%%% LaTeX-Vorlage Version 1.8 %%%

% Grundlegende Dokumenteneigenschaften gemäß DHBW-Vorgaben
\documentclass[a4paper,fontsize=11pt,oneside,parskip=half,headings=normal]{scrreprt} 
% \usepackage{showframe} % nur für Kontrolle der Ränder 

%%% Präambel einbinden (mit Festlegungen gemäß DHBW-Vorgaben) %%%
\input{template/_dhbw_praeambel.tex}

%%% Name der eigenen Literatur-Datenbank (ggf. anpassen) %%%
\bibliography{includes/literatur-datenbank.bib}

\begin{document}
%%% Deckblatt einbinden %%% 
% Anpassungen nötig (Name, Titel etc.)
% HIER EDITIEREN: 
% Typ der Arbeit (für Deckblatt und ehrenwörtliche Erklärung)
% - bitte Zutreffendes auswählen
%\newcommand{\typMeinerArbeit}{1. Projektarbeit} 
%\newcommand{\typMeinerArbeit}{2. Projektarbeit} 
%\newcommand{\typMeinerArbeit}{Seminararbeit} 
\newcommand{\typMeinerArbeit}{Bachelorarbeit} 

% Thema der Arbeit (für ehrenwörtliche Erklärung, ohne Umbrüche)
% HIER EDITIEREN: 
\newcommand{\themaMeinerArbeit}{Existiert eine Korrelation zwischen Storypoint-Aufwandsabschätzungen und Soft-warekomplexitätsmetriken? - Eine deskriptive Fallstudie sechs agiler Softwareprojekte}

% Vorname, Name der Autorin/des Autors (für Titelseite und Metadaten)
% HIER EDITIEREN:
\newcommand{\meinName}{Tim Struthoff}

\thispagestyle{empty}

\begin{spacing}{1}
\begin{center}	
~\vspace{0mm}

% HIER EDITIEREN: Titel der Arbeit
{\sffamily
\Large
% \Large  % bei sehr langen Titeln ggf. etwas kleinere Schriftart wählen
\textbf{Existiert eine Korrelation zwischen Storypoint-Aufwandsabschätzungen und Softwarekomplexitätsmetriken? }

\bigskip
\textbf{Eine deskriptive Fallstudie sechs agiler Softwareprojekte}
}


\vspace{15mm}

% Typ wird automatisch eingefügt (oben festlegen)
{\Large \typMeinerArbeit}

\vspace{1cm}

% HIER ggf. EDITIEREN
vorgelegt am \today 

\vspace{15mm}

Fakultät Wirtschaft
\medskip

Studiengang Wirtschaftsinformatik
\medskip

% HIER EDITIEREN: Kurs eintragen
Kurs WI2019I 

\vspace{10mm}

von

\vspace{10mm}

% Vorname und Name wird automatisch eingefügt (oben festlegen) 
{\large\textsc{\meinName}}

\vspace{10mm}
\end{center}

\vfill

% HIER EDITIEREN: Name des Unternehmens, Name der Betreuerin/des Betreuers
\begin{tabular}{ll}
Betreuer in der Ausbildungsstätte: & DHBW Stuttgart: \\
\hspace{0.4\linewidth} & \\
DXC Technologies & Katja Sattler \\
(Titel) Andreas Jordan & Delivery Lead, Testing and Digital Assurance \\
$\langle$ Funktion der Betreuerin/des Betreuers $\rangle$ \\
\\
Unterschrift der Betreuerin/des Betreuers \\
\end{tabular}


\vspace{1cm}
%(etwas Platz für die Unterschrift der Betreuerin/des Betreuers aus der Ausbildungsstätte)
\end{spacing}

% falls ein Vertraulichkeitsvermerk erforderlich ist,
% die Kommentarzeichen in den nachfolgenden Zeilen entfernen:
 
%\begin{center}
%\small
%\textbf{Vertraulichkeitsvermerk}:
%Der Inhalt dieser Arbeit darf weder als Ganzes noch in Auszügen \\
%Personen außerhalb des Prüfungs- und Evaluationsverfahrens zugänglich gemacht werden, sofern keine anders lautende Genehmigung des Dualen Partners vorliegt. 
%\end{center}

% Meta-Daten für PDF-Datei basierend auf obigen Angaben
\hypersetup{pdftitle={\themaMeinerArbeit}}
\hypersetup{pdfauthor={\meinName}}
\hypersetup{pdfsubject={\typMeinerArbeit\ DHBW Stuttgart \the\year}}

%%% Umstellung der Seiten-Nummerierung auf i, ii, iii ... %%%
\pagenumbering{Roman} 

%%% Abstract einbinden (optionale Kurzfassung Ihrer Arbeit) %%%
% \input{includes/0-preface/0.2-abstract.tex}
\cleardoublepage

%%% Inhalts-, Abbildungs-, Tabellenverzeichnisse %%%
% sollen einzeilig gesetzt werden, um Platz zu sparen 
\begin{spacing}{1}
\tableofcontents
\clearpage
\chapter*{Abkürzungsverzeichnis}
\addcontentsline{toc}{chapter}{Abkürzungsverzeichnis}

\begin{acronym}[DHBW] 
% Argument definiert die Breite der ersten Spalte anhand des längsten vorkommenden Eintrags
\acro{CRM}{Customer Relationship Management}
\acro{DIL}{Digital Innovation Lab}
\acro{WI}{Wirtschaftsinformatik}
\acro{IEEE}{Institute of Electrical and Electronics Engineers}
\acro{SLOC}{Anzahl der logischen Codezeilen}
\acro{LOC}{Anzahl der Codezeilen}
\acro{CSV}{Comma-separated values}
\acro{PNG}{Portable Network Graphik}
\acro{PDF}{Portable Document Format}
\acro{UTC}{Coordinated Universal Time}


\end{acronym}


\clearpage
\thispagestyle{kapitelkopfzeile}
\listoffigures
\phantomsection
\addcontentsline{toc}{chapter}{Abbildungsverzeichnis} % Abb.verz. ins Inh.verz. aufnehmen

\clearpage
\listoftables
\phantomsection
\addcontentsline{toc}{chapter}{Tabellenverzeichnis}   % Tab.verz. ins Inh.verz. aufnehmen
\end{spacing}

%%% Umstellung der Seiten-Nummerierung auf 1, 2, 3 ... %%%
\cleardoublepage
\pagenumbering{arabic}

%%% Ihr eigentlicher Inhalt %%%
% Empfehlung: strukturieren Sie Ihren Text in einzelnen Dateien 
% und binden Sie diese hier mit \input{includes/dateiname.tex} ein

\chapter{Cheatsheet}

Text\footnote{Fußnote}.

Referenz \ref{section:zeichencodierung}

Abschnitten~\ref{section:werkzeuge}

\url{https://ctan.org/pkg/hyperref}

\verb|_dhbw_|

\begin{verbatim}
_dhbw_biblatex-config.tex  (weitere Einstellung für Biblatex)
_dhbw_erklaerung.tex       (ehrenwörtliche Erklärung)
_dhbw_kopfzeilen.tex       (Kapitelname in Kopfzeilen) 
_dhbw_praeambel.tex        (Einbindung der benötigten Pakete)
\end{verbatim}

\lstset{language=TeX} 
\begin{lstlisting}
% HIER EDITIEREN: 
\end{lstlisting}

\section{Werkzeuge}\label{section:werkzeuge}

\subsection{Titel und Erklärung}

\subsubsection{Titel und Erklärung}

\emph{Hinweis:}


\chapter{Einleitung}

Die Abteilung \ac{DIL} Ratingen des IT-Beratungs- und Dienstleistungsunternehmens DXC Technology befasst sich unter anderem mit der Herstellung von Individualsoftware für eine Vielzahl von Kunden\footcite{BefragungMitarbeiternDigital2022}
\chapter{Softwarekomplexitätsma\ss e}\label{softwarekomplexituxe4tsmauxdfe}

Die Analyse der Softwarekomplexität ist ein Teil der statischen
Sourcecode-Analyse. In der Code-Analyse wird im Nachhinein (a
posterio\footcite[Vgl. ][S. 261]{hoffmannSoftwareQualitat2013}) anhand von statischen
Analysen einer Stichprobe zu einem bestimmten Zeitpunkt\footcite[Vgl. ][S. 86]{ebertSoftwareMetrikenPraxisEinfuhrung1996} des Sourcecodes festgestellt, ob die Software vorher
definierten Qualitätsanforderungen entspricht\footcite[Vgl. ][S. 261]{ebertSoftwareMetrikenPraxisEinfuhrung1996}.
Diese Analyse kann sowohl automatisiert als auch manuell erfolgen und
wird im Gegensatz zu Verfahren der konstruktiven Qualitätssicherung auf
bereits vorhandene Software angewendet\footcite[Vgl. ][S. 261]{hoffmannSoftwareQualitat2013}.

Als Teil der statischen Code-Analyse ist die Messung der
Softwarekomplexität ein Teil der Qualitätssicherung der
Software \footcite[Vgl. ][S. 261]{hoffmannSoftwareQualitat2013}. Dabei
sollen oft unsichtbare Eigenschaften der Software sichtbar und
quantitativ messbar gemacht werden\footcite[Vgl. ][S. 261]{hoffmannSoftwareQualitat2013}
\footcite[Vgl. ][S. 561]{zuseSoftwareComplexityMeasures1991}. Der Qualität von Software können
verschiedene Teilbereiche untergeordnet werden. Je nach Definition
gehören hierzu unter anderem Funktionalität, Laufzeit, Zuverlässigkeit,
Benutzbarkeit, Wartbarkeit, Transparenz, Übertragbarkeit und
Testbarkeit\footcite[Vgl. ][S. 22f]{hoffmannSoftwareQualitat2013} \footcite[Vgl. ][S. 245]{liggesmeyerSoftwareQualitatTestenAnalysieren2009}. Gerade auf die Qualitätsmerkmale Wartbarkeit und
Testbarkeit hat die Komplexität der Software einen direkten Einfluss. So
steigt der Aufwand des Wartens und Testens mit dem Umfang und der
Komplexität der Software.

Im Lebenszyklus einer Software kann die Komplexitätsuntersuchung als
Teil der Qualitätssicherung sowohl an das Ende der Entwicklungsphase
gestellt werden als auch kontinuierlich parallel zu der
Weiterentwicklung stattfinden.

\section{Definitionen}\label{definitionen}

Eine Betrachtung von Softwarekomplexitätsma\ss en setzt zunächst eine
genaue Definition dieser voraus. Softwarekomplexitätsmetriken sind
Metriken zur Messung der Komplexität einer Software. Diese drei Teile
der Definition von Softwarekomplexitätsmetriken werden im Folgenden
definiert.

Laut dem \ac{IEEE} Standard Glossar der Softwareentwicklung besteht
\emph{Software} aus den Computerprogrammen, Prozeduren und
gegebenenfalls der Dokumentation und den Daten, die den Betrieb eines
Computersystems betreffen\footcite[Vgl. ][S. 66]{IEEEStandardGlossary}

Die Definition von \emph{Komplexität} wird allgemein als schwierig
erachtet \footcite[Vgl. ][S. 335 und 627]{jonesAppliedSoftwareMeasurement2008}.
Aber auch hier schlägt das \ac{IEEE} Standard Glossar eine allgemein
anerkannte Definition vor: „{[}Complexity is{]} the degree to which a
system or component has a design or implementation that is difficult to
understand and verify``\footcite[Vgl. ][S. 18]{IEEEStandardGlossary}. Nach dieser
Definition ist Komplexität ein Ma\ss  dafür, wie schwierig eine Software zu
verstehen und zu validieren ist. Fenton und Jones bauen auf dieser
Definition auf und schlagen vier Arten von Komplexität vor: 1. Die
Komplexität des Problems, welches die Software zu lösen versucht, 2. Die
Komplexität der Algorithmen der Software, 3. Die Komplexität der
Struktur der Software und 4. Die kognitive Komplexität der
Software \footnote{\cite[Vgl. ][S. 258]{fentonSoftwareMetricsRigorous2003} und \cite[][S. 449]{jonesAppliedSoftwareMeasurement2008}}. Nach dieser Kategorisierung bestimmen die, in dieser
Arbeit behandelten Metriken die Komplexität der Struktur der Software
(3). Weiter lässt sich zwischen inter- und intra-modularen
Komplexitätsma\ss en unterscheiden. Hier werden ausschlie\ss lich
intra-modulare Komplexitätsma\ss en behandelt. Diese Messen die Komplexität
einzelner Programmteile\footcite[Vgl. ][S. 7ff]{zuseSoftwareComplexityMeasures1991}.

Im Kontext der Softwaremetrie werden Metriken als ein Messystem bzw. ein
Verfahren zum Quantifizieren von Eigenschaften von Software
definiert \footnote{\cite[Vgl. ][S. 35ff]{dumkeTheorieUndPraxis1994}, \cite[][S. 4ff]{ebertSoftwareMetrikenPraxisEinfuhrung1996}, \cite[][]{augstenWasSindSoftwaremetriken} und \cite[][S. 2f]{IEEEStandardSoftware}}. Im Deutschen werden sie oft als Synonym für Ma\ss  bzw. Ma\ss zahl
genutzt. Im Englischen findet zwischen den Begriffen „metric`` und
„measure`` eine genauere Unterscheidung statt: Eine Metrik sei eine
Funktion, die als Eingabe Daten eines Gegenstandes verwendet und hieraus
eine Zahl zur Quantifizierung dieser Eigenschaft(en) liefert\footcite[Vgl. ][S. 3]{IEEEStandardSoftware}. Ein Ma\ss  (measure) sei dahingegen die Konkrete Anwendung dieser
Metrik\footcite[Vgl. ][S. 2]{IEEEStandardSoftware}. Aufgrund dieser
Unterscheidung erscheint der Begriff Metrik für die, in dieser Arbeit
behandelten Verfahren als passender.

Zusammengefasst werden Softwarekomplexitätsmetriken in dieser Arbeit als
Verfahren zur Quantifizierung der Vielschichtigkeit der Struktur eines
Computerprogramms definiert.

\section{Kritik}\label{kritik}

Die Praktik der Softwarekomplexitätsmetrie wird allgemein stark
kritisiert.

Zum einen besteht Kritik an dem quantitativen Erfassen subjektiver
Softwareeigenschaften im Generellen. So können Metriken nicht
\emph{direkt} messen, was für Menschen als subjektive Komplexität
wahrgenommen wird, auch wenn das Verwenden von mehreren Ma\ss zahlen hilft,
eine robustere Perspektive zu schaffen\footcite[Vgl. ][S. 2]{rumreichExaminingSoftwareDesign2019}.

Weiter werden auch die Komplexitätsmetriken im Speziellen angezweifelt:
Viele Metriken vereinen mehrere, oft konfliktäre Messziele\footcite[Vgl. ][S. 322]{fentonSoftwareMetricsRigorous2003}. Dadurch wird ihre Aussagekraft verwässert. Nicht nur an dem
mathematischen Aufbau der Metriken, sondern auch an den
Implementierungen dieser gibt es Kritik. So liefern verschiedene
Implementierungen derselben Metrik oft verschiedene
Ergebnisse\footcite[Vgl. ][S. 2]{rumreichExaminingSoftwareDesign2019}. Zusätzlich wurden
die klassischen Metriken, wie z.B. die zyklomatische Komplexität für
imperative Programmiertechniken entwickelt. Das lässt sich darauf
zurückführen, dass zu ihrem Entwicklungszeitpunkt die imperative
Programmierung noch am weitesten verbreitet war\footcite[Vgl. ][S. 277]{hoffmannSoftwareQualitat2013}. Heutzutage sind andere Programmiertechniken, wie z.B. die
objekt-orientierte Programmierung verbreiteter. Diese neuen
Programmiertechniken wurden jedoch in den klassischen Metriken nicht
berücksichtigt\footcite[Vgl. ][S. 277]{hoffmannSoftwareQualitat2013}. Ein Beispiel dieser
Problematik sind die Vererbungsmechanismen in objekt-orientierten
Programmiersprachen. Eine Vererbung erhöht z.B. in der klassischen
Metrik der zyklomatischen Komplexität nicht die Komplexität, was an
dieser Stelle die Korrelation von Code-Grö\ss e und der Komplexität des
Codes aushebelt\footcite[Vgl. ][S. 277]{hoffmannSoftwareQualitat2013}.

Ferner ist auch der Zusammenhang zwischen Softwaremetriken und externen
Softwareeigenschaften, wie z.B. der Fehleranfälligkeit
umstritten\footcite[Vgl. ][S. 627]{jonesAppliedSoftwareMeasurement2008}. Hier konnten sowohl Studien
gefunden werden, die für eine Korrelation sprechen, also auch Studien,
die dagegensprechen. In Ebert 1996 konnte z.B. ein Zusammenhang zwischen
der gemessenen Komplexität einer Aufgabe und den dabei entstandenen
Fehlern nachgewiesen werden\footcite[Vgl. ][S. 65]{ebertSoftwareMetrikenPraxisEinfuhrung1996}. Im Gegensatz dazu
konnten in Revilla 2007 keine Zusammenhänge zwischen internen
Softwaremetriken und externen Eigenschaften von Software festgestellt
werden\footcite[Vgl. ][S. 203 und 208]{revillaCorrelationsInternalSoftware2007}.

Als mögliches Fazit aus diesen Studien wird in dieser Arbeit
vorschlagen, dass die Anwendbarkeit der Ma\ss zahlen vom jeweiligen Kontext
des Softwareprojektes und von dem erwarteten Ergebnis abhängt. Dieses
Fazit unterstreicht noch einmal die Relevanz dieser Arbeit. So kann mit
dem Ergebnis dieser Arbeit evaluiert werden, ob und welche
Softwarema\ss zahlen im Kontext des \ac{DIL} Ratingen anwendbar sind.

\section{Eine Auswahl von Softwarekomplexitätsma\ss zahlen}\label{eine-auswahl-von-softwarekomplexitatsmasszahlen}

Trotz der umfangreichen Kritik an der Vermessung von Software wurden in
den letzten Jahrzehnten eine Vielzahl von Ma\ss zahlen entwickelt.

In dieser Arbeit werden insgesamt vier Metriken für die Komplexität von
Software betrachtet. Zunächst bietet die Anzahl an logischen Codezeilen
einen ersten Eindruck der Komplexität. Sie ist aber noch eher ein
Umfangsma\ss  als ein Komplexitätsma\ss . Weiter werden die zyklomatische
Komplexität von Thomas McCabe, sowie die Softwarema\ss zahlen von Halstead
behandelt. Als letzte Ma\ss zahl wird die Einrückungskomplexität der
Autoren Hindle, Godfrey, und Holt betrachtet.

Die Auswahl der ersten drei Komplexitätsma\ss en beruft sich auf ihre
generelle Popularität in der Softwaremetrie. Im Rahmen einer
umfangreichen Literaturrecherche konnten für diese Metriken die meisten
Referenzen gefunden werden. Die Ma\ss zahl der logischen Codezeilen wird
von Zuse91 als ein wichtiges Ma\ss  zum Bestimmen des Umfangs und der
Komplexität einer Softwareanwendung eingestuft\footcite[Vgl. ][S. 145]{zuseSoftwareComplexityMeasures1991}.
Diese Position wird auch von \cite{satoExperiencesTrackingAgile2006} und
\cite{aleneziEmpiricalAnalysisComplexity2015} bekräftigt. Die zyklomatische
Komplexität von McCabe wird ebenfalls von Zuse als eine der bekanntesten
Ma\ss zahlen eingestuft \footcite[Vgl. ][S. 145]{zuseSoftwareComplexityMeasures1991}. Weiter bekräftigt
fentonSoftwareMetricsRigorous2003, dass diese Ma\ss zahl zum Messen der
Softwarekomplexität geeignet sei\footcite[Vgl. ][S. 31]{fentonSoftwareMetricsRigorous2003}. Auch die vier Autoren Revilla\footcite[Vgl. ][S. 203]{revillaCorrelationsInternalSoftware2007},
Jones\footcite[Vgl. ][S. 335, 627 und 449]{jonesAppliedSoftwareMeasurement2008}, Sato\footcite[Vgl. ][]{satoExperiencesTrackingAgile2006}
und Alenezi\footcite[Vgl. ][]{aleneziEmpiricalAnalysisComplexity2015} verweisen
auf die zyklomatische Komplexität als Ma\ss zahl für die Komplexität einer
Applikation. Jones sagt dabei zusätzlich aus, dass diese
Komplexitätsma\ss zahl ein besonders breites Anwendungsspektrum bedienen
könne\footcite[Vgl. ][S. 335, 627 und 449]{jonesAppliedSoftwareMeasurement2008}. Auch die Softwarema\ss zahlen
von Halstead werden in der Literatur häufig referenziert. Zuse und
Revilla bekräftigen die Verbreitung dieser Ma\ss zahlen\footnote{\cite[Vgl. ][S. 145]{zuseSoftwareComplexityMeasures1991} und \cite[S. 203]{revillaCorrelationsInternalSoftware2007}}. Zusätzlich sagt Fenton aus, dass die Ma\ss zahlen
geeignet zum Messen der Softwarekomplexität seien\footcite[Vgl. ][S. 31]{fentonSoftwareMetricsRigorous2003}.

Zusätzlich wird die Einrückungskomplexität betrachtet, da sie einen, im
Gegensatz zu den anderen Ma\ss zahlen methodisch sehr abweichenden Ansatz
verfolgt. So betrachtet sie im Gegensatz zu den
anderen Ma\ss zahlen nicht die Aufteilung auf Codezeilen oder den Inhalt
des Codes, sondern die Einrückung der einzelnen Codezeilen.

Neben diesen vier Ma\ss zahlen bestehen auch noch eine Vielzahl an weiteren
Messungsmethoden, die in dieser Arbeit aber keine weitere Betrachtung
finden sollen. Zum Beispiel wurden die gewichtete Anzahl an Methoden pro
Klasse (WMC)\footcite[Vgl. ][]{satoExperiencesTrackingAgile2006}, die Anzahl an
Kommentaren\footcite[Vgl. ][S. 207]{revillaCorrelationsInternalSoftware2007}, die Grö\ss e der Klasse\footcite[Vgl. ][]{satoExperiencesTrackingAgile2006}
und die Anzahl an Testcodezeilen\footcite[Vgl. ][]{satoExperiencesTrackingAgile2006}
nicht weiter betrachtet.

Zwischen drei der ausgewählten Ma\ss zahlen wurden bereits Korrelationen
nachgewiesen. Zunächst besteht eine starke Korrelation zwischen der
Anzahl an logischen Codezeilen und dem Aufwand nach Halstead\footcite[Vgl. ][S. 627]{jonesAppliedSoftwareMeasurement2008}. Eine noch stärkere, linear-stabile Korrelation lässt sich
zwischen der Anzahl logischer Code Zeilen und der zyklomatischen
Komplexitätsma\ss zahl nachweisen. Diese Korrelation konnte über
Programmierer*innen, Programme, Sprachen und Programmierparadigma hinweg
bewiesen werden\footnote{\cite[Vgl. ][S. 627]{jonesAppliedSoftwareMeasurement2008} und \cite[S. 137]{jayCyclomaticComplexityLines2009}}.

Die vier Komplexitätsmetriken werden nun der Reihe nach genauer
erläutert.

\subsection{Logische Codezeilen}\label{logische-codezeilen}

Als erste hier vorgestellte Metrik weist die \ac{SLOC} den geringsten Berechnungsaufwand auf. Sie wird
teilweise auch als \ac{LOC} bezeichnet\footcite[Vgl. ][S. 263]{hoffmannSoftwareQualitat2013}. Dabei werden die Zeilen an Sourcecode in der Software
berechnet\footcite[Vgl. ][S. 2]{rumreichExaminingSoftwareDesign2019}. Die Metrik lässt
sich somit ohne die Unterstützung komplexer Algorithmen berechnen und
lässt sich auf nahezu alle Programmiersprachen anwenden\footcite[Vgl. ][S. 263]{hoffmannSoftwareQualitat2013}. Eine Ausnahme bilden hier grafische
Programmiersprachen.

In ihrer Aussagekraft wird die Anzahl von Codezeilen allgemein als
begrenzt eingestuft\footnote{\cite[Vgl. ][S. 264]{hoffmannSoftwareQualitat2013} 
und \cite[S. 2]{rumreichExaminingSoftwareDesign2019}}. Unter anderem der Programmierstil und die Programmiersprache
haben einen Einfluss auf die Anzahl der Codezeilen (Hoffmann 2013:264,
Hoffmann 2013:264). Gleichzeitig erfüllt sie jedoch die Assoziativität,
Kommutativität und Monotonität\footcite[Vgl. ][S. 142]{zuseSoftwareComplexityMeasures1991} als wesentliche Qualitätskriterien für Ma\ss zahlen.

Eine Weiterentwicklung der LOC-Metrik ist u.a. die NCSS-Metrik (Non
Commented Source Statements). Sie misst genauso wie die LOC-Metrik die
Anzahl an Codezeilen, ignoriert dabei aber alle Kommentarzeilen\footcite[Vgl. ][S. 264]{hoffmannSoftwareQualitat2013}. Zusätzlich bestehen auch empirisch ermittelte
Sprachfaktoren, die es ermöglichen, die Ergebnisse von LOC und NCSS
Metriken vor ihrer Weiterverarbeitung zu gewichten\footcite[Vgl. ][S. 264]{hoffmannSoftwareQualitat2013}.

\subsection{Zyklomatische Komplexität}\label{Zyklomatische-Komplexitat}

Die zyklomatische Komplexität wurde von Tom McCabe entwickelt und zuerst
in seinem Aufsatz „A Complexity Metric``\footcite[Vgl. ][]{mccabeComplexityMeasure1976}
veröffentlicht\footcite[Vgl. ][S. 185]{sneedSoftwareZahlenVermessung2010}. Die zyklomatische
Komplexität gehört zu den Kontrollflussmetriken\footcite[Vgl. ][S. 88]{ebertSoftwareMetrikenPraxisEinfuhrung1996}. In der Praxis wird sie von vielen Firmen
entwicklungsbegleitend kontinuierlich erhoben, um vor problematischen
Programmkomponenten frühzeitig zu warnen. Dabei flie\ss t sie oft auch in
die Abnahmekriterien für Software ein\footcite[Vgl. ][S. 275]{hoffmannSoftwareQualitat2013}.

Die zyklomatische Komplexität ist definiert als eine Ma\ss zahl für die
Kontrollflusskomplexität eines Programmes\footnote{\cite[Vgl. ][S. 2]{rumreichExaminingSoftwareDesign2019} und \cite[S. 335]{jonesAppliedSoftwareMeasurement2008}}. Sie misst die Anzahl von linear
unabhängigen Pfaden auf dem Kontrollflussgraphen eines Programmes und
liefert so ein normalisiertes Komplexitätsma\ss \footcite[Vgl. ][S. 2]{rumreichExaminingSoftwareDesign2019}.

Zur Berechnung der zyklomatischen Komplexität wird der Programmcode
zunächst in einen Graphen aus Knoten und Kanten umgewandelt\footnote{\cite[Vgl. ][S. 185]{sneedSoftwareZahlenVermessung2010} und \cite[S. 335]{jonesAppliedSoftwareMeasurement2008}}. Dazu wird der Quelltext in einen
Syntaxbaum umgewandelt. Diese Baumstruktur spiegelt die Struktur des
Codes wider. Aus dem Syntaxbaum wird ein Kontrollflussgraph der
Anwendung konstruiert. Der Kontrollflussgraph besteht aus den
Anweisungen und Verzweigungen des Programms und stellt alle möglichen
Ablaufwege durch das Programm dar\footcite[Vgl. ][]{rackeKontrollflussdiagramme2017}.
Der Umwandlungsprozess wird in Abbildung .. beschrieben.

GRAFIK

Aus den Eigenschaften dieses Kontrollflussgraphen kann nun die
zyklomatische Komplexität mit Mitteln der Graphentheorie\footcite[Vgl. ][S. 273]{hoffmannSoftwareQualitat2013} berechnet werden. Als relevante Grö\ss en kommen dabei die
Anzahl der Kanten des Graphen (E), die Anzahl der Knoten (N), sowie die
Anzahl unabhängiger Teilgraphen (p) zum Einsatz. Die Anzahl unabhängiger
Teilgraphen entspricht der Anzahl unabhängiger Module bzw. Funktionen in
dem Programm. Wird ein Programm mit mehreren Modulen untersucht, werden
die Kanten und Knoten der Kontrollflussgraphen der Module
aufsummiert\footcite[Vgl. ][S. 275f]{hoffmannSoftwareQualitat2013}.

Die Formel für die zyklomatische Komplexität lautet wie folgt:

V(Z) = \textbar E\textbar-\textbar N\textbar{} + 2p

Die zyklomatische Komplexität (V(Z)) entspricht also der Anzahl an
Kanten (E) minus der Anzahl an Knoten (N) plus der doppelten Anzahl der
zusammenhängenden Einzelgraphen (p) (Hoffmann 2013:273, Jones 2008:335,
Hoffmann 2013:273, Ebert 1996:88, Sneed et al 2010:185). Also ist sie
stets um eins grö\ss er als die Anzahl an Verzweigungen in einem Programm
(Hoffmann 2013:274).

In der konkreten Anwendung wird die zyklomatische Komplexität oft als
Indikator für die Wartbarkeit des Code und für die Produktivität in der
Weiterentwicklung des Codes verwendet\footcite[Vgl. ][S. 336]{jonesAppliedSoftwareMeasurement2008}.
Zusätzlich liefert sie die obere Grenze der Anzahl an Tests, die für
eine vollständige Testabdeckung des Quelltextes benötigt werden.

Die zyklomatische Komplexität wurde in der Fachliteratur vielfach
zitiert und vor allem auch vielfach kritisiert.

Zunächst wird allgemein kritisiert, dass die Ma\ss zahl nur die Komplexität
der Ablauflogik des Codes und nicht die Komplexität des gesamten Codes
misst\footcite[Vgl. ][S. 186]{sneedSoftwareZahlenVermessung2010}. Interne Eigenschaften der Knoten\footnote{\cite[Vgl. ][S. 73]{hoffmannSoftwareQualitat2013} und \cite[S. 2]{rumreichExaminingSoftwareDesign2019}}, die Komplexität im Datenfluss\footcite[Vgl. ][S. 2]{rumreichExaminingSoftwareDesign2019} sowie die Verschachtelung von Modulen werden
ignoriert\footcite[Vgl. ][S. 89ff]{zuseSoftwareComplexityMeasures1991}. Also lässt sich sagen, dass die
Ma\ss zahl für prozeduralen Code eine gewisse Bedeutung hat, in
objektorientiertem Code jedoch nur die Komplexität der einzelnen
Methodenkörper misst\footcite[Vgl. ][S. 186]{sneedSoftwareZahlenVermessung2010}.

Insgesamt wird geschlussfolgert, dass die zyklomatische Komplexität nur
mit Vorsicht anzuwenden sei und in Relation zu anderen Ma\ss zahlen gesetzt
werden müsse\footcite[Vgl. ][S. 52]{fentonSoftwareMetricsRigorous2003}. Genau
aus diesem Grund wird sie in dieser Arbeit in den Kontext der agilen
Aufwandsabschätzungen und in Relation zu anderen Ma\ss zahlen gesetzt.

\subsection{Halstead Metriken}\label{Halstead-Metriken}

Als dritte Komplexitätsmetrik werden die Halstead Metriken herangezogen.
Bei diesen Metriken wendet der Autor Maurice Howard Halstead\footcite[Vgl. ][]{halsteadElementsSoftwareScience1979}
Elemente der Kommunikationstheorie auf die Programmierung an\footcite[Vgl. ][S. 183]{sneedSoftwareZahlenVermessung2010}.

Die Berechnung der Halstead Metriken wird im folgenden an einem Beispiel
erklärt\footcite[Vgl. ][S. 184]{sneedSoftwareZahlenVermessung2010}. Das Beispiel behandelt
folgenden Code:

\lstset{language=C}
\begin{lstlisting}
main()
{
int countOne, countTwo, countThree, average;
scanf(``%d %d %d'', &countOne, &countTwo, &countThree);
average = (countOne + countTwo + countThree) / 3;
printf(``average = \%d'', average);
}
\end{lstlisting}


Für die Berechnung seiner Metriken zählt Halstead zunächst die
Grundelemente der Programmiersprache, nämlich die Operatoren und
Operanden\footnote{\cite[Vgl. ][S. 183]{sneedSoftwareZahlenVermessung2010} und \cite[S. 2]{rumreichExaminingSoftwareDesign2019}}. Operatoren führen Operationen auf Operanden durch. Beispiele
für Operatoren sind arithmetische Operatoren (z.B. Addition)
Vergleichsoperatoren und Zuweisungsoperatoren. Operanden sind in der
Regel Datenelemente wie Zahlen oder Text.

In diesem Beispiel sind die unterschiedlichen Operatoren: main, (),
\{\}, int, scanf, \&, =, +, /, printf, ,, ;

Die unterschiedlichen Operanden sind: countOne, countTwo, countThree,
average, "\%d \%d \%d", 3, "avg = \%d"

Zu den Operatoren und Operanden werden folgende Zahlen erfasst:

\begin{itemize}
\item
  n1: Die Anzahl an unterschiedlichen Operatoren (12)
\item
  n2: Die Anzahl an unterschiedlichen Operanden (7)
\item
  N1: Die Anzahl insgesamt vorkommender Operatoren (27)
\item
  N2: Die Anzahl insgesamt vorkommender Operanden (15)
\end{itemize}

Aus diesen Zahlen können nun die Halstead Metriken berechnet werden:

Der Wortschatz eines Programms ergibt sich aus der Summe der
unterschiedlichen Operatoren und Operanden:

\emph{Wortschatz (n) = Operatoren (n1) + Operanden (n2) 19 = 12 + 7}

Die Länge des Programmes ergibt sich aus der Summe der insgesamt
vorkommenden Operanden und Operatoren :

\emph{Länge (N) = Operatorenvorkommnisse (N1) + Operandenvorkommnisse
(N2) 42 = 27 + 15}

Weiter Berechnet Halstead die Grö\ss e des Programmcodes durch einen
Logarithmus:

\emph{Grö\ss e (Volume / V) = Länge (N) * log2 (Wortschatz (n) ) 53,71 = 42
* log(19)}

Für die Sprachkomplexität setzt er jeweils die Operatoren ins Verhältnis
zu den Operatorenvorkommnissen und die Operanden ins Verhältnis zu den
Operandenvorkommnissen. Beide Werte werden miteinander multipliziert:

\emph{Komplexität = (Operatoren / Operatorenvorkommnisse) * (Operanden /
Operandenvorkommnisse) 0,207 =ca (12 / 27) * (7 / 15)}

Zuletzt berechnet er den Aufwand als Quotient aus Programmgrö\ss e und
Komplexität. Je komplexer oder grö\ss er ein Programm ist, desto länger
dauere, es dieses Programm zu schreiben:

\emph{Aufwand (E) = Grö\ss e / Komplexität 259,47 = 53,71 / 0,207}

Die von Halstead vorgeschlagenen Metriken wurden vielfach kritisiert. Er
hat die Berechnungen nie empirisch untermauert und in späteren
empirischen Studien wurden sie sogar widerlegt\footcite[Vgl. ][S. 185]{sneedSoftwareZahlenVermessung2010}. Insbesondere sei zu kritisieren, dass die Codegrö\ss e,
Komplexität und der Aufwand nach Halstead nicht das Kriterium der
Monotonität erfüllen. Sie können also nicht auf einer Verhältnisskala
verwendetet werden und nur die Berechnung von Medianwerten sei
sinnvoll\footcite[Vgl. ][S. 142]{zuseSoftwareComplexityMeasures1991}. Eine zusätzliche Einschränkung
sei, dass die Halstead Metriken die Berechnungskomplexität in den
Vordergrund rücken und keine Aussage über den Kontrollfluss der
Anwendung treffen\footcite[Vgl. ][S. 2]{rumreichExaminingSoftwareDesign2019}. Insgesamt seien sie also kritisch anzusehen\footcite[Vgl. ][S. 185]{sneedSoftwareZahlenVermessung2010}.


\subsection{Einrückungskomplexität}\label{Einruckungskomplexitat}

Als letztes Komplexitätsma\ss  wird hier die Einrückungskomplexität
vorgestellt. Sie wurde von Abram Hindle, Michael W. Godfrey und Richard
C. Holt an der University of Waterloo entwickelt und zuerst mit ihrer
Arbeit „Reading Beside the Lines: Indentation as a Proxy for Complexity
Metrics `` auf der sechzehnten internationalen IEEE Konferenz zum
Verständnis von Computerprogrammen vorgestellt\footcite[Vgl. ][S. 1]{hindleReadingLinesIndentation2008}. Die Komplexitätsma\ss zahl ist im Vergleich zu den anderen hier
behandelten Metriken verhältnismä\ss ig neu, wurde aber bereits in Arbeiten
von Lalouche et. al\footcite[Vgl. ][S. 10]{gilWhenSoftwareComplexity2016}
\footcite[Vgl. ][S. 7]{gilCorrelationSizeMetric2017} und Landman et. al
\footcite[Vgl. ][S. 6]{landmanEmpiricalAnalysisRelationship2016} referenziert.

Die Einrückungskomplexität soll nun erklärt werden. In den meisten
Programmiersprachen werden bestimmte Zeilen zur besseren Lesbarkeit des
Codes eingerückt. Die Einrückungskomplexität macht sich genau diesen
Umstand zu Nutzen und verwendet die Einrückungen der Codezeilen als
Indikator für Komplexität\footcite[Vgl. ][S. 1]{hindleReadingLinesUsing2009}.

In prozeduralen Code, wie z.B. in C, kann die Einrückung von Codezeilen
Kontrollstrukturen, wie Verzweigungen und Schleifen anzeigen. In
objektorientierten Sprachen, wie C++, Java und JavaScript kann die
Einrückung eine Verkapselung in Form von Klassen, Subklassen oder
Methoden anzeigen. In funktionalen Sprachen wie OCaml und Lisp zeigt die
Einrückung einen neuen Handlungskontext, neue Funktionen oder einen
neuen Ausdruck. In jedem dieser drei Typen von
Programmiersprachen sind in der Regel Stellen im Code eingerückt, die
die Komplexität des Codes erhöhen. Also lasse sich schlussfolgern, dass
die Menge der Einrückungen ein Ma\ss  der Komplexität eines Programmes sei.

Zur Verifizierung dieser These setzten die Autoren die
Einrückungskomplexität in ein Verhältnis zu den älteren Metriken wie der
zyklomatischen Komplexität und den Halstead Metriken. Bei der
zyklomatischen Komplexität erhöhen z.B. Verzweigungen die Komplexität.
Diese werden in den meisten Programmiersprachen durch eine Einrückung
der Codezeilen gekennzeichnet. Also stünde die Vermutung einer
Korrelation der Metriken nahe\footcite[Vgl. ][S. 2]{hindleReadingLinesUsing2009}.
Diese Korrelation konnte von den Autoren in einer Studie von 278
populären Open Source Projekten erfolgreich nachgewiesen
werden. Damit sei die Validität der
Einrückungskomplexität als Komplexitätsma\ss  von Software nachgewiesen.

Für die Berechnung der Einrückungskomplexität werden die Einrückungen
der einzelnen Codezeilen im Quelltext betrachtet. Die Anzahl an
physischen Einrückungen in einer Zeile bezeichnet zunächst die Anzahl an
Leerzeichen bzw. Tabs, welche sich am Anfang einer Zeile befinden. Diese
werden für jede Zeile berechnet. Anhand dieser Angaben wird auf der
Ebene einer Datei berechnet, wie viele Leerzeichen bzw. Tabs einer
logischen Einrückungsebene entsprechen\footcite[Vgl. ][S. 5]{hindleReadingLinesUsing2009}. In den meisten Projekten entsprechen vier Leerzeichen bzw. ein
Tab einem Einrückungslevel\footcite[Vgl. ][S. 9]{hindleReadingLinesUsing2009}.
Dieser Zusammenhang wird von den Autoren als das Einrückungsmodell einer
Datei bezeichnet. Mit dem Einrückungsmodell kann nun die Anzahl an
logischen Einrückungen für jede Datei berechnet werden. Die Anzahl
dieser Einrückungen gibt für jede Zeile die Einrückungsebene an. Die
Summe der logischen Einrückungen wird als die Einrückungskomplexität
bezeichnet.

Ein wesentlicher Vorteil der Einrückungskomplexität ist ihre
Sprachenunabhängigkeit. Im Gegensatz zu der zyklomatischen Komplexität
und den Halstead Metriken setzt sie kein Verständnis der Grammatik einer
Programmiersprache voraus\footcite[Vgl. ][S. 1]{hindleReadingLinesUsing2009}
\footcite[Vgl. ][S. 2]{hindleReadingLinesUsing2009}.

Als zusätzlichen Vorteil führen die Autoren auf, dass die
Einrückungskomplexität weniger Berechnungsschritte benötige als andere
Komplexitätsma\ss en. Demnach sei sie günstiger in der Durchführung als
manche andere Metriken\footcite[Vgl. ][S. 20f]{hindleReadingLinesUsing2009}.
Vergleichsoperatoren und Zuweisungsoperatoren. Operanden sind in der
Regel Datenelemente wie Zahlen oder Text.

In diesem Beispiel sind die unterschiedlichen Operatoren: main, (),
\{\}, int, scanf, \&, =, +, /, printf, ,, ;

Die unterschiedlichen Operanden sind: countOne, countTwo, countThree,
average, "\%d \%d \%d", 3, "avg = \%d"

Zu den Operatoren und Operanden werden folgende Zahlen erfasst:

\begin{itemize}
\item
  n1: Die Anzahl an unterschiedlichen Operatoren (12)
\item
  n2: Die Anzahl an unterschiedlichen Operanden (7)
\item
  N1: Die Anzahl insgesamt vorkommender Operatoren (27)
\item
  N2: Die Anzahl insgesamt vorkommender Operanden (15)
\end{itemize}

Aus diesen Zahlen können nun die Halstead Metriken berechnet werden:

Der Wortschatz eines Programms ergibt sich aus der Summe der
unterschiedlichen Operatoren und Operanden:

\emph{Wortschatz (n) = Operatoren (n1) + Operanden (n2) 19 = 12 + 7}

Die Länge des Programmes ergibt sich aus der Summe der insgesamt
vorkommenden Operanden und Operatoren :

\emph{Länge (N) = Operatorenvorkommnisse (N1) + Operandenvorkommnisse
(N2) 42 = 27 + 15}

Weiter Berechnet Halstead die Grö\ss e des Programmcodes durch einen
Logarithmus:

\emph{Grö\ss e (Volume / V) = Länge (N) * log2 (Wortschatz (n) ) 53,71 = 42
* log(19)}

Für die Sprachkomplexität setzt er jeweils die Operatoren ins Verhältnis
zu den Operatorenvorkommnissen und die Operanden ins Verhältnis zu den
Operandenvorkommnissen. Beide Werte werden miteinander multipliziert:

\emph{Komplexität = (Operatoren / Operatorenvorkommnisse) * (Operanden /
Operandenvorkommnisse) 0,207 =ca (12 / 27) * (7 / 15)}

Zuletzt berechnet er den Aufwand als Quotient aus Programmgrö\ss e und
Komplexität. Je komplexer oder grö\ss er ein Programm ist, desto länger
dauere, es dieses Programm zu schreiben:

\emph{Aufwand (E) = Grö\ss e / Komplexität 259,47 = 53,71 / 0,207}

Die von Halstead vorgeschlagenen Metriken wurden vielfach kritisiert. Er
hat die Berechnungen nie empirisch untermauert und in späteren
empirischen Studien wurden sie sogar widerlegt\footcite[Vgl. ][S. 185]{sneedSoftwareZahlenVermessung2010}. Insbesondere sei zu kritisieren, dass die Codegrö\ss e,
Komplexität und der Aufwand nach Halstead nicht das Kriterium der
Monotonität erfüllen. Sie können also nicht auf einer Verhältnisskala
verwendetet werden und nur die Berechnung von Medianwerten sei
sinnvoll\footcite[Vgl. ][S. 142]{zuseSoftwareComplexityMeasures1991}. Eine zusätzliche Einschränkung
sei, dass die Halstead Metriken die Berechnungskomplexität in den
Vordergrund rücken und keine Aussage über den Kontrollfluss der
Anwendung treffen\footcite[Vgl. ][S. 2]{rumreichExaminingSoftwareDesign2019}. Insgesamt seien sie also kritisch anzusehen\footcite[Vgl. ][S. 185]{sneedSoftwareZahlenVermessung2010}.


\subsection{Einrückungskomplexität}\label{Einruckungskomplexitat}

Als letztes Komplexitätsma\ss  wird hier die Einrückungskomplexität
vorgestellt. Sie wurde von Abram Hindle, Michael W. Godfrey und Richard
C. Holt an der University of Waterloo entwickelt und zuerst mit ihrer
Arbeit „Reading Beside the Lines: Indentation as a Proxy for Complexity
Metrics `` auf der sechzehnten internationalen IEEE Konferenz zum
Verständnis von Computerprogrammen vorgestellt\footcite[Vgl. ][S. 1]{hindleReadingLinesIndentation2008}. Die Komplexitätsma\ss zahl ist im Vergleich zu den anderen hier
behandelten Metriken verhältnismä\ss ig neu, wurde aber bereits in Arbeiten
von Lalouche et. al\footcite[Vgl. ][S. 10]{gilWhenSoftwareComplexity2016}
\footcite[Vgl. ][S. 7]{gilCorrelationSizeMetric2017} und Landman et. al
\footcite[Vgl. ][S. 6]{landmanEmpiricalAnalysisRelationship2016} referenziert.

Die Einrückungskomplexität soll nun erklärt werden. In den meisten
Programmiersprachen werden bestimmte Zeilen zur besseren Lesbarkeit des
Codes eingerückt. Die Einrückungskomplexität macht sich genau diesen
Umstand zu Nutzen und verwendet die Einrückungen der Codezeilen als
Indikator für Komplexität\footcite[Vgl. ][S. 1]{hindleReadingLinesUsing2009}.

In prozeduralen Code, wie z.B. in C, kann die Einrückung von Codezeilen
Kontrollstrukturen, wie Verzweigungen und Schleifen anzeigen. In
objektorientierten Sprachen, wie C++, Java und JavaScript kann die
Einrückung eine Verkapselung in Form von Klassen, Subklassen oder
Methoden anzeigen. In funktionalen Sprachen wie OCaml und Lisp zeigt die
Einrückung einen neuen Handlungskontext, neue Funktionen oder einen
neuen Ausdruck. In jedem dieser drei Typen von
Programmiersprachen sind in der Regel Stellen im Code eingerückt, die
die Komplexität des Codes erhöhen. Also lasse sich schlussfolgern, dass
die Menge der Einrückungen ein Ma\ss  der Komplexität eines Programmes sei.

Zur Verifizierung dieser These setzten die Autoren die
Einrückungskomplexität in ein Verhältnis zu den älteren Metriken wie der
zyklomatischen Komplexität und den Halstead Metriken. Bei der
zyklomatischen Komplexität erhöhen z.B. Verzweigungen die Komplexität.
Diese werden in den meisten Programmiersprachen durch eine Einrückung
der Codezeilen gekennzeichnet. Also stünde die Vermutung einer
Korrelation der Metriken nahe\footcite[Vgl. ][S. 2]{hindleReadingLinesUsing2009}.
Diese Korrelation konnte von den Autoren in einer Studie von 278
populären Open Source Projekten erfolgreich nachgewiesen
werden\footcite[][]{hindleReadingLinesUsing2009}. Damit sei die Validität der
Einrückungskomplexität als Komplexitätsma\ss  von Software nachgewiesen.

Für die Berechnung der Einrückungskomplexität werden die Einrückungen
der einzelnen Codezeilen im Quelltext betrachtet. Die Anzahl an
physischen Einrückungen in einer Zeile bezeichnet zunächst die Anzahl an
Leerzeichen bzw. Tabs, welche sich am Anfang einer Zeile befinden. Diese
werden für jede Zeile berechnet. Anhand dieser Angaben wird auf der
Ebene einer Datei berechnet, wie viele Leerzeichen bzw. Tabs einer
logischen Einrückungsebene entsprechen\footcite[Vgl. ][S. 5]{hindleReadingLinesUsing2009}. In den meisten Projekten entsprechen vier Leerzeichen bzw. ein
Tab einem Einrückungslevel\footcite[Vgl. ][S. 9]{hindleReadingLinesUsing2009}.
Dieser Zusammenhang wird von den Autoren als das Einrückungsmodell einer
Datei bezeichnet. Mit dem Einrückungsmodell kann nun die Anzahl an
logischen Einrückungen für jede Datei berechnet werden. Die Anzahl
dieser Einrückungen gibt für jede Zeile die Einrückungsebene an. Die
Summe der logischen Einrückungen wird als die Einrückungskomplexität
bezeichnet.

Ein wesentlicher Vorteil der Einrückungskomplexität ist ihre
Sprachenunabhängigkeit. Im Gegensatz zu der zyklomatischen Komplexität
und den Halstead Metriken setzt sie kein Verständnis der Grammatik einer
Programmiersprache voraus \footnote{\cite[Vgl.][S. 1]{hindleReadingLinesUsing2009} und \cite[S. 2]{hindleReadingLinesUsing2009}}.

Als zusätzlichen Vorteil führen die Autoren auf, dass die
Einrückungskomplexität weniger Berechnungsschritte benötige als andere
Komplexitätsma\ss en. Demnach sei sie günstiger in der Durchführung als
manche andere Metriken\footcite[Vgl. ][S. 20f]{hindleReadingLinesUsing2009}.
\chapter{Aufwandsabschätzungen agiler Projekte}\label{aufwandsabschuxe4tzungen-agiler-projekte}

In dieser Arbeit sollen Softwarekomplexitätsmaßen in Korrelation zu
Aufwandsabschätzungen gesetzt werden. Für ein genaues Verständnis dieser
Korrelation ist es auch von Nöten, die Aufwandsabschätzungen selbst
genauer zu betrachten.

Die, in dieser Arbeit betrachteten Projekte werden alle in einer agilen
Arbeitsweise realisiert. Im Rahmen dieser Vorgehensweise werden
üblicherweise für jedes zu entwickelnde Feature Schätzungen des
Aufwandes abgegeben. Für ein besseres Verständnis der Entstehung dieser
Projekte wird im Folgenden zunächst die agile Arbeitsweise erläutert.
Dann werden die Aufwandsabschätzungen der agilen Projekte genauer
betrachtet und im Kontext dieser Arbeit erläutert.

\section{Die agile Arbeitsweise in der
Softwareentwicklung}\label{die-agile-arbeitsweise-in-der-softwareentwicklung}

Als die agile Arbeitsweise wird eine Reihe von iterativen
Methoden \footcite[Vgl. ][]{atlassianWhatAgile} beschrieben, welche den Denk- und
Arbeitsprozess von Softwareentwicklungsteams effizienter und effektiver
gestalten sollen. Dabei werden alle Bereiche der traditionellen
Softwareentwicklung, von dem Projektmanagement, über Softwaredesign und
-architektur bis hin zur Prozessoptimierung betrachtet. Zu jedem dieser
Bereiche bestehen Praktiken, die über Softwareprojekte hinweg möglichst
einfach zu implementieren sein sollen\footcite[Vgl. ][]{stellmanLearningAgile2014
  S. 2}.

Zusätzlich wird mit der agilen Arbeitsweise eine Mentalität beschrieben,
bei der Planung, Design und Prozessoptimierung von dem gesamten
Entwicklungsteam durchgeführt werden sollen\footcite[Vgl. ][]{stellmanLearningAgile2014
  S. 2}.

\section{Methoden der agilen Softwareentwicklung}\label{methoden-der-agilen-softwareentwicklung}

Die agile Vorgehensweise wird in der Praxis in Form von verschiedenen
Methoden umgesetzt. Diese Methoden definieren konkrete Abläufe und
Verantwortlichkeiten. Zu dem Methoden gehören unter anderem Scrum,
Kanban und das Scaled Agile Framework. Die in dieser Arbeit behandelten
Projekte werden nach Scrum umgesetzt. Aus diesem Grund wird im folgenden
Kapitel die Scrum Methodik erklärt.

\subsection{Ablauf}\label{Ablauf}

Zur Entwicklung eines Produktes nach Scrum befasst sich ein sog.
Productowner zunächst mit den Wünschen der Stakeholder und hält diese in
einem priorisierten Product-Backlog fest. Zu Beginn eines jeden Sprints
wird in einem sog. Sprint Planning Meeting von dem Entwicklungsteam
festgelegt, welche der Aufgaben in dem nächsten Sprint umgesetzt werden
können. Es wird dabei nach Priorität vorgegangen. Diese Arbeitspakete
werden in dem Sprint-Backlog für jeden Sprint festgehalten. Während des
Sprints setzen die Entwickler die Aufgaben im Sprint-Backlog um, während
der Productowner den Backlog bearbeitet\footcite[Vgl. ][]{schwaberAgileSoftwareDevelopment2002}
\footcite[Vgl. ][]{schwaberAgileProjectManagement2004}.

Die Verwaltung der Anforderungen im Backlog geschieht in Form von sog.
Userstorys. Diese Userstorys sollten die gewünschte Funktionalität, die
Rolle des Anwenders, und den Geschäftswert dieser Funktionalität
beinhalten. Daneben wird der Aufwand für die Realisierung des Features
geschätzt\footcite[Vgl. ][]{cohnUserStoriesApplied2004}. Der Aufwand wird nicht
als absolute Zahl geschätzt, sondern als Aufwand relativ zu anderen
Stories geschätzt. Hat Story A eine Storypointanzahl von 1 und B eine
Anzahl von 5, ist B mit fünfmal so viel Aufwand verbunden. In einigen
Projekten werden Storypoints im gleichen Sinne auch zur Schätzung der
Komplexität herangezogen\footcite[Vgl. ][]{Quelle fehlt}. Von den meisten
Scrum-Teams wird eine feste Zahlenfolge zur Schätzung der Aufwände
verwendet. Dabei ist die Fibonacci-Folge\footnote{erklären} besonders
beliebt\footcite[Vgl. ][]{Quelle fehlt}.

\subsection{Aufwandsabschätzungen mit Planning Poker}\label{Aufwandsabschatzungen-mit-Planning-Poker}

Eine weit verbreitete Technik zum Schätzen der Userstorys ist das
Planungspoker. Zur Schätzung einer Story gibt jedes Teammitglied
gleichzeitig und verdeckt eine individuelle Schätzung zu der Story ab.
In einem zweiten Schritt wird durch eine Diskussion eine Einigung über
das Ergebnis erzielt. Die so erzielten Ergebnisse sollten innerhalb
eines Projektes konsequent sein. Auch innerhalb eines Teams über
Projekte hinweg können oft konsequente Ergebnisse erzielt werden. Diese
Eigenheit wird auch in dieser Arbeit berücksichtigt. So werden die
Messwerte der Storypoints auf den Kontext der einzelnen Projekte
normalisiert\footcite{Quelle ergänzen cohnAgileEstimatingPlanning2006}
\footcite[Vgl. ][]{(daltonGreatBigAgile2019a S. 203).}.

\chapter{Forschungsaufbau der Fallstudie}\label{forschungsaufbau-der-fallstudie}

In dieser Arbeit soll eine Fallstudie durchgeführt werden. Entsprechend
der Fallstudienmethodik wird in diesem Kapitel der Forschungsaufbau
zunächst in einem Forschungsprotokoll festgehalten
(gothlichFallstudienAlsForschungsmethode2003 S. 8,
millsEncyclopediaCaseStudy2010 S. 69f). Wichtige Komponenten sind dabei
zunächst eine Forschungsfrage, welche die Problemstellung und die Ziele
des Handelns aufzeigt. Dann wird eine Hypothese über die Antwort
aufgestellt. Es werden die einzelnen Fälle bzw. die Analyseeinheiten
differenziert, ausgewählt und vorgestellt. Die Daten aus diesen
Analyseeinheiten werden in Verbindung zu den Hypothesen gesetzt und es
werden Kriterien zur Interpretation der Daten aufgestellt
(millsEncyclopediaCaseStudy2010 S. 69f).

\section{Forschungsfrage}\label{forschungsfrage}

Eine klare Forschungsfrage ist essenziell für die Fallstudienforschung,
um die Problemstellung und Ziele des Handels aufzuzeigen
(dubeRigorInformationSystems2003 S. 605, millsEncyclopediaCaseStudy2010
S. 70, dubeRigorInformationSystems2003 S. 607). In der
Fallstudienforschung können Forschungsfragen explorativ, deskriptiv und
explanativ sowie eine Kombination aller drei Arten sein
(dubeRigorInformationSystems2003 S. 605).

In dieser Arbeit soll das Thema der Softwarekomplexitätsmetriken
deskriptiv (Frage nach dem Wie) behandelt werden. Es soll beschrieben
werden, wie sich Komplexitätsmetriken im Vergleich zu
Aufwandsabschätzungen verhalten.

\section{Hypothesen}\label{hypothesen}

Vor der Beantwortung der Forschungsfrage werden zunächst Hypothesen über
mögliche Antworten aufgestellt
(gothlichFallstudienAlsForschungsmethode2003 S. 8). Das Aufstellen von
Hypothesen dient der Identifikation der weiteren Forschungsrichtung
(dubeRigorInformationSystems2003 S. 607, millsEncyclopediaCaseStudy2010
S. 70).

In dieser Arbeit wird die Hypothese verfolgt, dass eine eingeschränkte
Korrelation zwischen Softwarekomplexitätsmetriken und
Aufwandsabschätzungen nachgewiesen werden kann. Es ist zu erwarten, dass
Störfaktoren diese Korrelation beeinflussen.

\section{Analyseeinheit}\label{analyseeinheit}

Die Hypothese soll durch die Betrachtung einer Reihe von Fällen (Cases)
als Untersuchungseinheiten validiert oder widerlegt werden.

Im allgemein anerkannten Vorgehen zur Fallstudienforschung sei an dieser
Stelle eine klare Beschreibung der Fälle von Bedeutung
(dubeRigorInformationSystems2003 S. 610).

Auch an die Auswahl der Fälle bestünden Anforderungen. Die Fälle müssen
im Zusammenhang zu der Forschungsfrage gewählt werden, können in diesem
Rahmen aber durchaus willkürlich ausgewählt werden
(millsEncyclopediaCaseStudy2010 S. 72f, yinCaseStudyResearch2014 S. 72).
Eine willkürliche Auswahl der Fälle ist gerade deswegen wichtig, da
insbesondere Extremfälle die gesamte Bandbreite der möglichen
Analyseergebnisse abdecken können. Die willkürliche Auswahl muss jedoch
keinem Zufallsprinzip folgen\footcite[Vgl. ][]{Quelle fehlt (Ausreißer und
  Extremfälle gerade an diesen Extremfällen gelegen, da sie die
  Bandbreite abstecken, innerhalb derer sich die Realität bewegt und
  relevante Phänomene in diesen Fällen am deutlichsten zu Tage treten.)}.
Als Grundvoraussetzung zur Auswahl der Fälle lässt sich sagen, dass ein
möglichst umfassender Zugang zu Dokumenten und Artefakten des Falles
gegeben sein sollte (millsEncyclopediaCaseStudy2010 S. 68). Für eine
möglichst valide Endaussage sei eine Anzahl von vier bis zehn Fällen
anzustreben (gothlichFallstudienAlsForschungsmethode2003 S.
9\footcite[Vgl. ][]{Eisenhardt, Kathleen M. (1989): 'Building Theories from Case
  Study Research', in: Academy of Management Review, Vol. 14, No. 4, S.
  532-550.}, dubeRigorInformationSystems2003 S. 609).

In dieser Arbeit soll die Entwicklung von Softwarekomplexitätsmetriken
betrachtet werden. Also liegt es auf der Hand, die Quelltexte von
Softwareprojekten als Analyseeinheiten zu betrachten. Die
Komplexitätsmetriken sollen mit insgesamt sechs Softwareprojekten
validiert werden. Bei fünf Projekten handelt es sich um Projekte, die
unternehmensintern im Kundenauftrag entwickelt werden bzw. wurden. Bei
einem weiteren Projekt handelt es sich um eine Softwarelösung, die
bereits von über 100 Tausend Organisationen verwendet wird\footcite[Vgl. ][]{GitLabGitLab}.
Es lässt sich also sagen, dass der wirtschaftliche bzw. praktische Bezug
aller Projekte sichergestellt ist. Aufgrund der Positionierung der
ersten fünf Projekte als unternehmensinterne Entwicklungen ist hier des
weiteren der weitgehend unlimitierte und vor allem unverfälschte Zugang
zu Daten der Projekte sichergestellt. Auch bei dem externen Projekt
besteht ein hinreichender Datenzugriff. Somit sind alle zuvor
aufgezeigten Voraussetzungen zur Auswahl der Fälle einer Fallstudie bei
den hier behandelten sechs Fällen bedient.

Im Sinne einer klaren Beschreibung der Fälle sollen einige Metadaten zu
den Fällen gesammelt werden. Die Art der abgefragten Informationen
richtet sich nach einer Arbeit von Sato et al (Sato et al 2006:49f) und
mehreren Befragungen von Experten in dem Unternehmen. Insgesamt konnten
so neun Datenpunkte identifiziert werden.

\begin{itemize}
    \item
      Der \emph{Name} des Projektes: Hier musste teilweise aus Gründen der
      Datensicherheit ein Pseudonym verwendet werden.
    \item
      Eine \emph{Beschreibung}: Diese Beschreibung sollte den
      Verwendungszweck und eine grobe betriebliche Motivation des Projektes
      beinhalten.
    \item
      \emph{Entwicklungsmethode}: Ob das Projekt agil nach Scrum oder einer
      anderen Entwicklungsmethode entwickelt wird, ist ebenfalls relevant
      für die Arbeit.
    \item
      \emph{Programmiersprache}: Welche Programmiersprachen in dem Projekt
      hauptsächlich verwendet werden.
    \item
      \emph{Offshore oder Onshore}: Ob die Entwicklung des Projektes in ein
      anderes Land (offshore) verlegt wurde.
    \item
      \emph{Team Lokation}: Der Hauptsitz des Entwicklungsteams.
    \item
      \emph{Kunden Lokation}: Der Hauptsitz des Kunden.
    \item
      \emph{Erfahrung der Entwickler}: Wie viel Berufserfahrung die
      Entwickler aufweisen.
    \item
      \emph{Erfahrung der Entwickler mit dem Projekt}: Wie viel Erfahrung
      die Entwickler bereits in dem speziellen Projekt haben.
    \end{itemize}

Die Datenpunkte Name, Beschreibung, Entwicklungsmethode, Sprache, Team
Lokation und Kunden Lokation wurden sowohl von Sato et al 2006 und von
Experten innerhalb des Unternehmens als wichtig erachtet. Die Relevanz
der weiteren Punkte begründet sich allein aus unternehmensinternen
Expertenmeinungen.

\section{Datensammlung}\label{datensammlung}

Zu den ausgewählten Quellen werden nun Daten erfasst. Das Ziel der
Datenerfassung ist das Schaffen einer belastbaren Basis, auf der eine
spätere Belegung oder Widerlegung der Hypothese geschehen kann. Für die
Zuverlässigkeit und Validität der getätigten Aussagen sei eine genaue
Beschreibung der Datenquellen unerlässlich\footcite[Vgl. ][]{(dubeRigorInformationSystems2003
  S. 612)} \footcite[Vgl. ][]{1. (p. 381) (dubeRigorInformationSystems2003 S.
  614)} \footcite[Vgl. ][]{Benbasat et al.'s (1987)}. Die Auswahl und
Beschreibung der Datenquellen wird in verschiedenster Literatur
umfassend erläutert\footcite[Vgl. ][]{(gothlichFallstudienAlsForschungsmethode2003
  S. 10); Girtler, Roland (2001): Methoden der Feldforschung, 4., völlig
  neu bearbeitete Auflage, Wien, Köln, Weimar; Lueger, Manfred (2000):
  Grundlagen qualitativer Feldforschung, Me-thodologie, Organisierung,
  Materialanalyse, Wien; Piore, Michael J. (1979): 'Qualitative Research
  Techniques in Econom-ics', in: Administrative Science Quarterly, Vol.
  24 (1979), No. 4, S. 560-569.}.

  Für die Auswahl der Datenquellen kommen verschiedene Quellenformate in
Frage: Im Generellen seien Dokumente jeglicher Art als Quelle
zulässig\footcite[Vgl. ][]{(gothlichFallstudienAlsForschungsmethode2003 S.10)}.
Auch Archivdaten spielen eine wichtige Rolle. So wurde in einer
Metastudie ermittelt, dass in ca. 64\% aller beobachteten Fallstudien
Archivdaten als Quelle eingesetzt wurden\footcite[Vgl. ][]{(dubeRigorInformationSystems2003
  S. 614)}. Hier seien insbesondere Zeitreihendaten von Interesse
\footcite[Vgl. ][]{i. (Benbasatet al. 1987; Eisenhardt1989).
  (dubeRigorInformationSystems2003 S. 612)}. Weiter seien Interviews,
Beobachtungen und Befragungen zulässige Datenquellen\footcite[Vgl. ][]{(Benbasatet
  al. 1987; Eisenhardt1989). (dubeRigorIn-formationSys-tems2003 S. 612),
  Yin(1994)}. Auch sämtliche andere Artefakte der Fälle, sowohl in
physischer als auch in digitaler Form können zu der Untersuchung
hinzugezogen werden\footcite[Vgl. ][]{a.
  (gothlichFallstudienAlsForschungsmethode2003 S.10);
  (dubeRigorInformationSystems2003 S. 612); Yin(1994)}.

Zu diesen Primärdaten könnten auch noch Sekundärdaten, wie z.B.
Interpretationen, Zusammenfassungen, Textanalysen, Statistiken und
Berichte hinzugezogen werden
(gothlichFallstudienAlsForschungsmethode2003 S. 11).

Für eine möglichst valide Endaussage sei eine Kombination der Quellen
besonders sinnvoll (gothlichFallstudienAlsForschungsmethode2003 S. 10).
Diese Verwendung mehrerer Quellen und auch mehrerer Forschungsmethoden
kann als methodologische Triangulation verstanden werden
(gothlichFallstudienAlsForschungsmethode2003 S. 10).

Bei allen Quellen sei es besonders wichtig, Daten ähnlich wie nach den
„Grundsätzen ordnungsgemäßer Buchführung`` in einer geordneten Form
lückenlos und unverfälscht aufzubewahren, sodass eine spätere Prüfung
der Schlussfolgerungen möglich sei
(gothlichFallstudienAlsForschungsmethode2003 S. 11).

Im Sinne dieser Empfehlungen zur Auswahl der Quellen wird auch die
Auswahl der Quellen in dieser Arbeit geplant. Es werden zwei
Primärquellen berücksichtigt. Zum einen werden Dokumente aus der
Projektmanagementsoftware der Projekte bezogen. Zum anderen werden
Quelltextartifakte aus der Versionsverwaltung analysiert. Zusätzlich zu
diesen Primärquellen wird in jedem Projekt ein Abschlussinterview mit
einer strukturierten Befragung durchgeführt. Diese Abschlussinterviews
sollen die Interpretation der Primärdaten stützten und die
ordnungsgemäße Sammlung dieser validieren.

\section{Verbindung von Daten und Hypothesen}\label{verbindung-von-daten-und-hypothesen}

Die eigentliche Beantwortung der Problemstellung bedarf einer
Ausrichtung der Daten der Analyseeinheiten (Punkt 3 Units of analysis)
als Reflektion der initialen Hypothese (yinCaseStudyResearch2014 S. 78,
gothlichFallstudienAlsForschungsmethode2003 S. 11). Diese Reflektion
bzw. Verbindung kann mit verschiedenen Analyseverfahren hergestellt
werden. Dabei stehen unter anderem Mustervergleiche, Erklärungsgebilde,
Zeitserienanalysen, logische Modelle und fallübergreifende Synthesen zur
Verfügung (yinCaseStudyResearch2014 S. 78,
gothlichFallstudienAlsForschungsmethode2003 S. 11). Wenn bekannt ist,
dass einige oder alle Hypothesen eine Entwicklung über Zeit betrachten,
könne zum Beispiel eine Zeitserienbetrachtung sinnvoll sein
(yinCaseStudyResearch2014 S. 78, 225).

In dieser Arbeit wird die Komplexität der Software in Projekten
betrachtet (siehe Unit of Analysis). Dabei sollen die
Komplexitätsmetriken mit Aufwandsabschätzungen verglichen werden. Dieser
Vergleich bzw. diese Verbindung ist in Abbildung .. dargestellt und wird
im Folgenden erläutert.

GRAFIK

Für jedes Softwareprojekt sind Daten zur Codekomplexität und zu den
Aufwandsabschätzungen vorhanden.

Die Codekomplexität kann aus dem Sourcecode der Software berechnet
werden (siehe Analysesoftware). Der Sourcecode aller Projekte wird
jeweils in einer Versionsverwaltungssoftware verwaltet. Mit einer
solchen Versionsverwaltungssoftware lassen sich jegliche Änderungen am
Code historisch nachvollziehen. Gleichzeitig kann so auch für jeden
Änderungszeitpunkt in retrospektive der Sourcecode rekonstruiert
werden\footcite[Vgl. ][]{Quelle Git}. Also lässt sich über die komplette
Entwicklungsgeschichte der Software hinweg rekonstruieren, wie der
Sourcecode zu dem jeweiligen Zeitpunkt ausgesehen haben muss. Aus der
Historie des Sourcecodes lässt sich auch eine Historie der
Sourcecodekomplexität berechnen. Dabei wird zu jedem Zeitpunkt in
Retrospektive der Sourcecode hinsichtlich seiner Komplexität analysiert.
Auf diese Weise lässt sich eine Zeitserie der Komplexität des
Sourcecodes über die Entwicklungsgeschichte hinweg rekonstruieren. Auf
der linken Seite von Abbildung .. ist beispielhaft eine solche Serie
über drei Zeitpunkte hinweg skizziert. In realen Projekten lassen sich
zwischen 1000 und 50.000 distinktive Zeitpunkte rekonstruieren. Die
Abbildung hier dient beispielhaft dazu, den Anstieg der Komplexität der
Software zu veranschaulichen.

Als zweite Vergleichsgröße sollen Aufwandsabschätzungen dienen. Auch
hier lässt sich eine Zeitserie aufbauen. Alle Projekte werden nach der
agilen Scrum Methodik verwaltet (Verweis zu Kapitel 3). Diese Methodik
sieht vor, im Voraus zu der Implementierung eines Softwarefeatures den
Umfang bzw. Implementierungsaufwand dieses abzuschätzen. Diese
Abschätzungen werden numerisch in Form von sog. Storypoints abgegeben
und in einer Projektmanagementsoftware (wie z.B. Jira vermerkt). Auch
wird zusammen mit der Aufwandsabschätzung unter anderem der Anfangs- und
der Endzeitpunkt der Implementierung vermerkt. Diese Daten sind bei
allen betrachteten Projekten über die komplette Entwicklungsdauer hinweg
verfügbar. Es lässt sich also rekonstruieren, zu welchem Zeitpunkt
welcher geschätzte Aufwand in die Software geflossen ist. Die
Kombination der Komplexitätsabschätzung zusammen mit ihrem Zeitpunkt
ergibt ebenfalls eine Zeitreihe. Diese ist in Abbildung .. beispielhaft
auf der linken Seite skizziert. Die Entwicklung von Features in einer
Software erfolgt konsekutiv und konstruktiv. Das heißt, wird an
Zeitpunkt 1 Feature 1 und an Zeitpunkt 2 Feature 2 entwickelt, besteht
die Software zu Zeitpunkt 2 aus den Features 1 und 2. Dieser
Sachzusammenhang wird ebenfalls in Abbildung .. skizziert. Der
insgesamte, abgeschätzte Funktionsumfang der Software zu einem Zeitpunkt
X besteht also aus allen Features, die zum Zeitpunkt X entwickelt
wurden, addiert zu den summierten Features, welche bereits vor Zeitpunkt
X entwickelt wurden. Die Aufwandsabschätzungen der Features (Stories)
sind zu den distinktiven Zeitpunkten in Form von Storypoints verfügbar.
Also lässt sich durch die Kumulierung aller Aufwandsabschätzungen
(Storypoints) bis zu diesem Zeitpunkt der abgeschätzte Funktionsumfang
der Software zu diesem Zeitpunkt ableiten. Dieser Sachzusammenhang ist
auf der rechten Seite von Abbildung .. dargestellt.

Zusammengefasst konnten bis hierhin zwei Zeitreihen konstruiert werden.
Einmal auf der linken Seite von Abbildung .. die berechnete
Codekomplexität der Software und dann auf der rechten Seite die
abgeschätzte Komplexität der Software. Beide Werte liegen zeitdistinktiv
zu einer Vielzahl von Zeitpunkten vor. Betrachtet man nun einen
einzelnen Zeitpunkt, sollte laut der anfänglichen Hypothese der
abgeschätzte Gesamtaufwand der Software der berechneten Codekomplexität
entsprechen. Zur Validierung der Hypothese lässt sich eine Korrelation
mathematisch berechnen. Somit ist über die Korrelation der beiden
Zeitreihen eine Verbindung der Fälle (Units of Analysis) über die
gesammelten Daten mit der anfänglichen Hypothese hergestellt.

\section{Kriterien zur Interpretation der Daten}\label{kriterien-zur-interpretation-der-daten}

Die in 4. erläuterte Korrelation der Aufwandsabschätzungen mit den
Codekomplexitätsmetriken soll in einem finalen Schritt interpretiert
werden. Im Sinne einer typischen Fallstudienforschung soll dabei eine
Erklärung für die aufgetretenen Phänomene gefunden werden
(gothlichFallstudienAlsForschungsmethode2003 S. 11?).

In der einschlägigen Literatur über die Durchführung von Fallstudien
finden sich verschiedene Techniken zur Interpretation der gesammelten
Daten. Darunter sind unter anderen Techniken der statistischen Analyse
(yinCaseStudyResearch2014 S. 79,
gothlichFallstudienAlsForschungsmethode2003 S. 11??) sowie Interviews
(gothlichFallstudienAlsForschungsmethode2003 S. 11?). Mit diesen beiden
Techniken werden die Analyseergebnisse aus 4.2.4 zunächst interpretiert
und dann validiert.

Im Sinne einer objektiven Vergleichbarkeit erscheint eine mathematische
Interpretation der Daten mit Methoden der Statistik sinnvoll. Auch in
verwandten Arbeiten wird mit mathematischen Verfahren gearbeitet. Diese
Arbeit strebt einen Nachweis einer Korrelation zwischen zwei Zeitreihen
an. Demnach erscheint es sinnvoll, Korrelationsmaße für den Vergleich
von Zeitreihen anzuwenden. In verwandten Arbeiten kommen unter anderem
die Pearson-Produkt-Moment-Korrelation, der Kendall
Rangkorrelationskoeffizient sowie der Spearman
Rangkorrelationskoeffizient zum Einsatz\footcite[Vgl. ][]{(Jay et al 2009:140),
  hindleReadingLinesUsing2009}.

Die Pearson-Produkt-Moment Korrelation ist ein parametrisches Verfahren
zur Berechnung des bivariaten Zusammenhangs zwischen zwei Größen. Der
Korrelationskoeffizient kann Werte zwischen -1 und 1 annehmen. Bei einem
Wert von +1 besteht ein vollständig positiver linearer Zusammenhang. Bei
einem Wert von -1 ein vollständig negativer. Das Korrelationsmaß wurde
erstmals von Sir Francis Galton verwendet und von Karl Pearson zuerst
formal-mathematisch begründet\footcite[Vgl. ][]{brucklerGeschichteMathematikKompakt2018
S. 116}.

Die Ergebnisse können über ihren Betrag grob eingeordnet
werden\footcite[Vgl. ][]{fahrmeirStatistikWegZur2016 S. 130}:

\begin{itemize}
\item
    „schwache Korrelation`` r <= 0.5
\item
    „mittlere Korrelation`` 0.5 <= r <= 0.8
\item
    „starke Korrelation`` 0.8 <= r.
\end{itemize}

Ein weiterer Korrelationskoeffizient ist der Kendall
Rangkorrelationskoeffizient. Zur Berechnung des Koeffizienten der die
Ordnungsrelationen aller möglicher Paare der beobachteten Merkmalswerte
für die zwei Merkmale verglichen. Im Vergleich zu Pearson ist dieser
Koeffizient robuster gegenüber Ausreißern und unabhängig von der
metrischen Skalierung der Daten\footcite[Vgl. ][]{fahrmeirStatistikWegZur2016 S.
  137ff}. Die Ergebnisse sind jedoch gleich zu interpretieren.

Ein dritter Korrelationskoeffizient ist der Spearman
Rangkorrelationskoeffizient. Dieser wird ähnlich wie der
Rangkorrelationskoeffizient nach Kendal berechnet, zieht jedoch
zusätzlich zu den Unterschieden zwischen den Rängen noch die Differenz
zwischen den Rängen hinzu\footcite[Vgl. ][]{fahrmeirStatistikWegZur2016 S. 133f}.
Auch hier sind die Ergebnisse gleich wie bei den anderen beiden
Koeffizienten zu interpretieren.

Zur Validierung dieser Interpretation wird zusätzlich in jedem der
Projekte ein Interview mit den Key-Stakeholdern durchgeführt. Das
Interview verfolgt dabei drei Ziele. Zunächst werden generelle
Informationen zu dem Projekt aufgenommen. Dann werden die Stakeholder
befragt, ob ihnen offensichtliche Messfehler in den Projekten auffallen.
In einem letzten Schritt werden in einer offenen Diskussion interessante
Punkte in den Arbeitsergebnissen identifiziert. Diese Interessenspunkte
werden dann versucht zu erklären.

Während bei vielen Forschungsmethoden die genaue Transkription und
Analyse von wenig strukturierten Interviews (z.B. durch die
Inhaltsanalyse nach Mayring) im Fokus steht, wird bei dieser Arbeit ein
strukturiertes Vorgehen bei der Durchführung angestrebt, um die
Interviews so durch logische Schlüsse analysieren zu können. Dieses
Vorgehen wird auch von verbreiteter Literatur zum generellen Vorgehen
bei der Fallstudienforschung unterstützt
(gothlichFallstudienAlsForschungsmethode2003 S. 12). Dabei birge eine zu
detailreiche Analyse der Interviews die Gefahr der Überinterpretation.
Wichtiger sei es, die Ergebnisse der Interviews kritisch zu reflektieren
(gothlichFallstudienAlsForschungsmethode2003 S. 12). Für ein möglichst
strukturiertes Vorgehen in den Interviews wurde vorab mit mehreren
Experten aus dem Feld der Softwareentwicklung ein Leitfaden in Form
eines Fragebogens erstellt. Dieser kann Anhang .. entnommen werden.

Die so erlangten Ergebnisse der einzelnen Fälle sollen auch über die
Fälle hinweg verglichen und interpretiert werden. Kommen bei einer
Fallstudie mehrere Fälle zu dem gleichen Ergebnis spricht man von einer
Replikation (gothlichFallstudienAlsForschungsmethode2003 S. 11). Wenn
bei allen Fällen eine Korrelation von Aufwandsabschätzungen und
Komplexitätsmaßen festgestellt werden kann, ließe dies auf eine
allgemeine Regel schließen. Aber auch eine sog. theoretische Replikation
ist vorstellbar. Dabei kommen verschiedene Fallstudien zu verschiedenen
Ergebnissen, welche aber theorieseitig erklärbar sind
(gothlichFallstudienAlsForschungsmethode2003 S.11).

\section{Implementierung einer
Untersuchungssoftware}\label{implementierung-einer-untersuchungssoftware}

Für die Untersuchung der Softwarekomplexitätsmetriken müssen einige
komplexe Berechnungen durchgeführt werden. Die Datenbasis dieser
Untersuchung umfasst mehrere zehntausend Datenpunkte. Eine manuelle
Verarbeitung dieser Daten erscheint als wenig ökonomisch. Also bedarf es
einer automatisierten Verarbeitung der Daten. Die Verarbeitung der Daten
soll mit einer Software erfolgen. In einer umfangreichen Suche konnte
lediglich die Software SonarQube einen wesentlichen Anteil der
Anforderungen an ein solches Produkt abdecken. Bei dem Ansetzen einer
ersten SonarQube Instanz stellte sich jedoch heraus, dass die Import-
und Exportfunktionen von SonarQube unzureichend für die hier angestrebte
Untersuchung sind. Da außer SonarQube keine passende Software gefunden
werden konnte, liegt es auf der Hand, eine eigene Untersuchungssoftware
zu implementieren.

\subsection{Anforderungen an die Untersuchungssoftware}\label{Anforderungen-an-die-Untersuchungssoftware}

Die Berechnung der Korrelation von Codekomplexitätsmetriken und
Komplexitätsabschätzungen stellt eine Reihe an Anforderungen an die
Untersuchungssoftware. Zunächst werden die Anforderungen in einem
Überblick zusammengefasst und dann im späteren Verlauf dieses Kapitels
genauer betrachtet.

Als Eingabe werden zwei Datenströme vorgesehen. Auf der einen Seite die
Entwicklungshistorie der Software und auf der anderen Seite die
historischen Komplexitätsabschätzungen für die Features der Software.
Besonders zu bemerken ist, dass beide Datenströme irreguläre
Zeitintervalle und unterschiedliche Wertebereiche haben.

Funktionale Anforderungen:

\begin{itemize}
\item
    Verarbeitung passender Eingabedatenströme

    \begin{itemize}
    \item
    Als ersten Eingabeparameter sind die Komplexitätsabschätzungen
    vorgesehen. Wie in 4.2.4 beschrieben, liegen bei allen behandelten
    Projekten für jedes entwickelte Feature Komplexitätsabschätzungen in
    einer Projektmanagementsoftware vor. Aus dieser Software können die
    Schätzungen in Tabellenform als CSV-Datei exportiert werden. Der
    Export soll unverarbeitet als ersten Eingabeparameter in die
    Untersuchungssoftware eingelesen werden. Dabei muss diese Tabelle
    zwei Spalten beinhalten. Ein Datensatz in dieser Tabelle muss
    jeweils einen Zeitstempel mit dem Entwicklungszeitpunkt des
    Features, sowie die Komplexitätsabschätzung als Anzahl von
    Storypoints beinhalten.
    \item
    Als zweiten Eingabeparameter soll die Versionshistorie der Software
    dienen. Bei allen zu untersuchenden Projekten liegt die
    Versionshistorie in Form eines sog. Repositories der
    Versionsverwaltungssoftware Git\footnote{Git erklären} vor. Diese
    Versionshistoriendaten sollen von der Untersuchungssoftware
    verarbeitet werden können.
    \end{itemize}
\item
    Generierung zielgerichteter Ausgangsdatenströme

    \begin{itemize}
    \item
    Zunächst wird als erster Ausgangsdatenstrom vorgesehen, dass in
    einem regulären Zeitintervall von einer Stunde über die komplette
    Entwicklungshistorie der Software hinweg jeweils die kumulierten
    Storypoints, sowie der jeweilige Stand der Komplexitätsmetriken
    vorliegen. Ist zu dem spezifischen Zeitpunkt in dem Zeitintervall
    kein Wert vorhanden, soll dieser aus den Nachbarwerten linear
    interpoliert werden. Als Ausgabeformat wird eine Tabelle in Form
    einer CSV Datei vorgesehen:
    \end{itemize}
\end{itemize}

TODO!TABLE

\begin{itemize}
\item
    Als zweiter Ausgabeparameter wird ein Graph des Verlaufes aller
    Maßzahlen im PNG- und PDF-Format vorgesehen.
\item
    In einem dritten Ausgabeparameter soll die Korrelation der Maßzahlen
    mit den Storypoints mit verschiedenen Korrelationsmaßen dargestellt
    werden. Als Korrelationsmaße ist die
    Pearson-Produkt-Moment-Korrelation, die Kendall Rangkorrelation und
    die Spearman Rangkorrelation vorgesehen. In dem dritten
    Ausgabeparameter sollen diese Koeffizienten als Tabelle im CSV-Format
    für jedes Komplexitätsmaß angegeben werden.
\item
    Als vierten Ausgabeparameter ist eine Heatmap\footnote{Definition
    Heatmap} mit den Korrelationskoeffizienten auf der x-Achse und den
    Komplexitätsmetriken auf der y-Achse vorgesehen.
\end{itemize}

\begin{itemize}
\item
    Für eine möglichst breite Anwendbarkeit soll die Software unter
    Unix-basierten Betriebssystemen, wie Linux und Mac ausführbar sein.
    Außerdem können so eine Vielzahl von Open-Source Bibliotheken
    angebunden werden.
\end{itemize}

Nicht-funktionale Anforderungen

\begin{itemize}
\item
    Der Fokus dieser Arbeit liegt nicht auf der Implementierung der
    Untersuchungssoftware, sondern auf den mit ihr erzielten Ergebnissen.
    Für eine möglichst ökonomische Beantwortung der Forschungsfragen
    erscheint es also von Interesse, die Untersuchungssoftware mit einem
    möglichst geringen Implementierungsaufwand umzusetzen.
\item
    In dieser Arbeit sollen einer Vielzahl von teils sehr umfangreichen
    Projekten analysiert werden. Um eine zeitgerechte Abwicklung der
    Analysen zu gewährleisten, soll der Untersuchungssoftware möglichst
    effizient mit Rechenressourcen umgehen.

    \begin{enumerate}
    \def\labelenumi{\arabic{enumi}.}
    \item
    \protect\hypertarget{_Toc102600182}{}{}Aufbau der
    Untersuchungssoftware
    \end{enumerate}
\end{itemize}

Zur Begegnung der zuvor genannten Anforderungen wird eine
Untersuchungssoftware skizziert. Die Transformation der Eingabedaten zu
den Ausgabedaten bedarf einer Vielzahl von komplexen Operationen. Für
eine ökonomische Umsetzung der Software (siehe Anforderung ..) wird
angestrebt, einen möglichst großen Anteil des Funktionsumfangs der
Software durch eine Wiederverwendung von bereits vorhandenen
Softwaremodulen zu realisieren. Im Zuge dessen wird auch eine
Implementierung über mehrere Programmiersprachen hinweg in Kauf
genommen. Eine Skizze der so erreichten Implementierung wird in
Abbildung TODO! dargestellt und im Folgenden erklärt.

TODO!GRAFIK

Der Untersuchungssoftware ist in drei Teile unterteilt. Zunächst liest
der Code Parser (Zahl) die Versionshistorie der Software ein und
berechnet für jeden Änderungsstand (Commit) die Codekomplexitätsmetriken
der Software. Die so berechneten Metriken werden in einem zweiten Modul
(2) mit den Daten aus dem Projektmanagement Tool zusammengefügt,
verarbeitet und zu den Ausgabedaten aufbereitet.

\textbf{Der Code Parser}

Der Code Parser ist dafür verantwortlich, die einzelnen
Entwicklungsstände der Software aus der Versionshistorie zu extrahieren
und für jeden Entwicklungsstand die Komplexitätsmetriken zu berechnen.
Als Eingabeparameter ist ein Repository der Versionsverwaltungssoftware
Git vorgesehen. Als Ausgabeparameter sind die
Softwarekomplexitätsmetriken in Tabellenform als CSV Datei vorgesehen
(Tabelle ..). Der Code Parser besteht aus einigen generellen
Kontrollstrukturen und speziellen Analysemodulen, welche die
Komplexitätsmetriken für verschiedene Programmiersprachen in den
Projekten berechnen. Für die Analysemodule wird auf bereits verbreitete
Bibliotheken zurückgegriffen, da eine Eigenentwicklung der Module zu
aufwändig wäre. Bei der Auswahl der Bibliotheken wurde darauf geachtet,
möglichst gut gewartete Software zu verwenden, um so ein möglichst
genaues Ergebnis zu erzielen. Eine Tabelle mit allen betrachteten
Bibliotheken findet sich in Anhang .. . Für ein noch genaueres Ergebnis
werden manche Metriken doppelt berechnet. In dem
Zahlenverarbeitungsmodul wird später das jeweils genauste Ergebnis
ausgewählt. Der Ablauf des Code Parsers wird im folgenden erläutert:

\begin{enumerate}
\def\labelenumi{\arabic{enumi}.}
\item
    \textbf{Commit Getter}: Zunächst werden durch den Commit Getter alle
    Entwicklungsstände der Software mit ihrem jeweiligen Commit-Hash und
    dem Zeitstempel des Eincheckzeitpunktes in eine CSV Datei geschrieben
\item
    \textbf{Commit Runner:} Der Commit Runner liest diese CSV-Datei,
    iteriert über alle Commits und führt für jeden Commit der Software
    alle Analysemodule aus. Die Analysemodule schreiben jeweils die
    Ergebnisse der Analyse des jeweiligen Commits in eine CSV-Datei mit
    dem Commit-Hash als Dateinamen.

    \begin{enumerate}
    \def\labelenumii{\alph{enumii}.}
    \item
    \textbf{Lizard}: Das Python Modul Lizard berechnet für alle
    Programmiersprachen die zyklomatische Komplexität\footcite[Vgl. ][]{LizardPythonModule}.
    \item
    \textbf{Plato}: Das NodeJS Commandline-Tool ES6-Plato\footcite[Vgl. ][]{Es6platoLibMaster}
    kann lediglich JavaScript Quelltexte analysieren. Für die JavaScript
    Teile des zu untersuchenden Projektes berechnet Plato noch einmal
    die zyklomatische Komplexität, den Halstead Aufwand und die Anzahl
    logischer Codezeilen
    \item
    \textbf{Indentation Complexity}: Mit dem Rust Commandline-Tool
    Complexity der Firma thoughtbot, inc\footcite[Vgl. ][]{Complexity2022} wird
    die Einrückungskomplexität für alle Programmiersprachen berechnet
    \item
    \textbf{MulitMetric}: Das MultiMetric Python Modul\footcite[Vgl. ][]{weihmanMultimetric2021}
    ist zuletzt in der Lage für sämtliche hier behandelte
    Programmiersprachen die zyklomatische Komplexität, den Halstead
    Aufwand und die Anzahl logischer Codezeilen zu berechnen.
    \end{enumerate}
\item
    \textbf{Consolidator:} Als letzter Schritt werden im Consolidator die
    Ergebnisse aller Analysemodule für alle Entwicklungsstände in einer
    Tabelle konsolidiert und als eine CSV-Datei ausgegeben. Dafür liest
    der Consolidator zunächst die Commit-Hashes und Zeitstempel aus der,
    von dem Commit-Getter generierten CSV Datei. Für jeden Commit wird für
    jedes Analysemodul nach einer Ergebnis-CSV-Datei gesucht. Für jede
    Komplexitätsmetrik wird eine Summe\footcite[Vgl. ][]{(Hoffmann 2013:276)} aller
    Dateien in dem Analyseergebnis gebildet. Die Summen der
    Komplexitätsmetriken der einzelnen Entwicklungsstände werden in einer
    CSV-Datei ausgegeben.
\end{enumerate}

Die Ausgabe des Consolidators ist gleichzeitig auch die Ausgabe des Code
Parsers. Als solche wird sie in Tabelle .. beschrieben.

TODO!TABELLE

\subsection{Zahlenverarbeitungsmodul}\label{Zahlenverarbeitungsmodul}

Nachdem in dem Code Parser die Komplexitätsberechnungen stattgefunden
haben, sollen diese nun in dem Zahlenverarbeitungsmodul zusammen mit den
Aufwandsabschätzungen zu dem Analyseergebnis verarbeitet werden.

Das Zahlenverarbeitungsmodul verarbeitet zwei Eingabeparameter. Zunächst
wird das Resultat des Code-Parser-Moduls (siehe Tabelle ..) gelesen. Wie
in 4.2.4 beschrieben ist als zweiter Eingabeparameter der Export von
einer Projektmanagementsoftware vorgesehen. In diesem Export sollte eine
CSV-Tabelle mit Datensätzen von den Features der Software enthalten
sein. Jeder Datensatz beinhaltet mindestens zwei Datenpunkte, einen
Zeitstempel (date) und eine Komplexitätsabschätzung (storypoints). Eine
Beispiel-Eingabetabelle kann Tabelle .. entnommen werden.

TODO!TABELLE

Die Operationen innerhalb dieses Moduls befassen sich primär mit der
Analyse und Verarbeitung großer Datenmengen. Das Wissenschaftsfeld Data
Science behandelt ebenfalls genau diese Operationen. In Data Science ist
die Programmiersprache Python weit verbreitet\footcite[Vgl. ][]{Quelle fehlt}.
Also liegt es auf der Hand, auch für dieses Modul die Programmiersprache
Python zu verwenden. Für Python existiert eine Vielzahl von Bibliotheken
zur Verarbeitung von großen Datenmengen. Für dieses Modul wird primär
die Datenanalyse- und Manipulationsbibliothek Pandas
verwendet\footcite[Vgl. ][]{\url{https://pandas.pydata.org/}}. Zusätzlich wird
die Bibliothek Matplotlib zum Erstellen von Graphen zur Hilfe gezogen.
In der Analyse der Daten erscheint es zusätzlich sinnvoll,
Zwischenergebnisse interaktiv zu verifizieren. So können Analysefehler
bereits früh erkannt und behoben werden. Die Software Jupyter ermöglicht
genau solch eine interaktive Programmausführung und wird somit zur
Ausführung des Moduls herangezogen.

Der Aufbau und die Implementierung des Zahlenverarbeitungsmoduls werden
nun erläutert. Das Modul lässt sich in vier Teile aufteilen:

\begin{enumerate}
\def\labelenumi{\arabic{enumi}.}
\item
    Verarbeiten der Codekomplexitätsmessungen zu einer Zeitreihe
\item
    Verarbeiten der Aufwandsabschätzungen zu einer Zeitreihe
\item
    Zusammenführen beider Zeitreihen
\item
    Analysieren der Zeitreihen
\end{enumerate}

Im ersten Schritt zur Verarbeitung der Codekomplexitätsmessungen werden
diese zunächst als Pandas Dataframe in das Modul geladen (Zeile 1). Bis
auf den Zeitstempel können alle Datentypen automatisch erkannt werden.
Der Datentyp des Zeitstempels wird manuell gesetzt (Zeile 2)

\lstdefinestyle{pythonStyle}{
%  basicstyle=\fontsize{7}{8}\selectfont\ttfamily
  language=Python,
  numbers=left,
  stepnumber=1,
  numbersep=10pt,
  breaklines=true,
  tabsize=4,
  showspaces=false,
  showstringspaces=false
}

\lstset{style=pythonStyle}
\lstinputlisting{includes/4 research setup/code-analyser-cells/cell-4.py}

Dann wird der Zeitstempel als Zeitindex gesetzt (Zelle 5 Zeile 1-3)

\lstset{style=pythonStyle}
\lstinputlisting{includes/4 research setup/code-analyser-cells/cell-5.py}

Als zweiter Schritt ist eine Untersuchung der gelesenen Daten
vorgesehen. Dabei wird zunächst untersucht, welche Ergebnisse in der
Eingabe vorhanden sind (Zelle 7 Zeile 1-4).

\lstset{style=pythonStyle}
\lstinputlisting{includes/4 research setup/code-analyser-cells/cell-7.py}

Weiter wird eine Übersicht erstellt, zu welchen dieser Ergebnisse valide
Werte vorliegen (Zelle 8 Zeile 1-17). Dabei werden alle Ergebnisse
aussortiert, bei denen mehr als zehn Prozent der Messwerte ungültig
sind.

\lstset{style=pythonStyle}
\lstinputlisting{includes/4 research setup/code-analyser-cells/cell-8.py}

Nun wird für jede zu berechnende Komplexitätsmetrik (Anzahl logischer
Codezeilen, zyklomatische Komplexität, Halstead Aufwand und
Einrückungskomplexität) das beste Ergebnis aus den vorliegenden
Ergebnissen der Analysemodule ausgewählt (Zelle 9). Für diese Auswahl
wurde zuvor in dem Zahlenverarbeitungsmodul ein Zuweisungsobjekt
hinterlegt (Zelle 2). In diesem Zuweisungsobjekt wird für jede Maßzahl
eine Präferenzliste der möglichen Ergebnisse der Analysemodule
hinterlegt. In dieser Zelle wird nun über die zu berechnenden
Komplexitätsmetriken iteriert und für jede Metrik das erste valide
Analyseergebnis aus der Präferenzenliste ausgewählt. Die Endauswahl wird
in einer Variable abgespeichert.

\lstset{style=pythonStyle}
\lstinputlisting{includes/4 research setup/code-analyser-cells/cell-9.py}

Anhand dieser Zuweisungstabelle wird nun ein Dataframe mit den besten
Analyseergebnissen für jede Metrik aufgebaut. Für alle Datenpunkte, bei
denen an dieser Stelle noch kein Wert vorhanden ist, muss davon
ausgegangen werden, dass kein Wert ermittelt werden konnte. Diese werden
mit 0 aufgefüllt (Zelle 10, Zeile 6)

\lstset{style=pythonStyle}
\lstinputlisting{includes/4 research setup/code-analyser-cells/cell-10.py}

Bis zu diesem Punkt konnte nun eine Tabelle mit den Datensätzen der
Komplexitätsmessungen aufgebaut werden. Ein Beispiel ist in Tabelle ..
aufgeführt. Jeder Datensatz beinhaltet einmal den Zeitstempel der
Komplexitätsmessung (timestamp), die Anzahl logischer Codezeilen
(logicalLinesOfCode), die zyklomatische Komplexität
(cyclomaticComplexity), den Aufwand nach Halstead (halsteadEffort) und
die Einrückungskomplexität (indendationComplexity).

TODO!TABELLE

Als nächster Schritt werden die Daten in einen einheitlichen
Wertebereich normalisiert (Zelle 11). Dabei werden die Daten so linear
verschoben und skaliert, dass jeweils der erste Wert jeder Zeitreihe 0
ist und der letzte Wert 1 ist.

\lstset{style=pythonStyle}
\lstinputlisting{includes/4 research setup/code-analyser-cells/cell-11.py}

Zusätzlich soll nun auch das Zeitintervall normalisiert bzw.
regularisiert werden. Dabei werden die Daten zunächst auf ein geringes
Intervall von 60 Sekunden abgetastet. Die fehlenden Werte werden linear
interpoliert. In einem durchschnittlichen Projekt werden so ca. 2
Millionen Datensätze generiert. Diese Anzahl von Datensätzen ist jedoch
zu groß, um sie weiter zu verarbeiten. Für die weitere Verarbeitung wird
die Zeitreihe erneut auf ein Intervall von einer Stunde abgetastet, was
die Anzahl an Datensätzen auf ca. 34 tausend reduziert.

\lstset{style=pythonStyle}
\lstinputlisting{includes/4 research setup/code-analyser-cells/cell-12.py}

Bis zu diesem Punkt wurde für jede Komplexitätsmetrik eine passende
Messreihe ausgewählt und die Werte auf ein einheitliches Zeitintervall,
sowie einen einheitlichen Wertebereich normalisiert. Diese Daten können
in dieser Form zusammen mit den Aufwandsabschätzungen weiterverarbeitet
werden.

Als zweiter Arbeitsschritt befasst sich das Zahlenverarbeitungsmodul mit
der Verarbeitung der Aufwandsabschätzungen. Diese Daten werden ebenfalls
zunächst in einen Pandas Dataframe geladen und bereinigt (Zelle 13).

\lstset{style=pythonStyle}
\lstinputlisting{includes/4 research setup/code-analyser-cells/cell-13.py}

Weiter werden die Aufwandsabschätzungen nach ihrem Zeitpunkt aufsteigend
sortiert (Zelle 14).

\lstset{style=pythonStyle}
\lstinputlisting{includes/4 research setup/code-analyser-cells/cell-14.py}

Weiter werden die relevanten Spalten aus den Daten ausgewählt (Zeile 1),
der Index als Zeitstempel in der UTC Zeitzone gesetzt (Zeile 2 und 3),
die Spalten mit einheitlichen Namen versehen und dann das resultierende
Datenset noch einmal nach Datum aufsteigend sortiert (Zeile 5).

\lstset{style=pythonStyle}
\lstinputlisting{includes/4 research setup/code-analyser-cells/cell-16.py}

Nun werden die Storypoints wie in 4.2.4 beschrieben kumuliert (Zelle
15). Hierzu bietet Pandas die Funktion cumsum an\footcite[Vgl. ][]{PandasDataFrameCumsum}.

\lstset{style=pythonStyle}
\lstinputlisting{includes/4 research setup/code-analyser-cells/cell-17.py}

Dann wird in dem gleichen Verfahren wie bei den Codekomplexitätsmetriken
der Wertebereich der Aufwandsabschätzungen auf einen Bereich zwischen 0
und 1 normalisiert (Zelle 16).

\lstset{style=pythonStyle}
\lstinputlisting{includes/4 research setup/code-analyser-cells/cell-18.py}

Auch das Zeitintervall der Messdaten wird zu einem regulären
Zeitintervall von einer Stunde vereinheitlicht (Zelle 17).

\lstset{style=pythonStyle}
\lstinputlisting{includes/4 research setup/code-analyser-cells/cell-19.py}

Das resultierende Datenset hat nun also genau wie die Codemetriken einen
Wertebereich zwischen null und eins und eine Abtastrate von 60 Sekunden.
Die beiden Zeitreihen sind nun also sowohl in ihrer Skalierung als auch
in ihrer Abtastrate identisch.

Als dritter Schritt des Zahlenverarbeitungsmoduls können die Zeitreihen
der Codekomplexitätsmessungen und der Aufwandsabschätzungen nun
zusammengeführt werden. Das geschieht mit der Merge-Funktion von
Pandas\footcite[Vgl. ][]{PandasDataFrameMergea}. Die Merge-Funktion stellt eine
neue Tabelle mit den Werten der Codekomplexitätsmessungen und der
Aufwandsabschätzungen auf. Durch den „how=inner`` Parameter wird
festgelegt, dass ein Inner-Merge durchgeführt wird. Bei einem
Inner-Merge werden nur Zeitpunkte berücksichtigt, an denen in beiden
Zeitreihen ein Wert verhanden ist.

\lstset{style=pythonStyle}
\lstinputlisting{includes/4 research setup/code-analyser-cells/cell-21.py}

Die zusammengefügten Daten werden nun noch einmal zusammen normalisiert
(Zelle 22)

\lstset{style=pythonStyle}
\lstinputlisting{includes/4 research setup/code-analyser-cells/cell-22.py}

Die normalisierten Daten werden als erster Ausgabeparameter der
Analysesoftware (siehe 4.7.1) als CSV-Datei exportiert.

Weiter sollen die Daten nun analysiert werden. Dazu wird zunächst mit
der Pandas Mean-Funktion\footcite[Vgl. ][]{PandasDataFrameMean} eine Zeitreihe
des arithmetischen Mittels aller Komplexitätsmetriken berechnet (Zelle
24).

\lstset{style=pythonStyle}
\lstinputlisting{includes/4 research setup/code-analyser-cells/cell-24.py}

Nun wird die Differenz dieses arithmetischen Mittels und den
Aufwandsabschätzungen gebildet (Zelle 25). Ist diese Differenz besonders
groß, heißt das, dass die Komplexitätsmessungen an dieser Stelle
besonders stark von den Aufwandsabschätzungen abweichen.

\lstset{style=pythonStyle}
\lstinputlisting{includes/4 research setup/code-analyser-cells/cell-25.py}

Aus den bis jetzt berechneten Daten kann der Korrelationsgraph
aufgestellt werden. Dieser wird als zweiter Ausgabeparameter (siehe 4.7.1) in Form von einer
PDF und einer PNG Datei exportiert. Der Code zur Generierung des Korrelationsgraphen findet sich im Anhang TODO.

Als letzter Schritt können die Korrelationen der Komplexitätsmetriken
mit den Aufwandsabschätzungen berechnet werden (Zelle 23). Das ist in
dem Pandas Modul mit der Corr-Funktion\footcite[Vgl. ][]{PandasDataFrameCorr}
möglich. Die einzelnen Korrelationskoeffizienten werden dann zu einer
Tabelle zusammengefügt (Zeile 13) und als CSV-Datei exportiert. Diese
CSV-Datei stellt den dritten Ausgabeparameter der Analysesoftware dar
(siehe 4.7.1).

Zur Visualisierung der Korrelationskoeffizienten wird zuletzt noch eine
Heatmap als vierter und letzter Ausgabeparameter (siehe 4.7.1) erstellt
und im PNG und PDF Format exportiert.

Zusammengefasst konnten mit der hier erläuterten Software die
Komplexitäten zu den verschiedenen Entwicklungszeitpunkten der Software
berechnet werden und zusammen mit den Aufwandsabschätzungen analysiert
werden.

\section{Grenzen und Probleme}\label{grenzen-und-probleme}

Die hier vorgestellte Analysesoftware unterliegt einer Reihe von
Einschränkungen.

Zunächst ist sie in der Verarbeitung der Eingabeparameter beschränkt.
Die Revisionsgeschichte der Repositories muss in der Git
Versionsverwaltung vorliegen. Git hat laut dem Black Duck Open Hub
Repository bei Open Source Projekten zwar einen Marktanteil von
73\%\footcite[Vgl. ][]{CompareRepositoriesOpen}, jedoch wäre eine Unterstützung
für andere Versionskontrollsysteme auch erstrebenswert. Das
Eingabeformat der Aufwandsabschätzungen ist zwar auf eine CSV-Datei
beschränkt, jedoch ist CSV ein offener Standard und die meisten
Tabellenformate lassen sich in dieses Format konvertieren.

Eine weitere Einschränkung der Analysesoftware liegt in der Verarbeitung
der Daten. Durch die Normalisierung aller Zeitreihen gehen die absoluten
Werte verloren und die Komplexitätsmetriken können nur in Relation
zueinander betrachtet werden. Diese Einschränkung ist notwendig, um die
Werte vergleichbar zu machen.


\chapter{Untersuchung der Softwarekomplexitätsmetriken}\label{untersuchung-der-softwarekomplexituxe4tsmetriken}

Mit der in Kapitel \ref{forschungsaufbau-der-fallstudie} vorgestellten Methodik werden nun sechs
Softwareprojekte als Fälle der Fallstudie untersucht. Dabei wird jedes
Projekt zunächst wie in \ref{analyseeinheit} beschrieben vorgestellt. Dann wird die
Erhebung der Daten beschrieben. Am Ende der Datenerhebung steht für
jedes Projekt ein Diagramm zu Verfügung. Für das erste Projekt wird
dieses Vorgehen beispielhaft ausführlich erklärt. Hier wird auch der
Korrelationsgraph erläutert. Zur Vermeidung von Doppelungen wird in den
weiteren Projekten auf eine umfangreiche Erklärung des Analysevorgehens
verzichtet.

\section{Digital NDA Application}\label{digital-nda-application}

Die „Digital \ac{NDA} Application`` hat das Ziel, das komplette Handling der
Verschwiegenheitserklärungen im \ac{DIL} und bei potenziellen weiteren Kunden
zu digitalisieren. Insbesondere soll die bisher für eine
Verschwiegenheitserklärung nötige Papierunterschrift durch eine digitale
Signatur ersetzt werden. Dazu wird den \ac{DIL}-Besuchern die Möglichkeit
gegeben, ihre Daten schon vorab über eine Webapplikation zu übermitteln,
woraufhin die bereits auf den Besucher angepasste
Verschwiegenheitserklärung vor Ort auf einem Tablet signiert werden
kann\footcite[Vgl. ][]{dxctechnologiesInternesDokumentAufbau2022}.

Das Projekt wurde agil von einem onshore Team aus erfahrenen Entwicklern
in Deutschland entwickelt. Es wurden hauptsächlich JavaScript
(52\%) und TypeScript (30\%)
als Programmiersprachen verwendet. Zusätzlich fielen in der Analyse noch
kleinere Programteile in Gherkin (7\%), PLpgSQL (5\%), sowie \ac{HTML},
Shell, SCSS und der Dockerfile Syntax zu jeweils weniger als 5\% Anteil
auf.

\subsection{Datenerhebung}\label{nda-Datenerhebung}

Wie in Kapitel \ref{datensammlung} beschrieben, sollen für die Fallstudie in jedem
Projekt zunächst verschiedene Daten erhoben werden. Für die Berechnung
der Korrelation zwischen den Softwarekomplexitätsmetriken und den
Aufwandsabschätzungen wird für beide Größen jeweils eine Datenquelle
benötigt. Die Daten zu den Aufwandsabschätzungen sollen aus der
Projektmanagementsoftware des Projektes bezogen werden und die Daten zu
den Softwarekomplexitätsmetriken aus der Versionsverwaltung des
Quelltextes.

In dem Digital \ac{NDA} Application Projekt werden die Aufwandsabschätzungen
in der Projektmanagementsoftware Jira verwaltet. In der
Projektmanagementsoftware finden die Aufwandsabschätzungen entsprechend
der Scrum-Methodik (siehe Kapitel \ref{aufwandsabschuxe4tzungen-agiler-projekte}) im Rahmen von User Storys statt.
In diesen User Storys ist jeweils eine Anzahl an Story Points und ein Datum vermerkt.

TODO Grafik

Wie in Kapitel \ref{datensammlung} beschrieben, sollen für die Fallstudie in jedem
Projekt zunächst verschiedene Daten erhoben werden. Für die Berechnung
der Korrelation zwischen den Softwarekomplexitätsmetriken und den
Aufwandsabschätzungen wird für beide Größen jeweils eine Datenquelle
benötigt. Die Daten zu den Aufwandsabschätzungen sollen aus der
Projektmanagementsoftware des Projektes bezogen werden und die Daten zu
den Softwarekomplexitätsmetriken aus der Versionsverwaltung des
Quelltextes.

In dem Digital \ac{NDA} Application Projekt werden die Aufwandsabschätzungen
in der Projektmanagementsoftware Jira verwaltet. In der
Projektmanagementsoftware finden die Aufwandsabschätzungen entsprechend
der Scrum-Methodik (siehe Kapitel \ref{aufwandsabschuxe4tzungen-agiler-projekte}) im Rahmen von User Storys statt.
Ein Beispiel einer User Story findet sich in Abbildung TODO . In
jeder User Story werden, wie in Kapitel \ref{aufwandsabschuxe4tzungen-agiler-projekte} beschrieben wichtige
Informationen zu der geplanten Funktion der Anwendung hinterlegt. Für
diese Arbeit sind die Felder „Story Points`` (\#1) und „Resolved`` unter
Dates (\#2) von besonderer Relevanz.

TODO Grafik

Aus der Software konnten insgesamt 67 Storys exportiert werden und
dienen wie in \ref{Anforderungen-an-die-Untersuchungssoftware} beschrieben als Input für die Analysesoftware.

Als zweite Datenquelle wird die Versionshistorie des Quelltextes der
DTCNDA Anwendung herangezogen. Diese liegt in Form eines Git
Repositories vor. Aus dem Git Repository konnten 1625 Commits geladen
werden. Die Versionshistorie des Quellcodes wird ebenfalls als
Eingabeparameter für das Analysetool übernommen.

\subsection{Auswertung}\label{nda-Auswertung}

In dieser Arbeit soll die Korrelation zwischen
Softwarekomplexitätsmetriken und Aufwandsabschätzungen beschrieben
werden. Nachdem in \ref{nda-Datenerhebung} für beide Größen Daten gesammelt wurden, sollen
diese nun wie in \ref{verbindung-von-daten-und-hypothesen} beschrieben ausgewertet werden. Hierzu werden die
Daten in die in \ref{implementierung-einer-untersuchungssoftware} beschriebe Analysesoftware geladen und mit ihr
verarbeitet. Es werden zunächst vom Code Parser die
Codekomplexitätsmetriken für die einzelnen Entwicklungsstände der
Software berechnet. Dann wird das Ergebnis dieser Berechnungen in das
Zahlenverarbeitungsmodul geladen. Dort werden die
Codekomplexitätsmetriken zusammen mit den Aufwandsabschätzungen zu einem
einheitlichen Format verarbeitet und in Relation zueinander gesetzt.
Diese Relation der Größen wird mit dem Graph in Abbildung TODO
beschrieben.

TODO Grafik

Abbildung TODO wird im Folgenden als Korrelationsgraph des Projektes
bezeichnet und soll nun erläutert werden. Die Abbildung besteht aus zwei
Diagrammen, welche übereinander angeordnet sind. In dem oberen Diagram
werden alle behandelten Größen in einem gemeinsamen Wertebereich
abgezeichnet. Die Story Points sind mit einer dickeren, schwarzen Linie
dargestellt. Die restlichen Linien beschreiben die
Codekomplexitätsmetriken. Die einzelnen Farbzuweisungen können der
Legende im oberen linken Rand der Abbildung entnommen werden. Das zweite
Diagramm in der Abbildung beschreibt die Differenz zwischen den Story
Points und dem arithmetischen Mittel aller Codekomplexitätsmetriken. Es
soll anzeigen, wo die Werte der beiden Größen besonders stark
voneinander abweichen. Der Wertebereich dieses zweiten Diagramms ist
analog zu dem Wertebereich des ersten Diagramms. Das Format dieser
Abbildung wird über alle folgenden Projekte hinweg konstant gehalten.

In dem Korrelationsgraph sind die einzelnen Zeitreihen über den
Projektzeitraum von Oktober 2018 bis November 2019 abgezeichnet. Alle
Zeitreihen folgen einer ähnlichen Linie. Diese könnte näherungsweise als
logarithmisch beschrieben werden. Nach einer stärkeren Steigung zum
Anfang des Projektes flacht die Steigung aller Größen zum Ende des
Projektes zunehmend ab. Dieser Verlauf konnte durch eine Verschiebung in
dem Ziel der Softwareentwicklung begründet werden. So wurde die Software
zum Ende hin immer langsamer weiterentwickelt. In dem Abschlussinterview
konnten die sprungartigen Anstiege der Aufwandsabschätzungen (1) durch
die Sprint-Intervalle erklärt werden. So werden die Storys vermehrt zum
Sprint-Ende als fertig markiert (Analysis Table Line 20). Des Weiteren
wurden Ende November 2018 einige große und kurzfristige Schwankungen in
den Komplexitätsmetriken identifiziert (2 in der Abbildung). In diesem
Zeitraum wurde die Anwendung restrukturiert. Im Rahmen der
Restrukturierung wurden regelmäßig große Codeteile entfernt und wieder
hinzugefügt. Diese Schwankungen stehen also in keinem Zusammenhang zu
dem tatsächlichen Umfang der Anwendung und können ignoriert werden. Von
Mitte Dezember 2018 bis Mitte Januar 2019 ist sowohl in den
Aufwandsabschätzungen, als auch in den Codekomplexitätsmetriken ein
Plateau erkenntlich (Nummer 4). Dieser Stillstand in der
Weiterentwicklung der Software konnte auf einen generellen
Betriebsschluss aufgrund der Weihnachtszeit zurückgeführt werden. Zum
Ende des Projektes sinkt die Entwicklungsgeschwindigkeit (Nummer 3).
Hier wurden Entwickler von dem Projekt abgezogen.

Zusammengefasst entsprechen die hier gesammelten Messdaten der
Hypothese. Über den Projektverlauf steigen die berechneten
Softwarekomplexitätsgrößen im größtenteils gleichen Maße wie die
Aufwandsabschätzungen. Abweichungen und Auffälligkeiten konnten mit den
Projektbeteiligten erklärt werden. Dementsprechend ist ein hoher
Korrelationsgrad zu erwarten.

In einem nächsten Schritt wurden wie in \ref{kriterien-zur-interpretation-der-daten} und \ref{implementierung-einer-untersuchungssoftware} beschrieben die
Korrelationskoeffizienten berechnet. Diese können Tabelle TODO entnommen
werden.

TODO

Für alle Untersuchungsergebnisse wurde ein P-Wert von \textless{} .00001
ermittelt.
Bei einem, für Sozial- und Wirtschaftswissenschaften üblichen
Signifikanzniveau von .05 lässt sich also sagen, dass die Ergebnisse
signifikant sind.

Für alle Komplexitätsmetriken konnten mit Korrelationskoeffizienten
zwischen 0,85 und 0,98 starke Korrelationen ermittelt werden. Mit einem
durchschnittlichen Korrelationskoeffizienten von 0.95 ist die
Korrelation bei den logischen Codezeilen am stärksten, gefolgt von der
Einrückungskomplexität (0,94), der zyklomatischen Komplexität (0,91) und
dem Aufwand nach Halstead (0,90).

\subsection{Fazit}\label{Fazit}

In diesem Fall konnten erfolgreich eine Zeitreihe der
Softwarekomplexitätsmaßzahlen und eine der Aufwandsabschätzungen
aufgestellt werden. Es wurden wenige Störfaktoren identifiziert.
Letztendlich wurde eine zum größten Teil sehr starke, signifikante
Korrelation zwischen den Komplexitätsmaßen und den Aufwandsabschätzungen
festgestellt. Der Fall der Digital \ac{NDA} Application spricht also für eine
Korrelation zwischen den Codekomplexitätsmetriken und den
Aufwandsabschätzungen.

\subsection{Kritik und Messfehler}\label{Kritik-und-Messfehler}

Auch wenn in diesem Projekt einige Störfaktoren, wie z.B. ein größeres
Refactoring des Codes aufgefallen sind, kann das Analyseergebnis
trotzdem als valide angenommen werden.

\section{inGRID}\label{ingrid}

Als zweiter Fall wird das interactive Generic Reporting Insight
Dashboard (inGRID) System untersucht. Das inGRID-System ist eine von DXC
entwickelte Anwendung, die eine fortschrittliche Monitoring- und
Reporting-Lösung bietet. Sie wurde als Software as a Service Angebot mit
einer zentralen Installation in der deutschen Cloud konzipiert. Eine
Vielzahl von Agenten können verwendet werden, um Systeme zu überwachen.
Die Agenten decken ein breites Spektrum an Überwachungsaktivitäten ab.
Dazu gehören \ac{VM}s, Middlewares, Anwendungen und andere Geräte (über
SNMP). Die inGRID Anwendung wird von einem Offshore Team hauptsächlich
für Kunden in Europa entwickelt. Das Entwicklungsteam ist in Ägypten
lokalisiert und besteht aus neun Entwickler*innen. Davon haben vier
Entwickler*innen mehr als drei Jahre Berufserfahrung und fünf
Entwickler*innen weniger als drei Jahre Berufserfahrung. Die
Entwickler*innen sind seit ungefähr zweieinhalb Jahren in dem Projekt
involviert.

Das inGRID System besteht aus einer Vielzahl von individuellen
Komponenten. Da die Analyse aller Komponenten für den Rahmen dieser
Arbeit zu umfangreich wäre, wird hier lediglich die Backend Komponente
des webbasierten Service Status Dashboards (SSD-Backend-Komponente)
betrachtet. Während in der gesamten Anwendung eine Vielzahl von
verschiedenen Programmiersprachen verwendet wird, besteht die
SSD-Backend-Komponente zum Großteil (93\%) aus
Java Code.

InGRID wird agil nach Scrum in Sprints mit einer Dauer von jeweils drei
Wochen realisiert. Innerhalb dieser drei Wochen werden Features von dem
Backlog des Projektes umgesetzt. Der reguläre Ablauf der Umsetzung von
Features in dem inGRID Projekt konnte von einem der Entwickler erläutert
werden. Der Ablauf beginnt mit der Anfrage eines neuen Features von
einer, am Projekt beteiligten Person. Für dieses Feature wird nun eine
User Story geschrieben und ein neuer Entwicklungszweig (englisch Branch)
in der Quelltextverwaltung erstellt. In diesem neuen Entwicklungszweig
wird nun neuer Quelltext hinzugefügt bzw. alter modifiziert. Wenn das
Feature fertig entwickelt ist, wird es mit einem zentralen
Entwicklungszweig verschmolzen (gemerged). Hier wird es möglichst
innerhalb von 24 bis 48 Stunden vollständig getestet. Wenn alle
Akzeptanzkriterien der User Story erfüllt sind, wird die User Story
geschlossen.

\subsection{Datenerhebung}\label{ingrid-Datenerhebung}

Für die SSD-Backend Komponente der inGRID Anwendung soll nun wie auch in
dem Fall der \ac{NDA} Anwendung eine Zeitreihe der Komplexitätsmetriken und
eine Zeitreihe der Aufwandsabschätzungen aufgestellt werden.

Zunächst wird hier die Datenerhebung für die Zeitreihe der
Aufwandsabschätzungen beschrieben. Dies geschieht anhand User Storys. Von besonderem Interesse für die Analyse sind
die Felder „Story Points`` und „Resolved Date``. Die Kombination beider
Felder kann verwendet werden, um die Zeitreihe der Aufwandsabschätzungen
aufzubauen. In dem Feld Story Points wird der Aufwand zur Realisierung
der Story geschätzt. Mit dem, in \ref{Aufwandsabschatzungen-mit-Planning-Poker} beschriebenen
Planning-Poker-Verfahren wird der Umfang durch eine Kombination von
Expertenmeinungen als eine Zahl in der Fibonacci Sequenz beschrieben.
Abweichend von der in \ref{aufwandsabschuxe4tzungen-agiler-projekte} beschriebenen Vorgehensweise werden in diesem
Projekt jedoch auch die Erfahrung der Entwickler, sowie der abgeschätzte
Aufwand in die Story Point Schätzungen mitaufgenommen. Das Feld
„Resolved Date`` ist für die zeitliche Verortung der Aufwandsabschätzung
der User Story relevant. Im Interview mit einem der Entwickler konnte
ermittelt werden, dass dieses Feld den Zeitpunkt der Fertigstellung
einer Story beinhaltet\footcite[Vgl. ][]{entwicklerInterviewMitEntwickler2022}. Es kann also mit diesem Feld
gesagt werden, wann der, in der User Story spezifizierte Aufwand in die
Anwendung geflossen ist. In dem Experteninterview\footcite[Vgl. ][]{entwicklerInterviewMitEntwickler2022} mit einem der Entwickler konnte an dieser Stelle eine
Verbindung zwischen den Daten aus der Projektmanagementsoftware und den
Messdaten aus dem Code geschlussfolgert werden. Wenn eine Story
geschlossen wird, wird auch der Code dieser Story in den „develop``
Branch gemerged. Also lässt sich sagen, dass der Zeitpunkt in dem
„Resolved Date`` der Story ungefähr dem Zeitpunkt der Änderungen in dem
Quelltext der Anwendung entsprechen muss. Laut dem Entwickler betrügen
die Abweichungen zwischen diesen beiden Zeitpunkten in mehr als 95\% der
Fälle weniger als 48 Stunden. In Relation zu der Gesamtdauer des
Projektes kann diese Abweichung bei einem Signifikanzniveau von 5\% als
nicht signifikant angesehen werden. Die Verbindung beider Ereignisse
über dieses Merkmal ist also valide.

Für die Analyse der User Storys anhand der zuvor beschriebenen Merkmale
müssen die User Storys eine Reihe von Kriterien erfüllen. Anhand dieser
Kriterien wird eine Vorauswahl der User Storys getroffen. Dabei werden
alle User Storys der Komponente „SSD Backend`` ausgewählt, die einen
Status von „Done`` in dem Feld „Resolution`` haben, und denen eine
Anzahl an Story Points zugewiesen ist. Jira bietet eine Möglichkeit zur
genauen Abfrage von Storys mithilfe der proprietären Abfragesprache \ac{JQL}. Die Syntax von \ac{JQL} ist stark an die Syntax der
Abfragesprache \ac{SQL} angelehnt\footcite[Vgl. ][]{atlassianptyltdUseAdvancedSearch2022}.
Die \ac{JQL} Abfrage für die User Storys findet sich im folgenden Code-Fragment.

\lstset{language=SQL}
\begin{lstlisting}
project = HCC AND component = "SSD Backend" AND issuetype = Story AND resolution = Done AND "Story Points" is not EMPTY ORDER BY resolved DESC, priority DESC, updated DESC
\end{lstlisting}

Mit dieser Abfrage konnten in dem inGRID Projekt 62 User Storys
ausgewählt werden. Diese wurden als \ac{CSV}-Datei exportiert und in die
Analysesoftware geladen.

Für die zweite Zeitreihe der Quelltextkomplexitätsmessungen wird ein
Quelltextrepository benötigt. Mit der Unterstützung des Entwicklers des
Projektes konnte das passende Repository ausgewählt werden. In dem
Repository konnten in dem entsprechenden Entwicklungszweig „develop``
2554 Commits identifiziert werden. Zu jedem dieser Commits kann der
Stand des Quelltextes rekonstruiert werden. Anhand dieser Rekonstruktion
kann, wie in 4.7.2 beschrieben eine Zeitreihe der Komplexitätsmessungen
erstellt werden.

\subsection{Auswertung}\label{ingrid-Auswertung}

Im Sinne der Fallstudienforschung werden in der Auswertung alle Daten in
Relation zueinander gebracht, um so zu logischen Schlüssen zu gelangen.
Wie in \ref{verbindung-von-daten-und-hypothesen} beschrieben, werden die Aufwandsabschätzungen zunächst in
Relation zu den Komplexitätsmetriken gebracht. Die so erlangten Daten
werden, wie in Kapitel \ref{verbindung-von-daten-und-hypothesen} beschrieben und verarbeitet. Der auf diese Weise
erlangte Korrelationsgraph ist in Abbildung TODO zu sehen und wird nun
beschrieben.

TODO

In dem Graph sind die Story Points, logischen Codezeilen, die
zyklomatische Komplexität und die Einrückungskomplexität genau wie in
dem NDA Projekt (siehe \ref{nda-Auswertung}) abgezeichnet. Die Differenz zwischen dem
Durchschnitt aller Codemetriken und den Storypoint Abschätzungen findet
sich, wie auch in dem ersten Projekt in einem separaten Diagramm unten
in der Grafik.

Zunächst verlaufen die Storypoint-Abschätzungen nahe bei den
Codekomplexitätsmetriken. Bis Ende Juni 2020 sind Abweichungen von 20
bis 30 Prozent ersichtlich. Anfang Juli 2020 ist in allen
Codekomplexitätsmetriken ein plötzlicher Abfall der Werte zu beobachten.
Dieser Abfall ist in den Aufwandsabschätzungen nicht vermerkt. In
Zusammenarbeit mit dem Entwickler konnte eine Migration der
Entwicklungsumgebung als möglicher Grund hierfür identifiziert
werden\footnote{Analyse Tabelle Zeile 38, Spalte K TODO}. Bis Januar 2021
fallen in den Codemetriken keine Veränderungen auf. Gleichzeitig steigen
die Komplexitätsabschätzungen in diesem Zeitraum aber kontinuierlich
weiter. Eine Erklärung für diese Abweichung konnte nicht gefunden
werden. Für die plötzlichen Anstiege und Abfälle der
Codekomplexitätsmetriken von Januar 2021 bis März 2021 konnte ein
Handover zwischen zwei Entwicklern als Erklärung gefunden werden. Von
März bis August 2021 bleiben sowohl die Komplexitätsmessungen, als auch
die Aufwandsabschätzungen größtenteils stabil. Diese Messdaten sprechen
also für eine Korrelation. Ab August bis zum Ende der Messdaten im
November 2021 lassen sich Anstiege in den Codekomplexitätsmetriken
beobachten. Hier wurden bereits Funktionen in der Anwendung entwickelt,
die aber noch nicht als geschlossene Storys in der
Projektmanagementsoftware vermerkt sind.

Insgesamt verliefen in diesem Projekt die Aufwandsabschätzungen nur in
Teilen ähnlich wie die Codekomplexitätsmetriken. Für die Abweichungen
konnten nicht an allen Stellen Erklärungen gefunden werden. Es ist also
zu erwarten, dass die berechneten Korrelationskoeffizienten nur eine
geringe Korrelation anzeigen.

TODO

Diese Erwartung ließ sich durch eine Berechnung der
Korrelationskoeffizienten bestätigen. Für die Komplexitätsmaßen der
logischen Codezeilen und der Einrückungskomplexität konnten über die
Koeffizienten hinweg nur Werte zwischen 0,64 und 0,43 erreicht werden.
Diese Werte schließen zwar keine Korrelation aus, sprechen aber deutlich
gegen eine Existenz dieser. Für die Komplexitätsmaßen der zyklomatischen
Komplexität und des Halstead Aufwandes konnten deutlich höhere Werte
zwischen 0,69 und 0,92 erreicht werden. Bei diesen Komplexitätsmaßen ist
in diesem Projekt also eine Korrelation mit den
Storypoint-Aufwandsabschätzungen durchaus anzunehmen. Als Erklärung für
die Abweichung zwischen den Komplexitätsmaßen kann ihre
Berechnungsmethodik vermutet werden. Sowohl die Maßzahl der logischen
Codezeilen als auch die der Einrückungskomplexität orientieren sich, wie
in 2.3 erklärt, zumindest in Teilen an den Zeilen im Sourcecode. Bei den
anderen beiden Maßen spielt die Aufteilung des Codes auf Codezeilen
keine Rolle. Hier wird lediglich der Inhalt des Codes betrachtet. Also
könnte die Hypothese aufgestellt werden, dass eine Aufteilung oder
nicht-Aufteilung des Quelltextes auf Codezeilen Einfluss auf die Maßzahl
der logischen Codezeilen und der Einrückungskomplexität genommen haben
könnte.

\subsection{Fazit}\label{ingrid-fazit}

Insgesamt konnte in diesem Projekt nur eine eingeschränkte Korrelation
zwischen den Codekomplexitätsmetriken und den Aufwandsabschätzungen
nachgewiesen werden. Für die Metriken der zyklomatischen Komplexität und
die von Halstead konnten starke Korrelationen nachgewiesen werden. Hier
spricht das Projekt also für eine generelle Korrelation. Bei den anderen
beiden Metriken konnten aber nur schwache Korrelationen festgestellt
werden, was wiederum gegen eine generelle Korrelation spricht. Insgesamt
schwächt dieses Projekt also die Hypothese, dass
Codekomplexitätsmetriken und Aufwandsabschätzungen zusammenhängen.

\subsection{Kritik}\label{ingrid-kritik}

In diesem Projekt wurden größere Abweichungen zwischen den Story Point
Abschätzungen und den Codemetriken festgestellt. Die Existenz dieser
Abweichungen deutet zunächst daraufhin, dass ein Fehler in der Messung
existieren könnte. Des Weiteren konnten nicht für alle Abweichungen
Erklärungen gefunden werden. Das könnte bedeuten, dass das Interview mit
dem Entwickler nicht ausreichend umfangreich durchgeführt wurde.
Insgesamt sollte die Validität der Ergebnisse dieses Falls stark
hinterfragt werden.

\section{Alstonii}\label{alstonii}

Der zweite Fall dieser Fallstudie befasst sich mit dem Alstonii
Projekt. Mit dem Alstonii Projekt soll ein mobiles Frontend für
Vertriebsmitarbeiter eines großen Telekommunikationsanbieters geschaffen
werden. Es soll diesen Mitarbeitern bei dem Verkauf von Breitband
Internetanbindungen an Privat- und Firmenkunden unterstützen. Als
digitales System soll das Alstonii Projekt einen analogen Prozess auf Basis
von Papierformularen ersetzen. Ziel ist die Unterstützung der gleichen
Funktionalität wie der analoge Prozess.

Das Projekt wird agil nach Scrum entwickelt und in mehreren
Programmiersprachen umgesetzt. Dabei kommen zu 82\% Vue, zu 10\%
TypeScript und zusätzlich (\textless5\%) noch \ac{HTML}, JavaScript, Python,
CSS, SCSS, Shell und Batch zum Einsatz. Realisiert
wird die Anwendung hauptsächlich von einem Nearshore Team in Bulgarien.
Unterstützt wird dieses Team von einem Offshore Team in Indien und
einigen Onsite Mitarbeitern. Insgesamt arbeiten so zwischen sechs und
neun Mitarbeiter an dem Projekt. Hauptsächlich kommen erfahrene
Mitarbeiter zum Einsatz, wobei die Erfahrung der Mitarbeiter in dem
Projekt trotz der langen Laufzeit als gering einzustufen ist. Das lässt
sich auf eine hohe Mitarbeiterfluktuation zurückführen. Das aktuelle
Entwicklungsteam ist seit sechs Monaten in dem Projekt.

\subsection{Datenerhebung}\label{Alstonii-Datenerhebung}

Auch in diesem Projekt konnten die Aufwandsabschätzungen aus der
Projektmanagementsoftware Jira exportiert werden. Nach den in Kapitel
4.4 definierten Kriterien konnten 215 Datensätze identifiziert werden.
Diese wurden als CSV-Datei exportiert und in die Analysesoftware
eingelesen.

Als Quelle für die Komplexitätsmaßzahlen wurde das GitHub Repository der
Software identifiziert. Es konnte ein Zugang beantragt werden und die
Daten konnten über eine \ac{SSH} Verbindung in eine lokale Kopie geladen
werden. Diese Kopie wurde dann von der Analysesoftware gelesen.

\subsection{Auswertung}\label{Alstonii-Auswertung}

In der Auswertung werden alle Daten nun miteinander verbunden, um so
einen Schluss auf die Hypothese zu erlangen. Die Relation der
Aufwandsabschätzungen mit den Komplexitätsmetriken ist in Abbildung TODO
erkenntlich.

TODO Grafik

Das Projekt wird über einen Zeitraum von zwei Jahren von März 2020 bis
Mai 2022 betrachtet. Eine dicke schwarze Linie zeigt die
Aufwandsabschätzungen. Die restlichen, dünneren Linien zeigen die
Komplexitätsberechnungen für die zyklomatische Komplexität, die
Einrückungskomplexität, den Halstead Aufwand und die logischen
Codezeilen. Rein visuell lässt sich erkennen, dass die Linien der
Komplexitätsmetriken ähnlich zu den Aufwandsabschätzungen verlaufen. Es
werden jedoch auch größere Abweichungen deutlich. Einige dieser
Abweichungen konnten in einem Abschlussgespräch zu dem Projekt erklärt
werden.

Zunächst konnte für die größeren Sprünge in der Einrückungskomplexität
(Nummer 1) die Verwendung eines Linters als Erklärung herangezogen
werden. Die Einrückungskomplexität wird über Einrückungen im Code
berechnet. Diese werden durch einen Linter teils automatisiert in einem
großen Umfang geändert. Der Linter passt hier die Einrückungen auf ein
festes Niveau an, wodurch die Summe der Einrückungen (siehe Kapitel \ref{Einruckungskomplexitat}) in
einem großen Umfang geändert wird. Bei den Sprüngen in der
Einrückungskomplexität fällt auch auf, dass das Komplexitätsmaß nach dem
Sprung wieder auf das gleiche Niveau wie vor dem Sprung zurückkehrt. Das
kann dadurch erklärt werden, dass die Änderungen in der Einrückung durch
den Linter nur temporär geschehen. Wird wieder zu dem ursprünglichen
Einrückungsformat zurückgewechselt, ist die Summe der Einrückungen
wieder auf dem alten Niveau.

Weiter ist von März 2021 bis Oktober 2021 ein Plateau in der
zyklomatischen Komplexität zu beobachten (Nummer 3). Dieses Plateau
lässt sich durch eine Fluktuation in der Besetzung des Projektteams
erklären. So wurde hier der Stil der Entwicklung grundsätzlich geändert.
Nach der Einarbeitungsphase des neuen Teams kam es zu einem
Sprungartigen Anstieg in den Komplexitätsmetriken.

Als letzter auffälliger Punkt wurde ein plötzliches Abfallen mit einem
anschließenden Abflachen der Komplexitätsmetriken zu Ende des Projektes
ab März 2022 identifiziert. Als potenzielle Erklärung hierfür wurde eine
Verschiebung des Entwicklungszieles vorgeschlagen. So wurde zum Ende des
Projektes ein verstärkter Fokus auf die Konsolidierung der Codebasis und
auf das Verbessern der Leistung der Applikation gelegt. Bei einer
Überarbeitung der Applikation ist zu erwarten, dass der Umfang der
Anwendung gleichbleibt. Diese Erwartung konnte mit den vorliegenden
Messdaten bestätigt werden.

Zusammenfassend entsprechen die hier gesammelten Daten zwar in grobem
Maße der Hypothese, es sind jedoch einige signifikante Abweichungen
zwischen den Maßen und den Aufwandsabschätzungen erkenntlich. Zu einem
Teil konnten sie durch technische Störfaktoren erklärt werden. Zu einem
anderen Teil sind sie aber durch Eigenheiten in dem Verlauf des
Projektes zu erklären. Nach dieser Durchsicht der Messdaten ist also
eine weniger starke Korrelation als z.B. in dem \ac{NDA} Projekt zu erwarten.
Die Korrelation der Komplexitätsmetriken mit den Aufwandsabschätzungen
wird nun berechnet.

TODO

Für alle Untersuchungsergebnisse wurde ein P-Wert von \textless{} .00001
ermittelt.
Bei einem, für Sozial- und Wirtschaftswissenschaften üblichen
Signifikanzeniveau von .05 lässt sich also sagen, dass die Ergebnisse
signifikant sind.

Wie zu erwarten war, wurden in diesem Fall geringfügig niedrigere
Korrelationskoeffizienten ermittelt als z.B. in dem \ac{NDA} Projekt
ermittelt. Im Durchschnitt sind die Werte aber nur um etwa 5\% geringer
als in dem \ac{NDA} Projekt.

\subsection{Fazit}\label{Alstonii-fazit}

Auch in dem Alstonii Projekt konnten erfolgreich Daten entsprechend
den zuvor gestellten Anforderungen gesammelt werden. Diese Daten konnten
ebenfalls erfolgreich in einen Zusammenhang mit der These gestellt
werden. Trotz stärkerer Störfaktoren wurde hier über
Korrelationkoeffizienten und Komplexitätsmetriken hinweg eine starke
Korrelation festgestellt. Also spricht dieser Fall für eine Bestätigung
der hypothetisierten Korrelation zwischen Softwarekomplexitätsmetriken
und Aufwandsabschätzungen.

\subsection{Kritik}\label{Alstonii-kritik}

In diesem Projekt ist die Bedeutung der Störfaktoren deutlich größer als
in dem NDA Projekt. Die Abweichungen zwischen den Komplexitätsmaßen und
den Aufwandsabschätzungen ließen sich nicht ausnahmslos erklären.

\section{Lacustris}\label{lacustris}

Ein weiteres Projekt ist das Lacustris Projekt\footnote{Lacustris
  Zeile 3}. Hierbei handelt es sich um einen Microservice, welcher ein
Teil eines größeren Projektes darstellt. Der Microservice ist dafür
verantwortlich, Daten aus einer \ac{AWS} Instanz abzurufen und diese für
andere Microservices zur Verfügung zu stellen\footnote{Lacustris Zeile 4}.
Entwickelt wird dieses Projekt ebenfalls in einer agilen Arbeitsweise,
sowohl von onshore, als auch von offshore Mitarbeitern. Primär sind die
Entwickler in Ägypten lokalisiert\footnote{Lacustris Zeile 7 und 8}. Die
Teams bestehen dabei aus einem Mix aus junior und senior Mitarbeitern.
Alle Mitarbeiter sind erst seit wenigen Monaten an dem Projekt
beteiligt. Die Programmiersprache des Projektes ist ausschließlich Java.

\subsection{Datenerhebung}\label{lm-Datenerhebung}

Zum Berechnen der Korrelationen sollen auch in diesem Projekt die
Aufwandsabschätzungen und die Codekomplexitätsmetriken erhoben werden.

Die Aufwandsabschätzungen lassen sich in diesem Projekt ähnlich wie in
den anderen Projekten der Projektmanagementsoftware Jira entnehmen.
Dabei liegen die Story Point Abschätzungen sowie ein Zeitstempel in
einem Feld names „Resolved`` vor. In diesem Projekt konnten insgesamt 58
relevante User Storys gefunden werden

Auch zu dem Git Repository des Projektes konnte ein Zugang erhalten
werden. Hier konnte die Entwicklungsgeschichte des Projektes an 965
Zeitpunkten rekonstruiert werden.

\subsection{Auswertung}\label{lm-Auswertung}

Für die Auswertung werden nun auch in diesem Projekt die Zeitreihe der
Aufwandsabschätzungen mit der der Codekomplexitätsmetriken in Verbindung
gebracht. Der daraus resultierende Korrelationsgraph ist in Abbildung TODO
erkenntlich.

TODO Grafik

Die Verläufe der verschiedenen Messreihen sollen nun beschrieben und
Anhand der Hinweise aus dem Abschlussinterivew des Projektes erläutert
werden. Die verschiedenen Ereignisse werden dabei in die Kategorien
People, Process und Technologies gegliedert.

In der Kategorie Process fallen vier Ereignisse auf. Alle diese
Ereignisse beziehen sich auf die Projektphasen des Projektes. Vom Anfang
des Projektes am 22. Juli 2021 bis Mitte August 2021 fällt eine
vergleichsweise geringe Produktivität auf. Hier wurde als Begründung
herbeigeführt, dass sich die Dynamik innerhalb des Teams erst bilden
musste, und das Team somit noch nicht so produktiv sein konnte (\#1). Von Mitte August 2021 bis Mitte
November 2021 ist die Produktivität des Teams deutlich höher und das
Team befindet sich in einer Phase der kontinuierlichen Entwicklung (\#2
und \#4). Ab November 2021
geht das Projekt in eine Phase der Stabilisierung über und die Codebasis
wird eher konsolidiert als erweitert. Hier steigen die Metriken also
weniger stark (\#5).

In der Kategorie People fallen drei Ereignisse auf. Zunächst war in der
letzten Woche des Novembers das komplette ägyptische Team im Urlaub (\#7). Dem folgte im Januar 2022 das Englische Team
(\#8). Als drittes Ereignis wurde Mitte Februar
2022 eine Entwicklerin von dem Projekt entlassen (\#6). Diese drei Ereignisse wirkten sich kaum auf
den Projektverlauf aus, da sich das Projekt hier schon in einer Phase
der Stabilisierung befand.

Neben diesen Projektereignissen fallen vom 4. bis zum 13. September 2021
starke Schwankungen in allen Metriken auf. Nach diesen Schwankungen
verblieben die Metriken für den Rest der Projektdauer auf einem, um 20\%
höheren Stand als vorher. Dieser starke Anstieg ließ sich auf das
Hinzufügen von vorgeneriertem Code zurückführen. So wurden durch ein
externes Programm Datenmodelle generiert, welche an dieser Stelle dem
Code hinzu gefügt wurden\footnote{LM Parties Zeile 19}.

In einem nächsten Schritt werden die Korrelationsmetriken berechnet:

TODO Grafik

Die Spearman Rangkorrelationskoeffizienten sind über alle Metriken
hinweg sehr stark mit Werten über 0,99. Die
Rangkorrelationskoeffizienten nach Kendall liegen mit Werten um 0,95 nur
leicht unter den Werten von Spearman. Auffallend ist bei beiden
Koeffizienten, dass die die Werte der Korrelationskoeffizienten über die
Metriken hinweg konstant bleiben. Die
Pearson-Produkt-Moment-Korrelationskoeffizienten liegen im Schnitt
deutlich unter den anderen Koeffizienten. Nur bei dem Aufwand nach
Halstead konnte hier ein höherer Wert erzielt werden. Das könnte darauf
zurückzuführen sein, dass der Störfaktor des vorgenerierten Codes in dem
Halstead Aufwand nicht so stark widergespiegelt wird. Dass bei den
anderen Koeffizienten der Halstead Aufwand nicht heraussticht deutet
darauf hin, dass die Koeffizienten nach Kendall und Spearman von dem
Störfaktor nicht so stark beeinflusst wurden.

\subsection{Kritik}\label{lm-kritik}

Klar zu kritisieren ist an der Analyse dieses Projektes, dass durch den
vorgenerierten Code Mitte September 2021 eine starke Verfälschung der
Messergebnisse stattgefunden hat. Die künstliche Erhöhung der Metriken
könnte die Korrelation zwischen den Aufwandsabschätzungen und den
Komplexitätsmetriken schwächen

\subsection{Fazit}\label{lm-fazit}

Insgesamt lassen sich in diesem Projekt starke Korrelationen zwischen
den Aufwandsabschätzungen und den Komplexitätsmetriken nachweisen. Eine
niedrigere Korrelation nach Pearson könnte auf den Störfaktor des
vorgenerierten Codes zurückzuführen sein. Es lässt sich also
abschließend sagen, dass dieser Fall für eine generelle Korrelation von
Aufwandsabschätzungen und Codekomplexitätsmetriken spricht.

\section{Engelmannii}\label{engelmannii}

Das Engelmannii Projekt\footnote{Engelmannii Zeile 5 TODO} wird von der
DXC für einen deutschen Energieversorger entwickelt. Es soll den
Anwendern als zentrale Verwaltungsschnittstelle für alle Dienste des
Energieversorgers dienen\footnote{Zeile 6}. Realisiert wird das Projekt
in einer agilen Arbeitsweise nach Scrum sowohl von offshore, als auch
onshore Teams. Das Offshore Team ist in Manila lokalisiert und das
Onshore Team über Deutschland verteilt. Die Teams bestehen aus
Entwicklern aller Erfahrungslevels, wobei alle seit mindestens zwei
Jahren an dem Projekt arbeiten. Aufgebaut ist das Projekt aus einem
Frontend und einem Backend. Das Frontend wird primär auf JavaScript
aufgebaut mit Technologien wie Angular und Typescript. Das Backend wird
mit .NET realisiert. Insgesamt ist die Codebasis des Projektes über drei
Repositories verteilt. Ein Repository beinhaltet sämtlichen Frontend
Code, eines den Code für die \ac{API} Middleware in C\# und ein letztes
Repository umfasst eine sog. Progressive Web App.

\subsection{Datenerhebung}\label{engelmannii-Datenerhebung}

Auch in dem Engelmannii-Projekt sollen einerseits die
Aufwandsabschätzungen und andererseits die Quelltext Repositories als
Primärquellen verwendet werden.

Die Aufwandsabschätzungen werden in dem Ernergieportal-Projekt in der
Software LifeCycle Management Software Microsoft DevOps verwaltet. Das
Format der Userstorys entspricht dabei zwar nicht in Gänze dem Format
der anderen Projekte, die relevanten Teilinformation können aber auch
hier problemlos abgelesen werden.

In einem Interview mit einem Stakeholder des Projektes\footcite[Vgl. ][]{stakeholdernInterviewMitStakeholdern2022}
konnten die relevanten Teilbereiche der User Storys identifiziert
werden. Die Aufwandsabschätzungen seien in einem Feld mit dem Namen
„Front End Dev Story Points`` zu finden. Das Abschlussdatum jedes
Features sei dem Feld „Closed Date`` zu entnehmen. Zusätzlich konnte
eine Abfrage zur Identifizierung aller relevanten User Storys erstellt
werden. Dabei schränkt die Abfrage die Auswahl der User Storys auf alle
Elemente mit einem Wert für das Feld „Front End Dev Story Points`` und
einem Status von „Closed`` ein. Von insgesamt 2329 User Storys wurden
588 als relevant ausgewählt.

Die so ausgewählten Userstorys können als \ac{CSV}-Datei exportiert und in
die Analysesoftware eingelesen werden.

Zur Ermittlung der Codekomplexitätsmetriken ist, wie in Kapitel \ref{verbindung-von-daten-und-hypothesen}
beschrieben eine Analyse des Quelltextes der Anwendung angestrebt. Wie
auch in den anderen Projekten, soll ein Änderungsverlauf der
Komplexitätsmetriken aus der Versionshistorie des Quelltextes erstellt
werden. Diese Versionshistorie ist für alle Quelltext Repositories der
Software im Format eines Git Repositories verfügbar. Wie eingangs
beschrieben besteht das Projekt aus drei Repositories. In dieser Analyse
wird jedoch lediglich die Entwicklung am Frontend der Applikation
betrachtet. Also ist auch nur das Repository des Frontends von
Interesse. Mit der in \ref{implementierung-einer-untersuchungssoftware}. beschriebenen Software kann ein Zeitverlauf
der Codekomplexitätsmetriken aus der Versionshistorie des Sourcecodes
erstellt werden. Die so ermittelten Daten werden nun ausgewertet.

\subsection{Auswertung}\label{engelmannii-Auswertung}

Zur Auswertung werden auch in diesem Projekt alle Daten in ein
Verhältnis zueinander gesetzt. Das Verhältnis der Storypoint
Aufwandsabschätzungen zu den Komplexitätsmaßzahlen findet sich in
Abbildung TODO

TODO Grafik

Bei einer genaueren Betrachtung der Messwerte fällt auf, dass die
zyklomatische Komplexität und die Einrückungskomplexität sehr nahe den
Story Point Abschätzungen verlaufen. Die Anzahl der logischen Codezeilen
und der Aufwand nach Halstead weisen jedoch von Mai 2019 bis August 2020
eine signifikante Abweichung von den Story Point Metriken von mehr als
300\% auf. Für diese Abweichung muss ein Fehler in der Datenerhebung
verantwortlich gemacht werden. Dabei wurden gewisse Berichtsdateien in
die Analyse eingeschlossen, die aber nicht zu dem eigentlichen Quelltext
der Anwendung dazugehören\footcite[Vgl. ][]{stakeholdernInterviewMitStakeholdern2022}. Im August 2020
fallen die Anzahl der logischen Codezeilen und der Halstead Aufwand
stark ab. Nach diesem Abfall befinden sie sich wieder auf einem
ähnlichen Niveau wie die anderen Metriken. Hier wurde in einem Interview
als Erklärung gefunden, dass im August 2020 die zusätzlichen
Berichtsdateien aus dem Quelltextrepository entfernt wurden. Des
weiteren ist ab März 2021 ein Abflachen des Anstieges der Komplexität
bei einem gleichzeitigen weiteren Anstieg der Aufwandsabschätzungen zu
beobachten. Ab März 2021 wurde an einer Grundlegenden Umstrukturierung
der Anwendung gearbeitet. Diese wurde jedoch in einem separaten
Entwicklungszweig durchgeführt und konnte deswegen in der Analyse der
Codemetriken nur begrenzt erfasst werden. Der Aufwand dieser
Umstrukturierung ist aber gleichzeitig in den Aufwandsabschätzungen zu
sehen. Das erklärt die Abweichung der beiden Werte in diesem Zeitraum.
Für die frequenten kleineren Sprünge in der Einrückungskomplexität ist
ein ähnlicher Messfehler wie in dem Alstonii Projekt verantwortlich.

TODO Grafik

Wie zu erwartet fallen die Korrelationskoeffizienten für die Anazhl der
logischen Codezeilen äußerst gerin aus. Mit Werten unter 0,25 ist hier
keine Korrelation nachweisbar. Das ist auf die Störung der Messung durch
Berichtsdaten zurückzuführen. Hier wurde die Messung über ein Viertel
der Projektdauer um 300\% erhöht, was den geringen
Korrelationskoeffizienten erklärt. Ähnlich wurde auch für den Aufwand
nach Halstead ein sehr geringer Korrelationkoeffizient von 0,5
ermittelt. Für die anderen beiden Komplexitätsmetriken (Zyklomatische
Komplexität und Einrückungskomplexität) konnten sehr starke
Korrelationen nachgewiesen werden. Das unterstützt wiederrum die
Vermutung, dass die geringen Korrelationskoeffizienten der anderen
Metriken mit dem Störfaktor der Berichtsdaten zu begründen sind.

\subsection{Fazit}\label{engelmanii-fazit}

Zusammenfassend konnten in diesem Fall nur bei der zyklomatischen
Komplexität und der Einrückungskomplexität eine ausreichend starke
Korrelation der Codekomplexitätsmetriken mit den Aufwandsabschätzungen
festgestellt werden. Aufgrund eines starken Störfaktors konnte bei den
anderen beiden Metriken keine Korrelation festgestellt werden. Insgesamt
sind zwar auch in diesem Projekt Korrelationen zwischen
Codekomplexitätsmetriken und Aufwandsabschätzungen feststellbar jedoch
fehlt es dem Fall für eine eindeutige Unterstützung der Hypothese an
Aussagekraft.

\subsection{Kritik}\label{engelmanii-kritik}

Klar zu kritisieren ist in diesem Fall, dass die Metriken des Halstead
Aufwandes und der logischen Codezeilen durch einen Störfaktor sehr stark
beeinträchtigt wurden.

\section{GitLab}\label{GitLab}

Als fünftes und letztes Projekt wurde die DevOps Software GitLab
untersucht. Die GitLab Software ist ein Open-Source Tool zum Entwickeln,
Absichern und Betreiben von Softwareanwendungen\footcite[Vgl. ][]{GitLab14Delivers2021}.
Besonders zu bemerken ist an dieser Stelle, dass die Firma eine
besonders transparente Strategie zum Veröffentlichen von Informationen
verfolgt\footcite[Vgl. ][]{GitLabValues}. Nach dieser Strategie werden alle
firmeninternen Informationen zunächst publiziert, sofern für sie keine
Ausnahmeregelung besteht. Eine Liste an Ausnahmen von dieser Richtlinie
ist ebenfalls publiziert. So werden zum Beispiel Informationen zu der
finanziellen Lage der Firma sowie Informationen zu Kunden nicht
publiziert\footcite[Vgl. ][]{GitLabCommunication}.

Entwickelt wird die GitLab Applikation von über 1500 Mitarbeitern,
welche ohne physische Büros aus 66 Ländern verteilt arbeiten\footcite[Vgl. ][]{GitLabGitLab}.
Die gesamte Anwendung umfasst über 650 Tausend Codezeilen und wird hauptsächlich in den Sprachen Ruby (66\%) und
JavaScript (20\%)\footcite[Vgl. ][]{GitLabOrgGitLab} geschrieben. Sämtlicher code
der Anwendung wird in einem zentral Git Repository verwaltet.

\subsection{Datenerhebung}\label{gitlab-Datenerhebung}

Aufgrund der Transparenten Kommunikation der GitLab gestaltet sich eine
Datenerhebung in diesem Projekt vergleichsweise einfach. Wie auch in den
anderen Projekten werden in diesem Projekt zwei Datenquellen benötigt.
Zum einen sollen die Codekomplexitätsmetriken aus der
Quelltextverwaltung der Software berechnet werden. Weiter sollen die
Aufwandsabschätzungen aus der Projektmanagementsoftware des Projektes
geladen werden. Beide Quellen sind inmFalle der GitLab Anwendung
öffentlich zugänglich.

Das Quelltext Repository kann unter der Adresse
\url{https://gitlab.com/gitlab-org/gitlab} abgerufen werden. Für die
Analyse wurde von dieser Adresse eine lokale Kopie des Repositories
geladen und in die Analysesoftware eingespeist. Über die Projektdauer
von fünf Jahren konnten 281.414 Entwicklungsstände identifiziert werden.

Die Aufwandsabschätzungen werden in dem GitLab Projekt im Rahmen von
Issues festgehalten. Issues entsprechen in ihrer Struktur in diesem
Kontext den User Storys. Der Aufwand zum Realisieren eines Issues wird
als das Gewicht des Issues angegeben. Es kann dabei Synonym zu den Story
Point Abschätzungen gesehen werden\footcite[Vgl. ][]{ScaledAgileGitLab}. Alle
Issues des GitLab Projektes sind auf einer Website öffentlich verfügbar
und lassen sich als \ac{CSV}-Datei exportieren\footcite[Vgl. ][]{IssuesGitLabOrg}. Auf
diese Weise konnten 68.366 Datensätze erhoben werden. Die \ac{CSV}-Datei mit
diesen Datensätzen kann unverarbeitet in die Analysesoftware geladen
werden.

Insgesamt konnten alle, für die Analyse benötigten Daten erfolgreich aus
öffentlich verfügbaren Quellen erhoben werden und ohne eine Modifikation
in die Analysesoftware geladen werden.

\subsection{Auswertung}\label{gitlab-Auswertung}

Die gesammelten Daten sollen nun ausgewertet werden. Der Umfang der
Auswertung ist in diesem Projekt ca. 15-mal so groß wie in allen anderen
Projekten. Aus diesem Grund muss die Auswertungsmethodik abgeändert
werden. Die Berechnung der Entwicklungsstände der Software wird in einem
Cluster aus vier Rechnungseinheiten in einem Rechenzentrum durchgeführt.
Zusätzlich wurden lediglich die Module Lizard und MultiMetric des Code
Parsers aktiviert. Auch mit diesen Änderungen dauerte die Analyse des
Projektes mehrere Wochen. Aufgrund des begrenzten zeitlichen Umfangs
dieser Arbeit konnte nur eine Analyse der ersten dreieinhalb Jahre des
Projektes erfolgen. Das Ergebnis dieser Analyse ist in Abbildung ..
gezeigt.

TODO Grafik

Zunächst ist visuell eine starke Korrelation aller Komplexitätsmetriken
mit den Aufwandsabschätzungen erkenntlich. Gleichzeitig fallen jedoch
auch kurzfristige Schwankungen in den Messdaten von ca. 10\% auf. Diese
Schwankungen sind über die Projektdauer hinweg in regelmäßigen Abständen
zu beobachten. Bei einer genaueren Analyse wurde festgestellt, dass
diese Schwankungen wahrscheinliche auf eine Asynchronität in der
Berechnungsmethodik der vier Berechnungsknoten zurückzuführen sind.
Dementsprechend handelt es sich hierbei also mit einer hohen
Wahrscheinlichkeit um einen Messfehler. Trotz dieser Schwankungen ist
ein hoher Korrelationsgrad zu erwarten. Dieser wird nun berechnet

TODO Grafik

Entsprechend der Erwartungen wurden in diesem Projekt starke,
konsistente Korrelationen zwischen allen Metriken und den
Aufwandsabschätzungen festgestellt. Würde man den Messfehler beheben,
ist anzunehmen, dass die Korrelationen noch stärker ausfallen.

\subsection{Kritik}\label{gitlab-kritik}

An der Analyse dieses Projektes fallen mehrere Kritikpunkte auf. Zum
einen unterliegt die Analyse potenziell einem Messfehler, welcher zu
Schwankungen in den Messdaten führt. Eine Behebung dieses Messfehlers
war aufgrund des begrenzten zeitlichen Rahmens dieser Arbeit nicht
möglich. Zusätzlich konnte in dem Projekt kein Abschlussinterview zu
Validierung der Daten durchgeführt werden. Die Validität des
Datenerhebungsverfahren konnte nur aus den öffentlich verfügbaren
Informationen der Firma begründet werden.

\subsection{Fazit}\label{gitlab-Fazit}

Zusammenfassend wurde in diesem Projekt eine starke Korrelation der
Aufwandsabschätzungen mit den Codekomplexitätsmetriken festgestellt.
Diese Korrelation könnte potenziell noch stärker ausfallen, wenn die
potenziellen Messfehler in den Daten behoben werden könnten. Insgesamt
lässt sich also sagen, dass dieser Fall die Hypothese einer Korrelation
zwischen Aufwandsabschätzungen und Codekomplexitätsmetriken stützt.
\chapter{Zusammenfassende Analyse der Untersuchungsergebnisse}\label{zusammenfassende-analyse-der-untersuchungsergebnisse}

Im vorherigen Kapitel wurde die Korrelation zwischen
Aufwandsabschätzungen und Codekomplexitätsmetriken in sechs Fällen
untersucht. Das Ergebnis dieser Untersuchung ist ein vier-Dimensionales
Datenset. Die vier Achsen dieses Datensets sind das Projekt, die
Codekomplexitätsmetrik, der Korrelationskoeffizient und der Wert dieser
Korrelation. Die Darstellung vier-dimensionaler Daten ist zwar
möglich\footcite{sarkarArtEffectiveVisualization2021}, im Rahmen dieser
Arbeit aber wenig praktikabel. Also ist die Reduktion des Datensets auf
drei Dimensionen angestrebt. Es muss also eine Dimension entfernt
werden. Da das Ziel dieser Arbeit eine Aussage über eine generelle
Korrelation ist, wird die Dimension der Korrelationskoeffizienten
vernachlässigt. Auch wurde diese Dimension in der Betrachtung der
individuellen Projekte bereits analysiert. Zur Reduktion des Datensets
auf drei Dimensionen wird das arithmetische Mittel der drei
Korrelationskoeffizienten als Wert übernommen. Das reduzierte Ergebnis
aller Projekte ist in Abbildung TODO gezeigt.

TODO Grafik

In Abbildung TODO sind die durchschnittlichen
Korrelationskoeffizienten für die einzelnen Komplexitätsmetriken in den
Projekten dargestellt. Zunächst wird versucht die Ergebnisse in den
individuellen Projekten zu begründen. Dann soll ein genereller Schluss
gezogen werden.

Der Fall des DIL NDA Projektes weist in allen vier Komplexitätsmetriken
starke Korrelationen auf. In der Analyse wurden kaum Störfaktoren
identifiziert. Es ist also davon auszugehen, dass das Ergebnis
hinreichend genau ist. Dieser Fall spricht für eine Korrelation aller
vier Komplexitätsmetriken mit den Aufwandsabschätzungen.

Im Falle des inGRID-Projektes konnten für die Zyklomatische Komplexität
und den Aufwand nach Halstead starke Korrelationen nachgewiesen werden.
Für das Fehlen der Korrelationen bei den Logischen Codezeilen und der
Einrückungskomplexität konnte keine Erklärung gefunden werden. Also
spricht dieser Fall gegen Korrelation zwischen diesen beiden Metriken
und den Aufwandsabschätzungen.

Das Alstonii Projekt spricht trotz einiger Störfaktoren für eine Korrelation
zwischen allen Metriken und den Aufwandsabschätzungen.

Auch in dem Lacustris Projekt konnten für alle Metriken
sehr starke Korrelationen festgestellt werden. Die
Pearson-Produkt-Moment-Korrelation fiel für die Anzahl logischer
Codezeilen, die zyklomatische Komplexität und die Einrückungskomplexität
zwar nur schwach aus, jedoch ist diese schwache Korrelation
wahrscheinlich auf einen Störfaktor zurückzuführen.

Bei dem Engelmannii Projekt korrelieren nur die zyklomatische Komplexität und die Einrückungskomplexität mit den
Aufwandsabschätzungen. Die beiden anderen Metriken wurden von einem
Messfehler gestört.

Zuletzt konnten in dem Fall des GitLab Projektes ebenfalls sehr starke
Korrelationen aller Komplexitätsmetriken gemessen werden.

In Tabelle TODO findet sich eine Übersicht aller
Komplexitäts-Korrelationen aller Projekte. Ein X in dem entsprechenden
Feld bedeutet, dass hier durchschnittlich ein starker
Korrelationskoeffizient (\textgreater{} 0.8) festgestellt wurde.

TODO Grafik

Insgesamt konnte für die zyklomatische Komplexität in allen Fällen eine
starke Korrelation nachgewiesen werden. Für den Aufwand nach Halstead
und die Einrückungskomplexität konnte in 5 von 6, also 80\% der Fällen
eine Korrelation nachgewiesen werden. Bei der Anzahl logischer
Codezeilen wurde nur in vier von sechs Fällen, also zu 70\% eine
Korrelation nachgewiesen.

Die Ausgangsfragestellung dieser Arbeit ist, ob eine Korrelation
zwischen 
Storypoint Aufwandsabschätzungen und
Softwarekomplexitätsmetriken existiert. Diese Frage kann mit einem
eingeschränkten „Ja`` beantwortet werden: Bei der Komplexitätsmetrik der
zyklomatischen Komplexität sprechen alle Fälle für eine Korrelation.
Zwar kann aufgrund der geringen Anzahl an Fällen in dieser Studie keine
generelle und allgemeingültige Aussage zu der Korrelation der
zyklomatischen Komplexität mit den Aufwandsabschätzungen getroffen
werden, jedoch ist eine solche Korrelation sehr wahrscheinlich. Bei dem
Halstead Aufwand und der Einrückungskomplexität spricht jeweils ein
Projekt gegen eine Starke Korrelation. Trotzdem ist eine generelle
Korrelation durchaus wahrscheinlich. Zuletzt sprechen bei der Anzahl an
logischen Codezeilen nur 70\% der Fälle für eine starke Korrelation der
Metrik mit den Aufwandsabschätzungen, jedoch ist auch hier eine leichte
Korrelation sehr wahrscheinlich. Insgesamt kann mit den Ergebnissen
dieser Arbeit gesagt werden, dass eine Korrelation zwischen Story Point
Aufwandsabschätzungen und Codekomplexitätsmetriken nicht ausgeschlossen
und in den meisten Fällen sehr wahrscheinlich ist. Die anfängliche
Zielsetzung einer Näherungsangabe zu dieser Frage ist also hiermit
erfüllt.
\chapter{Conclusion}\label{conclusion}

Die vorliegende Bachelorarbeit befasst sich mit der Fragestellung, ob
eine Korrelation zwischen Story Point Aufwandsabschätzungen und
Softwarekomplexitätsmetriken besteht. Das Ziel dieser Arbeit ist eine
Näherungsangabe als Antwort zu dieser Frage zu formulieren. Als
Forschungsmethodik wird eine explorative Fallstudien über sechs
Softwareprojekte durchgeführt.

Zur Beantwortung der Fragestellung wird zunächst die Theorie der
Softwarekomplexitätsmetriken erläutert. Die Softwarekomplexitätsmetriken
werden als Teil der statischen Code-Analyse in der
Softwarequalitätssicherung in den Software Lebenszyklus eingeordnet.
Dann werden sie als Verfahren zur Quantifizierung der Vielschichtigkeit
der Struktur eines Computerpro-gramms definiert. Es wird auf die
umfangreiche Kritik an Softwarekomplexitätsmetriken eingegangen. So
seien diese Metriken schwach in ihrer Aussagekraft und könnten die
subjektiv wahrgenommene Komplexität einer Software nicht quantifizieren.
Im nächsten Unterkapitel der theoretischen Abhandlung der
Softwarekomplexitätsmetriken wird eine Auswahl von Metriken getätigt.
Die Metriken Anzahl von logischen Codezeilen, Zyklomatische Komplexität
und Halstead Aufwand werden aufgrund ihrer hohen Anzahl an Zitierungen
in Fachliteratur ausgewählt. Die Metrik Einrückungskomplexität wird
aufgrund ihrer einzigartigen Berechnungsmethode betrachtet. Für alle
vier Metriken wird dann ihr Ursprung und ihre Berechnungsmethodik
erläutert. Außerdem wird an dieser Stelle noch einmal auf Kritik an den
individuellen Metriken eingegangen. Als zweite Vergleichsgröße werden
auch die Story Point Aufwandsabschätzungen erklärt. So wird im Kontext
der agilen Softwareentwicklung nach Scrum der Aufwand zur Realisierung
einer User Story, bzw. eines Features vor der Entwicklung abgeschätzt.

Nach der theoretischen Abhandlung der Softwarekomplexitätsmetriken und
der Aufwandsabschätzungen folgt nun eine Beschreibung des
Forschungsaufbaus. Es wird eine explorative Fallstudie an sechs
Softwareprojekten durchgeführt. In jedem Projekt werden zwei
Datenquellen untersucht. Als erste Quelle werden die
Aufwandsabschätzungen aus der Projektmanagementsoftware extrahiert. Die
zweite Quelle sind die Quelltextverwaltungen der Projekte. Aus der
Versionshistorie des Quelltextes können die Softwarekomplexitätsangaben
für die einzelnen Entwicklungsstände der Software berechnet werden. Aus
beiden Quellen wird jeweils eine Zeitreihe der Messwerte aufgestellt.
Beide Zeitreihen werden verbunden und es wird ein
Korrelationskoeffizient zwischen ihnen berechnet. Dieser
Korrelationskoeffizient bezeichnet die Korrelation zwischen den
Softwarekomplexitätsmetriken und den Aufwandsabschätzungen. Er wird als
Ergebnis der Fälle betrachtet. Die Berechnung des
Korrelationskoeffizienten aus den Quellen erfolgt mit einer
Analysesoftware. Die Implementierung dieser Analysesoftware wird
ebenfalls beschrieben.

In einem nächsten Schritt findet die Untersuchung der sechs
Softwareprojekte als Fälle der Fallstudie statt. Im ersten, dritten,
fünften und sechsten Fall kann eine starke Korrelation aller
Komplexitätsmetriken mit den Aufwandsabschätzungen gezeigt werden. Im
zweiten Fall und dritten Fall hingegen sind nur starke Korrelationen
zweier Metriken nachweisbar.

Insgesamt konnte durch die Fallstudie das Forschungsziel einer
Näherungsangabe zu der Korrelation von Softwarekomplexitätsmetriken und
Aufwandsabschätzungen erreicht werden. In allen Fällen korreliert die
zyklomatische Komplexität stark mit den Aufwandsabschätzungen. Also ist
eine allgemeine Korrelation wahrscheinlich. Für den Aufwand nach
Halstead und die Einrückungskomplexität konnte in 5 von 6, also 80\% der
Fällen eine Korrelation nachgewiesen werden. Bei der Anzahl logischer
Codezeilen wurde nur in vier von sechs Fällen, also zu 70\% eine
Korrelation nachgewiesen.

Insgesamt lautet das Ergebnis, dass eine starke Korrelation
(\textgreater{} 0,8) der Softwarekomplexitätsmetriken mit den
Aufwandsabschätzungen in jedem Fall wahrscheinlich, in keinem Fall aber
garantiert ist.

\section{Kritische Evaluierung}\label{kritische-evaluierung}

Die Ergebnisse dieser Arbeit werden nun kritisch evaluiert. Insbesondere
wird dabei auf die Gütekriterien zur Bewertung von Fallstudien nach
Gothlich 2003 genommen. (gothlichFallstudienAlsForschungsmethode2003 S.
13).

An der \emph{Objektivität}\footcite[Vgl. ][]{gothlichFallstudienAlsForschungsmethode2003
  S. 13} der Fallstudie lässt sich kritisieren, dass die Quelldaten der
Analyse nicht als Teil der Arbeit veröffentlicht wurden. Diese
beinhalten teils vetrauliche Daten und können deswegen nicht
veröffentlicht werden. Die Analysemethoden kann jedoch anhand der
öffentlichen Datenquellen des GitLab falls überprüft werden.

Die \emph{validität der Analysekonstrukte} wurde zwar nicht in einem
Gutachtenstil bewiesen, in einigen der Projekte jedoch durch
Projektbeteiligte validiert.

Auch an der \emph{internen Validität} der Arbeit gibt es Kritikpunkte.
So wurde in der Interpretation der Daten keinerlei alternative
Interpretationsmöglichkeiten berücksichtigt.

Zu der \emph{externen Validität} der Fälle lässt sich sagen, dass zwar
eine Replikationslogik angewendet wurde, das Analysevorgehen jedoch
nicht durch Feedbackschleifen mit den Ergebnissen der Fälle verfeinert
wurde.

Für die \emph{Reliabilität} spricht zunächst das sämtliche Analysen
vollständig automatisiert durchgeführt wurden, was Rechenfehler
größtenteils ausschließt. Jedoch werden dadurch eventuell auch
eigenheiten in den Messdaten ignoriert. So sind in mehreren der Fälle
Messfehler aufgefallen, die auf einer Fehlinterpretation der Daten durch
die Analysesoftware beruhen.

Zuletzt ist auch die \emph{Utilitarität} der Studie gegeben. Abgesehen
von einem regen Interesse innerhalb der Firma steht dem Aufwand zur
Realisierung der Studie auch der generelle Nutzen der Validierung der
Komplexitätsmetriken gegenüber.

Zusätzlich zu den Gütekriterien von Gothlich werden auch weitere
Kritikpunkte

\section{Ausblick}\label{Ausblick}

Im Folgenden soll aufbauend auf den Ergebnissen dieser Arbeit ein
Ausblick auf mögliche weitere Entwicklungen gegeben werden.

\subsection{Zusätzliche Validierung der Ergebnisse}\label{Zusatzliche-Validierung-der-Ergebnisse}

Zunächst können die hier erlangten Ergebnisse weiter validiert werden.
Dazu sollten einerseits die Messfehler in der Untersuchung der Projekte
behoben werden. Hierzu könnte die Analysesoftware verbessert werden.
Auch kann die Validierung der Ergebnisse durch mehr und umfangreichere
Interviews mit Projektbeteiligten verbessert werden. Zuletzt würde eine
Ausweitung der Fallstudie auf mehr Projekte die Validität der
Gesamtaussage weiter stützen.

\subsection{Umgebungsparameteranalyse}\label{Umgebungsparameteranalyse}

Neben der Weiterentwicklung der Korrelationsanalyse ist im Rahmen dieser
Arbeit noch ein weiterer interessanter Zusammenhang aufgefallen. An den
Stellen, an denen die Codekomplexitätsmetriken besonders stark von den
Aufwandsabschätzungen abwichen ließen sich in vielen Fällen besondere
Projektereignisse nachweisen. Also liegt die Vermutung nahe, dass
bestimmte Ereignisse im Verlauf eines Softwareprojektes eine Abweichung
der Kompelxitätsmetriken von den Aufwandsabschätzungen herbeiführen.
Dieser Zusammenhang könnte in einer weiteren Analyse untersucht werden.

Zur Untersuchung der Analyseergebnisse wurde von einem Experten der DXC
Technologies bereits eine sogenannte Umgebungsparameteranalyse
skizziert. Diese wird als mögliche Weiterentwicklung dieser Arbeit im
Folgenden beschrieben.

Im Rahmen der Umgebungsparameteranalyse werden als erster Schritt Fokus
Punkte für die weitere Analyse ausgesucht. Das sollen Zeitintervalle
seien, an denen die Codekomplexitätsmetriken über einen bestimmten
Schwellenwert hinweg von den Aufwandsabschätzungen abweichen. Für das
Alstonii Projekt ist die Auswahl dieser Zeitintervalle beispielhaft in
Abbildung .. skizziert.

TODO Grafik

Durch langjährige Erfahrung und Befragung von Projektbeteiligten konnten
Umgebungsvariablen identifiziert werden, welche die Korrelation
beeinflussen. Diese werden in drei Überkategorien gegliedert:
Menschliche Faktoren (People), Prozessfaktoren (Process) und
Technologiefaktoren (Technology). Zu jeder Hauptkategorie konnten Vier
Messbereiche identifiziert werden. Diese werden mit unterschiedlichen
quantitativen Indikatoren objektiv durch Befragung erfasst.

TODO Grafik

In einem dritten Schritt werden diese zwei Informationsstränge in Verbindung
zueinander gesetzt. So sollen potenzielle Kausalitäten erkannt werden.
Ein Umgebungsparameter könnte z.B. zu einer Abweichung der
Komplexitätsmetriken von den Aufwandsabschätzungen führen. Der so
generierte Kausalitätsgraph ist beispielhaft in Abbildung .. skizziert.

TODO Grafik

In einem letzten Schritt soll versucht werden, aus den Kausalitäten
Verbesserungsempfehlungen für das Management des Projektes abzuleiten.
So könnte z.B. eine Änderung in der Arbeitsmethode (Prozessfaktor) eine
Produktivitätserhöhung herbeiführen.

Insgesamt könnte mit der hier in Aussicht gestellten Arbeit ein System
zur kontinuierlichen Verbesserung der Entwicklungstechnik in agilen
Projekten geschaffen werden.

\input{includes/8-post/anhang.tex}
\input{includes/release_notes.tex}
%%% Ende des eigentlichen Inhalts %%%


%%% Quellenverzeichnisse (keine Anpassung nötig) %%%
\clearpage
\literaturverzeichnis
%%% Ende Quellenverzeichnisse %%%


%%% Erklärung (keine Anpassungen nötig) %%%
% steht ganz am Ende des Dokuments
\cleardoublepage
\input{template/_dhbw_erklaerung.tex}
\end{document}