\chapter{Zusammenfassende Analyse der Untersuchungsergebnisse}\label{zusammenfassende-analyse-der-untersuchungsergebnisse}

Im vorherigen Kapitel wurde die Korrelation zwischen
Aufwandsabschätzungen und Codekomplexitätsmetriken in sechs Fällen
untersucht. Das Ergebnis dieser Untersuchung ist ein vier-Dimensionales
Datenset. Die vier Achsen dieses Datensets sind das Projekt, die
Codekomplexitätsmetrik, der Korrelationskoeffizient und der Wert dieser
Korrelation. Die Darstellung vier-dimensionaler Daten ist zwar
möglich\footcite{sarkarArtEffectiveVisualization2021}, im Rahmen dieser
Arbeit aber wenig praktikabel. Also ist die Reduktion des Datensets auf
drei Dimensionen angestrebt. Es muss also eine Dimension entfernt
werden. Da das Ziel dieser Arbeit eine Aussage über eine generelle
Korrelation ist, wird die Dimension der Korrelationskoeffizienten
vernachlässigt. Auch wurde diese Dimension in der Betrachtung der
individuellen Projekte bereits analysiert. Zur Reduktion des Datensets
auf drei Dimensionen wird das arithmetische Mittel der drei
Korrelationskoeffizienten als Wert übernommen. Das reduzierte Ergebnis
aller Projekte ist in Abbildung TODO gezeigt.

TODO Grafik

In Abbildung TODO sind die durchschnittlichen
Korrelationskoeffizienten für die einzelnen Komplexitätsmetriken in den
Projekten dargestellt. Zunächst wird versucht die Ergebnisse in den
individuellen Projekten zu begründen. Dann soll ein genereller Schluss
gezogen werden.

Der Fall des DIL NDA Projektes weist in allen vier Komplexitätsmetriken
starke Korrelationen auf. In der Analyse wurden kaum Störfaktoren
identifiziert. Es ist also davon auszugehen, dass das Ergebnis
hinreichend genau ist. Dieser Fall spricht für eine Korrelation aller
vier Komplexitätsmetriken mit den Aufwandsabschätzungen.

Im Falle des inGRID-Projektes konnten für die Zyklomatische Komplexität
und den Aufwand nach Halstead starke Korrelationen nachgewiesen werden.
Für das Fehlen der Korrelationen bei den Logischen Codezeilen und der
Einrückungskomplexität konnte keine Erklärung gefunden werden. Also
spricht dieser Fall gegen Korrelation zwischen diesen beiden Metriken
und den Aufwandsabschätzungen.

Das Alstonii Projekt spricht trotz einiger Störfaktoren für eine Korrelation
zwischen allen Metriken und den Aufwandsabschätzungen.

Auch in dem Lacustris Projekt konnten für alle Metriken
sehr starke Korrelationen festgestellt werden. Die
Pearson-Produkt-Moment-Korrelation fiel für die Anzahl logischer
Codezeilen, die zyklomatische Komplexität und die Einrückungskomplexität
zwar nur schwach aus, jedoch ist diese schwache Korrelation
wahrscheinlich auf einen Störfaktor zurückzuführen.

Bei dem Engelmannii Projekt korrelieren nur die zyklomatische Komplexität und die Einrückungskomplexität mit den
Aufwandsabschätzungen. Die beiden anderen Metriken wurden von einem
Messfehler gestört.

Zuletzt konnten in dem Fall des GitLab Projektes ebenfalls sehr starke
Korrelationen aller Komplexitätsmetriken gemessen werden.

In Tabelle TODO findet sich eine Übersicht aller
Komplexitäts-Korrelationen aller Projekte. Ein X in dem entsprechenden
Feld bedeutet, dass hier durchschnittlich ein starker
Korrelationskoeffizient (\textgreater{} 0.8) festgestellt wurde.

TODO Grafik

Insgesamt konnte für die zyklomatische Komplexität in allen Fällen eine
starke Korrelation nachgewiesen werden. Für den Aufwand nach Halstead
und die Einrückungskomplexität konnte in 5 von 6, also 80\% der Fällen
eine Korrelation nachgewiesen werden. Bei der Anzahl logischer
Codezeilen wurde nur in vier von sechs Fällen, also zu 70\% eine
Korrelation nachgewiesen.

Die Ausgangsfragestellung dieser Arbeit ist, ob eine Korrelation
zwischen 
Storypoint Aufwandsabschätzungen und
Softwarekomplexitätsmetriken existiert. Diese Frage kann mit einem
eingeschränkten „Ja`` beantwortet werden: Bei der Komplexitätsmetrik der
zyklomatischen Komplexität sprechen alle Fälle für eine Korrelation.
Zwar kann aufgrund der geringen Anzahl an Fällen in dieser Studie keine
generelle und allgemeingültige Aussage zu der Korrelation der
zyklomatischen Komplexität mit den Aufwandsabschätzungen getroffen
werden, jedoch ist eine solche Korrelation sehr wahrscheinlich. Bei dem
Halstead Aufwand und der Einrückungskomplexität spricht jeweils ein
Projekt gegen eine Starke Korrelation. Trotzdem ist eine generelle
Korrelation durchaus wahrscheinlich. Zuletzt sprechen bei der Anzahl an
logischen Codezeilen nur 70\% der Fälle für eine starke Korrelation der
Metrik mit den Aufwandsabschätzungen, jedoch ist auch hier eine leichte
Korrelation sehr wahrscheinlich. Insgesamt kann mit den Ergebnissen
dieser Arbeit gesagt werden, dass eine Korrelation zwischen Story Point
Aufwandsabschätzungen und Codekomplexitätsmetriken nicht ausgeschlossen
und in den meisten Fällen sehr wahrscheinlich ist. Die anfängliche
Zielsetzung einer Näherungsangabe zu dieser Frage ist also hiermit
erfüllt.