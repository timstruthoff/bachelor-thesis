\chapter{Aufwandsabschätzungen agiler Projekte}\label{aufwandsabschuxe4tzungen-agiler-projekte}

In dieser Arbeit sollen Softwarekomplexitätsmaßen in Korrelation zu
Aufwandsabschätzungen gesetzt werden. Für ein genaues Verständnis dieser
Korrelation ist es auch von Nöten, die Aufwandsabschätzungen selbst
genauer zu betrachten.

Die, in dieser Arbeit betrachteten Projekte werden alle in einer agilen
Arbeitsweise realisiert. Im Rahmen dieser Vorgehensweise werden
üblicherweise für jedes zu entwickelnde Feature Schätzungen des
Aufwandes abgegeben. Für ein besseres Verständnis der Entstehung dieser
Projekte wird im Folgenden zunächst die agile Arbeitsweise erläutert.
Dann werden die Aufwandsabschätzungen der agilen Projekte genauer
betrachtet und im Kontext dieser Arbeit erläutert.

\section{Die agile Arbeitsweise in der
Softwareentwicklung}\label{die-agile-arbeitsweise-in-der-softwareentwicklung}

Als die agile Arbeitsweise wird eine Reihe von iterativen
Methoden \footcite[Vgl. ][]{atlassianWhatAgile} beschrieben, welche den Denk- und
Arbeitsprozess von Softwareentwicklungsteams effizienter und effektiver
gestalten sollen. Dabei werden alle Bereiche der traditionellen
Softwareentwicklung, von dem Projektmanagement, über Softwaredesign und
-architektur bis hin zur Prozessoptimierung betrachtet. Zu jedem dieser
Bereiche bestehen Praktiken, die über Softwareprojekte hinweg möglichst
einfach zu implementieren sein sollen\footcite[Vgl. ][]{stellmanLearningAgile2014
  S. 2}.

Zusätzlich wird mit der agilen Arbeitsweise eine Mentalität beschrieben,
bei der Planung, Design und Prozessoptimierung von dem gesamten
Entwicklungsteam durchgeführt werden sollen\footcite[Vgl. ][]{stellmanLearningAgile2014
  S. 2}.

\section{Methoden der agilen Softwareentwicklung}\label{methoden-der-agilen-softwareentwicklung}

Die agile Vorgehensweise wird in der Praxis in Form von verschiedenen
Methoden umgesetzt. Diese Methoden definieren konkrete Abläufe und
Verantwortlichkeiten. Zu dem Methoden gehören unter anderem Scrum,
Kanban und das Scaled Agile Framework. Die in dieser Arbeit behandelten
Projekte werden nach Scrum umgesetzt. Aus diesem Grund wird im folgenden
Kapitel die Scrum Methodik erklärt.

\subsection{Ablauf}\label{Ablauf}

Zur Entwicklung eines Produktes nach Scrum befasst sich ein sog.
Productowner zunächst mit den Wünschen der Stakeholder und hält diese in
einem priorisierten Product-Backlog fest. Zu Beginn eines jeden Sprints
wird in einem sog. Sprint Planning Meeting von dem Entwicklungsteam
festgelegt, welche der Aufgaben in dem nächsten Sprint umgesetzt werden
können. Es wird dabei nach Priorität vorgegangen. Diese Arbeitspakete
werden in dem Sprint-Backlog für jeden Sprint festgehalten. Während des
Sprints setzen die Entwickler die Aufgaben im Sprint-Backlog um, während
der Productowner den Backlog bearbeitet\footcite[Vgl. ][]{schwaberAgileSoftwareDevelopment2002}
\footcite[Vgl. ][]{schwaberAgileProjectManagement2004}.

Die Verwaltung der Anforderungen im Backlog geschieht in Form von sog.
Userstorys. Diese Userstorys sollten die gewünschte Funktionalität, die
Rolle des Anwenders, und den Geschäftswert dieser Funktionalität
beinhalten. Daneben wird der Aufwand für die Realisierung des Features
geschätzt\footcite[Vgl. ][]{cohnUserStoriesApplied2004}. Der Aufwand wird nicht
als absolute Zahl geschätzt, sondern als Aufwand relativ zu anderen
Stories geschätzt. Hat Story A eine Storypointanzahl von 1 und B eine
Anzahl von 5, ist B mit fünfmal so viel Aufwand verbunden. In einigen
Projekten werden Storypoints im gleichen Sinne auch zur Schätzung der
Komplexität herangezogen\footcite[Vgl. ][]{Quelle fehlt}. Von den meisten
Scrum-Teams wird eine feste Zahlenfolge zur Schätzung der Aufwände
verwendet. Dabei ist die Fibonacci-Folge\footnote{erklären} besonders
beliebt\footcite[Vgl. ][]{Quelle fehlt}.

\subsection{Aufwandsabschätzungen mit Planning Poker}\label{Aufwandsabschatzungen-mit-Planning-Poker}

Eine weit verbreitete Technik zum Schätzen der Userstorys ist das
Planungspoker. Zur Schätzung einer Story gibt jedes Teammitglied
gleichzeitig und verdeckt eine individuelle Schätzung zu der Story ab.
In einem zweiten Schritt wird durch eine Diskussion eine Einigung über
das Ergebnis erzielt. Die so erzielten Ergebnisse sollten innerhalb
eines Projektes konsequent sein. Auch innerhalb eines Teams über
Projekte hinweg können oft konsequente Ergebnisse erzielt werden. Diese
Eigenheit wird auch in dieser Arbeit berücksichtigt. So werden die
Messwerte der Storypoints auf den Kontext der einzelnen Projekte
normalisiert\footcite{Quelle ergänzen cohnAgileEstimatingPlanning2006}
\footcite[Vgl. ][]{(daltonGreatBigAgile2019a S. 203).}.
