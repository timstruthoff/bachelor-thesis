\chapter{Untersuchung der Softwarekomplexitätsmetriken}\label{untersuchung-der-softwarekomplexituxe4tsmetriken}

Mit der in Kapitel \ref{forschungsaufbau-der-fallstudie} vorgestellten Methodik werden nun sechs
Softwareprojekte als Fälle der Fallstudie untersucht. Dabei wird jedes
Projekt zunächst wie in \ref{analyseeinheit} beschrieben vorgestellt. Dann wird die
Erhebung der Daten beschrieben. Am Ende der Datenerhebung steht für
jedes Projekt ein Diagramm zu Verfügung. Für das erste Projekt wird
dieses Vorgehen beispielhaft ausführlich erklärt. Hier wird auch der
Korrelationsgraph erläutert. Zur Vermeidung von Doppelungen wird in den
weiteren Projekten auf eine umfangreiche Erklärung des Analysevorgehens
verzichtet.

\section{Digital NDA Application}\label{digital-nda-application}

Die „Digital \ac{NDA} Application`` hat das Ziel, das komplette Handling der
Verschwiegenheitserklärungen im \ac{DIL} und bei potenziellen weiteren Kunden
zu digitalisieren. Insbesondere soll die bisher für eine
Verschwiegenheitserklärung nötige Papierunterschrift durch eine digitale
Signatur ersetzt werden. Dazu wird den \ac{DIL}-Besuchern die Möglichkeit
gegeben, ihre Daten schon vorab über eine Webapplikation zu übermitteln,
woraufhin die bereits auf den Besucher angepasste
Verschwiegenheitserklärung vor Ort auf einem Tablet signiert werden
kann\footcite[Vgl. ][]{dxctechnologiesInternesDokumentAufbau2022}.

Das Projekt wurde agil von einem onshore Team aus erfahrenen Entwicklern
in Deutschland entwickelt. Es wurden hauptsächlich JavaScript
(52\%) und TypeScript (30\%)
als Programmiersprachen verwendet. Zusätzlich fielen in der Analyse noch
kleinere Programteile in Gherkin (7\%), PLpgSQL (5\%), sowie \ac{HTML},
Shell, SCSS und der Dockerfile Syntax zu jeweils weniger als 5\% Anteil
auf.

\subsection{Datenerhebung}\label{nda-Datenerhebung}

Wie in Kapitel \ref{datensammlung} beschrieben, sollen für die Fallstudie in jedem
Projekt zunächst verschiedene Daten erhoben werden. Für die Berechnung
der Korrelation zwischen den Softwarekomplexitätsmetriken und den
Aufwandsabschätzungen wird für beide Grö\ss en jeweils eine Datenquelle
benötigt. Die Daten zu den Aufwandsabschätzungen sollen aus der
Projektmanagementsoftware des Projektes bezogen werden und die Daten zu
den Softwarekomplexitätsmetriken aus der Versionsverwaltung des
Quelltextes.

In dem Digital \ac{NDA} Application Projekt werden die Aufwandsabschätzungen
in der Projektmanagementsoftware Jira verwaltet. In der
Projektmanagementsoftware finden die Aufwandsabschätzungen entsprechend
der Scrum-Methodik (siehe Kapitel \ref{aufwandsabschuxe4tzungen-agiler-projekte}) im Rahmen von User Storys statt.
In diesen User Storys ist jeweils eine Anzahl an Story Points und ein Datum vermerkt.

TODO Grafik

Wie in Kapitel \ref{datensammlung} beschrieben, sollen für die Fallstudie in jedem
Projekt zunächst verschiedene Daten erhoben werden. Für die Berechnung
der Korrelation zwischen den Softwarekomplexitätsmetriken und den
Aufwandsabschätzungen wird für beide Grö\ss en jeweils eine Datenquelle
benötigt. Die Daten zu den Aufwandsabschätzungen sollen aus der
Projektmanagementsoftware des Projektes bezogen werden und die Daten zu
den Softwarekomplexitätsmetriken aus der Versionsverwaltung des
Quelltextes.

In dem Digital \ac{NDA} Application Projekt werden die Aufwandsabschätzungen
in der Projektmanagementsoftware Jira verwaltet. In der
Projektmanagementsoftware finden die Aufwandsabschätzungen entsprechend
der Scrum-Methodik (siehe Kapitel \ref{aufwandsabschuxe4tzungen-agiler-projekte}) im Rahmen von User Storys statt.
Ein Beispiel einer User Story findet sich in Abbildung TODO . In
jeder User Story werden, wie in Kapitel \ref{aufwandsabschuxe4tzungen-agiler-projekte} beschrieben wichtige
Informationen zu der geplanten Funktion der Anwendung hinterlegt. Für
diese Arbeit sind die Felder „Story Points`` (\#1) und „Resolved`` unter
Dates (\#2) von besonderer Relevanz.

TODO Grafik

Aus der Software konnten insgesamt 67 Storys exportiert werden und
dienen wie in \ref{Anforderungen-an-die-Untersuchungssoftware} beschrieben als Input für die Analysesoftware.

Als zweite Datenquelle wird die Versionshistorie des Quelltextes der
DTCNDA Anwendung herangezogen. Diese liegt in Form eines Git
Repositories vor. Aus dem Git Repository konnten 1625 Commits geladen
werden. Die Versionshistorie des Quellcodes wird ebenfalls als
Eingabeparameter für das Analysetool übernommen.

\subsection{Auswertung}\label{nda-Auswertung}

In dieser Arbeit soll die Korrelation zwischen
Softwarekomplexitätsmetriken und Aufwandsabschätzungen beschrieben
werden. Nachdem in \ref{nda-Datenerhebung} für beide Grö\ss en Daten gesammelt wurden, sollen
diese nun wie in \ref{verbindung-von-daten-und-hypothesen} beschrieben ausgewertet werden. Hierzu werden die
Daten in die in \ref{implementierung-einer-untersuchungssoftware} beschriebe Analysesoftware geladen und mit ihr
verarbeitet. Es werden zunächst vom Code Parser die
Codekomplexitätsmetriken für die einzelnen Entwicklungsstände der
Software berechnet. Dann wird das Ergebnis dieser Berechnungen in das
Zahlenverarbeitungsmodul geladen. Dort werden die
Codekomplexitätsmetriken zusammen mit den Aufwandsabschätzungen zu einem
einheitlichen Format verarbeitet und in Relation zueinander gesetzt.
Diese Relation der Grö\ss en wird mit dem Graph in Abbildung TODO
beschrieben.

\begin{figure}\label{nda-graph}
  \begin{center}
      %% Creator: Matplotlib, PGF backend
%%
%% To include the figure in your LaTeX document, write
%%   \input{<filename>.pgf}
%%
%% Make sure the required packages are loaded in your preamble
%%   \usepackage{pgf}
%%
%% Also ensure that all the required font packages are loaded; for instance,
%% the lmodern package is sometimes necessary when using math font.
%%   \usepackage{lmodern}
%%
%% Figures using additional raster images can only be included by \input if
%% they are in the same directory as the main LaTeX file. For loading figures
%% from other directories you can use the `import` package
%%   \usepackage{import}
%%
%% and then include the figures with
%%   \import{<path to file>}{<filename>.pgf}
%%
%% Matplotlib used the following preamble
%%   \usepackage{fontspec}
%%
\begingroup%
\makeatletter%
\begin{pgfpicture}%
\pgfpathrectangle{\pgfpointorigin}{\pgfqpoint{6.317353in}{3.277753in}}%
\pgfusepath{use as bounding box, clip}%
\begin{pgfscope}%
\pgfsetbuttcap%
\pgfsetmiterjoin%
\definecolor{currentfill}{rgb}{1.000000,1.000000,1.000000}%
\pgfsetfillcolor{currentfill}%
\pgfsetlinewidth{0.000000pt}%
\definecolor{currentstroke}{rgb}{1.000000,1.000000,1.000000}%
\pgfsetstrokecolor{currentstroke}%
\pgfsetdash{}{0pt}%
\pgfpathmoveto{\pgfqpoint{0.000000in}{-0.000000in}}%
\pgfpathlineto{\pgfqpoint{6.317353in}{-0.000000in}}%
\pgfpathlineto{\pgfqpoint{6.317353in}{3.277753in}}%
\pgfpathlineto{\pgfqpoint{0.000000in}{3.277753in}}%
\pgfpathlineto{\pgfqpoint{0.000000in}{-0.000000in}}%
\pgfpathclose%
\pgfusepath{fill}%
\end{pgfscope}%
\begin{pgfscope}%
\pgfsetbuttcap%
\pgfsetmiterjoin%
\definecolor{currentfill}{rgb}{1.000000,1.000000,1.000000}%
\pgfsetfillcolor{currentfill}%
\pgfsetlinewidth{0.000000pt}%
\definecolor{currentstroke}{rgb}{0.000000,0.000000,0.000000}%
\pgfsetstrokecolor{currentstroke}%
\pgfsetstrokeopacity{0.000000}%
\pgfsetdash{}{0pt}%
\pgfpathmoveto{\pgfqpoint{0.481681in}{1.080890in}}%
\pgfpathlineto{\pgfqpoint{6.267353in}{1.080890in}}%
\pgfpathlineto{\pgfqpoint{6.267353in}{3.227753in}}%
\pgfpathlineto{\pgfqpoint{0.481681in}{3.227753in}}%
\pgfpathlineto{\pgfqpoint{0.481681in}{1.080890in}}%
\pgfpathclose%
\pgfusepath{fill}%
\end{pgfscope}%
\begin{pgfscope}%
\pgfpathrectangle{\pgfqpoint{0.481681in}{1.080890in}}{\pgfqpoint{5.785672in}{2.146863in}}%
\pgfusepath{clip}%
\pgfsetrectcap%
\pgfsetroundjoin%
\pgfsetlinewidth{0.100375pt}%
\definecolor{currentstroke}{rgb}{0.501961,0.501961,0.501961}%
\pgfsetstrokecolor{currentstroke}%
\pgfsetdash{}{0pt}%
\pgfpathmoveto{\pgfqpoint{0.767804in}{1.080890in}}%
\pgfpathlineto{\pgfqpoint{0.767804in}{3.227753in}}%
\pgfusepath{stroke}%
\end{pgfscope}%
\begin{pgfscope}%
\pgfsetbuttcap%
\pgfsetroundjoin%
\definecolor{currentfill}{rgb}{0.000000,0.000000,0.000000}%
\pgfsetfillcolor{currentfill}%
\pgfsetlinewidth{0.501875pt}%
\definecolor{currentstroke}{rgb}{0.000000,0.000000,0.000000}%
\pgfsetstrokecolor{currentstroke}%
\pgfsetdash{}{0pt}%
\pgfsys@defobject{currentmarker}{\pgfqpoint{0.000000in}{0.000000in}}{\pgfqpoint{0.000000in}{0.041667in}}{%
\pgfpathmoveto{\pgfqpoint{0.000000in}{0.000000in}}%
\pgfpathlineto{\pgfqpoint{0.000000in}{0.041667in}}%
\pgfusepath{stroke,fill}%
}%
\begin{pgfscope}%
\pgfsys@transformshift{0.767804in}{1.080890in}%
\pgfsys@useobject{currentmarker}{}%
\end{pgfscope}%
\end{pgfscope}%
\begin{pgfscope}%
\pgfsetbuttcap%
\pgfsetroundjoin%
\definecolor{currentfill}{rgb}{0.000000,0.000000,0.000000}%
\pgfsetfillcolor{currentfill}%
\pgfsetlinewidth{0.501875pt}%
\definecolor{currentstroke}{rgb}{0.000000,0.000000,0.000000}%
\pgfsetstrokecolor{currentstroke}%
\pgfsetdash{}{0pt}%
\pgfsys@defobject{currentmarker}{\pgfqpoint{0.000000in}{-0.041667in}}{\pgfqpoint{0.000000in}{0.000000in}}{%
\pgfpathmoveto{\pgfqpoint{0.000000in}{0.000000in}}%
\pgfpathlineto{\pgfqpoint{0.000000in}{-0.041667in}}%
\pgfusepath{stroke,fill}%
}%
\begin{pgfscope}%
\pgfsys@transformshift{0.767804in}{3.227753in}%
\pgfsys@useobject{currentmarker}{}%
\end{pgfscope}%
\end{pgfscope}%
\begin{pgfscope}%
\pgfpathrectangle{\pgfqpoint{0.481681in}{1.080890in}}{\pgfqpoint{5.785672in}{2.146863in}}%
\pgfusepath{clip}%
\pgfsetrectcap%
\pgfsetroundjoin%
\pgfsetlinewidth{0.100375pt}%
\definecolor{currentstroke}{rgb}{0.501961,0.501961,0.501961}%
\pgfsetstrokecolor{currentstroke}%
\pgfsetdash{}{0pt}%
\pgfpathmoveto{\pgfqpoint{1.174133in}{1.080890in}}%
\pgfpathlineto{\pgfqpoint{1.174133in}{3.227753in}}%
\pgfusepath{stroke}%
\end{pgfscope}%
\begin{pgfscope}%
\pgfsetbuttcap%
\pgfsetroundjoin%
\definecolor{currentfill}{rgb}{0.000000,0.000000,0.000000}%
\pgfsetfillcolor{currentfill}%
\pgfsetlinewidth{0.501875pt}%
\definecolor{currentstroke}{rgb}{0.000000,0.000000,0.000000}%
\pgfsetstrokecolor{currentstroke}%
\pgfsetdash{}{0pt}%
\pgfsys@defobject{currentmarker}{\pgfqpoint{0.000000in}{0.000000in}}{\pgfqpoint{0.000000in}{0.041667in}}{%
\pgfpathmoveto{\pgfqpoint{0.000000in}{0.000000in}}%
\pgfpathlineto{\pgfqpoint{0.000000in}{0.041667in}}%
\pgfusepath{stroke,fill}%
}%
\begin{pgfscope}%
\pgfsys@transformshift{1.174133in}{1.080890in}%
\pgfsys@useobject{currentmarker}{}%
\end{pgfscope}%
\end{pgfscope}%
\begin{pgfscope}%
\pgfsetbuttcap%
\pgfsetroundjoin%
\definecolor{currentfill}{rgb}{0.000000,0.000000,0.000000}%
\pgfsetfillcolor{currentfill}%
\pgfsetlinewidth{0.501875pt}%
\definecolor{currentstroke}{rgb}{0.000000,0.000000,0.000000}%
\pgfsetstrokecolor{currentstroke}%
\pgfsetdash{}{0pt}%
\pgfsys@defobject{currentmarker}{\pgfqpoint{0.000000in}{-0.041667in}}{\pgfqpoint{0.000000in}{0.000000in}}{%
\pgfpathmoveto{\pgfqpoint{0.000000in}{0.000000in}}%
\pgfpathlineto{\pgfqpoint{0.000000in}{-0.041667in}}%
\pgfusepath{stroke,fill}%
}%
\begin{pgfscope}%
\pgfsys@transformshift{1.174133in}{3.227753in}%
\pgfsys@useobject{currentmarker}{}%
\end{pgfscope}%
\end{pgfscope}%
\begin{pgfscope}%
\pgfpathrectangle{\pgfqpoint{0.481681in}{1.080890in}}{\pgfqpoint{5.785672in}{2.146863in}}%
\pgfusepath{clip}%
\pgfsetrectcap%
\pgfsetroundjoin%
\pgfsetlinewidth{0.100375pt}%
\definecolor{currentstroke}{rgb}{0.501961,0.501961,0.501961}%
\pgfsetstrokecolor{currentstroke}%
\pgfsetdash{}{0pt}%
\pgfpathmoveto{\pgfqpoint{1.580462in}{1.080890in}}%
\pgfpathlineto{\pgfqpoint{1.580462in}{3.227753in}}%
\pgfusepath{stroke}%
\end{pgfscope}%
\begin{pgfscope}%
\pgfsetbuttcap%
\pgfsetroundjoin%
\definecolor{currentfill}{rgb}{0.000000,0.000000,0.000000}%
\pgfsetfillcolor{currentfill}%
\pgfsetlinewidth{0.501875pt}%
\definecolor{currentstroke}{rgb}{0.000000,0.000000,0.000000}%
\pgfsetstrokecolor{currentstroke}%
\pgfsetdash{}{0pt}%
\pgfsys@defobject{currentmarker}{\pgfqpoint{0.000000in}{0.000000in}}{\pgfqpoint{0.000000in}{0.041667in}}{%
\pgfpathmoveto{\pgfqpoint{0.000000in}{0.000000in}}%
\pgfpathlineto{\pgfqpoint{0.000000in}{0.041667in}}%
\pgfusepath{stroke,fill}%
}%
\begin{pgfscope}%
\pgfsys@transformshift{1.580462in}{1.080890in}%
\pgfsys@useobject{currentmarker}{}%
\end{pgfscope}%
\end{pgfscope}%
\begin{pgfscope}%
\pgfsetbuttcap%
\pgfsetroundjoin%
\definecolor{currentfill}{rgb}{0.000000,0.000000,0.000000}%
\pgfsetfillcolor{currentfill}%
\pgfsetlinewidth{0.501875pt}%
\definecolor{currentstroke}{rgb}{0.000000,0.000000,0.000000}%
\pgfsetstrokecolor{currentstroke}%
\pgfsetdash{}{0pt}%
\pgfsys@defobject{currentmarker}{\pgfqpoint{0.000000in}{-0.041667in}}{\pgfqpoint{0.000000in}{0.000000in}}{%
\pgfpathmoveto{\pgfqpoint{0.000000in}{0.000000in}}%
\pgfpathlineto{\pgfqpoint{0.000000in}{-0.041667in}}%
\pgfusepath{stroke,fill}%
}%
\begin{pgfscope}%
\pgfsys@transformshift{1.580462in}{3.227753in}%
\pgfsys@useobject{currentmarker}{}%
\end{pgfscope}%
\end{pgfscope}%
\begin{pgfscope}%
\pgfpathrectangle{\pgfqpoint{0.481681in}{1.080890in}}{\pgfqpoint{5.785672in}{2.146863in}}%
\pgfusepath{clip}%
\pgfsetrectcap%
\pgfsetroundjoin%
\pgfsetlinewidth{0.100375pt}%
\definecolor{currentstroke}{rgb}{0.501961,0.501961,0.501961}%
\pgfsetstrokecolor{currentstroke}%
\pgfsetdash{}{0pt}%
\pgfpathmoveto{\pgfqpoint{1.986791in}{1.080890in}}%
\pgfpathlineto{\pgfqpoint{1.986791in}{3.227753in}}%
\pgfusepath{stroke}%
\end{pgfscope}%
\begin{pgfscope}%
\pgfsetbuttcap%
\pgfsetroundjoin%
\definecolor{currentfill}{rgb}{0.000000,0.000000,0.000000}%
\pgfsetfillcolor{currentfill}%
\pgfsetlinewidth{0.501875pt}%
\definecolor{currentstroke}{rgb}{0.000000,0.000000,0.000000}%
\pgfsetstrokecolor{currentstroke}%
\pgfsetdash{}{0pt}%
\pgfsys@defobject{currentmarker}{\pgfqpoint{0.000000in}{0.000000in}}{\pgfqpoint{0.000000in}{0.041667in}}{%
\pgfpathmoveto{\pgfqpoint{0.000000in}{0.000000in}}%
\pgfpathlineto{\pgfqpoint{0.000000in}{0.041667in}}%
\pgfusepath{stroke,fill}%
}%
\begin{pgfscope}%
\pgfsys@transformshift{1.986791in}{1.080890in}%
\pgfsys@useobject{currentmarker}{}%
\end{pgfscope}%
\end{pgfscope}%
\begin{pgfscope}%
\pgfsetbuttcap%
\pgfsetroundjoin%
\definecolor{currentfill}{rgb}{0.000000,0.000000,0.000000}%
\pgfsetfillcolor{currentfill}%
\pgfsetlinewidth{0.501875pt}%
\definecolor{currentstroke}{rgb}{0.000000,0.000000,0.000000}%
\pgfsetstrokecolor{currentstroke}%
\pgfsetdash{}{0pt}%
\pgfsys@defobject{currentmarker}{\pgfqpoint{0.000000in}{-0.041667in}}{\pgfqpoint{0.000000in}{0.000000in}}{%
\pgfpathmoveto{\pgfqpoint{0.000000in}{0.000000in}}%
\pgfpathlineto{\pgfqpoint{0.000000in}{-0.041667in}}%
\pgfusepath{stroke,fill}%
}%
\begin{pgfscope}%
\pgfsys@transformshift{1.986791in}{3.227753in}%
\pgfsys@useobject{currentmarker}{}%
\end{pgfscope}%
\end{pgfscope}%
\begin{pgfscope}%
\pgfpathrectangle{\pgfqpoint{0.481681in}{1.080890in}}{\pgfqpoint{5.785672in}{2.146863in}}%
\pgfusepath{clip}%
\pgfsetrectcap%
\pgfsetroundjoin%
\pgfsetlinewidth{0.100375pt}%
\definecolor{currentstroke}{rgb}{0.501961,0.501961,0.501961}%
\pgfsetstrokecolor{currentstroke}%
\pgfsetdash{}{0pt}%
\pgfpathmoveto{\pgfqpoint{2.393120in}{1.080890in}}%
\pgfpathlineto{\pgfqpoint{2.393120in}{3.227753in}}%
\pgfusepath{stroke}%
\end{pgfscope}%
\begin{pgfscope}%
\pgfsetbuttcap%
\pgfsetroundjoin%
\definecolor{currentfill}{rgb}{0.000000,0.000000,0.000000}%
\pgfsetfillcolor{currentfill}%
\pgfsetlinewidth{0.501875pt}%
\definecolor{currentstroke}{rgb}{0.000000,0.000000,0.000000}%
\pgfsetstrokecolor{currentstroke}%
\pgfsetdash{}{0pt}%
\pgfsys@defobject{currentmarker}{\pgfqpoint{0.000000in}{0.000000in}}{\pgfqpoint{0.000000in}{0.041667in}}{%
\pgfpathmoveto{\pgfqpoint{0.000000in}{0.000000in}}%
\pgfpathlineto{\pgfqpoint{0.000000in}{0.041667in}}%
\pgfusepath{stroke,fill}%
}%
\begin{pgfscope}%
\pgfsys@transformshift{2.393120in}{1.080890in}%
\pgfsys@useobject{currentmarker}{}%
\end{pgfscope}%
\end{pgfscope}%
\begin{pgfscope}%
\pgfsetbuttcap%
\pgfsetroundjoin%
\definecolor{currentfill}{rgb}{0.000000,0.000000,0.000000}%
\pgfsetfillcolor{currentfill}%
\pgfsetlinewidth{0.501875pt}%
\definecolor{currentstroke}{rgb}{0.000000,0.000000,0.000000}%
\pgfsetstrokecolor{currentstroke}%
\pgfsetdash{}{0pt}%
\pgfsys@defobject{currentmarker}{\pgfqpoint{0.000000in}{-0.041667in}}{\pgfqpoint{0.000000in}{0.000000in}}{%
\pgfpathmoveto{\pgfqpoint{0.000000in}{0.000000in}}%
\pgfpathlineto{\pgfqpoint{0.000000in}{-0.041667in}}%
\pgfusepath{stroke,fill}%
}%
\begin{pgfscope}%
\pgfsys@transformshift{2.393120in}{3.227753in}%
\pgfsys@useobject{currentmarker}{}%
\end{pgfscope}%
\end{pgfscope}%
\begin{pgfscope}%
\pgfpathrectangle{\pgfqpoint{0.481681in}{1.080890in}}{\pgfqpoint{5.785672in}{2.146863in}}%
\pgfusepath{clip}%
\pgfsetrectcap%
\pgfsetroundjoin%
\pgfsetlinewidth{0.100375pt}%
\definecolor{currentstroke}{rgb}{0.501961,0.501961,0.501961}%
\pgfsetstrokecolor{currentstroke}%
\pgfsetdash{}{0pt}%
\pgfpathmoveto{\pgfqpoint{2.799449in}{1.080890in}}%
\pgfpathlineto{\pgfqpoint{2.799449in}{3.227753in}}%
\pgfusepath{stroke}%
\end{pgfscope}%
\begin{pgfscope}%
\pgfsetbuttcap%
\pgfsetroundjoin%
\definecolor{currentfill}{rgb}{0.000000,0.000000,0.000000}%
\pgfsetfillcolor{currentfill}%
\pgfsetlinewidth{0.501875pt}%
\definecolor{currentstroke}{rgb}{0.000000,0.000000,0.000000}%
\pgfsetstrokecolor{currentstroke}%
\pgfsetdash{}{0pt}%
\pgfsys@defobject{currentmarker}{\pgfqpoint{0.000000in}{0.000000in}}{\pgfqpoint{0.000000in}{0.041667in}}{%
\pgfpathmoveto{\pgfqpoint{0.000000in}{0.000000in}}%
\pgfpathlineto{\pgfqpoint{0.000000in}{0.041667in}}%
\pgfusepath{stroke,fill}%
}%
\begin{pgfscope}%
\pgfsys@transformshift{2.799449in}{1.080890in}%
\pgfsys@useobject{currentmarker}{}%
\end{pgfscope}%
\end{pgfscope}%
\begin{pgfscope}%
\pgfsetbuttcap%
\pgfsetroundjoin%
\definecolor{currentfill}{rgb}{0.000000,0.000000,0.000000}%
\pgfsetfillcolor{currentfill}%
\pgfsetlinewidth{0.501875pt}%
\definecolor{currentstroke}{rgb}{0.000000,0.000000,0.000000}%
\pgfsetstrokecolor{currentstroke}%
\pgfsetdash{}{0pt}%
\pgfsys@defobject{currentmarker}{\pgfqpoint{0.000000in}{-0.041667in}}{\pgfqpoint{0.000000in}{0.000000in}}{%
\pgfpathmoveto{\pgfqpoint{0.000000in}{0.000000in}}%
\pgfpathlineto{\pgfqpoint{0.000000in}{-0.041667in}}%
\pgfusepath{stroke,fill}%
}%
\begin{pgfscope}%
\pgfsys@transformshift{2.799449in}{3.227753in}%
\pgfsys@useobject{currentmarker}{}%
\end{pgfscope}%
\end{pgfscope}%
\begin{pgfscope}%
\pgfpathrectangle{\pgfqpoint{0.481681in}{1.080890in}}{\pgfqpoint{5.785672in}{2.146863in}}%
\pgfusepath{clip}%
\pgfsetrectcap%
\pgfsetroundjoin%
\pgfsetlinewidth{0.100375pt}%
\definecolor{currentstroke}{rgb}{0.501961,0.501961,0.501961}%
\pgfsetstrokecolor{currentstroke}%
\pgfsetdash{}{0pt}%
\pgfpathmoveto{\pgfqpoint{3.205778in}{1.080890in}}%
\pgfpathlineto{\pgfqpoint{3.205778in}{3.227753in}}%
\pgfusepath{stroke}%
\end{pgfscope}%
\begin{pgfscope}%
\pgfsetbuttcap%
\pgfsetroundjoin%
\definecolor{currentfill}{rgb}{0.000000,0.000000,0.000000}%
\pgfsetfillcolor{currentfill}%
\pgfsetlinewidth{0.501875pt}%
\definecolor{currentstroke}{rgb}{0.000000,0.000000,0.000000}%
\pgfsetstrokecolor{currentstroke}%
\pgfsetdash{}{0pt}%
\pgfsys@defobject{currentmarker}{\pgfqpoint{0.000000in}{0.000000in}}{\pgfqpoint{0.000000in}{0.041667in}}{%
\pgfpathmoveto{\pgfqpoint{0.000000in}{0.000000in}}%
\pgfpathlineto{\pgfqpoint{0.000000in}{0.041667in}}%
\pgfusepath{stroke,fill}%
}%
\begin{pgfscope}%
\pgfsys@transformshift{3.205778in}{1.080890in}%
\pgfsys@useobject{currentmarker}{}%
\end{pgfscope}%
\end{pgfscope}%
\begin{pgfscope}%
\pgfsetbuttcap%
\pgfsetroundjoin%
\definecolor{currentfill}{rgb}{0.000000,0.000000,0.000000}%
\pgfsetfillcolor{currentfill}%
\pgfsetlinewidth{0.501875pt}%
\definecolor{currentstroke}{rgb}{0.000000,0.000000,0.000000}%
\pgfsetstrokecolor{currentstroke}%
\pgfsetdash{}{0pt}%
\pgfsys@defobject{currentmarker}{\pgfqpoint{0.000000in}{-0.041667in}}{\pgfqpoint{0.000000in}{0.000000in}}{%
\pgfpathmoveto{\pgfqpoint{0.000000in}{0.000000in}}%
\pgfpathlineto{\pgfqpoint{0.000000in}{-0.041667in}}%
\pgfusepath{stroke,fill}%
}%
\begin{pgfscope}%
\pgfsys@transformshift{3.205778in}{3.227753in}%
\pgfsys@useobject{currentmarker}{}%
\end{pgfscope}%
\end{pgfscope}%
\begin{pgfscope}%
\pgfpathrectangle{\pgfqpoint{0.481681in}{1.080890in}}{\pgfqpoint{5.785672in}{2.146863in}}%
\pgfusepath{clip}%
\pgfsetrectcap%
\pgfsetroundjoin%
\pgfsetlinewidth{0.100375pt}%
\definecolor{currentstroke}{rgb}{0.501961,0.501961,0.501961}%
\pgfsetstrokecolor{currentstroke}%
\pgfsetdash{}{0pt}%
\pgfpathmoveto{\pgfqpoint{3.612106in}{1.080890in}}%
\pgfpathlineto{\pgfqpoint{3.612106in}{3.227753in}}%
\pgfusepath{stroke}%
\end{pgfscope}%
\begin{pgfscope}%
\pgfsetbuttcap%
\pgfsetroundjoin%
\definecolor{currentfill}{rgb}{0.000000,0.000000,0.000000}%
\pgfsetfillcolor{currentfill}%
\pgfsetlinewidth{0.501875pt}%
\definecolor{currentstroke}{rgb}{0.000000,0.000000,0.000000}%
\pgfsetstrokecolor{currentstroke}%
\pgfsetdash{}{0pt}%
\pgfsys@defobject{currentmarker}{\pgfqpoint{0.000000in}{0.000000in}}{\pgfqpoint{0.000000in}{0.041667in}}{%
\pgfpathmoveto{\pgfqpoint{0.000000in}{0.000000in}}%
\pgfpathlineto{\pgfqpoint{0.000000in}{0.041667in}}%
\pgfusepath{stroke,fill}%
}%
\begin{pgfscope}%
\pgfsys@transformshift{3.612106in}{1.080890in}%
\pgfsys@useobject{currentmarker}{}%
\end{pgfscope}%
\end{pgfscope}%
\begin{pgfscope}%
\pgfsetbuttcap%
\pgfsetroundjoin%
\definecolor{currentfill}{rgb}{0.000000,0.000000,0.000000}%
\pgfsetfillcolor{currentfill}%
\pgfsetlinewidth{0.501875pt}%
\definecolor{currentstroke}{rgb}{0.000000,0.000000,0.000000}%
\pgfsetstrokecolor{currentstroke}%
\pgfsetdash{}{0pt}%
\pgfsys@defobject{currentmarker}{\pgfqpoint{0.000000in}{-0.041667in}}{\pgfqpoint{0.000000in}{0.000000in}}{%
\pgfpathmoveto{\pgfqpoint{0.000000in}{0.000000in}}%
\pgfpathlineto{\pgfqpoint{0.000000in}{-0.041667in}}%
\pgfusepath{stroke,fill}%
}%
\begin{pgfscope}%
\pgfsys@transformshift{3.612106in}{3.227753in}%
\pgfsys@useobject{currentmarker}{}%
\end{pgfscope}%
\end{pgfscope}%
\begin{pgfscope}%
\pgfpathrectangle{\pgfqpoint{0.481681in}{1.080890in}}{\pgfqpoint{5.785672in}{2.146863in}}%
\pgfusepath{clip}%
\pgfsetrectcap%
\pgfsetroundjoin%
\pgfsetlinewidth{0.100375pt}%
\definecolor{currentstroke}{rgb}{0.501961,0.501961,0.501961}%
\pgfsetstrokecolor{currentstroke}%
\pgfsetdash{}{0pt}%
\pgfpathmoveto{\pgfqpoint{4.018435in}{1.080890in}}%
\pgfpathlineto{\pgfqpoint{4.018435in}{3.227753in}}%
\pgfusepath{stroke}%
\end{pgfscope}%
\begin{pgfscope}%
\pgfsetbuttcap%
\pgfsetroundjoin%
\definecolor{currentfill}{rgb}{0.000000,0.000000,0.000000}%
\pgfsetfillcolor{currentfill}%
\pgfsetlinewidth{0.501875pt}%
\definecolor{currentstroke}{rgb}{0.000000,0.000000,0.000000}%
\pgfsetstrokecolor{currentstroke}%
\pgfsetdash{}{0pt}%
\pgfsys@defobject{currentmarker}{\pgfqpoint{0.000000in}{0.000000in}}{\pgfqpoint{0.000000in}{0.041667in}}{%
\pgfpathmoveto{\pgfqpoint{0.000000in}{0.000000in}}%
\pgfpathlineto{\pgfqpoint{0.000000in}{0.041667in}}%
\pgfusepath{stroke,fill}%
}%
\begin{pgfscope}%
\pgfsys@transformshift{4.018435in}{1.080890in}%
\pgfsys@useobject{currentmarker}{}%
\end{pgfscope}%
\end{pgfscope}%
\begin{pgfscope}%
\pgfsetbuttcap%
\pgfsetroundjoin%
\definecolor{currentfill}{rgb}{0.000000,0.000000,0.000000}%
\pgfsetfillcolor{currentfill}%
\pgfsetlinewidth{0.501875pt}%
\definecolor{currentstroke}{rgb}{0.000000,0.000000,0.000000}%
\pgfsetstrokecolor{currentstroke}%
\pgfsetdash{}{0pt}%
\pgfsys@defobject{currentmarker}{\pgfqpoint{0.000000in}{-0.041667in}}{\pgfqpoint{0.000000in}{0.000000in}}{%
\pgfpathmoveto{\pgfqpoint{0.000000in}{0.000000in}}%
\pgfpathlineto{\pgfqpoint{0.000000in}{-0.041667in}}%
\pgfusepath{stroke,fill}%
}%
\begin{pgfscope}%
\pgfsys@transformshift{4.018435in}{3.227753in}%
\pgfsys@useobject{currentmarker}{}%
\end{pgfscope}%
\end{pgfscope}%
\begin{pgfscope}%
\pgfpathrectangle{\pgfqpoint{0.481681in}{1.080890in}}{\pgfqpoint{5.785672in}{2.146863in}}%
\pgfusepath{clip}%
\pgfsetrectcap%
\pgfsetroundjoin%
\pgfsetlinewidth{0.100375pt}%
\definecolor{currentstroke}{rgb}{0.501961,0.501961,0.501961}%
\pgfsetstrokecolor{currentstroke}%
\pgfsetdash{}{0pt}%
\pgfpathmoveto{\pgfqpoint{4.424764in}{1.080890in}}%
\pgfpathlineto{\pgfqpoint{4.424764in}{3.227753in}}%
\pgfusepath{stroke}%
\end{pgfscope}%
\begin{pgfscope}%
\pgfsetbuttcap%
\pgfsetroundjoin%
\definecolor{currentfill}{rgb}{0.000000,0.000000,0.000000}%
\pgfsetfillcolor{currentfill}%
\pgfsetlinewidth{0.501875pt}%
\definecolor{currentstroke}{rgb}{0.000000,0.000000,0.000000}%
\pgfsetstrokecolor{currentstroke}%
\pgfsetdash{}{0pt}%
\pgfsys@defobject{currentmarker}{\pgfqpoint{0.000000in}{0.000000in}}{\pgfqpoint{0.000000in}{0.041667in}}{%
\pgfpathmoveto{\pgfqpoint{0.000000in}{0.000000in}}%
\pgfpathlineto{\pgfqpoint{0.000000in}{0.041667in}}%
\pgfusepath{stroke,fill}%
}%
\begin{pgfscope}%
\pgfsys@transformshift{4.424764in}{1.080890in}%
\pgfsys@useobject{currentmarker}{}%
\end{pgfscope}%
\end{pgfscope}%
\begin{pgfscope}%
\pgfsetbuttcap%
\pgfsetroundjoin%
\definecolor{currentfill}{rgb}{0.000000,0.000000,0.000000}%
\pgfsetfillcolor{currentfill}%
\pgfsetlinewidth{0.501875pt}%
\definecolor{currentstroke}{rgb}{0.000000,0.000000,0.000000}%
\pgfsetstrokecolor{currentstroke}%
\pgfsetdash{}{0pt}%
\pgfsys@defobject{currentmarker}{\pgfqpoint{0.000000in}{-0.041667in}}{\pgfqpoint{0.000000in}{0.000000in}}{%
\pgfpathmoveto{\pgfqpoint{0.000000in}{0.000000in}}%
\pgfpathlineto{\pgfqpoint{0.000000in}{-0.041667in}}%
\pgfusepath{stroke,fill}%
}%
\begin{pgfscope}%
\pgfsys@transformshift{4.424764in}{3.227753in}%
\pgfsys@useobject{currentmarker}{}%
\end{pgfscope}%
\end{pgfscope}%
\begin{pgfscope}%
\pgfpathrectangle{\pgfqpoint{0.481681in}{1.080890in}}{\pgfqpoint{5.785672in}{2.146863in}}%
\pgfusepath{clip}%
\pgfsetrectcap%
\pgfsetroundjoin%
\pgfsetlinewidth{0.100375pt}%
\definecolor{currentstroke}{rgb}{0.501961,0.501961,0.501961}%
\pgfsetstrokecolor{currentstroke}%
\pgfsetdash{}{0pt}%
\pgfpathmoveto{\pgfqpoint{4.831093in}{1.080890in}}%
\pgfpathlineto{\pgfqpoint{4.831093in}{3.227753in}}%
\pgfusepath{stroke}%
\end{pgfscope}%
\begin{pgfscope}%
\pgfsetbuttcap%
\pgfsetroundjoin%
\definecolor{currentfill}{rgb}{0.000000,0.000000,0.000000}%
\pgfsetfillcolor{currentfill}%
\pgfsetlinewidth{0.501875pt}%
\definecolor{currentstroke}{rgb}{0.000000,0.000000,0.000000}%
\pgfsetstrokecolor{currentstroke}%
\pgfsetdash{}{0pt}%
\pgfsys@defobject{currentmarker}{\pgfqpoint{0.000000in}{0.000000in}}{\pgfqpoint{0.000000in}{0.041667in}}{%
\pgfpathmoveto{\pgfqpoint{0.000000in}{0.000000in}}%
\pgfpathlineto{\pgfqpoint{0.000000in}{0.041667in}}%
\pgfusepath{stroke,fill}%
}%
\begin{pgfscope}%
\pgfsys@transformshift{4.831093in}{1.080890in}%
\pgfsys@useobject{currentmarker}{}%
\end{pgfscope}%
\end{pgfscope}%
\begin{pgfscope}%
\pgfsetbuttcap%
\pgfsetroundjoin%
\definecolor{currentfill}{rgb}{0.000000,0.000000,0.000000}%
\pgfsetfillcolor{currentfill}%
\pgfsetlinewidth{0.501875pt}%
\definecolor{currentstroke}{rgb}{0.000000,0.000000,0.000000}%
\pgfsetstrokecolor{currentstroke}%
\pgfsetdash{}{0pt}%
\pgfsys@defobject{currentmarker}{\pgfqpoint{0.000000in}{-0.041667in}}{\pgfqpoint{0.000000in}{0.000000in}}{%
\pgfpathmoveto{\pgfqpoint{0.000000in}{0.000000in}}%
\pgfpathlineto{\pgfqpoint{0.000000in}{-0.041667in}}%
\pgfusepath{stroke,fill}%
}%
\begin{pgfscope}%
\pgfsys@transformshift{4.831093in}{3.227753in}%
\pgfsys@useobject{currentmarker}{}%
\end{pgfscope}%
\end{pgfscope}%
\begin{pgfscope}%
\pgfpathrectangle{\pgfqpoint{0.481681in}{1.080890in}}{\pgfqpoint{5.785672in}{2.146863in}}%
\pgfusepath{clip}%
\pgfsetrectcap%
\pgfsetroundjoin%
\pgfsetlinewidth{0.100375pt}%
\definecolor{currentstroke}{rgb}{0.501961,0.501961,0.501961}%
\pgfsetstrokecolor{currentstroke}%
\pgfsetdash{}{0pt}%
\pgfpathmoveto{\pgfqpoint{5.237422in}{1.080890in}}%
\pgfpathlineto{\pgfqpoint{5.237422in}{3.227753in}}%
\pgfusepath{stroke}%
\end{pgfscope}%
\begin{pgfscope}%
\pgfsetbuttcap%
\pgfsetroundjoin%
\definecolor{currentfill}{rgb}{0.000000,0.000000,0.000000}%
\pgfsetfillcolor{currentfill}%
\pgfsetlinewidth{0.501875pt}%
\definecolor{currentstroke}{rgb}{0.000000,0.000000,0.000000}%
\pgfsetstrokecolor{currentstroke}%
\pgfsetdash{}{0pt}%
\pgfsys@defobject{currentmarker}{\pgfqpoint{0.000000in}{0.000000in}}{\pgfqpoint{0.000000in}{0.041667in}}{%
\pgfpathmoveto{\pgfqpoint{0.000000in}{0.000000in}}%
\pgfpathlineto{\pgfqpoint{0.000000in}{0.041667in}}%
\pgfusepath{stroke,fill}%
}%
\begin{pgfscope}%
\pgfsys@transformshift{5.237422in}{1.080890in}%
\pgfsys@useobject{currentmarker}{}%
\end{pgfscope}%
\end{pgfscope}%
\begin{pgfscope}%
\pgfsetbuttcap%
\pgfsetroundjoin%
\definecolor{currentfill}{rgb}{0.000000,0.000000,0.000000}%
\pgfsetfillcolor{currentfill}%
\pgfsetlinewidth{0.501875pt}%
\definecolor{currentstroke}{rgb}{0.000000,0.000000,0.000000}%
\pgfsetstrokecolor{currentstroke}%
\pgfsetdash{}{0pt}%
\pgfsys@defobject{currentmarker}{\pgfqpoint{0.000000in}{-0.041667in}}{\pgfqpoint{0.000000in}{0.000000in}}{%
\pgfpathmoveto{\pgfqpoint{0.000000in}{0.000000in}}%
\pgfpathlineto{\pgfqpoint{0.000000in}{-0.041667in}}%
\pgfusepath{stroke,fill}%
}%
\begin{pgfscope}%
\pgfsys@transformshift{5.237422in}{3.227753in}%
\pgfsys@useobject{currentmarker}{}%
\end{pgfscope}%
\end{pgfscope}%
\begin{pgfscope}%
\pgfpathrectangle{\pgfqpoint{0.481681in}{1.080890in}}{\pgfqpoint{5.785672in}{2.146863in}}%
\pgfusepath{clip}%
\pgfsetrectcap%
\pgfsetroundjoin%
\pgfsetlinewidth{0.100375pt}%
\definecolor{currentstroke}{rgb}{0.501961,0.501961,0.501961}%
\pgfsetstrokecolor{currentstroke}%
\pgfsetdash{}{0pt}%
\pgfpathmoveto{\pgfqpoint{5.643751in}{1.080890in}}%
\pgfpathlineto{\pgfqpoint{5.643751in}{3.227753in}}%
\pgfusepath{stroke}%
\end{pgfscope}%
\begin{pgfscope}%
\pgfsetbuttcap%
\pgfsetroundjoin%
\definecolor{currentfill}{rgb}{0.000000,0.000000,0.000000}%
\pgfsetfillcolor{currentfill}%
\pgfsetlinewidth{0.501875pt}%
\definecolor{currentstroke}{rgb}{0.000000,0.000000,0.000000}%
\pgfsetstrokecolor{currentstroke}%
\pgfsetdash{}{0pt}%
\pgfsys@defobject{currentmarker}{\pgfqpoint{0.000000in}{0.000000in}}{\pgfqpoint{0.000000in}{0.041667in}}{%
\pgfpathmoveto{\pgfqpoint{0.000000in}{0.000000in}}%
\pgfpathlineto{\pgfqpoint{0.000000in}{0.041667in}}%
\pgfusepath{stroke,fill}%
}%
\begin{pgfscope}%
\pgfsys@transformshift{5.643751in}{1.080890in}%
\pgfsys@useobject{currentmarker}{}%
\end{pgfscope}%
\end{pgfscope}%
\begin{pgfscope}%
\pgfsetbuttcap%
\pgfsetroundjoin%
\definecolor{currentfill}{rgb}{0.000000,0.000000,0.000000}%
\pgfsetfillcolor{currentfill}%
\pgfsetlinewidth{0.501875pt}%
\definecolor{currentstroke}{rgb}{0.000000,0.000000,0.000000}%
\pgfsetstrokecolor{currentstroke}%
\pgfsetdash{}{0pt}%
\pgfsys@defobject{currentmarker}{\pgfqpoint{0.000000in}{-0.041667in}}{\pgfqpoint{0.000000in}{0.000000in}}{%
\pgfpathmoveto{\pgfqpoint{0.000000in}{0.000000in}}%
\pgfpathlineto{\pgfqpoint{0.000000in}{-0.041667in}}%
\pgfusepath{stroke,fill}%
}%
\begin{pgfscope}%
\pgfsys@transformshift{5.643751in}{3.227753in}%
\pgfsys@useobject{currentmarker}{}%
\end{pgfscope}%
\end{pgfscope}%
\begin{pgfscope}%
\pgfpathrectangle{\pgfqpoint{0.481681in}{1.080890in}}{\pgfqpoint{5.785672in}{2.146863in}}%
\pgfusepath{clip}%
\pgfsetrectcap%
\pgfsetroundjoin%
\pgfsetlinewidth{0.100375pt}%
\definecolor{currentstroke}{rgb}{0.501961,0.501961,0.501961}%
\pgfsetstrokecolor{currentstroke}%
\pgfsetdash{}{0pt}%
\pgfpathmoveto{\pgfqpoint{6.050080in}{1.080890in}}%
\pgfpathlineto{\pgfqpoint{6.050080in}{3.227753in}}%
\pgfusepath{stroke}%
\end{pgfscope}%
\begin{pgfscope}%
\pgfsetbuttcap%
\pgfsetroundjoin%
\definecolor{currentfill}{rgb}{0.000000,0.000000,0.000000}%
\pgfsetfillcolor{currentfill}%
\pgfsetlinewidth{0.501875pt}%
\definecolor{currentstroke}{rgb}{0.000000,0.000000,0.000000}%
\pgfsetstrokecolor{currentstroke}%
\pgfsetdash{}{0pt}%
\pgfsys@defobject{currentmarker}{\pgfqpoint{0.000000in}{0.000000in}}{\pgfqpoint{0.000000in}{0.041667in}}{%
\pgfpathmoveto{\pgfqpoint{0.000000in}{0.000000in}}%
\pgfpathlineto{\pgfqpoint{0.000000in}{0.041667in}}%
\pgfusepath{stroke,fill}%
}%
\begin{pgfscope}%
\pgfsys@transformshift{6.050080in}{1.080890in}%
\pgfsys@useobject{currentmarker}{}%
\end{pgfscope}%
\end{pgfscope}%
\begin{pgfscope}%
\pgfsetbuttcap%
\pgfsetroundjoin%
\definecolor{currentfill}{rgb}{0.000000,0.000000,0.000000}%
\pgfsetfillcolor{currentfill}%
\pgfsetlinewidth{0.501875pt}%
\definecolor{currentstroke}{rgb}{0.000000,0.000000,0.000000}%
\pgfsetstrokecolor{currentstroke}%
\pgfsetdash{}{0pt}%
\pgfsys@defobject{currentmarker}{\pgfqpoint{0.000000in}{-0.041667in}}{\pgfqpoint{0.000000in}{0.000000in}}{%
\pgfpathmoveto{\pgfqpoint{0.000000in}{0.000000in}}%
\pgfpathlineto{\pgfqpoint{0.000000in}{-0.041667in}}%
\pgfusepath{stroke,fill}%
}%
\begin{pgfscope}%
\pgfsys@transformshift{6.050080in}{3.227753in}%
\pgfsys@useobject{currentmarker}{}%
\end{pgfscope}%
\end{pgfscope}%
\begin{pgfscope}%
\pgfpathrectangle{\pgfqpoint{0.481681in}{1.080890in}}{\pgfqpoint{5.785672in}{2.146863in}}%
\pgfusepath{clip}%
\pgfsetrectcap%
\pgfsetroundjoin%
\pgfsetlinewidth{0.100375pt}%
\definecolor{currentstroke}{rgb}{0.827451,0.827451,0.827451}%
\pgfsetstrokecolor{currentstroke}%
\pgfsetdash{}{0pt}%
\pgfpathmoveto{\pgfqpoint{0.496918in}{1.080890in}}%
\pgfpathlineto{\pgfqpoint{0.496918in}{3.227753in}}%
\pgfusepath{stroke}%
\end{pgfscope}%
\begin{pgfscope}%
\pgfsetbuttcap%
\pgfsetroundjoin%
\definecolor{currentfill}{rgb}{0.000000,0.000000,0.000000}%
\pgfsetfillcolor{currentfill}%
\pgfsetlinewidth{0.501875pt}%
\definecolor{currentstroke}{rgb}{0.000000,0.000000,0.000000}%
\pgfsetstrokecolor{currentstroke}%
\pgfsetdash{}{0pt}%
\pgfsys@defobject{currentmarker}{\pgfqpoint{0.000000in}{0.000000in}}{\pgfqpoint{0.000000in}{0.020833in}}{%
\pgfpathmoveto{\pgfqpoint{0.000000in}{0.000000in}}%
\pgfpathlineto{\pgfqpoint{0.000000in}{0.020833in}}%
\pgfusepath{stroke,fill}%
}%
\begin{pgfscope}%
\pgfsys@transformshift{0.496918in}{1.080890in}%
\pgfsys@useobject{currentmarker}{}%
\end{pgfscope}%
\end{pgfscope}%
\begin{pgfscope}%
\pgfsetbuttcap%
\pgfsetroundjoin%
\definecolor{currentfill}{rgb}{0.000000,0.000000,0.000000}%
\pgfsetfillcolor{currentfill}%
\pgfsetlinewidth{0.501875pt}%
\definecolor{currentstroke}{rgb}{0.000000,0.000000,0.000000}%
\pgfsetstrokecolor{currentstroke}%
\pgfsetdash{}{0pt}%
\pgfsys@defobject{currentmarker}{\pgfqpoint{0.000000in}{-0.020833in}}{\pgfqpoint{0.000000in}{0.000000in}}{%
\pgfpathmoveto{\pgfqpoint{0.000000in}{0.000000in}}%
\pgfpathlineto{\pgfqpoint{0.000000in}{-0.020833in}}%
\pgfusepath{stroke,fill}%
}%
\begin{pgfscope}%
\pgfsys@transformshift{0.496918in}{3.227753in}%
\pgfsys@useobject{currentmarker}{}%
\end{pgfscope}%
\end{pgfscope}%
\begin{pgfscope}%
\pgfpathrectangle{\pgfqpoint{0.481681in}{1.080890in}}{\pgfqpoint{5.785672in}{2.146863in}}%
\pgfusepath{clip}%
\pgfsetrectcap%
\pgfsetroundjoin%
\pgfsetlinewidth{0.100375pt}%
\definecolor{currentstroke}{rgb}{0.827451,0.827451,0.827451}%
\pgfsetstrokecolor{currentstroke}%
\pgfsetdash{}{0pt}%
\pgfpathmoveto{\pgfqpoint{0.564640in}{1.080890in}}%
\pgfpathlineto{\pgfqpoint{0.564640in}{3.227753in}}%
\pgfusepath{stroke}%
\end{pgfscope}%
\begin{pgfscope}%
\pgfsetbuttcap%
\pgfsetroundjoin%
\definecolor{currentfill}{rgb}{0.000000,0.000000,0.000000}%
\pgfsetfillcolor{currentfill}%
\pgfsetlinewidth{0.501875pt}%
\definecolor{currentstroke}{rgb}{0.000000,0.000000,0.000000}%
\pgfsetstrokecolor{currentstroke}%
\pgfsetdash{}{0pt}%
\pgfsys@defobject{currentmarker}{\pgfqpoint{0.000000in}{0.000000in}}{\pgfqpoint{0.000000in}{0.020833in}}{%
\pgfpathmoveto{\pgfqpoint{0.000000in}{0.000000in}}%
\pgfpathlineto{\pgfqpoint{0.000000in}{0.020833in}}%
\pgfusepath{stroke,fill}%
}%
\begin{pgfscope}%
\pgfsys@transformshift{0.564640in}{1.080890in}%
\pgfsys@useobject{currentmarker}{}%
\end{pgfscope}%
\end{pgfscope}%
\begin{pgfscope}%
\pgfsetbuttcap%
\pgfsetroundjoin%
\definecolor{currentfill}{rgb}{0.000000,0.000000,0.000000}%
\pgfsetfillcolor{currentfill}%
\pgfsetlinewidth{0.501875pt}%
\definecolor{currentstroke}{rgb}{0.000000,0.000000,0.000000}%
\pgfsetstrokecolor{currentstroke}%
\pgfsetdash{}{0pt}%
\pgfsys@defobject{currentmarker}{\pgfqpoint{0.000000in}{-0.020833in}}{\pgfqpoint{0.000000in}{0.000000in}}{%
\pgfpathmoveto{\pgfqpoint{0.000000in}{0.000000in}}%
\pgfpathlineto{\pgfqpoint{0.000000in}{-0.020833in}}%
\pgfusepath{stroke,fill}%
}%
\begin{pgfscope}%
\pgfsys@transformshift{0.564640in}{3.227753in}%
\pgfsys@useobject{currentmarker}{}%
\end{pgfscope}%
\end{pgfscope}%
\begin{pgfscope}%
\pgfpathrectangle{\pgfqpoint{0.481681in}{1.080890in}}{\pgfqpoint{5.785672in}{2.146863in}}%
\pgfusepath{clip}%
\pgfsetrectcap%
\pgfsetroundjoin%
\pgfsetlinewidth{0.100375pt}%
\definecolor{currentstroke}{rgb}{0.827451,0.827451,0.827451}%
\pgfsetstrokecolor{currentstroke}%
\pgfsetdash{}{0pt}%
\pgfpathmoveto{\pgfqpoint{0.632361in}{1.080890in}}%
\pgfpathlineto{\pgfqpoint{0.632361in}{3.227753in}}%
\pgfusepath{stroke}%
\end{pgfscope}%
\begin{pgfscope}%
\pgfsetbuttcap%
\pgfsetroundjoin%
\definecolor{currentfill}{rgb}{0.000000,0.000000,0.000000}%
\pgfsetfillcolor{currentfill}%
\pgfsetlinewidth{0.501875pt}%
\definecolor{currentstroke}{rgb}{0.000000,0.000000,0.000000}%
\pgfsetstrokecolor{currentstroke}%
\pgfsetdash{}{0pt}%
\pgfsys@defobject{currentmarker}{\pgfqpoint{0.000000in}{0.000000in}}{\pgfqpoint{0.000000in}{0.020833in}}{%
\pgfpathmoveto{\pgfqpoint{0.000000in}{0.000000in}}%
\pgfpathlineto{\pgfqpoint{0.000000in}{0.020833in}}%
\pgfusepath{stroke,fill}%
}%
\begin{pgfscope}%
\pgfsys@transformshift{0.632361in}{1.080890in}%
\pgfsys@useobject{currentmarker}{}%
\end{pgfscope}%
\end{pgfscope}%
\begin{pgfscope}%
\pgfsetbuttcap%
\pgfsetroundjoin%
\definecolor{currentfill}{rgb}{0.000000,0.000000,0.000000}%
\pgfsetfillcolor{currentfill}%
\pgfsetlinewidth{0.501875pt}%
\definecolor{currentstroke}{rgb}{0.000000,0.000000,0.000000}%
\pgfsetstrokecolor{currentstroke}%
\pgfsetdash{}{0pt}%
\pgfsys@defobject{currentmarker}{\pgfqpoint{0.000000in}{-0.020833in}}{\pgfqpoint{0.000000in}{0.000000in}}{%
\pgfpathmoveto{\pgfqpoint{0.000000in}{0.000000in}}%
\pgfpathlineto{\pgfqpoint{0.000000in}{-0.020833in}}%
\pgfusepath{stroke,fill}%
}%
\begin{pgfscope}%
\pgfsys@transformshift{0.632361in}{3.227753in}%
\pgfsys@useobject{currentmarker}{}%
\end{pgfscope}%
\end{pgfscope}%
\begin{pgfscope}%
\pgfpathrectangle{\pgfqpoint{0.481681in}{1.080890in}}{\pgfqpoint{5.785672in}{2.146863in}}%
\pgfusepath{clip}%
\pgfsetrectcap%
\pgfsetroundjoin%
\pgfsetlinewidth{0.100375pt}%
\definecolor{currentstroke}{rgb}{0.827451,0.827451,0.827451}%
\pgfsetstrokecolor{currentstroke}%
\pgfsetdash{}{0pt}%
\pgfpathmoveto{\pgfqpoint{0.700083in}{1.080890in}}%
\pgfpathlineto{\pgfqpoint{0.700083in}{3.227753in}}%
\pgfusepath{stroke}%
\end{pgfscope}%
\begin{pgfscope}%
\pgfsetbuttcap%
\pgfsetroundjoin%
\definecolor{currentfill}{rgb}{0.000000,0.000000,0.000000}%
\pgfsetfillcolor{currentfill}%
\pgfsetlinewidth{0.501875pt}%
\definecolor{currentstroke}{rgb}{0.000000,0.000000,0.000000}%
\pgfsetstrokecolor{currentstroke}%
\pgfsetdash{}{0pt}%
\pgfsys@defobject{currentmarker}{\pgfqpoint{0.000000in}{0.000000in}}{\pgfqpoint{0.000000in}{0.020833in}}{%
\pgfpathmoveto{\pgfqpoint{0.000000in}{0.000000in}}%
\pgfpathlineto{\pgfqpoint{0.000000in}{0.020833in}}%
\pgfusepath{stroke,fill}%
}%
\begin{pgfscope}%
\pgfsys@transformshift{0.700083in}{1.080890in}%
\pgfsys@useobject{currentmarker}{}%
\end{pgfscope}%
\end{pgfscope}%
\begin{pgfscope}%
\pgfsetbuttcap%
\pgfsetroundjoin%
\definecolor{currentfill}{rgb}{0.000000,0.000000,0.000000}%
\pgfsetfillcolor{currentfill}%
\pgfsetlinewidth{0.501875pt}%
\definecolor{currentstroke}{rgb}{0.000000,0.000000,0.000000}%
\pgfsetstrokecolor{currentstroke}%
\pgfsetdash{}{0pt}%
\pgfsys@defobject{currentmarker}{\pgfqpoint{0.000000in}{-0.020833in}}{\pgfqpoint{0.000000in}{0.000000in}}{%
\pgfpathmoveto{\pgfqpoint{0.000000in}{0.000000in}}%
\pgfpathlineto{\pgfqpoint{0.000000in}{-0.020833in}}%
\pgfusepath{stroke,fill}%
}%
\begin{pgfscope}%
\pgfsys@transformshift{0.700083in}{3.227753in}%
\pgfsys@useobject{currentmarker}{}%
\end{pgfscope}%
\end{pgfscope}%
\begin{pgfscope}%
\pgfpathrectangle{\pgfqpoint{0.481681in}{1.080890in}}{\pgfqpoint{5.785672in}{2.146863in}}%
\pgfusepath{clip}%
\pgfsetrectcap%
\pgfsetroundjoin%
\pgfsetlinewidth{0.100375pt}%
\definecolor{currentstroke}{rgb}{0.827451,0.827451,0.827451}%
\pgfsetstrokecolor{currentstroke}%
\pgfsetdash{}{0pt}%
\pgfpathmoveto{\pgfqpoint{0.835526in}{1.080890in}}%
\pgfpathlineto{\pgfqpoint{0.835526in}{3.227753in}}%
\pgfusepath{stroke}%
\end{pgfscope}%
\begin{pgfscope}%
\pgfsetbuttcap%
\pgfsetroundjoin%
\definecolor{currentfill}{rgb}{0.000000,0.000000,0.000000}%
\pgfsetfillcolor{currentfill}%
\pgfsetlinewidth{0.501875pt}%
\definecolor{currentstroke}{rgb}{0.000000,0.000000,0.000000}%
\pgfsetstrokecolor{currentstroke}%
\pgfsetdash{}{0pt}%
\pgfsys@defobject{currentmarker}{\pgfqpoint{0.000000in}{0.000000in}}{\pgfqpoint{0.000000in}{0.020833in}}{%
\pgfpathmoveto{\pgfqpoint{0.000000in}{0.000000in}}%
\pgfpathlineto{\pgfqpoint{0.000000in}{0.020833in}}%
\pgfusepath{stroke,fill}%
}%
\begin{pgfscope}%
\pgfsys@transformshift{0.835526in}{1.080890in}%
\pgfsys@useobject{currentmarker}{}%
\end{pgfscope}%
\end{pgfscope}%
\begin{pgfscope}%
\pgfsetbuttcap%
\pgfsetroundjoin%
\definecolor{currentfill}{rgb}{0.000000,0.000000,0.000000}%
\pgfsetfillcolor{currentfill}%
\pgfsetlinewidth{0.501875pt}%
\definecolor{currentstroke}{rgb}{0.000000,0.000000,0.000000}%
\pgfsetstrokecolor{currentstroke}%
\pgfsetdash{}{0pt}%
\pgfsys@defobject{currentmarker}{\pgfqpoint{0.000000in}{-0.020833in}}{\pgfqpoint{0.000000in}{0.000000in}}{%
\pgfpathmoveto{\pgfqpoint{0.000000in}{0.000000in}}%
\pgfpathlineto{\pgfqpoint{0.000000in}{-0.020833in}}%
\pgfusepath{stroke,fill}%
}%
\begin{pgfscope}%
\pgfsys@transformshift{0.835526in}{3.227753in}%
\pgfsys@useobject{currentmarker}{}%
\end{pgfscope}%
\end{pgfscope}%
\begin{pgfscope}%
\pgfpathrectangle{\pgfqpoint{0.481681in}{1.080890in}}{\pgfqpoint{5.785672in}{2.146863in}}%
\pgfusepath{clip}%
\pgfsetrectcap%
\pgfsetroundjoin%
\pgfsetlinewidth{0.100375pt}%
\definecolor{currentstroke}{rgb}{0.827451,0.827451,0.827451}%
\pgfsetstrokecolor{currentstroke}%
\pgfsetdash{}{0pt}%
\pgfpathmoveto{\pgfqpoint{0.903247in}{1.080890in}}%
\pgfpathlineto{\pgfqpoint{0.903247in}{3.227753in}}%
\pgfusepath{stroke}%
\end{pgfscope}%
\begin{pgfscope}%
\pgfsetbuttcap%
\pgfsetroundjoin%
\definecolor{currentfill}{rgb}{0.000000,0.000000,0.000000}%
\pgfsetfillcolor{currentfill}%
\pgfsetlinewidth{0.501875pt}%
\definecolor{currentstroke}{rgb}{0.000000,0.000000,0.000000}%
\pgfsetstrokecolor{currentstroke}%
\pgfsetdash{}{0pt}%
\pgfsys@defobject{currentmarker}{\pgfqpoint{0.000000in}{0.000000in}}{\pgfqpoint{0.000000in}{0.020833in}}{%
\pgfpathmoveto{\pgfqpoint{0.000000in}{0.000000in}}%
\pgfpathlineto{\pgfqpoint{0.000000in}{0.020833in}}%
\pgfusepath{stroke,fill}%
}%
\begin{pgfscope}%
\pgfsys@transformshift{0.903247in}{1.080890in}%
\pgfsys@useobject{currentmarker}{}%
\end{pgfscope}%
\end{pgfscope}%
\begin{pgfscope}%
\pgfsetbuttcap%
\pgfsetroundjoin%
\definecolor{currentfill}{rgb}{0.000000,0.000000,0.000000}%
\pgfsetfillcolor{currentfill}%
\pgfsetlinewidth{0.501875pt}%
\definecolor{currentstroke}{rgb}{0.000000,0.000000,0.000000}%
\pgfsetstrokecolor{currentstroke}%
\pgfsetdash{}{0pt}%
\pgfsys@defobject{currentmarker}{\pgfqpoint{0.000000in}{-0.020833in}}{\pgfqpoint{0.000000in}{0.000000in}}{%
\pgfpathmoveto{\pgfqpoint{0.000000in}{0.000000in}}%
\pgfpathlineto{\pgfqpoint{0.000000in}{-0.020833in}}%
\pgfusepath{stroke,fill}%
}%
\begin{pgfscope}%
\pgfsys@transformshift{0.903247in}{3.227753in}%
\pgfsys@useobject{currentmarker}{}%
\end{pgfscope}%
\end{pgfscope}%
\begin{pgfscope}%
\pgfpathrectangle{\pgfqpoint{0.481681in}{1.080890in}}{\pgfqpoint{5.785672in}{2.146863in}}%
\pgfusepath{clip}%
\pgfsetrectcap%
\pgfsetroundjoin%
\pgfsetlinewidth{0.100375pt}%
\definecolor{currentstroke}{rgb}{0.827451,0.827451,0.827451}%
\pgfsetstrokecolor{currentstroke}%
\pgfsetdash{}{0pt}%
\pgfpathmoveto{\pgfqpoint{0.970969in}{1.080890in}}%
\pgfpathlineto{\pgfqpoint{0.970969in}{3.227753in}}%
\pgfusepath{stroke}%
\end{pgfscope}%
\begin{pgfscope}%
\pgfsetbuttcap%
\pgfsetroundjoin%
\definecolor{currentfill}{rgb}{0.000000,0.000000,0.000000}%
\pgfsetfillcolor{currentfill}%
\pgfsetlinewidth{0.501875pt}%
\definecolor{currentstroke}{rgb}{0.000000,0.000000,0.000000}%
\pgfsetstrokecolor{currentstroke}%
\pgfsetdash{}{0pt}%
\pgfsys@defobject{currentmarker}{\pgfqpoint{0.000000in}{0.000000in}}{\pgfqpoint{0.000000in}{0.020833in}}{%
\pgfpathmoveto{\pgfqpoint{0.000000in}{0.000000in}}%
\pgfpathlineto{\pgfqpoint{0.000000in}{0.020833in}}%
\pgfusepath{stroke,fill}%
}%
\begin{pgfscope}%
\pgfsys@transformshift{0.970969in}{1.080890in}%
\pgfsys@useobject{currentmarker}{}%
\end{pgfscope}%
\end{pgfscope}%
\begin{pgfscope}%
\pgfsetbuttcap%
\pgfsetroundjoin%
\definecolor{currentfill}{rgb}{0.000000,0.000000,0.000000}%
\pgfsetfillcolor{currentfill}%
\pgfsetlinewidth{0.501875pt}%
\definecolor{currentstroke}{rgb}{0.000000,0.000000,0.000000}%
\pgfsetstrokecolor{currentstroke}%
\pgfsetdash{}{0pt}%
\pgfsys@defobject{currentmarker}{\pgfqpoint{0.000000in}{-0.020833in}}{\pgfqpoint{0.000000in}{0.000000in}}{%
\pgfpathmoveto{\pgfqpoint{0.000000in}{0.000000in}}%
\pgfpathlineto{\pgfqpoint{0.000000in}{-0.020833in}}%
\pgfusepath{stroke,fill}%
}%
\begin{pgfscope}%
\pgfsys@transformshift{0.970969in}{3.227753in}%
\pgfsys@useobject{currentmarker}{}%
\end{pgfscope}%
\end{pgfscope}%
\begin{pgfscope}%
\pgfpathrectangle{\pgfqpoint{0.481681in}{1.080890in}}{\pgfqpoint{5.785672in}{2.146863in}}%
\pgfusepath{clip}%
\pgfsetrectcap%
\pgfsetroundjoin%
\pgfsetlinewidth{0.100375pt}%
\definecolor{currentstroke}{rgb}{0.827451,0.827451,0.827451}%
\pgfsetstrokecolor{currentstroke}%
\pgfsetdash{}{0pt}%
\pgfpathmoveto{\pgfqpoint{1.038690in}{1.080890in}}%
\pgfpathlineto{\pgfqpoint{1.038690in}{3.227753in}}%
\pgfusepath{stroke}%
\end{pgfscope}%
\begin{pgfscope}%
\pgfsetbuttcap%
\pgfsetroundjoin%
\definecolor{currentfill}{rgb}{0.000000,0.000000,0.000000}%
\pgfsetfillcolor{currentfill}%
\pgfsetlinewidth{0.501875pt}%
\definecolor{currentstroke}{rgb}{0.000000,0.000000,0.000000}%
\pgfsetstrokecolor{currentstroke}%
\pgfsetdash{}{0pt}%
\pgfsys@defobject{currentmarker}{\pgfqpoint{0.000000in}{0.000000in}}{\pgfqpoint{0.000000in}{0.020833in}}{%
\pgfpathmoveto{\pgfqpoint{0.000000in}{0.000000in}}%
\pgfpathlineto{\pgfqpoint{0.000000in}{0.020833in}}%
\pgfusepath{stroke,fill}%
}%
\begin{pgfscope}%
\pgfsys@transformshift{1.038690in}{1.080890in}%
\pgfsys@useobject{currentmarker}{}%
\end{pgfscope}%
\end{pgfscope}%
\begin{pgfscope}%
\pgfsetbuttcap%
\pgfsetroundjoin%
\definecolor{currentfill}{rgb}{0.000000,0.000000,0.000000}%
\pgfsetfillcolor{currentfill}%
\pgfsetlinewidth{0.501875pt}%
\definecolor{currentstroke}{rgb}{0.000000,0.000000,0.000000}%
\pgfsetstrokecolor{currentstroke}%
\pgfsetdash{}{0pt}%
\pgfsys@defobject{currentmarker}{\pgfqpoint{0.000000in}{-0.020833in}}{\pgfqpoint{0.000000in}{0.000000in}}{%
\pgfpathmoveto{\pgfqpoint{0.000000in}{0.000000in}}%
\pgfpathlineto{\pgfqpoint{0.000000in}{-0.020833in}}%
\pgfusepath{stroke,fill}%
}%
\begin{pgfscope}%
\pgfsys@transformshift{1.038690in}{3.227753in}%
\pgfsys@useobject{currentmarker}{}%
\end{pgfscope}%
\end{pgfscope}%
\begin{pgfscope}%
\pgfpathrectangle{\pgfqpoint{0.481681in}{1.080890in}}{\pgfqpoint{5.785672in}{2.146863in}}%
\pgfusepath{clip}%
\pgfsetrectcap%
\pgfsetroundjoin%
\pgfsetlinewidth{0.100375pt}%
\definecolor{currentstroke}{rgb}{0.827451,0.827451,0.827451}%
\pgfsetstrokecolor{currentstroke}%
\pgfsetdash{}{0pt}%
\pgfpathmoveto{\pgfqpoint{1.106412in}{1.080890in}}%
\pgfpathlineto{\pgfqpoint{1.106412in}{3.227753in}}%
\pgfusepath{stroke}%
\end{pgfscope}%
\begin{pgfscope}%
\pgfsetbuttcap%
\pgfsetroundjoin%
\definecolor{currentfill}{rgb}{0.000000,0.000000,0.000000}%
\pgfsetfillcolor{currentfill}%
\pgfsetlinewidth{0.501875pt}%
\definecolor{currentstroke}{rgb}{0.000000,0.000000,0.000000}%
\pgfsetstrokecolor{currentstroke}%
\pgfsetdash{}{0pt}%
\pgfsys@defobject{currentmarker}{\pgfqpoint{0.000000in}{0.000000in}}{\pgfqpoint{0.000000in}{0.020833in}}{%
\pgfpathmoveto{\pgfqpoint{0.000000in}{0.000000in}}%
\pgfpathlineto{\pgfqpoint{0.000000in}{0.020833in}}%
\pgfusepath{stroke,fill}%
}%
\begin{pgfscope}%
\pgfsys@transformshift{1.106412in}{1.080890in}%
\pgfsys@useobject{currentmarker}{}%
\end{pgfscope}%
\end{pgfscope}%
\begin{pgfscope}%
\pgfsetbuttcap%
\pgfsetroundjoin%
\definecolor{currentfill}{rgb}{0.000000,0.000000,0.000000}%
\pgfsetfillcolor{currentfill}%
\pgfsetlinewidth{0.501875pt}%
\definecolor{currentstroke}{rgb}{0.000000,0.000000,0.000000}%
\pgfsetstrokecolor{currentstroke}%
\pgfsetdash{}{0pt}%
\pgfsys@defobject{currentmarker}{\pgfqpoint{0.000000in}{-0.020833in}}{\pgfqpoint{0.000000in}{0.000000in}}{%
\pgfpathmoveto{\pgfqpoint{0.000000in}{0.000000in}}%
\pgfpathlineto{\pgfqpoint{0.000000in}{-0.020833in}}%
\pgfusepath{stroke,fill}%
}%
\begin{pgfscope}%
\pgfsys@transformshift{1.106412in}{3.227753in}%
\pgfsys@useobject{currentmarker}{}%
\end{pgfscope}%
\end{pgfscope}%
\begin{pgfscope}%
\pgfpathrectangle{\pgfqpoint{0.481681in}{1.080890in}}{\pgfqpoint{5.785672in}{2.146863in}}%
\pgfusepath{clip}%
\pgfsetrectcap%
\pgfsetroundjoin%
\pgfsetlinewidth{0.100375pt}%
\definecolor{currentstroke}{rgb}{0.827451,0.827451,0.827451}%
\pgfsetstrokecolor{currentstroke}%
\pgfsetdash{}{0pt}%
\pgfpathmoveto{\pgfqpoint{1.241855in}{1.080890in}}%
\pgfpathlineto{\pgfqpoint{1.241855in}{3.227753in}}%
\pgfusepath{stroke}%
\end{pgfscope}%
\begin{pgfscope}%
\pgfsetbuttcap%
\pgfsetroundjoin%
\definecolor{currentfill}{rgb}{0.000000,0.000000,0.000000}%
\pgfsetfillcolor{currentfill}%
\pgfsetlinewidth{0.501875pt}%
\definecolor{currentstroke}{rgb}{0.000000,0.000000,0.000000}%
\pgfsetstrokecolor{currentstroke}%
\pgfsetdash{}{0pt}%
\pgfsys@defobject{currentmarker}{\pgfqpoint{0.000000in}{0.000000in}}{\pgfqpoint{0.000000in}{0.020833in}}{%
\pgfpathmoveto{\pgfqpoint{0.000000in}{0.000000in}}%
\pgfpathlineto{\pgfqpoint{0.000000in}{0.020833in}}%
\pgfusepath{stroke,fill}%
}%
\begin{pgfscope}%
\pgfsys@transformshift{1.241855in}{1.080890in}%
\pgfsys@useobject{currentmarker}{}%
\end{pgfscope}%
\end{pgfscope}%
\begin{pgfscope}%
\pgfsetbuttcap%
\pgfsetroundjoin%
\definecolor{currentfill}{rgb}{0.000000,0.000000,0.000000}%
\pgfsetfillcolor{currentfill}%
\pgfsetlinewidth{0.501875pt}%
\definecolor{currentstroke}{rgb}{0.000000,0.000000,0.000000}%
\pgfsetstrokecolor{currentstroke}%
\pgfsetdash{}{0pt}%
\pgfsys@defobject{currentmarker}{\pgfqpoint{0.000000in}{-0.020833in}}{\pgfqpoint{0.000000in}{0.000000in}}{%
\pgfpathmoveto{\pgfqpoint{0.000000in}{0.000000in}}%
\pgfpathlineto{\pgfqpoint{0.000000in}{-0.020833in}}%
\pgfusepath{stroke,fill}%
}%
\begin{pgfscope}%
\pgfsys@transformshift{1.241855in}{3.227753in}%
\pgfsys@useobject{currentmarker}{}%
\end{pgfscope}%
\end{pgfscope}%
\begin{pgfscope}%
\pgfpathrectangle{\pgfqpoint{0.481681in}{1.080890in}}{\pgfqpoint{5.785672in}{2.146863in}}%
\pgfusepath{clip}%
\pgfsetrectcap%
\pgfsetroundjoin%
\pgfsetlinewidth{0.100375pt}%
\definecolor{currentstroke}{rgb}{0.827451,0.827451,0.827451}%
\pgfsetstrokecolor{currentstroke}%
\pgfsetdash{}{0pt}%
\pgfpathmoveto{\pgfqpoint{1.309576in}{1.080890in}}%
\pgfpathlineto{\pgfqpoint{1.309576in}{3.227753in}}%
\pgfusepath{stroke}%
\end{pgfscope}%
\begin{pgfscope}%
\pgfsetbuttcap%
\pgfsetroundjoin%
\definecolor{currentfill}{rgb}{0.000000,0.000000,0.000000}%
\pgfsetfillcolor{currentfill}%
\pgfsetlinewidth{0.501875pt}%
\definecolor{currentstroke}{rgb}{0.000000,0.000000,0.000000}%
\pgfsetstrokecolor{currentstroke}%
\pgfsetdash{}{0pt}%
\pgfsys@defobject{currentmarker}{\pgfqpoint{0.000000in}{0.000000in}}{\pgfqpoint{0.000000in}{0.020833in}}{%
\pgfpathmoveto{\pgfqpoint{0.000000in}{0.000000in}}%
\pgfpathlineto{\pgfqpoint{0.000000in}{0.020833in}}%
\pgfusepath{stroke,fill}%
}%
\begin{pgfscope}%
\pgfsys@transformshift{1.309576in}{1.080890in}%
\pgfsys@useobject{currentmarker}{}%
\end{pgfscope}%
\end{pgfscope}%
\begin{pgfscope}%
\pgfsetbuttcap%
\pgfsetroundjoin%
\definecolor{currentfill}{rgb}{0.000000,0.000000,0.000000}%
\pgfsetfillcolor{currentfill}%
\pgfsetlinewidth{0.501875pt}%
\definecolor{currentstroke}{rgb}{0.000000,0.000000,0.000000}%
\pgfsetstrokecolor{currentstroke}%
\pgfsetdash{}{0pt}%
\pgfsys@defobject{currentmarker}{\pgfqpoint{0.000000in}{-0.020833in}}{\pgfqpoint{0.000000in}{0.000000in}}{%
\pgfpathmoveto{\pgfqpoint{0.000000in}{0.000000in}}%
\pgfpathlineto{\pgfqpoint{0.000000in}{-0.020833in}}%
\pgfusepath{stroke,fill}%
}%
\begin{pgfscope}%
\pgfsys@transformshift{1.309576in}{3.227753in}%
\pgfsys@useobject{currentmarker}{}%
\end{pgfscope}%
\end{pgfscope}%
\begin{pgfscope}%
\pgfpathrectangle{\pgfqpoint{0.481681in}{1.080890in}}{\pgfqpoint{5.785672in}{2.146863in}}%
\pgfusepath{clip}%
\pgfsetrectcap%
\pgfsetroundjoin%
\pgfsetlinewidth{0.100375pt}%
\definecolor{currentstroke}{rgb}{0.827451,0.827451,0.827451}%
\pgfsetstrokecolor{currentstroke}%
\pgfsetdash{}{0pt}%
\pgfpathmoveto{\pgfqpoint{1.377297in}{1.080890in}}%
\pgfpathlineto{\pgfqpoint{1.377297in}{3.227753in}}%
\pgfusepath{stroke}%
\end{pgfscope}%
\begin{pgfscope}%
\pgfsetbuttcap%
\pgfsetroundjoin%
\definecolor{currentfill}{rgb}{0.000000,0.000000,0.000000}%
\pgfsetfillcolor{currentfill}%
\pgfsetlinewidth{0.501875pt}%
\definecolor{currentstroke}{rgb}{0.000000,0.000000,0.000000}%
\pgfsetstrokecolor{currentstroke}%
\pgfsetdash{}{0pt}%
\pgfsys@defobject{currentmarker}{\pgfqpoint{0.000000in}{0.000000in}}{\pgfqpoint{0.000000in}{0.020833in}}{%
\pgfpathmoveto{\pgfqpoint{0.000000in}{0.000000in}}%
\pgfpathlineto{\pgfqpoint{0.000000in}{0.020833in}}%
\pgfusepath{stroke,fill}%
}%
\begin{pgfscope}%
\pgfsys@transformshift{1.377297in}{1.080890in}%
\pgfsys@useobject{currentmarker}{}%
\end{pgfscope}%
\end{pgfscope}%
\begin{pgfscope}%
\pgfsetbuttcap%
\pgfsetroundjoin%
\definecolor{currentfill}{rgb}{0.000000,0.000000,0.000000}%
\pgfsetfillcolor{currentfill}%
\pgfsetlinewidth{0.501875pt}%
\definecolor{currentstroke}{rgb}{0.000000,0.000000,0.000000}%
\pgfsetstrokecolor{currentstroke}%
\pgfsetdash{}{0pt}%
\pgfsys@defobject{currentmarker}{\pgfqpoint{0.000000in}{-0.020833in}}{\pgfqpoint{0.000000in}{0.000000in}}{%
\pgfpathmoveto{\pgfqpoint{0.000000in}{0.000000in}}%
\pgfpathlineto{\pgfqpoint{0.000000in}{-0.020833in}}%
\pgfusepath{stroke,fill}%
}%
\begin{pgfscope}%
\pgfsys@transformshift{1.377297in}{3.227753in}%
\pgfsys@useobject{currentmarker}{}%
\end{pgfscope}%
\end{pgfscope}%
\begin{pgfscope}%
\pgfpathrectangle{\pgfqpoint{0.481681in}{1.080890in}}{\pgfqpoint{5.785672in}{2.146863in}}%
\pgfusepath{clip}%
\pgfsetrectcap%
\pgfsetroundjoin%
\pgfsetlinewidth{0.100375pt}%
\definecolor{currentstroke}{rgb}{0.827451,0.827451,0.827451}%
\pgfsetstrokecolor{currentstroke}%
\pgfsetdash{}{0pt}%
\pgfpathmoveto{\pgfqpoint{1.445019in}{1.080890in}}%
\pgfpathlineto{\pgfqpoint{1.445019in}{3.227753in}}%
\pgfusepath{stroke}%
\end{pgfscope}%
\begin{pgfscope}%
\pgfsetbuttcap%
\pgfsetroundjoin%
\definecolor{currentfill}{rgb}{0.000000,0.000000,0.000000}%
\pgfsetfillcolor{currentfill}%
\pgfsetlinewidth{0.501875pt}%
\definecolor{currentstroke}{rgb}{0.000000,0.000000,0.000000}%
\pgfsetstrokecolor{currentstroke}%
\pgfsetdash{}{0pt}%
\pgfsys@defobject{currentmarker}{\pgfqpoint{0.000000in}{0.000000in}}{\pgfqpoint{0.000000in}{0.020833in}}{%
\pgfpathmoveto{\pgfqpoint{0.000000in}{0.000000in}}%
\pgfpathlineto{\pgfqpoint{0.000000in}{0.020833in}}%
\pgfusepath{stroke,fill}%
}%
\begin{pgfscope}%
\pgfsys@transformshift{1.445019in}{1.080890in}%
\pgfsys@useobject{currentmarker}{}%
\end{pgfscope}%
\end{pgfscope}%
\begin{pgfscope}%
\pgfsetbuttcap%
\pgfsetroundjoin%
\definecolor{currentfill}{rgb}{0.000000,0.000000,0.000000}%
\pgfsetfillcolor{currentfill}%
\pgfsetlinewidth{0.501875pt}%
\definecolor{currentstroke}{rgb}{0.000000,0.000000,0.000000}%
\pgfsetstrokecolor{currentstroke}%
\pgfsetdash{}{0pt}%
\pgfsys@defobject{currentmarker}{\pgfqpoint{0.000000in}{-0.020833in}}{\pgfqpoint{0.000000in}{0.000000in}}{%
\pgfpathmoveto{\pgfqpoint{0.000000in}{0.000000in}}%
\pgfpathlineto{\pgfqpoint{0.000000in}{-0.020833in}}%
\pgfusepath{stroke,fill}%
}%
\begin{pgfscope}%
\pgfsys@transformshift{1.445019in}{3.227753in}%
\pgfsys@useobject{currentmarker}{}%
\end{pgfscope}%
\end{pgfscope}%
\begin{pgfscope}%
\pgfpathrectangle{\pgfqpoint{0.481681in}{1.080890in}}{\pgfqpoint{5.785672in}{2.146863in}}%
\pgfusepath{clip}%
\pgfsetrectcap%
\pgfsetroundjoin%
\pgfsetlinewidth{0.100375pt}%
\definecolor{currentstroke}{rgb}{0.827451,0.827451,0.827451}%
\pgfsetstrokecolor{currentstroke}%
\pgfsetdash{}{0pt}%
\pgfpathmoveto{\pgfqpoint{1.512740in}{1.080890in}}%
\pgfpathlineto{\pgfqpoint{1.512740in}{3.227753in}}%
\pgfusepath{stroke}%
\end{pgfscope}%
\begin{pgfscope}%
\pgfsetbuttcap%
\pgfsetroundjoin%
\definecolor{currentfill}{rgb}{0.000000,0.000000,0.000000}%
\pgfsetfillcolor{currentfill}%
\pgfsetlinewidth{0.501875pt}%
\definecolor{currentstroke}{rgb}{0.000000,0.000000,0.000000}%
\pgfsetstrokecolor{currentstroke}%
\pgfsetdash{}{0pt}%
\pgfsys@defobject{currentmarker}{\pgfqpoint{0.000000in}{0.000000in}}{\pgfqpoint{0.000000in}{0.020833in}}{%
\pgfpathmoveto{\pgfqpoint{0.000000in}{0.000000in}}%
\pgfpathlineto{\pgfqpoint{0.000000in}{0.020833in}}%
\pgfusepath{stroke,fill}%
}%
\begin{pgfscope}%
\pgfsys@transformshift{1.512740in}{1.080890in}%
\pgfsys@useobject{currentmarker}{}%
\end{pgfscope}%
\end{pgfscope}%
\begin{pgfscope}%
\pgfsetbuttcap%
\pgfsetroundjoin%
\definecolor{currentfill}{rgb}{0.000000,0.000000,0.000000}%
\pgfsetfillcolor{currentfill}%
\pgfsetlinewidth{0.501875pt}%
\definecolor{currentstroke}{rgb}{0.000000,0.000000,0.000000}%
\pgfsetstrokecolor{currentstroke}%
\pgfsetdash{}{0pt}%
\pgfsys@defobject{currentmarker}{\pgfqpoint{0.000000in}{-0.020833in}}{\pgfqpoint{0.000000in}{0.000000in}}{%
\pgfpathmoveto{\pgfqpoint{0.000000in}{0.000000in}}%
\pgfpathlineto{\pgfqpoint{0.000000in}{-0.020833in}}%
\pgfusepath{stroke,fill}%
}%
\begin{pgfscope}%
\pgfsys@transformshift{1.512740in}{3.227753in}%
\pgfsys@useobject{currentmarker}{}%
\end{pgfscope}%
\end{pgfscope}%
\begin{pgfscope}%
\pgfpathrectangle{\pgfqpoint{0.481681in}{1.080890in}}{\pgfqpoint{5.785672in}{2.146863in}}%
\pgfusepath{clip}%
\pgfsetrectcap%
\pgfsetroundjoin%
\pgfsetlinewidth{0.100375pt}%
\definecolor{currentstroke}{rgb}{0.827451,0.827451,0.827451}%
\pgfsetstrokecolor{currentstroke}%
\pgfsetdash{}{0pt}%
\pgfpathmoveto{\pgfqpoint{1.648183in}{1.080890in}}%
\pgfpathlineto{\pgfqpoint{1.648183in}{3.227753in}}%
\pgfusepath{stroke}%
\end{pgfscope}%
\begin{pgfscope}%
\pgfsetbuttcap%
\pgfsetroundjoin%
\definecolor{currentfill}{rgb}{0.000000,0.000000,0.000000}%
\pgfsetfillcolor{currentfill}%
\pgfsetlinewidth{0.501875pt}%
\definecolor{currentstroke}{rgb}{0.000000,0.000000,0.000000}%
\pgfsetstrokecolor{currentstroke}%
\pgfsetdash{}{0pt}%
\pgfsys@defobject{currentmarker}{\pgfqpoint{0.000000in}{0.000000in}}{\pgfqpoint{0.000000in}{0.020833in}}{%
\pgfpathmoveto{\pgfqpoint{0.000000in}{0.000000in}}%
\pgfpathlineto{\pgfqpoint{0.000000in}{0.020833in}}%
\pgfusepath{stroke,fill}%
}%
\begin{pgfscope}%
\pgfsys@transformshift{1.648183in}{1.080890in}%
\pgfsys@useobject{currentmarker}{}%
\end{pgfscope}%
\end{pgfscope}%
\begin{pgfscope}%
\pgfsetbuttcap%
\pgfsetroundjoin%
\definecolor{currentfill}{rgb}{0.000000,0.000000,0.000000}%
\pgfsetfillcolor{currentfill}%
\pgfsetlinewidth{0.501875pt}%
\definecolor{currentstroke}{rgb}{0.000000,0.000000,0.000000}%
\pgfsetstrokecolor{currentstroke}%
\pgfsetdash{}{0pt}%
\pgfsys@defobject{currentmarker}{\pgfqpoint{0.000000in}{-0.020833in}}{\pgfqpoint{0.000000in}{0.000000in}}{%
\pgfpathmoveto{\pgfqpoint{0.000000in}{0.000000in}}%
\pgfpathlineto{\pgfqpoint{0.000000in}{-0.020833in}}%
\pgfusepath{stroke,fill}%
}%
\begin{pgfscope}%
\pgfsys@transformshift{1.648183in}{3.227753in}%
\pgfsys@useobject{currentmarker}{}%
\end{pgfscope}%
\end{pgfscope}%
\begin{pgfscope}%
\pgfpathrectangle{\pgfqpoint{0.481681in}{1.080890in}}{\pgfqpoint{5.785672in}{2.146863in}}%
\pgfusepath{clip}%
\pgfsetrectcap%
\pgfsetroundjoin%
\pgfsetlinewidth{0.100375pt}%
\definecolor{currentstroke}{rgb}{0.827451,0.827451,0.827451}%
\pgfsetstrokecolor{currentstroke}%
\pgfsetdash{}{0pt}%
\pgfpathmoveto{\pgfqpoint{1.715905in}{1.080890in}}%
\pgfpathlineto{\pgfqpoint{1.715905in}{3.227753in}}%
\pgfusepath{stroke}%
\end{pgfscope}%
\begin{pgfscope}%
\pgfsetbuttcap%
\pgfsetroundjoin%
\definecolor{currentfill}{rgb}{0.000000,0.000000,0.000000}%
\pgfsetfillcolor{currentfill}%
\pgfsetlinewidth{0.501875pt}%
\definecolor{currentstroke}{rgb}{0.000000,0.000000,0.000000}%
\pgfsetstrokecolor{currentstroke}%
\pgfsetdash{}{0pt}%
\pgfsys@defobject{currentmarker}{\pgfqpoint{0.000000in}{0.000000in}}{\pgfqpoint{0.000000in}{0.020833in}}{%
\pgfpathmoveto{\pgfqpoint{0.000000in}{0.000000in}}%
\pgfpathlineto{\pgfqpoint{0.000000in}{0.020833in}}%
\pgfusepath{stroke,fill}%
}%
\begin{pgfscope}%
\pgfsys@transformshift{1.715905in}{1.080890in}%
\pgfsys@useobject{currentmarker}{}%
\end{pgfscope}%
\end{pgfscope}%
\begin{pgfscope}%
\pgfsetbuttcap%
\pgfsetroundjoin%
\definecolor{currentfill}{rgb}{0.000000,0.000000,0.000000}%
\pgfsetfillcolor{currentfill}%
\pgfsetlinewidth{0.501875pt}%
\definecolor{currentstroke}{rgb}{0.000000,0.000000,0.000000}%
\pgfsetstrokecolor{currentstroke}%
\pgfsetdash{}{0pt}%
\pgfsys@defobject{currentmarker}{\pgfqpoint{0.000000in}{-0.020833in}}{\pgfqpoint{0.000000in}{0.000000in}}{%
\pgfpathmoveto{\pgfqpoint{0.000000in}{0.000000in}}%
\pgfpathlineto{\pgfqpoint{0.000000in}{-0.020833in}}%
\pgfusepath{stroke,fill}%
}%
\begin{pgfscope}%
\pgfsys@transformshift{1.715905in}{3.227753in}%
\pgfsys@useobject{currentmarker}{}%
\end{pgfscope}%
\end{pgfscope}%
\begin{pgfscope}%
\pgfpathrectangle{\pgfqpoint{0.481681in}{1.080890in}}{\pgfqpoint{5.785672in}{2.146863in}}%
\pgfusepath{clip}%
\pgfsetrectcap%
\pgfsetroundjoin%
\pgfsetlinewidth{0.100375pt}%
\definecolor{currentstroke}{rgb}{0.827451,0.827451,0.827451}%
\pgfsetstrokecolor{currentstroke}%
\pgfsetdash{}{0pt}%
\pgfpathmoveto{\pgfqpoint{1.783626in}{1.080890in}}%
\pgfpathlineto{\pgfqpoint{1.783626in}{3.227753in}}%
\pgfusepath{stroke}%
\end{pgfscope}%
\begin{pgfscope}%
\pgfsetbuttcap%
\pgfsetroundjoin%
\definecolor{currentfill}{rgb}{0.000000,0.000000,0.000000}%
\pgfsetfillcolor{currentfill}%
\pgfsetlinewidth{0.501875pt}%
\definecolor{currentstroke}{rgb}{0.000000,0.000000,0.000000}%
\pgfsetstrokecolor{currentstroke}%
\pgfsetdash{}{0pt}%
\pgfsys@defobject{currentmarker}{\pgfqpoint{0.000000in}{0.000000in}}{\pgfqpoint{0.000000in}{0.020833in}}{%
\pgfpathmoveto{\pgfqpoint{0.000000in}{0.000000in}}%
\pgfpathlineto{\pgfqpoint{0.000000in}{0.020833in}}%
\pgfusepath{stroke,fill}%
}%
\begin{pgfscope}%
\pgfsys@transformshift{1.783626in}{1.080890in}%
\pgfsys@useobject{currentmarker}{}%
\end{pgfscope}%
\end{pgfscope}%
\begin{pgfscope}%
\pgfsetbuttcap%
\pgfsetroundjoin%
\definecolor{currentfill}{rgb}{0.000000,0.000000,0.000000}%
\pgfsetfillcolor{currentfill}%
\pgfsetlinewidth{0.501875pt}%
\definecolor{currentstroke}{rgb}{0.000000,0.000000,0.000000}%
\pgfsetstrokecolor{currentstroke}%
\pgfsetdash{}{0pt}%
\pgfsys@defobject{currentmarker}{\pgfqpoint{0.000000in}{-0.020833in}}{\pgfqpoint{0.000000in}{0.000000in}}{%
\pgfpathmoveto{\pgfqpoint{0.000000in}{0.000000in}}%
\pgfpathlineto{\pgfqpoint{0.000000in}{-0.020833in}}%
\pgfusepath{stroke,fill}%
}%
\begin{pgfscope}%
\pgfsys@transformshift{1.783626in}{3.227753in}%
\pgfsys@useobject{currentmarker}{}%
\end{pgfscope}%
\end{pgfscope}%
\begin{pgfscope}%
\pgfpathrectangle{\pgfqpoint{0.481681in}{1.080890in}}{\pgfqpoint{5.785672in}{2.146863in}}%
\pgfusepath{clip}%
\pgfsetrectcap%
\pgfsetroundjoin%
\pgfsetlinewidth{0.100375pt}%
\definecolor{currentstroke}{rgb}{0.827451,0.827451,0.827451}%
\pgfsetstrokecolor{currentstroke}%
\pgfsetdash{}{0pt}%
\pgfpathmoveto{\pgfqpoint{1.851348in}{1.080890in}}%
\pgfpathlineto{\pgfqpoint{1.851348in}{3.227753in}}%
\pgfusepath{stroke}%
\end{pgfscope}%
\begin{pgfscope}%
\pgfsetbuttcap%
\pgfsetroundjoin%
\definecolor{currentfill}{rgb}{0.000000,0.000000,0.000000}%
\pgfsetfillcolor{currentfill}%
\pgfsetlinewidth{0.501875pt}%
\definecolor{currentstroke}{rgb}{0.000000,0.000000,0.000000}%
\pgfsetstrokecolor{currentstroke}%
\pgfsetdash{}{0pt}%
\pgfsys@defobject{currentmarker}{\pgfqpoint{0.000000in}{0.000000in}}{\pgfqpoint{0.000000in}{0.020833in}}{%
\pgfpathmoveto{\pgfqpoint{0.000000in}{0.000000in}}%
\pgfpathlineto{\pgfqpoint{0.000000in}{0.020833in}}%
\pgfusepath{stroke,fill}%
}%
\begin{pgfscope}%
\pgfsys@transformshift{1.851348in}{1.080890in}%
\pgfsys@useobject{currentmarker}{}%
\end{pgfscope}%
\end{pgfscope}%
\begin{pgfscope}%
\pgfsetbuttcap%
\pgfsetroundjoin%
\definecolor{currentfill}{rgb}{0.000000,0.000000,0.000000}%
\pgfsetfillcolor{currentfill}%
\pgfsetlinewidth{0.501875pt}%
\definecolor{currentstroke}{rgb}{0.000000,0.000000,0.000000}%
\pgfsetstrokecolor{currentstroke}%
\pgfsetdash{}{0pt}%
\pgfsys@defobject{currentmarker}{\pgfqpoint{0.000000in}{-0.020833in}}{\pgfqpoint{0.000000in}{0.000000in}}{%
\pgfpathmoveto{\pgfqpoint{0.000000in}{0.000000in}}%
\pgfpathlineto{\pgfqpoint{0.000000in}{-0.020833in}}%
\pgfusepath{stroke,fill}%
}%
\begin{pgfscope}%
\pgfsys@transformshift{1.851348in}{3.227753in}%
\pgfsys@useobject{currentmarker}{}%
\end{pgfscope}%
\end{pgfscope}%
\begin{pgfscope}%
\pgfpathrectangle{\pgfqpoint{0.481681in}{1.080890in}}{\pgfqpoint{5.785672in}{2.146863in}}%
\pgfusepath{clip}%
\pgfsetrectcap%
\pgfsetroundjoin%
\pgfsetlinewidth{0.100375pt}%
\definecolor{currentstroke}{rgb}{0.827451,0.827451,0.827451}%
\pgfsetstrokecolor{currentstroke}%
\pgfsetdash{}{0pt}%
\pgfpathmoveto{\pgfqpoint{1.919069in}{1.080890in}}%
\pgfpathlineto{\pgfqpoint{1.919069in}{3.227753in}}%
\pgfusepath{stroke}%
\end{pgfscope}%
\begin{pgfscope}%
\pgfsetbuttcap%
\pgfsetroundjoin%
\definecolor{currentfill}{rgb}{0.000000,0.000000,0.000000}%
\pgfsetfillcolor{currentfill}%
\pgfsetlinewidth{0.501875pt}%
\definecolor{currentstroke}{rgb}{0.000000,0.000000,0.000000}%
\pgfsetstrokecolor{currentstroke}%
\pgfsetdash{}{0pt}%
\pgfsys@defobject{currentmarker}{\pgfqpoint{0.000000in}{0.000000in}}{\pgfqpoint{0.000000in}{0.020833in}}{%
\pgfpathmoveto{\pgfqpoint{0.000000in}{0.000000in}}%
\pgfpathlineto{\pgfqpoint{0.000000in}{0.020833in}}%
\pgfusepath{stroke,fill}%
}%
\begin{pgfscope}%
\pgfsys@transformshift{1.919069in}{1.080890in}%
\pgfsys@useobject{currentmarker}{}%
\end{pgfscope}%
\end{pgfscope}%
\begin{pgfscope}%
\pgfsetbuttcap%
\pgfsetroundjoin%
\definecolor{currentfill}{rgb}{0.000000,0.000000,0.000000}%
\pgfsetfillcolor{currentfill}%
\pgfsetlinewidth{0.501875pt}%
\definecolor{currentstroke}{rgb}{0.000000,0.000000,0.000000}%
\pgfsetstrokecolor{currentstroke}%
\pgfsetdash{}{0pt}%
\pgfsys@defobject{currentmarker}{\pgfqpoint{0.000000in}{-0.020833in}}{\pgfqpoint{0.000000in}{0.000000in}}{%
\pgfpathmoveto{\pgfqpoint{0.000000in}{0.000000in}}%
\pgfpathlineto{\pgfqpoint{0.000000in}{-0.020833in}}%
\pgfusepath{stroke,fill}%
}%
\begin{pgfscope}%
\pgfsys@transformshift{1.919069in}{3.227753in}%
\pgfsys@useobject{currentmarker}{}%
\end{pgfscope}%
\end{pgfscope}%
\begin{pgfscope}%
\pgfpathrectangle{\pgfqpoint{0.481681in}{1.080890in}}{\pgfqpoint{5.785672in}{2.146863in}}%
\pgfusepath{clip}%
\pgfsetrectcap%
\pgfsetroundjoin%
\pgfsetlinewidth{0.100375pt}%
\definecolor{currentstroke}{rgb}{0.827451,0.827451,0.827451}%
\pgfsetstrokecolor{currentstroke}%
\pgfsetdash{}{0pt}%
\pgfpathmoveto{\pgfqpoint{2.054512in}{1.080890in}}%
\pgfpathlineto{\pgfqpoint{2.054512in}{3.227753in}}%
\pgfusepath{stroke}%
\end{pgfscope}%
\begin{pgfscope}%
\pgfsetbuttcap%
\pgfsetroundjoin%
\definecolor{currentfill}{rgb}{0.000000,0.000000,0.000000}%
\pgfsetfillcolor{currentfill}%
\pgfsetlinewidth{0.501875pt}%
\definecolor{currentstroke}{rgb}{0.000000,0.000000,0.000000}%
\pgfsetstrokecolor{currentstroke}%
\pgfsetdash{}{0pt}%
\pgfsys@defobject{currentmarker}{\pgfqpoint{0.000000in}{0.000000in}}{\pgfqpoint{0.000000in}{0.020833in}}{%
\pgfpathmoveto{\pgfqpoint{0.000000in}{0.000000in}}%
\pgfpathlineto{\pgfqpoint{0.000000in}{0.020833in}}%
\pgfusepath{stroke,fill}%
}%
\begin{pgfscope}%
\pgfsys@transformshift{2.054512in}{1.080890in}%
\pgfsys@useobject{currentmarker}{}%
\end{pgfscope}%
\end{pgfscope}%
\begin{pgfscope}%
\pgfsetbuttcap%
\pgfsetroundjoin%
\definecolor{currentfill}{rgb}{0.000000,0.000000,0.000000}%
\pgfsetfillcolor{currentfill}%
\pgfsetlinewidth{0.501875pt}%
\definecolor{currentstroke}{rgb}{0.000000,0.000000,0.000000}%
\pgfsetstrokecolor{currentstroke}%
\pgfsetdash{}{0pt}%
\pgfsys@defobject{currentmarker}{\pgfqpoint{0.000000in}{-0.020833in}}{\pgfqpoint{0.000000in}{0.000000in}}{%
\pgfpathmoveto{\pgfqpoint{0.000000in}{0.000000in}}%
\pgfpathlineto{\pgfqpoint{0.000000in}{-0.020833in}}%
\pgfusepath{stroke,fill}%
}%
\begin{pgfscope}%
\pgfsys@transformshift{2.054512in}{3.227753in}%
\pgfsys@useobject{currentmarker}{}%
\end{pgfscope}%
\end{pgfscope}%
\begin{pgfscope}%
\pgfpathrectangle{\pgfqpoint{0.481681in}{1.080890in}}{\pgfqpoint{5.785672in}{2.146863in}}%
\pgfusepath{clip}%
\pgfsetrectcap%
\pgfsetroundjoin%
\pgfsetlinewidth{0.100375pt}%
\definecolor{currentstroke}{rgb}{0.827451,0.827451,0.827451}%
\pgfsetstrokecolor{currentstroke}%
\pgfsetdash{}{0pt}%
\pgfpathmoveto{\pgfqpoint{2.122234in}{1.080890in}}%
\pgfpathlineto{\pgfqpoint{2.122234in}{3.227753in}}%
\pgfusepath{stroke}%
\end{pgfscope}%
\begin{pgfscope}%
\pgfsetbuttcap%
\pgfsetroundjoin%
\definecolor{currentfill}{rgb}{0.000000,0.000000,0.000000}%
\pgfsetfillcolor{currentfill}%
\pgfsetlinewidth{0.501875pt}%
\definecolor{currentstroke}{rgb}{0.000000,0.000000,0.000000}%
\pgfsetstrokecolor{currentstroke}%
\pgfsetdash{}{0pt}%
\pgfsys@defobject{currentmarker}{\pgfqpoint{0.000000in}{0.000000in}}{\pgfqpoint{0.000000in}{0.020833in}}{%
\pgfpathmoveto{\pgfqpoint{0.000000in}{0.000000in}}%
\pgfpathlineto{\pgfqpoint{0.000000in}{0.020833in}}%
\pgfusepath{stroke,fill}%
}%
\begin{pgfscope}%
\pgfsys@transformshift{2.122234in}{1.080890in}%
\pgfsys@useobject{currentmarker}{}%
\end{pgfscope}%
\end{pgfscope}%
\begin{pgfscope}%
\pgfsetbuttcap%
\pgfsetroundjoin%
\definecolor{currentfill}{rgb}{0.000000,0.000000,0.000000}%
\pgfsetfillcolor{currentfill}%
\pgfsetlinewidth{0.501875pt}%
\definecolor{currentstroke}{rgb}{0.000000,0.000000,0.000000}%
\pgfsetstrokecolor{currentstroke}%
\pgfsetdash{}{0pt}%
\pgfsys@defobject{currentmarker}{\pgfqpoint{0.000000in}{-0.020833in}}{\pgfqpoint{0.000000in}{0.000000in}}{%
\pgfpathmoveto{\pgfqpoint{0.000000in}{0.000000in}}%
\pgfpathlineto{\pgfqpoint{0.000000in}{-0.020833in}}%
\pgfusepath{stroke,fill}%
}%
\begin{pgfscope}%
\pgfsys@transformshift{2.122234in}{3.227753in}%
\pgfsys@useobject{currentmarker}{}%
\end{pgfscope}%
\end{pgfscope}%
\begin{pgfscope}%
\pgfpathrectangle{\pgfqpoint{0.481681in}{1.080890in}}{\pgfqpoint{5.785672in}{2.146863in}}%
\pgfusepath{clip}%
\pgfsetrectcap%
\pgfsetroundjoin%
\pgfsetlinewidth{0.100375pt}%
\definecolor{currentstroke}{rgb}{0.827451,0.827451,0.827451}%
\pgfsetstrokecolor{currentstroke}%
\pgfsetdash{}{0pt}%
\pgfpathmoveto{\pgfqpoint{2.189955in}{1.080890in}}%
\pgfpathlineto{\pgfqpoint{2.189955in}{3.227753in}}%
\pgfusepath{stroke}%
\end{pgfscope}%
\begin{pgfscope}%
\pgfsetbuttcap%
\pgfsetroundjoin%
\definecolor{currentfill}{rgb}{0.000000,0.000000,0.000000}%
\pgfsetfillcolor{currentfill}%
\pgfsetlinewidth{0.501875pt}%
\definecolor{currentstroke}{rgb}{0.000000,0.000000,0.000000}%
\pgfsetstrokecolor{currentstroke}%
\pgfsetdash{}{0pt}%
\pgfsys@defobject{currentmarker}{\pgfqpoint{0.000000in}{0.000000in}}{\pgfqpoint{0.000000in}{0.020833in}}{%
\pgfpathmoveto{\pgfqpoint{0.000000in}{0.000000in}}%
\pgfpathlineto{\pgfqpoint{0.000000in}{0.020833in}}%
\pgfusepath{stroke,fill}%
}%
\begin{pgfscope}%
\pgfsys@transformshift{2.189955in}{1.080890in}%
\pgfsys@useobject{currentmarker}{}%
\end{pgfscope}%
\end{pgfscope}%
\begin{pgfscope}%
\pgfsetbuttcap%
\pgfsetroundjoin%
\definecolor{currentfill}{rgb}{0.000000,0.000000,0.000000}%
\pgfsetfillcolor{currentfill}%
\pgfsetlinewidth{0.501875pt}%
\definecolor{currentstroke}{rgb}{0.000000,0.000000,0.000000}%
\pgfsetstrokecolor{currentstroke}%
\pgfsetdash{}{0pt}%
\pgfsys@defobject{currentmarker}{\pgfqpoint{0.000000in}{-0.020833in}}{\pgfqpoint{0.000000in}{0.000000in}}{%
\pgfpathmoveto{\pgfqpoint{0.000000in}{0.000000in}}%
\pgfpathlineto{\pgfqpoint{0.000000in}{-0.020833in}}%
\pgfusepath{stroke,fill}%
}%
\begin{pgfscope}%
\pgfsys@transformshift{2.189955in}{3.227753in}%
\pgfsys@useobject{currentmarker}{}%
\end{pgfscope}%
\end{pgfscope}%
\begin{pgfscope}%
\pgfpathrectangle{\pgfqpoint{0.481681in}{1.080890in}}{\pgfqpoint{5.785672in}{2.146863in}}%
\pgfusepath{clip}%
\pgfsetrectcap%
\pgfsetroundjoin%
\pgfsetlinewidth{0.100375pt}%
\definecolor{currentstroke}{rgb}{0.827451,0.827451,0.827451}%
\pgfsetstrokecolor{currentstroke}%
\pgfsetdash{}{0pt}%
\pgfpathmoveto{\pgfqpoint{2.257677in}{1.080890in}}%
\pgfpathlineto{\pgfqpoint{2.257677in}{3.227753in}}%
\pgfusepath{stroke}%
\end{pgfscope}%
\begin{pgfscope}%
\pgfsetbuttcap%
\pgfsetroundjoin%
\definecolor{currentfill}{rgb}{0.000000,0.000000,0.000000}%
\pgfsetfillcolor{currentfill}%
\pgfsetlinewidth{0.501875pt}%
\definecolor{currentstroke}{rgb}{0.000000,0.000000,0.000000}%
\pgfsetstrokecolor{currentstroke}%
\pgfsetdash{}{0pt}%
\pgfsys@defobject{currentmarker}{\pgfqpoint{0.000000in}{0.000000in}}{\pgfqpoint{0.000000in}{0.020833in}}{%
\pgfpathmoveto{\pgfqpoint{0.000000in}{0.000000in}}%
\pgfpathlineto{\pgfqpoint{0.000000in}{0.020833in}}%
\pgfusepath{stroke,fill}%
}%
\begin{pgfscope}%
\pgfsys@transformshift{2.257677in}{1.080890in}%
\pgfsys@useobject{currentmarker}{}%
\end{pgfscope}%
\end{pgfscope}%
\begin{pgfscope}%
\pgfsetbuttcap%
\pgfsetroundjoin%
\definecolor{currentfill}{rgb}{0.000000,0.000000,0.000000}%
\pgfsetfillcolor{currentfill}%
\pgfsetlinewidth{0.501875pt}%
\definecolor{currentstroke}{rgb}{0.000000,0.000000,0.000000}%
\pgfsetstrokecolor{currentstroke}%
\pgfsetdash{}{0pt}%
\pgfsys@defobject{currentmarker}{\pgfqpoint{0.000000in}{-0.020833in}}{\pgfqpoint{0.000000in}{0.000000in}}{%
\pgfpathmoveto{\pgfqpoint{0.000000in}{0.000000in}}%
\pgfpathlineto{\pgfqpoint{0.000000in}{-0.020833in}}%
\pgfusepath{stroke,fill}%
}%
\begin{pgfscope}%
\pgfsys@transformshift{2.257677in}{3.227753in}%
\pgfsys@useobject{currentmarker}{}%
\end{pgfscope}%
\end{pgfscope}%
\begin{pgfscope}%
\pgfpathrectangle{\pgfqpoint{0.481681in}{1.080890in}}{\pgfqpoint{5.785672in}{2.146863in}}%
\pgfusepath{clip}%
\pgfsetrectcap%
\pgfsetroundjoin%
\pgfsetlinewidth{0.100375pt}%
\definecolor{currentstroke}{rgb}{0.827451,0.827451,0.827451}%
\pgfsetstrokecolor{currentstroke}%
\pgfsetdash{}{0pt}%
\pgfpathmoveto{\pgfqpoint{2.325398in}{1.080890in}}%
\pgfpathlineto{\pgfqpoint{2.325398in}{3.227753in}}%
\pgfusepath{stroke}%
\end{pgfscope}%
\begin{pgfscope}%
\pgfsetbuttcap%
\pgfsetroundjoin%
\definecolor{currentfill}{rgb}{0.000000,0.000000,0.000000}%
\pgfsetfillcolor{currentfill}%
\pgfsetlinewidth{0.501875pt}%
\definecolor{currentstroke}{rgb}{0.000000,0.000000,0.000000}%
\pgfsetstrokecolor{currentstroke}%
\pgfsetdash{}{0pt}%
\pgfsys@defobject{currentmarker}{\pgfqpoint{0.000000in}{0.000000in}}{\pgfqpoint{0.000000in}{0.020833in}}{%
\pgfpathmoveto{\pgfqpoint{0.000000in}{0.000000in}}%
\pgfpathlineto{\pgfqpoint{0.000000in}{0.020833in}}%
\pgfusepath{stroke,fill}%
}%
\begin{pgfscope}%
\pgfsys@transformshift{2.325398in}{1.080890in}%
\pgfsys@useobject{currentmarker}{}%
\end{pgfscope}%
\end{pgfscope}%
\begin{pgfscope}%
\pgfsetbuttcap%
\pgfsetroundjoin%
\definecolor{currentfill}{rgb}{0.000000,0.000000,0.000000}%
\pgfsetfillcolor{currentfill}%
\pgfsetlinewidth{0.501875pt}%
\definecolor{currentstroke}{rgb}{0.000000,0.000000,0.000000}%
\pgfsetstrokecolor{currentstroke}%
\pgfsetdash{}{0pt}%
\pgfsys@defobject{currentmarker}{\pgfqpoint{0.000000in}{-0.020833in}}{\pgfqpoint{0.000000in}{0.000000in}}{%
\pgfpathmoveto{\pgfqpoint{0.000000in}{0.000000in}}%
\pgfpathlineto{\pgfqpoint{0.000000in}{-0.020833in}}%
\pgfusepath{stroke,fill}%
}%
\begin{pgfscope}%
\pgfsys@transformshift{2.325398in}{3.227753in}%
\pgfsys@useobject{currentmarker}{}%
\end{pgfscope}%
\end{pgfscope}%
\begin{pgfscope}%
\pgfpathrectangle{\pgfqpoint{0.481681in}{1.080890in}}{\pgfqpoint{5.785672in}{2.146863in}}%
\pgfusepath{clip}%
\pgfsetrectcap%
\pgfsetroundjoin%
\pgfsetlinewidth{0.100375pt}%
\definecolor{currentstroke}{rgb}{0.827451,0.827451,0.827451}%
\pgfsetstrokecolor{currentstroke}%
\pgfsetdash{}{0pt}%
\pgfpathmoveto{\pgfqpoint{2.460841in}{1.080890in}}%
\pgfpathlineto{\pgfqpoint{2.460841in}{3.227753in}}%
\pgfusepath{stroke}%
\end{pgfscope}%
\begin{pgfscope}%
\pgfsetbuttcap%
\pgfsetroundjoin%
\definecolor{currentfill}{rgb}{0.000000,0.000000,0.000000}%
\pgfsetfillcolor{currentfill}%
\pgfsetlinewidth{0.501875pt}%
\definecolor{currentstroke}{rgb}{0.000000,0.000000,0.000000}%
\pgfsetstrokecolor{currentstroke}%
\pgfsetdash{}{0pt}%
\pgfsys@defobject{currentmarker}{\pgfqpoint{0.000000in}{0.000000in}}{\pgfqpoint{0.000000in}{0.020833in}}{%
\pgfpathmoveto{\pgfqpoint{0.000000in}{0.000000in}}%
\pgfpathlineto{\pgfqpoint{0.000000in}{0.020833in}}%
\pgfusepath{stroke,fill}%
}%
\begin{pgfscope}%
\pgfsys@transformshift{2.460841in}{1.080890in}%
\pgfsys@useobject{currentmarker}{}%
\end{pgfscope}%
\end{pgfscope}%
\begin{pgfscope}%
\pgfsetbuttcap%
\pgfsetroundjoin%
\definecolor{currentfill}{rgb}{0.000000,0.000000,0.000000}%
\pgfsetfillcolor{currentfill}%
\pgfsetlinewidth{0.501875pt}%
\definecolor{currentstroke}{rgb}{0.000000,0.000000,0.000000}%
\pgfsetstrokecolor{currentstroke}%
\pgfsetdash{}{0pt}%
\pgfsys@defobject{currentmarker}{\pgfqpoint{0.000000in}{-0.020833in}}{\pgfqpoint{0.000000in}{0.000000in}}{%
\pgfpathmoveto{\pgfqpoint{0.000000in}{0.000000in}}%
\pgfpathlineto{\pgfqpoint{0.000000in}{-0.020833in}}%
\pgfusepath{stroke,fill}%
}%
\begin{pgfscope}%
\pgfsys@transformshift{2.460841in}{3.227753in}%
\pgfsys@useobject{currentmarker}{}%
\end{pgfscope}%
\end{pgfscope}%
\begin{pgfscope}%
\pgfpathrectangle{\pgfqpoint{0.481681in}{1.080890in}}{\pgfqpoint{5.785672in}{2.146863in}}%
\pgfusepath{clip}%
\pgfsetrectcap%
\pgfsetroundjoin%
\pgfsetlinewidth{0.100375pt}%
\definecolor{currentstroke}{rgb}{0.827451,0.827451,0.827451}%
\pgfsetstrokecolor{currentstroke}%
\pgfsetdash{}{0pt}%
\pgfpathmoveto{\pgfqpoint{2.528563in}{1.080890in}}%
\pgfpathlineto{\pgfqpoint{2.528563in}{3.227753in}}%
\pgfusepath{stroke}%
\end{pgfscope}%
\begin{pgfscope}%
\pgfsetbuttcap%
\pgfsetroundjoin%
\definecolor{currentfill}{rgb}{0.000000,0.000000,0.000000}%
\pgfsetfillcolor{currentfill}%
\pgfsetlinewidth{0.501875pt}%
\definecolor{currentstroke}{rgb}{0.000000,0.000000,0.000000}%
\pgfsetstrokecolor{currentstroke}%
\pgfsetdash{}{0pt}%
\pgfsys@defobject{currentmarker}{\pgfqpoint{0.000000in}{0.000000in}}{\pgfqpoint{0.000000in}{0.020833in}}{%
\pgfpathmoveto{\pgfqpoint{0.000000in}{0.000000in}}%
\pgfpathlineto{\pgfqpoint{0.000000in}{0.020833in}}%
\pgfusepath{stroke,fill}%
}%
\begin{pgfscope}%
\pgfsys@transformshift{2.528563in}{1.080890in}%
\pgfsys@useobject{currentmarker}{}%
\end{pgfscope}%
\end{pgfscope}%
\begin{pgfscope}%
\pgfsetbuttcap%
\pgfsetroundjoin%
\definecolor{currentfill}{rgb}{0.000000,0.000000,0.000000}%
\pgfsetfillcolor{currentfill}%
\pgfsetlinewidth{0.501875pt}%
\definecolor{currentstroke}{rgb}{0.000000,0.000000,0.000000}%
\pgfsetstrokecolor{currentstroke}%
\pgfsetdash{}{0pt}%
\pgfsys@defobject{currentmarker}{\pgfqpoint{0.000000in}{-0.020833in}}{\pgfqpoint{0.000000in}{0.000000in}}{%
\pgfpathmoveto{\pgfqpoint{0.000000in}{0.000000in}}%
\pgfpathlineto{\pgfqpoint{0.000000in}{-0.020833in}}%
\pgfusepath{stroke,fill}%
}%
\begin{pgfscope}%
\pgfsys@transformshift{2.528563in}{3.227753in}%
\pgfsys@useobject{currentmarker}{}%
\end{pgfscope}%
\end{pgfscope}%
\begin{pgfscope}%
\pgfpathrectangle{\pgfqpoint{0.481681in}{1.080890in}}{\pgfqpoint{5.785672in}{2.146863in}}%
\pgfusepath{clip}%
\pgfsetrectcap%
\pgfsetroundjoin%
\pgfsetlinewidth{0.100375pt}%
\definecolor{currentstroke}{rgb}{0.827451,0.827451,0.827451}%
\pgfsetstrokecolor{currentstroke}%
\pgfsetdash{}{0pt}%
\pgfpathmoveto{\pgfqpoint{2.596284in}{1.080890in}}%
\pgfpathlineto{\pgfqpoint{2.596284in}{3.227753in}}%
\pgfusepath{stroke}%
\end{pgfscope}%
\begin{pgfscope}%
\pgfsetbuttcap%
\pgfsetroundjoin%
\definecolor{currentfill}{rgb}{0.000000,0.000000,0.000000}%
\pgfsetfillcolor{currentfill}%
\pgfsetlinewidth{0.501875pt}%
\definecolor{currentstroke}{rgb}{0.000000,0.000000,0.000000}%
\pgfsetstrokecolor{currentstroke}%
\pgfsetdash{}{0pt}%
\pgfsys@defobject{currentmarker}{\pgfqpoint{0.000000in}{0.000000in}}{\pgfqpoint{0.000000in}{0.020833in}}{%
\pgfpathmoveto{\pgfqpoint{0.000000in}{0.000000in}}%
\pgfpathlineto{\pgfqpoint{0.000000in}{0.020833in}}%
\pgfusepath{stroke,fill}%
}%
\begin{pgfscope}%
\pgfsys@transformshift{2.596284in}{1.080890in}%
\pgfsys@useobject{currentmarker}{}%
\end{pgfscope}%
\end{pgfscope}%
\begin{pgfscope}%
\pgfsetbuttcap%
\pgfsetroundjoin%
\definecolor{currentfill}{rgb}{0.000000,0.000000,0.000000}%
\pgfsetfillcolor{currentfill}%
\pgfsetlinewidth{0.501875pt}%
\definecolor{currentstroke}{rgb}{0.000000,0.000000,0.000000}%
\pgfsetstrokecolor{currentstroke}%
\pgfsetdash{}{0pt}%
\pgfsys@defobject{currentmarker}{\pgfqpoint{0.000000in}{-0.020833in}}{\pgfqpoint{0.000000in}{0.000000in}}{%
\pgfpathmoveto{\pgfqpoint{0.000000in}{0.000000in}}%
\pgfpathlineto{\pgfqpoint{0.000000in}{-0.020833in}}%
\pgfusepath{stroke,fill}%
}%
\begin{pgfscope}%
\pgfsys@transformshift{2.596284in}{3.227753in}%
\pgfsys@useobject{currentmarker}{}%
\end{pgfscope}%
\end{pgfscope}%
\begin{pgfscope}%
\pgfpathrectangle{\pgfqpoint{0.481681in}{1.080890in}}{\pgfqpoint{5.785672in}{2.146863in}}%
\pgfusepath{clip}%
\pgfsetrectcap%
\pgfsetroundjoin%
\pgfsetlinewidth{0.100375pt}%
\definecolor{currentstroke}{rgb}{0.827451,0.827451,0.827451}%
\pgfsetstrokecolor{currentstroke}%
\pgfsetdash{}{0pt}%
\pgfpathmoveto{\pgfqpoint{2.664006in}{1.080890in}}%
\pgfpathlineto{\pgfqpoint{2.664006in}{3.227753in}}%
\pgfusepath{stroke}%
\end{pgfscope}%
\begin{pgfscope}%
\pgfsetbuttcap%
\pgfsetroundjoin%
\definecolor{currentfill}{rgb}{0.000000,0.000000,0.000000}%
\pgfsetfillcolor{currentfill}%
\pgfsetlinewidth{0.501875pt}%
\definecolor{currentstroke}{rgb}{0.000000,0.000000,0.000000}%
\pgfsetstrokecolor{currentstroke}%
\pgfsetdash{}{0pt}%
\pgfsys@defobject{currentmarker}{\pgfqpoint{0.000000in}{0.000000in}}{\pgfqpoint{0.000000in}{0.020833in}}{%
\pgfpathmoveto{\pgfqpoint{0.000000in}{0.000000in}}%
\pgfpathlineto{\pgfqpoint{0.000000in}{0.020833in}}%
\pgfusepath{stroke,fill}%
}%
\begin{pgfscope}%
\pgfsys@transformshift{2.664006in}{1.080890in}%
\pgfsys@useobject{currentmarker}{}%
\end{pgfscope}%
\end{pgfscope}%
\begin{pgfscope}%
\pgfsetbuttcap%
\pgfsetroundjoin%
\definecolor{currentfill}{rgb}{0.000000,0.000000,0.000000}%
\pgfsetfillcolor{currentfill}%
\pgfsetlinewidth{0.501875pt}%
\definecolor{currentstroke}{rgb}{0.000000,0.000000,0.000000}%
\pgfsetstrokecolor{currentstroke}%
\pgfsetdash{}{0pt}%
\pgfsys@defobject{currentmarker}{\pgfqpoint{0.000000in}{-0.020833in}}{\pgfqpoint{0.000000in}{0.000000in}}{%
\pgfpathmoveto{\pgfqpoint{0.000000in}{0.000000in}}%
\pgfpathlineto{\pgfqpoint{0.000000in}{-0.020833in}}%
\pgfusepath{stroke,fill}%
}%
\begin{pgfscope}%
\pgfsys@transformshift{2.664006in}{3.227753in}%
\pgfsys@useobject{currentmarker}{}%
\end{pgfscope}%
\end{pgfscope}%
\begin{pgfscope}%
\pgfpathrectangle{\pgfqpoint{0.481681in}{1.080890in}}{\pgfqpoint{5.785672in}{2.146863in}}%
\pgfusepath{clip}%
\pgfsetrectcap%
\pgfsetroundjoin%
\pgfsetlinewidth{0.100375pt}%
\definecolor{currentstroke}{rgb}{0.827451,0.827451,0.827451}%
\pgfsetstrokecolor{currentstroke}%
\pgfsetdash{}{0pt}%
\pgfpathmoveto{\pgfqpoint{2.731727in}{1.080890in}}%
\pgfpathlineto{\pgfqpoint{2.731727in}{3.227753in}}%
\pgfusepath{stroke}%
\end{pgfscope}%
\begin{pgfscope}%
\pgfsetbuttcap%
\pgfsetroundjoin%
\definecolor{currentfill}{rgb}{0.000000,0.000000,0.000000}%
\pgfsetfillcolor{currentfill}%
\pgfsetlinewidth{0.501875pt}%
\definecolor{currentstroke}{rgb}{0.000000,0.000000,0.000000}%
\pgfsetstrokecolor{currentstroke}%
\pgfsetdash{}{0pt}%
\pgfsys@defobject{currentmarker}{\pgfqpoint{0.000000in}{0.000000in}}{\pgfqpoint{0.000000in}{0.020833in}}{%
\pgfpathmoveto{\pgfqpoint{0.000000in}{0.000000in}}%
\pgfpathlineto{\pgfqpoint{0.000000in}{0.020833in}}%
\pgfusepath{stroke,fill}%
}%
\begin{pgfscope}%
\pgfsys@transformshift{2.731727in}{1.080890in}%
\pgfsys@useobject{currentmarker}{}%
\end{pgfscope}%
\end{pgfscope}%
\begin{pgfscope}%
\pgfsetbuttcap%
\pgfsetroundjoin%
\definecolor{currentfill}{rgb}{0.000000,0.000000,0.000000}%
\pgfsetfillcolor{currentfill}%
\pgfsetlinewidth{0.501875pt}%
\definecolor{currentstroke}{rgb}{0.000000,0.000000,0.000000}%
\pgfsetstrokecolor{currentstroke}%
\pgfsetdash{}{0pt}%
\pgfsys@defobject{currentmarker}{\pgfqpoint{0.000000in}{-0.020833in}}{\pgfqpoint{0.000000in}{0.000000in}}{%
\pgfpathmoveto{\pgfqpoint{0.000000in}{0.000000in}}%
\pgfpathlineto{\pgfqpoint{0.000000in}{-0.020833in}}%
\pgfusepath{stroke,fill}%
}%
\begin{pgfscope}%
\pgfsys@transformshift{2.731727in}{3.227753in}%
\pgfsys@useobject{currentmarker}{}%
\end{pgfscope}%
\end{pgfscope}%
\begin{pgfscope}%
\pgfpathrectangle{\pgfqpoint{0.481681in}{1.080890in}}{\pgfqpoint{5.785672in}{2.146863in}}%
\pgfusepath{clip}%
\pgfsetrectcap%
\pgfsetroundjoin%
\pgfsetlinewidth{0.100375pt}%
\definecolor{currentstroke}{rgb}{0.827451,0.827451,0.827451}%
\pgfsetstrokecolor{currentstroke}%
\pgfsetdash{}{0pt}%
\pgfpathmoveto{\pgfqpoint{2.867170in}{1.080890in}}%
\pgfpathlineto{\pgfqpoint{2.867170in}{3.227753in}}%
\pgfusepath{stroke}%
\end{pgfscope}%
\begin{pgfscope}%
\pgfsetbuttcap%
\pgfsetroundjoin%
\definecolor{currentfill}{rgb}{0.000000,0.000000,0.000000}%
\pgfsetfillcolor{currentfill}%
\pgfsetlinewidth{0.501875pt}%
\definecolor{currentstroke}{rgb}{0.000000,0.000000,0.000000}%
\pgfsetstrokecolor{currentstroke}%
\pgfsetdash{}{0pt}%
\pgfsys@defobject{currentmarker}{\pgfqpoint{0.000000in}{0.000000in}}{\pgfqpoint{0.000000in}{0.020833in}}{%
\pgfpathmoveto{\pgfqpoint{0.000000in}{0.000000in}}%
\pgfpathlineto{\pgfqpoint{0.000000in}{0.020833in}}%
\pgfusepath{stroke,fill}%
}%
\begin{pgfscope}%
\pgfsys@transformshift{2.867170in}{1.080890in}%
\pgfsys@useobject{currentmarker}{}%
\end{pgfscope}%
\end{pgfscope}%
\begin{pgfscope}%
\pgfsetbuttcap%
\pgfsetroundjoin%
\definecolor{currentfill}{rgb}{0.000000,0.000000,0.000000}%
\pgfsetfillcolor{currentfill}%
\pgfsetlinewidth{0.501875pt}%
\definecolor{currentstroke}{rgb}{0.000000,0.000000,0.000000}%
\pgfsetstrokecolor{currentstroke}%
\pgfsetdash{}{0pt}%
\pgfsys@defobject{currentmarker}{\pgfqpoint{0.000000in}{-0.020833in}}{\pgfqpoint{0.000000in}{0.000000in}}{%
\pgfpathmoveto{\pgfqpoint{0.000000in}{0.000000in}}%
\pgfpathlineto{\pgfqpoint{0.000000in}{-0.020833in}}%
\pgfusepath{stroke,fill}%
}%
\begin{pgfscope}%
\pgfsys@transformshift{2.867170in}{3.227753in}%
\pgfsys@useobject{currentmarker}{}%
\end{pgfscope}%
\end{pgfscope}%
\begin{pgfscope}%
\pgfpathrectangle{\pgfqpoint{0.481681in}{1.080890in}}{\pgfqpoint{5.785672in}{2.146863in}}%
\pgfusepath{clip}%
\pgfsetrectcap%
\pgfsetroundjoin%
\pgfsetlinewidth{0.100375pt}%
\definecolor{currentstroke}{rgb}{0.827451,0.827451,0.827451}%
\pgfsetstrokecolor{currentstroke}%
\pgfsetdash{}{0pt}%
\pgfpathmoveto{\pgfqpoint{2.934892in}{1.080890in}}%
\pgfpathlineto{\pgfqpoint{2.934892in}{3.227753in}}%
\pgfusepath{stroke}%
\end{pgfscope}%
\begin{pgfscope}%
\pgfsetbuttcap%
\pgfsetroundjoin%
\definecolor{currentfill}{rgb}{0.000000,0.000000,0.000000}%
\pgfsetfillcolor{currentfill}%
\pgfsetlinewidth{0.501875pt}%
\definecolor{currentstroke}{rgb}{0.000000,0.000000,0.000000}%
\pgfsetstrokecolor{currentstroke}%
\pgfsetdash{}{0pt}%
\pgfsys@defobject{currentmarker}{\pgfqpoint{0.000000in}{0.000000in}}{\pgfqpoint{0.000000in}{0.020833in}}{%
\pgfpathmoveto{\pgfqpoint{0.000000in}{0.000000in}}%
\pgfpathlineto{\pgfqpoint{0.000000in}{0.020833in}}%
\pgfusepath{stroke,fill}%
}%
\begin{pgfscope}%
\pgfsys@transformshift{2.934892in}{1.080890in}%
\pgfsys@useobject{currentmarker}{}%
\end{pgfscope}%
\end{pgfscope}%
\begin{pgfscope}%
\pgfsetbuttcap%
\pgfsetroundjoin%
\definecolor{currentfill}{rgb}{0.000000,0.000000,0.000000}%
\pgfsetfillcolor{currentfill}%
\pgfsetlinewidth{0.501875pt}%
\definecolor{currentstroke}{rgb}{0.000000,0.000000,0.000000}%
\pgfsetstrokecolor{currentstroke}%
\pgfsetdash{}{0pt}%
\pgfsys@defobject{currentmarker}{\pgfqpoint{0.000000in}{-0.020833in}}{\pgfqpoint{0.000000in}{0.000000in}}{%
\pgfpathmoveto{\pgfqpoint{0.000000in}{0.000000in}}%
\pgfpathlineto{\pgfqpoint{0.000000in}{-0.020833in}}%
\pgfusepath{stroke,fill}%
}%
\begin{pgfscope}%
\pgfsys@transformshift{2.934892in}{3.227753in}%
\pgfsys@useobject{currentmarker}{}%
\end{pgfscope}%
\end{pgfscope}%
\begin{pgfscope}%
\pgfpathrectangle{\pgfqpoint{0.481681in}{1.080890in}}{\pgfqpoint{5.785672in}{2.146863in}}%
\pgfusepath{clip}%
\pgfsetrectcap%
\pgfsetroundjoin%
\pgfsetlinewidth{0.100375pt}%
\definecolor{currentstroke}{rgb}{0.827451,0.827451,0.827451}%
\pgfsetstrokecolor{currentstroke}%
\pgfsetdash{}{0pt}%
\pgfpathmoveto{\pgfqpoint{3.002613in}{1.080890in}}%
\pgfpathlineto{\pgfqpoint{3.002613in}{3.227753in}}%
\pgfusepath{stroke}%
\end{pgfscope}%
\begin{pgfscope}%
\pgfsetbuttcap%
\pgfsetroundjoin%
\definecolor{currentfill}{rgb}{0.000000,0.000000,0.000000}%
\pgfsetfillcolor{currentfill}%
\pgfsetlinewidth{0.501875pt}%
\definecolor{currentstroke}{rgb}{0.000000,0.000000,0.000000}%
\pgfsetstrokecolor{currentstroke}%
\pgfsetdash{}{0pt}%
\pgfsys@defobject{currentmarker}{\pgfqpoint{0.000000in}{0.000000in}}{\pgfqpoint{0.000000in}{0.020833in}}{%
\pgfpathmoveto{\pgfqpoint{0.000000in}{0.000000in}}%
\pgfpathlineto{\pgfqpoint{0.000000in}{0.020833in}}%
\pgfusepath{stroke,fill}%
}%
\begin{pgfscope}%
\pgfsys@transformshift{3.002613in}{1.080890in}%
\pgfsys@useobject{currentmarker}{}%
\end{pgfscope}%
\end{pgfscope}%
\begin{pgfscope}%
\pgfsetbuttcap%
\pgfsetroundjoin%
\definecolor{currentfill}{rgb}{0.000000,0.000000,0.000000}%
\pgfsetfillcolor{currentfill}%
\pgfsetlinewidth{0.501875pt}%
\definecolor{currentstroke}{rgb}{0.000000,0.000000,0.000000}%
\pgfsetstrokecolor{currentstroke}%
\pgfsetdash{}{0pt}%
\pgfsys@defobject{currentmarker}{\pgfqpoint{0.000000in}{-0.020833in}}{\pgfqpoint{0.000000in}{0.000000in}}{%
\pgfpathmoveto{\pgfqpoint{0.000000in}{0.000000in}}%
\pgfpathlineto{\pgfqpoint{0.000000in}{-0.020833in}}%
\pgfusepath{stroke,fill}%
}%
\begin{pgfscope}%
\pgfsys@transformshift{3.002613in}{3.227753in}%
\pgfsys@useobject{currentmarker}{}%
\end{pgfscope}%
\end{pgfscope}%
\begin{pgfscope}%
\pgfpathrectangle{\pgfqpoint{0.481681in}{1.080890in}}{\pgfqpoint{5.785672in}{2.146863in}}%
\pgfusepath{clip}%
\pgfsetrectcap%
\pgfsetroundjoin%
\pgfsetlinewidth{0.100375pt}%
\definecolor{currentstroke}{rgb}{0.827451,0.827451,0.827451}%
\pgfsetstrokecolor{currentstroke}%
\pgfsetdash{}{0pt}%
\pgfpathmoveto{\pgfqpoint{3.070335in}{1.080890in}}%
\pgfpathlineto{\pgfqpoint{3.070335in}{3.227753in}}%
\pgfusepath{stroke}%
\end{pgfscope}%
\begin{pgfscope}%
\pgfsetbuttcap%
\pgfsetroundjoin%
\definecolor{currentfill}{rgb}{0.000000,0.000000,0.000000}%
\pgfsetfillcolor{currentfill}%
\pgfsetlinewidth{0.501875pt}%
\definecolor{currentstroke}{rgb}{0.000000,0.000000,0.000000}%
\pgfsetstrokecolor{currentstroke}%
\pgfsetdash{}{0pt}%
\pgfsys@defobject{currentmarker}{\pgfqpoint{0.000000in}{0.000000in}}{\pgfqpoint{0.000000in}{0.020833in}}{%
\pgfpathmoveto{\pgfqpoint{0.000000in}{0.000000in}}%
\pgfpathlineto{\pgfqpoint{0.000000in}{0.020833in}}%
\pgfusepath{stroke,fill}%
}%
\begin{pgfscope}%
\pgfsys@transformshift{3.070335in}{1.080890in}%
\pgfsys@useobject{currentmarker}{}%
\end{pgfscope}%
\end{pgfscope}%
\begin{pgfscope}%
\pgfsetbuttcap%
\pgfsetroundjoin%
\definecolor{currentfill}{rgb}{0.000000,0.000000,0.000000}%
\pgfsetfillcolor{currentfill}%
\pgfsetlinewidth{0.501875pt}%
\definecolor{currentstroke}{rgb}{0.000000,0.000000,0.000000}%
\pgfsetstrokecolor{currentstroke}%
\pgfsetdash{}{0pt}%
\pgfsys@defobject{currentmarker}{\pgfqpoint{0.000000in}{-0.020833in}}{\pgfqpoint{0.000000in}{0.000000in}}{%
\pgfpathmoveto{\pgfqpoint{0.000000in}{0.000000in}}%
\pgfpathlineto{\pgfqpoint{0.000000in}{-0.020833in}}%
\pgfusepath{stroke,fill}%
}%
\begin{pgfscope}%
\pgfsys@transformshift{3.070335in}{3.227753in}%
\pgfsys@useobject{currentmarker}{}%
\end{pgfscope}%
\end{pgfscope}%
\begin{pgfscope}%
\pgfpathrectangle{\pgfqpoint{0.481681in}{1.080890in}}{\pgfqpoint{5.785672in}{2.146863in}}%
\pgfusepath{clip}%
\pgfsetrectcap%
\pgfsetroundjoin%
\pgfsetlinewidth{0.100375pt}%
\definecolor{currentstroke}{rgb}{0.827451,0.827451,0.827451}%
\pgfsetstrokecolor{currentstroke}%
\pgfsetdash{}{0pt}%
\pgfpathmoveto{\pgfqpoint{3.138056in}{1.080890in}}%
\pgfpathlineto{\pgfqpoint{3.138056in}{3.227753in}}%
\pgfusepath{stroke}%
\end{pgfscope}%
\begin{pgfscope}%
\pgfsetbuttcap%
\pgfsetroundjoin%
\definecolor{currentfill}{rgb}{0.000000,0.000000,0.000000}%
\pgfsetfillcolor{currentfill}%
\pgfsetlinewidth{0.501875pt}%
\definecolor{currentstroke}{rgb}{0.000000,0.000000,0.000000}%
\pgfsetstrokecolor{currentstroke}%
\pgfsetdash{}{0pt}%
\pgfsys@defobject{currentmarker}{\pgfqpoint{0.000000in}{0.000000in}}{\pgfqpoint{0.000000in}{0.020833in}}{%
\pgfpathmoveto{\pgfqpoint{0.000000in}{0.000000in}}%
\pgfpathlineto{\pgfqpoint{0.000000in}{0.020833in}}%
\pgfusepath{stroke,fill}%
}%
\begin{pgfscope}%
\pgfsys@transformshift{3.138056in}{1.080890in}%
\pgfsys@useobject{currentmarker}{}%
\end{pgfscope}%
\end{pgfscope}%
\begin{pgfscope}%
\pgfsetbuttcap%
\pgfsetroundjoin%
\definecolor{currentfill}{rgb}{0.000000,0.000000,0.000000}%
\pgfsetfillcolor{currentfill}%
\pgfsetlinewidth{0.501875pt}%
\definecolor{currentstroke}{rgb}{0.000000,0.000000,0.000000}%
\pgfsetstrokecolor{currentstroke}%
\pgfsetdash{}{0pt}%
\pgfsys@defobject{currentmarker}{\pgfqpoint{0.000000in}{-0.020833in}}{\pgfqpoint{0.000000in}{0.000000in}}{%
\pgfpathmoveto{\pgfqpoint{0.000000in}{0.000000in}}%
\pgfpathlineto{\pgfqpoint{0.000000in}{-0.020833in}}%
\pgfusepath{stroke,fill}%
}%
\begin{pgfscope}%
\pgfsys@transformshift{3.138056in}{3.227753in}%
\pgfsys@useobject{currentmarker}{}%
\end{pgfscope}%
\end{pgfscope}%
\begin{pgfscope}%
\pgfpathrectangle{\pgfqpoint{0.481681in}{1.080890in}}{\pgfqpoint{5.785672in}{2.146863in}}%
\pgfusepath{clip}%
\pgfsetrectcap%
\pgfsetroundjoin%
\pgfsetlinewidth{0.100375pt}%
\definecolor{currentstroke}{rgb}{0.827451,0.827451,0.827451}%
\pgfsetstrokecolor{currentstroke}%
\pgfsetdash{}{0pt}%
\pgfpathmoveto{\pgfqpoint{3.273499in}{1.080890in}}%
\pgfpathlineto{\pgfqpoint{3.273499in}{3.227753in}}%
\pgfusepath{stroke}%
\end{pgfscope}%
\begin{pgfscope}%
\pgfsetbuttcap%
\pgfsetroundjoin%
\definecolor{currentfill}{rgb}{0.000000,0.000000,0.000000}%
\pgfsetfillcolor{currentfill}%
\pgfsetlinewidth{0.501875pt}%
\definecolor{currentstroke}{rgb}{0.000000,0.000000,0.000000}%
\pgfsetstrokecolor{currentstroke}%
\pgfsetdash{}{0pt}%
\pgfsys@defobject{currentmarker}{\pgfqpoint{0.000000in}{0.000000in}}{\pgfqpoint{0.000000in}{0.020833in}}{%
\pgfpathmoveto{\pgfqpoint{0.000000in}{0.000000in}}%
\pgfpathlineto{\pgfqpoint{0.000000in}{0.020833in}}%
\pgfusepath{stroke,fill}%
}%
\begin{pgfscope}%
\pgfsys@transformshift{3.273499in}{1.080890in}%
\pgfsys@useobject{currentmarker}{}%
\end{pgfscope}%
\end{pgfscope}%
\begin{pgfscope}%
\pgfsetbuttcap%
\pgfsetroundjoin%
\definecolor{currentfill}{rgb}{0.000000,0.000000,0.000000}%
\pgfsetfillcolor{currentfill}%
\pgfsetlinewidth{0.501875pt}%
\definecolor{currentstroke}{rgb}{0.000000,0.000000,0.000000}%
\pgfsetstrokecolor{currentstroke}%
\pgfsetdash{}{0pt}%
\pgfsys@defobject{currentmarker}{\pgfqpoint{0.000000in}{-0.020833in}}{\pgfqpoint{0.000000in}{0.000000in}}{%
\pgfpathmoveto{\pgfqpoint{0.000000in}{0.000000in}}%
\pgfpathlineto{\pgfqpoint{0.000000in}{-0.020833in}}%
\pgfusepath{stroke,fill}%
}%
\begin{pgfscope}%
\pgfsys@transformshift{3.273499in}{3.227753in}%
\pgfsys@useobject{currentmarker}{}%
\end{pgfscope}%
\end{pgfscope}%
\begin{pgfscope}%
\pgfpathrectangle{\pgfqpoint{0.481681in}{1.080890in}}{\pgfqpoint{5.785672in}{2.146863in}}%
\pgfusepath{clip}%
\pgfsetrectcap%
\pgfsetroundjoin%
\pgfsetlinewidth{0.100375pt}%
\definecolor{currentstroke}{rgb}{0.827451,0.827451,0.827451}%
\pgfsetstrokecolor{currentstroke}%
\pgfsetdash{}{0pt}%
\pgfpathmoveto{\pgfqpoint{3.341221in}{1.080890in}}%
\pgfpathlineto{\pgfqpoint{3.341221in}{3.227753in}}%
\pgfusepath{stroke}%
\end{pgfscope}%
\begin{pgfscope}%
\pgfsetbuttcap%
\pgfsetroundjoin%
\definecolor{currentfill}{rgb}{0.000000,0.000000,0.000000}%
\pgfsetfillcolor{currentfill}%
\pgfsetlinewidth{0.501875pt}%
\definecolor{currentstroke}{rgb}{0.000000,0.000000,0.000000}%
\pgfsetstrokecolor{currentstroke}%
\pgfsetdash{}{0pt}%
\pgfsys@defobject{currentmarker}{\pgfqpoint{0.000000in}{0.000000in}}{\pgfqpoint{0.000000in}{0.020833in}}{%
\pgfpathmoveto{\pgfqpoint{0.000000in}{0.000000in}}%
\pgfpathlineto{\pgfqpoint{0.000000in}{0.020833in}}%
\pgfusepath{stroke,fill}%
}%
\begin{pgfscope}%
\pgfsys@transformshift{3.341221in}{1.080890in}%
\pgfsys@useobject{currentmarker}{}%
\end{pgfscope}%
\end{pgfscope}%
\begin{pgfscope}%
\pgfsetbuttcap%
\pgfsetroundjoin%
\definecolor{currentfill}{rgb}{0.000000,0.000000,0.000000}%
\pgfsetfillcolor{currentfill}%
\pgfsetlinewidth{0.501875pt}%
\definecolor{currentstroke}{rgb}{0.000000,0.000000,0.000000}%
\pgfsetstrokecolor{currentstroke}%
\pgfsetdash{}{0pt}%
\pgfsys@defobject{currentmarker}{\pgfqpoint{0.000000in}{-0.020833in}}{\pgfqpoint{0.000000in}{0.000000in}}{%
\pgfpathmoveto{\pgfqpoint{0.000000in}{0.000000in}}%
\pgfpathlineto{\pgfqpoint{0.000000in}{-0.020833in}}%
\pgfusepath{stroke,fill}%
}%
\begin{pgfscope}%
\pgfsys@transformshift{3.341221in}{3.227753in}%
\pgfsys@useobject{currentmarker}{}%
\end{pgfscope}%
\end{pgfscope}%
\begin{pgfscope}%
\pgfpathrectangle{\pgfqpoint{0.481681in}{1.080890in}}{\pgfqpoint{5.785672in}{2.146863in}}%
\pgfusepath{clip}%
\pgfsetrectcap%
\pgfsetroundjoin%
\pgfsetlinewidth{0.100375pt}%
\definecolor{currentstroke}{rgb}{0.827451,0.827451,0.827451}%
\pgfsetstrokecolor{currentstroke}%
\pgfsetdash{}{0pt}%
\pgfpathmoveto{\pgfqpoint{3.408942in}{1.080890in}}%
\pgfpathlineto{\pgfqpoint{3.408942in}{3.227753in}}%
\pgfusepath{stroke}%
\end{pgfscope}%
\begin{pgfscope}%
\pgfsetbuttcap%
\pgfsetroundjoin%
\definecolor{currentfill}{rgb}{0.000000,0.000000,0.000000}%
\pgfsetfillcolor{currentfill}%
\pgfsetlinewidth{0.501875pt}%
\definecolor{currentstroke}{rgb}{0.000000,0.000000,0.000000}%
\pgfsetstrokecolor{currentstroke}%
\pgfsetdash{}{0pt}%
\pgfsys@defobject{currentmarker}{\pgfqpoint{0.000000in}{0.000000in}}{\pgfqpoint{0.000000in}{0.020833in}}{%
\pgfpathmoveto{\pgfqpoint{0.000000in}{0.000000in}}%
\pgfpathlineto{\pgfqpoint{0.000000in}{0.020833in}}%
\pgfusepath{stroke,fill}%
}%
\begin{pgfscope}%
\pgfsys@transformshift{3.408942in}{1.080890in}%
\pgfsys@useobject{currentmarker}{}%
\end{pgfscope}%
\end{pgfscope}%
\begin{pgfscope}%
\pgfsetbuttcap%
\pgfsetroundjoin%
\definecolor{currentfill}{rgb}{0.000000,0.000000,0.000000}%
\pgfsetfillcolor{currentfill}%
\pgfsetlinewidth{0.501875pt}%
\definecolor{currentstroke}{rgb}{0.000000,0.000000,0.000000}%
\pgfsetstrokecolor{currentstroke}%
\pgfsetdash{}{0pt}%
\pgfsys@defobject{currentmarker}{\pgfqpoint{0.000000in}{-0.020833in}}{\pgfqpoint{0.000000in}{0.000000in}}{%
\pgfpathmoveto{\pgfqpoint{0.000000in}{0.000000in}}%
\pgfpathlineto{\pgfqpoint{0.000000in}{-0.020833in}}%
\pgfusepath{stroke,fill}%
}%
\begin{pgfscope}%
\pgfsys@transformshift{3.408942in}{3.227753in}%
\pgfsys@useobject{currentmarker}{}%
\end{pgfscope}%
\end{pgfscope}%
\begin{pgfscope}%
\pgfpathrectangle{\pgfqpoint{0.481681in}{1.080890in}}{\pgfqpoint{5.785672in}{2.146863in}}%
\pgfusepath{clip}%
\pgfsetrectcap%
\pgfsetroundjoin%
\pgfsetlinewidth{0.100375pt}%
\definecolor{currentstroke}{rgb}{0.827451,0.827451,0.827451}%
\pgfsetstrokecolor{currentstroke}%
\pgfsetdash{}{0pt}%
\pgfpathmoveto{\pgfqpoint{3.476664in}{1.080890in}}%
\pgfpathlineto{\pgfqpoint{3.476664in}{3.227753in}}%
\pgfusepath{stroke}%
\end{pgfscope}%
\begin{pgfscope}%
\pgfsetbuttcap%
\pgfsetroundjoin%
\definecolor{currentfill}{rgb}{0.000000,0.000000,0.000000}%
\pgfsetfillcolor{currentfill}%
\pgfsetlinewidth{0.501875pt}%
\definecolor{currentstroke}{rgb}{0.000000,0.000000,0.000000}%
\pgfsetstrokecolor{currentstroke}%
\pgfsetdash{}{0pt}%
\pgfsys@defobject{currentmarker}{\pgfqpoint{0.000000in}{0.000000in}}{\pgfqpoint{0.000000in}{0.020833in}}{%
\pgfpathmoveto{\pgfqpoint{0.000000in}{0.000000in}}%
\pgfpathlineto{\pgfqpoint{0.000000in}{0.020833in}}%
\pgfusepath{stroke,fill}%
}%
\begin{pgfscope}%
\pgfsys@transformshift{3.476664in}{1.080890in}%
\pgfsys@useobject{currentmarker}{}%
\end{pgfscope}%
\end{pgfscope}%
\begin{pgfscope}%
\pgfsetbuttcap%
\pgfsetroundjoin%
\definecolor{currentfill}{rgb}{0.000000,0.000000,0.000000}%
\pgfsetfillcolor{currentfill}%
\pgfsetlinewidth{0.501875pt}%
\definecolor{currentstroke}{rgb}{0.000000,0.000000,0.000000}%
\pgfsetstrokecolor{currentstroke}%
\pgfsetdash{}{0pt}%
\pgfsys@defobject{currentmarker}{\pgfqpoint{0.000000in}{-0.020833in}}{\pgfqpoint{0.000000in}{0.000000in}}{%
\pgfpathmoveto{\pgfqpoint{0.000000in}{0.000000in}}%
\pgfpathlineto{\pgfqpoint{0.000000in}{-0.020833in}}%
\pgfusepath{stroke,fill}%
}%
\begin{pgfscope}%
\pgfsys@transformshift{3.476664in}{3.227753in}%
\pgfsys@useobject{currentmarker}{}%
\end{pgfscope}%
\end{pgfscope}%
\begin{pgfscope}%
\pgfpathrectangle{\pgfqpoint{0.481681in}{1.080890in}}{\pgfqpoint{5.785672in}{2.146863in}}%
\pgfusepath{clip}%
\pgfsetrectcap%
\pgfsetroundjoin%
\pgfsetlinewidth{0.100375pt}%
\definecolor{currentstroke}{rgb}{0.827451,0.827451,0.827451}%
\pgfsetstrokecolor{currentstroke}%
\pgfsetdash{}{0pt}%
\pgfpathmoveto{\pgfqpoint{3.544385in}{1.080890in}}%
\pgfpathlineto{\pgfqpoint{3.544385in}{3.227753in}}%
\pgfusepath{stroke}%
\end{pgfscope}%
\begin{pgfscope}%
\pgfsetbuttcap%
\pgfsetroundjoin%
\definecolor{currentfill}{rgb}{0.000000,0.000000,0.000000}%
\pgfsetfillcolor{currentfill}%
\pgfsetlinewidth{0.501875pt}%
\definecolor{currentstroke}{rgb}{0.000000,0.000000,0.000000}%
\pgfsetstrokecolor{currentstroke}%
\pgfsetdash{}{0pt}%
\pgfsys@defobject{currentmarker}{\pgfqpoint{0.000000in}{0.000000in}}{\pgfqpoint{0.000000in}{0.020833in}}{%
\pgfpathmoveto{\pgfqpoint{0.000000in}{0.000000in}}%
\pgfpathlineto{\pgfqpoint{0.000000in}{0.020833in}}%
\pgfusepath{stroke,fill}%
}%
\begin{pgfscope}%
\pgfsys@transformshift{3.544385in}{1.080890in}%
\pgfsys@useobject{currentmarker}{}%
\end{pgfscope}%
\end{pgfscope}%
\begin{pgfscope}%
\pgfsetbuttcap%
\pgfsetroundjoin%
\definecolor{currentfill}{rgb}{0.000000,0.000000,0.000000}%
\pgfsetfillcolor{currentfill}%
\pgfsetlinewidth{0.501875pt}%
\definecolor{currentstroke}{rgb}{0.000000,0.000000,0.000000}%
\pgfsetstrokecolor{currentstroke}%
\pgfsetdash{}{0pt}%
\pgfsys@defobject{currentmarker}{\pgfqpoint{0.000000in}{-0.020833in}}{\pgfqpoint{0.000000in}{0.000000in}}{%
\pgfpathmoveto{\pgfqpoint{0.000000in}{0.000000in}}%
\pgfpathlineto{\pgfqpoint{0.000000in}{-0.020833in}}%
\pgfusepath{stroke,fill}%
}%
\begin{pgfscope}%
\pgfsys@transformshift{3.544385in}{3.227753in}%
\pgfsys@useobject{currentmarker}{}%
\end{pgfscope}%
\end{pgfscope}%
\begin{pgfscope}%
\pgfpathrectangle{\pgfqpoint{0.481681in}{1.080890in}}{\pgfqpoint{5.785672in}{2.146863in}}%
\pgfusepath{clip}%
\pgfsetrectcap%
\pgfsetroundjoin%
\pgfsetlinewidth{0.100375pt}%
\definecolor{currentstroke}{rgb}{0.827451,0.827451,0.827451}%
\pgfsetstrokecolor{currentstroke}%
\pgfsetdash{}{0pt}%
\pgfpathmoveto{\pgfqpoint{3.679828in}{1.080890in}}%
\pgfpathlineto{\pgfqpoint{3.679828in}{3.227753in}}%
\pgfusepath{stroke}%
\end{pgfscope}%
\begin{pgfscope}%
\pgfsetbuttcap%
\pgfsetroundjoin%
\definecolor{currentfill}{rgb}{0.000000,0.000000,0.000000}%
\pgfsetfillcolor{currentfill}%
\pgfsetlinewidth{0.501875pt}%
\definecolor{currentstroke}{rgb}{0.000000,0.000000,0.000000}%
\pgfsetstrokecolor{currentstroke}%
\pgfsetdash{}{0pt}%
\pgfsys@defobject{currentmarker}{\pgfqpoint{0.000000in}{0.000000in}}{\pgfqpoint{0.000000in}{0.020833in}}{%
\pgfpathmoveto{\pgfqpoint{0.000000in}{0.000000in}}%
\pgfpathlineto{\pgfqpoint{0.000000in}{0.020833in}}%
\pgfusepath{stroke,fill}%
}%
\begin{pgfscope}%
\pgfsys@transformshift{3.679828in}{1.080890in}%
\pgfsys@useobject{currentmarker}{}%
\end{pgfscope}%
\end{pgfscope}%
\begin{pgfscope}%
\pgfsetbuttcap%
\pgfsetroundjoin%
\definecolor{currentfill}{rgb}{0.000000,0.000000,0.000000}%
\pgfsetfillcolor{currentfill}%
\pgfsetlinewidth{0.501875pt}%
\definecolor{currentstroke}{rgb}{0.000000,0.000000,0.000000}%
\pgfsetstrokecolor{currentstroke}%
\pgfsetdash{}{0pt}%
\pgfsys@defobject{currentmarker}{\pgfqpoint{0.000000in}{-0.020833in}}{\pgfqpoint{0.000000in}{0.000000in}}{%
\pgfpathmoveto{\pgfqpoint{0.000000in}{0.000000in}}%
\pgfpathlineto{\pgfqpoint{0.000000in}{-0.020833in}}%
\pgfusepath{stroke,fill}%
}%
\begin{pgfscope}%
\pgfsys@transformshift{3.679828in}{3.227753in}%
\pgfsys@useobject{currentmarker}{}%
\end{pgfscope}%
\end{pgfscope}%
\begin{pgfscope}%
\pgfpathrectangle{\pgfqpoint{0.481681in}{1.080890in}}{\pgfqpoint{5.785672in}{2.146863in}}%
\pgfusepath{clip}%
\pgfsetrectcap%
\pgfsetroundjoin%
\pgfsetlinewidth{0.100375pt}%
\definecolor{currentstroke}{rgb}{0.827451,0.827451,0.827451}%
\pgfsetstrokecolor{currentstroke}%
\pgfsetdash{}{0pt}%
\pgfpathmoveto{\pgfqpoint{3.747549in}{1.080890in}}%
\pgfpathlineto{\pgfqpoint{3.747549in}{3.227753in}}%
\pgfusepath{stroke}%
\end{pgfscope}%
\begin{pgfscope}%
\pgfsetbuttcap%
\pgfsetroundjoin%
\definecolor{currentfill}{rgb}{0.000000,0.000000,0.000000}%
\pgfsetfillcolor{currentfill}%
\pgfsetlinewidth{0.501875pt}%
\definecolor{currentstroke}{rgb}{0.000000,0.000000,0.000000}%
\pgfsetstrokecolor{currentstroke}%
\pgfsetdash{}{0pt}%
\pgfsys@defobject{currentmarker}{\pgfqpoint{0.000000in}{0.000000in}}{\pgfqpoint{0.000000in}{0.020833in}}{%
\pgfpathmoveto{\pgfqpoint{0.000000in}{0.000000in}}%
\pgfpathlineto{\pgfqpoint{0.000000in}{0.020833in}}%
\pgfusepath{stroke,fill}%
}%
\begin{pgfscope}%
\pgfsys@transformshift{3.747549in}{1.080890in}%
\pgfsys@useobject{currentmarker}{}%
\end{pgfscope}%
\end{pgfscope}%
\begin{pgfscope}%
\pgfsetbuttcap%
\pgfsetroundjoin%
\definecolor{currentfill}{rgb}{0.000000,0.000000,0.000000}%
\pgfsetfillcolor{currentfill}%
\pgfsetlinewidth{0.501875pt}%
\definecolor{currentstroke}{rgb}{0.000000,0.000000,0.000000}%
\pgfsetstrokecolor{currentstroke}%
\pgfsetdash{}{0pt}%
\pgfsys@defobject{currentmarker}{\pgfqpoint{0.000000in}{-0.020833in}}{\pgfqpoint{0.000000in}{0.000000in}}{%
\pgfpathmoveto{\pgfqpoint{0.000000in}{0.000000in}}%
\pgfpathlineto{\pgfqpoint{0.000000in}{-0.020833in}}%
\pgfusepath{stroke,fill}%
}%
\begin{pgfscope}%
\pgfsys@transformshift{3.747549in}{3.227753in}%
\pgfsys@useobject{currentmarker}{}%
\end{pgfscope}%
\end{pgfscope}%
\begin{pgfscope}%
\pgfpathrectangle{\pgfqpoint{0.481681in}{1.080890in}}{\pgfqpoint{5.785672in}{2.146863in}}%
\pgfusepath{clip}%
\pgfsetrectcap%
\pgfsetroundjoin%
\pgfsetlinewidth{0.100375pt}%
\definecolor{currentstroke}{rgb}{0.827451,0.827451,0.827451}%
\pgfsetstrokecolor{currentstroke}%
\pgfsetdash{}{0pt}%
\pgfpathmoveto{\pgfqpoint{3.815271in}{1.080890in}}%
\pgfpathlineto{\pgfqpoint{3.815271in}{3.227753in}}%
\pgfusepath{stroke}%
\end{pgfscope}%
\begin{pgfscope}%
\pgfsetbuttcap%
\pgfsetroundjoin%
\definecolor{currentfill}{rgb}{0.000000,0.000000,0.000000}%
\pgfsetfillcolor{currentfill}%
\pgfsetlinewidth{0.501875pt}%
\definecolor{currentstroke}{rgb}{0.000000,0.000000,0.000000}%
\pgfsetstrokecolor{currentstroke}%
\pgfsetdash{}{0pt}%
\pgfsys@defobject{currentmarker}{\pgfqpoint{0.000000in}{0.000000in}}{\pgfqpoint{0.000000in}{0.020833in}}{%
\pgfpathmoveto{\pgfqpoint{0.000000in}{0.000000in}}%
\pgfpathlineto{\pgfqpoint{0.000000in}{0.020833in}}%
\pgfusepath{stroke,fill}%
}%
\begin{pgfscope}%
\pgfsys@transformshift{3.815271in}{1.080890in}%
\pgfsys@useobject{currentmarker}{}%
\end{pgfscope}%
\end{pgfscope}%
\begin{pgfscope}%
\pgfsetbuttcap%
\pgfsetroundjoin%
\definecolor{currentfill}{rgb}{0.000000,0.000000,0.000000}%
\pgfsetfillcolor{currentfill}%
\pgfsetlinewidth{0.501875pt}%
\definecolor{currentstroke}{rgb}{0.000000,0.000000,0.000000}%
\pgfsetstrokecolor{currentstroke}%
\pgfsetdash{}{0pt}%
\pgfsys@defobject{currentmarker}{\pgfqpoint{0.000000in}{-0.020833in}}{\pgfqpoint{0.000000in}{0.000000in}}{%
\pgfpathmoveto{\pgfqpoint{0.000000in}{0.000000in}}%
\pgfpathlineto{\pgfqpoint{0.000000in}{-0.020833in}}%
\pgfusepath{stroke,fill}%
}%
\begin{pgfscope}%
\pgfsys@transformshift{3.815271in}{3.227753in}%
\pgfsys@useobject{currentmarker}{}%
\end{pgfscope}%
\end{pgfscope}%
\begin{pgfscope}%
\pgfpathrectangle{\pgfqpoint{0.481681in}{1.080890in}}{\pgfqpoint{5.785672in}{2.146863in}}%
\pgfusepath{clip}%
\pgfsetrectcap%
\pgfsetroundjoin%
\pgfsetlinewidth{0.100375pt}%
\definecolor{currentstroke}{rgb}{0.827451,0.827451,0.827451}%
\pgfsetstrokecolor{currentstroke}%
\pgfsetdash{}{0pt}%
\pgfpathmoveto{\pgfqpoint{3.882992in}{1.080890in}}%
\pgfpathlineto{\pgfqpoint{3.882992in}{3.227753in}}%
\pgfusepath{stroke}%
\end{pgfscope}%
\begin{pgfscope}%
\pgfsetbuttcap%
\pgfsetroundjoin%
\definecolor{currentfill}{rgb}{0.000000,0.000000,0.000000}%
\pgfsetfillcolor{currentfill}%
\pgfsetlinewidth{0.501875pt}%
\definecolor{currentstroke}{rgb}{0.000000,0.000000,0.000000}%
\pgfsetstrokecolor{currentstroke}%
\pgfsetdash{}{0pt}%
\pgfsys@defobject{currentmarker}{\pgfqpoint{0.000000in}{0.000000in}}{\pgfqpoint{0.000000in}{0.020833in}}{%
\pgfpathmoveto{\pgfqpoint{0.000000in}{0.000000in}}%
\pgfpathlineto{\pgfqpoint{0.000000in}{0.020833in}}%
\pgfusepath{stroke,fill}%
}%
\begin{pgfscope}%
\pgfsys@transformshift{3.882992in}{1.080890in}%
\pgfsys@useobject{currentmarker}{}%
\end{pgfscope}%
\end{pgfscope}%
\begin{pgfscope}%
\pgfsetbuttcap%
\pgfsetroundjoin%
\definecolor{currentfill}{rgb}{0.000000,0.000000,0.000000}%
\pgfsetfillcolor{currentfill}%
\pgfsetlinewidth{0.501875pt}%
\definecolor{currentstroke}{rgb}{0.000000,0.000000,0.000000}%
\pgfsetstrokecolor{currentstroke}%
\pgfsetdash{}{0pt}%
\pgfsys@defobject{currentmarker}{\pgfqpoint{0.000000in}{-0.020833in}}{\pgfqpoint{0.000000in}{0.000000in}}{%
\pgfpathmoveto{\pgfqpoint{0.000000in}{0.000000in}}%
\pgfpathlineto{\pgfqpoint{0.000000in}{-0.020833in}}%
\pgfusepath{stroke,fill}%
}%
\begin{pgfscope}%
\pgfsys@transformshift{3.882992in}{3.227753in}%
\pgfsys@useobject{currentmarker}{}%
\end{pgfscope}%
\end{pgfscope}%
\begin{pgfscope}%
\pgfpathrectangle{\pgfqpoint{0.481681in}{1.080890in}}{\pgfqpoint{5.785672in}{2.146863in}}%
\pgfusepath{clip}%
\pgfsetrectcap%
\pgfsetroundjoin%
\pgfsetlinewidth{0.100375pt}%
\definecolor{currentstroke}{rgb}{0.827451,0.827451,0.827451}%
\pgfsetstrokecolor{currentstroke}%
\pgfsetdash{}{0pt}%
\pgfpathmoveto{\pgfqpoint{3.950714in}{1.080890in}}%
\pgfpathlineto{\pgfqpoint{3.950714in}{3.227753in}}%
\pgfusepath{stroke}%
\end{pgfscope}%
\begin{pgfscope}%
\pgfsetbuttcap%
\pgfsetroundjoin%
\definecolor{currentfill}{rgb}{0.000000,0.000000,0.000000}%
\pgfsetfillcolor{currentfill}%
\pgfsetlinewidth{0.501875pt}%
\definecolor{currentstroke}{rgb}{0.000000,0.000000,0.000000}%
\pgfsetstrokecolor{currentstroke}%
\pgfsetdash{}{0pt}%
\pgfsys@defobject{currentmarker}{\pgfqpoint{0.000000in}{0.000000in}}{\pgfqpoint{0.000000in}{0.020833in}}{%
\pgfpathmoveto{\pgfqpoint{0.000000in}{0.000000in}}%
\pgfpathlineto{\pgfqpoint{0.000000in}{0.020833in}}%
\pgfusepath{stroke,fill}%
}%
\begin{pgfscope}%
\pgfsys@transformshift{3.950714in}{1.080890in}%
\pgfsys@useobject{currentmarker}{}%
\end{pgfscope}%
\end{pgfscope}%
\begin{pgfscope}%
\pgfsetbuttcap%
\pgfsetroundjoin%
\definecolor{currentfill}{rgb}{0.000000,0.000000,0.000000}%
\pgfsetfillcolor{currentfill}%
\pgfsetlinewidth{0.501875pt}%
\definecolor{currentstroke}{rgb}{0.000000,0.000000,0.000000}%
\pgfsetstrokecolor{currentstroke}%
\pgfsetdash{}{0pt}%
\pgfsys@defobject{currentmarker}{\pgfqpoint{0.000000in}{-0.020833in}}{\pgfqpoint{0.000000in}{0.000000in}}{%
\pgfpathmoveto{\pgfqpoint{0.000000in}{0.000000in}}%
\pgfpathlineto{\pgfqpoint{0.000000in}{-0.020833in}}%
\pgfusepath{stroke,fill}%
}%
\begin{pgfscope}%
\pgfsys@transformshift{3.950714in}{3.227753in}%
\pgfsys@useobject{currentmarker}{}%
\end{pgfscope}%
\end{pgfscope}%
\begin{pgfscope}%
\pgfpathrectangle{\pgfqpoint{0.481681in}{1.080890in}}{\pgfqpoint{5.785672in}{2.146863in}}%
\pgfusepath{clip}%
\pgfsetrectcap%
\pgfsetroundjoin%
\pgfsetlinewidth{0.100375pt}%
\definecolor{currentstroke}{rgb}{0.827451,0.827451,0.827451}%
\pgfsetstrokecolor{currentstroke}%
\pgfsetdash{}{0pt}%
\pgfpathmoveto{\pgfqpoint{4.086157in}{1.080890in}}%
\pgfpathlineto{\pgfqpoint{4.086157in}{3.227753in}}%
\pgfusepath{stroke}%
\end{pgfscope}%
\begin{pgfscope}%
\pgfsetbuttcap%
\pgfsetroundjoin%
\definecolor{currentfill}{rgb}{0.000000,0.000000,0.000000}%
\pgfsetfillcolor{currentfill}%
\pgfsetlinewidth{0.501875pt}%
\definecolor{currentstroke}{rgb}{0.000000,0.000000,0.000000}%
\pgfsetstrokecolor{currentstroke}%
\pgfsetdash{}{0pt}%
\pgfsys@defobject{currentmarker}{\pgfqpoint{0.000000in}{0.000000in}}{\pgfqpoint{0.000000in}{0.020833in}}{%
\pgfpathmoveto{\pgfqpoint{0.000000in}{0.000000in}}%
\pgfpathlineto{\pgfqpoint{0.000000in}{0.020833in}}%
\pgfusepath{stroke,fill}%
}%
\begin{pgfscope}%
\pgfsys@transformshift{4.086157in}{1.080890in}%
\pgfsys@useobject{currentmarker}{}%
\end{pgfscope}%
\end{pgfscope}%
\begin{pgfscope}%
\pgfsetbuttcap%
\pgfsetroundjoin%
\definecolor{currentfill}{rgb}{0.000000,0.000000,0.000000}%
\pgfsetfillcolor{currentfill}%
\pgfsetlinewidth{0.501875pt}%
\definecolor{currentstroke}{rgb}{0.000000,0.000000,0.000000}%
\pgfsetstrokecolor{currentstroke}%
\pgfsetdash{}{0pt}%
\pgfsys@defobject{currentmarker}{\pgfqpoint{0.000000in}{-0.020833in}}{\pgfqpoint{0.000000in}{0.000000in}}{%
\pgfpathmoveto{\pgfqpoint{0.000000in}{0.000000in}}%
\pgfpathlineto{\pgfqpoint{0.000000in}{-0.020833in}}%
\pgfusepath{stroke,fill}%
}%
\begin{pgfscope}%
\pgfsys@transformshift{4.086157in}{3.227753in}%
\pgfsys@useobject{currentmarker}{}%
\end{pgfscope}%
\end{pgfscope}%
\begin{pgfscope}%
\pgfpathrectangle{\pgfqpoint{0.481681in}{1.080890in}}{\pgfqpoint{5.785672in}{2.146863in}}%
\pgfusepath{clip}%
\pgfsetrectcap%
\pgfsetroundjoin%
\pgfsetlinewidth{0.100375pt}%
\definecolor{currentstroke}{rgb}{0.827451,0.827451,0.827451}%
\pgfsetstrokecolor{currentstroke}%
\pgfsetdash{}{0pt}%
\pgfpathmoveto{\pgfqpoint{4.153878in}{1.080890in}}%
\pgfpathlineto{\pgfqpoint{4.153878in}{3.227753in}}%
\pgfusepath{stroke}%
\end{pgfscope}%
\begin{pgfscope}%
\pgfsetbuttcap%
\pgfsetroundjoin%
\definecolor{currentfill}{rgb}{0.000000,0.000000,0.000000}%
\pgfsetfillcolor{currentfill}%
\pgfsetlinewidth{0.501875pt}%
\definecolor{currentstroke}{rgb}{0.000000,0.000000,0.000000}%
\pgfsetstrokecolor{currentstroke}%
\pgfsetdash{}{0pt}%
\pgfsys@defobject{currentmarker}{\pgfqpoint{0.000000in}{0.000000in}}{\pgfqpoint{0.000000in}{0.020833in}}{%
\pgfpathmoveto{\pgfqpoint{0.000000in}{0.000000in}}%
\pgfpathlineto{\pgfqpoint{0.000000in}{0.020833in}}%
\pgfusepath{stroke,fill}%
}%
\begin{pgfscope}%
\pgfsys@transformshift{4.153878in}{1.080890in}%
\pgfsys@useobject{currentmarker}{}%
\end{pgfscope}%
\end{pgfscope}%
\begin{pgfscope}%
\pgfsetbuttcap%
\pgfsetroundjoin%
\definecolor{currentfill}{rgb}{0.000000,0.000000,0.000000}%
\pgfsetfillcolor{currentfill}%
\pgfsetlinewidth{0.501875pt}%
\definecolor{currentstroke}{rgb}{0.000000,0.000000,0.000000}%
\pgfsetstrokecolor{currentstroke}%
\pgfsetdash{}{0pt}%
\pgfsys@defobject{currentmarker}{\pgfqpoint{0.000000in}{-0.020833in}}{\pgfqpoint{0.000000in}{0.000000in}}{%
\pgfpathmoveto{\pgfqpoint{0.000000in}{0.000000in}}%
\pgfpathlineto{\pgfqpoint{0.000000in}{-0.020833in}}%
\pgfusepath{stroke,fill}%
}%
\begin{pgfscope}%
\pgfsys@transformshift{4.153878in}{3.227753in}%
\pgfsys@useobject{currentmarker}{}%
\end{pgfscope}%
\end{pgfscope}%
\begin{pgfscope}%
\pgfpathrectangle{\pgfqpoint{0.481681in}{1.080890in}}{\pgfqpoint{5.785672in}{2.146863in}}%
\pgfusepath{clip}%
\pgfsetrectcap%
\pgfsetroundjoin%
\pgfsetlinewidth{0.100375pt}%
\definecolor{currentstroke}{rgb}{0.827451,0.827451,0.827451}%
\pgfsetstrokecolor{currentstroke}%
\pgfsetdash{}{0pt}%
\pgfpathmoveto{\pgfqpoint{4.221600in}{1.080890in}}%
\pgfpathlineto{\pgfqpoint{4.221600in}{3.227753in}}%
\pgfusepath{stroke}%
\end{pgfscope}%
\begin{pgfscope}%
\pgfsetbuttcap%
\pgfsetroundjoin%
\definecolor{currentfill}{rgb}{0.000000,0.000000,0.000000}%
\pgfsetfillcolor{currentfill}%
\pgfsetlinewidth{0.501875pt}%
\definecolor{currentstroke}{rgb}{0.000000,0.000000,0.000000}%
\pgfsetstrokecolor{currentstroke}%
\pgfsetdash{}{0pt}%
\pgfsys@defobject{currentmarker}{\pgfqpoint{0.000000in}{0.000000in}}{\pgfqpoint{0.000000in}{0.020833in}}{%
\pgfpathmoveto{\pgfqpoint{0.000000in}{0.000000in}}%
\pgfpathlineto{\pgfqpoint{0.000000in}{0.020833in}}%
\pgfusepath{stroke,fill}%
}%
\begin{pgfscope}%
\pgfsys@transformshift{4.221600in}{1.080890in}%
\pgfsys@useobject{currentmarker}{}%
\end{pgfscope}%
\end{pgfscope}%
\begin{pgfscope}%
\pgfsetbuttcap%
\pgfsetroundjoin%
\definecolor{currentfill}{rgb}{0.000000,0.000000,0.000000}%
\pgfsetfillcolor{currentfill}%
\pgfsetlinewidth{0.501875pt}%
\definecolor{currentstroke}{rgb}{0.000000,0.000000,0.000000}%
\pgfsetstrokecolor{currentstroke}%
\pgfsetdash{}{0pt}%
\pgfsys@defobject{currentmarker}{\pgfqpoint{0.000000in}{-0.020833in}}{\pgfqpoint{0.000000in}{0.000000in}}{%
\pgfpathmoveto{\pgfqpoint{0.000000in}{0.000000in}}%
\pgfpathlineto{\pgfqpoint{0.000000in}{-0.020833in}}%
\pgfusepath{stroke,fill}%
}%
\begin{pgfscope}%
\pgfsys@transformshift{4.221600in}{3.227753in}%
\pgfsys@useobject{currentmarker}{}%
\end{pgfscope}%
\end{pgfscope}%
\begin{pgfscope}%
\pgfpathrectangle{\pgfqpoint{0.481681in}{1.080890in}}{\pgfqpoint{5.785672in}{2.146863in}}%
\pgfusepath{clip}%
\pgfsetrectcap%
\pgfsetroundjoin%
\pgfsetlinewidth{0.100375pt}%
\definecolor{currentstroke}{rgb}{0.827451,0.827451,0.827451}%
\pgfsetstrokecolor{currentstroke}%
\pgfsetdash{}{0pt}%
\pgfpathmoveto{\pgfqpoint{4.289321in}{1.080890in}}%
\pgfpathlineto{\pgfqpoint{4.289321in}{3.227753in}}%
\pgfusepath{stroke}%
\end{pgfscope}%
\begin{pgfscope}%
\pgfsetbuttcap%
\pgfsetroundjoin%
\definecolor{currentfill}{rgb}{0.000000,0.000000,0.000000}%
\pgfsetfillcolor{currentfill}%
\pgfsetlinewidth{0.501875pt}%
\definecolor{currentstroke}{rgb}{0.000000,0.000000,0.000000}%
\pgfsetstrokecolor{currentstroke}%
\pgfsetdash{}{0pt}%
\pgfsys@defobject{currentmarker}{\pgfqpoint{0.000000in}{0.000000in}}{\pgfqpoint{0.000000in}{0.020833in}}{%
\pgfpathmoveto{\pgfqpoint{0.000000in}{0.000000in}}%
\pgfpathlineto{\pgfqpoint{0.000000in}{0.020833in}}%
\pgfusepath{stroke,fill}%
}%
\begin{pgfscope}%
\pgfsys@transformshift{4.289321in}{1.080890in}%
\pgfsys@useobject{currentmarker}{}%
\end{pgfscope}%
\end{pgfscope}%
\begin{pgfscope}%
\pgfsetbuttcap%
\pgfsetroundjoin%
\definecolor{currentfill}{rgb}{0.000000,0.000000,0.000000}%
\pgfsetfillcolor{currentfill}%
\pgfsetlinewidth{0.501875pt}%
\definecolor{currentstroke}{rgb}{0.000000,0.000000,0.000000}%
\pgfsetstrokecolor{currentstroke}%
\pgfsetdash{}{0pt}%
\pgfsys@defobject{currentmarker}{\pgfqpoint{0.000000in}{-0.020833in}}{\pgfqpoint{0.000000in}{0.000000in}}{%
\pgfpathmoveto{\pgfqpoint{0.000000in}{0.000000in}}%
\pgfpathlineto{\pgfqpoint{0.000000in}{-0.020833in}}%
\pgfusepath{stroke,fill}%
}%
\begin{pgfscope}%
\pgfsys@transformshift{4.289321in}{3.227753in}%
\pgfsys@useobject{currentmarker}{}%
\end{pgfscope}%
\end{pgfscope}%
\begin{pgfscope}%
\pgfpathrectangle{\pgfqpoint{0.481681in}{1.080890in}}{\pgfqpoint{5.785672in}{2.146863in}}%
\pgfusepath{clip}%
\pgfsetrectcap%
\pgfsetroundjoin%
\pgfsetlinewidth{0.100375pt}%
\definecolor{currentstroke}{rgb}{0.827451,0.827451,0.827451}%
\pgfsetstrokecolor{currentstroke}%
\pgfsetdash{}{0pt}%
\pgfpathmoveto{\pgfqpoint{4.357043in}{1.080890in}}%
\pgfpathlineto{\pgfqpoint{4.357043in}{3.227753in}}%
\pgfusepath{stroke}%
\end{pgfscope}%
\begin{pgfscope}%
\pgfsetbuttcap%
\pgfsetroundjoin%
\definecolor{currentfill}{rgb}{0.000000,0.000000,0.000000}%
\pgfsetfillcolor{currentfill}%
\pgfsetlinewidth{0.501875pt}%
\definecolor{currentstroke}{rgb}{0.000000,0.000000,0.000000}%
\pgfsetstrokecolor{currentstroke}%
\pgfsetdash{}{0pt}%
\pgfsys@defobject{currentmarker}{\pgfqpoint{0.000000in}{0.000000in}}{\pgfqpoint{0.000000in}{0.020833in}}{%
\pgfpathmoveto{\pgfqpoint{0.000000in}{0.000000in}}%
\pgfpathlineto{\pgfqpoint{0.000000in}{0.020833in}}%
\pgfusepath{stroke,fill}%
}%
\begin{pgfscope}%
\pgfsys@transformshift{4.357043in}{1.080890in}%
\pgfsys@useobject{currentmarker}{}%
\end{pgfscope}%
\end{pgfscope}%
\begin{pgfscope}%
\pgfsetbuttcap%
\pgfsetroundjoin%
\definecolor{currentfill}{rgb}{0.000000,0.000000,0.000000}%
\pgfsetfillcolor{currentfill}%
\pgfsetlinewidth{0.501875pt}%
\definecolor{currentstroke}{rgb}{0.000000,0.000000,0.000000}%
\pgfsetstrokecolor{currentstroke}%
\pgfsetdash{}{0pt}%
\pgfsys@defobject{currentmarker}{\pgfqpoint{0.000000in}{-0.020833in}}{\pgfqpoint{0.000000in}{0.000000in}}{%
\pgfpathmoveto{\pgfqpoint{0.000000in}{0.000000in}}%
\pgfpathlineto{\pgfqpoint{0.000000in}{-0.020833in}}%
\pgfusepath{stroke,fill}%
}%
\begin{pgfscope}%
\pgfsys@transformshift{4.357043in}{3.227753in}%
\pgfsys@useobject{currentmarker}{}%
\end{pgfscope}%
\end{pgfscope}%
\begin{pgfscope}%
\pgfpathrectangle{\pgfqpoint{0.481681in}{1.080890in}}{\pgfqpoint{5.785672in}{2.146863in}}%
\pgfusepath{clip}%
\pgfsetrectcap%
\pgfsetroundjoin%
\pgfsetlinewidth{0.100375pt}%
\definecolor{currentstroke}{rgb}{0.827451,0.827451,0.827451}%
\pgfsetstrokecolor{currentstroke}%
\pgfsetdash{}{0pt}%
\pgfpathmoveto{\pgfqpoint{4.492486in}{1.080890in}}%
\pgfpathlineto{\pgfqpoint{4.492486in}{3.227753in}}%
\pgfusepath{stroke}%
\end{pgfscope}%
\begin{pgfscope}%
\pgfsetbuttcap%
\pgfsetroundjoin%
\definecolor{currentfill}{rgb}{0.000000,0.000000,0.000000}%
\pgfsetfillcolor{currentfill}%
\pgfsetlinewidth{0.501875pt}%
\definecolor{currentstroke}{rgb}{0.000000,0.000000,0.000000}%
\pgfsetstrokecolor{currentstroke}%
\pgfsetdash{}{0pt}%
\pgfsys@defobject{currentmarker}{\pgfqpoint{0.000000in}{0.000000in}}{\pgfqpoint{0.000000in}{0.020833in}}{%
\pgfpathmoveto{\pgfqpoint{0.000000in}{0.000000in}}%
\pgfpathlineto{\pgfqpoint{0.000000in}{0.020833in}}%
\pgfusepath{stroke,fill}%
}%
\begin{pgfscope}%
\pgfsys@transformshift{4.492486in}{1.080890in}%
\pgfsys@useobject{currentmarker}{}%
\end{pgfscope}%
\end{pgfscope}%
\begin{pgfscope}%
\pgfsetbuttcap%
\pgfsetroundjoin%
\definecolor{currentfill}{rgb}{0.000000,0.000000,0.000000}%
\pgfsetfillcolor{currentfill}%
\pgfsetlinewidth{0.501875pt}%
\definecolor{currentstroke}{rgb}{0.000000,0.000000,0.000000}%
\pgfsetstrokecolor{currentstroke}%
\pgfsetdash{}{0pt}%
\pgfsys@defobject{currentmarker}{\pgfqpoint{0.000000in}{-0.020833in}}{\pgfqpoint{0.000000in}{0.000000in}}{%
\pgfpathmoveto{\pgfqpoint{0.000000in}{0.000000in}}%
\pgfpathlineto{\pgfqpoint{0.000000in}{-0.020833in}}%
\pgfusepath{stroke,fill}%
}%
\begin{pgfscope}%
\pgfsys@transformshift{4.492486in}{3.227753in}%
\pgfsys@useobject{currentmarker}{}%
\end{pgfscope}%
\end{pgfscope}%
\begin{pgfscope}%
\pgfpathrectangle{\pgfqpoint{0.481681in}{1.080890in}}{\pgfqpoint{5.785672in}{2.146863in}}%
\pgfusepath{clip}%
\pgfsetrectcap%
\pgfsetroundjoin%
\pgfsetlinewidth{0.100375pt}%
\definecolor{currentstroke}{rgb}{0.827451,0.827451,0.827451}%
\pgfsetstrokecolor{currentstroke}%
\pgfsetdash{}{0pt}%
\pgfpathmoveto{\pgfqpoint{4.560207in}{1.080890in}}%
\pgfpathlineto{\pgfqpoint{4.560207in}{3.227753in}}%
\pgfusepath{stroke}%
\end{pgfscope}%
\begin{pgfscope}%
\pgfsetbuttcap%
\pgfsetroundjoin%
\definecolor{currentfill}{rgb}{0.000000,0.000000,0.000000}%
\pgfsetfillcolor{currentfill}%
\pgfsetlinewidth{0.501875pt}%
\definecolor{currentstroke}{rgb}{0.000000,0.000000,0.000000}%
\pgfsetstrokecolor{currentstroke}%
\pgfsetdash{}{0pt}%
\pgfsys@defobject{currentmarker}{\pgfqpoint{0.000000in}{0.000000in}}{\pgfqpoint{0.000000in}{0.020833in}}{%
\pgfpathmoveto{\pgfqpoint{0.000000in}{0.000000in}}%
\pgfpathlineto{\pgfqpoint{0.000000in}{0.020833in}}%
\pgfusepath{stroke,fill}%
}%
\begin{pgfscope}%
\pgfsys@transformshift{4.560207in}{1.080890in}%
\pgfsys@useobject{currentmarker}{}%
\end{pgfscope}%
\end{pgfscope}%
\begin{pgfscope}%
\pgfsetbuttcap%
\pgfsetroundjoin%
\definecolor{currentfill}{rgb}{0.000000,0.000000,0.000000}%
\pgfsetfillcolor{currentfill}%
\pgfsetlinewidth{0.501875pt}%
\definecolor{currentstroke}{rgb}{0.000000,0.000000,0.000000}%
\pgfsetstrokecolor{currentstroke}%
\pgfsetdash{}{0pt}%
\pgfsys@defobject{currentmarker}{\pgfqpoint{0.000000in}{-0.020833in}}{\pgfqpoint{0.000000in}{0.000000in}}{%
\pgfpathmoveto{\pgfqpoint{0.000000in}{0.000000in}}%
\pgfpathlineto{\pgfqpoint{0.000000in}{-0.020833in}}%
\pgfusepath{stroke,fill}%
}%
\begin{pgfscope}%
\pgfsys@transformshift{4.560207in}{3.227753in}%
\pgfsys@useobject{currentmarker}{}%
\end{pgfscope}%
\end{pgfscope}%
\begin{pgfscope}%
\pgfpathrectangle{\pgfqpoint{0.481681in}{1.080890in}}{\pgfqpoint{5.785672in}{2.146863in}}%
\pgfusepath{clip}%
\pgfsetrectcap%
\pgfsetroundjoin%
\pgfsetlinewidth{0.100375pt}%
\definecolor{currentstroke}{rgb}{0.827451,0.827451,0.827451}%
\pgfsetstrokecolor{currentstroke}%
\pgfsetdash{}{0pt}%
\pgfpathmoveto{\pgfqpoint{4.627929in}{1.080890in}}%
\pgfpathlineto{\pgfqpoint{4.627929in}{3.227753in}}%
\pgfusepath{stroke}%
\end{pgfscope}%
\begin{pgfscope}%
\pgfsetbuttcap%
\pgfsetroundjoin%
\definecolor{currentfill}{rgb}{0.000000,0.000000,0.000000}%
\pgfsetfillcolor{currentfill}%
\pgfsetlinewidth{0.501875pt}%
\definecolor{currentstroke}{rgb}{0.000000,0.000000,0.000000}%
\pgfsetstrokecolor{currentstroke}%
\pgfsetdash{}{0pt}%
\pgfsys@defobject{currentmarker}{\pgfqpoint{0.000000in}{0.000000in}}{\pgfqpoint{0.000000in}{0.020833in}}{%
\pgfpathmoveto{\pgfqpoint{0.000000in}{0.000000in}}%
\pgfpathlineto{\pgfqpoint{0.000000in}{0.020833in}}%
\pgfusepath{stroke,fill}%
}%
\begin{pgfscope}%
\pgfsys@transformshift{4.627929in}{1.080890in}%
\pgfsys@useobject{currentmarker}{}%
\end{pgfscope}%
\end{pgfscope}%
\begin{pgfscope}%
\pgfsetbuttcap%
\pgfsetroundjoin%
\definecolor{currentfill}{rgb}{0.000000,0.000000,0.000000}%
\pgfsetfillcolor{currentfill}%
\pgfsetlinewidth{0.501875pt}%
\definecolor{currentstroke}{rgb}{0.000000,0.000000,0.000000}%
\pgfsetstrokecolor{currentstroke}%
\pgfsetdash{}{0pt}%
\pgfsys@defobject{currentmarker}{\pgfqpoint{0.000000in}{-0.020833in}}{\pgfqpoint{0.000000in}{0.000000in}}{%
\pgfpathmoveto{\pgfqpoint{0.000000in}{0.000000in}}%
\pgfpathlineto{\pgfqpoint{0.000000in}{-0.020833in}}%
\pgfusepath{stroke,fill}%
}%
\begin{pgfscope}%
\pgfsys@transformshift{4.627929in}{3.227753in}%
\pgfsys@useobject{currentmarker}{}%
\end{pgfscope}%
\end{pgfscope}%
\begin{pgfscope}%
\pgfpathrectangle{\pgfqpoint{0.481681in}{1.080890in}}{\pgfqpoint{5.785672in}{2.146863in}}%
\pgfusepath{clip}%
\pgfsetrectcap%
\pgfsetroundjoin%
\pgfsetlinewidth{0.100375pt}%
\definecolor{currentstroke}{rgb}{0.827451,0.827451,0.827451}%
\pgfsetstrokecolor{currentstroke}%
\pgfsetdash{}{0pt}%
\pgfpathmoveto{\pgfqpoint{4.695650in}{1.080890in}}%
\pgfpathlineto{\pgfqpoint{4.695650in}{3.227753in}}%
\pgfusepath{stroke}%
\end{pgfscope}%
\begin{pgfscope}%
\pgfsetbuttcap%
\pgfsetroundjoin%
\definecolor{currentfill}{rgb}{0.000000,0.000000,0.000000}%
\pgfsetfillcolor{currentfill}%
\pgfsetlinewidth{0.501875pt}%
\definecolor{currentstroke}{rgb}{0.000000,0.000000,0.000000}%
\pgfsetstrokecolor{currentstroke}%
\pgfsetdash{}{0pt}%
\pgfsys@defobject{currentmarker}{\pgfqpoint{0.000000in}{0.000000in}}{\pgfqpoint{0.000000in}{0.020833in}}{%
\pgfpathmoveto{\pgfqpoint{0.000000in}{0.000000in}}%
\pgfpathlineto{\pgfqpoint{0.000000in}{0.020833in}}%
\pgfusepath{stroke,fill}%
}%
\begin{pgfscope}%
\pgfsys@transformshift{4.695650in}{1.080890in}%
\pgfsys@useobject{currentmarker}{}%
\end{pgfscope}%
\end{pgfscope}%
\begin{pgfscope}%
\pgfsetbuttcap%
\pgfsetroundjoin%
\definecolor{currentfill}{rgb}{0.000000,0.000000,0.000000}%
\pgfsetfillcolor{currentfill}%
\pgfsetlinewidth{0.501875pt}%
\definecolor{currentstroke}{rgb}{0.000000,0.000000,0.000000}%
\pgfsetstrokecolor{currentstroke}%
\pgfsetdash{}{0pt}%
\pgfsys@defobject{currentmarker}{\pgfqpoint{0.000000in}{-0.020833in}}{\pgfqpoint{0.000000in}{0.000000in}}{%
\pgfpathmoveto{\pgfqpoint{0.000000in}{0.000000in}}%
\pgfpathlineto{\pgfqpoint{0.000000in}{-0.020833in}}%
\pgfusepath{stroke,fill}%
}%
\begin{pgfscope}%
\pgfsys@transformshift{4.695650in}{3.227753in}%
\pgfsys@useobject{currentmarker}{}%
\end{pgfscope}%
\end{pgfscope}%
\begin{pgfscope}%
\pgfpathrectangle{\pgfqpoint{0.481681in}{1.080890in}}{\pgfqpoint{5.785672in}{2.146863in}}%
\pgfusepath{clip}%
\pgfsetrectcap%
\pgfsetroundjoin%
\pgfsetlinewidth{0.100375pt}%
\definecolor{currentstroke}{rgb}{0.827451,0.827451,0.827451}%
\pgfsetstrokecolor{currentstroke}%
\pgfsetdash{}{0pt}%
\pgfpathmoveto{\pgfqpoint{4.763372in}{1.080890in}}%
\pgfpathlineto{\pgfqpoint{4.763372in}{3.227753in}}%
\pgfusepath{stroke}%
\end{pgfscope}%
\begin{pgfscope}%
\pgfsetbuttcap%
\pgfsetroundjoin%
\definecolor{currentfill}{rgb}{0.000000,0.000000,0.000000}%
\pgfsetfillcolor{currentfill}%
\pgfsetlinewidth{0.501875pt}%
\definecolor{currentstroke}{rgb}{0.000000,0.000000,0.000000}%
\pgfsetstrokecolor{currentstroke}%
\pgfsetdash{}{0pt}%
\pgfsys@defobject{currentmarker}{\pgfqpoint{0.000000in}{0.000000in}}{\pgfqpoint{0.000000in}{0.020833in}}{%
\pgfpathmoveto{\pgfqpoint{0.000000in}{0.000000in}}%
\pgfpathlineto{\pgfqpoint{0.000000in}{0.020833in}}%
\pgfusepath{stroke,fill}%
}%
\begin{pgfscope}%
\pgfsys@transformshift{4.763372in}{1.080890in}%
\pgfsys@useobject{currentmarker}{}%
\end{pgfscope}%
\end{pgfscope}%
\begin{pgfscope}%
\pgfsetbuttcap%
\pgfsetroundjoin%
\definecolor{currentfill}{rgb}{0.000000,0.000000,0.000000}%
\pgfsetfillcolor{currentfill}%
\pgfsetlinewidth{0.501875pt}%
\definecolor{currentstroke}{rgb}{0.000000,0.000000,0.000000}%
\pgfsetstrokecolor{currentstroke}%
\pgfsetdash{}{0pt}%
\pgfsys@defobject{currentmarker}{\pgfqpoint{0.000000in}{-0.020833in}}{\pgfqpoint{0.000000in}{0.000000in}}{%
\pgfpathmoveto{\pgfqpoint{0.000000in}{0.000000in}}%
\pgfpathlineto{\pgfqpoint{0.000000in}{-0.020833in}}%
\pgfusepath{stroke,fill}%
}%
\begin{pgfscope}%
\pgfsys@transformshift{4.763372in}{3.227753in}%
\pgfsys@useobject{currentmarker}{}%
\end{pgfscope}%
\end{pgfscope}%
\begin{pgfscope}%
\pgfpathrectangle{\pgfqpoint{0.481681in}{1.080890in}}{\pgfqpoint{5.785672in}{2.146863in}}%
\pgfusepath{clip}%
\pgfsetrectcap%
\pgfsetroundjoin%
\pgfsetlinewidth{0.100375pt}%
\definecolor{currentstroke}{rgb}{0.827451,0.827451,0.827451}%
\pgfsetstrokecolor{currentstroke}%
\pgfsetdash{}{0pt}%
\pgfpathmoveto{\pgfqpoint{4.898815in}{1.080890in}}%
\pgfpathlineto{\pgfqpoint{4.898815in}{3.227753in}}%
\pgfusepath{stroke}%
\end{pgfscope}%
\begin{pgfscope}%
\pgfsetbuttcap%
\pgfsetroundjoin%
\definecolor{currentfill}{rgb}{0.000000,0.000000,0.000000}%
\pgfsetfillcolor{currentfill}%
\pgfsetlinewidth{0.501875pt}%
\definecolor{currentstroke}{rgb}{0.000000,0.000000,0.000000}%
\pgfsetstrokecolor{currentstroke}%
\pgfsetdash{}{0pt}%
\pgfsys@defobject{currentmarker}{\pgfqpoint{0.000000in}{0.000000in}}{\pgfqpoint{0.000000in}{0.020833in}}{%
\pgfpathmoveto{\pgfqpoint{0.000000in}{0.000000in}}%
\pgfpathlineto{\pgfqpoint{0.000000in}{0.020833in}}%
\pgfusepath{stroke,fill}%
}%
\begin{pgfscope}%
\pgfsys@transformshift{4.898815in}{1.080890in}%
\pgfsys@useobject{currentmarker}{}%
\end{pgfscope}%
\end{pgfscope}%
\begin{pgfscope}%
\pgfsetbuttcap%
\pgfsetroundjoin%
\definecolor{currentfill}{rgb}{0.000000,0.000000,0.000000}%
\pgfsetfillcolor{currentfill}%
\pgfsetlinewidth{0.501875pt}%
\definecolor{currentstroke}{rgb}{0.000000,0.000000,0.000000}%
\pgfsetstrokecolor{currentstroke}%
\pgfsetdash{}{0pt}%
\pgfsys@defobject{currentmarker}{\pgfqpoint{0.000000in}{-0.020833in}}{\pgfqpoint{0.000000in}{0.000000in}}{%
\pgfpathmoveto{\pgfqpoint{0.000000in}{0.000000in}}%
\pgfpathlineto{\pgfqpoint{0.000000in}{-0.020833in}}%
\pgfusepath{stroke,fill}%
}%
\begin{pgfscope}%
\pgfsys@transformshift{4.898815in}{3.227753in}%
\pgfsys@useobject{currentmarker}{}%
\end{pgfscope}%
\end{pgfscope}%
\begin{pgfscope}%
\pgfpathrectangle{\pgfqpoint{0.481681in}{1.080890in}}{\pgfqpoint{5.785672in}{2.146863in}}%
\pgfusepath{clip}%
\pgfsetrectcap%
\pgfsetroundjoin%
\pgfsetlinewidth{0.100375pt}%
\definecolor{currentstroke}{rgb}{0.827451,0.827451,0.827451}%
\pgfsetstrokecolor{currentstroke}%
\pgfsetdash{}{0pt}%
\pgfpathmoveto{\pgfqpoint{4.966536in}{1.080890in}}%
\pgfpathlineto{\pgfqpoint{4.966536in}{3.227753in}}%
\pgfusepath{stroke}%
\end{pgfscope}%
\begin{pgfscope}%
\pgfsetbuttcap%
\pgfsetroundjoin%
\definecolor{currentfill}{rgb}{0.000000,0.000000,0.000000}%
\pgfsetfillcolor{currentfill}%
\pgfsetlinewidth{0.501875pt}%
\definecolor{currentstroke}{rgb}{0.000000,0.000000,0.000000}%
\pgfsetstrokecolor{currentstroke}%
\pgfsetdash{}{0pt}%
\pgfsys@defobject{currentmarker}{\pgfqpoint{0.000000in}{0.000000in}}{\pgfqpoint{0.000000in}{0.020833in}}{%
\pgfpathmoveto{\pgfqpoint{0.000000in}{0.000000in}}%
\pgfpathlineto{\pgfqpoint{0.000000in}{0.020833in}}%
\pgfusepath{stroke,fill}%
}%
\begin{pgfscope}%
\pgfsys@transformshift{4.966536in}{1.080890in}%
\pgfsys@useobject{currentmarker}{}%
\end{pgfscope}%
\end{pgfscope}%
\begin{pgfscope}%
\pgfsetbuttcap%
\pgfsetroundjoin%
\definecolor{currentfill}{rgb}{0.000000,0.000000,0.000000}%
\pgfsetfillcolor{currentfill}%
\pgfsetlinewidth{0.501875pt}%
\definecolor{currentstroke}{rgb}{0.000000,0.000000,0.000000}%
\pgfsetstrokecolor{currentstroke}%
\pgfsetdash{}{0pt}%
\pgfsys@defobject{currentmarker}{\pgfqpoint{0.000000in}{-0.020833in}}{\pgfqpoint{0.000000in}{0.000000in}}{%
\pgfpathmoveto{\pgfqpoint{0.000000in}{0.000000in}}%
\pgfpathlineto{\pgfqpoint{0.000000in}{-0.020833in}}%
\pgfusepath{stroke,fill}%
}%
\begin{pgfscope}%
\pgfsys@transformshift{4.966536in}{3.227753in}%
\pgfsys@useobject{currentmarker}{}%
\end{pgfscope}%
\end{pgfscope}%
\begin{pgfscope}%
\pgfpathrectangle{\pgfqpoint{0.481681in}{1.080890in}}{\pgfqpoint{5.785672in}{2.146863in}}%
\pgfusepath{clip}%
\pgfsetrectcap%
\pgfsetroundjoin%
\pgfsetlinewidth{0.100375pt}%
\definecolor{currentstroke}{rgb}{0.827451,0.827451,0.827451}%
\pgfsetstrokecolor{currentstroke}%
\pgfsetdash{}{0pt}%
\pgfpathmoveto{\pgfqpoint{5.034258in}{1.080890in}}%
\pgfpathlineto{\pgfqpoint{5.034258in}{3.227753in}}%
\pgfusepath{stroke}%
\end{pgfscope}%
\begin{pgfscope}%
\pgfsetbuttcap%
\pgfsetroundjoin%
\definecolor{currentfill}{rgb}{0.000000,0.000000,0.000000}%
\pgfsetfillcolor{currentfill}%
\pgfsetlinewidth{0.501875pt}%
\definecolor{currentstroke}{rgb}{0.000000,0.000000,0.000000}%
\pgfsetstrokecolor{currentstroke}%
\pgfsetdash{}{0pt}%
\pgfsys@defobject{currentmarker}{\pgfqpoint{0.000000in}{0.000000in}}{\pgfqpoint{0.000000in}{0.020833in}}{%
\pgfpathmoveto{\pgfqpoint{0.000000in}{0.000000in}}%
\pgfpathlineto{\pgfqpoint{0.000000in}{0.020833in}}%
\pgfusepath{stroke,fill}%
}%
\begin{pgfscope}%
\pgfsys@transformshift{5.034258in}{1.080890in}%
\pgfsys@useobject{currentmarker}{}%
\end{pgfscope}%
\end{pgfscope}%
\begin{pgfscope}%
\pgfsetbuttcap%
\pgfsetroundjoin%
\definecolor{currentfill}{rgb}{0.000000,0.000000,0.000000}%
\pgfsetfillcolor{currentfill}%
\pgfsetlinewidth{0.501875pt}%
\definecolor{currentstroke}{rgb}{0.000000,0.000000,0.000000}%
\pgfsetstrokecolor{currentstroke}%
\pgfsetdash{}{0pt}%
\pgfsys@defobject{currentmarker}{\pgfqpoint{0.000000in}{-0.020833in}}{\pgfqpoint{0.000000in}{0.000000in}}{%
\pgfpathmoveto{\pgfqpoint{0.000000in}{0.000000in}}%
\pgfpathlineto{\pgfqpoint{0.000000in}{-0.020833in}}%
\pgfusepath{stroke,fill}%
}%
\begin{pgfscope}%
\pgfsys@transformshift{5.034258in}{3.227753in}%
\pgfsys@useobject{currentmarker}{}%
\end{pgfscope}%
\end{pgfscope}%
\begin{pgfscope}%
\pgfpathrectangle{\pgfqpoint{0.481681in}{1.080890in}}{\pgfqpoint{5.785672in}{2.146863in}}%
\pgfusepath{clip}%
\pgfsetrectcap%
\pgfsetroundjoin%
\pgfsetlinewidth{0.100375pt}%
\definecolor{currentstroke}{rgb}{0.827451,0.827451,0.827451}%
\pgfsetstrokecolor{currentstroke}%
\pgfsetdash{}{0pt}%
\pgfpathmoveto{\pgfqpoint{5.101979in}{1.080890in}}%
\pgfpathlineto{\pgfqpoint{5.101979in}{3.227753in}}%
\pgfusepath{stroke}%
\end{pgfscope}%
\begin{pgfscope}%
\pgfsetbuttcap%
\pgfsetroundjoin%
\definecolor{currentfill}{rgb}{0.000000,0.000000,0.000000}%
\pgfsetfillcolor{currentfill}%
\pgfsetlinewidth{0.501875pt}%
\definecolor{currentstroke}{rgb}{0.000000,0.000000,0.000000}%
\pgfsetstrokecolor{currentstroke}%
\pgfsetdash{}{0pt}%
\pgfsys@defobject{currentmarker}{\pgfqpoint{0.000000in}{0.000000in}}{\pgfqpoint{0.000000in}{0.020833in}}{%
\pgfpathmoveto{\pgfqpoint{0.000000in}{0.000000in}}%
\pgfpathlineto{\pgfqpoint{0.000000in}{0.020833in}}%
\pgfusepath{stroke,fill}%
}%
\begin{pgfscope}%
\pgfsys@transformshift{5.101979in}{1.080890in}%
\pgfsys@useobject{currentmarker}{}%
\end{pgfscope}%
\end{pgfscope}%
\begin{pgfscope}%
\pgfsetbuttcap%
\pgfsetroundjoin%
\definecolor{currentfill}{rgb}{0.000000,0.000000,0.000000}%
\pgfsetfillcolor{currentfill}%
\pgfsetlinewidth{0.501875pt}%
\definecolor{currentstroke}{rgb}{0.000000,0.000000,0.000000}%
\pgfsetstrokecolor{currentstroke}%
\pgfsetdash{}{0pt}%
\pgfsys@defobject{currentmarker}{\pgfqpoint{0.000000in}{-0.020833in}}{\pgfqpoint{0.000000in}{0.000000in}}{%
\pgfpathmoveto{\pgfqpoint{0.000000in}{0.000000in}}%
\pgfpathlineto{\pgfqpoint{0.000000in}{-0.020833in}}%
\pgfusepath{stroke,fill}%
}%
\begin{pgfscope}%
\pgfsys@transformshift{5.101979in}{3.227753in}%
\pgfsys@useobject{currentmarker}{}%
\end{pgfscope}%
\end{pgfscope}%
\begin{pgfscope}%
\pgfpathrectangle{\pgfqpoint{0.481681in}{1.080890in}}{\pgfqpoint{5.785672in}{2.146863in}}%
\pgfusepath{clip}%
\pgfsetrectcap%
\pgfsetroundjoin%
\pgfsetlinewidth{0.100375pt}%
\definecolor{currentstroke}{rgb}{0.827451,0.827451,0.827451}%
\pgfsetstrokecolor{currentstroke}%
\pgfsetdash{}{0pt}%
\pgfpathmoveto{\pgfqpoint{5.169701in}{1.080890in}}%
\pgfpathlineto{\pgfqpoint{5.169701in}{3.227753in}}%
\pgfusepath{stroke}%
\end{pgfscope}%
\begin{pgfscope}%
\pgfsetbuttcap%
\pgfsetroundjoin%
\definecolor{currentfill}{rgb}{0.000000,0.000000,0.000000}%
\pgfsetfillcolor{currentfill}%
\pgfsetlinewidth{0.501875pt}%
\definecolor{currentstroke}{rgb}{0.000000,0.000000,0.000000}%
\pgfsetstrokecolor{currentstroke}%
\pgfsetdash{}{0pt}%
\pgfsys@defobject{currentmarker}{\pgfqpoint{0.000000in}{0.000000in}}{\pgfqpoint{0.000000in}{0.020833in}}{%
\pgfpathmoveto{\pgfqpoint{0.000000in}{0.000000in}}%
\pgfpathlineto{\pgfqpoint{0.000000in}{0.020833in}}%
\pgfusepath{stroke,fill}%
}%
\begin{pgfscope}%
\pgfsys@transformshift{5.169701in}{1.080890in}%
\pgfsys@useobject{currentmarker}{}%
\end{pgfscope}%
\end{pgfscope}%
\begin{pgfscope}%
\pgfsetbuttcap%
\pgfsetroundjoin%
\definecolor{currentfill}{rgb}{0.000000,0.000000,0.000000}%
\pgfsetfillcolor{currentfill}%
\pgfsetlinewidth{0.501875pt}%
\definecolor{currentstroke}{rgb}{0.000000,0.000000,0.000000}%
\pgfsetstrokecolor{currentstroke}%
\pgfsetdash{}{0pt}%
\pgfsys@defobject{currentmarker}{\pgfqpoint{0.000000in}{-0.020833in}}{\pgfqpoint{0.000000in}{0.000000in}}{%
\pgfpathmoveto{\pgfqpoint{0.000000in}{0.000000in}}%
\pgfpathlineto{\pgfqpoint{0.000000in}{-0.020833in}}%
\pgfusepath{stroke,fill}%
}%
\begin{pgfscope}%
\pgfsys@transformshift{5.169701in}{3.227753in}%
\pgfsys@useobject{currentmarker}{}%
\end{pgfscope}%
\end{pgfscope}%
\begin{pgfscope}%
\pgfpathrectangle{\pgfqpoint{0.481681in}{1.080890in}}{\pgfqpoint{5.785672in}{2.146863in}}%
\pgfusepath{clip}%
\pgfsetrectcap%
\pgfsetroundjoin%
\pgfsetlinewidth{0.100375pt}%
\definecolor{currentstroke}{rgb}{0.827451,0.827451,0.827451}%
\pgfsetstrokecolor{currentstroke}%
\pgfsetdash{}{0pt}%
\pgfpathmoveto{\pgfqpoint{5.305144in}{1.080890in}}%
\pgfpathlineto{\pgfqpoint{5.305144in}{3.227753in}}%
\pgfusepath{stroke}%
\end{pgfscope}%
\begin{pgfscope}%
\pgfsetbuttcap%
\pgfsetroundjoin%
\definecolor{currentfill}{rgb}{0.000000,0.000000,0.000000}%
\pgfsetfillcolor{currentfill}%
\pgfsetlinewidth{0.501875pt}%
\definecolor{currentstroke}{rgb}{0.000000,0.000000,0.000000}%
\pgfsetstrokecolor{currentstroke}%
\pgfsetdash{}{0pt}%
\pgfsys@defobject{currentmarker}{\pgfqpoint{0.000000in}{0.000000in}}{\pgfqpoint{0.000000in}{0.020833in}}{%
\pgfpathmoveto{\pgfqpoint{0.000000in}{0.000000in}}%
\pgfpathlineto{\pgfqpoint{0.000000in}{0.020833in}}%
\pgfusepath{stroke,fill}%
}%
\begin{pgfscope}%
\pgfsys@transformshift{5.305144in}{1.080890in}%
\pgfsys@useobject{currentmarker}{}%
\end{pgfscope}%
\end{pgfscope}%
\begin{pgfscope}%
\pgfsetbuttcap%
\pgfsetroundjoin%
\definecolor{currentfill}{rgb}{0.000000,0.000000,0.000000}%
\pgfsetfillcolor{currentfill}%
\pgfsetlinewidth{0.501875pt}%
\definecolor{currentstroke}{rgb}{0.000000,0.000000,0.000000}%
\pgfsetstrokecolor{currentstroke}%
\pgfsetdash{}{0pt}%
\pgfsys@defobject{currentmarker}{\pgfqpoint{0.000000in}{-0.020833in}}{\pgfqpoint{0.000000in}{0.000000in}}{%
\pgfpathmoveto{\pgfqpoint{0.000000in}{0.000000in}}%
\pgfpathlineto{\pgfqpoint{0.000000in}{-0.020833in}}%
\pgfusepath{stroke,fill}%
}%
\begin{pgfscope}%
\pgfsys@transformshift{5.305144in}{3.227753in}%
\pgfsys@useobject{currentmarker}{}%
\end{pgfscope}%
\end{pgfscope}%
\begin{pgfscope}%
\pgfpathrectangle{\pgfqpoint{0.481681in}{1.080890in}}{\pgfqpoint{5.785672in}{2.146863in}}%
\pgfusepath{clip}%
\pgfsetrectcap%
\pgfsetroundjoin%
\pgfsetlinewidth{0.100375pt}%
\definecolor{currentstroke}{rgb}{0.827451,0.827451,0.827451}%
\pgfsetstrokecolor{currentstroke}%
\pgfsetdash{}{0pt}%
\pgfpathmoveto{\pgfqpoint{5.372865in}{1.080890in}}%
\pgfpathlineto{\pgfqpoint{5.372865in}{3.227753in}}%
\pgfusepath{stroke}%
\end{pgfscope}%
\begin{pgfscope}%
\pgfsetbuttcap%
\pgfsetroundjoin%
\definecolor{currentfill}{rgb}{0.000000,0.000000,0.000000}%
\pgfsetfillcolor{currentfill}%
\pgfsetlinewidth{0.501875pt}%
\definecolor{currentstroke}{rgb}{0.000000,0.000000,0.000000}%
\pgfsetstrokecolor{currentstroke}%
\pgfsetdash{}{0pt}%
\pgfsys@defobject{currentmarker}{\pgfqpoint{0.000000in}{0.000000in}}{\pgfqpoint{0.000000in}{0.020833in}}{%
\pgfpathmoveto{\pgfqpoint{0.000000in}{0.000000in}}%
\pgfpathlineto{\pgfqpoint{0.000000in}{0.020833in}}%
\pgfusepath{stroke,fill}%
}%
\begin{pgfscope}%
\pgfsys@transformshift{5.372865in}{1.080890in}%
\pgfsys@useobject{currentmarker}{}%
\end{pgfscope}%
\end{pgfscope}%
\begin{pgfscope}%
\pgfsetbuttcap%
\pgfsetroundjoin%
\definecolor{currentfill}{rgb}{0.000000,0.000000,0.000000}%
\pgfsetfillcolor{currentfill}%
\pgfsetlinewidth{0.501875pt}%
\definecolor{currentstroke}{rgb}{0.000000,0.000000,0.000000}%
\pgfsetstrokecolor{currentstroke}%
\pgfsetdash{}{0pt}%
\pgfsys@defobject{currentmarker}{\pgfqpoint{0.000000in}{-0.020833in}}{\pgfqpoint{0.000000in}{0.000000in}}{%
\pgfpathmoveto{\pgfqpoint{0.000000in}{0.000000in}}%
\pgfpathlineto{\pgfqpoint{0.000000in}{-0.020833in}}%
\pgfusepath{stroke,fill}%
}%
\begin{pgfscope}%
\pgfsys@transformshift{5.372865in}{3.227753in}%
\pgfsys@useobject{currentmarker}{}%
\end{pgfscope}%
\end{pgfscope}%
\begin{pgfscope}%
\pgfpathrectangle{\pgfqpoint{0.481681in}{1.080890in}}{\pgfqpoint{5.785672in}{2.146863in}}%
\pgfusepath{clip}%
\pgfsetrectcap%
\pgfsetroundjoin%
\pgfsetlinewidth{0.100375pt}%
\definecolor{currentstroke}{rgb}{0.827451,0.827451,0.827451}%
\pgfsetstrokecolor{currentstroke}%
\pgfsetdash{}{0pt}%
\pgfpathmoveto{\pgfqpoint{5.440587in}{1.080890in}}%
\pgfpathlineto{\pgfqpoint{5.440587in}{3.227753in}}%
\pgfusepath{stroke}%
\end{pgfscope}%
\begin{pgfscope}%
\pgfsetbuttcap%
\pgfsetroundjoin%
\definecolor{currentfill}{rgb}{0.000000,0.000000,0.000000}%
\pgfsetfillcolor{currentfill}%
\pgfsetlinewidth{0.501875pt}%
\definecolor{currentstroke}{rgb}{0.000000,0.000000,0.000000}%
\pgfsetstrokecolor{currentstroke}%
\pgfsetdash{}{0pt}%
\pgfsys@defobject{currentmarker}{\pgfqpoint{0.000000in}{0.000000in}}{\pgfqpoint{0.000000in}{0.020833in}}{%
\pgfpathmoveto{\pgfqpoint{0.000000in}{0.000000in}}%
\pgfpathlineto{\pgfqpoint{0.000000in}{0.020833in}}%
\pgfusepath{stroke,fill}%
}%
\begin{pgfscope}%
\pgfsys@transformshift{5.440587in}{1.080890in}%
\pgfsys@useobject{currentmarker}{}%
\end{pgfscope}%
\end{pgfscope}%
\begin{pgfscope}%
\pgfsetbuttcap%
\pgfsetroundjoin%
\definecolor{currentfill}{rgb}{0.000000,0.000000,0.000000}%
\pgfsetfillcolor{currentfill}%
\pgfsetlinewidth{0.501875pt}%
\definecolor{currentstroke}{rgb}{0.000000,0.000000,0.000000}%
\pgfsetstrokecolor{currentstroke}%
\pgfsetdash{}{0pt}%
\pgfsys@defobject{currentmarker}{\pgfqpoint{0.000000in}{-0.020833in}}{\pgfqpoint{0.000000in}{0.000000in}}{%
\pgfpathmoveto{\pgfqpoint{0.000000in}{0.000000in}}%
\pgfpathlineto{\pgfqpoint{0.000000in}{-0.020833in}}%
\pgfusepath{stroke,fill}%
}%
\begin{pgfscope}%
\pgfsys@transformshift{5.440587in}{3.227753in}%
\pgfsys@useobject{currentmarker}{}%
\end{pgfscope}%
\end{pgfscope}%
\begin{pgfscope}%
\pgfpathrectangle{\pgfqpoint{0.481681in}{1.080890in}}{\pgfqpoint{5.785672in}{2.146863in}}%
\pgfusepath{clip}%
\pgfsetrectcap%
\pgfsetroundjoin%
\pgfsetlinewidth{0.100375pt}%
\definecolor{currentstroke}{rgb}{0.827451,0.827451,0.827451}%
\pgfsetstrokecolor{currentstroke}%
\pgfsetdash{}{0pt}%
\pgfpathmoveto{\pgfqpoint{5.508308in}{1.080890in}}%
\pgfpathlineto{\pgfqpoint{5.508308in}{3.227753in}}%
\pgfusepath{stroke}%
\end{pgfscope}%
\begin{pgfscope}%
\pgfsetbuttcap%
\pgfsetroundjoin%
\definecolor{currentfill}{rgb}{0.000000,0.000000,0.000000}%
\pgfsetfillcolor{currentfill}%
\pgfsetlinewidth{0.501875pt}%
\definecolor{currentstroke}{rgb}{0.000000,0.000000,0.000000}%
\pgfsetstrokecolor{currentstroke}%
\pgfsetdash{}{0pt}%
\pgfsys@defobject{currentmarker}{\pgfqpoint{0.000000in}{0.000000in}}{\pgfqpoint{0.000000in}{0.020833in}}{%
\pgfpathmoveto{\pgfqpoint{0.000000in}{0.000000in}}%
\pgfpathlineto{\pgfqpoint{0.000000in}{0.020833in}}%
\pgfusepath{stroke,fill}%
}%
\begin{pgfscope}%
\pgfsys@transformshift{5.508308in}{1.080890in}%
\pgfsys@useobject{currentmarker}{}%
\end{pgfscope}%
\end{pgfscope}%
\begin{pgfscope}%
\pgfsetbuttcap%
\pgfsetroundjoin%
\definecolor{currentfill}{rgb}{0.000000,0.000000,0.000000}%
\pgfsetfillcolor{currentfill}%
\pgfsetlinewidth{0.501875pt}%
\definecolor{currentstroke}{rgb}{0.000000,0.000000,0.000000}%
\pgfsetstrokecolor{currentstroke}%
\pgfsetdash{}{0pt}%
\pgfsys@defobject{currentmarker}{\pgfqpoint{0.000000in}{-0.020833in}}{\pgfqpoint{0.000000in}{0.000000in}}{%
\pgfpathmoveto{\pgfqpoint{0.000000in}{0.000000in}}%
\pgfpathlineto{\pgfqpoint{0.000000in}{-0.020833in}}%
\pgfusepath{stroke,fill}%
}%
\begin{pgfscope}%
\pgfsys@transformshift{5.508308in}{3.227753in}%
\pgfsys@useobject{currentmarker}{}%
\end{pgfscope}%
\end{pgfscope}%
\begin{pgfscope}%
\pgfpathrectangle{\pgfqpoint{0.481681in}{1.080890in}}{\pgfqpoint{5.785672in}{2.146863in}}%
\pgfusepath{clip}%
\pgfsetrectcap%
\pgfsetroundjoin%
\pgfsetlinewidth{0.100375pt}%
\definecolor{currentstroke}{rgb}{0.827451,0.827451,0.827451}%
\pgfsetstrokecolor{currentstroke}%
\pgfsetdash{}{0pt}%
\pgfpathmoveto{\pgfqpoint{5.576030in}{1.080890in}}%
\pgfpathlineto{\pgfqpoint{5.576030in}{3.227753in}}%
\pgfusepath{stroke}%
\end{pgfscope}%
\begin{pgfscope}%
\pgfsetbuttcap%
\pgfsetroundjoin%
\definecolor{currentfill}{rgb}{0.000000,0.000000,0.000000}%
\pgfsetfillcolor{currentfill}%
\pgfsetlinewidth{0.501875pt}%
\definecolor{currentstroke}{rgb}{0.000000,0.000000,0.000000}%
\pgfsetstrokecolor{currentstroke}%
\pgfsetdash{}{0pt}%
\pgfsys@defobject{currentmarker}{\pgfqpoint{0.000000in}{0.000000in}}{\pgfqpoint{0.000000in}{0.020833in}}{%
\pgfpathmoveto{\pgfqpoint{0.000000in}{0.000000in}}%
\pgfpathlineto{\pgfqpoint{0.000000in}{0.020833in}}%
\pgfusepath{stroke,fill}%
}%
\begin{pgfscope}%
\pgfsys@transformshift{5.576030in}{1.080890in}%
\pgfsys@useobject{currentmarker}{}%
\end{pgfscope}%
\end{pgfscope}%
\begin{pgfscope}%
\pgfsetbuttcap%
\pgfsetroundjoin%
\definecolor{currentfill}{rgb}{0.000000,0.000000,0.000000}%
\pgfsetfillcolor{currentfill}%
\pgfsetlinewidth{0.501875pt}%
\definecolor{currentstroke}{rgb}{0.000000,0.000000,0.000000}%
\pgfsetstrokecolor{currentstroke}%
\pgfsetdash{}{0pt}%
\pgfsys@defobject{currentmarker}{\pgfqpoint{0.000000in}{-0.020833in}}{\pgfqpoint{0.000000in}{0.000000in}}{%
\pgfpathmoveto{\pgfqpoint{0.000000in}{0.000000in}}%
\pgfpathlineto{\pgfqpoint{0.000000in}{-0.020833in}}%
\pgfusepath{stroke,fill}%
}%
\begin{pgfscope}%
\pgfsys@transformshift{5.576030in}{3.227753in}%
\pgfsys@useobject{currentmarker}{}%
\end{pgfscope}%
\end{pgfscope}%
\begin{pgfscope}%
\pgfpathrectangle{\pgfqpoint{0.481681in}{1.080890in}}{\pgfqpoint{5.785672in}{2.146863in}}%
\pgfusepath{clip}%
\pgfsetrectcap%
\pgfsetroundjoin%
\pgfsetlinewidth{0.100375pt}%
\definecolor{currentstroke}{rgb}{0.827451,0.827451,0.827451}%
\pgfsetstrokecolor{currentstroke}%
\pgfsetdash{}{0pt}%
\pgfpathmoveto{\pgfqpoint{5.711473in}{1.080890in}}%
\pgfpathlineto{\pgfqpoint{5.711473in}{3.227753in}}%
\pgfusepath{stroke}%
\end{pgfscope}%
\begin{pgfscope}%
\pgfsetbuttcap%
\pgfsetroundjoin%
\definecolor{currentfill}{rgb}{0.000000,0.000000,0.000000}%
\pgfsetfillcolor{currentfill}%
\pgfsetlinewidth{0.501875pt}%
\definecolor{currentstroke}{rgb}{0.000000,0.000000,0.000000}%
\pgfsetstrokecolor{currentstroke}%
\pgfsetdash{}{0pt}%
\pgfsys@defobject{currentmarker}{\pgfqpoint{0.000000in}{0.000000in}}{\pgfqpoint{0.000000in}{0.020833in}}{%
\pgfpathmoveto{\pgfqpoint{0.000000in}{0.000000in}}%
\pgfpathlineto{\pgfqpoint{0.000000in}{0.020833in}}%
\pgfusepath{stroke,fill}%
}%
\begin{pgfscope}%
\pgfsys@transformshift{5.711473in}{1.080890in}%
\pgfsys@useobject{currentmarker}{}%
\end{pgfscope}%
\end{pgfscope}%
\begin{pgfscope}%
\pgfsetbuttcap%
\pgfsetroundjoin%
\definecolor{currentfill}{rgb}{0.000000,0.000000,0.000000}%
\pgfsetfillcolor{currentfill}%
\pgfsetlinewidth{0.501875pt}%
\definecolor{currentstroke}{rgb}{0.000000,0.000000,0.000000}%
\pgfsetstrokecolor{currentstroke}%
\pgfsetdash{}{0pt}%
\pgfsys@defobject{currentmarker}{\pgfqpoint{0.000000in}{-0.020833in}}{\pgfqpoint{0.000000in}{0.000000in}}{%
\pgfpathmoveto{\pgfqpoint{0.000000in}{0.000000in}}%
\pgfpathlineto{\pgfqpoint{0.000000in}{-0.020833in}}%
\pgfusepath{stroke,fill}%
}%
\begin{pgfscope}%
\pgfsys@transformshift{5.711473in}{3.227753in}%
\pgfsys@useobject{currentmarker}{}%
\end{pgfscope}%
\end{pgfscope}%
\begin{pgfscope}%
\pgfpathrectangle{\pgfqpoint{0.481681in}{1.080890in}}{\pgfqpoint{5.785672in}{2.146863in}}%
\pgfusepath{clip}%
\pgfsetrectcap%
\pgfsetroundjoin%
\pgfsetlinewidth{0.100375pt}%
\definecolor{currentstroke}{rgb}{0.827451,0.827451,0.827451}%
\pgfsetstrokecolor{currentstroke}%
\pgfsetdash{}{0pt}%
\pgfpathmoveto{\pgfqpoint{5.779194in}{1.080890in}}%
\pgfpathlineto{\pgfqpoint{5.779194in}{3.227753in}}%
\pgfusepath{stroke}%
\end{pgfscope}%
\begin{pgfscope}%
\pgfsetbuttcap%
\pgfsetroundjoin%
\definecolor{currentfill}{rgb}{0.000000,0.000000,0.000000}%
\pgfsetfillcolor{currentfill}%
\pgfsetlinewidth{0.501875pt}%
\definecolor{currentstroke}{rgb}{0.000000,0.000000,0.000000}%
\pgfsetstrokecolor{currentstroke}%
\pgfsetdash{}{0pt}%
\pgfsys@defobject{currentmarker}{\pgfqpoint{0.000000in}{0.000000in}}{\pgfqpoint{0.000000in}{0.020833in}}{%
\pgfpathmoveto{\pgfqpoint{0.000000in}{0.000000in}}%
\pgfpathlineto{\pgfqpoint{0.000000in}{0.020833in}}%
\pgfusepath{stroke,fill}%
}%
\begin{pgfscope}%
\pgfsys@transformshift{5.779194in}{1.080890in}%
\pgfsys@useobject{currentmarker}{}%
\end{pgfscope}%
\end{pgfscope}%
\begin{pgfscope}%
\pgfsetbuttcap%
\pgfsetroundjoin%
\definecolor{currentfill}{rgb}{0.000000,0.000000,0.000000}%
\pgfsetfillcolor{currentfill}%
\pgfsetlinewidth{0.501875pt}%
\definecolor{currentstroke}{rgb}{0.000000,0.000000,0.000000}%
\pgfsetstrokecolor{currentstroke}%
\pgfsetdash{}{0pt}%
\pgfsys@defobject{currentmarker}{\pgfqpoint{0.000000in}{-0.020833in}}{\pgfqpoint{0.000000in}{0.000000in}}{%
\pgfpathmoveto{\pgfqpoint{0.000000in}{0.000000in}}%
\pgfpathlineto{\pgfqpoint{0.000000in}{-0.020833in}}%
\pgfusepath{stroke,fill}%
}%
\begin{pgfscope}%
\pgfsys@transformshift{5.779194in}{3.227753in}%
\pgfsys@useobject{currentmarker}{}%
\end{pgfscope}%
\end{pgfscope}%
\begin{pgfscope}%
\pgfpathrectangle{\pgfqpoint{0.481681in}{1.080890in}}{\pgfqpoint{5.785672in}{2.146863in}}%
\pgfusepath{clip}%
\pgfsetrectcap%
\pgfsetroundjoin%
\pgfsetlinewidth{0.100375pt}%
\definecolor{currentstroke}{rgb}{0.827451,0.827451,0.827451}%
\pgfsetstrokecolor{currentstroke}%
\pgfsetdash{}{0pt}%
\pgfpathmoveto{\pgfqpoint{5.846916in}{1.080890in}}%
\pgfpathlineto{\pgfqpoint{5.846916in}{3.227753in}}%
\pgfusepath{stroke}%
\end{pgfscope}%
\begin{pgfscope}%
\pgfsetbuttcap%
\pgfsetroundjoin%
\definecolor{currentfill}{rgb}{0.000000,0.000000,0.000000}%
\pgfsetfillcolor{currentfill}%
\pgfsetlinewidth{0.501875pt}%
\definecolor{currentstroke}{rgb}{0.000000,0.000000,0.000000}%
\pgfsetstrokecolor{currentstroke}%
\pgfsetdash{}{0pt}%
\pgfsys@defobject{currentmarker}{\pgfqpoint{0.000000in}{0.000000in}}{\pgfqpoint{0.000000in}{0.020833in}}{%
\pgfpathmoveto{\pgfqpoint{0.000000in}{0.000000in}}%
\pgfpathlineto{\pgfqpoint{0.000000in}{0.020833in}}%
\pgfusepath{stroke,fill}%
}%
\begin{pgfscope}%
\pgfsys@transformshift{5.846916in}{1.080890in}%
\pgfsys@useobject{currentmarker}{}%
\end{pgfscope}%
\end{pgfscope}%
\begin{pgfscope}%
\pgfsetbuttcap%
\pgfsetroundjoin%
\definecolor{currentfill}{rgb}{0.000000,0.000000,0.000000}%
\pgfsetfillcolor{currentfill}%
\pgfsetlinewidth{0.501875pt}%
\definecolor{currentstroke}{rgb}{0.000000,0.000000,0.000000}%
\pgfsetstrokecolor{currentstroke}%
\pgfsetdash{}{0pt}%
\pgfsys@defobject{currentmarker}{\pgfqpoint{0.000000in}{-0.020833in}}{\pgfqpoint{0.000000in}{0.000000in}}{%
\pgfpathmoveto{\pgfqpoint{0.000000in}{0.000000in}}%
\pgfpathlineto{\pgfqpoint{0.000000in}{-0.020833in}}%
\pgfusepath{stroke,fill}%
}%
\begin{pgfscope}%
\pgfsys@transformshift{5.846916in}{3.227753in}%
\pgfsys@useobject{currentmarker}{}%
\end{pgfscope}%
\end{pgfscope}%
\begin{pgfscope}%
\pgfpathrectangle{\pgfqpoint{0.481681in}{1.080890in}}{\pgfqpoint{5.785672in}{2.146863in}}%
\pgfusepath{clip}%
\pgfsetrectcap%
\pgfsetroundjoin%
\pgfsetlinewidth{0.100375pt}%
\definecolor{currentstroke}{rgb}{0.827451,0.827451,0.827451}%
\pgfsetstrokecolor{currentstroke}%
\pgfsetdash{}{0pt}%
\pgfpathmoveto{\pgfqpoint{5.914637in}{1.080890in}}%
\pgfpathlineto{\pgfqpoint{5.914637in}{3.227753in}}%
\pgfusepath{stroke}%
\end{pgfscope}%
\begin{pgfscope}%
\pgfsetbuttcap%
\pgfsetroundjoin%
\definecolor{currentfill}{rgb}{0.000000,0.000000,0.000000}%
\pgfsetfillcolor{currentfill}%
\pgfsetlinewidth{0.501875pt}%
\definecolor{currentstroke}{rgb}{0.000000,0.000000,0.000000}%
\pgfsetstrokecolor{currentstroke}%
\pgfsetdash{}{0pt}%
\pgfsys@defobject{currentmarker}{\pgfqpoint{0.000000in}{0.000000in}}{\pgfqpoint{0.000000in}{0.020833in}}{%
\pgfpathmoveto{\pgfqpoint{0.000000in}{0.000000in}}%
\pgfpathlineto{\pgfqpoint{0.000000in}{0.020833in}}%
\pgfusepath{stroke,fill}%
}%
\begin{pgfscope}%
\pgfsys@transformshift{5.914637in}{1.080890in}%
\pgfsys@useobject{currentmarker}{}%
\end{pgfscope}%
\end{pgfscope}%
\begin{pgfscope}%
\pgfsetbuttcap%
\pgfsetroundjoin%
\definecolor{currentfill}{rgb}{0.000000,0.000000,0.000000}%
\pgfsetfillcolor{currentfill}%
\pgfsetlinewidth{0.501875pt}%
\definecolor{currentstroke}{rgb}{0.000000,0.000000,0.000000}%
\pgfsetstrokecolor{currentstroke}%
\pgfsetdash{}{0pt}%
\pgfsys@defobject{currentmarker}{\pgfqpoint{0.000000in}{-0.020833in}}{\pgfqpoint{0.000000in}{0.000000in}}{%
\pgfpathmoveto{\pgfqpoint{0.000000in}{0.000000in}}%
\pgfpathlineto{\pgfqpoint{0.000000in}{-0.020833in}}%
\pgfusepath{stroke,fill}%
}%
\begin{pgfscope}%
\pgfsys@transformshift{5.914637in}{3.227753in}%
\pgfsys@useobject{currentmarker}{}%
\end{pgfscope}%
\end{pgfscope}%
\begin{pgfscope}%
\pgfpathrectangle{\pgfqpoint{0.481681in}{1.080890in}}{\pgfqpoint{5.785672in}{2.146863in}}%
\pgfusepath{clip}%
\pgfsetrectcap%
\pgfsetroundjoin%
\pgfsetlinewidth{0.100375pt}%
\definecolor{currentstroke}{rgb}{0.827451,0.827451,0.827451}%
\pgfsetstrokecolor{currentstroke}%
\pgfsetdash{}{0pt}%
\pgfpathmoveto{\pgfqpoint{5.982358in}{1.080890in}}%
\pgfpathlineto{\pgfqpoint{5.982358in}{3.227753in}}%
\pgfusepath{stroke}%
\end{pgfscope}%
\begin{pgfscope}%
\pgfsetbuttcap%
\pgfsetroundjoin%
\definecolor{currentfill}{rgb}{0.000000,0.000000,0.000000}%
\pgfsetfillcolor{currentfill}%
\pgfsetlinewidth{0.501875pt}%
\definecolor{currentstroke}{rgb}{0.000000,0.000000,0.000000}%
\pgfsetstrokecolor{currentstroke}%
\pgfsetdash{}{0pt}%
\pgfsys@defobject{currentmarker}{\pgfqpoint{0.000000in}{0.000000in}}{\pgfqpoint{0.000000in}{0.020833in}}{%
\pgfpathmoveto{\pgfqpoint{0.000000in}{0.000000in}}%
\pgfpathlineto{\pgfqpoint{0.000000in}{0.020833in}}%
\pgfusepath{stroke,fill}%
}%
\begin{pgfscope}%
\pgfsys@transformshift{5.982358in}{1.080890in}%
\pgfsys@useobject{currentmarker}{}%
\end{pgfscope}%
\end{pgfscope}%
\begin{pgfscope}%
\pgfsetbuttcap%
\pgfsetroundjoin%
\definecolor{currentfill}{rgb}{0.000000,0.000000,0.000000}%
\pgfsetfillcolor{currentfill}%
\pgfsetlinewidth{0.501875pt}%
\definecolor{currentstroke}{rgb}{0.000000,0.000000,0.000000}%
\pgfsetstrokecolor{currentstroke}%
\pgfsetdash{}{0pt}%
\pgfsys@defobject{currentmarker}{\pgfqpoint{0.000000in}{-0.020833in}}{\pgfqpoint{0.000000in}{0.000000in}}{%
\pgfpathmoveto{\pgfqpoint{0.000000in}{0.000000in}}%
\pgfpathlineto{\pgfqpoint{0.000000in}{-0.020833in}}%
\pgfusepath{stroke,fill}%
}%
\begin{pgfscope}%
\pgfsys@transformshift{5.982358in}{3.227753in}%
\pgfsys@useobject{currentmarker}{}%
\end{pgfscope}%
\end{pgfscope}%
\begin{pgfscope}%
\pgfpathrectangle{\pgfqpoint{0.481681in}{1.080890in}}{\pgfqpoint{5.785672in}{2.146863in}}%
\pgfusepath{clip}%
\pgfsetrectcap%
\pgfsetroundjoin%
\pgfsetlinewidth{0.100375pt}%
\definecolor{currentstroke}{rgb}{0.827451,0.827451,0.827451}%
\pgfsetstrokecolor{currentstroke}%
\pgfsetdash{}{0pt}%
\pgfpathmoveto{\pgfqpoint{6.117801in}{1.080890in}}%
\pgfpathlineto{\pgfqpoint{6.117801in}{3.227753in}}%
\pgfusepath{stroke}%
\end{pgfscope}%
\begin{pgfscope}%
\pgfsetbuttcap%
\pgfsetroundjoin%
\definecolor{currentfill}{rgb}{0.000000,0.000000,0.000000}%
\pgfsetfillcolor{currentfill}%
\pgfsetlinewidth{0.501875pt}%
\definecolor{currentstroke}{rgb}{0.000000,0.000000,0.000000}%
\pgfsetstrokecolor{currentstroke}%
\pgfsetdash{}{0pt}%
\pgfsys@defobject{currentmarker}{\pgfqpoint{0.000000in}{0.000000in}}{\pgfqpoint{0.000000in}{0.020833in}}{%
\pgfpathmoveto{\pgfqpoint{0.000000in}{0.000000in}}%
\pgfpathlineto{\pgfqpoint{0.000000in}{0.020833in}}%
\pgfusepath{stroke,fill}%
}%
\begin{pgfscope}%
\pgfsys@transformshift{6.117801in}{1.080890in}%
\pgfsys@useobject{currentmarker}{}%
\end{pgfscope}%
\end{pgfscope}%
\begin{pgfscope}%
\pgfsetbuttcap%
\pgfsetroundjoin%
\definecolor{currentfill}{rgb}{0.000000,0.000000,0.000000}%
\pgfsetfillcolor{currentfill}%
\pgfsetlinewidth{0.501875pt}%
\definecolor{currentstroke}{rgb}{0.000000,0.000000,0.000000}%
\pgfsetstrokecolor{currentstroke}%
\pgfsetdash{}{0pt}%
\pgfsys@defobject{currentmarker}{\pgfqpoint{0.000000in}{-0.020833in}}{\pgfqpoint{0.000000in}{0.000000in}}{%
\pgfpathmoveto{\pgfqpoint{0.000000in}{0.000000in}}%
\pgfpathlineto{\pgfqpoint{0.000000in}{-0.020833in}}%
\pgfusepath{stroke,fill}%
}%
\begin{pgfscope}%
\pgfsys@transformshift{6.117801in}{3.227753in}%
\pgfsys@useobject{currentmarker}{}%
\end{pgfscope}%
\end{pgfscope}%
\begin{pgfscope}%
\pgfpathrectangle{\pgfqpoint{0.481681in}{1.080890in}}{\pgfqpoint{5.785672in}{2.146863in}}%
\pgfusepath{clip}%
\pgfsetrectcap%
\pgfsetroundjoin%
\pgfsetlinewidth{0.100375pt}%
\definecolor{currentstroke}{rgb}{0.827451,0.827451,0.827451}%
\pgfsetstrokecolor{currentstroke}%
\pgfsetdash{}{0pt}%
\pgfpathmoveto{\pgfqpoint{6.185523in}{1.080890in}}%
\pgfpathlineto{\pgfqpoint{6.185523in}{3.227753in}}%
\pgfusepath{stroke}%
\end{pgfscope}%
\begin{pgfscope}%
\pgfsetbuttcap%
\pgfsetroundjoin%
\definecolor{currentfill}{rgb}{0.000000,0.000000,0.000000}%
\pgfsetfillcolor{currentfill}%
\pgfsetlinewidth{0.501875pt}%
\definecolor{currentstroke}{rgb}{0.000000,0.000000,0.000000}%
\pgfsetstrokecolor{currentstroke}%
\pgfsetdash{}{0pt}%
\pgfsys@defobject{currentmarker}{\pgfqpoint{0.000000in}{0.000000in}}{\pgfqpoint{0.000000in}{0.020833in}}{%
\pgfpathmoveto{\pgfqpoint{0.000000in}{0.000000in}}%
\pgfpathlineto{\pgfqpoint{0.000000in}{0.020833in}}%
\pgfusepath{stroke,fill}%
}%
\begin{pgfscope}%
\pgfsys@transformshift{6.185523in}{1.080890in}%
\pgfsys@useobject{currentmarker}{}%
\end{pgfscope}%
\end{pgfscope}%
\begin{pgfscope}%
\pgfsetbuttcap%
\pgfsetroundjoin%
\definecolor{currentfill}{rgb}{0.000000,0.000000,0.000000}%
\pgfsetfillcolor{currentfill}%
\pgfsetlinewidth{0.501875pt}%
\definecolor{currentstroke}{rgb}{0.000000,0.000000,0.000000}%
\pgfsetstrokecolor{currentstroke}%
\pgfsetdash{}{0pt}%
\pgfsys@defobject{currentmarker}{\pgfqpoint{0.000000in}{-0.020833in}}{\pgfqpoint{0.000000in}{0.000000in}}{%
\pgfpathmoveto{\pgfqpoint{0.000000in}{0.000000in}}%
\pgfpathlineto{\pgfqpoint{0.000000in}{-0.020833in}}%
\pgfusepath{stroke,fill}%
}%
\begin{pgfscope}%
\pgfsys@transformshift{6.185523in}{3.227753in}%
\pgfsys@useobject{currentmarker}{}%
\end{pgfscope}%
\end{pgfscope}%
\begin{pgfscope}%
\pgfpathrectangle{\pgfqpoint{0.481681in}{1.080890in}}{\pgfqpoint{5.785672in}{2.146863in}}%
\pgfusepath{clip}%
\pgfsetrectcap%
\pgfsetroundjoin%
\pgfsetlinewidth{0.100375pt}%
\definecolor{currentstroke}{rgb}{0.827451,0.827451,0.827451}%
\pgfsetstrokecolor{currentstroke}%
\pgfsetdash{}{0pt}%
\pgfpathmoveto{\pgfqpoint{6.253244in}{1.080890in}}%
\pgfpathlineto{\pgfqpoint{6.253244in}{3.227753in}}%
\pgfusepath{stroke}%
\end{pgfscope}%
\begin{pgfscope}%
\pgfsetbuttcap%
\pgfsetroundjoin%
\definecolor{currentfill}{rgb}{0.000000,0.000000,0.000000}%
\pgfsetfillcolor{currentfill}%
\pgfsetlinewidth{0.501875pt}%
\definecolor{currentstroke}{rgb}{0.000000,0.000000,0.000000}%
\pgfsetstrokecolor{currentstroke}%
\pgfsetdash{}{0pt}%
\pgfsys@defobject{currentmarker}{\pgfqpoint{0.000000in}{0.000000in}}{\pgfqpoint{0.000000in}{0.020833in}}{%
\pgfpathmoveto{\pgfqpoint{0.000000in}{0.000000in}}%
\pgfpathlineto{\pgfqpoint{0.000000in}{0.020833in}}%
\pgfusepath{stroke,fill}%
}%
\begin{pgfscope}%
\pgfsys@transformshift{6.253244in}{1.080890in}%
\pgfsys@useobject{currentmarker}{}%
\end{pgfscope}%
\end{pgfscope}%
\begin{pgfscope}%
\pgfsetbuttcap%
\pgfsetroundjoin%
\definecolor{currentfill}{rgb}{0.000000,0.000000,0.000000}%
\pgfsetfillcolor{currentfill}%
\pgfsetlinewidth{0.501875pt}%
\definecolor{currentstroke}{rgb}{0.000000,0.000000,0.000000}%
\pgfsetstrokecolor{currentstroke}%
\pgfsetdash{}{0pt}%
\pgfsys@defobject{currentmarker}{\pgfqpoint{0.000000in}{-0.020833in}}{\pgfqpoint{0.000000in}{0.000000in}}{%
\pgfpathmoveto{\pgfqpoint{0.000000in}{0.000000in}}%
\pgfpathlineto{\pgfqpoint{0.000000in}{-0.020833in}}%
\pgfusepath{stroke,fill}%
}%
\begin{pgfscope}%
\pgfsys@transformshift{6.253244in}{3.227753in}%
\pgfsys@useobject{currentmarker}{}%
\end{pgfscope}%
\end{pgfscope}%
\begin{pgfscope}%
\pgfsetbuttcap%
\pgfsetroundjoin%
\definecolor{currentfill}{rgb}{0.000000,0.000000,0.000000}%
\pgfsetfillcolor{currentfill}%
\pgfsetlinewidth{0.501875pt}%
\definecolor{currentstroke}{rgb}{0.000000,0.000000,0.000000}%
\pgfsetstrokecolor{currentstroke}%
\pgfsetdash{}{0pt}%
\pgfsys@defobject{currentmarker}{\pgfqpoint{0.000000in}{0.000000in}}{\pgfqpoint{0.041667in}{0.000000in}}{%
\pgfpathmoveto{\pgfqpoint{0.000000in}{0.000000in}}%
\pgfpathlineto{\pgfqpoint{0.041667in}{0.000000in}}%
\pgfusepath{stroke,fill}%
}%
\begin{pgfscope}%
\pgfsys@transformshift{0.481681in}{1.178475in}%
\pgfsys@useobject{currentmarker}{}%
\end{pgfscope}%
\end{pgfscope}%
\begin{pgfscope}%
\pgfsetbuttcap%
\pgfsetroundjoin%
\definecolor{currentfill}{rgb}{0.000000,0.000000,0.000000}%
\pgfsetfillcolor{currentfill}%
\pgfsetlinewidth{0.501875pt}%
\definecolor{currentstroke}{rgb}{0.000000,0.000000,0.000000}%
\pgfsetstrokecolor{currentstroke}%
\pgfsetdash{}{0pt}%
\pgfsys@defobject{currentmarker}{\pgfqpoint{-0.041667in}{0.000000in}}{\pgfqpoint{-0.000000in}{0.000000in}}{%
\pgfpathmoveto{\pgfqpoint{-0.000000in}{0.000000in}}%
\pgfpathlineto{\pgfqpoint{-0.041667in}{0.000000in}}%
\pgfusepath{stroke,fill}%
}%
\begin{pgfscope}%
\pgfsys@transformshift{6.267353in}{1.178475in}%
\pgfsys@useobject{currentmarker}{}%
\end{pgfscope}%
\end{pgfscope}%
\begin{pgfscope}%
\definecolor{textcolor}{rgb}{0.000000,0.000000,0.000000}%
\pgfsetstrokecolor{textcolor}%
\pgfsetfillcolor{textcolor}%
\pgftext[x=0.291028in, y=1.144739in, left, base]{\color{textcolor}\rmfamily\fontsize{7.000000}{8.400000}\selectfont 0.0}%
\end{pgfscope}%
\begin{pgfscope}%
\pgfsetbuttcap%
\pgfsetroundjoin%
\definecolor{currentfill}{rgb}{0.000000,0.000000,0.000000}%
\pgfsetfillcolor{currentfill}%
\pgfsetlinewidth{0.501875pt}%
\definecolor{currentstroke}{rgb}{0.000000,0.000000,0.000000}%
\pgfsetstrokecolor{currentstroke}%
\pgfsetdash{}{0pt}%
\pgfsys@defobject{currentmarker}{\pgfqpoint{0.000000in}{0.000000in}}{\pgfqpoint{0.041667in}{0.000000in}}{%
\pgfpathmoveto{\pgfqpoint{0.000000in}{0.000000in}}%
\pgfpathlineto{\pgfqpoint{0.041667in}{0.000000in}}%
\pgfusepath{stroke,fill}%
}%
\begin{pgfscope}%
\pgfsys@transformshift{0.481681in}{1.543302in}%
\pgfsys@useobject{currentmarker}{}%
\end{pgfscope}%
\end{pgfscope}%
\begin{pgfscope}%
\pgfsetbuttcap%
\pgfsetroundjoin%
\definecolor{currentfill}{rgb}{0.000000,0.000000,0.000000}%
\pgfsetfillcolor{currentfill}%
\pgfsetlinewidth{0.501875pt}%
\definecolor{currentstroke}{rgb}{0.000000,0.000000,0.000000}%
\pgfsetstrokecolor{currentstroke}%
\pgfsetdash{}{0pt}%
\pgfsys@defobject{currentmarker}{\pgfqpoint{-0.041667in}{0.000000in}}{\pgfqpoint{-0.000000in}{0.000000in}}{%
\pgfpathmoveto{\pgfqpoint{-0.000000in}{0.000000in}}%
\pgfpathlineto{\pgfqpoint{-0.041667in}{0.000000in}}%
\pgfusepath{stroke,fill}%
}%
\begin{pgfscope}%
\pgfsys@transformshift{6.267353in}{1.543302in}%
\pgfsys@useobject{currentmarker}{}%
\end{pgfscope}%
\end{pgfscope}%
\begin{pgfscope}%
\definecolor{textcolor}{rgb}{0.000000,0.000000,0.000000}%
\pgfsetstrokecolor{textcolor}%
\pgfsetfillcolor{textcolor}%
\pgftext[x=0.291028in, y=1.509566in, left, base]{\color{textcolor}\rmfamily\fontsize{7.000000}{8.400000}\selectfont 0.2}%
\end{pgfscope}%
\begin{pgfscope}%
\pgfsetbuttcap%
\pgfsetroundjoin%
\definecolor{currentfill}{rgb}{0.000000,0.000000,0.000000}%
\pgfsetfillcolor{currentfill}%
\pgfsetlinewidth{0.501875pt}%
\definecolor{currentstroke}{rgb}{0.000000,0.000000,0.000000}%
\pgfsetstrokecolor{currentstroke}%
\pgfsetdash{}{0pt}%
\pgfsys@defobject{currentmarker}{\pgfqpoint{0.000000in}{0.000000in}}{\pgfqpoint{0.041667in}{0.000000in}}{%
\pgfpathmoveto{\pgfqpoint{0.000000in}{0.000000in}}%
\pgfpathlineto{\pgfqpoint{0.041667in}{0.000000in}}%
\pgfusepath{stroke,fill}%
}%
\begin{pgfscope}%
\pgfsys@transformshift{0.481681in}{1.908129in}%
\pgfsys@useobject{currentmarker}{}%
\end{pgfscope}%
\end{pgfscope}%
\begin{pgfscope}%
\pgfsetbuttcap%
\pgfsetroundjoin%
\definecolor{currentfill}{rgb}{0.000000,0.000000,0.000000}%
\pgfsetfillcolor{currentfill}%
\pgfsetlinewidth{0.501875pt}%
\definecolor{currentstroke}{rgb}{0.000000,0.000000,0.000000}%
\pgfsetstrokecolor{currentstroke}%
\pgfsetdash{}{0pt}%
\pgfsys@defobject{currentmarker}{\pgfqpoint{-0.041667in}{0.000000in}}{\pgfqpoint{-0.000000in}{0.000000in}}{%
\pgfpathmoveto{\pgfqpoint{-0.000000in}{0.000000in}}%
\pgfpathlineto{\pgfqpoint{-0.041667in}{0.000000in}}%
\pgfusepath{stroke,fill}%
}%
\begin{pgfscope}%
\pgfsys@transformshift{6.267353in}{1.908129in}%
\pgfsys@useobject{currentmarker}{}%
\end{pgfscope}%
\end{pgfscope}%
\begin{pgfscope}%
\definecolor{textcolor}{rgb}{0.000000,0.000000,0.000000}%
\pgfsetstrokecolor{textcolor}%
\pgfsetfillcolor{textcolor}%
\pgftext[x=0.291028in, y=1.874393in, left, base]{\color{textcolor}\rmfamily\fontsize{7.000000}{8.400000}\selectfont 0.4}%
\end{pgfscope}%
\begin{pgfscope}%
\pgfsetbuttcap%
\pgfsetroundjoin%
\definecolor{currentfill}{rgb}{0.000000,0.000000,0.000000}%
\pgfsetfillcolor{currentfill}%
\pgfsetlinewidth{0.501875pt}%
\definecolor{currentstroke}{rgb}{0.000000,0.000000,0.000000}%
\pgfsetstrokecolor{currentstroke}%
\pgfsetdash{}{0pt}%
\pgfsys@defobject{currentmarker}{\pgfqpoint{0.000000in}{0.000000in}}{\pgfqpoint{0.041667in}{0.000000in}}{%
\pgfpathmoveto{\pgfqpoint{0.000000in}{0.000000in}}%
\pgfpathlineto{\pgfqpoint{0.041667in}{0.000000in}}%
\pgfusepath{stroke,fill}%
}%
\begin{pgfscope}%
\pgfsys@transformshift{0.481681in}{2.272956in}%
\pgfsys@useobject{currentmarker}{}%
\end{pgfscope}%
\end{pgfscope}%
\begin{pgfscope}%
\pgfsetbuttcap%
\pgfsetroundjoin%
\definecolor{currentfill}{rgb}{0.000000,0.000000,0.000000}%
\pgfsetfillcolor{currentfill}%
\pgfsetlinewidth{0.501875pt}%
\definecolor{currentstroke}{rgb}{0.000000,0.000000,0.000000}%
\pgfsetstrokecolor{currentstroke}%
\pgfsetdash{}{0pt}%
\pgfsys@defobject{currentmarker}{\pgfqpoint{-0.041667in}{0.000000in}}{\pgfqpoint{-0.000000in}{0.000000in}}{%
\pgfpathmoveto{\pgfqpoint{-0.000000in}{0.000000in}}%
\pgfpathlineto{\pgfqpoint{-0.041667in}{0.000000in}}%
\pgfusepath{stroke,fill}%
}%
\begin{pgfscope}%
\pgfsys@transformshift{6.267353in}{2.272956in}%
\pgfsys@useobject{currentmarker}{}%
\end{pgfscope}%
\end{pgfscope}%
\begin{pgfscope}%
\definecolor{textcolor}{rgb}{0.000000,0.000000,0.000000}%
\pgfsetstrokecolor{textcolor}%
\pgfsetfillcolor{textcolor}%
\pgftext[x=0.291028in, y=2.239220in, left, base]{\color{textcolor}\rmfamily\fontsize{7.000000}{8.400000}\selectfont 0.6}%
\end{pgfscope}%
\begin{pgfscope}%
\pgfsetbuttcap%
\pgfsetroundjoin%
\definecolor{currentfill}{rgb}{0.000000,0.000000,0.000000}%
\pgfsetfillcolor{currentfill}%
\pgfsetlinewidth{0.501875pt}%
\definecolor{currentstroke}{rgb}{0.000000,0.000000,0.000000}%
\pgfsetstrokecolor{currentstroke}%
\pgfsetdash{}{0pt}%
\pgfsys@defobject{currentmarker}{\pgfqpoint{0.000000in}{0.000000in}}{\pgfqpoint{0.041667in}{0.000000in}}{%
\pgfpathmoveto{\pgfqpoint{0.000000in}{0.000000in}}%
\pgfpathlineto{\pgfqpoint{0.041667in}{0.000000in}}%
\pgfusepath{stroke,fill}%
}%
\begin{pgfscope}%
\pgfsys@transformshift{0.481681in}{2.637782in}%
\pgfsys@useobject{currentmarker}{}%
\end{pgfscope}%
\end{pgfscope}%
\begin{pgfscope}%
\pgfsetbuttcap%
\pgfsetroundjoin%
\definecolor{currentfill}{rgb}{0.000000,0.000000,0.000000}%
\pgfsetfillcolor{currentfill}%
\pgfsetlinewidth{0.501875pt}%
\definecolor{currentstroke}{rgb}{0.000000,0.000000,0.000000}%
\pgfsetstrokecolor{currentstroke}%
\pgfsetdash{}{0pt}%
\pgfsys@defobject{currentmarker}{\pgfqpoint{-0.041667in}{0.000000in}}{\pgfqpoint{-0.000000in}{0.000000in}}{%
\pgfpathmoveto{\pgfqpoint{-0.000000in}{0.000000in}}%
\pgfpathlineto{\pgfqpoint{-0.041667in}{0.000000in}}%
\pgfusepath{stroke,fill}%
}%
\begin{pgfscope}%
\pgfsys@transformshift{6.267353in}{2.637782in}%
\pgfsys@useobject{currentmarker}{}%
\end{pgfscope}%
\end{pgfscope}%
\begin{pgfscope}%
\definecolor{textcolor}{rgb}{0.000000,0.000000,0.000000}%
\pgfsetstrokecolor{textcolor}%
\pgfsetfillcolor{textcolor}%
\pgftext[x=0.291028in, y=2.604046in, left, base]{\color{textcolor}\rmfamily\fontsize{7.000000}{8.400000}\selectfont 0.8}%
\end{pgfscope}%
\begin{pgfscope}%
\pgfsetbuttcap%
\pgfsetroundjoin%
\definecolor{currentfill}{rgb}{0.000000,0.000000,0.000000}%
\pgfsetfillcolor{currentfill}%
\pgfsetlinewidth{0.501875pt}%
\definecolor{currentstroke}{rgb}{0.000000,0.000000,0.000000}%
\pgfsetstrokecolor{currentstroke}%
\pgfsetdash{}{0pt}%
\pgfsys@defobject{currentmarker}{\pgfqpoint{0.000000in}{0.000000in}}{\pgfqpoint{0.041667in}{0.000000in}}{%
\pgfpathmoveto{\pgfqpoint{0.000000in}{0.000000in}}%
\pgfpathlineto{\pgfqpoint{0.041667in}{0.000000in}}%
\pgfusepath{stroke,fill}%
}%
\begin{pgfscope}%
\pgfsys@transformshift{0.481681in}{3.002609in}%
\pgfsys@useobject{currentmarker}{}%
\end{pgfscope}%
\end{pgfscope}%
\begin{pgfscope}%
\pgfsetbuttcap%
\pgfsetroundjoin%
\definecolor{currentfill}{rgb}{0.000000,0.000000,0.000000}%
\pgfsetfillcolor{currentfill}%
\pgfsetlinewidth{0.501875pt}%
\definecolor{currentstroke}{rgb}{0.000000,0.000000,0.000000}%
\pgfsetstrokecolor{currentstroke}%
\pgfsetdash{}{0pt}%
\pgfsys@defobject{currentmarker}{\pgfqpoint{-0.041667in}{0.000000in}}{\pgfqpoint{-0.000000in}{0.000000in}}{%
\pgfpathmoveto{\pgfqpoint{-0.000000in}{0.000000in}}%
\pgfpathlineto{\pgfqpoint{-0.041667in}{0.000000in}}%
\pgfusepath{stroke,fill}%
}%
\begin{pgfscope}%
\pgfsys@transformshift{6.267353in}{3.002609in}%
\pgfsys@useobject{currentmarker}{}%
\end{pgfscope}%
\end{pgfscope}%
\begin{pgfscope}%
\definecolor{textcolor}{rgb}{0.000000,0.000000,0.000000}%
\pgfsetstrokecolor{textcolor}%
\pgfsetfillcolor{textcolor}%
\pgftext[x=0.291028in, y=2.968873in, left, base]{\color{textcolor}\rmfamily\fontsize{7.000000}{8.400000}\selectfont 1.0}%
\end{pgfscope}%
\begin{pgfscope}%
\pgfsetbuttcap%
\pgfsetroundjoin%
\definecolor{currentfill}{rgb}{0.000000,0.000000,0.000000}%
\pgfsetfillcolor{currentfill}%
\pgfsetlinewidth{0.501875pt}%
\definecolor{currentstroke}{rgb}{0.000000,0.000000,0.000000}%
\pgfsetstrokecolor{currentstroke}%
\pgfsetdash{}{0pt}%
\pgfsys@defobject{currentmarker}{\pgfqpoint{0.000000in}{0.000000in}}{\pgfqpoint{0.020833in}{0.000000in}}{%
\pgfpathmoveto{\pgfqpoint{0.000000in}{0.000000in}}%
\pgfpathlineto{\pgfqpoint{0.020833in}{0.000000in}}%
\pgfusepath{stroke,fill}%
}%
\begin{pgfscope}%
\pgfsys@transformshift{0.481681in}{1.087268in}%
\pgfsys@useobject{currentmarker}{}%
\end{pgfscope}%
\end{pgfscope}%
\begin{pgfscope}%
\pgfsetbuttcap%
\pgfsetroundjoin%
\definecolor{currentfill}{rgb}{0.000000,0.000000,0.000000}%
\pgfsetfillcolor{currentfill}%
\pgfsetlinewidth{0.501875pt}%
\definecolor{currentstroke}{rgb}{0.000000,0.000000,0.000000}%
\pgfsetstrokecolor{currentstroke}%
\pgfsetdash{}{0pt}%
\pgfsys@defobject{currentmarker}{\pgfqpoint{-0.020833in}{0.000000in}}{\pgfqpoint{-0.000000in}{0.000000in}}{%
\pgfpathmoveto{\pgfqpoint{-0.000000in}{0.000000in}}%
\pgfpathlineto{\pgfqpoint{-0.020833in}{0.000000in}}%
\pgfusepath{stroke,fill}%
}%
\begin{pgfscope}%
\pgfsys@transformshift{6.267353in}{1.087268in}%
\pgfsys@useobject{currentmarker}{}%
\end{pgfscope}%
\end{pgfscope}%
\begin{pgfscope}%
\pgfsetbuttcap%
\pgfsetroundjoin%
\definecolor{currentfill}{rgb}{0.000000,0.000000,0.000000}%
\pgfsetfillcolor{currentfill}%
\pgfsetlinewidth{0.501875pt}%
\definecolor{currentstroke}{rgb}{0.000000,0.000000,0.000000}%
\pgfsetstrokecolor{currentstroke}%
\pgfsetdash{}{0pt}%
\pgfsys@defobject{currentmarker}{\pgfqpoint{0.000000in}{0.000000in}}{\pgfqpoint{0.020833in}{0.000000in}}{%
\pgfpathmoveto{\pgfqpoint{0.000000in}{0.000000in}}%
\pgfpathlineto{\pgfqpoint{0.020833in}{0.000000in}}%
\pgfusepath{stroke,fill}%
}%
\begin{pgfscope}%
\pgfsys@transformshift{0.481681in}{1.269682in}%
\pgfsys@useobject{currentmarker}{}%
\end{pgfscope}%
\end{pgfscope}%
\begin{pgfscope}%
\pgfsetbuttcap%
\pgfsetroundjoin%
\definecolor{currentfill}{rgb}{0.000000,0.000000,0.000000}%
\pgfsetfillcolor{currentfill}%
\pgfsetlinewidth{0.501875pt}%
\definecolor{currentstroke}{rgb}{0.000000,0.000000,0.000000}%
\pgfsetstrokecolor{currentstroke}%
\pgfsetdash{}{0pt}%
\pgfsys@defobject{currentmarker}{\pgfqpoint{-0.020833in}{0.000000in}}{\pgfqpoint{-0.000000in}{0.000000in}}{%
\pgfpathmoveto{\pgfqpoint{-0.000000in}{0.000000in}}%
\pgfpathlineto{\pgfqpoint{-0.020833in}{0.000000in}}%
\pgfusepath{stroke,fill}%
}%
\begin{pgfscope}%
\pgfsys@transformshift{6.267353in}{1.269682in}%
\pgfsys@useobject{currentmarker}{}%
\end{pgfscope}%
\end{pgfscope}%
\begin{pgfscope}%
\pgfsetbuttcap%
\pgfsetroundjoin%
\definecolor{currentfill}{rgb}{0.000000,0.000000,0.000000}%
\pgfsetfillcolor{currentfill}%
\pgfsetlinewidth{0.501875pt}%
\definecolor{currentstroke}{rgb}{0.000000,0.000000,0.000000}%
\pgfsetstrokecolor{currentstroke}%
\pgfsetdash{}{0pt}%
\pgfsys@defobject{currentmarker}{\pgfqpoint{0.000000in}{0.000000in}}{\pgfqpoint{0.020833in}{0.000000in}}{%
\pgfpathmoveto{\pgfqpoint{0.000000in}{0.000000in}}%
\pgfpathlineto{\pgfqpoint{0.020833in}{0.000000in}}%
\pgfusepath{stroke,fill}%
}%
\begin{pgfscope}%
\pgfsys@transformshift{0.481681in}{1.360888in}%
\pgfsys@useobject{currentmarker}{}%
\end{pgfscope}%
\end{pgfscope}%
\begin{pgfscope}%
\pgfsetbuttcap%
\pgfsetroundjoin%
\definecolor{currentfill}{rgb}{0.000000,0.000000,0.000000}%
\pgfsetfillcolor{currentfill}%
\pgfsetlinewidth{0.501875pt}%
\definecolor{currentstroke}{rgb}{0.000000,0.000000,0.000000}%
\pgfsetstrokecolor{currentstroke}%
\pgfsetdash{}{0pt}%
\pgfsys@defobject{currentmarker}{\pgfqpoint{-0.020833in}{0.000000in}}{\pgfqpoint{-0.000000in}{0.000000in}}{%
\pgfpathmoveto{\pgfqpoint{-0.000000in}{0.000000in}}%
\pgfpathlineto{\pgfqpoint{-0.020833in}{0.000000in}}%
\pgfusepath{stroke,fill}%
}%
\begin{pgfscope}%
\pgfsys@transformshift{6.267353in}{1.360888in}%
\pgfsys@useobject{currentmarker}{}%
\end{pgfscope}%
\end{pgfscope}%
\begin{pgfscope}%
\pgfsetbuttcap%
\pgfsetroundjoin%
\definecolor{currentfill}{rgb}{0.000000,0.000000,0.000000}%
\pgfsetfillcolor{currentfill}%
\pgfsetlinewidth{0.501875pt}%
\definecolor{currentstroke}{rgb}{0.000000,0.000000,0.000000}%
\pgfsetstrokecolor{currentstroke}%
\pgfsetdash{}{0pt}%
\pgfsys@defobject{currentmarker}{\pgfqpoint{0.000000in}{0.000000in}}{\pgfqpoint{0.020833in}{0.000000in}}{%
\pgfpathmoveto{\pgfqpoint{0.000000in}{0.000000in}}%
\pgfpathlineto{\pgfqpoint{0.020833in}{0.000000in}}%
\pgfusepath{stroke,fill}%
}%
\begin{pgfscope}%
\pgfsys@transformshift{0.481681in}{1.452095in}%
\pgfsys@useobject{currentmarker}{}%
\end{pgfscope}%
\end{pgfscope}%
\begin{pgfscope}%
\pgfsetbuttcap%
\pgfsetroundjoin%
\definecolor{currentfill}{rgb}{0.000000,0.000000,0.000000}%
\pgfsetfillcolor{currentfill}%
\pgfsetlinewidth{0.501875pt}%
\definecolor{currentstroke}{rgb}{0.000000,0.000000,0.000000}%
\pgfsetstrokecolor{currentstroke}%
\pgfsetdash{}{0pt}%
\pgfsys@defobject{currentmarker}{\pgfqpoint{-0.020833in}{0.000000in}}{\pgfqpoint{-0.000000in}{0.000000in}}{%
\pgfpathmoveto{\pgfqpoint{-0.000000in}{0.000000in}}%
\pgfpathlineto{\pgfqpoint{-0.020833in}{0.000000in}}%
\pgfusepath{stroke,fill}%
}%
\begin{pgfscope}%
\pgfsys@transformshift{6.267353in}{1.452095in}%
\pgfsys@useobject{currentmarker}{}%
\end{pgfscope}%
\end{pgfscope}%
\begin{pgfscope}%
\pgfsetbuttcap%
\pgfsetroundjoin%
\definecolor{currentfill}{rgb}{0.000000,0.000000,0.000000}%
\pgfsetfillcolor{currentfill}%
\pgfsetlinewidth{0.501875pt}%
\definecolor{currentstroke}{rgb}{0.000000,0.000000,0.000000}%
\pgfsetstrokecolor{currentstroke}%
\pgfsetdash{}{0pt}%
\pgfsys@defobject{currentmarker}{\pgfqpoint{0.000000in}{0.000000in}}{\pgfqpoint{0.020833in}{0.000000in}}{%
\pgfpathmoveto{\pgfqpoint{0.000000in}{0.000000in}}%
\pgfpathlineto{\pgfqpoint{0.020833in}{0.000000in}}%
\pgfusepath{stroke,fill}%
}%
\begin{pgfscope}%
\pgfsys@transformshift{0.481681in}{1.634509in}%
\pgfsys@useobject{currentmarker}{}%
\end{pgfscope}%
\end{pgfscope}%
\begin{pgfscope}%
\pgfsetbuttcap%
\pgfsetroundjoin%
\definecolor{currentfill}{rgb}{0.000000,0.000000,0.000000}%
\pgfsetfillcolor{currentfill}%
\pgfsetlinewidth{0.501875pt}%
\definecolor{currentstroke}{rgb}{0.000000,0.000000,0.000000}%
\pgfsetstrokecolor{currentstroke}%
\pgfsetdash{}{0pt}%
\pgfsys@defobject{currentmarker}{\pgfqpoint{-0.020833in}{0.000000in}}{\pgfqpoint{-0.000000in}{0.000000in}}{%
\pgfpathmoveto{\pgfqpoint{-0.000000in}{0.000000in}}%
\pgfpathlineto{\pgfqpoint{-0.020833in}{0.000000in}}%
\pgfusepath{stroke,fill}%
}%
\begin{pgfscope}%
\pgfsys@transformshift{6.267353in}{1.634509in}%
\pgfsys@useobject{currentmarker}{}%
\end{pgfscope}%
\end{pgfscope}%
\begin{pgfscope}%
\pgfsetbuttcap%
\pgfsetroundjoin%
\definecolor{currentfill}{rgb}{0.000000,0.000000,0.000000}%
\pgfsetfillcolor{currentfill}%
\pgfsetlinewidth{0.501875pt}%
\definecolor{currentstroke}{rgb}{0.000000,0.000000,0.000000}%
\pgfsetstrokecolor{currentstroke}%
\pgfsetdash{}{0pt}%
\pgfsys@defobject{currentmarker}{\pgfqpoint{0.000000in}{0.000000in}}{\pgfqpoint{0.020833in}{0.000000in}}{%
\pgfpathmoveto{\pgfqpoint{0.000000in}{0.000000in}}%
\pgfpathlineto{\pgfqpoint{0.020833in}{0.000000in}}%
\pgfusepath{stroke,fill}%
}%
\begin{pgfscope}%
\pgfsys@transformshift{0.481681in}{1.725715in}%
\pgfsys@useobject{currentmarker}{}%
\end{pgfscope}%
\end{pgfscope}%
\begin{pgfscope}%
\pgfsetbuttcap%
\pgfsetroundjoin%
\definecolor{currentfill}{rgb}{0.000000,0.000000,0.000000}%
\pgfsetfillcolor{currentfill}%
\pgfsetlinewidth{0.501875pt}%
\definecolor{currentstroke}{rgb}{0.000000,0.000000,0.000000}%
\pgfsetstrokecolor{currentstroke}%
\pgfsetdash{}{0pt}%
\pgfsys@defobject{currentmarker}{\pgfqpoint{-0.020833in}{0.000000in}}{\pgfqpoint{-0.000000in}{0.000000in}}{%
\pgfpathmoveto{\pgfqpoint{-0.000000in}{0.000000in}}%
\pgfpathlineto{\pgfqpoint{-0.020833in}{0.000000in}}%
\pgfusepath{stroke,fill}%
}%
\begin{pgfscope}%
\pgfsys@transformshift{6.267353in}{1.725715in}%
\pgfsys@useobject{currentmarker}{}%
\end{pgfscope}%
\end{pgfscope}%
\begin{pgfscope}%
\pgfsetbuttcap%
\pgfsetroundjoin%
\definecolor{currentfill}{rgb}{0.000000,0.000000,0.000000}%
\pgfsetfillcolor{currentfill}%
\pgfsetlinewidth{0.501875pt}%
\definecolor{currentstroke}{rgb}{0.000000,0.000000,0.000000}%
\pgfsetstrokecolor{currentstroke}%
\pgfsetdash{}{0pt}%
\pgfsys@defobject{currentmarker}{\pgfqpoint{0.000000in}{0.000000in}}{\pgfqpoint{0.020833in}{0.000000in}}{%
\pgfpathmoveto{\pgfqpoint{0.000000in}{0.000000in}}%
\pgfpathlineto{\pgfqpoint{0.020833in}{0.000000in}}%
\pgfusepath{stroke,fill}%
}%
\begin{pgfscope}%
\pgfsys@transformshift{0.481681in}{1.816922in}%
\pgfsys@useobject{currentmarker}{}%
\end{pgfscope}%
\end{pgfscope}%
\begin{pgfscope}%
\pgfsetbuttcap%
\pgfsetroundjoin%
\definecolor{currentfill}{rgb}{0.000000,0.000000,0.000000}%
\pgfsetfillcolor{currentfill}%
\pgfsetlinewidth{0.501875pt}%
\definecolor{currentstroke}{rgb}{0.000000,0.000000,0.000000}%
\pgfsetstrokecolor{currentstroke}%
\pgfsetdash{}{0pt}%
\pgfsys@defobject{currentmarker}{\pgfqpoint{-0.020833in}{0.000000in}}{\pgfqpoint{-0.000000in}{0.000000in}}{%
\pgfpathmoveto{\pgfqpoint{-0.000000in}{0.000000in}}%
\pgfpathlineto{\pgfqpoint{-0.020833in}{0.000000in}}%
\pgfusepath{stroke,fill}%
}%
\begin{pgfscope}%
\pgfsys@transformshift{6.267353in}{1.816922in}%
\pgfsys@useobject{currentmarker}{}%
\end{pgfscope}%
\end{pgfscope}%
\begin{pgfscope}%
\pgfsetbuttcap%
\pgfsetroundjoin%
\definecolor{currentfill}{rgb}{0.000000,0.000000,0.000000}%
\pgfsetfillcolor{currentfill}%
\pgfsetlinewidth{0.501875pt}%
\definecolor{currentstroke}{rgb}{0.000000,0.000000,0.000000}%
\pgfsetstrokecolor{currentstroke}%
\pgfsetdash{}{0pt}%
\pgfsys@defobject{currentmarker}{\pgfqpoint{0.000000in}{0.000000in}}{\pgfqpoint{0.020833in}{0.000000in}}{%
\pgfpathmoveto{\pgfqpoint{0.000000in}{0.000000in}}%
\pgfpathlineto{\pgfqpoint{0.020833in}{0.000000in}}%
\pgfusepath{stroke,fill}%
}%
\begin{pgfscope}%
\pgfsys@transformshift{0.481681in}{1.999335in}%
\pgfsys@useobject{currentmarker}{}%
\end{pgfscope}%
\end{pgfscope}%
\begin{pgfscope}%
\pgfsetbuttcap%
\pgfsetroundjoin%
\definecolor{currentfill}{rgb}{0.000000,0.000000,0.000000}%
\pgfsetfillcolor{currentfill}%
\pgfsetlinewidth{0.501875pt}%
\definecolor{currentstroke}{rgb}{0.000000,0.000000,0.000000}%
\pgfsetstrokecolor{currentstroke}%
\pgfsetdash{}{0pt}%
\pgfsys@defobject{currentmarker}{\pgfqpoint{-0.020833in}{0.000000in}}{\pgfqpoint{-0.000000in}{0.000000in}}{%
\pgfpathmoveto{\pgfqpoint{-0.000000in}{0.000000in}}%
\pgfpathlineto{\pgfqpoint{-0.020833in}{0.000000in}}%
\pgfusepath{stroke,fill}%
}%
\begin{pgfscope}%
\pgfsys@transformshift{6.267353in}{1.999335in}%
\pgfsys@useobject{currentmarker}{}%
\end{pgfscope}%
\end{pgfscope}%
\begin{pgfscope}%
\pgfsetbuttcap%
\pgfsetroundjoin%
\definecolor{currentfill}{rgb}{0.000000,0.000000,0.000000}%
\pgfsetfillcolor{currentfill}%
\pgfsetlinewidth{0.501875pt}%
\definecolor{currentstroke}{rgb}{0.000000,0.000000,0.000000}%
\pgfsetstrokecolor{currentstroke}%
\pgfsetdash{}{0pt}%
\pgfsys@defobject{currentmarker}{\pgfqpoint{0.000000in}{0.000000in}}{\pgfqpoint{0.020833in}{0.000000in}}{%
\pgfpathmoveto{\pgfqpoint{0.000000in}{0.000000in}}%
\pgfpathlineto{\pgfqpoint{0.020833in}{0.000000in}}%
\pgfusepath{stroke,fill}%
}%
\begin{pgfscope}%
\pgfsys@transformshift{0.481681in}{2.090542in}%
\pgfsys@useobject{currentmarker}{}%
\end{pgfscope}%
\end{pgfscope}%
\begin{pgfscope}%
\pgfsetbuttcap%
\pgfsetroundjoin%
\definecolor{currentfill}{rgb}{0.000000,0.000000,0.000000}%
\pgfsetfillcolor{currentfill}%
\pgfsetlinewidth{0.501875pt}%
\definecolor{currentstroke}{rgb}{0.000000,0.000000,0.000000}%
\pgfsetstrokecolor{currentstroke}%
\pgfsetdash{}{0pt}%
\pgfsys@defobject{currentmarker}{\pgfqpoint{-0.020833in}{0.000000in}}{\pgfqpoint{-0.000000in}{0.000000in}}{%
\pgfpathmoveto{\pgfqpoint{-0.000000in}{0.000000in}}%
\pgfpathlineto{\pgfqpoint{-0.020833in}{0.000000in}}%
\pgfusepath{stroke,fill}%
}%
\begin{pgfscope}%
\pgfsys@transformshift{6.267353in}{2.090542in}%
\pgfsys@useobject{currentmarker}{}%
\end{pgfscope}%
\end{pgfscope}%
\begin{pgfscope}%
\pgfsetbuttcap%
\pgfsetroundjoin%
\definecolor{currentfill}{rgb}{0.000000,0.000000,0.000000}%
\pgfsetfillcolor{currentfill}%
\pgfsetlinewidth{0.501875pt}%
\definecolor{currentstroke}{rgb}{0.000000,0.000000,0.000000}%
\pgfsetstrokecolor{currentstroke}%
\pgfsetdash{}{0pt}%
\pgfsys@defobject{currentmarker}{\pgfqpoint{0.000000in}{0.000000in}}{\pgfqpoint{0.020833in}{0.000000in}}{%
\pgfpathmoveto{\pgfqpoint{0.000000in}{0.000000in}}%
\pgfpathlineto{\pgfqpoint{0.020833in}{0.000000in}}%
\pgfusepath{stroke,fill}%
}%
\begin{pgfscope}%
\pgfsys@transformshift{0.481681in}{2.181749in}%
\pgfsys@useobject{currentmarker}{}%
\end{pgfscope}%
\end{pgfscope}%
\begin{pgfscope}%
\pgfsetbuttcap%
\pgfsetroundjoin%
\definecolor{currentfill}{rgb}{0.000000,0.000000,0.000000}%
\pgfsetfillcolor{currentfill}%
\pgfsetlinewidth{0.501875pt}%
\definecolor{currentstroke}{rgb}{0.000000,0.000000,0.000000}%
\pgfsetstrokecolor{currentstroke}%
\pgfsetdash{}{0pt}%
\pgfsys@defobject{currentmarker}{\pgfqpoint{-0.020833in}{0.000000in}}{\pgfqpoint{-0.000000in}{0.000000in}}{%
\pgfpathmoveto{\pgfqpoint{-0.000000in}{0.000000in}}%
\pgfpathlineto{\pgfqpoint{-0.020833in}{0.000000in}}%
\pgfusepath{stroke,fill}%
}%
\begin{pgfscope}%
\pgfsys@transformshift{6.267353in}{2.181749in}%
\pgfsys@useobject{currentmarker}{}%
\end{pgfscope}%
\end{pgfscope}%
\begin{pgfscope}%
\pgfsetbuttcap%
\pgfsetroundjoin%
\definecolor{currentfill}{rgb}{0.000000,0.000000,0.000000}%
\pgfsetfillcolor{currentfill}%
\pgfsetlinewidth{0.501875pt}%
\definecolor{currentstroke}{rgb}{0.000000,0.000000,0.000000}%
\pgfsetstrokecolor{currentstroke}%
\pgfsetdash{}{0pt}%
\pgfsys@defobject{currentmarker}{\pgfqpoint{0.000000in}{0.000000in}}{\pgfqpoint{0.020833in}{0.000000in}}{%
\pgfpathmoveto{\pgfqpoint{0.000000in}{0.000000in}}%
\pgfpathlineto{\pgfqpoint{0.020833in}{0.000000in}}%
\pgfusepath{stroke,fill}%
}%
\begin{pgfscope}%
\pgfsys@transformshift{0.481681in}{2.364162in}%
\pgfsys@useobject{currentmarker}{}%
\end{pgfscope}%
\end{pgfscope}%
\begin{pgfscope}%
\pgfsetbuttcap%
\pgfsetroundjoin%
\definecolor{currentfill}{rgb}{0.000000,0.000000,0.000000}%
\pgfsetfillcolor{currentfill}%
\pgfsetlinewidth{0.501875pt}%
\definecolor{currentstroke}{rgb}{0.000000,0.000000,0.000000}%
\pgfsetstrokecolor{currentstroke}%
\pgfsetdash{}{0pt}%
\pgfsys@defobject{currentmarker}{\pgfqpoint{-0.020833in}{0.000000in}}{\pgfqpoint{-0.000000in}{0.000000in}}{%
\pgfpathmoveto{\pgfqpoint{-0.000000in}{0.000000in}}%
\pgfpathlineto{\pgfqpoint{-0.020833in}{0.000000in}}%
\pgfusepath{stroke,fill}%
}%
\begin{pgfscope}%
\pgfsys@transformshift{6.267353in}{2.364162in}%
\pgfsys@useobject{currentmarker}{}%
\end{pgfscope}%
\end{pgfscope}%
\begin{pgfscope}%
\pgfsetbuttcap%
\pgfsetroundjoin%
\definecolor{currentfill}{rgb}{0.000000,0.000000,0.000000}%
\pgfsetfillcolor{currentfill}%
\pgfsetlinewidth{0.501875pt}%
\definecolor{currentstroke}{rgb}{0.000000,0.000000,0.000000}%
\pgfsetstrokecolor{currentstroke}%
\pgfsetdash{}{0pt}%
\pgfsys@defobject{currentmarker}{\pgfqpoint{0.000000in}{0.000000in}}{\pgfqpoint{0.020833in}{0.000000in}}{%
\pgfpathmoveto{\pgfqpoint{0.000000in}{0.000000in}}%
\pgfpathlineto{\pgfqpoint{0.020833in}{0.000000in}}%
\pgfusepath{stroke,fill}%
}%
\begin{pgfscope}%
\pgfsys@transformshift{0.481681in}{2.455369in}%
\pgfsys@useobject{currentmarker}{}%
\end{pgfscope}%
\end{pgfscope}%
\begin{pgfscope}%
\pgfsetbuttcap%
\pgfsetroundjoin%
\definecolor{currentfill}{rgb}{0.000000,0.000000,0.000000}%
\pgfsetfillcolor{currentfill}%
\pgfsetlinewidth{0.501875pt}%
\definecolor{currentstroke}{rgb}{0.000000,0.000000,0.000000}%
\pgfsetstrokecolor{currentstroke}%
\pgfsetdash{}{0pt}%
\pgfsys@defobject{currentmarker}{\pgfqpoint{-0.020833in}{0.000000in}}{\pgfqpoint{-0.000000in}{0.000000in}}{%
\pgfpathmoveto{\pgfqpoint{-0.000000in}{0.000000in}}%
\pgfpathlineto{\pgfqpoint{-0.020833in}{0.000000in}}%
\pgfusepath{stroke,fill}%
}%
\begin{pgfscope}%
\pgfsys@transformshift{6.267353in}{2.455369in}%
\pgfsys@useobject{currentmarker}{}%
\end{pgfscope}%
\end{pgfscope}%
\begin{pgfscope}%
\pgfsetbuttcap%
\pgfsetroundjoin%
\definecolor{currentfill}{rgb}{0.000000,0.000000,0.000000}%
\pgfsetfillcolor{currentfill}%
\pgfsetlinewidth{0.501875pt}%
\definecolor{currentstroke}{rgb}{0.000000,0.000000,0.000000}%
\pgfsetstrokecolor{currentstroke}%
\pgfsetdash{}{0pt}%
\pgfsys@defobject{currentmarker}{\pgfqpoint{0.000000in}{0.000000in}}{\pgfqpoint{0.020833in}{0.000000in}}{%
\pgfpathmoveto{\pgfqpoint{0.000000in}{0.000000in}}%
\pgfpathlineto{\pgfqpoint{0.020833in}{0.000000in}}%
\pgfusepath{stroke,fill}%
}%
\begin{pgfscope}%
\pgfsys@transformshift{0.481681in}{2.546576in}%
\pgfsys@useobject{currentmarker}{}%
\end{pgfscope}%
\end{pgfscope}%
\begin{pgfscope}%
\pgfsetbuttcap%
\pgfsetroundjoin%
\definecolor{currentfill}{rgb}{0.000000,0.000000,0.000000}%
\pgfsetfillcolor{currentfill}%
\pgfsetlinewidth{0.501875pt}%
\definecolor{currentstroke}{rgb}{0.000000,0.000000,0.000000}%
\pgfsetstrokecolor{currentstroke}%
\pgfsetdash{}{0pt}%
\pgfsys@defobject{currentmarker}{\pgfqpoint{-0.020833in}{0.000000in}}{\pgfqpoint{-0.000000in}{0.000000in}}{%
\pgfpathmoveto{\pgfqpoint{-0.000000in}{0.000000in}}%
\pgfpathlineto{\pgfqpoint{-0.020833in}{0.000000in}}%
\pgfusepath{stroke,fill}%
}%
\begin{pgfscope}%
\pgfsys@transformshift{6.267353in}{2.546576in}%
\pgfsys@useobject{currentmarker}{}%
\end{pgfscope}%
\end{pgfscope}%
\begin{pgfscope}%
\pgfsetbuttcap%
\pgfsetroundjoin%
\definecolor{currentfill}{rgb}{0.000000,0.000000,0.000000}%
\pgfsetfillcolor{currentfill}%
\pgfsetlinewidth{0.501875pt}%
\definecolor{currentstroke}{rgb}{0.000000,0.000000,0.000000}%
\pgfsetstrokecolor{currentstroke}%
\pgfsetdash{}{0pt}%
\pgfsys@defobject{currentmarker}{\pgfqpoint{0.000000in}{0.000000in}}{\pgfqpoint{0.020833in}{0.000000in}}{%
\pgfpathmoveto{\pgfqpoint{0.000000in}{0.000000in}}%
\pgfpathlineto{\pgfqpoint{0.020833in}{0.000000in}}%
\pgfusepath{stroke,fill}%
}%
\begin{pgfscope}%
\pgfsys@transformshift{0.481681in}{2.728989in}%
\pgfsys@useobject{currentmarker}{}%
\end{pgfscope}%
\end{pgfscope}%
\begin{pgfscope}%
\pgfsetbuttcap%
\pgfsetroundjoin%
\definecolor{currentfill}{rgb}{0.000000,0.000000,0.000000}%
\pgfsetfillcolor{currentfill}%
\pgfsetlinewidth{0.501875pt}%
\definecolor{currentstroke}{rgb}{0.000000,0.000000,0.000000}%
\pgfsetstrokecolor{currentstroke}%
\pgfsetdash{}{0pt}%
\pgfsys@defobject{currentmarker}{\pgfqpoint{-0.020833in}{0.000000in}}{\pgfqpoint{-0.000000in}{0.000000in}}{%
\pgfpathmoveto{\pgfqpoint{-0.000000in}{0.000000in}}%
\pgfpathlineto{\pgfqpoint{-0.020833in}{0.000000in}}%
\pgfusepath{stroke,fill}%
}%
\begin{pgfscope}%
\pgfsys@transformshift{6.267353in}{2.728989in}%
\pgfsys@useobject{currentmarker}{}%
\end{pgfscope}%
\end{pgfscope}%
\begin{pgfscope}%
\pgfsetbuttcap%
\pgfsetroundjoin%
\definecolor{currentfill}{rgb}{0.000000,0.000000,0.000000}%
\pgfsetfillcolor{currentfill}%
\pgfsetlinewidth{0.501875pt}%
\definecolor{currentstroke}{rgb}{0.000000,0.000000,0.000000}%
\pgfsetstrokecolor{currentstroke}%
\pgfsetdash{}{0pt}%
\pgfsys@defobject{currentmarker}{\pgfqpoint{0.000000in}{0.000000in}}{\pgfqpoint{0.020833in}{0.000000in}}{%
\pgfpathmoveto{\pgfqpoint{0.000000in}{0.000000in}}%
\pgfpathlineto{\pgfqpoint{0.020833in}{0.000000in}}%
\pgfusepath{stroke,fill}%
}%
\begin{pgfscope}%
\pgfsys@transformshift{0.481681in}{2.820196in}%
\pgfsys@useobject{currentmarker}{}%
\end{pgfscope}%
\end{pgfscope}%
\begin{pgfscope}%
\pgfsetbuttcap%
\pgfsetroundjoin%
\definecolor{currentfill}{rgb}{0.000000,0.000000,0.000000}%
\pgfsetfillcolor{currentfill}%
\pgfsetlinewidth{0.501875pt}%
\definecolor{currentstroke}{rgb}{0.000000,0.000000,0.000000}%
\pgfsetstrokecolor{currentstroke}%
\pgfsetdash{}{0pt}%
\pgfsys@defobject{currentmarker}{\pgfqpoint{-0.020833in}{0.000000in}}{\pgfqpoint{-0.000000in}{0.000000in}}{%
\pgfpathmoveto{\pgfqpoint{-0.000000in}{0.000000in}}%
\pgfpathlineto{\pgfqpoint{-0.020833in}{0.000000in}}%
\pgfusepath{stroke,fill}%
}%
\begin{pgfscope}%
\pgfsys@transformshift{6.267353in}{2.820196in}%
\pgfsys@useobject{currentmarker}{}%
\end{pgfscope}%
\end{pgfscope}%
\begin{pgfscope}%
\pgfsetbuttcap%
\pgfsetroundjoin%
\definecolor{currentfill}{rgb}{0.000000,0.000000,0.000000}%
\pgfsetfillcolor{currentfill}%
\pgfsetlinewidth{0.501875pt}%
\definecolor{currentstroke}{rgb}{0.000000,0.000000,0.000000}%
\pgfsetstrokecolor{currentstroke}%
\pgfsetdash{}{0pt}%
\pgfsys@defobject{currentmarker}{\pgfqpoint{0.000000in}{0.000000in}}{\pgfqpoint{0.020833in}{0.000000in}}{%
\pgfpathmoveto{\pgfqpoint{0.000000in}{0.000000in}}%
\pgfpathlineto{\pgfqpoint{0.020833in}{0.000000in}}%
\pgfusepath{stroke,fill}%
}%
\begin{pgfscope}%
\pgfsys@transformshift{0.481681in}{2.911403in}%
\pgfsys@useobject{currentmarker}{}%
\end{pgfscope}%
\end{pgfscope}%
\begin{pgfscope}%
\pgfsetbuttcap%
\pgfsetroundjoin%
\definecolor{currentfill}{rgb}{0.000000,0.000000,0.000000}%
\pgfsetfillcolor{currentfill}%
\pgfsetlinewidth{0.501875pt}%
\definecolor{currentstroke}{rgb}{0.000000,0.000000,0.000000}%
\pgfsetstrokecolor{currentstroke}%
\pgfsetdash{}{0pt}%
\pgfsys@defobject{currentmarker}{\pgfqpoint{-0.020833in}{0.000000in}}{\pgfqpoint{-0.000000in}{0.000000in}}{%
\pgfpathmoveto{\pgfqpoint{-0.000000in}{0.000000in}}%
\pgfpathlineto{\pgfqpoint{-0.020833in}{0.000000in}}%
\pgfusepath{stroke,fill}%
}%
\begin{pgfscope}%
\pgfsys@transformshift{6.267353in}{2.911403in}%
\pgfsys@useobject{currentmarker}{}%
\end{pgfscope}%
\end{pgfscope}%
\begin{pgfscope}%
\pgfsetbuttcap%
\pgfsetroundjoin%
\definecolor{currentfill}{rgb}{0.000000,0.000000,0.000000}%
\pgfsetfillcolor{currentfill}%
\pgfsetlinewidth{0.501875pt}%
\definecolor{currentstroke}{rgb}{0.000000,0.000000,0.000000}%
\pgfsetstrokecolor{currentstroke}%
\pgfsetdash{}{0pt}%
\pgfsys@defobject{currentmarker}{\pgfqpoint{0.000000in}{0.000000in}}{\pgfqpoint{0.020833in}{0.000000in}}{%
\pgfpathmoveto{\pgfqpoint{0.000000in}{0.000000in}}%
\pgfpathlineto{\pgfqpoint{0.020833in}{0.000000in}}%
\pgfusepath{stroke,fill}%
}%
\begin{pgfscope}%
\pgfsys@transformshift{0.481681in}{3.093816in}%
\pgfsys@useobject{currentmarker}{}%
\end{pgfscope}%
\end{pgfscope}%
\begin{pgfscope}%
\pgfsetbuttcap%
\pgfsetroundjoin%
\definecolor{currentfill}{rgb}{0.000000,0.000000,0.000000}%
\pgfsetfillcolor{currentfill}%
\pgfsetlinewidth{0.501875pt}%
\definecolor{currentstroke}{rgb}{0.000000,0.000000,0.000000}%
\pgfsetstrokecolor{currentstroke}%
\pgfsetdash{}{0pt}%
\pgfsys@defobject{currentmarker}{\pgfqpoint{-0.020833in}{0.000000in}}{\pgfqpoint{-0.000000in}{0.000000in}}{%
\pgfpathmoveto{\pgfqpoint{-0.000000in}{0.000000in}}%
\pgfpathlineto{\pgfqpoint{-0.020833in}{0.000000in}}%
\pgfusepath{stroke,fill}%
}%
\begin{pgfscope}%
\pgfsys@transformshift{6.267353in}{3.093816in}%
\pgfsys@useobject{currentmarker}{}%
\end{pgfscope}%
\end{pgfscope}%
\begin{pgfscope}%
\pgfsetbuttcap%
\pgfsetroundjoin%
\definecolor{currentfill}{rgb}{0.000000,0.000000,0.000000}%
\pgfsetfillcolor{currentfill}%
\pgfsetlinewidth{0.501875pt}%
\definecolor{currentstroke}{rgb}{0.000000,0.000000,0.000000}%
\pgfsetstrokecolor{currentstroke}%
\pgfsetdash{}{0pt}%
\pgfsys@defobject{currentmarker}{\pgfqpoint{0.000000in}{0.000000in}}{\pgfqpoint{0.020833in}{0.000000in}}{%
\pgfpathmoveto{\pgfqpoint{0.000000in}{0.000000in}}%
\pgfpathlineto{\pgfqpoint{0.020833in}{0.000000in}}%
\pgfusepath{stroke,fill}%
}%
\begin{pgfscope}%
\pgfsys@transformshift{0.481681in}{3.185023in}%
\pgfsys@useobject{currentmarker}{}%
\end{pgfscope}%
\end{pgfscope}%
\begin{pgfscope}%
\pgfsetbuttcap%
\pgfsetroundjoin%
\definecolor{currentfill}{rgb}{0.000000,0.000000,0.000000}%
\pgfsetfillcolor{currentfill}%
\pgfsetlinewidth{0.501875pt}%
\definecolor{currentstroke}{rgb}{0.000000,0.000000,0.000000}%
\pgfsetstrokecolor{currentstroke}%
\pgfsetdash{}{0pt}%
\pgfsys@defobject{currentmarker}{\pgfqpoint{-0.020833in}{0.000000in}}{\pgfqpoint{-0.000000in}{0.000000in}}{%
\pgfpathmoveto{\pgfqpoint{-0.000000in}{0.000000in}}%
\pgfpathlineto{\pgfqpoint{-0.020833in}{0.000000in}}%
\pgfusepath{stroke,fill}%
}%
\begin{pgfscope}%
\pgfsys@transformshift{6.267353in}{3.185023in}%
\pgfsys@useobject{currentmarker}{}%
\end{pgfscope}%
\end{pgfscope}%
\begin{pgfscope}%
\definecolor{textcolor}{rgb}{0.000000,0.000000,0.000000}%
\pgfsetstrokecolor{textcolor}%
\pgfsetfillcolor{textcolor}%
\pgftext[x=0.235472in,y=2.154322in,,bottom,rotate=90.000000]{\color{textcolor}\rmfamily\fontsize{8.000000}{9.600000}\selectfont Wert}%
\end{pgfscope}%
\begin{pgfscope}%
\pgfpathrectangle{\pgfqpoint{0.481681in}{1.080890in}}{\pgfqpoint{5.785672in}{2.146863in}}%
\pgfusepath{clip}%
\pgfsetrectcap%
\pgfsetroundjoin%
\pgfsetlinewidth{0.401500pt}%
\definecolor{currentstroke}{rgb}{0.000000,0.070588,0.098039}%
\pgfsetstrokecolor{currentstroke}%
\pgfsetdash{}{0pt}%
\pgfpathmoveto{\pgfqpoint{0.744666in}{1.178475in}}%
\pgfpathlineto{\pgfqpoint{0.846248in}{1.181082in}}%
\pgfpathlineto{\pgfqpoint{0.846813in}{1.184920in}}%
\pgfpathlineto{\pgfqpoint{0.847377in}{1.200834in}}%
\pgfpathlineto{\pgfqpoint{0.847941in}{1.225298in}}%
\pgfpathlineto{\pgfqpoint{0.848506in}{1.301895in}}%
\pgfpathlineto{\pgfqpoint{0.915098in}{1.314943in}}%
\pgfpathlineto{\pgfqpoint{0.928643in}{1.323270in}}%
\pgfpathlineto{\pgfqpoint{0.929207in}{1.325184in}}%
\pgfpathlineto{\pgfqpoint{0.939365in}{1.378271in}}%
\pgfpathlineto{\pgfqpoint{0.940494in}{1.394612in}}%
\pgfpathlineto{\pgfqpoint{0.942751in}{1.432521in}}%
\pgfpathlineto{\pgfqpoint{0.943316in}{1.435598in}}%
\pgfpathlineto{\pgfqpoint{0.999186in}{1.449165in}}%
\pgfpathlineto{\pgfqpoint{1.018938in}{1.456807in}}%
\pgfpathlineto{\pgfqpoint{1.019502in}{1.457362in}}%
\pgfpathlineto{\pgfqpoint{1.020631in}{1.469099in}}%
\pgfpathlineto{\pgfqpoint{1.024581in}{1.513021in}}%
\pgfpathlineto{\pgfqpoint{1.026274in}{1.513143in}}%
\pgfpathlineto{\pgfqpoint{1.198964in}{1.515267in}}%
\pgfpathlineto{\pgfqpoint{1.199529in}{1.516389in}}%
\pgfpathlineto{\pgfqpoint{1.200093in}{1.520525in}}%
\pgfpathlineto{\pgfqpoint{1.222102in}{1.569826in}}%
\pgfpathlineto{\pgfqpoint{1.222667in}{1.588151in}}%
\pgfpathlineto{\pgfqpoint{1.223231in}{1.645547in}}%
\pgfpathlineto{\pgfqpoint{1.236775in}{1.678725in}}%
\pgfpathlineto{\pgfqpoint{1.237904in}{1.679647in}}%
\pgfpathlineto{\pgfqpoint{1.277408in}{1.700536in}}%
\pgfpathlineto{\pgfqpoint{1.277973in}{1.703897in}}%
\pgfpathlineto{\pgfqpoint{1.280794in}{1.735182in}}%
\pgfpathlineto{\pgfqpoint{1.283616in}{1.736067in}}%
\pgfpathlineto{\pgfqpoint{1.399307in}{1.769692in}}%
\pgfpathlineto{\pgfqpoint{1.401564in}{1.772487in}}%
\pgfpathlineto{\pgfqpoint{1.425267in}{1.803118in}}%
\pgfpathlineto{\pgfqpoint{1.425831in}{1.806293in}}%
\pgfpathlineto{\pgfqpoint{1.426396in}{1.884083in}}%
\pgfpathlineto{\pgfqpoint{1.429217in}{1.946687in}}%
\pgfpathlineto{\pgfqpoint{1.429782in}{1.950848in}}%
\pgfpathlineto{\pgfqpoint{1.574818in}{1.972261in}}%
\pgfpathlineto{\pgfqpoint{1.575383in}{1.974672in}}%
\pgfpathlineto{\pgfqpoint{1.575947in}{2.004518in}}%
\pgfpathlineto{\pgfqpoint{1.576512in}{2.016685in}}%
\pgfpathlineto{\pgfqpoint{1.603600in}{2.070934in}}%
\pgfpathlineto{\pgfqpoint{1.604164in}{2.079627in}}%
\pgfpathlineto{\pgfqpoint{1.606422in}{2.150589in}}%
\pgfpathlineto{\pgfqpoint{1.606986in}{2.241456in}}%
\pgfpathlineto{\pgfqpoint{1.616016in}{2.294952in}}%
\pgfpathlineto{\pgfqpoint{1.616580in}{2.295786in}}%
\pgfpathlineto{\pgfqpoint{1.916812in}{2.318022in}}%
\pgfpathlineto{\pgfqpoint{2.062413in}{2.325912in}}%
\pgfpathlineto{\pgfqpoint{2.062978in}{2.326930in}}%
\pgfpathlineto{\pgfqpoint{2.065799in}{2.348604in}}%
\pgfpathlineto{\pgfqpoint{2.067492in}{2.381668in}}%
\pgfpathlineto{\pgfqpoint{2.068621in}{2.382732in}}%
\pgfpathlineto{\pgfqpoint{2.088937in}{2.394607in}}%
\pgfpathlineto{\pgfqpoint{2.089502in}{2.395322in}}%
\pgfpathlineto{\pgfqpoint{2.092323in}{2.410630in}}%
\pgfpathlineto{\pgfqpoint{2.093452in}{2.416346in}}%
\pgfpathlineto{\pgfqpoint{2.095145in}{2.417716in}}%
\pgfpathlineto{\pgfqpoint{2.173589in}{2.472293in}}%
\pgfpathlineto{\pgfqpoint{2.174154in}{2.526668in}}%
\pgfpathlineto{\pgfqpoint{2.174718in}{2.538787in}}%
\pgfpathlineto{\pgfqpoint{2.186569in}{2.596881in}}%
\pgfpathlineto{\pgfqpoint{2.187134in}{2.597894in}}%
\pgfpathlineto{\pgfqpoint{2.243568in}{2.636421in}}%
\pgfpathlineto{\pgfqpoint{2.410614in}{2.650496in}}%
\pgfpathlineto{\pgfqpoint{2.550572in}{2.662381in}}%
\pgfpathlineto{\pgfqpoint{2.570324in}{2.675572in}}%
\pgfpathlineto{\pgfqpoint{2.608135in}{2.700841in}}%
\pgfpathlineto{\pgfqpoint{2.608700in}{2.763401in}}%
\pgfpathlineto{\pgfqpoint{2.609264in}{2.774439in}}%
\pgfpathlineto{\pgfqpoint{2.920219in}{2.796221in}}%
\pgfpathlineto{\pgfqpoint{2.949000in}{2.819434in}}%
\pgfpathlineto{\pgfqpoint{2.983990in}{2.852007in}}%
\pgfpathlineto{\pgfqpoint{3.165145in}{2.855044in}}%
\pgfpathlineto{\pgfqpoint{3.517861in}{2.861229in}}%
\pgfpathlineto{\pgfqpoint{3.923061in}{2.873982in}}%
\pgfpathlineto{\pgfqpoint{4.311331in}{2.908274in}}%
\pgfpathlineto{\pgfqpoint{4.315281in}{2.916429in}}%
\pgfpathlineto{\pgfqpoint{4.327697in}{2.942275in}}%
\pgfpathlineto{\pgfqpoint{4.328826in}{2.942597in}}%
\pgfpathlineto{\pgfqpoint{5.199047in}{2.977093in}}%
\pgfpathlineto{\pgfqpoint{5.992517in}{2.981362in}}%
\pgfpathlineto{\pgfqpoint{5.993081in}{2.981813in}}%
\pgfpathlineto{\pgfqpoint{6.004368in}{3.002609in}}%
\pgfpathlineto{\pgfqpoint{6.004368in}{3.002609in}}%
\pgfusepath{stroke}%
\end{pgfscope}%
\begin{pgfscope}%
\pgfpathrectangle{\pgfqpoint{0.481681in}{1.080890in}}{\pgfqpoint{5.785672in}{2.146863in}}%
\pgfusepath{clip}%
\pgfsetrectcap%
\pgfsetroundjoin%
\pgfsetlinewidth{0.200750pt}%
\definecolor{currentstroke}{rgb}{0.682353,0.125490,0.070588}%
\pgfsetstrokecolor{currentstroke}%
\pgfsetdash{}{0pt}%
\pgfpathmoveto{\pgfqpoint{0.744666in}{1.178475in}}%
\pgfpathlineto{\pgfqpoint{0.798279in}{1.178475in}}%
\pgfpathlineto{\pgfqpoint{0.798843in}{1.180346in}}%
\pgfpathlineto{\pgfqpoint{0.799407in}{1.191224in}}%
\pgfpathlineto{\pgfqpoint{0.799972in}{1.188664in}}%
\pgfpathlineto{\pgfqpoint{0.800536in}{1.183132in}}%
\pgfpathlineto{\pgfqpoint{0.801101in}{1.182666in}}%
\pgfpathlineto{\pgfqpoint{0.801665in}{1.191157in}}%
\pgfpathlineto{\pgfqpoint{0.802229in}{1.182894in}}%
\pgfpathlineto{\pgfqpoint{0.802794in}{1.193696in}}%
\pgfpathlineto{\pgfqpoint{0.803922in}{1.195411in}}%
\pgfpathlineto{\pgfqpoint{0.813516in}{1.195633in}}%
\pgfpathlineto{\pgfqpoint{0.814645in}{1.195792in}}%
\pgfpathlineto{\pgfqpoint{0.815209in}{1.199882in}}%
\pgfpathlineto{\pgfqpoint{0.815773in}{1.200408in}}%
\pgfpathlineto{\pgfqpoint{0.816902in}{1.195738in}}%
\pgfpathlineto{\pgfqpoint{0.817467in}{1.195411in}}%
\pgfpathlineto{\pgfqpoint{0.825932in}{1.195411in}}%
\pgfpathlineto{\pgfqpoint{0.826496in}{1.196290in}}%
\pgfpathlineto{\pgfqpoint{0.827060in}{1.200073in}}%
\pgfpathlineto{\pgfqpoint{0.827625in}{1.208273in}}%
\pgfpathlineto{\pgfqpoint{0.828189in}{1.203929in}}%
\pgfpathlineto{\pgfqpoint{0.829318in}{1.213698in}}%
\pgfpathlineto{\pgfqpoint{0.829882in}{1.213480in}}%
\pgfpathlineto{\pgfqpoint{0.830446in}{1.349759in}}%
\pgfpathlineto{\pgfqpoint{0.831011in}{2.315302in}}%
\pgfpathlineto{\pgfqpoint{0.832704in}{2.123020in}}%
\pgfpathlineto{\pgfqpoint{0.840040in}{1.245617in}}%
\pgfpathlineto{\pgfqpoint{0.841169in}{1.989108in}}%
\pgfpathlineto{\pgfqpoint{0.841733in}{1.254043in}}%
\pgfpathlineto{\pgfqpoint{0.842298in}{2.091958in}}%
\pgfpathlineto{\pgfqpoint{0.842862in}{2.175489in}}%
\pgfpathlineto{\pgfqpoint{0.843426in}{2.376735in}}%
\pgfpathlineto{\pgfqpoint{0.843991in}{2.383255in}}%
\pgfpathlineto{\pgfqpoint{0.847377in}{2.380447in}}%
\pgfpathlineto{\pgfqpoint{0.853020in}{2.375653in}}%
\pgfpathlineto{\pgfqpoint{0.854149in}{2.372125in}}%
\pgfpathlineto{\pgfqpoint{0.854713in}{2.373182in}}%
\pgfpathlineto{\pgfqpoint{0.855278in}{2.383653in}}%
\pgfpathlineto{\pgfqpoint{0.883495in}{2.384050in}}%
\pgfpathlineto{\pgfqpoint{0.895911in}{2.384073in}}%
\pgfpathlineto{\pgfqpoint{0.897039in}{2.383809in}}%
\pgfpathlineto{\pgfqpoint{0.897604in}{2.383859in}}%
\pgfpathlineto{\pgfqpoint{0.898168in}{2.393195in}}%
\pgfpathlineto{\pgfqpoint{0.899297in}{2.393533in}}%
\pgfpathlineto{\pgfqpoint{0.907197in}{2.393878in}}%
\pgfpathlineto{\pgfqpoint{0.907762in}{2.393147in}}%
\pgfpathlineto{\pgfqpoint{0.908891in}{2.385788in}}%
\pgfpathlineto{\pgfqpoint{0.910584in}{2.396522in}}%
\pgfpathlineto{\pgfqpoint{0.911148in}{2.387713in}}%
\pgfpathlineto{\pgfqpoint{0.911712in}{2.392800in}}%
\pgfpathlineto{\pgfqpoint{0.912277in}{2.394191in}}%
\pgfpathlineto{\pgfqpoint{0.921870in}{2.394319in}}%
\pgfpathlineto{\pgfqpoint{0.922999in}{2.411922in}}%
\pgfpathlineto{\pgfqpoint{0.923564in}{2.437233in}}%
\pgfpathlineto{\pgfqpoint{0.924128in}{2.444201in}}%
\pgfpathlineto{\pgfqpoint{0.924692in}{2.445201in}}%
\pgfpathlineto{\pgfqpoint{0.925257in}{2.447175in}}%
\pgfpathlineto{\pgfqpoint{0.926950in}{2.447271in}}%
\pgfpathlineto{\pgfqpoint{0.933722in}{2.447271in}}%
\pgfpathlineto{\pgfqpoint{0.934286in}{2.447696in}}%
\pgfpathlineto{\pgfqpoint{0.934850in}{2.474456in}}%
\pgfpathlineto{\pgfqpoint{0.935415in}{2.481393in}}%
\pgfpathlineto{\pgfqpoint{0.935979in}{2.481428in}}%
\pgfpathlineto{\pgfqpoint{0.936543in}{2.415523in}}%
\pgfpathlineto{\pgfqpoint{0.938801in}{1.389136in}}%
\pgfpathlineto{\pgfqpoint{0.939365in}{1.374584in}}%
\pgfpathlineto{\pgfqpoint{0.947266in}{1.374460in}}%
\pgfpathlineto{\pgfqpoint{0.947830in}{1.374781in}}%
\pgfpathlineto{\pgfqpoint{0.949523in}{1.382095in}}%
\pgfpathlineto{\pgfqpoint{0.950088in}{1.381061in}}%
\pgfpathlineto{\pgfqpoint{0.950652in}{1.376658in}}%
\pgfpathlineto{\pgfqpoint{0.951216in}{1.377865in}}%
\pgfpathlineto{\pgfqpoint{0.990721in}{1.377587in}}%
\pgfpathlineto{\pgfqpoint{0.991285in}{1.379647in}}%
\pgfpathlineto{\pgfqpoint{0.991849in}{1.385261in}}%
\pgfpathlineto{\pgfqpoint{0.992414in}{1.386254in}}%
\pgfpathlineto{\pgfqpoint{0.992978in}{1.386678in}}%
\pgfpathlineto{\pgfqpoint{0.993542in}{1.390720in}}%
\pgfpathlineto{\pgfqpoint{0.994671in}{1.391573in}}%
\pgfpathlineto{\pgfqpoint{1.002572in}{1.388395in}}%
\pgfpathlineto{\pgfqpoint{1.003136in}{1.386651in}}%
\pgfpathlineto{\pgfqpoint{1.003701in}{1.392726in}}%
\pgfpathlineto{\pgfqpoint{1.004265in}{1.388788in}}%
\pgfpathlineto{\pgfqpoint{1.004829in}{1.400747in}}%
\pgfpathlineto{\pgfqpoint{1.005958in}{1.407441in}}%
\pgfpathlineto{\pgfqpoint{1.006522in}{1.407138in}}%
\pgfpathlineto{\pgfqpoint{1.007087in}{1.409008in}}%
\pgfpathlineto{\pgfqpoint{1.007651in}{1.405191in}}%
\pgfpathlineto{\pgfqpoint{1.015552in}{1.425262in}}%
\pgfpathlineto{\pgfqpoint{1.016681in}{1.410381in}}%
\pgfpathlineto{\pgfqpoint{1.017809in}{1.471148in}}%
\pgfpathlineto{\pgfqpoint{1.018374in}{1.472832in}}%
\pgfpathlineto{\pgfqpoint{1.018938in}{1.473201in}}%
\pgfpathlineto{\pgfqpoint{1.019502in}{1.480500in}}%
\pgfpathlineto{\pgfqpoint{1.020067in}{1.493926in}}%
\pgfpathlineto{\pgfqpoint{1.020631in}{1.457251in}}%
\pgfpathlineto{\pgfqpoint{1.021195in}{1.496259in}}%
\pgfpathlineto{\pgfqpoint{1.042640in}{1.531703in}}%
\pgfpathlineto{\pgfqpoint{1.043205in}{1.471579in}}%
\pgfpathlineto{\pgfqpoint{1.043769in}{1.494875in}}%
\pgfpathlineto{\pgfqpoint{1.044334in}{1.466826in}}%
\pgfpathlineto{\pgfqpoint{1.044898in}{1.517780in}}%
\pgfpathlineto{\pgfqpoint{1.045462in}{1.514333in}}%
\pgfpathlineto{\pgfqpoint{1.046027in}{1.533871in}}%
\pgfpathlineto{\pgfqpoint{1.046591in}{1.509629in}}%
\pgfpathlineto{\pgfqpoint{1.047155in}{1.537952in}}%
\pgfpathlineto{\pgfqpoint{1.047720in}{1.546020in}}%
\pgfpathlineto{\pgfqpoint{1.085531in}{1.527552in}}%
\pgfpathlineto{\pgfqpoint{1.086659in}{1.528851in}}%
\pgfpathlineto{\pgfqpoint{1.090046in}{1.534186in}}%
\pgfpathlineto{\pgfqpoint{1.097382in}{1.547512in}}%
\pgfpathlineto{\pgfqpoint{1.097946in}{1.534633in}}%
\pgfpathlineto{\pgfqpoint{1.099075in}{1.548514in}}%
\pgfpathlineto{\pgfqpoint{1.099639in}{1.565209in}}%
\pgfpathlineto{\pgfqpoint{1.100204in}{1.572330in}}%
\pgfpathlineto{\pgfqpoint{1.100768in}{1.584307in}}%
\pgfpathlineto{\pgfqpoint{1.101332in}{1.540386in}}%
\pgfpathlineto{\pgfqpoint{1.101897in}{1.550301in}}%
\pgfpathlineto{\pgfqpoint{1.102461in}{1.545995in}}%
\pgfpathlineto{\pgfqpoint{1.108669in}{1.534427in}}%
\pgfpathlineto{\pgfqpoint{1.110362in}{1.531256in}}%
\pgfpathlineto{\pgfqpoint{1.110926in}{1.546633in}}%
\pgfpathlineto{\pgfqpoint{1.111491in}{1.588003in}}%
\pgfpathlineto{\pgfqpoint{1.112055in}{1.562871in}}%
\pgfpathlineto{\pgfqpoint{1.112619in}{1.579576in}}%
\pgfpathlineto{\pgfqpoint{1.113184in}{1.588035in}}%
\pgfpathlineto{\pgfqpoint{1.113748in}{1.602670in}}%
\pgfpathlineto{\pgfqpoint{1.114312in}{1.599837in}}%
\pgfpathlineto{\pgfqpoint{1.114877in}{1.546516in}}%
\pgfpathlineto{\pgfqpoint{1.115441in}{1.570863in}}%
\pgfpathlineto{\pgfqpoint{1.116005in}{1.552517in}}%
\pgfpathlineto{\pgfqpoint{1.116570in}{1.551220in}}%
\pgfpathlineto{\pgfqpoint{1.123906in}{1.549851in}}%
\pgfpathlineto{\pgfqpoint{1.124471in}{1.566379in}}%
\pgfpathlineto{\pgfqpoint{1.125035in}{1.608497in}}%
\pgfpathlineto{\pgfqpoint{1.125599in}{1.621612in}}%
\pgfpathlineto{\pgfqpoint{1.126164in}{1.580734in}}%
\pgfpathlineto{\pgfqpoint{1.126728in}{1.585427in}}%
\pgfpathlineto{\pgfqpoint{1.127292in}{1.563423in}}%
\pgfpathlineto{\pgfqpoint{1.127857in}{1.565589in}}%
\pgfpathlineto{\pgfqpoint{1.128421in}{1.632000in}}%
\pgfpathlineto{\pgfqpoint{1.128985in}{1.594954in}}%
\pgfpathlineto{\pgfqpoint{1.129550in}{1.706584in}}%
\pgfpathlineto{\pgfqpoint{1.130114in}{1.715753in}}%
\pgfpathlineto{\pgfqpoint{1.138015in}{1.654458in}}%
\pgfpathlineto{\pgfqpoint{1.138579in}{1.641217in}}%
\pgfpathlineto{\pgfqpoint{1.139144in}{1.635945in}}%
\pgfpathlineto{\pgfqpoint{1.140272in}{1.639135in}}%
\pgfpathlineto{\pgfqpoint{1.141401in}{1.722424in}}%
\pgfpathlineto{\pgfqpoint{1.141965in}{1.638093in}}%
\pgfpathlineto{\pgfqpoint{1.144223in}{1.642295in}}%
\pgfpathlineto{\pgfqpoint{1.179212in}{1.709802in}}%
\pgfpathlineto{\pgfqpoint{1.179776in}{1.707139in}}%
\pgfpathlineto{\pgfqpoint{1.180341in}{1.681194in}}%
\pgfpathlineto{\pgfqpoint{1.180905in}{1.669866in}}%
\pgfpathlineto{\pgfqpoint{1.181470in}{1.733813in}}%
\pgfpathlineto{\pgfqpoint{1.182034in}{1.724256in}}%
\pgfpathlineto{\pgfqpoint{1.182598in}{1.737938in}}%
\pgfpathlineto{\pgfqpoint{1.183163in}{1.740350in}}%
\pgfpathlineto{\pgfqpoint{1.183727in}{1.733138in}}%
\pgfpathlineto{\pgfqpoint{1.184291in}{1.754022in}}%
\pgfpathlineto{\pgfqpoint{1.192756in}{1.757977in}}%
\pgfpathlineto{\pgfqpoint{1.193321in}{1.740707in}}%
\pgfpathlineto{\pgfqpoint{1.193885in}{1.750350in}}%
\pgfpathlineto{\pgfqpoint{1.194449in}{1.766505in}}%
\pgfpathlineto{\pgfqpoint{1.195014in}{1.771794in}}%
\pgfpathlineto{\pgfqpoint{1.196707in}{1.709537in}}%
\pgfpathlineto{\pgfqpoint{1.197271in}{1.769370in}}%
\pgfpathlineto{\pgfqpoint{1.197836in}{1.773379in}}%
\pgfpathlineto{\pgfqpoint{1.205736in}{1.773379in}}%
\pgfpathlineto{\pgfqpoint{1.206301in}{1.716166in}}%
\pgfpathlineto{\pgfqpoint{1.206865in}{1.688896in}}%
\pgfpathlineto{\pgfqpoint{1.207429in}{1.765077in}}%
\pgfpathlineto{\pgfqpoint{1.207994in}{1.768202in}}%
\pgfpathlineto{\pgfqpoint{1.208558in}{1.777933in}}%
\pgfpathlineto{\pgfqpoint{1.209122in}{1.779505in}}%
\pgfpathlineto{\pgfqpoint{1.209687in}{1.817479in}}%
\pgfpathlineto{\pgfqpoint{1.210251in}{1.802026in}}%
\pgfpathlineto{\pgfqpoint{1.210815in}{1.851694in}}%
\pgfpathlineto{\pgfqpoint{1.211380in}{1.852794in}}%
\pgfpathlineto{\pgfqpoint{1.218716in}{1.852794in}}%
\pgfpathlineto{\pgfqpoint{1.219281in}{1.852447in}}%
\pgfpathlineto{\pgfqpoint{1.219845in}{1.849170in}}%
\pgfpathlineto{\pgfqpoint{1.220409in}{1.847933in}}%
\pgfpathlineto{\pgfqpoint{1.220974in}{1.849912in}}%
\pgfpathlineto{\pgfqpoint{1.221538in}{1.853275in}}%
\pgfpathlineto{\pgfqpoint{1.222667in}{1.876024in}}%
\pgfpathlineto{\pgfqpoint{1.223231in}{1.982358in}}%
\pgfpathlineto{\pgfqpoint{1.223795in}{1.902472in}}%
\pgfpathlineto{\pgfqpoint{1.224360in}{1.904241in}}%
\pgfpathlineto{\pgfqpoint{1.227182in}{1.955829in}}%
\pgfpathlineto{\pgfqpoint{1.233389in}{2.069907in}}%
\pgfpathlineto{\pgfqpoint{1.234518in}{2.075243in}}%
\pgfpathlineto{\pgfqpoint{1.235082in}{2.076138in}}%
\pgfpathlineto{\pgfqpoint{1.235647in}{2.080019in}}%
\pgfpathlineto{\pgfqpoint{1.236211in}{2.077422in}}%
\pgfpathlineto{\pgfqpoint{1.272894in}{1.819386in}}%
\pgfpathlineto{\pgfqpoint{1.273458in}{1.840324in}}%
\pgfpathlineto{\pgfqpoint{1.274022in}{1.980603in}}%
\pgfpathlineto{\pgfqpoint{1.274587in}{1.966532in}}%
\pgfpathlineto{\pgfqpoint{1.275151in}{2.056763in}}%
\pgfpathlineto{\pgfqpoint{1.275715in}{2.097234in}}%
\pgfpathlineto{\pgfqpoint{1.276280in}{2.085864in}}%
\pgfpathlineto{\pgfqpoint{1.276844in}{1.839395in}}%
\pgfpathlineto{\pgfqpoint{1.277408in}{1.848841in}}%
\pgfpathlineto{\pgfqpoint{1.277973in}{2.012375in}}%
\pgfpathlineto{\pgfqpoint{1.278537in}{2.001184in}}%
\pgfpathlineto{\pgfqpoint{1.279101in}{2.122451in}}%
\pgfpathlineto{\pgfqpoint{1.287002in}{1.834441in}}%
\pgfpathlineto{\pgfqpoint{1.287567in}{1.846494in}}%
\pgfpathlineto{\pgfqpoint{1.288131in}{1.827295in}}%
\pgfpathlineto{\pgfqpoint{1.288695in}{1.820200in}}%
\pgfpathlineto{\pgfqpoint{1.289260in}{1.964069in}}%
\pgfpathlineto{\pgfqpoint{1.289824in}{2.015504in}}%
\pgfpathlineto{\pgfqpoint{1.290388in}{1.885356in}}%
\pgfpathlineto{\pgfqpoint{1.290953in}{1.882295in}}%
\pgfpathlineto{\pgfqpoint{1.291517in}{2.111669in}}%
\pgfpathlineto{\pgfqpoint{1.292081in}{2.142136in}}%
\pgfpathlineto{\pgfqpoint{1.299982in}{2.096753in}}%
\pgfpathlineto{\pgfqpoint{1.300546in}{2.105669in}}%
\pgfpathlineto{\pgfqpoint{1.301111in}{2.146569in}}%
\pgfpathlineto{\pgfqpoint{1.301675in}{2.064055in}}%
\pgfpathlineto{\pgfqpoint{1.302239in}{2.147151in}}%
\pgfpathlineto{\pgfqpoint{1.302804in}{2.148430in}}%
\pgfpathlineto{\pgfqpoint{1.303368in}{2.148613in}}%
\pgfpathlineto{\pgfqpoint{1.303933in}{2.149272in}}%
\pgfpathlineto{\pgfqpoint{1.304497in}{2.152542in}}%
\pgfpathlineto{\pgfqpoint{1.305061in}{2.150193in}}%
\pgfpathlineto{\pgfqpoint{1.305626in}{1.830368in}}%
\pgfpathlineto{\pgfqpoint{1.307319in}{1.895159in}}%
\pgfpathlineto{\pgfqpoint{1.313526in}{2.147954in}}%
\pgfpathlineto{\pgfqpoint{1.314091in}{2.158608in}}%
\pgfpathlineto{\pgfqpoint{1.314655in}{2.161532in}}%
\pgfpathlineto{\pgfqpoint{1.315219in}{2.172126in}}%
\pgfpathlineto{\pgfqpoint{1.315784in}{2.173958in}}%
\pgfpathlineto{\pgfqpoint{1.316348in}{2.178156in}}%
\pgfpathlineto{\pgfqpoint{1.316912in}{2.173042in}}%
\pgfpathlineto{\pgfqpoint{1.318041in}{2.173135in}}%
\pgfpathlineto{\pgfqpoint{1.318606in}{2.200346in}}%
\pgfpathlineto{\pgfqpoint{1.319170in}{2.207948in}}%
\pgfpathlineto{\pgfqpoint{1.347387in}{1.994146in}}%
\pgfpathlineto{\pgfqpoint{1.368268in}{1.835902in}}%
\pgfpathlineto{\pgfqpoint{1.368832in}{1.873519in}}%
\pgfpathlineto{\pgfqpoint{1.369397in}{2.230285in}}%
\pgfpathlineto{\pgfqpoint{1.369961in}{2.231773in}}%
\pgfpathlineto{\pgfqpoint{1.370525in}{2.229387in}}%
\pgfpathlineto{\pgfqpoint{1.371090in}{2.250634in}}%
\pgfpathlineto{\pgfqpoint{1.371654in}{2.257059in}}%
\pgfpathlineto{\pgfqpoint{1.372783in}{2.287289in}}%
\pgfpathlineto{\pgfqpoint{1.373347in}{2.299329in}}%
\pgfpathlineto{\pgfqpoint{1.373911in}{2.286452in}}%
\pgfpathlineto{\pgfqpoint{1.382377in}{1.847363in}}%
\pgfpathlineto{\pgfqpoint{1.382941in}{1.847991in}}%
\pgfpathlineto{\pgfqpoint{1.383505in}{1.825295in}}%
\pgfpathlineto{\pgfqpoint{1.384070in}{1.813967in}}%
\pgfpathlineto{\pgfqpoint{1.384634in}{1.809120in}}%
\pgfpathlineto{\pgfqpoint{1.385198in}{1.820219in}}%
\pgfpathlineto{\pgfqpoint{1.385763in}{1.824108in}}%
\pgfpathlineto{\pgfqpoint{1.386891in}{1.836014in}}%
\pgfpathlineto{\pgfqpoint{1.394792in}{1.839828in}}%
\pgfpathlineto{\pgfqpoint{1.395357in}{1.838731in}}%
\pgfpathlineto{\pgfqpoint{1.395921in}{1.845769in}}%
\pgfpathlineto{\pgfqpoint{1.396485in}{1.836663in}}%
\pgfpathlineto{\pgfqpoint{1.397050in}{1.851064in}}%
\pgfpathlineto{\pgfqpoint{1.397614in}{1.856495in}}%
\pgfpathlineto{\pgfqpoint{1.398743in}{1.857007in}}%
\pgfpathlineto{\pgfqpoint{1.399307in}{1.846207in}}%
\pgfpathlineto{\pgfqpoint{1.399871in}{1.866934in}}%
\pgfpathlineto{\pgfqpoint{1.410594in}{1.868725in}}%
\pgfpathlineto{\pgfqpoint{1.411158in}{1.869521in}}%
\pgfpathlineto{\pgfqpoint{1.411723in}{1.906224in}}%
\pgfpathlineto{\pgfqpoint{1.412287in}{1.906348in}}%
\pgfpathlineto{\pgfqpoint{1.412851in}{1.901203in}}%
\pgfpathlineto{\pgfqpoint{1.413416in}{1.874668in}}%
\pgfpathlineto{\pgfqpoint{1.413980in}{1.906291in}}%
\pgfpathlineto{\pgfqpoint{1.414544in}{1.908157in}}%
\pgfpathlineto{\pgfqpoint{1.421881in}{1.908157in}}%
\pgfpathlineto{\pgfqpoint{1.423574in}{1.900142in}}%
\pgfpathlineto{\pgfqpoint{1.424138in}{1.867026in}}%
\pgfpathlineto{\pgfqpoint{1.424703in}{1.883294in}}%
\pgfpathlineto{\pgfqpoint{1.425267in}{1.908894in}}%
\pgfpathlineto{\pgfqpoint{1.426396in}{1.910717in}}%
\pgfpathlineto{\pgfqpoint{1.465335in}{1.924679in}}%
\pgfpathlineto{\pgfqpoint{1.465900in}{1.918453in}}%
\pgfpathlineto{\pgfqpoint{1.466464in}{1.921980in}}%
\pgfpathlineto{\pgfqpoint{1.467028in}{1.915136in}}%
\pgfpathlineto{\pgfqpoint{1.468157in}{1.927314in}}%
\pgfpathlineto{\pgfqpoint{1.476622in}{1.925740in}}%
\pgfpathlineto{\pgfqpoint{1.477187in}{1.919396in}}%
\pgfpathlineto{\pgfqpoint{1.477751in}{1.925990in}}%
\pgfpathlineto{\pgfqpoint{1.478880in}{1.926350in}}%
\pgfpathlineto{\pgfqpoint{1.479444in}{1.928443in}}%
\pgfpathlineto{\pgfqpoint{1.480008in}{1.936864in}}%
\pgfpathlineto{\pgfqpoint{1.480573in}{1.927683in}}%
\pgfpathlineto{\pgfqpoint{1.481701in}{1.949355in}}%
\pgfpathlineto{\pgfqpoint{1.483959in}{1.948060in}}%
\pgfpathlineto{\pgfqpoint{1.490167in}{1.944507in}}%
\pgfpathlineto{\pgfqpoint{1.490731in}{1.949299in}}%
\pgfpathlineto{\pgfqpoint{1.491295in}{1.942471in}}%
\pgfpathlineto{\pgfqpoint{1.491860in}{1.946520in}}%
\pgfpathlineto{\pgfqpoint{1.492424in}{1.924034in}}%
\pgfpathlineto{\pgfqpoint{1.492988in}{1.943534in}}%
\pgfpathlineto{\pgfqpoint{1.493553in}{1.936852in}}%
\pgfpathlineto{\pgfqpoint{1.494117in}{1.948300in}}%
\pgfpathlineto{\pgfqpoint{1.495246in}{1.981123in}}%
\pgfpathlineto{\pgfqpoint{1.495810in}{1.992433in}}%
\pgfpathlineto{\pgfqpoint{1.503711in}{1.983545in}}%
\pgfpathlineto{\pgfqpoint{1.504275in}{1.948813in}}%
\pgfpathlineto{\pgfqpoint{1.505404in}{1.994287in}}%
\pgfpathlineto{\pgfqpoint{1.506533in}{1.991665in}}%
\pgfpathlineto{\pgfqpoint{1.507097in}{1.970250in}}%
\pgfpathlineto{\pgfqpoint{1.507661in}{1.966655in}}%
\pgfpathlineto{\pgfqpoint{1.508226in}{2.000108in}}%
\pgfpathlineto{\pgfqpoint{1.517255in}{2.013849in}}%
\pgfpathlineto{\pgfqpoint{1.517820in}{1.995331in}}%
\pgfpathlineto{\pgfqpoint{1.518384in}{2.012092in}}%
\pgfpathlineto{\pgfqpoint{1.518948in}{2.010783in}}%
\pgfpathlineto{\pgfqpoint{1.519513in}{2.010861in}}%
\pgfpathlineto{\pgfqpoint{1.520641in}{1.973135in}}%
\pgfpathlineto{\pgfqpoint{1.521770in}{1.983346in}}%
\pgfpathlineto{\pgfqpoint{1.526849in}{1.978760in}}%
\pgfpathlineto{\pgfqpoint{1.556759in}{1.951514in}}%
\pgfpathlineto{\pgfqpoint{1.557324in}{1.950339in}}%
\pgfpathlineto{\pgfqpoint{1.557888in}{1.943633in}}%
\pgfpathlineto{\pgfqpoint{1.558452in}{1.949871in}}%
\pgfpathlineto{\pgfqpoint{1.559017in}{1.962453in}}%
\pgfpathlineto{\pgfqpoint{1.559581in}{1.982113in}}%
\pgfpathlineto{\pgfqpoint{1.560710in}{1.979404in}}%
\pgfpathlineto{\pgfqpoint{1.561274in}{1.970728in}}%
\pgfpathlineto{\pgfqpoint{1.561839in}{1.949072in}}%
\pgfpathlineto{\pgfqpoint{1.562403in}{2.004729in}}%
\pgfpathlineto{\pgfqpoint{1.565789in}{2.005083in}}%
\pgfpathlineto{\pgfqpoint{1.571432in}{2.005765in}}%
\pgfpathlineto{\pgfqpoint{1.571997in}{2.006741in}}%
\pgfpathlineto{\pgfqpoint{1.572561in}{2.008396in}}%
\pgfpathlineto{\pgfqpoint{1.573690in}{1.971562in}}%
\pgfpathlineto{\pgfqpoint{1.574254in}{1.993105in}}%
\pgfpathlineto{\pgfqpoint{1.574818in}{1.994132in}}%
\pgfpathlineto{\pgfqpoint{1.575383in}{2.014706in}}%
\pgfpathlineto{\pgfqpoint{1.575947in}{2.017017in}}%
\pgfpathlineto{\pgfqpoint{1.576512in}{2.023226in}}%
\pgfpathlineto{\pgfqpoint{1.577076in}{2.024603in}}%
\pgfpathlineto{\pgfqpoint{1.584977in}{2.022950in}}%
\pgfpathlineto{\pgfqpoint{1.585541in}{1.981759in}}%
\pgfpathlineto{\pgfqpoint{1.586105in}{2.026975in}}%
\pgfpathlineto{\pgfqpoint{1.586670in}{2.024029in}}%
\pgfpathlineto{\pgfqpoint{1.587234in}{1.995556in}}%
\pgfpathlineto{\pgfqpoint{1.587798in}{1.982975in}}%
\pgfpathlineto{\pgfqpoint{1.588363in}{2.035141in}}%
\pgfpathlineto{\pgfqpoint{1.588927in}{2.017803in}}%
\pgfpathlineto{\pgfqpoint{1.589491in}{2.030316in}}%
\pgfpathlineto{\pgfqpoint{1.590056in}{2.051517in}}%
\pgfpathlineto{\pgfqpoint{1.598521in}{1.931428in}}%
\pgfpathlineto{\pgfqpoint{1.599085in}{2.023869in}}%
\pgfpathlineto{\pgfqpoint{1.599650in}{1.996669in}}%
\pgfpathlineto{\pgfqpoint{1.600778in}{2.025031in}}%
\pgfpathlineto{\pgfqpoint{1.601343in}{2.085086in}}%
\pgfpathlineto{\pgfqpoint{1.601907in}{2.084698in}}%
\pgfpathlineto{\pgfqpoint{1.602471in}{2.119592in}}%
\pgfpathlineto{\pgfqpoint{1.603036in}{2.187898in}}%
\pgfpathlineto{\pgfqpoint{1.603600in}{2.212014in}}%
\pgfpathlineto{\pgfqpoint{1.612065in}{2.212125in}}%
\pgfpathlineto{\pgfqpoint{1.613194in}{2.212298in}}%
\pgfpathlineto{\pgfqpoint{1.695588in}{2.212508in}}%
\pgfpathlineto{\pgfqpoint{1.696717in}{2.212857in}}%
\pgfpathlineto{\pgfqpoint{1.899882in}{2.207360in}}%
\pgfpathlineto{\pgfqpoint{1.936564in}{2.214218in}}%
\pgfpathlineto{\pgfqpoint{1.937128in}{2.212468in}}%
\pgfpathlineto{\pgfqpoint{1.939950in}{2.214199in}}%
\pgfpathlineto{\pgfqpoint{1.940514in}{2.214180in}}%
\pgfpathlineto{\pgfqpoint{1.941643in}{2.209879in}}%
\pgfpathlineto{\pgfqpoint{1.942772in}{2.212056in}}%
\pgfpathlineto{\pgfqpoint{1.950673in}{2.210504in}}%
\pgfpathlineto{\pgfqpoint{1.951801in}{2.212438in}}%
\pgfpathlineto{\pgfqpoint{1.952366in}{2.214403in}}%
\pgfpathlineto{\pgfqpoint{1.952930in}{2.214630in}}%
\pgfpathlineto{\pgfqpoint{1.953494in}{2.211106in}}%
\pgfpathlineto{\pgfqpoint{1.954059in}{2.219568in}}%
\pgfpathlineto{\pgfqpoint{1.954623in}{2.213258in}}%
\pgfpathlineto{\pgfqpoint{1.955187in}{2.214914in}}%
\pgfpathlineto{\pgfqpoint{1.955752in}{2.218339in}}%
\pgfpathlineto{\pgfqpoint{1.956881in}{2.210261in}}%
\pgfpathlineto{\pgfqpoint{1.958574in}{2.212436in}}%
\pgfpathlineto{\pgfqpoint{1.963653in}{2.219278in}}%
\pgfpathlineto{\pgfqpoint{1.964781in}{2.210872in}}%
\pgfpathlineto{\pgfqpoint{1.965346in}{2.222776in}}%
\pgfpathlineto{\pgfqpoint{1.965910in}{2.221150in}}%
\pgfpathlineto{\pgfqpoint{1.966474in}{2.218184in}}%
\pgfpathlineto{\pgfqpoint{1.967039in}{2.219784in}}%
\pgfpathlineto{\pgfqpoint{1.967603in}{2.212016in}}%
\pgfpathlineto{\pgfqpoint{1.968167in}{2.231030in}}%
\pgfpathlineto{\pgfqpoint{1.968732in}{2.215395in}}%
\pgfpathlineto{\pgfqpoint{1.969296in}{2.214641in}}%
\pgfpathlineto{\pgfqpoint{1.969860in}{2.214908in}}%
\pgfpathlineto{\pgfqpoint{1.977761in}{2.213344in}}%
\pgfpathlineto{\pgfqpoint{1.979454in}{2.268182in}}%
\pgfpathlineto{\pgfqpoint{1.980019in}{2.267649in}}%
\pgfpathlineto{\pgfqpoint{1.980583in}{2.270332in}}%
\pgfpathlineto{\pgfqpoint{1.981147in}{2.266143in}}%
\pgfpathlineto{\pgfqpoint{1.981712in}{2.232839in}}%
\pgfpathlineto{\pgfqpoint{1.982276in}{2.238562in}}%
\pgfpathlineto{\pgfqpoint{1.982840in}{2.214558in}}%
\pgfpathlineto{\pgfqpoint{1.983405in}{2.218677in}}%
\pgfpathlineto{\pgfqpoint{1.991306in}{2.212940in}}%
\pgfpathlineto{\pgfqpoint{1.991870in}{2.220927in}}%
\pgfpathlineto{\pgfqpoint{1.992999in}{2.278839in}}%
\pgfpathlineto{\pgfqpoint{1.993563in}{2.264737in}}%
\pgfpathlineto{\pgfqpoint{1.994127in}{2.263282in}}%
\pgfpathlineto{\pgfqpoint{1.994692in}{2.255748in}}%
\pgfpathlineto{\pgfqpoint{1.995256in}{2.241743in}}%
\pgfpathlineto{\pgfqpoint{2.030810in}{2.240497in}}%
\pgfpathlineto{\pgfqpoint{2.031374in}{2.256268in}}%
\pgfpathlineto{\pgfqpoint{2.031938in}{2.282459in}}%
\pgfpathlineto{\pgfqpoint{2.032503in}{2.286079in}}%
\pgfpathlineto{\pgfqpoint{2.033067in}{2.295317in}}%
\pgfpathlineto{\pgfqpoint{2.033632in}{2.239918in}}%
\pgfpathlineto{\pgfqpoint{2.034196in}{2.260675in}}%
\pgfpathlineto{\pgfqpoint{2.034760in}{2.252735in}}%
\pgfpathlineto{\pgfqpoint{2.035325in}{2.253338in}}%
\pgfpathlineto{\pgfqpoint{2.035889in}{2.257888in}}%
\pgfpathlineto{\pgfqpoint{2.036453in}{2.246041in}}%
\pgfpathlineto{\pgfqpoint{2.037018in}{2.247462in}}%
\pgfpathlineto{\pgfqpoint{2.044918in}{2.311239in}}%
\pgfpathlineto{\pgfqpoint{2.046047in}{2.265326in}}%
\pgfpathlineto{\pgfqpoint{2.047176in}{2.225950in}}%
\pgfpathlineto{\pgfqpoint{2.047740in}{2.262078in}}%
\pgfpathlineto{\pgfqpoint{2.048305in}{2.215264in}}%
\pgfpathlineto{\pgfqpoint{2.048869in}{2.242972in}}%
\pgfpathlineto{\pgfqpoint{2.049998in}{2.328687in}}%
\pgfpathlineto{\pgfqpoint{2.050562in}{2.330400in}}%
\pgfpathlineto{\pgfqpoint{2.057898in}{2.330159in}}%
\pgfpathlineto{\pgfqpoint{2.058463in}{2.299926in}}%
\pgfpathlineto{\pgfqpoint{2.059027in}{2.244675in}}%
\pgfpathlineto{\pgfqpoint{2.059591in}{2.240283in}}%
\pgfpathlineto{\pgfqpoint{2.060156in}{2.282856in}}%
\pgfpathlineto{\pgfqpoint{2.060720in}{2.277595in}}%
\pgfpathlineto{\pgfqpoint{2.061284in}{2.300289in}}%
\pgfpathlineto{\pgfqpoint{2.061849in}{2.272122in}}%
\pgfpathlineto{\pgfqpoint{2.062413in}{2.356496in}}%
\pgfpathlineto{\pgfqpoint{2.062978in}{2.256441in}}%
\pgfpathlineto{\pgfqpoint{2.063542in}{2.273139in}}%
\pgfpathlineto{\pgfqpoint{2.064106in}{2.398362in}}%
\pgfpathlineto{\pgfqpoint{2.064671in}{2.438344in}}%
\pgfpathlineto{\pgfqpoint{2.072007in}{2.435472in}}%
\pgfpathlineto{\pgfqpoint{2.072571in}{2.417976in}}%
\pgfpathlineto{\pgfqpoint{2.073136in}{2.301372in}}%
\pgfpathlineto{\pgfqpoint{2.073700in}{2.336231in}}%
\pgfpathlineto{\pgfqpoint{2.074264in}{2.420599in}}%
\pgfpathlineto{\pgfqpoint{2.074829in}{2.439078in}}%
\pgfpathlineto{\pgfqpoint{2.075957in}{2.270688in}}%
\pgfpathlineto{\pgfqpoint{2.077086in}{2.339774in}}%
\pgfpathlineto{\pgfqpoint{2.077651in}{2.336691in}}%
\pgfpathlineto{\pgfqpoint{2.085551in}{2.330868in}}%
\pgfpathlineto{\pgfqpoint{2.086116in}{2.375963in}}%
\pgfpathlineto{\pgfqpoint{2.086680in}{2.376212in}}%
\pgfpathlineto{\pgfqpoint{2.087244in}{2.385311in}}%
\pgfpathlineto{\pgfqpoint{2.088373in}{2.456782in}}%
\pgfpathlineto{\pgfqpoint{2.088937in}{2.466953in}}%
\pgfpathlineto{\pgfqpoint{2.089502in}{2.467675in}}%
\pgfpathlineto{\pgfqpoint{2.090066in}{2.486217in}}%
\pgfpathlineto{\pgfqpoint{2.090630in}{2.476868in}}%
\pgfpathlineto{\pgfqpoint{2.091195in}{2.476516in}}%
\pgfpathlineto{\pgfqpoint{2.127313in}{2.496975in}}%
\pgfpathlineto{\pgfqpoint{2.127877in}{2.494733in}}%
\pgfpathlineto{\pgfqpoint{2.129006in}{2.487468in}}%
\pgfpathlineto{\pgfqpoint{2.129570in}{2.496272in}}%
\pgfpathlineto{\pgfqpoint{2.130135in}{2.495221in}}%
\pgfpathlineto{\pgfqpoint{2.130699in}{2.522694in}}%
\pgfpathlineto{\pgfqpoint{2.131263in}{2.498512in}}%
\pgfpathlineto{\pgfqpoint{2.131828in}{2.494094in}}%
\pgfpathlineto{\pgfqpoint{2.139729in}{2.496097in}}%
\pgfpathlineto{\pgfqpoint{2.140293in}{2.497418in}}%
\pgfpathlineto{\pgfqpoint{2.140857in}{2.504042in}}%
\pgfpathlineto{\pgfqpoint{2.141422in}{2.415765in}}%
\pgfpathlineto{\pgfqpoint{2.143115in}{2.545666in}}%
\pgfpathlineto{\pgfqpoint{2.143679in}{2.556780in}}%
\pgfpathlineto{\pgfqpoint{2.144243in}{2.594724in}}%
\pgfpathlineto{\pgfqpoint{2.145936in}{2.595716in}}%
\pgfpathlineto{\pgfqpoint{2.153837in}{2.601197in}}%
\pgfpathlineto{\pgfqpoint{2.154402in}{2.608539in}}%
\pgfpathlineto{\pgfqpoint{2.154966in}{2.646131in}}%
\pgfpathlineto{\pgfqpoint{2.155530in}{2.643832in}}%
\pgfpathlineto{\pgfqpoint{2.156095in}{2.676698in}}%
\pgfpathlineto{\pgfqpoint{2.156659in}{2.687486in}}%
\pgfpathlineto{\pgfqpoint{2.157223in}{2.715181in}}%
\pgfpathlineto{\pgfqpoint{2.157788in}{2.722061in}}%
\pgfpathlineto{\pgfqpoint{2.162302in}{2.717279in}}%
\pgfpathlineto{\pgfqpoint{2.166817in}{2.712554in}}%
\pgfpathlineto{\pgfqpoint{2.167381in}{2.712407in}}%
\pgfpathlineto{\pgfqpoint{2.167946in}{2.713299in}}%
\pgfpathlineto{\pgfqpoint{2.168510in}{2.721348in}}%
\pgfpathlineto{\pgfqpoint{2.169075in}{2.715980in}}%
\pgfpathlineto{\pgfqpoint{2.169639in}{2.718887in}}%
\pgfpathlineto{\pgfqpoint{2.170203in}{2.724237in}}%
\pgfpathlineto{\pgfqpoint{2.170768in}{2.736041in}}%
\pgfpathlineto{\pgfqpoint{2.172461in}{2.736250in}}%
\pgfpathlineto{\pgfqpoint{2.180361in}{2.737798in}}%
\pgfpathlineto{\pgfqpoint{2.181490in}{2.727886in}}%
\pgfpathlineto{\pgfqpoint{2.182054in}{2.733514in}}%
\pgfpathlineto{\pgfqpoint{2.182619in}{2.736277in}}%
\pgfpathlineto{\pgfqpoint{2.183183in}{2.744353in}}%
\pgfpathlineto{\pgfqpoint{2.183747in}{2.746292in}}%
\pgfpathlineto{\pgfqpoint{2.223816in}{2.754431in}}%
\pgfpathlineto{\pgfqpoint{2.224380in}{2.749093in}}%
\pgfpathlineto{\pgfqpoint{2.224945in}{2.739411in}}%
\pgfpathlineto{\pgfqpoint{2.226073in}{2.743517in}}%
\pgfpathlineto{\pgfqpoint{2.228331in}{2.744251in}}%
\pgfpathlineto{\pgfqpoint{2.235667in}{2.746321in}}%
\pgfpathlineto{\pgfqpoint{2.236232in}{2.747410in}}%
\pgfpathlineto{\pgfqpoint{2.236796in}{2.726943in}}%
\pgfpathlineto{\pgfqpoint{2.238489in}{2.746617in}}%
\pgfpathlineto{\pgfqpoint{2.239053in}{2.755822in}}%
\pgfpathlineto{\pgfqpoint{2.239618in}{2.757478in}}%
\pgfpathlineto{\pgfqpoint{2.250905in}{2.765494in}}%
\pgfpathlineto{\pgfqpoint{2.272350in}{2.765502in}}%
\pgfpathlineto{\pgfqpoint{2.319755in}{2.765521in}}%
\pgfpathlineto{\pgfqpoint{2.320884in}{2.768665in}}%
\pgfpathlineto{\pgfqpoint{2.329349in}{2.765602in}}%
\pgfpathlineto{\pgfqpoint{2.329913in}{2.773819in}}%
\pgfpathlineto{\pgfqpoint{2.331042in}{2.765864in}}%
\pgfpathlineto{\pgfqpoint{2.331606in}{2.766327in}}%
\pgfpathlineto{\pgfqpoint{2.332735in}{2.777494in}}%
\pgfpathlineto{\pgfqpoint{2.333299in}{2.776513in}}%
\pgfpathlineto{\pgfqpoint{2.333863in}{2.779361in}}%
\pgfpathlineto{\pgfqpoint{2.334992in}{2.798955in}}%
\pgfpathlineto{\pgfqpoint{2.336121in}{2.776506in}}%
\pgfpathlineto{\pgfqpoint{2.345150in}{2.779824in}}%
\pgfpathlineto{\pgfqpoint{2.345715in}{2.783880in}}%
\pgfpathlineto{\pgfqpoint{2.346843in}{2.783358in}}%
\pgfpathlineto{\pgfqpoint{2.347408in}{2.784186in}}%
\pgfpathlineto{\pgfqpoint{2.347972in}{2.782169in}}%
\pgfpathlineto{\pgfqpoint{2.348536in}{2.781708in}}%
\pgfpathlineto{\pgfqpoint{2.359259in}{2.781306in}}%
\pgfpathlineto{\pgfqpoint{2.359823in}{2.790287in}}%
\pgfpathlineto{\pgfqpoint{2.360388in}{2.783575in}}%
\pgfpathlineto{\pgfqpoint{2.361516in}{2.794821in}}%
\pgfpathlineto{\pgfqpoint{2.362081in}{2.794955in}}%
\pgfpathlineto{\pgfqpoint{2.362645in}{2.795413in}}%
\pgfpathlineto{\pgfqpoint{2.369982in}{2.806362in}}%
\pgfpathlineto{\pgfqpoint{2.370546in}{2.802655in}}%
\pgfpathlineto{\pgfqpoint{2.371110in}{2.790789in}}%
\pgfpathlineto{\pgfqpoint{2.371675in}{2.788114in}}%
\pgfpathlineto{\pgfqpoint{2.372803in}{2.794266in}}%
\pgfpathlineto{\pgfqpoint{2.373368in}{2.823143in}}%
\pgfpathlineto{\pgfqpoint{2.373932in}{2.801472in}}%
\pgfpathlineto{\pgfqpoint{2.374496in}{2.809843in}}%
\pgfpathlineto{\pgfqpoint{2.375061in}{2.861710in}}%
\pgfpathlineto{\pgfqpoint{2.412308in}{2.795265in}}%
\pgfpathlineto{\pgfqpoint{2.412872in}{2.794828in}}%
\pgfpathlineto{\pgfqpoint{2.413436in}{2.792909in}}%
\pgfpathlineto{\pgfqpoint{2.414001in}{2.786615in}}%
\pgfpathlineto{\pgfqpoint{2.414565in}{2.789772in}}%
\pgfpathlineto{\pgfqpoint{2.416258in}{2.841379in}}%
\pgfpathlineto{\pgfqpoint{2.416822in}{2.830141in}}%
\pgfpathlineto{\pgfqpoint{2.417387in}{2.804548in}}%
\pgfpathlineto{\pgfqpoint{2.426416in}{2.854645in}}%
\pgfpathlineto{\pgfqpoint{2.427545in}{2.785465in}}%
\pgfpathlineto{\pgfqpoint{2.428674in}{2.791239in}}%
\pgfpathlineto{\pgfqpoint{2.437703in}{2.780209in}}%
\pgfpathlineto{\pgfqpoint{2.438267in}{2.783573in}}%
\pgfpathlineto{\pgfqpoint{2.439960in}{2.858003in}}%
\pgfpathlineto{\pgfqpoint{2.440525in}{2.858749in}}%
\pgfpathlineto{\pgfqpoint{2.441653in}{2.859007in}}%
\pgfpathlineto{\pgfqpoint{2.442218in}{2.842107in}}%
\pgfpathlineto{\pgfqpoint{2.442782in}{2.802658in}}%
\pgfpathlineto{\pgfqpoint{2.443347in}{2.798013in}}%
\pgfpathlineto{\pgfqpoint{2.452376in}{2.779383in}}%
\pgfpathlineto{\pgfqpoint{2.454633in}{2.779307in}}%
\pgfpathlineto{\pgfqpoint{2.455198in}{2.796310in}}%
\pgfpathlineto{\pgfqpoint{2.455762in}{2.840310in}}%
\pgfpathlineto{\pgfqpoint{2.456326in}{2.859974in}}%
\pgfpathlineto{\pgfqpoint{2.508811in}{2.855170in}}%
\pgfpathlineto{\pgfqpoint{2.509375in}{2.853635in}}%
\pgfpathlineto{\pgfqpoint{2.524048in}{2.788574in}}%
\pgfpathlineto{\pgfqpoint{2.524612in}{2.791034in}}%
\pgfpathlineto{\pgfqpoint{2.533077in}{2.852731in}}%
\pgfpathlineto{\pgfqpoint{2.533642in}{2.825374in}}%
\pgfpathlineto{\pgfqpoint{2.534206in}{2.781433in}}%
\pgfpathlineto{\pgfqpoint{2.537028in}{2.798626in}}%
\pgfpathlineto{\pgfqpoint{2.537592in}{2.798033in}}%
\pgfpathlineto{\pgfqpoint{2.538157in}{2.798107in}}%
\pgfpathlineto{\pgfqpoint{2.539850in}{2.813554in}}%
\pgfpathlineto{\pgfqpoint{2.545493in}{2.867718in}}%
\pgfpathlineto{\pgfqpoint{2.546057in}{2.870078in}}%
\pgfpathlineto{\pgfqpoint{2.546622in}{2.805611in}}%
\pgfpathlineto{\pgfqpoint{2.547186in}{2.799517in}}%
\pgfpathlineto{\pgfqpoint{2.547750in}{2.799773in}}%
\pgfpathlineto{\pgfqpoint{2.548315in}{2.861103in}}%
\pgfpathlineto{\pgfqpoint{2.548879in}{2.874805in}}%
\pgfpathlineto{\pgfqpoint{2.549444in}{2.842830in}}%
\pgfpathlineto{\pgfqpoint{2.550008in}{2.828863in}}%
\pgfpathlineto{\pgfqpoint{2.550572in}{2.785311in}}%
\pgfpathlineto{\pgfqpoint{2.555651in}{2.846439in}}%
\pgfpathlineto{\pgfqpoint{2.559602in}{2.891967in}}%
\pgfpathlineto{\pgfqpoint{2.560730in}{2.834834in}}%
\pgfpathlineto{\pgfqpoint{2.561295in}{2.803164in}}%
\pgfpathlineto{\pgfqpoint{2.561859in}{2.796446in}}%
\pgfpathlineto{\pgfqpoint{2.562423in}{2.807992in}}%
\pgfpathlineto{\pgfqpoint{2.562988in}{2.894786in}}%
\pgfpathlineto{\pgfqpoint{2.563552in}{2.894517in}}%
\pgfpathlineto{\pgfqpoint{2.601363in}{2.800302in}}%
\pgfpathlineto{\pgfqpoint{2.601928in}{2.810993in}}%
\pgfpathlineto{\pgfqpoint{2.603621in}{2.888180in}}%
\pgfpathlineto{\pgfqpoint{2.604185in}{2.852787in}}%
\pgfpathlineto{\pgfqpoint{2.605314in}{2.905492in}}%
\pgfpathlineto{\pgfqpoint{2.608135in}{2.902610in}}%
\pgfpathlineto{\pgfqpoint{2.614343in}{2.896198in}}%
\pgfpathlineto{\pgfqpoint{2.614908in}{2.848502in}}%
\pgfpathlineto{\pgfqpoint{2.615472in}{2.883340in}}%
\pgfpathlineto{\pgfqpoint{2.616036in}{2.815531in}}%
\pgfpathlineto{\pgfqpoint{2.616601in}{2.877713in}}%
\pgfpathlineto{\pgfqpoint{2.617165in}{2.904999in}}%
\pgfpathlineto{\pgfqpoint{2.617729in}{2.907883in}}%
\pgfpathlineto{\pgfqpoint{2.618294in}{2.903592in}}%
\pgfpathlineto{\pgfqpoint{2.619422in}{2.910093in}}%
\pgfpathlineto{\pgfqpoint{2.628452in}{2.909906in}}%
\pgfpathlineto{\pgfqpoint{2.629581in}{2.907444in}}%
\pgfpathlineto{\pgfqpoint{2.632402in}{2.900534in}}%
\pgfpathlineto{\pgfqpoint{2.634095in}{2.899653in}}%
\pgfpathlineto{\pgfqpoint{2.641432in}{2.896362in}}%
\pgfpathlineto{\pgfqpoint{2.641996in}{2.897001in}}%
\pgfpathlineto{\pgfqpoint{2.643689in}{2.907789in}}%
\pgfpathlineto{\pgfqpoint{2.645382in}{2.910657in}}%
\pgfpathlineto{\pgfqpoint{2.657798in}{2.923416in}}%
\pgfpathlineto{\pgfqpoint{2.710846in}{2.910455in}}%
\pgfpathlineto{\pgfqpoint{2.711411in}{2.914108in}}%
\pgfpathlineto{\pgfqpoint{2.712539in}{2.928615in}}%
\pgfpathlineto{\pgfqpoint{2.713104in}{2.928610in}}%
\pgfpathlineto{\pgfqpoint{2.713668in}{2.929038in}}%
\pgfpathlineto{\pgfqpoint{2.720440in}{2.914891in}}%
\pgfpathlineto{\pgfqpoint{2.722133in}{2.911342in}}%
\pgfpathlineto{\pgfqpoint{2.722698in}{2.910872in}}%
\pgfpathlineto{\pgfqpoint{2.726084in}{2.921524in}}%
\pgfpathlineto{\pgfqpoint{2.726648in}{2.930077in}}%
\pgfpathlineto{\pgfqpoint{2.727212in}{2.953610in}}%
\pgfpathlineto{\pgfqpoint{2.737935in}{2.912220in}}%
\pgfpathlineto{\pgfqpoint{2.738499in}{2.913351in}}%
\pgfpathlineto{\pgfqpoint{2.741321in}{2.974596in}}%
\pgfpathlineto{\pgfqpoint{2.748093in}{2.921818in}}%
\pgfpathlineto{\pgfqpoint{2.749222in}{2.912991in}}%
\pgfpathlineto{\pgfqpoint{2.749786in}{2.910924in}}%
\pgfpathlineto{\pgfqpoint{2.750351in}{2.911782in}}%
\pgfpathlineto{\pgfqpoint{2.751479in}{2.912093in}}%
\pgfpathlineto{\pgfqpoint{2.753172in}{2.911948in}}%
\pgfpathlineto{\pgfqpoint{2.754301in}{2.909843in}}%
\pgfpathlineto{\pgfqpoint{2.790983in}{2.912056in}}%
\pgfpathlineto{\pgfqpoint{2.791548in}{2.911693in}}%
\pgfpathlineto{\pgfqpoint{2.792112in}{2.911692in}}%
\pgfpathlineto{\pgfqpoint{2.792677in}{2.912093in}}%
\pgfpathlineto{\pgfqpoint{2.793241in}{2.939012in}}%
\pgfpathlineto{\pgfqpoint{2.793805in}{2.980381in}}%
\pgfpathlineto{\pgfqpoint{2.794934in}{2.983100in}}%
\pgfpathlineto{\pgfqpoint{2.796063in}{2.983254in}}%
\pgfpathlineto{\pgfqpoint{2.806785in}{2.983254in}}%
\pgfpathlineto{\pgfqpoint{2.807350in}{2.979058in}}%
\pgfpathlineto{\pgfqpoint{2.807914in}{2.956093in}}%
\pgfpathlineto{\pgfqpoint{2.808478in}{2.948331in}}%
\pgfpathlineto{\pgfqpoint{2.809607in}{2.948527in}}%
\pgfpathlineto{\pgfqpoint{2.818072in}{2.948515in}}%
\pgfpathlineto{\pgfqpoint{2.819201in}{2.946466in}}%
\pgfpathlineto{\pgfqpoint{2.834438in}{2.912574in}}%
\pgfpathlineto{\pgfqpoint{2.835002in}{2.912091in}}%
\pgfpathlineto{\pgfqpoint{2.888051in}{2.910408in}}%
\pgfpathlineto{\pgfqpoint{2.888615in}{2.912393in}}%
\pgfpathlineto{\pgfqpoint{2.889180in}{2.891156in}}%
\pgfpathlineto{\pgfqpoint{2.889744in}{2.830598in}}%
\pgfpathlineto{\pgfqpoint{2.900467in}{2.946477in}}%
\pgfpathlineto{\pgfqpoint{2.901031in}{2.925991in}}%
\pgfpathlineto{\pgfqpoint{2.901595in}{2.933104in}}%
\pgfpathlineto{\pgfqpoint{2.902160in}{2.925411in}}%
\pgfpathlineto{\pgfqpoint{2.902724in}{2.938437in}}%
\pgfpathlineto{\pgfqpoint{2.903288in}{2.937585in}}%
\pgfpathlineto{\pgfqpoint{2.903853in}{2.946774in}}%
\pgfpathlineto{\pgfqpoint{2.911753in}{2.913537in}}%
\pgfpathlineto{\pgfqpoint{2.912318in}{2.921810in}}%
\pgfpathlineto{\pgfqpoint{2.912882in}{2.920130in}}%
\pgfpathlineto{\pgfqpoint{2.913446in}{2.915027in}}%
\pgfpathlineto{\pgfqpoint{2.914575in}{2.948959in}}%
\pgfpathlineto{\pgfqpoint{2.915140in}{2.938963in}}%
\pgfpathlineto{\pgfqpoint{2.915704in}{2.950363in}}%
\pgfpathlineto{\pgfqpoint{2.916833in}{2.950804in}}%
\pgfpathlineto{\pgfqpoint{2.917961in}{2.951321in}}%
\pgfpathlineto{\pgfqpoint{2.918526in}{2.950316in}}%
\pgfpathlineto{\pgfqpoint{2.925298in}{2.915503in}}%
\pgfpathlineto{\pgfqpoint{2.925862in}{2.921920in}}%
\pgfpathlineto{\pgfqpoint{2.926426in}{2.922134in}}%
\pgfpathlineto{\pgfqpoint{2.926991in}{2.951872in}}%
\pgfpathlineto{\pgfqpoint{2.927555in}{2.950160in}}%
\pgfpathlineto{\pgfqpoint{2.928119in}{2.950578in}}%
\pgfpathlineto{\pgfqpoint{2.928684in}{2.952386in}}%
\pgfpathlineto{\pgfqpoint{2.929248in}{2.950692in}}%
\pgfpathlineto{\pgfqpoint{2.929813in}{2.952561in}}%
\pgfpathlineto{\pgfqpoint{2.930377in}{2.955998in}}%
\pgfpathlineto{\pgfqpoint{2.930941in}{2.956693in}}%
\pgfpathlineto{\pgfqpoint{2.941099in}{2.954206in}}%
\pgfpathlineto{\pgfqpoint{2.941664in}{2.955281in}}%
\pgfpathlineto{\pgfqpoint{2.942228in}{2.958672in}}%
\pgfpathlineto{\pgfqpoint{2.942792in}{2.958748in}}%
\pgfpathlineto{\pgfqpoint{2.943921in}{2.954117in}}%
\pgfpathlineto{\pgfqpoint{2.973267in}{2.961839in}}%
\pgfpathlineto{\pgfqpoint{2.983425in}{2.964514in}}%
\pgfpathlineto{\pgfqpoint{2.983990in}{2.963995in}}%
\pgfpathlineto{\pgfqpoint{2.984554in}{2.956074in}}%
\pgfpathlineto{\pgfqpoint{2.985118in}{2.953878in}}%
\pgfpathlineto{\pgfqpoint{2.995841in}{2.950731in}}%
\pgfpathlineto{\pgfqpoint{2.996405in}{2.951043in}}%
\pgfpathlineto{\pgfqpoint{2.996970in}{2.962144in}}%
\pgfpathlineto{\pgfqpoint{2.997534in}{2.964894in}}%
\pgfpathlineto{\pgfqpoint{3.007128in}{2.964894in}}%
\pgfpathlineto{\pgfqpoint{3.008257in}{2.950520in}}%
\pgfpathlineto{\pgfqpoint{3.009950in}{2.950520in}}%
\pgfpathlineto{\pgfqpoint{3.011643in}{2.954239in}}%
\pgfpathlineto{\pgfqpoint{3.012207in}{2.955568in}}%
\pgfpathlineto{\pgfqpoint{3.022930in}{2.966088in}}%
\pgfpathlineto{\pgfqpoint{3.023494in}{2.963497in}}%
\pgfpathlineto{\pgfqpoint{3.024058in}{2.954330in}}%
\pgfpathlineto{\pgfqpoint{3.024623in}{2.950520in}}%
\pgfpathlineto{\pgfqpoint{3.028009in}{2.950520in}}%
\pgfpathlineto{\pgfqpoint{3.028573in}{2.951171in}}%
\pgfpathlineto{\pgfqpoint{3.033088in}{2.964267in}}%
\pgfpathlineto{\pgfqpoint{3.033652in}{2.967064in}}%
\pgfpathlineto{\pgfqpoint{3.034216in}{2.973584in}}%
\pgfpathlineto{\pgfqpoint{3.034781in}{2.975944in}}%
\pgfpathlineto{\pgfqpoint{3.036474in}{2.950520in}}%
\pgfpathlineto{\pgfqpoint{3.037603in}{2.950520in}}%
\pgfpathlineto{\pgfqpoint{3.038731in}{2.966659in}}%
\pgfpathlineto{\pgfqpoint{3.039296in}{2.967456in}}%
\pgfpathlineto{\pgfqpoint{3.077107in}{2.967211in}}%
\pgfpathlineto{\pgfqpoint{3.077671in}{2.967022in}}%
\pgfpathlineto{\pgfqpoint{3.079364in}{2.967562in}}%
\pgfpathlineto{\pgfqpoint{3.087265in}{2.966933in}}%
\pgfpathlineto{\pgfqpoint{3.087829in}{2.967305in}}%
\pgfpathlineto{\pgfqpoint{3.088958in}{2.967106in}}%
\pgfpathlineto{\pgfqpoint{3.090651in}{2.967106in}}%
\pgfpathlineto{\pgfqpoint{3.092908in}{2.967076in}}%
\pgfpathlineto{\pgfqpoint{3.183204in}{2.980898in}}%
\pgfpathlineto{\pgfqpoint{3.184332in}{2.974809in}}%
\pgfpathlineto{\pgfqpoint{3.184897in}{2.981973in}}%
\pgfpathlineto{\pgfqpoint{3.212550in}{2.981973in}}%
\pgfpathlineto{\pgfqpoint{3.213114in}{2.982645in}}%
\pgfpathlineto{\pgfqpoint{3.213678in}{2.984303in}}%
\pgfpathlineto{\pgfqpoint{3.221015in}{2.987138in}}%
\pgfpathlineto{\pgfqpoint{3.227223in}{2.989375in}}%
\pgfpathlineto{\pgfqpoint{3.266163in}{2.989518in}}%
\pgfpathlineto{\pgfqpoint{3.266727in}{2.990636in}}%
\pgfpathlineto{\pgfqpoint{3.268420in}{2.996471in}}%
\pgfpathlineto{\pgfqpoint{3.269549in}{2.996965in}}%
\pgfpathlineto{\pgfqpoint{3.288172in}{2.999932in}}%
\pgfpathlineto{\pgfqpoint{3.296073in}{3.000759in}}%
\pgfpathlineto{\pgfqpoint{4.502644in}{3.001035in}}%
\pgfpathlineto{\pgfqpoint{5.072069in}{3.002609in}}%
\pgfpathlineto{\pgfqpoint{6.004368in}{3.002609in}}%
\pgfpathlineto{\pgfqpoint{6.004368in}{3.002609in}}%
\pgfusepath{stroke}%
\end{pgfscope}%
\begin{pgfscope}%
\pgfpathrectangle{\pgfqpoint{0.481681in}{1.080890in}}{\pgfqpoint{5.785672in}{2.146863in}}%
\pgfusepath{clip}%
\pgfsetrectcap%
\pgfsetroundjoin%
\pgfsetlinewidth{0.200750pt}%
\definecolor{currentstroke}{rgb}{0.000000,0.372549,0.450980}%
\pgfsetstrokecolor{currentstroke}%
\pgfsetdash{}{0pt}%
\pgfpathmoveto{\pgfqpoint{0.744666in}{1.178475in}}%
\pgfpathlineto{\pgfqpoint{0.798843in}{1.178740in}}%
\pgfpathlineto{\pgfqpoint{0.799407in}{1.180278in}}%
\pgfpathlineto{\pgfqpoint{0.799972in}{1.179916in}}%
\pgfpathlineto{\pgfqpoint{0.800536in}{1.179134in}}%
\pgfpathlineto{\pgfqpoint{0.801101in}{1.179068in}}%
\pgfpathlineto{\pgfqpoint{0.801665in}{1.180269in}}%
\pgfpathlineto{\pgfqpoint{0.802229in}{1.179119in}}%
\pgfpathlineto{\pgfqpoint{0.802794in}{1.182776in}}%
\pgfpathlineto{\pgfqpoint{0.803922in}{1.184957in}}%
\pgfpathlineto{\pgfqpoint{0.812952in}{1.185228in}}%
\pgfpathlineto{\pgfqpoint{0.814645in}{1.185657in}}%
\pgfpathlineto{\pgfqpoint{0.815209in}{1.187866in}}%
\pgfpathlineto{\pgfqpoint{0.815773in}{1.188208in}}%
\pgfpathlineto{\pgfqpoint{0.816902in}{1.185169in}}%
\pgfpathlineto{\pgfqpoint{0.818031in}{1.184957in}}%
\pgfpathlineto{\pgfqpoint{0.825932in}{1.184957in}}%
\pgfpathlineto{\pgfqpoint{0.826496in}{1.187121in}}%
\pgfpathlineto{\pgfqpoint{0.827625in}{1.196159in}}%
\pgfpathlineto{\pgfqpoint{0.828189in}{1.193518in}}%
\pgfpathlineto{\pgfqpoint{0.828753in}{1.198916in}}%
\pgfpathlineto{\pgfqpoint{0.829318in}{1.200150in}}%
\pgfpathlineto{\pgfqpoint{0.829882in}{1.199843in}}%
\pgfpathlineto{\pgfqpoint{0.830446in}{1.193025in}}%
\pgfpathlineto{\pgfqpoint{0.831011in}{1.205896in}}%
\pgfpathlineto{\pgfqpoint{0.832140in}{1.205191in}}%
\pgfpathlineto{\pgfqpoint{0.840040in}{1.199052in}}%
\pgfpathlineto{\pgfqpoint{0.841733in}{1.207584in}}%
\pgfpathlineto{\pgfqpoint{0.842298in}{1.212304in}}%
\pgfpathlineto{\pgfqpoint{0.842862in}{1.229396in}}%
\pgfpathlineto{\pgfqpoint{0.843426in}{1.225927in}}%
\pgfpathlineto{\pgfqpoint{0.843991in}{1.238120in}}%
\pgfpathlineto{\pgfqpoint{0.846813in}{1.232962in}}%
\pgfpathlineto{\pgfqpoint{0.853020in}{1.221332in}}%
\pgfpathlineto{\pgfqpoint{0.854149in}{1.215172in}}%
\pgfpathlineto{\pgfqpoint{0.854713in}{1.216954in}}%
\pgfpathlineto{\pgfqpoint{0.855278in}{1.239147in}}%
\pgfpathlineto{\pgfqpoint{0.855842in}{1.238666in}}%
\pgfpathlineto{\pgfqpoint{0.895911in}{1.239345in}}%
\pgfpathlineto{\pgfqpoint{0.896475in}{1.238945in}}%
\pgfpathlineto{\pgfqpoint{0.897604in}{1.238961in}}%
\pgfpathlineto{\pgfqpoint{0.898168in}{1.248558in}}%
\pgfpathlineto{\pgfqpoint{0.899297in}{1.248848in}}%
\pgfpathlineto{\pgfqpoint{0.907197in}{1.248848in}}%
\pgfpathlineto{\pgfqpoint{0.907762in}{1.248129in}}%
\pgfpathlineto{\pgfqpoint{0.908891in}{1.241097in}}%
\pgfpathlineto{\pgfqpoint{0.909455in}{1.248197in}}%
\pgfpathlineto{\pgfqpoint{0.910019in}{1.261245in}}%
\pgfpathlineto{\pgfqpoint{0.910584in}{1.265715in}}%
\pgfpathlineto{\pgfqpoint{0.911148in}{1.246862in}}%
\pgfpathlineto{\pgfqpoint{0.911712in}{1.255667in}}%
\pgfpathlineto{\pgfqpoint{0.912277in}{1.257181in}}%
\pgfpathlineto{\pgfqpoint{0.921870in}{1.257181in}}%
\pgfpathlineto{\pgfqpoint{0.922999in}{1.276992in}}%
\pgfpathlineto{\pgfqpoint{0.923564in}{1.302828in}}%
\pgfpathlineto{\pgfqpoint{0.924128in}{1.309951in}}%
\pgfpathlineto{\pgfqpoint{0.933722in}{1.309961in}}%
\pgfpathlineto{\pgfqpoint{0.934286in}{1.310556in}}%
\pgfpathlineto{\pgfqpoint{0.934850in}{1.347931in}}%
\pgfpathlineto{\pgfqpoint{0.935415in}{1.357185in}}%
\pgfpathlineto{\pgfqpoint{0.935979in}{1.357185in}}%
\pgfpathlineto{\pgfqpoint{0.936543in}{1.359831in}}%
\pgfpathlineto{\pgfqpoint{0.938801in}{1.401047in}}%
\pgfpathlineto{\pgfqpoint{0.939365in}{1.401631in}}%
\pgfpathlineto{\pgfqpoint{0.947266in}{1.401631in}}%
\pgfpathlineto{\pgfqpoint{0.947830in}{1.401986in}}%
\pgfpathlineto{\pgfqpoint{0.949523in}{1.409918in}}%
\pgfpathlineto{\pgfqpoint{0.950088in}{1.408774in}}%
\pgfpathlineto{\pgfqpoint{0.950652in}{1.402827in}}%
\pgfpathlineto{\pgfqpoint{0.951216in}{1.403482in}}%
\pgfpathlineto{\pgfqpoint{0.990721in}{1.403482in}}%
\pgfpathlineto{\pgfqpoint{0.991285in}{1.406194in}}%
\pgfpathlineto{\pgfqpoint{0.991849in}{1.413565in}}%
\pgfpathlineto{\pgfqpoint{0.992414in}{1.413989in}}%
\pgfpathlineto{\pgfqpoint{0.992978in}{1.414998in}}%
\pgfpathlineto{\pgfqpoint{0.993542in}{1.419302in}}%
\pgfpathlineto{\pgfqpoint{0.995235in}{1.420988in}}%
\pgfpathlineto{\pgfqpoint{1.002572in}{1.420187in}}%
\pgfpathlineto{\pgfqpoint{1.003136in}{1.414841in}}%
\pgfpathlineto{\pgfqpoint{1.003701in}{1.424799in}}%
\pgfpathlineto{\pgfqpoint{1.004265in}{1.415279in}}%
\pgfpathlineto{\pgfqpoint{1.004829in}{1.432936in}}%
\pgfpathlineto{\pgfqpoint{1.005394in}{1.438846in}}%
\pgfpathlineto{\pgfqpoint{1.006522in}{1.436074in}}%
\pgfpathlineto{\pgfqpoint{1.007087in}{1.441017in}}%
\pgfpathlineto{\pgfqpoint{1.007651in}{1.430667in}}%
\pgfpathlineto{\pgfqpoint{1.015552in}{1.464945in}}%
\pgfpathlineto{\pgfqpoint{1.016681in}{1.449419in}}%
\pgfpathlineto{\pgfqpoint{1.017809in}{1.529511in}}%
\pgfpathlineto{\pgfqpoint{1.018374in}{1.531606in}}%
\pgfpathlineto{\pgfqpoint{1.018938in}{1.532190in}}%
\pgfpathlineto{\pgfqpoint{1.019502in}{1.542382in}}%
\pgfpathlineto{\pgfqpoint{1.020067in}{1.560801in}}%
\pgfpathlineto{\pgfqpoint{1.020631in}{1.512136in}}%
\pgfpathlineto{\pgfqpoint{1.021195in}{1.566205in}}%
\pgfpathlineto{\pgfqpoint{1.042640in}{1.601064in}}%
\pgfpathlineto{\pgfqpoint{1.043205in}{1.526597in}}%
\pgfpathlineto{\pgfqpoint{1.043769in}{1.566011in}}%
\pgfpathlineto{\pgfqpoint{1.044334in}{1.524716in}}%
\pgfpathlineto{\pgfqpoint{1.044898in}{1.585939in}}%
\pgfpathlineto{\pgfqpoint{1.045462in}{1.582058in}}%
\pgfpathlineto{\pgfqpoint{1.046027in}{1.599485in}}%
\pgfpathlineto{\pgfqpoint{1.046591in}{1.575405in}}%
\pgfpathlineto{\pgfqpoint{1.047155in}{1.600555in}}%
\pgfpathlineto{\pgfqpoint{1.047720in}{1.608866in}}%
\pgfpathlineto{\pgfqpoint{1.085531in}{1.593390in}}%
\pgfpathlineto{\pgfqpoint{1.086659in}{1.594410in}}%
\pgfpathlineto{\pgfqpoint{1.090046in}{1.598668in}}%
\pgfpathlineto{\pgfqpoint{1.097382in}{1.609505in}}%
\pgfpathlineto{\pgfqpoint{1.097946in}{1.598302in}}%
\pgfpathlineto{\pgfqpoint{1.099075in}{1.607557in}}%
\pgfpathlineto{\pgfqpoint{1.099639in}{1.618112in}}%
\pgfpathlineto{\pgfqpoint{1.100204in}{1.623320in}}%
\pgfpathlineto{\pgfqpoint{1.100768in}{1.631979in}}%
\pgfpathlineto{\pgfqpoint{1.101332in}{1.606345in}}%
\pgfpathlineto{\pgfqpoint{1.101897in}{1.618738in}}%
\pgfpathlineto{\pgfqpoint{1.102461in}{1.617799in}}%
\pgfpathlineto{\pgfqpoint{1.110362in}{1.596149in}}%
\pgfpathlineto{\pgfqpoint{1.110926in}{1.610715in}}%
\pgfpathlineto{\pgfqpoint{1.111491in}{1.650822in}}%
\pgfpathlineto{\pgfqpoint{1.112055in}{1.631223in}}%
\pgfpathlineto{\pgfqpoint{1.113748in}{1.668607in}}%
\pgfpathlineto{\pgfqpoint{1.114312in}{1.670505in}}%
\pgfpathlineto{\pgfqpoint{1.114877in}{1.617147in}}%
\pgfpathlineto{\pgfqpoint{1.115441in}{1.631624in}}%
\pgfpathlineto{\pgfqpoint{1.116005in}{1.624265in}}%
\pgfpathlineto{\pgfqpoint{1.116570in}{1.624216in}}%
\pgfpathlineto{\pgfqpoint{1.123906in}{1.618548in}}%
\pgfpathlineto{\pgfqpoint{1.124471in}{1.629806in}}%
\pgfpathlineto{\pgfqpoint{1.125035in}{1.674636in}}%
\pgfpathlineto{\pgfqpoint{1.125599in}{1.692288in}}%
\pgfpathlineto{\pgfqpoint{1.126164in}{1.658523in}}%
\pgfpathlineto{\pgfqpoint{1.127857in}{1.637906in}}%
\pgfpathlineto{\pgfqpoint{1.128421in}{1.701079in}}%
\pgfpathlineto{\pgfqpoint{1.128985in}{1.668300in}}%
\pgfpathlineto{\pgfqpoint{1.129550in}{1.782401in}}%
\pgfpathlineto{\pgfqpoint{1.130114in}{1.793703in}}%
\pgfpathlineto{\pgfqpoint{1.138015in}{1.719168in}}%
\pgfpathlineto{\pgfqpoint{1.138579in}{1.710696in}}%
\pgfpathlineto{\pgfqpoint{1.139144in}{1.707392in}}%
\pgfpathlineto{\pgfqpoint{1.140272in}{1.712032in}}%
\pgfpathlineto{\pgfqpoint{1.141401in}{1.799032in}}%
\pgfpathlineto{\pgfqpoint{1.141965in}{1.704081in}}%
\pgfpathlineto{\pgfqpoint{1.144223in}{1.709241in}}%
\pgfpathlineto{\pgfqpoint{1.179212in}{1.791965in}}%
\pgfpathlineto{\pgfqpoint{1.179776in}{1.788716in}}%
\pgfpathlineto{\pgfqpoint{1.180341in}{1.757018in}}%
\pgfpathlineto{\pgfqpoint{1.180905in}{1.741915in}}%
\pgfpathlineto{\pgfqpoint{1.181470in}{1.814607in}}%
\pgfpathlineto{\pgfqpoint{1.182034in}{1.803473in}}%
\pgfpathlineto{\pgfqpoint{1.182598in}{1.819121in}}%
\pgfpathlineto{\pgfqpoint{1.183163in}{1.821814in}}%
\pgfpathlineto{\pgfqpoint{1.183727in}{1.816588in}}%
\pgfpathlineto{\pgfqpoint{1.184291in}{1.841698in}}%
\pgfpathlineto{\pgfqpoint{1.192756in}{1.845018in}}%
\pgfpathlineto{\pgfqpoint{1.193321in}{1.823246in}}%
\pgfpathlineto{\pgfqpoint{1.194449in}{1.849624in}}%
\pgfpathlineto{\pgfqpoint{1.195014in}{1.851645in}}%
\pgfpathlineto{\pgfqpoint{1.196707in}{1.791421in}}%
\pgfpathlineto{\pgfqpoint{1.197271in}{1.866461in}}%
\pgfpathlineto{\pgfqpoint{1.197836in}{1.871091in}}%
\pgfpathlineto{\pgfqpoint{1.205736in}{1.871091in}}%
\pgfpathlineto{\pgfqpoint{1.206301in}{1.800462in}}%
\pgfpathlineto{\pgfqpoint{1.206865in}{1.766165in}}%
\pgfpathlineto{\pgfqpoint{1.207429in}{1.857216in}}%
\pgfpathlineto{\pgfqpoint{1.207994in}{1.845349in}}%
\pgfpathlineto{\pgfqpoint{1.208558in}{1.867387in}}%
\pgfpathlineto{\pgfqpoint{1.209122in}{1.868930in}}%
\pgfpathlineto{\pgfqpoint{1.209687in}{1.896564in}}%
\pgfpathlineto{\pgfqpoint{1.210251in}{1.884115in}}%
\pgfpathlineto{\pgfqpoint{1.210815in}{1.937256in}}%
\pgfpathlineto{\pgfqpoint{1.211380in}{1.938685in}}%
\pgfpathlineto{\pgfqpoint{1.219845in}{1.938824in}}%
\pgfpathlineto{\pgfqpoint{1.220409in}{1.938132in}}%
\pgfpathlineto{\pgfqpoint{1.221538in}{1.948292in}}%
\pgfpathlineto{\pgfqpoint{1.222667in}{1.974490in}}%
\pgfpathlineto{\pgfqpoint{1.223231in}{2.053393in}}%
\pgfpathlineto{\pgfqpoint{1.223795in}{1.993738in}}%
\pgfpathlineto{\pgfqpoint{1.224360in}{1.979276in}}%
\pgfpathlineto{\pgfqpoint{1.224924in}{1.981497in}}%
\pgfpathlineto{\pgfqpoint{1.231696in}{2.092340in}}%
\pgfpathlineto{\pgfqpoint{1.232261in}{2.095780in}}%
\pgfpathlineto{\pgfqpoint{1.232825in}{2.127170in}}%
\pgfpathlineto{\pgfqpoint{1.233389in}{2.121847in}}%
\pgfpathlineto{\pgfqpoint{1.233954in}{2.110711in}}%
\pgfpathlineto{\pgfqpoint{1.234518in}{2.150729in}}%
\pgfpathlineto{\pgfqpoint{1.235082in}{2.151629in}}%
\pgfpathlineto{\pgfqpoint{1.235647in}{2.158885in}}%
\pgfpathlineto{\pgfqpoint{1.236775in}{2.152412in}}%
\pgfpathlineto{\pgfqpoint{1.272894in}{1.908021in}}%
\pgfpathlineto{\pgfqpoint{1.273458in}{1.928333in}}%
\pgfpathlineto{\pgfqpoint{1.274022in}{2.064265in}}%
\pgfpathlineto{\pgfqpoint{1.274587in}{2.051202in}}%
\pgfpathlineto{\pgfqpoint{1.275715in}{2.192796in}}%
\pgfpathlineto{\pgfqpoint{1.276280in}{2.182009in}}%
\pgfpathlineto{\pgfqpoint{1.276844in}{1.926597in}}%
\pgfpathlineto{\pgfqpoint{1.277408in}{1.934940in}}%
\pgfpathlineto{\pgfqpoint{1.277973in}{2.105138in}}%
\pgfpathlineto{\pgfqpoint{1.278537in}{2.125047in}}%
\pgfpathlineto{\pgfqpoint{1.279101in}{2.276392in}}%
\pgfpathlineto{\pgfqpoint{1.287002in}{1.918859in}}%
\pgfpathlineto{\pgfqpoint{1.287567in}{1.933823in}}%
\pgfpathlineto{\pgfqpoint{1.288131in}{1.912142in}}%
\pgfpathlineto{\pgfqpoint{1.288695in}{1.907441in}}%
\pgfpathlineto{\pgfqpoint{1.289260in}{2.082806in}}%
\pgfpathlineto{\pgfqpoint{1.289824in}{2.144675in}}%
\pgfpathlineto{\pgfqpoint{1.290388in}{1.986667in}}%
\pgfpathlineto{\pgfqpoint{1.290953in}{1.972845in}}%
\pgfpathlineto{\pgfqpoint{1.291517in}{2.254680in}}%
\pgfpathlineto{\pgfqpoint{1.292081in}{2.284201in}}%
\pgfpathlineto{\pgfqpoint{1.299982in}{2.177412in}}%
\pgfpathlineto{\pgfqpoint{1.300546in}{2.192938in}}%
\pgfpathlineto{\pgfqpoint{1.301111in}{2.281562in}}%
\pgfpathlineto{\pgfqpoint{1.301675in}{2.187089in}}%
\pgfpathlineto{\pgfqpoint{1.302239in}{2.280546in}}%
\pgfpathlineto{\pgfqpoint{1.302804in}{2.275158in}}%
\pgfpathlineto{\pgfqpoint{1.303368in}{2.275294in}}%
\pgfpathlineto{\pgfqpoint{1.303933in}{2.277266in}}%
\pgfpathlineto{\pgfqpoint{1.304497in}{2.280594in}}%
\pgfpathlineto{\pgfqpoint{1.305061in}{2.277562in}}%
\pgfpathlineto{\pgfqpoint{1.305626in}{1.920994in}}%
\pgfpathlineto{\pgfqpoint{1.307319in}{1.994260in}}%
\pgfpathlineto{\pgfqpoint{1.313526in}{2.279877in}}%
\pgfpathlineto{\pgfqpoint{1.314091in}{2.288222in}}%
\pgfpathlineto{\pgfqpoint{1.314655in}{2.286405in}}%
\pgfpathlineto{\pgfqpoint{1.315219in}{2.293508in}}%
\pgfpathlineto{\pgfqpoint{1.315784in}{2.292309in}}%
\pgfpathlineto{\pgfqpoint{1.316348in}{2.296239in}}%
\pgfpathlineto{\pgfqpoint{1.316912in}{2.293010in}}%
\pgfpathlineto{\pgfqpoint{1.318041in}{2.291802in}}%
\pgfpathlineto{\pgfqpoint{1.319170in}{2.336815in}}%
\pgfpathlineto{\pgfqpoint{1.334407in}{2.207543in}}%
\pgfpathlineto{\pgfqpoint{1.368268in}{1.920080in}}%
\pgfpathlineto{\pgfqpoint{1.368832in}{1.960902in}}%
\pgfpathlineto{\pgfqpoint{1.369397in}{2.350949in}}%
\pgfpathlineto{\pgfqpoint{1.369961in}{2.352635in}}%
\pgfpathlineto{\pgfqpoint{1.370525in}{2.357486in}}%
\pgfpathlineto{\pgfqpoint{1.371090in}{2.380194in}}%
\pgfpathlineto{\pgfqpoint{1.371654in}{2.383601in}}%
\pgfpathlineto{\pgfqpoint{1.372783in}{2.417904in}}%
\pgfpathlineto{\pgfqpoint{1.373347in}{2.424154in}}%
\pgfpathlineto{\pgfqpoint{1.373911in}{2.408166in}}%
\pgfpathlineto{\pgfqpoint{1.382941in}{1.914077in}}%
\pgfpathlineto{\pgfqpoint{1.383505in}{1.908013in}}%
\pgfpathlineto{\pgfqpoint{1.384070in}{1.866862in}}%
\pgfpathlineto{\pgfqpoint{1.384634in}{1.852625in}}%
\pgfpathlineto{\pgfqpoint{1.385198in}{1.862205in}}%
\pgfpathlineto{\pgfqpoint{1.385763in}{1.863504in}}%
\pgfpathlineto{\pgfqpoint{1.386891in}{1.875758in}}%
\pgfpathlineto{\pgfqpoint{1.394228in}{1.876564in}}%
\pgfpathlineto{\pgfqpoint{1.394792in}{1.877124in}}%
\pgfpathlineto{\pgfqpoint{1.395357in}{1.896757in}}%
\pgfpathlineto{\pgfqpoint{1.395921in}{1.892552in}}%
\pgfpathlineto{\pgfqpoint{1.396485in}{1.874382in}}%
\pgfpathlineto{\pgfqpoint{1.397050in}{1.887922in}}%
\pgfpathlineto{\pgfqpoint{1.397614in}{1.893313in}}%
\pgfpathlineto{\pgfqpoint{1.398743in}{1.893360in}}%
\pgfpathlineto{\pgfqpoint{1.399307in}{1.906385in}}%
\pgfpathlineto{\pgfqpoint{1.399871in}{1.898840in}}%
\pgfpathlineto{\pgfqpoint{1.410030in}{1.897954in}}%
\pgfpathlineto{\pgfqpoint{1.410594in}{1.898377in}}%
\pgfpathlineto{\pgfqpoint{1.411158in}{1.907143in}}%
\pgfpathlineto{\pgfqpoint{1.411723in}{1.931037in}}%
\pgfpathlineto{\pgfqpoint{1.412287in}{1.931947in}}%
\pgfpathlineto{\pgfqpoint{1.413416in}{1.926662in}}%
\pgfpathlineto{\pgfqpoint{1.413980in}{1.935392in}}%
\pgfpathlineto{\pgfqpoint{1.414544in}{1.935907in}}%
\pgfpathlineto{\pgfqpoint{1.423009in}{1.935907in}}%
\pgfpathlineto{\pgfqpoint{1.423574in}{1.935052in}}%
\pgfpathlineto{\pgfqpoint{1.424138in}{1.925259in}}%
\pgfpathlineto{\pgfqpoint{1.424703in}{1.928946in}}%
\pgfpathlineto{\pgfqpoint{1.425267in}{1.936287in}}%
\pgfpathlineto{\pgfqpoint{1.426396in}{1.937960in}}%
\pgfpathlineto{\pgfqpoint{1.465335in}{1.957945in}}%
\pgfpathlineto{\pgfqpoint{1.466464in}{1.943544in}}%
\pgfpathlineto{\pgfqpoint{1.467028in}{1.944217in}}%
\pgfpathlineto{\pgfqpoint{1.467593in}{1.938667in}}%
\pgfpathlineto{\pgfqpoint{1.468157in}{1.943101in}}%
\pgfpathlineto{\pgfqpoint{1.471543in}{1.940745in}}%
\pgfpathlineto{\pgfqpoint{1.476622in}{1.937125in}}%
\pgfpathlineto{\pgfqpoint{1.477187in}{1.937175in}}%
\pgfpathlineto{\pgfqpoint{1.477751in}{1.939554in}}%
\pgfpathlineto{\pgfqpoint{1.478880in}{1.958601in}}%
\pgfpathlineto{\pgfqpoint{1.479444in}{1.960195in}}%
\pgfpathlineto{\pgfqpoint{1.480573in}{1.948092in}}%
\pgfpathlineto{\pgfqpoint{1.481701in}{1.975215in}}%
\pgfpathlineto{\pgfqpoint{1.482830in}{1.973762in}}%
\pgfpathlineto{\pgfqpoint{1.490167in}{1.963231in}}%
\pgfpathlineto{\pgfqpoint{1.490731in}{1.967903in}}%
\pgfpathlineto{\pgfqpoint{1.491295in}{1.965322in}}%
\pgfpathlineto{\pgfqpoint{1.491860in}{1.969358in}}%
\pgfpathlineto{\pgfqpoint{1.492424in}{1.950240in}}%
\pgfpathlineto{\pgfqpoint{1.492988in}{1.963255in}}%
\pgfpathlineto{\pgfqpoint{1.493553in}{1.960278in}}%
\pgfpathlineto{\pgfqpoint{1.494117in}{1.969921in}}%
\pgfpathlineto{\pgfqpoint{1.494681in}{1.994307in}}%
\pgfpathlineto{\pgfqpoint{1.495246in}{2.000709in}}%
\pgfpathlineto{\pgfqpoint{1.495810in}{2.013755in}}%
\pgfpathlineto{\pgfqpoint{1.503711in}{2.002760in}}%
\pgfpathlineto{\pgfqpoint{1.504275in}{1.971909in}}%
\pgfpathlineto{\pgfqpoint{1.504840in}{2.003679in}}%
\pgfpathlineto{\pgfqpoint{1.505404in}{2.019445in}}%
\pgfpathlineto{\pgfqpoint{1.506533in}{2.015058in}}%
\pgfpathlineto{\pgfqpoint{1.507097in}{1.991125in}}%
\pgfpathlineto{\pgfqpoint{1.507661in}{1.988420in}}%
\pgfpathlineto{\pgfqpoint{1.508226in}{2.020303in}}%
\pgfpathlineto{\pgfqpoint{1.517255in}{2.022833in}}%
\pgfpathlineto{\pgfqpoint{1.517820in}{2.003949in}}%
\pgfpathlineto{\pgfqpoint{1.518384in}{2.022657in}}%
\pgfpathlineto{\pgfqpoint{1.518948in}{2.019814in}}%
\pgfpathlineto{\pgfqpoint{1.519513in}{2.022980in}}%
\pgfpathlineto{\pgfqpoint{1.520641in}{1.978215in}}%
\pgfpathlineto{\pgfqpoint{1.521770in}{1.994947in}}%
\pgfpathlineto{\pgfqpoint{1.524592in}{1.993043in}}%
\pgfpathlineto{\pgfqpoint{1.556759in}{1.970565in}}%
\pgfpathlineto{\pgfqpoint{1.557324in}{1.968916in}}%
\pgfpathlineto{\pgfqpoint{1.557888in}{1.958097in}}%
\pgfpathlineto{\pgfqpoint{1.558452in}{1.962607in}}%
\pgfpathlineto{\pgfqpoint{1.559581in}{1.998826in}}%
\pgfpathlineto{\pgfqpoint{1.560145in}{1.997556in}}%
\pgfpathlineto{\pgfqpoint{1.560710in}{1.997438in}}%
\pgfpathlineto{\pgfqpoint{1.561274in}{1.990332in}}%
\pgfpathlineto{\pgfqpoint{1.561839in}{1.965743in}}%
\pgfpathlineto{\pgfqpoint{1.562403in}{2.017445in}}%
\pgfpathlineto{\pgfqpoint{1.564660in}{2.016909in}}%
\pgfpathlineto{\pgfqpoint{1.571432in}{2.015596in}}%
\pgfpathlineto{\pgfqpoint{1.571997in}{2.017521in}}%
\pgfpathlineto{\pgfqpoint{1.572561in}{2.020966in}}%
\pgfpathlineto{\pgfqpoint{1.573690in}{1.991123in}}%
\pgfpathlineto{\pgfqpoint{1.574254in}{2.007172in}}%
\pgfpathlineto{\pgfqpoint{1.574818in}{2.008117in}}%
\pgfpathlineto{\pgfqpoint{1.575383in}{2.025334in}}%
\pgfpathlineto{\pgfqpoint{1.575947in}{2.025725in}}%
\pgfpathlineto{\pgfqpoint{1.577640in}{2.025492in}}%
\pgfpathlineto{\pgfqpoint{1.584977in}{2.023955in}}%
\pgfpathlineto{\pgfqpoint{1.585541in}{1.989317in}}%
\pgfpathlineto{\pgfqpoint{1.586105in}{2.027528in}}%
\pgfpathlineto{\pgfqpoint{1.586670in}{2.025443in}}%
\pgfpathlineto{\pgfqpoint{1.587798in}{1.982282in}}%
\pgfpathlineto{\pgfqpoint{1.588363in}{2.038867in}}%
\pgfpathlineto{\pgfqpoint{1.588927in}{2.016949in}}%
\pgfpathlineto{\pgfqpoint{1.589491in}{2.028630in}}%
\pgfpathlineto{\pgfqpoint{1.590056in}{2.056046in}}%
\pgfpathlineto{\pgfqpoint{1.598521in}{1.943965in}}%
\pgfpathlineto{\pgfqpoint{1.599085in}{2.031435in}}%
\pgfpathlineto{\pgfqpoint{1.599650in}{1.985323in}}%
\pgfpathlineto{\pgfqpoint{1.600214in}{2.011121in}}%
\pgfpathlineto{\pgfqpoint{1.600778in}{2.023315in}}%
\pgfpathlineto{\pgfqpoint{1.601343in}{2.067887in}}%
\pgfpathlineto{\pgfqpoint{1.601907in}{2.062393in}}%
\pgfpathlineto{\pgfqpoint{1.603600in}{2.191471in}}%
\pgfpathlineto{\pgfqpoint{1.611501in}{2.191471in}}%
\pgfpathlineto{\pgfqpoint{1.613194in}{2.192397in}}%
\pgfpathlineto{\pgfqpoint{1.695024in}{2.192416in}}%
\pgfpathlineto{\pgfqpoint{1.696717in}{2.193322in}}%
\pgfpathlineto{\pgfqpoint{1.939950in}{2.192397in}}%
\pgfpathlineto{\pgfqpoint{1.940514in}{2.192620in}}%
\pgfpathlineto{\pgfqpoint{1.941079in}{2.193271in}}%
\pgfpathlineto{\pgfqpoint{1.941643in}{2.193323in}}%
\pgfpathlineto{\pgfqpoint{1.942772in}{2.195175in}}%
\pgfpathlineto{\pgfqpoint{1.950673in}{2.195433in}}%
\pgfpathlineto{\pgfqpoint{1.951237in}{2.208620in}}%
\pgfpathlineto{\pgfqpoint{1.952366in}{2.214343in}}%
\pgfpathlineto{\pgfqpoint{1.952930in}{2.208892in}}%
\pgfpathlineto{\pgfqpoint{1.953494in}{2.193902in}}%
\pgfpathlineto{\pgfqpoint{1.954059in}{2.206232in}}%
\pgfpathlineto{\pgfqpoint{1.954623in}{2.197983in}}%
\pgfpathlineto{\pgfqpoint{1.955187in}{2.199104in}}%
\pgfpathlineto{\pgfqpoint{1.955752in}{2.207993in}}%
\pgfpathlineto{\pgfqpoint{1.956881in}{2.194044in}}%
\pgfpathlineto{\pgfqpoint{1.958009in}{2.196658in}}%
\pgfpathlineto{\pgfqpoint{1.963653in}{2.211092in}}%
\pgfpathlineto{\pgfqpoint{1.964781in}{2.196873in}}%
\pgfpathlineto{\pgfqpoint{1.965346in}{2.215288in}}%
\pgfpathlineto{\pgfqpoint{1.966474in}{2.204824in}}%
\pgfpathlineto{\pgfqpoint{1.967039in}{2.212657in}}%
\pgfpathlineto{\pgfqpoint{1.967603in}{2.200101in}}%
\pgfpathlineto{\pgfqpoint{1.968167in}{2.222773in}}%
\pgfpathlineto{\pgfqpoint{1.968732in}{2.206214in}}%
\pgfpathlineto{\pgfqpoint{1.970425in}{2.206879in}}%
\pgfpathlineto{\pgfqpoint{1.977761in}{2.204510in}}%
\pgfpathlineto{\pgfqpoint{1.979454in}{2.275310in}}%
\pgfpathlineto{\pgfqpoint{1.980019in}{2.274178in}}%
\pgfpathlineto{\pgfqpoint{1.980583in}{2.275062in}}%
\pgfpathlineto{\pgfqpoint{1.981147in}{2.268087in}}%
\pgfpathlineto{\pgfqpoint{1.981712in}{2.228741in}}%
\pgfpathlineto{\pgfqpoint{1.982276in}{2.246501in}}%
\pgfpathlineto{\pgfqpoint{1.982840in}{2.216536in}}%
\pgfpathlineto{\pgfqpoint{1.983405in}{2.211475in}}%
\pgfpathlineto{\pgfqpoint{1.991306in}{2.204689in}}%
\pgfpathlineto{\pgfqpoint{1.991870in}{2.215117in}}%
\pgfpathlineto{\pgfqpoint{1.992999in}{2.291695in}}%
\pgfpathlineto{\pgfqpoint{1.993563in}{2.264059in}}%
\pgfpathlineto{\pgfqpoint{1.994692in}{2.257941in}}%
\pgfpathlineto{\pgfqpoint{1.995256in}{2.237851in}}%
\pgfpathlineto{\pgfqpoint{2.030810in}{2.245312in}}%
\pgfpathlineto{\pgfqpoint{2.031374in}{2.264004in}}%
\pgfpathlineto{\pgfqpoint{2.031938in}{2.294952in}}%
\pgfpathlineto{\pgfqpoint{2.032503in}{2.293242in}}%
\pgfpathlineto{\pgfqpoint{2.033067in}{2.309279in}}%
\pgfpathlineto{\pgfqpoint{2.033632in}{2.238988in}}%
\pgfpathlineto{\pgfqpoint{2.034196in}{2.268178in}}%
\pgfpathlineto{\pgfqpoint{2.035325in}{2.246669in}}%
\pgfpathlineto{\pgfqpoint{2.035889in}{2.244915in}}%
\pgfpathlineto{\pgfqpoint{2.036453in}{2.232473in}}%
\pgfpathlineto{\pgfqpoint{2.037018in}{2.235485in}}%
\pgfpathlineto{\pgfqpoint{2.044918in}{2.328721in}}%
\pgfpathlineto{\pgfqpoint{2.046047in}{2.254463in}}%
\pgfpathlineto{\pgfqpoint{2.047176in}{2.211929in}}%
\pgfpathlineto{\pgfqpoint{2.047740in}{2.249929in}}%
\pgfpathlineto{\pgfqpoint{2.048305in}{2.203141in}}%
\pgfpathlineto{\pgfqpoint{2.048869in}{2.238909in}}%
\pgfpathlineto{\pgfqpoint{2.049998in}{2.352256in}}%
\pgfpathlineto{\pgfqpoint{2.050562in}{2.354440in}}%
\pgfpathlineto{\pgfqpoint{2.057898in}{2.354440in}}%
\pgfpathlineto{\pgfqpoint{2.058463in}{2.314384in}}%
\pgfpathlineto{\pgfqpoint{2.059027in}{2.230073in}}%
\pgfpathlineto{\pgfqpoint{2.060156in}{2.271073in}}%
\pgfpathlineto{\pgfqpoint{2.060720in}{2.266840in}}%
\pgfpathlineto{\pgfqpoint{2.061284in}{2.300048in}}%
\pgfpathlineto{\pgfqpoint{2.061849in}{2.258692in}}%
\pgfpathlineto{\pgfqpoint{2.062413in}{2.372159in}}%
\pgfpathlineto{\pgfqpoint{2.062978in}{2.250121in}}%
\pgfpathlineto{\pgfqpoint{2.063542in}{2.264212in}}%
\pgfpathlineto{\pgfqpoint{2.064106in}{2.419598in}}%
\pgfpathlineto{\pgfqpoint{2.064671in}{2.470046in}}%
\pgfpathlineto{\pgfqpoint{2.072007in}{2.468421in}}%
\pgfpathlineto{\pgfqpoint{2.072571in}{2.443933in}}%
\pgfpathlineto{\pgfqpoint{2.073136in}{2.294804in}}%
\pgfpathlineto{\pgfqpoint{2.073700in}{2.338600in}}%
\pgfpathlineto{\pgfqpoint{2.074264in}{2.444952in}}%
\pgfpathlineto{\pgfqpoint{2.074829in}{2.470748in}}%
\pgfpathlineto{\pgfqpoint{2.075957in}{2.253770in}}%
\pgfpathlineto{\pgfqpoint{2.077086in}{2.339478in}}%
\pgfpathlineto{\pgfqpoint{2.077651in}{2.333471in}}%
\pgfpathlineto{\pgfqpoint{2.084987in}{2.325629in}}%
\pgfpathlineto{\pgfqpoint{2.085551in}{2.325511in}}%
\pgfpathlineto{\pgfqpoint{2.086116in}{2.388753in}}%
\pgfpathlineto{\pgfqpoint{2.086680in}{2.379194in}}%
\pgfpathlineto{\pgfqpoint{2.087244in}{2.389925in}}%
\pgfpathlineto{\pgfqpoint{2.088373in}{2.488778in}}%
\pgfpathlineto{\pgfqpoint{2.089502in}{2.512755in}}%
\pgfpathlineto{\pgfqpoint{2.090066in}{2.554454in}}%
\pgfpathlineto{\pgfqpoint{2.090630in}{2.554331in}}%
\pgfpathlineto{\pgfqpoint{2.095710in}{2.548746in}}%
\pgfpathlineto{\pgfqpoint{2.127313in}{2.514056in}}%
\pgfpathlineto{\pgfqpoint{2.127877in}{2.515185in}}%
\pgfpathlineto{\pgfqpoint{2.129006in}{2.540969in}}%
\pgfpathlineto{\pgfqpoint{2.129570in}{2.542958in}}%
\pgfpathlineto{\pgfqpoint{2.130135in}{2.530624in}}%
\pgfpathlineto{\pgfqpoint{2.130699in}{2.570180in}}%
\pgfpathlineto{\pgfqpoint{2.131263in}{2.522986in}}%
\pgfpathlineto{\pgfqpoint{2.131828in}{2.516771in}}%
\pgfpathlineto{\pgfqpoint{2.139729in}{2.520843in}}%
\pgfpathlineto{\pgfqpoint{2.140293in}{2.522356in}}%
\pgfpathlineto{\pgfqpoint{2.140857in}{2.529372in}}%
\pgfpathlineto{\pgfqpoint{2.141422in}{2.422498in}}%
\pgfpathlineto{\pgfqpoint{2.142550in}{2.596142in}}%
\pgfpathlineto{\pgfqpoint{2.143679in}{2.612828in}}%
\pgfpathlineto{\pgfqpoint{2.144243in}{2.637791in}}%
\pgfpathlineto{\pgfqpoint{2.145936in}{2.638842in}}%
\pgfpathlineto{\pgfqpoint{2.153837in}{2.644395in}}%
\pgfpathlineto{\pgfqpoint{2.154402in}{2.650064in}}%
\pgfpathlineto{\pgfqpoint{2.154966in}{2.673664in}}%
\pgfpathlineto{\pgfqpoint{2.155530in}{2.672010in}}%
\pgfpathlineto{\pgfqpoint{2.156095in}{2.692012in}}%
\pgfpathlineto{\pgfqpoint{2.156659in}{2.699694in}}%
\pgfpathlineto{\pgfqpoint{2.157223in}{2.725496in}}%
\pgfpathlineto{\pgfqpoint{2.157788in}{2.736505in}}%
\pgfpathlineto{\pgfqpoint{2.166817in}{2.721560in}}%
\pgfpathlineto{\pgfqpoint{2.167946in}{2.711921in}}%
\pgfpathlineto{\pgfqpoint{2.168510in}{2.721740in}}%
\pgfpathlineto{\pgfqpoint{2.169075in}{2.712894in}}%
\pgfpathlineto{\pgfqpoint{2.170203in}{2.719652in}}%
\pgfpathlineto{\pgfqpoint{2.170768in}{2.731304in}}%
\pgfpathlineto{\pgfqpoint{2.172461in}{2.731513in}}%
\pgfpathlineto{\pgfqpoint{2.180361in}{2.733063in}}%
\pgfpathlineto{\pgfqpoint{2.180926in}{2.726503in}}%
\pgfpathlineto{\pgfqpoint{2.181490in}{2.725468in}}%
\pgfpathlineto{\pgfqpoint{2.182054in}{2.731692in}}%
\pgfpathlineto{\pgfqpoint{2.182619in}{2.732171in}}%
\pgfpathlineto{\pgfqpoint{2.183183in}{2.742367in}}%
\pgfpathlineto{\pgfqpoint{2.183747in}{2.744867in}}%
\pgfpathlineto{\pgfqpoint{2.223816in}{2.752496in}}%
\pgfpathlineto{\pgfqpoint{2.224380in}{2.747678in}}%
\pgfpathlineto{\pgfqpoint{2.224945in}{2.739671in}}%
\pgfpathlineto{\pgfqpoint{2.226073in}{2.745082in}}%
\pgfpathlineto{\pgfqpoint{2.227766in}{2.745359in}}%
\pgfpathlineto{\pgfqpoint{2.235667in}{2.746084in}}%
\pgfpathlineto{\pgfqpoint{2.236232in}{2.748900in}}%
\pgfpathlineto{\pgfqpoint{2.236796in}{2.721542in}}%
\pgfpathlineto{\pgfqpoint{2.237925in}{2.737820in}}%
\pgfpathlineto{\pgfqpoint{2.238489in}{2.746450in}}%
\pgfpathlineto{\pgfqpoint{2.239053in}{2.760006in}}%
\pgfpathlineto{\pgfqpoint{2.239618in}{2.762542in}}%
\pgfpathlineto{\pgfqpoint{2.250905in}{2.774813in}}%
\pgfpathlineto{\pgfqpoint{2.251469in}{2.774824in}}%
\pgfpathlineto{\pgfqpoint{2.252598in}{2.773008in}}%
\pgfpathlineto{\pgfqpoint{2.257112in}{2.772972in}}%
\pgfpathlineto{\pgfqpoint{2.320319in}{2.773080in}}%
\pgfpathlineto{\pgfqpoint{2.320884in}{2.773867in}}%
\pgfpathlineto{\pgfqpoint{2.329349in}{2.773001in}}%
\pgfpathlineto{\pgfqpoint{2.329913in}{2.777082in}}%
\pgfpathlineto{\pgfqpoint{2.331042in}{2.773646in}}%
\pgfpathlineto{\pgfqpoint{2.331606in}{2.774483in}}%
\pgfpathlineto{\pgfqpoint{2.332735in}{2.785044in}}%
\pgfpathlineto{\pgfqpoint{2.333299in}{2.782230in}}%
\pgfpathlineto{\pgfqpoint{2.333863in}{2.780997in}}%
\pgfpathlineto{\pgfqpoint{2.334992in}{2.794859in}}%
\pgfpathlineto{\pgfqpoint{2.336121in}{2.782467in}}%
\pgfpathlineto{\pgfqpoint{2.345150in}{2.789267in}}%
\pgfpathlineto{\pgfqpoint{2.345715in}{2.783905in}}%
\pgfpathlineto{\pgfqpoint{2.346279in}{2.787144in}}%
\pgfpathlineto{\pgfqpoint{2.346843in}{2.788233in}}%
\pgfpathlineto{\pgfqpoint{2.347408in}{2.785637in}}%
\pgfpathlineto{\pgfqpoint{2.347972in}{2.791954in}}%
\pgfpathlineto{\pgfqpoint{2.348536in}{2.793343in}}%
\pgfpathlineto{\pgfqpoint{2.359259in}{2.793343in}}%
\pgfpathlineto{\pgfqpoint{2.359823in}{2.795654in}}%
\pgfpathlineto{\pgfqpoint{2.360388in}{2.784051in}}%
\pgfpathlineto{\pgfqpoint{2.361516in}{2.800753in}}%
\pgfpathlineto{\pgfqpoint{2.362645in}{2.801596in}}%
\pgfpathlineto{\pgfqpoint{2.369982in}{2.799029in}}%
\pgfpathlineto{\pgfqpoint{2.370546in}{2.796174in}}%
\pgfpathlineto{\pgfqpoint{2.371110in}{2.787968in}}%
\pgfpathlineto{\pgfqpoint{2.371675in}{2.788686in}}%
\pgfpathlineto{\pgfqpoint{2.372239in}{2.790236in}}%
\pgfpathlineto{\pgfqpoint{2.372803in}{2.790624in}}%
\pgfpathlineto{\pgfqpoint{2.373368in}{2.807708in}}%
\pgfpathlineto{\pgfqpoint{2.373932in}{2.794165in}}%
\pgfpathlineto{\pgfqpoint{2.374496in}{2.805957in}}%
\pgfpathlineto{\pgfqpoint{2.375061in}{2.844728in}}%
\pgfpathlineto{\pgfqpoint{2.412308in}{2.801044in}}%
\pgfpathlineto{\pgfqpoint{2.412872in}{2.801049in}}%
\pgfpathlineto{\pgfqpoint{2.413436in}{2.796094in}}%
\pgfpathlineto{\pgfqpoint{2.414001in}{2.787719in}}%
\pgfpathlineto{\pgfqpoint{2.414565in}{2.788584in}}%
\pgfpathlineto{\pgfqpoint{2.416258in}{2.829044in}}%
\pgfpathlineto{\pgfqpoint{2.416822in}{2.822841in}}%
\pgfpathlineto{\pgfqpoint{2.417387in}{2.805584in}}%
\pgfpathlineto{\pgfqpoint{2.426416in}{2.841141in}}%
\pgfpathlineto{\pgfqpoint{2.427545in}{2.784410in}}%
\pgfpathlineto{\pgfqpoint{2.428674in}{2.788320in}}%
\pgfpathlineto{\pgfqpoint{2.437703in}{2.780980in}}%
\pgfpathlineto{\pgfqpoint{2.438267in}{2.783779in}}%
\pgfpathlineto{\pgfqpoint{2.439960in}{2.843748in}}%
\pgfpathlineto{\pgfqpoint{2.440525in}{2.845513in}}%
\pgfpathlineto{\pgfqpoint{2.441089in}{2.846123in}}%
\pgfpathlineto{\pgfqpoint{2.441653in}{2.846123in}}%
\pgfpathlineto{\pgfqpoint{2.442218in}{2.830219in}}%
\pgfpathlineto{\pgfqpoint{2.442782in}{2.798600in}}%
\pgfpathlineto{\pgfqpoint{2.443347in}{2.794950in}}%
\pgfpathlineto{\pgfqpoint{2.452376in}{2.780439in}}%
\pgfpathlineto{\pgfqpoint{2.454633in}{2.780380in}}%
\pgfpathlineto{\pgfqpoint{2.455198in}{2.794232in}}%
\pgfpathlineto{\pgfqpoint{2.455762in}{2.830079in}}%
\pgfpathlineto{\pgfqpoint{2.456326in}{2.846079in}}%
\pgfpathlineto{\pgfqpoint{2.508811in}{2.838736in}}%
\pgfpathlineto{\pgfqpoint{2.509375in}{2.837600in}}%
\pgfpathlineto{\pgfqpoint{2.524048in}{2.790170in}}%
\pgfpathlineto{\pgfqpoint{2.524612in}{2.791963in}}%
\pgfpathlineto{\pgfqpoint{2.533077in}{2.836941in}}%
\pgfpathlineto{\pgfqpoint{2.533642in}{2.819960in}}%
\pgfpathlineto{\pgfqpoint{2.534206in}{2.782036in}}%
\pgfpathlineto{\pgfqpoint{2.537028in}{2.795316in}}%
\pgfpathlineto{\pgfqpoint{2.537592in}{2.793226in}}%
\pgfpathlineto{\pgfqpoint{2.538157in}{2.793527in}}%
\pgfpathlineto{\pgfqpoint{2.540978in}{2.818756in}}%
\pgfpathlineto{\pgfqpoint{2.545493in}{2.858192in}}%
\pgfpathlineto{\pgfqpoint{2.546057in}{2.860340in}}%
\pgfpathlineto{\pgfqpoint{2.546622in}{2.801668in}}%
\pgfpathlineto{\pgfqpoint{2.547186in}{2.796121in}}%
\pgfpathlineto{\pgfqpoint{2.547750in}{2.797548in}}%
\pgfpathlineto{\pgfqpoint{2.548315in}{2.854048in}}%
\pgfpathlineto{\pgfqpoint{2.548879in}{2.866494in}}%
\pgfpathlineto{\pgfqpoint{2.549444in}{2.836017in}}%
\pgfpathlineto{\pgfqpoint{2.550008in}{2.822391in}}%
\pgfpathlineto{\pgfqpoint{2.550572in}{2.784181in}}%
\pgfpathlineto{\pgfqpoint{2.551137in}{2.787220in}}%
\pgfpathlineto{\pgfqpoint{2.552265in}{2.798288in}}%
\pgfpathlineto{\pgfqpoint{2.559602in}{2.879412in}}%
\pgfpathlineto{\pgfqpoint{2.560730in}{2.827968in}}%
\pgfpathlineto{\pgfqpoint{2.561295in}{2.797423in}}%
\pgfpathlineto{\pgfqpoint{2.561859in}{2.788320in}}%
\pgfpathlineto{\pgfqpoint{2.562423in}{2.799060in}}%
\pgfpathlineto{\pgfqpoint{2.562988in}{2.883314in}}%
\pgfpathlineto{\pgfqpoint{2.563552in}{2.883110in}}%
\pgfpathlineto{\pgfqpoint{2.601363in}{2.796840in}}%
\pgfpathlineto{\pgfqpoint{2.601928in}{2.806865in}}%
\pgfpathlineto{\pgfqpoint{2.603621in}{2.879083in}}%
\pgfpathlineto{\pgfqpoint{2.604185in}{2.844260in}}%
\pgfpathlineto{\pgfqpoint{2.605314in}{2.892253in}}%
\pgfpathlineto{\pgfqpoint{2.607571in}{2.890731in}}%
\pgfpathlineto{\pgfqpoint{2.614343in}{2.885970in}}%
\pgfpathlineto{\pgfqpoint{2.614908in}{2.844428in}}%
\pgfpathlineto{\pgfqpoint{2.615472in}{2.876334in}}%
\pgfpathlineto{\pgfqpoint{2.616036in}{2.817037in}}%
\pgfpathlineto{\pgfqpoint{2.616601in}{2.874513in}}%
\pgfpathlineto{\pgfqpoint{2.617165in}{2.897747in}}%
\pgfpathlineto{\pgfqpoint{2.617729in}{2.900809in}}%
\pgfpathlineto{\pgfqpoint{2.618294in}{2.895225in}}%
\pgfpathlineto{\pgfqpoint{2.619422in}{2.903582in}}%
\pgfpathlineto{\pgfqpoint{2.628452in}{2.904415in}}%
\pgfpathlineto{\pgfqpoint{2.629581in}{2.902749in}}%
\pgfpathlineto{\pgfqpoint{2.633531in}{2.896519in}}%
\pgfpathlineto{\pgfqpoint{2.641432in}{2.886528in}}%
\pgfpathlineto{\pgfqpoint{2.641996in}{2.887063in}}%
\pgfpathlineto{\pgfqpoint{2.644254in}{2.905390in}}%
\pgfpathlineto{\pgfqpoint{2.645382in}{2.904845in}}%
\pgfpathlineto{\pgfqpoint{2.657798in}{2.922889in}}%
\pgfpathlineto{\pgfqpoint{2.710846in}{2.904557in}}%
\pgfpathlineto{\pgfqpoint{2.711411in}{2.908237in}}%
\pgfpathlineto{\pgfqpoint{2.712539in}{2.923088in}}%
\pgfpathlineto{\pgfqpoint{2.713668in}{2.926168in}}%
\pgfpathlineto{\pgfqpoint{2.718183in}{2.915225in}}%
\pgfpathlineto{\pgfqpoint{2.722133in}{2.905573in}}%
\pgfpathlineto{\pgfqpoint{2.722698in}{2.905003in}}%
\pgfpathlineto{\pgfqpoint{2.726084in}{2.916882in}}%
\pgfpathlineto{\pgfqpoint{2.726648in}{2.925044in}}%
\pgfpathlineto{\pgfqpoint{2.727212in}{2.949662in}}%
\pgfpathlineto{\pgfqpoint{2.737935in}{2.906378in}}%
\pgfpathlineto{\pgfqpoint{2.738499in}{2.907669in}}%
\pgfpathlineto{\pgfqpoint{2.740757in}{2.954521in}}%
\pgfpathlineto{\pgfqpoint{2.741321in}{2.966852in}}%
\pgfpathlineto{\pgfqpoint{2.748093in}{2.915453in}}%
\pgfpathlineto{\pgfqpoint{2.749222in}{2.906857in}}%
\pgfpathlineto{\pgfqpoint{2.749786in}{2.904695in}}%
\pgfpathlineto{\pgfqpoint{2.750915in}{2.905384in}}%
\pgfpathlineto{\pgfqpoint{2.753172in}{2.905306in}}%
\pgfpathlineto{\pgfqpoint{2.753737in}{2.904637in}}%
\pgfpathlineto{\pgfqpoint{2.754865in}{2.904483in}}%
\pgfpathlineto{\pgfqpoint{2.792677in}{2.905384in}}%
\pgfpathlineto{\pgfqpoint{2.793241in}{2.931704in}}%
\pgfpathlineto{\pgfqpoint{2.793805in}{2.971644in}}%
\pgfpathlineto{\pgfqpoint{2.794934in}{2.972907in}}%
\pgfpathlineto{\pgfqpoint{2.797191in}{2.972979in}}%
\pgfpathlineto{\pgfqpoint{2.806785in}{2.972979in}}%
\pgfpathlineto{\pgfqpoint{2.807350in}{2.969083in}}%
\pgfpathlineto{\pgfqpoint{2.807914in}{2.946688in}}%
\pgfpathlineto{\pgfqpoint{2.808478in}{2.938718in}}%
\pgfpathlineto{\pgfqpoint{2.818072in}{2.938707in}}%
\pgfpathlineto{\pgfqpoint{2.819201in}{2.936832in}}%
\pgfpathlineto{\pgfqpoint{2.834438in}{2.905824in}}%
\pgfpathlineto{\pgfqpoint{2.835002in}{2.905383in}}%
\pgfpathlineto{\pgfqpoint{2.888051in}{2.904470in}}%
\pgfpathlineto{\pgfqpoint{2.888615in}{2.906253in}}%
\pgfpathlineto{\pgfqpoint{2.889180in}{2.920960in}}%
\pgfpathlineto{\pgfqpoint{2.889744in}{2.905588in}}%
\pgfpathlineto{\pgfqpoint{2.900467in}{2.937612in}}%
\pgfpathlineto{\pgfqpoint{2.901031in}{2.918194in}}%
\pgfpathlineto{\pgfqpoint{2.901595in}{2.924512in}}%
\pgfpathlineto{\pgfqpoint{2.902160in}{2.917679in}}%
\pgfpathlineto{\pgfqpoint{2.902724in}{2.929119in}}%
\pgfpathlineto{\pgfqpoint{2.903288in}{2.928520in}}%
\pgfpathlineto{\pgfqpoint{2.903853in}{2.936868in}}%
\pgfpathlineto{\pgfqpoint{2.911753in}{2.906695in}}%
\pgfpathlineto{\pgfqpoint{2.912318in}{2.917637in}}%
\pgfpathlineto{\pgfqpoint{2.912882in}{2.915521in}}%
\pgfpathlineto{\pgfqpoint{2.913446in}{2.909085in}}%
\pgfpathlineto{\pgfqpoint{2.914575in}{2.939366in}}%
\pgfpathlineto{\pgfqpoint{2.915140in}{2.931223in}}%
\pgfpathlineto{\pgfqpoint{2.915704in}{2.940570in}}%
\pgfpathlineto{\pgfqpoint{2.917961in}{2.940570in}}%
\pgfpathlineto{\pgfqpoint{2.918526in}{2.939613in}}%
\pgfpathlineto{\pgfqpoint{2.925298in}{2.908439in}}%
\pgfpathlineto{\pgfqpoint{2.925862in}{2.914186in}}%
\pgfpathlineto{\pgfqpoint{2.926426in}{2.913320in}}%
\pgfpathlineto{\pgfqpoint{2.926991in}{2.941496in}}%
\pgfpathlineto{\pgfqpoint{2.927555in}{2.941255in}}%
\pgfpathlineto{\pgfqpoint{2.928119in}{2.941429in}}%
\pgfpathlineto{\pgfqpoint{2.929248in}{2.942746in}}%
\pgfpathlineto{\pgfqpoint{2.930377in}{2.946289in}}%
\pgfpathlineto{\pgfqpoint{2.930941in}{2.945291in}}%
\pgfpathlineto{\pgfqpoint{2.941664in}{2.947957in}}%
\pgfpathlineto{\pgfqpoint{2.942228in}{2.951682in}}%
\pgfpathlineto{\pgfqpoint{2.942792in}{2.951682in}}%
\pgfpathlineto{\pgfqpoint{2.943921in}{2.948016in}}%
\pgfpathlineto{\pgfqpoint{2.973831in}{2.955628in}}%
\pgfpathlineto{\pgfqpoint{2.983425in}{2.958071in}}%
\pgfpathlineto{\pgfqpoint{2.983990in}{2.957352in}}%
\pgfpathlineto{\pgfqpoint{2.984554in}{2.949774in}}%
\pgfpathlineto{\pgfqpoint{2.985118in}{2.947561in}}%
\pgfpathlineto{\pgfqpoint{2.995841in}{2.941010in}}%
\pgfpathlineto{\pgfqpoint{2.996405in}{2.941254in}}%
\pgfpathlineto{\pgfqpoint{2.996970in}{2.954798in}}%
\pgfpathlineto{\pgfqpoint{2.997534in}{2.958163in}}%
\pgfpathlineto{\pgfqpoint{3.007128in}{2.958163in}}%
\pgfpathlineto{\pgfqpoint{3.008257in}{2.940570in}}%
\pgfpathlineto{\pgfqpoint{3.009950in}{2.940570in}}%
\pgfpathlineto{\pgfqpoint{3.011078in}{2.944696in}}%
\pgfpathlineto{\pgfqpoint{3.012207in}{2.951221in}}%
\pgfpathlineto{\pgfqpoint{3.022365in}{2.960775in}}%
\pgfpathlineto{\pgfqpoint{3.022930in}{2.959015in}}%
\pgfpathlineto{\pgfqpoint{3.023494in}{2.954765in}}%
\pgfpathlineto{\pgfqpoint{3.024058in}{2.944737in}}%
\pgfpathlineto{\pgfqpoint{3.024623in}{2.940570in}}%
\pgfpathlineto{\pgfqpoint{3.028009in}{2.940570in}}%
\pgfpathlineto{\pgfqpoint{3.028573in}{2.941368in}}%
\pgfpathlineto{\pgfqpoint{3.033088in}{2.957395in}}%
\pgfpathlineto{\pgfqpoint{3.033652in}{2.960900in}}%
\pgfpathlineto{\pgfqpoint{3.034216in}{2.969124in}}%
\pgfpathlineto{\pgfqpoint{3.034781in}{2.972152in}}%
\pgfpathlineto{\pgfqpoint{3.036474in}{2.940570in}}%
\pgfpathlineto{\pgfqpoint{3.037603in}{2.940570in}}%
\pgfpathlineto{\pgfqpoint{3.038731in}{2.958780in}}%
\pgfpathlineto{\pgfqpoint{3.039860in}{2.959089in}}%
\pgfpathlineto{\pgfqpoint{3.075414in}{2.959221in}}%
\pgfpathlineto{\pgfqpoint{3.077107in}{2.960015in}}%
\pgfpathlineto{\pgfqpoint{3.078235in}{2.959505in}}%
\pgfpathlineto{\pgfqpoint{3.079364in}{2.959136in}}%
\pgfpathlineto{\pgfqpoint{3.087265in}{2.959980in}}%
\pgfpathlineto{\pgfqpoint{3.087829in}{2.959416in}}%
\pgfpathlineto{\pgfqpoint{3.088394in}{2.959910in}}%
\pgfpathlineto{\pgfqpoint{3.090651in}{2.959849in}}%
\pgfpathlineto{\pgfqpoint{3.091780in}{2.960015in}}%
\pgfpathlineto{\pgfqpoint{3.094037in}{2.960292in}}%
\pgfpathlineto{\pgfqpoint{3.183204in}{2.977510in}}%
\pgfpathlineto{\pgfqpoint{3.184332in}{2.969512in}}%
\pgfpathlineto{\pgfqpoint{3.184897in}{2.978534in}}%
\pgfpathlineto{\pgfqpoint{3.213114in}{2.978808in}}%
\pgfpathlineto{\pgfqpoint{3.213678in}{2.979536in}}%
\pgfpathlineto{\pgfqpoint{3.226658in}{2.984933in}}%
\pgfpathlineto{\pgfqpoint{3.228916in}{2.985016in}}%
\pgfpathlineto{\pgfqpoint{3.266163in}{2.985021in}}%
\pgfpathlineto{\pgfqpoint{3.266727in}{2.986732in}}%
\pgfpathlineto{\pgfqpoint{3.268420in}{2.995657in}}%
\pgfpathlineto{\pgfqpoint{3.268984in}{2.996208in}}%
\pgfpathlineto{\pgfqpoint{3.281964in}{2.998402in}}%
\pgfpathlineto{\pgfqpoint{3.294944in}{3.002609in}}%
\pgfpathlineto{\pgfqpoint{6.004368in}{3.002609in}}%
\pgfpathlineto{\pgfqpoint{6.004368in}{3.002609in}}%
\pgfusepath{stroke}%
\end{pgfscope}%
\begin{pgfscope}%
\pgfpathrectangle{\pgfqpoint{0.481681in}{1.080890in}}{\pgfqpoint{5.785672in}{2.146863in}}%
\pgfusepath{clip}%
\pgfsetrectcap%
\pgfsetroundjoin%
\pgfsetlinewidth{0.200750pt}%
\definecolor{currentstroke}{rgb}{0.580392,0.823529,0.741176}%
\pgfsetstrokecolor{currentstroke}%
\pgfsetdash{}{0pt}%
\pgfpathmoveto{\pgfqpoint{0.744666in}{1.178475in}}%
\pgfpathlineto{\pgfqpoint{0.798279in}{1.178475in}}%
\pgfpathlineto{\pgfqpoint{0.798843in}{1.179108in}}%
\pgfpathlineto{\pgfqpoint{0.799407in}{1.182787in}}%
\pgfpathlineto{\pgfqpoint{0.799972in}{1.181921in}}%
\pgfpathlineto{\pgfqpoint{0.800536in}{1.180050in}}%
\pgfpathlineto{\pgfqpoint{0.801101in}{1.179893in}}%
\pgfpathlineto{\pgfqpoint{0.801665in}{1.182764in}}%
\pgfpathlineto{\pgfqpoint{0.802229in}{1.179967in}}%
\pgfpathlineto{\pgfqpoint{0.802794in}{1.183352in}}%
\pgfpathlineto{\pgfqpoint{0.806180in}{1.183534in}}%
\pgfpathlineto{\pgfqpoint{0.814645in}{1.183627in}}%
\pgfpathlineto{\pgfqpoint{0.815209in}{1.188234in}}%
\pgfpathlineto{\pgfqpoint{0.815773in}{1.188785in}}%
\pgfpathlineto{\pgfqpoint{0.816902in}{1.183878in}}%
\pgfpathlineto{\pgfqpoint{0.817467in}{1.183534in}}%
\pgfpathlineto{\pgfqpoint{0.826496in}{1.183580in}}%
\pgfpathlineto{\pgfqpoint{0.827060in}{1.185939in}}%
\pgfpathlineto{\pgfqpoint{0.827625in}{1.192647in}}%
\pgfpathlineto{\pgfqpoint{0.828189in}{1.189088in}}%
\pgfpathlineto{\pgfqpoint{0.829318in}{1.196410in}}%
\pgfpathlineto{\pgfqpoint{0.829882in}{1.196092in}}%
\pgfpathlineto{\pgfqpoint{0.830446in}{1.346443in}}%
\pgfpathlineto{\pgfqpoint{0.831011in}{2.374952in}}%
\pgfpathlineto{\pgfqpoint{0.832704in}{2.169285in}}%
\pgfpathlineto{\pgfqpoint{0.840040in}{1.232390in}}%
\pgfpathlineto{\pgfqpoint{0.841169in}{2.180942in}}%
\pgfpathlineto{\pgfqpoint{0.841733in}{1.247443in}}%
\pgfpathlineto{\pgfqpoint{0.842298in}{2.490540in}}%
\pgfpathlineto{\pgfqpoint{0.842862in}{2.606595in}}%
\pgfpathlineto{\pgfqpoint{0.843426in}{2.906891in}}%
\pgfpathlineto{\pgfqpoint{0.843991in}{2.911103in}}%
\pgfpathlineto{\pgfqpoint{0.848506in}{2.908115in}}%
\pgfpathlineto{\pgfqpoint{0.853020in}{2.905083in}}%
\pgfpathlineto{\pgfqpoint{0.854149in}{2.902960in}}%
\pgfpathlineto{\pgfqpoint{0.854713in}{2.903670in}}%
\pgfpathlineto{\pgfqpoint{0.855278in}{2.911305in}}%
\pgfpathlineto{\pgfqpoint{0.860357in}{2.911273in}}%
\pgfpathlineto{\pgfqpoint{0.897604in}{2.911327in}}%
\pgfpathlineto{\pgfqpoint{0.898168in}{2.919446in}}%
\pgfpathlineto{\pgfqpoint{0.899297in}{2.919718in}}%
\pgfpathlineto{\pgfqpoint{0.907197in}{2.919779in}}%
\pgfpathlineto{\pgfqpoint{0.907762in}{2.919123in}}%
\pgfpathlineto{\pgfqpoint{0.908891in}{2.912568in}}%
\pgfpathlineto{\pgfqpoint{0.910584in}{2.918889in}}%
\pgfpathlineto{\pgfqpoint{0.911148in}{2.913334in}}%
\pgfpathlineto{\pgfqpoint{0.911712in}{2.916237in}}%
\pgfpathlineto{\pgfqpoint{0.912277in}{2.917594in}}%
\pgfpathlineto{\pgfqpoint{0.921870in}{2.917615in}}%
\pgfpathlineto{\pgfqpoint{0.922435in}{2.924639in}}%
\pgfpathlineto{\pgfqpoint{0.924128in}{3.101415in}}%
\pgfpathlineto{\pgfqpoint{0.924692in}{3.103371in}}%
\pgfpathlineto{\pgfqpoint{0.925257in}{3.104072in}}%
\pgfpathlineto{\pgfqpoint{0.931464in}{3.104099in}}%
\pgfpathlineto{\pgfqpoint{0.933722in}{3.104099in}}%
\pgfpathlineto{\pgfqpoint{0.934286in}{3.104440in}}%
\pgfpathlineto{\pgfqpoint{0.934850in}{3.125823in}}%
\pgfpathlineto{\pgfqpoint{0.935415in}{3.130169in}}%
\pgfpathlineto{\pgfqpoint{0.935979in}{3.130038in}}%
\pgfpathlineto{\pgfqpoint{0.936543in}{3.023991in}}%
\pgfpathlineto{\pgfqpoint{0.938801in}{1.372439in}}%
\pgfpathlineto{\pgfqpoint{0.939365in}{1.349036in}}%
\pgfpathlineto{\pgfqpoint{0.947830in}{1.349229in}}%
\pgfpathlineto{\pgfqpoint{0.949523in}{1.353955in}}%
\pgfpathlineto{\pgfqpoint{0.950088in}{1.353318in}}%
\pgfpathlineto{\pgfqpoint{0.950652in}{1.353537in}}%
\pgfpathlineto{\pgfqpoint{0.951216in}{1.356125in}}%
\pgfpathlineto{\pgfqpoint{0.991849in}{1.356195in}}%
\pgfpathlineto{\pgfqpoint{0.992414in}{1.357718in}}%
\pgfpathlineto{\pgfqpoint{0.992978in}{1.357767in}}%
\pgfpathlineto{\pgfqpoint{0.993542in}{1.360320in}}%
\pgfpathlineto{\pgfqpoint{0.994671in}{1.362156in}}%
\pgfpathlineto{\pgfqpoint{1.002572in}{1.357580in}}%
\pgfpathlineto{\pgfqpoint{1.005394in}{1.366000in}}%
\pgfpathlineto{\pgfqpoint{1.005958in}{1.374481in}}%
\pgfpathlineto{\pgfqpoint{1.006522in}{1.373726in}}%
\pgfpathlineto{\pgfqpoint{1.007087in}{1.375022in}}%
\pgfpathlineto{\pgfqpoint{1.007651in}{1.372378in}}%
\pgfpathlineto{\pgfqpoint{1.015552in}{1.390928in}}%
\pgfpathlineto{\pgfqpoint{1.016116in}{1.384447in}}%
\pgfpathlineto{\pgfqpoint{1.016681in}{1.373444in}}%
\pgfpathlineto{\pgfqpoint{1.017245in}{1.394901in}}%
\pgfpathlineto{\pgfqpoint{1.017809in}{1.433137in}}%
\pgfpathlineto{\pgfqpoint{1.018374in}{1.434040in}}%
\pgfpathlineto{\pgfqpoint{1.018938in}{1.434189in}}%
\pgfpathlineto{\pgfqpoint{1.019502in}{1.458194in}}%
\pgfpathlineto{\pgfqpoint{1.020067in}{1.503605in}}%
\pgfpathlineto{\pgfqpoint{1.020631in}{1.445510in}}%
\pgfpathlineto{\pgfqpoint{1.021195in}{1.502767in}}%
\pgfpathlineto{\pgfqpoint{1.042640in}{1.528379in}}%
\pgfpathlineto{\pgfqpoint{1.043205in}{1.456631in}}%
\pgfpathlineto{\pgfqpoint{1.043769in}{1.503251in}}%
\pgfpathlineto{\pgfqpoint{1.044334in}{1.456631in}}%
\pgfpathlineto{\pgfqpoint{1.044898in}{1.521351in}}%
\pgfpathlineto{\pgfqpoint{1.045462in}{1.519821in}}%
\pgfpathlineto{\pgfqpoint{1.046027in}{1.529633in}}%
\pgfpathlineto{\pgfqpoint{1.046591in}{1.517992in}}%
\pgfpathlineto{\pgfqpoint{1.047155in}{1.532909in}}%
\pgfpathlineto{\pgfqpoint{1.047720in}{1.536535in}}%
\pgfpathlineto{\pgfqpoint{1.085531in}{1.527156in}}%
\pgfpathlineto{\pgfqpoint{1.086659in}{1.527773in}}%
\pgfpathlineto{\pgfqpoint{1.090046in}{1.530435in}}%
\pgfpathlineto{\pgfqpoint{1.097382in}{1.537544in}}%
\pgfpathlineto{\pgfqpoint{1.097946in}{1.531516in}}%
\pgfpathlineto{\pgfqpoint{1.099075in}{1.548638in}}%
\pgfpathlineto{\pgfqpoint{1.100768in}{1.627856in}}%
\pgfpathlineto{\pgfqpoint{1.101332in}{1.544458in}}%
\pgfpathlineto{\pgfqpoint{1.101897in}{1.558918in}}%
\pgfpathlineto{\pgfqpoint{1.102461in}{1.547992in}}%
\pgfpathlineto{\pgfqpoint{1.109233in}{1.535148in}}%
\pgfpathlineto{\pgfqpoint{1.110362in}{1.533000in}}%
\pgfpathlineto{\pgfqpoint{1.110926in}{1.555395in}}%
\pgfpathlineto{\pgfqpoint{1.111491in}{1.614889in}}%
\pgfpathlineto{\pgfqpoint{1.112055in}{1.571031in}}%
\pgfpathlineto{\pgfqpoint{1.113184in}{1.620710in}}%
\pgfpathlineto{\pgfqpoint{1.113748in}{1.633827in}}%
\pgfpathlineto{\pgfqpoint{1.114312in}{1.616774in}}%
\pgfpathlineto{\pgfqpoint{1.114877in}{1.559503in}}%
\pgfpathlineto{\pgfqpoint{1.115441in}{1.605790in}}%
\pgfpathlineto{\pgfqpoint{1.116005in}{1.586211in}}%
\pgfpathlineto{\pgfqpoint{1.116570in}{1.585553in}}%
\pgfpathlineto{\pgfqpoint{1.123906in}{1.587432in}}%
\pgfpathlineto{\pgfqpoint{1.124471in}{1.595017in}}%
\pgfpathlineto{\pgfqpoint{1.125035in}{1.640229in}}%
\pgfpathlineto{\pgfqpoint{1.125599in}{1.648038in}}%
\pgfpathlineto{\pgfqpoint{1.126164in}{1.578142in}}%
\pgfpathlineto{\pgfqpoint{1.126728in}{1.604260in}}%
\pgfpathlineto{\pgfqpoint{1.127292in}{1.572145in}}%
\pgfpathlineto{\pgfqpoint{1.127857in}{1.603647in}}%
\pgfpathlineto{\pgfqpoint{1.128421in}{1.671431in}}%
\pgfpathlineto{\pgfqpoint{1.128985in}{1.605765in}}%
\pgfpathlineto{\pgfqpoint{1.129550in}{1.766238in}}%
\pgfpathlineto{\pgfqpoint{1.130114in}{1.776801in}}%
\pgfpathlineto{\pgfqpoint{1.138015in}{1.698218in}}%
\pgfpathlineto{\pgfqpoint{1.138579in}{1.678932in}}%
\pgfpathlineto{\pgfqpoint{1.139708in}{1.664128in}}%
\pgfpathlineto{\pgfqpoint{1.140272in}{1.656898in}}%
\pgfpathlineto{\pgfqpoint{1.140837in}{1.735422in}}%
\pgfpathlineto{\pgfqpoint{1.141401in}{1.773493in}}%
\pgfpathlineto{\pgfqpoint{1.141965in}{1.628050in}}%
\pgfpathlineto{\pgfqpoint{1.143658in}{1.631851in}}%
\pgfpathlineto{\pgfqpoint{1.179212in}{1.716887in}}%
\pgfpathlineto{\pgfqpoint{1.179776in}{1.713525in}}%
\pgfpathlineto{\pgfqpoint{1.180341in}{1.680921in}}%
\pgfpathlineto{\pgfqpoint{1.180905in}{1.674252in}}%
\pgfpathlineto{\pgfqpoint{1.181470in}{1.788042in}}%
\pgfpathlineto{\pgfqpoint{1.182034in}{1.767235in}}%
\pgfpathlineto{\pgfqpoint{1.182598in}{1.790224in}}%
\pgfpathlineto{\pgfqpoint{1.183163in}{1.791607in}}%
\pgfpathlineto{\pgfqpoint{1.183727in}{1.768377in}}%
\pgfpathlineto{\pgfqpoint{1.184291in}{1.801168in}}%
\pgfpathlineto{\pgfqpoint{1.192756in}{1.808321in}}%
\pgfpathlineto{\pgfqpoint{1.193321in}{1.793923in}}%
\pgfpathlineto{\pgfqpoint{1.193885in}{1.802380in}}%
\pgfpathlineto{\pgfqpoint{1.194449in}{1.842340in}}%
\pgfpathlineto{\pgfqpoint{1.195014in}{1.862139in}}%
\pgfpathlineto{\pgfqpoint{1.195578in}{1.819456in}}%
\pgfpathlineto{\pgfqpoint{1.196143in}{1.799240in}}%
\pgfpathlineto{\pgfqpoint{1.196707in}{1.731599in}}%
\pgfpathlineto{\pgfqpoint{1.197271in}{1.818307in}}%
\pgfpathlineto{\pgfqpoint{1.197836in}{1.825021in}}%
\pgfpathlineto{\pgfqpoint{1.205736in}{1.824815in}}%
\pgfpathlineto{\pgfqpoint{1.206301in}{1.741624in}}%
\pgfpathlineto{\pgfqpoint{1.206865in}{1.702759in}}%
\pgfpathlineto{\pgfqpoint{1.207429in}{1.814016in}}%
\pgfpathlineto{\pgfqpoint{1.207994in}{1.812650in}}%
\pgfpathlineto{\pgfqpoint{1.208558in}{1.828273in}}%
\pgfpathlineto{\pgfqpoint{1.209122in}{1.831354in}}%
\pgfpathlineto{\pgfqpoint{1.209687in}{1.888243in}}%
\pgfpathlineto{\pgfqpoint{1.210251in}{1.862855in}}%
\pgfpathlineto{\pgfqpoint{1.210815in}{1.908974in}}%
\pgfpathlineto{\pgfqpoint{1.211944in}{1.909124in}}%
\pgfpathlineto{\pgfqpoint{1.219281in}{1.909075in}}%
\pgfpathlineto{\pgfqpoint{1.219845in}{1.908691in}}%
\pgfpathlineto{\pgfqpoint{1.220409in}{1.908902in}}%
\pgfpathlineto{\pgfqpoint{1.220974in}{1.911708in}}%
\pgfpathlineto{\pgfqpoint{1.221538in}{1.917373in}}%
\pgfpathlineto{\pgfqpoint{1.222102in}{1.932780in}}%
\pgfpathlineto{\pgfqpoint{1.222667in}{1.936283in}}%
\pgfpathlineto{\pgfqpoint{1.223231in}{2.115069in}}%
\pgfpathlineto{\pgfqpoint{1.223795in}{1.989617in}}%
\pgfpathlineto{\pgfqpoint{1.224360in}{1.989543in}}%
\pgfpathlineto{\pgfqpoint{1.225488in}{2.019544in}}%
\pgfpathlineto{\pgfqpoint{1.232825in}{2.242299in}}%
\pgfpathlineto{\pgfqpoint{1.233954in}{2.250084in}}%
\pgfpathlineto{\pgfqpoint{1.235082in}{2.254113in}}%
\pgfpathlineto{\pgfqpoint{1.235647in}{2.261278in}}%
\pgfpathlineto{\pgfqpoint{1.236211in}{2.258125in}}%
\pgfpathlineto{\pgfqpoint{1.272894in}{1.907299in}}%
\pgfpathlineto{\pgfqpoint{1.273458in}{1.935454in}}%
\pgfpathlineto{\pgfqpoint{1.274022in}{2.124371in}}%
\pgfpathlineto{\pgfqpoint{1.274587in}{2.104127in}}%
\pgfpathlineto{\pgfqpoint{1.275151in}{2.222543in}}%
\pgfpathlineto{\pgfqpoint{1.275715in}{2.271765in}}%
\pgfpathlineto{\pgfqpoint{1.276280in}{2.256304in}}%
\pgfpathlineto{\pgfqpoint{1.276844in}{1.924836in}}%
\pgfpathlineto{\pgfqpoint{1.277408in}{1.935561in}}%
\pgfpathlineto{\pgfqpoint{1.277973in}{2.156594in}}%
\pgfpathlineto{\pgfqpoint{1.278537in}{2.137767in}}%
\pgfpathlineto{\pgfqpoint{1.279101in}{2.298635in}}%
\pgfpathlineto{\pgfqpoint{1.287002in}{1.915702in}}%
\pgfpathlineto{\pgfqpoint{1.287567in}{1.931730in}}%
\pgfpathlineto{\pgfqpoint{1.288131in}{1.908987in}}%
\pgfpathlineto{\pgfqpoint{1.288695in}{1.904863in}}%
\pgfpathlineto{\pgfqpoint{1.289260in}{2.089346in}}%
\pgfpathlineto{\pgfqpoint{1.289824in}{2.154260in}}%
\pgfpathlineto{\pgfqpoint{1.290388in}{1.990598in}}%
\pgfpathlineto{\pgfqpoint{1.290953in}{1.976666in}}%
\pgfpathlineto{\pgfqpoint{1.291517in}{2.250959in}}%
\pgfpathlineto{\pgfqpoint{1.292081in}{2.290204in}}%
\pgfpathlineto{\pgfqpoint{1.299982in}{2.268300in}}%
\pgfpathlineto{\pgfqpoint{1.300546in}{2.273081in}}%
\pgfpathlineto{\pgfqpoint{1.301111in}{2.290994in}}%
\pgfpathlineto{\pgfqpoint{1.301675in}{2.195533in}}%
\pgfpathlineto{\pgfqpoint{1.302239in}{2.293819in}}%
\pgfpathlineto{\pgfqpoint{1.302804in}{2.290848in}}%
\pgfpathlineto{\pgfqpoint{1.303368in}{2.291607in}}%
\pgfpathlineto{\pgfqpoint{1.303933in}{2.295500in}}%
\pgfpathlineto{\pgfqpoint{1.304497in}{2.308769in}}%
\pgfpathlineto{\pgfqpoint{1.305061in}{2.295657in}}%
\pgfpathlineto{\pgfqpoint{1.305626in}{1.919432in}}%
\pgfpathlineto{\pgfqpoint{1.307319in}{1.994100in}}%
\pgfpathlineto{\pgfqpoint{1.314091in}{2.300784in}}%
\pgfpathlineto{\pgfqpoint{1.314655in}{2.297181in}}%
\pgfpathlineto{\pgfqpoint{1.315219in}{2.301480in}}%
\pgfpathlineto{\pgfqpoint{1.315784in}{2.296878in}}%
\pgfpathlineto{\pgfqpoint{1.316348in}{2.307968in}}%
\pgfpathlineto{\pgfqpoint{1.316912in}{2.303449in}}%
\pgfpathlineto{\pgfqpoint{1.318041in}{2.303119in}}%
\pgfpathlineto{\pgfqpoint{1.318606in}{2.313848in}}%
\pgfpathlineto{\pgfqpoint{1.319170in}{2.313773in}}%
\pgfpathlineto{\pgfqpoint{1.368268in}{1.916368in}}%
\pgfpathlineto{\pgfqpoint{1.368832in}{1.957099in}}%
\pgfpathlineto{\pgfqpoint{1.369397in}{2.331347in}}%
\pgfpathlineto{\pgfqpoint{1.369961in}{2.331887in}}%
\pgfpathlineto{\pgfqpoint{1.370525in}{2.326770in}}%
\pgfpathlineto{\pgfqpoint{1.371090in}{2.334369in}}%
\pgfpathlineto{\pgfqpoint{1.372218in}{2.343110in}}%
\pgfpathlineto{\pgfqpoint{1.373347in}{2.350822in}}%
\pgfpathlineto{\pgfqpoint{1.373911in}{2.333629in}}%
\pgfpathlineto{\pgfqpoint{1.382941in}{1.897453in}}%
\pgfpathlineto{\pgfqpoint{1.383505in}{1.899630in}}%
\pgfpathlineto{\pgfqpoint{1.384070in}{1.793353in}}%
\pgfpathlineto{\pgfqpoint{1.384634in}{1.756625in}}%
\pgfpathlineto{\pgfqpoint{1.385198in}{1.793958in}}%
\pgfpathlineto{\pgfqpoint{1.386327in}{1.823568in}}%
\pgfpathlineto{\pgfqpoint{1.386891in}{1.826395in}}%
\pgfpathlineto{\pgfqpoint{1.394792in}{1.832742in}}%
\pgfpathlineto{\pgfqpoint{1.395357in}{1.863246in}}%
\pgfpathlineto{\pgfqpoint{1.395921in}{1.865061in}}%
\pgfpathlineto{\pgfqpoint{1.396485in}{1.834943in}}%
\pgfpathlineto{\pgfqpoint{1.397050in}{1.864303in}}%
\pgfpathlineto{\pgfqpoint{1.397614in}{1.875309in}}%
\pgfpathlineto{\pgfqpoint{1.398743in}{1.882041in}}%
\pgfpathlineto{\pgfqpoint{1.399307in}{1.899086in}}%
\pgfpathlineto{\pgfqpoint{1.399871in}{1.887230in}}%
\pgfpathlineto{\pgfqpoint{1.410594in}{1.900031in}}%
\pgfpathlineto{\pgfqpoint{1.411158in}{1.950763in}}%
\pgfpathlineto{\pgfqpoint{1.411723in}{2.175025in}}%
\pgfpathlineto{\pgfqpoint{1.412287in}{2.171662in}}%
\pgfpathlineto{\pgfqpoint{1.412851in}{2.128682in}}%
\pgfpathlineto{\pgfqpoint{1.413416in}{2.061944in}}%
\pgfpathlineto{\pgfqpoint{1.413980in}{2.171067in}}%
\pgfpathlineto{\pgfqpoint{1.414544in}{2.177507in}}%
\pgfpathlineto{\pgfqpoint{1.421881in}{2.177507in}}%
\pgfpathlineto{\pgfqpoint{1.423009in}{2.148496in}}%
\pgfpathlineto{\pgfqpoint{1.423574in}{2.138971in}}%
\pgfpathlineto{\pgfqpoint{1.424138in}{2.029907in}}%
\pgfpathlineto{\pgfqpoint{1.424703in}{2.087191in}}%
\pgfpathlineto{\pgfqpoint{1.425267in}{2.178418in}}%
\pgfpathlineto{\pgfqpoint{1.426396in}{2.180609in}}%
\pgfpathlineto{\pgfqpoint{1.465335in}{2.186919in}}%
\pgfpathlineto{\pgfqpoint{1.465900in}{2.163545in}}%
\pgfpathlineto{\pgfqpoint{1.466464in}{1.935962in}}%
\pgfpathlineto{\pgfqpoint{1.467028in}{2.156408in}}%
\pgfpathlineto{\pgfqpoint{1.467593in}{1.998862in}}%
\pgfpathlineto{\pgfqpoint{1.468157in}{2.185027in}}%
\pgfpathlineto{\pgfqpoint{1.470415in}{2.083706in}}%
\pgfpathlineto{\pgfqpoint{1.476622in}{1.794957in}}%
\pgfpathlineto{\pgfqpoint{1.477187in}{1.920550in}}%
\pgfpathlineto{\pgfqpoint{1.477751in}{1.822573in}}%
\pgfpathlineto{\pgfqpoint{1.478880in}{2.147300in}}%
\pgfpathlineto{\pgfqpoint{1.479444in}{2.124041in}}%
\pgfpathlineto{\pgfqpoint{1.480008in}{1.813690in}}%
\pgfpathlineto{\pgfqpoint{1.480573in}{1.915734in}}%
\pgfpathlineto{\pgfqpoint{1.481137in}{1.840546in}}%
\pgfpathlineto{\pgfqpoint{1.481701in}{1.835025in}}%
\pgfpathlineto{\pgfqpoint{1.482830in}{1.829728in}}%
\pgfpathlineto{\pgfqpoint{1.490731in}{1.788296in}}%
\pgfpathlineto{\pgfqpoint{1.491295in}{1.793313in}}%
\pgfpathlineto{\pgfqpoint{1.491860in}{1.875785in}}%
\pgfpathlineto{\pgfqpoint{1.492424in}{2.045187in}}%
\pgfpathlineto{\pgfqpoint{1.492988in}{1.802117in}}%
\pgfpathlineto{\pgfqpoint{1.493553in}{1.921530in}}%
\pgfpathlineto{\pgfqpoint{1.494117in}{1.800847in}}%
\pgfpathlineto{\pgfqpoint{1.494681in}{1.852637in}}%
\pgfpathlineto{\pgfqpoint{1.495246in}{1.836138in}}%
\pgfpathlineto{\pgfqpoint{1.495810in}{1.846661in}}%
\pgfpathlineto{\pgfqpoint{1.503711in}{1.809710in}}%
\pgfpathlineto{\pgfqpoint{1.504275in}{1.977830in}}%
\pgfpathlineto{\pgfqpoint{1.505404in}{1.840791in}}%
\pgfpathlineto{\pgfqpoint{1.506533in}{1.816814in}}%
\pgfpathlineto{\pgfqpoint{1.507097in}{1.945057in}}%
\pgfpathlineto{\pgfqpoint{1.507661in}{1.955689in}}%
\pgfpathlineto{\pgfqpoint{1.508226in}{1.861504in}}%
\pgfpathlineto{\pgfqpoint{1.517255in}{1.833654in}}%
\pgfpathlineto{\pgfqpoint{1.517820in}{1.820266in}}%
\pgfpathlineto{\pgfqpoint{1.518384in}{1.844763in}}%
\pgfpathlineto{\pgfqpoint{1.518948in}{1.848880in}}%
\pgfpathlineto{\pgfqpoint{1.519513in}{1.856923in}}%
\pgfpathlineto{\pgfqpoint{1.520077in}{1.819809in}}%
\pgfpathlineto{\pgfqpoint{1.520641in}{1.807422in}}%
\pgfpathlineto{\pgfqpoint{1.521206in}{1.809335in}}%
\pgfpathlineto{\pgfqpoint{1.521770in}{1.809859in}}%
\pgfpathlineto{\pgfqpoint{1.543215in}{1.793124in}}%
\pgfpathlineto{\pgfqpoint{1.557324in}{1.782013in}}%
\pgfpathlineto{\pgfqpoint{1.557888in}{1.780517in}}%
\pgfpathlineto{\pgfqpoint{1.558452in}{1.788694in}}%
\pgfpathlineto{\pgfqpoint{1.559017in}{1.792458in}}%
\pgfpathlineto{\pgfqpoint{1.559581in}{1.805447in}}%
\pgfpathlineto{\pgfqpoint{1.560145in}{1.804616in}}%
\pgfpathlineto{\pgfqpoint{1.560710in}{1.804807in}}%
\pgfpathlineto{\pgfqpoint{1.561274in}{1.799134in}}%
\pgfpathlineto{\pgfqpoint{1.561839in}{1.790387in}}%
\pgfpathlineto{\pgfqpoint{1.562403in}{1.825824in}}%
\pgfpathlineto{\pgfqpoint{1.571432in}{1.826767in}}%
\pgfpathlineto{\pgfqpoint{1.572561in}{1.827640in}}%
\pgfpathlineto{\pgfqpoint{1.573125in}{1.809910in}}%
\pgfpathlineto{\pgfqpoint{1.573690in}{1.810627in}}%
\pgfpathlineto{\pgfqpoint{1.575383in}{1.833416in}}%
\pgfpathlineto{\pgfqpoint{1.575947in}{1.834827in}}%
\pgfpathlineto{\pgfqpoint{1.576512in}{1.844887in}}%
\pgfpathlineto{\pgfqpoint{1.577076in}{1.847222in}}%
\pgfpathlineto{\pgfqpoint{1.584977in}{1.845986in}}%
\pgfpathlineto{\pgfqpoint{1.585541in}{1.818522in}}%
\pgfpathlineto{\pgfqpoint{1.586105in}{1.842072in}}%
\pgfpathlineto{\pgfqpoint{1.587234in}{1.837846in}}%
\pgfpathlineto{\pgfqpoint{1.587798in}{1.837499in}}%
\pgfpathlineto{\pgfqpoint{1.588363in}{1.849232in}}%
\pgfpathlineto{\pgfqpoint{1.588927in}{1.845329in}}%
\pgfpathlineto{\pgfqpoint{1.589491in}{1.852464in}}%
\pgfpathlineto{\pgfqpoint{1.590056in}{1.855265in}}%
\pgfpathlineto{\pgfqpoint{1.598521in}{1.784239in}}%
\pgfpathlineto{\pgfqpoint{1.599085in}{1.851733in}}%
\pgfpathlineto{\pgfqpoint{1.599650in}{1.854217in}}%
\pgfpathlineto{\pgfqpoint{1.600214in}{1.858365in}}%
\pgfpathlineto{\pgfqpoint{1.600778in}{1.879782in}}%
\pgfpathlineto{\pgfqpoint{1.601343in}{1.921177in}}%
\pgfpathlineto{\pgfqpoint{1.601907in}{1.917557in}}%
\pgfpathlineto{\pgfqpoint{1.603036in}{2.023726in}}%
\pgfpathlineto{\pgfqpoint{1.603600in}{2.041983in}}%
\pgfpathlineto{\pgfqpoint{1.612065in}{2.042141in}}%
\pgfpathlineto{\pgfqpoint{1.613194in}{2.042386in}}%
\pgfpathlineto{\pgfqpoint{1.698975in}{2.042187in}}%
\pgfpathlineto{\pgfqpoint{1.898753in}{2.012648in}}%
\pgfpathlineto{\pgfqpoint{1.899882in}{2.013436in}}%
\pgfpathlineto{\pgfqpoint{1.936564in}{2.044584in}}%
\pgfpathlineto{\pgfqpoint{1.937128in}{2.042592in}}%
\pgfpathlineto{\pgfqpoint{1.939950in}{2.044680in}}%
\pgfpathlineto{\pgfqpoint{1.940514in}{2.042768in}}%
\pgfpathlineto{\pgfqpoint{1.941643in}{2.015376in}}%
\pgfpathlineto{\pgfqpoint{1.943901in}{2.015510in}}%
\pgfpathlineto{\pgfqpoint{1.950673in}{2.015222in}}%
\pgfpathlineto{\pgfqpoint{1.951801in}{1.991150in}}%
\pgfpathlineto{\pgfqpoint{1.952366in}{1.990115in}}%
\pgfpathlineto{\pgfqpoint{1.952930in}{2.000443in}}%
\pgfpathlineto{\pgfqpoint{1.953494in}{2.016813in}}%
\pgfpathlineto{\pgfqpoint{1.954059in}{2.020066in}}%
\pgfpathlineto{\pgfqpoint{1.954623in}{2.016873in}}%
\pgfpathlineto{\pgfqpoint{1.955187in}{2.017696in}}%
\pgfpathlineto{\pgfqpoint{1.955752in}{2.020039in}}%
\pgfpathlineto{\pgfqpoint{1.956881in}{2.015595in}}%
\pgfpathlineto{\pgfqpoint{1.958574in}{2.016845in}}%
\pgfpathlineto{\pgfqpoint{1.963653in}{2.020993in}}%
\pgfpathlineto{\pgfqpoint{1.964781in}{2.024853in}}%
\pgfpathlineto{\pgfqpoint{1.965346in}{2.029957in}}%
\pgfpathlineto{\pgfqpoint{1.966474in}{2.023509in}}%
\pgfpathlineto{\pgfqpoint{1.967039in}{2.028479in}}%
\pgfpathlineto{\pgfqpoint{1.967603in}{2.022492in}}%
\pgfpathlineto{\pgfqpoint{1.968167in}{2.035358in}}%
\pgfpathlineto{\pgfqpoint{1.968732in}{2.029166in}}%
\pgfpathlineto{\pgfqpoint{1.969296in}{2.031531in}}%
\pgfpathlineto{\pgfqpoint{1.969860in}{2.032344in}}%
\pgfpathlineto{\pgfqpoint{1.977761in}{2.028461in}}%
\pgfpathlineto{\pgfqpoint{1.979454in}{2.094300in}}%
\pgfpathlineto{\pgfqpoint{1.980019in}{2.092025in}}%
\pgfpathlineto{\pgfqpoint{1.980583in}{2.100266in}}%
\pgfpathlineto{\pgfqpoint{1.981147in}{2.095995in}}%
\pgfpathlineto{\pgfqpoint{1.981712in}{2.052322in}}%
\pgfpathlineto{\pgfqpoint{1.982276in}{2.061808in}}%
\pgfpathlineto{\pgfqpoint{1.982840in}{2.031446in}}%
\pgfpathlineto{\pgfqpoint{1.983405in}{2.035381in}}%
\pgfpathlineto{\pgfqpoint{1.991306in}{2.028613in}}%
\pgfpathlineto{\pgfqpoint{1.991870in}{2.032338in}}%
\pgfpathlineto{\pgfqpoint{1.992999in}{2.112908in}}%
\pgfpathlineto{\pgfqpoint{1.993563in}{2.085138in}}%
\pgfpathlineto{\pgfqpoint{1.994127in}{2.082123in}}%
\pgfpathlineto{\pgfqpoint{1.994692in}{2.074125in}}%
\pgfpathlineto{\pgfqpoint{1.995256in}{2.050492in}}%
\pgfpathlineto{\pgfqpoint{2.030810in}{2.011859in}}%
\pgfpathlineto{\pgfqpoint{2.031374in}{2.043057in}}%
\pgfpathlineto{\pgfqpoint{2.031938in}{2.095246in}}%
\pgfpathlineto{\pgfqpoint{2.033067in}{2.131208in}}%
\pgfpathlineto{\pgfqpoint{2.033632in}{2.062147in}}%
\pgfpathlineto{\pgfqpoint{2.034196in}{2.061283in}}%
\pgfpathlineto{\pgfqpoint{2.034760in}{2.037057in}}%
\pgfpathlineto{\pgfqpoint{2.035325in}{2.047228in}}%
\pgfpathlineto{\pgfqpoint{2.035889in}{2.065581in}}%
\pgfpathlineto{\pgfqpoint{2.036453in}{2.060503in}}%
\pgfpathlineto{\pgfqpoint{2.037018in}{2.064284in}}%
\pgfpathlineto{\pgfqpoint{2.044918in}{2.146572in}}%
\pgfpathlineto{\pgfqpoint{2.046047in}{2.054772in}}%
\pgfpathlineto{\pgfqpoint{2.046611in}{2.057743in}}%
\pgfpathlineto{\pgfqpoint{2.047176in}{2.037809in}}%
\pgfpathlineto{\pgfqpoint{2.047740in}{2.062704in}}%
\pgfpathlineto{\pgfqpoint{2.048305in}{2.024009in}}%
\pgfpathlineto{\pgfqpoint{2.048869in}{2.064752in}}%
\pgfpathlineto{\pgfqpoint{2.049998in}{2.171507in}}%
\pgfpathlineto{\pgfqpoint{2.050562in}{2.173426in}}%
\pgfpathlineto{\pgfqpoint{2.057898in}{2.173375in}}%
\pgfpathlineto{\pgfqpoint{2.058463in}{2.136104in}}%
\pgfpathlineto{\pgfqpoint{2.059027in}{2.068629in}}%
\pgfpathlineto{\pgfqpoint{2.059591in}{2.053474in}}%
\pgfpathlineto{\pgfqpoint{2.060156in}{2.105657in}}%
\pgfpathlineto{\pgfqpoint{2.060720in}{2.095655in}}%
\pgfpathlineto{\pgfqpoint{2.061284in}{2.131227in}}%
\pgfpathlineto{\pgfqpoint{2.061849in}{2.092862in}}%
\pgfpathlineto{\pgfqpoint{2.062413in}{2.182149in}}%
\pgfpathlineto{\pgfqpoint{2.062978in}{2.072977in}}%
\pgfpathlineto{\pgfqpoint{2.063542in}{2.093313in}}%
\pgfpathlineto{\pgfqpoint{2.064106in}{2.227908in}}%
\pgfpathlineto{\pgfqpoint{2.064671in}{2.274871in}}%
\pgfpathlineto{\pgfqpoint{2.072007in}{2.264273in}}%
\pgfpathlineto{\pgfqpoint{2.072571in}{2.252301in}}%
\pgfpathlineto{\pgfqpoint{2.073136in}{2.113295in}}%
\pgfpathlineto{\pgfqpoint{2.073700in}{2.149280in}}%
\pgfpathlineto{\pgfqpoint{2.074264in}{2.258673in}}%
\pgfpathlineto{\pgfqpoint{2.074829in}{2.271447in}}%
\pgfpathlineto{\pgfqpoint{2.075957in}{2.072439in}}%
\pgfpathlineto{\pgfqpoint{2.077086in}{2.141468in}}%
\pgfpathlineto{\pgfqpoint{2.077651in}{2.134569in}}%
\pgfpathlineto{\pgfqpoint{2.085551in}{2.133332in}}%
\pgfpathlineto{\pgfqpoint{2.086116in}{2.167965in}}%
\pgfpathlineto{\pgfqpoint{2.086680in}{2.170437in}}%
\pgfpathlineto{\pgfqpoint{2.087244in}{2.175713in}}%
\pgfpathlineto{\pgfqpoint{2.087809in}{2.205181in}}%
\pgfpathlineto{\pgfqpoint{2.088373in}{2.263115in}}%
\pgfpathlineto{\pgfqpoint{2.088937in}{2.265808in}}%
\pgfpathlineto{\pgfqpoint{2.089502in}{2.257126in}}%
\pgfpathlineto{\pgfqpoint{2.090066in}{2.266977in}}%
\pgfpathlineto{\pgfqpoint{2.090630in}{2.262252in}}%
\pgfpathlineto{\pgfqpoint{2.091195in}{2.262840in}}%
\pgfpathlineto{\pgfqpoint{2.127313in}{2.327455in}}%
\pgfpathlineto{\pgfqpoint{2.127877in}{2.321212in}}%
\pgfpathlineto{\pgfqpoint{2.129006in}{2.277995in}}%
\pgfpathlineto{\pgfqpoint{2.130699in}{2.306825in}}%
\pgfpathlineto{\pgfqpoint{2.131263in}{2.327827in}}%
\pgfpathlineto{\pgfqpoint{2.131828in}{2.326078in}}%
\pgfpathlineto{\pgfqpoint{2.139729in}{2.328320in}}%
\pgfpathlineto{\pgfqpoint{2.140293in}{2.330569in}}%
\pgfpathlineto{\pgfqpoint{2.140857in}{2.342225in}}%
\pgfpathlineto{\pgfqpoint{2.141422in}{2.209446in}}%
\pgfpathlineto{\pgfqpoint{2.141986in}{2.233882in}}%
\pgfpathlineto{\pgfqpoint{2.142550in}{2.291535in}}%
\pgfpathlineto{\pgfqpoint{2.143115in}{2.295349in}}%
\pgfpathlineto{\pgfqpoint{2.143679in}{2.302814in}}%
\pgfpathlineto{\pgfqpoint{2.144243in}{2.327581in}}%
\pgfpathlineto{\pgfqpoint{2.147629in}{2.328868in}}%
\pgfpathlineto{\pgfqpoint{2.153273in}{2.331103in}}%
\pgfpathlineto{\pgfqpoint{2.153837in}{2.332249in}}%
\pgfpathlineto{\pgfqpoint{2.154402in}{2.341214in}}%
\pgfpathlineto{\pgfqpoint{2.154966in}{2.394801in}}%
\pgfpathlineto{\pgfqpoint{2.155530in}{2.389709in}}%
\pgfpathlineto{\pgfqpoint{2.156095in}{2.444931in}}%
\pgfpathlineto{\pgfqpoint{2.156659in}{2.470223in}}%
\pgfpathlineto{\pgfqpoint{2.157223in}{2.514563in}}%
\pgfpathlineto{\pgfqpoint{2.157788in}{2.517088in}}%
\pgfpathlineto{\pgfqpoint{2.158916in}{2.515963in}}%
\pgfpathlineto{\pgfqpoint{2.167381in}{2.509728in}}%
\pgfpathlineto{\pgfqpoint{2.167946in}{2.510946in}}%
\pgfpathlineto{\pgfqpoint{2.168510in}{2.518119in}}%
\pgfpathlineto{\pgfqpoint{2.169075in}{2.510933in}}%
\pgfpathlineto{\pgfqpoint{2.170203in}{2.520474in}}%
\pgfpathlineto{\pgfqpoint{2.170768in}{2.539920in}}%
\pgfpathlineto{\pgfqpoint{2.172461in}{2.540199in}}%
\pgfpathlineto{\pgfqpoint{2.180361in}{2.542262in}}%
\pgfpathlineto{\pgfqpoint{2.181490in}{2.521831in}}%
\pgfpathlineto{\pgfqpoint{2.183747in}{2.557327in}}%
\pgfpathlineto{\pgfqpoint{2.223816in}{2.570473in}}%
\pgfpathlineto{\pgfqpoint{2.224945in}{2.552894in}}%
\pgfpathlineto{\pgfqpoint{2.225509in}{2.560625in}}%
\pgfpathlineto{\pgfqpoint{2.226073in}{2.563499in}}%
\pgfpathlineto{\pgfqpoint{2.227766in}{2.564333in}}%
\pgfpathlineto{\pgfqpoint{2.235667in}{2.567453in}}%
\pgfpathlineto{\pgfqpoint{2.236232in}{2.566472in}}%
\pgfpathlineto{\pgfqpoint{2.236796in}{2.527599in}}%
\pgfpathlineto{\pgfqpoint{2.237360in}{2.537034in}}%
\pgfpathlineto{\pgfqpoint{2.239053in}{2.583624in}}%
\pgfpathlineto{\pgfqpoint{2.239618in}{2.586928in}}%
\pgfpathlineto{\pgfqpoint{2.250905in}{2.602918in}}%
\pgfpathlineto{\pgfqpoint{2.251469in}{2.602933in}}%
\pgfpathlineto{\pgfqpoint{2.252598in}{2.598541in}}%
\pgfpathlineto{\pgfqpoint{2.254855in}{2.598454in}}%
\pgfpathlineto{\pgfqpoint{2.319755in}{2.598467in}}%
\pgfpathlineto{\pgfqpoint{2.320319in}{2.599347in}}%
\pgfpathlineto{\pgfqpoint{2.320884in}{2.600921in}}%
\pgfpathlineto{\pgfqpoint{2.329349in}{2.598532in}}%
\pgfpathlineto{\pgfqpoint{2.329913in}{2.602164in}}%
\pgfpathlineto{\pgfqpoint{2.330477in}{2.599912in}}%
\pgfpathlineto{\pgfqpoint{2.331042in}{2.598894in}}%
\pgfpathlineto{\pgfqpoint{2.331606in}{2.599431in}}%
\pgfpathlineto{\pgfqpoint{2.332170in}{2.601002in}}%
\pgfpathlineto{\pgfqpoint{2.332735in}{2.603926in}}%
\pgfpathlineto{\pgfqpoint{2.333299in}{2.603638in}}%
\pgfpathlineto{\pgfqpoint{2.333863in}{2.605113in}}%
\pgfpathlineto{\pgfqpoint{2.334992in}{2.628799in}}%
\pgfpathlineto{\pgfqpoint{2.336121in}{2.620586in}}%
\pgfpathlineto{\pgfqpoint{2.344586in}{2.629292in}}%
\pgfpathlineto{\pgfqpoint{2.345150in}{2.629503in}}%
\pgfpathlineto{\pgfqpoint{2.345715in}{2.622089in}}%
\pgfpathlineto{\pgfqpoint{2.346279in}{2.625867in}}%
\pgfpathlineto{\pgfqpoint{2.346843in}{2.627220in}}%
\pgfpathlineto{\pgfqpoint{2.347408in}{2.623940in}}%
\pgfpathlineto{\pgfqpoint{2.347972in}{2.631699in}}%
\pgfpathlineto{\pgfqpoint{2.348536in}{2.633234in}}%
\pgfpathlineto{\pgfqpoint{2.359259in}{2.629090in}}%
\pgfpathlineto{\pgfqpoint{2.359823in}{2.630797in}}%
\pgfpathlineto{\pgfqpoint{2.360388in}{2.616163in}}%
\pgfpathlineto{\pgfqpoint{2.361516in}{2.636079in}}%
\pgfpathlineto{\pgfqpoint{2.362081in}{2.635614in}}%
\pgfpathlineto{\pgfqpoint{2.362645in}{2.635715in}}%
\pgfpathlineto{\pgfqpoint{2.369982in}{2.633243in}}%
\pgfpathlineto{\pgfqpoint{2.370546in}{2.629992in}}%
\pgfpathlineto{\pgfqpoint{2.371110in}{2.621433in}}%
\pgfpathlineto{\pgfqpoint{2.371675in}{2.622829in}}%
\pgfpathlineto{\pgfqpoint{2.372803in}{2.635681in}}%
\pgfpathlineto{\pgfqpoint{2.373368in}{2.678342in}}%
\pgfpathlineto{\pgfqpoint{2.373932in}{2.640728in}}%
\pgfpathlineto{\pgfqpoint{2.374496in}{2.642530in}}%
\pgfpathlineto{\pgfqpoint{2.375061in}{2.716476in}}%
\pgfpathlineto{\pgfqpoint{2.412872in}{2.634947in}}%
\pgfpathlineto{\pgfqpoint{2.413436in}{2.631608in}}%
\pgfpathlineto{\pgfqpoint{2.414001in}{2.619018in}}%
\pgfpathlineto{\pgfqpoint{2.414565in}{2.623411in}}%
\pgfpathlineto{\pgfqpoint{2.416258in}{2.720679in}}%
\pgfpathlineto{\pgfqpoint{2.416822in}{2.701519in}}%
\pgfpathlineto{\pgfqpoint{2.417387in}{2.654696in}}%
\pgfpathlineto{\pgfqpoint{2.426416in}{2.729632in}}%
\pgfpathlineto{\pgfqpoint{2.427545in}{2.609242in}}%
\pgfpathlineto{\pgfqpoint{2.428109in}{2.609357in}}%
\pgfpathlineto{\pgfqpoint{2.428674in}{2.611096in}}%
\pgfpathlineto{\pgfqpoint{2.437703in}{2.604247in}}%
\pgfpathlineto{\pgfqpoint{2.438267in}{2.610565in}}%
\pgfpathlineto{\pgfqpoint{2.439960in}{2.740120in}}%
\pgfpathlineto{\pgfqpoint{2.440525in}{2.741518in}}%
\pgfpathlineto{\pgfqpoint{2.441089in}{2.742001in}}%
\pgfpathlineto{\pgfqpoint{2.441653in}{2.742001in}}%
\pgfpathlineto{\pgfqpoint{2.442218in}{2.711473in}}%
\pgfpathlineto{\pgfqpoint{2.442782in}{2.636836in}}%
\pgfpathlineto{\pgfqpoint{2.443347in}{2.628835in}}%
\pgfpathlineto{\pgfqpoint{2.452376in}{2.603789in}}%
\pgfpathlineto{\pgfqpoint{2.454069in}{2.603687in}}%
\pgfpathlineto{\pgfqpoint{2.454633in}{2.603687in}}%
\pgfpathlineto{\pgfqpoint{2.455198in}{2.632911in}}%
\pgfpathlineto{\pgfqpoint{2.455762in}{2.708538in}}%
\pgfpathlineto{\pgfqpoint{2.456326in}{2.742355in}}%
\pgfpathlineto{\pgfqpoint{2.508811in}{2.737088in}}%
\pgfpathlineto{\pgfqpoint{2.509375in}{2.734193in}}%
\pgfpathlineto{\pgfqpoint{2.524048in}{2.610787in}}%
\pgfpathlineto{\pgfqpoint{2.524612in}{2.615462in}}%
\pgfpathlineto{\pgfqpoint{2.533077in}{2.732478in}}%
\pgfpathlineto{\pgfqpoint{2.533642in}{2.684892in}}%
\pgfpathlineto{\pgfqpoint{2.534206in}{2.606545in}}%
\pgfpathlineto{\pgfqpoint{2.537028in}{2.629745in}}%
\pgfpathlineto{\pgfqpoint{2.537592in}{2.630211in}}%
\pgfpathlineto{\pgfqpoint{2.538157in}{2.630239in}}%
\pgfpathlineto{\pgfqpoint{2.538721in}{2.636857in}}%
\pgfpathlineto{\pgfqpoint{2.545493in}{2.758566in}}%
\pgfpathlineto{\pgfqpoint{2.546057in}{2.762984in}}%
\pgfpathlineto{\pgfqpoint{2.546622in}{2.642269in}}%
\pgfpathlineto{\pgfqpoint{2.547186in}{2.630856in}}%
\pgfpathlineto{\pgfqpoint{2.547750in}{2.637151in}}%
\pgfpathlineto{\pgfqpoint{2.548315in}{2.747677in}}%
\pgfpathlineto{\pgfqpoint{2.548879in}{2.771499in}}%
\pgfpathlineto{\pgfqpoint{2.549444in}{2.712436in}}%
\pgfpathlineto{\pgfqpoint{2.550008in}{2.685543in}}%
\pgfpathlineto{\pgfqpoint{2.550572in}{2.609901in}}%
\pgfpathlineto{\pgfqpoint{2.551137in}{2.614823in}}%
\pgfpathlineto{\pgfqpoint{2.552265in}{2.635903in}}%
\pgfpathlineto{\pgfqpoint{2.559602in}{2.793997in}}%
\pgfpathlineto{\pgfqpoint{2.560730in}{2.693228in}}%
\pgfpathlineto{\pgfqpoint{2.561295in}{2.633634in}}%
\pgfpathlineto{\pgfqpoint{2.561859in}{2.616945in}}%
\pgfpathlineto{\pgfqpoint{2.562423in}{2.638520in}}%
\pgfpathlineto{\pgfqpoint{2.562988in}{2.799688in}}%
\pgfpathlineto{\pgfqpoint{2.563552in}{2.799344in}}%
\pgfpathlineto{\pgfqpoint{2.601363in}{2.632265in}}%
\pgfpathlineto{\pgfqpoint{2.601928in}{2.651568in}}%
\pgfpathlineto{\pgfqpoint{2.603621in}{2.790740in}}%
\pgfpathlineto{\pgfqpoint{2.604185in}{2.724778in}}%
\pgfpathlineto{\pgfqpoint{2.605314in}{2.817452in}}%
\pgfpathlineto{\pgfqpoint{2.607007in}{2.815206in}}%
\pgfpathlineto{\pgfqpoint{2.614343in}{2.804952in}}%
\pgfpathlineto{\pgfqpoint{2.614908in}{2.727588in}}%
\pgfpathlineto{\pgfqpoint{2.615472in}{2.787560in}}%
\pgfpathlineto{\pgfqpoint{2.616036in}{2.662902in}}%
\pgfpathlineto{\pgfqpoint{2.616601in}{2.769810in}}%
\pgfpathlineto{\pgfqpoint{2.617165in}{2.822053in}}%
\pgfpathlineto{\pgfqpoint{2.617729in}{2.823766in}}%
\pgfpathlineto{\pgfqpoint{2.618294in}{2.814195in}}%
\pgfpathlineto{\pgfqpoint{2.618858in}{2.812770in}}%
\pgfpathlineto{\pgfqpoint{2.619422in}{2.812652in}}%
\pgfpathlineto{\pgfqpoint{2.628452in}{2.832182in}}%
\pgfpathlineto{\pgfqpoint{2.632402in}{2.836330in}}%
\pgfpathlineto{\pgfqpoint{2.634660in}{2.829062in}}%
\pgfpathlineto{\pgfqpoint{2.641432in}{2.806462in}}%
\pgfpathlineto{\pgfqpoint{2.641996in}{2.806621in}}%
\pgfpathlineto{\pgfqpoint{2.643689in}{2.827397in}}%
\pgfpathlineto{\pgfqpoint{2.644254in}{2.842312in}}%
\pgfpathlineto{\pgfqpoint{2.644818in}{2.838846in}}%
\pgfpathlineto{\pgfqpoint{2.645382in}{2.833282in}}%
\pgfpathlineto{\pgfqpoint{2.657798in}{2.872604in}}%
\pgfpathlineto{\pgfqpoint{2.710282in}{2.833072in}}%
\pgfpathlineto{\pgfqpoint{2.710846in}{2.832647in}}%
\pgfpathlineto{\pgfqpoint{2.711411in}{2.842180in}}%
\pgfpathlineto{\pgfqpoint{2.712539in}{2.880004in}}%
\pgfpathlineto{\pgfqpoint{2.713104in}{2.878952in}}%
\pgfpathlineto{\pgfqpoint{2.713668in}{2.881635in}}%
\pgfpathlineto{\pgfqpoint{2.718183in}{2.856830in}}%
\pgfpathlineto{\pgfqpoint{2.722133in}{2.834957in}}%
\pgfpathlineto{\pgfqpoint{2.722698in}{2.833679in}}%
\pgfpathlineto{\pgfqpoint{2.726084in}{2.860898in}}%
\pgfpathlineto{\pgfqpoint{2.726648in}{2.882000in}}%
\pgfpathlineto{\pgfqpoint{2.727212in}{2.935500in}}%
\pgfpathlineto{\pgfqpoint{2.737935in}{2.836806in}}%
\pgfpathlineto{\pgfqpoint{2.738499in}{2.839396in}}%
\pgfpathlineto{\pgfqpoint{2.741321in}{2.987531in}}%
\pgfpathlineto{\pgfqpoint{2.745836in}{2.902737in}}%
\pgfpathlineto{\pgfqpoint{2.749786in}{2.833141in}}%
\pgfpathlineto{\pgfqpoint{2.750351in}{2.834142in}}%
\pgfpathlineto{\pgfqpoint{2.750915in}{2.834508in}}%
\pgfpathlineto{\pgfqpoint{2.753172in}{2.834333in}}%
\pgfpathlineto{\pgfqpoint{2.753737in}{2.832772in}}%
\pgfpathlineto{\pgfqpoint{2.754301in}{2.832274in}}%
\pgfpathlineto{\pgfqpoint{2.790983in}{2.834472in}}%
\pgfpathlineto{\pgfqpoint{2.791548in}{2.833645in}}%
\pgfpathlineto{\pgfqpoint{2.792112in}{2.833637in}}%
\pgfpathlineto{\pgfqpoint{2.792677in}{2.834508in}}%
\pgfpathlineto{\pgfqpoint{2.793241in}{2.902486in}}%
\pgfpathlineto{\pgfqpoint{2.793805in}{3.005152in}}%
\pgfpathlineto{\pgfqpoint{2.794934in}{3.007072in}}%
\pgfpathlineto{\pgfqpoint{2.796627in}{3.007181in}}%
\pgfpathlineto{\pgfqpoint{2.806785in}{3.007181in}}%
\pgfpathlineto{\pgfqpoint{2.807350in}{2.993252in}}%
\pgfpathlineto{\pgfqpoint{2.807914in}{2.920390in}}%
\pgfpathlineto{\pgfqpoint{2.808478in}{2.893104in}}%
\pgfpathlineto{\pgfqpoint{2.809043in}{2.892171in}}%
\pgfpathlineto{\pgfqpoint{2.818072in}{2.892152in}}%
\pgfpathlineto{\pgfqpoint{2.818636in}{2.890896in}}%
\pgfpathlineto{\pgfqpoint{2.834438in}{2.835269in}}%
\pgfpathlineto{\pgfqpoint{2.835002in}{2.834506in}}%
\pgfpathlineto{\pgfqpoint{2.887487in}{2.832480in}}%
\pgfpathlineto{\pgfqpoint{2.888051in}{2.832458in}}%
\pgfpathlineto{\pgfqpoint{2.888615in}{2.835630in}}%
\pgfpathlineto{\pgfqpoint{2.889180in}{2.782696in}}%
\pgfpathlineto{\pgfqpoint{2.889744in}{2.664238in}}%
\pgfpathlineto{\pgfqpoint{2.900467in}{2.891622in}}%
\pgfpathlineto{\pgfqpoint{2.901031in}{2.858288in}}%
\pgfpathlineto{\pgfqpoint{2.901595in}{2.870073in}}%
\pgfpathlineto{\pgfqpoint{2.902160in}{2.857327in}}%
\pgfpathlineto{\pgfqpoint{2.902724in}{2.878699in}}%
\pgfpathlineto{\pgfqpoint{2.903288in}{2.876400in}}%
\pgfpathlineto{\pgfqpoint{2.903853in}{2.891460in}}%
\pgfpathlineto{\pgfqpoint{2.911753in}{2.836880in}}%
\pgfpathlineto{\pgfqpoint{2.912318in}{2.839142in}}%
\pgfpathlineto{\pgfqpoint{2.912882in}{2.838327in}}%
\pgfpathlineto{\pgfqpoint{2.913446in}{2.835903in}}%
\pgfpathlineto{\pgfqpoint{2.914575in}{2.894974in}}%
\pgfpathlineto{\pgfqpoint{2.915140in}{2.877649in}}%
\pgfpathlineto{\pgfqpoint{2.915704in}{2.896480in}}%
\pgfpathlineto{\pgfqpoint{2.916268in}{2.896255in}}%
\pgfpathlineto{\pgfqpoint{2.916833in}{2.895664in}}%
\pgfpathlineto{\pgfqpoint{2.917397in}{2.895887in}}%
\pgfpathlineto{\pgfqpoint{2.917961in}{2.896469in}}%
\pgfpathlineto{\pgfqpoint{2.918526in}{2.896341in}}%
\pgfpathlineto{\pgfqpoint{2.925298in}{2.840038in}}%
\pgfpathlineto{\pgfqpoint{2.925862in}{2.850440in}}%
\pgfpathlineto{\pgfqpoint{2.926426in}{2.848109in}}%
\pgfpathlineto{\pgfqpoint{2.926991in}{2.900102in}}%
\pgfpathlineto{\pgfqpoint{2.927555in}{2.894596in}}%
\pgfpathlineto{\pgfqpoint{2.928119in}{2.895992in}}%
\pgfpathlineto{\pgfqpoint{2.928684in}{2.899626in}}%
\pgfpathlineto{\pgfqpoint{2.929248in}{2.897243in}}%
\pgfpathlineto{\pgfqpoint{2.930941in}{2.910328in}}%
\pgfpathlineto{\pgfqpoint{2.941099in}{2.906906in}}%
\pgfpathlineto{\pgfqpoint{2.941664in}{2.909186in}}%
\pgfpathlineto{\pgfqpoint{2.942228in}{2.918906in}}%
\pgfpathlineto{\pgfqpoint{2.942792in}{2.919566in}}%
\pgfpathlineto{\pgfqpoint{2.943921in}{2.906838in}}%
\pgfpathlineto{\pgfqpoint{2.956337in}{2.915848in}}%
\pgfpathlineto{\pgfqpoint{2.983425in}{2.935542in}}%
\pgfpathlineto{\pgfqpoint{2.983990in}{2.934137in}}%
\pgfpathlineto{\pgfqpoint{2.984554in}{2.912251in}}%
\pgfpathlineto{\pgfqpoint{2.985118in}{2.906172in}}%
\pgfpathlineto{\pgfqpoint{2.995841in}{2.897404in}}%
\pgfpathlineto{\pgfqpoint{2.996405in}{2.898267in}}%
\pgfpathlineto{\pgfqpoint{2.996970in}{2.929014in}}%
\pgfpathlineto{\pgfqpoint{2.997534in}{2.936631in}}%
\pgfpathlineto{\pgfqpoint{3.007128in}{2.936631in}}%
\pgfpathlineto{\pgfqpoint{3.008257in}{2.896816in}}%
\pgfpathlineto{\pgfqpoint{3.009950in}{2.896816in}}%
\pgfpathlineto{\pgfqpoint{3.011078in}{2.902278in}}%
\pgfpathlineto{\pgfqpoint{3.012207in}{2.911441in}}%
\pgfpathlineto{\pgfqpoint{3.022365in}{2.939488in}}%
\pgfpathlineto{\pgfqpoint{3.022930in}{2.939400in}}%
\pgfpathlineto{\pgfqpoint{3.023494in}{2.930337in}}%
\pgfpathlineto{\pgfqpoint{3.024058in}{2.906656in}}%
\pgfpathlineto{\pgfqpoint{3.024623in}{2.896816in}}%
\pgfpathlineto{\pgfqpoint{3.028009in}{2.896816in}}%
\pgfpathlineto{\pgfqpoint{3.028573in}{2.898621in}}%
\pgfpathlineto{\pgfqpoint{3.033652in}{2.937662in}}%
\pgfpathlineto{\pgfqpoint{3.034216in}{2.940760in}}%
\pgfpathlineto{\pgfqpoint{3.034781in}{2.938837in}}%
\pgfpathlineto{\pgfqpoint{3.036474in}{2.896816in}}%
\pgfpathlineto{\pgfqpoint{3.037603in}{2.896816in}}%
\pgfpathlineto{\pgfqpoint{3.038731in}{2.939299in}}%
\pgfpathlineto{\pgfqpoint{3.039296in}{2.940548in}}%
\pgfpathlineto{\pgfqpoint{3.076542in}{2.940580in}}%
\pgfpathlineto{\pgfqpoint{3.077107in}{2.939527in}}%
\pgfpathlineto{\pgfqpoint{3.077671in}{2.939278in}}%
\pgfpathlineto{\pgfqpoint{3.078800in}{2.940912in}}%
\pgfpathlineto{\pgfqpoint{3.079928in}{2.940878in}}%
\pgfpathlineto{\pgfqpoint{3.087265in}{2.939010in}}%
\pgfpathlineto{\pgfqpoint{3.087829in}{2.940157in}}%
\pgfpathlineto{\pgfqpoint{3.088394in}{2.939566in}}%
\pgfpathlineto{\pgfqpoint{3.090651in}{2.939638in}}%
\pgfpathlineto{\pgfqpoint{3.091780in}{2.939439in}}%
\pgfpathlineto{\pgfqpoint{3.096295in}{2.939708in}}%
\pgfpathlineto{\pgfqpoint{3.183204in}{2.946030in}}%
\pgfpathlineto{\pgfqpoint{3.183768in}{2.945609in}}%
\pgfpathlineto{\pgfqpoint{3.184332in}{2.946923in}}%
\pgfpathlineto{\pgfqpoint{3.184897in}{2.950475in}}%
\pgfpathlineto{\pgfqpoint{3.212550in}{2.950475in}}%
\pgfpathlineto{\pgfqpoint{3.213114in}{2.951660in}}%
\pgfpathlineto{\pgfqpoint{3.213678in}{2.954828in}}%
\pgfpathlineto{\pgfqpoint{3.223272in}{2.972545in}}%
\pgfpathlineto{\pgfqpoint{3.226658in}{2.978772in}}%
\pgfpathlineto{\pgfqpoint{3.227223in}{2.979142in}}%
\pgfpathlineto{\pgfqpoint{3.266163in}{2.979246in}}%
\pgfpathlineto{\pgfqpoint{3.266727in}{2.981363in}}%
\pgfpathlineto{\pgfqpoint{3.268420in}{2.992425in}}%
\pgfpathlineto{\pgfqpoint{3.268984in}{2.993246in}}%
\pgfpathlineto{\pgfqpoint{3.281400in}{2.999004in}}%
\pgfpathlineto{\pgfqpoint{3.295509in}{3.001278in}}%
\pgfpathlineto{\pgfqpoint{4.535940in}{3.001555in}}%
\pgfpathlineto{\pgfqpoint{5.105365in}{3.002609in}}%
\pgfpathlineto{\pgfqpoint{6.004368in}{3.002609in}}%
\pgfpathlineto{\pgfqpoint{6.004368in}{3.002609in}}%
\pgfusepath{stroke}%
\end{pgfscope}%
\begin{pgfscope}%
\pgfpathrectangle{\pgfqpoint{0.481681in}{1.080890in}}{\pgfqpoint{5.785672in}{2.146863in}}%
\pgfusepath{clip}%
\pgfsetrectcap%
\pgfsetroundjoin%
\pgfsetlinewidth{0.200750pt}%
\definecolor{currentstroke}{rgb}{0.933333,0.607843,0.000000}%
\pgfsetstrokecolor{currentstroke}%
\pgfsetdash{}{0pt}%
\pgfpathmoveto{\pgfqpoint{0.744666in}{1.178475in}}%
\pgfpathlineto{\pgfqpoint{0.798279in}{1.178475in}}%
\pgfpathlineto{\pgfqpoint{0.798843in}{1.183112in}}%
\pgfpathlineto{\pgfqpoint{0.799407in}{1.210110in}}%
\pgfpathlineto{\pgfqpoint{0.799972in}{1.203789in}}%
\pgfpathlineto{\pgfqpoint{0.800536in}{1.190044in}}%
\pgfpathlineto{\pgfqpoint{0.801101in}{1.188888in}}%
\pgfpathlineto{\pgfqpoint{0.801665in}{1.209994in}}%
\pgfpathlineto{\pgfqpoint{0.802229in}{1.189893in}}%
\pgfpathlineto{\pgfqpoint{0.803358in}{1.244659in}}%
\pgfpathlineto{\pgfqpoint{0.803922in}{1.250383in}}%
\pgfpathlineto{\pgfqpoint{0.811823in}{1.250383in}}%
\pgfpathlineto{\pgfqpoint{0.812387in}{1.318069in}}%
\pgfpathlineto{\pgfqpoint{0.814080in}{1.961451in}}%
\pgfpathlineto{\pgfqpoint{0.814645in}{1.915962in}}%
\pgfpathlineto{\pgfqpoint{0.815209in}{1.258741in}}%
\pgfpathlineto{\pgfqpoint{0.815773in}{1.259858in}}%
\pgfpathlineto{\pgfqpoint{0.816902in}{1.251945in}}%
\pgfpathlineto{\pgfqpoint{0.817467in}{1.251389in}}%
\pgfpathlineto{\pgfqpoint{0.825932in}{1.251350in}}%
\pgfpathlineto{\pgfqpoint{0.826496in}{1.881498in}}%
\pgfpathlineto{\pgfqpoint{0.827060in}{2.040176in}}%
\pgfpathlineto{\pgfqpoint{0.827625in}{1.548088in}}%
\pgfpathlineto{\pgfqpoint{0.828189in}{1.814942in}}%
\pgfpathlineto{\pgfqpoint{0.828753in}{2.228630in}}%
\pgfpathlineto{\pgfqpoint{0.829318in}{2.232448in}}%
\pgfpathlineto{\pgfqpoint{0.829882in}{2.168341in}}%
\pgfpathlineto{\pgfqpoint{0.830446in}{1.587600in}}%
\pgfpathlineto{\pgfqpoint{0.831011in}{2.221726in}}%
\pgfpathlineto{\pgfqpoint{0.831575in}{2.231480in}}%
\pgfpathlineto{\pgfqpoint{0.840605in}{2.232139in}}%
\pgfpathlineto{\pgfqpoint{0.841169in}{1.946931in}}%
\pgfpathlineto{\pgfqpoint{0.841733in}{1.318268in}}%
\pgfpathlineto{\pgfqpoint{0.842298in}{1.651050in}}%
\pgfpathlineto{\pgfqpoint{0.842862in}{1.299491in}}%
\pgfpathlineto{\pgfqpoint{0.843426in}{1.301228in}}%
\pgfpathlineto{\pgfqpoint{0.843991in}{1.306487in}}%
\pgfpathlineto{\pgfqpoint{0.847941in}{1.303622in}}%
\pgfpathlineto{\pgfqpoint{0.853020in}{1.299868in}}%
\pgfpathlineto{\pgfqpoint{0.853585in}{1.297357in}}%
\pgfpathlineto{\pgfqpoint{0.854149in}{1.293063in}}%
\pgfpathlineto{\pgfqpoint{0.854713in}{1.294428in}}%
\pgfpathlineto{\pgfqpoint{0.855278in}{1.307519in}}%
\pgfpathlineto{\pgfqpoint{0.859792in}{1.307505in}}%
\pgfpathlineto{\pgfqpoint{0.895346in}{1.307737in}}%
\pgfpathlineto{\pgfqpoint{0.895911in}{1.313956in}}%
\pgfpathlineto{\pgfqpoint{0.897039in}{1.332491in}}%
\pgfpathlineto{\pgfqpoint{0.897604in}{1.315269in}}%
\pgfpathlineto{\pgfqpoint{0.898168in}{1.349456in}}%
\pgfpathlineto{\pgfqpoint{0.898732in}{1.349880in}}%
\pgfpathlineto{\pgfqpoint{0.907197in}{1.349700in}}%
\pgfpathlineto{\pgfqpoint{0.907762in}{1.354681in}}%
\pgfpathlineto{\pgfqpoint{0.908891in}{1.399343in}}%
\pgfpathlineto{\pgfqpoint{0.909455in}{1.349734in}}%
\pgfpathlineto{\pgfqpoint{0.910584in}{1.356447in}}%
\pgfpathlineto{\pgfqpoint{0.911148in}{1.338647in}}%
\pgfpathlineto{\pgfqpoint{0.911712in}{1.347533in}}%
\pgfpathlineto{\pgfqpoint{0.912277in}{1.346223in}}%
\pgfpathlineto{\pgfqpoint{0.921870in}{1.346152in}}%
\pgfpathlineto{\pgfqpoint{0.922999in}{1.365401in}}%
\pgfpathlineto{\pgfqpoint{0.923564in}{1.395011in}}%
\pgfpathlineto{\pgfqpoint{0.924128in}{1.402038in}}%
\pgfpathlineto{\pgfqpoint{0.924692in}{1.401914in}}%
\pgfpathlineto{\pgfqpoint{0.925257in}{1.403683in}}%
\pgfpathlineto{\pgfqpoint{0.927514in}{1.403772in}}%
\pgfpathlineto{\pgfqpoint{0.933722in}{1.403772in}}%
\pgfpathlineto{\pgfqpoint{0.934286in}{1.404116in}}%
\pgfpathlineto{\pgfqpoint{0.934850in}{1.425659in}}%
\pgfpathlineto{\pgfqpoint{0.935415in}{1.430761in}}%
\pgfpathlineto{\pgfqpoint{0.935979in}{1.430693in}}%
\pgfpathlineto{\pgfqpoint{0.936543in}{1.434600in}}%
\pgfpathlineto{\pgfqpoint{0.938801in}{1.495686in}}%
\pgfpathlineto{\pgfqpoint{0.939365in}{1.496550in}}%
\pgfpathlineto{\pgfqpoint{0.947830in}{1.496744in}}%
\pgfpathlineto{\pgfqpoint{0.949523in}{1.501653in}}%
\pgfpathlineto{\pgfqpoint{0.950088in}{1.500984in}}%
\pgfpathlineto{\pgfqpoint{0.950652in}{1.499297in}}%
\pgfpathlineto{\pgfqpoint{0.951216in}{1.500817in}}%
\pgfpathlineto{\pgfqpoint{0.990721in}{1.500817in}}%
\pgfpathlineto{\pgfqpoint{0.991285in}{1.501698in}}%
\pgfpathlineto{\pgfqpoint{0.991849in}{1.506154in}}%
\pgfpathlineto{\pgfqpoint{0.992414in}{1.505271in}}%
\pgfpathlineto{\pgfqpoint{0.992978in}{1.508236in}}%
\pgfpathlineto{\pgfqpoint{0.993542in}{1.516957in}}%
\pgfpathlineto{\pgfqpoint{0.994671in}{1.519300in}}%
\pgfpathlineto{\pgfqpoint{1.003136in}{1.510730in}}%
\pgfpathlineto{\pgfqpoint{1.003701in}{1.515203in}}%
\pgfpathlineto{\pgfqpoint{1.004265in}{1.510764in}}%
\pgfpathlineto{\pgfqpoint{1.004829in}{1.527569in}}%
\pgfpathlineto{\pgfqpoint{1.005958in}{1.533254in}}%
\pgfpathlineto{\pgfqpoint{1.006522in}{1.527891in}}%
\pgfpathlineto{\pgfqpoint{1.007087in}{1.535896in}}%
\pgfpathlineto{\pgfqpoint{1.007651in}{1.518754in}}%
\pgfpathlineto{\pgfqpoint{1.015552in}{1.576131in}}%
\pgfpathlineto{\pgfqpoint{1.016116in}{1.564640in}}%
\pgfpathlineto{\pgfqpoint{1.016681in}{1.543008in}}%
\pgfpathlineto{\pgfqpoint{1.017245in}{1.566113in}}%
\pgfpathlineto{\pgfqpoint{1.017809in}{1.629675in}}%
\pgfpathlineto{\pgfqpoint{1.018374in}{1.630500in}}%
\pgfpathlineto{\pgfqpoint{1.018938in}{1.630390in}}%
\pgfpathlineto{\pgfqpoint{1.019502in}{1.636775in}}%
\pgfpathlineto{\pgfqpoint{1.020067in}{1.649395in}}%
\pgfpathlineto{\pgfqpoint{1.020631in}{1.602256in}}%
\pgfpathlineto{\pgfqpoint{1.021195in}{1.653879in}}%
\pgfpathlineto{\pgfqpoint{1.042640in}{1.688781in}}%
\pgfpathlineto{\pgfqpoint{1.043205in}{1.617962in}}%
\pgfpathlineto{\pgfqpoint{1.043769in}{1.661569in}}%
\pgfpathlineto{\pgfqpoint{1.044334in}{1.617818in}}%
\pgfpathlineto{\pgfqpoint{1.044898in}{1.679926in}}%
\pgfpathlineto{\pgfqpoint{1.045462in}{1.676890in}}%
\pgfpathlineto{\pgfqpoint{1.046027in}{1.686449in}}%
\pgfpathlineto{\pgfqpoint{1.046591in}{1.674281in}}%
\pgfpathlineto{\pgfqpoint{1.047155in}{1.687361in}}%
\pgfpathlineto{\pgfqpoint{1.047720in}{1.691605in}}%
\pgfpathlineto{\pgfqpoint{1.085531in}{1.686329in}}%
\pgfpathlineto{\pgfqpoint{1.086659in}{1.687210in}}%
\pgfpathlineto{\pgfqpoint{1.089481in}{1.689630in}}%
\pgfpathlineto{\pgfqpoint{1.097382in}{1.693412in}}%
\pgfpathlineto{\pgfqpoint{1.097946in}{1.687598in}}%
\pgfpathlineto{\pgfqpoint{1.099075in}{1.692046in}}%
\pgfpathlineto{\pgfqpoint{1.099639in}{1.698632in}}%
\pgfpathlineto{\pgfqpoint{1.100768in}{1.701652in}}%
\pgfpathlineto{\pgfqpoint{1.101332in}{1.691465in}}%
\pgfpathlineto{\pgfqpoint{1.102461in}{1.704453in}}%
\pgfpathlineto{\pgfqpoint{1.103590in}{1.702224in}}%
\pgfpathlineto{\pgfqpoint{1.110362in}{1.687135in}}%
\pgfpathlineto{\pgfqpoint{1.110926in}{1.692653in}}%
\pgfpathlineto{\pgfqpoint{1.111491in}{1.708942in}}%
\pgfpathlineto{\pgfqpoint{1.112055in}{1.703241in}}%
\pgfpathlineto{\pgfqpoint{1.113184in}{1.715052in}}%
\pgfpathlineto{\pgfqpoint{1.114312in}{1.726567in}}%
\pgfpathlineto{\pgfqpoint{1.114877in}{1.701193in}}%
\pgfpathlineto{\pgfqpoint{1.115441in}{1.705945in}}%
\pgfpathlineto{\pgfqpoint{1.116005in}{1.703489in}}%
\pgfpathlineto{\pgfqpoint{1.116570in}{1.703945in}}%
\pgfpathlineto{\pgfqpoint{1.123906in}{1.702018in}}%
\pgfpathlineto{\pgfqpoint{1.124471in}{1.703721in}}%
\pgfpathlineto{\pgfqpoint{1.125035in}{1.725359in}}%
\pgfpathlineto{\pgfqpoint{1.125599in}{1.733908in}}%
\pgfpathlineto{\pgfqpoint{1.126728in}{1.717438in}}%
\pgfpathlineto{\pgfqpoint{1.127292in}{1.715844in}}%
\pgfpathlineto{\pgfqpoint{1.127857in}{1.709271in}}%
\pgfpathlineto{\pgfqpoint{1.128421in}{1.738114in}}%
\pgfpathlineto{\pgfqpoint{1.128985in}{1.729812in}}%
\pgfpathlineto{\pgfqpoint{1.129550in}{1.782459in}}%
\pgfpathlineto{\pgfqpoint{1.130114in}{1.790054in}}%
\pgfpathlineto{\pgfqpoint{1.138015in}{1.754442in}}%
\pgfpathlineto{\pgfqpoint{1.138579in}{1.749434in}}%
\pgfpathlineto{\pgfqpoint{1.139144in}{1.747721in}}%
\pgfpathlineto{\pgfqpoint{1.140272in}{1.754589in}}%
\pgfpathlineto{\pgfqpoint{1.141401in}{1.793233in}}%
\pgfpathlineto{\pgfqpoint{1.141965in}{1.761056in}}%
\pgfpathlineto{\pgfqpoint{1.145916in}{1.765540in}}%
\pgfpathlineto{\pgfqpoint{1.179212in}{1.803885in}}%
\pgfpathlineto{\pgfqpoint{1.179776in}{1.802318in}}%
\pgfpathlineto{\pgfqpoint{1.180341in}{1.786996in}}%
\pgfpathlineto{\pgfqpoint{1.180905in}{1.779458in}}%
\pgfpathlineto{\pgfqpoint{1.181470in}{1.813671in}}%
\pgfpathlineto{\pgfqpoint{1.182034in}{1.808774in}}%
\pgfpathlineto{\pgfqpoint{1.182598in}{1.815659in}}%
\pgfpathlineto{\pgfqpoint{1.183163in}{1.816818in}}%
\pgfpathlineto{\pgfqpoint{1.183727in}{1.815030in}}%
\pgfpathlineto{\pgfqpoint{1.184291in}{1.826106in}}%
\pgfpathlineto{\pgfqpoint{1.192756in}{1.828088in}}%
\pgfpathlineto{\pgfqpoint{1.193321in}{1.819138in}}%
\pgfpathlineto{\pgfqpoint{1.194449in}{1.842665in}}%
\pgfpathlineto{\pgfqpoint{1.195014in}{1.845108in}}%
\pgfpathlineto{\pgfqpoint{1.195578in}{1.827365in}}%
\pgfpathlineto{\pgfqpoint{1.196143in}{1.824728in}}%
\pgfpathlineto{\pgfqpoint{1.196707in}{1.813004in}}%
\pgfpathlineto{\pgfqpoint{1.197271in}{1.847852in}}%
\pgfpathlineto{\pgfqpoint{1.197836in}{1.850136in}}%
\pgfpathlineto{\pgfqpoint{1.205736in}{1.850136in}}%
\pgfpathlineto{\pgfqpoint{1.206301in}{1.812486in}}%
\pgfpathlineto{\pgfqpoint{1.206865in}{1.794341in}}%
\pgfpathlineto{\pgfqpoint{1.207429in}{1.844193in}}%
\pgfpathlineto{\pgfqpoint{1.207994in}{1.843670in}}%
\pgfpathlineto{\pgfqpoint{1.208558in}{1.840519in}}%
\pgfpathlineto{\pgfqpoint{1.209122in}{1.841826in}}%
\pgfpathlineto{\pgfqpoint{1.209687in}{1.877642in}}%
\pgfpathlineto{\pgfqpoint{1.210251in}{1.870060in}}%
\pgfpathlineto{\pgfqpoint{1.210815in}{1.919379in}}%
\pgfpathlineto{\pgfqpoint{1.211380in}{1.922423in}}%
\pgfpathlineto{\pgfqpoint{1.219281in}{1.922457in}}%
\pgfpathlineto{\pgfqpoint{1.220409in}{1.923396in}}%
\pgfpathlineto{\pgfqpoint{1.220974in}{1.928551in}}%
\pgfpathlineto{\pgfqpoint{1.221538in}{1.940663in}}%
\pgfpathlineto{\pgfqpoint{1.222667in}{1.989196in}}%
\pgfpathlineto{\pgfqpoint{1.223231in}{2.054828in}}%
\pgfpathlineto{\pgfqpoint{1.223795in}{2.014867in}}%
\pgfpathlineto{\pgfqpoint{1.224360in}{2.018040in}}%
\pgfpathlineto{\pgfqpoint{1.231696in}{2.092968in}}%
\pgfpathlineto{\pgfqpoint{1.232825in}{2.115608in}}%
\pgfpathlineto{\pgfqpoint{1.233954in}{2.119928in}}%
\pgfpathlineto{\pgfqpoint{1.234518in}{2.121711in}}%
\pgfpathlineto{\pgfqpoint{1.235082in}{2.122439in}}%
\pgfpathlineto{\pgfqpoint{1.235647in}{2.128561in}}%
\pgfpathlineto{\pgfqpoint{1.236211in}{2.126112in}}%
\pgfpathlineto{\pgfqpoint{1.272894in}{1.875527in}}%
\pgfpathlineto{\pgfqpoint{1.273458in}{1.895889in}}%
\pgfpathlineto{\pgfqpoint{1.274022in}{2.032288in}}%
\pgfpathlineto{\pgfqpoint{1.274587in}{2.018918in}}%
\pgfpathlineto{\pgfqpoint{1.275151in}{2.107386in}}%
\pgfpathlineto{\pgfqpoint{1.275715in}{2.147718in}}%
\pgfpathlineto{\pgfqpoint{1.276280in}{2.135305in}}%
\pgfpathlineto{\pgfqpoint{1.276844in}{1.891643in}}%
\pgfpathlineto{\pgfqpoint{1.277408in}{1.899861in}}%
\pgfpathlineto{\pgfqpoint{1.277973in}{2.061570in}}%
\pgfpathlineto{\pgfqpoint{1.278537in}{2.061276in}}%
\pgfpathlineto{\pgfqpoint{1.279101in}{2.189662in}}%
\pgfpathlineto{\pgfqpoint{1.287002in}{1.885067in}}%
\pgfpathlineto{\pgfqpoint{1.287567in}{1.898500in}}%
\pgfpathlineto{\pgfqpoint{1.288131in}{1.880344in}}%
\pgfpathlineto{\pgfqpoint{1.288695in}{1.875935in}}%
\pgfpathlineto{\pgfqpoint{1.289260in}{2.020260in}}%
\pgfpathlineto{\pgfqpoint{1.289824in}{2.071338in}}%
\pgfpathlineto{\pgfqpoint{1.290388in}{1.942512in}}%
\pgfpathlineto{\pgfqpoint{1.290953in}{1.933018in}}%
\pgfpathlineto{\pgfqpoint{1.291517in}{2.158148in}}%
\pgfpathlineto{\pgfqpoint{1.292081in}{2.188641in}}%
\pgfpathlineto{\pgfqpoint{1.299982in}{2.144770in}}%
\pgfpathlineto{\pgfqpoint{1.300546in}{2.153241in}}%
\pgfpathlineto{\pgfqpoint{1.301111in}{2.192854in}}%
\pgfpathlineto{\pgfqpoint{1.301675in}{2.121738in}}%
\pgfpathlineto{\pgfqpoint{1.302239in}{2.202952in}}%
\pgfpathlineto{\pgfqpoint{1.302804in}{2.197808in}}%
\pgfpathlineto{\pgfqpoint{1.303368in}{2.201203in}}%
\pgfpathlineto{\pgfqpoint{1.304497in}{2.203601in}}%
\pgfpathlineto{\pgfqpoint{1.305061in}{2.204190in}}%
\pgfpathlineto{\pgfqpoint{1.305626in}{1.886769in}}%
\pgfpathlineto{\pgfqpoint{1.307319in}{1.951947in}}%
\pgfpathlineto{\pgfqpoint{1.313526in}{2.206123in}}%
\pgfpathlineto{\pgfqpoint{1.314091in}{2.211944in}}%
\pgfpathlineto{\pgfqpoint{1.314655in}{2.211526in}}%
\pgfpathlineto{\pgfqpoint{1.315219in}{2.217031in}}%
\pgfpathlineto{\pgfqpoint{1.315784in}{2.216190in}}%
\pgfpathlineto{\pgfqpoint{1.316348in}{2.218441in}}%
\pgfpathlineto{\pgfqpoint{1.316912in}{2.216954in}}%
\pgfpathlineto{\pgfqpoint{1.318041in}{2.216912in}}%
\pgfpathlineto{\pgfqpoint{1.318606in}{2.238918in}}%
\pgfpathlineto{\pgfqpoint{1.319170in}{2.242885in}}%
\pgfpathlineto{\pgfqpoint{1.342872in}{2.071479in}}%
\pgfpathlineto{\pgfqpoint{1.368268in}{1.887777in}}%
\pgfpathlineto{\pgfqpoint{1.368832in}{1.923287in}}%
\pgfpathlineto{\pgfqpoint{1.369397in}{2.258254in}}%
\pgfpathlineto{\pgfqpoint{1.369961in}{2.259601in}}%
\pgfpathlineto{\pgfqpoint{1.370525in}{2.258128in}}%
\pgfpathlineto{\pgfqpoint{1.371090in}{2.270863in}}%
\pgfpathlineto{\pgfqpoint{1.371654in}{2.272752in}}%
\pgfpathlineto{\pgfqpoint{1.373347in}{2.293222in}}%
\pgfpathlineto{\pgfqpoint{1.373911in}{2.277692in}}%
\pgfpathlineto{\pgfqpoint{1.382377in}{1.901526in}}%
\pgfpathlineto{\pgfqpoint{1.382941in}{1.918108in}}%
\pgfpathlineto{\pgfqpoint{1.383505in}{1.883566in}}%
\pgfpathlineto{\pgfqpoint{1.384070in}{1.910908in}}%
\pgfpathlineto{\pgfqpoint{1.386327in}{1.936366in}}%
\pgfpathlineto{\pgfqpoint{1.386891in}{1.939986in}}%
\pgfpathlineto{\pgfqpoint{1.394792in}{1.942274in}}%
\pgfpathlineto{\pgfqpoint{1.395357in}{1.922809in}}%
\pgfpathlineto{\pgfqpoint{1.395921in}{1.937257in}}%
\pgfpathlineto{\pgfqpoint{1.396485in}{1.938301in}}%
\pgfpathlineto{\pgfqpoint{1.397050in}{1.948032in}}%
\pgfpathlineto{\pgfqpoint{1.397614in}{1.951761in}}%
\pgfpathlineto{\pgfqpoint{1.398178in}{1.951866in}}%
\pgfpathlineto{\pgfqpoint{1.398743in}{1.951538in}}%
\pgfpathlineto{\pgfqpoint{1.399307in}{1.923916in}}%
\pgfpathlineto{\pgfqpoint{1.399871in}{1.956409in}}%
\pgfpathlineto{\pgfqpoint{1.410030in}{1.959104in}}%
\pgfpathlineto{\pgfqpoint{1.410594in}{1.958737in}}%
\pgfpathlineto{\pgfqpoint{1.411158in}{1.951523in}}%
\pgfpathlineto{\pgfqpoint{1.411723in}{1.970507in}}%
\pgfpathlineto{\pgfqpoint{1.412851in}{1.970661in}}%
\pgfpathlineto{\pgfqpoint{1.413416in}{1.943643in}}%
\pgfpathlineto{\pgfqpoint{1.413980in}{1.977436in}}%
\pgfpathlineto{\pgfqpoint{1.414544in}{1.979411in}}%
\pgfpathlineto{\pgfqpoint{1.421881in}{1.979411in}}%
\pgfpathlineto{\pgfqpoint{1.423574in}{1.968454in}}%
\pgfpathlineto{\pgfqpoint{1.424138in}{1.934620in}}%
\pgfpathlineto{\pgfqpoint{1.424703in}{1.953374in}}%
\pgfpathlineto{\pgfqpoint{1.425267in}{1.985792in}}%
\pgfpathlineto{\pgfqpoint{1.425831in}{1.981936in}}%
\pgfpathlineto{\pgfqpoint{1.426396in}{1.980617in}}%
\pgfpathlineto{\pgfqpoint{1.465335in}{1.999642in}}%
\pgfpathlineto{\pgfqpoint{1.465900in}{1.995317in}}%
\pgfpathlineto{\pgfqpoint{1.466464in}{1.997531in}}%
\pgfpathlineto{\pgfqpoint{1.467593in}{1.991586in}}%
\pgfpathlineto{\pgfqpoint{1.468157in}{1.982014in}}%
\pgfpathlineto{\pgfqpoint{1.470415in}{1.986418in}}%
\pgfpathlineto{\pgfqpoint{1.476622in}{1.998970in}}%
\pgfpathlineto{\pgfqpoint{1.477187in}{1.995685in}}%
\pgfpathlineto{\pgfqpoint{1.477751in}{2.001100in}}%
\pgfpathlineto{\pgfqpoint{1.478880in}{2.002209in}}%
\pgfpathlineto{\pgfqpoint{1.479444in}{2.004563in}}%
\pgfpathlineto{\pgfqpoint{1.480008in}{2.009564in}}%
\pgfpathlineto{\pgfqpoint{1.480573in}{2.002914in}}%
\pgfpathlineto{\pgfqpoint{1.481701in}{2.025691in}}%
\pgfpathlineto{\pgfqpoint{1.483394in}{2.023908in}}%
\pgfpathlineto{\pgfqpoint{1.490167in}{2.016440in}}%
\pgfpathlineto{\pgfqpoint{1.490731in}{2.022514in}}%
\pgfpathlineto{\pgfqpoint{1.491295in}{2.021293in}}%
\pgfpathlineto{\pgfqpoint{1.491860in}{2.018934in}}%
\pgfpathlineto{\pgfqpoint{1.492424in}{2.002345in}}%
\pgfpathlineto{\pgfqpoint{1.492988in}{2.019915in}}%
\pgfpathlineto{\pgfqpoint{1.493553in}{2.014702in}}%
\pgfpathlineto{\pgfqpoint{1.495810in}{2.058541in}}%
\pgfpathlineto{\pgfqpoint{1.503711in}{2.046971in}}%
\pgfpathlineto{\pgfqpoint{1.504275in}{2.021035in}}%
\pgfpathlineto{\pgfqpoint{1.505404in}{2.059139in}}%
\pgfpathlineto{\pgfqpoint{1.506533in}{2.055873in}}%
\pgfpathlineto{\pgfqpoint{1.507097in}{2.034221in}}%
\pgfpathlineto{\pgfqpoint{1.507661in}{2.033268in}}%
\pgfpathlineto{\pgfqpoint{1.508226in}{2.059362in}}%
\pgfpathlineto{\pgfqpoint{1.517255in}{2.058013in}}%
\pgfpathlineto{\pgfqpoint{1.517820in}{2.046912in}}%
\pgfpathlineto{\pgfqpoint{1.518384in}{2.060211in}}%
\pgfpathlineto{\pgfqpoint{1.518948in}{2.058438in}}%
\pgfpathlineto{\pgfqpoint{1.519513in}{2.059543in}}%
\pgfpathlineto{\pgfqpoint{1.520641in}{2.027012in}}%
\pgfpathlineto{\pgfqpoint{1.521770in}{2.044078in}}%
\pgfpathlineto{\pgfqpoint{1.524027in}{2.043040in}}%
\pgfpathlineto{\pgfqpoint{1.557888in}{2.026024in}}%
\pgfpathlineto{\pgfqpoint{1.558452in}{2.027559in}}%
\pgfpathlineto{\pgfqpoint{1.559581in}{2.050789in}}%
\pgfpathlineto{\pgfqpoint{1.560145in}{2.050333in}}%
\pgfpathlineto{\pgfqpoint{1.560710in}{2.052692in}}%
\pgfpathlineto{\pgfqpoint{1.561274in}{2.050771in}}%
\pgfpathlineto{\pgfqpoint{1.561839in}{2.029695in}}%
\pgfpathlineto{\pgfqpoint{1.562403in}{2.069001in}}%
\pgfpathlineto{\pgfqpoint{1.565789in}{2.069492in}}%
\pgfpathlineto{\pgfqpoint{1.571432in}{2.070404in}}%
\pgfpathlineto{\pgfqpoint{1.571997in}{2.071460in}}%
\pgfpathlineto{\pgfqpoint{1.572561in}{2.073237in}}%
\pgfpathlineto{\pgfqpoint{1.573125in}{2.054264in}}%
\pgfpathlineto{\pgfqpoint{1.573690in}{2.048403in}}%
\pgfpathlineto{\pgfqpoint{1.574254in}{2.065818in}}%
\pgfpathlineto{\pgfqpoint{1.574818in}{2.066293in}}%
\pgfpathlineto{\pgfqpoint{1.575383in}{2.074315in}}%
\pgfpathlineto{\pgfqpoint{1.575947in}{2.074247in}}%
\pgfpathlineto{\pgfqpoint{1.577076in}{2.082249in}}%
\pgfpathlineto{\pgfqpoint{1.584977in}{2.079129in}}%
\pgfpathlineto{\pgfqpoint{1.585541in}{2.051571in}}%
\pgfpathlineto{\pgfqpoint{1.586105in}{2.070747in}}%
\pgfpathlineto{\pgfqpoint{1.586670in}{2.066992in}}%
\pgfpathlineto{\pgfqpoint{1.587234in}{2.070845in}}%
\pgfpathlineto{\pgfqpoint{1.587798in}{2.063066in}}%
\pgfpathlineto{\pgfqpoint{1.588363in}{2.088520in}}%
\pgfpathlineto{\pgfqpoint{1.588927in}{2.078912in}}%
\pgfpathlineto{\pgfqpoint{1.589491in}{2.083979in}}%
\pgfpathlineto{\pgfqpoint{1.590056in}{2.096316in}}%
\pgfpathlineto{\pgfqpoint{1.598521in}{2.012483in}}%
\pgfpathlineto{\pgfqpoint{1.599085in}{2.073155in}}%
\pgfpathlineto{\pgfqpoint{1.599650in}{2.070064in}}%
\pgfpathlineto{\pgfqpoint{1.600778in}{2.083430in}}%
\pgfpathlineto{\pgfqpoint{1.601343in}{2.131806in}}%
\pgfpathlineto{\pgfqpoint{1.601907in}{2.126498in}}%
\pgfpathlineto{\pgfqpoint{1.603036in}{2.222886in}}%
\pgfpathlineto{\pgfqpoint{1.603600in}{2.240572in}}%
\pgfpathlineto{\pgfqpoint{1.611501in}{2.240572in}}%
\pgfpathlineto{\pgfqpoint{1.613194in}{2.241466in}}%
\pgfpathlineto{\pgfqpoint{1.670193in}{2.241278in}}%
\pgfpathlineto{\pgfqpoint{1.671322in}{2.243052in}}%
\pgfpathlineto{\pgfqpoint{1.679222in}{2.243531in}}%
\pgfpathlineto{\pgfqpoint{1.679787in}{2.243020in}}%
\pgfpathlineto{\pgfqpoint{1.680351in}{2.241482in}}%
\pgfpathlineto{\pgfqpoint{1.681480in}{2.240937in}}%
\pgfpathlineto{\pgfqpoint{1.684866in}{2.240362in}}%
\pgfpathlineto{\pgfqpoint{1.693331in}{2.241058in}}%
\pgfpathlineto{\pgfqpoint{1.693895in}{2.242945in}}%
\pgfpathlineto{\pgfqpoint{1.694460in}{2.243556in}}%
\pgfpathlineto{\pgfqpoint{1.695588in}{2.243803in}}%
\pgfpathlineto{\pgfqpoint{1.696717in}{2.244212in}}%
\pgfpathlineto{\pgfqpoint{1.899317in}{2.235553in}}%
\pgfpathlineto{\pgfqpoint{1.936564in}{2.246365in}}%
\pgfpathlineto{\pgfqpoint{1.937128in}{2.243805in}}%
\pgfpathlineto{\pgfqpoint{1.940514in}{2.246832in}}%
\pgfpathlineto{\pgfqpoint{1.941079in}{2.252421in}}%
\pgfpathlineto{\pgfqpoint{1.941643in}{2.249099in}}%
\pgfpathlineto{\pgfqpoint{1.942208in}{2.273778in}}%
\pgfpathlineto{\pgfqpoint{1.942772in}{2.274311in}}%
\pgfpathlineto{\pgfqpoint{1.951237in}{2.241087in}}%
\pgfpathlineto{\pgfqpoint{1.951801in}{2.241722in}}%
\pgfpathlineto{\pgfqpoint{1.952366in}{2.244392in}}%
\pgfpathlineto{\pgfqpoint{1.952930in}{2.244715in}}%
\pgfpathlineto{\pgfqpoint{1.953494in}{2.241120in}}%
\pgfpathlineto{\pgfqpoint{1.954059in}{2.248636in}}%
\pgfpathlineto{\pgfqpoint{1.954623in}{2.260478in}}%
\pgfpathlineto{\pgfqpoint{1.955187in}{2.260145in}}%
\pgfpathlineto{\pgfqpoint{1.955752in}{2.277729in}}%
\pgfpathlineto{\pgfqpoint{1.956881in}{2.273287in}}%
\pgfpathlineto{\pgfqpoint{1.959138in}{2.274950in}}%
\pgfpathlineto{\pgfqpoint{1.963088in}{2.278048in}}%
\pgfpathlineto{\pgfqpoint{1.963653in}{2.277689in}}%
\pgfpathlineto{\pgfqpoint{1.964781in}{2.247227in}}%
\pgfpathlineto{\pgfqpoint{1.965346in}{2.275318in}}%
\pgfpathlineto{\pgfqpoint{1.966474in}{2.259314in}}%
\pgfpathlineto{\pgfqpoint{1.967039in}{2.270258in}}%
\pgfpathlineto{\pgfqpoint{1.967603in}{2.249202in}}%
\pgfpathlineto{\pgfqpoint{1.968167in}{2.275692in}}%
\pgfpathlineto{\pgfqpoint{1.968732in}{2.249329in}}%
\pgfpathlineto{\pgfqpoint{1.969296in}{2.247368in}}%
\pgfpathlineto{\pgfqpoint{1.969860in}{2.247845in}}%
\pgfpathlineto{\pgfqpoint{1.977761in}{2.247419in}}%
\pgfpathlineto{\pgfqpoint{1.978890in}{2.295138in}}%
\pgfpathlineto{\pgfqpoint{1.979454in}{2.309358in}}%
\pgfpathlineto{\pgfqpoint{1.980019in}{2.308428in}}%
\pgfpathlineto{\pgfqpoint{1.980583in}{2.309366in}}%
\pgfpathlineto{\pgfqpoint{1.981147in}{2.303619in}}%
\pgfpathlineto{\pgfqpoint{1.981712in}{2.269658in}}%
\pgfpathlineto{\pgfqpoint{1.982276in}{2.293251in}}%
\pgfpathlineto{\pgfqpoint{1.984533in}{2.282537in}}%
\pgfpathlineto{\pgfqpoint{1.991306in}{2.251922in}}%
\pgfpathlineto{\pgfqpoint{1.991870in}{2.270563in}}%
\pgfpathlineto{\pgfqpoint{1.992434in}{2.310642in}}%
\pgfpathlineto{\pgfqpoint{1.992999in}{2.321938in}}%
\pgfpathlineto{\pgfqpoint{1.993563in}{2.306019in}}%
\pgfpathlineto{\pgfqpoint{1.994127in}{2.304342in}}%
\pgfpathlineto{\pgfqpoint{1.994692in}{2.306432in}}%
\pgfpathlineto{\pgfqpoint{1.995820in}{2.305124in}}%
\pgfpathlineto{\pgfqpoint{2.030810in}{2.272873in}}%
\pgfpathlineto{\pgfqpoint{2.031374in}{2.289707in}}%
\pgfpathlineto{\pgfqpoint{2.031938in}{2.317991in}}%
\pgfpathlineto{\pgfqpoint{2.032503in}{2.319529in}}%
\pgfpathlineto{\pgfqpoint{2.033067in}{2.336400in}}%
\pgfpathlineto{\pgfqpoint{2.033632in}{2.299820in}}%
\pgfpathlineto{\pgfqpoint{2.034196in}{2.299297in}}%
\pgfpathlineto{\pgfqpoint{2.034760in}{2.287458in}}%
\pgfpathlineto{\pgfqpoint{2.035889in}{2.302014in}}%
\pgfpathlineto{\pgfqpoint{2.036453in}{2.273968in}}%
\pgfpathlineto{\pgfqpoint{2.037018in}{2.272811in}}%
\pgfpathlineto{\pgfqpoint{2.044918in}{2.345530in}}%
\pgfpathlineto{\pgfqpoint{2.046047in}{2.284712in}}%
\pgfpathlineto{\pgfqpoint{2.046611in}{2.274399in}}%
\pgfpathlineto{\pgfqpoint{2.047176in}{2.273309in}}%
\pgfpathlineto{\pgfqpoint{2.047740in}{2.284956in}}%
\pgfpathlineto{\pgfqpoint{2.048305in}{2.261650in}}%
\pgfpathlineto{\pgfqpoint{2.048869in}{2.278073in}}%
\pgfpathlineto{\pgfqpoint{2.049998in}{2.372345in}}%
\pgfpathlineto{\pgfqpoint{2.050562in}{2.374560in}}%
\pgfpathlineto{\pgfqpoint{2.057898in}{2.374608in}}%
\pgfpathlineto{\pgfqpoint{2.058463in}{2.340728in}}%
\pgfpathlineto{\pgfqpoint{2.059027in}{2.268654in}}%
\pgfpathlineto{\pgfqpoint{2.060156in}{2.315694in}}%
\pgfpathlineto{\pgfqpoint{2.060720in}{2.309545in}}%
\pgfpathlineto{\pgfqpoint{2.061284in}{2.327769in}}%
\pgfpathlineto{\pgfqpoint{2.061849in}{2.303129in}}%
\pgfpathlineto{\pgfqpoint{2.062413in}{2.383713in}}%
\pgfpathlineto{\pgfqpoint{2.062978in}{2.311365in}}%
\pgfpathlineto{\pgfqpoint{2.063542in}{2.344358in}}%
\pgfpathlineto{\pgfqpoint{2.064106in}{2.436184in}}%
\pgfpathlineto{\pgfqpoint{2.064671in}{2.458278in}}%
\pgfpathlineto{\pgfqpoint{2.072007in}{2.429348in}}%
\pgfpathlineto{\pgfqpoint{2.072571in}{2.432450in}}%
\pgfpathlineto{\pgfqpoint{2.073136in}{2.376985in}}%
\pgfpathlineto{\pgfqpoint{2.073700in}{2.392349in}}%
\pgfpathlineto{\pgfqpoint{2.074264in}{2.434411in}}%
\pgfpathlineto{\pgfqpoint{2.074829in}{2.452525in}}%
\pgfpathlineto{\pgfqpoint{2.075393in}{2.425193in}}%
\pgfpathlineto{\pgfqpoint{2.075957in}{2.368967in}}%
\pgfpathlineto{\pgfqpoint{2.076522in}{2.359662in}}%
\pgfpathlineto{\pgfqpoint{2.077086in}{2.368249in}}%
\pgfpathlineto{\pgfqpoint{2.077651in}{2.347994in}}%
\pgfpathlineto{\pgfqpoint{2.085551in}{2.341290in}}%
\pgfpathlineto{\pgfqpoint{2.086116in}{2.402783in}}%
\pgfpathlineto{\pgfqpoint{2.086680in}{2.379028in}}%
\pgfpathlineto{\pgfqpoint{2.087244in}{2.399047in}}%
\pgfpathlineto{\pgfqpoint{2.088373in}{2.510013in}}%
\pgfpathlineto{\pgfqpoint{2.089502in}{2.534187in}}%
\pgfpathlineto{\pgfqpoint{2.090066in}{2.573670in}}%
\pgfpathlineto{\pgfqpoint{2.090630in}{2.575722in}}%
\pgfpathlineto{\pgfqpoint{2.091759in}{2.573994in}}%
\pgfpathlineto{\pgfqpoint{2.127877in}{2.509315in}}%
\pgfpathlineto{\pgfqpoint{2.129006in}{2.553217in}}%
\pgfpathlineto{\pgfqpoint{2.129570in}{2.547900in}}%
\pgfpathlineto{\pgfqpoint{2.130135in}{2.528549in}}%
\pgfpathlineto{\pgfqpoint{2.130699in}{2.571988in}}%
\pgfpathlineto{\pgfqpoint{2.131263in}{2.513355in}}%
\pgfpathlineto{\pgfqpoint{2.131828in}{2.509857in}}%
\pgfpathlineto{\pgfqpoint{2.139729in}{2.512828in}}%
\pgfpathlineto{\pgfqpoint{2.140293in}{2.513509in}}%
\pgfpathlineto{\pgfqpoint{2.140857in}{2.516304in}}%
\pgfpathlineto{\pgfqpoint{2.141422in}{2.421767in}}%
\pgfpathlineto{\pgfqpoint{2.142550in}{2.612461in}}%
\pgfpathlineto{\pgfqpoint{2.143115in}{2.621817in}}%
\pgfpathlineto{\pgfqpoint{2.143679in}{2.625339in}}%
\pgfpathlineto{\pgfqpoint{2.144243in}{2.636513in}}%
\pgfpathlineto{\pgfqpoint{2.148194in}{2.637483in}}%
\pgfpathlineto{\pgfqpoint{2.153837in}{2.638952in}}%
\pgfpathlineto{\pgfqpoint{2.154402in}{2.642436in}}%
\pgfpathlineto{\pgfqpoint{2.154966in}{2.656840in}}%
\pgfpathlineto{\pgfqpoint{2.155530in}{2.656228in}}%
\pgfpathlineto{\pgfqpoint{2.156095in}{2.667101in}}%
\pgfpathlineto{\pgfqpoint{2.156659in}{2.672450in}}%
\pgfpathlineto{\pgfqpoint{2.157223in}{2.686024in}}%
\pgfpathlineto{\pgfqpoint{2.157788in}{2.692232in}}%
\pgfpathlineto{\pgfqpoint{2.166817in}{2.683398in}}%
\pgfpathlineto{\pgfqpoint{2.167381in}{2.683116in}}%
\pgfpathlineto{\pgfqpoint{2.167946in}{2.683593in}}%
\pgfpathlineto{\pgfqpoint{2.168510in}{2.690208in}}%
\pgfpathlineto{\pgfqpoint{2.169075in}{2.684476in}}%
\pgfpathlineto{\pgfqpoint{2.170203in}{2.688407in}}%
\pgfpathlineto{\pgfqpoint{2.170768in}{2.695237in}}%
\pgfpathlineto{\pgfqpoint{2.174154in}{2.695515in}}%
\pgfpathlineto{\pgfqpoint{2.180361in}{2.696140in}}%
\pgfpathlineto{\pgfqpoint{2.180926in}{2.691644in}}%
\pgfpathlineto{\pgfqpoint{2.181490in}{2.690864in}}%
\pgfpathlineto{\pgfqpoint{2.182054in}{2.696304in}}%
\pgfpathlineto{\pgfqpoint{2.182619in}{2.697024in}}%
\pgfpathlineto{\pgfqpoint{2.183183in}{2.703303in}}%
\pgfpathlineto{\pgfqpoint{2.183747in}{2.704830in}}%
\pgfpathlineto{\pgfqpoint{2.223816in}{2.709484in}}%
\pgfpathlineto{\pgfqpoint{2.224380in}{2.706586in}}%
\pgfpathlineto{\pgfqpoint{2.224945in}{2.705243in}}%
\pgfpathlineto{\pgfqpoint{2.225509in}{2.709388in}}%
\pgfpathlineto{\pgfqpoint{2.226073in}{2.710638in}}%
\pgfpathlineto{\pgfqpoint{2.228331in}{2.710851in}}%
\pgfpathlineto{\pgfqpoint{2.235667in}{2.711299in}}%
\pgfpathlineto{\pgfqpoint{2.236232in}{2.707931in}}%
\pgfpathlineto{\pgfqpoint{2.236796in}{2.689215in}}%
\pgfpathlineto{\pgfqpoint{2.237360in}{2.694118in}}%
\pgfpathlineto{\pgfqpoint{2.239053in}{2.717320in}}%
\pgfpathlineto{\pgfqpoint{2.239618in}{2.718546in}}%
\pgfpathlineto{\pgfqpoint{2.251469in}{2.724492in}}%
\pgfpathlineto{\pgfqpoint{2.252598in}{2.720369in}}%
\pgfpathlineto{\pgfqpoint{2.254855in}{2.720241in}}%
\pgfpathlineto{\pgfqpoint{2.263320in}{2.720180in}}%
\pgfpathlineto{\pgfqpoint{2.265013in}{2.722292in}}%
\pgfpathlineto{\pgfqpoint{2.288716in}{2.721927in}}%
\pgfpathlineto{\pgfqpoint{2.320319in}{2.721462in}}%
\pgfpathlineto{\pgfqpoint{2.320884in}{2.722478in}}%
\pgfpathlineto{\pgfqpoint{2.329349in}{2.721487in}}%
\pgfpathlineto{\pgfqpoint{2.330477in}{2.730269in}}%
\pgfpathlineto{\pgfqpoint{2.331042in}{2.731512in}}%
\pgfpathlineto{\pgfqpoint{2.331606in}{2.731875in}}%
\pgfpathlineto{\pgfqpoint{2.332735in}{2.740048in}}%
\pgfpathlineto{\pgfqpoint{2.333299in}{2.734873in}}%
\pgfpathlineto{\pgfqpoint{2.333863in}{2.737898in}}%
\pgfpathlineto{\pgfqpoint{2.334992in}{2.761340in}}%
\pgfpathlineto{\pgfqpoint{2.336121in}{2.740787in}}%
\pgfpathlineto{\pgfqpoint{2.344586in}{2.744778in}}%
\pgfpathlineto{\pgfqpoint{2.345150in}{2.744811in}}%
\pgfpathlineto{\pgfqpoint{2.345715in}{2.747277in}}%
\pgfpathlineto{\pgfqpoint{2.346279in}{2.747522in}}%
\pgfpathlineto{\pgfqpoint{2.347408in}{2.754906in}}%
\pgfpathlineto{\pgfqpoint{2.347972in}{2.748148in}}%
\pgfpathlineto{\pgfqpoint{2.348536in}{2.746630in}}%
\pgfpathlineto{\pgfqpoint{2.359259in}{2.745892in}}%
\pgfpathlineto{\pgfqpoint{2.359823in}{2.751762in}}%
\pgfpathlineto{\pgfqpoint{2.360388in}{2.743417in}}%
\pgfpathlineto{\pgfqpoint{2.361516in}{2.756296in}}%
\pgfpathlineto{\pgfqpoint{2.362081in}{2.756249in}}%
\pgfpathlineto{\pgfqpoint{2.362645in}{2.756544in}}%
\pgfpathlineto{\pgfqpoint{2.369982in}{2.768792in}}%
\pgfpathlineto{\pgfqpoint{2.370546in}{2.765586in}}%
\pgfpathlineto{\pgfqpoint{2.371110in}{2.756334in}}%
\pgfpathlineto{\pgfqpoint{2.371675in}{2.757112in}}%
\pgfpathlineto{\pgfqpoint{2.372803in}{2.773427in}}%
\pgfpathlineto{\pgfqpoint{2.373368in}{2.774322in}}%
\pgfpathlineto{\pgfqpoint{2.373932in}{2.747292in}}%
\pgfpathlineto{\pgfqpoint{2.374496in}{2.765952in}}%
\pgfpathlineto{\pgfqpoint{2.375061in}{2.810665in}}%
\pgfpathlineto{\pgfqpoint{2.412872in}{2.756289in}}%
\pgfpathlineto{\pgfqpoint{2.413436in}{2.755697in}}%
\pgfpathlineto{\pgfqpoint{2.414565in}{2.737259in}}%
\pgfpathlineto{\pgfqpoint{2.416258in}{2.784566in}}%
\pgfpathlineto{\pgfqpoint{2.416822in}{2.787677in}}%
\pgfpathlineto{\pgfqpoint{2.417387in}{2.778347in}}%
\pgfpathlineto{\pgfqpoint{2.426416in}{2.798098in}}%
\pgfpathlineto{\pgfqpoint{2.427545in}{2.733764in}}%
\pgfpathlineto{\pgfqpoint{2.428109in}{2.733874in}}%
\pgfpathlineto{\pgfqpoint{2.430367in}{2.749038in}}%
\pgfpathlineto{\pgfqpoint{2.438267in}{2.799927in}}%
\pgfpathlineto{\pgfqpoint{2.438832in}{2.759389in}}%
\pgfpathlineto{\pgfqpoint{2.439960in}{2.803488in}}%
\pgfpathlineto{\pgfqpoint{2.441089in}{2.803232in}}%
\pgfpathlineto{\pgfqpoint{2.441653in}{2.803287in}}%
\pgfpathlineto{\pgfqpoint{2.442218in}{2.785872in}}%
\pgfpathlineto{\pgfqpoint{2.442782in}{2.745872in}}%
\pgfpathlineto{\pgfqpoint{2.443347in}{2.741522in}}%
\pgfpathlineto{\pgfqpoint{2.452376in}{2.727722in}}%
\pgfpathlineto{\pgfqpoint{2.454633in}{2.727666in}}%
\pgfpathlineto{\pgfqpoint{2.455198in}{2.743624in}}%
\pgfpathlineto{\pgfqpoint{2.455762in}{2.784920in}}%
\pgfpathlineto{\pgfqpoint{2.456326in}{2.803381in}}%
\pgfpathlineto{\pgfqpoint{2.508811in}{2.799947in}}%
\pgfpathlineto{\pgfqpoint{2.509375in}{2.798512in}}%
\pgfpathlineto{\pgfqpoint{2.524048in}{2.737503in}}%
\pgfpathlineto{\pgfqpoint{2.524612in}{2.739767in}}%
\pgfpathlineto{\pgfqpoint{2.533077in}{2.797662in}}%
\pgfpathlineto{\pgfqpoint{2.533642in}{2.782587in}}%
\pgfpathlineto{\pgfqpoint{2.534206in}{2.794897in}}%
\pgfpathlineto{\pgfqpoint{2.537028in}{2.744939in}}%
\pgfpathlineto{\pgfqpoint{2.537592in}{2.742677in}}%
\pgfpathlineto{\pgfqpoint{2.538157in}{2.743810in}}%
\pgfpathlineto{\pgfqpoint{2.539285in}{2.753400in}}%
\pgfpathlineto{\pgfqpoint{2.545493in}{2.812508in}}%
\pgfpathlineto{\pgfqpoint{2.546057in}{2.814828in}}%
\pgfpathlineto{\pgfqpoint{2.546622in}{2.750413in}}%
\pgfpathlineto{\pgfqpoint{2.547186in}{2.744848in}}%
\pgfpathlineto{\pgfqpoint{2.548315in}{2.815351in}}%
\pgfpathlineto{\pgfqpoint{2.548879in}{2.817541in}}%
\pgfpathlineto{\pgfqpoint{2.549444in}{2.810129in}}%
\pgfpathlineto{\pgfqpoint{2.550008in}{2.813917in}}%
\pgfpathlineto{\pgfqpoint{2.550572in}{2.754688in}}%
\pgfpathlineto{\pgfqpoint{2.551137in}{2.795740in}}%
\pgfpathlineto{\pgfqpoint{2.551701in}{2.816553in}}%
\pgfpathlineto{\pgfqpoint{2.559037in}{2.832638in}}%
\pgfpathlineto{\pgfqpoint{2.559602in}{2.832731in}}%
\pgfpathlineto{\pgfqpoint{2.560730in}{2.782578in}}%
\pgfpathlineto{\pgfqpoint{2.561295in}{2.784451in}}%
\pgfpathlineto{\pgfqpoint{2.561859in}{2.818255in}}%
\pgfpathlineto{\pgfqpoint{2.562423in}{2.820570in}}%
\pgfpathlineto{\pgfqpoint{2.562988in}{2.834629in}}%
\pgfpathlineto{\pgfqpoint{2.564117in}{2.832474in}}%
\pgfpathlineto{\pgfqpoint{2.601363in}{2.745383in}}%
\pgfpathlineto{\pgfqpoint{2.601928in}{2.755433in}}%
\pgfpathlineto{\pgfqpoint{2.603621in}{2.828031in}}%
\pgfpathlineto{\pgfqpoint{2.604185in}{2.830424in}}%
\pgfpathlineto{\pgfqpoint{2.604749in}{2.859135in}}%
\pgfpathlineto{\pgfqpoint{2.605314in}{2.914981in}}%
\pgfpathlineto{\pgfqpoint{2.611522in}{2.860764in}}%
\pgfpathlineto{\pgfqpoint{2.613779in}{2.840967in}}%
\pgfpathlineto{\pgfqpoint{2.614343in}{2.842359in}}%
\pgfpathlineto{\pgfqpoint{2.614908in}{2.816894in}}%
\pgfpathlineto{\pgfqpoint{2.615472in}{2.854828in}}%
\pgfpathlineto{\pgfqpoint{2.616036in}{2.765285in}}%
\pgfpathlineto{\pgfqpoint{2.616601in}{2.829570in}}%
\pgfpathlineto{\pgfqpoint{2.617165in}{2.856517in}}%
\pgfpathlineto{\pgfqpoint{2.617729in}{2.859778in}}%
\pgfpathlineto{\pgfqpoint{2.618858in}{2.873676in}}%
\pgfpathlineto{\pgfqpoint{2.619422in}{2.862964in}}%
\pgfpathlineto{\pgfqpoint{2.627888in}{2.859751in}}%
\pgfpathlineto{\pgfqpoint{2.628452in}{2.859855in}}%
\pgfpathlineto{\pgfqpoint{2.629581in}{2.873193in}}%
\pgfpathlineto{\pgfqpoint{2.632402in}{2.910001in}}%
\pgfpathlineto{\pgfqpoint{2.633531in}{2.903963in}}%
\pgfpathlineto{\pgfqpoint{2.641996in}{2.851134in}}%
\pgfpathlineto{\pgfqpoint{2.643689in}{2.857873in}}%
\pgfpathlineto{\pgfqpoint{2.644254in}{2.891654in}}%
\pgfpathlineto{\pgfqpoint{2.644818in}{2.880282in}}%
\pgfpathlineto{\pgfqpoint{2.645382in}{2.861629in}}%
\pgfpathlineto{\pgfqpoint{2.657798in}{2.942806in}}%
\pgfpathlineto{\pgfqpoint{2.694480in}{2.887715in}}%
\pgfpathlineto{\pgfqpoint{2.710846in}{2.863132in}}%
\pgfpathlineto{\pgfqpoint{2.711411in}{2.878265in}}%
\pgfpathlineto{\pgfqpoint{2.712539in}{2.939726in}}%
\pgfpathlineto{\pgfqpoint{2.713104in}{2.946910in}}%
\pgfpathlineto{\pgfqpoint{2.713668in}{2.946335in}}%
\pgfpathlineto{\pgfqpoint{2.722698in}{2.868506in}}%
\pgfpathlineto{\pgfqpoint{2.726084in}{2.910255in}}%
\pgfpathlineto{\pgfqpoint{2.726648in}{2.930832in}}%
\pgfpathlineto{\pgfqpoint{2.727212in}{2.975196in}}%
\pgfpathlineto{\pgfqpoint{2.737935in}{2.855101in}}%
\pgfpathlineto{\pgfqpoint{2.738499in}{2.857847in}}%
\pgfpathlineto{\pgfqpoint{2.740757in}{2.980652in}}%
\pgfpathlineto{\pgfqpoint{2.741321in}{2.985584in}}%
\pgfpathlineto{\pgfqpoint{2.749786in}{2.860237in}}%
\pgfpathlineto{\pgfqpoint{2.750351in}{2.861109in}}%
\pgfpathlineto{\pgfqpoint{2.751479in}{2.861427in}}%
\pgfpathlineto{\pgfqpoint{2.752608in}{2.861355in}}%
\pgfpathlineto{\pgfqpoint{2.753172in}{2.862356in}}%
\pgfpathlineto{\pgfqpoint{2.753737in}{2.862609in}}%
\pgfpathlineto{\pgfqpoint{2.754301in}{2.862393in}}%
\pgfpathlineto{\pgfqpoint{2.790983in}{2.870995in}}%
\pgfpathlineto{\pgfqpoint{2.791548in}{2.870202in}}%
\pgfpathlineto{\pgfqpoint{2.792112in}{2.870255in}}%
\pgfpathlineto{\pgfqpoint{2.792677in}{2.871913in}}%
\pgfpathlineto{\pgfqpoint{2.793805in}{2.990011in}}%
\pgfpathlineto{\pgfqpoint{2.794370in}{2.990298in}}%
\pgfpathlineto{\pgfqpoint{2.794934in}{2.990182in}}%
\pgfpathlineto{\pgfqpoint{2.797191in}{2.980619in}}%
\pgfpathlineto{\pgfqpoint{2.803963in}{2.951169in}}%
\pgfpathlineto{\pgfqpoint{2.804528in}{2.950547in}}%
\pgfpathlineto{\pgfqpoint{2.806785in}{2.950527in}}%
\pgfpathlineto{\pgfqpoint{2.807350in}{2.947743in}}%
\pgfpathlineto{\pgfqpoint{2.807914in}{2.933747in}}%
\pgfpathlineto{\pgfqpoint{2.808478in}{2.927980in}}%
\pgfpathlineto{\pgfqpoint{2.809043in}{2.927496in}}%
\pgfpathlineto{\pgfqpoint{2.818072in}{2.927477in}}%
\pgfpathlineto{\pgfqpoint{2.818636in}{2.926296in}}%
\pgfpathlineto{\pgfqpoint{2.834438in}{2.873957in}}%
\pgfpathlineto{\pgfqpoint{2.835002in}{2.873214in}}%
\pgfpathlineto{\pgfqpoint{2.840082in}{2.871067in}}%
\pgfpathlineto{\pgfqpoint{2.888051in}{2.850473in}}%
\pgfpathlineto{\pgfqpoint{2.888615in}{2.854279in}}%
\pgfpathlineto{\pgfqpoint{2.889180in}{2.893670in}}%
\pgfpathlineto{\pgfqpoint{2.889744in}{2.866289in}}%
\pgfpathlineto{\pgfqpoint{2.900467in}{2.927568in}}%
\pgfpathlineto{\pgfqpoint{2.901031in}{2.887861in}}%
\pgfpathlineto{\pgfqpoint{2.901595in}{2.900085in}}%
\pgfpathlineto{\pgfqpoint{2.902160in}{2.886062in}}%
\pgfpathlineto{\pgfqpoint{2.902724in}{2.912029in}}%
\pgfpathlineto{\pgfqpoint{2.903288in}{2.910747in}}%
\pgfpathlineto{\pgfqpoint{2.903853in}{2.924262in}}%
\pgfpathlineto{\pgfqpoint{2.911753in}{2.875366in}}%
\pgfpathlineto{\pgfqpoint{2.912318in}{2.881700in}}%
\pgfpathlineto{\pgfqpoint{2.912882in}{2.880230in}}%
\pgfpathlineto{\pgfqpoint{2.913446in}{2.875793in}}%
\pgfpathlineto{\pgfqpoint{2.914575in}{2.930381in}}%
\pgfpathlineto{\pgfqpoint{2.915140in}{2.926864in}}%
\pgfpathlineto{\pgfqpoint{2.915704in}{2.944655in}}%
\pgfpathlineto{\pgfqpoint{2.916268in}{2.950317in}}%
\pgfpathlineto{\pgfqpoint{2.916833in}{2.946845in}}%
\pgfpathlineto{\pgfqpoint{2.917961in}{2.945853in}}%
\pgfpathlineto{\pgfqpoint{2.919090in}{2.936412in}}%
\pgfpathlineto{\pgfqpoint{2.925298in}{2.879338in}}%
\pgfpathlineto{\pgfqpoint{2.926426in}{2.906263in}}%
\pgfpathlineto{\pgfqpoint{2.926991in}{2.941167in}}%
\pgfpathlineto{\pgfqpoint{2.927555in}{2.946542in}}%
\pgfpathlineto{\pgfqpoint{2.928119in}{2.947626in}}%
\pgfpathlineto{\pgfqpoint{2.928684in}{2.946871in}}%
\pgfpathlineto{\pgfqpoint{2.929813in}{2.951574in}}%
\pgfpathlineto{\pgfqpoint{2.930941in}{2.947675in}}%
\pgfpathlineto{\pgfqpoint{2.941099in}{2.954909in}}%
\pgfpathlineto{\pgfqpoint{2.941664in}{2.951473in}}%
\pgfpathlineto{\pgfqpoint{2.942228in}{2.956495in}}%
\pgfpathlineto{\pgfqpoint{2.942792in}{2.956476in}}%
\pgfpathlineto{\pgfqpoint{2.943357in}{2.952449in}}%
\pgfpathlineto{\pgfqpoint{2.943921in}{2.945133in}}%
\pgfpathlineto{\pgfqpoint{2.944486in}{2.944978in}}%
\pgfpathlineto{\pgfqpoint{2.983425in}{2.960333in}}%
\pgfpathlineto{\pgfqpoint{2.983990in}{2.961779in}}%
\pgfpathlineto{\pgfqpoint{2.984554in}{2.953364in}}%
\pgfpathlineto{\pgfqpoint{2.985118in}{2.950681in}}%
\pgfpathlineto{\pgfqpoint{2.995841in}{2.950023in}}%
\pgfpathlineto{\pgfqpoint{2.996405in}{2.950472in}}%
\pgfpathlineto{\pgfqpoint{2.996970in}{2.961807in}}%
\pgfpathlineto{\pgfqpoint{2.997534in}{2.964605in}}%
\pgfpathlineto{\pgfqpoint{3.007128in}{2.964605in}}%
\pgfpathlineto{\pgfqpoint{3.008257in}{2.949644in}}%
\pgfpathlineto{\pgfqpoint{3.009950in}{2.949625in}}%
\pgfpathlineto{\pgfqpoint{3.011078in}{2.950348in}}%
\pgfpathlineto{\pgfqpoint{3.012207in}{2.952171in}}%
\pgfpathlineto{\pgfqpoint{3.022365in}{2.961192in}}%
\pgfpathlineto{\pgfqpoint{3.022930in}{2.960625in}}%
\pgfpathlineto{\pgfqpoint{3.023494in}{2.957822in}}%
\pgfpathlineto{\pgfqpoint{3.024058in}{2.951709in}}%
\pgfpathlineto{\pgfqpoint{3.024623in}{2.949155in}}%
\pgfpathlineto{\pgfqpoint{3.025187in}{2.949279in}}%
\pgfpathlineto{\pgfqpoint{3.027444in}{2.954314in}}%
\pgfpathlineto{\pgfqpoint{3.028573in}{2.955321in}}%
\pgfpathlineto{\pgfqpoint{3.033088in}{2.964180in}}%
\pgfpathlineto{\pgfqpoint{3.033652in}{2.966537in}}%
\pgfpathlineto{\pgfqpoint{3.034216in}{2.972344in}}%
\pgfpathlineto{\pgfqpoint{3.034781in}{2.974835in}}%
\pgfpathlineto{\pgfqpoint{3.036474in}{2.956621in}}%
\pgfpathlineto{\pgfqpoint{3.037603in}{2.957229in}}%
\pgfpathlineto{\pgfqpoint{3.038167in}{2.958795in}}%
\pgfpathlineto{\pgfqpoint{3.038731in}{2.974285in}}%
\pgfpathlineto{\pgfqpoint{3.042682in}{2.974244in}}%
\pgfpathlineto{\pgfqpoint{3.074285in}{2.974454in}}%
\pgfpathlineto{\pgfqpoint{3.074849in}{2.974913in}}%
\pgfpathlineto{\pgfqpoint{3.075978in}{2.974720in}}%
\pgfpathlineto{\pgfqpoint{3.076542in}{2.974480in}}%
\pgfpathlineto{\pgfqpoint{3.077107in}{2.972672in}}%
\pgfpathlineto{\pgfqpoint{3.077671in}{2.972050in}}%
\pgfpathlineto{\pgfqpoint{3.078800in}{2.973988in}}%
\pgfpathlineto{\pgfqpoint{3.079364in}{2.974128in}}%
\pgfpathlineto{\pgfqpoint{3.087265in}{2.971650in}}%
\pgfpathlineto{\pgfqpoint{3.087829in}{2.973873in}}%
\pgfpathlineto{\pgfqpoint{3.088394in}{2.972626in}}%
\pgfpathlineto{\pgfqpoint{3.089522in}{2.973031in}}%
\pgfpathlineto{\pgfqpoint{3.090087in}{2.973243in}}%
\pgfpathlineto{\pgfqpoint{3.091215in}{2.972057in}}%
\pgfpathlineto{\pgfqpoint{3.097988in}{2.973193in}}%
\pgfpathlineto{\pgfqpoint{3.183204in}{2.984537in}}%
\pgfpathlineto{\pgfqpoint{3.183768in}{2.981051in}}%
\pgfpathlineto{\pgfqpoint{3.184332in}{2.973977in}}%
\pgfpathlineto{\pgfqpoint{3.184897in}{2.983073in}}%
\pgfpathlineto{\pgfqpoint{3.185461in}{2.985984in}}%
\pgfpathlineto{\pgfqpoint{3.186590in}{2.986258in}}%
\pgfpathlineto{\pgfqpoint{3.212550in}{2.986258in}}%
\pgfpathlineto{\pgfqpoint{3.213114in}{2.986545in}}%
\pgfpathlineto{\pgfqpoint{3.213678in}{2.987272in}}%
\pgfpathlineto{\pgfqpoint{3.226658in}{2.990509in}}%
\pgfpathlineto{\pgfqpoint{3.230044in}{2.990559in}}%
\pgfpathlineto{\pgfqpoint{3.266163in}{2.990563in}}%
\pgfpathlineto{\pgfqpoint{3.266727in}{2.991953in}}%
\pgfpathlineto{\pgfqpoint{3.268420in}{2.999205in}}%
\pgfpathlineto{\pgfqpoint{3.268984in}{2.999636in}}%
\pgfpathlineto{\pgfqpoint{3.282529in}{3.000616in}}%
\pgfpathlineto{\pgfqpoint{3.302281in}{3.000219in}}%
\pgfpathlineto{\pgfqpoint{4.415170in}{3.000357in}}%
\pgfpathlineto{\pgfqpoint{4.416299in}{3.002911in}}%
\pgfpathlineto{\pgfqpoint{4.453546in}{3.002752in}}%
\pgfpathlineto{\pgfqpoint{5.080534in}{3.001579in}}%
\pgfpathlineto{\pgfqpoint{5.081098in}{3.001817in}}%
\pgfpathlineto{\pgfqpoint{5.082227in}{3.004118in}}%
\pgfpathlineto{\pgfqpoint{5.084484in}{3.004186in}}%
\pgfpathlineto{\pgfqpoint{6.004368in}{3.002609in}}%
\pgfpathlineto{\pgfqpoint{6.004368in}{3.002609in}}%
\pgfusepath{stroke}%
\end{pgfscope}%
\begin{pgfscope}%
\pgfsetrectcap%
\pgfsetmiterjoin%
\pgfsetlinewidth{0.501875pt}%
\definecolor{currentstroke}{rgb}{0.000000,0.000000,0.000000}%
\pgfsetstrokecolor{currentstroke}%
\pgfsetdash{}{0pt}%
\pgfpathmoveto{\pgfqpoint{0.481681in}{1.080890in}}%
\pgfpathlineto{\pgfqpoint{0.481681in}{3.227753in}}%
\pgfusepath{stroke}%
\end{pgfscope}%
\begin{pgfscope}%
\pgfsetrectcap%
\pgfsetmiterjoin%
\pgfsetlinewidth{0.501875pt}%
\definecolor{currentstroke}{rgb}{0.000000,0.000000,0.000000}%
\pgfsetstrokecolor{currentstroke}%
\pgfsetdash{}{0pt}%
\pgfpathmoveto{\pgfqpoint{6.267353in}{1.080890in}}%
\pgfpathlineto{\pgfqpoint{6.267353in}{3.227753in}}%
\pgfusepath{stroke}%
\end{pgfscope}%
\begin{pgfscope}%
\pgfsetrectcap%
\pgfsetmiterjoin%
\pgfsetlinewidth{0.501875pt}%
\definecolor{currentstroke}{rgb}{0.000000,0.000000,0.000000}%
\pgfsetstrokecolor{currentstroke}%
\pgfsetdash{}{0pt}%
\pgfpathmoveto{\pgfqpoint{0.481681in}{1.080890in}}%
\pgfpathlineto{\pgfqpoint{6.267353in}{1.080890in}}%
\pgfusepath{stroke}%
\end{pgfscope}%
\begin{pgfscope}%
\pgfsetrectcap%
\pgfsetmiterjoin%
\pgfsetlinewidth{0.501875pt}%
\definecolor{currentstroke}{rgb}{0.000000,0.000000,0.000000}%
\pgfsetstrokecolor{currentstroke}%
\pgfsetdash{}{0pt}%
\pgfpathmoveto{\pgfqpoint{0.481681in}{3.227753in}}%
\pgfpathlineto{\pgfqpoint{6.267353in}{3.227753in}}%
\pgfusepath{stroke}%
\end{pgfscope}%
\begin{pgfscope}%
\pgfsetrectcap%
\pgfsetroundjoin%
\pgfsetlinewidth{0.401500pt}%
\definecolor{currentstroke}{rgb}{0.000000,0.070588,0.098039}%
\pgfsetstrokecolor{currentstroke}%
\pgfsetdash{}{0pt}%
\pgfpathmoveto{\pgfqpoint{0.569181in}{3.105934in}}%
\pgfpathlineto{\pgfqpoint{0.666403in}{3.105934in}}%
\pgfpathlineto{\pgfqpoint{0.763625in}{3.105934in}}%
\pgfusepath{stroke}%
\end{pgfscope}%
\begin{pgfscope}%
\definecolor{textcolor}{rgb}{0.000000,0.000000,0.000000}%
\pgfsetstrokecolor{textcolor}%
\pgfsetfillcolor{textcolor}%
\pgftext[x=0.841403in,y=3.071906in,left,base]{\color{textcolor}\rmfamily\fontsize{7.000000}{8.400000}\selectfont Story Points}%
\end{pgfscope}%
\begin{pgfscope}%
\pgfsetrectcap%
\pgfsetroundjoin%
\pgfsetlinewidth{0.200750pt}%
\definecolor{currentstroke}{rgb}{0.682353,0.125490,0.070588}%
\pgfsetstrokecolor{currentstroke}%
\pgfsetdash{}{0pt}%
\pgfpathmoveto{\pgfqpoint{0.569181in}{2.969142in}}%
\pgfpathlineto{\pgfqpoint{0.666403in}{2.969142in}}%
\pgfpathlineto{\pgfqpoint{0.763625in}{2.969142in}}%
\pgfusepath{stroke}%
\end{pgfscope}%
\begin{pgfscope}%
\definecolor{textcolor}{rgb}{0.000000,0.000000,0.000000}%
\pgfsetstrokecolor{textcolor}%
\pgfsetfillcolor{textcolor}%
\pgftext[x=0.841403in,y=2.935114in,left,base]{\color{textcolor}\rmfamily\fontsize{7.000000}{8.400000}\selectfont Logische Codezeilen}%
\end{pgfscope}%
\begin{pgfscope}%
\pgfsetrectcap%
\pgfsetroundjoin%
\pgfsetlinewidth{0.200750pt}%
\definecolor{currentstroke}{rgb}{0.000000,0.372549,0.450980}%
\pgfsetstrokecolor{currentstroke}%
\pgfsetdash{}{0pt}%
\pgfpathmoveto{\pgfqpoint{0.569181in}{2.832545in}}%
\pgfpathlineto{\pgfqpoint{0.666403in}{2.832545in}}%
\pgfpathlineto{\pgfqpoint{0.763625in}{2.832545in}}%
\pgfusepath{stroke}%
\end{pgfscope}%
\begin{pgfscope}%
\definecolor{textcolor}{rgb}{0.000000,0.000000,0.000000}%
\pgfsetstrokecolor{textcolor}%
\pgfsetfillcolor{textcolor}%
\pgftext[x=0.841403in,y=2.798517in,left,base]{\color{textcolor}\rmfamily\fontsize{7.000000}{8.400000}\selectfont Zyklomatische Komplexität}%
\end{pgfscope}%
\begin{pgfscope}%
\pgfsetrectcap%
\pgfsetroundjoin%
\pgfsetlinewidth{0.200750pt}%
\definecolor{currentstroke}{rgb}{0.580392,0.823529,0.741176}%
\pgfsetstrokecolor{currentstroke}%
\pgfsetdash{}{0pt}%
\pgfpathmoveto{\pgfqpoint{0.569181in}{2.696045in}}%
\pgfpathlineto{\pgfqpoint{0.666403in}{2.696045in}}%
\pgfpathlineto{\pgfqpoint{0.763625in}{2.696045in}}%
\pgfusepath{stroke}%
\end{pgfscope}%
\begin{pgfscope}%
\definecolor{textcolor}{rgb}{0.000000,0.000000,0.000000}%
\pgfsetstrokecolor{textcolor}%
\pgfsetfillcolor{textcolor}%
\pgftext[x=0.841403in,y=2.662017in,left,base]{\color{textcolor}\rmfamily\fontsize{7.000000}{8.400000}\selectfont Halstead Aufwand}%
\end{pgfscope}%
\begin{pgfscope}%
\pgfsetrectcap%
\pgfsetroundjoin%
\pgfsetlinewidth{0.200750pt}%
\definecolor{currentstroke}{rgb}{0.933333,0.607843,0.000000}%
\pgfsetstrokecolor{currentstroke}%
\pgfsetdash{}{0pt}%
\pgfpathmoveto{\pgfqpoint{0.569181in}{2.560517in}}%
\pgfpathlineto{\pgfqpoint{0.666403in}{2.560517in}}%
\pgfpathlineto{\pgfqpoint{0.763625in}{2.560517in}}%
\pgfusepath{stroke}%
\end{pgfscope}%
\begin{pgfscope}%
\definecolor{textcolor}{rgb}{0.000000,0.000000,0.000000}%
\pgfsetstrokecolor{textcolor}%
\pgfsetfillcolor{textcolor}%
\pgftext[x=0.841403in,y=2.526489in,left,base]{\color{textcolor}\rmfamily\fontsize{7.000000}{8.400000}\selectfont Einrückungskomplexität}%
\end{pgfscope}%
\begin{pgfscope}%
\pgfsetbuttcap%
\pgfsetmiterjoin%
\definecolor{currentfill}{rgb}{1.000000,1.000000,1.000000}%
\pgfsetfillcolor{currentfill}%
\pgfsetlinewidth{0.000000pt}%
\definecolor{currentstroke}{rgb}{0.000000,0.000000,0.000000}%
\pgfsetstrokecolor{currentstroke}%
\pgfsetstrokeopacity{0.000000}%
\pgfsetdash{}{0pt}%
\pgfpathmoveto{\pgfqpoint{0.481681in}{0.586309in}}%
\pgfpathlineto{\pgfqpoint{6.267353in}{0.586309in}}%
\pgfpathlineto{\pgfqpoint{6.267353in}{0.893003in}}%
\pgfpathlineto{\pgfqpoint{0.481681in}{0.893003in}}%
\pgfpathlineto{\pgfqpoint{0.481681in}{0.586309in}}%
\pgfpathclose%
\pgfusepath{fill}%
\end{pgfscope}%
\begin{pgfscope}%
\pgfpathrectangle{\pgfqpoint{0.481681in}{0.586309in}}{\pgfqpoint{5.785672in}{0.306695in}}%
\pgfusepath{clip}%
\pgfsetbuttcap%
\pgfsetroundjoin%
\definecolor{currentfill}{rgb}{0.800000,0.788235,0.760784}%
\pgfsetfillcolor{currentfill}%
\pgfsetlinewidth{0.000000pt}%
\definecolor{currentstroke}{rgb}{0.000000,0.000000,0.000000}%
\pgfsetstrokecolor{currentstroke}%
\pgfsetdash{}{0pt}%
\pgfpathmoveto{\pgfqpoint{0.744666in}{0.739656in}}%
\pgfpathlineto{\pgfqpoint{0.744666in}{0.739656in}}%
\pgfpathlineto{\pgfqpoint{0.745230in}{0.739654in}}%
\pgfpathlineto{\pgfqpoint{0.745795in}{0.739652in}}%
\pgfpathlineto{\pgfqpoint{0.746359in}{0.739650in}}%
\pgfpathlineto{\pgfqpoint{0.746923in}{0.739647in}}%
\pgfpathlineto{\pgfqpoint{0.747488in}{0.739645in}}%
\pgfpathlineto{\pgfqpoint{0.748052in}{0.739643in}}%
\pgfpathlineto{\pgfqpoint{0.748616in}{0.739641in}}%
\pgfpathlineto{\pgfqpoint{0.749181in}{0.739639in}}%
\pgfpathlineto{\pgfqpoint{0.749745in}{0.739637in}}%
\pgfpathlineto{\pgfqpoint{0.750309in}{0.739635in}}%
\pgfpathlineto{\pgfqpoint{0.750874in}{0.739632in}}%
\pgfpathlineto{\pgfqpoint{0.751438in}{0.739630in}}%
\pgfpathlineto{\pgfqpoint{0.752002in}{0.739628in}}%
\pgfpathlineto{\pgfqpoint{0.752567in}{0.739626in}}%
\pgfpathlineto{\pgfqpoint{0.753131in}{0.739624in}}%
\pgfpathlineto{\pgfqpoint{0.753695in}{0.739622in}}%
\pgfpathlineto{\pgfqpoint{0.754260in}{0.739619in}}%
\pgfpathlineto{\pgfqpoint{0.754824in}{0.739617in}}%
\pgfpathlineto{\pgfqpoint{0.755389in}{0.739615in}}%
\pgfpathlineto{\pgfqpoint{0.755953in}{0.739613in}}%
\pgfpathlineto{\pgfqpoint{0.756517in}{0.739611in}}%
\pgfpathlineto{\pgfqpoint{0.757082in}{0.739609in}}%
\pgfpathlineto{\pgfqpoint{0.757646in}{0.739607in}}%
\pgfpathlineto{\pgfqpoint{0.758210in}{0.739604in}}%
\pgfpathlineto{\pgfqpoint{0.758775in}{0.739602in}}%
\pgfpathlineto{\pgfqpoint{0.759339in}{0.739600in}}%
\pgfpathlineto{\pgfqpoint{0.759903in}{0.739598in}}%
\pgfpathlineto{\pgfqpoint{0.760468in}{0.739596in}}%
\pgfpathlineto{\pgfqpoint{0.761032in}{0.739594in}}%
\pgfpathlineto{\pgfqpoint{0.761596in}{0.739591in}}%
\pgfpathlineto{\pgfqpoint{0.762161in}{0.739589in}}%
\pgfpathlineto{\pgfqpoint{0.762725in}{0.739587in}}%
\pgfpathlineto{\pgfqpoint{0.763289in}{0.739585in}}%
\pgfpathlineto{\pgfqpoint{0.763854in}{0.739583in}}%
\pgfpathlineto{\pgfqpoint{0.764418in}{0.739581in}}%
\pgfpathlineto{\pgfqpoint{0.764982in}{0.739579in}}%
\pgfpathlineto{\pgfqpoint{0.765547in}{0.739576in}}%
\pgfpathlineto{\pgfqpoint{0.766111in}{0.739574in}}%
\pgfpathlineto{\pgfqpoint{0.766675in}{0.739572in}}%
\pgfpathlineto{\pgfqpoint{0.767240in}{0.739570in}}%
\pgfpathlineto{\pgfqpoint{0.767804in}{0.739568in}}%
\pgfpathlineto{\pgfqpoint{0.768368in}{0.739566in}}%
\pgfpathlineto{\pgfqpoint{0.768933in}{0.739563in}}%
\pgfpathlineto{\pgfqpoint{0.769497in}{0.739561in}}%
\pgfpathlineto{\pgfqpoint{0.770061in}{0.739559in}}%
\pgfpathlineto{\pgfqpoint{0.770626in}{0.739557in}}%
\pgfpathlineto{\pgfqpoint{0.771190in}{0.739555in}}%
\pgfpathlineto{\pgfqpoint{0.771755in}{0.739553in}}%
\pgfpathlineto{\pgfqpoint{0.772319in}{0.739551in}}%
\pgfpathlineto{\pgfqpoint{0.772883in}{0.739548in}}%
\pgfpathlineto{\pgfqpoint{0.773448in}{0.739546in}}%
\pgfpathlineto{\pgfqpoint{0.774012in}{0.739544in}}%
\pgfpathlineto{\pgfqpoint{0.774576in}{0.739542in}}%
\pgfpathlineto{\pgfqpoint{0.775141in}{0.739540in}}%
\pgfpathlineto{\pgfqpoint{0.775705in}{0.739538in}}%
\pgfpathlineto{\pgfqpoint{0.776269in}{0.739535in}}%
\pgfpathlineto{\pgfqpoint{0.776834in}{0.739533in}}%
\pgfpathlineto{\pgfqpoint{0.777398in}{0.739531in}}%
\pgfpathlineto{\pgfqpoint{0.777962in}{0.739529in}}%
\pgfpathlineto{\pgfqpoint{0.778527in}{0.739527in}}%
\pgfpathlineto{\pgfqpoint{0.779091in}{0.739525in}}%
\pgfpathlineto{\pgfqpoint{0.779655in}{0.739523in}}%
\pgfpathlineto{\pgfqpoint{0.780220in}{0.739520in}}%
\pgfpathlineto{\pgfqpoint{0.780784in}{0.739518in}}%
\pgfpathlineto{\pgfqpoint{0.781348in}{0.739516in}}%
\pgfpathlineto{\pgfqpoint{0.781913in}{0.739514in}}%
\pgfpathlineto{\pgfqpoint{0.782477in}{0.739512in}}%
\pgfpathlineto{\pgfqpoint{0.783041in}{0.739510in}}%
\pgfpathlineto{\pgfqpoint{0.783606in}{0.739507in}}%
\pgfpathlineto{\pgfqpoint{0.784170in}{0.739505in}}%
\pgfpathlineto{\pgfqpoint{0.784734in}{0.739503in}}%
\pgfpathlineto{\pgfqpoint{0.785299in}{0.739501in}}%
\pgfpathlineto{\pgfqpoint{0.785863in}{0.739499in}}%
\pgfpathlineto{\pgfqpoint{0.786428in}{0.739497in}}%
\pgfpathlineto{\pgfqpoint{0.786992in}{0.739495in}}%
\pgfpathlineto{\pgfqpoint{0.787556in}{0.739492in}}%
\pgfpathlineto{\pgfqpoint{0.788121in}{0.739490in}}%
\pgfpathlineto{\pgfqpoint{0.788685in}{0.739488in}}%
\pgfpathlineto{\pgfqpoint{0.789249in}{0.739486in}}%
\pgfpathlineto{\pgfqpoint{0.789814in}{0.739484in}}%
\pgfpathlineto{\pgfqpoint{0.790378in}{0.739482in}}%
\pgfpathlineto{\pgfqpoint{0.790942in}{0.739479in}}%
\pgfpathlineto{\pgfqpoint{0.791507in}{0.739477in}}%
\pgfpathlineto{\pgfqpoint{0.792071in}{0.739475in}}%
\pgfpathlineto{\pgfqpoint{0.792635in}{0.739473in}}%
\pgfpathlineto{\pgfqpoint{0.793200in}{0.739471in}}%
\pgfpathlineto{\pgfqpoint{0.793764in}{0.739469in}}%
\pgfpathlineto{\pgfqpoint{0.794328in}{0.739467in}}%
\pgfpathlineto{\pgfqpoint{0.794893in}{0.739464in}}%
\pgfpathlineto{\pgfqpoint{0.795457in}{0.739462in}}%
\pgfpathlineto{\pgfqpoint{0.796021in}{0.739460in}}%
\pgfpathlineto{\pgfqpoint{0.796586in}{0.739458in}}%
\pgfpathlineto{\pgfqpoint{0.797150in}{0.739456in}}%
\pgfpathlineto{\pgfqpoint{0.797714in}{0.739454in}}%
\pgfpathlineto{\pgfqpoint{0.798279in}{0.739451in}}%
\pgfpathlineto{\pgfqpoint{0.798843in}{0.739725in}}%
\pgfpathlineto{\pgfqpoint{0.799407in}{0.741325in}}%
\pgfpathlineto{\pgfqpoint{0.799972in}{0.740947in}}%
\pgfpathlineto{\pgfqpoint{0.800536in}{0.740129in}}%
\pgfpathlineto{\pgfqpoint{0.801101in}{0.740058in}}%
\pgfpathlineto{\pgfqpoint{0.801665in}{0.741308in}}%
\pgfpathlineto{\pgfqpoint{0.802229in}{0.740105in}}%
\pgfpathlineto{\pgfqpoint{0.802794in}{0.741829in}}%
\pgfpathlineto{\pgfqpoint{0.803358in}{0.742894in}}%
\pgfpathlineto{\pgfqpoint{0.803922in}{0.743162in}}%
\pgfpathlineto{\pgfqpoint{0.804487in}{0.743160in}}%
\pgfpathlineto{\pgfqpoint{0.805051in}{0.743158in}}%
\pgfpathlineto{\pgfqpoint{0.805615in}{0.743156in}}%
\pgfpathlineto{\pgfqpoint{0.806180in}{0.743153in}}%
\pgfpathlineto{\pgfqpoint{0.806744in}{0.743151in}}%
\pgfpathlineto{\pgfqpoint{0.807308in}{0.743149in}}%
\pgfpathlineto{\pgfqpoint{0.807873in}{0.743147in}}%
\pgfpathlineto{\pgfqpoint{0.808437in}{0.743145in}}%
\pgfpathlineto{\pgfqpoint{0.809001in}{0.743143in}}%
\pgfpathlineto{\pgfqpoint{0.809566in}{0.743140in}}%
\pgfpathlineto{\pgfqpoint{0.810130in}{0.743138in}}%
\pgfpathlineto{\pgfqpoint{0.810694in}{0.743136in}}%
\pgfpathlineto{\pgfqpoint{0.811259in}{0.743134in}}%
\pgfpathlineto{\pgfqpoint{0.811823in}{0.743132in}}%
\pgfpathlineto{\pgfqpoint{0.812387in}{0.745650in}}%
\pgfpathlineto{\pgfqpoint{0.812952in}{0.753516in}}%
\pgfpathlineto{\pgfqpoint{0.813516in}{0.761556in}}%
\pgfpathlineto{\pgfqpoint{0.814080in}{0.769596in}}%
\pgfpathlineto{\pgfqpoint{0.814645in}{0.767909in}}%
\pgfpathlineto{\pgfqpoint{0.815209in}{0.743879in}}%
\pgfpathlineto{\pgfqpoint{0.815773in}{0.743971in}}%
\pgfpathlineto{\pgfqpoint{0.816338in}{0.743569in}}%
\pgfpathlineto{\pgfqpoint{0.816902in}{0.743203in}}%
\pgfpathlineto{\pgfqpoint{0.817467in}{0.743148in}}%
\pgfpathlineto{\pgfqpoint{0.818031in}{0.743145in}}%
\pgfpathlineto{\pgfqpoint{0.818595in}{0.743143in}}%
\pgfpathlineto{\pgfqpoint{0.819160in}{0.743141in}}%
\pgfpathlineto{\pgfqpoint{0.819724in}{0.743139in}}%
\pgfpathlineto{\pgfqpoint{0.820288in}{0.743136in}}%
\pgfpathlineto{\pgfqpoint{0.820853in}{0.743134in}}%
\pgfpathlineto{\pgfqpoint{0.821417in}{0.743132in}}%
\pgfpathlineto{\pgfqpoint{0.821981in}{0.743130in}}%
\pgfpathlineto{\pgfqpoint{0.822546in}{0.743127in}}%
\pgfpathlineto{\pgfqpoint{0.823110in}{0.743125in}}%
\pgfpathlineto{\pgfqpoint{0.823674in}{0.743123in}}%
\pgfpathlineto{\pgfqpoint{0.824239in}{0.743121in}}%
\pgfpathlineto{\pgfqpoint{0.824803in}{0.743118in}}%
\pgfpathlineto{\pgfqpoint{0.825367in}{0.743116in}}%
\pgfpathlineto{\pgfqpoint{0.825932in}{0.743114in}}%
\pgfpathlineto{\pgfqpoint{0.826496in}{0.766654in}}%
\pgfpathlineto{\pgfqpoint{0.827060in}{0.772932in}}%
\pgfpathlineto{\pgfqpoint{0.827625in}{0.755373in}}%
\pgfpathlineto{\pgfqpoint{0.828189in}{0.764900in}}%
\pgfpathlineto{\pgfqpoint{0.828753in}{0.780851in}}%
\pgfpathlineto{\pgfqpoint{0.829318in}{0.781299in}}%
\pgfpathlineto{\pgfqpoint{0.829882in}{0.778882in}}%
\pgfpathlineto{\pgfqpoint{0.830446in}{0.767692in}}%
\pgfpathlineto{\pgfqpoint{0.831011in}{0.865877in}}%
\pgfpathlineto{\pgfqpoint{0.831575in}{0.861901in}}%
\pgfpathlineto{\pgfqpoint{0.832140in}{0.856650in}}%
\pgfpathlineto{\pgfqpoint{0.832704in}{0.851400in}}%
\pgfpathlineto{\pgfqpoint{0.833268in}{0.846149in}}%
\pgfpathlineto{\pgfqpoint{0.833833in}{0.840899in}}%
\pgfpathlineto{\pgfqpoint{0.834397in}{0.835648in}}%
\pgfpathlineto{\pgfqpoint{0.834961in}{0.830398in}}%
\pgfpathlineto{\pgfqpoint{0.835526in}{0.825147in}}%
\pgfpathlineto{\pgfqpoint{0.836090in}{0.819897in}}%
\pgfpathlineto{\pgfqpoint{0.836654in}{0.814646in}}%
\pgfpathlineto{\pgfqpoint{0.837219in}{0.809396in}}%
\pgfpathlineto{\pgfqpoint{0.837783in}{0.804145in}}%
\pgfpathlineto{\pgfqpoint{0.838347in}{0.798894in}}%
\pgfpathlineto{\pgfqpoint{0.838912in}{0.793644in}}%
\pgfpathlineto{\pgfqpoint{0.839476in}{0.788393in}}%
\pgfpathlineto{\pgfqpoint{0.840040in}{0.783735in}}%
\pgfpathlineto{\pgfqpoint{0.840605in}{0.825463in}}%
\pgfpathlineto{\pgfqpoint{0.841169in}{0.836277in}}%
\pgfpathlineto{\pgfqpoint{0.841733in}{0.750938in}}%
\pgfpathlineto{\pgfqpoint{0.842298in}{0.840850in}}%
\pgfpathlineto{\pgfqpoint{0.842862in}{0.835833in}}%
\pgfpathlineto{\pgfqpoint{0.843426in}{0.854413in}}%
\pgfpathlineto{\pgfqpoint{0.843991in}{0.855458in}}%
\pgfpathlineto{\pgfqpoint{0.844555in}{0.855380in}}%
\pgfpathlineto{\pgfqpoint{0.845119in}{0.855292in}}%
\pgfpathlineto{\pgfqpoint{0.845684in}{0.855203in}}%
\pgfpathlineto{\pgfqpoint{0.846248in}{0.855114in}}%
\pgfpathlineto{\pgfqpoint{0.846813in}{0.854456in}}%
\pgfpathlineto{\pgfqpoint{0.847377in}{0.852003in}}%
\pgfpathlineto{\pgfqpoint{0.847941in}{0.848278in}}%
\pgfpathlineto{\pgfqpoint{0.848506in}{0.836801in}}%
\pgfpathlineto{\pgfqpoint{0.849070in}{0.836698in}}%
\pgfpathlineto{\pgfqpoint{0.849634in}{0.836595in}}%
\pgfpathlineto{\pgfqpoint{0.850199in}{0.836492in}}%
\pgfpathlineto{\pgfqpoint{0.850763in}{0.836389in}}%
\pgfpathlineto{\pgfqpoint{0.851327in}{0.836286in}}%
\pgfpathlineto{\pgfqpoint{0.851892in}{0.836183in}}%
\pgfpathlineto{\pgfqpoint{0.852456in}{0.836080in}}%
\pgfpathlineto{\pgfqpoint{0.853020in}{0.835977in}}%
\pgfpathlineto{\pgfqpoint{0.853585in}{0.835685in}}%
\pgfpathlineto{\pgfqpoint{0.854149in}{0.835252in}}%
\pgfpathlineto{\pgfqpoint{0.854713in}{0.835418in}}%
\pgfpathlineto{\pgfqpoint{0.855278in}{0.837387in}}%
\pgfpathlineto{\pgfqpoint{0.855842in}{0.837350in}}%
\pgfpathlineto{\pgfqpoint{0.856406in}{0.837335in}}%
\pgfpathlineto{\pgfqpoint{0.856971in}{0.837320in}}%
\pgfpathlineto{\pgfqpoint{0.857535in}{0.837304in}}%
\pgfpathlineto{\pgfqpoint{0.858099in}{0.837289in}}%
\pgfpathlineto{\pgfqpoint{0.858664in}{0.837273in}}%
\pgfpathlineto{\pgfqpoint{0.859228in}{0.837258in}}%
\pgfpathlineto{\pgfqpoint{0.859792in}{0.837243in}}%
\pgfpathlineto{\pgfqpoint{0.860357in}{0.837227in}}%
\pgfpathlineto{\pgfqpoint{0.860921in}{0.837212in}}%
\pgfpathlineto{\pgfqpoint{0.861485in}{0.837196in}}%
\pgfpathlineto{\pgfqpoint{0.862050in}{0.837181in}}%
\pgfpathlineto{\pgfqpoint{0.862614in}{0.837166in}}%
\pgfpathlineto{\pgfqpoint{0.863179in}{0.837150in}}%
\pgfpathlineto{\pgfqpoint{0.863743in}{0.837135in}}%
\pgfpathlineto{\pgfqpoint{0.864307in}{0.837119in}}%
\pgfpathlineto{\pgfqpoint{0.864872in}{0.837104in}}%
\pgfpathlineto{\pgfqpoint{0.865436in}{0.837089in}}%
\pgfpathlineto{\pgfqpoint{0.866000in}{0.837073in}}%
\pgfpathlineto{\pgfqpoint{0.866565in}{0.837058in}}%
\pgfpathlineto{\pgfqpoint{0.867129in}{0.837042in}}%
\pgfpathlineto{\pgfqpoint{0.867693in}{0.837027in}}%
\pgfpathlineto{\pgfqpoint{0.868258in}{0.837012in}}%
\pgfpathlineto{\pgfqpoint{0.868822in}{0.836996in}}%
\pgfpathlineto{\pgfqpoint{0.869386in}{0.836981in}}%
\pgfpathlineto{\pgfqpoint{0.869951in}{0.836966in}}%
\pgfpathlineto{\pgfqpoint{0.870515in}{0.836950in}}%
\pgfpathlineto{\pgfqpoint{0.871079in}{0.836935in}}%
\pgfpathlineto{\pgfqpoint{0.871644in}{0.836919in}}%
\pgfpathlineto{\pgfqpoint{0.872208in}{0.836904in}}%
\pgfpathlineto{\pgfqpoint{0.872772in}{0.836889in}}%
\pgfpathlineto{\pgfqpoint{0.873337in}{0.836873in}}%
\pgfpathlineto{\pgfqpoint{0.873901in}{0.836858in}}%
\pgfpathlineto{\pgfqpoint{0.874465in}{0.836842in}}%
\pgfpathlineto{\pgfqpoint{0.875030in}{0.836827in}}%
\pgfpathlineto{\pgfqpoint{0.875594in}{0.836812in}}%
\pgfpathlineto{\pgfqpoint{0.876158in}{0.836796in}}%
\pgfpathlineto{\pgfqpoint{0.876723in}{0.836781in}}%
\pgfpathlineto{\pgfqpoint{0.877287in}{0.836765in}}%
\pgfpathlineto{\pgfqpoint{0.877852in}{0.836750in}}%
\pgfpathlineto{\pgfqpoint{0.878416in}{0.836735in}}%
\pgfpathlineto{\pgfqpoint{0.878980in}{0.836719in}}%
\pgfpathlineto{\pgfqpoint{0.879545in}{0.836704in}}%
\pgfpathlineto{\pgfqpoint{0.880109in}{0.836688in}}%
\pgfpathlineto{\pgfqpoint{0.880673in}{0.836673in}}%
\pgfpathlineto{\pgfqpoint{0.881238in}{0.836658in}}%
\pgfpathlineto{\pgfqpoint{0.881802in}{0.836642in}}%
\pgfpathlineto{\pgfqpoint{0.882366in}{0.836627in}}%
\pgfpathlineto{\pgfqpoint{0.882931in}{0.836611in}}%
\pgfpathlineto{\pgfqpoint{0.883495in}{0.836596in}}%
\pgfpathlineto{\pgfqpoint{0.884059in}{0.836581in}}%
\pgfpathlineto{\pgfqpoint{0.884624in}{0.836565in}}%
\pgfpathlineto{\pgfqpoint{0.885188in}{0.836550in}}%
\pgfpathlineto{\pgfqpoint{0.885752in}{0.836534in}}%
\pgfpathlineto{\pgfqpoint{0.886317in}{0.836519in}}%
\pgfpathlineto{\pgfqpoint{0.886881in}{0.836504in}}%
\pgfpathlineto{\pgfqpoint{0.887445in}{0.836488in}}%
\pgfpathlineto{\pgfqpoint{0.888010in}{0.836473in}}%
\pgfpathlineto{\pgfqpoint{0.888574in}{0.836457in}}%
\pgfpathlineto{\pgfqpoint{0.889138in}{0.836442in}}%
\pgfpathlineto{\pgfqpoint{0.889703in}{0.836427in}}%
\pgfpathlineto{\pgfqpoint{0.890267in}{0.836411in}}%
\pgfpathlineto{\pgfqpoint{0.890831in}{0.836396in}}%
\pgfpathlineto{\pgfqpoint{0.891396in}{0.836380in}}%
\pgfpathlineto{\pgfqpoint{0.891960in}{0.836365in}}%
\pgfpathlineto{\pgfqpoint{0.892525in}{0.836350in}}%
\pgfpathlineto{\pgfqpoint{0.893089in}{0.836334in}}%
\pgfpathlineto{\pgfqpoint{0.893653in}{0.836319in}}%
\pgfpathlineto{\pgfqpoint{0.894218in}{0.836303in}}%
\pgfpathlineto{\pgfqpoint{0.894782in}{0.836288in}}%
\pgfpathlineto{\pgfqpoint{0.895346in}{0.836275in}}%
\pgfpathlineto{\pgfqpoint{0.895911in}{0.836475in}}%
\pgfpathlineto{\pgfqpoint{0.896475in}{0.836819in}}%
\pgfpathlineto{\pgfqpoint{0.897039in}{0.837106in}}%
\pgfpathlineto{\pgfqpoint{0.897604in}{0.836455in}}%
\pgfpathlineto{\pgfqpoint{0.898168in}{0.838715in}}%
\pgfpathlineto{\pgfqpoint{0.898732in}{0.838747in}}%
\pgfpathlineto{\pgfqpoint{0.899297in}{0.838731in}}%
\pgfpathlineto{\pgfqpoint{0.899861in}{0.838716in}}%
\pgfpathlineto{\pgfqpoint{0.900425in}{0.838700in}}%
\pgfpathlineto{\pgfqpoint{0.900990in}{0.838684in}}%
\pgfpathlineto{\pgfqpoint{0.901554in}{0.838669in}}%
\pgfpathlineto{\pgfqpoint{0.902118in}{0.838653in}}%
\pgfpathlineto{\pgfqpoint{0.902683in}{0.838637in}}%
\pgfpathlineto{\pgfqpoint{0.903247in}{0.838622in}}%
\pgfpathlineto{\pgfqpoint{0.903811in}{0.838606in}}%
\pgfpathlineto{\pgfqpoint{0.904376in}{0.838591in}}%
\pgfpathlineto{\pgfqpoint{0.904940in}{0.838575in}}%
\pgfpathlineto{\pgfqpoint{0.905504in}{0.838559in}}%
\pgfpathlineto{\pgfqpoint{0.906069in}{0.838544in}}%
\pgfpathlineto{\pgfqpoint{0.906633in}{0.838528in}}%
\pgfpathlineto{\pgfqpoint{0.907197in}{0.838512in}}%
\pgfpathlineto{\pgfqpoint{0.907762in}{0.838603in}}%
\pgfpathlineto{\pgfqpoint{0.908326in}{0.839117in}}%
\pgfpathlineto{\pgfqpoint{0.908891in}{0.839452in}}%
\pgfpathlineto{\pgfqpoint{0.909455in}{0.838084in}}%
\pgfpathlineto{\pgfqpoint{0.910019in}{0.838952in}}%
\pgfpathlineto{\pgfqpoint{0.910584in}{0.839358in}}%
\pgfpathlineto{\pgfqpoint{0.911148in}{0.837445in}}%
\pgfpathlineto{\pgfqpoint{0.911712in}{0.838383in}}%
\pgfpathlineto{\pgfqpoint{0.912277in}{0.838477in}}%
\pgfpathlineto{\pgfqpoint{0.912841in}{0.838460in}}%
\pgfpathlineto{\pgfqpoint{0.913405in}{0.838444in}}%
\pgfpathlineto{\pgfqpoint{0.913970in}{0.838428in}}%
\pgfpathlineto{\pgfqpoint{0.914534in}{0.838412in}}%
\pgfpathlineto{\pgfqpoint{0.915098in}{0.838378in}}%
\pgfpathlineto{\pgfqpoint{0.915663in}{0.838327in}}%
\pgfpathlineto{\pgfqpoint{0.916227in}{0.838275in}}%
\pgfpathlineto{\pgfqpoint{0.916791in}{0.838224in}}%
\pgfpathlineto{\pgfqpoint{0.917356in}{0.838173in}}%
\pgfpathlineto{\pgfqpoint{0.917920in}{0.838121in}}%
\pgfpathlineto{\pgfqpoint{0.918484in}{0.838070in}}%
\pgfpathlineto{\pgfqpoint{0.919049in}{0.838019in}}%
\pgfpathlineto{\pgfqpoint{0.919613in}{0.837967in}}%
\pgfpathlineto{\pgfqpoint{0.920177in}{0.837916in}}%
\pgfpathlineto{\pgfqpoint{0.920742in}{0.837865in}}%
\pgfpathlineto{\pgfqpoint{0.921306in}{0.837813in}}%
\pgfpathlineto{\pgfqpoint{0.921870in}{0.837762in}}%
\pgfpathlineto{\pgfqpoint{0.922435in}{0.838896in}}%
\pgfpathlineto{\pgfqpoint{0.922999in}{0.842431in}}%
\pgfpathlineto{\pgfqpoint{0.923564in}{0.848178in}}%
\pgfpathlineto{\pgfqpoint{0.924128in}{0.850283in}}%
\pgfpathlineto{\pgfqpoint{0.924692in}{0.850337in}}%
\pgfpathlineto{\pgfqpoint{0.925257in}{0.850451in}}%
\pgfpathlineto{\pgfqpoint{0.925821in}{0.850407in}}%
\pgfpathlineto{\pgfqpoint{0.926385in}{0.850356in}}%
\pgfpathlineto{\pgfqpoint{0.926950in}{0.850304in}}%
\pgfpathlineto{\pgfqpoint{0.927514in}{0.850253in}}%
\pgfpathlineto{\pgfqpoint{0.928078in}{0.850201in}}%
\pgfpathlineto{\pgfqpoint{0.928643in}{0.850148in}}%
\pgfpathlineto{\pgfqpoint{0.929207in}{0.849863in}}%
\pgfpathlineto{\pgfqpoint{0.929771in}{0.849425in}}%
\pgfpathlineto{\pgfqpoint{0.930336in}{0.848986in}}%
\pgfpathlineto{\pgfqpoint{0.930900in}{0.848548in}}%
\pgfpathlineto{\pgfqpoint{0.931464in}{0.848110in}}%
\pgfpathlineto{\pgfqpoint{0.932029in}{0.847671in}}%
\pgfpathlineto{\pgfqpoint{0.932593in}{0.847233in}}%
\pgfpathlineto{\pgfqpoint{0.933157in}{0.846795in}}%
\pgfpathlineto{\pgfqpoint{0.933722in}{0.846356in}}%
\pgfpathlineto{\pgfqpoint{0.934286in}{0.845981in}}%
\pgfpathlineto{\pgfqpoint{0.934850in}{0.849523in}}%
\pgfpathlineto{\pgfqpoint{0.935415in}{0.850038in}}%
\pgfpathlineto{\pgfqpoint{0.935979in}{0.849593in}}%
\pgfpathlineto{\pgfqpoint{0.936543in}{0.843006in}}%
\pgfpathlineto{\pgfqpoint{0.937108in}{0.818081in}}%
\pgfpathlineto{\pgfqpoint{0.937672in}{0.791935in}}%
\pgfpathlineto{\pgfqpoint{0.938237in}{0.765788in}}%
\pgfpathlineto{\pgfqpoint{0.938801in}{0.745498in}}%
\pgfpathlineto{\pgfqpoint{0.939365in}{0.743698in}}%
\pgfpathlineto{\pgfqpoint{0.939930in}{0.742684in}}%
\pgfpathlineto{\pgfqpoint{0.940494in}{0.741267in}}%
\pgfpathlineto{\pgfqpoint{0.941058in}{0.739850in}}%
\pgfpathlineto{\pgfqpoint{0.941623in}{0.738432in}}%
\pgfpathlineto{\pgfqpoint{0.942187in}{0.737015in}}%
\pgfpathlineto{\pgfqpoint{0.942751in}{0.735628in}}%
\pgfpathlineto{\pgfqpoint{0.943316in}{0.735170in}}%
\pgfpathlineto{\pgfqpoint{0.943880in}{0.735149in}}%
\pgfpathlineto{\pgfqpoint{0.944444in}{0.735129in}}%
\pgfpathlineto{\pgfqpoint{0.945009in}{0.735109in}}%
\pgfpathlineto{\pgfqpoint{0.945573in}{0.735088in}}%
\pgfpathlineto{\pgfqpoint{0.946137in}{0.735068in}}%
\pgfpathlineto{\pgfqpoint{0.946702in}{0.735047in}}%
\pgfpathlineto{\pgfqpoint{0.947266in}{0.735027in}}%
\pgfpathlineto{\pgfqpoint{0.947830in}{0.735048in}}%
\pgfpathlineto{\pgfqpoint{0.948395in}{0.735311in}}%
\pgfpathlineto{\pgfqpoint{0.948959in}{0.735612in}}%
\pgfpathlineto{\pgfqpoint{0.949523in}{0.735913in}}%
\pgfpathlineto{\pgfqpoint{0.950088in}{0.735764in}}%
\pgfpathlineto{\pgfqpoint{0.950652in}{0.735304in}}%
\pgfpathlineto{\pgfqpoint{0.951216in}{0.735506in}}%
\pgfpathlineto{\pgfqpoint{0.951781in}{0.735486in}}%
\pgfpathlineto{\pgfqpoint{0.952345in}{0.735466in}}%
\pgfpathlineto{\pgfqpoint{0.952910in}{0.735446in}}%
\pgfpathlineto{\pgfqpoint{0.953474in}{0.735426in}}%
\pgfpathlineto{\pgfqpoint{0.954038in}{0.735406in}}%
\pgfpathlineto{\pgfqpoint{0.954603in}{0.735386in}}%
\pgfpathlineto{\pgfqpoint{0.955167in}{0.735366in}}%
\pgfpathlineto{\pgfqpoint{0.955731in}{0.735346in}}%
\pgfpathlineto{\pgfqpoint{0.956296in}{0.735325in}}%
\pgfpathlineto{\pgfqpoint{0.956860in}{0.735305in}}%
\pgfpathlineto{\pgfqpoint{0.957424in}{0.735285in}}%
\pgfpathlineto{\pgfqpoint{0.957989in}{0.735265in}}%
\pgfpathlineto{\pgfqpoint{0.958553in}{0.735245in}}%
\pgfpathlineto{\pgfqpoint{0.959117in}{0.735225in}}%
\pgfpathlineto{\pgfqpoint{0.959682in}{0.735205in}}%
\pgfpathlineto{\pgfqpoint{0.960246in}{0.735185in}}%
\pgfpathlineto{\pgfqpoint{0.960810in}{0.735165in}}%
\pgfpathlineto{\pgfqpoint{0.961375in}{0.735144in}}%
\pgfpathlineto{\pgfqpoint{0.961939in}{0.735124in}}%
\pgfpathlineto{\pgfqpoint{0.962503in}{0.735104in}}%
\pgfpathlineto{\pgfqpoint{0.963068in}{0.735084in}}%
\pgfpathlineto{\pgfqpoint{0.963632in}{0.735064in}}%
\pgfpathlineto{\pgfqpoint{0.964196in}{0.735044in}}%
\pgfpathlineto{\pgfqpoint{0.964761in}{0.735024in}}%
\pgfpathlineto{\pgfqpoint{0.965325in}{0.735004in}}%
\pgfpathlineto{\pgfqpoint{0.965889in}{0.734984in}}%
\pgfpathlineto{\pgfqpoint{0.966454in}{0.734964in}}%
\pgfpathlineto{\pgfqpoint{0.967018in}{0.734943in}}%
\pgfpathlineto{\pgfqpoint{0.967582in}{0.734923in}}%
\pgfpathlineto{\pgfqpoint{0.968147in}{0.734903in}}%
\pgfpathlineto{\pgfqpoint{0.968711in}{0.734883in}}%
\pgfpathlineto{\pgfqpoint{0.969276in}{0.734863in}}%
\pgfpathlineto{\pgfqpoint{0.969840in}{0.734843in}}%
\pgfpathlineto{\pgfqpoint{0.970404in}{0.734823in}}%
\pgfpathlineto{\pgfqpoint{0.970969in}{0.734803in}}%
\pgfpathlineto{\pgfqpoint{0.971533in}{0.734783in}}%
\pgfpathlineto{\pgfqpoint{0.972097in}{0.734762in}}%
\pgfpathlineto{\pgfqpoint{0.972662in}{0.734742in}}%
\pgfpathlineto{\pgfqpoint{0.973226in}{0.734722in}}%
\pgfpathlineto{\pgfqpoint{0.973790in}{0.734702in}}%
\pgfpathlineto{\pgfqpoint{0.974355in}{0.734682in}}%
\pgfpathlineto{\pgfqpoint{0.974919in}{0.734662in}}%
\pgfpathlineto{\pgfqpoint{0.975483in}{0.734642in}}%
\pgfpathlineto{\pgfqpoint{0.976048in}{0.734622in}}%
\pgfpathlineto{\pgfqpoint{0.976612in}{0.734602in}}%
\pgfpathlineto{\pgfqpoint{0.977176in}{0.734581in}}%
\pgfpathlineto{\pgfqpoint{0.977741in}{0.734561in}}%
\pgfpathlineto{\pgfqpoint{0.978305in}{0.734541in}}%
\pgfpathlineto{\pgfqpoint{0.978869in}{0.734521in}}%
\pgfpathlineto{\pgfqpoint{0.979434in}{0.734501in}}%
\pgfpathlineto{\pgfqpoint{0.979998in}{0.734481in}}%
\pgfpathlineto{\pgfqpoint{0.980562in}{0.734461in}}%
\pgfpathlineto{\pgfqpoint{0.981127in}{0.734441in}}%
\pgfpathlineto{\pgfqpoint{0.981691in}{0.734421in}}%
\pgfpathlineto{\pgfqpoint{0.982255in}{0.734401in}}%
\pgfpathlineto{\pgfqpoint{0.982820in}{0.734380in}}%
\pgfpathlineto{\pgfqpoint{0.983384in}{0.734360in}}%
\pgfpathlineto{\pgfqpoint{0.983949in}{0.734340in}}%
\pgfpathlineto{\pgfqpoint{0.984513in}{0.734320in}}%
\pgfpathlineto{\pgfqpoint{0.985077in}{0.734300in}}%
\pgfpathlineto{\pgfqpoint{0.985642in}{0.734280in}}%
\pgfpathlineto{\pgfqpoint{0.986206in}{0.734260in}}%
\pgfpathlineto{\pgfqpoint{0.986770in}{0.734240in}}%
\pgfpathlineto{\pgfqpoint{0.987335in}{0.734220in}}%
\pgfpathlineto{\pgfqpoint{0.987899in}{0.734199in}}%
\pgfpathlineto{\pgfqpoint{0.988463in}{0.734179in}}%
\pgfpathlineto{\pgfqpoint{0.989028in}{0.734159in}}%
\pgfpathlineto{\pgfqpoint{0.989592in}{0.734139in}}%
\pgfpathlineto{\pgfqpoint{0.990156in}{0.734119in}}%
\pgfpathlineto{\pgfqpoint{0.990721in}{0.734099in}}%
\pgfpathlineto{\pgfqpoint{0.991285in}{0.734285in}}%
\pgfpathlineto{\pgfqpoint{0.991849in}{0.734920in}}%
\pgfpathlineto{\pgfqpoint{0.992414in}{0.734977in}}%
\pgfpathlineto{\pgfqpoint{0.992978in}{0.735122in}}%
\pgfpathlineto{\pgfqpoint{0.993542in}{0.735831in}}%
\pgfpathlineto{\pgfqpoint{0.994107in}{0.735946in}}%
\pgfpathlineto{\pgfqpoint{0.994671in}{0.736043in}}%
\pgfpathlineto{\pgfqpoint{0.995235in}{0.735980in}}%
\pgfpathlineto{\pgfqpoint{0.995800in}{0.735916in}}%
\pgfpathlineto{\pgfqpoint{0.996364in}{0.735852in}}%
\pgfpathlineto{\pgfqpoint{0.996928in}{0.735788in}}%
\pgfpathlineto{\pgfqpoint{0.997493in}{0.735719in}}%
\pgfpathlineto{\pgfqpoint{0.998057in}{0.735643in}}%
\pgfpathlineto{\pgfqpoint{0.998622in}{0.735566in}}%
\pgfpathlineto{\pgfqpoint{0.999186in}{0.735490in}}%
\pgfpathlineto{\pgfqpoint{0.999750in}{0.735413in}}%
\pgfpathlineto{\pgfqpoint{1.000315in}{0.735336in}}%
\pgfpathlineto{\pgfqpoint{1.000879in}{0.735260in}}%
\pgfpathlineto{\pgfqpoint{1.001443in}{0.735183in}}%
\pgfpathlineto{\pgfqpoint{1.002008in}{0.735106in}}%
\pgfpathlineto{\pgfqpoint{1.002572in}{0.735030in}}%
\pgfpathlineto{\pgfqpoint{1.003136in}{0.734775in}}%
\pgfpathlineto{\pgfqpoint{1.003701in}{0.735557in}}%
\pgfpathlineto{\pgfqpoint{1.004265in}{0.734915in}}%
\pgfpathlineto{\pgfqpoint{1.004829in}{0.736719in}}%
\pgfpathlineto{\pgfqpoint{1.005394in}{0.737159in}}%
\pgfpathlineto{\pgfqpoint{1.005958in}{0.737629in}}%
\pgfpathlineto{\pgfqpoint{1.006522in}{0.737307in}}%
\pgfpathlineto{\pgfqpoint{1.007087in}{0.737874in}}%
\pgfpathlineto{\pgfqpoint{1.007651in}{0.736579in}}%
\pgfpathlineto{\pgfqpoint{1.008215in}{0.736892in}}%
\pgfpathlineto{\pgfqpoint{1.008780in}{0.737206in}}%
\pgfpathlineto{\pgfqpoint{1.009344in}{0.737519in}}%
\pgfpathlineto{\pgfqpoint{1.009908in}{0.737833in}}%
\pgfpathlineto{\pgfqpoint{1.010473in}{0.738146in}}%
\pgfpathlineto{\pgfqpoint{1.011037in}{0.738460in}}%
\pgfpathlineto{\pgfqpoint{1.011601in}{0.738773in}}%
\pgfpathlineto{\pgfqpoint{1.012166in}{0.739087in}}%
\pgfpathlineto{\pgfqpoint{1.012730in}{0.739400in}}%
\pgfpathlineto{\pgfqpoint{1.013294in}{0.739714in}}%
\pgfpathlineto{\pgfqpoint{1.013859in}{0.740027in}}%
\pgfpathlineto{\pgfqpoint{1.014423in}{0.740341in}}%
\pgfpathlineto{\pgfqpoint{1.014988in}{0.740654in}}%
\pgfpathlineto{\pgfqpoint{1.015552in}{0.740968in}}%
\pgfpathlineto{\pgfqpoint{1.016116in}{0.739754in}}%
\pgfpathlineto{\pgfqpoint{1.016681in}{0.737891in}}%
\pgfpathlineto{\pgfqpoint{1.017245in}{0.741866in}}%
\pgfpathlineto{\pgfqpoint{1.017809in}{0.748504in}}%
\pgfpathlineto{\pgfqpoint{1.018374in}{0.748676in}}%
\pgfpathlineto{\pgfqpoint{1.018938in}{0.748681in}}%
\pgfpathlineto{\pgfqpoint{1.019502in}{0.750378in}}%
\pgfpathlineto{\pgfqpoint{1.020067in}{0.752955in}}%
\pgfpathlineto{\pgfqpoint{1.020631in}{0.744889in}}%
\pgfpathlineto{\pgfqpoint{1.021195in}{0.751417in}}%
\pgfpathlineto{\pgfqpoint{1.021760in}{0.750564in}}%
\pgfpathlineto{\pgfqpoint{1.022324in}{0.749711in}}%
\pgfpathlineto{\pgfqpoint{1.022888in}{0.748858in}}%
\pgfpathlineto{\pgfqpoint{1.023453in}{0.748006in}}%
\pgfpathlineto{\pgfqpoint{1.024017in}{0.747153in}}%
\pgfpathlineto{\pgfqpoint{1.024581in}{0.746634in}}%
\pgfpathlineto{\pgfqpoint{1.025146in}{0.746746in}}%
\pgfpathlineto{\pgfqpoint{1.025710in}{0.746872in}}%
\pgfpathlineto{\pgfqpoint{1.026274in}{0.746999in}}%
\pgfpathlineto{\pgfqpoint{1.026839in}{0.747126in}}%
\pgfpathlineto{\pgfqpoint{1.027403in}{0.747253in}}%
\pgfpathlineto{\pgfqpoint{1.027967in}{0.747380in}}%
\pgfpathlineto{\pgfqpoint{1.028532in}{0.747507in}}%
\pgfpathlineto{\pgfqpoint{1.029096in}{0.747634in}}%
\pgfpathlineto{\pgfqpoint{1.029661in}{0.747761in}}%
\pgfpathlineto{\pgfqpoint{1.030225in}{0.747888in}}%
\pgfpathlineto{\pgfqpoint{1.030789in}{0.748015in}}%
\pgfpathlineto{\pgfqpoint{1.031354in}{0.748142in}}%
\pgfpathlineto{\pgfqpoint{1.031918in}{0.748269in}}%
\pgfpathlineto{\pgfqpoint{1.032482in}{0.748396in}}%
\pgfpathlineto{\pgfqpoint{1.033047in}{0.748523in}}%
\pgfpathlineto{\pgfqpoint{1.033611in}{0.748650in}}%
\pgfpathlineto{\pgfqpoint{1.034175in}{0.748777in}}%
\pgfpathlineto{\pgfqpoint{1.034740in}{0.748904in}}%
\pgfpathlineto{\pgfqpoint{1.035304in}{0.749031in}}%
\pgfpathlineto{\pgfqpoint{1.035868in}{0.749158in}}%
\pgfpathlineto{\pgfqpoint{1.036433in}{0.749285in}}%
\pgfpathlineto{\pgfqpoint{1.036997in}{0.749412in}}%
\pgfpathlineto{\pgfqpoint{1.037561in}{0.749538in}}%
\pgfpathlineto{\pgfqpoint{1.038126in}{0.749665in}}%
\pgfpathlineto{\pgfqpoint{1.038690in}{0.749792in}}%
\pgfpathlineto{\pgfqpoint{1.039254in}{0.749919in}}%
\pgfpathlineto{\pgfqpoint{1.039819in}{0.750046in}}%
\pgfpathlineto{\pgfqpoint{1.040383in}{0.750173in}}%
\pgfpathlineto{\pgfqpoint{1.040947in}{0.750300in}}%
\pgfpathlineto{\pgfqpoint{1.041512in}{0.750427in}}%
\pgfpathlineto{\pgfqpoint{1.042076in}{0.750554in}}%
\pgfpathlineto{\pgfqpoint{1.042640in}{0.750681in}}%
\pgfpathlineto{\pgfqpoint{1.043205in}{0.740376in}}%
\pgfpathlineto{\pgfqpoint{1.043769in}{0.746061in}}%
\pgfpathlineto{\pgfqpoint{1.044334in}{0.740122in}}%
\pgfpathlineto{\pgfqpoint{1.044898in}{0.749006in}}%
\pgfpathlineto{\pgfqpoint{1.045462in}{0.748563in}}%
\pgfpathlineto{\pgfqpoint{1.046027in}{0.750657in}}%
\pgfpathlineto{\pgfqpoint{1.046591in}{0.747974in}}%
\pgfpathlineto{\pgfqpoint{1.047155in}{0.751002in}}%
\pgfpathlineto{\pgfqpoint{1.047720in}{0.751902in}}%
\pgfpathlineto{\pgfqpoint{1.048284in}{0.751874in}}%
\pgfpathlineto{\pgfqpoint{1.048848in}{0.751846in}}%
\pgfpathlineto{\pgfqpoint{1.049413in}{0.751818in}}%
\pgfpathlineto{\pgfqpoint{1.049977in}{0.751790in}}%
\pgfpathlineto{\pgfqpoint{1.050541in}{0.751762in}}%
\pgfpathlineto{\pgfqpoint{1.051106in}{0.751734in}}%
\pgfpathlineto{\pgfqpoint{1.051670in}{0.751706in}}%
\pgfpathlineto{\pgfqpoint{1.052234in}{0.751678in}}%
\pgfpathlineto{\pgfqpoint{1.052799in}{0.751650in}}%
\pgfpathlineto{\pgfqpoint{1.053363in}{0.751622in}}%
\pgfpathlineto{\pgfqpoint{1.053927in}{0.751594in}}%
\pgfpathlineto{\pgfqpoint{1.054492in}{0.751566in}}%
\pgfpathlineto{\pgfqpoint{1.055056in}{0.751538in}}%
\pgfpathlineto{\pgfqpoint{1.055620in}{0.751510in}}%
\pgfpathlineto{\pgfqpoint{1.056185in}{0.751482in}}%
\pgfpathlineto{\pgfqpoint{1.056749in}{0.751454in}}%
\pgfpathlineto{\pgfqpoint{1.057313in}{0.751426in}}%
\pgfpathlineto{\pgfqpoint{1.057878in}{0.751398in}}%
\pgfpathlineto{\pgfqpoint{1.058442in}{0.751370in}}%
\pgfpathlineto{\pgfqpoint{1.059006in}{0.751341in}}%
\pgfpathlineto{\pgfqpoint{1.059571in}{0.751313in}}%
\pgfpathlineto{\pgfqpoint{1.060135in}{0.751285in}}%
\pgfpathlineto{\pgfqpoint{1.060700in}{0.751257in}}%
\pgfpathlineto{\pgfqpoint{1.061264in}{0.751229in}}%
\pgfpathlineto{\pgfqpoint{1.061828in}{0.751201in}}%
\pgfpathlineto{\pgfqpoint{1.062393in}{0.751173in}}%
\pgfpathlineto{\pgfqpoint{1.062957in}{0.751145in}}%
\pgfpathlineto{\pgfqpoint{1.063521in}{0.751117in}}%
\pgfpathlineto{\pgfqpoint{1.064086in}{0.751089in}}%
\pgfpathlineto{\pgfqpoint{1.064650in}{0.751061in}}%
\pgfpathlineto{\pgfqpoint{1.065214in}{0.751033in}}%
\pgfpathlineto{\pgfqpoint{1.065779in}{0.751005in}}%
\pgfpathlineto{\pgfqpoint{1.066343in}{0.750977in}}%
\pgfpathlineto{\pgfqpoint{1.066907in}{0.750949in}}%
\pgfpathlineto{\pgfqpoint{1.067472in}{0.750921in}}%
\pgfpathlineto{\pgfqpoint{1.068036in}{0.750893in}}%
\pgfpathlineto{\pgfqpoint{1.068600in}{0.750865in}}%
\pgfpathlineto{\pgfqpoint{1.069165in}{0.750837in}}%
\pgfpathlineto{\pgfqpoint{1.069729in}{0.750809in}}%
\pgfpathlineto{\pgfqpoint{1.070293in}{0.750781in}}%
\pgfpathlineto{\pgfqpoint{1.070858in}{0.750753in}}%
\pgfpathlineto{\pgfqpoint{1.071422in}{0.750725in}}%
\pgfpathlineto{\pgfqpoint{1.071986in}{0.750697in}}%
\pgfpathlineto{\pgfqpoint{1.072551in}{0.750669in}}%
\pgfpathlineto{\pgfqpoint{1.073115in}{0.750641in}}%
\pgfpathlineto{\pgfqpoint{1.073679in}{0.750613in}}%
\pgfpathlineto{\pgfqpoint{1.074244in}{0.750584in}}%
\pgfpathlineto{\pgfqpoint{1.074808in}{0.750556in}}%
\pgfpathlineto{\pgfqpoint{1.075373in}{0.750528in}}%
\pgfpathlineto{\pgfqpoint{1.075937in}{0.750500in}}%
\pgfpathlineto{\pgfqpoint{1.076501in}{0.750472in}}%
\pgfpathlineto{\pgfqpoint{1.077066in}{0.750444in}}%
\pgfpathlineto{\pgfqpoint{1.077630in}{0.750416in}}%
\pgfpathlineto{\pgfqpoint{1.078194in}{0.750388in}}%
\pgfpathlineto{\pgfqpoint{1.078759in}{0.750360in}}%
\pgfpathlineto{\pgfqpoint{1.079323in}{0.750332in}}%
\pgfpathlineto{\pgfqpoint{1.079887in}{0.750304in}}%
\pgfpathlineto{\pgfqpoint{1.080452in}{0.750276in}}%
\pgfpathlineto{\pgfqpoint{1.081016in}{0.750248in}}%
\pgfpathlineto{\pgfqpoint{1.081580in}{0.750220in}}%
\pgfpathlineto{\pgfqpoint{1.082145in}{0.750192in}}%
\pgfpathlineto{\pgfqpoint{1.082709in}{0.750164in}}%
\pgfpathlineto{\pgfqpoint{1.083273in}{0.750136in}}%
\pgfpathlineto{\pgfqpoint{1.083838in}{0.750108in}}%
\pgfpathlineto{\pgfqpoint{1.084402in}{0.750080in}}%
\pgfpathlineto{\pgfqpoint{1.084966in}{0.750052in}}%
\pgfpathlineto{\pgfqpoint{1.085531in}{0.750026in}}%
\pgfpathlineto{\pgfqpoint{1.086095in}{0.750078in}}%
\pgfpathlineto{\pgfqpoint{1.086659in}{0.750166in}}%
\pgfpathlineto{\pgfqpoint{1.087224in}{0.750254in}}%
\pgfpathlineto{\pgfqpoint{1.087788in}{0.750342in}}%
\pgfpathlineto{\pgfqpoint{1.088352in}{0.750431in}}%
\pgfpathlineto{\pgfqpoint{1.088917in}{0.750520in}}%
\pgfpathlineto{\pgfqpoint{1.089481in}{0.750617in}}%
\pgfpathlineto{\pgfqpoint{1.090046in}{0.750716in}}%
\pgfpathlineto{\pgfqpoint{1.090610in}{0.750814in}}%
\pgfpathlineto{\pgfqpoint{1.091174in}{0.750912in}}%
\pgfpathlineto{\pgfqpoint{1.091739in}{0.751011in}}%
\pgfpathlineto{\pgfqpoint{1.092303in}{0.751109in}}%
\pgfpathlineto{\pgfqpoint{1.092867in}{0.751208in}}%
\pgfpathlineto{\pgfqpoint{1.093432in}{0.751306in}}%
\pgfpathlineto{\pgfqpoint{1.093996in}{0.751405in}}%
\pgfpathlineto{\pgfqpoint{1.094560in}{0.751503in}}%
\pgfpathlineto{\pgfqpoint{1.095125in}{0.751602in}}%
\pgfpathlineto{\pgfqpoint{1.095689in}{0.751700in}}%
\pgfpathlineto{\pgfqpoint{1.096253in}{0.751798in}}%
\pgfpathlineto{\pgfqpoint{1.096818in}{0.751897in}}%
\pgfpathlineto{\pgfqpoint{1.097382in}{0.751995in}}%
\pgfpathlineto{\pgfqpoint{1.097946in}{0.750659in}}%
\pgfpathlineto{\pgfqpoint{1.098511in}{0.751651in}}%
\pgfpathlineto{\pgfqpoint{1.099075in}{0.752319in}}%
\pgfpathlineto{\pgfqpoint{1.099639in}{0.754447in}}%
\pgfpathlineto{\pgfqpoint{1.100204in}{0.755724in}}%
\pgfpathlineto{\pgfqpoint{1.100768in}{0.757856in}}%
\pgfpathlineto{\pgfqpoint{1.101332in}{0.751790in}}%
\pgfpathlineto{\pgfqpoint{1.101897in}{0.753403in}}%
\pgfpathlineto{\pgfqpoint{1.102461in}{0.753037in}}%
\pgfpathlineto{\pgfqpoint{1.103025in}{0.752868in}}%
\pgfpathlineto{\pgfqpoint{1.103590in}{0.752683in}}%
\pgfpathlineto{\pgfqpoint{1.104154in}{0.752499in}}%
\pgfpathlineto{\pgfqpoint{1.104718in}{0.752314in}}%
\pgfpathlineto{\pgfqpoint{1.105283in}{0.752130in}}%
\pgfpathlineto{\pgfqpoint{1.105847in}{0.751945in}}%
\pgfpathlineto{\pgfqpoint{1.106412in}{0.751760in}}%
\pgfpathlineto{\pgfqpoint{1.106976in}{0.751576in}}%
\pgfpathlineto{\pgfqpoint{1.107540in}{0.751391in}}%
\pgfpathlineto{\pgfqpoint{1.108105in}{0.751207in}}%
\pgfpathlineto{\pgfqpoint{1.108669in}{0.751022in}}%
\pgfpathlineto{\pgfqpoint{1.109233in}{0.750838in}}%
\pgfpathlineto{\pgfqpoint{1.109798in}{0.750653in}}%
\pgfpathlineto{\pgfqpoint{1.110362in}{0.750468in}}%
\pgfpathlineto{\pgfqpoint{1.110926in}{0.752618in}}%
\pgfpathlineto{\pgfqpoint{1.111491in}{0.758464in}}%
\pgfpathlineto{\pgfqpoint{1.112055in}{0.754957in}}%
\pgfpathlineto{\pgfqpoint{1.112619in}{0.757374in}}%
\pgfpathlineto{\pgfqpoint{1.113184in}{0.759090in}}%
\pgfpathlineto{\pgfqpoint{1.113748in}{0.760863in}}%
\pgfpathlineto{\pgfqpoint{1.114312in}{0.760356in}}%
\pgfpathlineto{\pgfqpoint{1.114877in}{0.753316in}}%
\pgfpathlineto{\pgfqpoint{1.115441in}{0.756656in}}%
\pgfpathlineto{\pgfqpoint{1.116005in}{0.754880in}}%
\pgfpathlineto{\pgfqpoint{1.116570in}{0.754821in}}%
\pgfpathlineto{\pgfqpoint{1.117134in}{0.754800in}}%
\pgfpathlineto{\pgfqpoint{1.117698in}{0.754779in}}%
\pgfpathlineto{\pgfqpoint{1.118263in}{0.754757in}}%
\pgfpathlineto{\pgfqpoint{1.118827in}{0.754736in}}%
\pgfpathlineto{\pgfqpoint{1.119391in}{0.754715in}}%
\pgfpathlineto{\pgfqpoint{1.119956in}{0.754694in}}%
\pgfpathlineto{\pgfqpoint{1.120520in}{0.754672in}}%
\pgfpathlineto{\pgfqpoint{1.121085in}{0.754651in}}%
\pgfpathlineto{\pgfqpoint{1.121649in}{0.754630in}}%
\pgfpathlineto{\pgfqpoint{1.122213in}{0.754608in}}%
\pgfpathlineto{\pgfqpoint{1.122778in}{0.754587in}}%
\pgfpathlineto{\pgfqpoint{1.123342in}{0.754566in}}%
\pgfpathlineto{\pgfqpoint{1.123906in}{0.754545in}}%
\pgfpathlineto{\pgfqpoint{1.124471in}{0.755922in}}%
\pgfpathlineto{\pgfqpoint{1.125035in}{0.761639in}}%
\pgfpathlineto{\pgfqpoint{1.125599in}{0.763390in}}%
\pgfpathlineto{\pgfqpoint{1.126164in}{0.757694in}}%
\pgfpathlineto{\pgfqpoint{1.126728in}{0.758244in}}%
\pgfpathlineto{\pgfqpoint{1.127292in}{0.755807in}}%
\pgfpathlineto{\pgfqpoint{1.127857in}{0.756715in}}%
\pgfpathlineto{\pgfqpoint{1.128421in}{0.765123in}}%
\pgfpathlineto{\pgfqpoint{1.128985in}{0.759777in}}%
\pgfpathlineto{\pgfqpoint{1.129550in}{0.776091in}}%
\pgfpathlineto{\pgfqpoint{1.130114in}{0.777526in}}%
\pgfpathlineto{\pgfqpoint{1.130678in}{0.776870in}}%
\pgfpathlineto{\pgfqpoint{1.131243in}{0.776214in}}%
\pgfpathlineto{\pgfqpoint{1.131807in}{0.775557in}}%
\pgfpathlineto{\pgfqpoint{1.132371in}{0.774901in}}%
\pgfpathlineto{\pgfqpoint{1.132936in}{0.774245in}}%
\pgfpathlineto{\pgfqpoint{1.133500in}{0.773589in}}%
\pgfpathlineto{\pgfqpoint{1.134064in}{0.772933in}}%
\pgfpathlineto{\pgfqpoint{1.134629in}{0.772277in}}%
\pgfpathlineto{\pgfqpoint{1.135193in}{0.771621in}}%
\pgfpathlineto{\pgfqpoint{1.135758in}{0.770964in}}%
\pgfpathlineto{\pgfqpoint{1.136322in}{0.770308in}}%
\pgfpathlineto{\pgfqpoint{1.136886in}{0.769652in}}%
\pgfpathlineto{\pgfqpoint{1.137451in}{0.768996in}}%
\pgfpathlineto{\pgfqpoint{1.138015in}{0.768216in}}%
\pgfpathlineto{\pgfqpoint{1.138579in}{0.766505in}}%
\pgfpathlineto{\pgfqpoint{1.139144in}{0.765787in}}%
\pgfpathlineto{\pgfqpoint{1.139708in}{0.765835in}}%
\pgfpathlineto{\pgfqpoint{1.140272in}{0.765846in}}%
\pgfpathlineto{\pgfqpoint{1.140837in}{0.773653in}}%
\pgfpathlineto{\pgfqpoint{1.141401in}{0.777947in}}%
\pgfpathlineto{\pgfqpoint{1.141965in}{0.764677in}}%
\pgfpathlineto{\pgfqpoint{1.142530in}{0.764816in}}%
\pgfpathlineto{\pgfqpoint{1.143094in}{0.764980in}}%
\pgfpathlineto{\pgfqpoint{1.143658in}{0.765143in}}%
\pgfpathlineto{\pgfqpoint{1.144223in}{0.765307in}}%
\pgfpathlineto{\pgfqpoint{1.144787in}{0.765470in}}%
\pgfpathlineto{\pgfqpoint{1.145351in}{0.765633in}}%
\pgfpathlineto{\pgfqpoint{1.145916in}{0.765797in}}%
\pgfpathlineto{\pgfqpoint{1.146480in}{0.765960in}}%
\pgfpathlineto{\pgfqpoint{1.147044in}{0.766124in}}%
\pgfpathlineto{\pgfqpoint{1.147609in}{0.766287in}}%
\pgfpathlineto{\pgfqpoint{1.148173in}{0.766450in}}%
\pgfpathlineto{\pgfqpoint{1.148737in}{0.766614in}}%
\pgfpathlineto{\pgfqpoint{1.149302in}{0.766777in}}%
\pgfpathlineto{\pgfqpoint{1.149866in}{0.766941in}}%
\pgfpathlineto{\pgfqpoint{1.150431in}{0.767104in}}%
\pgfpathlineto{\pgfqpoint{1.150995in}{0.767267in}}%
\pgfpathlineto{\pgfqpoint{1.151559in}{0.767431in}}%
\pgfpathlineto{\pgfqpoint{1.152124in}{0.767594in}}%
\pgfpathlineto{\pgfqpoint{1.152688in}{0.767757in}}%
\pgfpathlineto{\pgfqpoint{1.153252in}{0.767921in}}%
\pgfpathlineto{\pgfqpoint{1.153817in}{0.768084in}}%
\pgfpathlineto{\pgfqpoint{1.154381in}{0.768248in}}%
\pgfpathlineto{\pgfqpoint{1.154945in}{0.768411in}}%
\pgfpathlineto{\pgfqpoint{1.155510in}{0.768574in}}%
\pgfpathlineto{\pgfqpoint{1.156074in}{0.768738in}}%
\pgfpathlineto{\pgfqpoint{1.156638in}{0.768901in}}%
\pgfpathlineto{\pgfqpoint{1.157203in}{0.769065in}}%
\pgfpathlineto{\pgfqpoint{1.157767in}{0.769228in}}%
\pgfpathlineto{\pgfqpoint{1.158331in}{0.769391in}}%
\pgfpathlineto{\pgfqpoint{1.158896in}{0.769555in}}%
\pgfpathlineto{\pgfqpoint{1.159460in}{0.769718in}}%
\pgfpathlineto{\pgfqpoint{1.160024in}{0.769882in}}%
\pgfpathlineto{\pgfqpoint{1.160589in}{0.770045in}}%
\pgfpathlineto{\pgfqpoint{1.161153in}{0.770208in}}%
\pgfpathlineto{\pgfqpoint{1.161717in}{0.770372in}}%
\pgfpathlineto{\pgfqpoint{1.162282in}{0.770535in}}%
\pgfpathlineto{\pgfqpoint{1.162846in}{0.770699in}}%
\pgfpathlineto{\pgfqpoint{1.163410in}{0.770862in}}%
\pgfpathlineto{\pgfqpoint{1.163975in}{0.771025in}}%
\pgfpathlineto{\pgfqpoint{1.164539in}{0.771189in}}%
\pgfpathlineto{\pgfqpoint{1.165103in}{0.771352in}}%
\pgfpathlineto{\pgfqpoint{1.165668in}{0.771516in}}%
\pgfpathlineto{\pgfqpoint{1.166232in}{0.771679in}}%
\pgfpathlineto{\pgfqpoint{1.166797in}{0.771842in}}%
\pgfpathlineto{\pgfqpoint{1.167361in}{0.772006in}}%
\pgfpathlineto{\pgfqpoint{1.167925in}{0.772169in}}%
\pgfpathlineto{\pgfqpoint{1.168490in}{0.772333in}}%
\pgfpathlineto{\pgfqpoint{1.169054in}{0.772496in}}%
\pgfpathlineto{\pgfqpoint{1.169618in}{0.772659in}}%
\pgfpathlineto{\pgfqpoint{1.170183in}{0.772823in}}%
\pgfpathlineto{\pgfqpoint{1.170747in}{0.772986in}}%
\pgfpathlineto{\pgfqpoint{1.171311in}{0.773149in}}%
\pgfpathlineto{\pgfqpoint{1.171876in}{0.773313in}}%
\pgfpathlineto{\pgfqpoint{1.172440in}{0.773476in}}%
\pgfpathlineto{\pgfqpoint{1.173004in}{0.773640in}}%
\pgfpathlineto{\pgfqpoint{1.173569in}{0.773803in}}%
\pgfpathlineto{\pgfqpoint{1.174133in}{0.773966in}}%
\pgfpathlineto{\pgfqpoint{1.174697in}{0.774130in}}%
\pgfpathlineto{\pgfqpoint{1.175262in}{0.774293in}}%
\pgfpathlineto{\pgfqpoint{1.175826in}{0.774457in}}%
\pgfpathlineto{\pgfqpoint{1.176390in}{0.774620in}}%
\pgfpathlineto{\pgfqpoint{1.176955in}{0.774783in}}%
\pgfpathlineto{\pgfqpoint{1.177519in}{0.774947in}}%
\pgfpathlineto{\pgfqpoint{1.178083in}{0.775110in}}%
\pgfpathlineto{\pgfqpoint{1.178648in}{0.775274in}}%
\pgfpathlineto{\pgfqpoint{1.179212in}{0.775437in}}%
\pgfpathlineto{\pgfqpoint{1.179776in}{0.775033in}}%
\pgfpathlineto{\pgfqpoint{1.180341in}{0.771107in}}%
\pgfpathlineto{\pgfqpoint{1.180905in}{0.769595in}}%
\pgfpathlineto{\pgfqpoint{1.181470in}{0.780176in}}%
\pgfpathlineto{\pgfqpoint{1.182034in}{0.778451in}}%
\pgfpathlineto{\pgfqpoint{1.182598in}{0.780651in}}%
\pgfpathlineto{\pgfqpoint{1.183163in}{0.780934in}}%
\pgfpathlineto{\pgfqpoint{1.183727in}{0.779540in}}%
\pgfpathlineto{\pgfqpoint{1.184291in}{0.782880in}}%
\pgfpathlineto{\pgfqpoint{1.184856in}{0.782920in}}%
\pgfpathlineto{\pgfqpoint{1.185420in}{0.782959in}}%
\pgfpathlineto{\pgfqpoint{1.185984in}{0.782999in}}%
\pgfpathlineto{\pgfqpoint{1.186549in}{0.783039in}}%
\pgfpathlineto{\pgfqpoint{1.187113in}{0.783078in}}%
\pgfpathlineto{\pgfqpoint{1.187677in}{0.783118in}}%
\pgfpathlineto{\pgfqpoint{1.188242in}{0.783158in}}%
\pgfpathlineto{\pgfqpoint{1.188806in}{0.783197in}}%
\pgfpathlineto{\pgfqpoint{1.189370in}{0.783237in}}%
\pgfpathlineto{\pgfqpoint{1.189935in}{0.783277in}}%
\pgfpathlineto{\pgfqpoint{1.190499in}{0.783316in}}%
\pgfpathlineto{\pgfqpoint{1.191063in}{0.783356in}}%
\pgfpathlineto{\pgfqpoint{1.191628in}{0.783395in}}%
\pgfpathlineto{\pgfqpoint{1.192192in}{0.783435in}}%
\pgfpathlineto{\pgfqpoint{1.192756in}{0.783475in}}%
\pgfpathlineto{\pgfqpoint{1.193321in}{0.781154in}}%
\pgfpathlineto{\pgfqpoint{1.193885in}{0.782700in}}%
\pgfpathlineto{\pgfqpoint{1.194449in}{0.785767in}}%
\pgfpathlineto{\pgfqpoint{1.195014in}{0.786864in}}%
\pgfpathlineto{\pgfqpoint{1.195578in}{0.783009in}}%
\pgfpathlineto{\pgfqpoint{1.196143in}{0.781252in}}%
\pgfpathlineto{\pgfqpoint{1.196707in}{0.776261in}}%
\pgfpathlineto{\pgfqpoint{1.197271in}{0.785793in}}%
\pgfpathlineto{\pgfqpoint{1.197836in}{0.786448in}}%
\pgfpathlineto{\pgfqpoint{1.198400in}{0.786446in}}%
\pgfpathlineto{\pgfqpoint{1.198964in}{0.786445in}}%
\pgfpathlineto{\pgfqpoint{1.199529in}{0.786277in}}%
\pgfpathlineto{\pgfqpoint{1.200093in}{0.785662in}}%
\pgfpathlineto{\pgfqpoint{1.200657in}{0.785473in}}%
\pgfpathlineto{\pgfqpoint{1.201222in}{0.785285in}}%
\pgfpathlineto{\pgfqpoint{1.201786in}{0.785096in}}%
\pgfpathlineto{\pgfqpoint{1.202350in}{0.784908in}}%
\pgfpathlineto{\pgfqpoint{1.202915in}{0.784719in}}%
\pgfpathlineto{\pgfqpoint{1.203479in}{0.784530in}}%
\pgfpathlineto{\pgfqpoint{1.204043in}{0.784342in}}%
\pgfpathlineto{\pgfqpoint{1.204608in}{0.784153in}}%
\pgfpathlineto{\pgfqpoint{1.205172in}{0.783965in}}%
\pgfpathlineto{\pgfqpoint{1.205736in}{0.783776in}}%
\pgfpathlineto{\pgfqpoint{1.206301in}{0.774343in}}%
\pgfpathlineto{\pgfqpoint{1.206865in}{0.769747in}}%
\pgfpathlineto{\pgfqpoint{1.207429in}{0.781766in}}%
\pgfpathlineto{\pgfqpoint{1.207994in}{0.781182in}}%
\pgfpathlineto{\pgfqpoint{1.208558in}{0.782639in}}%
\pgfpathlineto{\pgfqpoint{1.209122in}{0.782730in}}%
\pgfpathlineto{\pgfqpoint{1.209687in}{0.788428in}}%
\pgfpathlineto{\pgfqpoint{1.210251in}{0.785977in}}%
\pgfpathlineto{\pgfqpoint{1.210815in}{0.793159in}}%
\pgfpathlineto{\pgfqpoint{1.211380in}{0.793184in}}%
\pgfpathlineto{\pgfqpoint{1.211944in}{0.792996in}}%
\pgfpathlineto{\pgfqpoint{1.212509in}{0.792808in}}%
\pgfpathlineto{\pgfqpoint{1.213073in}{0.792620in}}%
\pgfpathlineto{\pgfqpoint{1.213637in}{0.792432in}}%
\pgfpathlineto{\pgfqpoint{1.214202in}{0.792244in}}%
\pgfpathlineto{\pgfqpoint{1.214766in}{0.792056in}}%
\pgfpathlineto{\pgfqpoint{1.215330in}{0.791868in}}%
\pgfpathlineto{\pgfqpoint{1.215895in}{0.791680in}}%
\pgfpathlineto{\pgfqpoint{1.216459in}{0.791492in}}%
\pgfpathlineto{\pgfqpoint{1.217023in}{0.791304in}}%
\pgfpathlineto{\pgfqpoint{1.217588in}{0.791116in}}%
\pgfpathlineto{\pgfqpoint{1.218152in}{0.790928in}}%
\pgfpathlineto{\pgfqpoint{1.218716in}{0.790740in}}%
\pgfpathlineto{\pgfqpoint{1.219281in}{0.790538in}}%
\pgfpathlineto{\pgfqpoint{1.219845in}{0.790238in}}%
\pgfpathlineto{\pgfqpoint{1.220409in}{0.790003in}}%
\pgfpathlineto{\pgfqpoint{1.220974in}{0.790345in}}%
\pgfpathlineto{\pgfqpoint{1.221538in}{0.791160in}}%
\pgfpathlineto{\pgfqpoint{1.222102in}{0.793511in}}%
\pgfpathlineto{\pgfqpoint{1.222667in}{0.792574in}}%
\pgfpathlineto{\pgfqpoint{1.223231in}{0.800012in}}%
\pgfpathlineto{\pgfqpoint{1.223795in}{0.788469in}}%
\pgfpathlineto{\pgfqpoint{1.224360in}{0.787906in}}%
\pgfpathlineto{\pgfqpoint{1.224924in}{0.788820in}}%
\pgfpathlineto{\pgfqpoint{1.225488in}{0.790211in}}%
\pgfpathlineto{\pgfqpoint{1.226053in}{0.791602in}}%
\pgfpathlineto{\pgfqpoint{1.226617in}{0.792993in}}%
\pgfpathlineto{\pgfqpoint{1.227182in}{0.794384in}}%
\pgfpathlineto{\pgfqpoint{1.227746in}{0.795775in}}%
\pgfpathlineto{\pgfqpoint{1.228310in}{0.797166in}}%
\pgfpathlineto{\pgfqpoint{1.228875in}{0.798557in}}%
\pgfpathlineto{\pgfqpoint{1.229439in}{0.799947in}}%
\pgfpathlineto{\pgfqpoint{1.230003in}{0.801338in}}%
\pgfpathlineto{\pgfqpoint{1.230568in}{0.802729in}}%
\pgfpathlineto{\pgfqpoint{1.231132in}{0.804120in}}%
\pgfpathlineto{\pgfqpoint{1.231696in}{0.805511in}}%
\pgfpathlineto{\pgfqpoint{1.232261in}{0.807186in}}%
\pgfpathlineto{\pgfqpoint{1.232825in}{0.809329in}}%
\pgfpathlineto{\pgfqpoint{1.233389in}{0.809353in}}%
\pgfpathlineto{\pgfqpoint{1.233954in}{0.809059in}}%
\pgfpathlineto{\pgfqpoint{1.234518in}{0.810543in}}%
\pgfpathlineto{\pgfqpoint{1.235082in}{0.810513in}}%
\pgfpathlineto{\pgfqpoint{1.235647in}{0.811215in}}%
\pgfpathlineto{\pgfqpoint{1.236211in}{0.810606in}}%
\pgfpathlineto{\pgfqpoint{1.236775in}{0.809770in}}%
\pgfpathlineto{\pgfqpoint{1.237340in}{0.809044in}}%
\pgfpathlineto{\pgfqpoint{1.237904in}{0.808366in}}%
\pgfpathlineto{\pgfqpoint{1.238468in}{0.807688in}}%
\pgfpathlineto{\pgfqpoint{1.239033in}{0.807010in}}%
\pgfpathlineto{\pgfqpoint{1.239597in}{0.806332in}}%
\pgfpathlineto{\pgfqpoint{1.240161in}{0.805655in}}%
\pgfpathlineto{\pgfqpoint{1.240726in}{0.804977in}}%
\pgfpathlineto{\pgfqpoint{1.241290in}{0.804299in}}%
\pgfpathlineto{\pgfqpoint{1.241855in}{0.803621in}}%
\pgfpathlineto{\pgfqpoint{1.242419in}{0.802943in}}%
\pgfpathlineto{\pgfqpoint{1.242983in}{0.802265in}}%
\pgfpathlineto{\pgfqpoint{1.243548in}{0.801587in}}%
\pgfpathlineto{\pgfqpoint{1.244112in}{0.800909in}}%
\pgfpathlineto{\pgfqpoint{1.244676in}{0.800231in}}%
\pgfpathlineto{\pgfqpoint{1.245241in}{0.799553in}}%
\pgfpathlineto{\pgfqpoint{1.245805in}{0.798875in}}%
\pgfpathlineto{\pgfqpoint{1.246369in}{0.798198in}}%
\pgfpathlineto{\pgfqpoint{1.246934in}{0.797520in}}%
\pgfpathlineto{\pgfqpoint{1.247498in}{0.796842in}}%
\pgfpathlineto{\pgfqpoint{1.248062in}{0.796164in}}%
\pgfpathlineto{\pgfqpoint{1.248627in}{0.795486in}}%
\pgfpathlineto{\pgfqpoint{1.249191in}{0.794808in}}%
\pgfpathlineto{\pgfqpoint{1.249755in}{0.794130in}}%
\pgfpathlineto{\pgfqpoint{1.250320in}{0.793452in}}%
\pgfpathlineto{\pgfqpoint{1.250884in}{0.792774in}}%
\pgfpathlineto{\pgfqpoint{1.251448in}{0.792096in}}%
\pgfpathlineto{\pgfqpoint{1.252013in}{0.791418in}}%
\pgfpathlineto{\pgfqpoint{1.252577in}{0.790741in}}%
\pgfpathlineto{\pgfqpoint{1.253141in}{0.790063in}}%
\pgfpathlineto{\pgfqpoint{1.253706in}{0.789385in}}%
\pgfpathlineto{\pgfqpoint{1.254270in}{0.788707in}}%
\pgfpathlineto{\pgfqpoint{1.254834in}{0.788029in}}%
\pgfpathlineto{\pgfqpoint{1.255399in}{0.787351in}}%
\pgfpathlineto{\pgfqpoint{1.255963in}{0.786673in}}%
\pgfpathlineto{\pgfqpoint{1.256527in}{0.785995in}}%
\pgfpathlineto{\pgfqpoint{1.257092in}{0.785317in}}%
\pgfpathlineto{\pgfqpoint{1.257656in}{0.784639in}}%
\pgfpathlineto{\pgfqpoint{1.258221in}{0.783961in}}%
\pgfpathlineto{\pgfqpoint{1.258785in}{0.783284in}}%
\pgfpathlineto{\pgfqpoint{1.259349in}{0.782606in}}%
\pgfpathlineto{\pgfqpoint{1.259914in}{0.781928in}}%
\pgfpathlineto{\pgfqpoint{1.260478in}{0.781250in}}%
\pgfpathlineto{\pgfqpoint{1.261042in}{0.780572in}}%
\pgfpathlineto{\pgfqpoint{1.261607in}{0.779894in}}%
\pgfpathlineto{\pgfqpoint{1.262171in}{0.779216in}}%
\pgfpathlineto{\pgfqpoint{1.262735in}{0.778538in}}%
\pgfpathlineto{\pgfqpoint{1.263300in}{0.777860in}}%
\pgfpathlineto{\pgfqpoint{1.263864in}{0.777182in}}%
\pgfpathlineto{\pgfqpoint{1.264428in}{0.776504in}}%
\pgfpathlineto{\pgfqpoint{1.264993in}{0.775827in}}%
\pgfpathlineto{\pgfqpoint{1.265557in}{0.775149in}}%
\pgfpathlineto{\pgfqpoint{1.266121in}{0.774471in}}%
\pgfpathlineto{\pgfqpoint{1.266686in}{0.773793in}}%
\pgfpathlineto{\pgfqpoint{1.267250in}{0.773115in}}%
\pgfpathlineto{\pgfqpoint{1.267814in}{0.772437in}}%
\pgfpathlineto{\pgfqpoint{1.268379in}{0.771759in}}%
\pgfpathlineto{\pgfqpoint{1.268943in}{0.771081in}}%
\pgfpathlineto{\pgfqpoint{1.269507in}{0.770403in}}%
\pgfpathlineto{\pgfqpoint{1.270072in}{0.769725in}}%
\pgfpathlineto{\pgfqpoint{1.270636in}{0.769047in}}%
\pgfpathlineto{\pgfqpoint{1.271200in}{0.768370in}}%
\pgfpathlineto{\pgfqpoint{1.271765in}{0.767692in}}%
\pgfpathlineto{\pgfqpoint{1.272329in}{0.767014in}}%
\pgfpathlineto{\pgfqpoint{1.272894in}{0.766336in}}%
\pgfpathlineto{\pgfqpoint{1.273458in}{0.769629in}}%
\pgfpathlineto{\pgfqpoint{1.274022in}{0.791947in}}%
\pgfpathlineto{\pgfqpoint{1.274587in}{0.789645in}}%
\pgfpathlineto{\pgfqpoint{1.275151in}{0.804030in}}%
\pgfpathlineto{\pgfqpoint{1.275715in}{0.810700in}}%
\pgfpathlineto{\pgfqpoint{1.276280in}{0.808795in}}%
\pgfpathlineto{\pgfqpoint{1.276844in}{0.768711in}}%
\pgfpathlineto{\pgfqpoint{1.277408in}{0.770032in}}%
\pgfpathlineto{\pgfqpoint{1.277973in}{0.796169in}}%
\pgfpathlineto{\pgfqpoint{1.278537in}{0.794780in}}%
\pgfpathlineto{\pgfqpoint{1.279101in}{0.814666in}}%
\pgfpathlineto{\pgfqpoint{1.279666in}{0.810100in}}%
\pgfpathlineto{\pgfqpoint{1.280230in}{0.805535in}}%
\pgfpathlineto{\pgfqpoint{1.280794in}{0.801330in}}%
\pgfpathlineto{\pgfqpoint{1.281359in}{0.797732in}}%
\pgfpathlineto{\pgfqpoint{1.281923in}{0.794145in}}%
\pgfpathlineto{\pgfqpoint{1.282487in}{0.790558in}}%
\pgfpathlineto{\pgfqpoint{1.283052in}{0.786971in}}%
\pgfpathlineto{\pgfqpoint{1.283616in}{0.783383in}}%
\pgfpathlineto{\pgfqpoint{1.284180in}{0.779796in}}%
\pgfpathlineto{\pgfqpoint{1.284745in}{0.776209in}}%
\pgfpathlineto{\pgfqpoint{1.285309in}{0.772621in}}%
\pgfpathlineto{\pgfqpoint{1.285873in}{0.769034in}}%
\pgfpathlineto{\pgfqpoint{1.286438in}{0.765447in}}%
\pgfpathlineto{\pgfqpoint{1.287002in}{0.762181in}}%
\pgfpathlineto{\pgfqpoint{1.287567in}{0.764256in}}%
\pgfpathlineto{\pgfqpoint{1.288131in}{0.761191in}}%
\pgfpathlineto{\pgfqpoint{1.288695in}{0.760411in}}%
\pgfpathlineto{\pgfqpoint{1.289260in}{0.784479in}}%
\pgfpathlineto{\pgfqpoint{1.289824in}{0.792979in}}%
\pgfpathlineto{\pgfqpoint{1.290388in}{0.771368in}}%
\pgfpathlineto{\pgfqpoint{1.290953in}{0.769845in}}%
\pgfpathlineto{\pgfqpoint{1.291517in}{0.807393in}}%
\pgfpathlineto{\pgfqpoint{1.292081in}{0.812192in}}%
\pgfpathlineto{\pgfqpoint{1.292646in}{0.811589in}}%
\pgfpathlineto{\pgfqpoint{1.293210in}{0.810986in}}%
\pgfpathlineto{\pgfqpoint{1.293774in}{0.810382in}}%
\pgfpathlineto{\pgfqpoint{1.294339in}{0.809779in}}%
\pgfpathlineto{\pgfqpoint{1.294903in}{0.809176in}}%
\pgfpathlineto{\pgfqpoint{1.295467in}{0.808573in}}%
\pgfpathlineto{\pgfqpoint{1.296032in}{0.807970in}}%
\pgfpathlineto{\pgfqpoint{1.296596in}{0.807367in}}%
\pgfpathlineto{\pgfqpoint{1.297160in}{0.806764in}}%
\pgfpathlineto{\pgfqpoint{1.297725in}{0.806161in}}%
\pgfpathlineto{\pgfqpoint{1.298289in}{0.805557in}}%
\pgfpathlineto{\pgfqpoint{1.298853in}{0.804954in}}%
\pgfpathlineto{\pgfqpoint{1.299418in}{0.804351in}}%
\pgfpathlineto{\pgfqpoint{1.299982in}{0.803748in}}%
\pgfpathlineto{\pgfqpoint{1.300546in}{0.805125in}}%
\pgfpathlineto{\pgfqpoint{1.301111in}{0.812055in}}%
\pgfpathlineto{\pgfqpoint{1.301675in}{0.799258in}}%
\pgfpathlineto{\pgfqpoint{1.302239in}{0.812470in}}%
\pgfpathlineto{\pgfqpoint{1.302804in}{0.811991in}}%
\pgfpathlineto{\pgfqpoint{1.303368in}{0.812133in}}%
\pgfpathlineto{\pgfqpoint{1.303933in}{0.812401in}}%
\pgfpathlineto{\pgfqpoint{1.304497in}{0.813155in}}%
\pgfpathlineto{\pgfqpoint{1.305061in}{0.812465in}}%
\pgfpathlineto{\pgfqpoint{1.305626in}{0.761506in}}%
\pgfpathlineto{\pgfqpoint{1.306190in}{0.764482in}}%
\pgfpathlineto{\pgfqpoint{1.306754in}{0.768123in}}%
\pgfpathlineto{\pgfqpoint{1.307319in}{0.771765in}}%
\pgfpathlineto{\pgfqpoint{1.307883in}{0.775406in}}%
\pgfpathlineto{\pgfqpoint{1.308447in}{0.779048in}}%
\pgfpathlineto{\pgfqpoint{1.309012in}{0.782690in}}%
\pgfpathlineto{\pgfqpoint{1.309576in}{0.786331in}}%
\pgfpathlineto{\pgfqpoint{1.310140in}{0.789973in}}%
\pgfpathlineto{\pgfqpoint{1.310705in}{0.793614in}}%
\pgfpathlineto{\pgfqpoint{1.311269in}{0.797256in}}%
\pgfpathlineto{\pgfqpoint{1.311833in}{0.800897in}}%
\pgfpathlineto{\pgfqpoint{1.312398in}{0.804539in}}%
\pgfpathlineto{\pgfqpoint{1.312962in}{0.808180in}}%
\pgfpathlineto{\pgfqpoint{1.313526in}{0.811805in}}%
\pgfpathlineto{\pgfqpoint{1.314091in}{0.813263in}}%
\pgfpathlineto{\pgfqpoint{1.314655in}{0.813130in}}%
\pgfpathlineto{\pgfqpoint{1.315219in}{0.814128in}}%
\pgfpathlineto{\pgfqpoint{1.315784in}{0.813925in}}%
\pgfpathlineto{\pgfqpoint{1.316348in}{0.814699in}}%
\pgfpathlineto{\pgfqpoint{1.316912in}{0.814141in}}%
\pgfpathlineto{\pgfqpoint{1.317477in}{0.814089in}}%
\pgfpathlineto{\pgfqpoint{1.318041in}{0.814037in}}%
\pgfpathlineto{\pgfqpoint{1.318606in}{0.817334in}}%
\pgfpathlineto{\pgfqpoint{1.319170in}{0.818318in}}%
\pgfpathlineto{\pgfqpoint{1.319734in}{0.817642in}}%
\pgfpathlineto{\pgfqpoint{1.320299in}{0.816959in}}%
\pgfpathlineto{\pgfqpoint{1.320863in}{0.816276in}}%
\pgfpathlineto{\pgfqpoint{1.321427in}{0.815593in}}%
\pgfpathlineto{\pgfqpoint{1.321992in}{0.814910in}}%
\pgfpathlineto{\pgfqpoint{1.322556in}{0.814227in}}%
\pgfpathlineto{\pgfqpoint{1.323120in}{0.813543in}}%
\pgfpathlineto{\pgfqpoint{1.323685in}{0.812860in}}%
\pgfpathlineto{\pgfqpoint{1.324249in}{0.812177in}}%
\pgfpathlineto{\pgfqpoint{1.324813in}{0.811494in}}%
\pgfpathlineto{\pgfqpoint{1.325378in}{0.810811in}}%
\pgfpathlineto{\pgfqpoint{1.325942in}{0.810128in}}%
\pgfpathlineto{\pgfqpoint{1.326506in}{0.809445in}}%
\pgfpathlineto{\pgfqpoint{1.327071in}{0.808762in}}%
\pgfpathlineto{\pgfqpoint{1.327635in}{0.808079in}}%
\pgfpathlineto{\pgfqpoint{1.328199in}{0.807396in}}%
\pgfpathlineto{\pgfqpoint{1.328764in}{0.806713in}}%
\pgfpathlineto{\pgfqpoint{1.329328in}{0.806030in}}%
\pgfpathlineto{\pgfqpoint{1.329892in}{0.805346in}}%
\pgfpathlineto{\pgfqpoint{1.330457in}{0.804663in}}%
\pgfpathlineto{\pgfqpoint{1.331021in}{0.803980in}}%
\pgfpathlineto{\pgfqpoint{1.331585in}{0.803297in}}%
\pgfpathlineto{\pgfqpoint{1.332150in}{0.802614in}}%
\pgfpathlineto{\pgfqpoint{1.332714in}{0.801931in}}%
\pgfpathlineto{\pgfqpoint{1.333279in}{0.801248in}}%
\pgfpathlineto{\pgfqpoint{1.333843in}{0.800565in}}%
\pgfpathlineto{\pgfqpoint{1.334407in}{0.799882in}}%
\pgfpathlineto{\pgfqpoint{1.334972in}{0.799199in}}%
\pgfpathlineto{\pgfqpoint{1.335536in}{0.798516in}}%
\pgfpathlineto{\pgfqpoint{1.336100in}{0.797833in}}%
\pgfpathlineto{\pgfqpoint{1.336665in}{0.797149in}}%
\pgfpathlineto{\pgfqpoint{1.337229in}{0.796466in}}%
\pgfpathlineto{\pgfqpoint{1.337793in}{0.795783in}}%
\pgfpathlineto{\pgfqpoint{1.338358in}{0.795100in}}%
\pgfpathlineto{\pgfqpoint{1.338922in}{0.794417in}}%
\pgfpathlineto{\pgfqpoint{1.339486in}{0.793734in}}%
\pgfpathlineto{\pgfqpoint{1.340051in}{0.793051in}}%
\pgfpathlineto{\pgfqpoint{1.340615in}{0.792368in}}%
\pgfpathlineto{\pgfqpoint{1.341179in}{0.791685in}}%
\pgfpathlineto{\pgfqpoint{1.341744in}{0.791002in}}%
\pgfpathlineto{\pgfqpoint{1.342308in}{0.790319in}}%
\pgfpathlineto{\pgfqpoint{1.342872in}{0.789636in}}%
\pgfpathlineto{\pgfqpoint{1.343437in}{0.788952in}}%
\pgfpathlineto{\pgfqpoint{1.344001in}{0.788269in}}%
\pgfpathlineto{\pgfqpoint{1.344565in}{0.787586in}}%
\pgfpathlineto{\pgfqpoint{1.345130in}{0.786903in}}%
\pgfpathlineto{\pgfqpoint{1.345694in}{0.786220in}}%
\pgfpathlineto{\pgfqpoint{1.346258in}{0.785537in}}%
\pgfpathlineto{\pgfqpoint{1.346823in}{0.784854in}}%
\pgfpathlineto{\pgfqpoint{1.347387in}{0.784171in}}%
\pgfpathlineto{\pgfqpoint{1.347951in}{0.783488in}}%
\pgfpathlineto{\pgfqpoint{1.348516in}{0.782805in}}%
\pgfpathlineto{\pgfqpoint{1.349080in}{0.782122in}}%
\pgfpathlineto{\pgfqpoint{1.349645in}{0.781439in}}%
\pgfpathlineto{\pgfqpoint{1.350209in}{0.780755in}}%
\pgfpathlineto{\pgfqpoint{1.350773in}{0.780072in}}%
\pgfpathlineto{\pgfqpoint{1.351338in}{0.779389in}}%
\pgfpathlineto{\pgfqpoint{1.351902in}{0.778706in}}%
\pgfpathlineto{\pgfqpoint{1.352466in}{0.778023in}}%
\pgfpathlineto{\pgfqpoint{1.353031in}{0.777340in}}%
\pgfpathlineto{\pgfqpoint{1.353595in}{0.776657in}}%
\pgfpathlineto{\pgfqpoint{1.354159in}{0.775974in}}%
\pgfpathlineto{\pgfqpoint{1.354724in}{0.775291in}}%
\pgfpathlineto{\pgfqpoint{1.355288in}{0.774608in}}%
\pgfpathlineto{\pgfqpoint{1.355852in}{0.773925in}}%
\pgfpathlineto{\pgfqpoint{1.356417in}{0.773242in}}%
\pgfpathlineto{\pgfqpoint{1.356981in}{0.772558in}}%
\pgfpathlineto{\pgfqpoint{1.357545in}{0.771875in}}%
\pgfpathlineto{\pgfqpoint{1.358110in}{0.771192in}}%
\pgfpathlineto{\pgfqpoint{1.358674in}{0.770509in}}%
\pgfpathlineto{\pgfqpoint{1.359238in}{0.769826in}}%
\pgfpathlineto{\pgfqpoint{1.359803in}{0.769143in}}%
\pgfpathlineto{\pgfqpoint{1.360367in}{0.768460in}}%
\pgfpathlineto{\pgfqpoint{1.360931in}{0.767777in}}%
\pgfpathlineto{\pgfqpoint{1.361496in}{0.767094in}}%
\pgfpathlineto{\pgfqpoint{1.362060in}{0.766411in}}%
\pgfpathlineto{\pgfqpoint{1.362624in}{0.765728in}}%
\pgfpathlineto{\pgfqpoint{1.363189in}{0.765045in}}%
\pgfpathlineto{\pgfqpoint{1.363753in}{0.764361in}}%
\pgfpathlineto{\pgfqpoint{1.364318in}{0.763678in}}%
\pgfpathlineto{\pgfqpoint{1.364882in}{0.762995in}}%
\pgfpathlineto{\pgfqpoint{1.365446in}{0.762312in}}%
\pgfpathlineto{\pgfqpoint{1.366011in}{0.761629in}}%
\pgfpathlineto{\pgfqpoint{1.366575in}{0.760946in}}%
\pgfpathlineto{\pgfqpoint{1.367139in}{0.760263in}}%
\pgfpathlineto{\pgfqpoint{1.367704in}{0.759580in}}%
\pgfpathlineto{\pgfqpoint{1.368268in}{0.758897in}}%
\pgfpathlineto{\pgfqpoint{1.368832in}{0.764623in}}%
\pgfpathlineto{\pgfqpoint{1.369397in}{0.818730in}}%
\pgfpathlineto{\pgfqpoint{1.369961in}{0.818893in}}%
\pgfpathlineto{\pgfqpoint{1.370525in}{0.818716in}}%
\pgfpathlineto{\pgfqpoint{1.371090in}{0.821081in}}%
\pgfpathlineto{\pgfqpoint{1.371654in}{0.821625in}}%
\pgfpathlineto{\pgfqpoint{1.372218in}{0.823560in}}%
\pgfpathlineto{\pgfqpoint{1.372783in}{0.824838in}}%
\pgfpathlineto{\pgfqpoint{1.373347in}{0.825872in}}%
\pgfpathlineto{\pgfqpoint{1.373911in}{0.823558in}}%
\pgfpathlineto{\pgfqpoint{1.374476in}{0.819324in}}%
\pgfpathlineto{\pgfqpoint{1.375040in}{0.815091in}}%
\pgfpathlineto{\pgfqpoint{1.375604in}{0.810857in}}%
\pgfpathlineto{\pgfqpoint{1.376169in}{0.806624in}}%
\pgfpathlineto{\pgfqpoint{1.376733in}{0.802390in}}%
\pgfpathlineto{\pgfqpoint{1.377297in}{0.798157in}}%
\pgfpathlineto{\pgfqpoint{1.377862in}{0.793923in}}%
\pgfpathlineto{\pgfqpoint{1.378426in}{0.789690in}}%
\pgfpathlineto{\pgfqpoint{1.378991in}{0.785456in}}%
\pgfpathlineto{\pgfqpoint{1.379555in}{0.781223in}}%
\pgfpathlineto{\pgfqpoint{1.380119in}{0.776990in}}%
\pgfpathlineto{\pgfqpoint{1.380684in}{0.772756in}}%
\pgfpathlineto{\pgfqpoint{1.381248in}{0.768523in}}%
\pgfpathlineto{\pgfqpoint{1.381812in}{0.764289in}}%
\pgfpathlineto{\pgfqpoint{1.382377in}{0.760056in}}%
\pgfpathlineto{\pgfqpoint{1.382941in}{0.758914in}}%
\pgfpathlineto{\pgfqpoint{1.383505in}{0.756617in}}%
\pgfpathlineto{\pgfqpoint{1.384070in}{0.751707in}}%
\pgfpathlineto{\pgfqpoint{1.384634in}{0.749862in}}%
\pgfpathlineto{\pgfqpoint{1.385198in}{0.752331in}}%
\pgfpathlineto{\pgfqpoint{1.385763in}{0.753198in}}%
\pgfpathlineto{\pgfqpoint{1.386327in}{0.754385in}}%
\pgfpathlineto{\pgfqpoint{1.386891in}{0.755045in}}%
\pgfpathlineto{\pgfqpoint{1.387456in}{0.755055in}}%
\pgfpathlineto{\pgfqpoint{1.388020in}{0.755065in}}%
\pgfpathlineto{\pgfqpoint{1.388584in}{0.755076in}}%
\pgfpathlineto{\pgfqpoint{1.389149in}{0.755086in}}%
\pgfpathlineto{\pgfqpoint{1.389713in}{0.755096in}}%
\pgfpathlineto{\pgfqpoint{1.390277in}{0.755106in}}%
\pgfpathlineto{\pgfqpoint{1.390842in}{0.755116in}}%
\pgfpathlineto{\pgfqpoint{1.391406in}{0.755127in}}%
\pgfpathlineto{\pgfqpoint{1.391970in}{0.755137in}}%
\pgfpathlineto{\pgfqpoint{1.392535in}{0.755147in}}%
\pgfpathlineto{\pgfqpoint{1.393099in}{0.755157in}}%
\pgfpathlineto{\pgfqpoint{1.393664in}{0.755167in}}%
\pgfpathlineto{\pgfqpoint{1.394228in}{0.755178in}}%
\pgfpathlineto{\pgfqpoint{1.394792in}{0.755218in}}%
\pgfpathlineto{\pgfqpoint{1.395357in}{0.756293in}}%
\pgfpathlineto{\pgfqpoint{1.395921in}{0.756978in}}%
\pgfpathlineto{\pgfqpoint{1.396485in}{0.754859in}}%
\pgfpathlineto{\pgfqpoint{1.397050in}{0.757327in}}%
\pgfpathlineto{\pgfqpoint{1.397614in}{0.758252in}}%
\pgfpathlineto{\pgfqpoint{1.398178in}{0.758347in}}%
\pgfpathlineto{\pgfqpoint{1.398743in}{0.758466in}}%
\pgfpathlineto{\pgfqpoint{1.399307in}{0.758127in}}%
\pgfpathlineto{\pgfqpoint{1.399871in}{0.759294in}}%
\pgfpathlineto{\pgfqpoint{1.400436in}{0.759219in}}%
\pgfpathlineto{\pgfqpoint{1.401000in}{0.759143in}}%
\pgfpathlineto{\pgfqpoint{1.401564in}{0.759068in}}%
\pgfpathlineto{\pgfqpoint{1.402129in}{0.758993in}}%
\pgfpathlineto{\pgfqpoint{1.402693in}{0.758918in}}%
\pgfpathlineto{\pgfqpoint{1.403257in}{0.758843in}}%
\pgfpathlineto{\pgfqpoint{1.403822in}{0.758768in}}%
\pgfpathlineto{\pgfqpoint{1.404386in}{0.758692in}}%
\pgfpathlineto{\pgfqpoint{1.404950in}{0.758617in}}%
\pgfpathlineto{\pgfqpoint{1.405515in}{0.758542in}}%
\pgfpathlineto{\pgfqpoint{1.406079in}{0.758467in}}%
\pgfpathlineto{\pgfqpoint{1.406643in}{0.758392in}}%
\pgfpathlineto{\pgfqpoint{1.407208in}{0.758317in}}%
\pgfpathlineto{\pgfqpoint{1.407772in}{0.758242in}}%
\pgfpathlineto{\pgfqpoint{1.408336in}{0.758166in}}%
\pgfpathlineto{\pgfqpoint{1.408901in}{0.758091in}}%
\pgfpathlineto{\pgfqpoint{1.409465in}{0.758016in}}%
\pgfpathlineto{\pgfqpoint{1.410030in}{0.757930in}}%
\pgfpathlineto{\pgfqpoint{1.410594in}{0.757845in}}%
\pgfpathlineto{\pgfqpoint{1.411158in}{0.759710in}}%
\pgfpathlineto{\pgfqpoint{1.411723in}{0.770897in}}%
\pgfpathlineto{\pgfqpoint{1.412287in}{0.770708in}}%
\pgfpathlineto{\pgfqpoint{1.412851in}{0.768710in}}%
\pgfpathlineto{\pgfqpoint{1.413416in}{0.764033in}}%
\pgfpathlineto{\pgfqpoint{1.413980in}{0.770738in}}%
\pgfpathlineto{\pgfqpoint{1.414544in}{0.771031in}}%
\pgfpathlineto{\pgfqpoint{1.415109in}{0.770923in}}%
\pgfpathlineto{\pgfqpoint{1.415673in}{0.770814in}}%
\pgfpathlineto{\pgfqpoint{1.416237in}{0.770706in}}%
\pgfpathlineto{\pgfqpoint{1.416802in}{0.770598in}}%
\pgfpathlineto{\pgfqpoint{1.417366in}{0.770489in}}%
\pgfpathlineto{\pgfqpoint{1.417930in}{0.770381in}}%
\pgfpathlineto{\pgfqpoint{1.418495in}{0.770272in}}%
\pgfpathlineto{\pgfqpoint{1.419059in}{0.770164in}}%
\pgfpathlineto{\pgfqpoint{1.419623in}{0.770055in}}%
\pgfpathlineto{\pgfqpoint{1.420188in}{0.769947in}}%
\pgfpathlineto{\pgfqpoint{1.420752in}{0.769838in}}%
\pgfpathlineto{\pgfqpoint{1.421316in}{0.769730in}}%
\pgfpathlineto{\pgfqpoint{1.421881in}{0.769621in}}%
\pgfpathlineto{\pgfqpoint{1.422445in}{0.768670in}}%
\pgfpathlineto{\pgfqpoint{1.423009in}{0.767857in}}%
\pgfpathlineto{\pgfqpoint{1.423574in}{0.767126in}}%
\pgfpathlineto{\pgfqpoint{1.424138in}{0.760110in}}%
\pgfpathlineto{\pgfqpoint{1.424703in}{0.763570in}}%
\pgfpathlineto{\pgfqpoint{1.425267in}{0.769283in}}%
\pgfpathlineto{\pgfqpoint{1.425831in}{0.768807in}}%
\pgfpathlineto{\pgfqpoint{1.426396in}{0.757262in}}%
\pgfpathlineto{\pgfqpoint{1.426960in}{0.755425in}}%
\pgfpathlineto{\pgfqpoint{1.427524in}{0.753588in}}%
\pgfpathlineto{\pgfqpoint{1.428089in}{0.751752in}}%
\pgfpathlineto{\pgfqpoint{1.428653in}{0.749915in}}%
\pgfpathlineto{\pgfqpoint{1.429217in}{0.748112in}}%
\pgfpathlineto{\pgfqpoint{1.429782in}{0.747525in}}%
\pgfpathlineto{\pgfqpoint{1.430346in}{0.747545in}}%
\pgfpathlineto{\pgfqpoint{1.430910in}{0.747564in}}%
\pgfpathlineto{\pgfqpoint{1.431475in}{0.747584in}}%
\pgfpathlineto{\pgfqpoint{1.432039in}{0.747603in}}%
\pgfpathlineto{\pgfqpoint{1.432603in}{0.747623in}}%
\pgfpathlineto{\pgfqpoint{1.433168in}{0.747643in}}%
\pgfpathlineto{\pgfqpoint{1.433732in}{0.747662in}}%
\pgfpathlineto{\pgfqpoint{1.434296in}{0.747682in}}%
\pgfpathlineto{\pgfqpoint{1.434861in}{0.747701in}}%
\pgfpathlineto{\pgfqpoint{1.435425in}{0.747721in}}%
\pgfpathlineto{\pgfqpoint{1.435989in}{0.747740in}}%
\pgfpathlineto{\pgfqpoint{1.436554in}{0.747760in}}%
\pgfpathlineto{\pgfqpoint{1.437118in}{0.747779in}}%
\pgfpathlineto{\pgfqpoint{1.437682in}{0.747799in}}%
\pgfpathlineto{\pgfqpoint{1.438247in}{0.747819in}}%
\pgfpathlineto{\pgfqpoint{1.438811in}{0.747838in}}%
\pgfpathlineto{\pgfqpoint{1.439376in}{0.747858in}}%
\pgfpathlineto{\pgfqpoint{1.439940in}{0.747877in}}%
\pgfpathlineto{\pgfqpoint{1.440504in}{0.747897in}}%
\pgfpathlineto{\pgfqpoint{1.441069in}{0.747916in}}%
\pgfpathlineto{\pgfqpoint{1.441633in}{0.747936in}}%
\pgfpathlineto{\pgfqpoint{1.442197in}{0.747955in}}%
\pgfpathlineto{\pgfqpoint{1.442762in}{0.747975in}}%
\pgfpathlineto{\pgfqpoint{1.443326in}{0.747994in}}%
\pgfpathlineto{\pgfqpoint{1.443890in}{0.748014in}}%
\pgfpathlineto{\pgfqpoint{1.444455in}{0.748034in}}%
\pgfpathlineto{\pgfqpoint{1.445019in}{0.748053in}}%
\pgfpathlineto{\pgfqpoint{1.445583in}{0.748073in}}%
\pgfpathlineto{\pgfqpoint{1.446148in}{0.748092in}}%
\pgfpathlineto{\pgfqpoint{1.446712in}{0.748112in}}%
\pgfpathlineto{\pgfqpoint{1.447276in}{0.748131in}}%
\pgfpathlineto{\pgfqpoint{1.447841in}{0.748151in}}%
\pgfpathlineto{\pgfqpoint{1.448405in}{0.748170in}}%
\pgfpathlineto{\pgfqpoint{1.448969in}{0.748190in}}%
\pgfpathlineto{\pgfqpoint{1.449534in}{0.748210in}}%
\pgfpathlineto{\pgfqpoint{1.450098in}{0.748229in}}%
\pgfpathlineto{\pgfqpoint{1.450662in}{0.748249in}}%
\pgfpathlineto{\pgfqpoint{1.451227in}{0.748268in}}%
\pgfpathlineto{\pgfqpoint{1.451791in}{0.748288in}}%
\pgfpathlineto{\pgfqpoint{1.452355in}{0.748307in}}%
\pgfpathlineto{\pgfqpoint{1.452920in}{0.748327in}}%
\pgfpathlineto{\pgfqpoint{1.453484in}{0.748346in}}%
\pgfpathlineto{\pgfqpoint{1.454048in}{0.748366in}}%
\pgfpathlineto{\pgfqpoint{1.454613in}{0.748385in}}%
\pgfpathlineto{\pgfqpoint{1.455177in}{0.748405in}}%
\pgfpathlineto{\pgfqpoint{1.455742in}{0.748425in}}%
\pgfpathlineto{\pgfqpoint{1.456306in}{0.748444in}}%
\pgfpathlineto{\pgfqpoint{1.456870in}{0.748464in}}%
\pgfpathlineto{\pgfqpoint{1.457435in}{0.748483in}}%
\pgfpathlineto{\pgfqpoint{1.457999in}{0.748503in}}%
\pgfpathlineto{\pgfqpoint{1.458563in}{0.748522in}}%
\pgfpathlineto{\pgfqpoint{1.459128in}{0.748542in}}%
\pgfpathlineto{\pgfqpoint{1.459692in}{0.748561in}}%
\pgfpathlineto{\pgfqpoint{1.460256in}{0.748581in}}%
\pgfpathlineto{\pgfqpoint{1.460821in}{0.748601in}}%
\pgfpathlineto{\pgfqpoint{1.461385in}{0.748620in}}%
\pgfpathlineto{\pgfqpoint{1.461949in}{0.748640in}}%
\pgfpathlineto{\pgfqpoint{1.462514in}{0.748659in}}%
\pgfpathlineto{\pgfqpoint{1.463078in}{0.748679in}}%
\pgfpathlineto{\pgfqpoint{1.463642in}{0.748698in}}%
\pgfpathlineto{\pgfqpoint{1.464207in}{0.748718in}}%
\pgfpathlineto{\pgfqpoint{1.464771in}{0.748737in}}%
\pgfpathlineto{\pgfqpoint{1.465335in}{0.748757in}}%
\pgfpathlineto{\pgfqpoint{1.465900in}{0.747156in}}%
\pgfpathlineto{\pgfqpoint{1.466464in}{0.738688in}}%
\pgfpathlineto{\pgfqpoint{1.467028in}{0.746497in}}%
\pgfpathlineto{\pgfqpoint{1.467593in}{0.740562in}}%
\pgfpathlineto{\pgfqpoint{1.468157in}{0.747515in}}%
\pgfpathlineto{\pgfqpoint{1.468721in}{0.746684in}}%
\pgfpathlineto{\pgfqpoint{1.469286in}{0.745719in}}%
\pgfpathlineto{\pgfqpoint{1.469850in}{0.744755in}}%
\pgfpathlineto{\pgfqpoint{1.470415in}{0.743790in}}%
\pgfpathlineto{\pgfqpoint{1.470979in}{0.742825in}}%
\pgfpathlineto{\pgfqpoint{1.471543in}{0.741860in}}%
\pgfpathlineto{\pgfqpoint{1.472108in}{0.740896in}}%
\pgfpathlineto{\pgfqpoint{1.472672in}{0.739931in}}%
\pgfpathlineto{\pgfqpoint{1.473236in}{0.738966in}}%
\pgfpathlineto{\pgfqpoint{1.473801in}{0.738001in}}%
\pgfpathlineto{\pgfqpoint{1.474365in}{0.737037in}}%
\pgfpathlineto{\pgfqpoint{1.474929in}{0.736072in}}%
\pgfpathlineto{\pgfqpoint{1.475494in}{0.735107in}}%
\pgfpathlineto{\pgfqpoint{1.476058in}{0.734142in}}%
\pgfpathlineto{\pgfqpoint{1.476622in}{0.733178in}}%
\pgfpathlineto{\pgfqpoint{1.477187in}{0.737478in}}%
\pgfpathlineto{\pgfqpoint{1.477751in}{0.734358in}}%
\pgfpathlineto{\pgfqpoint{1.478315in}{0.740802in}}%
\pgfpathlineto{\pgfqpoint{1.478880in}{0.747169in}}%
\pgfpathlineto{\pgfqpoint{1.479444in}{0.746516in}}%
\pgfpathlineto{\pgfqpoint{1.480008in}{0.735244in}}%
\pgfpathlineto{\pgfqpoint{1.480573in}{0.738207in}}%
\pgfpathlineto{\pgfqpoint{1.481137in}{0.736773in}}%
\pgfpathlineto{\pgfqpoint{1.481701in}{0.737843in}}%
\pgfpathlineto{\pgfqpoint{1.482266in}{0.737693in}}%
\pgfpathlineto{\pgfqpoint{1.482830in}{0.737503in}}%
\pgfpathlineto{\pgfqpoint{1.483394in}{0.737313in}}%
\pgfpathlineto{\pgfqpoint{1.483959in}{0.737122in}}%
\pgfpathlineto{\pgfqpoint{1.484523in}{0.736932in}}%
\pgfpathlineto{\pgfqpoint{1.485088in}{0.736742in}}%
\pgfpathlineto{\pgfqpoint{1.485652in}{0.736551in}}%
\pgfpathlineto{\pgfqpoint{1.486216in}{0.736361in}}%
\pgfpathlineto{\pgfqpoint{1.486781in}{0.736171in}}%
\pgfpathlineto{\pgfqpoint{1.487345in}{0.735980in}}%
\pgfpathlineto{\pgfqpoint{1.487909in}{0.735790in}}%
\pgfpathlineto{\pgfqpoint{1.488474in}{0.735600in}}%
\pgfpathlineto{\pgfqpoint{1.489038in}{0.735410in}}%
\pgfpathlineto{\pgfqpoint{1.489602in}{0.735219in}}%
\pgfpathlineto{\pgfqpoint{1.490167in}{0.735062in}}%
\pgfpathlineto{\pgfqpoint{1.490731in}{0.735516in}}%
\pgfpathlineto{\pgfqpoint{1.491295in}{0.735294in}}%
\pgfpathlineto{\pgfqpoint{1.491860in}{0.738561in}}%
\pgfpathlineto{\pgfqpoint{1.492424in}{0.742683in}}%
\pgfpathlineto{\pgfqpoint{1.492988in}{0.735496in}}%
\pgfpathlineto{\pgfqpoint{1.493553in}{0.739370in}}%
\pgfpathlineto{\pgfqpoint{1.494117in}{0.736089in}}%
\pgfpathlineto{\pgfqpoint{1.494681in}{0.740224in}}%
\pgfpathlineto{\pgfqpoint{1.495246in}{0.740580in}}%
\pgfpathlineto{\pgfqpoint{1.495810in}{0.742221in}}%
\pgfpathlineto{\pgfqpoint{1.496374in}{0.742026in}}%
\pgfpathlineto{\pgfqpoint{1.496939in}{0.741831in}}%
\pgfpathlineto{\pgfqpoint{1.497503in}{0.741636in}}%
\pgfpathlineto{\pgfqpoint{1.498067in}{0.741441in}}%
\pgfpathlineto{\pgfqpoint{1.498632in}{0.741246in}}%
\pgfpathlineto{\pgfqpoint{1.499196in}{0.741051in}}%
\pgfpathlineto{\pgfqpoint{1.499760in}{0.740856in}}%
\pgfpathlineto{\pgfqpoint{1.500325in}{0.740661in}}%
\pgfpathlineto{\pgfqpoint{1.500889in}{0.740466in}}%
\pgfpathlineto{\pgfqpoint{1.501454in}{0.740271in}}%
\pgfpathlineto{\pgfqpoint{1.502018in}{0.740076in}}%
\pgfpathlineto{\pgfqpoint{1.502582in}{0.739880in}}%
\pgfpathlineto{\pgfqpoint{1.503147in}{0.739685in}}%
\pgfpathlineto{\pgfqpoint{1.503711in}{0.739505in}}%
\pgfpathlineto{\pgfqpoint{1.504275in}{0.742340in}}%
\pgfpathlineto{\pgfqpoint{1.504840in}{0.743268in}}%
\pgfpathlineto{\pgfqpoint{1.505404in}{0.742095in}}%
\pgfpathlineto{\pgfqpoint{1.505968in}{0.741446in}}%
\pgfpathlineto{\pgfqpoint{1.506533in}{0.740797in}}%
\pgfpathlineto{\pgfqpoint{1.507097in}{0.743062in}}%
\pgfpathlineto{\pgfqpoint{1.507661in}{0.743175in}}%
\pgfpathlineto{\pgfqpoint{1.508226in}{0.743060in}}%
\pgfpathlineto{\pgfqpoint{1.508790in}{0.743018in}}%
\pgfpathlineto{\pgfqpoint{1.509354in}{0.742975in}}%
\pgfpathlineto{\pgfqpoint{1.509919in}{0.742933in}}%
\pgfpathlineto{\pgfqpoint{1.510483in}{0.742890in}}%
\pgfpathlineto{\pgfqpoint{1.511047in}{0.742848in}}%
\pgfpathlineto{\pgfqpoint{1.511612in}{0.742805in}}%
\pgfpathlineto{\pgfqpoint{1.512176in}{0.742763in}}%
\pgfpathlineto{\pgfqpoint{1.512740in}{0.742721in}}%
\pgfpathlineto{\pgfqpoint{1.513305in}{0.742678in}}%
\pgfpathlineto{\pgfqpoint{1.513869in}{0.742636in}}%
\pgfpathlineto{\pgfqpoint{1.514433in}{0.742593in}}%
\pgfpathlineto{\pgfqpoint{1.514998in}{0.742551in}}%
\pgfpathlineto{\pgfqpoint{1.515562in}{0.742508in}}%
\pgfpathlineto{\pgfqpoint{1.516127in}{0.742466in}}%
\pgfpathlineto{\pgfqpoint{1.516691in}{0.742424in}}%
\pgfpathlineto{\pgfqpoint{1.517255in}{0.742381in}}%
\pgfpathlineto{\pgfqpoint{1.517820in}{0.740068in}}%
\pgfpathlineto{\pgfqpoint{1.518384in}{0.742779in}}%
\pgfpathlineto{\pgfqpoint{1.518948in}{0.742700in}}%
\pgfpathlineto{\pgfqpoint{1.519513in}{0.743148in}}%
\pgfpathlineto{\pgfqpoint{1.520077in}{0.739527in}}%
\pgfpathlineto{\pgfqpoint{1.520641in}{0.737007in}}%
\pgfpathlineto{\pgfqpoint{1.521206in}{0.738001in}}%
\pgfpathlineto{\pgfqpoint{1.521770in}{0.738709in}}%
\pgfpathlineto{\pgfqpoint{1.522334in}{0.738643in}}%
\pgfpathlineto{\pgfqpoint{1.522899in}{0.738570in}}%
\pgfpathlineto{\pgfqpoint{1.523463in}{0.738497in}}%
\pgfpathlineto{\pgfqpoint{1.524027in}{0.738424in}}%
\pgfpathlineto{\pgfqpoint{1.524592in}{0.738351in}}%
\pgfpathlineto{\pgfqpoint{1.525156in}{0.738278in}}%
\pgfpathlineto{\pgfqpoint{1.525720in}{0.738205in}}%
\pgfpathlineto{\pgfqpoint{1.526285in}{0.738132in}}%
\pgfpathlineto{\pgfqpoint{1.526849in}{0.738059in}}%
\pgfpathlineto{\pgfqpoint{1.527413in}{0.737986in}}%
\pgfpathlineto{\pgfqpoint{1.527978in}{0.737913in}}%
\pgfpathlineto{\pgfqpoint{1.528542in}{0.737840in}}%
\pgfpathlineto{\pgfqpoint{1.529106in}{0.737767in}}%
\pgfpathlineto{\pgfqpoint{1.529671in}{0.737694in}}%
\pgfpathlineto{\pgfqpoint{1.530235in}{0.737621in}}%
\pgfpathlineto{\pgfqpoint{1.530800in}{0.737548in}}%
\pgfpathlineto{\pgfqpoint{1.531364in}{0.737475in}}%
\pgfpathlineto{\pgfqpoint{1.531928in}{0.737402in}}%
\pgfpathlineto{\pgfqpoint{1.532493in}{0.737329in}}%
\pgfpathlineto{\pgfqpoint{1.533057in}{0.737256in}}%
\pgfpathlineto{\pgfqpoint{1.533621in}{0.737183in}}%
\pgfpathlineto{\pgfqpoint{1.534186in}{0.737110in}}%
\pgfpathlineto{\pgfqpoint{1.534750in}{0.737037in}}%
\pgfpathlineto{\pgfqpoint{1.535314in}{0.736964in}}%
\pgfpathlineto{\pgfqpoint{1.535879in}{0.736891in}}%
\pgfpathlineto{\pgfqpoint{1.536443in}{0.736818in}}%
\pgfpathlineto{\pgfqpoint{1.537007in}{0.736745in}}%
\pgfpathlineto{\pgfqpoint{1.537572in}{0.736672in}}%
\pgfpathlineto{\pgfqpoint{1.538136in}{0.736599in}}%
\pgfpathlineto{\pgfqpoint{1.538700in}{0.736526in}}%
\pgfpathlineto{\pgfqpoint{1.539265in}{0.736453in}}%
\pgfpathlineto{\pgfqpoint{1.539829in}{0.736380in}}%
\pgfpathlineto{\pgfqpoint{1.540393in}{0.736307in}}%
\pgfpathlineto{\pgfqpoint{1.540958in}{0.736234in}}%
\pgfpathlineto{\pgfqpoint{1.541522in}{0.736161in}}%
\pgfpathlineto{\pgfqpoint{1.542086in}{0.736088in}}%
\pgfpathlineto{\pgfqpoint{1.542651in}{0.736015in}}%
\pgfpathlineto{\pgfqpoint{1.543215in}{0.735942in}}%
\pgfpathlineto{\pgfqpoint{1.543779in}{0.735869in}}%
\pgfpathlineto{\pgfqpoint{1.544344in}{0.735796in}}%
\pgfpathlineto{\pgfqpoint{1.544908in}{0.735723in}}%
\pgfpathlineto{\pgfqpoint{1.545472in}{0.735650in}}%
\pgfpathlineto{\pgfqpoint{1.546037in}{0.735576in}}%
\pgfpathlineto{\pgfqpoint{1.546601in}{0.735503in}}%
\pgfpathlineto{\pgfqpoint{1.547166in}{0.735430in}}%
\pgfpathlineto{\pgfqpoint{1.547730in}{0.735357in}}%
\pgfpathlineto{\pgfqpoint{1.548294in}{0.735284in}}%
\pgfpathlineto{\pgfqpoint{1.548859in}{0.735211in}}%
\pgfpathlineto{\pgfqpoint{1.549423in}{0.735138in}}%
\pgfpathlineto{\pgfqpoint{1.549987in}{0.735065in}}%
\pgfpathlineto{\pgfqpoint{1.550552in}{0.734992in}}%
\pgfpathlineto{\pgfqpoint{1.551116in}{0.734919in}}%
\pgfpathlineto{\pgfqpoint{1.551680in}{0.734846in}}%
\pgfpathlineto{\pgfqpoint{1.552245in}{0.734773in}}%
\pgfpathlineto{\pgfqpoint{1.552809in}{0.734700in}}%
\pgfpathlineto{\pgfqpoint{1.553373in}{0.734627in}}%
\pgfpathlineto{\pgfqpoint{1.553938in}{0.734554in}}%
\pgfpathlineto{\pgfqpoint{1.554502in}{0.734481in}}%
\pgfpathlineto{\pgfqpoint{1.555066in}{0.734408in}}%
\pgfpathlineto{\pgfqpoint{1.555631in}{0.734335in}}%
\pgfpathlineto{\pgfqpoint{1.556195in}{0.734262in}}%
\pgfpathlineto{\pgfqpoint{1.556759in}{0.734189in}}%
\pgfpathlineto{\pgfqpoint{1.557324in}{0.734049in}}%
\pgfpathlineto{\pgfqpoint{1.557888in}{0.733307in}}%
\pgfpathlineto{\pgfqpoint{1.558452in}{0.734055in}}%
\pgfpathlineto{\pgfqpoint{1.559017in}{0.735895in}}%
\pgfpathlineto{\pgfqpoint{1.559581in}{0.738062in}}%
\pgfpathlineto{\pgfqpoint{1.560145in}{0.737907in}}%
\pgfpathlineto{\pgfqpoint{1.560710in}{0.737932in}}%
\pgfpathlineto{\pgfqpoint{1.561274in}{0.737051in}}%
\pgfpathlineto{\pgfqpoint{1.561839in}{0.734210in}}%
\pgfpathlineto{\pgfqpoint{1.562403in}{0.740968in}}%
\pgfpathlineto{\pgfqpoint{1.562967in}{0.740952in}}%
\pgfpathlineto{\pgfqpoint{1.563532in}{0.740943in}}%
\pgfpathlineto{\pgfqpoint{1.564096in}{0.740935in}}%
\pgfpathlineto{\pgfqpoint{1.564660in}{0.740927in}}%
\pgfpathlineto{\pgfqpoint{1.565225in}{0.740918in}}%
\pgfpathlineto{\pgfqpoint{1.565789in}{0.740910in}}%
\pgfpathlineto{\pgfqpoint{1.566353in}{0.740901in}}%
\pgfpathlineto{\pgfqpoint{1.566918in}{0.740893in}}%
\pgfpathlineto{\pgfqpoint{1.567482in}{0.740885in}}%
\pgfpathlineto{\pgfqpoint{1.568046in}{0.740876in}}%
\pgfpathlineto{\pgfqpoint{1.568611in}{0.740868in}}%
\pgfpathlineto{\pgfqpoint{1.569175in}{0.740860in}}%
\pgfpathlineto{\pgfqpoint{1.569739in}{0.740851in}}%
\pgfpathlineto{\pgfqpoint{1.570304in}{0.740843in}}%
\pgfpathlineto{\pgfqpoint{1.570868in}{0.740834in}}%
\pgfpathlineto{\pgfqpoint{1.571432in}{0.740827in}}%
\pgfpathlineto{\pgfqpoint{1.571997in}{0.740974in}}%
\pgfpathlineto{\pgfqpoint{1.572561in}{0.741237in}}%
\pgfpathlineto{\pgfqpoint{1.573125in}{0.738374in}}%
\pgfpathlineto{\pgfqpoint{1.573690in}{0.737178in}}%
\pgfpathlineto{\pgfqpoint{1.574254in}{0.739491in}}%
\pgfpathlineto{\pgfqpoint{1.574818in}{0.739771in}}%
\pgfpathlineto{\pgfqpoint{1.575383in}{0.741481in}}%
\pgfpathlineto{\pgfqpoint{1.575947in}{0.737193in}}%
\pgfpathlineto{\pgfqpoint{1.576512in}{0.736144in}}%
\pgfpathlineto{\pgfqpoint{1.577076in}{0.736251in}}%
\pgfpathlineto{\pgfqpoint{1.577640in}{0.736063in}}%
\pgfpathlineto{\pgfqpoint{1.578205in}{0.735875in}}%
\pgfpathlineto{\pgfqpoint{1.578769in}{0.735686in}}%
\pgfpathlineto{\pgfqpoint{1.579333in}{0.735498in}}%
\pgfpathlineto{\pgfqpoint{1.579898in}{0.735309in}}%
\pgfpathlineto{\pgfqpoint{1.580462in}{0.735121in}}%
\pgfpathlineto{\pgfqpoint{1.581026in}{0.734933in}}%
\pgfpathlineto{\pgfqpoint{1.581591in}{0.734744in}}%
\pgfpathlineto{\pgfqpoint{1.582155in}{0.734556in}}%
\pgfpathlineto{\pgfqpoint{1.582719in}{0.734367in}}%
\pgfpathlineto{\pgfqpoint{1.583284in}{0.734179in}}%
\pgfpathlineto{\pgfqpoint{1.583848in}{0.733990in}}%
\pgfpathlineto{\pgfqpoint{1.584412in}{0.733802in}}%
\pgfpathlineto{\pgfqpoint{1.584977in}{0.733614in}}%
\pgfpathlineto{\pgfqpoint{1.585541in}{0.728581in}}%
\pgfpathlineto{\pgfqpoint{1.586105in}{0.733103in}}%
\pgfpathlineto{\pgfqpoint{1.586670in}{0.732532in}}%
\pgfpathlineto{\pgfqpoint{1.587234in}{0.730521in}}%
\pgfpathlineto{\pgfqpoint{1.587798in}{0.728825in}}%
\pgfpathlineto{\pgfqpoint{1.588363in}{0.734083in}}%
\pgfpathlineto{\pgfqpoint{1.588927in}{0.731953in}}%
\pgfpathlineto{\pgfqpoint{1.589491in}{0.733138in}}%
\pgfpathlineto{\pgfqpoint{1.590056in}{0.735340in}}%
\pgfpathlineto{\pgfqpoint{1.590620in}{0.734204in}}%
\pgfpathlineto{\pgfqpoint{1.591184in}{0.733062in}}%
\pgfpathlineto{\pgfqpoint{1.591749in}{0.731921in}}%
\pgfpathlineto{\pgfqpoint{1.592313in}{0.730780in}}%
\pgfpathlineto{\pgfqpoint{1.592878in}{0.729638in}}%
\pgfpathlineto{\pgfqpoint{1.593442in}{0.728497in}}%
\pgfpathlineto{\pgfqpoint{1.594006in}{0.727355in}}%
\pgfpathlineto{\pgfqpoint{1.594571in}{0.726214in}}%
\pgfpathlineto{\pgfqpoint{1.595135in}{0.725072in}}%
\pgfpathlineto{\pgfqpoint{1.595699in}{0.723931in}}%
\pgfpathlineto{\pgfqpoint{1.596264in}{0.722789in}}%
\pgfpathlineto{\pgfqpoint{1.596828in}{0.721648in}}%
\pgfpathlineto{\pgfqpoint{1.597392in}{0.720506in}}%
\pgfpathlineto{\pgfqpoint{1.597957in}{0.719365in}}%
\pgfpathlineto{\pgfqpoint{1.598521in}{0.718430in}}%
\pgfpathlineto{\pgfqpoint{1.599085in}{0.729716in}}%
\pgfpathlineto{\pgfqpoint{1.599650in}{0.726800in}}%
\pgfpathlineto{\pgfqpoint{1.600214in}{0.728629in}}%
\pgfpathlineto{\pgfqpoint{1.600778in}{0.730378in}}%
\pgfpathlineto{\pgfqpoint{1.601343in}{0.737437in}}%
\pgfpathlineto{\pgfqpoint{1.601907in}{0.736718in}}%
\pgfpathlineto{\pgfqpoint{1.602471in}{0.743235in}}%
\pgfpathlineto{\pgfqpoint{1.603036in}{0.751639in}}%
\pgfpathlineto{\pgfqpoint{1.603600in}{0.754613in}}%
\pgfpathlineto{\pgfqpoint{1.604164in}{0.753320in}}%
\pgfpathlineto{\pgfqpoint{1.604729in}{0.750786in}}%
\pgfpathlineto{\pgfqpoint{1.605293in}{0.748251in}}%
\pgfpathlineto{\pgfqpoint{1.605857in}{0.745717in}}%
\pgfpathlineto{\pgfqpoint{1.606422in}{0.742767in}}%
\pgfpathlineto{\pgfqpoint{1.606986in}{0.729254in}}%
\pgfpathlineto{\pgfqpoint{1.607551in}{0.728756in}}%
\pgfpathlineto{\pgfqpoint{1.608115in}{0.728257in}}%
\pgfpathlineto{\pgfqpoint{1.608679in}{0.727758in}}%
\pgfpathlineto{\pgfqpoint{1.609244in}{0.727260in}}%
\pgfpathlineto{\pgfqpoint{1.609808in}{0.726761in}}%
\pgfpathlineto{\pgfqpoint{1.610372in}{0.726262in}}%
\pgfpathlineto{\pgfqpoint{1.610937in}{0.725764in}}%
\pgfpathlineto{\pgfqpoint{1.611501in}{0.725265in}}%
\pgfpathlineto{\pgfqpoint{1.612065in}{0.724803in}}%
\pgfpathlineto{\pgfqpoint{1.612630in}{0.724361in}}%
\pgfpathlineto{\pgfqpoint{1.613194in}{0.723862in}}%
\pgfpathlineto{\pgfqpoint{1.613758in}{0.723364in}}%
\pgfpathlineto{\pgfqpoint{1.614323in}{0.722865in}}%
\pgfpathlineto{\pgfqpoint{1.614887in}{0.722366in}}%
\pgfpathlineto{\pgfqpoint{1.615451in}{0.721868in}}%
\pgfpathlineto{\pgfqpoint{1.616016in}{0.721392in}}%
\pgfpathlineto{\pgfqpoint{1.616580in}{0.721268in}}%
\pgfpathlineto{\pgfqpoint{1.617144in}{0.721262in}}%
\pgfpathlineto{\pgfqpoint{1.617709in}{0.721256in}}%
\pgfpathlineto{\pgfqpoint{1.618273in}{0.721249in}}%
\pgfpathlineto{\pgfqpoint{1.618837in}{0.721243in}}%
\pgfpathlineto{\pgfqpoint{1.619402in}{0.721237in}}%
\pgfpathlineto{\pgfqpoint{1.619966in}{0.721231in}}%
\pgfpathlineto{\pgfqpoint{1.620530in}{0.721224in}}%
\pgfpathlineto{\pgfqpoint{1.621095in}{0.721218in}}%
\pgfpathlineto{\pgfqpoint{1.621659in}{0.721212in}}%
\pgfpathlineto{\pgfqpoint{1.622224in}{0.721205in}}%
\pgfpathlineto{\pgfqpoint{1.622788in}{0.721199in}}%
\pgfpathlineto{\pgfqpoint{1.623352in}{0.721193in}}%
\pgfpathlineto{\pgfqpoint{1.623917in}{0.721187in}}%
\pgfpathlineto{\pgfqpoint{1.624481in}{0.721180in}}%
\pgfpathlineto{\pgfqpoint{1.625045in}{0.721174in}}%
\pgfpathlineto{\pgfqpoint{1.625610in}{0.721168in}}%
\pgfpathlineto{\pgfqpoint{1.626174in}{0.721161in}}%
\pgfpathlineto{\pgfqpoint{1.626738in}{0.721155in}}%
\pgfpathlineto{\pgfqpoint{1.627303in}{0.721149in}}%
\pgfpathlineto{\pgfqpoint{1.627867in}{0.721142in}}%
\pgfpathlineto{\pgfqpoint{1.628431in}{0.721136in}}%
\pgfpathlineto{\pgfqpoint{1.628996in}{0.721130in}}%
\pgfpathlineto{\pgfqpoint{1.629560in}{0.721124in}}%
\pgfpathlineto{\pgfqpoint{1.630124in}{0.721117in}}%
\pgfpathlineto{\pgfqpoint{1.630689in}{0.721111in}}%
\pgfpathlineto{\pgfqpoint{1.631253in}{0.721105in}}%
\pgfpathlineto{\pgfqpoint{1.631817in}{0.721098in}}%
\pgfpathlineto{\pgfqpoint{1.632382in}{0.721092in}}%
\pgfpathlineto{\pgfqpoint{1.632946in}{0.721086in}}%
\pgfpathlineto{\pgfqpoint{1.633510in}{0.721080in}}%
\pgfpathlineto{\pgfqpoint{1.634075in}{0.721073in}}%
\pgfpathlineto{\pgfqpoint{1.634639in}{0.721067in}}%
\pgfpathlineto{\pgfqpoint{1.635203in}{0.721061in}}%
\pgfpathlineto{\pgfqpoint{1.635768in}{0.721054in}}%
\pgfpathlineto{\pgfqpoint{1.636332in}{0.721048in}}%
\pgfpathlineto{\pgfqpoint{1.636897in}{0.721042in}}%
\pgfpathlineto{\pgfqpoint{1.637461in}{0.721036in}}%
\pgfpathlineto{\pgfqpoint{1.638025in}{0.721029in}}%
\pgfpathlineto{\pgfqpoint{1.638590in}{0.721023in}}%
\pgfpathlineto{\pgfqpoint{1.639154in}{0.721017in}}%
\pgfpathlineto{\pgfqpoint{1.639718in}{0.721010in}}%
\pgfpathlineto{\pgfqpoint{1.640283in}{0.721004in}}%
\pgfpathlineto{\pgfqpoint{1.640847in}{0.720998in}}%
\pgfpathlineto{\pgfqpoint{1.641411in}{0.720991in}}%
\pgfpathlineto{\pgfqpoint{1.641976in}{0.720985in}}%
\pgfpathlineto{\pgfqpoint{1.642540in}{0.720979in}}%
\pgfpathlineto{\pgfqpoint{1.643104in}{0.720973in}}%
\pgfpathlineto{\pgfqpoint{1.643669in}{0.720966in}}%
\pgfpathlineto{\pgfqpoint{1.644233in}{0.720960in}}%
\pgfpathlineto{\pgfqpoint{1.644797in}{0.720954in}}%
\pgfpathlineto{\pgfqpoint{1.645362in}{0.720947in}}%
\pgfpathlineto{\pgfqpoint{1.645926in}{0.720941in}}%
\pgfpathlineto{\pgfqpoint{1.646490in}{0.720935in}}%
\pgfpathlineto{\pgfqpoint{1.647055in}{0.720929in}}%
\pgfpathlineto{\pgfqpoint{1.647619in}{0.720922in}}%
\pgfpathlineto{\pgfqpoint{1.648183in}{0.720916in}}%
\pgfpathlineto{\pgfqpoint{1.648748in}{0.720910in}}%
\pgfpathlineto{\pgfqpoint{1.649312in}{0.720903in}}%
\pgfpathlineto{\pgfqpoint{1.649876in}{0.720897in}}%
\pgfpathlineto{\pgfqpoint{1.650441in}{0.720891in}}%
\pgfpathlineto{\pgfqpoint{1.651005in}{0.720885in}}%
\pgfpathlineto{\pgfqpoint{1.651569in}{0.720878in}}%
\pgfpathlineto{\pgfqpoint{1.652134in}{0.720872in}}%
\pgfpathlineto{\pgfqpoint{1.652698in}{0.720866in}}%
\pgfpathlineto{\pgfqpoint{1.653263in}{0.720859in}}%
\pgfpathlineto{\pgfqpoint{1.653827in}{0.720853in}}%
\pgfpathlineto{\pgfqpoint{1.654391in}{0.720847in}}%
\pgfpathlineto{\pgfqpoint{1.654956in}{0.720840in}}%
\pgfpathlineto{\pgfqpoint{1.655520in}{0.720834in}}%
\pgfpathlineto{\pgfqpoint{1.656084in}{0.720828in}}%
\pgfpathlineto{\pgfqpoint{1.656649in}{0.720822in}}%
\pgfpathlineto{\pgfqpoint{1.657213in}{0.720815in}}%
\pgfpathlineto{\pgfqpoint{1.657777in}{0.720808in}}%
\pgfpathlineto{\pgfqpoint{1.658342in}{0.720802in}}%
\pgfpathlineto{\pgfqpoint{1.658906in}{0.720795in}}%
\pgfpathlineto{\pgfqpoint{1.659470in}{0.720788in}}%
\pgfpathlineto{\pgfqpoint{1.660035in}{0.720781in}}%
\pgfpathlineto{\pgfqpoint{1.660599in}{0.720775in}}%
\pgfpathlineto{\pgfqpoint{1.661163in}{0.720768in}}%
\pgfpathlineto{\pgfqpoint{1.661728in}{0.720761in}}%
\pgfpathlineto{\pgfqpoint{1.662292in}{0.720755in}}%
\pgfpathlineto{\pgfqpoint{1.662856in}{0.720748in}}%
\pgfpathlineto{\pgfqpoint{1.663421in}{0.720741in}}%
\pgfpathlineto{\pgfqpoint{1.663985in}{0.720734in}}%
\pgfpathlineto{\pgfqpoint{1.664549in}{0.720728in}}%
\pgfpathlineto{\pgfqpoint{1.665114in}{0.720721in}}%
\pgfpathlineto{\pgfqpoint{1.665678in}{0.720714in}}%
\pgfpathlineto{\pgfqpoint{1.666242in}{0.720707in}}%
\pgfpathlineto{\pgfqpoint{1.666807in}{0.720701in}}%
\pgfpathlineto{\pgfqpoint{1.667371in}{0.720694in}}%
\pgfpathlineto{\pgfqpoint{1.667936in}{0.720687in}}%
\pgfpathlineto{\pgfqpoint{1.668500in}{0.720681in}}%
\pgfpathlineto{\pgfqpoint{1.669064in}{0.720674in}}%
\pgfpathlineto{\pgfqpoint{1.669629in}{0.720667in}}%
\pgfpathlineto{\pgfqpoint{1.670193in}{0.720664in}}%
\pgfpathlineto{\pgfqpoint{1.670757in}{0.720699in}}%
\pgfpathlineto{\pgfqpoint{1.671322in}{0.720717in}}%
\pgfpathlineto{\pgfqpoint{1.671886in}{0.720712in}}%
\pgfpathlineto{\pgfqpoint{1.672450in}{0.720707in}}%
\pgfpathlineto{\pgfqpoint{1.673015in}{0.720702in}}%
\pgfpathlineto{\pgfqpoint{1.673579in}{0.720697in}}%
\pgfpathlineto{\pgfqpoint{1.674143in}{0.720692in}}%
\pgfpathlineto{\pgfqpoint{1.674708in}{0.720687in}}%
\pgfpathlineto{\pgfqpoint{1.675272in}{0.720682in}}%
\pgfpathlineto{\pgfqpoint{1.675836in}{0.720677in}}%
\pgfpathlineto{\pgfqpoint{1.676401in}{0.720672in}}%
\pgfpathlineto{\pgfqpoint{1.676965in}{0.720667in}}%
\pgfpathlineto{\pgfqpoint{1.677529in}{0.720662in}}%
\pgfpathlineto{\pgfqpoint{1.678094in}{0.720657in}}%
\pgfpathlineto{\pgfqpoint{1.678658in}{0.720652in}}%
\pgfpathlineto{\pgfqpoint{1.679222in}{0.720647in}}%
\pgfpathlineto{\pgfqpoint{1.679787in}{0.720621in}}%
\pgfpathlineto{\pgfqpoint{1.680351in}{0.720558in}}%
\pgfpathlineto{\pgfqpoint{1.680915in}{0.720536in}}%
\pgfpathlineto{\pgfqpoint{1.681480in}{0.720525in}}%
\pgfpathlineto{\pgfqpoint{1.682044in}{0.720514in}}%
\pgfpathlineto{\pgfqpoint{1.682609in}{0.720503in}}%
\pgfpathlineto{\pgfqpoint{1.683173in}{0.720491in}}%
\pgfpathlineto{\pgfqpoint{1.683737in}{0.720480in}}%
\pgfpathlineto{\pgfqpoint{1.684302in}{0.720471in}}%
\pgfpathlineto{\pgfqpoint{1.684866in}{0.720466in}}%
\pgfpathlineto{\pgfqpoint{1.685430in}{0.720462in}}%
\pgfpathlineto{\pgfqpoint{1.685995in}{0.720457in}}%
\pgfpathlineto{\pgfqpoint{1.686559in}{0.720453in}}%
\pgfpathlineto{\pgfqpoint{1.687123in}{0.720448in}}%
\pgfpathlineto{\pgfqpoint{1.687688in}{0.720444in}}%
\pgfpathlineto{\pgfqpoint{1.688252in}{0.720439in}}%
\pgfpathlineto{\pgfqpoint{1.688816in}{0.720435in}}%
\pgfpathlineto{\pgfqpoint{1.689381in}{0.720430in}}%
\pgfpathlineto{\pgfqpoint{1.689945in}{0.720426in}}%
\pgfpathlineto{\pgfqpoint{1.690509in}{0.720421in}}%
\pgfpathlineto{\pgfqpoint{1.691074in}{0.720417in}}%
\pgfpathlineto{\pgfqpoint{1.691638in}{0.720412in}}%
\pgfpathlineto{\pgfqpoint{1.692202in}{0.720408in}}%
\pgfpathlineto{\pgfqpoint{1.692767in}{0.720403in}}%
\pgfpathlineto{\pgfqpoint{1.693331in}{0.720397in}}%
\pgfpathlineto{\pgfqpoint{1.693895in}{0.720461in}}%
\pgfpathlineto{\pgfqpoint{1.694460in}{0.720478in}}%
\pgfpathlineto{\pgfqpoint{1.695024in}{0.720473in}}%
\pgfpathlineto{\pgfqpoint{1.695588in}{0.720497in}}%
\pgfpathlineto{\pgfqpoint{1.696153in}{0.720531in}}%
\pgfpathlineto{\pgfqpoint{1.696717in}{0.720537in}}%
\pgfpathlineto{\pgfqpoint{1.697281in}{0.720526in}}%
\pgfpathlineto{\pgfqpoint{1.697846in}{0.720515in}}%
\pgfpathlineto{\pgfqpoint{1.698410in}{0.720504in}}%
\pgfpathlineto{\pgfqpoint{1.698975in}{0.720493in}}%
\pgfpathlineto{\pgfqpoint{1.699539in}{0.720482in}}%
\pgfpathlineto{\pgfqpoint{1.700103in}{0.720471in}}%
\pgfpathlineto{\pgfqpoint{1.700668in}{0.720460in}}%
\pgfpathlineto{\pgfqpoint{1.701232in}{0.720449in}}%
\pgfpathlineto{\pgfqpoint{1.701796in}{0.720438in}}%
\pgfpathlineto{\pgfqpoint{1.702361in}{0.720427in}}%
\pgfpathlineto{\pgfqpoint{1.702925in}{0.720416in}}%
\pgfpathlineto{\pgfqpoint{1.703489in}{0.720405in}}%
\pgfpathlineto{\pgfqpoint{1.704054in}{0.720394in}}%
\pgfpathlineto{\pgfqpoint{1.704618in}{0.720383in}}%
\pgfpathlineto{\pgfqpoint{1.705182in}{0.720372in}}%
\pgfpathlineto{\pgfqpoint{1.705747in}{0.720361in}}%
\pgfpathlineto{\pgfqpoint{1.706311in}{0.720350in}}%
\pgfpathlineto{\pgfqpoint{1.706875in}{0.720339in}}%
\pgfpathlineto{\pgfqpoint{1.707440in}{0.720328in}}%
\pgfpathlineto{\pgfqpoint{1.708004in}{0.720317in}}%
\pgfpathlineto{\pgfqpoint{1.708568in}{0.720306in}}%
\pgfpathlineto{\pgfqpoint{1.709133in}{0.720295in}}%
\pgfpathlineto{\pgfqpoint{1.709697in}{0.720284in}}%
\pgfpathlineto{\pgfqpoint{1.710261in}{0.720273in}}%
\pgfpathlineto{\pgfqpoint{1.710826in}{0.720262in}}%
\pgfpathlineto{\pgfqpoint{1.711390in}{0.720251in}}%
\pgfpathlineto{\pgfqpoint{1.711954in}{0.720240in}}%
\pgfpathlineto{\pgfqpoint{1.712519in}{0.720229in}}%
\pgfpathlineto{\pgfqpoint{1.713083in}{0.720218in}}%
\pgfpathlineto{\pgfqpoint{1.713648in}{0.720207in}}%
\pgfpathlineto{\pgfqpoint{1.714212in}{0.720196in}}%
\pgfpathlineto{\pgfqpoint{1.714776in}{0.720185in}}%
\pgfpathlineto{\pgfqpoint{1.715341in}{0.720174in}}%
\pgfpathlineto{\pgfqpoint{1.715905in}{0.720163in}}%
\pgfpathlineto{\pgfqpoint{1.716469in}{0.720152in}}%
\pgfpathlineto{\pgfqpoint{1.717034in}{0.720141in}}%
\pgfpathlineto{\pgfqpoint{1.717598in}{0.720130in}}%
\pgfpathlineto{\pgfqpoint{1.718162in}{0.720119in}}%
\pgfpathlineto{\pgfqpoint{1.718727in}{0.720109in}}%
\pgfpathlineto{\pgfqpoint{1.719291in}{0.720098in}}%
\pgfpathlineto{\pgfqpoint{1.719855in}{0.720087in}}%
\pgfpathlineto{\pgfqpoint{1.720420in}{0.720076in}}%
\pgfpathlineto{\pgfqpoint{1.720984in}{0.720065in}}%
\pgfpathlineto{\pgfqpoint{1.721548in}{0.720054in}}%
\pgfpathlineto{\pgfqpoint{1.722113in}{0.720043in}}%
\pgfpathlineto{\pgfqpoint{1.722677in}{0.720032in}}%
\pgfpathlineto{\pgfqpoint{1.723241in}{0.720021in}}%
\pgfpathlineto{\pgfqpoint{1.723806in}{0.720010in}}%
\pgfpathlineto{\pgfqpoint{1.724370in}{0.719999in}}%
\pgfpathlineto{\pgfqpoint{1.724934in}{0.719988in}}%
\pgfpathlineto{\pgfqpoint{1.725499in}{0.719977in}}%
\pgfpathlineto{\pgfqpoint{1.726063in}{0.719966in}}%
\pgfpathlineto{\pgfqpoint{1.726627in}{0.719955in}}%
\pgfpathlineto{\pgfqpoint{1.727192in}{0.719944in}}%
\pgfpathlineto{\pgfqpoint{1.727756in}{0.719933in}}%
\pgfpathlineto{\pgfqpoint{1.728321in}{0.719922in}}%
\pgfpathlineto{\pgfqpoint{1.728885in}{0.719911in}}%
\pgfpathlineto{\pgfqpoint{1.729449in}{0.719900in}}%
\pgfpathlineto{\pgfqpoint{1.730014in}{0.719889in}}%
\pgfpathlineto{\pgfqpoint{1.730578in}{0.719878in}}%
\pgfpathlineto{\pgfqpoint{1.731142in}{0.719867in}}%
\pgfpathlineto{\pgfqpoint{1.731707in}{0.719856in}}%
\pgfpathlineto{\pgfqpoint{1.732271in}{0.719845in}}%
\pgfpathlineto{\pgfqpoint{1.732835in}{0.719834in}}%
\pgfpathlineto{\pgfqpoint{1.733400in}{0.719823in}}%
\pgfpathlineto{\pgfqpoint{1.733964in}{0.719812in}}%
\pgfpathlineto{\pgfqpoint{1.734528in}{0.719801in}}%
\pgfpathlineto{\pgfqpoint{1.735093in}{0.719790in}}%
\pgfpathlineto{\pgfqpoint{1.735657in}{0.719779in}}%
\pgfpathlineto{\pgfqpoint{1.736221in}{0.719768in}}%
\pgfpathlineto{\pgfqpoint{1.736786in}{0.719757in}}%
\pgfpathlineto{\pgfqpoint{1.737350in}{0.719746in}}%
\pgfpathlineto{\pgfqpoint{1.737914in}{0.719735in}}%
\pgfpathlineto{\pgfqpoint{1.738479in}{0.719724in}}%
\pgfpathlineto{\pgfqpoint{1.739043in}{0.719713in}}%
\pgfpathlineto{\pgfqpoint{1.739607in}{0.719702in}}%
\pgfpathlineto{\pgfqpoint{1.740172in}{0.719691in}}%
\pgfpathlineto{\pgfqpoint{1.740736in}{0.719680in}}%
\pgfpathlineto{\pgfqpoint{1.741300in}{0.719669in}}%
\pgfpathlineto{\pgfqpoint{1.741865in}{0.719658in}}%
\pgfpathlineto{\pgfqpoint{1.742429in}{0.719647in}}%
\pgfpathlineto{\pgfqpoint{1.742993in}{0.719636in}}%
\pgfpathlineto{\pgfqpoint{1.743558in}{0.719625in}}%
\pgfpathlineto{\pgfqpoint{1.744122in}{0.719614in}}%
\pgfpathlineto{\pgfqpoint{1.744687in}{0.719603in}}%
\pgfpathlineto{\pgfqpoint{1.745251in}{0.719592in}}%
\pgfpathlineto{\pgfqpoint{1.745815in}{0.719581in}}%
\pgfpathlineto{\pgfqpoint{1.746380in}{0.719570in}}%
\pgfpathlineto{\pgfqpoint{1.746944in}{0.719559in}}%
\pgfpathlineto{\pgfqpoint{1.747508in}{0.719548in}}%
\pgfpathlineto{\pgfqpoint{1.748073in}{0.719537in}}%
\pgfpathlineto{\pgfqpoint{1.748637in}{0.719526in}}%
\pgfpathlineto{\pgfqpoint{1.749201in}{0.719515in}}%
\pgfpathlineto{\pgfqpoint{1.749766in}{0.719504in}}%
\pgfpathlineto{\pgfqpoint{1.750330in}{0.719493in}}%
\pgfpathlineto{\pgfqpoint{1.750894in}{0.719482in}}%
\pgfpathlineto{\pgfqpoint{1.751459in}{0.719471in}}%
\pgfpathlineto{\pgfqpoint{1.752023in}{0.719460in}}%
\pgfpathlineto{\pgfqpoint{1.752587in}{0.719449in}}%
\pgfpathlineto{\pgfqpoint{1.753152in}{0.719438in}}%
\pgfpathlineto{\pgfqpoint{1.753716in}{0.719427in}}%
\pgfpathlineto{\pgfqpoint{1.754280in}{0.719416in}}%
\pgfpathlineto{\pgfqpoint{1.754845in}{0.719405in}}%
\pgfpathlineto{\pgfqpoint{1.755409in}{0.719394in}}%
\pgfpathlineto{\pgfqpoint{1.755973in}{0.719383in}}%
\pgfpathlineto{\pgfqpoint{1.756538in}{0.719372in}}%
\pgfpathlineto{\pgfqpoint{1.757102in}{0.719361in}}%
\pgfpathlineto{\pgfqpoint{1.757666in}{0.719350in}}%
\pgfpathlineto{\pgfqpoint{1.758231in}{0.719339in}}%
\pgfpathlineto{\pgfqpoint{1.758795in}{0.719328in}}%
\pgfpathlineto{\pgfqpoint{1.759360in}{0.719317in}}%
\pgfpathlineto{\pgfqpoint{1.759924in}{0.719306in}}%
\pgfpathlineto{\pgfqpoint{1.760488in}{0.719295in}}%
\pgfpathlineto{\pgfqpoint{1.761053in}{0.719284in}}%
\pgfpathlineto{\pgfqpoint{1.761617in}{0.719273in}}%
\pgfpathlineto{\pgfqpoint{1.762181in}{0.719262in}}%
\pgfpathlineto{\pgfqpoint{1.762746in}{0.719251in}}%
\pgfpathlineto{\pgfqpoint{1.763310in}{0.719240in}}%
\pgfpathlineto{\pgfqpoint{1.763874in}{0.719229in}}%
\pgfpathlineto{\pgfqpoint{1.764439in}{0.719218in}}%
\pgfpathlineto{\pgfqpoint{1.765003in}{0.719207in}}%
\pgfpathlineto{\pgfqpoint{1.765567in}{0.719196in}}%
\pgfpathlineto{\pgfqpoint{1.766132in}{0.719185in}}%
\pgfpathlineto{\pgfqpoint{1.766696in}{0.719174in}}%
\pgfpathlineto{\pgfqpoint{1.767260in}{0.719163in}}%
\pgfpathlineto{\pgfqpoint{1.767825in}{0.719152in}}%
\pgfpathlineto{\pgfqpoint{1.768389in}{0.719141in}}%
\pgfpathlineto{\pgfqpoint{1.768953in}{0.719130in}}%
\pgfpathlineto{\pgfqpoint{1.769518in}{0.719119in}}%
\pgfpathlineto{\pgfqpoint{1.770082in}{0.719108in}}%
\pgfpathlineto{\pgfqpoint{1.770646in}{0.719097in}}%
\pgfpathlineto{\pgfqpoint{1.771211in}{0.719086in}}%
\pgfpathlineto{\pgfqpoint{1.771775in}{0.719075in}}%
\pgfpathlineto{\pgfqpoint{1.772339in}{0.719064in}}%
\pgfpathlineto{\pgfqpoint{1.772904in}{0.719053in}}%
\pgfpathlineto{\pgfqpoint{1.773468in}{0.719042in}}%
\pgfpathlineto{\pgfqpoint{1.774033in}{0.719031in}}%
\pgfpathlineto{\pgfqpoint{1.774597in}{0.719020in}}%
\pgfpathlineto{\pgfqpoint{1.775161in}{0.719009in}}%
\pgfpathlineto{\pgfqpoint{1.775726in}{0.718998in}}%
\pgfpathlineto{\pgfqpoint{1.776290in}{0.718987in}}%
\pgfpathlineto{\pgfqpoint{1.776854in}{0.718976in}}%
\pgfpathlineto{\pgfqpoint{1.777419in}{0.718965in}}%
\pgfpathlineto{\pgfqpoint{1.777983in}{0.718954in}}%
\pgfpathlineto{\pgfqpoint{1.778547in}{0.718943in}}%
\pgfpathlineto{\pgfqpoint{1.779112in}{0.718932in}}%
\pgfpathlineto{\pgfqpoint{1.779676in}{0.718921in}}%
\pgfpathlineto{\pgfqpoint{1.780240in}{0.718910in}}%
\pgfpathlineto{\pgfqpoint{1.780805in}{0.718899in}}%
\pgfpathlineto{\pgfqpoint{1.781369in}{0.718888in}}%
\pgfpathlineto{\pgfqpoint{1.781933in}{0.718877in}}%
\pgfpathlineto{\pgfqpoint{1.782498in}{0.718866in}}%
\pgfpathlineto{\pgfqpoint{1.783062in}{0.718855in}}%
\pgfpathlineto{\pgfqpoint{1.783626in}{0.718844in}}%
\pgfpathlineto{\pgfqpoint{1.784191in}{0.718833in}}%
\pgfpathlineto{\pgfqpoint{1.784755in}{0.718823in}}%
\pgfpathlineto{\pgfqpoint{1.785319in}{0.718812in}}%
\pgfpathlineto{\pgfqpoint{1.785884in}{0.718801in}}%
\pgfpathlineto{\pgfqpoint{1.786448in}{0.718790in}}%
\pgfpathlineto{\pgfqpoint{1.787012in}{0.718779in}}%
\pgfpathlineto{\pgfqpoint{1.787577in}{0.718768in}}%
\pgfpathlineto{\pgfqpoint{1.788141in}{0.718757in}}%
\pgfpathlineto{\pgfqpoint{1.788705in}{0.718746in}}%
\pgfpathlineto{\pgfqpoint{1.789270in}{0.718735in}}%
\pgfpathlineto{\pgfqpoint{1.789834in}{0.718724in}}%
\pgfpathlineto{\pgfqpoint{1.790399in}{0.718713in}}%
\pgfpathlineto{\pgfqpoint{1.790963in}{0.718702in}}%
\pgfpathlineto{\pgfqpoint{1.791527in}{0.718691in}}%
\pgfpathlineto{\pgfqpoint{1.792092in}{0.718680in}}%
\pgfpathlineto{\pgfqpoint{1.792656in}{0.718669in}}%
\pgfpathlineto{\pgfqpoint{1.793220in}{0.718658in}}%
\pgfpathlineto{\pgfqpoint{1.793785in}{0.718647in}}%
\pgfpathlineto{\pgfqpoint{1.794349in}{0.718636in}}%
\pgfpathlineto{\pgfqpoint{1.794913in}{0.718625in}}%
\pgfpathlineto{\pgfqpoint{1.795478in}{0.718614in}}%
\pgfpathlineto{\pgfqpoint{1.796042in}{0.718603in}}%
\pgfpathlineto{\pgfqpoint{1.796606in}{0.718592in}}%
\pgfpathlineto{\pgfqpoint{1.797171in}{0.718581in}}%
\pgfpathlineto{\pgfqpoint{1.797735in}{0.718570in}}%
\pgfpathlineto{\pgfqpoint{1.798299in}{0.718559in}}%
\pgfpathlineto{\pgfqpoint{1.798864in}{0.718548in}}%
\pgfpathlineto{\pgfqpoint{1.799428in}{0.718537in}}%
\pgfpathlineto{\pgfqpoint{1.799992in}{0.718526in}}%
\pgfpathlineto{\pgfqpoint{1.800557in}{0.718515in}}%
\pgfpathlineto{\pgfqpoint{1.801121in}{0.718504in}}%
\pgfpathlineto{\pgfqpoint{1.801685in}{0.718493in}}%
\pgfpathlineto{\pgfqpoint{1.802250in}{0.718482in}}%
\pgfpathlineto{\pgfqpoint{1.802814in}{0.718471in}}%
\pgfpathlineto{\pgfqpoint{1.803378in}{0.718460in}}%
\pgfpathlineto{\pgfqpoint{1.803943in}{0.718449in}}%
\pgfpathlineto{\pgfqpoint{1.804507in}{0.718438in}}%
\pgfpathlineto{\pgfqpoint{1.805072in}{0.718427in}}%
\pgfpathlineto{\pgfqpoint{1.805636in}{0.718416in}}%
\pgfpathlineto{\pgfqpoint{1.806200in}{0.718405in}}%
\pgfpathlineto{\pgfqpoint{1.806765in}{0.718394in}}%
\pgfpathlineto{\pgfqpoint{1.807329in}{0.718383in}}%
\pgfpathlineto{\pgfqpoint{1.807893in}{0.718372in}}%
\pgfpathlineto{\pgfqpoint{1.808458in}{0.718361in}}%
\pgfpathlineto{\pgfqpoint{1.809022in}{0.718350in}}%
\pgfpathlineto{\pgfqpoint{1.809586in}{0.718339in}}%
\pgfpathlineto{\pgfqpoint{1.810151in}{0.718328in}}%
\pgfpathlineto{\pgfqpoint{1.810715in}{0.718317in}}%
\pgfpathlineto{\pgfqpoint{1.811279in}{0.718306in}}%
\pgfpathlineto{\pgfqpoint{1.811844in}{0.718295in}}%
\pgfpathlineto{\pgfqpoint{1.812408in}{0.718284in}}%
\pgfpathlineto{\pgfqpoint{1.812972in}{0.718273in}}%
\pgfpathlineto{\pgfqpoint{1.813537in}{0.718262in}}%
\pgfpathlineto{\pgfqpoint{1.814101in}{0.718251in}}%
\pgfpathlineto{\pgfqpoint{1.814665in}{0.718240in}}%
\pgfpathlineto{\pgfqpoint{1.815230in}{0.718229in}}%
\pgfpathlineto{\pgfqpoint{1.815794in}{0.718218in}}%
\pgfpathlineto{\pgfqpoint{1.816358in}{0.718207in}}%
\pgfpathlineto{\pgfqpoint{1.816923in}{0.718196in}}%
\pgfpathlineto{\pgfqpoint{1.817487in}{0.718185in}}%
\pgfpathlineto{\pgfqpoint{1.818051in}{0.718174in}}%
\pgfpathlineto{\pgfqpoint{1.818616in}{0.718163in}}%
\pgfpathlineto{\pgfqpoint{1.819180in}{0.718152in}}%
\pgfpathlineto{\pgfqpoint{1.819745in}{0.718141in}}%
\pgfpathlineto{\pgfqpoint{1.820309in}{0.718130in}}%
\pgfpathlineto{\pgfqpoint{1.820873in}{0.718119in}}%
\pgfpathlineto{\pgfqpoint{1.821438in}{0.718108in}}%
\pgfpathlineto{\pgfqpoint{1.822002in}{0.718097in}}%
\pgfpathlineto{\pgfqpoint{1.822566in}{0.718086in}}%
\pgfpathlineto{\pgfqpoint{1.823131in}{0.718075in}}%
\pgfpathlineto{\pgfqpoint{1.823695in}{0.718064in}}%
\pgfpathlineto{\pgfqpoint{1.824259in}{0.718053in}}%
\pgfpathlineto{\pgfqpoint{1.824824in}{0.718042in}}%
\pgfpathlineto{\pgfqpoint{1.825388in}{0.718031in}}%
\pgfpathlineto{\pgfqpoint{1.825952in}{0.718020in}}%
\pgfpathlineto{\pgfqpoint{1.826517in}{0.718009in}}%
\pgfpathlineto{\pgfqpoint{1.827081in}{0.717998in}}%
\pgfpathlineto{\pgfqpoint{1.827645in}{0.717987in}}%
\pgfpathlineto{\pgfqpoint{1.828210in}{0.717976in}}%
\pgfpathlineto{\pgfqpoint{1.828774in}{0.717965in}}%
\pgfpathlineto{\pgfqpoint{1.829338in}{0.717954in}}%
\pgfpathlineto{\pgfqpoint{1.829903in}{0.717943in}}%
\pgfpathlineto{\pgfqpoint{1.830467in}{0.717932in}}%
\pgfpathlineto{\pgfqpoint{1.831031in}{0.717921in}}%
\pgfpathlineto{\pgfqpoint{1.831596in}{0.717910in}}%
\pgfpathlineto{\pgfqpoint{1.832160in}{0.717899in}}%
\pgfpathlineto{\pgfqpoint{1.832724in}{0.717888in}}%
\pgfpathlineto{\pgfqpoint{1.833289in}{0.717877in}}%
\pgfpathlineto{\pgfqpoint{1.833853in}{0.717866in}}%
\pgfpathlineto{\pgfqpoint{1.834418in}{0.717855in}}%
\pgfpathlineto{\pgfqpoint{1.834982in}{0.717844in}}%
\pgfpathlineto{\pgfqpoint{1.835546in}{0.717833in}}%
\pgfpathlineto{\pgfqpoint{1.836111in}{0.717822in}}%
\pgfpathlineto{\pgfqpoint{1.836675in}{0.717811in}}%
\pgfpathlineto{\pgfqpoint{1.837239in}{0.717800in}}%
\pgfpathlineto{\pgfqpoint{1.837804in}{0.717789in}}%
\pgfpathlineto{\pgfqpoint{1.838368in}{0.717778in}}%
\pgfpathlineto{\pgfqpoint{1.838932in}{0.717767in}}%
\pgfpathlineto{\pgfqpoint{1.839497in}{0.717756in}}%
\pgfpathlineto{\pgfqpoint{1.840061in}{0.717745in}}%
\pgfpathlineto{\pgfqpoint{1.840625in}{0.717734in}}%
\pgfpathlineto{\pgfqpoint{1.841190in}{0.717723in}}%
\pgfpathlineto{\pgfqpoint{1.841754in}{0.717712in}}%
\pgfpathlineto{\pgfqpoint{1.842318in}{0.717701in}}%
\pgfpathlineto{\pgfqpoint{1.842883in}{0.717690in}}%
\pgfpathlineto{\pgfqpoint{1.843447in}{0.717679in}}%
\pgfpathlineto{\pgfqpoint{1.844011in}{0.717668in}}%
\pgfpathlineto{\pgfqpoint{1.844576in}{0.717657in}}%
\pgfpathlineto{\pgfqpoint{1.845140in}{0.717646in}}%
\pgfpathlineto{\pgfqpoint{1.845704in}{0.717635in}}%
\pgfpathlineto{\pgfqpoint{1.846269in}{0.717624in}}%
\pgfpathlineto{\pgfqpoint{1.846833in}{0.717613in}}%
\pgfpathlineto{\pgfqpoint{1.847397in}{0.717602in}}%
\pgfpathlineto{\pgfqpoint{1.847962in}{0.717591in}}%
\pgfpathlineto{\pgfqpoint{1.848526in}{0.717580in}}%
\pgfpathlineto{\pgfqpoint{1.849090in}{0.717569in}}%
\pgfpathlineto{\pgfqpoint{1.849655in}{0.717558in}}%
\pgfpathlineto{\pgfqpoint{1.850219in}{0.717547in}}%
\pgfpathlineto{\pgfqpoint{1.850784in}{0.717536in}}%
\pgfpathlineto{\pgfqpoint{1.851348in}{0.717526in}}%
\pgfpathlineto{\pgfqpoint{1.851912in}{0.717515in}}%
\pgfpathlineto{\pgfqpoint{1.852477in}{0.717504in}}%
\pgfpathlineto{\pgfqpoint{1.853041in}{0.717493in}}%
\pgfpathlineto{\pgfqpoint{1.853605in}{0.717482in}}%
\pgfpathlineto{\pgfqpoint{1.854170in}{0.717471in}}%
\pgfpathlineto{\pgfqpoint{1.854734in}{0.717460in}}%
\pgfpathlineto{\pgfqpoint{1.855298in}{0.717449in}}%
\pgfpathlineto{\pgfqpoint{1.855863in}{0.717438in}}%
\pgfpathlineto{\pgfqpoint{1.856427in}{0.717427in}}%
\pgfpathlineto{\pgfqpoint{1.856991in}{0.717416in}}%
\pgfpathlineto{\pgfqpoint{1.857556in}{0.717405in}}%
\pgfpathlineto{\pgfqpoint{1.858120in}{0.717394in}}%
\pgfpathlineto{\pgfqpoint{1.858684in}{0.717383in}}%
\pgfpathlineto{\pgfqpoint{1.859249in}{0.717372in}}%
\pgfpathlineto{\pgfqpoint{1.859813in}{0.717361in}}%
\pgfpathlineto{\pgfqpoint{1.860377in}{0.717350in}}%
\pgfpathlineto{\pgfqpoint{1.860942in}{0.717339in}}%
\pgfpathlineto{\pgfqpoint{1.861506in}{0.717328in}}%
\pgfpathlineto{\pgfqpoint{1.862070in}{0.717317in}}%
\pgfpathlineto{\pgfqpoint{1.862635in}{0.717306in}}%
\pgfpathlineto{\pgfqpoint{1.863199in}{0.717295in}}%
\pgfpathlineto{\pgfqpoint{1.863763in}{0.717284in}}%
\pgfpathlineto{\pgfqpoint{1.864328in}{0.717273in}}%
\pgfpathlineto{\pgfqpoint{1.864892in}{0.717262in}}%
\pgfpathlineto{\pgfqpoint{1.865457in}{0.717251in}}%
\pgfpathlineto{\pgfqpoint{1.866021in}{0.717240in}}%
\pgfpathlineto{\pgfqpoint{1.866585in}{0.717229in}}%
\pgfpathlineto{\pgfqpoint{1.867150in}{0.717218in}}%
\pgfpathlineto{\pgfqpoint{1.867714in}{0.717207in}}%
\pgfpathlineto{\pgfqpoint{1.868278in}{0.717196in}}%
\pgfpathlineto{\pgfqpoint{1.868843in}{0.717185in}}%
\pgfpathlineto{\pgfqpoint{1.869407in}{0.717174in}}%
\pgfpathlineto{\pgfqpoint{1.869971in}{0.717163in}}%
\pgfpathlineto{\pgfqpoint{1.870536in}{0.717152in}}%
\pgfpathlineto{\pgfqpoint{1.871100in}{0.717141in}}%
\pgfpathlineto{\pgfqpoint{1.871664in}{0.717130in}}%
\pgfpathlineto{\pgfqpoint{1.872229in}{0.717119in}}%
\pgfpathlineto{\pgfqpoint{1.872793in}{0.717108in}}%
\pgfpathlineto{\pgfqpoint{1.873357in}{0.717097in}}%
\pgfpathlineto{\pgfqpoint{1.873922in}{0.717086in}}%
\pgfpathlineto{\pgfqpoint{1.874486in}{0.717075in}}%
\pgfpathlineto{\pgfqpoint{1.875050in}{0.717064in}}%
\pgfpathlineto{\pgfqpoint{1.875615in}{0.717053in}}%
\pgfpathlineto{\pgfqpoint{1.876179in}{0.717042in}}%
\pgfpathlineto{\pgfqpoint{1.876743in}{0.717031in}}%
\pgfpathlineto{\pgfqpoint{1.877308in}{0.717020in}}%
\pgfpathlineto{\pgfqpoint{1.877872in}{0.717009in}}%
\pgfpathlineto{\pgfqpoint{1.878436in}{0.716998in}}%
\pgfpathlineto{\pgfqpoint{1.879001in}{0.716987in}}%
\pgfpathlineto{\pgfqpoint{1.879565in}{0.716976in}}%
\pgfpathlineto{\pgfqpoint{1.880130in}{0.716965in}}%
\pgfpathlineto{\pgfqpoint{1.880694in}{0.716954in}}%
\pgfpathlineto{\pgfqpoint{1.881258in}{0.716943in}}%
\pgfpathlineto{\pgfqpoint{1.881823in}{0.716932in}}%
\pgfpathlineto{\pgfqpoint{1.882387in}{0.716921in}}%
\pgfpathlineto{\pgfqpoint{1.882951in}{0.716910in}}%
\pgfpathlineto{\pgfqpoint{1.883516in}{0.716899in}}%
\pgfpathlineto{\pgfqpoint{1.884080in}{0.716888in}}%
\pgfpathlineto{\pgfqpoint{1.884644in}{0.716877in}}%
\pgfpathlineto{\pgfqpoint{1.885209in}{0.716866in}}%
\pgfpathlineto{\pgfqpoint{1.885773in}{0.716855in}}%
\pgfpathlineto{\pgfqpoint{1.886337in}{0.716844in}}%
\pgfpathlineto{\pgfqpoint{1.886902in}{0.716833in}}%
\pgfpathlineto{\pgfqpoint{1.887466in}{0.716822in}}%
\pgfpathlineto{\pgfqpoint{1.888030in}{0.716811in}}%
\pgfpathlineto{\pgfqpoint{1.888595in}{0.716800in}}%
\pgfpathlineto{\pgfqpoint{1.889159in}{0.716789in}}%
\pgfpathlineto{\pgfqpoint{1.889723in}{0.716778in}}%
\pgfpathlineto{\pgfqpoint{1.890288in}{0.716767in}}%
\pgfpathlineto{\pgfqpoint{1.890852in}{0.716756in}}%
\pgfpathlineto{\pgfqpoint{1.891416in}{0.716745in}}%
\pgfpathlineto{\pgfqpoint{1.891981in}{0.716734in}}%
\pgfpathlineto{\pgfqpoint{1.892545in}{0.716723in}}%
\pgfpathlineto{\pgfqpoint{1.893109in}{0.716712in}}%
\pgfpathlineto{\pgfqpoint{1.893674in}{0.716701in}}%
\pgfpathlineto{\pgfqpoint{1.894238in}{0.716690in}}%
\pgfpathlineto{\pgfqpoint{1.894802in}{0.716679in}}%
\pgfpathlineto{\pgfqpoint{1.895367in}{0.716668in}}%
\pgfpathlineto{\pgfqpoint{1.895931in}{0.716657in}}%
\pgfpathlineto{\pgfqpoint{1.896496in}{0.716646in}}%
\pgfpathlineto{\pgfqpoint{1.897060in}{0.716635in}}%
\pgfpathlineto{\pgfqpoint{1.897624in}{0.716624in}}%
\pgfpathlineto{\pgfqpoint{1.898189in}{0.716613in}}%
\pgfpathlineto{\pgfqpoint{1.898753in}{0.716603in}}%
\pgfpathlineto{\pgfqpoint{1.899317in}{0.716615in}}%
\pgfpathlineto{\pgfqpoint{1.899882in}{0.716636in}}%
\pgfpathlineto{\pgfqpoint{1.900446in}{0.716658in}}%
\pgfpathlineto{\pgfqpoint{1.901010in}{0.716680in}}%
\pgfpathlineto{\pgfqpoint{1.901575in}{0.716702in}}%
\pgfpathlineto{\pgfqpoint{1.902139in}{0.716724in}}%
\pgfpathlineto{\pgfqpoint{1.902703in}{0.716746in}}%
\pgfpathlineto{\pgfqpoint{1.903268in}{0.716768in}}%
\pgfpathlineto{\pgfqpoint{1.903832in}{0.716790in}}%
\pgfpathlineto{\pgfqpoint{1.904396in}{0.716814in}}%
\pgfpathlineto{\pgfqpoint{1.904961in}{0.716838in}}%
\pgfpathlineto{\pgfqpoint{1.905525in}{0.716861in}}%
\pgfpathlineto{\pgfqpoint{1.906089in}{0.716885in}}%
\pgfpathlineto{\pgfqpoint{1.906654in}{0.716909in}}%
\pgfpathlineto{\pgfqpoint{1.907218in}{0.716932in}}%
\pgfpathlineto{\pgfqpoint{1.907782in}{0.716956in}}%
\pgfpathlineto{\pgfqpoint{1.908347in}{0.716979in}}%
\pgfpathlineto{\pgfqpoint{1.908911in}{0.717003in}}%
\pgfpathlineto{\pgfqpoint{1.909475in}{0.717027in}}%
\pgfpathlineto{\pgfqpoint{1.910040in}{0.717050in}}%
\pgfpathlineto{\pgfqpoint{1.910604in}{0.717074in}}%
\pgfpathlineto{\pgfqpoint{1.911169in}{0.717098in}}%
\pgfpathlineto{\pgfqpoint{1.911733in}{0.717121in}}%
\pgfpathlineto{\pgfqpoint{1.912297in}{0.717145in}}%
\pgfpathlineto{\pgfqpoint{1.912862in}{0.717169in}}%
\pgfpathlineto{\pgfqpoint{1.913426in}{0.717192in}}%
\pgfpathlineto{\pgfqpoint{1.913990in}{0.717216in}}%
\pgfpathlineto{\pgfqpoint{1.914555in}{0.717240in}}%
\pgfpathlineto{\pgfqpoint{1.915119in}{0.717263in}}%
\pgfpathlineto{\pgfqpoint{1.915683in}{0.717287in}}%
\pgfpathlineto{\pgfqpoint{1.916248in}{0.717311in}}%
\pgfpathlineto{\pgfqpoint{1.916812in}{0.717334in}}%
\pgfpathlineto{\pgfqpoint{1.917376in}{0.717358in}}%
\pgfpathlineto{\pgfqpoint{1.917941in}{0.717382in}}%
\pgfpathlineto{\pgfqpoint{1.918505in}{0.717405in}}%
\pgfpathlineto{\pgfqpoint{1.919069in}{0.717429in}}%
\pgfpathlineto{\pgfqpoint{1.919634in}{0.717453in}}%
\pgfpathlineto{\pgfqpoint{1.920198in}{0.717476in}}%
\pgfpathlineto{\pgfqpoint{1.920762in}{0.717500in}}%
\pgfpathlineto{\pgfqpoint{1.921327in}{0.717524in}}%
\pgfpathlineto{\pgfqpoint{1.921891in}{0.717547in}}%
\pgfpathlineto{\pgfqpoint{1.922455in}{0.717571in}}%
\pgfpathlineto{\pgfqpoint{1.923020in}{0.717595in}}%
\pgfpathlineto{\pgfqpoint{1.923584in}{0.717618in}}%
\pgfpathlineto{\pgfqpoint{1.924148in}{0.717642in}}%
\pgfpathlineto{\pgfqpoint{1.924713in}{0.717666in}}%
\pgfpathlineto{\pgfqpoint{1.925277in}{0.717689in}}%
\pgfpathlineto{\pgfqpoint{1.925842in}{0.717713in}}%
\pgfpathlineto{\pgfqpoint{1.926406in}{0.717737in}}%
\pgfpathlineto{\pgfqpoint{1.926970in}{0.717760in}}%
\pgfpathlineto{\pgfqpoint{1.927535in}{0.717784in}}%
\pgfpathlineto{\pgfqpoint{1.928099in}{0.717808in}}%
\pgfpathlineto{\pgfqpoint{1.928663in}{0.717831in}}%
\pgfpathlineto{\pgfqpoint{1.929228in}{0.717855in}}%
\pgfpathlineto{\pgfqpoint{1.929792in}{0.717879in}}%
\pgfpathlineto{\pgfqpoint{1.930356in}{0.717902in}}%
\pgfpathlineto{\pgfqpoint{1.930921in}{0.717926in}}%
\pgfpathlineto{\pgfqpoint{1.931485in}{0.717949in}}%
\pgfpathlineto{\pgfqpoint{1.932049in}{0.717973in}}%
\pgfpathlineto{\pgfqpoint{1.932614in}{0.717997in}}%
\pgfpathlineto{\pgfqpoint{1.933178in}{0.718020in}}%
\pgfpathlineto{\pgfqpoint{1.933742in}{0.718044in}}%
\pgfpathlineto{\pgfqpoint{1.934307in}{0.718068in}}%
\pgfpathlineto{\pgfqpoint{1.934871in}{0.718091in}}%
\pgfpathlineto{\pgfqpoint{1.935435in}{0.718115in}}%
\pgfpathlineto{\pgfqpoint{1.936000in}{0.718139in}}%
\pgfpathlineto{\pgfqpoint{1.936564in}{0.718138in}}%
\pgfpathlineto{\pgfqpoint{1.937128in}{0.717899in}}%
\pgfpathlineto{\pgfqpoint{1.937693in}{0.717942in}}%
\pgfpathlineto{\pgfqpoint{1.938257in}{0.717984in}}%
\pgfpathlineto{\pgfqpoint{1.938821in}{0.718027in}}%
\pgfpathlineto{\pgfqpoint{1.939386in}{0.718070in}}%
\pgfpathlineto{\pgfqpoint{1.939950in}{0.718112in}}%
\pgfpathlineto{\pgfqpoint{1.940514in}{0.718063in}}%
\pgfpathlineto{\pgfqpoint{1.941079in}{0.717543in}}%
\pgfpathlineto{\pgfqpoint{1.941643in}{0.716986in}}%
\pgfpathlineto{\pgfqpoint{1.942208in}{0.717990in}}%
\pgfpathlineto{\pgfqpoint{1.942772in}{0.718069in}}%
\pgfpathlineto{\pgfqpoint{1.943336in}{0.717976in}}%
\pgfpathlineto{\pgfqpoint{1.943901in}{0.717883in}}%
\pgfpathlineto{\pgfqpoint{1.944465in}{0.717790in}}%
\pgfpathlineto{\pgfqpoint{1.945029in}{0.717697in}}%
\pgfpathlineto{\pgfqpoint{1.945594in}{0.717603in}}%
\pgfpathlineto{\pgfqpoint{1.946158in}{0.717510in}}%
\pgfpathlineto{\pgfqpoint{1.946722in}{0.717417in}}%
\pgfpathlineto{\pgfqpoint{1.947287in}{0.717324in}}%
\pgfpathlineto{\pgfqpoint{1.947851in}{0.717231in}}%
\pgfpathlineto{\pgfqpoint{1.948415in}{0.717138in}}%
\pgfpathlineto{\pgfqpoint{1.948980in}{0.717044in}}%
\pgfpathlineto{\pgfqpoint{1.949544in}{0.716951in}}%
\pgfpathlineto{\pgfqpoint{1.950108in}{0.716858in}}%
\pgfpathlineto{\pgfqpoint{1.950673in}{0.716769in}}%
\pgfpathlineto{\pgfqpoint{1.951237in}{0.716811in}}%
\pgfpathlineto{\pgfqpoint{1.951801in}{0.716482in}}%
\pgfpathlineto{\pgfqpoint{1.952366in}{0.716735in}}%
\pgfpathlineto{\pgfqpoint{1.952930in}{0.716932in}}%
\pgfpathlineto{\pgfqpoint{1.953494in}{0.716714in}}%
\pgfpathlineto{\pgfqpoint{1.954059in}{0.717883in}}%
\pgfpathlineto{\pgfqpoint{1.954623in}{0.717659in}}%
\pgfpathlineto{\pgfqpoint{1.955187in}{0.717776in}}%
\pgfpathlineto{\pgfqpoint{1.955752in}{0.718970in}}%
\pgfpathlineto{\pgfqpoint{1.956316in}{0.718320in}}%
\pgfpathlineto{\pgfqpoint{1.956881in}{0.717811in}}%
\pgfpathlineto{\pgfqpoint{1.957445in}{0.717896in}}%
\pgfpathlineto{\pgfqpoint{1.958009in}{0.718009in}}%
\pgfpathlineto{\pgfqpoint{1.958574in}{0.718123in}}%
\pgfpathlineto{\pgfqpoint{1.959138in}{0.718236in}}%
\pgfpathlineto{\pgfqpoint{1.959702in}{0.718350in}}%
\pgfpathlineto{\pgfqpoint{1.960267in}{0.718463in}}%
\pgfpathlineto{\pgfqpoint{1.960831in}{0.718576in}}%
\pgfpathlineto{\pgfqpoint{1.961395in}{0.718690in}}%
\pgfpathlineto{\pgfqpoint{1.961960in}{0.718803in}}%
\pgfpathlineto{\pgfqpoint{1.962524in}{0.718916in}}%
\pgfpathlineto{\pgfqpoint{1.963088in}{0.719030in}}%
\pgfpathlineto{\pgfqpoint{1.963653in}{0.719090in}}%
\pgfpathlineto{\pgfqpoint{1.964217in}{0.718266in}}%
\pgfpathlineto{\pgfqpoint{1.964781in}{0.717251in}}%
\pgfpathlineto{\pgfqpoint{1.965346in}{0.719608in}}%
\pgfpathlineto{\pgfqpoint{1.965910in}{0.718921in}}%
\pgfpathlineto{\pgfqpoint{1.966474in}{0.718204in}}%
\pgfpathlineto{\pgfqpoint{1.967039in}{0.719142in}}%
\pgfpathlineto{\pgfqpoint{1.967603in}{0.717376in}}%
\pgfpathlineto{\pgfqpoint{1.968167in}{0.720385in}}%
\pgfpathlineto{\pgfqpoint{1.968732in}{0.717973in}}%
\pgfpathlineto{\pgfqpoint{1.969296in}{0.717968in}}%
\pgfpathlineto{\pgfqpoint{1.969860in}{0.718040in}}%
\pgfpathlineto{\pgfqpoint{1.970425in}{0.718013in}}%
\pgfpathlineto{\pgfqpoint{1.970989in}{0.717987in}}%
\pgfpathlineto{\pgfqpoint{1.971554in}{0.717960in}}%
\pgfpathlineto{\pgfqpoint{1.972118in}{0.717933in}}%
\pgfpathlineto{\pgfqpoint{1.972682in}{0.717906in}}%
\pgfpathlineto{\pgfqpoint{1.973247in}{0.717879in}}%
\pgfpathlineto{\pgfqpoint{1.973811in}{0.717852in}}%
\pgfpathlineto{\pgfqpoint{1.974375in}{0.717826in}}%
\pgfpathlineto{\pgfqpoint{1.974940in}{0.717799in}}%
\pgfpathlineto{\pgfqpoint{1.975504in}{0.717772in}}%
\pgfpathlineto{\pgfqpoint{1.976068in}{0.717745in}}%
\pgfpathlineto{\pgfqpoint{1.976633in}{0.717718in}}%
\pgfpathlineto{\pgfqpoint{1.977197in}{0.717691in}}%
\pgfpathlineto{\pgfqpoint{1.977761in}{0.717663in}}%
\pgfpathlineto{\pgfqpoint{1.978326in}{0.721175in}}%
\pgfpathlineto{\pgfqpoint{1.978890in}{0.724881in}}%
\pgfpathlineto{\pgfqpoint{1.979454in}{0.727071in}}%
\pgfpathlineto{\pgfqpoint{1.980019in}{0.726885in}}%
\pgfpathlineto{\pgfqpoint{1.980583in}{0.727355in}}%
\pgfpathlineto{\pgfqpoint{1.981147in}{0.726563in}}%
\pgfpathlineto{\pgfqpoint{1.981712in}{0.720971in}}%
\pgfpathlineto{\pgfqpoint{1.982276in}{0.723069in}}%
\pgfpathlineto{\pgfqpoint{1.982840in}{0.719818in}}%
\pgfpathlineto{\pgfqpoint{1.983405in}{0.719828in}}%
\pgfpathlineto{\pgfqpoint{1.983969in}{0.719677in}}%
\pgfpathlineto{\pgfqpoint{1.984533in}{0.719526in}}%
\pgfpathlineto{\pgfqpoint{1.985098in}{0.719376in}}%
\pgfpathlineto{\pgfqpoint{1.985662in}{0.719225in}}%
\pgfpathlineto{\pgfqpoint{1.986226in}{0.719074in}}%
\pgfpathlineto{\pgfqpoint{1.986791in}{0.718924in}}%
\pgfpathlineto{\pgfqpoint{1.987355in}{0.718773in}}%
\pgfpathlineto{\pgfqpoint{1.987920in}{0.718623in}}%
\pgfpathlineto{\pgfqpoint{1.988484in}{0.718472in}}%
\pgfpathlineto{\pgfqpoint{1.989048in}{0.718321in}}%
\pgfpathlineto{\pgfqpoint{1.989613in}{0.718171in}}%
\pgfpathlineto{\pgfqpoint{1.990177in}{0.718020in}}%
\pgfpathlineto{\pgfqpoint{1.990741in}{0.717870in}}%
\pgfpathlineto{\pgfqpoint{1.991306in}{0.717719in}}%
\pgfpathlineto{\pgfqpoint{1.991870in}{0.719230in}}%
\pgfpathlineto{\pgfqpoint{1.992434in}{0.725614in}}%
\pgfpathlineto{\pgfqpoint{1.992999in}{0.729127in}}%
\pgfpathlineto{\pgfqpoint{1.993563in}{0.725946in}}%
\pgfpathlineto{\pgfqpoint{1.994127in}{0.725610in}}%
\pgfpathlineto{\pgfqpoint{1.994692in}{0.724981in}}%
\pgfpathlineto{\pgfqpoint{1.995256in}{0.722802in}}%
\pgfpathlineto{\pgfqpoint{1.995820in}{0.722758in}}%
\pgfpathlineto{\pgfqpoint{1.996385in}{0.722715in}}%
\pgfpathlineto{\pgfqpoint{1.996949in}{0.722671in}}%
\pgfpathlineto{\pgfqpoint{1.997513in}{0.722628in}}%
\pgfpathlineto{\pgfqpoint{1.998078in}{0.722584in}}%
\pgfpathlineto{\pgfqpoint{1.998642in}{0.722541in}}%
\pgfpathlineto{\pgfqpoint{1.999206in}{0.722497in}}%
\pgfpathlineto{\pgfqpoint{1.999771in}{0.722454in}}%
\pgfpathlineto{\pgfqpoint{2.000335in}{0.722410in}}%
\pgfpathlineto{\pgfqpoint{2.000899in}{0.722367in}}%
\pgfpathlineto{\pgfqpoint{2.001464in}{0.722323in}}%
\pgfpathlineto{\pgfqpoint{2.002028in}{0.722280in}}%
\pgfpathlineto{\pgfqpoint{2.002593in}{0.722236in}}%
\pgfpathlineto{\pgfqpoint{2.003157in}{0.722193in}}%
\pgfpathlineto{\pgfqpoint{2.003721in}{0.722149in}}%
\pgfpathlineto{\pgfqpoint{2.004286in}{0.722106in}}%
\pgfpathlineto{\pgfqpoint{2.004850in}{0.722062in}}%
\pgfpathlineto{\pgfqpoint{2.005414in}{0.722019in}}%
\pgfpathlineto{\pgfqpoint{2.005979in}{0.721975in}}%
\pgfpathlineto{\pgfqpoint{2.006543in}{0.721932in}}%
\pgfpathlineto{\pgfqpoint{2.007107in}{0.721888in}}%
\pgfpathlineto{\pgfqpoint{2.007672in}{0.721845in}}%
\pgfpathlineto{\pgfqpoint{2.008236in}{0.721801in}}%
\pgfpathlineto{\pgfqpoint{2.008800in}{0.721758in}}%
\pgfpathlineto{\pgfqpoint{2.009365in}{0.721714in}}%
\pgfpathlineto{\pgfqpoint{2.009929in}{0.721671in}}%
\pgfpathlineto{\pgfqpoint{2.010493in}{0.721627in}}%
\pgfpathlineto{\pgfqpoint{2.011058in}{0.721584in}}%
\pgfpathlineto{\pgfqpoint{2.011622in}{0.721540in}}%
\pgfpathlineto{\pgfqpoint{2.012186in}{0.721497in}}%
\pgfpathlineto{\pgfqpoint{2.012751in}{0.721453in}}%
\pgfpathlineto{\pgfqpoint{2.013315in}{0.721410in}}%
\pgfpathlineto{\pgfqpoint{2.013879in}{0.721366in}}%
\pgfpathlineto{\pgfqpoint{2.014444in}{0.721323in}}%
\pgfpathlineto{\pgfqpoint{2.015008in}{0.721279in}}%
\pgfpathlineto{\pgfqpoint{2.015572in}{0.721236in}}%
\pgfpathlineto{\pgfqpoint{2.016137in}{0.721192in}}%
\pgfpathlineto{\pgfqpoint{2.016701in}{0.721149in}}%
\pgfpathlineto{\pgfqpoint{2.017266in}{0.721105in}}%
\pgfpathlineto{\pgfqpoint{2.017830in}{0.721062in}}%
\pgfpathlineto{\pgfqpoint{2.018394in}{0.721018in}}%
\pgfpathlineto{\pgfqpoint{2.018959in}{0.720975in}}%
\pgfpathlineto{\pgfqpoint{2.019523in}{0.720931in}}%
\pgfpathlineto{\pgfqpoint{2.020087in}{0.720888in}}%
\pgfpathlineto{\pgfqpoint{2.020652in}{0.720844in}}%
\pgfpathlineto{\pgfqpoint{2.021216in}{0.720801in}}%
\pgfpathlineto{\pgfqpoint{2.021780in}{0.720757in}}%
\pgfpathlineto{\pgfqpoint{2.022345in}{0.720714in}}%
\pgfpathlineto{\pgfqpoint{2.022909in}{0.720670in}}%
\pgfpathlineto{\pgfqpoint{2.023473in}{0.720627in}}%
\pgfpathlineto{\pgfqpoint{2.024038in}{0.720583in}}%
\pgfpathlineto{\pgfqpoint{2.024602in}{0.720540in}}%
\pgfpathlineto{\pgfqpoint{2.025166in}{0.720496in}}%
\pgfpathlineto{\pgfqpoint{2.025731in}{0.720453in}}%
\pgfpathlineto{\pgfqpoint{2.026295in}{0.720409in}}%
\pgfpathlineto{\pgfqpoint{2.026859in}{0.720366in}}%
\pgfpathlineto{\pgfqpoint{2.027424in}{0.720322in}}%
\pgfpathlineto{\pgfqpoint{2.027988in}{0.720279in}}%
\pgfpathlineto{\pgfqpoint{2.028552in}{0.720235in}}%
\pgfpathlineto{\pgfqpoint{2.029117in}{0.720192in}}%
\pgfpathlineto{\pgfqpoint{2.029681in}{0.720148in}}%
\pgfpathlineto{\pgfqpoint{2.030245in}{0.720105in}}%
\pgfpathlineto{\pgfqpoint{2.030810in}{0.720091in}}%
\pgfpathlineto{\pgfqpoint{2.031374in}{0.723154in}}%
\pgfpathlineto{\pgfqpoint{2.031938in}{0.728265in}}%
\pgfpathlineto{\pgfqpoint{2.032503in}{0.729051in}}%
\pgfpathlineto{\pgfqpoint{2.033067in}{0.731288in}}%
\pgfpathlineto{\pgfqpoint{2.033632in}{0.722683in}}%
\pgfpathlineto{\pgfqpoint{2.034196in}{0.724484in}}%
\pgfpathlineto{\pgfqpoint{2.034760in}{0.722493in}}%
\pgfpathlineto{\pgfqpoint{2.035325in}{0.722740in}}%
\pgfpathlineto{\pgfqpoint{2.035889in}{0.723763in}}%
\pgfpathlineto{\pgfqpoint{2.036453in}{0.721624in}}%
\pgfpathlineto{\pgfqpoint{2.037018in}{0.721882in}}%
\pgfpathlineto{\pgfqpoint{2.037582in}{0.722706in}}%
\pgfpathlineto{\pgfqpoint{2.038146in}{0.723530in}}%
\pgfpathlineto{\pgfqpoint{2.038711in}{0.724354in}}%
\pgfpathlineto{\pgfqpoint{2.039275in}{0.725178in}}%
\pgfpathlineto{\pgfqpoint{2.039839in}{0.726002in}}%
\pgfpathlineto{\pgfqpoint{2.040404in}{0.726826in}}%
\pgfpathlineto{\pgfqpoint{2.040968in}{0.727650in}}%
\pgfpathlineto{\pgfqpoint{2.041532in}{0.728474in}}%
\pgfpathlineto{\pgfqpoint{2.042097in}{0.729298in}}%
\pgfpathlineto{\pgfqpoint{2.042661in}{0.730122in}}%
\pgfpathlineto{\pgfqpoint{2.043225in}{0.730946in}}%
\pgfpathlineto{\pgfqpoint{2.043790in}{0.731770in}}%
\pgfpathlineto{\pgfqpoint{2.044354in}{0.732594in}}%
\pgfpathlineto{\pgfqpoint{2.044918in}{0.733418in}}%
\pgfpathlineto{\pgfqpoint{2.045483in}{0.727964in}}%
\pgfpathlineto{\pgfqpoint{2.046047in}{0.723267in}}%
\pgfpathlineto{\pgfqpoint{2.046611in}{0.721549in}}%
\pgfpathlineto{\pgfqpoint{2.047176in}{0.719159in}}%
\pgfpathlineto{\pgfqpoint{2.047740in}{0.723268in}}%
\pgfpathlineto{\pgfqpoint{2.048305in}{0.717479in}}%
\pgfpathlineto{\pgfqpoint{2.048869in}{0.721960in}}%
\pgfpathlineto{\pgfqpoint{2.049433in}{0.730792in}}%
\pgfpathlineto{\pgfqpoint{2.049998in}{0.736825in}}%
\pgfpathlineto{\pgfqpoint{2.050562in}{0.737119in}}%
\pgfpathlineto{\pgfqpoint{2.051126in}{0.737113in}}%
\pgfpathlineto{\pgfqpoint{2.051691in}{0.737108in}}%
\pgfpathlineto{\pgfqpoint{2.052255in}{0.737103in}}%
\pgfpathlineto{\pgfqpoint{2.052819in}{0.737098in}}%
\pgfpathlineto{\pgfqpoint{2.053384in}{0.737092in}}%
\pgfpathlineto{\pgfqpoint{2.053948in}{0.737087in}}%
\pgfpathlineto{\pgfqpoint{2.054512in}{0.737082in}}%
\pgfpathlineto{\pgfqpoint{2.055077in}{0.737077in}}%
\pgfpathlineto{\pgfqpoint{2.055641in}{0.737071in}}%
\pgfpathlineto{\pgfqpoint{2.056205in}{0.737066in}}%
\pgfpathlineto{\pgfqpoint{2.056770in}{0.737061in}}%
\pgfpathlineto{\pgfqpoint{2.057334in}{0.737056in}}%
\pgfpathlineto{\pgfqpoint{2.057898in}{0.737050in}}%
\pgfpathlineto{\pgfqpoint{2.058463in}{0.731788in}}%
\pgfpathlineto{\pgfqpoint{2.059027in}{0.721406in}}%
\pgfpathlineto{\pgfqpoint{2.059591in}{0.722416in}}%
\pgfpathlineto{\pgfqpoint{2.060156in}{0.727467in}}%
\pgfpathlineto{\pgfqpoint{2.060720in}{0.726509in}}%
\pgfpathlineto{\pgfqpoint{2.061284in}{0.730582in}}%
\pgfpathlineto{\pgfqpoint{2.061849in}{0.725651in}}%
\pgfpathlineto{\pgfqpoint{2.062413in}{0.739317in}}%
\pgfpathlineto{\pgfqpoint{2.062978in}{0.724160in}}%
\pgfpathlineto{\pgfqpoint{2.063542in}{0.726698in}}%
\pgfpathlineto{\pgfqpoint{2.064106in}{0.744929in}}%
\pgfpathlineto{\pgfqpoint{2.064671in}{0.750240in}}%
\pgfpathlineto{\pgfqpoint{2.065235in}{0.749496in}}%
\pgfpathlineto{\pgfqpoint{2.065799in}{0.748592in}}%
\pgfpathlineto{\pgfqpoint{2.066364in}{0.746761in}}%
\pgfpathlineto{\pgfqpoint{2.066928in}{0.744781in}}%
\pgfpathlineto{\pgfqpoint{2.067492in}{0.743297in}}%
\pgfpathlineto{\pgfqpoint{2.068057in}{0.743062in}}%
\pgfpathlineto{\pgfqpoint{2.068621in}{0.742887in}}%
\pgfpathlineto{\pgfqpoint{2.069185in}{0.742712in}}%
\pgfpathlineto{\pgfqpoint{2.069750in}{0.742537in}}%
\pgfpathlineto{\pgfqpoint{2.070314in}{0.742362in}}%
\pgfpathlineto{\pgfqpoint{2.070878in}{0.742187in}}%
\pgfpathlineto{\pgfqpoint{2.071443in}{0.742012in}}%
\pgfpathlineto{\pgfqpoint{2.072007in}{0.741837in}}%
\pgfpathlineto{\pgfqpoint{2.072571in}{0.739898in}}%
\pgfpathlineto{\pgfqpoint{2.073136in}{0.722739in}}%
\pgfpathlineto{\pgfqpoint{2.073700in}{0.727524in}}%
\pgfpathlineto{\pgfqpoint{2.074264in}{0.740196in}}%
\pgfpathlineto{\pgfqpoint{2.074829in}{0.742941in}}%
\pgfpathlineto{\pgfqpoint{2.075393in}{0.732832in}}%
\pgfpathlineto{\pgfqpoint{2.075957in}{0.718011in}}%
\pgfpathlineto{\pgfqpoint{2.076522in}{0.721496in}}%
\pgfpathlineto{\pgfqpoint{2.077086in}{0.726207in}}%
\pgfpathlineto{\pgfqpoint{2.077651in}{0.724811in}}%
\pgfpathlineto{\pgfqpoint{2.078215in}{0.724702in}}%
\pgfpathlineto{\pgfqpoint{2.078779in}{0.724593in}}%
\pgfpathlineto{\pgfqpoint{2.079344in}{0.724485in}}%
\pgfpathlineto{\pgfqpoint{2.079908in}{0.724376in}}%
\pgfpathlineto{\pgfqpoint{2.080472in}{0.724267in}}%
\pgfpathlineto{\pgfqpoint{2.081037in}{0.724158in}}%
\pgfpathlineto{\pgfqpoint{2.081601in}{0.724050in}}%
\pgfpathlineto{\pgfqpoint{2.082165in}{0.723941in}}%
\pgfpathlineto{\pgfqpoint{2.082730in}{0.723832in}}%
\pgfpathlineto{\pgfqpoint{2.083294in}{0.723724in}}%
\pgfpathlineto{\pgfqpoint{2.083858in}{0.723615in}}%
\pgfpathlineto{\pgfqpoint{2.084423in}{0.723506in}}%
\pgfpathlineto{\pgfqpoint{2.084987in}{0.723398in}}%
\pgfpathlineto{\pgfqpoint{2.085551in}{0.723316in}}%
\pgfpathlineto{\pgfqpoint{2.086116in}{0.730869in}}%
\pgfpathlineto{\pgfqpoint{2.086680in}{0.729682in}}%
\pgfpathlineto{\pgfqpoint{2.087244in}{0.731311in}}%
\pgfpathlineto{\pgfqpoint{2.087809in}{0.738146in}}%
\pgfpathlineto{\pgfqpoint{2.088373in}{0.744920in}}%
\pgfpathlineto{\pgfqpoint{2.088937in}{0.746512in}}%
\pgfpathlineto{\pgfqpoint{2.089502in}{0.746737in}}%
\pgfpathlineto{\pgfqpoint{2.090066in}{0.750393in}}%
\pgfpathlineto{\pgfqpoint{2.090630in}{0.749476in}}%
\pgfpathlineto{\pgfqpoint{2.091195in}{0.748972in}}%
\pgfpathlineto{\pgfqpoint{2.091759in}{0.748496in}}%
\pgfpathlineto{\pgfqpoint{2.092323in}{0.748020in}}%
\pgfpathlineto{\pgfqpoint{2.092888in}{0.747544in}}%
\pgfpathlineto{\pgfqpoint{2.093452in}{0.747148in}}%
\pgfpathlineto{\pgfqpoint{2.094017in}{0.747049in}}%
\pgfpathlineto{\pgfqpoint{2.094581in}{0.746980in}}%
\pgfpathlineto{\pgfqpoint{2.095145in}{0.746910in}}%
\pgfpathlineto{\pgfqpoint{2.095710in}{0.746840in}}%
\pgfpathlineto{\pgfqpoint{2.096274in}{0.746770in}}%
\pgfpathlineto{\pgfqpoint{2.096838in}{0.746701in}}%
\pgfpathlineto{\pgfqpoint{2.097403in}{0.746631in}}%
\pgfpathlineto{\pgfqpoint{2.097967in}{0.746561in}}%
\pgfpathlineto{\pgfqpoint{2.098531in}{0.746491in}}%
\pgfpathlineto{\pgfqpoint{2.099096in}{0.746421in}}%
\pgfpathlineto{\pgfqpoint{2.099660in}{0.746352in}}%
\pgfpathlineto{\pgfqpoint{2.100224in}{0.746282in}}%
\pgfpathlineto{\pgfqpoint{2.100789in}{0.746212in}}%
\pgfpathlineto{\pgfqpoint{2.101353in}{0.746142in}}%
\pgfpathlineto{\pgfqpoint{2.101917in}{0.746073in}}%
\pgfpathlineto{\pgfqpoint{2.102482in}{0.746003in}}%
\pgfpathlineto{\pgfqpoint{2.103046in}{0.745933in}}%
\pgfpathlineto{\pgfqpoint{2.103610in}{0.745863in}}%
\pgfpathlineto{\pgfqpoint{2.104175in}{0.745794in}}%
\pgfpathlineto{\pgfqpoint{2.104739in}{0.745724in}}%
\pgfpathlineto{\pgfqpoint{2.105303in}{0.745654in}}%
\pgfpathlineto{\pgfqpoint{2.105868in}{0.745584in}}%
\pgfpathlineto{\pgfqpoint{2.106432in}{0.745515in}}%
\pgfpathlineto{\pgfqpoint{2.106996in}{0.745445in}}%
\pgfpathlineto{\pgfqpoint{2.107561in}{0.745375in}}%
\pgfpathlineto{\pgfqpoint{2.108125in}{0.745305in}}%
\pgfpathlineto{\pgfqpoint{2.108690in}{0.745236in}}%
\pgfpathlineto{\pgfqpoint{2.109254in}{0.745166in}}%
\pgfpathlineto{\pgfqpoint{2.109818in}{0.745096in}}%
\pgfpathlineto{\pgfqpoint{2.110383in}{0.745026in}}%
\pgfpathlineto{\pgfqpoint{2.110947in}{0.744957in}}%
\pgfpathlineto{\pgfqpoint{2.111511in}{0.744887in}}%
\pgfpathlineto{\pgfqpoint{2.112076in}{0.744817in}}%
\pgfpathlineto{\pgfqpoint{2.112640in}{0.744747in}}%
\pgfpathlineto{\pgfqpoint{2.113204in}{0.744678in}}%
\pgfpathlineto{\pgfqpoint{2.113769in}{0.744608in}}%
\pgfpathlineto{\pgfqpoint{2.114333in}{0.744538in}}%
\pgfpathlineto{\pgfqpoint{2.114897in}{0.744468in}}%
\pgfpathlineto{\pgfqpoint{2.115462in}{0.744398in}}%
\pgfpathlineto{\pgfqpoint{2.116026in}{0.744329in}}%
\pgfpathlineto{\pgfqpoint{2.116590in}{0.744259in}}%
\pgfpathlineto{\pgfqpoint{2.117155in}{0.744189in}}%
\pgfpathlineto{\pgfqpoint{2.117719in}{0.744119in}}%
\pgfpathlineto{\pgfqpoint{2.118283in}{0.744050in}}%
\pgfpathlineto{\pgfqpoint{2.118848in}{0.743980in}}%
\pgfpathlineto{\pgfqpoint{2.119412in}{0.743910in}}%
\pgfpathlineto{\pgfqpoint{2.119976in}{0.743840in}}%
\pgfpathlineto{\pgfqpoint{2.120541in}{0.743771in}}%
\pgfpathlineto{\pgfqpoint{2.121105in}{0.743701in}}%
\pgfpathlineto{\pgfqpoint{2.121669in}{0.743631in}}%
\pgfpathlineto{\pgfqpoint{2.122234in}{0.743561in}}%
\pgfpathlineto{\pgfqpoint{2.122798in}{0.743492in}}%
\pgfpathlineto{\pgfqpoint{2.123363in}{0.743422in}}%
\pgfpathlineto{\pgfqpoint{2.123927in}{0.743352in}}%
\pgfpathlineto{\pgfqpoint{2.124491in}{0.743282in}}%
\pgfpathlineto{\pgfqpoint{2.125056in}{0.743213in}}%
\pgfpathlineto{\pgfqpoint{2.125620in}{0.743143in}}%
\pgfpathlineto{\pgfqpoint{2.126184in}{0.743073in}}%
\pgfpathlineto{\pgfqpoint{2.126749in}{0.743003in}}%
\pgfpathlineto{\pgfqpoint{2.127313in}{0.742930in}}%
\pgfpathlineto{\pgfqpoint{2.127877in}{0.742574in}}%
\pgfpathlineto{\pgfqpoint{2.128442in}{0.742960in}}%
\pgfpathlineto{\pgfqpoint{2.129006in}{0.743171in}}%
\pgfpathlineto{\pgfqpoint{2.129570in}{0.743677in}}%
\pgfpathlineto{\pgfqpoint{2.130135in}{0.742907in}}%
\pgfpathlineto{\pgfqpoint{2.130699in}{0.747161in}}%
\pgfpathlineto{\pgfqpoint{2.131263in}{0.743050in}}%
\pgfpathlineto{\pgfqpoint{2.131828in}{0.742402in}}%
\pgfpathlineto{\pgfqpoint{2.132392in}{0.742373in}}%
\pgfpathlineto{\pgfqpoint{2.132956in}{0.742345in}}%
\pgfpathlineto{\pgfqpoint{2.133521in}{0.742316in}}%
\pgfpathlineto{\pgfqpoint{2.134085in}{0.742288in}}%
\pgfpathlineto{\pgfqpoint{2.134649in}{0.742260in}}%
\pgfpathlineto{\pgfqpoint{2.135214in}{0.742231in}}%
\pgfpathlineto{\pgfqpoint{2.135778in}{0.742203in}}%
\pgfpathlineto{\pgfqpoint{2.136342in}{0.742174in}}%
\pgfpathlineto{\pgfqpoint{2.136907in}{0.742146in}}%
\pgfpathlineto{\pgfqpoint{2.137471in}{0.742117in}}%
\pgfpathlineto{\pgfqpoint{2.138035in}{0.742089in}}%
\pgfpathlineto{\pgfqpoint{2.138600in}{0.742061in}}%
\pgfpathlineto{\pgfqpoint{2.139164in}{0.742032in}}%
\pgfpathlineto{\pgfqpoint{2.139729in}{0.742004in}}%
\pgfpathlineto{\pgfqpoint{2.140293in}{0.742160in}}%
\pgfpathlineto{\pgfqpoint{2.140857in}{0.743146in}}%
\pgfpathlineto{\pgfqpoint{2.141422in}{0.727381in}}%
\pgfpathlineto{\pgfqpoint{2.141986in}{0.736098in}}%
\pgfpathlineto{\pgfqpoint{2.142550in}{0.748436in}}%
\pgfpathlineto{\pgfqpoint{2.143115in}{0.749468in}}%
\pgfpathlineto{\pgfqpoint{2.143679in}{0.750505in}}%
\pgfpathlineto{\pgfqpoint{2.144243in}{0.754122in}}%
\pgfpathlineto{\pgfqpoint{2.144808in}{0.754093in}}%
\pgfpathlineto{\pgfqpoint{2.145372in}{0.754077in}}%
\pgfpathlineto{\pgfqpoint{2.145936in}{0.754060in}}%
\pgfpathlineto{\pgfqpoint{2.146501in}{0.754044in}}%
\pgfpathlineto{\pgfqpoint{2.147065in}{0.754027in}}%
\pgfpathlineto{\pgfqpoint{2.147629in}{0.754011in}}%
\pgfpathlineto{\pgfqpoint{2.148194in}{0.753995in}}%
\pgfpathlineto{\pgfqpoint{2.148758in}{0.753978in}}%
\pgfpathlineto{\pgfqpoint{2.149322in}{0.753962in}}%
\pgfpathlineto{\pgfqpoint{2.149887in}{0.753945in}}%
\pgfpathlineto{\pgfqpoint{2.150451in}{0.753929in}}%
\pgfpathlineto{\pgfqpoint{2.151015in}{0.753913in}}%
\pgfpathlineto{\pgfqpoint{2.151580in}{0.753896in}}%
\pgfpathlineto{\pgfqpoint{2.152144in}{0.753880in}}%
\pgfpathlineto{\pgfqpoint{2.152708in}{0.753863in}}%
\pgfpathlineto{\pgfqpoint{2.153273in}{0.753847in}}%
\pgfpathlineto{\pgfqpoint{2.153837in}{0.753880in}}%
\pgfpathlineto{\pgfqpoint{2.154402in}{0.754768in}}%
\pgfpathlineto{\pgfqpoint{2.154966in}{0.759512in}}%
\pgfpathlineto{\pgfqpoint{2.155530in}{0.759095in}}%
\pgfpathlineto{\pgfqpoint{2.156095in}{0.763459in}}%
\pgfpathlineto{\pgfqpoint{2.156659in}{0.765226in}}%
\pgfpathlineto{\pgfqpoint{2.157223in}{0.769310in}}%
\pgfpathlineto{\pgfqpoint{2.157788in}{0.770241in}}%
\pgfpathlineto{\pgfqpoint{2.158352in}{0.770077in}}%
\pgfpathlineto{\pgfqpoint{2.158916in}{0.769925in}}%
\pgfpathlineto{\pgfqpoint{2.159481in}{0.769774in}}%
\pgfpathlineto{\pgfqpoint{2.160045in}{0.769622in}}%
\pgfpathlineto{\pgfqpoint{2.160609in}{0.769470in}}%
\pgfpathlineto{\pgfqpoint{2.161174in}{0.769319in}}%
\pgfpathlineto{\pgfqpoint{2.161738in}{0.769167in}}%
\pgfpathlineto{\pgfqpoint{2.162302in}{0.769015in}}%
\pgfpathlineto{\pgfqpoint{2.162867in}{0.768864in}}%
\pgfpathlineto{\pgfqpoint{2.163431in}{0.768712in}}%
\pgfpathlineto{\pgfqpoint{2.163995in}{0.768561in}}%
\pgfpathlineto{\pgfqpoint{2.164560in}{0.768409in}}%
\pgfpathlineto{\pgfqpoint{2.165124in}{0.768257in}}%
\pgfpathlineto{\pgfqpoint{2.165688in}{0.768106in}}%
\pgfpathlineto{\pgfqpoint{2.166253in}{0.767954in}}%
\pgfpathlineto{\pgfqpoint{2.166817in}{0.767802in}}%
\pgfpathlineto{\pgfqpoint{2.167381in}{0.767566in}}%
\pgfpathlineto{\pgfqpoint{2.167946in}{0.767401in}}%
\pgfpathlineto{\pgfqpoint{2.168510in}{0.768520in}}%
\pgfpathlineto{\pgfqpoint{2.169075in}{0.767453in}}%
\pgfpathlineto{\pgfqpoint{2.169639in}{0.767904in}}%
\pgfpathlineto{\pgfqpoint{2.170203in}{0.768395in}}%
\pgfpathlineto{\pgfqpoint{2.170768in}{0.770185in}}%
\pgfpathlineto{\pgfqpoint{2.171332in}{0.770128in}}%
\pgfpathlineto{\pgfqpoint{2.171896in}{0.770083in}}%
\pgfpathlineto{\pgfqpoint{2.172461in}{0.770040in}}%
\pgfpathlineto{\pgfqpoint{2.173025in}{0.769998in}}%
\pgfpathlineto{\pgfqpoint{2.173589in}{0.769955in}}%
\pgfpathlineto{\pgfqpoint{2.174154in}{0.761885in}}%
\pgfpathlineto{\pgfqpoint{2.174718in}{0.760098in}}%
\pgfpathlineto{\pgfqpoint{2.175282in}{0.759702in}}%
\pgfpathlineto{\pgfqpoint{2.175847in}{0.759306in}}%
\pgfpathlineto{\pgfqpoint{2.176411in}{0.758910in}}%
\pgfpathlineto{\pgfqpoint{2.176975in}{0.758514in}}%
\pgfpathlineto{\pgfqpoint{2.177540in}{0.758117in}}%
\pgfpathlineto{\pgfqpoint{2.178104in}{0.757721in}}%
\pgfpathlineto{\pgfqpoint{2.178668in}{0.757325in}}%
\pgfpathlineto{\pgfqpoint{2.179233in}{0.756929in}}%
\pgfpathlineto{\pgfqpoint{2.179797in}{0.756533in}}%
\pgfpathlineto{\pgfqpoint{2.180361in}{0.756136in}}%
\pgfpathlineto{\pgfqpoint{2.180926in}{0.754754in}}%
\pgfpathlineto{\pgfqpoint{2.181490in}{0.753706in}}%
\pgfpathlineto{\pgfqpoint{2.182054in}{0.754304in}}%
\pgfpathlineto{\pgfqpoint{2.182619in}{0.754413in}}%
\pgfpathlineto{\pgfqpoint{2.183183in}{0.755385in}}%
\pgfpathlineto{\pgfqpoint{2.183747in}{0.755302in}}%
\pgfpathlineto{\pgfqpoint{2.184312in}{0.754908in}}%
\pgfpathlineto{\pgfqpoint{2.184876in}{0.754513in}}%
\pgfpathlineto{\pgfqpoint{2.185441in}{0.754119in}}%
\pgfpathlineto{\pgfqpoint{2.186005in}{0.753724in}}%
\pgfpathlineto{\pgfqpoint{2.186569in}{0.753343in}}%
\pgfpathlineto{\pgfqpoint{2.187134in}{0.753210in}}%
\pgfpathlineto{\pgfqpoint{2.187698in}{0.753170in}}%
\pgfpathlineto{\pgfqpoint{2.188262in}{0.753130in}}%
\pgfpathlineto{\pgfqpoint{2.188827in}{0.753090in}}%
\pgfpathlineto{\pgfqpoint{2.189391in}{0.753050in}}%
\pgfpathlineto{\pgfqpoint{2.189955in}{0.753010in}}%
\pgfpathlineto{\pgfqpoint{2.190520in}{0.752970in}}%
\pgfpathlineto{\pgfqpoint{2.191084in}{0.752931in}}%
\pgfpathlineto{\pgfqpoint{2.191648in}{0.752891in}}%
\pgfpathlineto{\pgfqpoint{2.192213in}{0.752851in}}%
\pgfpathlineto{\pgfqpoint{2.192777in}{0.752811in}}%
\pgfpathlineto{\pgfqpoint{2.193341in}{0.752771in}}%
\pgfpathlineto{\pgfqpoint{2.193906in}{0.752731in}}%
\pgfpathlineto{\pgfqpoint{2.194470in}{0.752691in}}%
\pgfpathlineto{\pgfqpoint{2.195034in}{0.752651in}}%
\pgfpathlineto{\pgfqpoint{2.195599in}{0.752611in}}%
\pgfpathlineto{\pgfqpoint{2.196163in}{0.752571in}}%
\pgfpathlineto{\pgfqpoint{2.196727in}{0.752531in}}%
\pgfpathlineto{\pgfqpoint{2.197292in}{0.752491in}}%
\pgfpathlineto{\pgfqpoint{2.197856in}{0.752451in}}%
\pgfpathlineto{\pgfqpoint{2.198420in}{0.752411in}}%
\pgfpathlineto{\pgfqpoint{2.198985in}{0.752371in}}%
\pgfpathlineto{\pgfqpoint{2.199549in}{0.752332in}}%
\pgfpathlineto{\pgfqpoint{2.200114in}{0.752292in}}%
\pgfpathlineto{\pgfqpoint{2.200678in}{0.752252in}}%
\pgfpathlineto{\pgfqpoint{2.201242in}{0.752212in}}%
\pgfpathlineto{\pgfqpoint{2.201807in}{0.752172in}}%
\pgfpathlineto{\pgfqpoint{2.202371in}{0.752132in}}%
\pgfpathlineto{\pgfqpoint{2.202935in}{0.752092in}}%
\pgfpathlineto{\pgfqpoint{2.203500in}{0.752052in}}%
\pgfpathlineto{\pgfqpoint{2.204064in}{0.752012in}}%
\pgfpathlineto{\pgfqpoint{2.204628in}{0.751972in}}%
\pgfpathlineto{\pgfqpoint{2.205193in}{0.751932in}}%
\pgfpathlineto{\pgfqpoint{2.205757in}{0.751892in}}%
\pgfpathlineto{\pgfqpoint{2.206321in}{0.751852in}}%
\pgfpathlineto{\pgfqpoint{2.206886in}{0.751812in}}%
\pgfpathlineto{\pgfqpoint{2.207450in}{0.751773in}}%
\pgfpathlineto{\pgfqpoint{2.208014in}{0.751733in}}%
\pgfpathlineto{\pgfqpoint{2.208579in}{0.751693in}}%
\pgfpathlineto{\pgfqpoint{2.209143in}{0.751653in}}%
\pgfpathlineto{\pgfqpoint{2.209707in}{0.751613in}}%
\pgfpathlineto{\pgfqpoint{2.210272in}{0.751573in}}%
\pgfpathlineto{\pgfqpoint{2.210836in}{0.751533in}}%
\pgfpathlineto{\pgfqpoint{2.211400in}{0.751493in}}%
\pgfpathlineto{\pgfqpoint{2.211965in}{0.751453in}}%
\pgfpathlineto{\pgfqpoint{2.212529in}{0.751413in}}%
\pgfpathlineto{\pgfqpoint{2.213093in}{0.751373in}}%
\pgfpathlineto{\pgfqpoint{2.213658in}{0.751333in}}%
\pgfpathlineto{\pgfqpoint{2.214222in}{0.751293in}}%
\pgfpathlineto{\pgfqpoint{2.214787in}{0.751253in}}%
\pgfpathlineto{\pgfqpoint{2.215351in}{0.751213in}}%
\pgfpathlineto{\pgfqpoint{2.215915in}{0.751174in}}%
\pgfpathlineto{\pgfqpoint{2.216480in}{0.751134in}}%
\pgfpathlineto{\pgfqpoint{2.217044in}{0.751094in}}%
\pgfpathlineto{\pgfqpoint{2.217608in}{0.751054in}}%
\pgfpathlineto{\pgfqpoint{2.218173in}{0.751014in}}%
\pgfpathlineto{\pgfqpoint{2.218737in}{0.750974in}}%
\pgfpathlineto{\pgfqpoint{2.219301in}{0.750934in}}%
\pgfpathlineto{\pgfqpoint{2.219866in}{0.750894in}}%
\pgfpathlineto{\pgfqpoint{2.220430in}{0.750854in}}%
\pgfpathlineto{\pgfqpoint{2.220994in}{0.750814in}}%
\pgfpathlineto{\pgfqpoint{2.221559in}{0.750774in}}%
\pgfpathlineto{\pgfqpoint{2.222123in}{0.750734in}}%
\pgfpathlineto{\pgfqpoint{2.222687in}{0.750694in}}%
\pgfpathlineto{\pgfqpoint{2.223252in}{0.750654in}}%
\pgfpathlineto{\pgfqpoint{2.223816in}{0.750614in}}%
\pgfpathlineto{\pgfqpoint{2.224380in}{0.749749in}}%
\pgfpathlineto{\pgfqpoint{2.224945in}{0.748653in}}%
\pgfpathlineto{\pgfqpoint{2.225509in}{0.749253in}}%
\pgfpathlineto{\pgfqpoint{2.226073in}{0.749487in}}%
\pgfpathlineto{\pgfqpoint{2.226638in}{0.749464in}}%
\pgfpathlineto{\pgfqpoint{2.227202in}{0.749424in}}%
\pgfpathlineto{\pgfqpoint{2.227766in}{0.749383in}}%
\pgfpathlineto{\pgfqpoint{2.228331in}{0.749343in}}%
\pgfpathlineto{\pgfqpoint{2.228895in}{0.749303in}}%
\pgfpathlineto{\pgfqpoint{2.229459in}{0.749263in}}%
\pgfpathlineto{\pgfqpoint{2.230024in}{0.749223in}}%
\pgfpathlineto{\pgfqpoint{2.230588in}{0.749183in}}%
\pgfpathlineto{\pgfqpoint{2.231153in}{0.749143in}}%
\pgfpathlineto{\pgfqpoint{2.231717in}{0.749103in}}%
\pgfpathlineto{\pgfqpoint{2.232281in}{0.749063in}}%
\pgfpathlineto{\pgfqpoint{2.232846in}{0.749023in}}%
\pgfpathlineto{\pgfqpoint{2.233410in}{0.748982in}}%
\pgfpathlineto{\pgfqpoint{2.233974in}{0.748942in}}%
\pgfpathlineto{\pgfqpoint{2.234539in}{0.748902in}}%
\pgfpathlineto{\pgfqpoint{2.235103in}{0.748862in}}%
\pgfpathlineto{\pgfqpoint{2.235667in}{0.748822in}}%
\pgfpathlineto{\pgfqpoint{2.236232in}{0.748748in}}%
\pgfpathlineto{\pgfqpoint{2.236796in}{0.744771in}}%
\pgfpathlineto{\pgfqpoint{2.237360in}{0.745725in}}%
\pgfpathlineto{\pgfqpoint{2.237925in}{0.747218in}}%
\pgfpathlineto{\pgfqpoint{2.238489in}{0.748501in}}%
\pgfpathlineto{\pgfqpoint{2.239053in}{0.750173in}}%
\pgfpathlineto{\pgfqpoint{2.239618in}{0.750440in}}%
\pgfpathlineto{\pgfqpoint{2.240182in}{0.750462in}}%
\pgfpathlineto{\pgfqpoint{2.240746in}{0.750484in}}%
\pgfpathlineto{\pgfqpoint{2.241311in}{0.750506in}}%
\pgfpathlineto{\pgfqpoint{2.241875in}{0.750528in}}%
\pgfpathlineto{\pgfqpoint{2.242439in}{0.750550in}}%
\pgfpathlineto{\pgfqpoint{2.243004in}{0.750572in}}%
\pgfpathlineto{\pgfqpoint{2.243568in}{0.750615in}}%
\pgfpathlineto{\pgfqpoint{2.244132in}{0.750687in}}%
\pgfpathlineto{\pgfqpoint{2.244697in}{0.750760in}}%
\pgfpathlineto{\pgfqpoint{2.245261in}{0.750832in}}%
\pgfpathlineto{\pgfqpoint{2.245826in}{0.750905in}}%
\pgfpathlineto{\pgfqpoint{2.246390in}{0.750977in}}%
\pgfpathlineto{\pgfqpoint{2.246954in}{0.751050in}}%
\pgfpathlineto{\pgfqpoint{2.247519in}{0.751122in}}%
\pgfpathlineto{\pgfqpoint{2.248083in}{0.751194in}}%
\pgfpathlineto{\pgfqpoint{2.248647in}{0.751267in}}%
\pgfpathlineto{\pgfqpoint{2.249212in}{0.751339in}}%
\pgfpathlineto{\pgfqpoint{2.249776in}{0.751412in}}%
\pgfpathlineto{\pgfqpoint{2.250340in}{0.751484in}}%
\pgfpathlineto{\pgfqpoint{2.250905in}{0.751535in}}%
\pgfpathlineto{\pgfqpoint{2.251469in}{0.751530in}}%
\pgfpathlineto{\pgfqpoint{2.252033in}{0.751304in}}%
\pgfpathlineto{\pgfqpoint{2.252598in}{0.751132in}}%
\pgfpathlineto{\pgfqpoint{2.253162in}{0.751117in}}%
\pgfpathlineto{\pgfqpoint{2.253726in}{0.751110in}}%
\pgfpathlineto{\pgfqpoint{2.254291in}{0.751102in}}%
\pgfpathlineto{\pgfqpoint{2.254855in}{0.751095in}}%
\pgfpathlineto{\pgfqpoint{2.255419in}{0.751087in}}%
\pgfpathlineto{\pgfqpoint{2.255984in}{0.751080in}}%
\pgfpathlineto{\pgfqpoint{2.256548in}{0.751072in}}%
\pgfpathlineto{\pgfqpoint{2.257112in}{0.751065in}}%
\pgfpathlineto{\pgfqpoint{2.257677in}{0.751057in}}%
\pgfpathlineto{\pgfqpoint{2.258241in}{0.751050in}}%
\pgfpathlineto{\pgfqpoint{2.258805in}{0.751042in}}%
\pgfpathlineto{\pgfqpoint{2.259370in}{0.751035in}}%
\pgfpathlineto{\pgfqpoint{2.259934in}{0.751027in}}%
\pgfpathlineto{\pgfqpoint{2.260499in}{0.751020in}}%
\pgfpathlineto{\pgfqpoint{2.261063in}{0.751012in}}%
\pgfpathlineto{\pgfqpoint{2.261627in}{0.751005in}}%
\pgfpathlineto{\pgfqpoint{2.262192in}{0.750997in}}%
\pgfpathlineto{\pgfqpoint{2.262756in}{0.750990in}}%
\pgfpathlineto{\pgfqpoint{2.263320in}{0.750986in}}%
\pgfpathlineto{\pgfqpoint{2.263885in}{0.751006in}}%
\pgfpathlineto{\pgfqpoint{2.264449in}{0.751029in}}%
\pgfpathlineto{\pgfqpoint{2.265013in}{0.751044in}}%
\pgfpathlineto{\pgfqpoint{2.265578in}{0.751036in}}%
\pgfpathlineto{\pgfqpoint{2.266142in}{0.751029in}}%
\pgfpathlineto{\pgfqpoint{2.266706in}{0.751021in}}%
\pgfpathlineto{\pgfqpoint{2.267271in}{0.751014in}}%
\pgfpathlineto{\pgfqpoint{2.267835in}{0.751006in}}%
\pgfpathlineto{\pgfqpoint{2.268399in}{0.750999in}}%
\pgfpathlineto{\pgfqpoint{2.268964in}{0.750992in}}%
\pgfpathlineto{\pgfqpoint{2.269528in}{0.750984in}}%
\pgfpathlineto{\pgfqpoint{2.270092in}{0.750977in}}%
\pgfpathlineto{\pgfqpoint{2.270657in}{0.750969in}}%
\pgfpathlineto{\pgfqpoint{2.271221in}{0.750962in}}%
\pgfpathlineto{\pgfqpoint{2.271785in}{0.750955in}}%
\pgfpathlineto{\pgfqpoint{2.272350in}{0.750947in}}%
\pgfpathlineto{\pgfqpoint{2.272914in}{0.750940in}}%
\pgfpathlineto{\pgfqpoint{2.273478in}{0.750933in}}%
\pgfpathlineto{\pgfqpoint{2.274043in}{0.750925in}}%
\pgfpathlineto{\pgfqpoint{2.274607in}{0.750918in}}%
\pgfpathlineto{\pgfqpoint{2.275172in}{0.750910in}}%
\pgfpathlineto{\pgfqpoint{2.275736in}{0.750903in}}%
\pgfpathlineto{\pgfqpoint{2.276300in}{0.750896in}}%
\pgfpathlineto{\pgfqpoint{2.276865in}{0.750888in}}%
\pgfpathlineto{\pgfqpoint{2.277429in}{0.750881in}}%
\pgfpathlineto{\pgfqpoint{2.277993in}{0.750873in}}%
\pgfpathlineto{\pgfqpoint{2.278558in}{0.750866in}}%
\pgfpathlineto{\pgfqpoint{2.279122in}{0.750859in}}%
\pgfpathlineto{\pgfqpoint{2.279686in}{0.750851in}}%
\pgfpathlineto{\pgfqpoint{2.280251in}{0.750844in}}%
\pgfpathlineto{\pgfqpoint{2.280815in}{0.750836in}}%
\pgfpathlineto{\pgfqpoint{2.281379in}{0.750829in}}%
\pgfpathlineto{\pgfqpoint{2.281944in}{0.750822in}}%
\pgfpathlineto{\pgfqpoint{2.282508in}{0.750814in}}%
\pgfpathlineto{\pgfqpoint{2.283072in}{0.750807in}}%
\pgfpathlineto{\pgfqpoint{2.283637in}{0.750800in}}%
\pgfpathlineto{\pgfqpoint{2.284201in}{0.750792in}}%
\pgfpathlineto{\pgfqpoint{2.284765in}{0.750785in}}%
\pgfpathlineto{\pgfqpoint{2.285330in}{0.750777in}}%
\pgfpathlineto{\pgfqpoint{2.285894in}{0.750770in}}%
\pgfpathlineto{\pgfqpoint{2.286458in}{0.750763in}}%
\pgfpathlineto{\pgfqpoint{2.287023in}{0.750755in}}%
\pgfpathlineto{\pgfqpoint{2.287587in}{0.750748in}}%
\pgfpathlineto{\pgfqpoint{2.288151in}{0.750740in}}%
\pgfpathlineto{\pgfqpoint{2.288716in}{0.750733in}}%
\pgfpathlineto{\pgfqpoint{2.289280in}{0.750726in}}%
\pgfpathlineto{\pgfqpoint{2.289844in}{0.750718in}}%
\pgfpathlineto{\pgfqpoint{2.290409in}{0.750711in}}%
\pgfpathlineto{\pgfqpoint{2.290973in}{0.750703in}}%
\pgfpathlineto{\pgfqpoint{2.291538in}{0.750696in}}%
\pgfpathlineto{\pgfqpoint{2.292102in}{0.750689in}}%
\pgfpathlineto{\pgfqpoint{2.292666in}{0.750681in}}%
\pgfpathlineto{\pgfqpoint{2.293231in}{0.750674in}}%
\pgfpathlineto{\pgfqpoint{2.293795in}{0.750667in}}%
\pgfpathlineto{\pgfqpoint{2.294359in}{0.750659in}}%
\pgfpathlineto{\pgfqpoint{2.294924in}{0.750652in}}%
\pgfpathlineto{\pgfqpoint{2.295488in}{0.750644in}}%
\pgfpathlineto{\pgfqpoint{2.296052in}{0.750637in}}%
\pgfpathlineto{\pgfqpoint{2.296617in}{0.750630in}}%
\pgfpathlineto{\pgfqpoint{2.297181in}{0.750622in}}%
\pgfpathlineto{\pgfqpoint{2.297745in}{0.750615in}}%
\pgfpathlineto{\pgfqpoint{2.298310in}{0.750607in}}%
\pgfpathlineto{\pgfqpoint{2.298874in}{0.750600in}}%
\pgfpathlineto{\pgfqpoint{2.299438in}{0.750593in}}%
\pgfpathlineto{\pgfqpoint{2.300003in}{0.750585in}}%
\pgfpathlineto{\pgfqpoint{2.300567in}{0.750578in}}%
\pgfpathlineto{\pgfqpoint{2.301131in}{0.750571in}}%
\pgfpathlineto{\pgfqpoint{2.301696in}{0.750563in}}%
\pgfpathlineto{\pgfqpoint{2.302260in}{0.750556in}}%
\pgfpathlineto{\pgfqpoint{2.302824in}{0.750548in}}%
\pgfpathlineto{\pgfqpoint{2.303389in}{0.750541in}}%
\pgfpathlineto{\pgfqpoint{2.303953in}{0.750534in}}%
\pgfpathlineto{\pgfqpoint{2.304517in}{0.750526in}}%
\pgfpathlineto{\pgfqpoint{2.305082in}{0.750519in}}%
\pgfpathlineto{\pgfqpoint{2.305646in}{0.750511in}}%
\pgfpathlineto{\pgfqpoint{2.306211in}{0.750504in}}%
\pgfpathlineto{\pgfqpoint{2.306775in}{0.750497in}}%
\pgfpathlineto{\pgfqpoint{2.307339in}{0.750489in}}%
\pgfpathlineto{\pgfqpoint{2.307904in}{0.750482in}}%
\pgfpathlineto{\pgfqpoint{2.308468in}{0.750474in}}%
\pgfpathlineto{\pgfqpoint{2.309032in}{0.750467in}}%
\pgfpathlineto{\pgfqpoint{2.309597in}{0.750460in}}%
\pgfpathlineto{\pgfqpoint{2.310161in}{0.750452in}}%
\pgfpathlineto{\pgfqpoint{2.310725in}{0.750445in}}%
\pgfpathlineto{\pgfqpoint{2.311290in}{0.750438in}}%
\pgfpathlineto{\pgfqpoint{2.311854in}{0.750430in}}%
\pgfpathlineto{\pgfqpoint{2.312418in}{0.750423in}}%
\pgfpathlineto{\pgfqpoint{2.312983in}{0.750415in}}%
\pgfpathlineto{\pgfqpoint{2.313547in}{0.750408in}}%
\pgfpathlineto{\pgfqpoint{2.314111in}{0.750401in}}%
\pgfpathlineto{\pgfqpoint{2.314676in}{0.750393in}}%
\pgfpathlineto{\pgfqpoint{2.315240in}{0.750386in}}%
\pgfpathlineto{\pgfqpoint{2.315804in}{0.750378in}}%
\pgfpathlineto{\pgfqpoint{2.316369in}{0.750371in}}%
\pgfpathlineto{\pgfqpoint{2.316933in}{0.750364in}}%
\pgfpathlineto{\pgfqpoint{2.317497in}{0.750356in}}%
\pgfpathlineto{\pgfqpoint{2.318062in}{0.750349in}}%
\pgfpathlineto{\pgfqpoint{2.318626in}{0.750342in}}%
\pgfpathlineto{\pgfqpoint{2.319190in}{0.750334in}}%
\pgfpathlineto{\pgfqpoint{2.319755in}{0.750327in}}%
\pgfpathlineto{\pgfqpoint{2.320319in}{0.750406in}}%
\pgfpathlineto{\pgfqpoint{2.320884in}{0.750593in}}%
\pgfpathlineto{\pgfqpoint{2.321448in}{0.750568in}}%
\pgfpathlineto{\pgfqpoint{2.322012in}{0.750543in}}%
\pgfpathlineto{\pgfqpoint{2.322577in}{0.750518in}}%
\pgfpathlineto{\pgfqpoint{2.323141in}{0.750492in}}%
\pgfpathlineto{\pgfqpoint{2.323705in}{0.750467in}}%
\pgfpathlineto{\pgfqpoint{2.324270in}{0.750442in}}%
\pgfpathlineto{\pgfqpoint{2.324834in}{0.750417in}}%
\pgfpathlineto{\pgfqpoint{2.325398in}{0.750392in}}%
\pgfpathlineto{\pgfqpoint{2.325963in}{0.750366in}}%
\pgfpathlineto{\pgfqpoint{2.326527in}{0.750341in}}%
\pgfpathlineto{\pgfqpoint{2.327091in}{0.750316in}}%
\pgfpathlineto{\pgfqpoint{2.327656in}{0.750291in}}%
\pgfpathlineto{\pgfqpoint{2.328220in}{0.750266in}}%
\pgfpathlineto{\pgfqpoint{2.328784in}{0.750241in}}%
\pgfpathlineto{\pgfqpoint{2.329349in}{0.750215in}}%
\pgfpathlineto{\pgfqpoint{2.329913in}{0.750959in}}%
\pgfpathlineto{\pgfqpoint{2.330477in}{0.750792in}}%
\pgfpathlineto{\pgfqpoint{2.331042in}{0.750614in}}%
\pgfpathlineto{\pgfqpoint{2.331606in}{0.750689in}}%
\pgfpathlineto{\pgfqpoint{2.332170in}{0.751316in}}%
\pgfpathlineto{\pgfqpoint{2.332735in}{0.751953in}}%
\pgfpathlineto{\pgfqpoint{2.333299in}{0.751602in}}%
\pgfpathlineto{\pgfqpoint{2.333863in}{0.751822in}}%
\pgfpathlineto{\pgfqpoint{2.334428in}{0.753225in}}%
\pgfpathlineto{\pgfqpoint{2.334992in}{0.754804in}}%
\pgfpathlineto{\pgfqpoint{2.335556in}{0.753306in}}%
\pgfpathlineto{\pgfqpoint{2.336121in}{0.752425in}}%
\pgfpathlineto{\pgfqpoint{2.336685in}{0.752473in}}%
\pgfpathlineto{\pgfqpoint{2.337250in}{0.752522in}}%
\pgfpathlineto{\pgfqpoint{2.337814in}{0.752570in}}%
\pgfpathlineto{\pgfqpoint{2.338378in}{0.752618in}}%
\pgfpathlineto{\pgfqpoint{2.338943in}{0.752666in}}%
\pgfpathlineto{\pgfqpoint{2.339507in}{0.752714in}}%
\pgfpathlineto{\pgfqpoint{2.340071in}{0.752762in}}%
\pgfpathlineto{\pgfqpoint{2.340636in}{0.752810in}}%
\pgfpathlineto{\pgfqpoint{2.341200in}{0.752859in}}%
\pgfpathlineto{\pgfqpoint{2.341764in}{0.752907in}}%
\pgfpathlineto{\pgfqpoint{2.342329in}{0.752955in}}%
\pgfpathlineto{\pgfqpoint{2.342893in}{0.753003in}}%
\pgfpathlineto{\pgfqpoint{2.343457in}{0.753051in}}%
\pgfpathlineto{\pgfqpoint{2.344022in}{0.753099in}}%
\pgfpathlineto{\pgfqpoint{2.344586in}{0.753144in}}%
\pgfpathlineto{\pgfqpoint{2.345150in}{0.753169in}}%
\pgfpathlineto{\pgfqpoint{2.345715in}{0.752930in}}%
\pgfpathlineto{\pgfqpoint{2.346279in}{0.753186in}}%
\pgfpathlineto{\pgfqpoint{2.346843in}{0.753413in}}%
\pgfpathlineto{\pgfqpoint{2.347408in}{0.753337in}}%
\pgfpathlineto{\pgfqpoint{2.347972in}{0.753527in}}%
\pgfpathlineto{\pgfqpoint{2.348536in}{0.753555in}}%
\pgfpathlineto{\pgfqpoint{2.349101in}{0.753537in}}%
\pgfpathlineto{\pgfqpoint{2.349665in}{0.753520in}}%
\pgfpathlineto{\pgfqpoint{2.350229in}{0.753502in}}%
\pgfpathlineto{\pgfqpoint{2.350794in}{0.753485in}}%
\pgfpathlineto{\pgfqpoint{2.351358in}{0.753467in}}%
\pgfpathlineto{\pgfqpoint{2.351923in}{0.753450in}}%
\pgfpathlineto{\pgfqpoint{2.352487in}{0.753432in}}%
\pgfpathlineto{\pgfqpoint{2.353051in}{0.753415in}}%
\pgfpathlineto{\pgfqpoint{2.353616in}{0.753397in}}%
\pgfpathlineto{\pgfqpoint{2.354180in}{0.753380in}}%
\pgfpathlineto{\pgfqpoint{2.354744in}{0.753362in}}%
\pgfpathlineto{\pgfqpoint{2.355309in}{0.753345in}}%
\pgfpathlineto{\pgfqpoint{2.355873in}{0.753327in}}%
\pgfpathlineto{\pgfqpoint{2.356437in}{0.753310in}}%
\pgfpathlineto{\pgfqpoint{2.357002in}{0.753292in}}%
\pgfpathlineto{\pgfqpoint{2.357566in}{0.753275in}}%
\pgfpathlineto{\pgfqpoint{2.358130in}{0.753257in}}%
\pgfpathlineto{\pgfqpoint{2.358695in}{0.753239in}}%
\pgfpathlineto{\pgfqpoint{2.359259in}{0.753224in}}%
\pgfpathlineto{\pgfqpoint{2.359823in}{0.753919in}}%
\pgfpathlineto{\pgfqpoint{2.360388in}{0.752376in}}%
\pgfpathlineto{\pgfqpoint{2.360952in}{0.753479in}}%
\pgfpathlineto{\pgfqpoint{2.361516in}{0.754620in}}%
\pgfpathlineto{\pgfqpoint{2.362081in}{0.754615in}}%
\pgfpathlineto{\pgfqpoint{2.362645in}{0.754655in}}%
\pgfpathlineto{\pgfqpoint{2.363209in}{0.754700in}}%
\pgfpathlineto{\pgfqpoint{2.363774in}{0.754745in}}%
\pgfpathlineto{\pgfqpoint{2.364338in}{0.754790in}}%
\pgfpathlineto{\pgfqpoint{2.364902in}{0.754835in}}%
\pgfpathlineto{\pgfqpoint{2.365467in}{0.754879in}}%
\pgfpathlineto{\pgfqpoint{2.366031in}{0.754924in}}%
\pgfpathlineto{\pgfqpoint{2.366596in}{0.754969in}}%
\pgfpathlineto{\pgfqpoint{2.367160in}{0.755014in}}%
\pgfpathlineto{\pgfqpoint{2.367724in}{0.755059in}}%
\pgfpathlineto{\pgfqpoint{2.368289in}{0.755104in}}%
\pgfpathlineto{\pgfqpoint{2.368853in}{0.755149in}}%
\pgfpathlineto{\pgfqpoint{2.369417in}{0.755194in}}%
\pgfpathlineto{\pgfqpoint{2.369982in}{0.755238in}}%
\pgfpathlineto{\pgfqpoint{2.370546in}{0.754747in}}%
\pgfpathlineto{\pgfqpoint{2.371110in}{0.753332in}}%
\pgfpathlineto{\pgfqpoint{2.371675in}{0.753333in}}%
\pgfpathlineto{\pgfqpoint{2.372239in}{0.754110in}}%
\pgfpathlineto{\pgfqpoint{2.372803in}{0.754704in}}%
\pgfpathlineto{\pgfqpoint{2.373368in}{0.758025in}}%
\pgfpathlineto{\pgfqpoint{2.373932in}{0.754305in}}%
\pgfpathlineto{\pgfqpoint{2.374496in}{0.755809in}}%
\pgfpathlineto{\pgfqpoint{2.375061in}{0.763582in}}%
\pgfpathlineto{\pgfqpoint{2.375625in}{0.763438in}}%
\pgfpathlineto{\pgfqpoint{2.376189in}{0.763293in}}%
\pgfpathlineto{\pgfqpoint{2.376754in}{0.763148in}}%
\pgfpathlineto{\pgfqpoint{2.377318in}{0.763003in}}%
\pgfpathlineto{\pgfqpoint{2.377882in}{0.762858in}}%
\pgfpathlineto{\pgfqpoint{2.378447in}{0.762713in}}%
\pgfpathlineto{\pgfqpoint{2.379011in}{0.762568in}}%
\pgfpathlineto{\pgfqpoint{2.379575in}{0.762424in}}%
\pgfpathlineto{\pgfqpoint{2.380140in}{0.762279in}}%
\pgfpathlineto{\pgfqpoint{2.380704in}{0.762134in}}%
\pgfpathlineto{\pgfqpoint{2.381268in}{0.761989in}}%
\pgfpathlineto{\pgfqpoint{2.381833in}{0.761844in}}%
\pgfpathlineto{\pgfqpoint{2.382397in}{0.761699in}}%
\pgfpathlineto{\pgfqpoint{2.382962in}{0.761554in}}%
\pgfpathlineto{\pgfqpoint{2.383526in}{0.761410in}}%
\pgfpathlineto{\pgfqpoint{2.384090in}{0.761265in}}%
\pgfpathlineto{\pgfqpoint{2.384655in}{0.761120in}}%
\pgfpathlineto{\pgfqpoint{2.385219in}{0.760975in}}%
\pgfpathlineto{\pgfqpoint{2.385783in}{0.760830in}}%
\pgfpathlineto{\pgfqpoint{2.386348in}{0.760685in}}%
\pgfpathlineto{\pgfqpoint{2.386912in}{0.760540in}}%
\pgfpathlineto{\pgfqpoint{2.387476in}{0.760396in}}%
\pgfpathlineto{\pgfqpoint{2.388041in}{0.760251in}}%
\pgfpathlineto{\pgfqpoint{2.388605in}{0.760106in}}%
\pgfpathlineto{\pgfqpoint{2.389169in}{0.759961in}}%
\pgfpathlineto{\pgfqpoint{2.389734in}{0.759816in}}%
\pgfpathlineto{\pgfqpoint{2.390298in}{0.759671in}}%
\pgfpathlineto{\pgfqpoint{2.390862in}{0.759526in}}%
\pgfpathlineto{\pgfqpoint{2.391427in}{0.759382in}}%
\pgfpathlineto{\pgfqpoint{2.391991in}{0.759237in}}%
\pgfpathlineto{\pgfqpoint{2.392555in}{0.759092in}}%
\pgfpathlineto{\pgfqpoint{2.393120in}{0.758947in}}%
\pgfpathlineto{\pgfqpoint{2.393684in}{0.758802in}}%
\pgfpathlineto{\pgfqpoint{2.394248in}{0.758657in}}%
\pgfpathlineto{\pgfqpoint{2.394813in}{0.758512in}}%
\pgfpathlineto{\pgfqpoint{2.395377in}{0.758368in}}%
\pgfpathlineto{\pgfqpoint{2.395941in}{0.758223in}}%
\pgfpathlineto{\pgfqpoint{2.396506in}{0.758078in}}%
\pgfpathlineto{\pgfqpoint{2.397070in}{0.757933in}}%
\pgfpathlineto{\pgfqpoint{2.397635in}{0.757788in}}%
\pgfpathlineto{\pgfqpoint{2.398199in}{0.757643in}}%
\pgfpathlineto{\pgfqpoint{2.398763in}{0.757498in}}%
\pgfpathlineto{\pgfqpoint{2.399328in}{0.757353in}}%
\pgfpathlineto{\pgfqpoint{2.399892in}{0.757209in}}%
\pgfpathlineto{\pgfqpoint{2.400456in}{0.757064in}}%
\pgfpathlineto{\pgfqpoint{2.401021in}{0.756919in}}%
\pgfpathlineto{\pgfqpoint{2.401585in}{0.756774in}}%
\pgfpathlineto{\pgfqpoint{2.402149in}{0.756629in}}%
\pgfpathlineto{\pgfqpoint{2.402714in}{0.756484in}}%
\pgfpathlineto{\pgfqpoint{2.403278in}{0.756339in}}%
\pgfpathlineto{\pgfqpoint{2.403842in}{0.756195in}}%
\pgfpathlineto{\pgfqpoint{2.404407in}{0.756050in}}%
\pgfpathlineto{\pgfqpoint{2.404971in}{0.755905in}}%
\pgfpathlineto{\pgfqpoint{2.405535in}{0.755760in}}%
\pgfpathlineto{\pgfqpoint{2.406100in}{0.755615in}}%
\pgfpathlineto{\pgfqpoint{2.406664in}{0.755470in}}%
\pgfpathlineto{\pgfqpoint{2.407228in}{0.755325in}}%
\pgfpathlineto{\pgfqpoint{2.407793in}{0.755181in}}%
\pgfpathlineto{\pgfqpoint{2.408357in}{0.755036in}}%
\pgfpathlineto{\pgfqpoint{2.408921in}{0.754891in}}%
\pgfpathlineto{\pgfqpoint{2.409486in}{0.754746in}}%
\pgfpathlineto{\pgfqpoint{2.410050in}{0.754601in}}%
\pgfpathlineto{\pgfqpoint{2.410614in}{0.754456in}}%
\pgfpathlineto{\pgfqpoint{2.411179in}{0.754311in}}%
\pgfpathlineto{\pgfqpoint{2.411743in}{0.754167in}}%
\pgfpathlineto{\pgfqpoint{2.412308in}{0.754022in}}%
\pgfpathlineto{\pgfqpoint{2.412872in}{0.753946in}}%
\pgfpathlineto{\pgfqpoint{2.413436in}{0.753537in}}%
\pgfpathlineto{\pgfqpoint{2.414001in}{0.752081in}}%
\pgfpathlineto{\pgfqpoint{2.414565in}{0.752137in}}%
\pgfpathlineto{\pgfqpoint{2.415129in}{0.755062in}}%
\pgfpathlineto{\pgfqpoint{2.415694in}{0.757988in}}%
\pgfpathlineto{\pgfqpoint{2.416258in}{0.760913in}}%
\pgfpathlineto{\pgfqpoint{2.416822in}{0.759661in}}%
\pgfpathlineto{\pgfqpoint{2.417387in}{0.755974in}}%
\pgfpathlineto{\pgfqpoint{2.417951in}{0.756386in}}%
\pgfpathlineto{\pgfqpoint{2.418515in}{0.756797in}}%
\pgfpathlineto{\pgfqpoint{2.419080in}{0.757209in}}%
\pgfpathlineto{\pgfqpoint{2.419644in}{0.757621in}}%
\pgfpathlineto{\pgfqpoint{2.420208in}{0.758033in}}%
\pgfpathlineto{\pgfqpoint{2.420773in}{0.758445in}}%
\pgfpathlineto{\pgfqpoint{2.421337in}{0.758857in}}%
\pgfpathlineto{\pgfqpoint{2.421901in}{0.759269in}}%
\pgfpathlineto{\pgfqpoint{2.422466in}{0.759681in}}%
\pgfpathlineto{\pgfqpoint{2.423030in}{0.760093in}}%
\pgfpathlineto{\pgfqpoint{2.423594in}{0.760505in}}%
\pgfpathlineto{\pgfqpoint{2.424159in}{0.760917in}}%
\pgfpathlineto{\pgfqpoint{2.424723in}{0.761329in}}%
\pgfpathlineto{\pgfqpoint{2.425287in}{0.761741in}}%
\pgfpathlineto{\pgfqpoint{2.425852in}{0.762153in}}%
\pgfpathlineto{\pgfqpoint{2.426416in}{0.762565in}}%
\pgfpathlineto{\pgfqpoint{2.426980in}{0.757010in}}%
\pgfpathlineto{\pgfqpoint{2.427545in}{0.751002in}}%
\pgfpathlineto{\pgfqpoint{2.428109in}{0.751199in}}%
\pgfpathlineto{\pgfqpoint{2.428674in}{0.751580in}}%
\pgfpathlineto{\pgfqpoint{2.429238in}{0.751649in}}%
\pgfpathlineto{\pgfqpoint{2.429802in}{0.751719in}}%
\pgfpathlineto{\pgfqpoint{2.430367in}{0.751788in}}%
\pgfpathlineto{\pgfqpoint{2.430931in}{0.751857in}}%
\pgfpathlineto{\pgfqpoint{2.431495in}{0.751927in}}%
\pgfpathlineto{\pgfqpoint{2.432060in}{0.751996in}}%
\pgfpathlineto{\pgfqpoint{2.432624in}{0.752066in}}%
\pgfpathlineto{\pgfqpoint{2.433188in}{0.752135in}}%
\pgfpathlineto{\pgfqpoint{2.433753in}{0.752205in}}%
\pgfpathlineto{\pgfqpoint{2.434317in}{0.752274in}}%
\pgfpathlineto{\pgfqpoint{2.434881in}{0.752344in}}%
\pgfpathlineto{\pgfqpoint{2.435446in}{0.752413in}}%
\pgfpathlineto{\pgfqpoint{2.436010in}{0.752482in}}%
\pgfpathlineto{\pgfqpoint{2.436574in}{0.752552in}}%
\pgfpathlineto{\pgfqpoint{2.437139in}{0.752621in}}%
\pgfpathlineto{\pgfqpoint{2.437703in}{0.752691in}}%
\pgfpathlineto{\pgfqpoint{2.438267in}{0.753283in}}%
\pgfpathlineto{\pgfqpoint{2.438832in}{0.755137in}}%
\pgfpathlineto{\pgfqpoint{2.439396in}{0.760419in}}%
\pgfpathlineto{\pgfqpoint{2.439960in}{0.763207in}}%
\pgfpathlineto{\pgfqpoint{2.440525in}{0.763338in}}%
\pgfpathlineto{\pgfqpoint{2.441089in}{0.763379in}}%
\pgfpathlineto{\pgfqpoint{2.441653in}{0.763374in}}%
\pgfpathlineto{\pgfqpoint{2.442218in}{0.760365in}}%
\pgfpathlineto{\pgfqpoint{2.442782in}{0.753454in}}%
\pgfpathlineto{\pgfqpoint{2.443347in}{0.752679in}}%
\pgfpathlineto{\pgfqpoint{2.443911in}{0.752503in}}%
\pgfpathlineto{\pgfqpoint{2.444475in}{0.752326in}}%
\pgfpathlineto{\pgfqpoint{2.445040in}{0.752150in}}%
\pgfpathlineto{\pgfqpoint{2.445604in}{0.751973in}}%
\pgfpathlineto{\pgfqpoint{2.446168in}{0.751797in}}%
\pgfpathlineto{\pgfqpoint{2.446733in}{0.751620in}}%
\pgfpathlineto{\pgfqpoint{2.447297in}{0.751444in}}%
\pgfpathlineto{\pgfqpoint{2.447861in}{0.751267in}}%
\pgfpathlineto{\pgfqpoint{2.448426in}{0.751091in}}%
\pgfpathlineto{\pgfqpoint{2.448990in}{0.750914in}}%
\pgfpathlineto{\pgfqpoint{2.449554in}{0.750738in}}%
\pgfpathlineto{\pgfqpoint{2.450119in}{0.750561in}}%
\pgfpathlineto{\pgfqpoint{2.450683in}{0.750385in}}%
\pgfpathlineto{\pgfqpoint{2.451247in}{0.750208in}}%
\pgfpathlineto{\pgfqpoint{2.451812in}{0.750032in}}%
\pgfpathlineto{\pgfqpoint{2.452376in}{0.749890in}}%
\pgfpathlineto{\pgfqpoint{2.452940in}{0.749872in}}%
\pgfpathlineto{\pgfqpoint{2.453505in}{0.749865in}}%
\pgfpathlineto{\pgfqpoint{2.454069in}{0.749858in}}%
\pgfpathlineto{\pgfqpoint{2.454633in}{0.749851in}}%
\pgfpathlineto{\pgfqpoint{2.455198in}{0.752671in}}%
\pgfpathlineto{\pgfqpoint{2.455762in}{0.759979in}}%
\pgfpathlineto{\pgfqpoint{2.456326in}{0.763241in}}%
\pgfpathlineto{\pgfqpoint{2.456891in}{0.763226in}}%
\pgfpathlineto{\pgfqpoint{2.457455in}{0.763211in}}%
\pgfpathlineto{\pgfqpoint{2.458020in}{0.763195in}}%
\pgfpathlineto{\pgfqpoint{2.458584in}{0.763180in}}%
\pgfpathlineto{\pgfqpoint{2.459148in}{0.763164in}}%
\pgfpathlineto{\pgfqpoint{2.459713in}{0.763149in}}%
\pgfpathlineto{\pgfqpoint{2.460277in}{0.763134in}}%
\pgfpathlineto{\pgfqpoint{2.460841in}{0.763118in}}%
\pgfpathlineto{\pgfqpoint{2.461406in}{0.763103in}}%
\pgfpathlineto{\pgfqpoint{2.461970in}{0.763088in}}%
\pgfpathlineto{\pgfqpoint{2.462534in}{0.763072in}}%
\pgfpathlineto{\pgfqpoint{2.463099in}{0.763057in}}%
\pgfpathlineto{\pgfqpoint{2.463663in}{0.763041in}}%
\pgfpathlineto{\pgfqpoint{2.464227in}{0.763026in}}%
\pgfpathlineto{\pgfqpoint{2.464792in}{0.763011in}}%
\pgfpathlineto{\pgfqpoint{2.465356in}{0.762995in}}%
\pgfpathlineto{\pgfqpoint{2.465920in}{0.762980in}}%
\pgfpathlineto{\pgfqpoint{2.466485in}{0.762965in}}%
\pgfpathlineto{\pgfqpoint{2.467049in}{0.762949in}}%
\pgfpathlineto{\pgfqpoint{2.467613in}{0.762934in}}%
\pgfpathlineto{\pgfqpoint{2.468178in}{0.762918in}}%
\pgfpathlineto{\pgfqpoint{2.468742in}{0.762903in}}%
\pgfpathlineto{\pgfqpoint{2.469306in}{0.762888in}}%
\pgfpathlineto{\pgfqpoint{2.469871in}{0.762872in}}%
\pgfpathlineto{\pgfqpoint{2.470435in}{0.762857in}}%
\pgfpathlineto{\pgfqpoint{2.470999in}{0.762842in}}%
\pgfpathlineto{\pgfqpoint{2.471564in}{0.762826in}}%
\pgfpathlineto{\pgfqpoint{2.472128in}{0.762811in}}%
\pgfpathlineto{\pgfqpoint{2.472692in}{0.762795in}}%
\pgfpathlineto{\pgfqpoint{2.473257in}{0.762780in}}%
\pgfpathlineto{\pgfqpoint{2.473821in}{0.762765in}}%
\pgfpathlineto{\pgfqpoint{2.474386in}{0.762749in}}%
\pgfpathlineto{\pgfqpoint{2.474950in}{0.762734in}}%
\pgfpathlineto{\pgfqpoint{2.475514in}{0.762719in}}%
\pgfpathlineto{\pgfqpoint{2.476079in}{0.762703in}}%
\pgfpathlineto{\pgfqpoint{2.476643in}{0.762688in}}%
\pgfpathlineto{\pgfqpoint{2.477207in}{0.762672in}}%
\pgfpathlineto{\pgfqpoint{2.477772in}{0.762657in}}%
\pgfpathlineto{\pgfqpoint{2.478336in}{0.762642in}}%
\pgfpathlineto{\pgfqpoint{2.478900in}{0.762626in}}%
\pgfpathlineto{\pgfqpoint{2.479465in}{0.762611in}}%
\pgfpathlineto{\pgfqpoint{2.480029in}{0.762596in}}%
\pgfpathlineto{\pgfqpoint{2.480593in}{0.762580in}}%
\pgfpathlineto{\pgfqpoint{2.481158in}{0.762565in}}%
\pgfpathlineto{\pgfqpoint{2.481722in}{0.762549in}}%
\pgfpathlineto{\pgfqpoint{2.482286in}{0.762534in}}%
\pgfpathlineto{\pgfqpoint{2.482851in}{0.762519in}}%
\pgfpathlineto{\pgfqpoint{2.483415in}{0.762503in}}%
\pgfpathlineto{\pgfqpoint{2.483979in}{0.762488in}}%
\pgfpathlineto{\pgfqpoint{2.484544in}{0.762472in}}%
\pgfpathlineto{\pgfqpoint{2.485108in}{0.762457in}}%
\pgfpathlineto{\pgfqpoint{2.485672in}{0.762442in}}%
\pgfpathlineto{\pgfqpoint{2.486237in}{0.762426in}}%
\pgfpathlineto{\pgfqpoint{2.486801in}{0.762411in}}%
\pgfpathlineto{\pgfqpoint{2.487365in}{0.762396in}}%
\pgfpathlineto{\pgfqpoint{2.487930in}{0.762380in}}%
\pgfpathlineto{\pgfqpoint{2.488494in}{0.762365in}}%
\pgfpathlineto{\pgfqpoint{2.489059in}{0.762349in}}%
\pgfpathlineto{\pgfqpoint{2.489623in}{0.762334in}}%
\pgfpathlineto{\pgfqpoint{2.490187in}{0.762319in}}%
\pgfpathlineto{\pgfqpoint{2.490752in}{0.762303in}}%
\pgfpathlineto{\pgfqpoint{2.491316in}{0.762288in}}%
\pgfpathlineto{\pgfqpoint{2.491880in}{0.762273in}}%
\pgfpathlineto{\pgfqpoint{2.492445in}{0.762257in}}%
\pgfpathlineto{\pgfqpoint{2.493009in}{0.762242in}}%
\pgfpathlineto{\pgfqpoint{2.493573in}{0.762226in}}%
\pgfpathlineto{\pgfqpoint{2.494138in}{0.762211in}}%
\pgfpathlineto{\pgfqpoint{2.494702in}{0.762196in}}%
\pgfpathlineto{\pgfqpoint{2.495266in}{0.762180in}}%
\pgfpathlineto{\pgfqpoint{2.495831in}{0.762165in}}%
\pgfpathlineto{\pgfqpoint{2.496395in}{0.762150in}}%
\pgfpathlineto{\pgfqpoint{2.496959in}{0.762134in}}%
\pgfpathlineto{\pgfqpoint{2.497524in}{0.762119in}}%
\pgfpathlineto{\pgfqpoint{2.498088in}{0.762103in}}%
\pgfpathlineto{\pgfqpoint{2.498652in}{0.762088in}}%
\pgfpathlineto{\pgfqpoint{2.499217in}{0.762073in}}%
\pgfpathlineto{\pgfqpoint{2.499781in}{0.762057in}}%
\pgfpathlineto{\pgfqpoint{2.500345in}{0.762042in}}%
\pgfpathlineto{\pgfqpoint{2.500910in}{0.762027in}}%
\pgfpathlineto{\pgfqpoint{2.501474in}{0.762011in}}%
\pgfpathlineto{\pgfqpoint{2.502038in}{0.761996in}}%
\pgfpathlineto{\pgfqpoint{2.502603in}{0.761980in}}%
\pgfpathlineto{\pgfqpoint{2.503167in}{0.761965in}}%
\pgfpathlineto{\pgfqpoint{2.503732in}{0.761950in}}%
\pgfpathlineto{\pgfqpoint{2.504296in}{0.761934in}}%
\pgfpathlineto{\pgfqpoint{2.504860in}{0.761919in}}%
\pgfpathlineto{\pgfqpoint{2.505425in}{0.761903in}}%
\pgfpathlineto{\pgfqpoint{2.505989in}{0.761888in}}%
\pgfpathlineto{\pgfqpoint{2.506553in}{0.761873in}}%
\pgfpathlineto{\pgfqpoint{2.507118in}{0.761857in}}%
\pgfpathlineto{\pgfqpoint{2.507682in}{0.761842in}}%
\pgfpathlineto{\pgfqpoint{2.508246in}{0.761827in}}%
\pgfpathlineto{\pgfqpoint{2.508811in}{0.761809in}}%
\pgfpathlineto{\pgfqpoint{2.509375in}{0.761541in}}%
\pgfpathlineto{\pgfqpoint{2.509939in}{0.761108in}}%
\pgfpathlineto{\pgfqpoint{2.510504in}{0.760675in}}%
\pgfpathlineto{\pgfqpoint{2.511068in}{0.760241in}}%
\pgfpathlineto{\pgfqpoint{2.511632in}{0.759808in}}%
\pgfpathlineto{\pgfqpoint{2.512197in}{0.759374in}}%
\pgfpathlineto{\pgfqpoint{2.512761in}{0.758941in}}%
\pgfpathlineto{\pgfqpoint{2.513325in}{0.758507in}}%
\pgfpathlineto{\pgfqpoint{2.513890in}{0.758074in}}%
\pgfpathlineto{\pgfqpoint{2.514454in}{0.757640in}}%
\pgfpathlineto{\pgfqpoint{2.515018in}{0.757207in}}%
\pgfpathlineto{\pgfqpoint{2.515583in}{0.756773in}}%
\pgfpathlineto{\pgfqpoint{2.516147in}{0.756340in}}%
\pgfpathlineto{\pgfqpoint{2.516711in}{0.755906in}}%
\pgfpathlineto{\pgfqpoint{2.517276in}{0.755473in}}%
\pgfpathlineto{\pgfqpoint{2.517840in}{0.755039in}}%
\pgfpathlineto{\pgfqpoint{2.518405in}{0.754606in}}%
\pgfpathlineto{\pgfqpoint{2.518969in}{0.754172in}}%
\pgfpathlineto{\pgfqpoint{2.519533in}{0.753739in}}%
\pgfpathlineto{\pgfqpoint{2.520098in}{0.753305in}}%
\pgfpathlineto{\pgfqpoint{2.520662in}{0.752872in}}%
\pgfpathlineto{\pgfqpoint{2.521226in}{0.752439in}}%
\pgfpathlineto{\pgfqpoint{2.521791in}{0.752005in}}%
\pgfpathlineto{\pgfqpoint{2.522355in}{0.751572in}}%
\pgfpathlineto{\pgfqpoint{2.522919in}{0.751138in}}%
\pgfpathlineto{\pgfqpoint{2.523484in}{0.750705in}}%
\pgfpathlineto{\pgfqpoint{2.524048in}{0.750319in}}%
\pgfpathlineto{\pgfqpoint{2.524612in}{0.750728in}}%
\pgfpathlineto{\pgfqpoint{2.525177in}{0.751419in}}%
\pgfpathlineto{\pgfqpoint{2.525741in}{0.752110in}}%
\pgfpathlineto{\pgfqpoint{2.526305in}{0.752801in}}%
\pgfpathlineto{\pgfqpoint{2.526870in}{0.753492in}}%
\pgfpathlineto{\pgfqpoint{2.527434in}{0.754183in}}%
\pgfpathlineto{\pgfqpoint{2.527998in}{0.754873in}}%
\pgfpathlineto{\pgfqpoint{2.528563in}{0.755564in}}%
\pgfpathlineto{\pgfqpoint{2.529127in}{0.756255in}}%
\pgfpathlineto{\pgfqpoint{2.529691in}{0.756946in}}%
\pgfpathlineto{\pgfqpoint{2.530256in}{0.757637in}}%
\pgfpathlineto{\pgfqpoint{2.530820in}{0.758328in}}%
\pgfpathlineto{\pgfqpoint{2.531384in}{0.759018in}}%
\pgfpathlineto{\pgfqpoint{2.531949in}{0.759709in}}%
\pgfpathlineto{\pgfqpoint{2.532513in}{0.760400in}}%
\pgfpathlineto{\pgfqpoint{2.533077in}{0.761091in}}%
\pgfpathlineto{\pgfqpoint{2.533642in}{0.757106in}}%
\pgfpathlineto{\pgfqpoint{2.534206in}{0.751600in}}%
\pgfpathlineto{\pgfqpoint{2.534771in}{0.751624in}}%
\pgfpathlineto{\pgfqpoint{2.535335in}{0.751647in}}%
\pgfpathlineto{\pgfqpoint{2.535899in}{0.751670in}}%
\pgfpathlineto{\pgfqpoint{2.536464in}{0.751693in}}%
\pgfpathlineto{\pgfqpoint{2.537028in}{0.751703in}}%
\pgfpathlineto{\pgfqpoint{2.537592in}{0.751530in}}%
\pgfpathlineto{\pgfqpoint{2.538157in}{0.751580in}}%
\pgfpathlineto{\pgfqpoint{2.538721in}{0.752352in}}%
\pgfpathlineto{\pgfqpoint{2.539285in}{0.753306in}}%
\pgfpathlineto{\pgfqpoint{2.539850in}{0.754261in}}%
\pgfpathlineto{\pgfqpoint{2.540414in}{0.755215in}}%
\pgfpathlineto{\pgfqpoint{2.540978in}{0.756169in}}%
\pgfpathlineto{\pgfqpoint{2.541543in}{0.757124in}}%
\pgfpathlineto{\pgfqpoint{2.542107in}{0.758078in}}%
\pgfpathlineto{\pgfqpoint{2.542671in}{0.759033in}}%
\pgfpathlineto{\pgfqpoint{2.543236in}{0.759987in}}%
\pgfpathlineto{\pgfqpoint{2.543800in}{0.760941in}}%
\pgfpathlineto{\pgfqpoint{2.544364in}{0.761896in}}%
\pgfpathlineto{\pgfqpoint{2.544929in}{0.762850in}}%
\pgfpathlineto{\pgfqpoint{2.545493in}{0.763805in}}%
\pgfpathlineto{\pgfqpoint{2.546057in}{0.764215in}}%
\pgfpathlineto{\pgfqpoint{2.546622in}{0.752748in}}%
\pgfpathlineto{\pgfqpoint{2.547186in}{0.751677in}}%
\pgfpathlineto{\pgfqpoint{2.547750in}{0.753389in}}%
\pgfpathlineto{\pgfqpoint{2.548315in}{0.763070in}}%
\pgfpathlineto{\pgfqpoint{2.548879in}{0.765002in}}%
\pgfpathlineto{\pgfqpoint{2.549444in}{0.760202in}}%
\pgfpathlineto{\pgfqpoint{2.550008in}{0.758310in}}%
\pgfpathlineto{\pgfqpoint{2.550572in}{0.750235in}}%
\pgfpathlineto{\pgfqpoint{2.551137in}{0.752268in}}%
\pgfpathlineto{\pgfqpoint{2.551701in}{0.753742in}}%
\pgfpathlineto{\pgfqpoint{2.552265in}{0.754678in}}%
\pgfpathlineto{\pgfqpoint{2.552830in}{0.755614in}}%
\pgfpathlineto{\pgfqpoint{2.553394in}{0.756550in}}%
\pgfpathlineto{\pgfqpoint{2.553958in}{0.757486in}}%
\pgfpathlineto{\pgfqpoint{2.554523in}{0.758421in}}%
\pgfpathlineto{\pgfqpoint{2.555087in}{0.759357in}}%
\pgfpathlineto{\pgfqpoint{2.555651in}{0.760293in}}%
\pgfpathlineto{\pgfqpoint{2.556216in}{0.761229in}}%
\pgfpathlineto{\pgfqpoint{2.556780in}{0.762164in}}%
\pgfpathlineto{\pgfqpoint{2.557344in}{0.763100in}}%
\pgfpathlineto{\pgfqpoint{2.557909in}{0.764036in}}%
\pgfpathlineto{\pgfqpoint{2.558473in}{0.764972in}}%
\pgfpathlineto{\pgfqpoint{2.559037in}{0.765907in}}%
\pgfpathlineto{\pgfqpoint{2.559602in}{0.766590in}}%
\pgfpathlineto{\pgfqpoint{2.560166in}{0.762590in}}%
\pgfpathlineto{\pgfqpoint{2.560730in}{0.756831in}}%
\pgfpathlineto{\pgfqpoint{2.561295in}{0.752316in}}%
\pgfpathlineto{\pgfqpoint{2.561859in}{0.752308in}}%
\pgfpathlineto{\pgfqpoint{2.562423in}{0.753968in}}%
\pgfpathlineto{\pgfqpoint{2.562988in}{0.766786in}}%
\pgfpathlineto{\pgfqpoint{2.563552in}{0.766668in}}%
\pgfpathlineto{\pgfqpoint{2.564117in}{0.766370in}}%
\pgfpathlineto{\pgfqpoint{2.564681in}{0.766072in}}%
\pgfpathlineto{\pgfqpoint{2.565245in}{0.765774in}}%
\pgfpathlineto{\pgfqpoint{2.565810in}{0.765476in}}%
\pgfpathlineto{\pgfqpoint{2.566374in}{0.765178in}}%
\pgfpathlineto{\pgfqpoint{2.566938in}{0.764880in}}%
\pgfpathlineto{\pgfqpoint{2.567503in}{0.764582in}}%
\pgfpathlineto{\pgfqpoint{2.568067in}{0.764284in}}%
\pgfpathlineto{\pgfqpoint{2.568631in}{0.763986in}}%
\pgfpathlineto{\pgfqpoint{2.569196in}{0.763688in}}%
\pgfpathlineto{\pgfqpoint{2.569760in}{0.763390in}}%
\pgfpathlineto{\pgfqpoint{2.570324in}{0.763092in}}%
\pgfpathlineto{\pgfqpoint{2.570889in}{0.762794in}}%
\pgfpathlineto{\pgfqpoint{2.571453in}{0.762496in}}%
\pgfpathlineto{\pgfqpoint{2.572017in}{0.762198in}}%
\pgfpathlineto{\pgfqpoint{2.572582in}{0.761900in}}%
\pgfpathlineto{\pgfqpoint{2.573146in}{0.761602in}}%
\pgfpathlineto{\pgfqpoint{2.573710in}{0.761304in}}%
\pgfpathlineto{\pgfqpoint{2.574275in}{0.761006in}}%
\pgfpathlineto{\pgfqpoint{2.574839in}{0.760708in}}%
\pgfpathlineto{\pgfqpoint{2.575403in}{0.760410in}}%
\pgfpathlineto{\pgfqpoint{2.575968in}{0.760112in}}%
\pgfpathlineto{\pgfqpoint{2.576532in}{0.759814in}}%
\pgfpathlineto{\pgfqpoint{2.577096in}{0.759516in}}%
\pgfpathlineto{\pgfqpoint{2.577661in}{0.759218in}}%
\pgfpathlineto{\pgfqpoint{2.578225in}{0.758920in}}%
\pgfpathlineto{\pgfqpoint{2.578789in}{0.758622in}}%
\pgfpathlineto{\pgfqpoint{2.579354in}{0.758324in}}%
\pgfpathlineto{\pgfqpoint{2.579918in}{0.758026in}}%
\pgfpathlineto{\pgfqpoint{2.580483in}{0.757728in}}%
\pgfpathlineto{\pgfqpoint{2.581047in}{0.757430in}}%
\pgfpathlineto{\pgfqpoint{2.581611in}{0.757132in}}%
\pgfpathlineto{\pgfqpoint{2.582176in}{0.756834in}}%
\pgfpathlineto{\pgfqpoint{2.582740in}{0.756536in}}%
\pgfpathlineto{\pgfqpoint{2.583304in}{0.756238in}}%
\pgfpathlineto{\pgfqpoint{2.583869in}{0.755940in}}%
\pgfpathlineto{\pgfqpoint{2.584433in}{0.755642in}}%
\pgfpathlineto{\pgfqpoint{2.584997in}{0.755344in}}%
\pgfpathlineto{\pgfqpoint{2.585562in}{0.755046in}}%
\pgfpathlineto{\pgfqpoint{2.586126in}{0.754748in}}%
\pgfpathlineto{\pgfqpoint{2.586690in}{0.754450in}}%
\pgfpathlineto{\pgfqpoint{2.587255in}{0.754152in}}%
\pgfpathlineto{\pgfqpoint{2.587819in}{0.753854in}}%
\pgfpathlineto{\pgfqpoint{2.588383in}{0.753556in}}%
\pgfpathlineto{\pgfqpoint{2.588948in}{0.753258in}}%
\pgfpathlineto{\pgfqpoint{2.589512in}{0.752960in}}%
\pgfpathlineto{\pgfqpoint{2.590076in}{0.752662in}}%
\pgfpathlineto{\pgfqpoint{2.590641in}{0.752364in}}%
\pgfpathlineto{\pgfqpoint{2.591205in}{0.752066in}}%
\pgfpathlineto{\pgfqpoint{2.591769in}{0.751768in}}%
\pgfpathlineto{\pgfqpoint{2.592334in}{0.751470in}}%
\pgfpathlineto{\pgfqpoint{2.592898in}{0.751172in}}%
\pgfpathlineto{\pgfqpoint{2.593462in}{0.750874in}}%
\pgfpathlineto{\pgfqpoint{2.594027in}{0.750576in}}%
\pgfpathlineto{\pgfqpoint{2.594591in}{0.750278in}}%
\pgfpathlineto{\pgfqpoint{2.595156in}{0.749980in}}%
\pgfpathlineto{\pgfqpoint{2.595720in}{0.749682in}}%
\pgfpathlineto{\pgfqpoint{2.596284in}{0.749384in}}%
\pgfpathlineto{\pgfqpoint{2.596849in}{0.749086in}}%
\pgfpathlineto{\pgfqpoint{2.597413in}{0.748788in}}%
\pgfpathlineto{\pgfqpoint{2.597977in}{0.748490in}}%
\pgfpathlineto{\pgfqpoint{2.598542in}{0.748192in}}%
\pgfpathlineto{\pgfqpoint{2.599106in}{0.747894in}}%
\pgfpathlineto{\pgfqpoint{2.599670in}{0.747596in}}%
\pgfpathlineto{\pgfqpoint{2.600235in}{0.747298in}}%
\pgfpathlineto{\pgfqpoint{2.600799in}{0.747000in}}%
\pgfpathlineto{\pgfqpoint{2.601363in}{0.746702in}}%
\pgfpathlineto{\pgfqpoint{2.601928in}{0.748508in}}%
\pgfpathlineto{\pgfqpoint{2.602492in}{0.752946in}}%
\pgfpathlineto{\pgfqpoint{2.603056in}{0.757392in}}%
\pgfpathlineto{\pgfqpoint{2.603621in}{0.761767in}}%
\pgfpathlineto{\pgfqpoint{2.604185in}{0.756737in}}%
\pgfpathlineto{\pgfqpoint{2.604749in}{0.760831in}}%
\pgfpathlineto{\pgfqpoint{2.605314in}{0.766957in}}%
\pgfpathlineto{\pgfqpoint{2.605878in}{0.766671in}}%
\pgfpathlineto{\pgfqpoint{2.606442in}{0.766364in}}%
\pgfpathlineto{\pgfqpoint{2.607007in}{0.766057in}}%
\pgfpathlineto{\pgfqpoint{2.607571in}{0.765751in}}%
\pgfpathlineto{\pgfqpoint{2.608135in}{0.765444in}}%
\pgfpathlineto{\pgfqpoint{2.608700in}{0.755890in}}%
\pgfpathlineto{\pgfqpoint{2.609264in}{0.753997in}}%
\pgfpathlineto{\pgfqpoint{2.609829in}{0.753741in}}%
\pgfpathlineto{\pgfqpoint{2.610393in}{0.753484in}}%
\pgfpathlineto{\pgfqpoint{2.610957in}{0.753228in}}%
\pgfpathlineto{\pgfqpoint{2.611522in}{0.752971in}}%
\pgfpathlineto{\pgfqpoint{2.612086in}{0.752714in}}%
\pgfpathlineto{\pgfqpoint{2.612650in}{0.752458in}}%
\pgfpathlineto{\pgfqpoint{2.613215in}{0.752201in}}%
\pgfpathlineto{\pgfqpoint{2.613779in}{0.751945in}}%
\pgfpathlineto{\pgfqpoint{2.614343in}{0.751936in}}%
\pgfpathlineto{\pgfqpoint{2.614908in}{0.744790in}}%
\pgfpathlineto{\pgfqpoint{2.615472in}{0.750905in}}%
\pgfpathlineto{\pgfqpoint{2.616036in}{0.738211in}}%
\pgfpathlineto{\pgfqpoint{2.616601in}{0.749018in}}%
\pgfpathlineto{\pgfqpoint{2.617165in}{0.753834in}}%
\pgfpathlineto{\pgfqpoint{2.617729in}{0.754234in}}%
\pgfpathlineto{\pgfqpoint{2.618294in}{0.753787in}}%
\pgfpathlineto{\pgfqpoint{2.618858in}{0.754256in}}%
\pgfpathlineto{\pgfqpoint{2.619422in}{0.754108in}}%
\pgfpathlineto{\pgfqpoint{2.619987in}{0.754141in}}%
\pgfpathlineto{\pgfqpoint{2.620551in}{0.754175in}}%
\pgfpathlineto{\pgfqpoint{2.621115in}{0.754208in}}%
\pgfpathlineto{\pgfqpoint{2.621680in}{0.754242in}}%
\pgfpathlineto{\pgfqpoint{2.622244in}{0.754276in}}%
\pgfpathlineto{\pgfqpoint{2.622808in}{0.754309in}}%
\pgfpathlineto{\pgfqpoint{2.623373in}{0.754343in}}%
\pgfpathlineto{\pgfqpoint{2.623937in}{0.754377in}}%
\pgfpathlineto{\pgfqpoint{2.624501in}{0.754410in}}%
\pgfpathlineto{\pgfqpoint{2.625066in}{0.754444in}}%
\pgfpathlineto{\pgfqpoint{2.625630in}{0.754478in}}%
\pgfpathlineto{\pgfqpoint{2.626195in}{0.754511in}}%
\pgfpathlineto{\pgfqpoint{2.626759in}{0.754545in}}%
\pgfpathlineto{\pgfqpoint{2.627323in}{0.754579in}}%
\pgfpathlineto{\pgfqpoint{2.627888in}{0.754612in}}%
\pgfpathlineto{\pgfqpoint{2.628452in}{0.754649in}}%
\pgfpathlineto{\pgfqpoint{2.629016in}{0.754811in}}%
\pgfpathlineto{\pgfqpoint{2.629581in}{0.755027in}}%
\pgfpathlineto{\pgfqpoint{2.630145in}{0.755243in}}%
\pgfpathlineto{\pgfqpoint{2.630709in}{0.755459in}}%
\pgfpathlineto{\pgfqpoint{2.631274in}{0.755675in}}%
\pgfpathlineto{\pgfqpoint{2.631838in}{0.755891in}}%
\pgfpathlineto{\pgfqpoint{2.632402in}{0.756039in}}%
\pgfpathlineto{\pgfqpoint{2.632967in}{0.755841in}}%
\pgfpathlineto{\pgfqpoint{2.633531in}{0.755595in}}%
\pgfpathlineto{\pgfqpoint{2.634095in}{0.755349in}}%
\pgfpathlineto{\pgfqpoint{2.634660in}{0.755103in}}%
\pgfpathlineto{\pgfqpoint{2.635224in}{0.754857in}}%
\pgfpathlineto{\pgfqpoint{2.635788in}{0.754611in}}%
\pgfpathlineto{\pgfqpoint{2.636353in}{0.754365in}}%
\pgfpathlineto{\pgfqpoint{2.636917in}{0.754119in}}%
\pgfpathlineto{\pgfqpoint{2.637481in}{0.753873in}}%
\pgfpathlineto{\pgfqpoint{2.638046in}{0.753627in}}%
\pgfpathlineto{\pgfqpoint{2.638610in}{0.753381in}}%
\pgfpathlineto{\pgfqpoint{2.639174in}{0.753135in}}%
\pgfpathlineto{\pgfqpoint{2.639739in}{0.752889in}}%
\pgfpathlineto{\pgfqpoint{2.640303in}{0.752642in}}%
\pgfpathlineto{\pgfqpoint{2.640868in}{0.752396in}}%
\pgfpathlineto{\pgfqpoint{2.641432in}{0.752150in}}%
\pgfpathlineto{\pgfqpoint{2.641996in}{0.752110in}}%
\pgfpathlineto{\pgfqpoint{2.642561in}{0.752721in}}%
\pgfpathlineto{\pgfqpoint{2.643125in}{0.753379in}}%
\pgfpathlineto{\pgfqpoint{2.643689in}{0.754038in}}%
\pgfpathlineto{\pgfqpoint{2.644254in}{0.756046in}}%
\pgfpathlineto{\pgfqpoint{2.644818in}{0.755500in}}%
\pgfpathlineto{\pgfqpoint{2.645382in}{0.754626in}}%
\pgfpathlineto{\pgfqpoint{2.645947in}{0.754880in}}%
\pgfpathlineto{\pgfqpoint{2.646511in}{0.755137in}}%
\pgfpathlineto{\pgfqpoint{2.647075in}{0.755393in}}%
\pgfpathlineto{\pgfqpoint{2.647640in}{0.755650in}}%
\pgfpathlineto{\pgfqpoint{2.648204in}{0.755906in}}%
\pgfpathlineto{\pgfqpoint{2.648768in}{0.756162in}}%
\pgfpathlineto{\pgfqpoint{2.649333in}{0.756419in}}%
\pgfpathlineto{\pgfqpoint{2.649897in}{0.756675in}}%
\pgfpathlineto{\pgfqpoint{2.650461in}{0.756932in}}%
\pgfpathlineto{\pgfqpoint{2.651026in}{0.757188in}}%
\pgfpathlineto{\pgfqpoint{2.651590in}{0.757445in}}%
\pgfpathlineto{\pgfqpoint{2.652154in}{0.757701in}}%
\pgfpathlineto{\pgfqpoint{2.652719in}{0.757957in}}%
\pgfpathlineto{\pgfqpoint{2.653283in}{0.758214in}}%
\pgfpathlineto{\pgfqpoint{2.653847in}{0.758470in}}%
\pgfpathlineto{\pgfqpoint{2.654412in}{0.758727in}}%
\pgfpathlineto{\pgfqpoint{2.654976in}{0.758983in}}%
\pgfpathlineto{\pgfqpoint{2.655541in}{0.759240in}}%
\pgfpathlineto{\pgfqpoint{2.656105in}{0.759496in}}%
\pgfpathlineto{\pgfqpoint{2.656669in}{0.759753in}}%
\pgfpathlineto{\pgfqpoint{2.657234in}{0.760009in}}%
\pgfpathlineto{\pgfqpoint{2.657798in}{0.760123in}}%
\pgfpathlineto{\pgfqpoint{2.658362in}{0.760058in}}%
\pgfpathlineto{\pgfqpoint{2.658927in}{0.759992in}}%
\pgfpathlineto{\pgfqpoint{2.659491in}{0.759927in}}%
\pgfpathlineto{\pgfqpoint{2.660055in}{0.759861in}}%
\pgfpathlineto{\pgfqpoint{2.660620in}{0.759796in}}%
\pgfpathlineto{\pgfqpoint{2.661184in}{0.759730in}}%
\pgfpathlineto{\pgfqpoint{2.661748in}{0.759665in}}%
\pgfpathlineto{\pgfqpoint{2.662313in}{0.759599in}}%
\pgfpathlineto{\pgfqpoint{2.662877in}{0.759534in}}%
\pgfpathlineto{\pgfqpoint{2.663441in}{0.759468in}}%
\pgfpathlineto{\pgfqpoint{2.664006in}{0.759403in}}%
\pgfpathlineto{\pgfqpoint{2.664570in}{0.759337in}}%
\pgfpathlineto{\pgfqpoint{2.665134in}{0.759272in}}%
\pgfpathlineto{\pgfqpoint{2.665699in}{0.759206in}}%
\pgfpathlineto{\pgfqpoint{2.666263in}{0.759141in}}%
\pgfpathlineto{\pgfqpoint{2.666827in}{0.759075in}}%
\pgfpathlineto{\pgfqpoint{2.667392in}{0.759010in}}%
\pgfpathlineto{\pgfqpoint{2.667956in}{0.758944in}}%
\pgfpathlineto{\pgfqpoint{2.668520in}{0.758879in}}%
\pgfpathlineto{\pgfqpoint{2.669085in}{0.758813in}}%
\pgfpathlineto{\pgfqpoint{2.669649in}{0.758748in}}%
\pgfpathlineto{\pgfqpoint{2.670213in}{0.758682in}}%
\pgfpathlineto{\pgfqpoint{2.670778in}{0.758617in}}%
\pgfpathlineto{\pgfqpoint{2.671342in}{0.758551in}}%
\pgfpathlineto{\pgfqpoint{2.671907in}{0.758486in}}%
\pgfpathlineto{\pgfqpoint{2.672471in}{0.758420in}}%
\pgfpathlineto{\pgfqpoint{2.673035in}{0.758355in}}%
\pgfpathlineto{\pgfqpoint{2.673600in}{0.758289in}}%
\pgfpathlineto{\pgfqpoint{2.674164in}{0.758224in}}%
\pgfpathlineto{\pgfqpoint{2.674728in}{0.758158in}}%
\pgfpathlineto{\pgfqpoint{2.675293in}{0.758092in}}%
\pgfpathlineto{\pgfqpoint{2.675857in}{0.758027in}}%
\pgfpathlineto{\pgfqpoint{2.676421in}{0.757961in}}%
\pgfpathlineto{\pgfqpoint{2.676986in}{0.757896in}}%
\pgfpathlineto{\pgfqpoint{2.677550in}{0.757830in}}%
\pgfpathlineto{\pgfqpoint{2.678114in}{0.757765in}}%
\pgfpathlineto{\pgfqpoint{2.678679in}{0.757699in}}%
\pgfpathlineto{\pgfqpoint{2.679243in}{0.757634in}}%
\pgfpathlineto{\pgfqpoint{2.679807in}{0.757568in}}%
\pgfpathlineto{\pgfqpoint{2.680372in}{0.757503in}}%
\pgfpathlineto{\pgfqpoint{2.680936in}{0.757437in}}%
\pgfpathlineto{\pgfqpoint{2.681500in}{0.757372in}}%
\pgfpathlineto{\pgfqpoint{2.682065in}{0.757306in}}%
\pgfpathlineto{\pgfqpoint{2.682629in}{0.757241in}}%
\pgfpathlineto{\pgfqpoint{2.683193in}{0.757175in}}%
\pgfpathlineto{\pgfqpoint{2.683758in}{0.757110in}}%
\pgfpathlineto{\pgfqpoint{2.684322in}{0.757044in}}%
\pgfpathlineto{\pgfqpoint{2.684886in}{0.756979in}}%
\pgfpathlineto{\pgfqpoint{2.685451in}{0.756913in}}%
\pgfpathlineto{\pgfqpoint{2.686015in}{0.756848in}}%
\pgfpathlineto{\pgfqpoint{2.686580in}{0.756782in}}%
\pgfpathlineto{\pgfqpoint{2.687144in}{0.756717in}}%
\pgfpathlineto{\pgfqpoint{2.687708in}{0.756651in}}%
\pgfpathlineto{\pgfqpoint{2.688273in}{0.756586in}}%
\pgfpathlineto{\pgfqpoint{2.688837in}{0.756520in}}%
\pgfpathlineto{\pgfqpoint{2.689401in}{0.756455in}}%
\pgfpathlineto{\pgfqpoint{2.689966in}{0.756389in}}%
\pgfpathlineto{\pgfqpoint{2.690530in}{0.756324in}}%
\pgfpathlineto{\pgfqpoint{2.691094in}{0.756258in}}%
\pgfpathlineto{\pgfqpoint{2.691659in}{0.756193in}}%
\pgfpathlineto{\pgfqpoint{2.692223in}{0.756127in}}%
\pgfpathlineto{\pgfqpoint{2.692787in}{0.756062in}}%
\pgfpathlineto{\pgfqpoint{2.693352in}{0.755996in}}%
\pgfpathlineto{\pgfqpoint{2.693916in}{0.755931in}}%
\pgfpathlineto{\pgfqpoint{2.694480in}{0.755865in}}%
\pgfpathlineto{\pgfqpoint{2.695045in}{0.755800in}}%
\pgfpathlineto{\pgfqpoint{2.695609in}{0.755734in}}%
\pgfpathlineto{\pgfqpoint{2.696173in}{0.755669in}}%
\pgfpathlineto{\pgfqpoint{2.696738in}{0.755603in}}%
\pgfpathlineto{\pgfqpoint{2.697302in}{0.755538in}}%
\pgfpathlineto{\pgfqpoint{2.697866in}{0.755472in}}%
\pgfpathlineto{\pgfqpoint{2.698431in}{0.755407in}}%
\pgfpathlineto{\pgfqpoint{2.698995in}{0.755341in}}%
\pgfpathlineto{\pgfqpoint{2.699559in}{0.755276in}}%
\pgfpathlineto{\pgfqpoint{2.700124in}{0.755210in}}%
\pgfpathlineto{\pgfqpoint{2.700688in}{0.755145in}}%
\pgfpathlineto{\pgfqpoint{2.701253in}{0.755079in}}%
\pgfpathlineto{\pgfqpoint{2.701817in}{0.755014in}}%
\pgfpathlineto{\pgfqpoint{2.702381in}{0.754948in}}%
\pgfpathlineto{\pgfqpoint{2.702946in}{0.754883in}}%
\pgfpathlineto{\pgfqpoint{2.703510in}{0.754817in}}%
\pgfpathlineto{\pgfqpoint{2.704074in}{0.754752in}}%
\pgfpathlineto{\pgfqpoint{2.704639in}{0.754686in}}%
\pgfpathlineto{\pgfqpoint{2.705203in}{0.754621in}}%
\pgfpathlineto{\pgfqpoint{2.705767in}{0.754555in}}%
\pgfpathlineto{\pgfqpoint{2.706332in}{0.754490in}}%
\pgfpathlineto{\pgfqpoint{2.706896in}{0.754424in}}%
\pgfpathlineto{\pgfqpoint{2.707460in}{0.754359in}}%
\pgfpathlineto{\pgfqpoint{2.708025in}{0.754293in}}%
\pgfpathlineto{\pgfqpoint{2.708589in}{0.754228in}}%
\pgfpathlineto{\pgfqpoint{2.709153in}{0.754162in}}%
\pgfpathlineto{\pgfqpoint{2.709718in}{0.754097in}}%
\pgfpathlineto{\pgfqpoint{2.710282in}{0.754031in}}%
\pgfpathlineto{\pgfqpoint{2.710846in}{0.753966in}}%
\pgfpathlineto{\pgfqpoint{2.711411in}{0.755149in}}%
\pgfpathlineto{\pgfqpoint{2.711975in}{0.757625in}}%
\pgfpathlineto{\pgfqpoint{2.712539in}{0.759920in}}%
\pgfpathlineto{\pgfqpoint{2.713104in}{0.760201in}}%
\pgfpathlineto{\pgfqpoint{2.713668in}{0.760345in}}%
\pgfpathlineto{\pgfqpoint{2.714232in}{0.759957in}}%
\pgfpathlineto{\pgfqpoint{2.714797in}{0.759552in}}%
\pgfpathlineto{\pgfqpoint{2.715361in}{0.759147in}}%
\pgfpathlineto{\pgfqpoint{2.715926in}{0.758742in}}%
\pgfpathlineto{\pgfqpoint{2.716490in}{0.758337in}}%
\pgfpathlineto{\pgfqpoint{2.717054in}{0.757932in}}%
\pgfpathlineto{\pgfqpoint{2.717619in}{0.757527in}}%
\pgfpathlineto{\pgfqpoint{2.718183in}{0.757122in}}%
\pgfpathlineto{\pgfqpoint{2.718747in}{0.756716in}}%
\pgfpathlineto{\pgfqpoint{2.719312in}{0.756311in}}%
\pgfpathlineto{\pgfqpoint{2.719876in}{0.755906in}}%
\pgfpathlineto{\pgfqpoint{2.720440in}{0.755501in}}%
\pgfpathlineto{\pgfqpoint{2.721005in}{0.755096in}}%
\pgfpathlineto{\pgfqpoint{2.721569in}{0.754691in}}%
\pgfpathlineto{\pgfqpoint{2.722133in}{0.754286in}}%
\pgfpathlineto{\pgfqpoint{2.722698in}{0.754114in}}%
\pgfpathlineto{\pgfqpoint{2.723262in}{0.754636in}}%
\pgfpathlineto{\pgfqpoint{2.723826in}{0.755205in}}%
\pgfpathlineto{\pgfqpoint{2.724391in}{0.755774in}}%
\pgfpathlineto{\pgfqpoint{2.724955in}{0.756343in}}%
\pgfpathlineto{\pgfqpoint{2.725519in}{0.756912in}}%
\pgfpathlineto{\pgfqpoint{2.726084in}{0.757481in}}%
\pgfpathlineto{\pgfqpoint{2.726648in}{0.759646in}}%
\pgfpathlineto{\pgfqpoint{2.727212in}{0.765068in}}%
\pgfpathlineto{\pgfqpoint{2.727777in}{0.764477in}}%
\pgfpathlineto{\pgfqpoint{2.728341in}{0.763876in}}%
\pgfpathlineto{\pgfqpoint{2.728905in}{0.763276in}}%
\pgfpathlineto{\pgfqpoint{2.729470in}{0.762676in}}%
\pgfpathlineto{\pgfqpoint{2.730034in}{0.762076in}}%
\pgfpathlineto{\pgfqpoint{2.730598in}{0.761476in}}%
\pgfpathlineto{\pgfqpoint{2.731163in}{0.760876in}}%
\pgfpathlineto{\pgfqpoint{2.731727in}{0.760276in}}%
\pgfpathlineto{\pgfqpoint{2.732292in}{0.759676in}}%
\pgfpathlineto{\pgfqpoint{2.732856in}{0.759076in}}%
\pgfpathlineto{\pgfqpoint{2.733420in}{0.758476in}}%
\pgfpathlineto{\pgfqpoint{2.733985in}{0.757876in}}%
\pgfpathlineto{\pgfqpoint{2.734549in}{0.757276in}}%
\pgfpathlineto{\pgfqpoint{2.735113in}{0.756676in}}%
\pgfpathlineto{\pgfqpoint{2.735678in}{0.756076in}}%
\pgfpathlineto{\pgfqpoint{2.736242in}{0.755476in}}%
\pgfpathlineto{\pgfqpoint{2.736806in}{0.754876in}}%
\pgfpathlineto{\pgfqpoint{2.737371in}{0.754276in}}%
\pgfpathlineto{\pgfqpoint{2.737935in}{0.753676in}}%
\pgfpathlineto{\pgfqpoint{2.738499in}{0.753959in}}%
\pgfpathlineto{\pgfqpoint{2.739064in}{0.757195in}}%
\pgfpathlineto{\pgfqpoint{2.739628in}{0.760669in}}%
\pgfpathlineto{\pgfqpoint{2.740192in}{0.764143in}}%
\pgfpathlineto{\pgfqpoint{2.740757in}{0.765964in}}%
\pgfpathlineto{\pgfqpoint{2.741321in}{0.768663in}}%
\pgfpathlineto{\pgfqpoint{2.741885in}{0.767662in}}%
\pgfpathlineto{\pgfqpoint{2.742450in}{0.766614in}}%
\pgfpathlineto{\pgfqpoint{2.743014in}{0.765565in}}%
\pgfpathlineto{\pgfqpoint{2.743578in}{0.764516in}}%
\pgfpathlineto{\pgfqpoint{2.744143in}{0.763468in}}%
\pgfpathlineto{\pgfqpoint{2.744707in}{0.762419in}}%
\pgfpathlineto{\pgfqpoint{2.745271in}{0.761370in}}%
\pgfpathlineto{\pgfqpoint{2.745836in}{0.760322in}}%
\pgfpathlineto{\pgfqpoint{2.746400in}{0.759273in}}%
\pgfpathlineto{\pgfqpoint{2.746965in}{0.758224in}}%
\pgfpathlineto{\pgfqpoint{2.747529in}{0.757175in}}%
\pgfpathlineto{\pgfqpoint{2.748093in}{0.756127in}}%
\pgfpathlineto{\pgfqpoint{2.748658in}{0.755078in}}%
\pgfpathlineto{\pgfqpoint{2.749222in}{0.754029in}}%
\pgfpathlineto{\pgfqpoint{2.749786in}{0.753498in}}%
\pgfpathlineto{\pgfqpoint{2.750351in}{0.753612in}}%
\pgfpathlineto{\pgfqpoint{2.750915in}{0.753650in}}%
\pgfpathlineto{\pgfqpoint{2.751479in}{0.753645in}}%
\pgfpathlineto{\pgfqpoint{2.752044in}{0.753638in}}%
\pgfpathlineto{\pgfqpoint{2.752608in}{0.753630in}}%
\pgfpathlineto{\pgfqpoint{2.753172in}{0.753647in}}%
\pgfpathlineto{\pgfqpoint{2.753737in}{0.753515in}}%
\pgfpathlineto{\pgfqpoint{2.754301in}{0.753451in}}%
\pgfpathlineto{\pgfqpoint{2.754865in}{0.753453in}}%
\pgfpathlineto{\pgfqpoint{2.755430in}{0.753455in}}%
\pgfpathlineto{\pgfqpoint{2.755994in}{0.753457in}}%
\pgfpathlineto{\pgfqpoint{2.756558in}{0.753459in}}%
\pgfpathlineto{\pgfqpoint{2.757123in}{0.753461in}}%
\pgfpathlineto{\pgfqpoint{2.757687in}{0.753464in}}%
\pgfpathlineto{\pgfqpoint{2.758251in}{0.753466in}}%
\pgfpathlineto{\pgfqpoint{2.758816in}{0.753468in}}%
\pgfpathlineto{\pgfqpoint{2.759380in}{0.753470in}}%
\pgfpathlineto{\pgfqpoint{2.759944in}{0.753472in}}%
\pgfpathlineto{\pgfqpoint{2.760509in}{0.753474in}}%
\pgfpathlineto{\pgfqpoint{2.761073in}{0.753476in}}%
\pgfpathlineto{\pgfqpoint{2.761638in}{0.753479in}}%
\pgfpathlineto{\pgfqpoint{2.762202in}{0.753481in}}%
\pgfpathlineto{\pgfqpoint{2.762766in}{0.753483in}}%
\pgfpathlineto{\pgfqpoint{2.763331in}{0.753485in}}%
\pgfpathlineto{\pgfqpoint{2.763895in}{0.753487in}}%
\pgfpathlineto{\pgfqpoint{2.764459in}{0.753489in}}%
\pgfpathlineto{\pgfqpoint{2.765024in}{0.753491in}}%
\pgfpathlineto{\pgfqpoint{2.765588in}{0.753494in}}%
\pgfpathlineto{\pgfqpoint{2.766152in}{0.753496in}}%
\pgfpathlineto{\pgfqpoint{2.766717in}{0.753498in}}%
\pgfpathlineto{\pgfqpoint{2.767281in}{0.753500in}}%
\pgfpathlineto{\pgfqpoint{2.767845in}{0.753502in}}%
\pgfpathlineto{\pgfqpoint{2.768410in}{0.753504in}}%
\pgfpathlineto{\pgfqpoint{2.768974in}{0.753507in}}%
\pgfpathlineto{\pgfqpoint{2.769538in}{0.753509in}}%
\pgfpathlineto{\pgfqpoint{2.770103in}{0.753511in}}%
\pgfpathlineto{\pgfqpoint{2.770667in}{0.753513in}}%
\pgfpathlineto{\pgfqpoint{2.771231in}{0.753515in}}%
\pgfpathlineto{\pgfqpoint{2.771796in}{0.753517in}}%
\pgfpathlineto{\pgfqpoint{2.772360in}{0.753519in}}%
\pgfpathlineto{\pgfqpoint{2.772924in}{0.753522in}}%
\pgfpathlineto{\pgfqpoint{2.773489in}{0.753524in}}%
\pgfpathlineto{\pgfqpoint{2.774053in}{0.753526in}}%
\pgfpathlineto{\pgfqpoint{2.774617in}{0.753528in}}%
\pgfpathlineto{\pgfqpoint{2.775182in}{0.753530in}}%
\pgfpathlineto{\pgfqpoint{2.775746in}{0.753532in}}%
\pgfpathlineto{\pgfqpoint{2.776310in}{0.753534in}}%
\pgfpathlineto{\pgfqpoint{2.776875in}{0.753537in}}%
\pgfpathlineto{\pgfqpoint{2.777439in}{0.753539in}}%
\pgfpathlineto{\pgfqpoint{2.778004in}{0.753541in}}%
\pgfpathlineto{\pgfqpoint{2.778568in}{0.753543in}}%
\pgfpathlineto{\pgfqpoint{2.779132in}{0.753545in}}%
\pgfpathlineto{\pgfqpoint{2.779697in}{0.753547in}}%
\pgfpathlineto{\pgfqpoint{2.780261in}{0.753549in}}%
\pgfpathlineto{\pgfqpoint{2.780825in}{0.753552in}}%
\pgfpathlineto{\pgfqpoint{2.781390in}{0.753554in}}%
\pgfpathlineto{\pgfqpoint{2.781954in}{0.753556in}}%
\pgfpathlineto{\pgfqpoint{2.782518in}{0.753558in}}%
\pgfpathlineto{\pgfqpoint{2.783083in}{0.753560in}}%
\pgfpathlineto{\pgfqpoint{2.783647in}{0.753562in}}%
\pgfpathlineto{\pgfqpoint{2.784211in}{0.753564in}}%
\pgfpathlineto{\pgfqpoint{2.784776in}{0.753567in}}%
\pgfpathlineto{\pgfqpoint{2.785340in}{0.753569in}}%
\pgfpathlineto{\pgfqpoint{2.785904in}{0.753571in}}%
\pgfpathlineto{\pgfqpoint{2.786469in}{0.753573in}}%
\pgfpathlineto{\pgfqpoint{2.787033in}{0.753575in}}%
\pgfpathlineto{\pgfqpoint{2.787597in}{0.753577in}}%
\pgfpathlineto{\pgfqpoint{2.788162in}{0.753580in}}%
\pgfpathlineto{\pgfqpoint{2.788726in}{0.753582in}}%
\pgfpathlineto{\pgfqpoint{2.789290in}{0.753584in}}%
\pgfpathlineto{\pgfqpoint{2.789855in}{0.753586in}}%
\pgfpathlineto{\pgfqpoint{2.790419in}{0.753588in}}%
\pgfpathlineto{\pgfqpoint{2.790983in}{0.753590in}}%
\pgfpathlineto{\pgfqpoint{2.791548in}{0.753502in}}%
\pgfpathlineto{\pgfqpoint{2.792112in}{0.753497in}}%
\pgfpathlineto{\pgfqpoint{2.792677in}{0.753610in}}%
\pgfpathlineto{\pgfqpoint{2.793241in}{0.759873in}}%
\pgfpathlineto{\pgfqpoint{2.793805in}{0.769335in}}%
\pgfpathlineto{\pgfqpoint{2.794370in}{0.769459in}}%
\pgfpathlineto{\pgfqpoint{2.794934in}{0.769549in}}%
\pgfpathlineto{\pgfqpoint{2.795498in}{0.769475in}}%
\pgfpathlineto{\pgfqpoint{2.796063in}{0.769378in}}%
\pgfpathlineto{\pgfqpoint{2.796627in}{0.769280in}}%
\pgfpathlineto{\pgfqpoint{2.797191in}{0.769183in}}%
\pgfpathlineto{\pgfqpoint{2.797756in}{0.769086in}}%
\pgfpathlineto{\pgfqpoint{2.798320in}{0.768988in}}%
\pgfpathlineto{\pgfqpoint{2.798884in}{0.768891in}}%
\pgfpathlineto{\pgfqpoint{2.799449in}{0.768794in}}%
\pgfpathlineto{\pgfqpoint{2.800013in}{0.768696in}}%
\pgfpathlineto{\pgfqpoint{2.800577in}{0.768599in}}%
\pgfpathlineto{\pgfqpoint{2.801142in}{0.768502in}}%
\pgfpathlineto{\pgfqpoint{2.801706in}{0.768404in}}%
\pgfpathlineto{\pgfqpoint{2.802270in}{0.768307in}}%
\pgfpathlineto{\pgfqpoint{2.802835in}{0.768209in}}%
\pgfpathlineto{\pgfqpoint{2.803399in}{0.768112in}}%
\pgfpathlineto{\pgfqpoint{2.803963in}{0.768019in}}%
\pgfpathlineto{\pgfqpoint{2.804528in}{0.767990in}}%
\pgfpathlineto{\pgfqpoint{2.805092in}{0.767984in}}%
\pgfpathlineto{\pgfqpoint{2.805656in}{0.767978in}}%
\pgfpathlineto{\pgfqpoint{2.806221in}{0.767972in}}%
\pgfpathlineto{\pgfqpoint{2.806785in}{0.767966in}}%
\pgfpathlineto{\pgfqpoint{2.807350in}{0.767038in}}%
\pgfpathlineto{\pgfqpoint{2.807914in}{0.762116in}}%
\pgfpathlineto{\pgfqpoint{2.808478in}{0.760297in}}%
\pgfpathlineto{\pgfqpoint{2.809043in}{0.760246in}}%
\pgfpathlineto{\pgfqpoint{2.809607in}{0.760240in}}%
\pgfpathlineto{\pgfqpoint{2.810171in}{0.760234in}}%
\pgfpathlineto{\pgfqpoint{2.810736in}{0.760228in}}%
\pgfpathlineto{\pgfqpoint{2.811300in}{0.760222in}}%
\pgfpathlineto{\pgfqpoint{2.811864in}{0.760217in}}%
\pgfpathlineto{\pgfqpoint{2.812429in}{0.760211in}}%
\pgfpathlineto{\pgfqpoint{2.812993in}{0.760205in}}%
\pgfpathlineto{\pgfqpoint{2.813557in}{0.760199in}}%
\pgfpathlineto{\pgfqpoint{2.814122in}{0.760193in}}%
\pgfpathlineto{\pgfqpoint{2.814686in}{0.760188in}}%
\pgfpathlineto{\pgfqpoint{2.815250in}{0.760182in}}%
\pgfpathlineto{\pgfqpoint{2.815815in}{0.760176in}}%
\pgfpathlineto{\pgfqpoint{2.816379in}{0.760170in}}%
\pgfpathlineto{\pgfqpoint{2.816943in}{0.760164in}}%
\pgfpathlineto{\pgfqpoint{2.817508in}{0.760158in}}%
\pgfpathlineto{\pgfqpoint{2.818072in}{0.760150in}}%
\pgfpathlineto{\pgfqpoint{2.818636in}{0.759997in}}%
\pgfpathlineto{\pgfqpoint{2.819201in}{0.759759in}}%
\pgfpathlineto{\pgfqpoint{2.819765in}{0.759520in}}%
\pgfpathlineto{\pgfqpoint{2.820329in}{0.759282in}}%
\pgfpathlineto{\pgfqpoint{2.820894in}{0.759043in}}%
\pgfpathlineto{\pgfqpoint{2.821458in}{0.758805in}}%
\pgfpathlineto{\pgfqpoint{2.822022in}{0.758566in}}%
\pgfpathlineto{\pgfqpoint{2.822587in}{0.758327in}}%
\pgfpathlineto{\pgfqpoint{2.823151in}{0.758089in}}%
\pgfpathlineto{\pgfqpoint{2.823716in}{0.757850in}}%
\pgfpathlineto{\pgfqpoint{2.824280in}{0.757612in}}%
\pgfpathlineto{\pgfqpoint{2.824844in}{0.757373in}}%
\pgfpathlineto{\pgfqpoint{2.825409in}{0.757134in}}%
\pgfpathlineto{\pgfqpoint{2.825973in}{0.756896in}}%
\pgfpathlineto{\pgfqpoint{2.826537in}{0.756657in}}%
\pgfpathlineto{\pgfqpoint{2.827102in}{0.756419in}}%
\pgfpathlineto{\pgfqpoint{2.827666in}{0.756180in}}%
\pgfpathlineto{\pgfqpoint{2.828230in}{0.755941in}}%
\pgfpathlineto{\pgfqpoint{2.828795in}{0.755703in}}%
\pgfpathlineto{\pgfqpoint{2.829359in}{0.755464in}}%
\pgfpathlineto{\pgfqpoint{2.829923in}{0.755226in}}%
\pgfpathlineto{\pgfqpoint{2.830488in}{0.754987in}}%
\pgfpathlineto{\pgfqpoint{2.831052in}{0.754748in}}%
\pgfpathlineto{\pgfqpoint{2.831616in}{0.754510in}}%
\pgfpathlineto{\pgfqpoint{2.832181in}{0.754271in}}%
\pgfpathlineto{\pgfqpoint{2.832745in}{0.754033in}}%
\pgfpathlineto{\pgfqpoint{2.833309in}{0.753794in}}%
\pgfpathlineto{\pgfqpoint{2.833874in}{0.753555in}}%
\pgfpathlineto{\pgfqpoint{2.834438in}{0.753319in}}%
\pgfpathlineto{\pgfqpoint{2.835002in}{0.753223in}}%
\pgfpathlineto{\pgfqpoint{2.835567in}{0.753207in}}%
\pgfpathlineto{\pgfqpoint{2.836131in}{0.753191in}}%
\pgfpathlineto{\pgfqpoint{2.836695in}{0.753174in}}%
\pgfpathlineto{\pgfqpoint{2.837260in}{0.753157in}}%
\pgfpathlineto{\pgfqpoint{2.837824in}{0.753141in}}%
\pgfpathlineto{\pgfqpoint{2.838389in}{0.753124in}}%
\pgfpathlineto{\pgfqpoint{2.838953in}{0.753107in}}%
\pgfpathlineto{\pgfqpoint{2.839517in}{0.753091in}}%
\pgfpathlineto{\pgfqpoint{2.840082in}{0.753074in}}%
\pgfpathlineto{\pgfqpoint{2.840646in}{0.753057in}}%
\pgfpathlineto{\pgfqpoint{2.841210in}{0.753041in}}%
\pgfpathlineto{\pgfqpoint{2.841775in}{0.753024in}}%
\pgfpathlineto{\pgfqpoint{2.842339in}{0.753007in}}%
\pgfpathlineto{\pgfqpoint{2.842903in}{0.752991in}}%
\pgfpathlineto{\pgfqpoint{2.843468in}{0.752974in}}%
\pgfpathlineto{\pgfqpoint{2.844032in}{0.752957in}}%
\pgfpathlineto{\pgfqpoint{2.844596in}{0.752941in}}%
\pgfpathlineto{\pgfqpoint{2.845161in}{0.752924in}}%
\pgfpathlineto{\pgfqpoint{2.845725in}{0.752908in}}%
\pgfpathlineto{\pgfqpoint{2.846289in}{0.752891in}}%
\pgfpathlineto{\pgfqpoint{2.846854in}{0.752874in}}%
\pgfpathlineto{\pgfqpoint{2.847418in}{0.752858in}}%
\pgfpathlineto{\pgfqpoint{2.847982in}{0.752841in}}%
\pgfpathlineto{\pgfqpoint{2.848547in}{0.752824in}}%
\pgfpathlineto{\pgfqpoint{2.849111in}{0.752808in}}%
\pgfpathlineto{\pgfqpoint{2.849675in}{0.752791in}}%
\pgfpathlineto{\pgfqpoint{2.850240in}{0.752774in}}%
\pgfpathlineto{\pgfqpoint{2.850804in}{0.752758in}}%
\pgfpathlineto{\pgfqpoint{2.851368in}{0.752741in}}%
\pgfpathlineto{\pgfqpoint{2.851933in}{0.752724in}}%
\pgfpathlineto{\pgfqpoint{2.852497in}{0.752708in}}%
\pgfpathlineto{\pgfqpoint{2.853062in}{0.752691in}}%
\pgfpathlineto{\pgfqpoint{2.853626in}{0.752674in}}%
\pgfpathlineto{\pgfqpoint{2.854190in}{0.752658in}}%
\pgfpathlineto{\pgfqpoint{2.854755in}{0.752641in}}%
\pgfpathlineto{\pgfqpoint{2.855319in}{0.752624in}}%
\pgfpathlineto{\pgfqpoint{2.855883in}{0.752608in}}%
\pgfpathlineto{\pgfqpoint{2.856448in}{0.752591in}}%
\pgfpathlineto{\pgfqpoint{2.857012in}{0.752574in}}%
\pgfpathlineto{\pgfqpoint{2.857576in}{0.752558in}}%
\pgfpathlineto{\pgfqpoint{2.858141in}{0.752541in}}%
\pgfpathlineto{\pgfqpoint{2.858705in}{0.752524in}}%
\pgfpathlineto{\pgfqpoint{2.859269in}{0.752508in}}%
\pgfpathlineto{\pgfqpoint{2.859834in}{0.752491in}}%
\pgfpathlineto{\pgfqpoint{2.860398in}{0.752474in}}%
\pgfpathlineto{\pgfqpoint{2.860962in}{0.752458in}}%
\pgfpathlineto{\pgfqpoint{2.861527in}{0.752441in}}%
\pgfpathlineto{\pgfqpoint{2.862091in}{0.752424in}}%
\pgfpathlineto{\pgfqpoint{2.862655in}{0.752408in}}%
\pgfpathlineto{\pgfqpoint{2.863220in}{0.752391in}}%
\pgfpathlineto{\pgfqpoint{2.863784in}{0.752374in}}%
\pgfpathlineto{\pgfqpoint{2.864348in}{0.752358in}}%
\pgfpathlineto{\pgfqpoint{2.864913in}{0.752341in}}%
\pgfpathlineto{\pgfqpoint{2.865477in}{0.752325in}}%
\pgfpathlineto{\pgfqpoint{2.866041in}{0.752308in}}%
\pgfpathlineto{\pgfqpoint{2.866606in}{0.752291in}}%
\pgfpathlineto{\pgfqpoint{2.867170in}{0.752275in}}%
\pgfpathlineto{\pgfqpoint{2.867734in}{0.752258in}}%
\pgfpathlineto{\pgfqpoint{2.868299in}{0.752241in}}%
\pgfpathlineto{\pgfqpoint{2.868863in}{0.752225in}}%
\pgfpathlineto{\pgfqpoint{2.869428in}{0.752208in}}%
\pgfpathlineto{\pgfqpoint{2.869992in}{0.752191in}}%
\pgfpathlineto{\pgfqpoint{2.870556in}{0.752175in}}%
\pgfpathlineto{\pgfqpoint{2.871121in}{0.752158in}}%
\pgfpathlineto{\pgfqpoint{2.871685in}{0.752141in}}%
\pgfpathlineto{\pgfqpoint{2.872249in}{0.752125in}}%
\pgfpathlineto{\pgfqpoint{2.872814in}{0.752108in}}%
\pgfpathlineto{\pgfqpoint{2.873378in}{0.752091in}}%
\pgfpathlineto{\pgfqpoint{2.873942in}{0.752075in}}%
\pgfpathlineto{\pgfqpoint{2.874507in}{0.752058in}}%
\pgfpathlineto{\pgfqpoint{2.875071in}{0.752041in}}%
\pgfpathlineto{\pgfqpoint{2.875635in}{0.752025in}}%
\pgfpathlineto{\pgfqpoint{2.876200in}{0.752008in}}%
\pgfpathlineto{\pgfqpoint{2.876764in}{0.751991in}}%
\pgfpathlineto{\pgfqpoint{2.877328in}{0.751975in}}%
\pgfpathlineto{\pgfqpoint{2.877893in}{0.751958in}}%
\pgfpathlineto{\pgfqpoint{2.878457in}{0.751941in}}%
\pgfpathlineto{\pgfqpoint{2.879021in}{0.751925in}}%
\pgfpathlineto{\pgfqpoint{2.879586in}{0.751908in}}%
\pgfpathlineto{\pgfqpoint{2.880150in}{0.751891in}}%
\pgfpathlineto{\pgfqpoint{2.880714in}{0.751875in}}%
\pgfpathlineto{\pgfqpoint{2.881279in}{0.751858in}}%
\pgfpathlineto{\pgfqpoint{2.881843in}{0.751841in}}%
\pgfpathlineto{\pgfqpoint{2.882407in}{0.751825in}}%
\pgfpathlineto{\pgfqpoint{2.882972in}{0.751808in}}%
\pgfpathlineto{\pgfqpoint{2.883536in}{0.751791in}}%
\pgfpathlineto{\pgfqpoint{2.884101in}{0.751775in}}%
\pgfpathlineto{\pgfqpoint{2.884665in}{0.751758in}}%
\pgfpathlineto{\pgfqpoint{2.885229in}{0.751742in}}%
\pgfpathlineto{\pgfqpoint{2.885794in}{0.751725in}}%
\pgfpathlineto{\pgfqpoint{2.886358in}{0.751708in}}%
\pgfpathlineto{\pgfqpoint{2.886922in}{0.751692in}}%
\pgfpathlineto{\pgfqpoint{2.887487in}{0.751675in}}%
\pgfpathlineto{\pgfqpoint{2.888051in}{0.751658in}}%
\pgfpathlineto{\pgfqpoint{2.888615in}{0.752052in}}%
\pgfpathlineto{\pgfqpoint{2.889180in}{0.751300in}}%
\pgfpathlineto{\pgfqpoint{2.889744in}{0.743049in}}%
\pgfpathlineto{\pgfqpoint{2.890308in}{0.743905in}}%
\pgfpathlineto{\pgfqpoint{2.890873in}{0.744761in}}%
\pgfpathlineto{\pgfqpoint{2.891437in}{0.745616in}}%
\pgfpathlineto{\pgfqpoint{2.892001in}{0.746472in}}%
\pgfpathlineto{\pgfqpoint{2.892566in}{0.747328in}}%
\pgfpathlineto{\pgfqpoint{2.893130in}{0.748183in}}%
\pgfpathlineto{\pgfqpoint{2.893694in}{0.749039in}}%
\pgfpathlineto{\pgfqpoint{2.894259in}{0.749895in}}%
\pgfpathlineto{\pgfqpoint{2.894823in}{0.750750in}}%
\pgfpathlineto{\pgfqpoint{2.895387in}{0.751606in}}%
\pgfpathlineto{\pgfqpoint{2.895952in}{0.752462in}}%
\pgfpathlineto{\pgfqpoint{2.896516in}{0.753318in}}%
\pgfpathlineto{\pgfqpoint{2.897080in}{0.754173in}}%
\pgfpathlineto{\pgfqpoint{2.897645in}{0.755029in}}%
\pgfpathlineto{\pgfqpoint{2.898209in}{0.755885in}}%
\pgfpathlineto{\pgfqpoint{2.898774in}{0.756740in}}%
\pgfpathlineto{\pgfqpoint{2.899338in}{0.757596in}}%
\pgfpathlineto{\pgfqpoint{2.899902in}{0.758452in}}%
\pgfpathlineto{\pgfqpoint{2.900467in}{0.759169in}}%
\pgfpathlineto{\pgfqpoint{2.901031in}{0.754964in}}%
\pgfpathlineto{\pgfqpoint{2.901595in}{0.756350in}}%
\pgfpathlineto{\pgfqpoint{2.902160in}{0.754809in}}%
\pgfpathlineto{\pgfqpoint{2.902724in}{0.757473in}}%
\pgfpathlineto{\pgfqpoint{2.903288in}{0.757280in}}%
\pgfpathlineto{\pgfqpoint{2.903853in}{0.758989in}}%
\pgfpathlineto{\pgfqpoint{2.904417in}{0.758540in}}%
\pgfpathlineto{\pgfqpoint{2.904981in}{0.758091in}}%
\pgfpathlineto{\pgfqpoint{2.905546in}{0.757642in}}%
\pgfpathlineto{\pgfqpoint{2.906110in}{0.757193in}}%
\pgfpathlineto{\pgfqpoint{2.906674in}{0.756744in}}%
\pgfpathlineto{\pgfqpoint{2.907239in}{0.756295in}}%
\pgfpathlineto{\pgfqpoint{2.907803in}{0.755846in}}%
\pgfpathlineto{\pgfqpoint{2.908367in}{0.755397in}}%
\pgfpathlineto{\pgfqpoint{2.908932in}{0.754948in}}%
\pgfpathlineto{\pgfqpoint{2.909496in}{0.754499in}}%
\pgfpathlineto{\pgfqpoint{2.910060in}{0.754050in}}%
\pgfpathlineto{\pgfqpoint{2.910625in}{0.753601in}}%
\pgfpathlineto{\pgfqpoint{2.911189in}{0.753152in}}%
\pgfpathlineto{\pgfqpoint{2.911753in}{0.752703in}}%
\pgfpathlineto{\pgfqpoint{2.912318in}{0.753731in}}%
\pgfpathlineto{\pgfqpoint{2.912882in}{0.753499in}}%
\pgfpathlineto{\pgfqpoint{2.913446in}{0.752809in}}%
\pgfpathlineto{\pgfqpoint{2.914011in}{0.757069in}}%
\pgfpathlineto{\pgfqpoint{2.914575in}{0.759411in}}%
\pgfpathlineto{\pgfqpoint{2.915140in}{0.757956in}}%
\pgfpathlineto{\pgfqpoint{2.915704in}{0.760083in}}%
\pgfpathlineto{\pgfqpoint{2.916268in}{0.760290in}}%
\pgfpathlineto{\pgfqpoint{2.916833in}{0.760138in}}%
\pgfpathlineto{\pgfqpoint{2.917397in}{0.760120in}}%
\pgfpathlineto{\pgfqpoint{2.917961in}{0.760139in}}%
\pgfpathlineto{\pgfqpoint{2.918526in}{0.759902in}}%
\pgfpathlineto{\pgfqpoint{2.919090in}{0.759327in}}%
\pgfpathlineto{\pgfqpoint{2.919654in}{0.758748in}}%
\pgfpathlineto{\pgfqpoint{2.920219in}{0.758133in}}%
\pgfpathlineto{\pgfqpoint{2.920783in}{0.757494in}}%
\pgfpathlineto{\pgfqpoint{2.921347in}{0.756855in}}%
\pgfpathlineto{\pgfqpoint{2.921912in}{0.756216in}}%
\pgfpathlineto{\pgfqpoint{2.922476in}{0.755576in}}%
\pgfpathlineto{\pgfqpoint{2.923040in}{0.754937in}}%
\pgfpathlineto{\pgfqpoint{2.923605in}{0.754298in}}%
\pgfpathlineto{\pgfqpoint{2.924169in}{0.753659in}}%
\pgfpathlineto{\pgfqpoint{2.924733in}{0.753019in}}%
\pgfpathlineto{\pgfqpoint{2.925298in}{0.752380in}}%
\pgfpathlineto{\pgfqpoint{2.925862in}{0.753579in}}%
\pgfpathlineto{\pgfqpoint{2.926426in}{0.753976in}}%
\pgfpathlineto{\pgfqpoint{2.926991in}{0.759292in}}%
\pgfpathlineto{\pgfqpoint{2.927555in}{0.759148in}}%
\pgfpathlineto{\pgfqpoint{2.928119in}{0.759195in}}%
\pgfpathlineto{\pgfqpoint{2.928684in}{0.759322in}}%
\pgfpathlineto{\pgfqpoint{2.929248in}{0.759216in}}%
\pgfpathlineto{\pgfqpoint{2.929813in}{0.759561in}}%
\pgfpathlineto{\pgfqpoint{2.930377in}{0.759906in}}%
\pgfpathlineto{\pgfqpoint{2.930941in}{0.759767in}}%
\pgfpathlineto{\pgfqpoint{2.931506in}{0.759708in}}%
\pgfpathlineto{\pgfqpoint{2.932070in}{0.759649in}}%
\pgfpathlineto{\pgfqpoint{2.932634in}{0.759590in}}%
\pgfpathlineto{\pgfqpoint{2.933199in}{0.759532in}}%
\pgfpathlineto{\pgfqpoint{2.933763in}{0.759473in}}%
\pgfpathlineto{\pgfqpoint{2.934327in}{0.759414in}}%
\pgfpathlineto{\pgfqpoint{2.934892in}{0.759355in}}%
\pgfpathlineto{\pgfqpoint{2.935456in}{0.759296in}}%
\pgfpathlineto{\pgfqpoint{2.936020in}{0.759237in}}%
\pgfpathlineto{\pgfqpoint{2.936585in}{0.759179in}}%
\pgfpathlineto{\pgfqpoint{2.937149in}{0.759120in}}%
\pgfpathlineto{\pgfqpoint{2.937713in}{0.759061in}}%
\pgfpathlineto{\pgfqpoint{2.938278in}{0.759002in}}%
\pgfpathlineto{\pgfqpoint{2.938842in}{0.758943in}}%
\pgfpathlineto{\pgfqpoint{2.939406in}{0.758884in}}%
\pgfpathlineto{\pgfqpoint{2.939971in}{0.758826in}}%
\pgfpathlineto{\pgfqpoint{2.940535in}{0.758767in}}%
\pgfpathlineto{\pgfqpoint{2.941099in}{0.758708in}}%
\pgfpathlineto{\pgfqpoint{2.941664in}{0.758642in}}%
\pgfpathlineto{\pgfqpoint{2.942228in}{0.759388in}}%
\pgfpathlineto{\pgfqpoint{2.942792in}{0.759348in}}%
\pgfpathlineto{\pgfqpoint{2.943357in}{0.758762in}}%
\pgfpathlineto{\pgfqpoint{2.943921in}{0.758011in}}%
\pgfpathlineto{\pgfqpoint{2.944486in}{0.757963in}}%
\pgfpathlineto{\pgfqpoint{2.945050in}{0.757931in}}%
\pgfpathlineto{\pgfqpoint{2.945614in}{0.757898in}}%
\pgfpathlineto{\pgfqpoint{2.946179in}{0.757866in}}%
\pgfpathlineto{\pgfqpoint{2.946743in}{0.757833in}}%
\pgfpathlineto{\pgfqpoint{2.947307in}{0.757793in}}%
\pgfpathlineto{\pgfqpoint{2.947872in}{0.757749in}}%
\pgfpathlineto{\pgfqpoint{2.948436in}{0.757705in}}%
\pgfpathlineto{\pgfqpoint{2.949000in}{0.757661in}}%
\pgfpathlineto{\pgfqpoint{2.949565in}{0.757616in}}%
\pgfpathlineto{\pgfqpoint{2.950129in}{0.757572in}}%
\pgfpathlineto{\pgfqpoint{2.950693in}{0.757528in}}%
\pgfpathlineto{\pgfqpoint{2.951258in}{0.757483in}}%
\pgfpathlineto{\pgfqpoint{2.951822in}{0.757439in}}%
\pgfpathlineto{\pgfqpoint{2.952386in}{0.757395in}}%
\pgfpathlineto{\pgfqpoint{2.952951in}{0.757351in}}%
\pgfpathlineto{\pgfqpoint{2.953515in}{0.757306in}}%
\pgfpathlineto{\pgfqpoint{2.954079in}{0.757262in}}%
\pgfpathlineto{\pgfqpoint{2.954644in}{0.757218in}}%
\pgfpathlineto{\pgfqpoint{2.955208in}{0.757173in}}%
\pgfpathlineto{\pgfqpoint{2.955772in}{0.757129in}}%
\pgfpathlineto{\pgfqpoint{2.956337in}{0.757085in}}%
\pgfpathlineto{\pgfqpoint{2.956901in}{0.757041in}}%
\pgfpathlineto{\pgfqpoint{2.957465in}{0.756996in}}%
\pgfpathlineto{\pgfqpoint{2.958030in}{0.756952in}}%
\pgfpathlineto{\pgfqpoint{2.958594in}{0.756908in}}%
\pgfpathlineto{\pgfqpoint{2.959159in}{0.756863in}}%
\pgfpathlineto{\pgfqpoint{2.959723in}{0.756819in}}%
\pgfpathlineto{\pgfqpoint{2.960287in}{0.756775in}}%
\pgfpathlineto{\pgfqpoint{2.960852in}{0.756731in}}%
\pgfpathlineto{\pgfqpoint{2.961416in}{0.756686in}}%
\pgfpathlineto{\pgfqpoint{2.961980in}{0.756642in}}%
\pgfpathlineto{\pgfqpoint{2.962545in}{0.756598in}}%
\pgfpathlineto{\pgfqpoint{2.963109in}{0.756554in}}%
\pgfpathlineto{\pgfqpoint{2.963673in}{0.756509in}}%
\pgfpathlineto{\pgfqpoint{2.964238in}{0.756465in}}%
\pgfpathlineto{\pgfqpoint{2.964802in}{0.756421in}}%
\pgfpathlineto{\pgfqpoint{2.965366in}{0.756376in}}%
\pgfpathlineto{\pgfqpoint{2.965931in}{0.756332in}}%
\pgfpathlineto{\pgfqpoint{2.966495in}{0.756288in}}%
\pgfpathlineto{\pgfqpoint{2.967059in}{0.756244in}}%
\pgfpathlineto{\pgfqpoint{2.967624in}{0.756199in}}%
\pgfpathlineto{\pgfqpoint{2.968188in}{0.756155in}}%
\pgfpathlineto{\pgfqpoint{2.968752in}{0.756111in}}%
\pgfpathlineto{\pgfqpoint{2.969317in}{0.756066in}}%
\pgfpathlineto{\pgfqpoint{2.969881in}{0.756022in}}%
\pgfpathlineto{\pgfqpoint{2.970445in}{0.755978in}}%
\pgfpathlineto{\pgfqpoint{2.971010in}{0.755934in}}%
\pgfpathlineto{\pgfqpoint{2.971574in}{0.755889in}}%
\pgfpathlineto{\pgfqpoint{2.972138in}{0.755845in}}%
\pgfpathlineto{\pgfqpoint{2.972703in}{0.755801in}}%
\pgfpathlineto{\pgfqpoint{2.973267in}{0.755756in}}%
\pgfpathlineto{\pgfqpoint{2.973831in}{0.755712in}}%
\pgfpathlineto{\pgfqpoint{2.974396in}{0.755668in}}%
\pgfpathlineto{\pgfqpoint{2.974960in}{0.755624in}}%
\pgfpathlineto{\pgfqpoint{2.975525in}{0.755579in}}%
\pgfpathlineto{\pgfqpoint{2.976089in}{0.755535in}}%
\pgfpathlineto{\pgfqpoint{2.976653in}{0.755491in}}%
\pgfpathlineto{\pgfqpoint{2.977218in}{0.755446in}}%
\pgfpathlineto{\pgfqpoint{2.977782in}{0.755402in}}%
\pgfpathlineto{\pgfqpoint{2.978346in}{0.755358in}}%
\pgfpathlineto{\pgfqpoint{2.978911in}{0.755314in}}%
\pgfpathlineto{\pgfqpoint{2.979475in}{0.755269in}}%
\pgfpathlineto{\pgfqpoint{2.980039in}{0.755225in}}%
\pgfpathlineto{\pgfqpoint{2.980604in}{0.755181in}}%
\pgfpathlineto{\pgfqpoint{2.981168in}{0.755136in}}%
\pgfpathlineto{\pgfqpoint{2.981732in}{0.755092in}}%
\pgfpathlineto{\pgfqpoint{2.982297in}{0.755048in}}%
\pgfpathlineto{\pgfqpoint{2.982861in}{0.755004in}}%
\pgfpathlineto{\pgfqpoint{2.983425in}{0.754959in}}%
\pgfpathlineto{\pgfqpoint{2.983990in}{0.754870in}}%
\pgfpathlineto{\pgfqpoint{2.984554in}{0.753166in}}%
\pgfpathlineto{\pgfqpoint{2.985118in}{0.752675in}}%
\pgfpathlineto{\pgfqpoint{2.985683in}{0.752636in}}%
\pgfpathlineto{\pgfqpoint{2.986247in}{0.752597in}}%
\pgfpathlineto{\pgfqpoint{2.986811in}{0.752559in}}%
\pgfpathlineto{\pgfqpoint{2.987376in}{0.752520in}}%
\pgfpathlineto{\pgfqpoint{2.987940in}{0.752481in}}%
\pgfpathlineto{\pgfqpoint{2.988504in}{0.752442in}}%
\pgfpathlineto{\pgfqpoint{2.989069in}{0.752403in}}%
\pgfpathlineto{\pgfqpoint{2.989633in}{0.752364in}}%
\pgfpathlineto{\pgfqpoint{2.990198in}{0.752326in}}%
\pgfpathlineto{\pgfqpoint{2.990762in}{0.752287in}}%
\pgfpathlineto{\pgfqpoint{2.991326in}{0.752248in}}%
\pgfpathlineto{\pgfqpoint{2.991891in}{0.752209in}}%
\pgfpathlineto{\pgfqpoint{2.992455in}{0.752170in}}%
\pgfpathlineto{\pgfqpoint{2.993019in}{0.752131in}}%
\pgfpathlineto{\pgfqpoint{2.993584in}{0.752093in}}%
\pgfpathlineto{\pgfqpoint{2.994148in}{0.752054in}}%
\pgfpathlineto{\pgfqpoint{2.994712in}{0.752015in}}%
\pgfpathlineto{\pgfqpoint{2.995277in}{0.751976in}}%
\pgfpathlineto{\pgfqpoint{2.995841in}{0.751937in}}%
\pgfpathlineto{\pgfqpoint{2.996405in}{0.752005in}}%
\pgfpathlineto{\pgfqpoint{2.996970in}{0.754485in}}%
\pgfpathlineto{\pgfqpoint{2.997534in}{0.755098in}}%
\pgfpathlineto{\pgfqpoint{2.998098in}{0.755096in}}%
\pgfpathlineto{\pgfqpoint{2.998663in}{0.755095in}}%
\pgfpathlineto{\pgfqpoint{2.999227in}{0.755094in}}%
\pgfpathlineto{\pgfqpoint{2.999791in}{0.755092in}}%
\pgfpathlineto{\pgfqpoint{3.000356in}{0.755091in}}%
\pgfpathlineto{\pgfqpoint{3.000920in}{0.755089in}}%
\pgfpathlineto{\pgfqpoint{3.001484in}{0.755088in}}%
\pgfpathlineto{\pgfqpoint{3.002049in}{0.755087in}}%
\pgfpathlineto{\pgfqpoint{3.002613in}{0.755085in}}%
\pgfpathlineto{\pgfqpoint{3.003177in}{0.755084in}}%
\pgfpathlineto{\pgfqpoint{3.003742in}{0.755082in}}%
\pgfpathlineto{\pgfqpoint{3.004306in}{0.755081in}}%
\pgfpathlineto{\pgfqpoint{3.004871in}{0.755080in}}%
\pgfpathlineto{\pgfqpoint{3.005435in}{0.755078in}}%
\pgfpathlineto{\pgfqpoint{3.005999in}{0.755077in}}%
\pgfpathlineto{\pgfqpoint{3.006564in}{0.755075in}}%
\pgfpathlineto{\pgfqpoint{3.007128in}{0.755074in}}%
\pgfpathlineto{\pgfqpoint{3.007692in}{0.753541in}}%
\pgfpathlineto{\pgfqpoint{3.008257in}{0.751846in}}%
\pgfpathlineto{\pgfqpoint{3.008821in}{0.751845in}}%
\pgfpathlineto{\pgfqpoint{3.009385in}{0.751843in}}%
\pgfpathlineto{\pgfqpoint{3.009950in}{0.751841in}}%
\pgfpathlineto{\pgfqpoint{3.010514in}{0.752082in}}%
\pgfpathlineto{\pgfqpoint{3.011078in}{0.752293in}}%
\pgfpathlineto{\pgfqpoint{3.011643in}{0.752749in}}%
\pgfpathlineto{\pgfqpoint{3.012207in}{0.753058in}}%
\pgfpathlineto{\pgfqpoint{3.012771in}{0.753174in}}%
\pgfpathlineto{\pgfqpoint{3.013336in}{0.753290in}}%
\pgfpathlineto{\pgfqpoint{3.013900in}{0.753406in}}%
\pgfpathlineto{\pgfqpoint{3.014464in}{0.753522in}}%
\pgfpathlineto{\pgfqpoint{3.015029in}{0.753638in}}%
\pgfpathlineto{\pgfqpoint{3.015593in}{0.753754in}}%
\pgfpathlineto{\pgfqpoint{3.016157in}{0.753870in}}%
\pgfpathlineto{\pgfqpoint{3.016722in}{0.753986in}}%
\pgfpathlineto{\pgfqpoint{3.017286in}{0.754101in}}%
\pgfpathlineto{\pgfqpoint{3.017850in}{0.754217in}}%
\pgfpathlineto{\pgfqpoint{3.018415in}{0.754333in}}%
\pgfpathlineto{\pgfqpoint{3.018979in}{0.754449in}}%
\pgfpathlineto{\pgfqpoint{3.019543in}{0.754565in}}%
\pgfpathlineto{\pgfqpoint{3.020108in}{0.754681in}}%
\pgfpathlineto{\pgfqpoint{3.020672in}{0.754797in}}%
\pgfpathlineto{\pgfqpoint{3.021237in}{0.754913in}}%
\pgfpathlineto{\pgfqpoint{3.021801in}{0.755029in}}%
\pgfpathlineto{\pgfqpoint{3.022365in}{0.755143in}}%
\pgfpathlineto{\pgfqpoint{3.022930in}{0.755066in}}%
\pgfpathlineto{\pgfqpoint{3.023494in}{0.754369in}}%
\pgfpathlineto{\pgfqpoint{3.024058in}{0.752546in}}%
\pgfpathlineto{\pgfqpoint{3.024623in}{0.751787in}}%
\pgfpathlineto{\pgfqpoint{3.025187in}{0.751790in}}%
\pgfpathlineto{\pgfqpoint{3.025751in}{0.751836in}}%
\pgfpathlineto{\pgfqpoint{3.026316in}{0.751891in}}%
\pgfpathlineto{\pgfqpoint{3.026880in}{0.751942in}}%
\pgfpathlineto{\pgfqpoint{3.027444in}{0.751972in}}%
\pgfpathlineto{\pgfqpoint{3.028009in}{0.751988in}}%
\pgfpathlineto{\pgfqpoint{3.028573in}{0.752128in}}%
\pgfpathlineto{\pgfqpoint{3.029137in}{0.752470in}}%
\pgfpathlineto{\pgfqpoint{3.029702in}{0.752814in}}%
\pgfpathlineto{\pgfqpoint{3.030266in}{0.753158in}}%
\pgfpathlineto{\pgfqpoint{3.030830in}{0.753503in}}%
\pgfpathlineto{\pgfqpoint{3.031395in}{0.753847in}}%
\pgfpathlineto{\pgfqpoint{3.031959in}{0.754191in}}%
\pgfpathlineto{\pgfqpoint{3.032523in}{0.754535in}}%
\pgfpathlineto{\pgfqpoint{3.033088in}{0.754877in}}%
\pgfpathlineto{\pgfqpoint{3.033652in}{0.755300in}}%
\pgfpathlineto{\pgfqpoint{3.034216in}{0.756178in}}%
\pgfpathlineto{\pgfqpoint{3.034781in}{0.756398in}}%
\pgfpathlineto{\pgfqpoint{3.035345in}{0.754719in}}%
\pgfpathlineto{\pgfqpoint{3.035910in}{0.752925in}}%
\pgfpathlineto{\pgfqpoint{3.036474in}{0.752035in}}%
\pgfpathlineto{\pgfqpoint{3.037038in}{0.752040in}}%
\pgfpathlineto{\pgfqpoint{3.037603in}{0.752055in}}%
\pgfpathlineto{\pgfqpoint{3.038167in}{0.753393in}}%
\pgfpathlineto{\pgfqpoint{3.038731in}{0.755543in}}%
\pgfpathlineto{\pgfqpoint{3.039296in}{0.755627in}}%
\pgfpathlineto{\pgfqpoint{3.039860in}{0.755626in}}%
\pgfpathlineto{\pgfqpoint{3.040424in}{0.755625in}}%
\pgfpathlineto{\pgfqpoint{3.040989in}{0.755623in}}%
\pgfpathlineto{\pgfqpoint{3.041553in}{0.755622in}}%
\pgfpathlineto{\pgfqpoint{3.042117in}{0.755620in}}%
\pgfpathlineto{\pgfqpoint{3.042682in}{0.755619in}}%
\pgfpathlineto{\pgfqpoint{3.043246in}{0.755618in}}%
\pgfpathlineto{\pgfqpoint{3.043810in}{0.755616in}}%
\pgfpathlineto{\pgfqpoint{3.044375in}{0.755615in}}%
\pgfpathlineto{\pgfqpoint{3.044939in}{0.755613in}}%
\pgfpathlineto{\pgfqpoint{3.045503in}{0.755612in}}%
\pgfpathlineto{\pgfqpoint{3.046068in}{0.755611in}}%
\pgfpathlineto{\pgfqpoint{3.046632in}{0.755609in}}%
\pgfpathlineto{\pgfqpoint{3.047196in}{0.755608in}}%
\pgfpathlineto{\pgfqpoint{3.047761in}{0.755606in}}%
\pgfpathlineto{\pgfqpoint{3.048325in}{0.755605in}}%
\pgfpathlineto{\pgfqpoint{3.048889in}{0.755603in}}%
\pgfpathlineto{\pgfqpoint{3.049454in}{0.755602in}}%
\pgfpathlineto{\pgfqpoint{3.050018in}{0.755601in}}%
\pgfpathlineto{\pgfqpoint{3.050583in}{0.755599in}}%
\pgfpathlineto{\pgfqpoint{3.051147in}{0.755598in}}%
\pgfpathlineto{\pgfqpoint{3.051711in}{0.755596in}}%
\pgfpathlineto{\pgfqpoint{3.052276in}{0.755595in}}%
\pgfpathlineto{\pgfqpoint{3.052840in}{0.755594in}}%
\pgfpathlineto{\pgfqpoint{3.053404in}{0.755592in}}%
\pgfpathlineto{\pgfqpoint{3.053969in}{0.755591in}}%
\pgfpathlineto{\pgfqpoint{3.054533in}{0.755589in}}%
\pgfpathlineto{\pgfqpoint{3.055097in}{0.755588in}}%
\pgfpathlineto{\pgfqpoint{3.055662in}{0.755587in}}%
\pgfpathlineto{\pgfqpoint{3.056226in}{0.755585in}}%
\pgfpathlineto{\pgfqpoint{3.056790in}{0.755584in}}%
\pgfpathlineto{\pgfqpoint{3.057355in}{0.755582in}}%
\pgfpathlineto{\pgfqpoint{3.057919in}{0.755581in}}%
\pgfpathlineto{\pgfqpoint{3.058483in}{0.755580in}}%
\pgfpathlineto{\pgfqpoint{3.059048in}{0.755578in}}%
\pgfpathlineto{\pgfqpoint{3.059612in}{0.755577in}}%
\pgfpathlineto{\pgfqpoint{3.060176in}{0.755575in}}%
\pgfpathlineto{\pgfqpoint{3.060741in}{0.755574in}}%
\pgfpathlineto{\pgfqpoint{3.061305in}{0.755573in}}%
\pgfpathlineto{\pgfqpoint{3.061869in}{0.755571in}}%
\pgfpathlineto{\pgfqpoint{3.062434in}{0.755570in}}%
\pgfpathlineto{\pgfqpoint{3.062998in}{0.755568in}}%
\pgfpathlineto{\pgfqpoint{3.063562in}{0.755567in}}%
\pgfpathlineto{\pgfqpoint{3.064127in}{0.755566in}}%
\pgfpathlineto{\pgfqpoint{3.064691in}{0.755564in}}%
\pgfpathlineto{\pgfqpoint{3.065255in}{0.755563in}}%
\pgfpathlineto{\pgfqpoint{3.065820in}{0.755561in}}%
\pgfpathlineto{\pgfqpoint{3.066384in}{0.755560in}}%
\pgfpathlineto{\pgfqpoint{3.066949in}{0.755558in}}%
\pgfpathlineto{\pgfqpoint{3.067513in}{0.755557in}}%
\pgfpathlineto{\pgfqpoint{3.068077in}{0.755556in}}%
\pgfpathlineto{\pgfqpoint{3.068642in}{0.755554in}}%
\pgfpathlineto{\pgfqpoint{3.069206in}{0.755553in}}%
\pgfpathlineto{\pgfqpoint{3.069770in}{0.755551in}}%
\pgfpathlineto{\pgfqpoint{3.070335in}{0.755550in}}%
\pgfpathlineto{\pgfqpoint{3.070899in}{0.755549in}}%
\pgfpathlineto{\pgfqpoint{3.071463in}{0.755547in}}%
\pgfpathlineto{\pgfqpoint{3.072028in}{0.755546in}}%
\pgfpathlineto{\pgfqpoint{3.072592in}{0.755544in}}%
\pgfpathlineto{\pgfqpoint{3.073156in}{0.755543in}}%
\pgfpathlineto{\pgfqpoint{3.073721in}{0.755542in}}%
\pgfpathlineto{\pgfqpoint{3.074285in}{0.755548in}}%
\pgfpathlineto{\pgfqpoint{3.074849in}{0.755564in}}%
\pgfpathlineto{\pgfqpoint{3.075414in}{0.755568in}}%
\pgfpathlineto{\pgfqpoint{3.075978in}{0.755579in}}%
\pgfpathlineto{\pgfqpoint{3.076542in}{0.755584in}}%
\pgfpathlineto{\pgfqpoint{3.077107in}{0.755462in}}%
\pgfpathlineto{\pgfqpoint{3.077671in}{0.755415in}}%
\pgfpathlineto{\pgfqpoint{3.078235in}{0.755478in}}%
\pgfpathlineto{\pgfqpoint{3.078800in}{0.755539in}}%
\pgfpathlineto{\pgfqpoint{3.079364in}{0.755547in}}%
\pgfpathlineto{\pgfqpoint{3.079928in}{0.755534in}}%
\pgfpathlineto{\pgfqpoint{3.080493in}{0.755520in}}%
\pgfpathlineto{\pgfqpoint{3.081057in}{0.755507in}}%
\pgfpathlineto{\pgfqpoint{3.081622in}{0.755494in}}%
\pgfpathlineto{\pgfqpoint{3.082186in}{0.755480in}}%
\pgfpathlineto{\pgfqpoint{3.082750in}{0.755467in}}%
\pgfpathlineto{\pgfqpoint{3.083315in}{0.755454in}}%
\pgfpathlineto{\pgfqpoint{3.083879in}{0.755440in}}%
\pgfpathlineto{\pgfqpoint{3.084443in}{0.755427in}}%
\pgfpathlineto{\pgfqpoint{3.085008in}{0.755414in}}%
\pgfpathlineto{\pgfqpoint{3.085572in}{0.755401in}}%
\pgfpathlineto{\pgfqpoint{3.086136in}{0.755387in}}%
\pgfpathlineto{\pgfqpoint{3.086701in}{0.755374in}}%
\pgfpathlineto{\pgfqpoint{3.087265in}{0.755368in}}%
\pgfpathlineto{\pgfqpoint{3.087829in}{0.755485in}}%
\pgfpathlineto{\pgfqpoint{3.088394in}{0.755425in}}%
\pgfpathlineto{\pgfqpoint{3.088958in}{0.755428in}}%
\pgfpathlineto{\pgfqpoint{3.089522in}{0.755442in}}%
\pgfpathlineto{\pgfqpoint{3.090087in}{0.755451in}}%
\pgfpathlineto{\pgfqpoint{3.090651in}{0.755417in}}%
\pgfpathlineto{\pgfqpoint{3.091215in}{0.755394in}}%
\pgfpathlineto{\pgfqpoint{3.091780in}{0.755397in}}%
\pgfpathlineto{\pgfqpoint{3.092344in}{0.755407in}}%
\pgfpathlineto{\pgfqpoint{3.092908in}{0.755412in}}%
\pgfpathlineto{\pgfqpoint{3.093473in}{0.755422in}}%
\pgfpathlineto{\pgfqpoint{3.094037in}{0.755432in}}%
\pgfpathlineto{\pgfqpoint{3.094601in}{0.755442in}}%
\pgfpathlineto{\pgfqpoint{3.095166in}{0.755452in}}%
\pgfpathlineto{\pgfqpoint{3.095730in}{0.755463in}}%
\pgfpathlineto{\pgfqpoint{3.096295in}{0.755473in}}%
\pgfpathlineto{\pgfqpoint{3.096859in}{0.755483in}}%
\pgfpathlineto{\pgfqpoint{3.097423in}{0.755493in}}%
\pgfpathlineto{\pgfqpoint{3.097988in}{0.755503in}}%
\pgfpathlineto{\pgfqpoint{3.098552in}{0.755513in}}%
\pgfpathlineto{\pgfqpoint{3.099116in}{0.755524in}}%
\pgfpathlineto{\pgfqpoint{3.099681in}{0.755534in}}%
\pgfpathlineto{\pgfqpoint{3.100245in}{0.755544in}}%
\pgfpathlineto{\pgfqpoint{3.100809in}{0.755554in}}%
\pgfpathlineto{\pgfqpoint{3.101374in}{0.755564in}}%
\pgfpathlineto{\pgfqpoint{3.101938in}{0.755574in}}%
\pgfpathlineto{\pgfqpoint{3.102502in}{0.755585in}}%
\pgfpathlineto{\pgfqpoint{3.103067in}{0.755595in}}%
\pgfpathlineto{\pgfqpoint{3.103631in}{0.755605in}}%
\pgfpathlineto{\pgfqpoint{3.104195in}{0.755615in}}%
\pgfpathlineto{\pgfqpoint{3.104760in}{0.755625in}}%
\pgfpathlineto{\pgfqpoint{3.105324in}{0.755636in}}%
\pgfpathlineto{\pgfqpoint{3.105888in}{0.755646in}}%
\pgfpathlineto{\pgfqpoint{3.106453in}{0.755656in}}%
\pgfpathlineto{\pgfqpoint{3.107017in}{0.755666in}}%
\pgfpathlineto{\pgfqpoint{3.107581in}{0.755676in}}%
\pgfpathlineto{\pgfqpoint{3.108146in}{0.755686in}}%
\pgfpathlineto{\pgfqpoint{3.108710in}{0.755697in}}%
\pgfpathlineto{\pgfqpoint{3.109274in}{0.755707in}}%
\pgfpathlineto{\pgfqpoint{3.109839in}{0.755717in}}%
\pgfpathlineto{\pgfqpoint{3.110403in}{0.755727in}}%
\pgfpathlineto{\pgfqpoint{3.110967in}{0.755737in}}%
\pgfpathlineto{\pgfqpoint{3.111532in}{0.755747in}}%
\pgfpathlineto{\pgfqpoint{3.112096in}{0.755758in}}%
\pgfpathlineto{\pgfqpoint{3.112661in}{0.755768in}}%
\pgfpathlineto{\pgfqpoint{3.113225in}{0.755778in}}%
\pgfpathlineto{\pgfqpoint{3.113789in}{0.755788in}}%
\pgfpathlineto{\pgfqpoint{3.114354in}{0.755798in}}%
\pgfpathlineto{\pgfqpoint{3.114918in}{0.755809in}}%
\pgfpathlineto{\pgfqpoint{3.115482in}{0.755819in}}%
\pgfpathlineto{\pgfqpoint{3.116047in}{0.755829in}}%
\pgfpathlineto{\pgfqpoint{3.116611in}{0.755839in}}%
\pgfpathlineto{\pgfqpoint{3.117175in}{0.755849in}}%
\pgfpathlineto{\pgfqpoint{3.117740in}{0.755859in}}%
\pgfpathlineto{\pgfqpoint{3.118304in}{0.755870in}}%
\pgfpathlineto{\pgfqpoint{3.118868in}{0.755880in}}%
\pgfpathlineto{\pgfqpoint{3.119433in}{0.755890in}}%
\pgfpathlineto{\pgfqpoint{3.119997in}{0.755900in}}%
\pgfpathlineto{\pgfqpoint{3.120561in}{0.755910in}}%
\pgfpathlineto{\pgfqpoint{3.121126in}{0.755920in}}%
\pgfpathlineto{\pgfqpoint{3.121690in}{0.755931in}}%
\pgfpathlineto{\pgfqpoint{3.122254in}{0.755941in}}%
\pgfpathlineto{\pgfqpoint{3.122819in}{0.755951in}}%
\pgfpathlineto{\pgfqpoint{3.123383in}{0.755961in}}%
\pgfpathlineto{\pgfqpoint{3.123947in}{0.755971in}}%
\pgfpathlineto{\pgfqpoint{3.124512in}{0.755981in}}%
\pgfpathlineto{\pgfqpoint{3.125076in}{0.755992in}}%
\pgfpathlineto{\pgfqpoint{3.125640in}{0.756002in}}%
\pgfpathlineto{\pgfqpoint{3.126205in}{0.756012in}}%
\pgfpathlineto{\pgfqpoint{3.126769in}{0.756022in}}%
\pgfpathlineto{\pgfqpoint{3.127334in}{0.756032in}}%
\pgfpathlineto{\pgfqpoint{3.127898in}{0.756043in}}%
\pgfpathlineto{\pgfqpoint{3.128462in}{0.756053in}}%
\pgfpathlineto{\pgfqpoint{3.129027in}{0.756063in}}%
\pgfpathlineto{\pgfqpoint{3.129591in}{0.756073in}}%
\pgfpathlineto{\pgfqpoint{3.130155in}{0.756083in}}%
\pgfpathlineto{\pgfqpoint{3.130720in}{0.756093in}}%
\pgfpathlineto{\pgfqpoint{3.131284in}{0.756104in}}%
\pgfpathlineto{\pgfqpoint{3.131848in}{0.756114in}}%
\pgfpathlineto{\pgfqpoint{3.132413in}{0.756124in}}%
\pgfpathlineto{\pgfqpoint{3.132977in}{0.756134in}}%
\pgfpathlineto{\pgfqpoint{3.133541in}{0.756144in}}%
\pgfpathlineto{\pgfqpoint{3.134106in}{0.756154in}}%
\pgfpathlineto{\pgfqpoint{3.134670in}{0.756165in}}%
\pgfpathlineto{\pgfqpoint{3.135234in}{0.756175in}}%
\pgfpathlineto{\pgfqpoint{3.135799in}{0.756185in}}%
\pgfpathlineto{\pgfqpoint{3.136363in}{0.756195in}}%
\pgfpathlineto{\pgfqpoint{3.136927in}{0.756205in}}%
\pgfpathlineto{\pgfqpoint{3.137492in}{0.756216in}}%
\pgfpathlineto{\pgfqpoint{3.138056in}{0.756226in}}%
\pgfpathlineto{\pgfqpoint{3.138620in}{0.756236in}}%
\pgfpathlineto{\pgfqpoint{3.139185in}{0.756246in}}%
\pgfpathlineto{\pgfqpoint{3.139749in}{0.756256in}}%
\pgfpathlineto{\pgfqpoint{3.140313in}{0.756266in}}%
\pgfpathlineto{\pgfqpoint{3.140878in}{0.756277in}}%
\pgfpathlineto{\pgfqpoint{3.141442in}{0.756287in}}%
\pgfpathlineto{\pgfqpoint{3.142007in}{0.756297in}}%
\pgfpathlineto{\pgfqpoint{3.142571in}{0.756307in}}%
\pgfpathlineto{\pgfqpoint{3.143135in}{0.756317in}}%
\pgfpathlineto{\pgfqpoint{3.143700in}{0.756327in}}%
\pgfpathlineto{\pgfqpoint{3.144264in}{0.756338in}}%
\pgfpathlineto{\pgfqpoint{3.144828in}{0.756348in}}%
\pgfpathlineto{\pgfqpoint{3.145393in}{0.756358in}}%
\pgfpathlineto{\pgfqpoint{3.145957in}{0.756368in}}%
\pgfpathlineto{\pgfqpoint{3.146521in}{0.756378in}}%
\pgfpathlineto{\pgfqpoint{3.147086in}{0.756389in}}%
\pgfpathlineto{\pgfqpoint{3.147650in}{0.756399in}}%
\pgfpathlineto{\pgfqpoint{3.148214in}{0.756409in}}%
\pgfpathlineto{\pgfqpoint{3.148779in}{0.756419in}}%
\pgfpathlineto{\pgfqpoint{3.149343in}{0.756429in}}%
\pgfpathlineto{\pgfqpoint{3.149907in}{0.756439in}}%
\pgfpathlineto{\pgfqpoint{3.150472in}{0.756450in}}%
\pgfpathlineto{\pgfqpoint{3.151036in}{0.756460in}}%
\pgfpathlineto{\pgfqpoint{3.151600in}{0.756470in}}%
\pgfpathlineto{\pgfqpoint{3.152165in}{0.756480in}}%
\pgfpathlineto{\pgfqpoint{3.152729in}{0.756490in}}%
\pgfpathlineto{\pgfqpoint{3.153293in}{0.756500in}}%
\pgfpathlineto{\pgfqpoint{3.153858in}{0.756511in}}%
\pgfpathlineto{\pgfqpoint{3.154422in}{0.756521in}}%
\pgfpathlineto{\pgfqpoint{3.154986in}{0.756531in}}%
\pgfpathlineto{\pgfqpoint{3.155551in}{0.756541in}}%
\pgfpathlineto{\pgfqpoint{3.156115in}{0.756551in}}%
\pgfpathlineto{\pgfqpoint{3.156680in}{0.756562in}}%
\pgfpathlineto{\pgfqpoint{3.157244in}{0.756572in}}%
\pgfpathlineto{\pgfqpoint{3.157808in}{0.756582in}}%
\pgfpathlineto{\pgfqpoint{3.158373in}{0.756592in}}%
\pgfpathlineto{\pgfqpoint{3.158937in}{0.756602in}}%
\pgfpathlineto{\pgfqpoint{3.159501in}{0.756612in}}%
\pgfpathlineto{\pgfqpoint{3.160066in}{0.756623in}}%
\pgfpathlineto{\pgfqpoint{3.160630in}{0.756633in}}%
\pgfpathlineto{\pgfqpoint{3.161194in}{0.756643in}}%
\pgfpathlineto{\pgfqpoint{3.161759in}{0.756653in}}%
\pgfpathlineto{\pgfqpoint{3.162323in}{0.756663in}}%
\pgfpathlineto{\pgfqpoint{3.162887in}{0.756673in}}%
\pgfpathlineto{\pgfqpoint{3.163452in}{0.756684in}}%
\pgfpathlineto{\pgfqpoint{3.164016in}{0.756694in}}%
\pgfpathlineto{\pgfqpoint{3.164580in}{0.756704in}}%
\pgfpathlineto{\pgfqpoint{3.165145in}{0.756714in}}%
\pgfpathlineto{\pgfqpoint{3.165709in}{0.756724in}}%
\pgfpathlineto{\pgfqpoint{3.166273in}{0.756734in}}%
\pgfpathlineto{\pgfqpoint{3.166838in}{0.756745in}}%
\pgfpathlineto{\pgfqpoint{3.167402in}{0.756755in}}%
\pgfpathlineto{\pgfqpoint{3.167966in}{0.756765in}}%
\pgfpathlineto{\pgfqpoint{3.168531in}{0.756775in}}%
\pgfpathlineto{\pgfqpoint{3.169095in}{0.756785in}}%
\pgfpathlineto{\pgfqpoint{3.169659in}{0.756796in}}%
\pgfpathlineto{\pgfqpoint{3.170224in}{0.756806in}}%
\pgfpathlineto{\pgfqpoint{3.170788in}{0.756816in}}%
\pgfpathlineto{\pgfqpoint{3.171352in}{0.756826in}}%
\pgfpathlineto{\pgfqpoint{3.171917in}{0.756836in}}%
\pgfpathlineto{\pgfqpoint{3.172481in}{0.756846in}}%
\pgfpathlineto{\pgfqpoint{3.173046in}{0.756857in}}%
\pgfpathlineto{\pgfqpoint{3.173610in}{0.756867in}}%
\pgfpathlineto{\pgfqpoint{3.174174in}{0.756877in}}%
\pgfpathlineto{\pgfqpoint{3.174739in}{0.756887in}}%
\pgfpathlineto{\pgfqpoint{3.175303in}{0.756897in}}%
\pgfpathlineto{\pgfqpoint{3.175867in}{0.756907in}}%
\pgfpathlineto{\pgfqpoint{3.176432in}{0.756918in}}%
\pgfpathlineto{\pgfqpoint{3.176996in}{0.756928in}}%
\pgfpathlineto{\pgfqpoint{3.177560in}{0.756938in}}%
\pgfpathlineto{\pgfqpoint{3.178125in}{0.756948in}}%
\pgfpathlineto{\pgfqpoint{3.178689in}{0.756958in}}%
\pgfpathlineto{\pgfqpoint{3.179253in}{0.756969in}}%
\pgfpathlineto{\pgfqpoint{3.179818in}{0.756979in}}%
\pgfpathlineto{\pgfqpoint{3.180382in}{0.756989in}}%
\pgfpathlineto{\pgfqpoint{3.180946in}{0.756999in}}%
\pgfpathlineto{\pgfqpoint{3.181511in}{0.757009in}}%
\pgfpathlineto{\pgfqpoint{3.182075in}{0.757019in}}%
\pgfpathlineto{\pgfqpoint{3.182639in}{0.757030in}}%
\pgfpathlineto{\pgfqpoint{3.183204in}{0.757040in}}%
\pgfpathlineto{\pgfqpoint{3.183768in}{0.756672in}}%
\pgfpathlineto{\pgfqpoint{3.184332in}{0.756154in}}%
\pgfpathlineto{\pgfqpoint{3.184897in}{0.757224in}}%
\pgfpathlineto{\pgfqpoint{3.185461in}{0.757331in}}%
\pgfpathlineto{\pgfqpoint{3.186025in}{0.757340in}}%
\pgfpathlineto{\pgfqpoint{3.186590in}{0.757339in}}%
\pgfpathlineto{\pgfqpoint{3.187154in}{0.757337in}}%
\pgfpathlineto{\pgfqpoint{3.187719in}{0.757336in}}%
\pgfpathlineto{\pgfqpoint{3.188283in}{0.757334in}}%
\pgfpathlineto{\pgfqpoint{3.188847in}{0.757333in}}%
\pgfpathlineto{\pgfqpoint{3.189412in}{0.757332in}}%
\pgfpathlineto{\pgfqpoint{3.189976in}{0.757330in}}%
\pgfpathlineto{\pgfqpoint{3.190540in}{0.757329in}}%
\pgfpathlineto{\pgfqpoint{3.191105in}{0.757327in}}%
\pgfpathlineto{\pgfqpoint{3.191669in}{0.757326in}}%
\pgfpathlineto{\pgfqpoint{3.192233in}{0.757325in}}%
\pgfpathlineto{\pgfqpoint{3.192798in}{0.757323in}}%
\pgfpathlineto{\pgfqpoint{3.193362in}{0.757322in}}%
\pgfpathlineto{\pgfqpoint{3.193926in}{0.757320in}}%
\pgfpathlineto{\pgfqpoint{3.194491in}{0.757319in}}%
\pgfpathlineto{\pgfqpoint{3.195055in}{0.757318in}}%
\pgfpathlineto{\pgfqpoint{3.195619in}{0.757316in}}%
\pgfpathlineto{\pgfqpoint{3.196184in}{0.757315in}}%
\pgfpathlineto{\pgfqpoint{3.196748in}{0.757313in}}%
\pgfpathlineto{\pgfqpoint{3.197312in}{0.757312in}}%
\pgfpathlineto{\pgfqpoint{3.197877in}{0.757310in}}%
\pgfpathlineto{\pgfqpoint{3.198441in}{0.757309in}}%
\pgfpathlineto{\pgfqpoint{3.199005in}{0.757308in}}%
\pgfpathlineto{\pgfqpoint{3.199570in}{0.757306in}}%
\pgfpathlineto{\pgfqpoint{3.200134in}{0.757305in}}%
\pgfpathlineto{\pgfqpoint{3.200698in}{0.757303in}}%
\pgfpathlineto{\pgfqpoint{3.201263in}{0.757302in}}%
\pgfpathlineto{\pgfqpoint{3.201827in}{0.757301in}}%
\pgfpathlineto{\pgfqpoint{3.202392in}{0.757299in}}%
\pgfpathlineto{\pgfqpoint{3.202956in}{0.757298in}}%
\pgfpathlineto{\pgfqpoint{3.203520in}{0.757296in}}%
\pgfpathlineto{\pgfqpoint{3.204085in}{0.757295in}}%
\pgfpathlineto{\pgfqpoint{3.204649in}{0.757294in}}%
\pgfpathlineto{\pgfqpoint{3.205213in}{0.757292in}}%
\pgfpathlineto{\pgfqpoint{3.205778in}{0.757291in}}%
\pgfpathlineto{\pgfqpoint{3.206342in}{0.757289in}}%
\pgfpathlineto{\pgfqpoint{3.206906in}{0.757288in}}%
\pgfpathlineto{\pgfqpoint{3.207471in}{0.757287in}}%
\pgfpathlineto{\pgfqpoint{3.208035in}{0.757285in}}%
\pgfpathlineto{\pgfqpoint{3.208599in}{0.757284in}}%
\pgfpathlineto{\pgfqpoint{3.209164in}{0.757282in}}%
\pgfpathlineto{\pgfqpoint{3.209728in}{0.757281in}}%
\pgfpathlineto{\pgfqpoint{3.210292in}{0.757280in}}%
\pgfpathlineto{\pgfqpoint{3.210857in}{0.757278in}}%
\pgfpathlineto{\pgfqpoint{3.211421in}{0.757277in}}%
\pgfpathlineto{\pgfqpoint{3.211985in}{0.757275in}}%
\pgfpathlineto{\pgfqpoint{3.212550in}{0.757274in}}%
\pgfpathlineto{\pgfqpoint{3.213114in}{0.757362in}}%
\pgfpathlineto{\pgfqpoint{3.213678in}{0.757594in}}%
\pgfpathlineto{\pgfqpoint{3.214243in}{0.757657in}}%
\pgfpathlineto{\pgfqpoint{3.214807in}{0.757716in}}%
\pgfpathlineto{\pgfqpoint{3.215371in}{0.757775in}}%
\pgfpathlineto{\pgfqpoint{3.215936in}{0.757834in}}%
\pgfpathlineto{\pgfqpoint{3.216500in}{0.757894in}}%
\pgfpathlineto{\pgfqpoint{3.217064in}{0.757953in}}%
\pgfpathlineto{\pgfqpoint{3.217629in}{0.758012in}}%
\pgfpathlineto{\pgfqpoint{3.218193in}{0.758071in}}%
\pgfpathlineto{\pgfqpoint{3.218758in}{0.758130in}}%
\pgfpathlineto{\pgfqpoint{3.219322in}{0.758190in}}%
\pgfpathlineto{\pgfqpoint{3.219886in}{0.758249in}}%
\pgfpathlineto{\pgfqpoint{3.220451in}{0.758308in}}%
\pgfpathlineto{\pgfqpoint{3.221015in}{0.758367in}}%
\pgfpathlineto{\pgfqpoint{3.221579in}{0.758427in}}%
\pgfpathlineto{\pgfqpoint{3.222144in}{0.758486in}}%
\pgfpathlineto{\pgfqpoint{3.222708in}{0.758545in}}%
\pgfpathlineto{\pgfqpoint{3.223272in}{0.758604in}}%
\pgfpathlineto{\pgfqpoint{3.223837in}{0.758664in}}%
\pgfpathlineto{\pgfqpoint{3.224401in}{0.758723in}}%
\pgfpathlineto{\pgfqpoint{3.224965in}{0.758782in}}%
\pgfpathlineto{\pgfqpoint{3.225530in}{0.758841in}}%
\pgfpathlineto{\pgfqpoint{3.226094in}{0.758901in}}%
\pgfpathlineto{\pgfqpoint{3.226658in}{0.758959in}}%
\pgfpathlineto{\pgfqpoint{3.227223in}{0.758979in}}%
\pgfpathlineto{\pgfqpoint{3.227787in}{0.758978in}}%
\pgfpathlineto{\pgfqpoint{3.228351in}{0.758977in}}%
\pgfpathlineto{\pgfqpoint{3.228916in}{0.758975in}}%
\pgfpathlineto{\pgfqpoint{3.229480in}{0.758974in}}%
\pgfpathlineto{\pgfqpoint{3.230044in}{0.758973in}}%
\pgfpathlineto{\pgfqpoint{3.230609in}{0.758971in}}%
\pgfpathlineto{\pgfqpoint{3.231173in}{0.758970in}}%
\pgfpathlineto{\pgfqpoint{3.231737in}{0.758969in}}%
\pgfpathlineto{\pgfqpoint{3.232302in}{0.758968in}}%
\pgfpathlineto{\pgfqpoint{3.232866in}{0.758966in}}%
\pgfpathlineto{\pgfqpoint{3.233431in}{0.758965in}}%
\pgfpathlineto{\pgfqpoint{3.233995in}{0.758964in}}%
\pgfpathlineto{\pgfqpoint{3.234559in}{0.758963in}}%
\pgfpathlineto{\pgfqpoint{3.235124in}{0.758961in}}%
\pgfpathlineto{\pgfqpoint{3.235688in}{0.758960in}}%
\pgfpathlineto{\pgfqpoint{3.236252in}{0.758959in}}%
\pgfpathlineto{\pgfqpoint{3.236817in}{0.758957in}}%
\pgfpathlineto{\pgfqpoint{3.237381in}{0.758956in}}%
\pgfpathlineto{\pgfqpoint{3.237945in}{0.758955in}}%
\pgfpathlineto{\pgfqpoint{3.238510in}{0.758954in}}%
\pgfpathlineto{\pgfqpoint{3.239074in}{0.758952in}}%
\pgfpathlineto{\pgfqpoint{3.239638in}{0.758951in}}%
\pgfpathlineto{\pgfqpoint{3.240203in}{0.758950in}}%
\pgfpathlineto{\pgfqpoint{3.240767in}{0.758948in}}%
\pgfpathlineto{\pgfqpoint{3.241331in}{0.758947in}}%
\pgfpathlineto{\pgfqpoint{3.241896in}{0.758946in}}%
\pgfpathlineto{\pgfqpoint{3.242460in}{0.758945in}}%
\pgfpathlineto{\pgfqpoint{3.243024in}{0.758943in}}%
\pgfpathlineto{\pgfqpoint{3.243589in}{0.758942in}}%
\pgfpathlineto{\pgfqpoint{3.244153in}{0.758941in}}%
\pgfpathlineto{\pgfqpoint{3.244717in}{0.758940in}}%
\pgfpathlineto{\pgfqpoint{3.245282in}{0.758938in}}%
\pgfpathlineto{\pgfqpoint{3.245846in}{0.758937in}}%
\pgfpathlineto{\pgfqpoint{3.246410in}{0.758936in}}%
\pgfpathlineto{\pgfqpoint{3.246975in}{0.758934in}}%
\pgfpathlineto{\pgfqpoint{3.247539in}{0.758933in}}%
\pgfpathlineto{\pgfqpoint{3.248104in}{0.758932in}}%
\pgfpathlineto{\pgfqpoint{3.248668in}{0.758931in}}%
\pgfpathlineto{\pgfqpoint{3.249232in}{0.758929in}}%
\pgfpathlineto{\pgfqpoint{3.249797in}{0.758928in}}%
\pgfpathlineto{\pgfqpoint{3.250361in}{0.758927in}}%
\pgfpathlineto{\pgfqpoint{3.250925in}{0.758925in}}%
\pgfpathlineto{\pgfqpoint{3.251490in}{0.758924in}}%
\pgfpathlineto{\pgfqpoint{3.252054in}{0.758923in}}%
\pgfpathlineto{\pgfqpoint{3.252618in}{0.758922in}}%
\pgfpathlineto{\pgfqpoint{3.253183in}{0.758920in}}%
\pgfpathlineto{\pgfqpoint{3.253747in}{0.758919in}}%
\pgfpathlineto{\pgfqpoint{3.254311in}{0.758918in}}%
\pgfpathlineto{\pgfqpoint{3.254876in}{0.758916in}}%
\pgfpathlineto{\pgfqpoint{3.255440in}{0.758915in}}%
\pgfpathlineto{\pgfqpoint{3.256004in}{0.758914in}}%
\pgfpathlineto{\pgfqpoint{3.256569in}{0.758913in}}%
\pgfpathlineto{\pgfqpoint{3.257133in}{0.758911in}}%
\pgfpathlineto{\pgfqpoint{3.257697in}{0.758910in}}%
\pgfpathlineto{\pgfqpoint{3.258262in}{0.758909in}}%
\pgfpathlineto{\pgfqpoint{3.258826in}{0.758908in}}%
\pgfpathlineto{\pgfqpoint{3.259390in}{0.758906in}}%
\pgfpathlineto{\pgfqpoint{3.259955in}{0.758905in}}%
\pgfpathlineto{\pgfqpoint{3.260519in}{0.758904in}}%
\pgfpathlineto{\pgfqpoint{3.261083in}{0.758902in}}%
\pgfpathlineto{\pgfqpoint{3.261648in}{0.758901in}}%
\pgfpathlineto{\pgfqpoint{3.262212in}{0.758900in}}%
\pgfpathlineto{\pgfqpoint{3.262776in}{0.758899in}}%
\pgfpathlineto{\pgfqpoint{3.263341in}{0.758897in}}%
\pgfpathlineto{\pgfqpoint{3.263905in}{0.758896in}}%
\pgfpathlineto{\pgfqpoint{3.264470in}{0.758895in}}%
\pgfpathlineto{\pgfqpoint{3.265034in}{0.758893in}}%
\pgfpathlineto{\pgfqpoint{3.265598in}{0.758892in}}%
\pgfpathlineto{\pgfqpoint{3.266163in}{0.758892in}}%
\pgfpathlineto{\pgfqpoint{3.266727in}{0.759126in}}%
\pgfpathlineto{\pgfqpoint{3.267291in}{0.759547in}}%
\pgfpathlineto{\pgfqpoint{3.267856in}{0.759969in}}%
\pgfpathlineto{\pgfqpoint{3.268420in}{0.760351in}}%
\pgfpathlineto{\pgfqpoint{3.268984in}{0.760431in}}%
\pgfpathlineto{\pgfqpoint{3.269549in}{0.760449in}}%
\pgfpathlineto{\pgfqpoint{3.270113in}{0.760466in}}%
\pgfpathlineto{\pgfqpoint{3.270677in}{0.760483in}}%
\pgfpathlineto{\pgfqpoint{3.271242in}{0.760501in}}%
\pgfpathlineto{\pgfqpoint{3.271806in}{0.760518in}}%
\pgfpathlineto{\pgfqpoint{3.272370in}{0.760535in}}%
\pgfpathlineto{\pgfqpoint{3.272935in}{0.760553in}}%
\pgfpathlineto{\pgfqpoint{3.273499in}{0.760570in}}%
\pgfpathlineto{\pgfqpoint{3.274063in}{0.760587in}}%
\pgfpathlineto{\pgfqpoint{3.274628in}{0.760605in}}%
\pgfpathlineto{\pgfqpoint{3.275192in}{0.760622in}}%
\pgfpathlineto{\pgfqpoint{3.275756in}{0.760639in}}%
\pgfpathlineto{\pgfqpoint{3.276321in}{0.760657in}}%
\pgfpathlineto{\pgfqpoint{3.276885in}{0.760674in}}%
\pgfpathlineto{\pgfqpoint{3.277449in}{0.760691in}}%
\pgfpathlineto{\pgfqpoint{3.278014in}{0.760709in}}%
\pgfpathlineto{\pgfqpoint{3.278578in}{0.760726in}}%
\pgfpathlineto{\pgfqpoint{3.279143in}{0.760743in}}%
\pgfpathlineto{\pgfqpoint{3.279707in}{0.760761in}}%
\pgfpathlineto{\pgfqpoint{3.280271in}{0.760778in}}%
\pgfpathlineto{\pgfqpoint{3.280836in}{0.760794in}}%
\pgfpathlineto{\pgfqpoint{3.281400in}{0.760806in}}%
\pgfpathlineto{\pgfqpoint{3.281964in}{0.760818in}}%
\pgfpathlineto{\pgfqpoint{3.282529in}{0.760830in}}%
\pgfpathlineto{\pgfqpoint{3.283093in}{0.760841in}}%
\pgfpathlineto{\pgfqpoint{3.283657in}{0.760853in}}%
\pgfpathlineto{\pgfqpoint{3.284222in}{0.760865in}}%
\pgfpathlineto{\pgfqpoint{3.284786in}{0.760876in}}%
\pgfpathlineto{\pgfqpoint{3.285350in}{0.760888in}}%
\pgfpathlineto{\pgfqpoint{3.285915in}{0.760900in}}%
\pgfpathlineto{\pgfqpoint{3.286479in}{0.760912in}}%
\pgfpathlineto{\pgfqpoint{3.287043in}{0.760923in}}%
\pgfpathlineto{\pgfqpoint{3.287608in}{0.760935in}}%
\pgfpathlineto{\pgfqpoint{3.288172in}{0.760947in}}%
\pgfpathlineto{\pgfqpoint{3.288736in}{0.760958in}}%
\pgfpathlineto{\pgfqpoint{3.289301in}{0.760970in}}%
\pgfpathlineto{\pgfqpoint{3.289865in}{0.760982in}}%
\pgfpathlineto{\pgfqpoint{3.290429in}{0.760994in}}%
\pgfpathlineto{\pgfqpoint{3.290994in}{0.761005in}}%
\pgfpathlineto{\pgfqpoint{3.291558in}{0.761017in}}%
\pgfpathlineto{\pgfqpoint{3.292122in}{0.761029in}}%
\pgfpathlineto{\pgfqpoint{3.292687in}{0.761040in}}%
\pgfpathlineto{\pgfqpoint{3.293251in}{0.761052in}}%
\pgfpathlineto{\pgfqpoint{3.293816in}{0.761064in}}%
\pgfpathlineto{\pgfqpoint{3.294380in}{0.761071in}}%
\pgfpathlineto{\pgfqpoint{3.294944in}{0.761070in}}%
\pgfpathlineto{\pgfqpoint{3.295509in}{0.761068in}}%
\pgfpathlineto{\pgfqpoint{3.296073in}{0.761067in}}%
\pgfpathlineto{\pgfqpoint{3.296637in}{0.761066in}}%
\pgfpathlineto{\pgfqpoint{3.297202in}{0.761064in}}%
\pgfpathlineto{\pgfqpoint{3.297766in}{0.761063in}}%
\pgfpathlineto{\pgfqpoint{3.298330in}{0.761061in}}%
\pgfpathlineto{\pgfqpoint{3.298895in}{0.761060in}}%
\pgfpathlineto{\pgfqpoint{3.299459in}{0.761059in}}%
\pgfpathlineto{\pgfqpoint{3.300023in}{0.761057in}}%
\pgfpathlineto{\pgfqpoint{3.300588in}{0.761056in}}%
\pgfpathlineto{\pgfqpoint{3.301152in}{0.761054in}}%
\pgfpathlineto{\pgfqpoint{3.301716in}{0.761053in}}%
\pgfpathlineto{\pgfqpoint{3.302281in}{0.761051in}}%
\pgfpathlineto{\pgfqpoint{3.302845in}{0.761050in}}%
\pgfpathlineto{\pgfqpoint{3.303409in}{0.761049in}}%
\pgfpathlineto{\pgfqpoint{3.303974in}{0.761047in}}%
\pgfpathlineto{\pgfqpoint{3.304538in}{0.761046in}}%
\pgfpathlineto{\pgfqpoint{3.305102in}{0.761044in}}%
\pgfpathlineto{\pgfqpoint{3.305667in}{0.761043in}}%
\pgfpathlineto{\pgfqpoint{3.306231in}{0.761042in}}%
\pgfpathlineto{\pgfqpoint{3.306795in}{0.761040in}}%
\pgfpathlineto{\pgfqpoint{3.307360in}{0.761039in}}%
\pgfpathlineto{\pgfqpoint{3.307924in}{0.761037in}}%
\pgfpathlineto{\pgfqpoint{3.308488in}{0.761036in}}%
\pgfpathlineto{\pgfqpoint{3.309053in}{0.761035in}}%
\pgfpathlineto{\pgfqpoint{3.309617in}{0.761033in}}%
\pgfpathlineto{\pgfqpoint{3.310182in}{0.761032in}}%
\pgfpathlineto{\pgfqpoint{3.310746in}{0.761030in}}%
\pgfpathlineto{\pgfqpoint{3.311310in}{0.761029in}}%
\pgfpathlineto{\pgfqpoint{3.311875in}{0.761028in}}%
\pgfpathlineto{\pgfqpoint{3.312439in}{0.761026in}}%
\pgfpathlineto{\pgfqpoint{3.313003in}{0.761025in}}%
\pgfpathlineto{\pgfqpoint{3.313568in}{0.761023in}}%
\pgfpathlineto{\pgfqpoint{3.314132in}{0.761022in}}%
\pgfpathlineto{\pgfqpoint{3.314696in}{0.761021in}}%
\pgfpathlineto{\pgfqpoint{3.315261in}{0.761019in}}%
\pgfpathlineto{\pgfqpoint{3.315825in}{0.761018in}}%
\pgfpathlineto{\pgfqpoint{3.316389in}{0.761016in}}%
\pgfpathlineto{\pgfqpoint{3.316954in}{0.761015in}}%
\pgfpathlineto{\pgfqpoint{3.317518in}{0.761014in}}%
\pgfpathlineto{\pgfqpoint{3.318082in}{0.761012in}}%
\pgfpathlineto{\pgfqpoint{3.318647in}{0.761011in}}%
\pgfpathlineto{\pgfqpoint{3.319211in}{0.761009in}}%
\pgfpathlineto{\pgfqpoint{3.319775in}{0.761008in}}%
\pgfpathlineto{\pgfqpoint{3.320340in}{0.761006in}}%
\pgfpathlineto{\pgfqpoint{3.320904in}{0.761005in}}%
\pgfpathlineto{\pgfqpoint{3.321468in}{0.761004in}}%
\pgfpathlineto{\pgfqpoint{3.322033in}{0.761002in}}%
\pgfpathlineto{\pgfqpoint{3.322597in}{0.761001in}}%
\pgfpathlineto{\pgfqpoint{3.323161in}{0.760999in}}%
\pgfpathlineto{\pgfqpoint{3.323726in}{0.760998in}}%
\pgfpathlineto{\pgfqpoint{3.324290in}{0.760997in}}%
\pgfpathlineto{\pgfqpoint{3.324855in}{0.760995in}}%
\pgfpathlineto{\pgfqpoint{3.325419in}{0.760994in}}%
\pgfpathlineto{\pgfqpoint{3.325983in}{0.760992in}}%
\pgfpathlineto{\pgfqpoint{3.326548in}{0.760991in}}%
\pgfpathlineto{\pgfqpoint{3.327112in}{0.760990in}}%
\pgfpathlineto{\pgfqpoint{3.327676in}{0.760988in}}%
\pgfpathlineto{\pgfqpoint{3.328241in}{0.760987in}}%
\pgfpathlineto{\pgfqpoint{3.328805in}{0.760985in}}%
\pgfpathlineto{\pgfqpoint{3.329369in}{0.760984in}}%
\pgfpathlineto{\pgfqpoint{3.329934in}{0.760983in}}%
\pgfpathlineto{\pgfqpoint{3.330498in}{0.760981in}}%
\pgfpathlineto{\pgfqpoint{3.331062in}{0.760980in}}%
\pgfpathlineto{\pgfqpoint{3.331627in}{0.760978in}}%
\pgfpathlineto{\pgfqpoint{3.332191in}{0.760977in}}%
\pgfpathlineto{\pgfqpoint{3.332755in}{0.760976in}}%
\pgfpathlineto{\pgfqpoint{3.333320in}{0.760974in}}%
\pgfpathlineto{\pgfqpoint{3.333884in}{0.760973in}}%
\pgfpathlineto{\pgfqpoint{3.334448in}{0.760971in}}%
\pgfpathlineto{\pgfqpoint{3.335013in}{0.760970in}}%
\pgfpathlineto{\pgfqpoint{3.335577in}{0.760968in}}%
\pgfpathlineto{\pgfqpoint{3.336141in}{0.760967in}}%
\pgfpathlineto{\pgfqpoint{3.336706in}{0.760966in}}%
\pgfpathlineto{\pgfqpoint{3.337270in}{0.760964in}}%
\pgfpathlineto{\pgfqpoint{3.337834in}{0.760963in}}%
\pgfpathlineto{\pgfqpoint{3.338399in}{0.760961in}}%
\pgfpathlineto{\pgfqpoint{3.338963in}{0.760960in}}%
\pgfpathlineto{\pgfqpoint{3.339528in}{0.760959in}}%
\pgfpathlineto{\pgfqpoint{3.340092in}{0.760957in}}%
\pgfpathlineto{\pgfqpoint{3.340656in}{0.760956in}}%
\pgfpathlineto{\pgfqpoint{3.341221in}{0.760954in}}%
\pgfpathlineto{\pgfqpoint{3.341785in}{0.760953in}}%
\pgfpathlineto{\pgfqpoint{3.342349in}{0.760952in}}%
\pgfpathlineto{\pgfqpoint{3.342914in}{0.760950in}}%
\pgfpathlineto{\pgfqpoint{3.343478in}{0.760949in}}%
\pgfpathlineto{\pgfqpoint{3.344042in}{0.760947in}}%
\pgfpathlineto{\pgfqpoint{3.344607in}{0.760946in}}%
\pgfpathlineto{\pgfqpoint{3.345171in}{0.760945in}}%
\pgfpathlineto{\pgfqpoint{3.345735in}{0.760943in}}%
\pgfpathlineto{\pgfqpoint{3.346300in}{0.760942in}}%
\pgfpathlineto{\pgfqpoint{3.346864in}{0.760940in}}%
\pgfpathlineto{\pgfqpoint{3.347428in}{0.760939in}}%
\pgfpathlineto{\pgfqpoint{3.347993in}{0.760938in}}%
\pgfpathlineto{\pgfqpoint{3.348557in}{0.760936in}}%
\pgfpathlineto{\pgfqpoint{3.349121in}{0.760935in}}%
\pgfpathlineto{\pgfqpoint{3.349686in}{0.760933in}}%
\pgfpathlineto{\pgfqpoint{3.350250in}{0.760932in}}%
\pgfpathlineto{\pgfqpoint{3.350814in}{0.760931in}}%
\pgfpathlineto{\pgfqpoint{3.351379in}{0.760929in}}%
\pgfpathlineto{\pgfqpoint{3.351943in}{0.760928in}}%
\pgfpathlineto{\pgfqpoint{3.352507in}{0.760926in}}%
\pgfpathlineto{\pgfqpoint{3.353072in}{0.760925in}}%
\pgfpathlineto{\pgfqpoint{3.353636in}{0.760923in}}%
\pgfpathlineto{\pgfqpoint{3.354200in}{0.760922in}}%
\pgfpathlineto{\pgfqpoint{3.354765in}{0.760921in}}%
\pgfpathlineto{\pgfqpoint{3.355329in}{0.760919in}}%
\pgfpathlineto{\pgfqpoint{3.355894in}{0.760918in}}%
\pgfpathlineto{\pgfqpoint{3.356458in}{0.760916in}}%
\pgfpathlineto{\pgfqpoint{3.357022in}{0.760915in}}%
\pgfpathlineto{\pgfqpoint{3.357587in}{0.760914in}}%
\pgfpathlineto{\pgfqpoint{3.358151in}{0.760912in}}%
\pgfpathlineto{\pgfqpoint{3.358715in}{0.760911in}}%
\pgfpathlineto{\pgfqpoint{3.359280in}{0.760909in}}%
\pgfpathlineto{\pgfqpoint{3.359844in}{0.760908in}}%
\pgfpathlineto{\pgfqpoint{3.360408in}{0.760907in}}%
\pgfpathlineto{\pgfqpoint{3.360973in}{0.760905in}}%
\pgfpathlineto{\pgfqpoint{3.361537in}{0.760904in}}%
\pgfpathlineto{\pgfqpoint{3.362101in}{0.760902in}}%
\pgfpathlineto{\pgfqpoint{3.362666in}{0.760901in}}%
\pgfpathlineto{\pgfqpoint{3.363230in}{0.760900in}}%
\pgfpathlineto{\pgfqpoint{3.363794in}{0.760898in}}%
\pgfpathlineto{\pgfqpoint{3.364359in}{0.760897in}}%
\pgfpathlineto{\pgfqpoint{3.364923in}{0.760895in}}%
\pgfpathlineto{\pgfqpoint{3.365487in}{0.760894in}}%
\pgfpathlineto{\pgfqpoint{3.366052in}{0.760893in}}%
\pgfpathlineto{\pgfqpoint{3.366616in}{0.760891in}}%
\pgfpathlineto{\pgfqpoint{3.367180in}{0.760890in}}%
\pgfpathlineto{\pgfqpoint{3.367745in}{0.760888in}}%
\pgfpathlineto{\pgfqpoint{3.368309in}{0.760887in}}%
\pgfpathlineto{\pgfqpoint{3.368873in}{0.760886in}}%
\pgfpathlineto{\pgfqpoint{3.369438in}{0.760884in}}%
\pgfpathlineto{\pgfqpoint{3.370002in}{0.760883in}}%
\pgfpathlineto{\pgfqpoint{3.370567in}{0.760881in}}%
\pgfpathlineto{\pgfqpoint{3.371131in}{0.760880in}}%
\pgfpathlineto{\pgfqpoint{3.371695in}{0.760878in}}%
\pgfpathlineto{\pgfqpoint{3.372260in}{0.760877in}}%
\pgfpathlineto{\pgfqpoint{3.372824in}{0.760876in}}%
\pgfpathlineto{\pgfqpoint{3.373388in}{0.760874in}}%
\pgfpathlineto{\pgfqpoint{3.373953in}{0.760873in}}%
\pgfpathlineto{\pgfqpoint{3.374517in}{0.760871in}}%
\pgfpathlineto{\pgfqpoint{3.375081in}{0.760870in}}%
\pgfpathlineto{\pgfqpoint{3.375646in}{0.760869in}}%
\pgfpathlineto{\pgfqpoint{3.376210in}{0.760867in}}%
\pgfpathlineto{\pgfqpoint{3.376774in}{0.760866in}}%
\pgfpathlineto{\pgfqpoint{3.377339in}{0.760864in}}%
\pgfpathlineto{\pgfqpoint{3.377903in}{0.760863in}}%
\pgfpathlineto{\pgfqpoint{3.378467in}{0.760862in}}%
\pgfpathlineto{\pgfqpoint{3.379032in}{0.760860in}}%
\pgfpathlineto{\pgfqpoint{3.379596in}{0.760859in}}%
\pgfpathlineto{\pgfqpoint{3.380160in}{0.760857in}}%
\pgfpathlineto{\pgfqpoint{3.380725in}{0.760856in}}%
\pgfpathlineto{\pgfqpoint{3.381289in}{0.760855in}}%
\pgfpathlineto{\pgfqpoint{3.381853in}{0.760853in}}%
\pgfpathlineto{\pgfqpoint{3.382418in}{0.760852in}}%
\pgfpathlineto{\pgfqpoint{3.382982in}{0.760850in}}%
\pgfpathlineto{\pgfqpoint{3.383546in}{0.760849in}}%
\pgfpathlineto{\pgfqpoint{3.384111in}{0.760848in}}%
\pgfpathlineto{\pgfqpoint{3.384675in}{0.760846in}}%
\pgfpathlineto{\pgfqpoint{3.385240in}{0.760845in}}%
\pgfpathlineto{\pgfqpoint{3.385804in}{0.760843in}}%
\pgfpathlineto{\pgfqpoint{3.386368in}{0.760842in}}%
\pgfpathlineto{\pgfqpoint{3.386933in}{0.760841in}}%
\pgfpathlineto{\pgfqpoint{3.387497in}{0.760839in}}%
\pgfpathlineto{\pgfqpoint{3.388061in}{0.760838in}}%
\pgfpathlineto{\pgfqpoint{3.388626in}{0.760836in}}%
\pgfpathlineto{\pgfqpoint{3.389190in}{0.760835in}}%
\pgfpathlineto{\pgfqpoint{3.389754in}{0.760833in}}%
\pgfpathlineto{\pgfqpoint{3.390319in}{0.760832in}}%
\pgfpathlineto{\pgfqpoint{3.390883in}{0.760831in}}%
\pgfpathlineto{\pgfqpoint{3.391447in}{0.760829in}}%
\pgfpathlineto{\pgfqpoint{3.392012in}{0.760828in}}%
\pgfpathlineto{\pgfqpoint{3.392576in}{0.760826in}}%
\pgfpathlineto{\pgfqpoint{3.393140in}{0.760825in}}%
\pgfpathlineto{\pgfqpoint{3.393705in}{0.760824in}}%
\pgfpathlineto{\pgfqpoint{3.394269in}{0.760822in}}%
\pgfpathlineto{\pgfqpoint{3.394833in}{0.760821in}}%
\pgfpathlineto{\pgfqpoint{3.395398in}{0.760819in}}%
\pgfpathlineto{\pgfqpoint{3.395962in}{0.760818in}}%
\pgfpathlineto{\pgfqpoint{3.396526in}{0.760817in}}%
\pgfpathlineto{\pgfqpoint{3.397091in}{0.760815in}}%
\pgfpathlineto{\pgfqpoint{3.397655in}{0.760814in}}%
\pgfpathlineto{\pgfqpoint{3.398219in}{0.760812in}}%
\pgfpathlineto{\pgfqpoint{3.398784in}{0.760811in}}%
\pgfpathlineto{\pgfqpoint{3.399348in}{0.760810in}}%
\pgfpathlineto{\pgfqpoint{3.399913in}{0.760808in}}%
\pgfpathlineto{\pgfqpoint{3.400477in}{0.760807in}}%
\pgfpathlineto{\pgfqpoint{3.401041in}{0.760805in}}%
\pgfpathlineto{\pgfqpoint{3.401606in}{0.760804in}}%
\pgfpathlineto{\pgfqpoint{3.402170in}{0.760803in}}%
\pgfpathlineto{\pgfqpoint{3.402734in}{0.760801in}}%
\pgfpathlineto{\pgfqpoint{3.403299in}{0.760800in}}%
\pgfpathlineto{\pgfqpoint{3.403863in}{0.760798in}}%
\pgfpathlineto{\pgfqpoint{3.404427in}{0.760797in}}%
\pgfpathlineto{\pgfqpoint{3.404992in}{0.760795in}}%
\pgfpathlineto{\pgfqpoint{3.405556in}{0.760794in}}%
\pgfpathlineto{\pgfqpoint{3.406120in}{0.760793in}}%
\pgfpathlineto{\pgfqpoint{3.406685in}{0.760791in}}%
\pgfpathlineto{\pgfqpoint{3.407249in}{0.760790in}}%
\pgfpathlineto{\pgfqpoint{3.407813in}{0.760788in}}%
\pgfpathlineto{\pgfqpoint{3.408378in}{0.760787in}}%
\pgfpathlineto{\pgfqpoint{3.408942in}{0.760786in}}%
\pgfpathlineto{\pgfqpoint{3.409506in}{0.760784in}}%
\pgfpathlineto{\pgfqpoint{3.410071in}{0.760783in}}%
\pgfpathlineto{\pgfqpoint{3.410635in}{0.760781in}}%
\pgfpathlineto{\pgfqpoint{3.411199in}{0.760780in}}%
\pgfpathlineto{\pgfqpoint{3.411764in}{0.760779in}}%
\pgfpathlineto{\pgfqpoint{3.412328in}{0.760777in}}%
\pgfpathlineto{\pgfqpoint{3.412892in}{0.760776in}}%
\pgfpathlineto{\pgfqpoint{3.413457in}{0.760774in}}%
\pgfpathlineto{\pgfqpoint{3.414021in}{0.760773in}}%
\pgfpathlineto{\pgfqpoint{3.414585in}{0.760772in}}%
\pgfpathlineto{\pgfqpoint{3.415150in}{0.760770in}}%
\pgfpathlineto{\pgfqpoint{3.415714in}{0.760769in}}%
\pgfpathlineto{\pgfqpoint{3.416279in}{0.760767in}}%
\pgfpathlineto{\pgfqpoint{3.416843in}{0.760766in}}%
\pgfpathlineto{\pgfqpoint{3.417407in}{0.760765in}}%
\pgfpathlineto{\pgfqpoint{3.417972in}{0.760763in}}%
\pgfpathlineto{\pgfqpoint{3.418536in}{0.760762in}}%
\pgfpathlineto{\pgfqpoint{3.419100in}{0.760760in}}%
\pgfpathlineto{\pgfqpoint{3.419665in}{0.760759in}}%
\pgfpathlineto{\pgfqpoint{3.420229in}{0.760758in}}%
\pgfpathlineto{\pgfqpoint{3.420793in}{0.760756in}}%
\pgfpathlineto{\pgfqpoint{3.421358in}{0.760755in}}%
\pgfpathlineto{\pgfqpoint{3.421922in}{0.760753in}}%
\pgfpathlineto{\pgfqpoint{3.422486in}{0.760752in}}%
\pgfpathlineto{\pgfqpoint{3.423051in}{0.760750in}}%
\pgfpathlineto{\pgfqpoint{3.423615in}{0.760749in}}%
\pgfpathlineto{\pgfqpoint{3.424179in}{0.760748in}}%
\pgfpathlineto{\pgfqpoint{3.424744in}{0.760746in}}%
\pgfpathlineto{\pgfqpoint{3.425308in}{0.760745in}}%
\pgfpathlineto{\pgfqpoint{3.425872in}{0.760743in}}%
\pgfpathlineto{\pgfqpoint{3.426437in}{0.760742in}}%
\pgfpathlineto{\pgfqpoint{3.427001in}{0.760741in}}%
\pgfpathlineto{\pgfqpoint{3.427565in}{0.760739in}}%
\pgfpathlineto{\pgfqpoint{3.428130in}{0.760738in}}%
\pgfpathlineto{\pgfqpoint{3.428694in}{0.760736in}}%
\pgfpathlineto{\pgfqpoint{3.429258in}{0.760735in}}%
\pgfpathlineto{\pgfqpoint{3.429823in}{0.760734in}}%
\pgfpathlineto{\pgfqpoint{3.430387in}{0.760732in}}%
\pgfpathlineto{\pgfqpoint{3.430952in}{0.760731in}}%
\pgfpathlineto{\pgfqpoint{3.431516in}{0.760729in}}%
\pgfpathlineto{\pgfqpoint{3.432080in}{0.760728in}}%
\pgfpathlineto{\pgfqpoint{3.432645in}{0.760727in}}%
\pgfpathlineto{\pgfqpoint{3.433209in}{0.760725in}}%
\pgfpathlineto{\pgfqpoint{3.433773in}{0.760724in}}%
\pgfpathlineto{\pgfqpoint{3.434338in}{0.760722in}}%
\pgfpathlineto{\pgfqpoint{3.434902in}{0.760721in}}%
\pgfpathlineto{\pgfqpoint{3.435466in}{0.760720in}}%
\pgfpathlineto{\pgfqpoint{3.436031in}{0.760718in}}%
\pgfpathlineto{\pgfqpoint{3.436595in}{0.760717in}}%
\pgfpathlineto{\pgfqpoint{3.437159in}{0.760715in}}%
\pgfpathlineto{\pgfqpoint{3.437724in}{0.760714in}}%
\pgfpathlineto{\pgfqpoint{3.438288in}{0.760713in}}%
\pgfpathlineto{\pgfqpoint{3.438852in}{0.760711in}}%
\pgfpathlineto{\pgfqpoint{3.439417in}{0.760710in}}%
\pgfpathlineto{\pgfqpoint{3.439981in}{0.760708in}}%
\pgfpathlineto{\pgfqpoint{3.440545in}{0.760707in}}%
\pgfpathlineto{\pgfqpoint{3.441110in}{0.760705in}}%
\pgfpathlineto{\pgfqpoint{3.441674in}{0.760704in}}%
\pgfpathlineto{\pgfqpoint{3.442238in}{0.760703in}}%
\pgfpathlineto{\pgfqpoint{3.442803in}{0.760701in}}%
\pgfpathlineto{\pgfqpoint{3.443367in}{0.760700in}}%
\pgfpathlineto{\pgfqpoint{3.443931in}{0.760698in}}%
\pgfpathlineto{\pgfqpoint{3.444496in}{0.760697in}}%
\pgfpathlineto{\pgfqpoint{3.445060in}{0.760696in}}%
\pgfpathlineto{\pgfqpoint{3.445625in}{0.760694in}}%
\pgfpathlineto{\pgfqpoint{3.446189in}{0.760693in}}%
\pgfpathlineto{\pgfqpoint{3.446753in}{0.760691in}}%
\pgfpathlineto{\pgfqpoint{3.447318in}{0.760690in}}%
\pgfpathlineto{\pgfqpoint{3.447882in}{0.760689in}}%
\pgfpathlineto{\pgfqpoint{3.448446in}{0.760687in}}%
\pgfpathlineto{\pgfqpoint{3.449011in}{0.760686in}}%
\pgfpathlineto{\pgfqpoint{3.449575in}{0.760684in}}%
\pgfpathlineto{\pgfqpoint{3.450139in}{0.760683in}}%
\pgfpathlineto{\pgfqpoint{3.450704in}{0.760682in}}%
\pgfpathlineto{\pgfqpoint{3.451268in}{0.760680in}}%
\pgfpathlineto{\pgfqpoint{3.451832in}{0.760679in}}%
\pgfpathlineto{\pgfqpoint{3.452397in}{0.760677in}}%
\pgfpathlineto{\pgfqpoint{3.452961in}{0.760676in}}%
\pgfpathlineto{\pgfqpoint{3.453525in}{0.760675in}}%
\pgfpathlineto{\pgfqpoint{3.454090in}{0.760673in}}%
\pgfpathlineto{\pgfqpoint{3.454654in}{0.760672in}}%
\pgfpathlineto{\pgfqpoint{3.455218in}{0.760670in}}%
\pgfpathlineto{\pgfqpoint{3.455783in}{0.760669in}}%
\pgfpathlineto{\pgfqpoint{3.456347in}{0.760668in}}%
\pgfpathlineto{\pgfqpoint{3.456911in}{0.760666in}}%
\pgfpathlineto{\pgfqpoint{3.457476in}{0.760665in}}%
\pgfpathlineto{\pgfqpoint{3.458040in}{0.760663in}}%
\pgfpathlineto{\pgfqpoint{3.458604in}{0.760662in}}%
\pgfpathlineto{\pgfqpoint{3.459169in}{0.760660in}}%
\pgfpathlineto{\pgfqpoint{3.459733in}{0.760659in}}%
\pgfpathlineto{\pgfqpoint{3.460297in}{0.760658in}}%
\pgfpathlineto{\pgfqpoint{3.460862in}{0.760656in}}%
\pgfpathlineto{\pgfqpoint{3.461426in}{0.760655in}}%
\pgfpathlineto{\pgfqpoint{3.461991in}{0.760653in}}%
\pgfpathlineto{\pgfqpoint{3.462555in}{0.760652in}}%
\pgfpathlineto{\pgfqpoint{3.463119in}{0.760651in}}%
\pgfpathlineto{\pgfqpoint{3.463684in}{0.760649in}}%
\pgfpathlineto{\pgfqpoint{3.464248in}{0.760648in}}%
\pgfpathlineto{\pgfqpoint{3.464812in}{0.760646in}}%
\pgfpathlineto{\pgfqpoint{3.465377in}{0.760645in}}%
\pgfpathlineto{\pgfqpoint{3.465941in}{0.760644in}}%
\pgfpathlineto{\pgfqpoint{3.466505in}{0.760642in}}%
\pgfpathlineto{\pgfqpoint{3.467070in}{0.760641in}}%
\pgfpathlineto{\pgfqpoint{3.467634in}{0.760639in}}%
\pgfpathlineto{\pgfqpoint{3.468198in}{0.760638in}}%
\pgfpathlineto{\pgfqpoint{3.468763in}{0.760637in}}%
\pgfpathlineto{\pgfqpoint{3.469327in}{0.760635in}}%
\pgfpathlineto{\pgfqpoint{3.469891in}{0.760634in}}%
\pgfpathlineto{\pgfqpoint{3.470456in}{0.760632in}}%
\pgfpathlineto{\pgfqpoint{3.471020in}{0.760631in}}%
\pgfpathlineto{\pgfqpoint{3.471584in}{0.760630in}}%
\pgfpathlineto{\pgfqpoint{3.472149in}{0.760628in}}%
\pgfpathlineto{\pgfqpoint{3.472713in}{0.760627in}}%
\pgfpathlineto{\pgfqpoint{3.473277in}{0.760625in}}%
\pgfpathlineto{\pgfqpoint{3.473842in}{0.760624in}}%
\pgfpathlineto{\pgfqpoint{3.474406in}{0.760622in}}%
\pgfpathlineto{\pgfqpoint{3.474970in}{0.760621in}}%
\pgfpathlineto{\pgfqpoint{3.475535in}{0.760620in}}%
\pgfpathlineto{\pgfqpoint{3.476099in}{0.760618in}}%
\pgfpathlineto{\pgfqpoint{3.476664in}{0.760617in}}%
\pgfpathlineto{\pgfqpoint{3.477228in}{0.760615in}}%
\pgfpathlineto{\pgfqpoint{3.477792in}{0.760614in}}%
\pgfpathlineto{\pgfqpoint{3.478357in}{0.760613in}}%
\pgfpathlineto{\pgfqpoint{3.478921in}{0.760611in}}%
\pgfpathlineto{\pgfqpoint{3.479485in}{0.760610in}}%
\pgfpathlineto{\pgfqpoint{3.480050in}{0.760608in}}%
\pgfpathlineto{\pgfqpoint{3.480614in}{0.760607in}}%
\pgfpathlineto{\pgfqpoint{3.481178in}{0.760606in}}%
\pgfpathlineto{\pgfqpoint{3.481743in}{0.760604in}}%
\pgfpathlineto{\pgfqpoint{3.482307in}{0.760603in}}%
\pgfpathlineto{\pgfqpoint{3.482871in}{0.760601in}}%
\pgfpathlineto{\pgfqpoint{3.483436in}{0.760600in}}%
\pgfpathlineto{\pgfqpoint{3.484000in}{0.760599in}}%
\pgfpathlineto{\pgfqpoint{3.484564in}{0.760597in}}%
\pgfpathlineto{\pgfqpoint{3.485129in}{0.760596in}}%
\pgfpathlineto{\pgfqpoint{3.485693in}{0.760594in}}%
\pgfpathlineto{\pgfqpoint{3.486257in}{0.760593in}}%
\pgfpathlineto{\pgfqpoint{3.486822in}{0.760592in}}%
\pgfpathlineto{\pgfqpoint{3.487386in}{0.760590in}}%
\pgfpathlineto{\pgfqpoint{3.487950in}{0.760589in}}%
\pgfpathlineto{\pgfqpoint{3.488515in}{0.760587in}}%
\pgfpathlineto{\pgfqpoint{3.489079in}{0.760586in}}%
\pgfpathlineto{\pgfqpoint{3.489643in}{0.760585in}}%
\pgfpathlineto{\pgfqpoint{3.490208in}{0.760583in}}%
\pgfpathlineto{\pgfqpoint{3.490772in}{0.760582in}}%
\pgfpathlineto{\pgfqpoint{3.491337in}{0.760580in}}%
\pgfpathlineto{\pgfqpoint{3.491901in}{0.760579in}}%
\pgfpathlineto{\pgfqpoint{3.492465in}{0.760577in}}%
\pgfpathlineto{\pgfqpoint{3.493030in}{0.760576in}}%
\pgfpathlineto{\pgfqpoint{3.493594in}{0.760575in}}%
\pgfpathlineto{\pgfqpoint{3.494158in}{0.760573in}}%
\pgfpathlineto{\pgfqpoint{3.494723in}{0.760572in}}%
\pgfpathlineto{\pgfqpoint{3.495287in}{0.760570in}}%
\pgfpathlineto{\pgfqpoint{3.495851in}{0.760569in}}%
\pgfpathlineto{\pgfqpoint{3.496416in}{0.760568in}}%
\pgfpathlineto{\pgfqpoint{3.496980in}{0.760566in}}%
\pgfpathlineto{\pgfqpoint{3.497544in}{0.760565in}}%
\pgfpathlineto{\pgfqpoint{3.498109in}{0.760563in}}%
\pgfpathlineto{\pgfqpoint{3.498673in}{0.760561in}}%
\pgfpathlineto{\pgfqpoint{3.499237in}{0.760559in}}%
\pgfpathlineto{\pgfqpoint{3.499802in}{0.760556in}}%
\pgfpathlineto{\pgfqpoint{3.500366in}{0.760554in}}%
\pgfpathlineto{\pgfqpoint{3.500930in}{0.760551in}}%
\pgfpathlineto{\pgfqpoint{3.501495in}{0.760548in}}%
\pgfpathlineto{\pgfqpoint{3.502059in}{0.760546in}}%
\pgfpathlineto{\pgfqpoint{3.502623in}{0.760543in}}%
\pgfpathlineto{\pgfqpoint{3.503188in}{0.760541in}}%
\pgfpathlineto{\pgfqpoint{3.503752in}{0.760538in}}%
\pgfpathlineto{\pgfqpoint{3.504316in}{0.760535in}}%
\pgfpathlineto{\pgfqpoint{3.504881in}{0.760533in}}%
\pgfpathlineto{\pgfqpoint{3.505445in}{0.760530in}}%
\pgfpathlineto{\pgfqpoint{3.506009in}{0.760528in}}%
\pgfpathlineto{\pgfqpoint{3.506574in}{0.760525in}}%
\pgfpathlineto{\pgfqpoint{3.507138in}{0.760523in}}%
\pgfpathlineto{\pgfqpoint{3.507703in}{0.760520in}}%
\pgfpathlineto{\pgfqpoint{3.508267in}{0.760517in}}%
\pgfpathlineto{\pgfqpoint{3.508831in}{0.760515in}}%
\pgfpathlineto{\pgfqpoint{3.509396in}{0.760512in}}%
\pgfpathlineto{\pgfqpoint{3.509960in}{0.760510in}}%
\pgfpathlineto{\pgfqpoint{3.510524in}{0.760507in}}%
\pgfpathlineto{\pgfqpoint{3.511089in}{0.760504in}}%
\pgfpathlineto{\pgfqpoint{3.511653in}{0.760502in}}%
\pgfpathlineto{\pgfqpoint{3.512217in}{0.760499in}}%
\pgfpathlineto{\pgfqpoint{3.512782in}{0.760497in}}%
\pgfpathlineto{\pgfqpoint{3.513346in}{0.760494in}}%
\pgfpathlineto{\pgfqpoint{3.513910in}{0.760491in}}%
\pgfpathlineto{\pgfqpoint{3.514475in}{0.760489in}}%
\pgfpathlineto{\pgfqpoint{3.515039in}{0.760486in}}%
\pgfpathlineto{\pgfqpoint{3.515603in}{0.760484in}}%
\pgfpathlineto{\pgfqpoint{3.516168in}{0.760481in}}%
\pgfpathlineto{\pgfqpoint{3.516732in}{0.760479in}}%
\pgfpathlineto{\pgfqpoint{3.517296in}{0.760476in}}%
\pgfpathlineto{\pgfqpoint{3.517861in}{0.760473in}}%
\pgfpathlineto{\pgfqpoint{3.518425in}{0.760471in}}%
\pgfpathlineto{\pgfqpoint{3.518989in}{0.760468in}}%
\pgfpathlineto{\pgfqpoint{3.519554in}{0.760466in}}%
\pgfpathlineto{\pgfqpoint{3.520118in}{0.760463in}}%
\pgfpathlineto{\pgfqpoint{3.520682in}{0.760460in}}%
\pgfpathlineto{\pgfqpoint{3.521247in}{0.760458in}}%
\pgfpathlineto{\pgfqpoint{3.521811in}{0.760455in}}%
\pgfpathlineto{\pgfqpoint{3.522376in}{0.760453in}}%
\pgfpathlineto{\pgfqpoint{3.522940in}{0.760450in}}%
\pgfpathlineto{\pgfqpoint{3.523504in}{0.760448in}}%
\pgfpathlineto{\pgfqpoint{3.524069in}{0.760445in}}%
\pgfpathlineto{\pgfqpoint{3.524633in}{0.760442in}}%
\pgfpathlineto{\pgfqpoint{3.525197in}{0.760440in}}%
\pgfpathlineto{\pgfqpoint{3.525762in}{0.760437in}}%
\pgfpathlineto{\pgfqpoint{3.526326in}{0.760435in}}%
\pgfpathlineto{\pgfqpoint{3.526890in}{0.760432in}}%
\pgfpathlineto{\pgfqpoint{3.527455in}{0.760429in}}%
\pgfpathlineto{\pgfqpoint{3.528019in}{0.760427in}}%
\pgfpathlineto{\pgfqpoint{3.528583in}{0.760424in}}%
\pgfpathlineto{\pgfqpoint{3.529148in}{0.760422in}}%
\pgfpathlineto{\pgfqpoint{3.529712in}{0.760419in}}%
\pgfpathlineto{\pgfqpoint{3.530276in}{0.760417in}}%
\pgfpathlineto{\pgfqpoint{3.530841in}{0.760414in}}%
\pgfpathlineto{\pgfqpoint{3.531405in}{0.760411in}}%
\pgfpathlineto{\pgfqpoint{3.531969in}{0.760409in}}%
\pgfpathlineto{\pgfqpoint{3.532534in}{0.760406in}}%
\pgfpathlineto{\pgfqpoint{3.533098in}{0.760404in}}%
\pgfpathlineto{\pgfqpoint{3.533662in}{0.760401in}}%
\pgfpathlineto{\pgfqpoint{3.534227in}{0.760398in}}%
\pgfpathlineto{\pgfqpoint{3.534791in}{0.760396in}}%
\pgfpathlineto{\pgfqpoint{3.535355in}{0.760393in}}%
\pgfpathlineto{\pgfqpoint{3.535920in}{0.760391in}}%
\pgfpathlineto{\pgfqpoint{3.536484in}{0.760388in}}%
\pgfpathlineto{\pgfqpoint{3.537049in}{0.760385in}}%
\pgfpathlineto{\pgfqpoint{3.537613in}{0.760383in}}%
\pgfpathlineto{\pgfqpoint{3.538177in}{0.760380in}}%
\pgfpathlineto{\pgfqpoint{3.538742in}{0.760378in}}%
\pgfpathlineto{\pgfqpoint{3.539306in}{0.760375in}}%
\pgfpathlineto{\pgfqpoint{3.539870in}{0.760373in}}%
\pgfpathlineto{\pgfqpoint{3.540435in}{0.760370in}}%
\pgfpathlineto{\pgfqpoint{3.540999in}{0.760367in}}%
\pgfpathlineto{\pgfqpoint{3.541563in}{0.760365in}}%
\pgfpathlineto{\pgfqpoint{3.542128in}{0.760362in}}%
\pgfpathlineto{\pgfqpoint{3.542692in}{0.760360in}}%
\pgfpathlineto{\pgfqpoint{3.543256in}{0.760357in}}%
\pgfpathlineto{\pgfqpoint{3.543821in}{0.760354in}}%
\pgfpathlineto{\pgfqpoint{3.544385in}{0.760352in}}%
\pgfpathlineto{\pgfqpoint{3.544949in}{0.760349in}}%
\pgfpathlineto{\pgfqpoint{3.545514in}{0.760347in}}%
\pgfpathlineto{\pgfqpoint{3.546078in}{0.760344in}}%
\pgfpathlineto{\pgfqpoint{3.546642in}{0.760342in}}%
\pgfpathlineto{\pgfqpoint{3.547207in}{0.760339in}}%
\pgfpathlineto{\pgfqpoint{3.547771in}{0.760336in}}%
\pgfpathlineto{\pgfqpoint{3.548335in}{0.760334in}}%
\pgfpathlineto{\pgfqpoint{3.548900in}{0.760331in}}%
\pgfpathlineto{\pgfqpoint{3.549464in}{0.760329in}}%
\pgfpathlineto{\pgfqpoint{3.550028in}{0.760326in}}%
\pgfpathlineto{\pgfqpoint{3.550593in}{0.760323in}}%
\pgfpathlineto{\pgfqpoint{3.551157in}{0.760321in}}%
\pgfpathlineto{\pgfqpoint{3.551721in}{0.760318in}}%
\pgfpathlineto{\pgfqpoint{3.552286in}{0.760316in}}%
\pgfpathlineto{\pgfqpoint{3.552850in}{0.760313in}}%
\pgfpathlineto{\pgfqpoint{3.553415in}{0.760311in}}%
\pgfpathlineto{\pgfqpoint{3.553979in}{0.760308in}}%
\pgfpathlineto{\pgfqpoint{3.554543in}{0.760305in}}%
\pgfpathlineto{\pgfqpoint{3.555108in}{0.760303in}}%
\pgfpathlineto{\pgfqpoint{3.555672in}{0.760300in}}%
\pgfpathlineto{\pgfqpoint{3.556236in}{0.760298in}}%
\pgfpathlineto{\pgfqpoint{3.556801in}{0.760295in}}%
\pgfpathlineto{\pgfqpoint{3.557365in}{0.760292in}}%
\pgfpathlineto{\pgfqpoint{3.557929in}{0.760290in}}%
\pgfpathlineto{\pgfqpoint{3.558494in}{0.760287in}}%
\pgfpathlineto{\pgfqpoint{3.559058in}{0.760285in}}%
\pgfpathlineto{\pgfqpoint{3.559622in}{0.760282in}}%
\pgfpathlineto{\pgfqpoint{3.560187in}{0.760280in}}%
\pgfpathlineto{\pgfqpoint{3.560751in}{0.760277in}}%
\pgfpathlineto{\pgfqpoint{3.561315in}{0.760274in}}%
\pgfpathlineto{\pgfqpoint{3.561880in}{0.760272in}}%
\pgfpathlineto{\pgfqpoint{3.562444in}{0.760269in}}%
\pgfpathlineto{\pgfqpoint{3.563008in}{0.760267in}}%
\pgfpathlineto{\pgfqpoint{3.563573in}{0.760264in}}%
\pgfpathlineto{\pgfqpoint{3.564137in}{0.760261in}}%
\pgfpathlineto{\pgfqpoint{3.564701in}{0.760259in}}%
\pgfpathlineto{\pgfqpoint{3.565266in}{0.760256in}}%
\pgfpathlineto{\pgfqpoint{3.565830in}{0.760254in}}%
\pgfpathlineto{\pgfqpoint{3.566394in}{0.760251in}}%
\pgfpathlineto{\pgfqpoint{3.566959in}{0.760248in}}%
\pgfpathlineto{\pgfqpoint{3.567523in}{0.760246in}}%
\pgfpathlineto{\pgfqpoint{3.568088in}{0.760243in}}%
\pgfpathlineto{\pgfqpoint{3.568652in}{0.760241in}}%
\pgfpathlineto{\pgfqpoint{3.569216in}{0.760238in}}%
\pgfpathlineto{\pgfqpoint{3.569781in}{0.760236in}}%
\pgfpathlineto{\pgfqpoint{3.570345in}{0.760233in}}%
\pgfpathlineto{\pgfqpoint{3.570909in}{0.760230in}}%
\pgfpathlineto{\pgfqpoint{3.571474in}{0.760228in}}%
\pgfpathlineto{\pgfqpoint{3.572038in}{0.760225in}}%
\pgfpathlineto{\pgfqpoint{3.572602in}{0.760223in}}%
\pgfpathlineto{\pgfqpoint{3.573167in}{0.760220in}}%
\pgfpathlineto{\pgfqpoint{3.573731in}{0.760217in}}%
\pgfpathlineto{\pgfqpoint{3.574295in}{0.760215in}}%
\pgfpathlineto{\pgfqpoint{3.574860in}{0.760212in}}%
\pgfpathlineto{\pgfqpoint{3.575424in}{0.760210in}}%
\pgfpathlineto{\pgfqpoint{3.575988in}{0.760207in}}%
\pgfpathlineto{\pgfqpoint{3.576553in}{0.760205in}}%
\pgfpathlineto{\pgfqpoint{3.577117in}{0.760202in}}%
\pgfpathlineto{\pgfqpoint{3.577681in}{0.760199in}}%
\pgfpathlineto{\pgfqpoint{3.578246in}{0.760197in}}%
\pgfpathlineto{\pgfqpoint{3.578810in}{0.760194in}}%
\pgfpathlineto{\pgfqpoint{3.579374in}{0.760192in}}%
\pgfpathlineto{\pgfqpoint{3.579939in}{0.760189in}}%
\pgfpathlineto{\pgfqpoint{3.580503in}{0.760186in}}%
\pgfpathlineto{\pgfqpoint{3.581067in}{0.760184in}}%
\pgfpathlineto{\pgfqpoint{3.581632in}{0.760181in}}%
\pgfpathlineto{\pgfqpoint{3.582196in}{0.760179in}}%
\pgfpathlineto{\pgfqpoint{3.582761in}{0.760176in}}%
\pgfpathlineto{\pgfqpoint{3.583325in}{0.760174in}}%
\pgfpathlineto{\pgfqpoint{3.583889in}{0.760171in}}%
\pgfpathlineto{\pgfqpoint{3.584454in}{0.760168in}}%
\pgfpathlineto{\pgfqpoint{3.585018in}{0.760166in}}%
\pgfpathlineto{\pgfqpoint{3.585582in}{0.760163in}}%
\pgfpathlineto{\pgfqpoint{3.586147in}{0.760161in}}%
\pgfpathlineto{\pgfqpoint{3.586711in}{0.760158in}}%
\pgfpathlineto{\pgfqpoint{3.587275in}{0.760155in}}%
\pgfpathlineto{\pgfqpoint{3.587840in}{0.760153in}}%
\pgfpathlineto{\pgfqpoint{3.588404in}{0.760150in}}%
\pgfpathlineto{\pgfqpoint{3.588968in}{0.760148in}}%
\pgfpathlineto{\pgfqpoint{3.589533in}{0.760145in}}%
\pgfpathlineto{\pgfqpoint{3.590097in}{0.760143in}}%
\pgfpathlineto{\pgfqpoint{3.590661in}{0.760140in}}%
\pgfpathlineto{\pgfqpoint{3.591226in}{0.760137in}}%
\pgfpathlineto{\pgfqpoint{3.591790in}{0.760135in}}%
\pgfpathlineto{\pgfqpoint{3.592354in}{0.760132in}}%
\pgfpathlineto{\pgfqpoint{3.592919in}{0.760130in}}%
\pgfpathlineto{\pgfqpoint{3.593483in}{0.760127in}}%
\pgfpathlineto{\pgfqpoint{3.594047in}{0.760124in}}%
\pgfpathlineto{\pgfqpoint{3.594612in}{0.760122in}}%
\pgfpathlineto{\pgfqpoint{3.595176in}{0.760119in}}%
\pgfpathlineto{\pgfqpoint{3.595740in}{0.760117in}}%
\pgfpathlineto{\pgfqpoint{3.596305in}{0.760114in}}%
\pgfpathlineto{\pgfqpoint{3.596869in}{0.760111in}}%
\pgfpathlineto{\pgfqpoint{3.597434in}{0.760109in}}%
\pgfpathlineto{\pgfqpoint{3.597998in}{0.760106in}}%
\pgfpathlineto{\pgfqpoint{3.598562in}{0.760104in}}%
\pgfpathlineto{\pgfqpoint{3.599127in}{0.760101in}}%
\pgfpathlineto{\pgfqpoint{3.599691in}{0.760099in}}%
\pgfpathlineto{\pgfqpoint{3.600255in}{0.760096in}}%
\pgfpathlineto{\pgfqpoint{3.600820in}{0.760093in}}%
\pgfpathlineto{\pgfqpoint{3.601384in}{0.760091in}}%
\pgfpathlineto{\pgfqpoint{3.601948in}{0.760088in}}%
\pgfpathlineto{\pgfqpoint{3.602513in}{0.760086in}}%
\pgfpathlineto{\pgfqpoint{3.603077in}{0.760083in}}%
\pgfpathlineto{\pgfqpoint{3.603641in}{0.760080in}}%
\pgfpathlineto{\pgfqpoint{3.604206in}{0.760078in}}%
\pgfpathlineto{\pgfqpoint{3.604770in}{0.760075in}}%
\pgfpathlineto{\pgfqpoint{3.605334in}{0.760073in}}%
\pgfpathlineto{\pgfqpoint{3.605899in}{0.760070in}}%
\pgfpathlineto{\pgfqpoint{3.606463in}{0.760068in}}%
\pgfpathlineto{\pgfqpoint{3.607027in}{0.760065in}}%
\pgfpathlineto{\pgfqpoint{3.607592in}{0.760062in}}%
\pgfpathlineto{\pgfqpoint{3.608156in}{0.760060in}}%
\pgfpathlineto{\pgfqpoint{3.608720in}{0.760057in}}%
\pgfpathlineto{\pgfqpoint{3.609285in}{0.760055in}}%
\pgfpathlineto{\pgfqpoint{3.609849in}{0.760052in}}%
\pgfpathlineto{\pgfqpoint{3.610413in}{0.760049in}}%
\pgfpathlineto{\pgfqpoint{3.610978in}{0.760047in}}%
\pgfpathlineto{\pgfqpoint{3.611542in}{0.760044in}}%
\pgfpathlineto{\pgfqpoint{3.612106in}{0.760042in}}%
\pgfpathlineto{\pgfqpoint{3.612671in}{0.760039in}}%
\pgfpathlineto{\pgfqpoint{3.613235in}{0.760037in}}%
\pgfpathlineto{\pgfqpoint{3.613800in}{0.760034in}}%
\pgfpathlineto{\pgfqpoint{3.614364in}{0.760031in}}%
\pgfpathlineto{\pgfqpoint{3.614928in}{0.760029in}}%
\pgfpathlineto{\pgfqpoint{3.615493in}{0.760026in}}%
\pgfpathlineto{\pgfqpoint{3.616057in}{0.760024in}}%
\pgfpathlineto{\pgfqpoint{3.616621in}{0.760021in}}%
\pgfpathlineto{\pgfqpoint{3.617186in}{0.760018in}}%
\pgfpathlineto{\pgfqpoint{3.617750in}{0.760016in}}%
\pgfpathlineto{\pgfqpoint{3.618314in}{0.760013in}}%
\pgfpathlineto{\pgfqpoint{3.618879in}{0.760011in}}%
\pgfpathlineto{\pgfqpoint{3.619443in}{0.760008in}}%
\pgfpathlineto{\pgfqpoint{3.620007in}{0.760006in}}%
\pgfpathlineto{\pgfqpoint{3.620572in}{0.760003in}}%
\pgfpathlineto{\pgfqpoint{3.621136in}{0.760000in}}%
\pgfpathlineto{\pgfqpoint{3.621700in}{0.759998in}}%
\pgfpathlineto{\pgfqpoint{3.622265in}{0.759995in}}%
\pgfpathlineto{\pgfqpoint{3.622829in}{0.759993in}}%
\pgfpathlineto{\pgfqpoint{3.623393in}{0.759990in}}%
\pgfpathlineto{\pgfqpoint{3.623958in}{0.759987in}}%
\pgfpathlineto{\pgfqpoint{3.624522in}{0.759985in}}%
\pgfpathlineto{\pgfqpoint{3.625086in}{0.759982in}}%
\pgfpathlineto{\pgfqpoint{3.625651in}{0.759980in}}%
\pgfpathlineto{\pgfqpoint{3.626215in}{0.759977in}}%
\pgfpathlineto{\pgfqpoint{3.626779in}{0.759974in}}%
\pgfpathlineto{\pgfqpoint{3.627344in}{0.759972in}}%
\pgfpathlineto{\pgfqpoint{3.627908in}{0.759969in}}%
\pgfpathlineto{\pgfqpoint{3.628473in}{0.759967in}}%
\pgfpathlineto{\pgfqpoint{3.629037in}{0.759964in}}%
\pgfpathlineto{\pgfqpoint{3.629601in}{0.759962in}}%
\pgfpathlineto{\pgfqpoint{3.630166in}{0.759959in}}%
\pgfpathlineto{\pgfqpoint{3.630730in}{0.759956in}}%
\pgfpathlineto{\pgfqpoint{3.631294in}{0.759954in}}%
\pgfpathlineto{\pgfqpoint{3.631859in}{0.759951in}}%
\pgfpathlineto{\pgfqpoint{3.632423in}{0.759949in}}%
\pgfpathlineto{\pgfqpoint{3.632987in}{0.759946in}}%
\pgfpathlineto{\pgfqpoint{3.633552in}{0.759943in}}%
\pgfpathlineto{\pgfqpoint{3.634116in}{0.759941in}}%
\pgfpathlineto{\pgfqpoint{3.634680in}{0.759938in}}%
\pgfpathlineto{\pgfqpoint{3.635245in}{0.759936in}}%
\pgfpathlineto{\pgfqpoint{3.635809in}{0.759933in}}%
\pgfpathlineto{\pgfqpoint{3.636373in}{0.759931in}}%
\pgfpathlineto{\pgfqpoint{3.636938in}{0.759928in}}%
\pgfpathlineto{\pgfqpoint{3.637502in}{0.759925in}}%
\pgfpathlineto{\pgfqpoint{3.638066in}{0.759923in}}%
\pgfpathlineto{\pgfqpoint{3.638631in}{0.759920in}}%
\pgfpathlineto{\pgfqpoint{3.639195in}{0.759918in}}%
\pgfpathlineto{\pgfqpoint{3.639759in}{0.759915in}}%
\pgfpathlineto{\pgfqpoint{3.640324in}{0.759912in}}%
\pgfpathlineto{\pgfqpoint{3.640888in}{0.759910in}}%
\pgfpathlineto{\pgfqpoint{3.641452in}{0.759907in}}%
\pgfpathlineto{\pgfqpoint{3.642017in}{0.759905in}}%
\pgfpathlineto{\pgfqpoint{3.642581in}{0.759902in}}%
\pgfpathlineto{\pgfqpoint{3.643146in}{0.759900in}}%
\pgfpathlineto{\pgfqpoint{3.643710in}{0.759897in}}%
\pgfpathlineto{\pgfqpoint{3.644274in}{0.759894in}}%
\pgfpathlineto{\pgfqpoint{3.644839in}{0.759892in}}%
\pgfpathlineto{\pgfqpoint{3.645403in}{0.759889in}}%
\pgfpathlineto{\pgfqpoint{3.645967in}{0.759887in}}%
\pgfpathlineto{\pgfqpoint{3.646532in}{0.759884in}}%
\pgfpathlineto{\pgfqpoint{3.647096in}{0.759881in}}%
\pgfpathlineto{\pgfqpoint{3.647660in}{0.759879in}}%
\pgfpathlineto{\pgfqpoint{3.648225in}{0.759876in}}%
\pgfpathlineto{\pgfqpoint{3.648789in}{0.759874in}}%
\pgfpathlineto{\pgfqpoint{3.649353in}{0.759871in}}%
\pgfpathlineto{\pgfqpoint{3.649918in}{0.759868in}}%
\pgfpathlineto{\pgfqpoint{3.650482in}{0.759866in}}%
\pgfpathlineto{\pgfqpoint{3.651046in}{0.759863in}}%
\pgfpathlineto{\pgfqpoint{3.651611in}{0.759861in}}%
\pgfpathlineto{\pgfqpoint{3.652175in}{0.759858in}}%
\pgfpathlineto{\pgfqpoint{3.652739in}{0.759856in}}%
\pgfpathlineto{\pgfqpoint{3.653304in}{0.759853in}}%
\pgfpathlineto{\pgfqpoint{3.653868in}{0.759850in}}%
\pgfpathlineto{\pgfqpoint{3.654432in}{0.759848in}}%
\pgfpathlineto{\pgfqpoint{3.654997in}{0.759845in}}%
\pgfpathlineto{\pgfqpoint{3.655561in}{0.759843in}}%
\pgfpathlineto{\pgfqpoint{3.656125in}{0.759840in}}%
\pgfpathlineto{\pgfqpoint{3.656690in}{0.759837in}}%
\pgfpathlineto{\pgfqpoint{3.657254in}{0.759835in}}%
\pgfpathlineto{\pgfqpoint{3.657818in}{0.759832in}}%
\pgfpathlineto{\pgfqpoint{3.658383in}{0.759830in}}%
\pgfpathlineto{\pgfqpoint{3.658947in}{0.759827in}}%
\pgfpathlineto{\pgfqpoint{3.659512in}{0.759825in}}%
\pgfpathlineto{\pgfqpoint{3.660076in}{0.759822in}}%
\pgfpathlineto{\pgfqpoint{3.660640in}{0.759819in}}%
\pgfpathlineto{\pgfqpoint{3.661205in}{0.759817in}}%
\pgfpathlineto{\pgfqpoint{3.661769in}{0.759814in}}%
\pgfpathlineto{\pgfqpoint{3.662333in}{0.759812in}}%
\pgfpathlineto{\pgfqpoint{3.662898in}{0.759809in}}%
\pgfpathlineto{\pgfqpoint{3.663462in}{0.759806in}}%
\pgfpathlineto{\pgfqpoint{3.664026in}{0.759804in}}%
\pgfpathlineto{\pgfqpoint{3.664591in}{0.759801in}}%
\pgfpathlineto{\pgfqpoint{3.665155in}{0.759799in}}%
\pgfpathlineto{\pgfqpoint{3.665719in}{0.759796in}}%
\pgfpathlineto{\pgfqpoint{3.666284in}{0.759794in}}%
\pgfpathlineto{\pgfqpoint{3.666848in}{0.759791in}}%
\pgfpathlineto{\pgfqpoint{3.667412in}{0.759788in}}%
\pgfpathlineto{\pgfqpoint{3.667977in}{0.759786in}}%
\pgfpathlineto{\pgfqpoint{3.668541in}{0.759783in}}%
\pgfpathlineto{\pgfqpoint{3.669105in}{0.759781in}}%
\pgfpathlineto{\pgfqpoint{3.669670in}{0.759778in}}%
\pgfpathlineto{\pgfqpoint{3.670234in}{0.759775in}}%
\pgfpathlineto{\pgfqpoint{3.670798in}{0.759773in}}%
\pgfpathlineto{\pgfqpoint{3.671363in}{0.759770in}}%
\pgfpathlineto{\pgfqpoint{3.671927in}{0.759768in}}%
\pgfpathlineto{\pgfqpoint{3.672491in}{0.759765in}}%
\pgfpathlineto{\pgfqpoint{3.673056in}{0.759763in}}%
\pgfpathlineto{\pgfqpoint{3.673620in}{0.759760in}}%
\pgfpathlineto{\pgfqpoint{3.674185in}{0.759757in}}%
\pgfpathlineto{\pgfqpoint{3.674749in}{0.759755in}}%
\pgfpathlineto{\pgfqpoint{3.675313in}{0.759752in}}%
\pgfpathlineto{\pgfqpoint{3.675878in}{0.759750in}}%
\pgfpathlineto{\pgfqpoint{3.676442in}{0.759747in}}%
\pgfpathlineto{\pgfqpoint{3.677006in}{0.759744in}}%
\pgfpathlineto{\pgfqpoint{3.677571in}{0.759742in}}%
\pgfpathlineto{\pgfqpoint{3.678135in}{0.759739in}}%
\pgfpathlineto{\pgfqpoint{3.678699in}{0.759737in}}%
\pgfpathlineto{\pgfqpoint{3.679264in}{0.759734in}}%
\pgfpathlineto{\pgfqpoint{3.679828in}{0.759731in}}%
\pgfpathlineto{\pgfqpoint{3.680392in}{0.759729in}}%
\pgfpathlineto{\pgfqpoint{3.680957in}{0.759726in}}%
\pgfpathlineto{\pgfqpoint{3.681521in}{0.759724in}}%
\pgfpathlineto{\pgfqpoint{3.682085in}{0.759721in}}%
\pgfpathlineto{\pgfqpoint{3.682650in}{0.759719in}}%
\pgfpathlineto{\pgfqpoint{3.683214in}{0.759716in}}%
\pgfpathlineto{\pgfqpoint{3.683778in}{0.759713in}}%
\pgfpathlineto{\pgfqpoint{3.684343in}{0.759711in}}%
\pgfpathlineto{\pgfqpoint{3.684907in}{0.759708in}}%
\pgfpathlineto{\pgfqpoint{3.685471in}{0.759706in}}%
\pgfpathlineto{\pgfqpoint{3.686036in}{0.759703in}}%
\pgfpathlineto{\pgfqpoint{3.686600in}{0.759700in}}%
\pgfpathlineto{\pgfqpoint{3.687164in}{0.759698in}}%
\pgfpathlineto{\pgfqpoint{3.687729in}{0.759695in}}%
\pgfpathlineto{\pgfqpoint{3.688293in}{0.759693in}}%
\pgfpathlineto{\pgfqpoint{3.688858in}{0.759690in}}%
\pgfpathlineto{\pgfqpoint{3.689422in}{0.759688in}}%
\pgfpathlineto{\pgfqpoint{3.689986in}{0.759685in}}%
\pgfpathlineto{\pgfqpoint{3.690551in}{0.759682in}}%
\pgfpathlineto{\pgfqpoint{3.691115in}{0.759680in}}%
\pgfpathlineto{\pgfqpoint{3.691679in}{0.759677in}}%
\pgfpathlineto{\pgfqpoint{3.692244in}{0.759675in}}%
\pgfpathlineto{\pgfqpoint{3.692808in}{0.759672in}}%
\pgfpathlineto{\pgfqpoint{3.693372in}{0.759669in}}%
\pgfpathlineto{\pgfqpoint{3.693937in}{0.759667in}}%
\pgfpathlineto{\pgfqpoint{3.694501in}{0.759664in}}%
\pgfpathlineto{\pgfqpoint{3.695065in}{0.759662in}}%
\pgfpathlineto{\pgfqpoint{3.695630in}{0.759659in}}%
\pgfpathlineto{\pgfqpoint{3.696194in}{0.759657in}}%
\pgfpathlineto{\pgfqpoint{3.696758in}{0.759654in}}%
\pgfpathlineto{\pgfqpoint{3.697323in}{0.759651in}}%
\pgfpathlineto{\pgfqpoint{3.697887in}{0.759649in}}%
\pgfpathlineto{\pgfqpoint{3.698451in}{0.759646in}}%
\pgfpathlineto{\pgfqpoint{3.699016in}{0.759644in}}%
\pgfpathlineto{\pgfqpoint{3.699580in}{0.759641in}}%
\pgfpathlineto{\pgfqpoint{3.700144in}{0.759638in}}%
\pgfpathlineto{\pgfqpoint{3.700709in}{0.759636in}}%
\pgfpathlineto{\pgfqpoint{3.701273in}{0.759633in}}%
\pgfpathlineto{\pgfqpoint{3.701837in}{0.759631in}}%
\pgfpathlineto{\pgfqpoint{3.702402in}{0.759628in}}%
\pgfpathlineto{\pgfqpoint{3.702966in}{0.759626in}}%
\pgfpathlineto{\pgfqpoint{3.703530in}{0.759623in}}%
\pgfpathlineto{\pgfqpoint{3.704095in}{0.759620in}}%
\pgfpathlineto{\pgfqpoint{3.704659in}{0.759618in}}%
\pgfpathlineto{\pgfqpoint{3.705224in}{0.759615in}}%
\pgfpathlineto{\pgfqpoint{3.705788in}{0.759613in}}%
\pgfpathlineto{\pgfqpoint{3.706352in}{0.759610in}}%
\pgfpathlineto{\pgfqpoint{3.706917in}{0.759607in}}%
\pgfpathlineto{\pgfqpoint{3.707481in}{0.759605in}}%
\pgfpathlineto{\pgfqpoint{3.708045in}{0.759602in}}%
\pgfpathlineto{\pgfqpoint{3.708610in}{0.759600in}}%
\pgfpathlineto{\pgfqpoint{3.709174in}{0.759597in}}%
\pgfpathlineto{\pgfqpoint{3.709738in}{0.759594in}}%
\pgfpathlineto{\pgfqpoint{3.710303in}{0.759592in}}%
\pgfpathlineto{\pgfqpoint{3.710867in}{0.759589in}}%
\pgfpathlineto{\pgfqpoint{3.711431in}{0.759587in}}%
\pgfpathlineto{\pgfqpoint{3.711996in}{0.759584in}}%
\pgfpathlineto{\pgfqpoint{3.712560in}{0.759582in}}%
\pgfpathlineto{\pgfqpoint{3.713124in}{0.759579in}}%
\pgfpathlineto{\pgfqpoint{3.713689in}{0.759576in}}%
\pgfpathlineto{\pgfqpoint{3.714253in}{0.759574in}}%
\pgfpathlineto{\pgfqpoint{3.714817in}{0.759571in}}%
\pgfpathlineto{\pgfqpoint{3.715382in}{0.759569in}}%
\pgfpathlineto{\pgfqpoint{3.715946in}{0.759566in}}%
\pgfpathlineto{\pgfqpoint{3.716510in}{0.759563in}}%
\pgfpathlineto{\pgfqpoint{3.717075in}{0.759561in}}%
\pgfpathlineto{\pgfqpoint{3.717639in}{0.759558in}}%
\pgfpathlineto{\pgfqpoint{3.718203in}{0.759556in}}%
\pgfpathlineto{\pgfqpoint{3.718768in}{0.759553in}}%
\pgfpathlineto{\pgfqpoint{3.719332in}{0.759551in}}%
\pgfpathlineto{\pgfqpoint{3.719897in}{0.759548in}}%
\pgfpathlineto{\pgfqpoint{3.720461in}{0.759545in}}%
\pgfpathlineto{\pgfqpoint{3.721025in}{0.759543in}}%
\pgfpathlineto{\pgfqpoint{3.721590in}{0.759540in}}%
\pgfpathlineto{\pgfqpoint{3.722154in}{0.759538in}}%
\pgfpathlineto{\pgfqpoint{3.722718in}{0.759535in}}%
\pgfpathlineto{\pgfqpoint{3.723283in}{0.759532in}}%
\pgfpathlineto{\pgfqpoint{3.723847in}{0.759530in}}%
\pgfpathlineto{\pgfqpoint{3.724411in}{0.759527in}}%
\pgfpathlineto{\pgfqpoint{3.724976in}{0.759525in}}%
\pgfpathlineto{\pgfqpoint{3.725540in}{0.759522in}}%
\pgfpathlineto{\pgfqpoint{3.726104in}{0.759520in}}%
\pgfpathlineto{\pgfqpoint{3.726669in}{0.759517in}}%
\pgfpathlineto{\pgfqpoint{3.727233in}{0.759514in}}%
\pgfpathlineto{\pgfqpoint{3.727797in}{0.759512in}}%
\pgfpathlineto{\pgfqpoint{3.728362in}{0.759509in}}%
\pgfpathlineto{\pgfqpoint{3.728926in}{0.759507in}}%
\pgfpathlineto{\pgfqpoint{3.729490in}{0.759504in}}%
\pgfpathlineto{\pgfqpoint{3.730055in}{0.759501in}}%
\pgfpathlineto{\pgfqpoint{3.730619in}{0.759499in}}%
\pgfpathlineto{\pgfqpoint{3.731183in}{0.759496in}}%
\pgfpathlineto{\pgfqpoint{3.731748in}{0.759494in}}%
\pgfpathlineto{\pgfqpoint{3.732312in}{0.759491in}}%
\pgfpathlineto{\pgfqpoint{3.732876in}{0.759489in}}%
\pgfpathlineto{\pgfqpoint{3.733441in}{0.759486in}}%
\pgfpathlineto{\pgfqpoint{3.734005in}{0.759483in}}%
\pgfpathlineto{\pgfqpoint{3.734570in}{0.759481in}}%
\pgfpathlineto{\pgfqpoint{3.735134in}{0.759478in}}%
\pgfpathlineto{\pgfqpoint{3.735698in}{0.759476in}}%
\pgfpathlineto{\pgfqpoint{3.736263in}{0.759473in}}%
\pgfpathlineto{\pgfqpoint{3.736827in}{0.759470in}}%
\pgfpathlineto{\pgfqpoint{3.737391in}{0.759468in}}%
\pgfpathlineto{\pgfqpoint{3.737956in}{0.759465in}}%
\pgfpathlineto{\pgfqpoint{3.738520in}{0.759463in}}%
\pgfpathlineto{\pgfqpoint{3.739084in}{0.759460in}}%
\pgfpathlineto{\pgfqpoint{3.739649in}{0.759457in}}%
\pgfpathlineto{\pgfqpoint{3.740213in}{0.759455in}}%
\pgfpathlineto{\pgfqpoint{3.740777in}{0.759452in}}%
\pgfpathlineto{\pgfqpoint{3.741342in}{0.759450in}}%
\pgfpathlineto{\pgfqpoint{3.741906in}{0.759447in}}%
\pgfpathlineto{\pgfqpoint{3.742470in}{0.759445in}}%
\pgfpathlineto{\pgfqpoint{3.743035in}{0.759442in}}%
\pgfpathlineto{\pgfqpoint{3.743599in}{0.759439in}}%
\pgfpathlineto{\pgfqpoint{3.744163in}{0.759437in}}%
\pgfpathlineto{\pgfqpoint{3.744728in}{0.759434in}}%
\pgfpathlineto{\pgfqpoint{3.745292in}{0.759432in}}%
\pgfpathlineto{\pgfqpoint{3.745856in}{0.759429in}}%
\pgfpathlineto{\pgfqpoint{3.746421in}{0.759426in}}%
\pgfpathlineto{\pgfqpoint{3.746985in}{0.759424in}}%
\pgfpathlineto{\pgfqpoint{3.747549in}{0.759421in}}%
\pgfpathlineto{\pgfqpoint{3.748114in}{0.759419in}}%
\pgfpathlineto{\pgfqpoint{3.748678in}{0.759416in}}%
\pgfpathlineto{\pgfqpoint{3.749242in}{0.759414in}}%
\pgfpathlineto{\pgfqpoint{3.749807in}{0.759411in}}%
\pgfpathlineto{\pgfqpoint{3.750371in}{0.759408in}}%
\pgfpathlineto{\pgfqpoint{3.750936in}{0.759406in}}%
\pgfpathlineto{\pgfqpoint{3.751500in}{0.759403in}}%
\pgfpathlineto{\pgfqpoint{3.752064in}{0.759401in}}%
\pgfpathlineto{\pgfqpoint{3.752629in}{0.759398in}}%
\pgfpathlineto{\pgfqpoint{3.753193in}{0.759395in}}%
\pgfpathlineto{\pgfqpoint{3.753757in}{0.759393in}}%
\pgfpathlineto{\pgfqpoint{3.754322in}{0.759390in}}%
\pgfpathlineto{\pgfqpoint{3.754886in}{0.759388in}}%
\pgfpathlineto{\pgfqpoint{3.755450in}{0.759385in}}%
\pgfpathlineto{\pgfqpoint{3.756015in}{0.759383in}}%
\pgfpathlineto{\pgfqpoint{3.756579in}{0.759380in}}%
\pgfpathlineto{\pgfqpoint{3.757143in}{0.759377in}}%
\pgfpathlineto{\pgfqpoint{3.757708in}{0.759375in}}%
\pgfpathlineto{\pgfqpoint{3.758272in}{0.759372in}}%
\pgfpathlineto{\pgfqpoint{3.758836in}{0.759370in}}%
\pgfpathlineto{\pgfqpoint{3.759401in}{0.759367in}}%
\pgfpathlineto{\pgfqpoint{3.759965in}{0.759364in}}%
\pgfpathlineto{\pgfqpoint{3.760529in}{0.759362in}}%
\pgfpathlineto{\pgfqpoint{3.761094in}{0.759359in}}%
\pgfpathlineto{\pgfqpoint{3.761658in}{0.759357in}}%
\pgfpathlineto{\pgfqpoint{3.762222in}{0.759354in}}%
\pgfpathlineto{\pgfqpoint{3.762787in}{0.759352in}}%
\pgfpathlineto{\pgfqpoint{3.763351in}{0.759349in}}%
\pgfpathlineto{\pgfqpoint{3.763915in}{0.759346in}}%
\pgfpathlineto{\pgfqpoint{3.764480in}{0.759344in}}%
\pgfpathlineto{\pgfqpoint{3.765044in}{0.759341in}}%
\pgfpathlineto{\pgfqpoint{3.765609in}{0.759339in}}%
\pgfpathlineto{\pgfqpoint{3.766173in}{0.759336in}}%
\pgfpathlineto{\pgfqpoint{3.766737in}{0.759333in}}%
\pgfpathlineto{\pgfqpoint{3.767302in}{0.759331in}}%
\pgfpathlineto{\pgfqpoint{3.767866in}{0.759328in}}%
\pgfpathlineto{\pgfqpoint{3.768430in}{0.759326in}}%
\pgfpathlineto{\pgfqpoint{3.768995in}{0.759323in}}%
\pgfpathlineto{\pgfqpoint{3.769559in}{0.759320in}}%
\pgfpathlineto{\pgfqpoint{3.770123in}{0.759318in}}%
\pgfpathlineto{\pgfqpoint{3.770688in}{0.759315in}}%
\pgfpathlineto{\pgfqpoint{3.771252in}{0.759313in}}%
\pgfpathlineto{\pgfqpoint{3.771816in}{0.759310in}}%
\pgfpathlineto{\pgfqpoint{3.772381in}{0.759308in}}%
\pgfpathlineto{\pgfqpoint{3.772945in}{0.759305in}}%
\pgfpathlineto{\pgfqpoint{3.773509in}{0.759302in}}%
\pgfpathlineto{\pgfqpoint{3.774074in}{0.759300in}}%
\pgfpathlineto{\pgfqpoint{3.774638in}{0.759297in}}%
\pgfpathlineto{\pgfqpoint{3.775202in}{0.759295in}}%
\pgfpathlineto{\pgfqpoint{3.775767in}{0.759292in}}%
\pgfpathlineto{\pgfqpoint{3.776331in}{0.759289in}}%
\pgfpathlineto{\pgfqpoint{3.776895in}{0.759287in}}%
\pgfpathlineto{\pgfqpoint{3.777460in}{0.759284in}}%
\pgfpathlineto{\pgfqpoint{3.778024in}{0.759282in}}%
\pgfpathlineto{\pgfqpoint{3.778588in}{0.759279in}}%
\pgfpathlineto{\pgfqpoint{3.779153in}{0.759277in}}%
\pgfpathlineto{\pgfqpoint{3.779717in}{0.759274in}}%
\pgfpathlineto{\pgfqpoint{3.780282in}{0.759271in}}%
\pgfpathlineto{\pgfqpoint{3.780846in}{0.759269in}}%
\pgfpathlineto{\pgfqpoint{3.781410in}{0.759266in}}%
\pgfpathlineto{\pgfqpoint{3.781975in}{0.759264in}}%
\pgfpathlineto{\pgfqpoint{3.782539in}{0.759261in}}%
\pgfpathlineto{\pgfqpoint{3.783103in}{0.759258in}}%
\pgfpathlineto{\pgfqpoint{3.783668in}{0.759256in}}%
\pgfpathlineto{\pgfqpoint{3.784232in}{0.759253in}}%
\pgfpathlineto{\pgfqpoint{3.784796in}{0.759251in}}%
\pgfpathlineto{\pgfqpoint{3.785361in}{0.759248in}}%
\pgfpathlineto{\pgfqpoint{3.785925in}{0.759246in}}%
\pgfpathlineto{\pgfqpoint{3.786489in}{0.759243in}}%
\pgfpathlineto{\pgfqpoint{3.787054in}{0.759240in}}%
\pgfpathlineto{\pgfqpoint{3.787618in}{0.759238in}}%
\pgfpathlineto{\pgfqpoint{3.788182in}{0.759235in}}%
\pgfpathlineto{\pgfqpoint{3.788747in}{0.759233in}}%
\pgfpathlineto{\pgfqpoint{3.789311in}{0.759230in}}%
\pgfpathlineto{\pgfqpoint{3.789875in}{0.759227in}}%
\pgfpathlineto{\pgfqpoint{3.790440in}{0.759225in}}%
\pgfpathlineto{\pgfqpoint{3.791004in}{0.759222in}}%
\pgfpathlineto{\pgfqpoint{3.791568in}{0.759220in}}%
\pgfpathlineto{\pgfqpoint{3.792133in}{0.759217in}}%
\pgfpathlineto{\pgfqpoint{3.792697in}{0.759214in}}%
\pgfpathlineto{\pgfqpoint{3.793261in}{0.759212in}}%
\pgfpathlineto{\pgfqpoint{3.793826in}{0.759209in}}%
\pgfpathlineto{\pgfqpoint{3.794390in}{0.759207in}}%
\pgfpathlineto{\pgfqpoint{3.794954in}{0.759204in}}%
\pgfpathlineto{\pgfqpoint{3.795519in}{0.759202in}}%
\pgfpathlineto{\pgfqpoint{3.796083in}{0.759199in}}%
\pgfpathlineto{\pgfqpoint{3.796648in}{0.759196in}}%
\pgfpathlineto{\pgfqpoint{3.797212in}{0.759194in}}%
\pgfpathlineto{\pgfqpoint{3.797776in}{0.759191in}}%
\pgfpathlineto{\pgfqpoint{3.798341in}{0.759189in}}%
\pgfpathlineto{\pgfqpoint{3.798905in}{0.759186in}}%
\pgfpathlineto{\pgfqpoint{3.799469in}{0.759183in}}%
\pgfpathlineto{\pgfqpoint{3.800034in}{0.759181in}}%
\pgfpathlineto{\pgfqpoint{3.800598in}{0.759178in}}%
\pgfpathlineto{\pgfqpoint{3.801162in}{0.759176in}}%
\pgfpathlineto{\pgfqpoint{3.801727in}{0.759173in}}%
\pgfpathlineto{\pgfqpoint{3.802291in}{0.759171in}}%
\pgfpathlineto{\pgfqpoint{3.802855in}{0.759168in}}%
\pgfpathlineto{\pgfqpoint{3.803420in}{0.759165in}}%
\pgfpathlineto{\pgfqpoint{3.803984in}{0.759163in}}%
\pgfpathlineto{\pgfqpoint{3.804548in}{0.759160in}}%
\pgfpathlineto{\pgfqpoint{3.805113in}{0.759158in}}%
\pgfpathlineto{\pgfqpoint{3.805677in}{0.759155in}}%
\pgfpathlineto{\pgfqpoint{3.806241in}{0.759152in}}%
\pgfpathlineto{\pgfqpoint{3.806806in}{0.759150in}}%
\pgfpathlineto{\pgfqpoint{3.807370in}{0.759147in}}%
\pgfpathlineto{\pgfqpoint{3.807934in}{0.759145in}}%
\pgfpathlineto{\pgfqpoint{3.808499in}{0.759142in}}%
\pgfpathlineto{\pgfqpoint{3.809063in}{0.759140in}}%
\pgfpathlineto{\pgfqpoint{3.809627in}{0.759137in}}%
\pgfpathlineto{\pgfqpoint{3.810192in}{0.759134in}}%
\pgfpathlineto{\pgfqpoint{3.810756in}{0.759132in}}%
\pgfpathlineto{\pgfqpoint{3.811321in}{0.759129in}}%
\pgfpathlineto{\pgfqpoint{3.811885in}{0.759127in}}%
\pgfpathlineto{\pgfqpoint{3.812449in}{0.759124in}}%
\pgfpathlineto{\pgfqpoint{3.813014in}{0.759121in}}%
\pgfpathlineto{\pgfqpoint{3.813578in}{0.759119in}}%
\pgfpathlineto{\pgfqpoint{3.814142in}{0.759116in}}%
\pgfpathlineto{\pgfqpoint{3.814707in}{0.759114in}}%
\pgfpathlineto{\pgfqpoint{3.815271in}{0.759111in}}%
\pgfpathlineto{\pgfqpoint{3.815835in}{0.759109in}}%
\pgfpathlineto{\pgfqpoint{3.816400in}{0.759106in}}%
\pgfpathlineto{\pgfqpoint{3.816964in}{0.759103in}}%
\pgfpathlineto{\pgfqpoint{3.817528in}{0.759101in}}%
\pgfpathlineto{\pgfqpoint{3.818093in}{0.759098in}}%
\pgfpathlineto{\pgfqpoint{3.818657in}{0.759096in}}%
\pgfpathlineto{\pgfqpoint{3.819221in}{0.759093in}}%
\pgfpathlineto{\pgfqpoint{3.819786in}{0.759090in}}%
\pgfpathlineto{\pgfqpoint{3.820350in}{0.759088in}}%
\pgfpathlineto{\pgfqpoint{3.820914in}{0.759085in}}%
\pgfpathlineto{\pgfqpoint{3.821479in}{0.759083in}}%
\pgfpathlineto{\pgfqpoint{3.822043in}{0.759080in}}%
\pgfpathlineto{\pgfqpoint{3.822607in}{0.759077in}}%
\pgfpathlineto{\pgfqpoint{3.823172in}{0.759075in}}%
\pgfpathlineto{\pgfqpoint{3.823736in}{0.759072in}}%
\pgfpathlineto{\pgfqpoint{3.824300in}{0.759070in}}%
\pgfpathlineto{\pgfqpoint{3.824865in}{0.759067in}}%
\pgfpathlineto{\pgfqpoint{3.825429in}{0.759065in}}%
\pgfpathlineto{\pgfqpoint{3.825994in}{0.759062in}}%
\pgfpathlineto{\pgfqpoint{3.826558in}{0.759059in}}%
\pgfpathlineto{\pgfqpoint{3.827122in}{0.759057in}}%
\pgfpathlineto{\pgfqpoint{3.827687in}{0.759054in}}%
\pgfpathlineto{\pgfqpoint{3.828251in}{0.759052in}}%
\pgfpathlineto{\pgfqpoint{3.828815in}{0.759049in}}%
\pgfpathlineto{\pgfqpoint{3.829380in}{0.759046in}}%
\pgfpathlineto{\pgfqpoint{3.829944in}{0.759044in}}%
\pgfpathlineto{\pgfqpoint{3.830508in}{0.759041in}}%
\pgfpathlineto{\pgfqpoint{3.831073in}{0.759039in}}%
\pgfpathlineto{\pgfqpoint{3.831637in}{0.759036in}}%
\pgfpathlineto{\pgfqpoint{3.832201in}{0.759034in}}%
\pgfpathlineto{\pgfqpoint{3.832766in}{0.759031in}}%
\pgfpathlineto{\pgfqpoint{3.833330in}{0.759028in}}%
\pgfpathlineto{\pgfqpoint{3.833894in}{0.759026in}}%
\pgfpathlineto{\pgfqpoint{3.834459in}{0.759023in}}%
\pgfpathlineto{\pgfqpoint{3.835023in}{0.759021in}}%
\pgfpathlineto{\pgfqpoint{3.835587in}{0.759018in}}%
\pgfpathlineto{\pgfqpoint{3.836152in}{0.759015in}}%
\pgfpathlineto{\pgfqpoint{3.836716in}{0.759013in}}%
\pgfpathlineto{\pgfqpoint{3.837280in}{0.759010in}}%
\pgfpathlineto{\pgfqpoint{3.837845in}{0.759008in}}%
\pgfpathlineto{\pgfqpoint{3.838409in}{0.759005in}}%
\pgfpathlineto{\pgfqpoint{3.838973in}{0.759003in}}%
\pgfpathlineto{\pgfqpoint{3.839538in}{0.759000in}}%
\pgfpathlineto{\pgfqpoint{3.840102in}{0.758997in}}%
\pgfpathlineto{\pgfqpoint{3.840667in}{0.758995in}}%
\pgfpathlineto{\pgfqpoint{3.841231in}{0.758992in}}%
\pgfpathlineto{\pgfqpoint{3.841795in}{0.758990in}}%
\pgfpathlineto{\pgfqpoint{3.842360in}{0.758987in}}%
\pgfpathlineto{\pgfqpoint{3.842924in}{0.758984in}}%
\pgfpathlineto{\pgfqpoint{3.843488in}{0.758982in}}%
\pgfpathlineto{\pgfqpoint{3.844053in}{0.758979in}}%
\pgfpathlineto{\pgfqpoint{3.844617in}{0.758977in}}%
\pgfpathlineto{\pgfqpoint{3.845181in}{0.758974in}}%
\pgfpathlineto{\pgfqpoint{3.845746in}{0.758972in}}%
\pgfpathlineto{\pgfqpoint{3.846310in}{0.758969in}}%
\pgfpathlineto{\pgfqpoint{3.846874in}{0.758966in}}%
\pgfpathlineto{\pgfqpoint{3.847439in}{0.758964in}}%
\pgfpathlineto{\pgfqpoint{3.848003in}{0.758961in}}%
\pgfpathlineto{\pgfqpoint{3.848567in}{0.758959in}}%
\pgfpathlineto{\pgfqpoint{3.849132in}{0.758956in}}%
\pgfpathlineto{\pgfqpoint{3.849696in}{0.758953in}}%
\pgfpathlineto{\pgfqpoint{3.850260in}{0.758951in}}%
\pgfpathlineto{\pgfqpoint{3.850825in}{0.758948in}}%
\pgfpathlineto{\pgfqpoint{3.851389in}{0.758946in}}%
\pgfpathlineto{\pgfqpoint{3.851953in}{0.758943in}}%
\pgfpathlineto{\pgfqpoint{3.852518in}{0.758940in}}%
\pgfpathlineto{\pgfqpoint{3.853082in}{0.758938in}}%
\pgfpathlineto{\pgfqpoint{3.853646in}{0.758935in}}%
\pgfpathlineto{\pgfqpoint{3.854211in}{0.758933in}}%
\pgfpathlineto{\pgfqpoint{3.854775in}{0.758930in}}%
\pgfpathlineto{\pgfqpoint{3.855339in}{0.758928in}}%
\pgfpathlineto{\pgfqpoint{3.855904in}{0.758925in}}%
\pgfpathlineto{\pgfqpoint{3.856468in}{0.758922in}}%
\pgfpathlineto{\pgfqpoint{3.857033in}{0.758920in}}%
\pgfpathlineto{\pgfqpoint{3.857597in}{0.758917in}}%
\pgfpathlineto{\pgfqpoint{3.858161in}{0.758915in}}%
\pgfpathlineto{\pgfqpoint{3.858726in}{0.758912in}}%
\pgfpathlineto{\pgfqpoint{3.859290in}{0.758909in}}%
\pgfpathlineto{\pgfqpoint{3.859854in}{0.758907in}}%
\pgfpathlineto{\pgfqpoint{3.860419in}{0.758904in}}%
\pgfpathlineto{\pgfqpoint{3.860983in}{0.758902in}}%
\pgfpathlineto{\pgfqpoint{3.861547in}{0.758899in}}%
\pgfpathlineto{\pgfqpoint{3.862112in}{0.758897in}}%
\pgfpathlineto{\pgfqpoint{3.862676in}{0.758894in}}%
\pgfpathlineto{\pgfqpoint{3.863240in}{0.758891in}}%
\pgfpathlineto{\pgfqpoint{3.863805in}{0.758889in}}%
\pgfpathlineto{\pgfqpoint{3.864369in}{0.758886in}}%
\pgfpathlineto{\pgfqpoint{3.864933in}{0.758884in}}%
\pgfpathlineto{\pgfqpoint{3.865498in}{0.758881in}}%
\pgfpathlineto{\pgfqpoint{3.866062in}{0.758878in}}%
\pgfpathlineto{\pgfqpoint{3.866626in}{0.758876in}}%
\pgfpathlineto{\pgfqpoint{3.867191in}{0.758873in}}%
\pgfpathlineto{\pgfqpoint{3.867755in}{0.758871in}}%
\pgfpathlineto{\pgfqpoint{3.868319in}{0.758868in}}%
\pgfpathlineto{\pgfqpoint{3.868884in}{0.758866in}}%
\pgfpathlineto{\pgfqpoint{3.869448in}{0.758863in}}%
\pgfpathlineto{\pgfqpoint{3.870012in}{0.758860in}}%
\pgfpathlineto{\pgfqpoint{3.870577in}{0.758858in}}%
\pgfpathlineto{\pgfqpoint{3.871141in}{0.758855in}}%
\pgfpathlineto{\pgfqpoint{3.871706in}{0.758853in}}%
\pgfpathlineto{\pgfqpoint{3.872270in}{0.758850in}}%
\pgfpathlineto{\pgfqpoint{3.872834in}{0.758847in}}%
\pgfpathlineto{\pgfqpoint{3.873399in}{0.758845in}}%
\pgfpathlineto{\pgfqpoint{3.873963in}{0.758842in}}%
\pgfpathlineto{\pgfqpoint{3.874527in}{0.758840in}}%
\pgfpathlineto{\pgfqpoint{3.875092in}{0.758837in}}%
\pgfpathlineto{\pgfqpoint{3.875656in}{0.758835in}}%
\pgfpathlineto{\pgfqpoint{3.876220in}{0.758832in}}%
\pgfpathlineto{\pgfqpoint{3.876785in}{0.758829in}}%
\pgfpathlineto{\pgfqpoint{3.877349in}{0.758827in}}%
\pgfpathlineto{\pgfqpoint{3.877913in}{0.758824in}}%
\pgfpathlineto{\pgfqpoint{3.878478in}{0.758822in}}%
\pgfpathlineto{\pgfqpoint{3.879042in}{0.758819in}}%
\pgfpathlineto{\pgfqpoint{3.879606in}{0.758816in}}%
\pgfpathlineto{\pgfqpoint{3.880171in}{0.758814in}}%
\pgfpathlineto{\pgfqpoint{3.880735in}{0.758811in}}%
\pgfpathlineto{\pgfqpoint{3.881299in}{0.758809in}}%
\pgfpathlineto{\pgfqpoint{3.881864in}{0.758806in}}%
\pgfpathlineto{\pgfqpoint{3.882428in}{0.758803in}}%
\pgfpathlineto{\pgfqpoint{3.882992in}{0.758801in}}%
\pgfpathlineto{\pgfqpoint{3.883557in}{0.758798in}}%
\pgfpathlineto{\pgfqpoint{3.884121in}{0.758796in}}%
\pgfpathlineto{\pgfqpoint{3.884685in}{0.758793in}}%
\pgfpathlineto{\pgfqpoint{3.885250in}{0.758791in}}%
\pgfpathlineto{\pgfqpoint{3.885814in}{0.758788in}}%
\pgfpathlineto{\pgfqpoint{3.886379in}{0.758785in}}%
\pgfpathlineto{\pgfqpoint{3.886943in}{0.758783in}}%
\pgfpathlineto{\pgfqpoint{3.887507in}{0.758780in}}%
\pgfpathlineto{\pgfqpoint{3.888072in}{0.758778in}}%
\pgfpathlineto{\pgfqpoint{3.888636in}{0.758775in}}%
\pgfpathlineto{\pgfqpoint{3.889200in}{0.758772in}}%
\pgfpathlineto{\pgfqpoint{3.889765in}{0.758770in}}%
\pgfpathlineto{\pgfqpoint{3.890329in}{0.758767in}}%
\pgfpathlineto{\pgfqpoint{3.890893in}{0.758765in}}%
\pgfpathlineto{\pgfqpoint{3.891458in}{0.758762in}}%
\pgfpathlineto{\pgfqpoint{3.892022in}{0.758760in}}%
\pgfpathlineto{\pgfqpoint{3.892586in}{0.758757in}}%
\pgfpathlineto{\pgfqpoint{3.893151in}{0.758754in}}%
\pgfpathlineto{\pgfqpoint{3.893715in}{0.758752in}}%
\pgfpathlineto{\pgfqpoint{3.894279in}{0.758749in}}%
\pgfpathlineto{\pgfqpoint{3.894844in}{0.758747in}}%
\pgfpathlineto{\pgfqpoint{3.895408in}{0.758744in}}%
\pgfpathlineto{\pgfqpoint{3.895972in}{0.758741in}}%
\pgfpathlineto{\pgfqpoint{3.896537in}{0.758739in}}%
\pgfpathlineto{\pgfqpoint{3.897101in}{0.758736in}}%
\pgfpathlineto{\pgfqpoint{3.897665in}{0.758734in}}%
\pgfpathlineto{\pgfqpoint{3.898230in}{0.758731in}}%
\pgfpathlineto{\pgfqpoint{3.898794in}{0.758729in}}%
\pgfpathlineto{\pgfqpoint{3.899358in}{0.758726in}}%
\pgfpathlineto{\pgfqpoint{3.899923in}{0.758723in}}%
\pgfpathlineto{\pgfqpoint{3.900487in}{0.758721in}}%
\pgfpathlineto{\pgfqpoint{3.901051in}{0.758718in}}%
\pgfpathlineto{\pgfqpoint{3.901616in}{0.758716in}}%
\pgfpathlineto{\pgfqpoint{3.902180in}{0.758713in}}%
\pgfpathlineto{\pgfqpoint{3.902745in}{0.758710in}}%
\pgfpathlineto{\pgfqpoint{3.903309in}{0.758708in}}%
\pgfpathlineto{\pgfqpoint{3.903873in}{0.758705in}}%
\pgfpathlineto{\pgfqpoint{3.904438in}{0.758703in}}%
\pgfpathlineto{\pgfqpoint{3.905002in}{0.758700in}}%
\pgfpathlineto{\pgfqpoint{3.905566in}{0.758697in}}%
\pgfpathlineto{\pgfqpoint{3.906131in}{0.758695in}}%
\pgfpathlineto{\pgfqpoint{3.906695in}{0.758692in}}%
\pgfpathlineto{\pgfqpoint{3.907259in}{0.758690in}}%
\pgfpathlineto{\pgfqpoint{3.907824in}{0.758687in}}%
\pgfpathlineto{\pgfqpoint{3.908388in}{0.758685in}}%
\pgfpathlineto{\pgfqpoint{3.908952in}{0.758682in}}%
\pgfpathlineto{\pgfqpoint{3.909517in}{0.758679in}}%
\pgfpathlineto{\pgfqpoint{3.910081in}{0.758677in}}%
\pgfpathlineto{\pgfqpoint{3.910645in}{0.758674in}}%
\pgfpathlineto{\pgfqpoint{3.911210in}{0.758672in}}%
\pgfpathlineto{\pgfqpoint{3.911774in}{0.758669in}}%
\pgfpathlineto{\pgfqpoint{3.912338in}{0.758666in}}%
\pgfpathlineto{\pgfqpoint{3.912903in}{0.758664in}}%
\pgfpathlineto{\pgfqpoint{3.913467in}{0.758661in}}%
\pgfpathlineto{\pgfqpoint{3.914031in}{0.758659in}}%
\pgfpathlineto{\pgfqpoint{3.914596in}{0.758656in}}%
\pgfpathlineto{\pgfqpoint{3.915160in}{0.758654in}}%
\pgfpathlineto{\pgfqpoint{3.915724in}{0.758651in}}%
\pgfpathlineto{\pgfqpoint{3.916289in}{0.758648in}}%
\pgfpathlineto{\pgfqpoint{3.916853in}{0.758646in}}%
\pgfpathlineto{\pgfqpoint{3.917418in}{0.758643in}}%
\pgfpathlineto{\pgfqpoint{3.917982in}{0.758641in}}%
\pgfpathlineto{\pgfqpoint{3.918546in}{0.758636in}}%
\pgfpathlineto{\pgfqpoint{3.919111in}{0.758629in}}%
\pgfpathlineto{\pgfqpoint{3.919675in}{0.758621in}}%
\pgfpathlineto{\pgfqpoint{3.920239in}{0.758614in}}%
\pgfpathlineto{\pgfqpoint{3.920804in}{0.758606in}}%
\pgfpathlineto{\pgfqpoint{3.921368in}{0.758599in}}%
\pgfpathlineto{\pgfqpoint{3.921932in}{0.758592in}}%
\pgfpathlineto{\pgfqpoint{3.922497in}{0.758584in}}%
\pgfpathlineto{\pgfqpoint{3.923061in}{0.758577in}}%
\pgfpathlineto{\pgfqpoint{3.923625in}{0.758570in}}%
\pgfpathlineto{\pgfqpoint{3.924190in}{0.758562in}}%
\pgfpathlineto{\pgfqpoint{3.924754in}{0.758555in}}%
\pgfpathlineto{\pgfqpoint{3.925318in}{0.758548in}}%
\pgfpathlineto{\pgfqpoint{3.925883in}{0.758540in}}%
\pgfpathlineto{\pgfqpoint{3.926447in}{0.758533in}}%
\pgfpathlineto{\pgfqpoint{3.927011in}{0.758525in}}%
\pgfpathlineto{\pgfqpoint{3.927576in}{0.758518in}}%
\pgfpathlineto{\pgfqpoint{3.928140in}{0.758511in}}%
\pgfpathlineto{\pgfqpoint{3.928704in}{0.758503in}}%
\pgfpathlineto{\pgfqpoint{3.929269in}{0.758496in}}%
\pgfpathlineto{\pgfqpoint{3.929833in}{0.758489in}}%
\pgfpathlineto{\pgfqpoint{3.930397in}{0.758481in}}%
\pgfpathlineto{\pgfqpoint{3.930962in}{0.758474in}}%
\pgfpathlineto{\pgfqpoint{3.931526in}{0.758466in}}%
\pgfpathlineto{\pgfqpoint{3.932091in}{0.758459in}}%
\pgfpathlineto{\pgfqpoint{3.932655in}{0.758452in}}%
\pgfpathlineto{\pgfqpoint{3.933219in}{0.758444in}}%
\pgfpathlineto{\pgfqpoint{3.933784in}{0.758437in}}%
\pgfpathlineto{\pgfqpoint{3.934348in}{0.758430in}}%
\pgfpathlineto{\pgfqpoint{3.934912in}{0.758422in}}%
\pgfpathlineto{\pgfqpoint{3.935477in}{0.758415in}}%
\pgfpathlineto{\pgfqpoint{3.936041in}{0.758408in}}%
\pgfpathlineto{\pgfqpoint{3.936605in}{0.758400in}}%
\pgfpathlineto{\pgfqpoint{3.937170in}{0.758393in}}%
\pgfpathlineto{\pgfqpoint{3.937734in}{0.758385in}}%
\pgfpathlineto{\pgfqpoint{3.938298in}{0.758378in}}%
\pgfpathlineto{\pgfqpoint{3.938863in}{0.758371in}}%
\pgfpathlineto{\pgfqpoint{3.939427in}{0.758363in}}%
\pgfpathlineto{\pgfqpoint{3.939991in}{0.758356in}}%
\pgfpathlineto{\pgfqpoint{3.940556in}{0.758349in}}%
\pgfpathlineto{\pgfqpoint{3.941120in}{0.758341in}}%
\pgfpathlineto{\pgfqpoint{3.941684in}{0.758334in}}%
\pgfpathlineto{\pgfqpoint{3.942249in}{0.758327in}}%
\pgfpathlineto{\pgfqpoint{3.942813in}{0.758319in}}%
\pgfpathlineto{\pgfqpoint{3.943377in}{0.758312in}}%
\pgfpathlineto{\pgfqpoint{3.943942in}{0.758304in}}%
\pgfpathlineto{\pgfqpoint{3.944506in}{0.758297in}}%
\pgfpathlineto{\pgfqpoint{3.945070in}{0.758290in}}%
\pgfpathlineto{\pgfqpoint{3.945635in}{0.758282in}}%
\pgfpathlineto{\pgfqpoint{3.946199in}{0.758275in}}%
\pgfpathlineto{\pgfqpoint{3.946763in}{0.758268in}}%
\pgfpathlineto{\pgfqpoint{3.947328in}{0.758260in}}%
\pgfpathlineto{\pgfqpoint{3.947892in}{0.758253in}}%
\pgfpathlineto{\pgfqpoint{3.948457in}{0.758245in}}%
\pgfpathlineto{\pgfqpoint{3.949021in}{0.758238in}}%
\pgfpathlineto{\pgfqpoint{3.949585in}{0.758231in}}%
\pgfpathlineto{\pgfqpoint{3.950150in}{0.758223in}}%
\pgfpathlineto{\pgfqpoint{3.950714in}{0.758216in}}%
\pgfpathlineto{\pgfqpoint{3.951278in}{0.758209in}}%
\pgfpathlineto{\pgfqpoint{3.951843in}{0.758201in}}%
\pgfpathlineto{\pgfqpoint{3.952407in}{0.758194in}}%
\pgfpathlineto{\pgfqpoint{3.952971in}{0.758187in}}%
\pgfpathlineto{\pgfqpoint{3.953536in}{0.758179in}}%
\pgfpathlineto{\pgfqpoint{3.954100in}{0.758172in}}%
\pgfpathlineto{\pgfqpoint{3.954664in}{0.758164in}}%
\pgfpathlineto{\pgfqpoint{3.955229in}{0.758157in}}%
\pgfpathlineto{\pgfqpoint{3.955793in}{0.758150in}}%
\pgfpathlineto{\pgfqpoint{3.956357in}{0.758142in}}%
\pgfpathlineto{\pgfqpoint{3.956922in}{0.758135in}}%
\pgfpathlineto{\pgfqpoint{3.957486in}{0.758128in}}%
\pgfpathlineto{\pgfqpoint{3.958050in}{0.758120in}}%
\pgfpathlineto{\pgfqpoint{3.958615in}{0.758113in}}%
\pgfpathlineto{\pgfqpoint{3.959179in}{0.758106in}}%
\pgfpathlineto{\pgfqpoint{3.959743in}{0.758098in}}%
\pgfpathlineto{\pgfqpoint{3.960308in}{0.758091in}}%
\pgfpathlineto{\pgfqpoint{3.960872in}{0.758083in}}%
\pgfpathlineto{\pgfqpoint{3.961436in}{0.758076in}}%
\pgfpathlineto{\pgfqpoint{3.962001in}{0.758069in}}%
\pgfpathlineto{\pgfqpoint{3.962565in}{0.758061in}}%
\pgfpathlineto{\pgfqpoint{3.963130in}{0.758054in}}%
\pgfpathlineto{\pgfqpoint{3.963694in}{0.758047in}}%
\pgfpathlineto{\pgfqpoint{3.964258in}{0.758039in}}%
\pgfpathlineto{\pgfqpoint{3.964823in}{0.758032in}}%
\pgfpathlineto{\pgfqpoint{3.965387in}{0.758024in}}%
\pgfpathlineto{\pgfqpoint{3.965951in}{0.758017in}}%
\pgfpathlineto{\pgfqpoint{3.966516in}{0.758010in}}%
\pgfpathlineto{\pgfqpoint{3.967080in}{0.758002in}}%
\pgfpathlineto{\pgfqpoint{3.967644in}{0.757995in}}%
\pgfpathlineto{\pgfqpoint{3.968209in}{0.757988in}}%
\pgfpathlineto{\pgfqpoint{3.968773in}{0.757980in}}%
\pgfpathlineto{\pgfqpoint{3.969337in}{0.757973in}}%
\pgfpathlineto{\pgfqpoint{3.969902in}{0.757966in}}%
\pgfpathlineto{\pgfqpoint{3.970466in}{0.757958in}}%
\pgfpathlineto{\pgfqpoint{3.971030in}{0.757951in}}%
\pgfpathlineto{\pgfqpoint{3.971595in}{0.757943in}}%
\pgfpathlineto{\pgfqpoint{3.972159in}{0.757936in}}%
\pgfpathlineto{\pgfqpoint{3.972723in}{0.757929in}}%
\pgfpathlineto{\pgfqpoint{3.973288in}{0.757921in}}%
\pgfpathlineto{\pgfqpoint{3.973852in}{0.757914in}}%
\pgfpathlineto{\pgfqpoint{3.974416in}{0.757907in}}%
\pgfpathlineto{\pgfqpoint{3.974981in}{0.757899in}}%
\pgfpathlineto{\pgfqpoint{3.975545in}{0.757892in}}%
\pgfpathlineto{\pgfqpoint{3.976109in}{0.757884in}}%
\pgfpathlineto{\pgfqpoint{3.976674in}{0.757877in}}%
\pgfpathlineto{\pgfqpoint{3.977238in}{0.757870in}}%
\pgfpathlineto{\pgfqpoint{3.977803in}{0.757862in}}%
\pgfpathlineto{\pgfqpoint{3.978367in}{0.757855in}}%
\pgfpathlineto{\pgfqpoint{3.978931in}{0.757848in}}%
\pgfpathlineto{\pgfqpoint{3.979496in}{0.757840in}}%
\pgfpathlineto{\pgfqpoint{3.980060in}{0.757833in}}%
\pgfpathlineto{\pgfqpoint{3.980624in}{0.757826in}}%
\pgfpathlineto{\pgfqpoint{3.981189in}{0.757818in}}%
\pgfpathlineto{\pgfqpoint{3.981753in}{0.757811in}}%
\pgfpathlineto{\pgfqpoint{3.982317in}{0.757803in}}%
\pgfpathlineto{\pgfqpoint{3.982882in}{0.757796in}}%
\pgfpathlineto{\pgfqpoint{3.983446in}{0.757789in}}%
\pgfpathlineto{\pgfqpoint{3.984010in}{0.757781in}}%
\pgfpathlineto{\pgfqpoint{3.984575in}{0.757774in}}%
\pgfpathlineto{\pgfqpoint{3.985139in}{0.757767in}}%
\pgfpathlineto{\pgfqpoint{3.985703in}{0.757759in}}%
\pgfpathlineto{\pgfqpoint{3.986268in}{0.757752in}}%
\pgfpathlineto{\pgfqpoint{3.986832in}{0.757745in}}%
\pgfpathlineto{\pgfqpoint{3.987396in}{0.757737in}}%
\pgfpathlineto{\pgfqpoint{3.987961in}{0.757730in}}%
\pgfpathlineto{\pgfqpoint{3.988525in}{0.757722in}}%
\pgfpathlineto{\pgfqpoint{3.989089in}{0.757715in}}%
\pgfpathlineto{\pgfqpoint{3.989654in}{0.757708in}}%
\pgfpathlineto{\pgfqpoint{3.990218in}{0.757700in}}%
\pgfpathlineto{\pgfqpoint{3.990782in}{0.757693in}}%
\pgfpathlineto{\pgfqpoint{3.991347in}{0.757686in}}%
\pgfpathlineto{\pgfqpoint{3.991911in}{0.757678in}}%
\pgfpathlineto{\pgfqpoint{3.992475in}{0.757671in}}%
\pgfpathlineto{\pgfqpoint{3.993040in}{0.757663in}}%
\pgfpathlineto{\pgfqpoint{3.993604in}{0.757656in}}%
\pgfpathlineto{\pgfqpoint{3.994169in}{0.757649in}}%
\pgfpathlineto{\pgfqpoint{3.994733in}{0.757641in}}%
\pgfpathlineto{\pgfqpoint{3.995297in}{0.757634in}}%
\pgfpathlineto{\pgfqpoint{3.995862in}{0.757627in}}%
\pgfpathlineto{\pgfqpoint{3.996426in}{0.757619in}}%
\pgfpathlineto{\pgfqpoint{3.996990in}{0.757612in}}%
\pgfpathlineto{\pgfqpoint{3.997555in}{0.757605in}}%
\pgfpathlineto{\pgfqpoint{3.998119in}{0.757597in}}%
\pgfpathlineto{\pgfqpoint{3.998683in}{0.757590in}}%
\pgfpathlineto{\pgfqpoint{3.999248in}{0.757582in}}%
\pgfpathlineto{\pgfqpoint{3.999812in}{0.757575in}}%
\pgfpathlineto{\pgfqpoint{4.000376in}{0.757568in}}%
\pgfpathlineto{\pgfqpoint{4.000941in}{0.757560in}}%
\pgfpathlineto{\pgfqpoint{4.001505in}{0.757553in}}%
\pgfpathlineto{\pgfqpoint{4.002069in}{0.757546in}}%
\pgfpathlineto{\pgfqpoint{4.002634in}{0.757538in}}%
\pgfpathlineto{\pgfqpoint{4.003198in}{0.757531in}}%
\pgfpathlineto{\pgfqpoint{4.003762in}{0.757524in}}%
\pgfpathlineto{\pgfqpoint{4.004327in}{0.757516in}}%
\pgfpathlineto{\pgfqpoint{4.004891in}{0.757509in}}%
\pgfpathlineto{\pgfqpoint{4.005455in}{0.757501in}}%
\pgfpathlineto{\pgfqpoint{4.006020in}{0.757494in}}%
\pgfpathlineto{\pgfqpoint{4.006584in}{0.757487in}}%
\pgfpathlineto{\pgfqpoint{4.007148in}{0.757479in}}%
\pgfpathlineto{\pgfqpoint{4.007713in}{0.757472in}}%
\pgfpathlineto{\pgfqpoint{4.008277in}{0.757465in}}%
\pgfpathlineto{\pgfqpoint{4.008842in}{0.757457in}}%
\pgfpathlineto{\pgfqpoint{4.009406in}{0.757450in}}%
\pgfpathlineto{\pgfqpoint{4.009970in}{0.757442in}}%
\pgfpathlineto{\pgfqpoint{4.010535in}{0.757435in}}%
\pgfpathlineto{\pgfqpoint{4.011099in}{0.757428in}}%
\pgfpathlineto{\pgfqpoint{4.011663in}{0.757420in}}%
\pgfpathlineto{\pgfqpoint{4.012228in}{0.757413in}}%
\pgfpathlineto{\pgfqpoint{4.012792in}{0.757406in}}%
\pgfpathlineto{\pgfqpoint{4.013356in}{0.757398in}}%
\pgfpathlineto{\pgfqpoint{4.013921in}{0.757391in}}%
\pgfpathlineto{\pgfqpoint{4.014485in}{0.757384in}}%
\pgfpathlineto{\pgfqpoint{4.015049in}{0.757376in}}%
\pgfpathlineto{\pgfqpoint{4.015614in}{0.757369in}}%
\pgfpathlineto{\pgfqpoint{4.016178in}{0.757361in}}%
\pgfpathlineto{\pgfqpoint{4.016742in}{0.757354in}}%
\pgfpathlineto{\pgfqpoint{4.017307in}{0.757347in}}%
\pgfpathlineto{\pgfqpoint{4.017871in}{0.757339in}}%
\pgfpathlineto{\pgfqpoint{4.018435in}{0.757332in}}%
\pgfpathlineto{\pgfqpoint{4.019000in}{0.757325in}}%
\pgfpathlineto{\pgfqpoint{4.019564in}{0.757317in}}%
\pgfpathlineto{\pgfqpoint{4.020128in}{0.757310in}}%
\pgfpathlineto{\pgfqpoint{4.020693in}{0.757303in}}%
\pgfpathlineto{\pgfqpoint{4.021257in}{0.757295in}}%
\pgfpathlineto{\pgfqpoint{4.021821in}{0.757288in}}%
\pgfpathlineto{\pgfqpoint{4.022386in}{0.757280in}}%
\pgfpathlineto{\pgfqpoint{4.022950in}{0.757273in}}%
\pgfpathlineto{\pgfqpoint{4.023515in}{0.757266in}}%
\pgfpathlineto{\pgfqpoint{4.024079in}{0.757258in}}%
\pgfpathlineto{\pgfqpoint{4.024643in}{0.757251in}}%
\pgfpathlineto{\pgfqpoint{4.025208in}{0.757244in}}%
\pgfpathlineto{\pgfqpoint{4.025772in}{0.757236in}}%
\pgfpathlineto{\pgfqpoint{4.026336in}{0.757229in}}%
\pgfpathlineto{\pgfqpoint{4.026901in}{0.757221in}}%
\pgfpathlineto{\pgfqpoint{4.027465in}{0.757214in}}%
\pgfpathlineto{\pgfqpoint{4.028029in}{0.757207in}}%
\pgfpathlineto{\pgfqpoint{4.028594in}{0.757199in}}%
\pgfpathlineto{\pgfqpoint{4.029158in}{0.757192in}}%
\pgfpathlineto{\pgfqpoint{4.029722in}{0.757185in}}%
\pgfpathlineto{\pgfqpoint{4.030287in}{0.757177in}}%
\pgfpathlineto{\pgfqpoint{4.030851in}{0.757170in}}%
\pgfpathlineto{\pgfqpoint{4.031415in}{0.757163in}}%
\pgfpathlineto{\pgfqpoint{4.031980in}{0.757155in}}%
\pgfpathlineto{\pgfqpoint{4.032544in}{0.757148in}}%
\pgfpathlineto{\pgfqpoint{4.033108in}{0.757140in}}%
\pgfpathlineto{\pgfqpoint{4.033673in}{0.757133in}}%
\pgfpathlineto{\pgfqpoint{4.034237in}{0.757126in}}%
\pgfpathlineto{\pgfqpoint{4.034801in}{0.757118in}}%
\pgfpathlineto{\pgfqpoint{4.035366in}{0.757111in}}%
\pgfpathlineto{\pgfqpoint{4.035930in}{0.757104in}}%
\pgfpathlineto{\pgfqpoint{4.036494in}{0.757096in}}%
\pgfpathlineto{\pgfqpoint{4.037059in}{0.757089in}}%
\pgfpathlineto{\pgfqpoint{4.037623in}{0.757081in}}%
\pgfpathlineto{\pgfqpoint{4.038187in}{0.757074in}}%
\pgfpathlineto{\pgfqpoint{4.038752in}{0.757067in}}%
\pgfpathlineto{\pgfqpoint{4.039316in}{0.757059in}}%
\pgfpathlineto{\pgfqpoint{4.039881in}{0.757052in}}%
\pgfpathlineto{\pgfqpoint{4.040445in}{0.757045in}}%
\pgfpathlineto{\pgfqpoint{4.041009in}{0.757037in}}%
\pgfpathlineto{\pgfqpoint{4.041574in}{0.757030in}}%
\pgfpathlineto{\pgfqpoint{4.042138in}{0.757023in}}%
\pgfpathlineto{\pgfqpoint{4.042702in}{0.757015in}}%
\pgfpathlineto{\pgfqpoint{4.043267in}{0.757008in}}%
\pgfpathlineto{\pgfqpoint{4.043831in}{0.757000in}}%
\pgfpathlineto{\pgfqpoint{4.044395in}{0.756993in}}%
\pgfpathlineto{\pgfqpoint{4.044960in}{0.756986in}}%
\pgfpathlineto{\pgfqpoint{4.045524in}{0.756978in}}%
\pgfpathlineto{\pgfqpoint{4.046088in}{0.756971in}}%
\pgfpathlineto{\pgfqpoint{4.046653in}{0.756964in}}%
\pgfpathlineto{\pgfqpoint{4.047217in}{0.756956in}}%
\pgfpathlineto{\pgfqpoint{4.047781in}{0.756949in}}%
\pgfpathlineto{\pgfqpoint{4.048346in}{0.756942in}}%
\pgfpathlineto{\pgfqpoint{4.048910in}{0.756934in}}%
\pgfpathlineto{\pgfqpoint{4.049474in}{0.756927in}}%
\pgfpathlineto{\pgfqpoint{4.050039in}{0.756919in}}%
\pgfpathlineto{\pgfqpoint{4.050603in}{0.756912in}}%
\pgfpathlineto{\pgfqpoint{4.051167in}{0.756905in}}%
\pgfpathlineto{\pgfqpoint{4.051732in}{0.756897in}}%
\pgfpathlineto{\pgfqpoint{4.052296in}{0.756890in}}%
\pgfpathlineto{\pgfqpoint{4.052860in}{0.756883in}}%
\pgfpathlineto{\pgfqpoint{4.053425in}{0.756875in}}%
\pgfpathlineto{\pgfqpoint{4.053989in}{0.756868in}}%
\pgfpathlineto{\pgfqpoint{4.054554in}{0.756860in}}%
\pgfpathlineto{\pgfqpoint{4.055118in}{0.756853in}}%
\pgfpathlineto{\pgfqpoint{4.055682in}{0.756846in}}%
\pgfpathlineto{\pgfqpoint{4.056247in}{0.756838in}}%
\pgfpathlineto{\pgfqpoint{4.056811in}{0.756831in}}%
\pgfpathlineto{\pgfqpoint{4.057375in}{0.756824in}}%
\pgfpathlineto{\pgfqpoint{4.057940in}{0.756816in}}%
\pgfpathlineto{\pgfqpoint{4.058504in}{0.756809in}}%
\pgfpathlineto{\pgfqpoint{4.059068in}{0.756802in}}%
\pgfpathlineto{\pgfqpoint{4.059633in}{0.756794in}}%
\pgfpathlineto{\pgfqpoint{4.060197in}{0.756787in}}%
\pgfpathlineto{\pgfqpoint{4.060761in}{0.756779in}}%
\pgfpathlineto{\pgfqpoint{4.061326in}{0.756772in}}%
\pgfpathlineto{\pgfqpoint{4.061890in}{0.756765in}}%
\pgfpathlineto{\pgfqpoint{4.062454in}{0.756757in}}%
\pgfpathlineto{\pgfqpoint{4.063019in}{0.756750in}}%
\pgfpathlineto{\pgfqpoint{4.063583in}{0.756743in}}%
\pgfpathlineto{\pgfqpoint{4.064147in}{0.756735in}}%
\pgfpathlineto{\pgfqpoint{4.064712in}{0.756728in}}%
\pgfpathlineto{\pgfqpoint{4.065276in}{0.756721in}}%
\pgfpathlineto{\pgfqpoint{4.065840in}{0.756713in}}%
\pgfpathlineto{\pgfqpoint{4.066405in}{0.756706in}}%
\pgfpathlineto{\pgfqpoint{4.066969in}{0.756698in}}%
\pgfpathlineto{\pgfqpoint{4.067533in}{0.756691in}}%
\pgfpathlineto{\pgfqpoint{4.068098in}{0.756684in}}%
\pgfpathlineto{\pgfqpoint{4.068662in}{0.756676in}}%
\pgfpathlineto{\pgfqpoint{4.069227in}{0.756669in}}%
\pgfpathlineto{\pgfqpoint{4.069791in}{0.756662in}}%
\pgfpathlineto{\pgfqpoint{4.070355in}{0.756654in}}%
\pgfpathlineto{\pgfqpoint{4.070920in}{0.756647in}}%
\pgfpathlineto{\pgfqpoint{4.071484in}{0.756639in}}%
\pgfpathlineto{\pgfqpoint{4.072048in}{0.756632in}}%
\pgfpathlineto{\pgfqpoint{4.072613in}{0.756625in}}%
\pgfpathlineto{\pgfqpoint{4.073177in}{0.756617in}}%
\pgfpathlineto{\pgfqpoint{4.073741in}{0.756610in}}%
\pgfpathlineto{\pgfqpoint{4.074306in}{0.756603in}}%
\pgfpathlineto{\pgfqpoint{4.074870in}{0.756595in}}%
\pgfpathlineto{\pgfqpoint{4.075434in}{0.756588in}}%
\pgfpathlineto{\pgfqpoint{4.075999in}{0.756581in}}%
\pgfpathlineto{\pgfqpoint{4.076563in}{0.756573in}}%
\pgfpathlineto{\pgfqpoint{4.077127in}{0.756566in}}%
\pgfpathlineto{\pgfqpoint{4.077692in}{0.756558in}}%
\pgfpathlineto{\pgfqpoint{4.078256in}{0.756551in}}%
\pgfpathlineto{\pgfqpoint{4.078820in}{0.756544in}}%
\pgfpathlineto{\pgfqpoint{4.079385in}{0.756536in}}%
\pgfpathlineto{\pgfqpoint{4.079949in}{0.756529in}}%
\pgfpathlineto{\pgfqpoint{4.080513in}{0.756522in}}%
\pgfpathlineto{\pgfqpoint{4.081078in}{0.756514in}}%
\pgfpathlineto{\pgfqpoint{4.081642in}{0.756507in}}%
\pgfpathlineto{\pgfqpoint{4.082206in}{0.756499in}}%
\pgfpathlineto{\pgfqpoint{4.082771in}{0.756492in}}%
\pgfpathlineto{\pgfqpoint{4.083335in}{0.756485in}}%
\pgfpathlineto{\pgfqpoint{4.083900in}{0.756477in}}%
\pgfpathlineto{\pgfqpoint{4.084464in}{0.756470in}}%
\pgfpathlineto{\pgfqpoint{4.085028in}{0.756463in}}%
\pgfpathlineto{\pgfqpoint{4.085593in}{0.756455in}}%
\pgfpathlineto{\pgfqpoint{4.086157in}{0.756448in}}%
\pgfpathlineto{\pgfqpoint{4.086721in}{0.756441in}}%
\pgfpathlineto{\pgfqpoint{4.087286in}{0.756433in}}%
\pgfpathlineto{\pgfqpoint{4.087850in}{0.756426in}}%
\pgfpathlineto{\pgfqpoint{4.088414in}{0.756418in}}%
\pgfpathlineto{\pgfqpoint{4.088979in}{0.756411in}}%
\pgfpathlineto{\pgfqpoint{4.089543in}{0.756404in}}%
\pgfpathlineto{\pgfqpoint{4.090107in}{0.756396in}}%
\pgfpathlineto{\pgfqpoint{4.090672in}{0.756389in}}%
\pgfpathlineto{\pgfqpoint{4.091236in}{0.756382in}}%
\pgfpathlineto{\pgfqpoint{4.091800in}{0.756374in}}%
\pgfpathlineto{\pgfqpoint{4.092365in}{0.756367in}}%
\pgfpathlineto{\pgfqpoint{4.092929in}{0.756360in}}%
\pgfpathlineto{\pgfqpoint{4.093493in}{0.756352in}}%
\pgfpathlineto{\pgfqpoint{4.094058in}{0.756345in}}%
\pgfpathlineto{\pgfqpoint{4.094622in}{0.756337in}}%
\pgfpathlineto{\pgfqpoint{4.095186in}{0.756330in}}%
\pgfpathlineto{\pgfqpoint{4.095751in}{0.756323in}}%
\pgfpathlineto{\pgfqpoint{4.096315in}{0.756315in}}%
\pgfpathlineto{\pgfqpoint{4.096879in}{0.756308in}}%
\pgfpathlineto{\pgfqpoint{4.097444in}{0.756301in}}%
\pgfpathlineto{\pgfqpoint{4.098008in}{0.756293in}}%
\pgfpathlineto{\pgfqpoint{4.098572in}{0.756286in}}%
\pgfpathlineto{\pgfqpoint{4.099137in}{0.756278in}}%
\pgfpathlineto{\pgfqpoint{4.099701in}{0.756271in}}%
\pgfpathlineto{\pgfqpoint{4.100266in}{0.756264in}}%
\pgfpathlineto{\pgfqpoint{4.100830in}{0.756256in}}%
\pgfpathlineto{\pgfqpoint{4.101394in}{0.756249in}}%
\pgfpathlineto{\pgfqpoint{4.101959in}{0.756242in}}%
\pgfpathlineto{\pgfqpoint{4.102523in}{0.756234in}}%
\pgfpathlineto{\pgfqpoint{4.103087in}{0.756227in}}%
\pgfpathlineto{\pgfqpoint{4.103652in}{0.756220in}}%
\pgfpathlineto{\pgfqpoint{4.104216in}{0.756212in}}%
\pgfpathlineto{\pgfqpoint{4.104780in}{0.756205in}}%
\pgfpathlineto{\pgfqpoint{4.105345in}{0.756197in}}%
\pgfpathlineto{\pgfqpoint{4.105909in}{0.756190in}}%
\pgfpathlineto{\pgfqpoint{4.106473in}{0.756183in}}%
\pgfpathlineto{\pgfqpoint{4.107038in}{0.756175in}}%
\pgfpathlineto{\pgfqpoint{4.107602in}{0.756168in}}%
\pgfpathlineto{\pgfqpoint{4.108166in}{0.756161in}}%
\pgfpathlineto{\pgfqpoint{4.108731in}{0.756153in}}%
\pgfpathlineto{\pgfqpoint{4.109295in}{0.756146in}}%
\pgfpathlineto{\pgfqpoint{4.109859in}{0.756139in}}%
\pgfpathlineto{\pgfqpoint{4.110424in}{0.756131in}}%
\pgfpathlineto{\pgfqpoint{4.110988in}{0.756124in}}%
\pgfpathlineto{\pgfqpoint{4.111552in}{0.756116in}}%
\pgfpathlineto{\pgfqpoint{4.112117in}{0.756109in}}%
\pgfpathlineto{\pgfqpoint{4.112681in}{0.756102in}}%
\pgfpathlineto{\pgfqpoint{4.113245in}{0.756094in}}%
\pgfpathlineto{\pgfqpoint{4.113810in}{0.756087in}}%
\pgfpathlineto{\pgfqpoint{4.114374in}{0.756080in}}%
\pgfpathlineto{\pgfqpoint{4.114939in}{0.756072in}}%
\pgfpathlineto{\pgfqpoint{4.115503in}{0.756065in}}%
\pgfpathlineto{\pgfqpoint{4.116067in}{0.756057in}}%
\pgfpathlineto{\pgfqpoint{4.116632in}{0.756050in}}%
\pgfpathlineto{\pgfqpoint{4.117196in}{0.756043in}}%
\pgfpathlineto{\pgfqpoint{4.117760in}{0.756035in}}%
\pgfpathlineto{\pgfqpoint{4.118325in}{0.756028in}}%
\pgfpathlineto{\pgfqpoint{4.118889in}{0.756021in}}%
\pgfpathlineto{\pgfqpoint{4.119453in}{0.756013in}}%
\pgfpathlineto{\pgfqpoint{4.120018in}{0.756006in}}%
\pgfpathlineto{\pgfqpoint{4.120582in}{0.755999in}}%
\pgfpathlineto{\pgfqpoint{4.121146in}{0.755991in}}%
\pgfpathlineto{\pgfqpoint{4.121711in}{0.755984in}}%
\pgfpathlineto{\pgfqpoint{4.122275in}{0.755976in}}%
\pgfpathlineto{\pgfqpoint{4.122839in}{0.755969in}}%
\pgfpathlineto{\pgfqpoint{4.123404in}{0.755962in}}%
\pgfpathlineto{\pgfqpoint{4.123968in}{0.755954in}}%
\pgfpathlineto{\pgfqpoint{4.124532in}{0.755947in}}%
\pgfpathlineto{\pgfqpoint{4.125097in}{0.755940in}}%
\pgfpathlineto{\pgfqpoint{4.125661in}{0.755932in}}%
\pgfpathlineto{\pgfqpoint{4.126225in}{0.755925in}}%
\pgfpathlineto{\pgfqpoint{4.126790in}{0.755918in}}%
\pgfpathlineto{\pgfqpoint{4.127354in}{0.755910in}}%
\pgfpathlineto{\pgfqpoint{4.127918in}{0.755903in}}%
\pgfpathlineto{\pgfqpoint{4.128483in}{0.755895in}}%
\pgfpathlineto{\pgfqpoint{4.129047in}{0.755888in}}%
\pgfpathlineto{\pgfqpoint{4.129612in}{0.755881in}}%
\pgfpathlineto{\pgfqpoint{4.130176in}{0.755873in}}%
\pgfpathlineto{\pgfqpoint{4.130740in}{0.755866in}}%
\pgfpathlineto{\pgfqpoint{4.131305in}{0.755859in}}%
\pgfpathlineto{\pgfqpoint{4.131869in}{0.755851in}}%
\pgfpathlineto{\pgfqpoint{4.132433in}{0.755844in}}%
\pgfpathlineto{\pgfqpoint{4.132998in}{0.755836in}}%
\pgfpathlineto{\pgfqpoint{4.133562in}{0.755829in}}%
\pgfpathlineto{\pgfqpoint{4.134126in}{0.755822in}}%
\pgfpathlineto{\pgfqpoint{4.134691in}{0.755814in}}%
\pgfpathlineto{\pgfqpoint{4.135255in}{0.755807in}}%
\pgfpathlineto{\pgfqpoint{4.135819in}{0.755800in}}%
\pgfpathlineto{\pgfqpoint{4.136384in}{0.755792in}}%
\pgfpathlineto{\pgfqpoint{4.136948in}{0.755785in}}%
\pgfpathlineto{\pgfqpoint{4.137512in}{0.755778in}}%
\pgfpathlineto{\pgfqpoint{4.138077in}{0.755770in}}%
\pgfpathlineto{\pgfqpoint{4.138641in}{0.755763in}}%
\pgfpathlineto{\pgfqpoint{4.139205in}{0.755755in}}%
\pgfpathlineto{\pgfqpoint{4.139770in}{0.755748in}}%
\pgfpathlineto{\pgfqpoint{4.140334in}{0.755741in}}%
\pgfpathlineto{\pgfqpoint{4.140898in}{0.755733in}}%
\pgfpathlineto{\pgfqpoint{4.141463in}{0.755726in}}%
\pgfpathlineto{\pgfqpoint{4.142027in}{0.755719in}}%
\pgfpathlineto{\pgfqpoint{4.142591in}{0.755711in}}%
\pgfpathlineto{\pgfqpoint{4.143156in}{0.755704in}}%
\pgfpathlineto{\pgfqpoint{4.143720in}{0.755696in}}%
\pgfpathlineto{\pgfqpoint{4.144284in}{0.755689in}}%
\pgfpathlineto{\pgfqpoint{4.144849in}{0.755682in}}%
\pgfpathlineto{\pgfqpoint{4.145413in}{0.755674in}}%
\pgfpathlineto{\pgfqpoint{4.145978in}{0.755667in}}%
\pgfpathlineto{\pgfqpoint{4.146542in}{0.755660in}}%
\pgfpathlineto{\pgfqpoint{4.147106in}{0.755652in}}%
\pgfpathlineto{\pgfqpoint{4.147671in}{0.755645in}}%
\pgfpathlineto{\pgfqpoint{4.148235in}{0.755638in}}%
\pgfpathlineto{\pgfqpoint{4.148799in}{0.755630in}}%
\pgfpathlineto{\pgfqpoint{4.149364in}{0.755623in}}%
\pgfpathlineto{\pgfqpoint{4.149928in}{0.755615in}}%
\pgfpathlineto{\pgfqpoint{4.150492in}{0.755608in}}%
\pgfpathlineto{\pgfqpoint{4.151057in}{0.755601in}}%
\pgfpathlineto{\pgfqpoint{4.151621in}{0.755593in}}%
\pgfpathlineto{\pgfqpoint{4.152185in}{0.755586in}}%
\pgfpathlineto{\pgfqpoint{4.152750in}{0.755579in}}%
\pgfpathlineto{\pgfqpoint{4.153314in}{0.755571in}}%
\pgfpathlineto{\pgfqpoint{4.153878in}{0.755564in}}%
\pgfpathlineto{\pgfqpoint{4.154443in}{0.755557in}}%
\pgfpathlineto{\pgfqpoint{4.155007in}{0.755549in}}%
\pgfpathlineto{\pgfqpoint{4.155571in}{0.755542in}}%
\pgfpathlineto{\pgfqpoint{4.156136in}{0.755534in}}%
\pgfpathlineto{\pgfqpoint{4.156700in}{0.755527in}}%
\pgfpathlineto{\pgfqpoint{4.157264in}{0.755520in}}%
\pgfpathlineto{\pgfqpoint{4.157829in}{0.755512in}}%
\pgfpathlineto{\pgfqpoint{4.158393in}{0.755505in}}%
\pgfpathlineto{\pgfqpoint{4.158957in}{0.755498in}}%
\pgfpathlineto{\pgfqpoint{4.159522in}{0.755490in}}%
\pgfpathlineto{\pgfqpoint{4.160086in}{0.755483in}}%
\pgfpathlineto{\pgfqpoint{4.160651in}{0.755475in}}%
\pgfpathlineto{\pgfqpoint{4.161215in}{0.755468in}}%
\pgfpathlineto{\pgfqpoint{4.161779in}{0.755461in}}%
\pgfpathlineto{\pgfqpoint{4.162344in}{0.755453in}}%
\pgfpathlineto{\pgfqpoint{4.162908in}{0.755446in}}%
\pgfpathlineto{\pgfqpoint{4.163472in}{0.755439in}}%
\pgfpathlineto{\pgfqpoint{4.164037in}{0.755431in}}%
\pgfpathlineto{\pgfqpoint{4.164601in}{0.755424in}}%
\pgfpathlineto{\pgfqpoint{4.165165in}{0.755417in}}%
\pgfpathlineto{\pgfqpoint{4.165730in}{0.755409in}}%
\pgfpathlineto{\pgfqpoint{4.166294in}{0.755402in}}%
\pgfpathlineto{\pgfqpoint{4.166858in}{0.755394in}}%
\pgfpathlineto{\pgfqpoint{4.167423in}{0.755387in}}%
\pgfpathlineto{\pgfqpoint{4.167987in}{0.755380in}}%
\pgfpathlineto{\pgfqpoint{4.168551in}{0.755372in}}%
\pgfpathlineto{\pgfqpoint{4.169116in}{0.755365in}}%
\pgfpathlineto{\pgfqpoint{4.169680in}{0.755358in}}%
\pgfpathlineto{\pgfqpoint{4.170244in}{0.755350in}}%
\pgfpathlineto{\pgfqpoint{4.170809in}{0.755343in}}%
\pgfpathlineto{\pgfqpoint{4.171373in}{0.755336in}}%
\pgfpathlineto{\pgfqpoint{4.171937in}{0.755328in}}%
\pgfpathlineto{\pgfqpoint{4.172502in}{0.755321in}}%
\pgfpathlineto{\pgfqpoint{4.173066in}{0.755313in}}%
\pgfpathlineto{\pgfqpoint{4.173630in}{0.755306in}}%
\pgfpathlineto{\pgfqpoint{4.174195in}{0.755299in}}%
\pgfpathlineto{\pgfqpoint{4.174759in}{0.755291in}}%
\pgfpathlineto{\pgfqpoint{4.175324in}{0.755284in}}%
\pgfpathlineto{\pgfqpoint{4.175888in}{0.755277in}}%
\pgfpathlineto{\pgfqpoint{4.176452in}{0.755269in}}%
\pgfpathlineto{\pgfqpoint{4.177017in}{0.755262in}}%
\pgfpathlineto{\pgfqpoint{4.177581in}{0.755254in}}%
\pgfpathlineto{\pgfqpoint{4.178145in}{0.755247in}}%
\pgfpathlineto{\pgfqpoint{4.178710in}{0.755240in}}%
\pgfpathlineto{\pgfqpoint{4.179274in}{0.755232in}}%
\pgfpathlineto{\pgfqpoint{4.179838in}{0.755225in}}%
\pgfpathlineto{\pgfqpoint{4.180403in}{0.755218in}}%
\pgfpathlineto{\pgfqpoint{4.180967in}{0.755210in}}%
\pgfpathlineto{\pgfqpoint{4.181531in}{0.755203in}}%
\pgfpathlineto{\pgfqpoint{4.182096in}{0.755196in}}%
\pgfpathlineto{\pgfqpoint{4.182660in}{0.755188in}}%
\pgfpathlineto{\pgfqpoint{4.183224in}{0.755181in}}%
\pgfpathlineto{\pgfqpoint{4.183789in}{0.755173in}}%
\pgfpathlineto{\pgfqpoint{4.184353in}{0.755166in}}%
\pgfpathlineto{\pgfqpoint{4.184917in}{0.755159in}}%
\pgfpathlineto{\pgfqpoint{4.185482in}{0.755151in}}%
\pgfpathlineto{\pgfqpoint{4.186046in}{0.755144in}}%
\pgfpathlineto{\pgfqpoint{4.186610in}{0.755137in}}%
\pgfpathlineto{\pgfqpoint{4.187175in}{0.755129in}}%
\pgfpathlineto{\pgfqpoint{4.187739in}{0.755122in}}%
\pgfpathlineto{\pgfqpoint{4.188303in}{0.755115in}}%
\pgfpathlineto{\pgfqpoint{4.188868in}{0.755107in}}%
\pgfpathlineto{\pgfqpoint{4.189432in}{0.755100in}}%
\pgfpathlineto{\pgfqpoint{4.189996in}{0.755092in}}%
\pgfpathlineto{\pgfqpoint{4.190561in}{0.755085in}}%
\pgfpathlineto{\pgfqpoint{4.191125in}{0.755078in}}%
\pgfpathlineto{\pgfqpoint{4.191690in}{0.755070in}}%
\pgfpathlineto{\pgfqpoint{4.192254in}{0.755063in}}%
\pgfpathlineto{\pgfqpoint{4.192818in}{0.755056in}}%
\pgfpathlineto{\pgfqpoint{4.193383in}{0.755048in}}%
\pgfpathlineto{\pgfqpoint{4.193947in}{0.755041in}}%
\pgfpathlineto{\pgfqpoint{4.194511in}{0.755033in}}%
\pgfpathlineto{\pgfqpoint{4.195076in}{0.755026in}}%
\pgfpathlineto{\pgfqpoint{4.195640in}{0.755019in}}%
\pgfpathlineto{\pgfqpoint{4.196204in}{0.755011in}}%
\pgfpathlineto{\pgfqpoint{4.196769in}{0.755004in}}%
\pgfpathlineto{\pgfqpoint{4.197333in}{0.754997in}}%
\pgfpathlineto{\pgfqpoint{4.197897in}{0.754989in}}%
\pgfpathlineto{\pgfqpoint{4.198462in}{0.754982in}}%
\pgfpathlineto{\pgfqpoint{4.199026in}{0.754975in}}%
\pgfpathlineto{\pgfqpoint{4.199590in}{0.754967in}}%
\pgfpathlineto{\pgfqpoint{4.200155in}{0.754960in}}%
\pgfpathlineto{\pgfqpoint{4.200719in}{0.754952in}}%
\pgfpathlineto{\pgfqpoint{4.201283in}{0.754945in}}%
\pgfpathlineto{\pgfqpoint{4.201848in}{0.754938in}}%
\pgfpathlineto{\pgfqpoint{4.202412in}{0.754930in}}%
\pgfpathlineto{\pgfqpoint{4.202976in}{0.754923in}}%
\pgfpathlineto{\pgfqpoint{4.203541in}{0.754916in}}%
\pgfpathlineto{\pgfqpoint{4.204105in}{0.754908in}}%
\pgfpathlineto{\pgfqpoint{4.204669in}{0.754901in}}%
\pgfpathlineto{\pgfqpoint{4.205234in}{0.754893in}}%
\pgfpathlineto{\pgfqpoint{4.205798in}{0.754886in}}%
\pgfpathlineto{\pgfqpoint{4.206363in}{0.754879in}}%
\pgfpathlineto{\pgfqpoint{4.206927in}{0.754871in}}%
\pgfpathlineto{\pgfqpoint{4.207491in}{0.754864in}}%
\pgfpathlineto{\pgfqpoint{4.208056in}{0.754857in}}%
\pgfpathlineto{\pgfqpoint{4.208620in}{0.754849in}}%
\pgfpathlineto{\pgfqpoint{4.209184in}{0.754842in}}%
\pgfpathlineto{\pgfqpoint{4.209749in}{0.754835in}}%
\pgfpathlineto{\pgfqpoint{4.210313in}{0.754827in}}%
\pgfpathlineto{\pgfqpoint{4.210877in}{0.754820in}}%
\pgfpathlineto{\pgfqpoint{4.211442in}{0.754812in}}%
\pgfpathlineto{\pgfqpoint{4.212006in}{0.754805in}}%
\pgfpathlineto{\pgfqpoint{4.212570in}{0.754798in}}%
\pgfpathlineto{\pgfqpoint{4.213135in}{0.754790in}}%
\pgfpathlineto{\pgfqpoint{4.213699in}{0.754783in}}%
\pgfpathlineto{\pgfqpoint{4.214263in}{0.754776in}}%
\pgfpathlineto{\pgfqpoint{4.214828in}{0.754768in}}%
\pgfpathlineto{\pgfqpoint{4.215392in}{0.754761in}}%
\pgfpathlineto{\pgfqpoint{4.215956in}{0.754754in}}%
\pgfpathlineto{\pgfqpoint{4.216521in}{0.754746in}}%
\pgfpathlineto{\pgfqpoint{4.217085in}{0.754739in}}%
\pgfpathlineto{\pgfqpoint{4.217649in}{0.754731in}}%
\pgfpathlineto{\pgfqpoint{4.218214in}{0.754724in}}%
\pgfpathlineto{\pgfqpoint{4.218778in}{0.754717in}}%
\pgfpathlineto{\pgfqpoint{4.219342in}{0.754709in}}%
\pgfpathlineto{\pgfqpoint{4.219907in}{0.754702in}}%
\pgfpathlineto{\pgfqpoint{4.220471in}{0.754695in}}%
\pgfpathlineto{\pgfqpoint{4.221036in}{0.754687in}}%
\pgfpathlineto{\pgfqpoint{4.221600in}{0.754680in}}%
\pgfpathlineto{\pgfqpoint{4.222164in}{0.754672in}}%
\pgfpathlineto{\pgfqpoint{4.222729in}{0.754665in}}%
\pgfpathlineto{\pgfqpoint{4.223293in}{0.754658in}}%
\pgfpathlineto{\pgfqpoint{4.223857in}{0.754650in}}%
\pgfpathlineto{\pgfqpoint{4.224422in}{0.754643in}}%
\pgfpathlineto{\pgfqpoint{4.224986in}{0.754636in}}%
\pgfpathlineto{\pgfqpoint{4.225550in}{0.754628in}}%
\pgfpathlineto{\pgfqpoint{4.226115in}{0.754621in}}%
\pgfpathlineto{\pgfqpoint{4.226679in}{0.754614in}}%
\pgfpathlineto{\pgfqpoint{4.227243in}{0.754606in}}%
\pgfpathlineto{\pgfqpoint{4.227808in}{0.754599in}}%
\pgfpathlineto{\pgfqpoint{4.228372in}{0.754591in}}%
\pgfpathlineto{\pgfqpoint{4.228936in}{0.754584in}}%
\pgfpathlineto{\pgfqpoint{4.229501in}{0.754577in}}%
\pgfpathlineto{\pgfqpoint{4.230065in}{0.754569in}}%
\pgfpathlineto{\pgfqpoint{4.230629in}{0.754562in}}%
\pgfpathlineto{\pgfqpoint{4.231194in}{0.754555in}}%
\pgfpathlineto{\pgfqpoint{4.231758in}{0.754547in}}%
\pgfpathlineto{\pgfqpoint{4.232322in}{0.754540in}}%
\pgfpathlineto{\pgfqpoint{4.232887in}{0.754533in}}%
\pgfpathlineto{\pgfqpoint{4.233451in}{0.754525in}}%
\pgfpathlineto{\pgfqpoint{4.234015in}{0.754518in}}%
\pgfpathlineto{\pgfqpoint{4.234580in}{0.754510in}}%
\pgfpathlineto{\pgfqpoint{4.235144in}{0.754503in}}%
\pgfpathlineto{\pgfqpoint{4.235708in}{0.754496in}}%
\pgfpathlineto{\pgfqpoint{4.236273in}{0.754488in}}%
\pgfpathlineto{\pgfqpoint{4.236837in}{0.754481in}}%
\pgfpathlineto{\pgfqpoint{4.237402in}{0.754474in}}%
\pgfpathlineto{\pgfqpoint{4.237966in}{0.754466in}}%
\pgfpathlineto{\pgfqpoint{4.238530in}{0.754459in}}%
\pgfpathlineto{\pgfqpoint{4.239095in}{0.754451in}}%
\pgfpathlineto{\pgfqpoint{4.239659in}{0.754444in}}%
\pgfpathlineto{\pgfqpoint{4.240223in}{0.754437in}}%
\pgfpathlineto{\pgfqpoint{4.240788in}{0.754429in}}%
\pgfpathlineto{\pgfqpoint{4.241352in}{0.754422in}}%
\pgfpathlineto{\pgfqpoint{4.241916in}{0.754415in}}%
\pgfpathlineto{\pgfqpoint{4.242481in}{0.754407in}}%
\pgfpathlineto{\pgfqpoint{4.243045in}{0.754400in}}%
\pgfpathlineto{\pgfqpoint{4.243609in}{0.754393in}}%
\pgfpathlineto{\pgfqpoint{4.244174in}{0.754385in}}%
\pgfpathlineto{\pgfqpoint{4.244738in}{0.754378in}}%
\pgfpathlineto{\pgfqpoint{4.245302in}{0.754370in}}%
\pgfpathlineto{\pgfqpoint{4.245867in}{0.754363in}}%
\pgfpathlineto{\pgfqpoint{4.246431in}{0.754356in}}%
\pgfpathlineto{\pgfqpoint{4.246995in}{0.754348in}}%
\pgfpathlineto{\pgfqpoint{4.247560in}{0.754341in}}%
\pgfpathlineto{\pgfqpoint{4.248124in}{0.754334in}}%
\pgfpathlineto{\pgfqpoint{4.248688in}{0.754326in}}%
\pgfpathlineto{\pgfqpoint{4.249253in}{0.754319in}}%
\pgfpathlineto{\pgfqpoint{4.249817in}{0.754311in}}%
\pgfpathlineto{\pgfqpoint{4.250381in}{0.754304in}}%
\pgfpathlineto{\pgfqpoint{4.250946in}{0.754297in}}%
\pgfpathlineto{\pgfqpoint{4.251510in}{0.754289in}}%
\pgfpathlineto{\pgfqpoint{4.252075in}{0.754282in}}%
\pgfpathlineto{\pgfqpoint{4.252639in}{0.754275in}}%
\pgfpathlineto{\pgfqpoint{4.253203in}{0.754267in}}%
\pgfpathlineto{\pgfqpoint{4.253768in}{0.754260in}}%
\pgfpathlineto{\pgfqpoint{4.254332in}{0.754253in}}%
\pgfpathlineto{\pgfqpoint{4.254896in}{0.754245in}}%
\pgfpathlineto{\pgfqpoint{4.255461in}{0.754238in}}%
\pgfpathlineto{\pgfqpoint{4.256025in}{0.754230in}}%
\pgfpathlineto{\pgfqpoint{4.256589in}{0.754223in}}%
\pgfpathlineto{\pgfqpoint{4.257154in}{0.754216in}}%
\pgfpathlineto{\pgfqpoint{4.257718in}{0.754208in}}%
\pgfpathlineto{\pgfqpoint{4.258282in}{0.754201in}}%
\pgfpathlineto{\pgfqpoint{4.258847in}{0.754194in}}%
\pgfpathlineto{\pgfqpoint{4.259411in}{0.754186in}}%
\pgfpathlineto{\pgfqpoint{4.259975in}{0.754179in}}%
\pgfpathlineto{\pgfqpoint{4.260540in}{0.754172in}}%
\pgfpathlineto{\pgfqpoint{4.261104in}{0.754164in}}%
\pgfpathlineto{\pgfqpoint{4.261668in}{0.754157in}}%
\pgfpathlineto{\pgfqpoint{4.262233in}{0.754149in}}%
\pgfpathlineto{\pgfqpoint{4.262797in}{0.754142in}}%
\pgfpathlineto{\pgfqpoint{4.263361in}{0.754135in}}%
\pgfpathlineto{\pgfqpoint{4.263926in}{0.754127in}}%
\pgfpathlineto{\pgfqpoint{4.264490in}{0.754120in}}%
\pgfpathlineto{\pgfqpoint{4.265054in}{0.754113in}}%
\pgfpathlineto{\pgfqpoint{4.265619in}{0.754105in}}%
\pgfpathlineto{\pgfqpoint{4.266183in}{0.754098in}}%
\pgfpathlineto{\pgfqpoint{4.266748in}{0.754090in}}%
\pgfpathlineto{\pgfqpoint{4.267312in}{0.754083in}}%
\pgfpathlineto{\pgfqpoint{4.267876in}{0.754076in}}%
\pgfpathlineto{\pgfqpoint{4.268441in}{0.754068in}}%
\pgfpathlineto{\pgfqpoint{4.269005in}{0.754061in}}%
\pgfpathlineto{\pgfqpoint{4.269569in}{0.754054in}}%
\pgfpathlineto{\pgfqpoint{4.270134in}{0.754046in}}%
\pgfpathlineto{\pgfqpoint{4.270698in}{0.754039in}}%
\pgfpathlineto{\pgfqpoint{4.271262in}{0.754032in}}%
\pgfpathlineto{\pgfqpoint{4.271827in}{0.754024in}}%
\pgfpathlineto{\pgfqpoint{4.272391in}{0.754017in}}%
\pgfpathlineto{\pgfqpoint{4.272955in}{0.754009in}}%
\pgfpathlineto{\pgfqpoint{4.273520in}{0.754002in}}%
\pgfpathlineto{\pgfqpoint{4.274084in}{0.753995in}}%
\pgfpathlineto{\pgfqpoint{4.274648in}{0.753987in}}%
\pgfpathlineto{\pgfqpoint{4.275213in}{0.753980in}}%
\pgfpathlineto{\pgfqpoint{4.275777in}{0.753973in}}%
\pgfpathlineto{\pgfqpoint{4.276341in}{0.753965in}}%
\pgfpathlineto{\pgfqpoint{4.276906in}{0.753958in}}%
\pgfpathlineto{\pgfqpoint{4.277470in}{0.753951in}}%
\pgfpathlineto{\pgfqpoint{4.278034in}{0.753943in}}%
\pgfpathlineto{\pgfqpoint{4.278599in}{0.753936in}}%
\pgfpathlineto{\pgfqpoint{4.279163in}{0.753928in}}%
\pgfpathlineto{\pgfqpoint{4.279727in}{0.753921in}}%
\pgfpathlineto{\pgfqpoint{4.280292in}{0.753914in}}%
\pgfpathlineto{\pgfqpoint{4.280856in}{0.753906in}}%
\pgfpathlineto{\pgfqpoint{4.281421in}{0.753899in}}%
\pgfpathlineto{\pgfqpoint{4.281985in}{0.753892in}}%
\pgfpathlineto{\pgfqpoint{4.282549in}{0.753884in}}%
\pgfpathlineto{\pgfqpoint{4.283114in}{0.753877in}}%
\pgfpathlineto{\pgfqpoint{4.283678in}{0.753869in}}%
\pgfpathlineto{\pgfqpoint{4.284242in}{0.753862in}}%
\pgfpathlineto{\pgfqpoint{4.284807in}{0.753855in}}%
\pgfpathlineto{\pgfqpoint{4.285371in}{0.753847in}}%
\pgfpathlineto{\pgfqpoint{4.285935in}{0.753840in}}%
\pgfpathlineto{\pgfqpoint{4.286500in}{0.753833in}}%
\pgfpathlineto{\pgfqpoint{4.287064in}{0.753825in}}%
\pgfpathlineto{\pgfqpoint{4.287628in}{0.753818in}}%
\pgfpathlineto{\pgfqpoint{4.288193in}{0.753811in}}%
\pgfpathlineto{\pgfqpoint{4.288757in}{0.753803in}}%
\pgfpathlineto{\pgfqpoint{4.289321in}{0.753796in}}%
\pgfpathlineto{\pgfqpoint{4.289886in}{0.753788in}}%
\pgfpathlineto{\pgfqpoint{4.290450in}{0.753781in}}%
\pgfpathlineto{\pgfqpoint{4.291014in}{0.753774in}}%
\pgfpathlineto{\pgfqpoint{4.291579in}{0.753766in}}%
\pgfpathlineto{\pgfqpoint{4.292143in}{0.753759in}}%
\pgfpathlineto{\pgfqpoint{4.292707in}{0.753752in}}%
\pgfpathlineto{\pgfqpoint{4.293272in}{0.753744in}}%
\pgfpathlineto{\pgfqpoint{4.293836in}{0.753737in}}%
\pgfpathlineto{\pgfqpoint{4.294400in}{0.753730in}}%
\pgfpathlineto{\pgfqpoint{4.294965in}{0.753722in}}%
\pgfpathlineto{\pgfqpoint{4.295529in}{0.753715in}}%
\pgfpathlineto{\pgfqpoint{4.296093in}{0.753707in}}%
\pgfpathlineto{\pgfqpoint{4.296658in}{0.753700in}}%
\pgfpathlineto{\pgfqpoint{4.297222in}{0.753693in}}%
\pgfpathlineto{\pgfqpoint{4.297787in}{0.753685in}}%
\pgfpathlineto{\pgfqpoint{4.298351in}{0.753678in}}%
\pgfpathlineto{\pgfqpoint{4.298915in}{0.753671in}}%
\pgfpathlineto{\pgfqpoint{4.299480in}{0.753663in}}%
\pgfpathlineto{\pgfqpoint{4.300044in}{0.753656in}}%
\pgfpathlineto{\pgfqpoint{4.300608in}{0.753648in}}%
\pgfpathlineto{\pgfqpoint{4.301173in}{0.753641in}}%
\pgfpathlineto{\pgfqpoint{4.301737in}{0.753634in}}%
\pgfpathlineto{\pgfqpoint{4.302301in}{0.753626in}}%
\pgfpathlineto{\pgfqpoint{4.302866in}{0.753619in}}%
\pgfpathlineto{\pgfqpoint{4.303430in}{0.753612in}}%
\pgfpathlineto{\pgfqpoint{4.303994in}{0.753604in}}%
\pgfpathlineto{\pgfqpoint{4.304559in}{0.753597in}}%
\pgfpathlineto{\pgfqpoint{4.305123in}{0.753590in}}%
\pgfpathlineto{\pgfqpoint{4.305687in}{0.753582in}}%
\pgfpathlineto{\pgfqpoint{4.306252in}{0.753575in}}%
\pgfpathlineto{\pgfqpoint{4.306816in}{0.753567in}}%
\pgfpathlineto{\pgfqpoint{4.307380in}{0.753560in}}%
\pgfpathlineto{\pgfqpoint{4.307945in}{0.753553in}}%
\pgfpathlineto{\pgfqpoint{4.308509in}{0.753545in}}%
\pgfpathlineto{\pgfqpoint{4.309073in}{0.753538in}}%
\pgfpathlineto{\pgfqpoint{4.309638in}{0.753531in}}%
\pgfpathlineto{\pgfqpoint{4.310202in}{0.753523in}}%
\pgfpathlineto{\pgfqpoint{4.310766in}{0.753516in}}%
\pgfpathlineto{\pgfqpoint{4.311331in}{0.753477in}}%
\pgfpathlineto{\pgfqpoint{4.311895in}{0.753315in}}%
\pgfpathlineto{\pgfqpoint{4.312460in}{0.753140in}}%
\pgfpathlineto{\pgfqpoint{4.313024in}{0.752965in}}%
\pgfpathlineto{\pgfqpoint{4.313588in}{0.752790in}}%
\pgfpathlineto{\pgfqpoint{4.314153in}{0.752615in}}%
\pgfpathlineto{\pgfqpoint{4.314717in}{0.752440in}}%
\pgfpathlineto{\pgfqpoint{4.315281in}{0.752265in}}%
\pgfpathlineto{\pgfqpoint{4.315846in}{0.752090in}}%
\pgfpathlineto{\pgfqpoint{4.316410in}{0.751915in}}%
\pgfpathlineto{\pgfqpoint{4.316974in}{0.751740in}}%
\pgfpathlineto{\pgfqpoint{4.317539in}{0.751564in}}%
\pgfpathlineto{\pgfqpoint{4.318103in}{0.751389in}}%
\pgfpathlineto{\pgfqpoint{4.318667in}{0.751214in}}%
\pgfpathlineto{\pgfqpoint{4.319232in}{0.751039in}}%
\pgfpathlineto{\pgfqpoint{4.319796in}{0.750864in}}%
\pgfpathlineto{\pgfqpoint{4.320360in}{0.750689in}}%
\pgfpathlineto{\pgfqpoint{4.320925in}{0.750514in}}%
\pgfpathlineto{\pgfqpoint{4.321489in}{0.750339in}}%
\pgfpathlineto{\pgfqpoint{4.322053in}{0.750164in}}%
\pgfpathlineto{\pgfqpoint{4.322618in}{0.749989in}}%
\pgfpathlineto{\pgfqpoint{4.323182in}{0.749814in}}%
\pgfpathlineto{\pgfqpoint{4.323746in}{0.749639in}}%
\pgfpathlineto{\pgfqpoint{4.324311in}{0.749463in}}%
\pgfpathlineto{\pgfqpoint{4.324875in}{0.749288in}}%
\pgfpathlineto{\pgfqpoint{4.325439in}{0.749113in}}%
\pgfpathlineto{\pgfqpoint{4.326004in}{0.748938in}}%
\pgfpathlineto{\pgfqpoint{4.326568in}{0.748763in}}%
\pgfpathlineto{\pgfqpoint{4.327133in}{0.748588in}}%
\pgfpathlineto{\pgfqpoint{4.327697in}{0.748421in}}%
\pgfpathlineto{\pgfqpoint{4.328261in}{0.748377in}}%
\pgfpathlineto{\pgfqpoint{4.328826in}{0.748373in}}%
\pgfpathlineto{\pgfqpoint{4.329390in}{0.748370in}}%
\pgfpathlineto{\pgfqpoint{4.329954in}{0.748367in}}%
\pgfpathlineto{\pgfqpoint{4.330519in}{0.748363in}}%
\pgfpathlineto{\pgfqpoint{4.331083in}{0.748360in}}%
\pgfpathlineto{\pgfqpoint{4.331647in}{0.748357in}}%
\pgfpathlineto{\pgfqpoint{4.332212in}{0.748353in}}%
\pgfpathlineto{\pgfqpoint{4.332776in}{0.748350in}}%
\pgfpathlineto{\pgfqpoint{4.333340in}{0.748347in}}%
\pgfpathlineto{\pgfqpoint{4.333905in}{0.748343in}}%
\pgfpathlineto{\pgfqpoint{4.334469in}{0.748340in}}%
\pgfpathlineto{\pgfqpoint{4.335033in}{0.748336in}}%
\pgfpathlineto{\pgfqpoint{4.335598in}{0.748333in}}%
\pgfpathlineto{\pgfqpoint{4.336162in}{0.748330in}}%
\pgfpathlineto{\pgfqpoint{4.336726in}{0.748326in}}%
\pgfpathlineto{\pgfqpoint{4.337291in}{0.748323in}}%
\pgfpathlineto{\pgfqpoint{4.337855in}{0.748320in}}%
\pgfpathlineto{\pgfqpoint{4.338419in}{0.748316in}}%
\pgfpathlineto{\pgfqpoint{4.338984in}{0.748313in}}%
\pgfpathlineto{\pgfqpoint{4.339548in}{0.748310in}}%
\pgfpathlineto{\pgfqpoint{4.340112in}{0.748306in}}%
\pgfpathlineto{\pgfqpoint{4.340677in}{0.748303in}}%
\pgfpathlineto{\pgfqpoint{4.341241in}{0.748300in}}%
\pgfpathlineto{\pgfqpoint{4.341805in}{0.748296in}}%
\pgfpathlineto{\pgfqpoint{4.342370in}{0.748293in}}%
\pgfpathlineto{\pgfqpoint{4.342934in}{0.748290in}}%
\pgfpathlineto{\pgfqpoint{4.343499in}{0.748286in}}%
\pgfpathlineto{\pgfqpoint{4.344063in}{0.748283in}}%
\pgfpathlineto{\pgfqpoint{4.344627in}{0.748279in}}%
\pgfpathlineto{\pgfqpoint{4.345192in}{0.748276in}}%
\pgfpathlineto{\pgfqpoint{4.345756in}{0.748273in}}%
\pgfpathlineto{\pgfqpoint{4.346320in}{0.748269in}}%
\pgfpathlineto{\pgfqpoint{4.346885in}{0.748266in}}%
\pgfpathlineto{\pgfqpoint{4.347449in}{0.748263in}}%
\pgfpathlineto{\pgfqpoint{4.348013in}{0.748259in}}%
\pgfpathlineto{\pgfqpoint{4.348578in}{0.748256in}}%
\pgfpathlineto{\pgfqpoint{4.349142in}{0.748253in}}%
\pgfpathlineto{\pgfqpoint{4.349706in}{0.748249in}}%
\pgfpathlineto{\pgfqpoint{4.350271in}{0.748246in}}%
\pgfpathlineto{\pgfqpoint{4.350835in}{0.748243in}}%
\pgfpathlineto{\pgfqpoint{4.351399in}{0.748239in}}%
\pgfpathlineto{\pgfqpoint{4.351964in}{0.748236in}}%
\pgfpathlineto{\pgfqpoint{4.352528in}{0.748233in}}%
\pgfpathlineto{\pgfqpoint{4.353092in}{0.748229in}}%
\pgfpathlineto{\pgfqpoint{4.353657in}{0.748226in}}%
\pgfpathlineto{\pgfqpoint{4.354221in}{0.748222in}}%
\pgfpathlineto{\pgfqpoint{4.354785in}{0.748219in}}%
\pgfpathlineto{\pgfqpoint{4.355350in}{0.748216in}}%
\pgfpathlineto{\pgfqpoint{4.355914in}{0.748212in}}%
\pgfpathlineto{\pgfqpoint{4.356478in}{0.748209in}}%
\pgfpathlineto{\pgfqpoint{4.357043in}{0.748206in}}%
\pgfpathlineto{\pgfqpoint{4.357607in}{0.748202in}}%
\pgfpathlineto{\pgfqpoint{4.358172in}{0.748199in}}%
\pgfpathlineto{\pgfqpoint{4.358736in}{0.748196in}}%
\pgfpathlineto{\pgfqpoint{4.359300in}{0.748192in}}%
\pgfpathlineto{\pgfqpoint{4.359865in}{0.748189in}}%
\pgfpathlineto{\pgfqpoint{4.360429in}{0.748186in}}%
\pgfpathlineto{\pgfqpoint{4.360993in}{0.748182in}}%
\pgfpathlineto{\pgfqpoint{4.361558in}{0.748179in}}%
\pgfpathlineto{\pgfqpoint{4.362122in}{0.748176in}}%
\pgfpathlineto{\pgfqpoint{4.362686in}{0.748172in}}%
\pgfpathlineto{\pgfqpoint{4.363251in}{0.748169in}}%
\pgfpathlineto{\pgfqpoint{4.363815in}{0.748166in}}%
\pgfpathlineto{\pgfqpoint{4.364379in}{0.748162in}}%
\pgfpathlineto{\pgfqpoint{4.364944in}{0.748159in}}%
\pgfpathlineto{\pgfqpoint{4.365508in}{0.748155in}}%
\pgfpathlineto{\pgfqpoint{4.366072in}{0.748152in}}%
\pgfpathlineto{\pgfqpoint{4.366637in}{0.748149in}}%
\pgfpathlineto{\pgfqpoint{4.367201in}{0.748145in}}%
\pgfpathlineto{\pgfqpoint{4.367765in}{0.748142in}}%
\pgfpathlineto{\pgfqpoint{4.368330in}{0.748139in}}%
\pgfpathlineto{\pgfqpoint{4.368894in}{0.748135in}}%
\pgfpathlineto{\pgfqpoint{4.369458in}{0.748132in}}%
\pgfpathlineto{\pgfqpoint{4.370023in}{0.748129in}}%
\pgfpathlineto{\pgfqpoint{4.370587in}{0.748125in}}%
\pgfpathlineto{\pgfqpoint{4.371151in}{0.748122in}}%
\pgfpathlineto{\pgfqpoint{4.371716in}{0.748119in}}%
\pgfpathlineto{\pgfqpoint{4.372280in}{0.748115in}}%
\pgfpathlineto{\pgfqpoint{4.372845in}{0.748112in}}%
\pgfpathlineto{\pgfqpoint{4.373409in}{0.748109in}}%
\pgfpathlineto{\pgfqpoint{4.373973in}{0.748105in}}%
\pgfpathlineto{\pgfqpoint{4.374538in}{0.748102in}}%
\pgfpathlineto{\pgfqpoint{4.375102in}{0.748098in}}%
\pgfpathlineto{\pgfqpoint{4.375666in}{0.748095in}}%
\pgfpathlineto{\pgfqpoint{4.376231in}{0.748092in}}%
\pgfpathlineto{\pgfqpoint{4.376795in}{0.748088in}}%
\pgfpathlineto{\pgfqpoint{4.377359in}{0.748085in}}%
\pgfpathlineto{\pgfqpoint{4.377924in}{0.748082in}}%
\pgfpathlineto{\pgfqpoint{4.378488in}{0.748078in}}%
\pgfpathlineto{\pgfqpoint{4.379052in}{0.748075in}}%
\pgfpathlineto{\pgfqpoint{4.379617in}{0.748072in}}%
\pgfpathlineto{\pgfqpoint{4.380181in}{0.748068in}}%
\pgfpathlineto{\pgfqpoint{4.380745in}{0.748065in}}%
\pgfpathlineto{\pgfqpoint{4.381310in}{0.748062in}}%
\pgfpathlineto{\pgfqpoint{4.381874in}{0.748058in}}%
\pgfpathlineto{\pgfqpoint{4.382438in}{0.748055in}}%
\pgfpathlineto{\pgfqpoint{4.383003in}{0.748052in}}%
\pgfpathlineto{\pgfqpoint{4.383567in}{0.748048in}}%
\pgfpathlineto{\pgfqpoint{4.384131in}{0.748045in}}%
\pgfpathlineto{\pgfqpoint{4.384696in}{0.748041in}}%
\pgfpathlineto{\pgfqpoint{4.385260in}{0.748038in}}%
\pgfpathlineto{\pgfqpoint{4.385824in}{0.748035in}}%
\pgfpathlineto{\pgfqpoint{4.386389in}{0.748031in}}%
\pgfpathlineto{\pgfqpoint{4.386953in}{0.748028in}}%
\pgfpathlineto{\pgfqpoint{4.387517in}{0.748025in}}%
\pgfpathlineto{\pgfqpoint{4.388082in}{0.748021in}}%
\pgfpathlineto{\pgfqpoint{4.388646in}{0.748018in}}%
\pgfpathlineto{\pgfqpoint{4.389211in}{0.748015in}}%
\pgfpathlineto{\pgfqpoint{4.389775in}{0.748011in}}%
\pgfpathlineto{\pgfqpoint{4.390339in}{0.748008in}}%
\pgfpathlineto{\pgfqpoint{4.390904in}{0.748005in}}%
\pgfpathlineto{\pgfqpoint{4.391468in}{0.748001in}}%
\pgfpathlineto{\pgfqpoint{4.392032in}{0.747998in}}%
\pgfpathlineto{\pgfqpoint{4.392597in}{0.747995in}}%
\pgfpathlineto{\pgfqpoint{4.393161in}{0.747991in}}%
\pgfpathlineto{\pgfqpoint{4.393725in}{0.747988in}}%
\pgfpathlineto{\pgfqpoint{4.394290in}{0.747985in}}%
\pgfpathlineto{\pgfqpoint{4.394854in}{0.747981in}}%
\pgfpathlineto{\pgfqpoint{4.395418in}{0.747978in}}%
\pgfpathlineto{\pgfqpoint{4.395983in}{0.747974in}}%
\pgfpathlineto{\pgfqpoint{4.396547in}{0.747971in}}%
\pgfpathlineto{\pgfqpoint{4.397111in}{0.747968in}}%
\pgfpathlineto{\pgfqpoint{4.397676in}{0.747964in}}%
\pgfpathlineto{\pgfqpoint{4.398240in}{0.747961in}}%
\pgfpathlineto{\pgfqpoint{4.398804in}{0.747958in}}%
\pgfpathlineto{\pgfqpoint{4.399369in}{0.747954in}}%
\pgfpathlineto{\pgfqpoint{4.399933in}{0.747951in}}%
\pgfpathlineto{\pgfqpoint{4.400497in}{0.747948in}}%
\pgfpathlineto{\pgfqpoint{4.401062in}{0.747944in}}%
\pgfpathlineto{\pgfqpoint{4.401626in}{0.747941in}}%
\pgfpathlineto{\pgfqpoint{4.402190in}{0.747938in}}%
\pgfpathlineto{\pgfqpoint{4.402755in}{0.747934in}}%
\pgfpathlineto{\pgfqpoint{4.403319in}{0.747931in}}%
\pgfpathlineto{\pgfqpoint{4.403884in}{0.747928in}}%
\pgfpathlineto{\pgfqpoint{4.404448in}{0.747924in}}%
\pgfpathlineto{\pgfqpoint{4.405012in}{0.747921in}}%
\pgfpathlineto{\pgfqpoint{4.405577in}{0.747917in}}%
\pgfpathlineto{\pgfqpoint{4.406141in}{0.747914in}}%
\pgfpathlineto{\pgfqpoint{4.406705in}{0.747911in}}%
\pgfpathlineto{\pgfqpoint{4.407270in}{0.747907in}}%
\pgfpathlineto{\pgfqpoint{4.407834in}{0.747904in}}%
\pgfpathlineto{\pgfqpoint{4.408398in}{0.747901in}}%
\pgfpathlineto{\pgfqpoint{4.408963in}{0.747897in}}%
\pgfpathlineto{\pgfqpoint{4.409527in}{0.747894in}}%
\pgfpathlineto{\pgfqpoint{4.410091in}{0.747891in}}%
\pgfpathlineto{\pgfqpoint{4.410656in}{0.747887in}}%
\pgfpathlineto{\pgfqpoint{4.411220in}{0.747884in}}%
\pgfpathlineto{\pgfqpoint{4.411784in}{0.747881in}}%
\pgfpathlineto{\pgfqpoint{4.412349in}{0.747877in}}%
\pgfpathlineto{\pgfqpoint{4.412913in}{0.747874in}}%
\pgfpathlineto{\pgfqpoint{4.413477in}{0.747871in}}%
\pgfpathlineto{\pgfqpoint{4.414042in}{0.747867in}}%
\pgfpathlineto{\pgfqpoint{4.414606in}{0.747864in}}%
\pgfpathlineto{\pgfqpoint{4.415170in}{0.747866in}}%
\pgfpathlineto{\pgfqpoint{4.415735in}{0.747913in}}%
\pgfpathlineto{\pgfqpoint{4.416299in}{0.747954in}}%
\pgfpathlineto{\pgfqpoint{4.416863in}{0.747951in}}%
\pgfpathlineto{\pgfqpoint{4.417428in}{0.747945in}}%
\pgfpathlineto{\pgfqpoint{4.417992in}{0.747941in}}%
\pgfpathlineto{\pgfqpoint{4.418557in}{0.747938in}}%
\pgfpathlineto{\pgfqpoint{4.419121in}{0.747934in}}%
\pgfpathlineto{\pgfqpoint{4.419685in}{0.747931in}}%
\pgfpathlineto{\pgfqpoint{4.420250in}{0.747928in}}%
\pgfpathlineto{\pgfqpoint{4.420814in}{0.747924in}}%
\pgfpathlineto{\pgfqpoint{4.421378in}{0.747921in}}%
\pgfpathlineto{\pgfqpoint{4.421943in}{0.747918in}}%
\pgfpathlineto{\pgfqpoint{4.422507in}{0.747915in}}%
\pgfpathlineto{\pgfqpoint{4.423071in}{0.747911in}}%
\pgfpathlineto{\pgfqpoint{4.423636in}{0.747908in}}%
\pgfpathlineto{\pgfqpoint{4.424200in}{0.747905in}}%
\pgfpathlineto{\pgfqpoint{4.424764in}{0.747901in}}%
\pgfpathlineto{\pgfqpoint{4.425329in}{0.747898in}}%
\pgfpathlineto{\pgfqpoint{4.425893in}{0.747895in}}%
\pgfpathlineto{\pgfqpoint{4.426457in}{0.747892in}}%
\pgfpathlineto{\pgfqpoint{4.427022in}{0.747888in}}%
\pgfpathlineto{\pgfqpoint{4.427586in}{0.747885in}}%
\pgfpathlineto{\pgfqpoint{4.428150in}{0.747882in}}%
\pgfpathlineto{\pgfqpoint{4.428715in}{0.747879in}}%
\pgfpathlineto{\pgfqpoint{4.429279in}{0.747875in}}%
\pgfpathlineto{\pgfqpoint{4.429843in}{0.747872in}}%
\pgfpathlineto{\pgfqpoint{4.430408in}{0.747869in}}%
\pgfpathlineto{\pgfqpoint{4.430972in}{0.747865in}}%
\pgfpathlineto{\pgfqpoint{4.431536in}{0.747862in}}%
\pgfpathlineto{\pgfqpoint{4.432101in}{0.747859in}}%
\pgfpathlineto{\pgfqpoint{4.432665in}{0.747856in}}%
\pgfpathlineto{\pgfqpoint{4.433229in}{0.747852in}}%
\pgfpathlineto{\pgfqpoint{4.433794in}{0.747849in}}%
\pgfpathlineto{\pgfqpoint{4.434358in}{0.747846in}}%
\pgfpathlineto{\pgfqpoint{4.434923in}{0.747842in}}%
\pgfpathlineto{\pgfqpoint{4.435487in}{0.747839in}}%
\pgfpathlineto{\pgfqpoint{4.436051in}{0.747836in}}%
\pgfpathlineto{\pgfqpoint{4.436616in}{0.747833in}}%
\pgfpathlineto{\pgfqpoint{4.437180in}{0.747829in}}%
\pgfpathlineto{\pgfqpoint{4.437744in}{0.747826in}}%
\pgfpathlineto{\pgfqpoint{4.438309in}{0.747823in}}%
\pgfpathlineto{\pgfqpoint{4.438873in}{0.747819in}}%
\pgfpathlineto{\pgfqpoint{4.439437in}{0.747816in}}%
\pgfpathlineto{\pgfqpoint{4.440002in}{0.747813in}}%
\pgfpathlineto{\pgfqpoint{4.440566in}{0.747810in}}%
\pgfpathlineto{\pgfqpoint{4.441130in}{0.747806in}}%
\pgfpathlineto{\pgfqpoint{4.441695in}{0.747803in}}%
\pgfpathlineto{\pgfqpoint{4.442259in}{0.747800in}}%
\pgfpathlineto{\pgfqpoint{4.442823in}{0.747797in}}%
\pgfpathlineto{\pgfqpoint{4.443388in}{0.747793in}}%
\pgfpathlineto{\pgfqpoint{4.443952in}{0.747790in}}%
\pgfpathlineto{\pgfqpoint{4.444516in}{0.747787in}}%
\pgfpathlineto{\pgfqpoint{4.445081in}{0.747783in}}%
\pgfpathlineto{\pgfqpoint{4.445645in}{0.747780in}}%
\pgfpathlineto{\pgfqpoint{4.446209in}{0.747777in}}%
\pgfpathlineto{\pgfqpoint{4.446774in}{0.747774in}}%
\pgfpathlineto{\pgfqpoint{4.447338in}{0.747770in}}%
\pgfpathlineto{\pgfqpoint{4.447902in}{0.747767in}}%
\pgfpathlineto{\pgfqpoint{4.448467in}{0.747764in}}%
\pgfpathlineto{\pgfqpoint{4.449031in}{0.747760in}}%
\pgfpathlineto{\pgfqpoint{4.449596in}{0.747757in}}%
\pgfpathlineto{\pgfqpoint{4.450160in}{0.747754in}}%
\pgfpathlineto{\pgfqpoint{4.450724in}{0.747751in}}%
\pgfpathlineto{\pgfqpoint{4.451289in}{0.747747in}}%
\pgfpathlineto{\pgfqpoint{4.451853in}{0.747744in}}%
\pgfpathlineto{\pgfqpoint{4.452417in}{0.747741in}}%
\pgfpathlineto{\pgfqpoint{4.452982in}{0.747737in}}%
\pgfpathlineto{\pgfqpoint{4.453546in}{0.747734in}}%
\pgfpathlineto{\pgfqpoint{4.454110in}{0.747731in}}%
\pgfpathlineto{\pgfqpoint{4.454675in}{0.747728in}}%
\pgfpathlineto{\pgfqpoint{4.455239in}{0.747724in}}%
\pgfpathlineto{\pgfqpoint{4.455803in}{0.747721in}}%
\pgfpathlineto{\pgfqpoint{4.456368in}{0.747718in}}%
\pgfpathlineto{\pgfqpoint{4.456932in}{0.747714in}}%
\pgfpathlineto{\pgfqpoint{4.457496in}{0.747711in}}%
\pgfpathlineto{\pgfqpoint{4.458061in}{0.747708in}}%
\pgfpathlineto{\pgfqpoint{4.458625in}{0.747705in}}%
\pgfpathlineto{\pgfqpoint{4.459189in}{0.747701in}}%
\pgfpathlineto{\pgfqpoint{4.459754in}{0.747698in}}%
\pgfpathlineto{\pgfqpoint{4.460318in}{0.747695in}}%
\pgfpathlineto{\pgfqpoint{4.460882in}{0.747692in}}%
\pgfpathlineto{\pgfqpoint{4.461447in}{0.747688in}}%
\pgfpathlineto{\pgfqpoint{4.462011in}{0.747685in}}%
\pgfpathlineto{\pgfqpoint{4.462575in}{0.747682in}}%
\pgfpathlineto{\pgfqpoint{4.463140in}{0.747678in}}%
\pgfpathlineto{\pgfqpoint{4.463704in}{0.747675in}}%
\pgfpathlineto{\pgfqpoint{4.464269in}{0.747672in}}%
\pgfpathlineto{\pgfqpoint{4.464833in}{0.747669in}}%
\pgfpathlineto{\pgfqpoint{4.465397in}{0.747665in}}%
\pgfpathlineto{\pgfqpoint{4.465962in}{0.747662in}}%
\pgfpathlineto{\pgfqpoint{4.466526in}{0.747659in}}%
\pgfpathlineto{\pgfqpoint{4.467090in}{0.747655in}}%
\pgfpathlineto{\pgfqpoint{4.467655in}{0.747652in}}%
\pgfpathlineto{\pgfqpoint{4.468219in}{0.747649in}}%
\pgfpathlineto{\pgfqpoint{4.468783in}{0.747646in}}%
\pgfpathlineto{\pgfqpoint{4.469348in}{0.747642in}}%
\pgfpathlineto{\pgfqpoint{4.469912in}{0.747639in}}%
\pgfpathlineto{\pgfqpoint{4.470476in}{0.747636in}}%
\pgfpathlineto{\pgfqpoint{4.471041in}{0.747632in}}%
\pgfpathlineto{\pgfqpoint{4.471605in}{0.747629in}}%
\pgfpathlineto{\pgfqpoint{4.472169in}{0.747626in}}%
\pgfpathlineto{\pgfqpoint{4.472734in}{0.747623in}}%
\pgfpathlineto{\pgfqpoint{4.473298in}{0.747619in}}%
\pgfpathlineto{\pgfqpoint{4.473862in}{0.747616in}}%
\pgfpathlineto{\pgfqpoint{4.474427in}{0.747613in}}%
\pgfpathlineto{\pgfqpoint{4.474991in}{0.747609in}}%
\pgfpathlineto{\pgfqpoint{4.475555in}{0.747606in}}%
\pgfpathlineto{\pgfqpoint{4.476120in}{0.747603in}}%
\pgfpathlineto{\pgfqpoint{4.476684in}{0.747600in}}%
\pgfpathlineto{\pgfqpoint{4.477248in}{0.747596in}}%
\pgfpathlineto{\pgfqpoint{4.477813in}{0.747593in}}%
\pgfpathlineto{\pgfqpoint{4.478377in}{0.747590in}}%
\pgfpathlineto{\pgfqpoint{4.478941in}{0.747587in}}%
\pgfpathlineto{\pgfqpoint{4.479506in}{0.747583in}}%
\pgfpathlineto{\pgfqpoint{4.480070in}{0.747580in}}%
\pgfpathlineto{\pgfqpoint{4.480635in}{0.747577in}}%
\pgfpathlineto{\pgfqpoint{4.481199in}{0.747573in}}%
\pgfpathlineto{\pgfqpoint{4.481763in}{0.747570in}}%
\pgfpathlineto{\pgfqpoint{4.482328in}{0.747567in}}%
\pgfpathlineto{\pgfqpoint{4.482892in}{0.747564in}}%
\pgfpathlineto{\pgfqpoint{4.483456in}{0.747560in}}%
\pgfpathlineto{\pgfqpoint{4.484021in}{0.747557in}}%
\pgfpathlineto{\pgfqpoint{4.484585in}{0.747554in}}%
\pgfpathlineto{\pgfqpoint{4.485149in}{0.747550in}}%
\pgfpathlineto{\pgfqpoint{4.485714in}{0.747547in}}%
\pgfpathlineto{\pgfqpoint{4.486278in}{0.747544in}}%
\pgfpathlineto{\pgfqpoint{4.486842in}{0.747541in}}%
\pgfpathlineto{\pgfqpoint{4.487407in}{0.747537in}}%
\pgfpathlineto{\pgfqpoint{4.487971in}{0.747534in}}%
\pgfpathlineto{\pgfqpoint{4.488535in}{0.747531in}}%
\pgfpathlineto{\pgfqpoint{4.489100in}{0.747527in}}%
\pgfpathlineto{\pgfqpoint{4.489664in}{0.747524in}}%
\pgfpathlineto{\pgfqpoint{4.490228in}{0.747521in}}%
\pgfpathlineto{\pgfqpoint{4.490793in}{0.747518in}}%
\pgfpathlineto{\pgfqpoint{4.491357in}{0.747514in}}%
\pgfpathlineto{\pgfqpoint{4.491921in}{0.747511in}}%
\pgfpathlineto{\pgfqpoint{4.492486in}{0.747508in}}%
\pgfpathlineto{\pgfqpoint{4.493050in}{0.747505in}}%
\pgfpathlineto{\pgfqpoint{4.493614in}{0.747501in}}%
\pgfpathlineto{\pgfqpoint{4.494179in}{0.747498in}}%
\pgfpathlineto{\pgfqpoint{4.494743in}{0.747495in}}%
\pgfpathlineto{\pgfqpoint{4.495308in}{0.747491in}}%
\pgfpathlineto{\pgfqpoint{4.495872in}{0.747488in}}%
\pgfpathlineto{\pgfqpoint{4.496436in}{0.747485in}}%
\pgfpathlineto{\pgfqpoint{4.497001in}{0.747482in}}%
\pgfpathlineto{\pgfqpoint{4.497565in}{0.747478in}}%
\pgfpathlineto{\pgfqpoint{4.498129in}{0.747475in}}%
\pgfpathlineto{\pgfqpoint{4.498694in}{0.747472in}}%
\pgfpathlineto{\pgfqpoint{4.499258in}{0.747468in}}%
\pgfpathlineto{\pgfqpoint{4.499822in}{0.747465in}}%
\pgfpathlineto{\pgfqpoint{4.500387in}{0.747462in}}%
\pgfpathlineto{\pgfqpoint{4.500951in}{0.747459in}}%
\pgfpathlineto{\pgfqpoint{4.501515in}{0.747455in}}%
\pgfpathlineto{\pgfqpoint{4.502080in}{0.747452in}}%
\pgfpathlineto{\pgfqpoint{4.502644in}{0.747449in}}%
\pgfpathlineto{\pgfqpoint{4.503208in}{0.747445in}}%
\pgfpathlineto{\pgfqpoint{4.503773in}{0.747442in}}%
\pgfpathlineto{\pgfqpoint{4.504337in}{0.747439in}}%
\pgfpathlineto{\pgfqpoint{4.504901in}{0.747436in}}%
\pgfpathlineto{\pgfqpoint{4.505466in}{0.747432in}}%
\pgfpathlineto{\pgfqpoint{4.506030in}{0.747429in}}%
\pgfpathlineto{\pgfqpoint{4.506594in}{0.747426in}}%
\pgfpathlineto{\pgfqpoint{4.507159in}{0.747422in}}%
\pgfpathlineto{\pgfqpoint{4.507723in}{0.747419in}}%
\pgfpathlineto{\pgfqpoint{4.508287in}{0.747416in}}%
\pgfpathlineto{\pgfqpoint{4.508852in}{0.747413in}}%
\pgfpathlineto{\pgfqpoint{4.509416in}{0.747409in}}%
\pgfpathlineto{\pgfqpoint{4.509981in}{0.747406in}}%
\pgfpathlineto{\pgfqpoint{4.510545in}{0.747403in}}%
\pgfpathlineto{\pgfqpoint{4.511109in}{0.747400in}}%
\pgfpathlineto{\pgfqpoint{4.511674in}{0.747396in}}%
\pgfpathlineto{\pgfqpoint{4.512238in}{0.747393in}}%
\pgfpathlineto{\pgfqpoint{4.512802in}{0.747390in}}%
\pgfpathlineto{\pgfqpoint{4.513367in}{0.747386in}}%
\pgfpathlineto{\pgfqpoint{4.513931in}{0.747383in}}%
\pgfpathlineto{\pgfqpoint{4.514495in}{0.747380in}}%
\pgfpathlineto{\pgfqpoint{4.515060in}{0.747377in}}%
\pgfpathlineto{\pgfqpoint{4.515624in}{0.747373in}}%
\pgfpathlineto{\pgfqpoint{4.516188in}{0.747370in}}%
\pgfpathlineto{\pgfqpoint{4.516753in}{0.747367in}}%
\pgfpathlineto{\pgfqpoint{4.517317in}{0.747363in}}%
\pgfpathlineto{\pgfqpoint{4.517881in}{0.747360in}}%
\pgfpathlineto{\pgfqpoint{4.518446in}{0.747357in}}%
\pgfpathlineto{\pgfqpoint{4.519010in}{0.747354in}}%
\pgfpathlineto{\pgfqpoint{4.519574in}{0.747350in}}%
\pgfpathlineto{\pgfqpoint{4.520139in}{0.747347in}}%
\pgfpathlineto{\pgfqpoint{4.520703in}{0.747344in}}%
\pgfpathlineto{\pgfqpoint{4.521267in}{0.747340in}}%
\pgfpathlineto{\pgfqpoint{4.521832in}{0.747337in}}%
\pgfpathlineto{\pgfqpoint{4.522396in}{0.747334in}}%
\pgfpathlineto{\pgfqpoint{4.522960in}{0.747331in}}%
\pgfpathlineto{\pgfqpoint{4.523525in}{0.747327in}}%
\pgfpathlineto{\pgfqpoint{4.524089in}{0.747324in}}%
\pgfpathlineto{\pgfqpoint{4.524654in}{0.747321in}}%
\pgfpathlineto{\pgfqpoint{4.525218in}{0.747317in}}%
\pgfpathlineto{\pgfqpoint{4.525782in}{0.747314in}}%
\pgfpathlineto{\pgfqpoint{4.526347in}{0.747311in}}%
\pgfpathlineto{\pgfqpoint{4.526911in}{0.747308in}}%
\pgfpathlineto{\pgfqpoint{4.527475in}{0.747304in}}%
\pgfpathlineto{\pgfqpoint{4.528040in}{0.747301in}}%
\pgfpathlineto{\pgfqpoint{4.528604in}{0.747298in}}%
\pgfpathlineto{\pgfqpoint{4.529168in}{0.747295in}}%
\pgfpathlineto{\pgfqpoint{4.529733in}{0.747291in}}%
\pgfpathlineto{\pgfqpoint{4.530297in}{0.747288in}}%
\pgfpathlineto{\pgfqpoint{4.530861in}{0.747285in}}%
\pgfpathlineto{\pgfqpoint{4.531426in}{0.747281in}}%
\pgfpathlineto{\pgfqpoint{4.531990in}{0.747278in}}%
\pgfpathlineto{\pgfqpoint{4.532554in}{0.747275in}}%
\pgfpathlineto{\pgfqpoint{4.533119in}{0.747272in}}%
\pgfpathlineto{\pgfqpoint{4.533683in}{0.747268in}}%
\pgfpathlineto{\pgfqpoint{4.534247in}{0.747265in}}%
\pgfpathlineto{\pgfqpoint{4.534812in}{0.747262in}}%
\pgfpathlineto{\pgfqpoint{4.535376in}{0.747258in}}%
\pgfpathlineto{\pgfqpoint{4.535940in}{0.747255in}}%
\pgfpathlineto{\pgfqpoint{4.536505in}{0.747252in}}%
\pgfpathlineto{\pgfqpoint{4.537069in}{0.747249in}}%
\pgfpathlineto{\pgfqpoint{4.537633in}{0.747245in}}%
\pgfpathlineto{\pgfqpoint{4.538198in}{0.747242in}}%
\pgfpathlineto{\pgfqpoint{4.538762in}{0.747239in}}%
\pgfpathlineto{\pgfqpoint{4.539326in}{0.747235in}}%
\pgfpathlineto{\pgfqpoint{4.539891in}{0.747232in}}%
\pgfpathlineto{\pgfqpoint{4.540455in}{0.747229in}}%
\pgfpathlineto{\pgfqpoint{4.541020in}{0.747226in}}%
\pgfpathlineto{\pgfqpoint{4.541584in}{0.747222in}}%
\pgfpathlineto{\pgfqpoint{4.542148in}{0.747219in}}%
\pgfpathlineto{\pgfqpoint{4.542713in}{0.747216in}}%
\pgfpathlineto{\pgfqpoint{4.543277in}{0.747213in}}%
\pgfpathlineto{\pgfqpoint{4.543841in}{0.747209in}}%
\pgfpathlineto{\pgfqpoint{4.544406in}{0.747206in}}%
\pgfpathlineto{\pgfqpoint{4.544970in}{0.747203in}}%
\pgfpathlineto{\pgfqpoint{4.545534in}{0.747199in}}%
\pgfpathlineto{\pgfqpoint{4.546099in}{0.747196in}}%
\pgfpathlineto{\pgfqpoint{4.546663in}{0.747193in}}%
\pgfpathlineto{\pgfqpoint{4.547227in}{0.747190in}}%
\pgfpathlineto{\pgfqpoint{4.547792in}{0.747186in}}%
\pgfpathlineto{\pgfqpoint{4.548356in}{0.747183in}}%
\pgfpathlineto{\pgfqpoint{4.548920in}{0.747180in}}%
\pgfpathlineto{\pgfqpoint{4.549485in}{0.747176in}}%
\pgfpathlineto{\pgfqpoint{4.550049in}{0.747173in}}%
\pgfpathlineto{\pgfqpoint{4.550613in}{0.747170in}}%
\pgfpathlineto{\pgfqpoint{4.551178in}{0.747167in}}%
\pgfpathlineto{\pgfqpoint{4.551742in}{0.747163in}}%
\pgfpathlineto{\pgfqpoint{4.552306in}{0.747160in}}%
\pgfpathlineto{\pgfqpoint{4.552871in}{0.747157in}}%
\pgfpathlineto{\pgfqpoint{4.553435in}{0.747153in}}%
\pgfpathlineto{\pgfqpoint{4.553999in}{0.747150in}}%
\pgfpathlineto{\pgfqpoint{4.554564in}{0.747147in}}%
\pgfpathlineto{\pgfqpoint{4.555128in}{0.747144in}}%
\pgfpathlineto{\pgfqpoint{4.555693in}{0.747140in}}%
\pgfpathlineto{\pgfqpoint{4.556257in}{0.747137in}}%
\pgfpathlineto{\pgfqpoint{4.556821in}{0.747134in}}%
\pgfpathlineto{\pgfqpoint{4.557386in}{0.747130in}}%
\pgfpathlineto{\pgfqpoint{4.557950in}{0.747127in}}%
\pgfpathlineto{\pgfqpoint{4.558514in}{0.747124in}}%
\pgfpathlineto{\pgfqpoint{4.559079in}{0.747121in}}%
\pgfpathlineto{\pgfqpoint{4.559643in}{0.747117in}}%
\pgfpathlineto{\pgfqpoint{4.560207in}{0.747114in}}%
\pgfpathlineto{\pgfqpoint{4.560772in}{0.747111in}}%
\pgfpathlineto{\pgfqpoint{4.561336in}{0.747108in}}%
\pgfpathlineto{\pgfqpoint{4.561900in}{0.747104in}}%
\pgfpathlineto{\pgfqpoint{4.562465in}{0.747101in}}%
\pgfpathlineto{\pgfqpoint{4.563029in}{0.747098in}}%
\pgfpathlineto{\pgfqpoint{4.563593in}{0.747094in}}%
\pgfpathlineto{\pgfqpoint{4.564158in}{0.747091in}}%
\pgfpathlineto{\pgfqpoint{4.564722in}{0.747088in}}%
\pgfpathlineto{\pgfqpoint{4.565286in}{0.747085in}}%
\pgfpathlineto{\pgfqpoint{4.565851in}{0.747081in}}%
\pgfpathlineto{\pgfqpoint{4.566415in}{0.747078in}}%
\pgfpathlineto{\pgfqpoint{4.566979in}{0.747075in}}%
\pgfpathlineto{\pgfqpoint{4.567544in}{0.747071in}}%
\pgfpathlineto{\pgfqpoint{4.568108in}{0.747068in}}%
\pgfpathlineto{\pgfqpoint{4.568672in}{0.747065in}}%
\pgfpathlineto{\pgfqpoint{4.569237in}{0.747062in}}%
\pgfpathlineto{\pgfqpoint{4.569801in}{0.747058in}}%
\pgfpathlineto{\pgfqpoint{4.570366in}{0.747055in}}%
\pgfpathlineto{\pgfqpoint{4.570930in}{0.747052in}}%
\pgfpathlineto{\pgfqpoint{4.571494in}{0.747048in}}%
\pgfpathlineto{\pgfqpoint{4.572059in}{0.747045in}}%
\pgfpathlineto{\pgfqpoint{4.572623in}{0.747042in}}%
\pgfpathlineto{\pgfqpoint{4.573187in}{0.747039in}}%
\pgfpathlineto{\pgfqpoint{4.573752in}{0.747035in}}%
\pgfpathlineto{\pgfqpoint{4.574316in}{0.747032in}}%
\pgfpathlineto{\pgfqpoint{4.574880in}{0.747029in}}%
\pgfpathlineto{\pgfqpoint{4.575445in}{0.747026in}}%
\pgfpathlineto{\pgfqpoint{4.576009in}{0.747022in}}%
\pgfpathlineto{\pgfqpoint{4.576573in}{0.747019in}}%
\pgfpathlineto{\pgfqpoint{4.577138in}{0.747016in}}%
\pgfpathlineto{\pgfqpoint{4.577702in}{0.747012in}}%
\pgfpathlineto{\pgfqpoint{4.578266in}{0.747009in}}%
\pgfpathlineto{\pgfqpoint{4.578831in}{0.747006in}}%
\pgfpathlineto{\pgfqpoint{4.579395in}{0.747003in}}%
\pgfpathlineto{\pgfqpoint{4.579959in}{0.746999in}}%
\pgfpathlineto{\pgfqpoint{4.580524in}{0.746996in}}%
\pgfpathlineto{\pgfqpoint{4.581088in}{0.746993in}}%
\pgfpathlineto{\pgfqpoint{4.581652in}{0.746989in}}%
\pgfpathlineto{\pgfqpoint{4.582217in}{0.746986in}}%
\pgfpathlineto{\pgfqpoint{4.582781in}{0.746983in}}%
\pgfpathlineto{\pgfqpoint{4.583345in}{0.746980in}}%
\pgfpathlineto{\pgfqpoint{4.583910in}{0.746976in}}%
\pgfpathlineto{\pgfqpoint{4.584474in}{0.746973in}}%
\pgfpathlineto{\pgfqpoint{4.585038in}{0.746970in}}%
\pgfpathlineto{\pgfqpoint{4.585603in}{0.746966in}}%
\pgfpathlineto{\pgfqpoint{4.586167in}{0.746963in}}%
\pgfpathlineto{\pgfqpoint{4.586732in}{0.746960in}}%
\pgfpathlineto{\pgfqpoint{4.587296in}{0.746957in}}%
\pgfpathlineto{\pgfqpoint{4.587860in}{0.746953in}}%
\pgfpathlineto{\pgfqpoint{4.588425in}{0.746950in}}%
\pgfpathlineto{\pgfqpoint{4.588989in}{0.746947in}}%
\pgfpathlineto{\pgfqpoint{4.589553in}{0.746943in}}%
\pgfpathlineto{\pgfqpoint{4.590118in}{0.746940in}}%
\pgfpathlineto{\pgfqpoint{4.590682in}{0.746937in}}%
\pgfpathlineto{\pgfqpoint{4.591246in}{0.746934in}}%
\pgfpathlineto{\pgfqpoint{4.591811in}{0.746930in}}%
\pgfpathlineto{\pgfqpoint{4.592375in}{0.746927in}}%
\pgfpathlineto{\pgfqpoint{4.592939in}{0.746924in}}%
\pgfpathlineto{\pgfqpoint{4.593504in}{0.746921in}}%
\pgfpathlineto{\pgfqpoint{4.594068in}{0.746917in}}%
\pgfpathlineto{\pgfqpoint{4.594632in}{0.746914in}}%
\pgfpathlineto{\pgfqpoint{4.595197in}{0.746911in}}%
\pgfpathlineto{\pgfqpoint{4.595761in}{0.746907in}}%
\pgfpathlineto{\pgfqpoint{4.596325in}{0.746904in}}%
\pgfpathlineto{\pgfqpoint{4.596890in}{0.746901in}}%
\pgfpathlineto{\pgfqpoint{4.597454in}{0.746898in}}%
\pgfpathlineto{\pgfqpoint{4.598018in}{0.746894in}}%
\pgfpathlineto{\pgfqpoint{4.598583in}{0.746891in}}%
\pgfpathlineto{\pgfqpoint{4.599147in}{0.746888in}}%
\pgfpathlineto{\pgfqpoint{4.599711in}{0.746884in}}%
\pgfpathlineto{\pgfqpoint{4.600276in}{0.746881in}}%
\pgfpathlineto{\pgfqpoint{4.600840in}{0.746878in}}%
\pgfpathlineto{\pgfqpoint{4.601405in}{0.746875in}}%
\pgfpathlineto{\pgfqpoint{4.601969in}{0.746871in}}%
\pgfpathlineto{\pgfqpoint{4.602533in}{0.746868in}}%
\pgfpathlineto{\pgfqpoint{4.603098in}{0.746865in}}%
\pgfpathlineto{\pgfqpoint{4.603662in}{0.746861in}}%
\pgfpathlineto{\pgfqpoint{4.604226in}{0.746858in}}%
\pgfpathlineto{\pgfqpoint{4.604791in}{0.746855in}}%
\pgfpathlineto{\pgfqpoint{4.605355in}{0.746852in}}%
\pgfpathlineto{\pgfqpoint{4.605919in}{0.746848in}}%
\pgfpathlineto{\pgfqpoint{4.606484in}{0.746845in}}%
\pgfpathlineto{\pgfqpoint{4.607048in}{0.746842in}}%
\pgfpathlineto{\pgfqpoint{4.607612in}{0.746838in}}%
\pgfpathlineto{\pgfqpoint{4.608177in}{0.746835in}}%
\pgfpathlineto{\pgfqpoint{4.608741in}{0.746832in}}%
\pgfpathlineto{\pgfqpoint{4.609305in}{0.746829in}}%
\pgfpathlineto{\pgfqpoint{4.609870in}{0.746825in}}%
\pgfpathlineto{\pgfqpoint{4.610434in}{0.746822in}}%
\pgfpathlineto{\pgfqpoint{4.610998in}{0.746819in}}%
\pgfpathlineto{\pgfqpoint{4.611563in}{0.746816in}}%
\pgfpathlineto{\pgfqpoint{4.612127in}{0.746812in}}%
\pgfpathlineto{\pgfqpoint{4.612691in}{0.746809in}}%
\pgfpathlineto{\pgfqpoint{4.613256in}{0.746806in}}%
\pgfpathlineto{\pgfqpoint{4.613820in}{0.746802in}}%
\pgfpathlineto{\pgfqpoint{4.614384in}{0.746799in}}%
\pgfpathlineto{\pgfqpoint{4.614949in}{0.746796in}}%
\pgfpathlineto{\pgfqpoint{4.615513in}{0.746793in}}%
\pgfpathlineto{\pgfqpoint{4.616078in}{0.746789in}}%
\pgfpathlineto{\pgfqpoint{4.616642in}{0.746786in}}%
\pgfpathlineto{\pgfqpoint{4.617206in}{0.746783in}}%
\pgfpathlineto{\pgfqpoint{4.617771in}{0.746779in}}%
\pgfpathlineto{\pgfqpoint{4.618335in}{0.746776in}}%
\pgfpathlineto{\pgfqpoint{4.618899in}{0.746773in}}%
\pgfpathlineto{\pgfqpoint{4.619464in}{0.746770in}}%
\pgfpathlineto{\pgfqpoint{4.620028in}{0.746766in}}%
\pgfpathlineto{\pgfqpoint{4.620592in}{0.746763in}}%
\pgfpathlineto{\pgfqpoint{4.621157in}{0.746760in}}%
\pgfpathlineto{\pgfqpoint{4.621721in}{0.746756in}}%
\pgfpathlineto{\pgfqpoint{4.622285in}{0.746753in}}%
\pgfpathlineto{\pgfqpoint{4.622850in}{0.746750in}}%
\pgfpathlineto{\pgfqpoint{4.623414in}{0.746747in}}%
\pgfpathlineto{\pgfqpoint{4.623978in}{0.746743in}}%
\pgfpathlineto{\pgfqpoint{4.624543in}{0.746740in}}%
\pgfpathlineto{\pgfqpoint{4.625107in}{0.746737in}}%
\pgfpathlineto{\pgfqpoint{4.625671in}{0.746734in}}%
\pgfpathlineto{\pgfqpoint{4.626236in}{0.746730in}}%
\pgfpathlineto{\pgfqpoint{4.626800in}{0.746727in}}%
\pgfpathlineto{\pgfqpoint{4.627364in}{0.746724in}}%
\pgfpathlineto{\pgfqpoint{4.627929in}{0.746720in}}%
\pgfpathlineto{\pgfqpoint{4.628493in}{0.746717in}}%
\pgfpathlineto{\pgfqpoint{4.629057in}{0.746714in}}%
\pgfpathlineto{\pgfqpoint{4.629622in}{0.746711in}}%
\pgfpathlineto{\pgfqpoint{4.630186in}{0.746707in}}%
\pgfpathlineto{\pgfqpoint{4.630750in}{0.746704in}}%
\pgfpathlineto{\pgfqpoint{4.631315in}{0.746701in}}%
\pgfpathlineto{\pgfqpoint{4.631879in}{0.746697in}}%
\pgfpathlineto{\pgfqpoint{4.632444in}{0.746694in}}%
\pgfpathlineto{\pgfqpoint{4.633008in}{0.746691in}}%
\pgfpathlineto{\pgfqpoint{4.633572in}{0.746688in}}%
\pgfpathlineto{\pgfqpoint{4.634137in}{0.746684in}}%
\pgfpathlineto{\pgfqpoint{4.634701in}{0.746681in}}%
\pgfpathlineto{\pgfqpoint{4.635265in}{0.746678in}}%
\pgfpathlineto{\pgfqpoint{4.635830in}{0.746674in}}%
\pgfpathlineto{\pgfqpoint{4.636394in}{0.746671in}}%
\pgfpathlineto{\pgfqpoint{4.636958in}{0.746668in}}%
\pgfpathlineto{\pgfqpoint{4.637523in}{0.746665in}}%
\pgfpathlineto{\pgfqpoint{4.638087in}{0.746661in}}%
\pgfpathlineto{\pgfqpoint{4.638651in}{0.746658in}}%
\pgfpathlineto{\pgfqpoint{4.639216in}{0.746655in}}%
\pgfpathlineto{\pgfqpoint{4.639780in}{0.746651in}}%
\pgfpathlineto{\pgfqpoint{4.640344in}{0.746648in}}%
\pgfpathlineto{\pgfqpoint{4.640909in}{0.746645in}}%
\pgfpathlineto{\pgfqpoint{4.641473in}{0.746642in}}%
\pgfpathlineto{\pgfqpoint{4.642037in}{0.746638in}}%
\pgfpathlineto{\pgfqpoint{4.642602in}{0.746635in}}%
\pgfpathlineto{\pgfqpoint{4.643166in}{0.746632in}}%
\pgfpathlineto{\pgfqpoint{4.643730in}{0.746629in}}%
\pgfpathlineto{\pgfqpoint{4.644295in}{0.746625in}}%
\pgfpathlineto{\pgfqpoint{4.644859in}{0.746622in}}%
\pgfpathlineto{\pgfqpoint{4.645423in}{0.746619in}}%
\pgfpathlineto{\pgfqpoint{4.645988in}{0.746615in}}%
\pgfpathlineto{\pgfqpoint{4.646552in}{0.746612in}}%
\pgfpathlineto{\pgfqpoint{4.647117in}{0.746609in}}%
\pgfpathlineto{\pgfqpoint{4.647681in}{0.746606in}}%
\pgfpathlineto{\pgfqpoint{4.648245in}{0.746602in}}%
\pgfpathlineto{\pgfqpoint{4.648810in}{0.746599in}}%
\pgfpathlineto{\pgfqpoint{4.649374in}{0.746596in}}%
\pgfpathlineto{\pgfqpoint{4.649938in}{0.746592in}}%
\pgfpathlineto{\pgfqpoint{4.650503in}{0.746589in}}%
\pgfpathlineto{\pgfqpoint{4.651067in}{0.746586in}}%
\pgfpathlineto{\pgfqpoint{4.651631in}{0.746583in}}%
\pgfpathlineto{\pgfqpoint{4.652196in}{0.746579in}}%
\pgfpathlineto{\pgfqpoint{4.652760in}{0.746576in}}%
\pgfpathlineto{\pgfqpoint{4.653324in}{0.746573in}}%
\pgfpathlineto{\pgfqpoint{4.653889in}{0.746569in}}%
\pgfpathlineto{\pgfqpoint{4.654453in}{0.746566in}}%
\pgfpathlineto{\pgfqpoint{4.655017in}{0.746563in}}%
\pgfpathlineto{\pgfqpoint{4.655582in}{0.746560in}}%
\pgfpathlineto{\pgfqpoint{4.656146in}{0.746556in}}%
\pgfpathlineto{\pgfqpoint{4.656710in}{0.746553in}}%
\pgfpathlineto{\pgfqpoint{4.657275in}{0.746550in}}%
\pgfpathlineto{\pgfqpoint{4.657839in}{0.746546in}}%
\pgfpathlineto{\pgfqpoint{4.658403in}{0.746543in}}%
\pgfpathlineto{\pgfqpoint{4.658968in}{0.746540in}}%
\pgfpathlineto{\pgfqpoint{4.659532in}{0.746537in}}%
\pgfpathlineto{\pgfqpoint{4.660096in}{0.746533in}}%
\pgfpathlineto{\pgfqpoint{4.660661in}{0.746530in}}%
\pgfpathlineto{\pgfqpoint{4.661225in}{0.746527in}}%
\pgfpathlineto{\pgfqpoint{4.661790in}{0.746524in}}%
\pgfpathlineto{\pgfqpoint{4.662354in}{0.746520in}}%
\pgfpathlineto{\pgfqpoint{4.662918in}{0.746517in}}%
\pgfpathlineto{\pgfqpoint{4.663483in}{0.746514in}}%
\pgfpathlineto{\pgfqpoint{4.664047in}{0.746510in}}%
\pgfpathlineto{\pgfqpoint{4.664611in}{0.746507in}}%
\pgfpathlineto{\pgfqpoint{4.665176in}{0.746504in}}%
\pgfpathlineto{\pgfqpoint{4.665740in}{0.746501in}}%
\pgfpathlineto{\pgfqpoint{4.666304in}{0.746497in}}%
\pgfpathlineto{\pgfqpoint{4.666869in}{0.746494in}}%
\pgfpathlineto{\pgfqpoint{4.667433in}{0.746491in}}%
\pgfpathlineto{\pgfqpoint{4.667997in}{0.746487in}}%
\pgfpathlineto{\pgfqpoint{4.668562in}{0.746484in}}%
\pgfpathlineto{\pgfqpoint{4.669126in}{0.746481in}}%
\pgfpathlineto{\pgfqpoint{4.669690in}{0.746478in}}%
\pgfpathlineto{\pgfqpoint{4.670255in}{0.746474in}}%
\pgfpathlineto{\pgfqpoint{4.670819in}{0.746471in}}%
\pgfpathlineto{\pgfqpoint{4.671383in}{0.746468in}}%
\pgfpathlineto{\pgfqpoint{4.671948in}{0.746464in}}%
\pgfpathlineto{\pgfqpoint{4.672512in}{0.746461in}}%
\pgfpathlineto{\pgfqpoint{4.673076in}{0.746458in}}%
\pgfpathlineto{\pgfqpoint{4.673641in}{0.746455in}}%
\pgfpathlineto{\pgfqpoint{4.674205in}{0.746451in}}%
\pgfpathlineto{\pgfqpoint{4.674769in}{0.746448in}}%
\pgfpathlineto{\pgfqpoint{4.675334in}{0.746445in}}%
\pgfpathlineto{\pgfqpoint{4.675898in}{0.746442in}}%
\pgfpathlineto{\pgfqpoint{4.676462in}{0.746438in}}%
\pgfpathlineto{\pgfqpoint{4.677027in}{0.746435in}}%
\pgfpathlineto{\pgfqpoint{4.677591in}{0.746432in}}%
\pgfpathlineto{\pgfqpoint{4.678156in}{0.746428in}}%
\pgfpathlineto{\pgfqpoint{4.678720in}{0.746425in}}%
\pgfpathlineto{\pgfqpoint{4.679284in}{0.746422in}}%
\pgfpathlineto{\pgfqpoint{4.679849in}{0.746419in}}%
\pgfpathlineto{\pgfqpoint{4.680413in}{0.746415in}}%
\pgfpathlineto{\pgfqpoint{4.680977in}{0.746412in}}%
\pgfpathlineto{\pgfqpoint{4.681542in}{0.746409in}}%
\pgfpathlineto{\pgfqpoint{4.682106in}{0.746405in}}%
\pgfpathlineto{\pgfqpoint{4.682670in}{0.746402in}}%
\pgfpathlineto{\pgfqpoint{4.683235in}{0.746399in}}%
\pgfpathlineto{\pgfqpoint{4.683799in}{0.746396in}}%
\pgfpathlineto{\pgfqpoint{4.684363in}{0.746392in}}%
\pgfpathlineto{\pgfqpoint{4.684928in}{0.746389in}}%
\pgfpathlineto{\pgfqpoint{4.685492in}{0.746386in}}%
\pgfpathlineto{\pgfqpoint{4.686056in}{0.746382in}}%
\pgfpathlineto{\pgfqpoint{4.686621in}{0.746379in}}%
\pgfpathlineto{\pgfqpoint{4.687185in}{0.746376in}}%
\pgfpathlineto{\pgfqpoint{4.687749in}{0.746373in}}%
\pgfpathlineto{\pgfqpoint{4.688314in}{0.746369in}}%
\pgfpathlineto{\pgfqpoint{4.688878in}{0.746366in}}%
\pgfpathlineto{\pgfqpoint{4.689442in}{0.746363in}}%
\pgfpathlineto{\pgfqpoint{4.690007in}{0.746359in}}%
\pgfpathlineto{\pgfqpoint{4.690571in}{0.746356in}}%
\pgfpathlineto{\pgfqpoint{4.691135in}{0.746353in}}%
\pgfpathlineto{\pgfqpoint{4.691700in}{0.746350in}}%
\pgfpathlineto{\pgfqpoint{4.692264in}{0.746346in}}%
\pgfpathlineto{\pgfqpoint{4.692829in}{0.746343in}}%
\pgfpathlineto{\pgfqpoint{4.693393in}{0.746340in}}%
\pgfpathlineto{\pgfqpoint{4.693957in}{0.746337in}}%
\pgfpathlineto{\pgfqpoint{4.694522in}{0.746333in}}%
\pgfpathlineto{\pgfqpoint{4.695086in}{0.746330in}}%
\pgfpathlineto{\pgfqpoint{4.695650in}{0.746327in}}%
\pgfpathlineto{\pgfqpoint{4.696215in}{0.746323in}}%
\pgfpathlineto{\pgfqpoint{4.696779in}{0.746320in}}%
\pgfpathlineto{\pgfqpoint{4.697343in}{0.746317in}}%
\pgfpathlineto{\pgfqpoint{4.697908in}{0.746314in}}%
\pgfpathlineto{\pgfqpoint{4.698472in}{0.746310in}}%
\pgfpathlineto{\pgfqpoint{4.699036in}{0.746307in}}%
\pgfpathlineto{\pgfqpoint{4.699601in}{0.746304in}}%
\pgfpathlineto{\pgfqpoint{4.700165in}{0.746300in}}%
\pgfpathlineto{\pgfqpoint{4.700729in}{0.746297in}}%
\pgfpathlineto{\pgfqpoint{4.701294in}{0.746294in}}%
\pgfpathlineto{\pgfqpoint{4.701858in}{0.746291in}}%
\pgfpathlineto{\pgfqpoint{4.702422in}{0.746287in}}%
\pgfpathlineto{\pgfqpoint{4.702987in}{0.746284in}}%
\pgfpathlineto{\pgfqpoint{4.703551in}{0.746281in}}%
\pgfpathlineto{\pgfqpoint{4.704115in}{0.746277in}}%
\pgfpathlineto{\pgfqpoint{4.704680in}{0.746274in}}%
\pgfpathlineto{\pgfqpoint{4.705244in}{0.746271in}}%
\pgfpathlineto{\pgfqpoint{4.705808in}{0.746268in}}%
\pgfpathlineto{\pgfqpoint{4.706373in}{0.746264in}}%
\pgfpathlineto{\pgfqpoint{4.706937in}{0.746261in}}%
\pgfpathlineto{\pgfqpoint{4.707502in}{0.746258in}}%
\pgfpathlineto{\pgfqpoint{4.708066in}{0.746254in}}%
\pgfpathlineto{\pgfqpoint{4.708630in}{0.746251in}}%
\pgfpathlineto{\pgfqpoint{4.709195in}{0.746248in}}%
\pgfpathlineto{\pgfqpoint{4.709759in}{0.746245in}}%
\pgfpathlineto{\pgfqpoint{4.710323in}{0.746241in}}%
\pgfpathlineto{\pgfqpoint{4.710888in}{0.746238in}}%
\pgfpathlineto{\pgfqpoint{4.711452in}{0.746235in}}%
\pgfpathlineto{\pgfqpoint{4.712016in}{0.746232in}}%
\pgfpathlineto{\pgfqpoint{4.712581in}{0.746228in}}%
\pgfpathlineto{\pgfqpoint{4.713145in}{0.746225in}}%
\pgfpathlineto{\pgfqpoint{4.713709in}{0.746222in}}%
\pgfpathlineto{\pgfqpoint{4.714274in}{0.746218in}}%
\pgfpathlineto{\pgfqpoint{4.714838in}{0.746215in}}%
\pgfpathlineto{\pgfqpoint{4.715402in}{0.746212in}}%
\pgfpathlineto{\pgfqpoint{4.715967in}{0.746209in}}%
\pgfpathlineto{\pgfqpoint{4.716531in}{0.746205in}}%
\pgfpathlineto{\pgfqpoint{4.717095in}{0.746202in}}%
\pgfpathlineto{\pgfqpoint{4.717660in}{0.746199in}}%
\pgfpathlineto{\pgfqpoint{4.718224in}{0.746195in}}%
\pgfpathlineto{\pgfqpoint{4.718788in}{0.746192in}}%
\pgfpathlineto{\pgfqpoint{4.719353in}{0.746189in}}%
\pgfpathlineto{\pgfqpoint{4.719917in}{0.746186in}}%
\pgfpathlineto{\pgfqpoint{4.720481in}{0.746182in}}%
\pgfpathlineto{\pgfqpoint{4.721046in}{0.746179in}}%
\pgfpathlineto{\pgfqpoint{4.721610in}{0.746176in}}%
\pgfpathlineto{\pgfqpoint{4.722175in}{0.746172in}}%
\pgfpathlineto{\pgfqpoint{4.722739in}{0.746169in}}%
\pgfpathlineto{\pgfqpoint{4.723303in}{0.746166in}}%
\pgfpathlineto{\pgfqpoint{4.723868in}{0.746163in}}%
\pgfpathlineto{\pgfqpoint{4.724432in}{0.746159in}}%
\pgfpathlineto{\pgfqpoint{4.724996in}{0.746156in}}%
\pgfpathlineto{\pgfqpoint{4.725561in}{0.746153in}}%
\pgfpathlineto{\pgfqpoint{4.726125in}{0.746150in}}%
\pgfpathlineto{\pgfqpoint{4.726689in}{0.746146in}}%
\pgfpathlineto{\pgfqpoint{4.727254in}{0.746143in}}%
\pgfpathlineto{\pgfqpoint{4.727818in}{0.746140in}}%
\pgfpathlineto{\pgfqpoint{4.728382in}{0.746136in}}%
\pgfpathlineto{\pgfqpoint{4.728947in}{0.746133in}}%
\pgfpathlineto{\pgfqpoint{4.729511in}{0.746130in}}%
\pgfpathlineto{\pgfqpoint{4.730075in}{0.746127in}}%
\pgfpathlineto{\pgfqpoint{4.730640in}{0.746123in}}%
\pgfpathlineto{\pgfqpoint{4.731204in}{0.746120in}}%
\pgfpathlineto{\pgfqpoint{4.731768in}{0.746117in}}%
\pgfpathlineto{\pgfqpoint{4.732333in}{0.746113in}}%
\pgfpathlineto{\pgfqpoint{4.732897in}{0.746110in}}%
\pgfpathlineto{\pgfqpoint{4.733461in}{0.746107in}}%
\pgfpathlineto{\pgfqpoint{4.734026in}{0.746104in}}%
\pgfpathlineto{\pgfqpoint{4.734590in}{0.746100in}}%
\pgfpathlineto{\pgfqpoint{4.735154in}{0.746097in}}%
\pgfpathlineto{\pgfqpoint{4.735719in}{0.746094in}}%
\pgfpathlineto{\pgfqpoint{4.736283in}{0.746090in}}%
\pgfpathlineto{\pgfqpoint{4.736847in}{0.746087in}}%
\pgfpathlineto{\pgfqpoint{4.737412in}{0.746084in}}%
\pgfpathlineto{\pgfqpoint{4.737976in}{0.746081in}}%
\pgfpathlineto{\pgfqpoint{4.738541in}{0.746077in}}%
\pgfpathlineto{\pgfqpoint{4.739105in}{0.746074in}}%
\pgfpathlineto{\pgfqpoint{4.739669in}{0.746071in}}%
\pgfpathlineto{\pgfqpoint{4.740234in}{0.746067in}}%
\pgfpathlineto{\pgfqpoint{4.740798in}{0.746064in}}%
\pgfpathlineto{\pgfqpoint{4.741362in}{0.746061in}}%
\pgfpathlineto{\pgfqpoint{4.741927in}{0.746058in}}%
\pgfpathlineto{\pgfqpoint{4.742491in}{0.746054in}}%
\pgfpathlineto{\pgfqpoint{4.743055in}{0.746051in}}%
\pgfpathlineto{\pgfqpoint{4.743620in}{0.746048in}}%
\pgfpathlineto{\pgfqpoint{4.744184in}{0.746045in}}%
\pgfpathlineto{\pgfqpoint{4.744748in}{0.746041in}}%
\pgfpathlineto{\pgfqpoint{4.745313in}{0.746038in}}%
\pgfpathlineto{\pgfqpoint{4.745877in}{0.746035in}}%
\pgfpathlineto{\pgfqpoint{4.746441in}{0.746031in}}%
\pgfpathlineto{\pgfqpoint{4.747006in}{0.746028in}}%
\pgfpathlineto{\pgfqpoint{4.747570in}{0.746025in}}%
\pgfpathlineto{\pgfqpoint{4.748134in}{0.746022in}}%
\pgfpathlineto{\pgfqpoint{4.748699in}{0.746018in}}%
\pgfpathlineto{\pgfqpoint{4.749263in}{0.746015in}}%
\pgfpathlineto{\pgfqpoint{4.749827in}{0.746012in}}%
\pgfpathlineto{\pgfqpoint{4.750392in}{0.746008in}}%
\pgfpathlineto{\pgfqpoint{4.750956in}{0.746005in}}%
\pgfpathlineto{\pgfqpoint{4.751520in}{0.746002in}}%
\pgfpathlineto{\pgfqpoint{4.752085in}{0.745999in}}%
\pgfpathlineto{\pgfqpoint{4.752649in}{0.745995in}}%
\pgfpathlineto{\pgfqpoint{4.753214in}{0.745992in}}%
\pgfpathlineto{\pgfqpoint{4.753778in}{0.745989in}}%
\pgfpathlineto{\pgfqpoint{4.754342in}{0.745985in}}%
\pgfpathlineto{\pgfqpoint{4.754907in}{0.745982in}}%
\pgfpathlineto{\pgfqpoint{4.755471in}{0.745979in}}%
\pgfpathlineto{\pgfqpoint{4.756035in}{0.745976in}}%
\pgfpathlineto{\pgfqpoint{4.756600in}{0.745972in}}%
\pgfpathlineto{\pgfqpoint{4.757164in}{0.745969in}}%
\pgfpathlineto{\pgfqpoint{4.757728in}{0.745966in}}%
\pgfpathlineto{\pgfqpoint{4.758293in}{0.745963in}}%
\pgfpathlineto{\pgfqpoint{4.758857in}{0.745959in}}%
\pgfpathlineto{\pgfqpoint{4.759421in}{0.745956in}}%
\pgfpathlineto{\pgfqpoint{4.759986in}{0.745953in}}%
\pgfpathlineto{\pgfqpoint{4.760550in}{0.745949in}}%
\pgfpathlineto{\pgfqpoint{4.761114in}{0.745946in}}%
\pgfpathlineto{\pgfqpoint{4.761679in}{0.745943in}}%
\pgfpathlineto{\pgfqpoint{4.762243in}{0.745940in}}%
\pgfpathlineto{\pgfqpoint{4.762807in}{0.745936in}}%
\pgfpathlineto{\pgfqpoint{4.763372in}{0.745933in}}%
\pgfpathlineto{\pgfqpoint{4.763936in}{0.745930in}}%
\pgfpathlineto{\pgfqpoint{4.764500in}{0.745926in}}%
\pgfpathlineto{\pgfqpoint{4.765065in}{0.745923in}}%
\pgfpathlineto{\pgfqpoint{4.765629in}{0.745920in}}%
\pgfpathlineto{\pgfqpoint{4.766193in}{0.745917in}}%
\pgfpathlineto{\pgfqpoint{4.766758in}{0.745913in}}%
\pgfpathlineto{\pgfqpoint{4.767322in}{0.745910in}}%
\pgfpathlineto{\pgfqpoint{4.767887in}{0.745907in}}%
\pgfpathlineto{\pgfqpoint{4.768451in}{0.745903in}}%
\pgfpathlineto{\pgfqpoint{4.769015in}{0.745900in}}%
\pgfpathlineto{\pgfqpoint{4.769580in}{0.745897in}}%
\pgfpathlineto{\pgfqpoint{4.770144in}{0.745894in}}%
\pgfpathlineto{\pgfqpoint{4.770708in}{0.745890in}}%
\pgfpathlineto{\pgfqpoint{4.771273in}{0.745887in}}%
\pgfpathlineto{\pgfqpoint{4.771837in}{0.745884in}}%
\pgfpathlineto{\pgfqpoint{4.772401in}{0.745880in}}%
\pgfpathlineto{\pgfqpoint{4.772966in}{0.745877in}}%
\pgfpathlineto{\pgfqpoint{4.773530in}{0.745874in}}%
\pgfpathlineto{\pgfqpoint{4.774094in}{0.745871in}}%
\pgfpathlineto{\pgfqpoint{4.774659in}{0.745867in}}%
\pgfpathlineto{\pgfqpoint{4.775223in}{0.745864in}}%
\pgfpathlineto{\pgfqpoint{4.775787in}{0.745861in}}%
\pgfpathlineto{\pgfqpoint{4.776352in}{0.745858in}}%
\pgfpathlineto{\pgfqpoint{4.776916in}{0.745854in}}%
\pgfpathlineto{\pgfqpoint{4.777480in}{0.745851in}}%
\pgfpathlineto{\pgfqpoint{4.778045in}{0.745848in}}%
\pgfpathlineto{\pgfqpoint{4.778609in}{0.745844in}}%
\pgfpathlineto{\pgfqpoint{4.779173in}{0.745841in}}%
\pgfpathlineto{\pgfqpoint{4.779738in}{0.745838in}}%
\pgfpathlineto{\pgfqpoint{4.780302in}{0.745835in}}%
\pgfpathlineto{\pgfqpoint{4.780866in}{0.745831in}}%
\pgfpathlineto{\pgfqpoint{4.781431in}{0.745828in}}%
\pgfpathlineto{\pgfqpoint{4.781995in}{0.745825in}}%
\pgfpathlineto{\pgfqpoint{4.782559in}{0.745821in}}%
\pgfpathlineto{\pgfqpoint{4.783124in}{0.745818in}}%
\pgfpathlineto{\pgfqpoint{4.783688in}{0.745815in}}%
\pgfpathlineto{\pgfqpoint{4.784253in}{0.745812in}}%
\pgfpathlineto{\pgfqpoint{4.784817in}{0.745808in}}%
\pgfpathlineto{\pgfqpoint{4.785381in}{0.745805in}}%
\pgfpathlineto{\pgfqpoint{4.785946in}{0.745802in}}%
\pgfpathlineto{\pgfqpoint{4.786510in}{0.745798in}}%
\pgfpathlineto{\pgfqpoint{4.787074in}{0.745795in}}%
\pgfpathlineto{\pgfqpoint{4.787639in}{0.745792in}}%
\pgfpathlineto{\pgfqpoint{4.788203in}{0.745789in}}%
\pgfpathlineto{\pgfqpoint{4.788767in}{0.745785in}}%
\pgfpathlineto{\pgfqpoint{4.789332in}{0.745782in}}%
\pgfpathlineto{\pgfqpoint{4.789896in}{0.745779in}}%
\pgfpathlineto{\pgfqpoint{4.790460in}{0.745775in}}%
\pgfpathlineto{\pgfqpoint{4.791025in}{0.745772in}}%
\pgfpathlineto{\pgfqpoint{4.791589in}{0.745769in}}%
\pgfpathlineto{\pgfqpoint{4.792153in}{0.745766in}}%
\pgfpathlineto{\pgfqpoint{4.792718in}{0.745762in}}%
\pgfpathlineto{\pgfqpoint{4.793282in}{0.745759in}}%
\pgfpathlineto{\pgfqpoint{4.793846in}{0.745756in}}%
\pgfpathlineto{\pgfqpoint{4.794411in}{0.745753in}}%
\pgfpathlineto{\pgfqpoint{4.794975in}{0.745749in}}%
\pgfpathlineto{\pgfqpoint{4.795539in}{0.745746in}}%
\pgfpathlineto{\pgfqpoint{4.796104in}{0.745743in}}%
\pgfpathlineto{\pgfqpoint{4.796668in}{0.745739in}}%
\pgfpathlineto{\pgfqpoint{4.797232in}{0.745736in}}%
\pgfpathlineto{\pgfqpoint{4.797797in}{0.745733in}}%
\pgfpathlineto{\pgfqpoint{4.798361in}{0.745730in}}%
\pgfpathlineto{\pgfqpoint{4.798926in}{0.745726in}}%
\pgfpathlineto{\pgfqpoint{4.799490in}{0.745723in}}%
\pgfpathlineto{\pgfqpoint{4.800054in}{0.745720in}}%
\pgfpathlineto{\pgfqpoint{4.800619in}{0.745716in}}%
\pgfpathlineto{\pgfqpoint{4.801183in}{0.745713in}}%
\pgfpathlineto{\pgfqpoint{4.801747in}{0.745710in}}%
\pgfpathlineto{\pgfqpoint{4.802312in}{0.745707in}}%
\pgfpathlineto{\pgfqpoint{4.802876in}{0.745703in}}%
\pgfpathlineto{\pgfqpoint{4.803440in}{0.745700in}}%
\pgfpathlineto{\pgfqpoint{4.804005in}{0.745697in}}%
\pgfpathlineto{\pgfqpoint{4.804569in}{0.745693in}}%
\pgfpathlineto{\pgfqpoint{4.805133in}{0.745690in}}%
\pgfpathlineto{\pgfqpoint{4.805698in}{0.745687in}}%
\pgfpathlineto{\pgfqpoint{4.806262in}{0.745684in}}%
\pgfpathlineto{\pgfqpoint{4.806826in}{0.745680in}}%
\pgfpathlineto{\pgfqpoint{4.807391in}{0.745677in}}%
\pgfpathlineto{\pgfqpoint{4.807955in}{0.745674in}}%
\pgfpathlineto{\pgfqpoint{4.808519in}{0.745671in}}%
\pgfpathlineto{\pgfqpoint{4.809084in}{0.745667in}}%
\pgfpathlineto{\pgfqpoint{4.809648in}{0.745664in}}%
\pgfpathlineto{\pgfqpoint{4.810212in}{0.745661in}}%
\pgfpathlineto{\pgfqpoint{4.810777in}{0.745657in}}%
\pgfpathlineto{\pgfqpoint{4.811341in}{0.745654in}}%
\pgfpathlineto{\pgfqpoint{4.811905in}{0.745651in}}%
\pgfpathlineto{\pgfqpoint{4.812470in}{0.745648in}}%
\pgfpathlineto{\pgfqpoint{4.813034in}{0.745644in}}%
\pgfpathlineto{\pgfqpoint{4.813599in}{0.745641in}}%
\pgfpathlineto{\pgfqpoint{4.814163in}{0.745638in}}%
\pgfpathlineto{\pgfqpoint{4.814727in}{0.745634in}}%
\pgfpathlineto{\pgfqpoint{4.815292in}{0.745631in}}%
\pgfpathlineto{\pgfqpoint{4.815856in}{0.745628in}}%
\pgfpathlineto{\pgfqpoint{4.816420in}{0.745625in}}%
\pgfpathlineto{\pgfqpoint{4.816985in}{0.745621in}}%
\pgfpathlineto{\pgfqpoint{4.817549in}{0.745618in}}%
\pgfpathlineto{\pgfqpoint{4.818113in}{0.745615in}}%
\pgfpathlineto{\pgfqpoint{4.818678in}{0.745611in}}%
\pgfpathlineto{\pgfqpoint{4.819242in}{0.745608in}}%
\pgfpathlineto{\pgfqpoint{4.819806in}{0.745605in}}%
\pgfpathlineto{\pgfqpoint{4.820371in}{0.745602in}}%
\pgfpathlineto{\pgfqpoint{4.820935in}{0.745598in}}%
\pgfpathlineto{\pgfqpoint{4.821499in}{0.745595in}}%
\pgfpathlineto{\pgfqpoint{4.822064in}{0.745592in}}%
\pgfpathlineto{\pgfqpoint{4.822628in}{0.745588in}}%
\pgfpathlineto{\pgfqpoint{4.823192in}{0.745585in}}%
\pgfpathlineto{\pgfqpoint{4.823757in}{0.745582in}}%
\pgfpathlineto{\pgfqpoint{4.824321in}{0.745579in}}%
\pgfpathlineto{\pgfqpoint{4.824885in}{0.745575in}}%
\pgfpathlineto{\pgfqpoint{4.825450in}{0.745572in}}%
\pgfpathlineto{\pgfqpoint{4.826014in}{0.745569in}}%
\pgfpathlineto{\pgfqpoint{4.826578in}{0.745566in}}%
\pgfpathlineto{\pgfqpoint{4.827143in}{0.745562in}}%
\pgfpathlineto{\pgfqpoint{4.827707in}{0.745559in}}%
\pgfpathlineto{\pgfqpoint{4.828271in}{0.745556in}}%
\pgfpathlineto{\pgfqpoint{4.828836in}{0.745552in}}%
\pgfpathlineto{\pgfqpoint{4.829400in}{0.745549in}}%
\pgfpathlineto{\pgfqpoint{4.829965in}{0.745546in}}%
\pgfpathlineto{\pgfqpoint{4.830529in}{0.745543in}}%
\pgfpathlineto{\pgfqpoint{4.831093in}{0.745539in}}%
\pgfpathlineto{\pgfqpoint{4.831658in}{0.745536in}}%
\pgfpathlineto{\pgfqpoint{4.832222in}{0.745533in}}%
\pgfpathlineto{\pgfqpoint{4.832786in}{0.745529in}}%
\pgfpathlineto{\pgfqpoint{4.833351in}{0.745526in}}%
\pgfpathlineto{\pgfqpoint{4.833915in}{0.745523in}}%
\pgfpathlineto{\pgfqpoint{4.834479in}{0.745520in}}%
\pgfpathlineto{\pgfqpoint{4.835044in}{0.745516in}}%
\pgfpathlineto{\pgfqpoint{4.835608in}{0.745513in}}%
\pgfpathlineto{\pgfqpoint{4.836172in}{0.745510in}}%
\pgfpathlineto{\pgfqpoint{4.836737in}{0.745506in}}%
\pgfpathlineto{\pgfqpoint{4.837301in}{0.745503in}}%
\pgfpathlineto{\pgfqpoint{4.837865in}{0.745500in}}%
\pgfpathlineto{\pgfqpoint{4.838430in}{0.745497in}}%
\pgfpathlineto{\pgfqpoint{4.838994in}{0.745493in}}%
\pgfpathlineto{\pgfqpoint{4.839558in}{0.745490in}}%
\pgfpathlineto{\pgfqpoint{4.840123in}{0.745487in}}%
\pgfpathlineto{\pgfqpoint{4.840687in}{0.745483in}}%
\pgfpathlineto{\pgfqpoint{4.841251in}{0.745480in}}%
\pgfpathlineto{\pgfqpoint{4.841816in}{0.745477in}}%
\pgfpathlineto{\pgfqpoint{4.842380in}{0.745474in}}%
\pgfpathlineto{\pgfqpoint{4.842944in}{0.745470in}}%
\pgfpathlineto{\pgfqpoint{4.843509in}{0.745467in}}%
\pgfpathlineto{\pgfqpoint{4.844073in}{0.745464in}}%
\pgfpathlineto{\pgfqpoint{4.844638in}{0.745461in}}%
\pgfpathlineto{\pgfqpoint{4.845202in}{0.745457in}}%
\pgfpathlineto{\pgfqpoint{4.845766in}{0.745454in}}%
\pgfpathlineto{\pgfqpoint{4.846331in}{0.745451in}}%
\pgfpathlineto{\pgfqpoint{4.846895in}{0.745447in}}%
\pgfpathlineto{\pgfqpoint{4.847459in}{0.745444in}}%
\pgfpathlineto{\pgfqpoint{4.848024in}{0.745441in}}%
\pgfpathlineto{\pgfqpoint{4.848588in}{0.745438in}}%
\pgfpathlineto{\pgfqpoint{4.849152in}{0.745434in}}%
\pgfpathlineto{\pgfqpoint{4.849717in}{0.745431in}}%
\pgfpathlineto{\pgfqpoint{4.850281in}{0.745428in}}%
\pgfpathlineto{\pgfqpoint{4.850845in}{0.745424in}}%
\pgfpathlineto{\pgfqpoint{4.851410in}{0.745421in}}%
\pgfpathlineto{\pgfqpoint{4.851974in}{0.745418in}}%
\pgfpathlineto{\pgfqpoint{4.852538in}{0.745415in}}%
\pgfpathlineto{\pgfqpoint{4.853103in}{0.745411in}}%
\pgfpathlineto{\pgfqpoint{4.853667in}{0.745408in}}%
\pgfpathlineto{\pgfqpoint{4.854231in}{0.745405in}}%
\pgfpathlineto{\pgfqpoint{4.854796in}{0.745401in}}%
\pgfpathlineto{\pgfqpoint{4.855360in}{0.745398in}}%
\pgfpathlineto{\pgfqpoint{4.855924in}{0.745395in}}%
\pgfpathlineto{\pgfqpoint{4.856489in}{0.745392in}}%
\pgfpathlineto{\pgfqpoint{4.857053in}{0.745388in}}%
\pgfpathlineto{\pgfqpoint{4.857617in}{0.745385in}}%
\pgfpathlineto{\pgfqpoint{4.858182in}{0.745382in}}%
\pgfpathlineto{\pgfqpoint{4.858746in}{0.745379in}}%
\pgfpathlineto{\pgfqpoint{4.859311in}{0.745375in}}%
\pgfpathlineto{\pgfqpoint{4.859875in}{0.745372in}}%
\pgfpathlineto{\pgfqpoint{4.860439in}{0.745369in}}%
\pgfpathlineto{\pgfqpoint{4.861004in}{0.745365in}}%
\pgfpathlineto{\pgfqpoint{4.861568in}{0.745362in}}%
\pgfpathlineto{\pgfqpoint{4.862132in}{0.745359in}}%
\pgfpathlineto{\pgfqpoint{4.862697in}{0.745356in}}%
\pgfpathlineto{\pgfqpoint{4.863261in}{0.745352in}}%
\pgfpathlineto{\pgfqpoint{4.863825in}{0.745349in}}%
\pgfpathlineto{\pgfqpoint{4.864390in}{0.745346in}}%
\pgfpathlineto{\pgfqpoint{4.864954in}{0.745342in}}%
\pgfpathlineto{\pgfqpoint{4.865518in}{0.745339in}}%
\pgfpathlineto{\pgfqpoint{4.866083in}{0.745336in}}%
\pgfpathlineto{\pgfqpoint{4.866647in}{0.745333in}}%
\pgfpathlineto{\pgfqpoint{4.867211in}{0.745329in}}%
\pgfpathlineto{\pgfqpoint{4.867776in}{0.745326in}}%
\pgfpathlineto{\pgfqpoint{4.868340in}{0.745323in}}%
\pgfpathlineto{\pgfqpoint{4.868904in}{0.745319in}}%
\pgfpathlineto{\pgfqpoint{4.869469in}{0.745316in}}%
\pgfpathlineto{\pgfqpoint{4.870033in}{0.745313in}}%
\pgfpathlineto{\pgfqpoint{4.870597in}{0.745310in}}%
\pgfpathlineto{\pgfqpoint{4.871162in}{0.745306in}}%
\pgfpathlineto{\pgfqpoint{4.871726in}{0.745303in}}%
\pgfpathlineto{\pgfqpoint{4.872290in}{0.745300in}}%
\pgfpathlineto{\pgfqpoint{4.872855in}{0.745296in}}%
\pgfpathlineto{\pgfqpoint{4.873419in}{0.745293in}}%
\pgfpathlineto{\pgfqpoint{4.873983in}{0.745290in}}%
\pgfpathlineto{\pgfqpoint{4.874548in}{0.745287in}}%
\pgfpathlineto{\pgfqpoint{4.875112in}{0.745283in}}%
\pgfpathlineto{\pgfqpoint{4.875677in}{0.745280in}}%
\pgfpathlineto{\pgfqpoint{4.876241in}{0.745277in}}%
\pgfpathlineto{\pgfqpoint{4.876805in}{0.745274in}}%
\pgfpathlineto{\pgfqpoint{4.877370in}{0.745270in}}%
\pgfpathlineto{\pgfqpoint{4.877934in}{0.745267in}}%
\pgfpathlineto{\pgfqpoint{4.878498in}{0.745264in}}%
\pgfpathlineto{\pgfqpoint{4.879063in}{0.745260in}}%
\pgfpathlineto{\pgfqpoint{4.879627in}{0.745257in}}%
\pgfpathlineto{\pgfqpoint{4.880191in}{0.745254in}}%
\pgfpathlineto{\pgfqpoint{4.880756in}{0.745251in}}%
\pgfpathlineto{\pgfqpoint{4.881320in}{0.745247in}}%
\pgfpathlineto{\pgfqpoint{4.881884in}{0.745244in}}%
\pgfpathlineto{\pgfqpoint{4.882449in}{0.745241in}}%
\pgfpathlineto{\pgfqpoint{4.883013in}{0.745237in}}%
\pgfpathlineto{\pgfqpoint{4.883577in}{0.745234in}}%
\pgfpathlineto{\pgfqpoint{4.884142in}{0.745231in}}%
\pgfpathlineto{\pgfqpoint{4.884706in}{0.745228in}}%
\pgfpathlineto{\pgfqpoint{4.885270in}{0.745224in}}%
\pgfpathlineto{\pgfqpoint{4.885835in}{0.745221in}}%
\pgfpathlineto{\pgfqpoint{4.886399in}{0.745218in}}%
\pgfpathlineto{\pgfqpoint{4.886963in}{0.745214in}}%
\pgfpathlineto{\pgfqpoint{4.887528in}{0.745211in}}%
\pgfpathlineto{\pgfqpoint{4.888092in}{0.745208in}}%
\pgfpathlineto{\pgfqpoint{4.888656in}{0.745205in}}%
\pgfpathlineto{\pgfqpoint{4.889221in}{0.745201in}}%
\pgfpathlineto{\pgfqpoint{4.889785in}{0.745198in}}%
\pgfpathlineto{\pgfqpoint{4.890350in}{0.745195in}}%
\pgfpathlineto{\pgfqpoint{4.890914in}{0.745191in}}%
\pgfpathlineto{\pgfqpoint{4.891478in}{0.745188in}}%
\pgfpathlineto{\pgfqpoint{4.892043in}{0.745185in}}%
\pgfpathlineto{\pgfqpoint{4.892607in}{0.745182in}}%
\pgfpathlineto{\pgfqpoint{4.893171in}{0.745178in}}%
\pgfpathlineto{\pgfqpoint{4.893736in}{0.745175in}}%
\pgfpathlineto{\pgfqpoint{4.894300in}{0.745172in}}%
\pgfpathlineto{\pgfqpoint{4.894864in}{0.745169in}}%
\pgfpathlineto{\pgfqpoint{4.895429in}{0.745165in}}%
\pgfpathlineto{\pgfqpoint{4.895993in}{0.745162in}}%
\pgfpathlineto{\pgfqpoint{4.896557in}{0.745159in}}%
\pgfpathlineto{\pgfqpoint{4.897122in}{0.745155in}}%
\pgfpathlineto{\pgfqpoint{4.897686in}{0.745152in}}%
\pgfpathlineto{\pgfqpoint{4.898250in}{0.745149in}}%
\pgfpathlineto{\pgfqpoint{4.898815in}{0.745146in}}%
\pgfpathlineto{\pgfqpoint{4.899379in}{0.745142in}}%
\pgfpathlineto{\pgfqpoint{4.899943in}{0.745139in}}%
\pgfpathlineto{\pgfqpoint{4.900508in}{0.745136in}}%
\pgfpathlineto{\pgfqpoint{4.901072in}{0.745132in}}%
\pgfpathlineto{\pgfqpoint{4.901636in}{0.745129in}}%
\pgfpathlineto{\pgfqpoint{4.902201in}{0.745126in}}%
\pgfpathlineto{\pgfqpoint{4.902765in}{0.745123in}}%
\pgfpathlineto{\pgfqpoint{4.903329in}{0.745119in}}%
\pgfpathlineto{\pgfqpoint{4.903894in}{0.745116in}}%
\pgfpathlineto{\pgfqpoint{4.904458in}{0.745113in}}%
\pgfpathlineto{\pgfqpoint{4.905023in}{0.745109in}}%
\pgfpathlineto{\pgfqpoint{4.905587in}{0.745106in}}%
\pgfpathlineto{\pgfqpoint{4.906151in}{0.745103in}}%
\pgfpathlineto{\pgfqpoint{4.906716in}{0.745100in}}%
\pgfpathlineto{\pgfqpoint{4.907280in}{0.745096in}}%
\pgfpathlineto{\pgfqpoint{4.907844in}{0.745093in}}%
\pgfpathlineto{\pgfqpoint{4.908409in}{0.745090in}}%
\pgfpathlineto{\pgfqpoint{4.908973in}{0.745087in}}%
\pgfpathlineto{\pgfqpoint{4.909537in}{0.745083in}}%
\pgfpathlineto{\pgfqpoint{4.910102in}{0.745080in}}%
\pgfpathlineto{\pgfqpoint{4.910666in}{0.745077in}}%
\pgfpathlineto{\pgfqpoint{4.911230in}{0.745073in}}%
\pgfpathlineto{\pgfqpoint{4.911795in}{0.745070in}}%
\pgfpathlineto{\pgfqpoint{4.912359in}{0.745067in}}%
\pgfpathlineto{\pgfqpoint{4.912923in}{0.745064in}}%
\pgfpathlineto{\pgfqpoint{4.913488in}{0.745060in}}%
\pgfpathlineto{\pgfqpoint{4.914052in}{0.745057in}}%
\pgfpathlineto{\pgfqpoint{4.914616in}{0.745054in}}%
\pgfpathlineto{\pgfqpoint{4.915181in}{0.745050in}}%
\pgfpathlineto{\pgfqpoint{4.915745in}{0.745047in}}%
\pgfpathlineto{\pgfqpoint{4.916309in}{0.745044in}}%
\pgfpathlineto{\pgfqpoint{4.916874in}{0.745041in}}%
\pgfpathlineto{\pgfqpoint{4.917438in}{0.745037in}}%
\pgfpathlineto{\pgfqpoint{4.918002in}{0.745034in}}%
\pgfpathlineto{\pgfqpoint{4.918567in}{0.745031in}}%
\pgfpathlineto{\pgfqpoint{4.919131in}{0.745027in}}%
\pgfpathlineto{\pgfqpoint{4.919695in}{0.745024in}}%
\pgfpathlineto{\pgfqpoint{4.920260in}{0.745021in}}%
\pgfpathlineto{\pgfqpoint{4.920824in}{0.745018in}}%
\pgfpathlineto{\pgfqpoint{4.921389in}{0.745014in}}%
\pgfpathlineto{\pgfqpoint{4.921953in}{0.745011in}}%
\pgfpathlineto{\pgfqpoint{4.922517in}{0.745008in}}%
\pgfpathlineto{\pgfqpoint{4.923082in}{0.745004in}}%
\pgfpathlineto{\pgfqpoint{4.923646in}{0.745001in}}%
\pgfpathlineto{\pgfqpoint{4.924210in}{0.744998in}}%
\pgfpathlineto{\pgfqpoint{4.924775in}{0.744995in}}%
\pgfpathlineto{\pgfqpoint{4.925339in}{0.744991in}}%
\pgfpathlineto{\pgfqpoint{4.925903in}{0.744988in}}%
\pgfpathlineto{\pgfqpoint{4.926468in}{0.744985in}}%
\pgfpathlineto{\pgfqpoint{4.927032in}{0.744982in}}%
\pgfpathlineto{\pgfqpoint{4.927596in}{0.744978in}}%
\pgfpathlineto{\pgfqpoint{4.928161in}{0.744975in}}%
\pgfpathlineto{\pgfqpoint{4.928725in}{0.744972in}}%
\pgfpathlineto{\pgfqpoint{4.929289in}{0.744968in}}%
\pgfpathlineto{\pgfqpoint{4.929854in}{0.744965in}}%
\pgfpathlineto{\pgfqpoint{4.930418in}{0.744962in}}%
\pgfpathlineto{\pgfqpoint{4.930982in}{0.744959in}}%
\pgfpathlineto{\pgfqpoint{4.931547in}{0.744955in}}%
\pgfpathlineto{\pgfqpoint{4.932111in}{0.744952in}}%
\pgfpathlineto{\pgfqpoint{4.932675in}{0.744949in}}%
\pgfpathlineto{\pgfqpoint{4.933240in}{0.744945in}}%
\pgfpathlineto{\pgfqpoint{4.933804in}{0.744942in}}%
\pgfpathlineto{\pgfqpoint{4.934368in}{0.744939in}}%
\pgfpathlineto{\pgfqpoint{4.934933in}{0.744936in}}%
\pgfpathlineto{\pgfqpoint{4.935497in}{0.744932in}}%
\pgfpathlineto{\pgfqpoint{4.936062in}{0.744929in}}%
\pgfpathlineto{\pgfqpoint{4.936626in}{0.744926in}}%
\pgfpathlineto{\pgfqpoint{4.937190in}{0.744922in}}%
\pgfpathlineto{\pgfqpoint{4.937755in}{0.744919in}}%
\pgfpathlineto{\pgfqpoint{4.938319in}{0.744916in}}%
\pgfpathlineto{\pgfqpoint{4.938883in}{0.744913in}}%
\pgfpathlineto{\pgfqpoint{4.939448in}{0.744909in}}%
\pgfpathlineto{\pgfqpoint{4.940012in}{0.744906in}}%
\pgfpathlineto{\pgfqpoint{4.940576in}{0.744903in}}%
\pgfpathlineto{\pgfqpoint{4.941141in}{0.744900in}}%
\pgfpathlineto{\pgfqpoint{4.941705in}{0.744896in}}%
\pgfpathlineto{\pgfqpoint{4.942269in}{0.744893in}}%
\pgfpathlineto{\pgfqpoint{4.942834in}{0.744890in}}%
\pgfpathlineto{\pgfqpoint{4.943398in}{0.744886in}}%
\pgfpathlineto{\pgfqpoint{4.943962in}{0.744883in}}%
\pgfpathlineto{\pgfqpoint{4.944527in}{0.744880in}}%
\pgfpathlineto{\pgfqpoint{4.945091in}{0.744877in}}%
\pgfpathlineto{\pgfqpoint{4.945655in}{0.744873in}}%
\pgfpathlineto{\pgfqpoint{4.946220in}{0.744870in}}%
\pgfpathlineto{\pgfqpoint{4.946784in}{0.744867in}}%
\pgfpathlineto{\pgfqpoint{4.947348in}{0.744863in}}%
\pgfpathlineto{\pgfqpoint{4.947913in}{0.744860in}}%
\pgfpathlineto{\pgfqpoint{4.948477in}{0.744857in}}%
\pgfpathlineto{\pgfqpoint{4.949041in}{0.744854in}}%
\pgfpathlineto{\pgfqpoint{4.949606in}{0.744850in}}%
\pgfpathlineto{\pgfqpoint{4.950170in}{0.744847in}}%
\pgfpathlineto{\pgfqpoint{4.950735in}{0.744844in}}%
\pgfpathlineto{\pgfqpoint{4.951299in}{0.744840in}}%
\pgfpathlineto{\pgfqpoint{4.951863in}{0.744837in}}%
\pgfpathlineto{\pgfqpoint{4.952428in}{0.744834in}}%
\pgfpathlineto{\pgfqpoint{4.952992in}{0.744831in}}%
\pgfpathlineto{\pgfqpoint{4.953556in}{0.744827in}}%
\pgfpathlineto{\pgfqpoint{4.954121in}{0.744824in}}%
\pgfpathlineto{\pgfqpoint{4.954685in}{0.744821in}}%
\pgfpathlineto{\pgfqpoint{4.955249in}{0.744817in}}%
\pgfpathlineto{\pgfqpoint{4.955814in}{0.744814in}}%
\pgfpathlineto{\pgfqpoint{4.956378in}{0.744811in}}%
\pgfpathlineto{\pgfqpoint{4.956942in}{0.744808in}}%
\pgfpathlineto{\pgfqpoint{4.957507in}{0.744804in}}%
\pgfpathlineto{\pgfqpoint{4.958071in}{0.744801in}}%
\pgfpathlineto{\pgfqpoint{4.958635in}{0.744798in}}%
\pgfpathlineto{\pgfqpoint{4.959200in}{0.744795in}}%
\pgfpathlineto{\pgfqpoint{4.959764in}{0.744791in}}%
\pgfpathlineto{\pgfqpoint{4.960328in}{0.744788in}}%
\pgfpathlineto{\pgfqpoint{4.960893in}{0.744785in}}%
\pgfpathlineto{\pgfqpoint{4.961457in}{0.744781in}}%
\pgfpathlineto{\pgfqpoint{4.962021in}{0.744778in}}%
\pgfpathlineto{\pgfqpoint{4.962586in}{0.744775in}}%
\pgfpathlineto{\pgfqpoint{4.963150in}{0.744772in}}%
\pgfpathlineto{\pgfqpoint{4.963714in}{0.744768in}}%
\pgfpathlineto{\pgfqpoint{4.964279in}{0.744765in}}%
\pgfpathlineto{\pgfqpoint{4.964843in}{0.744762in}}%
\pgfpathlineto{\pgfqpoint{4.965408in}{0.744758in}}%
\pgfpathlineto{\pgfqpoint{4.965972in}{0.744755in}}%
\pgfpathlineto{\pgfqpoint{4.966536in}{0.744752in}}%
\pgfpathlineto{\pgfqpoint{4.967101in}{0.744749in}}%
\pgfpathlineto{\pgfqpoint{4.967665in}{0.744745in}}%
\pgfpathlineto{\pgfqpoint{4.968229in}{0.744742in}}%
\pgfpathlineto{\pgfqpoint{4.968794in}{0.744739in}}%
\pgfpathlineto{\pgfqpoint{4.969358in}{0.744735in}}%
\pgfpathlineto{\pgfqpoint{4.969922in}{0.744732in}}%
\pgfpathlineto{\pgfqpoint{4.970487in}{0.744729in}}%
\pgfpathlineto{\pgfqpoint{4.971051in}{0.744726in}}%
\pgfpathlineto{\pgfqpoint{4.971615in}{0.744722in}}%
\pgfpathlineto{\pgfqpoint{4.972180in}{0.744719in}}%
\pgfpathlineto{\pgfqpoint{4.972744in}{0.744716in}}%
\pgfpathlineto{\pgfqpoint{4.973308in}{0.744712in}}%
\pgfpathlineto{\pgfqpoint{4.973873in}{0.744709in}}%
\pgfpathlineto{\pgfqpoint{4.974437in}{0.744706in}}%
\pgfpathlineto{\pgfqpoint{4.975001in}{0.744703in}}%
\pgfpathlineto{\pgfqpoint{4.975566in}{0.744699in}}%
\pgfpathlineto{\pgfqpoint{4.976130in}{0.744696in}}%
\pgfpathlineto{\pgfqpoint{4.976694in}{0.744693in}}%
\pgfpathlineto{\pgfqpoint{4.977259in}{0.744690in}}%
\pgfpathlineto{\pgfqpoint{4.977823in}{0.744686in}}%
\pgfpathlineto{\pgfqpoint{4.978387in}{0.744683in}}%
\pgfpathlineto{\pgfqpoint{4.978952in}{0.744680in}}%
\pgfpathlineto{\pgfqpoint{4.979516in}{0.744676in}}%
\pgfpathlineto{\pgfqpoint{4.980080in}{0.744673in}}%
\pgfpathlineto{\pgfqpoint{4.980645in}{0.744670in}}%
\pgfpathlineto{\pgfqpoint{4.981209in}{0.744667in}}%
\pgfpathlineto{\pgfqpoint{4.981774in}{0.744663in}}%
\pgfpathlineto{\pgfqpoint{4.982338in}{0.744660in}}%
\pgfpathlineto{\pgfqpoint{4.982902in}{0.744657in}}%
\pgfpathlineto{\pgfqpoint{4.983467in}{0.744653in}}%
\pgfpathlineto{\pgfqpoint{4.984031in}{0.744650in}}%
\pgfpathlineto{\pgfqpoint{4.984595in}{0.744647in}}%
\pgfpathlineto{\pgfqpoint{4.985160in}{0.744644in}}%
\pgfpathlineto{\pgfqpoint{4.985724in}{0.744640in}}%
\pgfpathlineto{\pgfqpoint{4.986288in}{0.744637in}}%
\pgfpathlineto{\pgfqpoint{4.986853in}{0.744634in}}%
\pgfpathlineto{\pgfqpoint{4.987417in}{0.744630in}}%
\pgfpathlineto{\pgfqpoint{4.987981in}{0.744627in}}%
\pgfpathlineto{\pgfqpoint{4.988546in}{0.744624in}}%
\pgfpathlineto{\pgfqpoint{4.989110in}{0.744620in}}%
\pgfpathlineto{\pgfqpoint{4.989674in}{0.744617in}}%
\pgfpathlineto{\pgfqpoint{4.990239in}{0.744614in}}%
\pgfpathlineto{\pgfqpoint{4.990803in}{0.744610in}}%
\pgfpathlineto{\pgfqpoint{4.991367in}{0.744607in}}%
\pgfpathlineto{\pgfqpoint{4.991932in}{0.744604in}}%
\pgfpathlineto{\pgfqpoint{4.992496in}{0.744600in}}%
\pgfpathlineto{\pgfqpoint{4.993060in}{0.744597in}}%
\pgfpathlineto{\pgfqpoint{4.993625in}{0.744594in}}%
\pgfpathlineto{\pgfqpoint{4.994189in}{0.744590in}}%
\pgfpathlineto{\pgfqpoint{4.994753in}{0.744587in}}%
\pgfpathlineto{\pgfqpoint{4.995318in}{0.744583in}}%
\pgfpathlineto{\pgfqpoint{4.995882in}{0.744580in}}%
\pgfpathlineto{\pgfqpoint{4.996447in}{0.744577in}}%
\pgfpathlineto{\pgfqpoint{4.997011in}{0.744573in}}%
\pgfpathlineto{\pgfqpoint{4.997575in}{0.744570in}}%
\pgfpathlineto{\pgfqpoint{4.998140in}{0.744567in}}%
\pgfpathlineto{\pgfqpoint{4.998704in}{0.744563in}}%
\pgfpathlineto{\pgfqpoint{4.999268in}{0.744560in}}%
\pgfpathlineto{\pgfqpoint{4.999833in}{0.744557in}}%
\pgfpathlineto{\pgfqpoint{5.000397in}{0.744553in}}%
\pgfpathlineto{\pgfqpoint{5.000961in}{0.744550in}}%
\pgfpathlineto{\pgfqpoint{5.001526in}{0.744547in}}%
\pgfpathlineto{\pgfqpoint{5.002090in}{0.744543in}}%
\pgfpathlineto{\pgfqpoint{5.002654in}{0.744540in}}%
\pgfpathlineto{\pgfqpoint{5.003219in}{0.744537in}}%
\pgfpathlineto{\pgfqpoint{5.003783in}{0.744533in}}%
\pgfpathlineto{\pgfqpoint{5.004347in}{0.744530in}}%
\pgfpathlineto{\pgfqpoint{5.004912in}{0.744526in}}%
\pgfpathlineto{\pgfqpoint{5.005476in}{0.744523in}}%
\pgfpathlineto{\pgfqpoint{5.006040in}{0.744520in}}%
\pgfpathlineto{\pgfqpoint{5.006605in}{0.744516in}}%
\pgfpathlineto{\pgfqpoint{5.007169in}{0.744513in}}%
\pgfpathlineto{\pgfqpoint{5.007733in}{0.744510in}}%
\pgfpathlineto{\pgfqpoint{5.008298in}{0.744506in}}%
\pgfpathlineto{\pgfqpoint{5.008862in}{0.744503in}}%
\pgfpathlineto{\pgfqpoint{5.009426in}{0.744500in}}%
\pgfpathlineto{\pgfqpoint{5.009991in}{0.744496in}}%
\pgfpathlineto{\pgfqpoint{5.010555in}{0.744493in}}%
\pgfpathlineto{\pgfqpoint{5.011120in}{0.744490in}}%
\pgfpathlineto{\pgfqpoint{5.011684in}{0.744486in}}%
\pgfpathlineto{\pgfqpoint{5.012248in}{0.744483in}}%
\pgfpathlineto{\pgfqpoint{5.012813in}{0.744480in}}%
\pgfpathlineto{\pgfqpoint{5.013377in}{0.744476in}}%
\pgfpathlineto{\pgfqpoint{5.013941in}{0.744473in}}%
\pgfpathlineto{\pgfqpoint{5.014506in}{0.744469in}}%
\pgfpathlineto{\pgfqpoint{5.015070in}{0.744466in}}%
\pgfpathlineto{\pgfqpoint{5.015634in}{0.744463in}}%
\pgfpathlineto{\pgfqpoint{5.016199in}{0.744459in}}%
\pgfpathlineto{\pgfqpoint{5.016763in}{0.744456in}}%
\pgfpathlineto{\pgfqpoint{5.017327in}{0.744453in}}%
\pgfpathlineto{\pgfqpoint{5.017892in}{0.744449in}}%
\pgfpathlineto{\pgfqpoint{5.018456in}{0.744446in}}%
\pgfpathlineto{\pgfqpoint{5.019020in}{0.744443in}}%
\pgfpathlineto{\pgfqpoint{5.019585in}{0.744439in}}%
\pgfpathlineto{\pgfqpoint{5.020149in}{0.744436in}}%
\pgfpathlineto{\pgfqpoint{5.020713in}{0.744433in}}%
\pgfpathlineto{\pgfqpoint{5.021278in}{0.744429in}}%
\pgfpathlineto{\pgfqpoint{5.021842in}{0.744426in}}%
\pgfpathlineto{\pgfqpoint{5.022406in}{0.744423in}}%
\pgfpathlineto{\pgfqpoint{5.022971in}{0.744419in}}%
\pgfpathlineto{\pgfqpoint{5.023535in}{0.744416in}}%
\pgfpathlineto{\pgfqpoint{5.024099in}{0.744413in}}%
\pgfpathlineto{\pgfqpoint{5.024664in}{0.744409in}}%
\pgfpathlineto{\pgfqpoint{5.025228in}{0.744406in}}%
\pgfpathlineto{\pgfqpoint{5.025792in}{0.744402in}}%
\pgfpathlineto{\pgfqpoint{5.026357in}{0.744399in}}%
\pgfpathlineto{\pgfqpoint{5.026921in}{0.744396in}}%
\pgfpathlineto{\pgfqpoint{5.027486in}{0.744392in}}%
\pgfpathlineto{\pgfqpoint{5.028050in}{0.744389in}}%
\pgfpathlineto{\pgfqpoint{5.028614in}{0.744386in}}%
\pgfpathlineto{\pgfqpoint{5.029179in}{0.744382in}}%
\pgfpathlineto{\pgfqpoint{5.029743in}{0.744379in}}%
\pgfpathlineto{\pgfqpoint{5.030307in}{0.744376in}}%
\pgfpathlineto{\pgfqpoint{5.030872in}{0.744372in}}%
\pgfpathlineto{\pgfqpoint{5.031436in}{0.744369in}}%
\pgfpathlineto{\pgfqpoint{5.032000in}{0.744366in}}%
\pgfpathlineto{\pgfqpoint{5.032565in}{0.744362in}}%
\pgfpathlineto{\pgfqpoint{5.033129in}{0.744359in}}%
\pgfpathlineto{\pgfqpoint{5.033693in}{0.744356in}}%
\pgfpathlineto{\pgfqpoint{5.034258in}{0.744352in}}%
\pgfpathlineto{\pgfqpoint{5.034822in}{0.744349in}}%
\pgfpathlineto{\pgfqpoint{5.035386in}{0.744345in}}%
\pgfpathlineto{\pgfqpoint{5.035951in}{0.744342in}}%
\pgfpathlineto{\pgfqpoint{5.036515in}{0.744339in}}%
\pgfpathlineto{\pgfqpoint{5.037079in}{0.744335in}}%
\pgfpathlineto{\pgfqpoint{5.037644in}{0.744332in}}%
\pgfpathlineto{\pgfqpoint{5.038208in}{0.744329in}}%
\pgfpathlineto{\pgfqpoint{5.038772in}{0.744325in}}%
\pgfpathlineto{\pgfqpoint{5.039337in}{0.744322in}}%
\pgfpathlineto{\pgfqpoint{5.039901in}{0.744319in}}%
\pgfpathlineto{\pgfqpoint{5.040465in}{0.744315in}}%
\pgfpathlineto{\pgfqpoint{5.041030in}{0.744312in}}%
\pgfpathlineto{\pgfqpoint{5.041594in}{0.744309in}}%
\pgfpathlineto{\pgfqpoint{5.042159in}{0.744305in}}%
\pgfpathlineto{\pgfqpoint{5.042723in}{0.744302in}}%
\pgfpathlineto{\pgfqpoint{5.043287in}{0.744299in}}%
\pgfpathlineto{\pgfqpoint{5.043852in}{0.744295in}}%
\pgfpathlineto{\pgfqpoint{5.044416in}{0.744292in}}%
\pgfpathlineto{\pgfqpoint{5.044980in}{0.744288in}}%
\pgfpathlineto{\pgfqpoint{5.045545in}{0.744285in}}%
\pgfpathlineto{\pgfqpoint{5.046109in}{0.744282in}}%
\pgfpathlineto{\pgfqpoint{5.046673in}{0.744278in}}%
\pgfpathlineto{\pgfqpoint{5.047238in}{0.744275in}}%
\pgfpathlineto{\pgfqpoint{5.047802in}{0.744272in}}%
\pgfpathlineto{\pgfqpoint{5.048366in}{0.744268in}}%
\pgfpathlineto{\pgfqpoint{5.048931in}{0.744265in}}%
\pgfpathlineto{\pgfqpoint{5.049495in}{0.744262in}}%
\pgfpathlineto{\pgfqpoint{5.050059in}{0.744258in}}%
\pgfpathlineto{\pgfqpoint{5.050624in}{0.744255in}}%
\pgfpathlineto{\pgfqpoint{5.051188in}{0.744252in}}%
\pgfpathlineto{\pgfqpoint{5.051752in}{0.744248in}}%
\pgfpathlineto{\pgfqpoint{5.052317in}{0.744245in}}%
\pgfpathlineto{\pgfqpoint{5.052881in}{0.744242in}}%
\pgfpathlineto{\pgfqpoint{5.053445in}{0.744238in}}%
\pgfpathlineto{\pgfqpoint{5.054010in}{0.744235in}}%
\pgfpathlineto{\pgfqpoint{5.054574in}{0.744231in}}%
\pgfpathlineto{\pgfqpoint{5.055138in}{0.744228in}}%
\pgfpathlineto{\pgfqpoint{5.055703in}{0.744225in}}%
\pgfpathlineto{\pgfqpoint{5.056267in}{0.744221in}}%
\pgfpathlineto{\pgfqpoint{5.056832in}{0.744218in}}%
\pgfpathlineto{\pgfqpoint{5.057396in}{0.744215in}}%
\pgfpathlineto{\pgfqpoint{5.057960in}{0.744211in}}%
\pgfpathlineto{\pgfqpoint{5.058525in}{0.744208in}}%
\pgfpathlineto{\pgfqpoint{5.059089in}{0.744205in}}%
\pgfpathlineto{\pgfqpoint{5.059653in}{0.744201in}}%
\pgfpathlineto{\pgfqpoint{5.060218in}{0.744198in}}%
\pgfpathlineto{\pgfqpoint{5.060782in}{0.744195in}}%
\pgfpathlineto{\pgfqpoint{5.061346in}{0.744191in}}%
\pgfpathlineto{\pgfqpoint{5.061911in}{0.744188in}}%
\pgfpathlineto{\pgfqpoint{5.062475in}{0.744185in}}%
\pgfpathlineto{\pgfqpoint{5.063039in}{0.744181in}}%
\pgfpathlineto{\pgfqpoint{5.063604in}{0.744178in}}%
\pgfpathlineto{\pgfqpoint{5.064168in}{0.744175in}}%
\pgfpathlineto{\pgfqpoint{5.064732in}{0.744171in}}%
\pgfpathlineto{\pgfqpoint{5.065297in}{0.744168in}}%
\pgfpathlineto{\pgfqpoint{5.065861in}{0.744164in}}%
\pgfpathlineto{\pgfqpoint{5.066425in}{0.744161in}}%
\pgfpathlineto{\pgfqpoint{5.066990in}{0.744158in}}%
\pgfpathlineto{\pgfqpoint{5.067554in}{0.744154in}}%
\pgfpathlineto{\pgfqpoint{5.068118in}{0.744151in}}%
\pgfpathlineto{\pgfqpoint{5.068683in}{0.744148in}}%
\pgfpathlineto{\pgfqpoint{5.069247in}{0.744144in}}%
\pgfpathlineto{\pgfqpoint{5.069811in}{0.744141in}}%
\pgfpathlineto{\pgfqpoint{5.070376in}{0.744138in}}%
\pgfpathlineto{\pgfqpoint{5.070940in}{0.744134in}}%
\pgfpathlineto{\pgfqpoint{5.071504in}{0.744131in}}%
\pgfpathlineto{\pgfqpoint{5.072069in}{0.744128in}}%
\pgfpathlineto{\pgfqpoint{5.072633in}{0.744124in}}%
\pgfpathlineto{\pgfqpoint{5.073198in}{0.744121in}}%
\pgfpathlineto{\pgfqpoint{5.073762in}{0.744118in}}%
\pgfpathlineto{\pgfqpoint{5.074326in}{0.744114in}}%
\pgfpathlineto{\pgfqpoint{5.074891in}{0.744111in}}%
\pgfpathlineto{\pgfqpoint{5.075455in}{0.744107in}}%
\pgfpathlineto{\pgfqpoint{5.076019in}{0.744104in}}%
\pgfpathlineto{\pgfqpoint{5.076584in}{0.744101in}}%
\pgfpathlineto{\pgfqpoint{5.077148in}{0.744097in}}%
\pgfpathlineto{\pgfqpoint{5.077712in}{0.744094in}}%
\pgfpathlineto{\pgfqpoint{5.078277in}{0.744091in}}%
\pgfpathlineto{\pgfqpoint{5.078841in}{0.744087in}}%
\pgfpathlineto{\pgfqpoint{5.079405in}{0.744084in}}%
\pgfpathlineto{\pgfqpoint{5.079970in}{0.744081in}}%
\pgfpathlineto{\pgfqpoint{5.080534in}{0.744077in}}%
\pgfpathlineto{\pgfqpoint{5.081098in}{0.744083in}}%
\pgfpathlineto{\pgfqpoint{5.081663in}{0.744126in}}%
\pgfpathlineto{\pgfqpoint{5.082227in}{0.744162in}}%
\pgfpathlineto{\pgfqpoint{5.082791in}{0.744161in}}%
\pgfpathlineto{\pgfqpoint{5.083356in}{0.744158in}}%
\pgfpathlineto{\pgfqpoint{5.083920in}{0.744154in}}%
\pgfpathlineto{\pgfqpoint{5.084484in}{0.744151in}}%
\pgfpathlineto{\pgfqpoint{5.085049in}{0.744147in}}%
\pgfpathlineto{\pgfqpoint{5.085613in}{0.744144in}}%
\pgfpathlineto{\pgfqpoint{5.086177in}{0.744141in}}%
\pgfpathlineto{\pgfqpoint{5.086742in}{0.744137in}}%
\pgfpathlineto{\pgfqpoint{5.087306in}{0.744134in}}%
\pgfpathlineto{\pgfqpoint{5.087871in}{0.744130in}}%
\pgfpathlineto{\pgfqpoint{5.088435in}{0.744127in}}%
\pgfpathlineto{\pgfqpoint{5.088999in}{0.744124in}}%
\pgfpathlineto{\pgfqpoint{5.089564in}{0.744120in}}%
\pgfpathlineto{\pgfqpoint{5.090128in}{0.744117in}}%
\pgfpathlineto{\pgfqpoint{5.090692in}{0.744114in}}%
\pgfpathlineto{\pgfqpoint{5.091257in}{0.744110in}}%
\pgfpathlineto{\pgfqpoint{5.091821in}{0.744107in}}%
\pgfpathlineto{\pgfqpoint{5.092385in}{0.744103in}}%
\pgfpathlineto{\pgfqpoint{5.092950in}{0.744100in}}%
\pgfpathlineto{\pgfqpoint{5.093514in}{0.744097in}}%
\pgfpathlineto{\pgfqpoint{5.094078in}{0.744093in}}%
\pgfpathlineto{\pgfqpoint{5.094643in}{0.744090in}}%
\pgfpathlineto{\pgfqpoint{5.095207in}{0.744086in}}%
\pgfpathlineto{\pgfqpoint{5.095771in}{0.744083in}}%
\pgfpathlineto{\pgfqpoint{5.096336in}{0.744080in}}%
\pgfpathlineto{\pgfqpoint{5.096900in}{0.744076in}}%
\pgfpathlineto{\pgfqpoint{5.097464in}{0.744073in}}%
\pgfpathlineto{\pgfqpoint{5.098029in}{0.744069in}}%
\pgfpathlineto{\pgfqpoint{5.098593in}{0.744066in}}%
\pgfpathlineto{\pgfqpoint{5.099157in}{0.744063in}}%
\pgfpathlineto{\pgfqpoint{5.099722in}{0.744059in}}%
\pgfpathlineto{\pgfqpoint{5.100286in}{0.744056in}}%
\pgfpathlineto{\pgfqpoint{5.100850in}{0.744053in}}%
\pgfpathlineto{\pgfqpoint{5.101415in}{0.744049in}}%
\pgfpathlineto{\pgfqpoint{5.101979in}{0.744046in}}%
\pgfpathlineto{\pgfqpoint{5.102544in}{0.744042in}}%
\pgfpathlineto{\pgfqpoint{5.103108in}{0.744039in}}%
\pgfpathlineto{\pgfqpoint{5.103672in}{0.744036in}}%
\pgfpathlineto{\pgfqpoint{5.104237in}{0.744032in}}%
\pgfpathlineto{\pgfqpoint{5.104801in}{0.744029in}}%
\pgfpathlineto{\pgfqpoint{5.105365in}{0.744025in}}%
\pgfpathlineto{\pgfqpoint{5.105930in}{0.744022in}}%
\pgfpathlineto{\pgfqpoint{5.106494in}{0.744019in}}%
\pgfpathlineto{\pgfqpoint{5.107058in}{0.744015in}}%
\pgfpathlineto{\pgfqpoint{5.107623in}{0.744012in}}%
\pgfpathlineto{\pgfqpoint{5.108187in}{0.744008in}}%
\pgfpathlineto{\pgfqpoint{5.108751in}{0.744005in}}%
\pgfpathlineto{\pgfqpoint{5.109316in}{0.744002in}}%
\pgfpathlineto{\pgfqpoint{5.109880in}{0.743998in}}%
\pgfpathlineto{\pgfqpoint{5.110444in}{0.743995in}}%
\pgfpathlineto{\pgfqpoint{5.111009in}{0.743992in}}%
\pgfpathlineto{\pgfqpoint{5.111573in}{0.743988in}}%
\pgfpathlineto{\pgfqpoint{5.112137in}{0.743985in}}%
\pgfpathlineto{\pgfqpoint{5.112702in}{0.743981in}}%
\pgfpathlineto{\pgfqpoint{5.113266in}{0.743978in}}%
\pgfpathlineto{\pgfqpoint{5.113830in}{0.743975in}}%
\pgfpathlineto{\pgfqpoint{5.114395in}{0.743971in}}%
\pgfpathlineto{\pgfqpoint{5.114959in}{0.743968in}}%
\pgfpathlineto{\pgfqpoint{5.115523in}{0.743964in}}%
\pgfpathlineto{\pgfqpoint{5.116088in}{0.743961in}}%
\pgfpathlineto{\pgfqpoint{5.116652in}{0.743958in}}%
\pgfpathlineto{\pgfqpoint{5.117216in}{0.743954in}}%
\pgfpathlineto{\pgfqpoint{5.117781in}{0.743951in}}%
\pgfpathlineto{\pgfqpoint{5.118345in}{0.743948in}}%
\pgfpathlineto{\pgfqpoint{5.118910in}{0.743944in}}%
\pgfpathlineto{\pgfqpoint{5.119474in}{0.743941in}}%
\pgfpathlineto{\pgfqpoint{5.120038in}{0.743937in}}%
\pgfpathlineto{\pgfqpoint{5.120603in}{0.743934in}}%
\pgfpathlineto{\pgfqpoint{5.121167in}{0.743931in}}%
\pgfpathlineto{\pgfqpoint{5.121731in}{0.743927in}}%
\pgfpathlineto{\pgfqpoint{5.122296in}{0.743924in}}%
\pgfpathlineto{\pgfqpoint{5.122860in}{0.743920in}}%
\pgfpathlineto{\pgfqpoint{5.123424in}{0.743917in}}%
\pgfpathlineto{\pgfqpoint{5.123989in}{0.743914in}}%
\pgfpathlineto{\pgfqpoint{5.124553in}{0.743910in}}%
\pgfpathlineto{\pgfqpoint{5.125117in}{0.743907in}}%
\pgfpathlineto{\pgfqpoint{5.125682in}{0.743903in}}%
\pgfpathlineto{\pgfqpoint{5.126246in}{0.743900in}}%
\pgfpathlineto{\pgfqpoint{5.126810in}{0.743897in}}%
\pgfpathlineto{\pgfqpoint{5.127375in}{0.743893in}}%
\pgfpathlineto{\pgfqpoint{5.127939in}{0.743890in}}%
\pgfpathlineto{\pgfqpoint{5.128503in}{0.743887in}}%
\pgfpathlineto{\pgfqpoint{5.129068in}{0.743883in}}%
\pgfpathlineto{\pgfqpoint{5.129632in}{0.743880in}}%
\pgfpathlineto{\pgfqpoint{5.130196in}{0.743876in}}%
\pgfpathlineto{\pgfqpoint{5.130761in}{0.743873in}}%
\pgfpathlineto{\pgfqpoint{5.131325in}{0.743870in}}%
\pgfpathlineto{\pgfqpoint{5.131889in}{0.743866in}}%
\pgfpathlineto{\pgfqpoint{5.132454in}{0.743863in}}%
\pgfpathlineto{\pgfqpoint{5.133018in}{0.743859in}}%
\pgfpathlineto{\pgfqpoint{5.133583in}{0.743856in}}%
\pgfpathlineto{\pgfqpoint{5.134147in}{0.743853in}}%
\pgfpathlineto{\pgfqpoint{5.134711in}{0.743849in}}%
\pgfpathlineto{\pgfqpoint{5.135276in}{0.743846in}}%
\pgfpathlineto{\pgfqpoint{5.135840in}{0.743842in}}%
\pgfpathlineto{\pgfqpoint{5.136404in}{0.743839in}}%
\pgfpathlineto{\pgfqpoint{5.136969in}{0.743836in}}%
\pgfpathlineto{\pgfqpoint{5.137533in}{0.743832in}}%
\pgfpathlineto{\pgfqpoint{5.138097in}{0.743829in}}%
\pgfpathlineto{\pgfqpoint{5.138662in}{0.743826in}}%
\pgfpathlineto{\pgfqpoint{5.139226in}{0.743822in}}%
\pgfpathlineto{\pgfqpoint{5.139790in}{0.743819in}}%
\pgfpathlineto{\pgfqpoint{5.140355in}{0.743815in}}%
\pgfpathlineto{\pgfqpoint{5.140919in}{0.743812in}}%
\pgfpathlineto{\pgfqpoint{5.141483in}{0.743809in}}%
\pgfpathlineto{\pgfqpoint{5.142048in}{0.743805in}}%
\pgfpathlineto{\pgfqpoint{5.142612in}{0.743802in}}%
\pgfpathlineto{\pgfqpoint{5.143176in}{0.743798in}}%
\pgfpathlineto{\pgfqpoint{5.143741in}{0.743795in}}%
\pgfpathlineto{\pgfqpoint{5.144305in}{0.743792in}}%
\pgfpathlineto{\pgfqpoint{5.144869in}{0.743788in}}%
\pgfpathlineto{\pgfqpoint{5.145434in}{0.743785in}}%
\pgfpathlineto{\pgfqpoint{5.145998in}{0.743782in}}%
\pgfpathlineto{\pgfqpoint{5.146562in}{0.743778in}}%
\pgfpathlineto{\pgfqpoint{5.147127in}{0.743775in}}%
\pgfpathlineto{\pgfqpoint{5.147691in}{0.743771in}}%
\pgfpathlineto{\pgfqpoint{5.148256in}{0.743768in}}%
\pgfpathlineto{\pgfqpoint{5.148820in}{0.743765in}}%
\pgfpathlineto{\pgfqpoint{5.149384in}{0.743761in}}%
\pgfpathlineto{\pgfqpoint{5.149949in}{0.743758in}}%
\pgfpathlineto{\pgfqpoint{5.150513in}{0.743754in}}%
\pgfpathlineto{\pgfqpoint{5.151077in}{0.743751in}}%
\pgfpathlineto{\pgfqpoint{5.151642in}{0.743748in}}%
\pgfpathlineto{\pgfqpoint{5.152206in}{0.743744in}}%
\pgfpathlineto{\pgfqpoint{5.152770in}{0.743741in}}%
\pgfpathlineto{\pgfqpoint{5.153335in}{0.743737in}}%
\pgfpathlineto{\pgfqpoint{5.153899in}{0.743734in}}%
\pgfpathlineto{\pgfqpoint{5.154463in}{0.743731in}}%
\pgfpathlineto{\pgfqpoint{5.155028in}{0.743727in}}%
\pgfpathlineto{\pgfqpoint{5.155592in}{0.743724in}}%
\pgfpathlineto{\pgfqpoint{5.156156in}{0.743721in}}%
\pgfpathlineto{\pgfqpoint{5.156721in}{0.743717in}}%
\pgfpathlineto{\pgfqpoint{5.157285in}{0.743714in}}%
\pgfpathlineto{\pgfqpoint{5.157849in}{0.743710in}}%
\pgfpathlineto{\pgfqpoint{5.158414in}{0.743707in}}%
\pgfpathlineto{\pgfqpoint{5.158978in}{0.743704in}}%
\pgfpathlineto{\pgfqpoint{5.159542in}{0.743700in}}%
\pgfpathlineto{\pgfqpoint{5.160107in}{0.743697in}}%
\pgfpathlineto{\pgfqpoint{5.160671in}{0.743693in}}%
\pgfpathlineto{\pgfqpoint{5.161235in}{0.743690in}}%
\pgfpathlineto{\pgfqpoint{5.161800in}{0.743687in}}%
\pgfpathlineto{\pgfqpoint{5.162364in}{0.743683in}}%
\pgfpathlineto{\pgfqpoint{5.162929in}{0.743680in}}%
\pgfpathlineto{\pgfqpoint{5.163493in}{0.743676in}}%
\pgfpathlineto{\pgfqpoint{5.164057in}{0.743673in}}%
\pgfpathlineto{\pgfqpoint{5.164622in}{0.743670in}}%
\pgfpathlineto{\pgfqpoint{5.165186in}{0.743666in}}%
\pgfpathlineto{\pgfqpoint{5.165750in}{0.743663in}}%
\pgfpathlineto{\pgfqpoint{5.166315in}{0.743660in}}%
\pgfpathlineto{\pgfqpoint{5.166879in}{0.743656in}}%
\pgfpathlineto{\pgfqpoint{5.167443in}{0.743653in}}%
\pgfpathlineto{\pgfqpoint{5.168008in}{0.743649in}}%
\pgfpathlineto{\pgfqpoint{5.168572in}{0.743646in}}%
\pgfpathlineto{\pgfqpoint{5.169136in}{0.743643in}}%
\pgfpathlineto{\pgfqpoint{5.169701in}{0.743639in}}%
\pgfpathlineto{\pgfqpoint{5.170265in}{0.743636in}}%
\pgfpathlineto{\pgfqpoint{5.170829in}{0.743632in}}%
\pgfpathlineto{\pgfqpoint{5.171394in}{0.743629in}}%
\pgfpathlineto{\pgfqpoint{5.171958in}{0.743626in}}%
\pgfpathlineto{\pgfqpoint{5.172522in}{0.743622in}}%
\pgfpathlineto{\pgfqpoint{5.173087in}{0.743619in}}%
\pgfpathlineto{\pgfqpoint{5.173651in}{0.743615in}}%
\pgfpathlineto{\pgfqpoint{5.174215in}{0.743612in}}%
\pgfpathlineto{\pgfqpoint{5.174780in}{0.743609in}}%
\pgfpathlineto{\pgfqpoint{5.175344in}{0.743605in}}%
\pgfpathlineto{\pgfqpoint{5.175908in}{0.743602in}}%
\pgfpathlineto{\pgfqpoint{5.176473in}{0.743599in}}%
\pgfpathlineto{\pgfqpoint{5.177037in}{0.743595in}}%
\pgfpathlineto{\pgfqpoint{5.177601in}{0.743592in}}%
\pgfpathlineto{\pgfqpoint{5.178166in}{0.743588in}}%
\pgfpathlineto{\pgfqpoint{5.178730in}{0.743585in}}%
\pgfpathlineto{\pgfqpoint{5.179295in}{0.743582in}}%
\pgfpathlineto{\pgfqpoint{5.179859in}{0.743578in}}%
\pgfpathlineto{\pgfqpoint{5.180423in}{0.743575in}}%
\pgfpathlineto{\pgfqpoint{5.180988in}{0.743571in}}%
\pgfpathlineto{\pgfqpoint{5.181552in}{0.743568in}}%
\pgfpathlineto{\pgfqpoint{5.182116in}{0.743565in}}%
\pgfpathlineto{\pgfqpoint{5.182681in}{0.743561in}}%
\pgfpathlineto{\pgfqpoint{5.183245in}{0.743558in}}%
\pgfpathlineto{\pgfqpoint{5.183809in}{0.743555in}}%
\pgfpathlineto{\pgfqpoint{5.184374in}{0.743551in}}%
\pgfpathlineto{\pgfqpoint{5.184938in}{0.743548in}}%
\pgfpathlineto{\pgfqpoint{5.185502in}{0.743544in}}%
\pgfpathlineto{\pgfqpoint{5.186067in}{0.743541in}}%
\pgfpathlineto{\pgfqpoint{5.186631in}{0.743538in}}%
\pgfpathlineto{\pgfqpoint{5.187195in}{0.743534in}}%
\pgfpathlineto{\pgfqpoint{5.187760in}{0.743531in}}%
\pgfpathlineto{\pgfqpoint{5.188324in}{0.743527in}}%
\pgfpathlineto{\pgfqpoint{5.188888in}{0.743524in}}%
\pgfpathlineto{\pgfqpoint{5.189453in}{0.743521in}}%
\pgfpathlineto{\pgfqpoint{5.190017in}{0.743517in}}%
\pgfpathlineto{\pgfqpoint{5.190581in}{0.743514in}}%
\pgfpathlineto{\pgfqpoint{5.191146in}{0.743510in}}%
\pgfpathlineto{\pgfqpoint{5.191710in}{0.743508in}}%
\pgfpathlineto{\pgfqpoint{5.192274in}{0.743508in}}%
\pgfpathlineto{\pgfqpoint{5.192839in}{0.743507in}}%
\pgfpathlineto{\pgfqpoint{5.193403in}{0.743507in}}%
\pgfpathlineto{\pgfqpoint{5.193968in}{0.743506in}}%
\pgfpathlineto{\pgfqpoint{5.194532in}{0.743506in}}%
\pgfpathlineto{\pgfqpoint{5.195096in}{0.743505in}}%
\pgfpathlineto{\pgfqpoint{5.195661in}{0.743505in}}%
\pgfpathlineto{\pgfqpoint{5.196225in}{0.743504in}}%
\pgfpathlineto{\pgfqpoint{5.196789in}{0.743504in}}%
\pgfpathlineto{\pgfqpoint{5.197354in}{0.743503in}}%
\pgfpathlineto{\pgfqpoint{5.197918in}{0.743503in}}%
\pgfpathlineto{\pgfqpoint{5.198482in}{0.743502in}}%
\pgfpathlineto{\pgfqpoint{5.199047in}{0.743502in}}%
\pgfpathlineto{\pgfqpoint{5.199611in}{0.743501in}}%
\pgfpathlineto{\pgfqpoint{5.200175in}{0.743501in}}%
\pgfpathlineto{\pgfqpoint{5.200740in}{0.743500in}}%
\pgfpathlineto{\pgfqpoint{5.201304in}{0.743500in}}%
\pgfpathlineto{\pgfqpoint{5.201868in}{0.743499in}}%
\pgfpathlineto{\pgfqpoint{5.202433in}{0.743499in}}%
\pgfpathlineto{\pgfqpoint{5.202997in}{0.743499in}}%
\pgfpathlineto{\pgfqpoint{5.203561in}{0.743498in}}%
\pgfpathlineto{\pgfqpoint{5.204126in}{0.743498in}}%
\pgfpathlineto{\pgfqpoint{5.204690in}{0.743497in}}%
\pgfpathlineto{\pgfqpoint{5.205254in}{0.743497in}}%
\pgfpathlineto{\pgfqpoint{5.205819in}{0.743496in}}%
\pgfpathlineto{\pgfqpoint{5.206383in}{0.743496in}}%
\pgfpathlineto{\pgfqpoint{5.206947in}{0.743495in}}%
\pgfpathlineto{\pgfqpoint{5.207512in}{0.743495in}}%
\pgfpathlineto{\pgfqpoint{5.208076in}{0.743494in}}%
\pgfpathlineto{\pgfqpoint{5.208641in}{0.743494in}}%
\pgfpathlineto{\pgfqpoint{5.209205in}{0.743493in}}%
\pgfpathlineto{\pgfqpoint{5.209769in}{0.743493in}}%
\pgfpathlineto{\pgfqpoint{5.210334in}{0.743492in}}%
\pgfpathlineto{\pgfqpoint{5.210898in}{0.743492in}}%
\pgfpathlineto{\pgfqpoint{5.211462in}{0.743491in}}%
\pgfpathlineto{\pgfqpoint{5.212027in}{0.743491in}}%
\pgfpathlineto{\pgfqpoint{5.212591in}{0.743490in}}%
\pgfpathlineto{\pgfqpoint{5.213155in}{0.743490in}}%
\pgfpathlineto{\pgfqpoint{5.213720in}{0.743489in}}%
\pgfpathlineto{\pgfqpoint{5.214284in}{0.743489in}}%
\pgfpathlineto{\pgfqpoint{5.214848in}{0.743488in}}%
\pgfpathlineto{\pgfqpoint{5.215413in}{0.743488in}}%
\pgfpathlineto{\pgfqpoint{5.215977in}{0.743487in}}%
\pgfpathlineto{\pgfqpoint{5.216541in}{0.743487in}}%
\pgfpathlineto{\pgfqpoint{5.217106in}{0.743486in}}%
\pgfpathlineto{\pgfqpoint{5.217670in}{0.743486in}}%
\pgfpathlineto{\pgfqpoint{5.218234in}{0.743485in}}%
\pgfpathlineto{\pgfqpoint{5.218799in}{0.743485in}}%
\pgfpathlineto{\pgfqpoint{5.219363in}{0.743484in}}%
\pgfpathlineto{\pgfqpoint{5.219927in}{0.743484in}}%
\pgfpathlineto{\pgfqpoint{5.220492in}{0.743483in}}%
\pgfpathlineto{\pgfqpoint{5.221056in}{0.743483in}}%
\pgfpathlineto{\pgfqpoint{5.221620in}{0.743482in}}%
\pgfpathlineto{\pgfqpoint{5.222185in}{0.743482in}}%
\pgfpathlineto{\pgfqpoint{5.222749in}{0.743481in}}%
\pgfpathlineto{\pgfqpoint{5.223313in}{0.743481in}}%
\pgfpathlineto{\pgfqpoint{5.223878in}{0.743480in}}%
\pgfpathlineto{\pgfqpoint{5.224442in}{0.743480in}}%
\pgfpathlineto{\pgfqpoint{5.225007in}{0.743479in}}%
\pgfpathlineto{\pgfqpoint{5.225571in}{0.743479in}}%
\pgfpathlineto{\pgfqpoint{5.226135in}{0.743479in}}%
\pgfpathlineto{\pgfqpoint{5.226700in}{0.743478in}}%
\pgfpathlineto{\pgfqpoint{5.227264in}{0.743478in}}%
\pgfpathlineto{\pgfqpoint{5.227828in}{0.743477in}}%
\pgfpathlineto{\pgfqpoint{5.228393in}{0.743477in}}%
\pgfpathlineto{\pgfqpoint{5.228957in}{0.743476in}}%
\pgfpathlineto{\pgfqpoint{5.229521in}{0.743476in}}%
\pgfpathlineto{\pgfqpoint{5.230086in}{0.743475in}}%
\pgfpathlineto{\pgfqpoint{5.230650in}{0.743475in}}%
\pgfpathlineto{\pgfqpoint{5.231214in}{0.743474in}}%
\pgfpathlineto{\pgfqpoint{5.231779in}{0.743474in}}%
\pgfpathlineto{\pgfqpoint{5.232343in}{0.743473in}}%
\pgfpathlineto{\pgfqpoint{5.232907in}{0.743473in}}%
\pgfpathlineto{\pgfqpoint{5.233472in}{0.743472in}}%
\pgfpathlineto{\pgfqpoint{5.234036in}{0.743472in}}%
\pgfpathlineto{\pgfqpoint{5.234600in}{0.743471in}}%
\pgfpathlineto{\pgfqpoint{5.235165in}{0.743471in}}%
\pgfpathlineto{\pgfqpoint{5.235729in}{0.743470in}}%
\pgfpathlineto{\pgfqpoint{5.236293in}{0.743470in}}%
\pgfpathlineto{\pgfqpoint{5.236858in}{0.743469in}}%
\pgfpathlineto{\pgfqpoint{5.237422in}{0.743469in}}%
\pgfpathlineto{\pgfqpoint{5.237986in}{0.743468in}}%
\pgfpathlineto{\pgfqpoint{5.238551in}{0.743468in}}%
\pgfpathlineto{\pgfqpoint{5.239115in}{0.743467in}}%
\pgfpathlineto{\pgfqpoint{5.239680in}{0.743467in}}%
\pgfpathlineto{\pgfqpoint{5.240244in}{0.743466in}}%
\pgfpathlineto{\pgfqpoint{5.240808in}{0.743466in}}%
\pgfpathlineto{\pgfqpoint{5.241373in}{0.743465in}}%
\pgfpathlineto{\pgfqpoint{5.241937in}{0.743465in}}%
\pgfpathlineto{\pgfqpoint{5.242501in}{0.743464in}}%
\pgfpathlineto{\pgfqpoint{5.243066in}{0.743464in}}%
\pgfpathlineto{\pgfqpoint{5.243630in}{0.743463in}}%
\pgfpathlineto{\pgfqpoint{5.244194in}{0.743463in}}%
\pgfpathlineto{\pgfqpoint{5.244759in}{0.743462in}}%
\pgfpathlineto{\pgfqpoint{5.245323in}{0.743462in}}%
\pgfpathlineto{\pgfqpoint{5.245887in}{0.743461in}}%
\pgfpathlineto{\pgfqpoint{5.246452in}{0.743461in}}%
\pgfpathlineto{\pgfqpoint{5.247016in}{0.743460in}}%
\pgfpathlineto{\pgfqpoint{5.247580in}{0.743460in}}%
\pgfpathlineto{\pgfqpoint{5.248145in}{0.743460in}}%
\pgfpathlineto{\pgfqpoint{5.248709in}{0.743459in}}%
\pgfpathlineto{\pgfqpoint{5.249273in}{0.743459in}}%
\pgfpathlineto{\pgfqpoint{5.249838in}{0.743458in}}%
\pgfpathlineto{\pgfqpoint{5.250402in}{0.743458in}}%
\pgfpathlineto{\pgfqpoint{5.250966in}{0.743457in}}%
\pgfpathlineto{\pgfqpoint{5.251531in}{0.743457in}}%
\pgfpathlineto{\pgfqpoint{5.252095in}{0.743456in}}%
\pgfpathlineto{\pgfqpoint{5.252659in}{0.743456in}}%
\pgfpathlineto{\pgfqpoint{5.253224in}{0.743455in}}%
\pgfpathlineto{\pgfqpoint{5.253788in}{0.743455in}}%
\pgfpathlineto{\pgfqpoint{5.254353in}{0.743454in}}%
\pgfpathlineto{\pgfqpoint{5.254917in}{0.743454in}}%
\pgfpathlineto{\pgfqpoint{5.255481in}{0.743453in}}%
\pgfpathlineto{\pgfqpoint{5.256046in}{0.743453in}}%
\pgfpathlineto{\pgfqpoint{5.256610in}{0.743452in}}%
\pgfpathlineto{\pgfqpoint{5.257174in}{0.743452in}}%
\pgfpathlineto{\pgfqpoint{5.257739in}{0.743451in}}%
\pgfpathlineto{\pgfqpoint{5.258303in}{0.743451in}}%
\pgfpathlineto{\pgfqpoint{5.258867in}{0.743450in}}%
\pgfpathlineto{\pgfqpoint{5.259432in}{0.743450in}}%
\pgfpathlineto{\pgfqpoint{5.259996in}{0.743449in}}%
\pgfpathlineto{\pgfqpoint{5.260560in}{0.743449in}}%
\pgfpathlineto{\pgfqpoint{5.261125in}{0.743448in}}%
\pgfpathlineto{\pgfqpoint{5.261689in}{0.743448in}}%
\pgfpathlineto{\pgfqpoint{5.262253in}{0.743447in}}%
\pgfpathlineto{\pgfqpoint{5.262818in}{0.743447in}}%
\pgfpathlineto{\pgfqpoint{5.263382in}{0.743446in}}%
\pgfpathlineto{\pgfqpoint{5.263946in}{0.743446in}}%
\pgfpathlineto{\pgfqpoint{5.264511in}{0.743445in}}%
\pgfpathlineto{\pgfqpoint{5.265075in}{0.743445in}}%
\pgfpathlineto{\pgfqpoint{5.265639in}{0.743444in}}%
\pgfpathlineto{\pgfqpoint{5.266204in}{0.743444in}}%
\pgfpathlineto{\pgfqpoint{5.266768in}{0.743443in}}%
\pgfpathlineto{\pgfqpoint{5.267332in}{0.743443in}}%
\pgfpathlineto{\pgfqpoint{5.267897in}{0.743442in}}%
\pgfpathlineto{\pgfqpoint{5.268461in}{0.743442in}}%
\pgfpathlineto{\pgfqpoint{5.269025in}{0.743441in}}%
\pgfpathlineto{\pgfqpoint{5.269590in}{0.743441in}}%
\pgfpathlineto{\pgfqpoint{5.270154in}{0.743440in}}%
\pgfpathlineto{\pgfqpoint{5.270719in}{0.743440in}}%
\pgfpathlineto{\pgfqpoint{5.271283in}{0.743440in}}%
\pgfpathlineto{\pgfqpoint{5.271847in}{0.743439in}}%
\pgfpathlineto{\pgfqpoint{5.272412in}{0.743439in}}%
\pgfpathlineto{\pgfqpoint{5.272976in}{0.743438in}}%
\pgfpathlineto{\pgfqpoint{5.273540in}{0.743438in}}%
\pgfpathlineto{\pgfqpoint{5.274105in}{0.743437in}}%
\pgfpathlineto{\pgfqpoint{5.274669in}{0.743437in}}%
\pgfpathlineto{\pgfqpoint{5.275233in}{0.743436in}}%
\pgfpathlineto{\pgfqpoint{5.275798in}{0.743436in}}%
\pgfpathlineto{\pgfqpoint{5.276362in}{0.743435in}}%
\pgfpathlineto{\pgfqpoint{5.276926in}{0.743435in}}%
\pgfpathlineto{\pgfqpoint{5.277491in}{0.743434in}}%
\pgfpathlineto{\pgfqpoint{5.278055in}{0.743434in}}%
\pgfpathlineto{\pgfqpoint{5.278619in}{0.743433in}}%
\pgfpathlineto{\pgfqpoint{5.279184in}{0.743433in}}%
\pgfpathlineto{\pgfqpoint{5.279748in}{0.743432in}}%
\pgfpathlineto{\pgfqpoint{5.280312in}{0.743432in}}%
\pgfpathlineto{\pgfqpoint{5.280877in}{0.743431in}}%
\pgfpathlineto{\pgfqpoint{5.281441in}{0.743431in}}%
\pgfpathlineto{\pgfqpoint{5.282005in}{0.743430in}}%
\pgfpathlineto{\pgfqpoint{5.282570in}{0.743430in}}%
\pgfpathlineto{\pgfqpoint{5.283134in}{0.743429in}}%
\pgfpathlineto{\pgfqpoint{5.283698in}{0.743429in}}%
\pgfpathlineto{\pgfqpoint{5.284263in}{0.743428in}}%
\pgfpathlineto{\pgfqpoint{5.284827in}{0.743428in}}%
\pgfpathlineto{\pgfqpoint{5.285392in}{0.743427in}}%
\pgfpathlineto{\pgfqpoint{5.285956in}{0.743427in}}%
\pgfpathlineto{\pgfqpoint{5.286520in}{0.743426in}}%
\pgfpathlineto{\pgfqpoint{5.287085in}{0.743426in}}%
\pgfpathlineto{\pgfqpoint{5.287649in}{0.743425in}}%
\pgfpathlineto{\pgfqpoint{5.288213in}{0.743425in}}%
\pgfpathlineto{\pgfqpoint{5.288778in}{0.743424in}}%
\pgfpathlineto{\pgfqpoint{5.289342in}{0.743424in}}%
\pgfpathlineto{\pgfqpoint{5.289906in}{0.743423in}}%
\pgfpathlineto{\pgfqpoint{5.290471in}{0.743423in}}%
\pgfpathlineto{\pgfqpoint{5.291035in}{0.743422in}}%
\pgfpathlineto{\pgfqpoint{5.291599in}{0.743422in}}%
\pgfpathlineto{\pgfqpoint{5.292164in}{0.743421in}}%
\pgfpathlineto{\pgfqpoint{5.292728in}{0.743421in}}%
\pgfpathlineto{\pgfqpoint{5.293292in}{0.743421in}}%
\pgfpathlineto{\pgfqpoint{5.293857in}{0.743420in}}%
\pgfpathlineto{\pgfqpoint{5.294421in}{0.743420in}}%
\pgfpathlineto{\pgfqpoint{5.294985in}{0.743419in}}%
\pgfpathlineto{\pgfqpoint{5.295550in}{0.743419in}}%
\pgfpathlineto{\pgfqpoint{5.296114in}{0.743418in}}%
\pgfpathlineto{\pgfqpoint{5.296678in}{0.743418in}}%
\pgfpathlineto{\pgfqpoint{5.297243in}{0.743417in}}%
\pgfpathlineto{\pgfqpoint{5.297807in}{0.743417in}}%
\pgfpathlineto{\pgfqpoint{5.298371in}{0.743416in}}%
\pgfpathlineto{\pgfqpoint{5.298936in}{0.743416in}}%
\pgfpathlineto{\pgfqpoint{5.299500in}{0.743415in}}%
\pgfpathlineto{\pgfqpoint{5.300065in}{0.743415in}}%
\pgfpathlineto{\pgfqpoint{5.300629in}{0.743414in}}%
\pgfpathlineto{\pgfqpoint{5.301193in}{0.743414in}}%
\pgfpathlineto{\pgfqpoint{5.301758in}{0.743413in}}%
\pgfpathlineto{\pgfqpoint{5.302322in}{0.743413in}}%
\pgfpathlineto{\pgfqpoint{5.302886in}{0.743412in}}%
\pgfpathlineto{\pgfqpoint{5.303451in}{0.743412in}}%
\pgfpathlineto{\pgfqpoint{5.304015in}{0.743411in}}%
\pgfpathlineto{\pgfqpoint{5.304579in}{0.743411in}}%
\pgfpathlineto{\pgfqpoint{5.305144in}{0.743410in}}%
\pgfpathlineto{\pgfqpoint{5.305708in}{0.743410in}}%
\pgfpathlineto{\pgfqpoint{5.306272in}{0.743409in}}%
\pgfpathlineto{\pgfqpoint{5.306837in}{0.743409in}}%
\pgfpathlineto{\pgfqpoint{5.307401in}{0.743408in}}%
\pgfpathlineto{\pgfqpoint{5.307965in}{0.743408in}}%
\pgfpathlineto{\pgfqpoint{5.308530in}{0.743407in}}%
\pgfpathlineto{\pgfqpoint{5.309094in}{0.743407in}}%
\pgfpathlineto{\pgfqpoint{5.309658in}{0.743406in}}%
\pgfpathlineto{\pgfqpoint{5.310223in}{0.743406in}}%
\pgfpathlineto{\pgfqpoint{5.310787in}{0.743405in}}%
\pgfpathlineto{\pgfqpoint{5.311351in}{0.743405in}}%
\pgfpathlineto{\pgfqpoint{5.311916in}{0.743404in}}%
\pgfpathlineto{\pgfqpoint{5.312480in}{0.743404in}}%
\pgfpathlineto{\pgfqpoint{5.313044in}{0.743403in}}%
\pgfpathlineto{\pgfqpoint{5.313609in}{0.743403in}}%
\pgfpathlineto{\pgfqpoint{5.314173in}{0.743402in}}%
\pgfpathlineto{\pgfqpoint{5.314737in}{0.743402in}}%
\pgfpathlineto{\pgfqpoint{5.315302in}{0.743401in}}%
\pgfpathlineto{\pgfqpoint{5.315866in}{0.743401in}}%
\pgfpathlineto{\pgfqpoint{5.316431in}{0.743401in}}%
\pgfpathlineto{\pgfqpoint{5.316995in}{0.743400in}}%
\pgfpathlineto{\pgfqpoint{5.317559in}{0.743400in}}%
\pgfpathlineto{\pgfqpoint{5.318124in}{0.743399in}}%
\pgfpathlineto{\pgfqpoint{5.318688in}{0.743399in}}%
\pgfpathlineto{\pgfqpoint{5.319252in}{0.743398in}}%
\pgfpathlineto{\pgfqpoint{5.319817in}{0.743398in}}%
\pgfpathlineto{\pgfqpoint{5.320381in}{0.743397in}}%
\pgfpathlineto{\pgfqpoint{5.320945in}{0.743397in}}%
\pgfpathlineto{\pgfqpoint{5.321510in}{0.743396in}}%
\pgfpathlineto{\pgfqpoint{5.322074in}{0.743396in}}%
\pgfpathlineto{\pgfqpoint{5.322638in}{0.743395in}}%
\pgfpathlineto{\pgfqpoint{5.323203in}{0.743395in}}%
\pgfpathlineto{\pgfqpoint{5.323767in}{0.743394in}}%
\pgfpathlineto{\pgfqpoint{5.324331in}{0.743394in}}%
\pgfpathlineto{\pgfqpoint{5.324896in}{0.743393in}}%
\pgfpathlineto{\pgfqpoint{5.325460in}{0.743393in}}%
\pgfpathlineto{\pgfqpoint{5.326024in}{0.743392in}}%
\pgfpathlineto{\pgfqpoint{5.326589in}{0.743392in}}%
\pgfpathlineto{\pgfqpoint{5.327153in}{0.743391in}}%
\pgfpathlineto{\pgfqpoint{5.327717in}{0.743391in}}%
\pgfpathlineto{\pgfqpoint{5.328282in}{0.743390in}}%
\pgfpathlineto{\pgfqpoint{5.328846in}{0.743390in}}%
\pgfpathlineto{\pgfqpoint{5.329410in}{0.743389in}}%
\pgfpathlineto{\pgfqpoint{5.329975in}{0.743389in}}%
\pgfpathlineto{\pgfqpoint{5.330539in}{0.743388in}}%
\pgfpathlineto{\pgfqpoint{5.331104in}{0.743388in}}%
\pgfpathlineto{\pgfqpoint{5.331668in}{0.743387in}}%
\pgfpathlineto{\pgfqpoint{5.332232in}{0.743387in}}%
\pgfpathlineto{\pgfqpoint{5.332797in}{0.743386in}}%
\pgfpathlineto{\pgfqpoint{5.333361in}{0.743386in}}%
\pgfpathlineto{\pgfqpoint{5.333925in}{0.743385in}}%
\pgfpathlineto{\pgfqpoint{5.334490in}{0.743385in}}%
\pgfpathlineto{\pgfqpoint{5.335054in}{0.743384in}}%
\pgfpathlineto{\pgfqpoint{5.335618in}{0.743384in}}%
\pgfpathlineto{\pgfqpoint{5.336183in}{0.743383in}}%
\pgfpathlineto{\pgfqpoint{5.336747in}{0.743383in}}%
\pgfpathlineto{\pgfqpoint{5.337311in}{0.743382in}}%
\pgfpathlineto{\pgfqpoint{5.337876in}{0.743382in}}%
\pgfpathlineto{\pgfqpoint{5.338440in}{0.743381in}}%
\pgfpathlineto{\pgfqpoint{5.339004in}{0.743381in}}%
\pgfpathlineto{\pgfqpoint{5.339569in}{0.743381in}}%
\pgfpathlineto{\pgfqpoint{5.340133in}{0.743380in}}%
\pgfpathlineto{\pgfqpoint{5.340697in}{0.743380in}}%
\pgfpathlineto{\pgfqpoint{5.341262in}{0.743379in}}%
\pgfpathlineto{\pgfqpoint{5.341826in}{0.743379in}}%
\pgfpathlineto{\pgfqpoint{5.342390in}{0.743378in}}%
\pgfpathlineto{\pgfqpoint{5.342955in}{0.743378in}}%
\pgfpathlineto{\pgfqpoint{5.343519in}{0.743377in}}%
\pgfpathlineto{\pgfqpoint{5.344083in}{0.743377in}}%
\pgfpathlineto{\pgfqpoint{5.344648in}{0.743376in}}%
\pgfpathlineto{\pgfqpoint{5.345212in}{0.743376in}}%
\pgfpathlineto{\pgfqpoint{5.345777in}{0.743375in}}%
\pgfpathlineto{\pgfqpoint{5.346341in}{0.743375in}}%
\pgfpathlineto{\pgfqpoint{5.346905in}{0.743374in}}%
\pgfpathlineto{\pgfqpoint{5.347470in}{0.743374in}}%
\pgfpathlineto{\pgfqpoint{5.348034in}{0.743373in}}%
\pgfpathlineto{\pgfqpoint{5.348598in}{0.743373in}}%
\pgfpathlineto{\pgfqpoint{5.349163in}{0.743372in}}%
\pgfpathlineto{\pgfqpoint{5.349727in}{0.743372in}}%
\pgfpathlineto{\pgfqpoint{5.350291in}{0.743371in}}%
\pgfpathlineto{\pgfqpoint{5.350856in}{0.743371in}}%
\pgfpathlineto{\pgfqpoint{5.351420in}{0.743370in}}%
\pgfpathlineto{\pgfqpoint{5.351984in}{0.743370in}}%
\pgfpathlineto{\pgfqpoint{5.352549in}{0.743369in}}%
\pgfpathlineto{\pgfqpoint{5.353113in}{0.743369in}}%
\pgfpathlineto{\pgfqpoint{5.353677in}{0.743368in}}%
\pgfpathlineto{\pgfqpoint{5.354242in}{0.743368in}}%
\pgfpathlineto{\pgfqpoint{5.354806in}{0.743367in}}%
\pgfpathlineto{\pgfqpoint{5.355370in}{0.743367in}}%
\pgfpathlineto{\pgfqpoint{5.355935in}{0.743366in}}%
\pgfpathlineto{\pgfqpoint{5.356499in}{0.743366in}}%
\pgfpathlineto{\pgfqpoint{5.357063in}{0.743365in}}%
\pgfpathlineto{\pgfqpoint{5.357628in}{0.743365in}}%
\pgfpathlineto{\pgfqpoint{5.358192in}{0.743364in}}%
\pgfpathlineto{\pgfqpoint{5.358756in}{0.743364in}}%
\pgfpathlineto{\pgfqpoint{5.359321in}{0.743363in}}%
\pgfpathlineto{\pgfqpoint{5.359885in}{0.743363in}}%
\pgfpathlineto{\pgfqpoint{5.360449in}{0.743362in}}%
\pgfpathlineto{\pgfqpoint{5.361014in}{0.743362in}}%
\pgfpathlineto{\pgfqpoint{5.361578in}{0.743362in}}%
\pgfpathlineto{\pgfqpoint{5.362143in}{0.743361in}}%
\pgfpathlineto{\pgfqpoint{5.362707in}{0.743361in}}%
\pgfpathlineto{\pgfqpoint{5.363271in}{0.743360in}}%
\pgfpathlineto{\pgfqpoint{5.363836in}{0.743360in}}%
\pgfpathlineto{\pgfqpoint{5.364400in}{0.743359in}}%
\pgfpathlineto{\pgfqpoint{5.364964in}{0.743359in}}%
\pgfpathlineto{\pgfqpoint{5.365529in}{0.743358in}}%
\pgfpathlineto{\pgfqpoint{5.366093in}{0.743358in}}%
\pgfpathlineto{\pgfqpoint{5.366657in}{0.743357in}}%
\pgfpathlineto{\pgfqpoint{5.367222in}{0.743357in}}%
\pgfpathlineto{\pgfqpoint{5.367786in}{0.743356in}}%
\pgfpathlineto{\pgfqpoint{5.368350in}{0.743356in}}%
\pgfpathlineto{\pgfqpoint{5.368915in}{0.743355in}}%
\pgfpathlineto{\pgfqpoint{5.369479in}{0.743355in}}%
\pgfpathlineto{\pgfqpoint{5.370043in}{0.743354in}}%
\pgfpathlineto{\pgfqpoint{5.370608in}{0.743354in}}%
\pgfpathlineto{\pgfqpoint{5.371172in}{0.743353in}}%
\pgfpathlineto{\pgfqpoint{5.371736in}{0.743353in}}%
\pgfpathlineto{\pgfqpoint{5.372301in}{0.743352in}}%
\pgfpathlineto{\pgfqpoint{5.372865in}{0.743352in}}%
\pgfpathlineto{\pgfqpoint{5.373429in}{0.743351in}}%
\pgfpathlineto{\pgfqpoint{5.373994in}{0.743351in}}%
\pgfpathlineto{\pgfqpoint{5.374558in}{0.743350in}}%
\pgfpathlineto{\pgfqpoint{5.375122in}{0.743350in}}%
\pgfpathlineto{\pgfqpoint{5.375687in}{0.743349in}}%
\pgfpathlineto{\pgfqpoint{5.376251in}{0.743349in}}%
\pgfpathlineto{\pgfqpoint{5.376816in}{0.743348in}}%
\pgfpathlineto{\pgfqpoint{5.377380in}{0.743348in}}%
\pgfpathlineto{\pgfqpoint{5.377944in}{0.743347in}}%
\pgfpathlineto{\pgfqpoint{5.378509in}{0.743347in}}%
\pgfpathlineto{\pgfqpoint{5.379073in}{0.743346in}}%
\pgfpathlineto{\pgfqpoint{5.379637in}{0.743346in}}%
\pgfpathlineto{\pgfqpoint{5.380202in}{0.743345in}}%
\pgfpathlineto{\pgfqpoint{5.380766in}{0.743345in}}%
\pgfpathlineto{\pgfqpoint{5.381330in}{0.743344in}}%
\pgfpathlineto{\pgfqpoint{5.381895in}{0.743344in}}%
\pgfpathlineto{\pgfqpoint{5.382459in}{0.743343in}}%
\pgfpathlineto{\pgfqpoint{5.383023in}{0.743343in}}%
\pgfpathlineto{\pgfqpoint{5.383588in}{0.743342in}}%
\pgfpathlineto{\pgfqpoint{5.384152in}{0.743342in}}%
\pgfpathlineto{\pgfqpoint{5.384716in}{0.743342in}}%
\pgfpathlineto{\pgfqpoint{5.385281in}{0.743341in}}%
\pgfpathlineto{\pgfqpoint{5.385845in}{0.743341in}}%
\pgfpathlineto{\pgfqpoint{5.386409in}{0.743340in}}%
\pgfpathlineto{\pgfqpoint{5.386974in}{0.743340in}}%
\pgfpathlineto{\pgfqpoint{5.387538in}{0.743339in}}%
\pgfpathlineto{\pgfqpoint{5.388102in}{0.743339in}}%
\pgfpathlineto{\pgfqpoint{5.388667in}{0.743338in}}%
\pgfpathlineto{\pgfqpoint{5.389231in}{0.743338in}}%
\pgfpathlineto{\pgfqpoint{5.389795in}{0.743337in}}%
\pgfpathlineto{\pgfqpoint{5.390360in}{0.743337in}}%
\pgfpathlineto{\pgfqpoint{5.390924in}{0.743336in}}%
\pgfpathlineto{\pgfqpoint{5.391489in}{0.743336in}}%
\pgfpathlineto{\pgfqpoint{5.392053in}{0.743335in}}%
\pgfpathlineto{\pgfqpoint{5.392617in}{0.743335in}}%
\pgfpathlineto{\pgfqpoint{5.393182in}{0.743334in}}%
\pgfpathlineto{\pgfqpoint{5.393746in}{0.743334in}}%
\pgfpathlineto{\pgfqpoint{5.394310in}{0.743333in}}%
\pgfpathlineto{\pgfqpoint{5.394875in}{0.743333in}}%
\pgfpathlineto{\pgfqpoint{5.395439in}{0.743332in}}%
\pgfpathlineto{\pgfqpoint{5.396003in}{0.743332in}}%
\pgfpathlineto{\pgfqpoint{5.396568in}{0.743331in}}%
\pgfpathlineto{\pgfqpoint{5.397132in}{0.743331in}}%
\pgfpathlineto{\pgfqpoint{5.397696in}{0.743330in}}%
\pgfpathlineto{\pgfqpoint{5.398261in}{0.743330in}}%
\pgfpathlineto{\pgfqpoint{5.398825in}{0.743329in}}%
\pgfpathlineto{\pgfqpoint{5.399389in}{0.743329in}}%
\pgfpathlineto{\pgfqpoint{5.399954in}{0.743328in}}%
\pgfpathlineto{\pgfqpoint{5.400518in}{0.743328in}}%
\pgfpathlineto{\pgfqpoint{5.401082in}{0.743327in}}%
\pgfpathlineto{\pgfqpoint{5.401647in}{0.743327in}}%
\pgfpathlineto{\pgfqpoint{5.402211in}{0.743326in}}%
\pgfpathlineto{\pgfqpoint{5.402775in}{0.743326in}}%
\pgfpathlineto{\pgfqpoint{5.403340in}{0.743325in}}%
\pgfpathlineto{\pgfqpoint{5.403904in}{0.743325in}}%
\pgfpathlineto{\pgfqpoint{5.404468in}{0.743324in}}%
\pgfpathlineto{\pgfqpoint{5.405033in}{0.743324in}}%
\pgfpathlineto{\pgfqpoint{5.405597in}{0.743323in}}%
\pgfpathlineto{\pgfqpoint{5.406162in}{0.743323in}}%
\pgfpathlineto{\pgfqpoint{5.406726in}{0.743323in}}%
\pgfpathlineto{\pgfqpoint{5.407290in}{0.743322in}}%
\pgfpathlineto{\pgfqpoint{5.407855in}{0.743322in}}%
\pgfpathlineto{\pgfqpoint{5.408419in}{0.743321in}}%
\pgfpathlineto{\pgfqpoint{5.408983in}{0.743321in}}%
\pgfpathlineto{\pgfqpoint{5.409548in}{0.743320in}}%
\pgfpathlineto{\pgfqpoint{5.410112in}{0.743320in}}%
\pgfpathlineto{\pgfqpoint{5.410676in}{0.743319in}}%
\pgfpathlineto{\pgfqpoint{5.411241in}{0.743319in}}%
\pgfpathlineto{\pgfqpoint{5.411805in}{0.743318in}}%
\pgfpathlineto{\pgfqpoint{5.412369in}{0.743318in}}%
\pgfpathlineto{\pgfqpoint{5.412934in}{0.743317in}}%
\pgfpathlineto{\pgfqpoint{5.413498in}{0.743317in}}%
\pgfpathlineto{\pgfqpoint{5.414062in}{0.743316in}}%
\pgfpathlineto{\pgfqpoint{5.414627in}{0.743316in}}%
\pgfpathlineto{\pgfqpoint{5.415191in}{0.743315in}}%
\pgfpathlineto{\pgfqpoint{5.415755in}{0.743315in}}%
\pgfpathlineto{\pgfqpoint{5.416320in}{0.743314in}}%
\pgfpathlineto{\pgfqpoint{5.416884in}{0.743314in}}%
\pgfpathlineto{\pgfqpoint{5.417448in}{0.743313in}}%
\pgfpathlineto{\pgfqpoint{5.418013in}{0.743313in}}%
\pgfpathlineto{\pgfqpoint{5.418577in}{0.743312in}}%
\pgfpathlineto{\pgfqpoint{5.419141in}{0.743312in}}%
\pgfpathlineto{\pgfqpoint{5.419706in}{0.743311in}}%
\pgfpathlineto{\pgfqpoint{5.420270in}{0.743311in}}%
\pgfpathlineto{\pgfqpoint{5.420834in}{0.743310in}}%
\pgfpathlineto{\pgfqpoint{5.421399in}{0.743310in}}%
\pgfpathlineto{\pgfqpoint{5.421963in}{0.743309in}}%
\pgfpathlineto{\pgfqpoint{5.422528in}{0.743309in}}%
\pgfpathlineto{\pgfqpoint{5.423092in}{0.743308in}}%
\pgfpathlineto{\pgfqpoint{5.423656in}{0.743308in}}%
\pgfpathlineto{\pgfqpoint{5.424221in}{0.743307in}}%
\pgfpathlineto{\pgfqpoint{5.424785in}{0.743307in}}%
\pgfpathlineto{\pgfqpoint{5.425349in}{0.743306in}}%
\pgfpathlineto{\pgfqpoint{5.425914in}{0.743306in}}%
\pgfpathlineto{\pgfqpoint{5.426478in}{0.743305in}}%
\pgfpathlineto{\pgfqpoint{5.427042in}{0.743305in}}%
\pgfpathlineto{\pgfqpoint{5.427607in}{0.743304in}}%
\pgfpathlineto{\pgfqpoint{5.428171in}{0.743304in}}%
\pgfpathlineto{\pgfqpoint{5.428735in}{0.743303in}}%
\pgfpathlineto{\pgfqpoint{5.429300in}{0.743303in}}%
\pgfpathlineto{\pgfqpoint{5.429864in}{0.743303in}}%
\pgfpathlineto{\pgfqpoint{5.430428in}{0.743302in}}%
\pgfpathlineto{\pgfqpoint{5.430993in}{0.743302in}}%
\pgfpathlineto{\pgfqpoint{5.431557in}{0.743301in}}%
\pgfpathlineto{\pgfqpoint{5.432121in}{0.743301in}}%
\pgfpathlineto{\pgfqpoint{5.432686in}{0.743300in}}%
\pgfpathlineto{\pgfqpoint{5.433250in}{0.743300in}}%
\pgfpathlineto{\pgfqpoint{5.433814in}{0.743299in}}%
\pgfpathlineto{\pgfqpoint{5.434379in}{0.743299in}}%
\pgfpathlineto{\pgfqpoint{5.434943in}{0.743298in}}%
\pgfpathlineto{\pgfqpoint{5.435507in}{0.743298in}}%
\pgfpathlineto{\pgfqpoint{5.436072in}{0.743297in}}%
\pgfpathlineto{\pgfqpoint{5.436636in}{0.743297in}}%
\pgfpathlineto{\pgfqpoint{5.437201in}{0.743296in}}%
\pgfpathlineto{\pgfqpoint{5.437765in}{0.743296in}}%
\pgfpathlineto{\pgfqpoint{5.438329in}{0.743295in}}%
\pgfpathlineto{\pgfqpoint{5.438894in}{0.743295in}}%
\pgfpathlineto{\pgfqpoint{5.439458in}{0.743294in}}%
\pgfpathlineto{\pgfqpoint{5.440022in}{0.743294in}}%
\pgfpathlineto{\pgfqpoint{5.440587in}{0.743293in}}%
\pgfpathlineto{\pgfqpoint{5.441151in}{0.743293in}}%
\pgfpathlineto{\pgfqpoint{5.441715in}{0.743292in}}%
\pgfpathlineto{\pgfqpoint{5.442280in}{0.743292in}}%
\pgfpathlineto{\pgfqpoint{5.442844in}{0.743291in}}%
\pgfpathlineto{\pgfqpoint{5.443408in}{0.743291in}}%
\pgfpathlineto{\pgfqpoint{5.443973in}{0.743290in}}%
\pgfpathlineto{\pgfqpoint{5.444537in}{0.743290in}}%
\pgfpathlineto{\pgfqpoint{5.445101in}{0.743289in}}%
\pgfpathlineto{\pgfqpoint{5.445666in}{0.743289in}}%
\pgfpathlineto{\pgfqpoint{5.446230in}{0.743288in}}%
\pgfpathlineto{\pgfqpoint{5.446794in}{0.743288in}}%
\pgfpathlineto{\pgfqpoint{5.447359in}{0.743287in}}%
\pgfpathlineto{\pgfqpoint{5.447923in}{0.743287in}}%
\pgfpathlineto{\pgfqpoint{5.448487in}{0.743286in}}%
\pgfpathlineto{\pgfqpoint{5.449052in}{0.743286in}}%
\pgfpathlineto{\pgfqpoint{5.449616in}{0.743285in}}%
\pgfpathlineto{\pgfqpoint{5.450180in}{0.743285in}}%
\pgfpathlineto{\pgfqpoint{5.450745in}{0.743284in}}%
\pgfpathlineto{\pgfqpoint{5.451309in}{0.743284in}}%
\pgfpathlineto{\pgfqpoint{5.451874in}{0.743284in}}%
\pgfpathlineto{\pgfqpoint{5.452438in}{0.743283in}}%
\pgfpathlineto{\pgfqpoint{5.453002in}{0.743283in}}%
\pgfpathlineto{\pgfqpoint{5.453567in}{0.743282in}}%
\pgfpathlineto{\pgfqpoint{5.454131in}{0.743282in}}%
\pgfpathlineto{\pgfqpoint{5.454695in}{0.743281in}}%
\pgfpathlineto{\pgfqpoint{5.455260in}{0.743281in}}%
\pgfpathlineto{\pgfqpoint{5.455824in}{0.743280in}}%
\pgfpathlineto{\pgfqpoint{5.456388in}{0.743280in}}%
\pgfpathlineto{\pgfqpoint{5.456953in}{0.743279in}}%
\pgfpathlineto{\pgfqpoint{5.457517in}{0.743279in}}%
\pgfpathlineto{\pgfqpoint{5.458081in}{0.743278in}}%
\pgfpathlineto{\pgfqpoint{5.458646in}{0.743278in}}%
\pgfpathlineto{\pgfqpoint{5.459210in}{0.743277in}}%
\pgfpathlineto{\pgfqpoint{5.459774in}{0.743277in}}%
\pgfpathlineto{\pgfqpoint{5.460339in}{0.743276in}}%
\pgfpathlineto{\pgfqpoint{5.460903in}{0.743276in}}%
\pgfpathlineto{\pgfqpoint{5.461467in}{0.743275in}}%
\pgfpathlineto{\pgfqpoint{5.462032in}{0.743275in}}%
\pgfpathlineto{\pgfqpoint{5.462596in}{0.743274in}}%
\pgfpathlineto{\pgfqpoint{5.463160in}{0.743274in}}%
\pgfpathlineto{\pgfqpoint{5.463725in}{0.743273in}}%
\pgfpathlineto{\pgfqpoint{5.464289in}{0.743273in}}%
\pgfpathlineto{\pgfqpoint{5.464853in}{0.743272in}}%
\pgfpathlineto{\pgfqpoint{5.465418in}{0.743272in}}%
\pgfpathlineto{\pgfqpoint{5.465982in}{0.743271in}}%
\pgfpathlineto{\pgfqpoint{5.466546in}{0.743271in}}%
\pgfpathlineto{\pgfqpoint{5.467111in}{0.743270in}}%
\pgfpathlineto{\pgfqpoint{5.467675in}{0.743270in}}%
\pgfpathlineto{\pgfqpoint{5.468240in}{0.743269in}}%
\pgfpathlineto{\pgfqpoint{5.468804in}{0.743269in}}%
\pgfpathlineto{\pgfqpoint{5.469368in}{0.743268in}}%
\pgfpathlineto{\pgfqpoint{5.469933in}{0.743268in}}%
\pgfpathlineto{\pgfqpoint{5.470497in}{0.743267in}}%
\pgfpathlineto{\pgfqpoint{5.471061in}{0.743267in}}%
\pgfpathlineto{\pgfqpoint{5.471626in}{0.743266in}}%
\pgfpathlineto{\pgfqpoint{5.472190in}{0.743266in}}%
\pgfpathlineto{\pgfqpoint{5.472754in}{0.743265in}}%
\pgfpathlineto{\pgfqpoint{5.473319in}{0.743265in}}%
\pgfpathlineto{\pgfqpoint{5.473883in}{0.743264in}}%
\pgfpathlineto{\pgfqpoint{5.474447in}{0.743264in}}%
\pgfpathlineto{\pgfqpoint{5.475012in}{0.743264in}}%
\pgfpathlineto{\pgfqpoint{5.475576in}{0.743263in}}%
\pgfpathlineto{\pgfqpoint{5.476140in}{0.743263in}}%
\pgfpathlineto{\pgfqpoint{5.476705in}{0.743262in}}%
\pgfpathlineto{\pgfqpoint{5.477269in}{0.743262in}}%
\pgfpathlineto{\pgfqpoint{5.477833in}{0.743261in}}%
\pgfpathlineto{\pgfqpoint{5.478398in}{0.743261in}}%
\pgfpathlineto{\pgfqpoint{5.478962in}{0.743260in}}%
\pgfpathlineto{\pgfqpoint{5.479526in}{0.743260in}}%
\pgfpathlineto{\pgfqpoint{5.480091in}{0.743259in}}%
\pgfpathlineto{\pgfqpoint{5.480655in}{0.743259in}}%
\pgfpathlineto{\pgfqpoint{5.481219in}{0.743258in}}%
\pgfpathlineto{\pgfqpoint{5.481784in}{0.743258in}}%
\pgfpathlineto{\pgfqpoint{5.482348in}{0.743257in}}%
\pgfpathlineto{\pgfqpoint{5.482913in}{0.743257in}}%
\pgfpathlineto{\pgfqpoint{5.483477in}{0.743256in}}%
\pgfpathlineto{\pgfqpoint{5.484041in}{0.743256in}}%
\pgfpathlineto{\pgfqpoint{5.484606in}{0.743255in}}%
\pgfpathlineto{\pgfqpoint{5.485170in}{0.743255in}}%
\pgfpathlineto{\pgfqpoint{5.485734in}{0.743254in}}%
\pgfpathlineto{\pgfqpoint{5.486299in}{0.743254in}}%
\pgfpathlineto{\pgfqpoint{5.486863in}{0.743253in}}%
\pgfpathlineto{\pgfqpoint{5.487427in}{0.743253in}}%
\pgfpathlineto{\pgfqpoint{5.487992in}{0.743252in}}%
\pgfpathlineto{\pgfqpoint{5.488556in}{0.743252in}}%
\pgfpathlineto{\pgfqpoint{5.489120in}{0.743251in}}%
\pgfpathlineto{\pgfqpoint{5.489685in}{0.743251in}}%
\pgfpathlineto{\pgfqpoint{5.490249in}{0.743250in}}%
\pgfpathlineto{\pgfqpoint{5.490813in}{0.743250in}}%
\pgfpathlineto{\pgfqpoint{5.491378in}{0.743249in}}%
\pgfpathlineto{\pgfqpoint{5.491942in}{0.743249in}}%
\pgfpathlineto{\pgfqpoint{5.492506in}{0.743248in}}%
\pgfpathlineto{\pgfqpoint{5.493071in}{0.743248in}}%
\pgfpathlineto{\pgfqpoint{5.493635in}{0.743247in}}%
\pgfpathlineto{\pgfqpoint{5.494199in}{0.743247in}}%
\pgfpathlineto{\pgfqpoint{5.494764in}{0.743246in}}%
\pgfpathlineto{\pgfqpoint{5.495328in}{0.743246in}}%
\pgfpathlineto{\pgfqpoint{5.495892in}{0.743245in}}%
\pgfpathlineto{\pgfqpoint{5.496457in}{0.743245in}}%
\pgfpathlineto{\pgfqpoint{5.497021in}{0.743244in}}%
\pgfpathlineto{\pgfqpoint{5.497586in}{0.743244in}}%
\pgfpathlineto{\pgfqpoint{5.498150in}{0.743244in}}%
\pgfpathlineto{\pgfqpoint{5.498714in}{0.743243in}}%
\pgfpathlineto{\pgfqpoint{5.499279in}{0.743243in}}%
\pgfpathlineto{\pgfqpoint{5.499843in}{0.743242in}}%
\pgfpathlineto{\pgfqpoint{5.500407in}{0.743242in}}%
\pgfpathlineto{\pgfqpoint{5.500972in}{0.743241in}}%
\pgfpathlineto{\pgfqpoint{5.501536in}{0.743241in}}%
\pgfpathlineto{\pgfqpoint{5.502100in}{0.743240in}}%
\pgfpathlineto{\pgfqpoint{5.502665in}{0.743240in}}%
\pgfpathlineto{\pgfqpoint{5.503229in}{0.743239in}}%
\pgfpathlineto{\pgfqpoint{5.503793in}{0.743239in}}%
\pgfpathlineto{\pgfqpoint{5.504358in}{0.743238in}}%
\pgfpathlineto{\pgfqpoint{5.504922in}{0.743238in}}%
\pgfpathlineto{\pgfqpoint{5.505486in}{0.743237in}}%
\pgfpathlineto{\pgfqpoint{5.506051in}{0.743237in}}%
\pgfpathlineto{\pgfqpoint{5.506615in}{0.743236in}}%
\pgfpathlineto{\pgfqpoint{5.507179in}{0.743236in}}%
\pgfpathlineto{\pgfqpoint{5.507744in}{0.743235in}}%
\pgfpathlineto{\pgfqpoint{5.508308in}{0.743235in}}%
\pgfpathlineto{\pgfqpoint{5.508872in}{0.743234in}}%
\pgfpathlineto{\pgfqpoint{5.509437in}{0.743234in}}%
\pgfpathlineto{\pgfqpoint{5.510001in}{0.743233in}}%
\pgfpathlineto{\pgfqpoint{5.510565in}{0.743233in}}%
\pgfpathlineto{\pgfqpoint{5.511130in}{0.743232in}}%
\pgfpathlineto{\pgfqpoint{5.511694in}{0.743232in}}%
\pgfpathlineto{\pgfqpoint{5.512258in}{0.743231in}}%
\pgfpathlineto{\pgfqpoint{5.512823in}{0.743231in}}%
\pgfpathlineto{\pgfqpoint{5.513387in}{0.743230in}}%
\pgfpathlineto{\pgfqpoint{5.513952in}{0.743230in}}%
\pgfpathlineto{\pgfqpoint{5.514516in}{0.743229in}}%
\pgfpathlineto{\pgfqpoint{5.515080in}{0.743229in}}%
\pgfpathlineto{\pgfqpoint{5.515645in}{0.743228in}}%
\pgfpathlineto{\pgfqpoint{5.516209in}{0.743228in}}%
\pgfpathlineto{\pgfqpoint{5.516773in}{0.743227in}}%
\pgfpathlineto{\pgfqpoint{5.517338in}{0.743227in}}%
\pgfpathlineto{\pgfqpoint{5.517902in}{0.743226in}}%
\pgfpathlineto{\pgfqpoint{5.518466in}{0.743226in}}%
\pgfpathlineto{\pgfqpoint{5.519031in}{0.743225in}}%
\pgfpathlineto{\pgfqpoint{5.519595in}{0.743225in}}%
\pgfpathlineto{\pgfqpoint{5.520159in}{0.743225in}}%
\pgfpathlineto{\pgfqpoint{5.520724in}{0.743224in}}%
\pgfpathlineto{\pgfqpoint{5.521288in}{0.743224in}}%
\pgfpathlineto{\pgfqpoint{5.521852in}{0.743223in}}%
\pgfpathlineto{\pgfqpoint{5.522417in}{0.743223in}}%
\pgfpathlineto{\pgfqpoint{5.522981in}{0.743222in}}%
\pgfpathlineto{\pgfqpoint{5.523545in}{0.743222in}}%
\pgfpathlineto{\pgfqpoint{5.524110in}{0.743221in}}%
\pgfpathlineto{\pgfqpoint{5.524674in}{0.743221in}}%
\pgfpathlineto{\pgfqpoint{5.525238in}{0.743220in}}%
\pgfpathlineto{\pgfqpoint{5.525803in}{0.743220in}}%
\pgfpathlineto{\pgfqpoint{5.526367in}{0.743219in}}%
\pgfpathlineto{\pgfqpoint{5.526931in}{0.743219in}}%
\pgfpathlineto{\pgfqpoint{5.527496in}{0.743218in}}%
\pgfpathlineto{\pgfqpoint{5.528060in}{0.743218in}}%
\pgfpathlineto{\pgfqpoint{5.528625in}{0.743217in}}%
\pgfpathlineto{\pgfqpoint{5.529189in}{0.743217in}}%
\pgfpathlineto{\pgfqpoint{5.529753in}{0.743216in}}%
\pgfpathlineto{\pgfqpoint{5.530318in}{0.743216in}}%
\pgfpathlineto{\pgfqpoint{5.530882in}{0.743215in}}%
\pgfpathlineto{\pgfqpoint{5.531446in}{0.743215in}}%
\pgfpathlineto{\pgfqpoint{5.532011in}{0.743214in}}%
\pgfpathlineto{\pgfqpoint{5.532575in}{0.743214in}}%
\pgfpathlineto{\pgfqpoint{5.533139in}{0.743213in}}%
\pgfpathlineto{\pgfqpoint{5.533704in}{0.743213in}}%
\pgfpathlineto{\pgfqpoint{5.534268in}{0.743212in}}%
\pgfpathlineto{\pgfqpoint{5.534832in}{0.743212in}}%
\pgfpathlineto{\pgfqpoint{5.535397in}{0.743211in}}%
\pgfpathlineto{\pgfqpoint{5.535961in}{0.743211in}}%
\pgfpathlineto{\pgfqpoint{5.536525in}{0.743210in}}%
\pgfpathlineto{\pgfqpoint{5.537090in}{0.743210in}}%
\pgfpathlineto{\pgfqpoint{5.537654in}{0.743209in}}%
\pgfpathlineto{\pgfqpoint{5.538218in}{0.743209in}}%
\pgfpathlineto{\pgfqpoint{5.538783in}{0.743208in}}%
\pgfpathlineto{\pgfqpoint{5.539347in}{0.743208in}}%
\pgfpathlineto{\pgfqpoint{5.539911in}{0.743207in}}%
\pgfpathlineto{\pgfqpoint{5.540476in}{0.743207in}}%
\pgfpathlineto{\pgfqpoint{5.541040in}{0.743206in}}%
\pgfpathlineto{\pgfqpoint{5.541604in}{0.743206in}}%
\pgfpathlineto{\pgfqpoint{5.542169in}{0.743205in}}%
\pgfpathlineto{\pgfqpoint{5.542733in}{0.743205in}}%
\pgfpathlineto{\pgfqpoint{5.543298in}{0.743205in}}%
\pgfpathlineto{\pgfqpoint{5.543862in}{0.743204in}}%
\pgfpathlineto{\pgfqpoint{5.544426in}{0.743204in}}%
\pgfpathlineto{\pgfqpoint{5.544991in}{0.743203in}}%
\pgfpathlineto{\pgfqpoint{5.545555in}{0.743203in}}%
\pgfpathlineto{\pgfqpoint{5.546119in}{0.743202in}}%
\pgfpathlineto{\pgfqpoint{5.546684in}{0.743202in}}%
\pgfpathlineto{\pgfqpoint{5.547248in}{0.743201in}}%
\pgfpathlineto{\pgfqpoint{5.547812in}{0.743201in}}%
\pgfpathlineto{\pgfqpoint{5.548377in}{0.743200in}}%
\pgfpathlineto{\pgfqpoint{5.548941in}{0.743200in}}%
\pgfpathlineto{\pgfqpoint{5.549505in}{0.743199in}}%
\pgfpathlineto{\pgfqpoint{5.550070in}{0.743199in}}%
\pgfpathlineto{\pgfqpoint{5.550634in}{0.743198in}}%
\pgfpathlineto{\pgfqpoint{5.551198in}{0.743198in}}%
\pgfpathlineto{\pgfqpoint{5.551763in}{0.743197in}}%
\pgfpathlineto{\pgfqpoint{5.552327in}{0.743197in}}%
\pgfpathlineto{\pgfqpoint{5.552891in}{0.743196in}}%
\pgfpathlineto{\pgfqpoint{5.553456in}{0.743196in}}%
\pgfpathlineto{\pgfqpoint{5.554020in}{0.743195in}}%
\pgfpathlineto{\pgfqpoint{5.554584in}{0.743195in}}%
\pgfpathlineto{\pgfqpoint{5.555149in}{0.743194in}}%
\pgfpathlineto{\pgfqpoint{5.555713in}{0.743194in}}%
\pgfpathlineto{\pgfqpoint{5.556277in}{0.743193in}}%
\pgfpathlineto{\pgfqpoint{5.556842in}{0.743193in}}%
\pgfpathlineto{\pgfqpoint{5.557406in}{0.743192in}}%
\pgfpathlineto{\pgfqpoint{5.557970in}{0.743192in}}%
\pgfpathlineto{\pgfqpoint{5.558535in}{0.743191in}}%
\pgfpathlineto{\pgfqpoint{5.559099in}{0.743191in}}%
\pgfpathlineto{\pgfqpoint{5.559664in}{0.743190in}}%
\pgfpathlineto{\pgfqpoint{5.560228in}{0.743190in}}%
\pgfpathlineto{\pgfqpoint{5.560792in}{0.743189in}}%
\pgfpathlineto{\pgfqpoint{5.561357in}{0.743189in}}%
\pgfpathlineto{\pgfqpoint{5.561921in}{0.743188in}}%
\pgfpathlineto{\pgfqpoint{5.562485in}{0.743188in}}%
\pgfpathlineto{\pgfqpoint{5.563050in}{0.743187in}}%
\pgfpathlineto{\pgfqpoint{5.563614in}{0.743187in}}%
\pgfpathlineto{\pgfqpoint{5.564178in}{0.743186in}}%
\pgfpathlineto{\pgfqpoint{5.564743in}{0.743186in}}%
\pgfpathlineto{\pgfqpoint{5.565307in}{0.743186in}}%
\pgfpathlineto{\pgfqpoint{5.565871in}{0.743185in}}%
\pgfpathlineto{\pgfqpoint{5.566436in}{0.743185in}}%
\pgfpathlineto{\pgfqpoint{5.567000in}{0.743184in}}%
\pgfpathlineto{\pgfqpoint{5.567564in}{0.743184in}}%
\pgfpathlineto{\pgfqpoint{5.568129in}{0.743183in}}%
\pgfpathlineto{\pgfqpoint{5.568693in}{0.743183in}}%
\pgfpathlineto{\pgfqpoint{5.569257in}{0.743182in}}%
\pgfpathlineto{\pgfqpoint{5.569822in}{0.743182in}}%
\pgfpathlineto{\pgfqpoint{5.570386in}{0.743181in}}%
\pgfpathlineto{\pgfqpoint{5.570950in}{0.743181in}}%
\pgfpathlineto{\pgfqpoint{5.571515in}{0.743180in}}%
\pgfpathlineto{\pgfqpoint{5.572079in}{0.743180in}}%
\pgfpathlineto{\pgfqpoint{5.572643in}{0.743179in}}%
\pgfpathlineto{\pgfqpoint{5.573208in}{0.743179in}}%
\pgfpathlineto{\pgfqpoint{5.573772in}{0.743178in}}%
\pgfpathlineto{\pgfqpoint{5.574337in}{0.743178in}}%
\pgfpathlineto{\pgfqpoint{5.574901in}{0.743177in}}%
\pgfpathlineto{\pgfqpoint{5.575465in}{0.743177in}}%
\pgfpathlineto{\pgfqpoint{5.576030in}{0.743176in}}%
\pgfpathlineto{\pgfqpoint{5.576594in}{0.743176in}}%
\pgfpathlineto{\pgfqpoint{5.577158in}{0.743175in}}%
\pgfpathlineto{\pgfqpoint{5.577723in}{0.743175in}}%
\pgfpathlineto{\pgfqpoint{5.578287in}{0.743174in}}%
\pgfpathlineto{\pgfqpoint{5.578851in}{0.743174in}}%
\pgfpathlineto{\pgfqpoint{5.579416in}{0.743173in}}%
\pgfpathlineto{\pgfqpoint{5.579980in}{0.743173in}}%
\pgfpathlineto{\pgfqpoint{5.580544in}{0.743172in}}%
\pgfpathlineto{\pgfqpoint{5.581109in}{0.743172in}}%
\pgfpathlineto{\pgfqpoint{5.581673in}{0.743171in}}%
\pgfpathlineto{\pgfqpoint{5.582237in}{0.743171in}}%
\pgfpathlineto{\pgfqpoint{5.582802in}{0.743170in}}%
\pgfpathlineto{\pgfqpoint{5.583366in}{0.743170in}}%
\pgfpathlineto{\pgfqpoint{5.583930in}{0.743169in}}%
\pgfpathlineto{\pgfqpoint{5.584495in}{0.743169in}}%
\pgfpathlineto{\pgfqpoint{5.585059in}{0.743168in}}%
\pgfpathlineto{\pgfqpoint{5.585623in}{0.743168in}}%
\pgfpathlineto{\pgfqpoint{5.586188in}{0.743167in}}%
\pgfpathlineto{\pgfqpoint{5.586752in}{0.743167in}}%
\pgfpathlineto{\pgfqpoint{5.587316in}{0.743166in}}%
\pgfpathlineto{\pgfqpoint{5.587881in}{0.743166in}}%
\pgfpathlineto{\pgfqpoint{5.588445in}{0.743166in}}%
\pgfpathlineto{\pgfqpoint{5.589010in}{0.743165in}}%
\pgfpathlineto{\pgfqpoint{5.589574in}{0.743165in}}%
\pgfpathlineto{\pgfqpoint{5.590138in}{0.743164in}}%
\pgfpathlineto{\pgfqpoint{5.590703in}{0.743164in}}%
\pgfpathlineto{\pgfqpoint{5.591267in}{0.743163in}}%
\pgfpathlineto{\pgfqpoint{5.591831in}{0.743163in}}%
\pgfpathlineto{\pgfqpoint{5.592396in}{0.743162in}}%
\pgfpathlineto{\pgfqpoint{5.592960in}{0.743162in}}%
\pgfpathlineto{\pgfqpoint{5.593524in}{0.743161in}}%
\pgfpathlineto{\pgfqpoint{5.594089in}{0.743161in}}%
\pgfpathlineto{\pgfqpoint{5.594653in}{0.743160in}}%
\pgfpathlineto{\pgfqpoint{5.595217in}{0.743160in}}%
\pgfpathlineto{\pgfqpoint{5.595782in}{0.743159in}}%
\pgfpathlineto{\pgfqpoint{5.596346in}{0.743159in}}%
\pgfpathlineto{\pgfqpoint{5.596910in}{0.743158in}}%
\pgfpathlineto{\pgfqpoint{5.597475in}{0.743158in}}%
\pgfpathlineto{\pgfqpoint{5.598039in}{0.743157in}}%
\pgfpathlineto{\pgfqpoint{5.598603in}{0.743157in}}%
\pgfpathlineto{\pgfqpoint{5.599168in}{0.743156in}}%
\pgfpathlineto{\pgfqpoint{5.599732in}{0.743156in}}%
\pgfpathlineto{\pgfqpoint{5.600296in}{0.743155in}}%
\pgfpathlineto{\pgfqpoint{5.600861in}{0.743155in}}%
\pgfpathlineto{\pgfqpoint{5.601425in}{0.743154in}}%
\pgfpathlineto{\pgfqpoint{5.601989in}{0.743154in}}%
\pgfpathlineto{\pgfqpoint{5.602554in}{0.743153in}}%
\pgfpathlineto{\pgfqpoint{5.603118in}{0.743153in}}%
\pgfpathlineto{\pgfqpoint{5.603683in}{0.743152in}}%
\pgfpathlineto{\pgfqpoint{5.604247in}{0.743152in}}%
\pgfpathlineto{\pgfqpoint{5.604811in}{0.743151in}}%
\pgfpathlineto{\pgfqpoint{5.605376in}{0.743151in}}%
\pgfpathlineto{\pgfqpoint{5.605940in}{0.743150in}}%
\pgfpathlineto{\pgfqpoint{5.606504in}{0.743150in}}%
\pgfpathlineto{\pgfqpoint{5.607069in}{0.743149in}}%
\pgfpathlineto{\pgfqpoint{5.607633in}{0.743149in}}%
\pgfpathlineto{\pgfqpoint{5.608197in}{0.743148in}}%
\pgfpathlineto{\pgfqpoint{5.608762in}{0.743148in}}%
\pgfpathlineto{\pgfqpoint{5.609326in}{0.743147in}}%
\pgfpathlineto{\pgfqpoint{5.609890in}{0.743147in}}%
\pgfpathlineto{\pgfqpoint{5.610455in}{0.743147in}}%
\pgfpathlineto{\pgfqpoint{5.611019in}{0.743146in}}%
\pgfpathlineto{\pgfqpoint{5.611583in}{0.743146in}}%
\pgfpathlineto{\pgfqpoint{5.612148in}{0.743145in}}%
\pgfpathlineto{\pgfqpoint{5.612712in}{0.743145in}}%
\pgfpathlineto{\pgfqpoint{5.613276in}{0.743144in}}%
\pgfpathlineto{\pgfqpoint{5.613841in}{0.743144in}}%
\pgfpathlineto{\pgfqpoint{5.614405in}{0.743143in}}%
\pgfpathlineto{\pgfqpoint{5.614969in}{0.743143in}}%
\pgfpathlineto{\pgfqpoint{5.615534in}{0.743142in}}%
\pgfpathlineto{\pgfqpoint{5.616098in}{0.743142in}}%
\pgfpathlineto{\pgfqpoint{5.616662in}{0.743141in}}%
\pgfpathlineto{\pgfqpoint{5.617227in}{0.743141in}}%
\pgfpathlineto{\pgfqpoint{5.617791in}{0.743140in}}%
\pgfpathlineto{\pgfqpoint{5.618355in}{0.743140in}}%
\pgfpathlineto{\pgfqpoint{5.618920in}{0.743139in}}%
\pgfpathlineto{\pgfqpoint{5.619484in}{0.743139in}}%
\pgfpathlineto{\pgfqpoint{5.620049in}{0.743138in}}%
\pgfpathlineto{\pgfqpoint{5.620613in}{0.743138in}}%
\pgfpathlineto{\pgfqpoint{5.621177in}{0.743137in}}%
\pgfpathlineto{\pgfqpoint{5.621742in}{0.743137in}}%
\pgfpathlineto{\pgfqpoint{5.622306in}{0.743136in}}%
\pgfpathlineto{\pgfqpoint{5.622870in}{0.743136in}}%
\pgfpathlineto{\pgfqpoint{5.623435in}{0.743135in}}%
\pgfpathlineto{\pgfqpoint{5.623999in}{0.743135in}}%
\pgfpathlineto{\pgfqpoint{5.624563in}{0.743134in}}%
\pgfpathlineto{\pgfqpoint{5.625128in}{0.743134in}}%
\pgfpathlineto{\pgfqpoint{5.625692in}{0.743133in}}%
\pgfpathlineto{\pgfqpoint{5.626256in}{0.743133in}}%
\pgfpathlineto{\pgfqpoint{5.626821in}{0.743132in}}%
\pgfpathlineto{\pgfqpoint{5.627385in}{0.743132in}}%
\pgfpathlineto{\pgfqpoint{5.627949in}{0.743131in}}%
\pgfpathlineto{\pgfqpoint{5.628514in}{0.743131in}}%
\pgfpathlineto{\pgfqpoint{5.629078in}{0.743130in}}%
\pgfpathlineto{\pgfqpoint{5.629642in}{0.743130in}}%
\pgfpathlineto{\pgfqpoint{5.630207in}{0.743129in}}%
\pgfpathlineto{\pgfqpoint{5.630771in}{0.743129in}}%
\pgfpathlineto{\pgfqpoint{5.631335in}{0.743128in}}%
\pgfpathlineto{\pgfqpoint{5.631900in}{0.743128in}}%
\pgfpathlineto{\pgfqpoint{5.632464in}{0.743127in}}%
\pgfpathlineto{\pgfqpoint{5.633028in}{0.743127in}}%
\pgfpathlineto{\pgfqpoint{5.633593in}{0.743127in}}%
\pgfpathlineto{\pgfqpoint{5.634157in}{0.743126in}}%
\pgfpathlineto{\pgfqpoint{5.634722in}{0.743126in}}%
\pgfpathlineto{\pgfqpoint{5.635286in}{0.743125in}}%
\pgfpathlineto{\pgfqpoint{5.635850in}{0.743125in}}%
\pgfpathlineto{\pgfqpoint{5.636415in}{0.743124in}}%
\pgfpathlineto{\pgfqpoint{5.636979in}{0.743124in}}%
\pgfpathlineto{\pgfqpoint{5.637543in}{0.743123in}}%
\pgfpathlineto{\pgfqpoint{5.638108in}{0.743123in}}%
\pgfpathlineto{\pgfqpoint{5.638672in}{0.743122in}}%
\pgfpathlineto{\pgfqpoint{5.639236in}{0.743122in}}%
\pgfpathlineto{\pgfqpoint{5.639801in}{0.743121in}}%
\pgfpathlineto{\pgfqpoint{5.640365in}{0.743121in}}%
\pgfpathlineto{\pgfqpoint{5.640929in}{0.743120in}}%
\pgfpathlineto{\pgfqpoint{5.641494in}{0.743120in}}%
\pgfpathlineto{\pgfqpoint{5.642058in}{0.743119in}}%
\pgfpathlineto{\pgfqpoint{5.642622in}{0.743119in}}%
\pgfpathlineto{\pgfqpoint{5.643187in}{0.743118in}}%
\pgfpathlineto{\pgfqpoint{5.643751in}{0.743118in}}%
\pgfpathlineto{\pgfqpoint{5.644315in}{0.743117in}}%
\pgfpathlineto{\pgfqpoint{5.644880in}{0.743117in}}%
\pgfpathlineto{\pgfqpoint{5.645444in}{0.743116in}}%
\pgfpathlineto{\pgfqpoint{5.646008in}{0.743116in}}%
\pgfpathlineto{\pgfqpoint{5.646573in}{0.743115in}}%
\pgfpathlineto{\pgfqpoint{5.647137in}{0.743115in}}%
\pgfpathlineto{\pgfqpoint{5.647701in}{0.743114in}}%
\pgfpathlineto{\pgfqpoint{5.648266in}{0.743114in}}%
\pgfpathlineto{\pgfqpoint{5.648830in}{0.743113in}}%
\pgfpathlineto{\pgfqpoint{5.649395in}{0.743113in}}%
\pgfpathlineto{\pgfqpoint{5.649959in}{0.743112in}}%
\pgfpathlineto{\pgfqpoint{5.650523in}{0.743112in}}%
\pgfpathlineto{\pgfqpoint{5.651088in}{0.743111in}}%
\pgfpathlineto{\pgfqpoint{5.651652in}{0.743111in}}%
\pgfpathlineto{\pgfqpoint{5.652216in}{0.743110in}}%
\pgfpathlineto{\pgfqpoint{5.652781in}{0.743110in}}%
\pgfpathlineto{\pgfqpoint{5.653345in}{0.743109in}}%
\pgfpathlineto{\pgfqpoint{5.653909in}{0.743109in}}%
\pgfpathlineto{\pgfqpoint{5.654474in}{0.743108in}}%
\pgfpathlineto{\pgfqpoint{5.655038in}{0.743108in}}%
\pgfpathlineto{\pgfqpoint{5.655602in}{0.743107in}}%
\pgfpathlineto{\pgfqpoint{5.656167in}{0.743107in}}%
\pgfpathlineto{\pgfqpoint{5.656731in}{0.743107in}}%
\pgfpathlineto{\pgfqpoint{5.657295in}{0.743106in}}%
\pgfpathlineto{\pgfqpoint{5.657860in}{0.743106in}}%
\pgfpathlineto{\pgfqpoint{5.658424in}{0.743105in}}%
\pgfpathlineto{\pgfqpoint{5.658988in}{0.743105in}}%
\pgfpathlineto{\pgfqpoint{5.659553in}{0.743104in}}%
\pgfpathlineto{\pgfqpoint{5.660117in}{0.743104in}}%
\pgfpathlineto{\pgfqpoint{5.660681in}{0.743103in}}%
\pgfpathlineto{\pgfqpoint{5.661246in}{0.743103in}}%
\pgfpathlineto{\pgfqpoint{5.661810in}{0.743102in}}%
\pgfpathlineto{\pgfqpoint{5.662374in}{0.743102in}}%
\pgfpathlineto{\pgfqpoint{5.662939in}{0.743101in}}%
\pgfpathlineto{\pgfqpoint{5.663503in}{0.743101in}}%
\pgfpathlineto{\pgfqpoint{5.664067in}{0.743100in}}%
\pgfpathlineto{\pgfqpoint{5.664632in}{0.743100in}}%
\pgfpathlineto{\pgfqpoint{5.665196in}{0.743099in}}%
\pgfpathlineto{\pgfqpoint{5.665761in}{0.743099in}}%
\pgfpathlineto{\pgfqpoint{5.666325in}{0.743098in}}%
\pgfpathlineto{\pgfqpoint{5.666889in}{0.743098in}}%
\pgfpathlineto{\pgfqpoint{5.667454in}{0.743097in}}%
\pgfpathlineto{\pgfqpoint{5.668018in}{0.743097in}}%
\pgfpathlineto{\pgfqpoint{5.668582in}{0.743096in}}%
\pgfpathlineto{\pgfqpoint{5.669147in}{0.743096in}}%
\pgfpathlineto{\pgfqpoint{5.669711in}{0.743095in}}%
\pgfpathlineto{\pgfqpoint{5.670275in}{0.743095in}}%
\pgfpathlineto{\pgfqpoint{5.670840in}{0.743094in}}%
\pgfpathlineto{\pgfqpoint{5.671404in}{0.743094in}}%
\pgfpathlineto{\pgfqpoint{5.671968in}{0.743093in}}%
\pgfpathlineto{\pgfqpoint{5.672533in}{0.743093in}}%
\pgfpathlineto{\pgfqpoint{5.673097in}{0.743092in}}%
\pgfpathlineto{\pgfqpoint{5.673661in}{0.743092in}}%
\pgfpathlineto{\pgfqpoint{5.674226in}{0.743091in}}%
\pgfpathlineto{\pgfqpoint{5.674790in}{0.743091in}}%
\pgfpathlineto{\pgfqpoint{5.675354in}{0.743090in}}%
\pgfpathlineto{\pgfqpoint{5.675919in}{0.743090in}}%
\pgfpathlineto{\pgfqpoint{5.676483in}{0.743089in}}%
\pgfpathlineto{\pgfqpoint{5.677047in}{0.743089in}}%
\pgfpathlineto{\pgfqpoint{5.677612in}{0.743088in}}%
\pgfpathlineto{\pgfqpoint{5.678176in}{0.743088in}}%
\pgfpathlineto{\pgfqpoint{5.678740in}{0.743088in}}%
\pgfpathlineto{\pgfqpoint{5.679305in}{0.743087in}}%
\pgfpathlineto{\pgfqpoint{5.679869in}{0.743087in}}%
\pgfpathlineto{\pgfqpoint{5.680434in}{0.743086in}}%
\pgfpathlineto{\pgfqpoint{5.680998in}{0.743086in}}%
\pgfpathlineto{\pgfqpoint{5.681562in}{0.743085in}}%
\pgfpathlineto{\pgfqpoint{5.682127in}{0.743085in}}%
\pgfpathlineto{\pgfqpoint{5.682691in}{0.743084in}}%
\pgfpathlineto{\pgfqpoint{5.683255in}{0.743084in}}%
\pgfpathlineto{\pgfqpoint{5.683820in}{0.743083in}}%
\pgfpathlineto{\pgfqpoint{5.684384in}{0.743083in}}%
\pgfpathlineto{\pgfqpoint{5.684948in}{0.743082in}}%
\pgfpathlineto{\pgfqpoint{5.685513in}{0.743082in}}%
\pgfpathlineto{\pgfqpoint{5.686077in}{0.743081in}}%
\pgfpathlineto{\pgfqpoint{5.686641in}{0.743081in}}%
\pgfpathlineto{\pgfqpoint{5.687206in}{0.743080in}}%
\pgfpathlineto{\pgfqpoint{5.687770in}{0.743080in}}%
\pgfpathlineto{\pgfqpoint{5.688334in}{0.743079in}}%
\pgfpathlineto{\pgfqpoint{5.688899in}{0.743079in}}%
\pgfpathlineto{\pgfqpoint{5.689463in}{0.743078in}}%
\pgfpathlineto{\pgfqpoint{5.690027in}{0.743078in}}%
\pgfpathlineto{\pgfqpoint{5.690592in}{0.743077in}}%
\pgfpathlineto{\pgfqpoint{5.691156in}{0.743077in}}%
\pgfpathlineto{\pgfqpoint{5.691720in}{0.743076in}}%
\pgfpathlineto{\pgfqpoint{5.692285in}{0.743076in}}%
\pgfpathlineto{\pgfqpoint{5.692849in}{0.743075in}}%
\pgfpathlineto{\pgfqpoint{5.693413in}{0.743075in}}%
\pgfpathlineto{\pgfqpoint{5.693978in}{0.743074in}}%
\pgfpathlineto{\pgfqpoint{5.694542in}{0.743074in}}%
\pgfpathlineto{\pgfqpoint{5.695107in}{0.743073in}}%
\pgfpathlineto{\pgfqpoint{5.695671in}{0.743073in}}%
\pgfpathlineto{\pgfqpoint{5.696235in}{0.743072in}}%
\pgfpathlineto{\pgfqpoint{5.696800in}{0.743072in}}%
\pgfpathlineto{\pgfqpoint{5.697364in}{0.743071in}}%
\pgfpathlineto{\pgfqpoint{5.697928in}{0.743071in}}%
\pgfpathlineto{\pgfqpoint{5.698493in}{0.743070in}}%
\pgfpathlineto{\pgfqpoint{5.699057in}{0.743070in}}%
\pgfpathlineto{\pgfqpoint{5.699621in}{0.743069in}}%
\pgfpathlineto{\pgfqpoint{5.700186in}{0.743069in}}%
\pgfpathlineto{\pgfqpoint{5.700750in}{0.743068in}}%
\pgfpathlineto{\pgfqpoint{5.701314in}{0.743068in}}%
\pgfpathlineto{\pgfqpoint{5.701879in}{0.743068in}}%
\pgfpathlineto{\pgfqpoint{5.702443in}{0.743067in}}%
\pgfpathlineto{\pgfqpoint{5.703007in}{0.743067in}}%
\pgfpathlineto{\pgfqpoint{5.703572in}{0.743066in}}%
\pgfpathlineto{\pgfqpoint{5.704136in}{0.743066in}}%
\pgfpathlineto{\pgfqpoint{5.704700in}{0.743065in}}%
\pgfpathlineto{\pgfqpoint{5.705265in}{0.743065in}}%
\pgfpathlineto{\pgfqpoint{5.705829in}{0.743064in}}%
\pgfpathlineto{\pgfqpoint{5.706393in}{0.743064in}}%
\pgfpathlineto{\pgfqpoint{5.706958in}{0.743063in}}%
\pgfpathlineto{\pgfqpoint{5.707522in}{0.743063in}}%
\pgfpathlineto{\pgfqpoint{5.708086in}{0.743062in}}%
\pgfpathlineto{\pgfqpoint{5.708651in}{0.743062in}}%
\pgfpathlineto{\pgfqpoint{5.709215in}{0.743061in}}%
\pgfpathlineto{\pgfqpoint{5.709779in}{0.743061in}}%
\pgfpathlineto{\pgfqpoint{5.710344in}{0.743060in}}%
\pgfpathlineto{\pgfqpoint{5.710908in}{0.743060in}}%
\pgfpathlineto{\pgfqpoint{5.711473in}{0.743059in}}%
\pgfpathlineto{\pgfqpoint{5.712037in}{0.743059in}}%
\pgfpathlineto{\pgfqpoint{5.712601in}{0.743058in}}%
\pgfpathlineto{\pgfqpoint{5.713166in}{0.743058in}}%
\pgfpathlineto{\pgfqpoint{5.713730in}{0.743057in}}%
\pgfpathlineto{\pgfqpoint{5.714294in}{0.743057in}}%
\pgfpathlineto{\pgfqpoint{5.714859in}{0.743056in}}%
\pgfpathlineto{\pgfqpoint{5.715423in}{0.743056in}}%
\pgfpathlineto{\pgfqpoint{5.715987in}{0.743055in}}%
\pgfpathlineto{\pgfqpoint{5.716552in}{0.743055in}}%
\pgfpathlineto{\pgfqpoint{5.717116in}{0.743054in}}%
\pgfpathlineto{\pgfqpoint{5.717680in}{0.743054in}}%
\pgfpathlineto{\pgfqpoint{5.718245in}{0.743053in}}%
\pgfpathlineto{\pgfqpoint{5.718809in}{0.743053in}}%
\pgfpathlineto{\pgfqpoint{5.719373in}{0.743052in}}%
\pgfpathlineto{\pgfqpoint{5.719938in}{0.743052in}}%
\pgfpathlineto{\pgfqpoint{5.720502in}{0.743051in}}%
\pgfpathlineto{\pgfqpoint{5.721066in}{0.743051in}}%
\pgfpathlineto{\pgfqpoint{5.721631in}{0.743050in}}%
\pgfpathlineto{\pgfqpoint{5.722195in}{0.743050in}}%
\pgfpathlineto{\pgfqpoint{5.722759in}{0.743049in}}%
\pgfpathlineto{\pgfqpoint{5.723324in}{0.743049in}}%
\pgfpathlineto{\pgfqpoint{5.723888in}{0.743049in}}%
\pgfpathlineto{\pgfqpoint{5.724452in}{0.743048in}}%
\pgfpathlineto{\pgfqpoint{5.725017in}{0.743048in}}%
\pgfpathlineto{\pgfqpoint{5.725581in}{0.743047in}}%
\pgfpathlineto{\pgfqpoint{5.726146in}{0.743047in}}%
\pgfpathlineto{\pgfqpoint{5.726710in}{0.743046in}}%
\pgfpathlineto{\pgfqpoint{5.727274in}{0.743046in}}%
\pgfpathlineto{\pgfqpoint{5.727839in}{0.743045in}}%
\pgfpathlineto{\pgfqpoint{5.728403in}{0.743045in}}%
\pgfpathlineto{\pgfqpoint{5.728967in}{0.743044in}}%
\pgfpathlineto{\pgfqpoint{5.729532in}{0.743044in}}%
\pgfpathlineto{\pgfqpoint{5.730096in}{0.743043in}}%
\pgfpathlineto{\pgfqpoint{5.730660in}{0.743043in}}%
\pgfpathlineto{\pgfqpoint{5.731225in}{0.743042in}}%
\pgfpathlineto{\pgfqpoint{5.731789in}{0.743042in}}%
\pgfpathlineto{\pgfqpoint{5.732353in}{0.743041in}}%
\pgfpathlineto{\pgfqpoint{5.732918in}{0.743041in}}%
\pgfpathlineto{\pgfqpoint{5.733482in}{0.743040in}}%
\pgfpathlineto{\pgfqpoint{5.734046in}{0.743040in}}%
\pgfpathlineto{\pgfqpoint{5.734611in}{0.743039in}}%
\pgfpathlineto{\pgfqpoint{5.735175in}{0.743039in}}%
\pgfpathlineto{\pgfqpoint{5.735739in}{0.743038in}}%
\pgfpathlineto{\pgfqpoint{5.736304in}{0.743038in}}%
\pgfpathlineto{\pgfqpoint{5.736868in}{0.743037in}}%
\pgfpathlineto{\pgfqpoint{5.737432in}{0.743037in}}%
\pgfpathlineto{\pgfqpoint{5.737997in}{0.743036in}}%
\pgfpathlineto{\pgfqpoint{5.738561in}{0.743036in}}%
\pgfpathlineto{\pgfqpoint{5.739125in}{0.743035in}}%
\pgfpathlineto{\pgfqpoint{5.739690in}{0.743035in}}%
\pgfpathlineto{\pgfqpoint{5.740254in}{0.743034in}}%
\pgfpathlineto{\pgfqpoint{5.740819in}{0.743034in}}%
\pgfpathlineto{\pgfqpoint{5.741383in}{0.743033in}}%
\pgfpathlineto{\pgfqpoint{5.741947in}{0.743033in}}%
\pgfpathlineto{\pgfqpoint{5.742512in}{0.743032in}}%
\pgfpathlineto{\pgfqpoint{5.743076in}{0.743032in}}%
\pgfpathlineto{\pgfqpoint{5.743640in}{0.743031in}}%
\pgfpathlineto{\pgfqpoint{5.744205in}{0.743031in}}%
\pgfpathlineto{\pgfqpoint{5.744769in}{0.743030in}}%
\pgfpathlineto{\pgfqpoint{5.745333in}{0.743030in}}%
\pgfpathlineto{\pgfqpoint{5.745898in}{0.743029in}}%
\pgfpathlineto{\pgfqpoint{5.746462in}{0.743029in}}%
\pgfpathlineto{\pgfqpoint{5.747026in}{0.743029in}}%
\pgfpathlineto{\pgfqpoint{5.747591in}{0.743028in}}%
\pgfpathlineto{\pgfqpoint{5.748155in}{0.743028in}}%
\pgfpathlineto{\pgfqpoint{5.748719in}{0.743027in}}%
\pgfpathlineto{\pgfqpoint{5.749284in}{0.743027in}}%
\pgfpathlineto{\pgfqpoint{5.749848in}{0.743026in}}%
\pgfpathlineto{\pgfqpoint{5.750412in}{0.743026in}}%
\pgfpathlineto{\pgfqpoint{5.750977in}{0.743025in}}%
\pgfpathlineto{\pgfqpoint{5.751541in}{0.743025in}}%
\pgfpathlineto{\pgfqpoint{5.752105in}{0.743024in}}%
\pgfpathlineto{\pgfqpoint{5.752670in}{0.743024in}}%
\pgfpathlineto{\pgfqpoint{5.753234in}{0.743023in}}%
\pgfpathlineto{\pgfqpoint{5.753798in}{0.743023in}}%
\pgfpathlineto{\pgfqpoint{5.754363in}{0.743022in}}%
\pgfpathlineto{\pgfqpoint{5.754927in}{0.743022in}}%
\pgfpathlineto{\pgfqpoint{5.755491in}{0.743021in}}%
\pgfpathlineto{\pgfqpoint{5.756056in}{0.743021in}}%
\pgfpathlineto{\pgfqpoint{5.756620in}{0.743020in}}%
\pgfpathlineto{\pgfqpoint{5.757185in}{0.743020in}}%
\pgfpathlineto{\pgfqpoint{5.757749in}{0.743019in}}%
\pgfpathlineto{\pgfqpoint{5.758313in}{0.743019in}}%
\pgfpathlineto{\pgfqpoint{5.758878in}{0.743018in}}%
\pgfpathlineto{\pgfqpoint{5.759442in}{0.743018in}}%
\pgfpathlineto{\pgfqpoint{5.760006in}{0.743017in}}%
\pgfpathlineto{\pgfqpoint{5.760571in}{0.743017in}}%
\pgfpathlineto{\pgfqpoint{5.761135in}{0.743016in}}%
\pgfpathlineto{\pgfqpoint{5.761699in}{0.743016in}}%
\pgfpathlineto{\pgfqpoint{5.762264in}{0.743015in}}%
\pgfpathlineto{\pgfqpoint{5.762828in}{0.743015in}}%
\pgfpathlineto{\pgfqpoint{5.763392in}{0.743014in}}%
\pgfpathlineto{\pgfqpoint{5.763957in}{0.743014in}}%
\pgfpathlineto{\pgfqpoint{5.764521in}{0.743013in}}%
\pgfpathlineto{\pgfqpoint{5.765085in}{0.743013in}}%
\pgfpathlineto{\pgfqpoint{5.765650in}{0.743012in}}%
\pgfpathlineto{\pgfqpoint{5.766214in}{0.743012in}}%
\pgfpathlineto{\pgfqpoint{5.766778in}{0.743011in}}%
\pgfpathlineto{\pgfqpoint{5.767343in}{0.743011in}}%
\pgfpathlineto{\pgfqpoint{5.767907in}{0.743010in}}%
\pgfpathlineto{\pgfqpoint{5.768471in}{0.743010in}}%
\pgfpathlineto{\pgfqpoint{5.769036in}{0.743009in}}%
\pgfpathlineto{\pgfqpoint{5.769600in}{0.743009in}}%
\pgfpathlineto{\pgfqpoint{5.770164in}{0.743009in}}%
\pgfpathlineto{\pgfqpoint{5.770729in}{0.743008in}}%
\pgfpathlineto{\pgfqpoint{5.771293in}{0.743008in}}%
\pgfpathlineto{\pgfqpoint{5.771858in}{0.743007in}}%
\pgfpathlineto{\pgfqpoint{5.772422in}{0.743007in}}%
\pgfpathlineto{\pgfqpoint{5.772986in}{0.743006in}}%
\pgfpathlineto{\pgfqpoint{5.773551in}{0.743006in}}%
\pgfpathlineto{\pgfqpoint{5.774115in}{0.743005in}}%
\pgfpathlineto{\pgfqpoint{5.774679in}{0.743005in}}%
\pgfpathlineto{\pgfqpoint{5.775244in}{0.743004in}}%
\pgfpathlineto{\pgfqpoint{5.775808in}{0.743004in}}%
\pgfpathlineto{\pgfqpoint{5.776372in}{0.743003in}}%
\pgfpathlineto{\pgfqpoint{5.776937in}{0.743003in}}%
\pgfpathlineto{\pgfqpoint{5.777501in}{0.743002in}}%
\pgfpathlineto{\pgfqpoint{5.778065in}{0.743002in}}%
\pgfpathlineto{\pgfqpoint{5.778630in}{0.743001in}}%
\pgfpathlineto{\pgfqpoint{5.779194in}{0.743001in}}%
\pgfpathlineto{\pgfqpoint{5.779758in}{0.743000in}}%
\pgfpathlineto{\pgfqpoint{5.780323in}{0.743000in}}%
\pgfpathlineto{\pgfqpoint{5.780887in}{0.742999in}}%
\pgfpathlineto{\pgfqpoint{5.781451in}{0.742999in}}%
\pgfpathlineto{\pgfqpoint{5.782016in}{0.742998in}}%
\pgfpathlineto{\pgfqpoint{5.782580in}{0.742998in}}%
\pgfpathlineto{\pgfqpoint{5.783144in}{0.742997in}}%
\pgfpathlineto{\pgfqpoint{5.783709in}{0.742997in}}%
\pgfpathlineto{\pgfqpoint{5.784273in}{0.742996in}}%
\pgfpathlineto{\pgfqpoint{5.784837in}{0.742996in}}%
\pgfpathlineto{\pgfqpoint{5.785402in}{0.742995in}}%
\pgfpathlineto{\pgfqpoint{5.785966in}{0.742995in}}%
\pgfpathlineto{\pgfqpoint{5.786531in}{0.742994in}}%
\pgfpathlineto{\pgfqpoint{5.787095in}{0.742994in}}%
\pgfpathlineto{\pgfqpoint{5.787659in}{0.742993in}}%
\pgfpathlineto{\pgfqpoint{5.788224in}{0.742993in}}%
\pgfpathlineto{\pgfqpoint{5.788788in}{0.742992in}}%
\pgfpathlineto{\pgfqpoint{5.789352in}{0.742992in}}%
\pgfpathlineto{\pgfqpoint{5.789917in}{0.742991in}}%
\pgfpathlineto{\pgfqpoint{5.790481in}{0.742991in}}%
\pgfpathlineto{\pgfqpoint{5.791045in}{0.742990in}}%
\pgfpathlineto{\pgfqpoint{5.791610in}{0.742990in}}%
\pgfpathlineto{\pgfqpoint{5.792174in}{0.742990in}}%
\pgfpathlineto{\pgfqpoint{5.792738in}{0.742989in}}%
\pgfpathlineto{\pgfqpoint{5.793303in}{0.742989in}}%
\pgfpathlineto{\pgfqpoint{5.793867in}{0.742988in}}%
\pgfpathlineto{\pgfqpoint{5.794431in}{0.742988in}}%
\pgfpathlineto{\pgfqpoint{5.794996in}{0.742987in}}%
\pgfpathlineto{\pgfqpoint{5.795560in}{0.742987in}}%
\pgfpathlineto{\pgfqpoint{5.796124in}{0.742986in}}%
\pgfpathlineto{\pgfqpoint{5.796689in}{0.742986in}}%
\pgfpathlineto{\pgfqpoint{5.797253in}{0.742985in}}%
\pgfpathlineto{\pgfqpoint{5.797817in}{0.742985in}}%
\pgfpathlineto{\pgfqpoint{5.798382in}{0.742984in}}%
\pgfpathlineto{\pgfqpoint{5.798946in}{0.742984in}}%
\pgfpathlineto{\pgfqpoint{5.799510in}{0.742983in}}%
\pgfpathlineto{\pgfqpoint{5.800075in}{0.742983in}}%
\pgfpathlineto{\pgfqpoint{5.800639in}{0.742982in}}%
\pgfpathlineto{\pgfqpoint{5.801203in}{0.742982in}}%
\pgfpathlineto{\pgfqpoint{5.801768in}{0.742981in}}%
\pgfpathlineto{\pgfqpoint{5.802332in}{0.742981in}}%
\pgfpathlineto{\pgfqpoint{5.802897in}{0.742980in}}%
\pgfpathlineto{\pgfqpoint{5.803461in}{0.742980in}}%
\pgfpathlineto{\pgfqpoint{5.804025in}{0.742979in}}%
\pgfpathlineto{\pgfqpoint{5.804590in}{0.742979in}}%
\pgfpathlineto{\pgfqpoint{5.805154in}{0.742978in}}%
\pgfpathlineto{\pgfqpoint{5.805718in}{0.742978in}}%
\pgfpathlineto{\pgfqpoint{5.806283in}{0.742977in}}%
\pgfpathlineto{\pgfqpoint{5.806847in}{0.742977in}}%
\pgfpathlineto{\pgfqpoint{5.807411in}{0.742976in}}%
\pgfpathlineto{\pgfqpoint{5.807976in}{0.742976in}}%
\pgfpathlineto{\pgfqpoint{5.808540in}{0.742975in}}%
\pgfpathlineto{\pgfqpoint{5.809104in}{0.742975in}}%
\pgfpathlineto{\pgfqpoint{5.809669in}{0.742974in}}%
\pgfpathlineto{\pgfqpoint{5.810233in}{0.742974in}}%
\pgfpathlineto{\pgfqpoint{5.810797in}{0.742973in}}%
\pgfpathlineto{\pgfqpoint{5.811362in}{0.742973in}}%
\pgfpathlineto{\pgfqpoint{5.811926in}{0.742972in}}%
\pgfpathlineto{\pgfqpoint{5.812490in}{0.742972in}}%
\pgfpathlineto{\pgfqpoint{5.813055in}{0.742971in}}%
\pgfpathlineto{\pgfqpoint{5.813619in}{0.742971in}}%
\pgfpathlineto{\pgfqpoint{5.814183in}{0.742970in}}%
\pgfpathlineto{\pgfqpoint{5.814748in}{0.742970in}}%
\pgfpathlineto{\pgfqpoint{5.815312in}{0.742970in}}%
\pgfpathlineto{\pgfqpoint{5.815876in}{0.742969in}}%
\pgfpathlineto{\pgfqpoint{5.816441in}{0.742969in}}%
\pgfpathlineto{\pgfqpoint{5.817005in}{0.742968in}}%
\pgfpathlineto{\pgfqpoint{5.817570in}{0.742968in}}%
\pgfpathlineto{\pgfqpoint{5.818134in}{0.742967in}}%
\pgfpathlineto{\pgfqpoint{5.818698in}{0.742967in}}%
\pgfpathlineto{\pgfqpoint{5.819263in}{0.742966in}}%
\pgfpathlineto{\pgfqpoint{5.819827in}{0.742966in}}%
\pgfpathlineto{\pgfqpoint{5.820391in}{0.742965in}}%
\pgfpathlineto{\pgfqpoint{5.820956in}{0.742965in}}%
\pgfpathlineto{\pgfqpoint{5.821520in}{0.742964in}}%
\pgfpathlineto{\pgfqpoint{5.822084in}{0.742964in}}%
\pgfpathlineto{\pgfqpoint{5.822649in}{0.742963in}}%
\pgfpathlineto{\pgfqpoint{5.823213in}{0.742963in}}%
\pgfpathlineto{\pgfqpoint{5.823777in}{0.742962in}}%
\pgfpathlineto{\pgfqpoint{5.824342in}{0.742962in}}%
\pgfpathlineto{\pgfqpoint{5.824906in}{0.742961in}}%
\pgfpathlineto{\pgfqpoint{5.825470in}{0.742961in}}%
\pgfpathlineto{\pgfqpoint{5.826035in}{0.742960in}}%
\pgfpathlineto{\pgfqpoint{5.826599in}{0.742960in}}%
\pgfpathlineto{\pgfqpoint{5.827163in}{0.742959in}}%
\pgfpathlineto{\pgfqpoint{5.827728in}{0.742959in}}%
\pgfpathlineto{\pgfqpoint{5.828292in}{0.742958in}}%
\pgfpathlineto{\pgfqpoint{5.828856in}{0.742958in}}%
\pgfpathlineto{\pgfqpoint{5.829421in}{0.742957in}}%
\pgfpathlineto{\pgfqpoint{5.829985in}{0.742957in}}%
\pgfpathlineto{\pgfqpoint{5.830549in}{0.742956in}}%
\pgfpathlineto{\pgfqpoint{5.831114in}{0.742956in}}%
\pgfpathlineto{\pgfqpoint{5.831678in}{0.742955in}}%
\pgfpathlineto{\pgfqpoint{5.832243in}{0.742955in}}%
\pgfpathlineto{\pgfqpoint{5.832807in}{0.742954in}}%
\pgfpathlineto{\pgfqpoint{5.833371in}{0.742954in}}%
\pgfpathlineto{\pgfqpoint{5.833936in}{0.742953in}}%
\pgfpathlineto{\pgfqpoint{5.834500in}{0.742953in}}%
\pgfpathlineto{\pgfqpoint{5.835064in}{0.742952in}}%
\pgfpathlineto{\pgfqpoint{5.835629in}{0.742952in}}%
\pgfpathlineto{\pgfqpoint{5.836193in}{0.742951in}}%
\pgfpathlineto{\pgfqpoint{5.836757in}{0.742951in}}%
\pgfpathlineto{\pgfqpoint{5.837322in}{0.742951in}}%
\pgfpathlineto{\pgfqpoint{5.837886in}{0.742950in}}%
\pgfpathlineto{\pgfqpoint{5.838450in}{0.742950in}}%
\pgfpathlineto{\pgfqpoint{5.839015in}{0.742949in}}%
\pgfpathlineto{\pgfqpoint{5.839579in}{0.742949in}}%
\pgfpathlineto{\pgfqpoint{5.840143in}{0.742948in}}%
\pgfpathlineto{\pgfqpoint{5.840708in}{0.742948in}}%
\pgfpathlineto{\pgfqpoint{5.841272in}{0.742947in}}%
\pgfpathlineto{\pgfqpoint{5.841836in}{0.742947in}}%
\pgfpathlineto{\pgfqpoint{5.842401in}{0.742946in}}%
\pgfpathlineto{\pgfqpoint{5.842965in}{0.742946in}}%
\pgfpathlineto{\pgfqpoint{5.843529in}{0.742945in}}%
\pgfpathlineto{\pgfqpoint{5.844094in}{0.742945in}}%
\pgfpathlineto{\pgfqpoint{5.844658in}{0.742944in}}%
\pgfpathlineto{\pgfqpoint{5.845222in}{0.742944in}}%
\pgfpathlineto{\pgfqpoint{5.845787in}{0.742943in}}%
\pgfpathlineto{\pgfqpoint{5.846351in}{0.742943in}}%
\pgfpathlineto{\pgfqpoint{5.846916in}{0.742942in}}%
\pgfpathlineto{\pgfqpoint{5.847480in}{0.742942in}}%
\pgfpathlineto{\pgfqpoint{5.848044in}{0.742941in}}%
\pgfpathlineto{\pgfqpoint{5.848609in}{0.742941in}}%
\pgfpathlineto{\pgfqpoint{5.849173in}{0.742940in}}%
\pgfpathlineto{\pgfqpoint{5.849737in}{0.742940in}}%
\pgfpathlineto{\pgfqpoint{5.850302in}{0.742939in}}%
\pgfpathlineto{\pgfqpoint{5.850866in}{0.742939in}}%
\pgfpathlineto{\pgfqpoint{5.851430in}{0.742938in}}%
\pgfpathlineto{\pgfqpoint{5.851995in}{0.742938in}}%
\pgfpathlineto{\pgfqpoint{5.852559in}{0.742937in}}%
\pgfpathlineto{\pgfqpoint{5.853123in}{0.742937in}}%
\pgfpathlineto{\pgfqpoint{5.853688in}{0.742936in}}%
\pgfpathlineto{\pgfqpoint{5.854252in}{0.742936in}}%
\pgfpathlineto{\pgfqpoint{5.854816in}{0.742935in}}%
\pgfpathlineto{\pgfqpoint{5.855381in}{0.742935in}}%
\pgfpathlineto{\pgfqpoint{5.855945in}{0.742934in}}%
\pgfpathlineto{\pgfqpoint{5.856509in}{0.742934in}}%
\pgfpathlineto{\pgfqpoint{5.857074in}{0.742933in}}%
\pgfpathlineto{\pgfqpoint{5.857638in}{0.742933in}}%
\pgfpathlineto{\pgfqpoint{5.858202in}{0.742932in}}%
\pgfpathlineto{\pgfqpoint{5.858767in}{0.742932in}}%
\pgfpathlineto{\pgfqpoint{5.859331in}{0.742931in}}%
\pgfpathlineto{\pgfqpoint{5.859895in}{0.742931in}}%
\pgfpathlineto{\pgfqpoint{5.860460in}{0.742931in}}%
\pgfpathlineto{\pgfqpoint{5.861024in}{0.742930in}}%
\pgfpathlineto{\pgfqpoint{5.861588in}{0.742930in}}%
\pgfpathlineto{\pgfqpoint{5.862153in}{0.742929in}}%
\pgfpathlineto{\pgfqpoint{5.862717in}{0.742929in}}%
\pgfpathlineto{\pgfqpoint{5.863282in}{0.742928in}}%
\pgfpathlineto{\pgfqpoint{5.863846in}{0.742928in}}%
\pgfpathlineto{\pgfqpoint{5.864410in}{0.742927in}}%
\pgfpathlineto{\pgfqpoint{5.864975in}{0.742927in}}%
\pgfpathlineto{\pgfqpoint{5.865539in}{0.742926in}}%
\pgfpathlineto{\pgfqpoint{5.866103in}{0.742926in}}%
\pgfpathlineto{\pgfqpoint{5.866668in}{0.742925in}}%
\pgfpathlineto{\pgfqpoint{5.867232in}{0.742925in}}%
\pgfpathlineto{\pgfqpoint{5.867796in}{0.742924in}}%
\pgfpathlineto{\pgfqpoint{5.868361in}{0.742924in}}%
\pgfpathlineto{\pgfqpoint{5.868925in}{0.742923in}}%
\pgfpathlineto{\pgfqpoint{5.869489in}{0.742923in}}%
\pgfpathlineto{\pgfqpoint{5.870054in}{0.742922in}}%
\pgfpathlineto{\pgfqpoint{5.870618in}{0.742922in}}%
\pgfpathlineto{\pgfqpoint{5.871182in}{0.742921in}}%
\pgfpathlineto{\pgfqpoint{5.871747in}{0.742921in}}%
\pgfpathlineto{\pgfqpoint{5.872311in}{0.742920in}}%
\pgfpathlineto{\pgfqpoint{5.872875in}{0.742920in}}%
\pgfpathlineto{\pgfqpoint{5.873440in}{0.742919in}}%
\pgfpathlineto{\pgfqpoint{5.874004in}{0.742919in}}%
\pgfpathlineto{\pgfqpoint{5.874568in}{0.742918in}}%
\pgfpathlineto{\pgfqpoint{5.875133in}{0.742918in}}%
\pgfpathlineto{\pgfqpoint{5.875697in}{0.742917in}}%
\pgfpathlineto{\pgfqpoint{5.876261in}{0.742917in}}%
\pgfpathlineto{\pgfqpoint{5.876826in}{0.742916in}}%
\pgfpathlineto{\pgfqpoint{5.877390in}{0.742916in}}%
\pgfpathlineto{\pgfqpoint{5.877955in}{0.742915in}}%
\pgfpathlineto{\pgfqpoint{5.878519in}{0.742915in}}%
\pgfpathlineto{\pgfqpoint{5.879083in}{0.742914in}}%
\pgfpathlineto{\pgfqpoint{5.879648in}{0.742914in}}%
\pgfpathlineto{\pgfqpoint{5.880212in}{0.742913in}}%
\pgfpathlineto{\pgfqpoint{5.880776in}{0.742913in}}%
\pgfpathlineto{\pgfqpoint{5.881341in}{0.742912in}}%
\pgfpathlineto{\pgfqpoint{5.881905in}{0.742912in}}%
\pgfpathlineto{\pgfqpoint{5.882469in}{0.742912in}}%
\pgfpathlineto{\pgfqpoint{5.883034in}{0.742911in}}%
\pgfpathlineto{\pgfqpoint{5.883598in}{0.742911in}}%
\pgfpathlineto{\pgfqpoint{5.884162in}{0.742910in}}%
\pgfpathlineto{\pgfqpoint{5.884727in}{0.742910in}}%
\pgfpathlineto{\pgfqpoint{5.885291in}{0.742909in}}%
\pgfpathlineto{\pgfqpoint{5.885855in}{0.742909in}}%
\pgfpathlineto{\pgfqpoint{5.886420in}{0.742908in}}%
\pgfpathlineto{\pgfqpoint{5.886984in}{0.742908in}}%
\pgfpathlineto{\pgfqpoint{5.887548in}{0.742907in}}%
\pgfpathlineto{\pgfqpoint{5.888113in}{0.742907in}}%
\pgfpathlineto{\pgfqpoint{5.888677in}{0.742906in}}%
\pgfpathlineto{\pgfqpoint{5.889241in}{0.742906in}}%
\pgfpathlineto{\pgfqpoint{5.889806in}{0.742905in}}%
\pgfpathlineto{\pgfqpoint{5.890370in}{0.742905in}}%
\pgfpathlineto{\pgfqpoint{5.890934in}{0.742904in}}%
\pgfpathlineto{\pgfqpoint{5.891499in}{0.742904in}}%
\pgfpathlineto{\pgfqpoint{5.892063in}{0.742903in}}%
\pgfpathlineto{\pgfqpoint{5.892628in}{0.742903in}}%
\pgfpathlineto{\pgfqpoint{5.893192in}{0.742902in}}%
\pgfpathlineto{\pgfqpoint{5.893756in}{0.742902in}}%
\pgfpathlineto{\pgfqpoint{5.894321in}{0.742901in}}%
\pgfpathlineto{\pgfqpoint{5.894885in}{0.742901in}}%
\pgfpathlineto{\pgfqpoint{5.895449in}{0.742900in}}%
\pgfpathlineto{\pgfqpoint{5.896014in}{0.742900in}}%
\pgfpathlineto{\pgfqpoint{5.896578in}{0.742899in}}%
\pgfpathlineto{\pgfqpoint{5.897142in}{0.742899in}}%
\pgfpathlineto{\pgfqpoint{5.897707in}{0.742898in}}%
\pgfpathlineto{\pgfqpoint{5.898271in}{0.742898in}}%
\pgfpathlineto{\pgfqpoint{5.898835in}{0.742897in}}%
\pgfpathlineto{\pgfqpoint{5.899400in}{0.742897in}}%
\pgfpathlineto{\pgfqpoint{5.899964in}{0.742896in}}%
\pgfpathlineto{\pgfqpoint{5.900528in}{0.742896in}}%
\pgfpathlineto{\pgfqpoint{5.901093in}{0.742895in}}%
\pgfpathlineto{\pgfqpoint{5.901657in}{0.742895in}}%
\pgfpathlineto{\pgfqpoint{5.902221in}{0.742894in}}%
\pgfpathlineto{\pgfqpoint{5.902786in}{0.742894in}}%
\pgfpathlineto{\pgfqpoint{5.903350in}{0.742893in}}%
\pgfpathlineto{\pgfqpoint{5.903914in}{0.742893in}}%
\pgfpathlineto{\pgfqpoint{5.904479in}{0.742892in}}%
\pgfpathlineto{\pgfqpoint{5.905043in}{0.742892in}}%
\pgfpathlineto{\pgfqpoint{5.905607in}{0.742892in}}%
\pgfpathlineto{\pgfqpoint{5.906172in}{0.742891in}}%
\pgfpathlineto{\pgfqpoint{5.906736in}{0.742891in}}%
\pgfpathlineto{\pgfqpoint{5.907300in}{0.742890in}}%
\pgfpathlineto{\pgfqpoint{5.907865in}{0.742890in}}%
\pgfpathlineto{\pgfqpoint{5.908429in}{0.742889in}}%
\pgfpathlineto{\pgfqpoint{5.908994in}{0.742889in}}%
\pgfpathlineto{\pgfqpoint{5.909558in}{0.742888in}}%
\pgfpathlineto{\pgfqpoint{5.910122in}{0.742888in}}%
\pgfpathlineto{\pgfqpoint{5.910687in}{0.742887in}}%
\pgfpathlineto{\pgfqpoint{5.911251in}{0.742887in}}%
\pgfpathlineto{\pgfqpoint{5.911815in}{0.742886in}}%
\pgfpathlineto{\pgfqpoint{5.912380in}{0.742886in}}%
\pgfpathlineto{\pgfqpoint{5.912944in}{0.742885in}}%
\pgfpathlineto{\pgfqpoint{5.913508in}{0.742885in}}%
\pgfpathlineto{\pgfqpoint{5.914073in}{0.742884in}}%
\pgfpathlineto{\pgfqpoint{5.914637in}{0.742884in}}%
\pgfpathlineto{\pgfqpoint{5.915201in}{0.742883in}}%
\pgfpathlineto{\pgfqpoint{5.915766in}{0.742883in}}%
\pgfpathlineto{\pgfqpoint{5.916330in}{0.742882in}}%
\pgfpathlineto{\pgfqpoint{5.916894in}{0.742882in}}%
\pgfpathlineto{\pgfqpoint{5.917459in}{0.742881in}}%
\pgfpathlineto{\pgfqpoint{5.918023in}{0.742881in}}%
\pgfpathlineto{\pgfqpoint{5.918587in}{0.742880in}}%
\pgfpathlineto{\pgfqpoint{5.919152in}{0.742880in}}%
\pgfpathlineto{\pgfqpoint{5.919716in}{0.742879in}}%
\pgfpathlineto{\pgfqpoint{5.920280in}{0.742879in}}%
\pgfpathlineto{\pgfqpoint{5.920845in}{0.742878in}}%
\pgfpathlineto{\pgfqpoint{5.921409in}{0.742878in}}%
\pgfpathlineto{\pgfqpoint{5.921973in}{0.742877in}}%
\pgfpathlineto{\pgfqpoint{5.922538in}{0.742877in}}%
\pgfpathlineto{\pgfqpoint{5.923102in}{0.742876in}}%
\pgfpathlineto{\pgfqpoint{5.923667in}{0.742876in}}%
\pgfpathlineto{\pgfqpoint{5.924231in}{0.742875in}}%
\pgfpathlineto{\pgfqpoint{5.924795in}{0.742875in}}%
\pgfpathlineto{\pgfqpoint{5.925360in}{0.742874in}}%
\pgfpathlineto{\pgfqpoint{5.925924in}{0.742874in}}%
\pgfpathlineto{\pgfqpoint{5.926488in}{0.742873in}}%
\pgfpathlineto{\pgfqpoint{5.927053in}{0.742873in}}%
\pgfpathlineto{\pgfqpoint{5.927617in}{0.742872in}}%
\pgfpathlineto{\pgfqpoint{5.928181in}{0.742872in}}%
\pgfpathlineto{\pgfqpoint{5.928746in}{0.742872in}}%
\pgfpathlineto{\pgfqpoint{5.929310in}{0.742871in}}%
\pgfpathlineto{\pgfqpoint{5.929874in}{0.742871in}}%
\pgfpathlineto{\pgfqpoint{5.930439in}{0.742870in}}%
\pgfpathlineto{\pgfqpoint{5.931003in}{0.742870in}}%
\pgfpathlineto{\pgfqpoint{5.931567in}{0.742869in}}%
\pgfpathlineto{\pgfqpoint{5.932132in}{0.742869in}}%
\pgfpathlineto{\pgfqpoint{5.932696in}{0.742868in}}%
\pgfpathlineto{\pgfqpoint{5.933260in}{0.742868in}}%
\pgfpathlineto{\pgfqpoint{5.933825in}{0.742867in}}%
\pgfpathlineto{\pgfqpoint{5.934389in}{0.742867in}}%
\pgfpathlineto{\pgfqpoint{5.934953in}{0.742866in}}%
\pgfpathlineto{\pgfqpoint{5.935518in}{0.742866in}}%
\pgfpathlineto{\pgfqpoint{5.936082in}{0.742865in}}%
\pgfpathlineto{\pgfqpoint{5.936646in}{0.742865in}}%
\pgfpathlineto{\pgfqpoint{5.937211in}{0.742864in}}%
\pgfpathlineto{\pgfqpoint{5.937775in}{0.742864in}}%
\pgfpathlineto{\pgfqpoint{5.938340in}{0.742863in}}%
\pgfpathlineto{\pgfqpoint{5.938904in}{0.742863in}}%
\pgfpathlineto{\pgfqpoint{5.939468in}{0.742862in}}%
\pgfpathlineto{\pgfqpoint{5.940033in}{0.742862in}}%
\pgfpathlineto{\pgfqpoint{5.940597in}{0.742861in}}%
\pgfpathlineto{\pgfqpoint{5.941161in}{0.742861in}}%
\pgfpathlineto{\pgfqpoint{5.941726in}{0.742860in}}%
\pgfpathlineto{\pgfqpoint{5.942290in}{0.742860in}}%
\pgfpathlineto{\pgfqpoint{5.942854in}{0.742859in}}%
\pgfpathlineto{\pgfqpoint{5.943419in}{0.742859in}}%
\pgfpathlineto{\pgfqpoint{5.943983in}{0.742858in}}%
\pgfpathlineto{\pgfqpoint{5.944547in}{0.742858in}}%
\pgfpathlineto{\pgfqpoint{5.945112in}{0.742857in}}%
\pgfpathlineto{\pgfqpoint{5.945676in}{0.742857in}}%
\pgfpathlineto{\pgfqpoint{5.946240in}{0.742856in}}%
\pgfpathlineto{\pgfqpoint{5.946805in}{0.742856in}}%
\pgfpathlineto{\pgfqpoint{5.947369in}{0.742855in}}%
\pgfpathlineto{\pgfqpoint{5.947933in}{0.742855in}}%
\pgfpathlineto{\pgfqpoint{5.948498in}{0.742854in}}%
\pgfpathlineto{\pgfqpoint{5.949062in}{0.742854in}}%
\pgfpathlineto{\pgfqpoint{5.949626in}{0.742853in}}%
\pgfpathlineto{\pgfqpoint{5.950191in}{0.742853in}}%
\pgfpathlineto{\pgfqpoint{5.950755in}{0.742853in}}%
\pgfpathlineto{\pgfqpoint{5.951319in}{0.742852in}}%
\pgfpathlineto{\pgfqpoint{5.951884in}{0.742852in}}%
\pgfpathlineto{\pgfqpoint{5.952448in}{0.742851in}}%
\pgfpathlineto{\pgfqpoint{5.953012in}{0.742851in}}%
\pgfpathlineto{\pgfqpoint{5.953577in}{0.742850in}}%
\pgfpathlineto{\pgfqpoint{5.954141in}{0.742850in}}%
\pgfpathlineto{\pgfqpoint{5.954706in}{0.742849in}}%
\pgfpathlineto{\pgfqpoint{5.955270in}{0.742849in}}%
\pgfpathlineto{\pgfqpoint{5.955834in}{0.742848in}}%
\pgfpathlineto{\pgfqpoint{5.956399in}{0.742848in}}%
\pgfpathlineto{\pgfqpoint{5.956963in}{0.742847in}}%
\pgfpathlineto{\pgfqpoint{5.957527in}{0.742847in}}%
\pgfpathlineto{\pgfqpoint{5.958092in}{0.742846in}}%
\pgfpathlineto{\pgfqpoint{5.958656in}{0.742846in}}%
\pgfpathlineto{\pgfqpoint{5.959220in}{0.742845in}}%
\pgfpathlineto{\pgfqpoint{5.959785in}{0.742845in}}%
\pgfpathlineto{\pgfqpoint{5.960349in}{0.742844in}}%
\pgfpathlineto{\pgfqpoint{5.960913in}{0.742844in}}%
\pgfpathlineto{\pgfqpoint{5.961478in}{0.742843in}}%
\pgfpathlineto{\pgfqpoint{5.962042in}{0.742843in}}%
\pgfpathlineto{\pgfqpoint{5.962606in}{0.742842in}}%
\pgfpathlineto{\pgfqpoint{5.963171in}{0.742842in}}%
\pgfpathlineto{\pgfqpoint{5.963735in}{0.742841in}}%
\pgfpathlineto{\pgfqpoint{5.964299in}{0.742841in}}%
\pgfpathlineto{\pgfqpoint{5.964864in}{0.742840in}}%
\pgfpathlineto{\pgfqpoint{5.965428in}{0.742840in}}%
\pgfpathlineto{\pgfqpoint{5.965992in}{0.742839in}}%
\pgfpathlineto{\pgfqpoint{5.966557in}{0.742839in}}%
\pgfpathlineto{\pgfqpoint{5.967121in}{0.742838in}}%
\pgfpathlineto{\pgfqpoint{5.967685in}{0.742838in}}%
\pgfpathlineto{\pgfqpoint{5.968250in}{0.742837in}}%
\pgfpathlineto{\pgfqpoint{5.968814in}{0.742837in}}%
\pgfpathlineto{\pgfqpoint{5.969379in}{0.742836in}}%
\pgfpathlineto{\pgfqpoint{5.969943in}{0.742836in}}%
\pgfpathlineto{\pgfqpoint{5.970507in}{0.742835in}}%
\pgfpathlineto{\pgfqpoint{5.971072in}{0.742835in}}%
\pgfpathlineto{\pgfqpoint{5.971636in}{0.742834in}}%
\pgfpathlineto{\pgfqpoint{5.972200in}{0.742834in}}%
\pgfpathlineto{\pgfqpoint{5.972765in}{0.742833in}}%
\pgfpathlineto{\pgfqpoint{5.973329in}{0.742833in}}%
\pgfpathlineto{\pgfqpoint{5.973893in}{0.742833in}}%
\pgfpathlineto{\pgfqpoint{5.974458in}{0.742832in}}%
\pgfpathlineto{\pgfqpoint{5.975022in}{0.742832in}}%
\pgfpathlineto{\pgfqpoint{5.975586in}{0.742831in}}%
\pgfpathlineto{\pgfqpoint{5.976151in}{0.742831in}}%
\pgfpathlineto{\pgfqpoint{5.976715in}{0.742830in}}%
\pgfpathlineto{\pgfqpoint{5.977279in}{0.742830in}}%
\pgfpathlineto{\pgfqpoint{5.977844in}{0.742829in}}%
\pgfpathlineto{\pgfqpoint{5.978408in}{0.742829in}}%
\pgfpathlineto{\pgfqpoint{5.978972in}{0.742828in}}%
\pgfpathlineto{\pgfqpoint{5.979537in}{0.742828in}}%
\pgfpathlineto{\pgfqpoint{5.980101in}{0.742827in}}%
\pgfpathlineto{\pgfqpoint{5.980665in}{0.742827in}}%
\pgfpathlineto{\pgfqpoint{5.981230in}{0.742826in}}%
\pgfpathlineto{\pgfqpoint{5.981794in}{0.742826in}}%
\pgfpathlineto{\pgfqpoint{5.982358in}{0.742825in}}%
\pgfpathlineto{\pgfqpoint{5.982923in}{0.742825in}}%
\pgfpathlineto{\pgfqpoint{5.983487in}{0.742824in}}%
\pgfpathlineto{\pgfqpoint{5.984052in}{0.742824in}}%
\pgfpathlineto{\pgfqpoint{5.984616in}{0.742823in}}%
\pgfpathlineto{\pgfqpoint{5.985180in}{0.742823in}}%
\pgfpathlineto{\pgfqpoint{5.985745in}{0.742822in}}%
\pgfpathlineto{\pgfqpoint{5.986309in}{0.742822in}}%
\pgfpathlineto{\pgfqpoint{5.986873in}{0.742821in}}%
\pgfpathlineto{\pgfqpoint{5.987438in}{0.742821in}}%
\pgfpathlineto{\pgfqpoint{5.988002in}{0.742820in}}%
\pgfpathlineto{\pgfqpoint{5.988566in}{0.742820in}}%
\pgfpathlineto{\pgfqpoint{5.989131in}{0.742819in}}%
\pgfpathlineto{\pgfqpoint{5.989695in}{0.742819in}}%
\pgfpathlineto{\pgfqpoint{5.990259in}{0.742818in}}%
\pgfpathlineto{\pgfqpoint{5.990824in}{0.742818in}}%
\pgfpathlineto{\pgfqpoint{5.991388in}{0.742817in}}%
\pgfpathlineto{\pgfqpoint{5.991952in}{0.742817in}}%
\pgfpathlineto{\pgfqpoint{5.992517in}{0.742816in}}%
\pgfpathlineto{\pgfqpoint{5.993081in}{0.742749in}}%
\pgfpathlineto{\pgfqpoint{5.993645in}{0.742594in}}%
\pgfpathlineto{\pgfqpoint{5.994210in}{0.742437in}}%
\pgfpathlineto{\pgfqpoint{5.994774in}{0.742281in}}%
\pgfpathlineto{\pgfqpoint{5.995338in}{0.742125in}}%
\pgfpathlineto{\pgfqpoint{5.995903in}{0.741969in}}%
\pgfpathlineto{\pgfqpoint{5.996467in}{0.741813in}}%
\pgfpathlineto{\pgfqpoint{5.997031in}{0.741657in}}%
\pgfpathlineto{\pgfqpoint{5.997596in}{0.741500in}}%
\pgfpathlineto{\pgfqpoint{5.998160in}{0.741344in}}%
\pgfpathlineto{\pgfqpoint{5.998724in}{0.741188in}}%
\pgfpathlineto{\pgfqpoint{5.999289in}{0.741032in}}%
\pgfpathlineto{\pgfqpoint{5.999853in}{0.740876in}}%
\pgfpathlineto{\pgfqpoint{6.000418in}{0.740719in}}%
\pgfpathlineto{\pgfqpoint{6.000982in}{0.740563in}}%
\pgfpathlineto{\pgfqpoint{6.001546in}{0.740407in}}%
\pgfpathlineto{\pgfqpoint{6.002111in}{0.740251in}}%
\pgfpathlineto{\pgfqpoint{6.002675in}{0.740095in}}%
\pgfpathlineto{\pgfqpoint{6.003239in}{0.739939in}}%
\pgfpathlineto{\pgfqpoint{6.003804in}{0.739782in}}%
\pgfpathlineto{\pgfqpoint{6.004368in}{0.739656in}}%
\pgfpathlineto{\pgfqpoint{6.004368in}{0.739656in}}%
\pgfpathlineto{\pgfqpoint{6.004368in}{0.739656in}}%
\pgfpathlineto{\pgfqpoint{6.003804in}{0.739656in}}%
\pgfpathlineto{\pgfqpoint{6.003239in}{0.739656in}}%
\pgfpathlineto{\pgfqpoint{6.002675in}{0.739656in}}%
\pgfpathlineto{\pgfqpoint{6.002111in}{0.739656in}}%
\pgfpathlineto{\pgfqpoint{6.001546in}{0.739656in}}%
\pgfpathlineto{\pgfqpoint{6.000982in}{0.739656in}}%
\pgfpathlineto{\pgfqpoint{6.000418in}{0.739656in}}%
\pgfpathlineto{\pgfqpoint{5.999853in}{0.739656in}}%
\pgfpathlineto{\pgfqpoint{5.999289in}{0.739656in}}%
\pgfpathlineto{\pgfqpoint{5.998724in}{0.739656in}}%
\pgfpathlineto{\pgfqpoint{5.998160in}{0.739656in}}%
\pgfpathlineto{\pgfqpoint{5.997596in}{0.739656in}}%
\pgfpathlineto{\pgfqpoint{5.997031in}{0.739656in}}%
\pgfpathlineto{\pgfqpoint{5.996467in}{0.739656in}}%
\pgfpathlineto{\pgfqpoint{5.995903in}{0.739656in}}%
\pgfpathlineto{\pgfqpoint{5.995338in}{0.739656in}}%
\pgfpathlineto{\pgfqpoint{5.994774in}{0.739656in}}%
\pgfpathlineto{\pgfqpoint{5.994210in}{0.739656in}}%
\pgfpathlineto{\pgfqpoint{5.993645in}{0.739656in}}%
\pgfpathlineto{\pgfqpoint{5.993081in}{0.739656in}}%
\pgfpathlineto{\pgfqpoint{5.992517in}{0.739656in}}%
\pgfpathlineto{\pgfqpoint{5.991952in}{0.739656in}}%
\pgfpathlineto{\pgfqpoint{5.991388in}{0.739656in}}%
\pgfpathlineto{\pgfqpoint{5.990824in}{0.739656in}}%
\pgfpathlineto{\pgfqpoint{5.990259in}{0.739656in}}%
\pgfpathlineto{\pgfqpoint{5.989695in}{0.739656in}}%
\pgfpathlineto{\pgfqpoint{5.989131in}{0.739656in}}%
\pgfpathlineto{\pgfqpoint{5.988566in}{0.739656in}}%
\pgfpathlineto{\pgfqpoint{5.988002in}{0.739656in}}%
\pgfpathlineto{\pgfqpoint{5.987438in}{0.739656in}}%
\pgfpathlineto{\pgfqpoint{5.986873in}{0.739656in}}%
\pgfpathlineto{\pgfqpoint{5.986309in}{0.739656in}}%
\pgfpathlineto{\pgfqpoint{5.985745in}{0.739656in}}%
\pgfpathlineto{\pgfqpoint{5.985180in}{0.739656in}}%
\pgfpathlineto{\pgfqpoint{5.984616in}{0.739656in}}%
\pgfpathlineto{\pgfqpoint{5.984052in}{0.739656in}}%
\pgfpathlineto{\pgfqpoint{5.983487in}{0.739656in}}%
\pgfpathlineto{\pgfqpoint{5.982923in}{0.739656in}}%
\pgfpathlineto{\pgfqpoint{5.982358in}{0.739656in}}%
\pgfpathlineto{\pgfqpoint{5.981794in}{0.739656in}}%
\pgfpathlineto{\pgfqpoint{5.981230in}{0.739656in}}%
\pgfpathlineto{\pgfqpoint{5.980665in}{0.739656in}}%
\pgfpathlineto{\pgfqpoint{5.980101in}{0.739656in}}%
\pgfpathlineto{\pgfqpoint{5.979537in}{0.739656in}}%
\pgfpathlineto{\pgfqpoint{5.978972in}{0.739656in}}%
\pgfpathlineto{\pgfqpoint{5.978408in}{0.739656in}}%
\pgfpathlineto{\pgfqpoint{5.977844in}{0.739656in}}%
\pgfpathlineto{\pgfqpoint{5.977279in}{0.739656in}}%
\pgfpathlineto{\pgfqpoint{5.976715in}{0.739656in}}%
\pgfpathlineto{\pgfqpoint{5.976151in}{0.739656in}}%
\pgfpathlineto{\pgfqpoint{5.975586in}{0.739656in}}%
\pgfpathlineto{\pgfqpoint{5.975022in}{0.739656in}}%
\pgfpathlineto{\pgfqpoint{5.974458in}{0.739656in}}%
\pgfpathlineto{\pgfqpoint{5.973893in}{0.739656in}}%
\pgfpathlineto{\pgfqpoint{5.973329in}{0.739656in}}%
\pgfpathlineto{\pgfqpoint{5.972765in}{0.739656in}}%
\pgfpathlineto{\pgfqpoint{5.972200in}{0.739656in}}%
\pgfpathlineto{\pgfqpoint{5.971636in}{0.739656in}}%
\pgfpathlineto{\pgfqpoint{5.971072in}{0.739656in}}%
\pgfpathlineto{\pgfqpoint{5.970507in}{0.739656in}}%
\pgfpathlineto{\pgfqpoint{5.969943in}{0.739656in}}%
\pgfpathlineto{\pgfqpoint{5.969379in}{0.739656in}}%
\pgfpathlineto{\pgfqpoint{5.968814in}{0.739656in}}%
\pgfpathlineto{\pgfqpoint{5.968250in}{0.739656in}}%
\pgfpathlineto{\pgfqpoint{5.967685in}{0.739656in}}%
\pgfpathlineto{\pgfqpoint{5.967121in}{0.739656in}}%
\pgfpathlineto{\pgfqpoint{5.966557in}{0.739656in}}%
\pgfpathlineto{\pgfqpoint{5.965992in}{0.739656in}}%
\pgfpathlineto{\pgfqpoint{5.965428in}{0.739656in}}%
\pgfpathlineto{\pgfqpoint{5.964864in}{0.739656in}}%
\pgfpathlineto{\pgfqpoint{5.964299in}{0.739656in}}%
\pgfpathlineto{\pgfqpoint{5.963735in}{0.739656in}}%
\pgfpathlineto{\pgfqpoint{5.963171in}{0.739656in}}%
\pgfpathlineto{\pgfqpoint{5.962606in}{0.739656in}}%
\pgfpathlineto{\pgfqpoint{5.962042in}{0.739656in}}%
\pgfpathlineto{\pgfqpoint{5.961478in}{0.739656in}}%
\pgfpathlineto{\pgfqpoint{5.960913in}{0.739656in}}%
\pgfpathlineto{\pgfqpoint{5.960349in}{0.739656in}}%
\pgfpathlineto{\pgfqpoint{5.959785in}{0.739656in}}%
\pgfpathlineto{\pgfqpoint{5.959220in}{0.739656in}}%
\pgfpathlineto{\pgfqpoint{5.958656in}{0.739656in}}%
\pgfpathlineto{\pgfqpoint{5.958092in}{0.739656in}}%
\pgfpathlineto{\pgfqpoint{5.957527in}{0.739656in}}%
\pgfpathlineto{\pgfqpoint{5.956963in}{0.739656in}}%
\pgfpathlineto{\pgfqpoint{5.956399in}{0.739656in}}%
\pgfpathlineto{\pgfqpoint{5.955834in}{0.739656in}}%
\pgfpathlineto{\pgfqpoint{5.955270in}{0.739656in}}%
\pgfpathlineto{\pgfqpoint{5.954706in}{0.739656in}}%
\pgfpathlineto{\pgfqpoint{5.954141in}{0.739656in}}%
\pgfpathlineto{\pgfqpoint{5.953577in}{0.739656in}}%
\pgfpathlineto{\pgfqpoint{5.953012in}{0.739656in}}%
\pgfpathlineto{\pgfqpoint{5.952448in}{0.739656in}}%
\pgfpathlineto{\pgfqpoint{5.951884in}{0.739656in}}%
\pgfpathlineto{\pgfqpoint{5.951319in}{0.739656in}}%
\pgfpathlineto{\pgfqpoint{5.950755in}{0.739656in}}%
\pgfpathlineto{\pgfqpoint{5.950191in}{0.739656in}}%
\pgfpathlineto{\pgfqpoint{5.949626in}{0.739656in}}%
\pgfpathlineto{\pgfqpoint{5.949062in}{0.739656in}}%
\pgfpathlineto{\pgfqpoint{5.948498in}{0.739656in}}%
\pgfpathlineto{\pgfqpoint{5.947933in}{0.739656in}}%
\pgfpathlineto{\pgfqpoint{5.947369in}{0.739656in}}%
\pgfpathlineto{\pgfqpoint{5.946805in}{0.739656in}}%
\pgfpathlineto{\pgfqpoint{5.946240in}{0.739656in}}%
\pgfpathlineto{\pgfqpoint{5.945676in}{0.739656in}}%
\pgfpathlineto{\pgfqpoint{5.945112in}{0.739656in}}%
\pgfpathlineto{\pgfqpoint{5.944547in}{0.739656in}}%
\pgfpathlineto{\pgfqpoint{5.943983in}{0.739656in}}%
\pgfpathlineto{\pgfqpoint{5.943419in}{0.739656in}}%
\pgfpathlineto{\pgfqpoint{5.942854in}{0.739656in}}%
\pgfpathlineto{\pgfqpoint{5.942290in}{0.739656in}}%
\pgfpathlineto{\pgfqpoint{5.941726in}{0.739656in}}%
\pgfpathlineto{\pgfqpoint{5.941161in}{0.739656in}}%
\pgfpathlineto{\pgfqpoint{5.940597in}{0.739656in}}%
\pgfpathlineto{\pgfqpoint{5.940033in}{0.739656in}}%
\pgfpathlineto{\pgfqpoint{5.939468in}{0.739656in}}%
\pgfpathlineto{\pgfqpoint{5.938904in}{0.739656in}}%
\pgfpathlineto{\pgfqpoint{5.938340in}{0.739656in}}%
\pgfpathlineto{\pgfqpoint{5.937775in}{0.739656in}}%
\pgfpathlineto{\pgfqpoint{5.937211in}{0.739656in}}%
\pgfpathlineto{\pgfqpoint{5.936646in}{0.739656in}}%
\pgfpathlineto{\pgfqpoint{5.936082in}{0.739656in}}%
\pgfpathlineto{\pgfqpoint{5.935518in}{0.739656in}}%
\pgfpathlineto{\pgfqpoint{5.934953in}{0.739656in}}%
\pgfpathlineto{\pgfqpoint{5.934389in}{0.739656in}}%
\pgfpathlineto{\pgfqpoint{5.933825in}{0.739656in}}%
\pgfpathlineto{\pgfqpoint{5.933260in}{0.739656in}}%
\pgfpathlineto{\pgfqpoint{5.932696in}{0.739656in}}%
\pgfpathlineto{\pgfqpoint{5.932132in}{0.739656in}}%
\pgfpathlineto{\pgfqpoint{5.931567in}{0.739656in}}%
\pgfpathlineto{\pgfqpoint{5.931003in}{0.739656in}}%
\pgfpathlineto{\pgfqpoint{5.930439in}{0.739656in}}%
\pgfpathlineto{\pgfqpoint{5.929874in}{0.739656in}}%
\pgfpathlineto{\pgfqpoint{5.929310in}{0.739656in}}%
\pgfpathlineto{\pgfqpoint{5.928746in}{0.739656in}}%
\pgfpathlineto{\pgfqpoint{5.928181in}{0.739656in}}%
\pgfpathlineto{\pgfqpoint{5.927617in}{0.739656in}}%
\pgfpathlineto{\pgfqpoint{5.927053in}{0.739656in}}%
\pgfpathlineto{\pgfqpoint{5.926488in}{0.739656in}}%
\pgfpathlineto{\pgfqpoint{5.925924in}{0.739656in}}%
\pgfpathlineto{\pgfqpoint{5.925360in}{0.739656in}}%
\pgfpathlineto{\pgfqpoint{5.924795in}{0.739656in}}%
\pgfpathlineto{\pgfqpoint{5.924231in}{0.739656in}}%
\pgfpathlineto{\pgfqpoint{5.923667in}{0.739656in}}%
\pgfpathlineto{\pgfqpoint{5.923102in}{0.739656in}}%
\pgfpathlineto{\pgfqpoint{5.922538in}{0.739656in}}%
\pgfpathlineto{\pgfqpoint{5.921973in}{0.739656in}}%
\pgfpathlineto{\pgfqpoint{5.921409in}{0.739656in}}%
\pgfpathlineto{\pgfqpoint{5.920845in}{0.739656in}}%
\pgfpathlineto{\pgfqpoint{5.920280in}{0.739656in}}%
\pgfpathlineto{\pgfqpoint{5.919716in}{0.739656in}}%
\pgfpathlineto{\pgfqpoint{5.919152in}{0.739656in}}%
\pgfpathlineto{\pgfqpoint{5.918587in}{0.739656in}}%
\pgfpathlineto{\pgfqpoint{5.918023in}{0.739656in}}%
\pgfpathlineto{\pgfqpoint{5.917459in}{0.739656in}}%
\pgfpathlineto{\pgfqpoint{5.916894in}{0.739656in}}%
\pgfpathlineto{\pgfqpoint{5.916330in}{0.739656in}}%
\pgfpathlineto{\pgfqpoint{5.915766in}{0.739656in}}%
\pgfpathlineto{\pgfqpoint{5.915201in}{0.739656in}}%
\pgfpathlineto{\pgfqpoint{5.914637in}{0.739656in}}%
\pgfpathlineto{\pgfqpoint{5.914073in}{0.739656in}}%
\pgfpathlineto{\pgfqpoint{5.913508in}{0.739656in}}%
\pgfpathlineto{\pgfqpoint{5.912944in}{0.739656in}}%
\pgfpathlineto{\pgfqpoint{5.912380in}{0.739656in}}%
\pgfpathlineto{\pgfqpoint{5.911815in}{0.739656in}}%
\pgfpathlineto{\pgfqpoint{5.911251in}{0.739656in}}%
\pgfpathlineto{\pgfqpoint{5.910687in}{0.739656in}}%
\pgfpathlineto{\pgfqpoint{5.910122in}{0.739656in}}%
\pgfpathlineto{\pgfqpoint{5.909558in}{0.739656in}}%
\pgfpathlineto{\pgfqpoint{5.908994in}{0.739656in}}%
\pgfpathlineto{\pgfqpoint{5.908429in}{0.739656in}}%
\pgfpathlineto{\pgfqpoint{5.907865in}{0.739656in}}%
\pgfpathlineto{\pgfqpoint{5.907300in}{0.739656in}}%
\pgfpathlineto{\pgfqpoint{5.906736in}{0.739656in}}%
\pgfpathlineto{\pgfqpoint{5.906172in}{0.739656in}}%
\pgfpathlineto{\pgfqpoint{5.905607in}{0.739656in}}%
\pgfpathlineto{\pgfqpoint{5.905043in}{0.739656in}}%
\pgfpathlineto{\pgfqpoint{5.904479in}{0.739656in}}%
\pgfpathlineto{\pgfqpoint{5.903914in}{0.739656in}}%
\pgfpathlineto{\pgfqpoint{5.903350in}{0.739656in}}%
\pgfpathlineto{\pgfqpoint{5.902786in}{0.739656in}}%
\pgfpathlineto{\pgfqpoint{5.902221in}{0.739656in}}%
\pgfpathlineto{\pgfqpoint{5.901657in}{0.739656in}}%
\pgfpathlineto{\pgfqpoint{5.901093in}{0.739656in}}%
\pgfpathlineto{\pgfqpoint{5.900528in}{0.739656in}}%
\pgfpathlineto{\pgfqpoint{5.899964in}{0.739656in}}%
\pgfpathlineto{\pgfqpoint{5.899400in}{0.739656in}}%
\pgfpathlineto{\pgfqpoint{5.898835in}{0.739656in}}%
\pgfpathlineto{\pgfqpoint{5.898271in}{0.739656in}}%
\pgfpathlineto{\pgfqpoint{5.897707in}{0.739656in}}%
\pgfpathlineto{\pgfqpoint{5.897142in}{0.739656in}}%
\pgfpathlineto{\pgfqpoint{5.896578in}{0.739656in}}%
\pgfpathlineto{\pgfqpoint{5.896014in}{0.739656in}}%
\pgfpathlineto{\pgfqpoint{5.895449in}{0.739656in}}%
\pgfpathlineto{\pgfqpoint{5.894885in}{0.739656in}}%
\pgfpathlineto{\pgfqpoint{5.894321in}{0.739656in}}%
\pgfpathlineto{\pgfqpoint{5.893756in}{0.739656in}}%
\pgfpathlineto{\pgfqpoint{5.893192in}{0.739656in}}%
\pgfpathlineto{\pgfqpoint{5.892628in}{0.739656in}}%
\pgfpathlineto{\pgfqpoint{5.892063in}{0.739656in}}%
\pgfpathlineto{\pgfqpoint{5.891499in}{0.739656in}}%
\pgfpathlineto{\pgfqpoint{5.890934in}{0.739656in}}%
\pgfpathlineto{\pgfqpoint{5.890370in}{0.739656in}}%
\pgfpathlineto{\pgfqpoint{5.889806in}{0.739656in}}%
\pgfpathlineto{\pgfqpoint{5.889241in}{0.739656in}}%
\pgfpathlineto{\pgfqpoint{5.888677in}{0.739656in}}%
\pgfpathlineto{\pgfqpoint{5.888113in}{0.739656in}}%
\pgfpathlineto{\pgfqpoint{5.887548in}{0.739656in}}%
\pgfpathlineto{\pgfqpoint{5.886984in}{0.739656in}}%
\pgfpathlineto{\pgfqpoint{5.886420in}{0.739656in}}%
\pgfpathlineto{\pgfqpoint{5.885855in}{0.739656in}}%
\pgfpathlineto{\pgfqpoint{5.885291in}{0.739656in}}%
\pgfpathlineto{\pgfqpoint{5.884727in}{0.739656in}}%
\pgfpathlineto{\pgfqpoint{5.884162in}{0.739656in}}%
\pgfpathlineto{\pgfqpoint{5.883598in}{0.739656in}}%
\pgfpathlineto{\pgfqpoint{5.883034in}{0.739656in}}%
\pgfpathlineto{\pgfqpoint{5.882469in}{0.739656in}}%
\pgfpathlineto{\pgfqpoint{5.881905in}{0.739656in}}%
\pgfpathlineto{\pgfqpoint{5.881341in}{0.739656in}}%
\pgfpathlineto{\pgfqpoint{5.880776in}{0.739656in}}%
\pgfpathlineto{\pgfqpoint{5.880212in}{0.739656in}}%
\pgfpathlineto{\pgfqpoint{5.879648in}{0.739656in}}%
\pgfpathlineto{\pgfqpoint{5.879083in}{0.739656in}}%
\pgfpathlineto{\pgfqpoint{5.878519in}{0.739656in}}%
\pgfpathlineto{\pgfqpoint{5.877955in}{0.739656in}}%
\pgfpathlineto{\pgfqpoint{5.877390in}{0.739656in}}%
\pgfpathlineto{\pgfqpoint{5.876826in}{0.739656in}}%
\pgfpathlineto{\pgfqpoint{5.876261in}{0.739656in}}%
\pgfpathlineto{\pgfqpoint{5.875697in}{0.739656in}}%
\pgfpathlineto{\pgfqpoint{5.875133in}{0.739656in}}%
\pgfpathlineto{\pgfqpoint{5.874568in}{0.739656in}}%
\pgfpathlineto{\pgfqpoint{5.874004in}{0.739656in}}%
\pgfpathlineto{\pgfqpoint{5.873440in}{0.739656in}}%
\pgfpathlineto{\pgfqpoint{5.872875in}{0.739656in}}%
\pgfpathlineto{\pgfqpoint{5.872311in}{0.739656in}}%
\pgfpathlineto{\pgfqpoint{5.871747in}{0.739656in}}%
\pgfpathlineto{\pgfqpoint{5.871182in}{0.739656in}}%
\pgfpathlineto{\pgfqpoint{5.870618in}{0.739656in}}%
\pgfpathlineto{\pgfqpoint{5.870054in}{0.739656in}}%
\pgfpathlineto{\pgfqpoint{5.869489in}{0.739656in}}%
\pgfpathlineto{\pgfqpoint{5.868925in}{0.739656in}}%
\pgfpathlineto{\pgfqpoint{5.868361in}{0.739656in}}%
\pgfpathlineto{\pgfqpoint{5.867796in}{0.739656in}}%
\pgfpathlineto{\pgfqpoint{5.867232in}{0.739656in}}%
\pgfpathlineto{\pgfqpoint{5.866668in}{0.739656in}}%
\pgfpathlineto{\pgfqpoint{5.866103in}{0.739656in}}%
\pgfpathlineto{\pgfqpoint{5.865539in}{0.739656in}}%
\pgfpathlineto{\pgfqpoint{5.864975in}{0.739656in}}%
\pgfpathlineto{\pgfqpoint{5.864410in}{0.739656in}}%
\pgfpathlineto{\pgfqpoint{5.863846in}{0.739656in}}%
\pgfpathlineto{\pgfqpoint{5.863282in}{0.739656in}}%
\pgfpathlineto{\pgfqpoint{5.862717in}{0.739656in}}%
\pgfpathlineto{\pgfqpoint{5.862153in}{0.739656in}}%
\pgfpathlineto{\pgfqpoint{5.861588in}{0.739656in}}%
\pgfpathlineto{\pgfqpoint{5.861024in}{0.739656in}}%
\pgfpathlineto{\pgfqpoint{5.860460in}{0.739656in}}%
\pgfpathlineto{\pgfqpoint{5.859895in}{0.739656in}}%
\pgfpathlineto{\pgfqpoint{5.859331in}{0.739656in}}%
\pgfpathlineto{\pgfqpoint{5.858767in}{0.739656in}}%
\pgfpathlineto{\pgfqpoint{5.858202in}{0.739656in}}%
\pgfpathlineto{\pgfqpoint{5.857638in}{0.739656in}}%
\pgfpathlineto{\pgfqpoint{5.857074in}{0.739656in}}%
\pgfpathlineto{\pgfqpoint{5.856509in}{0.739656in}}%
\pgfpathlineto{\pgfqpoint{5.855945in}{0.739656in}}%
\pgfpathlineto{\pgfqpoint{5.855381in}{0.739656in}}%
\pgfpathlineto{\pgfqpoint{5.854816in}{0.739656in}}%
\pgfpathlineto{\pgfqpoint{5.854252in}{0.739656in}}%
\pgfpathlineto{\pgfqpoint{5.853688in}{0.739656in}}%
\pgfpathlineto{\pgfqpoint{5.853123in}{0.739656in}}%
\pgfpathlineto{\pgfqpoint{5.852559in}{0.739656in}}%
\pgfpathlineto{\pgfqpoint{5.851995in}{0.739656in}}%
\pgfpathlineto{\pgfqpoint{5.851430in}{0.739656in}}%
\pgfpathlineto{\pgfqpoint{5.850866in}{0.739656in}}%
\pgfpathlineto{\pgfqpoint{5.850302in}{0.739656in}}%
\pgfpathlineto{\pgfqpoint{5.849737in}{0.739656in}}%
\pgfpathlineto{\pgfqpoint{5.849173in}{0.739656in}}%
\pgfpathlineto{\pgfqpoint{5.848609in}{0.739656in}}%
\pgfpathlineto{\pgfqpoint{5.848044in}{0.739656in}}%
\pgfpathlineto{\pgfqpoint{5.847480in}{0.739656in}}%
\pgfpathlineto{\pgfqpoint{5.846916in}{0.739656in}}%
\pgfpathlineto{\pgfqpoint{5.846351in}{0.739656in}}%
\pgfpathlineto{\pgfqpoint{5.845787in}{0.739656in}}%
\pgfpathlineto{\pgfqpoint{5.845222in}{0.739656in}}%
\pgfpathlineto{\pgfqpoint{5.844658in}{0.739656in}}%
\pgfpathlineto{\pgfqpoint{5.844094in}{0.739656in}}%
\pgfpathlineto{\pgfqpoint{5.843529in}{0.739656in}}%
\pgfpathlineto{\pgfqpoint{5.842965in}{0.739656in}}%
\pgfpathlineto{\pgfqpoint{5.842401in}{0.739656in}}%
\pgfpathlineto{\pgfqpoint{5.841836in}{0.739656in}}%
\pgfpathlineto{\pgfqpoint{5.841272in}{0.739656in}}%
\pgfpathlineto{\pgfqpoint{5.840708in}{0.739656in}}%
\pgfpathlineto{\pgfqpoint{5.840143in}{0.739656in}}%
\pgfpathlineto{\pgfqpoint{5.839579in}{0.739656in}}%
\pgfpathlineto{\pgfqpoint{5.839015in}{0.739656in}}%
\pgfpathlineto{\pgfqpoint{5.838450in}{0.739656in}}%
\pgfpathlineto{\pgfqpoint{5.837886in}{0.739656in}}%
\pgfpathlineto{\pgfqpoint{5.837322in}{0.739656in}}%
\pgfpathlineto{\pgfqpoint{5.836757in}{0.739656in}}%
\pgfpathlineto{\pgfqpoint{5.836193in}{0.739656in}}%
\pgfpathlineto{\pgfqpoint{5.835629in}{0.739656in}}%
\pgfpathlineto{\pgfqpoint{5.835064in}{0.739656in}}%
\pgfpathlineto{\pgfqpoint{5.834500in}{0.739656in}}%
\pgfpathlineto{\pgfqpoint{5.833936in}{0.739656in}}%
\pgfpathlineto{\pgfqpoint{5.833371in}{0.739656in}}%
\pgfpathlineto{\pgfqpoint{5.832807in}{0.739656in}}%
\pgfpathlineto{\pgfqpoint{5.832243in}{0.739656in}}%
\pgfpathlineto{\pgfqpoint{5.831678in}{0.739656in}}%
\pgfpathlineto{\pgfqpoint{5.831114in}{0.739656in}}%
\pgfpathlineto{\pgfqpoint{5.830549in}{0.739656in}}%
\pgfpathlineto{\pgfqpoint{5.829985in}{0.739656in}}%
\pgfpathlineto{\pgfqpoint{5.829421in}{0.739656in}}%
\pgfpathlineto{\pgfqpoint{5.828856in}{0.739656in}}%
\pgfpathlineto{\pgfqpoint{5.828292in}{0.739656in}}%
\pgfpathlineto{\pgfqpoint{5.827728in}{0.739656in}}%
\pgfpathlineto{\pgfqpoint{5.827163in}{0.739656in}}%
\pgfpathlineto{\pgfqpoint{5.826599in}{0.739656in}}%
\pgfpathlineto{\pgfqpoint{5.826035in}{0.739656in}}%
\pgfpathlineto{\pgfqpoint{5.825470in}{0.739656in}}%
\pgfpathlineto{\pgfqpoint{5.824906in}{0.739656in}}%
\pgfpathlineto{\pgfqpoint{5.824342in}{0.739656in}}%
\pgfpathlineto{\pgfqpoint{5.823777in}{0.739656in}}%
\pgfpathlineto{\pgfqpoint{5.823213in}{0.739656in}}%
\pgfpathlineto{\pgfqpoint{5.822649in}{0.739656in}}%
\pgfpathlineto{\pgfqpoint{5.822084in}{0.739656in}}%
\pgfpathlineto{\pgfqpoint{5.821520in}{0.739656in}}%
\pgfpathlineto{\pgfqpoint{5.820956in}{0.739656in}}%
\pgfpathlineto{\pgfqpoint{5.820391in}{0.739656in}}%
\pgfpathlineto{\pgfqpoint{5.819827in}{0.739656in}}%
\pgfpathlineto{\pgfqpoint{5.819263in}{0.739656in}}%
\pgfpathlineto{\pgfqpoint{5.818698in}{0.739656in}}%
\pgfpathlineto{\pgfqpoint{5.818134in}{0.739656in}}%
\pgfpathlineto{\pgfqpoint{5.817570in}{0.739656in}}%
\pgfpathlineto{\pgfqpoint{5.817005in}{0.739656in}}%
\pgfpathlineto{\pgfqpoint{5.816441in}{0.739656in}}%
\pgfpathlineto{\pgfqpoint{5.815876in}{0.739656in}}%
\pgfpathlineto{\pgfqpoint{5.815312in}{0.739656in}}%
\pgfpathlineto{\pgfqpoint{5.814748in}{0.739656in}}%
\pgfpathlineto{\pgfqpoint{5.814183in}{0.739656in}}%
\pgfpathlineto{\pgfqpoint{5.813619in}{0.739656in}}%
\pgfpathlineto{\pgfqpoint{5.813055in}{0.739656in}}%
\pgfpathlineto{\pgfqpoint{5.812490in}{0.739656in}}%
\pgfpathlineto{\pgfqpoint{5.811926in}{0.739656in}}%
\pgfpathlineto{\pgfqpoint{5.811362in}{0.739656in}}%
\pgfpathlineto{\pgfqpoint{5.810797in}{0.739656in}}%
\pgfpathlineto{\pgfqpoint{5.810233in}{0.739656in}}%
\pgfpathlineto{\pgfqpoint{5.809669in}{0.739656in}}%
\pgfpathlineto{\pgfqpoint{5.809104in}{0.739656in}}%
\pgfpathlineto{\pgfqpoint{5.808540in}{0.739656in}}%
\pgfpathlineto{\pgfqpoint{5.807976in}{0.739656in}}%
\pgfpathlineto{\pgfqpoint{5.807411in}{0.739656in}}%
\pgfpathlineto{\pgfqpoint{5.806847in}{0.739656in}}%
\pgfpathlineto{\pgfqpoint{5.806283in}{0.739656in}}%
\pgfpathlineto{\pgfqpoint{5.805718in}{0.739656in}}%
\pgfpathlineto{\pgfqpoint{5.805154in}{0.739656in}}%
\pgfpathlineto{\pgfqpoint{5.804590in}{0.739656in}}%
\pgfpathlineto{\pgfqpoint{5.804025in}{0.739656in}}%
\pgfpathlineto{\pgfqpoint{5.803461in}{0.739656in}}%
\pgfpathlineto{\pgfqpoint{5.802897in}{0.739656in}}%
\pgfpathlineto{\pgfqpoint{5.802332in}{0.739656in}}%
\pgfpathlineto{\pgfqpoint{5.801768in}{0.739656in}}%
\pgfpathlineto{\pgfqpoint{5.801203in}{0.739656in}}%
\pgfpathlineto{\pgfqpoint{5.800639in}{0.739656in}}%
\pgfpathlineto{\pgfqpoint{5.800075in}{0.739656in}}%
\pgfpathlineto{\pgfqpoint{5.799510in}{0.739656in}}%
\pgfpathlineto{\pgfqpoint{5.798946in}{0.739656in}}%
\pgfpathlineto{\pgfqpoint{5.798382in}{0.739656in}}%
\pgfpathlineto{\pgfqpoint{5.797817in}{0.739656in}}%
\pgfpathlineto{\pgfqpoint{5.797253in}{0.739656in}}%
\pgfpathlineto{\pgfqpoint{5.796689in}{0.739656in}}%
\pgfpathlineto{\pgfqpoint{5.796124in}{0.739656in}}%
\pgfpathlineto{\pgfqpoint{5.795560in}{0.739656in}}%
\pgfpathlineto{\pgfqpoint{5.794996in}{0.739656in}}%
\pgfpathlineto{\pgfqpoint{5.794431in}{0.739656in}}%
\pgfpathlineto{\pgfqpoint{5.793867in}{0.739656in}}%
\pgfpathlineto{\pgfqpoint{5.793303in}{0.739656in}}%
\pgfpathlineto{\pgfqpoint{5.792738in}{0.739656in}}%
\pgfpathlineto{\pgfqpoint{5.792174in}{0.739656in}}%
\pgfpathlineto{\pgfqpoint{5.791610in}{0.739656in}}%
\pgfpathlineto{\pgfqpoint{5.791045in}{0.739656in}}%
\pgfpathlineto{\pgfqpoint{5.790481in}{0.739656in}}%
\pgfpathlineto{\pgfqpoint{5.789917in}{0.739656in}}%
\pgfpathlineto{\pgfqpoint{5.789352in}{0.739656in}}%
\pgfpathlineto{\pgfqpoint{5.788788in}{0.739656in}}%
\pgfpathlineto{\pgfqpoint{5.788224in}{0.739656in}}%
\pgfpathlineto{\pgfqpoint{5.787659in}{0.739656in}}%
\pgfpathlineto{\pgfqpoint{5.787095in}{0.739656in}}%
\pgfpathlineto{\pgfqpoint{5.786531in}{0.739656in}}%
\pgfpathlineto{\pgfqpoint{5.785966in}{0.739656in}}%
\pgfpathlineto{\pgfqpoint{5.785402in}{0.739656in}}%
\pgfpathlineto{\pgfqpoint{5.784837in}{0.739656in}}%
\pgfpathlineto{\pgfqpoint{5.784273in}{0.739656in}}%
\pgfpathlineto{\pgfqpoint{5.783709in}{0.739656in}}%
\pgfpathlineto{\pgfqpoint{5.783144in}{0.739656in}}%
\pgfpathlineto{\pgfqpoint{5.782580in}{0.739656in}}%
\pgfpathlineto{\pgfqpoint{5.782016in}{0.739656in}}%
\pgfpathlineto{\pgfqpoint{5.781451in}{0.739656in}}%
\pgfpathlineto{\pgfqpoint{5.780887in}{0.739656in}}%
\pgfpathlineto{\pgfqpoint{5.780323in}{0.739656in}}%
\pgfpathlineto{\pgfqpoint{5.779758in}{0.739656in}}%
\pgfpathlineto{\pgfqpoint{5.779194in}{0.739656in}}%
\pgfpathlineto{\pgfqpoint{5.778630in}{0.739656in}}%
\pgfpathlineto{\pgfqpoint{5.778065in}{0.739656in}}%
\pgfpathlineto{\pgfqpoint{5.777501in}{0.739656in}}%
\pgfpathlineto{\pgfqpoint{5.776937in}{0.739656in}}%
\pgfpathlineto{\pgfqpoint{5.776372in}{0.739656in}}%
\pgfpathlineto{\pgfqpoint{5.775808in}{0.739656in}}%
\pgfpathlineto{\pgfqpoint{5.775244in}{0.739656in}}%
\pgfpathlineto{\pgfqpoint{5.774679in}{0.739656in}}%
\pgfpathlineto{\pgfqpoint{5.774115in}{0.739656in}}%
\pgfpathlineto{\pgfqpoint{5.773551in}{0.739656in}}%
\pgfpathlineto{\pgfqpoint{5.772986in}{0.739656in}}%
\pgfpathlineto{\pgfqpoint{5.772422in}{0.739656in}}%
\pgfpathlineto{\pgfqpoint{5.771858in}{0.739656in}}%
\pgfpathlineto{\pgfqpoint{5.771293in}{0.739656in}}%
\pgfpathlineto{\pgfqpoint{5.770729in}{0.739656in}}%
\pgfpathlineto{\pgfqpoint{5.770164in}{0.739656in}}%
\pgfpathlineto{\pgfqpoint{5.769600in}{0.739656in}}%
\pgfpathlineto{\pgfqpoint{5.769036in}{0.739656in}}%
\pgfpathlineto{\pgfqpoint{5.768471in}{0.739656in}}%
\pgfpathlineto{\pgfqpoint{5.767907in}{0.739656in}}%
\pgfpathlineto{\pgfqpoint{5.767343in}{0.739656in}}%
\pgfpathlineto{\pgfqpoint{5.766778in}{0.739656in}}%
\pgfpathlineto{\pgfqpoint{5.766214in}{0.739656in}}%
\pgfpathlineto{\pgfqpoint{5.765650in}{0.739656in}}%
\pgfpathlineto{\pgfqpoint{5.765085in}{0.739656in}}%
\pgfpathlineto{\pgfqpoint{5.764521in}{0.739656in}}%
\pgfpathlineto{\pgfqpoint{5.763957in}{0.739656in}}%
\pgfpathlineto{\pgfqpoint{5.763392in}{0.739656in}}%
\pgfpathlineto{\pgfqpoint{5.762828in}{0.739656in}}%
\pgfpathlineto{\pgfqpoint{5.762264in}{0.739656in}}%
\pgfpathlineto{\pgfqpoint{5.761699in}{0.739656in}}%
\pgfpathlineto{\pgfqpoint{5.761135in}{0.739656in}}%
\pgfpathlineto{\pgfqpoint{5.760571in}{0.739656in}}%
\pgfpathlineto{\pgfqpoint{5.760006in}{0.739656in}}%
\pgfpathlineto{\pgfqpoint{5.759442in}{0.739656in}}%
\pgfpathlineto{\pgfqpoint{5.758878in}{0.739656in}}%
\pgfpathlineto{\pgfqpoint{5.758313in}{0.739656in}}%
\pgfpathlineto{\pgfqpoint{5.757749in}{0.739656in}}%
\pgfpathlineto{\pgfqpoint{5.757185in}{0.739656in}}%
\pgfpathlineto{\pgfqpoint{5.756620in}{0.739656in}}%
\pgfpathlineto{\pgfqpoint{5.756056in}{0.739656in}}%
\pgfpathlineto{\pgfqpoint{5.755491in}{0.739656in}}%
\pgfpathlineto{\pgfqpoint{5.754927in}{0.739656in}}%
\pgfpathlineto{\pgfqpoint{5.754363in}{0.739656in}}%
\pgfpathlineto{\pgfqpoint{5.753798in}{0.739656in}}%
\pgfpathlineto{\pgfqpoint{5.753234in}{0.739656in}}%
\pgfpathlineto{\pgfqpoint{5.752670in}{0.739656in}}%
\pgfpathlineto{\pgfqpoint{5.752105in}{0.739656in}}%
\pgfpathlineto{\pgfqpoint{5.751541in}{0.739656in}}%
\pgfpathlineto{\pgfqpoint{5.750977in}{0.739656in}}%
\pgfpathlineto{\pgfqpoint{5.750412in}{0.739656in}}%
\pgfpathlineto{\pgfqpoint{5.749848in}{0.739656in}}%
\pgfpathlineto{\pgfqpoint{5.749284in}{0.739656in}}%
\pgfpathlineto{\pgfqpoint{5.748719in}{0.739656in}}%
\pgfpathlineto{\pgfqpoint{5.748155in}{0.739656in}}%
\pgfpathlineto{\pgfqpoint{5.747591in}{0.739656in}}%
\pgfpathlineto{\pgfqpoint{5.747026in}{0.739656in}}%
\pgfpathlineto{\pgfqpoint{5.746462in}{0.739656in}}%
\pgfpathlineto{\pgfqpoint{5.745898in}{0.739656in}}%
\pgfpathlineto{\pgfqpoint{5.745333in}{0.739656in}}%
\pgfpathlineto{\pgfqpoint{5.744769in}{0.739656in}}%
\pgfpathlineto{\pgfqpoint{5.744205in}{0.739656in}}%
\pgfpathlineto{\pgfqpoint{5.743640in}{0.739656in}}%
\pgfpathlineto{\pgfqpoint{5.743076in}{0.739656in}}%
\pgfpathlineto{\pgfqpoint{5.742512in}{0.739656in}}%
\pgfpathlineto{\pgfqpoint{5.741947in}{0.739656in}}%
\pgfpathlineto{\pgfqpoint{5.741383in}{0.739656in}}%
\pgfpathlineto{\pgfqpoint{5.740819in}{0.739656in}}%
\pgfpathlineto{\pgfqpoint{5.740254in}{0.739656in}}%
\pgfpathlineto{\pgfqpoint{5.739690in}{0.739656in}}%
\pgfpathlineto{\pgfqpoint{5.739125in}{0.739656in}}%
\pgfpathlineto{\pgfqpoint{5.738561in}{0.739656in}}%
\pgfpathlineto{\pgfqpoint{5.737997in}{0.739656in}}%
\pgfpathlineto{\pgfqpoint{5.737432in}{0.739656in}}%
\pgfpathlineto{\pgfqpoint{5.736868in}{0.739656in}}%
\pgfpathlineto{\pgfqpoint{5.736304in}{0.739656in}}%
\pgfpathlineto{\pgfqpoint{5.735739in}{0.739656in}}%
\pgfpathlineto{\pgfqpoint{5.735175in}{0.739656in}}%
\pgfpathlineto{\pgfqpoint{5.734611in}{0.739656in}}%
\pgfpathlineto{\pgfqpoint{5.734046in}{0.739656in}}%
\pgfpathlineto{\pgfqpoint{5.733482in}{0.739656in}}%
\pgfpathlineto{\pgfqpoint{5.732918in}{0.739656in}}%
\pgfpathlineto{\pgfqpoint{5.732353in}{0.739656in}}%
\pgfpathlineto{\pgfqpoint{5.731789in}{0.739656in}}%
\pgfpathlineto{\pgfqpoint{5.731225in}{0.739656in}}%
\pgfpathlineto{\pgfqpoint{5.730660in}{0.739656in}}%
\pgfpathlineto{\pgfqpoint{5.730096in}{0.739656in}}%
\pgfpathlineto{\pgfqpoint{5.729532in}{0.739656in}}%
\pgfpathlineto{\pgfqpoint{5.728967in}{0.739656in}}%
\pgfpathlineto{\pgfqpoint{5.728403in}{0.739656in}}%
\pgfpathlineto{\pgfqpoint{5.727839in}{0.739656in}}%
\pgfpathlineto{\pgfqpoint{5.727274in}{0.739656in}}%
\pgfpathlineto{\pgfqpoint{5.726710in}{0.739656in}}%
\pgfpathlineto{\pgfqpoint{5.726146in}{0.739656in}}%
\pgfpathlineto{\pgfqpoint{5.725581in}{0.739656in}}%
\pgfpathlineto{\pgfqpoint{5.725017in}{0.739656in}}%
\pgfpathlineto{\pgfqpoint{5.724452in}{0.739656in}}%
\pgfpathlineto{\pgfqpoint{5.723888in}{0.739656in}}%
\pgfpathlineto{\pgfqpoint{5.723324in}{0.739656in}}%
\pgfpathlineto{\pgfqpoint{5.722759in}{0.739656in}}%
\pgfpathlineto{\pgfqpoint{5.722195in}{0.739656in}}%
\pgfpathlineto{\pgfqpoint{5.721631in}{0.739656in}}%
\pgfpathlineto{\pgfqpoint{5.721066in}{0.739656in}}%
\pgfpathlineto{\pgfqpoint{5.720502in}{0.739656in}}%
\pgfpathlineto{\pgfqpoint{5.719938in}{0.739656in}}%
\pgfpathlineto{\pgfqpoint{5.719373in}{0.739656in}}%
\pgfpathlineto{\pgfqpoint{5.718809in}{0.739656in}}%
\pgfpathlineto{\pgfqpoint{5.718245in}{0.739656in}}%
\pgfpathlineto{\pgfqpoint{5.717680in}{0.739656in}}%
\pgfpathlineto{\pgfqpoint{5.717116in}{0.739656in}}%
\pgfpathlineto{\pgfqpoint{5.716552in}{0.739656in}}%
\pgfpathlineto{\pgfqpoint{5.715987in}{0.739656in}}%
\pgfpathlineto{\pgfqpoint{5.715423in}{0.739656in}}%
\pgfpathlineto{\pgfqpoint{5.714859in}{0.739656in}}%
\pgfpathlineto{\pgfqpoint{5.714294in}{0.739656in}}%
\pgfpathlineto{\pgfqpoint{5.713730in}{0.739656in}}%
\pgfpathlineto{\pgfqpoint{5.713166in}{0.739656in}}%
\pgfpathlineto{\pgfqpoint{5.712601in}{0.739656in}}%
\pgfpathlineto{\pgfqpoint{5.712037in}{0.739656in}}%
\pgfpathlineto{\pgfqpoint{5.711473in}{0.739656in}}%
\pgfpathlineto{\pgfqpoint{5.710908in}{0.739656in}}%
\pgfpathlineto{\pgfqpoint{5.710344in}{0.739656in}}%
\pgfpathlineto{\pgfqpoint{5.709779in}{0.739656in}}%
\pgfpathlineto{\pgfqpoint{5.709215in}{0.739656in}}%
\pgfpathlineto{\pgfqpoint{5.708651in}{0.739656in}}%
\pgfpathlineto{\pgfqpoint{5.708086in}{0.739656in}}%
\pgfpathlineto{\pgfqpoint{5.707522in}{0.739656in}}%
\pgfpathlineto{\pgfqpoint{5.706958in}{0.739656in}}%
\pgfpathlineto{\pgfqpoint{5.706393in}{0.739656in}}%
\pgfpathlineto{\pgfqpoint{5.705829in}{0.739656in}}%
\pgfpathlineto{\pgfqpoint{5.705265in}{0.739656in}}%
\pgfpathlineto{\pgfqpoint{5.704700in}{0.739656in}}%
\pgfpathlineto{\pgfqpoint{5.704136in}{0.739656in}}%
\pgfpathlineto{\pgfqpoint{5.703572in}{0.739656in}}%
\pgfpathlineto{\pgfqpoint{5.703007in}{0.739656in}}%
\pgfpathlineto{\pgfqpoint{5.702443in}{0.739656in}}%
\pgfpathlineto{\pgfqpoint{5.701879in}{0.739656in}}%
\pgfpathlineto{\pgfqpoint{5.701314in}{0.739656in}}%
\pgfpathlineto{\pgfqpoint{5.700750in}{0.739656in}}%
\pgfpathlineto{\pgfqpoint{5.700186in}{0.739656in}}%
\pgfpathlineto{\pgfqpoint{5.699621in}{0.739656in}}%
\pgfpathlineto{\pgfqpoint{5.699057in}{0.739656in}}%
\pgfpathlineto{\pgfqpoint{5.698493in}{0.739656in}}%
\pgfpathlineto{\pgfqpoint{5.697928in}{0.739656in}}%
\pgfpathlineto{\pgfqpoint{5.697364in}{0.739656in}}%
\pgfpathlineto{\pgfqpoint{5.696800in}{0.739656in}}%
\pgfpathlineto{\pgfqpoint{5.696235in}{0.739656in}}%
\pgfpathlineto{\pgfqpoint{5.695671in}{0.739656in}}%
\pgfpathlineto{\pgfqpoint{5.695107in}{0.739656in}}%
\pgfpathlineto{\pgfqpoint{5.694542in}{0.739656in}}%
\pgfpathlineto{\pgfqpoint{5.693978in}{0.739656in}}%
\pgfpathlineto{\pgfqpoint{5.693413in}{0.739656in}}%
\pgfpathlineto{\pgfqpoint{5.692849in}{0.739656in}}%
\pgfpathlineto{\pgfqpoint{5.692285in}{0.739656in}}%
\pgfpathlineto{\pgfqpoint{5.691720in}{0.739656in}}%
\pgfpathlineto{\pgfqpoint{5.691156in}{0.739656in}}%
\pgfpathlineto{\pgfqpoint{5.690592in}{0.739656in}}%
\pgfpathlineto{\pgfqpoint{5.690027in}{0.739656in}}%
\pgfpathlineto{\pgfqpoint{5.689463in}{0.739656in}}%
\pgfpathlineto{\pgfqpoint{5.688899in}{0.739656in}}%
\pgfpathlineto{\pgfqpoint{5.688334in}{0.739656in}}%
\pgfpathlineto{\pgfqpoint{5.687770in}{0.739656in}}%
\pgfpathlineto{\pgfqpoint{5.687206in}{0.739656in}}%
\pgfpathlineto{\pgfqpoint{5.686641in}{0.739656in}}%
\pgfpathlineto{\pgfqpoint{5.686077in}{0.739656in}}%
\pgfpathlineto{\pgfqpoint{5.685513in}{0.739656in}}%
\pgfpathlineto{\pgfqpoint{5.684948in}{0.739656in}}%
\pgfpathlineto{\pgfqpoint{5.684384in}{0.739656in}}%
\pgfpathlineto{\pgfqpoint{5.683820in}{0.739656in}}%
\pgfpathlineto{\pgfqpoint{5.683255in}{0.739656in}}%
\pgfpathlineto{\pgfqpoint{5.682691in}{0.739656in}}%
\pgfpathlineto{\pgfqpoint{5.682127in}{0.739656in}}%
\pgfpathlineto{\pgfqpoint{5.681562in}{0.739656in}}%
\pgfpathlineto{\pgfqpoint{5.680998in}{0.739656in}}%
\pgfpathlineto{\pgfqpoint{5.680434in}{0.739656in}}%
\pgfpathlineto{\pgfqpoint{5.679869in}{0.739656in}}%
\pgfpathlineto{\pgfqpoint{5.679305in}{0.739656in}}%
\pgfpathlineto{\pgfqpoint{5.678740in}{0.739656in}}%
\pgfpathlineto{\pgfqpoint{5.678176in}{0.739656in}}%
\pgfpathlineto{\pgfqpoint{5.677612in}{0.739656in}}%
\pgfpathlineto{\pgfqpoint{5.677047in}{0.739656in}}%
\pgfpathlineto{\pgfqpoint{5.676483in}{0.739656in}}%
\pgfpathlineto{\pgfqpoint{5.675919in}{0.739656in}}%
\pgfpathlineto{\pgfqpoint{5.675354in}{0.739656in}}%
\pgfpathlineto{\pgfqpoint{5.674790in}{0.739656in}}%
\pgfpathlineto{\pgfqpoint{5.674226in}{0.739656in}}%
\pgfpathlineto{\pgfqpoint{5.673661in}{0.739656in}}%
\pgfpathlineto{\pgfqpoint{5.673097in}{0.739656in}}%
\pgfpathlineto{\pgfqpoint{5.672533in}{0.739656in}}%
\pgfpathlineto{\pgfqpoint{5.671968in}{0.739656in}}%
\pgfpathlineto{\pgfqpoint{5.671404in}{0.739656in}}%
\pgfpathlineto{\pgfqpoint{5.670840in}{0.739656in}}%
\pgfpathlineto{\pgfqpoint{5.670275in}{0.739656in}}%
\pgfpathlineto{\pgfqpoint{5.669711in}{0.739656in}}%
\pgfpathlineto{\pgfqpoint{5.669147in}{0.739656in}}%
\pgfpathlineto{\pgfqpoint{5.668582in}{0.739656in}}%
\pgfpathlineto{\pgfqpoint{5.668018in}{0.739656in}}%
\pgfpathlineto{\pgfqpoint{5.667454in}{0.739656in}}%
\pgfpathlineto{\pgfqpoint{5.666889in}{0.739656in}}%
\pgfpathlineto{\pgfqpoint{5.666325in}{0.739656in}}%
\pgfpathlineto{\pgfqpoint{5.665761in}{0.739656in}}%
\pgfpathlineto{\pgfqpoint{5.665196in}{0.739656in}}%
\pgfpathlineto{\pgfqpoint{5.664632in}{0.739656in}}%
\pgfpathlineto{\pgfqpoint{5.664067in}{0.739656in}}%
\pgfpathlineto{\pgfqpoint{5.663503in}{0.739656in}}%
\pgfpathlineto{\pgfqpoint{5.662939in}{0.739656in}}%
\pgfpathlineto{\pgfqpoint{5.662374in}{0.739656in}}%
\pgfpathlineto{\pgfqpoint{5.661810in}{0.739656in}}%
\pgfpathlineto{\pgfqpoint{5.661246in}{0.739656in}}%
\pgfpathlineto{\pgfqpoint{5.660681in}{0.739656in}}%
\pgfpathlineto{\pgfqpoint{5.660117in}{0.739656in}}%
\pgfpathlineto{\pgfqpoint{5.659553in}{0.739656in}}%
\pgfpathlineto{\pgfqpoint{5.658988in}{0.739656in}}%
\pgfpathlineto{\pgfqpoint{5.658424in}{0.739656in}}%
\pgfpathlineto{\pgfqpoint{5.657860in}{0.739656in}}%
\pgfpathlineto{\pgfqpoint{5.657295in}{0.739656in}}%
\pgfpathlineto{\pgfqpoint{5.656731in}{0.739656in}}%
\pgfpathlineto{\pgfqpoint{5.656167in}{0.739656in}}%
\pgfpathlineto{\pgfqpoint{5.655602in}{0.739656in}}%
\pgfpathlineto{\pgfqpoint{5.655038in}{0.739656in}}%
\pgfpathlineto{\pgfqpoint{5.654474in}{0.739656in}}%
\pgfpathlineto{\pgfqpoint{5.653909in}{0.739656in}}%
\pgfpathlineto{\pgfqpoint{5.653345in}{0.739656in}}%
\pgfpathlineto{\pgfqpoint{5.652781in}{0.739656in}}%
\pgfpathlineto{\pgfqpoint{5.652216in}{0.739656in}}%
\pgfpathlineto{\pgfqpoint{5.651652in}{0.739656in}}%
\pgfpathlineto{\pgfqpoint{5.651088in}{0.739656in}}%
\pgfpathlineto{\pgfqpoint{5.650523in}{0.739656in}}%
\pgfpathlineto{\pgfqpoint{5.649959in}{0.739656in}}%
\pgfpathlineto{\pgfqpoint{5.649395in}{0.739656in}}%
\pgfpathlineto{\pgfqpoint{5.648830in}{0.739656in}}%
\pgfpathlineto{\pgfqpoint{5.648266in}{0.739656in}}%
\pgfpathlineto{\pgfqpoint{5.647701in}{0.739656in}}%
\pgfpathlineto{\pgfqpoint{5.647137in}{0.739656in}}%
\pgfpathlineto{\pgfqpoint{5.646573in}{0.739656in}}%
\pgfpathlineto{\pgfqpoint{5.646008in}{0.739656in}}%
\pgfpathlineto{\pgfqpoint{5.645444in}{0.739656in}}%
\pgfpathlineto{\pgfqpoint{5.644880in}{0.739656in}}%
\pgfpathlineto{\pgfqpoint{5.644315in}{0.739656in}}%
\pgfpathlineto{\pgfqpoint{5.643751in}{0.739656in}}%
\pgfpathlineto{\pgfqpoint{5.643187in}{0.739656in}}%
\pgfpathlineto{\pgfqpoint{5.642622in}{0.739656in}}%
\pgfpathlineto{\pgfqpoint{5.642058in}{0.739656in}}%
\pgfpathlineto{\pgfqpoint{5.641494in}{0.739656in}}%
\pgfpathlineto{\pgfqpoint{5.640929in}{0.739656in}}%
\pgfpathlineto{\pgfqpoint{5.640365in}{0.739656in}}%
\pgfpathlineto{\pgfqpoint{5.639801in}{0.739656in}}%
\pgfpathlineto{\pgfqpoint{5.639236in}{0.739656in}}%
\pgfpathlineto{\pgfqpoint{5.638672in}{0.739656in}}%
\pgfpathlineto{\pgfqpoint{5.638108in}{0.739656in}}%
\pgfpathlineto{\pgfqpoint{5.637543in}{0.739656in}}%
\pgfpathlineto{\pgfqpoint{5.636979in}{0.739656in}}%
\pgfpathlineto{\pgfqpoint{5.636415in}{0.739656in}}%
\pgfpathlineto{\pgfqpoint{5.635850in}{0.739656in}}%
\pgfpathlineto{\pgfqpoint{5.635286in}{0.739656in}}%
\pgfpathlineto{\pgfqpoint{5.634722in}{0.739656in}}%
\pgfpathlineto{\pgfqpoint{5.634157in}{0.739656in}}%
\pgfpathlineto{\pgfqpoint{5.633593in}{0.739656in}}%
\pgfpathlineto{\pgfqpoint{5.633028in}{0.739656in}}%
\pgfpathlineto{\pgfqpoint{5.632464in}{0.739656in}}%
\pgfpathlineto{\pgfqpoint{5.631900in}{0.739656in}}%
\pgfpathlineto{\pgfqpoint{5.631335in}{0.739656in}}%
\pgfpathlineto{\pgfqpoint{5.630771in}{0.739656in}}%
\pgfpathlineto{\pgfqpoint{5.630207in}{0.739656in}}%
\pgfpathlineto{\pgfqpoint{5.629642in}{0.739656in}}%
\pgfpathlineto{\pgfqpoint{5.629078in}{0.739656in}}%
\pgfpathlineto{\pgfqpoint{5.628514in}{0.739656in}}%
\pgfpathlineto{\pgfqpoint{5.627949in}{0.739656in}}%
\pgfpathlineto{\pgfqpoint{5.627385in}{0.739656in}}%
\pgfpathlineto{\pgfqpoint{5.626821in}{0.739656in}}%
\pgfpathlineto{\pgfqpoint{5.626256in}{0.739656in}}%
\pgfpathlineto{\pgfqpoint{5.625692in}{0.739656in}}%
\pgfpathlineto{\pgfqpoint{5.625128in}{0.739656in}}%
\pgfpathlineto{\pgfqpoint{5.624563in}{0.739656in}}%
\pgfpathlineto{\pgfqpoint{5.623999in}{0.739656in}}%
\pgfpathlineto{\pgfqpoint{5.623435in}{0.739656in}}%
\pgfpathlineto{\pgfqpoint{5.622870in}{0.739656in}}%
\pgfpathlineto{\pgfqpoint{5.622306in}{0.739656in}}%
\pgfpathlineto{\pgfqpoint{5.621742in}{0.739656in}}%
\pgfpathlineto{\pgfqpoint{5.621177in}{0.739656in}}%
\pgfpathlineto{\pgfqpoint{5.620613in}{0.739656in}}%
\pgfpathlineto{\pgfqpoint{5.620049in}{0.739656in}}%
\pgfpathlineto{\pgfqpoint{5.619484in}{0.739656in}}%
\pgfpathlineto{\pgfqpoint{5.618920in}{0.739656in}}%
\pgfpathlineto{\pgfqpoint{5.618355in}{0.739656in}}%
\pgfpathlineto{\pgfqpoint{5.617791in}{0.739656in}}%
\pgfpathlineto{\pgfqpoint{5.617227in}{0.739656in}}%
\pgfpathlineto{\pgfqpoint{5.616662in}{0.739656in}}%
\pgfpathlineto{\pgfqpoint{5.616098in}{0.739656in}}%
\pgfpathlineto{\pgfqpoint{5.615534in}{0.739656in}}%
\pgfpathlineto{\pgfqpoint{5.614969in}{0.739656in}}%
\pgfpathlineto{\pgfqpoint{5.614405in}{0.739656in}}%
\pgfpathlineto{\pgfqpoint{5.613841in}{0.739656in}}%
\pgfpathlineto{\pgfqpoint{5.613276in}{0.739656in}}%
\pgfpathlineto{\pgfqpoint{5.612712in}{0.739656in}}%
\pgfpathlineto{\pgfqpoint{5.612148in}{0.739656in}}%
\pgfpathlineto{\pgfqpoint{5.611583in}{0.739656in}}%
\pgfpathlineto{\pgfqpoint{5.611019in}{0.739656in}}%
\pgfpathlineto{\pgfqpoint{5.610455in}{0.739656in}}%
\pgfpathlineto{\pgfqpoint{5.609890in}{0.739656in}}%
\pgfpathlineto{\pgfqpoint{5.609326in}{0.739656in}}%
\pgfpathlineto{\pgfqpoint{5.608762in}{0.739656in}}%
\pgfpathlineto{\pgfqpoint{5.608197in}{0.739656in}}%
\pgfpathlineto{\pgfqpoint{5.607633in}{0.739656in}}%
\pgfpathlineto{\pgfqpoint{5.607069in}{0.739656in}}%
\pgfpathlineto{\pgfqpoint{5.606504in}{0.739656in}}%
\pgfpathlineto{\pgfqpoint{5.605940in}{0.739656in}}%
\pgfpathlineto{\pgfqpoint{5.605376in}{0.739656in}}%
\pgfpathlineto{\pgfqpoint{5.604811in}{0.739656in}}%
\pgfpathlineto{\pgfqpoint{5.604247in}{0.739656in}}%
\pgfpathlineto{\pgfqpoint{5.603683in}{0.739656in}}%
\pgfpathlineto{\pgfqpoint{5.603118in}{0.739656in}}%
\pgfpathlineto{\pgfqpoint{5.602554in}{0.739656in}}%
\pgfpathlineto{\pgfqpoint{5.601989in}{0.739656in}}%
\pgfpathlineto{\pgfqpoint{5.601425in}{0.739656in}}%
\pgfpathlineto{\pgfqpoint{5.600861in}{0.739656in}}%
\pgfpathlineto{\pgfqpoint{5.600296in}{0.739656in}}%
\pgfpathlineto{\pgfqpoint{5.599732in}{0.739656in}}%
\pgfpathlineto{\pgfqpoint{5.599168in}{0.739656in}}%
\pgfpathlineto{\pgfqpoint{5.598603in}{0.739656in}}%
\pgfpathlineto{\pgfqpoint{5.598039in}{0.739656in}}%
\pgfpathlineto{\pgfqpoint{5.597475in}{0.739656in}}%
\pgfpathlineto{\pgfqpoint{5.596910in}{0.739656in}}%
\pgfpathlineto{\pgfqpoint{5.596346in}{0.739656in}}%
\pgfpathlineto{\pgfqpoint{5.595782in}{0.739656in}}%
\pgfpathlineto{\pgfqpoint{5.595217in}{0.739656in}}%
\pgfpathlineto{\pgfqpoint{5.594653in}{0.739656in}}%
\pgfpathlineto{\pgfqpoint{5.594089in}{0.739656in}}%
\pgfpathlineto{\pgfqpoint{5.593524in}{0.739656in}}%
\pgfpathlineto{\pgfqpoint{5.592960in}{0.739656in}}%
\pgfpathlineto{\pgfqpoint{5.592396in}{0.739656in}}%
\pgfpathlineto{\pgfqpoint{5.591831in}{0.739656in}}%
\pgfpathlineto{\pgfqpoint{5.591267in}{0.739656in}}%
\pgfpathlineto{\pgfqpoint{5.590703in}{0.739656in}}%
\pgfpathlineto{\pgfqpoint{5.590138in}{0.739656in}}%
\pgfpathlineto{\pgfqpoint{5.589574in}{0.739656in}}%
\pgfpathlineto{\pgfqpoint{5.589010in}{0.739656in}}%
\pgfpathlineto{\pgfqpoint{5.588445in}{0.739656in}}%
\pgfpathlineto{\pgfqpoint{5.587881in}{0.739656in}}%
\pgfpathlineto{\pgfqpoint{5.587316in}{0.739656in}}%
\pgfpathlineto{\pgfqpoint{5.586752in}{0.739656in}}%
\pgfpathlineto{\pgfqpoint{5.586188in}{0.739656in}}%
\pgfpathlineto{\pgfqpoint{5.585623in}{0.739656in}}%
\pgfpathlineto{\pgfqpoint{5.585059in}{0.739656in}}%
\pgfpathlineto{\pgfqpoint{5.584495in}{0.739656in}}%
\pgfpathlineto{\pgfqpoint{5.583930in}{0.739656in}}%
\pgfpathlineto{\pgfqpoint{5.583366in}{0.739656in}}%
\pgfpathlineto{\pgfqpoint{5.582802in}{0.739656in}}%
\pgfpathlineto{\pgfqpoint{5.582237in}{0.739656in}}%
\pgfpathlineto{\pgfqpoint{5.581673in}{0.739656in}}%
\pgfpathlineto{\pgfqpoint{5.581109in}{0.739656in}}%
\pgfpathlineto{\pgfqpoint{5.580544in}{0.739656in}}%
\pgfpathlineto{\pgfqpoint{5.579980in}{0.739656in}}%
\pgfpathlineto{\pgfqpoint{5.579416in}{0.739656in}}%
\pgfpathlineto{\pgfqpoint{5.578851in}{0.739656in}}%
\pgfpathlineto{\pgfqpoint{5.578287in}{0.739656in}}%
\pgfpathlineto{\pgfqpoint{5.577723in}{0.739656in}}%
\pgfpathlineto{\pgfqpoint{5.577158in}{0.739656in}}%
\pgfpathlineto{\pgfqpoint{5.576594in}{0.739656in}}%
\pgfpathlineto{\pgfqpoint{5.576030in}{0.739656in}}%
\pgfpathlineto{\pgfqpoint{5.575465in}{0.739656in}}%
\pgfpathlineto{\pgfqpoint{5.574901in}{0.739656in}}%
\pgfpathlineto{\pgfqpoint{5.574337in}{0.739656in}}%
\pgfpathlineto{\pgfqpoint{5.573772in}{0.739656in}}%
\pgfpathlineto{\pgfqpoint{5.573208in}{0.739656in}}%
\pgfpathlineto{\pgfqpoint{5.572643in}{0.739656in}}%
\pgfpathlineto{\pgfqpoint{5.572079in}{0.739656in}}%
\pgfpathlineto{\pgfqpoint{5.571515in}{0.739656in}}%
\pgfpathlineto{\pgfqpoint{5.570950in}{0.739656in}}%
\pgfpathlineto{\pgfqpoint{5.570386in}{0.739656in}}%
\pgfpathlineto{\pgfqpoint{5.569822in}{0.739656in}}%
\pgfpathlineto{\pgfqpoint{5.569257in}{0.739656in}}%
\pgfpathlineto{\pgfqpoint{5.568693in}{0.739656in}}%
\pgfpathlineto{\pgfqpoint{5.568129in}{0.739656in}}%
\pgfpathlineto{\pgfqpoint{5.567564in}{0.739656in}}%
\pgfpathlineto{\pgfqpoint{5.567000in}{0.739656in}}%
\pgfpathlineto{\pgfqpoint{5.566436in}{0.739656in}}%
\pgfpathlineto{\pgfqpoint{5.565871in}{0.739656in}}%
\pgfpathlineto{\pgfqpoint{5.565307in}{0.739656in}}%
\pgfpathlineto{\pgfqpoint{5.564743in}{0.739656in}}%
\pgfpathlineto{\pgfqpoint{5.564178in}{0.739656in}}%
\pgfpathlineto{\pgfqpoint{5.563614in}{0.739656in}}%
\pgfpathlineto{\pgfqpoint{5.563050in}{0.739656in}}%
\pgfpathlineto{\pgfqpoint{5.562485in}{0.739656in}}%
\pgfpathlineto{\pgfqpoint{5.561921in}{0.739656in}}%
\pgfpathlineto{\pgfqpoint{5.561357in}{0.739656in}}%
\pgfpathlineto{\pgfqpoint{5.560792in}{0.739656in}}%
\pgfpathlineto{\pgfqpoint{5.560228in}{0.739656in}}%
\pgfpathlineto{\pgfqpoint{5.559664in}{0.739656in}}%
\pgfpathlineto{\pgfqpoint{5.559099in}{0.739656in}}%
\pgfpathlineto{\pgfqpoint{5.558535in}{0.739656in}}%
\pgfpathlineto{\pgfqpoint{5.557970in}{0.739656in}}%
\pgfpathlineto{\pgfqpoint{5.557406in}{0.739656in}}%
\pgfpathlineto{\pgfqpoint{5.556842in}{0.739656in}}%
\pgfpathlineto{\pgfqpoint{5.556277in}{0.739656in}}%
\pgfpathlineto{\pgfqpoint{5.555713in}{0.739656in}}%
\pgfpathlineto{\pgfqpoint{5.555149in}{0.739656in}}%
\pgfpathlineto{\pgfqpoint{5.554584in}{0.739656in}}%
\pgfpathlineto{\pgfqpoint{5.554020in}{0.739656in}}%
\pgfpathlineto{\pgfqpoint{5.553456in}{0.739656in}}%
\pgfpathlineto{\pgfqpoint{5.552891in}{0.739656in}}%
\pgfpathlineto{\pgfqpoint{5.552327in}{0.739656in}}%
\pgfpathlineto{\pgfqpoint{5.551763in}{0.739656in}}%
\pgfpathlineto{\pgfqpoint{5.551198in}{0.739656in}}%
\pgfpathlineto{\pgfqpoint{5.550634in}{0.739656in}}%
\pgfpathlineto{\pgfqpoint{5.550070in}{0.739656in}}%
\pgfpathlineto{\pgfqpoint{5.549505in}{0.739656in}}%
\pgfpathlineto{\pgfqpoint{5.548941in}{0.739656in}}%
\pgfpathlineto{\pgfqpoint{5.548377in}{0.739656in}}%
\pgfpathlineto{\pgfqpoint{5.547812in}{0.739656in}}%
\pgfpathlineto{\pgfqpoint{5.547248in}{0.739656in}}%
\pgfpathlineto{\pgfqpoint{5.546684in}{0.739656in}}%
\pgfpathlineto{\pgfqpoint{5.546119in}{0.739656in}}%
\pgfpathlineto{\pgfqpoint{5.545555in}{0.739656in}}%
\pgfpathlineto{\pgfqpoint{5.544991in}{0.739656in}}%
\pgfpathlineto{\pgfqpoint{5.544426in}{0.739656in}}%
\pgfpathlineto{\pgfqpoint{5.543862in}{0.739656in}}%
\pgfpathlineto{\pgfqpoint{5.543298in}{0.739656in}}%
\pgfpathlineto{\pgfqpoint{5.542733in}{0.739656in}}%
\pgfpathlineto{\pgfqpoint{5.542169in}{0.739656in}}%
\pgfpathlineto{\pgfqpoint{5.541604in}{0.739656in}}%
\pgfpathlineto{\pgfqpoint{5.541040in}{0.739656in}}%
\pgfpathlineto{\pgfqpoint{5.540476in}{0.739656in}}%
\pgfpathlineto{\pgfqpoint{5.539911in}{0.739656in}}%
\pgfpathlineto{\pgfqpoint{5.539347in}{0.739656in}}%
\pgfpathlineto{\pgfqpoint{5.538783in}{0.739656in}}%
\pgfpathlineto{\pgfqpoint{5.538218in}{0.739656in}}%
\pgfpathlineto{\pgfqpoint{5.537654in}{0.739656in}}%
\pgfpathlineto{\pgfqpoint{5.537090in}{0.739656in}}%
\pgfpathlineto{\pgfqpoint{5.536525in}{0.739656in}}%
\pgfpathlineto{\pgfqpoint{5.535961in}{0.739656in}}%
\pgfpathlineto{\pgfqpoint{5.535397in}{0.739656in}}%
\pgfpathlineto{\pgfqpoint{5.534832in}{0.739656in}}%
\pgfpathlineto{\pgfqpoint{5.534268in}{0.739656in}}%
\pgfpathlineto{\pgfqpoint{5.533704in}{0.739656in}}%
\pgfpathlineto{\pgfqpoint{5.533139in}{0.739656in}}%
\pgfpathlineto{\pgfqpoint{5.532575in}{0.739656in}}%
\pgfpathlineto{\pgfqpoint{5.532011in}{0.739656in}}%
\pgfpathlineto{\pgfqpoint{5.531446in}{0.739656in}}%
\pgfpathlineto{\pgfqpoint{5.530882in}{0.739656in}}%
\pgfpathlineto{\pgfqpoint{5.530318in}{0.739656in}}%
\pgfpathlineto{\pgfqpoint{5.529753in}{0.739656in}}%
\pgfpathlineto{\pgfqpoint{5.529189in}{0.739656in}}%
\pgfpathlineto{\pgfqpoint{5.528625in}{0.739656in}}%
\pgfpathlineto{\pgfqpoint{5.528060in}{0.739656in}}%
\pgfpathlineto{\pgfqpoint{5.527496in}{0.739656in}}%
\pgfpathlineto{\pgfqpoint{5.526931in}{0.739656in}}%
\pgfpathlineto{\pgfqpoint{5.526367in}{0.739656in}}%
\pgfpathlineto{\pgfqpoint{5.525803in}{0.739656in}}%
\pgfpathlineto{\pgfqpoint{5.525238in}{0.739656in}}%
\pgfpathlineto{\pgfqpoint{5.524674in}{0.739656in}}%
\pgfpathlineto{\pgfqpoint{5.524110in}{0.739656in}}%
\pgfpathlineto{\pgfqpoint{5.523545in}{0.739656in}}%
\pgfpathlineto{\pgfqpoint{5.522981in}{0.739656in}}%
\pgfpathlineto{\pgfqpoint{5.522417in}{0.739656in}}%
\pgfpathlineto{\pgfqpoint{5.521852in}{0.739656in}}%
\pgfpathlineto{\pgfqpoint{5.521288in}{0.739656in}}%
\pgfpathlineto{\pgfqpoint{5.520724in}{0.739656in}}%
\pgfpathlineto{\pgfqpoint{5.520159in}{0.739656in}}%
\pgfpathlineto{\pgfqpoint{5.519595in}{0.739656in}}%
\pgfpathlineto{\pgfqpoint{5.519031in}{0.739656in}}%
\pgfpathlineto{\pgfqpoint{5.518466in}{0.739656in}}%
\pgfpathlineto{\pgfqpoint{5.517902in}{0.739656in}}%
\pgfpathlineto{\pgfqpoint{5.517338in}{0.739656in}}%
\pgfpathlineto{\pgfqpoint{5.516773in}{0.739656in}}%
\pgfpathlineto{\pgfqpoint{5.516209in}{0.739656in}}%
\pgfpathlineto{\pgfqpoint{5.515645in}{0.739656in}}%
\pgfpathlineto{\pgfqpoint{5.515080in}{0.739656in}}%
\pgfpathlineto{\pgfqpoint{5.514516in}{0.739656in}}%
\pgfpathlineto{\pgfqpoint{5.513952in}{0.739656in}}%
\pgfpathlineto{\pgfqpoint{5.513387in}{0.739656in}}%
\pgfpathlineto{\pgfqpoint{5.512823in}{0.739656in}}%
\pgfpathlineto{\pgfqpoint{5.512258in}{0.739656in}}%
\pgfpathlineto{\pgfqpoint{5.511694in}{0.739656in}}%
\pgfpathlineto{\pgfqpoint{5.511130in}{0.739656in}}%
\pgfpathlineto{\pgfqpoint{5.510565in}{0.739656in}}%
\pgfpathlineto{\pgfqpoint{5.510001in}{0.739656in}}%
\pgfpathlineto{\pgfqpoint{5.509437in}{0.739656in}}%
\pgfpathlineto{\pgfqpoint{5.508872in}{0.739656in}}%
\pgfpathlineto{\pgfqpoint{5.508308in}{0.739656in}}%
\pgfpathlineto{\pgfqpoint{5.507744in}{0.739656in}}%
\pgfpathlineto{\pgfqpoint{5.507179in}{0.739656in}}%
\pgfpathlineto{\pgfqpoint{5.506615in}{0.739656in}}%
\pgfpathlineto{\pgfqpoint{5.506051in}{0.739656in}}%
\pgfpathlineto{\pgfqpoint{5.505486in}{0.739656in}}%
\pgfpathlineto{\pgfqpoint{5.504922in}{0.739656in}}%
\pgfpathlineto{\pgfqpoint{5.504358in}{0.739656in}}%
\pgfpathlineto{\pgfqpoint{5.503793in}{0.739656in}}%
\pgfpathlineto{\pgfqpoint{5.503229in}{0.739656in}}%
\pgfpathlineto{\pgfqpoint{5.502665in}{0.739656in}}%
\pgfpathlineto{\pgfqpoint{5.502100in}{0.739656in}}%
\pgfpathlineto{\pgfqpoint{5.501536in}{0.739656in}}%
\pgfpathlineto{\pgfqpoint{5.500972in}{0.739656in}}%
\pgfpathlineto{\pgfqpoint{5.500407in}{0.739656in}}%
\pgfpathlineto{\pgfqpoint{5.499843in}{0.739656in}}%
\pgfpathlineto{\pgfqpoint{5.499279in}{0.739656in}}%
\pgfpathlineto{\pgfqpoint{5.498714in}{0.739656in}}%
\pgfpathlineto{\pgfqpoint{5.498150in}{0.739656in}}%
\pgfpathlineto{\pgfqpoint{5.497586in}{0.739656in}}%
\pgfpathlineto{\pgfqpoint{5.497021in}{0.739656in}}%
\pgfpathlineto{\pgfqpoint{5.496457in}{0.739656in}}%
\pgfpathlineto{\pgfqpoint{5.495892in}{0.739656in}}%
\pgfpathlineto{\pgfqpoint{5.495328in}{0.739656in}}%
\pgfpathlineto{\pgfqpoint{5.494764in}{0.739656in}}%
\pgfpathlineto{\pgfqpoint{5.494199in}{0.739656in}}%
\pgfpathlineto{\pgfqpoint{5.493635in}{0.739656in}}%
\pgfpathlineto{\pgfqpoint{5.493071in}{0.739656in}}%
\pgfpathlineto{\pgfqpoint{5.492506in}{0.739656in}}%
\pgfpathlineto{\pgfqpoint{5.491942in}{0.739656in}}%
\pgfpathlineto{\pgfqpoint{5.491378in}{0.739656in}}%
\pgfpathlineto{\pgfqpoint{5.490813in}{0.739656in}}%
\pgfpathlineto{\pgfqpoint{5.490249in}{0.739656in}}%
\pgfpathlineto{\pgfqpoint{5.489685in}{0.739656in}}%
\pgfpathlineto{\pgfqpoint{5.489120in}{0.739656in}}%
\pgfpathlineto{\pgfqpoint{5.488556in}{0.739656in}}%
\pgfpathlineto{\pgfqpoint{5.487992in}{0.739656in}}%
\pgfpathlineto{\pgfqpoint{5.487427in}{0.739656in}}%
\pgfpathlineto{\pgfqpoint{5.486863in}{0.739656in}}%
\pgfpathlineto{\pgfqpoint{5.486299in}{0.739656in}}%
\pgfpathlineto{\pgfqpoint{5.485734in}{0.739656in}}%
\pgfpathlineto{\pgfqpoint{5.485170in}{0.739656in}}%
\pgfpathlineto{\pgfqpoint{5.484606in}{0.739656in}}%
\pgfpathlineto{\pgfqpoint{5.484041in}{0.739656in}}%
\pgfpathlineto{\pgfqpoint{5.483477in}{0.739656in}}%
\pgfpathlineto{\pgfqpoint{5.482913in}{0.739656in}}%
\pgfpathlineto{\pgfqpoint{5.482348in}{0.739656in}}%
\pgfpathlineto{\pgfqpoint{5.481784in}{0.739656in}}%
\pgfpathlineto{\pgfqpoint{5.481219in}{0.739656in}}%
\pgfpathlineto{\pgfqpoint{5.480655in}{0.739656in}}%
\pgfpathlineto{\pgfqpoint{5.480091in}{0.739656in}}%
\pgfpathlineto{\pgfqpoint{5.479526in}{0.739656in}}%
\pgfpathlineto{\pgfqpoint{5.478962in}{0.739656in}}%
\pgfpathlineto{\pgfqpoint{5.478398in}{0.739656in}}%
\pgfpathlineto{\pgfqpoint{5.477833in}{0.739656in}}%
\pgfpathlineto{\pgfqpoint{5.477269in}{0.739656in}}%
\pgfpathlineto{\pgfqpoint{5.476705in}{0.739656in}}%
\pgfpathlineto{\pgfqpoint{5.476140in}{0.739656in}}%
\pgfpathlineto{\pgfqpoint{5.475576in}{0.739656in}}%
\pgfpathlineto{\pgfqpoint{5.475012in}{0.739656in}}%
\pgfpathlineto{\pgfqpoint{5.474447in}{0.739656in}}%
\pgfpathlineto{\pgfqpoint{5.473883in}{0.739656in}}%
\pgfpathlineto{\pgfqpoint{5.473319in}{0.739656in}}%
\pgfpathlineto{\pgfqpoint{5.472754in}{0.739656in}}%
\pgfpathlineto{\pgfqpoint{5.472190in}{0.739656in}}%
\pgfpathlineto{\pgfqpoint{5.471626in}{0.739656in}}%
\pgfpathlineto{\pgfqpoint{5.471061in}{0.739656in}}%
\pgfpathlineto{\pgfqpoint{5.470497in}{0.739656in}}%
\pgfpathlineto{\pgfqpoint{5.469933in}{0.739656in}}%
\pgfpathlineto{\pgfqpoint{5.469368in}{0.739656in}}%
\pgfpathlineto{\pgfqpoint{5.468804in}{0.739656in}}%
\pgfpathlineto{\pgfqpoint{5.468240in}{0.739656in}}%
\pgfpathlineto{\pgfqpoint{5.467675in}{0.739656in}}%
\pgfpathlineto{\pgfqpoint{5.467111in}{0.739656in}}%
\pgfpathlineto{\pgfqpoint{5.466546in}{0.739656in}}%
\pgfpathlineto{\pgfqpoint{5.465982in}{0.739656in}}%
\pgfpathlineto{\pgfqpoint{5.465418in}{0.739656in}}%
\pgfpathlineto{\pgfqpoint{5.464853in}{0.739656in}}%
\pgfpathlineto{\pgfqpoint{5.464289in}{0.739656in}}%
\pgfpathlineto{\pgfqpoint{5.463725in}{0.739656in}}%
\pgfpathlineto{\pgfqpoint{5.463160in}{0.739656in}}%
\pgfpathlineto{\pgfqpoint{5.462596in}{0.739656in}}%
\pgfpathlineto{\pgfqpoint{5.462032in}{0.739656in}}%
\pgfpathlineto{\pgfqpoint{5.461467in}{0.739656in}}%
\pgfpathlineto{\pgfqpoint{5.460903in}{0.739656in}}%
\pgfpathlineto{\pgfqpoint{5.460339in}{0.739656in}}%
\pgfpathlineto{\pgfqpoint{5.459774in}{0.739656in}}%
\pgfpathlineto{\pgfqpoint{5.459210in}{0.739656in}}%
\pgfpathlineto{\pgfqpoint{5.458646in}{0.739656in}}%
\pgfpathlineto{\pgfqpoint{5.458081in}{0.739656in}}%
\pgfpathlineto{\pgfqpoint{5.457517in}{0.739656in}}%
\pgfpathlineto{\pgfqpoint{5.456953in}{0.739656in}}%
\pgfpathlineto{\pgfqpoint{5.456388in}{0.739656in}}%
\pgfpathlineto{\pgfqpoint{5.455824in}{0.739656in}}%
\pgfpathlineto{\pgfqpoint{5.455260in}{0.739656in}}%
\pgfpathlineto{\pgfqpoint{5.454695in}{0.739656in}}%
\pgfpathlineto{\pgfqpoint{5.454131in}{0.739656in}}%
\pgfpathlineto{\pgfqpoint{5.453567in}{0.739656in}}%
\pgfpathlineto{\pgfqpoint{5.453002in}{0.739656in}}%
\pgfpathlineto{\pgfqpoint{5.452438in}{0.739656in}}%
\pgfpathlineto{\pgfqpoint{5.451874in}{0.739656in}}%
\pgfpathlineto{\pgfqpoint{5.451309in}{0.739656in}}%
\pgfpathlineto{\pgfqpoint{5.450745in}{0.739656in}}%
\pgfpathlineto{\pgfqpoint{5.450180in}{0.739656in}}%
\pgfpathlineto{\pgfqpoint{5.449616in}{0.739656in}}%
\pgfpathlineto{\pgfqpoint{5.449052in}{0.739656in}}%
\pgfpathlineto{\pgfqpoint{5.448487in}{0.739656in}}%
\pgfpathlineto{\pgfqpoint{5.447923in}{0.739656in}}%
\pgfpathlineto{\pgfqpoint{5.447359in}{0.739656in}}%
\pgfpathlineto{\pgfqpoint{5.446794in}{0.739656in}}%
\pgfpathlineto{\pgfqpoint{5.446230in}{0.739656in}}%
\pgfpathlineto{\pgfqpoint{5.445666in}{0.739656in}}%
\pgfpathlineto{\pgfqpoint{5.445101in}{0.739656in}}%
\pgfpathlineto{\pgfqpoint{5.444537in}{0.739656in}}%
\pgfpathlineto{\pgfqpoint{5.443973in}{0.739656in}}%
\pgfpathlineto{\pgfqpoint{5.443408in}{0.739656in}}%
\pgfpathlineto{\pgfqpoint{5.442844in}{0.739656in}}%
\pgfpathlineto{\pgfqpoint{5.442280in}{0.739656in}}%
\pgfpathlineto{\pgfqpoint{5.441715in}{0.739656in}}%
\pgfpathlineto{\pgfqpoint{5.441151in}{0.739656in}}%
\pgfpathlineto{\pgfqpoint{5.440587in}{0.739656in}}%
\pgfpathlineto{\pgfqpoint{5.440022in}{0.739656in}}%
\pgfpathlineto{\pgfqpoint{5.439458in}{0.739656in}}%
\pgfpathlineto{\pgfqpoint{5.438894in}{0.739656in}}%
\pgfpathlineto{\pgfqpoint{5.438329in}{0.739656in}}%
\pgfpathlineto{\pgfqpoint{5.437765in}{0.739656in}}%
\pgfpathlineto{\pgfqpoint{5.437201in}{0.739656in}}%
\pgfpathlineto{\pgfqpoint{5.436636in}{0.739656in}}%
\pgfpathlineto{\pgfqpoint{5.436072in}{0.739656in}}%
\pgfpathlineto{\pgfqpoint{5.435507in}{0.739656in}}%
\pgfpathlineto{\pgfqpoint{5.434943in}{0.739656in}}%
\pgfpathlineto{\pgfqpoint{5.434379in}{0.739656in}}%
\pgfpathlineto{\pgfqpoint{5.433814in}{0.739656in}}%
\pgfpathlineto{\pgfqpoint{5.433250in}{0.739656in}}%
\pgfpathlineto{\pgfqpoint{5.432686in}{0.739656in}}%
\pgfpathlineto{\pgfqpoint{5.432121in}{0.739656in}}%
\pgfpathlineto{\pgfqpoint{5.431557in}{0.739656in}}%
\pgfpathlineto{\pgfqpoint{5.430993in}{0.739656in}}%
\pgfpathlineto{\pgfqpoint{5.430428in}{0.739656in}}%
\pgfpathlineto{\pgfqpoint{5.429864in}{0.739656in}}%
\pgfpathlineto{\pgfqpoint{5.429300in}{0.739656in}}%
\pgfpathlineto{\pgfqpoint{5.428735in}{0.739656in}}%
\pgfpathlineto{\pgfqpoint{5.428171in}{0.739656in}}%
\pgfpathlineto{\pgfqpoint{5.427607in}{0.739656in}}%
\pgfpathlineto{\pgfqpoint{5.427042in}{0.739656in}}%
\pgfpathlineto{\pgfqpoint{5.426478in}{0.739656in}}%
\pgfpathlineto{\pgfqpoint{5.425914in}{0.739656in}}%
\pgfpathlineto{\pgfqpoint{5.425349in}{0.739656in}}%
\pgfpathlineto{\pgfqpoint{5.424785in}{0.739656in}}%
\pgfpathlineto{\pgfqpoint{5.424221in}{0.739656in}}%
\pgfpathlineto{\pgfqpoint{5.423656in}{0.739656in}}%
\pgfpathlineto{\pgfqpoint{5.423092in}{0.739656in}}%
\pgfpathlineto{\pgfqpoint{5.422528in}{0.739656in}}%
\pgfpathlineto{\pgfqpoint{5.421963in}{0.739656in}}%
\pgfpathlineto{\pgfqpoint{5.421399in}{0.739656in}}%
\pgfpathlineto{\pgfqpoint{5.420834in}{0.739656in}}%
\pgfpathlineto{\pgfqpoint{5.420270in}{0.739656in}}%
\pgfpathlineto{\pgfqpoint{5.419706in}{0.739656in}}%
\pgfpathlineto{\pgfqpoint{5.419141in}{0.739656in}}%
\pgfpathlineto{\pgfqpoint{5.418577in}{0.739656in}}%
\pgfpathlineto{\pgfqpoint{5.418013in}{0.739656in}}%
\pgfpathlineto{\pgfqpoint{5.417448in}{0.739656in}}%
\pgfpathlineto{\pgfqpoint{5.416884in}{0.739656in}}%
\pgfpathlineto{\pgfqpoint{5.416320in}{0.739656in}}%
\pgfpathlineto{\pgfqpoint{5.415755in}{0.739656in}}%
\pgfpathlineto{\pgfqpoint{5.415191in}{0.739656in}}%
\pgfpathlineto{\pgfqpoint{5.414627in}{0.739656in}}%
\pgfpathlineto{\pgfqpoint{5.414062in}{0.739656in}}%
\pgfpathlineto{\pgfqpoint{5.413498in}{0.739656in}}%
\pgfpathlineto{\pgfqpoint{5.412934in}{0.739656in}}%
\pgfpathlineto{\pgfqpoint{5.412369in}{0.739656in}}%
\pgfpathlineto{\pgfqpoint{5.411805in}{0.739656in}}%
\pgfpathlineto{\pgfqpoint{5.411241in}{0.739656in}}%
\pgfpathlineto{\pgfqpoint{5.410676in}{0.739656in}}%
\pgfpathlineto{\pgfqpoint{5.410112in}{0.739656in}}%
\pgfpathlineto{\pgfqpoint{5.409548in}{0.739656in}}%
\pgfpathlineto{\pgfqpoint{5.408983in}{0.739656in}}%
\pgfpathlineto{\pgfqpoint{5.408419in}{0.739656in}}%
\pgfpathlineto{\pgfqpoint{5.407855in}{0.739656in}}%
\pgfpathlineto{\pgfqpoint{5.407290in}{0.739656in}}%
\pgfpathlineto{\pgfqpoint{5.406726in}{0.739656in}}%
\pgfpathlineto{\pgfqpoint{5.406162in}{0.739656in}}%
\pgfpathlineto{\pgfqpoint{5.405597in}{0.739656in}}%
\pgfpathlineto{\pgfqpoint{5.405033in}{0.739656in}}%
\pgfpathlineto{\pgfqpoint{5.404468in}{0.739656in}}%
\pgfpathlineto{\pgfqpoint{5.403904in}{0.739656in}}%
\pgfpathlineto{\pgfqpoint{5.403340in}{0.739656in}}%
\pgfpathlineto{\pgfqpoint{5.402775in}{0.739656in}}%
\pgfpathlineto{\pgfqpoint{5.402211in}{0.739656in}}%
\pgfpathlineto{\pgfqpoint{5.401647in}{0.739656in}}%
\pgfpathlineto{\pgfqpoint{5.401082in}{0.739656in}}%
\pgfpathlineto{\pgfqpoint{5.400518in}{0.739656in}}%
\pgfpathlineto{\pgfqpoint{5.399954in}{0.739656in}}%
\pgfpathlineto{\pgfqpoint{5.399389in}{0.739656in}}%
\pgfpathlineto{\pgfqpoint{5.398825in}{0.739656in}}%
\pgfpathlineto{\pgfqpoint{5.398261in}{0.739656in}}%
\pgfpathlineto{\pgfqpoint{5.397696in}{0.739656in}}%
\pgfpathlineto{\pgfqpoint{5.397132in}{0.739656in}}%
\pgfpathlineto{\pgfqpoint{5.396568in}{0.739656in}}%
\pgfpathlineto{\pgfqpoint{5.396003in}{0.739656in}}%
\pgfpathlineto{\pgfqpoint{5.395439in}{0.739656in}}%
\pgfpathlineto{\pgfqpoint{5.394875in}{0.739656in}}%
\pgfpathlineto{\pgfqpoint{5.394310in}{0.739656in}}%
\pgfpathlineto{\pgfqpoint{5.393746in}{0.739656in}}%
\pgfpathlineto{\pgfqpoint{5.393182in}{0.739656in}}%
\pgfpathlineto{\pgfqpoint{5.392617in}{0.739656in}}%
\pgfpathlineto{\pgfqpoint{5.392053in}{0.739656in}}%
\pgfpathlineto{\pgfqpoint{5.391489in}{0.739656in}}%
\pgfpathlineto{\pgfqpoint{5.390924in}{0.739656in}}%
\pgfpathlineto{\pgfqpoint{5.390360in}{0.739656in}}%
\pgfpathlineto{\pgfqpoint{5.389795in}{0.739656in}}%
\pgfpathlineto{\pgfqpoint{5.389231in}{0.739656in}}%
\pgfpathlineto{\pgfqpoint{5.388667in}{0.739656in}}%
\pgfpathlineto{\pgfqpoint{5.388102in}{0.739656in}}%
\pgfpathlineto{\pgfqpoint{5.387538in}{0.739656in}}%
\pgfpathlineto{\pgfqpoint{5.386974in}{0.739656in}}%
\pgfpathlineto{\pgfqpoint{5.386409in}{0.739656in}}%
\pgfpathlineto{\pgfqpoint{5.385845in}{0.739656in}}%
\pgfpathlineto{\pgfqpoint{5.385281in}{0.739656in}}%
\pgfpathlineto{\pgfqpoint{5.384716in}{0.739656in}}%
\pgfpathlineto{\pgfqpoint{5.384152in}{0.739656in}}%
\pgfpathlineto{\pgfqpoint{5.383588in}{0.739656in}}%
\pgfpathlineto{\pgfqpoint{5.383023in}{0.739656in}}%
\pgfpathlineto{\pgfqpoint{5.382459in}{0.739656in}}%
\pgfpathlineto{\pgfqpoint{5.381895in}{0.739656in}}%
\pgfpathlineto{\pgfqpoint{5.381330in}{0.739656in}}%
\pgfpathlineto{\pgfqpoint{5.380766in}{0.739656in}}%
\pgfpathlineto{\pgfqpoint{5.380202in}{0.739656in}}%
\pgfpathlineto{\pgfqpoint{5.379637in}{0.739656in}}%
\pgfpathlineto{\pgfqpoint{5.379073in}{0.739656in}}%
\pgfpathlineto{\pgfqpoint{5.378509in}{0.739656in}}%
\pgfpathlineto{\pgfqpoint{5.377944in}{0.739656in}}%
\pgfpathlineto{\pgfqpoint{5.377380in}{0.739656in}}%
\pgfpathlineto{\pgfqpoint{5.376816in}{0.739656in}}%
\pgfpathlineto{\pgfqpoint{5.376251in}{0.739656in}}%
\pgfpathlineto{\pgfqpoint{5.375687in}{0.739656in}}%
\pgfpathlineto{\pgfqpoint{5.375122in}{0.739656in}}%
\pgfpathlineto{\pgfqpoint{5.374558in}{0.739656in}}%
\pgfpathlineto{\pgfqpoint{5.373994in}{0.739656in}}%
\pgfpathlineto{\pgfqpoint{5.373429in}{0.739656in}}%
\pgfpathlineto{\pgfqpoint{5.372865in}{0.739656in}}%
\pgfpathlineto{\pgfqpoint{5.372301in}{0.739656in}}%
\pgfpathlineto{\pgfqpoint{5.371736in}{0.739656in}}%
\pgfpathlineto{\pgfqpoint{5.371172in}{0.739656in}}%
\pgfpathlineto{\pgfqpoint{5.370608in}{0.739656in}}%
\pgfpathlineto{\pgfqpoint{5.370043in}{0.739656in}}%
\pgfpathlineto{\pgfqpoint{5.369479in}{0.739656in}}%
\pgfpathlineto{\pgfqpoint{5.368915in}{0.739656in}}%
\pgfpathlineto{\pgfqpoint{5.368350in}{0.739656in}}%
\pgfpathlineto{\pgfqpoint{5.367786in}{0.739656in}}%
\pgfpathlineto{\pgfqpoint{5.367222in}{0.739656in}}%
\pgfpathlineto{\pgfqpoint{5.366657in}{0.739656in}}%
\pgfpathlineto{\pgfqpoint{5.366093in}{0.739656in}}%
\pgfpathlineto{\pgfqpoint{5.365529in}{0.739656in}}%
\pgfpathlineto{\pgfqpoint{5.364964in}{0.739656in}}%
\pgfpathlineto{\pgfqpoint{5.364400in}{0.739656in}}%
\pgfpathlineto{\pgfqpoint{5.363836in}{0.739656in}}%
\pgfpathlineto{\pgfqpoint{5.363271in}{0.739656in}}%
\pgfpathlineto{\pgfqpoint{5.362707in}{0.739656in}}%
\pgfpathlineto{\pgfqpoint{5.362143in}{0.739656in}}%
\pgfpathlineto{\pgfqpoint{5.361578in}{0.739656in}}%
\pgfpathlineto{\pgfqpoint{5.361014in}{0.739656in}}%
\pgfpathlineto{\pgfqpoint{5.360449in}{0.739656in}}%
\pgfpathlineto{\pgfqpoint{5.359885in}{0.739656in}}%
\pgfpathlineto{\pgfqpoint{5.359321in}{0.739656in}}%
\pgfpathlineto{\pgfqpoint{5.358756in}{0.739656in}}%
\pgfpathlineto{\pgfqpoint{5.358192in}{0.739656in}}%
\pgfpathlineto{\pgfqpoint{5.357628in}{0.739656in}}%
\pgfpathlineto{\pgfqpoint{5.357063in}{0.739656in}}%
\pgfpathlineto{\pgfqpoint{5.356499in}{0.739656in}}%
\pgfpathlineto{\pgfqpoint{5.355935in}{0.739656in}}%
\pgfpathlineto{\pgfqpoint{5.355370in}{0.739656in}}%
\pgfpathlineto{\pgfqpoint{5.354806in}{0.739656in}}%
\pgfpathlineto{\pgfqpoint{5.354242in}{0.739656in}}%
\pgfpathlineto{\pgfqpoint{5.353677in}{0.739656in}}%
\pgfpathlineto{\pgfqpoint{5.353113in}{0.739656in}}%
\pgfpathlineto{\pgfqpoint{5.352549in}{0.739656in}}%
\pgfpathlineto{\pgfqpoint{5.351984in}{0.739656in}}%
\pgfpathlineto{\pgfqpoint{5.351420in}{0.739656in}}%
\pgfpathlineto{\pgfqpoint{5.350856in}{0.739656in}}%
\pgfpathlineto{\pgfqpoint{5.350291in}{0.739656in}}%
\pgfpathlineto{\pgfqpoint{5.349727in}{0.739656in}}%
\pgfpathlineto{\pgfqpoint{5.349163in}{0.739656in}}%
\pgfpathlineto{\pgfqpoint{5.348598in}{0.739656in}}%
\pgfpathlineto{\pgfqpoint{5.348034in}{0.739656in}}%
\pgfpathlineto{\pgfqpoint{5.347470in}{0.739656in}}%
\pgfpathlineto{\pgfqpoint{5.346905in}{0.739656in}}%
\pgfpathlineto{\pgfqpoint{5.346341in}{0.739656in}}%
\pgfpathlineto{\pgfqpoint{5.345777in}{0.739656in}}%
\pgfpathlineto{\pgfqpoint{5.345212in}{0.739656in}}%
\pgfpathlineto{\pgfqpoint{5.344648in}{0.739656in}}%
\pgfpathlineto{\pgfqpoint{5.344083in}{0.739656in}}%
\pgfpathlineto{\pgfqpoint{5.343519in}{0.739656in}}%
\pgfpathlineto{\pgfqpoint{5.342955in}{0.739656in}}%
\pgfpathlineto{\pgfqpoint{5.342390in}{0.739656in}}%
\pgfpathlineto{\pgfqpoint{5.341826in}{0.739656in}}%
\pgfpathlineto{\pgfqpoint{5.341262in}{0.739656in}}%
\pgfpathlineto{\pgfqpoint{5.340697in}{0.739656in}}%
\pgfpathlineto{\pgfqpoint{5.340133in}{0.739656in}}%
\pgfpathlineto{\pgfqpoint{5.339569in}{0.739656in}}%
\pgfpathlineto{\pgfqpoint{5.339004in}{0.739656in}}%
\pgfpathlineto{\pgfqpoint{5.338440in}{0.739656in}}%
\pgfpathlineto{\pgfqpoint{5.337876in}{0.739656in}}%
\pgfpathlineto{\pgfqpoint{5.337311in}{0.739656in}}%
\pgfpathlineto{\pgfqpoint{5.336747in}{0.739656in}}%
\pgfpathlineto{\pgfqpoint{5.336183in}{0.739656in}}%
\pgfpathlineto{\pgfqpoint{5.335618in}{0.739656in}}%
\pgfpathlineto{\pgfqpoint{5.335054in}{0.739656in}}%
\pgfpathlineto{\pgfqpoint{5.334490in}{0.739656in}}%
\pgfpathlineto{\pgfqpoint{5.333925in}{0.739656in}}%
\pgfpathlineto{\pgfqpoint{5.333361in}{0.739656in}}%
\pgfpathlineto{\pgfqpoint{5.332797in}{0.739656in}}%
\pgfpathlineto{\pgfqpoint{5.332232in}{0.739656in}}%
\pgfpathlineto{\pgfqpoint{5.331668in}{0.739656in}}%
\pgfpathlineto{\pgfqpoint{5.331104in}{0.739656in}}%
\pgfpathlineto{\pgfqpoint{5.330539in}{0.739656in}}%
\pgfpathlineto{\pgfqpoint{5.329975in}{0.739656in}}%
\pgfpathlineto{\pgfqpoint{5.329410in}{0.739656in}}%
\pgfpathlineto{\pgfqpoint{5.328846in}{0.739656in}}%
\pgfpathlineto{\pgfqpoint{5.328282in}{0.739656in}}%
\pgfpathlineto{\pgfqpoint{5.327717in}{0.739656in}}%
\pgfpathlineto{\pgfqpoint{5.327153in}{0.739656in}}%
\pgfpathlineto{\pgfqpoint{5.326589in}{0.739656in}}%
\pgfpathlineto{\pgfqpoint{5.326024in}{0.739656in}}%
\pgfpathlineto{\pgfqpoint{5.325460in}{0.739656in}}%
\pgfpathlineto{\pgfqpoint{5.324896in}{0.739656in}}%
\pgfpathlineto{\pgfqpoint{5.324331in}{0.739656in}}%
\pgfpathlineto{\pgfqpoint{5.323767in}{0.739656in}}%
\pgfpathlineto{\pgfqpoint{5.323203in}{0.739656in}}%
\pgfpathlineto{\pgfqpoint{5.322638in}{0.739656in}}%
\pgfpathlineto{\pgfqpoint{5.322074in}{0.739656in}}%
\pgfpathlineto{\pgfqpoint{5.321510in}{0.739656in}}%
\pgfpathlineto{\pgfqpoint{5.320945in}{0.739656in}}%
\pgfpathlineto{\pgfqpoint{5.320381in}{0.739656in}}%
\pgfpathlineto{\pgfqpoint{5.319817in}{0.739656in}}%
\pgfpathlineto{\pgfqpoint{5.319252in}{0.739656in}}%
\pgfpathlineto{\pgfqpoint{5.318688in}{0.739656in}}%
\pgfpathlineto{\pgfqpoint{5.318124in}{0.739656in}}%
\pgfpathlineto{\pgfqpoint{5.317559in}{0.739656in}}%
\pgfpathlineto{\pgfqpoint{5.316995in}{0.739656in}}%
\pgfpathlineto{\pgfqpoint{5.316431in}{0.739656in}}%
\pgfpathlineto{\pgfqpoint{5.315866in}{0.739656in}}%
\pgfpathlineto{\pgfqpoint{5.315302in}{0.739656in}}%
\pgfpathlineto{\pgfqpoint{5.314737in}{0.739656in}}%
\pgfpathlineto{\pgfqpoint{5.314173in}{0.739656in}}%
\pgfpathlineto{\pgfqpoint{5.313609in}{0.739656in}}%
\pgfpathlineto{\pgfqpoint{5.313044in}{0.739656in}}%
\pgfpathlineto{\pgfqpoint{5.312480in}{0.739656in}}%
\pgfpathlineto{\pgfqpoint{5.311916in}{0.739656in}}%
\pgfpathlineto{\pgfqpoint{5.311351in}{0.739656in}}%
\pgfpathlineto{\pgfqpoint{5.310787in}{0.739656in}}%
\pgfpathlineto{\pgfqpoint{5.310223in}{0.739656in}}%
\pgfpathlineto{\pgfqpoint{5.309658in}{0.739656in}}%
\pgfpathlineto{\pgfqpoint{5.309094in}{0.739656in}}%
\pgfpathlineto{\pgfqpoint{5.308530in}{0.739656in}}%
\pgfpathlineto{\pgfqpoint{5.307965in}{0.739656in}}%
\pgfpathlineto{\pgfqpoint{5.307401in}{0.739656in}}%
\pgfpathlineto{\pgfqpoint{5.306837in}{0.739656in}}%
\pgfpathlineto{\pgfqpoint{5.306272in}{0.739656in}}%
\pgfpathlineto{\pgfqpoint{5.305708in}{0.739656in}}%
\pgfpathlineto{\pgfqpoint{5.305144in}{0.739656in}}%
\pgfpathlineto{\pgfqpoint{5.304579in}{0.739656in}}%
\pgfpathlineto{\pgfqpoint{5.304015in}{0.739656in}}%
\pgfpathlineto{\pgfqpoint{5.303451in}{0.739656in}}%
\pgfpathlineto{\pgfqpoint{5.302886in}{0.739656in}}%
\pgfpathlineto{\pgfqpoint{5.302322in}{0.739656in}}%
\pgfpathlineto{\pgfqpoint{5.301758in}{0.739656in}}%
\pgfpathlineto{\pgfqpoint{5.301193in}{0.739656in}}%
\pgfpathlineto{\pgfqpoint{5.300629in}{0.739656in}}%
\pgfpathlineto{\pgfqpoint{5.300065in}{0.739656in}}%
\pgfpathlineto{\pgfqpoint{5.299500in}{0.739656in}}%
\pgfpathlineto{\pgfqpoint{5.298936in}{0.739656in}}%
\pgfpathlineto{\pgfqpoint{5.298371in}{0.739656in}}%
\pgfpathlineto{\pgfqpoint{5.297807in}{0.739656in}}%
\pgfpathlineto{\pgfqpoint{5.297243in}{0.739656in}}%
\pgfpathlineto{\pgfqpoint{5.296678in}{0.739656in}}%
\pgfpathlineto{\pgfqpoint{5.296114in}{0.739656in}}%
\pgfpathlineto{\pgfqpoint{5.295550in}{0.739656in}}%
\pgfpathlineto{\pgfqpoint{5.294985in}{0.739656in}}%
\pgfpathlineto{\pgfqpoint{5.294421in}{0.739656in}}%
\pgfpathlineto{\pgfqpoint{5.293857in}{0.739656in}}%
\pgfpathlineto{\pgfqpoint{5.293292in}{0.739656in}}%
\pgfpathlineto{\pgfqpoint{5.292728in}{0.739656in}}%
\pgfpathlineto{\pgfqpoint{5.292164in}{0.739656in}}%
\pgfpathlineto{\pgfqpoint{5.291599in}{0.739656in}}%
\pgfpathlineto{\pgfqpoint{5.291035in}{0.739656in}}%
\pgfpathlineto{\pgfqpoint{5.290471in}{0.739656in}}%
\pgfpathlineto{\pgfqpoint{5.289906in}{0.739656in}}%
\pgfpathlineto{\pgfqpoint{5.289342in}{0.739656in}}%
\pgfpathlineto{\pgfqpoint{5.288778in}{0.739656in}}%
\pgfpathlineto{\pgfqpoint{5.288213in}{0.739656in}}%
\pgfpathlineto{\pgfqpoint{5.287649in}{0.739656in}}%
\pgfpathlineto{\pgfqpoint{5.287085in}{0.739656in}}%
\pgfpathlineto{\pgfqpoint{5.286520in}{0.739656in}}%
\pgfpathlineto{\pgfqpoint{5.285956in}{0.739656in}}%
\pgfpathlineto{\pgfqpoint{5.285392in}{0.739656in}}%
\pgfpathlineto{\pgfqpoint{5.284827in}{0.739656in}}%
\pgfpathlineto{\pgfqpoint{5.284263in}{0.739656in}}%
\pgfpathlineto{\pgfqpoint{5.283698in}{0.739656in}}%
\pgfpathlineto{\pgfqpoint{5.283134in}{0.739656in}}%
\pgfpathlineto{\pgfqpoint{5.282570in}{0.739656in}}%
\pgfpathlineto{\pgfqpoint{5.282005in}{0.739656in}}%
\pgfpathlineto{\pgfqpoint{5.281441in}{0.739656in}}%
\pgfpathlineto{\pgfqpoint{5.280877in}{0.739656in}}%
\pgfpathlineto{\pgfqpoint{5.280312in}{0.739656in}}%
\pgfpathlineto{\pgfqpoint{5.279748in}{0.739656in}}%
\pgfpathlineto{\pgfqpoint{5.279184in}{0.739656in}}%
\pgfpathlineto{\pgfqpoint{5.278619in}{0.739656in}}%
\pgfpathlineto{\pgfqpoint{5.278055in}{0.739656in}}%
\pgfpathlineto{\pgfqpoint{5.277491in}{0.739656in}}%
\pgfpathlineto{\pgfqpoint{5.276926in}{0.739656in}}%
\pgfpathlineto{\pgfqpoint{5.276362in}{0.739656in}}%
\pgfpathlineto{\pgfqpoint{5.275798in}{0.739656in}}%
\pgfpathlineto{\pgfqpoint{5.275233in}{0.739656in}}%
\pgfpathlineto{\pgfqpoint{5.274669in}{0.739656in}}%
\pgfpathlineto{\pgfqpoint{5.274105in}{0.739656in}}%
\pgfpathlineto{\pgfqpoint{5.273540in}{0.739656in}}%
\pgfpathlineto{\pgfqpoint{5.272976in}{0.739656in}}%
\pgfpathlineto{\pgfqpoint{5.272412in}{0.739656in}}%
\pgfpathlineto{\pgfqpoint{5.271847in}{0.739656in}}%
\pgfpathlineto{\pgfqpoint{5.271283in}{0.739656in}}%
\pgfpathlineto{\pgfqpoint{5.270719in}{0.739656in}}%
\pgfpathlineto{\pgfqpoint{5.270154in}{0.739656in}}%
\pgfpathlineto{\pgfqpoint{5.269590in}{0.739656in}}%
\pgfpathlineto{\pgfqpoint{5.269025in}{0.739656in}}%
\pgfpathlineto{\pgfqpoint{5.268461in}{0.739656in}}%
\pgfpathlineto{\pgfqpoint{5.267897in}{0.739656in}}%
\pgfpathlineto{\pgfqpoint{5.267332in}{0.739656in}}%
\pgfpathlineto{\pgfqpoint{5.266768in}{0.739656in}}%
\pgfpathlineto{\pgfqpoint{5.266204in}{0.739656in}}%
\pgfpathlineto{\pgfqpoint{5.265639in}{0.739656in}}%
\pgfpathlineto{\pgfqpoint{5.265075in}{0.739656in}}%
\pgfpathlineto{\pgfqpoint{5.264511in}{0.739656in}}%
\pgfpathlineto{\pgfqpoint{5.263946in}{0.739656in}}%
\pgfpathlineto{\pgfqpoint{5.263382in}{0.739656in}}%
\pgfpathlineto{\pgfqpoint{5.262818in}{0.739656in}}%
\pgfpathlineto{\pgfqpoint{5.262253in}{0.739656in}}%
\pgfpathlineto{\pgfqpoint{5.261689in}{0.739656in}}%
\pgfpathlineto{\pgfqpoint{5.261125in}{0.739656in}}%
\pgfpathlineto{\pgfqpoint{5.260560in}{0.739656in}}%
\pgfpathlineto{\pgfqpoint{5.259996in}{0.739656in}}%
\pgfpathlineto{\pgfqpoint{5.259432in}{0.739656in}}%
\pgfpathlineto{\pgfqpoint{5.258867in}{0.739656in}}%
\pgfpathlineto{\pgfqpoint{5.258303in}{0.739656in}}%
\pgfpathlineto{\pgfqpoint{5.257739in}{0.739656in}}%
\pgfpathlineto{\pgfqpoint{5.257174in}{0.739656in}}%
\pgfpathlineto{\pgfqpoint{5.256610in}{0.739656in}}%
\pgfpathlineto{\pgfqpoint{5.256046in}{0.739656in}}%
\pgfpathlineto{\pgfqpoint{5.255481in}{0.739656in}}%
\pgfpathlineto{\pgfqpoint{5.254917in}{0.739656in}}%
\pgfpathlineto{\pgfqpoint{5.254353in}{0.739656in}}%
\pgfpathlineto{\pgfqpoint{5.253788in}{0.739656in}}%
\pgfpathlineto{\pgfqpoint{5.253224in}{0.739656in}}%
\pgfpathlineto{\pgfqpoint{5.252659in}{0.739656in}}%
\pgfpathlineto{\pgfqpoint{5.252095in}{0.739656in}}%
\pgfpathlineto{\pgfqpoint{5.251531in}{0.739656in}}%
\pgfpathlineto{\pgfqpoint{5.250966in}{0.739656in}}%
\pgfpathlineto{\pgfqpoint{5.250402in}{0.739656in}}%
\pgfpathlineto{\pgfqpoint{5.249838in}{0.739656in}}%
\pgfpathlineto{\pgfqpoint{5.249273in}{0.739656in}}%
\pgfpathlineto{\pgfqpoint{5.248709in}{0.739656in}}%
\pgfpathlineto{\pgfqpoint{5.248145in}{0.739656in}}%
\pgfpathlineto{\pgfqpoint{5.247580in}{0.739656in}}%
\pgfpathlineto{\pgfqpoint{5.247016in}{0.739656in}}%
\pgfpathlineto{\pgfqpoint{5.246452in}{0.739656in}}%
\pgfpathlineto{\pgfqpoint{5.245887in}{0.739656in}}%
\pgfpathlineto{\pgfqpoint{5.245323in}{0.739656in}}%
\pgfpathlineto{\pgfqpoint{5.244759in}{0.739656in}}%
\pgfpathlineto{\pgfqpoint{5.244194in}{0.739656in}}%
\pgfpathlineto{\pgfqpoint{5.243630in}{0.739656in}}%
\pgfpathlineto{\pgfqpoint{5.243066in}{0.739656in}}%
\pgfpathlineto{\pgfqpoint{5.242501in}{0.739656in}}%
\pgfpathlineto{\pgfqpoint{5.241937in}{0.739656in}}%
\pgfpathlineto{\pgfqpoint{5.241373in}{0.739656in}}%
\pgfpathlineto{\pgfqpoint{5.240808in}{0.739656in}}%
\pgfpathlineto{\pgfqpoint{5.240244in}{0.739656in}}%
\pgfpathlineto{\pgfqpoint{5.239680in}{0.739656in}}%
\pgfpathlineto{\pgfqpoint{5.239115in}{0.739656in}}%
\pgfpathlineto{\pgfqpoint{5.238551in}{0.739656in}}%
\pgfpathlineto{\pgfqpoint{5.237986in}{0.739656in}}%
\pgfpathlineto{\pgfqpoint{5.237422in}{0.739656in}}%
\pgfpathlineto{\pgfqpoint{5.236858in}{0.739656in}}%
\pgfpathlineto{\pgfqpoint{5.236293in}{0.739656in}}%
\pgfpathlineto{\pgfqpoint{5.235729in}{0.739656in}}%
\pgfpathlineto{\pgfqpoint{5.235165in}{0.739656in}}%
\pgfpathlineto{\pgfqpoint{5.234600in}{0.739656in}}%
\pgfpathlineto{\pgfqpoint{5.234036in}{0.739656in}}%
\pgfpathlineto{\pgfqpoint{5.233472in}{0.739656in}}%
\pgfpathlineto{\pgfqpoint{5.232907in}{0.739656in}}%
\pgfpathlineto{\pgfqpoint{5.232343in}{0.739656in}}%
\pgfpathlineto{\pgfqpoint{5.231779in}{0.739656in}}%
\pgfpathlineto{\pgfqpoint{5.231214in}{0.739656in}}%
\pgfpathlineto{\pgfqpoint{5.230650in}{0.739656in}}%
\pgfpathlineto{\pgfqpoint{5.230086in}{0.739656in}}%
\pgfpathlineto{\pgfqpoint{5.229521in}{0.739656in}}%
\pgfpathlineto{\pgfqpoint{5.228957in}{0.739656in}}%
\pgfpathlineto{\pgfqpoint{5.228393in}{0.739656in}}%
\pgfpathlineto{\pgfqpoint{5.227828in}{0.739656in}}%
\pgfpathlineto{\pgfqpoint{5.227264in}{0.739656in}}%
\pgfpathlineto{\pgfqpoint{5.226700in}{0.739656in}}%
\pgfpathlineto{\pgfqpoint{5.226135in}{0.739656in}}%
\pgfpathlineto{\pgfqpoint{5.225571in}{0.739656in}}%
\pgfpathlineto{\pgfqpoint{5.225007in}{0.739656in}}%
\pgfpathlineto{\pgfqpoint{5.224442in}{0.739656in}}%
\pgfpathlineto{\pgfqpoint{5.223878in}{0.739656in}}%
\pgfpathlineto{\pgfqpoint{5.223313in}{0.739656in}}%
\pgfpathlineto{\pgfqpoint{5.222749in}{0.739656in}}%
\pgfpathlineto{\pgfqpoint{5.222185in}{0.739656in}}%
\pgfpathlineto{\pgfqpoint{5.221620in}{0.739656in}}%
\pgfpathlineto{\pgfqpoint{5.221056in}{0.739656in}}%
\pgfpathlineto{\pgfqpoint{5.220492in}{0.739656in}}%
\pgfpathlineto{\pgfqpoint{5.219927in}{0.739656in}}%
\pgfpathlineto{\pgfqpoint{5.219363in}{0.739656in}}%
\pgfpathlineto{\pgfqpoint{5.218799in}{0.739656in}}%
\pgfpathlineto{\pgfqpoint{5.218234in}{0.739656in}}%
\pgfpathlineto{\pgfqpoint{5.217670in}{0.739656in}}%
\pgfpathlineto{\pgfqpoint{5.217106in}{0.739656in}}%
\pgfpathlineto{\pgfqpoint{5.216541in}{0.739656in}}%
\pgfpathlineto{\pgfqpoint{5.215977in}{0.739656in}}%
\pgfpathlineto{\pgfqpoint{5.215413in}{0.739656in}}%
\pgfpathlineto{\pgfqpoint{5.214848in}{0.739656in}}%
\pgfpathlineto{\pgfqpoint{5.214284in}{0.739656in}}%
\pgfpathlineto{\pgfqpoint{5.213720in}{0.739656in}}%
\pgfpathlineto{\pgfqpoint{5.213155in}{0.739656in}}%
\pgfpathlineto{\pgfqpoint{5.212591in}{0.739656in}}%
\pgfpathlineto{\pgfqpoint{5.212027in}{0.739656in}}%
\pgfpathlineto{\pgfqpoint{5.211462in}{0.739656in}}%
\pgfpathlineto{\pgfqpoint{5.210898in}{0.739656in}}%
\pgfpathlineto{\pgfqpoint{5.210334in}{0.739656in}}%
\pgfpathlineto{\pgfqpoint{5.209769in}{0.739656in}}%
\pgfpathlineto{\pgfqpoint{5.209205in}{0.739656in}}%
\pgfpathlineto{\pgfqpoint{5.208641in}{0.739656in}}%
\pgfpathlineto{\pgfqpoint{5.208076in}{0.739656in}}%
\pgfpathlineto{\pgfqpoint{5.207512in}{0.739656in}}%
\pgfpathlineto{\pgfqpoint{5.206947in}{0.739656in}}%
\pgfpathlineto{\pgfqpoint{5.206383in}{0.739656in}}%
\pgfpathlineto{\pgfqpoint{5.205819in}{0.739656in}}%
\pgfpathlineto{\pgfqpoint{5.205254in}{0.739656in}}%
\pgfpathlineto{\pgfqpoint{5.204690in}{0.739656in}}%
\pgfpathlineto{\pgfqpoint{5.204126in}{0.739656in}}%
\pgfpathlineto{\pgfqpoint{5.203561in}{0.739656in}}%
\pgfpathlineto{\pgfqpoint{5.202997in}{0.739656in}}%
\pgfpathlineto{\pgfqpoint{5.202433in}{0.739656in}}%
\pgfpathlineto{\pgfqpoint{5.201868in}{0.739656in}}%
\pgfpathlineto{\pgfqpoint{5.201304in}{0.739656in}}%
\pgfpathlineto{\pgfqpoint{5.200740in}{0.739656in}}%
\pgfpathlineto{\pgfqpoint{5.200175in}{0.739656in}}%
\pgfpathlineto{\pgfqpoint{5.199611in}{0.739656in}}%
\pgfpathlineto{\pgfqpoint{5.199047in}{0.739656in}}%
\pgfpathlineto{\pgfqpoint{5.198482in}{0.739656in}}%
\pgfpathlineto{\pgfqpoint{5.197918in}{0.739656in}}%
\pgfpathlineto{\pgfqpoint{5.197354in}{0.739656in}}%
\pgfpathlineto{\pgfqpoint{5.196789in}{0.739656in}}%
\pgfpathlineto{\pgfqpoint{5.196225in}{0.739656in}}%
\pgfpathlineto{\pgfqpoint{5.195661in}{0.739656in}}%
\pgfpathlineto{\pgfqpoint{5.195096in}{0.739656in}}%
\pgfpathlineto{\pgfqpoint{5.194532in}{0.739656in}}%
\pgfpathlineto{\pgfqpoint{5.193968in}{0.739656in}}%
\pgfpathlineto{\pgfqpoint{5.193403in}{0.739656in}}%
\pgfpathlineto{\pgfqpoint{5.192839in}{0.739656in}}%
\pgfpathlineto{\pgfqpoint{5.192274in}{0.739656in}}%
\pgfpathlineto{\pgfqpoint{5.191710in}{0.739656in}}%
\pgfpathlineto{\pgfqpoint{5.191146in}{0.739656in}}%
\pgfpathlineto{\pgfqpoint{5.190581in}{0.739656in}}%
\pgfpathlineto{\pgfqpoint{5.190017in}{0.739656in}}%
\pgfpathlineto{\pgfqpoint{5.189453in}{0.739656in}}%
\pgfpathlineto{\pgfqpoint{5.188888in}{0.739656in}}%
\pgfpathlineto{\pgfqpoint{5.188324in}{0.739656in}}%
\pgfpathlineto{\pgfqpoint{5.187760in}{0.739656in}}%
\pgfpathlineto{\pgfqpoint{5.187195in}{0.739656in}}%
\pgfpathlineto{\pgfqpoint{5.186631in}{0.739656in}}%
\pgfpathlineto{\pgfqpoint{5.186067in}{0.739656in}}%
\pgfpathlineto{\pgfqpoint{5.185502in}{0.739656in}}%
\pgfpathlineto{\pgfqpoint{5.184938in}{0.739656in}}%
\pgfpathlineto{\pgfqpoint{5.184374in}{0.739656in}}%
\pgfpathlineto{\pgfqpoint{5.183809in}{0.739656in}}%
\pgfpathlineto{\pgfqpoint{5.183245in}{0.739656in}}%
\pgfpathlineto{\pgfqpoint{5.182681in}{0.739656in}}%
\pgfpathlineto{\pgfqpoint{5.182116in}{0.739656in}}%
\pgfpathlineto{\pgfqpoint{5.181552in}{0.739656in}}%
\pgfpathlineto{\pgfqpoint{5.180988in}{0.739656in}}%
\pgfpathlineto{\pgfqpoint{5.180423in}{0.739656in}}%
\pgfpathlineto{\pgfqpoint{5.179859in}{0.739656in}}%
\pgfpathlineto{\pgfqpoint{5.179295in}{0.739656in}}%
\pgfpathlineto{\pgfqpoint{5.178730in}{0.739656in}}%
\pgfpathlineto{\pgfqpoint{5.178166in}{0.739656in}}%
\pgfpathlineto{\pgfqpoint{5.177601in}{0.739656in}}%
\pgfpathlineto{\pgfqpoint{5.177037in}{0.739656in}}%
\pgfpathlineto{\pgfqpoint{5.176473in}{0.739656in}}%
\pgfpathlineto{\pgfqpoint{5.175908in}{0.739656in}}%
\pgfpathlineto{\pgfqpoint{5.175344in}{0.739656in}}%
\pgfpathlineto{\pgfqpoint{5.174780in}{0.739656in}}%
\pgfpathlineto{\pgfqpoint{5.174215in}{0.739656in}}%
\pgfpathlineto{\pgfqpoint{5.173651in}{0.739656in}}%
\pgfpathlineto{\pgfqpoint{5.173087in}{0.739656in}}%
\pgfpathlineto{\pgfqpoint{5.172522in}{0.739656in}}%
\pgfpathlineto{\pgfqpoint{5.171958in}{0.739656in}}%
\pgfpathlineto{\pgfqpoint{5.171394in}{0.739656in}}%
\pgfpathlineto{\pgfqpoint{5.170829in}{0.739656in}}%
\pgfpathlineto{\pgfqpoint{5.170265in}{0.739656in}}%
\pgfpathlineto{\pgfqpoint{5.169701in}{0.739656in}}%
\pgfpathlineto{\pgfqpoint{5.169136in}{0.739656in}}%
\pgfpathlineto{\pgfqpoint{5.168572in}{0.739656in}}%
\pgfpathlineto{\pgfqpoint{5.168008in}{0.739656in}}%
\pgfpathlineto{\pgfqpoint{5.167443in}{0.739656in}}%
\pgfpathlineto{\pgfqpoint{5.166879in}{0.739656in}}%
\pgfpathlineto{\pgfqpoint{5.166315in}{0.739656in}}%
\pgfpathlineto{\pgfqpoint{5.165750in}{0.739656in}}%
\pgfpathlineto{\pgfqpoint{5.165186in}{0.739656in}}%
\pgfpathlineto{\pgfqpoint{5.164622in}{0.739656in}}%
\pgfpathlineto{\pgfqpoint{5.164057in}{0.739656in}}%
\pgfpathlineto{\pgfqpoint{5.163493in}{0.739656in}}%
\pgfpathlineto{\pgfqpoint{5.162929in}{0.739656in}}%
\pgfpathlineto{\pgfqpoint{5.162364in}{0.739656in}}%
\pgfpathlineto{\pgfqpoint{5.161800in}{0.739656in}}%
\pgfpathlineto{\pgfqpoint{5.161235in}{0.739656in}}%
\pgfpathlineto{\pgfqpoint{5.160671in}{0.739656in}}%
\pgfpathlineto{\pgfqpoint{5.160107in}{0.739656in}}%
\pgfpathlineto{\pgfqpoint{5.159542in}{0.739656in}}%
\pgfpathlineto{\pgfqpoint{5.158978in}{0.739656in}}%
\pgfpathlineto{\pgfqpoint{5.158414in}{0.739656in}}%
\pgfpathlineto{\pgfqpoint{5.157849in}{0.739656in}}%
\pgfpathlineto{\pgfqpoint{5.157285in}{0.739656in}}%
\pgfpathlineto{\pgfqpoint{5.156721in}{0.739656in}}%
\pgfpathlineto{\pgfqpoint{5.156156in}{0.739656in}}%
\pgfpathlineto{\pgfqpoint{5.155592in}{0.739656in}}%
\pgfpathlineto{\pgfqpoint{5.155028in}{0.739656in}}%
\pgfpathlineto{\pgfqpoint{5.154463in}{0.739656in}}%
\pgfpathlineto{\pgfqpoint{5.153899in}{0.739656in}}%
\pgfpathlineto{\pgfqpoint{5.153335in}{0.739656in}}%
\pgfpathlineto{\pgfqpoint{5.152770in}{0.739656in}}%
\pgfpathlineto{\pgfqpoint{5.152206in}{0.739656in}}%
\pgfpathlineto{\pgfqpoint{5.151642in}{0.739656in}}%
\pgfpathlineto{\pgfqpoint{5.151077in}{0.739656in}}%
\pgfpathlineto{\pgfqpoint{5.150513in}{0.739656in}}%
\pgfpathlineto{\pgfqpoint{5.149949in}{0.739656in}}%
\pgfpathlineto{\pgfqpoint{5.149384in}{0.739656in}}%
\pgfpathlineto{\pgfqpoint{5.148820in}{0.739656in}}%
\pgfpathlineto{\pgfqpoint{5.148256in}{0.739656in}}%
\pgfpathlineto{\pgfqpoint{5.147691in}{0.739656in}}%
\pgfpathlineto{\pgfqpoint{5.147127in}{0.739656in}}%
\pgfpathlineto{\pgfqpoint{5.146562in}{0.739656in}}%
\pgfpathlineto{\pgfqpoint{5.145998in}{0.739656in}}%
\pgfpathlineto{\pgfqpoint{5.145434in}{0.739656in}}%
\pgfpathlineto{\pgfqpoint{5.144869in}{0.739656in}}%
\pgfpathlineto{\pgfqpoint{5.144305in}{0.739656in}}%
\pgfpathlineto{\pgfqpoint{5.143741in}{0.739656in}}%
\pgfpathlineto{\pgfqpoint{5.143176in}{0.739656in}}%
\pgfpathlineto{\pgfqpoint{5.142612in}{0.739656in}}%
\pgfpathlineto{\pgfqpoint{5.142048in}{0.739656in}}%
\pgfpathlineto{\pgfqpoint{5.141483in}{0.739656in}}%
\pgfpathlineto{\pgfqpoint{5.140919in}{0.739656in}}%
\pgfpathlineto{\pgfqpoint{5.140355in}{0.739656in}}%
\pgfpathlineto{\pgfqpoint{5.139790in}{0.739656in}}%
\pgfpathlineto{\pgfqpoint{5.139226in}{0.739656in}}%
\pgfpathlineto{\pgfqpoint{5.138662in}{0.739656in}}%
\pgfpathlineto{\pgfqpoint{5.138097in}{0.739656in}}%
\pgfpathlineto{\pgfqpoint{5.137533in}{0.739656in}}%
\pgfpathlineto{\pgfqpoint{5.136969in}{0.739656in}}%
\pgfpathlineto{\pgfqpoint{5.136404in}{0.739656in}}%
\pgfpathlineto{\pgfqpoint{5.135840in}{0.739656in}}%
\pgfpathlineto{\pgfqpoint{5.135276in}{0.739656in}}%
\pgfpathlineto{\pgfqpoint{5.134711in}{0.739656in}}%
\pgfpathlineto{\pgfqpoint{5.134147in}{0.739656in}}%
\pgfpathlineto{\pgfqpoint{5.133583in}{0.739656in}}%
\pgfpathlineto{\pgfqpoint{5.133018in}{0.739656in}}%
\pgfpathlineto{\pgfqpoint{5.132454in}{0.739656in}}%
\pgfpathlineto{\pgfqpoint{5.131889in}{0.739656in}}%
\pgfpathlineto{\pgfqpoint{5.131325in}{0.739656in}}%
\pgfpathlineto{\pgfqpoint{5.130761in}{0.739656in}}%
\pgfpathlineto{\pgfqpoint{5.130196in}{0.739656in}}%
\pgfpathlineto{\pgfqpoint{5.129632in}{0.739656in}}%
\pgfpathlineto{\pgfqpoint{5.129068in}{0.739656in}}%
\pgfpathlineto{\pgfqpoint{5.128503in}{0.739656in}}%
\pgfpathlineto{\pgfqpoint{5.127939in}{0.739656in}}%
\pgfpathlineto{\pgfqpoint{5.127375in}{0.739656in}}%
\pgfpathlineto{\pgfqpoint{5.126810in}{0.739656in}}%
\pgfpathlineto{\pgfqpoint{5.126246in}{0.739656in}}%
\pgfpathlineto{\pgfqpoint{5.125682in}{0.739656in}}%
\pgfpathlineto{\pgfqpoint{5.125117in}{0.739656in}}%
\pgfpathlineto{\pgfqpoint{5.124553in}{0.739656in}}%
\pgfpathlineto{\pgfqpoint{5.123989in}{0.739656in}}%
\pgfpathlineto{\pgfqpoint{5.123424in}{0.739656in}}%
\pgfpathlineto{\pgfqpoint{5.122860in}{0.739656in}}%
\pgfpathlineto{\pgfqpoint{5.122296in}{0.739656in}}%
\pgfpathlineto{\pgfqpoint{5.121731in}{0.739656in}}%
\pgfpathlineto{\pgfqpoint{5.121167in}{0.739656in}}%
\pgfpathlineto{\pgfqpoint{5.120603in}{0.739656in}}%
\pgfpathlineto{\pgfqpoint{5.120038in}{0.739656in}}%
\pgfpathlineto{\pgfqpoint{5.119474in}{0.739656in}}%
\pgfpathlineto{\pgfqpoint{5.118910in}{0.739656in}}%
\pgfpathlineto{\pgfqpoint{5.118345in}{0.739656in}}%
\pgfpathlineto{\pgfqpoint{5.117781in}{0.739656in}}%
\pgfpathlineto{\pgfqpoint{5.117216in}{0.739656in}}%
\pgfpathlineto{\pgfqpoint{5.116652in}{0.739656in}}%
\pgfpathlineto{\pgfqpoint{5.116088in}{0.739656in}}%
\pgfpathlineto{\pgfqpoint{5.115523in}{0.739656in}}%
\pgfpathlineto{\pgfqpoint{5.114959in}{0.739656in}}%
\pgfpathlineto{\pgfqpoint{5.114395in}{0.739656in}}%
\pgfpathlineto{\pgfqpoint{5.113830in}{0.739656in}}%
\pgfpathlineto{\pgfqpoint{5.113266in}{0.739656in}}%
\pgfpathlineto{\pgfqpoint{5.112702in}{0.739656in}}%
\pgfpathlineto{\pgfqpoint{5.112137in}{0.739656in}}%
\pgfpathlineto{\pgfqpoint{5.111573in}{0.739656in}}%
\pgfpathlineto{\pgfqpoint{5.111009in}{0.739656in}}%
\pgfpathlineto{\pgfqpoint{5.110444in}{0.739656in}}%
\pgfpathlineto{\pgfqpoint{5.109880in}{0.739656in}}%
\pgfpathlineto{\pgfqpoint{5.109316in}{0.739656in}}%
\pgfpathlineto{\pgfqpoint{5.108751in}{0.739656in}}%
\pgfpathlineto{\pgfqpoint{5.108187in}{0.739656in}}%
\pgfpathlineto{\pgfqpoint{5.107623in}{0.739656in}}%
\pgfpathlineto{\pgfqpoint{5.107058in}{0.739656in}}%
\pgfpathlineto{\pgfqpoint{5.106494in}{0.739656in}}%
\pgfpathlineto{\pgfqpoint{5.105930in}{0.739656in}}%
\pgfpathlineto{\pgfqpoint{5.105365in}{0.739656in}}%
\pgfpathlineto{\pgfqpoint{5.104801in}{0.739656in}}%
\pgfpathlineto{\pgfqpoint{5.104237in}{0.739656in}}%
\pgfpathlineto{\pgfqpoint{5.103672in}{0.739656in}}%
\pgfpathlineto{\pgfqpoint{5.103108in}{0.739656in}}%
\pgfpathlineto{\pgfqpoint{5.102544in}{0.739656in}}%
\pgfpathlineto{\pgfqpoint{5.101979in}{0.739656in}}%
\pgfpathlineto{\pgfqpoint{5.101415in}{0.739656in}}%
\pgfpathlineto{\pgfqpoint{5.100850in}{0.739656in}}%
\pgfpathlineto{\pgfqpoint{5.100286in}{0.739656in}}%
\pgfpathlineto{\pgfqpoint{5.099722in}{0.739656in}}%
\pgfpathlineto{\pgfqpoint{5.099157in}{0.739656in}}%
\pgfpathlineto{\pgfqpoint{5.098593in}{0.739656in}}%
\pgfpathlineto{\pgfqpoint{5.098029in}{0.739656in}}%
\pgfpathlineto{\pgfqpoint{5.097464in}{0.739656in}}%
\pgfpathlineto{\pgfqpoint{5.096900in}{0.739656in}}%
\pgfpathlineto{\pgfqpoint{5.096336in}{0.739656in}}%
\pgfpathlineto{\pgfqpoint{5.095771in}{0.739656in}}%
\pgfpathlineto{\pgfqpoint{5.095207in}{0.739656in}}%
\pgfpathlineto{\pgfqpoint{5.094643in}{0.739656in}}%
\pgfpathlineto{\pgfqpoint{5.094078in}{0.739656in}}%
\pgfpathlineto{\pgfqpoint{5.093514in}{0.739656in}}%
\pgfpathlineto{\pgfqpoint{5.092950in}{0.739656in}}%
\pgfpathlineto{\pgfqpoint{5.092385in}{0.739656in}}%
\pgfpathlineto{\pgfqpoint{5.091821in}{0.739656in}}%
\pgfpathlineto{\pgfqpoint{5.091257in}{0.739656in}}%
\pgfpathlineto{\pgfqpoint{5.090692in}{0.739656in}}%
\pgfpathlineto{\pgfqpoint{5.090128in}{0.739656in}}%
\pgfpathlineto{\pgfqpoint{5.089564in}{0.739656in}}%
\pgfpathlineto{\pgfqpoint{5.088999in}{0.739656in}}%
\pgfpathlineto{\pgfqpoint{5.088435in}{0.739656in}}%
\pgfpathlineto{\pgfqpoint{5.087871in}{0.739656in}}%
\pgfpathlineto{\pgfqpoint{5.087306in}{0.739656in}}%
\pgfpathlineto{\pgfqpoint{5.086742in}{0.739656in}}%
\pgfpathlineto{\pgfqpoint{5.086177in}{0.739656in}}%
\pgfpathlineto{\pgfqpoint{5.085613in}{0.739656in}}%
\pgfpathlineto{\pgfqpoint{5.085049in}{0.739656in}}%
\pgfpathlineto{\pgfqpoint{5.084484in}{0.739656in}}%
\pgfpathlineto{\pgfqpoint{5.083920in}{0.739656in}}%
\pgfpathlineto{\pgfqpoint{5.083356in}{0.739656in}}%
\pgfpathlineto{\pgfqpoint{5.082791in}{0.739656in}}%
\pgfpathlineto{\pgfqpoint{5.082227in}{0.739656in}}%
\pgfpathlineto{\pgfqpoint{5.081663in}{0.739656in}}%
\pgfpathlineto{\pgfqpoint{5.081098in}{0.739656in}}%
\pgfpathlineto{\pgfqpoint{5.080534in}{0.739656in}}%
\pgfpathlineto{\pgfqpoint{5.079970in}{0.739656in}}%
\pgfpathlineto{\pgfqpoint{5.079405in}{0.739656in}}%
\pgfpathlineto{\pgfqpoint{5.078841in}{0.739656in}}%
\pgfpathlineto{\pgfqpoint{5.078277in}{0.739656in}}%
\pgfpathlineto{\pgfqpoint{5.077712in}{0.739656in}}%
\pgfpathlineto{\pgfqpoint{5.077148in}{0.739656in}}%
\pgfpathlineto{\pgfqpoint{5.076584in}{0.739656in}}%
\pgfpathlineto{\pgfqpoint{5.076019in}{0.739656in}}%
\pgfpathlineto{\pgfqpoint{5.075455in}{0.739656in}}%
\pgfpathlineto{\pgfqpoint{5.074891in}{0.739656in}}%
\pgfpathlineto{\pgfqpoint{5.074326in}{0.739656in}}%
\pgfpathlineto{\pgfqpoint{5.073762in}{0.739656in}}%
\pgfpathlineto{\pgfqpoint{5.073198in}{0.739656in}}%
\pgfpathlineto{\pgfqpoint{5.072633in}{0.739656in}}%
\pgfpathlineto{\pgfqpoint{5.072069in}{0.739656in}}%
\pgfpathlineto{\pgfqpoint{5.071504in}{0.739656in}}%
\pgfpathlineto{\pgfqpoint{5.070940in}{0.739656in}}%
\pgfpathlineto{\pgfqpoint{5.070376in}{0.739656in}}%
\pgfpathlineto{\pgfqpoint{5.069811in}{0.739656in}}%
\pgfpathlineto{\pgfqpoint{5.069247in}{0.739656in}}%
\pgfpathlineto{\pgfqpoint{5.068683in}{0.739656in}}%
\pgfpathlineto{\pgfqpoint{5.068118in}{0.739656in}}%
\pgfpathlineto{\pgfqpoint{5.067554in}{0.739656in}}%
\pgfpathlineto{\pgfqpoint{5.066990in}{0.739656in}}%
\pgfpathlineto{\pgfqpoint{5.066425in}{0.739656in}}%
\pgfpathlineto{\pgfqpoint{5.065861in}{0.739656in}}%
\pgfpathlineto{\pgfqpoint{5.065297in}{0.739656in}}%
\pgfpathlineto{\pgfqpoint{5.064732in}{0.739656in}}%
\pgfpathlineto{\pgfqpoint{5.064168in}{0.739656in}}%
\pgfpathlineto{\pgfqpoint{5.063604in}{0.739656in}}%
\pgfpathlineto{\pgfqpoint{5.063039in}{0.739656in}}%
\pgfpathlineto{\pgfqpoint{5.062475in}{0.739656in}}%
\pgfpathlineto{\pgfqpoint{5.061911in}{0.739656in}}%
\pgfpathlineto{\pgfqpoint{5.061346in}{0.739656in}}%
\pgfpathlineto{\pgfqpoint{5.060782in}{0.739656in}}%
\pgfpathlineto{\pgfqpoint{5.060218in}{0.739656in}}%
\pgfpathlineto{\pgfqpoint{5.059653in}{0.739656in}}%
\pgfpathlineto{\pgfqpoint{5.059089in}{0.739656in}}%
\pgfpathlineto{\pgfqpoint{5.058525in}{0.739656in}}%
\pgfpathlineto{\pgfqpoint{5.057960in}{0.739656in}}%
\pgfpathlineto{\pgfqpoint{5.057396in}{0.739656in}}%
\pgfpathlineto{\pgfqpoint{5.056832in}{0.739656in}}%
\pgfpathlineto{\pgfqpoint{5.056267in}{0.739656in}}%
\pgfpathlineto{\pgfqpoint{5.055703in}{0.739656in}}%
\pgfpathlineto{\pgfqpoint{5.055138in}{0.739656in}}%
\pgfpathlineto{\pgfqpoint{5.054574in}{0.739656in}}%
\pgfpathlineto{\pgfqpoint{5.054010in}{0.739656in}}%
\pgfpathlineto{\pgfqpoint{5.053445in}{0.739656in}}%
\pgfpathlineto{\pgfqpoint{5.052881in}{0.739656in}}%
\pgfpathlineto{\pgfqpoint{5.052317in}{0.739656in}}%
\pgfpathlineto{\pgfqpoint{5.051752in}{0.739656in}}%
\pgfpathlineto{\pgfqpoint{5.051188in}{0.739656in}}%
\pgfpathlineto{\pgfqpoint{5.050624in}{0.739656in}}%
\pgfpathlineto{\pgfqpoint{5.050059in}{0.739656in}}%
\pgfpathlineto{\pgfqpoint{5.049495in}{0.739656in}}%
\pgfpathlineto{\pgfqpoint{5.048931in}{0.739656in}}%
\pgfpathlineto{\pgfqpoint{5.048366in}{0.739656in}}%
\pgfpathlineto{\pgfqpoint{5.047802in}{0.739656in}}%
\pgfpathlineto{\pgfqpoint{5.047238in}{0.739656in}}%
\pgfpathlineto{\pgfqpoint{5.046673in}{0.739656in}}%
\pgfpathlineto{\pgfqpoint{5.046109in}{0.739656in}}%
\pgfpathlineto{\pgfqpoint{5.045545in}{0.739656in}}%
\pgfpathlineto{\pgfqpoint{5.044980in}{0.739656in}}%
\pgfpathlineto{\pgfqpoint{5.044416in}{0.739656in}}%
\pgfpathlineto{\pgfqpoint{5.043852in}{0.739656in}}%
\pgfpathlineto{\pgfqpoint{5.043287in}{0.739656in}}%
\pgfpathlineto{\pgfqpoint{5.042723in}{0.739656in}}%
\pgfpathlineto{\pgfqpoint{5.042159in}{0.739656in}}%
\pgfpathlineto{\pgfqpoint{5.041594in}{0.739656in}}%
\pgfpathlineto{\pgfqpoint{5.041030in}{0.739656in}}%
\pgfpathlineto{\pgfqpoint{5.040465in}{0.739656in}}%
\pgfpathlineto{\pgfqpoint{5.039901in}{0.739656in}}%
\pgfpathlineto{\pgfqpoint{5.039337in}{0.739656in}}%
\pgfpathlineto{\pgfqpoint{5.038772in}{0.739656in}}%
\pgfpathlineto{\pgfqpoint{5.038208in}{0.739656in}}%
\pgfpathlineto{\pgfqpoint{5.037644in}{0.739656in}}%
\pgfpathlineto{\pgfqpoint{5.037079in}{0.739656in}}%
\pgfpathlineto{\pgfqpoint{5.036515in}{0.739656in}}%
\pgfpathlineto{\pgfqpoint{5.035951in}{0.739656in}}%
\pgfpathlineto{\pgfqpoint{5.035386in}{0.739656in}}%
\pgfpathlineto{\pgfqpoint{5.034822in}{0.739656in}}%
\pgfpathlineto{\pgfqpoint{5.034258in}{0.739656in}}%
\pgfpathlineto{\pgfqpoint{5.033693in}{0.739656in}}%
\pgfpathlineto{\pgfqpoint{5.033129in}{0.739656in}}%
\pgfpathlineto{\pgfqpoint{5.032565in}{0.739656in}}%
\pgfpathlineto{\pgfqpoint{5.032000in}{0.739656in}}%
\pgfpathlineto{\pgfqpoint{5.031436in}{0.739656in}}%
\pgfpathlineto{\pgfqpoint{5.030872in}{0.739656in}}%
\pgfpathlineto{\pgfqpoint{5.030307in}{0.739656in}}%
\pgfpathlineto{\pgfqpoint{5.029743in}{0.739656in}}%
\pgfpathlineto{\pgfqpoint{5.029179in}{0.739656in}}%
\pgfpathlineto{\pgfqpoint{5.028614in}{0.739656in}}%
\pgfpathlineto{\pgfqpoint{5.028050in}{0.739656in}}%
\pgfpathlineto{\pgfqpoint{5.027486in}{0.739656in}}%
\pgfpathlineto{\pgfqpoint{5.026921in}{0.739656in}}%
\pgfpathlineto{\pgfqpoint{5.026357in}{0.739656in}}%
\pgfpathlineto{\pgfqpoint{5.025792in}{0.739656in}}%
\pgfpathlineto{\pgfqpoint{5.025228in}{0.739656in}}%
\pgfpathlineto{\pgfqpoint{5.024664in}{0.739656in}}%
\pgfpathlineto{\pgfqpoint{5.024099in}{0.739656in}}%
\pgfpathlineto{\pgfqpoint{5.023535in}{0.739656in}}%
\pgfpathlineto{\pgfqpoint{5.022971in}{0.739656in}}%
\pgfpathlineto{\pgfqpoint{5.022406in}{0.739656in}}%
\pgfpathlineto{\pgfqpoint{5.021842in}{0.739656in}}%
\pgfpathlineto{\pgfqpoint{5.021278in}{0.739656in}}%
\pgfpathlineto{\pgfqpoint{5.020713in}{0.739656in}}%
\pgfpathlineto{\pgfqpoint{5.020149in}{0.739656in}}%
\pgfpathlineto{\pgfqpoint{5.019585in}{0.739656in}}%
\pgfpathlineto{\pgfqpoint{5.019020in}{0.739656in}}%
\pgfpathlineto{\pgfqpoint{5.018456in}{0.739656in}}%
\pgfpathlineto{\pgfqpoint{5.017892in}{0.739656in}}%
\pgfpathlineto{\pgfqpoint{5.017327in}{0.739656in}}%
\pgfpathlineto{\pgfqpoint{5.016763in}{0.739656in}}%
\pgfpathlineto{\pgfqpoint{5.016199in}{0.739656in}}%
\pgfpathlineto{\pgfqpoint{5.015634in}{0.739656in}}%
\pgfpathlineto{\pgfqpoint{5.015070in}{0.739656in}}%
\pgfpathlineto{\pgfqpoint{5.014506in}{0.739656in}}%
\pgfpathlineto{\pgfqpoint{5.013941in}{0.739656in}}%
\pgfpathlineto{\pgfqpoint{5.013377in}{0.739656in}}%
\pgfpathlineto{\pgfqpoint{5.012813in}{0.739656in}}%
\pgfpathlineto{\pgfqpoint{5.012248in}{0.739656in}}%
\pgfpathlineto{\pgfqpoint{5.011684in}{0.739656in}}%
\pgfpathlineto{\pgfqpoint{5.011120in}{0.739656in}}%
\pgfpathlineto{\pgfqpoint{5.010555in}{0.739656in}}%
\pgfpathlineto{\pgfqpoint{5.009991in}{0.739656in}}%
\pgfpathlineto{\pgfqpoint{5.009426in}{0.739656in}}%
\pgfpathlineto{\pgfqpoint{5.008862in}{0.739656in}}%
\pgfpathlineto{\pgfqpoint{5.008298in}{0.739656in}}%
\pgfpathlineto{\pgfqpoint{5.007733in}{0.739656in}}%
\pgfpathlineto{\pgfqpoint{5.007169in}{0.739656in}}%
\pgfpathlineto{\pgfqpoint{5.006605in}{0.739656in}}%
\pgfpathlineto{\pgfqpoint{5.006040in}{0.739656in}}%
\pgfpathlineto{\pgfqpoint{5.005476in}{0.739656in}}%
\pgfpathlineto{\pgfqpoint{5.004912in}{0.739656in}}%
\pgfpathlineto{\pgfqpoint{5.004347in}{0.739656in}}%
\pgfpathlineto{\pgfqpoint{5.003783in}{0.739656in}}%
\pgfpathlineto{\pgfqpoint{5.003219in}{0.739656in}}%
\pgfpathlineto{\pgfqpoint{5.002654in}{0.739656in}}%
\pgfpathlineto{\pgfqpoint{5.002090in}{0.739656in}}%
\pgfpathlineto{\pgfqpoint{5.001526in}{0.739656in}}%
\pgfpathlineto{\pgfqpoint{5.000961in}{0.739656in}}%
\pgfpathlineto{\pgfqpoint{5.000397in}{0.739656in}}%
\pgfpathlineto{\pgfqpoint{4.999833in}{0.739656in}}%
\pgfpathlineto{\pgfqpoint{4.999268in}{0.739656in}}%
\pgfpathlineto{\pgfqpoint{4.998704in}{0.739656in}}%
\pgfpathlineto{\pgfqpoint{4.998140in}{0.739656in}}%
\pgfpathlineto{\pgfqpoint{4.997575in}{0.739656in}}%
\pgfpathlineto{\pgfqpoint{4.997011in}{0.739656in}}%
\pgfpathlineto{\pgfqpoint{4.996447in}{0.739656in}}%
\pgfpathlineto{\pgfqpoint{4.995882in}{0.739656in}}%
\pgfpathlineto{\pgfqpoint{4.995318in}{0.739656in}}%
\pgfpathlineto{\pgfqpoint{4.994753in}{0.739656in}}%
\pgfpathlineto{\pgfqpoint{4.994189in}{0.739656in}}%
\pgfpathlineto{\pgfqpoint{4.993625in}{0.739656in}}%
\pgfpathlineto{\pgfqpoint{4.993060in}{0.739656in}}%
\pgfpathlineto{\pgfqpoint{4.992496in}{0.739656in}}%
\pgfpathlineto{\pgfqpoint{4.991932in}{0.739656in}}%
\pgfpathlineto{\pgfqpoint{4.991367in}{0.739656in}}%
\pgfpathlineto{\pgfqpoint{4.990803in}{0.739656in}}%
\pgfpathlineto{\pgfqpoint{4.990239in}{0.739656in}}%
\pgfpathlineto{\pgfqpoint{4.989674in}{0.739656in}}%
\pgfpathlineto{\pgfqpoint{4.989110in}{0.739656in}}%
\pgfpathlineto{\pgfqpoint{4.988546in}{0.739656in}}%
\pgfpathlineto{\pgfqpoint{4.987981in}{0.739656in}}%
\pgfpathlineto{\pgfqpoint{4.987417in}{0.739656in}}%
\pgfpathlineto{\pgfqpoint{4.986853in}{0.739656in}}%
\pgfpathlineto{\pgfqpoint{4.986288in}{0.739656in}}%
\pgfpathlineto{\pgfqpoint{4.985724in}{0.739656in}}%
\pgfpathlineto{\pgfqpoint{4.985160in}{0.739656in}}%
\pgfpathlineto{\pgfqpoint{4.984595in}{0.739656in}}%
\pgfpathlineto{\pgfqpoint{4.984031in}{0.739656in}}%
\pgfpathlineto{\pgfqpoint{4.983467in}{0.739656in}}%
\pgfpathlineto{\pgfqpoint{4.982902in}{0.739656in}}%
\pgfpathlineto{\pgfqpoint{4.982338in}{0.739656in}}%
\pgfpathlineto{\pgfqpoint{4.981774in}{0.739656in}}%
\pgfpathlineto{\pgfqpoint{4.981209in}{0.739656in}}%
\pgfpathlineto{\pgfqpoint{4.980645in}{0.739656in}}%
\pgfpathlineto{\pgfqpoint{4.980080in}{0.739656in}}%
\pgfpathlineto{\pgfqpoint{4.979516in}{0.739656in}}%
\pgfpathlineto{\pgfqpoint{4.978952in}{0.739656in}}%
\pgfpathlineto{\pgfqpoint{4.978387in}{0.739656in}}%
\pgfpathlineto{\pgfqpoint{4.977823in}{0.739656in}}%
\pgfpathlineto{\pgfqpoint{4.977259in}{0.739656in}}%
\pgfpathlineto{\pgfqpoint{4.976694in}{0.739656in}}%
\pgfpathlineto{\pgfqpoint{4.976130in}{0.739656in}}%
\pgfpathlineto{\pgfqpoint{4.975566in}{0.739656in}}%
\pgfpathlineto{\pgfqpoint{4.975001in}{0.739656in}}%
\pgfpathlineto{\pgfqpoint{4.974437in}{0.739656in}}%
\pgfpathlineto{\pgfqpoint{4.973873in}{0.739656in}}%
\pgfpathlineto{\pgfqpoint{4.973308in}{0.739656in}}%
\pgfpathlineto{\pgfqpoint{4.972744in}{0.739656in}}%
\pgfpathlineto{\pgfqpoint{4.972180in}{0.739656in}}%
\pgfpathlineto{\pgfqpoint{4.971615in}{0.739656in}}%
\pgfpathlineto{\pgfqpoint{4.971051in}{0.739656in}}%
\pgfpathlineto{\pgfqpoint{4.970487in}{0.739656in}}%
\pgfpathlineto{\pgfqpoint{4.969922in}{0.739656in}}%
\pgfpathlineto{\pgfqpoint{4.969358in}{0.739656in}}%
\pgfpathlineto{\pgfqpoint{4.968794in}{0.739656in}}%
\pgfpathlineto{\pgfqpoint{4.968229in}{0.739656in}}%
\pgfpathlineto{\pgfqpoint{4.967665in}{0.739656in}}%
\pgfpathlineto{\pgfqpoint{4.967101in}{0.739656in}}%
\pgfpathlineto{\pgfqpoint{4.966536in}{0.739656in}}%
\pgfpathlineto{\pgfqpoint{4.965972in}{0.739656in}}%
\pgfpathlineto{\pgfqpoint{4.965408in}{0.739656in}}%
\pgfpathlineto{\pgfqpoint{4.964843in}{0.739656in}}%
\pgfpathlineto{\pgfqpoint{4.964279in}{0.739656in}}%
\pgfpathlineto{\pgfqpoint{4.963714in}{0.739656in}}%
\pgfpathlineto{\pgfqpoint{4.963150in}{0.739656in}}%
\pgfpathlineto{\pgfqpoint{4.962586in}{0.739656in}}%
\pgfpathlineto{\pgfqpoint{4.962021in}{0.739656in}}%
\pgfpathlineto{\pgfqpoint{4.961457in}{0.739656in}}%
\pgfpathlineto{\pgfqpoint{4.960893in}{0.739656in}}%
\pgfpathlineto{\pgfqpoint{4.960328in}{0.739656in}}%
\pgfpathlineto{\pgfqpoint{4.959764in}{0.739656in}}%
\pgfpathlineto{\pgfqpoint{4.959200in}{0.739656in}}%
\pgfpathlineto{\pgfqpoint{4.958635in}{0.739656in}}%
\pgfpathlineto{\pgfqpoint{4.958071in}{0.739656in}}%
\pgfpathlineto{\pgfqpoint{4.957507in}{0.739656in}}%
\pgfpathlineto{\pgfqpoint{4.956942in}{0.739656in}}%
\pgfpathlineto{\pgfqpoint{4.956378in}{0.739656in}}%
\pgfpathlineto{\pgfqpoint{4.955814in}{0.739656in}}%
\pgfpathlineto{\pgfqpoint{4.955249in}{0.739656in}}%
\pgfpathlineto{\pgfqpoint{4.954685in}{0.739656in}}%
\pgfpathlineto{\pgfqpoint{4.954121in}{0.739656in}}%
\pgfpathlineto{\pgfqpoint{4.953556in}{0.739656in}}%
\pgfpathlineto{\pgfqpoint{4.952992in}{0.739656in}}%
\pgfpathlineto{\pgfqpoint{4.952428in}{0.739656in}}%
\pgfpathlineto{\pgfqpoint{4.951863in}{0.739656in}}%
\pgfpathlineto{\pgfqpoint{4.951299in}{0.739656in}}%
\pgfpathlineto{\pgfqpoint{4.950735in}{0.739656in}}%
\pgfpathlineto{\pgfqpoint{4.950170in}{0.739656in}}%
\pgfpathlineto{\pgfqpoint{4.949606in}{0.739656in}}%
\pgfpathlineto{\pgfqpoint{4.949041in}{0.739656in}}%
\pgfpathlineto{\pgfqpoint{4.948477in}{0.739656in}}%
\pgfpathlineto{\pgfqpoint{4.947913in}{0.739656in}}%
\pgfpathlineto{\pgfqpoint{4.947348in}{0.739656in}}%
\pgfpathlineto{\pgfqpoint{4.946784in}{0.739656in}}%
\pgfpathlineto{\pgfqpoint{4.946220in}{0.739656in}}%
\pgfpathlineto{\pgfqpoint{4.945655in}{0.739656in}}%
\pgfpathlineto{\pgfqpoint{4.945091in}{0.739656in}}%
\pgfpathlineto{\pgfqpoint{4.944527in}{0.739656in}}%
\pgfpathlineto{\pgfqpoint{4.943962in}{0.739656in}}%
\pgfpathlineto{\pgfqpoint{4.943398in}{0.739656in}}%
\pgfpathlineto{\pgfqpoint{4.942834in}{0.739656in}}%
\pgfpathlineto{\pgfqpoint{4.942269in}{0.739656in}}%
\pgfpathlineto{\pgfqpoint{4.941705in}{0.739656in}}%
\pgfpathlineto{\pgfqpoint{4.941141in}{0.739656in}}%
\pgfpathlineto{\pgfqpoint{4.940576in}{0.739656in}}%
\pgfpathlineto{\pgfqpoint{4.940012in}{0.739656in}}%
\pgfpathlineto{\pgfqpoint{4.939448in}{0.739656in}}%
\pgfpathlineto{\pgfqpoint{4.938883in}{0.739656in}}%
\pgfpathlineto{\pgfqpoint{4.938319in}{0.739656in}}%
\pgfpathlineto{\pgfqpoint{4.937755in}{0.739656in}}%
\pgfpathlineto{\pgfqpoint{4.937190in}{0.739656in}}%
\pgfpathlineto{\pgfqpoint{4.936626in}{0.739656in}}%
\pgfpathlineto{\pgfqpoint{4.936062in}{0.739656in}}%
\pgfpathlineto{\pgfqpoint{4.935497in}{0.739656in}}%
\pgfpathlineto{\pgfqpoint{4.934933in}{0.739656in}}%
\pgfpathlineto{\pgfqpoint{4.934368in}{0.739656in}}%
\pgfpathlineto{\pgfqpoint{4.933804in}{0.739656in}}%
\pgfpathlineto{\pgfqpoint{4.933240in}{0.739656in}}%
\pgfpathlineto{\pgfqpoint{4.932675in}{0.739656in}}%
\pgfpathlineto{\pgfqpoint{4.932111in}{0.739656in}}%
\pgfpathlineto{\pgfqpoint{4.931547in}{0.739656in}}%
\pgfpathlineto{\pgfqpoint{4.930982in}{0.739656in}}%
\pgfpathlineto{\pgfqpoint{4.930418in}{0.739656in}}%
\pgfpathlineto{\pgfqpoint{4.929854in}{0.739656in}}%
\pgfpathlineto{\pgfqpoint{4.929289in}{0.739656in}}%
\pgfpathlineto{\pgfqpoint{4.928725in}{0.739656in}}%
\pgfpathlineto{\pgfqpoint{4.928161in}{0.739656in}}%
\pgfpathlineto{\pgfqpoint{4.927596in}{0.739656in}}%
\pgfpathlineto{\pgfqpoint{4.927032in}{0.739656in}}%
\pgfpathlineto{\pgfqpoint{4.926468in}{0.739656in}}%
\pgfpathlineto{\pgfqpoint{4.925903in}{0.739656in}}%
\pgfpathlineto{\pgfqpoint{4.925339in}{0.739656in}}%
\pgfpathlineto{\pgfqpoint{4.924775in}{0.739656in}}%
\pgfpathlineto{\pgfqpoint{4.924210in}{0.739656in}}%
\pgfpathlineto{\pgfqpoint{4.923646in}{0.739656in}}%
\pgfpathlineto{\pgfqpoint{4.923082in}{0.739656in}}%
\pgfpathlineto{\pgfqpoint{4.922517in}{0.739656in}}%
\pgfpathlineto{\pgfqpoint{4.921953in}{0.739656in}}%
\pgfpathlineto{\pgfqpoint{4.921389in}{0.739656in}}%
\pgfpathlineto{\pgfqpoint{4.920824in}{0.739656in}}%
\pgfpathlineto{\pgfqpoint{4.920260in}{0.739656in}}%
\pgfpathlineto{\pgfqpoint{4.919695in}{0.739656in}}%
\pgfpathlineto{\pgfqpoint{4.919131in}{0.739656in}}%
\pgfpathlineto{\pgfqpoint{4.918567in}{0.739656in}}%
\pgfpathlineto{\pgfqpoint{4.918002in}{0.739656in}}%
\pgfpathlineto{\pgfqpoint{4.917438in}{0.739656in}}%
\pgfpathlineto{\pgfqpoint{4.916874in}{0.739656in}}%
\pgfpathlineto{\pgfqpoint{4.916309in}{0.739656in}}%
\pgfpathlineto{\pgfqpoint{4.915745in}{0.739656in}}%
\pgfpathlineto{\pgfqpoint{4.915181in}{0.739656in}}%
\pgfpathlineto{\pgfqpoint{4.914616in}{0.739656in}}%
\pgfpathlineto{\pgfqpoint{4.914052in}{0.739656in}}%
\pgfpathlineto{\pgfqpoint{4.913488in}{0.739656in}}%
\pgfpathlineto{\pgfqpoint{4.912923in}{0.739656in}}%
\pgfpathlineto{\pgfqpoint{4.912359in}{0.739656in}}%
\pgfpathlineto{\pgfqpoint{4.911795in}{0.739656in}}%
\pgfpathlineto{\pgfqpoint{4.911230in}{0.739656in}}%
\pgfpathlineto{\pgfqpoint{4.910666in}{0.739656in}}%
\pgfpathlineto{\pgfqpoint{4.910102in}{0.739656in}}%
\pgfpathlineto{\pgfqpoint{4.909537in}{0.739656in}}%
\pgfpathlineto{\pgfqpoint{4.908973in}{0.739656in}}%
\pgfpathlineto{\pgfqpoint{4.908409in}{0.739656in}}%
\pgfpathlineto{\pgfqpoint{4.907844in}{0.739656in}}%
\pgfpathlineto{\pgfqpoint{4.907280in}{0.739656in}}%
\pgfpathlineto{\pgfqpoint{4.906716in}{0.739656in}}%
\pgfpathlineto{\pgfqpoint{4.906151in}{0.739656in}}%
\pgfpathlineto{\pgfqpoint{4.905587in}{0.739656in}}%
\pgfpathlineto{\pgfqpoint{4.905023in}{0.739656in}}%
\pgfpathlineto{\pgfqpoint{4.904458in}{0.739656in}}%
\pgfpathlineto{\pgfqpoint{4.903894in}{0.739656in}}%
\pgfpathlineto{\pgfqpoint{4.903329in}{0.739656in}}%
\pgfpathlineto{\pgfqpoint{4.902765in}{0.739656in}}%
\pgfpathlineto{\pgfqpoint{4.902201in}{0.739656in}}%
\pgfpathlineto{\pgfqpoint{4.901636in}{0.739656in}}%
\pgfpathlineto{\pgfqpoint{4.901072in}{0.739656in}}%
\pgfpathlineto{\pgfqpoint{4.900508in}{0.739656in}}%
\pgfpathlineto{\pgfqpoint{4.899943in}{0.739656in}}%
\pgfpathlineto{\pgfqpoint{4.899379in}{0.739656in}}%
\pgfpathlineto{\pgfqpoint{4.898815in}{0.739656in}}%
\pgfpathlineto{\pgfqpoint{4.898250in}{0.739656in}}%
\pgfpathlineto{\pgfqpoint{4.897686in}{0.739656in}}%
\pgfpathlineto{\pgfqpoint{4.897122in}{0.739656in}}%
\pgfpathlineto{\pgfqpoint{4.896557in}{0.739656in}}%
\pgfpathlineto{\pgfqpoint{4.895993in}{0.739656in}}%
\pgfpathlineto{\pgfqpoint{4.895429in}{0.739656in}}%
\pgfpathlineto{\pgfqpoint{4.894864in}{0.739656in}}%
\pgfpathlineto{\pgfqpoint{4.894300in}{0.739656in}}%
\pgfpathlineto{\pgfqpoint{4.893736in}{0.739656in}}%
\pgfpathlineto{\pgfqpoint{4.893171in}{0.739656in}}%
\pgfpathlineto{\pgfqpoint{4.892607in}{0.739656in}}%
\pgfpathlineto{\pgfqpoint{4.892043in}{0.739656in}}%
\pgfpathlineto{\pgfqpoint{4.891478in}{0.739656in}}%
\pgfpathlineto{\pgfqpoint{4.890914in}{0.739656in}}%
\pgfpathlineto{\pgfqpoint{4.890350in}{0.739656in}}%
\pgfpathlineto{\pgfqpoint{4.889785in}{0.739656in}}%
\pgfpathlineto{\pgfqpoint{4.889221in}{0.739656in}}%
\pgfpathlineto{\pgfqpoint{4.888656in}{0.739656in}}%
\pgfpathlineto{\pgfqpoint{4.888092in}{0.739656in}}%
\pgfpathlineto{\pgfqpoint{4.887528in}{0.739656in}}%
\pgfpathlineto{\pgfqpoint{4.886963in}{0.739656in}}%
\pgfpathlineto{\pgfqpoint{4.886399in}{0.739656in}}%
\pgfpathlineto{\pgfqpoint{4.885835in}{0.739656in}}%
\pgfpathlineto{\pgfqpoint{4.885270in}{0.739656in}}%
\pgfpathlineto{\pgfqpoint{4.884706in}{0.739656in}}%
\pgfpathlineto{\pgfqpoint{4.884142in}{0.739656in}}%
\pgfpathlineto{\pgfqpoint{4.883577in}{0.739656in}}%
\pgfpathlineto{\pgfqpoint{4.883013in}{0.739656in}}%
\pgfpathlineto{\pgfqpoint{4.882449in}{0.739656in}}%
\pgfpathlineto{\pgfqpoint{4.881884in}{0.739656in}}%
\pgfpathlineto{\pgfqpoint{4.881320in}{0.739656in}}%
\pgfpathlineto{\pgfqpoint{4.880756in}{0.739656in}}%
\pgfpathlineto{\pgfqpoint{4.880191in}{0.739656in}}%
\pgfpathlineto{\pgfqpoint{4.879627in}{0.739656in}}%
\pgfpathlineto{\pgfqpoint{4.879063in}{0.739656in}}%
\pgfpathlineto{\pgfqpoint{4.878498in}{0.739656in}}%
\pgfpathlineto{\pgfqpoint{4.877934in}{0.739656in}}%
\pgfpathlineto{\pgfqpoint{4.877370in}{0.739656in}}%
\pgfpathlineto{\pgfqpoint{4.876805in}{0.739656in}}%
\pgfpathlineto{\pgfqpoint{4.876241in}{0.739656in}}%
\pgfpathlineto{\pgfqpoint{4.875677in}{0.739656in}}%
\pgfpathlineto{\pgfqpoint{4.875112in}{0.739656in}}%
\pgfpathlineto{\pgfqpoint{4.874548in}{0.739656in}}%
\pgfpathlineto{\pgfqpoint{4.873983in}{0.739656in}}%
\pgfpathlineto{\pgfqpoint{4.873419in}{0.739656in}}%
\pgfpathlineto{\pgfqpoint{4.872855in}{0.739656in}}%
\pgfpathlineto{\pgfqpoint{4.872290in}{0.739656in}}%
\pgfpathlineto{\pgfqpoint{4.871726in}{0.739656in}}%
\pgfpathlineto{\pgfqpoint{4.871162in}{0.739656in}}%
\pgfpathlineto{\pgfqpoint{4.870597in}{0.739656in}}%
\pgfpathlineto{\pgfqpoint{4.870033in}{0.739656in}}%
\pgfpathlineto{\pgfqpoint{4.869469in}{0.739656in}}%
\pgfpathlineto{\pgfqpoint{4.868904in}{0.739656in}}%
\pgfpathlineto{\pgfqpoint{4.868340in}{0.739656in}}%
\pgfpathlineto{\pgfqpoint{4.867776in}{0.739656in}}%
\pgfpathlineto{\pgfqpoint{4.867211in}{0.739656in}}%
\pgfpathlineto{\pgfqpoint{4.866647in}{0.739656in}}%
\pgfpathlineto{\pgfqpoint{4.866083in}{0.739656in}}%
\pgfpathlineto{\pgfqpoint{4.865518in}{0.739656in}}%
\pgfpathlineto{\pgfqpoint{4.864954in}{0.739656in}}%
\pgfpathlineto{\pgfqpoint{4.864390in}{0.739656in}}%
\pgfpathlineto{\pgfqpoint{4.863825in}{0.739656in}}%
\pgfpathlineto{\pgfqpoint{4.863261in}{0.739656in}}%
\pgfpathlineto{\pgfqpoint{4.862697in}{0.739656in}}%
\pgfpathlineto{\pgfqpoint{4.862132in}{0.739656in}}%
\pgfpathlineto{\pgfqpoint{4.861568in}{0.739656in}}%
\pgfpathlineto{\pgfqpoint{4.861004in}{0.739656in}}%
\pgfpathlineto{\pgfqpoint{4.860439in}{0.739656in}}%
\pgfpathlineto{\pgfqpoint{4.859875in}{0.739656in}}%
\pgfpathlineto{\pgfqpoint{4.859311in}{0.739656in}}%
\pgfpathlineto{\pgfqpoint{4.858746in}{0.739656in}}%
\pgfpathlineto{\pgfqpoint{4.858182in}{0.739656in}}%
\pgfpathlineto{\pgfqpoint{4.857617in}{0.739656in}}%
\pgfpathlineto{\pgfqpoint{4.857053in}{0.739656in}}%
\pgfpathlineto{\pgfqpoint{4.856489in}{0.739656in}}%
\pgfpathlineto{\pgfqpoint{4.855924in}{0.739656in}}%
\pgfpathlineto{\pgfqpoint{4.855360in}{0.739656in}}%
\pgfpathlineto{\pgfqpoint{4.854796in}{0.739656in}}%
\pgfpathlineto{\pgfqpoint{4.854231in}{0.739656in}}%
\pgfpathlineto{\pgfqpoint{4.853667in}{0.739656in}}%
\pgfpathlineto{\pgfqpoint{4.853103in}{0.739656in}}%
\pgfpathlineto{\pgfqpoint{4.852538in}{0.739656in}}%
\pgfpathlineto{\pgfqpoint{4.851974in}{0.739656in}}%
\pgfpathlineto{\pgfqpoint{4.851410in}{0.739656in}}%
\pgfpathlineto{\pgfqpoint{4.850845in}{0.739656in}}%
\pgfpathlineto{\pgfqpoint{4.850281in}{0.739656in}}%
\pgfpathlineto{\pgfqpoint{4.849717in}{0.739656in}}%
\pgfpathlineto{\pgfqpoint{4.849152in}{0.739656in}}%
\pgfpathlineto{\pgfqpoint{4.848588in}{0.739656in}}%
\pgfpathlineto{\pgfqpoint{4.848024in}{0.739656in}}%
\pgfpathlineto{\pgfqpoint{4.847459in}{0.739656in}}%
\pgfpathlineto{\pgfqpoint{4.846895in}{0.739656in}}%
\pgfpathlineto{\pgfqpoint{4.846331in}{0.739656in}}%
\pgfpathlineto{\pgfqpoint{4.845766in}{0.739656in}}%
\pgfpathlineto{\pgfqpoint{4.845202in}{0.739656in}}%
\pgfpathlineto{\pgfqpoint{4.844638in}{0.739656in}}%
\pgfpathlineto{\pgfqpoint{4.844073in}{0.739656in}}%
\pgfpathlineto{\pgfqpoint{4.843509in}{0.739656in}}%
\pgfpathlineto{\pgfqpoint{4.842944in}{0.739656in}}%
\pgfpathlineto{\pgfqpoint{4.842380in}{0.739656in}}%
\pgfpathlineto{\pgfqpoint{4.841816in}{0.739656in}}%
\pgfpathlineto{\pgfqpoint{4.841251in}{0.739656in}}%
\pgfpathlineto{\pgfqpoint{4.840687in}{0.739656in}}%
\pgfpathlineto{\pgfqpoint{4.840123in}{0.739656in}}%
\pgfpathlineto{\pgfqpoint{4.839558in}{0.739656in}}%
\pgfpathlineto{\pgfqpoint{4.838994in}{0.739656in}}%
\pgfpathlineto{\pgfqpoint{4.838430in}{0.739656in}}%
\pgfpathlineto{\pgfqpoint{4.837865in}{0.739656in}}%
\pgfpathlineto{\pgfqpoint{4.837301in}{0.739656in}}%
\pgfpathlineto{\pgfqpoint{4.836737in}{0.739656in}}%
\pgfpathlineto{\pgfqpoint{4.836172in}{0.739656in}}%
\pgfpathlineto{\pgfqpoint{4.835608in}{0.739656in}}%
\pgfpathlineto{\pgfqpoint{4.835044in}{0.739656in}}%
\pgfpathlineto{\pgfqpoint{4.834479in}{0.739656in}}%
\pgfpathlineto{\pgfqpoint{4.833915in}{0.739656in}}%
\pgfpathlineto{\pgfqpoint{4.833351in}{0.739656in}}%
\pgfpathlineto{\pgfqpoint{4.832786in}{0.739656in}}%
\pgfpathlineto{\pgfqpoint{4.832222in}{0.739656in}}%
\pgfpathlineto{\pgfqpoint{4.831658in}{0.739656in}}%
\pgfpathlineto{\pgfqpoint{4.831093in}{0.739656in}}%
\pgfpathlineto{\pgfqpoint{4.830529in}{0.739656in}}%
\pgfpathlineto{\pgfqpoint{4.829965in}{0.739656in}}%
\pgfpathlineto{\pgfqpoint{4.829400in}{0.739656in}}%
\pgfpathlineto{\pgfqpoint{4.828836in}{0.739656in}}%
\pgfpathlineto{\pgfqpoint{4.828271in}{0.739656in}}%
\pgfpathlineto{\pgfqpoint{4.827707in}{0.739656in}}%
\pgfpathlineto{\pgfqpoint{4.827143in}{0.739656in}}%
\pgfpathlineto{\pgfqpoint{4.826578in}{0.739656in}}%
\pgfpathlineto{\pgfqpoint{4.826014in}{0.739656in}}%
\pgfpathlineto{\pgfqpoint{4.825450in}{0.739656in}}%
\pgfpathlineto{\pgfqpoint{4.824885in}{0.739656in}}%
\pgfpathlineto{\pgfqpoint{4.824321in}{0.739656in}}%
\pgfpathlineto{\pgfqpoint{4.823757in}{0.739656in}}%
\pgfpathlineto{\pgfqpoint{4.823192in}{0.739656in}}%
\pgfpathlineto{\pgfqpoint{4.822628in}{0.739656in}}%
\pgfpathlineto{\pgfqpoint{4.822064in}{0.739656in}}%
\pgfpathlineto{\pgfqpoint{4.821499in}{0.739656in}}%
\pgfpathlineto{\pgfqpoint{4.820935in}{0.739656in}}%
\pgfpathlineto{\pgfqpoint{4.820371in}{0.739656in}}%
\pgfpathlineto{\pgfqpoint{4.819806in}{0.739656in}}%
\pgfpathlineto{\pgfqpoint{4.819242in}{0.739656in}}%
\pgfpathlineto{\pgfqpoint{4.818678in}{0.739656in}}%
\pgfpathlineto{\pgfqpoint{4.818113in}{0.739656in}}%
\pgfpathlineto{\pgfqpoint{4.817549in}{0.739656in}}%
\pgfpathlineto{\pgfqpoint{4.816985in}{0.739656in}}%
\pgfpathlineto{\pgfqpoint{4.816420in}{0.739656in}}%
\pgfpathlineto{\pgfqpoint{4.815856in}{0.739656in}}%
\pgfpathlineto{\pgfqpoint{4.815292in}{0.739656in}}%
\pgfpathlineto{\pgfqpoint{4.814727in}{0.739656in}}%
\pgfpathlineto{\pgfqpoint{4.814163in}{0.739656in}}%
\pgfpathlineto{\pgfqpoint{4.813599in}{0.739656in}}%
\pgfpathlineto{\pgfqpoint{4.813034in}{0.739656in}}%
\pgfpathlineto{\pgfqpoint{4.812470in}{0.739656in}}%
\pgfpathlineto{\pgfqpoint{4.811905in}{0.739656in}}%
\pgfpathlineto{\pgfqpoint{4.811341in}{0.739656in}}%
\pgfpathlineto{\pgfqpoint{4.810777in}{0.739656in}}%
\pgfpathlineto{\pgfqpoint{4.810212in}{0.739656in}}%
\pgfpathlineto{\pgfqpoint{4.809648in}{0.739656in}}%
\pgfpathlineto{\pgfqpoint{4.809084in}{0.739656in}}%
\pgfpathlineto{\pgfqpoint{4.808519in}{0.739656in}}%
\pgfpathlineto{\pgfqpoint{4.807955in}{0.739656in}}%
\pgfpathlineto{\pgfqpoint{4.807391in}{0.739656in}}%
\pgfpathlineto{\pgfqpoint{4.806826in}{0.739656in}}%
\pgfpathlineto{\pgfqpoint{4.806262in}{0.739656in}}%
\pgfpathlineto{\pgfqpoint{4.805698in}{0.739656in}}%
\pgfpathlineto{\pgfqpoint{4.805133in}{0.739656in}}%
\pgfpathlineto{\pgfqpoint{4.804569in}{0.739656in}}%
\pgfpathlineto{\pgfqpoint{4.804005in}{0.739656in}}%
\pgfpathlineto{\pgfqpoint{4.803440in}{0.739656in}}%
\pgfpathlineto{\pgfqpoint{4.802876in}{0.739656in}}%
\pgfpathlineto{\pgfqpoint{4.802312in}{0.739656in}}%
\pgfpathlineto{\pgfqpoint{4.801747in}{0.739656in}}%
\pgfpathlineto{\pgfqpoint{4.801183in}{0.739656in}}%
\pgfpathlineto{\pgfqpoint{4.800619in}{0.739656in}}%
\pgfpathlineto{\pgfqpoint{4.800054in}{0.739656in}}%
\pgfpathlineto{\pgfqpoint{4.799490in}{0.739656in}}%
\pgfpathlineto{\pgfqpoint{4.798926in}{0.739656in}}%
\pgfpathlineto{\pgfqpoint{4.798361in}{0.739656in}}%
\pgfpathlineto{\pgfqpoint{4.797797in}{0.739656in}}%
\pgfpathlineto{\pgfqpoint{4.797232in}{0.739656in}}%
\pgfpathlineto{\pgfqpoint{4.796668in}{0.739656in}}%
\pgfpathlineto{\pgfqpoint{4.796104in}{0.739656in}}%
\pgfpathlineto{\pgfqpoint{4.795539in}{0.739656in}}%
\pgfpathlineto{\pgfqpoint{4.794975in}{0.739656in}}%
\pgfpathlineto{\pgfqpoint{4.794411in}{0.739656in}}%
\pgfpathlineto{\pgfqpoint{4.793846in}{0.739656in}}%
\pgfpathlineto{\pgfqpoint{4.793282in}{0.739656in}}%
\pgfpathlineto{\pgfqpoint{4.792718in}{0.739656in}}%
\pgfpathlineto{\pgfqpoint{4.792153in}{0.739656in}}%
\pgfpathlineto{\pgfqpoint{4.791589in}{0.739656in}}%
\pgfpathlineto{\pgfqpoint{4.791025in}{0.739656in}}%
\pgfpathlineto{\pgfqpoint{4.790460in}{0.739656in}}%
\pgfpathlineto{\pgfqpoint{4.789896in}{0.739656in}}%
\pgfpathlineto{\pgfqpoint{4.789332in}{0.739656in}}%
\pgfpathlineto{\pgfqpoint{4.788767in}{0.739656in}}%
\pgfpathlineto{\pgfqpoint{4.788203in}{0.739656in}}%
\pgfpathlineto{\pgfqpoint{4.787639in}{0.739656in}}%
\pgfpathlineto{\pgfqpoint{4.787074in}{0.739656in}}%
\pgfpathlineto{\pgfqpoint{4.786510in}{0.739656in}}%
\pgfpathlineto{\pgfqpoint{4.785946in}{0.739656in}}%
\pgfpathlineto{\pgfqpoint{4.785381in}{0.739656in}}%
\pgfpathlineto{\pgfqpoint{4.784817in}{0.739656in}}%
\pgfpathlineto{\pgfqpoint{4.784253in}{0.739656in}}%
\pgfpathlineto{\pgfqpoint{4.783688in}{0.739656in}}%
\pgfpathlineto{\pgfqpoint{4.783124in}{0.739656in}}%
\pgfpathlineto{\pgfqpoint{4.782559in}{0.739656in}}%
\pgfpathlineto{\pgfqpoint{4.781995in}{0.739656in}}%
\pgfpathlineto{\pgfqpoint{4.781431in}{0.739656in}}%
\pgfpathlineto{\pgfqpoint{4.780866in}{0.739656in}}%
\pgfpathlineto{\pgfqpoint{4.780302in}{0.739656in}}%
\pgfpathlineto{\pgfqpoint{4.779738in}{0.739656in}}%
\pgfpathlineto{\pgfqpoint{4.779173in}{0.739656in}}%
\pgfpathlineto{\pgfqpoint{4.778609in}{0.739656in}}%
\pgfpathlineto{\pgfqpoint{4.778045in}{0.739656in}}%
\pgfpathlineto{\pgfqpoint{4.777480in}{0.739656in}}%
\pgfpathlineto{\pgfqpoint{4.776916in}{0.739656in}}%
\pgfpathlineto{\pgfqpoint{4.776352in}{0.739656in}}%
\pgfpathlineto{\pgfqpoint{4.775787in}{0.739656in}}%
\pgfpathlineto{\pgfqpoint{4.775223in}{0.739656in}}%
\pgfpathlineto{\pgfqpoint{4.774659in}{0.739656in}}%
\pgfpathlineto{\pgfqpoint{4.774094in}{0.739656in}}%
\pgfpathlineto{\pgfqpoint{4.773530in}{0.739656in}}%
\pgfpathlineto{\pgfqpoint{4.772966in}{0.739656in}}%
\pgfpathlineto{\pgfqpoint{4.772401in}{0.739656in}}%
\pgfpathlineto{\pgfqpoint{4.771837in}{0.739656in}}%
\pgfpathlineto{\pgfqpoint{4.771273in}{0.739656in}}%
\pgfpathlineto{\pgfqpoint{4.770708in}{0.739656in}}%
\pgfpathlineto{\pgfqpoint{4.770144in}{0.739656in}}%
\pgfpathlineto{\pgfqpoint{4.769580in}{0.739656in}}%
\pgfpathlineto{\pgfqpoint{4.769015in}{0.739656in}}%
\pgfpathlineto{\pgfqpoint{4.768451in}{0.739656in}}%
\pgfpathlineto{\pgfqpoint{4.767887in}{0.739656in}}%
\pgfpathlineto{\pgfqpoint{4.767322in}{0.739656in}}%
\pgfpathlineto{\pgfqpoint{4.766758in}{0.739656in}}%
\pgfpathlineto{\pgfqpoint{4.766193in}{0.739656in}}%
\pgfpathlineto{\pgfqpoint{4.765629in}{0.739656in}}%
\pgfpathlineto{\pgfqpoint{4.765065in}{0.739656in}}%
\pgfpathlineto{\pgfqpoint{4.764500in}{0.739656in}}%
\pgfpathlineto{\pgfqpoint{4.763936in}{0.739656in}}%
\pgfpathlineto{\pgfqpoint{4.763372in}{0.739656in}}%
\pgfpathlineto{\pgfqpoint{4.762807in}{0.739656in}}%
\pgfpathlineto{\pgfqpoint{4.762243in}{0.739656in}}%
\pgfpathlineto{\pgfqpoint{4.761679in}{0.739656in}}%
\pgfpathlineto{\pgfqpoint{4.761114in}{0.739656in}}%
\pgfpathlineto{\pgfqpoint{4.760550in}{0.739656in}}%
\pgfpathlineto{\pgfqpoint{4.759986in}{0.739656in}}%
\pgfpathlineto{\pgfqpoint{4.759421in}{0.739656in}}%
\pgfpathlineto{\pgfqpoint{4.758857in}{0.739656in}}%
\pgfpathlineto{\pgfqpoint{4.758293in}{0.739656in}}%
\pgfpathlineto{\pgfqpoint{4.757728in}{0.739656in}}%
\pgfpathlineto{\pgfqpoint{4.757164in}{0.739656in}}%
\pgfpathlineto{\pgfqpoint{4.756600in}{0.739656in}}%
\pgfpathlineto{\pgfqpoint{4.756035in}{0.739656in}}%
\pgfpathlineto{\pgfqpoint{4.755471in}{0.739656in}}%
\pgfpathlineto{\pgfqpoint{4.754907in}{0.739656in}}%
\pgfpathlineto{\pgfqpoint{4.754342in}{0.739656in}}%
\pgfpathlineto{\pgfqpoint{4.753778in}{0.739656in}}%
\pgfpathlineto{\pgfqpoint{4.753214in}{0.739656in}}%
\pgfpathlineto{\pgfqpoint{4.752649in}{0.739656in}}%
\pgfpathlineto{\pgfqpoint{4.752085in}{0.739656in}}%
\pgfpathlineto{\pgfqpoint{4.751520in}{0.739656in}}%
\pgfpathlineto{\pgfqpoint{4.750956in}{0.739656in}}%
\pgfpathlineto{\pgfqpoint{4.750392in}{0.739656in}}%
\pgfpathlineto{\pgfqpoint{4.749827in}{0.739656in}}%
\pgfpathlineto{\pgfqpoint{4.749263in}{0.739656in}}%
\pgfpathlineto{\pgfqpoint{4.748699in}{0.739656in}}%
\pgfpathlineto{\pgfqpoint{4.748134in}{0.739656in}}%
\pgfpathlineto{\pgfqpoint{4.747570in}{0.739656in}}%
\pgfpathlineto{\pgfqpoint{4.747006in}{0.739656in}}%
\pgfpathlineto{\pgfqpoint{4.746441in}{0.739656in}}%
\pgfpathlineto{\pgfqpoint{4.745877in}{0.739656in}}%
\pgfpathlineto{\pgfqpoint{4.745313in}{0.739656in}}%
\pgfpathlineto{\pgfqpoint{4.744748in}{0.739656in}}%
\pgfpathlineto{\pgfqpoint{4.744184in}{0.739656in}}%
\pgfpathlineto{\pgfqpoint{4.743620in}{0.739656in}}%
\pgfpathlineto{\pgfqpoint{4.743055in}{0.739656in}}%
\pgfpathlineto{\pgfqpoint{4.742491in}{0.739656in}}%
\pgfpathlineto{\pgfqpoint{4.741927in}{0.739656in}}%
\pgfpathlineto{\pgfqpoint{4.741362in}{0.739656in}}%
\pgfpathlineto{\pgfqpoint{4.740798in}{0.739656in}}%
\pgfpathlineto{\pgfqpoint{4.740234in}{0.739656in}}%
\pgfpathlineto{\pgfqpoint{4.739669in}{0.739656in}}%
\pgfpathlineto{\pgfqpoint{4.739105in}{0.739656in}}%
\pgfpathlineto{\pgfqpoint{4.738541in}{0.739656in}}%
\pgfpathlineto{\pgfqpoint{4.737976in}{0.739656in}}%
\pgfpathlineto{\pgfqpoint{4.737412in}{0.739656in}}%
\pgfpathlineto{\pgfqpoint{4.736847in}{0.739656in}}%
\pgfpathlineto{\pgfqpoint{4.736283in}{0.739656in}}%
\pgfpathlineto{\pgfqpoint{4.735719in}{0.739656in}}%
\pgfpathlineto{\pgfqpoint{4.735154in}{0.739656in}}%
\pgfpathlineto{\pgfqpoint{4.734590in}{0.739656in}}%
\pgfpathlineto{\pgfqpoint{4.734026in}{0.739656in}}%
\pgfpathlineto{\pgfqpoint{4.733461in}{0.739656in}}%
\pgfpathlineto{\pgfqpoint{4.732897in}{0.739656in}}%
\pgfpathlineto{\pgfqpoint{4.732333in}{0.739656in}}%
\pgfpathlineto{\pgfqpoint{4.731768in}{0.739656in}}%
\pgfpathlineto{\pgfqpoint{4.731204in}{0.739656in}}%
\pgfpathlineto{\pgfqpoint{4.730640in}{0.739656in}}%
\pgfpathlineto{\pgfqpoint{4.730075in}{0.739656in}}%
\pgfpathlineto{\pgfqpoint{4.729511in}{0.739656in}}%
\pgfpathlineto{\pgfqpoint{4.728947in}{0.739656in}}%
\pgfpathlineto{\pgfqpoint{4.728382in}{0.739656in}}%
\pgfpathlineto{\pgfqpoint{4.727818in}{0.739656in}}%
\pgfpathlineto{\pgfqpoint{4.727254in}{0.739656in}}%
\pgfpathlineto{\pgfqpoint{4.726689in}{0.739656in}}%
\pgfpathlineto{\pgfqpoint{4.726125in}{0.739656in}}%
\pgfpathlineto{\pgfqpoint{4.725561in}{0.739656in}}%
\pgfpathlineto{\pgfqpoint{4.724996in}{0.739656in}}%
\pgfpathlineto{\pgfqpoint{4.724432in}{0.739656in}}%
\pgfpathlineto{\pgfqpoint{4.723868in}{0.739656in}}%
\pgfpathlineto{\pgfqpoint{4.723303in}{0.739656in}}%
\pgfpathlineto{\pgfqpoint{4.722739in}{0.739656in}}%
\pgfpathlineto{\pgfqpoint{4.722175in}{0.739656in}}%
\pgfpathlineto{\pgfqpoint{4.721610in}{0.739656in}}%
\pgfpathlineto{\pgfqpoint{4.721046in}{0.739656in}}%
\pgfpathlineto{\pgfqpoint{4.720481in}{0.739656in}}%
\pgfpathlineto{\pgfqpoint{4.719917in}{0.739656in}}%
\pgfpathlineto{\pgfqpoint{4.719353in}{0.739656in}}%
\pgfpathlineto{\pgfqpoint{4.718788in}{0.739656in}}%
\pgfpathlineto{\pgfqpoint{4.718224in}{0.739656in}}%
\pgfpathlineto{\pgfqpoint{4.717660in}{0.739656in}}%
\pgfpathlineto{\pgfqpoint{4.717095in}{0.739656in}}%
\pgfpathlineto{\pgfqpoint{4.716531in}{0.739656in}}%
\pgfpathlineto{\pgfqpoint{4.715967in}{0.739656in}}%
\pgfpathlineto{\pgfqpoint{4.715402in}{0.739656in}}%
\pgfpathlineto{\pgfqpoint{4.714838in}{0.739656in}}%
\pgfpathlineto{\pgfqpoint{4.714274in}{0.739656in}}%
\pgfpathlineto{\pgfqpoint{4.713709in}{0.739656in}}%
\pgfpathlineto{\pgfqpoint{4.713145in}{0.739656in}}%
\pgfpathlineto{\pgfqpoint{4.712581in}{0.739656in}}%
\pgfpathlineto{\pgfqpoint{4.712016in}{0.739656in}}%
\pgfpathlineto{\pgfqpoint{4.711452in}{0.739656in}}%
\pgfpathlineto{\pgfqpoint{4.710888in}{0.739656in}}%
\pgfpathlineto{\pgfqpoint{4.710323in}{0.739656in}}%
\pgfpathlineto{\pgfqpoint{4.709759in}{0.739656in}}%
\pgfpathlineto{\pgfqpoint{4.709195in}{0.739656in}}%
\pgfpathlineto{\pgfqpoint{4.708630in}{0.739656in}}%
\pgfpathlineto{\pgfqpoint{4.708066in}{0.739656in}}%
\pgfpathlineto{\pgfqpoint{4.707502in}{0.739656in}}%
\pgfpathlineto{\pgfqpoint{4.706937in}{0.739656in}}%
\pgfpathlineto{\pgfqpoint{4.706373in}{0.739656in}}%
\pgfpathlineto{\pgfqpoint{4.705808in}{0.739656in}}%
\pgfpathlineto{\pgfqpoint{4.705244in}{0.739656in}}%
\pgfpathlineto{\pgfqpoint{4.704680in}{0.739656in}}%
\pgfpathlineto{\pgfqpoint{4.704115in}{0.739656in}}%
\pgfpathlineto{\pgfqpoint{4.703551in}{0.739656in}}%
\pgfpathlineto{\pgfqpoint{4.702987in}{0.739656in}}%
\pgfpathlineto{\pgfqpoint{4.702422in}{0.739656in}}%
\pgfpathlineto{\pgfqpoint{4.701858in}{0.739656in}}%
\pgfpathlineto{\pgfqpoint{4.701294in}{0.739656in}}%
\pgfpathlineto{\pgfqpoint{4.700729in}{0.739656in}}%
\pgfpathlineto{\pgfqpoint{4.700165in}{0.739656in}}%
\pgfpathlineto{\pgfqpoint{4.699601in}{0.739656in}}%
\pgfpathlineto{\pgfqpoint{4.699036in}{0.739656in}}%
\pgfpathlineto{\pgfqpoint{4.698472in}{0.739656in}}%
\pgfpathlineto{\pgfqpoint{4.697908in}{0.739656in}}%
\pgfpathlineto{\pgfqpoint{4.697343in}{0.739656in}}%
\pgfpathlineto{\pgfqpoint{4.696779in}{0.739656in}}%
\pgfpathlineto{\pgfqpoint{4.696215in}{0.739656in}}%
\pgfpathlineto{\pgfqpoint{4.695650in}{0.739656in}}%
\pgfpathlineto{\pgfqpoint{4.695086in}{0.739656in}}%
\pgfpathlineto{\pgfqpoint{4.694522in}{0.739656in}}%
\pgfpathlineto{\pgfqpoint{4.693957in}{0.739656in}}%
\pgfpathlineto{\pgfqpoint{4.693393in}{0.739656in}}%
\pgfpathlineto{\pgfqpoint{4.692829in}{0.739656in}}%
\pgfpathlineto{\pgfqpoint{4.692264in}{0.739656in}}%
\pgfpathlineto{\pgfqpoint{4.691700in}{0.739656in}}%
\pgfpathlineto{\pgfqpoint{4.691135in}{0.739656in}}%
\pgfpathlineto{\pgfqpoint{4.690571in}{0.739656in}}%
\pgfpathlineto{\pgfqpoint{4.690007in}{0.739656in}}%
\pgfpathlineto{\pgfqpoint{4.689442in}{0.739656in}}%
\pgfpathlineto{\pgfqpoint{4.688878in}{0.739656in}}%
\pgfpathlineto{\pgfqpoint{4.688314in}{0.739656in}}%
\pgfpathlineto{\pgfqpoint{4.687749in}{0.739656in}}%
\pgfpathlineto{\pgfqpoint{4.687185in}{0.739656in}}%
\pgfpathlineto{\pgfqpoint{4.686621in}{0.739656in}}%
\pgfpathlineto{\pgfqpoint{4.686056in}{0.739656in}}%
\pgfpathlineto{\pgfqpoint{4.685492in}{0.739656in}}%
\pgfpathlineto{\pgfqpoint{4.684928in}{0.739656in}}%
\pgfpathlineto{\pgfqpoint{4.684363in}{0.739656in}}%
\pgfpathlineto{\pgfqpoint{4.683799in}{0.739656in}}%
\pgfpathlineto{\pgfqpoint{4.683235in}{0.739656in}}%
\pgfpathlineto{\pgfqpoint{4.682670in}{0.739656in}}%
\pgfpathlineto{\pgfqpoint{4.682106in}{0.739656in}}%
\pgfpathlineto{\pgfqpoint{4.681542in}{0.739656in}}%
\pgfpathlineto{\pgfqpoint{4.680977in}{0.739656in}}%
\pgfpathlineto{\pgfqpoint{4.680413in}{0.739656in}}%
\pgfpathlineto{\pgfqpoint{4.679849in}{0.739656in}}%
\pgfpathlineto{\pgfqpoint{4.679284in}{0.739656in}}%
\pgfpathlineto{\pgfqpoint{4.678720in}{0.739656in}}%
\pgfpathlineto{\pgfqpoint{4.678156in}{0.739656in}}%
\pgfpathlineto{\pgfqpoint{4.677591in}{0.739656in}}%
\pgfpathlineto{\pgfqpoint{4.677027in}{0.739656in}}%
\pgfpathlineto{\pgfqpoint{4.676462in}{0.739656in}}%
\pgfpathlineto{\pgfqpoint{4.675898in}{0.739656in}}%
\pgfpathlineto{\pgfqpoint{4.675334in}{0.739656in}}%
\pgfpathlineto{\pgfqpoint{4.674769in}{0.739656in}}%
\pgfpathlineto{\pgfqpoint{4.674205in}{0.739656in}}%
\pgfpathlineto{\pgfqpoint{4.673641in}{0.739656in}}%
\pgfpathlineto{\pgfqpoint{4.673076in}{0.739656in}}%
\pgfpathlineto{\pgfqpoint{4.672512in}{0.739656in}}%
\pgfpathlineto{\pgfqpoint{4.671948in}{0.739656in}}%
\pgfpathlineto{\pgfqpoint{4.671383in}{0.739656in}}%
\pgfpathlineto{\pgfqpoint{4.670819in}{0.739656in}}%
\pgfpathlineto{\pgfqpoint{4.670255in}{0.739656in}}%
\pgfpathlineto{\pgfqpoint{4.669690in}{0.739656in}}%
\pgfpathlineto{\pgfqpoint{4.669126in}{0.739656in}}%
\pgfpathlineto{\pgfqpoint{4.668562in}{0.739656in}}%
\pgfpathlineto{\pgfqpoint{4.667997in}{0.739656in}}%
\pgfpathlineto{\pgfqpoint{4.667433in}{0.739656in}}%
\pgfpathlineto{\pgfqpoint{4.666869in}{0.739656in}}%
\pgfpathlineto{\pgfqpoint{4.666304in}{0.739656in}}%
\pgfpathlineto{\pgfqpoint{4.665740in}{0.739656in}}%
\pgfpathlineto{\pgfqpoint{4.665176in}{0.739656in}}%
\pgfpathlineto{\pgfqpoint{4.664611in}{0.739656in}}%
\pgfpathlineto{\pgfqpoint{4.664047in}{0.739656in}}%
\pgfpathlineto{\pgfqpoint{4.663483in}{0.739656in}}%
\pgfpathlineto{\pgfqpoint{4.662918in}{0.739656in}}%
\pgfpathlineto{\pgfqpoint{4.662354in}{0.739656in}}%
\pgfpathlineto{\pgfqpoint{4.661790in}{0.739656in}}%
\pgfpathlineto{\pgfqpoint{4.661225in}{0.739656in}}%
\pgfpathlineto{\pgfqpoint{4.660661in}{0.739656in}}%
\pgfpathlineto{\pgfqpoint{4.660096in}{0.739656in}}%
\pgfpathlineto{\pgfqpoint{4.659532in}{0.739656in}}%
\pgfpathlineto{\pgfqpoint{4.658968in}{0.739656in}}%
\pgfpathlineto{\pgfqpoint{4.658403in}{0.739656in}}%
\pgfpathlineto{\pgfqpoint{4.657839in}{0.739656in}}%
\pgfpathlineto{\pgfqpoint{4.657275in}{0.739656in}}%
\pgfpathlineto{\pgfqpoint{4.656710in}{0.739656in}}%
\pgfpathlineto{\pgfqpoint{4.656146in}{0.739656in}}%
\pgfpathlineto{\pgfqpoint{4.655582in}{0.739656in}}%
\pgfpathlineto{\pgfqpoint{4.655017in}{0.739656in}}%
\pgfpathlineto{\pgfqpoint{4.654453in}{0.739656in}}%
\pgfpathlineto{\pgfqpoint{4.653889in}{0.739656in}}%
\pgfpathlineto{\pgfqpoint{4.653324in}{0.739656in}}%
\pgfpathlineto{\pgfqpoint{4.652760in}{0.739656in}}%
\pgfpathlineto{\pgfqpoint{4.652196in}{0.739656in}}%
\pgfpathlineto{\pgfqpoint{4.651631in}{0.739656in}}%
\pgfpathlineto{\pgfqpoint{4.651067in}{0.739656in}}%
\pgfpathlineto{\pgfqpoint{4.650503in}{0.739656in}}%
\pgfpathlineto{\pgfqpoint{4.649938in}{0.739656in}}%
\pgfpathlineto{\pgfqpoint{4.649374in}{0.739656in}}%
\pgfpathlineto{\pgfqpoint{4.648810in}{0.739656in}}%
\pgfpathlineto{\pgfqpoint{4.648245in}{0.739656in}}%
\pgfpathlineto{\pgfqpoint{4.647681in}{0.739656in}}%
\pgfpathlineto{\pgfqpoint{4.647117in}{0.739656in}}%
\pgfpathlineto{\pgfqpoint{4.646552in}{0.739656in}}%
\pgfpathlineto{\pgfqpoint{4.645988in}{0.739656in}}%
\pgfpathlineto{\pgfqpoint{4.645423in}{0.739656in}}%
\pgfpathlineto{\pgfqpoint{4.644859in}{0.739656in}}%
\pgfpathlineto{\pgfqpoint{4.644295in}{0.739656in}}%
\pgfpathlineto{\pgfqpoint{4.643730in}{0.739656in}}%
\pgfpathlineto{\pgfqpoint{4.643166in}{0.739656in}}%
\pgfpathlineto{\pgfqpoint{4.642602in}{0.739656in}}%
\pgfpathlineto{\pgfqpoint{4.642037in}{0.739656in}}%
\pgfpathlineto{\pgfqpoint{4.641473in}{0.739656in}}%
\pgfpathlineto{\pgfqpoint{4.640909in}{0.739656in}}%
\pgfpathlineto{\pgfqpoint{4.640344in}{0.739656in}}%
\pgfpathlineto{\pgfqpoint{4.639780in}{0.739656in}}%
\pgfpathlineto{\pgfqpoint{4.639216in}{0.739656in}}%
\pgfpathlineto{\pgfqpoint{4.638651in}{0.739656in}}%
\pgfpathlineto{\pgfqpoint{4.638087in}{0.739656in}}%
\pgfpathlineto{\pgfqpoint{4.637523in}{0.739656in}}%
\pgfpathlineto{\pgfqpoint{4.636958in}{0.739656in}}%
\pgfpathlineto{\pgfqpoint{4.636394in}{0.739656in}}%
\pgfpathlineto{\pgfqpoint{4.635830in}{0.739656in}}%
\pgfpathlineto{\pgfqpoint{4.635265in}{0.739656in}}%
\pgfpathlineto{\pgfqpoint{4.634701in}{0.739656in}}%
\pgfpathlineto{\pgfqpoint{4.634137in}{0.739656in}}%
\pgfpathlineto{\pgfqpoint{4.633572in}{0.739656in}}%
\pgfpathlineto{\pgfqpoint{4.633008in}{0.739656in}}%
\pgfpathlineto{\pgfqpoint{4.632444in}{0.739656in}}%
\pgfpathlineto{\pgfqpoint{4.631879in}{0.739656in}}%
\pgfpathlineto{\pgfqpoint{4.631315in}{0.739656in}}%
\pgfpathlineto{\pgfqpoint{4.630750in}{0.739656in}}%
\pgfpathlineto{\pgfqpoint{4.630186in}{0.739656in}}%
\pgfpathlineto{\pgfqpoint{4.629622in}{0.739656in}}%
\pgfpathlineto{\pgfqpoint{4.629057in}{0.739656in}}%
\pgfpathlineto{\pgfqpoint{4.628493in}{0.739656in}}%
\pgfpathlineto{\pgfqpoint{4.627929in}{0.739656in}}%
\pgfpathlineto{\pgfqpoint{4.627364in}{0.739656in}}%
\pgfpathlineto{\pgfqpoint{4.626800in}{0.739656in}}%
\pgfpathlineto{\pgfqpoint{4.626236in}{0.739656in}}%
\pgfpathlineto{\pgfqpoint{4.625671in}{0.739656in}}%
\pgfpathlineto{\pgfqpoint{4.625107in}{0.739656in}}%
\pgfpathlineto{\pgfqpoint{4.624543in}{0.739656in}}%
\pgfpathlineto{\pgfqpoint{4.623978in}{0.739656in}}%
\pgfpathlineto{\pgfqpoint{4.623414in}{0.739656in}}%
\pgfpathlineto{\pgfqpoint{4.622850in}{0.739656in}}%
\pgfpathlineto{\pgfqpoint{4.622285in}{0.739656in}}%
\pgfpathlineto{\pgfqpoint{4.621721in}{0.739656in}}%
\pgfpathlineto{\pgfqpoint{4.621157in}{0.739656in}}%
\pgfpathlineto{\pgfqpoint{4.620592in}{0.739656in}}%
\pgfpathlineto{\pgfqpoint{4.620028in}{0.739656in}}%
\pgfpathlineto{\pgfqpoint{4.619464in}{0.739656in}}%
\pgfpathlineto{\pgfqpoint{4.618899in}{0.739656in}}%
\pgfpathlineto{\pgfqpoint{4.618335in}{0.739656in}}%
\pgfpathlineto{\pgfqpoint{4.617771in}{0.739656in}}%
\pgfpathlineto{\pgfqpoint{4.617206in}{0.739656in}}%
\pgfpathlineto{\pgfqpoint{4.616642in}{0.739656in}}%
\pgfpathlineto{\pgfqpoint{4.616078in}{0.739656in}}%
\pgfpathlineto{\pgfqpoint{4.615513in}{0.739656in}}%
\pgfpathlineto{\pgfqpoint{4.614949in}{0.739656in}}%
\pgfpathlineto{\pgfqpoint{4.614384in}{0.739656in}}%
\pgfpathlineto{\pgfqpoint{4.613820in}{0.739656in}}%
\pgfpathlineto{\pgfqpoint{4.613256in}{0.739656in}}%
\pgfpathlineto{\pgfqpoint{4.612691in}{0.739656in}}%
\pgfpathlineto{\pgfqpoint{4.612127in}{0.739656in}}%
\pgfpathlineto{\pgfqpoint{4.611563in}{0.739656in}}%
\pgfpathlineto{\pgfqpoint{4.610998in}{0.739656in}}%
\pgfpathlineto{\pgfqpoint{4.610434in}{0.739656in}}%
\pgfpathlineto{\pgfqpoint{4.609870in}{0.739656in}}%
\pgfpathlineto{\pgfqpoint{4.609305in}{0.739656in}}%
\pgfpathlineto{\pgfqpoint{4.608741in}{0.739656in}}%
\pgfpathlineto{\pgfqpoint{4.608177in}{0.739656in}}%
\pgfpathlineto{\pgfqpoint{4.607612in}{0.739656in}}%
\pgfpathlineto{\pgfqpoint{4.607048in}{0.739656in}}%
\pgfpathlineto{\pgfqpoint{4.606484in}{0.739656in}}%
\pgfpathlineto{\pgfqpoint{4.605919in}{0.739656in}}%
\pgfpathlineto{\pgfqpoint{4.605355in}{0.739656in}}%
\pgfpathlineto{\pgfqpoint{4.604791in}{0.739656in}}%
\pgfpathlineto{\pgfqpoint{4.604226in}{0.739656in}}%
\pgfpathlineto{\pgfqpoint{4.603662in}{0.739656in}}%
\pgfpathlineto{\pgfqpoint{4.603098in}{0.739656in}}%
\pgfpathlineto{\pgfqpoint{4.602533in}{0.739656in}}%
\pgfpathlineto{\pgfqpoint{4.601969in}{0.739656in}}%
\pgfpathlineto{\pgfqpoint{4.601405in}{0.739656in}}%
\pgfpathlineto{\pgfqpoint{4.600840in}{0.739656in}}%
\pgfpathlineto{\pgfqpoint{4.600276in}{0.739656in}}%
\pgfpathlineto{\pgfqpoint{4.599711in}{0.739656in}}%
\pgfpathlineto{\pgfqpoint{4.599147in}{0.739656in}}%
\pgfpathlineto{\pgfqpoint{4.598583in}{0.739656in}}%
\pgfpathlineto{\pgfqpoint{4.598018in}{0.739656in}}%
\pgfpathlineto{\pgfqpoint{4.597454in}{0.739656in}}%
\pgfpathlineto{\pgfqpoint{4.596890in}{0.739656in}}%
\pgfpathlineto{\pgfqpoint{4.596325in}{0.739656in}}%
\pgfpathlineto{\pgfqpoint{4.595761in}{0.739656in}}%
\pgfpathlineto{\pgfqpoint{4.595197in}{0.739656in}}%
\pgfpathlineto{\pgfqpoint{4.594632in}{0.739656in}}%
\pgfpathlineto{\pgfqpoint{4.594068in}{0.739656in}}%
\pgfpathlineto{\pgfqpoint{4.593504in}{0.739656in}}%
\pgfpathlineto{\pgfqpoint{4.592939in}{0.739656in}}%
\pgfpathlineto{\pgfqpoint{4.592375in}{0.739656in}}%
\pgfpathlineto{\pgfqpoint{4.591811in}{0.739656in}}%
\pgfpathlineto{\pgfqpoint{4.591246in}{0.739656in}}%
\pgfpathlineto{\pgfqpoint{4.590682in}{0.739656in}}%
\pgfpathlineto{\pgfqpoint{4.590118in}{0.739656in}}%
\pgfpathlineto{\pgfqpoint{4.589553in}{0.739656in}}%
\pgfpathlineto{\pgfqpoint{4.588989in}{0.739656in}}%
\pgfpathlineto{\pgfqpoint{4.588425in}{0.739656in}}%
\pgfpathlineto{\pgfqpoint{4.587860in}{0.739656in}}%
\pgfpathlineto{\pgfqpoint{4.587296in}{0.739656in}}%
\pgfpathlineto{\pgfqpoint{4.586732in}{0.739656in}}%
\pgfpathlineto{\pgfqpoint{4.586167in}{0.739656in}}%
\pgfpathlineto{\pgfqpoint{4.585603in}{0.739656in}}%
\pgfpathlineto{\pgfqpoint{4.585038in}{0.739656in}}%
\pgfpathlineto{\pgfqpoint{4.584474in}{0.739656in}}%
\pgfpathlineto{\pgfqpoint{4.583910in}{0.739656in}}%
\pgfpathlineto{\pgfqpoint{4.583345in}{0.739656in}}%
\pgfpathlineto{\pgfqpoint{4.582781in}{0.739656in}}%
\pgfpathlineto{\pgfqpoint{4.582217in}{0.739656in}}%
\pgfpathlineto{\pgfqpoint{4.581652in}{0.739656in}}%
\pgfpathlineto{\pgfqpoint{4.581088in}{0.739656in}}%
\pgfpathlineto{\pgfqpoint{4.580524in}{0.739656in}}%
\pgfpathlineto{\pgfqpoint{4.579959in}{0.739656in}}%
\pgfpathlineto{\pgfqpoint{4.579395in}{0.739656in}}%
\pgfpathlineto{\pgfqpoint{4.578831in}{0.739656in}}%
\pgfpathlineto{\pgfqpoint{4.578266in}{0.739656in}}%
\pgfpathlineto{\pgfqpoint{4.577702in}{0.739656in}}%
\pgfpathlineto{\pgfqpoint{4.577138in}{0.739656in}}%
\pgfpathlineto{\pgfqpoint{4.576573in}{0.739656in}}%
\pgfpathlineto{\pgfqpoint{4.576009in}{0.739656in}}%
\pgfpathlineto{\pgfqpoint{4.575445in}{0.739656in}}%
\pgfpathlineto{\pgfqpoint{4.574880in}{0.739656in}}%
\pgfpathlineto{\pgfqpoint{4.574316in}{0.739656in}}%
\pgfpathlineto{\pgfqpoint{4.573752in}{0.739656in}}%
\pgfpathlineto{\pgfqpoint{4.573187in}{0.739656in}}%
\pgfpathlineto{\pgfqpoint{4.572623in}{0.739656in}}%
\pgfpathlineto{\pgfqpoint{4.572059in}{0.739656in}}%
\pgfpathlineto{\pgfqpoint{4.571494in}{0.739656in}}%
\pgfpathlineto{\pgfqpoint{4.570930in}{0.739656in}}%
\pgfpathlineto{\pgfqpoint{4.570366in}{0.739656in}}%
\pgfpathlineto{\pgfqpoint{4.569801in}{0.739656in}}%
\pgfpathlineto{\pgfqpoint{4.569237in}{0.739656in}}%
\pgfpathlineto{\pgfqpoint{4.568672in}{0.739656in}}%
\pgfpathlineto{\pgfqpoint{4.568108in}{0.739656in}}%
\pgfpathlineto{\pgfqpoint{4.567544in}{0.739656in}}%
\pgfpathlineto{\pgfqpoint{4.566979in}{0.739656in}}%
\pgfpathlineto{\pgfqpoint{4.566415in}{0.739656in}}%
\pgfpathlineto{\pgfqpoint{4.565851in}{0.739656in}}%
\pgfpathlineto{\pgfqpoint{4.565286in}{0.739656in}}%
\pgfpathlineto{\pgfqpoint{4.564722in}{0.739656in}}%
\pgfpathlineto{\pgfqpoint{4.564158in}{0.739656in}}%
\pgfpathlineto{\pgfqpoint{4.563593in}{0.739656in}}%
\pgfpathlineto{\pgfqpoint{4.563029in}{0.739656in}}%
\pgfpathlineto{\pgfqpoint{4.562465in}{0.739656in}}%
\pgfpathlineto{\pgfqpoint{4.561900in}{0.739656in}}%
\pgfpathlineto{\pgfqpoint{4.561336in}{0.739656in}}%
\pgfpathlineto{\pgfqpoint{4.560772in}{0.739656in}}%
\pgfpathlineto{\pgfqpoint{4.560207in}{0.739656in}}%
\pgfpathlineto{\pgfqpoint{4.559643in}{0.739656in}}%
\pgfpathlineto{\pgfqpoint{4.559079in}{0.739656in}}%
\pgfpathlineto{\pgfqpoint{4.558514in}{0.739656in}}%
\pgfpathlineto{\pgfqpoint{4.557950in}{0.739656in}}%
\pgfpathlineto{\pgfqpoint{4.557386in}{0.739656in}}%
\pgfpathlineto{\pgfqpoint{4.556821in}{0.739656in}}%
\pgfpathlineto{\pgfqpoint{4.556257in}{0.739656in}}%
\pgfpathlineto{\pgfqpoint{4.555693in}{0.739656in}}%
\pgfpathlineto{\pgfqpoint{4.555128in}{0.739656in}}%
\pgfpathlineto{\pgfqpoint{4.554564in}{0.739656in}}%
\pgfpathlineto{\pgfqpoint{4.553999in}{0.739656in}}%
\pgfpathlineto{\pgfqpoint{4.553435in}{0.739656in}}%
\pgfpathlineto{\pgfqpoint{4.552871in}{0.739656in}}%
\pgfpathlineto{\pgfqpoint{4.552306in}{0.739656in}}%
\pgfpathlineto{\pgfqpoint{4.551742in}{0.739656in}}%
\pgfpathlineto{\pgfqpoint{4.551178in}{0.739656in}}%
\pgfpathlineto{\pgfqpoint{4.550613in}{0.739656in}}%
\pgfpathlineto{\pgfqpoint{4.550049in}{0.739656in}}%
\pgfpathlineto{\pgfqpoint{4.549485in}{0.739656in}}%
\pgfpathlineto{\pgfqpoint{4.548920in}{0.739656in}}%
\pgfpathlineto{\pgfqpoint{4.548356in}{0.739656in}}%
\pgfpathlineto{\pgfqpoint{4.547792in}{0.739656in}}%
\pgfpathlineto{\pgfqpoint{4.547227in}{0.739656in}}%
\pgfpathlineto{\pgfqpoint{4.546663in}{0.739656in}}%
\pgfpathlineto{\pgfqpoint{4.546099in}{0.739656in}}%
\pgfpathlineto{\pgfqpoint{4.545534in}{0.739656in}}%
\pgfpathlineto{\pgfqpoint{4.544970in}{0.739656in}}%
\pgfpathlineto{\pgfqpoint{4.544406in}{0.739656in}}%
\pgfpathlineto{\pgfqpoint{4.543841in}{0.739656in}}%
\pgfpathlineto{\pgfqpoint{4.543277in}{0.739656in}}%
\pgfpathlineto{\pgfqpoint{4.542713in}{0.739656in}}%
\pgfpathlineto{\pgfqpoint{4.542148in}{0.739656in}}%
\pgfpathlineto{\pgfqpoint{4.541584in}{0.739656in}}%
\pgfpathlineto{\pgfqpoint{4.541020in}{0.739656in}}%
\pgfpathlineto{\pgfqpoint{4.540455in}{0.739656in}}%
\pgfpathlineto{\pgfqpoint{4.539891in}{0.739656in}}%
\pgfpathlineto{\pgfqpoint{4.539326in}{0.739656in}}%
\pgfpathlineto{\pgfqpoint{4.538762in}{0.739656in}}%
\pgfpathlineto{\pgfqpoint{4.538198in}{0.739656in}}%
\pgfpathlineto{\pgfqpoint{4.537633in}{0.739656in}}%
\pgfpathlineto{\pgfqpoint{4.537069in}{0.739656in}}%
\pgfpathlineto{\pgfqpoint{4.536505in}{0.739656in}}%
\pgfpathlineto{\pgfqpoint{4.535940in}{0.739656in}}%
\pgfpathlineto{\pgfqpoint{4.535376in}{0.739656in}}%
\pgfpathlineto{\pgfqpoint{4.534812in}{0.739656in}}%
\pgfpathlineto{\pgfqpoint{4.534247in}{0.739656in}}%
\pgfpathlineto{\pgfqpoint{4.533683in}{0.739656in}}%
\pgfpathlineto{\pgfqpoint{4.533119in}{0.739656in}}%
\pgfpathlineto{\pgfqpoint{4.532554in}{0.739656in}}%
\pgfpathlineto{\pgfqpoint{4.531990in}{0.739656in}}%
\pgfpathlineto{\pgfqpoint{4.531426in}{0.739656in}}%
\pgfpathlineto{\pgfqpoint{4.530861in}{0.739656in}}%
\pgfpathlineto{\pgfqpoint{4.530297in}{0.739656in}}%
\pgfpathlineto{\pgfqpoint{4.529733in}{0.739656in}}%
\pgfpathlineto{\pgfqpoint{4.529168in}{0.739656in}}%
\pgfpathlineto{\pgfqpoint{4.528604in}{0.739656in}}%
\pgfpathlineto{\pgfqpoint{4.528040in}{0.739656in}}%
\pgfpathlineto{\pgfqpoint{4.527475in}{0.739656in}}%
\pgfpathlineto{\pgfqpoint{4.526911in}{0.739656in}}%
\pgfpathlineto{\pgfqpoint{4.526347in}{0.739656in}}%
\pgfpathlineto{\pgfqpoint{4.525782in}{0.739656in}}%
\pgfpathlineto{\pgfqpoint{4.525218in}{0.739656in}}%
\pgfpathlineto{\pgfqpoint{4.524654in}{0.739656in}}%
\pgfpathlineto{\pgfqpoint{4.524089in}{0.739656in}}%
\pgfpathlineto{\pgfqpoint{4.523525in}{0.739656in}}%
\pgfpathlineto{\pgfqpoint{4.522960in}{0.739656in}}%
\pgfpathlineto{\pgfqpoint{4.522396in}{0.739656in}}%
\pgfpathlineto{\pgfqpoint{4.521832in}{0.739656in}}%
\pgfpathlineto{\pgfqpoint{4.521267in}{0.739656in}}%
\pgfpathlineto{\pgfqpoint{4.520703in}{0.739656in}}%
\pgfpathlineto{\pgfqpoint{4.520139in}{0.739656in}}%
\pgfpathlineto{\pgfqpoint{4.519574in}{0.739656in}}%
\pgfpathlineto{\pgfqpoint{4.519010in}{0.739656in}}%
\pgfpathlineto{\pgfqpoint{4.518446in}{0.739656in}}%
\pgfpathlineto{\pgfqpoint{4.517881in}{0.739656in}}%
\pgfpathlineto{\pgfqpoint{4.517317in}{0.739656in}}%
\pgfpathlineto{\pgfqpoint{4.516753in}{0.739656in}}%
\pgfpathlineto{\pgfqpoint{4.516188in}{0.739656in}}%
\pgfpathlineto{\pgfqpoint{4.515624in}{0.739656in}}%
\pgfpathlineto{\pgfqpoint{4.515060in}{0.739656in}}%
\pgfpathlineto{\pgfqpoint{4.514495in}{0.739656in}}%
\pgfpathlineto{\pgfqpoint{4.513931in}{0.739656in}}%
\pgfpathlineto{\pgfqpoint{4.513367in}{0.739656in}}%
\pgfpathlineto{\pgfqpoint{4.512802in}{0.739656in}}%
\pgfpathlineto{\pgfqpoint{4.512238in}{0.739656in}}%
\pgfpathlineto{\pgfqpoint{4.511674in}{0.739656in}}%
\pgfpathlineto{\pgfqpoint{4.511109in}{0.739656in}}%
\pgfpathlineto{\pgfqpoint{4.510545in}{0.739656in}}%
\pgfpathlineto{\pgfqpoint{4.509981in}{0.739656in}}%
\pgfpathlineto{\pgfqpoint{4.509416in}{0.739656in}}%
\pgfpathlineto{\pgfqpoint{4.508852in}{0.739656in}}%
\pgfpathlineto{\pgfqpoint{4.508287in}{0.739656in}}%
\pgfpathlineto{\pgfqpoint{4.507723in}{0.739656in}}%
\pgfpathlineto{\pgfqpoint{4.507159in}{0.739656in}}%
\pgfpathlineto{\pgfqpoint{4.506594in}{0.739656in}}%
\pgfpathlineto{\pgfqpoint{4.506030in}{0.739656in}}%
\pgfpathlineto{\pgfqpoint{4.505466in}{0.739656in}}%
\pgfpathlineto{\pgfqpoint{4.504901in}{0.739656in}}%
\pgfpathlineto{\pgfqpoint{4.504337in}{0.739656in}}%
\pgfpathlineto{\pgfqpoint{4.503773in}{0.739656in}}%
\pgfpathlineto{\pgfqpoint{4.503208in}{0.739656in}}%
\pgfpathlineto{\pgfqpoint{4.502644in}{0.739656in}}%
\pgfpathlineto{\pgfqpoint{4.502080in}{0.739656in}}%
\pgfpathlineto{\pgfqpoint{4.501515in}{0.739656in}}%
\pgfpathlineto{\pgfqpoint{4.500951in}{0.739656in}}%
\pgfpathlineto{\pgfqpoint{4.500387in}{0.739656in}}%
\pgfpathlineto{\pgfqpoint{4.499822in}{0.739656in}}%
\pgfpathlineto{\pgfqpoint{4.499258in}{0.739656in}}%
\pgfpathlineto{\pgfqpoint{4.498694in}{0.739656in}}%
\pgfpathlineto{\pgfqpoint{4.498129in}{0.739656in}}%
\pgfpathlineto{\pgfqpoint{4.497565in}{0.739656in}}%
\pgfpathlineto{\pgfqpoint{4.497001in}{0.739656in}}%
\pgfpathlineto{\pgfqpoint{4.496436in}{0.739656in}}%
\pgfpathlineto{\pgfqpoint{4.495872in}{0.739656in}}%
\pgfpathlineto{\pgfqpoint{4.495308in}{0.739656in}}%
\pgfpathlineto{\pgfqpoint{4.494743in}{0.739656in}}%
\pgfpathlineto{\pgfqpoint{4.494179in}{0.739656in}}%
\pgfpathlineto{\pgfqpoint{4.493614in}{0.739656in}}%
\pgfpathlineto{\pgfqpoint{4.493050in}{0.739656in}}%
\pgfpathlineto{\pgfqpoint{4.492486in}{0.739656in}}%
\pgfpathlineto{\pgfqpoint{4.491921in}{0.739656in}}%
\pgfpathlineto{\pgfqpoint{4.491357in}{0.739656in}}%
\pgfpathlineto{\pgfqpoint{4.490793in}{0.739656in}}%
\pgfpathlineto{\pgfqpoint{4.490228in}{0.739656in}}%
\pgfpathlineto{\pgfqpoint{4.489664in}{0.739656in}}%
\pgfpathlineto{\pgfqpoint{4.489100in}{0.739656in}}%
\pgfpathlineto{\pgfqpoint{4.488535in}{0.739656in}}%
\pgfpathlineto{\pgfqpoint{4.487971in}{0.739656in}}%
\pgfpathlineto{\pgfqpoint{4.487407in}{0.739656in}}%
\pgfpathlineto{\pgfqpoint{4.486842in}{0.739656in}}%
\pgfpathlineto{\pgfqpoint{4.486278in}{0.739656in}}%
\pgfpathlineto{\pgfqpoint{4.485714in}{0.739656in}}%
\pgfpathlineto{\pgfqpoint{4.485149in}{0.739656in}}%
\pgfpathlineto{\pgfqpoint{4.484585in}{0.739656in}}%
\pgfpathlineto{\pgfqpoint{4.484021in}{0.739656in}}%
\pgfpathlineto{\pgfqpoint{4.483456in}{0.739656in}}%
\pgfpathlineto{\pgfqpoint{4.482892in}{0.739656in}}%
\pgfpathlineto{\pgfqpoint{4.482328in}{0.739656in}}%
\pgfpathlineto{\pgfqpoint{4.481763in}{0.739656in}}%
\pgfpathlineto{\pgfqpoint{4.481199in}{0.739656in}}%
\pgfpathlineto{\pgfqpoint{4.480635in}{0.739656in}}%
\pgfpathlineto{\pgfqpoint{4.480070in}{0.739656in}}%
\pgfpathlineto{\pgfqpoint{4.479506in}{0.739656in}}%
\pgfpathlineto{\pgfqpoint{4.478941in}{0.739656in}}%
\pgfpathlineto{\pgfqpoint{4.478377in}{0.739656in}}%
\pgfpathlineto{\pgfqpoint{4.477813in}{0.739656in}}%
\pgfpathlineto{\pgfqpoint{4.477248in}{0.739656in}}%
\pgfpathlineto{\pgfqpoint{4.476684in}{0.739656in}}%
\pgfpathlineto{\pgfqpoint{4.476120in}{0.739656in}}%
\pgfpathlineto{\pgfqpoint{4.475555in}{0.739656in}}%
\pgfpathlineto{\pgfqpoint{4.474991in}{0.739656in}}%
\pgfpathlineto{\pgfqpoint{4.474427in}{0.739656in}}%
\pgfpathlineto{\pgfqpoint{4.473862in}{0.739656in}}%
\pgfpathlineto{\pgfqpoint{4.473298in}{0.739656in}}%
\pgfpathlineto{\pgfqpoint{4.472734in}{0.739656in}}%
\pgfpathlineto{\pgfqpoint{4.472169in}{0.739656in}}%
\pgfpathlineto{\pgfqpoint{4.471605in}{0.739656in}}%
\pgfpathlineto{\pgfqpoint{4.471041in}{0.739656in}}%
\pgfpathlineto{\pgfqpoint{4.470476in}{0.739656in}}%
\pgfpathlineto{\pgfqpoint{4.469912in}{0.739656in}}%
\pgfpathlineto{\pgfqpoint{4.469348in}{0.739656in}}%
\pgfpathlineto{\pgfqpoint{4.468783in}{0.739656in}}%
\pgfpathlineto{\pgfqpoint{4.468219in}{0.739656in}}%
\pgfpathlineto{\pgfqpoint{4.467655in}{0.739656in}}%
\pgfpathlineto{\pgfqpoint{4.467090in}{0.739656in}}%
\pgfpathlineto{\pgfqpoint{4.466526in}{0.739656in}}%
\pgfpathlineto{\pgfqpoint{4.465962in}{0.739656in}}%
\pgfpathlineto{\pgfqpoint{4.465397in}{0.739656in}}%
\pgfpathlineto{\pgfqpoint{4.464833in}{0.739656in}}%
\pgfpathlineto{\pgfqpoint{4.464269in}{0.739656in}}%
\pgfpathlineto{\pgfqpoint{4.463704in}{0.739656in}}%
\pgfpathlineto{\pgfqpoint{4.463140in}{0.739656in}}%
\pgfpathlineto{\pgfqpoint{4.462575in}{0.739656in}}%
\pgfpathlineto{\pgfqpoint{4.462011in}{0.739656in}}%
\pgfpathlineto{\pgfqpoint{4.461447in}{0.739656in}}%
\pgfpathlineto{\pgfqpoint{4.460882in}{0.739656in}}%
\pgfpathlineto{\pgfqpoint{4.460318in}{0.739656in}}%
\pgfpathlineto{\pgfqpoint{4.459754in}{0.739656in}}%
\pgfpathlineto{\pgfqpoint{4.459189in}{0.739656in}}%
\pgfpathlineto{\pgfqpoint{4.458625in}{0.739656in}}%
\pgfpathlineto{\pgfqpoint{4.458061in}{0.739656in}}%
\pgfpathlineto{\pgfqpoint{4.457496in}{0.739656in}}%
\pgfpathlineto{\pgfqpoint{4.456932in}{0.739656in}}%
\pgfpathlineto{\pgfqpoint{4.456368in}{0.739656in}}%
\pgfpathlineto{\pgfqpoint{4.455803in}{0.739656in}}%
\pgfpathlineto{\pgfqpoint{4.455239in}{0.739656in}}%
\pgfpathlineto{\pgfqpoint{4.454675in}{0.739656in}}%
\pgfpathlineto{\pgfqpoint{4.454110in}{0.739656in}}%
\pgfpathlineto{\pgfqpoint{4.453546in}{0.739656in}}%
\pgfpathlineto{\pgfqpoint{4.452982in}{0.739656in}}%
\pgfpathlineto{\pgfqpoint{4.452417in}{0.739656in}}%
\pgfpathlineto{\pgfqpoint{4.451853in}{0.739656in}}%
\pgfpathlineto{\pgfqpoint{4.451289in}{0.739656in}}%
\pgfpathlineto{\pgfqpoint{4.450724in}{0.739656in}}%
\pgfpathlineto{\pgfqpoint{4.450160in}{0.739656in}}%
\pgfpathlineto{\pgfqpoint{4.449596in}{0.739656in}}%
\pgfpathlineto{\pgfqpoint{4.449031in}{0.739656in}}%
\pgfpathlineto{\pgfqpoint{4.448467in}{0.739656in}}%
\pgfpathlineto{\pgfqpoint{4.447902in}{0.739656in}}%
\pgfpathlineto{\pgfqpoint{4.447338in}{0.739656in}}%
\pgfpathlineto{\pgfqpoint{4.446774in}{0.739656in}}%
\pgfpathlineto{\pgfqpoint{4.446209in}{0.739656in}}%
\pgfpathlineto{\pgfqpoint{4.445645in}{0.739656in}}%
\pgfpathlineto{\pgfqpoint{4.445081in}{0.739656in}}%
\pgfpathlineto{\pgfqpoint{4.444516in}{0.739656in}}%
\pgfpathlineto{\pgfqpoint{4.443952in}{0.739656in}}%
\pgfpathlineto{\pgfqpoint{4.443388in}{0.739656in}}%
\pgfpathlineto{\pgfqpoint{4.442823in}{0.739656in}}%
\pgfpathlineto{\pgfqpoint{4.442259in}{0.739656in}}%
\pgfpathlineto{\pgfqpoint{4.441695in}{0.739656in}}%
\pgfpathlineto{\pgfqpoint{4.441130in}{0.739656in}}%
\pgfpathlineto{\pgfqpoint{4.440566in}{0.739656in}}%
\pgfpathlineto{\pgfqpoint{4.440002in}{0.739656in}}%
\pgfpathlineto{\pgfqpoint{4.439437in}{0.739656in}}%
\pgfpathlineto{\pgfqpoint{4.438873in}{0.739656in}}%
\pgfpathlineto{\pgfqpoint{4.438309in}{0.739656in}}%
\pgfpathlineto{\pgfqpoint{4.437744in}{0.739656in}}%
\pgfpathlineto{\pgfqpoint{4.437180in}{0.739656in}}%
\pgfpathlineto{\pgfqpoint{4.436616in}{0.739656in}}%
\pgfpathlineto{\pgfqpoint{4.436051in}{0.739656in}}%
\pgfpathlineto{\pgfqpoint{4.435487in}{0.739656in}}%
\pgfpathlineto{\pgfqpoint{4.434923in}{0.739656in}}%
\pgfpathlineto{\pgfqpoint{4.434358in}{0.739656in}}%
\pgfpathlineto{\pgfqpoint{4.433794in}{0.739656in}}%
\pgfpathlineto{\pgfqpoint{4.433229in}{0.739656in}}%
\pgfpathlineto{\pgfqpoint{4.432665in}{0.739656in}}%
\pgfpathlineto{\pgfqpoint{4.432101in}{0.739656in}}%
\pgfpathlineto{\pgfqpoint{4.431536in}{0.739656in}}%
\pgfpathlineto{\pgfqpoint{4.430972in}{0.739656in}}%
\pgfpathlineto{\pgfqpoint{4.430408in}{0.739656in}}%
\pgfpathlineto{\pgfqpoint{4.429843in}{0.739656in}}%
\pgfpathlineto{\pgfqpoint{4.429279in}{0.739656in}}%
\pgfpathlineto{\pgfqpoint{4.428715in}{0.739656in}}%
\pgfpathlineto{\pgfqpoint{4.428150in}{0.739656in}}%
\pgfpathlineto{\pgfqpoint{4.427586in}{0.739656in}}%
\pgfpathlineto{\pgfqpoint{4.427022in}{0.739656in}}%
\pgfpathlineto{\pgfqpoint{4.426457in}{0.739656in}}%
\pgfpathlineto{\pgfqpoint{4.425893in}{0.739656in}}%
\pgfpathlineto{\pgfqpoint{4.425329in}{0.739656in}}%
\pgfpathlineto{\pgfqpoint{4.424764in}{0.739656in}}%
\pgfpathlineto{\pgfqpoint{4.424200in}{0.739656in}}%
\pgfpathlineto{\pgfqpoint{4.423636in}{0.739656in}}%
\pgfpathlineto{\pgfqpoint{4.423071in}{0.739656in}}%
\pgfpathlineto{\pgfqpoint{4.422507in}{0.739656in}}%
\pgfpathlineto{\pgfqpoint{4.421943in}{0.739656in}}%
\pgfpathlineto{\pgfqpoint{4.421378in}{0.739656in}}%
\pgfpathlineto{\pgfqpoint{4.420814in}{0.739656in}}%
\pgfpathlineto{\pgfqpoint{4.420250in}{0.739656in}}%
\pgfpathlineto{\pgfqpoint{4.419685in}{0.739656in}}%
\pgfpathlineto{\pgfqpoint{4.419121in}{0.739656in}}%
\pgfpathlineto{\pgfqpoint{4.418557in}{0.739656in}}%
\pgfpathlineto{\pgfqpoint{4.417992in}{0.739656in}}%
\pgfpathlineto{\pgfqpoint{4.417428in}{0.739656in}}%
\pgfpathlineto{\pgfqpoint{4.416863in}{0.739656in}}%
\pgfpathlineto{\pgfqpoint{4.416299in}{0.739656in}}%
\pgfpathlineto{\pgfqpoint{4.415735in}{0.739656in}}%
\pgfpathlineto{\pgfqpoint{4.415170in}{0.739656in}}%
\pgfpathlineto{\pgfqpoint{4.414606in}{0.739656in}}%
\pgfpathlineto{\pgfqpoint{4.414042in}{0.739656in}}%
\pgfpathlineto{\pgfqpoint{4.413477in}{0.739656in}}%
\pgfpathlineto{\pgfqpoint{4.412913in}{0.739656in}}%
\pgfpathlineto{\pgfqpoint{4.412349in}{0.739656in}}%
\pgfpathlineto{\pgfqpoint{4.411784in}{0.739656in}}%
\pgfpathlineto{\pgfqpoint{4.411220in}{0.739656in}}%
\pgfpathlineto{\pgfqpoint{4.410656in}{0.739656in}}%
\pgfpathlineto{\pgfqpoint{4.410091in}{0.739656in}}%
\pgfpathlineto{\pgfqpoint{4.409527in}{0.739656in}}%
\pgfpathlineto{\pgfqpoint{4.408963in}{0.739656in}}%
\pgfpathlineto{\pgfqpoint{4.408398in}{0.739656in}}%
\pgfpathlineto{\pgfqpoint{4.407834in}{0.739656in}}%
\pgfpathlineto{\pgfqpoint{4.407270in}{0.739656in}}%
\pgfpathlineto{\pgfqpoint{4.406705in}{0.739656in}}%
\pgfpathlineto{\pgfqpoint{4.406141in}{0.739656in}}%
\pgfpathlineto{\pgfqpoint{4.405577in}{0.739656in}}%
\pgfpathlineto{\pgfqpoint{4.405012in}{0.739656in}}%
\pgfpathlineto{\pgfqpoint{4.404448in}{0.739656in}}%
\pgfpathlineto{\pgfqpoint{4.403884in}{0.739656in}}%
\pgfpathlineto{\pgfqpoint{4.403319in}{0.739656in}}%
\pgfpathlineto{\pgfqpoint{4.402755in}{0.739656in}}%
\pgfpathlineto{\pgfqpoint{4.402190in}{0.739656in}}%
\pgfpathlineto{\pgfqpoint{4.401626in}{0.739656in}}%
\pgfpathlineto{\pgfqpoint{4.401062in}{0.739656in}}%
\pgfpathlineto{\pgfqpoint{4.400497in}{0.739656in}}%
\pgfpathlineto{\pgfqpoint{4.399933in}{0.739656in}}%
\pgfpathlineto{\pgfqpoint{4.399369in}{0.739656in}}%
\pgfpathlineto{\pgfqpoint{4.398804in}{0.739656in}}%
\pgfpathlineto{\pgfqpoint{4.398240in}{0.739656in}}%
\pgfpathlineto{\pgfqpoint{4.397676in}{0.739656in}}%
\pgfpathlineto{\pgfqpoint{4.397111in}{0.739656in}}%
\pgfpathlineto{\pgfqpoint{4.396547in}{0.739656in}}%
\pgfpathlineto{\pgfqpoint{4.395983in}{0.739656in}}%
\pgfpathlineto{\pgfqpoint{4.395418in}{0.739656in}}%
\pgfpathlineto{\pgfqpoint{4.394854in}{0.739656in}}%
\pgfpathlineto{\pgfqpoint{4.394290in}{0.739656in}}%
\pgfpathlineto{\pgfqpoint{4.393725in}{0.739656in}}%
\pgfpathlineto{\pgfqpoint{4.393161in}{0.739656in}}%
\pgfpathlineto{\pgfqpoint{4.392597in}{0.739656in}}%
\pgfpathlineto{\pgfqpoint{4.392032in}{0.739656in}}%
\pgfpathlineto{\pgfqpoint{4.391468in}{0.739656in}}%
\pgfpathlineto{\pgfqpoint{4.390904in}{0.739656in}}%
\pgfpathlineto{\pgfqpoint{4.390339in}{0.739656in}}%
\pgfpathlineto{\pgfqpoint{4.389775in}{0.739656in}}%
\pgfpathlineto{\pgfqpoint{4.389211in}{0.739656in}}%
\pgfpathlineto{\pgfqpoint{4.388646in}{0.739656in}}%
\pgfpathlineto{\pgfqpoint{4.388082in}{0.739656in}}%
\pgfpathlineto{\pgfqpoint{4.387517in}{0.739656in}}%
\pgfpathlineto{\pgfqpoint{4.386953in}{0.739656in}}%
\pgfpathlineto{\pgfqpoint{4.386389in}{0.739656in}}%
\pgfpathlineto{\pgfqpoint{4.385824in}{0.739656in}}%
\pgfpathlineto{\pgfqpoint{4.385260in}{0.739656in}}%
\pgfpathlineto{\pgfqpoint{4.384696in}{0.739656in}}%
\pgfpathlineto{\pgfqpoint{4.384131in}{0.739656in}}%
\pgfpathlineto{\pgfqpoint{4.383567in}{0.739656in}}%
\pgfpathlineto{\pgfqpoint{4.383003in}{0.739656in}}%
\pgfpathlineto{\pgfqpoint{4.382438in}{0.739656in}}%
\pgfpathlineto{\pgfqpoint{4.381874in}{0.739656in}}%
\pgfpathlineto{\pgfqpoint{4.381310in}{0.739656in}}%
\pgfpathlineto{\pgfqpoint{4.380745in}{0.739656in}}%
\pgfpathlineto{\pgfqpoint{4.380181in}{0.739656in}}%
\pgfpathlineto{\pgfqpoint{4.379617in}{0.739656in}}%
\pgfpathlineto{\pgfqpoint{4.379052in}{0.739656in}}%
\pgfpathlineto{\pgfqpoint{4.378488in}{0.739656in}}%
\pgfpathlineto{\pgfqpoint{4.377924in}{0.739656in}}%
\pgfpathlineto{\pgfqpoint{4.377359in}{0.739656in}}%
\pgfpathlineto{\pgfqpoint{4.376795in}{0.739656in}}%
\pgfpathlineto{\pgfqpoint{4.376231in}{0.739656in}}%
\pgfpathlineto{\pgfqpoint{4.375666in}{0.739656in}}%
\pgfpathlineto{\pgfqpoint{4.375102in}{0.739656in}}%
\pgfpathlineto{\pgfqpoint{4.374538in}{0.739656in}}%
\pgfpathlineto{\pgfqpoint{4.373973in}{0.739656in}}%
\pgfpathlineto{\pgfqpoint{4.373409in}{0.739656in}}%
\pgfpathlineto{\pgfqpoint{4.372845in}{0.739656in}}%
\pgfpathlineto{\pgfqpoint{4.372280in}{0.739656in}}%
\pgfpathlineto{\pgfqpoint{4.371716in}{0.739656in}}%
\pgfpathlineto{\pgfqpoint{4.371151in}{0.739656in}}%
\pgfpathlineto{\pgfqpoint{4.370587in}{0.739656in}}%
\pgfpathlineto{\pgfqpoint{4.370023in}{0.739656in}}%
\pgfpathlineto{\pgfqpoint{4.369458in}{0.739656in}}%
\pgfpathlineto{\pgfqpoint{4.368894in}{0.739656in}}%
\pgfpathlineto{\pgfqpoint{4.368330in}{0.739656in}}%
\pgfpathlineto{\pgfqpoint{4.367765in}{0.739656in}}%
\pgfpathlineto{\pgfqpoint{4.367201in}{0.739656in}}%
\pgfpathlineto{\pgfqpoint{4.366637in}{0.739656in}}%
\pgfpathlineto{\pgfqpoint{4.366072in}{0.739656in}}%
\pgfpathlineto{\pgfqpoint{4.365508in}{0.739656in}}%
\pgfpathlineto{\pgfqpoint{4.364944in}{0.739656in}}%
\pgfpathlineto{\pgfqpoint{4.364379in}{0.739656in}}%
\pgfpathlineto{\pgfqpoint{4.363815in}{0.739656in}}%
\pgfpathlineto{\pgfqpoint{4.363251in}{0.739656in}}%
\pgfpathlineto{\pgfqpoint{4.362686in}{0.739656in}}%
\pgfpathlineto{\pgfqpoint{4.362122in}{0.739656in}}%
\pgfpathlineto{\pgfqpoint{4.361558in}{0.739656in}}%
\pgfpathlineto{\pgfqpoint{4.360993in}{0.739656in}}%
\pgfpathlineto{\pgfqpoint{4.360429in}{0.739656in}}%
\pgfpathlineto{\pgfqpoint{4.359865in}{0.739656in}}%
\pgfpathlineto{\pgfqpoint{4.359300in}{0.739656in}}%
\pgfpathlineto{\pgfqpoint{4.358736in}{0.739656in}}%
\pgfpathlineto{\pgfqpoint{4.358172in}{0.739656in}}%
\pgfpathlineto{\pgfqpoint{4.357607in}{0.739656in}}%
\pgfpathlineto{\pgfqpoint{4.357043in}{0.739656in}}%
\pgfpathlineto{\pgfqpoint{4.356478in}{0.739656in}}%
\pgfpathlineto{\pgfqpoint{4.355914in}{0.739656in}}%
\pgfpathlineto{\pgfqpoint{4.355350in}{0.739656in}}%
\pgfpathlineto{\pgfqpoint{4.354785in}{0.739656in}}%
\pgfpathlineto{\pgfqpoint{4.354221in}{0.739656in}}%
\pgfpathlineto{\pgfqpoint{4.353657in}{0.739656in}}%
\pgfpathlineto{\pgfqpoint{4.353092in}{0.739656in}}%
\pgfpathlineto{\pgfqpoint{4.352528in}{0.739656in}}%
\pgfpathlineto{\pgfqpoint{4.351964in}{0.739656in}}%
\pgfpathlineto{\pgfqpoint{4.351399in}{0.739656in}}%
\pgfpathlineto{\pgfqpoint{4.350835in}{0.739656in}}%
\pgfpathlineto{\pgfqpoint{4.350271in}{0.739656in}}%
\pgfpathlineto{\pgfqpoint{4.349706in}{0.739656in}}%
\pgfpathlineto{\pgfqpoint{4.349142in}{0.739656in}}%
\pgfpathlineto{\pgfqpoint{4.348578in}{0.739656in}}%
\pgfpathlineto{\pgfqpoint{4.348013in}{0.739656in}}%
\pgfpathlineto{\pgfqpoint{4.347449in}{0.739656in}}%
\pgfpathlineto{\pgfqpoint{4.346885in}{0.739656in}}%
\pgfpathlineto{\pgfqpoint{4.346320in}{0.739656in}}%
\pgfpathlineto{\pgfqpoint{4.345756in}{0.739656in}}%
\pgfpathlineto{\pgfqpoint{4.345192in}{0.739656in}}%
\pgfpathlineto{\pgfqpoint{4.344627in}{0.739656in}}%
\pgfpathlineto{\pgfqpoint{4.344063in}{0.739656in}}%
\pgfpathlineto{\pgfqpoint{4.343499in}{0.739656in}}%
\pgfpathlineto{\pgfqpoint{4.342934in}{0.739656in}}%
\pgfpathlineto{\pgfqpoint{4.342370in}{0.739656in}}%
\pgfpathlineto{\pgfqpoint{4.341805in}{0.739656in}}%
\pgfpathlineto{\pgfqpoint{4.341241in}{0.739656in}}%
\pgfpathlineto{\pgfqpoint{4.340677in}{0.739656in}}%
\pgfpathlineto{\pgfqpoint{4.340112in}{0.739656in}}%
\pgfpathlineto{\pgfqpoint{4.339548in}{0.739656in}}%
\pgfpathlineto{\pgfqpoint{4.338984in}{0.739656in}}%
\pgfpathlineto{\pgfqpoint{4.338419in}{0.739656in}}%
\pgfpathlineto{\pgfqpoint{4.337855in}{0.739656in}}%
\pgfpathlineto{\pgfqpoint{4.337291in}{0.739656in}}%
\pgfpathlineto{\pgfqpoint{4.336726in}{0.739656in}}%
\pgfpathlineto{\pgfqpoint{4.336162in}{0.739656in}}%
\pgfpathlineto{\pgfqpoint{4.335598in}{0.739656in}}%
\pgfpathlineto{\pgfqpoint{4.335033in}{0.739656in}}%
\pgfpathlineto{\pgfqpoint{4.334469in}{0.739656in}}%
\pgfpathlineto{\pgfqpoint{4.333905in}{0.739656in}}%
\pgfpathlineto{\pgfqpoint{4.333340in}{0.739656in}}%
\pgfpathlineto{\pgfqpoint{4.332776in}{0.739656in}}%
\pgfpathlineto{\pgfqpoint{4.332212in}{0.739656in}}%
\pgfpathlineto{\pgfqpoint{4.331647in}{0.739656in}}%
\pgfpathlineto{\pgfqpoint{4.331083in}{0.739656in}}%
\pgfpathlineto{\pgfqpoint{4.330519in}{0.739656in}}%
\pgfpathlineto{\pgfqpoint{4.329954in}{0.739656in}}%
\pgfpathlineto{\pgfqpoint{4.329390in}{0.739656in}}%
\pgfpathlineto{\pgfqpoint{4.328826in}{0.739656in}}%
\pgfpathlineto{\pgfqpoint{4.328261in}{0.739656in}}%
\pgfpathlineto{\pgfqpoint{4.327697in}{0.739656in}}%
\pgfpathlineto{\pgfqpoint{4.327133in}{0.739656in}}%
\pgfpathlineto{\pgfqpoint{4.326568in}{0.739656in}}%
\pgfpathlineto{\pgfqpoint{4.326004in}{0.739656in}}%
\pgfpathlineto{\pgfqpoint{4.325439in}{0.739656in}}%
\pgfpathlineto{\pgfqpoint{4.324875in}{0.739656in}}%
\pgfpathlineto{\pgfqpoint{4.324311in}{0.739656in}}%
\pgfpathlineto{\pgfqpoint{4.323746in}{0.739656in}}%
\pgfpathlineto{\pgfqpoint{4.323182in}{0.739656in}}%
\pgfpathlineto{\pgfqpoint{4.322618in}{0.739656in}}%
\pgfpathlineto{\pgfqpoint{4.322053in}{0.739656in}}%
\pgfpathlineto{\pgfqpoint{4.321489in}{0.739656in}}%
\pgfpathlineto{\pgfqpoint{4.320925in}{0.739656in}}%
\pgfpathlineto{\pgfqpoint{4.320360in}{0.739656in}}%
\pgfpathlineto{\pgfqpoint{4.319796in}{0.739656in}}%
\pgfpathlineto{\pgfqpoint{4.319232in}{0.739656in}}%
\pgfpathlineto{\pgfqpoint{4.318667in}{0.739656in}}%
\pgfpathlineto{\pgfqpoint{4.318103in}{0.739656in}}%
\pgfpathlineto{\pgfqpoint{4.317539in}{0.739656in}}%
\pgfpathlineto{\pgfqpoint{4.316974in}{0.739656in}}%
\pgfpathlineto{\pgfqpoint{4.316410in}{0.739656in}}%
\pgfpathlineto{\pgfqpoint{4.315846in}{0.739656in}}%
\pgfpathlineto{\pgfqpoint{4.315281in}{0.739656in}}%
\pgfpathlineto{\pgfqpoint{4.314717in}{0.739656in}}%
\pgfpathlineto{\pgfqpoint{4.314153in}{0.739656in}}%
\pgfpathlineto{\pgfqpoint{4.313588in}{0.739656in}}%
\pgfpathlineto{\pgfqpoint{4.313024in}{0.739656in}}%
\pgfpathlineto{\pgfqpoint{4.312460in}{0.739656in}}%
\pgfpathlineto{\pgfqpoint{4.311895in}{0.739656in}}%
\pgfpathlineto{\pgfqpoint{4.311331in}{0.739656in}}%
\pgfpathlineto{\pgfqpoint{4.310766in}{0.739656in}}%
\pgfpathlineto{\pgfqpoint{4.310202in}{0.739656in}}%
\pgfpathlineto{\pgfqpoint{4.309638in}{0.739656in}}%
\pgfpathlineto{\pgfqpoint{4.309073in}{0.739656in}}%
\pgfpathlineto{\pgfqpoint{4.308509in}{0.739656in}}%
\pgfpathlineto{\pgfqpoint{4.307945in}{0.739656in}}%
\pgfpathlineto{\pgfqpoint{4.307380in}{0.739656in}}%
\pgfpathlineto{\pgfqpoint{4.306816in}{0.739656in}}%
\pgfpathlineto{\pgfqpoint{4.306252in}{0.739656in}}%
\pgfpathlineto{\pgfqpoint{4.305687in}{0.739656in}}%
\pgfpathlineto{\pgfqpoint{4.305123in}{0.739656in}}%
\pgfpathlineto{\pgfqpoint{4.304559in}{0.739656in}}%
\pgfpathlineto{\pgfqpoint{4.303994in}{0.739656in}}%
\pgfpathlineto{\pgfqpoint{4.303430in}{0.739656in}}%
\pgfpathlineto{\pgfqpoint{4.302866in}{0.739656in}}%
\pgfpathlineto{\pgfqpoint{4.302301in}{0.739656in}}%
\pgfpathlineto{\pgfqpoint{4.301737in}{0.739656in}}%
\pgfpathlineto{\pgfqpoint{4.301173in}{0.739656in}}%
\pgfpathlineto{\pgfqpoint{4.300608in}{0.739656in}}%
\pgfpathlineto{\pgfqpoint{4.300044in}{0.739656in}}%
\pgfpathlineto{\pgfqpoint{4.299480in}{0.739656in}}%
\pgfpathlineto{\pgfqpoint{4.298915in}{0.739656in}}%
\pgfpathlineto{\pgfqpoint{4.298351in}{0.739656in}}%
\pgfpathlineto{\pgfqpoint{4.297787in}{0.739656in}}%
\pgfpathlineto{\pgfqpoint{4.297222in}{0.739656in}}%
\pgfpathlineto{\pgfqpoint{4.296658in}{0.739656in}}%
\pgfpathlineto{\pgfqpoint{4.296093in}{0.739656in}}%
\pgfpathlineto{\pgfqpoint{4.295529in}{0.739656in}}%
\pgfpathlineto{\pgfqpoint{4.294965in}{0.739656in}}%
\pgfpathlineto{\pgfqpoint{4.294400in}{0.739656in}}%
\pgfpathlineto{\pgfqpoint{4.293836in}{0.739656in}}%
\pgfpathlineto{\pgfqpoint{4.293272in}{0.739656in}}%
\pgfpathlineto{\pgfqpoint{4.292707in}{0.739656in}}%
\pgfpathlineto{\pgfqpoint{4.292143in}{0.739656in}}%
\pgfpathlineto{\pgfqpoint{4.291579in}{0.739656in}}%
\pgfpathlineto{\pgfqpoint{4.291014in}{0.739656in}}%
\pgfpathlineto{\pgfqpoint{4.290450in}{0.739656in}}%
\pgfpathlineto{\pgfqpoint{4.289886in}{0.739656in}}%
\pgfpathlineto{\pgfqpoint{4.289321in}{0.739656in}}%
\pgfpathlineto{\pgfqpoint{4.288757in}{0.739656in}}%
\pgfpathlineto{\pgfqpoint{4.288193in}{0.739656in}}%
\pgfpathlineto{\pgfqpoint{4.287628in}{0.739656in}}%
\pgfpathlineto{\pgfqpoint{4.287064in}{0.739656in}}%
\pgfpathlineto{\pgfqpoint{4.286500in}{0.739656in}}%
\pgfpathlineto{\pgfqpoint{4.285935in}{0.739656in}}%
\pgfpathlineto{\pgfqpoint{4.285371in}{0.739656in}}%
\pgfpathlineto{\pgfqpoint{4.284807in}{0.739656in}}%
\pgfpathlineto{\pgfqpoint{4.284242in}{0.739656in}}%
\pgfpathlineto{\pgfqpoint{4.283678in}{0.739656in}}%
\pgfpathlineto{\pgfqpoint{4.283114in}{0.739656in}}%
\pgfpathlineto{\pgfqpoint{4.282549in}{0.739656in}}%
\pgfpathlineto{\pgfqpoint{4.281985in}{0.739656in}}%
\pgfpathlineto{\pgfqpoint{4.281421in}{0.739656in}}%
\pgfpathlineto{\pgfqpoint{4.280856in}{0.739656in}}%
\pgfpathlineto{\pgfqpoint{4.280292in}{0.739656in}}%
\pgfpathlineto{\pgfqpoint{4.279727in}{0.739656in}}%
\pgfpathlineto{\pgfqpoint{4.279163in}{0.739656in}}%
\pgfpathlineto{\pgfqpoint{4.278599in}{0.739656in}}%
\pgfpathlineto{\pgfqpoint{4.278034in}{0.739656in}}%
\pgfpathlineto{\pgfqpoint{4.277470in}{0.739656in}}%
\pgfpathlineto{\pgfqpoint{4.276906in}{0.739656in}}%
\pgfpathlineto{\pgfqpoint{4.276341in}{0.739656in}}%
\pgfpathlineto{\pgfqpoint{4.275777in}{0.739656in}}%
\pgfpathlineto{\pgfqpoint{4.275213in}{0.739656in}}%
\pgfpathlineto{\pgfqpoint{4.274648in}{0.739656in}}%
\pgfpathlineto{\pgfqpoint{4.274084in}{0.739656in}}%
\pgfpathlineto{\pgfqpoint{4.273520in}{0.739656in}}%
\pgfpathlineto{\pgfqpoint{4.272955in}{0.739656in}}%
\pgfpathlineto{\pgfqpoint{4.272391in}{0.739656in}}%
\pgfpathlineto{\pgfqpoint{4.271827in}{0.739656in}}%
\pgfpathlineto{\pgfqpoint{4.271262in}{0.739656in}}%
\pgfpathlineto{\pgfqpoint{4.270698in}{0.739656in}}%
\pgfpathlineto{\pgfqpoint{4.270134in}{0.739656in}}%
\pgfpathlineto{\pgfqpoint{4.269569in}{0.739656in}}%
\pgfpathlineto{\pgfqpoint{4.269005in}{0.739656in}}%
\pgfpathlineto{\pgfqpoint{4.268441in}{0.739656in}}%
\pgfpathlineto{\pgfqpoint{4.267876in}{0.739656in}}%
\pgfpathlineto{\pgfqpoint{4.267312in}{0.739656in}}%
\pgfpathlineto{\pgfqpoint{4.266748in}{0.739656in}}%
\pgfpathlineto{\pgfqpoint{4.266183in}{0.739656in}}%
\pgfpathlineto{\pgfqpoint{4.265619in}{0.739656in}}%
\pgfpathlineto{\pgfqpoint{4.265054in}{0.739656in}}%
\pgfpathlineto{\pgfqpoint{4.264490in}{0.739656in}}%
\pgfpathlineto{\pgfqpoint{4.263926in}{0.739656in}}%
\pgfpathlineto{\pgfqpoint{4.263361in}{0.739656in}}%
\pgfpathlineto{\pgfqpoint{4.262797in}{0.739656in}}%
\pgfpathlineto{\pgfqpoint{4.262233in}{0.739656in}}%
\pgfpathlineto{\pgfqpoint{4.261668in}{0.739656in}}%
\pgfpathlineto{\pgfqpoint{4.261104in}{0.739656in}}%
\pgfpathlineto{\pgfqpoint{4.260540in}{0.739656in}}%
\pgfpathlineto{\pgfqpoint{4.259975in}{0.739656in}}%
\pgfpathlineto{\pgfqpoint{4.259411in}{0.739656in}}%
\pgfpathlineto{\pgfqpoint{4.258847in}{0.739656in}}%
\pgfpathlineto{\pgfqpoint{4.258282in}{0.739656in}}%
\pgfpathlineto{\pgfqpoint{4.257718in}{0.739656in}}%
\pgfpathlineto{\pgfqpoint{4.257154in}{0.739656in}}%
\pgfpathlineto{\pgfqpoint{4.256589in}{0.739656in}}%
\pgfpathlineto{\pgfqpoint{4.256025in}{0.739656in}}%
\pgfpathlineto{\pgfqpoint{4.255461in}{0.739656in}}%
\pgfpathlineto{\pgfqpoint{4.254896in}{0.739656in}}%
\pgfpathlineto{\pgfqpoint{4.254332in}{0.739656in}}%
\pgfpathlineto{\pgfqpoint{4.253768in}{0.739656in}}%
\pgfpathlineto{\pgfqpoint{4.253203in}{0.739656in}}%
\pgfpathlineto{\pgfqpoint{4.252639in}{0.739656in}}%
\pgfpathlineto{\pgfqpoint{4.252075in}{0.739656in}}%
\pgfpathlineto{\pgfqpoint{4.251510in}{0.739656in}}%
\pgfpathlineto{\pgfqpoint{4.250946in}{0.739656in}}%
\pgfpathlineto{\pgfqpoint{4.250381in}{0.739656in}}%
\pgfpathlineto{\pgfqpoint{4.249817in}{0.739656in}}%
\pgfpathlineto{\pgfqpoint{4.249253in}{0.739656in}}%
\pgfpathlineto{\pgfqpoint{4.248688in}{0.739656in}}%
\pgfpathlineto{\pgfqpoint{4.248124in}{0.739656in}}%
\pgfpathlineto{\pgfqpoint{4.247560in}{0.739656in}}%
\pgfpathlineto{\pgfqpoint{4.246995in}{0.739656in}}%
\pgfpathlineto{\pgfqpoint{4.246431in}{0.739656in}}%
\pgfpathlineto{\pgfqpoint{4.245867in}{0.739656in}}%
\pgfpathlineto{\pgfqpoint{4.245302in}{0.739656in}}%
\pgfpathlineto{\pgfqpoint{4.244738in}{0.739656in}}%
\pgfpathlineto{\pgfqpoint{4.244174in}{0.739656in}}%
\pgfpathlineto{\pgfqpoint{4.243609in}{0.739656in}}%
\pgfpathlineto{\pgfqpoint{4.243045in}{0.739656in}}%
\pgfpathlineto{\pgfqpoint{4.242481in}{0.739656in}}%
\pgfpathlineto{\pgfqpoint{4.241916in}{0.739656in}}%
\pgfpathlineto{\pgfqpoint{4.241352in}{0.739656in}}%
\pgfpathlineto{\pgfqpoint{4.240788in}{0.739656in}}%
\pgfpathlineto{\pgfqpoint{4.240223in}{0.739656in}}%
\pgfpathlineto{\pgfqpoint{4.239659in}{0.739656in}}%
\pgfpathlineto{\pgfqpoint{4.239095in}{0.739656in}}%
\pgfpathlineto{\pgfqpoint{4.238530in}{0.739656in}}%
\pgfpathlineto{\pgfqpoint{4.237966in}{0.739656in}}%
\pgfpathlineto{\pgfqpoint{4.237402in}{0.739656in}}%
\pgfpathlineto{\pgfqpoint{4.236837in}{0.739656in}}%
\pgfpathlineto{\pgfqpoint{4.236273in}{0.739656in}}%
\pgfpathlineto{\pgfqpoint{4.235708in}{0.739656in}}%
\pgfpathlineto{\pgfqpoint{4.235144in}{0.739656in}}%
\pgfpathlineto{\pgfqpoint{4.234580in}{0.739656in}}%
\pgfpathlineto{\pgfqpoint{4.234015in}{0.739656in}}%
\pgfpathlineto{\pgfqpoint{4.233451in}{0.739656in}}%
\pgfpathlineto{\pgfqpoint{4.232887in}{0.739656in}}%
\pgfpathlineto{\pgfqpoint{4.232322in}{0.739656in}}%
\pgfpathlineto{\pgfqpoint{4.231758in}{0.739656in}}%
\pgfpathlineto{\pgfqpoint{4.231194in}{0.739656in}}%
\pgfpathlineto{\pgfqpoint{4.230629in}{0.739656in}}%
\pgfpathlineto{\pgfqpoint{4.230065in}{0.739656in}}%
\pgfpathlineto{\pgfqpoint{4.229501in}{0.739656in}}%
\pgfpathlineto{\pgfqpoint{4.228936in}{0.739656in}}%
\pgfpathlineto{\pgfqpoint{4.228372in}{0.739656in}}%
\pgfpathlineto{\pgfqpoint{4.227808in}{0.739656in}}%
\pgfpathlineto{\pgfqpoint{4.227243in}{0.739656in}}%
\pgfpathlineto{\pgfqpoint{4.226679in}{0.739656in}}%
\pgfpathlineto{\pgfqpoint{4.226115in}{0.739656in}}%
\pgfpathlineto{\pgfqpoint{4.225550in}{0.739656in}}%
\pgfpathlineto{\pgfqpoint{4.224986in}{0.739656in}}%
\pgfpathlineto{\pgfqpoint{4.224422in}{0.739656in}}%
\pgfpathlineto{\pgfqpoint{4.223857in}{0.739656in}}%
\pgfpathlineto{\pgfqpoint{4.223293in}{0.739656in}}%
\pgfpathlineto{\pgfqpoint{4.222729in}{0.739656in}}%
\pgfpathlineto{\pgfqpoint{4.222164in}{0.739656in}}%
\pgfpathlineto{\pgfqpoint{4.221600in}{0.739656in}}%
\pgfpathlineto{\pgfqpoint{4.221036in}{0.739656in}}%
\pgfpathlineto{\pgfqpoint{4.220471in}{0.739656in}}%
\pgfpathlineto{\pgfqpoint{4.219907in}{0.739656in}}%
\pgfpathlineto{\pgfqpoint{4.219342in}{0.739656in}}%
\pgfpathlineto{\pgfqpoint{4.218778in}{0.739656in}}%
\pgfpathlineto{\pgfqpoint{4.218214in}{0.739656in}}%
\pgfpathlineto{\pgfqpoint{4.217649in}{0.739656in}}%
\pgfpathlineto{\pgfqpoint{4.217085in}{0.739656in}}%
\pgfpathlineto{\pgfqpoint{4.216521in}{0.739656in}}%
\pgfpathlineto{\pgfqpoint{4.215956in}{0.739656in}}%
\pgfpathlineto{\pgfqpoint{4.215392in}{0.739656in}}%
\pgfpathlineto{\pgfqpoint{4.214828in}{0.739656in}}%
\pgfpathlineto{\pgfqpoint{4.214263in}{0.739656in}}%
\pgfpathlineto{\pgfqpoint{4.213699in}{0.739656in}}%
\pgfpathlineto{\pgfqpoint{4.213135in}{0.739656in}}%
\pgfpathlineto{\pgfqpoint{4.212570in}{0.739656in}}%
\pgfpathlineto{\pgfqpoint{4.212006in}{0.739656in}}%
\pgfpathlineto{\pgfqpoint{4.211442in}{0.739656in}}%
\pgfpathlineto{\pgfqpoint{4.210877in}{0.739656in}}%
\pgfpathlineto{\pgfqpoint{4.210313in}{0.739656in}}%
\pgfpathlineto{\pgfqpoint{4.209749in}{0.739656in}}%
\pgfpathlineto{\pgfqpoint{4.209184in}{0.739656in}}%
\pgfpathlineto{\pgfqpoint{4.208620in}{0.739656in}}%
\pgfpathlineto{\pgfqpoint{4.208056in}{0.739656in}}%
\pgfpathlineto{\pgfqpoint{4.207491in}{0.739656in}}%
\pgfpathlineto{\pgfqpoint{4.206927in}{0.739656in}}%
\pgfpathlineto{\pgfqpoint{4.206363in}{0.739656in}}%
\pgfpathlineto{\pgfqpoint{4.205798in}{0.739656in}}%
\pgfpathlineto{\pgfqpoint{4.205234in}{0.739656in}}%
\pgfpathlineto{\pgfqpoint{4.204669in}{0.739656in}}%
\pgfpathlineto{\pgfqpoint{4.204105in}{0.739656in}}%
\pgfpathlineto{\pgfqpoint{4.203541in}{0.739656in}}%
\pgfpathlineto{\pgfqpoint{4.202976in}{0.739656in}}%
\pgfpathlineto{\pgfqpoint{4.202412in}{0.739656in}}%
\pgfpathlineto{\pgfqpoint{4.201848in}{0.739656in}}%
\pgfpathlineto{\pgfqpoint{4.201283in}{0.739656in}}%
\pgfpathlineto{\pgfqpoint{4.200719in}{0.739656in}}%
\pgfpathlineto{\pgfqpoint{4.200155in}{0.739656in}}%
\pgfpathlineto{\pgfqpoint{4.199590in}{0.739656in}}%
\pgfpathlineto{\pgfqpoint{4.199026in}{0.739656in}}%
\pgfpathlineto{\pgfqpoint{4.198462in}{0.739656in}}%
\pgfpathlineto{\pgfqpoint{4.197897in}{0.739656in}}%
\pgfpathlineto{\pgfqpoint{4.197333in}{0.739656in}}%
\pgfpathlineto{\pgfqpoint{4.196769in}{0.739656in}}%
\pgfpathlineto{\pgfqpoint{4.196204in}{0.739656in}}%
\pgfpathlineto{\pgfqpoint{4.195640in}{0.739656in}}%
\pgfpathlineto{\pgfqpoint{4.195076in}{0.739656in}}%
\pgfpathlineto{\pgfqpoint{4.194511in}{0.739656in}}%
\pgfpathlineto{\pgfqpoint{4.193947in}{0.739656in}}%
\pgfpathlineto{\pgfqpoint{4.193383in}{0.739656in}}%
\pgfpathlineto{\pgfqpoint{4.192818in}{0.739656in}}%
\pgfpathlineto{\pgfqpoint{4.192254in}{0.739656in}}%
\pgfpathlineto{\pgfqpoint{4.191690in}{0.739656in}}%
\pgfpathlineto{\pgfqpoint{4.191125in}{0.739656in}}%
\pgfpathlineto{\pgfqpoint{4.190561in}{0.739656in}}%
\pgfpathlineto{\pgfqpoint{4.189996in}{0.739656in}}%
\pgfpathlineto{\pgfqpoint{4.189432in}{0.739656in}}%
\pgfpathlineto{\pgfqpoint{4.188868in}{0.739656in}}%
\pgfpathlineto{\pgfqpoint{4.188303in}{0.739656in}}%
\pgfpathlineto{\pgfqpoint{4.187739in}{0.739656in}}%
\pgfpathlineto{\pgfqpoint{4.187175in}{0.739656in}}%
\pgfpathlineto{\pgfqpoint{4.186610in}{0.739656in}}%
\pgfpathlineto{\pgfqpoint{4.186046in}{0.739656in}}%
\pgfpathlineto{\pgfqpoint{4.185482in}{0.739656in}}%
\pgfpathlineto{\pgfqpoint{4.184917in}{0.739656in}}%
\pgfpathlineto{\pgfqpoint{4.184353in}{0.739656in}}%
\pgfpathlineto{\pgfqpoint{4.183789in}{0.739656in}}%
\pgfpathlineto{\pgfqpoint{4.183224in}{0.739656in}}%
\pgfpathlineto{\pgfqpoint{4.182660in}{0.739656in}}%
\pgfpathlineto{\pgfqpoint{4.182096in}{0.739656in}}%
\pgfpathlineto{\pgfqpoint{4.181531in}{0.739656in}}%
\pgfpathlineto{\pgfqpoint{4.180967in}{0.739656in}}%
\pgfpathlineto{\pgfqpoint{4.180403in}{0.739656in}}%
\pgfpathlineto{\pgfqpoint{4.179838in}{0.739656in}}%
\pgfpathlineto{\pgfqpoint{4.179274in}{0.739656in}}%
\pgfpathlineto{\pgfqpoint{4.178710in}{0.739656in}}%
\pgfpathlineto{\pgfqpoint{4.178145in}{0.739656in}}%
\pgfpathlineto{\pgfqpoint{4.177581in}{0.739656in}}%
\pgfpathlineto{\pgfqpoint{4.177017in}{0.739656in}}%
\pgfpathlineto{\pgfqpoint{4.176452in}{0.739656in}}%
\pgfpathlineto{\pgfqpoint{4.175888in}{0.739656in}}%
\pgfpathlineto{\pgfqpoint{4.175324in}{0.739656in}}%
\pgfpathlineto{\pgfqpoint{4.174759in}{0.739656in}}%
\pgfpathlineto{\pgfqpoint{4.174195in}{0.739656in}}%
\pgfpathlineto{\pgfqpoint{4.173630in}{0.739656in}}%
\pgfpathlineto{\pgfqpoint{4.173066in}{0.739656in}}%
\pgfpathlineto{\pgfqpoint{4.172502in}{0.739656in}}%
\pgfpathlineto{\pgfqpoint{4.171937in}{0.739656in}}%
\pgfpathlineto{\pgfqpoint{4.171373in}{0.739656in}}%
\pgfpathlineto{\pgfqpoint{4.170809in}{0.739656in}}%
\pgfpathlineto{\pgfqpoint{4.170244in}{0.739656in}}%
\pgfpathlineto{\pgfqpoint{4.169680in}{0.739656in}}%
\pgfpathlineto{\pgfqpoint{4.169116in}{0.739656in}}%
\pgfpathlineto{\pgfqpoint{4.168551in}{0.739656in}}%
\pgfpathlineto{\pgfqpoint{4.167987in}{0.739656in}}%
\pgfpathlineto{\pgfqpoint{4.167423in}{0.739656in}}%
\pgfpathlineto{\pgfqpoint{4.166858in}{0.739656in}}%
\pgfpathlineto{\pgfqpoint{4.166294in}{0.739656in}}%
\pgfpathlineto{\pgfqpoint{4.165730in}{0.739656in}}%
\pgfpathlineto{\pgfqpoint{4.165165in}{0.739656in}}%
\pgfpathlineto{\pgfqpoint{4.164601in}{0.739656in}}%
\pgfpathlineto{\pgfqpoint{4.164037in}{0.739656in}}%
\pgfpathlineto{\pgfqpoint{4.163472in}{0.739656in}}%
\pgfpathlineto{\pgfqpoint{4.162908in}{0.739656in}}%
\pgfpathlineto{\pgfqpoint{4.162344in}{0.739656in}}%
\pgfpathlineto{\pgfqpoint{4.161779in}{0.739656in}}%
\pgfpathlineto{\pgfqpoint{4.161215in}{0.739656in}}%
\pgfpathlineto{\pgfqpoint{4.160651in}{0.739656in}}%
\pgfpathlineto{\pgfqpoint{4.160086in}{0.739656in}}%
\pgfpathlineto{\pgfqpoint{4.159522in}{0.739656in}}%
\pgfpathlineto{\pgfqpoint{4.158957in}{0.739656in}}%
\pgfpathlineto{\pgfqpoint{4.158393in}{0.739656in}}%
\pgfpathlineto{\pgfqpoint{4.157829in}{0.739656in}}%
\pgfpathlineto{\pgfqpoint{4.157264in}{0.739656in}}%
\pgfpathlineto{\pgfqpoint{4.156700in}{0.739656in}}%
\pgfpathlineto{\pgfqpoint{4.156136in}{0.739656in}}%
\pgfpathlineto{\pgfqpoint{4.155571in}{0.739656in}}%
\pgfpathlineto{\pgfqpoint{4.155007in}{0.739656in}}%
\pgfpathlineto{\pgfqpoint{4.154443in}{0.739656in}}%
\pgfpathlineto{\pgfqpoint{4.153878in}{0.739656in}}%
\pgfpathlineto{\pgfqpoint{4.153314in}{0.739656in}}%
\pgfpathlineto{\pgfqpoint{4.152750in}{0.739656in}}%
\pgfpathlineto{\pgfqpoint{4.152185in}{0.739656in}}%
\pgfpathlineto{\pgfqpoint{4.151621in}{0.739656in}}%
\pgfpathlineto{\pgfqpoint{4.151057in}{0.739656in}}%
\pgfpathlineto{\pgfqpoint{4.150492in}{0.739656in}}%
\pgfpathlineto{\pgfqpoint{4.149928in}{0.739656in}}%
\pgfpathlineto{\pgfqpoint{4.149364in}{0.739656in}}%
\pgfpathlineto{\pgfqpoint{4.148799in}{0.739656in}}%
\pgfpathlineto{\pgfqpoint{4.148235in}{0.739656in}}%
\pgfpathlineto{\pgfqpoint{4.147671in}{0.739656in}}%
\pgfpathlineto{\pgfqpoint{4.147106in}{0.739656in}}%
\pgfpathlineto{\pgfqpoint{4.146542in}{0.739656in}}%
\pgfpathlineto{\pgfqpoint{4.145978in}{0.739656in}}%
\pgfpathlineto{\pgfqpoint{4.145413in}{0.739656in}}%
\pgfpathlineto{\pgfqpoint{4.144849in}{0.739656in}}%
\pgfpathlineto{\pgfqpoint{4.144284in}{0.739656in}}%
\pgfpathlineto{\pgfqpoint{4.143720in}{0.739656in}}%
\pgfpathlineto{\pgfqpoint{4.143156in}{0.739656in}}%
\pgfpathlineto{\pgfqpoint{4.142591in}{0.739656in}}%
\pgfpathlineto{\pgfqpoint{4.142027in}{0.739656in}}%
\pgfpathlineto{\pgfqpoint{4.141463in}{0.739656in}}%
\pgfpathlineto{\pgfqpoint{4.140898in}{0.739656in}}%
\pgfpathlineto{\pgfqpoint{4.140334in}{0.739656in}}%
\pgfpathlineto{\pgfqpoint{4.139770in}{0.739656in}}%
\pgfpathlineto{\pgfqpoint{4.139205in}{0.739656in}}%
\pgfpathlineto{\pgfqpoint{4.138641in}{0.739656in}}%
\pgfpathlineto{\pgfqpoint{4.138077in}{0.739656in}}%
\pgfpathlineto{\pgfqpoint{4.137512in}{0.739656in}}%
\pgfpathlineto{\pgfqpoint{4.136948in}{0.739656in}}%
\pgfpathlineto{\pgfqpoint{4.136384in}{0.739656in}}%
\pgfpathlineto{\pgfqpoint{4.135819in}{0.739656in}}%
\pgfpathlineto{\pgfqpoint{4.135255in}{0.739656in}}%
\pgfpathlineto{\pgfqpoint{4.134691in}{0.739656in}}%
\pgfpathlineto{\pgfqpoint{4.134126in}{0.739656in}}%
\pgfpathlineto{\pgfqpoint{4.133562in}{0.739656in}}%
\pgfpathlineto{\pgfqpoint{4.132998in}{0.739656in}}%
\pgfpathlineto{\pgfqpoint{4.132433in}{0.739656in}}%
\pgfpathlineto{\pgfqpoint{4.131869in}{0.739656in}}%
\pgfpathlineto{\pgfqpoint{4.131305in}{0.739656in}}%
\pgfpathlineto{\pgfqpoint{4.130740in}{0.739656in}}%
\pgfpathlineto{\pgfqpoint{4.130176in}{0.739656in}}%
\pgfpathlineto{\pgfqpoint{4.129612in}{0.739656in}}%
\pgfpathlineto{\pgfqpoint{4.129047in}{0.739656in}}%
\pgfpathlineto{\pgfqpoint{4.128483in}{0.739656in}}%
\pgfpathlineto{\pgfqpoint{4.127918in}{0.739656in}}%
\pgfpathlineto{\pgfqpoint{4.127354in}{0.739656in}}%
\pgfpathlineto{\pgfqpoint{4.126790in}{0.739656in}}%
\pgfpathlineto{\pgfqpoint{4.126225in}{0.739656in}}%
\pgfpathlineto{\pgfqpoint{4.125661in}{0.739656in}}%
\pgfpathlineto{\pgfqpoint{4.125097in}{0.739656in}}%
\pgfpathlineto{\pgfqpoint{4.124532in}{0.739656in}}%
\pgfpathlineto{\pgfqpoint{4.123968in}{0.739656in}}%
\pgfpathlineto{\pgfqpoint{4.123404in}{0.739656in}}%
\pgfpathlineto{\pgfqpoint{4.122839in}{0.739656in}}%
\pgfpathlineto{\pgfqpoint{4.122275in}{0.739656in}}%
\pgfpathlineto{\pgfqpoint{4.121711in}{0.739656in}}%
\pgfpathlineto{\pgfqpoint{4.121146in}{0.739656in}}%
\pgfpathlineto{\pgfqpoint{4.120582in}{0.739656in}}%
\pgfpathlineto{\pgfqpoint{4.120018in}{0.739656in}}%
\pgfpathlineto{\pgfqpoint{4.119453in}{0.739656in}}%
\pgfpathlineto{\pgfqpoint{4.118889in}{0.739656in}}%
\pgfpathlineto{\pgfqpoint{4.118325in}{0.739656in}}%
\pgfpathlineto{\pgfqpoint{4.117760in}{0.739656in}}%
\pgfpathlineto{\pgfqpoint{4.117196in}{0.739656in}}%
\pgfpathlineto{\pgfqpoint{4.116632in}{0.739656in}}%
\pgfpathlineto{\pgfqpoint{4.116067in}{0.739656in}}%
\pgfpathlineto{\pgfqpoint{4.115503in}{0.739656in}}%
\pgfpathlineto{\pgfqpoint{4.114939in}{0.739656in}}%
\pgfpathlineto{\pgfqpoint{4.114374in}{0.739656in}}%
\pgfpathlineto{\pgfqpoint{4.113810in}{0.739656in}}%
\pgfpathlineto{\pgfqpoint{4.113245in}{0.739656in}}%
\pgfpathlineto{\pgfqpoint{4.112681in}{0.739656in}}%
\pgfpathlineto{\pgfqpoint{4.112117in}{0.739656in}}%
\pgfpathlineto{\pgfqpoint{4.111552in}{0.739656in}}%
\pgfpathlineto{\pgfqpoint{4.110988in}{0.739656in}}%
\pgfpathlineto{\pgfqpoint{4.110424in}{0.739656in}}%
\pgfpathlineto{\pgfqpoint{4.109859in}{0.739656in}}%
\pgfpathlineto{\pgfqpoint{4.109295in}{0.739656in}}%
\pgfpathlineto{\pgfqpoint{4.108731in}{0.739656in}}%
\pgfpathlineto{\pgfqpoint{4.108166in}{0.739656in}}%
\pgfpathlineto{\pgfqpoint{4.107602in}{0.739656in}}%
\pgfpathlineto{\pgfqpoint{4.107038in}{0.739656in}}%
\pgfpathlineto{\pgfqpoint{4.106473in}{0.739656in}}%
\pgfpathlineto{\pgfqpoint{4.105909in}{0.739656in}}%
\pgfpathlineto{\pgfqpoint{4.105345in}{0.739656in}}%
\pgfpathlineto{\pgfqpoint{4.104780in}{0.739656in}}%
\pgfpathlineto{\pgfqpoint{4.104216in}{0.739656in}}%
\pgfpathlineto{\pgfqpoint{4.103652in}{0.739656in}}%
\pgfpathlineto{\pgfqpoint{4.103087in}{0.739656in}}%
\pgfpathlineto{\pgfqpoint{4.102523in}{0.739656in}}%
\pgfpathlineto{\pgfqpoint{4.101959in}{0.739656in}}%
\pgfpathlineto{\pgfqpoint{4.101394in}{0.739656in}}%
\pgfpathlineto{\pgfqpoint{4.100830in}{0.739656in}}%
\pgfpathlineto{\pgfqpoint{4.100266in}{0.739656in}}%
\pgfpathlineto{\pgfqpoint{4.099701in}{0.739656in}}%
\pgfpathlineto{\pgfqpoint{4.099137in}{0.739656in}}%
\pgfpathlineto{\pgfqpoint{4.098572in}{0.739656in}}%
\pgfpathlineto{\pgfqpoint{4.098008in}{0.739656in}}%
\pgfpathlineto{\pgfqpoint{4.097444in}{0.739656in}}%
\pgfpathlineto{\pgfqpoint{4.096879in}{0.739656in}}%
\pgfpathlineto{\pgfqpoint{4.096315in}{0.739656in}}%
\pgfpathlineto{\pgfqpoint{4.095751in}{0.739656in}}%
\pgfpathlineto{\pgfqpoint{4.095186in}{0.739656in}}%
\pgfpathlineto{\pgfqpoint{4.094622in}{0.739656in}}%
\pgfpathlineto{\pgfqpoint{4.094058in}{0.739656in}}%
\pgfpathlineto{\pgfqpoint{4.093493in}{0.739656in}}%
\pgfpathlineto{\pgfqpoint{4.092929in}{0.739656in}}%
\pgfpathlineto{\pgfqpoint{4.092365in}{0.739656in}}%
\pgfpathlineto{\pgfqpoint{4.091800in}{0.739656in}}%
\pgfpathlineto{\pgfqpoint{4.091236in}{0.739656in}}%
\pgfpathlineto{\pgfqpoint{4.090672in}{0.739656in}}%
\pgfpathlineto{\pgfqpoint{4.090107in}{0.739656in}}%
\pgfpathlineto{\pgfqpoint{4.089543in}{0.739656in}}%
\pgfpathlineto{\pgfqpoint{4.088979in}{0.739656in}}%
\pgfpathlineto{\pgfqpoint{4.088414in}{0.739656in}}%
\pgfpathlineto{\pgfqpoint{4.087850in}{0.739656in}}%
\pgfpathlineto{\pgfqpoint{4.087286in}{0.739656in}}%
\pgfpathlineto{\pgfqpoint{4.086721in}{0.739656in}}%
\pgfpathlineto{\pgfqpoint{4.086157in}{0.739656in}}%
\pgfpathlineto{\pgfqpoint{4.085593in}{0.739656in}}%
\pgfpathlineto{\pgfqpoint{4.085028in}{0.739656in}}%
\pgfpathlineto{\pgfqpoint{4.084464in}{0.739656in}}%
\pgfpathlineto{\pgfqpoint{4.083900in}{0.739656in}}%
\pgfpathlineto{\pgfqpoint{4.083335in}{0.739656in}}%
\pgfpathlineto{\pgfqpoint{4.082771in}{0.739656in}}%
\pgfpathlineto{\pgfqpoint{4.082206in}{0.739656in}}%
\pgfpathlineto{\pgfqpoint{4.081642in}{0.739656in}}%
\pgfpathlineto{\pgfqpoint{4.081078in}{0.739656in}}%
\pgfpathlineto{\pgfqpoint{4.080513in}{0.739656in}}%
\pgfpathlineto{\pgfqpoint{4.079949in}{0.739656in}}%
\pgfpathlineto{\pgfqpoint{4.079385in}{0.739656in}}%
\pgfpathlineto{\pgfqpoint{4.078820in}{0.739656in}}%
\pgfpathlineto{\pgfqpoint{4.078256in}{0.739656in}}%
\pgfpathlineto{\pgfqpoint{4.077692in}{0.739656in}}%
\pgfpathlineto{\pgfqpoint{4.077127in}{0.739656in}}%
\pgfpathlineto{\pgfqpoint{4.076563in}{0.739656in}}%
\pgfpathlineto{\pgfqpoint{4.075999in}{0.739656in}}%
\pgfpathlineto{\pgfqpoint{4.075434in}{0.739656in}}%
\pgfpathlineto{\pgfqpoint{4.074870in}{0.739656in}}%
\pgfpathlineto{\pgfqpoint{4.074306in}{0.739656in}}%
\pgfpathlineto{\pgfqpoint{4.073741in}{0.739656in}}%
\pgfpathlineto{\pgfqpoint{4.073177in}{0.739656in}}%
\pgfpathlineto{\pgfqpoint{4.072613in}{0.739656in}}%
\pgfpathlineto{\pgfqpoint{4.072048in}{0.739656in}}%
\pgfpathlineto{\pgfqpoint{4.071484in}{0.739656in}}%
\pgfpathlineto{\pgfqpoint{4.070920in}{0.739656in}}%
\pgfpathlineto{\pgfqpoint{4.070355in}{0.739656in}}%
\pgfpathlineto{\pgfqpoint{4.069791in}{0.739656in}}%
\pgfpathlineto{\pgfqpoint{4.069227in}{0.739656in}}%
\pgfpathlineto{\pgfqpoint{4.068662in}{0.739656in}}%
\pgfpathlineto{\pgfqpoint{4.068098in}{0.739656in}}%
\pgfpathlineto{\pgfqpoint{4.067533in}{0.739656in}}%
\pgfpathlineto{\pgfqpoint{4.066969in}{0.739656in}}%
\pgfpathlineto{\pgfqpoint{4.066405in}{0.739656in}}%
\pgfpathlineto{\pgfqpoint{4.065840in}{0.739656in}}%
\pgfpathlineto{\pgfqpoint{4.065276in}{0.739656in}}%
\pgfpathlineto{\pgfqpoint{4.064712in}{0.739656in}}%
\pgfpathlineto{\pgfqpoint{4.064147in}{0.739656in}}%
\pgfpathlineto{\pgfqpoint{4.063583in}{0.739656in}}%
\pgfpathlineto{\pgfqpoint{4.063019in}{0.739656in}}%
\pgfpathlineto{\pgfqpoint{4.062454in}{0.739656in}}%
\pgfpathlineto{\pgfqpoint{4.061890in}{0.739656in}}%
\pgfpathlineto{\pgfqpoint{4.061326in}{0.739656in}}%
\pgfpathlineto{\pgfqpoint{4.060761in}{0.739656in}}%
\pgfpathlineto{\pgfqpoint{4.060197in}{0.739656in}}%
\pgfpathlineto{\pgfqpoint{4.059633in}{0.739656in}}%
\pgfpathlineto{\pgfqpoint{4.059068in}{0.739656in}}%
\pgfpathlineto{\pgfqpoint{4.058504in}{0.739656in}}%
\pgfpathlineto{\pgfqpoint{4.057940in}{0.739656in}}%
\pgfpathlineto{\pgfqpoint{4.057375in}{0.739656in}}%
\pgfpathlineto{\pgfqpoint{4.056811in}{0.739656in}}%
\pgfpathlineto{\pgfqpoint{4.056247in}{0.739656in}}%
\pgfpathlineto{\pgfqpoint{4.055682in}{0.739656in}}%
\pgfpathlineto{\pgfqpoint{4.055118in}{0.739656in}}%
\pgfpathlineto{\pgfqpoint{4.054554in}{0.739656in}}%
\pgfpathlineto{\pgfqpoint{4.053989in}{0.739656in}}%
\pgfpathlineto{\pgfqpoint{4.053425in}{0.739656in}}%
\pgfpathlineto{\pgfqpoint{4.052860in}{0.739656in}}%
\pgfpathlineto{\pgfqpoint{4.052296in}{0.739656in}}%
\pgfpathlineto{\pgfqpoint{4.051732in}{0.739656in}}%
\pgfpathlineto{\pgfqpoint{4.051167in}{0.739656in}}%
\pgfpathlineto{\pgfqpoint{4.050603in}{0.739656in}}%
\pgfpathlineto{\pgfqpoint{4.050039in}{0.739656in}}%
\pgfpathlineto{\pgfqpoint{4.049474in}{0.739656in}}%
\pgfpathlineto{\pgfqpoint{4.048910in}{0.739656in}}%
\pgfpathlineto{\pgfqpoint{4.048346in}{0.739656in}}%
\pgfpathlineto{\pgfqpoint{4.047781in}{0.739656in}}%
\pgfpathlineto{\pgfqpoint{4.047217in}{0.739656in}}%
\pgfpathlineto{\pgfqpoint{4.046653in}{0.739656in}}%
\pgfpathlineto{\pgfqpoint{4.046088in}{0.739656in}}%
\pgfpathlineto{\pgfqpoint{4.045524in}{0.739656in}}%
\pgfpathlineto{\pgfqpoint{4.044960in}{0.739656in}}%
\pgfpathlineto{\pgfqpoint{4.044395in}{0.739656in}}%
\pgfpathlineto{\pgfqpoint{4.043831in}{0.739656in}}%
\pgfpathlineto{\pgfqpoint{4.043267in}{0.739656in}}%
\pgfpathlineto{\pgfqpoint{4.042702in}{0.739656in}}%
\pgfpathlineto{\pgfqpoint{4.042138in}{0.739656in}}%
\pgfpathlineto{\pgfqpoint{4.041574in}{0.739656in}}%
\pgfpathlineto{\pgfqpoint{4.041009in}{0.739656in}}%
\pgfpathlineto{\pgfqpoint{4.040445in}{0.739656in}}%
\pgfpathlineto{\pgfqpoint{4.039881in}{0.739656in}}%
\pgfpathlineto{\pgfqpoint{4.039316in}{0.739656in}}%
\pgfpathlineto{\pgfqpoint{4.038752in}{0.739656in}}%
\pgfpathlineto{\pgfqpoint{4.038187in}{0.739656in}}%
\pgfpathlineto{\pgfqpoint{4.037623in}{0.739656in}}%
\pgfpathlineto{\pgfqpoint{4.037059in}{0.739656in}}%
\pgfpathlineto{\pgfqpoint{4.036494in}{0.739656in}}%
\pgfpathlineto{\pgfqpoint{4.035930in}{0.739656in}}%
\pgfpathlineto{\pgfqpoint{4.035366in}{0.739656in}}%
\pgfpathlineto{\pgfqpoint{4.034801in}{0.739656in}}%
\pgfpathlineto{\pgfqpoint{4.034237in}{0.739656in}}%
\pgfpathlineto{\pgfqpoint{4.033673in}{0.739656in}}%
\pgfpathlineto{\pgfqpoint{4.033108in}{0.739656in}}%
\pgfpathlineto{\pgfqpoint{4.032544in}{0.739656in}}%
\pgfpathlineto{\pgfqpoint{4.031980in}{0.739656in}}%
\pgfpathlineto{\pgfqpoint{4.031415in}{0.739656in}}%
\pgfpathlineto{\pgfqpoint{4.030851in}{0.739656in}}%
\pgfpathlineto{\pgfqpoint{4.030287in}{0.739656in}}%
\pgfpathlineto{\pgfqpoint{4.029722in}{0.739656in}}%
\pgfpathlineto{\pgfqpoint{4.029158in}{0.739656in}}%
\pgfpathlineto{\pgfqpoint{4.028594in}{0.739656in}}%
\pgfpathlineto{\pgfqpoint{4.028029in}{0.739656in}}%
\pgfpathlineto{\pgfqpoint{4.027465in}{0.739656in}}%
\pgfpathlineto{\pgfqpoint{4.026901in}{0.739656in}}%
\pgfpathlineto{\pgfqpoint{4.026336in}{0.739656in}}%
\pgfpathlineto{\pgfqpoint{4.025772in}{0.739656in}}%
\pgfpathlineto{\pgfqpoint{4.025208in}{0.739656in}}%
\pgfpathlineto{\pgfqpoint{4.024643in}{0.739656in}}%
\pgfpathlineto{\pgfqpoint{4.024079in}{0.739656in}}%
\pgfpathlineto{\pgfqpoint{4.023515in}{0.739656in}}%
\pgfpathlineto{\pgfqpoint{4.022950in}{0.739656in}}%
\pgfpathlineto{\pgfqpoint{4.022386in}{0.739656in}}%
\pgfpathlineto{\pgfqpoint{4.021821in}{0.739656in}}%
\pgfpathlineto{\pgfqpoint{4.021257in}{0.739656in}}%
\pgfpathlineto{\pgfqpoint{4.020693in}{0.739656in}}%
\pgfpathlineto{\pgfqpoint{4.020128in}{0.739656in}}%
\pgfpathlineto{\pgfqpoint{4.019564in}{0.739656in}}%
\pgfpathlineto{\pgfqpoint{4.019000in}{0.739656in}}%
\pgfpathlineto{\pgfqpoint{4.018435in}{0.739656in}}%
\pgfpathlineto{\pgfqpoint{4.017871in}{0.739656in}}%
\pgfpathlineto{\pgfqpoint{4.017307in}{0.739656in}}%
\pgfpathlineto{\pgfqpoint{4.016742in}{0.739656in}}%
\pgfpathlineto{\pgfqpoint{4.016178in}{0.739656in}}%
\pgfpathlineto{\pgfqpoint{4.015614in}{0.739656in}}%
\pgfpathlineto{\pgfqpoint{4.015049in}{0.739656in}}%
\pgfpathlineto{\pgfqpoint{4.014485in}{0.739656in}}%
\pgfpathlineto{\pgfqpoint{4.013921in}{0.739656in}}%
\pgfpathlineto{\pgfqpoint{4.013356in}{0.739656in}}%
\pgfpathlineto{\pgfqpoint{4.012792in}{0.739656in}}%
\pgfpathlineto{\pgfqpoint{4.012228in}{0.739656in}}%
\pgfpathlineto{\pgfqpoint{4.011663in}{0.739656in}}%
\pgfpathlineto{\pgfqpoint{4.011099in}{0.739656in}}%
\pgfpathlineto{\pgfqpoint{4.010535in}{0.739656in}}%
\pgfpathlineto{\pgfqpoint{4.009970in}{0.739656in}}%
\pgfpathlineto{\pgfqpoint{4.009406in}{0.739656in}}%
\pgfpathlineto{\pgfqpoint{4.008842in}{0.739656in}}%
\pgfpathlineto{\pgfqpoint{4.008277in}{0.739656in}}%
\pgfpathlineto{\pgfqpoint{4.007713in}{0.739656in}}%
\pgfpathlineto{\pgfqpoint{4.007148in}{0.739656in}}%
\pgfpathlineto{\pgfqpoint{4.006584in}{0.739656in}}%
\pgfpathlineto{\pgfqpoint{4.006020in}{0.739656in}}%
\pgfpathlineto{\pgfqpoint{4.005455in}{0.739656in}}%
\pgfpathlineto{\pgfqpoint{4.004891in}{0.739656in}}%
\pgfpathlineto{\pgfqpoint{4.004327in}{0.739656in}}%
\pgfpathlineto{\pgfqpoint{4.003762in}{0.739656in}}%
\pgfpathlineto{\pgfqpoint{4.003198in}{0.739656in}}%
\pgfpathlineto{\pgfqpoint{4.002634in}{0.739656in}}%
\pgfpathlineto{\pgfqpoint{4.002069in}{0.739656in}}%
\pgfpathlineto{\pgfqpoint{4.001505in}{0.739656in}}%
\pgfpathlineto{\pgfqpoint{4.000941in}{0.739656in}}%
\pgfpathlineto{\pgfqpoint{4.000376in}{0.739656in}}%
\pgfpathlineto{\pgfqpoint{3.999812in}{0.739656in}}%
\pgfpathlineto{\pgfqpoint{3.999248in}{0.739656in}}%
\pgfpathlineto{\pgfqpoint{3.998683in}{0.739656in}}%
\pgfpathlineto{\pgfqpoint{3.998119in}{0.739656in}}%
\pgfpathlineto{\pgfqpoint{3.997555in}{0.739656in}}%
\pgfpathlineto{\pgfqpoint{3.996990in}{0.739656in}}%
\pgfpathlineto{\pgfqpoint{3.996426in}{0.739656in}}%
\pgfpathlineto{\pgfqpoint{3.995862in}{0.739656in}}%
\pgfpathlineto{\pgfqpoint{3.995297in}{0.739656in}}%
\pgfpathlineto{\pgfqpoint{3.994733in}{0.739656in}}%
\pgfpathlineto{\pgfqpoint{3.994169in}{0.739656in}}%
\pgfpathlineto{\pgfqpoint{3.993604in}{0.739656in}}%
\pgfpathlineto{\pgfqpoint{3.993040in}{0.739656in}}%
\pgfpathlineto{\pgfqpoint{3.992475in}{0.739656in}}%
\pgfpathlineto{\pgfqpoint{3.991911in}{0.739656in}}%
\pgfpathlineto{\pgfqpoint{3.991347in}{0.739656in}}%
\pgfpathlineto{\pgfqpoint{3.990782in}{0.739656in}}%
\pgfpathlineto{\pgfqpoint{3.990218in}{0.739656in}}%
\pgfpathlineto{\pgfqpoint{3.989654in}{0.739656in}}%
\pgfpathlineto{\pgfqpoint{3.989089in}{0.739656in}}%
\pgfpathlineto{\pgfqpoint{3.988525in}{0.739656in}}%
\pgfpathlineto{\pgfqpoint{3.987961in}{0.739656in}}%
\pgfpathlineto{\pgfqpoint{3.987396in}{0.739656in}}%
\pgfpathlineto{\pgfqpoint{3.986832in}{0.739656in}}%
\pgfpathlineto{\pgfqpoint{3.986268in}{0.739656in}}%
\pgfpathlineto{\pgfqpoint{3.985703in}{0.739656in}}%
\pgfpathlineto{\pgfqpoint{3.985139in}{0.739656in}}%
\pgfpathlineto{\pgfqpoint{3.984575in}{0.739656in}}%
\pgfpathlineto{\pgfqpoint{3.984010in}{0.739656in}}%
\pgfpathlineto{\pgfqpoint{3.983446in}{0.739656in}}%
\pgfpathlineto{\pgfqpoint{3.982882in}{0.739656in}}%
\pgfpathlineto{\pgfqpoint{3.982317in}{0.739656in}}%
\pgfpathlineto{\pgfqpoint{3.981753in}{0.739656in}}%
\pgfpathlineto{\pgfqpoint{3.981189in}{0.739656in}}%
\pgfpathlineto{\pgfqpoint{3.980624in}{0.739656in}}%
\pgfpathlineto{\pgfqpoint{3.980060in}{0.739656in}}%
\pgfpathlineto{\pgfqpoint{3.979496in}{0.739656in}}%
\pgfpathlineto{\pgfqpoint{3.978931in}{0.739656in}}%
\pgfpathlineto{\pgfqpoint{3.978367in}{0.739656in}}%
\pgfpathlineto{\pgfqpoint{3.977803in}{0.739656in}}%
\pgfpathlineto{\pgfqpoint{3.977238in}{0.739656in}}%
\pgfpathlineto{\pgfqpoint{3.976674in}{0.739656in}}%
\pgfpathlineto{\pgfqpoint{3.976109in}{0.739656in}}%
\pgfpathlineto{\pgfqpoint{3.975545in}{0.739656in}}%
\pgfpathlineto{\pgfqpoint{3.974981in}{0.739656in}}%
\pgfpathlineto{\pgfqpoint{3.974416in}{0.739656in}}%
\pgfpathlineto{\pgfqpoint{3.973852in}{0.739656in}}%
\pgfpathlineto{\pgfqpoint{3.973288in}{0.739656in}}%
\pgfpathlineto{\pgfqpoint{3.972723in}{0.739656in}}%
\pgfpathlineto{\pgfqpoint{3.972159in}{0.739656in}}%
\pgfpathlineto{\pgfqpoint{3.971595in}{0.739656in}}%
\pgfpathlineto{\pgfqpoint{3.971030in}{0.739656in}}%
\pgfpathlineto{\pgfqpoint{3.970466in}{0.739656in}}%
\pgfpathlineto{\pgfqpoint{3.969902in}{0.739656in}}%
\pgfpathlineto{\pgfqpoint{3.969337in}{0.739656in}}%
\pgfpathlineto{\pgfqpoint{3.968773in}{0.739656in}}%
\pgfpathlineto{\pgfqpoint{3.968209in}{0.739656in}}%
\pgfpathlineto{\pgfqpoint{3.967644in}{0.739656in}}%
\pgfpathlineto{\pgfqpoint{3.967080in}{0.739656in}}%
\pgfpathlineto{\pgfqpoint{3.966516in}{0.739656in}}%
\pgfpathlineto{\pgfqpoint{3.965951in}{0.739656in}}%
\pgfpathlineto{\pgfqpoint{3.965387in}{0.739656in}}%
\pgfpathlineto{\pgfqpoint{3.964823in}{0.739656in}}%
\pgfpathlineto{\pgfqpoint{3.964258in}{0.739656in}}%
\pgfpathlineto{\pgfqpoint{3.963694in}{0.739656in}}%
\pgfpathlineto{\pgfqpoint{3.963130in}{0.739656in}}%
\pgfpathlineto{\pgfqpoint{3.962565in}{0.739656in}}%
\pgfpathlineto{\pgfqpoint{3.962001in}{0.739656in}}%
\pgfpathlineto{\pgfqpoint{3.961436in}{0.739656in}}%
\pgfpathlineto{\pgfqpoint{3.960872in}{0.739656in}}%
\pgfpathlineto{\pgfqpoint{3.960308in}{0.739656in}}%
\pgfpathlineto{\pgfqpoint{3.959743in}{0.739656in}}%
\pgfpathlineto{\pgfqpoint{3.959179in}{0.739656in}}%
\pgfpathlineto{\pgfqpoint{3.958615in}{0.739656in}}%
\pgfpathlineto{\pgfqpoint{3.958050in}{0.739656in}}%
\pgfpathlineto{\pgfqpoint{3.957486in}{0.739656in}}%
\pgfpathlineto{\pgfqpoint{3.956922in}{0.739656in}}%
\pgfpathlineto{\pgfqpoint{3.956357in}{0.739656in}}%
\pgfpathlineto{\pgfqpoint{3.955793in}{0.739656in}}%
\pgfpathlineto{\pgfqpoint{3.955229in}{0.739656in}}%
\pgfpathlineto{\pgfqpoint{3.954664in}{0.739656in}}%
\pgfpathlineto{\pgfqpoint{3.954100in}{0.739656in}}%
\pgfpathlineto{\pgfqpoint{3.953536in}{0.739656in}}%
\pgfpathlineto{\pgfqpoint{3.952971in}{0.739656in}}%
\pgfpathlineto{\pgfqpoint{3.952407in}{0.739656in}}%
\pgfpathlineto{\pgfqpoint{3.951843in}{0.739656in}}%
\pgfpathlineto{\pgfqpoint{3.951278in}{0.739656in}}%
\pgfpathlineto{\pgfqpoint{3.950714in}{0.739656in}}%
\pgfpathlineto{\pgfqpoint{3.950150in}{0.739656in}}%
\pgfpathlineto{\pgfqpoint{3.949585in}{0.739656in}}%
\pgfpathlineto{\pgfqpoint{3.949021in}{0.739656in}}%
\pgfpathlineto{\pgfqpoint{3.948457in}{0.739656in}}%
\pgfpathlineto{\pgfqpoint{3.947892in}{0.739656in}}%
\pgfpathlineto{\pgfqpoint{3.947328in}{0.739656in}}%
\pgfpathlineto{\pgfqpoint{3.946763in}{0.739656in}}%
\pgfpathlineto{\pgfqpoint{3.946199in}{0.739656in}}%
\pgfpathlineto{\pgfqpoint{3.945635in}{0.739656in}}%
\pgfpathlineto{\pgfqpoint{3.945070in}{0.739656in}}%
\pgfpathlineto{\pgfqpoint{3.944506in}{0.739656in}}%
\pgfpathlineto{\pgfqpoint{3.943942in}{0.739656in}}%
\pgfpathlineto{\pgfqpoint{3.943377in}{0.739656in}}%
\pgfpathlineto{\pgfqpoint{3.942813in}{0.739656in}}%
\pgfpathlineto{\pgfqpoint{3.942249in}{0.739656in}}%
\pgfpathlineto{\pgfqpoint{3.941684in}{0.739656in}}%
\pgfpathlineto{\pgfqpoint{3.941120in}{0.739656in}}%
\pgfpathlineto{\pgfqpoint{3.940556in}{0.739656in}}%
\pgfpathlineto{\pgfqpoint{3.939991in}{0.739656in}}%
\pgfpathlineto{\pgfqpoint{3.939427in}{0.739656in}}%
\pgfpathlineto{\pgfqpoint{3.938863in}{0.739656in}}%
\pgfpathlineto{\pgfqpoint{3.938298in}{0.739656in}}%
\pgfpathlineto{\pgfqpoint{3.937734in}{0.739656in}}%
\pgfpathlineto{\pgfqpoint{3.937170in}{0.739656in}}%
\pgfpathlineto{\pgfqpoint{3.936605in}{0.739656in}}%
\pgfpathlineto{\pgfqpoint{3.936041in}{0.739656in}}%
\pgfpathlineto{\pgfqpoint{3.935477in}{0.739656in}}%
\pgfpathlineto{\pgfqpoint{3.934912in}{0.739656in}}%
\pgfpathlineto{\pgfqpoint{3.934348in}{0.739656in}}%
\pgfpathlineto{\pgfqpoint{3.933784in}{0.739656in}}%
\pgfpathlineto{\pgfqpoint{3.933219in}{0.739656in}}%
\pgfpathlineto{\pgfqpoint{3.932655in}{0.739656in}}%
\pgfpathlineto{\pgfqpoint{3.932091in}{0.739656in}}%
\pgfpathlineto{\pgfqpoint{3.931526in}{0.739656in}}%
\pgfpathlineto{\pgfqpoint{3.930962in}{0.739656in}}%
\pgfpathlineto{\pgfqpoint{3.930397in}{0.739656in}}%
\pgfpathlineto{\pgfqpoint{3.929833in}{0.739656in}}%
\pgfpathlineto{\pgfqpoint{3.929269in}{0.739656in}}%
\pgfpathlineto{\pgfqpoint{3.928704in}{0.739656in}}%
\pgfpathlineto{\pgfqpoint{3.928140in}{0.739656in}}%
\pgfpathlineto{\pgfqpoint{3.927576in}{0.739656in}}%
\pgfpathlineto{\pgfqpoint{3.927011in}{0.739656in}}%
\pgfpathlineto{\pgfqpoint{3.926447in}{0.739656in}}%
\pgfpathlineto{\pgfqpoint{3.925883in}{0.739656in}}%
\pgfpathlineto{\pgfqpoint{3.925318in}{0.739656in}}%
\pgfpathlineto{\pgfqpoint{3.924754in}{0.739656in}}%
\pgfpathlineto{\pgfqpoint{3.924190in}{0.739656in}}%
\pgfpathlineto{\pgfqpoint{3.923625in}{0.739656in}}%
\pgfpathlineto{\pgfqpoint{3.923061in}{0.739656in}}%
\pgfpathlineto{\pgfqpoint{3.922497in}{0.739656in}}%
\pgfpathlineto{\pgfqpoint{3.921932in}{0.739656in}}%
\pgfpathlineto{\pgfqpoint{3.921368in}{0.739656in}}%
\pgfpathlineto{\pgfqpoint{3.920804in}{0.739656in}}%
\pgfpathlineto{\pgfqpoint{3.920239in}{0.739656in}}%
\pgfpathlineto{\pgfqpoint{3.919675in}{0.739656in}}%
\pgfpathlineto{\pgfqpoint{3.919111in}{0.739656in}}%
\pgfpathlineto{\pgfqpoint{3.918546in}{0.739656in}}%
\pgfpathlineto{\pgfqpoint{3.917982in}{0.739656in}}%
\pgfpathlineto{\pgfqpoint{3.917418in}{0.739656in}}%
\pgfpathlineto{\pgfqpoint{3.916853in}{0.739656in}}%
\pgfpathlineto{\pgfqpoint{3.916289in}{0.739656in}}%
\pgfpathlineto{\pgfqpoint{3.915724in}{0.739656in}}%
\pgfpathlineto{\pgfqpoint{3.915160in}{0.739656in}}%
\pgfpathlineto{\pgfqpoint{3.914596in}{0.739656in}}%
\pgfpathlineto{\pgfqpoint{3.914031in}{0.739656in}}%
\pgfpathlineto{\pgfqpoint{3.913467in}{0.739656in}}%
\pgfpathlineto{\pgfqpoint{3.912903in}{0.739656in}}%
\pgfpathlineto{\pgfqpoint{3.912338in}{0.739656in}}%
\pgfpathlineto{\pgfqpoint{3.911774in}{0.739656in}}%
\pgfpathlineto{\pgfqpoint{3.911210in}{0.739656in}}%
\pgfpathlineto{\pgfqpoint{3.910645in}{0.739656in}}%
\pgfpathlineto{\pgfqpoint{3.910081in}{0.739656in}}%
\pgfpathlineto{\pgfqpoint{3.909517in}{0.739656in}}%
\pgfpathlineto{\pgfqpoint{3.908952in}{0.739656in}}%
\pgfpathlineto{\pgfqpoint{3.908388in}{0.739656in}}%
\pgfpathlineto{\pgfqpoint{3.907824in}{0.739656in}}%
\pgfpathlineto{\pgfqpoint{3.907259in}{0.739656in}}%
\pgfpathlineto{\pgfqpoint{3.906695in}{0.739656in}}%
\pgfpathlineto{\pgfqpoint{3.906131in}{0.739656in}}%
\pgfpathlineto{\pgfqpoint{3.905566in}{0.739656in}}%
\pgfpathlineto{\pgfqpoint{3.905002in}{0.739656in}}%
\pgfpathlineto{\pgfqpoint{3.904438in}{0.739656in}}%
\pgfpathlineto{\pgfqpoint{3.903873in}{0.739656in}}%
\pgfpathlineto{\pgfqpoint{3.903309in}{0.739656in}}%
\pgfpathlineto{\pgfqpoint{3.902745in}{0.739656in}}%
\pgfpathlineto{\pgfqpoint{3.902180in}{0.739656in}}%
\pgfpathlineto{\pgfqpoint{3.901616in}{0.739656in}}%
\pgfpathlineto{\pgfqpoint{3.901051in}{0.739656in}}%
\pgfpathlineto{\pgfqpoint{3.900487in}{0.739656in}}%
\pgfpathlineto{\pgfqpoint{3.899923in}{0.739656in}}%
\pgfpathlineto{\pgfqpoint{3.899358in}{0.739656in}}%
\pgfpathlineto{\pgfqpoint{3.898794in}{0.739656in}}%
\pgfpathlineto{\pgfqpoint{3.898230in}{0.739656in}}%
\pgfpathlineto{\pgfqpoint{3.897665in}{0.739656in}}%
\pgfpathlineto{\pgfqpoint{3.897101in}{0.739656in}}%
\pgfpathlineto{\pgfqpoint{3.896537in}{0.739656in}}%
\pgfpathlineto{\pgfqpoint{3.895972in}{0.739656in}}%
\pgfpathlineto{\pgfqpoint{3.895408in}{0.739656in}}%
\pgfpathlineto{\pgfqpoint{3.894844in}{0.739656in}}%
\pgfpathlineto{\pgfqpoint{3.894279in}{0.739656in}}%
\pgfpathlineto{\pgfqpoint{3.893715in}{0.739656in}}%
\pgfpathlineto{\pgfqpoint{3.893151in}{0.739656in}}%
\pgfpathlineto{\pgfqpoint{3.892586in}{0.739656in}}%
\pgfpathlineto{\pgfqpoint{3.892022in}{0.739656in}}%
\pgfpathlineto{\pgfqpoint{3.891458in}{0.739656in}}%
\pgfpathlineto{\pgfqpoint{3.890893in}{0.739656in}}%
\pgfpathlineto{\pgfqpoint{3.890329in}{0.739656in}}%
\pgfpathlineto{\pgfqpoint{3.889765in}{0.739656in}}%
\pgfpathlineto{\pgfqpoint{3.889200in}{0.739656in}}%
\pgfpathlineto{\pgfqpoint{3.888636in}{0.739656in}}%
\pgfpathlineto{\pgfqpoint{3.888072in}{0.739656in}}%
\pgfpathlineto{\pgfqpoint{3.887507in}{0.739656in}}%
\pgfpathlineto{\pgfqpoint{3.886943in}{0.739656in}}%
\pgfpathlineto{\pgfqpoint{3.886379in}{0.739656in}}%
\pgfpathlineto{\pgfqpoint{3.885814in}{0.739656in}}%
\pgfpathlineto{\pgfqpoint{3.885250in}{0.739656in}}%
\pgfpathlineto{\pgfqpoint{3.884685in}{0.739656in}}%
\pgfpathlineto{\pgfqpoint{3.884121in}{0.739656in}}%
\pgfpathlineto{\pgfqpoint{3.883557in}{0.739656in}}%
\pgfpathlineto{\pgfqpoint{3.882992in}{0.739656in}}%
\pgfpathlineto{\pgfqpoint{3.882428in}{0.739656in}}%
\pgfpathlineto{\pgfqpoint{3.881864in}{0.739656in}}%
\pgfpathlineto{\pgfqpoint{3.881299in}{0.739656in}}%
\pgfpathlineto{\pgfqpoint{3.880735in}{0.739656in}}%
\pgfpathlineto{\pgfqpoint{3.880171in}{0.739656in}}%
\pgfpathlineto{\pgfqpoint{3.879606in}{0.739656in}}%
\pgfpathlineto{\pgfqpoint{3.879042in}{0.739656in}}%
\pgfpathlineto{\pgfqpoint{3.878478in}{0.739656in}}%
\pgfpathlineto{\pgfqpoint{3.877913in}{0.739656in}}%
\pgfpathlineto{\pgfqpoint{3.877349in}{0.739656in}}%
\pgfpathlineto{\pgfqpoint{3.876785in}{0.739656in}}%
\pgfpathlineto{\pgfqpoint{3.876220in}{0.739656in}}%
\pgfpathlineto{\pgfqpoint{3.875656in}{0.739656in}}%
\pgfpathlineto{\pgfqpoint{3.875092in}{0.739656in}}%
\pgfpathlineto{\pgfqpoint{3.874527in}{0.739656in}}%
\pgfpathlineto{\pgfqpoint{3.873963in}{0.739656in}}%
\pgfpathlineto{\pgfqpoint{3.873399in}{0.739656in}}%
\pgfpathlineto{\pgfqpoint{3.872834in}{0.739656in}}%
\pgfpathlineto{\pgfqpoint{3.872270in}{0.739656in}}%
\pgfpathlineto{\pgfqpoint{3.871706in}{0.739656in}}%
\pgfpathlineto{\pgfqpoint{3.871141in}{0.739656in}}%
\pgfpathlineto{\pgfqpoint{3.870577in}{0.739656in}}%
\pgfpathlineto{\pgfqpoint{3.870012in}{0.739656in}}%
\pgfpathlineto{\pgfqpoint{3.869448in}{0.739656in}}%
\pgfpathlineto{\pgfqpoint{3.868884in}{0.739656in}}%
\pgfpathlineto{\pgfqpoint{3.868319in}{0.739656in}}%
\pgfpathlineto{\pgfqpoint{3.867755in}{0.739656in}}%
\pgfpathlineto{\pgfqpoint{3.867191in}{0.739656in}}%
\pgfpathlineto{\pgfqpoint{3.866626in}{0.739656in}}%
\pgfpathlineto{\pgfqpoint{3.866062in}{0.739656in}}%
\pgfpathlineto{\pgfqpoint{3.865498in}{0.739656in}}%
\pgfpathlineto{\pgfqpoint{3.864933in}{0.739656in}}%
\pgfpathlineto{\pgfqpoint{3.864369in}{0.739656in}}%
\pgfpathlineto{\pgfqpoint{3.863805in}{0.739656in}}%
\pgfpathlineto{\pgfqpoint{3.863240in}{0.739656in}}%
\pgfpathlineto{\pgfqpoint{3.862676in}{0.739656in}}%
\pgfpathlineto{\pgfqpoint{3.862112in}{0.739656in}}%
\pgfpathlineto{\pgfqpoint{3.861547in}{0.739656in}}%
\pgfpathlineto{\pgfqpoint{3.860983in}{0.739656in}}%
\pgfpathlineto{\pgfqpoint{3.860419in}{0.739656in}}%
\pgfpathlineto{\pgfqpoint{3.859854in}{0.739656in}}%
\pgfpathlineto{\pgfqpoint{3.859290in}{0.739656in}}%
\pgfpathlineto{\pgfqpoint{3.858726in}{0.739656in}}%
\pgfpathlineto{\pgfqpoint{3.858161in}{0.739656in}}%
\pgfpathlineto{\pgfqpoint{3.857597in}{0.739656in}}%
\pgfpathlineto{\pgfqpoint{3.857033in}{0.739656in}}%
\pgfpathlineto{\pgfqpoint{3.856468in}{0.739656in}}%
\pgfpathlineto{\pgfqpoint{3.855904in}{0.739656in}}%
\pgfpathlineto{\pgfqpoint{3.855339in}{0.739656in}}%
\pgfpathlineto{\pgfqpoint{3.854775in}{0.739656in}}%
\pgfpathlineto{\pgfqpoint{3.854211in}{0.739656in}}%
\pgfpathlineto{\pgfqpoint{3.853646in}{0.739656in}}%
\pgfpathlineto{\pgfqpoint{3.853082in}{0.739656in}}%
\pgfpathlineto{\pgfqpoint{3.852518in}{0.739656in}}%
\pgfpathlineto{\pgfqpoint{3.851953in}{0.739656in}}%
\pgfpathlineto{\pgfqpoint{3.851389in}{0.739656in}}%
\pgfpathlineto{\pgfqpoint{3.850825in}{0.739656in}}%
\pgfpathlineto{\pgfqpoint{3.850260in}{0.739656in}}%
\pgfpathlineto{\pgfqpoint{3.849696in}{0.739656in}}%
\pgfpathlineto{\pgfqpoint{3.849132in}{0.739656in}}%
\pgfpathlineto{\pgfqpoint{3.848567in}{0.739656in}}%
\pgfpathlineto{\pgfqpoint{3.848003in}{0.739656in}}%
\pgfpathlineto{\pgfqpoint{3.847439in}{0.739656in}}%
\pgfpathlineto{\pgfqpoint{3.846874in}{0.739656in}}%
\pgfpathlineto{\pgfqpoint{3.846310in}{0.739656in}}%
\pgfpathlineto{\pgfqpoint{3.845746in}{0.739656in}}%
\pgfpathlineto{\pgfqpoint{3.845181in}{0.739656in}}%
\pgfpathlineto{\pgfqpoint{3.844617in}{0.739656in}}%
\pgfpathlineto{\pgfqpoint{3.844053in}{0.739656in}}%
\pgfpathlineto{\pgfqpoint{3.843488in}{0.739656in}}%
\pgfpathlineto{\pgfqpoint{3.842924in}{0.739656in}}%
\pgfpathlineto{\pgfqpoint{3.842360in}{0.739656in}}%
\pgfpathlineto{\pgfqpoint{3.841795in}{0.739656in}}%
\pgfpathlineto{\pgfqpoint{3.841231in}{0.739656in}}%
\pgfpathlineto{\pgfqpoint{3.840667in}{0.739656in}}%
\pgfpathlineto{\pgfqpoint{3.840102in}{0.739656in}}%
\pgfpathlineto{\pgfqpoint{3.839538in}{0.739656in}}%
\pgfpathlineto{\pgfqpoint{3.838973in}{0.739656in}}%
\pgfpathlineto{\pgfqpoint{3.838409in}{0.739656in}}%
\pgfpathlineto{\pgfqpoint{3.837845in}{0.739656in}}%
\pgfpathlineto{\pgfqpoint{3.837280in}{0.739656in}}%
\pgfpathlineto{\pgfqpoint{3.836716in}{0.739656in}}%
\pgfpathlineto{\pgfqpoint{3.836152in}{0.739656in}}%
\pgfpathlineto{\pgfqpoint{3.835587in}{0.739656in}}%
\pgfpathlineto{\pgfqpoint{3.835023in}{0.739656in}}%
\pgfpathlineto{\pgfqpoint{3.834459in}{0.739656in}}%
\pgfpathlineto{\pgfqpoint{3.833894in}{0.739656in}}%
\pgfpathlineto{\pgfqpoint{3.833330in}{0.739656in}}%
\pgfpathlineto{\pgfqpoint{3.832766in}{0.739656in}}%
\pgfpathlineto{\pgfqpoint{3.832201in}{0.739656in}}%
\pgfpathlineto{\pgfqpoint{3.831637in}{0.739656in}}%
\pgfpathlineto{\pgfqpoint{3.831073in}{0.739656in}}%
\pgfpathlineto{\pgfqpoint{3.830508in}{0.739656in}}%
\pgfpathlineto{\pgfqpoint{3.829944in}{0.739656in}}%
\pgfpathlineto{\pgfqpoint{3.829380in}{0.739656in}}%
\pgfpathlineto{\pgfqpoint{3.828815in}{0.739656in}}%
\pgfpathlineto{\pgfqpoint{3.828251in}{0.739656in}}%
\pgfpathlineto{\pgfqpoint{3.827687in}{0.739656in}}%
\pgfpathlineto{\pgfqpoint{3.827122in}{0.739656in}}%
\pgfpathlineto{\pgfqpoint{3.826558in}{0.739656in}}%
\pgfpathlineto{\pgfqpoint{3.825994in}{0.739656in}}%
\pgfpathlineto{\pgfqpoint{3.825429in}{0.739656in}}%
\pgfpathlineto{\pgfqpoint{3.824865in}{0.739656in}}%
\pgfpathlineto{\pgfqpoint{3.824300in}{0.739656in}}%
\pgfpathlineto{\pgfqpoint{3.823736in}{0.739656in}}%
\pgfpathlineto{\pgfqpoint{3.823172in}{0.739656in}}%
\pgfpathlineto{\pgfqpoint{3.822607in}{0.739656in}}%
\pgfpathlineto{\pgfqpoint{3.822043in}{0.739656in}}%
\pgfpathlineto{\pgfqpoint{3.821479in}{0.739656in}}%
\pgfpathlineto{\pgfqpoint{3.820914in}{0.739656in}}%
\pgfpathlineto{\pgfqpoint{3.820350in}{0.739656in}}%
\pgfpathlineto{\pgfqpoint{3.819786in}{0.739656in}}%
\pgfpathlineto{\pgfqpoint{3.819221in}{0.739656in}}%
\pgfpathlineto{\pgfqpoint{3.818657in}{0.739656in}}%
\pgfpathlineto{\pgfqpoint{3.818093in}{0.739656in}}%
\pgfpathlineto{\pgfqpoint{3.817528in}{0.739656in}}%
\pgfpathlineto{\pgfqpoint{3.816964in}{0.739656in}}%
\pgfpathlineto{\pgfqpoint{3.816400in}{0.739656in}}%
\pgfpathlineto{\pgfqpoint{3.815835in}{0.739656in}}%
\pgfpathlineto{\pgfqpoint{3.815271in}{0.739656in}}%
\pgfpathlineto{\pgfqpoint{3.814707in}{0.739656in}}%
\pgfpathlineto{\pgfqpoint{3.814142in}{0.739656in}}%
\pgfpathlineto{\pgfqpoint{3.813578in}{0.739656in}}%
\pgfpathlineto{\pgfqpoint{3.813014in}{0.739656in}}%
\pgfpathlineto{\pgfqpoint{3.812449in}{0.739656in}}%
\pgfpathlineto{\pgfqpoint{3.811885in}{0.739656in}}%
\pgfpathlineto{\pgfqpoint{3.811321in}{0.739656in}}%
\pgfpathlineto{\pgfqpoint{3.810756in}{0.739656in}}%
\pgfpathlineto{\pgfqpoint{3.810192in}{0.739656in}}%
\pgfpathlineto{\pgfqpoint{3.809627in}{0.739656in}}%
\pgfpathlineto{\pgfqpoint{3.809063in}{0.739656in}}%
\pgfpathlineto{\pgfqpoint{3.808499in}{0.739656in}}%
\pgfpathlineto{\pgfqpoint{3.807934in}{0.739656in}}%
\pgfpathlineto{\pgfqpoint{3.807370in}{0.739656in}}%
\pgfpathlineto{\pgfqpoint{3.806806in}{0.739656in}}%
\pgfpathlineto{\pgfqpoint{3.806241in}{0.739656in}}%
\pgfpathlineto{\pgfqpoint{3.805677in}{0.739656in}}%
\pgfpathlineto{\pgfqpoint{3.805113in}{0.739656in}}%
\pgfpathlineto{\pgfqpoint{3.804548in}{0.739656in}}%
\pgfpathlineto{\pgfqpoint{3.803984in}{0.739656in}}%
\pgfpathlineto{\pgfqpoint{3.803420in}{0.739656in}}%
\pgfpathlineto{\pgfqpoint{3.802855in}{0.739656in}}%
\pgfpathlineto{\pgfqpoint{3.802291in}{0.739656in}}%
\pgfpathlineto{\pgfqpoint{3.801727in}{0.739656in}}%
\pgfpathlineto{\pgfqpoint{3.801162in}{0.739656in}}%
\pgfpathlineto{\pgfqpoint{3.800598in}{0.739656in}}%
\pgfpathlineto{\pgfqpoint{3.800034in}{0.739656in}}%
\pgfpathlineto{\pgfqpoint{3.799469in}{0.739656in}}%
\pgfpathlineto{\pgfqpoint{3.798905in}{0.739656in}}%
\pgfpathlineto{\pgfqpoint{3.798341in}{0.739656in}}%
\pgfpathlineto{\pgfqpoint{3.797776in}{0.739656in}}%
\pgfpathlineto{\pgfqpoint{3.797212in}{0.739656in}}%
\pgfpathlineto{\pgfqpoint{3.796648in}{0.739656in}}%
\pgfpathlineto{\pgfqpoint{3.796083in}{0.739656in}}%
\pgfpathlineto{\pgfqpoint{3.795519in}{0.739656in}}%
\pgfpathlineto{\pgfqpoint{3.794954in}{0.739656in}}%
\pgfpathlineto{\pgfqpoint{3.794390in}{0.739656in}}%
\pgfpathlineto{\pgfqpoint{3.793826in}{0.739656in}}%
\pgfpathlineto{\pgfqpoint{3.793261in}{0.739656in}}%
\pgfpathlineto{\pgfqpoint{3.792697in}{0.739656in}}%
\pgfpathlineto{\pgfqpoint{3.792133in}{0.739656in}}%
\pgfpathlineto{\pgfqpoint{3.791568in}{0.739656in}}%
\pgfpathlineto{\pgfqpoint{3.791004in}{0.739656in}}%
\pgfpathlineto{\pgfqpoint{3.790440in}{0.739656in}}%
\pgfpathlineto{\pgfqpoint{3.789875in}{0.739656in}}%
\pgfpathlineto{\pgfqpoint{3.789311in}{0.739656in}}%
\pgfpathlineto{\pgfqpoint{3.788747in}{0.739656in}}%
\pgfpathlineto{\pgfqpoint{3.788182in}{0.739656in}}%
\pgfpathlineto{\pgfqpoint{3.787618in}{0.739656in}}%
\pgfpathlineto{\pgfqpoint{3.787054in}{0.739656in}}%
\pgfpathlineto{\pgfqpoint{3.786489in}{0.739656in}}%
\pgfpathlineto{\pgfqpoint{3.785925in}{0.739656in}}%
\pgfpathlineto{\pgfqpoint{3.785361in}{0.739656in}}%
\pgfpathlineto{\pgfqpoint{3.784796in}{0.739656in}}%
\pgfpathlineto{\pgfqpoint{3.784232in}{0.739656in}}%
\pgfpathlineto{\pgfqpoint{3.783668in}{0.739656in}}%
\pgfpathlineto{\pgfqpoint{3.783103in}{0.739656in}}%
\pgfpathlineto{\pgfqpoint{3.782539in}{0.739656in}}%
\pgfpathlineto{\pgfqpoint{3.781975in}{0.739656in}}%
\pgfpathlineto{\pgfqpoint{3.781410in}{0.739656in}}%
\pgfpathlineto{\pgfqpoint{3.780846in}{0.739656in}}%
\pgfpathlineto{\pgfqpoint{3.780282in}{0.739656in}}%
\pgfpathlineto{\pgfqpoint{3.779717in}{0.739656in}}%
\pgfpathlineto{\pgfqpoint{3.779153in}{0.739656in}}%
\pgfpathlineto{\pgfqpoint{3.778588in}{0.739656in}}%
\pgfpathlineto{\pgfqpoint{3.778024in}{0.739656in}}%
\pgfpathlineto{\pgfqpoint{3.777460in}{0.739656in}}%
\pgfpathlineto{\pgfqpoint{3.776895in}{0.739656in}}%
\pgfpathlineto{\pgfqpoint{3.776331in}{0.739656in}}%
\pgfpathlineto{\pgfqpoint{3.775767in}{0.739656in}}%
\pgfpathlineto{\pgfqpoint{3.775202in}{0.739656in}}%
\pgfpathlineto{\pgfqpoint{3.774638in}{0.739656in}}%
\pgfpathlineto{\pgfqpoint{3.774074in}{0.739656in}}%
\pgfpathlineto{\pgfqpoint{3.773509in}{0.739656in}}%
\pgfpathlineto{\pgfqpoint{3.772945in}{0.739656in}}%
\pgfpathlineto{\pgfqpoint{3.772381in}{0.739656in}}%
\pgfpathlineto{\pgfqpoint{3.771816in}{0.739656in}}%
\pgfpathlineto{\pgfqpoint{3.771252in}{0.739656in}}%
\pgfpathlineto{\pgfqpoint{3.770688in}{0.739656in}}%
\pgfpathlineto{\pgfqpoint{3.770123in}{0.739656in}}%
\pgfpathlineto{\pgfqpoint{3.769559in}{0.739656in}}%
\pgfpathlineto{\pgfqpoint{3.768995in}{0.739656in}}%
\pgfpathlineto{\pgfqpoint{3.768430in}{0.739656in}}%
\pgfpathlineto{\pgfqpoint{3.767866in}{0.739656in}}%
\pgfpathlineto{\pgfqpoint{3.767302in}{0.739656in}}%
\pgfpathlineto{\pgfqpoint{3.766737in}{0.739656in}}%
\pgfpathlineto{\pgfqpoint{3.766173in}{0.739656in}}%
\pgfpathlineto{\pgfqpoint{3.765609in}{0.739656in}}%
\pgfpathlineto{\pgfqpoint{3.765044in}{0.739656in}}%
\pgfpathlineto{\pgfqpoint{3.764480in}{0.739656in}}%
\pgfpathlineto{\pgfqpoint{3.763915in}{0.739656in}}%
\pgfpathlineto{\pgfqpoint{3.763351in}{0.739656in}}%
\pgfpathlineto{\pgfqpoint{3.762787in}{0.739656in}}%
\pgfpathlineto{\pgfqpoint{3.762222in}{0.739656in}}%
\pgfpathlineto{\pgfqpoint{3.761658in}{0.739656in}}%
\pgfpathlineto{\pgfqpoint{3.761094in}{0.739656in}}%
\pgfpathlineto{\pgfqpoint{3.760529in}{0.739656in}}%
\pgfpathlineto{\pgfqpoint{3.759965in}{0.739656in}}%
\pgfpathlineto{\pgfqpoint{3.759401in}{0.739656in}}%
\pgfpathlineto{\pgfqpoint{3.758836in}{0.739656in}}%
\pgfpathlineto{\pgfqpoint{3.758272in}{0.739656in}}%
\pgfpathlineto{\pgfqpoint{3.757708in}{0.739656in}}%
\pgfpathlineto{\pgfqpoint{3.757143in}{0.739656in}}%
\pgfpathlineto{\pgfqpoint{3.756579in}{0.739656in}}%
\pgfpathlineto{\pgfqpoint{3.756015in}{0.739656in}}%
\pgfpathlineto{\pgfqpoint{3.755450in}{0.739656in}}%
\pgfpathlineto{\pgfqpoint{3.754886in}{0.739656in}}%
\pgfpathlineto{\pgfqpoint{3.754322in}{0.739656in}}%
\pgfpathlineto{\pgfqpoint{3.753757in}{0.739656in}}%
\pgfpathlineto{\pgfqpoint{3.753193in}{0.739656in}}%
\pgfpathlineto{\pgfqpoint{3.752629in}{0.739656in}}%
\pgfpathlineto{\pgfqpoint{3.752064in}{0.739656in}}%
\pgfpathlineto{\pgfqpoint{3.751500in}{0.739656in}}%
\pgfpathlineto{\pgfqpoint{3.750936in}{0.739656in}}%
\pgfpathlineto{\pgfqpoint{3.750371in}{0.739656in}}%
\pgfpathlineto{\pgfqpoint{3.749807in}{0.739656in}}%
\pgfpathlineto{\pgfqpoint{3.749242in}{0.739656in}}%
\pgfpathlineto{\pgfqpoint{3.748678in}{0.739656in}}%
\pgfpathlineto{\pgfqpoint{3.748114in}{0.739656in}}%
\pgfpathlineto{\pgfqpoint{3.747549in}{0.739656in}}%
\pgfpathlineto{\pgfqpoint{3.746985in}{0.739656in}}%
\pgfpathlineto{\pgfqpoint{3.746421in}{0.739656in}}%
\pgfpathlineto{\pgfqpoint{3.745856in}{0.739656in}}%
\pgfpathlineto{\pgfqpoint{3.745292in}{0.739656in}}%
\pgfpathlineto{\pgfqpoint{3.744728in}{0.739656in}}%
\pgfpathlineto{\pgfqpoint{3.744163in}{0.739656in}}%
\pgfpathlineto{\pgfqpoint{3.743599in}{0.739656in}}%
\pgfpathlineto{\pgfqpoint{3.743035in}{0.739656in}}%
\pgfpathlineto{\pgfqpoint{3.742470in}{0.739656in}}%
\pgfpathlineto{\pgfqpoint{3.741906in}{0.739656in}}%
\pgfpathlineto{\pgfqpoint{3.741342in}{0.739656in}}%
\pgfpathlineto{\pgfqpoint{3.740777in}{0.739656in}}%
\pgfpathlineto{\pgfqpoint{3.740213in}{0.739656in}}%
\pgfpathlineto{\pgfqpoint{3.739649in}{0.739656in}}%
\pgfpathlineto{\pgfqpoint{3.739084in}{0.739656in}}%
\pgfpathlineto{\pgfqpoint{3.738520in}{0.739656in}}%
\pgfpathlineto{\pgfqpoint{3.737956in}{0.739656in}}%
\pgfpathlineto{\pgfqpoint{3.737391in}{0.739656in}}%
\pgfpathlineto{\pgfqpoint{3.736827in}{0.739656in}}%
\pgfpathlineto{\pgfqpoint{3.736263in}{0.739656in}}%
\pgfpathlineto{\pgfqpoint{3.735698in}{0.739656in}}%
\pgfpathlineto{\pgfqpoint{3.735134in}{0.739656in}}%
\pgfpathlineto{\pgfqpoint{3.734570in}{0.739656in}}%
\pgfpathlineto{\pgfqpoint{3.734005in}{0.739656in}}%
\pgfpathlineto{\pgfqpoint{3.733441in}{0.739656in}}%
\pgfpathlineto{\pgfqpoint{3.732876in}{0.739656in}}%
\pgfpathlineto{\pgfqpoint{3.732312in}{0.739656in}}%
\pgfpathlineto{\pgfqpoint{3.731748in}{0.739656in}}%
\pgfpathlineto{\pgfqpoint{3.731183in}{0.739656in}}%
\pgfpathlineto{\pgfqpoint{3.730619in}{0.739656in}}%
\pgfpathlineto{\pgfqpoint{3.730055in}{0.739656in}}%
\pgfpathlineto{\pgfqpoint{3.729490in}{0.739656in}}%
\pgfpathlineto{\pgfqpoint{3.728926in}{0.739656in}}%
\pgfpathlineto{\pgfqpoint{3.728362in}{0.739656in}}%
\pgfpathlineto{\pgfqpoint{3.727797in}{0.739656in}}%
\pgfpathlineto{\pgfqpoint{3.727233in}{0.739656in}}%
\pgfpathlineto{\pgfqpoint{3.726669in}{0.739656in}}%
\pgfpathlineto{\pgfqpoint{3.726104in}{0.739656in}}%
\pgfpathlineto{\pgfqpoint{3.725540in}{0.739656in}}%
\pgfpathlineto{\pgfqpoint{3.724976in}{0.739656in}}%
\pgfpathlineto{\pgfqpoint{3.724411in}{0.739656in}}%
\pgfpathlineto{\pgfqpoint{3.723847in}{0.739656in}}%
\pgfpathlineto{\pgfqpoint{3.723283in}{0.739656in}}%
\pgfpathlineto{\pgfqpoint{3.722718in}{0.739656in}}%
\pgfpathlineto{\pgfqpoint{3.722154in}{0.739656in}}%
\pgfpathlineto{\pgfqpoint{3.721590in}{0.739656in}}%
\pgfpathlineto{\pgfqpoint{3.721025in}{0.739656in}}%
\pgfpathlineto{\pgfqpoint{3.720461in}{0.739656in}}%
\pgfpathlineto{\pgfqpoint{3.719897in}{0.739656in}}%
\pgfpathlineto{\pgfqpoint{3.719332in}{0.739656in}}%
\pgfpathlineto{\pgfqpoint{3.718768in}{0.739656in}}%
\pgfpathlineto{\pgfqpoint{3.718203in}{0.739656in}}%
\pgfpathlineto{\pgfqpoint{3.717639in}{0.739656in}}%
\pgfpathlineto{\pgfqpoint{3.717075in}{0.739656in}}%
\pgfpathlineto{\pgfqpoint{3.716510in}{0.739656in}}%
\pgfpathlineto{\pgfqpoint{3.715946in}{0.739656in}}%
\pgfpathlineto{\pgfqpoint{3.715382in}{0.739656in}}%
\pgfpathlineto{\pgfqpoint{3.714817in}{0.739656in}}%
\pgfpathlineto{\pgfqpoint{3.714253in}{0.739656in}}%
\pgfpathlineto{\pgfqpoint{3.713689in}{0.739656in}}%
\pgfpathlineto{\pgfqpoint{3.713124in}{0.739656in}}%
\pgfpathlineto{\pgfqpoint{3.712560in}{0.739656in}}%
\pgfpathlineto{\pgfqpoint{3.711996in}{0.739656in}}%
\pgfpathlineto{\pgfqpoint{3.711431in}{0.739656in}}%
\pgfpathlineto{\pgfqpoint{3.710867in}{0.739656in}}%
\pgfpathlineto{\pgfqpoint{3.710303in}{0.739656in}}%
\pgfpathlineto{\pgfqpoint{3.709738in}{0.739656in}}%
\pgfpathlineto{\pgfqpoint{3.709174in}{0.739656in}}%
\pgfpathlineto{\pgfqpoint{3.708610in}{0.739656in}}%
\pgfpathlineto{\pgfqpoint{3.708045in}{0.739656in}}%
\pgfpathlineto{\pgfqpoint{3.707481in}{0.739656in}}%
\pgfpathlineto{\pgfqpoint{3.706917in}{0.739656in}}%
\pgfpathlineto{\pgfqpoint{3.706352in}{0.739656in}}%
\pgfpathlineto{\pgfqpoint{3.705788in}{0.739656in}}%
\pgfpathlineto{\pgfqpoint{3.705224in}{0.739656in}}%
\pgfpathlineto{\pgfqpoint{3.704659in}{0.739656in}}%
\pgfpathlineto{\pgfqpoint{3.704095in}{0.739656in}}%
\pgfpathlineto{\pgfqpoint{3.703530in}{0.739656in}}%
\pgfpathlineto{\pgfqpoint{3.702966in}{0.739656in}}%
\pgfpathlineto{\pgfqpoint{3.702402in}{0.739656in}}%
\pgfpathlineto{\pgfqpoint{3.701837in}{0.739656in}}%
\pgfpathlineto{\pgfqpoint{3.701273in}{0.739656in}}%
\pgfpathlineto{\pgfqpoint{3.700709in}{0.739656in}}%
\pgfpathlineto{\pgfqpoint{3.700144in}{0.739656in}}%
\pgfpathlineto{\pgfqpoint{3.699580in}{0.739656in}}%
\pgfpathlineto{\pgfqpoint{3.699016in}{0.739656in}}%
\pgfpathlineto{\pgfqpoint{3.698451in}{0.739656in}}%
\pgfpathlineto{\pgfqpoint{3.697887in}{0.739656in}}%
\pgfpathlineto{\pgfqpoint{3.697323in}{0.739656in}}%
\pgfpathlineto{\pgfqpoint{3.696758in}{0.739656in}}%
\pgfpathlineto{\pgfqpoint{3.696194in}{0.739656in}}%
\pgfpathlineto{\pgfqpoint{3.695630in}{0.739656in}}%
\pgfpathlineto{\pgfqpoint{3.695065in}{0.739656in}}%
\pgfpathlineto{\pgfqpoint{3.694501in}{0.739656in}}%
\pgfpathlineto{\pgfqpoint{3.693937in}{0.739656in}}%
\pgfpathlineto{\pgfqpoint{3.693372in}{0.739656in}}%
\pgfpathlineto{\pgfqpoint{3.692808in}{0.739656in}}%
\pgfpathlineto{\pgfqpoint{3.692244in}{0.739656in}}%
\pgfpathlineto{\pgfqpoint{3.691679in}{0.739656in}}%
\pgfpathlineto{\pgfqpoint{3.691115in}{0.739656in}}%
\pgfpathlineto{\pgfqpoint{3.690551in}{0.739656in}}%
\pgfpathlineto{\pgfqpoint{3.689986in}{0.739656in}}%
\pgfpathlineto{\pgfqpoint{3.689422in}{0.739656in}}%
\pgfpathlineto{\pgfqpoint{3.688858in}{0.739656in}}%
\pgfpathlineto{\pgfqpoint{3.688293in}{0.739656in}}%
\pgfpathlineto{\pgfqpoint{3.687729in}{0.739656in}}%
\pgfpathlineto{\pgfqpoint{3.687164in}{0.739656in}}%
\pgfpathlineto{\pgfqpoint{3.686600in}{0.739656in}}%
\pgfpathlineto{\pgfqpoint{3.686036in}{0.739656in}}%
\pgfpathlineto{\pgfqpoint{3.685471in}{0.739656in}}%
\pgfpathlineto{\pgfqpoint{3.684907in}{0.739656in}}%
\pgfpathlineto{\pgfqpoint{3.684343in}{0.739656in}}%
\pgfpathlineto{\pgfqpoint{3.683778in}{0.739656in}}%
\pgfpathlineto{\pgfqpoint{3.683214in}{0.739656in}}%
\pgfpathlineto{\pgfqpoint{3.682650in}{0.739656in}}%
\pgfpathlineto{\pgfqpoint{3.682085in}{0.739656in}}%
\pgfpathlineto{\pgfqpoint{3.681521in}{0.739656in}}%
\pgfpathlineto{\pgfqpoint{3.680957in}{0.739656in}}%
\pgfpathlineto{\pgfqpoint{3.680392in}{0.739656in}}%
\pgfpathlineto{\pgfqpoint{3.679828in}{0.739656in}}%
\pgfpathlineto{\pgfqpoint{3.679264in}{0.739656in}}%
\pgfpathlineto{\pgfqpoint{3.678699in}{0.739656in}}%
\pgfpathlineto{\pgfqpoint{3.678135in}{0.739656in}}%
\pgfpathlineto{\pgfqpoint{3.677571in}{0.739656in}}%
\pgfpathlineto{\pgfqpoint{3.677006in}{0.739656in}}%
\pgfpathlineto{\pgfqpoint{3.676442in}{0.739656in}}%
\pgfpathlineto{\pgfqpoint{3.675878in}{0.739656in}}%
\pgfpathlineto{\pgfqpoint{3.675313in}{0.739656in}}%
\pgfpathlineto{\pgfqpoint{3.674749in}{0.739656in}}%
\pgfpathlineto{\pgfqpoint{3.674185in}{0.739656in}}%
\pgfpathlineto{\pgfqpoint{3.673620in}{0.739656in}}%
\pgfpathlineto{\pgfqpoint{3.673056in}{0.739656in}}%
\pgfpathlineto{\pgfqpoint{3.672491in}{0.739656in}}%
\pgfpathlineto{\pgfqpoint{3.671927in}{0.739656in}}%
\pgfpathlineto{\pgfqpoint{3.671363in}{0.739656in}}%
\pgfpathlineto{\pgfqpoint{3.670798in}{0.739656in}}%
\pgfpathlineto{\pgfqpoint{3.670234in}{0.739656in}}%
\pgfpathlineto{\pgfqpoint{3.669670in}{0.739656in}}%
\pgfpathlineto{\pgfqpoint{3.669105in}{0.739656in}}%
\pgfpathlineto{\pgfqpoint{3.668541in}{0.739656in}}%
\pgfpathlineto{\pgfqpoint{3.667977in}{0.739656in}}%
\pgfpathlineto{\pgfqpoint{3.667412in}{0.739656in}}%
\pgfpathlineto{\pgfqpoint{3.666848in}{0.739656in}}%
\pgfpathlineto{\pgfqpoint{3.666284in}{0.739656in}}%
\pgfpathlineto{\pgfqpoint{3.665719in}{0.739656in}}%
\pgfpathlineto{\pgfqpoint{3.665155in}{0.739656in}}%
\pgfpathlineto{\pgfqpoint{3.664591in}{0.739656in}}%
\pgfpathlineto{\pgfqpoint{3.664026in}{0.739656in}}%
\pgfpathlineto{\pgfqpoint{3.663462in}{0.739656in}}%
\pgfpathlineto{\pgfqpoint{3.662898in}{0.739656in}}%
\pgfpathlineto{\pgfqpoint{3.662333in}{0.739656in}}%
\pgfpathlineto{\pgfqpoint{3.661769in}{0.739656in}}%
\pgfpathlineto{\pgfqpoint{3.661205in}{0.739656in}}%
\pgfpathlineto{\pgfqpoint{3.660640in}{0.739656in}}%
\pgfpathlineto{\pgfqpoint{3.660076in}{0.739656in}}%
\pgfpathlineto{\pgfqpoint{3.659512in}{0.739656in}}%
\pgfpathlineto{\pgfqpoint{3.658947in}{0.739656in}}%
\pgfpathlineto{\pgfqpoint{3.658383in}{0.739656in}}%
\pgfpathlineto{\pgfqpoint{3.657818in}{0.739656in}}%
\pgfpathlineto{\pgfqpoint{3.657254in}{0.739656in}}%
\pgfpathlineto{\pgfqpoint{3.656690in}{0.739656in}}%
\pgfpathlineto{\pgfqpoint{3.656125in}{0.739656in}}%
\pgfpathlineto{\pgfqpoint{3.655561in}{0.739656in}}%
\pgfpathlineto{\pgfqpoint{3.654997in}{0.739656in}}%
\pgfpathlineto{\pgfqpoint{3.654432in}{0.739656in}}%
\pgfpathlineto{\pgfqpoint{3.653868in}{0.739656in}}%
\pgfpathlineto{\pgfqpoint{3.653304in}{0.739656in}}%
\pgfpathlineto{\pgfqpoint{3.652739in}{0.739656in}}%
\pgfpathlineto{\pgfqpoint{3.652175in}{0.739656in}}%
\pgfpathlineto{\pgfqpoint{3.651611in}{0.739656in}}%
\pgfpathlineto{\pgfqpoint{3.651046in}{0.739656in}}%
\pgfpathlineto{\pgfqpoint{3.650482in}{0.739656in}}%
\pgfpathlineto{\pgfqpoint{3.649918in}{0.739656in}}%
\pgfpathlineto{\pgfqpoint{3.649353in}{0.739656in}}%
\pgfpathlineto{\pgfqpoint{3.648789in}{0.739656in}}%
\pgfpathlineto{\pgfqpoint{3.648225in}{0.739656in}}%
\pgfpathlineto{\pgfqpoint{3.647660in}{0.739656in}}%
\pgfpathlineto{\pgfqpoint{3.647096in}{0.739656in}}%
\pgfpathlineto{\pgfqpoint{3.646532in}{0.739656in}}%
\pgfpathlineto{\pgfqpoint{3.645967in}{0.739656in}}%
\pgfpathlineto{\pgfqpoint{3.645403in}{0.739656in}}%
\pgfpathlineto{\pgfqpoint{3.644839in}{0.739656in}}%
\pgfpathlineto{\pgfqpoint{3.644274in}{0.739656in}}%
\pgfpathlineto{\pgfqpoint{3.643710in}{0.739656in}}%
\pgfpathlineto{\pgfqpoint{3.643146in}{0.739656in}}%
\pgfpathlineto{\pgfqpoint{3.642581in}{0.739656in}}%
\pgfpathlineto{\pgfqpoint{3.642017in}{0.739656in}}%
\pgfpathlineto{\pgfqpoint{3.641452in}{0.739656in}}%
\pgfpathlineto{\pgfqpoint{3.640888in}{0.739656in}}%
\pgfpathlineto{\pgfqpoint{3.640324in}{0.739656in}}%
\pgfpathlineto{\pgfqpoint{3.639759in}{0.739656in}}%
\pgfpathlineto{\pgfqpoint{3.639195in}{0.739656in}}%
\pgfpathlineto{\pgfqpoint{3.638631in}{0.739656in}}%
\pgfpathlineto{\pgfqpoint{3.638066in}{0.739656in}}%
\pgfpathlineto{\pgfqpoint{3.637502in}{0.739656in}}%
\pgfpathlineto{\pgfqpoint{3.636938in}{0.739656in}}%
\pgfpathlineto{\pgfqpoint{3.636373in}{0.739656in}}%
\pgfpathlineto{\pgfqpoint{3.635809in}{0.739656in}}%
\pgfpathlineto{\pgfqpoint{3.635245in}{0.739656in}}%
\pgfpathlineto{\pgfqpoint{3.634680in}{0.739656in}}%
\pgfpathlineto{\pgfqpoint{3.634116in}{0.739656in}}%
\pgfpathlineto{\pgfqpoint{3.633552in}{0.739656in}}%
\pgfpathlineto{\pgfqpoint{3.632987in}{0.739656in}}%
\pgfpathlineto{\pgfqpoint{3.632423in}{0.739656in}}%
\pgfpathlineto{\pgfqpoint{3.631859in}{0.739656in}}%
\pgfpathlineto{\pgfqpoint{3.631294in}{0.739656in}}%
\pgfpathlineto{\pgfqpoint{3.630730in}{0.739656in}}%
\pgfpathlineto{\pgfqpoint{3.630166in}{0.739656in}}%
\pgfpathlineto{\pgfqpoint{3.629601in}{0.739656in}}%
\pgfpathlineto{\pgfqpoint{3.629037in}{0.739656in}}%
\pgfpathlineto{\pgfqpoint{3.628473in}{0.739656in}}%
\pgfpathlineto{\pgfqpoint{3.627908in}{0.739656in}}%
\pgfpathlineto{\pgfqpoint{3.627344in}{0.739656in}}%
\pgfpathlineto{\pgfqpoint{3.626779in}{0.739656in}}%
\pgfpathlineto{\pgfqpoint{3.626215in}{0.739656in}}%
\pgfpathlineto{\pgfqpoint{3.625651in}{0.739656in}}%
\pgfpathlineto{\pgfqpoint{3.625086in}{0.739656in}}%
\pgfpathlineto{\pgfqpoint{3.624522in}{0.739656in}}%
\pgfpathlineto{\pgfqpoint{3.623958in}{0.739656in}}%
\pgfpathlineto{\pgfqpoint{3.623393in}{0.739656in}}%
\pgfpathlineto{\pgfqpoint{3.622829in}{0.739656in}}%
\pgfpathlineto{\pgfqpoint{3.622265in}{0.739656in}}%
\pgfpathlineto{\pgfqpoint{3.621700in}{0.739656in}}%
\pgfpathlineto{\pgfqpoint{3.621136in}{0.739656in}}%
\pgfpathlineto{\pgfqpoint{3.620572in}{0.739656in}}%
\pgfpathlineto{\pgfqpoint{3.620007in}{0.739656in}}%
\pgfpathlineto{\pgfqpoint{3.619443in}{0.739656in}}%
\pgfpathlineto{\pgfqpoint{3.618879in}{0.739656in}}%
\pgfpathlineto{\pgfqpoint{3.618314in}{0.739656in}}%
\pgfpathlineto{\pgfqpoint{3.617750in}{0.739656in}}%
\pgfpathlineto{\pgfqpoint{3.617186in}{0.739656in}}%
\pgfpathlineto{\pgfqpoint{3.616621in}{0.739656in}}%
\pgfpathlineto{\pgfqpoint{3.616057in}{0.739656in}}%
\pgfpathlineto{\pgfqpoint{3.615493in}{0.739656in}}%
\pgfpathlineto{\pgfqpoint{3.614928in}{0.739656in}}%
\pgfpathlineto{\pgfqpoint{3.614364in}{0.739656in}}%
\pgfpathlineto{\pgfqpoint{3.613800in}{0.739656in}}%
\pgfpathlineto{\pgfqpoint{3.613235in}{0.739656in}}%
\pgfpathlineto{\pgfqpoint{3.612671in}{0.739656in}}%
\pgfpathlineto{\pgfqpoint{3.612106in}{0.739656in}}%
\pgfpathlineto{\pgfqpoint{3.611542in}{0.739656in}}%
\pgfpathlineto{\pgfqpoint{3.610978in}{0.739656in}}%
\pgfpathlineto{\pgfqpoint{3.610413in}{0.739656in}}%
\pgfpathlineto{\pgfqpoint{3.609849in}{0.739656in}}%
\pgfpathlineto{\pgfqpoint{3.609285in}{0.739656in}}%
\pgfpathlineto{\pgfqpoint{3.608720in}{0.739656in}}%
\pgfpathlineto{\pgfqpoint{3.608156in}{0.739656in}}%
\pgfpathlineto{\pgfqpoint{3.607592in}{0.739656in}}%
\pgfpathlineto{\pgfqpoint{3.607027in}{0.739656in}}%
\pgfpathlineto{\pgfqpoint{3.606463in}{0.739656in}}%
\pgfpathlineto{\pgfqpoint{3.605899in}{0.739656in}}%
\pgfpathlineto{\pgfqpoint{3.605334in}{0.739656in}}%
\pgfpathlineto{\pgfqpoint{3.604770in}{0.739656in}}%
\pgfpathlineto{\pgfqpoint{3.604206in}{0.739656in}}%
\pgfpathlineto{\pgfqpoint{3.603641in}{0.739656in}}%
\pgfpathlineto{\pgfqpoint{3.603077in}{0.739656in}}%
\pgfpathlineto{\pgfqpoint{3.602513in}{0.739656in}}%
\pgfpathlineto{\pgfqpoint{3.601948in}{0.739656in}}%
\pgfpathlineto{\pgfqpoint{3.601384in}{0.739656in}}%
\pgfpathlineto{\pgfqpoint{3.600820in}{0.739656in}}%
\pgfpathlineto{\pgfqpoint{3.600255in}{0.739656in}}%
\pgfpathlineto{\pgfqpoint{3.599691in}{0.739656in}}%
\pgfpathlineto{\pgfqpoint{3.599127in}{0.739656in}}%
\pgfpathlineto{\pgfqpoint{3.598562in}{0.739656in}}%
\pgfpathlineto{\pgfqpoint{3.597998in}{0.739656in}}%
\pgfpathlineto{\pgfqpoint{3.597434in}{0.739656in}}%
\pgfpathlineto{\pgfqpoint{3.596869in}{0.739656in}}%
\pgfpathlineto{\pgfqpoint{3.596305in}{0.739656in}}%
\pgfpathlineto{\pgfqpoint{3.595740in}{0.739656in}}%
\pgfpathlineto{\pgfqpoint{3.595176in}{0.739656in}}%
\pgfpathlineto{\pgfqpoint{3.594612in}{0.739656in}}%
\pgfpathlineto{\pgfqpoint{3.594047in}{0.739656in}}%
\pgfpathlineto{\pgfqpoint{3.593483in}{0.739656in}}%
\pgfpathlineto{\pgfqpoint{3.592919in}{0.739656in}}%
\pgfpathlineto{\pgfqpoint{3.592354in}{0.739656in}}%
\pgfpathlineto{\pgfqpoint{3.591790in}{0.739656in}}%
\pgfpathlineto{\pgfqpoint{3.591226in}{0.739656in}}%
\pgfpathlineto{\pgfqpoint{3.590661in}{0.739656in}}%
\pgfpathlineto{\pgfqpoint{3.590097in}{0.739656in}}%
\pgfpathlineto{\pgfqpoint{3.589533in}{0.739656in}}%
\pgfpathlineto{\pgfqpoint{3.588968in}{0.739656in}}%
\pgfpathlineto{\pgfqpoint{3.588404in}{0.739656in}}%
\pgfpathlineto{\pgfqpoint{3.587840in}{0.739656in}}%
\pgfpathlineto{\pgfqpoint{3.587275in}{0.739656in}}%
\pgfpathlineto{\pgfqpoint{3.586711in}{0.739656in}}%
\pgfpathlineto{\pgfqpoint{3.586147in}{0.739656in}}%
\pgfpathlineto{\pgfqpoint{3.585582in}{0.739656in}}%
\pgfpathlineto{\pgfqpoint{3.585018in}{0.739656in}}%
\pgfpathlineto{\pgfqpoint{3.584454in}{0.739656in}}%
\pgfpathlineto{\pgfqpoint{3.583889in}{0.739656in}}%
\pgfpathlineto{\pgfqpoint{3.583325in}{0.739656in}}%
\pgfpathlineto{\pgfqpoint{3.582761in}{0.739656in}}%
\pgfpathlineto{\pgfqpoint{3.582196in}{0.739656in}}%
\pgfpathlineto{\pgfqpoint{3.581632in}{0.739656in}}%
\pgfpathlineto{\pgfqpoint{3.581067in}{0.739656in}}%
\pgfpathlineto{\pgfqpoint{3.580503in}{0.739656in}}%
\pgfpathlineto{\pgfqpoint{3.579939in}{0.739656in}}%
\pgfpathlineto{\pgfqpoint{3.579374in}{0.739656in}}%
\pgfpathlineto{\pgfqpoint{3.578810in}{0.739656in}}%
\pgfpathlineto{\pgfqpoint{3.578246in}{0.739656in}}%
\pgfpathlineto{\pgfqpoint{3.577681in}{0.739656in}}%
\pgfpathlineto{\pgfqpoint{3.577117in}{0.739656in}}%
\pgfpathlineto{\pgfqpoint{3.576553in}{0.739656in}}%
\pgfpathlineto{\pgfqpoint{3.575988in}{0.739656in}}%
\pgfpathlineto{\pgfqpoint{3.575424in}{0.739656in}}%
\pgfpathlineto{\pgfqpoint{3.574860in}{0.739656in}}%
\pgfpathlineto{\pgfqpoint{3.574295in}{0.739656in}}%
\pgfpathlineto{\pgfqpoint{3.573731in}{0.739656in}}%
\pgfpathlineto{\pgfqpoint{3.573167in}{0.739656in}}%
\pgfpathlineto{\pgfqpoint{3.572602in}{0.739656in}}%
\pgfpathlineto{\pgfqpoint{3.572038in}{0.739656in}}%
\pgfpathlineto{\pgfqpoint{3.571474in}{0.739656in}}%
\pgfpathlineto{\pgfqpoint{3.570909in}{0.739656in}}%
\pgfpathlineto{\pgfqpoint{3.570345in}{0.739656in}}%
\pgfpathlineto{\pgfqpoint{3.569781in}{0.739656in}}%
\pgfpathlineto{\pgfqpoint{3.569216in}{0.739656in}}%
\pgfpathlineto{\pgfqpoint{3.568652in}{0.739656in}}%
\pgfpathlineto{\pgfqpoint{3.568088in}{0.739656in}}%
\pgfpathlineto{\pgfqpoint{3.567523in}{0.739656in}}%
\pgfpathlineto{\pgfqpoint{3.566959in}{0.739656in}}%
\pgfpathlineto{\pgfqpoint{3.566394in}{0.739656in}}%
\pgfpathlineto{\pgfqpoint{3.565830in}{0.739656in}}%
\pgfpathlineto{\pgfqpoint{3.565266in}{0.739656in}}%
\pgfpathlineto{\pgfqpoint{3.564701in}{0.739656in}}%
\pgfpathlineto{\pgfqpoint{3.564137in}{0.739656in}}%
\pgfpathlineto{\pgfqpoint{3.563573in}{0.739656in}}%
\pgfpathlineto{\pgfqpoint{3.563008in}{0.739656in}}%
\pgfpathlineto{\pgfqpoint{3.562444in}{0.739656in}}%
\pgfpathlineto{\pgfqpoint{3.561880in}{0.739656in}}%
\pgfpathlineto{\pgfqpoint{3.561315in}{0.739656in}}%
\pgfpathlineto{\pgfqpoint{3.560751in}{0.739656in}}%
\pgfpathlineto{\pgfqpoint{3.560187in}{0.739656in}}%
\pgfpathlineto{\pgfqpoint{3.559622in}{0.739656in}}%
\pgfpathlineto{\pgfqpoint{3.559058in}{0.739656in}}%
\pgfpathlineto{\pgfqpoint{3.558494in}{0.739656in}}%
\pgfpathlineto{\pgfqpoint{3.557929in}{0.739656in}}%
\pgfpathlineto{\pgfqpoint{3.557365in}{0.739656in}}%
\pgfpathlineto{\pgfqpoint{3.556801in}{0.739656in}}%
\pgfpathlineto{\pgfqpoint{3.556236in}{0.739656in}}%
\pgfpathlineto{\pgfqpoint{3.555672in}{0.739656in}}%
\pgfpathlineto{\pgfqpoint{3.555108in}{0.739656in}}%
\pgfpathlineto{\pgfqpoint{3.554543in}{0.739656in}}%
\pgfpathlineto{\pgfqpoint{3.553979in}{0.739656in}}%
\pgfpathlineto{\pgfqpoint{3.553415in}{0.739656in}}%
\pgfpathlineto{\pgfqpoint{3.552850in}{0.739656in}}%
\pgfpathlineto{\pgfqpoint{3.552286in}{0.739656in}}%
\pgfpathlineto{\pgfqpoint{3.551721in}{0.739656in}}%
\pgfpathlineto{\pgfqpoint{3.551157in}{0.739656in}}%
\pgfpathlineto{\pgfqpoint{3.550593in}{0.739656in}}%
\pgfpathlineto{\pgfqpoint{3.550028in}{0.739656in}}%
\pgfpathlineto{\pgfqpoint{3.549464in}{0.739656in}}%
\pgfpathlineto{\pgfqpoint{3.548900in}{0.739656in}}%
\pgfpathlineto{\pgfqpoint{3.548335in}{0.739656in}}%
\pgfpathlineto{\pgfqpoint{3.547771in}{0.739656in}}%
\pgfpathlineto{\pgfqpoint{3.547207in}{0.739656in}}%
\pgfpathlineto{\pgfqpoint{3.546642in}{0.739656in}}%
\pgfpathlineto{\pgfqpoint{3.546078in}{0.739656in}}%
\pgfpathlineto{\pgfqpoint{3.545514in}{0.739656in}}%
\pgfpathlineto{\pgfqpoint{3.544949in}{0.739656in}}%
\pgfpathlineto{\pgfqpoint{3.544385in}{0.739656in}}%
\pgfpathlineto{\pgfqpoint{3.543821in}{0.739656in}}%
\pgfpathlineto{\pgfqpoint{3.543256in}{0.739656in}}%
\pgfpathlineto{\pgfqpoint{3.542692in}{0.739656in}}%
\pgfpathlineto{\pgfqpoint{3.542128in}{0.739656in}}%
\pgfpathlineto{\pgfqpoint{3.541563in}{0.739656in}}%
\pgfpathlineto{\pgfqpoint{3.540999in}{0.739656in}}%
\pgfpathlineto{\pgfqpoint{3.540435in}{0.739656in}}%
\pgfpathlineto{\pgfqpoint{3.539870in}{0.739656in}}%
\pgfpathlineto{\pgfqpoint{3.539306in}{0.739656in}}%
\pgfpathlineto{\pgfqpoint{3.538742in}{0.739656in}}%
\pgfpathlineto{\pgfqpoint{3.538177in}{0.739656in}}%
\pgfpathlineto{\pgfqpoint{3.537613in}{0.739656in}}%
\pgfpathlineto{\pgfqpoint{3.537049in}{0.739656in}}%
\pgfpathlineto{\pgfqpoint{3.536484in}{0.739656in}}%
\pgfpathlineto{\pgfqpoint{3.535920in}{0.739656in}}%
\pgfpathlineto{\pgfqpoint{3.535355in}{0.739656in}}%
\pgfpathlineto{\pgfqpoint{3.534791in}{0.739656in}}%
\pgfpathlineto{\pgfqpoint{3.534227in}{0.739656in}}%
\pgfpathlineto{\pgfqpoint{3.533662in}{0.739656in}}%
\pgfpathlineto{\pgfqpoint{3.533098in}{0.739656in}}%
\pgfpathlineto{\pgfqpoint{3.532534in}{0.739656in}}%
\pgfpathlineto{\pgfqpoint{3.531969in}{0.739656in}}%
\pgfpathlineto{\pgfqpoint{3.531405in}{0.739656in}}%
\pgfpathlineto{\pgfqpoint{3.530841in}{0.739656in}}%
\pgfpathlineto{\pgfqpoint{3.530276in}{0.739656in}}%
\pgfpathlineto{\pgfqpoint{3.529712in}{0.739656in}}%
\pgfpathlineto{\pgfqpoint{3.529148in}{0.739656in}}%
\pgfpathlineto{\pgfqpoint{3.528583in}{0.739656in}}%
\pgfpathlineto{\pgfqpoint{3.528019in}{0.739656in}}%
\pgfpathlineto{\pgfqpoint{3.527455in}{0.739656in}}%
\pgfpathlineto{\pgfqpoint{3.526890in}{0.739656in}}%
\pgfpathlineto{\pgfqpoint{3.526326in}{0.739656in}}%
\pgfpathlineto{\pgfqpoint{3.525762in}{0.739656in}}%
\pgfpathlineto{\pgfqpoint{3.525197in}{0.739656in}}%
\pgfpathlineto{\pgfqpoint{3.524633in}{0.739656in}}%
\pgfpathlineto{\pgfqpoint{3.524069in}{0.739656in}}%
\pgfpathlineto{\pgfqpoint{3.523504in}{0.739656in}}%
\pgfpathlineto{\pgfqpoint{3.522940in}{0.739656in}}%
\pgfpathlineto{\pgfqpoint{3.522376in}{0.739656in}}%
\pgfpathlineto{\pgfqpoint{3.521811in}{0.739656in}}%
\pgfpathlineto{\pgfqpoint{3.521247in}{0.739656in}}%
\pgfpathlineto{\pgfqpoint{3.520682in}{0.739656in}}%
\pgfpathlineto{\pgfqpoint{3.520118in}{0.739656in}}%
\pgfpathlineto{\pgfqpoint{3.519554in}{0.739656in}}%
\pgfpathlineto{\pgfqpoint{3.518989in}{0.739656in}}%
\pgfpathlineto{\pgfqpoint{3.518425in}{0.739656in}}%
\pgfpathlineto{\pgfqpoint{3.517861in}{0.739656in}}%
\pgfpathlineto{\pgfqpoint{3.517296in}{0.739656in}}%
\pgfpathlineto{\pgfqpoint{3.516732in}{0.739656in}}%
\pgfpathlineto{\pgfqpoint{3.516168in}{0.739656in}}%
\pgfpathlineto{\pgfqpoint{3.515603in}{0.739656in}}%
\pgfpathlineto{\pgfqpoint{3.515039in}{0.739656in}}%
\pgfpathlineto{\pgfqpoint{3.514475in}{0.739656in}}%
\pgfpathlineto{\pgfqpoint{3.513910in}{0.739656in}}%
\pgfpathlineto{\pgfqpoint{3.513346in}{0.739656in}}%
\pgfpathlineto{\pgfqpoint{3.512782in}{0.739656in}}%
\pgfpathlineto{\pgfqpoint{3.512217in}{0.739656in}}%
\pgfpathlineto{\pgfqpoint{3.511653in}{0.739656in}}%
\pgfpathlineto{\pgfqpoint{3.511089in}{0.739656in}}%
\pgfpathlineto{\pgfqpoint{3.510524in}{0.739656in}}%
\pgfpathlineto{\pgfqpoint{3.509960in}{0.739656in}}%
\pgfpathlineto{\pgfqpoint{3.509396in}{0.739656in}}%
\pgfpathlineto{\pgfqpoint{3.508831in}{0.739656in}}%
\pgfpathlineto{\pgfqpoint{3.508267in}{0.739656in}}%
\pgfpathlineto{\pgfqpoint{3.507703in}{0.739656in}}%
\pgfpathlineto{\pgfqpoint{3.507138in}{0.739656in}}%
\pgfpathlineto{\pgfqpoint{3.506574in}{0.739656in}}%
\pgfpathlineto{\pgfqpoint{3.506009in}{0.739656in}}%
\pgfpathlineto{\pgfqpoint{3.505445in}{0.739656in}}%
\pgfpathlineto{\pgfqpoint{3.504881in}{0.739656in}}%
\pgfpathlineto{\pgfqpoint{3.504316in}{0.739656in}}%
\pgfpathlineto{\pgfqpoint{3.503752in}{0.739656in}}%
\pgfpathlineto{\pgfqpoint{3.503188in}{0.739656in}}%
\pgfpathlineto{\pgfqpoint{3.502623in}{0.739656in}}%
\pgfpathlineto{\pgfqpoint{3.502059in}{0.739656in}}%
\pgfpathlineto{\pgfqpoint{3.501495in}{0.739656in}}%
\pgfpathlineto{\pgfqpoint{3.500930in}{0.739656in}}%
\pgfpathlineto{\pgfqpoint{3.500366in}{0.739656in}}%
\pgfpathlineto{\pgfqpoint{3.499802in}{0.739656in}}%
\pgfpathlineto{\pgfqpoint{3.499237in}{0.739656in}}%
\pgfpathlineto{\pgfqpoint{3.498673in}{0.739656in}}%
\pgfpathlineto{\pgfqpoint{3.498109in}{0.739656in}}%
\pgfpathlineto{\pgfqpoint{3.497544in}{0.739656in}}%
\pgfpathlineto{\pgfqpoint{3.496980in}{0.739656in}}%
\pgfpathlineto{\pgfqpoint{3.496416in}{0.739656in}}%
\pgfpathlineto{\pgfqpoint{3.495851in}{0.739656in}}%
\pgfpathlineto{\pgfqpoint{3.495287in}{0.739656in}}%
\pgfpathlineto{\pgfqpoint{3.494723in}{0.739656in}}%
\pgfpathlineto{\pgfqpoint{3.494158in}{0.739656in}}%
\pgfpathlineto{\pgfqpoint{3.493594in}{0.739656in}}%
\pgfpathlineto{\pgfqpoint{3.493030in}{0.739656in}}%
\pgfpathlineto{\pgfqpoint{3.492465in}{0.739656in}}%
\pgfpathlineto{\pgfqpoint{3.491901in}{0.739656in}}%
\pgfpathlineto{\pgfqpoint{3.491337in}{0.739656in}}%
\pgfpathlineto{\pgfqpoint{3.490772in}{0.739656in}}%
\pgfpathlineto{\pgfqpoint{3.490208in}{0.739656in}}%
\pgfpathlineto{\pgfqpoint{3.489643in}{0.739656in}}%
\pgfpathlineto{\pgfqpoint{3.489079in}{0.739656in}}%
\pgfpathlineto{\pgfqpoint{3.488515in}{0.739656in}}%
\pgfpathlineto{\pgfqpoint{3.487950in}{0.739656in}}%
\pgfpathlineto{\pgfqpoint{3.487386in}{0.739656in}}%
\pgfpathlineto{\pgfqpoint{3.486822in}{0.739656in}}%
\pgfpathlineto{\pgfqpoint{3.486257in}{0.739656in}}%
\pgfpathlineto{\pgfqpoint{3.485693in}{0.739656in}}%
\pgfpathlineto{\pgfqpoint{3.485129in}{0.739656in}}%
\pgfpathlineto{\pgfqpoint{3.484564in}{0.739656in}}%
\pgfpathlineto{\pgfqpoint{3.484000in}{0.739656in}}%
\pgfpathlineto{\pgfqpoint{3.483436in}{0.739656in}}%
\pgfpathlineto{\pgfqpoint{3.482871in}{0.739656in}}%
\pgfpathlineto{\pgfqpoint{3.482307in}{0.739656in}}%
\pgfpathlineto{\pgfqpoint{3.481743in}{0.739656in}}%
\pgfpathlineto{\pgfqpoint{3.481178in}{0.739656in}}%
\pgfpathlineto{\pgfqpoint{3.480614in}{0.739656in}}%
\pgfpathlineto{\pgfqpoint{3.480050in}{0.739656in}}%
\pgfpathlineto{\pgfqpoint{3.479485in}{0.739656in}}%
\pgfpathlineto{\pgfqpoint{3.478921in}{0.739656in}}%
\pgfpathlineto{\pgfqpoint{3.478357in}{0.739656in}}%
\pgfpathlineto{\pgfqpoint{3.477792in}{0.739656in}}%
\pgfpathlineto{\pgfqpoint{3.477228in}{0.739656in}}%
\pgfpathlineto{\pgfqpoint{3.476664in}{0.739656in}}%
\pgfpathlineto{\pgfqpoint{3.476099in}{0.739656in}}%
\pgfpathlineto{\pgfqpoint{3.475535in}{0.739656in}}%
\pgfpathlineto{\pgfqpoint{3.474970in}{0.739656in}}%
\pgfpathlineto{\pgfqpoint{3.474406in}{0.739656in}}%
\pgfpathlineto{\pgfqpoint{3.473842in}{0.739656in}}%
\pgfpathlineto{\pgfqpoint{3.473277in}{0.739656in}}%
\pgfpathlineto{\pgfqpoint{3.472713in}{0.739656in}}%
\pgfpathlineto{\pgfqpoint{3.472149in}{0.739656in}}%
\pgfpathlineto{\pgfqpoint{3.471584in}{0.739656in}}%
\pgfpathlineto{\pgfqpoint{3.471020in}{0.739656in}}%
\pgfpathlineto{\pgfqpoint{3.470456in}{0.739656in}}%
\pgfpathlineto{\pgfqpoint{3.469891in}{0.739656in}}%
\pgfpathlineto{\pgfqpoint{3.469327in}{0.739656in}}%
\pgfpathlineto{\pgfqpoint{3.468763in}{0.739656in}}%
\pgfpathlineto{\pgfqpoint{3.468198in}{0.739656in}}%
\pgfpathlineto{\pgfqpoint{3.467634in}{0.739656in}}%
\pgfpathlineto{\pgfqpoint{3.467070in}{0.739656in}}%
\pgfpathlineto{\pgfqpoint{3.466505in}{0.739656in}}%
\pgfpathlineto{\pgfqpoint{3.465941in}{0.739656in}}%
\pgfpathlineto{\pgfqpoint{3.465377in}{0.739656in}}%
\pgfpathlineto{\pgfqpoint{3.464812in}{0.739656in}}%
\pgfpathlineto{\pgfqpoint{3.464248in}{0.739656in}}%
\pgfpathlineto{\pgfqpoint{3.463684in}{0.739656in}}%
\pgfpathlineto{\pgfqpoint{3.463119in}{0.739656in}}%
\pgfpathlineto{\pgfqpoint{3.462555in}{0.739656in}}%
\pgfpathlineto{\pgfqpoint{3.461991in}{0.739656in}}%
\pgfpathlineto{\pgfqpoint{3.461426in}{0.739656in}}%
\pgfpathlineto{\pgfqpoint{3.460862in}{0.739656in}}%
\pgfpathlineto{\pgfqpoint{3.460297in}{0.739656in}}%
\pgfpathlineto{\pgfqpoint{3.459733in}{0.739656in}}%
\pgfpathlineto{\pgfqpoint{3.459169in}{0.739656in}}%
\pgfpathlineto{\pgfqpoint{3.458604in}{0.739656in}}%
\pgfpathlineto{\pgfqpoint{3.458040in}{0.739656in}}%
\pgfpathlineto{\pgfqpoint{3.457476in}{0.739656in}}%
\pgfpathlineto{\pgfqpoint{3.456911in}{0.739656in}}%
\pgfpathlineto{\pgfqpoint{3.456347in}{0.739656in}}%
\pgfpathlineto{\pgfqpoint{3.455783in}{0.739656in}}%
\pgfpathlineto{\pgfqpoint{3.455218in}{0.739656in}}%
\pgfpathlineto{\pgfqpoint{3.454654in}{0.739656in}}%
\pgfpathlineto{\pgfqpoint{3.454090in}{0.739656in}}%
\pgfpathlineto{\pgfqpoint{3.453525in}{0.739656in}}%
\pgfpathlineto{\pgfqpoint{3.452961in}{0.739656in}}%
\pgfpathlineto{\pgfqpoint{3.452397in}{0.739656in}}%
\pgfpathlineto{\pgfqpoint{3.451832in}{0.739656in}}%
\pgfpathlineto{\pgfqpoint{3.451268in}{0.739656in}}%
\pgfpathlineto{\pgfqpoint{3.450704in}{0.739656in}}%
\pgfpathlineto{\pgfqpoint{3.450139in}{0.739656in}}%
\pgfpathlineto{\pgfqpoint{3.449575in}{0.739656in}}%
\pgfpathlineto{\pgfqpoint{3.449011in}{0.739656in}}%
\pgfpathlineto{\pgfqpoint{3.448446in}{0.739656in}}%
\pgfpathlineto{\pgfqpoint{3.447882in}{0.739656in}}%
\pgfpathlineto{\pgfqpoint{3.447318in}{0.739656in}}%
\pgfpathlineto{\pgfqpoint{3.446753in}{0.739656in}}%
\pgfpathlineto{\pgfqpoint{3.446189in}{0.739656in}}%
\pgfpathlineto{\pgfqpoint{3.445625in}{0.739656in}}%
\pgfpathlineto{\pgfqpoint{3.445060in}{0.739656in}}%
\pgfpathlineto{\pgfqpoint{3.444496in}{0.739656in}}%
\pgfpathlineto{\pgfqpoint{3.443931in}{0.739656in}}%
\pgfpathlineto{\pgfqpoint{3.443367in}{0.739656in}}%
\pgfpathlineto{\pgfqpoint{3.442803in}{0.739656in}}%
\pgfpathlineto{\pgfqpoint{3.442238in}{0.739656in}}%
\pgfpathlineto{\pgfqpoint{3.441674in}{0.739656in}}%
\pgfpathlineto{\pgfqpoint{3.441110in}{0.739656in}}%
\pgfpathlineto{\pgfqpoint{3.440545in}{0.739656in}}%
\pgfpathlineto{\pgfqpoint{3.439981in}{0.739656in}}%
\pgfpathlineto{\pgfqpoint{3.439417in}{0.739656in}}%
\pgfpathlineto{\pgfqpoint{3.438852in}{0.739656in}}%
\pgfpathlineto{\pgfqpoint{3.438288in}{0.739656in}}%
\pgfpathlineto{\pgfqpoint{3.437724in}{0.739656in}}%
\pgfpathlineto{\pgfqpoint{3.437159in}{0.739656in}}%
\pgfpathlineto{\pgfqpoint{3.436595in}{0.739656in}}%
\pgfpathlineto{\pgfqpoint{3.436031in}{0.739656in}}%
\pgfpathlineto{\pgfqpoint{3.435466in}{0.739656in}}%
\pgfpathlineto{\pgfqpoint{3.434902in}{0.739656in}}%
\pgfpathlineto{\pgfqpoint{3.434338in}{0.739656in}}%
\pgfpathlineto{\pgfqpoint{3.433773in}{0.739656in}}%
\pgfpathlineto{\pgfqpoint{3.433209in}{0.739656in}}%
\pgfpathlineto{\pgfqpoint{3.432645in}{0.739656in}}%
\pgfpathlineto{\pgfqpoint{3.432080in}{0.739656in}}%
\pgfpathlineto{\pgfqpoint{3.431516in}{0.739656in}}%
\pgfpathlineto{\pgfqpoint{3.430952in}{0.739656in}}%
\pgfpathlineto{\pgfqpoint{3.430387in}{0.739656in}}%
\pgfpathlineto{\pgfqpoint{3.429823in}{0.739656in}}%
\pgfpathlineto{\pgfqpoint{3.429258in}{0.739656in}}%
\pgfpathlineto{\pgfqpoint{3.428694in}{0.739656in}}%
\pgfpathlineto{\pgfqpoint{3.428130in}{0.739656in}}%
\pgfpathlineto{\pgfqpoint{3.427565in}{0.739656in}}%
\pgfpathlineto{\pgfqpoint{3.427001in}{0.739656in}}%
\pgfpathlineto{\pgfqpoint{3.426437in}{0.739656in}}%
\pgfpathlineto{\pgfqpoint{3.425872in}{0.739656in}}%
\pgfpathlineto{\pgfqpoint{3.425308in}{0.739656in}}%
\pgfpathlineto{\pgfqpoint{3.424744in}{0.739656in}}%
\pgfpathlineto{\pgfqpoint{3.424179in}{0.739656in}}%
\pgfpathlineto{\pgfqpoint{3.423615in}{0.739656in}}%
\pgfpathlineto{\pgfqpoint{3.423051in}{0.739656in}}%
\pgfpathlineto{\pgfqpoint{3.422486in}{0.739656in}}%
\pgfpathlineto{\pgfqpoint{3.421922in}{0.739656in}}%
\pgfpathlineto{\pgfqpoint{3.421358in}{0.739656in}}%
\pgfpathlineto{\pgfqpoint{3.420793in}{0.739656in}}%
\pgfpathlineto{\pgfqpoint{3.420229in}{0.739656in}}%
\pgfpathlineto{\pgfqpoint{3.419665in}{0.739656in}}%
\pgfpathlineto{\pgfqpoint{3.419100in}{0.739656in}}%
\pgfpathlineto{\pgfqpoint{3.418536in}{0.739656in}}%
\pgfpathlineto{\pgfqpoint{3.417972in}{0.739656in}}%
\pgfpathlineto{\pgfqpoint{3.417407in}{0.739656in}}%
\pgfpathlineto{\pgfqpoint{3.416843in}{0.739656in}}%
\pgfpathlineto{\pgfqpoint{3.416279in}{0.739656in}}%
\pgfpathlineto{\pgfqpoint{3.415714in}{0.739656in}}%
\pgfpathlineto{\pgfqpoint{3.415150in}{0.739656in}}%
\pgfpathlineto{\pgfqpoint{3.414585in}{0.739656in}}%
\pgfpathlineto{\pgfqpoint{3.414021in}{0.739656in}}%
\pgfpathlineto{\pgfqpoint{3.413457in}{0.739656in}}%
\pgfpathlineto{\pgfqpoint{3.412892in}{0.739656in}}%
\pgfpathlineto{\pgfqpoint{3.412328in}{0.739656in}}%
\pgfpathlineto{\pgfqpoint{3.411764in}{0.739656in}}%
\pgfpathlineto{\pgfqpoint{3.411199in}{0.739656in}}%
\pgfpathlineto{\pgfqpoint{3.410635in}{0.739656in}}%
\pgfpathlineto{\pgfqpoint{3.410071in}{0.739656in}}%
\pgfpathlineto{\pgfqpoint{3.409506in}{0.739656in}}%
\pgfpathlineto{\pgfqpoint{3.408942in}{0.739656in}}%
\pgfpathlineto{\pgfqpoint{3.408378in}{0.739656in}}%
\pgfpathlineto{\pgfqpoint{3.407813in}{0.739656in}}%
\pgfpathlineto{\pgfqpoint{3.407249in}{0.739656in}}%
\pgfpathlineto{\pgfqpoint{3.406685in}{0.739656in}}%
\pgfpathlineto{\pgfqpoint{3.406120in}{0.739656in}}%
\pgfpathlineto{\pgfqpoint{3.405556in}{0.739656in}}%
\pgfpathlineto{\pgfqpoint{3.404992in}{0.739656in}}%
\pgfpathlineto{\pgfqpoint{3.404427in}{0.739656in}}%
\pgfpathlineto{\pgfqpoint{3.403863in}{0.739656in}}%
\pgfpathlineto{\pgfqpoint{3.403299in}{0.739656in}}%
\pgfpathlineto{\pgfqpoint{3.402734in}{0.739656in}}%
\pgfpathlineto{\pgfqpoint{3.402170in}{0.739656in}}%
\pgfpathlineto{\pgfqpoint{3.401606in}{0.739656in}}%
\pgfpathlineto{\pgfqpoint{3.401041in}{0.739656in}}%
\pgfpathlineto{\pgfqpoint{3.400477in}{0.739656in}}%
\pgfpathlineto{\pgfqpoint{3.399913in}{0.739656in}}%
\pgfpathlineto{\pgfqpoint{3.399348in}{0.739656in}}%
\pgfpathlineto{\pgfqpoint{3.398784in}{0.739656in}}%
\pgfpathlineto{\pgfqpoint{3.398219in}{0.739656in}}%
\pgfpathlineto{\pgfqpoint{3.397655in}{0.739656in}}%
\pgfpathlineto{\pgfqpoint{3.397091in}{0.739656in}}%
\pgfpathlineto{\pgfqpoint{3.396526in}{0.739656in}}%
\pgfpathlineto{\pgfqpoint{3.395962in}{0.739656in}}%
\pgfpathlineto{\pgfqpoint{3.395398in}{0.739656in}}%
\pgfpathlineto{\pgfqpoint{3.394833in}{0.739656in}}%
\pgfpathlineto{\pgfqpoint{3.394269in}{0.739656in}}%
\pgfpathlineto{\pgfqpoint{3.393705in}{0.739656in}}%
\pgfpathlineto{\pgfqpoint{3.393140in}{0.739656in}}%
\pgfpathlineto{\pgfqpoint{3.392576in}{0.739656in}}%
\pgfpathlineto{\pgfqpoint{3.392012in}{0.739656in}}%
\pgfpathlineto{\pgfqpoint{3.391447in}{0.739656in}}%
\pgfpathlineto{\pgfqpoint{3.390883in}{0.739656in}}%
\pgfpathlineto{\pgfqpoint{3.390319in}{0.739656in}}%
\pgfpathlineto{\pgfqpoint{3.389754in}{0.739656in}}%
\pgfpathlineto{\pgfqpoint{3.389190in}{0.739656in}}%
\pgfpathlineto{\pgfqpoint{3.388626in}{0.739656in}}%
\pgfpathlineto{\pgfqpoint{3.388061in}{0.739656in}}%
\pgfpathlineto{\pgfqpoint{3.387497in}{0.739656in}}%
\pgfpathlineto{\pgfqpoint{3.386933in}{0.739656in}}%
\pgfpathlineto{\pgfqpoint{3.386368in}{0.739656in}}%
\pgfpathlineto{\pgfqpoint{3.385804in}{0.739656in}}%
\pgfpathlineto{\pgfqpoint{3.385240in}{0.739656in}}%
\pgfpathlineto{\pgfqpoint{3.384675in}{0.739656in}}%
\pgfpathlineto{\pgfqpoint{3.384111in}{0.739656in}}%
\pgfpathlineto{\pgfqpoint{3.383546in}{0.739656in}}%
\pgfpathlineto{\pgfqpoint{3.382982in}{0.739656in}}%
\pgfpathlineto{\pgfqpoint{3.382418in}{0.739656in}}%
\pgfpathlineto{\pgfqpoint{3.381853in}{0.739656in}}%
\pgfpathlineto{\pgfqpoint{3.381289in}{0.739656in}}%
\pgfpathlineto{\pgfqpoint{3.380725in}{0.739656in}}%
\pgfpathlineto{\pgfqpoint{3.380160in}{0.739656in}}%
\pgfpathlineto{\pgfqpoint{3.379596in}{0.739656in}}%
\pgfpathlineto{\pgfqpoint{3.379032in}{0.739656in}}%
\pgfpathlineto{\pgfqpoint{3.378467in}{0.739656in}}%
\pgfpathlineto{\pgfqpoint{3.377903in}{0.739656in}}%
\pgfpathlineto{\pgfqpoint{3.377339in}{0.739656in}}%
\pgfpathlineto{\pgfqpoint{3.376774in}{0.739656in}}%
\pgfpathlineto{\pgfqpoint{3.376210in}{0.739656in}}%
\pgfpathlineto{\pgfqpoint{3.375646in}{0.739656in}}%
\pgfpathlineto{\pgfqpoint{3.375081in}{0.739656in}}%
\pgfpathlineto{\pgfqpoint{3.374517in}{0.739656in}}%
\pgfpathlineto{\pgfqpoint{3.373953in}{0.739656in}}%
\pgfpathlineto{\pgfqpoint{3.373388in}{0.739656in}}%
\pgfpathlineto{\pgfqpoint{3.372824in}{0.739656in}}%
\pgfpathlineto{\pgfqpoint{3.372260in}{0.739656in}}%
\pgfpathlineto{\pgfqpoint{3.371695in}{0.739656in}}%
\pgfpathlineto{\pgfqpoint{3.371131in}{0.739656in}}%
\pgfpathlineto{\pgfqpoint{3.370567in}{0.739656in}}%
\pgfpathlineto{\pgfqpoint{3.370002in}{0.739656in}}%
\pgfpathlineto{\pgfqpoint{3.369438in}{0.739656in}}%
\pgfpathlineto{\pgfqpoint{3.368873in}{0.739656in}}%
\pgfpathlineto{\pgfqpoint{3.368309in}{0.739656in}}%
\pgfpathlineto{\pgfqpoint{3.367745in}{0.739656in}}%
\pgfpathlineto{\pgfqpoint{3.367180in}{0.739656in}}%
\pgfpathlineto{\pgfqpoint{3.366616in}{0.739656in}}%
\pgfpathlineto{\pgfqpoint{3.366052in}{0.739656in}}%
\pgfpathlineto{\pgfqpoint{3.365487in}{0.739656in}}%
\pgfpathlineto{\pgfqpoint{3.364923in}{0.739656in}}%
\pgfpathlineto{\pgfqpoint{3.364359in}{0.739656in}}%
\pgfpathlineto{\pgfqpoint{3.363794in}{0.739656in}}%
\pgfpathlineto{\pgfqpoint{3.363230in}{0.739656in}}%
\pgfpathlineto{\pgfqpoint{3.362666in}{0.739656in}}%
\pgfpathlineto{\pgfqpoint{3.362101in}{0.739656in}}%
\pgfpathlineto{\pgfqpoint{3.361537in}{0.739656in}}%
\pgfpathlineto{\pgfqpoint{3.360973in}{0.739656in}}%
\pgfpathlineto{\pgfqpoint{3.360408in}{0.739656in}}%
\pgfpathlineto{\pgfqpoint{3.359844in}{0.739656in}}%
\pgfpathlineto{\pgfqpoint{3.359280in}{0.739656in}}%
\pgfpathlineto{\pgfqpoint{3.358715in}{0.739656in}}%
\pgfpathlineto{\pgfqpoint{3.358151in}{0.739656in}}%
\pgfpathlineto{\pgfqpoint{3.357587in}{0.739656in}}%
\pgfpathlineto{\pgfqpoint{3.357022in}{0.739656in}}%
\pgfpathlineto{\pgfqpoint{3.356458in}{0.739656in}}%
\pgfpathlineto{\pgfqpoint{3.355894in}{0.739656in}}%
\pgfpathlineto{\pgfqpoint{3.355329in}{0.739656in}}%
\pgfpathlineto{\pgfqpoint{3.354765in}{0.739656in}}%
\pgfpathlineto{\pgfqpoint{3.354200in}{0.739656in}}%
\pgfpathlineto{\pgfqpoint{3.353636in}{0.739656in}}%
\pgfpathlineto{\pgfqpoint{3.353072in}{0.739656in}}%
\pgfpathlineto{\pgfqpoint{3.352507in}{0.739656in}}%
\pgfpathlineto{\pgfqpoint{3.351943in}{0.739656in}}%
\pgfpathlineto{\pgfqpoint{3.351379in}{0.739656in}}%
\pgfpathlineto{\pgfqpoint{3.350814in}{0.739656in}}%
\pgfpathlineto{\pgfqpoint{3.350250in}{0.739656in}}%
\pgfpathlineto{\pgfqpoint{3.349686in}{0.739656in}}%
\pgfpathlineto{\pgfqpoint{3.349121in}{0.739656in}}%
\pgfpathlineto{\pgfqpoint{3.348557in}{0.739656in}}%
\pgfpathlineto{\pgfqpoint{3.347993in}{0.739656in}}%
\pgfpathlineto{\pgfqpoint{3.347428in}{0.739656in}}%
\pgfpathlineto{\pgfqpoint{3.346864in}{0.739656in}}%
\pgfpathlineto{\pgfqpoint{3.346300in}{0.739656in}}%
\pgfpathlineto{\pgfqpoint{3.345735in}{0.739656in}}%
\pgfpathlineto{\pgfqpoint{3.345171in}{0.739656in}}%
\pgfpathlineto{\pgfqpoint{3.344607in}{0.739656in}}%
\pgfpathlineto{\pgfqpoint{3.344042in}{0.739656in}}%
\pgfpathlineto{\pgfqpoint{3.343478in}{0.739656in}}%
\pgfpathlineto{\pgfqpoint{3.342914in}{0.739656in}}%
\pgfpathlineto{\pgfqpoint{3.342349in}{0.739656in}}%
\pgfpathlineto{\pgfqpoint{3.341785in}{0.739656in}}%
\pgfpathlineto{\pgfqpoint{3.341221in}{0.739656in}}%
\pgfpathlineto{\pgfqpoint{3.340656in}{0.739656in}}%
\pgfpathlineto{\pgfqpoint{3.340092in}{0.739656in}}%
\pgfpathlineto{\pgfqpoint{3.339528in}{0.739656in}}%
\pgfpathlineto{\pgfqpoint{3.338963in}{0.739656in}}%
\pgfpathlineto{\pgfqpoint{3.338399in}{0.739656in}}%
\pgfpathlineto{\pgfqpoint{3.337834in}{0.739656in}}%
\pgfpathlineto{\pgfqpoint{3.337270in}{0.739656in}}%
\pgfpathlineto{\pgfqpoint{3.336706in}{0.739656in}}%
\pgfpathlineto{\pgfqpoint{3.336141in}{0.739656in}}%
\pgfpathlineto{\pgfqpoint{3.335577in}{0.739656in}}%
\pgfpathlineto{\pgfqpoint{3.335013in}{0.739656in}}%
\pgfpathlineto{\pgfqpoint{3.334448in}{0.739656in}}%
\pgfpathlineto{\pgfqpoint{3.333884in}{0.739656in}}%
\pgfpathlineto{\pgfqpoint{3.333320in}{0.739656in}}%
\pgfpathlineto{\pgfqpoint{3.332755in}{0.739656in}}%
\pgfpathlineto{\pgfqpoint{3.332191in}{0.739656in}}%
\pgfpathlineto{\pgfqpoint{3.331627in}{0.739656in}}%
\pgfpathlineto{\pgfqpoint{3.331062in}{0.739656in}}%
\pgfpathlineto{\pgfqpoint{3.330498in}{0.739656in}}%
\pgfpathlineto{\pgfqpoint{3.329934in}{0.739656in}}%
\pgfpathlineto{\pgfqpoint{3.329369in}{0.739656in}}%
\pgfpathlineto{\pgfqpoint{3.328805in}{0.739656in}}%
\pgfpathlineto{\pgfqpoint{3.328241in}{0.739656in}}%
\pgfpathlineto{\pgfqpoint{3.327676in}{0.739656in}}%
\pgfpathlineto{\pgfqpoint{3.327112in}{0.739656in}}%
\pgfpathlineto{\pgfqpoint{3.326548in}{0.739656in}}%
\pgfpathlineto{\pgfqpoint{3.325983in}{0.739656in}}%
\pgfpathlineto{\pgfqpoint{3.325419in}{0.739656in}}%
\pgfpathlineto{\pgfqpoint{3.324855in}{0.739656in}}%
\pgfpathlineto{\pgfqpoint{3.324290in}{0.739656in}}%
\pgfpathlineto{\pgfqpoint{3.323726in}{0.739656in}}%
\pgfpathlineto{\pgfqpoint{3.323161in}{0.739656in}}%
\pgfpathlineto{\pgfqpoint{3.322597in}{0.739656in}}%
\pgfpathlineto{\pgfqpoint{3.322033in}{0.739656in}}%
\pgfpathlineto{\pgfqpoint{3.321468in}{0.739656in}}%
\pgfpathlineto{\pgfqpoint{3.320904in}{0.739656in}}%
\pgfpathlineto{\pgfqpoint{3.320340in}{0.739656in}}%
\pgfpathlineto{\pgfqpoint{3.319775in}{0.739656in}}%
\pgfpathlineto{\pgfqpoint{3.319211in}{0.739656in}}%
\pgfpathlineto{\pgfqpoint{3.318647in}{0.739656in}}%
\pgfpathlineto{\pgfqpoint{3.318082in}{0.739656in}}%
\pgfpathlineto{\pgfqpoint{3.317518in}{0.739656in}}%
\pgfpathlineto{\pgfqpoint{3.316954in}{0.739656in}}%
\pgfpathlineto{\pgfqpoint{3.316389in}{0.739656in}}%
\pgfpathlineto{\pgfqpoint{3.315825in}{0.739656in}}%
\pgfpathlineto{\pgfqpoint{3.315261in}{0.739656in}}%
\pgfpathlineto{\pgfqpoint{3.314696in}{0.739656in}}%
\pgfpathlineto{\pgfqpoint{3.314132in}{0.739656in}}%
\pgfpathlineto{\pgfqpoint{3.313568in}{0.739656in}}%
\pgfpathlineto{\pgfqpoint{3.313003in}{0.739656in}}%
\pgfpathlineto{\pgfqpoint{3.312439in}{0.739656in}}%
\pgfpathlineto{\pgfqpoint{3.311875in}{0.739656in}}%
\pgfpathlineto{\pgfqpoint{3.311310in}{0.739656in}}%
\pgfpathlineto{\pgfqpoint{3.310746in}{0.739656in}}%
\pgfpathlineto{\pgfqpoint{3.310182in}{0.739656in}}%
\pgfpathlineto{\pgfqpoint{3.309617in}{0.739656in}}%
\pgfpathlineto{\pgfqpoint{3.309053in}{0.739656in}}%
\pgfpathlineto{\pgfqpoint{3.308488in}{0.739656in}}%
\pgfpathlineto{\pgfqpoint{3.307924in}{0.739656in}}%
\pgfpathlineto{\pgfqpoint{3.307360in}{0.739656in}}%
\pgfpathlineto{\pgfqpoint{3.306795in}{0.739656in}}%
\pgfpathlineto{\pgfqpoint{3.306231in}{0.739656in}}%
\pgfpathlineto{\pgfqpoint{3.305667in}{0.739656in}}%
\pgfpathlineto{\pgfqpoint{3.305102in}{0.739656in}}%
\pgfpathlineto{\pgfqpoint{3.304538in}{0.739656in}}%
\pgfpathlineto{\pgfqpoint{3.303974in}{0.739656in}}%
\pgfpathlineto{\pgfqpoint{3.303409in}{0.739656in}}%
\pgfpathlineto{\pgfqpoint{3.302845in}{0.739656in}}%
\pgfpathlineto{\pgfqpoint{3.302281in}{0.739656in}}%
\pgfpathlineto{\pgfqpoint{3.301716in}{0.739656in}}%
\pgfpathlineto{\pgfqpoint{3.301152in}{0.739656in}}%
\pgfpathlineto{\pgfqpoint{3.300588in}{0.739656in}}%
\pgfpathlineto{\pgfqpoint{3.300023in}{0.739656in}}%
\pgfpathlineto{\pgfqpoint{3.299459in}{0.739656in}}%
\pgfpathlineto{\pgfqpoint{3.298895in}{0.739656in}}%
\pgfpathlineto{\pgfqpoint{3.298330in}{0.739656in}}%
\pgfpathlineto{\pgfqpoint{3.297766in}{0.739656in}}%
\pgfpathlineto{\pgfqpoint{3.297202in}{0.739656in}}%
\pgfpathlineto{\pgfqpoint{3.296637in}{0.739656in}}%
\pgfpathlineto{\pgfqpoint{3.296073in}{0.739656in}}%
\pgfpathlineto{\pgfqpoint{3.295509in}{0.739656in}}%
\pgfpathlineto{\pgfqpoint{3.294944in}{0.739656in}}%
\pgfpathlineto{\pgfqpoint{3.294380in}{0.739656in}}%
\pgfpathlineto{\pgfqpoint{3.293816in}{0.739656in}}%
\pgfpathlineto{\pgfqpoint{3.293251in}{0.739656in}}%
\pgfpathlineto{\pgfqpoint{3.292687in}{0.739656in}}%
\pgfpathlineto{\pgfqpoint{3.292122in}{0.739656in}}%
\pgfpathlineto{\pgfqpoint{3.291558in}{0.739656in}}%
\pgfpathlineto{\pgfqpoint{3.290994in}{0.739656in}}%
\pgfpathlineto{\pgfqpoint{3.290429in}{0.739656in}}%
\pgfpathlineto{\pgfqpoint{3.289865in}{0.739656in}}%
\pgfpathlineto{\pgfqpoint{3.289301in}{0.739656in}}%
\pgfpathlineto{\pgfqpoint{3.288736in}{0.739656in}}%
\pgfpathlineto{\pgfqpoint{3.288172in}{0.739656in}}%
\pgfpathlineto{\pgfqpoint{3.287608in}{0.739656in}}%
\pgfpathlineto{\pgfqpoint{3.287043in}{0.739656in}}%
\pgfpathlineto{\pgfqpoint{3.286479in}{0.739656in}}%
\pgfpathlineto{\pgfqpoint{3.285915in}{0.739656in}}%
\pgfpathlineto{\pgfqpoint{3.285350in}{0.739656in}}%
\pgfpathlineto{\pgfqpoint{3.284786in}{0.739656in}}%
\pgfpathlineto{\pgfqpoint{3.284222in}{0.739656in}}%
\pgfpathlineto{\pgfqpoint{3.283657in}{0.739656in}}%
\pgfpathlineto{\pgfqpoint{3.283093in}{0.739656in}}%
\pgfpathlineto{\pgfqpoint{3.282529in}{0.739656in}}%
\pgfpathlineto{\pgfqpoint{3.281964in}{0.739656in}}%
\pgfpathlineto{\pgfqpoint{3.281400in}{0.739656in}}%
\pgfpathlineto{\pgfqpoint{3.280836in}{0.739656in}}%
\pgfpathlineto{\pgfqpoint{3.280271in}{0.739656in}}%
\pgfpathlineto{\pgfqpoint{3.279707in}{0.739656in}}%
\pgfpathlineto{\pgfqpoint{3.279143in}{0.739656in}}%
\pgfpathlineto{\pgfqpoint{3.278578in}{0.739656in}}%
\pgfpathlineto{\pgfqpoint{3.278014in}{0.739656in}}%
\pgfpathlineto{\pgfqpoint{3.277449in}{0.739656in}}%
\pgfpathlineto{\pgfqpoint{3.276885in}{0.739656in}}%
\pgfpathlineto{\pgfqpoint{3.276321in}{0.739656in}}%
\pgfpathlineto{\pgfqpoint{3.275756in}{0.739656in}}%
\pgfpathlineto{\pgfqpoint{3.275192in}{0.739656in}}%
\pgfpathlineto{\pgfqpoint{3.274628in}{0.739656in}}%
\pgfpathlineto{\pgfqpoint{3.274063in}{0.739656in}}%
\pgfpathlineto{\pgfqpoint{3.273499in}{0.739656in}}%
\pgfpathlineto{\pgfqpoint{3.272935in}{0.739656in}}%
\pgfpathlineto{\pgfqpoint{3.272370in}{0.739656in}}%
\pgfpathlineto{\pgfqpoint{3.271806in}{0.739656in}}%
\pgfpathlineto{\pgfqpoint{3.271242in}{0.739656in}}%
\pgfpathlineto{\pgfqpoint{3.270677in}{0.739656in}}%
\pgfpathlineto{\pgfqpoint{3.270113in}{0.739656in}}%
\pgfpathlineto{\pgfqpoint{3.269549in}{0.739656in}}%
\pgfpathlineto{\pgfqpoint{3.268984in}{0.739656in}}%
\pgfpathlineto{\pgfqpoint{3.268420in}{0.739656in}}%
\pgfpathlineto{\pgfqpoint{3.267856in}{0.739656in}}%
\pgfpathlineto{\pgfqpoint{3.267291in}{0.739656in}}%
\pgfpathlineto{\pgfqpoint{3.266727in}{0.739656in}}%
\pgfpathlineto{\pgfqpoint{3.266163in}{0.739656in}}%
\pgfpathlineto{\pgfqpoint{3.265598in}{0.739656in}}%
\pgfpathlineto{\pgfqpoint{3.265034in}{0.739656in}}%
\pgfpathlineto{\pgfqpoint{3.264470in}{0.739656in}}%
\pgfpathlineto{\pgfqpoint{3.263905in}{0.739656in}}%
\pgfpathlineto{\pgfqpoint{3.263341in}{0.739656in}}%
\pgfpathlineto{\pgfqpoint{3.262776in}{0.739656in}}%
\pgfpathlineto{\pgfqpoint{3.262212in}{0.739656in}}%
\pgfpathlineto{\pgfqpoint{3.261648in}{0.739656in}}%
\pgfpathlineto{\pgfqpoint{3.261083in}{0.739656in}}%
\pgfpathlineto{\pgfqpoint{3.260519in}{0.739656in}}%
\pgfpathlineto{\pgfqpoint{3.259955in}{0.739656in}}%
\pgfpathlineto{\pgfqpoint{3.259390in}{0.739656in}}%
\pgfpathlineto{\pgfqpoint{3.258826in}{0.739656in}}%
\pgfpathlineto{\pgfqpoint{3.258262in}{0.739656in}}%
\pgfpathlineto{\pgfqpoint{3.257697in}{0.739656in}}%
\pgfpathlineto{\pgfqpoint{3.257133in}{0.739656in}}%
\pgfpathlineto{\pgfqpoint{3.256569in}{0.739656in}}%
\pgfpathlineto{\pgfqpoint{3.256004in}{0.739656in}}%
\pgfpathlineto{\pgfqpoint{3.255440in}{0.739656in}}%
\pgfpathlineto{\pgfqpoint{3.254876in}{0.739656in}}%
\pgfpathlineto{\pgfqpoint{3.254311in}{0.739656in}}%
\pgfpathlineto{\pgfqpoint{3.253747in}{0.739656in}}%
\pgfpathlineto{\pgfqpoint{3.253183in}{0.739656in}}%
\pgfpathlineto{\pgfqpoint{3.252618in}{0.739656in}}%
\pgfpathlineto{\pgfqpoint{3.252054in}{0.739656in}}%
\pgfpathlineto{\pgfqpoint{3.251490in}{0.739656in}}%
\pgfpathlineto{\pgfqpoint{3.250925in}{0.739656in}}%
\pgfpathlineto{\pgfqpoint{3.250361in}{0.739656in}}%
\pgfpathlineto{\pgfqpoint{3.249797in}{0.739656in}}%
\pgfpathlineto{\pgfqpoint{3.249232in}{0.739656in}}%
\pgfpathlineto{\pgfqpoint{3.248668in}{0.739656in}}%
\pgfpathlineto{\pgfqpoint{3.248104in}{0.739656in}}%
\pgfpathlineto{\pgfqpoint{3.247539in}{0.739656in}}%
\pgfpathlineto{\pgfqpoint{3.246975in}{0.739656in}}%
\pgfpathlineto{\pgfqpoint{3.246410in}{0.739656in}}%
\pgfpathlineto{\pgfqpoint{3.245846in}{0.739656in}}%
\pgfpathlineto{\pgfqpoint{3.245282in}{0.739656in}}%
\pgfpathlineto{\pgfqpoint{3.244717in}{0.739656in}}%
\pgfpathlineto{\pgfqpoint{3.244153in}{0.739656in}}%
\pgfpathlineto{\pgfqpoint{3.243589in}{0.739656in}}%
\pgfpathlineto{\pgfqpoint{3.243024in}{0.739656in}}%
\pgfpathlineto{\pgfqpoint{3.242460in}{0.739656in}}%
\pgfpathlineto{\pgfqpoint{3.241896in}{0.739656in}}%
\pgfpathlineto{\pgfqpoint{3.241331in}{0.739656in}}%
\pgfpathlineto{\pgfqpoint{3.240767in}{0.739656in}}%
\pgfpathlineto{\pgfqpoint{3.240203in}{0.739656in}}%
\pgfpathlineto{\pgfqpoint{3.239638in}{0.739656in}}%
\pgfpathlineto{\pgfqpoint{3.239074in}{0.739656in}}%
\pgfpathlineto{\pgfqpoint{3.238510in}{0.739656in}}%
\pgfpathlineto{\pgfqpoint{3.237945in}{0.739656in}}%
\pgfpathlineto{\pgfqpoint{3.237381in}{0.739656in}}%
\pgfpathlineto{\pgfqpoint{3.236817in}{0.739656in}}%
\pgfpathlineto{\pgfqpoint{3.236252in}{0.739656in}}%
\pgfpathlineto{\pgfqpoint{3.235688in}{0.739656in}}%
\pgfpathlineto{\pgfqpoint{3.235124in}{0.739656in}}%
\pgfpathlineto{\pgfqpoint{3.234559in}{0.739656in}}%
\pgfpathlineto{\pgfqpoint{3.233995in}{0.739656in}}%
\pgfpathlineto{\pgfqpoint{3.233431in}{0.739656in}}%
\pgfpathlineto{\pgfqpoint{3.232866in}{0.739656in}}%
\pgfpathlineto{\pgfqpoint{3.232302in}{0.739656in}}%
\pgfpathlineto{\pgfqpoint{3.231737in}{0.739656in}}%
\pgfpathlineto{\pgfqpoint{3.231173in}{0.739656in}}%
\pgfpathlineto{\pgfqpoint{3.230609in}{0.739656in}}%
\pgfpathlineto{\pgfqpoint{3.230044in}{0.739656in}}%
\pgfpathlineto{\pgfqpoint{3.229480in}{0.739656in}}%
\pgfpathlineto{\pgfqpoint{3.228916in}{0.739656in}}%
\pgfpathlineto{\pgfqpoint{3.228351in}{0.739656in}}%
\pgfpathlineto{\pgfqpoint{3.227787in}{0.739656in}}%
\pgfpathlineto{\pgfqpoint{3.227223in}{0.739656in}}%
\pgfpathlineto{\pgfqpoint{3.226658in}{0.739656in}}%
\pgfpathlineto{\pgfqpoint{3.226094in}{0.739656in}}%
\pgfpathlineto{\pgfqpoint{3.225530in}{0.739656in}}%
\pgfpathlineto{\pgfqpoint{3.224965in}{0.739656in}}%
\pgfpathlineto{\pgfqpoint{3.224401in}{0.739656in}}%
\pgfpathlineto{\pgfqpoint{3.223837in}{0.739656in}}%
\pgfpathlineto{\pgfqpoint{3.223272in}{0.739656in}}%
\pgfpathlineto{\pgfqpoint{3.222708in}{0.739656in}}%
\pgfpathlineto{\pgfqpoint{3.222144in}{0.739656in}}%
\pgfpathlineto{\pgfqpoint{3.221579in}{0.739656in}}%
\pgfpathlineto{\pgfqpoint{3.221015in}{0.739656in}}%
\pgfpathlineto{\pgfqpoint{3.220451in}{0.739656in}}%
\pgfpathlineto{\pgfqpoint{3.219886in}{0.739656in}}%
\pgfpathlineto{\pgfqpoint{3.219322in}{0.739656in}}%
\pgfpathlineto{\pgfqpoint{3.218758in}{0.739656in}}%
\pgfpathlineto{\pgfqpoint{3.218193in}{0.739656in}}%
\pgfpathlineto{\pgfqpoint{3.217629in}{0.739656in}}%
\pgfpathlineto{\pgfqpoint{3.217064in}{0.739656in}}%
\pgfpathlineto{\pgfqpoint{3.216500in}{0.739656in}}%
\pgfpathlineto{\pgfqpoint{3.215936in}{0.739656in}}%
\pgfpathlineto{\pgfqpoint{3.215371in}{0.739656in}}%
\pgfpathlineto{\pgfqpoint{3.214807in}{0.739656in}}%
\pgfpathlineto{\pgfqpoint{3.214243in}{0.739656in}}%
\pgfpathlineto{\pgfqpoint{3.213678in}{0.739656in}}%
\pgfpathlineto{\pgfqpoint{3.213114in}{0.739656in}}%
\pgfpathlineto{\pgfqpoint{3.212550in}{0.739656in}}%
\pgfpathlineto{\pgfqpoint{3.211985in}{0.739656in}}%
\pgfpathlineto{\pgfqpoint{3.211421in}{0.739656in}}%
\pgfpathlineto{\pgfqpoint{3.210857in}{0.739656in}}%
\pgfpathlineto{\pgfqpoint{3.210292in}{0.739656in}}%
\pgfpathlineto{\pgfqpoint{3.209728in}{0.739656in}}%
\pgfpathlineto{\pgfqpoint{3.209164in}{0.739656in}}%
\pgfpathlineto{\pgfqpoint{3.208599in}{0.739656in}}%
\pgfpathlineto{\pgfqpoint{3.208035in}{0.739656in}}%
\pgfpathlineto{\pgfqpoint{3.207471in}{0.739656in}}%
\pgfpathlineto{\pgfqpoint{3.206906in}{0.739656in}}%
\pgfpathlineto{\pgfqpoint{3.206342in}{0.739656in}}%
\pgfpathlineto{\pgfqpoint{3.205778in}{0.739656in}}%
\pgfpathlineto{\pgfqpoint{3.205213in}{0.739656in}}%
\pgfpathlineto{\pgfqpoint{3.204649in}{0.739656in}}%
\pgfpathlineto{\pgfqpoint{3.204085in}{0.739656in}}%
\pgfpathlineto{\pgfqpoint{3.203520in}{0.739656in}}%
\pgfpathlineto{\pgfqpoint{3.202956in}{0.739656in}}%
\pgfpathlineto{\pgfqpoint{3.202392in}{0.739656in}}%
\pgfpathlineto{\pgfqpoint{3.201827in}{0.739656in}}%
\pgfpathlineto{\pgfqpoint{3.201263in}{0.739656in}}%
\pgfpathlineto{\pgfqpoint{3.200698in}{0.739656in}}%
\pgfpathlineto{\pgfqpoint{3.200134in}{0.739656in}}%
\pgfpathlineto{\pgfqpoint{3.199570in}{0.739656in}}%
\pgfpathlineto{\pgfqpoint{3.199005in}{0.739656in}}%
\pgfpathlineto{\pgfqpoint{3.198441in}{0.739656in}}%
\pgfpathlineto{\pgfqpoint{3.197877in}{0.739656in}}%
\pgfpathlineto{\pgfqpoint{3.197312in}{0.739656in}}%
\pgfpathlineto{\pgfqpoint{3.196748in}{0.739656in}}%
\pgfpathlineto{\pgfqpoint{3.196184in}{0.739656in}}%
\pgfpathlineto{\pgfqpoint{3.195619in}{0.739656in}}%
\pgfpathlineto{\pgfqpoint{3.195055in}{0.739656in}}%
\pgfpathlineto{\pgfqpoint{3.194491in}{0.739656in}}%
\pgfpathlineto{\pgfqpoint{3.193926in}{0.739656in}}%
\pgfpathlineto{\pgfqpoint{3.193362in}{0.739656in}}%
\pgfpathlineto{\pgfqpoint{3.192798in}{0.739656in}}%
\pgfpathlineto{\pgfqpoint{3.192233in}{0.739656in}}%
\pgfpathlineto{\pgfqpoint{3.191669in}{0.739656in}}%
\pgfpathlineto{\pgfqpoint{3.191105in}{0.739656in}}%
\pgfpathlineto{\pgfqpoint{3.190540in}{0.739656in}}%
\pgfpathlineto{\pgfqpoint{3.189976in}{0.739656in}}%
\pgfpathlineto{\pgfqpoint{3.189412in}{0.739656in}}%
\pgfpathlineto{\pgfqpoint{3.188847in}{0.739656in}}%
\pgfpathlineto{\pgfqpoint{3.188283in}{0.739656in}}%
\pgfpathlineto{\pgfqpoint{3.187719in}{0.739656in}}%
\pgfpathlineto{\pgfqpoint{3.187154in}{0.739656in}}%
\pgfpathlineto{\pgfqpoint{3.186590in}{0.739656in}}%
\pgfpathlineto{\pgfqpoint{3.186025in}{0.739656in}}%
\pgfpathlineto{\pgfqpoint{3.185461in}{0.739656in}}%
\pgfpathlineto{\pgfqpoint{3.184897in}{0.739656in}}%
\pgfpathlineto{\pgfqpoint{3.184332in}{0.739656in}}%
\pgfpathlineto{\pgfqpoint{3.183768in}{0.739656in}}%
\pgfpathlineto{\pgfqpoint{3.183204in}{0.739656in}}%
\pgfpathlineto{\pgfqpoint{3.182639in}{0.739656in}}%
\pgfpathlineto{\pgfqpoint{3.182075in}{0.739656in}}%
\pgfpathlineto{\pgfqpoint{3.181511in}{0.739656in}}%
\pgfpathlineto{\pgfqpoint{3.180946in}{0.739656in}}%
\pgfpathlineto{\pgfqpoint{3.180382in}{0.739656in}}%
\pgfpathlineto{\pgfqpoint{3.179818in}{0.739656in}}%
\pgfpathlineto{\pgfqpoint{3.179253in}{0.739656in}}%
\pgfpathlineto{\pgfqpoint{3.178689in}{0.739656in}}%
\pgfpathlineto{\pgfqpoint{3.178125in}{0.739656in}}%
\pgfpathlineto{\pgfqpoint{3.177560in}{0.739656in}}%
\pgfpathlineto{\pgfqpoint{3.176996in}{0.739656in}}%
\pgfpathlineto{\pgfqpoint{3.176432in}{0.739656in}}%
\pgfpathlineto{\pgfqpoint{3.175867in}{0.739656in}}%
\pgfpathlineto{\pgfqpoint{3.175303in}{0.739656in}}%
\pgfpathlineto{\pgfqpoint{3.174739in}{0.739656in}}%
\pgfpathlineto{\pgfqpoint{3.174174in}{0.739656in}}%
\pgfpathlineto{\pgfqpoint{3.173610in}{0.739656in}}%
\pgfpathlineto{\pgfqpoint{3.173046in}{0.739656in}}%
\pgfpathlineto{\pgfqpoint{3.172481in}{0.739656in}}%
\pgfpathlineto{\pgfqpoint{3.171917in}{0.739656in}}%
\pgfpathlineto{\pgfqpoint{3.171352in}{0.739656in}}%
\pgfpathlineto{\pgfqpoint{3.170788in}{0.739656in}}%
\pgfpathlineto{\pgfqpoint{3.170224in}{0.739656in}}%
\pgfpathlineto{\pgfqpoint{3.169659in}{0.739656in}}%
\pgfpathlineto{\pgfqpoint{3.169095in}{0.739656in}}%
\pgfpathlineto{\pgfqpoint{3.168531in}{0.739656in}}%
\pgfpathlineto{\pgfqpoint{3.167966in}{0.739656in}}%
\pgfpathlineto{\pgfqpoint{3.167402in}{0.739656in}}%
\pgfpathlineto{\pgfqpoint{3.166838in}{0.739656in}}%
\pgfpathlineto{\pgfqpoint{3.166273in}{0.739656in}}%
\pgfpathlineto{\pgfqpoint{3.165709in}{0.739656in}}%
\pgfpathlineto{\pgfqpoint{3.165145in}{0.739656in}}%
\pgfpathlineto{\pgfqpoint{3.164580in}{0.739656in}}%
\pgfpathlineto{\pgfqpoint{3.164016in}{0.739656in}}%
\pgfpathlineto{\pgfqpoint{3.163452in}{0.739656in}}%
\pgfpathlineto{\pgfqpoint{3.162887in}{0.739656in}}%
\pgfpathlineto{\pgfqpoint{3.162323in}{0.739656in}}%
\pgfpathlineto{\pgfqpoint{3.161759in}{0.739656in}}%
\pgfpathlineto{\pgfqpoint{3.161194in}{0.739656in}}%
\pgfpathlineto{\pgfqpoint{3.160630in}{0.739656in}}%
\pgfpathlineto{\pgfqpoint{3.160066in}{0.739656in}}%
\pgfpathlineto{\pgfqpoint{3.159501in}{0.739656in}}%
\pgfpathlineto{\pgfqpoint{3.158937in}{0.739656in}}%
\pgfpathlineto{\pgfqpoint{3.158373in}{0.739656in}}%
\pgfpathlineto{\pgfqpoint{3.157808in}{0.739656in}}%
\pgfpathlineto{\pgfqpoint{3.157244in}{0.739656in}}%
\pgfpathlineto{\pgfqpoint{3.156680in}{0.739656in}}%
\pgfpathlineto{\pgfqpoint{3.156115in}{0.739656in}}%
\pgfpathlineto{\pgfqpoint{3.155551in}{0.739656in}}%
\pgfpathlineto{\pgfqpoint{3.154986in}{0.739656in}}%
\pgfpathlineto{\pgfqpoint{3.154422in}{0.739656in}}%
\pgfpathlineto{\pgfqpoint{3.153858in}{0.739656in}}%
\pgfpathlineto{\pgfqpoint{3.153293in}{0.739656in}}%
\pgfpathlineto{\pgfqpoint{3.152729in}{0.739656in}}%
\pgfpathlineto{\pgfqpoint{3.152165in}{0.739656in}}%
\pgfpathlineto{\pgfqpoint{3.151600in}{0.739656in}}%
\pgfpathlineto{\pgfqpoint{3.151036in}{0.739656in}}%
\pgfpathlineto{\pgfqpoint{3.150472in}{0.739656in}}%
\pgfpathlineto{\pgfqpoint{3.149907in}{0.739656in}}%
\pgfpathlineto{\pgfqpoint{3.149343in}{0.739656in}}%
\pgfpathlineto{\pgfqpoint{3.148779in}{0.739656in}}%
\pgfpathlineto{\pgfqpoint{3.148214in}{0.739656in}}%
\pgfpathlineto{\pgfqpoint{3.147650in}{0.739656in}}%
\pgfpathlineto{\pgfqpoint{3.147086in}{0.739656in}}%
\pgfpathlineto{\pgfqpoint{3.146521in}{0.739656in}}%
\pgfpathlineto{\pgfqpoint{3.145957in}{0.739656in}}%
\pgfpathlineto{\pgfqpoint{3.145393in}{0.739656in}}%
\pgfpathlineto{\pgfqpoint{3.144828in}{0.739656in}}%
\pgfpathlineto{\pgfqpoint{3.144264in}{0.739656in}}%
\pgfpathlineto{\pgfqpoint{3.143700in}{0.739656in}}%
\pgfpathlineto{\pgfqpoint{3.143135in}{0.739656in}}%
\pgfpathlineto{\pgfqpoint{3.142571in}{0.739656in}}%
\pgfpathlineto{\pgfqpoint{3.142007in}{0.739656in}}%
\pgfpathlineto{\pgfqpoint{3.141442in}{0.739656in}}%
\pgfpathlineto{\pgfqpoint{3.140878in}{0.739656in}}%
\pgfpathlineto{\pgfqpoint{3.140313in}{0.739656in}}%
\pgfpathlineto{\pgfqpoint{3.139749in}{0.739656in}}%
\pgfpathlineto{\pgfqpoint{3.139185in}{0.739656in}}%
\pgfpathlineto{\pgfqpoint{3.138620in}{0.739656in}}%
\pgfpathlineto{\pgfqpoint{3.138056in}{0.739656in}}%
\pgfpathlineto{\pgfqpoint{3.137492in}{0.739656in}}%
\pgfpathlineto{\pgfqpoint{3.136927in}{0.739656in}}%
\pgfpathlineto{\pgfqpoint{3.136363in}{0.739656in}}%
\pgfpathlineto{\pgfqpoint{3.135799in}{0.739656in}}%
\pgfpathlineto{\pgfqpoint{3.135234in}{0.739656in}}%
\pgfpathlineto{\pgfqpoint{3.134670in}{0.739656in}}%
\pgfpathlineto{\pgfqpoint{3.134106in}{0.739656in}}%
\pgfpathlineto{\pgfqpoint{3.133541in}{0.739656in}}%
\pgfpathlineto{\pgfqpoint{3.132977in}{0.739656in}}%
\pgfpathlineto{\pgfqpoint{3.132413in}{0.739656in}}%
\pgfpathlineto{\pgfqpoint{3.131848in}{0.739656in}}%
\pgfpathlineto{\pgfqpoint{3.131284in}{0.739656in}}%
\pgfpathlineto{\pgfqpoint{3.130720in}{0.739656in}}%
\pgfpathlineto{\pgfqpoint{3.130155in}{0.739656in}}%
\pgfpathlineto{\pgfqpoint{3.129591in}{0.739656in}}%
\pgfpathlineto{\pgfqpoint{3.129027in}{0.739656in}}%
\pgfpathlineto{\pgfqpoint{3.128462in}{0.739656in}}%
\pgfpathlineto{\pgfqpoint{3.127898in}{0.739656in}}%
\pgfpathlineto{\pgfqpoint{3.127334in}{0.739656in}}%
\pgfpathlineto{\pgfqpoint{3.126769in}{0.739656in}}%
\pgfpathlineto{\pgfqpoint{3.126205in}{0.739656in}}%
\pgfpathlineto{\pgfqpoint{3.125640in}{0.739656in}}%
\pgfpathlineto{\pgfqpoint{3.125076in}{0.739656in}}%
\pgfpathlineto{\pgfqpoint{3.124512in}{0.739656in}}%
\pgfpathlineto{\pgfqpoint{3.123947in}{0.739656in}}%
\pgfpathlineto{\pgfqpoint{3.123383in}{0.739656in}}%
\pgfpathlineto{\pgfqpoint{3.122819in}{0.739656in}}%
\pgfpathlineto{\pgfqpoint{3.122254in}{0.739656in}}%
\pgfpathlineto{\pgfqpoint{3.121690in}{0.739656in}}%
\pgfpathlineto{\pgfqpoint{3.121126in}{0.739656in}}%
\pgfpathlineto{\pgfqpoint{3.120561in}{0.739656in}}%
\pgfpathlineto{\pgfqpoint{3.119997in}{0.739656in}}%
\pgfpathlineto{\pgfqpoint{3.119433in}{0.739656in}}%
\pgfpathlineto{\pgfqpoint{3.118868in}{0.739656in}}%
\pgfpathlineto{\pgfqpoint{3.118304in}{0.739656in}}%
\pgfpathlineto{\pgfqpoint{3.117740in}{0.739656in}}%
\pgfpathlineto{\pgfqpoint{3.117175in}{0.739656in}}%
\pgfpathlineto{\pgfqpoint{3.116611in}{0.739656in}}%
\pgfpathlineto{\pgfqpoint{3.116047in}{0.739656in}}%
\pgfpathlineto{\pgfqpoint{3.115482in}{0.739656in}}%
\pgfpathlineto{\pgfqpoint{3.114918in}{0.739656in}}%
\pgfpathlineto{\pgfqpoint{3.114354in}{0.739656in}}%
\pgfpathlineto{\pgfqpoint{3.113789in}{0.739656in}}%
\pgfpathlineto{\pgfqpoint{3.113225in}{0.739656in}}%
\pgfpathlineto{\pgfqpoint{3.112661in}{0.739656in}}%
\pgfpathlineto{\pgfqpoint{3.112096in}{0.739656in}}%
\pgfpathlineto{\pgfqpoint{3.111532in}{0.739656in}}%
\pgfpathlineto{\pgfqpoint{3.110967in}{0.739656in}}%
\pgfpathlineto{\pgfqpoint{3.110403in}{0.739656in}}%
\pgfpathlineto{\pgfqpoint{3.109839in}{0.739656in}}%
\pgfpathlineto{\pgfqpoint{3.109274in}{0.739656in}}%
\pgfpathlineto{\pgfqpoint{3.108710in}{0.739656in}}%
\pgfpathlineto{\pgfqpoint{3.108146in}{0.739656in}}%
\pgfpathlineto{\pgfqpoint{3.107581in}{0.739656in}}%
\pgfpathlineto{\pgfqpoint{3.107017in}{0.739656in}}%
\pgfpathlineto{\pgfqpoint{3.106453in}{0.739656in}}%
\pgfpathlineto{\pgfqpoint{3.105888in}{0.739656in}}%
\pgfpathlineto{\pgfqpoint{3.105324in}{0.739656in}}%
\pgfpathlineto{\pgfqpoint{3.104760in}{0.739656in}}%
\pgfpathlineto{\pgfqpoint{3.104195in}{0.739656in}}%
\pgfpathlineto{\pgfqpoint{3.103631in}{0.739656in}}%
\pgfpathlineto{\pgfqpoint{3.103067in}{0.739656in}}%
\pgfpathlineto{\pgfqpoint{3.102502in}{0.739656in}}%
\pgfpathlineto{\pgfqpoint{3.101938in}{0.739656in}}%
\pgfpathlineto{\pgfqpoint{3.101374in}{0.739656in}}%
\pgfpathlineto{\pgfqpoint{3.100809in}{0.739656in}}%
\pgfpathlineto{\pgfqpoint{3.100245in}{0.739656in}}%
\pgfpathlineto{\pgfqpoint{3.099681in}{0.739656in}}%
\pgfpathlineto{\pgfqpoint{3.099116in}{0.739656in}}%
\pgfpathlineto{\pgfqpoint{3.098552in}{0.739656in}}%
\pgfpathlineto{\pgfqpoint{3.097988in}{0.739656in}}%
\pgfpathlineto{\pgfqpoint{3.097423in}{0.739656in}}%
\pgfpathlineto{\pgfqpoint{3.096859in}{0.739656in}}%
\pgfpathlineto{\pgfqpoint{3.096295in}{0.739656in}}%
\pgfpathlineto{\pgfqpoint{3.095730in}{0.739656in}}%
\pgfpathlineto{\pgfqpoint{3.095166in}{0.739656in}}%
\pgfpathlineto{\pgfqpoint{3.094601in}{0.739656in}}%
\pgfpathlineto{\pgfqpoint{3.094037in}{0.739656in}}%
\pgfpathlineto{\pgfqpoint{3.093473in}{0.739656in}}%
\pgfpathlineto{\pgfqpoint{3.092908in}{0.739656in}}%
\pgfpathlineto{\pgfqpoint{3.092344in}{0.739656in}}%
\pgfpathlineto{\pgfqpoint{3.091780in}{0.739656in}}%
\pgfpathlineto{\pgfqpoint{3.091215in}{0.739656in}}%
\pgfpathlineto{\pgfqpoint{3.090651in}{0.739656in}}%
\pgfpathlineto{\pgfqpoint{3.090087in}{0.739656in}}%
\pgfpathlineto{\pgfqpoint{3.089522in}{0.739656in}}%
\pgfpathlineto{\pgfqpoint{3.088958in}{0.739656in}}%
\pgfpathlineto{\pgfqpoint{3.088394in}{0.739656in}}%
\pgfpathlineto{\pgfqpoint{3.087829in}{0.739656in}}%
\pgfpathlineto{\pgfqpoint{3.087265in}{0.739656in}}%
\pgfpathlineto{\pgfqpoint{3.086701in}{0.739656in}}%
\pgfpathlineto{\pgfqpoint{3.086136in}{0.739656in}}%
\pgfpathlineto{\pgfqpoint{3.085572in}{0.739656in}}%
\pgfpathlineto{\pgfqpoint{3.085008in}{0.739656in}}%
\pgfpathlineto{\pgfqpoint{3.084443in}{0.739656in}}%
\pgfpathlineto{\pgfqpoint{3.083879in}{0.739656in}}%
\pgfpathlineto{\pgfqpoint{3.083315in}{0.739656in}}%
\pgfpathlineto{\pgfqpoint{3.082750in}{0.739656in}}%
\pgfpathlineto{\pgfqpoint{3.082186in}{0.739656in}}%
\pgfpathlineto{\pgfqpoint{3.081622in}{0.739656in}}%
\pgfpathlineto{\pgfqpoint{3.081057in}{0.739656in}}%
\pgfpathlineto{\pgfqpoint{3.080493in}{0.739656in}}%
\pgfpathlineto{\pgfqpoint{3.079928in}{0.739656in}}%
\pgfpathlineto{\pgfqpoint{3.079364in}{0.739656in}}%
\pgfpathlineto{\pgfqpoint{3.078800in}{0.739656in}}%
\pgfpathlineto{\pgfqpoint{3.078235in}{0.739656in}}%
\pgfpathlineto{\pgfqpoint{3.077671in}{0.739656in}}%
\pgfpathlineto{\pgfqpoint{3.077107in}{0.739656in}}%
\pgfpathlineto{\pgfqpoint{3.076542in}{0.739656in}}%
\pgfpathlineto{\pgfqpoint{3.075978in}{0.739656in}}%
\pgfpathlineto{\pgfqpoint{3.075414in}{0.739656in}}%
\pgfpathlineto{\pgfqpoint{3.074849in}{0.739656in}}%
\pgfpathlineto{\pgfqpoint{3.074285in}{0.739656in}}%
\pgfpathlineto{\pgfqpoint{3.073721in}{0.739656in}}%
\pgfpathlineto{\pgfqpoint{3.073156in}{0.739656in}}%
\pgfpathlineto{\pgfqpoint{3.072592in}{0.739656in}}%
\pgfpathlineto{\pgfqpoint{3.072028in}{0.739656in}}%
\pgfpathlineto{\pgfqpoint{3.071463in}{0.739656in}}%
\pgfpathlineto{\pgfqpoint{3.070899in}{0.739656in}}%
\pgfpathlineto{\pgfqpoint{3.070335in}{0.739656in}}%
\pgfpathlineto{\pgfqpoint{3.069770in}{0.739656in}}%
\pgfpathlineto{\pgfqpoint{3.069206in}{0.739656in}}%
\pgfpathlineto{\pgfqpoint{3.068642in}{0.739656in}}%
\pgfpathlineto{\pgfqpoint{3.068077in}{0.739656in}}%
\pgfpathlineto{\pgfqpoint{3.067513in}{0.739656in}}%
\pgfpathlineto{\pgfqpoint{3.066949in}{0.739656in}}%
\pgfpathlineto{\pgfqpoint{3.066384in}{0.739656in}}%
\pgfpathlineto{\pgfqpoint{3.065820in}{0.739656in}}%
\pgfpathlineto{\pgfqpoint{3.065255in}{0.739656in}}%
\pgfpathlineto{\pgfqpoint{3.064691in}{0.739656in}}%
\pgfpathlineto{\pgfqpoint{3.064127in}{0.739656in}}%
\pgfpathlineto{\pgfqpoint{3.063562in}{0.739656in}}%
\pgfpathlineto{\pgfqpoint{3.062998in}{0.739656in}}%
\pgfpathlineto{\pgfqpoint{3.062434in}{0.739656in}}%
\pgfpathlineto{\pgfqpoint{3.061869in}{0.739656in}}%
\pgfpathlineto{\pgfqpoint{3.061305in}{0.739656in}}%
\pgfpathlineto{\pgfqpoint{3.060741in}{0.739656in}}%
\pgfpathlineto{\pgfqpoint{3.060176in}{0.739656in}}%
\pgfpathlineto{\pgfqpoint{3.059612in}{0.739656in}}%
\pgfpathlineto{\pgfqpoint{3.059048in}{0.739656in}}%
\pgfpathlineto{\pgfqpoint{3.058483in}{0.739656in}}%
\pgfpathlineto{\pgfqpoint{3.057919in}{0.739656in}}%
\pgfpathlineto{\pgfqpoint{3.057355in}{0.739656in}}%
\pgfpathlineto{\pgfqpoint{3.056790in}{0.739656in}}%
\pgfpathlineto{\pgfqpoint{3.056226in}{0.739656in}}%
\pgfpathlineto{\pgfqpoint{3.055662in}{0.739656in}}%
\pgfpathlineto{\pgfqpoint{3.055097in}{0.739656in}}%
\pgfpathlineto{\pgfqpoint{3.054533in}{0.739656in}}%
\pgfpathlineto{\pgfqpoint{3.053969in}{0.739656in}}%
\pgfpathlineto{\pgfqpoint{3.053404in}{0.739656in}}%
\pgfpathlineto{\pgfqpoint{3.052840in}{0.739656in}}%
\pgfpathlineto{\pgfqpoint{3.052276in}{0.739656in}}%
\pgfpathlineto{\pgfqpoint{3.051711in}{0.739656in}}%
\pgfpathlineto{\pgfqpoint{3.051147in}{0.739656in}}%
\pgfpathlineto{\pgfqpoint{3.050583in}{0.739656in}}%
\pgfpathlineto{\pgfqpoint{3.050018in}{0.739656in}}%
\pgfpathlineto{\pgfqpoint{3.049454in}{0.739656in}}%
\pgfpathlineto{\pgfqpoint{3.048889in}{0.739656in}}%
\pgfpathlineto{\pgfqpoint{3.048325in}{0.739656in}}%
\pgfpathlineto{\pgfqpoint{3.047761in}{0.739656in}}%
\pgfpathlineto{\pgfqpoint{3.047196in}{0.739656in}}%
\pgfpathlineto{\pgfqpoint{3.046632in}{0.739656in}}%
\pgfpathlineto{\pgfqpoint{3.046068in}{0.739656in}}%
\pgfpathlineto{\pgfqpoint{3.045503in}{0.739656in}}%
\pgfpathlineto{\pgfqpoint{3.044939in}{0.739656in}}%
\pgfpathlineto{\pgfqpoint{3.044375in}{0.739656in}}%
\pgfpathlineto{\pgfqpoint{3.043810in}{0.739656in}}%
\pgfpathlineto{\pgfqpoint{3.043246in}{0.739656in}}%
\pgfpathlineto{\pgfqpoint{3.042682in}{0.739656in}}%
\pgfpathlineto{\pgfqpoint{3.042117in}{0.739656in}}%
\pgfpathlineto{\pgfqpoint{3.041553in}{0.739656in}}%
\pgfpathlineto{\pgfqpoint{3.040989in}{0.739656in}}%
\pgfpathlineto{\pgfqpoint{3.040424in}{0.739656in}}%
\pgfpathlineto{\pgfqpoint{3.039860in}{0.739656in}}%
\pgfpathlineto{\pgfqpoint{3.039296in}{0.739656in}}%
\pgfpathlineto{\pgfqpoint{3.038731in}{0.739656in}}%
\pgfpathlineto{\pgfqpoint{3.038167in}{0.739656in}}%
\pgfpathlineto{\pgfqpoint{3.037603in}{0.739656in}}%
\pgfpathlineto{\pgfqpoint{3.037038in}{0.739656in}}%
\pgfpathlineto{\pgfqpoint{3.036474in}{0.739656in}}%
\pgfpathlineto{\pgfqpoint{3.035910in}{0.739656in}}%
\pgfpathlineto{\pgfqpoint{3.035345in}{0.739656in}}%
\pgfpathlineto{\pgfqpoint{3.034781in}{0.739656in}}%
\pgfpathlineto{\pgfqpoint{3.034216in}{0.739656in}}%
\pgfpathlineto{\pgfqpoint{3.033652in}{0.739656in}}%
\pgfpathlineto{\pgfqpoint{3.033088in}{0.739656in}}%
\pgfpathlineto{\pgfqpoint{3.032523in}{0.739656in}}%
\pgfpathlineto{\pgfqpoint{3.031959in}{0.739656in}}%
\pgfpathlineto{\pgfqpoint{3.031395in}{0.739656in}}%
\pgfpathlineto{\pgfqpoint{3.030830in}{0.739656in}}%
\pgfpathlineto{\pgfqpoint{3.030266in}{0.739656in}}%
\pgfpathlineto{\pgfqpoint{3.029702in}{0.739656in}}%
\pgfpathlineto{\pgfqpoint{3.029137in}{0.739656in}}%
\pgfpathlineto{\pgfqpoint{3.028573in}{0.739656in}}%
\pgfpathlineto{\pgfqpoint{3.028009in}{0.739656in}}%
\pgfpathlineto{\pgfqpoint{3.027444in}{0.739656in}}%
\pgfpathlineto{\pgfqpoint{3.026880in}{0.739656in}}%
\pgfpathlineto{\pgfqpoint{3.026316in}{0.739656in}}%
\pgfpathlineto{\pgfqpoint{3.025751in}{0.739656in}}%
\pgfpathlineto{\pgfqpoint{3.025187in}{0.739656in}}%
\pgfpathlineto{\pgfqpoint{3.024623in}{0.739656in}}%
\pgfpathlineto{\pgfqpoint{3.024058in}{0.739656in}}%
\pgfpathlineto{\pgfqpoint{3.023494in}{0.739656in}}%
\pgfpathlineto{\pgfqpoint{3.022930in}{0.739656in}}%
\pgfpathlineto{\pgfqpoint{3.022365in}{0.739656in}}%
\pgfpathlineto{\pgfqpoint{3.021801in}{0.739656in}}%
\pgfpathlineto{\pgfqpoint{3.021237in}{0.739656in}}%
\pgfpathlineto{\pgfqpoint{3.020672in}{0.739656in}}%
\pgfpathlineto{\pgfqpoint{3.020108in}{0.739656in}}%
\pgfpathlineto{\pgfqpoint{3.019543in}{0.739656in}}%
\pgfpathlineto{\pgfqpoint{3.018979in}{0.739656in}}%
\pgfpathlineto{\pgfqpoint{3.018415in}{0.739656in}}%
\pgfpathlineto{\pgfqpoint{3.017850in}{0.739656in}}%
\pgfpathlineto{\pgfqpoint{3.017286in}{0.739656in}}%
\pgfpathlineto{\pgfqpoint{3.016722in}{0.739656in}}%
\pgfpathlineto{\pgfqpoint{3.016157in}{0.739656in}}%
\pgfpathlineto{\pgfqpoint{3.015593in}{0.739656in}}%
\pgfpathlineto{\pgfqpoint{3.015029in}{0.739656in}}%
\pgfpathlineto{\pgfqpoint{3.014464in}{0.739656in}}%
\pgfpathlineto{\pgfqpoint{3.013900in}{0.739656in}}%
\pgfpathlineto{\pgfqpoint{3.013336in}{0.739656in}}%
\pgfpathlineto{\pgfqpoint{3.012771in}{0.739656in}}%
\pgfpathlineto{\pgfqpoint{3.012207in}{0.739656in}}%
\pgfpathlineto{\pgfqpoint{3.011643in}{0.739656in}}%
\pgfpathlineto{\pgfqpoint{3.011078in}{0.739656in}}%
\pgfpathlineto{\pgfqpoint{3.010514in}{0.739656in}}%
\pgfpathlineto{\pgfqpoint{3.009950in}{0.739656in}}%
\pgfpathlineto{\pgfqpoint{3.009385in}{0.739656in}}%
\pgfpathlineto{\pgfqpoint{3.008821in}{0.739656in}}%
\pgfpathlineto{\pgfqpoint{3.008257in}{0.739656in}}%
\pgfpathlineto{\pgfqpoint{3.007692in}{0.739656in}}%
\pgfpathlineto{\pgfqpoint{3.007128in}{0.739656in}}%
\pgfpathlineto{\pgfqpoint{3.006564in}{0.739656in}}%
\pgfpathlineto{\pgfqpoint{3.005999in}{0.739656in}}%
\pgfpathlineto{\pgfqpoint{3.005435in}{0.739656in}}%
\pgfpathlineto{\pgfqpoint{3.004871in}{0.739656in}}%
\pgfpathlineto{\pgfqpoint{3.004306in}{0.739656in}}%
\pgfpathlineto{\pgfqpoint{3.003742in}{0.739656in}}%
\pgfpathlineto{\pgfqpoint{3.003177in}{0.739656in}}%
\pgfpathlineto{\pgfqpoint{3.002613in}{0.739656in}}%
\pgfpathlineto{\pgfqpoint{3.002049in}{0.739656in}}%
\pgfpathlineto{\pgfqpoint{3.001484in}{0.739656in}}%
\pgfpathlineto{\pgfqpoint{3.000920in}{0.739656in}}%
\pgfpathlineto{\pgfqpoint{3.000356in}{0.739656in}}%
\pgfpathlineto{\pgfqpoint{2.999791in}{0.739656in}}%
\pgfpathlineto{\pgfqpoint{2.999227in}{0.739656in}}%
\pgfpathlineto{\pgfqpoint{2.998663in}{0.739656in}}%
\pgfpathlineto{\pgfqpoint{2.998098in}{0.739656in}}%
\pgfpathlineto{\pgfqpoint{2.997534in}{0.739656in}}%
\pgfpathlineto{\pgfqpoint{2.996970in}{0.739656in}}%
\pgfpathlineto{\pgfqpoint{2.996405in}{0.739656in}}%
\pgfpathlineto{\pgfqpoint{2.995841in}{0.739656in}}%
\pgfpathlineto{\pgfqpoint{2.995277in}{0.739656in}}%
\pgfpathlineto{\pgfqpoint{2.994712in}{0.739656in}}%
\pgfpathlineto{\pgfqpoint{2.994148in}{0.739656in}}%
\pgfpathlineto{\pgfqpoint{2.993584in}{0.739656in}}%
\pgfpathlineto{\pgfqpoint{2.993019in}{0.739656in}}%
\pgfpathlineto{\pgfqpoint{2.992455in}{0.739656in}}%
\pgfpathlineto{\pgfqpoint{2.991891in}{0.739656in}}%
\pgfpathlineto{\pgfqpoint{2.991326in}{0.739656in}}%
\pgfpathlineto{\pgfqpoint{2.990762in}{0.739656in}}%
\pgfpathlineto{\pgfqpoint{2.990198in}{0.739656in}}%
\pgfpathlineto{\pgfqpoint{2.989633in}{0.739656in}}%
\pgfpathlineto{\pgfqpoint{2.989069in}{0.739656in}}%
\pgfpathlineto{\pgfqpoint{2.988504in}{0.739656in}}%
\pgfpathlineto{\pgfqpoint{2.987940in}{0.739656in}}%
\pgfpathlineto{\pgfqpoint{2.987376in}{0.739656in}}%
\pgfpathlineto{\pgfqpoint{2.986811in}{0.739656in}}%
\pgfpathlineto{\pgfqpoint{2.986247in}{0.739656in}}%
\pgfpathlineto{\pgfqpoint{2.985683in}{0.739656in}}%
\pgfpathlineto{\pgfqpoint{2.985118in}{0.739656in}}%
\pgfpathlineto{\pgfqpoint{2.984554in}{0.739656in}}%
\pgfpathlineto{\pgfqpoint{2.983990in}{0.739656in}}%
\pgfpathlineto{\pgfqpoint{2.983425in}{0.739656in}}%
\pgfpathlineto{\pgfqpoint{2.982861in}{0.739656in}}%
\pgfpathlineto{\pgfqpoint{2.982297in}{0.739656in}}%
\pgfpathlineto{\pgfqpoint{2.981732in}{0.739656in}}%
\pgfpathlineto{\pgfqpoint{2.981168in}{0.739656in}}%
\pgfpathlineto{\pgfqpoint{2.980604in}{0.739656in}}%
\pgfpathlineto{\pgfqpoint{2.980039in}{0.739656in}}%
\pgfpathlineto{\pgfqpoint{2.979475in}{0.739656in}}%
\pgfpathlineto{\pgfqpoint{2.978911in}{0.739656in}}%
\pgfpathlineto{\pgfqpoint{2.978346in}{0.739656in}}%
\pgfpathlineto{\pgfqpoint{2.977782in}{0.739656in}}%
\pgfpathlineto{\pgfqpoint{2.977218in}{0.739656in}}%
\pgfpathlineto{\pgfqpoint{2.976653in}{0.739656in}}%
\pgfpathlineto{\pgfqpoint{2.976089in}{0.739656in}}%
\pgfpathlineto{\pgfqpoint{2.975525in}{0.739656in}}%
\pgfpathlineto{\pgfqpoint{2.974960in}{0.739656in}}%
\pgfpathlineto{\pgfqpoint{2.974396in}{0.739656in}}%
\pgfpathlineto{\pgfqpoint{2.973831in}{0.739656in}}%
\pgfpathlineto{\pgfqpoint{2.973267in}{0.739656in}}%
\pgfpathlineto{\pgfqpoint{2.972703in}{0.739656in}}%
\pgfpathlineto{\pgfqpoint{2.972138in}{0.739656in}}%
\pgfpathlineto{\pgfqpoint{2.971574in}{0.739656in}}%
\pgfpathlineto{\pgfqpoint{2.971010in}{0.739656in}}%
\pgfpathlineto{\pgfqpoint{2.970445in}{0.739656in}}%
\pgfpathlineto{\pgfqpoint{2.969881in}{0.739656in}}%
\pgfpathlineto{\pgfqpoint{2.969317in}{0.739656in}}%
\pgfpathlineto{\pgfqpoint{2.968752in}{0.739656in}}%
\pgfpathlineto{\pgfqpoint{2.968188in}{0.739656in}}%
\pgfpathlineto{\pgfqpoint{2.967624in}{0.739656in}}%
\pgfpathlineto{\pgfqpoint{2.967059in}{0.739656in}}%
\pgfpathlineto{\pgfqpoint{2.966495in}{0.739656in}}%
\pgfpathlineto{\pgfqpoint{2.965931in}{0.739656in}}%
\pgfpathlineto{\pgfqpoint{2.965366in}{0.739656in}}%
\pgfpathlineto{\pgfqpoint{2.964802in}{0.739656in}}%
\pgfpathlineto{\pgfqpoint{2.964238in}{0.739656in}}%
\pgfpathlineto{\pgfqpoint{2.963673in}{0.739656in}}%
\pgfpathlineto{\pgfqpoint{2.963109in}{0.739656in}}%
\pgfpathlineto{\pgfqpoint{2.962545in}{0.739656in}}%
\pgfpathlineto{\pgfqpoint{2.961980in}{0.739656in}}%
\pgfpathlineto{\pgfqpoint{2.961416in}{0.739656in}}%
\pgfpathlineto{\pgfqpoint{2.960852in}{0.739656in}}%
\pgfpathlineto{\pgfqpoint{2.960287in}{0.739656in}}%
\pgfpathlineto{\pgfqpoint{2.959723in}{0.739656in}}%
\pgfpathlineto{\pgfqpoint{2.959159in}{0.739656in}}%
\pgfpathlineto{\pgfqpoint{2.958594in}{0.739656in}}%
\pgfpathlineto{\pgfqpoint{2.958030in}{0.739656in}}%
\pgfpathlineto{\pgfqpoint{2.957465in}{0.739656in}}%
\pgfpathlineto{\pgfqpoint{2.956901in}{0.739656in}}%
\pgfpathlineto{\pgfqpoint{2.956337in}{0.739656in}}%
\pgfpathlineto{\pgfqpoint{2.955772in}{0.739656in}}%
\pgfpathlineto{\pgfqpoint{2.955208in}{0.739656in}}%
\pgfpathlineto{\pgfqpoint{2.954644in}{0.739656in}}%
\pgfpathlineto{\pgfqpoint{2.954079in}{0.739656in}}%
\pgfpathlineto{\pgfqpoint{2.953515in}{0.739656in}}%
\pgfpathlineto{\pgfqpoint{2.952951in}{0.739656in}}%
\pgfpathlineto{\pgfqpoint{2.952386in}{0.739656in}}%
\pgfpathlineto{\pgfqpoint{2.951822in}{0.739656in}}%
\pgfpathlineto{\pgfqpoint{2.951258in}{0.739656in}}%
\pgfpathlineto{\pgfqpoint{2.950693in}{0.739656in}}%
\pgfpathlineto{\pgfqpoint{2.950129in}{0.739656in}}%
\pgfpathlineto{\pgfqpoint{2.949565in}{0.739656in}}%
\pgfpathlineto{\pgfqpoint{2.949000in}{0.739656in}}%
\pgfpathlineto{\pgfqpoint{2.948436in}{0.739656in}}%
\pgfpathlineto{\pgfqpoint{2.947872in}{0.739656in}}%
\pgfpathlineto{\pgfqpoint{2.947307in}{0.739656in}}%
\pgfpathlineto{\pgfqpoint{2.946743in}{0.739656in}}%
\pgfpathlineto{\pgfqpoint{2.946179in}{0.739656in}}%
\pgfpathlineto{\pgfqpoint{2.945614in}{0.739656in}}%
\pgfpathlineto{\pgfqpoint{2.945050in}{0.739656in}}%
\pgfpathlineto{\pgfqpoint{2.944486in}{0.739656in}}%
\pgfpathlineto{\pgfqpoint{2.943921in}{0.739656in}}%
\pgfpathlineto{\pgfqpoint{2.943357in}{0.739656in}}%
\pgfpathlineto{\pgfqpoint{2.942792in}{0.739656in}}%
\pgfpathlineto{\pgfqpoint{2.942228in}{0.739656in}}%
\pgfpathlineto{\pgfqpoint{2.941664in}{0.739656in}}%
\pgfpathlineto{\pgfqpoint{2.941099in}{0.739656in}}%
\pgfpathlineto{\pgfqpoint{2.940535in}{0.739656in}}%
\pgfpathlineto{\pgfqpoint{2.939971in}{0.739656in}}%
\pgfpathlineto{\pgfqpoint{2.939406in}{0.739656in}}%
\pgfpathlineto{\pgfqpoint{2.938842in}{0.739656in}}%
\pgfpathlineto{\pgfqpoint{2.938278in}{0.739656in}}%
\pgfpathlineto{\pgfqpoint{2.937713in}{0.739656in}}%
\pgfpathlineto{\pgfqpoint{2.937149in}{0.739656in}}%
\pgfpathlineto{\pgfqpoint{2.936585in}{0.739656in}}%
\pgfpathlineto{\pgfqpoint{2.936020in}{0.739656in}}%
\pgfpathlineto{\pgfqpoint{2.935456in}{0.739656in}}%
\pgfpathlineto{\pgfqpoint{2.934892in}{0.739656in}}%
\pgfpathlineto{\pgfqpoint{2.934327in}{0.739656in}}%
\pgfpathlineto{\pgfqpoint{2.933763in}{0.739656in}}%
\pgfpathlineto{\pgfqpoint{2.933199in}{0.739656in}}%
\pgfpathlineto{\pgfqpoint{2.932634in}{0.739656in}}%
\pgfpathlineto{\pgfqpoint{2.932070in}{0.739656in}}%
\pgfpathlineto{\pgfqpoint{2.931506in}{0.739656in}}%
\pgfpathlineto{\pgfqpoint{2.930941in}{0.739656in}}%
\pgfpathlineto{\pgfqpoint{2.930377in}{0.739656in}}%
\pgfpathlineto{\pgfqpoint{2.929813in}{0.739656in}}%
\pgfpathlineto{\pgfqpoint{2.929248in}{0.739656in}}%
\pgfpathlineto{\pgfqpoint{2.928684in}{0.739656in}}%
\pgfpathlineto{\pgfqpoint{2.928119in}{0.739656in}}%
\pgfpathlineto{\pgfqpoint{2.927555in}{0.739656in}}%
\pgfpathlineto{\pgfqpoint{2.926991in}{0.739656in}}%
\pgfpathlineto{\pgfqpoint{2.926426in}{0.739656in}}%
\pgfpathlineto{\pgfqpoint{2.925862in}{0.739656in}}%
\pgfpathlineto{\pgfqpoint{2.925298in}{0.739656in}}%
\pgfpathlineto{\pgfqpoint{2.924733in}{0.739656in}}%
\pgfpathlineto{\pgfqpoint{2.924169in}{0.739656in}}%
\pgfpathlineto{\pgfqpoint{2.923605in}{0.739656in}}%
\pgfpathlineto{\pgfqpoint{2.923040in}{0.739656in}}%
\pgfpathlineto{\pgfqpoint{2.922476in}{0.739656in}}%
\pgfpathlineto{\pgfqpoint{2.921912in}{0.739656in}}%
\pgfpathlineto{\pgfqpoint{2.921347in}{0.739656in}}%
\pgfpathlineto{\pgfqpoint{2.920783in}{0.739656in}}%
\pgfpathlineto{\pgfqpoint{2.920219in}{0.739656in}}%
\pgfpathlineto{\pgfqpoint{2.919654in}{0.739656in}}%
\pgfpathlineto{\pgfqpoint{2.919090in}{0.739656in}}%
\pgfpathlineto{\pgfqpoint{2.918526in}{0.739656in}}%
\pgfpathlineto{\pgfqpoint{2.917961in}{0.739656in}}%
\pgfpathlineto{\pgfqpoint{2.917397in}{0.739656in}}%
\pgfpathlineto{\pgfqpoint{2.916833in}{0.739656in}}%
\pgfpathlineto{\pgfqpoint{2.916268in}{0.739656in}}%
\pgfpathlineto{\pgfqpoint{2.915704in}{0.739656in}}%
\pgfpathlineto{\pgfqpoint{2.915140in}{0.739656in}}%
\pgfpathlineto{\pgfqpoint{2.914575in}{0.739656in}}%
\pgfpathlineto{\pgfqpoint{2.914011in}{0.739656in}}%
\pgfpathlineto{\pgfqpoint{2.913446in}{0.739656in}}%
\pgfpathlineto{\pgfqpoint{2.912882in}{0.739656in}}%
\pgfpathlineto{\pgfqpoint{2.912318in}{0.739656in}}%
\pgfpathlineto{\pgfqpoint{2.911753in}{0.739656in}}%
\pgfpathlineto{\pgfqpoint{2.911189in}{0.739656in}}%
\pgfpathlineto{\pgfqpoint{2.910625in}{0.739656in}}%
\pgfpathlineto{\pgfqpoint{2.910060in}{0.739656in}}%
\pgfpathlineto{\pgfqpoint{2.909496in}{0.739656in}}%
\pgfpathlineto{\pgfqpoint{2.908932in}{0.739656in}}%
\pgfpathlineto{\pgfqpoint{2.908367in}{0.739656in}}%
\pgfpathlineto{\pgfqpoint{2.907803in}{0.739656in}}%
\pgfpathlineto{\pgfqpoint{2.907239in}{0.739656in}}%
\pgfpathlineto{\pgfqpoint{2.906674in}{0.739656in}}%
\pgfpathlineto{\pgfqpoint{2.906110in}{0.739656in}}%
\pgfpathlineto{\pgfqpoint{2.905546in}{0.739656in}}%
\pgfpathlineto{\pgfqpoint{2.904981in}{0.739656in}}%
\pgfpathlineto{\pgfqpoint{2.904417in}{0.739656in}}%
\pgfpathlineto{\pgfqpoint{2.903853in}{0.739656in}}%
\pgfpathlineto{\pgfqpoint{2.903288in}{0.739656in}}%
\pgfpathlineto{\pgfqpoint{2.902724in}{0.739656in}}%
\pgfpathlineto{\pgfqpoint{2.902160in}{0.739656in}}%
\pgfpathlineto{\pgfqpoint{2.901595in}{0.739656in}}%
\pgfpathlineto{\pgfqpoint{2.901031in}{0.739656in}}%
\pgfpathlineto{\pgfqpoint{2.900467in}{0.739656in}}%
\pgfpathlineto{\pgfqpoint{2.899902in}{0.739656in}}%
\pgfpathlineto{\pgfqpoint{2.899338in}{0.739656in}}%
\pgfpathlineto{\pgfqpoint{2.898774in}{0.739656in}}%
\pgfpathlineto{\pgfqpoint{2.898209in}{0.739656in}}%
\pgfpathlineto{\pgfqpoint{2.897645in}{0.739656in}}%
\pgfpathlineto{\pgfqpoint{2.897080in}{0.739656in}}%
\pgfpathlineto{\pgfqpoint{2.896516in}{0.739656in}}%
\pgfpathlineto{\pgfqpoint{2.895952in}{0.739656in}}%
\pgfpathlineto{\pgfqpoint{2.895387in}{0.739656in}}%
\pgfpathlineto{\pgfqpoint{2.894823in}{0.739656in}}%
\pgfpathlineto{\pgfqpoint{2.894259in}{0.739656in}}%
\pgfpathlineto{\pgfqpoint{2.893694in}{0.739656in}}%
\pgfpathlineto{\pgfqpoint{2.893130in}{0.739656in}}%
\pgfpathlineto{\pgfqpoint{2.892566in}{0.739656in}}%
\pgfpathlineto{\pgfqpoint{2.892001in}{0.739656in}}%
\pgfpathlineto{\pgfqpoint{2.891437in}{0.739656in}}%
\pgfpathlineto{\pgfqpoint{2.890873in}{0.739656in}}%
\pgfpathlineto{\pgfqpoint{2.890308in}{0.739656in}}%
\pgfpathlineto{\pgfqpoint{2.889744in}{0.739656in}}%
\pgfpathlineto{\pgfqpoint{2.889180in}{0.739656in}}%
\pgfpathlineto{\pgfqpoint{2.888615in}{0.739656in}}%
\pgfpathlineto{\pgfqpoint{2.888051in}{0.739656in}}%
\pgfpathlineto{\pgfqpoint{2.887487in}{0.739656in}}%
\pgfpathlineto{\pgfqpoint{2.886922in}{0.739656in}}%
\pgfpathlineto{\pgfqpoint{2.886358in}{0.739656in}}%
\pgfpathlineto{\pgfqpoint{2.885794in}{0.739656in}}%
\pgfpathlineto{\pgfqpoint{2.885229in}{0.739656in}}%
\pgfpathlineto{\pgfqpoint{2.884665in}{0.739656in}}%
\pgfpathlineto{\pgfqpoint{2.884101in}{0.739656in}}%
\pgfpathlineto{\pgfqpoint{2.883536in}{0.739656in}}%
\pgfpathlineto{\pgfqpoint{2.882972in}{0.739656in}}%
\pgfpathlineto{\pgfqpoint{2.882407in}{0.739656in}}%
\pgfpathlineto{\pgfqpoint{2.881843in}{0.739656in}}%
\pgfpathlineto{\pgfqpoint{2.881279in}{0.739656in}}%
\pgfpathlineto{\pgfqpoint{2.880714in}{0.739656in}}%
\pgfpathlineto{\pgfqpoint{2.880150in}{0.739656in}}%
\pgfpathlineto{\pgfqpoint{2.879586in}{0.739656in}}%
\pgfpathlineto{\pgfqpoint{2.879021in}{0.739656in}}%
\pgfpathlineto{\pgfqpoint{2.878457in}{0.739656in}}%
\pgfpathlineto{\pgfqpoint{2.877893in}{0.739656in}}%
\pgfpathlineto{\pgfqpoint{2.877328in}{0.739656in}}%
\pgfpathlineto{\pgfqpoint{2.876764in}{0.739656in}}%
\pgfpathlineto{\pgfqpoint{2.876200in}{0.739656in}}%
\pgfpathlineto{\pgfqpoint{2.875635in}{0.739656in}}%
\pgfpathlineto{\pgfqpoint{2.875071in}{0.739656in}}%
\pgfpathlineto{\pgfqpoint{2.874507in}{0.739656in}}%
\pgfpathlineto{\pgfqpoint{2.873942in}{0.739656in}}%
\pgfpathlineto{\pgfqpoint{2.873378in}{0.739656in}}%
\pgfpathlineto{\pgfqpoint{2.872814in}{0.739656in}}%
\pgfpathlineto{\pgfqpoint{2.872249in}{0.739656in}}%
\pgfpathlineto{\pgfqpoint{2.871685in}{0.739656in}}%
\pgfpathlineto{\pgfqpoint{2.871121in}{0.739656in}}%
\pgfpathlineto{\pgfqpoint{2.870556in}{0.739656in}}%
\pgfpathlineto{\pgfqpoint{2.869992in}{0.739656in}}%
\pgfpathlineto{\pgfqpoint{2.869428in}{0.739656in}}%
\pgfpathlineto{\pgfqpoint{2.868863in}{0.739656in}}%
\pgfpathlineto{\pgfqpoint{2.868299in}{0.739656in}}%
\pgfpathlineto{\pgfqpoint{2.867734in}{0.739656in}}%
\pgfpathlineto{\pgfqpoint{2.867170in}{0.739656in}}%
\pgfpathlineto{\pgfqpoint{2.866606in}{0.739656in}}%
\pgfpathlineto{\pgfqpoint{2.866041in}{0.739656in}}%
\pgfpathlineto{\pgfqpoint{2.865477in}{0.739656in}}%
\pgfpathlineto{\pgfqpoint{2.864913in}{0.739656in}}%
\pgfpathlineto{\pgfqpoint{2.864348in}{0.739656in}}%
\pgfpathlineto{\pgfqpoint{2.863784in}{0.739656in}}%
\pgfpathlineto{\pgfqpoint{2.863220in}{0.739656in}}%
\pgfpathlineto{\pgfqpoint{2.862655in}{0.739656in}}%
\pgfpathlineto{\pgfqpoint{2.862091in}{0.739656in}}%
\pgfpathlineto{\pgfqpoint{2.861527in}{0.739656in}}%
\pgfpathlineto{\pgfqpoint{2.860962in}{0.739656in}}%
\pgfpathlineto{\pgfqpoint{2.860398in}{0.739656in}}%
\pgfpathlineto{\pgfqpoint{2.859834in}{0.739656in}}%
\pgfpathlineto{\pgfqpoint{2.859269in}{0.739656in}}%
\pgfpathlineto{\pgfqpoint{2.858705in}{0.739656in}}%
\pgfpathlineto{\pgfqpoint{2.858141in}{0.739656in}}%
\pgfpathlineto{\pgfqpoint{2.857576in}{0.739656in}}%
\pgfpathlineto{\pgfqpoint{2.857012in}{0.739656in}}%
\pgfpathlineto{\pgfqpoint{2.856448in}{0.739656in}}%
\pgfpathlineto{\pgfqpoint{2.855883in}{0.739656in}}%
\pgfpathlineto{\pgfqpoint{2.855319in}{0.739656in}}%
\pgfpathlineto{\pgfqpoint{2.854755in}{0.739656in}}%
\pgfpathlineto{\pgfqpoint{2.854190in}{0.739656in}}%
\pgfpathlineto{\pgfqpoint{2.853626in}{0.739656in}}%
\pgfpathlineto{\pgfqpoint{2.853062in}{0.739656in}}%
\pgfpathlineto{\pgfqpoint{2.852497in}{0.739656in}}%
\pgfpathlineto{\pgfqpoint{2.851933in}{0.739656in}}%
\pgfpathlineto{\pgfqpoint{2.851368in}{0.739656in}}%
\pgfpathlineto{\pgfqpoint{2.850804in}{0.739656in}}%
\pgfpathlineto{\pgfqpoint{2.850240in}{0.739656in}}%
\pgfpathlineto{\pgfqpoint{2.849675in}{0.739656in}}%
\pgfpathlineto{\pgfqpoint{2.849111in}{0.739656in}}%
\pgfpathlineto{\pgfqpoint{2.848547in}{0.739656in}}%
\pgfpathlineto{\pgfqpoint{2.847982in}{0.739656in}}%
\pgfpathlineto{\pgfqpoint{2.847418in}{0.739656in}}%
\pgfpathlineto{\pgfqpoint{2.846854in}{0.739656in}}%
\pgfpathlineto{\pgfqpoint{2.846289in}{0.739656in}}%
\pgfpathlineto{\pgfqpoint{2.845725in}{0.739656in}}%
\pgfpathlineto{\pgfqpoint{2.845161in}{0.739656in}}%
\pgfpathlineto{\pgfqpoint{2.844596in}{0.739656in}}%
\pgfpathlineto{\pgfqpoint{2.844032in}{0.739656in}}%
\pgfpathlineto{\pgfqpoint{2.843468in}{0.739656in}}%
\pgfpathlineto{\pgfqpoint{2.842903in}{0.739656in}}%
\pgfpathlineto{\pgfqpoint{2.842339in}{0.739656in}}%
\pgfpathlineto{\pgfqpoint{2.841775in}{0.739656in}}%
\pgfpathlineto{\pgfqpoint{2.841210in}{0.739656in}}%
\pgfpathlineto{\pgfqpoint{2.840646in}{0.739656in}}%
\pgfpathlineto{\pgfqpoint{2.840082in}{0.739656in}}%
\pgfpathlineto{\pgfqpoint{2.839517in}{0.739656in}}%
\pgfpathlineto{\pgfqpoint{2.838953in}{0.739656in}}%
\pgfpathlineto{\pgfqpoint{2.838389in}{0.739656in}}%
\pgfpathlineto{\pgfqpoint{2.837824in}{0.739656in}}%
\pgfpathlineto{\pgfqpoint{2.837260in}{0.739656in}}%
\pgfpathlineto{\pgfqpoint{2.836695in}{0.739656in}}%
\pgfpathlineto{\pgfqpoint{2.836131in}{0.739656in}}%
\pgfpathlineto{\pgfqpoint{2.835567in}{0.739656in}}%
\pgfpathlineto{\pgfqpoint{2.835002in}{0.739656in}}%
\pgfpathlineto{\pgfqpoint{2.834438in}{0.739656in}}%
\pgfpathlineto{\pgfqpoint{2.833874in}{0.739656in}}%
\pgfpathlineto{\pgfqpoint{2.833309in}{0.739656in}}%
\pgfpathlineto{\pgfqpoint{2.832745in}{0.739656in}}%
\pgfpathlineto{\pgfqpoint{2.832181in}{0.739656in}}%
\pgfpathlineto{\pgfqpoint{2.831616in}{0.739656in}}%
\pgfpathlineto{\pgfqpoint{2.831052in}{0.739656in}}%
\pgfpathlineto{\pgfqpoint{2.830488in}{0.739656in}}%
\pgfpathlineto{\pgfqpoint{2.829923in}{0.739656in}}%
\pgfpathlineto{\pgfqpoint{2.829359in}{0.739656in}}%
\pgfpathlineto{\pgfqpoint{2.828795in}{0.739656in}}%
\pgfpathlineto{\pgfqpoint{2.828230in}{0.739656in}}%
\pgfpathlineto{\pgfqpoint{2.827666in}{0.739656in}}%
\pgfpathlineto{\pgfqpoint{2.827102in}{0.739656in}}%
\pgfpathlineto{\pgfqpoint{2.826537in}{0.739656in}}%
\pgfpathlineto{\pgfqpoint{2.825973in}{0.739656in}}%
\pgfpathlineto{\pgfqpoint{2.825409in}{0.739656in}}%
\pgfpathlineto{\pgfqpoint{2.824844in}{0.739656in}}%
\pgfpathlineto{\pgfqpoint{2.824280in}{0.739656in}}%
\pgfpathlineto{\pgfqpoint{2.823716in}{0.739656in}}%
\pgfpathlineto{\pgfqpoint{2.823151in}{0.739656in}}%
\pgfpathlineto{\pgfqpoint{2.822587in}{0.739656in}}%
\pgfpathlineto{\pgfqpoint{2.822022in}{0.739656in}}%
\pgfpathlineto{\pgfqpoint{2.821458in}{0.739656in}}%
\pgfpathlineto{\pgfqpoint{2.820894in}{0.739656in}}%
\pgfpathlineto{\pgfqpoint{2.820329in}{0.739656in}}%
\pgfpathlineto{\pgfqpoint{2.819765in}{0.739656in}}%
\pgfpathlineto{\pgfqpoint{2.819201in}{0.739656in}}%
\pgfpathlineto{\pgfqpoint{2.818636in}{0.739656in}}%
\pgfpathlineto{\pgfqpoint{2.818072in}{0.739656in}}%
\pgfpathlineto{\pgfqpoint{2.817508in}{0.739656in}}%
\pgfpathlineto{\pgfqpoint{2.816943in}{0.739656in}}%
\pgfpathlineto{\pgfqpoint{2.816379in}{0.739656in}}%
\pgfpathlineto{\pgfqpoint{2.815815in}{0.739656in}}%
\pgfpathlineto{\pgfqpoint{2.815250in}{0.739656in}}%
\pgfpathlineto{\pgfqpoint{2.814686in}{0.739656in}}%
\pgfpathlineto{\pgfqpoint{2.814122in}{0.739656in}}%
\pgfpathlineto{\pgfqpoint{2.813557in}{0.739656in}}%
\pgfpathlineto{\pgfqpoint{2.812993in}{0.739656in}}%
\pgfpathlineto{\pgfqpoint{2.812429in}{0.739656in}}%
\pgfpathlineto{\pgfqpoint{2.811864in}{0.739656in}}%
\pgfpathlineto{\pgfqpoint{2.811300in}{0.739656in}}%
\pgfpathlineto{\pgfqpoint{2.810736in}{0.739656in}}%
\pgfpathlineto{\pgfqpoint{2.810171in}{0.739656in}}%
\pgfpathlineto{\pgfqpoint{2.809607in}{0.739656in}}%
\pgfpathlineto{\pgfqpoint{2.809043in}{0.739656in}}%
\pgfpathlineto{\pgfqpoint{2.808478in}{0.739656in}}%
\pgfpathlineto{\pgfqpoint{2.807914in}{0.739656in}}%
\pgfpathlineto{\pgfqpoint{2.807350in}{0.739656in}}%
\pgfpathlineto{\pgfqpoint{2.806785in}{0.739656in}}%
\pgfpathlineto{\pgfqpoint{2.806221in}{0.739656in}}%
\pgfpathlineto{\pgfqpoint{2.805656in}{0.739656in}}%
\pgfpathlineto{\pgfqpoint{2.805092in}{0.739656in}}%
\pgfpathlineto{\pgfqpoint{2.804528in}{0.739656in}}%
\pgfpathlineto{\pgfqpoint{2.803963in}{0.739656in}}%
\pgfpathlineto{\pgfqpoint{2.803399in}{0.739656in}}%
\pgfpathlineto{\pgfqpoint{2.802835in}{0.739656in}}%
\pgfpathlineto{\pgfqpoint{2.802270in}{0.739656in}}%
\pgfpathlineto{\pgfqpoint{2.801706in}{0.739656in}}%
\pgfpathlineto{\pgfqpoint{2.801142in}{0.739656in}}%
\pgfpathlineto{\pgfqpoint{2.800577in}{0.739656in}}%
\pgfpathlineto{\pgfqpoint{2.800013in}{0.739656in}}%
\pgfpathlineto{\pgfqpoint{2.799449in}{0.739656in}}%
\pgfpathlineto{\pgfqpoint{2.798884in}{0.739656in}}%
\pgfpathlineto{\pgfqpoint{2.798320in}{0.739656in}}%
\pgfpathlineto{\pgfqpoint{2.797756in}{0.739656in}}%
\pgfpathlineto{\pgfqpoint{2.797191in}{0.739656in}}%
\pgfpathlineto{\pgfqpoint{2.796627in}{0.739656in}}%
\pgfpathlineto{\pgfqpoint{2.796063in}{0.739656in}}%
\pgfpathlineto{\pgfqpoint{2.795498in}{0.739656in}}%
\pgfpathlineto{\pgfqpoint{2.794934in}{0.739656in}}%
\pgfpathlineto{\pgfqpoint{2.794370in}{0.739656in}}%
\pgfpathlineto{\pgfqpoint{2.793805in}{0.739656in}}%
\pgfpathlineto{\pgfqpoint{2.793241in}{0.739656in}}%
\pgfpathlineto{\pgfqpoint{2.792677in}{0.739656in}}%
\pgfpathlineto{\pgfqpoint{2.792112in}{0.739656in}}%
\pgfpathlineto{\pgfqpoint{2.791548in}{0.739656in}}%
\pgfpathlineto{\pgfqpoint{2.790983in}{0.739656in}}%
\pgfpathlineto{\pgfqpoint{2.790419in}{0.739656in}}%
\pgfpathlineto{\pgfqpoint{2.789855in}{0.739656in}}%
\pgfpathlineto{\pgfqpoint{2.789290in}{0.739656in}}%
\pgfpathlineto{\pgfqpoint{2.788726in}{0.739656in}}%
\pgfpathlineto{\pgfqpoint{2.788162in}{0.739656in}}%
\pgfpathlineto{\pgfqpoint{2.787597in}{0.739656in}}%
\pgfpathlineto{\pgfqpoint{2.787033in}{0.739656in}}%
\pgfpathlineto{\pgfqpoint{2.786469in}{0.739656in}}%
\pgfpathlineto{\pgfqpoint{2.785904in}{0.739656in}}%
\pgfpathlineto{\pgfqpoint{2.785340in}{0.739656in}}%
\pgfpathlineto{\pgfqpoint{2.784776in}{0.739656in}}%
\pgfpathlineto{\pgfqpoint{2.784211in}{0.739656in}}%
\pgfpathlineto{\pgfqpoint{2.783647in}{0.739656in}}%
\pgfpathlineto{\pgfqpoint{2.783083in}{0.739656in}}%
\pgfpathlineto{\pgfqpoint{2.782518in}{0.739656in}}%
\pgfpathlineto{\pgfqpoint{2.781954in}{0.739656in}}%
\pgfpathlineto{\pgfqpoint{2.781390in}{0.739656in}}%
\pgfpathlineto{\pgfqpoint{2.780825in}{0.739656in}}%
\pgfpathlineto{\pgfqpoint{2.780261in}{0.739656in}}%
\pgfpathlineto{\pgfqpoint{2.779697in}{0.739656in}}%
\pgfpathlineto{\pgfqpoint{2.779132in}{0.739656in}}%
\pgfpathlineto{\pgfqpoint{2.778568in}{0.739656in}}%
\pgfpathlineto{\pgfqpoint{2.778004in}{0.739656in}}%
\pgfpathlineto{\pgfqpoint{2.777439in}{0.739656in}}%
\pgfpathlineto{\pgfqpoint{2.776875in}{0.739656in}}%
\pgfpathlineto{\pgfqpoint{2.776310in}{0.739656in}}%
\pgfpathlineto{\pgfqpoint{2.775746in}{0.739656in}}%
\pgfpathlineto{\pgfqpoint{2.775182in}{0.739656in}}%
\pgfpathlineto{\pgfqpoint{2.774617in}{0.739656in}}%
\pgfpathlineto{\pgfqpoint{2.774053in}{0.739656in}}%
\pgfpathlineto{\pgfqpoint{2.773489in}{0.739656in}}%
\pgfpathlineto{\pgfqpoint{2.772924in}{0.739656in}}%
\pgfpathlineto{\pgfqpoint{2.772360in}{0.739656in}}%
\pgfpathlineto{\pgfqpoint{2.771796in}{0.739656in}}%
\pgfpathlineto{\pgfqpoint{2.771231in}{0.739656in}}%
\pgfpathlineto{\pgfqpoint{2.770667in}{0.739656in}}%
\pgfpathlineto{\pgfqpoint{2.770103in}{0.739656in}}%
\pgfpathlineto{\pgfqpoint{2.769538in}{0.739656in}}%
\pgfpathlineto{\pgfqpoint{2.768974in}{0.739656in}}%
\pgfpathlineto{\pgfqpoint{2.768410in}{0.739656in}}%
\pgfpathlineto{\pgfqpoint{2.767845in}{0.739656in}}%
\pgfpathlineto{\pgfqpoint{2.767281in}{0.739656in}}%
\pgfpathlineto{\pgfqpoint{2.766717in}{0.739656in}}%
\pgfpathlineto{\pgfqpoint{2.766152in}{0.739656in}}%
\pgfpathlineto{\pgfqpoint{2.765588in}{0.739656in}}%
\pgfpathlineto{\pgfqpoint{2.765024in}{0.739656in}}%
\pgfpathlineto{\pgfqpoint{2.764459in}{0.739656in}}%
\pgfpathlineto{\pgfqpoint{2.763895in}{0.739656in}}%
\pgfpathlineto{\pgfqpoint{2.763331in}{0.739656in}}%
\pgfpathlineto{\pgfqpoint{2.762766in}{0.739656in}}%
\pgfpathlineto{\pgfqpoint{2.762202in}{0.739656in}}%
\pgfpathlineto{\pgfqpoint{2.761638in}{0.739656in}}%
\pgfpathlineto{\pgfqpoint{2.761073in}{0.739656in}}%
\pgfpathlineto{\pgfqpoint{2.760509in}{0.739656in}}%
\pgfpathlineto{\pgfqpoint{2.759944in}{0.739656in}}%
\pgfpathlineto{\pgfqpoint{2.759380in}{0.739656in}}%
\pgfpathlineto{\pgfqpoint{2.758816in}{0.739656in}}%
\pgfpathlineto{\pgfqpoint{2.758251in}{0.739656in}}%
\pgfpathlineto{\pgfqpoint{2.757687in}{0.739656in}}%
\pgfpathlineto{\pgfqpoint{2.757123in}{0.739656in}}%
\pgfpathlineto{\pgfqpoint{2.756558in}{0.739656in}}%
\pgfpathlineto{\pgfqpoint{2.755994in}{0.739656in}}%
\pgfpathlineto{\pgfqpoint{2.755430in}{0.739656in}}%
\pgfpathlineto{\pgfqpoint{2.754865in}{0.739656in}}%
\pgfpathlineto{\pgfqpoint{2.754301in}{0.739656in}}%
\pgfpathlineto{\pgfqpoint{2.753737in}{0.739656in}}%
\pgfpathlineto{\pgfqpoint{2.753172in}{0.739656in}}%
\pgfpathlineto{\pgfqpoint{2.752608in}{0.739656in}}%
\pgfpathlineto{\pgfqpoint{2.752044in}{0.739656in}}%
\pgfpathlineto{\pgfqpoint{2.751479in}{0.739656in}}%
\pgfpathlineto{\pgfqpoint{2.750915in}{0.739656in}}%
\pgfpathlineto{\pgfqpoint{2.750351in}{0.739656in}}%
\pgfpathlineto{\pgfqpoint{2.749786in}{0.739656in}}%
\pgfpathlineto{\pgfqpoint{2.749222in}{0.739656in}}%
\pgfpathlineto{\pgfqpoint{2.748658in}{0.739656in}}%
\pgfpathlineto{\pgfqpoint{2.748093in}{0.739656in}}%
\pgfpathlineto{\pgfqpoint{2.747529in}{0.739656in}}%
\pgfpathlineto{\pgfqpoint{2.746965in}{0.739656in}}%
\pgfpathlineto{\pgfqpoint{2.746400in}{0.739656in}}%
\pgfpathlineto{\pgfqpoint{2.745836in}{0.739656in}}%
\pgfpathlineto{\pgfqpoint{2.745271in}{0.739656in}}%
\pgfpathlineto{\pgfqpoint{2.744707in}{0.739656in}}%
\pgfpathlineto{\pgfqpoint{2.744143in}{0.739656in}}%
\pgfpathlineto{\pgfqpoint{2.743578in}{0.739656in}}%
\pgfpathlineto{\pgfqpoint{2.743014in}{0.739656in}}%
\pgfpathlineto{\pgfqpoint{2.742450in}{0.739656in}}%
\pgfpathlineto{\pgfqpoint{2.741885in}{0.739656in}}%
\pgfpathlineto{\pgfqpoint{2.741321in}{0.739656in}}%
\pgfpathlineto{\pgfqpoint{2.740757in}{0.739656in}}%
\pgfpathlineto{\pgfqpoint{2.740192in}{0.739656in}}%
\pgfpathlineto{\pgfqpoint{2.739628in}{0.739656in}}%
\pgfpathlineto{\pgfqpoint{2.739064in}{0.739656in}}%
\pgfpathlineto{\pgfqpoint{2.738499in}{0.739656in}}%
\pgfpathlineto{\pgfqpoint{2.737935in}{0.739656in}}%
\pgfpathlineto{\pgfqpoint{2.737371in}{0.739656in}}%
\pgfpathlineto{\pgfqpoint{2.736806in}{0.739656in}}%
\pgfpathlineto{\pgfqpoint{2.736242in}{0.739656in}}%
\pgfpathlineto{\pgfqpoint{2.735678in}{0.739656in}}%
\pgfpathlineto{\pgfqpoint{2.735113in}{0.739656in}}%
\pgfpathlineto{\pgfqpoint{2.734549in}{0.739656in}}%
\pgfpathlineto{\pgfqpoint{2.733985in}{0.739656in}}%
\pgfpathlineto{\pgfqpoint{2.733420in}{0.739656in}}%
\pgfpathlineto{\pgfqpoint{2.732856in}{0.739656in}}%
\pgfpathlineto{\pgfqpoint{2.732292in}{0.739656in}}%
\pgfpathlineto{\pgfqpoint{2.731727in}{0.739656in}}%
\pgfpathlineto{\pgfqpoint{2.731163in}{0.739656in}}%
\pgfpathlineto{\pgfqpoint{2.730598in}{0.739656in}}%
\pgfpathlineto{\pgfqpoint{2.730034in}{0.739656in}}%
\pgfpathlineto{\pgfqpoint{2.729470in}{0.739656in}}%
\pgfpathlineto{\pgfqpoint{2.728905in}{0.739656in}}%
\pgfpathlineto{\pgfqpoint{2.728341in}{0.739656in}}%
\pgfpathlineto{\pgfqpoint{2.727777in}{0.739656in}}%
\pgfpathlineto{\pgfqpoint{2.727212in}{0.739656in}}%
\pgfpathlineto{\pgfqpoint{2.726648in}{0.739656in}}%
\pgfpathlineto{\pgfqpoint{2.726084in}{0.739656in}}%
\pgfpathlineto{\pgfqpoint{2.725519in}{0.739656in}}%
\pgfpathlineto{\pgfqpoint{2.724955in}{0.739656in}}%
\pgfpathlineto{\pgfqpoint{2.724391in}{0.739656in}}%
\pgfpathlineto{\pgfqpoint{2.723826in}{0.739656in}}%
\pgfpathlineto{\pgfqpoint{2.723262in}{0.739656in}}%
\pgfpathlineto{\pgfqpoint{2.722698in}{0.739656in}}%
\pgfpathlineto{\pgfqpoint{2.722133in}{0.739656in}}%
\pgfpathlineto{\pgfqpoint{2.721569in}{0.739656in}}%
\pgfpathlineto{\pgfqpoint{2.721005in}{0.739656in}}%
\pgfpathlineto{\pgfqpoint{2.720440in}{0.739656in}}%
\pgfpathlineto{\pgfqpoint{2.719876in}{0.739656in}}%
\pgfpathlineto{\pgfqpoint{2.719312in}{0.739656in}}%
\pgfpathlineto{\pgfqpoint{2.718747in}{0.739656in}}%
\pgfpathlineto{\pgfqpoint{2.718183in}{0.739656in}}%
\pgfpathlineto{\pgfqpoint{2.717619in}{0.739656in}}%
\pgfpathlineto{\pgfqpoint{2.717054in}{0.739656in}}%
\pgfpathlineto{\pgfqpoint{2.716490in}{0.739656in}}%
\pgfpathlineto{\pgfqpoint{2.715926in}{0.739656in}}%
\pgfpathlineto{\pgfqpoint{2.715361in}{0.739656in}}%
\pgfpathlineto{\pgfqpoint{2.714797in}{0.739656in}}%
\pgfpathlineto{\pgfqpoint{2.714232in}{0.739656in}}%
\pgfpathlineto{\pgfqpoint{2.713668in}{0.739656in}}%
\pgfpathlineto{\pgfqpoint{2.713104in}{0.739656in}}%
\pgfpathlineto{\pgfqpoint{2.712539in}{0.739656in}}%
\pgfpathlineto{\pgfqpoint{2.711975in}{0.739656in}}%
\pgfpathlineto{\pgfqpoint{2.711411in}{0.739656in}}%
\pgfpathlineto{\pgfqpoint{2.710846in}{0.739656in}}%
\pgfpathlineto{\pgfqpoint{2.710282in}{0.739656in}}%
\pgfpathlineto{\pgfqpoint{2.709718in}{0.739656in}}%
\pgfpathlineto{\pgfqpoint{2.709153in}{0.739656in}}%
\pgfpathlineto{\pgfqpoint{2.708589in}{0.739656in}}%
\pgfpathlineto{\pgfqpoint{2.708025in}{0.739656in}}%
\pgfpathlineto{\pgfqpoint{2.707460in}{0.739656in}}%
\pgfpathlineto{\pgfqpoint{2.706896in}{0.739656in}}%
\pgfpathlineto{\pgfqpoint{2.706332in}{0.739656in}}%
\pgfpathlineto{\pgfqpoint{2.705767in}{0.739656in}}%
\pgfpathlineto{\pgfqpoint{2.705203in}{0.739656in}}%
\pgfpathlineto{\pgfqpoint{2.704639in}{0.739656in}}%
\pgfpathlineto{\pgfqpoint{2.704074in}{0.739656in}}%
\pgfpathlineto{\pgfqpoint{2.703510in}{0.739656in}}%
\pgfpathlineto{\pgfqpoint{2.702946in}{0.739656in}}%
\pgfpathlineto{\pgfqpoint{2.702381in}{0.739656in}}%
\pgfpathlineto{\pgfqpoint{2.701817in}{0.739656in}}%
\pgfpathlineto{\pgfqpoint{2.701253in}{0.739656in}}%
\pgfpathlineto{\pgfqpoint{2.700688in}{0.739656in}}%
\pgfpathlineto{\pgfqpoint{2.700124in}{0.739656in}}%
\pgfpathlineto{\pgfqpoint{2.699559in}{0.739656in}}%
\pgfpathlineto{\pgfqpoint{2.698995in}{0.739656in}}%
\pgfpathlineto{\pgfqpoint{2.698431in}{0.739656in}}%
\pgfpathlineto{\pgfqpoint{2.697866in}{0.739656in}}%
\pgfpathlineto{\pgfqpoint{2.697302in}{0.739656in}}%
\pgfpathlineto{\pgfqpoint{2.696738in}{0.739656in}}%
\pgfpathlineto{\pgfqpoint{2.696173in}{0.739656in}}%
\pgfpathlineto{\pgfqpoint{2.695609in}{0.739656in}}%
\pgfpathlineto{\pgfqpoint{2.695045in}{0.739656in}}%
\pgfpathlineto{\pgfqpoint{2.694480in}{0.739656in}}%
\pgfpathlineto{\pgfqpoint{2.693916in}{0.739656in}}%
\pgfpathlineto{\pgfqpoint{2.693352in}{0.739656in}}%
\pgfpathlineto{\pgfqpoint{2.692787in}{0.739656in}}%
\pgfpathlineto{\pgfqpoint{2.692223in}{0.739656in}}%
\pgfpathlineto{\pgfqpoint{2.691659in}{0.739656in}}%
\pgfpathlineto{\pgfqpoint{2.691094in}{0.739656in}}%
\pgfpathlineto{\pgfqpoint{2.690530in}{0.739656in}}%
\pgfpathlineto{\pgfqpoint{2.689966in}{0.739656in}}%
\pgfpathlineto{\pgfqpoint{2.689401in}{0.739656in}}%
\pgfpathlineto{\pgfqpoint{2.688837in}{0.739656in}}%
\pgfpathlineto{\pgfqpoint{2.688273in}{0.739656in}}%
\pgfpathlineto{\pgfqpoint{2.687708in}{0.739656in}}%
\pgfpathlineto{\pgfqpoint{2.687144in}{0.739656in}}%
\pgfpathlineto{\pgfqpoint{2.686580in}{0.739656in}}%
\pgfpathlineto{\pgfqpoint{2.686015in}{0.739656in}}%
\pgfpathlineto{\pgfqpoint{2.685451in}{0.739656in}}%
\pgfpathlineto{\pgfqpoint{2.684886in}{0.739656in}}%
\pgfpathlineto{\pgfqpoint{2.684322in}{0.739656in}}%
\pgfpathlineto{\pgfqpoint{2.683758in}{0.739656in}}%
\pgfpathlineto{\pgfqpoint{2.683193in}{0.739656in}}%
\pgfpathlineto{\pgfqpoint{2.682629in}{0.739656in}}%
\pgfpathlineto{\pgfqpoint{2.682065in}{0.739656in}}%
\pgfpathlineto{\pgfqpoint{2.681500in}{0.739656in}}%
\pgfpathlineto{\pgfqpoint{2.680936in}{0.739656in}}%
\pgfpathlineto{\pgfqpoint{2.680372in}{0.739656in}}%
\pgfpathlineto{\pgfqpoint{2.679807in}{0.739656in}}%
\pgfpathlineto{\pgfqpoint{2.679243in}{0.739656in}}%
\pgfpathlineto{\pgfqpoint{2.678679in}{0.739656in}}%
\pgfpathlineto{\pgfqpoint{2.678114in}{0.739656in}}%
\pgfpathlineto{\pgfqpoint{2.677550in}{0.739656in}}%
\pgfpathlineto{\pgfqpoint{2.676986in}{0.739656in}}%
\pgfpathlineto{\pgfqpoint{2.676421in}{0.739656in}}%
\pgfpathlineto{\pgfqpoint{2.675857in}{0.739656in}}%
\pgfpathlineto{\pgfqpoint{2.675293in}{0.739656in}}%
\pgfpathlineto{\pgfqpoint{2.674728in}{0.739656in}}%
\pgfpathlineto{\pgfqpoint{2.674164in}{0.739656in}}%
\pgfpathlineto{\pgfqpoint{2.673600in}{0.739656in}}%
\pgfpathlineto{\pgfqpoint{2.673035in}{0.739656in}}%
\pgfpathlineto{\pgfqpoint{2.672471in}{0.739656in}}%
\pgfpathlineto{\pgfqpoint{2.671907in}{0.739656in}}%
\pgfpathlineto{\pgfqpoint{2.671342in}{0.739656in}}%
\pgfpathlineto{\pgfqpoint{2.670778in}{0.739656in}}%
\pgfpathlineto{\pgfqpoint{2.670213in}{0.739656in}}%
\pgfpathlineto{\pgfqpoint{2.669649in}{0.739656in}}%
\pgfpathlineto{\pgfqpoint{2.669085in}{0.739656in}}%
\pgfpathlineto{\pgfqpoint{2.668520in}{0.739656in}}%
\pgfpathlineto{\pgfqpoint{2.667956in}{0.739656in}}%
\pgfpathlineto{\pgfqpoint{2.667392in}{0.739656in}}%
\pgfpathlineto{\pgfqpoint{2.666827in}{0.739656in}}%
\pgfpathlineto{\pgfqpoint{2.666263in}{0.739656in}}%
\pgfpathlineto{\pgfqpoint{2.665699in}{0.739656in}}%
\pgfpathlineto{\pgfqpoint{2.665134in}{0.739656in}}%
\pgfpathlineto{\pgfqpoint{2.664570in}{0.739656in}}%
\pgfpathlineto{\pgfqpoint{2.664006in}{0.739656in}}%
\pgfpathlineto{\pgfqpoint{2.663441in}{0.739656in}}%
\pgfpathlineto{\pgfqpoint{2.662877in}{0.739656in}}%
\pgfpathlineto{\pgfqpoint{2.662313in}{0.739656in}}%
\pgfpathlineto{\pgfqpoint{2.661748in}{0.739656in}}%
\pgfpathlineto{\pgfqpoint{2.661184in}{0.739656in}}%
\pgfpathlineto{\pgfqpoint{2.660620in}{0.739656in}}%
\pgfpathlineto{\pgfqpoint{2.660055in}{0.739656in}}%
\pgfpathlineto{\pgfqpoint{2.659491in}{0.739656in}}%
\pgfpathlineto{\pgfqpoint{2.658927in}{0.739656in}}%
\pgfpathlineto{\pgfqpoint{2.658362in}{0.739656in}}%
\pgfpathlineto{\pgfqpoint{2.657798in}{0.739656in}}%
\pgfpathlineto{\pgfqpoint{2.657234in}{0.739656in}}%
\pgfpathlineto{\pgfqpoint{2.656669in}{0.739656in}}%
\pgfpathlineto{\pgfqpoint{2.656105in}{0.739656in}}%
\pgfpathlineto{\pgfqpoint{2.655541in}{0.739656in}}%
\pgfpathlineto{\pgfqpoint{2.654976in}{0.739656in}}%
\pgfpathlineto{\pgfqpoint{2.654412in}{0.739656in}}%
\pgfpathlineto{\pgfqpoint{2.653847in}{0.739656in}}%
\pgfpathlineto{\pgfqpoint{2.653283in}{0.739656in}}%
\pgfpathlineto{\pgfqpoint{2.652719in}{0.739656in}}%
\pgfpathlineto{\pgfqpoint{2.652154in}{0.739656in}}%
\pgfpathlineto{\pgfqpoint{2.651590in}{0.739656in}}%
\pgfpathlineto{\pgfqpoint{2.651026in}{0.739656in}}%
\pgfpathlineto{\pgfqpoint{2.650461in}{0.739656in}}%
\pgfpathlineto{\pgfqpoint{2.649897in}{0.739656in}}%
\pgfpathlineto{\pgfqpoint{2.649333in}{0.739656in}}%
\pgfpathlineto{\pgfqpoint{2.648768in}{0.739656in}}%
\pgfpathlineto{\pgfqpoint{2.648204in}{0.739656in}}%
\pgfpathlineto{\pgfqpoint{2.647640in}{0.739656in}}%
\pgfpathlineto{\pgfqpoint{2.647075in}{0.739656in}}%
\pgfpathlineto{\pgfqpoint{2.646511in}{0.739656in}}%
\pgfpathlineto{\pgfqpoint{2.645947in}{0.739656in}}%
\pgfpathlineto{\pgfqpoint{2.645382in}{0.739656in}}%
\pgfpathlineto{\pgfqpoint{2.644818in}{0.739656in}}%
\pgfpathlineto{\pgfqpoint{2.644254in}{0.739656in}}%
\pgfpathlineto{\pgfqpoint{2.643689in}{0.739656in}}%
\pgfpathlineto{\pgfqpoint{2.643125in}{0.739656in}}%
\pgfpathlineto{\pgfqpoint{2.642561in}{0.739656in}}%
\pgfpathlineto{\pgfqpoint{2.641996in}{0.739656in}}%
\pgfpathlineto{\pgfqpoint{2.641432in}{0.739656in}}%
\pgfpathlineto{\pgfqpoint{2.640868in}{0.739656in}}%
\pgfpathlineto{\pgfqpoint{2.640303in}{0.739656in}}%
\pgfpathlineto{\pgfqpoint{2.639739in}{0.739656in}}%
\pgfpathlineto{\pgfqpoint{2.639174in}{0.739656in}}%
\pgfpathlineto{\pgfqpoint{2.638610in}{0.739656in}}%
\pgfpathlineto{\pgfqpoint{2.638046in}{0.739656in}}%
\pgfpathlineto{\pgfqpoint{2.637481in}{0.739656in}}%
\pgfpathlineto{\pgfqpoint{2.636917in}{0.739656in}}%
\pgfpathlineto{\pgfqpoint{2.636353in}{0.739656in}}%
\pgfpathlineto{\pgfqpoint{2.635788in}{0.739656in}}%
\pgfpathlineto{\pgfqpoint{2.635224in}{0.739656in}}%
\pgfpathlineto{\pgfqpoint{2.634660in}{0.739656in}}%
\pgfpathlineto{\pgfqpoint{2.634095in}{0.739656in}}%
\pgfpathlineto{\pgfqpoint{2.633531in}{0.739656in}}%
\pgfpathlineto{\pgfqpoint{2.632967in}{0.739656in}}%
\pgfpathlineto{\pgfqpoint{2.632402in}{0.739656in}}%
\pgfpathlineto{\pgfqpoint{2.631838in}{0.739656in}}%
\pgfpathlineto{\pgfqpoint{2.631274in}{0.739656in}}%
\pgfpathlineto{\pgfqpoint{2.630709in}{0.739656in}}%
\pgfpathlineto{\pgfqpoint{2.630145in}{0.739656in}}%
\pgfpathlineto{\pgfqpoint{2.629581in}{0.739656in}}%
\pgfpathlineto{\pgfqpoint{2.629016in}{0.739656in}}%
\pgfpathlineto{\pgfqpoint{2.628452in}{0.739656in}}%
\pgfpathlineto{\pgfqpoint{2.627888in}{0.739656in}}%
\pgfpathlineto{\pgfqpoint{2.627323in}{0.739656in}}%
\pgfpathlineto{\pgfqpoint{2.626759in}{0.739656in}}%
\pgfpathlineto{\pgfqpoint{2.626195in}{0.739656in}}%
\pgfpathlineto{\pgfqpoint{2.625630in}{0.739656in}}%
\pgfpathlineto{\pgfqpoint{2.625066in}{0.739656in}}%
\pgfpathlineto{\pgfqpoint{2.624501in}{0.739656in}}%
\pgfpathlineto{\pgfqpoint{2.623937in}{0.739656in}}%
\pgfpathlineto{\pgfqpoint{2.623373in}{0.739656in}}%
\pgfpathlineto{\pgfqpoint{2.622808in}{0.739656in}}%
\pgfpathlineto{\pgfqpoint{2.622244in}{0.739656in}}%
\pgfpathlineto{\pgfqpoint{2.621680in}{0.739656in}}%
\pgfpathlineto{\pgfqpoint{2.621115in}{0.739656in}}%
\pgfpathlineto{\pgfqpoint{2.620551in}{0.739656in}}%
\pgfpathlineto{\pgfqpoint{2.619987in}{0.739656in}}%
\pgfpathlineto{\pgfqpoint{2.619422in}{0.739656in}}%
\pgfpathlineto{\pgfqpoint{2.618858in}{0.739656in}}%
\pgfpathlineto{\pgfqpoint{2.618294in}{0.739656in}}%
\pgfpathlineto{\pgfqpoint{2.617729in}{0.739656in}}%
\pgfpathlineto{\pgfqpoint{2.617165in}{0.739656in}}%
\pgfpathlineto{\pgfqpoint{2.616601in}{0.739656in}}%
\pgfpathlineto{\pgfqpoint{2.616036in}{0.739656in}}%
\pgfpathlineto{\pgfqpoint{2.615472in}{0.739656in}}%
\pgfpathlineto{\pgfqpoint{2.614908in}{0.739656in}}%
\pgfpathlineto{\pgfqpoint{2.614343in}{0.739656in}}%
\pgfpathlineto{\pgfqpoint{2.613779in}{0.739656in}}%
\pgfpathlineto{\pgfqpoint{2.613215in}{0.739656in}}%
\pgfpathlineto{\pgfqpoint{2.612650in}{0.739656in}}%
\pgfpathlineto{\pgfqpoint{2.612086in}{0.739656in}}%
\pgfpathlineto{\pgfqpoint{2.611522in}{0.739656in}}%
\pgfpathlineto{\pgfqpoint{2.610957in}{0.739656in}}%
\pgfpathlineto{\pgfqpoint{2.610393in}{0.739656in}}%
\pgfpathlineto{\pgfqpoint{2.609829in}{0.739656in}}%
\pgfpathlineto{\pgfqpoint{2.609264in}{0.739656in}}%
\pgfpathlineto{\pgfqpoint{2.608700in}{0.739656in}}%
\pgfpathlineto{\pgfqpoint{2.608135in}{0.739656in}}%
\pgfpathlineto{\pgfqpoint{2.607571in}{0.739656in}}%
\pgfpathlineto{\pgfqpoint{2.607007in}{0.739656in}}%
\pgfpathlineto{\pgfqpoint{2.606442in}{0.739656in}}%
\pgfpathlineto{\pgfqpoint{2.605878in}{0.739656in}}%
\pgfpathlineto{\pgfqpoint{2.605314in}{0.739656in}}%
\pgfpathlineto{\pgfqpoint{2.604749in}{0.739656in}}%
\pgfpathlineto{\pgfqpoint{2.604185in}{0.739656in}}%
\pgfpathlineto{\pgfqpoint{2.603621in}{0.739656in}}%
\pgfpathlineto{\pgfqpoint{2.603056in}{0.739656in}}%
\pgfpathlineto{\pgfqpoint{2.602492in}{0.739656in}}%
\pgfpathlineto{\pgfqpoint{2.601928in}{0.739656in}}%
\pgfpathlineto{\pgfqpoint{2.601363in}{0.739656in}}%
\pgfpathlineto{\pgfqpoint{2.600799in}{0.739656in}}%
\pgfpathlineto{\pgfqpoint{2.600235in}{0.739656in}}%
\pgfpathlineto{\pgfqpoint{2.599670in}{0.739656in}}%
\pgfpathlineto{\pgfqpoint{2.599106in}{0.739656in}}%
\pgfpathlineto{\pgfqpoint{2.598542in}{0.739656in}}%
\pgfpathlineto{\pgfqpoint{2.597977in}{0.739656in}}%
\pgfpathlineto{\pgfqpoint{2.597413in}{0.739656in}}%
\pgfpathlineto{\pgfqpoint{2.596849in}{0.739656in}}%
\pgfpathlineto{\pgfqpoint{2.596284in}{0.739656in}}%
\pgfpathlineto{\pgfqpoint{2.595720in}{0.739656in}}%
\pgfpathlineto{\pgfqpoint{2.595156in}{0.739656in}}%
\pgfpathlineto{\pgfqpoint{2.594591in}{0.739656in}}%
\pgfpathlineto{\pgfqpoint{2.594027in}{0.739656in}}%
\pgfpathlineto{\pgfqpoint{2.593462in}{0.739656in}}%
\pgfpathlineto{\pgfqpoint{2.592898in}{0.739656in}}%
\pgfpathlineto{\pgfqpoint{2.592334in}{0.739656in}}%
\pgfpathlineto{\pgfqpoint{2.591769in}{0.739656in}}%
\pgfpathlineto{\pgfqpoint{2.591205in}{0.739656in}}%
\pgfpathlineto{\pgfqpoint{2.590641in}{0.739656in}}%
\pgfpathlineto{\pgfqpoint{2.590076in}{0.739656in}}%
\pgfpathlineto{\pgfqpoint{2.589512in}{0.739656in}}%
\pgfpathlineto{\pgfqpoint{2.588948in}{0.739656in}}%
\pgfpathlineto{\pgfqpoint{2.588383in}{0.739656in}}%
\pgfpathlineto{\pgfqpoint{2.587819in}{0.739656in}}%
\pgfpathlineto{\pgfqpoint{2.587255in}{0.739656in}}%
\pgfpathlineto{\pgfqpoint{2.586690in}{0.739656in}}%
\pgfpathlineto{\pgfqpoint{2.586126in}{0.739656in}}%
\pgfpathlineto{\pgfqpoint{2.585562in}{0.739656in}}%
\pgfpathlineto{\pgfqpoint{2.584997in}{0.739656in}}%
\pgfpathlineto{\pgfqpoint{2.584433in}{0.739656in}}%
\pgfpathlineto{\pgfqpoint{2.583869in}{0.739656in}}%
\pgfpathlineto{\pgfqpoint{2.583304in}{0.739656in}}%
\pgfpathlineto{\pgfqpoint{2.582740in}{0.739656in}}%
\pgfpathlineto{\pgfqpoint{2.582176in}{0.739656in}}%
\pgfpathlineto{\pgfqpoint{2.581611in}{0.739656in}}%
\pgfpathlineto{\pgfqpoint{2.581047in}{0.739656in}}%
\pgfpathlineto{\pgfqpoint{2.580483in}{0.739656in}}%
\pgfpathlineto{\pgfqpoint{2.579918in}{0.739656in}}%
\pgfpathlineto{\pgfqpoint{2.579354in}{0.739656in}}%
\pgfpathlineto{\pgfqpoint{2.578789in}{0.739656in}}%
\pgfpathlineto{\pgfqpoint{2.578225in}{0.739656in}}%
\pgfpathlineto{\pgfqpoint{2.577661in}{0.739656in}}%
\pgfpathlineto{\pgfqpoint{2.577096in}{0.739656in}}%
\pgfpathlineto{\pgfqpoint{2.576532in}{0.739656in}}%
\pgfpathlineto{\pgfqpoint{2.575968in}{0.739656in}}%
\pgfpathlineto{\pgfqpoint{2.575403in}{0.739656in}}%
\pgfpathlineto{\pgfqpoint{2.574839in}{0.739656in}}%
\pgfpathlineto{\pgfqpoint{2.574275in}{0.739656in}}%
\pgfpathlineto{\pgfqpoint{2.573710in}{0.739656in}}%
\pgfpathlineto{\pgfqpoint{2.573146in}{0.739656in}}%
\pgfpathlineto{\pgfqpoint{2.572582in}{0.739656in}}%
\pgfpathlineto{\pgfqpoint{2.572017in}{0.739656in}}%
\pgfpathlineto{\pgfqpoint{2.571453in}{0.739656in}}%
\pgfpathlineto{\pgfqpoint{2.570889in}{0.739656in}}%
\pgfpathlineto{\pgfqpoint{2.570324in}{0.739656in}}%
\pgfpathlineto{\pgfqpoint{2.569760in}{0.739656in}}%
\pgfpathlineto{\pgfqpoint{2.569196in}{0.739656in}}%
\pgfpathlineto{\pgfqpoint{2.568631in}{0.739656in}}%
\pgfpathlineto{\pgfqpoint{2.568067in}{0.739656in}}%
\pgfpathlineto{\pgfqpoint{2.567503in}{0.739656in}}%
\pgfpathlineto{\pgfqpoint{2.566938in}{0.739656in}}%
\pgfpathlineto{\pgfqpoint{2.566374in}{0.739656in}}%
\pgfpathlineto{\pgfqpoint{2.565810in}{0.739656in}}%
\pgfpathlineto{\pgfqpoint{2.565245in}{0.739656in}}%
\pgfpathlineto{\pgfqpoint{2.564681in}{0.739656in}}%
\pgfpathlineto{\pgfqpoint{2.564117in}{0.739656in}}%
\pgfpathlineto{\pgfqpoint{2.563552in}{0.739656in}}%
\pgfpathlineto{\pgfqpoint{2.562988in}{0.739656in}}%
\pgfpathlineto{\pgfqpoint{2.562423in}{0.739656in}}%
\pgfpathlineto{\pgfqpoint{2.561859in}{0.739656in}}%
\pgfpathlineto{\pgfqpoint{2.561295in}{0.739656in}}%
\pgfpathlineto{\pgfqpoint{2.560730in}{0.739656in}}%
\pgfpathlineto{\pgfqpoint{2.560166in}{0.739656in}}%
\pgfpathlineto{\pgfqpoint{2.559602in}{0.739656in}}%
\pgfpathlineto{\pgfqpoint{2.559037in}{0.739656in}}%
\pgfpathlineto{\pgfqpoint{2.558473in}{0.739656in}}%
\pgfpathlineto{\pgfqpoint{2.557909in}{0.739656in}}%
\pgfpathlineto{\pgfqpoint{2.557344in}{0.739656in}}%
\pgfpathlineto{\pgfqpoint{2.556780in}{0.739656in}}%
\pgfpathlineto{\pgfqpoint{2.556216in}{0.739656in}}%
\pgfpathlineto{\pgfqpoint{2.555651in}{0.739656in}}%
\pgfpathlineto{\pgfqpoint{2.555087in}{0.739656in}}%
\pgfpathlineto{\pgfqpoint{2.554523in}{0.739656in}}%
\pgfpathlineto{\pgfqpoint{2.553958in}{0.739656in}}%
\pgfpathlineto{\pgfqpoint{2.553394in}{0.739656in}}%
\pgfpathlineto{\pgfqpoint{2.552830in}{0.739656in}}%
\pgfpathlineto{\pgfqpoint{2.552265in}{0.739656in}}%
\pgfpathlineto{\pgfqpoint{2.551701in}{0.739656in}}%
\pgfpathlineto{\pgfqpoint{2.551137in}{0.739656in}}%
\pgfpathlineto{\pgfqpoint{2.550572in}{0.739656in}}%
\pgfpathlineto{\pgfqpoint{2.550008in}{0.739656in}}%
\pgfpathlineto{\pgfqpoint{2.549444in}{0.739656in}}%
\pgfpathlineto{\pgfqpoint{2.548879in}{0.739656in}}%
\pgfpathlineto{\pgfqpoint{2.548315in}{0.739656in}}%
\pgfpathlineto{\pgfqpoint{2.547750in}{0.739656in}}%
\pgfpathlineto{\pgfqpoint{2.547186in}{0.739656in}}%
\pgfpathlineto{\pgfqpoint{2.546622in}{0.739656in}}%
\pgfpathlineto{\pgfqpoint{2.546057in}{0.739656in}}%
\pgfpathlineto{\pgfqpoint{2.545493in}{0.739656in}}%
\pgfpathlineto{\pgfqpoint{2.544929in}{0.739656in}}%
\pgfpathlineto{\pgfqpoint{2.544364in}{0.739656in}}%
\pgfpathlineto{\pgfqpoint{2.543800in}{0.739656in}}%
\pgfpathlineto{\pgfqpoint{2.543236in}{0.739656in}}%
\pgfpathlineto{\pgfqpoint{2.542671in}{0.739656in}}%
\pgfpathlineto{\pgfqpoint{2.542107in}{0.739656in}}%
\pgfpathlineto{\pgfqpoint{2.541543in}{0.739656in}}%
\pgfpathlineto{\pgfqpoint{2.540978in}{0.739656in}}%
\pgfpathlineto{\pgfqpoint{2.540414in}{0.739656in}}%
\pgfpathlineto{\pgfqpoint{2.539850in}{0.739656in}}%
\pgfpathlineto{\pgfqpoint{2.539285in}{0.739656in}}%
\pgfpathlineto{\pgfqpoint{2.538721in}{0.739656in}}%
\pgfpathlineto{\pgfqpoint{2.538157in}{0.739656in}}%
\pgfpathlineto{\pgfqpoint{2.537592in}{0.739656in}}%
\pgfpathlineto{\pgfqpoint{2.537028in}{0.739656in}}%
\pgfpathlineto{\pgfqpoint{2.536464in}{0.739656in}}%
\pgfpathlineto{\pgfqpoint{2.535899in}{0.739656in}}%
\pgfpathlineto{\pgfqpoint{2.535335in}{0.739656in}}%
\pgfpathlineto{\pgfqpoint{2.534771in}{0.739656in}}%
\pgfpathlineto{\pgfqpoint{2.534206in}{0.739656in}}%
\pgfpathlineto{\pgfqpoint{2.533642in}{0.739656in}}%
\pgfpathlineto{\pgfqpoint{2.533077in}{0.739656in}}%
\pgfpathlineto{\pgfqpoint{2.532513in}{0.739656in}}%
\pgfpathlineto{\pgfqpoint{2.531949in}{0.739656in}}%
\pgfpathlineto{\pgfqpoint{2.531384in}{0.739656in}}%
\pgfpathlineto{\pgfqpoint{2.530820in}{0.739656in}}%
\pgfpathlineto{\pgfqpoint{2.530256in}{0.739656in}}%
\pgfpathlineto{\pgfqpoint{2.529691in}{0.739656in}}%
\pgfpathlineto{\pgfqpoint{2.529127in}{0.739656in}}%
\pgfpathlineto{\pgfqpoint{2.528563in}{0.739656in}}%
\pgfpathlineto{\pgfqpoint{2.527998in}{0.739656in}}%
\pgfpathlineto{\pgfqpoint{2.527434in}{0.739656in}}%
\pgfpathlineto{\pgfqpoint{2.526870in}{0.739656in}}%
\pgfpathlineto{\pgfqpoint{2.526305in}{0.739656in}}%
\pgfpathlineto{\pgfqpoint{2.525741in}{0.739656in}}%
\pgfpathlineto{\pgfqpoint{2.525177in}{0.739656in}}%
\pgfpathlineto{\pgfqpoint{2.524612in}{0.739656in}}%
\pgfpathlineto{\pgfqpoint{2.524048in}{0.739656in}}%
\pgfpathlineto{\pgfqpoint{2.523484in}{0.739656in}}%
\pgfpathlineto{\pgfqpoint{2.522919in}{0.739656in}}%
\pgfpathlineto{\pgfqpoint{2.522355in}{0.739656in}}%
\pgfpathlineto{\pgfqpoint{2.521791in}{0.739656in}}%
\pgfpathlineto{\pgfqpoint{2.521226in}{0.739656in}}%
\pgfpathlineto{\pgfqpoint{2.520662in}{0.739656in}}%
\pgfpathlineto{\pgfqpoint{2.520098in}{0.739656in}}%
\pgfpathlineto{\pgfqpoint{2.519533in}{0.739656in}}%
\pgfpathlineto{\pgfqpoint{2.518969in}{0.739656in}}%
\pgfpathlineto{\pgfqpoint{2.518405in}{0.739656in}}%
\pgfpathlineto{\pgfqpoint{2.517840in}{0.739656in}}%
\pgfpathlineto{\pgfqpoint{2.517276in}{0.739656in}}%
\pgfpathlineto{\pgfqpoint{2.516711in}{0.739656in}}%
\pgfpathlineto{\pgfqpoint{2.516147in}{0.739656in}}%
\pgfpathlineto{\pgfqpoint{2.515583in}{0.739656in}}%
\pgfpathlineto{\pgfqpoint{2.515018in}{0.739656in}}%
\pgfpathlineto{\pgfqpoint{2.514454in}{0.739656in}}%
\pgfpathlineto{\pgfqpoint{2.513890in}{0.739656in}}%
\pgfpathlineto{\pgfqpoint{2.513325in}{0.739656in}}%
\pgfpathlineto{\pgfqpoint{2.512761in}{0.739656in}}%
\pgfpathlineto{\pgfqpoint{2.512197in}{0.739656in}}%
\pgfpathlineto{\pgfqpoint{2.511632in}{0.739656in}}%
\pgfpathlineto{\pgfqpoint{2.511068in}{0.739656in}}%
\pgfpathlineto{\pgfqpoint{2.510504in}{0.739656in}}%
\pgfpathlineto{\pgfqpoint{2.509939in}{0.739656in}}%
\pgfpathlineto{\pgfqpoint{2.509375in}{0.739656in}}%
\pgfpathlineto{\pgfqpoint{2.508811in}{0.739656in}}%
\pgfpathlineto{\pgfqpoint{2.508246in}{0.739656in}}%
\pgfpathlineto{\pgfqpoint{2.507682in}{0.739656in}}%
\pgfpathlineto{\pgfqpoint{2.507118in}{0.739656in}}%
\pgfpathlineto{\pgfqpoint{2.506553in}{0.739656in}}%
\pgfpathlineto{\pgfqpoint{2.505989in}{0.739656in}}%
\pgfpathlineto{\pgfqpoint{2.505425in}{0.739656in}}%
\pgfpathlineto{\pgfqpoint{2.504860in}{0.739656in}}%
\pgfpathlineto{\pgfqpoint{2.504296in}{0.739656in}}%
\pgfpathlineto{\pgfqpoint{2.503732in}{0.739656in}}%
\pgfpathlineto{\pgfqpoint{2.503167in}{0.739656in}}%
\pgfpathlineto{\pgfqpoint{2.502603in}{0.739656in}}%
\pgfpathlineto{\pgfqpoint{2.502038in}{0.739656in}}%
\pgfpathlineto{\pgfqpoint{2.501474in}{0.739656in}}%
\pgfpathlineto{\pgfqpoint{2.500910in}{0.739656in}}%
\pgfpathlineto{\pgfqpoint{2.500345in}{0.739656in}}%
\pgfpathlineto{\pgfqpoint{2.499781in}{0.739656in}}%
\pgfpathlineto{\pgfqpoint{2.499217in}{0.739656in}}%
\pgfpathlineto{\pgfqpoint{2.498652in}{0.739656in}}%
\pgfpathlineto{\pgfqpoint{2.498088in}{0.739656in}}%
\pgfpathlineto{\pgfqpoint{2.497524in}{0.739656in}}%
\pgfpathlineto{\pgfqpoint{2.496959in}{0.739656in}}%
\pgfpathlineto{\pgfqpoint{2.496395in}{0.739656in}}%
\pgfpathlineto{\pgfqpoint{2.495831in}{0.739656in}}%
\pgfpathlineto{\pgfqpoint{2.495266in}{0.739656in}}%
\pgfpathlineto{\pgfqpoint{2.494702in}{0.739656in}}%
\pgfpathlineto{\pgfqpoint{2.494138in}{0.739656in}}%
\pgfpathlineto{\pgfqpoint{2.493573in}{0.739656in}}%
\pgfpathlineto{\pgfqpoint{2.493009in}{0.739656in}}%
\pgfpathlineto{\pgfqpoint{2.492445in}{0.739656in}}%
\pgfpathlineto{\pgfqpoint{2.491880in}{0.739656in}}%
\pgfpathlineto{\pgfqpoint{2.491316in}{0.739656in}}%
\pgfpathlineto{\pgfqpoint{2.490752in}{0.739656in}}%
\pgfpathlineto{\pgfqpoint{2.490187in}{0.739656in}}%
\pgfpathlineto{\pgfqpoint{2.489623in}{0.739656in}}%
\pgfpathlineto{\pgfqpoint{2.489059in}{0.739656in}}%
\pgfpathlineto{\pgfqpoint{2.488494in}{0.739656in}}%
\pgfpathlineto{\pgfqpoint{2.487930in}{0.739656in}}%
\pgfpathlineto{\pgfqpoint{2.487365in}{0.739656in}}%
\pgfpathlineto{\pgfqpoint{2.486801in}{0.739656in}}%
\pgfpathlineto{\pgfqpoint{2.486237in}{0.739656in}}%
\pgfpathlineto{\pgfqpoint{2.485672in}{0.739656in}}%
\pgfpathlineto{\pgfqpoint{2.485108in}{0.739656in}}%
\pgfpathlineto{\pgfqpoint{2.484544in}{0.739656in}}%
\pgfpathlineto{\pgfqpoint{2.483979in}{0.739656in}}%
\pgfpathlineto{\pgfqpoint{2.483415in}{0.739656in}}%
\pgfpathlineto{\pgfqpoint{2.482851in}{0.739656in}}%
\pgfpathlineto{\pgfqpoint{2.482286in}{0.739656in}}%
\pgfpathlineto{\pgfqpoint{2.481722in}{0.739656in}}%
\pgfpathlineto{\pgfqpoint{2.481158in}{0.739656in}}%
\pgfpathlineto{\pgfqpoint{2.480593in}{0.739656in}}%
\pgfpathlineto{\pgfqpoint{2.480029in}{0.739656in}}%
\pgfpathlineto{\pgfqpoint{2.479465in}{0.739656in}}%
\pgfpathlineto{\pgfqpoint{2.478900in}{0.739656in}}%
\pgfpathlineto{\pgfqpoint{2.478336in}{0.739656in}}%
\pgfpathlineto{\pgfqpoint{2.477772in}{0.739656in}}%
\pgfpathlineto{\pgfqpoint{2.477207in}{0.739656in}}%
\pgfpathlineto{\pgfqpoint{2.476643in}{0.739656in}}%
\pgfpathlineto{\pgfqpoint{2.476079in}{0.739656in}}%
\pgfpathlineto{\pgfqpoint{2.475514in}{0.739656in}}%
\pgfpathlineto{\pgfqpoint{2.474950in}{0.739656in}}%
\pgfpathlineto{\pgfqpoint{2.474386in}{0.739656in}}%
\pgfpathlineto{\pgfqpoint{2.473821in}{0.739656in}}%
\pgfpathlineto{\pgfqpoint{2.473257in}{0.739656in}}%
\pgfpathlineto{\pgfqpoint{2.472692in}{0.739656in}}%
\pgfpathlineto{\pgfqpoint{2.472128in}{0.739656in}}%
\pgfpathlineto{\pgfqpoint{2.471564in}{0.739656in}}%
\pgfpathlineto{\pgfqpoint{2.470999in}{0.739656in}}%
\pgfpathlineto{\pgfqpoint{2.470435in}{0.739656in}}%
\pgfpathlineto{\pgfqpoint{2.469871in}{0.739656in}}%
\pgfpathlineto{\pgfqpoint{2.469306in}{0.739656in}}%
\pgfpathlineto{\pgfqpoint{2.468742in}{0.739656in}}%
\pgfpathlineto{\pgfqpoint{2.468178in}{0.739656in}}%
\pgfpathlineto{\pgfqpoint{2.467613in}{0.739656in}}%
\pgfpathlineto{\pgfqpoint{2.467049in}{0.739656in}}%
\pgfpathlineto{\pgfqpoint{2.466485in}{0.739656in}}%
\pgfpathlineto{\pgfqpoint{2.465920in}{0.739656in}}%
\pgfpathlineto{\pgfqpoint{2.465356in}{0.739656in}}%
\pgfpathlineto{\pgfqpoint{2.464792in}{0.739656in}}%
\pgfpathlineto{\pgfqpoint{2.464227in}{0.739656in}}%
\pgfpathlineto{\pgfqpoint{2.463663in}{0.739656in}}%
\pgfpathlineto{\pgfqpoint{2.463099in}{0.739656in}}%
\pgfpathlineto{\pgfqpoint{2.462534in}{0.739656in}}%
\pgfpathlineto{\pgfqpoint{2.461970in}{0.739656in}}%
\pgfpathlineto{\pgfqpoint{2.461406in}{0.739656in}}%
\pgfpathlineto{\pgfqpoint{2.460841in}{0.739656in}}%
\pgfpathlineto{\pgfqpoint{2.460277in}{0.739656in}}%
\pgfpathlineto{\pgfqpoint{2.459713in}{0.739656in}}%
\pgfpathlineto{\pgfqpoint{2.459148in}{0.739656in}}%
\pgfpathlineto{\pgfqpoint{2.458584in}{0.739656in}}%
\pgfpathlineto{\pgfqpoint{2.458020in}{0.739656in}}%
\pgfpathlineto{\pgfqpoint{2.457455in}{0.739656in}}%
\pgfpathlineto{\pgfqpoint{2.456891in}{0.739656in}}%
\pgfpathlineto{\pgfqpoint{2.456326in}{0.739656in}}%
\pgfpathlineto{\pgfqpoint{2.455762in}{0.739656in}}%
\pgfpathlineto{\pgfqpoint{2.455198in}{0.739656in}}%
\pgfpathlineto{\pgfqpoint{2.454633in}{0.739656in}}%
\pgfpathlineto{\pgfqpoint{2.454069in}{0.739656in}}%
\pgfpathlineto{\pgfqpoint{2.453505in}{0.739656in}}%
\pgfpathlineto{\pgfqpoint{2.452940in}{0.739656in}}%
\pgfpathlineto{\pgfqpoint{2.452376in}{0.739656in}}%
\pgfpathlineto{\pgfqpoint{2.451812in}{0.739656in}}%
\pgfpathlineto{\pgfqpoint{2.451247in}{0.739656in}}%
\pgfpathlineto{\pgfqpoint{2.450683in}{0.739656in}}%
\pgfpathlineto{\pgfqpoint{2.450119in}{0.739656in}}%
\pgfpathlineto{\pgfqpoint{2.449554in}{0.739656in}}%
\pgfpathlineto{\pgfqpoint{2.448990in}{0.739656in}}%
\pgfpathlineto{\pgfqpoint{2.448426in}{0.739656in}}%
\pgfpathlineto{\pgfqpoint{2.447861in}{0.739656in}}%
\pgfpathlineto{\pgfqpoint{2.447297in}{0.739656in}}%
\pgfpathlineto{\pgfqpoint{2.446733in}{0.739656in}}%
\pgfpathlineto{\pgfqpoint{2.446168in}{0.739656in}}%
\pgfpathlineto{\pgfqpoint{2.445604in}{0.739656in}}%
\pgfpathlineto{\pgfqpoint{2.445040in}{0.739656in}}%
\pgfpathlineto{\pgfqpoint{2.444475in}{0.739656in}}%
\pgfpathlineto{\pgfqpoint{2.443911in}{0.739656in}}%
\pgfpathlineto{\pgfqpoint{2.443347in}{0.739656in}}%
\pgfpathlineto{\pgfqpoint{2.442782in}{0.739656in}}%
\pgfpathlineto{\pgfqpoint{2.442218in}{0.739656in}}%
\pgfpathlineto{\pgfqpoint{2.441653in}{0.739656in}}%
\pgfpathlineto{\pgfqpoint{2.441089in}{0.739656in}}%
\pgfpathlineto{\pgfqpoint{2.440525in}{0.739656in}}%
\pgfpathlineto{\pgfqpoint{2.439960in}{0.739656in}}%
\pgfpathlineto{\pgfqpoint{2.439396in}{0.739656in}}%
\pgfpathlineto{\pgfqpoint{2.438832in}{0.739656in}}%
\pgfpathlineto{\pgfqpoint{2.438267in}{0.739656in}}%
\pgfpathlineto{\pgfqpoint{2.437703in}{0.739656in}}%
\pgfpathlineto{\pgfqpoint{2.437139in}{0.739656in}}%
\pgfpathlineto{\pgfqpoint{2.436574in}{0.739656in}}%
\pgfpathlineto{\pgfqpoint{2.436010in}{0.739656in}}%
\pgfpathlineto{\pgfqpoint{2.435446in}{0.739656in}}%
\pgfpathlineto{\pgfqpoint{2.434881in}{0.739656in}}%
\pgfpathlineto{\pgfqpoint{2.434317in}{0.739656in}}%
\pgfpathlineto{\pgfqpoint{2.433753in}{0.739656in}}%
\pgfpathlineto{\pgfqpoint{2.433188in}{0.739656in}}%
\pgfpathlineto{\pgfqpoint{2.432624in}{0.739656in}}%
\pgfpathlineto{\pgfqpoint{2.432060in}{0.739656in}}%
\pgfpathlineto{\pgfqpoint{2.431495in}{0.739656in}}%
\pgfpathlineto{\pgfqpoint{2.430931in}{0.739656in}}%
\pgfpathlineto{\pgfqpoint{2.430367in}{0.739656in}}%
\pgfpathlineto{\pgfqpoint{2.429802in}{0.739656in}}%
\pgfpathlineto{\pgfqpoint{2.429238in}{0.739656in}}%
\pgfpathlineto{\pgfqpoint{2.428674in}{0.739656in}}%
\pgfpathlineto{\pgfqpoint{2.428109in}{0.739656in}}%
\pgfpathlineto{\pgfqpoint{2.427545in}{0.739656in}}%
\pgfpathlineto{\pgfqpoint{2.426980in}{0.739656in}}%
\pgfpathlineto{\pgfqpoint{2.426416in}{0.739656in}}%
\pgfpathlineto{\pgfqpoint{2.425852in}{0.739656in}}%
\pgfpathlineto{\pgfqpoint{2.425287in}{0.739656in}}%
\pgfpathlineto{\pgfqpoint{2.424723in}{0.739656in}}%
\pgfpathlineto{\pgfqpoint{2.424159in}{0.739656in}}%
\pgfpathlineto{\pgfqpoint{2.423594in}{0.739656in}}%
\pgfpathlineto{\pgfqpoint{2.423030in}{0.739656in}}%
\pgfpathlineto{\pgfqpoint{2.422466in}{0.739656in}}%
\pgfpathlineto{\pgfqpoint{2.421901in}{0.739656in}}%
\pgfpathlineto{\pgfqpoint{2.421337in}{0.739656in}}%
\pgfpathlineto{\pgfqpoint{2.420773in}{0.739656in}}%
\pgfpathlineto{\pgfqpoint{2.420208in}{0.739656in}}%
\pgfpathlineto{\pgfqpoint{2.419644in}{0.739656in}}%
\pgfpathlineto{\pgfqpoint{2.419080in}{0.739656in}}%
\pgfpathlineto{\pgfqpoint{2.418515in}{0.739656in}}%
\pgfpathlineto{\pgfqpoint{2.417951in}{0.739656in}}%
\pgfpathlineto{\pgfqpoint{2.417387in}{0.739656in}}%
\pgfpathlineto{\pgfqpoint{2.416822in}{0.739656in}}%
\pgfpathlineto{\pgfqpoint{2.416258in}{0.739656in}}%
\pgfpathlineto{\pgfqpoint{2.415694in}{0.739656in}}%
\pgfpathlineto{\pgfqpoint{2.415129in}{0.739656in}}%
\pgfpathlineto{\pgfqpoint{2.414565in}{0.739656in}}%
\pgfpathlineto{\pgfqpoint{2.414001in}{0.739656in}}%
\pgfpathlineto{\pgfqpoint{2.413436in}{0.739656in}}%
\pgfpathlineto{\pgfqpoint{2.412872in}{0.739656in}}%
\pgfpathlineto{\pgfqpoint{2.412308in}{0.739656in}}%
\pgfpathlineto{\pgfqpoint{2.411743in}{0.739656in}}%
\pgfpathlineto{\pgfqpoint{2.411179in}{0.739656in}}%
\pgfpathlineto{\pgfqpoint{2.410614in}{0.739656in}}%
\pgfpathlineto{\pgfqpoint{2.410050in}{0.739656in}}%
\pgfpathlineto{\pgfqpoint{2.409486in}{0.739656in}}%
\pgfpathlineto{\pgfqpoint{2.408921in}{0.739656in}}%
\pgfpathlineto{\pgfqpoint{2.408357in}{0.739656in}}%
\pgfpathlineto{\pgfqpoint{2.407793in}{0.739656in}}%
\pgfpathlineto{\pgfqpoint{2.407228in}{0.739656in}}%
\pgfpathlineto{\pgfqpoint{2.406664in}{0.739656in}}%
\pgfpathlineto{\pgfqpoint{2.406100in}{0.739656in}}%
\pgfpathlineto{\pgfqpoint{2.405535in}{0.739656in}}%
\pgfpathlineto{\pgfqpoint{2.404971in}{0.739656in}}%
\pgfpathlineto{\pgfqpoint{2.404407in}{0.739656in}}%
\pgfpathlineto{\pgfqpoint{2.403842in}{0.739656in}}%
\pgfpathlineto{\pgfqpoint{2.403278in}{0.739656in}}%
\pgfpathlineto{\pgfqpoint{2.402714in}{0.739656in}}%
\pgfpathlineto{\pgfqpoint{2.402149in}{0.739656in}}%
\pgfpathlineto{\pgfqpoint{2.401585in}{0.739656in}}%
\pgfpathlineto{\pgfqpoint{2.401021in}{0.739656in}}%
\pgfpathlineto{\pgfqpoint{2.400456in}{0.739656in}}%
\pgfpathlineto{\pgfqpoint{2.399892in}{0.739656in}}%
\pgfpathlineto{\pgfqpoint{2.399328in}{0.739656in}}%
\pgfpathlineto{\pgfqpoint{2.398763in}{0.739656in}}%
\pgfpathlineto{\pgfqpoint{2.398199in}{0.739656in}}%
\pgfpathlineto{\pgfqpoint{2.397635in}{0.739656in}}%
\pgfpathlineto{\pgfqpoint{2.397070in}{0.739656in}}%
\pgfpathlineto{\pgfqpoint{2.396506in}{0.739656in}}%
\pgfpathlineto{\pgfqpoint{2.395941in}{0.739656in}}%
\pgfpathlineto{\pgfqpoint{2.395377in}{0.739656in}}%
\pgfpathlineto{\pgfqpoint{2.394813in}{0.739656in}}%
\pgfpathlineto{\pgfqpoint{2.394248in}{0.739656in}}%
\pgfpathlineto{\pgfqpoint{2.393684in}{0.739656in}}%
\pgfpathlineto{\pgfqpoint{2.393120in}{0.739656in}}%
\pgfpathlineto{\pgfqpoint{2.392555in}{0.739656in}}%
\pgfpathlineto{\pgfqpoint{2.391991in}{0.739656in}}%
\pgfpathlineto{\pgfqpoint{2.391427in}{0.739656in}}%
\pgfpathlineto{\pgfqpoint{2.390862in}{0.739656in}}%
\pgfpathlineto{\pgfqpoint{2.390298in}{0.739656in}}%
\pgfpathlineto{\pgfqpoint{2.389734in}{0.739656in}}%
\pgfpathlineto{\pgfqpoint{2.389169in}{0.739656in}}%
\pgfpathlineto{\pgfqpoint{2.388605in}{0.739656in}}%
\pgfpathlineto{\pgfqpoint{2.388041in}{0.739656in}}%
\pgfpathlineto{\pgfqpoint{2.387476in}{0.739656in}}%
\pgfpathlineto{\pgfqpoint{2.386912in}{0.739656in}}%
\pgfpathlineto{\pgfqpoint{2.386348in}{0.739656in}}%
\pgfpathlineto{\pgfqpoint{2.385783in}{0.739656in}}%
\pgfpathlineto{\pgfqpoint{2.385219in}{0.739656in}}%
\pgfpathlineto{\pgfqpoint{2.384655in}{0.739656in}}%
\pgfpathlineto{\pgfqpoint{2.384090in}{0.739656in}}%
\pgfpathlineto{\pgfqpoint{2.383526in}{0.739656in}}%
\pgfpathlineto{\pgfqpoint{2.382962in}{0.739656in}}%
\pgfpathlineto{\pgfqpoint{2.382397in}{0.739656in}}%
\pgfpathlineto{\pgfqpoint{2.381833in}{0.739656in}}%
\pgfpathlineto{\pgfqpoint{2.381268in}{0.739656in}}%
\pgfpathlineto{\pgfqpoint{2.380704in}{0.739656in}}%
\pgfpathlineto{\pgfqpoint{2.380140in}{0.739656in}}%
\pgfpathlineto{\pgfqpoint{2.379575in}{0.739656in}}%
\pgfpathlineto{\pgfqpoint{2.379011in}{0.739656in}}%
\pgfpathlineto{\pgfqpoint{2.378447in}{0.739656in}}%
\pgfpathlineto{\pgfqpoint{2.377882in}{0.739656in}}%
\pgfpathlineto{\pgfqpoint{2.377318in}{0.739656in}}%
\pgfpathlineto{\pgfqpoint{2.376754in}{0.739656in}}%
\pgfpathlineto{\pgfqpoint{2.376189in}{0.739656in}}%
\pgfpathlineto{\pgfqpoint{2.375625in}{0.739656in}}%
\pgfpathlineto{\pgfqpoint{2.375061in}{0.739656in}}%
\pgfpathlineto{\pgfqpoint{2.374496in}{0.739656in}}%
\pgfpathlineto{\pgfqpoint{2.373932in}{0.739656in}}%
\pgfpathlineto{\pgfqpoint{2.373368in}{0.739656in}}%
\pgfpathlineto{\pgfqpoint{2.372803in}{0.739656in}}%
\pgfpathlineto{\pgfqpoint{2.372239in}{0.739656in}}%
\pgfpathlineto{\pgfqpoint{2.371675in}{0.739656in}}%
\pgfpathlineto{\pgfqpoint{2.371110in}{0.739656in}}%
\pgfpathlineto{\pgfqpoint{2.370546in}{0.739656in}}%
\pgfpathlineto{\pgfqpoint{2.369982in}{0.739656in}}%
\pgfpathlineto{\pgfqpoint{2.369417in}{0.739656in}}%
\pgfpathlineto{\pgfqpoint{2.368853in}{0.739656in}}%
\pgfpathlineto{\pgfqpoint{2.368289in}{0.739656in}}%
\pgfpathlineto{\pgfqpoint{2.367724in}{0.739656in}}%
\pgfpathlineto{\pgfqpoint{2.367160in}{0.739656in}}%
\pgfpathlineto{\pgfqpoint{2.366596in}{0.739656in}}%
\pgfpathlineto{\pgfqpoint{2.366031in}{0.739656in}}%
\pgfpathlineto{\pgfqpoint{2.365467in}{0.739656in}}%
\pgfpathlineto{\pgfqpoint{2.364902in}{0.739656in}}%
\pgfpathlineto{\pgfqpoint{2.364338in}{0.739656in}}%
\pgfpathlineto{\pgfqpoint{2.363774in}{0.739656in}}%
\pgfpathlineto{\pgfqpoint{2.363209in}{0.739656in}}%
\pgfpathlineto{\pgfqpoint{2.362645in}{0.739656in}}%
\pgfpathlineto{\pgfqpoint{2.362081in}{0.739656in}}%
\pgfpathlineto{\pgfqpoint{2.361516in}{0.739656in}}%
\pgfpathlineto{\pgfqpoint{2.360952in}{0.739656in}}%
\pgfpathlineto{\pgfqpoint{2.360388in}{0.739656in}}%
\pgfpathlineto{\pgfqpoint{2.359823in}{0.739656in}}%
\pgfpathlineto{\pgfqpoint{2.359259in}{0.739656in}}%
\pgfpathlineto{\pgfqpoint{2.358695in}{0.739656in}}%
\pgfpathlineto{\pgfqpoint{2.358130in}{0.739656in}}%
\pgfpathlineto{\pgfqpoint{2.357566in}{0.739656in}}%
\pgfpathlineto{\pgfqpoint{2.357002in}{0.739656in}}%
\pgfpathlineto{\pgfqpoint{2.356437in}{0.739656in}}%
\pgfpathlineto{\pgfqpoint{2.355873in}{0.739656in}}%
\pgfpathlineto{\pgfqpoint{2.355309in}{0.739656in}}%
\pgfpathlineto{\pgfqpoint{2.354744in}{0.739656in}}%
\pgfpathlineto{\pgfqpoint{2.354180in}{0.739656in}}%
\pgfpathlineto{\pgfqpoint{2.353616in}{0.739656in}}%
\pgfpathlineto{\pgfqpoint{2.353051in}{0.739656in}}%
\pgfpathlineto{\pgfqpoint{2.352487in}{0.739656in}}%
\pgfpathlineto{\pgfqpoint{2.351923in}{0.739656in}}%
\pgfpathlineto{\pgfqpoint{2.351358in}{0.739656in}}%
\pgfpathlineto{\pgfqpoint{2.350794in}{0.739656in}}%
\pgfpathlineto{\pgfqpoint{2.350229in}{0.739656in}}%
\pgfpathlineto{\pgfqpoint{2.349665in}{0.739656in}}%
\pgfpathlineto{\pgfqpoint{2.349101in}{0.739656in}}%
\pgfpathlineto{\pgfqpoint{2.348536in}{0.739656in}}%
\pgfpathlineto{\pgfqpoint{2.347972in}{0.739656in}}%
\pgfpathlineto{\pgfqpoint{2.347408in}{0.739656in}}%
\pgfpathlineto{\pgfqpoint{2.346843in}{0.739656in}}%
\pgfpathlineto{\pgfqpoint{2.346279in}{0.739656in}}%
\pgfpathlineto{\pgfqpoint{2.345715in}{0.739656in}}%
\pgfpathlineto{\pgfqpoint{2.345150in}{0.739656in}}%
\pgfpathlineto{\pgfqpoint{2.344586in}{0.739656in}}%
\pgfpathlineto{\pgfqpoint{2.344022in}{0.739656in}}%
\pgfpathlineto{\pgfqpoint{2.343457in}{0.739656in}}%
\pgfpathlineto{\pgfqpoint{2.342893in}{0.739656in}}%
\pgfpathlineto{\pgfqpoint{2.342329in}{0.739656in}}%
\pgfpathlineto{\pgfqpoint{2.341764in}{0.739656in}}%
\pgfpathlineto{\pgfqpoint{2.341200in}{0.739656in}}%
\pgfpathlineto{\pgfqpoint{2.340636in}{0.739656in}}%
\pgfpathlineto{\pgfqpoint{2.340071in}{0.739656in}}%
\pgfpathlineto{\pgfqpoint{2.339507in}{0.739656in}}%
\pgfpathlineto{\pgfqpoint{2.338943in}{0.739656in}}%
\pgfpathlineto{\pgfqpoint{2.338378in}{0.739656in}}%
\pgfpathlineto{\pgfqpoint{2.337814in}{0.739656in}}%
\pgfpathlineto{\pgfqpoint{2.337250in}{0.739656in}}%
\pgfpathlineto{\pgfqpoint{2.336685in}{0.739656in}}%
\pgfpathlineto{\pgfqpoint{2.336121in}{0.739656in}}%
\pgfpathlineto{\pgfqpoint{2.335556in}{0.739656in}}%
\pgfpathlineto{\pgfqpoint{2.334992in}{0.739656in}}%
\pgfpathlineto{\pgfqpoint{2.334428in}{0.739656in}}%
\pgfpathlineto{\pgfqpoint{2.333863in}{0.739656in}}%
\pgfpathlineto{\pgfqpoint{2.333299in}{0.739656in}}%
\pgfpathlineto{\pgfqpoint{2.332735in}{0.739656in}}%
\pgfpathlineto{\pgfqpoint{2.332170in}{0.739656in}}%
\pgfpathlineto{\pgfqpoint{2.331606in}{0.739656in}}%
\pgfpathlineto{\pgfqpoint{2.331042in}{0.739656in}}%
\pgfpathlineto{\pgfqpoint{2.330477in}{0.739656in}}%
\pgfpathlineto{\pgfqpoint{2.329913in}{0.739656in}}%
\pgfpathlineto{\pgfqpoint{2.329349in}{0.739656in}}%
\pgfpathlineto{\pgfqpoint{2.328784in}{0.739656in}}%
\pgfpathlineto{\pgfqpoint{2.328220in}{0.739656in}}%
\pgfpathlineto{\pgfqpoint{2.327656in}{0.739656in}}%
\pgfpathlineto{\pgfqpoint{2.327091in}{0.739656in}}%
\pgfpathlineto{\pgfqpoint{2.326527in}{0.739656in}}%
\pgfpathlineto{\pgfqpoint{2.325963in}{0.739656in}}%
\pgfpathlineto{\pgfqpoint{2.325398in}{0.739656in}}%
\pgfpathlineto{\pgfqpoint{2.324834in}{0.739656in}}%
\pgfpathlineto{\pgfqpoint{2.324270in}{0.739656in}}%
\pgfpathlineto{\pgfqpoint{2.323705in}{0.739656in}}%
\pgfpathlineto{\pgfqpoint{2.323141in}{0.739656in}}%
\pgfpathlineto{\pgfqpoint{2.322577in}{0.739656in}}%
\pgfpathlineto{\pgfqpoint{2.322012in}{0.739656in}}%
\pgfpathlineto{\pgfqpoint{2.321448in}{0.739656in}}%
\pgfpathlineto{\pgfqpoint{2.320884in}{0.739656in}}%
\pgfpathlineto{\pgfqpoint{2.320319in}{0.739656in}}%
\pgfpathlineto{\pgfqpoint{2.319755in}{0.739656in}}%
\pgfpathlineto{\pgfqpoint{2.319190in}{0.739656in}}%
\pgfpathlineto{\pgfqpoint{2.318626in}{0.739656in}}%
\pgfpathlineto{\pgfqpoint{2.318062in}{0.739656in}}%
\pgfpathlineto{\pgfqpoint{2.317497in}{0.739656in}}%
\pgfpathlineto{\pgfqpoint{2.316933in}{0.739656in}}%
\pgfpathlineto{\pgfqpoint{2.316369in}{0.739656in}}%
\pgfpathlineto{\pgfqpoint{2.315804in}{0.739656in}}%
\pgfpathlineto{\pgfqpoint{2.315240in}{0.739656in}}%
\pgfpathlineto{\pgfqpoint{2.314676in}{0.739656in}}%
\pgfpathlineto{\pgfqpoint{2.314111in}{0.739656in}}%
\pgfpathlineto{\pgfqpoint{2.313547in}{0.739656in}}%
\pgfpathlineto{\pgfqpoint{2.312983in}{0.739656in}}%
\pgfpathlineto{\pgfqpoint{2.312418in}{0.739656in}}%
\pgfpathlineto{\pgfqpoint{2.311854in}{0.739656in}}%
\pgfpathlineto{\pgfqpoint{2.311290in}{0.739656in}}%
\pgfpathlineto{\pgfqpoint{2.310725in}{0.739656in}}%
\pgfpathlineto{\pgfqpoint{2.310161in}{0.739656in}}%
\pgfpathlineto{\pgfqpoint{2.309597in}{0.739656in}}%
\pgfpathlineto{\pgfqpoint{2.309032in}{0.739656in}}%
\pgfpathlineto{\pgfqpoint{2.308468in}{0.739656in}}%
\pgfpathlineto{\pgfqpoint{2.307904in}{0.739656in}}%
\pgfpathlineto{\pgfqpoint{2.307339in}{0.739656in}}%
\pgfpathlineto{\pgfqpoint{2.306775in}{0.739656in}}%
\pgfpathlineto{\pgfqpoint{2.306211in}{0.739656in}}%
\pgfpathlineto{\pgfqpoint{2.305646in}{0.739656in}}%
\pgfpathlineto{\pgfqpoint{2.305082in}{0.739656in}}%
\pgfpathlineto{\pgfqpoint{2.304517in}{0.739656in}}%
\pgfpathlineto{\pgfqpoint{2.303953in}{0.739656in}}%
\pgfpathlineto{\pgfqpoint{2.303389in}{0.739656in}}%
\pgfpathlineto{\pgfqpoint{2.302824in}{0.739656in}}%
\pgfpathlineto{\pgfqpoint{2.302260in}{0.739656in}}%
\pgfpathlineto{\pgfqpoint{2.301696in}{0.739656in}}%
\pgfpathlineto{\pgfqpoint{2.301131in}{0.739656in}}%
\pgfpathlineto{\pgfqpoint{2.300567in}{0.739656in}}%
\pgfpathlineto{\pgfqpoint{2.300003in}{0.739656in}}%
\pgfpathlineto{\pgfqpoint{2.299438in}{0.739656in}}%
\pgfpathlineto{\pgfqpoint{2.298874in}{0.739656in}}%
\pgfpathlineto{\pgfqpoint{2.298310in}{0.739656in}}%
\pgfpathlineto{\pgfqpoint{2.297745in}{0.739656in}}%
\pgfpathlineto{\pgfqpoint{2.297181in}{0.739656in}}%
\pgfpathlineto{\pgfqpoint{2.296617in}{0.739656in}}%
\pgfpathlineto{\pgfqpoint{2.296052in}{0.739656in}}%
\pgfpathlineto{\pgfqpoint{2.295488in}{0.739656in}}%
\pgfpathlineto{\pgfqpoint{2.294924in}{0.739656in}}%
\pgfpathlineto{\pgfqpoint{2.294359in}{0.739656in}}%
\pgfpathlineto{\pgfqpoint{2.293795in}{0.739656in}}%
\pgfpathlineto{\pgfqpoint{2.293231in}{0.739656in}}%
\pgfpathlineto{\pgfqpoint{2.292666in}{0.739656in}}%
\pgfpathlineto{\pgfqpoint{2.292102in}{0.739656in}}%
\pgfpathlineto{\pgfqpoint{2.291538in}{0.739656in}}%
\pgfpathlineto{\pgfqpoint{2.290973in}{0.739656in}}%
\pgfpathlineto{\pgfqpoint{2.290409in}{0.739656in}}%
\pgfpathlineto{\pgfqpoint{2.289844in}{0.739656in}}%
\pgfpathlineto{\pgfqpoint{2.289280in}{0.739656in}}%
\pgfpathlineto{\pgfqpoint{2.288716in}{0.739656in}}%
\pgfpathlineto{\pgfqpoint{2.288151in}{0.739656in}}%
\pgfpathlineto{\pgfqpoint{2.287587in}{0.739656in}}%
\pgfpathlineto{\pgfqpoint{2.287023in}{0.739656in}}%
\pgfpathlineto{\pgfqpoint{2.286458in}{0.739656in}}%
\pgfpathlineto{\pgfqpoint{2.285894in}{0.739656in}}%
\pgfpathlineto{\pgfqpoint{2.285330in}{0.739656in}}%
\pgfpathlineto{\pgfqpoint{2.284765in}{0.739656in}}%
\pgfpathlineto{\pgfqpoint{2.284201in}{0.739656in}}%
\pgfpathlineto{\pgfqpoint{2.283637in}{0.739656in}}%
\pgfpathlineto{\pgfqpoint{2.283072in}{0.739656in}}%
\pgfpathlineto{\pgfqpoint{2.282508in}{0.739656in}}%
\pgfpathlineto{\pgfqpoint{2.281944in}{0.739656in}}%
\pgfpathlineto{\pgfqpoint{2.281379in}{0.739656in}}%
\pgfpathlineto{\pgfqpoint{2.280815in}{0.739656in}}%
\pgfpathlineto{\pgfqpoint{2.280251in}{0.739656in}}%
\pgfpathlineto{\pgfqpoint{2.279686in}{0.739656in}}%
\pgfpathlineto{\pgfqpoint{2.279122in}{0.739656in}}%
\pgfpathlineto{\pgfqpoint{2.278558in}{0.739656in}}%
\pgfpathlineto{\pgfqpoint{2.277993in}{0.739656in}}%
\pgfpathlineto{\pgfqpoint{2.277429in}{0.739656in}}%
\pgfpathlineto{\pgfqpoint{2.276865in}{0.739656in}}%
\pgfpathlineto{\pgfqpoint{2.276300in}{0.739656in}}%
\pgfpathlineto{\pgfqpoint{2.275736in}{0.739656in}}%
\pgfpathlineto{\pgfqpoint{2.275172in}{0.739656in}}%
\pgfpathlineto{\pgfqpoint{2.274607in}{0.739656in}}%
\pgfpathlineto{\pgfqpoint{2.274043in}{0.739656in}}%
\pgfpathlineto{\pgfqpoint{2.273478in}{0.739656in}}%
\pgfpathlineto{\pgfqpoint{2.272914in}{0.739656in}}%
\pgfpathlineto{\pgfqpoint{2.272350in}{0.739656in}}%
\pgfpathlineto{\pgfqpoint{2.271785in}{0.739656in}}%
\pgfpathlineto{\pgfqpoint{2.271221in}{0.739656in}}%
\pgfpathlineto{\pgfqpoint{2.270657in}{0.739656in}}%
\pgfpathlineto{\pgfqpoint{2.270092in}{0.739656in}}%
\pgfpathlineto{\pgfqpoint{2.269528in}{0.739656in}}%
\pgfpathlineto{\pgfqpoint{2.268964in}{0.739656in}}%
\pgfpathlineto{\pgfqpoint{2.268399in}{0.739656in}}%
\pgfpathlineto{\pgfqpoint{2.267835in}{0.739656in}}%
\pgfpathlineto{\pgfqpoint{2.267271in}{0.739656in}}%
\pgfpathlineto{\pgfqpoint{2.266706in}{0.739656in}}%
\pgfpathlineto{\pgfqpoint{2.266142in}{0.739656in}}%
\pgfpathlineto{\pgfqpoint{2.265578in}{0.739656in}}%
\pgfpathlineto{\pgfqpoint{2.265013in}{0.739656in}}%
\pgfpathlineto{\pgfqpoint{2.264449in}{0.739656in}}%
\pgfpathlineto{\pgfqpoint{2.263885in}{0.739656in}}%
\pgfpathlineto{\pgfqpoint{2.263320in}{0.739656in}}%
\pgfpathlineto{\pgfqpoint{2.262756in}{0.739656in}}%
\pgfpathlineto{\pgfqpoint{2.262192in}{0.739656in}}%
\pgfpathlineto{\pgfqpoint{2.261627in}{0.739656in}}%
\pgfpathlineto{\pgfqpoint{2.261063in}{0.739656in}}%
\pgfpathlineto{\pgfqpoint{2.260499in}{0.739656in}}%
\pgfpathlineto{\pgfqpoint{2.259934in}{0.739656in}}%
\pgfpathlineto{\pgfqpoint{2.259370in}{0.739656in}}%
\pgfpathlineto{\pgfqpoint{2.258805in}{0.739656in}}%
\pgfpathlineto{\pgfqpoint{2.258241in}{0.739656in}}%
\pgfpathlineto{\pgfqpoint{2.257677in}{0.739656in}}%
\pgfpathlineto{\pgfqpoint{2.257112in}{0.739656in}}%
\pgfpathlineto{\pgfqpoint{2.256548in}{0.739656in}}%
\pgfpathlineto{\pgfqpoint{2.255984in}{0.739656in}}%
\pgfpathlineto{\pgfqpoint{2.255419in}{0.739656in}}%
\pgfpathlineto{\pgfqpoint{2.254855in}{0.739656in}}%
\pgfpathlineto{\pgfqpoint{2.254291in}{0.739656in}}%
\pgfpathlineto{\pgfqpoint{2.253726in}{0.739656in}}%
\pgfpathlineto{\pgfqpoint{2.253162in}{0.739656in}}%
\pgfpathlineto{\pgfqpoint{2.252598in}{0.739656in}}%
\pgfpathlineto{\pgfqpoint{2.252033in}{0.739656in}}%
\pgfpathlineto{\pgfqpoint{2.251469in}{0.739656in}}%
\pgfpathlineto{\pgfqpoint{2.250905in}{0.739656in}}%
\pgfpathlineto{\pgfqpoint{2.250340in}{0.739656in}}%
\pgfpathlineto{\pgfqpoint{2.249776in}{0.739656in}}%
\pgfpathlineto{\pgfqpoint{2.249212in}{0.739656in}}%
\pgfpathlineto{\pgfqpoint{2.248647in}{0.739656in}}%
\pgfpathlineto{\pgfqpoint{2.248083in}{0.739656in}}%
\pgfpathlineto{\pgfqpoint{2.247519in}{0.739656in}}%
\pgfpathlineto{\pgfqpoint{2.246954in}{0.739656in}}%
\pgfpathlineto{\pgfqpoint{2.246390in}{0.739656in}}%
\pgfpathlineto{\pgfqpoint{2.245826in}{0.739656in}}%
\pgfpathlineto{\pgfqpoint{2.245261in}{0.739656in}}%
\pgfpathlineto{\pgfqpoint{2.244697in}{0.739656in}}%
\pgfpathlineto{\pgfqpoint{2.244132in}{0.739656in}}%
\pgfpathlineto{\pgfqpoint{2.243568in}{0.739656in}}%
\pgfpathlineto{\pgfqpoint{2.243004in}{0.739656in}}%
\pgfpathlineto{\pgfqpoint{2.242439in}{0.739656in}}%
\pgfpathlineto{\pgfqpoint{2.241875in}{0.739656in}}%
\pgfpathlineto{\pgfqpoint{2.241311in}{0.739656in}}%
\pgfpathlineto{\pgfqpoint{2.240746in}{0.739656in}}%
\pgfpathlineto{\pgfqpoint{2.240182in}{0.739656in}}%
\pgfpathlineto{\pgfqpoint{2.239618in}{0.739656in}}%
\pgfpathlineto{\pgfqpoint{2.239053in}{0.739656in}}%
\pgfpathlineto{\pgfqpoint{2.238489in}{0.739656in}}%
\pgfpathlineto{\pgfqpoint{2.237925in}{0.739656in}}%
\pgfpathlineto{\pgfqpoint{2.237360in}{0.739656in}}%
\pgfpathlineto{\pgfqpoint{2.236796in}{0.739656in}}%
\pgfpathlineto{\pgfqpoint{2.236232in}{0.739656in}}%
\pgfpathlineto{\pgfqpoint{2.235667in}{0.739656in}}%
\pgfpathlineto{\pgfqpoint{2.235103in}{0.739656in}}%
\pgfpathlineto{\pgfqpoint{2.234539in}{0.739656in}}%
\pgfpathlineto{\pgfqpoint{2.233974in}{0.739656in}}%
\pgfpathlineto{\pgfqpoint{2.233410in}{0.739656in}}%
\pgfpathlineto{\pgfqpoint{2.232846in}{0.739656in}}%
\pgfpathlineto{\pgfqpoint{2.232281in}{0.739656in}}%
\pgfpathlineto{\pgfqpoint{2.231717in}{0.739656in}}%
\pgfpathlineto{\pgfqpoint{2.231153in}{0.739656in}}%
\pgfpathlineto{\pgfqpoint{2.230588in}{0.739656in}}%
\pgfpathlineto{\pgfqpoint{2.230024in}{0.739656in}}%
\pgfpathlineto{\pgfqpoint{2.229459in}{0.739656in}}%
\pgfpathlineto{\pgfqpoint{2.228895in}{0.739656in}}%
\pgfpathlineto{\pgfqpoint{2.228331in}{0.739656in}}%
\pgfpathlineto{\pgfqpoint{2.227766in}{0.739656in}}%
\pgfpathlineto{\pgfqpoint{2.227202in}{0.739656in}}%
\pgfpathlineto{\pgfqpoint{2.226638in}{0.739656in}}%
\pgfpathlineto{\pgfqpoint{2.226073in}{0.739656in}}%
\pgfpathlineto{\pgfqpoint{2.225509in}{0.739656in}}%
\pgfpathlineto{\pgfqpoint{2.224945in}{0.739656in}}%
\pgfpathlineto{\pgfqpoint{2.224380in}{0.739656in}}%
\pgfpathlineto{\pgfqpoint{2.223816in}{0.739656in}}%
\pgfpathlineto{\pgfqpoint{2.223252in}{0.739656in}}%
\pgfpathlineto{\pgfqpoint{2.222687in}{0.739656in}}%
\pgfpathlineto{\pgfqpoint{2.222123in}{0.739656in}}%
\pgfpathlineto{\pgfqpoint{2.221559in}{0.739656in}}%
\pgfpathlineto{\pgfqpoint{2.220994in}{0.739656in}}%
\pgfpathlineto{\pgfqpoint{2.220430in}{0.739656in}}%
\pgfpathlineto{\pgfqpoint{2.219866in}{0.739656in}}%
\pgfpathlineto{\pgfqpoint{2.219301in}{0.739656in}}%
\pgfpathlineto{\pgfqpoint{2.218737in}{0.739656in}}%
\pgfpathlineto{\pgfqpoint{2.218173in}{0.739656in}}%
\pgfpathlineto{\pgfqpoint{2.217608in}{0.739656in}}%
\pgfpathlineto{\pgfqpoint{2.217044in}{0.739656in}}%
\pgfpathlineto{\pgfqpoint{2.216480in}{0.739656in}}%
\pgfpathlineto{\pgfqpoint{2.215915in}{0.739656in}}%
\pgfpathlineto{\pgfqpoint{2.215351in}{0.739656in}}%
\pgfpathlineto{\pgfqpoint{2.214787in}{0.739656in}}%
\pgfpathlineto{\pgfqpoint{2.214222in}{0.739656in}}%
\pgfpathlineto{\pgfqpoint{2.213658in}{0.739656in}}%
\pgfpathlineto{\pgfqpoint{2.213093in}{0.739656in}}%
\pgfpathlineto{\pgfqpoint{2.212529in}{0.739656in}}%
\pgfpathlineto{\pgfqpoint{2.211965in}{0.739656in}}%
\pgfpathlineto{\pgfqpoint{2.211400in}{0.739656in}}%
\pgfpathlineto{\pgfqpoint{2.210836in}{0.739656in}}%
\pgfpathlineto{\pgfqpoint{2.210272in}{0.739656in}}%
\pgfpathlineto{\pgfqpoint{2.209707in}{0.739656in}}%
\pgfpathlineto{\pgfqpoint{2.209143in}{0.739656in}}%
\pgfpathlineto{\pgfqpoint{2.208579in}{0.739656in}}%
\pgfpathlineto{\pgfqpoint{2.208014in}{0.739656in}}%
\pgfpathlineto{\pgfqpoint{2.207450in}{0.739656in}}%
\pgfpathlineto{\pgfqpoint{2.206886in}{0.739656in}}%
\pgfpathlineto{\pgfqpoint{2.206321in}{0.739656in}}%
\pgfpathlineto{\pgfqpoint{2.205757in}{0.739656in}}%
\pgfpathlineto{\pgfqpoint{2.205193in}{0.739656in}}%
\pgfpathlineto{\pgfqpoint{2.204628in}{0.739656in}}%
\pgfpathlineto{\pgfqpoint{2.204064in}{0.739656in}}%
\pgfpathlineto{\pgfqpoint{2.203500in}{0.739656in}}%
\pgfpathlineto{\pgfqpoint{2.202935in}{0.739656in}}%
\pgfpathlineto{\pgfqpoint{2.202371in}{0.739656in}}%
\pgfpathlineto{\pgfqpoint{2.201807in}{0.739656in}}%
\pgfpathlineto{\pgfqpoint{2.201242in}{0.739656in}}%
\pgfpathlineto{\pgfqpoint{2.200678in}{0.739656in}}%
\pgfpathlineto{\pgfqpoint{2.200114in}{0.739656in}}%
\pgfpathlineto{\pgfqpoint{2.199549in}{0.739656in}}%
\pgfpathlineto{\pgfqpoint{2.198985in}{0.739656in}}%
\pgfpathlineto{\pgfqpoint{2.198420in}{0.739656in}}%
\pgfpathlineto{\pgfqpoint{2.197856in}{0.739656in}}%
\pgfpathlineto{\pgfqpoint{2.197292in}{0.739656in}}%
\pgfpathlineto{\pgfqpoint{2.196727in}{0.739656in}}%
\pgfpathlineto{\pgfqpoint{2.196163in}{0.739656in}}%
\pgfpathlineto{\pgfqpoint{2.195599in}{0.739656in}}%
\pgfpathlineto{\pgfqpoint{2.195034in}{0.739656in}}%
\pgfpathlineto{\pgfqpoint{2.194470in}{0.739656in}}%
\pgfpathlineto{\pgfqpoint{2.193906in}{0.739656in}}%
\pgfpathlineto{\pgfqpoint{2.193341in}{0.739656in}}%
\pgfpathlineto{\pgfqpoint{2.192777in}{0.739656in}}%
\pgfpathlineto{\pgfqpoint{2.192213in}{0.739656in}}%
\pgfpathlineto{\pgfqpoint{2.191648in}{0.739656in}}%
\pgfpathlineto{\pgfqpoint{2.191084in}{0.739656in}}%
\pgfpathlineto{\pgfqpoint{2.190520in}{0.739656in}}%
\pgfpathlineto{\pgfqpoint{2.189955in}{0.739656in}}%
\pgfpathlineto{\pgfqpoint{2.189391in}{0.739656in}}%
\pgfpathlineto{\pgfqpoint{2.188827in}{0.739656in}}%
\pgfpathlineto{\pgfqpoint{2.188262in}{0.739656in}}%
\pgfpathlineto{\pgfqpoint{2.187698in}{0.739656in}}%
\pgfpathlineto{\pgfqpoint{2.187134in}{0.739656in}}%
\pgfpathlineto{\pgfqpoint{2.186569in}{0.739656in}}%
\pgfpathlineto{\pgfqpoint{2.186005in}{0.739656in}}%
\pgfpathlineto{\pgfqpoint{2.185441in}{0.739656in}}%
\pgfpathlineto{\pgfqpoint{2.184876in}{0.739656in}}%
\pgfpathlineto{\pgfqpoint{2.184312in}{0.739656in}}%
\pgfpathlineto{\pgfqpoint{2.183747in}{0.739656in}}%
\pgfpathlineto{\pgfqpoint{2.183183in}{0.739656in}}%
\pgfpathlineto{\pgfqpoint{2.182619in}{0.739656in}}%
\pgfpathlineto{\pgfqpoint{2.182054in}{0.739656in}}%
\pgfpathlineto{\pgfqpoint{2.181490in}{0.739656in}}%
\pgfpathlineto{\pgfqpoint{2.180926in}{0.739656in}}%
\pgfpathlineto{\pgfqpoint{2.180361in}{0.739656in}}%
\pgfpathlineto{\pgfqpoint{2.179797in}{0.739656in}}%
\pgfpathlineto{\pgfqpoint{2.179233in}{0.739656in}}%
\pgfpathlineto{\pgfqpoint{2.178668in}{0.739656in}}%
\pgfpathlineto{\pgfqpoint{2.178104in}{0.739656in}}%
\pgfpathlineto{\pgfqpoint{2.177540in}{0.739656in}}%
\pgfpathlineto{\pgfqpoint{2.176975in}{0.739656in}}%
\pgfpathlineto{\pgfqpoint{2.176411in}{0.739656in}}%
\pgfpathlineto{\pgfqpoint{2.175847in}{0.739656in}}%
\pgfpathlineto{\pgfqpoint{2.175282in}{0.739656in}}%
\pgfpathlineto{\pgfqpoint{2.174718in}{0.739656in}}%
\pgfpathlineto{\pgfqpoint{2.174154in}{0.739656in}}%
\pgfpathlineto{\pgfqpoint{2.173589in}{0.739656in}}%
\pgfpathlineto{\pgfqpoint{2.173025in}{0.739656in}}%
\pgfpathlineto{\pgfqpoint{2.172461in}{0.739656in}}%
\pgfpathlineto{\pgfqpoint{2.171896in}{0.739656in}}%
\pgfpathlineto{\pgfqpoint{2.171332in}{0.739656in}}%
\pgfpathlineto{\pgfqpoint{2.170768in}{0.739656in}}%
\pgfpathlineto{\pgfqpoint{2.170203in}{0.739656in}}%
\pgfpathlineto{\pgfqpoint{2.169639in}{0.739656in}}%
\pgfpathlineto{\pgfqpoint{2.169075in}{0.739656in}}%
\pgfpathlineto{\pgfqpoint{2.168510in}{0.739656in}}%
\pgfpathlineto{\pgfqpoint{2.167946in}{0.739656in}}%
\pgfpathlineto{\pgfqpoint{2.167381in}{0.739656in}}%
\pgfpathlineto{\pgfqpoint{2.166817in}{0.739656in}}%
\pgfpathlineto{\pgfqpoint{2.166253in}{0.739656in}}%
\pgfpathlineto{\pgfqpoint{2.165688in}{0.739656in}}%
\pgfpathlineto{\pgfqpoint{2.165124in}{0.739656in}}%
\pgfpathlineto{\pgfqpoint{2.164560in}{0.739656in}}%
\pgfpathlineto{\pgfqpoint{2.163995in}{0.739656in}}%
\pgfpathlineto{\pgfqpoint{2.163431in}{0.739656in}}%
\pgfpathlineto{\pgfqpoint{2.162867in}{0.739656in}}%
\pgfpathlineto{\pgfqpoint{2.162302in}{0.739656in}}%
\pgfpathlineto{\pgfqpoint{2.161738in}{0.739656in}}%
\pgfpathlineto{\pgfqpoint{2.161174in}{0.739656in}}%
\pgfpathlineto{\pgfqpoint{2.160609in}{0.739656in}}%
\pgfpathlineto{\pgfqpoint{2.160045in}{0.739656in}}%
\pgfpathlineto{\pgfqpoint{2.159481in}{0.739656in}}%
\pgfpathlineto{\pgfqpoint{2.158916in}{0.739656in}}%
\pgfpathlineto{\pgfqpoint{2.158352in}{0.739656in}}%
\pgfpathlineto{\pgfqpoint{2.157788in}{0.739656in}}%
\pgfpathlineto{\pgfqpoint{2.157223in}{0.739656in}}%
\pgfpathlineto{\pgfqpoint{2.156659in}{0.739656in}}%
\pgfpathlineto{\pgfqpoint{2.156095in}{0.739656in}}%
\pgfpathlineto{\pgfqpoint{2.155530in}{0.739656in}}%
\pgfpathlineto{\pgfqpoint{2.154966in}{0.739656in}}%
\pgfpathlineto{\pgfqpoint{2.154402in}{0.739656in}}%
\pgfpathlineto{\pgfqpoint{2.153837in}{0.739656in}}%
\pgfpathlineto{\pgfqpoint{2.153273in}{0.739656in}}%
\pgfpathlineto{\pgfqpoint{2.152708in}{0.739656in}}%
\pgfpathlineto{\pgfqpoint{2.152144in}{0.739656in}}%
\pgfpathlineto{\pgfqpoint{2.151580in}{0.739656in}}%
\pgfpathlineto{\pgfqpoint{2.151015in}{0.739656in}}%
\pgfpathlineto{\pgfqpoint{2.150451in}{0.739656in}}%
\pgfpathlineto{\pgfqpoint{2.149887in}{0.739656in}}%
\pgfpathlineto{\pgfqpoint{2.149322in}{0.739656in}}%
\pgfpathlineto{\pgfqpoint{2.148758in}{0.739656in}}%
\pgfpathlineto{\pgfqpoint{2.148194in}{0.739656in}}%
\pgfpathlineto{\pgfqpoint{2.147629in}{0.739656in}}%
\pgfpathlineto{\pgfqpoint{2.147065in}{0.739656in}}%
\pgfpathlineto{\pgfqpoint{2.146501in}{0.739656in}}%
\pgfpathlineto{\pgfqpoint{2.145936in}{0.739656in}}%
\pgfpathlineto{\pgfqpoint{2.145372in}{0.739656in}}%
\pgfpathlineto{\pgfqpoint{2.144808in}{0.739656in}}%
\pgfpathlineto{\pgfqpoint{2.144243in}{0.739656in}}%
\pgfpathlineto{\pgfqpoint{2.143679in}{0.739656in}}%
\pgfpathlineto{\pgfqpoint{2.143115in}{0.739656in}}%
\pgfpathlineto{\pgfqpoint{2.142550in}{0.739656in}}%
\pgfpathlineto{\pgfqpoint{2.141986in}{0.739656in}}%
\pgfpathlineto{\pgfqpoint{2.141422in}{0.739656in}}%
\pgfpathlineto{\pgfqpoint{2.140857in}{0.739656in}}%
\pgfpathlineto{\pgfqpoint{2.140293in}{0.739656in}}%
\pgfpathlineto{\pgfqpoint{2.139729in}{0.739656in}}%
\pgfpathlineto{\pgfqpoint{2.139164in}{0.739656in}}%
\pgfpathlineto{\pgfqpoint{2.138600in}{0.739656in}}%
\pgfpathlineto{\pgfqpoint{2.138035in}{0.739656in}}%
\pgfpathlineto{\pgfqpoint{2.137471in}{0.739656in}}%
\pgfpathlineto{\pgfqpoint{2.136907in}{0.739656in}}%
\pgfpathlineto{\pgfqpoint{2.136342in}{0.739656in}}%
\pgfpathlineto{\pgfqpoint{2.135778in}{0.739656in}}%
\pgfpathlineto{\pgfqpoint{2.135214in}{0.739656in}}%
\pgfpathlineto{\pgfqpoint{2.134649in}{0.739656in}}%
\pgfpathlineto{\pgfqpoint{2.134085in}{0.739656in}}%
\pgfpathlineto{\pgfqpoint{2.133521in}{0.739656in}}%
\pgfpathlineto{\pgfqpoint{2.132956in}{0.739656in}}%
\pgfpathlineto{\pgfqpoint{2.132392in}{0.739656in}}%
\pgfpathlineto{\pgfqpoint{2.131828in}{0.739656in}}%
\pgfpathlineto{\pgfqpoint{2.131263in}{0.739656in}}%
\pgfpathlineto{\pgfqpoint{2.130699in}{0.739656in}}%
\pgfpathlineto{\pgfqpoint{2.130135in}{0.739656in}}%
\pgfpathlineto{\pgfqpoint{2.129570in}{0.739656in}}%
\pgfpathlineto{\pgfqpoint{2.129006in}{0.739656in}}%
\pgfpathlineto{\pgfqpoint{2.128442in}{0.739656in}}%
\pgfpathlineto{\pgfqpoint{2.127877in}{0.739656in}}%
\pgfpathlineto{\pgfqpoint{2.127313in}{0.739656in}}%
\pgfpathlineto{\pgfqpoint{2.126749in}{0.739656in}}%
\pgfpathlineto{\pgfqpoint{2.126184in}{0.739656in}}%
\pgfpathlineto{\pgfqpoint{2.125620in}{0.739656in}}%
\pgfpathlineto{\pgfqpoint{2.125056in}{0.739656in}}%
\pgfpathlineto{\pgfqpoint{2.124491in}{0.739656in}}%
\pgfpathlineto{\pgfqpoint{2.123927in}{0.739656in}}%
\pgfpathlineto{\pgfqpoint{2.123363in}{0.739656in}}%
\pgfpathlineto{\pgfqpoint{2.122798in}{0.739656in}}%
\pgfpathlineto{\pgfqpoint{2.122234in}{0.739656in}}%
\pgfpathlineto{\pgfqpoint{2.121669in}{0.739656in}}%
\pgfpathlineto{\pgfqpoint{2.121105in}{0.739656in}}%
\pgfpathlineto{\pgfqpoint{2.120541in}{0.739656in}}%
\pgfpathlineto{\pgfqpoint{2.119976in}{0.739656in}}%
\pgfpathlineto{\pgfqpoint{2.119412in}{0.739656in}}%
\pgfpathlineto{\pgfqpoint{2.118848in}{0.739656in}}%
\pgfpathlineto{\pgfqpoint{2.118283in}{0.739656in}}%
\pgfpathlineto{\pgfqpoint{2.117719in}{0.739656in}}%
\pgfpathlineto{\pgfqpoint{2.117155in}{0.739656in}}%
\pgfpathlineto{\pgfqpoint{2.116590in}{0.739656in}}%
\pgfpathlineto{\pgfqpoint{2.116026in}{0.739656in}}%
\pgfpathlineto{\pgfqpoint{2.115462in}{0.739656in}}%
\pgfpathlineto{\pgfqpoint{2.114897in}{0.739656in}}%
\pgfpathlineto{\pgfqpoint{2.114333in}{0.739656in}}%
\pgfpathlineto{\pgfqpoint{2.113769in}{0.739656in}}%
\pgfpathlineto{\pgfqpoint{2.113204in}{0.739656in}}%
\pgfpathlineto{\pgfqpoint{2.112640in}{0.739656in}}%
\pgfpathlineto{\pgfqpoint{2.112076in}{0.739656in}}%
\pgfpathlineto{\pgfqpoint{2.111511in}{0.739656in}}%
\pgfpathlineto{\pgfqpoint{2.110947in}{0.739656in}}%
\pgfpathlineto{\pgfqpoint{2.110383in}{0.739656in}}%
\pgfpathlineto{\pgfqpoint{2.109818in}{0.739656in}}%
\pgfpathlineto{\pgfqpoint{2.109254in}{0.739656in}}%
\pgfpathlineto{\pgfqpoint{2.108690in}{0.739656in}}%
\pgfpathlineto{\pgfqpoint{2.108125in}{0.739656in}}%
\pgfpathlineto{\pgfqpoint{2.107561in}{0.739656in}}%
\pgfpathlineto{\pgfqpoint{2.106996in}{0.739656in}}%
\pgfpathlineto{\pgfqpoint{2.106432in}{0.739656in}}%
\pgfpathlineto{\pgfqpoint{2.105868in}{0.739656in}}%
\pgfpathlineto{\pgfqpoint{2.105303in}{0.739656in}}%
\pgfpathlineto{\pgfqpoint{2.104739in}{0.739656in}}%
\pgfpathlineto{\pgfqpoint{2.104175in}{0.739656in}}%
\pgfpathlineto{\pgfqpoint{2.103610in}{0.739656in}}%
\pgfpathlineto{\pgfqpoint{2.103046in}{0.739656in}}%
\pgfpathlineto{\pgfqpoint{2.102482in}{0.739656in}}%
\pgfpathlineto{\pgfqpoint{2.101917in}{0.739656in}}%
\pgfpathlineto{\pgfqpoint{2.101353in}{0.739656in}}%
\pgfpathlineto{\pgfqpoint{2.100789in}{0.739656in}}%
\pgfpathlineto{\pgfqpoint{2.100224in}{0.739656in}}%
\pgfpathlineto{\pgfqpoint{2.099660in}{0.739656in}}%
\pgfpathlineto{\pgfqpoint{2.099096in}{0.739656in}}%
\pgfpathlineto{\pgfqpoint{2.098531in}{0.739656in}}%
\pgfpathlineto{\pgfqpoint{2.097967in}{0.739656in}}%
\pgfpathlineto{\pgfqpoint{2.097403in}{0.739656in}}%
\pgfpathlineto{\pgfqpoint{2.096838in}{0.739656in}}%
\pgfpathlineto{\pgfqpoint{2.096274in}{0.739656in}}%
\pgfpathlineto{\pgfqpoint{2.095710in}{0.739656in}}%
\pgfpathlineto{\pgfqpoint{2.095145in}{0.739656in}}%
\pgfpathlineto{\pgfqpoint{2.094581in}{0.739656in}}%
\pgfpathlineto{\pgfqpoint{2.094017in}{0.739656in}}%
\pgfpathlineto{\pgfqpoint{2.093452in}{0.739656in}}%
\pgfpathlineto{\pgfqpoint{2.092888in}{0.739656in}}%
\pgfpathlineto{\pgfqpoint{2.092323in}{0.739656in}}%
\pgfpathlineto{\pgfqpoint{2.091759in}{0.739656in}}%
\pgfpathlineto{\pgfqpoint{2.091195in}{0.739656in}}%
\pgfpathlineto{\pgfqpoint{2.090630in}{0.739656in}}%
\pgfpathlineto{\pgfqpoint{2.090066in}{0.739656in}}%
\pgfpathlineto{\pgfqpoint{2.089502in}{0.739656in}}%
\pgfpathlineto{\pgfqpoint{2.088937in}{0.739656in}}%
\pgfpathlineto{\pgfqpoint{2.088373in}{0.739656in}}%
\pgfpathlineto{\pgfqpoint{2.087809in}{0.739656in}}%
\pgfpathlineto{\pgfqpoint{2.087244in}{0.739656in}}%
\pgfpathlineto{\pgfqpoint{2.086680in}{0.739656in}}%
\pgfpathlineto{\pgfqpoint{2.086116in}{0.739656in}}%
\pgfpathlineto{\pgfqpoint{2.085551in}{0.739656in}}%
\pgfpathlineto{\pgfqpoint{2.084987in}{0.739656in}}%
\pgfpathlineto{\pgfqpoint{2.084423in}{0.739656in}}%
\pgfpathlineto{\pgfqpoint{2.083858in}{0.739656in}}%
\pgfpathlineto{\pgfqpoint{2.083294in}{0.739656in}}%
\pgfpathlineto{\pgfqpoint{2.082730in}{0.739656in}}%
\pgfpathlineto{\pgfqpoint{2.082165in}{0.739656in}}%
\pgfpathlineto{\pgfqpoint{2.081601in}{0.739656in}}%
\pgfpathlineto{\pgfqpoint{2.081037in}{0.739656in}}%
\pgfpathlineto{\pgfqpoint{2.080472in}{0.739656in}}%
\pgfpathlineto{\pgfqpoint{2.079908in}{0.739656in}}%
\pgfpathlineto{\pgfqpoint{2.079344in}{0.739656in}}%
\pgfpathlineto{\pgfqpoint{2.078779in}{0.739656in}}%
\pgfpathlineto{\pgfqpoint{2.078215in}{0.739656in}}%
\pgfpathlineto{\pgfqpoint{2.077651in}{0.739656in}}%
\pgfpathlineto{\pgfqpoint{2.077086in}{0.739656in}}%
\pgfpathlineto{\pgfqpoint{2.076522in}{0.739656in}}%
\pgfpathlineto{\pgfqpoint{2.075957in}{0.739656in}}%
\pgfpathlineto{\pgfqpoint{2.075393in}{0.739656in}}%
\pgfpathlineto{\pgfqpoint{2.074829in}{0.739656in}}%
\pgfpathlineto{\pgfqpoint{2.074264in}{0.739656in}}%
\pgfpathlineto{\pgfqpoint{2.073700in}{0.739656in}}%
\pgfpathlineto{\pgfqpoint{2.073136in}{0.739656in}}%
\pgfpathlineto{\pgfqpoint{2.072571in}{0.739656in}}%
\pgfpathlineto{\pgfqpoint{2.072007in}{0.739656in}}%
\pgfpathlineto{\pgfqpoint{2.071443in}{0.739656in}}%
\pgfpathlineto{\pgfqpoint{2.070878in}{0.739656in}}%
\pgfpathlineto{\pgfqpoint{2.070314in}{0.739656in}}%
\pgfpathlineto{\pgfqpoint{2.069750in}{0.739656in}}%
\pgfpathlineto{\pgfqpoint{2.069185in}{0.739656in}}%
\pgfpathlineto{\pgfqpoint{2.068621in}{0.739656in}}%
\pgfpathlineto{\pgfqpoint{2.068057in}{0.739656in}}%
\pgfpathlineto{\pgfqpoint{2.067492in}{0.739656in}}%
\pgfpathlineto{\pgfqpoint{2.066928in}{0.739656in}}%
\pgfpathlineto{\pgfqpoint{2.066364in}{0.739656in}}%
\pgfpathlineto{\pgfqpoint{2.065799in}{0.739656in}}%
\pgfpathlineto{\pgfqpoint{2.065235in}{0.739656in}}%
\pgfpathlineto{\pgfqpoint{2.064671in}{0.739656in}}%
\pgfpathlineto{\pgfqpoint{2.064106in}{0.739656in}}%
\pgfpathlineto{\pgfqpoint{2.063542in}{0.739656in}}%
\pgfpathlineto{\pgfqpoint{2.062978in}{0.739656in}}%
\pgfpathlineto{\pgfqpoint{2.062413in}{0.739656in}}%
\pgfpathlineto{\pgfqpoint{2.061849in}{0.739656in}}%
\pgfpathlineto{\pgfqpoint{2.061284in}{0.739656in}}%
\pgfpathlineto{\pgfqpoint{2.060720in}{0.739656in}}%
\pgfpathlineto{\pgfqpoint{2.060156in}{0.739656in}}%
\pgfpathlineto{\pgfqpoint{2.059591in}{0.739656in}}%
\pgfpathlineto{\pgfqpoint{2.059027in}{0.739656in}}%
\pgfpathlineto{\pgfqpoint{2.058463in}{0.739656in}}%
\pgfpathlineto{\pgfqpoint{2.057898in}{0.739656in}}%
\pgfpathlineto{\pgfqpoint{2.057334in}{0.739656in}}%
\pgfpathlineto{\pgfqpoint{2.056770in}{0.739656in}}%
\pgfpathlineto{\pgfqpoint{2.056205in}{0.739656in}}%
\pgfpathlineto{\pgfqpoint{2.055641in}{0.739656in}}%
\pgfpathlineto{\pgfqpoint{2.055077in}{0.739656in}}%
\pgfpathlineto{\pgfqpoint{2.054512in}{0.739656in}}%
\pgfpathlineto{\pgfqpoint{2.053948in}{0.739656in}}%
\pgfpathlineto{\pgfqpoint{2.053384in}{0.739656in}}%
\pgfpathlineto{\pgfqpoint{2.052819in}{0.739656in}}%
\pgfpathlineto{\pgfqpoint{2.052255in}{0.739656in}}%
\pgfpathlineto{\pgfqpoint{2.051691in}{0.739656in}}%
\pgfpathlineto{\pgfqpoint{2.051126in}{0.739656in}}%
\pgfpathlineto{\pgfqpoint{2.050562in}{0.739656in}}%
\pgfpathlineto{\pgfqpoint{2.049998in}{0.739656in}}%
\pgfpathlineto{\pgfqpoint{2.049433in}{0.739656in}}%
\pgfpathlineto{\pgfqpoint{2.048869in}{0.739656in}}%
\pgfpathlineto{\pgfqpoint{2.048305in}{0.739656in}}%
\pgfpathlineto{\pgfqpoint{2.047740in}{0.739656in}}%
\pgfpathlineto{\pgfqpoint{2.047176in}{0.739656in}}%
\pgfpathlineto{\pgfqpoint{2.046611in}{0.739656in}}%
\pgfpathlineto{\pgfqpoint{2.046047in}{0.739656in}}%
\pgfpathlineto{\pgfqpoint{2.045483in}{0.739656in}}%
\pgfpathlineto{\pgfqpoint{2.044918in}{0.739656in}}%
\pgfpathlineto{\pgfqpoint{2.044354in}{0.739656in}}%
\pgfpathlineto{\pgfqpoint{2.043790in}{0.739656in}}%
\pgfpathlineto{\pgfqpoint{2.043225in}{0.739656in}}%
\pgfpathlineto{\pgfqpoint{2.042661in}{0.739656in}}%
\pgfpathlineto{\pgfqpoint{2.042097in}{0.739656in}}%
\pgfpathlineto{\pgfqpoint{2.041532in}{0.739656in}}%
\pgfpathlineto{\pgfqpoint{2.040968in}{0.739656in}}%
\pgfpathlineto{\pgfqpoint{2.040404in}{0.739656in}}%
\pgfpathlineto{\pgfqpoint{2.039839in}{0.739656in}}%
\pgfpathlineto{\pgfqpoint{2.039275in}{0.739656in}}%
\pgfpathlineto{\pgfqpoint{2.038711in}{0.739656in}}%
\pgfpathlineto{\pgfqpoint{2.038146in}{0.739656in}}%
\pgfpathlineto{\pgfqpoint{2.037582in}{0.739656in}}%
\pgfpathlineto{\pgfqpoint{2.037018in}{0.739656in}}%
\pgfpathlineto{\pgfqpoint{2.036453in}{0.739656in}}%
\pgfpathlineto{\pgfqpoint{2.035889in}{0.739656in}}%
\pgfpathlineto{\pgfqpoint{2.035325in}{0.739656in}}%
\pgfpathlineto{\pgfqpoint{2.034760in}{0.739656in}}%
\pgfpathlineto{\pgfqpoint{2.034196in}{0.739656in}}%
\pgfpathlineto{\pgfqpoint{2.033632in}{0.739656in}}%
\pgfpathlineto{\pgfqpoint{2.033067in}{0.739656in}}%
\pgfpathlineto{\pgfqpoint{2.032503in}{0.739656in}}%
\pgfpathlineto{\pgfqpoint{2.031938in}{0.739656in}}%
\pgfpathlineto{\pgfqpoint{2.031374in}{0.739656in}}%
\pgfpathlineto{\pgfqpoint{2.030810in}{0.739656in}}%
\pgfpathlineto{\pgfqpoint{2.030245in}{0.739656in}}%
\pgfpathlineto{\pgfqpoint{2.029681in}{0.739656in}}%
\pgfpathlineto{\pgfqpoint{2.029117in}{0.739656in}}%
\pgfpathlineto{\pgfqpoint{2.028552in}{0.739656in}}%
\pgfpathlineto{\pgfqpoint{2.027988in}{0.739656in}}%
\pgfpathlineto{\pgfqpoint{2.027424in}{0.739656in}}%
\pgfpathlineto{\pgfqpoint{2.026859in}{0.739656in}}%
\pgfpathlineto{\pgfqpoint{2.026295in}{0.739656in}}%
\pgfpathlineto{\pgfqpoint{2.025731in}{0.739656in}}%
\pgfpathlineto{\pgfqpoint{2.025166in}{0.739656in}}%
\pgfpathlineto{\pgfqpoint{2.024602in}{0.739656in}}%
\pgfpathlineto{\pgfqpoint{2.024038in}{0.739656in}}%
\pgfpathlineto{\pgfqpoint{2.023473in}{0.739656in}}%
\pgfpathlineto{\pgfqpoint{2.022909in}{0.739656in}}%
\pgfpathlineto{\pgfqpoint{2.022345in}{0.739656in}}%
\pgfpathlineto{\pgfqpoint{2.021780in}{0.739656in}}%
\pgfpathlineto{\pgfqpoint{2.021216in}{0.739656in}}%
\pgfpathlineto{\pgfqpoint{2.020652in}{0.739656in}}%
\pgfpathlineto{\pgfqpoint{2.020087in}{0.739656in}}%
\pgfpathlineto{\pgfqpoint{2.019523in}{0.739656in}}%
\pgfpathlineto{\pgfqpoint{2.018959in}{0.739656in}}%
\pgfpathlineto{\pgfqpoint{2.018394in}{0.739656in}}%
\pgfpathlineto{\pgfqpoint{2.017830in}{0.739656in}}%
\pgfpathlineto{\pgfqpoint{2.017266in}{0.739656in}}%
\pgfpathlineto{\pgfqpoint{2.016701in}{0.739656in}}%
\pgfpathlineto{\pgfqpoint{2.016137in}{0.739656in}}%
\pgfpathlineto{\pgfqpoint{2.015572in}{0.739656in}}%
\pgfpathlineto{\pgfqpoint{2.015008in}{0.739656in}}%
\pgfpathlineto{\pgfqpoint{2.014444in}{0.739656in}}%
\pgfpathlineto{\pgfqpoint{2.013879in}{0.739656in}}%
\pgfpathlineto{\pgfqpoint{2.013315in}{0.739656in}}%
\pgfpathlineto{\pgfqpoint{2.012751in}{0.739656in}}%
\pgfpathlineto{\pgfqpoint{2.012186in}{0.739656in}}%
\pgfpathlineto{\pgfqpoint{2.011622in}{0.739656in}}%
\pgfpathlineto{\pgfqpoint{2.011058in}{0.739656in}}%
\pgfpathlineto{\pgfqpoint{2.010493in}{0.739656in}}%
\pgfpathlineto{\pgfqpoint{2.009929in}{0.739656in}}%
\pgfpathlineto{\pgfqpoint{2.009365in}{0.739656in}}%
\pgfpathlineto{\pgfqpoint{2.008800in}{0.739656in}}%
\pgfpathlineto{\pgfqpoint{2.008236in}{0.739656in}}%
\pgfpathlineto{\pgfqpoint{2.007672in}{0.739656in}}%
\pgfpathlineto{\pgfqpoint{2.007107in}{0.739656in}}%
\pgfpathlineto{\pgfqpoint{2.006543in}{0.739656in}}%
\pgfpathlineto{\pgfqpoint{2.005979in}{0.739656in}}%
\pgfpathlineto{\pgfqpoint{2.005414in}{0.739656in}}%
\pgfpathlineto{\pgfqpoint{2.004850in}{0.739656in}}%
\pgfpathlineto{\pgfqpoint{2.004286in}{0.739656in}}%
\pgfpathlineto{\pgfqpoint{2.003721in}{0.739656in}}%
\pgfpathlineto{\pgfqpoint{2.003157in}{0.739656in}}%
\pgfpathlineto{\pgfqpoint{2.002593in}{0.739656in}}%
\pgfpathlineto{\pgfqpoint{2.002028in}{0.739656in}}%
\pgfpathlineto{\pgfqpoint{2.001464in}{0.739656in}}%
\pgfpathlineto{\pgfqpoint{2.000899in}{0.739656in}}%
\pgfpathlineto{\pgfqpoint{2.000335in}{0.739656in}}%
\pgfpathlineto{\pgfqpoint{1.999771in}{0.739656in}}%
\pgfpathlineto{\pgfqpoint{1.999206in}{0.739656in}}%
\pgfpathlineto{\pgfqpoint{1.998642in}{0.739656in}}%
\pgfpathlineto{\pgfqpoint{1.998078in}{0.739656in}}%
\pgfpathlineto{\pgfqpoint{1.997513in}{0.739656in}}%
\pgfpathlineto{\pgfqpoint{1.996949in}{0.739656in}}%
\pgfpathlineto{\pgfqpoint{1.996385in}{0.739656in}}%
\pgfpathlineto{\pgfqpoint{1.995820in}{0.739656in}}%
\pgfpathlineto{\pgfqpoint{1.995256in}{0.739656in}}%
\pgfpathlineto{\pgfqpoint{1.994692in}{0.739656in}}%
\pgfpathlineto{\pgfqpoint{1.994127in}{0.739656in}}%
\pgfpathlineto{\pgfqpoint{1.993563in}{0.739656in}}%
\pgfpathlineto{\pgfqpoint{1.992999in}{0.739656in}}%
\pgfpathlineto{\pgfqpoint{1.992434in}{0.739656in}}%
\pgfpathlineto{\pgfqpoint{1.991870in}{0.739656in}}%
\pgfpathlineto{\pgfqpoint{1.991306in}{0.739656in}}%
\pgfpathlineto{\pgfqpoint{1.990741in}{0.739656in}}%
\pgfpathlineto{\pgfqpoint{1.990177in}{0.739656in}}%
\pgfpathlineto{\pgfqpoint{1.989613in}{0.739656in}}%
\pgfpathlineto{\pgfqpoint{1.989048in}{0.739656in}}%
\pgfpathlineto{\pgfqpoint{1.988484in}{0.739656in}}%
\pgfpathlineto{\pgfqpoint{1.987920in}{0.739656in}}%
\pgfpathlineto{\pgfqpoint{1.987355in}{0.739656in}}%
\pgfpathlineto{\pgfqpoint{1.986791in}{0.739656in}}%
\pgfpathlineto{\pgfqpoint{1.986226in}{0.739656in}}%
\pgfpathlineto{\pgfqpoint{1.985662in}{0.739656in}}%
\pgfpathlineto{\pgfqpoint{1.985098in}{0.739656in}}%
\pgfpathlineto{\pgfqpoint{1.984533in}{0.739656in}}%
\pgfpathlineto{\pgfqpoint{1.983969in}{0.739656in}}%
\pgfpathlineto{\pgfqpoint{1.983405in}{0.739656in}}%
\pgfpathlineto{\pgfqpoint{1.982840in}{0.739656in}}%
\pgfpathlineto{\pgfqpoint{1.982276in}{0.739656in}}%
\pgfpathlineto{\pgfqpoint{1.981712in}{0.739656in}}%
\pgfpathlineto{\pgfqpoint{1.981147in}{0.739656in}}%
\pgfpathlineto{\pgfqpoint{1.980583in}{0.739656in}}%
\pgfpathlineto{\pgfqpoint{1.980019in}{0.739656in}}%
\pgfpathlineto{\pgfqpoint{1.979454in}{0.739656in}}%
\pgfpathlineto{\pgfqpoint{1.978890in}{0.739656in}}%
\pgfpathlineto{\pgfqpoint{1.978326in}{0.739656in}}%
\pgfpathlineto{\pgfqpoint{1.977761in}{0.739656in}}%
\pgfpathlineto{\pgfqpoint{1.977197in}{0.739656in}}%
\pgfpathlineto{\pgfqpoint{1.976633in}{0.739656in}}%
\pgfpathlineto{\pgfqpoint{1.976068in}{0.739656in}}%
\pgfpathlineto{\pgfqpoint{1.975504in}{0.739656in}}%
\pgfpathlineto{\pgfqpoint{1.974940in}{0.739656in}}%
\pgfpathlineto{\pgfqpoint{1.974375in}{0.739656in}}%
\pgfpathlineto{\pgfqpoint{1.973811in}{0.739656in}}%
\pgfpathlineto{\pgfqpoint{1.973247in}{0.739656in}}%
\pgfpathlineto{\pgfqpoint{1.972682in}{0.739656in}}%
\pgfpathlineto{\pgfqpoint{1.972118in}{0.739656in}}%
\pgfpathlineto{\pgfqpoint{1.971554in}{0.739656in}}%
\pgfpathlineto{\pgfqpoint{1.970989in}{0.739656in}}%
\pgfpathlineto{\pgfqpoint{1.970425in}{0.739656in}}%
\pgfpathlineto{\pgfqpoint{1.969860in}{0.739656in}}%
\pgfpathlineto{\pgfqpoint{1.969296in}{0.739656in}}%
\pgfpathlineto{\pgfqpoint{1.968732in}{0.739656in}}%
\pgfpathlineto{\pgfqpoint{1.968167in}{0.739656in}}%
\pgfpathlineto{\pgfqpoint{1.967603in}{0.739656in}}%
\pgfpathlineto{\pgfqpoint{1.967039in}{0.739656in}}%
\pgfpathlineto{\pgfqpoint{1.966474in}{0.739656in}}%
\pgfpathlineto{\pgfqpoint{1.965910in}{0.739656in}}%
\pgfpathlineto{\pgfqpoint{1.965346in}{0.739656in}}%
\pgfpathlineto{\pgfqpoint{1.964781in}{0.739656in}}%
\pgfpathlineto{\pgfqpoint{1.964217in}{0.739656in}}%
\pgfpathlineto{\pgfqpoint{1.963653in}{0.739656in}}%
\pgfpathlineto{\pgfqpoint{1.963088in}{0.739656in}}%
\pgfpathlineto{\pgfqpoint{1.962524in}{0.739656in}}%
\pgfpathlineto{\pgfqpoint{1.961960in}{0.739656in}}%
\pgfpathlineto{\pgfqpoint{1.961395in}{0.739656in}}%
\pgfpathlineto{\pgfqpoint{1.960831in}{0.739656in}}%
\pgfpathlineto{\pgfqpoint{1.960267in}{0.739656in}}%
\pgfpathlineto{\pgfqpoint{1.959702in}{0.739656in}}%
\pgfpathlineto{\pgfqpoint{1.959138in}{0.739656in}}%
\pgfpathlineto{\pgfqpoint{1.958574in}{0.739656in}}%
\pgfpathlineto{\pgfqpoint{1.958009in}{0.739656in}}%
\pgfpathlineto{\pgfqpoint{1.957445in}{0.739656in}}%
\pgfpathlineto{\pgfqpoint{1.956881in}{0.739656in}}%
\pgfpathlineto{\pgfqpoint{1.956316in}{0.739656in}}%
\pgfpathlineto{\pgfqpoint{1.955752in}{0.739656in}}%
\pgfpathlineto{\pgfqpoint{1.955187in}{0.739656in}}%
\pgfpathlineto{\pgfqpoint{1.954623in}{0.739656in}}%
\pgfpathlineto{\pgfqpoint{1.954059in}{0.739656in}}%
\pgfpathlineto{\pgfqpoint{1.953494in}{0.739656in}}%
\pgfpathlineto{\pgfqpoint{1.952930in}{0.739656in}}%
\pgfpathlineto{\pgfqpoint{1.952366in}{0.739656in}}%
\pgfpathlineto{\pgfqpoint{1.951801in}{0.739656in}}%
\pgfpathlineto{\pgfqpoint{1.951237in}{0.739656in}}%
\pgfpathlineto{\pgfqpoint{1.950673in}{0.739656in}}%
\pgfpathlineto{\pgfqpoint{1.950108in}{0.739656in}}%
\pgfpathlineto{\pgfqpoint{1.949544in}{0.739656in}}%
\pgfpathlineto{\pgfqpoint{1.948980in}{0.739656in}}%
\pgfpathlineto{\pgfqpoint{1.948415in}{0.739656in}}%
\pgfpathlineto{\pgfqpoint{1.947851in}{0.739656in}}%
\pgfpathlineto{\pgfqpoint{1.947287in}{0.739656in}}%
\pgfpathlineto{\pgfqpoint{1.946722in}{0.739656in}}%
\pgfpathlineto{\pgfqpoint{1.946158in}{0.739656in}}%
\pgfpathlineto{\pgfqpoint{1.945594in}{0.739656in}}%
\pgfpathlineto{\pgfqpoint{1.945029in}{0.739656in}}%
\pgfpathlineto{\pgfqpoint{1.944465in}{0.739656in}}%
\pgfpathlineto{\pgfqpoint{1.943901in}{0.739656in}}%
\pgfpathlineto{\pgfqpoint{1.943336in}{0.739656in}}%
\pgfpathlineto{\pgfqpoint{1.942772in}{0.739656in}}%
\pgfpathlineto{\pgfqpoint{1.942208in}{0.739656in}}%
\pgfpathlineto{\pgfqpoint{1.941643in}{0.739656in}}%
\pgfpathlineto{\pgfqpoint{1.941079in}{0.739656in}}%
\pgfpathlineto{\pgfqpoint{1.940514in}{0.739656in}}%
\pgfpathlineto{\pgfqpoint{1.939950in}{0.739656in}}%
\pgfpathlineto{\pgfqpoint{1.939386in}{0.739656in}}%
\pgfpathlineto{\pgfqpoint{1.938821in}{0.739656in}}%
\pgfpathlineto{\pgfqpoint{1.938257in}{0.739656in}}%
\pgfpathlineto{\pgfqpoint{1.937693in}{0.739656in}}%
\pgfpathlineto{\pgfqpoint{1.937128in}{0.739656in}}%
\pgfpathlineto{\pgfqpoint{1.936564in}{0.739656in}}%
\pgfpathlineto{\pgfqpoint{1.936000in}{0.739656in}}%
\pgfpathlineto{\pgfqpoint{1.935435in}{0.739656in}}%
\pgfpathlineto{\pgfqpoint{1.934871in}{0.739656in}}%
\pgfpathlineto{\pgfqpoint{1.934307in}{0.739656in}}%
\pgfpathlineto{\pgfqpoint{1.933742in}{0.739656in}}%
\pgfpathlineto{\pgfqpoint{1.933178in}{0.739656in}}%
\pgfpathlineto{\pgfqpoint{1.932614in}{0.739656in}}%
\pgfpathlineto{\pgfqpoint{1.932049in}{0.739656in}}%
\pgfpathlineto{\pgfqpoint{1.931485in}{0.739656in}}%
\pgfpathlineto{\pgfqpoint{1.930921in}{0.739656in}}%
\pgfpathlineto{\pgfqpoint{1.930356in}{0.739656in}}%
\pgfpathlineto{\pgfqpoint{1.929792in}{0.739656in}}%
\pgfpathlineto{\pgfqpoint{1.929228in}{0.739656in}}%
\pgfpathlineto{\pgfqpoint{1.928663in}{0.739656in}}%
\pgfpathlineto{\pgfqpoint{1.928099in}{0.739656in}}%
\pgfpathlineto{\pgfqpoint{1.927535in}{0.739656in}}%
\pgfpathlineto{\pgfqpoint{1.926970in}{0.739656in}}%
\pgfpathlineto{\pgfqpoint{1.926406in}{0.739656in}}%
\pgfpathlineto{\pgfqpoint{1.925842in}{0.739656in}}%
\pgfpathlineto{\pgfqpoint{1.925277in}{0.739656in}}%
\pgfpathlineto{\pgfqpoint{1.924713in}{0.739656in}}%
\pgfpathlineto{\pgfqpoint{1.924148in}{0.739656in}}%
\pgfpathlineto{\pgfqpoint{1.923584in}{0.739656in}}%
\pgfpathlineto{\pgfqpoint{1.923020in}{0.739656in}}%
\pgfpathlineto{\pgfqpoint{1.922455in}{0.739656in}}%
\pgfpathlineto{\pgfqpoint{1.921891in}{0.739656in}}%
\pgfpathlineto{\pgfqpoint{1.921327in}{0.739656in}}%
\pgfpathlineto{\pgfqpoint{1.920762in}{0.739656in}}%
\pgfpathlineto{\pgfqpoint{1.920198in}{0.739656in}}%
\pgfpathlineto{\pgfqpoint{1.919634in}{0.739656in}}%
\pgfpathlineto{\pgfqpoint{1.919069in}{0.739656in}}%
\pgfpathlineto{\pgfqpoint{1.918505in}{0.739656in}}%
\pgfpathlineto{\pgfqpoint{1.917941in}{0.739656in}}%
\pgfpathlineto{\pgfqpoint{1.917376in}{0.739656in}}%
\pgfpathlineto{\pgfqpoint{1.916812in}{0.739656in}}%
\pgfpathlineto{\pgfqpoint{1.916248in}{0.739656in}}%
\pgfpathlineto{\pgfqpoint{1.915683in}{0.739656in}}%
\pgfpathlineto{\pgfqpoint{1.915119in}{0.739656in}}%
\pgfpathlineto{\pgfqpoint{1.914555in}{0.739656in}}%
\pgfpathlineto{\pgfqpoint{1.913990in}{0.739656in}}%
\pgfpathlineto{\pgfqpoint{1.913426in}{0.739656in}}%
\pgfpathlineto{\pgfqpoint{1.912862in}{0.739656in}}%
\pgfpathlineto{\pgfqpoint{1.912297in}{0.739656in}}%
\pgfpathlineto{\pgfqpoint{1.911733in}{0.739656in}}%
\pgfpathlineto{\pgfqpoint{1.911169in}{0.739656in}}%
\pgfpathlineto{\pgfqpoint{1.910604in}{0.739656in}}%
\pgfpathlineto{\pgfqpoint{1.910040in}{0.739656in}}%
\pgfpathlineto{\pgfqpoint{1.909475in}{0.739656in}}%
\pgfpathlineto{\pgfqpoint{1.908911in}{0.739656in}}%
\pgfpathlineto{\pgfqpoint{1.908347in}{0.739656in}}%
\pgfpathlineto{\pgfqpoint{1.907782in}{0.739656in}}%
\pgfpathlineto{\pgfqpoint{1.907218in}{0.739656in}}%
\pgfpathlineto{\pgfqpoint{1.906654in}{0.739656in}}%
\pgfpathlineto{\pgfqpoint{1.906089in}{0.739656in}}%
\pgfpathlineto{\pgfqpoint{1.905525in}{0.739656in}}%
\pgfpathlineto{\pgfqpoint{1.904961in}{0.739656in}}%
\pgfpathlineto{\pgfqpoint{1.904396in}{0.739656in}}%
\pgfpathlineto{\pgfqpoint{1.903832in}{0.739656in}}%
\pgfpathlineto{\pgfqpoint{1.903268in}{0.739656in}}%
\pgfpathlineto{\pgfqpoint{1.902703in}{0.739656in}}%
\pgfpathlineto{\pgfqpoint{1.902139in}{0.739656in}}%
\pgfpathlineto{\pgfqpoint{1.901575in}{0.739656in}}%
\pgfpathlineto{\pgfqpoint{1.901010in}{0.739656in}}%
\pgfpathlineto{\pgfqpoint{1.900446in}{0.739656in}}%
\pgfpathlineto{\pgfqpoint{1.899882in}{0.739656in}}%
\pgfpathlineto{\pgfqpoint{1.899317in}{0.739656in}}%
\pgfpathlineto{\pgfqpoint{1.898753in}{0.739656in}}%
\pgfpathlineto{\pgfqpoint{1.898189in}{0.739656in}}%
\pgfpathlineto{\pgfqpoint{1.897624in}{0.739656in}}%
\pgfpathlineto{\pgfqpoint{1.897060in}{0.739656in}}%
\pgfpathlineto{\pgfqpoint{1.896496in}{0.739656in}}%
\pgfpathlineto{\pgfqpoint{1.895931in}{0.739656in}}%
\pgfpathlineto{\pgfqpoint{1.895367in}{0.739656in}}%
\pgfpathlineto{\pgfqpoint{1.894802in}{0.739656in}}%
\pgfpathlineto{\pgfqpoint{1.894238in}{0.739656in}}%
\pgfpathlineto{\pgfqpoint{1.893674in}{0.739656in}}%
\pgfpathlineto{\pgfqpoint{1.893109in}{0.739656in}}%
\pgfpathlineto{\pgfqpoint{1.892545in}{0.739656in}}%
\pgfpathlineto{\pgfqpoint{1.891981in}{0.739656in}}%
\pgfpathlineto{\pgfqpoint{1.891416in}{0.739656in}}%
\pgfpathlineto{\pgfqpoint{1.890852in}{0.739656in}}%
\pgfpathlineto{\pgfqpoint{1.890288in}{0.739656in}}%
\pgfpathlineto{\pgfqpoint{1.889723in}{0.739656in}}%
\pgfpathlineto{\pgfqpoint{1.889159in}{0.739656in}}%
\pgfpathlineto{\pgfqpoint{1.888595in}{0.739656in}}%
\pgfpathlineto{\pgfqpoint{1.888030in}{0.739656in}}%
\pgfpathlineto{\pgfqpoint{1.887466in}{0.739656in}}%
\pgfpathlineto{\pgfqpoint{1.886902in}{0.739656in}}%
\pgfpathlineto{\pgfqpoint{1.886337in}{0.739656in}}%
\pgfpathlineto{\pgfqpoint{1.885773in}{0.739656in}}%
\pgfpathlineto{\pgfqpoint{1.885209in}{0.739656in}}%
\pgfpathlineto{\pgfqpoint{1.884644in}{0.739656in}}%
\pgfpathlineto{\pgfqpoint{1.884080in}{0.739656in}}%
\pgfpathlineto{\pgfqpoint{1.883516in}{0.739656in}}%
\pgfpathlineto{\pgfqpoint{1.882951in}{0.739656in}}%
\pgfpathlineto{\pgfqpoint{1.882387in}{0.739656in}}%
\pgfpathlineto{\pgfqpoint{1.881823in}{0.739656in}}%
\pgfpathlineto{\pgfqpoint{1.881258in}{0.739656in}}%
\pgfpathlineto{\pgfqpoint{1.880694in}{0.739656in}}%
\pgfpathlineto{\pgfqpoint{1.880130in}{0.739656in}}%
\pgfpathlineto{\pgfqpoint{1.879565in}{0.739656in}}%
\pgfpathlineto{\pgfqpoint{1.879001in}{0.739656in}}%
\pgfpathlineto{\pgfqpoint{1.878436in}{0.739656in}}%
\pgfpathlineto{\pgfqpoint{1.877872in}{0.739656in}}%
\pgfpathlineto{\pgfqpoint{1.877308in}{0.739656in}}%
\pgfpathlineto{\pgfqpoint{1.876743in}{0.739656in}}%
\pgfpathlineto{\pgfqpoint{1.876179in}{0.739656in}}%
\pgfpathlineto{\pgfqpoint{1.875615in}{0.739656in}}%
\pgfpathlineto{\pgfqpoint{1.875050in}{0.739656in}}%
\pgfpathlineto{\pgfqpoint{1.874486in}{0.739656in}}%
\pgfpathlineto{\pgfqpoint{1.873922in}{0.739656in}}%
\pgfpathlineto{\pgfqpoint{1.873357in}{0.739656in}}%
\pgfpathlineto{\pgfqpoint{1.872793in}{0.739656in}}%
\pgfpathlineto{\pgfqpoint{1.872229in}{0.739656in}}%
\pgfpathlineto{\pgfqpoint{1.871664in}{0.739656in}}%
\pgfpathlineto{\pgfqpoint{1.871100in}{0.739656in}}%
\pgfpathlineto{\pgfqpoint{1.870536in}{0.739656in}}%
\pgfpathlineto{\pgfqpoint{1.869971in}{0.739656in}}%
\pgfpathlineto{\pgfqpoint{1.869407in}{0.739656in}}%
\pgfpathlineto{\pgfqpoint{1.868843in}{0.739656in}}%
\pgfpathlineto{\pgfqpoint{1.868278in}{0.739656in}}%
\pgfpathlineto{\pgfqpoint{1.867714in}{0.739656in}}%
\pgfpathlineto{\pgfqpoint{1.867150in}{0.739656in}}%
\pgfpathlineto{\pgfqpoint{1.866585in}{0.739656in}}%
\pgfpathlineto{\pgfqpoint{1.866021in}{0.739656in}}%
\pgfpathlineto{\pgfqpoint{1.865457in}{0.739656in}}%
\pgfpathlineto{\pgfqpoint{1.864892in}{0.739656in}}%
\pgfpathlineto{\pgfqpoint{1.864328in}{0.739656in}}%
\pgfpathlineto{\pgfqpoint{1.863763in}{0.739656in}}%
\pgfpathlineto{\pgfqpoint{1.863199in}{0.739656in}}%
\pgfpathlineto{\pgfqpoint{1.862635in}{0.739656in}}%
\pgfpathlineto{\pgfqpoint{1.862070in}{0.739656in}}%
\pgfpathlineto{\pgfqpoint{1.861506in}{0.739656in}}%
\pgfpathlineto{\pgfqpoint{1.860942in}{0.739656in}}%
\pgfpathlineto{\pgfqpoint{1.860377in}{0.739656in}}%
\pgfpathlineto{\pgfqpoint{1.859813in}{0.739656in}}%
\pgfpathlineto{\pgfqpoint{1.859249in}{0.739656in}}%
\pgfpathlineto{\pgfqpoint{1.858684in}{0.739656in}}%
\pgfpathlineto{\pgfqpoint{1.858120in}{0.739656in}}%
\pgfpathlineto{\pgfqpoint{1.857556in}{0.739656in}}%
\pgfpathlineto{\pgfqpoint{1.856991in}{0.739656in}}%
\pgfpathlineto{\pgfqpoint{1.856427in}{0.739656in}}%
\pgfpathlineto{\pgfqpoint{1.855863in}{0.739656in}}%
\pgfpathlineto{\pgfqpoint{1.855298in}{0.739656in}}%
\pgfpathlineto{\pgfqpoint{1.854734in}{0.739656in}}%
\pgfpathlineto{\pgfqpoint{1.854170in}{0.739656in}}%
\pgfpathlineto{\pgfqpoint{1.853605in}{0.739656in}}%
\pgfpathlineto{\pgfqpoint{1.853041in}{0.739656in}}%
\pgfpathlineto{\pgfqpoint{1.852477in}{0.739656in}}%
\pgfpathlineto{\pgfqpoint{1.851912in}{0.739656in}}%
\pgfpathlineto{\pgfqpoint{1.851348in}{0.739656in}}%
\pgfpathlineto{\pgfqpoint{1.850784in}{0.739656in}}%
\pgfpathlineto{\pgfqpoint{1.850219in}{0.739656in}}%
\pgfpathlineto{\pgfqpoint{1.849655in}{0.739656in}}%
\pgfpathlineto{\pgfqpoint{1.849090in}{0.739656in}}%
\pgfpathlineto{\pgfqpoint{1.848526in}{0.739656in}}%
\pgfpathlineto{\pgfqpoint{1.847962in}{0.739656in}}%
\pgfpathlineto{\pgfqpoint{1.847397in}{0.739656in}}%
\pgfpathlineto{\pgfqpoint{1.846833in}{0.739656in}}%
\pgfpathlineto{\pgfqpoint{1.846269in}{0.739656in}}%
\pgfpathlineto{\pgfqpoint{1.845704in}{0.739656in}}%
\pgfpathlineto{\pgfqpoint{1.845140in}{0.739656in}}%
\pgfpathlineto{\pgfqpoint{1.844576in}{0.739656in}}%
\pgfpathlineto{\pgfqpoint{1.844011in}{0.739656in}}%
\pgfpathlineto{\pgfqpoint{1.843447in}{0.739656in}}%
\pgfpathlineto{\pgfqpoint{1.842883in}{0.739656in}}%
\pgfpathlineto{\pgfqpoint{1.842318in}{0.739656in}}%
\pgfpathlineto{\pgfqpoint{1.841754in}{0.739656in}}%
\pgfpathlineto{\pgfqpoint{1.841190in}{0.739656in}}%
\pgfpathlineto{\pgfqpoint{1.840625in}{0.739656in}}%
\pgfpathlineto{\pgfqpoint{1.840061in}{0.739656in}}%
\pgfpathlineto{\pgfqpoint{1.839497in}{0.739656in}}%
\pgfpathlineto{\pgfqpoint{1.838932in}{0.739656in}}%
\pgfpathlineto{\pgfqpoint{1.838368in}{0.739656in}}%
\pgfpathlineto{\pgfqpoint{1.837804in}{0.739656in}}%
\pgfpathlineto{\pgfqpoint{1.837239in}{0.739656in}}%
\pgfpathlineto{\pgfqpoint{1.836675in}{0.739656in}}%
\pgfpathlineto{\pgfqpoint{1.836111in}{0.739656in}}%
\pgfpathlineto{\pgfqpoint{1.835546in}{0.739656in}}%
\pgfpathlineto{\pgfqpoint{1.834982in}{0.739656in}}%
\pgfpathlineto{\pgfqpoint{1.834418in}{0.739656in}}%
\pgfpathlineto{\pgfqpoint{1.833853in}{0.739656in}}%
\pgfpathlineto{\pgfqpoint{1.833289in}{0.739656in}}%
\pgfpathlineto{\pgfqpoint{1.832724in}{0.739656in}}%
\pgfpathlineto{\pgfqpoint{1.832160in}{0.739656in}}%
\pgfpathlineto{\pgfqpoint{1.831596in}{0.739656in}}%
\pgfpathlineto{\pgfqpoint{1.831031in}{0.739656in}}%
\pgfpathlineto{\pgfqpoint{1.830467in}{0.739656in}}%
\pgfpathlineto{\pgfqpoint{1.829903in}{0.739656in}}%
\pgfpathlineto{\pgfqpoint{1.829338in}{0.739656in}}%
\pgfpathlineto{\pgfqpoint{1.828774in}{0.739656in}}%
\pgfpathlineto{\pgfqpoint{1.828210in}{0.739656in}}%
\pgfpathlineto{\pgfqpoint{1.827645in}{0.739656in}}%
\pgfpathlineto{\pgfqpoint{1.827081in}{0.739656in}}%
\pgfpathlineto{\pgfqpoint{1.826517in}{0.739656in}}%
\pgfpathlineto{\pgfqpoint{1.825952in}{0.739656in}}%
\pgfpathlineto{\pgfqpoint{1.825388in}{0.739656in}}%
\pgfpathlineto{\pgfqpoint{1.824824in}{0.739656in}}%
\pgfpathlineto{\pgfqpoint{1.824259in}{0.739656in}}%
\pgfpathlineto{\pgfqpoint{1.823695in}{0.739656in}}%
\pgfpathlineto{\pgfqpoint{1.823131in}{0.739656in}}%
\pgfpathlineto{\pgfqpoint{1.822566in}{0.739656in}}%
\pgfpathlineto{\pgfqpoint{1.822002in}{0.739656in}}%
\pgfpathlineto{\pgfqpoint{1.821438in}{0.739656in}}%
\pgfpathlineto{\pgfqpoint{1.820873in}{0.739656in}}%
\pgfpathlineto{\pgfqpoint{1.820309in}{0.739656in}}%
\pgfpathlineto{\pgfqpoint{1.819745in}{0.739656in}}%
\pgfpathlineto{\pgfqpoint{1.819180in}{0.739656in}}%
\pgfpathlineto{\pgfqpoint{1.818616in}{0.739656in}}%
\pgfpathlineto{\pgfqpoint{1.818051in}{0.739656in}}%
\pgfpathlineto{\pgfqpoint{1.817487in}{0.739656in}}%
\pgfpathlineto{\pgfqpoint{1.816923in}{0.739656in}}%
\pgfpathlineto{\pgfqpoint{1.816358in}{0.739656in}}%
\pgfpathlineto{\pgfqpoint{1.815794in}{0.739656in}}%
\pgfpathlineto{\pgfqpoint{1.815230in}{0.739656in}}%
\pgfpathlineto{\pgfqpoint{1.814665in}{0.739656in}}%
\pgfpathlineto{\pgfqpoint{1.814101in}{0.739656in}}%
\pgfpathlineto{\pgfqpoint{1.813537in}{0.739656in}}%
\pgfpathlineto{\pgfqpoint{1.812972in}{0.739656in}}%
\pgfpathlineto{\pgfqpoint{1.812408in}{0.739656in}}%
\pgfpathlineto{\pgfqpoint{1.811844in}{0.739656in}}%
\pgfpathlineto{\pgfqpoint{1.811279in}{0.739656in}}%
\pgfpathlineto{\pgfqpoint{1.810715in}{0.739656in}}%
\pgfpathlineto{\pgfqpoint{1.810151in}{0.739656in}}%
\pgfpathlineto{\pgfqpoint{1.809586in}{0.739656in}}%
\pgfpathlineto{\pgfqpoint{1.809022in}{0.739656in}}%
\pgfpathlineto{\pgfqpoint{1.808458in}{0.739656in}}%
\pgfpathlineto{\pgfqpoint{1.807893in}{0.739656in}}%
\pgfpathlineto{\pgfqpoint{1.807329in}{0.739656in}}%
\pgfpathlineto{\pgfqpoint{1.806765in}{0.739656in}}%
\pgfpathlineto{\pgfqpoint{1.806200in}{0.739656in}}%
\pgfpathlineto{\pgfqpoint{1.805636in}{0.739656in}}%
\pgfpathlineto{\pgfqpoint{1.805072in}{0.739656in}}%
\pgfpathlineto{\pgfqpoint{1.804507in}{0.739656in}}%
\pgfpathlineto{\pgfqpoint{1.803943in}{0.739656in}}%
\pgfpathlineto{\pgfqpoint{1.803378in}{0.739656in}}%
\pgfpathlineto{\pgfqpoint{1.802814in}{0.739656in}}%
\pgfpathlineto{\pgfqpoint{1.802250in}{0.739656in}}%
\pgfpathlineto{\pgfqpoint{1.801685in}{0.739656in}}%
\pgfpathlineto{\pgfqpoint{1.801121in}{0.739656in}}%
\pgfpathlineto{\pgfqpoint{1.800557in}{0.739656in}}%
\pgfpathlineto{\pgfqpoint{1.799992in}{0.739656in}}%
\pgfpathlineto{\pgfqpoint{1.799428in}{0.739656in}}%
\pgfpathlineto{\pgfqpoint{1.798864in}{0.739656in}}%
\pgfpathlineto{\pgfqpoint{1.798299in}{0.739656in}}%
\pgfpathlineto{\pgfqpoint{1.797735in}{0.739656in}}%
\pgfpathlineto{\pgfqpoint{1.797171in}{0.739656in}}%
\pgfpathlineto{\pgfqpoint{1.796606in}{0.739656in}}%
\pgfpathlineto{\pgfqpoint{1.796042in}{0.739656in}}%
\pgfpathlineto{\pgfqpoint{1.795478in}{0.739656in}}%
\pgfpathlineto{\pgfqpoint{1.794913in}{0.739656in}}%
\pgfpathlineto{\pgfqpoint{1.794349in}{0.739656in}}%
\pgfpathlineto{\pgfqpoint{1.793785in}{0.739656in}}%
\pgfpathlineto{\pgfqpoint{1.793220in}{0.739656in}}%
\pgfpathlineto{\pgfqpoint{1.792656in}{0.739656in}}%
\pgfpathlineto{\pgfqpoint{1.792092in}{0.739656in}}%
\pgfpathlineto{\pgfqpoint{1.791527in}{0.739656in}}%
\pgfpathlineto{\pgfqpoint{1.790963in}{0.739656in}}%
\pgfpathlineto{\pgfqpoint{1.790399in}{0.739656in}}%
\pgfpathlineto{\pgfqpoint{1.789834in}{0.739656in}}%
\pgfpathlineto{\pgfqpoint{1.789270in}{0.739656in}}%
\pgfpathlineto{\pgfqpoint{1.788705in}{0.739656in}}%
\pgfpathlineto{\pgfqpoint{1.788141in}{0.739656in}}%
\pgfpathlineto{\pgfqpoint{1.787577in}{0.739656in}}%
\pgfpathlineto{\pgfqpoint{1.787012in}{0.739656in}}%
\pgfpathlineto{\pgfqpoint{1.786448in}{0.739656in}}%
\pgfpathlineto{\pgfqpoint{1.785884in}{0.739656in}}%
\pgfpathlineto{\pgfqpoint{1.785319in}{0.739656in}}%
\pgfpathlineto{\pgfqpoint{1.784755in}{0.739656in}}%
\pgfpathlineto{\pgfqpoint{1.784191in}{0.739656in}}%
\pgfpathlineto{\pgfqpoint{1.783626in}{0.739656in}}%
\pgfpathlineto{\pgfqpoint{1.783062in}{0.739656in}}%
\pgfpathlineto{\pgfqpoint{1.782498in}{0.739656in}}%
\pgfpathlineto{\pgfqpoint{1.781933in}{0.739656in}}%
\pgfpathlineto{\pgfqpoint{1.781369in}{0.739656in}}%
\pgfpathlineto{\pgfqpoint{1.780805in}{0.739656in}}%
\pgfpathlineto{\pgfqpoint{1.780240in}{0.739656in}}%
\pgfpathlineto{\pgfqpoint{1.779676in}{0.739656in}}%
\pgfpathlineto{\pgfqpoint{1.779112in}{0.739656in}}%
\pgfpathlineto{\pgfqpoint{1.778547in}{0.739656in}}%
\pgfpathlineto{\pgfqpoint{1.777983in}{0.739656in}}%
\pgfpathlineto{\pgfqpoint{1.777419in}{0.739656in}}%
\pgfpathlineto{\pgfqpoint{1.776854in}{0.739656in}}%
\pgfpathlineto{\pgfqpoint{1.776290in}{0.739656in}}%
\pgfpathlineto{\pgfqpoint{1.775726in}{0.739656in}}%
\pgfpathlineto{\pgfqpoint{1.775161in}{0.739656in}}%
\pgfpathlineto{\pgfqpoint{1.774597in}{0.739656in}}%
\pgfpathlineto{\pgfqpoint{1.774033in}{0.739656in}}%
\pgfpathlineto{\pgfqpoint{1.773468in}{0.739656in}}%
\pgfpathlineto{\pgfqpoint{1.772904in}{0.739656in}}%
\pgfpathlineto{\pgfqpoint{1.772339in}{0.739656in}}%
\pgfpathlineto{\pgfqpoint{1.771775in}{0.739656in}}%
\pgfpathlineto{\pgfqpoint{1.771211in}{0.739656in}}%
\pgfpathlineto{\pgfqpoint{1.770646in}{0.739656in}}%
\pgfpathlineto{\pgfqpoint{1.770082in}{0.739656in}}%
\pgfpathlineto{\pgfqpoint{1.769518in}{0.739656in}}%
\pgfpathlineto{\pgfqpoint{1.768953in}{0.739656in}}%
\pgfpathlineto{\pgfqpoint{1.768389in}{0.739656in}}%
\pgfpathlineto{\pgfqpoint{1.767825in}{0.739656in}}%
\pgfpathlineto{\pgfqpoint{1.767260in}{0.739656in}}%
\pgfpathlineto{\pgfqpoint{1.766696in}{0.739656in}}%
\pgfpathlineto{\pgfqpoint{1.766132in}{0.739656in}}%
\pgfpathlineto{\pgfqpoint{1.765567in}{0.739656in}}%
\pgfpathlineto{\pgfqpoint{1.765003in}{0.739656in}}%
\pgfpathlineto{\pgfqpoint{1.764439in}{0.739656in}}%
\pgfpathlineto{\pgfqpoint{1.763874in}{0.739656in}}%
\pgfpathlineto{\pgfqpoint{1.763310in}{0.739656in}}%
\pgfpathlineto{\pgfqpoint{1.762746in}{0.739656in}}%
\pgfpathlineto{\pgfqpoint{1.762181in}{0.739656in}}%
\pgfpathlineto{\pgfqpoint{1.761617in}{0.739656in}}%
\pgfpathlineto{\pgfqpoint{1.761053in}{0.739656in}}%
\pgfpathlineto{\pgfqpoint{1.760488in}{0.739656in}}%
\pgfpathlineto{\pgfqpoint{1.759924in}{0.739656in}}%
\pgfpathlineto{\pgfqpoint{1.759360in}{0.739656in}}%
\pgfpathlineto{\pgfqpoint{1.758795in}{0.739656in}}%
\pgfpathlineto{\pgfqpoint{1.758231in}{0.739656in}}%
\pgfpathlineto{\pgfqpoint{1.757666in}{0.739656in}}%
\pgfpathlineto{\pgfqpoint{1.757102in}{0.739656in}}%
\pgfpathlineto{\pgfqpoint{1.756538in}{0.739656in}}%
\pgfpathlineto{\pgfqpoint{1.755973in}{0.739656in}}%
\pgfpathlineto{\pgfqpoint{1.755409in}{0.739656in}}%
\pgfpathlineto{\pgfqpoint{1.754845in}{0.739656in}}%
\pgfpathlineto{\pgfqpoint{1.754280in}{0.739656in}}%
\pgfpathlineto{\pgfqpoint{1.753716in}{0.739656in}}%
\pgfpathlineto{\pgfqpoint{1.753152in}{0.739656in}}%
\pgfpathlineto{\pgfqpoint{1.752587in}{0.739656in}}%
\pgfpathlineto{\pgfqpoint{1.752023in}{0.739656in}}%
\pgfpathlineto{\pgfqpoint{1.751459in}{0.739656in}}%
\pgfpathlineto{\pgfqpoint{1.750894in}{0.739656in}}%
\pgfpathlineto{\pgfqpoint{1.750330in}{0.739656in}}%
\pgfpathlineto{\pgfqpoint{1.749766in}{0.739656in}}%
\pgfpathlineto{\pgfqpoint{1.749201in}{0.739656in}}%
\pgfpathlineto{\pgfqpoint{1.748637in}{0.739656in}}%
\pgfpathlineto{\pgfqpoint{1.748073in}{0.739656in}}%
\pgfpathlineto{\pgfqpoint{1.747508in}{0.739656in}}%
\pgfpathlineto{\pgfqpoint{1.746944in}{0.739656in}}%
\pgfpathlineto{\pgfqpoint{1.746380in}{0.739656in}}%
\pgfpathlineto{\pgfqpoint{1.745815in}{0.739656in}}%
\pgfpathlineto{\pgfqpoint{1.745251in}{0.739656in}}%
\pgfpathlineto{\pgfqpoint{1.744687in}{0.739656in}}%
\pgfpathlineto{\pgfqpoint{1.744122in}{0.739656in}}%
\pgfpathlineto{\pgfqpoint{1.743558in}{0.739656in}}%
\pgfpathlineto{\pgfqpoint{1.742993in}{0.739656in}}%
\pgfpathlineto{\pgfqpoint{1.742429in}{0.739656in}}%
\pgfpathlineto{\pgfqpoint{1.741865in}{0.739656in}}%
\pgfpathlineto{\pgfqpoint{1.741300in}{0.739656in}}%
\pgfpathlineto{\pgfqpoint{1.740736in}{0.739656in}}%
\pgfpathlineto{\pgfqpoint{1.740172in}{0.739656in}}%
\pgfpathlineto{\pgfqpoint{1.739607in}{0.739656in}}%
\pgfpathlineto{\pgfqpoint{1.739043in}{0.739656in}}%
\pgfpathlineto{\pgfqpoint{1.738479in}{0.739656in}}%
\pgfpathlineto{\pgfqpoint{1.737914in}{0.739656in}}%
\pgfpathlineto{\pgfqpoint{1.737350in}{0.739656in}}%
\pgfpathlineto{\pgfqpoint{1.736786in}{0.739656in}}%
\pgfpathlineto{\pgfqpoint{1.736221in}{0.739656in}}%
\pgfpathlineto{\pgfqpoint{1.735657in}{0.739656in}}%
\pgfpathlineto{\pgfqpoint{1.735093in}{0.739656in}}%
\pgfpathlineto{\pgfqpoint{1.734528in}{0.739656in}}%
\pgfpathlineto{\pgfqpoint{1.733964in}{0.739656in}}%
\pgfpathlineto{\pgfqpoint{1.733400in}{0.739656in}}%
\pgfpathlineto{\pgfqpoint{1.732835in}{0.739656in}}%
\pgfpathlineto{\pgfqpoint{1.732271in}{0.739656in}}%
\pgfpathlineto{\pgfqpoint{1.731707in}{0.739656in}}%
\pgfpathlineto{\pgfqpoint{1.731142in}{0.739656in}}%
\pgfpathlineto{\pgfqpoint{1.730578in}{0.739656in}}%
\pgfpathlineto{\pgfqpoint{1.730014in}{0.739656in}}%
\pgfpathlineto{\pgfqpoint{1.729449in}{0.739656in}}%
\pgfpathlineto{\pgfqpoint{1.728885in}{0.739656in}}%
\pgfpathlineto{\pgfqpoint{1.728321in}{0.739656in}}%
\pgfpathlineto{\pgfqpoint{1.727756in}{0.739656in}}%
\pgfpathlineto{\pgfqpoint{1.727192in}{0.739656in}}%
\pgfpathlineto{\pgfqpoint{1.726627in}{0.739656in}}%
\pgfpathlineto{\pgfqpoint{1.726063in}{0.739656in}}%
\pgfpathlineto{\pgfqpoint{1.725499in}{0.739656in}}%
\pgfpathlineto{\pgfqpoint{1.724934in}{0.739656in}}%
\pgfpathlineto{\pgfqpoint{1.724370in}{0.739656in}}%
\pgfpathlineto{\pgfqpoint{1.723806in}{0.739656in}}%
\pgfpathlineto{\pgfqpoint{1.723241in}{0.739656in}}%
\pgfpathlineto{\pgfqpoint{1.722677in}{0.739656in}}%
\pgfpathlineto{\pgfqpoint{1.722113in}{0.739656in}}%
\pgfpathlineto{\pgfqpoint{1.721548in}{0.739656in}}%
\pgfpathlineto{\pgfqpoint{1.720984in}{0.739656in}}%
\pgfpathlineto{\pgfqpoint{1.720420in}{0.739656in}}%
\pgfpathlineto{\pgfqpoint{1.719855in}{0.739656in}}%
\pgfpathlineto{\pgfqpoint{1.719291in}{0.739656in}}%
\pgfpathlineto{\pgfqpoint{1.718727in}{0.739656in}}%
\pgfpathlineto{\pgfqpoint{1.718162in}{0.739656in}}%
\pgfpathlineto{\pgfqpoint{1.717598in}{0.739656in}}%
\pgfpathlineto{\pgfqpoint{1.717034in}{0.739656in}}%
\pgfpathlineto{\pgfqpoint{1.716469in}{0.739656in}}%
\pgfpathlineto{\pgfqpoint{1.715905in}{0.739656in}}%
\pgfpathlineto{\pgfqpoint{1.715341in}{0.739656in}}%
\pgfpathlineto{\pgfqpoint{1.714776in}{0.739656in}}%
\pgfpathlineto{\pgfqpoint{1.714212in}{0.739656in}}%
\pgfpathlineto{\pgfqpoint{1.713648in}{0.739656in}}%
\pgfpathlineto{\pgfqpoint{1.713083in}{0.739656in}}%
\pgfpathlineto{\pgfqpoint{1.712519in}{0.739656in}}%
\pgfpathlineto{\pgfqpoint{1.711954in}{0.739656in}}%
\pgfpathlineto{\pgfqpoint{1.711390in}{0.739656in}}%
\pgfpathlineto{\pgfqpoint{1.710826in}{0.739656in}}%
\pgfpathlineto{\pgfqpoint{1.710261in}{0.739656in}}%
\pgfpathlineto{\pgfqpoint{1.709697in}{0.739656in}}%
\pgfpathlineto{\pgfqpoint{1.709133in}{0.739656in}}%
\pgfpathlineto{\pgfqpoint{1.708568in}{0.739656in}}%
\pgfpathlineto{\pgfqpoint{1.708004in}{0.739656in}}%
\pgfpathlineto{\pgfqpoint{1.707440in}{0.739656in}}%
\pgfpathlineto{\pgfqpoint{1.706875in}{0.739656in}}%
\pgfpathlineto{\pgfqpoint{1.706311in}{0.739656in}}%
\pgfpathlineto{\pgfqpoint{1.705747in}{0.739656in}}%
\pgfpathlineto{\pgfqpoint{1.705182in}{0.739656in}}%
\pgfpathlineto{\pgfqpoint{1.704618in}{0.739656in}}%
\pgfpathlineto{\pgfqpoint{1.704054in}{0.739656in}}%
\pgfpathlineto{\pgfqpoint{1.703489in}{0.739656in}}%
\pgfpathlineto{\pgfqpoint{1.702925in}{0.739656in}}%
\pgfpathlineto{\pgfqpoint{1.702361in}{0.739656in}}%
\pgfpathlineto{\pgfqpoint{1.701796in}{0.739656in}}%
\pgfpathlineto{\pgfqpoint{1.701232in}{0.739656in}}%
\pgfpathlineto{\pgfqpoint{1.700668in}{0.739656in}}%
\pgfpathlineto{\pgfqpoint{1.700103in}{0.739656in}}%
\pgfpathlineto{\pgfqpoint{1.699539in}{0.739656in}}%
\pgfpathlineto{\pgfqpoint{1.698975in}{0.739656in}}%
\pgfpathlineto{\pgfqpoint{1.698410in}{0.739656in}}%
\pgfpathlineto{\pgfqpoint{1.697846in}{0.739656in}}%
\pgfpathlineto{\pgfqpoint{1.697281in}{0.739656in}}%
\pgfpathlineto{\pgfqpoint{1.696717in}{0.739656in}}%
\pgfpathlineto{\pgfqpoint{1.696153in}{0.739656in}}%
\pgfpathlineto{\pgfqpoint{1.695588in}{0.739656in}}%
\pgfpathlineto{\pgfqpoint{1.695024in}{0.739656in}}%
\pgfpathlineto{\pgfqpoint{1.694460in}{0.739656in}}%
\pgfpathlineto{\pgfqpoint{1.693895in}{0.739656in}}%
\pgfpathlineto{\pgfqpoint{1.693331in}{0.739656in}}%
\pgfpathlineto{\pgfqpoint{1.692767in}{0.739656in}}%
\pgfpathlineto{\pgfqpoint{1.692202in}{0.739656in}}%
\pgfpathlineto{\pgfqpoint{1.691638in}{0.739656in}}%
\pgfpathlineto{\pgfqpoint{1.691074in}{0.739656in}}%
\pgfpathlineto{\pgfqpoint{1.690509in}{0.739656in}}%
\pgfpathlineto{\pgfqpoint{1.689945in}{0.739656in}}%
\pgfpathlineto{\pgfqpoint{1.689381in}{0.739656in}}%
\pgfpathlineto{\pgfqpoint{1.688816in}{0.739656in}}%
\pgfpathlineto{\pgfqpoint{1.688252in}{0.739656in}}%
\pgfpathlineto{\pgfqpoint{1.687688in}{0.739656in}}%
\pgfpathlineto{\pgfqpoint{1.687123in}{0.739656in}}%
\pgfpathlineto{\pgfqpoint{1.686559in}{0.739656in}}%
\pgfpathlineto{\pgfqpoint{1.685995in}{0.739656in}}%
\pgfpathlineto{\pgfqpoint{1.685430in}{0.739656in}}%
\pgfpathlineto{\pgfqpoint{1.684866in}{0.739656in}}%
\pgfpathlineto{\pgfqpoint{1.684302in}{0.739656in}}%
\pgfpathlineto{\pgfqpoint{1.683737in}{0.739656in}}%
\pgfpathlineto{\pgfqpoint{1.683173in}{0.739656in}}%
\pgfpathlineto{\pgfqpoint{1.682609in}{0.739656in}}%
\pgfpathlineto{\pgfqpoint{1.682044in}{0.739656in}}%
\pgfpathlineto{\pgfqpoint{1.681480in}{0.739656in}}%
\pgfpathlineto{\pgfqpoint{1.680915in}{0.739656in}}%
\pgfpathlineto{\pgfqpoint{1.680351in}{0.739656in}}%
\pgfpathlineto{\pgfqpoint{1.679787in}{0.739656in}}%
\pgfpathlineto{\pgfqpoint{1.679222in}{0.739656in}}%
\pgfpathlineto{\pgfqpoint{1.678658in}{0.739656in}}%
\pgfpathlineto{\pgfqpoint{1.678094in}{0.739656in}}%
\pgfpathlineto{\pgfqpoint{1.677529in}{0.739656in}}%
\pgfpathlineto{\pgfqpoint{1.676965in}{0.739656in}}%
\pgfpathlineto{\pgfqpoint{1.676401in}{0.739656in}}%
\pgfpathlineto{\pgfqpoint{1.675836in}{0.739656in}}%
\pgfpathlineto{\pgfqpoint{1.675272in}{0.739656in}}%
\pgfpathlineto{\pgfqpoint{1.674708in}{0.739656in}}%
\pgfpathlineto{\pgfqpoint{1.674143in}{0.739656in}}%
\pgfpathlineto{\pgfqpoint{1.673579in}{0.739656in}}%
\pgfpathlineto{\pgfqpoint{1.673015in}{0.739656in}}%
\pgfpathlineto{\pgfqpoint{1.672450in}{0.739656in}}%
\pgfpathlineto{\pgfqpoint{1.671886in}{0.739656in}}%
\pgfpathlineto{\pgfqpoint{1.671322in}{0.739656in}}%
\pgfpathlineto{\pgfqpoint{1.670757in}{0.739656in}}%
\pgfpathlineto{\pgfqpoint{1.670193in}{0.739656in}}%
\pgfpathlineto{\pgfqpoint{1.669629in}{0.739656in}}%
\pgfpathlineto{\pgfqpoint{1.669064in}{0.739656in}}%
\pgfpathlineto{\pgfqpoint{1.668500in}{0.739656in}}%
\pgfpathlineto{\pgfqpoint{1.667936in}{0.739656in}}%
\pgfpathlineto{\pgfqpoint{1.667371in}{0.739656in}}%
\pgfpathlineto{\pgfqpoint{1.666807in}{0.739656in}}%
\pgfpathlineto{\pgfqpoint{1.666242in}{0.739656in}}%
\pgfpathlineto{\pgfqpoint{1.665678in}{0.739656in}}%
\pgfpathlineto{\pgfqpoint{1.665114in}{0.739656in}}%
\pgfpathlineto{\pgfqpoint{1.664549in}{0.739656in}}%
\pgfpathlineto{\pgfqpoint{1.663985in}{0.739656in}}%
\pgfpathlineto{\pgfqpoint{1.663421in}{0.739656in}}%
\pgfpathlineto{\pgfqpoint{1.662856in}{0.739656in}}%
\pgfpathlineto{\pgfqpoint{1.662292in}{0.739656in}}%
\pgfpathlineto{\pgfqpoint{1.661728in}{0.739656in}}%
\pgfpathlineto{\pgfqpoint{1.661163in}{0.739656in}}%
\pgfpathlineto{\pgfqpoint{1.660599in}{0.739656in}}%
\pgfpathlineto{\pgfqpoint{1.660035in}{0.739656in}}%
\pgfpathlineto{\pgfqpoint{1.659470in}{0.739656in}}%
\pgfpathlineto{\pgfqpoint{1.658906in}{0.739656in}}%
\pgfpathlineto{\pgfqpoint{1.658342in}{0.739656in}}%
\pgfpathlineto{\pgfqpoint{1.657777in}{0.739656in}}%
\pgfpathlineto{\pgfqpoint{1.657213in}{0.739656in}}%
\pgfpathlineto{\pgfqpoint{1.656649in}{0.739656in}}%
\pgfpathlineto{\pgfqpoint{1.656084in}{0.739656in}}%
\pgfpathlineto{\pgfqpoint{1.655520in}{0.739656in}}%
\pgfpathlineto{\pgfqpoint{1.654956in}{0.739656in}}%
\pgfpathlineto{\pgfqpoint{1.654391in}{0.739656in}}%
\pgfpathlineto{\pgfqpoint{1.653827in}{0.739656in}}%
\pgfpathlineto{\pgfqpoint{1.653263in}{0.739656in}}%
\pgfpathlineto{\pgfqpoint{1.652698in}{0.739656in}}%
\pgfpathlineto{\pgfqpoint{1.652134in}{0.739656in}}%
\pgfpathlineto{\pgfqpoint{1.651569in}{0.739656in}}%
\pgfpathlineto{\pgfqpoint{1.651005in}{0.739656in}}%
\pgfpathlineto{\pgfqpoint{1.650441in}{0.739656in}}%
\pgfpathlineto{\pgfqpoint{1.649876in}{0.739656in}}%
\pgfpathlineto{\pgfqpoint{1.649312in}{0.739656in}}%
\pgfpathlineto{\pgfqpoint{1.648748in}{0.739656in}}%
\pgfpathlineto{\pgfqpoint{1.648183in}{0.739656in}}%
\pgfpathlineto{\pgfqpoint{1.647619in}{0.739656in}}%
\pgfpathlineto{\pgfqpoint{1.647055in}{0.739656in}}%
\pgfpathlineto{\pgfqpoint{1.646490in}{0.739656in}}%
\pgfpathlineto{\pgfqpoint{1.645926in}{0.739656in}}%
\pgfpathlineto{\pgfqpoint{1.645362in}{0.739656in}}%
\pgfpathlineto{\pgfqpoint{1.644797in}{0.739656in}}%
\pgfpathlineto{\pgfqpoint{1.644233in}{0.739656in}}%
\pgfpathlineto{\pgfqpoint{1.643669in}{0.739656in}}%
\pgfpathlineto{\pgfqpoint{1.643104in}{0.739656in}}%
\pgfpathlineto{\pgfqpoint{1.642540in}{0.739656in}}%
\pgfpathlineto{\pgfqpoint{1.641976in}{0.739656in}}%
\pgfpathlineto{\pgfqpoint{1.641411in}{0.739656in}}%
\pgfpathlineto{\pgfqpoint{1.640847in}{0.739656in}}%
\pgfpathlineto{\pgfqpoint{1.640283in}{0.739656in}}%
\pgfpathlineto{\pgfqpoint{1.639718in}{0.739656in}}%
\pgfpathlineto{\pgfqpoint{1.639154in}{0.739656in}}%
\pgfpathlineto{\pgfqpoint{1.638590in}{0.739656in}}%
\pgfpathlineto{\pgfqpoint{1.638025in}{0.739656in}}%
\pgfpathlineto{\pgfqpoint{1.637461in}{0.739656in}}%
\pgfpathlineto{\pgfqpoint{1.636897in}{0.739656in}}%
\pgfpathlineto{\pgfqpoint{1.636332in}{0.739656in}}%
\pgfpathlineto{\pgfqpoint{1.635768in}{0.739656in}}%
\pgfpathlineto{\pgfqpoint{1.635203in}{0.739656in}}%
\pgfpathlineto{\pgfqpoint{1.634639in}{0.739656in}}%
\pgfpathlineto{\pgfqpoint{1.634075in}{0.739656in}}%
\pgfpathlineto{\pgfqpoint{1.633510in}{0.739656in}}%
\pgfpathlineto{\pgfqpoint{1.632946in}{0.739656in}}%
\pgfpathlineto{\pgfqpoint{1.632382in}{0.739656in}}%
\pgfpathlineto{\pgfqpoint{1.631817in}{0.739656in}}%
\pgfpathlineto{\pgfqpoint{1.631253in}{0.739656in}}%
\pgfpathlineto{\pgfqpoint{1.630689in}{0.739656in}}%
\pgfpathlineto{\pgfqpoint{1.630124in}{0.739656in}}%
\pgfpathlineto{\pgfqpoint{1.629560in}{0.739656in}}%
\pgfpathlineto{\pgfqpoint{1.628996in}{0.739656in}}%
\pgfpathlineto{\pgfqpoint{1.628431in}{0.739656in}}%
\pgfpathlineto{\pgfqpoint{1.627867in}{0.739656in}}%
\pgfpathlineto{\pgfqpoint{1.627303in}{0.739656in}}%
\pgfpathlineto{\pgfqpoint{1.626738in}{0.739656in}}%
\pgfpathlineto{\pgfqpoint{1.626174in}{0.739656in}}%
\pgfpathlineto{\pgfqpoint{1.625610in}{0.739656in}}%
\pgfpathlineto{\pgfqpoint{1.625045in}{0.739656in}}%
\pgfpathlineto{\pgfqpoint{1.624481in}{0.739656in}}%
\pgfpathlineto{\pgfqpoint{1.623917in}{0.739656in}}%
\pgfpathlineto{\pgfqpoint{1.623352in}{0.739656in}}%
\pgfpathlineto{\pgfqpoint{1.622788in}{0.739656in}}%
\pgfpathlineto{\pgfqpoint{1.622224in}{0.739656in}}%
\pgfpathlineto{\pgfqpoint{1.621659in}{0.739656in}}%
\pgfpathlineto{\pgfqpoint{1.621095in}{0.739656in}}%
\pgfpathlineto{\pgfqpoint{1.620530in}{0.739656in}}%
\pgfpathlineto{\pgfqpoint{1.619966in}{0.739656in}}%
\pgfpathlineto{\pgfqpoint{1.619402in}{0.739656in}}%
\pgfpathlineto{\pgfqpoint{1.618837in}{0.739656in}}%
\pgfpathlineto{\pgfqpoint{1.618273in}{0.739656in}}%
\pgfpathlineto{\pgfqpoint{1.617709in}{0.739656in}}%
\pgfpathlineto{\pgfqpoint{1.617144in}{0.739656in}}%
\pgfpathlineto{\pgfqpoint{1.616580in}{0.739656in}}%
\pgfpathlineto{\pgfqpoint{1.616016in}{0.739656in}}%
\pgfpathlineto{\pgfqpoint{1.615451in}{0.739656in}}%
\pgfpathlineto{\pgfqpoint{1.614887in}{0.739656in}}%
\pgfpathlineto{\pgfqpoint{1.614323in}{0.739656in}}%
\pgfpathlineto{\pgfqpoint{1.613758in}{0.739656in}}%
\pgfpathlineto{\pgfqpoint{1.613194in}{0.739656in}}%
\pgfpathlineto{\pgfqpoint{1.612630in}{0.739656in}}%
\pgfpathlineto{\pgfqpoint{1.612065in}{0.739656in}}%
\pgfpathlineto{\pgfqpoint{1.611501in}{0.739656in}}%
\pgfpathlineto{\pgfqpoint{1.610937in}{0.739656in}}%
\pgfpathlineto{\pgfqpoint{1.610372in}{0.739656in}}%
\pgfpathlineto{\pgfqpoint{1.609808in}{0.739656in}}%
\pgfpathlineto{\pgfqpoint{1.609244in}{0.739656in}}%
\pgfpathlineto{\pgfqpoint{1.608679in}{0.739656in}}%
\pgfpathlineto{\pgfqpoint{1.608115in}{0.739656in}}%
\pgfpathlineto{\pgfqpoint{1.607551in}{0.739656in}}%
\pgfpathlineto{\pgfqpoint{1.606986in}{0.739656in}}%
\pgfpathlineto{\pgfqpoint{1.606422in}{0.739656in}}%
\pgfpathlineto{\pgfqpoint{1.605857in}{0.739656in}}%
\pgfpathlineto{\pgfqpoint{1.605293in}{0.739656in}}%
\pgfpathlineto{\pgfqpoint{1.604729in}{0.739656in}}%
\pgfpathlineto{\pgfqpoint{1.604164in}{0.739656in}}%
\pgfpathlineto{\pgfqpoint{1.603600in}{0.739656in}}%
\pgfpathlineto{\pgfqpoint{1.603036in}{0.739656in}}%
\pgfpathlineto{\pgfqpoint{1.602471in}{0.739656in}}%
\pgfpathlineto{\pgfqpoint{1.601907in}{0.739656in}}%
\pgfpathlineto{\pgfqpoint{1.601343in}{0.739656in}}%
\pgfpathlineto{\pgfqpoint{1.600778in}{0.739656in}}%
\pgfpathlineto{\pgfqpoint{1.600214in}{0.739656in}}%
\pgfpathlineto{\pgfqpoint{1.599650in}{0.739656in}}%
\pgfpathlineto{\pgfqpoint{1.599085in}{0.739656in}}%
\pgfpathlineto{\pgfqpoint{1.598521in}{0.739656in}}%
\pgfpathlineto{\pgfqpoint{1.597957in}{0.739656in}}%
\pgfpathlineto{\pgfqpoint{1.597392in}{0.739656in}}%
\pgfpathlineto{\pgfqpoint{1.596828in}{0.739656in}}%
\pgfpathlineto{\pgfqpoint{1.596264in}{0.739656in}}%
\pgfpathlineto{\pgfqpoint{1.595699in}{0.739656in}}%
\pgfpathlineto{\pgfqpoint{1.595135in}{0.739656in}}%
\pgfpathlineto{\pgfqpoint{1.594571in}{0.739656in}}%
\pgfpathlineto{\pgfqpoint{1.594006in}{0.739656in}}%
\pgfpathlineto{\pgfqpoint{1.593442in}{0.739656in}}%
\pgfpathlineto{\pgfqpoint{1.592878in}{0.739656in}}%
\pgfpathlineto{\pgfqpoint{1.592313in}{0.739656in}}%
\pgfpathlineto{\pgfqpoint{1.591749in}{0.739656in}}%
\pgfpathlineto{\pgfqpoint{1.591184in}{0.739656in}}%
\pgfpathlineto{\pgfqpoint{1.590620in}{0.739656in}}%
\pgfpathlineto{\pgfqpoint{1.590056in}{0.739656in}}%
\pgfpathlineto{\pgfqpoint{1.589491in}{0.739656in}}%
\pgfpathlineto{\pgfqpoint{1.588927in}{0.739656in}}%
\pgfpathlineto{\pgfqpoint{1.588363in}{0.739656in}}%
\pgfpathlineto{\pgfqpoint{1.587798in}{0.739656in}}%
\pgfpathlineto{\pgfqpoint{1.587234in}{0.739656in}}%
\pgfpathlineto{\pgfqpoint{1.586670in}{0.739656in}}%
\pgfpathlineto{\pgfqpoint{1.586105in}{0.739656in}}%
\pgfpathlineto{\pgfqpoint{1.585541in}{0.739656in}}%
\pgfpathlineto{\pgfqpoint{1.584977in}{0.739656in}}%
\pgfpathlineto{\pgfqpoint{1.584412in}{0.739656in}}%
\pgfpathlineto{\pgfqpoint{1.583848in}{0.739656in}}%
\pgfpathlineto{\pgfqpoint{1.583284in}{0.739656in}}%
\pgfpathlineto{\pgfqpoint{1.582719in}{0.739656in}}%
\pgfpathlineto{\pgfqpoint{1.582155in}{0.739656in}}%
\pgfpathlineto{\pgfqpoint{1.581591in}{0.739656in}}%
\pgfpathlineto{\pgfqpoint{1.581026in}{0.739656in}}%
\pgfpathlineto{\pgfqpoint{1.580462in}{0.739656in}}%
\pgfpathlineto{\pgfqpoint{1.579898in}{0.739656in}}%
\pgfpathlineto{\pgfqpoint{1.579333in}{0.739656in}}%
\pgfpathlineto{\pgfqpoint{1.578769in}{0.739656in}}%
\pgfpathlineto{\pgfqpoint{1.578205in}{0.739656in}}%
\pgfpathlineto{\pgfqpoint{1.577640in}{0.739656in}}%
\pgfpathlineto{\pgfqpoint{1.577076in}{0.739656in}}%
\pgfpathlineto{\pgfqpoint{1.576512in}{0.739656in}}%
\pgfpathlineto{\pgfqpoint{1.575947in}{0.739656in}}%
\pgfpathlineto{\pgfqpoint{1.575383in}{0.739656in}}%
\pgfpathlineto{\pgfqpoint{1.574818in}{0.739656in}}%
\pgfpathlineto{\pgfqpoint{1.574254in}{0.739656in}}%
\pgfpathlineto{\pgfqpoint{1.573690in}{0.739656in}}%
\pgfpathlineto{\pgfqpoint{1.573125in}{0.739656in}}%
\pgfpathlineto{\pgfqpoint{1.572561in}{0.739656in}}%
\pgfpathlineto{\pgfqpoint{1.571997in}{0.739656in}}%
\pgfpathlineto{\pgfqpoint{1.571432in}{0.739656in}}%
\pgfpathlineto{\pgfqpoint{1.570868in}{0.739656in}}%
\pgfpathlineto{\pgfqpoint{1.570304in}{0.739656in}}%
\pgfpathlineto{\pgfqpoint{1.569739in}{0.739656in}}%
\pgfpathlineto{\pgfqpoint{1.569175in}{0.739656in}}%
\pgfpathlineto{\pgfqpoint{1.568611in}{0.739656in}}%
\pgfpathlineto{\pgfqpoint{1.568046in}{0.739656in}}%
\pgfpathlineto{\pgfqpoint{1.567482in}{0.739656in}}%
\pgfpathlineto{\pgfqpoint{1.566918in}{0.739656in}}%
\pgfpathlineto{\pgfqpoint{1.566353in}{0.739656in}}%
\pgfpathlineto{\pgfqpoint{1.565789in}{0.739656in}}%
\pgfpathlineto{\pgfqpoint{1.565225in}{0.739656in}}%
\pgfpathlineto{\pgfqpoint{1.564660in}{0.739656in}}%
\pgfpathlineto{\pgfqpoint{1.564096in}{0.739656in}}%
\pgfpathlineto{\pgfqpoint{1.563532in}{0.739656in}}%
\pgfpathlineto{\pgfqpoint{1.562967in}{0.739656in}}%
\pgfpathlineto{\pgfqpoint{1.562403in}{0.739656in}}%
\pgfpathlineto{\pgfqpoint{1.561839in}{0.739656in}}%
\pgfpathlineto{\pgfqpoint{1.561274in}{0.739656in}}%
\pgfpathlineto{\pgfqpoint{1.560710in}{0.739656in}}%
\pgfpathlineto{\pgfqpoint{1.560145in}{0.739656in}}%
\pgfpathlineto{\pgfqpoint{1.559581in}{0.739656in}}%
\pgfpathlineto{\pgfqpoint{1.559017in}{0.739656in}}%
\pgfpathlineto{\pgfqpoint{1.558452in}{0.739656in}}%
\pgfpathlineto{\pgfqpoint{1.557888in}{0.739656in}}%
\pgfpathlineto{\pgfqpoint{1.557324in}{0.739656in}}%
\pgfpathlineto{\pgfqpoint{1.556759in}{0.739656in}}%
\pgfpathlineto{\pgfqpoint{1.556195in}{0.739656in}}%
\pgfpathlineto{\pgfqpoint{1.555631in}{0.739656in}}%
\pgfpathlineto{\pgfqpoint{1.555066in}{0.739656in}}%
\pgfpathlineto{\pgfqpoint{1.554502in}{0.739656in}}%
\pgfpathlineto{\pgfqpoint{1.553938in}{0.739656in}}%
\pgfpathlineto{\pgfqpoint{1.553373in}{0.739656in}}%
\pgfpathlineto{\pgfqpoint{1.552809in}{0.739656in}}%
\pgfpathlineto{\pgfqpoint{1.552245in}{0.739656in}}%
\pgfpathlineto{\pgfqpoint{1.551680in}{0.739656in}}%
\pgfpathlineto{\pgfqpoint{1.551116in}{0.739656in}}%
\pgfpathlineto{\pgfqpoint{1.550552in}{0.739656in}}%
\pgfpathlineto{\pgfqpoint{1.549987in}{0.739656in}}%
\pgfpathlineto{\pgfqpoint{1.549423in}{0.739656in}}%
\pgfpathlineto{\pgfqpoint{1.548859in}{0.739656in}}%
\pgfpathlineto{\pgfqpoint{1.548294in}{0.739656in}}%
\pgfpathlineto{\pgfqpoint{1.547730in}{0.739656in}}%
\pgfpathlineto{\pgfqpoint{1.547166in}{0.739656in}}%
\pgfpathlineto{\pgfqpoint{1.546601in}{0.739656in}}%
\pgfpathlineto{\pgfqpoint{1.546037in}{0.739656in}}%
\pgfpathlineto{\pgfqpoint{1.545472in}{0.739656in}}%
\pgfpathlineto{\pgfqpoint{1.544908in}{0.739656in}}%
\pgfpathlineto{\pgfqpoint{1.544344in}{0.739656in}}%
\pgfpathlineto{\pgfqpoint{1.543779in}{0.739656in}}%
\pgfpathlineto{\pgfqpoint{1.543215in}{0.739656in}}%
\pgfpathlineto{\pgfqpoint{1.542651in}{0.739656in}}%
\pgfpathlineto{\pgfqpoint{1.542086in}{0.739656in}}%
\pgfpathlineto{\pgfqpoint{1.541522in}{0.739656in}}%
\pgfpathlineto{\pgfqpoint{1.540958in}{0.739656in}}%
\pgfpathlineto{\pgfqpoint{1.540393in}{0.739656in}}%
\pgfpathlineto{\pgfqpoint{1.539829in}{0.739656in}}%
\pgfpathlineto{\pgfqpoint{1.539265in}{0.739656in}}%
\pgfpathlineto{\pgfqpoint{1.538700in}{0.739656in}}%
\pgfpathlineto{\pgfqpoint{1.538136in}{0.739656in}}%
\pgfpathlineto{\pgfqpoint{1.537572in}{0.739656in}}%
\pgfpathlineto{\pgfqpoint{1.537007in}{0.739656in}}%
\pgfpathlineto{\pgfqpoint{1.536443in}{0.739656in}}%
\pgfpathlineto{\pgfqpoint{1.535879in}{0.739656in}}%
\pgfpathlineto{\pgfqpoint{1.535314in}{0.739656in}}%
\pgfpathlineto{\pgfqpoint{1.534750in}{0.739656in}}%
\pgfpathlineto{\pgfqpoint{1.534186in}{0.739656in}}%
\pgfpathlineto{\pgfqpoint{1.533621in}{0.739656in}}%
\pgfpathlineto{\pgfqpoint{1.533057in}{0.739656in}}%
\pgfpathlineto{\pgfqpoint{1.532493in}{0.739656in}}%
\pgfpathlineto{\pgfqpoint{1.531928in}{0.739656in}}%
\pgfpathlineto{\pgfqpoint{1.531364in}{0.739656in}}%
\pgfpathlineto{\pgfqpoint{1.530800in}{0.739656in}}%
\pgfpathlineto{\pgfqpoint{1.530235in}{0.739656in}}%
\pgfpathlineto{\pgfqpoint{1.529671in}{0.739656in}}%
\pgfpathlineto{\pgfqpoint{1.529106in}{0.739656in}}%
\pgfpathlineto{\pgfqpoint{1.528542in}{0.739656in}}%
\pgfpathlineto{\pgfqpoint{1.527978in}{0.739656in}}%
\pgfpathlineto{\pgfqpoint{1.527413in}{0.739656in}}%
\pgfpathlineto{\pgfqpoint{1.526849in}{0.739656in}}%
\pgfpathlineto{\pgfqpoint{1.526285in}{0.739656in}}%
\pgfpathlineto{\pgfqpoint{1.525720in}{0.739656in}}%
\pgfpathlineto{\pgfqpoint{1.525156in}{0.739656in}}%
\pgfpathlineto{\pgfqpoint{1.524592in}{0.739656in}}%
\pgfpathlineto{\pgfqpoint{1.524027in}{0.739656in}}%
\pgfpathlineto{\pgfqpoint{1.523463in}{0.739656in}}%
\pgfpathlineto{\pgfqpoint{1.522899in}{0.739656in}}%
\pgfpathlineto{\pgfqpoint{1.522334in}{0.739656in}}%
\pgfpathlineto{\pgfqpoint{1.521770in}{0.739656in}}%
\pgfpathlineto{\pgfqpoint{1.521206in}{0.739656in}}%
\pgfpathlineto{\pgfqpoint{1.520641in}{0.739656in}}%
\pgfpathlineto{\pgfqpoint{1.520077in}{0.739656in}}%
\pgfpathlineto{\pgfqpoint{1.519513in}{0.739656in}}%
\pgfpathlineto{\pgfqpoint{1.518948in}{0.739656in}}%
\pgfpathlineto{\pgfqpoint{1.518384in}{0.739656in}}%
\pgfpathlineto{\pgfqpoint{1.517820in}{0.739656in}}%
\pgfpathlineto{\pgfqpoint{1.517255in}{0.739656in}}%
\pgfpathlineto{\pgfqpoint{1.516691in}{0.739656in}}%
\pgfpathlineto{\pgfqpoint{1.516127in}{0.739656in}}%
\pgfpathlineto{\pgfqpoint{1.515562in}{0.739656in}}%
\pgfpathlineto{\pgfqpoint{1.514998in}{0.739656in}}%
\pgfpathlineto{\pgfqpoint{1.514433in}{0.739656in}}%
\pgfpathlineto{\pgfqpoint{1.513869in}{0.739656in}}%
\pgfpathlineto{\pgfqpoint{1.513305in}{0.739656in}}%
\pgfpathlineto{\pgfqpoint{1.512740in}{0.739656in}}%
\pgfpathlineto{\pgfqpoint{1.512176in}{0.739656in}}%
\pgfpathlineto{\pgfqpoint{1.511612in}{0.739656in}}%
\pgfpathlineto{\pgfqpoint{1.511047in}{0.739656in}}%
\pgfpathlineto{\pgfqpoint{1.510483in}{0.739656in}}%
\pgfpathlineto{\pgfqpoint{1.509919in}{0.739656in}}%
\pgfpathlineto{\pgfqpoint{1.509354in}{0.739656in}}%
\pgfpathlineto{\pgfqpoint{1.508790in}{0.739656in}}%
\pgfpathlineto{\pgfqpoint{1.508226in}{0.739656in}}%
\pgfpathlineto{\pgfqpoint{1.507661in}{0.739656in}}%
\pgfpathlineto{\pgfqpoint{1.507097in}{0.739656in}}%
\pgfpathlineto{\pgfqpoint{1.506533in}{0.739656in}}%
\pgfpathlineto{\pgfqpoint{1.505968in}{0.739656in}}%
\pgfpathlineto{\pgfqpoint{1.505404in}{0.739656in}}%
\pgfpathlineto{\pgfqpoint{1.504840in}{0.739656in}}%
\pgfpathlineto{\pgfqpoint{1.504275in}{0.739656in}}%
\pgfpathlineto{\pgfqpoint{1.503711in}{0.739656in}}%
\pgfpathlineto{\pgfqpoint{1.503147in}{0.739656in}}%
\pgfpathlineto{\pgfqpoint{1.502582in}{0.739656in}}%
\pgfpathlineto{\pgfqpoint{1.502018in}{0.739656in}}%
\pgfpathlineto{\pgfqpoint{1.501454in}{0.739656in}}%
\pgfpathlineto{\pgfqpoint{1.500889in}{0.739656in}}%
\pgfpathlineto{\pgfqpoint{1.500325in}{0.739656in}}%
\pgfpathlineto{\pgfqpoint{1.499760in}{0.739656in}}%
\pgfpathlineto{\pgfqpoint{1.499196in}{0.739656in}}%
\pgfpathlineto{\pgfqpoint{1.498632in}{0.739656in}}%
\pgfpathlineto{\pgfqpoint{1.498067in}{0.739656in}}%
\pgfpathlineto{\pgfqpoint{1.497503in}{0.739656in}}%
\pgfpathlineto{\pgfqpoint{1.496939in}{0.739656in}}%
\pgfpathlineto{\pgfqpoint{1.496374in}{0.739656in}}%
\pgfpathlineto{\pgfqpoint{1.495810in}{0.739656in}}%
\pgfpathlineto{\pgfqpoint{1.495246in}{0.739656in}}%
\pgfpathlineto{\pgfqpoint{1.494681in}{0.739656in}}%
\pgfpathlineto{\pgfqpoint{1.494117in}{0.739656in}}%
\pgfpathlineto{\pgfqpoint{1.493553in}{0.739656in}}%
\pgfpathlineto{\pgfqpoint{1.492988in}{0.739656in}}%
\pgfpathlineto{\pgfqpoint{1.492424in}{0.739656in}}%
\pgfpathlineto{\pgfqpoint{1.491860in}{0.739656in}}%
\pgfpathlineto{\pgfqpoint{1.491295in}{0.739656in}}%
\pgfpathlineto{\pgfqpoint{1.490731in}{0.739656in}}%
\pgfpathlineto{\pgfqpoint{1.490167in}{0.739656in}}%
\pgfpathlineto{\pgfqpoint{1.489602in}{0.739656in}}%
\pgfpathlineto{\pgfqpoint{1.489038in}{0.739656in}}%
\pgfpathlineto{\pgfqpoint{1.488474in}{0.739656in}}%
\pgfpathlineto{\pgfqpoint{1.487909in}{0.739656in}}%
\pgfpathlineto{\pgfqpoint{1.487345in}{0.739656in}}%
\pgfpathlineto{\pgfqpoint{1.486781in}{0.739656in}}%
\pgfpathlineto{\pgfqpoint{1.486216in}{0.739656in}}%
\pgfpathlineto{\pgfqpoint{1.485652in}{0.739656in}}%
\pgfpathlineto{\pgfqpoint{1.485088in}{0.739656in}}%
\pgfpathlineto{\pgfqpoint{1.484523in}{0.739656in}}%
\pgfpathlineto{\pgfqpoint{1.483959in}{0.739656in}}%
\pgfpathlineto{\pgfqpoint{1.483394in}{0.739656in}}%
\pgfpathlineto{\pgfqpoint{1.482830in}{0.739656in}}%
\pgfpathlineto{\pgfqpoint{1.482266in}{0.739656in}}%
\pgfpathlineto{\pgfqpoint{1.481701in}{0.739656in}}%
\pgfpathlineto{\pgfqpoint{1.481137in}{0.739656in}}%
\pgfpathlineto{\pgfqpoint{1.480573in}{0.739656in}}%
\pgfpathlineto{\pgfqpoint{1.480008in}{0.739656in}}%
\pgfpathlineto{\pgfqpoint{1.479444in}{0.739656in}}%
\pgfpathlineto{\pgfqpoint{1.478880in}{0.739656in}}%
\pgfpathlineto{\pgfqpoint{1.478315in}{0.739656in}}%
\pgfpathlineto{\pgfqpoint{1.477751in}{0.739656in}}%
\pgfpathlineto{\pgfqpoint{1.477187in}{0.739656in}}%
\pgfpathlineto{\pgfqpoint{1.476622in}{0.739656in}}%
\pgfpathlineto{\pgfqpoint{1.476058in}{0.739656in}}%
\pgfpathlineto{\pgfqpoint{1.475494in}{0.739656in}}%
\pgfpathlineto{\pgfqpoint{1.474929in}{0.739656in}}%
\pgfpathlineto{\pgfqpoint{1.474365in}{0.739656in}}%
\pgfpathlineto{\pgfqpoint{1.473801in}{0.739656in}}%
\pgfpathlineto{\pgfqpoint{1.473236in}{0.739656in}}%
\pgfpathlineto{\pgfqpoint{1.472672in}{0.739656in}}%
\pgfpathlineto{\pgfqpoint{1.472108in}{0.739656in}}%
\pgfpathlineto{\pgfqpoint{1.471543in}{0.739656in}}%
\pgfpathlineto{\pgfqpoint{1.470979in}{0.739656in}}%
\pgfpathlineto{\pgfqpoint{1.470415in}{0.739656in}}%
\pgfpathlineto{\pgfqpoint{1.469850in}{0.739656in}}%
\pgfpathlineto{\pgfqpoint{1.469286in}{0.739656in}}%
\pgfpathlineto{\pgfqpoint{1.468721in}{0.739656in}}%
\pgfpathlineto{\pgfqpoint{1.468157in}{0.739656in}}%
\pgfpathlineto{\pgfqpoint{1.467593in}{0.739656in}}%
\pgfpathlineto{\pgfqpoint{1.467028in}{0.739656in}}%
\pgfpathlineto{\pgfqpoint{1.466464in}{0.739656in}}%
\pgfpathlineto{\pgfqpoint{1.465900in}{0.739656in}}%
\pgfpathlineto{\pgfqpoint{1.465335in}{0.739656in}}%
\pgfpathlineto{\pgfqpoint{1.464771in}{0.739656in}}%
\pgfpathlineto{\pgfqpoint{1.464207in}{0.739656in}}%
\pgfpathlineto{\pgfqpoint{1.463642in}{0.739656in}}%
\pgfpathlineto{\pgfqpoint{1.463078in}{0.739656in}}%
\pgfpathlineto{\pgfqpoint{1.462514in}{0.739656in}}%
\pgfpathlineto{\pgfqpoint{1.461949in}{0.739656in}}%
\pgfpathlineto{\pgfqpoint{1.461385in}{0.739656in}}%
\pgfpathlineto{\pgfqpoint{1.460821in}{0.739656in}}%
\pgfpathlineto{\pgfqpoint{1.460256in}{0.739656in}}%
\pgfpathlineto{\pgfqpoint{1.459692in}{0.739656in}}%
\pgfpathlineto{\pgfqpoint{1.459128in}{0.739656in}}%
\pgfpathlineto{\pgfqpoint{1.458563in}{0.739656in}}%
\pgfpathlineto{\pgfqpoint{1.457999in}{0.739656in}}%
\pgfpathlineto{\pgfqpoint{1.457435in}{0.739656in}}%
\pgfpathlineto{\pgfqpoint{1.456870in}{0.739656in}}%
\pgfpathlineto{\pgfqpoint{1.456306in}{0.739656in}}%
\pgfpathlineto{\pgfqpoint{1.455742in}{0.739656in}}%
\pgfpathlineto{\pgfqpoint{1.455177in}{0.739656in}}%
\pgfpathlineto{\pgfqpoint{1.454613in}{0.739656in}}%
\pgfpathlineto{\pgfqpoint{1.454048in}{0.739656in}}%
\pgfpathlineto{\pgfqpoint{1.453484in}{0.739656in}}%
\pgfpathlineto{\pgfqpoint{1.452920in}{0.739656in}}%
\pgfpathlineto{\pgfqpoint{1.452355in}{0.739656in}}%
\pgfpathlineto{\pgfqpoint{1.451791in}{0.739656in}}%
\pgfpathlineto{\pgfqpoint{1.451227in}{0.739656in}}%
\pgfpathlineto{\pgfqpoint{1.450662in}{0.739656in}}%
\pgfpathlineto{\pgfqpoint{1.450098in}{0.739656in}}%
\pgfpathlineto{\pgfqpoint{1.449534in}{0.739656in}}%
\pgfpathlineto{\pgfqpoint{1.448969in}{0.739656in}}%
\pgfpathlineto{\pgfqpoint{1.448405in}{0.739656in}}%
\pgfpathlineto{\pgfqpoint{1.447841in}{0.739656in}}%
\pgfpathlineto{\pgfqpoint{1.447276in}{0.739656in}}%
\pgfpathlineto{\pgfqpoint{1.446712in}{0.739656in}}%
\pgfpathlineto{\pgfqpoint{1.446148in}{0.739656in}}%
\pgfpathlineto{\pgfqpoint{1.445583in}{0.739656in}}%
\pgfpathlineto{\pgfqpoint{1.445019in}{0.739656in}}%
\pgfpathlineto{\pgfqpoint{1.444455in}{0.739656in}}%
\pgfpathlineto{\pgfqpoint{1.443890in}{0.739656in}}%
\pgfpathlineto{\pgfqpoint{1.443326in}{0.739656in}}%
\pgfpathlineto{\pgfqpoint{1.442762in}{0.739656in}}%
\pgfpathlineto{\pgfqpoint{1.442197in}{0.739656in}}%
\pgfpathlineto{\pgfqpoint{1.441633in}{0.739656in}}%
\pgfpathlineto{\pgfqpoint{1.441069in}{0.739656in}}%
\pgfpathlineto{\pgfqpoint{1.440504in}{0.739656in}}%
\pgfpathlineto{\pgfqpoint{1.439940in}{0.739656in}}%
\pgfpathlineto{\pgfqpoint{1.439376in}{0.739656in}}%
\pgfpathlineto{\pgfqpoint{1.438811in}{0.739656in}}%
\pgfpathlineto{\pgfqpoint{1.438247in}{0.739656in}}%
\pgfpathlineto{\pgfqpoint{1.437682in}{0.739656in}}%
\pgfpathlineto{\pgfqpoint{1.437118in}{0.739656in}}%
\pgfpathlineto{\pgfqpoint{1.436554in}{0.739656in}}%
\pgfpathlineto{\pgfqpoint{1.435989in}{0.739656in}}%
\pgfpathlineto{\pgfqpoint{1.435425in}{0.739656in}}%
\pgfpathlineto{\pgfqpoint{1.434861in}{0.739656in}}%
\pgfpathlineto{\pgfqpoint{1.434296in}{0.739656in}}%
\pgfpathlineto{\pgfqpoint{1.433732in}{0.739656in}}%
\pgfpathlineto{\pgfqpoint{1.433168in}{0.739656in}}%
\pgfpathlineto{\pgfqpoint{1.432603in}{0.739656in}}%
\pgfpathlineto{\pgfqpoint{1.432039in}{0.739656in}}%
\pgfpathlineto{\pgfqpoint{1.431475in}{0.739656in}}%
\pgfpathlineto{\pgfqpoint{1.430910in}{0.739656in}}%
\pgfpathlineto{\pgfqpoint{1.430346in}{0.739656in}}%
\pgfpathlineto{\pgfqpoint{1.429782in}{0.739656in}}%
\pgfpathlineto{\pgfqpoint{1.429217in}{0.739656in}}%
\pgfpathlineto{\pgfqpoint{1.428653in}{0.739656in}}%
\pgfpathlineto{\pgfqpoint{1.428089in}{0.739656in}}%
\pgfpathlineto{\pgfqpoint{1.427524in}{0.739656in}}%
\pgfpathlineto{\pgfqpoint{1.426960in}{0.739656in}}%
\pgfpathlineto{\pgfqpoint{1.426396in}{0.739656in}}%
\pgfpathlineto{\pgfqpoint{1.425831in}{0.739656in}}%
\pgfpathlineto{\pgfqpoint{1.425267in}{0.739656in}}%
\pgfpathlineto{\pgfqpoint{1.424703in}{0.739656in}}%
\pgfpathlineto{\pgfqpoint{1.424138in}{0.739656in}}%
\pgfpathlineto{\pgfqpoint{1.423574in}{0.739656in}}%
\pgfpathlineto{\pgfqpoint{1.423009in}{0.739656in}}%
\pgfpathlineto{\pgfqpoint{1.422445in}{0.739656in}}%
\pgfpathlineto{\pgfqpoint{1.421881in}{0.739656in}}%
\pgfpathlineto{\pgfqpoint{1.421316in}{0.739656in}}%
\pgfpathlineto{\pgfqpoint{1.420752in}{0.739656in}}%
\pgfpathlineto{\pgfqpoint{1.420188in}{0.739656in}}%
\pgfpathlineto{\pgfqpoint{1.419623in}{0.739656in}}%
\pgfpathlineto{\pgfqpoint{1.419059in}{0.739656in}}%
\pgfpathlineto{\pgfqpoint{1.418495in}{0.739656in}}%
\pgfpathlineto{\pgfqpoint{1.417930in}{0.739656in}}%
\pgfpathlineto{\pgfqpoint{1.417366in}{0.739656in}}%
\pgfpathlineto{\pgfqpoint{1.416802in}{0.739656in}}%
\pgfpathlineto{\pgfqpoint{1.416237in}{0.739656in}}%
\pgfpathlineto{\pgfqpoint{1.415673in}{0.739656in}}%
\pgfpathlineto{\pgfqpoint{1.415109in}{0.739656in}}%
\pgfpathlineto{\pgfqpoint{1.414544in}{0.739656in}}%
\pgfpathlineto{\pgfqpoint{1.413980in}{0.739656in}}%
\pgfpathlineto{\pgfqpoint{1.413416in}{0.739656in}}%
\pgfpathlineto{\pgfqpoint{1.412851in}{0.739656in}}%
\pgfpathlineto{\pgfqpoint{1.412287in}{0.739656in}}%
\pgfpathlineto{\pgfqpoint{1.411723in}{0.739656in}}%
\pgfpathlineto{\pgfqpoint{1.411158in}{0.739656in}}%
\pgfpathlineto{\pgfqpoint{1.410594in}{0.739656in}}%
\pgfpathlineto{\pgfqpoint{1.410030in}{0.739656in}}%
\pgfpathlineto{\pgfqpoint{1.409465in}{0.739656in}}%
\pgfpathlineto{\pgfqpoint{1.408901in}{0.739656in}}%
\pgfpathlineto{\pgfqpoint{1.408336in}{0.739656in}}%
\pgfpathlineto{\pgfqpoint{1.407772in}{0.739656in}}%
\pgfpathlineto{\pgfqpoint{1.407208in}{0.739656in}}%
\pgfpathlineto{\pgfqpoint{1.406643in}{0.739656in}}%
\pgfpathlineto{\pgfqpoint{1.406079in}{0.739656in}}%
\pgfpathlineto{\pgfqpoint{1.405515in}{0.739656in}}%
\pgfpathlineto{\pgfqpoint{1.404950in}{0.739656in}}%
\pgfpathlineto{\pgfqpoint{1.404386in}{0.739656in}}%
\pgfpathlineto{\pgfqpoint{1.403822in}{0.739656in}}%
\pgfpathlineto{\pgfqpoint{1.403257in}{0.739656in}}%
\pgfpathlineto{\pgfqpoint{1.402693in}{0.739656in}}%
\pgfpathlineto{\pgfqpoint{1.402129in}{0.739656in}}%
\pgfpathlineto{\pgfqpoint{1.401564in}{0.739656in}}%
\pgfpathlineto{\pgfqpoint{1.401000in}{0.739656in}}%
\pgfpathlineto{\pgfqpoint{1.400436in}{0.739656in}}%
\pgfpathlineto{\pgfqpoint{1.399871in}{0.739656in}}%
\pgfpathlineto{\pgfqpoint{1.399307in}{0.739656in}}%
\pgfpathlineto{\pgfqpoint{1.398743in}{0.739656in}}%
\pgfpathlineto{\pgfqpoint{1.398178in}{0.739656in}}%
\pgfpathlineto{\pgfqpoint{1.397614in}{0.739656in}}%
\pgfpathlineto{\pgfqpoint{1.397050in}{0.739656in}}%
\pgfpathlineto{\pgfqpoint{1.396485in}{0.739656in}}%
\pgfpathlineto{\pgfqpoint{1.395921in}{0.739656in}}%
\pgfpathlineto{\pgfqpoint{1.395357in}{0.739656in}}%
\pgfpathlineto{\pgfqpoint{1.394792in}{0.739656in}}%
\pgfpathlineto{\pgfqpoint{1.394228in}{0.739656in}}%
\pgfpathlineto{\pgfqpoint{1.393664in}{0.739656in}}%
\pgfpathlineto{\pgfqpoint{1.393099in}{0.739656in}}%
\pgfpathlineto{\pgfqpoint{1.392535in}{0.739656in}}%
\pgfpathlineto{\pgfqpoint{1.391970in}{0.739656in}}%
\pgfpathlineto{\pgfqpoint{1.391406in}{0.739656in}}%
\pgfpathlineto{\pgfqpoint{1.390842in}{0.739656in}}%
\pgfpathlineto{\pgfqpoint{1.390277in}{0.739656in}}%
\pgfpathlineto{\pgfqpoint{1.389713in}{0.739656in}}%
\pgfpathlineto{\pgfqpoint{1.389149in}{0.739656in}}%
\pgfpathlineto{\pgfqpoint{1.388584in}{0.739656in}}%
\pgfpathlineto{\pgfqpoint{1.388020in}{0.739656in}}%
\pgfpathlineto{\pgfqpoint{1.387456in}{0.739656in}}%
\pgfpathlineto{\pgfqpoint{1.386891in}{0.739656in}}%
\pgfpathlineto{\pgfqpoint{1.386327in}{0.739656in}}%
\pgfpathlineto{\pgfqpoint{1.385763in}{0.739656in}}%
\pgfpathlineto{\pgfqpoint{1.385198in}{0.739656in}}%
\pgfpathlineto{\pgfqpoint{1.384634in}{0.739656in}}%
\pgfpathlineto{\pgfqpoint{1.384070in}{0.739656in}}%
\pgfpathlineto{\pgfqpoint{1.383505in}{0.739656in}}%
\pgfpathlineto{\pgfqpoint{1.382941in}{0.739656in}}%
\pgfpathlineto{\pgfqpoint{1.382377in}{0.739656in}}%
\pgfpathlineto{\pgfqpoint{1.381812in}{0.739656in}}%
\pgfpathlineto{\pgfqpoint{1.381248in}{0.739656in}}%
\pgfpathlineto{\pgfqpoint{1.380684in}{0.739656in}}%
\pgfpathlineto{\pgfqpoint{1.380119in}{0.739656in}}%
\pgfpathlineto{\pgfqpoint{1.379555in}{0.739656in}}%
\pgfpathlineto{\pgfqpoint{1.378991in}{0.739656in}}%
\pgfpathlineto{\pgfqpoint{1.378426in}{0.739656in}}%
\pgfpathlineto{\pgfqpoint{1.377862in}{0.739656in}}%
\pgfpathlineto{\pgfqpoint{1.377297in}{0.739656in}}%
\pgfpathlineto{\pgfqpoint{1.376733in}{0.739656in}}%
\pgfpathlineto{\pgfqpoint{1.376169in}{0.739656in}}%
\pgfpathlineto{\pgfqpoint{1.375604in}{0.739656in}}%
\pgfpathlineto{\pgfqpoint{1.375040in}{0.739656in}}%
\pgfpathlineto{\pgfqpoint{1.374476in}{0.739656in}}%
\pgfpathlineto{\pgfqpoint{1.373911in}{0.739656in}}%
\pgfpathlineto{\pgfqpoint{1.373347in}{0.739656in}}%
\pgfpathlineto{\pgfqpoint{1.372783in}{0.739656in}}%
\pgfpathlineto{\pgfqpoint{1.372218in}{0.739656in}}%
\pgfpathlineto{\pgfqpoint{1.371654in}{0.739656in}}%
\pgfpathlineto{\pgfqpoint{1.371090in}{0.739656in}}%
\pgfpathlineto{\pgfqpoint{1.370525in}{0.739656in}}%
\pgfpathlineto{\pgfqpoint{1.369961in}{0.739656in}}%
\pgfpathlineto{\pgfqpoint{1.369397in}{0.739656in}}%
\pgfpathlineto{\pgfqpoint{1.368832in}{0.739656in}}%
\pgfpathlineto{\pgfqpoint{1.368268in}{0.739656in}}%
\pgfpathlineto{\pgfqpoint{1.367704in}{0.739656in}}%
\pgfpathlineto{\pgfqpoint{1.367139in}{0.739656in}}%
\pgfpathlineto{\pgfqpoint{1.366575in}{0.739656in}}%
\pgfpathlineto{\pgfqpoint{1.366011in}{0.739656in}}%
\pgfpathlineto{\pgfqpoint{1.365446in}{0.739656in}}%
\pgfpathlineto{\pgfqpoint{1.364882in}{0.739656in}}%
\pgfpathlineto{\pgfqpoint{1.364318in}{0.739656in}}%
\pgfpathlineto{\pgfqpoint{1.363753in}{0.739656in}}%
\pgfpathlineto{\pgfqpoint{1.363189in}{0.739656in}}%
\pgfpathlineto{\pgfqpoint{1.362624in}{0.739656in}}%
\pgfpathlineto{\pgfqpoint{1.362060in}{0.739656in}}%
\pgfpathlineto{\pgfqpoint{1.361496in}{0.739656in}}%
\pgfpathlineto{\pgfqpoint{1.360931in}{0.739656in}}%
\pgfpathlineto{\pgfqpoint{1.360367in}{0.739656in}}%
\pgfpathlineto{\pgfqpoint{1.359803in}{0.739656in}}%
\pgfpathlineto{\pgfqpoint{1.359238in}{0.739656in}}%
\pgfpathlineto{\pgfqpoint{1.358674in}{0.739656in}}%
\pgfpathlineto{\pgfqpoint{1.358110in}{0.739656in}}%
\pgfpathlineto{\pgfqpoint{1.357545in}{0.739656in}}%
\pgfpathlineto{\pgfqpoint{1.356981in}{0.739656in}}%
\pgfpathlineto{\pgfqpoint{1.356417in}{0.739656in}}%
\pgfpathlineto{\pgfqpoint{1.355852in}{0.739656in}}%
\pgfpathlineto{\pgfqpoint{1.355288in}{0.739656in}}%
\pgfpathlineto{\pgfqpoint{1.354724in}{0.739656in}}%
\pgfpathlineto{\pgfqpoint{1.354159in}{0.739656in}}%
\pgfpathlineto{\pgfqpoint{1.353595in}{0.739656in}}%
\pgfpathlineto{\pgfqpoint{1.353031in}{0.739656in}}%
\pgfpathlineto{\pgfqpoint{1.352466in}{0.739656in}}%
\pgfpathlineto{\pgfqpoint{1.351902in}{0.739656in}}%
\pgfpathlineto{\pgfqpoint{1.351338in}{0.739656in}}%
\pgfpathlineto{\pgfqpoint{1.350773in}{0.739656in}}%
\pgfpathlineto{\pgfqpoint{1.350209in}{0.739656in}}%
\pgfpathlineto{\pgfqpoint{1.349645in}{0.739656in}}%
\pgfpathlineto{\pgfqpoint{1.349080in}{0.739656in}}%
\pgfpathlineto{\pgfqpoint{1.348516in}{0.739656in}}%
\pgfpathlineto{\pgfqpoint{1.347951in}{0.739656in}}%
\pgfpathlineto{\pgfqpoint{1.347387in}{0.739656in}}%
\pgfpathlineto{\pgfqpoint{1.346823in}{0.739656in}}%
\pgfpathlineto{\pgfqpoint{1.346258in}{0.739656in}}%
\pgfpathlineto{\pgfqpoint{1.345694in}{0.739656in}}%
\pgfpathlineto{\pgfqpoint{1.345130in}{0.739656in}}%
\pgfpathlineto{\pgfqpoint{1.344565in}{0.739656in}}%
\pgfpathlineto{\pgfqpoint{1.344001in}{0.739656in}}%
\pgfpathlineto{\pgfqpoint{1.343437in}{0.739656in}}%
\pgfpathlineto{\pgfqpoint{1.342872in}{0.739656in}}%
\pgfpathlineto{\pgfqpoint{1.342308in}{0.739656in}}%
\pgfpathlineto{\pgfqpoint{1.341744in}{0.739656in}}%
\pgfpathlineto{\pgfqpoint{1.341179in}{0.739656in}}%
\pgfpathlineto{\pgfqpoint{1.340615in}{0.739656in}}%
\pgfpathlineto{\pgfqpoint{1.340051in}{0.739656in}}%
\pgfpathlineto{\pgfqpoint{1.339486in}{0.739656in}}%
\pgfpathlineto{\pgfqpoint{1.338922in}{0.739656in}}%
\pgfpathlineto{\pgfqpoint{1.338358in}{0.739656in}}%
\pgfpathlineto{\pgfqpoint{1.337793in}{0.739656in}}%
\pgfpathlineto{\pgfqpoint{1.337229in}{0.739656in}}%
\pgfpathlineto{\pgfqpoint{1.336665in}{0.739656in}}%
\pgfpathlineto{\pgfqpoint{1.336100in}{0.739656in}}%
\pgfpathlineto{\pgfqpoint{1.335536in}{0.739656in}}%
\pgfpathlineto{\pgfqpoint{1.334972in}{0.739656in}}%
\pgfpathlineto{\pgfqpoint{1.334407in}{0.739656in}}%
\pgfpathlineto{\pgfqpoint{1.333843in}{0.739656in}}%
\pgfpathlineto{\pgfqpoint{1.333279in}{0.739656in}}%
\pgfpathlineto{\pgfqpoint{1.332714in}{0.739656in}}%
\pgfpathlineto{\pgfqpoint{1.332150in}{0.739656in}}%
\pgfpathlineto{\pgfqpoint{1.331585in}{0.739656in}}%
\pgfpathlineto{\pgfqpoint{1.331021in}{0.739656in}}%
\pgfpathlineto{\pgfqpoint{1.330457in}{0.739656in}}%
\pgfpathlineto{\pgfqpoint{1.329892in}{0.739656in}}%
\pgfpathlineto{\pgfqpoint{1.329328in}{0.739656in}}%
\pgfpathlineto{\pgfqpoint{1.328764in}{0.739656in}}%
\pgfpathlineto{\pgfqpoint{1.328199in}{0.739656in}}%
\pgfpathlineto{\pgfqpoint{1.327635in}{0.739656in}}%
\pgfpathlineto{\pgfqpoint{1.327071in}{0.739656in}}%
\pgfpathlineto{\pgfqpoint{1.326506in}{0.739656in}}%
\pgfpathlineto{\pgfqpoint{1.325942in}{0.739656in}}%
\pgfpathlineto{\pgfqpoint{1.325378in}{0.739656in}}%
\pgfpathlineto{\pgfqpoint{1.324813in}{0.739656in}}%
\pgfpathlineto{\pgfqpoint{1.324249in}{0.739656in}}%
\pgfpathlineto{\pgfqpoint{1.323685in}{0.739656in}}%
\pgfpathlineto{\pgfqpoint{1.323120in}{0.739656in}}%
\pgfpathlineto{\pgfqpoint{1.322556in}{0.739656in}}%
\pgfpathlineto{\pgfqpoint{1.321992in}{0.739656in}}%
\pgfpathlineto{\pgfqpoint{1.321427in}{0.739656in}}%
\pgfpathlineto{\pgfqpoint{1.320863in}{0.739656in}}%
\pgfpathlineto{\pgfqpoint{1.320299in}{0.739656in}}%
\pgfpathlineto{\pgfqpoint{1.319734in}{0.739656in}}%
\pgfpathlineto{\pgfqpoint{1.319170in}{0.739656in}}%
\pgfpathlineto{\pgfqpoint{1.318606in}{0.739656in}}%
\pgfpathlineto{\pgfqpoint{1.318041in}{0.739656in}}%
\pgfpathlineto{\pgfqpoint{1.317477in}{0.739656in}}%
\pgfpathlineto{\pgfqpoint{1.316912in}{0.739656in}}%
\pgfpathlineto{\pgfqpoint{1.316348in}{0.739656in}}%
\pgfpathlineto{\pgfqpoint{1.315784in}{0.739656in}}%
\pgfpathlineto{\pgfqpoint{1.315219in}{0.739656in}}%
\pgfpathlineto{\pgfqpoint{1.314655in}{0.739656in}}%
\pgfpathlineto{\pgfqpoint{1.314091in}{0.739656in}}%
\pgfpathlineto{\pgfqpoint{1.313526in}{0.739656in}}%
\pgfpathlineto{\pgfqpoint{1.312962in}{0.739656in}}%
\pgfpathlineto{\pgfqpoint{1.312398in}{0.739656in}}%
\pgfpathlineto{\pgfqpoint{1.311833in}{0.739656in}}%
\pgfpathlineto{\pgfqpoint{1.311269in}{0.739656in}}%
\pgfpathlineto{\pgfqpoint{1.310705in}{0.739656in}}%
\pgfpathlineto{\pgfqpoint{1.310140in}{0.739656in}}%
\pgfpathlineto{\pgfqpoint{1.309576in}{0.739656in}}%
\pgfpathlineto{\pgfqpoint{1.309012in}{0.739656in}}%
\pgfpathlineto{\pgfqpoint{1.308447in}{0.739656in}}%
\pgfpathlineto{\pgfqpoint{1.307883in}{0.739656in}}%
\pgfpathlineto{\pgfqpoint{1.307319in}{0.739656in}}%
\pgfpathlineto{\pgfqpoint{1.306754in}{0.739656in}}%
\pgfpathlineto{\pgfqpoint{1.306190in}{0.739656in}}%
\pgfpathlineto{\pgfqpoint{1.305626in}{0.739656in}}%
\pgfpathlineto{\pgfqpoint{1.305061in}{0.739656in}}%
\pgfpathlineto{\pgfqpoint{1.304497in}{0.739656in}}%
\pgfpathlineto{\pgfqpoint{1.303933in}{0.739656in}}%
\pgfpathlineto{\pgfqpoint{1.303368in}{0.739656in}}%
\pgfpathlineto{\pgfqpoint{1.302804in}{0.739656in}}%
\pgfpathlineto{\pgfqpoint{1.302239in}{0.739656in}}%
\pgfpathlineto{\pgfqpoint{1.301675in}{0.739656in}}%
\pgfpathlineto{\pgfqpoint{1.301111in}{0.739656in}}%
\pgfpathlineto{\pgfqpoint{1.300546in}{0.739656in}}%
\pgfpathlineto{\pgfqpoint{1.299982in}{0.739656in}}%
\pgfpathlineto{\pgfqpoint{1.299418in}{0.739656in}}%
\pgfpathlineto{\pgfqpoint{1.298853in}{0.739656in}}%
\pgfpathlineto{\pgfqpoint{1.298289in}{0.739656in}}%
\pgfpathlineto{\pgfqpoint{1.297725in}{0.739656in}}%
\pgfpathlineto{\pgfqpoint{1.297160in}{0.739656in}}%
\pgfpathlineto{\pgfqpoint{1.296596in}{0.739656in}}%
\pgfpathlineto{\pgfqpoint{1.296032in}{0.739656in}}%
\pgfpathlineto{\pgfqpoint{1.295467in}{0.739656in}}%
\pgfpathlineto{\pgfqpoint{1.294903in}{0.739656in}}%
\pgfpathlineto{\pgfqpoint{1.294339in}{0.739656in}}%
\pgfpathlineto{\pgfqpoint{1.293774in}{0.739656in}}%
\pgfpathlineto{\pgfqpoint{1.293210in}{0.739656in}}%
\pgfpathlineto{\pgfqpoint{1.292646in}{0.739656in}}%
\pgfpathlineto{\pgfqpoint{1.292081in}{0.739656in}}%
\pgfpathlineto{\pgfqpoint{1.291517in}{0.739656in}}%
\pgfpathlineto{\pgfqpoint{1.290953in}{0.739656in}}%
\pgfpathlineto{\pgfqpoint{1.290388in}{0.739656in}}%
\pgfpathlineto{\pgfqpoint{1.289824in}{0.739656in}}%
\pgfpathlineto{\pgfqpoint{1.289260in}{0.739656in}}%
\pgfpathlineto{\pgfqpoint{1.288695in}{0.739656in}}%
\pgfpathlineto{\pgfqpoint{1.288131in}{0.739656in}}%
\pgfpathlineto{\pgfqpoint{1.287567in}{0.739656in}}%
\pgfpathlineto{\pgfqpoint{1.287002in}{0.739656in}}%
\pgfpathlineto{\pgfqpoint{1.286438in}{0.739656in}}%
\pgfpathlineto{\pgfqpoint{1.285873in}{0.739656in}}%
\pgfpathlineto{\pgfqpoint{1.285309in}{0.739656in}}%
\pgfpathlineto{\pgfqpoint{1.284745in}{0.739656in}}%
\pgfpathlineto{\pgfqpoint{1.284180in}{0.739656in}}%
\pgfpathlineto{\pgfqpoint{1.283616in}{0.739656in}}%
\pgfpathlineto{\pgfqpoint{1.283052in}{0.739656in}}%
\pgfpathlineto{\pgfqpoint{1.282487in}{0.739656in}}%
\pgfpathlineto{\pgfqpoint{1.281923in}{0.739656in}}%
\pgfpathlineto{\pgfqpoint{1.281359in}{0.739656in}}%
\pgfpathlineto{\pgfqpoint{1.280794in}{0.739656in}}%
\pgfpathlineto{\pgfqpoint{1.280230in}{0.739656in}}%
\pgfpathlineto{\pgfqpoint{1.279666in}{0.739656in}}%
\pgfpathlineto{\pgfqpoint{1.279101in}{0.739656in}}%
\pgfpathlineto{\pgfqpoint{1.278537in}{0.739656in}}%
\pgfpathlineto{\pgfqpoint{1.277973in}{0.739656in}}%
\pgfpathlineto{\pgfqpoint{1.277408in}{0.739656in}}%
\pgfpathlineto{\pgfqpoint{1.276844in}{0.739656in}}%
\pgfpathlineto{\pgfqpoint{1.276280in}{0.739656in}}%
\pgfpathlineto{\pgfqpoint{1.275715in}{0.739656in}}%
\pgfpathlineto{\pgfqpoint{1.275151in}{0.739656in}}%
\pgfpathlineto{\pgfqpoint{1.274587in}{0.739656in}}%
\pgfpathlineto{\pgfqpoint{1.274022in}{0.739656in}}%
\pgfpathlineto{\pgfqpoint{1.273458in}{0.739656in}}%
\pgfpathlineto{\pgfqpoint{1.272894in}{0.739656in}}%
\pgfpathlineto{\pgfqpoint{1.272329in}{0.739656in}}%
\pgfpathlineto{\pgfqpoint{1.271765in}{0.739656in}}%
\pgfpathlineto{\pgfqpoint{1.271200in}{0.739656in}}%
\pgfpathlineto{\pgfqpoint{1.270636in}{0.739656in}}%
\pgfpathlineto{\pgfqpoint{1.270072in}{0.739656in}}%
\pgfpathlineto{\pgfqpoint{1.269507in}{0.739656in}}%
\pgfpathlineto{\pgfqpoint{1.268943in}{0.739656in}}%
\pgfpathlineto{\pgfqpoint{1.268379in}{0.739656in}}%
\pgfpathlineto{\pgfqpoint{1.267814in}{0.739656in}}%
\pgfpathlineto{\pgfqpoint{1.267250in}{0.739656in}}%
\pgfpathlineto{\pgfqpoint{1.266686in}{0.739656in}}%
\pgfpathlineto{\pgfqpoint{1.266121in}{0.739656in}}%
\pgfpathlineto{\pgfqpoint{1.265557in}{0.739656in}}%
\pgfpathlineto{\pgfqpoint{1.264993in}{0.739656in}}%
\pgfpathlineto{\pgfqpoint{1.264428in}{0.739656in}}%
\pgfpathlineto{\pgfqpoint{1.263864in}{0.739656in}}%
\pgfpathlineto{\pgfqpoint{1.263300in}{0.739656in}}%
\pgfpathlineto{\pgfqpoint{1.262735in}{0.739656in}}%
\pgfpathlineto{\pgfqpoint{1.262171in}{0.739656in}}%
\pgfpathlineto{\pgfqpoint{1.261607in}{0.739656in}}%
\pgfpathlineto{\pgfqpoint{1.261042in}{0.739656in}}%
\pgfpathlineto{\pgfqpoint{1.260478in}{0.739656in}}%
\pgfpathlineto{\pgfqpoint{1.259914in}{0.739656in}}%
\pgfpathlineto{\pgfqpoint{1.259349in}{0.739656in}}%
\pgfpathlineto{\pgfqpoint{1.258785in}{0.739656in}}%
\pgfpathlineto{\pgfqpoint{1.258221in}{0.739656in}}%
\pgfpathlineto{\pgfqpoint{1.257656in}{0.739656in}}%
\pgfpathlineto{\pgfqpoint{1.257092in}{0.739656in}}%
\pgfpathlineto{\pgfqpoint{1.256527in}{0.739656in}}%
\pgfpathlineto{\pgfqpoint{1.255963in}{0.739656in}}%
\pgfpathlineto{\pgfqpoint{1.255399in}{0.739656in}}%
\pgfpathlineto{\pgfqpoint{1.254834in}{0.739656in}}%
\pgfpathlineto{\pgfqpoint{1.254270in}{0.739656in}}%
\pgfpathlineto{\pgfqpoint{1.253706in}{0.739656in}}%
\pgfpathlineto{\pgfqpoint{1.253141in}{0.739656in}}%
\pgfpathlineto{\pgfqpoint{1.252577in}{0.739656in}}%
\pgfpathlineto{\pgfqpoint{1.252013in}{0.739656in}}%
\pgfpathlineto{\pgfqpoint{1.251448in}{0.739656in}}%
\pgfpathlineto{\pgfqpoint{1.250884in}{0.739656in}}%
\pgfpathlineto{\pgfqpoint{1.250320in}{0.739656in}}%
\pgfpathlineto{\pgfqpoint{1.249755in}{0.739656in}}%
\pgfpathlineto{\pgfqpoint{1.249191in}{0.739656in}}%
\pgfpathlineto{\pgfqpoint{1.248627in}{0.739656in}}%
\pgfpathlineto{\pgfqpoint{1.248062in}{0.739656in}}%
\pgfpathlineto{\pgfqpoint{1.247498in}{0.739656in}}%
\pgfpathlineto{\pgfqpoint{1.246934in}{0.739656in}}%
\pgfpathlineto{\pgfqpoint{1.246369in}{0.739656in}}%
\pgfpathlineto{\pgfqpoint{1.245805in}{0.739656in}}%
\pgfpathlineto{\pgfqpoint{1.245241in}{0.739656in}}%
\pgfpathlineto{\pgfqpoint{1.244676in}{0.739656in}}%
\pgfpathlineto{\pgfqpoint{1.244112in}{0.739656in}}%
\pgfpathlineto{\pgfqpoint{1.243548in}{0.739656in}}%
\pgfpathlineto{\pgfqpoint{1.242983in}{0.739656in}}%
\pgfpathlineto{\pgfqpoint{1.242419in}{0.739656in}}%
\pgfpathlineto{\pgfqpoint{1.241855in}{0.739656in}}%
\pgfpathlineto{\pgfqpoint{1.241290in}{0.739656in}}%
\pgfpathlineto{\pgfqpoint{1.240726in}{0.739656in}}%
\pgfpathlineto{\pgfqpoint{1.240161in}{0.739656in}}%
\pgfpathlineto{\pgfqpoint{1.239597in}{0.739656in}}%
\pgfpathlineto{\pgfqpoint{1.239033in}{0.739656in}}%
\pgfpathlineto{\pgfqpoint{1.238468in}{0.739656in}}%
\pgfpathlineto{\pgfqpoint{1.237904in}{0.739656in}}%
\pgfpathlineto{\pgfqpoint{1.237340in}{0.739656in}}%
\pgfpathlineto{\pgfqpoint{1.236775in}{0.739656in}}%
\pgfpathlineto{\pgfqpoint{1.236211in}{0.739656in}}%
\pgfpathlineto{\pgfqpoint{1.235647in}{0.739656in}}%
\pgfpathlineto{\pgfqpoint{1.235082in}{0.739656in}}%
\pgfpathlineto{\pgfqpoint{1.234518in}{0.739656in}}%
\pgfpathlineto{\pgfqpoint{1.233954in}{0.739656in}}%
\pgfpathlineto{\pgfqpoint{1.233389in}{0.739656in}}%
\pgfpathlineto{\pgfqpoint{1.232825in}{0.739656in}}%
\pgfpathlineto{\pgfqpoint{1.232261in}{0.739656in}}%
\pgfpathlineto{\pgfqpoint{1.231696in}{0.739656in}}%
\pgfpathlineto{\pgfqpoint{1.231132in}{0.739656in}}%
\pgfpathlineto{\pgfqpoint{1.230568in}{0.739656in}}%
\pgfpathlineto{\pgfqpoint{1.230003in}{0.739656in}}%
\pgfpathlineto{\pgfqpoint{1.229439in}{0.739656in}}%
\pgfpathlineto{\pgfqpoint{1.228875in}{0.739656in}}%
\pgfpathlineto{\pgfqpoint{1.228310in}{0.739656in}}%
\pgfpathlineto{\pgfqpoint{1.227746in}{0.739656in}}%
\pgfpathlineto{\pgfqpoint{1.227182in}{0.739656in}}%
\pgfpathlineto{\pgfqpoint{1.226617in}{0.739656in}}%
\pgfpathlineto{\pgfqpoint{1.226053in}{0.739656in}}%
\pgfpathlineto{\pgfqpoint{1.225488in}{0.739656in}}%
\pgfpathlineto{\pgfqpoint{1.224924in}{0.739656in}}%
\pgfpathlineto{\pgfqpoint{1.224360in}{0.739656in}}%
\pgfpathlineto{\pgfqpoint{1.223795in}{0.739656in}}%
\pgfpathlineto{\pgfqpoint{1.223231in}{0.739656in}}%
\pgfpathlineto{\pgfqpoint{1.222667in}{0.739656in}}%
\pgfpathlineto{\pgfqpoint{1.222102in}{0.739656in}}%
\pgfpathlineto{\pgfqpoint{1.221538in}{0.739656in}}%
\pgfpathlineto{\pgfqpoint{1.220974in}{0.739656in}}%
\pgfpathlineto{\pgfqpoint{1.220409in}{0.739656in}}%
\pgfpathlineto{\pgfqpoint{1.219845in}{0.739656in}}%
\pgfpathlineto{\pgfqpoint{1.219281in}{0.739656in}}%
\pgfpathlineto{\pgfqpoint{1.218716in}{0.739656in}}%
\pgfpathlineto{\pgfqpoint{1.218152in}{0.739656in}}%
\pgfpathlineto{\pgfqpoint{1.217588in}{0.739656in}}%
\pgfpathlineto{\pgfqpoint{1.217023in}{0.739656in}}%
\pgfpathlineto{\pgfqpoint{1.216459in}{0.739656in}}%
\pgfpathlineto{\pgfqpoint{1.215895in}{0.739656in}}%
\pgfpathlineto{\pgfqpoint{1.215330in}{0.739656in}}%
\pgfpathlineto{\pgfqpoint{1.214766in}{0.739656in}}%
\pgfpathlineto{\pgfqpoint{1.214202in}{0.739656in}}%
\pgfpathlineto{\pgfqpoint{1.213637in}{0.739656in}}%
\pgfpathlineto{\pgfqpoint{1.213073in}{0.739656in}}%
\pgfpathlineto{\pgfqpoint{1.212509in}{0.739656in}}%
\pgfpathlineto{\pgfqpoint{1.211944in}{0.739656in}}%
\pgfpathlineto{\pgfqpoint{1.211380in}{0.739656in}}%
\pgfpathlineto{\pgfqpoint{1.210815in}{0.739656in}}%
\pgfpathlineto{\pgfqpoint{1.210251in}{0.739656in}}%
\pgfpathlineto{\pgfqpoint{1.209687in}{0.739656in}}%
\pgfpathlineto{\pgfqpoint{1.209122in}{0.739656in}}%
\pgfpathlineto{\pgfqpoint{1.208558in}{0.739656in}}%
\pgfpathlineto{\pgfqpoint{1.207994in}{0.739656in}}%
\pgfpathlineto{\pgfqpoint{1.207429in}{0.739656in}}%
\pgfpathlineto{\pgfqpoint{1.206865in}{0.739656in}}%
\pgfpathlineto{\pgfqpoint{1.206301in}{0.739656in}}%
\pgfpathlineto{\pgfqpoint{1.205736in}{0.739656in}}%
\pgfpathlineto{\pgfqpoint{1.205172in}{0.739656in}}%
\pgfpathlineto{\pgfqpoint{1.204608in}{0.739656in}}%
\pgfpathlineto{\pgfqpoint{1.204043in}{0.739656in}}%
\pgfpathlineto{\pgfqpoint{1.203479in}{0.739656in}}%
\pgfpathlineto{\pgfqpoint{1.202915in}{0.739656in}}%
\pgfpathlineto{\pgfqpoint{1.202350in}{0.739656in}}%
\pgfpathlineto{\pgfqpoint{1.201786in}{0.739656in}}%
\pgfpathlineto{\pgfqpoint{1.201222in}{0.739656in}}%
\pgfpathlineto{\pgfqpoint{1.200657in}{0.739656in}}%
\pgfpathlineto{\pgfqpoint{1.200093in}{0.739656in}}%
\pgfpathlineto{\pgfqpoint{1.199529in}{0.739656in}}%
\pgfpathlineto{\pgfqpoint{1.198964in}{0.739656in}}%
\pgfpathlineto{\pgfqpoint{1.198400in}{0.739656in}}%
\pgfpathlineto{\pgfqpoint{1.197836in}{0.739656in}}%
\pgfpathlineto{\pgfqpoint{1.197271in}{0.739656in}}%
\pgfpathlineto{\pgfqpoint{1.196707in}{0.739656in}}%
\pgfpathlineto{\pgfqpoint{1.196143in}{0.739656in}}%
\pgfpathlineto{\pgfqpoint{1.195578in}{0.739656in}}%
\pgfpathlineto{\pgfqpoint{1.195014in}{0.739656in}}%
\pgfpathlineto{\pgfqpoint{1.194449in}{0.739656in}}%
\pgfpathlineto{\pgfqpoint{1.193885in}{0.739656in}}%
\pgfpathlineto{\pgfqpoint{1.193321in}{0.739656in}}%
\pgfpathlineto{\pgfqpoint{1.192756in}{0.739656in}}%
\pgfpathlineto{\pgfqpoint{1.192192in}{0.739656in}}%
\pgfpathlineto{\pgfqpoint{1.191628in}{0.739656in}}%
\pgfpathlineto{\pgfqpoint{1.191063in}{0.739656in}}%
\pgfpathlineto{\pgfqpoint{1.190499in}{0.739656in}}%
\pgfpathlineto{\pgfqpoint{1.189935in}{0.739656in}}%
\pgfpathlineto{\pgfqpoint{1.189370in}{0.739656in}}%
\pgfpathlineto{\pgfqpoint{1.188806in}{0.739656in}}%
\pgfpathlineto{\pgfqpoint{1.188242in}{0.739656in}}%
\pgfpathlineto{\pgfqpoint{1.187677in}{0.739656in}}%
\pgfpathlineto{\pgfqpoint{1.187113in}{0.739656in}}%
\pgfpathlineto{\pgfqpoint{1.186549in}{0.739656in}}%
\pgfpathlineto{\pgfqpoint{1.185984in}{0.739656in}}%
\pgfpathlineto{\pgfqpoint{1.185420in}{0.739656in}}%
\pgfpathlineto{\pgfqpoint{1.184856in}{0.739656in}}%
\pgfpathlineto{\pgfqpoint{1.184291in}{0.739656in}}%
\pgfpathlineto{\pgfqpoint{1.183727in}{0.739656in}}%
\pgfpathlineto{\pgfqpoint{1.183163in}{0.739656in}}%
\pgfpathlineto{\pgfqpoint{1.182598in}{0.739656in}}%
\pgfpathlineto{\pgfqpoint{1.182034in}{0.739656in}}%
\pgfpathlineto{\pgfqpoint{1.181470in}{0.739656in}}%
\pgfpathlineto{\pgfqpoint{1.180905in}{0.739656in}}%
\pgfpathlineto{\pgfqpoint{1.180341in}{0.739656in}}%
\pgfpathlineto{\pgfqpoint{1.179776in}{0.739656in}}%
\pgfpathlineto{\pgfqpoint{1.179212in}{0.739656in}}%
\pgfpathlineto{\pgfqpoint{1.178648in}{0.739656in}}%
\pgfpathlineto{\pgfqpoint{1.178083in}{0.739656in}}%
\pgfpathlineto{\pgfqpoint{1.177519in}{0.739656in}}%
\pgfpathlineto{\pgfqpoint{1.176955in}{0.739656in}}%
\pgfpathlineto{\pgfqpoint{1.176390in}{0.739656in}}%
\pgfpathlineto{\pgfqpoint{1.175826in}{0.739656in}}%
\pgfpathlineto{\pgfqpoint{1.175262in}{0.739656in}}%
\pgfpathlineto{\pgfqpoint{1.174697in}{0.739656in}}%
\pgfpathlineto{\pgfqpoint{1.174133in}{0.739656in}}%
\pgfpathlineto{\pgfqpoint{1.173569in}{0.739656in}}%
\pgfpathlineto{\pgfqpoint{1.173004in}{0.739656in}}%
\pgfpathlineto{\pgfqpoint{1.172440in}{0.739656in}}%
\pgfpathlineto{\pgfqpoint{1.171876in}{0.739656in}}%
\pgfpathlineto{\pgfqpoint{1.171311in}{0.739656in}}%
\pgfpathlineto{\pgfqpoint{1.170747in}{0.739656in}}%
\pgfpathlineto{\pgfqpoint{1.170183in}{0.739656in}}%
\pgfpathlineto{\pgfqpoint{1.169618in}{0.739656in}}%
\pgfpathlineto{\pgfqpoint{1.169054in}{0.739656in}}%
\pgfpathlineto{\pgfqpoint{1.168490in}{0.739656in}}%
\pgfpathlineto{\pgfqpoint{1.167925in}{0.739656in}}%
\pgfpathlineto{\pgfqpoint{1.167361in}{0.739656in}}%
\pgfpathlineto{\pgfqpoint{1.166797in}{0.739656in}}%
\pgfpathlineto{\pgfqpoint{1.166232in}{0.739656in}}%
\pgfpathlineto{\pgfqpoint{1.165668in}{0.739656in}}%
\pgfpathlineto{\pgfqpoint{1.165103in}{0.739656in}}%
\pgfpathlineto{\pgfqpoint{1.164539in}{0.739656in}}%
\pgfpathlineto{\pgfqpoint{1.163975in}{0.739656in}}%
\pgfpathlineto{\pgfqpoint{1.163410in}{0.739656in}}%
\pgfpathlineto{\pgfqpoint{1.162846in}{0.739656in}}%
\pgfpathlineto{\pgfqpoint{1.162282in}{0.739656in}}%
\pgfpathlineto{\pgfqpoint{1.161717in}{0.739656in}}%
\pgfpathlineto{\pgfqpoint{1.161153in}{0.739656in}}%
\pgfpathlineto{\pgfqpoint{1.160589in}{0.739656in}}%
\pgfpathlineto{\pgfqpoint{1.160024in}{0.739656in}}%
\pgfpathlineto{\pgfqpoint{1.159460in}{0.739656in}}%
\pgfpathlineto{\pgfqpoint{1.158896in}{0.739656in}}%
\pgfpathlineto{\pgfqpoint{1.158331in}{0.739656in}}%
\pgfpathlineto{\pgfqpoint{1.157767in}{0.739656in}}%
\pgfpathlineto{\pgfqpoint{1.157203in}{0.739656in}}%
\pgfpathlineto{\pgfqpoint{1.156638in}{0.739656in}}%
\pgfpathlineto{\pgfqpoint{1.156074in}{0.739656in}}%
\pgfpathlineto{\pgfqpoint{1.155510in}{0.739656in}}%
\pgfpathlineto{\pgfqpoint{1.154945in}{0.739656in}}%
\pgfpathlineto{\pgfqpoint{1.154381in}{0.739656in}}%
\pgfpathlineto{\pgfqpoint{1.153817in}{0.739656in}}%
\pgfpathlineto{\pgfqpoint{1.153252in}{0.739656in}}%
\pgfpathlineto{\pgfqpoint{1.152688in}{0.739656in}}%
\pgfpathlineto{\pgfqpoint{1.152124in}{0.739656in}}%
\pgfpathlineto{\pgfqpoint{1.151559in}{0.739656in}}%
\pgfpathlineto{\pgfqpoint{1.150995in}{0.739656in}}%
\pgfpathlineto{\pgfqpoint{1.150431in}{0.739656in}}%
\pgfpathlineto{\pgfqpoint{1.149866in}{0.739656in}}%
\pgfpathlineto{\pgfqpoint{1.149302in}{0.739656in}}%
\pgfpathlineto{\pgfqpoint{1.148737in}{0.739656in}}%
\pgfpathlineto{\pgfqpoint{1.148173in}{0.739656in}}%
\pgfpathlineto{\pgfqpoint{1.147609in}{0.739656in}}%
\pgfpathlineto{\pgfqpoint{1.147044in}{0.739656in}}%
\pgfpathlineto{\pgfqpoint{1.146480in}{0.739656in}}%
\pgfpathlineto{\pgfqpoint{1.145916in}{0.739656in}}%
\pgfpathlineto{\pgfqpoint{1.145351in}{0.739656in}}%
\pgfpathlineto{\pgfqpoint{1.144787in}{0.739656in}}%
\pgfpathlineto{\pgfqpoint{1.144223in}{0.739656in}}%
\pgfpathlineto{\pgfqpoint{1.143658in}{0.739656in}}%
\pgfpathlineto{\pgfqpoint{1.143094in}{0.739656in}}%
\pgfpathlineto{\pgfqpoint{1.142530in}{0.739656in}}%
\pgfpathlineto{\pgfqpoint{1.141965in}{0.739656in}}%
\pgfpathlineto{\pgfqpoint{1.141401in}{0.739656in}}%
\pgfpathlineto{\pgfqpoint{1.140837in}{0.739656in}}%
\pgfpathlineto{\pgfqpoint{1.140272in}{0.739656in}}%
\pgfpathlineto{\pgfqpoint{1.139708in}{0.739656in}}%
\pgfpathlineto{\pgfqpoint{1.139144in}{0.739656in}}%
\pgfpathlineto{\pgfqpoint{1.138579in}{0.739656in}}%
\pgfpathlineto{\pgfqpoint{1.138015in}{0.739656in}}%
\pgfpathlineto{\pgfqpoint{1.137451in}{0.739656in}}%
\pgfpathlineto{\pgfqpoint{1.136886in}{0.739656in}}%
\pgfpathlineto{\pgfqpoint{1.136322in}{0.739656in}}%
\pgfpathlineto{\pgfqpoint{1.135758in}{0.739656in}}%
\pgfpathlineto{\pgfqpoint{1.135193in}{0.739656in}}%
\pgfpathlineto{\pgfqpoint{1.134629in}{0.739656in}}%
\pgfpathlineto{\pgfqpoint{1.134064in}{0.739656in}}%
\pgfpathlineto{\pgfqpoint{1.133500in}{0.739656in}}%
\pgfpathlineto{\pgfqpoint{1.132936in}{0.739656in}}%
\pgfpathlineto{\pgfqpoint{1.132371in}{0.739656in}}%
\pgfpathlineto{\pgfqpoint{1.131807in}{0.739656in}}%
\pgfpathlineto{\pgfqpoint{1.131243in}{0.739656in}}%
\pgfpathlineto{\pgfqpoint{1.130678in}{0.739656in}}%
\pgfpathlineto{\pgfqpoint{1.130114in}{0.739656in}}%
\pgfpathlineto{\pgfqpoint{1.129550in}{0.739656in}}%
\pgfpathlineto{\pgfqpoint{1.128985in}{0.739656in}}%
\pgfpathlineto{\pgfqpoint{1.128421in}{0.739656in}}%
\pgfpathlineto{\pgfqpoint{1.127857in}{0.739656in}}%
\pgfpathlineto{\pgfqpoint{1.127292in}{0.739656in}}%
\pgfpathlineto{\pgfqpoint{1.126728in}{0.739656in}}%
\pgfpathlineto{\pgfqpoint{1.126164in}{0.739656in}}%
\pgfpathlineto{\pgfqpoint{1.125599in}{0.739656in}}%
\pgfpathlineto{\pgfqpoint{1.125035in}{0.739656in}}%
\pgfpathlineto{\pgfqpoint{1.124471in}{0.739656in}}%
\pgfpathlineto{\pgfqpoint{1.123906in}{0.739656in}}%
\pgfpathlineto{\pgfqpoint{1.123342in}{0.739656in}}%
\pgfpathlineto{\pgfqpoint{1.122778in}{0.739656in}}%
\pgfpathlineto{\pgfqpoint{1.122213in}{0.739656in}}%
\pgfpathlineto{\pgfqpoint{1.121649in}{0.739656in}}%
\pgfpathlineto{\pgfqpoint{1.121085in}{0.739656in}}%
\pgfpathlineto{\pgfqpoint{1.120520in}{0.739656in}}%
\pgfpathlineto{\pgfqpoint{1.119956in}{0.739656in}}%
\pgfpathlineto{\pgfqpoint{1.119391in}{0.739656in}}%
\pgfpathlineto{\pgfqpoint{1.118827in}{0.739656in}}%
\pgfpathlineto{\pgfqpoint{1.118263in}{0.739656in}}%
\pgfpathlineto{\pgfqpoint{1.117698in}{0.739656in}}%
\pgfpathlineto{\pgfqpoint{1.117134in}{0.739656in}}%
\pgfpathlineto{\pgfqpoint{1.116570in}{0.739656in}}%
\pgfpathlineto{\pgfqpoint{1.116005in}{0.739656in}}%
\pgfpathlineto{\pgfqpoint{1.115441in}{0.739656in}}%
\pgfpathlineto{\pgfqpoint{1.114877in}{0.739656in}}%
\pgfpathlineto{\pgfqpoint{1.114312in}{0.739656in}}%
\pgfpathlineto{\pgfqpoint{1.113748in}{0.739656in}}%
\pgfpathlineto{\pgfqpoint{1.113184in}{0.739656in}}%
\pgfpathlineto{\pgfqpoint{1.112619in}{0.739656in}}%
\pgfpathlineto{\pgfqpoint{1.112055in}{0.739656in}}%
\pgfpathlineto{\pgfqpoint{1.111491in}{0.739656in}}%
\pgfpathlineto{\pgfqpoint{1.110926in}{0.739656in}}%
\pgfpathlineto{\pgfqpoint{1.110362in}{0.739656in}}%
\pgfpathlineto{\pgfqpoint{1.109798in}{0.739656in}}%
\pgfpathlineto{\pgfqpoint{1.109233in}{0.739656in}}%
\pgfpathlineto{\pgfqpoint{1.108669in}{0.739656in}}%
\pgfpathlineto{\pgfqpoint{1.108105in}{0.739656in}}%
\pgfpathlineto{\pgfqpoint{1.107540in}{0.739656in}}%
\pgfpathlineto{\pgfqpoint{1.106976in}{0.739656in}}%
\pgfpathlineto{\pgfqpoint{1.106412in}{0.739656in}}%
\pgfpathlineto{\pgfqpoint{1.105847in}{0.739656in}}%
\pgfpathlineto{\pgfqpoint{1.105283in}{0.739656in}}%
\pgfpathlineto{\pgfqpoint{1.104718in}{0.739656in}}%
\pgfpathlineto{\pgfqpoint{1.104154in}{0.739656in}}%
\pgfpathlineto{\pgfqpoint{1.103590in}{0.739656in}}%
\pgfpathlineto{\pgfqpoint{1.103025in}{0.739656in}}%
\pgfpathlineto{\pgfqpoint{1.102461in}{0.739656in}}%
\pgfpathlineto{\pgfqpoint{1.101897in}{0.739656in}}%
\pgfpathlineto{\pgfqpoint{1.101332in}{0.739656in}}%
\pgfpathlineto{\pgfqpoint{1.100768in}{0.739656in}}%
\pgfpathlineto{\pgfqpoint{1.100204in}{0.739656in}}%
\pgfpathlineto{\pgfqpoint{1.099639in}{0.739656in}}%
\pgfpathlineto{\pgfqpoint{1.099075in}{0.739656in}}%
\pgfpathlineto{\pgfqpoint{1.098511in}{0.739656in}}%
\pgfpathlineto{\pgfqpoint{1.097946in}{0.739656in}}%
\pgfpathlineto{\pgfqpoint{1.097382in}{0.739656in}}%
\pgfpathlineto{\pgfqpoint{1.096818in}{0.739656in}}%
\pgfpathlineto{\pgfqpoint{1.096253in}{0.739656in}}%
\pgfpathlineto{\pgfqpoint{1.095689in}{0.739656in}}%
\pgfpathlineto{\pgfqpoint{1.095125in}{0.739656in}}%
\pgfpathlineto{\pgfqpoint{1.094560in}{0.739656in}}%
\pgfpathlineto{\pgfqpoint{1.093996in}{0.739656in}}%
\pgfpathlineto{\pgfqpoint{1.093432in}{0.739656in}}%
\pgfpathlineto{\pgfqpoint{1.092867in}{0.739656in}}%
\pgfpathlineto{\pgfqpoint{1.092303in}{0.739656in}}%
\pgfpathlineto{\pgfqpoint{1.091739in}{0.739656in}}%
\pgfpathlineto{\pgfqpoint{1.091174in}{0.739656in}}%
\pgfpathlineto{\pgfqpoint{1.090610in}{0.739656in}}%
\pgfpathlineto{\pgfqpoint{1.090046in}{0.739656in}}%
\pgfpathlineto{\pgfqpoint{1.089481in}{0.739656in}}%
\pgfpathlineto{\pgfqpoint{1.088917in}{0.739656in}}%
\pgfpathlineto{\pgfqpoint{1.088352in}{0.739656in}}%
\pgfpathlineto{\pgfqpoint{1.087788in}{0.739656in}}%
\pgfpathlineto{\pgfqpoint{1.087224in}{0.739656in}}%
\pgfpathlineto{\pgfqpoint{1.086659in}{0.739656in}}%
\pgfpathlineto{\pgfqpoint{1.086095in}{0.739656in}}%
\pgfpathlineto{\pgfqpoint{1.085531in}{0.739656in}}%
\pgfpathlineto{\pgfqpoint{1.084966in}{0.739656in}}%
\pgfpathlineto{\pgfqpoint{1.084402in}{0.739656in}}%
\pgfpathlineto{\pgfqpoint{1.083838in}{0.739656in}}%
\pgfpathlineto{\pgfqpoint{1.083273in}{0.739656in}}%
\pgfpathlineto{\pgfqpoint{1.082709in}{0.739656in}}%
\pgfpathlineto{\pgfqpoint{1.082145in}{0.739656in}}%
\pgfpathlineto{\pgfqpoint{1.081580in}{0.739656in}}%
\pgfpathlineto{\pgfqpoint{1.081016in}{0.739656in}}%
\pgfpathlineto{\pgfqpoint{1.080452in}{0.739656in}}%
\pgfpathlineto{\pgfqpoint{1.079887in}{0.739656in}}%
\pgfpathlineto{\pgfqpoint{1.079323in}{0.739656in}}%
\pgfpathlineto{\pgfqpoint{1.078759in}{0.739656in}}%
\pgfpathlineto{\pgfqpoint{1.078194in}{0.739656in}}%
\pgfpathlineto{\pgfqpoint{1.077630in}{0.739656in}}%
\pgfpathlineto{\pgfqpoint{1.077066in}{0.739656in}}%
\pgfpathlineto{\pgfqpoint{1.076501in}{0.739656in}}%
\pgfpathlineto{\pgfqpoint{1.075937in}{0.739656in}}%
\pgfpathlineto{\pgfqpoint{1.075373in}{0.739656in}}%
\pgfpathlineto{\pgfqpoint{1.074808in}{0.739656in}}%
\pgfpathlineto{\pgfqpoint{1.074244in}{0.739656in}}%
\pgfpathlineto{\pgfqpoint{1.073679in}{0.739656in}}%
\pgfpathlineto{\pgfqpoint{1.073115in}{0.739656in}}%
\pgfpathlineto{\pgfqpoint{1.072551in}{0.739656in}}%
\pgfpathlineto{\pgfqpoint{1.071986in}{0.739656in}}%
\pgfpathlineto{\pgfqpoint{1.071422in}{0.739656in}}%
\pgfpathlineto{\pgfqpoint{1.070858in}{0.739656in}}%
\pgfpathlineto{\pgfqpoint{1.070293in}{0.739656in}}%
\pgfpathlineto{\pgfqpoint{1.069729in}{0.739656in}}%
\pgfpathlineto{\pgfqpoint{1.069165in}{0.739656in}}%
\pgfpathlineto{\pgfqpoint{1.068600in}{0.739656in}}%
\pgfpathlineto{\pgfqpoint{1.068036in}{0.739656in}}%
\pgfpathlineto{\pgfqpoint{1.067472in}{0.739656in}}%
\pgfpathlineto{\pgfqpoint{1.066907in}{0.739656in}}%
\pgfpathlineto{\pgfqpoint{1.066343in}{0.739656in}}%
\pgfpathlineto{\pgfqpoint{1.065779in}{0.739656in}}%
\pgfpathlineto{\pgfqpoint{1.065214in}{0.739656in}}%
\pgfpathlineto{\pgfqpoint{1.064650in}{0.739656in}}%
\pgfpathlineto{\pgfqpoint{1.064086in}{0.739656in}}%
\pgfpathlineto{\pgfqpoint{1.063521in}{0.739656in}}%
\pgfpathlineto{\pgfqpoint{1.062957in}{0.739656in}}%
\pgfpathlineto{\pgfqpoint{1.062393in}{0.739656in}}%
\pgfpathlineto{\pgfqpoint{1.061828in}{0.739656in}}%
\pgfpathlineto{\pgfqpoint{1.061264in}{0.739656in}}%
\pgfpathlineto{\pgfqpoint{1.060700in}{0.739656in}}%
\pgfpathlineto{\pgfqpoint{1.060135in}{0.739656in}}%
\pgfpathlineto{\pgfqpoint{1.059571in}{0.739656in}}%
\pgfpathlineto{\pgfqpoint{1.059006in}{0.739656in}}%
\pgfpathlineto{\pgfqpoint{1.058442in}{0.739656in}}%
\pgfpathlineto{\pgfqpoint{1.057878in}{0.739656in}}%
\pgfpathlineto{\pgfqpoint{1.057313in}{0.739656in}}%
\pgfpathlineto{\pgfqpoint{1.056749in}{0.739656in}}%
\pgfpathlineto{\pgfqpoint{1.056185in}{0.739656in}}%
\pgfpathlineto{\pgfqpoint{1.055620in}{0.739656in}}%
\pgfpathlineto{\pgfqpoint{1.055056in}{0.739656in}}%
\pgfpathlineto{\pgfqpoint{1.054492in}{0.739656in}}%
\pgfpathlineto{\pgfqpoint{1.053927in}{0.739656in}}%
\pgfpathlineto{\pgfqpoint{1.053363in}{0.739656in}}%
\pgfpathlineto{\pgfqpoint{1.052799in}{0.739656in}}%
\pgfpathlineto{\pgfqpoint{1.052234in}{0.739656in}}%
\pgfpathlineto{\pgfqpoint{1.051670in}{0.739656in}}%
\pgfpathlineto{\pgfqpoint{1.051106in}{0.739656in}}%
\pgfpathlineto{\pgfqpoint{1.050541in}{0.739656in}}%
\pgfpathlineto{\pgfqpoint{1.049977in}{0.739656in}}%
\pgfpathlineto{\pgfqpoint{1.049413in}{0.739656in}}%
\pgfpathlineto{\pgfqpoint{1.048848in}{0.739656in}}%
\pgfpathlineto{\pgfqpoint{1.048284in}{0.739656in}}%
\pgfpathlineto{\pgfqpoint{1.047720in}{0.739656in}}%
\pgfpathlineto{\pgfqpoint{1.047155in}{0.739656in}}%
\pgfpathlineto{\pgfqpoint{1.046591in}{0.739656in}}%
\pgfpathlineto{\pgfqpoint{1.046027in}{0.739656in}}%
\pgfpathlineto{\pgfqpoint{1.045462in}{0.739656in}}%
\pgfpathlineto{\pgfqpoint{1.044898in}{0.739656in}}%
\pgfpathlineto{\pgfqpoint{1.044334in}{0.739656in}}%
\pgfpathlineto{\pgfqpoint{1.043769in}{0.739656in}}%
\pgfpathlineto{\pgfqpoint{1.043205in}{0.739656in}}%
\pgfpathlineto{\pgfqpoint{1.042640in}{0.739656in}}%
\pgfpathlineto{\pgfqpoint{1.042076in}{0.739656in}}%
\pgfpathlineto{\pgfqpoint{1.041512in}{0.739656in}}%
\pgfpathlineto{\pgfqpoint{1.040947in}{0.739656in}}%
\pgfpathlineto{\pgfqpoint{1.040383in}{0.739656in}}%
\pgfpathlineto{\pgfqpoint{1.039819in}{0.739656in}}%
\pgfpathlineto{\pgfqpoint{1.039254in}{0.739656in}}%
\pgfpathlineto{\pgfqpoint{1.038690in}{0.739656in}}%
\pgfpathlineto{\pgfqpoint{1.038126in}{0.739656in}}%
\pgfpathlineto{\pgfqpoint{1.037561in}{0.739656in}}%
\pgfpathlineto{\pgfqpoint{1.036997in}{0.739656in}}%
\pgfpathlineto{\pgfqpoint{1.036433in}{0.739656in}}%
\pgfpathlineto{\pgfqpoint{1.035868in}{0.739656in}}%
\pgfpathlineto{\pgfqpoint{1.035304in}{0.739656in}}%
\pgfpathlineto{\pgfqpoint{1.034740in}{0.739656in}}%
\pgfpathlineto{\pgfqpoint{1.034175in}{0.739656in}}%
\pgfpathlineto{\pgfqpoint{1.033611in}{0.739656in}}%
\pgfpathlineto{\pgfqpoint{1.033047in}{0.739656in}}%
\pgfpathlineto{\pgfqpoint{1.032482in}{0.739656in}}%
\pgfpathlineto{\pgfqpoint{1.031918in}{0.739656in}}%
\pgfpathlineto{\pgfqpoint{1.031354in}{0.739656in}}%
\pgfpathlineto{\pgfqpoint{1.030789in}{0.739656in}}%
\pgfpathlineto{\pgfqpoint{1.030225in}{0.739656in}}%
\pgfpathlineto{\pgfqpoint{1.029661in}{0.739656in}}%
\pgfpathlineto{\pgfqpoint{1.029096in}{0.739656in}}%
\pgfpathlineto{\pgfqpoint{1.028532in}{0.739656in}}%
\pgfpathlineto{\pgfqpoint{1.027967in}{0.739656in}}%
\pgfpathlineto{\pgfqpoint{1.027403in}{0.739656in}}%
\pgfpathlineto{\pgfqpoint{1.026839in}{0.739656in}}%
\pgfpathlineto{\pgfqpoint{1.026274in}{0.739656in}}%
\pgfpathlineto{\pgfqpoint{1.025710in}{0.739656in}}%
\pgfpathlineto{\pgfqpoint{1.025146in}{0.739656in}}%
\pgfpathlineto{\pgfqpoint{1.024581in}{0.739656in}}%
\pgfpathlineto{\pgfqpoint{1.024017in}{0.739656in}}%
\pgfpathlineto{\pgfqpoint{1.023453in}{0.739656in}}%
\pgfpathlineto{\pgfqpoint{1.022888in}{0.739656in}}%
\pgfpathlineto{\pgfqpoint{1.022324in}{0.739656in}}%
\pgfpathlineto{\pgfqpoint{1.021760in}{0.739656in}}%
\pgfpathlineto{\pgfqpoint{1.021195in}{0.739656in}}%
\pgfpathlineto{\pgfqpoint{1.020631in}{0.739656in}}%
\pgfpathlineto{\pgfqpoint{1.020067in}{0.739656in}}%
\pgfpathlineto{\pgfqpoint{1.019502in}{0.739656in}}%
\pgfpathlineto{\pgfqpoint{1.018938in}{0.739656in}}%
\pgfpathlineto{\pgfqpoint{1.018374in}{0.739656in}}%
\pgfpathlineto{\pgfqpoint{1.017809in}{0.739656in}}%
\pgfpathlineto{\pgfqpoint{1.017245in}{0.739656in}}%
\pgfpathlineto{\pgfqpoint{1.016681in}{0.739656in}}%
\pgfpathlineto{\pgfqpoint{1.016116in}{0.739656in}}%
\pgfpathlineto{\pgfqpoint{1.015552in}{0.739656in}}%
\pgfpathlineto{\pgfqpoint{1.014988in}{0.739656in}}%
\pgfpathlineto{\pgfqpoint{1.014423in}{0.739656in}}%
\pgfpathlineto{\pgfqpoint{1.013859in}{0.739656in}}%
\pgfpathlineto{\pgfqpoint{1.013294in}{0.739656in}}%
\pgfpathlineto{\pgfqpoint{1.012730in}{0.739656in}}%
\pgfpathlineto{\pgfqpoint{1.012166in}{0.739656in}}%
\pgfpathlineto{\pgfqpoint{1.011601in}{0.739656in}}%
\pgfpathlineto{\pgfqpoint{1.011037in}{0.739656in}}%
\pgfpathlineto{\pgfqpoint{1.010473in}{0.739656in}}%
\pgfpathlineto{\pgfqpoint{1.009908in}{0.739656in}}%
\pgfpathlineto{\pgfqpoint{1.009344in}{0.739656in}}%
\pgfpathlineto{\pgfqpoint{1.008780in}{0.739656in}}%
\pgfpathlineto{\pgfqpoint{1.008215in}{0.739656in}}%
\pgfpathlineto{\pgfqpoint{1.007651in}{0.739656in}}%
\pgfpathlineto{\pgfqpoint{1.007087in}{0.739656in}}%
\pgfpathlineto{\pgfqpoint{1.006522in}{0.739656in}}%
\pgfpathlineto{\pgfqpoint{1.005958in}{0.739656in}}%
\pgfpathlineto{\pgfqpoint{1.005394in}{0.739656in}}%
\pgfpathlineto{\pgfqpoint{1.004829in}{0.739656in}}%
\pgfpathlineto{\pgfqpoint{1.004265in}{0.739656in}}%
\pgfpathlineto{\pgfqpoint{1.003701in}{0.739656in}}%
\pgfpathlineto{\pgfqpoint{1.003136in}{0.739656in}}%
\pgfpathlineto{\pgfqpoint{1.002572in}{0.739656in}}%
\pgfpathlineto{\pgfqpoint{1.002008in}{0.739656in}}%
\pgfpathlineto{\pgfqpoint{1.001443in}{0.739656in}}%
\pgfpathlineto{\pgfqpoint{1.000879in}{0.739656in}}%
\pgfpathlineto{\pgfqpoint{1.000315in}{0.739656in}}%
\pgfpathlineto{\pgfqpoint{0.999750in}{0.739656in}}%
\pgfpathlineto{\pgfqpoint{0.999186in}{0.739656in}}%
\pgfpathlineto{\pgfqpoint{0.998622in}{0.739656in}}%
\pgfpathlineto{\pgfqpoint{0.998057in}{0.739656in}}%
\pgfpathlineto{\pgfqpoint{0.997493in}{0.739656in}}%
\pgfpathlineto{\pgfqpoint{0.996928in}{0.739656in}}%
\pgfpathlineto{\pgfqpoint{0.996364in}{0.739656in}}%
\pgfpathlineto{\pgfqpoint{0.995800in}{0.739656in}}%
\pgfpathlineto{\pgfqpoint{0.995235in}{0.739656in}}%
\pgfpathlineto{\pgfqpoint{0.994671in}{0.739656in}}%
\pgfpathlineto{\pgfqpoint{0.994107in}{0.739656in}}%
\pgfpathlineto{\pgfqpoint{0.993542in}{0.739656in}}%
\pgfpathlineto{\pgfqpoint{0.992978in}{0.739656in}}%
\pgfpathlineto{\pgfqpoint{0.992414in}{0.739656in}}%
\pgfpathlineto{\pgfqpoint{0.991849in}{0.739656in}}%
\pgfpathlineto{\pgfqpoint{0.991285in}{0.739656in}}%
\pgfpathlineto{\pgfqpoint{0.990721in}{0.739656in}}%
\pgfpathlineto{\pgfqpoint{0.990156in}{0.739656in}}%
\pgfpathlineto{\pgfqpoint{0.989592in}{0.739656in}}%
\pgfpathlineto{\pgfqpoint{0.989028in}{0.739656in}}%
\pgfpathlineto{\pgfqpoint{0.988463in}{0.739656in}}%
\pgfpathlineto{\pgfqpoint{0.987899in}{0.739656in}}%
\pgfpathlineto{\pgfqpoint{0.987335in}{0.739656in}}%
\pgfpathlineto{\pgfqpoint{0.986770in}{0.739656in}}%
\pgfpathlineto{\pgfqpoint{0.986206in}{0.739656in}}%
\pgfpathlineto{\pgfqpoint{0.985642in}{0.739656in}}%
\pgfpathlineto{\pgfqpoint{0.985077in}{0.739656in}}%
\pgfpathlineto{\pgfqpoint{0.984513in}{0.739656in}}%
\pgfpathlineto{\pgfqpoint{0.983949in}{0.739656in}}%
\pgfpathlineto{\pgfqpoint{0.983384in}{0.739656in}}%
\pgfpathlineto{\pgfqpoint{0.982820in}{0.739656in}}%
\pgfpathlineto{\pgfqpoint{0.982255in}{0.739656in}}%
\pgfpathlineto{\pgfqpoint{0.981691in}{0.739656in}}%
\pgfpathlineto{\pgfqpoint{0.981127in}{0.739656in}}%
\pgfpathlineto{\pgfqpoint{0.980562in}{0.739656in}}%
\pgfpathlineto{\pgfqpoint{0.979998in}{0.739656in}}%
\pgfpathlineto{\pgfqpoint{0.979434in}{0.739656in}}%
\pgfpathlineto{\pgfqpoint{0.978869in}{0.739656in}}%
\pgfpathlineto{\pgfqpoint{0.978305in}{0.739656in}}%
\pgfpathlineto{\pgfqpoint{0.977741in}{0.739656in}}%
\pgfpathlineto{\pgfqpoint{0.977176in}{0.739656in}}%
\pgfpathlineto{\pgfqpoint{0.976612in}{0.739656in}}%
\pgfpathlineto{\pgfqpoint{0.976048in}{0.739656in}}%
\pgfpathlineto{\pgfqpoint{0.975483in}{0.739656in}}%
\pgfpathlineto{\pgfqpoint{0.974919in}{0.739656in}}%
\pgfpathlineto{\pgfqpoint{0.974355in}{0.739656in}}%
\pgfpathlineto{\pgfqpoint{0.973790in}{0.739656in}}%
\pgfpathlineto{\pgfqpoint{0.973226in}{0.739656in}}%
\pgfpathlineto{\pgfqpoint{0.972662in}{0.739656in}}%
\pgfpathlineto{\pgfqpoint{0.972097in}{0.739656in}}%
\pgfpathlineto{\pgfqpoint{0.971533in}{0.739656in}}%
\pgfpathlineto{\pgfqpoint{0.970969in}{0.739656in}}%
\pgfpathlineto{\pgfqpoint{0.970404in}{0.739656in}}%
\pgfpathlineto{\pgfqpoint{0.969840in}{0.739656in}}%
\pgfpathlineto{\pgfqpoint{0.969276in}{0.739656in}}%
\pgfpathlineto{\pgfqpoint{0.968711in}{0.739656in}}%
\pgfpathlineto{\pgfqpoint{0.968147in}{0.739656in}}%
\pgfpathlineto{\pgfqpoint{0.967582in}{0.739656in}}%
\pgfpathlineto{\pgfqpoint{0.967018in}{0.739656in}}%
\pgfpathlineto{\pgfqpoint{0.966454in}{0.739656in}}%
\pgfpathlineto{\pgfqpoint{0.965889in}{0.739656in}}%
\pgfpathlineto{\pgfqpoint{0.965325in}{0.739656in}}%
\pgfpathlineto{\pgfqpoint{0.964761in}{0.739656in}}%
\pgfpathlineto{\pgfqpoint{0.964196in}{0.739656in}}%
\pgfpathlineto{\pgfqpoint{0.963632in}{0.739656in}}%
\pgfpathlineto{\pgfqpoint{0.963068in}{0.739656in}}%
\pgfpathlineto{\pgfqpoint{0.962503in}{0.739656in}}%
\pgfpathlineto{\pgfqpoint{0.961939in}{0.739656in}}%
\pgfpathlineto{\pgfqpoint{0.961375in}{0.739656in}}%
\pgfpathlineto{\pgfqpoint{0.960810in}{0.739656in}}%
\pgfpathlineto{\pgfqpoint{0.960246in}{0.739656in}}%
\pgfpathlineto{\pgfqpoint{0.959682in}{0.739656in}}%
\pgfpathlineto{\pgfqpoint{0.959117in}{0.739656in}}%
\pgfpathlineto{\pgfqpoint{0.958553in}{0.739656in}}%
\pgfpathlineto{\pgfqpoint{0.957989in}{0.739656in}}%
\pgfpathlineto{\pgfqpoint{0.957424in}{0.739656in}}%
\pgfpathlineto{\pgfqpoint{0.956860in}{0.739656in}}%
\pgfpathlineto{\pgfqpoint{0.956296in}{0.739656in}}%
\pgfpathlineto{\pgfqpoint{0.955731in}{0.739656in}}%
\pgfpathlineto{\pgfqpoint{0.955167in}{0.739656in}}%
\pgfpathlineto{\pgfqpoint{0.954603in}{0.739656in}}%
\pgfpathlineto{\pgfqpoint{0.954038in}{0.739656in}}%
\pgfpathlineto{\pgfqpoint{0.953474in}{0.739656in}}%
\pgfpathlineto{\pgfqpoint{0.952910in}{0.739656in}}%
\pgfpathlineto{\pgfqpoint{0.952345in}{0.739656in}}%
\pgfpathlineto{\pgfqpoint{0.951781in}{0.739656in}}%
\pgfpathlineto{\pgfqpoint{0.951216in}{0.739656in}}%
\pgfpathlineto{\pgfqpoint{0.950652in}{0.739656in}}%
\pgfpathlineto{\pgfqpoint{0.950088in}{0.739656in}}%
\pgfpathlineto{\pgfqpoint{0.949523in}{0.739656in}}%
\pgfpathlineto{\pgfqpoint{0.948959in}{0.739656in}}%
\pgfpathlineto{\pgfqpoint{0.948395in}{0.739656in}}%
\pgfpathlineto{\pgfqpoint{0.947830in}{0.739656in}}%
\pgfpathlineto{\pgfqpoint{0.947266in}{0.739656in}}%
\pgfpathlineto{\pgfqpoint{0.946702in}{0.739656in}}%
\pgfpathlineto{\pgfqpoint{0.946137in}{0.739656in}}%
\pgfpathlineto{\pgfqpoint{0.945573in}{0.739656in}}%
\pgfpathlineto{\pgfqpoint{0.945009in}{0.739656in}}%
\pgfpathlineto{\pgfqpoint{0.944444in}{0.739656in}}%
\pgfpathlineto{\pgfqpoint{0.943880in}{0.739656in}}%
\pgfpathlineto{\pgfqpoint{0.943316in}{0.739656in}}%
\pgfpathlineto{\pgfqpoint{0.942751in}{0.739656in}}%
\pgfpathlineto{\pgfqpoint{0.942187in}{0.739656in}}%
\pgfpathlineto{\pgfqpoint{0.941623in}{0.739656in}}%
\pgfpathlineto{\pgfqpoint{0.941058in}{0.739656in}}%
\pgfpathlineto{\pgfqpoint{0.940494in}{0.739656in}}%
\pgfpathlineto{\pgfqpoint{0.939930in}{0.739656in}}%
\pgfpathlineto{\pgfqpoint{0.939365in}{0.739656in}}%
\pgfpathlineto{\pgfqpoint{0.938801in}{0.739656in}}%
\pgfpathlineto{\pgfqpoint{0.938237in}{0.739656in}}%
\pgfpathlineto{\pgfqpoint{0.937672in}{0.739656in}}%
\pgfpathlineto{\pgfqpoint{0.937108in}{0.739656in}}%
\pgfpathlineto{\pgfqpoint{0.936543in}{0.739656in}}%
\pgfpathlineto{\pgfqpoint{0.935979in}{0.739656in}}%
\pgfpathlineto{\pgfqpoint{0.935415in}{0.739656in}}%
\pgfpathlineto{\pgfqpoint{0.934850in}{0.739656in}}%
\pgfpathlineto{\pgfqpoint{0.934286in}{0.739656in}}%
\pgfpathlineto{\pgfqpoint{0.933722in}{0.739656in}}%
\pgfpathlineto{\pgfqpoint{0.933157in}{0.739656in}}%
\pgfpathlineto{\pgfqpoint{0.932593in}{0.739656in}}%
\pgfpathlineto{\pgfqpoint{0.932029in}{0.739656in}}%
\pgfpathlineto{\pgfqpoint{0.931464in}{0.739656in}}%
\pgfpathlineto{\pgfqpoint{0.930900in}{0.739656in}}%
\pgfpathlineto{\pgfqpoint{0.930336in}{0.739656in}}%
\pgfpathlineto{\pgfqpoint{0.929771in}{0.739656in}}%
\pgfpathlineto{\pgfqpoint{0.929207in}{0.739656in}}%
\pgfpathlineto{\pgfqpoint{0.928643in}{0.739656in}}%
\pgfpathlineto{\pgfqpoint{0.928078in}{0.739656in}}%
\pgfpathlineto{\pgfqpoint{0.927514in}{0.739656in}}%
\pgfpathlineto{\pgfqpoint{0.926950in}{0.739656in}}%
\pgfpathlineto{\pgfqpoint{0.926385in}{0.739656in}}%
\pgfpathlineto{\pgfqpoint{0.925821in}{0.739656in}}%
\pgfpathlineto{\pgfqpoint{0.925257in}{0.739656in}}%
\pgfpathlineto{\pgfqpoint{0.924692in}{0.739656in}}%
\pgfpathlineto{\pgfqpoint{0.924128in}{0.739656in}}%
\pgfpathlineto{\pgfqpoint{0.923564in}{0.739656in}}%
\pgfpathlineto{\pgfqpoint{0.922999in}{0.739656in}}%
\pgfpathlineto{\pgfqpoint{0.922435in}{0.739656in}}%
\pgfpathlineto{\pgfqpoint{0.921870in}{0.739656in}}%
\pgfpathlineto{\pgfqpoint{0.921306in}{0.739656in}}%
\pgfpathlineto{\pgfqpoint{0.920742in}{0.739656in}}%
\pgfpathlineto{\pgfqpoint{0.920177in}{0.739656in}}%
\pgfpathlineto{\pgfqpoint{0.919613in}{0.739656in}}%
\pgfpathlineto{\pgfqpoint{0.919049in}{0.739656in}}%
\pgfpathlineto{\pgfqpoint{0.918484in}{0.739656in}}%
\pgfpathlineto{\pgfqpoint{0.917920in}{0.739656in}}%
\pgfpathlineto{\pgfqpoint{0.917356in}{0.739656in}}%
\pgfpathlineto{\pgfqpoint{0.916791in}{0.739656in}}%
\pgfpathlineto{\pgfqpoint{0.916227in}{0.739656in}}%
\pgfpathlineto{\pgfqpoint{0.915663in}{0.739656in}}%
\pgfpathlineto{\pgfqpoint{0.915098in}{0.739656in}}%
\pgfpathlineto{\pgfqpoint{0.914534in}{0.739656in}}%
\pgfpathlineto{\pgfqpoint{0.913970in}{0.739656in}}%
\pgfpathlineto{\pgfqpoint{0.913405in}{0.739656in}}%
\pgfpathlineto{\pgfqpoint{0.912841in}{0.739656in}}%
\pgfpathlineto{\pgfqpoint{0.912277in}{0.739656in}}%
\pgfpathlineto{\pgfqpoint{0.911712in}{0.739656in}}%
\pgfpathlineto{\pgfqpoint{0.911148in}{0.739656in}}%
\pgfpathlineto{\pgfqpoint{0.910584in}{0.739656in}}%
\pgfpathlineto{\pgfqpoint{0.910019in}{0.739656in}}%
\pgfpathlineto{\pgfqpoint{0.909455in}{0.739656in}}%
\pgfpathlineto{\pgfqpoint{0.908891in}{0.739656in}}%
\pgfpathlineto{\pgfqpoint{0.908326in}{0.739656in}}%
\pgfpathlineto{\pgfqpoint{0.907762in}{0.739656in}}%
\pgfpathlineto{\pgfqpoint{0.907197in}{0.739656in}}%
\pgfpathlineto{\pgfqpoint{0.906633in}{0.739656in}}%
\pgfpathlineto{\pgfqpoint{0.906069in}{0.739656in}}%
\pgfpathlineto{\pgfqpoint{0.905504in}{0.739656in}}%
\pgfpathlineto{\pgfqpoint{0.904940in}{0.739656in}}%
\pgfpathlineto{\pgfqpoint{0.904376in}{0.739656in}}%
\pgfpathlineto{\pgfqpoint{0.903811in}{0.739656in}}%
\pgfpathlineto{\pgfqpoint{0.903247in}{0.739656in}}%
\pgfpathlineto{\pgfqpoint{0.902683in}{0.739656in}}%
\pgfpathlineto{\pgfqpoint{0.902118in}{0.739656in}}%
\pgfpathlineto{\pgfqpoint{0.901554in}{0.739656in}}%
\pgfpathlineto{\pgfqpoint{0.900990in}{0.739656in}}%
\pgfpathlineto{\pgfqpoint{0.900425in}{0.739656in}}%
\pgfpathlineto{\pgfqpoint{0.899861in}{0.739656in}}%
\pgfpathlineto{\pgfqpoint{0.899297in}{0.739656in}}%
\pgfpathlineto{\pgfqpoint{0.898732in}{0.739656in}}%
\pgfpathlineto{\pgfqpoint{0.898168in}{0.739656in}}%
\pgfpathlineto{\pgfqpoint{0.897604in}{0.739656in}}%
\pgfpathlineto{\pgfqpoint{0.897039in}{0.739656in}}%
\pgfpathlineto{\pgfqpoint{0.896475in}{0.739656in}}%
\pgfpathlineto{\pgfqpoint{0.895911in}{0.739656in}}%
\pgfpathlineto{\pgfqpoint{0.895346in}{0.739656in}}%
\pgfpathlineto{\pgfqpoint{0.894782in}{0.739656in}}%
\pgfpathlineto{\pgfqpoint{0.894218in}{0.739656in}}%
\pgfpathlineto{\pgfqpoint{0.893653in}{0.739656in}}%
\pgfpathlineto{\pgfqpoint{0.893089in}{0.739656in}}%
\pgfpathlineto{\pgfqpoint{0.892525in}{0.739656in}}%
\pgfpathlineto{\pgfqpoint{0.891960in}{0.739656in}}%
\pgfpathlineto{\pgfqpoint{0.891396in}{0.739656in}}%
\pgfpathlineto{\pgfqpoint{0.890831in}{0.739656in}}%
\pgfpathlineto{\pgfqpoint{0.890267in}{0.739656in}}%
\pgfpathlineto{\pgfqpoint{0.889703in}{0.739656in}}%
\pgfpathlineto{\pgfqpoint{0.889138in}{0.739656in}}%
\pgfpathlineto{\pgfqpoint{0.888574in}{0.739656in}}%
\pgfpathlineto{\pgfqpoint{0.888010in}{0.739656in}}%
\pgfpathlineto{\pgfqpoint{0.887445in}{0.739656in}}%
\pgfpathlineto{\pgfqpoint{0.886881in}{0.739656in}}%
\pgfpathlineto{\pgfqpoint{0.886317in}{0.739656in}}%
\pgfpathlineto{\pgfqpoint{0.885752in}{0.739656in}}%
\pgfpathlineto{\pgfqpoint{0.885188in}{0.739656in}}%
\pgfpathlineto{\pgfqpoint{0.884624in}{0.739656in}}%
\pgfpathlineto{\pgfqpoint{0.884059in}{0.739656in}}%
\pgfpathlineto{\pgfqpoint{0.883495in}{0.739656in}}%
\pgfpathlineto{\pgfqpoint{0.882931in}{0.739656in}}%
\pgfpathlineto{\pgfqpoint{0.882366in}{0.739656in}}%
\pgfpathlineto{\pgfqpoint{0.881802in}{0.739656in}}%
\pgfpathlineto{\pgfqpoint{0.881238in}{0.739656in}}%
\pgfpathlineto{\pgfqpoint{0.880673in}{0.739656in}}%
\pgfpathlineto{\pgfqpoint{0.880109in}{0.739656in}}%
\pgfpathlineto{\pgfqpoint{0.879545in}{0.739656in}}%
\pgfpathlineto{\pgfqpoint{0.878980in}{0.739656in}}%
\pgfpathlineto{\pgfqpoint{0.878416in}{0.739656in}}%
\pgfpathlineto{\pgfqpoint{0.877852in}{0.739656in}}%
\pgfpathlineto{\pgfqpoint{0.877287in}{0.739656in}}%
\pgfpathlineto{\pgfqpoint{0.876723in}{0.739656in}}%
\pgfpathlineto{\pgfqpoint{0.876158in}{0.739656in}}%
\pgfpathlineto{\pgfqpoint{0.875594in}{0.739656in}}%
\pgfpathlineto{\pgfqpoint{0.875030in}{0.739656in}}%
\pgfpathlineto{\pgfqpoint{0.874465in}{0.739656in}}%
\pgfpathlineto{\pgfqpoint{0.873901in}{0.739656in}}%
\pgfpathlineto{\pgfqpoint{0.873337in}{0.739656in}}%
\pgfpathlineto{\pgfqpoint{0.872772in}{0.739656in}}%
\pgfpathlineto{\pgfqpoint{0.872208in}{0.739656in}}%
\pgfpathlineto{\pgfqpoint{0.871644in}{0.739656in}}%
\pgfpathlineto{\pgfqpoint{0.871079in}{0.739656in}}%
\pgfpathlineto{\pgfqpoint{0.870515in}{0.739656in}}%
\pgfpathlineto{\pgfqpoint{0.869951in}{0.739656in}}%
\pgfpathlineto{\pgfqpoint{0.869386in}{0.739656in}}%
\pgfpathlineto{\pgfqpoint{0.868822in}{0.739656in}}%
\pgfpathlineto{\pgfqpoint{0.868258in}{0.739656in}}%
\pgfpathlineto{\pgfqpoint{0.867693in}{0.739656in}}%
\pgfpathlineto{\pgfqpoint{0.867129in}{0.739656in}}%
\pgfpathlineto{\pgfqpoint{0.866565in}{0.739656in}}%
\pgfpathlineto{\pgfqpoint{0.866000in}{0.739656in}}%
\pgfpathlineto{\pgfqpoint{0.865436in}{0.739656in}}%
\pgfpathlineto{\pgfqpoint{0.864872in}{0.739656in}}%
\pgfpathlineto{\pgfqpoint{0.864307in}{0.739656in}}%
\pgfpathlineto{\pgfqpoint{0.863743in}{0.739656in}}%
\pgfpathlineto{\pgfqpoint{0.863179in}{0.739656in}}%
\pgfpathlineto{\pgfqpoint{0.862614in}{0.739656in}}%
\pgfpathlineto{\pgfqpoint{0.862050in}{0.739656in}}%
\pgfpathlineto{\pgfqpoint{0.861485in}{0.739656in}}%
\pgfpathlineto{\pgfqpoint{0.860921in}{0.739656in}}%
\pgfpathlineto{\pgfqpoint{0.860357in}{0.739656in}}%
\pgfpathlineto{\pgfqpoint{0.859792in}{0.739656in}}%
\pgfpathlineto{\pgfqpoint{0.859228in}{0.739656in}}%
\pgfpathlineto{\pgfqpoint{0.858664in}{0.739656in}}%
\pgfpathlineto{\pgfqpoint{0.858099in}{0.739656in}}%
\pgfpathlineto{\pgfqpoint{0.857535in}{0.739656in}}%
\pgfpathlineto{\pgfqpoint{0.856971in}{0.739656in}}%
\pgfpathlineto{\pgfqpoint{0.856406in}{0.739656in}}%
\pgfpathlineto{\pgfqpoint{0.855842in}{0.739656in}}%
\pgfpathlineto{\pgfqpoint{0.855278in}{0.739656in}}%
\pgfpathlineto{\pgfqpoint{0.854713in}{0.739656in}}%
\pgfpathlineto{\pgfqpoint{0.854149in}{0.739656in}}%
\pgfpathlineto{\pgfqpoint{0.853585in}{0.739656in}}%
\pgfpathlineto{\pgfqpoint{0.853020in}{0.739656in}}%
\pgfpathlineto{\pgfqpoint{0.852456in}{0.739656in}}%
\pgfpathlineto{\pgfqpoint{0.851892in}{0.739656in}}%
\pgfpathlineto{\pgfqpoint{0.851327in}{0.739656in}}%
\pgfpathlineto{\pgfqpoint{0.850763in}{0.739656in}}%
\pgfpathlineto{\pgfqpoint{0.850199in}{0.739656in}}%
\pgfpathlineto{\pgfqpoint{0.849634in}{0.739656in}}%
\pgfpathlineto{\pgfqpoint{0.849070in}{0.739656in}}%
\pgfpathlineto{\pgfqpoint{0.848506in}{0.739656in}}%
\pgfpathlineto{\pgfqpoint{0.847941in}{0.739656in}}%
\pgfpathlineto{\pgfqpoint{0.847377in}{0.739656in}}%
\pgfpathlineto{\pgfqpoint{0.846813in}{0.739656in}}%
\pgfpathlineto{\pgfqpoint{0.846248in}{0.739656in}}%
\pgfpathlineto{\pgfqpoint{0.845684in}{0.739656in}}%
\pgfpathlineto{\pgfqpoint{0.845119in}{0.739656in}}%
\pgfpathlineto{\pgfqpoint{0.844555in}{0.739656in}}%
\pgfpathlineto{\pgfqpoint{0.843991in}{0.739656in}}%
\pgfpathlineto{\pgfqpoint{0.843426in}{0.739656in}}%
\pgfpathlineto{\pgfqpoint{0.842862in}{0.739656in}}%
\pgfpathlineto{\pgfqpoint{0.842298in}{0.739656in}}%
\pgfpathlineto{\pgfqpoint{0.841733in}{0.739656in}}%
\pgfpathlineto{\pgfqpoint{0.841169in}{0.739656in}}%
\pgfpathlineto{\pgfqpoint{0.840605in}{0.739656in}}%
\pgfpathlineto{\pgfqpoint{0.840040in}{0.739656in}}%
\pgfpathlineto{\pgfqpoint{0.839476in}{0.739656in}}%
\pgfpathlineto{\pgfqpoint{0.838912in}{0.739656in}}%
\pgfpathlineto{\pgfqpoint{0.838347in}{0.739656in}}%
\pgfpathlineto{\pgfqpoint{0.837783in}{0.739656in}}%
\pgfpathlineto{\pgfqpoint{0.837219in}{0.739656in}}%
\pgfpathlineto{\pgfqpoint{0.836654in}{0.739656in}}%
\pgfpathlineto{\pgfqpoint{0.836090in}{0.739656in}}%
\pgfpathlineto{\pgfqpoint{0.835526in}{0.739656in}}%
\pgfpathlineto{\pgfqpoint{0.834961in}{0.739656in}}%
\pgfpathlineto{\pgfqpoint{0.834397in}{0.739656in}}%
\pgfpathlineto{\pgfqpoint{0.833833in}{0.739656in}}%
\pgfpathlineto{\pgfqpoint{0.833268in}{0.739656in}}%
\pgfpathlineto{\pgfqpoint{0.832704in}{0.739656in}}%
\pgfpathlineto{\pgfqpoint{0.832140in}{0.739656in}}%
\pgfpathlineto{\pgfqpoint{0.831575in}{0.739656in}}%
\pgfpathlineto{\pgfqpoint{0.831011in}{0.739656in}}%
\pgfpathlineto{\pgfqpoint{0.830446in}{0.739656in}}%
\pgfpathlineto{\pgfqpoint{0.829882in}{0.739656in}}%
\pgfpathlineto{\pgfqpoint{0.829318in}{0.739656in}}%
\pgfpathlineto{\pgfqpoint{0.828753in}{0.739656in}}%
\pgfpathlineto{\pgfqpoint{0.828189in}{0.739656in}}%
\pgfpathlineto{\pgfqpoint{0.827625in}{0.739656in}}%
\pgfpathlineto{\pgfqpoint{0.827060in}{0.739656in}}%
\pgfpathlineto{\pgfqpoint{0.826496in}{0.739656in}}%
\pgfpathlineto{\pgfqpoint{0.825932in}{0.739656in}}%
\pgfpathlineto{\pgfqpoint{0.825367in}{0.739656in}}%
\pgfpathlineto{\pgfqpoint{0.824803in}{0.739656in}}%
\pgfpathlineto{\pgfqpoint{0.824239in}{0.739656in}}%
\pgfpathlineto{\pgfqpoint{0.823674in}{0.739656in}}%
\pgfpathlineto{\pgfqpoint{0.823110in}{0.739656in}}%
\pgfpathlineto{\pgfqpoint{0.822546in}{0.739656in}}%
\pgfpathlineto{\pgfqpoint{0.821981in}{0.739656in}}%
\pgfpathlineto{\pgfqpoint{0.821417in}{0.739656in}}%
\pgfpathlineto{\pgfqpoint{0.820853in}{0.739656in}}%
\pgfpathlineto{\pgfqpoint{0.820288in}{0.739656in}}%
\pgfpathlineto{\pgfqpoint{0.819724in}{0.739656in}}%
\pgfpathlineto{\pgfqpoint{0.819160in}{0.739656in}}%
\pgfpathlineto{\pgfqpoint{0.818595in}{0.739656in}}%
\pgfpathlineto{\pgfqpoint{0.818031in}{0.739656in}}%
\pgfpathlineto{\pgfqpoint{0.817467in}{0.739656in}}%
\pgfpathlineto{\pgfqpoint{0.816902in}{0.739656in}}%
\pgfpathlineto{\pgfqpoint{0.816338in}{0.739656in}}%
\pgfpathlineto{\pgfqpoint{0.815773in}{0.739656in}}%
\pgfpathlineto{\pgfqpoint{0.815209in}{0.739656in}}%
\pgfpathlineto{\pgfqpoint{0.814645in}{0.739656in}}%
\pgfpathlineto{\pgfqpoint{0.814080in}{0.739656in}}%
\pgfpathlineto{\pgfqpoint{0.813516in}{0.739656in}}%
\pgfpathlineto{\pgfqpoint{0.812952in}{0.739656in}}%
\pgfpathlineto{\pgfqpoint{0.812387in}{0.739656in}}%
\pgfpathlineto{\pgfqpoint{0.811823in}{0.739656in}}%
\pgfpathlineto{\pgfqpoint{0.811259in}{0.739656in}}%
\pgfpathlineto{\pgfqpoint{0.810694in}{0.739656in}}%
\pgfpathlineto{\pgfqpoint{0.810130in}{0.739656in}}%
\pgfpathlineto{\pgfqpoint{0.809566in}{0.739656in}}%
\pgfpathlineto{\pgfqpoint{0.809001in}{0.739656in}}%
\pgfpathlineto{\pgfqpoint{0.808437in}{0.739656in}}%
\pgfpathlineto{\pgfqpoint{0.807873in}{0.739656in}}%
\pgfpathlineto{\pgfqpoint{0.807308in}{0.739656in}}%
\pgfpathlineto{\pgfqpoint{0.806744in}{0.739656in}}%
\pgfpathlineto{\pgfqpoint{0.806180in}{0.739656in}}%
\pgfpathlineto{\pgfqpoint{0.805615in}{0.739656in}}%
\pgfpathlineto{\pgfqpoint{0.805051in}{0.739656in}}%
\pgfpathlineto{\pgfqpoint{0.804487in}{0.739656in}}%
\pgfpathlineto{\pgfqpoint{0.803922in}{0.739656in}}%
\pgfpathlineto{\pgfqpoint{0.803358in}{0.739656in}}%
\pgfpathlineto{\pgfqpoint{0.802794in}{0.739656in}}%
\pgfpathlineto{\pgfqpoint{0.802229in}{0.739656in}}%
\pgfpathlineto{\pgfqpoint{0.801665in}{0.739656in}}%
\pgfpathlineto{\pgfqpoint{0.801101in}{0.739656in}}%
\pgfpathlineto{\pgfqpoint{0.800536in}{0.739656in}}%
\pgfpathlineto{\pgfqpoint{0.799972in}{0.739656in}}%
\pgfpathlineto{\pgfqpoint{0.799407in}{0.739656in}}%
\pgfpathlineto{\pgfqpoint{0.798843in}{0.739656in}}%
\pgfpathlineto{\pgfqpoint{0.798279in}{0.739656in}}%
\pgfpathlineto{\pgfqpoint{0.797714in}{0.739656in}}%
\pgfpathlineto{\pgfqpoint{0.797150in}{0.739656in}}%
\pgfpathlineto{\pgfqpoint{0.796586in}{0.739656in}}%
\pgfpathlineto{\pgfqpoint{0.796021in}{0.739656in}}%
\pgfpathlineto{\pgfqpoint{0.795457in}{0.739656in}}%
\pgfpathlineto{\pgfqpoint{0.794893in}{0.739656in}}%
\pgfpathlineto{\pgfqpoint{0.794328in}{0.739656in}}%
\pgfpathlineto{\pgfqpoint{0.793764in}{0.739656in}}%
\pgfpathlineto{\pgfqpoint{0.793200in}{0.739656in}}%
\pgfpathlineto{\pgfqpoint{0.792635in}{0.739656in}}%
\pgfpathlineto{\pgfqpoint{0.792071in}{0.739656in}}%
\pgfpathlineto{\pgfqpoint{0.791507in}{0.739656in}}%
\pgfpathlineto{\pgfqpoint{0.790942in}{0.739656in}}%
\pgfpathlineto{\pgfqpoint{0.790378in}{0.739656in}}%
\pgfpathlineto{\pgfqpoint{0.789814in}{0.739656in}}%
\pgfpathlineto{\pgfqpoint{0.789249in}{0.739656in}}%
\pgfpathlineto{\pgfqpoint{0.788685in}{0.739656in}}%
\pgfpathlineto{\pgfqpoint{0.788121in}{0.739656in}}%
\pgfpathlineto{\pgfqpoint{0.787556in}{0.739656in}}%
\pgfpathlineto{\pgfqpoint{0.786992in}{0.739656in}}%
\pgfpathlineto{\pgfqpoint{0.786428in}{0.739656in}}%
\pgfpathlineto{\pgfqpoint{0.785863in}{0.739656in}}%
\pgfpathlineto{\pgfqpoint{0.785299in}{0.739656in}}%
\pgfpathlineto{\pgfqpoint{0.784734in}{0.739656in}}%
\pgfpathlineto{\pgfqpoint{0.784170in}{0.739656in}}%
\pgfpathlineto{\pgfqpoint{0.783606in}{0.739656in}}%
\pgfpathlineto{\pgfqpoint{0.783041in}{0.739656in}}%
\pgfpathlineto{\pgfqpoint{0.782477in}{0.739656in}}%
\pgfpathlineto{\pgfqpoint{0.781913in}{0.739656in}}%
\pgfpathlineto{\pgfqpoint{0.781348in}{0.739656in}}%
\pgfpathlineto{\pgfqpoint{0.780784in}{0.739656in}}%
\pgfpathlineto{\pgfqpoint{0.780220in}{0.739656in}}%
\pgfpathlineto{\pgfqpoint{0.779655in}{0.739656in}}%
\pgfpathlineto{\pgfqpoint{0.779091in}{0.739656in}}%
\pgfpathlineto{\pgfqpoint{0.778527in}{0.739656in}}%
\pgfpathlineto{\pgfqpoint{0.777962in}{0.739656in}}%
\pgfpathlineto{\pgfqpoint{0.777398in}{0.739656in}}%
\pgfpathlineto{\pgfqpoint{0.776834in}{0.739656in}}%
\pgfpathlineto{\pgfqpoint{0.776269in}{0.739656in}}%
\pgfpathlineto{\pgfqpoint{0.775705in}{0.739656in}}%
\pgfpathlineto{\pgfqpoint{0.775141in}{0.739656in}}%
\pgfpathlineto{\pgfqpoint{0.774576in}{0.739656in}}%
\pgfpathlineto{\pgfqpoint{0.774012in}{0.739656in}}%
\pgfpathlineto{\pgfqpoint{0.773448in}{0.739656in}}%
\pgfpathlineto{\pgfqpoint{0.772883in}{0.739656in}}%
\pgfpathlineto{\pgfqpoint{0.772319in}{0.739656in}}%
\pgfpathlineto{\pgfqpoint{0.771755in}{0.739656in}}%
\pgfpathlineto{\pgfqpoint{0.771190in}{0.739656in}}%
\pgfpathlineto{\pgfqpoint{0.770626in}{0.739656in}}%
\pgfpathlineto{\pgfqpoint{0.770061in}{0.739656in}}%
\pgfpathlineto{\pgfqpoint{0.769497in}{0.739656in}}%
\pgfpathlineto{\pgfqpoint{0.768933in}{0.739656in}}%
\pgfpathlineto{\pgfqpoint{0.768368in}{0.739656in}}%
\pgfpathlineto{\pgfqpoint{0.767804in}{0.739656in}}%
\pgfpathlineto{\pgfqpoint{0.767240in}{0.739656in}}%
\pgfpathlineto{\pgfqpoint{0.766675in}{0.739656in}}%
\pgfpathlineto{\pgfqpoint{0.766111in}{0.739656in}}%
\pgfpathlineto{\pgfqpoint{0.765547in}{0.739656in}}%
\pgfpathlineto{\pgfqpoint{0.764982in}{0.739656in}}%
\pgfpathlineto{\pgfqpoint{0.764418in}{0.739656in}}%
\pgfpathlineto{\pgfqpoint{0.763854in}{0.739656in}}%
\pgfpathlineto{\pgfqpoint{0.763289in}{0.739656in}}%
\pgfpathlineto{\pgfqpoint{0.762725in}{0.739656in}}%
\pgfpathlineto{\pgfqpoint{0.762161in}{0.739656in}}%
\pgfpathlineto{\pgfqpoint{0.761596in}{0.739656in}}%
\pgfpathlineto{\pgfqpoint{0.761032in}{0.739656in}}%
\pgfpathlineto{\pgfqpoint{0.760468in}{0.739656in}}%
\pgfpathlineto{\pgfqpoint{0.759903in}{0.739656in}}%
\pgfpathlineto{\pgfqpoint{0.759339in}{0.739656in}}%
\pgfpathlineto{\pgfqpoint{0.758775in}{0.739656in}}%
\pgfpathlineto{\pgfqpoint{0.758210in}{0.739656in}}%
\pgfpathlineto{\pgfqpoint{0.757646in}{0.739656in}}%
\pgfpathlineto{\pgfqpoint{0.757082in}{0.739656in}}%
\pgfpathlineto{\pgfqpoint{0.756517in}{0.739656in}}%
\pgfpathlineto{\pgfqpoint{0.755953in}{0.739656in}}%
\pgfpathlineto{\pgfqpoint{0.755389in}{0.739656in}}%
\pgfpathlineto{\pgfqpoint{0.754824in}{0.739656in}}%
\pgfpathlineto{\pgfqpoint{0.754260in}{0.739656in}}%
\pgfpathlineto{\pgfqpoint{0.753695in}{0.739656in}}%
\pgfpathlineto{\pgfqpoint{0.753131in}{0.739656in}}%
\pgfpathlineto{\pgfqpoint{0.752567in}{0.739656in}}%
\pgfpathlineto{\pgfqpoint{0.752002in}{0.739656in}}%
\pgfpathlineto{\pgfqpoint{0.751438in}{0.739656in}}%
\pgfpathlineto{\pgfqpoint{0.750874in}{0.739656in}}%
\pgfpathlineto{\pgfqpoint{0.750309in}{0.739656in}}%
\pgfpathlineto{\pgfqpoint{0.749745in}{0.739656in}}%
\pgfpathlineto{\pgfqpoint{0.749181in}{0.739656in}}%
\pgfpathlineto{\pgfqpoint{0.748616in}{0.739656in}}%
\pgfpathlineto{\pgfqpoint{0.748052in}{0.739656in}}%
\pgfpathlineto{\pgfqpoint{0.747488in}{0.739656in}}%
\pgfpathlineto{\pgfqpoint{0.746923in}{0.739656in}}%
\pgfpathlineto{\pgfqpoint{0.746359in}{0.739656in}}%
\pgfpathlineto{\pgfqpoint{0.745795in}{0.739656in}}%
\pgfpathlineto{\pgfqpoint{0.745230in}{0.739656in}}%
\pgfpathlineto{\pgfqpoint{0.744666in}{0.739656in}}%
\pgfpathlineto{\pgfqpoint{0.744666in}{0.739656in}}%
\pgfpathclose%
\pgfusepath{fill}%
\end{pgfscope}%
\begin{pgfscope}%
\pgfsetbuttcap%
\pgfsetroundjoin%
\definecolor{currentfill}{rgb}{0.000000,0.000000,0.000000}%
\pgfsetfillcolor{currentfill}%
\pgfsetlinewidth{0.501875pt}%
\definecolor{currentstroke}{rgb}{0.000000,0.000000,0.000000}%
\pgfsetstrokecolor{currentstroke}%
\pgfsetdash{}{0pt}%
\pgfsys@defobject{currentmarker}{\pgfqpoint{0.000000in}{0.000000in}}{\pgfqpoint{0.000000in}{0.041667in}}{%
\pgfpathmoveto{\pgfqpoint{0.000000in}{0.000000in}}%
\pgfpathlineto{\pgfqpoint{0.000000in}{0.041667in}}%
\pgfusepath{stroke,fill}%
}%
\begin{pgfscope}%
\pgfsys@transformshift{0.767804in}{0.586309in}%
\pgfsys@useobject{currentmarker}{}%
\end{pgfscope}%
\end{pgfscope}%
\begin{pgfscope}%
\pgfsetbuttcap%
\pgfsetroundjoin%
\definecolor{currentfill}{rgb}{0.000000,0.000000,0.000000}%
\pgfsetfillcolor{currentfill}%
\pgfsetlinewidth{0.501875pt}%
\definecolor{currentstroke}{rgb}{0.000000,0.000000,0.000000}%
\pgfsetstrokecolor{currentstroke}%
\pgfsetdash{}{0pt}%
\pgfsys@defobject{currentmarker}{\pgfqpoint{0.000000in}{-0.041667in}}{\pgfqpoint{0.000000in}{0.000000in}}{%
\pgfpathmoveto{\pgfqpoint{0.000000in}{0.000000in}}%
\pgfpathlineto{\pgfqpoint{0.000000in}{-0.041667in}}%
\pgfusepath{stroke,fill}%
}%
\begin{pgfscope}%
\pgfsys@transformshift{0.767804in}{0.893003in}%
\pgfsys@useobject{currentmarker}{}%
\end{pgfscope}%
\end{pgfscope}%
\begin{pgfscope}%
\definecolor{textcolor}{rgb}{0.000000,0.000000,0.000000}%
\pgfsetstrokecolor{textcolor}%
\pgfsetfillcolor{textcolor}%
\pgftext[x=0.310277in, y=0.220556in, left, base,rotate=30.000000]{\color{textcolor}\rmfamily\fontsize{7.000000}{8.400000}\selectfont 2018-10-13}%
\end{pgfscope}%
\begin{pgfscope}%
\pgfsetbuttcap%
\pgfsetroundjoin%
\definecolor{currentfill}{rgb}{0.000000,0.000000,0.000000}%
\pgfsetfillcolor{currentfill}%
\pgfsetlinewidth{0.501875pt}%
\definecolor{currentstroke}{rgb}{0.000000,0.000000,0.000000}%
\pgfsetstrokecolor{currentstroke}%
\pgfsetdash{}{0pt}%
\pgfsys@defobject{currentmarker}{\pgfqpoint{0.000000in}{0.000000in}}{\pgfqpoint{0.000000in}{0.041667in}}{%
\pgfpathmoveto{\pgfqpoint{0.000000in}{0.000000in}}%
\pgfpathlineto{\pgfqpoint{0.000000in}{0.041667in}}%
\pgfusepath{stroke,fill}%
}%
\begin{pgfscope}%
\pgfsys@transformshift{1.174133in}{0.586309in}%
\pgfsys@useobject{currentmarker}{}%
\end{pgfscope}%
\end{pgfscope}%
\begin{pgfscope}%
\pgfsetbuttcap%
\pgfsetroundjoin%
\definecolor{currentfill}{rgb}{0.000000,0.000000,0.000000}%
\pgfsetfillcolor{currentfill}%
\pgfsetlinewidth{0.501875pt}%
\definecolor{currentstroke}{rgb}{0.000000,0.000000,0.000000}%
\pgfsetstrokecolor{currentstroke}%
\pgfsetdash{}{0pt}%
\pgfsys@defobject{currentmarker}{\pgfqpoint{0.000000in}{-0.041667in}}{\pgfqpoint{0.000000in}{0.000000in}}{%
\pgfpathmoveto{\pgfqpoint{0.000000in}{0.000000in}}%
\pgfpathlineto{\pgfqpoint{0.000000in}{-0.041667in}}%
\pgfusepath{stroke,fill}%
}%
\begin{pgfscope}%
\pgfsys@transformshift{1.174133in}{0.893003in}%
\pgfsys@useobject{currentmarker}{}%
\end{pgfscope}%
\end{pgfscope}%
\begin{pgfscope}%
\definecolor{textcolor}{rgb}{0.000000,0.000000,0.000000}%
\pgfsetstrokecolor{textcolor}%
\pgfsetfillcolor{textcolor}%
\pgftext[x=0.716606in, y=0.220556in, left, base,rotate=30.000000]{\color{textcolor}\rmfamily\fontsize{7.000000}{8.400000}\selectfont 2018-11-12}%
\end{pgfscope}%
\begin{pgfscope}%
\pgfsetbuttcap%
\pgfsetroundjoin%
\definecolor{currentfill}{rgb}{0.000000,0.000000,0.000000}%
\pgfsetfillcolor{currentfill}%
\pgfsetlinewidth{0.501875pt}%
\definecolor{currentstroke}{rgb}{0.000000,0.000000,0.000000}%
\pgfsetstrokecolor{currentstroke}%
\pgfsetdash{}{0pt}%
\pgfsys@defobject{currentmarker}{\pgfqpoint{0.000000in}{0.000000in}}{\pgfqpoint{0.000000in}{0.041667in}}{%
\pgfpathmoveto{\pgfqpoint{0.000000in}{0.000000in}}%
\pgfpathlineto{\pgfqpoint{0.000000in}{0.041667in}}%
\pgfusepath{stroke,fill}%
}%
\begin{pgfscope}%
\pgfsys@transformshift{1.580462in}{0.586309in}%
\pgfsys@useobject{currentmarker}{}%
\end{pgfscope}%
\end{pgfscope}%
\begin{pgfscope}%
\pgfsetbuttcap%
\pgfsetroundjoin%
\definecolor{currentfill}{rgb}{0.000000,0.000000,0.000000}%
\pgfsetfillcolor{currentfill}%
\pgfsetlinewidth{0.501875pt}%
\definecolor{currentstroke}{rgb}{0.000000,0.000000,0.000000}%
\pgfsetstrokecolor{currentstroke}%
\pgfsetdash{}{0pt}%
\pgfsys@defobject{currentmarker}{\pgfqpoint{0.000000in}{-0.041667in}}{\pgfqpoint{0.000000in}{0.000000in}}{%
\pgfpathmoveto{\pgfqpoint{0.000000in}{0.000000in}}%
\pgfpathlineto{\pgfqpoint{0.000000in}{-0.041667in}}%
\pgfusepath{stroke,fill}%
}%
\begin{pgfscope}%
\pgfsys@transformshift{1.580462in}{0.893003in}%
\pgfsys@useobject{currentmarker}{}%
\end{pgfscope}%
\end{pgfscope}%
\begin{pgfscope}%
\definecolor{textcolor}{rgb}{0.000000,0.000000,0.000000}%
\pgfsetstrokecolor{textcolor}%
\pgfsetfillcolor{textcolor}%
\pgftext[x=1.122935in, y=0.220556in, left, base,rotate=30.000000]{\color{textcolor}\rmfamily\fontsize{7.000000}{8.400000}\selectfont 2018-12-12}%
\end{pgfscope}%
\begin{pgfscope}%
\pgfsetbuttcap%
\pgfsetroundjoin%
\definecolor{currentfill}{rgb}{0.000000,0.000000,0.000000}%
\pgfsetfillcolor{currentfill}%
\pgfsetlinewidth{0.501875pt}%
\definecolor{currentstroke}{rgb}{0.000000,0.000000,0.000000}%
\pgfsetstrokecolor{currentstroke}%
\pgfsetdash{}{0pt}%
\pgfsys@defobject{currentmarker}{\pgfqpoint{0.000000in}{0.000000in}}{\pgfqpoint{0.000000in}{0.041667in}}{%
\pgfpathmoveto{\pgfqpoint{0.000000in}{0.000000in}}%
\pgfpathlineto{\pgfqpoint{0.000000in}{0.041667in}}%
\pgfusepath{stroke,fill}%
}%
\begin{pgfscope}%
\pgfsys@transformshift{1.986791in}{0.586309in}%
\pgfsys@useobject{currentmarker}{}%
\end{pgfscope}%
\end{pgfscope}%
\begin{pgfscope}%
\pgfsetbuttcap%
\pgfsetroundjoin%
\definecolor{currentfill}{rgb}{0.000000,0.000000,0.000000}%
\pgfsetfillcolor{currentfill}%
\pgfsetlinewidth{0.501875pt}%
\definecolor{currentstroke}{rgb}{0.000000,0.000000,0.000000}%
\pgfsetstrokecolor{currentstroke}%
\pgfsetdash{}{0pt}%
\pgfsys@defobject{currentmarker}{\pgfqpoint{0.000000in}{-0.041667in}}{\pgfqpoint{0.000000in}{0.000000in}}{%
\pgfpathmoveto{\pgfqpoint{0.000000in}{0.000000in}}%
\pgfpathlineto{\pgfqpoint{0.000000in}{-0.041667in}}%
\pgfusepath{stroke,fill}%
}%
\begin{pgfscope}%
\pgfsys@transformshift{1.986791in}{0.893003in}%
\pgfsys@useobject{currentmarker}{}%
\end{pgfscope}%
\end{pgfscope}%
\begin{pgfscope}%
\definecolor{textcolor}{rgb}{0.000000,0.000000,0.000000}%
\pgfsetstrokecolor{textcolor}%
\pgfsetfillcolor{textcolor}%
\pgftext[x=1.529264in, y=0.220556in, left, base,rotate=30.000000]{\color{textcolor}\rmfamily\fontsize{7.000000}{8.400000}\selectfont 2019-01-11}%
\end{pgfscope}%
\begin{pgfscope}%
\pgfsetbuttcap%
\pgfsetroundjoin%
\definecolor{currentfill}{rgb}{0.000000,0.000000,0.000000}%
\pgfsetfillcolor{currentfill}%
\pgfsetlinewidth{0.501875pt}%
\definecolor{currentstroke}{rgb}{0.000000,0.000000,0.000000}%
\pgfsetstrokecolor{currentstroke}%
\pgfsetdash{}{0pt}%
\pgfsys@defobject{currentmarker}{\pgfqpoint{0.000000in}{0.000000in}}{\pgfqpoint{0.000000in}{0.041667in}}{%
\pgfpathmoveto{\pgfqpoint{0.000000in}{0.000000in}}%
\pgfpathlineto{\pgfqpoint{0.000000in}{0.041667in}}%
\pgfusepath{stroke,fill}%
}%
\begin{pgfscope}%
\pgfsys@transformshift{2.393120in}{0.586309in}%
\pgfsys@useobject{currentmarker}{}%
\end{pgfscope}%
\end{pgfscope}%
\begin{pgfscope}%
\pgfsetbuttcap%
\pgfsetroundjoin%
\definecolor{currentfill}{rgb}{0.000000,0.000000,0.000000}%
\pgfsetfillcolor{currentfill}%
\pgfsetlinewidth{0.501875pt}%
\definecolor{currentstroke}{rgb}{0.000000,0.000000,0.000000}%
\pgfsetstrokecolor{currentstroke}%
\pgfsetdash{}{0pt}%
\pgfsys@defobject{currentmarker}{\pgfqpoint{0.000000in}{-0.041667in}}{\pgfqpoint{0.000000in}{0.000000in}}{%
\pgfpathmoveto{\pgfqpoint{0.000000in}{0.000000in}}%
\pgfpathlineto{\pgfqpoint{0.000000in}{-0.041667in}}%
\pgfusepath{stroke,fill}%
}%
\begin{pgfscope}%
\pgfsys@transformshift{2.393120in}{0.893003in}%
\pgfsys@useobject{currentmarker}{}%
\end{pgfscope}%
\end{pgfscope}%
\begin{pgfscope}%
\definecolor{textcolor}{rgb}{0.000000,0.000000,0.000000}%
\pgfsetstrokecolor{textcolor}%
\pgfsetfillcolor{textcolor}%
\pgftext[x=1.935593in, y=0.220556in, left, base,rotate=30.000000]{\color{textcolor}\rmfamily\fontsize{7.000000}{8.400000}\selectfont 2019-02-10}%
\end{pgfscope}%
\begin{pgfscope}%
\pgfsetbuttcap%
\pgfsetroundjoin%
\definecolor{currentfill}{rgb}{0.000000,0.000000,0.000000}%
\pgfsetfillcolor{currentfill}%
\pgfsetlinewidth{0.501875pt}%
\definecolor{currentstroke}{rgb}{0.000000,0.000000,0.000000}%
\pgfsetstrokecolor{currentstroke}%
\pgfsetdash{}{0pt}%
\pgfsys@defobject{currentmarker}{\pgfqpoint{0.000000in}{0.000000in}}{\pgfqpoint{0.000000in}{0.041667in}}{%
\pgfpathmoveto{\pgfqpoint{0.000000in}{0.000000in}}%
\pgfpathlineto{\pgfqpoint{0.000000in}{0.041667in}}%
\pgfusepath{stroke,fill}%
}%
\begin{pgfscope}%
\pgfsys@transformshift{2.799449in}{0.586309in}%
\pgfsys@useobject{currentmarker}{}%
\end{pgfscope}%
\end{pgfscope}%
\begin{pgfscope}%
\pgfsetbuttcap%
\pgfsetroundjoin%
\definecolor{currentfill}{rgb}{0.000000,0.000000,0.000000}%
\pgfsetfillcolor{currentfill}%
\pgfsetlinewidth{0.501875pt}%
\definecolor{currentstroke}{rgb}{0.000000,0.000000,0.000000}%
\pgfsetstrokecolor{currentstroke}%
\pgfsetdash{}{0pt}%
\pgfsys@defobject{currentmarker}{\pgfqpoint{0.000000in}{-0.041667in}}{\pgfqpoint{0.000000in}{0.000000in}}{%
\pgfpathmoveto{\pgfqpoint{0.000000in}{0.000000in}}%
\pgfpathlineto{\pgfqpoint{0.000000in}{-0.041667in}}%
\pgfusepath{stroke,fill}%
}%
\begin{pgfscope}%
\pgfsys@transformshift{2.799449in}{0.893003in}%
\pgfsys@useobject{currentmarker}{}%
\end{pgfscope}%
\end{pgfscope}%
\begin{pgfscope}%
\definecolor{textcolor}{rgb}{0.000000,0.000000,0.000000}%
\pgfsetstrokecolor{textcolor}%
\pgfsetfillcolor{textcolor}%
\pgftext[x=2.341922in, y=0.220556in, left, base,rotate=30.000000]{\color{textcolor}\rmfamily\fontsize{7.000000}{8.400000}\selectfont 2019-03-12}%
\end{pgfscope}%
\begin{pgfscope}%
\pgfsetbuttcap%
\pgfsetroundjoin%
\definecolor{currentfill}{rgb}{0.000000,0.000000,0.000000}%
\pgfsetfillcolor{currentfill}%
\pgfsetlinewidth{0.501875pt}%
\definecolor{currentstroke}{rgb}{0.000000,0.000000,0.000000}%
\pgfsetstrokecolor{currentstroke}%
\pgfsetdash{}{0pt}%
\pgfsys@defobject{currentmarker}{\pgfqpoint{0.000000in}{0.000000in}}{\pgfqpoint{0.000000in}{0.041667in}}{%
\pgfpathmoveto{\pgfqpoint{0.000000in}{0.000000in}}%
\pgfpathlineto{\pgfqpoint{0.000000in}{0.041667in}}%
\pgfusepath{stroke,fill}%
}%
\begin{pgfscope}%
\pgfsys@transformshift{3.205778in}{0.586309in}%
\pgfsys@useobject{currentmarker}{}%
\end{pgfscope}%
\end{pgfscope}%
\begin{pgfscope}%
\pgfsetbuttcap%
\pgfsetroundjoin%
\definecolor{currentfill}{rgb}{0.000000,0.000000,0.000000}%
\pgfsetfillcolor{currentfill}%
\pgfsetlinewidth{0.501875pt}%
\definecolor{currentstroke}{rgb}{0.000000,0.000000,0.000000}%
\pgfsetstrokecolor{currentstroke}%
\pgfsetdash{}{0pt}%
\pgfsys@defobject{currentmarker}{\pgfqpoint{0.000000in}{-0.041667in}}{\pgfqpoint{0.000000in}{0.000000in}}{%
\pgfpathmoveto{\pgfqpoint{0.000000in}{0.000000in}}%
\pgfpathlineto{\pgfqpoint{0.000000in}{-0.041667in}}%
\pgfusepath{stroke,fill}%
}%
\begin{pgfscope}%
\pgfsys@transformshift{3.205778in}{0.893003in}%
\pgfsys@useobject{currentmarker}{}%
\end{pgfscope}%
\end{pgfscope}%
\begin{pgfscope}%
\definecolor{textcolor}{rgb}{0.000000,0.000000,0.000000}%
\pgfsetstrokecolor{textcolor}%
\pgfsetfillcolor{textcolor}%
\pgftext[x=2.748251in, y=0.220556in, left, base,rotate=30.000000]{\color{textcolor}\rmfamily\fontsize{7.000000}{8.400000}\selectfont 2019-04-11}%
\end{pgfscope}%
\begin{pgfscope}%
\pgfsetbuttcap%
\pgfsetroundjoin%
\definecolor{currentfill}{rgb}{0.000000,0.000000,0.000000}%
\pgfsetfillcolor{currentfill}%
\pgfsetlinewidth{0.501875pt}%
\definecolor{currentstroke}{rgb}{0.000000,0.000000,0.000000}%
\pgfsetstrokecolor{currentstroke}%
\pgfsetdash{}{0pt}%
\pgfsys@defobject{currentmarker}{\pgfqpoint{0.000000in}{0.000000in}}{\pgfqpoint{0.000000in}{0.041667in}}{%
\pgfpathmoveto{\pgfqpoint{0.000000in}{0.000000in}}%
\pgfpathlineto{\pgfqpoint{0.000000in}{0.041667in}}%
\pgfusepath{stroke,fill}%
}%
\begin{pgfscope}%
\pgfsys@transformshift{3.612106in}{0.586309in}%
\pgfsys@useobject{currentmarker}{}%
\end{pgfscope}%
\end{pgfscope}%
\begin{pgfscope}%
\pgfsetbuttcap%
\pgfsetroundjoin%
\definecolor{currentfill}{rgb}{0.000000,0.000000,0.000000}%
\pgfsetfillcolor{currentfill}%
\pgfsetlinewidth{0.501875pt}%
\definecolor{currentstroke}{rgb}{0.000000,0.000000,0.000000}%
\pgfsetstrokecolor{currentstroke}%
\pgfsetdash{}{0pt}%
\pgfsys@defobject{currentmarker}{\pgfqpoint{0.000000in}{-0.041667in}}{\pgfqpoint{0.000000in}{0.000000in}}{%
\pgfpathmoveto{\pgfqpoint{0.000000in}{0.000000in}}%
\pgfpathlineto{\pgfqpoint{0.000000in}{-0.041667in}}%
\pgfusepath{stroke,fill}%
}%
\begin{pgfscope}%
\pgfsys@transformshift{3.612106in}{0.893003in}%
\pgfsys@useobject{currentmarker}{}%
\end{pgfscope}%
\end{pgfscope}%
\begin{pgfscope}%
\definecolor{textcolor}{rgb}{0.000000,0.000000,0.000000}%
\pgfsetstrokecolor{textcolor}%
\pgfsetfillcolor{textcolor}%
\pgftext[x=3.154580in, y=0.220556in, left, base,rotate=30.000000]{\color{textcolor}\rmfamily\fontsize{7.000000}{8.400000}\selectfont 2019-05-11}%
\end{pgfscope}%
\begin{pgfscope}%
\pgfsetbuttcap%
\pgfsetroundjoin%
\definecolor{currentfill}{rgb}{0.000000,0.000000,0.000000}%
\pgfsetfillcolor{currentfill}%
\pgfsetlinewidth{0.501875pt}%
\definecolor{currentstroke}{rgb}{0.000000,0.000000,0.000000}%
\pgfsetstrokecolor{currentstroke}%
\pgfsetdash{}{0pt}%
\pgfsys@defobject{currentmarker}{\pgfqpoint{0.000000in}{0.000000in}}{\pgfqpoint{0.000000in}{0.041667in}}{%
\pgfpathmoveto{\pgfqpoint{0.000000in}{0.000000in}}%
\pgfpathlineto{\pgfqpoint{0.000000in}{0.041667in}}%
\pgfusepath{stroke,fill}%
}%
\begin{pgfscope}%
\pgfsys@transformshift{4.018435in}{0.586309in}%
\pgfsys@useobject{currentmarker}{}%
\end{pgfscope}%
\end{pgfscope}%
\begin{pgfscope}%
\pgfsetbuttcap%
\pgfsetroundjoin%
\definecolor{currentfill}{rgb}{0.000000,0.000000,0.000000}%
\pgfsetfillcolor{currentfill}%
\pgfsetlinewidth{0.501875pt}%
\definecolor{currentstroke}{rgb}{0.000000,0.000000,0.000000}%
\pgfsetstrokecolor{currentstroke}%
\pgfsetdash{}{0pt}%
\pgfsys@defobject{currentmarker}{\pgfqpoint{0.000000in}{-0.041667in}}{\pgfqpoint{0.000000in}{0.000000in}}{%
\pgfpathmoveto{\pgfqpoint{0.000000in}{0.000000in}}%
\pgfpathlineto{\pgfqpoint{0.000000in}{-0.041667in}}%
\pgfusepath{stroke,fill}%
}%
\begin{pgfscope}%
\pgfsys@transformshift{4.018435in}{0.893003in}%
\pgfsys@useobject{currentmarker}{}%
\end{pgfscope}%
\end{pgfscope}%
\begin{pgfscope}%
\definecolor{textcolor}{rgb}{0.000000,0.000000,0.000000}%
\pgfsetstrokecolor{textcolor}%
\pgfsetfillcolor{textcolor}%
\pgftext[x=3.560909in, y=0.220556in, left, base,rotate=30.000000]{\color{textcolor}\rmfamily\fontsize{7.000000}{8.400000}\selectfont 2019-06-10}%
\end{pgfscope}%
\begin{pgfscope}%
\pgfsetbuttcap%
\pgfsetroundjoin%
\definecolor{currentfill}{rgb}{0.000000,0.000000,0.000000}%
\pgfsetfillcolor{currentfill}%
\pgfsetlinewidth{0.501875pt}%
\definecolor{currentstroke}{rgb}{0.000000,0.000000,0.000000}%
\pgfsetstrokecolor{currentstroke}%
\pgfsetdash{}{0pt}%
\pgfsys@defobject{currentmarker}{\pgfqpoint{0.000000in}{0.000000in}}{\pgfqpoint{0.000000in}{0.041667in}}{%
\pgfpathmoveto{\pgfqpoint{0.000000in}{0.000000in}}%
\pgfpathlineto{\pgfqpoint{0.000000in}{0.041667in}}%
\pgfusepath{stroke,fill}%
}%
\begin{pgfscope}%
\pgfsys@transformshift{4.424764in}{0.586309in}%
\pgfsys@useobject{currentmarker}{}%
\end{pgfscope}%
\end{pgfscope}%
\begin{pgfscope}%
\pgfsetbuttcap%
\pgfsetroundjoin%
\definecolor{currentfill}{rgb}{0.000000,0.000000,0.000000}%
\pgfsetfillcolor{currentfill}%
\pgfsetlinewidth{0.501875pt}%
\definecolor{currentstroke}{rgb}{0.000000,0.000000,0.000000}%
\pgfsetstrokecolor{currentstroke}%
\pgfsetdash{}{0pt}%
\pgfsys@defobject{currentmarker}{\pgfqpoint{0.000000in}{-0.041667in}}{\pgfqpoint{0.000000in}{0.000000in}}{%
\pgfpathmoveto{\pgfqpoint{0.000000in}{0.000000in}}%
\pgfpathlineto{\pgfqpoint{0.000000in}{-0.041667in}}%
\pgfusepath{stroke,fill}%
}%
\begin{pgfscope}%
\pgfsys@transformshift{4.424764in}{0.893003in}%
\pgfsys@useobject{currentmarker}{}%
\end{pgfscope}%
\end{pgfscope}%
\begin{pgfscope}%
\definecolor{textcolor}{rgb}{0.000000,0.000000,0.000000}%
\pgfsetstrokecolor{textcolor}%
\pgfsetfillcolor{textcolor}%
\pgftext[x=3.967237in, y=0.220556in, left, base,rotate=30.000000]{\color{textcolor}\rmfamily\fontsize{7.000000}{8.400000}\selectfont 2019-07-10}%
\end{pgfscope}%
\begin{pgfscope}%
\pgfsetbuttcap%
\pgfsetroundjoin%
\definecolor{currentfill}{rgb}{0.000000,0.000000,0.000000}%
\pgfsetfillcolor{currentfill}%
\pgfsetlinewidth{0.501875pt}%
\definecolor{currentstroke}{rgb}{0.000000,0.000000,0.000000}%
\pgfsetstrokecolor{currentstroke}%
\pgfsetdash{}{0pt}%
\pgfsys@defobject{currentmarker}{\pgfqpoint{0.000000in}{0.000000in}}{\pgfqpoint{0.000000in}{0.041667in}}{%
\pgfpathmoveto{\pgfqpoint{0.000000in}{0.000000in}}%
\pgfpathlineto{\pgfqpoint{0.000000in}{0.041667in}}%
\pgfusepath{stroke,fill}%
}%
\begin{pgfscope}%
\pgfsys@transformshift{4.831093in}{0.586309in}%
\pgfsys@useobject{currentmarker}{}%
\end{pgfscope}%
\end{pgfscope}%
\begin{pgfscope}%
\pgfsetbuttcap%
\pgfsetroundjoin%
\definecolor{currentfill}{rgb}{0.000000,0.000000,0.000000}%
\pgfsetfillcolor{currentfill}%
\pgfsetlinewidth{0.501875pt}%
\definecolor{currentstroke}{rgb}{0.000000,0.000000,0.000000}%
\pgfsetstrokecolor{currentstroke}%
\pgfsetdash{}{0pt}%
\pgfsys@defobject{currentmarker}{\pgfqpoint{0.000000in}{-0.041667in}}{\pgfqpoint{0.000000in}{0.000000in}}{%
\pgfpathmoveto{\pgfqpoint{0.000000in}{0.000000in}}%
\pgfpathlineto{\pgfqpoint{0.000000in}{-0.041667in}}%
\pgfusepath{stroke,fill}%
}%
\begin{pgfscope}%
\pgfsys@transformshift{4.831093in}{0.893003in}%
\pgfsys@useobject{currentmarker}{}%
\end{pgfscope}%
\end{pgfscope}%
\begin{pgfscope}%
\definecolor{textcolor}{rgb}{0.000000,0.000000,0.000000}%
\pgfsetstrokecolor{textcolor}%
\pgfsetfillcolor{textcolor}%
\pgftext[x=4.373566in, y=0.220556in, left, base,rotate=30.000000]{\color{textcolor}\rmfamily\fontsize{7.000000}{8.400000}\selectfont 2019-08-09}%
\end{pgfscope}%
\begin{pgfscope}%
\pgfsetbuttcap%
\pgfsetroundjoin%
\definecolor{currentfill}{rgb}{0.000000,0.000000,0.000000}%
\pgfsetfillcolor{currentfill}%
\pgfsetlinewidth{0.501875pt}%
\definecolor{currentstroke}{rgb}{0.000000,0.000000,0.000000}%
\pgfsetstrokecolor{currentstroke}%
\pgfsetdash{}{0pt}%
\pgfsys@defobject{currentmarker}{\pgfqpoint{0.000000in}{0.000000in}}{\pgfqpoint{0.000000in}{0.041667in}}{%
\pgfpathmoveto{\pgfqpoint{0.000000in}{0.000000in}}%
\pgfpathlineto{\pgfqpoint{0.000000in}{0.041667in}}%
\pgfusepath{stroke,fill}%
}%
\begin{pgfscope}%
\pgfsys@transformshift{5.237422in}{0.586309in}%
\pgfsys@useobject{currentmarker}{}%
\end{pgfscope}%
\end{pgfscope}%
\begin{pgfscope}%
\pgfsetbuttcap%
\pgfsetroundjoin%
\definecolor{currentfill}{rgb}{0.000000,0.000000,0.000000}%
\pgfsetfillcolor{currentfill}%
\pgfsetlinewidth{0.501875pt}%
\definecolor{currentstroke}{rgb}{0.000000,0.000000,0.000000}%
\pgfsetstrokecolor{currentstroke}%
\pgfsetdash{}{0pt}%
\pgfsys@defobject{currentmarker}{\pgfqpoint{0.000000in}{-0.041667in}}{\pgfqpoint{0.000000in}{0.000000in}}{%
\pgfpathmoveto{\pgfqpoint{0.000000in}{0.000000in}}%
\pgfpathlineto{\pgfqpoint{0.000000in}{-0.041667in}}%
\pgfusepath{stroke,fill}%
}%
\begin{pgfscope}%
\pgfsys@transformshift{5.237422in}{0.893003in}%
\pgfsys@useobject{currentmarker}{}%
\end{pgfscope}%
\end{pgfscope}%
\begin{pgfscope}%
\definecolor{textcolor}{rgb}{0.000000,0.000000,0.000000}%
\pgfsetstrokecolor{textcolor}%
\pgfsetfillcolor{textcolor}%
\pgftext[x=4.779895in, y=0.220556in, left, base,rotate=30.000000]{\color{textcolor}\rmfamily\fontsize{7.000000}{8.400000}\selectfont 2019-09-08}%
\end{pgfscope}%
\begin{pgfscope}%
\pgfsetbuttcap%
\pgfsetroundjoin%
\definecolor{currentfill}{rgb}{0.000000,0.000000,0.000000}%
\pgfsetfillcolor{currentfill}%
\pgfsetlinewidth{0.501875pt}%
\definecolor{currentstroke}{rgb}{0.000000,0.000000,0.000000}%
\pgfsetstrokecolor{currentstroke}%
\pgfsetdash{}{0pt}%
\pgfsys@defobject{currentmarker}{\pgfqpoint{0.000000in}{0.000000in}}{\pgfqpoint{0.000000in}{0.041667in}}{%
\pgfpathmoveto{\pgfqpoint{0.000000in}{0.000000in}}%
\pgfpathlineto{\pgfqpoint{0.000000in}{0.041667in}}%
\pgfusepath{stroke,fill}%
}%
\begin{pgfscope}%
\pgfsys@transformshift{5.643751in}{0.586309in}%
\pgfsys@useobject{currentmarker}{}%
\end{pgfscope}%
\end{pgfscope}%
\begin{pgfscope}%
\pgfsetbuttcap%
\pgfsetroundjoin%
\definecolor{currentfill}{rgb}{0.000000,0.000000,0.000000}%
\pgfsetfillcolor{currentfill}%
\pgfsetlinewidth{0.501875pt}%
\definecolor{currentstroke}{rgb}{0.000000,0.000000,0.000000}%
\pgfsetstrokecolor{currentstroke}%
\pgfsetdash{}{0pt}%
\pgfsys@defobject{currentmarker}{\pgfqpoint{0.000000in}{-0.041667in}}{\pgfqpoint{0.000000in}{0.000000in}}{%
\pgfpathmoveto{\pgfqpoint{0.000000in}{0.000000in}}%
\pgfpathlineto{\pgfqpoint{0.000000in}{-0.041667in}}%
\pgfusepath{stroke,fill}%
}%
\begin{pgfscope}%
\pgfsys@transformshift{5.643751in}{0.893003in}%
\pgfsys@useobject{currentmarker}{}%
\end{pgfscope}%
\end{pgfscope}%
\begin{pgfscope}%
\definecolor{textcolor}{rgb}{0.000000,0.000000,0.000000}%
\pgfsetstrokecolor{textcolor}%
\pgfsetfillcolor{textcolor}%
\pgftext[x=5.186224in, y=0.220556in, left, base,rotate=30.000000]{\color{textcolor}\rmfamily\fontsize{7.000000}{8.400000}\selectfont 2019-10-08}%
\end{pgfscope}%
\begin{pgfscope}%
\pgfsetbuttcap%
\pgfsetroundjoin%
\definecolor{currentfill}{rgb}{0.000000,0.000000,0.000000}%
\pgfsetfillcolor{currentfill}%
\pgfsetlinewidth{0.501875pt}%
\definecolor{currentstroke}{rgb}{0.000000,0.000000,0.000000}%
\pgfsetstrokecolor{currentstroke}%
\pgfsetdash{}{0pt}%
\pgfsys@defobject{currentmarker}{\pgfqpoint{0.000000in}{0.000000in}}{\pgfqpoint{0.000000in}{0.041667in}}{%
\pgfpathmoveto{\pgfqpoint{0.000000in}{0.000000in}}%
\pgfpathlineto{\pgfqpoint{0.000000in}{0.041667in}}%
\pgfusepath{stroke,fill}%
}%
\begin{pgfscope}%
\pgfsys@transformshift{6.050080in}{0.586309in}%
\pgfsys@useobject{currentmarker}{}%
\end{pgfscope}%
\end{pgfscope}%
\begin{pgfscope}%
\pgfsetbuttcap%
\pgfsetroundjoin%
\definecolor{currentfill}{rgb}{0.000000,0.000000,0.000000}%
\pgfsetfillcolor{currentfill}%
\pgfsetlinewidth{0.501875pt}%
\definecolor{currentstroke}{rgb}{0.000000,0.000000,0.000000}%
\pgfsetstrokecolor{currentstroke}%
\pgfsetdash{}{0pt}%
\pgfsys@defobject{currentmarker}{\pgfqpoint{0.000000in}{-0.041667in}}{\pgfqpoint{0.000000in}{0.000000in}}{%
\pgfpathmoveto{\pgfqpoint{0.000000in}{0.000000in}}%
\pgfpathlineto{\pgfqpoint{0.000000in}{-0.041667in}}%
\pgfusepath{stroke,fill}%
}%
\begin{pgfscope}%
\pgfsys@transformshift{6.050080in}{0.893003in}%
\pgfsys@useobject{currentmarker}{}%
\end{pgfscope}%
\end{pgfscope}%
\begin{pgfscope}%
\definecolor{textcolor}{rgb}{0.000000,0.000000,0.000000}%
\pgfsetstrokecolor{textcolor}%
\pgfsetfillcolor{textcolor}%
\pgftext[x=5.592553in, y=0.220556in, left, base,rotate=30.000000]{\color{textcolor}\rmfamily\fontsize{7.000000}{8.400000}\selectfont 2019-11-07}%
\end{pgfscope}%
\begin{pgfscope}%
\pgfsetbuttcap%
\pgfsetroundjoin%
\definecolor{currentfill}{rgb}{0.000000,0.000000,0.000000}%
\pgfsetfillcolor{currentfill}%
\pgfsetlinewidth{0.501875pt}%
\definecolor{currentstroke}{rgb}{0.000000,0.000000,0.000000}%
\pgfsetstrokecolor{currentstroke}%
\pgfsetdash{}{0pt}%
\pgfsys@defobject{currentmarker}{\pgfqpoint{0.000000in}{0.000000in}}{\pgfqpoint{0.000000in}{0.020833in}}{%
\pgfpathmoveto{\pgfqpoint{0.000000in}{0.000000in}}%
\pgfpathlineto{\pgfqpoint{0.000000in}{0.020833in}}%
\pgfusepath{stroke,fill}%
}%
\begin{pgfscope}%
\pgfsys@transformshift{0.496918in}{0.586309in}%
\pgfsys@useobject{currentmarker}{}%
\end{pgfscope}%
\end{pgfscope}%
\begin{pgfscope}%
\pgfsetbuttcap%
\pgfsetroundjoin%
\definecolor{currentfill}{rgb}{0.000000,0.000000,0.000000}%
\pgfsetfillcolor{currentfill}%
\pgfsetlinewidth{0.501875pt}%
\definecolor{currentstroke}{rgb}{0.000000,0.000000,0.000000}%
\pgfsetstrokecolor{currentstroke}%
\pgfsetdash{}{0pt}%
\pgfsys@defobject{currentmarker}{\pgfqpoint{0.000000in}{-0.020833in}}{\pgfqpoint{0.000000in}{0.000000in}}{%
\pgfpathmoveto{\pgfqpoint{0.000000in}{0.000000in}}%
\pgfpathlineto{\pgfqpoint{0.000000in}{-0.020833in}}%
\pgfusepath{stroke,fill}%
}%
\begin{pgfscope}%
\pgfsys@transformshift{0.496918in}{0.893003in}%
\pgfsys@useobject{currentmarker}{}%
\end{pgfscope}%
\end{pgfscope}%
\begin{pgfscope}%
\pgfsetbuttcap%
\pgfsetroundjoin%
\definecolor{currentfill}{rgb}{0.000000,0.000000,0.000000}%
\pgfsetfillcolor{currentfill}%
\pgfsetlinewidth{0.501875pt}%
\definecolor{currentstroke}{rgb}{0.000000,0.000000,0.000000}%
\pgfsetstrokecolor{currentstroke}%
\pgfsetdash{}{0pt}%
\pgfsys@defobject{currentmarker}{\pgfqpoint{0.000000in}{0.000000in}}{\pgfqpoint{0.000000in}{0.020833in}}{%
\pgfpathmoveto{\pgfqpoint{0.000000in}{0.000000in}}%
\pgfpathlineto{\pgfqpoint{0.000000in}{0.020833in}}%
\pgfusepath{stroke,fill}%
}%
\begin{pgfscope}%
\pgfsys@transformshift{0.564640in}{0.586309in}%
\pgfsys@useobject{currentmarker}{}%
\end{pgfscope}%
\end{pgfscope}%
\begin{pgfscope}%
\pgfsetbuttcap%
\pgfsetroundjoin%
\definecolor{currentfill}{rgb}{0.000000,0.000000,0.000000}%
\pgfsetfillcolor{currentfill}%
\pgfsetlinewidth{0.501875pt}%
\definecolor{currentstroke}{rgb}{0.000000,0.000000,0.000000}%
\pgfsetstrokecolor{currentstroke}%
\pgfsetdash{}{0pt}%
\pgfsys@defobject{currentmarker}{\pgfqpoint{0.000000in}{-0.020833in}}{\pgfqpoint{0.000000in}{0.000000in}}{%
\pgfpathmoveto{\pgfqpoint{0.000000in}{0.000000in}}%
\pgfpathlineto{\pgfqpoint{0.000000in}{-0.020833in}}%
\pgfusepath{stroke,fill}%
}%
\begin{pgfscope}%
\pgfsys@transformshift{0.564640in}{0.893003in}%
\pgfsys@useobject{currentmarker}{}%
\end{pgfscope}%
\end{pgfscope}%
\begin{pgfscope}%
\pgfsetbuttcap%
\pgfsetroundjoin%
\definecolor{currentfill}{rgb}{0.000000,0.000000,0.000000}%
\pgfsetfillcolor{currentfill}%
\pgfsetlinewidth{0.501875pt}%
\definecolor{currentstroke}{rgb}{0.000000,0.000000,0.000000}%
\pgfsetstrokecolor{currentstroke}%
\pgfsetdash{}{0pt}%
\pgfsys@defobject{currentmarker}{\pgfqpoint{0.000000in}{0.000000in}}{\pgfqpoint{0.000000in}{0.020833in}}{%
\pgfpathmoveto{\pgfqpoint{0.000000in}{0.000000in}}%
\pgfpathlineto{\pgfqpoint{0.000000in}{0.020833in}}%
\pgfusepath{stroke,fill}%
}%
\begin{pgfscope}%
\pgfsys@transformshift{0.632361in}{0.586309in}%
\pgfsys@useobject{currentmarker}{}%
\end{pgfscope}%
\end{pgfscope}%
\begin{pgfscope}%
\pgfsetbuttcap%
\pgfsetroundjoin%
\definecolor{currentfill}{rgb}{0.000000,0.000000,0.000000}%
\pgfsetfillcolor{currentfill}%
\pgfsetlinewidth{0.501875pt}%
\definecolor{currentstroke}{rgb}{0.000000,0.000000,0.000000}%
\pgfsetstrokecolor{currentstroke}%
\pgfsetdash{}{0pt}%
\pgfsys@defobject{currentmarker}{\pgfqpoint{0.000000in}{-0.020833in}}{\pgfqpoint{0.000000in}{0.000000in}}{%
\pgfpathmoveto{\pgfqpoint{0.000000in}{0.000000in}}%
\pgfpathlineto{\pgfqpoint{0.000000in}{-0.020833in}}%
\pgfusepath{stroke,fill}%
}%
\begin{pgfscope}%
\pgfsys@transformshift{0.632361in}{0.893003in}%
\pgfsys@useobject{currentmarker}{}%
\end{pgfscope}%
\end{pgfscope}%
\begin{pgfscope}%
\pgfsetbuttcap%
\pgfsetroundjoin%
\definecolor{currentfill}{rgb}{0.000000,0.000000,0.000000}%
\pgfsetfillcolor{currentfill}%
\pgfsetlinewidth{0.501875pt}%
\definecolor{currentstroke}{rgb}{0.000000,0.000000,0.000000}%
\pgfsetstrokecolor{currentstroke}%
\pgfsetdash{}{0pt}%
\pgfsys@defobject{currentmarker}{\pgfqpoint{0.000000in}{0.000000in}}{\pgfqpoint{0.000000in}{0.020833in}}{%
\pgfpathmoveto{\pgfqpoint{0.000000in}{0.000000in}}%
\pgfpathlineto{\pgfqpoint{0.000000in}{0.020833in}}%
\pgfusepath{stroke,fill}%
}%
\begin{pgfscope}%
\pgfsys@transformshift{0.700083in}{0.586309in}%
\pgfsys@useobject{currentmarker}{}%
\end{pgfscope}%
\end{pgfscope}%
\begin{pgfscope}%
\pgfsetbuttcap%
\pgfsetroundjoin%
\definecolor{currentfill}{rgb}{0.000000,0.000000,0.000000}%
\pgfsetfillcolor{currentfill}%
\pgfsetlinewidth{0.501875pt}%
\definecolor{currentstroke}{rgb}{0.000000,0.000000,0.000000}%
\pgfsetstrokecolor{currentstroke}%
\pgfsetdash{}{0pt}%
\pgfsys@defobject{currentmarker}{\pgfqpoint{0.000000in}{-0.020833in}}{\pgfqpoint{0.000000in}{0.000000in}}{%
\pgfpathmoveto{\pgfqpoint{0.000000in}{0.000000in}}%
\pgfpathlineto{\pgfqpoint{0.000000in}{-0.020833in}}%
\pgfusepath{stroke,fill}%
}%
\begin{pgfscope}%
\pgfsys@transformshift{0.700083in}{0.893003in}%
\pgfsys@useobject{currentmarker}{}%
\end{pgfscope}%
\end{pgfscope}%
\begin{pgfscope}%
\pgfsetbuttcap%
\pgfsetroundjoin%
\definecolor{currentfill}{rgb}{0.000000,0.000000,0.000000}%
\pgfsetfillcolor{currentfill}%
\pgfsetlinewidth{0.501875pt}%
\definecolor{currentstroke}{rgb}{0.000000,0.000000,0.000000}%
\pgfsetstrokecolor{currentstroke}%
\pgfsetdash{}{0pt}%
\pgfsys@defobject{currentmarker}{\pgfqpoint{0.000000in}{0.000000in}}{\pgfqpoint{0.000000in}{0.020833in}}{%
\pgfpathmoveto{\pgfqpoint{0.000000in}{0.000000in}}%
\pgfpathlineto{\pgfqpoint{0.000000in}{0.020833in}}%
\pgfusepath{stroke,fill}%
}%
\begin{pgfscope}%
\pgfsys@transformshift{0.835526in}{0.586309in}%
\pgfsys@useobject{currentmarker}{}%
\end{pgfscope}%
\end{pgfscope}%
\begin{pgfscope}%
\pgfsetbuttcap%
\pgfsetroundjoin%
\definecolor{currentfill}{rgb}{0.000000,0.000000,0.000000}%
\pgfsetfillcolor{currentfill}%
\pgfsetlinewidth{0.501875pt}%
\definecolor{currentstroke}{rgb}{0.000000,0.000000,0.000000}%
\pgfsetstrokecolor{currentstroke}%
\pgfsetdash{}{0pt}%
\pgfsys@defobject{currentmarker}{\pgfqpoint{0.000000in}{-0.020833in}}{\pgfqpoint{0.000000in}{0.000000in}}{%
\pgfpathmoveto{\pgfqpoint{0.000000in}{0.000000in}}%
\pgfpathlineto{\pgfqpoint{0.000000in}{-0.020833in}}%
\pgfusepath{stroke,fill}%
}%
\begin{pgfscope}%
\pgfsys@transformshift{0.835526in}{0.893003in}%
\pgfsys@useobject{currentmarker}{}%
\end{pgfscope}%
\end{pgfscope}%
\begin{pgfscope}%
\pgfsetbuttcap%
\pgfsetroundjoin%
\definecolor{currentfill}{rgb}{0.000000,0.000000,0.000000}%
\pgfsetfillcolor{currentfill}%
\pgfsetlinewidth{0.501875pt}%
\definecolor{currentstroke}{rgb}{0.000000,0.000000,0.000000}%
\pgfsetstrokecolor{currentstroke}%
\pgfsetdash{}{0pt}%
\pgfsys@defobject{currentmarker}{\pgfqpoint{0.000000in}{0.000000in}}{\pgfqpoint{0.000000in}{0.020833in}}{%
\pgfpathmoveto{\pgfqpoint{0.000000in}{0.000000in}}%
\pgfpathlineto{\pgfqpoint{0.000000in}{0.020833in}}%
\pgfusepath{stroke,fill}%
}%
\begin{pgfscope}%
\pgfsys@transformshift{0.903247in}{0.586309in}%
\pgfsys@useobject{currentmarker}{}%
\end{pgfscope}%
\end{pgfscope}%
\begin{pgfscope}%
\pgfsetbuttcap%
\pgfsetroundjoin%
\definecolor{currentfill}{rgb}{0.000000,0.000000,0.000000}%
\pgfsetfillcolor{currentfill}%
\pgfsetlinewidth{0.501875pt}%
\definecolor{currentstroke}{rgb}{0.000000,0.000000,0.000000}%
\pgfsetstrokecolor{currentstroke}%
\pgfsetdash{}{0pt}%
\pgfsys@defobject{currentmarker}{\pgfqpoint{0.000000in}{-0.020833in}}{\pgfqpoint{0.000000in}{0.000000in}}{%
\pgfpathmoveto{\pgfqpoint{0.000000in}{0.000000in}}%
\pgfpathlineto{\pgfqpoint{0.000000in}{-0.020833in}}%
\pgfusepath{stroke,fill}%
}%
\begin{pgfscope}%
\pgfsys@transformshift{0.903247in}{0.893003in}%
\pgfsys@useobject{currentmarker}{}%
\end{pgfscope}%
\end{pgfscope}%
\begin{pgfscope}%
\pgfsetbuttcap%
\pgfsetroundjoin%
\definecolor{currentfill}{rgb}{0.000000,0.000000,0.000000}%
\pgfsetfillcolor{currentfill}%
\pgfsetlinewidth{0.501875pt}%
\definecolor{currentstroke}{rgb}{0.000000,0.000000,0.000000}%
\pgfsetstrokecolor{currentstroke}%
\pgfsetdash{}{0pt}%
\pgfsys@defobject{currentmarker}{\pgfqpoint{0.000000in}{0.000000in}}{\pgfqpoint{0.000000in}{0.020833in}}{%
\pgfpathmoveto{\pgfqpoint{0.000000in}{0.000000in}}%
\pgfpathlineto{\pgfqpoint{0.000000in}{0.020833in}}%
\pgfusepath{stroke,fill}%
}%
\begin{pgfscope}%
\pgfsys@transformshift{0.970969in}{0.586309in}%
\pgfsys@useobject{currentmarker}{}%
\end{pgfscope}%
\end{pgfscope}%
\begin{pgfscope}%
\pgfsetbuttcap%
\pgfsetroundjoin%
\definecolor{currentfill}{rgb}{0.000000,0.000000,0.000000}%
\pgfsetfillcolor{currentfill}%
\pgfsetlinewidth{0.501875pt}%
\definecolor{currentstroke}{rgb}{0.000000,0.000000,0.000000}%
\pgfsetstrokecolor{currentstroke}%
\pgfsetdash{}{0pt}%
\pgfsys@defobject{currentmarker}{\pgfqpoint{0.000000in}{-0.020833in}}{\pgfqpoint{0.000000in}{0.000000in}}{%
\pgfpathmoveto{\pgfqpoint{0.000000in}{0.000000in}}%
\pgfpathlineto{\pgfqpoint{0.000000in}{-0.020833in}}%
\pgfusepath{stroke,fill}%
}%
\begin{pgfscope}%
\pgfsys@transformshift{0.970969in}{0.893003in}%
\pgfsys@useobject{currentmarker}{}%
\end{pgfscope}%
\end{pgfscope}%
\begin{pgfscope}%
\pgfsetbuttcap%
\pgfsetroundjoin%
\definecolor{currentfill}{rgb}{0.000000,0.000000,0.000000}%
\pgfsetfillcolor{currentfill}%
\pgfsetlinewidth{0.501875pt}%
\definecolor{currentstroke}{rgb}{0.000000,0.000000,0.000000}%
\pgfsetstrokecolor{currentstroke}%
\pgfsetdash{}{0pt}%
\pgfsys@defobject{currentmarker}{\pgfqpoint{0.000000in}{0.000000in}}{\pgfqpoint{0.000000in}{0.020833in}}{%
\pgfpathmoveto{\pgfqpoint{0.000000in}{0.000000in}}%
\pgfpathlineto{\pgfqpoint{0.000000in}{0.020833in}}%
\pgfusepath{stroke,fill}%
}%
\begin{pgfscope}%
\pgfsys@transformshift{1.038690in}{0.586309in}%
\pgfsys@useobject{currentmarker}{}%
\end{pgfscope}%
\end{pgfscope}%
\begin{pgfscope}%
\pgfsetbuttcap%
\pgfsetroundjoin%
\definecolor{currentfill}{rgb}{0.000000,0.000000,0.000000}%
\pgfsetfillcolor{currentfill}%
\pgfsetlinewidth{0.501875pt}%
\definecolor{currentstroke}{rgb}{0.000000,0.000000,0.000000}%
\pgfsetstrokecolor{currentstroke}%
\pgfsetdash{}{0pt}%
\pgfsys@defobject{currentmarker}{\pgfqpoint{0.000000in}{-0.020833in}}{\pgfqpoint{0.000000in}{0.000000in}}{%
\pgfpathmoveto{\pgfqpoint{0.000000in}{0.000000in}}%
\pgfpathlineto{\pgfqpoint{0.000000in}{-0.020833in}}%
\pgfusepath{stroke,fill}%
}%
\begin{pgfscope}%
\pgfsys@transformshift{1.038690in}{0.893003in}%
\pgfsys@useobject{currentmarker}{}%
\end{pgfscope}%
\end{pgfscope}%
\begin{pgfscope}%
\pgfsetbuttcap%
\pgfsetroundjoin%
\definecolor{currentfill}{rgb}{0.000000,0.000000,0.000000}%
\pgfsetfillcolor{currentfill}%
\pgfsetlinewidth{0.501875pt}%
\definecolor{currentstroke}{rgb}{0.000000,0.000000,0.000000}%
\pgfsetstrokecolor{currentstroke}%
\pgfsetdash{}{0pt}%
\pgfsys@defobject{currentmarker}{\pgfqpoint{0.000000in}{0.000000in}}{\pgfqpoint{0.000000in}{0.020833in}}{%
\pgfpathmoveto{\pgfqpoint{0.000000in}{0.000000in}}%
\pgfpathlineto{\pgfqpoint{0.000000in}{0.020833in}}%
\pgfusepath{stroke,fill}%
}%
\begin{pgfscope}%
\pgfsys@transformshift{1.106412in}{0.586309in}%
\pgfsys@useobject{currentmarker}{}%
\end{pgfscope}%
\end{pgfscope}%
\begin{pgfscope}%
\pgfsetbuttcap%
\pgfsetroundjoin%
\definecolor{currentfill}{rgb}{0.000000,0.000000,0.000000}%
\pgfsetfillcolor{currentfill}%
\pgfsetlinewidth{0.501875pt}%
\definecolor{currentstroke}{rgb}{0.000000,0.000000,0.000000}%
\pgfsetstrokecolor{currentstroke}%
\pgfsetdash{}{0pt}%
\pgfsys@defobject{currentmarker}{\pgfqpoint{0.000000in}{-0.020833in}}{\pgfqpoint{0.000000in}{0.000000in}}{%
\pgfpathmoveto{\pgfqpoint{0.000000in}{0.000000in}}%
\pgfpathlineto{\pgfqpoint{0.000000in}{-0.020833in}}%
\pgfusepath{stroke,fill}%
}%
\begin{pgfscope}%
\pgfsys@transformshift{1.106412in}{0.893003in}%
\pgfsys@useobject{currentmarker}{}%
\end{pgfscope}%
\end{pgfscope}%
\begin{pgfscope}%
\pgfsetbuttcap%
\pgfsetroundjoin%
\definecolor{currentfill}{rgb}{0.000000,0.000000,0.000000}%
\pgfsetfillcolor{currentfill}%
\pgfsetlinewidth{0.501875pt}%
\definecolor{currentstroke}{rgb}{0.000000,0.000000,0.000000}%
\pgfsetstrokecolor{currentstroke}%
\pgfsetdash{}{0pt}%
\pgfsys@defobject{currentmarker}{\pgfqpoint{0.000000in}{0.000000in}}{\pgfqpoint{0.000000in}{0.020833in}}{%
\pgfpathmoveto{\pgfqpoint{0.000000in}{0.000000in}}%
\pgfpathlineto{\pgfqpoint{0.000000in}{0.020833in}}%
\pgfusepath{stroke,fill}%
}%
\begin{pgfscope}%
\pgfsys@transformshift{1.241855in}{0.586309in}%
\pgfsys@useobject{currentmarker}{}%
\end{pgfscope}%
\end{pgfscope}%
\begin{pgfscope}%
\pgfsetbuttcap%
\pgfsetroundjoin%
\definecolor{currentfill}{rgb}{0.000000,0.000000,0.000000}%
\pgfsetfillcolor{currentfill}%
\pgfsetlinewidth{0.501875pt}%
\definecolor{currentstroke}{rgb}{0.000000,0.000000,0.000000}%
\pgfsetstrokecolor{currentstroke}%
\pgfsetdash{}{0pt}%
\pgfsys@defobject{currentmarker}{\pgfqpoint{0.000000in}{-0.020833in}}{\pgfqpoint{0.000000in}{0.000000in}}{%
\pgfpathmoveto{\pgfqpoint{0.000000in}{0.000000in}}%
\pgfpathlineto{\pgfqpoint{0.000000in}{-0.020833in}}%
\pgfusepath{stroke,fill}%
}%
\begin{pgfscope}%
\pgfsys@transformshift{1.241855in}{0.893003in}%
\pgfsys@useobject{currentmarker}{}%
\end{pgfscope}%
\end{pgfscope}%
\begin{pgfscope}%
\pgfsetbuttcap%
\pgfsetroundjoin%
\definecolor{currentfill}{rgb}{0.000000,0.000000,0.000000}%
\pgfsetfillcolor{currentfill}%
\pgfsetlinewidth{0.501875pt}%
\definecolor{currentstroke}{rgb}{0.000000,0.000000,0.000000}%
\pgfsetstrokecolor{currentstroke}%
\pgfsetdash{}{0pt}%
\pgfsys@defobject{currentmarker}{\pgfqpoint{0.000000in}{0.000000in}}{\pgfqpoint{0.000000in}{0.020833in}}{%
\pgfpathmoveto{\pgfqpoint{0.000000in}{0.000000in}}%
\pgfpathlineto{\pgfqpoint{0.000000in}{0.020833in}}%
\pgfusepath{stroke,fill}%
}%
\begin{pgfscope}%
\pgfsys@transformshift{1.309576in}{0.586309in}%
\pgfsys@useobject{currentmarker}{}%
\end{pgfscope}%
\end{pgfscope}%
\begin{pgfscope}%
\pgfsetbuttcap%
\pgfsetroundjoin%
\definecolor{currentfill}{rgb}{0.000000,0.000000,0.000000}%
\pgfsetfillcolor{currentfill}%
\pgfsetlinewidth{0.501875pt}%
\definecolor{currentstroke}{rgb}{0.000000,0.000000,0.000000}%
\pgfsetstrokecolor{currentstroke}%
\pgfsetdash{}{0pt}%
\pgfsys@defobject{currentmarker}{\pgfqpoint{0.000000in}{-0.020833in}}{\pgfqpoint{0.000000in}{0.000000in}}{%
\pgfpathmoveto{\pgfqpoint{0.000000in}{0.000000in}}%
\pgfpathlineto{\pgfqpoint{0.000000in}{-0.020833in}}%
\pgfusepath{stroke,fill}%
}%
\begin{pgfscope}%
\pgfsys@transformshift{1.309576in}{0.893003in}%
\pgfsys@useobject{currentmarker}{}%
\end{pgfscope}%
\end{pgfscope}%
\begin{pgfscope}%
\pgfsetbuttcap%
\pgfsetroundjoin%
\definecolor{currentfill}{rgb}{0.000000,0.000000,0.000000}%
\pgfsetfillcolor{currentfill}%
\pgfsetlinewidth{0.501875pt}%
\definecolor{currentstroke}{rgb}{0.000000,0.000000,0.000000}%
\pgfsetstrokecolor{currentstroke}%
\pgfsetdash{}{0pt}%
\pgfsys@defobject{currentmarker}{\pgfqpoint{0.000000in}{0.000000in}}{\pgfqpoint{0.000000in}{0.020833in}}{%
\pgfpathmoveto{\pgfqpoint{0.000000in}{0.000000in}}%
\pgfpathlineto{\pgfqpoint{0.000000in}{0.020833in}}%
\pgfusepath{stroke,fill}%
}%
\begin{pgfscope}%
\pgfsys@transformshift{1.377297in}{0.586309in}%
\pgfsys@useobject{currentmarker}{}%
\end{pgfscope}%
\end{pgfscope}%
\begin{pgfscope}%
\pgfsetbuttcap%
\pgfsetroundjoin%
\definecolor{currentfill}{rgb}{0.000000,0.000000,0.000000}%
\pgfsetfillcolor{currentfill}%
\pgfsetlinewidth{0.501875pt}%
\definecolor{currentstroke}{rgb}{0.000000,0.000000,0.000000}%
\pgfsetstrokecolor{currentstroke}%
\pgfsetdash{}{0pt}%
\pgfsys@defobject{currentmarker}{\pgfqpoint{0.000000in}{-0.020833in}}{\pgfqpoint{0.000000in}{0.000000in}}{%
\pgfpathmoveto{\pgfqpoint{0.000000in}{0.000000in}}%
\pgfpathlineto{\pgfqpoint{0.000000in}{-0.020833in}}%
\pgfusepath{stroke,fill}%
}%
\begin{pgfscope}%
\pgfsys@transformshift{1.377297in}{0.893003in}%
\pgfsys@useobject{currentmarker}{}%
\end{pgfscope}%
\end{pgfscope}%
\begin{pgfscope}%
\pgfsetbuttcap%
\pgfsetroundjoin%
\definecolor{currentfill}{rgb}{0.000000,0.000000,0.000000}%
\pgfsetfillcolor{currentfill}%
\pgfsetlinewidth{0.501875pt}%
\definecolor{currentstroke}{rgb}{0.000000,0.000000,0.000000}%
\pgfsetstrokecolor{currentstroke}%
\pgfsetdash{}{0pt}%
\pgfsys@defobject{currentmarker}{\pgfqpoint{0.000000in}{0.000000in}}{\pgfqpoint{0.000000in}{0.020833in}}{%
\pgfpathmoveto{\pgfqpoint{0.000000in}{0.000000in}}%
\pgfpathlineto{\pgfqpoint{0.000000in}{0.020833in}}%
\pgfusepath{stroke,fill}%
}%
\begin{pgfscope}%
\pgfsys@transformshift{1.445019in}{0.586309in}%
\pgfsys@useobject{currentmarker}{}%
\end{pgfscope}%
\end{pgfscope}%
\begin{pgfscope}%
\pgfsetbuttcap%
\pgfsetroundjoin%
\definecolor{currentfill}{rgb}{0.000000,0.000000,0.000000}%
\pgfsetfillcolor{currentfill}%
\pgfsetlinewidth{0.501875pt}%
\definecolor{currentstroke}{rgb}{0.000000,0.000000,0.000000}%
\pgfsetstrokecolor{currentstroke}%
\pgfsetdash{}{0pt}%
\pgfsys@defobject{currentmarker}{\pgfqpoint{0.000000in}{-0.020833in}}{\pgfqpoint{0.000000in}{0.000000in}}{%
\pgfpathmoveto{\pgfqpoint{0.000000in}{0.000000in}}%
\pgfpathlineto{\pgfqpoint{0.000000in}{-0.020833in}}%
\pgfusepath{stroke,fill}%
}%
\begin{pgfscope}%
\pgfsys@transformshift{1.445019in}{0.893003in}%
\pgfsys@useobject{currentmarker}{}%
\end{pgfscope}%
\end{pgfscope}%
\begin{pgfscope}%
\pgfsetbuttcap%
\pgfsetroundjoin%
\definecolor{currentfill}{rgb}{0.000000,0.000000,0.000000}%
\pgfsetfillcolor{currentfill}%
\pgfsetlinewidth{0.501875pt}%
\definecolor{currentstroke}{rgb}{0.000000,0.000000,0.000000}%
\pgfsetstrokecolor{currentstroke}%
\pgfsetdash{}{0pt}%
\pgfsys@defobject{currentmarker}{\pgfqpoint{0.000000in}{0.000000in}}{\pgfqpoint{0.000000in}{0.020833in}}{%
\pgfpathmoveto{\pgfqpoint{0.000000in}{0.000000in}}%
\pgfpathlineto{\pgfqpoint{0.000000in}{0.020833in}}%
\pgfusepath{stroke,fill}%
}%
\begin{pgfscope}%
\pgfsys@transformshift{1.512740in}{0.586309in}%
\pgfsys@useobject{currentmarker}{}%
\end{pgfscope}%
\end{pgfscope}%
\begin{pgfscope}%
\pgfsetbuttcap%
\pgfsetroundjoin%
\definecolor{currentfill}{rgb}{0.000000,0.000000,0.000000}%
\pgfsetfillcolor{currentfill}%
\pgfsetlinewidth{0.501875pt}%
\definecolor{currentstroke}{rgb}{0.000000,0.000000,0.000000}%
\pgfsetstrokecolor{currentstroke}%
\pgfsetdash{}{0pt}%
\pgfsys@defobject{currentmarker}{\pgfqpoint{0.000000in}{-0.020833in}}{\pgfqpoint{0.000000in}{0.000000in}}{%
\pgfpathmoveto{\pgfqpoint{0.000000in}{0.000000in}}%
\pgfpathlineto{\pgfqpoint{0.000000in}{-0.020833in}}%
\pgfusepath{stroke,fill}%
}%
\begin{pgfscope}%
\pgfsys@transformshift{1.512740in}{0.893003in}%
\pgfsys@useobject{currentmarker}{}%
\end{pgfscope}%
\end{pgfscope}%
\begin{pgfscope}%
\pgfsetbuttcap%
\pgfsetroundjoin%
\definecolor{currentfill}{rgb}{0.000000,0.000000,0.000000}%
\pgfsetfillcolor{currentfill}%
\pgfsetlinewidth{0.501875pt}%
\definecolor{currentstroke}{rgb}{0.000000,0.000000,0.000000}%
\pgfsetstrokecolor{currentstroke}%
\pgfsetdash{}{0pt}%
\pgfsys@defobject{currentmarker}{\pgfqpoint{0.000000in}{0.000000in}}{\pgfqpoint{0.000000in}{0.020833in}}{%
\pgfpathmoveto{\pgfqpoint{0.000000in}{0.000000in}}%
\pgfpathlineto{\pgfqpoint{0.000000in}{0.020833in}}%
\pgfusepath{stroke,fill}%
}%
\begin{pgfscope}%
\pgfsys@transformshift{1.648183in}{0.586309in}%
\pgfsys@useobject{currentmarker}{}%
\end{pgfscope}%
\end{pgfscope}%
\begin{pgfscope}%
\pgfsetbuttcap%
\pgfsetroundjoin%
\definecolor{currentfill}{rgb}{0.000000,0.000000,0.000000}%
\pgfsetfillcolor{currentfill}%
\pgfsetlinewidth{0.501875pt}%
\definecolor{currentstroke}{rgb}{0.000000,0.000000,0.000000}%
\pgfsetstrokecolor{currentstroke}%
\pgfsetdash{}{0pt}%
\pgfsys@defobject{currentmarker}{\pgfqpoint{0.000000in}{-0.020833in}}{\pgfqpoint{0.000000in}{0.000000in}}{%
\pgfpathmoveto{\pgfqpoint{0.000000in}{0.000000in}}%
\pgfpathlineto{\pgfqpoint{0.000000in}{-0.020833in}}%
\pgfusepath{stroke,fill}%
}%
\begin{pgfscope}%
\pgfsys@transformshift{1.648183in}{0.893003in}%
\pgfsys@useobject{currentmarker}{}%
\end{pgfscope}%
\end{pgfscope}%
\begin{pgfscope}%
\pgfsetbuttcap%
\pgfsetroundjoin%
\definecolor{currentfill}{rgb}{0.000000,0.000000,0.000000}%
\pgfsetfillcolor{currentfill}%
\pgfsetlinewidth{0.501875pt}%
\definecolor{currentstroke}{rgb}{0.000000,0.000000,0.000000}%
\pgfsetstrokecolor{currentstroke}%
\pgfsetdash{}{0pt}%
\pgfsys@defobject{currentmarker}{\pgfqpoint{0.000000in}{0.000000in}}{\pgfqpoint{0.000000in}{0.020833in}}{%
\pgfpathmoveto{\pgfqpoint{0.000000in}{0.000000in}}%
\pgfpathlineto{\pgfqpoint{0.000000in}{0.020833in}}%
\pgfusepath{stroke,fill}%
}%
\begin{pgfscope}%
\pgfsys@transformshift{1.715905in}{0.586309in}%
\pgfsys@useobject{currentmarker}{}%
\end{pgfscope}%
\end{pgfscope}%
\begin{pgfscope}%
\pgfsetbuttcap%
\pgfsetroundjoin%
\definecolor{currentfill}{rgb}{0.000000,0.000000,0.000000}%
\pgfsetfillcolor{currentfill}%
\pgfsetlinewidth{0.501875pt}%
\definecolor{currentstroke}{rgb}{0.000000,0.000000,0.000000}%
\pgfsetstrokecolor{currentstroke}%
\pgfsetdash{}{0pt}%
\pgfsys@defobject{currentmarker}{\pgfqpoint{0.000000in}{-0.020833in}}{\pgfqpoint{0.000000in}{0.000000in}}{%
\pgfpathmoveto{\pgfqpoint{0.000000in}{0.000000in}}%
\pgfpathlineto{\pgfqpoint{0.000000in}{-0.020833in}}%
\pgfusepath{stroke,fill}%
}%
\begin{pgfscope}%
\pgfsys@transformshift{1.715905in}{0.893003in}%
\pgfsys@useobject{currentmarker}{}%
\end{pgfscope}%
\end{pgfscope}%
\begin{pgfscope}%
\pgfsetbuttcap%
\pgfsetroundjoin%
\definecolor{currentfill}{rgb}{0.000000,0.000000,0.000000}%
\pgfsetfillcolor{currentfill}%
\pgfsetlinewidth{0.501875pt}%
\definecolor{currentstroke}{rgb}{0.000000,0.000000,0.000000}%
\pgfsetstrokecolor{currentstroke}%
\pgfsetdash{}{0pt}%
\pgfsys@defobject{currentmarker}{\pgfqpoint{0.000000in}{0.000000in}}{\pgfqpoint{0.000000in}{0.020833in}}{%
\pgfpathmoveto{\pgfqpoint{0.000000in}{0.000000in}}%
\pgfpathlineto{\pgfqpoint{0.000000in}{0.020833in}}%
\pgfusepath{stroke,fill}%
}%
\begin{pgfscope}%
\pgfsys@transformshift{1.783626in}{0.586309in}%
\pgfsys@useobject{currentmarker}{}%
\end{pgfscope}%
\end{pgfscope}%
\begin{pgfscope}%
\pgfsetbuttcap%
\pgfsetroundjoin%
\definecolor{currentfill}{rgb}{0.000000,0.000000,0.000000}%
\pgfsetfillcolor{currentfill}%
\pgfsetlinewidth{0.501875pt}%
\definecolor{currentstroke}{rgb}{0.000000,0.000000,0.000000}%
\pgfsetstrokecolor{currentstroke}%
\pgfsetdash{}{0pt}%
\pgfsys@defobject{currentmarker}{\pgfqpoint{0.000000in}{-0.020833in}}{\pgfqpoint{0.000000in}{0.000000in}}{%
\pgfpathmoveto{\pgfqpoint{0.000000in}{0.000000in}}%
\pgfpathlineto{\pgfqpoint{0.000000in}{-0.020833in}}%
\pgfusepath{stroke,fill}%
}%
\begin{pgfscope}%
\pgfsys@transformshift{1.783626in}{0.893003in}%
\pgfsys@useobject{currentmarker}{}%
\end{pgfscope}%
\end{pgfscope}%
\begin{pgfscope}%
\pgfsetbuttcap%
\pgfsetroundjoin%
\definecolor{currentfill}{rgb}{0.000000,0.000000,0.000000}%
\pgfsetfillcolor{currentfill}%
\pgfsetlinewidth{0.501875pt}%
\definecolor{currentstroke}{rgb}{0.000000,0.000000,0.000000}%
\pgfsetstrokecolor{currentstroke}%
\pgfsetdash{}{0pt}%
\pgfsys@defobject{currentmarker}{\pgfqpoint{0.000000in}{0.000000in}}{\pgfqpoint{0.000000in}{0.020833in}}{%
\pgfpathmoveto{\pgfqpoint{0.000000in}{0.000000in}}%
\pgfpathlineto{\pgfqpoint{0.000000in}{0.020833in}}%
\pgfusepath{stroke,fill}%
}%
\begin{pgfscope}%
\pgfsys@transformshift{1.851348in}{0.586309in}%
\pgfsys@useobject{currentmarker}{}%
\end{pgfscope}%
\end{pgfscope}%
\begin{pgfscope}%
\pgfsetbuttcap%
\pgfsetroundjoin%
\definecolor{currentfill}{rgb}{0.000000,0.000000,0.000000}%
\pgfsetfillcolor{currentfill}%
\pgfsetlinewidth{0.501875pt}%
\definecolor{currentstroke}{rgb}{0.000000,0.000000,0.000000}%
\pgfsetstrokecolor{currentstroke}%
\pgfsetdash{}{0pt}%
\pgfsys@defobject{currentmarker}{\pgfqpoint{0.000000in}{-0.020833in}}{\pgfqpoint{0.000000in}{0.000000in}}{%
\pgfpathmoveto{\pgfqpoint{0.000000in}{0.000000in}}%
\pgfpathlineto{\pgfqpoint{0.000000in}{-0.020833in}}%
\pgfusepath{stroke,fill}%
}%
\begin{pgfscope}%
\pgfsys@transformshift{1.851348in}{0.893003in}%
\pgfsys@useobject{currentmarker}{}%
\end{pgfscope}%
\end{pgfscope}%
\begin{pgfscope}%
\pgfsetbuttcap%
\pgfsetroundjoin%
\definecolor{currentfill}{rgb}{0.000000,0.000000,0.000000}%
\pgfsetfillcolor{currentfill}%
\pgfsetlinewidth{0.501875pt}%
\definecolor{currentstroke}{rgb}{0.000000,0.000000,0.000000}%
\pgfsetstrokecolor{currentstroke}%
\pgfsetdash{}{0pt}%
\pgfsys@defobject{currentmarker}{\pgfqpoint{0.000000in}{0.000000in}}{\pgfqpoint{0.000000in}{0.020833in}}{%
\pgfpathmoveto{\pgfqpoint{0.000000in}{0.000000in}}%
\pgfpathlineto{\pgfqpoint{0.000000in}{0.020833in}}%
\pgfusepath{stroke,fill}%
}%
\begin{pgfscope}%
\pgfsys@transformshift{1.919069in}{0.586309in}%
\pgfsys@useobject{currentmarker}{}%
\end{pgfscope}%
\end{pgfscope}%
\begin{pgfscope}%
\pgfsetbuttcap%
\pgfsetroundjoin%
\definecolor{currentfill}{rgb}{0.000000,0.000000,0.000000}%
\pgfsetfillcolor{currentfill}%
\pgfsetlinewidth{0.501875pt}%
\definecolor{currentstroke}{rgb}{0.000000,0.000000,0.000000}%
\pgfsetstrokecolor{currentstroke}%
\pgfsetdash{}{0pt}%
\pgfsys@defobject{currentmarker}{\pgfqpoint{0.000000in}{-0.020833in}}{\pgfqpoint{0.000000in}{0.000000in}}{%
\pgfpathmoveto{\pgfqpoint{0.000000in}{0.000000in}}%
\pgfpathlineto{\pgfqpoint{0.000000in}{-0.020833in}}%
\pgfusepath{stroke,fill}%
}%
\begin{pgfscope}%
\pgfsys@transformshift{1.919069in}{0.893003in}%
\pgfsys@useobject{currentmarker}{}%
\end{pgfscope}%
\end{pgfscope}%
\begin{pgfscope}%
\pgfsetbuttcap%
\pgfsetroundjoin%
\definecolor{currentfill}{rgb}{0.000000,0.000000,0.000000}%
\pgfsetfillcolor{currentfill}%
\pgfsetlinewidth{0.501875pt}%
\definecolor{currentstroke}{rgb}{0.000000,0.000000,0.000000}%
\pgfsetstrokecolor{currentstroke}%
\pgfsetdash{}{0pt}%
\pgfsys@defobject{currentmarker}{\pgfqpoint{0.000000in}{0.000000in}}{\pgfqpoint{0.000000in}{0.020833in}}{%
\pgfpathmoveto{\pgfqpoint{0.000000in}{0.000000in}}%
\pgfpathlineto{\pgfqpoint{0.000000in}{0.020833in}}%
\pgfusepath{stroke,fill}%
}%
\begin{pgfscope}%
\pgfsys@transformshift{2.054512in}{0.586309in}%
\pgfsys@useobject{currentmarker}{}%
\end{pgfscope}%
\end{pgfscope}%
\begin{pgfscope}%
\pgfsetbuttcap%
\pgfsetroundjoin%
\definecolor{currentfill}{rgb}{0.000000,0.000000,0.000000}%
\pgfsetfillcolor{currentfill}%
\pgfsetlinewidth{0.501875pt}%
\definecolor{currentstroke}{rgb}{0.000000,0.000000,0.000000}%
\pgfsetstrokecolor{currentstroke}%
\pgfsetdash{}{0pt}%
\pgfsys@defobject{currentmarker}{\pgfqpoint{0.000000in}{-0.020833in}}{\pgfqpoint{0.000000in}{0.000000in}}{%
\pgfpathmoveto{\pgfqpoint{0.000000in}{0.000000in}}%
\pgfpathlineto{\pgfqpoint{0.000000in}{-0.020833in}}%
\pgfusepath{stroke,fill}%
}%
\begin{pgfscope}%
\pgfsys@transformshift{2.054512in}{0.893003in}%
\pgfsys@useobject{currentmarker}{}%
\end{pgfscope}%
\end{pgfscope}%
\begin{pgfscope}%
\pgfsetbuttcap%
\pgfsetroundjoin%
\definecolor{currentfill}{rgb}{0.000000,0.000000,0.000000}%
\pgfsetfillcolor{currentfill}%
\pgfsetlinewidth{0.501875pt}%
\definecolor{currentstroke}{rgb}{0.000000,0.000000,0.000000}%
\pgfsetstrokecolor{currentstroke}%
\pgfsetdash{}{0pt}%
\pgfsys@defobject{currentmarker}{\pgfqpoint{0.000000in}{0.000000in}}{\pgfqpoint{0.000000in}{0.020833in}}{%
\pgfpathmoveto{\pgfqpoint{0.000000in}{0.000000in}}%
\pgfpathlineto{\pgfqpoint{0.000000in}{0.020833in}}%
\pgfusepath{stroke,fill}%
}%
\begin{pgfscope}%
\pgfsys@transformshift{2.122234in}{0.586309in}%
\pgfsys@useobject{currentmarker}{}%
\end{pgfscope}%
\end{pgfscope}%
\begin{pgfscope}%
\pgfsetbuttcap%
\pgfsetroundjoin%
\definecolor{currentfill}{rgb}{0.000000,0.000000,0.000000}%
\pgfsetfillcolor{currentfill}%
\pgfsetlinewidth{0.501875pt}%
\definecolor{currentstroke}{rgb}{0.000000,0.000000,0.000000}%
\pgfsetstrokecolor{currentstroke}%
\pgfsetdash{}{0pt}%
\pgfsys@defobject{currentmarker}{\pgfqpoint{0.000000in}{-0.020833in}}{\pgfqpoint{0.000000in}{0.000000in}}{%
\pgfpathmoveto{\pgfqpoint{0.000000in}{0.000000in}}%
\pgfpathlineto{\pgfqpoint{0.000000in}{-0.020833in}}%
\pgfusepath{stroke,fill}%
}%
\begin{pgfscope}%
\pgfsys@transformshift{2.122234in}{0.893003in}%
\pgfsys@useobject{currentmarker}{}%
\end{pgfscope}%
\end{pgfscope}%
\begin{pgfscope}%
\pgfsetbuttcap%
\pgfsetroundjoin%
\definecolor{currentfill}{rgb}{0.000000,0.000000,0.000000}%
\pgfsetfillcolor{currentfill}%
\pgfsetlinewidth{0.501875pt}%
\definecolor{currentstroke}{rgb}{0.000000,0.000000,0.000000}%
\pgfsetstrokecolor{currentstroke}%
\pgfsetdash{}{0pt}%
\pgfsys@defobject{currentmarker}{\pgfqpoint{0.000000in}{0.000000in}}{\pgfqpoint{0.000000in}{0.020833in}}{%
\pgfpathmoveto{\pgfqpoint{0.000000in}{0.000000in}}%
\pgfpathlineto{\pgfqpoint{0.000000in}{0.020833in}}%
\pgfusepath{stroke,fill}%
}%
\begin{pgfscope}%
\pgfsys@transformshift{2.189955in}{0.586309in}%
\pgfsys@useobject{currentmarker}{}%
\end{pgfscope}%
\end{pgfscope}%
\begin{pgfscope}%
\pgfsetbuttcap%
\pgfsetroundjoin%
\definecolor{currentfill}{rgb}{0.000000,0.000000,0.000000}%
\pgfsetfillcolor{currentfill}%
\pgfsetlinewidth{0.501875pt}%
\definecolor{currentstroke}{rgb}{0.000000,0.000000,0.000000}%
\pgfsetstrokecolor{currentstroke}%
\pgfsetdash{}{0pt}%
\pgfsys@defobject{currentmarker}{\pgfqpoint{0.000000in}{-0.020833in}}{\pgfqpoint{0.000000in}{0.000000in}}{%
\pgfpathmoveto{\pgfqpoint{0.000000in}{0.000000in}}%
\pgfpathlineto{\pgfqpoint{0.000000in}{-0.020833in}}%
\pgfusepath{stroke,fill}%
}%
\begin{pgfscope}%
\pgfsys@transformshift{2.189955in}{0.893003in}%
\pgfsys@useobject{currentmarker}{}%
\end{pgfscope}%
\end{pgfscope}%
\begin{pgfscope}%
\pgfsetbuttcap%
\pgfsetroundjoin%
\definecolor{currentfill}{rgb}{0.000000,0.000000,0.000000}%
\pgfsetfillcolor{currentfill}%
\pgfsetlinewidth{0.501875pt}%
\definecolor{currentstroke}{rgb}{0.000000,0.000000,0.000000}%
\pgfsetstrokecolor{currentstroke}%
\pgfsetdash{}{0pt}%
\pgfsys@defobject{currentmarker}{\pgfqpoint{0.000000in}{0.000000in}}{\pgfqpoint{0.000000in}{0.020833in}}{%
\pgfpathmoveto{\pgfqpoint{0.000000in}{0.000000in}}%
\pgfpathlineto{\pgfqpoint{0.000000in}{0.020833in}}%
\pgfusepath{stroke,fill}%
}%
\begin{pgfscope}%
\pgfsys@transformshift{2.257677in}{0.586309in}%
\pgfsys@useobject{currentmarker}{}%
\end{pgfscope}%
\end{pgfscope}%
\begin{pgfscope}%
\pgfsetbuttcap%
\pgfsetroundjoin%
\definecolor{currentfill}{rgb}{0.000000,0.000000,0.000000}%
\pgfsetfillcolor{currentfill}%
\pgfsetlinewidth{0.501875pt}%
\definecolor{currentstroke}{rgb}{0.000000,0.000000,0.000000}%
\pgfsetstrokecolor{currentstroke}%
\pgfsetdash{}{0pt}%
\pgfsys@defobject{currentmarker}{\pgfqpoint{0.000000in}{-0.020833in}}{\pgfqpoint{0.000000in}{0.000000in}}{%
\pgfpathmoveto{\pgfqpoint{0.000000in}{0.000000in}}%
\pgfpathlineto{\pgfqpoint{0.000000in}{-0.020833in}}%
\pgfusepath{stroke,fill}%
}%
\begin{pgfscope}%
\pgfsys@transformshift{2.257677in}{0.893003in}%
\pgfsys@useobject{currentmarker}{}%
\end{pgfscope}%
\end{pgfscope}%
\begin{pgfscope}%
\pgfsetbuttcap%
\pgfsetroundjoin%
\definecolor{currentfill}{rgb}{0.000000,0.000000,0.000000}%
\pgfsetfillcolor{currentfill}%
\pgfsetlinewidth{0.501875pt}%
\definecolor{currentstroke}{rgb}{0.000000,0.000000,0.000000}%
\pgfsetstrokecolor{currentstroke}%
\pgfsetdash{}{0pt}%
\pgfsys@defobject{currentmarker}{\pgfqpoint{0.000000in}{0.000000in}}{\pgfqpoint{0.000000in}{0.020833in}}{%
\pgfpathmoveto{\pgfqpoint{0.000000in}{0.000000in}}%
\pgfpathlineto{\pgfqpoint{0.000000in}{0.020833in}}%
\pgfusepath{stroke,fill}%
}%
\begin{pgfscope}%
\pgfsys@transformshift{2.325398in}{0.586309in}%
\pgfsys@useobject{currentmarker}{}%
\end{pgfscope}%
\end{pgfscope}%
\begin{pgfscope}%
\pgfsetbuttcap%
\pgfsetroundjoin%
\definecolor{currentfill}{rgb}{0.000000,0.000000,0.000000}%
\pgfsetfillcolor{currentfill}%
\pgfsetlinewidth{0.501875pt}%
\definecolor{currentstroke}{rgb}{0.000000,0.000000,0.000000}%
\pgfsetstrokecolor{currentstroke}%
\pgfsetdash{}{0pt}%
\pgfsys@defobject{currentmarker}{\pgfqpoint{0.000000in}{-0.020833in}}{\pgfqpoint{0.000000in}{0.000000in}}{%
\pgfpathmoveto{\pgfqpoint{0.000000in}{0.000000in}}%
\pgfpathlineto{\pgfqpoint{0.000000in}{-0.020833in}}%
\pgfusepath{stroke,fill}%
}%
\begin{pgfscope}%
\pgfsys@transformshift{2.325398in}{0.893003in}%
\pgfsys@useobject{currentmarker}{}%
\end{pgfscope}%
\end{pgfscope}%
\begin{pgfscope}%
\pgfsetbuttcap%
\pgfsetroundjoin%
\definecolor{currentfill}{rgb}{0.000000,0.000000,0.000000}%
\pgfsetfillcolor{currentfill}%
\pgfsetlinewidth{0.501875pt}%
\definecolor{currentstroke}{rgb}{0.000000,0.000000,0.000000}%
\pgfsetstrokecolor{currentstroke}%
\pgfsetdash{}{0pt}%
\pgfsys@defobject{currentmarker}{\pgfqpoint{0.000000in}{0.000000in}}{\pgfqpoint{0.000000in}{0.020833in}}{%
\pgfpathmoveto{\pgfqpoint{0.000000in}{0.000000in}}%
\pgfpathlineto{\pgfqpoint{0.000000in}{0.020833in}}%
\pgfusepath{stroke,fill}%
}%
\begin{pgfscope}%
\pgfsys@transformshift{2.460841in}{0.586309in}%
\pgfsys@useobject{currentmarker}{}%
\end{pgfscope}%
\end{pgfscope}%
\begin{pgfscope}%
\pgfsetbuttcap%
\pgfsetroundjoin%
\definecolor{currentfill}{rgb}{0.000000,0.000000,0.000000}%
\pgfsetfillcolor{currentfill}%
\pgfsetlinewidth{0.501875pt}%
\definecolor{currentstroke}{rgb}{0.000000,0.000000,0.000000}%
\pgfsetstrokecolor{currentstroke}%
\pgfsetdash{}{0pt}%
\pgfsys@defobject{currentmarker}{\pgfqpoint{0.000000in}{-0.020833in}}{\pgfqpoint{0.000000in}{0.000000in}}{%
\pgfpathmoveto{\pgfqpoint{0.000000in}{0.000000in}}%
\pgfpathlineto{\pgfqpoint{0.000000in}{-0.020833in}}%
\pgfusepath{stroke,fill}%
}%
\begin{pgfscope}%
\pgfsys@transformshift{2.460841in}{0.893003in}%
\pgfsys@useobject{currentmarker}{}%
\end{pgfscope}%
\end{pgfscope}%
\begin{pgfscope}%
\pgfsetbuttcap%
\pgfsetroundjoin%
\definecolor{currentfill}{rgb}{0.000000,0.000000,0.000000}%
\pgfsetfillcolor{currentfill}%
\pgfsetlinewidth{0.501875pt}%
\definecolor{currentstroke}{rgb}{0.000000,0.000000,0.000000}%
\pgfsetstrokecolor{currentstroke}%
\pgfsetdash{}{0pt}%
\pgfsys@defobject{currentmarker}{\pgfqpoint{0.000000in}{0.000000in}}{\pgfqpoint{0.000000in}{0.020833in}}{%
\pgfpathmoveto{\pgfqpoint{0.000000in}{0.000000in}}%
\pgfpathlineto{\pgfqpoint{0.000000in}{0.020833in}}%
\pgfusepath{stroke,fill}%
}%
\begin{pgfscope}%
\pgfsys@transformshift{2.528563in}{0.586309in}%
\pgfsys@useobject{currentmarker}{}%
\end{pgfscope}%
\end{pgfscope}%
\begin{pgfscope}%
\pgfsetbuttcap%
\pgfsetroundjoin%
\definecolor{currentfill}{rgb}{0.000000,0.000000,0.000000}%
\pgfsetfillcolor{currentfill}%
\pgfsetlinewidth{0.501875pt}%
\definecolor{currentstroke}{rgb}{0.000000,0.000000,0.000000}%
\pgfsetstrokecolor{currentstroke}%
\pgfsetdash{}{0pt}%
\pgfsys@defobject{currentmarker}{\pgfqpoint{0.000000in}{-0.020833in}}{\pgfqpoint{0.000000in}{0.000000in}}{%
\pgfpathmoveto{\pgfqpoint{0.000000in}{0.000000in}}%
\pgfpathlineto{\pgfqpoint{0.000000in}{-0.020833in}}%
\pgfusepath{stroke,fill}%
}%
\begin{pgfscope}%
\pgfsys@transformshift{2.528563in}{0.893003in}%
\pgfsys@useobject{currentmarker}{}%
\end{pgfscope}%
\end{pgfscope}%
\begin{pgfscope}%
\pgfsetbuttcap%
\pgfsetroundjoin%
\definecolor{currentfill}{rgb}{0.000000,0.000000,0.000000}%
\pgfsetfillcolor{currentfill}%
\pgfsetlinewidth{0.501875pt}%
\definecolor{currentstroke}{rgb}{0.000000,0.000000,0.000000}%
\pgfsetstrokecolor{currentstroke}%
\pgfsetdash{}{0pt}%
\pgfsys@defobject{currentmarker}{\pgfqpoint{0.000000in}{0.000000in}}{\pgfqpoint{0.000000in}{0.020833in}}{%
\pgfpathmoveto{\pgfqpoint{0.000000in}{0.000000in}}%
\pgfpathlineto{\pgfqpoint{0.000000in}{0.020833in}}%
\pgfusepath{stroke,fill}%
}%
\begin{pgfscope}%
\pgfsys@transformshift{2.596284in}{0.586309in}%
\pgfsys@useobject{currentmarker}{}%
\end{pgfscope}%
\end{pgfscope}%
\begin{pgfscope}%
\pgfsetbuttcap%
\pgfsetroundjoin%
\definecolor{currentfill}{rgb}{0.000000,0.000000,0.000000}%
\pgfsetfillcolor{currentfill}%
\pgfsetlinewidth{0.501875pt}%
\definecolor{currentstroke}{rgb}{0.000000,0.000000,0.000000}%
\pgfsetstrokecolor{currentstroke}%
\pgfsetdash{}{0pt}%
\pgfsys@defobject{currentmarker}{\pgfqpoint{0.000000in}{-0.020833in}}{\pgfqpoint{0.000000in}{0.000000in}}{%
\pgfpathmoveto{\pgfqpoint{0.000000in}{0.000000in}}%
\pgfpathlineto{\pgfqpoint{0.000000in}{-0.020833in}}%
\pgfusepath{stroke,fill}%
}%
\begin{pgfscope}%
\pgfsys@transformshift{2.596284in}{0.893003in}%
\pgfsys@useobject{currentmarker}{}%
\end{pgfscope}%
\end{pgfscope}%
\begin{pgfscope}%
\pgfsetbuttcap%
\pgfsetroundjoin%
\definecolor{currentfill}{rgb}{0.000000,0.000000,0.000000}%
\pgfsetfillcolor{currentfill}%
\pgfsetlinewidth{0.501875pt}%
\definecolor{currentstroke}{rgb}{0.000000,0.000000,0.000000}%
\pgfsetstrokecolor{currentstroke}%
\pgfsetdash{}{0pt}%
\pgfsys@defobject{currentmarker}{\pgfqpoint{0.000000in}{0.000000in}}{\pgfqpoint{0.000000in}{0.020833in}}{%
\pgfpathmoveto{\pgfqpoint{0.000000in}{0.000000in}}%
\pgfpathlineto{\pgfqpoint{0.000000in}{0.020833in}}%
\pgfusepath{stroke,fill}%
}%
\begin{pgfscope}%
\pgfsys@transformshift{2.664006in}{0.586309in}%
\pgfsys@useobject{currentmarker}{}%
\end{pgfscope}%
\end{pgfscope}%
\begin{pgfscope}%
\pgfsetbuttcap%
\pgfsetroundjoin%
\definecolor{currentfill}{rgb}{0.000000,0.000000,0.000000}%
\pgfsetfillcolor{currentfill}%
\pgfsetlinewidth{0.501875pt}%
\definecolor{currentstroke}{rgb}{0.000000,0.000000,0.000000}%
\pgfsetstrokecolor{currentstroke}%
\pgfsetdash{}{0pt}%
\pgfsys@defobject{currentmarker}{\pgfqpoint{0.000000in}{-0.020833in}}{\pgfqpoint{0.000000in}{0.000000in}}{%
\pgfpathmoveto{\pgfqpoint{0.000000in}{0.000000in}}%
\pgfpathlineto{\pgfqpoint{0.000000in}{-0.020833in}}%
\pgfusepath{stroke,fill}%
}%
\begin{pgfscope}%
\pgfsys@transformshift{2.664006in}{0.893003in}%
\pgfsys@useobject{currentmarker}{}%
\end{pgfscope}%
\end{pgfscope}%
\begin{pgfscope}%
\pgfsetbuttcap%
\pgfsetroundjoin%
\definecolor{currentfill}{rgb}{0.000000,0.000000,0.000000}%
\pgfsetfillcolor{currentfill}%
\pgfsetlinewidth{0.501875pt}%
\definecolor{currentstroke}{rgb}{0.000000,0.000000,0.000000}%
\pgfsetstrokecolor{currentstroke}%
\pgfsetdash{}{0pt}%
\pgfsys@defobject{currentmarker}{\pgfqpoint{0.000000in}{0.000000in}}{\pgfqpoint{0.000000in}{0.020833in}}{%
\pgfpathmoveto{\pgfqpoint{0.000000in}{0.000000in}}%
\pgfpathlineto{\pgfqpoint{0.000000in}{0.020833in}}%
\pgfusepath{stroke,fill}%
}%
\begin{pgfscope}%
\pgfsys@transformshift{2.731727in}{0.586309in}%
\pgfsys@useobject{currentmarker}{}%
\end{pgfscope}%
\end{pgfscope}%
\begin{pgfscope}%
\pgfsetbuttcap%
\pgfsetroundjoin%
\definecolor{currentfill}{rgb}{0.000000,0.000000,0.000000}%
\pgfsetfillcolor{currentfill}%
\pgfsetlinewidth{0.501875pt}%
\definecolor{currentstroke}{rgb}{0.000000,0.000000,0.000000}%
\pgfsetstrokecolor{currentstroke}%
\pgfsetdash{}{0pt}%
\pgfsys@defobject{currentmarker}{\pgfqpoint{0.000000in}{-0.020833in}}{\pgfqpoint{0.000000in}{0.000000in}}{%
\pgfpathmoveto{\pgfqpoint{0.000000in}{0.000000in}}%
\pgfpathlineto{\pgfqpoint{0.000000in}{-0.020833in}}%
\pgfusepath{stroke,fill}%
}%
\begin{pgfscope}%
\pgfsys@transformshift{2.731727in}{0.893003in}%
\pgfsys@useobject{currentmarker}{}%
\end{pgfscope}%
\end{pgfscope}%
\begin{pgfscope}%
\pgfsetbuttcap%
\pgfsetroundjoin%
\definecolor{currentfill}{rgb}{0.000000,0.000000,0.000000}%
\pgfsetfillcolor{currentfill}%
\pgfsetlinewidth{0.501875pt}%
\definecolor{currentstroke}{rgb}{0.000000,0.000000,0.000000}%
\pgfsetstrokecolor{currentstroke}%
\pgfsetdash{}{0pt}%
\pgfsys@defobject{currentmarker}{\pgfqpoint{0.000000in}{0.000000in}}{\pgfqpoint{0.000000in}{0.020833in}}{%
\pgfpathmoveto{\pgfqpoint{0.000000in}{0.000000in}}%
\pgfpathlineto{\pgfqpoint{0.000000in}{0.020833in}}%
\pgfusepath{stroke,fill}%
}%
\begin{pgfscope}%
\pgfsys@transformshift{2.867170in}{0.586309in}%
\pgfsys@useobject{currentmarker}{}%
\end{pgfscope}%
\end{pgfscope}%
\begin{pgfscope}%
\pgfsetbuttcap%
\pgfsetroundjoin%
\definecolor{currentfill}{rgb}{0.000000,0.000000,0.000000}%
\pgfsetfillcolor{currentfill}%
\pgfsetlinewidth{0.501875pt}%
\definecolor{currentstroke}{rgb}{0.000000,0.000000,0.000000}%
\pgfsetstrokecolor{currentstroke}%
\pgfsetdash{}{0pt}%
\pgfsys@defobject{currentmarker}{\pgfqpoint{0.000000in}{-0.020833in}}{\pgfqpoint{0.000000in}{0.000000in}}{%
\pgfpathmoveto{\pgfqpoint{0.000000in}{0.000000in}}%
\pgfpathlineto{\pgfqpoint{0.000000in}{-0.020833in}}%
\pgfusepath{stroke,fill}%
}%
\begin{pgfscope}%
\pgfsys@transformshift{2.867170in}{0.893003in}%
\pgfsys@useobject{currentmarker}{}%
\end{pgfscope}%
\end{pgfscope}%
\begin{pgfscope}%
\pgfsetbuttcap%
\pgfsetroundjoin%
\definecolor{currentfill}{rgb}{0.000000,0.000000,0.000000}%
\pgfsetfillcolor{currentfill}%
\pgfsetlinewidth{0.501875pt}%
\definecolor{currentstroke}{rgb}{0.000000,0.000000,0.000000}%
\pgfsetstrokecolor{currentstroke}%
\pgfsetdash{}{0pt}%
\pgfsys@defobject{currentmarker}{\pgfqpoint{0.000000in}{0.000000in}}{\pgfqpoint{0.000000in}{0.020833in}}{%
\pgfpathmoveto{\pgfqpoint{0.000000in}{0.000000in}}%
\pgfpathlineto{\pgfqpoint{0.000000in}{0.020833in}}%
\pgfusepath{stroke,fill}%
}%
\begin{pgfscope}%
\pgfsys@transformshift{2.934892in}{0.586309in}%
\pgfsys@useobject{currentmarker}{}%
\end{pgfscope}%
\end{pgfscope}%
\begin{pgfscope}%
\pgfsetbuttcap%
\pgfsetroundjoin%
\definecolor{currentfill}{rgb}{0.000000,0.000000,0.000000}%
\pgfsetfillcolor{currentfill}%
\pgfsetlinewidth{0.501875pt}%
\definecolor{currentstroke}{rgb}{0.000000,0.000000,0.000000}%
\pgfsetstrokecolor{currentstroke}%
\pgfsetdash{}{0pt}%
\pgfsys@defobject{currentmarker}{\pgfqpoint{0.000000in}{-0.020833in}}{\pgfqpoint{0.000000in}{0.000000in}}{%
\pgfpathmoveto{\pgfqpoint{0.000000in}{0.000000in}}%
\pgfpathlineto{\pgfqpoint{0.000000in}{-0.020833in}}%
\pgfusepath{stroke,fill}%
}%
\begin{pgfscope}%
\pgfsys@transformshift{2.934892in}{0.893003in}%
\pgfsys@useobject{currentmarker}{}%
\end{pgfscope}%
\end{pgfscope}%
\begin{pgfscope}%
\pgfsetbuttcap%
\pgfsetroundjoin%
\definecolor{currentfill}{rgb}{0.000000,0.000000,0.000000}%
\pgfsetfillcolor{currentfill}%
\pgfsetlinewidth{0.501875pt}%
\definecolor{currentstroke}{rgb}{0.000000,0.000000,0.000000}%
\pgfsetstrokecolor{currentstroke}%
\pgfsetdash{}{0pt}%
\pgfsys@defobject{currentmarker}{\pgfqpoint{0.000000in}{0.000000in}}{\pgfqpoint{0.000000in}{0.020833in}}{%
\pgfpathmoveto{\pgfqpoint{0.000000in}{0.000000in}}%
\pgfpathlineto{\pgfqpoint{0.000000in}{0.020833in}}%
\pgfusepath{stroke,fill}%
}%
\begin{pgfscope}%
\pgfsys@transformshift{3.002613in}{0.586309in}%
\pgfsys@useobject{currentmarker}{}%
\end{pgfscope}%
\end{pgfscope}%
\begin{pgfscope}%
\pgfsetbuttcap%
\pgfsetroundjoin%
\definecolor{currentfill}{rgb}{0.000000,0.000000,0.000000}%
\pgfsetfillcolor{currentfill}%
\pgfsetlinewidth{0.501875pt}%
\definecolor{currentstroke}{rgb}{0.000000,0.000000,0.000000}%
\pgfsetstrokecolor{currentstroke}%
\pgfsetdash{}{0pt}%
\pgfsys@defobject{currentmarker}{\pgfqpoint{0.000000in}{-0.020833in}}{\pgfqpoint{0.000000in}{0.000000in}}{%
\pgfpathmoveto{\pgfqpoint{0.000000in}{0.000000in}}%
\pgfpathlineto{\pgfqpoint{0.000000in}{-0.020833in}}%
\pgfusepath{stroke,fill}%
}%
\begin{pgfscope}%
\pgfsys@transformshift{3.002613in}{0.893003in}%
\pgfsys@useobject{currentmarker}{}%
\end{pgfscope}%
\end{pgfscope}%
\begin{pgfscope}%
\pgfsetbuttcap%
\pgfsetroundjoin%
\definecolor{currentfill}{rgb}{0.000000,0.000000,0.000000}%
\pgfsetfillcolor{currentfill}%
\pgfsetlinewidth{0.501875pt}%
\definecolor{currentstroke}{rgb}{0.000000,0.000000,0.000000}%
\pgfsetstrokecolor{currentstroke}%
\pgfsetdash{}{0pt}%
\pgfsys@defobject{currentmarker}{\pgfqpoint{0.000000in}{0.000000in}}{\pgfqpoint{0.000000in}{0.020833in}}{%
\pgfpathmoveto{\pgfqpoint{0.000000in}{0.000000in}}%
\pgfpathlineto{\pgfqpoint{0.000000in}{0.020833in}}%
\pgfusepath{stroke,fill}%
}%
\begin{pgfscope}%
\pgfsys@transformshift{3.070335in}{0.586309in}%
\pgfsys@useobject{currentmarker}{}%
\end{pgfscope}%
\end{pgfscope}%
\begin{pgfscope}%
\pgfsetbuttcap%
\pgfsetroundjoin%
\definecolor{currentfill}{rgb}{0.000000,0.000000,0.000000}%
\pgfsetfillcolor{currentfill}%
\pgfsetlinewidth{0.501875pt}%
\definecolor{currentstroke}{rgb}{0.000000,0.000000,0.000000}%
\pgfsetstrokecolor{currentstroke}%
\pgfsetdash{}{0pt}%
\pgfsys@defobject{currentmarker}{\pgfqpoint{0.000000in}{-0.020833in}}{\pgfqpoint{0.000000in}{0.000000in}}{%
\pgfpathmoveto{\pgfqpoint{0.000000in}{0.000000in}}%
\pgfpathlineto{\pgfqpoint{0.000000in}{-0.020833in}}%
\pgfusepath{stroke,fill}%
}%
\begin{pgfscope}%
\pgfsys@transformshift{3.070335in}{0.893003in}%
\pgfsys@useobject{currentmarker}{}%
\end{pgfscope}%
\end{pgfscope}%
\begin{pgfscope}%
\pgfsetbuttcap%
\pgfsetroundjoin%
\definecolor{currentfill}{rgb}{0.000000,0.000000,0.000000}%
\pgfsetfillcolor{currentfill}%
\pgfsetlinewidth{0.501875pt}%
\definecolor{currentstroke}{rgb}{0.000000,0.000000,0.000000}%
\pgfsetstrokecolor{currentstroke}%
\pgfsetdash{}{0pt}%
\pgfsys@defobject{currentmarker}{\pgfqpoint{0.000000in}{0.000000in}}{\pgfqpoint{0.000000in}{0.020833in}}{%
\pgfpathmoveto{\pgfqpoint{0.000000in}{0.000000in}}%
\pgfpathlineto{\pgfqpoint{0.000000in}{0.020833in}}%
\pgfusepath{stroke,fill}%
}%
\begin{pgfscope}%
\pgfsys@transformshift{3.138056in}{0.586309in}%
\pgfsys@useobject{currentmarker}{}%
\end{pgfscope}%
\end{pgfscope}%
\begin{pgfscope}%
\pgfsetbuttcap%
\pgfsetroundjoin%
\definecolor{currentfill}{rgb}{0.000000,0.000000,0.000000}%
\pgfsetfillcolor{currentfill}%
\pgfsetlinewidth{0.501875pt}%
\definecolor{currentstroke}{rgb}{0.000000,0.000000,0.000000}%
\pgfsetstrokecolor{currentstroke}%
\pgfsetdash{}{0pt}%
\pgfsys@defobject{currentmarker}{\pgfqpoint{0.000000in}{-0.020833in}}{\pgfqpoint{0.000000in}{0.000000in}}{%
\pgfpathmoveto{\pgfqpoint{0.000000in}{0.000000in}}%
\pgfpathlineto{\pgfqpoint{0.000000in}{-0.020833in}}%
\pgfusepath{stroke,fill}%
}%
\begin{pgfscope}%
\pgfsys@transformshift{3.138056in}{0.893003in}%
\pgfsys@useobject{currentmarker}{}%
\end{pgfscope}%
\end{pgfscope}%
\begin{pgfscope}%
\pgfsetbuttcap%
\pgfsetroundjoin%
\definecolor{currentfill}{rgb}{0.000000,0.000000,0.000000}%
\pgfsetfillcolor{currentfill}%
\pgfsetlinewidth{0.501875pt}%
\definecolor{currentstroke}{rgb}{0.000000,0.000000,0.000000}%
\pgfsetstrokecolor{currentstroke}%
\pgfsetdash{}{0pt}%
\pgfsys@defobject{currentmarker}{\pgfqpoint{0.000000in}{0.000000in}}{\pgfqpoint{0.000000in}{0.020833in}}{%
\pgfpathmoveto{\pgfqpoint{0.000000in}{0.000000in}}%
\pgfpathlineto{\pgfqpoint{0.000000in}{0.020833in}}%
\pgfusepath{stroke,fill}%
}%
\begin{pgfscope}%
\pgfsys@transformshift{3.273499in}{0.586309in}%
\pgfsys@useobject{currentmarker}{}%
\end{pgfscope}%
\end{pgfscope}%
\begin{pgfscope}%
\pgfsetbuttcap%
\pgfsetroundjoin%
\definecolor{currentfill}{rgb}{0.000000,0.000000,0.000000}%
\pgfsetfillcolor{currentfill}%
\pgfsetlinewidth{0.501875pt}%
\definecolor{currentstroke}{rgb}{0.000000,0.000000,0.000000}%
\pgfsetstrokecolor{currentstroke}%
\pgfsetdash{}{0pt}%
\pgfsys@defobject{currentmarker}{\pgfqpoint{0.000000in}{-0.020833in}}{\pgfqpoint{0.000000in}{0.000000in}}{%
\pgfpathmoveto{\pgfqpoint{0.000000in}{0.000000in}}%
\pgfpathlineto{\pgfqpoint{0.000000in}{-0.020833in}}%
\pgfusepath{stroke,fill}%
}%
\begin{pgfscope}%
\pgfsys@transformshift{3.273499in}{0.893003in}%
\pgfsys@useobject{currentmarker}{}%
\end{pgfscope}%
\end{pgfscope}%
\begin{pgfscope}%
\pgfsetbuttcap%
\pgfsetroundjoin%
\definecolor{currentfill}{rgb}{0.000000,0.000000,0.000000}%
\pgfsetfillcolor{currentfill}%
\pgfsetlinewidth{0.501875pt}%
\definecolor{currentstroke}{rgb}{0.000000,0.000000,0.000000}%
\pgfsetstrokecolor{currentstroke}%
\pgfsetdash{}{0pt}%
\pgfsys@defobject{currentmarker}{\pgfqpoint{0.000000in}{0.000000in}}{\pgfqpoint{0.000000in}{0.020833in}}{%
\pgfpathmoveto{\pgfqpoint{0.000000in}{0.000000in}}%
\pgfpathlineto{\pgfqpoint{0.000000in}{0.020833in}}%
\pgfusepath{stroke,fill}%
}%
\begin{pgfscope}%
\pgfsys@transformshift{3.341221in}{0.586309in}%
\pgfsys@useobject{currentmarker}{}%
\end{pgfscope}%
\end{pgfscope}%
\begin{pgfscope}%
\pgfsetbuttcap%
\pgfsetroundjoin%
\definecolor{currentfill}{rgb}{0.000000,0.000000,0.000000}%
\pgfsetfillcolor{currentfill}%
\pgfsetlinewidth{0.501875pt}%
\definecolor{currentstroke}{rgb}{0.000000,0.000000,0.000000}%
\pgfsetstrokecolor{currentstroke}%
\pgfsetdash{}{0pt}%
\pgfsys@defobject{currentmarker}{\pgfqpoint{0.000000in}{-0.020833in}}{\pgfqpoint{0.000000in}{0.000000in}}{%
\pgfpathmoveto{\pgfqpoint{0.000000in}{0.000000in}}%
\pgfpathlineto{\pgfqpoint{0.000000in}{-0.020833in}}%
\pgfusepath{stroke,fill}%
}%
\begin{pgfscope}%
\pgfsys@transformshift{3.341221in}{0.893003in}%
\pgfsys@useobject{currentmarker}{}%
\end{pgfscope}%
\end{pgfscope}%
\begin{pgfscope}%
\pgfsetbuttcap%
\pgfsetroundjoin%
\definecolor{currentfill}{rgb}{0.000000,0.000000,0.000000}%
\pgfsetfillcolor{currentfill}%
\pgfsetlinewidth{0.501875pt}%
\definecolor{currentstroke}{rgb}{0.000000,0.000000,0.000000}%
\pgfsetstrokecolor{currentstroke}%
\pgfsetdash{}{0pt}%
\pgfsys@defobject{currentmarker}{\pgfqpoint{0.000000in}{0.000000in}}{\pgfqpoint{0.000000in}{0.020833in}}{%
\pgfpathmoveto{\pgfqpoint{0.000000in}{0.000000in}}%
\pgfpathlineto{\pgfqpoint{0.000000in}{0.020833in}}%
\pgfusepath{stroke,fill}%
}%
\begin{pgfscope}%
\pgfsys@transformshift{3.408942in}{0.586309in}%
\pgfsys@useobject{currentmarker}{}%
\end{pgfscope}%
\end{pgfscope}%
\begin{pgfscope}%
\pgfsetbuttcap%
\pgfsetroundjoin%
\definecolor{currentfill}{rgb}{0.000000,0.000000,0.000000}%
\pgfsetfillcolor{currentfill}%
\pgfsetlinewidth{0.501875pt}%
\definecolor{currentstroke}{rgb}{0.000000,0.000000,0.000000}%
\pgfsetstrokecolor{currentstroke}%
\pgfsetdash{}{0pt}%
\pgfsys@defobject{currentmarker}{\pgfqpoint{0.000000in}{-0.020833in}}{\pgfqpoint{0.000000in}{0.000000in}}{%
\pgfpathmoveto{\pgfqpoint{0.000000in}{0.000000in}}%
\pgfpathlineto{\pgfqpoint{0.000000in}{-0.020833in}}%
\pgfusepath{stroke,fill}%
}%
\begin{pgfscope}%
\pgfsys@transformshift{3.408942in}{0.893003in}%
\pgfsys@useobject{currentmarker}{}%
\end{pgfscope}%
\end{pgfscope}%
\begin{pgfscope}%
\pgfsetbuttcap%
\pgfsetroundjoin%
\definecolor{currentfill}{rgb}{0.000000,0.000000,0.000000}%
\pgfsetfillcolor{currentfill}%
\pgfsetlinewidth{0.501875pt}%
\definecolor{currentstroke}{rgb}{0.000000,0.000000,0.000000}%
\pgfsetstrokecolor{currentstroke}%
\pgfsetdash{}{0pt}%
\pgfsys@defobject{currentmarker}{\pgfqpoint{0.000000in}{0.000000in}}{\pgfqpoint{0.000000in}{0.020833in}}{%
\pgfpathmoveto{\pgfqpoint{0.000000in}{0.000000in}}%
\pgfpathlineto{\pgfqpoint{0.000000in}{0.020833in}}%
\pgfusepath{stroke,fill}%
}%
\begin{pgfscope}%
\pgfsys@transformshift{3.476664in}{0.586309in}%
\pgfsys@useobject{currentmarker}{}%
\end{pgfscope}%
\end{pgfscope}%
\begin{pgfscope}%
\pgfsetbuttcap%
\pgfsetroundjoin%
\definecolor{currentfill}{rgb}{0.000000,0.000000,0.000000}%
\pgfsetfillcolor{currentfill}%
\pgfsetlinewidth{0.501875pt}%
\definecolor{currentstroke}{rgb}{0.000000,0.000000,0.000000}%
\pgfsetstrokecolor{currentstroke}%
\pgfsetdash{}{0pt}%
\pgfsys@defobject{currentmarker}{\pgfqpoint{0.000000in}{-0.020833in}}{\pgfqpoint{0.000000in}{0.000000in}}{%
\pgfpathmoveto{\pgfqpoint{0.000000in}{0.000000in}}%
\pgfpathlineto{\pgfqpoint{0.000000in}{-0.020833in}}%
\pgfusepath{stroke,fill}%
}%
\begin{pgfscope}%
\pgfsys@transformshift{3.476664in}{0.893003in}%
\pgfsys@useobject{currentmarker}{}%
\end{pgfscope}%
\end{pgfscope}%
\begin{pgfscope}%
\pgfsetbuttcap%
\pgfsetroundjoin%
\definecolor{currentfill}{rgb}{0.000000,0.000000,0.000000}%
\pgfsetfillcolor{currentfill}%
\pgfsetlinewidth{0.501875pt}%
\definecolor{currentstroke}{rgb}{0.000000,0.000000,0.000000}%
\pgfsetstrokecolor{currentstroke}%
\pgfsetdash{}{0pt}%
\pgfsys@defobject{currentmarker}{\pgfqpoint{0.000000in}{0.000000in}}{\pgfqpoint{0.000000in}{0.020833in}}{%
\pgfpathmoveto{\pgfqpoint{0.000000in}{0.000000in}}%
\pgfpathlineto{\pgfqpoint{0.000000in}{0.020833in}}%
\pgfusepath{stroke,fill}%
}%
\begin{pgfscope}%
\pgfsys@transformshift{3.544385in}{0.586309in}%
\pgfsys@useobject{currentmarker}{}%
\end{pgfscope}%
\end{pgfscope}%
\begin{pgfscope}%
\pgfsetbuttcap%
\pgfsetroundjoin%
\definecolor{currentfill}{rgb}{0.000000,0.000000,0.000000}%
\pgfsetfillcolor{currentfill}%
\pgfsetlinewidth{0.501875pt}%
\definecolor{currentstroke}{rgb}{0.000000,0.000000,0.000000}%
\pgfsetstrokecolor{currentstroke}%
\pgfsetdash{}{0pt}%
\pgfsys@defobject{currentmarker}{\pgfqpoint{0.000000in}{-0.020833in}}{\pgfqpoint{0.000000in}{0.000000in}}{%
\pgfpathmoveto{\pgfqpoint{0.000000in}{0.000000in}}%
\pgfpathlineto{\pgfqpoint{0.000000in}{-0.020833in}}%
\pgfusepath{stroke,fill}%
}%
\begin{pgfscope}%
\pgfsys@transformshift{3.544385in}{0.893003in}%
\pgfsys@useobject{currentmarker}{}%
\end{pgfscope}%
\end{pgfscope}%
\begin{pgfscope}%
\pgfsetbuttcap%
\pgfsetroundjoin%
\definecolor{currentfill}{rgb}{0.000000,0.000000,0.000000}%
\pgfsetfillcolor{currentfill}%
\pgfsetlinewidth{0.501875pt}%
\definecolor{currentstroke}{rgb}{0.000000,0.000000,0.000000}%
\pgfsetstrokecolor{currentstroke}%
\pgfsetdash{}{0pt}%
\pgfsys@defobject{currentmarker}{\pgfqpoint{0.000000in}{0.000000in}}{\pgfqpoint{0.000000in}{0.020833in}}{%
\pgfpathmoveto{\pgfqpoint{0.000000in}{0.000000in}}%
\pgfpathlineto{\pgfqpoint{0.000000in}{0.020833in}}%
\pgfusepath{stroke,fill}%
}%
\begin{pgfscope}%
\pgfsys@transformshift{3.679828in}{0.586309in}%
\pgfsys@useobject{currentmarker}{}%
\end{pgfscope}%
\end{pgfscope}%
\begin{pgfscope}%
\pgfsetbuttcap%
\pgfsetroundjoin%
\definecolor{currentfill}{rgb}{0.000000,0.000000,0.000000}%
\pgfsetfillcolor{currentfill}%
\pgfsetlinewidth{0.501875pt}%
\definecolor{currentstroke}{rgb}{0.000000,0.000000,0.000000}%
\pgfsetstrokecolor{currentstroke}%
\pgfsetdash{}{0pt}%
\pgfsys@defobject{currentmarker}{\pgfqpoint{0.000000in}{-0.020833in}}{\pgfqpoint{0.000000in}{0.000000in}}{%
\pgfpathmoveto{\pgfqpoint{0.000000in}{0.000000in}}%
\pgfpathlineto{\pgfqpoint{0.000000in}{-0.020833in}}%
\pgfusepath{stroke,fill}%
}%
\begin{pgfscope}%
\pgfsys@transformshift{3.679828in}{0.893003in}%
\pgfsys@useobject{currentmarker}{}%
\end{pgfscope}%
\end{pgfscope}%
\begin{pgfscope}%
\pgfsetbuttcap%
\pgfsetroundjoin%
\definecolor{currentfill}{rgb}{0.000000,0.000000,0.000000}%
\pgfsetfillcolor{currentfill}%
\pgfsetlinewidth{0.501875pt}%
\definecolor{currentstroke}{rgb}{0.000000,0.000000,0.000000}%
\pgfsetstrokecolor{currentstroke}%
\pgfsetdash{}{0pt}%
\pgfsys@defobject{currentmarker}{\pgfqpoint{0.000000in}{0.000000in}}{\pgfqpoint{0.000000in}{0.020833in}}{%
\pgfpathmoveto{\pgfqpoint{0.000000in}{0.000000in}}%
\pgfpathlineto{\pgfqpoint{0.000000in}{0.020833in}}%
\pgfusepath{stroke,fill}%
}%
\begin{pgfscope}%
\pgfsys@transformshift{3.747549in}{0.586309in}%
\pgfsys@useobject{currentmarker}{}%
\end{pgfscope}%
\end{pgfscope}%
\begin{pgfscope}%
\pgfsetbuttcap%
\pgfsetroundjoin%
\definecolor{currentfill}{rgb}{0.000000,0.000000,0.000000}%
\pgfsetfillcolor{currentfill}%
\pgfsetlinewidth{0.501875pt}%
\definecolor{currentstroke}{rgb}{0.000000,0.000000,0.000000}%
\pgfsetstrokecolor{currentstroke}%
\pgfsetdash{}{0pt}%
\pgfsys@defobject{currentmarker}{\pgfqpoint{0.000000in}{-0.020833in}}{\pgfqpoint{0.000000in}{0.000000in}}{%
\pgfpathmoveto{\pgfqpoint{0.000000in}{0.000000in}}%
\pgfpathlineto{\pgfqpoint{0.000000in}{-0.020833in}}%
\pgfusepath{stroke,fill}%
}%
\begin{pgfscope}%
\pgfsys@transformshift{3.747549in}{0.893003in}%
\pgfsys@useobject{currentmarker}{}%
\end{pgfscope}%
\end{pgfscope}%
\begin{pgfscope}%
\pgfsetbuttcap%
\pgfsetroundjoin%
\definecolor{currentfill}{rgb}{0.000000,0.000000,0.000000}%
\pgfsetfillcolor{currentfill}%
\pgfsetlinewidth{0.501875pt}%
\definecolor{currentstroke}{rgb}{0.000000,0.000000,0.000000}%
\pgfsetstrokecolor{currentstroke}%
\pgfsetdash{}{0pt}%
\pgfsys@defobject{currentmarker}{\pgfqpoint{0.000000in}{0.000000in}}{\pgfqpoint{0.000000in}{0.020833in}}{%
\pgfpathmoveto{\pgfqpoint{0.000000in}{0.000000in}}%
\pgfpathlineto{\pgfqpoint{0.000000in}{0.020833in}}%
\pgfusepath{stroke,fill}%
}%
\begin{pgfscope}%
\pgfsys@transformshift{3.815271in}{0.586309in}%
\pgfsys@useobject{currentmarker}{}%
\end{pgfscope}%
\end{pgfscope}%
\begin{pgfscope}%
\pgfsetbuttcap%
\pgfsetroundjoin%
\definecolor{currentfill}{rgb}{0.000000,0.000000,0.000000}%
\pgfsetfillcolor{currentfill}%
\pgfsetlinewidth{0.501875pt}%
\definecolor{currentstroke}{rgb}{0.000000,0.000000,0.000000}%
\pgfsetstrokecolor{currentstroke}%
\pgfsetdash{}{0pt}%
\pgfsys@defobject{currentmarker}{\pgfqpoint{0.000000in}{-0.020833in}}{\pgfqpoint{0.000000in}{0.000000in}}{%
\pgfpathmoveto{\pgfqpoint{0.000000in}{0.000000in}}%
\pgfpathlineto{\pgfqpoint{0.000000in}{-0.020833in}}%
\pgfusepath{stroke,fill}%
}%
\begin{pgfscope}%
\pgfsys@transformshift{3.815271in}{0.893003in}%
\pgfsys@useobject{currentmarker}{}%
\end{pgfscope}%
\end{pgfscope}%
\begin{pgfscope}%
\pgfsetbuttcap%
\pgfsetroundjoin%
\definecolor{currentfill}{rgb}{0.000000,0.000000,0.000000}%
\pgfsetfillcolor{currentfill}%
\pgfsetlinewidth{0.501875pt}%
\definecolor{currentstroke}{rgb}{0.000000,0.000000,0.000000}%
\pgfsetstrokecolor{currentstroke}%
\pgfsetdash{}{0pt}%
\pgfsys@defobject{currentmarker}{\pgfqpoint{0.000000in}{0.000000in}}{\pgfqpoint{0.000000in}{0.020833in}}{%
\pgfpathmoveto{\pgfqpoint{0.000000in}{0.000000in}}%
\pgfpathlineto{\pgfqpoint{0.000000in}{0.020833in}}%
\pgfusepath{stroke,fill}%
}%
\begin{pgfscope}%
\pgfsys@transformshift{3.882992in}{0.586309in}%
\pgfsys@useobject{currentmarker}{}%
\end{pgfscope}%
\end{pgfscope}%
\begin{pgfscope}%
\pgfsetbuttcap%
\pgfsetroundjoin%
\definecolor{currentfill}{rgb}{0.000000,0.000000,0.000000}%
\pgfsetfillcolor{currentfill}%
\pgfsetlinewidth{0.501875pt}%
\definecolor{currentstroke}{rgb}{0.000000,0.000000,0.000000}%
\pgfsetstrokecolor{currentstroke}%
\pgfsetdash{}{0pt}%
\pgfsys@defobject{currentmarker}{\pgfqpoint{0.000000in}{-0.020833in}}{\pgfqpoint{0.000000in}{0.000000in}}{%
\pgfpathmoveto{\pgfqpoint{0.000000in}{0.000000in}}%
\pgfpathlineto{\pgfqpoint{0.000000in}{-0.020833in}}%
\pgfusepath{stroke,fill}%
}%
\begin{pgfscope}%
\pgfsys@transformshift{3.882992in}{0.893003in}%
\pgfsys@useobject{currentmarker}{}%
\end{pgfscope}%
\end{pgfscope}%
\begin{pgfscope}%
\pgfsetbuttcap%
\pgfsetroundjoin%
\definecolor{currentfill}{rgb}{0.000000,0.000000,0.000000}%
\pgfsetfillcolor{currentfill}%
\pgfsetlinewidth{0.501875pt}%
\definecolor{currentstroke}{rgb}{0.000000,0.000000,0.000000}%
\pgfsetstrokecolor{currentstroke}%
\pgfsetdash{}{0pt}%
\pgfsys@defobject{currentmarker}{\pgfqpoint{0.000000in}{0.000000in}}{\pgfqpoint{0.000000in}{0.020833in}}{%
\pgfpathmoveto{\pgfqpoint{0.000000in}{0.000000in}}%
\pgfpathlineto{\pgfqpoint{0.000000in}{0.020833in}}%
\pgfusepath{stroke,fill}%
}%
\begin{pgfscope}%
\pgfsys@transformshift{3.950714in}{0.586309in}%
\pgfsys@useobject{currentmarker}{}%
\end{pgfscope}%
\end{pgfscope}%
\begin{pgfscope}%
\pgfsetbuttcap%
\pgfsetroundjoin%
\definecolor{currentfill}{rgb}{0.000000,0.000000,0.000000}%
\pgfsetfillcolor{currentfill}%
\pgfsetlinewidth{0.501875pt}%
\definecolor{currentstroke}{rgb}{0.000000,0.000000,0.000000}%
\pgfsetstrokecolor{currentstroke}%
\pgfsetdash{}{0pt}%
\pgfsys@defobject{currentmarker}{\pgfqpoint{0.000000in}{-0.020833in}}{\pgfqpoint{0.000000in}{0.000000in}}{%
\pgfpathmoveto{\pgfqpoint{0.000000in}{0.000000in}}%
\pgfpathlineto{\pgfqpoint{0.000000in}{-0.020833in}}%
\pgfusepath{stroke,fill}%
}%
\begin{pgfscope}%
\pgfsys@transformshift{3.950714in}{0.893003in}%
\pgfsys@useobject{currentmarker}{}%
\end{pgfscope}%
\end{pgfscope}%
\begin{pgfscope}%
\pgfsetbuttcap%
\pgfsetroundjoin%
\definecolor{currentfill}{rgb}{0.000000,0.000000,0.000000}%
\pgfsetfillcolor{currentfill}%
\pgfsetlinewidth{0.501875pt}%
\definecolor{currentstroke}{rgb}{0.000000,0.000000,0.000000}%
\pgfsetstrokecolor{currentstroke}%
\pgfsetdash{}{0pt}%
\pgfsys@defobject{currentmarker}{\pgfqpoint{0.000000in}{0.000000in}}{\pgfqpoint{0.000000in}{0.020833in}}{%
\pgfpathmoveto{\pgfqpoint{0.000000in}{0.000000in}}%
\pgfpathlineto{\pgfqpoint{0.000000in}{0.020833in}}%
\pgfusepath{stroke,fill}%
}%
\begin{pgfscope}%
\pgfsys@transformshift{4.086157in}{0.586309in}%
\pgfsys@useobject{currentmarker}{}%
\end{pgfscope}%
\end{pgfscope}%
\begin{pgfscope}%
\pgfsetbuttcap%
\pgfsetroundjoin%
\definecolor{currentfill}{rgb}{0.000000,0.000000,0.000000}%
\pgfsetfillcolor{currentfill}%
\pgfsetlinewidth{0.501875pt}%
\definecolor{currentstroke}{rgb}{0.000000,0.000000,0.000000}%
\pgfsetstrokecolor{currentstroke}%
\pgfsetdash{}{0pt}%
\pgfsys@defobject{currentmarker}{\pgfqpoint{0.000000in}{-0.020833in}}{\pgfqpoint{0.000000in}{0.000000in}}{%
\pgfpathmoveto{\pgfqpoint{0.000000in}{0.000000in}}%
\pgfpathlineto{\pgfqpoint{0.000000in}{-0.020833in}}%
\pgfusepath{stroke,fill}%
}%
\begin{pgfscope}%
\pgfsys@transformshift{4.086157in}{0.893003in}%
\pgfsys@useobject{currentmarker}{}%
\end{pgfscope}%
\end{pgfscope}%
\begin{pgfscope}%
\pgfsetbuttcap%
\pgfsetroundjoin%
\definecolor{currentfill}{rgb}{0.000000,0.000000,0.000000}%
\pgfsetfillcolor{currentfill}%
\pgfsetlinewidth{0.501875pt}%
\definecolor{currentstroke}{rgb}{0.000000,0.000000,0.000000}%
\pgfsetstrokecolor{currentstroke}%
\pgfsetdash{}{0pt}%
\pgfsys@defobject{currentmarker}{\pgfqpoint{0.000000in}{0.000000in}}{\pgfqpoint{0.000000in}{0.020833in}}{%
\pgfpathmoveto{\pgfqpoint{0.000000in}{0.000000in}}%
\pgfpathlineto{\pgfqpoint{0.000000in}{0.020833in}}%
\pgfusepath{stroke,fill}%
}%
\begin{pgfscope}%
\pgfsys@transformshift{4.153878in}{0.586309in}%
\pgfsys@useobject{currentmarker}{}%
\end{pgfscope}%
\end{pgfscope}%
\begin{pgfscope}%
\pgfsetbuttcap%
\pgfsetroundjoin%
\definecolor{currentfill}{rgb}{0.000000,0.000000,0.000000}%
\pgfsetfillcolor{currentfill}%
\pgfsetlinewidth{0.501875pt}%
\definecolor{currentstroke}{rgb}{0.000000,0.000000,0.000000}%
\pgfsetstrokecolor{currentstroke}%
\pgfsetdash{}{0pt}%
\pgfsys@defobject{currentmarker}{\pgfqpoint{0.000000in}{-0.020833in}}{\pgfqpoint{0.000000in}{0.000000in}}{%
\pgfpathmoveto{\pgfqpoint{0.000000in}{0.000000in}}%
\pgfpathlineto{\pgfqpoint{0.000000in}{-0.020833in}}%
\pgfusepath{stroke,fill}%
}%
\begin{pgfscope}%
\pgfsys@transformshift{4.153878in}{0.893003in}%
\pgfsys@useobject{currentmarker}{}%
\end{pgfscope}%
\end{pgfscope}%
\begin{pgfscope}%
\pgfsetbuttcap%
\pgfsetroundjoin%
\definecolor{currentfill}{rgb}{0.000000,0.000000,0.000000}%
\pgfsetfillcolor{currentfill}%
\pgfsetlinewidth{0.501875pt}%
\definecolor{currentstroke}{rgb}{0.000000,0.000000,0.000000}%
\pgfsetstrokecolor{currentstroke}%
\pgfsetdash{}{0pt}%
\pgfsys@defobject{currentmarker}{\pgfqpoint{0.000000in}{0.000000in}}{\pgfqpoint{0.000000in}{0.020833in}}{%
\pgfpathmoveto{\pgfqpoint{0.000000in}{0.000000in}}%
\pgfpathlineto{\pgfqpoint{0.000000in}{0.020833in}}%
\pgfusepath{stroke,fill}%
}%
\begin{pgfscope}%
\pgfsys@transformshift{4.221600in}{0.586309in}%
\pgfsys@useobject{currentmarker}{}%
\end{pgfscope}%
\end{pgfscope}%
\begin{pgfscope}%
\pgfsetbuttcap%
\pgfsetroundjoin%
\definecolor{currentfill}{rgb}{0.000000,0.000000,0.000000}%
\pgfsetfillcolor{currentfill}%
\pgfsetlinewidth{0.501875pt}%
\definecolor{currentstroke}{rgb}{0.000000,0.000000,0.000000}%
\pgfsetstrokecolor{currentstroke}%
\pgfsetdash{}{0pt}%
\pgfsys@defobject{currentmarker}{\pgfqpoint{0.000000in}{-0.020833in}}{\pgfqpoint{0.000000in}{0.000000in}}{%
\pgfpathmoveto{\pgfqpoint{0.000000in}{0.000000in}}%
\pgfpathlineto{\pgfqpoint{0.000000in}{-0.020833in}}%
\pgfusepath{stroke,fill}%
}%
\begin{pgfscope}%
\pgfsys@transformshift{4.221600in}{0.893003in}%
\pgfsys@useobject{currentmarker}{}%
\end{pgfscope}%
\end{pgfscope}%
\begin{pgfscope}%
\pgfsetbuttcap%
\pgfsetroundjoin%
\definecolor{currentfill}{rgb}{0.000000,0.000000,0.000000}%
\pgfsetfillcolor{currentfill}%
\pgfsetlinewidth{0.501875pt}%
\definecolor{currentstroke}{rgb}{0.000000,0.000000,0.000000}%
\pgfsetstrokecolor{currentstroke}%
\pgfsetdash{}{0pt}%
\pgfsys@defobject{currentmarker}{\pgfqpoint{0.000000in}{0.000000in}}{\pgfqpoint{0.000000in}{0.020833in}}{%
\pgfpathmoveto{\pgfqpoint{0.000000in}{0.000000in}}%
\pgfpathlineto{\pgfqpoint{0.000000in}{0.020833in}}%
\pgfusepath{stroke,fill}%
}%
\begin{pgfscope}%
\pgfsys@transformshift{4.289321in}{0.586309in}%
\pgfsys@useobject{currentmarker}{}%
\end{pgfscope}%
\end{pgfscope}%
\begin{pgfscope}%
\pgfsetbuttcap%
\pgfsetroundjoin%
\definecolor{currentfill}{rgb}{0.000000,0.000000,0.000000}%
\pgfsetfillcolor{currentfill}%
\pgfsetlinewidth{0.501875pt}%
\definecolor{currentstroke}{rgb}{0.000000,0.000000,0.000000}%
\pgfsetstrokecolor{currentstroke}%
\pgfsetdash{}{0pt}%
\pgfsys@defobject{currentmarker}{\pgfqpoint{0.000000in}{-0.020833in}}{\pgfqpoint{0.000000in}{0.000000in}}{%
\pgfpathmoveto{\pgfqpoint{0.000000in}{0.000000in}}%
\pgfpathlineto{\pgfqpoint{0.000000in}{-0.020833in}}%
\pgfusepath{stroke,fill}%
}%
\begin{pgfscope}%
\pgfsys@transformshift{4.289321in}{0.893003in}%
\pgfsys@useobject{currentmarker}{}%
\end{pgfscope}%
\end{pgfscope}%
\begin{pgfscope}%
\pgfsetbuttcap%
\pgfsetroundjoin%
\definecolor{currentfill}{rgb}{0.000000,0.000000,0.000000}%
\pgfsetfillcolor{currentfill}%
\pgfsetlinewidth{0.501875pt}%
\definecolor{currentstroke}{rgb}{0.000000,0.000000,0.000000}%
\pgfsetstrokecolor{currentstroke}%
\pgfsetdash{}{0pt}%
\pgfsys@defobject{currentmarker}{\pgfqpoint{0.000000in}{0.000000in}}{\pgfqpoint{0.000000in}{0.020833in}}{%
\pgfpathmoveto{\pgfqpoint{0.000000in}{0.000000in}}%
\pgfpathlineto{\pgfqpoint{0.000000in}{0.020833in}}%
\pgfusepath{stroke,fill}%
}%
\begin{pgfscope}%
\pgfsys@transformshift{4.357043in}{0.586309in}%
\pgfsys@useobject{currentmarker}{}%
\end{pgfscope}%
\end{pgfscope}%
\begin{pgfscope}%
\pgfsetbuttcap%
\pgfsetroundjoin%
\definecolor{currentfill}{rgb}{0.000000,0.000000,0.000000}%
\pgfsetfillcolor{currentfill}%
\pgfsetlinewidth{0.501875pt}%
\definecolor{currentstroke}{rgb}{0.000000,0.000000,0.000000}%
\pgfsetstrokecolor{currentstroke}%
\pgfsetdash{}{0pt}%
\pgfsys@defobject{currentmarker}{\pgfqpoint{0.000000in}{-0.020833in}}{\pgfqpoint{0.000000in}{0.000000in}}{%
\pgfpathmoveto{\pgfqpoint{0.000000in}{0.000000in}}%
\pgfpathlineto{\pgfqpoint{0.000000in}{-0.020833in}}%
\pgfusepath{stroke,fill}%
}%
\begin{pgfscope}%
\pgfsys@transformshift{4.357043in}{0.893003in}%
\pgfsys@useobject{currentmarker}{}%
\end{pgfscope}%
\end{pgfscope}%
\begin{pgfscope}%
\pgfsetbuttcap%
\pgfsetroundjoin%
\definecolor{currentfill}{rgb}{0.000000,0.000000,0.000000}%
\pgfsetfillcolor{currentfill}%
\pgfsetlinewidth{0.501875pt}%
\definecolor{currentstroke}{rgb}{0.000000,0.000000,0.000000}%
\pgfsetstrokecolor{currentstroke}%
\pgfsetdash{}{0pt}%
\pgfsys@defobject{currentmarker}{\pgfqpoint{0.000000in}{0.000000in}}{\pgfqpoint{0.000000in}{0.020833in}}{%
\pgfpathmoveto{\pgfqpoint{0.000000in}{0.000000in}}%
\pgfpathlineto{\pgfqpoint{0.000000in}{0.020833in}}%
\pgfusepath{stroke,fill}%
}%
\begin{pgfscope}%
\pgfsys@transformshift{4.492486in}{0.586309in}%
\pgfsys@useobject{currentmarker}{}%
\end{pgfscope}%
\end{pgfscope}%
\begin{pgfscope}%
\pgfsetbuttcap%
\pgfsetroundjoin%
\definecolor{currentfill}{rgb}{0.000000,0.000000,0.000000}%
\pgfsetfillcolor{currentfill}%
\pgfsetlinewidth{0.501875pt}%
\definecolor{currentstroke}{rgb}{0.000000,0.000000,0.000000}%
\pgfsetstrokecolor{currentstroke}%
\pgfsetdash{}{0pt}%
\pgfsys@defobject{currentmarker}{\pgfqpoint{0.000000in}{-0.020833in}}{\pgfqpoint{0.000000in}{0.000000in}}{%
\pgfpathmoveto{\pgfqpoint{0.000000in}{0.000000in}}%
\pgfpathlineto{\pgfqpoint{0.000000in}{-0.020833in}}%
\pgfusepath{stroke,fill}%
}%
\begin{pgfscope}%
\pgfsys@transformshift{4.492486in}{0.893003in}%
\pgfsys@useobject{currentmarker}{}%
\end{pgfscope}%
\end{pgfscope}%
\begin{pgfscope}%
\pgfsetbuttcap%
\pgfsetroundjoin%
\definecolor{currentfill}{rgb}{0.000000,0.000000,0.000000}%
\pgfsetfillcolor{currentfill}%
\pgfsetlinewidth{0.501875pt}%
\definecolor{currentstroke}{rgb}{0.000000,0.000000,0.000000}%
\pgfsetstrokecolor{currentstroke}%
\pgfsetdash{}{0pt}%
\pgfsys@defobject{currentmarker}{\pgfqpoint{0.000000in}{0.000000in}}{\pgfqpoint{0.000000in}{0.020833in}}{%
\pgfpathmoveto{\pgfqpoint{0.000000in}{0.000000in}}%
\pgfpathlineto{\pgfqpoint{0.000000in}{0.020833in}}%
\pgfusepath{stroke,fill}%
}%
\begin{pgfscope}%
\pgfsys@transformshift{4.560207in}{0.586309in}%
\pgfsys@useobject{currentmarker}{}%
\end{pgfscope}%
\end{pgfscope}%
\begin{pgfscope}%
\pgfsetbuttcap%
\pgfsetroundjoin%
\definecolor{currentfill}{rgb}{0.000000,0.000000,0.000000}%
\pgfsetfillcolor{currentfill}%
\pgfsetlinewidth{0.501875pt}%
\definecolor{currentstroke}{rgb}{0.000000,0.000000,0.000000}%
\pgfsetstrokecolor{currentstroke}%
\pgfsetdash{}{0pt}%
\pgfsys@defobject{currentmarker}{\pgfqpoint{0.000000in}{-0.020833in}}{\pgfqpoint{0.000000in}{0.000000in}}{%
\pgfpathmoveto{\pgfqpoint{0.000000in}{0.000000in}}%
\pgfpathlineto{\pgfqpoint{0.000000in}{-0.020833in}}%
\pgfusepath{stroke,fill}%
}%
\begin{pgfscope}%
\pgfsys@transformshift{4.560207in}{0.893003in}%
\pgfsys@useobject{currentmarker}{}%
\end{pgfscope}%
\end{pgfscope}%
\begin{pgfscope}%
\pgfsetbuttcap%
\pgfsetroundjoin%
\definecolor{currentfill}{rgb}{0.000000,0.000000,0.000000}%
\pgfsetfillcolor{currentfill}%
\pgfsetlinewidth{0.501875pt}%
\definecolor{currentstroke}{rgb}{0.000000,0.000000,0.000000}%
\pgfsetstrokecolor{currentstroke}%
\pgfsetdash{}{0pt}%
\pgfsys@defobject{currentmarker}{\pgfqpoint{0.000000in}{0.000000in}}{\pgfqpoint{0.000000in}{0.020833in}}{%
\pgfpathmoveto{\pgfqpoint{0.000000in}{0.000000in}}%
\pgfpathlineto{\pgfqpoint{0.000000in}{0.020833in}}%
\pgfusepath{stroke,fill}%
}%
\begin{pgfscope}%
\pgfsys@transformshift{4.627929in}{0.586309in}%
\pgfsys@useobject{currentmarker}{}%
\end{pgfscope}%
\end{pgfscope}%
\begin{pgfscope}%
\pgfsetbuttcap%
\pgfsetroundjoin%
\definecolor{currentfill}{rgb}{0.000000,0.000000,0.000000}%
\pgfsetfillcolor{currentfill}%
\pgfsetlinewidth{0.501875pt}%
\definecolor{currentstroke}{rgb}{0.000000,0.000000,0.000000}%
\pgfsetstrokecolor{currentstroke}%
\pgfsetdash{}{0pt}%
\pgfsys@defobject{currentmarker}{\pgfqpoint{0.000000in}{-0.020833in}}{\pgfqpoint{0.000000in}{0.000000in}}{%
\pgfpathmoveto{\pgfqpoint{0.000000in}{0.000000in}}%
\pgfpathlineto{\pgfqpoint{0.000000in}{-0.020833in}}%
\pgfusepath{stroke,fill}%
}%
\begin{pgfscope}%
\pgfsys@transformshift{4.627929in}{0.893003in}%
\pgfsys@useobject{currentmarker}{}%
\end{pgfscope}%
\end{pgfscope}%
\begin{pgfscope}%
\pgfsetbuttcap%
\pgfsetroundjoin%
\definecolor{currentfill}{rgb}{0.000000,0.000000,0.000000}%
\pgfsetfillcolor{currentfill}%
\pgfsetlinewidth{0.501875pt}%
\definecolor{currentstroke}{rgb}{0.000000,0.000000,0.000000}%
\pgfsetstrokecolor{currentstroke}%
\pgfsetdash{}{0pt}%
\pgfsys@defobject{currentmarker}{\pgfqpoint{0.000000in}{0.000000in}}{\pgfqpoint{0.000000in}{0.020833in}}{%
\pgfpathmoveto{\pgfqpoint{0.000000in}{0.000000in}}%
\pgfpathlineto{\pgfqpoint{0.000000in}{0.020833in}}%
\pgfusepath{stroke,fill}%
}%
\begin{pgfscope}%
\pgfsys@transformshift{4.695650in}{0.586309in}%
\pgfsys@useobject{currentmarker}{}%
\end{pgfscope}%
\end{pgfscope}%
\begin{pgfscope}%
\pgfsetbuttcap%
\pgfsetroundjoin%
\definecolor{currentfill}{rgb}{0.000000,0.000000,0.000000}%
\pgfsetfillcolor{currentfill}%
\pgfsetlinewidth{0.501875pt}%
\definecolor{currentstroke}{rgb}{0.000000,0.000000,0.000000}%
\pgfsetstrokecolor{currentstroke}%
\pgfsetdash{}{0pt}%
\pgfsys@defobject{currentmarker}{\pgfqpoint{0.000000in}{-0.020833in}}{\pgfqpoint{0.000000in}{0.000000in}}{%
\pgfpathmoveto{\pgfqpoint{0.000000in}{0.000000in}}%
\pgfpathlineto{\pgfqpoint{0.000000in}{-0.020833in}}%
\pgfusepath{stroke,fill}%
}%
\begin{pgfscope}%
\pgfsys@transformshift{4.695650in}{0.893003in}%
\pgfsys@useobject{currentmarker}{}%
\end{pgfscope}%
\end{pgfscope}%
\begin{pgfscope}%
\pgfsetbuttcap%
\pgfsetroundjoin%
\definecolor{currentfill}{rgb}{0.000000,0.000000,0.000000}%
\pgfsetfillcolor{currentfill}%
\pgfsetlinewidth{0.501875pt}%
\definecolor{currentstroke}{rgb}{0.000000,0.000000,0.000000}%
\pgfsetstrokecolor{currentstroke}%
\pgfsetdash{}{0pt}%
\pgfsys@defobject{currentmarker}{\pgfqpoint{0.000000in}{0.000000in}}{\pgfqpoint{0.000000in}{0.020833in}}{%
\pgfpathmoveto{\pgfqpoint{0.000000in}{0.000000in}}%
\pgfpathlineto{\pgfqpoint{0.000000in}{0.020833in}}%
\pgfusepath{stroke,fill}%
}%
\begin{pgfscope}%
\pgfsys@transformshift{4.763372in}{0.586309in}%
\pgfsys@useobject{currentmarker}{}%
\end{pgfscope}%
\end{pgfscope}%
\begin{pgfscope}%
\pgfsetbuttcap%
\pgfsetroundjoin%
\definecolor{currentfill}{rgb}{0.000000,0.000000,0.000000}%
\pgfsetfillcolor{currentfill}%
\pgfsetlinewidth{0.501875pt}%
\definecolor{currentstroke}{rgb}{0.000000,0.000000,0.000000}%
\pgfsetstrokecolor{currentstroke}%
\pgfsetdash{}{0pt}%
\pgfsys@defobject{currentmarker}{\pgfqpoint{0.000000in}{-0.020833in}}{\pgfqpoint{0.000000in}{0.000000in}}{%
\pgfpathmoveto{\pgfqpoint{0.000000in}{0.000000in}}%
\pgfpathlineto{\pgfqpoint{0.000000in}{-0.020833in}}%
\pgfusepath{stroke,fill}%
}%
\begin{pgfscope}%
\pgfsys@transformshift{4.763372in}{0.893003in}%
\pgfsys@useobject{currentmarker}{}%
\end{pgfscope}%
\end{pgfscope}%
\begin{pgfscope}%
\pgfsetbuttcap%
\pgfsetroundjoin%
\definecolor{currentfill}{rgb}{0.000000,0.000000,0.000000}%
\pgfsetfillcolor{currentfill}%
\pgfsetlinewidth{0.501875pt}%
\definecolor{currentstroke}{rgb}{0.000000,0.000000,0.000000}%
\pgfsetstrokecolor{currentstroke}%
\pgfsetdash{}{0pt}%
\pgfsys@defobject{currentmarker}{\pgfqpoint{0.000000in}{0.000000in}}{\pgfqpoint{0.000000in}{0.020833in}}{%
\pgfpathmoveto{\pgfqpoint{0.000000in}{0.000000in}}%
\pgfpathlineto{\pgfqpoint{0.000000in}{0.020833in}}%
\pgfusepath{stroke,fill}%
}%
\begin{pgfscope}%
\pgfsys@transformshift{4.898815in}{0.586309in}%
\pgfsys@useobject{currentmarker}{}%
\end{pgfscope}%
\end{pgfscope}%
\begin{pgfscope}%
\pgfsetbuttcap%
\pgfsetroundjoin%
\definecolor{currentfill}{rgb}{0.000000,0.000000,0.000000}%
\pgfsetfillcolor{currentfill}%
\pgfsetlinewidth{0.501875pt}%
\definecolor{currentstroke}{rgb}{0.000000,0.000000,0.000000}%
\pgfsetstrokecolor{currentstroke}%
\pgfsetdash{}{0pt}%
\pgfsys@defobject{currentmarker}{\pgfqpoint{0.000000in}{-0.020833in}}{\pgfqpoint{0.000000in}{0.000000in}}{%
\pgfpathmoveto{\pgfqpoint{0.000000in}{0.000000in}}%
\pgfpathlineto{\pgfqpoint{0.000000in}{-0.020833in}}%
\pgfusepath{stroke,fill}%
}%
\begin{pgfscope}%
\pgfsys@transformshift{4.898815in}{0.893003in}%
\pgfsys@useobject{currentmarker}{}%
\end{pgfscope}%
\end{pgfscope}%
\begin{pgfscope}%
\pgfsetbuttcap%
\pgfsetroundjoin%
\definecolor{currentfill}{rgb}{0.000000,0.000000,0.000000}%
\pgfsetfillcolor{currentfill}%
\pgfsetlinewidth{0.501875pt}%
\definecolor{currentstroke}{rgb}{0.000000,0.000000,0.000000}%
\pgfsetstrokecolor{currentstroke}%
\pgfsetdash{}{0pt}%
\pgfsys@defobject{currentmarker}{\pgfqpoint{0.000000in}{0.000000in}}{\pgfqpoint{0.000000in}{0.020833in}}{%
\pgfpathmoveto{\pgfqpoint{0.000000in}{0.000000in}}%
\pgfpathlineto{\pgfqpoint{0.000000in}{0.020833in}}%
\pgfusepath{stroke,fill}%
}%
\begin{pgfscope}%
\pgfsys@transformshift{4.966536in}{0.586309in}%
\pgfsys@useobject{currentmarker}{}%
\end{pgfscope}%
\end{pgfscope}%
\begin{pgfscope}%
\pgfsetbuttcap%
\pgfsetroundjoin%
\definecolor{currentfill}{rgb}{0.000000,0.000000,0.000000}%
\pgfsetfillcolor{currentfill}%
\pgfsetlinewidth{0.501875pt}%
\definecolor{currentstroke}{rgb}{0.000000,0.000000,0.000000}%
\pgfsetstrokecolor{currentstroke}%
\pgfsetdash{}{0pt}%
\pgfsys@defobject{currentmarker}{\pgfqpoint{0.000000in}{-0.020833in}}{\pgfqpoint{0.000000in}{0.000000in}}{%
\pgfpathmoveto{\pgfqpoint{0.000000in}{0.000000in}}%
\pgfpathlineto{\pgfqpoint{0.000000in}{-0.020833in}}%
\pgfusepath{stroke,fill}%
}%
\begin{pgfscope}%
\pgfsys@transformshift{4.966536in}{0.893003in}%
\pgfsys@useobject{currentmarker}{}%
\end{pgfscope}%
\end{pgfscope}%
\begin{pgfscope}%
\pgfsetbuttcap%
\pgfsetroundjoin%
\definecolor{currentfill}{rgb}{0.000000,0.000000,0.000000}%
\pgfsetfillcolor{currentfill}%
\pgfsetlinewidth{0.501875pt}%
\definecolor{currentstroke}{rgb}{0.000000,0.000000,0.000000}%
\pgfsetstrokecolor{currentstroke}%
\pgfsetdash{}{0pt}%
\pgfsys@defobject{currentmarker}{\pgfqpoint{0.000000in}{0.000000in}}{\pgfqpoint{0.000000in}{0.020833in}}{%
\pgfpathmoveto{\pgfqpoint{0.000000in}{0.000000in}}%
\pgfpathlineto{\pgfqpoint{0.000000in}{0.020833in}}%
\pgfusepath{stroke,fill}%
}%
\begin{pgfscope}%
\pgfsys@transformshift{5.034258in}{0.586309in}%
\pgfsys@useobject{currentmarker}{}%
\end{pgfscope}%
\end{pgfscope}%
\begin{pgfscope}%
\pgfsetbuttcap%
\pgfsetroundjoin%
\definecolor{currentfill}{rgb}{0.000000,0.000000,0.000000}%
\pgfsetfillcolor{currentfill}%
\pgfsetlinewidth{0.501875pt}%
\definecolor{currentstroke}{rgb}{0.000000,0.000000,0.000000}%
\pgfsetstrokecolor{currentstroke}%
\pgfsetdash{}{0pt}%
\pgfsys@defobject{currentmarker}{\pgfqpoint{0.000000in}{-0.020833in}}{\pgfqpoint{0.000000in}{0.000000in}}{%
\pgfpathmoveto{\pgfqpoint{0.000000in}{0.000000in}}%
\pgfpathlineto{\pgfqpoint{0.000000in}{-0.020833in}}%
\pgfusepath{stroke,fill}%
}%
\begin{pgfscope}%
\pgfsys@transformshift{5.034258in}{0.893003in}%
\pgfsys@useobject{currentmarker}{}%
\end{pgfscope}%
\end{pgfscope}%
\begin{pgfscope}%
\pgfsetbuttcap%
\pgfsetroundjoin%
\definecolor{currentfill}{rgb}{0.000000,0.000000,0.000000}%
\pgfsetfillcolor{currentfill}%
\pgfsetlinewidth{0.501875pt}%
\definecolor{currentstroke}{rgb}{0.000000,0.000000,0.000000}%
\pgfsetstrokecolor{currentstroke}%
\pgfsetdash{}{0pt}%
\pgfsys@defobject{currentmarker}{\pgfqpoint{0.000000in}{0.000000in}}{\pgfqpoint{0.000000in}{0.020833in}}{%
\pgfpathmoveto{\pgfqpoint{0.000000in}{0.000000in}}%
\pgfpathlineto{\pgfqpoint{0.000000in}{0.020833in}}%
\pgfusepath{stroke,fill}%
}%
\begin{pgfscope}%
\pgfsys@transformshift{5.101979in}{0.586309in}%
\pgfsys@useobject{currentmarker}{}%
\end{pgfscope}%
\end{pgfscope}%
\begin{pgfscope}%
\pgfsetbuttcap%
\pgfsetroundjoin%
\definecolor{currentfill}{rgb}{0.000000,0.000000,0.000000}%
\pgfsetfillcolor{currentfill}%
\pgfsetlinewidth{0.501875pt}%
\definecolor{currentstroke}{rgb}{0.000000,0.000000,0.000000}%
\pgfsetstrokecolor{currentstroke}%
\pgfsetdash{}{0pt}%
\pgfsys@defobject{currentmarker}{\pgfqpoint{0.000000in}{-0.020833in}}{\pgfqpoint{0.000000in}{0.000000in}}{%
\pgfpathmoveto{\pgfqpoint{0.000000in}{0.000000in}}%
\pgfpathlineto{\pgfqpoint{0.000000in}{-0.020833in}}%
\pgfusepath{stroke,fill}%
}%
\begin{pgfscope}%
\pgfsys@transformshift{5.101979in}{0.893003in}%
\pgfsys@useobject{currentmarker}{}%
\end{pgfscope}%
\end{pgfscope}%
\begin{pgfscope}%
\pgfsetbuttcap%
\pgfsetroundjoin%
\definecolor{currentfill}{rgb}{0.000000,0.000000,0.000000}%
\pgfsetfillcolor{currentfill}%
\pgfsetlinewidth{0.501875pt}%
\definecolor{currentstroke}{rgb}{0.000000,0.000000,0.000000}%
\pgfsetstrokecolor{currentstroke}%
\pgfsetdash{}{0pt}%
\pgfsys@defobject{currentmarker}{\pgfqpoint{0.000000in}{0.000000in}}{\pgfqpoint{0.000000in}{0.020833in}}{%
\pgfpathmoveto{\pgfqpoint{0.000000in}{0.000000in}}%
\pgfpathlineto{\pgfqpoint{0.000000in}{0.020833in}}%
\pgfusepath{stroke,fill}%
}%
\begin{pgfscope}%
\pgfsys@transformshift{5.169701in}{0.586309in}%
\pgfsys@useobject{currentmarker}{}%
\end{pgfscope}%
\end{pgfscope}%
\begin{pgfscope}%
\pgfsetbuttcap%
\pgfsetroundjoin%
\definecolor{currentfill}{rgb}{0.000000,0.000000,0.000000}%
\pgfsetfillcolor{currentfill}%
\pgfsetlinewidth{0.501875pt}%
\definecolor{currentstroke}{rgb}{0.000000,0.000000,0.000000}%
\pgfsetstrokecolor{currentstroke}%
\pgfsetdash{}{0pt}%
\pgfsys@defobject{currentmarker}{\pgfqpoint{0.000000in}{-0.020833in}}{\pgfqpoint{0.000000in}{0.000000in}}{%
\pgfpathmoveto{\pgfqpoint{0.000000in}{0.000000in}}%
\pgfpathlineto{\pgfqpoint{0.000000in}{-0.020833in}}%
\pgfusepath{stroke,fill}%
}%
\begin{pgfscope}%
\pgfsys@transformshift{5.169701in}{0.893003in}%
\pgfsys@useobject{currentmarker}{}%
\end{pgfscope}%
\end{pgfscope}%
\begin{pgfscope}%
\pgfsetbuttcap%
\pgfsetroundjoin%
\definecolor{currentfill}{rgb}{0.000000,0.000000,0.000000}%
\pgfsetfillcolor{currentfill}%
\pgfsetlinewidth{0.501875pt}%
\definecolor{currentstroke}{rgb}{0.000000,0.000000,0.000000}%
\pgfsetstrokecolor{currentstroke}%
\pgfsetdash{}{0pt}%
\pgfsys@defobject{currentmarker}{\pgfqpoint{0.000000in}{0.000000in}}{\pgfqpoint{0.000000in}{0.020833in}}{%
\pgfpathmoveto{\pgfqpoint{0.000000in}{0.000000in}}%
\pgfpathlineto{\pgfqpoint{0.000000in}{0.020833in}}%
\pgfusepath{stroke,fill}%
}%
\begin{pgfscope}%
\pgfsys@transformshift{5.305144in}{0.586309in}%
\pgfsys@useobject{currentmarker}{}%
\end{pgfscope}%
\end{pgfscope}%
\begin{pgfscope}%
\pgfsetbuttcap%
\pgfsetroundjoin%
\definecolor{currentfill}{rgb}{0.000000,0.000000,0.000000}%
\pgfsetfillcolor{currentfill}%
\pgfsetlinewidth{0.501875pt}%
\definecolor{currentstroke}{rgb}{0.000000,0.000000,0.000000}%
\pgfsetstrokecolor{currentstroke}%
\pgfsetdash{}{0pt}%
\pgfsys@defobject{currentmarker}{\pgfqpoint{0.000000in}{-0.020833in}}{\pgfqpoint{0.000000in}{0.000000in}}{%
\pgfpathmoveto{\pgfqpoint{0.000000in}{0.000000in}}%
\pgfpathlineto{\pgfqpoint{0.000000in}{-0.020833in}}%
\pgfusepath{stroke,fill}%
}%
\begin{pgfscope}%
\pgfsys@transformshift{5.305144in}{0.893003in}%
\pgfsys@useobject{currentmarker}{}%
\end{pgfscope}%
\end{pgfscope}%
\begin{pgfscope}%
\pgfsetbuttcap%
\pgfsetroundjoin%
\definecolor{currentfill}{rgb}{0.000000,0.000000,0.000000}%
\pgfsetfillcolor{currentfill}%
\pgfsetlinewidth{0.501875pt}%
\definecolor{currentstroke}{rgb}{0.000000,0.000000,0.000000}%
\pgfsetstrokecolor{currentstroke}%
\pgfsetdash{}{0pt}%
\pgfsys@defobject{currentmarker}{\pgfqpoint{0.000000in}{0.000000in}}{\pgfqpoint{0.000000in}{0.020833in}}{%
\pgfpathmoveto{\pgfqpoint{0.000000in}{0.000000in}}%
\pgfpathlineto{\pgfqpoint{0.000000in}{0.020833in}}%
\pgfusepath{stroke,fill}%
}%
\begin{pgfscope}%
\pgfsys@transformshift{5.372865in}{0.586309in}%
\pgfsys@useobject{currentmarker}{}%
\end{pgfscope}%
\end{pgfscope}%
\begin{pgfscope}%
\pgfsetbuttcap%
\pgfsetroundjoin%
\definecolor{currentfill}{rgb}{0.000000,0.000000,0.000000}%
\pgfsetfillcolor{currentfill}%
\pgfsetlinewidth{0.501875pt}%
\definecolor{currentstroke}{rgb}{0.000000,0.000000,0.000000}%
\pgfsetstrokecolor{currentstroke}%
\pgfsetdash{}{0pt}%
\pgfsys@defobject{currentmarker}{\pgfqpoint{0.000000in}{-0.020833in}}{\pgfqpoint{0.000000in}{0.000000in}}{%
\pgfpathmoveto{\pgfqpoint{0.000000in}{0.000000in}}%
\pgfpathlineto{\pgfqpoint{0.000000in}{-0.020833in}}%
\pgfusepath{stroke,fill}%
}%
\begin{pgfscope}%
\pgfsys@transformshift{5.372865in}{0.893003in}%
\pgfsys@useobject{currentmarker}{}%
\end{pgfscope}%
\end{pgfscope}%
\begin{pgfscope}%
\pgfsetbuttcap%
\pgfsetroundjoin%
\definecolor{currentfill}{rgb}{0.000000,0.000000,0.000000}%
\pgfsetfillcolor{currentfill}%
\pgfsetlinewidth{0.501875pt}%
\definecolor{currentstroke}{rgb}{0.000000,0.000000,0.000000}%
\pgfsetstrokecolor{currentstroke}%
\pgfsetdash{}{0pt}%
\pgfsys@defobject{currentmarker}{\pgfqpoint{0.000000in}{0.000000in}}{\pgfqpoint{0.000000in}{0.020833in}}{%
\pgfpathmoveto{\pgfqpoint{0.000000in}{0.000000in}}%
\pgfpathlineto{\pgfqpoint{0.000000in}{0.020833in}}%
\pgfusepath{stroke,fill}%
}%
\begin{pgfscope}%
\pgfsys@transformshift{5.440587in}{0.586309in}%
\pgfsys@useobject{currentmarker}{}%
\end{pgfscope}%
\end{pgfscope}%
\begin{pgfscope}%
\pgfsetbuttcap%
\pgfsetroundjoin%
\definecolor{currentfill}{rgb}{0.000000,0.000000,0.000000}%
\pgfsetfillcolor{currentfill}%
\pgfsetlinewidth{0.501875pt}%
\definecolor{currentstroke}{rgb}{0.000000,0.000000,0.000000}%
\pgfsetstrokecolor{currentstroke}%
\pgfsetdash{}{0pt}%
\pgfsys@defobject{currentmarker}{\pgfqpoint{0.000000in}{-0.020833in}}{\pgfqpoint{0.000000in}{0.000000in}}{%
\pgfpathmoveto{\pgfqpoint{0.000000in}{0.000000in}}%
\pgfpathlineto{\pgfqpoint{0.000000in}{-0.020833in}}%
\pgfusepath{stroke,fill}%
}%
\begin{pgfscope}%
\pgfsys@transformshift{5.440587in}{0.893003in}%
\pgfsys@useobject{currentmarker}{}%
\end{pgfscope}%
\end{pgfscope}%
\begin{pgfscope}%
\pgfsetbuttcap%
\pgfsetroundjoin%
\definecolor{currentfill}{rgb}{0.000000,0.000000,0.000000}%
\pgfsetfillcolor{currentfill}%
\pgfsetlinewidth{0.501875pt}%
\definecolor{currentstroke}{rgb}{0.000000,0.000000,0.000000}%
\pgfsetstrokecolor{currentstroke}%
\pgfsetdash{}{0pt}%
\pgfsys@defobject{currentmarker}{\pgfqpoint{0.000000in}{0.000000in}}{\pgfqpoint{0.000000in}{0.020833in}}{%
\pgfpathmoveto{\pgfqpoint{0.000000in}{0.000000in}}%
\pgfpathlineto{\pgfqpoint{0.000000in}{0.020833in}}%
\pgfusepath{stroke,fill}%
}%
\begin{pgfscope}%
\pgfsys@transformshift{5.508308in}{0.586309in}%
\pgfsys@useobject{currentmarker}{}%
\end{pgfscope}%
\end{pgfscope}%
\begin{pgfscope}%
\pgfsetbuttcap%
\pgfsetroundjoin%
\definecolor{currentfill}{rgb}{0.000000,0.000000,0.000000}%
\pgfsetfillcolor{currentfill}%
\pgfsetlinewidth{0.501875pt}%
\definecolor{currentstroke}{rgb}{0.000000,0.000000,0.000000}%
\pgfsetstrokecolor{currentstroke}%
\pgfsetdash{}{0pt}%
\pgfsys@defobject{currentmarker}{\pgfqpoint{0.000000in}{-0.020833in}}{\pgfqpoint{0.000000in}{0.000000in}}{%
\pgfpathmoveto{\pgfqpoint{0.000000in}{0.000000in}}%
\pgfpathlineto{\pgfqpoint{0.000000in}{-0.020833in}}%
\pgfusepath{stroke,fill}%
}%
\begin{pgfscope}%
\pgfsys@transformshift{5.508308in}{0.893003in}%
\pgfsys@useobject{currentmarker}{}%
\end{pgfscope}%
\end{pgfscope}%
\begin{pgfscope}%
\pgfsetbuttcap%
\pgfsetroundjoin%
\definecolor{currentfill}{rgb}{0.000000,0.000000,0.000000}%
\pgfsetfillcolor{currentfill}%
\pgfsetlinewidth{0.501875pt}%
\definecolor{currentstroke}{rgb}{0.000000,0.000000,0.000000}%
\pgfsetstrokecolor{currentstroke}%
\pgfsetdash{}{0pt}%
\pgfsys@defobject{currentmarker}{\pgfqpoint{0.000000in}{0.000000in}}{\pgfqpoint{0.000000in}{0.020833in}}{%
\pgfpathmoveto{\pgfqpoint{0.000000in}{0.000000in}}%
\pgfpathlineto{\pgfqpoint{0.000000in}{0.020833in}}%
\pgfusepath{stroke,fill}%
}%
\begin{pgfscope}%
\pgfsys@transformshift{5.576030in}{0.586309in}%
\pgfsys@useobject{currentmarker}{}%
\end{pgfscope}%
\end{pgfscope}%
\begin{pgfscope}%
\pgfsetbuttcap%
\pgfsetroundjoin%
\definecolor{currentfill}{rgb}{0.000000,0.000000,0.000000}%
\pgfsetfillcolor{currentfill}%
\pgfsetlinewidth{0.501875pt}%
\definecolor{currentstroke}{rgb}{0.000000,0.000000,0.000000}%
\pgfsetstrokecolor{currentstroke}%
\pgfsetdash{}{0pt}%
\pgfsys@defobject{currentmarker}{\pgfqpoint{0.000000in}{-0.020833in}}{\pgfqpoint{0.000000in}{0.000000in}}{%
\pgfpathmoveto{\pgfqpoint{0.000000in}{0.000000in}}%
\pgfpathlineto{\pgfqpoint{0.000000in}{-0.020833in}}%
\pgfusepath{stroke,fill}%
}%
\begin{pgfscope}%
\pgfsys@transformshift{5.576030in}{0.893003in}%
\pgfsys@useobject{currentmarker}{}%
\end{pgfscope}%
\end{pgfscope}%
\begin{pgfscope}%
\pgfsetbuttcap%
\pgfsetroundjoin%
\definecolor{currentfill}{rgb}{0.000000,0.000000,0.000000}%
\pgfsetfillcolor{currentfill}%
\pgfsetlinewidth{0.501875pt}%
\definecolor{currentstroke}{rgb}{0.000000,0.000000,0.000000}%
\pgfsetstrokecolor{currentstroke}%
\pgfsetdash{}{0pt}%
\pgfsys@defobject{currentmarker}{\pgfqpoint{0.000000in}{0.000000in}}{\pgfqpoint{0.000000in}{0.020833in}}{%
\pgfpathmoveto{\pgfqpoint{0.000000in}{0.000000in}}%
\pgfpathlineto{\pgfqpoint{0.000000in}{0.020833in}}%
\pgfusepath{stroke,fill}%
}%
\begin{pgfscope}%
\pgfsys@transformshift{5.711473in}{0.586309in}%
\pgfsys@useobject{currentmarker}{}%
\end{pgfscope}%
\end{pgfscope}%
\begin{pgfscope}%
\pgfsetbuttcap%
\pgfsetroundjoin%
\definecolor{currentfill}{rgb}{0.000000,0.000000,0.000000}%
\pgfsetfillcolor{currentfill}%
\pgfsetlinewidth{0.501875pt}%
\definecolor{currentstroke}{rgb}{0.000000,0.000000,0.000000}%
\pgfsetstrokecolor{currentstroke}%
\pgfsetdash{}{0pt}%
\pgfsys@defobject{currentmarker}{\pgfqpoint{0.000000in}{-0.020833in}}{\pgfqpoint{0.000000in}{0.000000in}}{%
\pgfpathmoveto{\pgfqpoint{0.000000in}{0.000000in}}%
\pgfpathlineto{\pgfqpoint{0.000000in}{-0.020833in}}%
\pgfusepath{stroke,fill}%
}%
\begin{pgfscope}%
\pgfsys@transformshift{5.711473in}{0.893003in}%
\pgfsys@useobject{currentmarker}{}%
\end{pgfscope}%
\end{pgfscope}%
\begin{pgfscope}%
\pgfsetbuttcap%
\pgfsetroundjoin%
\definecolor{currentfill}{rgb}{0.000000,0.000000,0.000000}%
\pgfsetfillcolor{currentfill}%
\pgfsetlinewidth{0.501875pt}%
\definecolor{currentstroke}{rgb}{0.000000,0.000000,0.000000}%
\pgfsetstrokecolor{currentstroke}%
\pgfsetdash{}{0pt}%
\pgfsys@defobject{currentmarker}{\pgfqpoint{0.000000in}{0.000000in}}{\pgfqpoint{0.000000in}{0.020833in}}{%
\pgfpathmoveto{\pgfqpoint{0.000000in}{0.000000in}}%
\pgfpathlineto{\pgfqpoint{0.000000in}{0.020833in}}%
\pgfusepath{stroke,fill}%
}%
\begin{pgfscope}%
\pgfsys@transformshift{5.779194in}{0.586309in}%
\pgfsys@useobject{currentmarker}{}%
\end{pgfscope}%
\end{pgfscope}%
\begin{pgfscope}%
\pgfsetbuttcap%
\pgfsetroundjoin%
\definecolor{currentfill}{rgb}{0.000000,0.000000,0.000000}%
\pgfsetfillcolor{currentfill}%
\pgfsetlinewidth{0.501875pt}%
\definecolor{currentstroke}{rgb}{0.000000,0.000000,0.000000}%
\pgfsetstrokecolor{currentstroke}%
\pgfsetdash{}{0pt}%
\pgfsys@defobject{currentmarker}{\pgfqpoint{0.000000in}{-0.020833in}}{\pgfqpoint{0.000000in}{0.000000in}}{%
\pgfpathmoveto{\pgfqpoint{0.000000in}{0.000000in}}%
\pgfpathlineto{\pgfqpoint{0.000000in}{-0.020833in}}%
\pgfusepath{stroke,fill}%
}%
\begin{pgfscope}%
\pgfsys@transformshift{5.779194in}{0.893003in}%
\pgfsys@useobject{currentmarker}{}%
\end{pgfscope}%
\end{pgfscope}%
\begin{pgfscope}%
\pgfsetbuttcap%
\pgfsetroundjoin%
\definecolor{currentfill}{rgb}{0.000000,0.000000,0.000000}%
\pgfsetfillcolor{currentfill}%
\pgfsetlinewidth{0.501875pt}%
\definecolor{currentstroke}{rgb}{0.000000,0.000000,0.000000}%
\pgfsetstrokecolor{currentstroke}%
\pgfsetdash{}{0pt}%
\pgfsys@defobject{currentmarker}{\pgfqpoint{0.000000in}{0.000000in}}{\pgfqpoint{0.000000in}{0.020833in}}{%
\pgfpathmoveto{\pgfqpoint{0.000000in}{0.000000in}}%
\pgfpathlineto{\pgfqpoint{0.000000in}{0.020833in}}%
\pgfusepath{stroke,fill}%
}%
\begin{pgfscope}%
\pgfsys@transformshift{5.846916in}{0.586309in}%
\pgfsys@useobject{currentmarker}{}%
\end{pgfscope}%
\end{pgfscope}%
\begin{pgfscope}%
\pgfsetbuttcap%
\pgfsetroundjoin%
\definecolor{currentfill}{rgb}{0.000000,0.000000,0.000000}%
\pgfsetfillcolor{currentfill}%
\pgfsetlinewidth{0.501875pt}%
\definecolor{currentstroke}{rgb}{0.000000,0.000000,0.000000}%
\pgfsetstrokecolor{currentstroke}%
\pgfsetdash{}{0pt}%
\pgfsys@defobject{currentmarker}{\pgfqpoint{0.000000in}{-0.020833in}}{\pgfqpoint{0.000000in}{0.000000in}}{%
\pgfpathmoveto{\pgfqpoint{0.000000in}{0.000000in}}%
\pgfpathlineto{\pgfqpoint{0.000000in}{-0.020833in}}%
\pgfusepath{stroke,fill}%
}%
\begin{pgfscope}%
\pgfsys@transformshift{5.846916in}{0.893003in}%
\pgfsys@useobject{currentmarker}{}%
\end{pgfscope}%
\end{pgfscope}%
\begin{pgfscope}%
\pgfsetbuttcap%
\pgfsetroundjoin%
\definecolor{currentfill}{rgb}{0.000000,0.000000,0.000000}%
\pgfsetfillcolor{currentfill}%
\pgfsetlinewidth{0.501875pt}%
\definecolor{currentstroke}{rgb}{0.000000,0.000000,0.000000}%
\pgfsetstrokecolor{currentstroke}%
\pgfsetdash{}{0pt}%
\pgfsys@defobject{currentmarker}{\pgfqpoint{0.000000in}{0.000000in}}{\pgfqpoint{0.000000in}{0.020833in}}{%
\pgfpathmoveto{\pgfqpoint{0.000000in}{0.000000in}}%
\pgfpathlineto{\pgfqpoint{0.000000in}{0.020833in}}%
\pgfusepath{stroke,fill}%
}%
\begin{pgfscope}%
\pgfsys@transformshift{5.914637in}{0.586309in}%
\pgfsys@useobject{currentmarker}{}%
\end{pgfscope}%
\end{pgfscope}%
\begin{pgfscope}%
\pgfsetbuttcap%
\pgfsetroundjoin%
\definecolor{currentfill}{rgb}{0.000000,0.000000,0.000000}%
\pgfsetfillcolor{currentfill}%
\pgfsetlinewidth{0.501875pt}%
\definecolor{currentstroke}{rgb}{0.000000,0.000000,0.000000}%
\pgfsetstrokecolor{currentstroke}%
\pgfsetdash{}{0pt}%
\pgfsys@defobject{currentmarker}{\pgfqpoint{0.000000in}{-0.020833in}}{\pgfqpoint{0.000000in}{0.000000in}}{%
\pgfpathmoveto{\pgfqpoint{0.000000in}{0.000000in}}%
\pgfpathlineto{\pgfqpoint{0.000000in}{-0.020833in}}%
\pgfusepath{stroke,fill}%
}%
\begin{pgfscope}%
\pgfsys@transformshift{5.914637in}{0.893003in}%
\pgfsys@useobject{currentmarker}{}%
\end{pgfscope}%
\end{pgfscope}%
\begin{pgfscope}%
\pgfsetbuttcap%
\pgfsetroundjoin%
\definecolor{currentfill}{rgb}{0.000000,0.000000,0.000000}%
\pgfsetfillcolor{currentfill}%
\pgfsetlinewidth{0.501875pt}%
\definecolor{currentstroke}{rgb}{0.000000,0.000000,0.000000}%
\pgfsetstrokecolor{currentstroke}%
\pgfsetdash{}{0pt}%
\pgfsys@defobject{currentmarker}{\pgfqpoint{0.000000in}{0.000000in}}{\pgfqpoint{0.000000in}{0.020833in}}{%
\pgfpathmoveto{\pgfqpoint{0.000000in}{0.000000in}}%
\pgfpathlineto{\pgfqpoint{0.000000in}{0.020833in}}%
\pgfusepath{stroke,fill}%
}%
\begin{pgfscope}%
\pgfsys@transformshift{5.982358in}{0.586309in}%
\pgfsys@useobject{currentmarker}{}%
\end{pgfscope}%
\end{pgfscope}%
\begin{pgfscope}%
\pgfsetbuttcap%
\pgfsetroundjoin%
\definecolor{currentfill}{rgb}{0.000000,0.000000,0.000000}%
\pgfsetfillcolor{currentfill}%
\pgfsetlinewidth{0.501875pt}%
\definecolor{currentstroke}{rgb}{0.000000,0.000000,0.000000}%
\pgfsetstrokecolor{currentstroke}%
\pgfsetdash{}{0pt}%
\pgfsys@defobject{currentmarker}{\pgfqpoint{0.000000in}{-0.020833in}}{\pgfqpoint{0.000000in}{0.000000in}}{%
\pgfpathmoveto{\pgfqpoint{0.000000in}{0.000000in}}%
\pgfpathlineto{\pgfqpoint{0.000000in}{-0.020833in}}%
\pgfusepath{stroke,fill}%
}%
\begin{pgfscope}%
\pgfsys@transformshift{5.982358in}{0.893003in}%
\pgfsys@useobject{currentmarker}{}%
\end{pgfscope}%
\end{pgfscope}%
\begin{pgfscope}%
\pgfsetbuttcap%
\pgfsetroundjoin%
\definecolor{currentfill}{rgb}{0.000000,0.000000,0.000000}%
\pgfsetfillcolor{currentfill}%
\pgfsetlinewidth{0.501875pt}%
\definecolor{currentstroke}{rgb}{0.000000,0.000000,0.000000}%
\pgfsetstrokecolor{currentstroke}%
\pgfsetdash{}{0pt}%
\pgfsys@defobject{currentmarker}{\pgfqpoint{0.000000in}{0.000000in}}{\pgfqpoint{0.000000in}{0.020833in}}{%
\pgfpathmoveto{\pgfqpoint{0.000000in}{0.000000in}}%
\pgfpathlineto{\pgfqpoint{0.000000in}{0.020833in}}%
\pgfusepath{stroke,fill}%
}%
\begin{pgfscope}%
\pgfsys@transformshift{6.117801in}{0.586309in}%
\pgfsys@useobject{currentmarker}{}%
\end{pgfscope}%
\end{pgfscope}%
\begin{pgfscope}%
\pgfsetbuttcap%
\pgfsetroundjoin%
\definecolor{currentfill}{rgb}{0.000000,0.000000,0.000000}%
\pgfsetfillcolor{currentfill}%
\pgfsetlinewidth{0.501875pt}%
\definecolor{currentstroke}{rgb}{0.000000,0.000000,0.000000}%
\pgfsetstrokecolor{currentstroke}%
\pgfsetdash{}{0pt}%
\pgfsys@defobject{currentmarker}{\pgfqpoint{0.000000in}{-0.020833in}}{\pgfqpoint{0.000000in}{0.000000in}}{%
\pgfpathmoveto{\pgfqpoint{0.000000in}{0.000000in}}%
\pgfpathlineto{\pgfqpoint{0.000000in}{-0.020833in}}%
\pgfusepath{stroke,fill}%
}%
\begin{pgfscope}%
\pgfsys@transformshift{6.117801in}{0.893003in}%
\pgfsys@useobject{currentmarker}{}%
\end{pgfscope}%
\end{pgfscope}%
\begin{pgfscope}%
\pgfsetbuttcap%
\pgfsetroundjoin%
\definecolor{currentfill}{rgb}{0.000000,0.000000,0.000000}%
\pgfsetfillcolor{currentfill}%
\pgfsetlinewidth{0.501875pt}%
\definecolor{currentstroke}{rgb}{0.000000,0.000000,0.000000}%
\pgfsetstrokecolor{currentstroke}%
\pgfsetdash{}{0pt}%
\pgfsys@defobject{currentmarker}{\pgfqpoint{0.000000in}{0.000000in}}{\pgfqpoint{0.000000in}{0.020833in}}{%
\pgfpathmoveto{\pgfqpoint{0.000000in}{0.000000in}}%
\pgfpathlineto{\pgfqpoint{0.000000in}{0.020833in}}%
\pgfusepath{stroke,fill}%
}%
\begin{pgfscope}%
\pgfsys@transformshift{6.185523in}{0.586309in}%
\pgfsys@useobject{currentmarker}{}%
\end{pgfscope}%
\end{pgfscope}%
\begin{pgfscope}%
\pgfsetbuttcap%
\pgfsetroundjoin%
\definecolor{currentfill}{rgb}{0.000000,0.000000,0.000000}%
\pgfsetfillcolor{currentfill}%
\pgfsetlinewidth{0.501875pt}%
\definecolor{currentstroke}{rgb}{0.000000,0.000000,0.000000}%
\pgfsetstrokecolor{currentstroke}%
\pgfsetdash{}{0pt}%
\pgfsys@defobject{currentmarker}{\pgfqpoint{0.000000in}{-0.020833in}}{\pgfqpoint{0.000000in}{0.000000in}}{%
\pgfpathmoveto{\pgfqpoint{0.000000in}{0.000000in}}%
\pgfpathlineto{\pgfqpoint{0.000000in}{-0.020833in}}%
\pgfusepath{stroke,fill}%
}%
\begin{pgfscope}%
\pgfsys@transformshift{6.185523in}{0.893003in}%
\pgfsys@useobject{currentmarker}{}%
\end{pgfscope}%
\end{pgfscope}%
\begin{pgfscope}%
\pgfsetbuttcap%
\pgfsetroundjoin%
\definecolor{currentfill}{rgb}{0.000000,0.000000,0.000000}%
\pgfsetfillcolor{currentfill}%
\pgfsetlinewidth{0.501875pt}%
\definecolor{currentstroke}{rgb}{0.000000,0.000000,0.000000}%
\pgfsetstrokecolor{currentstroke}%
\pgfsetdash{}{0pt}%
\pgfsys@defobject{currentmarker}{\pgfqpoint{0.000000in}{0.000000in}}{\pgfqpoint{0.000000in}{0.020833in}}{%
\pgfpathmoveto{\pgfqpoint{0.000000in}{0.000000in}}%
\pgfpathlineto{\pgfqpoint{0.000000in}{0.020833in}}%
\pgfusepath{stroke,fill}%
}%
\begin{pgfscope}%
\pgfsys@transformshift{6.253244in}{0.586309in}%
\pgfsys@useobject{currentmarker}{}%
\end{pgfscope}%
\end{pgfscope}%
\begin{pgfscope}%
\pgfsetbuttcap%
\pgfsetroundjoin%
\definecolor{currentfill}{rgb}{0.000000,0.000000,0.000000}%
\pgfsetfillcolor{currentfill}%
\pgfsetlinewidth{0.501875pt}%
\definecolor{currentstroke}{rgb}{0.000000,0.000000,0.000000}%
\pgfsetstrokecolor{currentstroke}%
\pgfsetdash{}{0pt}%
\pgfsys@defobject{currentmarker}{\pgfqpoint{0.000000in}{-0.020833in}}{\pgfqpoint{0.000000in}{0.000000in}}{%
\pgfpathmoveto{\pgfqpoint{0.000000in}{0.000000in}}%
\pgfpathlineto{\pgfqpoint{0.000000in}{-0.020833in}}%
\pgfusepath{stroke,fill}%
}%
\begin{pgfscope}%
\pgfsys@transformshift{6.253244in}{0.893003in}%
\pgfsys@useobject{currentmarker}{}%
\end{pgfscope}%
\end{pgfscope}%
\begin{pgfscope}%
\definecolor{textcolor}{rgb}{0.000000,0.000000,0.000000}%
\pgfsetstrokecolor{textcolor}%
\pgfsetfillcolor{textcolor}%
\pgftext[x=3.374517in,y=0.148667in,,top]{\color{textcolor}\rmfamily\fontsize{8.000000}{9.600000}\selectfont Zeit}%
\end{pgfscope}%
\begin{pgfscope}%
\pgfsetbuttcap%
\pgfsetroundjoin%
\definecolor{currentfill}{rgb}{0.000000,0.000000,0.000000}%
\pgfsetfillcolor{currentfill}%
\pgfsetlinewidth{0.501875pt}%
\definecolor{currentstroke}{rgb}{0.000000,0.000000,0.000000}%
\pgfsetstrokecolor{currentstroke}%
\pgfsetdash{}{0pt}%
\pgfsys@defobject{currentmarker}{\pgfqpoint{0.000000in}{0.000000in}}{\pgfqpoint{0.041667in}{0.000000in}}{%
\pgfpathmoveto{\pgfqpoint{0.000000in}{0.000000in}}%
\pgfpathlineto{\pgfqpoint{0.041667in}{0.000000in}}%
\pgfusepath{stroke,fill}%
}%
\begin{pgfscope}%
\pgfsys@transformshift{0.481681in}{0.604024in}%
\pgfsys@useobject{currentmarker}{}%
\end{pgfscope}%
\end{pgfscope}%
\begin{pgfscope}%
\pgfsetbuttcap%
\pgfsetroundjoin%
\definecolor{currentfill}{rgb}{0.000000,0.000000,0.000000}%
\pgfsetfillcolor{currentfill}%
\pgfsetlinewidth{0.501875pt}%
\definecolor{currentstroke}{rgb}{0.000000,0.000000,0.000000}%
\pgfsetstrokecolor{currentstroke}%
\pgfsetdash{}{0pt}%
\pgfsys@defobject{currentmarker}{\pgfqpoint{-0.041667in}{0.000000in}}{\pgfqpoint{-0.000000in}{0.000000in}}{%
\pgfpathmoveto{\pgfqpoint{-0.000000in}{0.000000in}}%
\pgfpathlineto{\pgfqpoint{-0.041667in}{0.000000in}}%
\pgfusepath{stroke,fill}%
}%
\begin{pgfscope}%
\pgfsys@transformshift{6.267353in}{0.604024in}%
\pgfsys@useobject{currentmarker}{}%
\end{pgfscope}%
\end{pgfscope}%
\begin{pgfscope}%
\definecolor{textcolor}{rgb}{0.000000,0.000000,0.000000}%
\pgfsetstrokecolor{textcolor}%
\pgfsetfillcolor{textcolor}%
\pgftext[x=0.204222in, y=0.570287in, left, base]{\color{textcolor}\rmfamily\fontsize{7.000000}{8.400000}\selectfont \ensuremath{-}0.5}%
\end{pgfscope}%
\begin{pgfscope}%
\pgfsetbuttcap%
\pgfsetroundjoin%
\definecolor{currentfill}{rgb}{0.000000,0.000000,0.000000}%
\pgfsetfillcolor{currentfill}%
\pgfsetlinewidth{0.501875pt}%
\definecolor{currentstroke}{rgb}{0.000000,0.000000,0.000000}%
\pgfsetstrokecolor{currentstroke}%
\pgfsetdash{}{0pt}%
\pgfsys@defobject{currentmarker}{\pgfqpoint{0.000000in}{0.000000in}}{\pgfqpoint{0.041667in}{0.000000in}}{%
\pgfpathmoveto{\pgfqpoint{0.000000in}{0.000000in}}%
\pgfpathlineto{\pgfqpoint{0.041667in}{0.000000in}}%
\pgfusepath{stroke,fill}%
}%
\begin{pgfscope}%
\pgfsys@transformshift{0.481681in}{0.739656in}%
\pgfsys@useobject{currentmarker}{}%
\end{pgfscope}%
\end{pgfscope}%
\begin{pgfscope}%
\pgfsetbuttcap%
\pgfsetroundjoin%
\definecolor{currentfill}{rgb}{0.000000,0.000000,0.000000}%
\pgfsetfillcolor{currentfill}%
\pgfsetlinewidth{0.501875pt}%
\definecolor{currentstroke}{rgb}{0.000000,0.000000,0.000000}%
\pgfsetstrokecolor{currentstroke}%
\pgfsetdash{}{0pt}%
\pgfsys@defobject{currentmarker}{\pgfqpoint{-0.041667in}{0.000000in}}{\pgfqpoint{-0.000000in}{0.000000in}}{%
\pgfpathmoveto{\pgfqpoint{-0.000000in}{0.000000in}}%
\pgfpathlineto{\pgfqpoint{-0.041667in}{0.000000in}}%
\pgfusepath{stroke,fill}%
}%
\begin{pgfscope}%
\pgfsys@transformshift{6.267353in}{0.739656in}%
\pgfsys@useobject{currentmarker}{}%
\end{pgfscope}%
\end{pgfscope}%
\begin{pgfscope}%
\definecolor{textcolor}{rgb}{0.000000,0.000000,0.000000}%
\pgfsetstrokecolor{textcolor}%
\pgfsetfillcolor{textcolor}%
\pgftext[x=0.291028in, y=0.705920in, left, base]{\color{textcolor}\rmfamily\fontsize{7.000000}{8.400000}\selectfont 0.0}%
\end{pgfscope}%
\begin{pgfscope}%
\pgfsetbuttcap%
\pgfsetroundjoin%
\definecolor{currentfill}{rgb}{0.000000,0.000000,0.000000}%
\pgfsetfillcolor{currentfill}%
\pgfsetlinewidth{0.501875pt}%
\definecolor{currentstroke}{rgb}{0.000000,0.000000,0.000000}%
\pgfsetstrokecolor{currentstroke}%
\pgfsetdash{}{0pt}%
\pgfsys@defobject{currentmarker}{\pgfqpoint{0.000000in}{0.000000in}}{\pgfqpoint{0.041667in}{0.000000in}}{%
\pgfpathmoveto{\pgfqpoint{0.000000in}{0.000000in}}%
\pgfpathlineto{\pgfqpoint{0.041667in}{0.000000in}}%
\pgfusepath{stroke,fill}%
}%
\begin{pgfscope}%
\pgfsys@transformshift{0.481681in}{0.875289in}%
\pgfsys@useobject{currentmarker}{}%
\end{pgfscope}%
\end{pgfscope}%
\begin{pgfscope}%
\pgfsetbuttcap%
\pgfsetroundjoin%
\definecolor{currentfill}{rgb}{0.000000,0.000000,0.000000}%
\pgfsetfillcolor{currentfill}%
\pgfsetlinewidth{0.501875pt}%
\definecolor{currentstroke}{rgb}{0.000000,0.000000,0.000000}%
\pgfsetstrokecolor{currentstroke}%
\pgfsetdash{}{0pt}%
\pgfsys@defobject{currentmarker}{\pgfqpoint{-0.041667in}{0.000000in}}{\pgfqpoint{-0.000000in}{0.000000in}}{%
\pgfpathmoveto{\pgfqpoint{-0.000000in}{0.000000in}}%
\pgfpathlineto{\pgfqpoint{-0.041667in}{0.000000in}}%
\pgfusepath{stroke,fill}%
}%
\begin{pgfscope}%
\pgfsys@transformshift{6.267353in}{0.875289in}%
\pgfsys@useobject{currentmarker}{}%
\end{pgfscope}%
\end{pgfscope}%
\begin{pgfscope}%
\definecolor{textcolor}{rgb}{0.000000,0.000000,0.000000}%
\pgfsetstrokecolor{textcolor}%
\pgfsetfillcolor{textcolor}%
\pgftext[x=0.291028in, y=0.841552in, left, base]{\color{textcolor}\rmfamily\fontsize{7.000000}{8.400000}\selectfont 0.5}%
\end{pgfscope}%
\begin{pgfscope}%
\pgfsetbuttcap%
\pgfsetroundjoin%
\definecolor{currentfill}{rgb}{0.000000,0.000000,0.000000}%
\pgfsetfillcolor{currentfill}%
\pgfsetlinewidth{0.501875pt}%
\definecolor{currentstroke}{rgb}{0.000000,0.000000,0.000000}%
\pgfsetstrokecolor{currentstroke}%
\pgfsetdash{}{0pt}%
\pgfsys@defobject{currentmarker}{\pgfqpoint{0.000000in}{0.000000in}}{\pgfqpoint{0.020833in}{0.000000in}}{%
\pgfpathmoveto{\pgfqpoint{0.000000in}{0.000000in}}%
\pgfpathlineto{\pgfqpoint{0.020833in}{0.000000in}}%
\pgfusepath{stroke,fill}%
}%
\begin{pgfscope}%
\pgfsys@transformshift{0.481681in}{0.631150in}%
\pgfsys@useobject{currentmarker}{}%
\end{pgfscope}%
\end{pgfscope}%
\begin{pgfscope}%
\pgfsetbuttcap%
\pgfsetroundjoin%
\definecolor{currentfill}{rgb}{0.000000,0.000000,0.000000}%
\pgfsetfillcolor{currentfill}%
\pgfsetlinewidth{0.501875pt}%
\definecolor{currentstroke}{rgb}{0.000000,0.000000,0.000000}%
\pgfsetstrokecolor{currentstroke}%
\pgfsetdash{}{0pt}%
\pgfsys@defobject{currentmarker}{\pgfqpoint{-0.020833in}{0.000000in}}{\pgfqpoint{-0.000000in}{0.000000in}}{%
\pgfpathmoveto{\pgfqpoint{-0.000000in}{0.000000in}}%
\pgfpathlineto{\pgfqpoint{-0.020833in}{0.000000in}}%
\pgfusepath{stroke,fill}%
}%
\begin{pgfscope}%
\pgfsys@transformshift{6.267353in}{0.631150in}%
\pgfsys@useobject{currentmarker}{}%
\end{pgfscope}%
\end{pgfscope}%
\begin{pgfscope}%
\pgfsetbuttcap%
\pgfsetroundjoin%
\definecolor{currentfill}{rgb}{0.000000,0.000000,0.000000}%
\pgfsetfillcolor{currentfill}%
\pgfsetlinewidth{0.501875pt}%
\definecolor{currentstroke}{rgb}{0.000000,0.000000,0.000000}%
\pgfsetstrokecolor{currentstroke}%
\pgfsetdash{}{0pt}%
\pgfsys@defobject{currentmarker}{\pgfqpoint{0.000000in}{0.000000in}}{\pgfqpoint{0.020833in}{0.000000in}}{%
\pgfpathmoveto{\pgfqpoint{0.000000in}{0.000000in}}%
\pgfpathlineto{\pgfqpoint{0.020833in}{0.000000in}}%
\pgfusepath{stroke,fill}%
}%
\begin{pgfscope}%
\pgfsys@transformshift{0.481681in}{0.658277in}%
\pgfsys@useobject{currentmarker}{}%
\end{pgfscope}%
\end{pgfscope}%
\begin{pgfscope}%
\pgfsetbuttcap%
\pgfsetroundjoin%
\definecolor{currentfill}{rgb}{0.000000,0.000000,0.000000}%
\pgfsetfillcolor{currentfill}%
\pgfsetlinewidth{0.501875pt}%
\definecolor{currentstroke}{rgb}{0.000000,0.000000,0.000000}%
\pgfsetstrokecolor{currentstroke}%
\pgfsetdash{}{0pt}%
\pgfsys@defobject{currentmarker}{\pgfqpoint{-0.020833in}{0.000000in}}{\pgfqpoint{-0.000000in}{0.000000in}}{%
\pgfpathmoveto{\pgfqpoint{-0.000000in}{0.000000in}}%
\pgfpathlineto{\pgfqpoint{-0.020833in}{0.000000in}}%
\pgfusepath{stroke,fill}%
}%
\begin{pgfscope}%
\pgfsys@transformshift{6.267353in}{0.658277in}%
\pgfsys@useobject{currentmarker}{}%
\end{pgfscope}%
\end{pgfscope}%
\begin{pgfscope}%
\pgfsetbuttcap%
\pgfsetroundjoin%
\definecolor{currentfill}{rgb}{0.000000,0.000000,0.000000}%
\pgfsetfillcolor{currentfill}%
\pgfsetlinewidth{0.501875pt}%
\definecolor{currentstroke}{rgb}{0.000000,0.000000,0.000000}%
\pgfsetstrokecolor{currentstroke}%
\pgfsetdash{}{0pt}%
\pgfsys@defobject{currentmarker}{\pgfqpoint{0.000000in}{0.000000in}}{\pgfqpoint{0.020833in}{0.000000in}}{%
\pgfpathmoveto{\pgfqpoint{0.000000in}{0.000000in}}%
\pgfpathlineto{\pgfqpoint{0.020833in}{0.000000in}}%
\pgfusepath{stroke,fill}%
}%
\begin{pgfscope}%
\pgfsys@transformshift{0.481681in}{0.685403in}%
\pgfsys@useobject{currentmarker}{}%
\end{pgfscope}%
\end{pgfscope}%
\begin{pgfscope}%
\pgfsetbuttcap%
\pgfsetroundjoin%
\definecolor{currentfill}{rgb}{0.000000,0.000000,0.000000}%
\pgfsetfillcolor{currentfill}%
\pgfsetlinewidth{0.501875pt}%
\definecolor{currentstroke}{rgb}{0.000000,0.000000,0.000000}%
\pgfsetstrokecolor{currentstroke}%
\pgfsetdash{}{0pt}%
\pgfsys@defobject{currentmarker}{\pgfqpoint{-0.020833in}{0.000000in}}{\pgfqpoint{-0.000000in}{0.000000in}}{%
\pgfpathmoveto{\pgfqpoint{-0.000000in}{0.000000in}}%
\pgfpathlineto{\pgfqpoint{-0.020833in}{0.000000in}}%
\pgfusepath{stroke,fill}%
}%
\begin{pgfscope}%
\pgfsys@transformshift{6.267353in}{0.685403in}%
\pgfsys@useobject{currentmarker}{}%
\end{pgfscope}%
\end{pgfscope}%
\begin{pgfscope}%
\pgfsetbuttcap%
\pgfsetroundjoin%
\definecolor{currentfill}{rgb}{0.000000,0.000000,0.000000}%
\pgfsetfillcolor{currentfill}%
\pgfsetlinewidth{0.501875pt}%
\definecolor{currentstroke}{rgb}{0.000000,0.000000,0.000000}%
\pgfsetstrokecolor{currentstroke}%
\pgfsetdash{}{0pt}%
\pgfsys@defobject{currentmarker}{\pgfqpoint{0.000000in}{0.000000in}}{\pgfqpoint{0.020833in}{0.000000in}}{%
\pgfpathmoveto{\pgfqpoint{0.000000in}{0.000000in}}%
\pgfpathlineto{\pgfqpoint{0.020833in}{0.000000in}}%
\pgfusepath{stroke,fill}%
}%
\begin{pgfscope}%
\pgfsys@transformshift{0.481681in}{0.712530in}%
\pgfsys@useobject{currentmarker}{}%
\end{pgfscope}%
\end{pgfscope}%
\begin{pgfscope}%
\pgfsetbuttcap%
\pgfsetroundjoin%
\definecolor{currentfill}{rgb}{0.000000,0.000000,0.000000}%
\pgfsetfillcolor{currentfill}%
\pgfsetlinewidth{0.501875pt}%
\definecolor{currentstroke}{rgb}{0.000000,0.000000,0.000000}%
\pgfsetstrokecolor{currentstroke}%
\pgfsetdash{}{0pt}%
\pgfsys@defobject{currentmarker}{\pgfqpoint{-0.020833in}{0.000000in}}{\pgfqpoint{-0.000000in}{0.000000in}}{%
\pgfpathmoveto{\pgfqpoint{-0.000000in}{0.000000in}}%
\pgfpathlineto{\pgfqpoint{-0.020833in}{0.000000in}}%
\pgfusepath{stroke,fill}%
}%
\begin{pgfscope}%
\pgfsys@transformshift{6.267353in}{0.712530in}%
\pgfsys@useobject{currentmarker}{}%
\end{pgfscope}%
\end{pgfscope}%
\begin{pgfscope}%
\pgfsetbuttcap%
\pgfsetroundjoin%
\definecolor{currentfill}{rgb}{0.000000,0.000000,0.000000}%
\pgfsetfillcolor{currentfill}%
\pgfsetlinewidth{0.501875pt}%
\definecolor{currentstroke}{rgb}{0.000000,0.000000,0.000000}%
\pgfsetstrokecolor{currentstroke}%
\pgfsetdash{}{0pt}%
\pgfsys@defobject{currentmarker}{\pgfqpoint{0.000000in}{0.000000in}}{\pgfqpoint{0.020833in}{0.000000in}}{%
\pgfpathmoveto{\pgfqpoint{0.000000in}{0.000000in}}%
\pgfpathlineto{\pgfqpoint{0.020833in}{0.000000in}}%
\pgfusepath{stroke,fill}%
}%
\begin{pgfscope}%
\pgfsys@transformshift{0.481681in}{0.766783in}%
\pgfsys@useobject{currentmarker}{}%
\end{pgfscope}%
\end{pgfscope}%
\begin{pgfscope}%
\pgfsetbuttcap%
\pgfsetroundjoin%
\definecolor{currentfill}{rgb}{0.000000,0.000000,0.000000}%
\pgfsetfillcolor{currentfill}%
\pgfsetlinewidth{0.501875pt}%
\definecolor{currentstroke}{rgb}{0.000000,0.000000,0.000000}%
\pgfsetstrokecolor{currentstroke}%
\pgfsetdash{}{0pt}%
\pgfsys@defobject{currentmarker}{\pgfqpoint{-0.020833in}{0.000000in}}{\pgfqpoint{-0.000000in}{0.000000in}}{%
\pgfpathmoveto{\pgfqpoint{-0.000000in}{0.000000in}}%
\pgfpathlineto{\pgfqpoint{-0.020833in}{0.000000in}}%
\pgfusepath{stroke,fill}%
}%
\begin{pgfscope}%
\pgfsys@transformshift{6.267353in}{0.766783in}%
\pgfsys@useobject{currentmarker}{}%
\end{pgfscope}%
\end{pgfscope}%
\begin{pgfscope}%
\pgfsetbuttcap%
\pgfsetroundjoin%
\definecolor{currentfill}{rgb}{0.000000,0.000000,0.000000}%
\pgfsetfillcolor{currentfill}%
\pgfsetlinewidth{0.501875pt}%
\definecolor{currentstroke}{rgb}{0.000000,0.000000,0.000000}%
\pgfsetstrokecolor{currentstroke}%
\pgfsetdash{}{0pt}%
\pgfsys@defobject{currentmarker}{\pgfqpoint{0.000000in}{0.000000in}}{\pgfqpoint{0.020833in}{0.000000in}}{%
\pgfpathmoveto{\pgfqpoint{0.000000in}{0.000000in}}%
\pgfpathlineto{\pgfqpoint{0.020833in}{0.000000in}}%
\pgfusepath{stroke,fill}%
}%
\begin{pgfscope}%
\pgfsys@transformshift{0.481681in}{0.793909in}%
\pgfsys@useobject{currentmarker}{}%
\end{pgfscope}%
\end{pgfscope}%
\begin{pgfscope}%
\pgfsetbuttcap%
\pgfsetroundjoin%
\definecolor{currentfill}{rgb}{0.000000,0.000000,0.000000}%
\pgfsetfillcolor{currentfill}%
\pgfsetlinewidth{0.501875pt}%
\definecolor{currentstroke}{rgb}{0.000000,0.000000,0.000000}%
\pgfsetstrokecolor{currentstroke}%
\pgfsetdash{}{0pt}%
\pgfsys@defobject{currentmarker}{\pgfqpoint{-0.020833in}{0.000000in}}{\pgfqpoint{-0.000000in}{0.000000in}}{%
\pgfpathmoveto{\pgfqpoint{-0.000000in}{0.000000in}}%
\pgfpathlineto{\pgfqpoint{-0.020833in}{0.000000in}}%
\pgfusepath{stroke,fill}%
}%
\begin{pgfscope}%
\pgfsys@transformshift{6.267353in}{0.793909in}%
\pgfsys@useobject{currentmarker}{}%
\end{pgfscope}%
\end{pgfscope}%
\begin{pgfscope}%
\pgfsetbuttcap%
\pgfsetroundjoin%
\definecolor{currentfill}{rgb}{0.000000,0.000000,0.000000}%
\pgfsetfillcolor{currentfill}%
\pgfsetlinewidth{0.501875pt}%
\definecolor{currentstroke}{rgb}{0.000000,0.000000,0.000000}%
\pgfsetstrokecolor{currentstroke}%
\pgfsetdash{}{0pt}%
\pgfsys@defobject{currentmarker}{\pgfqpoint{0.000000in}{0.000000in}}{\pgfqpoint{0.020833in}{0.000000in}}{%
\pgfpathmoveto{\pgfqpoint{0.000000in}{0.000000in}}%
\pgfpathlineto{\pgfqpoint{0.020833in}{0.000000in}}%
\pgfusepath{stroke,fill}%
}%
\begin{pgfscope}%
\pgfsys@transformshift{0.481681in}{0.821036in}%
\pgfsys@useobject{currentmarker}{}%
\end{pgfscope}%
\end{pgfscope}%
\begin{pgfscope}%
\pgfsetbuttcap%
\pgfsetroundjoin%
\definecolor{currentfill}{rgb}{0.000000,0.000000,0.000000}%
\pgfsetfillcolor{currentfill}%
\pgfsetlinewidth{0.501875pt}%
\definecolor{currentstroke}{rgb}{0.000000,0.000000,0.000000}%
\pgfsetstrokecolor{currentstroke}%
\pgfsetdash{}{0pt}%
\pgfsys@defobject{currentmarker}{\pgfqpoint{-0.020833in}{0.000000in}}{\pgfqpoint{-0.000000in}{0.000000in}}{%
\pgfpathmoveto{\pgfqpoint{-0.000000in}{0.000000in}}%
\pgfpathlineto{\pgfqpoint{-0.020833in}{0.000000in}}%
\pgfusepath{stroke,fill}%
}%
\begin{pgfscope}%
\pgfsys@transformshift{6.267353in}{0.821036in}%
\pgfsys@useobject{currentmarker}{}%
\end{pgfscope}%
\end{pgfscope}%
\begin{pgfscope}%
\pgfsetbuttcap%
\pgfsetroundjoin%
\definecolor{currentfill}{rgb}{0.000000,0.000000,0.000000}%
\pgfsetfillcolor{currentfill}%
\pgfsetlinewidth{0.501875pt}%
\definecolor{currentstroke}{rgb}{0.000000,0.000000,0.000000}%
\pgfsetstrokecolor{currentstroke}%
\pgfsetdash{}{0pt}%
\pgfsys@defobject{currentmarker}{\pgfqpoint{0.000000in}{0.000000in}}{\pgfqpoint{0.020833in}{0.000000in}}{%
\pgfpathmoveto{\pgfqpoint{0.000000in}{0.000000in}}%
\pgfpathlineto{\pgfqpoint{0.020833in}{0.000000in}}%
\pgfusepath{stroke,fill}%
}%
\begin{pgfscope}%
\pgfsys@transformshift{0.481681in}{0.848162in}%
\pgfsys@useobject{currentmarker}{}%
\end{pgfscope}%
\end{pgfscope}%
\begin{pgfscope}%
\pgfsetbuttcap%
\pgfsetroundjoin%
\definecolor{currentfill}{rgb}{0.000000,0.000000,0.000000}%
\pgfsetfillcolor{currentfill}%
\pgfsetlinewidth{0.501875pt}%
\definecolor{currentstroke}{rgb}{0.000000,0.000000,0.000000}%
\pgfsetstrokecolor{currentstroke}%
\pgfsetdash{}{0pt}%
\pgfsys@defobject{currentmarker}{\pgfqpoint{-0.020833in}{0.000000in}}{\pgfqpoint{-0.000000in}{0.000000in}}{%
\pgfpathmoveto{\pgfqpoint{-0.000000in}{0.000000in}}%
\pgfpathlineto{\pgfqpoint{-0.020833in}{0.000000in}}%
\pgfusepath{stroke,fill}%
}%
\begin{pgfscope}%
\pgfsys@transformshift{6.267353in}{0.848162in}%
\pgfsys@useobject{currentmarker}{}%
\end{pgfscope}%
\end{pgfscope}%
\begin{pgfscope}%
\definecolor{textcolor}{rgb}{0.000000,0.000000,0.000000}%
\pgfsetstrokecolor{textcolor}%
\pgfsetfillcolor{textcolor}%
\pgftext[x=0.148667in,y=0.739656in,,bottom,rotate=90.000000]{\color{textcolor}\rmfamily\fontsize{8.000000}{9.600000}\selectfont Unterschied}%
\end{pgfscope}%
\begin{pgfscope}%
\pgfsetrectcap%
\pgfsetmiterjoin%
\pgfsetlinewidth{0.501875pt}%
\definecolor{currentstroke}{rgb}{0.000000,0.000000,0.000000}%
\pgfsetstrokecolor{currentstroke}%
\pgfsetdash{}{0pt}%
\pgfpathmoveto{\pgfqpoint{0.481681in}{0.586309in}}%
\pgfpathlineto{\pgfqpoint{0.481681in}{0.893003in}}%
\pgfusepath{stroke}%
\end{pgfscope}%
\begin{pgfscope}%
\pgfsetrectcap%
\pgfsetmiterjoin%
\pgfsetlinewidth{0.501875pt}%
\definecolor{currentstroke}{rgb}{0.000000,0.000000,0.000000}%
\pgfsetstrokecolor{currentstroke}%
\pgfsetdash{}{0pt}%
\pgfpathmoveto{\pgfqpoint{6.267353in}{0.586309in}}%
\pgfpathlineto{\pgfqpoint{6.267353in}{0.893003in}}%
\pgfusepath{stroke}%
\end{pgfscope}%
\begin{pgfscope}%
\pgfsetrectcap%
\pgfsetmiterjoin%
\pgfsetlinewidth{0.501875pt}%
\definecolor{currentstroke}{rgb}{0.000000,0.000000,0.000000}%
\pgfsetstrokecolor{currentstroke}%
\pgfsetdash{}{0pt}%
\pgfpathmoveto{\pgfqpoint{0.481681in}{0.586309in}}%
\pgfpathlineto{\pgfqpoint{6.267353in}{0.586309in}}%
\pgfusepath{stroke}%
\end{pgfscope}%
\begin{pgfscope}%
\pgfsetrectcap%
\pgfsetmiterjoin%
\pgfsetlinewidth{0.501875pt}%
\definecolor{currentstroke}{rgb}{0.000000,0.000000,0.000000}%
\pgfsetstrokecolor{currentstroke}%
\pgfsetdash{}{0pt}%
\pgfpathmoveto{\pgfqpoint{0.481681in}{0.893003in}}%
\pgfpathlineto{\pgfqpoint{6.267353in}{0.893003in}}%
\pgfusepath{stroke}%
\end{pgfscope}%
\end{pgfpicture}%
\makeatother%
\endgroup%

  \end{center}
  \caption{Verhältnis von Aufwand und Komplexität im DIL NDA Projekt.}
\end{figure}

Abbildung TODO wird im Folgenden als Korrelationsgraph des Projektes
bezeichnet und soll nun erläutert werden. Die Abbildung besteht aus zwei
Diagrammen, welche übereinander angeordnet sind. In dem oberen Diagram
werden alle behandelten Grö\ss en in einem gemeinsamen Wertebereich
abgezeichnet. Die Story Points sind mit einer dickeren, schwarzen Linie
dargestellt. Die restlichen Linien beschreiben die
Codekomplexitätsmetriken. Die einzelnen Farbzuweisungen können der
Legende im oberen linken Rand der Abbildung entnommen werden. Das zweite
Diagramm in der Abbildung beschreibt die Differenz zwischen den Story
Points und dem arithmetischen Mittel aller Codekomplexitätsmetriken. Es
soll anzeigen, wo die Werte der beiden Grö\ss en besonders stark
voneinander abweichen. Der Wertebereich dieses zweiten Diagramms ist
analog zu dem Wertebereich des ersten Diagramms. Das Format dieser
Abbildung wird über alle folgenden Projekte hinweg konstant gehalten.

In dem Korrelationsgraph sind die einzelnen Zeitreihen über den
Projektzeitraum von Oktober 2018 bis November 2019 abgezeichnet. Alle
Zeitreihen folgen einer ähnlichen Linie. Diese könnte näherungsweise als
logarithmisch beschrieben werden. Nach einer stärkeren Steigung zum
Anfang des Projektes flacht die Steigung aller Grö\ss en zum Ende des
Projektes zunehmend ab. Dieser Verlauf konnte durch eine Verschiebung in
dem Ziel der Softwareentwicklung begründet werden. So wurde die Software
zum Ende hin immer langsamer weiterentwickelt. In dem Abschlussinterview
konnten die sprungartigen Anstiege der Aufwandsabschätzungen (1) durch
die Sprint-Intervalle erklärt werden. So werden die Storys vermehrt zum
Sprint-Ende als fertig markiert (Analysis Table Line 20). Des Weiteren
wurden Ende November 2018 einige gro\ss e und kurzfristige Schwankungen in
den Komplexitätsmetriken identifiziert (2 in der Abbildung). In diesem
Zeitraum wurde die Anwendung restrukturiert. Im Rahmen der
Restrukturierung wurden regelmä\ss ig gro\ss e Codeteile entfernt und wieder
hinzugefügt. Diese Schwankungen stehen also in keinem Zusammenhang zu
dem tatsächlichen Umfang der Anwendung und können ignoriert werden. Von
Mitte Dezember 2018 bis Mitte Januar 2019 ist sowohl in den
Aufwandsabschätzungen, als auch in den Codekomplexitätsmetriken ein
Plateau erkenntlich (Nummer 4). Dieser Stillstand in der
Weiterentwicklung der Software konnte auf einen generellen
Betriebsschluss aufgrund der Weihnachtszeit zurückgeführt werden. Zum
Ende des Projektes sinkt die Entwicklungsgeschwindigkeit (Nummer 3).
Hier wurden Entwickler von dem Projekt abgezogen.

Zusammengefasst entsprechen die hier gesammelten Messdaten der
Hypothese. Über den Projektverlauf steigen die berechneten
Softwarekomplexitätsgrö\ss en im grö\ss tenteils gleichen Ma\ss e wie die
Aufwandsabschätzungen. Abweichungen und Auffälligkeiten konnten mit den
Projektbeteiligten erklärt werden. Dementsprechend ist ein hoher
Korrelationsgrad zu erwarten.

In einem nächsten Schritt wurden wie in \ref{kriterien-zur-interpretation-der-daten} und \ref{implementierung-einer-untersuchungssoftware} beschrieben die
Korrelationskoeffizienten berechnet. Diese können Tabelle TODO entnommen
werden.

TODO

Für alle Untersuchungsergebnisse wurde ein P-Wert von \textless{} .00001
ermittelt.
Bei einem, für Sozial- und Wirtschaftswissenschaften üblichen
Signifikanzniveau von .05 lässt sich also sagen, dass die Ergebnisse
signifikant sind.

Für alle Komplexitätsmetriken konnten mit Korrelationskoeffizienten
zwischen 0,85 und 0,98 starke Korrelationen ermittelt werden. Mit einem
durchschnittlichen Korrelationskoeffizienten von 0.95 ist die
Korrelation bei den logischen Codezeilen am stärksten, gefolgt von der
Einrückungskomplexität (0,94), der zyklomatischen Komplexität (0,91) und
dem Aufwand nach Halstead (0,90).

\subsection{Fazit}\label{Fazit}

In diesem Fall konnten erfolgreich eine Zeitreihe der
Softwarekomplexitätsma\ss zahlen und eine der Aufwandsabschätzungen
aufgestellt werden. Es wurden wenige Störfaktoren identifiziert.
Letztendlich wurde eine zum grö\ss ten Teil sehr starke, signifikante
Korrelation zwischen den Komplexitätsma\ss en und den Aufwandsabschätzungen
festgestellt. Der Fall der Digital \ac{NDA} Application spricht also für eine
Korrelation zwischen den Codekomplexitätsmetriken und den
Aufwandsabschätzungen.

\subsection{Kritik und Messfehler}\label{Kritik-und-Messfehler}

Auch wenn in diesem Projekt einige Störfaktoren, wie z.B. ein grö\ss eres
Refactoring des Codes aufgefallen sind, kann das Analyseergebnis
trotzdem als valide angenommen werden.

\section{inGRID}\label{ingrid}

Als zweiter Fall wird das interactive Generic Reporting Insight
Dashboard (inGRID) System untersucht. Das inGRID-System ist eine von DXC
entwickelte Anwendung, die eine fortschrittliche Monitoring- und
Reporting-Lösung bietet. Sie wurde als Software as a Service Angebot mit
einer zentralen Installation in der deutschen Cloud konzipiert. Eine
Vielzahl von Agenten können verwendet werden, um Systeme zu überwachen.
Die Agenten decken ein breites Spektrum an Überwachungsaktivitäten ab.
Dazu gehören \ac{VM}s, Middlewares, Anwendungen und andere Geräte (über
SNMP). Die inGRID Anwendung wird von einem Offshore Team hauptsächlich
für Kunden in Europa entwickelt. Das Entwicklungsteam ist in Ägypten
lokalisiert und besteht aus neun Entwickler*innen. Davon haben vier
Entwickler*innen mehr als drei Jahre Berufserfahrung und fünf
Entwickler*innen weniger als drei Jahre Berufserfahrung. Die
Entwickler*innen sind seit ungefähr zweieinhalb Jahren in dem Projekt
involviert.

Das inGRID System besteht aus einer Vielzahl von individuellen
Komponenten. Da die Analyse aller Komponenten für den Rahmen dieser
Arbeit zu umfangreich wäre, wird hier lediglich die Backend Komponente
des webbasierten Service Status Dashboards (SSD-Backend-Komponente)
betrachtet. Während in der gesamten Anwendung eine Vielzahl von
verschiedenen Programmiersprachen verwendet wird, besteht die
SSD-Backend-Komponente zum Gro\ss teil (93\%) aus
Java Code.

InGRID wird agil nach Scrum in Sprints mit einer Dauer von jeweils drei
Wochen realisiert. Innerhalb dieser drei Wochen werden Features von dem
Backlog des Projektes umgesetzt. Der reguläre Ablauf der Umsetzung von
Features in dem inGRID Projekt konnte von einem der Entwickler erläutert
werden. Der Ablauf beginnt mit der Anfrage eines neuen Features von
einer, am Projekt beteiligten Person. Für dieses Feature wird nun eine
User Story geschrieben und ein neuer Entwicklungszweig (englisch Branch)
in der Quelltextverwaltung erstellt. In diesem neuen Entwicklungszweig
wird nun neuer Quelltext hinzugefügt bzw. alter modifiziert. Wenn das
Feature fertig entwickelt ist, wird es mit einem zentralen
Entwicklungszweig verschmolzen (gemerged). Hier wird es möglichst
innerhalb von 24 bis 48 Stunden vollständig getestet. Wenn alle
Akzeptanzkriterien der User Story erfüllt sind, wird die User Story
geschlossen.

\subsection{Datenerhebung}\label{ingrid-Datenerhebung}

Für die SSD-Backend Komponente der inGRID Anwendung soll nun wie auch in
dem Fall der \ac{NDA} Anwendung eine Zeitreihe der Komplexitätsmetriken und
eine Zeitreihe der Aufwandsabschätzungen aufgestellt werden.

Zunächst wird hier die Datenerhebung für die Zeitreihe der
Aufwandsabschätzungen beschrieben. Dies geschieht anhand User Storys. Von besonderem Interesse für die Analyse sind
die Felder „Story Points`` und „Resolved Date``. Die Kombination beider
Felder kann verwendet werden, um die Zeitreihe der Aufwandsabschätzungen
aufzubauen. In dem Feld Story Points wird der Aufwand zur Realisierung
der Story geschätzt. Mit dem, in \ref{Aufwandsabschatzungen-mit-Planning-Poker} beschriebenen
Planning-Poker-Verfahren wird der Umfang durch eine Kombination von
Expertenmeinungen als eine Zahl in der Fibonacci Sequenz beschrieben.
Abweichend von der in \ref{aufwandsabschuxe4tzungen-agiler-projekte} beschriebenen Vorgehensweise werden in diesem
Projekt jedoch auch die Erfahrung der Entwickler, sowie der abgeschätzte
Aufwand in die Story Point Schätzungen mitaufgenommen. Das Feld
„Resolved Date`` ist für die zeitliche Verortung der Aufwandsabschätzung
der User Story relevant. Im Interview mit einem der Entwickler konnte
ermittelt werden, dass dieses Feld den Zeitpunkt der Fertigstellung
einer Story beinhaltet\footcite[Vgl. ][]{entwicklerInterviewMitEntwickler2022}. Es kann also mit diesem Feld
gesagt werden, wann der, in der User Story spezifizierte Aufwand in die
Anwendung geflossen ist. In dem Experteninterview\footcite[Vgl. ][]{entwicklerInterviewMitEntwickler2022} mit einem der Entwickler konnte an dieser Stelle eine
Verbindung zwischen den Daten aus der Projektmanagementsoftware und den
Messdaten aus dem Code geschlussfolgert werden. Wenn eine Story
geschlossen wird, wird auch der Code dieser Story in den „develop``
Branch gemerged. Also lässt sich sagen, dass der Zeitpunkt in dem
„Resolved Date`` der Story ungefähr dem Zeitpunkt der Änderungen in dem
Quelltext der Anwendung entsprechen muss. Laut dem Entwickler betrügen
die Abweichungen zwischen diesen beiden Zeitpunkten in mehr als 95\% der
Fälle weniger als 48 Stunden. In Relation zu der Gesamtdauer des
Projektes kann diese Abweichung bei einem Signifikanzniveau von 5\% als
nicht signifikant angesehen werden. Die Verbindung beider Ereignisse
über dieses Merkmal ist also valide.

Für die Analyse der User Storys anhand der zuvor beschriebenen Merkmale
müssen die User Storys eine Reihe von Kriterien erfüllen. Anhand dieser
Kriterien wird eine Vorauswahl der User Storys getroffen. Dabei werden
alle User Storys der Komponente „SSD Backend`` ausgewählt, die einen
Status von „Done`` in dem Feld „Resolution`` haben, und denen eine
Anzahl an Story Points zugewiesen ist. Jira bietet eine Möglichkeit zur
genauen Abfrage von Storys mithilfe der proprietären Abfragesprache \ac{JQL}. Die Syntax von \ac{JQL} ist stark an die Syntax der
Abfragesprache \ac{SQL} angelehnt\footcite[Vgl. ][]{atlassianptyltdUseAdvancedSearch2022}.
Die \ac{JQL} Abfrage für die User Storys findet sich im folgenden Code-Fragment.

\lstset{language=SQL}
\begin{lstlisting}
project = HCC AND component = "SSD Backend" AND issuetype = Story AND resolution = Done AND "Story Points" is not EMPTY ORDER BY resolved DESC, priority DESC, updated DESC
\end{lstlisting}

Mit dieser Abfrage konnten in dem inGRID Projekt 62 User Storys
ausgewählt werden. Diese wurden als \ac{CSV}-Datei exportiert und in die
Analysesoftware geladen.

Für die zweite Zeitreihe der Quelltextkomplexitätsmessungen wird ein
Quelltextrepository benötigt. Mit der Unterstützung des Entwicklers des
Projektes konnte das passende Repository ausgewählt werden. In dem
Repository konnten in dem entsprechenden Entwicklungszweig „develop``
2554 Commits identifiziert werden. Zu jedem dieser Commits kann der
Stand des Quelltextes rekonstruiert werden. Anhand dieser Rekonstruktion
kann, wie in 4.7.2 beschrieben eine Zeitreihe der Komplexitätsmessungen
erstellt werden.

\subsection{Auswertung}\label{ingrid-Auswertung}

Im Sinne der Fallstudienforschung werden in der Auswertung alle Daten in
Relation zueinander gebracht, um so zu logischen Schlüssen zu gelangen.
Wie in \ref{verbindung-von-daten-und-hypothesen} beschrieben, werden die Aufwandsabschätzungen zunächst in
Relation zu den Komplexitätsmetriken gebracht. Die so erlangten Daten
werden, wie in Kapitel \ref{verbindung-von-daten-und-hypothesen} beschrieben und verarbeitet. Der auf diese Weise
erlangte Korrelationsgraph ist in Abbildung TODO zu sehen und wird nun
beschrieben.

\begin{figure}\label{ingrid-graph}
  \begin{center}
      %% Creator: Matplotlib, PGF backend
%%
%% To include the figure in your LaTeX document, write
%%   \input{<filename>.pgf}
%%
%% Make sure the required packages are loaded in your preamble
%%   \usepackage{pgf}
%%
%% Also ensure that all the required font packages are loaded; for instance,
%% the lmodern package is sometimes necessary when using math font.
%%   \usepackage{lmodern}
%%
%% Figures using additional raster images can only be included by \input if
%% they are in the same directory as the main LaTeX file. For loading figures
%% from other directories you can use the `import` package
%%   \usepackage{import}
%%
%% and then include the figures with
%%   \import{<path to file>}{<filename>.pgf}
%%
%% Matplotlib used the following preamble
%%   \usepackage{fontspec}
%%
\begingroup%
\makeatletter%
\begin{pgfpicture}%
\pgfpathrectangle{\pgfpointorigin}{\pgfqpoint{6.317353in}{3.277753in}}%
\pgfusepath{use as bounding box, clip}%
\begin{pgfscope}%
\pgfsetbuttcap%
\pgfsetmiterjoin%
\definecolor{currentfill}{rgb}{1.000000,1.000000,1.000000}%
\pgfsetfillcolor{currentfill}%
\pgfsetlinewidth{0.000000pt}%
\definecolor{currentstroke}{rgb}{1.000000,1.000000,1.000000}%
\pgfsetstrokecolor{currentstroke}%
\pgfsetdash{}{0pt}%
\pgfpathmoveto{\pgfqpoint{0.000000in}{-0.000000in}}%
\pgfpathlineto{\pgfqpoint{6.317353in}{-0.000000in}}%
\pgfpathlineto{\pgfqpoint{6.317353in}{3.277753in}}%
\pgfpathlineto{\pgfqpoint{0.000000in}{3.277753in}}%
\pgfpathlineto{\pgfqpoint{0.000000in}{-0.000000in}}%
\pgfpathclose%
\pgfusepath{fill}%
\end{pgfscope}%
\begin{pgfscope}%
\pgfsetbuttcap%
\pgfsetmiterjoin%
\definecolor{currentfill}{rgb}{1.000000,1.000000,1.000000}%
\pgfsetfillcolor{currentfill}%
\pgfsetlinewidth{0.000000pt}%
\definecolor{currentstroke}{rgb}{0.000000,0.000000,0.000000}%
\pgfsetstrokecolor{currentstroke}%
\pgfsetstrokeopacity{0.000000}%
\pgfsetdash{}{0pt}%
\pgfpathmoveto{\pgfqpoint{0.481681in}{1.080890in}}%
\pgfpathlineto{\pgfqpoint{6.267353in}{1.080890in}}%
\pgfpathlineto{\pgfqpoint{6.267353in}{3.227753in}}%
\pgfpathlineto{\pgfqpoint{0.481681in}{3.227753in}}%
\pgfpathlineto{\pgfqpoint{0.481681in}{1.080890in}}%
\pgfpathclose%
\pgfusepath{fill}%
\end{pgfscope}%
\begin{pgfscope}%
\pgfpathrectangle{\pgfqpoint{0.481681in}{1.080890in}}{\pgfqpoint{5.785672in}{2.146863in}}%
\pgfusepath{clip}%
\pgfsetrectcap%
\pgfsetroundjoin%
\pgfsetlinewidth{0.100375pt}%
\definecolor{currentstroke}{rgb}{0.501961,0.501961,0.501961}%
\pgfsetstrokecolor{currentstroke}%
\pgfsetdash{}{0pt}%
\pgfpathmoveto{\pgfqpoint{0.689633in}{1.080890in}}%
\pgfpathlineto{\pgfqpoint{0.689633in}{3.227753in}}%
\pgfusepath{stroke}%
\end{pgfscope}%
\begin{pgfscope}%
\pgfsetbuttcap%
\pgfsetroundjoin%
\definecolor{currentfill}{rgb}{0.000000,0.000000,0.000000}%
\pgfsetfillcolor{currentfill}%
\pgfsetlinewidth{0.501875pt}%
\definecolor{currentstroke}{rgb}{0.000000,0.000000,0.000000}%
\pgfsetstrokecolor{currentstroke}%
\pgfsetdash{}{0pt}%
\pgfsys@defobject{currentmarker}{\pgfqpoint{0.000000in}{0.000000in}}{\pgfqpoint{0.000000in}{0.041667in}}{%
\pgfpathmoveto{\pgfqpoint{0.000000in}{0.000000in}}%
\pgfpathlineto{\pgfqpoint{0.000000in}{0.041667in}}%
\pgfusepath{stroke,fill}%
}%
\begin{pgfscope}%
\pgfsys@transformshift{0.689633in}{1.080890in}%
\pgfsys@useobject{currentmarker}{}%
\end{pgfscope}%
\end{pgfscope}%
\begin{pgfscope}%
\pgfsetbuttcap%
\pgfsetroundjoin%
\definecolor{currentfill}{rgb}{0.000000,0.000000,0.000000}%
\pgfsetfillcolor{currentfill}%
\pgfsetlinewidth{0.501875pt}%
\definecolor{currentstroke}{rgb}{0.000000,0.000000,0.000000}%
\pgfsetstrokecolor{currentstroke}%
\pgfsetdash{}{0pt}%
\pgfsys@defobject{currentmarker}{\pgfqpoint{0.000000in}{-0.041667in}}{\pgfqpoint{0.000000in}{0.000000in}}{%
\pgfpathmoveto{\pgfqpoint{0.000000in}{0.000000in}}%
\pgfpathlineto{\pgfqpoint{0.000000in}{-0.041667in}}%
\pgfusepath{stroke,fill}%
}%
\begin{pgfscope}%
\pgfsys@transformshift{0.689633in}{3.227753in}%
\pgfsys@useobject{currentmarker}{}%
\end{pgfscope}%
\end{pgfscope}%
\begin{pgfscope}%
\pgfpathrectangle{\pgfqpoint{0.481681in}{1.080890in}}{\pgfqpoint{5.785672in}{2.146863in}}%
\pgfusepath{clip}%
\pgfsetrectcap%
\pgfsetroundjoin%
\pgfsetlinewidth{0.100375pt}%
\definecolor{currentstroke}{rgb}{0.501961,0.501961,0.501961}%
\pgfsetstrokecolor{currentstroke}%
\pgfsetdash{}{0pt}%
\pgfpathmoveto{\pgfqpoint{1.118000in}{1.080890in}}%
\pgfpathlineto{\pgfqpoint{1.118000in}{3.227753in}}%
\pgfusepath{stroke}%
\end{pgfscope}%
\begin{pgfscope}%
\pgfsetbuttcap%
\pgfsetroundjoin%
\definecolor{currentfill}{rgb}{0.000000,0.000000,0.000000}%
\pgfsetfillcolor{currentfill}%
\pgfsetlinewidth{0.501875pt}%
\definecolor{currentstroke}{rgb}{0.000000,0.000000,0.000000}%
\pgfsetstrokecolor{currentstroke}%
\pgfsetdash{}{0pt}%
\pgfsys@defobject{currentmarker}{\pgfqpoint{0.000000in}{0.000000in}}{\pgfqpoint{0.000000in}{0.041667in}}{%
\pgfpathmoveto{\pgfqpoint{0.000000in}{0.000000in}}%
\pgfpathlineto{\pgfqpoint{0.000000in}{0.041667in}}%
\pgfusepath{stroke,fill}%
}%
\begin{pgfscope}%
\pgfsys@transformshift{1.118000in}{1.080890in}%
\pgfsys@useobject{currentmarker}{}%
\end{pgfscope}%
\end{pgfscope}%
\begin{pgfscope}%
\pgfsetbuttcap%
\pgfsetroundjoin%
\definecolor{currentfill}{rgb}{0.000000,0.000000,0.000000}%
\pgfsetfillcolor{currentfill}%
\pgfsetlinewidth{0.501875pt}%
\definecolor{currentstroke}{rgb}{0.000000,0.000000,0.000000}%
\pgfsetstrokecolor{currentstroke}%
\pgfsetdash{}{0pt}%
\pgfsys@defobject{currentmarker}{\pgfqpoint{0.000000in}{-0.041667in}}{\pgfqpoint{0.000000in}{0.000000in}}{%
\pgfpathmoveto{\pgfqpoint{0.000000in}{0.000000in}}%
\pgfpathlineto{\pgfqpoint{0.000000in}{-0.041667in}}%
\pgfusepath{stroke,fill}%
}%
\begin{pgfscope}%
\pgfsys@transformshift{1.118000in}{3.227753in}%
\pgfsys@useobject{currentmarker}{}%
\end{pgfscope}%
\end{pgfscope}%
\begin{pgfscope}%
\pgfpathrectangle{\pgfqpoint{0.481681in}{1.080890in}}{\pgfqpoint{5.785672in}{2.146863in}}%
\pgfusepath{clip}%
\pgfsetrectcap%
\pgfsetroundjoin%
\pgfsetlinewidth{0.100375pt}%
\definecolor{currentstroke}{rgb}{0.501961,0.501961,0.501961}%
\pgfsetstrokecolor{currentstroke}%
\pgfsetdash{}{0pt}%
\pgfpathmoveto{\pgfqpoint{1.546368in}{1.080890in}}%
\pgfpathlineto{\pgfqpoint{1.546368in}{3.227753in}}%
\pgfusepath{stroke}%
\end{pgfscope}%
\begin{pgfscope}%
\pgfsetbuttcap%
\pgfsetroundjoin%
\definecolor{currentfill}{rgb}{0.000000,0.000000,0.000000}%
\pgfsetfillcolor{currentfill}%
\pgfsetlinewidth{0.501875pt}%
\definecolor{currentstroke}{rgb}{0.000000,0.000000,0.000000}%
\pgfsetstrokecolor{currentstroke}%
\pgfsetdash{}{0pt}%
\pgfsys@defobject{currentmarker}{\pgfqpoint{0.000000in}{0.000000in}}{\pgfqpoint{0.000000in}{0.041667in}}{%
\pgfpathmoveto{\pgfqpoint{0.000000in}{0.000000in}}%
\pgfpathlineto{\pgfqpoint{0.000000in}{0.041667in}}%
\pgfusepath{stroke,fill}%
}%
\begin{pgfscope}%
\pgfsys@transformshift{1.546368in}{1.080890in}%
\pgfsys@useobject{currentmarker}{}%
\end{pgfscope}%
\end{pgfscope}%
\begin{pgfscope}%
\pgfsetbuttcap%
\pgfsetroundjoin%
\definecolor{currentfill}{rgb}{0.000000,0.000000,0.000000}%
\pgfsetfillcolor{currentfill}%
\pgfsetlinewidth{0.501875pt}%
\definecolor{currentstroke}{rgb}{0.000000,0.000000,0.000000}%
\pgfsetstrokecolor{currentstroke}%
\pgfsetdash{}{0pt}%
\pgfsys@defobject{currentmarker}{\pgfqpoint{0.000000in}{-0.041667in}}{\pgfqpoint{0.000000in}{0.000000in}}{%
\pgfpathmoveto{\pgfqpoint{0.000000in}{0.000000in}}%
\pgfpathlineto{\pgfqpoint{0.000000in}{-0.041667in}}%
\pgfusepath{stroke,fill}%
}%
\begin{pgfscope}%
\pgfsys@transformshift{1.546368in}{3.227753in}%
\pgfsys@useobject{currentmarker}{}%
\end{pgfscope}%
\end{pgfscope}%
\begin{pgfscope}%
\pgfpathrectangle{\pgfqpoint{0.481681in}{1.080890in}}{\pgfqpoint{5.785672in}{2.146863in}}%
\pgfusepath{clip}%
\pgfsetrectcap%
\pgfsetroundjoin%
\pgfsetlinewidth{0.100375pt}%
\definecolor{currentstroke}{rgb}{0.501961,0.501961,0.501961}%
\pgfsetstrokecolor{currentstroke}%
\pgfsetdash{}{0pt}%
\pgfpathmoveto{\pgfqpoint{1.974736in}{1.080890in}}%
\pgfpathlineto{\pgfqpoint{1.974736in}{3.227753in}}%
\pgfusepath{stroke}%
\end{pgfscope}%
\begin{pgfscope}%
\pgfsetbuttcap%
\pgfsetroundjoin%
\definecolor{currentfill}{rgb}{0.000000,0.000000,0.000000}%
\pgfsetfillcolor{currentfill}%
\pgfsetlinewidth{0.501875pt}%
\definecolor{currentstroke}{rgb}{0.000000,0.000000,0.000000}%
\pgfsetstrokecolor{currentstroke}%
\pgfsetdash{}{0pt}%
\pgfsys@defobject{currentmarker}{\pgfqpoint{0.000000in}{0.000000in}}{\pgfqpoint{0.000000in}{0.041667in}}{%
\pgfpathmoveto{\pgfqpoint{0.000000in}{0.000000in}}%
\pgfpathlineto{\pgfqpoint{0.000000in}{0.041667in}}%
\pgfusepath{stroke,fill}%
}%
\begin{pgfscope}%
\pgfsys@transformshift{1.974736in}{1.080890in}%
\pgfsys@useobject{currentmarker}{}%
\end{pgfscope}%
\end{pgfscope}%
\begin{pgfscope}%
\pgfsetbuttcap%
\pgfsetroundjoin%
\definecolor{currentfill}{rgb}{0.000000,0.000000,0.000000}%
\pgfsetfillcolor{currentfill}%
\pgfsetlinewidth{0.501875pt}%
\definecolor{currentstroke}{rgb}{0.000000,0.000000,0.000000}%
\pgfsetstrokecolor{currentstroke}%
\pgfsetdash{}{0pt}%
\pgfsys@defobject{currentmarker}{\pgfqpoint{0.000000in}{-0.041667in}}{\pgfqpoint{0.000000in}{0.000000in}}{%
\pgfpathmoveto{\pgfqpoint{0.000000in}{0.000000in}}%
\pgfpathlineto{\pgfqpoint{0.000000in}{-0.041667in}}%
\pgfusepath{stroke,fill}%
}%
\begin{pgfscope}%
\pgfsys@transformshift{1.974736in}{3.227753in}%
\pgfsys@useobject{currentmarker}{}%
\end{pgfscope}%
\end{pgfscope}%
\begin{pgfscope}%
\pgfpathrectangle{\pgfqpoint{0.481681in}{1.080890in}}{\pgfqpoint{5.785672in}{2.146863in}}%
\pgfusepath{clip}%
\pgfsetrectcap%
\pgfsetroundjoin%
\pgfsetlinewidth{0.100375pt}%
\definecolor{currentstroke}{rgb}{0.501961,0.501961,0.501961}%
\pgfsetstrokecolor{currentstroke}%
\pgfsetdash{}{0pt}%
\pgfpathmoveto{\pgfqpoint{2.403104in}{1.080890in}}%
\pgfpathlineto{\pgfqpoint{2.403104in}{3.227753in}}%
\pgfusepath{stroke}%
\end{pgfscope}%
\begin{pgfscope}%
\pgfsetbuttcap%
\pgfsetroundjoin%
\definecolor{currentfill}{rgb}{0.000000,0.000000,0.000000}%
\pgfsetfillcolor{currentfill}%
\pgfsetlinewidth{0.501875pt}%
\definecolor{currentstroke}{rgb}{0.000000,0.000000,0.000000}%
\pgfsetstrokecolor{currentstroke}%
\pgfsetdash{}{0pt}%
\pgfsys@defobject{currentmarker}{\pgfqpoint{0.000000in}{0.000000in}}{\pgfqpoint{0.000000in}{0.041667in}}{%
\pgfpathmoveto{\pgfqpoint{0.000000in}{0.000000in}}%
\pgfpathlineto{\pgfqpoint{0.000000in}{0.041667in}}%
\pgfusepath{stroke,fill}%
}%
\begin{pgfscope}%
\pgfsys@transformshift{2.403104in}{1.080890in}%
\pgfsys@useobject{currentmarker}{}%
\end{pgfscope}%
\end{pgfscope}%
\begin{pgfscope}%
\pgfsetbuttcap%
\pgfsetroundjoin%
\definecolor{currentfill}{rgb}{0.000000,0.000000,0.000000}%
\pgfsetfillcolor{currentfill}%
\pgfsetlinewidth{0.501875pt}%
\definecolor{currentstroke}{rgb}{0.000000,0.000000,0.000000}%
\pgfsetstrokecolor{currentstroke}%
\pgfsetdash{}{0pt}%
\pgfsys@defobject{currentmarker}{\pgfqpoint{0.000000in}{-0.041667in}}{\pgfqpoint{0.000000in}{0.000000in}}{%
\pgfpathmoveto{\pgfqpoint{0.000000in}{0.000000in}}%
\pgfpathlineto{\pgfqpoint{0.000000in}{-0.041667in}}%
\pgfusepath{stroke,fill}%
}%
\begin{pgfscope}%
\pgfsys@transformshift{2.403104in}{3.227753in}%
\pgfsys@useobject{currentmarker}{}%
\end{pgfscope}%
\end{pgfscope}%
\begin{pgfscope}%
\pgfpathrectangle{\pgfqpoint{0.481681in}{1.080890in}}{\pgfqpoint{5.785672in}{2.146863in}}%
\pgfusepath{clip}%
\pgfsetrectcap%
\pgfsetroundjoin%
\pgfsetlinewidth{0.100375pt}%
\definecolor{currentstroke}{rgb}{0.501961,0.501961,0.501961}%
\pgfsetstrokecolor{currentstroke}%
\pgfsetdash{}{0pt}%
\pgfpathmoveto{\pgfqpoint{2.831472in}{1.080890in}}%
\pgfpathlineto{\pgfqpoint{2.831472in}{3.227753in}}%
\pgfusepath{stroke}%
\end{pgfscope}%
\begin{pgfscope}%
\pgfsetbuttcap%
\pgfsetroundjoin%
\definecolor{currentfill}{rgb}{0.000000,0.000000,0.000000}%
\pgfsetfillcolor{currentfill}%
\pgfsetlinewidth{0.501875pt}%
\definecolor{currentstroke}{rgb}{0.000000,0.000000,0.000000}%
\pgfsetstrokecolor{currentstroke}%
\pgfsetdash{}{0pt}%
\pgfsys@defobject{currentmarker}{\pgfqpoint{0.000000in}{0.000000in}}{\pgfqpoint{0.000000in}{0.041667in}}{%
\pgfpathmoveto{\pgfqpoint{0.000000in}{0.000000in}}%
\pgfpathlineto{\pgfqpoint{0.000000in}{0.041667in}}%
\pgfusepath{stroke,fill}%
}%
\begin{pgfscope}%
\pgfsys@transformshift{2.831472in}{1.080890in}%
\pgfsys@useobject{currentmarker}{}%
\end{pgfscope}%
\end{pgfscope}%
\begin{pgfscope}%
\pgfsetbuttcap%
\pgfsetroundjoin%
\definecolor{currentfill}{rgb}{0.000000,0.000000,0.000000}%
\pgfsetfillcolor{currentfill}%
\pgfsetlinewidth{0.501875pt}%
\definecolor{currentstroke}{rgb}{0.000000,0.000000,0.000000}%
\pgfsetstrokecolor{currentstroke}%
\pgfsetdash{}{0pt}%
\pgfsys@defobject{currentmarker}{\pgfqpoint{0.000000in}{-0.041667in}}{\pgfqpoint{0.000000in}{0.000000in}}{%
\pgfpathmoveto{\pgfqpoint{0.000000in}{0.000000in}}%
\pgfpathlineto{\pgfqpoint{0.000000in}{-0.041667in}}%
\pgfusepath{stroke,fill}%
}%
\begin{pgfscope}%
\pgfsys@transformshift{2.831472in}{3.227753in}%
\pgfsys@useobject{currentmarker}{}%
\end{pgfscope}%
\end{pgfscope}%
\begin{pgfscope}%
\pgfpathrectangle{\pgfqpoint{0.481681in}{1.080890in}}{\pgfqpoint{5.785672in}{2.146863in}}%
\pgfusepath{clip}%
\pgfsetrectcap%
\pgfsetroundjoin%
\pgfsetlinewidth{0.100375pt}%
\definecolor{currentstroke}{rgb}{0.501961,0.501961,0.501961}%
\pgfsetstrokecolor{currentstroke}%
\pgfsetdash{}{0pt}%
\pgfpathmoveto{\pgfqpoint{3.259839in}{1.080890in}}%
\pgfpathlineto{\pgfqpoint{3.259839in}{3.227753in}}%
\pgfusepath{stroke}%
\end{pgfscope}%
\begin{pgfscope}%
\pgfsetbuttcap%
\pgfsetroundjoin%
\definecolor{currentfill}{rgb}{0.000000,0.000000,0.000000}%
\pgfsetfillcolor{currentfill}%
\pgfsetlinewidth{0.501875pt}%
\definecolor{currentstroke}{rgb}{0.000000,0.000000,0.000000}%
\pgfsetstrokecolor{currentstroke}%
\pgfsetdash{}{0pt}%
\pgfsys@defobject{currentmarker}{\pgfqpoint{0.000000in}{0.000000in}}{\pgfqpoint{0.000000in}{0.041667in}}{%
\pgfpathmoveto{\pgfqpoint{0.000000in}{0.000000in}}%
\pgfpathlineto{\pgfqpoint{0.000000in}{0.041667in}}%
\pgfusepath{stroke,fill}%
}%
\begin{pgfscope}%
\pgfsys@transformshift{3.259839in}{1.080890in}%
\pgfsys@useobject{currentmarker}{}%
\end{pgfscope}%
\end{pgfscope}%
\begin{pgfscope}%
\pgfsetbuttcap%
\pgfsetroundjoin%
\definecolor{currentfill}{rgb}{0.000000,0.000000,0.000000}%
\pgfsetfillcolor{currentfill}%
\pgfsetlinewidth{0.501875pt}%
\definecolor{currentstroke}{rgb}{0.000000,0.000000,0.000000}%
\pgfsetstrokecolor{currentstroke}%
\pgfsetdash{}{0pt}%
\pgfsys@defobject{currentmarker}{\pgfqpoint{0.000000in}{-0.041667in}}{\pgfqpoint{0.000000in}{0.000000in}}{%
\pgfpathmoveto{\pgfqpoint{0.000000in}{0.000000in}}%
\pgfpathlineto{\pgfqpoint{0.000000in}{-0.041667in}}%
\pgfusepath{stroke,fill}%
}%
\begin{pgfscope}%
\pgfsys@transformshift{3.259839in}{3.227753in}%
\pgfsys@useobject{currentmarker}{}%
\end{pgfscope}%
\end{pgfscope}%
\begin{pgfscope}%
\pgfpathrectangle{\pgfqpoint{0.481681in}{1.080890in}}{\pgfqpoint{5.785672in}{2.146863in}}%
\pgfusepath{clip}%
\pgfsetrectcap%
\pgfsetroundjoin%
\pgfsetlinewidth{0.100375pt}%
\definecolor{currentstroke}{rgb}{0.501961,0.501961,0.501961}%
\pgfsetstrokecolor{currentstroke}%
\pgfsetdash{}{0pt}%
\pgfpathmoveto{\pgfqpoint{3.688207in}{1.080890in}}%
\pgfpathlineto{\pgfqpoint{3.688207in}{3.227753in}}%
\pgfusepath{stroke}%
\end{pgfscope}%
\begin{pgfscope}%
\pgfsetbuttcap%
\pgfsetroundjoin%
\definecolor{currentfill}{rgb}{0.000000,0.000000,0.000000}%
\pgfsetfillcolor{currentfill}%
\pgfsetlinewidth{0.501875pt}%
\definecolor{currentstroke}{rgb}{0.000000,0.000000,0.000000}%
\pgfsetstrokecolor{currentstroke}%
\pgfsetdash{}{0pt}%
\pgfsys@defobject{currentmarker}{\pgfqpoint{0.000000in}{0.000000in}}{\pgfqpoint{0.000000in}{0.041667in}}{%
\pgfpathmoveto{\pgfqpoint{0.000000in}{0.000000in}}%
\pgfpathlineto{\pgfqpoint{0.000000in}{0.041667in}}%
\pgfusepath{stroke,fill}%
}%
\begin{pgfscope}%
\pgfsys@transformshift{3.688207in}{1.080890in}%
\pgfsys@useobject{currentmarker}{}%
\end{pgfscope}%
\end{pgfscope}%
\begin{pgfscope}%
\pgfsetbuttcap%
\pgfsetroundjoin%
\definecolor{currentfill}{rgb}{0.000000,0.000000,0.000000}%
\pgfsetfillcolor{currentfill}%
\pgfsetlinewidth{0.501875pt}%
\definecolor{currentstroke}{rgb}{0.000000,0.000000,0.000000}%
\pgfsetstrokecolor{currentstroke}%
\pgfsetdash{}{0pt}%
\pgfsys@defobject{currentmarker}{\pgfqpoint{0.000000in}{-0.041667in}}{\pgfqpoint{0.000000in}{0.000000in}}{%
\pgfpathmoveto{\pgfqpoint{0.000000in}{0.000000in}}%
\pgfpathlineto{\pgfqpoint{0.000000in}{-0.041667in}}%
\pgfusepath{stroke,fill}%
}%
\begin{pgfscope}%
\pgfsys@transformshift{3.688207in}{3.227753in}%
\pgfsys@useobject{currentmarker}{}%
\end{pgfscope}%
\end{pgfscope}%
\begin{pgfscope}%
\pgfpathrectangle{\pgfqpoint{0.481681in}{1.080890in}}{\pgfqpoint{5.785672in}{2.146863in}}%
\pgfusepath{clip}%
\pgfsetrectcap%
\pgfsetroundjoin%
\pgfsetlinewidth{0.100375pt}%
\definecolor{currentstroke}{rgb}{0.501961,0.501961,0.501961}%
\pgfsetstrokecolor{currentstroke}%
\pgfsetdash{}{0pt}%
\pgfpathmoveto{\pgfqpoint{4.116575in}{1.080890in}}%
\pgfpathlineto{\pgfqpoint{4.116575in}{3.227753in}}%
\pgfusepath{stroke}%
\end{pgfscope}%
\begin{pgfscope}%
\pgfsetbuttcap%
\pgfsetroundjoin%
\definecolor{currentfill}{rgb}{0.000000,0.000000,0.000000}%
\pgfsetfillcolor{currentfill}%
\pgfsetlinewidth{0.501875pt}%
\definecolor{currentstroke}{rgb}{0.000000,0.000000,0.000000}%
\pgfsetstrokecolor{currentstroke}%
\pgfsetdash{}{0pt}%
\pgfsys@defobject{currentmarker}{\pgfqpoint{0.000000in}{0.000000in}}{\pgfqpoint{0.000000in}{0.041667in}}{%
\pgfpathmoveto{\pgfqpoint{0.000000in}{0.000000in}}%
\pgfpathlineto{\pgfqpoint{0.000000in}{0.041667in}}%
\pgfusepath{stroke,fill}%
}%
\begin{pgfscope}%
\pgfsys@transformshift{4.116575in}{1.080890in}%
\pgfsys@useobject{currentmarker}{}%
\end{pgfscope}%
\end{pgfscope}%
\begin{pgfscope}%
\pgfsetbuttcap%
\pgfsetroundjoin%
\definecolor{currentfill}{rgb}{0.000000,0.000000,0.000000}%
\pgfsetfillcolor{currentfill}%
\pgfsetlinewidth{0.501875pt}%
\definecolor{currentstroke}{rgb}{0.000000,0.000000,0.000000}%
\pgfsetstrokecolor{currentstroke}%
\pgfsetdash{}{0pt}%
\pgfsys@defobject{currentmarker}{\pgfqpoint{0.000000in}{-0.041667in}}{\pgfqpoint{0.000000in}{0.000000in}}{%
\pgfpathmoveto{\pgfqpoint{0.000000in}{0.000000in}}%
\pgfpathlineto{\pgfqpoint{0.000000in}{-0.041667in}}%
\pgfusepath{stroke,fill}%
}%
\begin{pgfscope}%
\pgfsys@transformshift{4.116575in}{3.227753in}%
\pgfsys@useobject{currentmarker}{}%
\end{pgfscope}%
\end{pgfscope}%
\begin{pgfscope}%
\pgfpathrectangle{\pgfqpoint{0.481681in}{1.080890in}}{\pgfqpoint{5.785672in}{2.146863in}}%
\pgfusepath{clip}%
\pgfsetrectcap%
\pgfsetroundjoin%
\pgfsetlinewidth{0.100375pt}%
\definecolor{currentstroke}{rgb}{0.501961,0.501961,0.501961}%
\pgfsetstrokecolor{currentstroke}%
\pgfsetdash{}{0pt}%
\pgfpathmoveto{\pgfqpoint{4.544943in}{1.080890in}}%
\pgfpathlineto{\pgfqpoint{4.544943in}{3.227753in}}%
\pgfusepath{stroke}%
\end{pgfscope}%
\begin{pgfscope}%
\pgfsetbuttcap%
\pgfsetroundjoin%
\definecolor{currentfill}{rgb}{0.000000,0.000000,0.000000}%
\pgfsetfillcolor{currentfill}%
\pgfsetlinewidth{0.501875pt}%
\definecolor{currentstroke}{rgb}{0.000000,0.000000,0.000000}%
\pgfsetstrokecolor{currentstroke}%
\pgfsetdash{}{0pt}%
\pgfsys@defobject{currentmarker}{\pgfqpoint{0.000000in}{0.000000in}}{\pgfqpoint{0.000000in}{0.041667in}}{%
\pgfpathmoveto{\pgfqpoint{0.000000in}{0.000000in}}%
\pgfpathlineto{\pgfqpoint{0.000000in}{0.041667in}}%
\pgfusepath{stroke,fill}%
}%
\begin{pgfscope}%
\pgfsys@transformshift{4.544943in}{1.080890in}%
\pgfsys@useobject{currentmarker}{}%
\end{pgfscope}%
\end{pgfscope}%
\begin{pgfscope}%
\pgfsetbuttcap%
\pgfsetroundjoin%
\definecolor{currentfill}{rgb}{0.000000,0.000000,0.000000}%
\pgfsetfillcolor{currentfill}%
\pgfsetlinewidth{0.501875pt}%
\definecolor{currentstroke}{rgb}{0.000000,0.000000,0.000000}%
\pgfsetstrokecolor{currentstroke}%
\pgfsetdash{}{0pt}%
\pgfsys@defobject{currentmarker}{\pgfqpoint{0.000000in}{-0.041667in}}{\pgfqpoint{0.000000in}{0.000000in}}{%
\pgfpathmoveto{\pgfqpoint{0.000000in}{0.000000in}}%
\pgfpathlineto{\pgfqpoint{0.000000in}{-0.041667in}}%
\pgfusepath{stroke,fill}%
}%
\begin{pgfscope}%
\pgfsys@transformshift{4.544943in}{3.227753in}%
\pgfsys@useobject{currentmarker}{}%
\end{pgfscope}%
\end{pgfscope}%
\begin{pgfscope}%
\pgfpathrectangle{\pgfqpoint{0.481681in}{1.080890in}}{\pgfqpoint{5.785672in}{2.146863in}}%
\pgfusepath{clip}%
\pgfsetrectcap%
\pgfsetroundjoin%
\pgfsetlinewidth{0.100375pt}%
\definecolor{currentstroke}{rgb}{0.501961,0.501961,0.501961}%
\pgfsetstrokecolor{currentstroke}%
\pgfsetdash{}{0pt}%
\pgfpathmoveto{\pgfqpoint{4.973310in}{1.080890in}}%
\pgfpathlineto{\pgfqpoint{4.973310in}{3.227753in}}%
\pgfusepath{stroke}%
\end{pgfscope}%
\begin{pgfscope}%
\pgfsetbuttcap%
\pgfsetroundjoin%
\definecolor{currentfill}{rgb}{0.000000,0.000000,0.000000}%
\pgfsetfillcolor{currentfill}%
\pgfsetlinewidth{0.501875pt}%
\definecolor{currentstroke}{rgb}{0.000000,0.000000,0.000000}%
\pgfsetstrokecolor{currentstroke}%
\pgfsetdash{}{0pt}%
\pgfsys@defobject{currentmarker}{\pgfqpoint{0.000000in}{0.000000in}}{\pgfqpoint{0.000000in}{0.041667in}}{%
\pgfpathmoveto{\pgfqpoint{0.000000in}{0.000000in}}%
\pgfpathlineto{\pgfqpoint{0.000000in}{0.041667in}}%
\pgfusepath{stroke,fill}%
}%
\begin{pgfscope}%
\pgfsys@transformshift{4.973310in}{1.080890in}%
\pgfsys@useobject{currentmarker}{}%
\end{pgfscope}%
\end{pgfscope}%
\begin{pgfscope}%
\pgfsetbuttcap%
\pgfsetroundjoin%
\definecolor{currentfill}{rgb}{0.000000,0.000000,0.000000}%
\pgfsetfillcolor{currentfill}%
\pgfsetlinewidth{0.501875pt}%
\definecolor{currentstroke}{rgb}{0.000000,0.000000,0.000000}%
\pgfsetstrokecolor{currentstroke}%
\pgfsetdash{}{0pt}%
\pgfsys@defobject{currentmarker}{\pgfqpoint{0.000000in}{-0.041667in}}{\pgfqpoint{0.000000in}{0.000000in}}{%
\pgfpathmoveto{\pgfqpoint{0.000000in}{0.000000in}}%
\pgfpathlineto{\pgfqpoint{0.000000in}{-0.041667in}}%
\pgfusepath{stroke,fill}%
}%
\begin{pgfscope}%
\pgfsys@transformshift{4.973310in}{3.227753in}%
\pgfsys@useobject{currentmarker}{}%
\end{pgfscope}%
\end{pgfscope}%
\begin{pgfscope}%
\pgfpathrectangle{\pgfqpoint{0.481681in}{1.080890in}}{\pgfqpoint{5.785672in}{2.146863in}}%
\pgfusepath{clip}%
\pgfsetrectcap%
\pgfsetroundjoin%
\pgfsetlinewidth{0.100375pt}%
\definecolor{currentstroke}{rgb}{0.501961,0.501961,0.501961}%
\pgfsetstrokecolor{currentstroke}%
\pgfsetdash{}{0pt}%
\pgfpathmoveto{\pgfqpoint{5.401678in}{1.080890in}}%
\pgfpathlineto{\pgfqpoint{5.401678in}{3.227753in}}%
\pgfusepath{stroke}%
\end{pgfscope}%
\begin{pgfscope}%
\pgfsetbuttcap%
\pgfsetroundjoin%
\definecolor{currentfill}{rgb}{0.000000,0.000000,0.000000}%
\pgfsetfillcolor{currentfill}%
\pgfsetlinewidth{0.501875pt}%
\definecolor{currentstroke}{rgb}{0.000000,0.000000,0.000000}%
\pgfsetstrokecolor{currentstroke}%
\pgfsetdash{}{0pt}%
\pgfsys@defobject{currentmarker}{\pgfqpoint{0.000000in}{0.000000in}}{\pgfqpoint{0.000000in}{0.041667in}}{%
\pgfpathmoveto{\pgfqpoint{0.000000in}{0.000000in}}%
\pgfpathlineto{\pgfqpoint{0.000000in}{0.041667in}}%
\pgfusepath{stroke,fill}%
}%
\begin{pgfscope}%
\pgfsys@transformshift{5.401678in}{1.080890in}%
\pgfsys@useobject{currentmarker}{}%
\end{pgfscope}%
\end{pgfscope}%
\begin{pgfscope}%
\pgfsetbuttcap%
\pgfsetroundjoin%
\definecolor{currentfill}{rgb}{0.000000,0.000000,0.000000}%
\pgfsetfillcolor{currentfill}%
\pgfsetlinewidth{0.501875pt}%
\definecolor{currentstroke}{rgb}{0.000000,0.000000,0.000000}%
\pgfsetstrokecolor{currentstroke}%
\pgfsetdash{}{0pt}%
\pgfsys@defobject{currentmarker}{\pgfqpoint{0.000000in}{-0.041667in}}{\pgfqpoint{0.000000in}{0.000000in}}{%
\pgfpathmoveto{\pgfqpoint{0.000000in}{0.000000in}}%
\pgfpathlineto{\pgfqpoint{0.000000in}{-0.041667in}}%
\pgfusepath{stroke,fill}%
}%
\begin{pgfscope}%
\pgfsys@transformshift{5.401678in}{3.227753in}%
\pgfsys@useobject{currentmarker}{}%
\end{pgfscope}%
\end{pgfscope}%
\begin{pgfscope}%
\pgfpathrectangle{\pgfqpoint{0.481681in}{1.080890in}}{\pgfqpoint{5.785672in}{2.146863in}}%
\pgfusepath{clip}%
\pgfsetrectcap%
\pgfsetroundjoin%
\pgfsetlinewidth{0.100375pt}%
\definecolor{currentstroke}{rgb}{0.501961,0.501961,0.501961}%
\pgfsetstrokecolor{currentstroke}%
\pgfsetdash{}{0pt}%
\pgfpathmoveto{\pgfqpoint{5.830046in}{1.080890in}}%
\pgfpathlineto{\pgfqpoint{5.830046in}{3.227753in}}%
\pgfusepath{stroke}%
\end{pgfscope}%
\begin{pgfscope}%
\pgfsetbuttcap%
\pgfsetroundjoin%
\definecolor{currentfill}{rgb}{0.000000,0.000000,0.000000}%
\pgfsetfillcolor{currentfill}%
\pgfsetlinewidth{0.501875pt}%
\definecolor{currentstroke}{rgb}{0.000000,0.000000,0.000000}%
\pgfsetstrokecolor{currentstroke}%
\pgfsetdash{}{0pt}%
\pgfsys@defobject{currentmarker}{\pgfqpoint{0.000000in}{0.000000in}}{\pgfqpoint{0.000000in}{0.041667in}}{%
\pgfpathmoveto{\pgfqpoint{0.000000in}{0.000000in}}%
\pgfpathlineto{\pgfqpoint{0.000000in}{0.041667in}}%
\pgfusepath{stroke,fill}%
}%
\begin{pgfscope}%
\pgfsys@transformshift{5.830046in}{1.080890in}%
\pgfsys@useobject{currentmarker}{}%
\end{pgfscope}%
\end{pgfscope}%
\begin{pgfscope}%
\pgfsetbuttcap%
\pgfsetroundjoin%
\definecolor{currentfill}{rgb}{0.000000,0.000000,0.000000}%
\pgfsetfillcolor{currentfill}%
\pgfsetlinewidth{0.501875pt}%
\definecolor{currentstroke}{rgb}{0.000000,0.000000,0.000000}%
\pgfsetstrokecolor{currentstroke}%
\pgfsetdash{}{0pt}%
\pgfsys@defobject{currentmarker}{\pgfqpoint{0.000000in}{-0.041667in}}{\pgfqpoint{0.000000in}{0.000000in}}{%
\pgfpathmoveto{\pgfqpoint{0.000000in}{0.000000in}}%
\pgfpathlineto{\pgfqpoint{0.000000in}{-0.041667in}}%
\pgfusepath{stroke,fill}%
}%
\begin{pgfscope}%
\pgfsys@transformshift{5.830046in}{3.227753in}%
\pgfsys@useobject{currentmarker}{}%
\end{pgfscope}%
\end{pgfscope}%
\begin{pgfscope}%
\pgfpathrectangle{\pgfqpoint{0.481681in}{1.080890in}}{\pgfqpoint{5.785672in}{2.146863in}}%
\pgfusepath{clip}%
\pgfsetrectcap%
\pgfsetroundjoin%
\pgfsetlinewidth{0.100375pt}%
\definecolor{currentstroke}{rgb}{0.501961,0.501961,0.501961}%
\pgfsetstrokecolor{currentstroke}%
\pgfsetdash{}{0pt}%
\pgfpathmoveto{\pgfqpoint{6.258414in}{1.080890in}}%
\pgfpathlineto{\pgfqpoint{6.258414in}{3.227753in}}%
\pgfusepath{stroke}%
\end{pgfscope}%
\begin{pgfscope}%
\pgfsetbuttcap%
\pgfsetroundjoin%
\definecolor{currentfill}{rgb}{0.000000,0.000000,0.000000}%
\pgfsetfillcolor{currentfill}%
\pgfsetlinewidth{0.501875pt}%
\definecolor{currentstroke}{rgb}{0.000000,0.000000,0.000000}%
\pgfsetstrokecolor{currentstroke}%
\pgfsetdash{}{0pt}%
\pgfsys@defobject{currentmarker}{\pgfqpoint{0.000000in}{0.000000in}}{\pgfqpoint{0.000000in}{0.041667in}}{%
\pgfpathmoveto{\pgfqpoint{0.000000in}{0.000000in}}%
\pgfpathlineto{\pgfqpoint{0.000000in}{0.041667in}}%
\pgfusepath{stroke,fill}%
}%
\begin{pgfscope}%
\pgfsys@transformshift{6.258414in}{1.080890in}%
\pgfsys@useobject{currentmarker}{}%
\end{pgfscope}%
\end{pgfscope}%
\begin{pgfscope}%
\pgfsetbuttcap%
\pgfsetroundjoin%
\definecolor{currentfill}{rgb}{0.000000,0.000000,0.000000}%
\pgfsetfillcolor{currentfill}%
\pgfsetlinewidth{0.501875pt}%
\definecolor{currentstroke}{rgb}{0.000000,0.000000,0.000000}%
\pgfsetstrokecolor{currentstroke}%
\pgfsetdash{}{0pt}%
\pgfsys@defobject{currentmarker}{\pgfqpoint{0.000000in}{-0.041667in}}{\pgfqpoint{0.000000in}{0.000000in}}{%
\pgfpathmoveto{\pgfqpoint{0.000000in}{0.000000in}}%
\pgfpathlineto{\pgfqpoint{0.000000in}{-0.041667in}}%
\pgfusepath{stroke,fill}%
}%
\begin{pgfscope}%
\pgfsys@transformshift{6.258414in}{3.227753in}%
\pgfsys@useobject{currentmarker}{}%
\end{pgfscope}%
\end{pgfscope}%
\begin{pgfscope}%
\pgfpathrectangle{\pgfqpoint{0.481681in}{1.080890in}}{\pgfqpoint{5.785672in}{2.146863in}}%
\pgfusepath{clip}%
\pgfsetrectcap%
\pgfsetroundjoin%
\pgfsetlinewidth{0.100375pt}%
\definecolor{currentstroke}{rgb}{0.827451,0.827451,0.827451}%
\pgfsetstrokecolor{currentstroke}%
\pgfsetdash{}{0pt}%
\pgfpathmoveto{\pgfqpoint{0.511146in}{1.080890in}}%
\pgfpathlineto{\pgfqpoint{0.511146in}{3.227753in}}%
\pgfusepath{stroke}%
\end{pgfscope}%
\begin{pgfscope}%
\pgfsetbuttcap%
\pgfsetroundjoin%
\definecolor{currentfill}{rgb}{0.000000,0.000000,0.000000}%
\pgfsetfillcolor{currentfill}%
\pgfsetlinewidth{0.501875pt}%
\definecolor{currentstroke}{rgb}{0.000000,0.000000,0.000000}%
\pgfsetstrokecolor{currentstroke}%
\pgfsetdash{}{0pt}%
\pgfsys@defobject{currentmarker}{\pgfqpoint{0.000000in}{0.000000in}}{\pgfqpoint{0.000000in}{0.020833in}}{%
\pgfpathmoveto{\pgfqpoint{0.000000in}{0.000000in}}%
\pgfpathlineto{\pgfqpoint{0.000000in}{0.020833in}}%
\pgfusepath{stroke,fill}%
}%
\begin{pgfscope}%
\pgfsys@transformshift{0.511146in}{1.080890in}%
\pgfsys@useobject{currentmarker}{}%
\end{pgfscope}%
\end{pgfscope}%
\begin{pgfscope}%
\pgfsetbuttcap%
\pgfsetroundjoin%
\definecolor{currentfill}{rgb}{0.000000,0.000000,0.000000}%
\pgfsetfillcolor{currentfill}%
\pgfsetlinewidth{0.501875pt}%
\definecolor{currentstroke}{rgb}{0.000000,0.000000,0.000000}%
\pgfsetstrokecolor{currentstroke}%
\pgfsetdash{}{0pt}%
\pgfsys@defobject{currentmarker}{\pgfqpoint{0.000000in}{-0.020833in}}{\pgfqpoint{0.000000in}{0.000000in}}{%
\pgfpathmoveto{\pgfqpoint{0.000000in}{0.000000in}}%
\pgfpathlineto{\pgfqpoint{0.000000in}{-0.020833in}}%
\pgfusepath{stroke,fill}%
}%
\begin{pgfscope}%
\pgfsys@transformshift{0.511146in}{3.227753in}%
\pgfsys@useobject{currentmarker}{}%
\end{pgfscope}%
\end{pgfscope}%
\begin{pgfscope}%
\pgfpathrectangle{\pgfqpoint{0.481681in}{1.080890in}}{\pgfqpoint{5.785672in}{2.146863in}}%
\pgfusepath{clip}%
\pgfsetrectcap%
\pgfsetroundjoin%
\pgfsetlinewidth{0.100375pt}%
\definecolor{currentstroke}{rgb}{0.827451,0.827451,0.827451}%
\pgfsetstrokecolor{currentstroke}%
\pgfsetdash{}{0pt}%
\pgfpathmoveto{\pgfqpoint{0.546843in}{1.080890in}}%
\pgfpathlineto{\pgfqpoint{0.546843in}{3.227753in}}%
\pgfusepath{stroke}%
\end{pgfscope}%
\begin{pgfscope}%
\pgfsetbuttcap%
\pgfsetroundjoin%
\definecolor{currentfill}{rgb}{0.000000,0.000000,0.000000}%
\pgfsetfillcolor{currentfill}%
\pgfsetlinewidth{0.501875pt}%
\definecolor{currentstroke}{rgb}{0.000000,0.000000,0.000000}%
\pgfsetstrokecolor{currentstroke}%
\pgfsetdash{}{0pt}%
\pgfsys@defobject{currentmarker}{\pgfqpoint{0.000000in}{0.000000in}}{\pgfqpoint{0.000000in}{0.020833in}}{%
\pgfpathmoveto{\pgfqpoint{0.000000in}{0.000000in}}%
\pgfpathlineto{\pgfqpoint{0.000000in}{0.020833in}}%
\pgfusepath{stroke,fill}%
}%
\begin{pgfscope}%
\pgfsys@transformshift{0.546843in}{1.080890in}%
\pgfsys@useobject{currentmarker}{}%
\end{pgfscope}%
\end{pgfscope}%
\begin{pgfscope}%
\pgfsetbuttcap%
\pgfsetroundjoin%
\definecolor{currentfill}{rgb}{0.000000,0.000000,0.000000}%
\pgfsetfillcolor{currentfill}%
\pgfsetlinewidth{0.501875pt}%
\definecolor{currentstroke}{rgb}{0.000000,0.000000,0.000000}%
\pgfsetstrokecolor{currentstroke}%
\pgfsetdash{}{0pt}%
\pgfsys@defobject{currentmarker}{\pgfqpoint{0.000000in}{-0.020833in}}{\pgfqpoint{0.000000in}{0.000000in}}{%
\pgfpathmoveto{\pgfqpoint{0.000000in}{0.000000in}}%
\pgfpathlineto{\pgfqpoint{0.000000in}{-0.020833in}}%
\pgfusepath{stroke,fill}%
}%
\begin{pgfscope}%
\pgfsys@transformshift{0.546843in}{3.227753in}%
\pgfsys@useobject{currentmarker}{}%
\end{pgfscope}%
\end{pgfscope}%
\begin{pgfscope}%
\pgfpathrectangle{\pgfqpoint{0.481681in}{1.080890in}}{\pgfqpoint{5.785672in}{2.146863in}}%
\pgfusepath{clip}%
\pgfsetrectcap%
\pgfsetroundjoin%
\pgfsetlinewidth{0.100375pt}%
\definecolor{currentstroke}{rgb}{0.827451,0.827451,0.827451}%
\pgfsetstrokecolor{currentstroke}%
\pgfsetdash{}{0pt}%
\pgfpathmoveto{\pgfqpoint{0.582541in}{1.080890in}}%
\pgfpathlineto{\pgfqpoint{0.582541in}{3.227753in}}%
\pgfusepath{stroke}%
\end{pgfscope}%
\begin{pgfscope}%
\pgfsetbuttcap%
\pgfsetroundjoin%
\definecolor{currentfill}{rgb}{0.000000,0.000000,0.000000}%
\pgfsetfillcolor{currentfill}%
\pgfsetlinewidth{0.501875pt}%
\definecolor{currentstroke}{rgb}{0.000000,0.000000,0.000000}%
\pgfsetstrokecolor{currentstroke}%
\pgfsetdash{}{0pt}%
\pgfsys@defobject{currentmarker}{\pgfqpoint{0.000000in}{0.000000in}}{\pgfqpoint{0.000000in}{0.020833in}}{%
\pgfpathmoveto{\pgfqpoint{0.000000in}{0.000000in}}%
\pgfpathlineto{\pgfqpoint{0.000000in}{0.020833in}}%
\pgfusepath{stroke,fill}%
}%
\begin{pgfscope}%
\pgfsys@transformshift{0.582541in}{1.080890in}%
\pgfsys@useobject{currentmarker}{}%
\end{pgfscope}%
\end{pgfscope}%
\begin{pgfscope}%
\pgfsetbuttcap%
\pgfsetroundjoin%
\definecolor{currentfill}{rgb}{0.000000,0.000000,0.000000}%
\pgfsetfillcolor{currentfill}%
\pgfsetlinewidth{0.501875pt}%
\definecolor{currentstroke}{rgb}{0.000000,0.000000,0.000000}%
\pgfsetstrokecolor{currentstroke}%
\pgfsetdash{}{0pt}%
\pgfsys@defobject{currentmarker}{\pgfqpoint{0.000000in}{-0.020833in}}{\pgfqpoint{0.000000in}{0.000000in}}{%
\pgfpathmoveto{\pgfqpoint{0.000000in}{0.000000in}}%
\pgfpathlineto{\pgfqpoint{0.000000in}{-0.020833in}}%
\pgfusepath{stroke,fill}%
}%
\begin{pgfscope}%
\pgfsys@transformshift{0.582541in}{3.227753in}%
\pgfsys@useobject{currentmarker}{}%
\end{pgfscope}%
\end{pgfscope}%
\begin{pgfscope}%
\pgfpathrectangle{\pgfqpoint{0.481681in}{1.080890in}}{\pgfqpoint{5.785672in}{2.146863in}}%
\pgfusepath{clip}%
\pgfsetrectcap%
\pgfsetroundjoin%
\pgfsetlinewidth{0.100375pt}%
\definecolor{currentstroke}{rgb}{0.827451,0.827451,0.827451}%
\pgfsetstrokecolor{currentstroke}%
\pgfsetdash{}{0pt}%
\pgfpathmoveto{\pgfqpoint{0.618238in}{1.080890in}}%
\pgfpathlineto{\pgfqpoint{0.618238in}{3.227753in}}%
\pgfusepath{stroke}%
\end{pgfscope}%
\begin{pgfscope}%
\pgfsetbuttcap%
\pgfsetroundjoin%
\definecolor{currentfill}{rgb}{0.000000,0.000000,0.000000}%
\pgfsetfillcolor{currentfill}%
\pgfsetlinewidth{0.501875pt}%
\definecolor{currentstroke}{rgb}{0.000000,0.000000,0.000000}%
\pgfsetstrokecolor{currentstroke}%
\pgfsetdash{}{0pt}%
\pgfsys@defobject{currentmarker}{\pgfqpoint{0.000000in}{0.000000in}}{\pgfqpoint{0.000000in}{0.020833in}}{%
\pgfpathmoveto{\pgfqpoint{0.000000in}{0.000000in}}%
\pgfpathlineto{\pgfqpoint{0.000000in}{0.020833in}}%
\pgfusepath{stroke,fill}%
}%
\begin{pgfscope}%
\pgfsys@transformshift{0.618238in}{1.080890in}%
\pgfsys@useobject{currentmarker}{}%
\end{pgfscope}%
\end{pgfscope}%
\begin{pgfscope}%
\pgfsetbuttcap%
\pgfsetroundjoin%
\definecolor{currentfill}{rgb}{0.000000,0.000000,0.000000}%
\pgfsetfillcolor{currentfill}%
\pgfsetlinewidth{0.501875pt}%
\definecolor{currentstroke}{rgb}{0.000000,0.000000,0.000000}%
\pgfsetstrokecolor{currentstroke}%
\pgfsetdash{}{0pt}%
\pgfsys@defobject{currentmarker}{\pgfqpoint{0.000000in}{-0.020833in}}{\pgfqpoint{0.000000in}{0.000000in}}{%
\pgfpathmoveto{\pgfqpoint{0.000000in}{0.000000in}}%
\pgfpathlineto{\pgfqpoint{0.000000in}{-0.020833in}}%
\pgfusepath{stroke,fill}%
}%
\begin{pgfscope}%
\pgfsys@transformshift{0.618238in}{3.227753in}%
\pgfsys@useobject{currentmarker}{}%
\end{pgfscope}%
\end{pgfscope}%
\begin{pgfscope}%
\pgfpathrectangle{\pgfqpoint{0.481681in}{1.080890in}}{\pgfqpoint{5.785672in}{2.146863in}}%
\pgfusepath{clip}%
\pgfsetrectcap%
\pgfsetroundjoin%
\pgfsetlinewidth{0.100375pt}%
\definecolor{currentstroke}{rgb}{0.827451,0.827451,0.827451}%
\pgfsetstrokecolor{currentstroke}%
\pgfsetdash{}{0pt}%
\pgfpathmoveto{\pgfqpoint{0.653935in}{1.080890in}}%
\pgfpathlineto{\pgfqpoint{0.653935in}{3.227753in}}%
\pgfusepath{stroke}%
\end{pgfscope}%
\begin{pgfscope}%
\pgfsetbuttcap%
\pgfsetroundjoin%
\definecolor{currentfill}{rgb}{0.000000,0.000000,0.000000}%
\pgfsetfillcolor{currentfill}%
\pgfsetlinewidth{0.501875pt}%
\definecolor{currentstroke}{rgb}{0.000000,0.000000,0.000000}%
\pgfsetstrokecolor{currentstroke}%
\pgfsetdash{}{0pt}%
\pgfsys@defobject{currentmarker}{\pgfqpoint{0.000000in}{0.000000in}}{\pgfqpoint{0.000000in}{0.020833in}}{%
\pgfpathmoveto{\pgfqpoint{0.000000in}{0.000000in}}%
\pgfpathlineto{\pgfqpoint{0.000000in}{0.020833in}}%
\pgfusepath{stroke,fill}%
}%
\begin{pgfscope}%
\pgfsys@transformshift{0.653935in}{1.080890in}%
\pgfsys@useobject{currentmarker}{}%
\end{pgfscope}%
\end{pgfscope}%
\begin{pgfscope}%
\pgfsetbuttcap%
\pgfsetroundjoin%
\definecolor{currentfill}{rgb}{0.000000,0.000000,0.000000}%
\pgfsetfillcolor{currentfill}%
\pgfsetlinewidth{0.501875pt}%
\definecolor{currentstroke}{rgb}{0.000000,0.000000,0.000000}%
\pgfsetstrokecolor{currentstroke}%
\pgfsetdash{}{0pt}%
\pgfsys@defobject{currentmarker}{\pgfqpoint{0.000000in}{-0.020833in}}{\pgfqpoint{0.000000in}{0.000000in}}{%
\pgfpathmoveto{\pgfqpoint{0.000000in}{0.000000in}}%
\pgfpathlineto{\pgfqpoint{0.000000in}{-0.020833in}}%
\pgfusepath{stroke,fill}%
}%
\begin{pgfscope}%
\pgfsys@transformshift{0.653935in}{3.227753in}%
\pgfsys@useobject{currentmarker}{}%
\end{pgfscope}%
\end{pgfscope}%
\begin{pgfscope}%
\pgfpathrectangle{\pgfqpoint{0.481681in}{1.080890in}}{\pgfqpoint{5.785672in}{2.146863in}}%
\pgfusepath{clip}%
\pgfsetrectcap%
\pgfsetroundjoin%
\pgfsetlinewidth{0.100375pt}%
\definecolor{currentstroke}{rgb}{0.827451,0.827451,0.827451}%
\pgfsetstrokecolor{currentstroke}%
\pgfsetdash{}{0pt}%
\pgfpathmoveto{\pgfqpoint{0.725330in}{1.080890in}}%
\pgfpathlineto{\pgfqpoint{0.725330in}{3.227753in}}%
\pgfusepath{stroke}%
\end{pgfscope}%
\begin{pgfscope}%
\pgfsetbuttcap%
\pgfsetroundjoin%
\definecolor{currentfill}{rgb}{0.000000,0.000000,0.000000}%
\pgfsetfillcolor{currentfill}%
\pgfsetlinewidth{0.501875pt}%
\definecolor{currentstroke}{rgb}{0.000000,0.000000,0.000000}%
\pgfsetstrokecolor{currentstroke}%
\pgfsetdash{}{0pt}%
\pgfsys@defobject{currentmarker}{\pgfqpoint{0.000000in}{0.000000in}}{\pgfqpoint{0.000000in}{0.020833in}}{%
\pgfpathmoveto{\pgfqpoint{0.000000in}{0.000000in}}%
\pgfpathlineto{\pgfqpoint{0.000000in}{0.020833in}}%
\pgfusepath{stroke,fill}%
}%
\begin{pgfscope}%
\pgfsys@transformshift{0.725330in}{1.080890in}%
\pgfsys@useobject{currentmarker}{}%
\end{pgfscope}%
\end{pgfscope}%
\begin{pgfscope}%
\pgfsetbuttcap%
\pgfsetroundjoin%
\definecolor{currentfill}{rgb}{0.000000,0.000000,0.000000}%
\pgfsetfillcolor{currentfill}%
\pgfsetlinewidth{0.501875pt}%
\definecolor{currentstroke}{rgb}{0.000000,0.000000,0.000000}%
\pgfsetstrokecolor{currentstroke}%
\pgfsetdash{}{0pt}%
\pgfsys@defobject{currentmarker}{\pgfqpoint{0.000000in}{-0.020833in}}{\pgfqpoint{0.000000in}{0.000000in}}{%
\pgfpathmoveto{\pgfqpoint{0.000000in}{0.000000in}}%
\pgfpathlineto{\pgfqpoint{0.000000in}{-0.020833in}}%
\pgfusepath{stroke,fill}%
}%
\begin{pgfscope}%
\pgfsys@transformshift{0.725330in}{3.227753in}%
\pgfsys@useobject{currentmarker}{}%
\end{pgfscope}%
\end{pgfscope}%
\begin{pgfscope}%
\pgfpathrectangle{\pgfqpoint{0.481681in}{1.080890in}}{\pgfqpoint{5.785672in}{2.146863in}}%
\pgfusepath{clip}%
\pgfsetrectcap%
\pgfsetroundjoin%
\pgfsetlinewidth{0.100375pt}%
\definecolor{currentstroke}{rgb}{0.827451,0.827451,0.827451}%
\pgfsetstrokecolor{currentstroke}%
\pgfsetdash{}{0pt}%
\pgfpathmoveto{\pgfqpoint{0.761027in}{1.080890in}}%
\pgfpathlineto{\pgfqpoint{0.761027in}{3.227753in}}%
\pgfusepath{stroke}%
\end{pgfscope}%
\begin{pgfscope}%
\pgfsetbuttcap%
\pgfsetroundjoin%
\definecolor{currentfill}{rgb}{0.000000,0.000000,0.000000}%
\pgfsetfillcolor{currentfill}%
\pgfsetlinewidth{0.501875pt}%
\definecolor{currentstroke}{rgb}{0.000000,0.000000,0.000000}%
\pgfsetstrokecolor{currentstroke}%
\pgfsetdash{}{0pt}%
\pgfsys@defobject{currentmarker}{\pgfqpoint{0.000000in}{0.000000in}}{\pgfqpoint{0.000000in}{0.020833in}}{%
\pgfpathmoveto{\pgfqpoint{0.000000in}{0.000000in}}%
\pgfpathlineto{\pgfqpoint{0.000000in}{0.020833in}}%
\pgfusepath{stroke,fill}%
}%
\begin{pgfscope}%
\pgfsys@transformshift{0.761027in}{1.080890in}%
\pgfsys@useobject{currentmarker}{}%
\end{pgfscope}%
\end{pgfscope}%
\begin{pgfscope}%
\pgfsetbuttcap%
\pgfsetroundjoin%
\definecolor{currentfill}{rgb}{0.000000,0.000000,0.000000}%
\pgfsetfillcolor{currentfill}%
\pgfsetlinewidth{0.501875pt}%
\definecolor{currentstroke}{rgb}{0.000000,0.000000,0.000000}%
\pgfsetstrokecolor{currentstroke}%
\pgfsetdash{}{0pt}%
\pgfsys@defobject{currentmarker}{\pgfqpoint{0.000000in}{-0.020833in}}{\pgfqpoint{0.000000in}{0.000000in}}{%
\pgfpathmoveto{\pgfqpoint{0.000000in}{0.000000in}}%
\pgfpathlineto{\pgfqpoint{0.000000in}{-0.020833in}}%
\pgfusepath{stroke,fill}%
}%
\begin{pgfscope}%
\pgfsys@transformshift{0.761027in}{3.227753in}%
\pgfsys@useobject{currentmarker}{}%
\end{pgfscope}%
\end{pgfscope}%
\begin{pgfscope}%
\pgfpathrectangle{\pgfqpoint{0.481681in}{1.080890in}}{\pgfqpoint{5.785672in}{2.146863in}}%
\pgfusepath{clip}%
\pgfsetrectcap%
\pgfsetroundjoin%
\pgfsetlinewidth{0.100375pt}%
\definecolor{currentstroke}{rgb}{0.827451,0.827451,0.827451}%
\pgfsetstrokecolor{currentstroke}%
\pgfsetdash{}{0pt}%
\pgfpathmoveto{\pgfqpoint{0.796725in}{1.080890in}}%
\pgfpathlineto{\pgfqpoint{0.796725in}{3.227753in}}%
\pgfusepath{stroke}%
\end{pgfscope}%
\begin{pgfscope}%
\pgfsetbuttcap%
\pgfsetroundjoin%
\definecolor{currentfill}{rgb}{0.000000,0.000000,0.000000}%
\pgfsetfillcolor{currentfill}%
\pgfsetlinewidth{0.501875pt}%
\definecolor{currentstroke}{rgb}{0.000000,0.000000,0.000000}%
\pgfsetstrokecolor{currentstroke}%
\pgfsetdash{}{0pt}%
\pgfsys@defobject{currentmarker}{\pgfqpoint{0.000000in}{0.000000in}}{\pgfqpoint{0.000000in}{0.020833in}}{%
\pgfpathmoveto{\pgfqpoint{0.000000in}{0.000000in}}%
\pgfpathlineto{\pgfqpoint{0.000000in}{0.020833in}}%
\pgfusepath{stroke,fill}%
}%
\begin{pgfscope}%
\pgfsys@transformshift{0.796725in}{1.080890in}%
\pgfsys@useobject{currentmarker}{}%
\end{pgfscope}%
\end{pgfscope}%
\begin{pgfscope}%
\pgfsetbuttcap%
\pgfsetroundjoin%
\definecolor{currentfill}{rgb}{0.000000,0.000000,0.000000}%
\pgfsetfillcolor{currentfill}%
\pgfsetlinewidth{0.501875pt}%
\definecolor{currentstroke}{rgb}{0.000000,0.000000,0.000000}%
\pgfsetstrokecolor{currentstroke}%
\pgfsetdash{}{0pt}%
\pgfsys@defobject{currentmarker}{\pgfqpoint{0.000000in}{-0.020833in}}{\pgfqpoint{0.000000in}{0.000000in}}{%
\pgfpathmoveto{\pgfqpoint{0.000000in}{0.000000in}}%
\pgfpathlineto{\pgfqpoint{0.000000in}{-0.020833in}}%
\pgfusepath{stroke,fill}%
}%
\begin{pgfscope}%
\pgfsys@transformshift{0.796725in}{3.227753in}%
\pgfsys@useobject{currentmarker}{}%
\end{pgfscope}%
\end{pgfscope}%
\begin{pgfscope}%
\pgfpathrectangle{\pgfqpoint{0.481681in}{1.080890in}}{\pgfqpoint{5.785672in}{2.146863in}}%
\pgfusepath{clip}%
\pgfsetrectcap%
\pgfsetroundjoin%
\pgfsetlinewidth{0.100375pt}%
\definecolor{currentstroke}{rgb}{0.827451,0.827451,0.827451}%
\pgfsetstrokecolor{currentstroke}%
\pgfsetdash{}{0pt}%
\pgfpathmoveto{\pgfqpoint{0.832422in}{1.080890in}}%
\pgfpathlineto{\pgfqpoint{0.832422in}{3.227753in}}%
\pgfusepath{stroke}%
\end{pgfscope}%
\begin{pgfscope}%
\pgfsetbuttcap%
\pgfsetroundjoin%
\definecolor{currentfill}{rgb}{0.000000,0.000000,0.000000}%
\pgfsetfillcolor{currentfill}%
\pgfsetlinewidth{0.501875pt}%
\definecolor{currentstroke}{rgb}{0.000000,0.000000,0.000000}%
\pgfsetstrokecolor{currentstroke}%
\pgfsetdash{}{0pt}%
\pgfsys@defobject{currentmarker}{\pgfqpoint{0.000000in}{0.000000in}}{\pgfqpoint{0.000000in}{0.020833in}}{%
\pgfpathmoveto{\pgfqpoint{0.000000in}{0.000000in}}%
\pgfpathlineto{\pgfqpoint{0.000000in}{0.020833in}}%
\pgfusepath{stroke,fill}%
}%
\begin{pgfscope}%
\pgfsys@transformshift{0.832422in}{1.080890in}%
\pgfsys@useobject{currentmarker}{}%
\end{pgfscope}%
\end{pgfscope}%
\begin{pgfscope}%
\pgfsetbuttcap%
\pgfsetroundjoin%
\definecolor{currentfill}{rgb}{0.000000,0.000000,0.000000}%
\pgfsetfillcolor{currentfill}%
\pgfsetlinewidth{0.501875pt}%
\definecolor{currentstroke}{rgb}{0.000000,0.000000,0.000000}%
\pgfsetstrokecolor{currentstroke}%
\pgfsetdash{}{0pt}%
\pgfsys@defobject{currentmarker}{\pgfqpoint{0.000000in}{-0.020833in}}{\pgfqpoint{0.000000in}{0.000000in}}{%
\pgfpathmoveto{\pgfqpoint{0.000000in}{0.000000in}}%
\pgfpathlineto{\pgfqpoint{0.000000in}{-0.020833in}}%
\pgfusepath{stroke,fill}%
}%
\begin{pgfscope}%
\pgfsys@transformshift{0.832422in}{3.227753in}%
\pgfsys@useobject{currentmarker}{}%
\end{pgfscope}%
\end{pgfscope}%
\begin{pgfscope}%
\pgfpathrectangle{\pgfqpoint{0.481681in}{1.080890in}}{\pgfqpoint{5.785672in}{2.146863in}}%
\pgfusepath{clip}%
\pgfsetrectcap%
\pgfsetroundjoin%
\pgfsetlinewidth{0.100375pt}%
\definecolor{currentstroke}{rgb}{0.827451,0.827451,0.827451}%
\pgfsetstrokecolor{currentstroke}%
\pgfsetdash{}{0pt}%
\pgfpathmoveto{\pgfqpoint{0.868119in}{1.080890in}}%
\pgfpathlineto{\pgfqpoint{0.868119in}{3.227753in}}%
\pgfusepath{stroke}%
\end{pgfscope}%
\begin{pgfscope}%
\pgfsetbuttcap%
\pgfsetroundjoin%
\definecolor{currentfill}{rgb}{0.000000,0.000000,0.000000}%
\pgfsetfillcolor{currentfill}%
\pgfsetlinewidth{0.501875pt}%
\definecolor{currentstroke}{rgb}{0.000000,0.000000,0.000000}%
\pgfsetstrokecolor{currentstroke}%
\pgfsetdash{}{0pt}%
\pgfsys@defobject{currentmarker}{\pgfqpoint{0.000000in}{0.000000in}}{\pgfqpoint{0.000000in}{0.020833in}}{%
\pgfpathmoveto{\pgfqpoint{0.000000in}{0.000000in}}%
\pgfpathlineto{\pgfqpoint{0.000000in}{0.020833in}}%
\pgfusepath{stroke,fill}%
}%
\begin{pgfscope}%
\pgfsys@transformshift{0.868119in}{1.080890in}%
\pgfsys@useobject{currentmarker}{}%
\end{pgfscope}%
\end{pgfscope}%
\begin{pgfscope}%
\pgfsetbuttcap%
\pgfsetroundjoin%
\definecolor{currentfill}{rgb}{0.000000,0.000000,0.000000}%
\pgfsetfillcolor{currentfill}%
\pgfsetlinewidth{0.501875pt}%
\definecolor{currentstroke}{rgb}{0.000000,0.000000,0.000000}%
\pgfsetstrokecolor{currentstroke}%
\pgfsetdash{}{0pt}%
\pgfsys@defobject{currentmarker}{\pgfqpoint{0.000000in}{-0.020833in}}{\pgfqpoint{0.000000in}{0.000000in}}{%
\pgfpathmoveto{\pgfqpoint{0.000000in}{0.000000in}}%
\pgfpathlineto{\pgfqpoint{0.000000in}{-0.020833in}}%
\pgfusepath{stroke,fill}%
}%
\begin{pgfscope}%
\pgfsys@transformshift{0.868119in}{3.227753in}%
\pgfsys@useobject{currentmarker}{}%
\end{pgfscope}%
\end{pgfscope}%
\begin{pgfscope}%
\pgfpathrectangle{\pgfqpoint{0.481681in}{1.080890in}}{\pgfqpoint{5.785672in}{2.146863in}}%
\pgfusepath{clip}%
\pgfsetrectcap%
\pgfsetroundjoin%
\pgfsetlinewidth{0.100375pt}%
\definecolor{currentstroke}{rgb}{0.827451,0.827451,0.827451}%
\pgfsetstrokecolor{currentstroke}%
\pgfsetdash{}{0pt}%
\pgfpathmoveto{\pgfqpoint{0.903816in}{1.080890in}}%
\pgfpathlineto{\pgfqpoint{0.903816in}{3.227753in}}%
\pgfusepath{stroke}%
\end{pgfscope}%
\begin{pgfscope}%
\pgfsetbuttcap%
\pgfsetroundjoin%
\definecolor{currentfill}{rgb}{0.000000,0.000000,0.000000}%
\pgfsetfillcolor{currentfill}%
\pgfsetlinewidth{0.501875pt}%
\definecolor{currentstroke}{rgb}{0.000000,0.000000,0.000000}%
\pgfsetstrokecolor{currentstroke}%
\pgfsetdash{}{0pt}%
\pgfsys@defobject{currentmarker}{\pgfqpoint{0.000000in}{0.000000in}}{\pgfqpoint{0.000000in}{0.020833in}}{%
\pgfpathmoveto{\pgfqpoint{0.000000in}{0.000000in}}%
\pgfpathlineto{\pgfqpoint{0.000000in}{0.020833in}}%
\pgfusepath{stroke,fill}%
}%
\begin{pgfscope}%
\pgfsys@transformshift{0.903816in}{1.080890in}%
\pgfsys@useobject{currentmarker}{}%
\end{pgfscope}%
\end{pgfscope}%
\begin{pgfscope}%
\pgfsetbuttcap%
\pgfsetroundjoin%
\definecolor{currentfill}{rgb}{0.000000,0.000000,0.000000}%
\pgfsetfillcolor{currentfill}%
\pgfsetlinewidth{0.501875pt}%
\definecolor{currentstroke}{rgb}{0.000000,0.000000,0.000000}%
\pgfsetstrokecolor{currentstroke}%
\pgfsetdash{}{0pt}%
\pgfsys@defobject{currentmarker}{\pgfqpoint{0.000000in}{-0.020833in}}{\pgfqpoint{0.000000in}{0.000000in}}{%
\pgfpathmoveto{\pgfqpoint{0.000000in}{0.000000in}}%
\pgfpathlineto{\pgfqpoint{0.000000in}{-0.020833in}}%
\pgfusepath{stroke,fill}%
}%
\begin{pgfscope}%
\pgfsys@transformshift{0.903816in}{3.227753in}%
\pgfsys@useobject{currentmarker}{}%
\end{pgfscope}%
\end{pgfscope}%
\begin{pgfscope}%
\pgfpathrectangle{\pgfqpoint{0.481681in}{1.080890in}}{\pgfqpoint{5.785672in}{2.146863in}}%
\pgfusepath{clip}%
\pgfsetrectcap%
\pgfsetroundjoin%
\pgfsetlinewidth{0.100375pt}%
\definecolor{currentstroke}{rgb}{0.827451,0.827451,0.827451}%
\pgfsetstrokecolor{currentstroke}%
\pgfsetdash{}{0pt}%
\pgfpathmoveto{\pgfqpoint{0.939514in}{1.080890in}}%
\pgfpathlineto{\pgfqpoint{0.939514in}{3.227753in}}%
\pgfusepath{stroke}%
\end{pgfscope}%
\begin{pgfscope}%
\pgfsetbuttcap%
\pgfsetroundjoin%
\definecolor{currentfill}{rgb}{0.000000,0.000000,0.000000}%
\pgfsetfillcolor{currentfill}%
\pgfsetlinewidth{0.501875pt}%
\definecolor{currentstroke}{rgb}{0.000000,0.000000,0.000000}%
\pgfsetstrokecolor{currentstroke}%
\pgfsetdash{}{0pt}%
\pgfsys@defobject{currentmarker}{\pgfqpoint{0.000000in}{0.000000in}}{\pgfqpoint{0.000000in}{0.020833in}}{%
\pgfpathmoveto{\pgfqpoint{0.000000in}{0.000000in}}%
\pgfpathlineto{\pgfqpoint{0.000000in}{0.020833in}}%
\pgfusepath{stroke,fill}%
}%
\begin{pgfscope}%
\pgfsys@transformshift{0.939514in}{1.080890in}%
\pgfsys@useobject{currentmarker}{}%
\end{pgfscope}%
\end{pgfscope}%
\begin{pgfscope}%
\pgfsetbuttcap%
\pgfsetroundjoin%
\definecolor{currentfill}{rgb}{0.000000,0.000000,0.000000}%
\pgfsetfillcolor{currentfill}%
\pgfsetlinewidth{0.501875pt}%
\definecolor{currentstroke}{rgb}{0.000000,0.000000,0.000000}%
\pgfsetstrokecolor{currentstroke}%
\pgfsetdash{}{0pt}%
\pgfsys@defobject{currentmarker}{\pgfqpoint{0.000000in}{-0.020833in}}{\pgfqpoint{0.000000in}{0.000000in}}{%
\pgfpathmoveto{\pgfqpoint{0.000000in}{0.000000in}}%
\pgfpathlineto{\pgfqpoint{0.000000in}{-0.020833in}}%
\pgfusepath{stroke,fill}%
}%
\begin{pgfscope}%
\pgfsys@transformshift{0.939514in}{3.227753in}%
\pgfsys@useobject{currentmarker}{}%
\end{pgfscope}%
\end{pgfscope}%
\begin{pgfscope}%
\pgfpathrectangle{\pgfqpoint{0.481681in}{1.080890in}}{\pgfqpoint{5.785672in}{2.146863in}}%
\pgfusepath{clip}%
\pgfsetrectcap%
\pgfsetroundjoin%
\pgfsetlinewidth{0.100375pt}%
\definecolor{currentstroke}{rgb}{0.827451,0.827451,0.827451}%
\pgfsetstrokecolor{currentstroke}%
\pgfsetdash{}{0pt}%
\pgfpathmoveto{\pgfqpoint{0.975211in}{1.080890in}}%
\pgfpathlineto{\pgfqpoint{0.975211in}{3.227753in}}%
\pgfusepath{stroke}%
\end{pgfscope}%
\begin{pgfscope}%
\pgfsetbuttcap%
\pgfsetroundjoin%
\definecolor{currentfill}{rgb}{0.000000,0.000000,0.000000}%
\pgfsetfillcolor{currentfill}%
\pgfsetlinewidth{0.501875pt}%
\definecolor{currentstroke}{rgb}{0.000000,0.000000,0.000000}%
\pgfsetstrokecolor{currentstroke}%
\pgfsetdash{}{0pt}%
\pgfsys@defobject{currentmarker}{\pgfqpoint{0.000000in}{0.000000in}}{\pgfqpoint{0.000000in}{0.020833in}}{%
\pgfpathmoveto{\pgfqpoint{0.000000in}{0.000000in}}%
\pgfpathlineto{\pgfqpoint{0.000000in}{0.020833in}}%
\pgfusepath{stroke,fill}%
}%
\begin{pgfscope}%
\pgfsys@transformshift{0.975211in}{1.080890in}%
\pgfsys@useobject{currentmarker}{}%
\end{pgfscope}%
\end{pgfscope}%
\begin{pgfscope}%
\pgfsetbuttcap%
\pgfsetroundjoin%
\definecolor{currentfill}{rgb}{0.000000,0.000000,0.000000}%
\pgfsetfillcolor{currentfill}%
\pgfsetlinewidth{0.501875pt}%
\definecolor{currentstroke}{rgb}{0.000000,0.000000,0.000000}%
\pgfsetstrokecolor{currentstroke}%
\pgfsetdash{}{0pt}%
\pgfsys@defobject{currentmarker}{\pgfqpoint{0.000000in}{-0.020833in}}{\pgfqpoint{0.000000in}{0.000000in}}{%
\pgfpathmoveto{\pgfqpoint{0.000000in}{0.000000in}}%
\pgfpathlineto{\pgfqpoint{0.000000in}{-0.020833in}}%
\pgfusepath{stroke,fill}%
}%
\begin{pgfscope}%
\pgfsys@transformshift{0.975211in}{3.227753in}%
\pgfsys@useobject{currentmarker}{}%
\end{pgfscope}%
\end{pgfscope}%
\begin{pgfscope}%
\pgfpathrectangle{\pgfqpoint{0.481681in}{1.080890in}}{\pgfqpoint{5.785672in}{2.146863in}}%
\pgfusepath{clip}%
\pgfsetrectcap%
\pgfsetroundjoin%
\pgfsetlinewidth{0.100375pt}%
\definecolor{currentstroke}{rgb}{0.827451,0.827451,0.827451}%
\pgfsetstrokecolor{currentstroke}%
\pgfsetdash{}{0pt}%
\pgfpathmoveto{\pgfqpoint{1.010908in}{1.080890in}}%
\pgfpathlineto{\pgfqpoint{1.010908in}{3.227753in}}%
\pgfusepath{stroke}%
\end{pgfscope}%
\begin{pgfscope}%
\pgfsetbuttcap%
\pgfsetroundjoin%
\definecolor{currentfill}{rgb}{0.000000,0.000000,0.000000}%
\pgfsetfillcolor{currentfill}%
\pgfsetlinewidth{0.501875pt}%
\definecolor{currentstroke}{rgb}{0.000000,0.000000,0.000000}%
\pgfsetstrokecolor{currentstroke}%
\pgfsetdash{}{0pt}%
\pgfsys@defobject{currentmarker}{\pgfqpoint{0.000000in}{0.000000in}}{\pgfqpoint{0.000000in}{0.020833in}}{%
\pgfpathmoveto{\pgfqpoint{0.000000in}{0.000000in}}%
\pgfpathlineto{\pgfqpoint{0.000000in}{0.020833in}}%
\pgfusepath{stroke,fill}%
}%
\begin{pgfscope}%
\pgfsys@transformshift{1.010908in}{1.080890in}%
\pgfsys@useobject{currentmarker}{}%
\end{pgfscope}%
\end{pgfscope}%
\begin{pgfscope}%
\pgfsetbuttcap%
\pgfsetroundjoin%
\definecolor{currentfill}{rgb}{0.000000,0.000000,0.000000}%
\pgfsetfillcolor{currentfill}%
\pgfsetlinewidth{0.501875pt}%
\definecolor{currentstroke}{rgb}{0.000000,0.000000,0.000000}%
\pgfsetstrokecolor{currentstroke}%
\pgfsetdash{}{0pt}%
\pgfsys@defobject{currentmarker}{\pgfqpoint{0.000000in}{-0.020833in}}{\pgfqpoint{0.000000in}{0.000000in}}{%
\pgfpathmoveto{\pgfqpoint{0.000000in}{0.000000in}}%
\pgfpathlineto{\pgfqpoint{0.000000in}{-0.020833in}}%
\pgfusepath{stroke,fill}%
}%
\begin{pgfscope}%
\pgfsys@transformshift{1.010908in}{3.227753in}%
\pgfsys@useobject{currentmarker}{}%
\end{pgfscope}%
\end{pgfscope}%
\begin{pgfscope}%
\pgfpathrectangle{\pgfqpoint{0.481681in}{1.080890in}}{\pgfqpoint{5.785672in}{2.146863in}}%
\pgfusepath{clip}%
\pgfsetrectcap%
\pgfsetroundjoin%
\pgfsetlinewidth{0.100375pt}%
\definecolor{currentstroke}{rgb}{0.827451,0.827451,0.827451}%
\pgfsetstrokecolor{currentstroke}%
\pgfsetdash{}{0pt}%
\pgfpathmoveto{\pgfqpoint{1.046606in}{1.080890in}}%
\pgfpathlineto{\pgfqpoint{1.046606in}{3.227753in}}%
\pgfusepath{stroke}%
\end{pgfscope}%
\begin{pgfscope}%
\pgfsetbuttcap%
\pgfsetroundjoin%
\definecolor{currentfill}{rgb}{0.000000,0.000000,0.000000}%
\pgfsetfillcolor{currentfill}%
\pgfsetlinewidth{0.501875pt}%
\definecolor{currentstroke}{rgb}{0.000000,0.000000,0.000000}%
\pgfsetstrokecolor{currentstroke}%
\pgfsetdash{}{0pt}%
\pgfsys@defobject{currentmarker}{\pgfqpoint{0.000000in}{0.000000in}}{\pgfqpoint{0.000000in}{0.020833in}}{%
\pgfpathmoveto{\pgfqpoint{0.000000in}{0.000000in}}%
\pgfpathlineto{\pgfqpoint{0.000000in}{0.020833in}}%
\pgfusepath{stroke,fill}%
}%
\begin{pgfscope}%
\pgfsys@transformshift{1.046606in}{1.080890in}%
\pgfsys@useobject{currentmarker}{}%
\end{pgfscope}%
\end{pgfscope}%
\begin{pgfscope}%
\pgfsetbuttcap%
\pgfsetroundjoin%
\definecolor{currentfill}{rgb}{0.000000,0.000000,0.000000}%
\pgfsetfillcolor{currentfill}%
\pgfsetlinewidth{0.501875pt}%
\definecolor{currentstroke}{rgb}{0.000000,0.000000,0.000000}%
\pgfsetstrokecolor{currentstroke}%
\pgfsetdash{}{0pt}%
\pgfsys@defobject{currentmarker}{\pgfqpoint{0.000000in}{-0.020833in}}{\pgfqpoint{0.000000in}{0.000000in}}{%
\pgfpathmoveto{\pgfqpoint{0.000000in}{0.000000in}}%
\pgfpathlineto{\pgfqpoint{0.000000in}{-0.020833in}}%
\pgfusepath{stroke,fill}%
}%
\begin{pgfscope}%
\pgfsys@transformshift{1.046606in}{3.227753in}%
\pgfsys@useobject{currentmarker}{}%
\end{pgfscope}%
\end{pgfscope}%
\begin{pgfscope}%
\pgfpathrectangle{\pgfqpoint{0.481681in}{1.080890in}}{\pgfqpoint{5.785672in}{2.146863in}}%
\pgfusepath{clip}%
\pgfsetrectcap%
\pgfsetroundjoin%
\pgfsetlinewidth{0.100375pt}%
\definecolor{currentstroke}{rgb}{0.827451,0.827451,0.827451}%
\pgfsetstrokecolor{currentstroke}%
\pgfsetdash{}{0pt}%
\pgfpathmoveto{\pgfqpoint{1.082303in}{1.080890in}}%
\pgfpathlineto{\pgfqpoint{1.082303in}{3.227753in}}%
\pgfusepath{stroke}%
\end{pgfscope}%
\begin{pgfscope}%
\pgfsetbuttcap%
\pgfsetroundjoin%
\definecolor{currentfill}{rgb}{0.000000,0.000000,0.000000}%
\pgfsetfillcolor{currentfill}%
\pgfsetlinewidth{0.501875pt}%
\definecolor{currentstroke}{rgb}{0.000000,0.000000,0.000000}%
\pgfsetstrokecolor{currentstroke}%
\pgfsetdash{}{0pt}%
\pgfsys@defobject{currentmarker}{\pgfqpoint{0.000000in}{0.000000in}}{\pgfqpoint{0.000000in}{0.020833in}}{%
\pgfpathmoveto{\pgfqpoint{0.000000in}{0.000000in}}%
\pgfpathlineto{\pgfqpoint{0.000000in}{0.020833in}}%
\pgfusepath{stroke,fill}%
}%
\begin{pgfscope}%
\pgfsys@transformshift{1.082303in}{1.080890in}%
\pgfsys@useobject{currentmarker}{}%
\end{pgfscope}%
\end{pgfscope}%
\begin{pgfscope}%
\pgfsetbuttcap%
\pgfsetroundjoin%
\definecolor{currentfill}{rgb}{0.000000,0.000000,0.000000}%
\pgfsetfillcolor{currentfill}%
\pgfsetlinewidth{0.501875pt}%
\definecolor{currentstroke}{rgb}{0.000000,0.000000,0.000000}%
\pgfsetstrokecolor{currentstroke}%
\pgfsetdash{}{0pt}%
\pgfsys@defobject{currentmarker}{\pgfqpoint{0.000000in}{-0.020833in}}{\pgfqpoint{0.000000in}{0.000000in}}{%
\pgfpathmoveto{\pgfqpoint{0.000000in}{0.000000in}}%
\pgfpathlineto{\pgfqpoint{0.000000in}{-0.020833in}}%
\pgfusepath{stroke,fill}%
}%
\begin{pgfscope}%
\pgfsys@transformshift{1.082303in}{3.227753in}%
\pgfsys@useobject{currentmarker}{}%
\end{pgfscope}%
\end{pgfscope}%
\begin{pgfscope}%
\pgfpathrectangle{\pgfqpoint{0.481681in}{1.080890in}}{\pgfqpoint{5.785672in}{2.146863in}}%
\pgfusepath{clip}%
\pgfsetrectcap%
\pgfsetroundjoin%
\pgfsetlinewidth{0.100375pt}%
\definecolor{currentstroke}{rgb}{0.827451,0.827451,0.827451}%
\pgfsetstrokecolor{currentstroke}%
\pgfsetdash{}{0pt}%
\pgfpathmoveto{\pgfqpoint{1.153698in}{1.080890in}}%
\pgfpathlineto{\pgfqpoint{1.153698in}{3.227753in}}%
\pgfusepath{stroke}%
\end{pgfscope}%
\begin{pgfscope}%
\pgfsetbuttcap%
\pgfsetroundjoin%
\definecolor{currentfill}{rgb}{0.000000,0.000000,0.000000}%
\pgfsetfillcolor{currentfill}%
\pgfsetlinewidth{0.501875pt}%
\definecolor{currentstroke}{rgb}{0.000000,0.000000,0.000000}%
\pgfsetstrokecolor{currentstroke}%
\pgfsetdash{}{0pt}%
\pgfsys@defobject{currentmarker}{\pgfqpoint{0.000000in}{0.000000in}}{\pgfqpoint{0.000000in}{0.020833in}}{%
\pgfpathmoveto{\pgfqpoint{0.000000in}{0.000000in}}%
\pgfpathlineto{\pgfqpoint{0.000000in}{0.020833in}}%
\pgfusepath{stroke,fill}%
}%
\begin{pgfscope}%
\pgfsys@transformshift{1.153698in}{1.080890in}%
\pgfsys@useobject{currentmarker}{}%
\end{pgfscope}%
\end{pgfscope}%
\begin{pgfscope}%
\pgfsetbuttcap%
\pgfsetroundjoin%
\definecolor{currentfill}{rgb}{0.000000,0.000000,0.000000}%
\pgfsetfillcolor{currentfill}%
\pgfsetlinewidth{0.501875pt}%
\definecolor{currentstroke}{rgb}{0.000000,0.000000,0.000000}%
\pgfsetstrokecolor{currentstroke}%
\pgfsetdash{}{0pt}%
\pgfsys@defobject{currentmarker}{\pgfqpoint{0.000000in}{-0.020833in}}{\pgfqpoint{0.000000in}{0.000000in}}{%
\pgfpathmoveto{\pgfqpoint{0.000000in}{0.000000in}}%
\pgfpathlineto{\pgfqpoint{0.000000in}{-0.020833in}}%
\pgfusepath{stroke,fill}%
}%
\begin{pgfscope}%
\pgfsys@transformshift{1.153698in}{3.227753in}%
\pgfsys@useobject{currentmarker}{}%
\end{pgfscope}%
\end{pgfscope}%
\begin{pgfscope}%
\pgfpathrectangle{\pgfqpoint{0.481681in}{1.080890in}}{\pgfqpoint{5.785672in}{2.146863in}}%
\pgfusepath{clip}%
\pgfsetrectcap%
\pgfsetroundjoin%
\pgfsetlinewidth{0.100375pt}%
\definecolor{currentstroke}{rgb}{0.827451,0.827451,0.827451}%
\pgfsetstrokecolor{currentstroke}%
\pgfsetdash{}{0pt}%
\pgfpathmoveto{\pgfqpoint{1.189395in}{1.080890in}}%
\pgfpathlineto{\pgfqpoint{1.189395in}{3.227753in}}%
\pgfusepath{stroke}%
\end{pgfscope}%
\begin{pgfscope}%
\pgfsetbuttcap%
\pgfsetroundjoin%
\definecolor{currentfill}{rgb}{0.000000,0.000000,0.000000}%
\pgfsetfillcolor{currentfill}%
\pgfsetlinewidth{0.501875pt}%
\definecolor{currentstroke}{rgb}{0.000000,0.000000,0.000000}%
\pgfsetstrokecolor{currentstroke}%
\pgfsetdash{}{0pt}%
\pgfsys@defobject{currentmarker}{\pgfqpoint{0.000000in}{0.000000in}}{\pgfqpoint{0.000000in}{0.020833in}}{%
\pgfpathmoveto{\pgfqpoint{0.000000in}{0.000000in}}%
\pgfpathlineto{\pgfqpoint{0.000000in}{0.020833in}}%
\pgfusepath{stroke,fill}%
}%
\begin{pgfscope}%
\pgfsys@transformshift{1.189395in}{1.080890in}%
\pgfsys@useobject{currentmarker}{}%
\end{pgfscope}%
\end{pgfscope}%
\begin{pgfscope}%
\pgfsetbuttcap%
\pgfsetroundjoin%
\definecolor{currentfill}{rgb}{0.000000,0.000000,0.000000}%
\pgfsetfillcolor{currentfill}%
\pgfsetlinewidth{0.501875pt}%
\definecolor{currentstroke}{rgb}{0.000000,0.000000,0.000000}%
\pgfsetstrokecolor{currentstroke}%
\pgfsetdash{}{0pt}%
\pgfsys@defobject{currentmarker}{\pgfqpoint{0.000000in}{-0.020833in}}{\pgfqpoint{0.000000in}{0.000000in}}{%
\pgfpathmoveto{\pgfqpoint{0.000000in}{0.000000in}}%
\pgfpathlineto{\pgfqpoint{0.000000in}{-0.020833in}}%
\pgfusepath{stroke,fill}%
}%
\begin{pgfscope}%
\pgfsys@transformshift{1.189395in}{3.227753in}%
\pgfsys@useobject{currentmarker}{}%
\end{pgfscope}%
\end{pgfscope}%
\begin{pgfscope}%
\pgfpathrectangle{\pgfqpoint{0.481681in}{1.080890in}}{\pgfqpoint{5.785672in}{2.146863in}}%
\pgfusepath{clip}%
\pgfsetrectcap%
\pgfsetroundjoin%
\pgfsetlinewidth{0.100375pt}%
\definecolor{currentstroke}{rgb}{0.827451,0.827451,0.827451}%
\pgfsetstrokecolor{currentstroke}%
\pgfsetdash{}{0pt}%
\pgfpathmoveto{\pgfqpoint{1.225092in}{1.080890in}}%
\pgfpathlineto{\pgfqpoint{1.225092in}{3.227753in}}%
\pgfusepath{stroke}%
\end{pgfscope}%
\begin{pgfscope}%
\pgfsetbuttcap%
\pgfsetroundjoin%
\definecolor{currentfill}{rgb}{0.000000,0.000000,0.000000}%
\pgfsetfillcolor{currentfill}%
\pgfsetlinewidth{0.501875pt}%
\definecolor{currentstroke}{rgb}{0.000000,0.000000,0.000000}%
\pgfsetstrokecolor{currentstroke}%
\pgfsetdash{}{0pt}%
\pgfsys@defobject{currentmarker}{\pgfqpoint{0.000000in}{0.000000in}}{\pgfqpoint{0.000000in}{0.020833in}}{%
\pgfpathmoveto{\pgfqpoint{0.000000in}{0.000000in}}%
\pgfpathlineto{\pgfqpoint{0.000000in}{0.020833in}}%
\pgfusepath{stroke,fill}%
}%
\begin{pgfscope}%
\pgfsys@transformshift{1.225092in}{1.080890in}%
\pgfsys@useobject{currentmarker}{}%
\end{pgfscope}%
\end{pgfscope}%
\begin{pgfscope}%
\pgfsetbuttcap%
\pgfsetroundjoin%
\definecolor{currentfill}{rgb}{0.000000,0.000000,0.000000}%
\pgfsetfillcolor{currentfill}%
\pgfsetlinewidth{0.501875pt}%
\definecolor{currentstroke}{rgb}{0.000000,0.000000,0.000000}%
\pgfsetstrokecolor{currentstroke}%
\pgfsetdash{}{0pt}%
\pgfsys@defobject{currentmarker}{\pgfqpoint{0.000000in}{-0.020833in}}{\pgfqpoint{0.000000in}{0.000000in}}{%
\pgfpathmoveto{\pgfqpoint{0.000000in}{0.000000in}}%
\pgfpathlineto{\pgfqpoint{0.000000in}{-0.020833in}}%
\pgfusepath{stroke,fill}%
}%
\begin{pgfscope}%
\pgfsys@transformshift{1.225092in}{3.227753in}%
\pgfsys@useobject{currentmarker}{}%
\end{pgfscope}%
\end{pgfscope}%
\begin{pgfscope}%
\pgfpathrectangle{\pgfqpoint{0.481681in}{1.080890in}}{\pgfqpoint{5.785672in}{2.146863in}}%
\pgfusepath{clip}%
\pgfsetrectcap%
\pgfsetroundjoin%
\pgfsetlinewidth{0.100375pt}%
\definecolor{currentstroke}{rgb}{0.827451,0.827451,0.827451}%
\pgfsetstrokecolor{currentstroke}%
\pgfsetdash{}{0pt}%
\pgfpathmoveto{\pgfqpoint{1.260790in}{1.080890in}}%
\pgfpathlineto{\pgfqpoint{1.260790in}{3.227753in}}%
\pgfusepath{stroke}%
\end{pgfscope}%
\begin{pgfscope}%
\pgfsetbuttcap%
\pgfsetroundjoin%
\definecolor{currentfill}{rgb}{0.000000,0.000000,0.000000}%
\pgfsetfillcolor{currentfill}%
\pgfsetlinewidth{0.501875pt}%
\definecolor{currentstroke}{rgb}{0.000000,0.000000,0.000000}%
\pgfsetstrokecolor{currentstroke}%
\pgfsetdash{}{0pt}%
\pgfsys@defobject{currentmarker}{\pgfqpoint{0.000000in}{0.000000in}}{\pgfqpoint{0.000000in}{0.020833in}}{%
\pgfpathmoveto{\pgfqpoint{0.000000in}{0.000000in}}%
\pgfpathlineto{\pgfqpoint{0.000000in}{0.020833in}}%
\pgfusepath{stroke,fill}%
}%
\begin{pgfscope}%
\pgfsys@transformshift{1.260790in}{1.080890in}%
\pgfsys@useobject{currentmarker}{}%
\end{pgfscope}%
\end{pgfscope}%
\begin{pgfscope}%
\pgfsetbuttcap%
\pgfsetroundjoin%
\definecolor{currentfill}{rgb}{0.000000,0.000000,0.000000}%
\pgfsetfillcolor{currentfill}%
\pgfsetlinewidth{0.501875pt}%
\definecolor{currentstroke}{rgb}{0.000000,0.000000,0.000000}%
\pgfsetstrokecolor{currentstroke}%
\pgfsetdash{}{0pt}%
\pgfsys@defobject{currentmarker}{\pgfqpoint{0.000000in}{-0.020833in}}{\pgfqpoint{0.000000in}{0.000000in}}{%
\pgfpathmoveto{\pgfqpoint{0.000000in}{0.000000in}}%
\pgfpathlineto{\pgfqpoint{0.000000in}{-0.020833in}}%
\pgfusepath{stroke,fill}%
}%
\begin{pgfscope}%
\pgfsys@transformshift{1.260790in}{3.227753in}%
\pgfsys@useobject{currentmarker}{}%
\end{pgfscope}%
\end{pgfscope}%
\begin{pgfscope}%
\pgfpathrectangle{\pgfqpoint{0.481681in}{1.080890in}}{\pgfqpoint{5.785672in}{2.146863in}}%
\pgfusepath{clip}%
\pgfsetrectcap%
\pgfsetroundjoin%
\pgfsetlinewidth{0.100375pt}%
\definecolor{currentstroke}{rgb}{0.827451,0.827451,0.827451}%
\pgfsetstrokecolor{currentstroke}%
\pgfsetdash{}{0pt}%
\pgfpathmoveto{\pgfqpoint{1.296487in}{1.080890in}}%
\pgfpathlineto{\pgfqpoint{1.296487in}{3.227753in}}%
\pgfusepath{stroke}%
\end{pgfscope}%
\begin{pgfscope}%
\pgfsetbuttcap%
\pgfsetroundjoin%
\definecolor{currentfill}{rgb}{0.000000,0.000000,0.000000}%
\pgfsetfillcolor{currentfill}%
\pgfsetlinewidth{0.501875pt}%
\definecolor{currentstroke}{rgb}{0.000000,0.000000,0.000000}%
\pgfsetstrokecolor{currentstroke}%
\pgfsetdash{}{0pt}%
\pgfsys@defobject{currentmarker}{\pgfqpoint{0.000000in}{0.000000in}}{\pgfqpoint{0.000000in}{0.020833in}}{%
\pgfpathmoveto{\pgfqpoint{0.000000in}{0.000000in}}%
\pgfpathlineto{\pgfqpoint{0.000000in}{0.020833in}}%
\pgfusepath{stroke,fill}%
}%
\begin{pgfscope}%
\pgfsys@transformshift{1.296487in}{1.080890in}%
\pgfsys@useobject{currentmarker}{}%
\end{pgfscope}%
\end{pgfscope}%
\begin{pgfscope}%
\pgfsetbuttcap%
\pgfsetroundjoin%
\definecolor{currentfill}{rgb}{0.000000,0.000000,0.000000}%
\pgfsetfillcolor{currentfill}%
\pgfsetlinewidth{0.501875pt}%
\definecolor{currentstroke}{rgb}{0.000000,0.000000,0.000000}%
\pgfsetstrokecolor{currentstroke}%
\pgfsetdash{}{0pt}%
\pgfsys@defobject{currentmarker}{\pgfqpoint{0.000000in}{-0.020833in}}{\pgfqpoint{0.000000in}{0.000000in}}{%
\pgfpathmoveto{\pgfqpoint{0.000000in}{0.000000in}}%
\pgfpathlineto{\pgfqpoint{0.000000in}{-0.020833in}}%
\pgfusepath{stroke,fill}%
}%
\begin{pgfscope}%
\pgfsys@transformshift{1.296487in}{3.227753in}%
\pgfsys@useobject{currentmarker}{}%
\end{pgfscope}%
\end{pgfscope}%
\begin{pgfscope}%
\pgfpathrectangle{\pgfqpoint{0.481681in}{1.080890in}}{\pgfqpoint{5.785672in}{2.146863in}}%
\pgfusepath{clip}%
\pgfsetrectcap%
\pgfsetroundjoin%
\pgfsetlinewidth{0.100375pt}%
\definecolor{currentstroke}{rgb}{0.827451,0.827451,0.827451}%
\pgfsetstrokecolor{currentstroke}%
\pgfsetdash{}{0pt}%
\pgfpathmoveto{\pgfqpoint{1.332184in}{1.080890in}}%
\pgfpathlineto{\pgfqpoint{1.332184in}{3.227753in}}%
\pgfusepath{stroke}%
\end{pgfscope}%
\begin{pgfscope}%
\pgfsetbuttcap%
\pgfsetroundjoin%
\definecolor{currentfill}{rgb}{0.000000,0.000000,0.000000}%
\pgfsetfillcolor{currentfill}%
\pgfsetlinewidth{0.501875pt}%
\definecolor{currentstroke}{rgb}{0.000000,0.000000,0.000000}%
\pgfsetstrokecolor{currentstroke}%
\pgfsetdash{}{0pt}%
\pgfsys@defobject{currentmarker}{\pgfqpoint{0.000000in}{0.000000in}}{\pgfqpoint{0.000000in}{0.020833in}}{%
\pgfpathmoveto{\pgfqpoint{0.000000in}{0.000000in}}%
\pgfpathlineto{\pgfqpoint{0.000000in}{0.020833in}}%
\pgfusepath{stroke,fill}%
}%
\begin{pgfscope}%
\pgfsys@transformshift{1.332184in}{1.080890in}%
\pgfsys@useobject{currentmarker}{}%
\end{pgfscope}%
\end{pgfscope}%
\begin{pgfscope}%
\pgfsetbuttcap%
\pgfsetroundjoin%
\definecolor{currentfill}{rgb}{0.000000,0.000000,0.000000}%
\pgfsetfillcolor{currentfill}%
\pgfsetlinewidth{0.501875pt}%
\definecolor{currentstroke}{rgb}{0.000000,0.000000,0.000000}%
\pgfsetstrokecolor{currentstroke}%
\pgfsetdash{}{0pt}%
\pgfsys@defobject{currentmarker}{\pgfqpoint{0.000000in}{-0.020833in}}{\pgfqpoint{0.000000in}{0.000000in}}{%
\pgfpathmoveto{\pgfqpoint{0.000000in}{0.000000in}}%
\pgfpathlineto{\pgfqpoint{0.000000in}{-0.020833in}}%
\pgfusepath{stroke,fill}%
}%
\begin{pgfscope}%
\pgfsys@transformshift{1.332184in}{3.227753in}%
\pgfsys@useobject{currentmarker}{}%
\end{pgfscope}%
\end{pgfscope}%
\begin{pgfscope}%
\pgfpathrectangle{\pgfqpoint{0.481681in}{1.080890in}}{\pgfqpoint{5.785672in}{2.146863in}}%
\pgfusepath{clip}%
\pgfsetrectcap%
\pgfsetroundjoin%
\pgfsetlinewidth{0.100375pt}%
\definecolor{currentstroke}{rgb}{0.827451,0.827451,0.827451}%
\pgfsetstrokecolor{currentstroke}%
\pgfsetdash{}{0pt}%
\pgfpathmoveto{\pgfqpoint{1.367882in}{1.080890in}}%
\pgfpathlineto{\pgfqpoint{1.367882in}{3.227753in}}%
\pgfusepath{stroke}%
\end{pgfscope}%
\begin{pgfscope}%
\pgfsetbuttcap%
\pgfsetroundjoin%
\definecolor{currentfill}{rgb}{0.000000,0.000000,0.000000}%
\pgfsetfillcolor{currentfill}%
\pgfsetlinewidth{0.501875pt}%
\definecolor{currentstroke}{rgb}{0.000000,0.000000,0.000000}%
\pgfsetstrokecolor{currentstroke}%
\pgfsetdash{}{0pt}%
\pgfsys@defobject{currentmarker}{\pgfqpoint{0.000000in}{0.000000in}}{\pgfqpoint{0.000000in}{0.020833in}}{%
\pgfpathmoveto{\pgfqpoint{0.000000in}{0.000000in}}%
\pgfpathlineto{\pgfqpoint{0.000000in}{0.020833in}}%
\pgfusepath{stroke,fill}%
}%
\begin{pgfscope}%
\pgfsys@transformshift{1.367882in}{1.080890in}%
\pgfsys@useobject{currentmarker}{}%
\end{pgfscope}%
\end{pgfscope}%
\begin{pgfscope}%
\pgfsetbuttcap%
\pgfsetroundjoin%
\definecolor{currentfill}{rgb}{0.000000,0.000000,0.000000}%
\pgfsetfillcolor{currentfill}%
\pgfsetlinewidth{0.501875pt}%
\definecolor{currentstroke}{rgb}{0.000000,0.000000,0.000000}%
\pgfsetstrokecolor{currentstroke}%
\pgfsetdash{}{0pt}%
\pgfsys@defobject{currentmarker}{\pgfqpoint{0.000000in}{-0.020833in}}{\pgfqpoint{0.000000in}{0.000000in}}{%
\pgfpathmoveto{\pgfqpoint{0.000000in}{0.000000in}}%
\pgfpathlineto{\pgfqpoint{0.000000in}{-0.020833in}}%
\pgfusepath{stroke,fill}%
}%
\begin{pgfscope}%
\pgfsys@transformshift{1.367882in}{3.227753in}%
\pgfsys@useobject{currentmarker}{}%
\end{pgfscope}%
\end{pgfscope}%
\begin{pgfscope}%
\pgfpathrectangle{\pgfqpoint{0.481681in}{1.080890in}}{\pgfqpoint{5.785672in}{2.146863in}}%
\pgfusepath{clip}%
\pgfsetrectcap%
\pgfsetroundjoin%
\pgfsetlinewidth{0.100375pt}%
\definecolor{currentstroke}{rgb}{0.827451,0.827451,0.827451}%
\pgfsetstrokecolor{currentstroke}%
\pgfsetdash{}{0pt}%
\pgfpathmoveto{\pgfqpoint{1.403579in}{1.080890in}}%
\pgfpathlineto{\pgfqpoint{1.403579in}{3.227753in}}%
\pgfusepath{stroke}%
\end{pgfscope}%
\begin{pgfscope}%
\pgfsetbuttcap%
\pgfsetroundjoin%
\definecolor{currentfill}{rgb}{0.000000,0.000000,0.000000}%
\pgfsetfillcolor{currentfill}%
\pgfsetlinewidth{0.501875pt}%
\definecolor{currentstroke}{rgb}{0.000000,0.000000,0.000000}%
\pgfsetstrokecolor{currentstroke}%
\pgfsetdash{}{0pt}%
\pgfsys@defobject{currentmarker}{\pgfqpoint{0.000000in}{0.000000in}}{\pgfqpoint{0.000000in}{0.020833in}}{%
\pgfpathmoveto{\pgfqpoint{0.000000in}{0.000000in}}%
\pgfpathlineto{\pgfqpoint{0.000000in}{0.020833in}}%
\pgfusepath{stroke,fill}%
}%
\begin{pgfscope}%
\pgfsys@transformshift{1.403579in}{1.080890in}%
\pgfsys@useobject{currentmarker}{}%
\end{pgfscope}%
\end{pgfscope}%
\begin{pgfscope}%
\pgfsetbuttcap%
\pgfsetroundjoin%
\definecolor{currentfill}{rgb}{0.000000,0.000000,0.000000}%
\pgfsetfillcolor{currentfill}%
\pgfsetlinewidth{0.501875pt}%
\definecolor{currentstroke}{rgb}{0.000000,0.000000,0.000000}%
\pgfsetstrokecolor{currentstroke}%
\pgfsetdash{}{0pt}%
\pgfsys@defobject{currentmarker}{\pgfqpoint{0.000000in}{-0.020833in}}{\pgfqpoint{0.000000in}{0.000000in}}{%
\pgfpathmoveto{\pgfqpoint{0.000000in}{0.000000in}}%
\pgfpathlineto{\pgfqpoint{0.000000in}{-0.020833in}}%
\pgfusepath{stroke,fill}%
}%
\begin{pgfscope}%
\pgfsys@transformshift{1.403579in}{3.227753in}%
\pgfsys@useobject{currentmarker}{}%
\end{pgfscope}%
\end{pgfscope}%
\begin{pgfscope}%
\pgfpathrectangle{\pgfqpoint{0.481681in}{1.080890in}}{\pgfqpoint{5.785672in}{2.146863in}}%
\pgfusepath{clip}%
\pgfsetrectcap%
\pgfsetroundjoin%
\pgfsetlinewidth{0.100375pt}%
\definecolor{currentstroke}{rgb}{0.827451,0.827451,0.827451}%
\pgfsetstrokecolor{currentstroke}%
\pgfsetdash{}{0pt}%
\pgfpathmoveto{\pgfqpoint{1.439276in}{1.080890in}}%
\pgfpathlineto{\pgfqpoint{1.439276in}{3.227753in}}%
\pgfusepath{stroke}%
\end{pgfscope}%
\begin{pgfscope}%
\pgfsetbuttcap%
\pgfsetroundjoin%
\definecolor{currentfill}{rgb}{0.000000,0.000000,0.000000}%
\pgfsetfillcolor{currentfill}%
\pgfsetlinewidth{0.501875pt}%
\definecolor{currentstroke}{rgb}{0.000000,0.000000,0.000000}%
\pgfsetstrokecolor{currentstroke}%
\pgfsetdash{}{0pt}%
\pgfsys@defobject{currentmarker}{\pgfqpoint{0.000000in}{0.000000in}}{\pgfqpoint{0.000000in}{0.020833in}}{%
\pgfpathmoveto{\pgfqpoint{0.000000in}{0.000000in}}%
\pgfpathlineto{\pgfqpoint{0.000000in}{0.020833in}}%
\pgfusepath{stroke,fill}%
}%
\begin{pgfscope}%
\pgfsys@transformshift{1.439276in}{1.080890in}%
\pgfsys@useobject{currentmarker}{}%
\end{pgfscope}%
\end{pgfscope}%
\begin{pgfscope}%
\pgfsetbuttcap%
\pgfsetroundjoin%
\definecolor{currentfill}{rgb}{0.000000,0.000000,0.000000}%
\pgfsetfillcolor{currentfill}%
\pgfsetlinewidth{0.501875pt}%
\definecolor{currentstroke}{rgb}{0.000000,0.000000,0.000000}%
\pgfsetstrokecolor{currentstroke}%
\pgfsetdash{}{0pt}%
\pgfsys@defobject{currentmarker}{\pgfqpoint{0.000000in}{-0.020833in}}{\pgfqpoint{0.000000in}{0.000000in}}{%
\pgfpathmoveto{\pgfqpoint{0.000000in}{0.000000in}}%
\pgfpathlineto{\pgfqpoint{0.000000in}{-0.020833in}}%
\pgfusepath{stroke,fill}%
}%
\begin{pgfscope}%
\pgfsys@transformshift{1.439276in}{3.227753in}%
\pgfsys@useobject{currentmarker}{}%
\end{pgfscope}%
\end{pgfscope}%
\begin{pgfscope}%
\pgfpathrectangle{\pgfqpoint{0.481681in}{1.080890in}}{\pgfqpoint{5.785672in}{2.146863in}}%
\pgfusepath{clip}%
\pgfsetrectcap%
\pgfsetroundjoin%
\pgfsetlinewidth{0.100375pt}%
\definecolor{currentstroke}{rgb}{0.827451,0.827451,0.827451}%
\pgfsetstrokecolor{currentstroke}%
\pgfsetdash{}{0pt}%
\pgfpathmoveto{\pgfqpoint{1.474974in}{1.080890in}}%
\pgfpathlineto{\pgfqpoint{1.474974in}{3.227753in}}%
\pgfusepath{stroke}%
\end{pgfscope}%
\begin{pgfscope}%
\pgfsetbuttcap%
\pgfsetroundjoin%
\definecolor{currentfill}{rgb}{0.000000,0.000000,0.000000}%
\pgfsetfillcolor{currentfill}%
\pgfsetlinewidth{0.501875pt}%
\definecolor{currentstroke}{rgb}{0.000000,0.000000,0.000000}%
\pgfsetstrokecolor{currentstroke}%
\pgfsetdash{}{0pt}%
\pgfsys@defobject{currentmarker}{\pgfqpoint{0.000000in}{0.000000in}}{\pgfqpoint{0.000000in}{0.020833in}}{%
\pgfpathmoveto{\pgfqpoint{0.000000in}{0.000000in}}%
\pgfpathlineto{\pgfqpoint{0.000000in}{0.020833in}}%
\pgfusepath{stroke,fill}%
}%
\begin{pgfscope}%
\pgfsys@transformshift{1.474974in}{1.080890in}%
\pgfsys@useobject{currentmarker}{}%
\end{pgfscope}%
\end{pgfscope}%
\begin{pgfscope}%
\pgfsetbuttcap%
\pgfsetroundjoin%
\definecolor{currentfill}{rgb}{0.000000,0.000000,0.000000}%
\pgfsetfillcolor{currentfill}%
\pgfsetlinewidth{0.501875pt}%
\definecolor{currentstroke}{rgb}{0.000000,0.000000,0.000000}%
\pgfsetstrokecolor{currentstroke}%
\pgfsetdash{}{0pt}%
\pgfsys@defobject{currentmarker}{\pgfqpoint{0.000000in}{-0.020833in}}{\pgfqpoint{0.000000in}{0.000000in}}{%
\pgfpathmoveto{\pgfqpoint{0.000000in}{0.000000in}}%
\pgfpathlineto{\pgfqpoint{0.000000in}{-0.020833in}}%
\pgfusepath{stroke,fill}%
}%
\begin{pgfscope}%
\pgfsys@transformshift{1.474974in}{3.227753in}%
\pgfsys@useobject{currentmarker}{}%
\end{pgfscope}%
\end{pgfscope}%
\begin{pgfscope}%
\pgfpathrectangle{\pgfqpoint{0.481681in}{1.080890in}}{\pgfqpoint{5.785672in}{2.146863in}}%
\pgfusepath{clip}%
\pgfsetrectcap%
\pgfsetroundjoin%
\pgfsetlinewidth{0.100375pt}%
\definecolor{currentstroke}{rgb}{0.827451,0.827451,0.827451}%
\pgfsetstrokecolor{currentstroke}%
\pgfsetdash{}{0pt}%
\pgfpathmoveto{\pgfqpoint{1.510671in}{1.080890in}}%
\pgfpathlineto{\pgfqpoint{1.510671in}{3.227753in}}%
\pgfusepath{stroke}%
\end{pgfscope}%
\begin{pgfscope}%
\pgfsetbuttcap%
\pgfsetroundjoin%
\definecolor{currentfill}{rgb}{0.000000,0.000000,0.000000}%
\pgfsetfillcolor{currentfill}%
\pgfsetlinewidth{0.501875pt}%
\definecolor{currentstroke}{rgb}{0.000000,0.000000,0.000000}%
\pgfsetstrokecolor{currentstroke}%
\pgfsetdash{}{0pt}%
\pgfsys@defobject{currentmarker}{\pgfqpoint{0.000000in}{0.000000in}}{\pgfqpoint{0.000000in}{0.020833in}}{%
\pgfpathmoveto{\pgfqpoint{0.000000in}{0.000000in}}%
\pgfpathlineto{\pgfqpoint{0.000000in}{0.020833in}}%
\pgfusepath{stroke,fill}%
}%
\begin{pgfscope}%
\pgfsys@transformshift{1.510671in}{1.080890in}%
\pgfsys@useobject{currentmarker}{}%
\end{pgfscope}%
\end{pgfscope}%
\begin{pgfscope}%
\pgfsetbuttcap%
\pgfsetroundjoin%
\definecolor{currentfill}{rgb}{0.000000,0.000000,0.000000}%
\pgfsetfillcolor{currentfill}%
\pgfsetlinewidth{0.501875pt}%
\definecolor{currentstroke}{rgb}{0.000000,0.000000,0.000000}%
\pgfsetstrokecolor{currentstroke}%
\pgfsetdash{}{0pt}%
\pgfsys@defobject{currentmarker}{\pgfqpoint{0.000000in}{-0.020833in}}{\pgfqpoint{0.000000in}{0.000000in}}{%
\pgfpathmoveto{\pgfqpoint{0.000000in}{0.000000in}}%
\pgfpathlineto{\pgfqpoint{0.000000in}{-0.020833in}}%
\pgfusepath{stroke,fill}%
}%
\begin{pgfscope}%
\pgfsys@transformshift{1.510671in}{3.227753in}%
\pgfsys@useobject{currentmarker}{}%
\end{pgfscope}%
\end{pgfscope}%
\begin{pgfscope}%
\pgfpathrectangle{\pgfqpoint{0.481681in}{1.080890in}}{\pgfqpoint{5.785672in}{2.146863in}}%
\pgfusepath{clip}%
\pgfsetrectcap%
\pgfsetroundjoin%
\pgfsetlinewidth{0.100375pt}%
\definecolor{currentstroke}{rgb}{0.827451,0.827451,0.827451}%
\pgfsetstrokecolor{currentstroke}%
\pgfsetdash{}{0pt}%
\pgfpathmoveto{\pgfqpoint{1.582065in}{1.080890in}}%
\pgfpathlineto{\pgfqpoint{1.582065in}{3.227753in}}%
\pgfusepath{stroke}%
\end{pgfscope}%
\begin{pgfscope}%
\pgfsetbuttcap%
\pgfsetroundjoin%
\definecolor{currentfill}{rgb}{0.000000,0.000000,0.000000}%
\pgfsetfillcolor{currentfill}%
\pgfsetlinewidth{0.501875pt}%
\definecolor{currentstroke}{rgb}{0.000000,0.000000,0.000000}%
\pgfsetstrokecolor{currentstroke}%
\pgfsetdash{}{0pt}%
\pgfsys@defobject{currentmarker}{\pgfqpoint{0.000000in}{0.000000in}}{\pgfqpoint{0.000000in}{0.020833in}}{%
\pgfpathmoveto{\pgfqpoint{0.000000in}{0.000000in}}%
\pgfpathlineto{\pgfqpoint{0.000000in}{0.020833in}}%
\pgfusepath{stroke,fill}%
}%
\begin{pgfscope}%
\pgfsys@transformshift{1.582065in}{1.080890in}%
\pgfsys@useobject{currentmarker}{}%
\end{pgfscope}%
\end{pgfscope}%
\begin{pgfscope}%
\pgfsetbuttcap%
\pgfsetroundjoin%
\definecolor{currentfill}{rgb}{0.000000,0.000000,0.000000}%
\pgfsetfillcolor{currentfill}%
\pgfsetlinewidth{0.501875pt}%
\definecolor{currentstroke}{rgb}{0.000000,0.000000,0.000000}%
\pgfsetstrokecolor{currentstroke}%
\pgfsetdash{}{0pt}%
\pgfsys@defobject{currentmarker}{\pgfqpoint{0.000000in}{-0.020833in}}{\pgfqpoint{0.000000in}{0.000000in}}{%
\pgfpathmoveto{\pgfqpoint{0.000000in}{0.000000in}}%
\pgfpathlineto{\pgfqpoint{0.000000in}{-0.020833in}}%
\pgfusepath{stroke,fill}%
}%
\begin{pgfscope}%
\pgfsys@transformshift{1.582065in}{3.227753in}%
\pgfsys@useobject{currentmarker}{}%
\end{pgfscope}%
\end{pgfscope}%
\begin{pgfscope}%
\pgfpathrectangle{\pgfqpoint{0.481681in}{1.080890in}}{\pgfqpoint{5.785672in}{2.146863in}}%
\pgfusepath{clip}%
\pgfsetrectcap%
\pgfsetroundjoin%
\pgfsetlinewidth{0.100375pt}%
\definecolor{currentstroke}{rgb}{0.827451,0.827451,0.827451}%
\pgfsetstrokecolor{currentstroke}%
\pgfsetdash{}{0pt}%
\pgfpathmoveto{\pgfqpoint{1.617763in}{1.080890in}}%
\pgfpathlineto{\pgfqpoint{1.617763in}{3.227753in}}%
\pgfusepath{stroke}%
\end{pgfscope}%
\begin{pgfscope}%
\pgfsetbuttcap%
\pgfsetroundjoin%
\definecolor{currentfill}{rgb}{0.000000,0.000000,0.000000}%
\pgfsetfillcolor{currentfill}%
\pgfsetlinewidth{0.501875pt}%
\definecolor{currentstroke}{rgb}{0.000000,0.000000,0.000000}%
\pgfsetstrokecolor{currentstroke}%
\pgfsetdash{}{0pt}%
\pgfsys@defobject{currentmarker}{\pgfqpoint{0.000000in}{0.000000in}}{\pgfqpoint{0.000000in}{0.020833in}}{%
\pgfpathmoveto{\pgfqpoint{0.000000in}{0.000000in}}%
\pgfpathlineto{\pgfqpoint{0.000000in}{0.020833in}}%
\pgfusepath{stroke,fill}%
}%
\begin{pgfscope}%
\pgfsys@transformshift{1.617763in}{1.080890in}%
\pgfsys@useobject{currentmarker}{}%
\end{pgfscope}%
\end{pgfscope}%
\begin{pgfscope}%
\pgfsetbuttcap%
\pgfsetroundjoin%
\definecolor{currentfill}{rgb}{0.000000,0.000000,0.000000}%
\pgfsetfillcolor{currentfill}%
\pgfsetlinewidth{0.501875pt}%
\definecolor{currentstroke}{rgb}{0.000000,0.000000,0.000000}%
\pgfsetstrokecolor{currentstroke}%
\pgfsetdash{}{0pt}%
\pgfsys@defobject{currentmarker}{\pgfqpoint{0.000000in}{-0.020833in}}{\pgfqpoint{0.000000in}{0.000000in}}{%
\pgfpathmoveto{\pgfqpoint{0.000000in}{0.000000in}}%
\pgfpathlineto{\pgfqpoint{0.000000in}{-0.020833in}}%
\pgfusepath{stroke,fill}%
}%
\begin{pgfscope}%
\pgfsys@transformshift{1.617763in}{3.227753in}%
\pgfsys@useobject{currentmarker}{}%
\end{pgfscope}%
\end{pgfscope}%
\begin{pgfscope}%
\pgfpathrectangle{\pgfqpoint{0.481681in}{1.080890in}}{\pgfqpoint{5.785672in}{2.146863in}}%
\pgfusepath{clip}%
\pgfsetrectcap%
\pgfsetroundjoin%
\pgfsetlinewidth{0.100375pt}%
\definecolor{currentstroke}{rgb}{0.827451,0.827451,0.827451}%
\pgfsetstrokecolor{currentstroke}%
\pgfsetdash{}{0pt}%
\pgfpathmoveto{\pgfqpoint{1.653460in}{1.080890in}}%
\pgfpathlineto{\pgfqpoint{1.653460in}{3.227753in}}%
\pgfusepath{stroke}%
\end{pgfscope}%
\begin{pgfscope}%
\pgfsetbuttcap%
\pgfsetroundjoin%
\definecolor{currentfill}{rgb}{0.000000,0.000000,0.000000}%
\pgfsetfillcolor{currentfill}%
\pgfsetlinewidth{0.501875pt}%
\definecolor{currentstroke}{rgb}{0.000000,0.000000,0.000000}%
\pgfsetstrokecolor{currentstroke}%
\pgfsetdash{}{0pt}%
\pgfsys@defobject{currentmarker}{\pgfqpoint{0.000000in}{0.000000in}}{\pgfqpoint{0.000000in}{0.020833in}}{%
\pgfpathmoveto{\pgfqpoint{0.000000in}{0.000000in}}%
\pgfpathlineto{\pgfqpoint{0.000000in}{0.020833in}}%
\pgfusepath{stroke,fill}%
}%
\begin{pgfscope}%
\pgfsys@transformshift{1.653460in}{1.080890in}%
\pgfsys@useobject{currentmarker}{}%
\end{pgfscope}%
\end{pgfscope}%
\begin{pgfscope}%
\pgfsetbuttcap%
\pgfsetroundjoin%
\definecolor{currentfill}{rgb}{0.000000,0.000000,0.000000}%
\pgfsetfillcolor{currentfill}%
\pgfsetlinewidth{0.501875pt}%
\definecolor{currentstroke}{rgb}{0.000000,0.000000,0.000000}%
\pgfsetstrokecolor{currentstroke}%
\pgfsetdash{}{0pt}%
\pgfsys@defobject{currentmarker}{\pgfqpoint{0.000000in}{-0.020833in}}{\pgfqpoint{0.000000in}{0.000000in}}{%
\pgfpathmoveto{\pgfqpoint{0.000000in}{0.000000in}}%
\pgfpathlineto{\pgfqpoint{0.000000in}{-0.020833in}}%
\pgfusepath{stroke,fill}%
}%
\begin{pgfscope}%
\pgfsys@transformshift{1.653460in}{3.227753in}%
\pgfsys@useobject{currentmarker}{}%
\end{pgfscope}%
\end{pgfscope}%
\begin{pgfscope}%
\pgfpathrectangle{\pgfqpoint{0.481681in}{1.080890in}}{\pgfqpoint{5.785672in}{2.146863in}}%
\pgfusepath{clip}%
\pgfsetrectcap%
\pgfsetroundjoin%
\pgfsetlinewidth{0.100375pt}%
\definecolor{currentstroke}{rgb}{0.827451,0.827451,0.827451}%
\pgfsetstrokecolor{currentstroke}%
\pgfsetdash{}{0pt}%
\pgfpathmoveto{\pgfqpoint{1.689157in}{1.080890in}}%
\pgfpathlineto{\pgfqpoint{1.689157in}{3.227753in}}%
\pgfusepath{stroke}%
\end{pgfscope}%
\begin{pgfscope}%
\pgfsetbuttcap%
\pgfsetroundjoin%
\definecolor{currentfill}{rgb}{0.000000,0.000000,0.000000}%
\pgfsetfillcolor{currentfill}%
\pgfsetlinewidth{0.501875pt}%
\definecolor{currentstroke}{rgb}{0.000000,0.000000,0.000000}%
\pgfsetstrokecolor{currentstroke}%
\pgfsetdash{}{0pt}%
\pgfsys@defobject{currentmarker}{\pgfqpoint{0.000000in}{0.000000in}}{\pgfqpoint{0.000000in}{0.020833in}}{%
\pgfpathmoveto{\pgfqpoint{0.000000in}{0.000000in}}%
\pgfpathlineto{\pgfqpoint{0.000000in}{0.020833in}}%
\pgfusepath{stroke,fill}%
}%
\begin{pgfscope}%
\pgfsys@transformshift{1.689157in}{1.080890in}%
\pgfsys@useobject{currentmarker}{}%
\end{pgfscope}%
\end{pgfscope}%
\begin{pgfscope}%
\pgfsetbuttcap%
\pgfsetroundjoin%
\definecolor{currentfill}{rgb}{0.000000,0.000000,0.000000}%
\pgfsetfillcolor{currentfill}%
\pgfsetlinewidth{0.501875pt}%
\definecolor{currentstroke}{rgb}{0.000000,0.000000,0.000000}%
\pgfsetstrokecolor{currentstroke}%
\pgfsetdash{}{0pt}%
\pgfsys@defobject{currentmarker}{\pgfqpoint{0.000000in}{-0.020833in}}{\pgfqpoint{0.000000in}{0.000000in}}{%
\pgfpathmoveto{\pgfqpoint{0.000000in}{0.000000in}}%
\pgfpathlineto{\pgfqpoint{0.000000in}{-0.020833in}}%
\pgfusepath{stroke,fill}%
}%
\begin{pgfscope}%
\pgfsys@transformshift{1.689157in}{3.227753in}%
\pgfsys@useobject{currentmarker}{}%
\end{pgfscope}%
\end{pgfscope}%
\begin{pgfscope}%
\pgfpathrectangle{\pgfqpoint{0.481681in}{1.080890in}}{\pgfqpoint{5.785672in}{2.146863in}}%
\pgfusepath{clip}%
\pgfsetrectcap%
\pgfsetroundjoin%
\pgfsetlinewidth{0.100375pt}%
\definecolor{currentstroke}{rgb}{0.827451,0.827451,0.827451}%
\pgfsetstrokecolor{currentstroke}%
\pgfsetdash{}{0pt}%
\pgfpathmoveto{\pgfqpoint{1.724855in}{1.080890in}}%
\pgfpathlineto{\pgfqpoint{1.724855in}{3.227753in}}%
\pgfusepath{stroke}%
\end{pgfscope}%
\begin{pgfscope}%
\pgfsetbuttcap%
\pgfsetroundjoin%
\definecolor{currentfill}{rgb}{0.000000,0.000000,0.000000}%
\pgfsetfillcolor{currentfill}%
\pgfsetlinewidth{0.501875pt}%
\definecolor{currentstroke}{rgb}{0.000000,0.000000,0.000000}%
\pgfsetstrokecolor{currentstroke}%
\pgfsetdash{}{0pt}%
\pgfsys@defobject{currentmarker}{\pgfqpoint{0.000000in}{0.000000in}}{\pgfqpoint{0.000000in}{0.020833in}}{%
\pgfpathmoveto{\pgfqpoint{0.000000in}{0.000000in}}%
\pgfpathlineto{\pgfqpoint{0.000000in}{0.020833in}}%
\pgfusepath{stroke,fill}%
}%
\begin{pgfscope}%
\pgfsys@transformshift{1.724855in}{1.080890in}%
\pgfsys@useobject{currentmarker}{}%
\end{pgfscope}%
\end{pgfscope}%
\begin{pgfscope}%
\pgfsetbuttcap%
\pgfsetroundjoin%
\definecolor{currentfill}{rgb}{0.000000,0.000000,0.000000}%
\pgfsetfillcolor{currentfill}%
\pgfsetlinewidth{0.501875pt}%
\definecolor{currentstroke}{rgb}{0.000000,0.000000,0.000000}%
\pgfsetstrokecolor{currentstroke}%
\pgfsetdash{}{0pt}%
\pgfsys@defobject{currentmarker}{\pgfqpoint{0.000000in}{-0.020833in}}{\pgfqpoint{0.000000in}{0.000000in}}{%
\pgfpathmoveto{\pgfqpoint{0.000000in}{0.000000in}}%
\pgfpathlineto{\pgfqpoint{0.000000in}{-0.020833in}}%
\pgfusepath{stroke,fill}%
}%
\begin{pgfscope}%
\pgfsys@transformshift{1.724855in}{3.227753in}%
\pgfsys@useobject{currentmarker}{}%
\end{pgfscope}%
\end{pgfscope}%
\begin{pgfscope}%
\pgfpathrectangle{\pgfqpoint{0.481681in}{1.080890in}}{\pgfqpoint{5.785672in}{2.146863in}}%
\pgfusepath{clip}%
\pgfsetrectcap%
\pgfsetroundjoin%
\pgfsetlinewidth{0.100375pt}%
\definecolor{currentstroke}{rgb}{0.827451,0.827451,0.827451}%
\pgfsetstrokecolor{currentstroke}%
\pgfsetdash{}{0pt}%
\pgfpathmoveto{\pgfqpoint{1.760552in}{1.080890in}}%
\pgfpathlineto{\pgfqpoint{1.760552in}{3.227753in}}%
\pgfusepath{stroke}%
\end{pgfscope}%
\begin{pgfscope}%
\pgfsetbuttcap%
\pgfsetroundjoin%
\definecolor{currentfill}{rgb}{0.000000,0.000000,0.000000}%
\pgfsetfillcolor{currentfill}%
\pgfsetlinewidth{0.501875pt}%
\definecolor{currentstroke}{rgb}{0.000000,0.000000,0.000000}%
\pgfsetstrokecolor{currentstroke}%
\pgfsetdash{}{0pt}%
\pgfsys@defobject{currentmarker}{\pgfqpoint{0.000000in}{0.000000in}}{\pgfqpoint{0.000000in}{0.020833in}}{%
\pgfpathmoveto{\pgfqpoint{0.000000in}{0.000000in}}%
\pgfpathlineto{\pgfqpoint{0.000000in}{0.020833in}}%
\pgfusepath{stroke,fill}%
}%
\begin{pgfscope}%
\pgfsys@transformshift{1.760552in}{1.080890in}%
\pgfsys@useobject{currentmarker}{}%
\end{pgfscope}%
\end{pgfscope}%
\begin{pgfscope}%
\pgfsetbuttcap%
\pgfsetroundjoin%
\definecolor{currentfill}{rgb}{0.000000,0.000000,0.000000}%
\pgfsetfillcolor{currentfill}%
\pgfsetlinewidth{0.501875pt}%
\definecolor{currentstroke}{rgb}{0.000000,0.000000,0.000000}%
\pgfsetstrokecolor{currentstroke}%
\pgfsetdash{}{0pt}%
\pgfsys@defobject{currentmarker}{\pgfqpoint{0.000000in}{-0.020833in}}{\pgfqpoint{0.000000in}{0.000000in}}{%
\pgfpathmoveto{\pgfqpoint{0.000000in}{0.000000in}}%
\pgfpathlineto{\pgfqpoint{0.000000in}{-0.020833in}}%
\pgfusepath{stroke,fill}%
}%
\begin{pgfscope}%
\pgfsys@transformshift{1.760552in}{3.227753in}%
\pgfsys@useobject{currentmarker}{}%
\end{pgfscope}%
\end{pgfscope}%
\begin{pgfscope}%
\pgfpathrectangle{\pgfqpoint{0.481681in}{1.080890in}}{\pgfqpoint{5.785672in}{2.146863in}}%
\pgfusepath{clip}%
\pgfsetrectcap%
\pgfsetroundjoin%
\pgfsetlinewidth{0.100375pt}%
\definecolor{currentstroke}{rgb}{0.827451,0.827451,0.827451}%
\pgfsetstrokecolor{currentstroke}%
\pgfsetdash{}{0pt}%
\pgfpathmoveto{\pgfqpoint{1.796249in}{1.080890in}}%
\pgfpathlineto{\pgfqpoint{1.796249in}{3.227753in}}%
\pgfusepath{stroke}%
\end{pgfscope}%
\begin{pgfscope}%
\pgfsetbuttcap%
\pgfsetroundjoin%
\definecolor{currentfill}{rgb}{0.000000,0.000000,0.000000}%
\pgfsetfillcolor{currentfill}%
\pgfsetlinewidth{0.501875pt}%
\definecolor{currentstroke}{rgb}{0.000000,0.000000,0.000000}%
\pgfsetstrokecolor{currentstroke}%
\pgfsetdash{}{0pt}%
\pgfsys@defobject{currentmarker}{\pgfqpoint{0.000000in}{0.000000in}}{\pgfqpoint{0.000000in}{0.020833in}}{%
\pgfpathmoveto{\pgfqpoint{0.000000in}{0.000000in}}%
\pgfpathlineto{\pgfqpoint{0.000000in}{0.020833in}}%
\pgfusepath{stroke,fill}%
}%
\begin{pgfscope}%
\pgfsys@transformshift{1.796249in}{1.080890in}%
\pgfsys@useobject{currentmarker}{}%
\end{pgfscope}%
\end{pgfscope}%
\begin{pgfscope}%
\pgfsetbuttcap%
\pgfsetroundjoin%
\definecolor{currentfill}{rgb}{0.000000,0.000000,0.000000}%
\pgfsetfillcolor{currentfill}%
\pgfsetlinewidth{0.501875pt}%
\definecolor{currentstroke}{rgb}{0.000000,0.000000,0.000000}%
\pgfsetstrokecolor{currentstroke}%
\pgfsetdash{}{0pt}%
\pgfsys@defobject{currentmarker}{\pgfqpoint{0.000000in}{-0.020833in}}{\pgfqpoint{0.000000in}{0.000000in}}{%
\pgfpathmoveto{\pgfqpoint{0.000000in}{0.000000in}}%
\pgfpathlineto{\pgfqpoint{0.000000in}{-0.020833in}}%
\pgfusepath{stroke,fill}%
}%
\begin{pgfscope}%
\pgfsys@transformshift{1.796249in}{3.227753in}%
\pgfsys@useobject{currentmarker}{}%
\end{pgfscope}%
\end{pgfscope}%
\begin{pgfscope}%
\pgfpathrectangle{\pgfqpoint{0.481681in}{1.080890in}}{\pgfqpoint{5.785672in}{2.146863in}}%
\pgfusepath{clip}%
\pgfsetrectcap%
\pgfsetroundjoin%
\pgfsetlinewidth{0.100375pt}%
\definecolor{currentstroke}{rgb}{0.827451,0.827451,0.827451}%
\pgfsetstrokecolor{currentstroke}%
\pgfsetdash{}{0pt}%
\pgfpathmoveto{\pgfqpoint{1.831947in}{1.080890in}}%
\pgfpathlineto{\pgfqpoint{1.831947in}{3.227753in}}%
\pgfusepath{stroke}%
\end{pgfscope}%
\begin{pgfscope}%
\pgfsetbuttcap%
\pgfsetroundjoin%
\definecolor{currentfill}{rgb}{0.000000,0.000000,0.000000}%
\pgfsetfillcolor{currentfill}%
\pgfsetlinewidth{0.501875pt}%
\definecolor{currentstroke}{rgb}{0.000000,0.000000,0.000000}%
\pgfsetstrokecolor{currentstroke}%
\pgfsetdash{}{0pt}%
\pgfsys@defobject{currentmarker}{\pgfqpoint{0.000000in}{0.000000in}}{\pgfqpoint{0.000000in}{0.020833in}}{%
\pgfpathmoveto{\pgfqpoint{0.000000in}{0.000000in}}%
\pgfpathlineto{\pgfqpoint{0.000000in}{0.020833in}}%
\pgfusepath{stroke,fill}%
}%
\begin{pgfscope}%
\pgfsys@transformshift{1.831947in}{1.080890in}%
\pgfsys@useobject{currentmarker}{}%
\end{pgfscope}%
\end{pgfscope}%
\begin{pgfscope}%
\pgfsetbuttcap%
\pgfsetroundjoin%
\definecolor{currentfill}{rgb}{0.000000,0.000000,0.000000}%
\pgfsetfillcolor{currentfill}%
\pgfsetlinewidth{0.501875pt}%
\definecolor{currentstroke}{rgb}{0.000000,0.000000,0.000000}%
\pgfsetstrokecolor{currentstroke}%
\pgfsetdash{}{0pt}%
\pgfsys@defobject{currentmarker}{\pgfqpoint{0.000000in}{-0.020833in}}{\pgfqpoint{0.000000in}{0.000000in}}{%
\pgfpathmoveto{\pgfqpoint{0.000000in}{0.000000in}}%
\pgfpathlineto{\pgfqpoint{0.000000in}{-0.020833in}}%
\pgfusepath{stroke,fill}%
}%
\begin{pgfscope}%
\pgfsys@transformshift{1.831947in}{3.227753in}%
\pgfsys@useobject{currentmarker}{}%
\end{pgfscope}%
\end{pgfscope}%
\begin{pgfscope}%
\pgfpathrectangle{\pgfqpoint{0.481681in}{1.080890in}}{\pgfqpoint{5.785672in}{2.146863in}}%
\pgfusepath{clip}%
\pgfsetrectcap%
\pgfsetroundjoin%
\pgfsetlinewidth{0.100375pt}%
\definecolor{currentstroke}{rgb}{0.827451,0.827451,0.827451}%
\pgfsetstrokecolor{currentstroke}%
\pgfsetdash{}{0pt}%
\pgfpathmoveto{\pgfqpoint{1.867644in}{1.080890in}}%
\pgfpathlineto{\pgfqpoint{1.867644in}{3.227753in}}%
\pgfusepath{stroke}%
\end{pgfscope}%
\begin{pgfscope}%
\pgfsetbuttcap%
\pgfsetroundjoin%
\definecolor{currentfill}{rgb}{0.000000,0.000000,0.000000}%
\pgfsetfillcolor{currentfill}%
\pgfsetlinewidth{0.501875pt}%
\definecolor{currentstroke}{rgb}{0.000000,0.000000,0.000000}%
\pgfsetstrokecolor{currentstroke}%
\pgfsetdash{}{0pt}%
\pgfsys@defobject{currentmarker}{\pgfqpoint{0.000000in}{0.000000in}}{\pgfqpoint{0.000000in}{0.020833in}}{%
\pgfpathmoveto{\pgfqpoint{0.000000in}{0.000000in}}%
\pgfpathlineto{\pgfqpoint{0.000000in}{0.020833in}}%
\pgfusepath{stroke,fill}%
}%
\begin{pgfscope}%
\pgfsys@transformshift{1.867644in}{1.080890in}%
\pgfsys@useobject{currentmarker}{}%
\end{pgfscope}%
\end{pgfscope}%
\begin{pgfscope}%
\pgfsetbuttcap%
\pgfsetroundjoin%
\definecolor{currentfill}{rgb}{0.000000,0.000000,0.000000}%
\pgfsetfillcolor{currentfill}%
\pgfsetlinewidth{0.501875pt}%
\definecolor{currentstroke}{rgb}{0.000000,0.000000,0.000000}%
\pgfsetstrokecolor{currentstroke}%
\pgfsetdash{}{0pt}%
\pgfsys@defobject{currentmarker}{\pgfqpoint{0.000000in}{-0.020833in}}{\pgfqpoint{0.000000in}{0.000000in}}{%
\pgfpathmoveto{\pgfqpoint{0.000000in}{0.000000in}}%
\pgfpathlineto{\pgfqpoint{0.000000in}{-0.020833in}}%
\pgfusepath{stroke,fill}%
}%
\begin{pgfscope}%
\pgfsys@transformshift{1.867644in}{3.227753in}%
\pgfsys@useobject{currentmarker}{}%
\end{pgfscope}%
\end{pgfscope}%
\begin{pgfscope}%
\pgfpathrectangle{\pgfqpoint{0.481681in}{1.080890in}}{\pgfqpoint{5.785672in}{2.146863in}}%
\pgfusepath{clip}%
\pgfsetrectcap%
\pgfsetroundjoin%
\pgfsetlinewidth{0.100375pt}%
\definecolor{currentstroke}{rgb}{0.827451,0.827451,0.827451}%
\pgfsetstrokecolor{currentstroke}%
\pgfsetdash{}{0pt}%
\pgfpathmoveto{\pgfqpoint{1.903341in}{1.080890in}}%
\pgfpathlineto{\pgfqpoint{1.903341in}{3.227753in}}%
\pgfusepath{stroke}%
\end{pgfscope}%
\begin{pgfscope}%
\pgfsetbuttcap%
\pgfsetroundjoin%
\definecolor{currentfill}{rgb}{0.000000,0.000000,0.000000}%
\pgfsetfillcolor{currentfill}%
\pgfsetlinewidth{0.501875pt}%
\definecolor{currentstroke}{rgb}{0.000000,0.000000,0.000000}%
\pgfsetstrokecolor{currentstroke}%
\pgfsetdash{}{0pt}%
\pgfsys@defobject{currentmarker}{\pgfqpoint{0.000000in}{0.000000in}}{\pgfqpoint{0.000000in}{0.020833in}}{%
\pgfpathmoveto{\pgfqpoint{0.000000in}{0.000000in}}%
\pgfpathlineto{\pgfqpoint{0.000000in}{0.020833in}}%
\pgfusepath{stroke,fill}%
}%
\begin{pgfscope}%
\pgfsys@transformshift{1.903341in}{1.080890in}%
\pgfsys@useobject{currentmarker}{}%
\end{pgfscope}%
\end{pgfscope}%
\begin{pgfscope}%
\pgfsetbuttcap%
\pgfsetroundjoin%
\definecolor{currentfill}{rgb}{0.000000,0.000000,0.000000}%
\pgfsetfillcolor{currentfill}%
\pgfsetlinewidth{0.501875pt}%
\definecolor{currentstroke}{rgb}{0.000000,0.000000,0.000000}%
\pgfsetstrokecolor{currentstroke}%
\pgfsetdash{}{0pt}%
\pgfsys@defobject{currentmarker}{\pgfqpoint{0.000000in}{-0.020833in}}{\pgfqpoint{0.000000in}{0.000000in}}{%
\pgfpathmoveto{\pgfqpoint{0.000000in}{0.000000in}}%
\pgfpathlineto{\pgfqpoint{0.000000in}{-0.020833in}}%
\pgfusepath{stroke,fill}%
}%
\begin{pgfscope}%
\pgfsys@transformshift{1.903341in}{3.227753in}%
\pgfsys@useobject{currentmarker}{}%
\end{pgfscope}%
\end{pgfscope}%
\begin{pgfscope}%
\pgfpathrectangle{\pgfqpoint{0.481681in}{1.080890in}}{\pgfqpoint{5.785672in}{2.146863in}}%
\pgfusepath{clip}%
\pgfsetrectcap%
\pgfsetroundjoin%
\pgfsetlinewidth{0.100375pt}%
\definecolor{currentstroke}{rgb}{0.827451,0.827451,0.827451}%
\pgfsetstrokecolor{currentstroke}%
\pgfsetdash{}{0pt}%
\pgfpathmoveto{\pgfqpoint{1.939039in}{1.080890in}}%
\pgfpathlineto{\pgfqpoint{1.939039in}{3.227753in}}%
\pgfusepath{stroke}%
\end{pgfscope}%
\begin{pgfscope}%
\pgfsetbuttcap%
\pgfsetroundjoin%
\definecolor{currentfill}{rgb}{0.000000,0.000000,0.000000}%
\pgfsetfillcolor{currentfill}%
\pgfsetlinewidth{0.501875pt}%
\definecolor{currentstroke}{rgb}{0.000000,0.000000,0.000000}%
\pgfsetstrokecolor{currentstroke}%
\pgfsetdash{}{0pt}%
\pgfsys@defobject{currentmarker}{\pgfqpoint{0.000000in}{0.000000in}}{\pgfqpoint{0.000000in}{0.020833in}}{%
\pgfpathmoveto{\pgfqpoint{0.000000in}{0.000000in}}%
\pgfpathlineto{\pgfqpoint{0.000000in}{0.020833in}}%
\pgfusepath{stroke,fill}%
}%
\begin{pgfscope}%
\pgfsys@transformshift{1.939039in}{1.080890in}%
\pgfsys@useobject{currentmarker}{}%
\end{pgfscope}%
\end{pgfscope}%
\begin{pgfscope}%
\pgfsetbuttcap%
\pgfsetroundjoin%
\definecolor{currentfill}{rgb}{0.000000,0.000000,0.000000}%
\pgfsetfillcolor{currentfill}%
\pgfsetlinewidth{0.501875pt}%
\definecolor{currentstroke}{rgb}{0.000000,0.000000,0.000000}%
\pgfsetstrokecolor{currentstroke}%
\pgfsetdash{}{0pt}%
\pgfsys@defobject{currentmarker}{\pgfqpoint{0.000000in}{-0.020833in}}{\pgfqpoint{0.000000in}{0.000000in}}{%
\pgfpathmoveto{\pgfqpoint{0.000000in}{0.000000in}}%
\pgfpathlineto{\pgfqpoint{0.000000in}{-0.020833in}}%
\pgfusepath{stroke,fill}%
}%
\begin{pgfscope}%
\pgfsys@transformshift{1.939039in}{3.227753in}%
\pgfsys@useobject{currentmarker}{}%
\end{pgfscope}%
\end{pgfscope}%
\begin{pgfscope}%
\pgfpathrectangle{\pgfqpoint{0.481681in}{1.080890in}}{\pgfqpoint{5.785672in}{2.146863in}}%
\pgfusepath{clip}%
\pgfsetrectcap%
\pgfsetroundjoin%
\pgfsetlinewidth{0.100375pt}%
\definecolor{currentstroke}{rgb}{0.827451,0.827451,0.827451}%
\pgfsetstrokecolor{currentstroke}%
\pgfsetdash{}{0pt}%
\pgfpathmoveto{\pgfqpoint{2.010433in}{1.080890in}}%
\pgfpathlineto{\pgfqpoint{2.010433in}{3.227753in}}%
\pgfusepath{stroke}%
\end{pgfscope}%
\begin{pgfscope}%
\pgfsetbuttcap%
\pgfsetroundjoin%
\definecolor{currentfill}{rgb}{0.000000,0.000000,0.000000}%
\pgfsetfillcolor{currentfill}%
\pgfsetlinewidth{0.501875pt}%
\definecolor{currentstroke}{rgb}{0.000000,0.000000,0.000000}%
\pgfsetstrokecolor{currentstroke}%
\pgfsetdash{}{0pt}%
\pgfsys@defobject{currentmarker}{\pgfqpoint{0.000000in}{0.000000in}}{\pgfqpoint{0.000000in}{0.020833in}}{%
\pgfpathmoveto{\pgfqpoint{0.000000in}{0.000000in}}%
\pgfpathlineto{\pgfqpoint{0.000000in}{0.020833in}}%
\pgfusepath{stroke,fill}%
}%
\begin{pgfscope}%
\pgfsys@transformshift{2.010433in}{1.080890in}%
\pgfsys@useobject{currentmarker}{}%
\end{pgfscope}%
\end{pgfscope}%
\begin{pgfscope}%
\pgfsetbuttcap%
\pgfsetroundjoin%
\definecolor{currentfill}{rgb}{0.000000,0.000000,0.000000}%
\pgfsetfillcolor{currentfill}%
\pgfsetlinewidth{0.501875pt}%
\definecolor{currentstroke}{rgb}{0.000000,0.000000,0.000000}%
\pgfsetstrokecolor{currentstroke}%
\pgfsetdash{}{0pt}%
\pgfsys@defobject{currentmarker}{\pgfqpoint{0.000000in}{-0.020833in}}{\pgfqpoint{0.000000in}{0.000000in}}{%
\pgfpathmoveto{\pgfqpoint{0.000000in}{0.000000in}}%
\pgfpathlineto{\pgfqpoint{0.000000in}{-0.020833in}}%
\pgfusepath{stroke,fill}%
}%
\begin{pgfscope}%
\pgfsys@transformshift{2.010433in}{3.227753in}%
\pgfsys@useobject{currentmarker}{}%
\end{pgfscope}%
\end{pgfscope}%
\begin{pgfscope}%
\pgfpathrectangle{\pgfqpoint{0.481681in}{1.080890in}}{\pgfqpoint{5.785672in}{2.146863in}}%
\pgfusepath{clip}%
\pgfsetrectcap%
\pgfsetroundjoin%
\pgfsetlinewidth{0.100375pt}%
\definecolor{currentstroke}{rgb}{0.827451,0.827451,0.827451}%
\pgfsetstrokecolor{currentstroke}%
\pgfsetdash{}{0pt}%
\pgfpathmoveto{\pgfqpoint{2.046131in}{1.080890in}}%
\pgfpathlineto{\pgfqpoint{2.046131in}{3.227753in}}%
\pgfusepath{stroke}%
\end{pgfscope}%
\begin{pgfscope}%
\pgfsetbuttcap%
\pgfsetroundjoin%
\definecolor{currentfill}{rgb}{0.000000,0.000000,0.000000}%
\pgfsetfillcolor{currentfill}%
\pgfsetlinewidth{0.501875pt}%
\definecolor{currentstroke}{rgb}{0.000000,0.000000,0.000000}%
\pgfsetstrokecolor{currentstroke}%
\pgfsetdash{}{0pt}%
\pgfsys@defobject{currentmarker}{\pgfqpoint{0.000000in}{0.000000in}}{\pgfqpoint{0.000000in}{0.020833in}}{%
\pgfpathmoveto{\pgfqpoint{0.000000in}{0.000000in}}%
\pgfpathlineto{\pgfqpoint{0.000000in}{0.020833in}}%
\pgfusepath{stroke,fill}%
}%
\begin{pgfscope}%
\pgfsys@transformshift{2.046131in}{1.080890in}%
\pgfsys@useobject{currentmarker}{}%
\end{pgfscope}%
\end{pgfscope}%
\begin{pgfscope}%
\pgfsetbuttcap%
\pgfsetroundjoin%
\definecolor{currentfill}{rgb}{0.000000,0.000000,0.000000}%
\pgfsetfillcolor{currentfill}%
\pgfsetlinewidth{0.501875pt}%
\definecolor{currentstroke}{rgb}{0.000000,0.000000,0.000000}%
\pgfsetstrokecolor{currentstroke}%
\pgfsetdash{}{0pt}%
\pgfsys@defobject{currentmarker}{\pgfqpoint{0.000000in}{-0.020833in}}{\pgfqpoint{0.000000in}{0.000000in}}{%
\pgfpathmoveto{\pgfqpoint{0.000000in}{0.000000in}}%
\pgfpathlineto{\pgfqpoint{0.000000in}{-0.020833in}}%
\pgfusepath{stroke,fill}%
}%
\begin{pgfscope}%
\pgfsys@transformshift{2.046131in}{3.227753in}%
\pgfsys@useobject{currentmarker}{}%
\end{pgfscope}%
\end{pgfscope}%
\begin{pgfscope}%
\pgfpathrectangle{\pgfqpoint{0.481681in}{1.080890in}}{\pgfqpoint{5.785672in}{2.146863in}}%
\pgfusepath{clip}%
\pgfsetrectcap%
\pgfsetroundjoin%
\pgfsetlinewidth{0.100375pt}%
\definecolor{currentstroke}{rgb}{0.827451,0.827451,0.827451}%
\pgfsetstrokecolor{currentstroke}%
\pgfsetdash{}{0pt}%
\pgfpathmoveto{\pgfqpoint{2.081828in}{1.080890in}}%
\pgfpathlineto{\pgfqpoint{2.081828in}{3.227753in}}%
\pgfusepath{stroke}%
\end{pgfscope}%
\begin{pgfscope}%
\pgfsetbuttcap%
\pgfsetroundjoin%
\definecolor{currentfill}{rgb}{0.000000,0.000000,0.000000}%
\pgfsetfillcolor{currentfill}%
\pgfsetlinewidth{0.501875pt}%
\definecolor{currentstroke}{rgb}{0.000000,0.000000,0.000000}%
\pgfsetstrokecolor{currentstroke}%
\pgfsetdash{}{0pt}%
\pgfsys@defobject{currentmarker}{\pgfqpoint{0.000000in}{0.000000in}}{\pgfqpoint{0.000000in}{0.020833in}}{%
\pgfpathmoveto{\pgfqpoint{0.000000in}{0.000000in}}%
\pgfpathlineto{\pgfqpoint{0.000000in}{0.020833in}}%
\pgfusepath{stroke,fill}%
}%
\begin{pgfscope}%
\pgfsys@transformshift{2.081828in}{1.080890in}%
\pgfsys@useobject{currentmarker}{}%
\end{pgfscope}%
\end{pgfscope}%
\begin{pgfscope}%
\pgfsetbuttcap%
\pgfsetroundjoin%
\definecolor{currentfill}{rgb}{0.000000,0.000000,0.000000}%
\pgfsetfillcolor{currentfill}%
\pgfsetlinewidth{0.501875pt}%
\definecolor{currentstroke}{rgb}{0.000000,0.000000,0.000000}%
\pgfsetstrokecolor{currentstroke}%
\pgfsetdash{}{0pt}%
\pgfsys@defobject{currentmarker}{\pgfqpoint{0.000000in}{-0.020833in}}{\pgfqpoint{0.000000in}{0.000000in}}{%
\pgfpathmoveto{\pgfqpoint{0.000000in}{0.000000in}}%
\pgfpathlineto{\pgfqpoint{0.000000in}{-0.020833in}}%
\pgfusepath{stroke,fill}%
}%
\begin{pgfscope}%
\pgfsys@transformshift{2.081828in}{3.227753in}%
\pgfsys@useobject{currentmarker}{}%
\end{pgfscope}%
\end{pgfscope}%
\begin{pgfscope}%
\pgfpathrectangle{\pgfqpoint{0.481681in}{1.080890in}}{\pgfqpoint{5.785672in}{2.146863in}}%
\pgfusepath{clip}%
\pgfsetrectcap%
\pgfsetroundjoin%
\pgfsetlinewidth{0.100375pt}%
\definecolor{currentstroke}{rgb}{0.827451,0.827451,0.827451}%
\pgfsetstrokecolor{currentstroke}%
\pgfsetdash{}{0pt}%
\pgfpathmoveto{\pgfqpoint{2.117525in}{1.080890in}}%
\pgfpathlineto{\pgfqpoint{2.117525in}{3.227753in}}%
\pgfusepath{stroke}%
\end{pgfscope}%
\begin{pgfscope}%
\pgfsetbuttcap%
\pgfsetroundjoin%
\definecolor{currentfill}{rgb}{0.000000,0.000000,0.000000}%
\pgfsetfillcolor{currentfill}%
\pgfsetlinewidth{0.501875pt}%
\definecolor{currentstroke}{rgb}{0.000000,0.000000,0.000000}%
\pgfsetstrokecolor{currentstroke}%
\pgfsetdash{}{0pt}%
\pgfsys@defobject{currentmarker}{\pgfqpoint{0.000000in}{0.000000in}}{\pgfqpoint{0.000000in}{0.020833in}}{%
\pgfpathmoveto{\pgfqpoint{0.000000in}{0.000000in}}%
\pgfpathlineto{\pgfqpoint{0.000000in}{0.020833in}}%
\pgfusepath{stroke,fill}%
}%
\begin{pgfscope}%
\pgfsys@transformshift{2.117525in}{1.080890in}%
\pgfsys@useobject{currentmarker}{}%
\end{pgfscope}%
\end{pgfscope}%
\begin{pgfscope}%
\pgfsetbuttcap%
\pgfsetroundjoin%
\definecolor{currentfill}{rgb}{0.000000,0.000000,0.000000}%
\pgfsetfillcolor{currentfill}%
\pgfsetlinewidth{0.501875pt}%
\definecolor{currentstroke}{rgb}{0.000000,0.000000,0.000000}%
\pgfsetstrokecolor{currentstroke}%
\pgfsetdash{}{0pt}%
\pgfsys@defobject{currentmarker}{\pgfqpoint{0.000000in}{-0.020833in}}{\pgfqpoint{0.000000in}{0.000000in}}{%
\pgfpathmoveto{\pgfqpoint{0.000000in}{0.000000in}}%
\pgfpathlineto{\pgfqpoint{0.000000in}{-0.020833in}}%
\pgfusepath{stroke,fill}%
}%
\begin{pgfscope}%
\pgfsys@transformshift{2.117525in}{3.227753in}%
\pgfsys@useobject{currentmarker}{}%
\end{pgfscope}%
\end{pgfscope}%
\begin{pgfscope}%
\pgfpathrectangle{\pgfqpoint{0.481681in}{1.080890in}}{\pgfqpoint{5.785672in}{2.146863in}}%
\pgfusepath{clip}%
\pgfsetrectcap%
\pgfsetroundjoin%
\pgfsetlinewidth{0.100375pt}%
\definecolor{currentstroke}{rgb}{0.827451,0.827451,0.827451}%
\pgfsetstrokecolor{currentstroke}%
\pgfsetdash{}{0pt}%
\pgfpathmoveto{\pgfqpoint{2.153223in}{1.080890in}}%
\pgfpathlineto{\pgfqpoint{2.153223in}{3.227753in}}%
\pgfusepath{stroke}%
\end{pgfscope}%
\begin{pgfscope}%
\pgfsetbuttcap%
\pgfsetroundjoin%
\definecolor{currentfill}{rgb}{0.000000,0.000000,0.000000}%
\pgfsetfillcolor{currentfill}%
\pgfsetlinewidth{0.501875pt}%
\definecolor{currentstroke}{rgb}{0.000000,0.000000,0.000000}%
\pgfsetstrokecolor{currentstroke}%
\pgfsetdash{}{0pt}%
\pgfsys@defobject{currentmarker}{\pgfqpoint{0.000000in}{0.000000in}}{\pgfqpoint{0.000000in}{0.020833in}}{%
\pgfpathmoveto{\pgfqpoint{0.000000in}{0.000000in}}%
\pgfpathlineto{\pgfqpoint{0.000000in}{0.020833in}}%
\pgfusepath{stroke,fill}%
}%
\begin{pgfscope}%
\pgfsys@transformshift{2.153223in}{1.080890in}%
\pgfsys@useobject{currentmarker}{}%
\end{pgfscope}%
\end{pgfscope}%
\begin{pgfscope}%
\pgfsetbuttcap%
\pgfsetroundjoin%
\definecolor{currentfill}{rgb}{0.000000,0.000000,0.000000}%
\pgfsetfillcolor{currentfill}%
\pgfsetlinewidth{0.501875pt}%
\definecolor{currentstroke}{rgb}{0.000000,0.000000,0.000000}%
\pgfsetstrokecolor{currentstroke}%
\pgfsetdash{}{0pt}%
\pgfsys@defobject{currentmarker}{\pgfqpoint{0.000000in}{-0.020833in}}{\pgfqpoint{0.000000in}{0.000000in}}{%
\pgfpathmoveto{\pgfqpoint{0.000000in}{0.000000in}}%
\pgfpathlineto{\pgfqpoint{0.000000in}{-0.020833in}}%
\pgfusepath{stroke,fill}%
}%
\begin{pgfscope}%
\pgfsys@transformshift{2.153223in}{3.227753in}%
\pgfsys@useobject{currentmarker}{}%
\end{pgfscope}%
\end{pgfscope}%
\begin{pgfscope}%
\pgfpathrectangle{\pgfqpoint{0.481681in}{1.080890in}}{\pgfqpoint{5.785672in}{2.146863in}}%
\pgfusepath{clip}%
\pgfsetrectcap%
\pgfsetroundjoin%
\pgfsetlinewidth{0.100375pt}%
\definecolor{currentstroke}{rgb}{0.827451,0.827451,0.827451}%
\pgfsetstrokecolor{currentstroke}%
\pgfsetdash{}{0pt}%
\pgfpathmoveto{\pgfqpoint{2.188920in}{1.080890in}}%
\pgfpathlineto{\pgfqpoint{2.188920in}{3.227753in}}%
\pgfusepath{stroke}%
\end{pgfscope}%
\begin{pgfscope}%
\pgfsetbuttcap%
\pgfsetroundjoin%
\definecolor{currentfill}{rgb}{0.000000,0.000000,0.000000}%
\pgfsetfillcolor{currentfill}%
\pgfsetlinewidth{0.501875pt}%
\definecolor{currentstroke}{rgb}{0.000000,0.000000,0.000000}%
\pgfsetstrokecolor{currentstroke}%
\pgfsetdash{}{0pt}%
\pgfsys@defobject{currentmarker}{\pgfqpoint{0.000000in}{0.000000in}}{\pgfqpoint{0.000000in}{0.020833in}}{%
\pgfpathmoveto{\pgfqpoint{0.000000in}{0.000000in}}%
\pgfpathlineto{\pgfqpoint{0.000000in}{0.020833in}}%
\pgfusepath{stroke,fill}%
}%
\begin{pgfscope}%
\pgfsys@transformshift{2.188920in}{1.080890in}%
\pgfsys@useobject{currentmarker}{}%
\end{pgfscope}%
\end{pgfscope}%
\begin{pgfscope}%
\pgfsetbuttcap%
\pgfsetroundjoin%
\definecolor{currentfill}{rgb}{0.000000,0.000000,0.000000}%
\pgfsetfillcolor{currentfill}%
\pgfsetlinewidth{0.501875pt}%
\definecolor{currentstroke}{rgb}{0.000000,0.000000,0.000000}%
\pgfsetstrokecolor{currentstroke}%
\pgfsetdash{}{0pt}%
\pgfsys@defobject{currentmarker}{\pgfqpoint{0.000000in}{-0.020833in}}{\pgfqpoint{0.000000in}{0.000000in}}{%
\pgfpathmoveto{\pgfqpoint{0.000000in}{0.000000in}}%
\pgfpathlineto{\pgfqpoint{0.000000in}{-0.020833in}}%
\pgfusepath{stroke,fill}%
}%
\begin{pgfscope}%
\pgfsys@transformshift{2.188920in}{3.227753in}%
\pgfsys@useobject{currentmarker}{}%
\end{pgfscope}%
\end{pgfscope}%
\begin{pgfscope}%
\pgfpathrectangle{\pgfqpoint{0.481681in}{1.080890in}}{\pgfqpoint{5.785672in}{2.146863in}}%
\pgfusepath{clip}%
\pgfsetrectcap%
\pgfsetroundjoin%
\pgfsetlinewidth{0.100375pt}%
\definecolor{currentstroke}{rgb}{0.827451,0.827451,0.827451}%
\pgfsetstrokecolor{currentstroke}%
\pgfsetdash{}{0pt}%
\pgfpathmoveto{\pgfqpoint{2.224617in}{1.080890in}}%
\pgfpathlineto{\pgfqpoint{2.224617in}{3.227753in}}%
\pgfusepath{stroke}%
\end{pgfscope}%
\begin{pgfscope}%
\pgfsetbuttcap%
\pgfsetroundjoin%
\definecolor{currentfill}{rgb}{0.000000,0.000000,0.000000}%
\pgfsetfillcolor{currentfill}%
\pgfsetlinewidth{0.501875pt}%
\definecolor{currentstroke}{rgb}{0.000000,0.000000,0.000000}%
\pgfsetstrokecolor{currentstroke}%
\pgfsetdash{}{0pt}%
\pgfsys@defobject{currentmarker}{\pgfqpoint{0.000000in}{0.000000in}}{\pgfqpoint{0.000000in}{0.020833in}}{%
\pgfpathmoveto{\pgfqpoint{0.000000in}{0.000000in}}%
\pgfpathlineto{\pgfqpoint{0.000000in}{0.020833in}}%
\pgfusepath{stroke,fill}%
}%
\begin{pgfscope}%
\pgfsys@transformshift{2.224617in}{1.080890in}%
\pgfsys@useobject{currentmarker}{}%
\end{pgfscope}%
\end{pgfscope}%
\begin{pgfscope}%
\pgfsetbuttcap%
\pgfsetroundjoin%
\definecolor{currentfill}{rgb}{0.000000,0.000000,0.000000}%
\pgfsetfillcolor{currentfill}%
\pgfsetlinewidth{0.501875pt}%
\definecolor{currentstroke}{rgb}{0.000000,0.000000,0.000000}%
\pgfsetstrokecolor{currentstroke}%
\pgfsetdash{}{0pt}%
\pgfsys@defobject{currentmarker}{\pgfqpoint{0.000000in}{-0.020833in}}{\pgfqpoint{0.000000in}{0.000000in}}{%
\pgfpathmoveto{\pgfqpoint{0.000000in}{0.000000in}}%
\pgfpathlineto{\pgfqpoint{0.000000in}{-0.020833in}}%
\pgfusepath{stroke,fill}%
}%
\begin{pgfscope}%
\pgfsys@transformshift{2.224617in}{3.227753in}%
\pgfsys@useobject{currentmarker}{}%
\end{pgfscope}%
\end{pgfscope}%
\begin{pgfscope}%
\pgfpathrectangle{\pgfqpoint{0.481681in}{1.080890in}}{\pgfqpoint{5.785672in}{2.146863in}}%
\pgfusepath{clip}%
\pgfsetrectcap%
\pgfsetroundjoin%
\pgfsetlinewidth{0.100375pt}%
\definecolor{currentstroke}{rgb}{0.827451,0.827451,0.827451}%
\pgfsetstrokecolor{currentstroke}%
\pgfsetdash{}{0pt}%
\pgfpathmoveto{\pgfqpoint{2.260314in}{1.080890in}}%
\pgfpathlineto{\pgfqpoint{2.260314in}{3.227753in}}%
\pgfusepath{stroke}%
\end{pgfscope}%
\begin{pgfscope}%
\pgfsetbuttcap%
\pgfsetroundjoin%
\definecolor{currentfill}{rgb}{0.000000,0.000000,0.000000}%
\pgfsetfillcolor{currentfill}%
\pgfsetlinewidth{0.501875pt}%
\definecolor{currentstroke}{rgb}{0.000000,0.000000,0.000000}%
\pgfsetstrokecolor{currentstroke}%
\pgfsetdash{}{0pt}%
\pgfsys@defobject{currentmarker}{\pgfqpoint{0.000000in}{0.000000in}}{\pgfqpoint{0.000000in}{0.020833in}}{%
\pgfpathmoveto{\pgfqpoint{0.000000in}{0.000000in}}%
\pgfpathlineto{\pgfqpoint{0.000000in}{0.020833in}}%
\pgfusepath{stroke,fill}%
}%
\begin{pgfscope}%
\pgfsys@transformshift{2.260314in}{1.080890in}%
\pgfsys@useobject{currentmarker}{}%
\end{pgfscope}%
\end{pgfscope}%
\begin{pgfscope}%
\pgfsetbuttcap%
\pgfsetroundjoin%
\definecolor{currentfill}{rgb}{0.000000,0.000000,0.000000}%
\pgfsetfillcolor{currentfill}%
\pgfsetlinewidth{0.501875pt}%
\definecolor{currentstroke}{rgb}{0.000000,0.000000,0.000000}%
\pgfsetstrokecolor{currentstroke}%
\pgfsetdash{}{0pt}%
\pgfsys@defobject{currentmarker}{\pgfqpoint{0.000000in}{-0.020833in}}{\pgfqpoint{0.000000in}{0.000000in}}{%
\pgfpathmoveto{\pgfqpoint{0.000000in}{0.000000in}}%
\pgfpathlineto{\pgfqpoint{0.000000in}{-0.020833in}}%
\pgfusepath{stroke,fill}%
}%
\begin{pgfscope}%
\pgfsys@transformshift{2.260314in}{3.227753in}%
\pgfsys@useobject{currentmarker}{}%
\end{pgfscope}%
\end{pgfscope}%
\begin{pgfscope}%
\pgfpathrectangle{\pgfqpoint{0.481681in}{1.080890in}}{\pgfqpoint{5.785672in}{2.146863in}}%
\pgfusepath{clip}%
\pgfsetrectcap%
\pgfsetroundjoin%
\pgfsetlinewidth{0.100375pt}%
\definecolor{currentstroke}{rgb}{0.827451,0.827451,0.827451}%
\pgfsetstrokecolor{currentstroke}%
\pgfsetdash{}{0pt}%
\pgfpathmoveto{\pgfqpoint{2.296012in}{1.080890in}}%
\pgfpathlineto{\pgfqpoint{2.296012in}{3.227753in}}%
\pgfusepath{stroke}%
\end{pgfscope}%
\begin{pgfscope}%
\pgfsetbuttcap%
\pgfsetroundjoin%
\definecolor{currentfill}{rgb}{0.000000,0.000000,0.000000}%
\pgfsetfillcolor{currentfill}%
\pgfsetlinewidth{0.501875pt}%
\definecolor{currentstroke}{rgb}{0.000000,0.000000,0.000000}%
\pgfsetstrokecolor{currentstroke}%
\pgfsetdash{}{0pt}%
\pgfsys@defobject{currentmarker}{\pgfqpoint{0.000000in}{0.000000in}}{\pgfqpoint{0.000000in}{0.020833in}}{%
\pgfpathmoveto{\pgfqpoint{0.000000in}{0.000000in}}%
\pgfpathlineto{\pgfqpoint{0.000000in}{0.020833in}}%
\pgfusepath{stroke,fill}%
}%
\begin{pgfscope}%
\pgfsys@transformshift{2.296012in}{1.080890in}%
\pgfsys@useobject{currentmarker}{}%
\end{pgfscope}%
\end{pgfscope}%
\begin{pgfscope}%
\pgfsetbuttcap%
\pgfsetroundjoin%
\definecolor{currentfill}{rgb}{0.000000,0.000000,0.000000}%
\pgfsetfillcolor{currentfill}%
\pgfsetlinewidth{0.501875pt}%
\definecolor{currentstroke}{rgb}{0.000000,0.000000,0.000000}%
\pgfsetstrokecolor{currentstroke}%
\pgfsetdash{}{0pt}%
\pgfsys@defobject{currentmarker}{\pgfqpoint{0.000000in}{-0.020833in}}{\pgfqpoint{0.000000in}{0.000000in}}{%
\pgfpathmoveto{\pgfqpoint{0.000000in}{0.000000in}}%
\pgfpathlineto{\pgfqpoint{0.000000in}{-0.020833in}}%
\pgfusepath{stroke,fill}%
}%
\begin{pgfscope}%
\pgfsys@transformshift{2.296012in}{3.227753in}%
\pgfsys@useobject{currentmarker}{}%
\end{pgfscope}%
\end{pgfscope}%
\begin{pgfscope}%
\pgfpathrectangle{\pgfqpoint{0.481681in}{1.080890in}}{\pgfqpoint{5.785672in}{2.146863in}}%
\pgfusepath{clip}%
\pgfsetrectcap%
\pgfsetroundjoin%
\pgfsetlinewidth{0.100375pt}%
\definecolor{currentstroke}{rgb}{0.827451,0.827451,0.827451}%
\pgfsetstrokecolor{currentstroke}%
\pgfsetdash{}{0pt}%
\pgfpathmoveto{\pgfqpoint{2.331709in}{1.080890in}}%
\pgfpathlineto{\pgfqpoint{2.331709in}{3.227753in}}%
\pgfusepath{stroke}%
\end{pgfscope}%
\begin{pgfscope}%
\pgfsetbuttcap%
\pgfsetroundjoin%
\definecolor{currentfill}{rgb}{0.000000,0.000000,0.000000}%
\pgfsetfillcolor{currentfill}%
\pgfsetlinewidth{0.501875pt}%
\definecolor{currentstroke}{rgb}{0.000000,0.000000,0.000000}%
\pgfsetstrokecolor{currentstroke}%
\pgfsetdash{}{0pt}%
\pgfsys@defobject{currentmarker}{\pgfqpoint{0.000000in}{0.000000in}}{\pgfqpoint{0.000000in}{0.020833in}}{%
\pgfpathmoveto{\pgfqpoint{0.000000in}{0.000000in}}%
\pgfpathlineto{\pgfqpoint{0.000000in}{0.020833in}}%
\pgfusepath{stroke,fill}%
}%
\begin{pgfscope}%
\pgfsys@transformshift{2.331709in}{1.080890in}%
\pgfsys@useobject{currentmarker}{}%
\end{pgfscope}%
\end{pgfscope}%
\begin{pgfscope}%
\pgfsetbuttcap%
\pgfsetroundjoin%
\definecolor{currentfill}{rgb}{0.000000,0.000000,0.000000}%
\pgfsetfillcolor{currentfill}%
\pgfsetlinewidth{0.501875pt}%
\definecolor{currentstroke}{rgb}{0.000000,0.000000,0.000000}%
\pgfsetstrokecolor{currentstroke}%
\pgfsetdash{}{0pt}%
\pgfsys@defobject{currentmarker}{\pgfqpoint{0.000000in}{-0.020833in}}{\pgfqpoint{0.000000in}{0.000000in}}{%
\pgfpathmoveto{\pgfqpoint{0.000000in}{0.000000in}}%
\pgfpathlineto{\pgfqpoint{0.000000in}{-0.020833in}}%
\pgfusepath{stroke,fill}%
}%
\begin{pgfscope}%
\pgfsys@transformshift{2.331709in}{3.227753in}%
\pgfsys@useobject{currentmarker}{}%
\end{pgfscope}%
\end{pgfscope}%
\begin{pgfscope}%
\pgfpathrectangle{\pgfqpoint{0.481681in}{1.080890in}}{\pgfqpoint{5.785672in}{2.146863in}}%
\pgfusepath{clip}%
\pgfsetrectcap%
\pgfsetroundjoin%
\pgfsetlinewidth{0.100375pt}%
\definecolor{currentstroke}{rgb}{0.827451,0.827451,0.827451}%
\pgfsetstrokecolor{currentstroke}%
\pgfsetdash{}{0pt}%
\pgfpathmoveto{\pgfqpoint{2.367406in}{1.080890in}}%
\pgfpathlineto{\pgfqpoint{2.367406in}{3.227753in}}%
\pgfusepath{stroke}%
\end{pgfscope}%
\begin{pgfscope}%
\pgfsetbuttcap%
\pgfsetroundjoin%
\definecolor{currentfill}{rgb}{0.000000,0.000000,0.000000}%
\pgfsetfillcolor{currentfill}%
\pgfsetlinewidth{0.501875pt}%
\definecolor{currentstroke}{rgb}{0.000000,0.000000,0.000000}%
\pgfsetstrokecolor{currentstroke}%
\pgfsetdash{}{0pt}%
\pgfsys@defobject{currentmarker}{\pgfqpoint{0.000000in}{0.000000in}}{\pgfqpoint{0.000000in}{0.020833in}}{%
\pgfpathmoveto{\pgfqpoint{0.000000in}{0.000000in}}%
\pgfpathlineto{\pgfqpoint{0.000000in}{0.020833in}}%
\pgfusepath{stroke,fill}%
}%
\begin{pgfscope}%
\pgfsys@transformshift{2.367406in}{1.080890in}%
\pgfsys@useobject{currentmarker}{}%
\end{pgfscope}%
\end{pgfscope}%
\begin{pgfscope}%
\pgfsetbuttcap%
\pgfsetroundjoin%
\definecolor{currentfill}{rgb}{0.000000,0.000000,0.000000}%
\pgfsetfillcolor{currentfill}%
\pgfsetlinewidth{0.501875pt}%
\definecolor{currentstroke}{rgb}{0.000000,0.000000,0.000000}%
\pgfsetstrokecolor{currentstroke}%
\pgfsetdash{}{0pt}%
\pgfsys@defobject{currentmarker}{\pgfqpoint{0.000000in}{-0.020833in}}{\pgfqpoint{0.000000in}{0.000000in}}{%
\pgfpathmoveto{\pgfqpoint{0.000000in}{0.000000in}}%
\pgfpathlineto{\pgfqpoint{0.000000in}{-0.020833in}}%
\pgfusepath{stroke,fill}%
}%
\begin{pgfscope}%
\pgfsys@transformshift{2.367406in}{3.227753in}%
\pgfsys@useobject{currentmarker}{}%
\end{pgfscope}%
\end{pgfscope}%
\begin{pgfscope}%
\pgfpathrectangle{\pgfqpoint{0.481681in}{1.080890in}}{\pgfqpoint{5.785672in}{2.146863in}}%
\pgfusepath{clip}%
\pgfsetrectcap%
\pgfsetroundjoin%
\pgfsetlinewidth{0.100375pt}%
\definecolor{currentstroke}{rgb}{0.827451,0.827451,0.827451}%
\pgfsetstrokecolor{currentstroke}%
\pgfsetdash{}{0pt}%
\pgfpathmoveto{\pgfqpoint{2.438801in}{1.080890in}}%
\pgfpathlineto{\pgfqpoint{2.438801in}{3.227753in}}%
\pgfusepath{stroke}%
\end{pgfscope}%
\begin{pgfscope}%
\pgfsetbuttcap%
\pgfsetroundjoin%
\definecolor{currentfill}{rgb}{0.000000,0.000000,0.000000}%
\pgfsetfillcolor{currentfill}%
\pgfsetlinewidth{0.501875pt}%
\definecolor{currentstroke}{rgb}{0.000000,0.000000,0.000000}%
\pgfsetstrokecolor{currentstroke}%
\pgfsetdash{}{0pt}%
\pgfsys@defobject{currentmarker}{\pgfqpoint{0.000000in}{0.000000in}}{\pgfqpoint{0.000000in}{0.020833in}}{%
\pgfpathmoveto{\pgfqpoint{0.000000in}{0.000000in}}%
\pgfpathlineto{\pgfqpoint{0.000000in}{0.020833in}}%
\pgfusepath{stroke,fill}%
}%
\begin{pgfscope}%
\pgfsys@transformshift{2.438801in}{1.080890in}%
\pgfsys@useobject{currentmarker}{}%
\end{pgfscope}%
\end{pgfscope}%
\begin{pgfscope}%
\pgfsetbuttcap%
\pgfsetroundjoin%
\definecolor{currentfill}{rgb}{0.000000,0.000000,0.000000}%
\pgfsetfillcolor{currentfill}%
\pgfsetlinewidth{0.501875pt}%
\definecolor{currentstroke}{rgb}{0.000000,0.000000,0.000000}%
\pgfsetstrokecolor{currentstroke}%
\pgfsetdash{}{0pt}%
\pgfsys@defobject{currentmarker}{\pgfqpoint{0.000000in}{-0.020833in}}{\pgfqpoint{0.000000in}{0.000000in}}{%
\pgfpathmoveto{\pgfqpoint{0.000000in}{0.000000in}}%
\pgfpathlineto{\pgfqpoint{0.000000in}{-0.020833in}}%
\pgfusepath{stroke,fill}%
}%
\begin{pgfscope}%
\pgfsys@transformshift{2.438801in}{3.227753in}%
\pgfsys@useobject{currentmarker}{}%
\end{pgfscope}%
\end{pgfscope}%
\begin{pgfscope}%
\pgfpathrectangle{\pgfqpoint{0.481681in}{1.080890in}}{\pgfqpoint{5.785672in}{2.146863in}}%
\pgfusepath{clip}%
\pgfsetrectcap%
\pgfsetroundjoin%
\pgfsetlinewidth{0.100375pt}%
\definecolor{currentstroke}{rgb}{0.827451,0.827451,0.827451}%
\pgfsetstrokecolor{currentstroke}%
\pgfsetdash{}{0pt}%
\pgfpathmoveto{\pgfqpoint{2.474498in}{1.080890in}}%
\pgfpathlineto{\pgfqpoint{2.474498in}{3.227753in}}%
\pgfusepath{stroke}%
\end{pgfscope}%
\begin{pgfscope}%
\pgfsetbuttcap%
\pgfsetroundjoin%
\definecolor{currentfill}{rgb}{0.000000,0.000000,0.000000}%
\pgfsetfillcolor{currentfill}%
\pgfsetlinewidth{0.501875pt}%
\definecolor{currentstroke}{rgb}{0.000000,0.000000,0.000000}%
\pgfsetstrokecolor{currentstroke}%
\pgfsetdash{}{0pt}%
\pgfsys@defobject{currentmarker}{\pgfqpoint{0.000000in}{0.000000in}}{\pgfqpoint{0.000000in}{0.020833in}}{%
\pgfpathmoveto{\pgfqpoint{0.000000in}{0.000000in}}%
\pgfpathlineto{\pgfqpoint{0.000000in}{0.020833in}}%
\pgfusepath{stroke,fill}%
}%
\begin{pgfscope}%
\pgfsys@transformshift{2.474498in}{1.080890in}%
\pgfsys@useobject{currentmarker}{}%
\end{pgfscope}%
\end{pgfscope}%
\begin{pgfscope}%
\pgfsetbuttcap%
\pgfsetroundjoin%
\definecolor{currentfill}{rgb}{0.000000,0.000000,0.000000}%
\pgfsetfillcolor{currentfill}%
\pgfsetlinewidth{0.501875pt}%
\definecolor{currentstroke}{rgb}{0.000000,0.000000,0.000000}%
\pgfsetstrokecolor{currentstroke}%
\pgfsetdash{}{0pt}%
\pgfsys@defobject{currentmarker}{\pgfqpoint{0.000000in}{-0.020833in}}{\pgfqpoint{0.000000in}{0.000000in}}{%
\pgfpathmoveto{\pgfqpoint{0.000000in}{0.000000in}}%
\pgfpathlineto{\pgfqpoint{0.000000in}{-0.020833in}}%
\pgfusepath{stroke,fill}%
}%
\begin{pgfscope}%
\pgfsys@transformshift{2.474498in}{3.227753in}%
\pgfsys@useobject{currentmarker}{}%
\end{pgfscope}%
\end{pgfscope}%
\begin{pgfscope}%
\pgfpathrectangle{\pgfqpoint{0.481681in}{1.080890in}}{\pgfqpoint{5.785672in}{2.146863in}}%
\pgfusepath{clip}%
\pgfsetrectcap%
\pgfsetroundjoin%
\pgfsetlinewidth{0.100375pt}%
\definecolor{currentstroke}{rgb}{0.827451,0.827451,0.827451}%
\pgfsetstrokecolor{currentstroke}%
\pgfsetdash{}{0pt}%
\pgfpathmoveto{\pgfqpoint{2.510196in}{1.080890in}}%
\pgfpathlineto{\pgfqpoint{2.510196in}{3.227753in}}%
\pgfusepath{stroke}%
\end{pgfscope}%
\begin{pgfscope}%
\pgfsetbuttcap%
\pgfsetroundjoin%
\definecolor{currentfill}{rgb}{0.000000,0.000000,0.000000}%
\pgfsetfillcolor{currentfill}%
\pgfsetlinewidth{0.501875pt}%
\definecolor{currentstroke}{rgb}{0.000000,0.000000,0.000000}%
\pgfsetstrokecolor{currentstroke}%
\pgfsetdash{}{0pt}%
\pgfsys@defobject{currentmarker}{\pgfqpoint{0.000000in}{0.000000in}}{\pgfqpoint{0.000000in}{0.020833in}}{%
\pgfpathmoveto{\pgfqpoint{0.000000in}{0.000000in}}%
\pgfpathlineto{\pgfqpoint{0.000000in}{0.020833in}}%
\pgfusepath{stroke,fill}%
}%
\begin{pgfscope}%
\pgfsys@transformshift{2.510196in}{1.080890in}%
\pgfsys@useobject{currentmarker}{}%
\end{pgfscope}%
\end{pgfscope}%
\begin{pgfscope}%
\pgfsetbuttcap%
\pgfsetroundjoin%
\definecolor{currentfill}{rgb}{0.000000,0.000000,0.000000}%
\pgfsetfillcolor{currentfill}%
\pgfsetlinewidth{0.501875pt}%
\definecolor{currentstroke}{rgb}{0.000000,0.000000,0.000000}%
\pgfsetstrokecolor{currentstroke}%
\pgfsetdash{}{0pt}%
\pgfsys@defobject{currentmarker}{\pgfqpoint{0.000000in}{-0.020833in}}{\pgfqpoint{0.000000in}{0.000000in}}{%
\pgfpathmoveto{\pgfqpoint{0.000000in}{0.000000in}}%
\pgfpathlineto{\pgfqpoint{0.000000in}{-0.020833in}}%
\pgfusepath{stroke,fill}%
}%
\begin{pgfscope}%
\pgfsys@transformshift{2.510196in}{3.227753in}%
\pgfsys@useobject{currentmarker}{}%
\end{pgfscope}%
\end{pgfscope}%
\begin{pgfscope}%
\pgfpathrectangle{\pgfqpoint{0.481681in}{1.080890in}}{\pgfqpoint{5.785672in}{2.146863in}}%
\pgfusepath{clip}%
\pgfsetrectcap%
\pgfsetroundjoin%
\pgfsetlinewidth{0.100375pt}%
\definecolor{currentstroke}{rgb}{0.827451,0.827451,0.827451}%
\pgfsetstrokecolor{currentstroke}%
\pgfsetdash{}{0pt}%
\pgfpathmoveto{\pgfqpoint{2.545893in}{1.080890in}}%
\pgfpathlineto{\pgfqpoint{2.545893in}{3.227753in}}%
\pgfusepath{stroke}%
\end{pgfscope}%
\begin{pgfscope}%
\pgfsetbuttcap%
\pgfsetroundjoin%
\definecolor{currentfill}{rgb}{0.000000,0.000000,0.000000}%
\pgfsetfillcolor{currentfill}%
\pgfsetlinewidth{0.501875pt}%
\definecolor{currentstroke}{rgb}{0.000000,0.000000,0.000000}%
\pgfsetstrokecolor{currentstroke}%
\pgfsetdash{}{0pt}%
\pgfsys@defobject{currentmarker}{\pgfqpoint{0.000000in}{0.000000in}}{\pgfqpoint{0.000000in}{0.020833in}}{%
\pgfpathmoveto{\pgfqpoint{0.000000in}{0.000000in}}%
\pgfpathlineto{\pgfqpoint{0.000000in}{0.020833in}}%
\pgfusepath{stroke,fill}%
}%
\begin{pgfscope}%
\pgfsys@transformshift{2.545893in}{1.080890in}%
\pgfsys@useobject{currentmarker}{}%
\end{pgfscope}%
\end{pgfscope}%
\begin{pgfscope}%
\pgfsetbuttcap%
\pgfsetroundjoin%
\definecolor{currentfill}{rgb}{0.000000,0.000000,0.000000}%
\pgfsetfillcolor{currentfill}%
\pgfsetlinewidth{0.501875pt}%
\definecolor{currentstroke}{rgb}{0.000000,0.000000,0.000000}%
\pgfsetstrokecolor{currentstroke}%
\pgfsetdash{}{0pt}%
\pgfsys@defobject{currentmarker}{\pgfqpoint{0.000000in}{-0.020833in}}{\pgfqpoint{0.000000in}{0.000000in}}{%
\pgfpathmoveto{\pgfqpoint{0.000000in}{0.000000in}}%
\pgfpathlineto{\pgfqpoint{0.000000in}{-0.020833in}}%
\pgfusepath{stroke,fill}%
}%
\begin{pgfscope}%
\pgfsys@transformshift{2.545893in}{3.227753in}%
\pgfsys@useobject{currentmarker}{}%
\end{pgfscope}%
\end{pgfscope}%
\begin{pgfscope}%
\pgfpathrectangle{\pgfqpoint{0.481681in}{1.080890in}}{\pgfqpoint{5.785672in}{2.146863in}}%
\pgfusepath{clip}%
\pgfsetrectcap%
\pgfsetroundjoin%
\pgfsetlinewidth{0.100375pt}%
\definecolor{currentstroke}{rgb}{0.827451,0.827451,0.827451}%
\pgfsetstrokecolor{currentstroke}%
\pgfsetdash{}{0pt}%
\pgfpathmoveto{\pgfqpoint{2.581590in}{1.080890in}}%
\pgfpathlineto{\pgfqpoint{2.581590in}{3.227753in}}%
\pgfusepath{stroke}%
\end{pgfscope}%
\begin{pgfscope}%
\pgfsetbuttcap%
\pgfsetroundjoin%
\definecolor{currentfill}{rgb}{0.000000,0.000000,0.000000}%
\pgfsetfillcolor{currentfill}%
\pgfsetlinewidth{0.501875pt}%
\definecolor{currentstroke}{rgb}{0.000000,0.000000,0.000000}%
\pgfsetstrokecolor{currentstroke}%
\pgfsetdash{}{0pt}%
\pgfsys@defobject{currentmarker}{\pgfqpoint{0.000000in}{0.000000in}}{\pgfqpoint{0.000000in}{0.020833in}}{%
\pgfpathmoveto{\pgfqpoint{0.000000in}{0.000000in}}%
\pgfpathlineto{\pgfqpoint{0.000000in}{0.020833in}}%
\pgfusepath{stroke,fill}%
}%
\begin{pgfscope}%
\pgfsys@transformshift{2.581590in}{1.080890in}%
\pgfsys@useobject{currentmarker}{}%
\end{pgfscope}%
\end{pgfscope}%
\begin{pgfscope}%
\pgfsetbuttcap%
\pgfsetroundjoin%
\definecolor{currentfill}{rgb}{0.000000,0.000000,0.000000}%
\pgfsetfillcolor{currentfill}%
\pgfsetlinewidth{0.501875pt}%
\definecolor{currentstroke}{rgb}{0.000000,0.000000,0.000000}%
\pgfsetstrokecolor{currentstroke}%
\pgfsetdash{}{0pt}%
\pgfsys@defobject{currentmarker}{\pgfqpoint{0.000000in}{-0.020833in}}{\pgfqpoint{0.000000in}{0.000000in}}{%
\pgfpathmoveto{\pgfqpoint{0.000000in}{0.000000in}}%
\pgfpathlineto{\pgfqpoint{0.000000in}{-0.020833in}}%
\pgfusepath{stroke,fill}%
}%
\begin{pgfscope}%
\pgfsys@transformshift{2.581590in}{3.227753in}%
\pgfsys@useobject{currentmarker}{}%
\end{pgfscope}%
\end{pgfscope}%
\begin{pgfscope}%
\pgfpathrectangle{\pgfqpoint{0.481681in}{1.080890in}}{\pgfqpoint{5.785672in}{2.146863in}}%
\pgfusepath{clip}%
\pgfsetrectcap%
\pgfsetroundjoin%
\pgfsetlinewidth{0.100375pt}%
\definecolor{currentstroke}{rgb}{0.827451,0.827451,0.827451}%
\pgfsetstrokecolor{currentstroke}%
\pgfsetdash{}{0pt}%
\pgfpathmoveto{\pgfqpoint{2.617288in}{1.080890in}}%
\pgfpathlineto{\pgfqpoint{2.617288in}{3.227753in}}%
\pgfusepath{stroke}%
\end{pgfscope}%
\begin{pgfscope}%
\pgfsetbuttcap%
\pgfsetroundjoin%
\definecolor{currentfill}{rgb}{0.000000,0.000000,0.000000}%
\pgfsetfillcolor{currentfill}%
\pgfsetlinewidth{0.501875pt}%
\definecolor{currentstroke}{rgb}{0.000000,0.000000,0.000000}%
\pgfsetstrokecolor{currentstroke}%
\pgfsetdash{}{0pt}%
\pgfsys@defobject{currentmarker}{\pgfqpoint{0.000000in}{0.000000in}}{\pgfqpoint{0.000000in}{0.020833in}}{%
\pgfpathmoveto{\pgfqpoint{0.000000in}{0.000000in}}%
\pgfpathlineto{\pgfqpoint{0.000000in}{0.020833in}}%
\pgfusepath{stroke,fill}%
}%
\begin{pgfscope}%
\pgfsys@transformshift{2.617288in}{1.080890in}%
\pgfsys@useobject{currentmarker}{}%
\end{pgfscope}%
\end{pgfscope}%
\begin{pgfscope}%
\pgfsetbuttcap%
\pgfsetroundjoin%
\definecolor{currentfill}{rgb}{0.000000,0.000000,0.000000}%
\pgfsetfillcolor{currentfill}%
\pgfsetlinewidth{0.501875pt}%
\definecolor{currentstroke}{rgb}{0.000000,0.000000,0.000000}%
\pgfsetstrokecolor{currentstroke}%
\pgfsetdash{}{0pt}%
\pgfsys@defobject{currentmarker}{\pgfqpoint{0.000000in}{-0.020833in}}{\pgfqpoint{0.000000in}{0.000000in}}{%
\pgfpathmoveto{\pgfqpoint{0.000000in}{0.000000in}}%
\pgfpathlineto{\pgfqpoint{0.000000in}{-0.020833in}}%
\pgfusepath{stroke,fill}%
}%
\begin{pgfscope}%
\pgfsys@transformshift{2.617288in}{3.227753in}%
\pgfsys@useobject{currentmarker}{}%
\end{pgfscope}%
\end{pgfscope}%
\begin{pgfscope}%
\pgfpathrectangle{\pgfqpoint{0.481681in}{1.080890in}}{\pgfqpoint{5.785672in}{2.146863in}}%
\pgfusepath{clip}%
\pgfsetrectcap%
\pgfsetroundjoin%
\pgfsetlinewidth{0.100375pt}%
\definecolor{currentstroke}{rgb}{0.827451,0.827451,0.827451}%
\pgfsetstrokecolor{currentstroke}%
\pgfsetdash{}{0pt}%
\pgfpathmoveto{\pgfqpoint{2.652985in}{1.080890in}}%
\pgfpathlineto{\pgfqpoint{2.652985in}{3.227753in}}%
\pgfusepath{stroke}%
\end{pgfscope}%
\begin{pgfscope}%
\pgfsetbuttcap%
\pgfsetroundjoin%
\definecolor{currentfill}{rgb}{0.000000,0.000000,0.000000}%
\pgfsetfillcolor{currentfill}%
\pgfsetlinewidth{0.501875pt}%
\definecolor{currentstroke}{rgb}{0.000000,0.000000,0.000000}%
\pgfsetstrokecolor{currentstroke}%
\pgfsetdash{}{0pt}%
\pgfsys@defobject{currentmarker}{\pgfqpoint{0.000000in}{0.000000in}}{\pgfqpoint{0.000000in}{0.020833in}}{%
\pgfpathmoveto{\pgfqpoint{0.000000in}{0.000000in}}%
\pgfpathlineto{\pgfqpoint{0.000000in}{0.020833in}}%
\pgfusepath{stroke,fill}%
}%
\begin{pgfscope}%
\pgfsys@transformshift{2.652985in}{1.080890in}%
\pgfsys@useobject{currentmarker}{}%
\end{pgfscope}%
\end{pgfscope}%
\begin{pgfscope}%
\pgfsetbuttcap%
\pgfsetroundjoin%
\definecolor{currentfill}{rgb}{0.000000,0.000000,0.000000}%
\pgfsetfillcolor{currentfill}%
\pgfsetlinewidth{0.501875pt}%
\definecolor{currentstroke}{rgb}{0.000000,0.000000,0.000000}%
\pgfsetstrokecolor{currentstroke}%
\pgfsetdash{}{0pt}%
\pgfsys@defobject{currentmarker}{\pgfqpoint{0.000000in}{-0.020833in}}{\pgfqpoint{0.000000in}{0.000000in}}{%
\pgfpathmoveto{\pgfqpoint{0.000000in}{0.000000in}}%
\pgfpathlineto{\pgfqpoint{0.000000in}{-0.020833in}}%
\pgfusepath{stroke,fill}%
}%
\begin{pgfscope}%
\pgfsys@transformshift{2.652985in}{3.227753in}%
\pgfsys@useobject{currentmarker}{}%
\end{pgfscope}%
\end{pgfscope}%
\begin{pgfscope}%
\pgfpathrectangle{\pgfqpoint{0.481681in}{1.080890in}}{\pgfqpoint{5.785672in}{2.146863in}}%
\pgfusepath{clip}%
\pgfsetrectcap%
\pgfsetroundjoin%
\pgfsetlinewidth{0.100375pt}%
\definecolor{currentstroke}{rgb}{0.827451,0.827451,0.827451}%
\pgfsetstrokecolor{currentstroke}%
\pgfsetdash{}{0pt}%
\pgfpathmoveto{\pgfqpoint{2.688682in}{1.080890in}}%
\pgfpathlineto{\pgfqpoint{2.688682in}{3.227753in}}%
\pgfusepath{stroke}%
\end{pgfscope}%
\begin{pgfscope}%
\pgfsetbuttcap%
\pgfsetroundjoin%
\definecolor{currentfill}{rgb}{0.000000,0.000000,0.000000}%
\pgfsetfillcolor{currentfill}%
\pgfsetlinewidth{0.501875pt}%
\definecolor{currentstroke}{rgb}{0.000000,0.000000,0.000000}%
\pgfsetstrokecolor{currentstroke}%
\pgfsetdash{}{0pt}%
\pgfsys@defobject{currentmarker}{\pgfqpoint{0.000000in}{0.000000in}}{\pgfqpoint{0.000000in}{0.020833in}}{%
\pgfpathmoveto{\pgfqpoint{0.000000in}{0.000000in}}%
\pgfpathlineto{\pgfqpoint{0.000000in}{0.020833in}}%
\pgfusepath{stroke,fill}%
}%
\begin{pgfscope}%
\pgfsys@transformshift{2.688682in}{1.080890in}%
\pgfsys@useobject{currentmarker}{}%
\end{pgfscope}%
\end{pgfscope}%
\begin{pgfscope}%
\pgfsetbuttcap%
\pgfsetroundjoin%
\definecolor{currentfill}{rgb}{0.000000,0.000000,0.000000}%
\pgfsetfillcolor{currentfill}%
\pgfsetlinewidth{0.501875pt}%
\definecolor{currentstroke}{rgb}{0.000000,0.000000,0.000000}%
\pgfsetstrokecolor{currentstroke}%
\pgfsetdash{}{0pt}%
\pgfsys@defobject{currentmarker}{\pgfqpoint{0.000000in}{-0.020833in}}{\pgfqpoint{0.000000in}{0.000000in}}{%
\pgfpathmoveto{\pgfqpoint{0.000000in}{0.000000in}}%
\pgfpathlineto{\pgfqpoint{0.000000in}{-0.020833in}}%
\pgfusepath{stroke,fill}%
}%
\begin{pgfscope}%
\pgfsys@transformshift{2.688682in}{3.227753in}%
\pgfsys@useobject{currentmarker}{}%
\end{pgfscope}%
\end{pgfscope}%
\begin{pgfscope}%
\pgfpathrectangle{\pgfqpoint{0.481681in}{1.080890in}}{\pgfqpoint{5.785672in}{2.146863in}}%
\pgfusepath{clip}%
\pgfsetrectcap%
\pgfsetroundjoin%
\pgfsetlinewidth{0.100375pt}%
\definecolor{currentstroke}{rgb}{0.827451,0.827451,0.827451}%
\pgfsetstrokecolor{currentstroke}%
\pgfsetdash{}{0pt}%
\pgfpathmoveto{\pgfqpoint{2.724380in}{1.080890in}}%
\pgfpathlineto{\pgfqpoint{2.724380in}{3.227753in}}%
\pgfusepath{stroke}%
\end{pgfscope}%
\begin{pgfscope}%
\pgfsetbuttcap%
\pgfsetroundjoin%
\definecolor{currentfill}{rgb}{0.000000,0.000000,0.000000}%
\pgfsetfillcolor{currentfill}%
\pgfsetlinewidth{0.501875pt}%
\definecolor{currentstroke}{rgb}{0.000000,0.000000,0.000000}%
\pgfsetstrokecolor{currentstroke}%
\pgfsetdash{}{0pt}%
\pgfsys@defobject{currentmarker}{\pgfqpoint{0.000000in}{0.000000in}}{\pgfqpoint{0.000000in}{0.020833in}}{%
\pgfpathmoveto{\pgfqpoint{0.000000in}{0.000000in}}%
\pgfpathlineto{\pgfqpoint{0.000000in}{0.020833in}}%
\pgfusepath{stroke,fill}%
}%
\begin{pgfscope}%
\pgfsys@transformshift{2.724380in}{1.080890in}%
\pgfsys@useobject{currentmarker}{}%
\end{pgfscope}%
\end{pgfscope}%
\begin{pgfscope}%
\pgfsetbuttcap%
\pgfsetroundjoin%
\definecolor{currentfill}{rgb}{0.000000,0.000000,0.000000}%
\pgfsetfillcolor{currentfill}%
\pgfsetlinewidth{0.501875pt}%
\definecolor{currentstroke}{rgb}{0.000000,0.000000,0.000000}%
\pgfsetstrokecolor{currentstroke}%
\pgfsetdash{}{0pt}%
\pgfsys@defobject{currentmarker}{\pgfqpoint{0.000000in}{-0.020833in}}{\pgfqpoint{0.000000in}{0.000000in}}{%
\pgfpathmoveto{\pgfqpoint{0.000000in}{0.000000in}}%
\pgfpathlineto{\pgfqpoint{0.000000in}{-0.020833in}}%
\pgfusepath{stroke,fill}%
}%
\begin{pgfscope}%
\pgfsys@transformshift{2.724380in}{3.227753in}%
\pgfsys@useobject{currentmarker}{}%
\end{pgfscope}%
\end{pgfscope}%
\begin{pgfscope}%
\pgfpathrectangle{\pgfqpoint{0.481681in}{1.080890in}}{\pgfqpoint{5.785672in}{2.146863in}}%
\pgfusepath{clip}%
\pgfsetrectcap%
\pgfsetroundjoin%
\pgfsetlinewidth{0.100375pt}%
\definecolor{currentstroke}{rgb}{0.827451,0.827451,0.827451}%
\pgfsetstrokecolor{currentstroke}%
\pgfsetdash{}{0pt}%
\pgfpathmoveto{\pgfqpoint{2.760077in}{1.080890in}}%
\pgfpathlineto{\pgfqpoint{2.760077in}{3.227753in}}%
\pgfusepath{stroke}%
\end{pgfscope}%
\begin{pgfscope}%
\pgfsetbuttcap%
\pgfsetroundjoin%
\definecolor{currentfill}{rgb}{0.000000,0.000000,0.000000}%
\pgfsetfillcolor{currentfill}%
\pgfsetlinewidth{0.501875pt}%
\definecolor{currentstroke}{rgb}{0.000000,0.000000,0.000000}%
\pgfsetstrokecolor{currentstroke}%
\pgfsetdash{}{0pt}%
\pgfsys@defobject{currentmarker}{\pgfqpoint{0.000000in}{0.000000in}}{\pgfqpoint{0.000000in}{0.020833in}}{%
\pgfpathmoveto{\pgfqpoint{0.000000in}{0.000000in}}%
\pgfpathlineto{\pgfqpoint{0.000000in}{0.020833in}}%
\pgfusepath{stroke,fill}%
}%
\begin{pgfscope}%
\pgfsys@transformshift{2.760077in}{1.080890in}%
\pgfsys@useobject{currentmarker}{}%
\end{pgfscope}%
\end{pgfscope}%
\begin{pgfscope}%
\pgfsetbuttcap%
\pgfsetroundjoin%
\definecolor{currentfill}{rgb}{0.000000,0.000000,0.000000}%
\pgfsetfillcolor{currentfill}%
\pgfsetlinewidth{0.501875pt}%
\definecolor{currentstroke}{rgb}{0.000000,0.000000,0.000000}%
\pgfsetstrokecolor{currentstroke}%
\pgfsetdash{}{0pt}%
\pgfsys@defobject{currentmarker}{\pgfqpoint{0.000000in}{-0.020833in}}{\pgfqpoint{0.000000in}{0.000000in}}{%
\pgfpathmoveto{\pgfqpoint{0.000000in}{0.000000in}}%
\pgfpathlineto{\pgfqpoint{0.000000in}{-0.020833in}}%
\pgfusepath{stroke,fill}%
}%
\begin{pgfscope}%
\pgfsys@transformshift{2.760077in}{3.227753in}%
\pgfsys@useobject{currentmarker}{}%
\end{pgfscope}%
\end{pgfscope}%
\begin{pgfscope}%
\pgfpathrectangle{\pgfqpoint{0.481681in}{1.080890in}}{\pgfqpoint{5.785672in}{2.146863in}}%
\pgfusepath{clip}%
\pgfsetrectcap%
\pgfsetroundjoin%
\pgfsetlinewidth{0.100375pt}%
\definecolor{currentstroke}{rgb}{0.827451,0.827451,0.827451}%
\pgfsetstrokecolor{currentstroke}%
\pgfsetdash{}{0pt}%
\pgfpathmoveto{\pgfqpoint{2.795774in}{1.080890in}}%
\pgfpathlineto{\pgfqpoint{2.795774in}{3.227753in}}%
\pgfusepath{stroke}%
\end{pgfscope}%
\begin{pgfscope}%
\pgfsetbuttcap%
\pgfsetroundjoin%
\definecolor{currentfill}{rgb}{0.000000,0.000000,0.000000}%
\pgfsetfillcolor{currentfill}%
\pgfsetlinewidth{0.501875pt}%
\definecolor{currentstroke}{rgb}{0.000000,0.000000,0.000000}%
\pgfsetstrokecolor{currentstroke}%
\pgfsetdash{}{0pt}%
\pgfsys@defobject{currentmarker}{\pgfqpoint{0.000000in}{0.000000in}}{\pgfqpoint{0.000000in}{0.020833in}}{%
\pgfpathmoveto{\pgfqpoint{0.000000in}{0.000000in}}%
\pgfpathlineto{\pgfqpoint{0.000000in}{0.020833in}}%
\pgfusepath{stroke,fill}%
}%
\begin{pgfscope}%
\pgfsys@transformshift{2.795774in}{1.080890in}%
\pgfsys@useobject{currentmarker}{}%
\end{pgfscope}%
\end{pgfscope}%
\begin{pgfscope}%
\pgfsetbuttcap%
\pgfsetroundjoin%
\definecolor{currentfill}{rgb}{0.000000,0.000000,0.000000}%
\pgfsetfillcolor{currentfill}%
\pgfsetlinewidth{0.501875pt}%
\definecolor{currentstroke}{rgb}{0.000000,0.000000,0.000000}%
\pgfsetstrokecolor{currentstroke}%
\pgfsetdash{}{0pt}%
\pgfsys@defobject{currentmarker}{\pgfqpoint{0.000000in}{-0.020833in}}{\pgfqpoint{0.000000in}{0.000000in}}{%
\pgfpathmoveto{\pgfqpoint{0.000000in}{0.000000in}}%
\pgfpathlineto{\pgfqpoint{0.000000in}{-0.020833in}}%
\pgfusepath{stroke,fill}%
}%
\begin{pgfscope}%
\pgfsys@transformshift{2.795774in}{3.227753in}%
\pgfsys@useobject{currentmarker}{}%
\end{pgfscope}%
\end{pgfscope}%
\begin{pgfscope}%
\pgfpathrectangle{\pgfqpoint{0.481681in}{1.080890in}}{\pgfqpoint{5.785672in}{2.146863in}}%
\pgfusepath{clip}%
\pgfsetrectcap%
\pgfsetroundjoin%
\pgfsetlinewidth{0.100375pt}%
\definecolor{currentstroke}{rgb}{0.827451,0.827451,0.827451}%
\pgfsetstrokecolor{currentstroke}%
\pgfsetdash{}{0pt}%
\pgfpathmoveto{\pgfqpoint{2.867169in}{1.080890in}}%
\pgfpathlineto{\pgfqpoint{2.867169in}{3.227753in}}%
\pgfusepath{stroke}%
\end{pgfscope}%
\begin{pgfscope}%
\pgfsetbuttcap%
\pgfsetroundjoin%
\definecolor{currentfill}{rgb}{0.000000,0.000000,0.000000}%
\pgfsetfillcolor{currentfill}%
\pgfsetlinewidth{0.501875pt}%
\definecolor{currentstroke}{rgb}{0.000000,0.000000,0.000000}%
\pgfsetstrokecolor{currentstroke}%
\pgfsetdash{}{0pt}%
\pgfsys@defobject{currentmarker}{\pgfqpoint{0.000000in}{0.000000in}}{\pgfqpoint{0.000000in}{0.020833in}}{%
\pgfpathmoveto{\pgfqpoint{0.000000in}{0.000000in}}%
\pgfpathlineto{\pgfqpoint{0.000000in}{0.020833in}}%
\pgfusepath{stroke,fill}%
}%
\begin{pgfscope}%
\pgfsys@transformshift{2.867169in}{1.080890in}%
\pgfsys@useobject{currentmarker}{}%
\end{pgfscope}%
\end{pgfscope}%
\begin{pgfscope}%
\pgfsetbuttcap%
\pgfsetroundjoin%
\definecolor{currentfill}{rgb}{0.000000,0.000000,0.000000}%
\pgfsetfillcolor{currentfill}%
\pgfsetlinewidth{0.501875pt}%
\definecolor{currentstroke}{rgb}{0.000000,0.000000,0.000000}%
\pgfsetstrokecolor{currentstroke}%
\pgfsetdash{}{0pt}%
\pgfsys@defobject{currentmarker}{\pgfqpoint{0.000000in}{-0.020833in}}{\pgfqpoint{0.000000in}{0.000000in}}{%
\pgfpathmoveto{\pgfqpoint{0.000000in}{0.000000in}}%
\pgfpathlineto{\pgfqpoint{0.000000in}{-0.020833in}}%
\pgfusepath{stroke,fill}%
}%
\begin{pgfscope}%
\pgfsys@transformshift{2.867169in}{3.227753in}%
\pgfsys@useobject{currentmarker}{}%
\end{pgfscope}%
\end{pgfscope}%
\begin{pgfscope}%
\pgfpathrectangle{\pgfqpoint{0.481681in}{1.080890in}}{\pgfqpoint{5.785672in}{2.146863in}}%
\pgfusepath{clip}%
\pgfsetrectcap%
\pgfsetroundjoin%
\pgfsetlinewidth{0.100375pt}%
\definecolor{currentstroke}{rgb}{0.827451,0.827451,0.827451}%
\pgfsetstrokecolor{currentstroke}%
\pgfsetdash{}{0pt}%
\pgfpathmoveto{\pgfqpoint{2.902866in}{1.080890in}}%
\pgfpathlineto{\pgfqpoint{2.902866in}{3.227753in}}%
\pgfusepath{stroke}%
\end{pgfscope}%
\begin{pgfscope}%
\pgfsetbuttcap%
\pgfsetroundjoin%
\definecolor{currentfill}{rgb}{0.000000,0.000000,0.000000}%
\pgfsetfillcolor{currentfill}%
\pgfsetlinewidth{0.501875pt}%
\definecolor{currentstroke}{rgb}{0.000000,0.000000,0.000000}%
\pgfsetstrokecolor{currentstroke}%
\pgfsetdash{}{0pt}%
\pgfsys@defobject{currentmarker}{\pgfqpoint{0.000000in}{0.000000in}}{\pgfqpoint{0.000000in}{0.020833in}}{%
\pgfpathmoveto{\pgfqpoint{0.000000in}{0.000000in}}%
\pgfpathlineto{\pgfqpoint{0.000000in}{0.020833in}}%
\pgfusepath{stroke,fill}%
}%
\begin{pgfscope}%
\pgfsys@transformshift{2.902866in}{1.080890in}%
\pgfsys@useobject{currentmarker}{}%
\end{pgfscope}%
\end{pgfscope}%
\begin{pgfscope}%
\pgfsetbuttcap%
\pgfsetroundjoin%
\definecolor{currentfill}{rgb}{0.000000,0.000000,0.000000}%
\pgfsetfillcolor{currentfill}%
\pgfsetlinewidth{0.501875pt}%
\definecolor{currentstroke}{rgb}{0.000000,0.000000,0.000000}%
\pgfsetstrokecolor{currentstroke}%
\pgfsetdash{}{0pt}%
\pgfsys@defobject{currentmarker}{\pgfqpoint{0.000000in}{-0.020833in}}{\pgfqpoint{0.000000in}{0.000000in}}{%
\pgfpathmoveto{\pgfqpoint{0.000000in}{0.000000in}}%
\pgfpathlineto{\pgfqpoint{0.000000in}{-0.020833in}}%
\pgfusepath{stroke,fill}%
}%
\begin{pgfscope}%
\pgfsys@transformshift{2.902866in}{3.227753in}%
\pgfsys@useobject{currentmarker}{}%
\end{pgfscope}%
\end{pgfscope}%
\begin{pgfscope}%
\pgfpathrectangle{\pgfqpoint{0.481681in}{1.080890in}}{\pgfqpoint{5.785672in}{2.146863in}}%
\pgfusepath{clip}%
\pgfsetrectcap%
\pgfsetroundjoin%
\pgfsetlinewidth{0.100375pt}%
\definecolor{currentstroke}{rgb}{0.827451,0.827451,0.827451}%
\pgfsetstrokecolor{currentstroke}%
\pgfsetdash{}{0pt}%
\pgfpathmoveto{\pgfqpoint{2.938563in}{1.080890in}}%
\pgfpathlineto{\pgfqpoint{2.938563in}{3.227753in}}%
\pgfusepath{stroke}%
\end{pgfscope}%
\begin{pgfscope}%
\pgfsetbuttcap%
\pgfsetroundjoin%
\definecolor{currentfill}{rgb}{0.000000,0.000000,0.000000}%
\pgfsetfillcolor{currentfill}%
\pgfsetlinewidth{0.501875pt}%
\definecolor{currentstroke}{rgb}{0.000000,0.000000,0.000000}%
\pgfsetstrokecolor{currentstroke}%
\pgfsetdash{}{0pt}%
\pgfsys@defobject{currentmarker}{\pgfqpoint{0.000000in}{0.000000in}}{\pgfqpoint{0.000000in}{0.020833in}}{%
\pgfpathmoveto{\pgfqpoint{0.000000in}{0.000000in}}%
\pgfpathlineto{\pgfqpoint{0.000000in}{0.020833in}}%
\pgfusepath{stroke,fill}%
}%
\begin{pgfscope}%
\pgfsys@transformshift{2.938563in}{1.080890in}%
\pgfsys@useobject{currentmarker}{}%
\end{pgfscope}%
\end{pgfscope}%
\begin{pgfscope}%
\pgfsetbuttcap%
\pgfsetroundjoin%
\definecolor{currentfill}{rgb}{0.000000,0.000000,0.000000}%
\pgfsetfillcolor{currentfill}%
\pgfsetlinewidth{0.501875pt}%
\definecolor{currentstroke}{rgb}{0.000000,0.000000,0.000000}%
\pgfsetstrokecolor{currentstroke}%
\pgfsetdash{}{0pt}%
\pgfsys@defobject{currentmarker}{\pgfqpoint{0.000000in}{-0.020833in}}{\pgfqpoint{0.000000in}{0.000000in}}{%
\pgfpathmoveto{\pgfqpoint{0.000000in}{0.000000in}}%
\pgfpathlineto{\pgfqpoint{0.000000in}{-0.020833in}}%
\pgfusepath{stroke,fill}%
}%
\begin{pgfscope}%
\pgfsys@transformshift{2.938563in}{3.227753in}%
\pgfsys@useobject{currentmarker}{}%
\end{pgfscope}%
\end{pgfscope}%
\begin{pgfscope}%
\pgfpathrectangle{\pgfqpoint{0.481681in}{1.080890in}}{\pgfqpoint{5.785672in}{2.146863in}}%
\pgfusepath{clip}%
\pgfsetrectcap%
\pgfsetroundjoin%
\pgfsetlinewidth{0.100375pt}%
\definecolor{currentstroke}{rgb}{0.827451,0.827451,0.827451}%
\pgfsetstrokecolor{currentstroke}%
\pgfsetdash{}{0pt}%
\pgfpathmoveto{\pgfqpoint{2.974261in}{1.080890in}}%
\pgfpathlineto{\pgfqpoint{2.974261in}{3.227753in}}%
\pgfusepath{stroke}%
\end{pgfscope}%
\begin{pgfscope}%
\pgfsetbuttcap%
\pgfsetroundjoin%
\definecolor{currentfill}{rgb}{0.000000,0.000000,0.000000}%
\pgfsetfillcolor{currentfill}%
\pgfsetlinewidth{0.501875pt}%
\definecolor{currentstroke}{rgb}{0.000000,0.000000,0.000000}%
\pgfsetstrokecolor{currentstroke}%
\pgfsetdash{}{0pt}%
\pgfsys@defobject{currentmarker}{\pgfqpoint{0.000000in}{0.000000in}}{\pgfqpoint{0.000000in}{0.020833in}}{%
\pgfpathmoveto{\pgfqpoint{0.000000in}{0.000000in}}%
\pgfpathlineto{\pgfqpoint{0.000000in}{0.020833in}}%
\pgfusepath{stroke,fill}%
}%
\begin{pgfscope}%
\pgfsys@transformshift{2.974261in}{1.080890in}%
\pgfsys@useobject{currentmarker}{}%
\end{pgfscope}%
\end{pgfscope}%
\begin{pgfscope}%
\pgfsetbuttcap%
\pgfsetroundjoin%
\definecolor{currentfill}{rgb}{0.000000,0.000000,0.000000}%
\pgfsetfillcolor{currentfill}%
\pgfsetlinewidth{0.501875pt}%
\definecolor{currentstroke}{rgb}{0.000000,0.000000,0.000000}%
\pgfsetstrokecolor{currentstroke}%
\pgfsetdash{}{0pt}%
\pgfsys@defobject{currentmarker}{\pgfqpoint{0.000000in}{-0.020833in}}{\pgfqpoint{0.000000in}{0.000000in}}{%
\pgfpathmoveto{\pgfqpoint{0.000000in}{0.000000in}}%
\pgfpathlineto{\pgfqpoint{0.000000in}{-0.020833in}}%
\pgfusepath{stroke,fill}%
}%
\begin{pgfscope}%
\pgfsys@transformshift{2.974261in}{3.227753in}%
\pgfsys@useobject{currentmarker}{}%
\end{pgfscope}%
\end{pgfscope}%
\begin{pgfscope}%
\pgfpathrectangle{\pgfqpoint{0.481681in}{1.080890in}}{\pgfqpoint{5.785672in}{2.146863in}}%
\pgfusepath{clip}%
\pgfsetrectcap%
\pgfsetroundjoin%
\pgfsetlinewidth{0.100375pt}%
\definecolor{currentstroke}{rgb}{0.827451,0.827451,0.827451}%
\pgfsetstrokecolor{currentstroke}%
\pgfsetdash{}{0pt}%
\pgfpathmoveto{\pgfqpoint{3.009958in}{1.080890in}}%
\pgfpathlineto{\pgfqpoint{3.009958in}{3.227753in}}%
\pgfusepath{stroke}%
\end{pgfscope}%
\begin{pgfscope}%
\pgfsetbuttcap%
\pgfsetroundjoin%
\definecolor{currentfill}{rgb}{0.000000,0.000000,0.000000}%
\pgfsetfillcolor{currentfill}%
\pgfsetlinewidth{0.501875pt}%
\definecolor{currentstroke}{rgb}{0.000000,0.000000,0.000000}%
\pgfsetstrokecolor{currentstroke}%
\pgfsetdash{}{0pt}%
\pgfsys@defobject{currentmarker}{\pgfqpoint{0.000000in}{0.000000in}}{\pgfqpoint{0.000000in}{0.020833in}}{%
\pgfpathmoveto{\pgfqpoint{0.000000in}{0.000000in}}%
\pgfpathlineto{\pgfqpoint{0.000000in}{0.020833in}}%
\pgfusepath{stroke,fill}%
}%
\begin{pgfscope}%
\pgfsys@transformshift{3.009958in}{1.080890in}%
\pgfsys@useobject{currentmarker}{}%
\end{pgfscope}%
\end{pgfscope}%
\begin{pgfscope}%
\pgfsetbuttcap%
\pgfsetroundjoin%
\definecolor{currentfill}{rgb}{0.000000,0.000000,0.000000}%
\pgfsetfillcolor{currentfill}%
\pgfsetlinewidth{0.501875pt}%
\definecolor{currentstroke}{rgb}{0.000000,0.000000,0.000000}%
\pgfsetstrokecolor{currentstroke}%
\pgfsetdash{}{0pt}%
\pgfsys@defobject{currentmarker}{\pgfqpoint{0.000000in}{-0.020833in}}{\pgfqpoint{0.000000in}{0.000000in}}{%
\pgfpathmoveto{\pgfqpoint{0.000000in}{0.000000in}}%
\pgfpathlineto{\pgfqpoint{0.000000in}{-0.020833in}}%
\pgfusepath{stroke,fill}%
}%
\begin{pgfscope}%
\pgfsys@transformshift{3.009958in}{3.227753in}%
\pgfsys@useobject{currentmarker}{}%
\end{pgfscope}%
\end{pgfscope}%
\begin{pgfscope}%
\pgfpathrectangle{\pgfqpoint{0.481681in}{1.080890in}}{\pgfqpoint{5.785672in}{2.146863in}}%
\pgfusepath{clip}%
\pgfsetrectcap%
\pgfsetroundjoin%
\pgfsetlinewidth{0.100375pt}%
\definecolor{currentstroke}{rgb}{0.827451,0.827451,0.827451}%
\pgfsetstrokecolor{currentstroke}%
\pgfsetdash{}{0pt}%
\pgfpathmoveto{\pgfqpoint{3.045655in}{1.080890in}}%
\pgfpathlineto{\pgfqpoint{3.045655in}{3.227753in}}%
\pgfusepath{stroke}%
\end{pgfscope}%
\begin{pgfscope}%
\pgfsetbuttcap%
\pgfsetroundjoin%
\definecolor{currentfill}{rgb}{0.000000,0.000000,0.000000}%
\pgfsetfillcolor{currentfill}%
\pgfsetlinewidth{0.501875pt}%
\definecolor{currentstroke}{rgb}{0.000000,0.000000,0.000000}%
\pgfsetstrokecolor{currentstroke}%
\pgfsetdash{}{0pt}%
\pgfsys@defobject{currentmarker}{\pgfqpoint{0.000000in}{0.000000in}}{\pgfqpoint{0.000000in}{0.020833in}}{%
\pgfpathmoveto{\pgfqpoint{0.000000in}{0.000000in}}%
\pgfpathlineto{\pgfqpoint{0.000000in}{0.020833in}}%
\pgfusepath{stroke,fill}%
}%
\begin{pgfscope}%
\pgfsys@transformshift{3.045655in}{1.080890in}%
\pgfsys@useobject{currentmarker}{}%
\end{pgfscope}%
\end{pgfscope}%
\begin{pgfscope}%
\pgfsetbuttcap%
\pgfsetroundjoin%
\definecolor{currentfill}{rgb}{0.000000,0.000000,0.000000}%
\pgfsetfillcolor{currentfill}%
\pgfsetlinewidth{0.501875pt}%
\definecolor{currentstroke}{rgb}{0.000000,0.000000,0.000000}%
\pgfsetstrokecolor{currentstroke}%
\pgfsetdash{}{0pt}%
\pgfsys@defobject{currentmarker}{\pgfqpoint{0.000000in}{-0.020833in}}{\pgfqpoint{0.000000in}{0.000000in}}{%
\pgfpathmoveto{\pgfqpoint{0.000000in}{0.000000in}}%
\pgfpathlineto{\pgfqpoint{0.000000in}{-0.020833in}}%
\pgfusepath{stroke,fill}%
}%
\begin{pgfscope}%
\pgfsys@transformshift{3.045655in}{3.227753in}%
\pgfsys@useobject{currentmarker}{}%
\end{pgfscope}%
\end{pgfscope}%
\begin{pgfscope}%
\pgfpathrectangle{\pgfqpoint{0.481681in}{1.080890in}}{\pgfqpoint{5.785672in}{2.146863in}}%
\pgfusepath{clip}%
\pgfsetrectcap%
\pgfsetroundjoin%
\pgfsetlinewidth{0.100375pt}%
\definecolor{currentstroke}{rgb}{0.827451,0.827451,0.827451}%
\pgfsetstrokecolor{currentstroke}%
\pgfsetdash{}{0pt}%
\pgfpathmoveto{\pgfqpoint{3.081353in}{1.080890in}}%
\pgfpathlineto{\pgfqpoint{3.081353in}{3.227753in}}%
\pgfusepath{stroke}%
\end{pgfscope}%
\begin{pgfscope}%
\pgfsetbuttcap%
\pgfsetroundjoin%
\definecolor{currentfill}{rgb}{0.000000,0.000000,0.000000}%
\pgfsetfillcolor{currentfill}%
\pgfsetlinewidth{0.501875pt}%
\definecolor{currentstroke}{rgb}{0.000000,0.000000,0.000000}%
\pgfsetstrokecolor{currentstroke}%
\pgfsetdash{}{0pt}%
\pgfsys@defobject{currentmarker}{\pgfqpoint{0.000000in}{0.000000in}}{\pgfqpoint{0.000000in}{0.020833in}}{%
\pgfpathmoveto{\pgfqpoint{0.000000in}{0.000000in}}%
\pgfpathlineto{\pgfqpoint{0.000000in}{0.020833in}}%
\pgfusepath{stroke,fill}%
}%
\begin{pgfscope}%
\pgfsys@transformshift{3.081353in}{1.080890in}%
\pgfsys@useobject{currentmarker}{}%
\end{pgfscope}%
\end{pgfscope}%
\begin{pgfscope}%
\pgfsetbuttcap%
\pgfsetroundjoin%
\definecolor{currentfill}{rgb}{0.000000,0.000000,0.000000}%
\pgfsetfillcolor{currentfill}%
\pgfsetlinewidth{0.501875pt}%
\definecolor{currentstroke}{rgb}{0.000000,0.000000,0.000000}%
\pgfsetstrokecolor{currentstroke}%
\pgfsetdash{}{0pt}%
\pgfsys@defobject{currentmarker}{\pgfqpoint{0.000000in}{-0.020833in}}{\pgfqpoint{0.000000in}{0.000000in}}{%
\pgfpathmoveto{\pgfqpoint{0.000000in}{0.000000in}}%
\pgfpathlineto{\pgfqpoint{0.000000in}{-0.020833in}}%
\pgfusepath{stroke,fill}%
}%
\begin{pgfscope}%
\pgfsys@transformshift{3.081353in}{3.227753in}%
\pgfsys@useobject{currentmarker}{}%
\end{pgfscope}%
\end{pgfscope}%
\begin{pgfscope}%
\pgfpathrectangle{\pgfqpoint{0.481681in}{1.080890in}}{\pgfqpoint{5.785672in}{2.146863in}}%
\pgfusepath{clip}%
\pgfsetrectcap%
\pgfsetroundjoin%
\pgfsetlinewidth{0.100375pt}%
\definecolor{currentstroke}{rgb}{0.827451,0.827451,0.827451}%
\pgfsetstrokecolor{currentstroke}%
\pgfsetdash{}{0pt}%
\pgfpathmoveto{\pgfqpoint{3.117050in}{1.080890in}}%
\pgfpathlineto{\pgfqpoint{3.117050in}{3.227753in}}%
\pgfusepath{stroke}%
\end{pgfscope}%
\begin{pgfscope}%
\pgfsetbuttcap%
\pgfsetroundjoin%
\definecolor{currentfill}{rgb}{0.000000,0.000000,0.000000}%
\pgfsetfillcolor{currentfill}%
\pgfsetlinewidth{0.501875pt}%
\definecolor{currentstroke}{rgb}{0.000000,0.000000,0.000000}%
\pgfsetstrokecolor{currentstroke}%
\pgfsetdash{}{0pt}%
\pgfsys@defobject{currentmarker}{\pgfqpoint{0.000000in}{0.000000in}}{\pgfqpoint{0.000000in}{0.020833in}}{%
\pgfpathmoveto{\pgfqpoint{0.000000in}{0.000000in}}%
\pgfpathlineto{\pgfqpoint{0.000000in}{0.020833in}}%
\pgfusepath{stroke,fill}%
}%
\begin{pgfscope}%
\pgfsys@transformshift{3.117050in}{1.080890in}%
\pgfsys@useobject{currentmarker}{}%
\end{pgfscope}%
\end{pgfscope}%
\begin{pgfscope}%
\pgfsetbuttcap%
\pgfsetroundjoin%
\definecolor{currentfill}{rgb}{0.000000,0.000000,0.000000}%
\pgfsetfillcolor{currentfill}%
\pgfsetlinewidth{0.501875pt}%
\definecolor{currentstroke}{rgb}{0.000000,0.000000,0.000000}%
\pgfsetstrokecolor{currentstroke}%
\pgfsetdash{}{0pt}%
\pgfsys@defobject{currentmarker}{\pgfqpoint{0.000000in}{-0.020833in}}{\pgfqpoint{0.000000in}{0.000000in}}{%
\pgfpathmoveto{\pgfqpoint{0.000000in}{0.000000in}}%
\pgfpathlineto{\pgfqpoint{0.000000in}{-0.020833in}}%
\pgfusepath{stroke,fill}%
}%
\begin{pgfscope}%
\pgfsys@transformshift{3.117050in}{3.227753in}%
\pgfsys@useobject{currentmarker}{}%
\end{pgfscope}%
\end{pgfscope}%
\begin{pgfscope}%
\pgfpathrectangle{\pgfqpoint{0.481681in}{1.080890in}}{\pgfqpoint{5.785672in}{2.146863in}}%
\pgfusepath{clip}%
\pgfsetrectcap%
\pgfsetroundjoin%
\pgfsetlinewidth{0.100375pt}%
\definecolor{currentstroke}{rgb}{0.827451,0.827451,0.827451}%
\pgfsetstrokecolor{currentstroke}%
\pgfsetdash{}{0pt}%
\pgfpathmoveto{\pgfqpoint{3.152747in}{1.080890in}}%
\pgfpathlineto{\pgfqpoint{3.152747in}{3.227753in}}%
\pgfusepath{stroke}%
\end{pgfscope}%
\begin{pgfscope}%
\pgfsetbuttcap%
\pgfsetroundjoin%
\definecolor{currentfill}{rgb}{0.000000,0.000000,0.000000}%
\pgfsetfillcolor{currentfill}%
\pgfsetlinewidth{0.501875pt}%
\definecolor{currentstroke}{rgb}{0.000000,0.000000,0.000000}%
\pgfsetstrokecolor{currentstroke}%
\pgfsetdash{}{0pt}%
\pgfsys@defobject{currentmarker}{\pgfqpoint{0.000000in}{0.000000in}}{\pgfqpoint{0.000000in}{0.020833in}}{%
\pgfpathmoveto{\pgfqpoint{0.000000in}{0.000000in}}%
\pgfpathlineto{\pgfqpoint{0.000000in}{0.020833in}}%
\pgfusepath{stroke,fill}%
}%
\begin{pgfscope}%
\pgfsys@transformshift{3.152747in}{1.080890in}%
\pgfsys@useobject{currentmarker}{}%
\end{pgfscope}%
\end{pgfscope}%
\begin{pgfscope}%
\pgfsetbuttcap%
\pgfsetroundjoin%
\definecolor{currentfill}{rgb}{0.000000,0.000000,0.000000}%
\pgfsetfillcolor{currentfill}%
\pgfsetlinewidth{0.501875pt}%
\definecolor{currentstroke}{rgb}{0.000000,0.000000,0.000000}%
\pgfsetstrokecolor{currentstroke}%
\pgfsetdash{}{0pt}%
\pgfsys@defobject{currentmarker}{\pgfqpoint{0.000000in}{-0.020833in}}{\pgfqpoint{0.000000in}{0.000000in}}{%
\pgfpathmoveto{\pgfqpoint{0.000000in}{0.000000in}}%
\pgfpathlineto{\pgfqpoint{0.000000in}{-0.020833in}}%
\pgfusepath{stroke,fill}%
}%
\begin{pgfscope}%
\pgfsys@transformshift{3.152747in}{3.227753in}%
\pgfsys@useobject{currentmarker}{}%
\end{pgfscope}%
\end{pgfscope}%
\begin{pgfscope}%
\pgfpathrectangle{\pgfqpoint{0.481681in}{1.080890in}}{\pgfqpoint{5.785672in}{2.146863in}}%
\pgfusepath{clip}%
\pgfsetrectcap%
\pgfsetroundjoin%
\pgfsetlinewidth{0.100375pt}%
\definecolor{currentstroke}{rgb}{0.827451,0.827451,0.827451}%
\pgfsetstrokecolor{currentstroke}%
\pgfsetdash{}{0pt}%
\pgfpathmoveto{\pgfqpoint{3.188445in}{1.080890in}}%
\pgfpathlineto{\pgfqpoint{3.188445in}{3.227753in}}%
\pgfusepath{stroke}%
\end{pgfscope}%
\begin{pgfscope}%
\pgfsetbuttcap%
\pgfsetroundjoin%
\definecolor{currentfill}{rgb}{0.000000,0.000000,0.000000}%
\pgfsetfillcolor{currentfill}%
\pgfsetlinewidth{0.501875pt}%
\definecolor{currentstroke}{rgb}{0.000000,0.000000,0.000000}%
\pgfsetstrokecolor{currentstroke}%
\pgfsetdash{}{0pt}%
\pgfsys@defobject{currentmarker}{\pgfqpoint{0.000000in}{0.000000in}}{\pgfqpoint{0.000000in}{0.020833in}}{%
\pgfpathmoveto{\pgfqpoint{0.000000in}{0.000000in}}%
\pgfpathlineto{\pgfqpoint{0.000000in}{0.020833in}}%
\pgfusepath{stroke,fill}%
}%
\begin{pgfscope}%
\pgfsys@transformshift{3.188445in}{1.080890in}%
\pgfsys@useobject{currentmarker}{}%
\end{pgfscope}%
\end{pgfscope}%
\begin{pgfscope}%
\pgfsetbuttcap%
\pgfsetroundjoin%
\definecolor{currentfill}{rgb}{0.000000,0.000000,0.000000}%
\pgfsetfillcolor{currentfill}%
\pgfsetlinewidth{0.501875pt}%
\definecolor{currentstroke}{rgb}{0.000000,0.000000,0.000000}%
\pgfsetstrokecolor{currentstroke}%
\pgfsetdash{}{0pt}%
\pgfsys@defobject{currentmarker}{\pgfqpoint{0.000000in}{-0.020833in}}{\pgfqpoint{0.000000in}{0.000000in}}{%
\pgfpathmoveto{\pgfqpoint{0.000000in}{0.000000in}}%
\pgfpathlineto{\pgfqpoint{0.000000in}{-0.020833in}}%
\pgfusepath{stroke,fill}%
}%
\begin{pgfscope}%
\pgfsys@transformshift{3.188445in}{3.227753in}%
\pgfsys@useobject{currentmarker}{}%
\end{pgfscope}%
\end{pgfscope}%
\begin{pgfscope}%
\pgfpathrectangle{\pgfqpoint{0.481681in}{1.080890in}}{\pgfqpoint{5.785672in}{2.146863in}}%
\pgfusepath{clip}%
\pgfsetrectcap%
\pgfsetroundjoin%
\pgfsetlinewidth{0.100375pt}%
\definecolor{currentstroke}{rgb}{0.827451,0.827451,0.827451}%
\pgfsetstrokecolor{currentstroke}%
\pgfsetdash{}{0pt}%
\pgfpathmoveto{\pgfqpoint{3.224142in}{1.080890in}}%
\pgfpathlineto{\pgfqpoint{3.224142in}{3.227753in}}%
\pgfusepath{stroke}%
\end{pgfscope}%
\begin{pgfscope}%
\pgfsetbuttcap%
\pgfsetroundjoin%
\definecolor{currentfill}{rgb}{0.000000,0.000000,0.000000}%
\pgfsetfillcolor{currentfill}%
\pgfsetlinewidth{0.501875pt}%
\definecolor{currentstroke}{rgb}{0.000000,0.000000,0.000000}%
\pgfsetstrokecolor{currentstroke}%
\pgfsetdash{}{0pt}%
\pgfsys@defobject{currentmarker}{\pgfqpoint{0.000000in}{0.000000in}}{\pgfqpoint{0.000000in}{0.020833in}}{%
\pgfpathmoveto{\pgfqpoint{0.000000in}{0.000000in}}%
\pgfpathlineto{\pgfqpoint{0.000000in}{0.020833in}}%
\pgfusepath{stroke,fill}%
}%
\begin{pgfscope}%
\pgfsys@transformshift{3.224142in}{1.080890in}%
\pgfsys@useobject{currentmarker}{}%
\end{pgfscope}%
\end{pgfscope}%
\begin{pgfscope}%
\pgfsetbuttcap%
\pgfsetroundjoin%
\definecolor{currentfill}{rgb}{0.000000,0.000000,0.000000}%
\pgfsetfillcolor{currentfill}%
\pgfsetlinewidth{0.501875pt}%
\definecolor{currentstroke}{rgb}{0.000000,0.000000,0.000000}%
\pgfsetstrokecolor{currentstroke}%
\pgfsetdash{}{0pt}%
\pgfsys@defobject{currentmarker}{\pgfqpoint{0.000000in}{-0.020833in}}{\pgfqpoint{0.000000in}{0.000000in}}{%
\pgfpathmoveto{\pgfqpoint{0.000000in}{0.000000in}}%
\pgfpathlineto{\pgfqpoint{0.000000in}{-0.020833in}}%
\pgfusepath{stroke,fill}%
}%
\begin{pgfscope}%
\pgfsys@transformshift{3.224142in}{3.227753in}%
\pgfsys@useobject{currentmarker}{}%
\end{pgfscope}%
\end{pgfscope}%
\begin{pgfscope}%
\pgfpathrectangle{\pgfqpoint{0.481681in}{1.080890in}}{\pgfqpoint{5.785672in}{2.146863in}}%
\pgfusepath{clip}%
\pgfsetrectcap%
\pgfsetroundjoin%
\pgfsetlinewidth{0.100375pt}%
\definecolor{currentstroke}{rgb}{0.827451,0.827451,0.827451}%
\pgfsetstrokecolor{currentstroke}%
\pgfsetdash{}{0pt}%
\pgfpathmoveto{\pgfqpoint{3.295537in}{1.080890in}}%
\pgfpathlineto{\pgfqpoint{3.295537in}{3.227753in}}%
\pgfusepath{stroke}%
\end{pgfscope}%
\begin{pgfscope}%
\pgfsetbuttcap%
\pgfsetroundjoin%
\definecolor{currentfill}{rgb}{0.000000,0.000000,0.000000}%
\pgfsetfillcolor{currentfill}%
\pgfsetlinewidth{0.501875pt}%
\definecolor{currentstroke}{rgb}{0.000000,0.000000,0.000000}%
\pgfsetstrokecolor{currentstroke}%
\pgfsetdash{}{0pt}%
\pgfsys@defobject{currentmarker}{\pgfqpoint{0.000000in}{0.000000in}}{\pgfqpoint{0.000000in}{0.020833in}}{%
\pgfpathmoveto{\pgfqpoint{0.000000in}{0.000000in}}%
\pgfpathlineto{\pgfqpoint{0.000000in}{0.020833in}}%
\pgfusepath{stroke,fill}%
}%
\begin{pgfscope}%
\pgfsys@transformshift{3.295537in}{1.080890in}%
\pgfsys@useobject{currentmarker}{}%
\end{pgfscope}%
\end{pgfscope}%
\begin{pgfscope}%
\pgfsetbuttcap%
\pgfsetroundjoin%
\definecolor{currentfill}{rgb}{0.000000,0.000000,0.000000}%
\pgfsetfillcolor{currentfill}%
\pgfsetlinewidth{0.501875pt}%
\definecolor{currentstroke}{rgb}{0.000000,0.000000,0.000000}%
\pgfsetstrokecolor{currentstroke}%
\pgfsetdash{}{0pt}%
\pgfsys@defobject{currentmarker}{\pgfqpoint{0.000000in}{-0.020833in}}{\pgfqpoint{0.000000in}{0.000000in}}{%
\pgfpathmoveto{\pgfqpoint{0.000000in}{0.000000in}}%
\pgfpathlineto{\pgfqpoint{0.000000in}{-0.020833in}}%
\pgfusepath{stroke,fill}%
}%
\begin{pgfscope}%
\pgfsys@transformshift{3.295537in}{3.227753in}%
\pgfsys@useobject{currentmarker}{}%
\end{pgfscope}%
\end{pgfscope}%
\begin{pgfscope}%
\pgfpathrectangle{\pgfqpoint{0.481681in}{1.080890in}}{\pgfqpoint{5.785672in}{2.146863in}}%
\pgfusepath{clip}%
\pgfsetrectcap%
\pgfsetroundjoin%
\pgfsetlinewidth{0.100375pt}%
\definecolor{currentstroke}{rgb}{0.827451,0.827451,0.827451}%
\pgfsetstrokecolor{currentstroke}%
\pgfsetdash{}{0pt}%
\pgfpathmoveto{\pgfqpoint{3.331234in}{1.080890in}}%
\pgfpathlineto{\pgfqpoint{3.331234in}{3.227753in}}%
\pgfusepath{stroke}%
\end{pgfscope}%
\begin{pgfscope}%
\pgfsetbuttcap%
\pgfsetroundjoin%
\definecolor{currentfill}{rgb}{0.000000,0.000000,0.000000}%
\pgfsetfillcolor{currentfill}%
\pgfsetlinewidth{0.501875pt}%
\definecolor{currentstroke}{rgb}{0.000000,0.000000,0.000000}%
\pgfsetstrokecolor{currentstroke}%
\pgfsetdash{}{0pt}%
\pgfsys@defobject{currentmarker}{\pgfqpoint{0.000000in}{0.000000in}}{\pgfqpoint{0.000000in}{0.020833in}}{%
\pgfpathmoveto{\pgfqpoint{0.000000in}{0.000000in}}%
\pgfpathlineto{\pgfqpoint{0.000000in}{0.020833in}}%
\pgfusepath{stroke,fill}%
}%
\begin{pgfscope}%
\pgfsys@transformshift{3.331234in}{1.080890in}%
\pgfsys@useobject{currentmarker}{}%
\end{pgfscope}%
\end{pgfscope}%
\begin{pgfscope}%
\pgfsetbuttcap%
\pgfsetroundjoin%
\definecolor{currentfill}{rgb}{0.000000,0.000000,0.000000}%
\pgfsetfillcolor{currentfill}%
\pgfsetlinewidth{0.501875pt}%
\definecolor{currentstroke}{rgb}{0.000000,0.000000,0.000000}%
\pgfsetstrokecolor{currentstroke}%
\pgfsetdash{}{0pt}%
\pgfsys@defobject{currentmarker}{\pgfqpoint{0.000000in}{-0.020833in}}{\pgfqpoint{0.000000in}{0.000000in}}{%
\pgfpathmoveto{\pgfqpoint{0.000000in}{0.000000in}}%
\pgfpathlineto{\pgfqpoint{0.000000in}{-0.020833in}}%
\pgfusepath{stroke,fill}%
}%
\begin{pgfscope}%
\pgfsys@transformshift{3.331234in}{3.227753in}%
\pgfsys@useobject{currentmarker}{}%
\end{pgfscope}%
\end{pgfscope}%
\begin{pgfscope}%
\pgfpathrectangle{\pgfqpoint{0.481681in}{1.080890in}}{\pgfqpoint{5.785672in}{2.146863in}}%
\pgfusepath{clip}%
\pgfsetrectcap%
\pgfsetroundjoin%
\pgfsetlinewidth{0.100375pt}%
\definecolor{currentstroke}{rgb}{0.827451,0.827451,0.827451}%
\pgfsetstrokecolor{currentstroke}%
\pgfsetdash{}{0pt}%
\pgfpathmoveto{\pgfqpoint{3.366931in}{1.080890in}}%
\pgfpathlineto{\pgfqpoint{3.366931in}{3.227753in}}%
\pgfusepath{stroke}%
\end{pgfscope}%
\begin{pgfscope}%
\pgfsetbuttcap%
\pgfsetroundjoin%
\definecolor{currentfill}{rgb}{0.000000,0.000000,0.000000}%
\pgfsetfillcolor{currentfill}%
\pgfsetlinewidth{0.501875pt}%
\definecolor{currentstroke}{rgb}{0.000000,0.000000,0.000000}%
\pgfsetstrokecolor{currentstroke}%
\pgfsetdash{}{0pt}%
\pgfsys@defobject{currentmarker}{\pgfqpoint{0.000000in}{0.000000in}}{\pgfqpoint{0.000000in}{0.020833in}}{%
\pgfpathmoveto{\pgfqpoint{0.000000in}{0.000000in}}%
\pgfpathlineto{\pgfqpoint{0.000000in}{0.020833in}}%
\pgfusepath{stroke,fill}%
}%
\begin{pgfscope}%
\pgfsys@transformshift{3.366931in}{1.080890in}%
\pgfsys@useobject{currentmarker}{}%
\end{pgfscope}%
\end{pgfscope}%
\begin{pgfscope}%
\pgfsetbuttcap%
\pgfsetroundjoin%
\definecolor{currentfill}{rgb}{0.000000,0.000000,0.000000}%
\pgfsetfillcolor{currentfill}%
\pgfsetlinewidth{0.501875pt}%
\definecolor{currentstroke}{rgb}{0.000000,0.000000,0.000000}%
\pgfsetstrokecolor{currentstroke}%
\pgfsetdash{}{0pt}%
\pgfsys@defobject{currentmarker}{\pgfqpoint{0.000000in}{-0.020833in}}{\pgfqpoint{0.000000in}{0.000000in}}{%
\pgfpathmoveto{\pgfqpoint{0.000000in}{0.000000in}}%
\pgfpathlineto{\pgfqpoint{0.000000in}{-0.020833in}}%
\pgfusepath{stroke,fill}%
}%
\begin{pgfscope}%
\pgfsys@transformshift{3.366931in}{3.227753in}%
\pgfsys@useobject{currentmarker}{}%
\end{pgfscope}%
\end{pgfscope}%
\begin{pgfscope}%
\pgfpathrectangle{\pgfqpoint{0.481681in}{1.080890in}}{\pgfqpoint{5.785672in}{2.146863in}}%
\pgfusepath{clip}%
\pgfsetrectcap%
\pgfsetroundjoin%
\pgfsetlinewidth{0.100375pt}%
\definecolor{currentstroke}{rgb}{0.827451,0.827451,0.827451}%
\pgfsetstrokecolor{currentstroke}%
\pgfsetdash{}{0pt}%
\pgfpathmoveto{\pgfqpoint{3.402629in}{1.080890in}}%
\pgfpathlineto{\pgfqpoint{3.402629in}{3.227753in}}%
\pgfusepath{stroke}%
\end{pgfscope}%
\begin{pgfscope}%
\pgfsetbuttcap%
\pgfsetroundjoin%
\definecolor{currentfill}{rgb}{0.000000,0.000000,0.000000}%
\pgfsetfillcolor{currentfill}%
\pgfsetlinewidth{0.501875pt}%
\definecolor{currentstroke}{rgb}{0.000000,0.000000,0.000000}%
\pgfsetstrokecolor{currentstroke}%
\pgfsetdash{}{0pt}%
\pgfsys@defobject{currentmarker}{\pgfqpoint{0.000000in}{0.000000in}}{\pgfqpoint{0.000000in}{0.020833in}}{%
\pgfpathmoveto{\pgfqpoint{0.000000in}{0.000000in}}%
\pgfpathlineto{\pgfqpoint{0.000000in}{0.020833in}}%
\pgfusepath{stroke,fill}%
}%
\begin{pgfscope}%
\pgfsys@transformshift{3.402629in}{1.080890in}%
\pgfsys@useobject{currentmarker}{}%
\end{pgfscope}%
\end{pgfscope}%
\begin{pgfscope}%
\pgfsetbuttcap%
\pgfsetroundjoin%
\definecolor{currentfill}{rgb}{0.000000,0.000000,0.000000}%
\pgfsetfillcolor{currentfill}%
\pgfsetlinewidth{0.501875pt}%
\definecolor{currentstroke}{rgb}{0.000000,0.000000,0.000000}%
\pgfsetstrokecolor{currentstroke}%
\pgfsetdash{}{0pt}%
\pgfsys@defobject{currentmarker}{\pgfqpoint{0.000000in}{-0.020833in}}{\pgfqpoint{0.000000in}{0.000000in}}{%
\pgfpathmoveto{\pgfqpoint{0.000000in}{0.000000in}}%
\pgfpathlineto{\pgfqpoint{0.000000in}{-0.020833in}}%
\pgfusepath{stroke,fill}%
}%
\begin{pgfscope}%
\pgfsys@transformshift{3.402629in}{3.227753in}%
\pgfsys@useobject{currentmarker}{}%
\end{pgfscope}%
\end{pgfscope}%
\begin{pgfscope}%
\pgfpathrectangle{\pgfqpoint{0.481681in}{1.080890in}}{\pgfqpoint{5.785672in}{2.146863in}}%
\pgfusepath{clip}%
\pgfsetrectcap%
\pgfsetroundjoin%
\pgfsetlinewidth{0.100375pt}%
\definecolor{currentstroke}{rgb}{0.827451,0.827451,0.827451}%
\pgfsetstrokecolor{currentstroke}%
\pgfsetdash{}{0pt}%
\pgfpathmoveto{\pgfqpoint{3.438326in}{1.080890in}}%
\pgfpathlineto{\pgfqpoint{3.438326in}{3.227753in}}%
\pgfusepath{stroke}%
\end{pgfscope}%
\begin{pgfscope}%
\pgfsetbuttcap%
\pgfsetroundjoin%
\definecolor{currentfill}{rgb}{0.000000,0.000000,0.000000}%
\pgfsetfillcolor{currentfill}%
\pgfsetlinewidth{0.501875pt}%
\definecolor{currentstroke}{rgb}{0.000000,0.000000,0.000000}%
\pgfsetstrokecolor{currentstroke}%
\pgfsetdash{}{0pt}%
\pgfsys@defobject{currentmarker}{\pgfqpoint{0.000000in}{0.000000in}}{\pgfqpoint{0.000000in}{0.020833in}}{%
\pgfpathmoveto{\pgfqpoint{0.000000in}{0.000000in}}%
\pgfpathlineto{\pgfqpoint{0.000000in}{0.020833in}}%
\pgfusepath{stroke,fill}%
}%
\begin{pgfscope}%
\pgfsys@transformshift{3.438326in}{1.080890in}%
\pgfsys@useobject{currentmarker}{}%
\end{pgfscope}%
\end{pgfscope}%
\begin{pgfscope}%
\pgfsetbuttcap%
\pgfsetroundjoin%
\definecolor{currentfill}{rgb}{0.000000,0.000000,0.000000}%
\pgfsetfillcolor{currentfill}%
\pgfsetlinewidth{0.501875pt}%
\definecolor{currentstroke}{rgb}{0.000000,0.000000,0.000000}%
\pgfsetstrokecolor{currentstroke}%
\pgfsetdash{}{0pt}%
\pgfsys@defobject{currentmarker}{\pgfqpoint{0.000000in}{-0.020833in}}{\pgfqpoint{0.000000in}{0.000000in}}{%
\pgfpathmoveto{\pgfqpoint{0.000000in}{0.000000in}}%
\pgfpathlineto{\pgfqpoint{0.000000in}{-0.020833in}}%
\pgfusepath{stroke,fill}%
}%
\begin{pgfscope}%
\pgfsys@transformshift{3.438326in}{3.227753in}%
\pgfsys@useobject{currentmarker}{}%
\end{pgfscope}%
\end{pgfscope}%
\begin{pgfscope}%
\pgfpathrectangle{\pgfqpoint{0.481681in}{1.080890in}}{\pgfqpoint{5.785672in}{2.146863in}}%
\pgfusepath{clip}%
\pgfsetrectcap%
\pgfsetroundjoin%
\pgfsetlinewidth{0.100375pt}%
\definecolor{currentstroke}{rgb}{0.827451,0.827451,0.827451}%
\pgfsetstrokecolor{currentstroke}%
\pgfsetdash{}{0pt}%
\pgfpathmoveto{\pgfqpoint{3.474023in}{1.080890in}}%
\pgfpathlineto{\pgfqpoint{3.474023in}{3.227753in}}%
\pgfusepath{stroke}%
\end{pgfscope}%
\begin{pgfscope}%
\pgfsetbuttcap%
\pgfsetroundjoin%
\definecolor{currentfill}{rgb}{0.000000,0.000000,0.000000}%
\pgfsetfillcolor{currentfill}%
\pgfsetlinewidth{0.501875pt}%
\definecolor{currentstroke}{rgb}{0.000000,0.000000,0.000000}%
\pgfsetstrokecolor{currentstroke}%
\pgfsetdash{}{0pt}%
\pgfsys@defobject{currentmarker}{\pgfqpoint{0.000000in}{0.000000in}}{\pgfqpoint{0.000000in}{0.020833in}}{%
\pgfpathmoveto{\pgfqpoint{0.000000in}{0.000000in}}%
\pgfpathlineto{\pgfqpoint{0.000000in}{0.020833in}}%
\pgfusepath{stroke,fill}%
}%
\begin{pgfscope}%
\pgfsys@transformshift{3.474023in}{1.080890in}%
\pgfsys@useobject{currentmarker}{}%
\end{pgfscope}%
\end{pgfscope}%
\begin{pgfscope}%
\pgfsetbuttcap%
\pgfsetroundjoin%
\definecolor{currentfill}{rgb}{0.000000,0.000000,0.000000}%
\pgfsetfillcolor{currentfill}%
\pgfsetlinewidth{0.501875pt}%
\definecolor{currentstroke}{rgb}{0.000000,0.000000,0.000000}%
\pgfsetstrokecolor{currentstroke}%
\pgfsetdash{}{0pt}%
\pgfsys@defobject{currentmarker}{\pgfqpoint{0.000000in}{-0.020833in}}{\pgfqpoint{0.000000in}{0.000000in}}{%
\pgfpathmoveto{\pgfqpoint{0.000000in}{0.000000in}}%
\pgfpathlineto{\pgfqpoint{0.000000in}{-0.020833in}}%
\pgfusepath{stroke,fill}%
}%
\begin{pgfscope}%
\pgfsys@transformshift{3.474023in}{3.227753in}%
\pgfsys@useobject{currentmarker}{}%
\end{pgfscope}%
\end{pgfscope}%
\begin{pgfscope}%
\pgfpathrectangle{\pgfqpoint{0.481681in}{1.080890in}}{\pgfqpoint{5.785672in}{2.146863in}}%
\pgfusepath{clip}%
\pgfsetrectcap%
\pgfsetroundjoin%
\pgfsetlinewidth{0.100375pt}%
\definecolor{currentstroke}{rgb}{0.827451,0.827451,0.827451}%
\pgfsetstrokecolor{currentstroke}%
\pgfsetdash{}{0pt}%
\pgfpathmoveto{\pgfqpoint{3.509721in}{1.080890in}}%
\pgfpathlineto{\pgfqpoint{3.509721in}{3.227753in}}%
\pgfusepath{stroke}%
\end{pgfscope}%
\begin{pgfscope}%
\pgfsetbuttcap%
\pgfsetroundjoin%
\definecolor{currentfill}{rgb}{0.000000,0.000000,0.000000}%
\pgfsetfillcolor{currentfill}%
\pgfsetlinewidth{0.501875pt}%
\definecolor{currentstroke}{rgb}{0.000000,0.000000,0.000000}%
\pgfsetstrokecolor{currentstroke}%
\pgfsetdash{}{0pt}%
\pgfsys@defobject{currentmarker}{\pgfqpoint{0.000000in}{0.000000in}}{\pgfqpoint{0.000000in}{0.020833in}}{%
\pgfpathmoveto{\pgfqpoint{0.000000in}{0.000000in}}%
\pgfpathlineto{\pgfqpoint{0.000000in}{0.020833in}}%
\pgfusepath{stroke,fill}%
}%
\begin{pgfscope}%
\pgfsys@transformshift{3.509721in}{1.080890in}%
\pgfsys@useobject{currentmarker}{}%
\end{pgfscope}%
\end{pgfscope}%
\begin{pgfscope}%
\pgfsetbuttcap%
\pgfsetroundjoin%
\definecolor{currentfill}{rgb}{0.000000,0.000000,0.000000}%
\pgfsetfillcolor{currentfill}%
\pgfsetlinewidth{0.501875pt}%
\definecolor{currentstroke}{rgb}{0.000000,0.000000,0.000000}%
\pgfsetstrokecolor{currentstroke}%
\pgfsetdash{}{0pt}%
\pgfsys@defobject{currentmarker}{\pgfqpoint{0.000000in}{-0.020833in}}{\pgfqpoint{0.000000in}{0.000000in}}{%
\pgfpathmoveto{\pgfqpoint{0.000000in}{0.000000in}}%
\pgfpathlineto{\pgfqpoint{0.000000in}{-0.020833in}}%
\pgfusepath{stroke,fill}%
}%
\begin{pgfscope}%
\pgfsys@transformshift{3.509721in}{3.227753in}%
\pgfsys@useobject{currentmarker}{}%
\end{pgfscope}%
\end{pgfscope}%
\begin{pgfscope}%
\pgfpathrectangle{\pgfqpoint{0.481681in}{1.080890in}}{\pgfqpoint{5.785672in}{2.146863in}}%
\pgfusepath{clip}%
\pgfsetrectcap%
\pgfsetroundjoin%
\pgfsetlinewidth{0.100375pt}%
\definecolor{currentstroke}{rgb}{0.827451,0.827451,0.827451}%
\pgfsetstrokecolor{currentstroke}%
\pgfsetdash{}{0pt}%
\pgfpathmoveto{\pgfqpoint{3.545418in}{1.080890in}}%
\pgfpathlineto{\pgfqpoint{3.545418in}{3.227753in}}%
\pgfusepath{stroke}%
\end{pgfscope}%
\begin{pgfscope}%
\pgfsetbuttcap%
\pgfsetroundjoin%
\definecolor{currentfill}{rgb}{0.000000,0.000000,0.000000}%
\pgfsetfillcolor{currentfill}%
\pgfsetlinewidth{0.501875pt}%
\definecolor{currentstroke}{rgb}{0.000000,0.000000,0.000000}%
\pgfsetstrokecolor{currentstroke}%
\pgfsetdash{}{0pt}%
\pgfsys@defobject{currentmarker}{\pgfqpoint{0.000000in}{0.000000in}}{\pgfqpoint{0.000000in}{0.020833in}}{%
\pgfpathmoveto{\pgfqpoint{0.000000in}{0.000000in}}%
\pgfpathlineto{\pgfqpoint{0.000000in}{0.020833in}}%
\pgfusepath{stroke,fill}%
}%
\begin{pgfscope}%
\pgfsys@transformshift{3.545418in}{1.080890in}%
\pgfsys@useobject{currentmarker}{}%
\end{pgfscope}%
\end{pgfscope}%
\begin{pgfscope}%
\pgfsetbuttcap%
\pgfsetroundjoin%
\definecolor{currentfill}{rgb}{0.000000,0.000000,0.000000}%
\pgfsetfillcolor{currentfill}%
\pgfsetlinewidth{0.501875pt}%
\definecolor{currentstroke}{rgb}{0.000000,0.000000,0.000000}%
\pgfsetstrokecolor{currentstroke}%
\pgfsetdash{}{0pt}%
\pgfsys@defobject{currentmarker}{\pgfqpoint{0.000000in}{-0.020833in}}{\pgfqpoint{0.000000in}{0.000000in}}{%
\pgfpathmoveto{\pgfqpoint{0.000000in}{0.000000in}}%
\pgfpathlineto{\pgfqpoint{0.000000in}{-0.020833in}}%
\pgfusepath{stroke,fill}%
}%
\begin{pgfscope}%
\pgfsys@transformshift{3.545418in}{3.227753in}%
\pgfsys@useobject{currentmarker}{}%
\end{pgfscope}%
\end{pgfscope}%
\begin{pgfscope}%
\pgfpathrectangle{\pgfqpoint{0.481681in}{1.080890in}}{\pgfqpoint{5.785672in}{2.146863in}}%
\pgfusepath{clip}%
\pgfsetrectcap%
\pgfsetroundjoin%
\pgfsetlinewidth{0.100375pt}%
\definecolor{currentstroke}{rgb}{0.827451,0.827451,0.827451}%
\pgfsetstrokecolor{currentstroke}%
\pgfsetdash{}{0pt}%
\pgfpathmoveto{\pgfqpoint{3.581115in}{1.080890in}}%
\pgfpathlineto{\pgfqpoint{3.581115in}{3.227753in}}%
\pgfusepath{stroke}%
\end{pgfscope}%
\begin{pgfscope}%
\pgfsetbuttcap%
\pgfsetroundjoin%
\definecolor{currentfill}{rgb}{0.000000,0.000000,0.000000}%
\pgfsetfillcolor{currentfill}%
\pgfsetlinewidth{0.501875pt}%
\definecolor{currentstroke}{rgb}{0.000000,0.000000,0.000000}%
\pgfsetstrokecolor{currentstroke}%
\pgfsetdash{}{0pt}%
\pgfsys@defobject{currentmarker}{\pgfqpoint{0.000000in}{0.000000in}}{\pgfqpoint{0.000000in}{0.020833in}}{%
\pgfpathmoveto{\pgfqpoint{0.000000in}{0.000000in}}%
\pgfpathlineto{\pgfqpoint{0.000000in}{0.020833in}}%
\pgfusepath{stroke,fill}%
}%
\begin{pgfscope}%
\pgfsys@transformshift{3.581115in}{1.080890in}%
\pgfsys@useobject{currentmarker}{}%
\end{pgfscope}%
\end{pgfscope}%
\begin{pgfscope}%
\pgfsetbuttcap%
\pgfsetroundjoin%
\definecolor{currentfill}{rgb}{0.000000,0.000000,0.000000}%
\pgfsetfillcolor{currentfill}%
\pgfsetlinewidth{0.501875pt}%
\definecolor{currentstroke}{rgb}{0.000000,0.000000,0.000000}%
\pgfsetstrokecolor{currentstroke}%
\pgfsetdash{}{0pt}%
\pgfsys@defobject{currentmarker}{\pgfqpoint{0.000000in}{-0.020833in}}{\pgfqpoint{0.000000in}{0.000000in}}{%
\pgfpathmoveto{\pgfqpoint{0.000000in}{0.000000in}}%
\pgfpathlineto{\pgfqpoint{0.000000in}{-0.020833in}}%
\pgfusepath{stroke,fill}%
}%
\begin{pgfscope}%
\pgfsys@transformshift{3.581115in}{3.227753in}%
\pgfsys@useobject{currentmarker}{}%
\end{pgfscope}%
\end{pgfscope}%
\begin{pgfscope}%
\pgfpathrectangle{\pgfqpoint{0.481681in}{1.080890in}}{\pgfqpoint{5.785672in}{2.146863in}}%
\pgfusepath{clip}%
\pgfsetrectcap%
\pgfsetroundjoin%
\pgfsetlinewidth{0.100375pt}%
\definecolor{currentstroke}{rgb}{0.827451,0.827451,0.827451}%
\pgfsetstrokecolor{currentstroke}%
\pgfsetdash{}{0pt}%
\pgfpathmoveto{\pgfqpoint{3.616812in}{1.080890in}}%
\pgfpathlineto{\pgfqpoint{3.616812in}{3.227753in}}%
\pgfusepath{stroke}%
\end{pgfscope}%
\begin{pgfscope}%
\pgfsetbuttcap%
\pgfsetroundjoin%
\definecolor{currentfill}{rgb}{0.000000,0.000000,0.000000}%
\pgfsetfillcolor{currentfill}%
\pgfsetlinewidth{0.501875pt}%
\definecolor{currentstroke}{rgb}{0.000000,0.000000,0.000000}%
\pgfsetstrokecolor{currentstroke}%
\pgfsetdash{}{0pt}%
\pgfsys@defobject{currentmarker}{\pgfqpoint{0.000000in}{0.000000in}}{\pgfqpoint{0.000000in}{0.020833in}}{%
\pgfpathmoveto{\pgfqpoint{0.000000in}{0.000000in}}%
\pgfpathlineto{\pgfqpoint{0.000000in}{0.020833in}}%
\pgfusepath{stroke,fill}%
}%
\begin{pgfscope}%
\pgfsys@transformshift{3.616812in}{1.080890in}%
\pgfsys@useobject{currentmarker}{}%
\end{pgfscope}%
\end{pgfscope}%
\begin{pgfscope}%
\pgfsetbuttcap%
\pgfsetroundjoin%
\definecolor{currentfill}{rgb}{0.000000,0.000000,0.000000}%
\pgfsetfillcolor{currentfill}%
\pgfsetlinewidth{0.501875pt}%
\definecolor{currentstroke}{rgb}{0.000000,0.000000,0.000000}%
\pgfsetstrokecolor{currentstroke}%
\pgfsetdash{}{0pt}%
\pgfsys@defobject{currentmarker}{\pgfqpoint{0.000000in}{-0.020833in}}{\pgfqpoint{0.000000in}{0.000000in}}{%
\pgfpathmoveto{\pgfqpoint{0.000000in}{0.000000in}}%
\pgfpathlineto{\pgfqpoint{0.000000in}{-0.020833in}}%
\pgfusepath{stroke,fill}%
}%
\begin{pgfscope}%
\pgfsys@transformshift{3.616812in}{3.227753in}%
\pgfsys@useobject{currentmarker}{}%
\end{pgfscope}%
\end{pgfscope}%
\begin{pgfscope}%
\pgfpathrectangle{\pgfqpoint{0.481681in}{1.080890in}}{\pgfqpoint{5.785672in}{2.146863in}}%
\pgfusepath{clip}%
\pgfsetrectcap%
\pgfsetroundjoin%
\pgfsetlinewidth{0.100375pt}%
\definecolor{currentstroke}{rgb}{0.827451,0.827451,0.827451}%
\pgfsetstrokecolor{currentstroke}%
\pgfsetdash{}{0pt}%
\pgfpathmoveto{\pgfqpoint{3.652510in}{1.080890in}}%
\pgfpathlineto{\pgfqpoint{3.652510in}{3.227753in}}%
\pgfusepath{stroke}%
\end{pgfscope}%
\begin{pgfscope}%
\pgfsetbuttcap%
\pgfsetroundjoin%
\definecolor{currentfill}{rgb}{0.000000,0.000000,0.000000}%
\pgfsetfillcolor{currentfill}%
\pgfsetlinewidth{0.501875pt}%
\definecolor{currentstroke}{rgb}{0.000000,0.000000,0.000000}%
\pgfsetstrokecolor{currentstroke}%
\pgfsetdash{}{0pt}%
\pgfsys@defobject{currentmarker}{\pgfqpoint{0.000000in}{0.000000in}}{\pgfqpoint{0.000000in}{0.020833in}}{%
\pgfpathmoveto{\pgfqpoint{0.000000in}{0.000000in}}%
\pgfpathlineto{\pgfqpoint{0.000000in}{0.020833in}}%
\pgfusepath{stroke,fill}%
}%
\begin{pgfscope}%
\pgfsys@transformshift{3.652510in}{1.080890in}%
\pgfsys@useobject{currentmarker}{}%
\end{pgfscope}%
\end{pgfscope}%
\begin{pgfscope}%
\pgfsetbuttcap%
\pgfsetroundjoin%
\definecolor{currentfill}{rgb}{0.000000,0.000000,0.000000}%
\pgfsetfillcolor{currentfill}%
\pgfsetlinewidth{0.501875pt}%
\definecolor{currentstroke}{rgb}{0.000000,0.000000,0.000000}%
\pgfsetstrokecolor{currentstroke}%
\pgfsetdash{}{0pt}%
\pgfsys@defobject{currentmarker}{\pgfqpoint{0.000000in}{-0.020833in}}{\pgfqpoint{0.000000in}{0.000000in}}{%
\pgfpathmoveto{\pgfqpoint{0.000000in}{0.000000in}}%
\pgfpathlineto{\pgfqpoint{0.000000in}{-0.020833in}}%
\pgfusepath{stroke,fill}%
}%
\begin{pgfscope}%
\pgfsys@transformshift{3.652510in}{3.227753in}%
\pgfsys@useobject{currentmarker}{}%
\end{pgfscope}%
\end{pgfscope}%
\begin{pgfscope}%
\pgfpathrectangle{\pgfqpoint{0.481681in}{1.080890in}}{\pgfqpoint{5.785672in}{2.146863in}}%
\pgfusepath{clip}%
\pgfsetrectcap%
\pgfsetroundjoin%
\pgfsetlinewidth{0.100375pt}%
\definecolor{currentstroke}{rgb}{0.827451,0.827451,0.827451}%
\pgfsetstrokecolor{currentstroke}%
\pgfsetdash{}{0pt}%
\pgfpathmoveto{\pgfqpoint{3.723904in}{1.080890in}}%
\pgfpathlineto{\pgfqpoint{3.723904in}{3.227753in}}%
\pgfusepath{stroke}%
\end{pgfscope}%
\begin{pgfscope}%
\pgfsetbuttcap%
\pgfsetroundjoin%
\definecolor{currentfill}{rgb}{0.000000,0.000000,0.000000}%
\pgfsetfillcolor{currentfill}%
\pgfsetlinewidth{0.501875pt}%
\definecolor{currentstroke}{rgb}{0.000000,0.000000,0.000000}%
\pgfsetstrokecolor{currentstroke}%
\pgfsetdash{}{0pt}%
\pgfsys@defobject{currentmarker}{\pgfqpoint{0.000000in}{0.000000in}}{\pgfqpoint{0.000000in}{0.020833in}}{%
\pgfpathmoveto{\pgfqpoint{0.000000in}{0.000000in}}%
\pgfpathlineto{\pgfqpoint{0.000000in}{0.020833in}}%
\pgfusepath{stroke,fill}%
}%
\begin{pgfscope}%
\pgfsys@transformshift{3.723904in}{1.080890in}%
\pgfsys@useobject{currentmarker}{}%
\end{pgfscope}%
\end{pgfscope}%
\begin{pgfscope}%
\pgfsetbuttcap%
\pgfsetroundjoin%
\definecolor{currentfill}{rgb}{0.000000,0.000000,0.000000}%
\pgfsetfillcolor{currentfill}%
\pgfsetlinewidth{0.501875pt}%
\definecolor{currentstroke}{rgb}{0.000000,0.000000,0.000000}%
\pgfsetstrokecolor{currentstroke}%
\pgfsetdash{}{0pt}%
\pgfsys@defobject{currentmarker}{\pgfqpoint{0.000000in}{-0.020833in}}{\pgfqpoint{0.000000in}{0.000000in}}{%
\pgfpathmoveto{\pgfqpoint{0.000000in}{0.000000in}}%
\pgfpathlineto{\pgfqpoint{0.000000in}{-0.020833in}}%
\pgfusepath{stroke,fill}%
}%
\begin{pgfscope}%
\pgfsys@transformshift{3.723904in}{3.227753in}%
\pgfsys@useobject{currentmarker}{}%
\end{pgfscope}%
\end{pgfscope}%
\begin{pgfscope}%
\pgfpathrectangle{\pgfqpoint{0.481681in}{1.080890in}}{\pgfqpoint{5.785672in}{2.146863in}}%
\pgfusepath{clip}%
\pgfsetrectcap%
\pgfsetroundjoin%
\pgfsetlinewidth{0.100375pt}%
\definecolor{currentstroke}{rgb}{0.827451,0.827451,0.827451}%
\pgfsetstrokecolor{currentstroke}%
\pgfsetdash{}{0pt}%
\pgfpathmoveto{\pgfqpoint{3.759602in}{1.080890in}}%
\pgfpathlineto{\pgfqpoint{3.759602in}{3.227753in}}%
\pgfusepath{stroke}%
\end{pgfscope}%
\begin{pgfscope}%
\pgfsetbuttcap%
\pgfsetroundjoin%
\definecolor{currentfill}{rgb}{0.000000,0.000000,0.000000}%
\pgfsetfillcolor{currentfill}%
\pgfsetlinewidth{0.501875pt}%
\definecolor{currentstroke}{rgb}{0.000000,0.000000,0.000000}%
\pgfsetstrokecolor{currentstroke}%
\pgfsetdash{}{0pt}%
\pgfsys@defobject{currentmarker}{\pgfqpoint{0.000000in}{0.000000in}}{\pgfqpoint{0.000000in}{0.020833in}}{%
\pgfpathmoveto{\pgfqpoint{0.000000in}{0.000000in}}%
\pgfpathlineto{\pgfqpoint{0.000000in}{0.020833in}}%
\pgfusepath{stroke,fill}%
}%
\begin{pgfscope}%
\pgfsys@transformshift{3.759602in}{1.080890in}%
\pgfsys@useobject{currentmarker}{}%
\end{pgfscope}%
\end{pgfscope}%
\begin{pgfscope}%
\pgfsetbuttcap%
\pgfsetroundjoin%
\definecolor{currentfill}{rgb}{0.000000,0.000000,0.000000}%
\pgfsetfillcolor{currentfill}%
\pgfsetlinewidth{0.501875pt}%
\definecolor{currentstroke}{rgb}{0.000000,0.000000,0.000000}%
\pgfsetstrokecolor{currentstroke}%
\pgfsetdash{}{0pt}%
\pgfsys@defobject{currentmarker}{\pgfqpoint{0.000000in}{-0.020833in}}{\pgfqpoint{0.000000in}{0.000000in}}{%
\pgfpathmoveto{\pgfqpoint{0.000000in}{0.000000in}}%
\pgfpathlineto{\pgfqpoint{0.000000in}{-0.020833in}}%
\pgfusepath{stroke,fill}%
}%
\begin{pgfscope}%
\pgfsys@transformshift{3.759602in}{3.227753in}%
\pgfsys@useobject{currentmarker}{}%
\end{pgfscope}%
\end{pgfscope}%
\begin{pgfscope}%
\pgfpathrectangle{\pgfqpoint{0.481681in}{1.080890in}}{\pgfqpoint{5.785672in}{2.146863in}}%
\pgfusepath{clip}%
\pgfsetrectcap%
\pgfsetroundjoin%
\pgfsetlinewidth{0.100375pt}%
\definecolor{currentstroke}{rgb}{0.827451,0.827451,0.827451}%
\pgfsetstrokecolor{currentstroke}%
\pgfsetdash{}{0pt}%
\pgfpathmoveto{\pgfqpoint{3.795299in}{1.080890in}}%
\pgfpathlineto{\pgfqpoint{3.795299in}{3.227753in}}%
\pgfusepath{stroke}%
\end{pgfscope}%
\begin{pgfscope}%
\pgfsetbuttcap%
\pgfsetroundjoin%
\definecolor{currentfill}{rgb}{0.000000,0.000000,0.000000}%
\pgfsetfillcolor{currentfill}%
\pgfsetlinewidth{0.501875pt}%
\definecolor{currentstroke}{rgb}{0.000000,0.000000,0.000000}%
\pgfsetstrokecolor{currentstroke}%
\pgfsetdash{}{0pt}%
\pgfsys@defobject{currentmarker}{\pgfqpoint{0.000000in}{0.000000in}}{\pgfqpoint{0.000000in}{0.020833in}}{%
\pgfpathmoveto{\pgfqpoint{0.000000in}{0.000000in}}%
\pgfpathlineto{\pgfqpoint{0.000000in}{0.020833in}}%
\pgfusepath{stroke,fill}%
}%
\begin{pgfscope}%
\pgfsys@transformshift{3.795299in}{1.080890in}%
\pgfsys@useobject{currentmarker}{}%
\end{pgfscope}%
\end{pgfscope}%
\begin{pgfscope}%
\pgfsetbuttcap%
\pgfsetroundjoin%
\definecolor{currentfill}{rgb}{0.000000,0.000000,0.000000}%
\pgfsetfillcolor{currentfill}%
\pgfsetlinewidth{0.501875pt}%
\definecolor{currentstroke}{rgb}{0.000000,0.000000,0.000000}%
\pgfsetstrokecolor{currentstroke}%
\pgfsetdash{}{0pt}%
\pgfsys@defobject{currentmarker}{\pgfqpoint{0.000000in}{-0.020833in}}{\pgfqpoint{0.000000in}{0.000000in}}{%
\pgfpathmoveto{\pgfqpoint{0.000000in}{0.000000in}}%
\pgfpathlineto{\pgfqpoint{0.000000in}{-0.020833in}}%
\pgfusepath{stroke,fill}%
}%
\begin{pgfscope}%
\pgfsys@transformshift{3.795299in}{3.227753in}%
\pgfsys@useobject{currentmarker}{}%
\end{pgfscope}%
\end{pgfscope}%
\begin{pgfscope}%
\pgfpathrectangle{\pgfqpoint{0.481681in}{1.080890in}}{\pgfqpoint{5.785672in}{2.146863in}}%
\pgfusepath{clip}%
\pgfsetrectcap%
\pgfsetroundjoin%
\pgfsetlinewidth{0.100375pt}%
\definecolor{currentstroke}{rgb}{0.827451,0.827451,0.827451}%
\pgfsetstrokecolor{currentstroke}%
\pgfsetdash{}{0pt}%
\pgfpathmoveto{\pgfqpoint{3.830996in}{1.080890in}}%
\pgfpathlineto{\pgfqpoint{3.830996in}{3.227753in}}%
\pgfusepath{stroke}%
\end{pgfscope}%
\begin{pgfscope}%
\pgfsetbuttcap%
\pgfsetroundjoin%
\definecolor{currentfill}{rgb}{0.000000,0.000000,0.000000}%
\pgfsetfillcolor{currentfill}%
\pgfsetlinewidth{0.501875pt}%
\definecolor{currentstroke}{rgb}{0.000000,0.000000,0.000000}%
\pgfsetstrokecolor{currentstroke}%
\pgfsetdash{}{0pt}%
\pgfsys@defobject{currentmarker}{\pgfqpoint{0.000000in}{0.000000in}}{\pgfqpoint{0.000000in}{0.020833in}}{%
\pgfpathmoveto{\pgfqpoint{0.000000in}{0.000000in}}%
\pgfpathlineto{\pgfqpoint{0.000000in}{0.020833in}}%
\pgfusepath{stroke,fill}%
}%
\begin{pgfscope}%
\pgfsys@transformshift{3.830996in}{1.080890in}%
\pgfsys@useobject{currentmarker}{}%
\end{pgfscope}%
\end{pgfscope}%
\begin{pgfscope}%
\pgfsetbuttcap%
\pgfsetroundjoin%
\definecolor{currentfill}{rgb}{0.000000,0.000000,0.000000}%
\pgfsetfillcolor{currentfill}%
\pgfsetlinewidth{0.501875pt}%
\definecolor{currentstroke}{rgb}{0.000000,0.000000,0.000000}%
\pgfsetstrokecolor{currentstroke}%
\pgfsetdash{}{0pt}%
\pgfsys@defobject{currentmarker}{\pgfqpoint{0.000000in}{-0.020833in}}{\pgfqpoint{0.000000in}{0.000000in}}{%
\pgfpathmoveto{\pgfqpoint{0.000000in}{0.000000in}}%
\pgfpathlineto{\pgfqpoint{0.000000in}{-0.020833in}}%
\pgfusepath{stroke,fill}%
}%
\begin{pgfscope}%
\pgfsys@transformshift{3.830996in}{3.227753in}%
\pgfsys@useobject{currentmarker}{}%
\end{pgfscope}%
\end{pgfscope}%
\begin{pgfscope}%
\pgfpathrectangle{\pgfqpoint{0.481681in}{1.080890in}}{\pgfqpoint{5.785672in}{2.146863in}}%
\pgfusepath{clip}%
\pgfsetrectcap%
\pgfsetroundjoin%
\pgfsetlinewidth{0.100375pt}%
\definecolor{currentstroke}{rgb}{0.827451,0.827451,0.827451}%
\pgfsetstrokecolor{currentstroke}%
\pgfsetdash{}{0pt}%
\pgfpathmoveto{\pgfqpoint{3.866694in}{1.080890in}}%
\pgfpathlineto{\pgfqpoint{3.866694in}{3.227753in}}%
\pgfusepath{stroke}%
\end{pgfscope}%
\begin{pgfscope}%
\pgfsetbuttcap%
\pgfsetroundjoin%
\definecolor{currentfill}{rgb}{0.000000,0.000000,0.000000}%
\pgfsetfillcolor{currentfill}%
\pgfsetlinewidth{0.501875pt}%
\definecolor{currentstroke}{rgb}{0.000000,0.000000,0.000000}%
\pgfsetstrokecolor{currentstroke}%
\pgfsetdash{}{0pt}%
\pgfsys@defobject{currentmarker}{\pgfqpoint{0.000000in}{0.000000in}}{\pgfqpoint{0.000000in}{0.020833in}}{%
\pgfpathmoveto{\pgfqpoint{0.000000in}{0.000000in}}%
\pgfpathlineto{\pgfqpoint{0.000000in}{0.020833in}}%
\pgfusepath{stroke,fill}%
}%
\begin{pgfscope}%
\pgfsys@transformshift{3.866694in}{1.080890in}%
\pgfsys@useobject{currentmarker}{}%
\end{pgfscope}%
\end{pgfscope}%
\begin{pgfscope}%
\pgfsetbuttcap%
\pgfsetroundjoin%
\definecolor{currentfill}{rgb}{0.000000,0.000000,0.000000}%
\pgfsetfillcolor{currentfill}%
\pgfsetlinewidth{0.501875pt}%
\definecolor{currentstroke}{rgb}{0.000000,0.000000,0.000000}%
\pgfsetstrokecolor{currentstroke}%
\pgfsetdash{}{0pt}%
\pgfsys@defobject{currentmarker}{\pgfqpoint{0.000000in}{-0.020833in}}{\pgfqpoint{0.000000in}{0.000000in}}{%
\pgfpathmoveto{\pgfqpoint{0.000000in}{0.000000in}}%
\pgfpathlineto{\pgfqpoint{0.000000in}{-0.020833in}}%
\pgfusepath{stroke,fill}%
}%
\begin{pgfscope}%
\pgfsys@transformshift{3.866694in}{3.227753in}%
\pgfsys@useobject{currentmarker}{}%
\end{pgfscope}%
\end{pgfscope}%
\begin{pgfscope}%
\pgfpathrectangle{\pgfqpoint{0.481681in}{1.080890in}}{\pgfqpoint{5.785672in}{2.146863in}}%
\pgfusepath{clip}%
\pgfsetrectcap%
\pgfsetroundjoin%
\pgfsetlinewidth{0.100375pt}%
\definecolor{currentstroke}{rgb}{0.827451,0.827451,0.827451}%
\pgfsetstrokecolor{currentstroke}%
\pgfsetdash{}{0pt}%
\pgfpathmoveto{\pgfqpoint{3.902391in}{1.080890in}}%
\pgfpathlineto{\pgfqpoint{3.902391in}{3.227753in}}%
\pgfusepath{stroke}%
\end{pgfscope}%
\begin{pgfscope}%
\pgfsetbuttcap%
\pgfsetroundjoin%
\definecolor{currentfill}{rgb}{0.000000,0.000000,0.000000}%
\pgfsetfillcolor{currentfill}%
\pgfsetlinewidth{0.501875pt}%
\definecolor{currentstroke}{rgb}{0.000000,0.000000,0.000000}%
\pgfsetstrokecolor{currentstroke}%
\pgfsetdash{}{0pt}%
\pgfsys@defobject{currentmarker}{\pgfqpoint{0.000000in}{0.000000in}}{\pgfqpoint{0.000000in}{0.020833in}}{%
\pgfpathmoveto{\pgfqpoint{0.000000in}{0.000000in}}%
\pgfpathlineto{\pgfqpoint{0.000000in}{0.020833in}}%
\pgfusepath{stroke,fill}%
}%
\begin{pgfscope}%
\pgfsys@transformshift{3.902391in}{1.080890in}%
\pgfsys@useobject{currentmarker}{}%
\end{pgfscope}%
\end{pgfscope}%
\begin{pgfscope}%
\pgfsetbuttcap%
\pgfsetroundjoin%
\definecolor{currentfill}{rgb}{0.000000,0.000000,0.000000}%
\pgfsetfillcolor{currentfill}%
\pgfsetlinewidth{0.501875pt}%
\definecolor{currentstroke}{rgb}{0.000000,0.000000,0.000000}%
\pgfsetstrokecolor{currentstroke}%
\pgfsetdash{}{0pt}%
\pgfsys@defobject{currentmarker}{\pgfqpoint{0.000000in}{-0.020833in}}{\pgfqpoint{0.000000in}{0.000000in}}{%
\pgfpathmoveto{\pgfqpoint{0.000000in}{0.000000in}}%
\pgfpathlineto{\pgfqpoint{0.000000in}{-0.020833in}}%
\pgfusepath{stroke,fill}%
}%
\begin{pgfscope}%
\pgfsys@transformshift{3.902391in}{3.227753in}%
\pgfsys@useobject{currentmarker}{}%
\end{pgfscope}%
\end{pgfscope}%
\begin{pgfscope}%
\pgfpathrectangle{\pgfqpoint{0.481681in}{1.080890in}}{\pgfqpoint{5.785672in}{2.146863in}}%
\pgfusepath{clip}%
\pgfsetrectcap%
\pgfsetroundjoin%
\pgfsetlinewidth{0.100375pt}%
\definecolor{currentstroke}{rgb}{0.827451,0.827451,0.827451}%
\pgfsetstrokecolor{currentstroke}%
\pgfsetdash{}{0pt}%
\pgfpathmoveto{\pgfqpoint{3.938088in}{1.080890in}}%
\pgfpathlineto{\pgfqpoint{3.938088in}{3.227753in}}%
\pgfusepath{stroke}%
\end{pgfscope}%
\begin{pgfscope}%
\pgfsetbuttcap%
\pgfsetroundjoin%
\definecolor{currentfill}{rgb}{0.000000,0.000000,0.000000}%
\pgfsetfillcolor{currentfill}%
\pgfsetlinewidth{0.501875pt}%
\definecolor{currentstroke}{rgb}{0.000000,0.000000,0.000000}%
\pgfsetstrokecolor{currentstroke}%
\pgfsetdash{}{0pt}%
\pgfsys@defobject{currentmarker}{\pgfqpoint{0.000000in}{0.000000in}}{\pgfqpoint{0.000000in}{0.020833in}}{%
\pgfpathmoveto{\pgfqpoint{0.000000in}{0.000000in}}%
\pgfpathlineto{\pgfqpoint{0.000000in}{0.020833in}}%
\pgfusepath{stroke,fill}%
}%
\begin{pgfscope}%
\pgfsys@transformshift{3.938088in}{1.080890in}%
\pgfsys@useobject{currentmarker}{}%
\end{pgfscope}%
\end{pgfscope}%
\begin{pgfscope}%
\pgfsetbuttcap%
\pgfsetroundjoin%
\definecolor{currentfill}{rgb}{0.000000,0.000000,0.000000}%
\pgfsetfillcolor{currentfill}%
\pgfsetlinewidth{0.501875pt}%
\definecolor{currentstroke}{rgb}{0.000000,0.000000,0.000000}%
\pgfsetstrokecolor{currentstroke}%
\pgfsetdash{}{0pt}%
\pgfsys@defobject{currentmarker}{\pgfqpoint{0.000000in}{-0.020833in}}{\pgfqpoint{0.000000in}{0.000000in}}{%
\pgfpathmoveto{\pgfqpoint{0.000000in}{0.000000in}}%
\pgfpathlineto{\pgfqpoint{0.000000in}{-0.020833in}}%
\pgfusepath{stroke,fill}%
}%
\begin{pgfscope}%
\pgfsys@transformshift{3.938088in}{3.227753in}%
\pgfsys@useobject{currentmarker}{}%
\end{pgfscope}%
\end{pgfscope}%
\begin{pgfscope}%
\pgfpathrectangle{\pgfqpoint{0.481681in}{1.080890in}}{\pgfqpoint{5.785672in}{2.146863in}}%
\pgfusepath{clip}%
\pgfsetrectcap%
\pgfsetroundjoin%
\pgfsetlinewidth{0.100375pt}%
\definecolor{currentstroke}{rgb}{0.827451,0.827451,0.827451}%
\pgfsetstrokecolor{currentstroke}%
\pgfsetdash{}{0pt}%
\pgfpathmoveto{\pgfqpoint{3.973786in}{1.080890in}}%
\pgfpathlineto{\pgfqpoint{3.973786in}{3.227753in}}%
\pgfusepath{stroke}%
\end{pgfscope}%
\begin{pgfscope}%
\pgfsetbuttcap%
\pgfsetroundjoin%
\definecolor{currentfill}{rgb}{0.000000,0.000000,0.000000}%
\pgfsetfillcolor{currentfill}%
\pgfsetlinewidth{0.501875pt}%
\definecolor{currentstroke}{rgb}{0.000000,0.000000,0.000000}%
\pgfsetstrokecolor{currentstroke}%
\pgfsetdash{}{0pt}%
\pgfsys@defobject{currentmarker}{\pgfqpoint{0.000000in}{0.000000in}}{\pgfqpoint{0.000000in}{0.020833in}}{%
\pgfpathmoveto{\pgfqpoint{0.000000in}{0.000000in}}%
\pgfpathlineto{\pgfqpoint{0.000000in}{0.020833in}}%
\pgfusepath{stroke,fill}%
}%
\begin{pgfscope}%
\pgfsys@transformshift{3.973786in}{1.080890in}%
\pgfsys@useobject{currentmarker}{}%
\end{pgfscope}%
\end{pgfscope}%
\begin{pgfscope}%
\pgfsetbuttcap%
\pgfsetroundjoin%
\definecolor{currentfill}{rgb}{0.000000,0.000000,0.000000}%
\pgfsetfillcolor{currentfill}%
\pgfsetlinewidth{0.501875pt}%
\definecolor{currentstroke}{rgb}{0.000000,0.000000,0.000000}%
\pgfsetstrokecolor{currentstroke}%
\pgfsetdash{}{0pt}%
\pgfsys@defobject{currentmarker}{\pgfqpoint{0.000000in}{-0.020833in}}{\pgfqpoint{0.000000in}{0.000000in}}{%
\pgfpathmoveto{\pgfqpoint{0.000000in}{0.000000in}}%
\pgfpathlineto{\pgfqpoint{0.000000in}{-0.020833in}}%
\pgfusepath{stroke,fill}%
}%
\begin{pgfscope}%
\pgfsys@transformshift{3.973786in}{3.227753in}%
\pgfsys@useobject{currentmarker}{}%
\end{pgfscope}%
\end{pgfscope}%
\begin{pgfscope}%
\pgfpathrectangle{\pgfqpoint{0.481681in}{1.080890in}}{\pgfqpoint{5.785672in}{2.146863in}}%
\pgfusepath{clip}%
\pgfsetrectcap%
\pgfsetroundjoin%
\pgfsetlinewidth{0.100375pt}%
\definecolor{currentstroke}{rgb}{0.827451,0.827451,0.827451}%
\pgfsetstrokecolor{currentstroke}%
\pgfsetdash{}{0pt}%
\pgfpathmoveto{\pgfqpoint{4.009483in}{1.080890in}}%
\pgfpathlineto{\pgfqpoint{4.009483in}{3.227753in}}%
\pgfusepath{stroke}%
\end{pgfscope}%
\begin{pgfscope}%
\pgfsetbuttcap%
\pgfsetroundjoin%
\definecolor{currentfill}{rgb}{0.000000,0.000000,0.000000}%
\pgfsetfillcolor{currentfill}%
\pgfsetlinewidth{0.501875pt}%
\definecolor{currentstroke}{rgb}{0.000000,0.000000,0.000000}%
\pgfsetstrokecolor{currentstroke}%
\pgfsetdash{}{0pt}%
\pgfsys@defobject{currentmarker}{\pgfqpoint{0.000000in}{0.000000in}}{\pgfqpoint{0.000000in}{0.020833in}}{%
\pgfpathmoveto{\pgfqpoint{0.000000in}{0.000000in}}%
\pgfpathlineto{\pgfqpoint{0.000000in}{0.020833in}}%
\pgfusepath{stroke,fill}%
}%
\begin{pgfscope}%
\pgfsys@transformshift{4.009483in}{1.080890in}%
\pgfsys@useobject{currentmarker}{}%
\end{pgfscope}%
\end{pgfscope}%
\begin{pgfscope}%
\pgfsetbuttcap%
\pgfsetroundjoin%
\definecolor{currentfill}{rgb}{0.000000,0.000000,0.000000}%
\pgfsetfillcolor{currentfill}%
\pgfsetlinewidth{0.501875pt}%
\definecolor{currentstroke}{rgb}{0.000000,0.000000,0.000000}%
\pgfsetstrokecolor{currentstroke}%
\pgfsetdash{}{0pt}%
\pgfsys@defobject{currentmarker}{\pgfqpoint{0.000000in}{-0.020833in}}{\pgfqpoint{0.000000in}{0.000000in}}{%
\pgfpathmoveto{\pgfqpoint{0.000000in}{0.000000in}}%
\pgfpathlineto{\pgfqpoint{0.000000in}{-0.020833in}}%
\pgfusepath{stroke,fill}%
}%
\begin{pgfscope}%
\pgfsys@transformshift{4.009483in}{3.227753in}%
\pgfsys@useobject{currentmarker}{}%
\end{pgfscope}%
\end{pgfscope}%
\begin{pgfscope}%
\pgfpathrectangle{\pgfqpoint{0.481681in}{1.080890in}}{\pgfqpoint{5.785672in}{2.146863in}}%
\pgfusepath{clip}%
\pgfsetrectcap%
\pgfsetroundjoin%
\pgfsetlinewidth{0.100375pt}%
\definecolor{currentstroke}{rgb}{0.827451,0.827451,0.827451}%
\pgfsetstrokecolor{currentstroke}%
\pgfsetdash{}{0pt}%
\pgfpathmoveto{\pgfqpoint{4.045180in}{1.080890in}}%
\pgfpathlineto{\pgfqpoint{4.045180in}{3.227753in}}%
\pgfusepath{stroke}%
\end{pgfscope}%
\begin{pgfscope}%
\pgfsetbuttcap%
\pgfsetroundjoin%
\definecolor{currentfill}{rgb}{0.000000,0.000000,0.000000}%
\pgfsetfillcolor{currentfill}%
\pgfsetlinewidth{0.501875pt}%
\definecolor{currentstroke}{rgb}{0.000000,0.000000,0.000000}%
\pgfsetstrokecolor{currentstroke}%
\pgfsetdash{}{0pt}%
\pgfsys@defobject{currentmarker}{\pgfqpoint{0.000000in}{0.000000in}}{\pgfqpoint{0.000000in}{0.020833in}}{%
\pgfpathmoveto{\pgfqpoint{0.000000in}{0.000000in}}%
\pgfpathlineto{\pgfqpoint{0.000000in}{0.020833in}}%
\pgfusepath{stroke,fill}%
}%
\begin{pgfscope}%
\pgfsys@transformshift{4.045180in}{1.080890in}%
\pgfsys@useobject{currentmarker}{}%
\end{pgfscope}%
\end{pgfscope}%
\begin{pgfscope}%
\pgfsetbuttcap%
\pgfsetroundjoin%
\definecolor{currentfill}{rgb}{0.000000,0.000000,0.000000}%
\pgfsetfillcolor{currentfill}%
\pgfsetlinewidth{0.501875pt}%
\definecolor{currentstroke}{rgb}{0.000000,0.000000,0.000000}%
\pgfsetstrokecolor{currentstroke}%
\pgfsetdash{}{0pt}%
\pgfsys@defobject{currentmarker}{\pgfqpoint{0.000000in}{-0.020833in}}{\pgfqpoint{0.000000in}{0.000000in}}{%
\pgfpathmoveto{\pgfqpoint{0.000000in}{0.000000in}}%
\pgfpathlineto{\pgfqpoint{0.000000in}{-0.020833in}}%
\pgfusepath{stroke,fill}%
}%
\begin{pgfscope}%
\pgfsys@transformshift{4.045180in}{3.227753in}%
\pgfsys@useobject{currentmarker}{}%
\end{pgfscope}%
\end{pgfscope}%
\begin{pgfscope}%
\pgfpathrectangle{\pgfqpoint{0.481681in}{1.080890in}}{\pgfqpoint{5.785672in}{2.146863in}}%
\pgfusepath{clip}%
\pgfsetrectcap%
\pgfsetroundjoin%
\pgfsetlinewidth{0.100375pt}%
\definecolor{currentstroke}{rgb}{0.827451,0.827451,0.827451}%
\pgfsetstrokecolor{currentstroke}%
\pgfsetdash{}{0pt}%
\pgfpathmoveto{\pgfqpoint{4.080878in}{1.080890in}}%
\pgfpathlineto{\pgfqpoint{4.080878in}{3.227753in}}%
\pgfusepath{stroke}%
\end{pgfscope}%
\begin{pgfscope}%
\pgfsetbuttcap%
\pgfsetroundjoin%
\definecolor{currentfill}{rgb}{0.000000,0.000000,0.000000}%
\pgfsetfillcolor{currentfill}%
\pgfsetlinewidth{0.501875pt}%
\definecolor{currentstroke}{rgb}{0.000000,0.000000,0.000000}%
\pgfsetstrokecolor{currentstroke}%
\pgfsetdash{}{0pt}%
\pgfsys@defobject{currentmarker}{\pgfqpoint{0.000000in}{0.000000in}}{\pgfqpoint{0.000000in}{0.020833in}}{%
\pgfpathmoveto{\pgfqpoint{0.000000in}{0.000000in}}%
\pgfpathlineto{\pgfqpoint{0.000000in}{0.020833in}}%
\pgfusepath{stroke,fill}%
}%
\begin{pgfscope}%
\pgfsys@transformshift{4.080878in}{1.080890in}%
\pgfsys@useobject{currentmarker}{}%
\end{pgfscope}%
\end{pgfscope}%
\begin{pgfscope}%
\pgfsetbuttcap%
\pgfsetroundjoin%
\definecolor{currentfill}{rgb}{0.000000,0.000000,0.000000}%
\pgfsetfillcolor{currentfill}%
\pgfsetlinewidth{0.501875pt}%
\definecolor{currentstroke}{rgb}{0.000000,0.000000,0.000000}%
\pgfsetstrokecolor{currentstroke}%
\pgfsetdash{}{0pt}%
\pgfsys@defobject{currentmarker}{\pgfqpoint{0.000000in}{-0.020833in}}{\pgfqpoint{0.000000in}{0.000000in}}{%
\pgfpathmoveto{\pgfqpoint{0.000000in}{0.000000in}}%
\pgfpathlineto{\pgfqpoint{0.000000in}{-0.020833in}}%
\pgfusepath{stroke,fill}%
}%
\begin{pgfscope}%
\pgfsys@transformshift{4.080878in}{3.227753in}%
\pgfsys@useobject{currentmarker}{}%
\end{pgfscope}%
\end{pgfscope}%
\begin{pgfscope}%
\pgfpathrectangle{\pgfqpoint{0.481681in}{1.080890in}}{\pgfqpoint{5.785672in}{2.146863in}}%
\pgfusepath{clip}%
\pgfsetrectcap%
\pgfsetroundjoin%
\pgfsetlinewidth{0.100375pt}%
\definecolor{currentstroke}{rgb}{0.827451,0.827451,0.827451}%
\pgfsetstrokecolor{currentstroke}%
\pgfsetdash{}{0pt}%
\pgfpathmoveto{\pgfqpoint{4.152272in}{1.080890in}}%
\pgfpathlineto{\pgfqpoint{4.152272in}{3.227753in}}%
\pgfusepath{stroke}%
\end{pgfscope}%
\begin{pgfscope}%
\pgfsetbuttcap%
\pgfsetroundjoin%
\definecolor{currentfill}{rgb}{0.000000,0.000000,0.000000}%
\pgfsetfillcolor{currentfill}%
\pgfsetlinewidth{0.501875pt}%
\definecolor{currentstroke}{rgb}{0.000000,0.000000,0.000000}%
\pgfsetstrokecolor{currentstroke}%
\pgfsetdash{}{0pt}%
\pgfsys@defobject{currentmarker}{\pgfqpoint{0.000000in}{0.000000in}}{\pgfqpoint{0.000000in}{0.020833in}}{%
\pgfpathmoveto{\pgfqpoint{0.000000in}{0.000000in}}%
\pgfpathlineto{\pgfqpoint{0.000000in}{0.020833in}}%
\pgfusepath{stroke,fill}%
}%
\begin{pgfscope}%
\pgfsys@transformshift{4.152272in}{1.080890in}%
\pgfsys@useobject{currentmarker}{}%
\end{pgfscope}%
\end{pgfscope}%
\begin{pgfscope}%
\pgfsetbuttcap%
\pgfsetroundjoin%
\definecolor{currentfill}{rgb}{0.000000,0.000000,0.000000}%
\pgfsetfillcolor{currentfill}%
\pgfsetlinewidth{0.501875pt}%
\definecolor{currentstroke}{rgb}{0.000000,0.000000,0.000000}%
\pgfsetstrokecolor{currentstroke}%
\pgfsetdash{}{0pt}%
\pgfsys@defobject{currentmarker}{\pgfqpoint{0.000000in}{-0.020833in}}{\pgfqpoint{0.000000in}{0.000000in}}{%
\pgfpathmoveto{\pgfqpoint{0.000000in}{0.000000in}}%
\pgfpathlineto{\pgfqpoint{0.000000in}{-0.020833in}}%
\pgfusepath{stroke,fill}%
}%
\begin{pgfscope}%
\pgfsys@transformshift{4.152272in}{3.227753in}%
\pgfsys@useobject{currentmarker}{}%
\end{pgfscope}%
\end{pgfscope}%
\begin{pgfscope}%
\pgfpathrectangle{\pgfqpoint{0.481681in}{1.080890in}}{\pgfqpoint{5.785672in}{2.146863in}}%
\pgfusepath{clip}%
\pgfsetrectcap%
\pgfsetroundjoin%
\pgfsetlinewidth{0.100375pt}%
\definecolor{currentstroke}{rgb}{0.827451,0.827451,0.827451}%
\pgfsetstrokecolor{currentstroke}%
\pgfsetdash{}{0pt}%
\pgfpathmoveto{\pgfqpoint{4.187970in}{1.080890in}}%
\pgfpathlineto{\pgfqpoint{4.187970in}{3.227753in}}%
\pgfusepath{stroke}%
\end{pgfscope}%
\begin{pgfscope}%
\pgfsetbuttcap%
\pgfsetroundjoin%
\definecolor{currentfill}{rgb}{0.000000,0.000000,0.000000}%
\pgfsetfillcolor{currentfill}%
\pgfsetlinewidth{0.501875pt}%
\definecolor{currentstroke}{rgb}{0.000000,0.000000,0.000000}%
\pgfsetstrokecolor{currentstroke}%
\pgfsetdash{}{0pt}%
\pgfsys@defobject{currentmarker}{\pgfqpoint{0.000000in}{0.000000in}}{\pgfqpoint{0.000000in}{0.020833in}}{%
\pgfpathmoveto{\pgfqpoint{0.000000in}{0.000000in}}%
\pgfpathlineto{\pgfqpoint{0.000000in}{0.020833in}}%
\pgfusepath{stroke,fill}%
}%
\begin{pgfscope}%
\pgfsys@transformshift{4.187970in}{1.080890in}%
\pgfsys@useobject{currentmarker}{}%
\end{pgfscope}%
\end{pgfscope}%
\begin{pgfscope}%
\pgfsetbuttcap%
\pgfsetroundjoin%
\definecolor{currentfill}{rgb}{0.000000,0.000000,0.000000}%
\pgfsetfillcolor{currentfill}%
\pgfsetlinewidth{0.501875pt}%
\definecolor{currentstroke}{rgb}{0.000000,0.000000,0.000000}%
\pgfsetstrokecolor{currentstroke}%
\pgfsetdash{}{0pt}%
\pgfsys@defobject{currentmarker}{\pgfqpoint{0.000000in}{-0.020833in}}{\pgfqpoint{0.000000in}{0.000000in}}{%
\pgfpathmoveto{\pgfqpoint{0.000000in}{0.000000in}}%
\pgfpathlineto{\pgfqpoint{0.000000in}{-0.020833in}}%
\pgfusepath{stroke,fill}%
}%
\begin{pgfscope}%
\pgfsys@transformshift{4.187970in}{3.227753in}%
\pgfsys@useobject{currentmarker}{}%
\end{pgfscope}%
\end{pgfscope}%
\begin{pgfscope}%
\pgfpathrectangle{\pgfqpoint{0.481681in}{1.080890in}}{\pgfqpoint{5.785672in}{2.146863in}}%
\pgfusepath{clip}%
\pgfsetrectcap%
\pgfsetroundjoin%
\pgfsetlinewidth{0.100375pt}%
\definecolor{currentstroke}{rgb}{0.827451,0.827451,0.827451}%
\pgfsetstrokecolor{currentstroke}%
\pgfsetdash{}{0pt}%
\pgfpathmoveto{\pgfqpoint{4.223667in}{1.080890in}}%
\pgfpathlineto{\pgfqpoint{4.223667in}{3.227753in}}%
\pgfusepath{stroke}%
\end{pgfscope}%
\begin{pgfscope}%
\pgfsetbuttcap%
\pgfsetroundjoin%
\definecolor{currentfill}{rgb}{0.000000,0.000000,0.000000}%
\pgfsetfillcolor{currentfill}%
\pgfsetlinewidth{0.501875pt}%
\definecolor{currentstroke}{rgb}{0.000000,0.000000,0.000000}%
\pgfsetstrokecolor{currentstroke}%
\pgfsetdash{}{0pt}%
\pgfsys@defobject{currentmarker}{\pgfqpoint{0.000000in}{0.000000in}}{\pgfqpoint{0.000000in}{0.020833in}}{%
\pgfpathmoveto{\pgfqpoint{0.000000in}{0.000000in}}%
\pgfpathlineto{\pgfqpoint{0.000000in}{0.020833in}}%
\pgfusepath{stroke,fill}%
}%
\begin{pgfscope}%
\pgfsys@transformshift{4.223667in}{1.080890in}%
\pgfsys@useobject{currentmarker}{}%
\end{pgfscope}%
\end{pgfscope}%
\begin{pgfscope}%
\pgfsetbuttcap%
\pgfsetroundjoin%
\definecolor{currentfill}{rgb}{0.000000,0.000000,0.000000}%
\pgfsetfillcolor{currentfill}%
\pgfsetlinewidth{0.501875pt}%
\definecolor{currentstroke}{rgb}{0.000000,0.000000,0.000000}%
\pgfsetstrokecolor{currentstroke}%
\pgfsetdash{}{0pt}%
\pgfsys@defobject{currentmarker}{\pgfqpoint{0.000000in}{-0.020833in}}{\pgfqpoint{0.000000in}{0.000000in}}{%
\pgfpathmoveto{\pgfqpoint{0.000000in}{0.000000in}}%
\pgfpathlineto{\pgfqpoint{0.000000in}{-0.020833in}}%
\pgfusepath{stroke,fill}%
}%
\begin{pgfscope}%
\pgfsys@transformshift{4.223667in}{3.227753in}%
\pgfsys@useobject{currentmarker}{}%
\end{pgfscope}%
\end{pgfscope}%
\begin{pgfscope}%
\pgfpathrectangle{\pgfqpoint{0.481681in}{1.080890in}}{\pgfqpoint{5.785672in}{2.146863in}}%
\pgfusepath{clip}%
\pgfsetrectcap%
\pgfsetroundjoin%
\pgfsetlinewidth{0.100375pt}%
\definecolor{currentstroke}{rgb}{0.827451,0.827451,0.827451}%
\pgfsetstrokecolor{currentstroke}%
\pgfsetdash{}{0pt}%
\pgfpathmoveto{\pgfqpoint{4.259364in}{1.080890in}}%
\pgfpathlineto{\pgfqpoint{4.259364in}{3.227753in}}%
\pgfusepath{stroke}%
\end{pgfscope}%
\begin{pgfscope}%
\pgfsetbuttcap%
\pgfsetroundjoin%
\definecolor{currentfill}{rgb}{0.000000,0.000000,0.000000}%
\pgfsetfillcolor{currentfill}%
\pgfsetlinewidth{0.501875pt}%
\definecolor{currentstroke}{rgb}{0.000000,0.000000,0.000000}%
\pgfsetstrokecolor{currentstroke}%
\pgfsetdash{}{0pt}%
\pgfsys@defobject{currentmarker}{\pgfqpoint{0.000000in}{0.000000in}}{\pgfqpoint{0.000000in}{0.020833in}}{%
\pgfpathmoveto{\pgfqpoint{0.000000in}{0.000000in}}%
\pgfpathlineto{\pgfqpoint{0.000000in}{0.020833in}}%
\pgfusepath{stroke,fill}%
}%
\begin{pgfscope}%
\pgfsys@transformshift{4.259364in}{1.080890in}%
\pgfsys@useobject{currentmarker}{}%
\end{pgfscope}%
\end{pgfscope}%
\begin{pgfscope}%
\pgfsetbuttcap%
\pgfsetroundjoin%
\definecolor{currentfill}{rgb}{0.000000,0.000000,0.000000}%
\pgfsetfillcolor{currentfill}%
\pgfsetlinewidth{0.501875pt}%
\definecolor{currentstroke}{rgb}{0.000000,0.000000,0.000000}%
\pgfsetstrokecolor{currentstroke}%
\pgfsetdash{}{0pt}%
\pgfsys@defobject{currentmarker}{\pgfqpoint{0.000000in}{-0.020833in}}{\pgfqpoint{0.000000in}{0.000000in}}{%
\pgfpathmoveto{\pgfqpoint{0.000000in}{0.000000in}}%
\pgfpathlineto{\pgfqpoint{0.000000in}{-0.020833in}}%
\pgfusepath{stroke,fill}%
}%
\begin{pgfscope}%
\pgfsys@transformshift{4.259364in}{3.227753in}%
\pgfsys@useobject{currentmarker}{}%
\end{pgfscope}%
\end{pgfscope}%
\begin{pgfscope}%
\pgfpathrectangle{\pgfqpoint{0.481681in}{1.080890in}}{\pgfqpoint{5.785672in}{2.146863in}}%
\pgfusepath{clip}%
\pgfsetrectcap%
\pgfsetroundjoin%
\pgfsetlinewidth{0.100375pt}%
\definecolor{currentstroke}{rgb}{0.827451,0.827451,0.827451}%
\pgfsetstrokecolor{currentstroke}%
\pgfsetdash{}{0pt}%
\pgfpathmoveto{\pgfqpoint{4.295061in}{1.080890in}}%
\pgfpathlineto{\pgfqpoint{4.295061in}{3.227753in}}%
\pgfusepath{stroke}%
\end{pgfscope}%
\begin{pgfscope}%
\pgfsetbuttcap%
\pgfsetroundjoin%
\definecolor{currentfill}{rgb}{0.000000,0.000000,0.000000}%
\pgfsetfillcolor{currentfill}%
\pgfsetlinewidth{0.501875pt}%
\definecolor{currentstroke}{rgb}{0.000000,0.000000,0.000000}%
\pgfsetstrokecolor{currentstroke}%
\pgfsetdash{}{0pt}%
\pgfsys@defobject{currentmarker}{\pgfqpoint{0.000000in}{0.000000in}}{\pgfqpoint{0.000000in}{0.020833in}}{%
\pgfpathmoveto{\pgfqpoint{0.000000in}{0.000000in}}%
\pgfpathlineto{\pgfqpoint{0.000000in}{0.020833in}}%
\pgfusepath{stroke,fill}%
}%
\begin{pgfscope}%
\pgfsys@transformshift{4.295061in}{1.080890in}%
\pgfsys@useobject{currentmarker}{}%
\end{pgfscope}%
\end{pgfscope}%
\begin{pgfscope}%
\pgfsetbuttcap%
\pgfsetroundjoin%
\definecolor{currentfill}{rgb}{0.000000,0.000000,0.000000}%
\pgfsetfillcolor{currentfill}%
\pgfsetlinewidth{0.501875pt}%
\definecolor{currentstroke}{rgb}{0.000000,0.000000,0.000000}%
\pgfsetstrokecolor{currentstroke}%
\pgfsetdash{}{0pt}%
\pgfsys@defobject{currentmarker}{\pgfqpoint{0.000000in}{-0.020833in}}{\pgfqpoint{0.000000in}{0.000000in}}{%
\pgfpathmoveto{\pgfqpoint{0.000000in}{0.000000in}}%
\pgfpathlineto{\pgfqpoint{0.000000in}{-0.020833in}}%
\pgfusepath{stroke,fill}%
}%
\begin{pgfscope}%
\pgfsys@transformshift{4.295061in}{3.227753in}%
\pgfsys@useobject{currentmarker}{}%
\end{pgfscope}%
\end{pgfscope}%
\begin{pgfscope}%
\pgfpathrectangle{\pgfqpoint{0.481681in}{1.080890in}}{\pgfqpoint{5.785672in}{2.146863in}}%
\pgfusepath{clip}%
\pgfsetrectcap%
\pgfsetroundjoin%
\pgfsetlinewidth{0.100375pt}%
\definecolor{currentstroke}{rgb}{0.827451,0.827451,0.827451}%
\pgfsetstrokecolor{currentstroke}%
\pgfsetdash{}{0pt}%
\pgfpathmoveto{\pgfqpoint{4.330759in}{1.080890in}}%
\pgfpathlineto{\pgfqpoint{4.330759in}{3.227753in}}%
\pgfusepath{stroke}%
\end{pgfscope}%
\begin{pgfscope}%
\pgfsetbuttcap%
\pgfsetroundjoin%
\definecolor{currentfill}{rgb}{0.000000,0.000000,0.000000}%
\pgfsetfillcolor{currentfill}%
\pgfsetlinewidth{0.501875pt}%
\definecolor{currentstroke}{rgb}{0.000000,0.000000,0.000000}%
\pgfsetstrokecolor{currentstroke}%
\pgfsetdash{}{0pt}%
\pgfsys@defobject{currentmarker}{\pgfqpoint{0.000000in}{0.000000in}}{\pgfqpoint{0.000000in}{0.020833in}}{%
\pgfpathmoveto{\pgfqpoint{0.000000in}{0.000000in}}%
\pgfpathlineto{\pgfqpoint{0.000000in}{0.020833in}}%
\pgfusepath{stroke,fill}%
}%
\begin{pgfscope}%
\pgfsys@transformshift{4.330759in}{1.080890in}%
\pgfsys@useobject{currentmarker}{}%
\end{pgfscope}%
\end{pgfscope}%
\begin{pgfscope}%
\pgfsetbuttcap%
\pgfsetroundjoin%
\definecolor{currentfill}{rgb}{0.000000,0.000000,0.000000}%
\pgfsetfillcolor{currentfill}%
\pgfsetlinewidth{0.501875pt}%
\definecolor{currentstroke}{rgb}{0.000000,0.000000,0.000000}%
\pgfsetstrokecolor{currentstroke}%
\pgfsetdash{}{0pt}%
\pgfsys@defobject{currentmarker}{\pgfqpoint{0.000000in}{-0.020833in}}{\pgfqpoint{0.000000in}{0.000000in}}{%
\pgfpathmoveto{\pgfqpoint{0.000000in}{0.000000in}}%
\pgfpathlineto{\pgfqpoint{0.000000in}{-0.020833in}}%
\pgfusepath{stroke,fill}%
}%
\begin{pgfscope}%
\pgfsys@transformshift{4.330759in}{3.227753in}%
\pgfsys@useobject{currentmarker}{}%
\end{pgfscope}%
\end{pgfscope}%
\begin{pgfscope}%
\pgfpathrectangle{\pgfqpoint{0.481681in}{1.080890in}}{\pgfqpoint{5.785672in}{2.146863in}}%
\pgfusepath{clip}%
\pgfsetrectcap%
\pgfsetroundjoin%
\pgfsetlinewidth{0.100375pt}%
\definecolor{currentstroke}{rgb}{0.827451,0.827451,0.827451}%
\pgfsetstrokecolor{currentstroke}%
\pgfsetdash{}{0pt}%
\pgfpathmoveto{\pgfqpoint{4.366456in}{1.080890in}}%
\pgfpathlineto{\pgfqpoint{4.366456in}{3.227753in}}%
\pgfusepath{stroke}%
\end{pgfscope}%
\begin{pgfscope}%
\pgfsetbuttcap%
\pgfsetroundjoin%
\definecolor{currentfill}{rgb}{0.000000,0.000000,0.000000}%
\pgfsetfillcolor{currentfill}%
\pgfsetlinewidth{0.501875pt}%
\definecolor{currentstroke}{rgb}{0.000000,0.000000,0.000000}%
\pgfsetstrokecolor{currentstroke}%
\pgfsetdash{}{0pt}%
\pgfsys@defobject{currentmarker}{\pgfqpoint{0.000000in}{0.000000in}}{\pgfqpoint{0.000000in}{0.020833in}}{%
\pgfpathmoveto{\pgfqpoint{0.000000in}{0.000000in}}%
\pgfpathlineto{\pgfqpoint{0.000000in}{0.020833in}}%
\pgfusepath{stroke,fill}%
}%
\begin{pgfscope}%
\pgfsys@transformshift{4.366456in}{1.080890in}%
\pgfsys@useobject{currentmarker}{}%
\end{pgfscope}%
\end{pgfscope}%
\begin{pgfscope}%
\pgfsetbuttcap%
\pgfsetroundjoin%
\definecolor{currentfill}{rgb}{0.000000,0.000000,0.000000}%
\pgfsetfillcolor{currentfill}%
\pgfsetlinewidth{0.501875pt}%
\definecolor{currentstroke}{rgb}{0.000000,0.000000,0.000000}%
\pgfsetstrokecolor{currentstroke}%
\pgfsetdash{}{0pt}%
\pgfsys@defobject{currentmarker}{\pgfqpoint{0.000000in}{-0.020833in}}{\pgfqpoint{0.000000in}{0.000000in}}{%
\pgfpathmoveto{\pgfqpoint{0.000000in}{0.000000in}}%
\pgfpathlineto{\pgfqpoint{0.000000in}{-0.020833in}}%
\pgfusepath{stroke,fill}%
}%
\begin{pgfscope}%
\pgfsys@transformshift{4.366456in}{3.227753in}%
\pgfsys@useobject{currentmarker}{}%
\end{pgfscope}%
\end{pgfscope}%
\begin{pgfscope}%
\pgfpathrectangle{\pgfqpoint{0.481681in}{1.080890in}}{\pgfqpoint{5.785672in}{2.146863in}}%
\pgfusepath{clip}%
\pgfsetrectcap%
\pgfsetroundjoin%
\pgfsetlinewidth{0.100375pt}%
\definecolor{currentstroke}{rgb}{0.827451,0.827451,0.827451}%
\pgfsetstrokecolor{currentstroke}%
\pgfsetdash{}{0pt}%
\pgfpathmoveto{\pgfqpoint{4.402153in}{1.080890in}}%
\pgfpathlineto{\pgfqpoint{4.402153in}{3.227753in}}%
\pgfusepath{stroke}%
\end{pgfscope}%
\begin{pgfscope}%
\pgfsetbuttcap%
\pgfsetroundjoin%
\definecolor{currentfill}{rgb}{0.000000,0.000000,0.000000}%
\pgfsetfillcolor{currentfill}%
\pgfsetlinewidth{0.501875pt}%
\definecolor{currentstroke}{rgb}{0.000000,0.000000,0.000000}%
\pgfsetstrokecolor{currentstroke}%
\pgfsetdash{}{0pt}%
\pgfsys@defobject{currentmarker}{\pgfqpoint{0.000000in}{0.000000in}}{\pgfqpoint{0.000000in}{0.020833in}}{%
\pgfpathmoveto{\pgfqpoint{0.000000in}{0.000000in}}%
\pgfpathlineto{\pgfqpoint{0.000000in}{0.020833in}}%
\pgfusepath{stroke,fill}%
}%
\begin{pgfscope}%
\pgfsys@transformshift{4.402153in}{1.080890in}%
\pgfsys@useobject{currentmarker}{}%
\end{pgfscope}%
\end{pgfscope}%
\begin{pgfscope}%
\pgfsetbuttcap%
\pgfsetroundjoin%
\definecolor{currentfill}{rgb}{0.000000,0.000000,0.000000}%
\pgfsetfillcolor{currentfill}%
\pgfsetlinewidth{0.501875pt}%
\definecolor{currentstroke}{rgb}{0.000000,0.000000,0.000000}%
\pgfsetstrokecolor{currentstroke}%
\pgfsetdash{}{0pt}%
\pgfsys@defobject{currentmarker}{\pgfqpoint{0.000000in}{-0.020833in}}{\pgfqpoint{0.000000in}{0.000000in}}{%
\pgfpathmoveto{\pgfqpoint{0.000000in}{0.000000in}}%
\pgfpathlineto{\pgfqpoint{0.000000in}{-0.020833in}}%
\pgfusepath{stroke,fill}%
}%
\begin{pgfscope}%
\pgfsys@transformshift{4.402153in}{3.227753in}%
\pgfsys@useobject{currentmarker}{}%
\end{pgfscope}%
\end{pgfscope}%
\begin{pgfscope}%
\pgfpathrectangle{\pgfqpoint{0.481681in}{1.080890in}}{\pgfqpoint{5.785672in}{2.146863in}}%
\pgfusepath{clip}%
\pgfsetrectcap%
\pgfsetroundjoin%
\pgfsetlinewidth{0.100375pt}%
\definecolor{currentstroke}{rgb}{0.827451,0.827451,0.827451}%
\pgfsetstrokecolor{currentstroke}%
\pgfsetdash{}{0pt}%
\pgfpathmoveto{\pgfqpoint{4.437851in}{1.080890in}}%
\pgfpathlineto{\pgfqpoint{4.437851in}{3.227753in}}%
\pgfusepath{stroke}%
\end{pgfscope}%
\begin{pgfscope}%
\pgfsetbuttcap%
\pgfsetroundjoin%
\definecolor{currentfill}{rgb}{0.000000,0.000000,0.000000}%
\pgfsetfillcolor{currentfill}%
\pgfsetlinewidth{0.501875pt}%
\definecolor{currentstroke}{rgb}{0.000000,0.000000,0.000000}%
\pgfsetstrokecolor{currentstroke}%
\pgfsetdash{}{0pt}%
\pgfsys@defobject{currentmarker}{\pgfqpoint{0.000000in}{0.000000in}}{\pgfqpoint{0.000000in}{0.020833in}}{%
\pgfpathmoveto{\pgfqpoint{0.000000in}{0.000000in}}%
\pgfpathlineto{\pgfqpoint{0.000000in}{0.020833in}}%
\pgfusepath{stroke,fill}%
}%
\begin{pgfscope}%
\pgfsys@transformshift{4.437851in}{1.080890in}%
\pgfsys@useobject{currentmarker}{}%
\end{pgfscope}%
\end{pgfscope}%
\begin{pgfscope}%
\pgfsetbuttcap%
\pgfsetroundjoin%
\definecolor{currentfill}{rgb}{0.000000,0.000000,0.000000}%
\pgfsetfillcolor{currentfill}%
\pgfsetlinewidth{0.501875pt}%
\definecolor{currentstroke}{rgb}{0.000000,0.000000,0.000000}%
\pgfsetstrokecolor{currentstroke}%
\pgfsetdash{}{0pt}%
\pgfsys@defobject{currentmarker}{\pgfqpoint{0.000000in}{-0.020833in}}{\pgfqpoint{0.000000in}{0.000000in}}{%
\pgfpathmoveto{\pgfqpoint{0.000000in}{0.000000in}}%
\pgfpathlineto{\pgfqpoint{0.000000in}{-0.020833in}}%
\pgfusepath{stroke,fill}%
}%
\begin{pgfscope}%
\pgfsys@transformshift{4.437851in}{3.227753in}%
\pgfsys@useobject{currentmarker}{}%
\end{pgfscope}%
\end{pgfscope}%
\begin{pgfscope}%
\pgfpathrectangle{\pgfqpoint{0.481681in}{1.080890in}}{\pgfqpoint{5.785672in}{2.146863in}}%
\pgfusepath{clip}%
\pgfsetrectcap%
\pgfsetroundjoin%
\pgfsetlinewidth{0.100375pt}%
\definecolor{currentstroke}{rgb}{0.827451,0.827451,0.827451}%
\pgfsetstrokecolor{currentstroke}%
\pgfsetdash{}{0pt}%
\pgfpathmoveto{\pgfqpoint{4.473548in}{1.080890in}}%
\pgfpathlineto{\pgfqpoint{4.473548in}{3.227753in}}%
\pgfusepath{stroke}%
\end{pgfscope}%
\begin{pgfscope}%
\pgfsetbuttcap%
\pgfsetroundjoin%
\definecolor{currentfill}{rgb}{0.000000,0.000000,0.000000}%
\pgfsetfillcolor{currentfill}%
\pgfsetlinewidth{0.501875pt}%
\definecolor{currentstroke}{rgb}{0.000000,0.000000,0.000000}%
\pgfsetstrokecolor{currentstroke}%
\pgfsetdash{}{0pt}%
\pgfsys@defobject{currentmarker}{\pgfqpoint{0.000000in}{0.000000in}}{\pgfqpoint{0.000000in}{0.020833in}}{%
\pgfpathmoveto{\pgfqpoint{0.000000in}{0.000000in}}%
\pgfpathlineto{\pgfqpoint{0.000000in}{0.020833in}}%
\pgfusepath{stroke,fill}%
}%
\begin{pgfscope}%
\pgfsys@transformshift{4.473548in}{1.080890in}%
\pgfsys@useobject{currentmarker}{}%
\end{pgfscope}%
\end{pgfscope}%
\begin{pgfscope}%
\pgfsetbuttcap%
\pgfsetroundjoin%
\definecolor{currentfill}{rgb}{0.000000,0.000000,0.000000}%
\pgfsetfillcolor{currentfill}%
\pgfsetlinewidth{0.501875pt}%
\definecolor{currentstroke}{rgb}{0.000000,0.000000,0.000000}%
\pgfsetstrokecolor{currentstroke}%
\pgfsetdash{}{0pt}%
\pgfsys@defobject{currentmarker}{\pgfqpoint{0.000000in}{-0.020833in}}{\pgfqpoint{0.000000in}{0.000000in}}{%
\pgfpathmoveto{\pgfqpoint{0.000000in}{0.000000in}}%
\pgfpathlineto{\pgfqpoint{0.000000in}{-0.020833in}}%
\pgfusepath{stroke,fill}%
}%
\begin{pgfscope}%
\pgfsys@transformshift{4.473548in}{3.227753in}%
\pgfsys@useobject{currentmarker}{}%
\end{pgfscope}%
\end{pgfscope}%
\begin{pgfscope}%
\pgfpathrectangle{\pgfqpoint{0.481681in}{1.080890in}}{\pgfqpoint{5.785672in}{2.146863in}}%
\pgfusepath{clip}%
\pgfsetrectcap%
\pgfsetroundjoin%
\pgfsetlinewidth{0.100375pt}%
\definecolor{currentstroke}{rgb}{0.827451,0.827451,0.827451}%
\pgfsetstrokecolor{currentstroke}%
\pgfsetdash{}{0pt}%
\pgfpathmoveto{\pgfqpoint{4.509245in}{1.080890in}}%
\pgfpathlineto{\pgfqpoint{4.509245in}{3.227753in}}%
\pgfusepath{stroke}%
\end{pgfscope}%
\begin{pgfscope}%
\pgfsetbuttcap%
\pgfsetroundjoin%
\definecolor{currentfill}{rgb}{0.000000,0.000000,0.000000}%
\pgfsetfillcolor{currentfill}%
\pgfsetlinewidth{0.501875pt}%
\definecolor{currentstroke}{rgb}{0.000000,0.000000,0.000000}%
\pgfsetstrokecolor{currentstroke}%
\pgfsetdash{}{0pt}%
\pgfsys@defobject{currentmarker}{\pgfqpoint{0.000000in}{0.000000in}}{\pgfqpoint{0.000000in}{0.020833in}}{%
\pgfpathmoveto{\pgfqpoint{0.000000in}{0.000000in}}%
\pgfpathlineto{\pgfqpoint{0.000000in}{0.020833in}}%
\pgfusepath{stroke,fill}%
}%
\begin{pgfscope}%
\pgfsys@transformshift{4.509245in}{1.080890in}%
\pgfsys@useobject{currentmarker}{}%
\end{pgfscope}%
\end{pgfscope}%
\begin{pgfscope}%
\pgfsetbuttcap%
\pgfsetroundjoin%
\definecolor{currentfill}{rgb}{0.000000,0.000000,0.000000}%
\pgfsetfillcolor{currentfill}%
\pgfsetlinewidth{0.501875pt}%
\definecolor{currentstroke}{rgb}{0.000000,0.000000,0.000000}%
\pgfsetstrokecolor{currentstroke}%
\pgfsetdash{}{0pt}%
\pgfsys@defobject{currentmarker}{\pgfqpoint{0.000000in}{-0.020833in}}{\pgfqpoint{0.000000in}{0.000000in}}{%
\pgfpathmoveto{\pgfqpoint{0.000000in}{0.000000in}}%
\pgfpathlineto{\pgfqpoint{0.000000in}{-0.020833in}}%
\pgfusepath{stroke,fill}%
}%
\begin{pgfscope}%
\pgfsys@transformshift{4.509245in}{3.227753in}%
\pgfsys@useobject{currentmarker}{}%
\end{pgfscope}%
\end{pgfscope}%
\begin{pgfscope}%
\pgfpathrectangle{\pgfqpoint{0.481681in}{1.080890in}}{\pgfqpoint{5.785672in}{2.146863in}}%
\pgfusepath{clip}%
\pgfsetrectcap%
\pgfsetroundjoin%
\pgfsetlinewidth{0.100375pt}%
\definecolor{currentstroke}{rgb}{0.827451,0.827451,0.827451}%
\pgfsetstrokecolor{currentstroke}%
\pgfsetdash{}{0pt}%
\pgfpathmoveto{\pgfqpoint{4.580640in}{1.080890in}}%
\pgfpathlineto{\pgfqpoint{4.580640in}{3.227753in}}%
\pgfusepath{stroke}%
\end{pgfscope}%
\begin{pgfscope}%
\pgfsetbuttcap%
\pgfsetroundjoin%
\definecolor{currentfill}{rgb}{0.000000,0.000000,0.000000}%
\pgfsetfillcolor{currentfill}%
\pgfsetlinewidth{0.501875pt}%
\definecolor{currentstroke}{rgb}{0.000000,0.000000,0.000000}%
\pgfsetstrokecolor{currentstroke}%
\pgfsetdash{}{0pt}%
\pgfsys@defobject{currentmarker}{\pgfqpoint{0.000000in}{0.000000in}}{\pgfqpoint{0.000000in}{0.020833in}}{%
\pgfpathmoveto{\pgfqpoint{0.000000in}{0.000000in}}%
\pgfpathlineto{\pgfqpoint{0.000000in}{0.020833in}}%
\pgfusepath{stroke,fill}%
}%
\begin{pgfscope}%
\pgfsys@transformshift{4.580640in}{1.080890in}%
\pgfsys@useobject{currentmarker}{}%
\end{pgfscope}%
\end{pgfscope}%
\begin{pgfscope}%
\pgfsetbuttcap%
\pgfsetroundjoin%
\definecolor{currentfill}{rgb}{0.000000,0.000000,0.000000}%
\pgfsetfillcolor{currentfill}%
\pgfsetlinewidth{0.501875pt}%
\definecolor{currentstroke}{rgb}{0.000000,0.000000,0.000000}%
\pgfsetstrokecolor{currentstroke}%
\pgfsetdash{}{0pt}%
\pgfsys@defobject{currentmarker}{\pgfqpoint{0.000000in}{-0.020833in}}{\pgfqpoint{0.000000in}{0.000000in}}{%
\pgfpathmoveto{\pgfqpoint{0.000000in}{0.000000in}}%
\pgfpathlineto{\pgfqpoint{0.000000in}{-0.020833in}}%
\pgfusepath{stroke,fill}%
}%
\begin{pgfscope}%
\pgfsys@transformshift{4.580640in}{3.227753in}%
\pgfsys@useobject{currentmarker}{}%
\end{pgfscope}%
\end{pgfscope}%
\begin{pgfscope}%
\pgfpathrectangle{\pgfqpoint{0.481681in}{1.080890in}}{\pgfqpoint{5.785672in}{2.146863in}}%
\pgfusepath{clip}%
\pgfsetrectcap%
\pgfsetroundjoin%
\pgfsetlinewidth{0.100375pt}%
\definecolor{currentstroke}{rgb}{0.827451,0.827451,0.827451}%
\pgfsetstrokecolor{currentstroke}%
\pgfsetdash{}{0pt}%
\pgfpathmoveto{\pgfqpoint{4.616337in}{1.080890in}}%
\pgfpathlineto{\pgfqpoint{4.616337in}{3.227753in}}%
\pgfusepath{stroke}%
\end{pgfscope}%
\begin{pgfscope}%
\pgfsetbuttcap%
\pgfsetroundjoin%
\definecolor{currentfill}{rgb}{0.000000,0.000000,0.000000}%
\pgfsetfillcolor{currentfill}%
\pgfsetlinewidth{0.501875pt}%
\definecolor{currentstroke}{rgb}{0.000000,0.000000,0.000000}%
\pgfsetstrokecolor{currentstroke}%
\pgfsetdash{}{0pt}%
\pgfsys@defobject{currentmarker}{\pgfqpoint{0.000000in}{0.000000in}}{\pgfqpoint{0.000000in}{0.020833in}}{%
\pgfpathmoveto{\pgfqpoint{0.000000in}{0.000000in}}%
\pgfpathlineto{\pgfqpoint{0.000000in}{0.020833in}}%
\pgfusepath{stroke,fill}%
}%
\begin{pgfscope}%
\pgfsys@transformshift{4.616337in}{1.080890in}%
\pgfsys@useobject{currentmarker}{}%
\end{pgfscope}%
\end{pgfscope}%
\begin{pgfscope}%
\pgfsetbuttcap%
\pgfsetroundjoin%
\definecolor{currentfill}{rgb}{0.000000,0.000000,0.000000}%
\pgfsetfillcolor{currentfill}%
\pgfsetlinewidth{0.501875pt}%
\definecolor{currentstroke}{rgb}{0.000000,0.000000,0.000000}%
\pgfsetstrokecolor{currentstroke}%
\pgfsetdash{}{0pt}%
\pgfsys@defobject{currentmarker}{\pgfqpoint{0.000000in}{-0.020833in}}{\pgfqpoint{0.000000in}{0.000000in}}{%
\pgfpathmoveto{\pgfqpoint{0.000000in}{0.000000in}}%
\pgfpathlineto{\pgfqpoint{0.000000in}{-0.020833in}}%
\pgfusepath{stroke,fill}%
}%
\begin{pgfscope}%
\pgfsys@transformshift{4.616337in}{3.227753in}%
\pgfsys@useobject{currentmarker}{}%
\end{pgfscope}%
\end{pgfscope}%
\begin{pgfscope}%
\pgfpathrectangle{\pgfqpoint{0.481681in}{1.080890in}}{\pgfqpoint{5.785672in}{2.146863in}}%
\pgfusepath{clip}%
\pgfsetrectcap%
\pgfsetroundjoin%
\pgfsetlinewidth{0.100375pt}%
\definecolor{currentstroke}{rgb}{0.827451,0.827451,0.827451}%
\pgfsetstrokecolor{currentstroke}%
\pgfsetdash{}{0pt}%
\pgfpathmoveto{\pgfqpoint{4.652035in}{1.080890in}}%
\pgfpathlineto{\pgfqpoint{4.652035in}{3.227753in}}%
\pgfusepath{stroke}%
\end{pgfscope}%
\begin{pgfscope}%
\pgfsetbuttcap%
\pgfsetroundjoin%
\definecolor{currentfill}{rgb}{0.000000,0.000000,0.000000}%
\pgfsetfillcolor{currentfill}%
\pgfsetlinewidth{0.501875pt}%
\definecolor{currentstroke}{rgb}{0.000000,0.000000,0.000000}%
\pgfsetstrokecolor{currentstroke}%
\pgfsetdash{}{0pt}%
\pgfsys@defobject{currentmarker}{\pgfqpoint{0.000000in}{0.000000in}}{\pgfqpoint{0.000000in}{0.020833in}}{%
\pgfpathmoveto{\pgfqpoint{0.000000in}{0.000000in}}%
\pgfpathlineto{\pgfqpoint{0.000000in}{0.020833in}}%
\pgfusepath{stroke,fill}%
}%
\begin{pgfscope}%
\pgfsys@transformshift{4.652035in}{1.080890in}%
\pgfsys@useobject{currentmarker}{}%
\end{pgfscope}%
\end{pgfscope}%
\begin{pgfscope}%
\pgfsetbuttcap%
\pgfsetroundjoin%
\definecolor{currentfill}{rgb}{0.000000,0.000000,0.000000}%
\pgfsetfillcolor{currentfill}%
\pgfsetlinewidth{0.501875pt}%
\definecolor{currentstroke}{rgb}{0.000000,0.000000,0.000000}%
\pgfsetstrokecolor{currentstroke}%
\pgfsetdash{}{0pt}%
\pgfsys@defobject{currentmarker}{\pgfqpoint{0.000000in}{-0.020833in}}{\pgfqpoint{0.000000in}{0.000000in}}{%
\pgfpathmoveto{\pgfqpoint{0.000000in}{0.000000in}}%
\pgfpathlineto{\pgfqpoint{0.000000in}{-0.020833in}}%
\pgfusepath{stroke,fill}%
}%
\begin{pgfscope}%
\pgfsys@transformshift{4.652035in}{3.227753in}%
\pgfsys@useobject{currentmarker}{}%
\end{pgfscope}%
\end{pgfscope}%
\begin{pgfscope}%
\pgfpathrectangle{\pgfqpoint{0.481681in}{1.080890in}}{\pgfqpoint{5.785672in}{2.146863in}}%
\pgfusepath{clip}%
\pgfsetrectcap%
\pgfsetroundjoin%
\pgfsetlinewidth{0.100375pt}%
\definecolor{currentstroke}{rgb}{0.827451,0.827451,0.827451}%
\pgfsetstrokecolor{currentstroke}%
\pgfsetdash{}{0pt}%
\pgfpathmoveto{\pgfqpoint{4.687732in}{1.080890in}}%
\pgfpathlineto{\pgfqpoint{4.687732in}{3.227753in}}%
\pgfusepath{stroke}%
\end{pgfscope}%
\begin{pgfscope}%
\pgfsetbuttcap%
\pgfsetroundjoin%
\definecolor{currentfill}{rgb}{0.000000,0.000000,0.000000}%
\pgfsetfillcolor{currentfill}%
\pgfsetlinewidth{0.501875pt}%
\definecolor{currentstroke}{rgb}{0.000000,0.000000,0.000000}%
\pgfsetstrokecolor{currentstroke}%
\pgfsetdash{}{0pt}%
\pgfsys@defobject{currentmarker}{\pgfqpoint{0.000000in}{0.000000in}}{\pgfqpoint{0.000000in}{0.020833in}}{%
\pgfpathmoveto{\pgfqpoint{0.000000in}{0.000000in}}%
\pgfpathlineto{\pgfqpoint{0.000000in}{0.020833in}}%
\pgfusepath{stroke,fill}%
}%
\begin{pgfscope}%
\pgfsys@transformshift{4.687732in}{1.080890in}%
\pgfsys@useobject{currentmarker}{}%
\end{pgfscope}%
\end{pgfscope}%
\begin{pgfscope}%
\pgfsetbuttcap%
\pgfsetroundjoin%
\definecolor{currentfill}{rgb}{0.000000,0.000000,0.000000}%
\pgfsetfillcolor{currentfill}%
\pgfsetlinewidth{0.501875pt}%
\definecolor{currentstroke}{rgb}{0.000000,0.000000,0.000000}%
\pgfsetstrokecolor{currentstroke}%
\pgfsetdash{}{0pt}%
\pgfsys@defobject{currentmarker}{\pgfqpoint{0.000000in}{-0.020833in}}{\pgfqpoint{0.000000in}{0.000000in}}{%
\pgfpathmoveto{\pgfqpoint{0.000000in}{0.000000in}}%
\pgfpathlineto{\pgfqpoint{0.000000in}{-0.020833in}}%
\pgfusepath{stroke,fill}%
}%
\begin{pgfscope}%
\pgfsys@transformshift{4.687732in}{3.227753in}%
\pgfsys@useobject{currentmarker}{}%
\end{pgfscope}%
\end{pgfscope}%
\begin{pgfscope}%
\pgfpathrectangle{\pgfqpoint{0.481681in}{1.080890in}}{\pgfqpoint{5.785672in}{2.146863in}}%
\pgfusepath{clip}%
\pgfsetrectcap%
\pgfsetroundjoin%
\pgfsetlinewidth{0.100375pt}%
\definecolor{currentstroke}{rgb}{0.827451,0.827451,0.827451}%
\pgfsetstrokecolor{currentstroke}%
\pgfsetdash{}{0pt}%
\pgfpathmoveto{\pgfqpoint{4.723429in}{1.080890in}}%
\pgfpathlineto{\pgfqpoint{4.723429in}{3.227753in}}%
\pgfusepath{stroke}%
\end{pgfscope}%
\begin{pgfscope}%
\pgfsetbuttcap%
\pgfsetroundjoin%
\definecolor{currentfill}{rgb}{0.000000,0.000000,0.000000}%
\pgfsetfillcolor{currentfill}%
\pgfsetlinewidth{0.501875pt}%
\definecolor{currentstroke}{rgb}{0.000000,0.000000,0.000000}%
\pgfsetstrokecolor{currentstroke}%
\pgfsetdash{}{0pt}%
\pgfsys@defobject{currentmarker}{\pgfqpoint{0.000000in}{0.000000in}}{\pgfqpoint{0.000000in}{0.020833in}}{%
\pgfpathmoveto{\pgfqpoint{0.000000in}{0.000000in}}%
\pgfpathlineto{\pgfqpoint{0.000000in}{0.020833in}}%
\pgfusepath{stroke,fill}%
}%
\begin{pgfscope}%
\pgfsys@transformshift{4.723429in}{1.080890in}%
\pgfsys@useobject{currentmarker}{}%
\end{pgfscope}%
\end{pgfscope}%
\begin{pgfscope}%
\pgfsetbuttcap%
\pgfsetroundjoin%
\definecolor{currentfill}{rgb}{0.000000,0.000000,0.000000}%
\pgfsetfillcolor{currentfill}%
\pgfsetlinewidth{0.501875pt}%
\definecolor{currentstroke}{rgb}{0.000000,0.000000,0.000000}%
\pgfsetstrokecolor{currentstroke}%
\pgfsetdash{}{0pt}%
\pgfsys@defobject{currentmarker}{\pgfqpoint{0.000000in}{-0.020833in}}{\pgfqpoint{0.000000in}{0.000000in}}{%
\pgfpathmoveto{\pgfqpoint{0.000000in}{0.000000in}}%
\pgfpathlineto{\pgfqpoint{0.000000in}{-0.020833in}}%
\pgfusepath{stroke,fill}%
}%
\begin{pgfscope}%
\pgfsys@transformshift{4.723429in}{3.227753in}%
\pgfsys@useobject{currentmarker}{}%
\end{pgfscope}%
\end{pgfscope}%
\begin{pgfscope}%
\pgfpathrectangle{\pgfqpoint{0.481681in}{1.080890in}}{\pgfqpoint{5.785672in}{2.146863in}}%
\pgfusepath{clip}%
\pgfsetrectcap%
\pgfsetroundjoin%
\pgfsetlinewidth{0.100375pt}%
\definecolor{currentstroke}{rgb}{0.827451,0.827451,0.827451}%
\pgfsetstrokecolor{currentstroke}%
\pgfsetdash{}{0pt}%
\pgfpathmoveto{\pgfqpoint{4.759127in}{1.080890in}}%
\pgfpathlineto{\pgfqpoint{4.759127in}{3.227753in}}%
\pgfusepath{stroke}%
\end{pgfscope}%
\begin{pgfscope}%
\pgfsetbuttcap%
\pgfsetroundjoin%
\definecolor{currentfill}{rgb}{0.000000,0.000000,0.000000}%
\pgfsetfillcolor{currentfill}%
\pgfsetlinewidth{0.501875pt}%
\definecolor{currentstroke}{rgb}{0.000000,0.000000,0.000000}%
\pgfsetstrokecolor{currentstroke}%
\pgfsetdash{}{0pt}%
\pgfsys@defobject{currentmarker}{\pgfqpoint{0.000000in}{0.000000in}}{\pgfqpoint{0.000000in}{0.020833in}}{%
\pgfpathmoveto{\pgfqpoint{0.000000in}{0.000000in}}%
\pgfpathlineto{\pgfqpoint{0.000000in}{0.020833in}}%
\pgfusepath{stroke,fill}%
}%
\begin{pgfscope}%
\pgfsys@transformshift{4.759127in}{1.080890in}%
\pgfsys@useobject{currentmarker}{}%
\end{pgfscope}%
\end{pgfscope}%
\begin{pgfscope}%
\pgfsetbuttcap%
\pgfsetroundjoin%
\definecolor{currentfill}{rgb}{0.000000,0.000000,0.000000}%
\pgfsetfillcolor{currentfill}%
\pgfsetlinewidth{0.501875pt}%
\definecolor{currentstroke}{rgb}{0.000000,0.000000,0.000000}%
\pgfsetstrokecolor{currentstroke}%
\pgfsetdash{}{0pt}%
\pgfsys@defobject{currentmarker}{\pgfqpoint{0.000000in}{-0.020833in}}{\pgfqpoint{0.000000in}{0.000000in}}{%
\pgfpathmoveto{\pgfqpoint{0.000000in}{0.000000in}}%
\pgfpathlineto{\pgfqpoint{0.000000in}{-0.020833in}}%
\pgfusepath{stroke,fill}%
}%
\begin{pgfscope}%
\pgfsys@transformshift{4.759127in}{3.227753in}%
\pgfsys@useobject{currentmarker}{}%
\end{pgfscope}%
\end{pgfscope}%
\begin{pgfscope}%
\pgfpathrectangle{\pgfqpoint{0.481681in}{1.080890in}}{\pgfqpoint{5.785672in}{2.146863in}}%
\pgfusepath{clip}%
\pgfsetrectcap%
\pgfsetroundjoin%
\pgfsetlinewidth{0.100375pt}%
\definecolor{currentstroke}{rgb}{0.827451,0.827451,0.827451}%
\pgfsetstrokecolor{currentstroke}%
\pgfsetdash{}{0pt}%
\pgfpathmoveto{\pgfqpoint{4.794824in}{1.080890in}}%
\pgfpathlineto{\pgfqpoint{4.794824in}{3.227753in}}%
\pgfusepath{stroke}%
\end{pgfscope}%
\begin{pgfscope}%
\pgfsetbuttcap%
\pgfsetroundjoin%
\definecolor{currentfill}{rgb}{0.000000,0.000000,0.000000}%
\pgfsetfillcolor{currentfill}%
\pgfsetlinewidth{0.501875pt}%
\definecolor{currentstroke}{rgb}{0.000000,0.000000,0.000000}%
\pgfsetstrokecolor{currentstroke}%
\pgfsetdash{}{0pt}%
\pgfsys@defobject{currentmarker}{\pgfqpoint{0.000000in}{0.000000in}}{\pgfqpoint{0.000000in}{0.020833in}}{%
\pgfpathmoveto{\pgfqpoint{0.000000in}{0.000000in}}%
\pgfpathlineto{\pgfqpoint{0.000000in}{0.020833in}}%
\pgfusepath{stroke,fill}%
}%
\begin{pgfscope}%
\pgfsys@transformshift{4.794824in}{1.080890in}%
\pgfsys@useobject{currentmarker}{}%
\end{pgfscope}%
\end{pgfscope}%
\begin{pgfscope}%
\pgfsetbuttcap%
\pgfsetroundjoin%
\definecolor{currentfill}{rgb}{0.000000,0.000000,0.000000}%
\pgfsetfillcolor{currentfill}%
\pgfsetlinewidth{0.501875pt}%
\definecolor{currentstroke}{rgb}{0.000000,0.000000,0.000000}%
\pgfsetstrokecolor{currentstroke}%
\pgfsetdash{}{0pt}%
\pgfsys@defobject{currentmarker}{\pgfqpoint{0.000000in}{-0.020833in}}{\pgfqpoint{0.000000in}{0.000000in}}{%
\pgfpathmoveto{\pgfqpoint{0.000000in}{0.000000in}}%
\pgfpathlineto{\pgfqpoint{0.000000in}{-0.020833in}}%
\pgfusepath{stroke,fill}%
}%
\begin{pgfscope}%
\pgfsys@transformshift{4.794824in}{3.227753in}%
\pgfsys@useobject{currentmarker}{}%
\end{pgfscope}%
\end{pgfscope}%
\begin{pgfscope}%
\pgfpathrectangle{\pgfqpoint{0.481681in}{1.080890in}}{\pgfqpoint{5.785672in}{2.146863in}}%
\pgfusepath{clip}%
\pgfsetrectcap%
\pgfsetroundjoin%
\pgfsetlinewidth{0.100375pt}%
\definecolor{currentstroke}{rgb}{0.827451,0.827451,0.827451}%
\pgfsetstrokecolor{currentstroke}%
\pgfsetdash{}{0pt}%
\pgfpathmoveto{\pgfqpoint{4.830521in}{1.080890in}}%
\pgfpathlineto{\pgfqpoint{4.830521in}{3.227753in}}%
\pgfusepath{stroke}%
\end{pgfscope}%
\begin{pgfscope}%
\pgfsetbuttcap%
\pgfsetroundjoin%
\definecolor{currentfill}{rgb}{0.000000,0.000000,0.000000}%
\pgfsetfillcolor{currentfill}%
\pgfsetlinewidth{0.501875pt}%
\definecolor{currentstroke}{rgb}{0.000000,0.000000,0.000000}%
\pgfsetstrokecolor{currentstroke}%
\pgfsetdash{}{0pt}%
\pgfsys@defobject{currentmarker}{\pgfqpoint{0.000000in}{0.000000in}}{\pgfqpoint{0.000000in}{0.020833in}}{%
\pgfpathmoveto{\pgfqpoint{0.000000in}{0.000000in}}%
\pgfpathlineto{\pgfqpoint{0.000000in}{0.020833in}}%
\pgfusepath{stroke,fill}%
}%
\begin{pgfscope}%
\pgfsys@transformshift{4.830521in}{1.080890in}%
\pgfsys@useobject{currentmarker}{}%
\end{pgfscope}%
\end{pgfscope}%
\begin{pgfscope}%
\pgfsetbuttcap%
\pgfsetroundjoin%
\definecolor{currentfill}{rgb}{0.000000,0.000000,0.000000}%
\pgfsetfillcolor{currentfill}%
\pgfsetlinewidth{0.501875pt}%
\definecolor{currentstroke}{rgb}{0.000000,0.000000,0.000000}%
\pgfsetstrokecolor{currentstroke}%
\pgfsetdash{}{0pt}%
\pgfsys@defobject{currentmarker}{\pgfqpoint{0.000000in}{-0.020833in}}{\pgfqpoint{0.000000in}{0.000000in}}{%
\pgfpathmoveto{\pgfqpoint{0.000000in}{0.000000in}}%
\pgfpathlineto{\pgfqpoint{0.000000in}{-0.020833in}}%
\pgfusepath{stroke,fill}%
}%
\begin{pgfscope}%
\pgfsys@transformshift{4.830521in}{3.227753in}%
\pgfsys@useobject{currentmarker}{}%
\end{pgfscope}%
\end{pgfscope}%
\begin{pgfscope}%
\pgfpathrectangle{\pgfqpoint{0.481681in}{1.080890in}}{\pgfqpoint{5.785672in}{2.146863in}}%
\pgfusepath{clip}%
\pgfsetrectcap%
\pgfsetroundjoin%
\pgfsetlinewidth{0.100375pt}%
\definecolor{currentstroke}{rgb}{0.827451,0.827451,0.827451}%
\pgfsetstrokecolor{currentstroke}%
\pgfsetdash{}{0pt}%
\pgfpathmoveto{\pgfqpoint{4.866219in}{1.080890in}}%
\pgfpathlineto{\pgfqpoint{4.866219in}{3.227753in}}%
\pgfusepath{stroke}%
\end{pgfscope}%
\begin{pgfscope}%
\pgfsetbuttcap%
\pgfsetroundjoin%
\definecolor{currentfill}{rgb}{0.000000,0.000000,0.000000}%
\pgfsetfillcolor{currentfill}%
\pgfsetlinewidth{0.501875pt}%
\definecolor{currentstroke}{rgb}{0.000000,0.000000,0.000000}%
\pgfsetstrokecolor{currentstroke}%
\pgfsetdash{}{0pt}%
\pgfsys@defobject{currentmarker}{\pgfqpoint{0.000000in}{0.000000in}}{\pgfqpoint{0.000000in}{0.020833in}}{%
\pgfpathmoveto{\pgfqpoint{0.000000in}{0.000000in}}%
\pgfpathlineto{\pgfqpoint{0.000000in}{0.020833in}}%
\pgfusepath{stroke,fill}%
}%
\begin{pgfscope}%
\pgfsys@transformshift{4.866219in}{1.080890in}%
\pgfsys@useobject{currentmarker}{}%
\end{pgfscope}%
\end{pgfscope}%
\begin{pgfscope}%
\pgfsetbuttcap%
\pgfsetroundjoin%
\definecolor{currentfill}{rgb}{0.000000,0.000000,0.000000}%
\pgfsetfillcolor{currentfill}%
\pgfsetlinewidth{0.501875pt}%
\definecolor{currentstroke}{rgb}{0.000000,0.000000,0.000000}%
\pgfsetstrokecolor{currentstroke}%
\pgfsetdash{}{0pt}%
\pgfsys@defobject{currentmarker}{\pgfqpoint{0.000000in}{-0.020833in}}{\pgfqpoint{0.000000in}{0.000000in}}{%
\pgfpathmoveto{\pgfqpoint{0.000000in}{0.000000in}}%
\pgfpathlineto{\pgfqpoint{0.000000in}{-0.020833in}}%
\pgfusepath{stroke,fill}%
}%
\begin{pgfscope}%
\pgfsys@transformshift{4.866219in}{3.227753in}%
\pgfsys@useobject{currentmarker}{}%
\end{pgfscope}%
\end{pgfscope}%
\begin{pgfscope}%
\pgfpathrectangle{\pgfqpoint{0.481681in}{1.080890in}}{\pgfqpoint{5.785672in}{2.146863in}}%
\pgfusepath{clip}%
\pgfsetrectcap%
\pgfsetroundjoin%
\pgfsetlinewidth{0.100375pt}%
\definecolor{currentstroke}{rgb}{0.827451,0.827451,0.827451}%
\pgfsetstrokecolor{currentstroke}%
\pgfsetdash{}{0pt}%
\pgfpathmoveto{\pgfqpoint{4.901916in}{1.080890in}}%
\pgfpathlineto{\pgfqpoint{4.901916in}{3.227753in}}%
\pgfusepath{stroke}%
\end{pgfscope}%
\begin{pgfscope}%
\pgfsetbuttcap%
\pgfsetroundjoin%
\definecolor{currentfill}{rgb}{0.000000,0.000000,0.000000}%
\pgfsetfillcolor{currentfill}%
\pgfsetlinewidth{0.501875pt}%
\definecolor{currentstroke}{rgb}{0.000000,0.000000,0.000000}%
\pgfsetstrokecolor{currentstroke}%
\pgfsetdash{}{0pt}%
\pgfsys@defobject{currentmarker}{\pgfqpoint{0.000000in}{0.000000in}}{\pgfqpoint{0.000000in}{0.020833in}}{%
\pgfpathmoveto{\pgfqpoint{0.000000in}{0.000000in}}%
\pgfpathlineto{\pgfqpoint{0.000000in}{0.020833in}}%
\pgfusepath{stroke,fill}%
}%
\begin{pgfscope}%
\pgfsys@transformshift{4.901916in}{1.080890in}%
\pgfsys@useobject{currentmarker}{}%
\end{pgfscope}%
\end{pgfscope}%
\begin{pgfscope}%
\pgfsetbuttcap%
\pgfsetroundjoin%
\definecolor{currentfill}{rgb}{0.000000,0.000000,0.000000}%
\pgfsetfillcolor{currentfill}%
\pgfsetlinewidth{0.501875pt}%
\definecolor{currentstroke}{rgb}{0.000000,0.000000,0.000000}%
\pgfsetstrokecolor{currentstroke}%
\pgfsetdash{}{0pt}%
\pgfsys@defobject{currentmarker}{\pgfqpoint{0.000000in}{-0.020833in}}{\pgfqpoint{0.000000in}{0.000000in}}{%
\pgfpathmoveto{\pgfqpoint{0.000000in}{0.000000in}}%
\pgfpathlineto{\pgfqpoint{0.000000in}{-0.020833in}}%
\pgfusepath{stroke,fill}%
}%
\begin{pgfscope}%
\pgfsys@transformshift{4.901916in}{3.227753in}%
\pgfsys@useobject{currentmarker}{}%
\end{pgfscope}%
\end{pgfscope}%
\begin{pgfscope}%
\pgfpathrectangle{\pgfqpoint{0.481681in}{1.080890in}}{\pgfqpoint{5.785672in}{2.146863in}}%
\pgfusepath{clip}%
\pgfsetrectcap%
\pgfsetroundjoin%
\pgfsetlinewidth{0.100375pt}%
\definecolor{currentstroke}{rgb}{0.827451,0.827451,0.827451}%
\pgfsetstrokecolor{currentstroke}%
\pgfsetdash{}{0pt}%
\pgfpathmoveto{\pgfqpoint{4.937613in}{1.080890in}}%
\pgfpathlineto{\pgfqpoint{4.937613in}{3.227753in}}%
\pgfusepath{stroke}%
\end{pgfscope}%
\begin{pgfscope}%
\pgfsetbuttcap%
\pgfsetroundjoin%
\definecolor{currentfill}{rgb}{0.000000,0.000000,0.000000}%
\pgfsetfillcolor{currentfill}%
\pgfsetlinewidth{0.501875pt}%
\definecolor{currentstroke}{rgb}{0.000000,0.000000,0.000000}%
\pgfsetstrokecolor{currentstroke}%
\pgfsetdash{}{0pt}%
\pgfsys@defobject{currentmarker}{\pgfqpoint{0.000000in}{0.000000in}}{\pgfqpoint{0.000000in}{0.020833in}}{%
\pgfpathmoveto{\pgfqpoint{0.000000in}{0.000000in}}%
\pgfpathlineto{\pgfqpoint{0.000000in}{0.020833in}}%
\pgfusepath{stroke,fill}%
}%
\begin{pgfscope}%
\pgfsys@transformshift{4.937613in}{1.080890in}%
\pgfsys@useobject{currentmarker}{}%
\end{pgfscope}%
\end{pgfscope}%
\begin{pgfscope}%
\pgfsetbuttcap%
\pgfsetroundjoin%
\definecolor{currentfill}{rgb}{0.000000,0.000000,0.000000}%
\pgfsetfillcolor{currentfill}%
\pgfsetlinewidth{0.501875pt}%
\definecolor{currentstroke}{rgb}{0.000000,0.000000,0.000000}%
\pgfsetstrokecolor{currentstroke}%
\pgfsetdash{}{0pt}%
\pgfsys@defobject{currentmarker}{\pgfqpoint{0.000000in}{-0.020833in}}{\pgfqpoint{0.000000in}{0.000000in}}{%
\pgfpathmoveto{\pgfqpoint{0.000000in}{0.000000in}}%
\pgfpathlineto{\pgfqpoint{0.000000in}{-0.020833in}}%
\pgfusepath{stroke,fill}%
}%
\begin{pgfscope}%
\pgfsys@transformshift{4.937613in}{3.227753in}%
\pgfsys@useobject{currentmarker}{}%
\end{pgfscope}%
\end{pgfscope}%
\begin{pgfscope}%
\pgfpathrectangle{\pgfqpoint{0.481681in}{1.080890in}}{\pgfqpoint{5.785672in}{2.146863in}}%
\pgfusepath{clip}%
\pgfsetrectcap%
\pgfsetroundjoin%
\pgfsetlinewidth{0.100375pt}%
\definecolor{currentstroke}{rgb}{0.827451,0.827451,0.827451}%
\pgfsetstrokecolor{currentstroke}%
\pgfsetdash{}{0pt}%
\pgfpathmoveto{\pgfqpoint{5.009008in}{1.080890in}}%
\pgfpathlineto{\pgfqpoint{5.009008in}{3.227753in}}%
\pgfusepath{stroke}%
\end{pgfscope}%
\begin{pgfscope}%
\pgfsetbuttcap%
\pgfsetroundjoin%
\definecolor{currentfill}{rgb}{0.000000,0.000000,0.000000}%
\pgfsetfillcolor{currentfill}%
\pgfsetlinewidth{0.501875pt}%
\definecolor{currentstroke}{rgb}{0.000000,0.000000,0.000000}%
\pgfsetstrokecolor{currentstroke}%
\pgfsetdash{}{0pt}%
\pgfsys@defobject{currentmarker}{\pgfqpoint{0.000000in}{0.000000in}}{\pgfqpoint{0.000000in}{0.020833in}}{%
\pgfpathmoveto{\pgfqpoint{0.000000in}{0.000000in}}%
\pgfpathlineto{\pgfqpoint{0.000000in}{0.020833in}}%
\pgfusepath{stroke,fill}%
}%
\begin{pgfscope}%
\pgfsys@transformshift{5.009008in}{1.080890in}%
\pgfsys@useobject{currentmarker}{}%
\end{pgfscope}%
\end{pgfscope}%
\begin{pgfscope}%
\pgfsetbuttcap%
\pgfsetroundjoin%
\definecolor{currentfill}{rgb}{0.000000,0.000000,0.000000}%
\pgfsetfillcolor{currentfill}%
\pgfsetlinewidth{0.501875pt}%
\definecolor{currentstroke}{rgb}{0.000000,0.000000,0.000000}%
\pgfsetstrokecolor{currentstroke}%
\pgfsetdash{}{0pt}%
\pgfsys@defobject{currentmarker}{\pgfqpoint{0.000000in}{-0.020833in}}{\pgfqpoint{0.000000in}{0.000000in}}{%
\pgfpathmoveto{\pgfqpoint{0.000000in}{0.000000in}}%
\pgfpathlineto{\pgfqpoint{0.000000in}{-0.020833in}}%
\pgfusepath{stroke,fill}%
}%
\begin{pgfscope}%
\pgfsys@transformshift{5.009008in}{3.227753in}%
\pgfsys@useobject{currentmarker}{}%
\end{pgfscope}%
\end{pgfscope}%
\begin{pgfscope}%
\pgfpathrectangle{\pgfqpoint{0.481681in}{1.080890in}}{\pgfqpoint{5.785672in}{2.146863in}}%
\pgfusepath{clip}%
\pgfsetrectcap%
\pgfsetroundjoin%
\pgfsetlinewidth{0.100375pt}%
\definecolor{currentstroke}{rgb}{0.827451,0.827451,0.827451}%
\pgfsetstrokecolor{currentstroke}%
\pgfsetdash{}{0pt}%
\pgfpathmoveto{\pgfqpoint{5.044705in}{1.080890in}}%
\pgfpathlineto{\pgfqpoint{5.044705in}{3.227753in}}%
\pgfusepath{stroke}%
\end{pgfscope}%
\begin{pgfscope}%
\pgfsetbuttcap%
\pgfsetroundjoin%
\definecolor{currentfill}{rgb}{0.000000,0.000000,0.000000}%
\pgfsetfillcolor{currentfill}%
\pgfsetlinewidth{0.501875pt}%
\definecolor{currentstroke}{rgb}{0.000000,0.000000,0.000000}%
\pgfsetstrokecolor{currentstroke}%
\pgfsetdash{}{0pt}%
\pgfsys@defobject{currentmarker}{\pgfqpoint{0.000000in}{0.000000in}}{\pgfqpoint{0.000000in}{0.020833in}}{%
\pgfpathmoveto{\pgfqpoint{0.000000in}{0.000000in}}%
\pgfpathlineto{\pgfqpoint{0.000000in}{0.020833in}}%
\pgfusepath{stroke,fill}%
}%
\begin{pgfscope}%
\pgfsys@transformshift{5.044705in}{1.080890in}%
\pgfsys@useobject{currentmarker}{}%
\end{pgfscope}%
\end{pgfscope}%
\begin{pgfscope}%
\pgfsetbuttcap%
\pgfsetroundjoin%
\definecolor{currentfill}{rgb}{0.000000,0.000000,0.000000}%
\pgfsetfillcolor{currentfill}%
\pgfsetlinewidth{0.501875pt}%
\definecolor{currentstroke}{rgb}{0.000000,0.000000,0.000000}%
\pgfsetstrokecolor{currentstroke}%
\pgfsetdash{}{0pt}%
\pgfsys@defobject{currentmarker}{\pgfqpoint{0.000000in}{-0.020833in}}{\pgfqpoint{0.000000in}{0.000000in}}{%
\pgfpathmoveto{\pgfqpoint{0.000000in}{0.000000in}}%
\pgfpathlineto{\pgfqpoint{0.000000in}{-0.020833in}}%
\pgfusepath{stroke,fill}%
}%
\begin{pgfscope}%
\pgfsys@transformshift{5.044705in}{3.227753in}%
\pgfsys@useobject{currentmarker}{}%
\end{pgfscope}%
\end{pgfscope}%
\begin{pgfscope}%
\pgfpathrectangle{\pgfqpoint{0.481681in}{1.080890in}}{\pgfqpoint{5.785672in}{2.146863in}}%
\pgfusepath{clip}%
\pgfsetrectcap%
\pgfsetroundjoin%
\pgfsetlinewidth{0.100375pt}%
\definecolor{currentstroke}{rgb}{0.827451,0.827451,0.827451}%
\pgfsetstrokecolor{currentstroke}%
\pgfsetdash{}{0pt}%
\pgfpathmoveto{\pgfqpoint{5.080402in}{1.080890in}}%
\pgfpathlineto{\pgfqpoint{5.080402in}{3.227753in}}%
\pgfusepath{stroke}%
\end{pgfscope}%
\begin{pgfscope}%
\pgfsetbuttcap%
\pgfsetroundjoin%
\definecolor{currentfill}{rgb}{0.000000,0.000000,0.000000}%
\pgfsetfillcolor{currentfill}%
\pgfsetlinewidth{0.501875pt}%
\definecolor{currentstroke}{rgb}{0.000000,0.000000,0.000000}%
\pgfsetstrokecolor{currentstroke}%
\pgfsetdash{}{0pt}%
\pgfsys@defobject{currentmarker}{\pgfqpoint{0.000000in}{0.000000in}}{\pgfqpoint{0.000000in}{0.020833in}}{%
\pgfpathmoveto{\pgfqpoint{0.000000in}{0.000000in}}%
\pgfpathlineto{\pgfqpoint{0.000000in}{0.020833in}}%
\pgfusepath{stroke,fill}%
}%
\begin{pgfscope}%
\pgfsys@transformshift{5.080402in}{1.080890in}%
\pgfsys@useobject{currentmarker}{}%
\end{pgfscope}%
\end{pgfscope}%
\begin{pgfscope}%
\pgfsetbuttcap%
\pgfsetroundjoin%
\definecolor{currentfill}{rgb}{0.000000,0.000000,0.000000}%
\pgfsetfillcolor{currentfill}%
\pgfsetlinewidth{0.501875pt}%
\definecolor{currentstroke}{rgb}{0.000000,0.000000,0.000000}%
\pgfsetstrokecolor{currentstroke}%
\pgfsetdash{}{0pt}%
\pgfsys@defobject{currentmarker}{\pgfqpoint{0.000000in}{-0.020833in}}{\pgfqpoint{0.000000in}{0.000000in}}{%
\pgfpathmoveto{\pgfqpoint{0.000000in}{0.000000in}}%
\pgfpathlineto{\pgfqpoint{0.000000in}{-0.020833in}}%
\pgfusepath{stroke,fill}%
}%
\begin{pgfscope}%
\pgfsys@transformshift{5.080402in}{3.227753in}%
\pgfsys@useobject{currentmarker}{}%
\end{pgfscope}%
\end{pgfscope}%
\begin{pgfscope}%
\pgfpathrectangle{\pgfqpoint{0.481681in}{1.080890in}}{\pgfqpoint{5.785672in}{2.146863in}}%
\pgfusepath{clip}%
\pgfsetrectcap%
\pgfsetroundjoin%
\pgfsetlinewidth{0.100375pt}%
\definecolor{currentstroke}{rgb}{0.827451,0.827451,0.827451}%
\pgfsetstrokecolor{currentstroke}%
\pgfsetdash{}{0pt}%
\pgfpathmoveto{\pgfqpoint{5.116100in}{1.080890in}}%
\pgfpathlineto{\pgfqpoint{5.116100in}{3.227753in}}%
\pgfusepath{stroke}%
\end{pgfscope}%
\begin{pgfscope}%
\pgfsetbuttcap%
\pgfsetroundjoin%
\definecolor{currentfill}{rgb}{0.000000,0.000000,0.000000}%
\pgfsetfillcolor{currentfill}%
\pgfsetlinewidth{0.501875pt}%
\definecolor{currentstroke}{rgb}{0.000000,0.000000,0.000000}%
\pgfsetstrokecolor{currentstroke}%
\pgfsetdash{}{0pt}%
\pgfsys@defobject{currentmarker}{\pgfqpoint{0.000000in}{0.000000in}}{\pgfqpoint{0.000000in}{0.020833in}}{%
\pgfpathmoveto{\pgfqpoint{0.000000in}{0.000000in}}%
\pgfpathlineto{\pgfqpoint{0.000000in}{0.020833in}}%
\pgfusepath{stroke,fill}%
}%
\begin{pgfscope}%
\pgfsys@transformshift{5.116100in}{1.080890in}%
\pgfsys@useobject{currentmarker}{}%
\end{pgfscope}%
\end{pgfscope}%
\begin{pgfscope}%
\pgfsetbuttcap%
\pgfsetroundjoin%
\definecolor{currentfill}{rgb}{0.000000,0.000000,0.000000}%
\pgfsetfillcolor{currentfill}%
\pgfsetlinewidth{0.501875pt}%
\definecolor{currentstroke}{rgb}{0.000000,0.000000,0.000000}%
\pgfsetstrokecolor{currentstroke}%
\pgfsetdash{}{0pt}%
\pgfsys@defobject{currentmarker}{\pgfqpoint{0.000000in}{-0.020833in}}{\pgfqpoint{0.000000in}{0.000000in}}{%
\pgfpathmoveto{\pgfqpoint{0.000000in}{0.000000in}}%
\pgfpathlineto{\pgfqpoint{0.000000in}{-0.020833in}}%
\pgfusepath{stroke,fill}%
}%
\begin{pgfscope}%
\pgfsys@transformshift{5.116100in}{3.227753in}%
\pgfsys@useobject{currentmarker}{}%
\end{pgfscope}%
\end{pgfscope}%
\begin{pgfscope}%
\pgfpathrectangle{\pgfqpoint{0.481681in}{1.080890in}}{\pgfqpoint{5.785672in}{2.146863in}}%
\pgfusepath{clip}%
\pgfsetrectcap%
\pgfsetroundjoin%
\pgfsetlinewidth{0.100375pt}%
\definecolor{currentstroke}{rgb}{0.827451,0.827451,0.827451}%
\pgfsetstrokecolor{currentstroke}%
\pgfsetdash{}{0pt}%
\pgfpathmoveto{\pgfqpoint{5.151797in}{1.080890in}}%
\pgfpathlineto{\pgfqpoint{5.151797in}{3.227753in}}%
\pgfusepath{stroke}%
\end{pgfscope}%
\begin{pgfscope}%
\pgfsetbuttcap%
\pgfsetroundjoin%
\definecolor{currentfill}{rgb}{0.000000,0.000000,0.000000}%
\pgfsetfillcolor{currentfill}%
\pgfsetlinewidth{0.501875pt}%
\definecolor{currentstroke}{rgb}{0.000000,0.000000,0.000000}%
\pgfsetstrokecolor{currentstroke}%
\pgfsetdash{}{0pt}%
\pgfsys@defobject{currentmarker}{\pgfqpoint{0.000000in}{0.000000in}}{\pgfqpoint{0.000000in}{0.020833in}}{%
\pgfpathmoveto{\pgfqpoint{0.000000in}{0.000000in}}%
\pgfpathlineto{\pgfqpoint{0.000000in}{0.020833in}}%
\pgfusepath{stroke,fill}%
}%
\begin{pgfscope}%
\pgfsys@transformshift{5.151797in}{1.080890in}%
\pgfsys@useobject{currentmarker}{}%
\end{pgfscope}%
\end{pgfscope}%
\begin{pgfscope}%
\pgfsetbuttcap%
\pgfsetroundjoin%
\definecolor{currentfill}{rgb}{0.000000,0.000000,0.000000}%
\pgfsetfillcolor{currentfill}%
\pgfsetlinewidth{0.501875pt}%
\definecolor{currentstroke}{rgb}{0.000000,0.000000,0.000000}%
\pgfsetstrokecolor{currentstroke}%
\pgfsetdash{}{0pt}%
\pgfsys@defobject{currentmarker}{\pgfqpoint{0.000000in}{-0.020833in}}{\pgfqpoint{0.000000in}{0.000000in}}{%
\pgfpathmoveto{\pgfqpoint{0.000000in}{0.000000in}}%
\pgfpathlineto{\pgfqpoint{0.000000in}{-0.020833in}}%
\pgfusepath{stroke,fill}%
}%
\begin{pgfscope}%
\pgfsys@transformshift{5.151797in}{3.227753in}%
\pgfsys@useobject{currentmarker}{}%
\end{pgfscope}%
\end{pgfscope}%
\begin{pgfscope}%
\pgfpathrectangle{\pgfqpoint{0.481681in}{1.080890in}}{\pgfqpoint{5.785672in}{2.146863in}}%
\pgfusepath{clip}%
\pgfsetrectcap%
\pgfsetroundjoin%
\pgfsetlinewidth{0.100375pt}%
\definecolor{currentstroke}{rgb}{0.827451,0.827451,0.827451}%
\pgfsetstrokecolor{currentstroke}%
\pgfsetdash{}{0pt}%
\pgfpathmoveto{\pgfqpoint{5.187494in}{1.080890in}}%
\pgfpathlineto{\pgfqpoint{5.187494in}{3.227753in}}%
\pgfusepath{stroke}%
\end{pgfscope}%
\begin{pgfscope}%
\pgfsetbuttcap%
\pgfsetroundjoin%
\definecolor{currentfill}{rgb}{0.000000,0.000000,0.000000}%
\pgfsetfillcolor{currentfill}%
\pgfsetlinewidth{0.501875pt}%
\definecolor{currentstroke}{rgb}{0.000000,0.000000,0.000000}%
\pgfsetstrokecolor{currentstroke}%
\pgfsetdash{}{0pt}%
\pgfsys@defobject{currentmarker}{\pgfqpoint{0.000000in}{0.000000in}}{\pgfqpoint{0.000000in}{0.020833in}}{%
\pgfpathmoveto{\pgfqpoint{0.000000in}{0.000000in}}%
\pgfpathlineto{\pgfqpoint{0.000000in}{0.020833in}}%
\pgfusepath{stroke,fill}%
}%
\begin{pgfscope}%
\pgfsys@transformshift{5.187494in}{1.080890in}%
\pgfsys@useobject{currentmarker}{}%
\end{pgfscope}%
\end{pgfscope}%
\begin{pgfscope}%
\pgfsetbuttcap%
\pgfsetroundjoin%
\definecolor{currentfill}{rgb}{0.000000,0.000000,0.000000}%
\pgfsetfillcolor{currentfill}%
\pgfsetlinewidth{0.501875pt}%
\definecolor{currentstroke}{rgb}{0.000000,0.000000,0.000000}%
\pgfsetstrokecolor{currentstroke}%
\pgfsetdash{}{0pt}%
\pgfsys@defobject{currentmarker}{\pgfqpoint{0.000000in}{-0.020833in}}{\pgfqpoint{0.000000in}{0.000000in}}{%
\pgfpathmoveto{\pgfqpoint{0.000000in}{0.000000in}}%
\pgfpathlineto{\pgfqpoint{0.000000in}{-0.020833in}}%
\pgfusepath{stroke,fill}%
}%
\begin{pgfscope}%
\pgfsys@transformshift{5.187494in}{3.227753in}%
\pgfsys@useobject{currentmarker}{}%
\end{pgfscope}%
\end{pgfscope}%
\begin{pgfscope}%
\pgfpathrectangle{\pgfqpoint{0.481681in}{1.080890in}}{\pgfqpoint{5.785672in}{2.146863in}}%
\pgfusepath{clip}%
\pgfsetrectcap%
\pgfsetroundjoin%
\pgfsetlinewidth{0.100375pt}%
\definecolor{currentstroke}{rgb}{0.827451,0.827451,0.827451}%
\pgfsetstrokecolor{currentstroke}%
\pgfsetdash{}{0pt}%
\pgfpathmoveto{\pgfqpoint{5.223192in}{1.080890in}}%
\pgfpathlineto{\pgfqpoint{5.223192in}{3.227753in}}%
\pgfusepath{stroke}%
\end{pgfscope}%
\begin{pgfscope}%
\pgfsetbuttcap%
\pgfsetroundjoin%
\definecolor{currentfill}{rgb}{0.000000,0.000000,0.000000}%
\pgfsetfillcolor{currentfill}%
\pgfsetlinewidth{0.501875pt}%
\definecolor{currentstroke}{rgb}{0.000000,0.000000,0.000000}%
\pgfsetstrokecolor{currentstroke}%
\pgfsetdash{}{0pt}%
\pgfsys@defobject{currentmarker}{\pgfqpoint{0.000000in}{0.000000in}}{\pgfqpoint{0.000000in}{0.020833in}}{%
\pgfpathmoveto{\pgfqpoint{0.000000in}{0.000000in}}%
\pgfpathlineto{\pgfqpoint{0.000000in}{0.020833in}}%
\pgfusepath{stroke,fill}%
}%
\begin{pgfscope}%
\pgfsys@transformshift{5.223192in}{1.080890in}%
\pgfsys@useobject{currentmarker}{}%
\end{pgfscope}%
\end{pgfscope}%
\begin{pgfscope}%
\pgfsetbuttcap%
\pgfsetroundjoin%
\definecolor{currentfill}{rgb}{0.000000,0.000000,0.000000}%
\pgfsetfillcolor{currentfill}%
\pgfsetlinewidth{0.501875pt}%
\definecolor{currentstroke}{rgb}{0.000000,0.000000,0.000000}%
\pgfsetstrokecolor{currentstroke}%
\pgfsetdash{}{0pt}%
\pgfsys@defobject{currentmarker}{\pgfqpoint{0.000000in}{-0.020833in}}{\pgfqpoint{0.000000in}{0.000000in}}{%
\pgfpathmoveto{\pgfqpoint{0.000000in}{0.000000in}}%
\pgfpathlineto{\pgfqpoint{0.000000in}{-0.020833in}}%
\pgfusepath{stroke,fill}%
}%
\begin{pgfscope}%
\pgfsys@transformshift{5.223192in}{3.227753in}%
\pgfsys@useobject{currentmarker}{}%
\end{pgfscope}%
\end{pgfscope}%
\begin{pgfscope}%
\pgfpathrectangle{\pgfqpoint{0.481681in}{1.080890in}}{\pgfqpoint{5.785672in}{2.146863in}}%
\pgfusepath{clip}%
\pgfsetrectcap%
\pgfsetroundjoin%
\pgfsetlinewidth{0.100375pt}%
\definecolor{currentstroke}{rgb}{0.827451,0.827451,0.827451}%
\pgfsetstrokecolor{currentstroke}%
\pgfsetdash{}{0pt}%
\pgfpathmoveto{\pgfqpoint{5.258889in}{1.080890in}}%
\pgfpathlineto{\pgfqpoint{5.258889in}{3.227753in}}%
\pgfusepath{stroke}%
\end{pgfscope}%
\begin{pgfscope}%
\pgfsetbuttcap%
\pgfsetroundjoin%
\definecolor{currentfill}{rgb}{0.000000,0.000000,0.000000}%
\pgfsetfillcolor{currentfill}%
\pgfsetlinewidth{0.501875pt}%
\definecolor{currentstroke}{rgb}{0.000000,0.000000,0.000000}%
\pgfsetstrokecolor{currentstroke}%
\pgfsetdash{}{0pt}%
\pgfsys@defobject{currentmarker}{\pgfqpoint{0.000000in}{0.000000in}}{\pgfqpoint{0.000000in}{0.020833in}}{%
\pgfpathmoveto{\pgfqpoint{0.000000in}{0.000000in}}%
\pgfpathlineto{\pgfqpoint{0.000000in}{0.020833in}}%
\pgfusepath{stroke,fill}%
}%
\begin{pgfscope}%
\pgfsys@transformshift{5.258889in}{1.080890in}%
\pgfsys@useobject{currentmarker}{}%
\end{pgfscope}%
\end{pgfscope}%
\begin{pgfscope}%
\pgfsetbuttcap%
\pgfsetroundjoin%
\definecolor{currentfill}{rgb}{0.000000,0.000000,0.000000}%
\pgfsetfillcolor{currentfill}%
\pgfsetlinewidth{0.501875pt}%
\definecolor{currentstroke}{rgb}{0.000000,0.000000,0.000000}%
\pgfsetstrokecolor{currentstroke}%
\pgfsetdash{}{0pt}%
\pgfsys@defobject{currentmarker}{\pgfqpoint{0.000000in}{-0.020833in}}{\pgfqpoint{0.000000in}{0.000000in}}{%
\pgfpathmoveto{\pgfqpoint{0.000000in}{0.000000in}}%
\pgfpathlineto{\pgfqpoint{0.000000in}{-0.020833in}}%
\pgfusepath{stroke,fill}%
}%
\begin{pgfscope}%
\pgfsys@transformshift{5.258889in}{3.227753in}%
\pgfsys@useobject{currentmarker}{}%
\end{pgfscope}%
\end{pgfscope}%
\begin{pgfscope}%
\pgfpathrectangle{\pgfqpoint{0.481681in}{1.080890in}}{\pgfqpoint{5.785672in}{2.146863in}}%
\pgfusepath{clip}%
\pgfsetrectcap%
\pgfsetroundjoin%
\pgfsetlinewidth{0.100375pt}%
\definecolor{currentstroke}{rgb}{0.827451,0.827451,0.827451}%
\pgfsetstrokecolor{currentstroke}%
\pgfsetdash{}{0pt}%
\pgfpathmoveto{\pgfqpoint{5.294586in}{1.080890in}}%
\pgfpathlineto{\pgfqpoint{5.294586in}{3.227753in}}%
\pgfusepath{stroke}%
\end{pgfscope}%
\begin{pgfscope}%
\pgfsetbuttcap%
\pgfsetroundjoin%
\definecolor{currentfill}{rgb}{0.000000,0.000000,0.000000}%
\pgfsetfillcolor{currentfill}%
\pgfsetlinewidth{0.501875pt}%
\definecolor{currentstroke}{rgb}{0.000000,0.000000,0.000000}%
\pgfsetstrokecolor{currentstroke}%
\pgfsetdash{}{0pt}%
\pgfsys@defobject{currentmarker}{\pgfqpoint{0.000000in}{0.000000in}}{\pgfqpoint{0.000000in}{0.020833in}}{%
\pgfpathmoveto{\pgfqpoint{0.000000in}{0.000000in}}%
\pgfpathlineto{\pgfqpoint{0.000000in}{0.020833in}}%
\pgfusepath{stroke,fill}%
}%
\begin{pgfscope}%
\pgfsys@transformshift{5.294586in}{1.080890in}%
\pgfsys@useobject{currentmarker}{}%
\end{pgfscope}%
\end{pgfscope}%
\begin{pgfscope}%
\pgfsetbuttcap%
\pgfsetroundjoin%
\definecolor{currentfill}{rgb}{0.000000,0.000000,0.000000}%
\pgfsetfillcolor{currentfill}%
\pgfsetlinewidth{0.501875pt}%
\definecolor{currentstroke}{rgb}{0.000000,0.000000,0.000000}%
\pgfsetstrokecolor{currentstroke}%
\pgfsetdash{}{0pt}%
\pgfsys@defobject{currentmarker}{\pgfqpoint{0.000000in}{-0.020833in}}{\pgfqpoint{0.000000in}{0.000000in}}{%
\pgfpathmoveto{\pgfqpoint{0.000000in}{0.000000in}}%
\pgfpathlineto{\pgfqpoint{0.000000in}{-0.020833in}}%
\pgfusepath{stroke,fill}%
}%
\begin{pgfscope}%
\pgfsys@transformshift{5.294586in}{3.227753in}%
\pgfsys@useobject{currentmarker}{}%
\end{pgfscope}%
\end{pgfscope}%
\begin{pgfscope}%
\pgfpathrectangle{\pgfqpoint{0.481681in}{1.080890in}}{\pgfqpoint{5.785672in}{2.146863in}}%
\pgfusepath{clip}%
\pgfsetrectcap%
\pgfsetroundjoin%
\pgfsetlinewidth{0.100375pt}%
\definecolor{currentstroke}{rgb}{0.827451,0.827451,0.827451}%
\pgfsetstrokecolor{currentstroke}%
\pgfsetdash{}{0pt}%
\pgfpathmoveto{\pgfqpoint{5.330284in}{1.080890in}}%
\pgfpathlineto{\pgfqpoint{5.330284in}{3.227753in}}%
\pgfusepath{stroke}%
\end{pgfscope}%
\begin{pgfscope}%
\pgfsetbuttcap%
\pgfsetroundjoin%
\definecolor{currentfill}{rgb}{0.000000,0.000000,0.000000}%
\pgfsetfillcolor{currentfill}%
\pgfsetlinewidth{0.501875pt}%
\definecolor{currentstroke}{rgb}{0.000000,0.000000,0.000000}%
\pgfsetstrokecolor{currentstroke}%
\pgfsetdash{}{0pt}%
\pgfsys@defobject{currentmarker}{\pgfqpoint{0.000000in}{0.000000in}}{\pgfqpoint{0.000000in}{0.020833in}}{%
\pgfpathmoveto{\pgfqpoint{0.000000in}{0.000000in}}%
\pgfpathlineto{\pgfqpoint{0.000000in}{0.020833in}}%
\pgfusepath{stroke,fill}%
}%
\begin{pgfscope}%
\pgfsys@transformshift{5.330284in}{1.080890in}%
\pgfsys@useobject{currentmarker}{}%
\end{pgfscope}%
\end{pgfscope}%
\begin{pgfscope}%
\pgfsetbuttcap%
\pgfsetroundjoin%
\definecolor{currentfill}{rgb}{0.000000,0.000000,0.000000}%
\pgfsetfillcolor{currentfill}%
\pgfsetlinewidth{0.501875pt}%
\definecolor{currentstroke}{rgb}{0.000000,0.000000,0.000000}%
\pgfsetstrokecolor{currentstroke}%
\pgfsetdash{}{0pt}%
\pgfsys@defobject{currentmarker}{\pgfqpoint{0.000000in}{-0.020833in}}{\pgfqpoint{0.000000in}{0.000000in}}{%
\pgfpathmoveto{\pgfqpoint{0.000000in}{0.000000in}}%
\pgfpathlineto{\pgfqpoint{0.000000in}{-0.020833in}}%
\pgfusepath{stroke,fill}%
}%
\begin{pgfscope}%
\pgfsys@transformshift{5.330284in}{3.227753in}%
\pgfsys@useobject{currentmarker}{}%
\end{pgfscope}%
\end{pgfscope}%
\begin{pgfscope}%
\pgfpathrectangle{\pgfqpoint{0.481681in}{1.080890in}}{\pgfqpoint{5.785672in}{2.146863in}}%
\pgfusepath{clip}%
\pgfsetrectcap%
\pgfsetroundjoin%
\pgfsetlinewidth{0.100375pt}%
\definecolor{currentstroke}{rgb}{0.827451,0.827451,0.827451}%
\pgfsetstrokecolor{currentstroke}%
\pgfsetdash{}{0pt}%
\pgfpathmoveto{\pgfqpoint{5.365981in}{1.080890in}}%
\pgfpathlineto{\pgfqpoint{5.365981in}{3.227753in}}%
\pgfusepath{stroke}%
\end{pgfscope}%
\begin{pgfscope}%
\pgfsetbuttcap%
\pgfsetroundjoin%
\definecolor{currentfill}{rgb}{0.000000,0.000000,0.000000}%
\pgfsetfillcolor{currentfill}%
\pgfsetlinewidth{0.501875pt}%
\definecolor{currentstroke}{rgb}{0.000000,0.000000,0.000000}%
\pgfsetstrokecolor{currentstroke}%
\pgfsetdash{}{0pt}%
\pgfsys@defobject{currentmarker}{\pgfqpoint{0.000000in}{0.000000in}}{\pgfqpoint{0.000000in}{0.020833in}}{%
\pgfpathmoveto{\pgfqpoint{0.000000in}{0.000000in}}%
\pgfpathlineto{\pgfqpoint{0.000000in}{0.020833in}}%
\pgfusepath{stroke,fill}%
}%
\begin{pgfscope}%
\pgfsys@transformshift{5.365981in}{1.080890in}%
\pgfsys@useobject{currentmarker}{}%
\end{pgfscope}%
\end{pgfscope}%
\begin{pgfscope}%
\pgfsetbuttcap%
\pgfsetroundjoin%
\definecolor{currentfill}{rgb}{0.000000,0.000000,0.000000}%
\pgfsetfillcolor{currentfill}%
\pgfsetlinewidth{0.501875pt}%
\definecolor{currentstroke}{rgb}{0.000000,0.000000,0.000000}%
\pgfsetstrokecolor{currentstroke}%
\pgfsetdash{}{0pt}%
\pgfsys@defobject{currentmarker}{\pgfqpoint{0.000000in}{-0.020833in}}{\pgfqpoint{0.000000in}{0.000000in}}{%
\pgfpathmoveto{\pgfqpoint{0.000000in}{0.000000in}}%
\pgfpathlineto{\pgfqpoint{0.000000in}{-0.020833in}}%
\pgfusepath{stroke,fill}%
}%
\begin{pgfscope}%
\pgfsys@transformshift{5.365981in}{3.227753in}%
\pgfsys@useobject{currentmarker}{}%
\end{pgfscope}%
\end{pgfscope}%
\begin{pgfscope}%
\pgfpathrectangle{\pgfqpoint{0.481681in}{1.080890in}}{\pgfqpoint{5.785672in}{2.146863in}}%
\pgfusepath{clip}%
\pgfsetrectcap%
\pgfsetroundjoin%
\pgfsetlinewidth{0.100375pt}%
\definecolor{currentstroke}{rgb}{0.827451,0.827451,0.827451}%
\pgfsetstrokecolor{currentstroke}%
\pgfsetdash{}{0pt}%
\pgfpathmoveto{\pgfqpoint{5.437376in}{1.080890in}}%
\pgfpathlineto{\pgfqpoint{5.437376in}{3.227753in}}%
\pgfusepath{stroke}%
\end{pgfscope}%
\begin{pgfscope}%
\pgfsetbuttcap%
\pgfsetroundjoin%
\definecolor{currentfill}{rgb}{0.000000,0.000000,0.000000}%
\pgfsetfillcolor{currentfill}%
\pgfsetlinewidth{0.501875pt}%
\definecolor{currentstroke}{rgb}{0.000000,0.000000,0.000000}%
\pgfsetstrokecolor{currentstroke}%
\pgfsetdash{}{0pt}%
\pgfsys@defobject{currentmarker}{\pgfqpoint{0.000000in}{0.000000in}}{\pgfqpoint{0.000000in}{0.020833in}}{%
\pgfpathmoveto{\pgfqpoint{0.000000in}{0.000000in}}%
\pgfpathlineto{\pgfqpoint{0.000000in}{0.020833in}}%
\pgfusepath{stroke,fill}%
}%
\begin{pgfscope}%
\pgfsys@transformshift{5.437376in}{1.080890in}%
\pgfsys@useobject{currentmarker}{}%
\end{pgfscope}%
\end{pgfscope}%
\begin{pgfscope}%
\pgfsetbuttcap%
\pgfsetroundjoin%
\definecolor{currentfill}{rgb}{0.000000,0.000000,0.000000}%
\pgfsetfillcolor{currentfill}%
\pgfsetlinewidth{0.501875pt}%
\definecolor{currentstroke}{rgb}{0.000000,0.000000,0.000000}%
\pgfsetstrokecolor{currentstroke}%
\pgfsetdash{}{0pt}%
\pgfsys@defobject{currentmarker}{\pgfqpoint{0.000000in}{-0.020833in}}{\pgfqpoint{0.000000in}{0.000000in}}{%
\pgfpathmoveto{\pgfqpoint{0.000000in}{0.000000in}}%
\pgfpathlineto{\pgfqpoint{0.000000in}{-0.020833in}}%
\pgfusepath{stroke,fill}%
}%
\begin{pgfscope}%
\pgfsys@transformshift{5.437376in}{3.227753in}%
\pgfsys@useobject{currentmarker}{}%
\end{pgfscope}%
\end{pgfscope}%
\begin{pgfscope}%
\pgfpathrectangle{\pgfqpoint{0.481681in}{1.080890in}}{\pgfqpoint{5.785672in}{2.146863in}}%
\pgfusepath{clip}%
\pgfsetrectcap%
\pgfsetroundjoin%
\pgfsetlinewidth{0.100375pt}%
\definecolor{currentstroke}{rgb}{0.827451,0.827451,0.827451}%
\pgfsetstrokecolor{currentstroke}%
\pgfsetdash{}{0pt}%
\pgfpathmoveto{\pgfqpoint{5.473073in}{1.080890in}}%
\pgfpathlineto{\pgfqpoint{5.473073in}{3.227753in}}%
\pgfusepath{stroke}%
\end{pgfscope}%
\begin{pgfscope}%
\pgfsetbuttcap%
\pgfsetroundjoin%
\definecolor{currentfill}{rgb}{0.000000,0.000000,0.000000}%
\pgfsetfillcolor{currentfill}%
\pgfsetlinewidth{0.501875pt}%
\definecolor{currentstroke}{rgb}{0.000000,0.000000,0.000000}%
\pgfsetstrokecolor{currentstroke}%
\pgfsetdash{}{0pt}%
\pgfsys@defobject{currentmarker}{\pgfqpoint{0.000000in}{0.000000in}}{\pgfqpoint{0.000000in}{0.020833in}}{%
\pgfpathmoveto{\pgfqpoint{0.000000in}{0.000000in}}%
\pgfpathlineto{\pgfqpoint{0.000000in}{0.020833in}}%
\pgfusepath{stroke,fill}%
}%
\begin{pgfscope}%
\pgfsys@transformshift{5.473073in}{1.080890in}%
\pgfsys@useobject{currentmarker}{}%
\end{pgfscope}%
\end{pgfscope}%
\begin{pgfscope}%
\pgfsetbuttcap%
\pgfsetroundjoin%
\definecolor{currentfill}{rgb}{0.000000,0.000000,0.000000}%
\pgfsetfillcolor{currentfill}%
\pgfsetlinewidth{0.501875pt}%
\definecolor{currentstroke}{rgb}{0.000000,0.000000,0.000000}%
\pgfsetstrokecolor{currentstroke}%
\pgfsetdash{}{0pt}%
\pgfsys@defobject{currentmarker}{\pgfqpoint{0.000000in}{-0.020833in}}{\pgfqpoint{0.000000in}{0.000000in}}{%
\pgfpathmoveto{\pgfqpoint{0.000000in}{0.000000in}}%
\pgfpathlineto{\pgfqpoint{0.000000in}{-0.020833in}}%
\pgfusepath{stroke,fill}%
}%
\begin{pgfscope}%
\pgfsys@transformshift{5.473073in}{3.227753in}%
\pgfsys@useobject{currentmarker}{}%
\end{pgfscope}%
\end{pgfscope}%
\begin{pgfscope}%
\pgfpathrectangle{\pgfqpoint{0.481681in}{1.080890in}}{\pgfqpoint{5.785672in}{2.146863in}}%
\pgfusepath{clip}%
\pgfsetrectcap%
\pgfsetroundjoin%
\pgfsetlinewidth{0.100375pt}%
\definecolor{currentstroke}{rgb}{0.827451,0.827451,0.827451}%
\pgfsetstrokecolor{currentstroke}%
\pgfsetdash{}{0pt}%
\pgfpathmoveto{\pgfqpoint{5.508770in}{1.080890in}}%
\pgfpathlineto{\pgfqpoint{5.508770in}{3.227753in}}%
\pgfusepath{stroke}%
\end{pgfscope}%
\begin{pgfscope}%
\pgfsetbuttcap%
\pgfsetroundjoin%
\definecolor{currentfill}{rgb}{0.000000,0.000000,0.000000}%
\pgfsetfillcolor{currentfill}%
\pgfsetlinewidth{0.501875pt}%
\definecolor{currentstroke}{rgb}{0.000000,0.000000,0.000000}%
\pgfsetstrokecolor{currentstroke}%
\pgfsetdash{}{0pt}%
\pgfsys@defobject{currentmarker}{\pgfqpoint{0.000000in}{0.000000in}}{\pgfqpoint{0.000000in}{0.020833in}}{%
\pgfpathmoveto{\pgfqpoint{0.000000in}{0.000000in}}%
\pgfpathlineto{\pgfqpoint{0.000000in}{0.020833in}}%
\pgfusepath{stroke,fill}%
}%
\begin{pgfscope}%
\pgfsys@transformshift{5.508770in}{1.080890in}%
\pgfsys@useobject{currentmarker}{}%
\end{pgfscope}%
\end{pgfscope}%
\begin{pgfscope}%
\pgfsetbuttcap%
\pgfsetroundjoin%
\definecolor{currentfill}{rgb}{0.000000,0.000000,0.000000}%
\pgfsetfillcolor{currentfill}%
\pgfsetlinewidth{0.501875pt}%
\definecolor{currentstroke}{rgb}{0.000000,0.000000,0.000000}%
\pgfsetstrokecolor{currentstroke}%
\pgfsetdash{}{0pt}%
\pgfsys@defobject{currentmarker}{\pgfqpoint{0.000000in}{-0.020833in}}{\pgfqpoint{0.000000in}{0.000000in}}{%
\pgfpathmoveto{\pgfqpoint{0.000000in}{0.000000in}}%
\pgfpathlineto{\pgfqpoint{0.000000in}{-0.020833in}}%
\pgfusepath{stroke,fill}%
}%
\begin{pgfscope}%
\pgfsys@transformshift{5.508770in}{3.227753in}%
\pgfsys@useobject{currentmarker}{}%
\end{pgfscope}%
\end{pgfscope}%
\begin{pgfscope}%
\pgfpathrectangle{\pgfqpoint{0.481681in}{1.080890in}}{\pgfqpoint{5.785672in}{2.146863in}}%
\pgfusepath{clip}%
\pgfsetrectcap%
\pgfsetroundjoin%
\pgfsetlinewidth{0.100375pt}%
\definecolor{currentstroke}{rgb}{0.827451,0.827451,0.827451}%
\pgfsetstrokecolor{currentstroke}%
\pgfsetdash{}{0pt}%
\pgfpathmoveto{\pgfqpoint{5.544468in}{1.080890in}}%
\pgfpathlineto{\pgfqpoint{5.544468in}{3.227753in}}%
\pgfusepath{stroke}%
\end{pgfscope}%
\begin{pgfscope}%
\pgfsetbuttcap%
\pgfsetroundjoin%
\definecolor{currentfill}{rgb}{0.000000,0.000000,0.000000}%
\pgfsetfillcolor{currentfill}%
\pgfsetlinewidth{0.501875pt}%
\definecolor{currentstroke}{rgb}{0.000000,0.000000,0.000000}%
\pgfsetstrokecolor{currentstroke}%
\pgfsetdash{}{0pt}%
\pgfsys@defobject{currentmarker}{\pgfqpoint{0.000000in}{0.000000in}}{\pgfqpoint{0.000000in}{0.020833in}}{%
\pgfpathmoveto{\pgfqpoint{0.000000in}{0.000000in}}%
\pgfpathlineto{\pgfqpoint{0.000000in}{0.020833in}}%
\pgfusepath{stroke,fill}%
}%
\begin{pgfscope}%
\pgfsys@transformshift{5.544468in}{1.080890in}%
\pgfsys@useobject{currentmarker}{}%
\end{pgfscope}%
\end{pgfscope}%
\begin{pgfscope}%
\pgfsetbuttcap%
\pgfsetroundjoin%
\definecolor{currentfill}{rgb}{0.000000,0.000000,0.000000}%
\pgfsetfillcolor{currentfill}%
\pgfsetlinewidth{0.501875pt}%
\definecolor{currentstroke}{rgb}{0.000000,0.000000,0.000000}%
\pgfsetstrokecolor{currentstroke}%
\pgfsetdash{}{0pt}%
\pgfsys@defobject{currentmarker}{\pgfqpoint{0.000000in}{-0.020833in}}{\pgfqpoint{0.000000in}{0.000000in}}{%
\pgfpathmoveto{\pgfqpoint{0.000000in}{0.000000in}}%
\pgfpathlineto{\pgfqpoint{0.000000in}{-0.020833in}}%
\pgfusepath{stroke,fill}%
}%
\begin{pgfscope}%
\pgfsys@transformshift{5.544468in}{3.227753in}%
\pgfsys@useobject{currentmarker}{}%
\end{pgfscope}%
\end{pgfscope}%
\begin{pgfscope}%
\pgfpathrectangle{\pgfqpoint{0.481681in}{1.080890in}}{\pgfqpoint{5.785672in}{2.146863in}}%
\pgfusepath{clip}%
\pgfsetrectcap%
\pgfsetroundjoin%
\pgfsetlinewidth{0.100375pt}%
\definecolor{currentstroke}{rgb}{0.827451,0.827451,0.827451}%
\pgfsetstrokecolor{currentstroke}%
\pgfsetdash{}{0pt}%
\pgfpathmoveto{\pgfqpoint{5.580165in}{1.080890in}}%
\pgfpathlineto{\pgfqpoint{5.580165in}{3.227753in}}%
\pgfusepath{stroke}%
\end{pgfscope}%
\begin{pgfscope}%
\pgfsetbuttcap%
\pgfsetroundjoin%
\definecolor{currentfill}{rgb}{0.000000,0.000000,0.000000}%
\pgfsetfillcolor{currentfill}%
\pgfsetlinewidth{0.501875pt}%
\definecolor{currentstroke}{rgb}{0.000000,0.000000,0.000000}%
\pgfsetstrokecolor{currentstroke}%
\pgfsetdash{}{0pt}%
\pgfsys@defobject{currentmarker}{\pgfqpoint{0.000000in}{0.000000in}}{\pgfqpoint{0.000000in}{0.020833in}}{%
\pgfpathmoveto{\pgfqpoint{0.000000in}{0.000000in}}%
\pgfpathlineto{\pgfqpoint{0.000000in}{0.020833in}}%
\pgfusepath{stroke,fill}%
}%
\begin{pgfscope}%
\pgfsys@transformshift{5.580165in}{1.080890in}%
\pgfsys@useobject{currentmarker}{}%
\end{pgfscope}%
\end{pgfscope}%
\begin{pgfscope}%
\pgfsetbuttcap%
\pgfsetroundjoin%
\definecolor{currentfill}{rgb}{0.000000,0.000000,0.000000}%
\pgfsetfillcolor{currentfill}%
\pgfsetlinewidth{0.501875pt}%
\definecolor{currentstroke}{rgb}{0.000000,0.000000,0.000000}%
\pgfsetstrokecolor{currentstroke}%
\pgfsetdash{}{0pt}%
\pgfsys@defobject{currentmarker}{\pgfqpoint{0.000000in}{-0.020833in}}{\pgfqpoint{0.000000in}{0.000000in}}{%
\pgfpathmoveto{\pgfqpoint{0.000000in}{0.000000in}}%
\pgfpathlineto{\pgfqpoint{0.000000in}{-0.020833in}}%
\pgfusepath{stroke,fill}%
}%
\begin{pgfscope}%
\pgfsys@transformshift{5.580165in}{3.227753in}%
\pgfsys@useobject{currentmarker}{}%
\end{pgfscope}%
\end{pgfscope}%
\begin{pgfscope}%
\pgfpathrectangle{\pgfqpoint{0.481681in}{1.080890in}}{\pgfqpoint{5.785672in}{2.146863in}}%
\pgfusepath{clip}%
\pgfsetrectcap%
\pgfsetroundjoin%
\pgfsetlinewidth{0.100375pt}%
\definecolor{currentstroke}{rgb}{0.827451,0.827451,0.827451}%
\pgfsetstrokecolor{currentstroke}%
\pgfsetdash{}{0pt}%
\pgfpathmoveto{\pgfqpoint{5.615862in}{1.080890in}}%
\pgfpathlineto{\pgfqpoint{5.615862in}{3.227753in}}%
\pgfusepath{stroke}%
\end{pgfscope}%
\begin{pgfscope}%
\pgfsetbuttcap%
\pgfsetroundjoin%
\definecolor{currentfill}{rgb}{0.000000,0.000000,0.000000}%
\pgfsetfillcolor{currentfill}%
\pgfsetlinewidth{0.501875pt}%
\definecolor{currentstroke}{rgb}{0.000000,0.000000,0.000000}%
\pgfsetstrokecolor{currentstroke}%
\pgfsetdash{}{0pt}%
\pgfsys@defobject{currentmarker}{\pgfqpoint{0.000000in}{0.000000in}}{\pgfqpoint{0.000000in}{0.020833in}}{%
\pgfpathmoveto{\pgfqpoint{0.000000in}{0.000000in}}%
\pgfpathlineto{\pgfqpoint{0.000000in}{0.020833in}}%
\pgfusepath{stroke,fill}%
}%
\begin{pgfscope}%
\pgfsys@transformshift{5.615862in}{1.080890in}%
\pgfsys@useobject{currentmarker}{}%
\end{pgfscope}%
\end{pgfscope}%
\begin{pgfscope}%
\pgfsetbuttcap%
\pgfsetroundjoin%
\definecolor{currentfill}{rgb}{0.000000,0.000000,0.000000}%
\pgfsetfillcolor{currentfill}%
\pgfsetlinewidth{0.501875pt}%
\definecolor{currentstroke}{rgb}{0.000000,0.000000,0.000000}%
\pgfsetstrokecolor{currentstroke}%
\pgfsetdash{}{0pt}%
\pgfsys@defobject{currentmarker}{\pgfqpoint{0.000000in}{-0.020833in}}{\pgfqpoint{0.000000in}{0.000000in}}{%
\pgfpathmoveto{\pgfqpoint{0.000000in}{0.000000in}}%
\pgfpathlineto{\pgfqpoint{0.000000in}{-0.020833in}}%
\pgfusepath{stroke,fill}%
}%
\begin{pgfscope}%
\pgfsys@transformshift{5.615862in}{3.227753in}%
\pgfsys@useobject{currentmarker}{}%
\end{pgfscope}%
\end{pgfscope}%
\begin{pgfscope}%
\pgfpathrectangle{\pgfqpoint{0.481681in}{1.080890in}}{\pgfqpoint{5.785672in}{2.146863in}}%
\pgfusepath{clip}%
\pgfsetrectcap%
\pgfsetroundjoin%
\pgfsetlinewidth{0.100375pt}%
\definecolor{currentstroke}{rgb}{0.827451,0.827451,0.827451}%
\pgfsetstrokecolor{currentstroke}%
\pgfsetdash{}{0pt}%
\pgfpathmoveto{\pgfqpoint{5.651559in}{1.080890in}}%
\pgfpathlineto{\pgfqpoint{5.651559in}{3.227753in}}%
\pgfusepath{stroke}%
\end{pgfscope}%
\begin{pgfscope}%
\pgfsetbuttcap%
\pgfsetroundjoin%
\definecolor{currentfill}{rgb}{0.000000,0.000000,0.000000}%
\pgfsetfillcolor{currentfill}%
\pgfsetlinewidth{0.501875pt}%
\definecolor{currentstroke}{rgb}{0.000000,0.000000,0.000000}%
\pgfsetstrokecolor{currentstroke}%
\pgfsetdash{}{0pt}%
\pgfsys@defobject{currentmarker}{\pgfqpoint{0.000000in}{0.000000in}}{\pgfqpoint{0.000000in}{0.020833in}}{%
\pgfpathmoveto{\pgfqpoint{0.000000in}{0.000000in}}%
\pgfpathlineto{\pgfqpoint{0.000000in}{0.020833in}}%
\pgfusepath{stroke,fill}%
}%
\begin{pgfscope}%
\pgfsys@transformshift{5.651559in}{1.080890in}%
\pgfsys@useobject{currentmarker}{}%
\end{pgfscope}%
\end{pgfscope}%
\begin{pgfscope}%
\pgfsetbuttcap%
\pgfsetroundjoin%
\definecolor{currentfill}{rgb}{0.000000,0.000000,0.000000}%
\pgfsetfillcolor{currentfill}%
\pgfsetlinewidth{0.501875pt}%
\definecolor{currentstroke}{rgb}{0.000000,0.000000,0.000000}%
\pgfsetstrokecolor{currentstroke}%
\pgfsetdash{}{0pt}%
\pgfsys@defobject{currentmarker}{\pgfqpoint{0.000000in}{-0.020833in}}{\pgfqpoint{0.000000in}{0.000000in}}{%
\pgfpathmoveto{\pgfqpoint{0.000000in}{0.000000in}}%
\pgfpathlineto{\pgfqpoint{0.000000in}{-0.020833in}}%
\pgfusepath{stroke,fill}%
}%
\begin{pgfscope}%
\pgfsys@transformshift{5.651559in}{3.227753in}%
\pgfsys@useobject{currentmarker}{}%
\end{pgfscope}%
\end{pgfscope}%
\begin{pgfscope}%
\pgfpathrectangle{\pgfqpoint{0.481681in}{1.080890in}}{\pgfqpoint{5.785672in}{2.146863in}}%
\pgfusepath{clip}%
\pgfsetrectcap%
\pgfsetroundjoin%
\pgfsetlinewidth{0.100375pt}%
\definecolor{currentstroke}{rgb}{0.827451,0.827451,0.827451}%
\pgfsetstrokecolor{currentstroke}%
\pgfsetdash{}{0pt}%
\pgfpathmoveto{\pgfqpoint{5.687257in}{1.080890in}}%
\pgfpathlineto{\pgfqpoint{5.687257in}{3.227753in}}%
\pgfusepath{stroke}%
\end{pgfscope}%
\begin{pgfscope}%
\pgfsetbuttcap%
\pgfsetroundjoin%
\definecolor{currentfill}{rgb}{0.000000,0.000000,0.000000}%
\pgfsetfillcolor{currentfill}%
\pgfsetlinewidth{0.501875pt}%
\definecolor{currentstroke}{rgb}{0.000000,0.000000,0.000000}%
\pgfsetstrokecolor{currentstroke}%
\pgfsetdash{}{0pt}%
\pgfsys@defobject{currentmarker}{\pgfqpoint{0.000000in}{0.000000in}}{\pgfqpoint{0.000000in}{0.020833in}}{%
\pgfpathmoveto{\pgfqpoint{0.000000in}{0.000000in}}%
\pgfpathlineto{\pgfqpoint{0.000000in}{0.020833in}}%
\pgfusepath{stroke,fill}%
}%
\begin{pgfscope}%
\pgfsys@transformshift{5.687257in}{1.080890in}%
\pgfsys@useobject{currentmarker}{}%
\end{pgfscope}%
\end{pgfscope}%
\begin{pgfscope}%
\pgfsetbuttcap%
\pgfsetroundjoin%
\definecolor{currentfill}{rgb}{0.000000,0.000000,0.000000}%
\pgfsetfillcolor{currentfill}%
\pgfsetlinewidth{0.501875pt}%
\definecolor{currentstroke}{rgb}{0.000000,0.000000,0.000000}%
\pgfsetstrokecolor{currentstroke}%
\pgfsetdash{}{0pt}%
\pgfsys@defobject{currentmarker}{\pgfqpoint{0.000000in}{-0.020833in}}{\pgfqpoint{0.000000in}{0.000000in}}{%
\pgfpathmoveto{\pgfqpoint{0.000000in}{0.000000in}}%
\pgfpathlineto{\pgfqpoint{0.000000in}{-0.020833in}}%
\pgfusepath{stroke,fill}%
}%
\begin{pgfscope}%
\pgfsys@transformshift{5.687257in}{3.227753in}%
\pgfsys@useobject{currentmarker}{}%
\end{pgfscope}%
\end{pgfscope}%
\begin{pgfscope}%
\pgfpathrectangle{\pgfqpoint{0.481681in}{1.080890in}}{\pgfqpoint{5.785672in}{2.146863in}}%
\pgfusepath{clip}%
\pgfsetrectcap%
\pgfsetroundjoin%
\pgfsetlinewidth{0.100375pt}%
\definecolor{currentstroke}{rgb}{0.827451,0.827451,0.827451}%
\pgfsetstrokecolor{currentstroke}%
\pgfsetdash{}{0pt}%
\pgfpathmoveto{\pgfqpoint{5.722954in}{1.080890in}}%
\pgfpathlineto{\pgfqpoint{5.722954in}{3.227753in}}%
\pgfusepath{stroke}%
\end{pgfscope}%
\begin{pgfscope}%
\pgfsetbuttcap%
\pgfsetroundjoin%
\definecolor{currentfill}{rgb}{0.000000,0.000000,0.000000}%
\pgfsetfillcolor{currentfill}%
\pgfsetlinewidth{0.501875pt}%
\definecolor{currentstroke}{rgb}{0.000000,0.000000,0.000000}%
\pgfsetstrokecolor{currentstroke}%
\pgfsetdash{}{0pt}%
\pgfsys@defobject{currentmarker}{\pgfqpoint{0.000000in}{0.000000in}}{\pgfqpoint{0.000000in}{0.020833in}}{%
\pgfpathmoveto{\pgfqpoint{0.000000in}{0.000000in}}%
\pgfpathlineto{\pgfqpoint{0.000000in}{0.020833in}}%
\pgfusepath{stroke,fill}%
}%
\begin{pgfscope}%
\pgfsys@transformshift{5.722954in}{1.080890in}%
\pgfsys@useobject{currentmarker}{}%
\end{pgfscope}%
\end{pgfscope}%
\begin{pgfscope}%
\pgfsetbuttcap%
\pgfsetroundjoin%
\definecolor{currentfill}{rgb}{0.000000,0.000000,0.000000}%
\pgfsetfillcolor{currentfill}%
\pgfsetlinewidth{0.501875pt}%
\definecolor{currentstroke}{rgb}{0.000000,0.000000,0.000000}%
\pgfsetstrokecolor{currentstroke}%
\pgfsetdash{}{0pt}%
\pgfsys@defobject{currentmarker}{\pgfqpoint{0.000000in}{-0.020833in}}{\pgfqpoint{0.000000in}{0.000000in}}{%
\pgfpathmoveto{\pgfqpoint{0.000000in}{0.000000in}}%
\pgfpathlineto{\pgfqpoint{0.000000in}{-0.020833in}}%
\pgfusepath{stroke,fill}%
}%
\begin{pgfscope}%
\pgfsys@transformshift{5.722954in}{3.227753in}%
\pgfsys@useobject{currentmarker}{}%
\end{pgfscope}%
\end{pgfscope}%
\begin{pgfscope}%
\pgfpathrectangle{\pgfqpoint{0.481681in}{1.080890in}}{\pgfqpoint{5.785672in}{2.146863in}}%
\pgfusepath{clip}%
\pgfsetrectcap%
\pgfsetroundjoin%
\pgfsetlinewidth{0.100375pt}%
\definecolor{currentstroke}{rgb}{0.827451,0.827451,0.827451}%
\pgfsetstrokecolor{currentstroke}%
\pgfsetdash{}{0pt}%
\pgfpathmoveto{\pgfqpoint{5.758651in}{1.080890in}}%
\pgfpathlineto{\pgfqpoint{5.758651in}{3.227753in}}%
\pgfusepath{stroke}%
\end{pgfscope}%
\begin{pgfscope}%
\pgfsetbuttcap%
\pgfsetroundjoin%
\definecolor{currentfill}{rgb}{0.000000,0.000000,0.000000}%
\pgfsetfillcolor{currentfill}%
\pgfsetlinewidth{0.501875pt}%
\definecolor{currentstroke}{rgb}{0.000000,0.000000,0.000000}%
\pgfsetstrokecolor{currentstroke}%
\pgfsetdash{}{0pt}%
\pgfsys@defobject{currentmarker}{\pgfqpoint{0.000000in}{0.000000in}}{\pgfqpoint{0.000000in}{0.020833in}}{%
\pgfpathmoveto{\pgfqpoint{0.000000in}{0.000000in}}%
\pgfpathlineto{\pgfqpoint{0.000000in}{0.020833in}}%
\pgfusepath{stroke,fill}%
}%
\begin{pgfscope}%
\pgfsys@transformshift{5.758651in}{1.080890in}%
\pgfsys@useobject{currentmarker}{}%
\end{pgfscope}%
\end{pgfscope}%
\begin{pgfscope}%
\pgfsetbuttcap%
\pgfsetroundjoin%
\definecolor{currentfill}{rgb}{0.000000,0.000000,0.000000}%
\pgfsetfillcolor{currentfill}%
\pgfsetlinewidth{0.501875pt}%
\definecolor{currentstroke}{rgb}{0.000000,0.000000,0.000000}%
\pgfsetstrokecolor{currentstroke}%
\pgfsetdash{}{0pt}%
\pgfsys@defobject{currentmarker}{\pgfqpoint{0.000000in}{-0.020833in}}{\pgfqpoint{0.000000in}{0.000000in}}{%
\pgfpathmoveto{\pgfqpoint{0.000000in}{0.000000in}}%
\pgfpathlineto{\pgfqpoint{0.000000in}{-0.020833in}}%
\pgfusepath{stroke,fill}%
}%
\begin{pgfscope}%
\pgfsys@transformshift{5.758651in}{3.227753in}%
\pgfsys@useobject{currentmarker}{}%
\end{pgfscope}%
\end{pgfscope}%
\begin{pgfscope}%
\pgfpathrectangle{\pgfqpoint{0.481681in}{1.080890in}}{\pgfqpoint{5.785672in}{2.146863in}}%
\pgfusepath{clip}%
\pgfsetrectcap%
\pgfsetroundjoin%
\pgfsetlinewidth{0.100375pt}%
\definecolor{currentstroke}{rgb}{0.827451,0.827451,0.827451}%
\pgfsetstrokecolor{currentstroke}%
\pgfsetdash{}{0pt}%
\pgfpathmoveto{\pgfqpoint{5.794349in}{1.080890in}}%
\pgfpathlineto{\pgfqpoint{5.794349in}{3.227753in}}%
\pgfusepath{stroke}%
\end{pgfscope}%
\begin{pgfscope}%
\pgfsetbuttcap%
\pgfsetroundjoin%
\definecolor{currentfill}{rgb}{0.000000,0.000000,0.000000}%
\pgfsetfillcolor{currentfill}%
\pgfsetlinewidth{0.501875pt}%
\definecolor{currentstroke}{rgb}{0.000000,0.000000,0.000000}%
\pgfsetstrokecolor{currentstroke}%
\pgfsetdash{}{0pt}%
\pgfsys@defobject{currentmarker}{\pgfqpoint{0.000000in}{0.000000in}}{\pgfqpoint{0.000000in}{0.020833in}}{%
\pgfpathmoveto{\pgfqpoint{0.000000in}{0.000000in}}%
\pgfpathlineto{\pgfqpoint{0.000000in}{0.020833in}}%
\pgfusepath{stroke,fill}%
}%
\begin{pgfscope}%
\pgfsys@transformshift{5.794349in}{1.080890in}%
\pgfsys@useobject{currentmarker}{}%
\end{pgfscope}%
\end{pgfscope}%
\begin{pgfscope}%
\pgfsetbuttcap%
\pgfsetroundjoin%
\definecolor{currentfill}{rgb}{0.000000,0.000000,0.000000}%
\pgfsetfillcolor{currentfill}%
\pgfsetlinewidth{0.501875pt}%
\definecolor{currentstroke}{rgb}{0.000000,0.000000,0.000000}%
\pgfsetstrokecolor{currentstroke}%
\pgfsetdash{}{0pt}%
\pgfsys@defobject{currentmarker}{\pgfqpoint{0.000000in}{-0.020833in}}{\pgfqpoint{0.000000in}{0.000000in}}{%
\pgfpathmoveto{\pgfqpoint{0.000000in}{0.000000in}}%
\pgfpathlineto{\pgfqpoint{0.000000in}{-0.020833in}}%
\pgfusepath{stroke,fill}%
}%
\begin{pgfscope}%
\pgfsys@transformshift{5.794349in}{3.227753in}%
\pgfsys@useobject{currentmarker}{}%
\end{pgfscope}%
\end{pgfscope}%
\begin{pgfscope}%
\pgfpathrectangle{\pgfqpoint{0.481681in}{1.080890in}}{\pgfqpoint{5.785672in}{2.146863in}}%
\pgfusepath{clip}%
\pgfsetrectcap%
\pgfsetroundjoin%
\pgfsetlinewidth{0.100375pt}%
\definecolor{currentstroke}{rgb}{0.827451,0.827451,0.827451}%
\pgfsetstrokecolor{currentstroke}%
\pgfsetdash{}{0pt}%
\pgfpathmoveto{\pgfqpoint{5.865743in}{1.080890in}}%
\pgfpathlineto{\pgfqpoint{5.865743in}{3.227753in}}%
\pgfusepath{stroke}%
\end{pgfscope}%
\begin{pgfscope}%
\pgfsetbuttcap%
\pgfsetroundjoin%
\definecolor{currentfill}{rgb}{0.000000,0.000000,0.000000}%
\pgfsetfillcolor{currentfill}%
\pgfsetlinewidth{0.501875pt}%
\definecolor{currentstroke}{rgb}{0.000000,0.000000,0.000000}%
\pgfsetstrokecolor{currentstroke}%
\pgfsetdash{}{0pt}%
\pgfsys@defobject{currentmarker}{\pgfqpoint{0.000000in}{0.000000in}}{\pgfqpoint{0.000000in}{0.020833in}}{%
\pgfpathmoveto{\pgfqpoint{0.000000in}{0.000000in}}%
\pgfpathlineto{\pgfqpoint{0.000000in}{0.020833in}}%
\pgfusepath{stroke,fill}%
}%
\begin{pgfscope}%
\pgfsys@transformshift{5.865743in}{1.080890in}%
\pgfsys@useobject{currentmarker}{}%
\end{pgfscope}%
\end{pgfscope}%
\begin{pgfscope}%
\pgfsetbuttcap%
\pgfsetroundjoin%
\definecolor{currentfill}{rgb}{0.000000,0.000000,0.000000}%
\pgfsetfillcolor{currentfill}%
\pgfsetlinewidth{0.501875pt}%
\definecolor{currentstroke}{rgb}{0.000000,0.000000,0.000000}%
\pgfsetstrokecolor{currentstroke}%
\pgfsetdash{}{0pt}%
\pgfsys@defobject{currentmarker}{\pgfqpoint{0.000000in}{-0.020833in}}{\pgfqpoint{0.000000in}{0.000000in}}{%
\pgfpathmoveto{\pgfqpoint{0.000000in}{0.000000in}}%
\pgfpathlineto{\pgfqpoint{0.000000in}{-0.020833in}}%
\pgfusepath{stroke,fill}%
}%
\begin{pgfscope}%
\pgfsys@transformshift{5.865743in}{3.227753in}%
\pgfsys@useobject{currentmarker}{}%
\end{pgfscope}%
\end{pgfscope}%
\begin{pgfscope}%
\pgfpathrectangle{\pgfqpoint{0.481681in}{1.080890in}}{\pgfqpoint{5.785672in}{2.146863in}}%
\pgfusepath{clip}%
\pgfsetrectcap%
\pgfsetroundjoin%
\pgfsetlinewidth{0.100375pt}%
\definecolor{currentstroke}{rgb}{0.827451,0.827451,0.827451}%
\pgfsetstrokecolor{currentstroke}%
\pgfsetdash{}{0pt}%
\pgfpathmoveto{\pgfqpoint{5.901441in}{1.080890in}}%
\pgfpathlineto{\pgfqpoint{5.901441in}{3.227753in}}%
\pgfusepath{stroke}%
\end{pgfscope}%
\begin{pgfscope}%
\pgfsetbuttcap%
\pgfsetroundjoin%
\definecolor{currentfill}{rgb}{0.000000,0.000000,0.000000}%
\pgfsetfillcolor{currentfill}%
\pgfsetlinewidth{0.501875pt}%
\definecolor{currentstroke}{rgb}{0.000000,0.000000,0.000000}%
\pgfsetstrokecolor{currentstroke}%
\pgfsetdash{}{0pt}%
\pgfsys@defobject{currentmarker}{\pgfqpoint{0.000000in}{0.000000in}}{\pgfqpoint{0.000000in}{0.020833in}}{%
\pgfpathmoveto{\pgfqpoint{0.000000in}{0.000000in}}%
\pgfpathlineto{\pgfqpoint{0.000000in}{0.020833in}}%
\pgfusepath{stroke,fill}%
}%
\begin{pgfscope}%
\pgfsys@transformshift{5.901441in}{1.080890in}%
\pgfsys@useobject{currentmarker}{}%
\end{pgfscope}%
\end{pgfscope}%
\begin{pgfscope}%
\pgfsetbuttcap%
\pgfsetroundjoin%
\definecolor{currentfill}{rgb}{0.000000,0.000000,0.000000}%
\pgfsetfillcolor{currentfill}%
\pgfsetlinewidth{0.501875pt}%
\definecolor{currentstroke}{rgb}{0.000000,0.000000,0.000000}%
\pgfsetstrokecolor{currentstroke}%
\pgfsetdash{}{0pt}%
\pgfsys@defobject{currentmarker}{\pgfqpoint{0.000000in}{-0.020833in}}{\pgfqpoint{0.000000in}{0.000000in}}{%
\pgfpathmoveto{\pgfqpoint{0.000000in}{0.000000in}}%
\pgfpathlineto{\pgfqpoint{0.000000in}{-0.020833in}}%
\pgfusepath{stroke,fill}%
}%
\begin{pgfscope}%
\pgfsys@transformshift{5.901441in}{3.227753in}%
\pgfsys@useobject{currentmarker}{}%
\end{pgfscope}%
\end{pgfscope}%
\begin{pgfscope}%
\pgfpathrectangle{\pgfqpoint{0.481681in}{1.080890in}}{\pgfqpoint{5.785672in}{2.146863in}}%
\pgfusepath{clip}%
\pgfsetrectcap%
\pgfsetroundjoin%
\pgfsetlinewidth{0.100375pt}%
\definecolor{currentstroke}{rgb}{0.827451,0.827451,0.827451}%
\pgfsetstrokecolor{currentstroke}%
\pgfsetdash{}{0pt}%
\pgfpathmoveto{\pgfqpoint{5.937138in}{1.080890in}}%
\pgfpathlineto{\pgfqpoint{5.937138in}{3.227753in}}%
\pgfusepath{stroke}%
\end{pgfscope}%
\begin{pgfscope}%
\pgfsetbuttcap%
\pgfsetroundjoin%
\definecolor{currentfill}{rgb}{0.000000,0.000000,0.000000}%
\pgfsetfillcolor{currentfill}%
\pgfsetlinewidth{0.501875pt}%
\definecolor{currentstroke}{rgb}{0.000000,0.000000,0.000000}%
\pgfsetstrokecolor{currentstroke}%
\pgfsetdash{}{0pt}%
\pgfsys@defobject{currentmarker}{\pgfqpoint{0.000000in}{0.000000in}}{\pgfqpoint{0.000000in}{0.020833in}}{%
\pgfpathmoveto{\pgfqpoint{0.000000in}{0.000000in}}%
\pgfpathlineto{\pgfqpoint{0.000000in}{0.020833in}}%
\pgfusepath{stroke,fill}%
}%
\begin{pgfscope}%
\pgfsys@transformshift{5.937138in}{1.080890in}%
\pgfsys@useobject{currentmarker}{}%
\end{pgfscope}%
\end{pgfscope}%
\begin{pgfscope}%
\pgfsetbuttcap%
\pgfsetroundjoin%
\definecolor{currentfill}{rgb}{0.000000,0.000000,0.000000}%
\pgfsetfillcolor{currentfill}%
\pgfsetlinewidth{0.501875pt}%
\definecolor{currentstroke}{rgb}{0.000000,0.000000,0.000000}%
\pgfsetstrokecolor{currentstroke}%
\pgfsetdash{}{0pt}%
\pgfsys@defobject{currentmarker}{\pgfqpoint{0.000000in}{-0.020833in}}{\pgfqpoint{0.000000in}{0.000000in}}{%
\pgfpathmoveto{\pgfqpoint{0.000000in}{0.000000in}}%
\pgfpathlineto{\pgfqpoint{0.000000in}{-0.020833in}}%
\pgfusepath{stroke,fill}%
}%
\begin{pgfscope}%
\pgfsys@transformshift{5.937138in}{3.227753in}%
\pgfsys@useobject{currentmarker}{}%
\end{pgfscope}%
\end{pgfscope}%
\begin{pgfscope}%
\pgfpathrectangle{\pgfqpoint{0.481681in}{1.080890in}}{\pgfqpoint{5.785672in}{2.146863in}}%
\pgfusepath{clip}%
\pgfsetrectcap%
\pgfsetroundjoin%
\pgfsetlinewidth{0.100375pt}%
\definecolor{currentstroke}{rgb}{0.827451,0.827451,0.827451}%
\pgfsetstrokecolor{currentstroke}%
\pgfsetdash{}{0pt}%
\pgfpathmoveto{\pgfqpoint{5.972835in}{1.080890in}}%
\pgfpathlineto{\pgfqpoint{5.972835in}{3.227753in}}%
\pgfusepath{stroke}%
\end{pgfscope}%
\begin{pgfscope}%
\pgfsetbuttcap%
\pgfsetroundjoin%
\definecolor{currentfill}{rgb}{0.000000,0.000000,0.000000}%
\pgfsetfillcolor{currentfill}%
\pgfsetlinewidth{0.501875pt}%
\definecolor{currentstroke}{rgb}{0.000000,0.000000,0.000000}%
\pgfsetstrokecolor{currentstroke}%
\pgfsetdash{}{0pt}%
\pgfsys@defobject{currentmarker}{\pgfqpoint{0.000000in}{0.000000in}}{\pgfqpoint{0.000000in}{0.020833in}}{%
\pgfpathmoveto{\pgfqpoint{0.000000in}{0.000000in}}%
\pgfpathlineto{\pgfqpoint{0.000000in}{0.020833in}}%
\pgfusepath{stroke,fill}%
}%
\begin{pgfscope}%
\pgfsys@transformshift{5.972835in}{1.080890in}%
\pgfsys@useobject{currentmarker}{}%
\end{pgfscope}%
\end{pgfscope}%
\begin{pgfscope}%
\pgfsetbuttcap%
\pgfsetroundjoin%
\definecolor{currentfill}{rgb}{0.000000,0.000000,0.000000}%
\pgfsetfillcolor{currentfill}%
\pgfsetlinewidth{0.501875pt}%
\definecolor{currentstroke}{rgb}{0.000000,0.000000,0.000000}%
\pgfsetstrokecolor{currentstroke}%
\pgfsetdash{}{0pt}%
\pgfsys@defobject{currentmarker}{\pgfqpoint{0.000000in}{-0.020833in}}{\pgfqpoint{0.000000in}{0.000000in}}{%
\pgfpathmoveto{\pgfqpoint{0.000000in}{0.000000in}}%
\pgfpathlineto{\pgfqpoint{0.000000in}{-0.020833in}}%
\pgfusepath{stroke,fill}%
}%
\begin{pgfscope}%
\pgfsys@transformshift{5.972835in}{3.227753in}%
\pgfsys@useobject{currentmarker}{}%
\end{pgfscope}%
\end{pgfscope}%
\begin{pgfscope}%
\pgfpathrectangle{\pgfqpoint{0.481681in}{1.080890in}}{\pgfqpoint{5.785672in}{2.146863in}}%
\pgfusepath{clip}%
\pgfsetrectcap%
\pgfsetroundjoin%
\pgfsetlinewidth{0.100375pt}%
\definecolor{currentstroke}{rgb}{0.827451,0.827451,0.827451}%
\pgfsetstrokecolor{currentstroke}%
\pgfsetdash{}{0pt}%
\pgfpathmoveto{\pgfqpoint{6.008533in}{1.080890in}}%
\pgfpathlineto{\pgfqpoint{6.008533in}{3.227753in}}%
\pgfusepath{stroke}%
\end{pgfscope}%
\begin{pgfscope}%
\pgfsetbuttcap%
\pgfsetroundjoin%
\definecolor{currentfill}{rgb}{0.000000,0.000000,0.000000}%
\pgfsetfillcolor{currentfill}%
\pgfsetlinewidth{0.501875pt}%
\definecolor{currentstroke}{rgb}{0.000000,0.000000,0.000000}%
\pgfsetstrokecolor{currentstroke}%
\pgfsetdash{}{0pt}%
\pgfsys@defobject{currentmarker}{\pgfqpoint{0.000000in}{0.000000in}}{\pgfqpoint{0.000000in}{0.020833in}}{%
\pgfpathmoveto{\pgfqpoint{0.000000in}{0.000000in}}%
\pgfpathlineto{\pgfqpoint{0.000000in}{0.020833in}}%
\pgfusepath{stroke,fill}%
}%
\begin{pgfscope}%
\pgfsys@transformshift{6.008533in}{1.080890in}%
\pgfsys@useobject{currentmarker}{}%
\end{pgfscope}%
\end{pgfscope}%
\begin{pgfscope}%
\pgfsetbuttcap%
\pgfsetroundjoin%
\definecolor{currentfill}{rgb}{0.000000,0.000000,0.000000}%
\pgfsetfillcolor{currentfill}%
\pgfsetlinewidth{0.501875pt}%
\definecolor{currentstroke}{rgb}{0.000000,0.000000,0.000000}%
\pgfsetstrokecolor{currentstroke}%
\pgfsetdash{}{0pt}%
\pgfsys@defobject{currentmarker}{\pgfqpoint{0.000000in}{-0.020833in}}{\pgfqpoint{0.000000in}{0.000000in}}{%
\pgfpathmoveto{\pgfqpoint{0.000000in}{0.000000in}}%
\pgfpathlineto{\pgfqpoint{0.000000in}{-0.020833in}}%
\pgfusepath{stroke,fill}%
}%
\begin{pgfscope}%
\pgfsys@transformshift{6.008533in}{3.227753in}%
\pgfsys@useobject{currentmarker}{}%
\end{pgfscope}%
\end{pgfscope}%
\begin{pgfscope}%
\pgfpathrectangle{\pgfqpoint{0.481681in}{1.080890in}}{\pgfqpoint{5.785672in}{2.146863in}}%
\pgfusepath{clip}%
\pgfsetrectcap%
\pgfsetroundjoin%
\pgfsetlinewidth{0.100375pt}%
\definecolor{currentstroke}{rgb}{0.827451,0.827451,0.827451}%
\pgfsetstrokecolor{currentstroke}%
\pgfsetdash{}{0pt}%
\pgfpathmoveto{\pgfqpoint{6.044230in}{1.080890in}}%
\pgfpathlineto{\pgfqpoint{6.044230in}{3.227753in}}%
\pgfusepath{stroke}%
\end{pgfscope}%
\begin{pgfscope}%
\pgfsetbuttcap%
\pgfsetroundjoin%
\definecolor{currentfill}{rgb}{0.000000,0.000000,0.000000}%
\pgfsetfillcolor{currentfill}%
\pgfsetlinewidth{0.501875pt}%
\definecolor{currentstroke}{rgb}{0.000000,0.000000,0.000000}%
\pgfsetstrokecolor{currentstroke}%
\pgfsetdash{}{0pt}%
\pgfsys@defobject{currentmarker}{\pgfqpoint{0.000000in}{0.000000in}}{\pgfqpoint{0.000000in}{0.020833in}}{%
\pgfpathmoveto{\pgfqpoint{0.000000in}{0.000000in}}%
\pgfpathlineto{\pgfqpoint{0.000000in}{0.020833in}}%
\pgfusepath{stroke,fill}%
}%
\begin{pgfscope}%
\pgfsys@transformshift{6.044230in}{1.080890in}%
\pgfsys@useobject{currentmarker}{}%
\end{pgfscope}%
\end{pgfscope}%
\begin{pgfscope}%
\pgfsetbuttcap%
\pgfsetroundjoin%
\definecolor{currentfill}{rgb}{0.000000,0.000000,0.000000}%
\pgfsetfillcolor{currentfill}%
\pgfsetlinewidth{0.501875pt}%
\definecolor{currentstroke}{rgb}{0.000000,0.000000,0.000000}%
\pgfsetstrokecolor{currentstroke}%
\pgfsetdash{}{0pt}%
\pgfsys@defobject{currentmarker}{\pgfqpoint{0.000000in}{-0.020833in}}{\pgfqpoint{0.000000in}{0.000000in}}{%
\pgfpathmoveto{\pgfqpoint{0.000000in}{0.000000in}}%
\pgfpathlineto{\pgfqpoint{0.000000in}{-0.020833in}}%
\pgfusepath{stroke,fill}%
}%
\begin{pgfscope}%
\pgfsys@transformshift{6.044230in}{3.227753in}%
\pgfsys@useobject{currentmarker}{}%
\end{pgfscope}%
\end{pgfscope}%
\begin{pgfscope}%
\pgfpathrectangle{\pgfqpoint{0.481681in}{1.080890in}}{\pgfqpoint{5.785672in}{2.146863in}}%
\pgfusepath{clip}%
\pgfsetrectcap%
\pgfsetroundjoin%
\pgfsetlinewidth{0.100375pt}%
\definecolor{currentstroke}{rgb}{0.827451,0.827451,0.827451}%
\pgfsetstrokecolor{currentstroke}%
\pgfsetdash{}{0pt}%
\pgfpathmoveto{\pgfqpoint{6.079927in}{1.080890in}}%
\pgfpathlineto{\pgfqpoint{6.079927in}{3.227753in}}%
\pgfusepath{stroke}%
\end{pgfscope}%
\begin{pgfscope}%
\pgfsetbuttcap%
\pgfsetroundjoin%
\definecolor{currentfill}{rgb}{0.000000,0.000000,0.000000}%
\pgfsetfillcolor{currentfill}%
\pgfsetlinewidth{0.501875pt}%
\definecolor{currentstroke}{rgb}{0.000000,0.000000,0.000000}%
\pgfsetstrokecolor{currentstroke}%
\pgfsetdash{}{0pt}%
\pgfsys@defobject{currentmarker}{\pgfqpoint{0.000000in}{0.000000in}}{\pgfqpoint{0.000000in}{0.020833in}}{%
\pgfpathmoveto{\pgfqpoint{0.000000in}{0.000000in}}%
\pgfpathlineto{\pgfqpoint{0.000000in}{0.020833in}}%
\pgfusepath{stroke,fill}%
}%
\begin{pgfscope}%
\pgfsys@transformshift{6.079927in}{1.080890in}%
\pgfsys@useobject{currentmarker}{}%
\end{pgfscope}%
\end{pgfscope}%
\begin{pgfscope}%
\pgfsetbuttcap%
\pgfsetroundjoin%
\definecolor{currentfill}{rgb}{0.000000,0.000000,0.000000}%
\pgfsetfillcolor{currentfill}%
\pgfsetlinewidth{0.501875pt}%
\definecolor{currentstroke}{rgb}{0.000000,0.000000,0.000000}%
\pgfsetstrokecolor{currentstroke}%
\pgfsetdash{}{0pt}%
\pgfsys@defobject{currentmarker}{\pgfqpoint{0.000000in}{-0.020833in}}{\pgfqpoint{0.000000in}{0.000000in}}{%
\pgfpathmoveto{\pgfqpoint{0.000000in}{0.000000in}}%
\pgfpathlineto{\pgfqpoint{0.000000in}{-0.020833in}}%
\pgfusepath{stroke,fill}%
}%
\begin{pgfscope}%
\pgfsys@transformshift{6.079927in}{3.227753in}%
\pgfsys@useobject{currentmarker}{}%
\end{pgfscope}%
\end{pgfscope}%
\begin{pgfscope}%
\pgfpathrectangle{\pgfqpoint{0.481681in}{1.080890in}}{\pgfqpoint{5.785672in}{2.146863in}}%
\pgfusepath{clip}%
\pgfsetrectcap%
\pgfsetroundjoin%
\pgfsetlinewidth{0.100375pt}%
\definecolor{currentstroke}{rgb}{0.827451,0.827451,0.827451}%
\pgfsetstrokecolor{currentstroke}%
\pgfsetdash{}{0pt}%
\pgfpathmoveto{\pgfqpoint{6.115625in}{1.080890in}}%
\pgfpathlineto{\pgfqpoint{6.115625in}{3.227753in}}%
\pgfusepath{stroke}%
\end{pgfscope}%
\begin{pgfscope}%
\pgfsetbuttcap%
\pgfsetroundjoin%
\definecolor{currentfill}{rgb}{0.000000,0.000000,0.000000}%
\pgfsetfillcolor{currentfill}%
\pgfsetlinewidth{0.501875pt}%
\definecolor{currentstroke}{rgb}{0.000000,0.000000,0.000000}%
\pgfsetstrokecolor{currentstroke}%
\pgfsetdash{}{0pt}%
\pgfsys@defobject{currentmarker}{\pgfqpoint{0.000000in}{0.000000in}}{\pgfqpoint{0.000000in}{0.020833in}}{%
\pgfpathmoveto{\pgfqpoint{0.000000in}{0.000000in}}%
\pgfpathlineto{\pgfqpoint{0.000000in}{0.020833in}}%
\pgfusepath{stroke,fill}%
}%
\begin{pgfscope}%
\pgfsys@transformshift{6.115625in}{1.080890in}%
\pgfsys@useobject{currentmarker}{}%
\end{pgfscope}%
\end{pgfscope}%
\begin{pgfscope}%
\pgfsetbuttcap%
\pgfsetroundjoin%
\definecolor{currentfill}{rgb}{0.000000,0.000000,0.000000}%
\pgfsetfillcolor{currentfill}%
\pgfsetlinewidth{0.501875pt}%
\definecolor{currentstroke}{rgb}{0.000000,0.000000,0.000000}%
\pgfsetstrokecolor{currentstroke}%
\pgfsetdash{}{0pt}%
\pgfsys@defobject{currentmarker}{\pgfqpoint{0.000000in}{-0.020833in}}{\pgfqpoint{0.000000in}{0.000000in}}{%
\pgfpathmoveto{\pgfqpoint{0.000000in}{0.000000in}}%
\pgfpathlineto{\pgfqpoint{0.000000in}{-0.020833in}}%
\pgfusepath{stroke,fill}%
}%
\begin{pgfscope}%
\pgfsys@transformshift{6.115625in}{3.227753in}%
\pgfsys@useobject{currentmarker}{}%
\end{pgfscope}%
\end{pgfscope}%
\begin{pgfscope}%
\pgfpathrectangle{\pgfqpoint{0.481681in}{1.080890in}}{\pgfqpoint{5.785672in}{2.146863in}}%
\pgfusepath{clip}%
\pgfsetrectcap%
\pgfsetroundjoin%
\pgfsetlinewidth{0.100375pt}%
\definecolor{currentstroke}{rgb}{0.827451,0.827451,0.827451}%
\pgfsetstrokecolor{currentstroke}%
\pgfsetdash{}{0pt}%
\pgfpathmoveto{\pgfqpoint{6.151322in}{1.080890in}}%
\pgfpathlineto{\pgfqpoint{6.151322in}{3.227753in}}%
\pgfusepath{stroke}%
\end{pgfscope}%
\begin{pgfscope}%
\pgfsetbuttcap%
\pgfsetroundjoin%
\definecolor{currentfill}{rgb}{0.000000,0.000000,0.000000}%
\pgfsetfillcolor{currentfill}%
\pgfsetlinewidth{0.501875pt}%
\definecolor{currentstroke}{rgb}{0.000000,0.000000,0.000000}%
\pgfsetstrokecolor{currentstroke}%
\pgfsetdash{}{0pt}%
\pgfsys@defobject{currentmarker}{\pgfqpoint{0.000000in}{0.000000in}}{\pgfqpoint{0.000000in}{0.020833in}}{%
\pgfpathmoveto{\pgfqpoint{0.000000in}{0.000000in}}%
\pgfpathlineto{\pgfqpoint{0.000000in}{0.020833in}}%
\pgfusepath{stroke,fill}%
}%
\begin{pgfscope}%
\pgfsys@transformshift{6.151322in}{1.080890in}%
\pgfsys@useobject{currentmarker}{}%
\end{pgfscope}%
\end{pgfscope}%
\begin{pgfscope}%
\pgfsetbuttcap%
\pgfsetroundjoin%
\definecolor{currentfill}{rgb}{0.000000,0.000000,0.000000}%
\pgfsetfillcolor{currentfill}%
\pgfsetlinewidth{0.501875pt}%
\definecolor{currentstroke}{rgb}{0.000000,0.000000,0.000000}%
\pgfsetstrokecolor{currentstroke}%
\pgfsetdash{}{0pt}%
\pgfsys@defobject{currentmarker}{\pgfqpoint{0.000000in}{-0.020833in}}{\pgfqpoint{0.000000in}{0.000000in}}{%
\pgfpathmoveto{\pgfqpoint{0.000000in}{0.000000in}}%
\pgfpathlineto{\pgfqpoint{0.000000in}{-0.020833in}}%
\pgfusepath{stroke,fill}%
}%
\begin{pgfscope}%
\pgfsys@transformshift{6.151322in}{3.227753in}%
\pgfsys@useobject{currentmarker}{}%
\end{pgfscope}%
\end{pgfscope}%
\begin{pgfscope}%
\pgfpathrectangle{\pgfqpoint{0.481681in}{1.080890in}}{\pgfqpoint{5.785672in}{2.146863in}}%
\pgfusepath{clip}%
\pgfsetrectcap%
\pgfsetroundjoin%
\pgfsetlinewidth{0.100375pt}%
\definecolor{currentstroke}{rgb}{0.827451,0.827451,0.827451}%
\pgfsetstrokecolor{currentstroke}%
\pgfsetdash{}{0pt}%
\pgfpathmoveto{\pgfqpoint{6.187019in}{1.080890in}}%
\pgfpathlineto{\pgfqpoint{6.187019in}{3.227753in}}%
\pgfusepath{stroke}%
\end{pgfscope}%
\begin{pgfscope}%
\pgfsetbuttcap%
\pgfsetroundjoin%
\definecolor{currentfill}{rgb}{0.000000,0.000000,0.000000}%
\pgfsetfillcolor{currentfill}%
\pgfsetlinewidth{0.501875pt}%
\definecolor{currentstroke}{rgb}{0.000000,0.000000,0.000000}%
\pgfsetstrokecolor{currentstroke}%
\pgfsetdash{}{0pt}%
\pgfsys@defobject{currentmarker}{\pgfqpoint{0.000000in}{0.000000in}}{\pgfqpoint{0.000000in}{0.020833in}}{%
\pgfpathmoveto{\pgfqpoint{0.000000in}{0.000000in}}%
\pgfpathlineto{\pgfqpoint{0.000000in}{0.020833in}}%
\pgfusepath{stroke,fill}%
}%
\begin{pgfscope}%
\pgfsys@transformshift{6.187019in}{1.080890in}%
\pgfsys@useobject{currentmarker}{}%
\end{pgfscope}%
\end{pgfscope}%
\begin{pgfscope}%
\pgfsetbuttcap%
\pgfsetroundjoin%
\definecolor{currentfill}{rgb}{0.000000,0.000000,0.000000}%
\pgfsetfillcolor{currentfill}%
\pgfsetlinewidth{0.501875pt}%
\definecolor{currentstroke}{rgb}{0.000000,0.000000,0.000000}%
\pgfsetstrokecolor{currentstroke}%
\pgfsetdash{}{0pt}%
\pgfsys@defobject{currentmarker}{\pgfqpoint{0.000000in}{-0.020833in}}{\pgfqpoint{0.000000in}{0.000000in}}{%
\pgfpathmoveto{\pgfqpoint{0.000000in}{0.000000in}}%
\pgfpathlineto{\pgfqpoint{0.000000in}{-0.020833in}}%
\pgfusepath{stroke,fill}%
}%
\begin{pgfscope}%
\pgfsys@transformshift{6.187019in}{3.227753in}%
\pgfsys@useobject{currentmarker}{}%
\end{pgfscope}%
\end{pgfscope}%
\begin{pgfscope}%
\pgfpathrectangle{\pgfqpoint{0.481681in}{1.080890in}}{\pgfqpoint{5.785672in}{2.146863in}}%
\pgfusepath{clip}%
\pgfsetrectcap%
\pgfsetroundjoin%
\pgfsetlinewidth{0.100375pt}%
\definecolor{currentstroke}{rgb}{0.827451,0.827451,0.827451}%
\pgfsetstrokecolor{currentstroke}%
\pgfsetdash{}{0pt}%
\pgfpathmoveto{\pgfqpoint{6.222717in}{1.080890in}}%
\pgfpathlineto{\pgfqpoint{6.222717in}{3.227753in}}%
\pgfusepath{stroke}%
\end{pgfscope}%
\begin{pgfscope}%
\pgfsetbuttcap%
\pgfsetroundjoin%
\definecolor{currentfill}{rgb}{0.000000,0.000000,0.000000}%
\pgfsetfillcolor{currentfill}%
\pgfsetlinewidth{0.501875pt}%
\definecolor{currentstroke}{rgb}{0.000000,0.000000,0.000000}%
\pgfsetstrokecolor{currentstroke}%
\pgfsetdash{}{0pt}%
\pgfsys@defobject{currentmarker}{\pgfqpoint{0.000000in}{0.000000in}}{\pgfqpoint{0.000000in}{0.020833in}}{%
\pgfpathmoveto{\pgfqpoint{0.000000in}{0.000000in}}%
\pgfpathlineto{\pgfqpoint{0.000000in}{0.020833in}}%
\pgfusepath{stroke,fill}%
}%
\begin{pgfscope}%
\pgfsys@transformshift{6.222717in}{1.080890in}%
\pgfsys@useobject{currentmarker}{}%
\end{pgfscope}%
\end{pgfscope}%
\begin{pgfscope}%
\pgfsetbuttcap%
\pgfsetroundjoin%
\definecolor{currentfill}{rgb}{0.000000,0.000000,0.000000}%
\pgfsetfillcolor{currentfill}%
\pgfsetlinewidth{0.501875pt}%
\definecolor{currentstroke}{rgb}{0.000000,0.000000,0.000000}%
\pgfsetstrokecolor{currentstroke}%
\pgfsetdash{}{0pt}%
\pgfsys@defobject{currentmarker}{\pgfqpoint{0.000000in}{-0.020833in}}{\pgfqpoint{0.000000in}{0.000000in}}{%
\pgfpathmoveto{\pgfqpoint{0.000000in}{0.000000in}}%
\pgfpathlineto{\pgfqpoint{0.000000in}{-0.020833in}}%
\pgfusepath{stroke,fill}%
}%
\begin{pgfscope}%
\pgfsys@transformshift{6.222717in}{3.227753in}%
\pgfsys@useobject{currentmarker}{}%
\end{pgfscope}%
\end{pgfscope}%
\begin{pgfscope}%
\pgfsetbuttcap%
\pgfsetroundjoin%
\definecolor{currentfill}{rgb}{0.000000,0.000000,0.000000}%
\pgfsetfillcolor{currentfill}%
\pgfsetlinewidth{0.501875pt}%
\definecolor{currentstroke}{rgb}{0.000000,0.000000,0.000000}%
\pgfsetstrokecolor{currentstroke}%
\pgfsetdash{}{0pt}%
\pgfsys@defobject{currentmarker}{\pgfqpoint{0.000000in}{0.000000in}}{\pgfqpoint{0.041667in}{0.000000in}}{%
\pgfpathmoveto{\pgfqpoint{0.000000in}{0.000000in}}%
\pgfpathlineto{\pgfqpoint{0.041667in}{0.000000in}}%
\pgfusepath{stroke,fill}%
}%
\begin{pgfscope}%
\pgfsys@transformshift{0.481681in}{1.193383in}%
\pgfsys@useobject{currentmarker}{}%
\end{pgfscope}%
\end{pgfscope}%
\begin{pgfscope}%
\pgfsetbuttcap%
\pgfsetroundjoin%
\definecolor{currentfill}{rgb}{0.000000,0.000000,0.000000}%
\pgfsetfillcolor{currentfill}%
\pgfsetlinewidth{0.501875pt}%
\definecolor{currentstroke}{rgb}{0.000000,0.000000,0.000000}%
\pgfsetstrokecolor{currentstroke}%
\pgfsetdash{}{0pt}%
\pgfsys@defobject{currentmarker}{\pgfqpoint{-0.041667in}{0.000000in}}{\pgfqpoint{-0.000000in}{0.000000in}}{%
\pgfpathmoveto{\pgfqpoint{-0.000000in}{0.000000in}}%
\pgfpathlineto{\pgfqpoint{-0.041667in}{0.000000in}}%
\pgfusepath{stroke,fill}%
}%
\begin{pgfscope}%
\pgfsys@transformshift{6.267353in}{1.193383in}%
\pgfsys@useobject{currentmarker}{}%
\end{pgfscope}%
\end{pgfscope}%
\begin{pgfscope}%
\definecolor{textcolor}{rgb}{0.000000,0.000000,0.000000}%
\pgfsetstrokecolor{textcolor}%
\pgfsetfillcolor{textcolor}%
\pgftext[x=0.204222in, y=1.159647in, left, base]{\color{textcolor}\rmfamily\fontsize{7.000000}{8.400000}\selectfont \ensuremath{-}0.2}%
\end{pgfscope}%
\begin{pgfscope}%
\pgfsetbuttcap%
\pgfsetroundjoin%
\definecolor{currentfill}{rgb}{0.000000,0.000000,0.000000}%
\pgfsetfillcolor{currentfill}%
\pgfsetlinewidth{0.501875pt}%
\definecolor{currentstroke}{rgb}{0.000000,0.000000,0.000000}%
\pgfsetstrokecolor{currentstroke}%
\pgfsetdash{}{0pt}%
\pgfsys@defobject{currentmarker}{\pgfqpoint{0.000000in}{0.000000in}}{\pgfqpoint{0.041667in}{0.000000in}}{%
\pgfpathmoveto{\pgfqpoint{0.000000in}{0.000000in}}%
\pgfpathlineto{\pgfqpoint{0.041667in}{0.000000in}}%
\pgfusepath{stroke,fill}%
}%
\begin{pgfscope}%
\pgfsys@transformshift{0.481681in}{1.516038in}%
\pgfsys@useobject{currentmarker}{}%
\end{pgfscope}%
\end{pgfscope}%
\begin{pgfscope}%
\pgfsetbuttcap%
\pgfsetroundjoin%
\definecolor{currentfill}{rgb}{0.000000,0.000000,0.000000}%
\pgfsetfillcolor{currentfill}%
\pgfsetlinewidth{0.501875pt}%
\definecolor{currentstroke}{rgb}{0.000000,0.000000,0.000000}%
\pgfsetstrokecolor{currentstroke}%
\pgfsetdash{}{0pt}%
\pgfsys@defobject{currentmarker}{\pgfqpoint{-0.041667in}{0.000000in}}{\pgfqpoint{-0.000000in}{0.000000in}}{%
\pgfpathmoveto{\pgfqpoint{-0.000000in}{0.000000in}}%
\pgfpathlineto{\pgfqpoint{-0.041667in}{0.000000in}}%
\pgfusepath{stroke,fill}%
}%
\begin{pgfscope}%
\pgfsys@transformshift{6.267353in}{1.516038in}%
\pgfsys@useobject{currentmarker}{}%
\end{pgfscope}%
\end{pgfscope}%
\begin{pgfscope}%
\definecolor{textcolor}{rgb}{0.000000,0.000000,0.000000}%
\pgfsetstrokecolor{textcolor}%
\pgfsetfillcolor{textcolor}%
\pgftext[x=0.291028in, y=1.482302in, left, base]{\color{textcolor}\rmfamily\fontsize{7.000000}{8.400000}\selectfont 0.0}%
\end{pgfscope}%
\begin{pgfscope}%
\pgfsetbuttcap%
\pgfsetroundjoin%
\definecolor{currentfill}{rgb}{0.000000,0.000000,0.000000}%
\pgfsetfillcolor{currentfill}%
\pgfsetlinewidth{0.501875pt}%
\definecolor{currentstroke}{rgb}{0.000000,0.000000,0.000000}%
\pgfsetstrokecolor{currentstroke}%
\pgfsetdash{}{0pt}%
\pgfsys@defobject{currentmarker}{\pgfqpoint{0.000000in}{0.000000in}}{\pgfqpoint{0.041667in}{0.000000in}}{%
\pgfpathmoveto{\pgfqpoint{0.000000in}{0.000000in}}%
\pgfpathlineto{\pgfqpoint{0.041667in}{0.000000in}}%
\pgfusepath{stroke,fill}%
}%
\begin{pgfscope}%
\pgfsys@transformshift{0.481681in}{1.838693in}%
\pgfsys@useobject{currentmarker}{}%
\end{pgfscope}%
\end{pgfscope}%
\begin{pgfscope}%
\pgfsetbuttcap%
\pgfsetroundjoin%
\definecolor{currentfill}{rgb}{0.000000,0.000000,0.000000}%
\pgfsetfillcolor{currentfill}%
\pgfsetlinewidth{0.501875pt}%
\definecolor{currentstroke}{rgb}{0.000000,0.000000,0.000000}%
\pgfsetstrokecolor{currentstroke}%
\pgfsetdash{}{0pt}%
\pgfsys@defobject{currentmarker}{\pgfqpoint{-0.041667in}{0.000000in}}{\pgfqpoint{-0.000000in}{0.000000in}}{%
\pgfpathmoveto{\pgfqpoint{-0.000000in}{0.000000in}}%
\pgfpathlineto{\pgfqpoint{-0.041667in}{0.000000in}}%
\pgfusepath{stroke,fill}%
}%
\begin{pgfscope}%
\pgfsys@transformshift{6.267353in}{1.838693in}%
\pgfsys@useobject{currentmarker}{}%
\end{pgfscope}%
\end{pgfscope}%
\begin{pgfscope}%
\definecolor{textcolor}{rgb}{0.000000,0.000000,0.000000}%
\pgfsetstrokecolor{textcolor}%
\pgfsetfillcolor{textcolor}%
\pgftext[x=0.291028in, y=1.804957in, left, base]{\color{textcolor}\rmfamily\fontsize{7.000000}{8.400000}\selectfont 0.2}%
\end{pgfscope}%
\begin{pgfscope}%
\pgfsetbuttcap%
\pgfsetroundjoin%
\definecolor{currentfill}{rgb}{0.000000,0.000000,0.000000}%
\pgfsetfillcolor{currentfill}%
\pgfsetlinewidth{0.501875pt}%
\definecolor{currentstroke}{rgb}{0.000000,0.000000,0.000000}%
\pgfsetstrokecolor{currentstroke}%
\pgfsetdash{}{0pt}%
\pgfsys@defobject{currentmarker}{\pgfqpoint{0.000000in}{0.000000in}}{\pgfqpoint{0.041667in}{0.000000in}}{%
\pgfpathmoveto{\pgfqpoint{0.000000in}{0.000000in}}%
\pgfpathlineto{\pgfqpoint{0.041667in}{0.000000in}}%
\pgfusepath{stroke,fill}%
}%
\begin{pgfscope}%
\pgfsys@transformshift{0.481681in}{2.161348in}%
\pgfsys@useobject{currentmarker}{}%
\end{pgfscope}%
\end{pgfscope}%
\begin{pgfscope}%
\pgfsetbuttcap%
\pgfsetroundjoin%
\definecolor{currentfill}{rgb}{0.000000,0.000000,0.000000}%
\pgfsetfillcolor{currentfill}%
\pgfsetlinewidth{0.501875pt}%
\definecolor{currentstroke}{rgb}{0.000000,0.000000,0.000000}%
\pgfsetstrokecolor{currentstroke}%
\pgfsetdash{}{0pt}%
\pgfsys@defobject{currentmarker}{\pgfqpoint{-0.041667in}{0.000000in}}{\pgfqpoint{-0.000000in}{0.000000in}}{%
\pgfpathmoveto{\pgfqpoint{-0.000000in}{0.000000in}}%
\pgfpathlineto{\pgfqpoint{-0.041667in}{0.000000in}}%
\pgfusepath{stroke,fill}%
}%
\begin{pgfscope}%
\pgfsys@transformshift{6.267353in}{2.161348in}%
\pgfsys@useobject{currentmarker}{}%
\end{pgfscope}%
\end{pgfscope}%
\begin{pgfscope}%
\definecolor{textcolor}{rgb}{0.000000,0.000000,0.000000}%
\pgfsetstrokecolor{textcolor}%
\pgfsetfillcolor{textcolor}%
\pgftext[x=0.291028in, y=2.127612in, left, base]{\color{textcolor}\rmfamily\fontsize{7.000000}{8.400000}\selectfont 0.4}%
\end{pgfscope}%
\begin{pgfscope}%
\pgfsetbuttcap%
\pgfsetroundjoin%
\definecolor{currentfill}{rgb}{0.000000,0.000000,0.000000}%
\pgfsetfillcolor{currentfill}%
\pgfsetlinewidth{0.501875pt}%
\definecolor{currentstroke}{rgb}{0.000000,0.000000,0.000000}%
\pgfsetstrokecolor{currentstroke}%
\pgfsetdash{}{0pt}%
\pgfsys@defobject{currentmarker}{\pgfqpoint{0.000000in}{0.000000in}}{\pgfqpoint{0.041667in}{0.000000in}}{%
\pgfpathmoveto{\pgfqpoint{0.000000in}{0.000000in}}%
\pgfpathlineto{\pgfqpoint{0.041667in}{0.000000in}}%
\pgfusepath{stroke,fill}%
}%
\begin{pgfscope}%
\pgfsys@transformshift{0.481681in}{2.484003in}%
\pgfsys@useobject{currentmarker}{}%
\end{pgfscope}%
\end{pgfscope}%
\begin{pgfscope}%
\pgfsetbuttcap%
\pgfsetroundjoin%
\definecolor{currentfill}{rgb}{0.000000,0.000000,0.000000}%
\pgfsetfillcolor{currentfill}%
\pgfsetlinewidth{0.501875pt}%
\definecolor{currentstroke}{rgb}{0.000000,0.000000,0.000000}%
\pgfsetstrokecolor{currentstroke}%
\pgfsetdash{}{0pt}%
\pgfsys@defobject{currentmarker}{\pgfqpoint{-0.041667in}{0.000000in}}{\pgfqpoint{-0.000000in}{0.000000in}}{%
\pgfpathmoveto{\pgfqpoint{-0.000000in}{0.000000in}}%
\pgfpathlineto{\pgfqpoint{-0.041667in}{0.000000in}}%
\pgfusepath{stroke,fill}%
}%
\begin{pgfscope}%
\pgfsys@transformshift{6.267353in}{2.484003in}%
\pgfsys@useobject{currentmarker}{}%
\end{pgfscope}%
\end{pgfscope}%
\begin{pgfscope}%
\definecolor{textcolor}{rgb}{0.000000,0.000000,0.000000}%
\pgfsetstrokecolor{textcolor}%
\pgfsetfillcolor{textcolor}%
\pgftext[x=0.291028in, y=2.450267in, left, base]{\color{textcolor}\rmfamily\fontsize{7.000000}{8.400000}\selectfont 0.6}%
\end{pgfscope}%
\begin{pgfscope}%
\pgfsetbuttcap%
\pgfsetroundjoin%
\definecolor{currentfill}{rgb}{0.000000,0.000000,0.000000}%
\pgfsetfillcolor{currentfill}%
\pgfsetlinewidth{0.501875pt}%
\definecolor{currentstroke}{rgb}{0.000000,0.000000,0.000000}%
\pgfsetstrokecolor{currentstroke}%
\pgfsetdash{}{0pt}%
\pgfsys@defobject{currentmarker}{\pgfqpoint{0.000000in}{0.000000in}}{\pgfqpoint{0.041667in}{0.000000in}}{%
\pgfpathmoveto{\pgfqpoint{0.000000in}{0.000000in}}%
\pgfpathlineto{\pgfqpoint{0.041667in}{0.000000in}}%
\pgfusepath{stroke,fill}%
}%
\begin{pgfscope}%
\pgfsys@transformshift{0.481681in}{2.806658in}%
\pgfsys@useobject{currentmarker}{}%
\end{pgfscope}%
\end{pgfscope}%
\begin{pgfscope}%
\pgfsetbuttcap%
\pgfsetroundjoin%
\definecolor{currentfill}{rgb}{0.000000,0.000000,0.000000}%
\pgfsetfillcolor{currentfill}%
\pgfsetlinewidth{0.501875pt}%
\definecolor{currentstroke}{rgb}{0.000000,0.000000,0.000000}%
\pgfsetstrokecolor{currentstroke}%
\pgfsetdash{}{0pt}%
\pgfsys@defobject{currentmarker}{\pgfqpoint{-0.041667in}{0.000000in}}{\pgfqpoint{-0.000000in}{0.000000in}}{%
\pgfpathmoveto{\pgfqpoint{-0.000000in}{0.000000in}}%
\pgfpathlineto{\pgfqpoint{-0.041667in}{0.000000in}}%
\pgfusepath{stroke,fill}%
}%
\begin{pgfscope}%
\pgfsys@transformshift{6.267353in}{2.806658in}%
\pgfsys@useobject{currentmarker}{}%
\end{pgfscope}%
\end{pgfscope}%
\begin{pgfscope}%
\definecolor{textcolor}{rgb}{0.000000,0.000000,0.000000}%
\pgfsetstrokecolor{textcolor}%
\pgfsetfillcolor{textcolor}%
\pgftext[x=0.291028in, y=2.772922in, left, base]{\color{textcolor}\rmfamily\fontsize{7.000000}{8.400000}\selectfont 0.8}%
\end{pgfscope}%
\begin{pgfscope}%
\pgfsetbuttcap%
\pgfsetroundjoin%
\definecolor{currentfill}{rgb}{0.000000,0.000000,0.000000}%
\pgfsetfillcolor{currentfill}%
\pgfsetlinewidth{0.501875pt}%
\definecolor{currentstroke}{rgb}{0.000000,0.000000,0.000000}%
\pgfsetstrokecolor{currentstroke}%
\pgfsetdash{}{0pt}%
\pgfsys@defobject{currentmarker}{\pgfqpoint{0.000000in}{0.000000in}}{\pgfqpoint{0.041667in}{0.000000in}}{%
\pgfpathmoveto{\pgfqpoint{0.000000in}{0.000000in}}%
\pgfpathlineto{\pgfqpoint{0.041667in}{0.000000in}}%
\pgfusepath{stroke,fill}%
}%
\begin{pgfscope}%
\pgfsys@transformshift{0.481681in}{3.129313in}%
\pgfsys@useobject{currentmarker}{}%
\end{pgfscope}%
\end{pgfscope}%
\begin{pgfscope}%
\pgfsetbuttcap%
\pgfsetroundjoin%
\definecolor{currentfill}{rgb}{0.000000,0.000000,0.000000}%
\pgfsetfillcolor{currentfill}%
\pgfsetlinewidth{0.501875pt}%
\definecolor{currentstroke}{rgb}{0.000000,0.000000,0.000000}%
\pgfsetstrokecolor{currentstroke}%
\pgfsetdash{}{0pt}%
\pgfsys@defobject{currentmarker}{\pgfqpoint{-0.041667in}{0.000000in}}{\pgfqpoint{-0.000000in}{0.000000in}}{%
\pgfpathmoveto{\pgfqpoint{-0.000000in}{0.000000in}}%
\pgfpathlineto{\pgfqpoint{-0.041667in}{0.000000in}}%
\pgfusepath{stroke,fill}%
}%
\begin{pgfscope}%
\pgfsys@transformshift{6.267353in}{3.129313in}%
\pgfsys@useobject{currentmarker}{}%
\end{pgfscope}%
\end{pgfscope}%
\begin{pgfscope}%
\definecolor{textcolor}{rgb}{0.000000,0.000000,0.000000}%
\pgfsetstrokecolor{textcolor}%
\pgfsetfillcolor{textcolor}%
\pgftext[x=0.291028in, y=3.095576in, left, base]{\color{textcolor}\rmfamily\fontsize{7.000000}{8.400000}\selectfont 1.0}%
\end{pgfscope}%
\begin{pgfscope}%
\pgfsetbuttcap%
\pgfsetroundjoin%
\definecolor{currentfill}{rgb}{0.000000,0.000000,0.000000}%
\pgfsetfillcolor{currentfill}%
\pgfsetlinewidth{0.501875pt}%
\definecolor{currentstroke}{rgb}{0.000000,0.000000,0.000000}%
\pgfsetstrokecolor{currentstroke}%
\pgfsetdash{}{0pt}%
\pgfsys@defobject{currentmarker}{\pgfqpoint{0.000000in}{0.000000in}}{\pgfqpoint{0.020833in}{0.000000in}}{%
\pgfpathmoveto{\pgfqpoint{0.000000in}{0.000000in}}%
\pgfpathlineto{\pgfqpoint{0.020833in}{0.000000in}}%
\pgfusepath{stroke,fill}%
}%
\begin{pgfscope}%
\pgfsys@transformshift{0.481681in}{1.112720in}%
\pgfsys@useobject{currentmarker}{}%
\end{pgfscope}%
\end{pgfscope}%
\begin{pgfscope}%
\pgfsetbuttcap%
\pgfsetroundjoin%
\definecolor{currentfill}{rgb}{0.000000,0.000000,0.000000}%
\pgfsetfillcolor{currentfill}%
\pgfsetlinewidth{0.501875pt}%
\definecolor{currentstroke}{rgb}{0.000000,0.000000,0.000000}%
\pgfsetstrokecolor{currentstroke}%
\pgfsetdash{}{0pt}%
\pgfsys@defobject{currentmarker}{\pgfqpoint{-0.020833in}{0.000000in}}{\pgfqpoint{-0.000000in}{0.000000in}}{%
\pgfpathmoveto{\pgfqpoint{-0.000000in}{0.000000in}}%
\pgfpathlineto{\pgfqpoint{-0.020833in}{0.000000in}}%
\pgfusepath{stroke,fill}%
}%
\begin{pgfscope}%
\pgfsys@transformshift{6.267353in}{1.112720in}%
\pgfsys@useobject{currentmarker}{}%
\end{pgfscope}%
\end{pgfscope}%
\begin{pgfscope}%
\pgfsetbuttcap%
\pgfsetroundjoin%
\definecolor{currentfill}{rgb}{0.000000,0.000000,0.000000}%
\pgfsetfillcolor{currentfill}%
\pgfsetlinewidth{0.501875pt}%
\definecolor{currentstroke}{rgb}{0.000000,0.000000,0.000000}%
\pgfsetstrokecolor{currentstroke}%
\pgfsetdash{}{0pt}%
\pgfsys@defobject{currentmarker}{\pgfqpoint{0.000000in}{0.000000in}}{\pgfqpoint{0.020833in}{0.000000in}}{%
\pgfpathmoveto{\pgfqpoint{0.000000in}{0.000000in}}%
\pgfpathlineto{\pgfqpoint{0.020833in}{0.000000in}}%
\pgfusepath{stroke,fill}%
}%
\begin{pgfscope}%
\pgfsys@transformshift{0.481681in}{1.274047in}%
\pgfsys@useobject{currentmarker}{}%
\end{pgfscope}%
\end{pgfscope}%
\begin{pgfscope}%
\pgfsetbuttcap%
\pgfsetroundjoin%
\definecolor{currentfill}{rgb}{0.000000,0.000000,0.000000}%
\pgfsetfillcolor{currentfill}%
\pgfsetlinewidth{0.501875pt}%
\definecolor{currentstroke}{rgb}{0.000000,0.000000,0.000000}%
\pgfsetstrokecolor{currentstroke}%
\pgfsetdash{}{0pt}%
\pgfsys@defobject{currentmarker}{\pgfqpoint{-0.020833in}{0.000000in}}{\pgfqpoint{-0.000000in}{0.000000in}}{%
\pgfpathmoveto{\pgfqpoint{-0.000000in}{0.000000in}}%
\pgfpathlineto{\pgfqpoint{-0.020833in}{0.000000in}}%
\pgfusepath{stroke,fill}%
}%
\begin{pgfscope}%
\pgfsys@transformshift{6.267353in}{1.274047in}%
\pgfsys@useobject{currentmarker}{}%
\end{pgfscope}%
\end{pgfscope}%
\begin{pgfscope}%
\pgfsetbuttcap%
\pgfsetroundjoin%
\definecolor{currentfill}{rgb}{0.000000,0.000000,0.000000}%
\pgfsetfillcolor{currentfill}%
\pgfsetlinewidth{0.501875pt}%
\definecolor{currentstroke}{rgb}{0.000000,0.000000,0.000000}%
\pgfsetstrokecolor{currentstroke}%
\pgfsetdash{}{0pt}%
\pgfsys@defobject{currentmarker}{\pgfqpoint{0.000000in}{0.000000in}}{\pgfqpoint{0.020833in}{0.000000in}}{%
\pgfpathmoveto{\pgfqpoint{0.000000in}{0.000000in}}%
\pgfpathlineto{\pgfqpoint{0.020833in}{0.000000in}}%
\pgfusepath{stroke,fill}%
}%
\begin{pgfscope}%
\pgfsys@transformshift{0.481681in}{1.354711in}%
\pgfsys@useobject{currentmarker}{}%
\end{pgfscope}%
\end{pgfscope}%
\begin{pgfscope}%
\pgfsetbuttcap%
\pgfsetroundjoin%
\definecolor{currentfill}{rgb}{0.000000,0.000000,0.000000}%
\pgfsetfillcolor{currentfill}%
\pgfsetlinewidth{0.501875pt}%
\definecolor{currentstroke}{rgb}{0.000000,0.000000,0.000000}%
\pgfsetstrokecolor{currentstroke}%
\pgfsetdash{}{0pt}%
\pgfsys@defobject{currentmarker}{\pgfqpoint{-0.020833in}{0.000000in}}{\pgfqpoint{-0.000000in}{0.000000in}}{%
\pgfpathmoveto{\pgfqpoint{-0.000000in}{0.000000in}}%
\pgfpathlineto{\pgfqpoint{-0.020833in}{0.000000in}}%
\pgfusepath{stroke,fill}%
}%
\begin{pgfscope}%
\pgfsys@transformshift{6.267353in}{1.354711in}%
\pgfsys@useobject{currentmarker}{}%
\end{pgfscope}%
\end{pgfscope}%
\begin{pgfscope}%
\pgfsetbuttcap%
\pgfsetroundjoin%
\definecolor{currentfill}{rgb}{0.000000,0.000000,0.000000}%
\pgfsetfillcolor{currentfill}%
\pgfsetlinewidth{0.501875pt}%
\definecolor{currentstroke}{rgb}{0.000000,0.000000,0.000000}%
\pgfsetstrokecolor{currentstroke}%
\pgfsetdash{}{0pt}%
\pgfsys@defobject{currentmarker}{\pgfqpoint{0.000000in}{0.000000in}}{\pgfqpoint{0.020833in}{0.000000in}}{%
\pgfpathmoveto{\pgfqpoint{0.000000in}{0.000000in}}%
\pgfpathlineto{\pgfqpoint{0.020833in}{0.000000in}}%
\pgfusepath{stroke,fill}%
}%
\begin{pgfscope}%
\pgfsys@transformshift{0.481681in}{1.435375in}%
\pgfsys@useobject{currentmarker}{}%
\end{pgfscope}%
\end{pgfscope}%
\begin{pgfscope}%
\pgfsetbuttcap%
\pgfsetroundjoin%
\definecolor{currentfill}{rgb}{0.000000,0.000000,0.000000}%
\pgfsetfillcolor{currentfill}%
\pgfsetlinewidth{0.501875pt}%
\definecolor{currentstroke}{rgb}{0.000000,0.000000,0.000000}%
\pgfsetstrokecolor{currentstroke}%
\pgfsetdash{}{0pt}%
\pgfsys@defobject{currentmarker}{\pgfqpoint{-0.020833in}{0.000000in}}{\pgfqpoint{-0.000000in}{0.000000in}}{%
\pgfpathmoveto{\pgfqpoint{-0.000000in}{0.000000in}}%
\pgfpathlineto{\pgfqpoint{-0.020833in}{0.000000in}}%
\pgfusepath{stroke,fill}%
}%
\begin{pgfscope}%
\pgfsys@transformshift{6.267353in}{1.435375in}%
\pgfsys@useobject{currentmarker}{}%
\end{pgfscope}%
\end{pgfscope}%
\begin{pgfscope}%
\pgfsetbuttcap%
\pgfsetroundjoin%
\definecolor{currentfill}{rgb}{0.000000,0.000000,0.000000}%
\pgfsetfillcolor{currentfill}%
\pgfsetlinewidth{0.501875pt}%
\definecolor{currentstroke}{rgb}{0.000000,0.000000,0.000000}%
\pgfsetstrokecolor{currentstroke}%
\pgfsetdash{}{0pt}%
\pgfsys@defobject{currentmarker}{\pgfqpoint{0.000000in}{0.000000in}}{\pgfqpoint{0.020833in}{0.000000in}}{%
\pgfpathmoveto{\pgfqpoint{0.000000in}{0.000000in}}%
\pgfpathlineto{\pgfqpoint{0.020833in}{0.000000in}}%
\pgfusepath{stroke,fill}%
}%
\begin{pgfscope}%
\pgfsys@transformshift{0.481681in}{1.596702in}%
\pgfsys@useobject{currentmarker}{}%
\end{pgfscope}%
\end{pgfscope}%
\begin{pgfscope}%
\pgfsetbuttcap%
\pgfsetroundjoin%
\definecolor{currentfill}{rgb}{0.000000,0.000000,0.000000}%
\pgfsetfillcolor{currentfill}%
\pgfsetlinewidth{0.501875pt}%
\definecolor{currentstroke}{rgb}{0.000000,0.000000,0.000000}%
\pgfsetstrokecolor{currentstroke}%
\pgfsetdash{}{0pt}%
\pgfsys@defobject{currentmarker}{\pgfqpoint{-0.020833in}{0.000000in}}{\pgfqpoint{-0.000000in}{0.000000in}}{%
\pgfpathmoveto{\pgfqpoint{-0.000000in}{0.000000in}}%
\pgfpathlineto{\pgfqpoint{-0.020833in}{0.000000in}}%
\pgfusepath{stroke,fill}%
}%
\begin{pgfscope}%
\pgfsys@transformshift{6.267353in}{1.596702in}%
\pgfsys@useobject{currentmarker}{}%
\end{pgfscope}%
\end{pgfscope}%
\begin{pgfscope}%
\pgfsetbuttcap%
\pgfsetroundjoin%
\definecolor{currentfill}{rgb}{0.000000,0.000000,0.000000}%
\pgfsetfillcolor{currentfill}%
\pgfsetlinewidth{0.501875pt}%
\definecolor{currentstroke}{rgb}{0.000000,0.000000,0.000000}%
\pgfsetstrokecolor{currentstroke}%
\pgfsetdash{}{0pt}%
\pgfsys@defobject{currentmarker}{\pgfqpoint{0.000000in}{0.000000in}}{\pgfqpoint{0.020833in}{0.000000in}}{%
\pgfpathmoveto{\pgfqpoint{0.000000in}{0.000000in}}%
\pgfpathlineto{\pgfqpoint{0.020833in}{0.000000in}}%
\pgfusepath{stroke,fill}%
}%
\begin{pgfscope}%
\pgfsys@transformshift{0.481681in}{1.677366in}%
\pgfsys@useobject{currentmarker}{}%
\end{pgfscope}%
\end{pgfscope}%
\begin{pgfscope}%
\pgfsetbuttcap%
\pgfsetroundjoin%
\definecolor{currentfill}{rgb}{0.000000,0.000000,0.000000}%
\pgfsetfillcolor{currentfill}%
\pgfsetlinewidth{0.501875pt}%
\definecolor{currentstroke}{rgb}{0.000000,0.000000,0.000000}%
\pgfsetstrokecolor{currentstroke}%
\pgfsetdash{}{0pt}%
\pgfsys@defobject{currentmarker}{\pgfqpoint{-0.020833in}{0.000000in}}{\pgfqpoint{-0.000000in}{0.000000in}}{%
\pgfpathmoveto{\pgfqpoint{-0.000000in}{0.000000in}}%
\pgfpathlineto{\pgfqpoint{-0.020833in}{0.000000in}}%
\pgfusepath{stroke,fill}%
}%
\begin{pgfscope}%
\pgfsys@transformshift{6.267353in}{1.677366in}%
\pgfsys@useobject{currentmarker}{}%
\end{pgfscope}%
\end{pgfscope}%
\begin{pgfscope}%
\pgfsetbuttcap%
\pgfsetroundjoin%
\definecolor{currentfill}{rgb}{0.000000,0.000000,0.000000}%
\pgfsetfillcolor{currentfill}%
\pgfsetlinewidth{0.501875pt}%
\definecolor{currentstroke}{rgb}{0.000000,0.000000,0.000000}%
\pgfsetstrokecolor{currentstroke}%
\pgfsetdash{}{0pt}%
\pgfsys@defobject{currentmarker}{\pgfqpoint{0.000000in}{0.000000in}}{\pgfqpoint{0.020833in}{0.000000in}}{%
\pgfpathmoveto{\pgfqpoint{0.000000in}{0.000000in}}%
\pgfpathlineto{\pgfqpoint{0.020833in}{0.000000in}}%
\pgfusepath{stroke,fill}%
}%
\begin{pgfscope}%
\pgfsys@transformshift{0.481681in}{1.758029in}%
\pgfsys@useobject{currentmarker}{}%
\end{pgfscope}%
\end{pgfscope}%
\begin{pgfscope}%
\pgfsetbuttcap%
\pgfsetroundjoin%
\definecolor{currentfill}{rgb}{0.000000,0.000000,0.000000}%
\pgfsetfillcolor{currentfill}%
\pgfsetlinewidth{0.501875pt}%
\definecolor{currentstroke}{rgb}{0.000000,0.000000,0.000000}%
\pgfsetstrokecolor{currentstroke}%
\pgfsetdash{}{0pt}%
\pgfsys@defobject{currentmarker}{\pgfqpoint{-0.020833in}{0.000000in}}{\pgfqpoint{-0.000000in}{0.000000in}}{%
\pgfpathmoveto{\pgfqpoint{-0.000000in}{0.000000in}}%
\pgfpathlineto{\pgfqpoint{-0.020833in}{0.000000in}}%
\pgfusepath{stroke,fill}%
}%
\begin{pgfscope}%
\pgfsys@transformshift{6.267353in}{1.758029in}%
\pgfsys@useobject{currentmarker}{}%
\end{pgfscope}%
\end{pgfscope}%
\begin{pgfscope}%
\pgfsetbuttcap%
\pgfsetroundjoin%
\definecolor{currentfill}{rgb}{0.000000,0.000000,0.000000}%
\pgfsetfillcolor{currentfill}%
\pgfsetlinewidth{0.501875pt}%
\definecolor{currentstroke}{rgb}{0.000000,0.000000,0.000000}%
\pgfsetstrokecolor{currentstroke}%
\pgfsetdash{}{0pt}%
\pgfsys@defobject{currentmarker}{\pgfqpoint{0.000000in}{0.000000in}}{\pgfqpoint{0.020833in}{0.000000in}}{%
\pgfpathmoveto{\pgfqpoint{0.000000in}{0.000000in}}%
\pgfpathlineto{\pgfqpoint{0.020833in}{0.000000in}}%
\pgfusepath{stroke,fill}%
}%
\begin{pgfscope}%
\pgfsys@transformshift{0.481681in}{1.919357in}%
\pgfsys@useobject{currentmarker}{}%
\end{pgfscope}%
\end{pgfscope}%
\begin{pgfscope}%
\pgfsetbuttcap%
\pgfsetroundjoin%
\definecolor{currentfill}{rgb}{0.000000,0.000000,0.000000}%
\pgfsetfillcolor{currentfill}%
\pgfsetlinewidth{0.501875pt}%
\definecolor{currentstroke}{rgb}{0.000000,0.000000,0.000000}%
\pgfsetstrokecolor{currentstroke}%
\pgfsetdash{}{0pt}%
\pgfsys@defobject{currentmarker}{\pgfqpoint{-0.020833in}{0.000000in}}{\pgfqpoint{-0.000000in}{0.000000in}}{%
\pgfpathmoveto{\pgfqpoint{-0.000000in}{0.000000in}}%
\pgfpathlineto{\pgfqpoint{-0.020833in}{0.000000in}}%
\pgfusepath{stroke,fill}%
}%
\begin{pgfscope}%
\pgfsys@transformshift{6.267353in}{1.919357in}%
\pgfsys@useobject{currentmarker}{}%
\end{pgfscope}%
\end{pgfscope}%
\begin{pgfscope}%
\pgfsetbuttcap%
\pgfsetroundjoin%
\definecolor{currentfill}{rgb}{0.000000,0.000000,0.000000}%
\pgfsetfillcolor{currentfill}%
\pgfsetlinewidth{0.501875pt}%
\definecolor{currentstroke}{rgb}{0.000000,0.000000,0.000000}%
\pgfsetstrokecolor{currentstroke}%
\pgfsetdash{}{0pt}%
\pgfsys@defobject{currentmarker}{\pgfqpoint{0.000000in}{0.000000in}}{\pgfqpoint{0.020833in}{0.000000in}}{%
\pgfpathmoveto{\pgfqpoint{0.000000in}{0.000000in}}%
\pgfpathlineto{\pgfqpoint{0.020833in}{0.000000in}}%
\pgfusepath{stroke,fill}%
}%
\begin{pgfscope}%
\pgfsys@transformshift{0.481681in}{2.000021in}%
\pgfsys@useobject{currentmarker}{}%
\end{pgfscope}%
\end{pgfscope}%
\begin{pgfscope}%
\pgfsetbuttcap%
\pgfsetroundjoin%
\definecolor{currentfill}{rgb}{0.000000,0.000000,0.000000}%
\pgfsetfillcolor{currentfill}%
\pgfsetlinewidth{0.501875pt}%
\definecolor{currentstroke}{rgb}{0.000000,0.000000,0.000000}%
\pgfsetstrokecolor{currentstroke}%
\pgfsetdash{}{0pt}%
\pgfsys@defobject{currentmarker}{\pgfqpoint{-0.020833in}{0.000000in}}{\pgfqpoint{-0.000000in}{0.000000in}}{%
\pgfpathmoveto{\pgfqpoint{-0.000000in}{0.000000in}}%
\pgfpathlineto{\pgfqpoint{-0.020833in}{0.000000in}}%
\pgfusepath{stroke,fill}%
}%
\begin{pgfscope}%
\pgfsys@transformshift{6.267353in}{2.000021in}%
\pgfsys@useobject{currentmarker}{}%
\end{pgfscope}%
\end{pgfscope}%
\begin{pgfscope}%
\pgfsetbuttcap%
\pgfsetroundjoin%
\definecolor{currentfill}{rgb}{0.000000,0.000000,0.000000}%
\pgfsetfillcolor{currentfill}%
\pgfsetlinewidth{0.501875pt}%
\definecolor{currentstroke}{rgb}{0.000000,0.000000,0.000000}%
\pgfsetstrokecolor{currentstroke}%
\pgfsetdash{}{0pt}%
\pgfsys@defobject{currentmarker}{\pgfqpoint{0.000000in}{0.000000in}}{\pgfqpoint{0.020833in}{0.000000in}}{%
\pgfpathmoveto{\pgfqpoint{0.000000in}{0.000000in}}%
\pgfpathlineto{\pgfqpoint{0.020833in}{0.000000in}}%
\pgfusepath{stroke,fill}%
}%
\begin{pgfscope}%
\pgfsys@transformshift{0.481681in}{2.080684in}%
\pgfsys@useobject{currentmarker}{}%
\end{pgfscope}%
\end{pgfscope}%
\begin{pgfscope}%
\pgfsetbuttcap%
\pgfsetroundjoin%
\definecolor{currentfill}{rgb}{0.000000,0.000000,0.000000}%
\pgfsetfillcolor{currentfill}%
\pgfsetlinewidth{0.501875pt}%
\definecolor{currentstroke}{rgb}{0.000000,0.000000,0.000000}%
\pgfsetstrokecolor{currentstroke}%
\pgfsetdash{}{0pt}%
\pgfsys@defobject{currentmarker}{\pgfqpoint{-0.020833in}{0.000000in}}{\pgfqpoint{-0.000000in}{0.000000in}}{%
\pgfpathmoveto{\pgfqpoint{-0.000000in}{0.000000in}}%
\pgfpathlineto{\pgfqpoint{-0.020833in}{0.000000in}}%
\pgfusepath{stroke,fill}%
}%
\begin{pgfscope}%
\pgfsys@transformshift{6.267353in}{2.080684in}%
\pgfsys@useobject{currentmarker}{}%
\end{pgfscope}%
\end{pgfscope}%
\begin{pgfscope}%
\pgfsetbuttcap%
\pgfsetroundjoin%
\definecolor{currentfill}{rgb}{0.000000,0.000000,0.000000}%
\pgfsetfillcolor{currentfill}%
\pgfsetlinewidth{0.501875pt}%
\definecolor{currentstroke}{rgb}{0.000000,0.000000,0.000000}%
\pgfsetstrokecolor{currentstroke}%
\pgfsetdash{}{0pt}%
\pgfsys@defobject{currentmarker}{\pgfqpoint{0.000000in}{0.000000in}}{\pgfqpoint{0.020833in}{0.000000in}}{%
\pgfpathmoveto{\pgfqpoint{0.000000in}{0.000000in}}%
\pgfpathlineto{\pgfqpoint{0.020833in}{0.000000in}}%
\pgfusepath{stroke,fill}%
}%
\begin{pgfscope}%
\pgfsys@transformshift{0.481681in}{2.242012in}%
\pgfsys@useobject{currentmarker}{}%
\end{pgfscope}%
\end{pgfscope}%
\begin{pgfscope}%
\pgfsetbuttcap%
\pgfsetroundjoin%
\definecolor{currentfill}{rgb}{0.000000,0.000000,0.000000}%
\pgfsetfillcolor{currentfill}%
\pgfsetlinewidth{0.501875pt}%
\definecolor{currentstroke}{rgb}{0.000000,0.000000,0.000000}%
\pgfsetstrokecolor{currentstroke}%
\pgfsetdash{}{0pt}%
\pgfsys@defobject{currentmarker}{\pgfqpoint{-0.020833in}{0.000000in}}{\pgfqpoint{-0.000000in}{0.000000in}}{%
\pgfpathmoveto{\pgfqpoint{-0.000000in}{0.000000in}}%
\pgfpathlineto{\pgfqpoint{-0.020833in}{0.000000in}}%
\pgfusepath{stroke,fill}%
}%
\begin{pgfscope}%
\pgfsys@transformshift{6.267353in}{2.242012in}%
\pgfsys@useobject{currentmarker}{}%
\end{pgfscope}%
\end{pgfscope}%
\begin{pgfscope}%
\pgfsetbuttcap%
\pgfsetroundjoin%
\definecolor{currentfill}{rgb}{0.000000,0.000000,0.000000}%
\pgfsetfillcolor{currentfill}%
\pgfsetlinewidth{0.501875pt}%
\definecolor{currentstroke}{rgb}{0.000000,0.000000,0.000000}%
\pgfsetstrokecolor{currentstroke}%
\pgfsetdash{}{0pt}%
\pgfsys@defobject{currentmarker}{\pgfqpoint{0.000000in}{0.000000in}}{\pgfqpoint{0.020833in}{0.000000in}}{%
\pgfpathmoveto{\pgfqpoint{0.000000in}{0.000000in}}%
\pgfpathlineto{\pgfqpoint{0.020833in}{0.000000in}}%
\pgfusepath{stroke,fill}%
}%
\begin{pgfscope}%
\pgfsys@transformshift{0.481681in}{2.322675in}%
\pgfsys@useobject{currentmarker}{}%
\end{pgfscope}%
\end{pgfscope}%
\begin{pgfscope}%
\pgfsetbuttcap%
\pgfsetroundjoin%
\definecolor{currentfill}{rgb}{0.000000,0.000000,0.000000}%
\pgfsetfillcolor{currentfill}%
\pgfsetlinewidth{0.501875pt}%
\definecolor{currentstroke}{rgb}{0.000000,0.000000,0.000000}%
\pgfsetstrokecolor{currentstroke}%
\pgfsetdash{}{0pt}%
\pgfsys@defobject{currentmarker}{\pgfqpoint{-0.020833in}{0.000000in}}{\pgfqpoint{-0.000000in}{0.000000in}}{%
\pgfpathmoveto{\pgfqpoint{-0.000000in}{0.000000in}}%
\pgfpathlineto{\pgfqpoint{-0.020833in}{0.000000in}}%
\pgfusepath{stroke,fill}%
}%
\begin{pgfscope}%
\pgfsys@transformshift{6.267353in}{2.322675in}%
\pgfsys@useobject{currentmarker}{}%
\end{pgfscope}%
\end{pgfscope}%
\begin{pgfscope}%
\pgfsetbuttcap%
\pgfsetroundjoin%
\definecolor{currentfill}{rgb}{0.000000,0.000000,0.000000}%
\pgfsetfillcolor{currentfill}%
\pgfsetlinewidth{0.501875pt}%
\definecolor{currentstroke}{rgb}{0.000000,0.000000,0.000000}%
\pgfsetstrokecolor{currentstroke}%
\pgfsetdash{}{0pt}%
\pgfsys@defobject{currentmarker}{\pgfqpoint{0.000000in}{0.000000in}}{\pgfqpoint{0.020833in}{0.000000in}}{%
\pgfpathmoveto{\pgfqpoint{0.000000in}{0.000000in}}%
\pgfpathlineto{\pgfqpoint{0.020833in}{0.000000in}}%
\pgfusepath{stroke,fill}%
}%
\begin{pgfscope}%
\pgfsys@transformshift{0.481681in}{2.403339in}%
\pgfsys@useobject{currentmarker}{}%
\end{pgfscope}%
\end{pgfscope}%
\begin{pgfscope}%
\pgfsetbuttcap%
\pgfsetroundjoin%
\definecolor{currentfill}{rgb}{0.000000,0.000000,0.000000}%
\pgfsetfillcolor{currentfill}%
\pgfsetlinewidth{0.501875pt}%
\definecolor{currentstroke}{rgb}{0.000000,0.000000,0.000000}%
\pgfsetstrokecolor{currentstroke}%
\pgfsetdash{}{0pt}%
\pgfsys@defobject{currentmarker}{\pgfqpoint{-0.020833in}{0.000000in}}{\pgfqpoint{-0.000000in}{0.000000in}}{%
\pgfpathmoveto{\pgfqpoint{-0.000000in}{0.000000in}}%
\pgfpathlineto{\pgfqpoint{-0.020833in}{0.000000in}}%
\pgfusepath{stroke,fill}%
}%
\begin{pgfscope}%
\pgfsys@transformshift{6.267353in}{2.403339in}%
\pgfsys@useobject{currentmarker}{}%
\end{pgfscope}%
\end{pgfscope}%
\begin{pgfscope}%
\pgfsetbuttcap%
\pgfsetroundjoin%
\definecolor{currentfill}{rgb}{0.000000,0.000000,0.000000}%
\pgfsetfillcolor{currentfill}%
\pgfsetlinewidth{0.501875pt}%
\definecolor{currentstroke}{rgb}{0.000000,0.000000,0.000000}%
\pgfsetstrokecolor{currentstroke}%
\pgfsetdash{}{0pt}%
\pgfsys@defobject{currentmarker}{\pgfqpoint{0.000000in}{0.000000in}}{\pgfqpoint{0.020833in}{0.000000in}}{%
\pgfpathmoveto{\pgfqpoint{0.000000in}{0.000000in}}%
\pgfpathlineto{\pgfqpoint{0.020833in}{0.000000in}}%
\pgfusepath{stroke,fill}%
}%
\begin{pgfscope}%
\pgfsys@transformshift{0.481681in}{2.564667in}%
\pgfsys@useobject{currentmarker}{}%
\end{pgfscope}%
\end{pgfscope}%
\begin{pgfscope}%
\pgfsetbuttcap%
\pgfsetroundjoin%
\definecolor{currentfill}{rgb}{0.000000,0.000000,0.000000}%
\pgfsetfillcolor{currentfill}%
\pgfsetlinewidth{0.501875pt}%
\definecolor{currentstroke}{rgb}{0.000000,0.000000,0.000000}%
\pgfsetstrokecolor{currentstroke}%
\pgfsetdash{}{0pt}%
\pgfsys@defobject{currentmarker}{\pgfqpoint{-0.020833in}{0.000000in}}{\pgfqpoint{-0.000000in}{0.000000in}}{%
\pgfpathmoveto{\pgfqpoint{-0.000000in}{0.000000in}}%
\pgfpathlineto{\pgfqpoint{-0.020833in}{0.000000in}}%
\pgfusepath{stroke,fill}%
}%
\begin{pgfscope}%
\pgfsys@transformshift{6.267353in}{2.564667in}%
\pgfsys@useobject{currentmarker}{}%
\end{pgfscope}%
\end{pgfscope}%
\begin{pgfscope}%
\pgfsetbuttcap%
\pgfsetroundjoin%
\definecolor{currentfill}{rgb}{0.000000,0.000000,0.000000}%
\pgfsetfillcolor{currentfill}%
\pgfsetlinewidth{0.501875pt}%
\definecolor{currentstroke}{rgb}{0.000000,0.000000,0.000000}%
\pgfsetstrokecolor{currentstroke}%
\pgfsetdash{}{0pt}%
\pgfsys@defobject{currentmarker}{\pgfqpoint{0.000000in}{0.000000in}}{\pgfqpoint{0.020833in}{0.000000in}}{%
\pgfpathmoveto{\pgfqpoint{0.000000in}{0.000000in}}%
\pgfpathlineto{\pgfqpoint{0.020833in}{0.000000in}}%
\pgfusepath{stroke,fill}%
}%
\begin{pgfscope}%
\pgfsys@transformshift{0.481681in}{2.645330in}%
\pgfsys@useobject{currentmarker}{}%
\end{pgfscope}%
\end{pgfscope}%
\begin{pgfscope}%
\pgfsetbuttcap%
\pgfsetroundjoin%
\definecolor{currentfill}{rgb}{0.000000,0.000000,0.000000}%
\pgfsetfillcolor{currentfill}%
\pgfsetlinewidth{0.501875pt}%
\definecolor{currentstroke}{rgb}{0.000000,0.000000,0.000000}%
\pgfsetstrokecolor{currentstroke}%
\pgfsetdash{}{0pt}%
\pgfsys@defobject{currentmarker}{\pgfqpoint{-0.020833in}{0.000000in}}{\pgfqpoint{-0.000000in}{0.000000in}}{%
\pgfpathmoveto{\pgfqpoint{-0.000000in}{0.000000in}}%
\pgfpathlineto{\pgfqpoint{-0.020833in}{0.000000in}}%
\pgfusepath{stroke,fill}%
}%
\begin{pgfscope}%
\pgfsys@transformshift{6.267353in}{2.645330in}%
\pgfsys@useobject{currentmarker}{}%
\end{pgfscope}%
\end{pgfscope}%
\begin{pgfscope}%
\pgfsetbuttcap%
\pgfsetroundjoin%
\definecolor{currentfill}{rgb}{0.000000,0.000000,0.000000}%
\pgfsetfillcolor{currentfill}%
\pgfsetlinewidth{0.501875pt}%
\definecolor{currentstroke}{rgb}{0.000000,0.000000,0.000000}%
\pgfsetstrokecolor{currentstroke}%
\pgfsetdash{}{0pt}%
\pgfsys@defobject{currentmarker}{\pgfqpoint{0.000000in}{0.000000in}}{\pgfqpoint{0.020833in}{0.000000in}}{%
\pgfpathmoveto{\pgfqpoint{0.000000in}{0.000000in}}%
\pgfpathlineto{\pgfqpoint{0.020833in}{0.000000in}}%
\pgfusepath{stroke,fill}%
}%
\begin{pgfscope}%
\pgfsys@transformshift{0.481681in}{2.725994in}%
\pgfsys@useobject{currentmarker}{}%
\end{pgfscope}%
\end{pgfscope}%
\begin{pgfscope}%
\pgfsetbuttcap%
\pgfsetroundjoin%
\definecolor{currentfill}{rgb}{0.000000,0.000000,0.000000}%
\pgfsetfillcolor{currentfill}%
\pgfsetlinewidth{0.501875pt}%
\definecolor{currentstroke}{rgb}{0.000000,0.000000,0.000000}%
\pgfsetstrokecolor{currentstroke}%
\pgfsetdash{}{0pt}%
\pgfsys@defobject{currentmarker}{\pgfqpoint{-0.020833in}{0.000000in}}{\pgfqpoint{-0.000000in}{0.000000in}}{%
\pgfpathmoveto{\pgfqpoint{-0.000000in}{0.000000in}}%
\pgfpathlineto{\pgfqpoint{-0.020833in}{0.000000in}}%
\pgfusepath{stroke,fill}%
}%
\begin{pgfscope}%
\pgfsys@transformshift{6.267353in}{2.725994in}%
\pgfsys@useobject{currentmarker}{}%
\end{pgfscope}%
\end{pgfscope}%
\begin{pgfscope}%
\pgfsetbuttcap%
\pgfsetroundjoin%
\definecolor{currentfill}{rgb}{0.000000,0.000000,0.000000}%
\pgfsetfillcolor{currentfill}%
\pgfsetlinewidth{0.501875pt}%
\definecolor{currentstroke}{rgb}{0.000000,0.000000,0.000000}%
\pgfsetstrokecolor{currentstroke}%
\pgfsetdash{}{0pt}%
\pgfsys@defobject{currentmarker}{\pgfqpoint{0.000000in}{0.000000in}}{\pgfqpoint{0.020833in}{0.000000in}}{%
\pgfpathmoveto{\pgfqpoint{0.000000in}{0.000000in}}%
\pgfpathlineto{\pgfqpoint{0.020833in}{0.000000in}}%
\pgfusepath{stroke,fill}%
}%
\begin{pgfscope}%
\pgfsys@transformshift{0.481681in}{2.887321in}%
\pgfsys@useobject{currentmarker}{}%
\end{pgfscope}%
\end{pgfscope}%
\begin{pgfscope}%
\pgfsetbuttcap%
\pgfsetroundjoin%
\definecolor{currentfill}{rgb}{0.000000,0.000000,0.000000}%
\pgfsetfillcolor{currentfill}%
\pgfsetlinewidth{0.501875pt}%
\definecolor{currentstroke}{rgb}{0.000000,0.000000,0.000000}%
\pgfsetstrokecolor{currentstroke}%
\pgfsetdash{}{0pt}%
\pgfsys@defobject{currentmarker}{\pgfqpoint{-0.020833in}{0.000000in}}{\pgfqpoint{-0.000000in}{0.000000in}}{%
\pgfpathmoveto{\pgfqpoint{-0.000000in}{0.000000in}}%
\pgfpathlineto{\pgfqpoint{-0.020833in}{0.000000in}}%
\pgfusepath{stroke,fill}%
}%
\begin{pgfscope}%
\pgfsys@transformshift{6.267353in}{2.887321in}%
\pgfsys@useobject{currentmarker}{}%
\end{pgfscope}%
\end{pgfscope}%
\begin{pgfscope}%
\pgfsetbuttcap%
\pgfsetroundjoin%
\definecolor{currentfill}{rgb}{0.000000,0.000000,0.000000}%
\pgfsetfillcolor{currentfill}%
\pgfsetlinewidth{0.501875pt}%
\definecolor{currentstroke}{rgb}{0.000000,0.000000,0.000000}%
\pgfsetstrokecolor{currentstroke}%
\pgfsetdash{}{0pt}%
\pgfsys@defobject{currentmarker}{\pgfqpoint{0.000000in}{0.000000in}}{\pgfqpoint{0.020833in}{0.000000in}}{%
\pgfpathmoveto{\pgfqpoint{0.000000in}{0.000000in}}%
\pgfpathlineto{\pgfqpoint{0.020833in}{0.000000in}}%
\pgfusepath{stroke,fill}%
}%
\begin{pgfscope}%
\pgfsys@transformshift{0.481681in}{2.967985in}%
\pgfsys@useobject{currentmarker}{}%
\end{pgfscope}%
\end{pgfscope}%
\begin{pgfscope}%
\pgfsetbuttcap%
\pgfsetroundjoin%
\definecolor{currentfill}{rgb}{0.000000,0.000000,0.000000}%
\pgfsetfillcolor{currentfill}%
\pgfsetlinewidth{0.501875pt}%
\definecolor{currentstroke}{rgb}{0.000000,0.000000,0.000000}%
\pgfsetstrokecolor{currentstroke}%
\pgfsetdash{}{0pt}%
\pgfsys@defobject{currentmarker}{\pgfqpoint{-0.020833in}{0.000000in}}{\pgfqpoint{-0.000000in}{0.000000in}}{%
\pgfpathmoveto{\pgfqpoint{-0.000000in}{0.000000in}}%
\pgfpathlineto{\pgfqpoint{-0.020833in}{0.000000in}}%
\pgfusepath{stroke,fill}%
}%
\begin{pgfscope}%
\pgfsys@transformshift{6.267353in}{2.967985in}%
\pgfsys@useobject{currentmarker}{}%
\end{pgfscope}%
\end{pgfscope}%
\begin{pgfscope}%
\pgfsetbuttcap%
\pgfsetroundjoin%
\definecolor{currentfill}{rgb}{0.000000,0.000000,0.000000}%
\pgfsetfillcolor{currentfill}%
\pgfsetlinewidth{0.501875pt}%
\definecolor{currentstroke}{rgb}{0.000000,0.000000,0.000000}%
\pgfsetstrokecolor{currentstroke}%
\pgfsetdash{}{0pt}%
\pgfsys@defobject{currentmarker}{\pgfqpoint{0.000000in}{0.000000in}}{\pgfqpoint{0.020833in}{0.000000in}}{%
\pgfpathmoveto{\pgfqpoint{0.000000in}{0.000000in}}%
\pgfpathlineto{\pgfqpoint{0.020833in}{0.000000in}}%
\pgfusepath{stroke,fill}%
}%
\begin{pgfscope}%
\pgfsys@transformshift{0.481681in}{3.048649in}%
\pgfsys@useobject{currentmarker}{}%
\end{pgfscope}%
\end{pgfscope}%
\begin{pgfscope}%
\pgfsetbuttcap%
\pgfsetroundjoin%
\definecolor{currentfill}{rgb}{0.000000,0.000000,0.000000}%
\pgfsetfillcolor{currentfill}%
\pgfsetlinewidth{0.501875pt}%
\definecolor{currentstroke}{rgb}{0.000000,0.000000,0.000000}%
\pgfsetstrokecolor{currentstroke}%
\pgfsetdash{}{0pt}%
\pgfsys@defobject{currentmarker}{\pgfqpoint{-0.020833in}{0.000000in}}{\pgfqpoint{-0.000000in}{0.000000in}}{%
\pgfpathmoveto{\pgfqpoint{-0.000000in}{0.000000in}}%
\pgfpathlineto{\pgfqpoint{-0.020833in}{0.000000in}}%
\pgfusepath{stroke,fill}%
}%
\begin{pgfscope}%
\pgfsys@transformshift{6.267353in}{3.048649in}%
\pgfsys@useobject{currentmarker}{}%
\end{pgfscope}%
\end{pgfscope}%
\begin{pgfscope}%
\pgfsetbuttcap%
\pgfsetroundjoin%
\definecolor{currentfill}{rgb}{0.000000,0.000000,0.000000}%
\pgfsetfillcolor{currentfill}%
\pgfsetlinewidth{0.501875pt}%
\definecolor{currentstroke}{rgb}{0.000000,0.000000,0.000000}%
\pgfsetstrokecolor{currentstroke}%
\pgfsetdash{}{0pt}%
\pgfsys@defobject{currentmarker}{\pgfqpoint{0.000000in}{0.000000in}}{\pgfqpoint{0.020833in}{0.000000in}}{%
\pgfpathmoveto{\pgfqpoint{0.000000in}{0.000000in}}%
\pgfpathlineto{\pgfqpoint{0.020833in}{0.000000in}}%
\pgfusepath{stroke,fill}%
}%
\begin{pgfscope}%
\pgfsys@transformshift{0.481681in}{3.209976in}%
\pgfsys@useobject{currentmarker}{}%
\end{pgfscope}%
\end{pgfscope}%
\begin{pgfscope}%
\pgfsetbuttcap%
\pgfsetroundjoin%
\definecolor{currentfill}{rgb}{0.000000,0.000000,0.000000}%
\pgfsetfillcolor{currentfill}%
\pgfsetlinewidth{0.501875pt}%
\definecolor{currentstroke}{rgb}{0.000000,0.000000,0.000000}%
\pgfsetstrokecolor{currentstroke}%
\pgfsetdash{}{0pt}%
\pgfsys@defobject{currentmarker}{\pgfqpoint{-0.020833in}{0.000000in}}{\pgfqpoint{-0.000000in}{0.000000in}}{%
\pgfpathmoveto{\pgfqpoint{-0.000000in}{0.000000in}}%
\pgfpathlineto{\pgfqpoint{-0.020833in}{0.000000in}}%
\pgfusepath{stroke,fill}%
}%
\begin{pgfscope}%
\pgfsys@transformshift{6.267353in}{3.209976in}%
\pgfsys@useobject{currentmarker}{}%
\end{pgfscope}%
\end{pgfscope}%
\begin{pgfscope}%
\definecolor{textcolor}{rgb}{0.000000,0.000000,0.000000}%
\pgfsetstrokecolor{textcolor}%
\pgfsetfillcolor{textcolor}%
\pgftext[x=0.148667in,y=2.154322in,,bottom,rotate=90.000000]{\color{textcolor}\rmfamily\fontsize{8.000000}{9.600000}\selectfont Wert}%
\end{pgfscope}%
\begin{pgfscope}%
\pgfpathrectangle{\pgfqpoint{0.481681in}{1.080890in}}{\pgfqpoint{5.785672in}{2.146863in}}%
\pgfusepath{clip}%
\pgfsetrectcap%
\pgfsetroundjoin%
\pgfsetlinewidth{0.401500pt}%
\definecolor{currentstroke}{rgb}{0.000000,0.070588,0.098039}%
\pgfsetstrokecolor{currentstroke}%
\pgfsetdash{}{0pt}%
\pgfpathmoveto{\pgfqpoint{0.744666in}{1.516038in}}%
\pgfpathlineto{\pgfqpoint{0.758350in}{1.550752in}}%
\pgfpathlineto{\pgfqpoint{0.801187in}{1.565975in}}%
\pgfpathlineto{\pgfqpoint{0.803269in}{1.569565in}}%
\pgfpathlineto{\pgfqpoint{0.815168in}{1.590017in}}%
\pgfpathlineto{\pgfqpoint{0.893107in}{1.594464in}}%
\pgfpathlineto{\pgfqpoint{0.895190in}{1.598762in}}%
\pgfpathlineto{\pgfqpoint{1.036491in}{1.603416in}}%
\pgfpathlineto{\pgfqpoint{1.066239in}{1.609326in}}%
\pgfpathlineto{\pgfqpoint{1.094797in}{1.625024in}}%
\pgfpathlineto{\pgfqpoint{1.261682in}{1.638594in}}%
\pgfpathlineto{\pgfqpoint{1.334862in}{1.651482in}}%
\pgfpathlineto{\pgfqpoint{1.384838in}{1.686524in}}%
\pgfpathlineto{\pgfqpoint{1.408636in}{1.721184in}}%
\pgfpathlineto{\pgfqpoint{1.415478in}{1.721493in}}%
\pgfpathlineto{\pgfqpoint{1.916430in}{1.743443in}}%
\pgfpathlineto{\pgfqpoint{2.056245in}{1.764872in}}%
\pgfpathlineto{\pgfqpoint{2.056542in}{1.765237in}}%
\pgfpathlineto{\pgfqpoint{2.058327in}{1.786854in}}%
\pgfpathlineto{\pgfqpoint{2.356995in}{1.830483in}}%
\pgfpathlineto{\pgfqpoint{2.357590in}{1.852640in}}%
\pgfpathlineto{\pgfqpoint{2.384363in}{1.874003in}}%
\pgfpathlineto{\pgfqpoint{2.384660in}{1.874664in}}%
\pgfpathlineto{\pgfqpoint{2.386445in}{1.894779in}}%
\pgfpathlineto{\pgfqpoint{2.399831in}{1.922433in}}%
\pgfpathlineto{\pgfqpoint{2.478663in}{1.944304in}}%
\pgfpathlineto{\pgfqpoint{2.535779in}{1.948786in}}%
\pgfpathlineto{\pgfqpoint{2.559577in}{1.963073in}}%
\pgfpathlineto{\pgfqpoint{2.584268in}{1.981384in}}%
\pgfpathlineto{\pgfqpoint{2.585458in}{1.986637in}}%
\pgfpathlineto{\pgfqpoint{2.593489in}{2.023028in}}%
\pgfpathlineto{\pgfqpoint{2.691062in}{2.057850in}}%
\pgfpathlineto{\pgfqpoint{2.691360in}{2.060236in}}%
\pgfpathlineto{\pgfqpoint{2.691955in}{2.070891in}}%
\pgfpathlineto{\pgfqpoint{2.692847in}{2.071187in}}%
\pgfpathlineto{\pgfqpoint{2.833851in}{2.092939in}}%
\pgfpathlineto{\pgfqpoint{2.836231in}{2.096468in}}%
\pgfpathlineto{\pgfqpoint{2.848725in}{2.114923in}}%
\pgfpathlineto{\pgfqpoint{2.901974in}{2.137032in}}%
\pgfpathlineto{\pgfqpoint{2.991217in}{2.158466in}}%
\pgfpathlineto{\pgfqpoint{2.991812in}{2.160633in}}%
\pgfpathlineto{\pgfqpoint{2.998654in}{2.191413in}}%
\pgfpathlineto{\pgfqpoint{3.047440in}{2.214070in}}%
\pgfpathlineto{\pgfqpoint{3.084922in}{2.233274in}}%
\pgfpathlineto{\pgfqpoint{3.143526in}{2.291607in}}%
\pgfpathlineto{\pgfqpoint{3.351165in}{2.465667in}}%
\pgfpathlineto{\pgfqpoint{3.585875in}{2.614322in}}%
\pgfpathlineto{\pgfqpoint{3.690587in}{2.665581in}}%
\pgfpathlineto{\pgfqpoint{3.690884in}{2.677101in}}%
\pgfpathlineto{\pgfqpoint{3.691182in}{2.736220in}}%
\pgfpathlineto{\pgfqpoint{3.704568in}{2.769858in}}%
\pgfpathlineto{\pgfqpoint{3.704866in}{2.781950in}}%
\pgfpathlineto{\pgfqpoint{3.705163in}{2.806004in}}%
\pgfpathlineto{\pgfqpoint{3.733424in}{2.840916in}}%
\pgfpathlineto{\pgfqpoint{3.741753in}{2.870694in}}%
\pgfpathlineto{\pgfqpoint{3.747703in}{2.891989in}}%
\pgfpathlineto{\pgfqpoint{3.748595in}{2.945776in}}%
\pgfpathlineto{\pgfqpoint{3.899714in}{2.981250in}}%
\pgfpathlineto{\pgfqpoint{4.040421in}{3.037315in}}%
\pgfpathlineto{\pgfqpoint{4.040718in}{3.039620in}}%
\pgfpathlineto{\pgfqpoint{4.041016in}{3.059355in}}%
\pgfpathlineto{\pgfqpoint{4.284650in}{3.072718in}}%
\pgfpathlineto{\pgfqpoint{4.348607in}{3.107456in}}%
\pgfpathlineto{\pgfqpoint{6.004368in}{3.129313in}}%
\pgfpathlineto{\pgfqpoint{6.004368in}{3.129313in}}%
\pgfusepath{stroke}%
\end{pgfscope}%
\begin{pgfscope}%
\pgfpathrectangle{\pgfqpoint{0.481681in}{1.080890in}}{\pgfqpoint{5.785672in}{2.146863in}}%
\pgfusepath{clip}%
\pgfsetrectcap%
\pgfsetroundjoin%
\pgfsetlinewidth{0.200750pt}%
\definecolor{currentstroke}{rgb}{0.682353,0.125490,0.070588}%
\pgfsetstrokecolor{currentstroke}%
\pgfsetdash{}{0pt}%
\pgfpathmoveto{\pgfqpoint{0.744666in}{1.516038in}}%
\pgfpathlineto{\pgfqpoint{0.751805in}{1.519049in}}%
\pgfpathlineto{\pgfqpoint{0.757160in}{1.523290in}}%
\pgfpathlineto{\pgfqpoint{0.757755in}{1.524118in}}%
\pgfpathlineto{\pgfqpoint{0.761325in}{1.524385in}}%
\pgfpathlineto{\pgfqpoint{0.784230in}{1.526275in}}%
\pgfpathlineto{\pgfqpoint{0.788395in}{1.521875in}}%
\pgfpathlineto{\pgfqpoint{0.791965in}{1.521875in}}%
\pgfpathlineto{\pgfqpoint{0.792560in}{1.522731in}}%
\pgfpathlineto{\pgfqpoint{0.794047in}{1.525249in}}%
\pgfpathlineto{\pgfqpoint{0.794345in}{1.524126in}}%
\pgfpathlineto{\pgfqpoint{0.794940in}{1.526360in}}%
\pgfpathlineto{\pgfqpoint{0.813681in}{1.526270in}}%
\pgfpathlineto{\pgfqpoint{0.814573in}{1.524873in}}%
\pgfpathlineto{\pgfqpoint{0.824687in}{1.524865in}}%
\pgfpathlineto{\pgfqpoint{0.841346in}{1.525138in}}%
\pgfpathlineto{\pgfqpoint{0.852948in}{1.525497in}}%
\pgfpathlineto{\pgfqpoint{0.859195in}{1.524865in}}%
\pgfpathlineto{\pgfqpoint{0.894000in}{1.524784in}}%
\pgfpathlineto{\pgfqpoint{0.894297in}{1.527537in}}%
\pgfpathlineto{\pgfqpoint{0.894595in}{1.526360in}}%
\pgfpathlineto{\pgfqpoint{0.949331in}{1.526360in}}%
\pgfpathlineto{\pgfqpoint{0.949628in}{1.527056in}}%
\pgfpathlineto{\pgfqpoint{0.950223in}{1.535453in}}%
\pgfpathlineto{\pgfqpoint{0.950818in}{1.536078in}}%
\pgfpathlineto{\pgfqpoint{0.963015in}{1.536089in}}%
\pgfpathlineto{\pgfqpoint{0.963609in}{1.536990in}}%
\pgfpathlineto{\pgfqpoint{0.964204in}{1.538320in}}%
\pgfpathlineto{\pgfqpoint{0.964502in}{1.538320in}}%
\pgfpathlineto{\pgfqpoint{0.964799in}{1.534886in}}%
\pgfpathlineto{\pgfqpoint{0.965097in}{1.466668in}}%
\pgfpathlineto{\pgfqpoint{0.965394in}{1.465065in}}%
\pgfpathlineto{\pgfqpoint{1.005851in}{1.465065in}}%
\pgfpathlineto{\pgfqpoint{1.006149in}{1.464745in}}%
\pgfpathlineto{\pgfqpoint{1.007339in}{1.447412in}}%
\pgfpathlineto{\pgfqpoint{1.014181in}{1.344750in}}%
\pgfpathlineto{\pgfqpoint{1.023105in}{1.348342in}}%
\pgfpathlineto{\pgfqpoint{1.043631in}{1.356628in}}%
\pgfpathlineto{\pgfqpoint{1.043928in}{1.358122in}}%
\pgfpathlineto{\pgfqpoint{1.045118in}{1.374619in}}%
\pgfpathlineto{\pgfqpoint{1.049878in}{1.374662in}}%
\pgfpathlineto{\pgfqpoint{1.050473in}{1.376861in}}%
\pgfpathlineto{\pgfqpoint{1.093012in}{1.377136in}}%
\pgfpathlineto{\pgfqpoint{1.236991in}{1.383942in}}%
\pgfpathlineto{\pgfqpoint{1.252758in}{1.386578in}}%
\pgfpathlineto{\pgfqpoint{1.286373in}{1.386578in}}%
\pgfpathlineto{\pgfqpoint{1.286670in}{1.389849in}}%
\pgfpathlineto{\pgfqpoint{1.286968in}{1.402276in}}%
\pgfpathlineto{\pgfqpoint{1.291430in}{1.402251in}}%
\pgfpathlineto{\pgfqpoint{1.292025in}{1.369386in}}%
\pgfpathlineto{\pgfqpoint{1.292917in}{1.369386in}}%
\pgfpathlineto{\pgfqpoint{1.293215in}{1.369710in}}%
\pgfpathlineto{\pgfqpoint{1.293810in}{1.371292in}}%
\pgfpathlineto{\pgfqpoint{1.300949in}{1.377609in}}%
\pgfpathlineto{\pgfqpoint{1.341109in}{1.377609in}}%
\pgfpathlineto{\pgfqpoint{1.341704in}{1.413488in}}%
\pgfpathlineto{\pgfqpoint{1.350330in}{1.413488in}}%
\pgfpathlineto{\pgfqpoint{1.350628in}{1.414312in}}%
\pgfpathlineto{\pgfqpoint{1.350925in}{1.419398in}}%
\pgfpathlineto{\pgfqpoint{1.352115in}{1.385290in}}%
\pgfpathlineto{\pgfqpoint{1.352413in}{1.391918in}}%
\pgfpathlineto{\pgfqpoint{1.352710in}{1.419468in}}%
\pgfpathlineto{\pgfqpoint{1.362527in}{1.418721in}}%
\pgfpathlineto{\pgfqpoint{1.385730in}{1.418721in}}%
\pgfpathlineto{\pgfqpoint{1.386028in}{1.420297in}}%
\pgfpathlineto{\pgfqpoint{1.386325in}{1.435913in}}%
\pgfpathlineto{\pgfqpoint{1.398522in}{1.435907in}}%
\pgfpathlineto{\pgfqpoint{1.399117in}{1.433475in}}%
\pgfpathlineto{\pgfqpoint{1.401497in}{1.421217in}}%
\pgfpathlineto{\pgfqpoint{1.401794in}{1.433130in}}%
\pgfpathlineto{\pgfqpoint{1.402092in}{1.457730in}}%
\pgfpathlineto{\pgfqpoint{1.406554in}{1.460581in}}%
\pgfpathlineto{\pgfqpoint{1.408041in}{1.460578in}}%
\pgfpathlineto{\pgfqpoint{1.408636in}{1.458291in}}%
\pgfpathlineto{\pgfqpoint{1.412801in}{1.437726in}}%
\pgfpathlineto{\pgfqpoint{1.413098in}{1.446219in}}%
\pgfpathlineto{\pgfqpoint{1.413396in}{1.469231in}}%
\pgfpathlineto{\pgfqpoint{1.413991in}{1.443801in}}%
\pgfpathlineto{\pgfqpoint{1.414883in}{1.482834in}}%
\pgfpathlineto{\pgfqpoint{1.415181in}{1.484500in}}%
\pgfpathlineto{\pgfqpoint{1.544286in}{1.484768in}}%
\pgfpathlineto{\pgfqpoint{1.567489in}{1.487490in}}%
\pgfpathlineto{\pgfqpoint{1.843251in}{1.487490in}}%
\pgfpathlineto{\pgfqpoint{1.843846in}{1.488212in}}%
\pgfpathlineto{\pgfqpoint{1.848903in}{1.487891in}}%
\pgfpathlineto{\pgfqpoint{1.849498in}{1.488238in}}%
\pgfpathlineto{\pgfqpoint{1.863777in}{1.488130in}}%
\pgfpathlineto{\pgfqpoint{1.865264in}{1.481364in}}%
\pgfpathlineto{\pgfqpoint{1.886088in}{1.382997in}}%
\pgfpathlineto{\pgfqpoint{1.886385in}{1.412368in}}%
\pgfpathlineto{\pgfqpoint{1.886683in}{1.482152in}}%
\pgfpathlineto{\pgfqpoint{1.887278in}{1.484080in}}%
\pgfpathlineto{\pgfqpoint{1.899772in}{1.461389in}}%
\pgfpathlineto{\pgfqpoint{1.900367in}{1.468873in}}%
\pgfpathlineto{\pgfqpoint{1.943203in}{1.477025in}}%
\pgfpathlineto{\pgfqpoint{1.946476in}{1.477283in}}%
\pgfpathlineto{\pgfqpoint{1.964622in}{1.480706in}}%
\pgfpathlineto{\pgfqpoint{1.991692in}{1.478909in}}%
\pgfpathlineto{\pgfqpoint{1.992287in}{1.481632in}}%
\pgfpathlineto{\pgfqpoint{2.001509in}{1.533961in}}%
\pgfpathlineto{\pgfqpoint{2.002104in}{1.758084in}}%
\pgfpathlineto{\pgfqpoint{2.002401in}{1.757342in}}%
\pgfpathlineto{\pgfqpoint{2.002699in}{1.755841in}}%
\pgfpathlineto{\pgfqpoint{2.005376in}{1.755841in}}%
\pgfpathlineto{\pgfqpoint{2.005674in}{1.756192in}}%
\pgfpathlineto{\pgfqpoint{2.006566in}{1.761821in}}%
\pgfpathlineto{\pgfqpoint{2.045536in}{1.762092in}}%
\pgfpathlineto{\pgfqpoint{2.242168in}{1.775336in}}%
\pgfpathlineto{\pgfqpoint{2.242763in}{1.776032in}}%
\pgfpathlineto{\pgfqpoint{2.326057in}{1.777518in}}%
\pgfpathlineto{\pgfqpoint{2.391502in}{1.777518in}}%
\pgfpathlineto{\pgfqpoint{2.391800in}{1.777021in}}%
\pgfpathlineto{\pgfqpoint{2.392097in}{1.770226in}}%
\pgfpathlineto{\pgfqpoint{2.392395in}{1.771214in}}%
\pgfpathlineto{\pgfqpoint{2.399534in}{1.832387in}}%
\pgfpathlineto{\pgfqpoint{2.400426in}{1.829614in}}%
\pgfpathlineto{\pgfqpoint{2.401319in}{1.832416in}}%
\pgfpathlineto{\pgfqpoint{2.405186in}{1.839832in}}%
\pgfpathlineto{\pgfqpoint{2.405781in}{1.842551in}}%
\pgfpathlineto{\pgfqpoint{2.457245in}{1.842551in}}%
\pgfpathlineto{\pgfqpoint{2.458137in}{1.843298in}}%
\pgfpathlineto{\pgfqpoint{2.499189in}{1.843298in}}%
\pgfpathlineto{\pgfqpoint{2.499486in}{1.845971in}}%
\pgfpathlineto{\pgfqpoint{2.500974in}{1.876935in}}%
\pgfpathlineto{\pgfqpoint{2.506328in}{1.876935in}}%
\pgfpathlineto{\pgfqpoint{2.506626in}{1.876413in}}%
\pgfpathlineto{\pgfqpoint{2.507518in}{1.811659in}}%
\pgfpathlineto{\pgfqpoint{2.513170in}{1.357940in}}%
\pgfpathlineto{\pgfqpoint{2.513468in}{1.355931in}}%
\pgfpathlineto{\pgfqpoint{3.148285in}{1.356209in}}%
\pgfpathlineto{\pgfqpoint{3.261922in}{1.357426in}}%
\pgfpathlineto{\pgfqpoint{3.408578in}{1.357699in}}%
\pgfpathlineto{\pgfqpoint{3.548393in}{1.361684in}}%
\pgfpathlineto{\pgfqpoint{3.550475in}{1.357426in}}%
\pgfpathlineto{\pgfqpoint{3.557912in}{1.357700in}}%
\pgfpathlineto{\pgfqpoint{3.593312in}{1.362659in}}%
\pgfpathlineto{\pgfqpoint{3.608186in}{1.362393in}}%
\pgfpathlineto{\pgfqpoint{3.662327in}{1.357441in}}%
\pgfpathlineto{\pgfqpoint{3.662922in}{1.362376in}}%
\pgfpathlineto{\pgfqpoint{3.663516in}{1.362992in}}%
\pgfpathlineto{\pgfqpoint{3.847358in}{1.466487in}}%
\pgfpathlineto{\pgfqpoint{3.847953in}{1.465593in}}%
\pgfpathlineto{\pgfqpoint{3.886327in}{1.386348in}}%
\pgfpathlineto{\pgfqpoint{3.886625in}{1.386389in}}%
\pgfpathlineto{\pgfqpoint{3.890194in}{1.415149in}}%
\pgfpathlineto{\pgfqpoint{3.913100in}{1.599153in}}%
\pgfpathlineto{\pgfqpoint{3.916670in}{1.544555in}}%
\pgfpathlineto{\pgfqpoint{3.927082in}{1.385901in}}%
\pgfpathlineto{\pgfqpoint{3.928569in}{1.385831in}}%
\pgfpathlineto{\pgfqpoint{3.984495in}{1.385831in}}%
\pgfpathlineto{\pgfqpoint{3.984792in}{1.386701in}}%
\pgfpathlineto{\pgfqpoint{3.998179in}{1.498232in}}%
\pgfpathlineto{\pgfqpoint{4.025844in}{1.515662in}}%
\pgfpathlineto{\pgfqpoint{4.026142in}{1.514562in}}%
\pgfpathlineto{\pgfqpoint{4.027332in}{1.386381in}}%
\pgfpathlineto{\pgfqpoint{4.027629in}{1.389448in}}%
\pgfpathlineto{\pgfqpoint{4.034174in}{1.487498in}}%
\pgfpathlineto{\pgfqpoint{4.039231in}{1.589435in}}%
\pgfpathlineto{\pgfqpoint{4.039528in}{1.610517in}}%
\pgfpathlineto{\pgfqpoint{4.039826in}{1.698284in}}%
\pgfpathlineto{\pgfqpoint{4.076118in}{1.698385in}}%
\pgfpathlineto{\pgfqpoint{4.078498in}{1.704534in}}%
\pgfpathlineto{\pgfqpoint{4.080580in}{1.709376in}}%
\pgfpathlineto{\pgfqpoint{4.098429in}{1.698284in}}%
\pgfpathlineto{\pgfqpoint{4.126987in}{1.698545in}}%
\pgfpathlineto{\pgfqpoint{4.140968in}{1.713726in}}%
\pgfpathlineto{\pgfqpoint{4.141266in}{1.711564in}}%
\pgfpathlineto{\pgfqpoint{4.141860in}{1.695573in}}%
\pgfpathlineto{\pgfqpoint{4.142158in}{1.696785in}}%
\pgfpathlineto{\pgfqpoint{4.145728in}{1.734141in}}%
\pgfpathlineto{\pgfqpoint{4.147513in}{1.713691in}}%
\pgfpathlineto{\pgfqpoint{4.147810in}{1.717151in}}%
\pgfpathlineto{\pgfqpoint{4.148405in}{1.751075in}}%
\pgfpathlineto{\pgfqpoint{4.152272in}{1.744094in}}%
\pgfpathlineto{\pgfqpoint{4.152867in}{1.757336in}}%
\pgfpathlineto{\pgfqpoint{4.155842in}{1.757276in}}%
\pgfpathlineto{\pgfqpoint{4.156437in}{1.747293in}}%
\pgfpathlineto{\pgfqpoint{4.190944in}{1.833180in}}%
\pgfpathlineto{\pgfqpoint{4.191242in}{1.832703in}}%
\pgfpathlineto{\pgfqpoint{4.206413in}{1.736013in}}%
\pgfpathlineto{\pgfqpoint{4.208495in}{1.743598in}}%
\pgfpathlineto{\pgfqpoint{4.232889in}{1.833581in}}%
\pgfpathlineto{\pgfqpoint{4.233484in}{1.833420in}}%
\pgfpathlineto{\pgfqpoint{4.253415in}{1.809996in}}%
\pgfpathlineto{\pgfqpoint{4.253712in}{1.840856in}}%
\pgfpathlineto{\pgfqpoint{4.254010in}{1.943338in}}%
\pgfpathlineto{\pgfqpoint{4.255497in}{1.950446in}}%
\pgfpathlineto{\pgfqpoint{4.276320in}{2.043552in}}%
\pgfpathlineto{\pgfqpoint{4.280783in}{2.055668in}}%
\pgfpathlineto{\pgfqpoint{4.281080in}{2.054313in}}%
\pgfpathlineto{\pgfqpoint{4.283162in}{1.961327in}}%
\pgfpathlineto{\pgfqpoint{4.283460in}{1.991624in}}%
\pgfpathlineto{\pgfqpoint{4.283757in}{2.060072in}}%
\pgfpathlineto{\pgfqpoint{4.290599in}{2.060328in}}%
\pgfpathlineto{\pgfqpoint{4.305176in}{2.066799in}}%
\pgfpathlineto{\pgfqpoint{4.305473in}{2.066799in}}%
\pgfpathlineto{\pgfqpoint{4.305771in}{2.066459in}}%
\pgfpathlineto{\pgfqpoint{4.306366in}{2.065183in}}%
\pgfpathlineto{\pgfqpoint{4.326297in}{2.057947in}}%
\pgfpathlineto{\pgfqpoint{4.326594in}{2.058519in}}%
\pgfpathlineto{\pgfqpoint{4.328081in}{2.097322in}}%
\pgfpathlineto{\pgfqpoint{4.331354in}{2.180951in}}%
\pgfpathlineto{\pgfqpoint{4.331949in}{2.181166in}}%
\pgfpathlineto{\pgfqpoint{4.333436in}{2.181166in}}%
\pgfpathlineto{\pgfqpoint{4.333734in}{2.182294in}}%
\pgfpathlineto{\pgfqpoint{4.334329in}{2.186394in}}%
\pgfpathlineto{\pgfqpoint{4.353367in}{2.186398in}}%
\pgfpathlineto{\pgfqpoint{4.377165in}{2.186408in}}%
\pgfpathlineto{\pgfqpoint{4.378058in}{2.188364in}}%
\pgfpathlineto{\pgfqpoint{4.382520in}{2.199105in}}%
\pgfpathlineto{\pgfqpoint{4.383115in}{2.199342in}}%
\pgfpathlineto{\pgfqpoint{4.391147in}{2.206581in}}%
\pgfpathlineto{\pgfqpoint{4.404831in}{2.206851in}}%
\pgfpathlineto{\pgfqpoint{4.483365in}{2.210309in}}%
\pgfpathlineto{\pgfqpoint{4.483662in}{2.213002in}}%
\pgfpathlineto{\pgfqpoint{4.492289in}{2.436005in}}%
\pgfpathlineto{\pgfqpoint{4.494074in}{2.436062in}}%
\pgfpathlineto{\pgfqpoint{4.505973in}{2.436148in}}%
\pgfpathlineto{\pgfqpoint{4.542563in}{2.462320in}}%
\pgfpathlineto{\pgfqpoint{4.583912in}{2.468221in}}%
\pgfpathlineto{\pgfqpoint{4.585102in}{2.470568in}}%
\pgfpathlineto{\pgfqpoint{4.599084in}{2.474203in}}%
\pgfpathlineto{\pgfqpoint{4.635376in}{2.475111in}}%
\pgfpathlineto{\pgfqpoint{4.646978in}{2.482170in}}%
\pgfpathlineto{\pgfqpoint{4.647275in}{2.479395in}}%
\pgfpathlineto{\pgfqpoint{4.647870in}{2.232214in}}%
\pgfpathlineto{\pgfqpoint{4.648167in}{2.385533in}}%
\pgfpathlineto{\pgfqpoint{4.648762in}{2.213720in}}%
\pgfpathlineto{\pgfqpoint{4.649060in}{2.223772in}}%
\pgfpathlineto{\pgfqpoint{4.649655in}{2.491788in}}%
\pgfpathlineto{\pgfqpoint{4.655902in}{2.498840in}}%
\pgfpathlineto{\pgfqpoint{4.706176in}{2.497356in}}%
\pgfpathlineto{\pgfqpoint{4.865326in}{2.497084in}}%
\pgfpathlineto{\pgfqpoint{4.888232in}{2.496777in}}%
\pgfpathlineto{\pgfqpoint{5.041730in}{2.507426in}}%
\pgfpathlineto{\pgfqpoint{5.128594in}{2.525761in}}%
\pgfpathlineto{\pgfqpoint{5.475453in}{2.525795in}}%
\pgfpathlineto{\pgfqpoint{5.476048in}{2.531303in}}%
\pgfpathlineto{\pgfqpoint{5.477535in}{2.550896in}}%
\pgfpathlineto{\pgfqpoint{5.484080in}{2.640780in}}%
\pgfpathlineto{\pgfqpoint{5.491517in}{2.676261in}}%
\pgfpathlineto{\pgfqpoint{5.492111in}{2.671029in}}%
\pgfpathlineto{\pgfqpoint{5.504308in}{2.542919in}}%
\pgfpathlineto{\pgfqpoint{5.505200in}{2.684261in}}%
\pgfpathlineto{\pgfqpoint{5.533461in}{2.687270in}}%
\pgfpathlineto{\pgfqpoint{5.549227in}{2.695113in}}%
\pgfpathlineto{\pgfqpoint{5.598311in}{2.708040in}}%
\pgfpathlineto{\pgfqpoint{5.598608in}{2.708705in}}%
\pgfpathlineto{\pgfqpoint{5.599203in}{2.789026in}}%
\pgfpathlineto{\pgfqpoint{5.604558in}{2.764980in}}%
\pgfpathlineto{\pgfqpoint{5.604855in}{2.748761in}}%
\pgfpathlineto{\pgfqpoint{5.605450in}{2.708150in}}%
\pgfpathlineto{\pgfqpoint{5.625976in}{2.708313in}}%
\pgfpathlineto{\pgfqpoint{5.626571in}{2.709345in}}%
\pgfpathlineto{\pgfqpoint{5.626869in}{2.708740in}}%
\pgfpathlineto{\pgfqpoint{5.627761in}{2.712223in}}%
\pgfpathlineto{\pgfqpoint{5.639660in}{2.763538in}}%
\pgfpathlineto{\pgfqpoint{5.639958in}{2.771581in}}%
\pgfpathlineto{\pgfqpoint{5.640255in}{2.768982in}}%
\pgfpathlineto{\pgfqpoint{5.640850in}{2.728506in}}%
\pgfpathlineto{\pgfqpoint{5.647990in}{2.810672in}}%
\pgfpathlineto{\pgfqpoint{5.651559in}{2.817279in}}%
\pgfpathlineto{\pgfqpoint{5.655129in}{2.823264in}}%
\pgfpathlineto{\pgfqpoint{5.660781in}{2.823003in}}%
\pgfpathlineto{\pgfqpoint{5.798513in}{2.814464in}}%
\pgfpathlineto{\pgfqpoint{5.825881in}{2.823176in}}%
\pgfpathlineto{\pgfqpoint{5.826179in}{2.825207in}}%
\pgfpathlineto{\pgfqpoint{5.829154in}{2.942192in}}%
\pgfpathlineto{\pgfqpoint{5.833021in}{3.095646in}}%
\pgfpathlineto{\pgfqpoint{5.833318in}{3.092036in}}%
\pgfpathlineto{\pgfqpoint{5.836293in}{2.827002in}}%
\pgfpathlineto{\pgfqpoint{5.839863in}{2.827002in}}%
\pgfpathlineto{\pgfqpoint{5.840160in}{2.826163in}}%
\pgfpathlineto{\pgfqpoint{5.842540in}{2.781025in}}%
\pgfpathlineto{\pgfqpoint{5.847597in}{2.683832in}}%
\pgfpathlineto{\pgfqpoint{5.847895in}{2.721391in}}%
\pgfpathlineto{\pgfqpoint{5.848192in}{2.828519in}}%
\pgfpathlineto{\pgfqpoint{5.856522in}{2.829300in}}%
\pgfpathlineto{\pgfqpoint{5.875560in}{2.829989in}}%
\pgfpathlineto{\pgfqpoint{5.876155in}{2.831321in}}%
\pgfpathlineto{\pgfqpoint{5.976108in}{3.114909in}}%
\pgfpathlineto{\pgfqpoint{5.976405in}{3.114902in}}%
\pgfpathlineto{\pgfqpoint{5.977000in}{3.095794in}}%
\pgfpathlineto{\pgfqpoint{5.982950in}{2.859087in}}%
\pgfpathlineto{\pgfqpoint{5.983247in}{2.913873in}}%
\pgfpathlineto{\pgfqpoint{5.983545in}{3.128192in}}%
\pgfpathlineto{\pgfqpoint{5.989494in}{3.126803in}}%
\pgfpathlineto{\pgfqpoint{5.990089in}{3.128995in}}%
\pgfpathlineto{\pgfqpoint{6.004368in}{3.129313in}}%
\pgfpathlineto{\pgfqpoint{6.004368in}{3.129313in}}%
\pgfusepath{stroke}%
\end{pgfscope}%
\begin{pgfscope}%
\pgfpathrectangle{\pgfqpoint{0.481681in}{1.080890in}}{\pgfqpoint{5.785672in}{2.146863in}}%
\pgfusepath{clip}%
\pgfsetrectcap%
\pgfsetroundjoin%
\pgfsetlinewidth{0.200750pt}%
\definecolor{currentstroke}{rgb}{0.000000,0.372549,0.450980}%
\pgfsetstrokecolor{currentstroke}%
\pgfsetdash{}{0pt}%
\pgfpathmoveto{\pgfqpoint{0.744666in}{1.516038in}}%
\pgfpathlineto{\pgfqpoint{0.750615in}{1.516135in}}%
\pgfpathlineto{\pgfqpoint{0.751805in}{1.521361in}}%
\pgfpathlineto{\pgfqpoint{0.757160in}{1.546075in}}%
\pgfpathlineto{\pgfqpoint{0.757457in}{1.551656in}}%
\pgfpathlineto{\pgfqpoint{0.757755in}{1.572745in}}%
\pgfpathlineto{\pgfqpoint{0.758350in}{1.573360in}}%
\pgfpathlineto{\pgfqpoint{0.786015in}{1.577163in}}%
\pgfpathlineto{\pgfqpoint{0.791965in}{1.577163in}}%
\pgfpathlineto{\pgfqpoint{0.792560in}{1.578914in}}%
\pgfpathlineto{\pgfqpoint{0.794345in}{1.584804in}}%
\pgfpathlineto{\pgfqpoint{0.796427in}{1.585179in}}%
\pgfpathlineto{\pgfqpoint{0.814871in}{1.588624in}}%
\pgfpathlineto{\pgfqpoint{0.852055in}{1.588653in}}%
\pgfpathlineto{\pgfqpoint{0.852948in}{1.592158in}}%
\pgfpathlineto{\pgfqpoint{0.857410in}{1.611463in}}%
\pgfpathlineto{\pgfqpoint{0.858600in}{1.611546in}}%
\pgfpathlineto{\pgfqpoint{0.894000in}{1.611737in}}%
\pgfpathlineto{\pgfqpoint{0.894892in}{1.603906in}}%
\pgfpathlineto{\pgfqpoint{0.949331in}{1.603906in}}%
\pgfpathlineto{\pgfqpoint{0.949628in}{1.604229in}}%
\pgfpathlineto{\pgfqpoint{0.950520in}{1.611546in}}%
\pgfpathlineto{\pgfqpoint{0.963015in}{1.611585in}}%
\pgfpathlineto{\pgfqpoint{0.964204in}{1.615367in}}%
\pgfpathlineto{\pgfqpoint{0.964502in}{1.615367in}}%
\pgfpathlineto{\pgfqpoint{0.964799in}{1.613039in}}%
\pgfpathlineto{\pgfqpoint{0.965097in}{1.568149in}}%
\pgfpathlineto{\pgfqpoint{0.965692in}{1.569523in}}%
\pgfpathlineto{\pgfqpoint{1.006149in}{1.569391in}}%
\pgfpathlineto{\pgfqpoint{1.007339in}{1.562238in}}%
\pgfpathlineto{\pgfqpoint{1.014181in}{1.519878in}}%
\pgfpathlineto{\pgfqpoint{1.029352in}{1.523784in}}%
\pgfpathlineto{\pgfqpoint{1.043631in}{1.527466in}}%
\pgfpathlineto{\pgfqpoint{1.043928in}{1.527872in}}%
\pgfpathlineto{\pgfqpoint{1.045118in}{1.535140in}}%
\pgfpathlineto{\pgfqpoint{1.097474in}{1.535411in}}%
\pgfpathlineto{\pgfqpoint{1.245618in}{1.538960in}}%
\pgfpathlineto{\pgfqpoint{1.286373in}{1.538960in}}%
\pgfpathlineto{\pgfqpoint{1.286670in}{1.541348in}}%
\pgfpathlineto{\pgfqpoint{1.286968in}{1.550421in}}%
\pgfpathlineto{\pgfqpoint{1.291430in}{1.550408in}}%
\pgfpathlineto{\pgfqpoint{1.292025in}{1.531320in}}%
\pgfpathlineto{\pgfqpoint{1.341109in}{1.531320in}}%
\pgfpathlineto{\pgfqpoint{1.341704in}{1.554241in}}%
\pgfpathlineto{\pgfqpoint{1.350330in}{1.554241in}}%
\pgfpathlineto{\pgfqpoint{1.350628in}{1.555030in}}%
\pgfpathlineto{\pgfqpoint{1.350925in}{1.563226in}}%
\pgfpathlineto{\pgfqpoint{1.351818in}{1.544575in}}%
\pgfpathlineto{\pgfqpoint{1.352115in}{1.537629in}}%
\pgfpathlineto{\pgfqpoint{1.352413in}{1.543073in}}%
\pgfpathlineto{\pgfqpoint{1.352710in}{1.565702in}}%
\pgfpathlineto{\pgfqpoint{1.385730in}{1.565702in}}%
\pgfpathlineto{\pgfqpoint{1.386028in}{1.566753in}}%
\pgfpathlineto{\pgfqpoint{1.386325in}{1.577163in}}%
\pgfpathlineto{\pgfqpoint{1.398522in}{1.577159in}}%
\pgfpathlineto{\pgfqpoint{1.399117in}{1.575464in}}%
\pgfpathlineto{\pgfqpoint{1.401497in}{1.566921in}}%
\pgfpathlineto{\pgfqpoint{1.401794in}{1.579293in}}%
\pgfpathlineto{\pgfqpoint{1.402092in}{1.603906in}}%
\pgfpathlineto{\pgfqpoint{1.408041in}{1.603903in}}%
\pgfpathlineto{\pgfqpoint{1.408636in}{1.601346in}}%
\pgfpathlineto{\pgfqpoint{1.412801in}{1.578354in}}%
\pgfpathlineto{\pgfqpoint{1.413098in}{1.585894in}}%
\pgfpathlineto{\pgfqpoint{1.413396in}{1.606888in}}%
\pgfpathlineto{\pgfqpoint{1.413991in}{1.583680in}}%
\pgfpathlineto{\pgfqpoint{1.414883in}{1.618122in}}%
\pgfpathlineto{\pgfqpoint{1.415181in}{1.619187in}}%
\pgfpathlineto{\pgfqpoint{1.543691in}{1.619430in}}%
\pgfpathlineto{\pgfqpoint{1.566894in}{1.623007in}}%
\pgfpathlineto{\pgfqpoint{1.863777in}{1.622919in}}%
\pgfpathlineto{\pgfqpoint{1.865264in}{1.617357in}}%
\pgfpathlineto{\pgfqpoint{1.886088in}{1.536497in}}%
\pgfpathlineto{\pgfqpoint{1.886385in}{1.567257in}}%
\pgfpathlineto{\pgfqpoint{1.886683in}{1.639709in}}%
\pgfpathlineto{\pgfqpoint{1.887278in}{1.641970in}}%
\pgfpathlineto{\pgfqpoint{1.899772in}{1.634488in}}%
\pgfpathlineto{\pgfqpoint{1.900367in}{1.642174in}}%
\pgfpathlineto{\pgfqpoint{1.943501in}{1.649749in}}%
\pgfpathlineto{\pgfqpoint{1.946476in}{1.650013in}}%
\pgfpathlineto{\pgfqpoint{1.964622in}{1.653511in}}%
\pgfpathlineto{\pgfqpoint{1.991692in}{1.651677in}}%
\pgfpathlineto{\pgfqpoint{1.992287in}{1.655184in}}%
\pgfpathlineto{\pgfqpoint{2.001509in}{1.724056in}}%
\pgfpathlineto{\pgfqpoint{2.002104in}{1.817843in}}%
\pgfpathlineto{\pgfqpoint{2.005674in}{1.818067in}}%
\pgfpathlineto{\pgfqpoint{2.006566in}{1.821664in}}%
\pgfpathlineto{\pgfqpoint{2.046131in}{1.821929in}}%
\pgfpathlineto{\pgfqpoint{2.246928in}{1.833125in}}%
\pgfpathlineto{\pgfqpoint{2.392097in}{1.833338in}}%
\pgfpathlineto{\pgfqpoint{2.394477in}{1.845528in}}%
\pgfpathlineto{\pgfqpoint{2.399534in}{1.871299in}}%
\pgfpathlineto{\pgfqpoint{2.400129in}{1.871328in}}%
\pgfpathlineto{\pgfqpoint{2.400426in}{1.872067in}}%
\pgfpathlineto{\pgfqpoint{2.401021in}{1.875812in}}%
\pgfpathlineto{\pgfqpoint{2.405484in}{1.889191in}}%
\pgfpathlineto{\pgfqpoint{2.405781in}{1.890429in}}%
\pgfpathlineto{\pgfqpoint{2.499189in}{1.890429in}}%
\pgfpathlineto{\pgfqpoint{2.499486in}{1.892251in}}%
\pgfpathlineto{\pgfqpoint{2.500974in}{1.913351in}}%
\pgfpathlineto{\pgfqpoint{2.506626in}{1.913125in}}%
\pgfpathlineto{\pgfqpoint{2.507518in}{1.885111in}}%
\pgfpathlineto{\pgfqpoint{2.513170in}{1.688822in}}%
\pgfpathlineto{\pgfqpoint{2.513468in}{1.687953in}}%
\pgfpathlineto{\pgfqpoint{3.410066in}{1.688223in}}%
\pgfpathlineto{\pgfqpoint{3.548393in}{1.691580in}}%
\pgfpathlineto{\pgfqpoint{3.550475in}{1.687953in}}%
\pgfpathlineto{\pgfqpoint{3.558507in}{1.688216in}}%
\pgfpathlineto{\pgfqpoint{3.593907in}{1.691773in}}%
\pgfpathlineto{\pgfqpoint{3.609376in}{1.691499in}}%
\pgfpathlineto{\pgfqpoint{3.662327in}{1.687964in}}%
\pgfpathlineto{\pgfqpoint{3.662922in}{1.691559in}}%
\pgfpathlineto{\pgfqpoint{3.663516in}{1.691883in}}%
\pgfpathlineto{\pgfqpoint{3.847655in}{1.726055in}}%
\pgfpathlineto{\pgfqpoint{3.886327in}{1.703381in}}%
\pgfpathlineto{\pgfqpoint{3.886625in}{1.703635in}}%
\pgfpathlineto{\pgfqpoint{3.890789in}{1.728769in}}%
\pgfpathlineto{\pgfqpoint{3.913100in}{1.862783in}}%
\pgfpathlineto{\pgfqpoint{3.916670in}{1.821948in}}%
\pgfpathlineto{\pgfqpoint{3.927082in}{1.703286in}}%
\pgfpathlineto{\pgfqpoint{3.928867in}{1.703234in}}%
\pgfpathlineto{\pgfqpoint{3.984495in}{1.703234in}}%
\pgfpathlineto{\pgfqpoint{3.984792in}{1.703738in}}%
\pgfpathlineto{\pgfqpoint{3.998179in}{1.768415in}}%
\pgfpathlineto{\pgfqpoint{4.025844in}{1.783262in}}%
\pgfpathlineto{\pgfqpoint{4.026142in}{1.782626in}}%
\pgfpathlineto{\pgfqpoint{4.027332in}{1.703549in}}%
\pgfpathlineto{\pgfqpoint{4.027629in}{1.705117in}}%
\pgfpathlineto{\pgfqpoint{4.033281in}{1.747813in}}%
\pgfpathlineto{\pgfqpoint{4.039231in}{1.851759in}}%
\pgfpathlineto{\pgfqpoint{4.039528in}{1.864241in}}%
\pgfpathlineto{\pgfqpoint{4.039826in}{1.909531in}}%
\pgfpathlineto{\pgfqpoint{4.126987in}{1.909658in}}%
\pgfpathlineto{\pgfqpoint{4.140968in}{1.917047in}}%
\pgfpathlineto{\pgfqpoint{4.141266in}{1.916182in}}%
\pgfpathlineto{\pgfqpoint{4.141860in}{1.909645in}}%
\pgfpathlineto{\pgfqpoint{4.142158in}{1.910519in}}%
\pgfpathlineto{\pgfqpoint{4.145430in}{1.935029in}}%
\pgfpathlineto{\pgfqpoint{4.145728in}{1.935099in}}%
\pgfpathlineto{\pgfqpoint{4.147513in}{1.919262in}}%
\pgfpathlineto{\pgfqpoint{4.147810in}{1.921269in}}%
\pgfpathlineto{\pgfqpoint{4.148405in}{1.943626in}}%
\pgfpathlineto{\pgfqpoint{4.152272in}{1.936508in}}%
\pgfpathlineto{\pgfqpoint{4.152867in}{1.943914in}}%
\pgfpathlineto{\pgfqpoint{4.155842in}{1.943871in}}%
\pgfpathlineto{\pgfqpoint{4.156437in}{1.937111in}}%
\pgfpathlineto{\pgfqpoint{4.190944in}{2.027539in}}%
\pgfpathlineto{\pgfqpoint{4.191242in}{2.027071in}}%
\pgfpathlineto{\pgfqpoint{4.206413in}{1.928992in}}%
\pgfpathlineto{\pgfqpoint{4.208495in}{1.936686in}}%
\pgfpathlineto{\pgfqpoint{4.232889in}{2.027961in}}%
\pgfpathlineto{\pgfqpoint{4.233484in}{2.027807in}}%
\pgfpathlineto{\pgfqpoint{4.253415in}{2.005360in}}%
\pgfpathlineto{\pgfqpoint{4.253712in}{2.031615in}}%
\pgfpathlineto{\pgfqpoint{4.254010in}{2.118908in}}%
\pgfpathlineto{\pgfqpoint{4.256092in}{2.128700in}}%
\pgfpathlineto{\pgfqpoint{4.276023in}{2.219306in}}%
\pgfpathlineto{\pgfqpoint{4.280783in}{2.226259in}}%
\pgfpathlineto{\pgfqpoint{4.281080in}{2.224781in}}%
\pgfpathlineto{\pgfqpoint{4.283162in}{2.139551in}}%
\pgfpathlineto{\pgfqpoint{4.283460in}{2.168896in}}%
\pgfpathlineto{\pgfqpoint{4.283757in}{2.234258in}}%
\pgfpathlineto{\pgfqpoint{4.290302in}{2.234393in}}%
\pgfpathlineto{\pgfqpoint{4.304878in}{2.241899in}}%
\pgfpathlineto{\pgfqpoint{4.326297in}{2.241899in}}%
\pgfpathlineto{\pgfqpoint{4.326594in}{2.242385in}}%
\pgfpathlineto{\pgfqpoint{4.328081in}{2.270555in}}%
\pgfpathlineto{\pgfqpoint{4.331651in}{2.337407in}}%
\pgfpathlineto{\pgfqpoint{4.333436in}{2.337407in}}%
\pgfpathlineto{\pgfqpoint{4.333734in}{2.339878in}}%
\pgfpathlineto{\pgfqpoint{4.334329in}{2.348858in}}%
\pgfpathlineto{\pgfqpoint{4.342955in}{2.348867in}}%
\pgfpathlineto{\pgfqpoint{4.377165in}{2.348873in}}%
\pgfpathlineto{\pgfqpoint{4.378058in}{2.350050in}}%
\pgfpathlineto{\pgfqpoint{4.382520in}{2.356508in}}%
\pgfpathlineto{\pgfqpoint{4.383115in}{2.356749in}}%
\pgfpathlineto{\pgfqpoint{4.391444in}{2.364482in}}%
\pgfpathlineto{\pgfqpoint{4.398584in}{2.367969in}}%
\pgfpathlineto{\pgfqpoint{4.483365in}{2.367969in}}%
\pgfpathlineto{\pgfqpoint{4.483662in}{2.370307in}}%
\pgfpathlineto{\pgfqpoint{4.491397in}{2.542425in}}%
\pgfpathlineto{\pgfqpoint{4.492289in}{2.547509in}}%
\pgfpathlineto{\pgfqpoint{4.497941in}{2.547524in}}%
\pgfpathlineto{\pgfqpoint{4.505973in}{2.547624in}}%
\pgfpathlineto{\pgfqpoint{4.544050in}{2.579461in}}%
\pgfpathlineto{\pgfqpoint{4.583912in}{2.604824in}}%
\pgfpathlineto{\pgfqpoint{4.585102in}{2.608649in}}%
\pgfpathlineto{\pgfqpoint{4.635078in}{2.608694in}}%
\pgfpathlineto{\pgfqpoint{4.637756in}{2.611976in}}%
\pgfpathlineto{\pgfqpoint{4.646978in}{2.623446in}}%
\pgfpathlineto{\pgfqpoint{4.647275in}{2.621204in}}%
\pgfpathlineto{\pgfqpoint{4.647870in}{2.403098in}}%
\pgfpathlineto{\pgfqpoint{4.648167in}{2.536889in}}%
\pgfpathlineto{\pgfqpoint{4.648762in}{2.382832in}}%
\pgfpathlineto{\pgfqpoint{4.649060in}{2.384253in}}%
\pgfpathlineto{\pgfqpoint{4.649655in}{2.624351in}}%
\pgfpathlineto{\pgfqpoint{4.655902in}{2.631571in}}%
\pgfpathlineto{\pgfqpoint{4.706176in}{2.631849in}}%
\pgfpathlineto{\pgfqpoint{4.812673in}{2.635391in}}%
\pgfpathlineto{\pgfqpoint{4.856402in}{2.635642in}}%
\pgfpathlineto{\pgfqpoint{4.886447in}{2.639229in}}%
\pgfpathlineto{\pgfqpoint{5.040540in}{2.643250in}}%
\pgfpathlineto{\pgfqpoint{5.127701in}{2.685055in}}%
\pgfpathlineto{\pgfqpoint{5.475453in}{2.685117in}}%
\pgfpathlineto{\pgfqpoint{5.476048in}{2.695171in}}%
\pgfpathlineto{\pgfqpoint{5.476940in}{2.709323in}}%
\pgfpathlineto{\pgfqpoint{5.484080in}{2.788680in}}%
\pgfpathlineto{\pgfqpoint{5.491517in}{2.814583in}}%
\pgfpathlineto{\pgfqpoint{5.492111in}{2.810652in}}%
\pgfpathlineto{\pgfqpoint{5.504011in}{2.714774in}}%
\pgfpathlineto{\pgfqpoint{5.504308in}{2.714695in}}%
\pgfpathlineto{\pgfqpoint{5.505200in}{2.826407in}}%
\pgfpathlineto{\pgfqpoint{5.527511in}{2.826534in}}%
\pgfpathlineto{\pgfqpoint{5.534056in}{2.830411in}}%
\pgfpathlineto{\pgfqpoint{5.549227in}{2.834761in}}%
\pgfpathlineto{\pgfqpoint{5.598311in}{2.856782in}}%
\pgfpathlineto{\pgfqpoint{5.598608in}{2.857515in}}%
\pgfpathlineto{\pgfqpoint{5.599203in}{2.941017in}}%
\pgfpathlineto{\pgfqpoint{5.604558in}{2.941017in}}%
\pgfpathlineto{\pgfqpoint{5.604855in}{2.916774in}}%
\pgfpathlineto{\pgfqpoint{5.605450in}{2.853149in}}%
\pgfpathlineto{\pgfqpoint{5.626571in}{2.853149in}}%
\pgfpathlineto{\pgfqpoint{5.626869in}{2.850571in}}%
\pgfpathlineto{\pgfqpoint{5.627464in}{2.853892in}}%
\pgfpathlineto{\pgfqpoint{5.639660in}{2.939915in}}%
\pgfpathlineto{\pgfqpoint{5.639958in}{2.949484in}}%
\pgfpathlineto{\pgfqpoint{5.640255in}{2.943218in}}%
\pgfpathlineto{\pgfqpoint{5.640850in}{2.881676in}}%
\pgfpathlineto{\pgfqpoint{5.647990in}{2.967960in}}%
\pgfpathlineto{\pgfqpoint{5.653047in}{2.981969in}}%
\pgfpathlineto{\pgfqpoint{5.654832in}{2.986774in}}%
\pgfpathlineto{\pgfqpoint{5.656022in}{2.986860in}}%
\pgfpathlineto{\pgfqpoint{5.658996in}{2.986609in}}%
\pgfpathlineto{\pgfqpoint{5.798216in}{2.971707in}}%
\pgfpathlineto{\pgfqpoint{5.825881in}{2.986711in}}%
\pgfpathlineto{\pgfqpoint{5.826179in}{2.987619in}}%
\pgfpathlineto{\pgfqpoint{5.829154in}{3.033991in}}%
\pgfpathlineto{\pgfqpoint{5.833021in}{3.094804in}}%
\pgfpathlineto{\pgfqpoint{5.833318in}{3.089631in}}%
\pgfpathlineto{\pgfqpoint{5.835996in}{2.991599in}}%
\pgfpathlineto{\pgfqpoint{5.836293in}{2.990681in}}%
\pgfpathlineto{\pgfqpoint{5.839863in}{2.990681in}}%
\pgfpathlineto{\pgfqpoint{5.840160in}{2.989822in}}%
\pgfpathlineto{\pgfqpoint{5.842540in}{2.943568in}}%
\pgfpathlineto{\pgfqpoint{5.847597in}{2.843971in}}%
\pgfpathlineto{\pgfqpoint{5.847895in}{2.882024in}}%
\pgfpathlineto{\pgfqpoint{5.848192in}{2.990681in}}%
\pgfpathlineto{\pgfqpoint{5.875858in}{2.990889in}}%
\pgfpathlineto{\pgfqpoint{5.976405in}{3.112916in}}%
\pgfpathlineto{\pgfqpoint{5.977000in}{3.105698in}}%
\pgfpathlineto{\pgfqpoint{5.982950in}{3.011866in}}%
\pgfpathlineto{\pgfqpoint{5.983247in}{3.035836in}}%
\pgfpathlineto{\pgfqpoint{5.983545in}{3.128212in}}%
\pgfpathlineto{\pgfqpoint{5.993361in}{3.128479in}}%
\pgfpathlineto{\pgfqpoint{6.004368in}{3.129313in}}%
\pgfpathlineto{\pgfqpoint{6.004368in}{3.129313in}}%
\pgfusepath{stroke}%
\end{pgfscope}%
\begin{pgfscope}%
\pgfpathrectangle{\pgfqpoint{0.481681in}{1.080890in}}{\pgfqpoint{5.785672in}{2.146863in}}%
\pgfusepath{clip}%
\pgfsetrectcap%
\pgfsetroundjoin%
\pgfsetlinewidth{0.200750pt}%
\definecolor{currentstroke}{rgb}{0.580392,0.823529,0.741176}%
\pgfsetstrokecolor{currentstroke}%
\pgfsetdash{}{0pt}%
\pgfpathmoveto{\pgfqpoint{0.744666in}{1.516038in}}%
\pgfpathlineto{\pgfqpoint{0.751508in}{1.519463in}}%
\pgfpathlineto{\pgfqpoint{0.757160in}{1.525427in}}%
\pgfpathlineto{\pgfqpoint{0.758052in}{1.528446in}}%
\pgfpathlineto{\pgfqpoint{0.766382in}{1.528899in}}%
\pgfpathlineto{\pgfqpoint{0.791072in}{1.529996in}}%
\pgfpathlineto{\pgfqpoint{0.791965in}{1.529996in}}%
\pgfpathlineto{\pgfqpoint{0.792262in}{1.531423in}}%
\pgfpathlineto{\pgfqpoint{0.794047in}{1.547669in}}%
\pgfpathlineto{\pgfqpoint{0.794642in}{1.545878in}}%
\pgfpathlineto{\pgfqpoint{0.796130in}{1.546292in}}%
\pgfpathlineto{\pgfqpoint{0.813681in}{1.551978in}}%
\pgfpathlineto{\pgfqpoint{0.814573in}{1.553542in}}%
\pgfpathlineto{\pgfqpoint{0.823498in}{1.553552in}}%
\pgfpathlineto{\pgfqpoint{0.837776in}{1.553810in}}%
\pgfpathlineto{\pgfqpoint{0.852948in}{1.556089in}}%
\pgfpathlineto{\pgfqpoint{0.857707in}{1.558699in}}%
\pgfpathlineto{\pgfqpoint{0.893702in}{1.558523in}}%
\pgfpathlineto{\pgfqpoint{0.894000in}{1.557951in}}%
\pgfpathlineto{\pgfqpoint{0.894892in}{1.553477in}}%
\pgfpathlineto{\pgfqpoint{0.949331in}{1.553477in}}%
\pgfpathlineto{\pgfqpoint{0.949628in}{1.554106in}}%
\pgfpathlineto{\pgfqpoint{0.950223in}{1.561330in}}%
\pgfpathlineto{\pgfqpoint{0.950818in}{1.561626in}}%
\pgfpathlineto{\pgfqpoint{0.963015in}{1.561640in}}%
\pgfpathlineto{\pgfqpoint{0.964204in}{1.563224in}}%
\pgfpathlineto{\pgfqpoint{0.964502in}{1.563224in}}%
\pgfpathlineto{\pgfqpoint{0.964799in}{1.561006in}}%
\pgfpathlineto{\pgfqpoint{0.965097in}{1.517312in}}%
\pgfpathlineto{\pgfqpoint{0.965692in}{1.516942in}}%
\pgfpathlineto{\pgfqpoint{1.006149in}{1.516719in}}%
\pgfpathlineto{\pgfqpoint{1.007339in}{1.504655in}}%
\pgfpathlineto{\pgfqpoint{1.014181in}{1.433204in}}%
\pgfpathlineto{\pgfqpoint{1.025485in}{1.436862in}}%
\pgfpathlineto{\pgfqpoint{1.043631in}{1.442746in}}%
\pgfpathlineto{\pgfqpoint{1.043928in}{1.444787in}}%
\pgfpathlineto{\pgfqpoint{1.045118in}{1.459691in}}%
\pgfpathlineto{\pgfqpoint{1.049878in}{1.458941in}}%
\pgfpathlineto{\pgfqpoint{1.050473in}{1.464315in}}%
\pgfpathlineto{\pgfqpoint{1.093607in}{1.464584in}}%
\pgfpathlineto{\pgfqpoint{1.240264in}{1.470784in}}%
\pgfpathlineto{\pgfqpoint{1.254245in}{1.471804in}}%
\pgfpathlineto{\pgfqpoint{1.286373in}{1.471804in}}%
\pgfpathlineto{\pgfqpoint{1.286670in}{1.476577in}}%
\pgfpathlineto{\pgfqpoint{1.286968in}{1.493360in}}%
\pgfpathlineto{\pgfqpoint{1.287860in}{1.493255in}}%
\pgfpathlineto{\pgfqpoint{1.291430in}{1.493232in}}%
\pgfpathlineto{\pgfqpoint{1.291727in}{1.472666in}}%
\pgfpathlineto{\pgfqpoint{1.292025in}{1.473577in}}%
\pgfpathlineto{\pgfqpoint{1.292917in}{1.473577in}}%
\pgfpathlineto{\pgfqpoint{1.293215in}{1.473969in}}%
\pgfpathlineto{\pgfqpoint{1.293512in}{1.475302in}}%
\pgfpathlineto{\pgfqpoint{1.296487in}{1.474184in}}%
\pgfpathlineto{\pgfqpoint{1.301247in}{1.472616in}}%
\pgfpathlineto{\pgfqpoint{1.341109in}{1.472616in}}%
\pgfpathlineto{\pgfqpoint{1.341704in}{1.523068in}}%
\pgfpathlineto{\pgfqpoint{1.343191in}{1.523173in}}%
\pgfpathlineto{\pgfqpoint{1.345868in}{1.528110in}}%
\pgfpathlineto{\pgfqpoint{1.350330in}{1.528110in}}%
\pgfpathlineto{\pgfqpoint{1.350628in}{1.528660in}}%
\pgfpathlineto{\pgfqpoint{1.350925in}{1.533041in}}%
\pgfpathlineto{\pgfqpoint{1.352115in}{1.484209in}}%
\pgfpathlineto{\pgfqpoint{1.352413in}{1.494212in}}%
\pgfpathlineto{\pgfqpoint{1.352710in}{1.535914in}}%
\pgfpathlineto{\pgfqpoint{1.353900in}{1.535894in}}%
\pgfpathlineto{\pgfqpoint{1.358362in}{1.535765in}}%
\pgfpathlineto{\pgfqpoint{1.385730in}{1.535765in}}%
\pgfpathlineto{\pgfqpoint{1.386028in}{1.537572in}}%
\pgfpathlineto{\pgfqpoint{1.386325in}{1.555483in}}%
\pgfpathlineto{\pgfqpoint{1.398522in}{1.556146in}}%
\pgfpathlineto{\pgfqpoint{1.399117in}{1.553218in}}%
\pgfpathlineto{\pgfqpoint{1.401497in}{1.538431in}}%
\pgfpathlineto{\pgfqpoint{1.401794in}{1.548343in}}%
\pgfpathlineto{\pgfqpoint{1.402092in}{1.569605in}}%
\pgfpathlineto{\pgfqpoint{1.408041in}{1.570495in}}%
\pgfpathlineto{\pgfqpoint{1.408636in}{1.569130in}}%
\pgfpathlineto{\pgfqpoint{1.412801in}{1.556846in}}%
\pgfpathlineto{\pgfqpoint{1.413098in}{1.564147in}}%
\pgfpathlineto{\pgfqpoint{1.413396in}{1.583003in}}%
\pgfpathlineto{\pgfqpoint{1.413991in}{1.560944in}}%
\pgfpathlineto{\pgfqpoint{1.415181in}{1.595527in}}%
\pgfpathlineto{\pgfqpoint{1.544583in}{1.595780in}}%
\pgfpathlineto{\pgfqpoint{1.567787in}{1.598000in}}%
\pgfpathlineto{\pgfqpoint{1.842358in}{1.597730in}}%
\pgfpathlineto{\pgfqpoint{1.843251in}{1.597680in}}%
\pgfpathlineto{\pgfqpoint{1.843846in}{1.596342in}}%
\pgfpathlineto{\pgfqpoint{1.848903in}{1.597079in}}%
\pgfpathlineto{\pgfqpoint{1.849200in}{1.596283in}}%
\pgfpathlineto{\pgfqpoint{1.863777in}{1.596213in}}%
\pgfpathlineto{\pgfqpoint{1.865264in}{1.591833in}}%
\pgfpathlineto{\pgfqpoint{1.886088in}{1.528153in}}%
\pgfpathlineto{\pgfqpoint{1.886385in}{1.548783in}}%
\pgfpathlineto{\pgfqpoint{1.886683in}{1.598783in}}%
\pgfpathlineto{\pgfqpoint{1.887278in}{1.602072in}}%
\pgfpathlineto{\pgfqpoint{1.899772in}{1.574313in}}%
\pgfpathlineto{\pgfqpoint{1.900367in}{1.582565in}}%
\pgfpathlineto{\pgfqpoint{1.959565in}{1.587278in}}%
\pgfpathlineto{\pgfqpoint{1.965514in}{1.587679in}}%
\pgfpathlineto{\pgfqpoint{1.991692in}{1.586726in}}%
\pgfpathlineto{\pgfqpoint{1.992287in}{1.589070in}}%
\pgfpathlineto{\pgfqpoint{2.001211in}{1.634859in}}%
\pgfpathlineto{\pgfqpoint{2.001509in}{1.636789in}}%
\pgfpathlineto{\pgfqpoint{2.002104in}{1.818049in}}%
\pgfpathlineto{\pgfqpoint{2.002401in}{1.818938in}}%
\pgfpathlineto{\pgfqpoint{2.002699in}{1.821490in}}%
\pgfpathlineto{\pgfqpoint{2.005376in}{1.821490in}}%
\pgfpathlineto{\pgfqpoint{2.005674in}{1.822009in}}%
\pgfpathlineto{\pgfqpoint{2.006566in}{1.830343in}}%
\pgfpathlineto{\pgfqpoint{2.045536in}{1.830605in}}%
\pgfpathlineto{\pgfqpoint{2.242168in}{1.843491in}}%
\pgfpathlineto{\pgfqpoint{2.242466in}{1.844855in}}%
\pgfpathlineto{\pgfqpoint{2.248713in}{1.845124in}}%
\pgfpathlineto{\pgfqpoint{2.319810in}{1.847754in}}%
\pgfpathlineto{\pgfqpoint{2.392097in}{1.847631in}}%
\pgfpathlineto{\pgfqpoint{2.393882in}{1.861199in}}%
\pgfpathlineto{\pgfqpoint{2.399534in}{1.904494in}}%
\pgfpathlineto{\pgfqpoint{2.399831in}{1.904588in}}%
\pgfpathlineto{\pgfqpoint{2.400426in}{1.904038in}}%
\pgfpathlineto{\pgfqpoint{2.401021in}{1.906517in}}%
\pgfpathlineto{\pgfqpoint{2.405186in}{1.911861in}}%
\pgfpathlineto{\pgfqpoint{2.405781in}{1.914608in}}%
\pgfpathlineto{\pgfqpoint{2.457542in}{1.914503in}}%
\pgfpathlineto{\pgfqpoint{2.458137in}{1.915465in}}%
\pgfpathlineto{\pgfqpoint{2.499189in}{1.915465in}}%
\pgfpathlineto{\pgfqpoint{2.499486in}{1.918423in}}%
\pgfpathlineto{\pgfqpoint{2.500974in}{1.951866in}}%
\pgfpathlineto{\pgfqpoint{2.501569in}{1.951707in}}%
\pgfpathlineto{\pgfqpoint{2.506328in}{1.951707in}}%
\pgfpathlineto{\pgfqpoint{2.506626in}{1.951378in}}%
\pgfpathlineto{\pgfqpoint{2.507518in}{1.910607in}}%
\pgfpathlineto{\pgfqpoint{2.513170in}{1.625016in}}%
\pgfpathlineto{\pgfqpoint{2.513765in}{1.624602in}}%
\pgfpathlineto{\pgfqpoint{3.141741in}{1.624877in}}%
\pgfpathlineto{\pgfqpoint{3.255377in}{1.626746in}}%
\pgfpathlineto{\pgfqpoint{3.434756in}{1.626469in}}%
\pgfpathlineto{\pgfqpoint{3.548690in}{1.625814in}}%
\pgfpathlineto{\pgfqpoint{3.550475in}{1.626746in}}%
\pgfpathlineto{\pgfqpoint{3.561779in}{1.626481in}}%
\pgfpathlineto{\pgfqpoint{3.591229in}{1.625115in}}%
\pgfpathlineto{\pgfqpoint{3.592122in}{1.624526in}}%
\pgfpathlineto{\pgfqpoint{3.612350in}{1.624801in}}%
\pgfpathlineto{\pgfqpoint{3.662327in}{1.626739in}}%
\pgfpathlineto{\pgfqpoint{3.662922in}{1.624667in}}%
\pgfpathlineto{\pgfqpoint{3.663516in}{1.624728in}}%
\pgfpathlineto{\pgfqpoint{3.847655in}{1.687177in}}%
\pgfpathlineto{\pgfqpoint{3.886327in}{1.651599in}}%
\pgfpathlineto{\pgfqpoint{3.886625in}{1.651769in}}%
\pgfpathlineto{\pgfqpoint{3.890789in}{1.676512in}}%
\pgfpathlineto{\pgfqpoint{3.913100in}{1.808473in}}%
\pgfpathlineto{\pgfqpoint{3.916670in}{1.768263in}}%
\pgfpathlineto{\pgfqpoint{3.927082in}{1.651420in}}%
\pgfpathlineto{\pgfqpoint{3.928867in}{1.651369in}}%
\pgfpathlineto{\pgfqpoint{3.984495in}{1.651369in}}%
\pgfpathlineto{\pgfqpoint{3.984792in}{1.651905in}}%
\pgfpathlineto{\pgfqpoint{3.998179in}{1.720722in}}%
\pgfpathlineto{\pgfqpoint{4.025844in}{1.733600in}}%
\pgfpathlineto{\pgfqpoint{4.026142in}{1.732922in}}%
\pgfpathlineto{\pgfqpoint{4.027332in}{1.651714in}}%
\pgfpathlineto{\pgfqpoint{4.027629in}{1.653620in}}%
\pgfpathlineto{\pgfqpoint{4.033579in}{1.708908in}}%
\pgfpathlineto{\pgfqpoint{4.039231in}{1.801804in}}%
\pgfpathlineto{\pgfqpoint{4.039528in}{1.818091in}}%
\pgfpathlineto{\pgfqpoint{4.039826in}{1.884861in}}%
\pgfpathlineto{\pgfqpoint{4.076415in}{1.885057in}}%
\pgfpathlineto{\pgfqpoint{4.080580in}{1.887624in}}%
\pgfpathlineto{\pgfqpoint{4.099619in}{1.884861in}}%
\pgfpathlineto{\pgfqpoint{4.126689in}{1.884881in}}%
\pgfpathlineto{\pgfqpoint{4.128177in}{1.886882in}}%
\pgfpathlineto{\pgfqpoint{4.140968in}{1.904957in}}%
\pgfpathlineto{\pgfqpoint{4.141266in}{1.901801in}}%
\pgfpathlineto{\pgfqpoint{4.141860in}{1.878714in}}%
\pgfpathlineto{\pgfqpoint{4.142158in}{1.879589in}}%
\pgfpathlineto{\pgfqpoint{4.145728in}{1.912055in}}%
\pgfpathlineto{\pgfqpoint{4.147513in}{1.900286in}}%
\pgfpathlineto{\pgfqpoint{4.147810in}{1.903988in}}%
\pgfpathlineto{\pgfqpoint{4.148405in}{1.932826in}}%
\pgfpathlineto{\pgfqpoint{4.152272in}{1.917494in}}%
\pgfpathlineto{\pgfqpoint{4.152867in}{1.937462in}}%
\pgfpathlineto{\pgfqpoint{4.155842in}{1.937417in}}%
\pgfpathlineto{\pgfqpoint{4.156437in}{1.930068in}}%
\pgfpathlineto{\pgfqpoint{4.190944in}{2.001038in}}%
\pgfpathlineto{\pgfqpoint{4.191242in}{2.000652in}}%
\pgfpathlineto{\pgfqpoint{4.206413in}{1.921649in}}%
\pgfpathlineto{\pgfqpoint{4.208495in}{1.927846in}}%
\pgfpathlineto{\pgfqpoint{4.232889in}{2.001369in}}%
\pgfpathlineto{\pgfqpoint{4.233484in}{2.001263in}}%
\pgfpathlineto{\pgfqpoint{4.253415in}{1.985853in}}%
\pgfpathlineto{\pgfqpoint{4.253712in}{2.003829in}}%
\pgfpathlineto{\pgfqpoint{4.254010in}{2.063753in}}%
\pgfpathlineto{\pgfqpoint{4.255794in}{2.070702in}}%
\pgfpathlineto{\pgfqpoint{4.276320in}{2.147076in}}%
\pgfpathlineto{\pgfqpoint{4.280783in}{2.157778in}}%
\pgfpathlineto{\pgfqpoint{4.281080in}{2.156530in}}%
\pgfpathlineto{\pgfqpoint{4.283162in}{2.072014in}}%
\pgfpathlineto{\pgfqpoint{4.283460in}{2.096923in}}%
\pgfpathlineto{\pgfqpoint{4.283757in}{2.154756in}}%
\pgfpathlineto{\pgfqpoint{4.290599in}{2.154955in}}%
\pgfpathlineto{\pgfqpoint{4.304878in}{2.159899in}}%
\pgfpathlineto{\pgfqpoint{4.305771in}{2.159688in}}%
\pgfpathlineto{\pgfqpoint{4.306366in}{2.159243in}}%
\pgfpathlineto{\pgfqpoint{4.326297in}{2.158399in}}%
\pgfpathlineto{\pgfqpoint{4.326594in}{2.159021in}}%
\pgfpathlineto{\pgfqpoint{4.328081in}{2.195735in}}%
\pgfpathlineto{\pgfqpoint{4.331354in}{2.274500in}}%
\pgfpathlineto{\pgfqpoint{4.331949in}{2.274616in}}%
\pgfpathlineto{\pgfqpoint{4.333436in}{2.274477in}}%
\pgfpathlineto{\pgfqpoint{4.333734in}{2.275307in}}%
\pgfpathlineto{\pgfqpoint{4.334329in}{2.278377in}}%
\pgfpathlineto{\pgfqpoint{4.359614in}{2.278380in}}%
\pgfpathlineto{\pgfqpoint{4.377165in}{2.278388in}}%
\pgfpathlineto{\pgfqpoint{4.378058in}{2.280085in}}%
\pgfpathlineto{\pgfqpoint{4.382520in}{2.289240in}}%
\pgfpathlineto{\pgfqpoint{4.382817in}{2.289165in}}%
\pgfpathlineto{\pgfqpoint{4.384602in}{2.290779in}}%
\pgfpathlineto{\pgfqpoint{4.390849in}{2.296556in}}%
\pgfpathlineto{\pgfqpoint{4.396204in}{2.295411in}}%
\pgfpathlineto{\pgfqpoint{4.399179in}{2.294998in}}%
\pgfpathlineto{\pgfqpoint{4.483365in}{2.300621in}}%
\pgfpathlineto{\pgfqpoint{4.483662in}{2.303082in}}%
\pgfpathlineto{\pgfqpoint{4.491694in}{2.489348in}}%
\pgfpathlineto{\pgfqpoint{4.492289in}{2.495465in}}%
\pgfpathlineto{\pgfqpoint{4.495264in}{2.495494in}}%
\pgfpathlineto{\pgfqpoint{4.505973in}{2.495600in}}%
\pgfpathlineto{\pgfqpoint{4.543158in}{2.528327in}}%
\pgfpathlineto{\pgfqpoint{4.584210in}{2.550283in}}%
\pgfpathlineto{\pgfqpoint{4.585102in}{2.551126in}}%
\pgfpathlineto{\pgfqpoint{4.598786in}{2.556758in}}%
\pgfpathlineto{\pgfqpoint{4.635376in}{2.558958in}}%
\pgfpathlineto{\pgfqpoint{4.646978in}{2.566410in}}%
\pgfpathlineto{\pgfqpoint{4.647275in}{2.563703in}}%
\pgfpathlineto{\pgfqpoint{4.647870in}{2.321251in}}%
\pgfpathlineto{\pgfqpoint{4.648167in}{2.471819in}}%
\pgfpathlineto{\pgfqpoint{4.648762in}{2.302508in}}%
\pgfpathlineto{\pgfqpoint{4.649060in}{2.306304in}}%
\pgfpathlineto{\pgfqpoint{4.649655in}{2.565483in}}%
\pgfpathlineto{\pgfqpoint{4.655604in}{2.576218in}}%
\pgfpathlineto{\pgfqpoint{4.657984in}{2.576184in}}%
\pgfpathlineto{\pgfqpoint{4.705581in}{2.575010in}}%
\pgfpathlineto{\pgfqpoint{4.840338in}{2.575793in}}%
\pgfpathlineto{\pgfqpoint{4.855807in}{2.576031in}}%
\pgfpathlineto{\pgfqpoint{4.884662in}{2.581268in}}%
\pgfpathlineto{\pgfqpoint{4.886150in}{2.582124in}}%
\pgfpathlineto{\pgfqpoint{5.041135in}{2.597700in}}%
\pgfpathlineto{\pgfqpoint{5.127999in}{2.625006in}}%
\pgfpathlineto{\pgfqpoint{5.401083in}{2.625269in}}%
\pgfpathlineto{\pgfqpoint{5.449275in}{2.628601in}}%
\pgfpathlineto{\pgfqpoint{5.475453in}{2.625109in}}%
\pgfpathlineto{\pgfqpoint{5.476048in}{2.639602in}}%
\pgfpathlineto{\pgfqpoint{5.476940in}{2.659878in}}%
\pgfpathlineto{\pgfqpoint{5.484080in}{2.771818in}}%
\pgfpathlineto{\pgfqpoint{5.491517in}{2.789817in}}%
\pgfpathlineto{\pgfqpoint{5.492409in}{2.781514in}}%
\pgfpathlineto{\pgfqpoint{5.504011in}{2.661152in}}%
\pgfpathlineto{\pgfqpoint{5.504308in}{2.660982in}}%
\pgfpathlineto{\pgfqpoint{5.505200in}{2.798476in}}%
\pgfpathlineto{\pgfqpoint{5.527809in}{2.800110in}}%
\pgfpathlineto{\pgfqpoint{5.537626in}{2.805195in}}%
\pgfpathlineto{\pgfqpoint{5.550120in}{2.810872in}}%
\pgfpathlineto{\pgfqpoint{5.598311in}{2.828390in}}%
\pgfpathlineto{\pgfqpoint{5.598608in}{2.829095in}}%
\pgfpathlineto{\pgfqpoint{5.599203in}{2.912071in}}%
\pgfpathlineto{\pgfqpoint{5.604261in}{2.911200in}}%
\pgfpathlineto{\pgfqpoint{5.604558in}{2.910631in}}%
\pgfpathlineto{\pgfqpoint{5.604855in}{2.886655in}}%
\pgfpathlineto{\pgfqpoint{5.605450in}{2.829433in}}%
\pgfpathlineto{\pgfqpoint{5.615267in}{2.829239in}}%
\pgfpathlineto{\pgfqpoint{5.615862in}{2.829542in}}%
\pgfpathlineto{\pgfqpoint{5.626571in}{2.829469in}}%
\pgfpathlineto{\pgfqpoint{5.626869in}{2.828785in}}%
\pgfpathlineto{\pgfqpoint{5.628059in}{2.835747in}}%
\pgfpathlineto{\pgfqpoint{5.639660in}{2.907406in}}%
\pgfpathlineto{\pgfqpoint{5.639958in}{2.918236in}}%
\pgfpathlineto{\pgfqpoint{5.640255in}{2.912543in}}%
\pgfpathlineto{\pgfqpoint{5.640850in}{2.856558in}}%
\pgfpathlineto{\pgfqpoint{5.647990in}{2.934009in}}%
\pgfpathlineto{\pgfqpoint{5.651262in}{2.938833in}}%
\pgfpathlineto{\pgfqpoint{5.655129in}{2.944021in}}%
\pgfpathlineto{\pgfqpoint{5.660484in}{2.943760in}}%
\pgfpathlineto{\pgfqpoint{5.798513in}{2.934516in}}%
\pgfpathlineto{\pgfqpoint{5.825881in}{2.943927in}}%
\pgfpathlineto{\pgfqpoint{5.826179in}{2.945194in}}%
\pgfpathlineto{\pgfqpoint{5.829154in}{3.016094in}}%
\pgfpathlineto{\pgfqpoint{5.833021in}{3.109092in}}%
\pgfpathlineto{\pgfqpoint{5.833318in}{3.108583in}}%
\pgfpathlineto{\pgfqpoint{5.836293in}{2.944426in}}%
\pgfpathlineto{\pgfqpoint{5.839863in}{2.944426in}}%
\pgfpathlineto{\pgfqpoint{5.840160in}{2.943533in}}%
\pgfpathlineto{\pgfqpoint{5.842540in}{2.895486in}}%
\pgfpathlineto{\pgfqpoint{5.847597in}{2.792029in}}%
\pgfpathlineto{\pgfqpoint{5.847895in}{2.831833in}}%
\pgfpathlineto{\pgfqpoint{5.848192in}{2.945472in}}%
\pgfpathlineto{\pgfqpoint{5.854439in}{2.947878in}}%
\pgfpathlineto{\pgfqpoint{5.875560in}{2.947880in}}%
\pgfpathlineto{\pgfqpoint{5.876155in}{2.948707in}}%
\pgfpathlineto{\pgfqpoint{5.976108in}{3.125473in}}%
\pgfpathlineto{\pgfqpoint{5.976405in}{3.123931in}}%
\pgfpathlineto{\pgfqpoint{5.980570in}{3.023526in}}%
\pgfpathlineto{\pgfqpoint{5.982950in}{2.965890in}}%
\pgfpathlineto{\pgfqpoint{5.983247in}{2.999344in}}%
\pgfpathlineto{\pgfqpoint{5.983545in}{3.130169in}}%
\pgfpathlineto{\pgfqpoint{5.989494in}{3.130057in}}%
\pgfpathlineto{\pgfqpoint{5.990089in}{3.128980in}}%
\pgfpathlineto{\pgfqpoint{6.004368in}{3.129313in}}%
\pgfpathlineto{\pgfqpoint{6.004368in}{3.129313in}}%
\pgfusepath{stroke}%
\end{pgfscope}%
\begin{pgfscope}%
\pgfpathrectangle{\pgfqpoint{0.481681in}{1.080890in}}{\pgfqpoint{5.785672in}{2.146863in}}%
\pgfusepath{clip}%
\pgfsetrectcap%
\pgfsetroundjoin%
\pgfsetlinewidth{0.200750pt}%
\definecolor{currentstroke}{rgb}{0.933333,0.607843,0.000000}%
\pgfsetstrokecolor{currentstroke}%
\pgfsetdash{}{0pt}%
\pgfpathmoveto{\pgfqpoint{0.744666in}{1.516038in}}%
\pgfpathlineto{\pgfqpoint{0.750615in}{1.515172in}}%
\pgfpathlineto{\pgfqpoint{0.751805in}{1.520187in}}%
\pgfpathlineto{\pgfqpoint{0.757457in}{1.544281in}}%
\pgfpathlineto{\pgfqpoint{0.758350in}{1.544304in}}%
\pgfpathlineto{\pgfqpoint{0.788693in}{1.557830in}}%
\pgfpathlineto{\pgfqpoint{0.791965in}{1.557830in}}%
\pgfpathlineto{\pgfqpoint{0.792262in}{1.560020in}}%
\pgfpathlineto{\pgfqpoint{0.794047in}{1.585412in}}%
\pgfpathlineto{\pgfqpoint{0.794345in}{1.585797in}}%
\pgfpathlineto{\pgfqpoint{0.794940in}{1.583675in}}%
\pgfpathlineto{\pgfqpoint{0.814276in}{1.582361in}}%
\pgfpathlineto{\pgfqpoint{0.815763in}{1.582420in}}%
\pgfpathlineto{\pgfqpoint{0.852055in}{1.582444in}}%
\pgfpathlineto{\pgfqpoint{0.852948in}{1.585369in}}%
\pgfpathlineto{\pgfqpoint{0.857410in}{1.601476in}}%
\pgfpathlineto{\pgfqpoint{0.858897in}{1.601545in}}%
\pgfpathlineto{\pgfqpoint{0.893405in}{1.601595in}}%
\pgfpathlineto{\pgfqpoint{0.894000in}{1.630948in}}%
\pgfpathlineto{\pgfqpoint{0.894595in}{1.646547in}}%
\pgfpathlineto{\pgfqpoint{0.949331in}{1.646547in}}%
\pgfpathlineto{\pgfqpoint{0.949628in}{1.647206in}}%
\pgfpathlineto{\pgfqpoint{0.950520in}{1.657877in}}%
\pgfpathlineto{\pgfqpoint{0.963609in}{1.657681in}}%
\pgfpathlineto{\pgfqpoint{0.964204in}{1.657395in}}%
\pgfpathlineto{\pgfqpoint{0.964502in}{1.657395in}}%
\pgfpathlineto{\pgfqpoint{0.964799in}{1.653127in}}%
\pgfpathlineto{\pgfqpoint{0.965097in}{1.568340in}}%
\pgfpathlineto{\pgfqpoint{0.965394in}{1.566348in}}%
\pgfpathlineto{\pgfqpoint{1.005851in}{1.566348in}}%
\pgfpathlineto{\pgfqpoint{1.006149in}{1.566058in}}%
\pgfpathlineto{\pgfqpoint{1.007339in}{1.550329in}}%
\pgfpathlineto{\pgfqpoint{1.014181in}{1.457171in}}%
\pgfpathlineto{\pgfqpoint{1.023402in}{1.460813in}}%
\pgfpathlineto{\pgfqpoint{1.043631in}{1.468822in}}%
\pgfpathlineto{\pgfqpoint{1.043928in}{1.469537in}}%
\pgfpathlineto{\pgfqpoint{1.045118in}{1.488721in}}%
\pgfpathlineto{\pgfqpoint{1.049878in}{1.488765in}}%
\pgfpathlineto{\pgfqpoint{1.050473in}{1.490971in}}%
\pgfpathlineto{\pgfqpoint{1.090930in}{1.491246in}}%
\pgfpathlineto{\pgfqpoint{1.236396in}{1.501013in}}%
\pgfpathlineto{\pgfqpoint{1.253353in}{1.499168in}}%
\pgfpathlineto{\pgfqpoint{1.286373in}{1.499168in}}%
\pgfpathlineto{\pgfqpoint{1.286670in}{1.503504in}}%
\pgfpathlineto{\pgfqpoint{1.286968in}{1.519981in}}%
\pgfpathlineto{\pgfqpoint{1.291430in}{1.519958in}}%
\pgfpathlineto{\pgfqpoint{1.292025in}{1.490730in}}%
\pgfpathlineto{\pgfqpoint{1.292917in}{1.490730in}}%
\pgfpathlineto{\pgfqpoint{1.293215in}{1.491218in}}%
\pgfpathlineto{\pgfqpoint{1.293512in}{1.492775in}}%
\pgfpathlineto{\pgfqpoint{1.296784in}{1.488930in}}%
\pgfpathlineto{\pgfqpoint{1.300652in}{1.484467in}}%
\pgfpathlineto{\pgfqpoint{1.319393in}{1.484462in}}%
\pgfpathlineto{\pgfqpoint{1.341109in}{1.484462in}}%
\pgfpathlineto{\pgfqpoint{1.341704in}{1.532196in}}%
\pgfpathlineto{\pgfqpoint{1.350330in}{1.532196in}}%
\pgfpathlineto{\pgfqpoint{1.350628in}{1.533573in}}%
\pgfpathlineto{\pgfqpoint{1.350925in}{1.549869in}}%
\pgfpathlineto{\pgfqpoint{1.351520in}{1.526766in}}%
\pgfpathlineto{\pgfqpoint{1.352115in}{1.497675in}}%
\pgfpathlineto{\pgfqpoint{1.352413in}{1.509075in}}%
\pgfpathlineto{\pgfqpoint{1.352710in}{1.556464in}}%
\pgfpathlineto{\pgfqpoint{1.365799in}{1.556143in}}%
\pgfpathlineto{\pgfqpoint{1.385730in}{1.556143in}}%
\pgfpathlineto{\pgfqpoint{1.386028in}{1.558669in}}%
\pgfpathlineto{\pgfqpoint{1.386325in}{1.583706in}}%
\pgfpathlineto{\pgfqpoint{1.398522in}{1.583697in}}%
\pgfpathlineto{\pgfqpoint{1.399117in}{1.580405in}}%
\pgfpathlineto{\pgfqpoint{1.401497in}{1.563814in}}%
\pgfpathlineto{\pgfqpoint{1.401794in}{1.579675in}}%
\pgfpathlineto{\pgfqpoint{1.402092in}{1.612216in}}%
\pgfpathlineto{\pgfqpoint{1.406554in}{1.610224in}}%
\pgfpathlineto{\pgfqpoint{1.408041in}{1.610221in}}%
\pgfpathlineto{\pgfqpoint{1.408636in}{1.607633in}}%
\pgfpathlineto{\pgfqpoint{1.412801in}{1.584349in}}%
\pgfpathlineto{\pgfqpoint{1.413098in}{1.595486in}}%
\pgfpathlineto{\pgfqpoint{1.413396in}{1.625223in}}%
\pgfpathlineto{\pgfqpoint{1.413991in}{1.592369in}}%
\pgfpathlineto{\pgfqpoint{1.414883in}{1.643961in}}%
\pgfpathlineto{\pgfqpoint{1.415478in}{1.644548in}}%
\pgfpathlineto{\pgfqpoint{1.494012in}{1.663583in}}%
\pgfpathlineto{\pgfqpoint{1.544583in}{1.669902in}}%
\pgfpathlineto{\pgfqpoint{1.566597in}{1.674592in}}%
\pgfpathlineto{\pgfqpoint{1.843251in}{1.674436in}}%
\pgfpathlineto{\pgfqpoint{1.843846in}{1.674126in}}%
\pgfpathlineto{\pgfqpoint{1.849795in}{1.674110in}}%
\pgfpathlineto{\pgfqpoint{1.863777in}{1.673985in}}%
\pgfpathlineto{\pgfqpoint{1.865264in}{1.666141in}}%
\pgfpathlineto{\pgfqpoint{1.886088in}{1.552110in}}%
\pgfpathlineto{\pgfqpoint{1.886385in}{1.513485in}}%
\pgfpathlineto{\pgfqpoint{1.886683in}{1.432684in}}%
\pgfpathlineto{\pgfqpoint{1.887278in}{1.438265in}}%
\pgfpathlineto{\pgfqpoint{1.899772in}{1.456128in}}%
\pgfpathlineto{\pgfqpoint{1.900367in}{1.468069in}}%
\pgfpathlineto{\pgfqpoint{1.946178in}{1.468329in}}%
\pgfpathlineto{\pgfqpoint{1.964324in}{1.472922in}}%
\pgfpathlineto{\pgfqpoint{1.991692in}{1.470535in}}%
\pgfpathlineto{\pgfqpoint{1.992287in}{1.472266in}}%
\pgfpathlineto{\pgfqpoint{2.000319in}{1.501675in}}%
\pgfpathlineto{\pgfqpoint{2.001509in}{1.502002in}}%
\pgfpathlineto{\pgfqpoint{2.002104in}{1.682467in}}%
\pgfpathlineto{\pgfqpoint{2.002401in}{1.682995in}}%
\pgfpathlineto{\pgfqpoint{2.002699in}{1.684155in}}%
\pgfpathlineto{\pgfqpoint{2.005376in}{1.684155in}}%
\pgfpathlineto{\pgfqpoint{2.005674in}{1.684772in}}%
\pgfpathlineto{\pgfqpoint{2.006566in}{1.694682in}}%
\pgfpathlineto{\pgfqpoint{2.044346in}{1.694943in}}%
\pgfpathlineto{\pgfqpoint{2.242168in}{1.713032in}}%
\pgfpathlineto{\pgfqpoint{2.242763in}{1.712527in}}%
\pgfpathlineto{\pgfqpoint{2.335279in}{1.713405in}}%
\pgfpathlineto{\pgfqpoint{2.391502in}{1.713405in}}%
\pgfpathlineto{\pgfqpoint{2.391800in}{1.713708in}}%
\pgfpathlineto{\pgfqpoint{2.392989in}{1.729806in}}%
\pgfpathlineto{\pgfqpoint{2.399831in}{1.810074in}}%
\pgfpathlineto{\pgfqpoint{2.400724in}{1.802674in}}%
\pgfpathlineto{\pgfqpoint{2.401021in}{1.803048in}}%
\pgfpathlineto{\pgfqpoint{2.405186in}{1.818960in}}%
\pgfpathlineto{\pgfqpoint{2.405781in}{1.827432in}}%
\pgfpathlineto{\pgfqpoint{2.457542in}{1.826631in}}%
\pgfpathlineto{\pgfqpoint{2.458137in}{1.828881in}}%
\pgfpathlineto{\pgfqpoint{2.499189in}{1.828881in}}%
\pgfpathlineto{\pgfqpoint{2.499486in}{1.832432in}}%
\pgfpathlineto{\pgfqpoint{2.500974in}{1.873561in}}%
\pgfpathlineto{\pgfqpoint{2.506328in}{1.873561in}}%
\pgfpathlineto{\pgfqpoint{2.506626in}{1.872864in}}%
\pgfpathlineto{\pgfqpoint{2.507518in}{1.786472in}}%
\pgfpathlineto{\pgfqpoint{2.513170in}{1.181136in}}%
\pgfpathlineto{\pgfqpoint{2.513468in}{1.179547in}}%
\pgfpathlineto{\pgfqpoint{2.514360in}{1.184402in}}%
\pgfpathlineto{\pgfqpoint{3.188742in}{1.184130in}}%
\pgfpathlineto{\pgfqpoint{3.247048in}{1.183036in}}%
\pgfpathlineto{\pgfqpoint{3.409173in}{1.182759in}}%
\pgfpathlineto{\pgfqpoint{3.548095in}{1.178782in}}%
\pgfpathlineto{\pgfqpoint{3.548393in}{1.179398in}}%
\pgfpathlineto{\pgfqpoint{3.550475in}{1.191071in}}%
\pgfpathlineto{\pgfqpoint{3.556722in}{1.190834in}}%
\pgfpathlineto{\pgfqpoint{3.591229in}{1.178475in}}%
\pgfpathlineto{\pgfqpoint{3.591527in}{1.181131in}}%
\pgfpathlineto{\pgfqpoint{3.591824in}{1.186411in}}%
\pgfpathlineto{\pgfqpoint{3.608483in}{1.186672in}}%
\pgfpathlineto{\pgfqpoint{3.662327in}{1.191058in}}%
\pgfpathlineto{\pgfqpoint{3.662922in}{1.186694in}}%
\pgfpathlineto{\pgfqpoint{3.663516in}{1.186639in}}%
\pgfpathlineto{\pgfqpoint{3.847655in}{1.257477in}}%
\pgfpathlineto{\pgfqpoint{3.886327in}{1.209543in}}%
\pgfpathlineto{\pgfqpoint{3.886625in}{1.209811in}}%
\pgfpathlineto{\pgfqpoint{3.890789in}{1.245654in}}%
\pgfpathlineto{\pgfqpoint{3.913100in}{1.436805in}}%
\pgfpathlineto{\pgfqpoint{3.916670in}{1.378560in}}%
\pgfpathlineto{\pgfqpoint{3.927082in}{1.209307in}}%
\pgfpathlineto{\pgfqpoint{3.928272in}{1.209233in}}%
\pgfpathlineto{\pgfqpoint{3.984495in}{1.209233in}}%
\pgfpathlineto{\pgfqpoint{3.984792in}{1.210086in}}%
\pgfpathlineto{\pgfqpoint{3.998179in}{1.319496in}}%
\pgfpathlineto{\pgfqpoint{4.025844in}{1.340421in}}%
\pgfpathlineto{\pgfqpoint{4.026142in}{1.339343in}}%
\pgfpathlineto{\pgfqpoint{4.027332in}{1.209733in}}%
\pgfpathlineto{\pgfqpoint{4.027629in}{1.212088in}}%
\pgfpathlineto{\pgfqpoint{4.032984in}{1.272627in}}%
\pgfpathlineto{\pgfqpoint{4.033281in}{1.286180in}}%
\pgfpathlineto{\pgfqpoint{4.039231in}{1.824052in}}%
\pgfpathlineto{\pgfqpoint{4.039826in}{1.514677in}}%
\pgfpathlineto{\pgfqpoint{4.076118in}{1.514757in}}%
\pgfpathlineto{\pgfqpoint{4.078498in}{1.519605in}}%
\pgfpathlineto{\pgfqpoint{4.080580in}{1.523422in}}%
\pgfpathlineto{\pgfqpoint{4.098726in}{1.514677in}}%
\pgfpathlineto{\pgfqpoint{4.126987in}{1.514451in}}%
\pgfpathlineto{\pgfqpoint{4.140968in}{1.501318in}}%
\pgfpathlineto{\pgfqpoint{4.141266in}{1.502907in}}%
\pgfpathlineto{\pgfqpoint{4.141860in}{1.514870in}}%
\pgfpathlineto{\pgfqpoint{4.142158in}{1.516846in}}%
\pgfpathlineto{\pgfqpoint{4.145430in}{1.560668in}}%
\pgfpathlineto{\pgfqpoint{4.145728in}{1.559123in}}%
\pgfpathlineto{\pgfqpoint{4.147513in}{1.508218in}}%
\pgfpathlineto{\pgfqpoint{4.147810in}{1.508984in}}%
\pgfpathlineto{\pgfqpoint{4.148405in}{1.550147in}}%
\pgfpathlineto{\pgfqpoint{4.152272in}{1.570812in}}%
\pgfpathlineto{\pgfqpoint{4.152867in}{1.554535in}}%
\pgfpathlineto{\pgfqpoint{4.155842in}{1.554460in}}%
\pgfpathlineto{\pgfqpoint{4.156437in}{1.541577in}}%
\pgfpathlineto{\pgfqpoint{4.190944in}{1.608705in}}%
\pgfpathlineto{\pgfqpoint{4.191242in}{1.608287in}}%
\pgfpathlineto{\pgfqpoint{4.206413in}{1.527670in}}%
\pgfpathlineto{\pgfqpoint{4.208495in}{1.533994in}}%
\pgfpathlineto{\pgfqpoint{4.232889in}{1.609019in}}%
\pgfpathlineto{\pgfqpoint{4.233186in}{1.609019in}}%
\pgfpathlineto{\pgfqpoint{4.233781in}{1.610110in}}%
\pgfpathlineto{\pgfqpoint{4.253415in}{1.659491in}}%
\pgfpathlineto{\pgfqpoint{4.253712in}{1.696473in}}%
\pgfpathlineto{\pgfqpoint{4.254010in}{1.816127in}}%
\pgfpathlineto{\pgfqpoint{4.254902in}{1.820108in}}%
\pgfpathlineto{\pgfqpoint{4.276023in}{1.898693in}}%
\pgfpathlineto{\pgfqpoint{4.280783in}{1.896573in}}%
\pgfpathlineto{\pgfqpoint{4.281080in}{1.895194in}}%
\pgfpathlineto{\pgfqpoint{4.283162in}{1.835181in}}%
\pgfpathlineto{\pgfqpoint{4.283460in}{1.857345in}}%
\pgfpathlineto{\pgfqpoint{4.283757in}{1.905865in}}%
\pgfpathlineto{\pgfqpoint{4.290004in}{1.905889in}}%
\pgfpathlineto{\pgfqpoint{4.291789in}{1.908070in}}%
\pgfpathlineto{\pgfqpoint{4.304878in}{1.924187in}}%
\pgfpathlineto{\pgfqpoint{4.305771in}{1.923986in}}%
\pgfpathlineto{\pgfqpoint{4.306068in}{1.923380in}}%
\pgfpathlineto{\pgfqpoint{4.307556in}{1.923904in}}%
\pgfpathlineto{\pgfqpoint{4.326297in}{1.931291in}}%
\pgfpathlineto{\pgfqpoint{4.326594in}{1.932169in}}%
\pgfpathlineto{\pgfqpoint{4.328081in}{1.976759in}}%
\pgfpathlineto{\pgfqpoint{4.331354in}{2.077125in}}%
\pgfpathlineto{\pgfqpoint{4.331949in}{2.077834in}}%
\pgfpathlineto{\pgfqpoint{4.333436in}{2.077834in}}%
\pgfpathlineto{\pgfqpoint{4.333734in}{2.080572in}}%
\pgfpathlineto{\pgfqpoint{4.334329in}{2.090520in}}%
\pgfpathlineto{\pgfqpoint{4.342360in}{2.090530in}}%
\pgfpathlineto{\pgfqpoint{4.382817in}{2.090657in}}%
\pgfpathlineto{\pgfqpoint{4.384007in}{2.093657in}}%
\pgfpathlineto{\pgfqpoint{4.390849in}{2.111697in}}%
\pgfpathlineto{\pgfqpoint{4.392932in}{2.111423in}}%
\pgfpathlineto{\pgfqpoint{4.399476in}{2.110684in}}%
\pgfpathlineto{\pgfqpoint{4.483365in}{2.117515in}}%
\pgfpathlineto{\pgfqpoint{4.483662in}{2.120956in}}%
\pgfpathlineto{\pgfqpoint{4.491694in}{2.380176in}}%
\pgfpathlineto{\pgfqpoint{4.492289in}{2.387742in}}%
\pgfpathlineto{\pgfqpoint{4.494669in}{2.387778in}}%
\pgfpathlineto{\pgfqpoint{4.505973in}{2.387911in}}%
\pgfpathlineto{\pgfqpoint{4.543158in}{2.428971in}}%
\pgfpathlineto{\pgfqpoint{4.583912in}{2.456337in}}%
\pgfpathlineto{\pgfqpoint{4.585102in}{2.461933in}}%
\pgfpathlineto{\pgfqpoint{4.598786in}{2.468626in}}%
\pgfpathlineto{\pgfqpoint{4.646383in}{2.468874in}}%
\pgfpathlineto{\pgfqpoint{4.646978in}{2.468866in}}%
\pgfpathlineto{\pgfqpoint{4.647275in}{2.464917in}}%
\pgfpathlineto{\pgfqpoint{4.647870in}{2.136950in}}%
\pgfpathlineto{\pgfqpoint{4.648167in}{2.352508in}}%
\pgfpathlineto{\pgfqpoint{4.649060in}{2.097096in}}%
\pgfpathlineto{\pgfqpoint{4.649655in}{2.437498in}}%
\pgfpathlineto{\pgfqpoint{4.655902in}{2.446808in}}%
\pgfpathlineto{\pgfqpoint{4.700226in}{2.452628in}}%
\pgfpathlineto{\pgfqpoint{4.856997in}{2.452353in}}%
\pgfpathlineto{\pgfqpoint{4.886447in}{2.449450in}}%
\pgfpathlineto{\pgfqpoint{5.041433in}{2.457855in}}%
\pgfpathlineto{\pgfqpoint{5.128296in}{2.478584in}}%
\pgfpathlineto{\pgfqpoint{5.447490in}{2.478585in}}%
\pgfpathlineto{\pgfqpoint{5.448085in}{2.482859in}}%
\pgfpathlineto{\pgfqpoint{5.448382in}{2.484589in}}%
\pgfpathlineto{\pgfqpoint{5.452249in}{2.484611in}}%
\pgfpathlineto{\pgfqpoint{5.475453in}{2.484700in}}%
\pgfpathlineto{\pgfqpoint{5.476048in}{2.499250in}}%
\pgfpathlineto{\pgfqpoint{5.476940in}{2.518260in}}%
\pgfpathlineto{\pgfqpoint{5.484377in}{2.613405in}}%
\pgfpathlineto{\pgfqpoint{5.491517in}{2.669657in}}%
\pgfpathlineto{\pgfqpoint{5.492111in}{2.663927in}}%
\pgfpathlineto{\pgfqpoint{5.504011in}{2.519393in}}%
\pgfpathlineto{\pgfqpoint{5.504308in}{2.519190in}}%
\pgfpathlineto{\pgfqpoint{5.505200in}{2.676263in}}%
\pgfpathlineto{\pgfqpoint{5.530784in}{2.675623in}}%
\pgfpathlineto{\pgfqpoint{5.533461in}{2.675450in}}%
\pgfpathlineto{\pgfqpoint{5.548930in}{2.685431in}}%
\pgfpathlineto{\pgfqpoint{5.598311in}{2.700887in}}%
\pgfpathlineto{\pgfqpoint{5.598608in}{2.701750in}}%
\pgfpathlineto{\pgfqpoint{5.599203in}{2.810510in}}%
\pgfpathlineto{\pgfqpoint{5.604558in}{2.839958in}}%
\pgfpathlineto{\pgfqpoint{5.604855in}{2.803245in}}%
\pgfpathlineto{\pgfqpoint{5.605450in}{2.704244in}}%
\pgfpathlineto{\pgfqpoint{5.615267in}{2.704858in}}%
\pgfpathlineto{\pgfqpoint{5.615862in}{2.704489in}}%
\pgfpathlineto{\pgfqpoint{5.626274in}{2.705165in}}%
\pgfpathlineto{\pgfqpoint{5.626571in}{2.705389in}}%
\pgfpathlineto{\pgfqpoint{5.626869in}{2.703777in}}%
\pgfpathlineto{\pgfqpoint{5.627761in}{2.712530in}}%
\pgfpathlineto{\pgfqpoint{5.639660in}{2.839339in}}%
\pgfpathlineto{\pgfqpoint{5.639958in}{2.850682in}}%
\pgfpathlineto{\pgfqpoint{5.640255in}{2.838142in}}%
\pgfpathlineto{\pgfqpoint{5.640850in}{2.749926in}}%
\pgfpathlineto{\pgfqpoint{5.648287in}{2.839485in}}%
\pgfpathlineto{\pgfqpoint{5.654832in}{2.863789in}}%
\pgfpathlineto{\pgfqpoint{5.655724in}{2.863906in}}%
\pgfpathlineto{\pgfqpoint{5.658401in}{2.863648in}}%
\pgfpathlineto{\pgfqpoint{5.798216in}{2.843028in}}%
\pgfpathlineto{\pgfqpoint{5.825881in}{2.863700in}}%
\pgfpathlineto{\pgfqpoint{5.826179in}{2.865404in}}%
\pgfpathlineto{\pgfqpoint{5.829154in}{2.956479in}}%
\pgfpathlineto{\pgfqpoint{5.833021in}{3.075928in}}%
\pgfpathlineto{\pgfqpoint{5.833318in}{3.066020in}}%
\pgfpathlineto{\pgfqpoint{5.835996in}{2.870554in}}%
\pgfpathlineto{\pgfqpoint{5.836293in}{2.868727in}}%
\pgfpathlineto{\pgfqpoint{5.839863in}{2.868727in}}%
\pgfpathlineto{\pgfqpoint{5.840160in}{2.867640in}}%
\pgfpathlineto{\pgfqpoint{5.842540in}{2.809106in}}%
\pgfpathlineto{\pgfqpoint{5.847597in}{2.683066in}}%
\pgfpathlineto{\pgfqpoint{5.847895in}{2.731701in}}%
\pgfpathlineto{\pgfqpoint{5.848192in}{2.870466in}}%
\pgfpathlineto{\pgfqpoint{5.853844in}{2.872106in}}%
\pgfpathlineto{\pgfqpoint{5.855332in}{2.876120in}}%
\pgfpathlineto{\pgfqpoint{5.875560in}{2.876122in}}%
\pgfpathlineto{\pgfqpoint{5.876155in}{2.877184in}}%
\pgfpathlineto{\pgfqpoint{5.976108in}{3.104000in}}%
\pgfpathlineto{\pgfqpoint{5.976405in}{3.103092in}}%
\pgfpathlineto{\pgfqpoint{5.977595in}{3.070188in}}%
\pgfpathlineto{\pgfqpoint{5.982950in}{2.913361in}}%
\pgfpathlineto{\pgfqpoint{5.983247in}{2.957386in}}%
\pgfpathlineto{\pgfqpoint{5.983545in}{3.127352in}}%
\pgfpathlineto{\pgfqpoint{5.989494in}{3.128173in}}%
\pgfpathlineto{\pgfqpoint{5.990089in}{3.127437in}}%
\pgfpathlineto{\pgfqpoint{6.004368in}{3.129313in}}%
\pgfpathlineto{\pgfqpoint{6.004368in}{3.129313in}}%
\pgfusepath{stroke}%
\end{pgfscope}%
\begin{pgfscope}%
\pgfsetrectcap%
\pgfsetmiterjoin%
\pgfsetlinewidth{0.501875pt}%
\definecolor{currentstroke}{rgb}{0.000000,0.000000,0.000000}%
\pgfsetstrokecolor{currentstroke}%
\pgfsetdash{}{0pt}%
\pgfpathmoveto{\pgfqpoint{0.481681in}{1.080890in}}%
\pgfpathlineto{\pgfqpoint{0.481681in}{3.227753in}}%
\pgfusepath{stroke}%
\end{pgfscope}%
\begin{pgfscope}%
\pgfsetrectcap%
\pgfsetmiterjoin%
\pgfsetlinewidth{0.501875pt}%
\definecolor{currentstroke}{rgb}{0.000000,0.000000,0.000000}%
\pgfsetstrokecolor{currentstroke}%
\pgfsetdash{}{0pt}%
\pgfpathmoveto{\pgfqpoint{6.267353in}{1.080890in}}%
\pgfpathlineto{\pgfqpoint{6.267353in}{3.227753in}}%
\pgfusepath{stroke}%
\end{pgfscope}%
\begin{pgfscope}%
\pgfsetrectcap%
\pgfsetmiterjoin%
\pgfsetlinewidth{0.501875pt}%
\definecolor{currentstroke}{rgb}{0.000000,0.000000,0.000000}%
\pgfsetstrokecolor{currentstroke}%
\pgfsetdash{}{0pt}%
\pgfpathmoveto{\pgfqpoint{0.481681in}{1.080890in}}%
\pgfpathlineto{\pgfqpoint{6.267353in}{1.080890in}}%
\pgfusepath{stroke}%
\end{pgfscope}%
\begin{pgfscope}%
\pgfsetrectcap%
\pgfsetmiterjoin%
\pgfsetlinewidth{0.501875pt}%
\definecolor{currentstroke}{rgb}{0.000000,0.000000,0.000000}%
\pgfsetstrokecolor{currentstroke}%
\pgfsetdash{}{0pt}%
\pgfpathmoveto{\pgfqpoint{0.481681in}{3.227753in}}%
\pgfpathlineto{\pgfqpoint{6.267353in}{3.227753in}}%
\pgfusepath{stroke}%
\end{pgfscope}%
\begin{pgfscope}%
\pgfsetrectcap%
\pgfsetroundjoin%
\pgfsetlinewidth{0.401500pt}%
\definecolor{currentstroke}{rgb}{0.000000,0.070588,0.098039}%
\pgfsetstrokecolor{currentstroke}%
\pgfsetdash{}{0pt}%
\pgfpathmoveto{\pgfqpoint{0.569181in}{3.105934in}}%
\pgfpathlineto{\pgfqpoint{0.666403in}{3.105934in}}%
\pgfpathlineto{\pgfqpoint{0.763625in}{3.105934in}}%
\pgfusepath{stroke}%
\end{pgfscope}%
\begin{pgfscope}%
\definecolor{textcolor}{rgb}{0.000000,0.000000,0.000000}%
\pgfsetstrokecolor{textcolor}%
\pgfsetfillcolor{textcolor}%
\pgftext[x=0.841403in,y=3.071906in,left,base]{\color{textcolor}\rmfamily\fontsize{7.000000}{8.400000}\selectfont Story Points}%
\end{pgfscope}%
\begin{pgfscope}%
\pgfsetrectcap%
\pgfsetroundjoin%
\pgfsetlinewidth{0.200750pt}%
\definecolor{currentstroke}{rgb}{0.682353,0.125490,0.070588}%
\pgfsetstrokecolor{currentstroke}%
\pgfsetdash{}{0pt}%
\pgfpathmoveto{\pgfqpoint{0.569181in}{2.969142in}}%
\pgfpathlineto{\pgfqpoint{0.666403in}{2.969142in}}%
\pgfpathlineto{\pgfqpoint{0.763625in}{2.969142in}}%
\pgfusepath{stroke}%
\end{pgfscope}%
\begin{pgfscope}%
\definecolor{textcolor}{rgb}{0.000000,0.000000,0.000000}%
\pgfsetstrokecolor{textcolor}%
\pgfsetfillcolor{textcolor}%
\pgftext[x=0.841403in,y=2.935114in,left,base]{\color{textcolor}\rmfamily\fontsize{7.000000}{8.400000}\selectfont Logische Codezeilen}%
\end{pgfscope}%
\begin{pgfscope}%
\pgfsetrectcap%
\pgfsetroundjoin%
\pgfsetlinewidth{0.200750pt}%
\definecolor{currentstroke}{rgb}{0.000000,0.372549,0.450980}%
\pgfsetstrokecolor{currentstroke}%
\pgfsetdash{}{0pt}%
\pgfpathmoveto{\pgfqpoint{0.569181in}{2.832545in}}%
\pgfpathlineto{\pgfqpoint{0.666403in}{2.832545in}}%
\pgfpathlineto{\pgfqpoint{0.763625in}{2.832545in}}%
\pgfusepath{stroke}%
\end{pgfscope}%
\begin{pgfscope}%
\definecolor{textcolor}{rgb}{0.000000,0.000000,0.000000}%
\pgfsetstrokecolor{textcolor}%
\pgfsetfillcolor{textcolor}%
\pgftext[x=0.841403in,y=2.798517in,left,base]{\color{textcolor}\rmfamily\fontsize{7.000000}{8.400000}\selectfont Zyklomatische Komplexität}%
\end{pgfscope}%
\begin{pgfscope}%
\pgfsetrectcap%
\pgfsetroundjoin%
\pgfsetlinewidth{0.200750pt}%
\definecolor{currentstroke}{rgb}{0.580392,0.823529,0.741176}%
\pgfsetstrokecolor{currentstroke}%
\pgfsetdash{}{0pt}%
\pgfpathmoveto{\pgfqpoint{0.569181in}{2.696045in}}%
\pgfpathlineto{\pgfqpoint{0.666403in}{2.696045in}}%
\pgfpathlineto{\pgfqpoint{0.763625in}{2.696045in}}%
\pgfusepath{stroke}%
\end{pgfscope}%
\begin{pgfscope}%
\definecolor{textcolor}{rgb}{0.000000,0.000000,0.000000}%
\pgfsetstrokecolor{textcolor}%
\pgfsetfillcolor{textcolor}%
\pgftext[x=0.841403in,y=2.662017in,left,base]{\color{textcolor}\rmfamily\fontsize{7.000000}{8.400000}\selectfont Halstead Aufwand}%
\end{pgfscope}%
\begin{pgfscope}%
\pgfsetrectcap%
\pgfsetroundjoin%
\pgfsetlinewidth{0.200750pt}%
\definecolor{currentstroke}{rgb}{0.933333,0.607843,0.000000}%
\pgfsetstrokecolor{currentstroke}%
\pgfsetdash{}{0pt}%
\pgfpathmoveto{\pgfqpoint{0.569181in}{2.560517in}}%
\pgfpathlineto{\pgfqpoint{0.666403in}{2.560517in}}%
\pgfpathlineto{\pgfqpoint{0.763625in}{2.560517in}}%
\pgfusepath{stroke}%
\end{pgfscope}%
\begin{pgfscope}%
\definecolor{textcolor}{rgb}{0.000000,0.000000,0.000000}%
\pgfsetstrokecolor{textcolor}%
\pgfsetfillcolor{textcolor}%
\pgftext[x=0.841403in,y=2.526489in,left,base]{\color{textcolor}\rmfamily\fontsize{7.000000}{8.400000}\selectfont Einrückungskomplexität}%
\end{pgfscope}%
\begin{pgfscope}%
\pgfsetbuttcap%
\pgfsetmiterjoin%
\definecolor{currentfill}{rgb}{1.000000,1.000000,1.000000}%
\pgfsetfillcolor{currentfill}%
\pgfsetlinewidth{0.000000pt}%
\definecolor{currentstroke}{rgb}{0.000000,0.000000,0.000000}%
\pgfsetstrokecolor{currentstroke}%
\pgfsetstrokeopacity{0.000000}%
\pgfsetdash{}{0pt}%
\pgfpathmoveto{\pgfqpoint{0.481681in}{0.586309in}}%
\pgfpathlineto{\pgfqpoint{6.267353in}{0.586309in}}%
\pgfpathlineto{\pgfqpoint{6.267353in}{0.893003in}}%
\pgfpathlineto{\pgfqpoint{0.481681in}{0.893003in}}%
\pgfpathlineto{\pgfqpoint{0.481681in}{0.586309in}}%
\pgfpathclose%
\pgfusepath{fill}%
\end{pgfscope}%
\begin{pgfscope}%
\pgfpathrectangle{\pgfqpoint{0.481681in}{0.586309in}}{\pgfqpoint{5.785672in}{0.306695in}}%
\pgfusepath{clip}%
\pgfsetbuttcap%
\pgfsetroundjoin%
\definecolor{currentfill}{rgb}{0.800000,0.788235,0.760784}%
\pgfsetfillcolor{currentfill}%
\pgfsetlinewidth{0.000000pt}%
\definecolor{currentstroke}{rgb}{0.000000,0.000000,0.000000}%
\pgfsetstrokecolor{currentstroke}%
\pgfsetdash{}{0pt}%
\pgfpathmoveto{\pgfqpoint{0.744666in}{0.739656in}}%
\pgfpathlineto{\pgfqpoint{0.744666in}{0.739656in}}%
\pgfpathlineto{\pgfqpoint{0.744963in}{0.739597in}}%
\pgfpathlineto{\pgfqpoint{0.745261in}{0.739538in}}%
\pgfpathlineto{\pgfqpoint{0.745558in}{0.739479in}}%
\pgfpathlineto{\pgfqpoint{0.745856in}{0.739420in}}%
\pgfpathlineto{\pgfqpoint{0.746153in}{0.739361in}}%
\pgfpathlineto{\pgfqpoint{0.746451in}{0.739302in}}%
\pgfpathlineto{\pgfqpoint{0.746748in}{0.739243in}}%
\pgfpathlineto{\pgfqpoint{0.747046in}{0.739184in}}%
\pgfpathlineto{\pgfqpoint{0.747343in}{0.739124in}}%
\pgfpathlineto{\pgfqpoint{0.747641in}{0.739065in}}%
\pgfpathlineto{\pgfqpoint{0.747938in}{0.739006in}}%
\pgfpathlineto{\pgfqpoint{0.748236in}{0.738947in}}%
\pgfpathlineto{\pgfqpoint{0.748533in}{0.738888in}}%
\pgfpathlineto{\pgfqpoint{0.748831in}{0.738829in}}%
\pgfpathlineto{\pgfqpoint{0.749128in}{0.738770in}}%
\pgfpathlineto{\pgfqpoint{0.749426in}{0.738711in}}%
\pgfpathlineto{\pgfqpoint{0.749723in}{0.738652in}}%
\pgfpathlineto{\pgfqpoint{0.750021in}{0.738593in}}%
\pgfpathlineto{\pgfqpoint{0.750318in}{0.738530in}}%
\pgfpathlineto{\pgfqpoint{0.750615in}{0.738472in}}%
\pgfpathlineto{\pgfqpoint{0.750913in}{0.738462in}}%
\pgfpathlineto{\pgfqpoint{0.751210in}{0.738466in}}%
\pgfpathlineto{\pgfqpoint{0.751508in}{0.738469in}}%
\pgfpathlineto{\pgfqpoint{0.751805in}{0.738472in}}%
\pgfpathlineto{\pgfqpoint{0.752103in}{0.738476in}}%
\pgfpathlineto{\pgfqpoint{0.752400in}{0.738479in}}%
\pgfpathlineto{\pgfqpoint{0.752698in}{0.738482in}}%
\pgfpathlineto{\pgfqpoint{0.752995in}{0.738486in}}%
\pgfpathlineto{\pgfqpoint{0.753293in}{0.738489in}}%
\pgfpathlineto{\pgfqpoint{0.753590in}{0.738492in}}%
\pgfpathlineto{\pgfqpoint{0.753888in}{0.738496in}}%
\pgfpathlineto{\pgfqpoint{0.754185in}{0.738499in}}%
\pgfpathlineto{\pgfqpoint{0.754483in}{0.738502in}}%
\pgfpathlineto{\pgfqpoint{0.754780in}{0.738506in}}%
\pgfpathlineto{\pgfqpoint{0.755078in}{0.738509in}}%
\pgfpathlineto{\pgfqpoint{0.755375in}{0.738512in}}%
\pgfpathlineto{\pgfqpoint{0.755673in}{0.738516in}}%
\pgfpathlineto{\pgfqpoint{0.755970in}{0.738519in}}%
\pgfpathlineto{\pgfqpoint{0.756268in}{0.738522in}}%
\pgfpathlineto{\pgfqpoint{0.756565in}{0.738526in}}%
\pgfpathlineto{\pgfqpoint{0.756863in}{0.738529in}}%
\pgfpathlineto{\pgfqpoint{0.757160in}{0.738532in}}%
\pgfpathlineto{\pgfqpoint{0.757457in}{0.738628in}}%
\pgfpathlineto{\pgfqpoint{0.757755in}{0.739039in}}%
\pgfpathlineto{\pgfqpoint{0.758052in}{0.738996in}}%
\pgfpathlineto{\pgfqpoint{0.758350in}{0.738983in}}%
\pgfpathlineto{\pgfqpoint{0.758647in}{0.738979in}}%
\pgfpathlineto{\pgfqpoint{0.758945in}{0.738975in}}%
\pgfpathlineto{\pgfqpoint{0.759242in}{0.738971in}}%
\pgfpathlineto{\pgfqpoint{0.759540in}{0.738967in}}%
\pgfpathlineto{\pgfqpoint{0.759837in}{0.738963in}}%
\pgfpathlineto{\pgfqpoint{0.760135in}{0.738959in}}%
\pgfpathlineto{\pgfqpoint{0.760432in}{0.738955in}}%
\pgfpathlineto{\pgfqpoint{0.760730in}{0.738950in}}%
\pgfpathlineto{\pgfqpoint{0.761027in}{0.738946in}}%
\pgfpathlineto{\pgfqpoint{0.761325in}{0.738942in}}%
\pgfpathlineto{\pgfqpoint{0.761622in}{0.738938in}}%
\pgfpathlineto{\pgfqpoint{0.761920in}{0.738934in}}%
\pgfpathlineto{\pgfqpoint{0.762217in}{0.738930in}}%
\pgfpathlineto{\pgfqpoint{0.762515in}{0.738926in}}%
\pgfpathlineto{\pgfqpoint{0.762812in}{0.738922in}}%
\pgfpathlineto{\pgfqpoint{0.763110in}{0.738918in}}%
\pgfpathlineto{\pgfqpoint{0.763407in}{0.738914in}}%
\pgfpathlineto{\pgfqpoint{0.763705in}{0.738909in}}%
\pgfpathlineto{\pgfqpoint{0.764002in}{0.738905in}}%
\pgfpathlineto{\pgfqpoint{0.764299in}{0.738901in}}%
\pgfpathlineto{\pgfqpoint{0.764597in}{0.738897in}}%
\pgfpathlineto{\pgfqpoint{0.764894in}{0.738893in}}%
\pgfpathlineto{\pgfqpoint{0.765192in}{0.738889in}}%
\pgfpathlineto{\pgfqpoint{0.765489in}{0.738885in}}%
\pgfpathlineto{\pgfqpoint{0.765787in}{0.738881in}}%
\pgfpathlineto{\pgfqpoint{0.766084in}{0.738877in}}%
\pgfpathlineto{\pgfqpoint{0.766382in}{0.738873in}}%
\pgfpathlineto{\pgfqpoint{0.766679in}{0.738869in}}%
\pgfpathlineto{\pgfqpoint{0.766977in}{0.738864in}}%
\pgfpathlineto{\pgfqpoint{0.767274in}{0.738860in}}%
\pgfpathlineto{\pgfqpoint{0.767572in}{0.738856in}}%
\pgfpathlineto{\pgfqpoint{0.767869in}{0.738852in}}%
\pgfpathlineto{\pgfqpoint{0.768167in}{0.738848in}}%
\pgfpathlineto{\pgfqpoint{0.768464in}{0.738844in}}%
\pgfpathlineto{\pgfqpoint{0.768762in}{0.738840in}}%
\pgfpathlineto{\pgfqpoint{0.769059in}{0.738836in}}%
\pgfpathlineto{\pgfqpoint{0.769357in}{0.738832in}}%
\pgfpathlineto{\pgfqpoint{0.769654in}{0.738828in}}%
\pgfpathlineto{\pgfqpoint{0.769952in}{0.738823in}}%
\pgfpathlineto{\pgfqpoint{0.770249in}{0.738819in}}%
\pgfpathlineto{\pgfqpoint{0.770546in}{0.738815in}}%
\pgfpathlineto{\pgfqpoint{0.770844in}{0.738811in}}%
\pgfpathlineto{\pgfqpoint{0.771141in}{0.738807in}}%
\pgfpathlineto{\pgfqpoint{0.771439in}{0.738803in}}%
\pgfpathlineto{\pgfqpoint{0.771736in}{0.738799in}}%
\pgfpathlineto{\pgfqpoint{0.772034in}{0.738795in}}%
\pgfpathlineto{\pgfqpoint{0.772331in}{0.738791in}}%
\pgfpathlineto{\pgfqpoint{0.772629in}{0.738787in}}%
\pgfpathlineto{\pgfqpoint{0.772926in}{0.738783in}}%
\pgfpathlineto{\pgfqpoint{0.773224in}{0.738778in}}%
\pgfpathlineto{\pgfqpoint{0.773521in}{0.738774in}}%
\pgfpathlineto{\pgfqpoint{0.773819in}{0.738770in}}%
\pgfpathlineto{\pgfqpoint{0.774116in}{0.738766in}}%
\pgfpathlineto{\pgfqpoint{0.774414in}{0.738762in}}%
\pgfpathlineto{\pgfqpoint{0.774711in}{0.738758in}}%
\pgfpathlineto{\pgfqpoint{0.775009in}{0.738754in}}%
\pgfpathlineto{\pgfqpoint{0.775306in}{0.738750in}}%
\pgfpathlineto{\pgfqpoint{0.775604in}{0.738746in}}%
\pgfpathlineto{\pgfqpoint{0.775901in}{0.738742in}}%
\pgfpathlineto{\pgfqpoint{0.776199in}{0.738737in}}%
\pgfpathlineto{\pgfqpoint{0.776496in}{0.738733in}}%
\pgfpathlineto{\pgfqpoint{0.776794in}{0.738729in}}%
\pgfpathlineto{\pgfqpoint{0.777091in}{0.738725in}}%
\pgfpathlineto{\pgfqpoint{0.777388in}{0.738721in}}%
\pgfpathlineto{\pgfqpoint{0.777686in}{0.738717in}}%
\pgfpathlineto{\pgfqpoint{0.777983in}{0.738713in}}%
\pgfpathlineto{\pgfqpoint{0.778281in}{0.738709in}}%
\pgfpathlineto{\pgfqpoint{0.778578in}{0.738705in}}%
\pgfpathlineto{\pgfqpoint{0.778876in}{0.738701in}}%
\pgfpathlineto{\pgfqpoint{0.779173in}{0.738697in}}%
\pgfpathlineto{\pgfqpoint{0.779471in}{0.738692in}}%
\pgfpathlineto{\pgfqpoint{0.779768in}{0.738688in}}%
\pgfpathlineto{\pgfqpoint{0.780066in}{0.738684in}}%
\pgfpathlineto{\pgfqpoint{0.780363in}{0.738680in}}%
\pgfpathlineto{\pgfqpoint{0.780661in}{0.738676in}}%
\pgfpathlineto{\pgfqpoint{0.780958in}{0.738672in}}%
\pgfpathlineto{\pgfqpoint{0.781256in}{0.738668in}}%
\pgfpathlineto{\pgfqpoint{0.781553in}{0.738664in}}%
\pgfpathlineto{\pgfqpoint{0.781851in}{0.738660in}}%
\pgfpathlineto{\pgfqpoint{0.782148in}{0.738656in}}%
\pgfpathlineto{\pgfqpoint{0.782446in}{0.738651in}}%
\pgfpathlineto{\pgfqpoint{0.782743in}{0.738647in}}%
\pgfpathlineto{\pgfqpoint{0.783041in}{0.738643in}}%
\pgfpathlineto{\pgfqpoint{0.783338in}{0.738639in}}%
\pgfpathlineto{\pgfqpoint{0.783636in}{0.738635in}}%
\pgfpathlineto{\pgfqpoint{0.783933in}{0.738631in}}%
\pgfpathlineto{\pgfqpoint{0.784230in}{0.738625in}}%
\pgfpathlineto{\pgfqpoint{0.784528in}{0.738612in}}%
\pgfpathlineto{\pgfqpoint{0.784825in}{0.738599in}}%
\pgfpathlineto{\pgfqpoint{0.785123in}{0.738586in}}%
\pgfpathlineto{\pgfqpoint{0.785420in}{0.738573in}}%
\pgfpathlineto{\pgfqpoint{0.785718in}{0.738560in}}%
\pgfpathlineto{\pgfqpoint{0.786015in}{0.738547in}}%
\pgfpathlineto{\pgfqpoint{0.786313in}{0.738535in}}%
\pgfpathlineto{\pgfqpoint{0.786610in}{0.738522in}}%
\pgfpathlineto{\pgfqpoint{0.786908in}{0.738509in}}%
\pgfpathlineto{\pgfqpoint{0.787205in}{0.738496in}}%
\pgfpathlineto{\pgfqpoint{0.787503in}{0.738483in}}%
\pgfpathlineto{\pgfqpoint{0.787800in}{0.738470in}}%
\pgfpathlineto{\pgfqpoint{0.788098in}{0.738458in}}%
\pgfpathlineto{\pgfqpoint{0.788395in}{0.738450in}}%
\pgfpathlineto{\pgfqpoint{0.788693in}{0.738441in}}%
\pgfpathlineto{\pgfqpoint{0.788990in}{0.738432in}}%
\pgfpathlineto{\pgfqpoint{0.789288in}{0.738424in}}%
\pgfpathlineto{\pgfqpoint{0.789585in}{0.738415in}}%
\pgfpathlineto{\pgfqpoint{0.789883in}{0.738406in}}%
\pgfpathlineto{\pgfqpoint{0.790180in}{0.738398in}}%
\pgfpathlineto{\pgfqpoint{0.790477in}{0.738389in}}%
\pgfpathlineto{\pgfqpoint{0.790775in}{0.738380in}}%
\pgfpathlineto{\pgfqpoint{0.791072in}{0.738372in}}%
\pgfpathlineto{\pgfqpoint{0.791370in}{0.738363in}}%
\pgfpathlineto{\pgfqpoint{0.791667in}{0.738354in}}%
\pgfpathlineto{\pgfqpoint{0.791965in}{0.738346in}}%
\pgfpathlineto{\pgfqpoint{0.792262in}{0.738429in}}%
\pgfpathlineto{\pgfqpoint{0.792560in}{0.738602in}}%
\pgfpathlineto{\pgfqpoint{0.792857in}{0.738775in}}%
\pgfpathlineto{\pgfqpoint{0.793155in}{0.738948in}}%
\pgfpathlineto{\pgfqpoint{0.793452in}{0.739121in}}%
\pgfpathlineto{\pgfqpoint{0.793750in}{0.739294in}}%
\pgfpathlineto{\pgfqpoint{0.794047in}{0.739436in}}%
\pgfpathlineto{\pgfqpoint{0.794345in}{0.739394in}}%
\pgfpathlineto{\pgfqpoint{0.794642in}{0.739374in}}%
\pgfpathlineto{\pgfqpoint{0.794940in}{0.739367in}}%
\pgfpathlineto{\pgfqpoint{0.795237in}{0.739361in}}%
\pgfpathlineto{\pgfqpoint{0.795535in}{0.739355in}}%
\pgfpathlineto{\pgfqpoint{0.795832in}{0.739349in}}%
\pgfpathlineto{\pgfqpoint{0.796130in}{0.739343in}}%
\pgfpathlineto{\pgfqpoint{0.796427in}{0.739337in}}%
\pgfpathlineto{\pgfqpoint{0.796725in}{0.739331in}}%
\pgfpathlineto{\pgfqpoint{0.797022in}{0.739325in}}%
\pgfpathlineto{\pgfqpoint{0.797319in}{0.739319in}}%
\pgfpathlineto{\pgfqpoint{0.797617in}{0.739313in}}%
\pgfpathlineto{\pgfqpoint{0.797914in}{0.739307in}}%
\pgfpathlineto{\pgfqpoint{0.798212in}{0.739301in}}%
\pgfpathlineto{\pgfqpoint{0.798509in}{0.739295in}}%
\pgfpathlineto{\pgfqpoint{0.798807in}{0.739289in}}%
\pgfpathlineto{\pgfqpoint{0.799104in}{0.739283in}}%
\pgfpathlineto{\pgfqpoint{0.799402in}{0.739277in}}%
\pgfpathlineto{\pgfqpoint{0.799699in}{0.739271in}}%
\pgfpathlineto{\pgfqpoint{0.799997in}{0.739265in}}%
\pgfpathlineto{\pgfqpoint{0.800294in}{0.739260in}}%
\pgfpathlineto{\pgfqpoint{0.800592in}{0.739254in}}%
\pgfpathlineto{\pgfqpoint{0.800889in}{0.739248in}}%
\pgfpathlineto{\pgfqpoint{0.801187in}{0.739239in}}%
\pgfpathlineto{\pgfqpoint{0.801484in}{0.739204in}}%
\pgfpathlineto{\pgfqpoint{0.801782in}{0.739164in}}%
\pgfpathlineto{\pgfqpoint{0.802079in}{0.739124in}}%
\pgfpathlineto{\pgfqpoint{0.802377in}{0.739084in}}%
\pgfpathlineto{\pgfqpoint{0.802674in}{0.739043in}}%
\pgfpathlineto{\pgfqpoint{0.802972in}{0.739003in}}%
\pgfpathlineto{\pgfqpoint{0.803269in}{0.738963in}}%
\pgfpathlineto{\pgfqpoint{0.803567in}{0.738922in}}%
\pgfpathlineto{\pgfqpoint{0.803864in}{0.738882in}}%
\pgfpathlineto{\pgfqpoint{0.804161in}{0.738842in}}%
\pgfpathlineto{\pgfqpoint{0.804459in}{0.738802in}}%
\pgfpathlineto{\pgfqpoint{0.804756in}{0.738761in}}%
\pgfpathlineto{\pgfqpoint{0.805054in}{0.738721in}}%
\pgfpathlineto{\pgfqpoint{0.805351in}{0.738681in}}%
\pgfpathlineto{\pgfqpoint{0.805649in}{0.738640in}}%
\pgfpathlineto{\pgfqpoint{0.805946in}{0.738600in}}%
\pgfpathlineto{\pgfqpoint{0.806244in}{0.738560in}}%
\pgfpathlineto{\pgfqpoint{0.806541in}{0.738520in}}%
\pgfpathlineto{\pgfqpoint{0.806839in}{0.738479in}}%
\pgfpathlineto{\pgfqpoint{0.807136in}{0.738439in}}%
\pgfpathlineto{\pgfqpoint{0.807434in}{0.738399in}}%
\pgfpathlineto{\pgfqpoint{0.807731in}{0.738358in}}%
\pgfpathlineto{\pgfqpoint{0.808029in}{0.738318in}}%
\pgfpathlineto{\pgfqpoint{0.808326in}{0.738278in}}%
\pgfpathlineto{\pgfqpoint{0.808624in}{0.738237in}}%
\pgfpathlineto{\pgfqpoint{0.808921in}{0.738197in}}%
\pgfpathlineto{\pgfqpoint{0.809219in}{0.738157in}}%
\pgfpathlineto{\pgfqpoint{0.809516in}{0.738117in}}%
\pgfpathlineto{\pgfqpoint{0.809814in}{0.738076in}}%
\pgfpathlineto{\pgfqpoint{0.810111in}{0.738036in}}%
\pgfpathlineto{\pgfqpoint{0.810408in}{0.737996in}}%
\pgfpathlineto{\pgfqpoint{0.810706in}{0.737955in}}%
\pgfpathlineto{\pgfqpoint{0.811003in}{0.737915in}}%
\pgfpathlineto{\pgfqpoint{0.811301in}{0.737875in}}%
\pgfpathlineto{\pgfqpoint{0.811598in}{0.737835in}}%
\pgfpathlineto{\pgfqpoint{0.811896in}{0.737794in}}%
\pgfpathlineto{\pgfqpoint{0.812193in}{0.737754in}}%
\pgfpathlineto{\pgfqpoint{0.812491in}{0.737714in}}%
\pgfpathlineto{\pgfqpoint{0.812788in}{0.737673in}}%
\pgfpathlineto{\pgfqpoint{0.813086in}{0.737633in}}%
\pgfpathlineto{\pgfqpoint{0.813383in}{0.737593in}}%
\pgfpathlineto{\pgfqpoint{0.813681in}{0.737552in}}%
\pgfpathlineto{\pgfqpoint{0.813978in}{0.737512in}}%
\pgfpathlineto{\pgfqpoint{0.814276in}{0.737472in}}%
\pgfpathlineto{\pgfqpoint{0.814573in}{0.737431in}}%
\pgfpathlineto{\pgfqpoint{0.814871in}{0.737393in}}%
\pgfpathlineto{\pgfqpoint{0.815168in}{0.737385in}}%
\pgfpathlineto{\pgfqpoint{0.815466in}{0.737383in}}%
\pgfpathlineto{\pgfqpoint{0.815763in}{0.737382in}}%
\pgfpathlineto{\pgfqpoint{0.816061in}{0.737381in}}%
\pgfpathlineto{\pgfqpoint{0.816358in}{0.737379in}}%
\pgfpathlineto{\pgfqpoint{0.816656in}{0.737378in}}%
\pgfpathlineto{\pgfqpoint{0.816953in}{0.737377in}}%
\pgfpathlineto{\pgfqpoint{0.817250in}{0.737375in}}%
\pgfpathlineto{\pgfqpoint{0.817548in}{0.737374in}}%
\pgfpathlineto{\pgfqpoint{0.817845in}{0.737372in}}%
\pgfpathlineto{\pgfqpoint{0.818143in}{0.737371in}}%
\pgfpathlineto{\pgfqpoint{0.818440in}{0.737370in}}%
\pgfpathlineto{\pgfqpoint{0.818738in}{0.737368in}}%
\pgfpathlineto{\pgfqpoint{0.819035in}{0.737367in}}%
\pgfpathlineto{\pgfqpoint{0.819333in}{0.737366in}}%
\pgfpathlineto{\pgfqpoint{0.819630in}{0.737364in}}%
\pgfpathlineto{\pgfqpoint{0.819928in}{0.737363in}}%
\pgfpathlineto{\pgfqpoint{0.820225in}{0.737362in}}%
\pgfpathlineto{\pgfqpoint{0.820523in}{0.737360in}}%
\pgfpathlineto{\pgfqpoint{0.820820in}{0.737359in}}%
\pgfpathlineto{\pgfqpoint{0.821118in}{0.737357in}}%
\pgfpathlineto{\pgfqpoint{0.821415in}{0.737356in}}%
\pgfpathlineto{\pgfqpoint{0.821713in}{0.737355in}}%
\pgfpathlineto{\pgfqpoint{0.822010in}{0.737353in}}%
\pgfpathlineto{\pgfqpoint{0.822308in}{0.737352in}}%
\pgfpathlineto{\pgfqpoint{0.822605in}{0.737351in}}%
\pgfpathlineto{\pgfqpoint{0.822903in}{0.737349in}}%
\pgfpathlineto{\pgfqpoint{0.823200in}{0.737348in}}%
\pgfpathlineto{\pgfqpoint{0.823498in}{0.737346in}}%
\pgfpathlineto{\pgfqpoint{0.823795in}{0.737345in}}%
\pgfpathlineto{\pgfqpoint{0.824092in}{0.737344in}}%
\pgfpathlineto{\pgfqpoint{0.824390in}{0.737342in}}%
\pgfpathlineto{\pgfqpoint{0.824687in}{0.737341in}}%
\pgfpathlineto{\pgfqpoint{0.824985in}{0.737340in}}%
\pgfpathlineto{\pgfqpoint{0.825282in}{0.737338in}}%
\pgfpathlineto{\pgfqpoint{0.825580in}{0.737337in}}%
\pgfpathlineto{\pgfqpoint{0.825877in}{0.737336in}}%
\pgfpathlineto{\pgfqpoint{0.826175in}{0.737334in}}%
\pgfpathlineto{\pgfqpoint{0.826472in}{0.737333in}}%
\pgfpathlineto{\pgfqpoint{0.826770in}{0.737331in}}%
\pgfpathlineto{\pgfqpoint{0.827067in}{0.737330in}}%
\pgfpathlineto{\pgfqpoint{0.827365in}{0.737329in}}%
\pgfpathlineto{\pgfqpoint{0.827662in}{0.737327in}}%
\pgfpathlineto{\pgfqpoint{0.827960in}{0.737326in}}%
\pgfpathlineto{\pgfqpoint{0.828257in}{0.737325in}}%
\pgfpathlineto{\pgfqpoint{0.828555in}{0.737323in}}%
\pgfpathlineto{\pgfqpoint{0.828852in}{0.737322in}}%
\pgfpathlineto{\pgfqpoint{0.829150in}{0.737321in}}%
\pgfpathlineto{\pgfqpoint{0.829447in}{0.737319in}}%
\pgfpathlineto{\pgfqpoint{0.829745in}{0.737318in}}%
\pgfpathlineto{\pgfqpoint{0.830042in}{0.737316in}}%
\pgfpathlineto{\pgfqpoint{0.830339in}{0.737315in}}%
\pgfpathlineto{\pgfqpoint{0.830637in}{0.737314in}}%
\pgfpathlineto{\pgfqpoint{0.830934in}{0.737312in}}%
\pgfpathlineto{\pgfqpoint{0.831232in}{0.737311in}}%
\pgfpathlineto{\pgfqpoint{0.831529in}{0.737310in}}%
\pgfpathlineto{\pgfqpoint{0.831827in}{0.737308in}}%
\pgfpathlineto{\pgfqpoint{0.832124in}{0.737307in}}%
\pgfpathlineto{\pgfqpoint{0.832422in}{0.737306in}}%
\pgfpathlineto{\pgfqpoint{0.832719in}{0.737304in}}%
\pgfpathlineto{\pgfqpoint{0.833017in}{0.737303in}}%
\pgfpathlineto{\pgfqpoint{0.833314in}{0.737301in}}%
\pgfpathlineto{\pgfqpoint{0.833612in}{0.737300in}}%
\pgfpathlineto{\pgfqpoint{0.833909in}{0.737299in}}%
\pgfpathlineto{\pgfqpoint{0.834207in}{0.737297in}}%
\pgfpathlineto{\pgfqpoint{0.834504in}{0.737296in}}%
\pgfpathlineto{\pgfqpoint{0.834802in}{0.737295in}}%
\pgfpathlineto{\pgfqpoint{0.835099in}{0.737293in}}%
\pgfpathlineto{\pgfqpoint{0.835397in}{0.737292in}}%
\pgfpathlineto{\pgfqpoint{0.835694in}{0.737290in}}%
\pgfpathlineto{\pgfqpoint{0.835992in}{0.737289in}}%
\pgfpathlineto{\pgfqpoint{0.836289in}{0.737289in}}%
\pgfpathlineto{\pgfqpoint{0.836587in}{0.737289in}}%
\pgfpathlineto{\pgfqpoint{0.836884in}{0.737289in}}%
\pgfpathlineto{\pgfqpoint{0.837181in}{0.737289in}}%
\pgfpathlineto{\pgfqpoint{0.837479in}{0.737288in}}%
\pgfpathlineto{\pgfqpoint{0.837776in}{0.737288in}}%
\pgfpathlineto{\pgfqpoint{0.838074in}{0.737288in}}%
\pgfpathlineto{\pgfqpoint{0.838371in}{0.737288in}}%
\pgfpathlineto{\pgfqpoint{0.838669in}{0.737288in}}%
\pgfpathlineto{\pgfqpoint{0.838966in}{0.737287in}}%
\pgfpathlineto{\pgfqpoint{0.839264in}{0.737287in}}%
\pgfpathlineto{\pgfqpoint{0.839561in}{0.737287in}}%
\pgfpathlineto{\pgfqpoint{0.839859in}{0.737287in}}%
\pgfpathlineto{\pgfqpoint{0.840156in}{0.737286in}}%
\pgfpathlineto{\pgfqpoint{0.840454in}{0.737286in}}%
\pgfpathlineto{\pgfqpoint{0.840751in}{0.737286in}}%
\pgfpathlineto{\pgfqpoint{0.841049in}{0.737286in}}%
\pgfpathlineto{\pgfqpoint{0.841346in}{0.737286in}}%
\pgfpathlineto{\pgfqpoint{0.841644in}{0.737285in}}%
\pgfpathlineto{\pgfqpoint{0.841941in}{0.737285in}}%
\pgfpathlineto{\pgfqpoint{0.842239in}{0.737285in}}%
\pgfpathlineto{\pgfqpoint{0.842536in}{0.737285in}}%
\pgfpathlineto{\pgfqpoint{0.842834in}{0.737285in}}%
\pgfpathlineto{\pgfqpoint{0.843131in}{0.737284in}}%
\pgfpathlineto{\pgfqpoint{0.843429in}{0.737284in}}%
\pgfpathlineto{\pgfqpoint{0.843726in}{0.737284in}}%
\pgfpathlineto{\pgfqpoint{0.844023in}{0.737284in}}%
\pgfpathlineto{\pgfqpoint{0.844321in}{0.737284in}}%
\pgfpathlineto{\pgfqpoint{0.844618in}{0.737283in}}%
\pgfpathlineto{\pgfqpoint{0.844916in}{0.737283in}}%
\pgfpathlineto{\pgfqpoint{0.845213in}{0.737283in}}%
\pgfpathlineto{\pgfqpoint{0.845511in}{0.737283in}}%
\pgfpathlineto{\pgfqpoint{0.845808in}{0.737282in}}%
\pgfpathlineto{\pgfqpoint{0.846106in}{0.737282in}}%
\pgfpathlineto{\pgfqpoint{0.846403in}{0.737282in}}%
\pgfpathlineto{\pgfqpoint{0.846701in}{0.737282in}}%
\pgfpathlineto{\pgfqpoint{0.846998in}{0.737282in}}%
\pgfpathlineto{\pgfqpoint{0.847296in}{0.737281in}}%
\pgfpathlineto{\pgfqpoint{0.847593in}{0.737281in}}%
\pgfpathlineto{\pgfqpoint{0.847891in}{0.737281in}}%
\pgfpathlineto{\pgfqpoint{0.848188in}{0.737281in}}%
\pgfpathlineto{\pgfqpoint{0.848486in}{0.737281in}}%
\pgfpathlineto{\pgfqpoint{0.848783in}{0.737280in}}%
\pgfpathlineto{\pgfqpoint{0.849081in}{0.737280in}}%
\pgfpathlineto{\pgfqpoint{0.849378in}{0.737280in}}%
\pgfpathlineto{\pgfqpoint{0.849676in}{0.737280in}}%
\pgfpathlineto{\pgfqpoint{0.849973in}{0.737279in}}%
\pgfpathlineto{\pgfqpoint{0.850270in}{0.737279in}}%
\pgfpathlineto{\pgfqpoint{0.850568in}{0.737279in}}%
\pgfpathlineto{\pgfqpoint{0.850865in}{0.737278in}}%
\pgfpathlineto{\pgfqpoint{0.851163in}{0.737277in}}%
\pgfpathlineto{\pgfqpoint{0.851460in}{0.737276in}}%
\pgfpathlineto{\pgfqpoint{0.851758in}{0.737274in}}%
\pgfpathlineto{\pgfqpoint{0.852055in}{0.737274in}}%
\pgfpathlineto{\pgfqpoint{0.852353in}{0.737308in}}%
\pgfpathlineto{\pgfqpoint{0.852650in}{0.737359in}}%
\pgfpathlineto{\pgfqpoint{0.852948in}{0.737409in}}%
\pgfpathlineto{\pgfqpoint{0.853245in}{0.737460in}}%
\pgfpathlineto{\pgfqpoint{0.853543in}{0.737511in}}%
\pgfpathlineto{\pgfqpoint{0.853840in}{0.737561in}}%
\pgfpathlineto{\pgfqpoint{0.854138in}{0.737612in}}%
\pgfpathlineto{\pgfqpoint{0.854435in}{0.737662in}}%
\pgfpathlineto{\pgfqpoint{0.854733in}{0.737713in}}%
\pgfpathlineto{\pgfqpoint{0.855030in}{0.737763in}}%
\pgfpathlineto{\pgfqpoint{0.855328in}{0.737814in}}%
\pgfpathlineto{\pgfqpoint{0.855625in}{0.737864in}}%
\pgfpathlineto{\pgfqpoint{0.855923in}{0.737915in}}%
\pgfpathlineto{\pgfqpoint{0.856220in}{0.737965in}}%
\pgfpathlineto{\pgfqpoint{0.856518in}{0.738016in}}%
\pgfpathlineto{\pgfqpoint{0.856815in}{0.738066in}}%
\pgfpathlineto{\pgfqpoint{0.857112in}{0.738117in}}%
\pgfpathlineto{\pgfqpoint{0.857410in}{0.738157in}}%
\pgfpathlineto{\pgfqpoint{0.857707in}{0.738159in}}%
\pgfpathlineto{\pgfqpoint{0.858005in}{0.738157in}}%
\pgfpathlineto{\pgfqpoint{0.858302in}{0.738156in}}%
\pgfpathlineto{\pgfqpoint{0.858600in}{0.738154in}}%
\pgfpathlineto{\pgfqpoint{0.858897in}{0.738153in}}%
\pgfpathlineto{\pgfqpoint{0.859195in}{0.738152in}}%
\pgfpathlineto{\pgfqpoint{0.859492in}{0.738150in}}%
\pgfpathlineto{\pgfqpoint{0.859790in}{0.738149in}}%
\pgfpathlineto{\pgfqpoint{0.860087in}{0.738148in}}%
\pgfpathlineto{\pgfqpoint{0.860385in}{0.738146in}}%
\pgfpathlineto{\pgfqpoint{0.860682in}{0.738145in}}%
\pgfpathlineto{\pgfqpoint{0.860980in}{0.738144in}}%
\pgfpathlineto{\pgfqpoint{0.861277in}{0.738142in}}%
\pgfpathlineto{\pgfqpoint{0.861575in}{0.738141in}}%
\pgfpathlineto{\pgfqpoint{0.861872in}{0.738139in}}%
\pgfpathlineto{\pgfqpoint{0.862170in}{0.738138in}}%
\pgfpathlineto{\pgfqpoint{0.862467in}{0.738137in}}%
\pgfpathlineto{\pgfqpoint{0.862765in}{0.738135in}}%
\pgfpathlineto{\pgfqpoint{0.863062in}{0.738134in}}%
\pgfpathlineto{\pgfqpoint{0.863360in}{0.738133in}}%
\pgfpathlineto{\pgfqpoint{0.863657in}{0.738131in}}%
\pgfpathlineto{\pgfqpoint{0.863954in}{0.738130in}}%
\pgfpathlineto{\pgfqpoint{0.864252in}{0.738128in}}%
\pgfpathlineto{\pgfqpoint{0.864549in}{0.738127in}}%
\pgfpathlineto{\pgfqpoint{0.864847in}{0.738126in}}%
\pgfpathlineto{\pgfqpoint{0.865144in}{0.738124in}}%
\pgfpathlineto{\pgfqpoint{0.865442in}{0.738123in}}%
\pgfpathlineto{\pgfqpoint{0.865739in}{0.738122in}}%
\pgfpathlineto{\pgfqpoint{0.866037in}{0.738120in}}%
\pgfpathlineto{\pgfqpoint{0.866334in}{0.738119in}}%
\pgfpathlineto{\pgfqpoint{0.866632in}{0.738118in}}%
\pgfpathlineto{\pgfqpoint{0.866929in}{0.738116in}}%
\pgfpathlineto{\pgfqpoint{0.867227in}{0.738115in}}%
\pgfpathlineto{\pgfqpoint{0.867524in}{0.738113in}}%
\pgfpathlineto{\pgfqpoint{0.867822in}{0.738112in}}%
\pgfpathlineto{\pgfqpoint{0.868119in}{0.738111in}}%
\pgfpathlineto{\pgfqpoint{0.868417in}{0.738109in}}%
\pgfpathlineto{\pgfqpoint{0.868714in}{0.738108in}}%
\pgfpathlineto{\pgfqpoint{0.869012in}{0.738107in}}%
\pgfpathlineto{\pgfqpoint{0.869309in}{0.738105in}}%
\pgfpathlineto{\pgfqpoint{0.869607in}{0.738104in}}%
\pgfpathlineto{\pgfqpoint{0.869904in}{0.738103in}}%
\pgfpathlineto{\pgfqpoint{0.870201in}{0.738101in}}%
\pgfpathlineto{\pgfqpoint{0.870499in}{0.738100in}}%
\pgfpathlineto{\pgfqpoint{0.870796in}{0.738098in}}%
\pgfpathlineto{\pgfqpoint{0.871094in}{0.738097in}}%
\pgfpathlineto{\pgfqpoint{0.871391in}{0.738096in}}%
\pgfpathlineto{\pgfqpoint{0.871689in}{0.738094in}}%
\pgfpathlineto{\pgfqpoint{0.871986in}{0.738093in}}%
\pgfpathlineto{\pgfqpoint{0.872284in}{0.738092in}}%
\pgfpathlineto{\pgfqpoint{0.872581in}{0.738090in}}%
\pgfpathlineto{\pgfqpoint{0.872879in}{0.738089in}}%
\pgfpathlineto{\pgfqpoint{0.873176in}{0.738088in}}%
\pgfpathlineto{\pgfqpoint{0.873474in}{0.738086in}}%
\pgfpathlineto{\pgfqpoint{0.873771in}{0.738085in}}%
\pgfpathlineto{\pgfqpoint{0.874069in}{0.738083in}}%
\pgfpathlineto{\pgfqpoint{0.874366in}{0.738082in}}%
\pgfpathlineto{\pgfqpoint{0.874664in}{0.738081in}}%
\pgfpathlineto{\pgfqpoint{0.874961in}{0.738079in}}%
\pgfpathlineto{\pgfqpoint{0.875259in}{0.738078in}}%
\pgfpathlineto{\pgfqpoint{0.875556in}{0.738077in}}%
\pgfpathlineto{\pgfqpoint{0.875854in}{0.738075in}}%
\pgfpathlineto{\pgfqpoint{0.876151in}{0.738074in}}%
\pgfpathlineto{\pgfqpoint{0.876449in}{0.738072in}}%
\pgfpathlineto{\pgfqpoint{0.876746in}{0.738071in}}%
\pgfpathlineto{\pgfqpoint{0.877043in}{0.738070in}}%
\pgfpathlineto{\pgfqpoint{0.877341in}{0.738068in}}%
\pgfpathlineto{\pgfqpoint{0.877638in}{0.738067in}}%
\pgfpathlineto{\pgfqpoint{0.877936in}{0.738066in}}%
\pgfpathlineto{\pgfqpoint{0.878233in}{0.738064in}}%
\pgfpathlineto{\pgfqpoint{0.878531in}{0.738063in}}%
\pgfpathlineto{\pgfqpoint{0.878828in}{0.738062in}}%
\pgfpathlineto{\pgfqpoint{0.879126in}{0.738060in}}%
\pgfpathlineto{\pgfqpoint{0.879423in}{0.738059in}}%
\pgfpathlineto{\pgfqpoint{0.879721in}{0.738057in}}%
\pgfpathlineto{\pgfqpoint{0.880018in}{0.738056in}}%
\pgfpathlineto{\pgfqpoint{0.880316in}{0.738055in}}%
\pgfpathlineto{\pgfqpoint{0.880613in}{0.738053in}}%
\pgfpathlineto{\pgfqpoint{0.880911in}{0.738052in}}%
\pgfpathlineto{\pgfqpoint{0.881208in}{0.738051in}}%
\pgfpathlineto{\pgfqpoint{0.881506in}{0.738049in}}%
\pgfpathlineto{\pgfqpoint{0.881803in}{0.738048in}}%
\pgfpathlineto{\pgfqpoint{0.882101in}{0.738047in}}%
\pgfpathlineto{\pgfqpoint{0.882398in}{0.738045in}}%
\pgfpathlineto{\pgfqpoint{0.882696in}{0.738044in}}%
\pgfpathlineto{\pgfqpoint{0.882993in}{0.738042in}}%
\pgfpathlineto{\pgfqpoint{0.883291in}{0.738041in}}%
\pgfpathlineto{\pgfqpoint{0.883588in}{0.738040in}}%
\pgfpathlineto{\pgfqpoint{0.883885in}{0.738038in}}%
\pgfpathlineto{\pgfqpoint{0.884183in}{0.738037in}}%
\pgfpathlineto{\pgfqpoint{0.884480in}{0.738036in}}%
\pgfpathlineto{\pgfqpoint{0.884778in}{0.738034in}}%
\pgfpathlineto{\pgfqpoint{0.885075in}{0.738033in}}%
\pgfpathlineto{\pgfqpoint{0.885373in}{0.738032in}}%
\pgfpathlineto{\pgfqpoint{0.885670in}{0.738030in}}%
\pgfpathlineto{\pgfqpoint{0.885968in}{0.738029in}}%
\pgfpathlineto{\pgfqpoint{0.886265in}{0.738027in}}%
\pgfpathlineto{\pgfqpoint{0.886563in}{0.738026in}}%
\pgfpathlineto{\pgfqpoint{0.886860in}{0.738025in}}%
\pgfpathlineto{\pgfqpoint{0.887158in}{0.738023in}}%
\pgfpathlineto{\pgfqpoint{0.887455in}{0.738022in}}%
\pgfpathlineto{\pgfqpoint{0.887753in}{0.738021in}}%
\pgfpathlineto{\pgfqpoint{0.888050in}{0.738019in}}%
\pgfpathlineto{\pgfqpoint{0.888348in}{0.738018in}}%
\pgfpathlineto{\pgfqpoint{0.888645in}{0.738016in}}%
\pgfpathlineto{\pgfqpoint{0.888943in}{0.738015in}}%
\pgfpathlineto{\pgfqpoint{0.889240in}{0.738014in}}%
\pgfpathlineto{\pgfqpoint{0.889538in}{0.738012in}}%
\pgfpathlineto{\pgfqpoint{0.889835in}{0.738011in}}%
\pgfpathlineto{\pgfqpoint{0.890132in}{0.738010in}}%
\pgfpathlineto{\pgfqpoint{0.890430in}{0.738008in}}%
\pgfpathlineto{\pgfqpoint{0.890727in}{0.738007in}}%
\pgfpathlineto{\pgfqpoint{0.891025in}{0.738006in}}%
\pgfpathlineto{\pgfqpoint{0.891322in}{0.738004in}}%
\pgfpathlineto{\pgfqpoint{0.891620in}{0.738003in}}%
\pgfpathlineto{\pgfqpoint{0.891917in}{0.738001in}}%
\pgfpathlineto{\pgfqpoint{0.892215in}{0.738000in}}%
\pgfpathlineto{\pgfqpoint{0.892512in}{0.737999in}}%
\pgfpathlineto{\pgfqpoint{0.892810in}{0.737997in}}%
\pgfpathlineto{\pgfqpoint{0.893107in}{0.737989in}}%
\pgfpathlineto{\pgfqpoint{0.893405in}{0.737932in}}%
\pgfpathlineto{\pgfqpoint{0.893702in}{0.738425in}}%
\pgfpathlineto{\pgfqpoint{0.894000in}{0.738389in}}%
\pgfpathlineto{\pgfqpoint{0.894297in}{0.738585in}}%
\pgfpathlineto{\pgfqpoint{0.894595in}{0.738356in}}%
\pgfpathlineto{\pgfqpoint{0.894892in}{0.738327in}}%
\pgfpathlineto{\pgfqpoint{0.895190in}{0.738326in}}%
\pgfpathlineto{\pgfqpoint{0.895487in}{0.738325in}}%
\pgfpathlineto{\pgfqpoint{0.895785in}{0.738325in}}%
\pgfpathlineto{\pgfqpoint{0.896082in}{0.738324in}}%
\pgfpathlineto{\pgfqpoint{0.896380in}{0.738323in}}%
\pgfpathlineto{\pgfqpoint{0.896677in}{0.738322in}}%
\pgfpathlineto{\pgfqpoint{0.896974in}{0.738322in}}%
\pgfpathlineto{\pgfqpoint{0.897272in}{0.738321in}}%
\pgfpathlineto{\pgfqpoint{0.897569in}{0.738320in}}%
\pgfpathlineto{\pgfqpoint{0.897867in}{0.738319in}}%
\pgfpathlineto{\pgfqpoint{0.898164in}{0.738319in}}%
\pgfpathlineto{\pgfqpoint{0.898462in}{0.738318in}}%
\pgfpathlineto{\pgfqpoint{0.898759in}{0.738317in}}%
\pgfpathlineto{\pgfqpoint{0.899057in}{0.738316in}}%
\pgfpathlineto{\pgfqpoint{0.899354in}{0.738316in}}%
\pgfpathlineto{\pgfqpoint{0.899652in}{0.738315in}}%
\pgfpathlineto{\pgfqpoint{0.899949in}{0.738314in}}%
\pgfpathlineto{\pgfqpoint{0.900247in}{0.738313in}}%
\pgfpathlineto{\pgfqpoint{0.900544in}{0.738313in}}%
\pgfpathlineto{\pgfqpoint{0.900842in}{0.738312in}}%
\pgfpathlineto{\pgfqpoint{0.901139in}{0.738311in}}%
\pgfpathlineto{\pgfqpoint{0.901437in}{0.738310in}}%
\pgfpathlineto{\pgfqpoint{0.901734in}{0.738309in}}%
\pgfpathlineto{\pgfqpoint{0.902032in}{0.738309in}}%
\pgfpathlineto{\pgfqpoint{0.902329in}{0.738308in}}%
\pgfpathlineto{\pgfqpoint{0.902627in}{0.738307in}}%
\pgfpathlineto{\pgfqpoint{0.902924in}{0.738306in}}%
\pgfpathlineto{\pgfqpoint{0.903222in}{0.738306in}}%
\pgfpathlineto{\pgfqpoint{0.903519in}{0.738305in}}%
\pgfpathlineto{\pgfqpoint{0.903816in}{0.738304in}}%
\pgfpathlineto{\pgfqpoint{0.904114in}{0.738303in}}%
\pgfpathlineto{\pgfqpoint{0.904411in}{0.738303in}}%
\pgfpathlineto{\pgfqpoint{0.904709in}{0.738302in}}%
\pgfpathlineto{\pgfqpoint{0.905006in}{0.738301in}}%
\pgfpathlineto{\pgfqpoint{0.905304in}{0.738300in}}%
\pgfpathlineto{\pgfqpoint{0.905601in}{0.738300in}}%
\pgfpathlineto{\pgfqpoint{0.905899in}{0.738299in}}%
\pgfpathlineto{\pgfqpoint{0.906196in}{0.738298in}}%
\pgfpathlineto{\pgfqpoint{0.906494in}{0.738297in}}%
\pgfpathlineto{\pgfqpoint{0.906791in}{0.738297in}}%
\pgfpathlineto{\pgfqpoint{0.907089in}{0.738296in}}%
\pgfpathlineto{\pgfqpoint{0.907386in}{0.738295in}}%
\pgfpathlineto{\pgfqpoint{0.907684in}{0.738294in}}%
\pgfpathlineto{\pgfqpoint{0.907981in}{0.738293in}}%
\pgfpathlineto{\pgfqpoint{0.908279in}{0.738293in}}%
\pgfpathlineto{\pgfqpoint{0.908576in}{0.738292in}}%
\pgfpathlineto{\pgfqpoint{0.908874in}{0.738291in}}%
\pgfpathlineto{\pgfqpoint{0.909171in}{0.738290in}}%
\pgfpathlineto{\pgfqpoint{0.909469in}{0.738290in}}%
\pgfpathlineto{\pgfqpoint{0.909766in}{0.738289in}}%
\pgfpathlineto{\pgfqpoint{0.910063in}{0.738288in}}%
\pgfpathlineto{\pgfqpoint{0.910361in}{0.738287in}}%
\pgfpathlineto{\pgfqpoint{0.910658in}{0.738287in}}%
\pgfpathlineto{\pgfqpoint{0.910956in}{0.738286in}}%
\pgfpathlineto{\pgfqpoint{0.911253in}{0.738285in}}%
\pgfpathlineto{\pgfqpoint{0.911551in}{0.738284in}}%
\pgfpathlineto{\pgfqpoint{0.911848in}{0.738284in}}%
\pgfpathlineto{\pgfqpoint{0.912146in}{0.738283in}}%
\pgfpathlineto{\pgfqpoint{0.912443in}{0.738282in}}%
\pgfpathlineto{\pgfqpoint{0.912741in}{0.738281in}}%
\pgfpathlineto{\pgfqpoint{0.913038in}{0.738281in}}%
\pgfpathlineto{\pgfqpoint{0.913336in}{0.738280in}}%
\pgfpathlineto{\pgfqpoint{0.913633in}{0.738279in}}%
\pgfpathlineto{\pgfqpoint{0.913931in}{0.738278in}}%
\pgfpathlineto{\pgfqpoint{0.914228in}{0.738277in}}%
\pgfpathlineto{\pgfqpoint{0.914526in}{0.738277in}}%
\pgfpathlineto{\pgfqpoint{0.914823in}{0.738276in}}%
\pgfpathlineto{\pgfqpoint{0.915121in}{0.738275in}}%
\pgfpathlineto{\pgfqpoint{0.915418in}{0.738274in}}%
\pgfpathlineto{\pgfqpoint{0.915716in}{0.738274in}}%
\pgfpathlineto{\pgfqpoint{0.916013in}{0.738273in}}%
\pgfpathlineto{\pgfqpoint{0.916311in}{0.738272in}}%
\pgfpathlineto{\pgfqpoint{0.916608in}{0.738271in}}%
\pgfpathlineto{\pgfqpoint{0.916905in}{0.738271in}}%
\pgfpathlineto{\pgfqpoint{0.917203in}{0.738270in}}%
\pgfpathlineto{\pgfqpoint{0.917500in}{0.738269in}}%
\pgfpathlineto{\pgfqpoint{0.917798in}{0.738268in}}%
\pgfpathlineto{\pgfqpoint{0.918095in}{0.738268in}}%
\pgfpathlineto{\pgfqpoint{0.918393in}{0.738267in}}%
\pgfpathlineto{\pgfqpoint{0.918690in}{0.738266in}}%
\pgfpathlineto{\pgfqpoint{0.918988in}{0.738265in}}%
\pgfpathlineto{\pgfqpoint{0.919285in}{0.738264in}}%
\pgfpathlineto{\pgfqpoint{0.919583in}{0.738264in}}%
\pgfpathlineto{\pgfqpoint{0.919880in}{0.738263in}}%
\pgfpathlineto{\pgfqpoint{0.920178in}{0.738262in}}%
\pgfpathlineto{\pgfqpoint{0.920475in}{0.738261in}}%
\pgfpathlineto{\pgfqpoint{0.920773in}{0.738261in}}%
\pgfpathlineto{\pgfqpoint{0.921070in}{0.738260in}}%
\pgfpathlineto{\pgfqpoint{0.921368in}{0.738259in}}%
\pgfpathlineto{\pgfqpoint{0.921665in}{0.738258in}}%
\pgfpathlineto{\pgfqpoint{0.921963in}{0.738258in}}%
\pgfpathlineto{\pgfqpoint{0.922260in}{0.738257in}}%
\pgfpathlineto{\pgfqpoint{0.922558in}{0.738256in}}%
\pgfpathlineto{\pgfqpoint{0.922855in}{0.738255in}}%
\pgfpathlineto{\pgfqpoint{0.923153in}{0.738255in}}%
\pgfpathlineto{\pgfqpoint{0.923450in}{0.738254in}}%
\pgfpathlineto{\pgfqpoint{0.923747in}{0.738253in}}%
\pgfpathlineto{\pgfqpoint{0.924045in}{0.738252in}}%
\pgfpathlineto{\pgfqpoint{0.924342in}{0.738252in}}%
\pgfpathlineto{\pgfqpoint{0.924640in}{0.738251in}}%
\pgfpathlineto{\pgfqpoint{0.924937in}{0.738250in}}%
\pgfpathlineto{\pgfqpoint{0.925235in}{0.738249in}}%
\pgfpathlineto{\pgfqpoint{0.925532in}{0.738248in}}%
\pgfpathlineto{\pgfqpoint{0.925830in}{0.738248in}}%
\pgfpathlineto{\pgfqpoint{0.926127in}{0.738247in}}%
\pgfpathlineto{\pgfqpoint{0.926425in}{0.738246in}}%
\pgfpathlineto{\pgfqpoint{0.926722in}{0.738245in}}%
\pgfpathlineto{\pgfqpoint{0.927020in}{0.738245in}}%
\pgfpathlineto{\pgfqpoint{0.927317in}{0.738244in}}%
\pgfpathlineto{\pgfqpoint{0.927615in}{0.738243in}}%
\pgfpathlineto{\pgfqpoint{0.927912in}{0.738242in}}%
\pgfpathlineto{\pgfqpoint{0.928210in}{0.738242in}}%
\pgfpathlineto{\pgfqpoint{0.928507in}{0.738241in}}%
\pgfpathlineto{\pgfqpoint{0.928805in}{0.738240in}}%
\pgfpathlineto{\pgfqpoint{0.929102in}{0.738239in}}%
\pgfpathlineto{\pgfqpoint{0.929400in}{0.738239in}}%
\pgfpathlineto{\pgfqpoint{0.929697in}{0.738238in}}%
\pgfpathlineto{\pgfqpoint{0.929994in}{0.738237in}}%
\pgfpathlineto{\pgfqpoint{0.930292in}{0.738236in}}%
\pgfpathlineto{\pgfqpoint{0.930589in}{0.738236in}}%
\pgfpathlineto{\pgfqpoint{0.930887in}{0.738235in}}%
\pgfpathlineto{\pgfqpoint{0.931184in}{0.738234in}}%
\pgfpathlineto{\pgfqpoint{0.931482in}{0.738233in}}%
\pgfpathlineto{\pgfqpoint{0.931779in}{0.738232in}}%
\pgfpathlineto{\pgfqpoint{0.932077in}{0.738232in}}%
\pgfpathlineto{\pgfqpoint{0.932374in}{0.738231in}}%
\pgfpathlineto{\pgfqpoint{0.932672in}{0.738230in}}%
\pgfpathlineto{\pgfqpoint{0.932969in}{0.738229in}}%
\pgfpathlineto{\pgfqpoint{0.933267in}{0.738229in}}%
\pgfpathlineto{\pgfqpoint{0.933564in}{0.738228in}}%
\pgfpathlineto{\pgfqpoint{0.933862in}{0.738227in}}%
\pgfpathlineto{\pgfqpoint{0.934159in}{0.738226in}}%
\pgfpathlineto{\pgfqpoint{0.934457in}{0.738226in}}%
\pgfpathlineto{\pgfqpoint{0.934754in}{0.738225in}}%
\pgfpathlineto{\pgfqpoint{0.935052in}{0.738224in}}%
\pgfpathlineto{\pgfqpoint{0.935349in}{0.738223in}}%
\pgfpathlineto{\pgfqpoint{0.935647in}{0.738223in}}%
\pgfpathlineto{\pgfqpoint{0.935944in}{0.738222in}}%
\pgfpathlineto{\pgfqpoint{0.936242in}{0.738221in}}%
\pgfpathlineto{\pgfqpoint{0.936539in}{0.738220in}}%
\pgfpathlineto{\pgfqpoint{0.936836in}{0.738220in}}%
\pgfpathlineto{\pgfqpoint{0.937134in}{0.738219in}}%
\pgfpathlineto{\pgfqpoint{0.937431in}{0.738218in}}%
\pgfpathlineto{\pgfqpoint{0.937729in}{0.738217in}}%
\pgfpathlineto{\pgfqpoint{0.938026in}{0.738216in}}%
\pgfpathlineto{\pgfqpoint{0.938324in}{0.738216in}}%
\pgfpathlineto{\pgfqpoint{0.938621in}{0.738215in}}%
\pgfpathlineto{\pgfqpoint{0.938919in}{0.738214in}}%
\pgfpathlineto{\pgfqpoint{0.939216in}{0.738213in}}%
\pgfpathlineto{\pgfqpoint{0.939514in}{0.738213in}}%
\pgfpathlineto{\pgfqpoint{0.939811in}{0.738212in}}%
\pgfpathlineto{\pgfqpoint{0.940109in}{0.738211in}}%
\pgfpathlineto{\pgfqpoint{0.940406in}{0.738210in}}%
\pgfpathlineto{\pgfqpoint{0.940704in}{0.738210in}}%
\pgfpathlineto{\pgfqpoint{0.941001in}{0.738209in}}%
\pgfpathlineto{\pgfqpoint{0.941299in}{0.738208in}}%
\pgfpathlineto{\pgfqpoint{0.941596in}{0.738207in}}%
\pgfpathlineto{\pgfqpoint{0.941894in}{0.738207in}}%
\pgfpathlineto{\pgfqpoint{0.942191in}{0.738206in}}%
\pgfpathlineto{\pgfqpoint{0.942489in}{0.738205in}}%
\pgfpathlineto{\pgfqpoint{0.942786in}{0.738204in}}%
\pgfpathlineto{\pgfqpoint{0.943084in}{0.738204in}}%
\pgfpathlineto{\pgfqpoint{0.943381in}{0.738203in}}%
\pgfpathlineto{\pgfqpoint{0.943678in}{0.738202in}}%
\pgfpathlineto{\pgfqpoint{0.943976in}{0.738201in}}%
\pgfpathlineto{\pgfqpoint{0.944273in}{0.738200in}}%
\pgfpathlineto{\pgfqpoint{0.944571in}{0.738200in}}%
\pgfpathlineto{\pgfqpoint{0.944868in}{0.738199in}}%
\pgfpathlineto{\pgfqpoint{0.945166in}{0.738198in}}%
\pgfpathlineto{\pgfqpoint{0.945463in}{0.738197in}}%
\pgfpathlineto{\pgfqpoint{0.945761in}{0.738197in}}%
\pgfpathlineto{\pgfqpoint{0.946058in}{0.738196in}}%
\pgfpathlineto{\pgfqpoint{0.946356in}{0.738195in}}%
\pgfpathlineto{\pgfqpoint{0.946653in}{0.738194in}}%
\pgfpathlineto{\pgfqpoint{0.946951in}{0.738194in}}%
\pgfpathlineto{\pgfqpoint{0.947248in}{0.738193in}}%
\pgfpathlineto{\pgfqpoint{0.947546in}{0.738192in}}%
\pgfpathlineto{\pgfqpoint{0.947843in}{0.738191in}}%
\pgfpathlineto{\pgfqpoint{0.948141in}{0.738191in}}%
\pgfpathlineto{\pgfqpoint{0.948438in}{0.738190in}}%
\pgfpathlineto{\pgfqpoint{0.948736in}{0.738189in}}%
\pgfpathlineto{\pgfqpoint{0.949033in}{0.738188in}}%
\pgfpathlineto{\pgfqpoint{0.949331in}{0.738188in}}%
\pgfpathlineto{\pgfqpoint{0.949628in}{0.738234in}}%
\pgfpathlineto{\pgfqpoint{0.949926in}{0.738593in}}%
\pgfpathlineto{\pgfqpoint{0.950223in}{0.738860in}}%
\pgfpathlineto{\pgfqpoint{0.950520in}{0.738941in}}%
\pgfpathlineto{\pgfqpoint{0.950818in}{0.738940in}}%
\pgfpathlineto{\pgfqpoint{0.951115in}{0.738939in}}%
\pgfpathlineto{\pgfqpoint{0.951413in}{0.738939in}}%
\pgfpathlineto{\pgfqpoint{0.951710in}{0.738938in}}%
\pgfpathlineto{\pgfqpoint{0.952008in}{0.738937in}}%
\pgfpathlineto{\pgfqpoint{0.952305in}{0.738936in}}%
\pgfpathlineto{\pgfqpoint{0.952603in}{0.738936in}}%
\pgfpathlineto{\pgfqpoint{0.952900in}{0.738935in}}%
\pgfpathlineto{\pgfqpoint{0.953198in}{0.738934in}}%
\pgfpathlineto{\pgfqpoint{0.953495in}{0.738933in}}%
\pgfpathlineto{\pgfqpoint{0.953793in}{0.738933in}}%
\pgfpathlineto{\pgfqpoint{0.954090in}{0.738932in}}%
\pgfpathlineto{\pgfqpoint{0.954388in}{0.738931in}}%
\pgfpathlineto{\pgfqpoint{0.954685in}{0.738930in}}%
\pgfpathlineto{\pgfqpoint{0.954983in}{0.738930in}}%
\pgfpathlineto{\pgfqpoint{0.955280in}{0.738929in}}%
\pgfpathlineto{\pgfqpoint{0.955578in}{0.738928in}}%
\pgfpathlineto{\pgfqpoint{0.955875in}{0.738927in}}%
\pgfpathlineto{\pgfqpoint{0.956173in}{0.738926in}}%
\pgfpathlineto{\pgfqpoint{0.956470in}{0.738926in}}%
\pgfpathlineto{\pgfqpoint{0.956767in}{0.738925in}}%
\pgfpathlineto{\pgfqpoint{0.957065in}{0.738924in}}%
\pgfpathlineto{\pgfqpoint{0.957362in}{0.738923in}}%
\pgfpathlineto{\pgfqpoint{0.957660in}{0.738923in}}%
\pgfpathlineto{\pgfqpoint{0.957957in}{0.738922in}}%
\pgfpathlineto{\pgfqpoint{0.958255in}{0.738921in}}%
\pgfpathlineto{\pgfqpoint{0.958552in}{0.738920in}}%
\pgfpathlineto{\pgfqpoint{0.958850in}{0.738920in}}%
\pgfpathlineto{\pgfqpoint{0.959147in}{0.738919in}}%
\pgfpathlineto{\pgfqpoint{0.959445in}{0.738918in}}%
\pgfpathlineto{\pgfqpoint{0.959742in}{0.738917in}}%
\pgfpathlineto{\pgfqpoint{0.960040in}{0.738917in}}%
\pgfpathlineto{\pgfqpoint{0.960337in}{0.738916in}}%
\pgfpathlineto{\pgfqpoint{0.960635in}{0.738915in}}%
\pgfpathlineto{\pgfqpoint{0.960932in}{0.738914in}}%
\pgfpathlineto{\pgfqpoint{0.961230in}{0.738914in}}%
\pgfpathlineto{\pgfqpoint{0.961527in}{0.738913in}}%
\pgfpathlineto{\pgfqpoint{0.961825in}{0.738912in}}%
\pgfpathlineto{\pgfqpoint{0.962122in}{0.738911in}}%
\pgfpathlineto{\pgfqpoint{0.962420in}{0.738910in}}%
\pgfpathlineto{\pgfqpoint{0.962717in}{0.738910in}}%
\pgfpathlineto{\pgfqpoint{0.963015in}{0.738910in}}%
\pgfpathlineto{\pgfqpoint{0.963312in}{0.738950in}}%
\pgfpathlineto{\pgfqpoint{0.963609in}{0.739007in}}%
\pgfpathlineto{\pgfqpoint{0.963907in}{0.739050in}}%
\pgfpathlineto{\pgfqpoint{0.964204in}{0.739053in}}%
\pgfpathlineto{\pgfqpoint{0.964502in}{0.739053in}}%
\pgfpathlineto{\pgfqpoint{0.964799in}{0.738800in}}%
\pgfpathlineto{\pgfqpoint{0.965097in}{0.733838in}}%
\pgfpathlineto{\pgfqpoint{0.965394in}{0.733782in}}%
\pgfpathlineto{\pgfqpoint{0.965692in}{0.733784in}}%
\pgfpathlineto{\pgfqpoint{0.965989in}{0.733783in}}%
\pgfpathlineto{\pgfqpoint{0.966287in}{0.733782in}}%
\pgfpathlineto{\pgfqpoint{0.966584in}{0.733782in}}%
\pgfpathlineto{\pgfqpoint{0.966882in}{0.733781in}}%
\pgfpathlineto{\pgfqpoint{0.967179in}{0.733780in}}%
\pgfpathlineto{\pgfqpoint{0.967477in}{0.733779in}}%
\pgfpathlineto{\pgfqpoint{0.967774in}{0.733778in}}%
\pgfpathlineto{\pgfqpoint{0.968072in}{0.733778in}}%
\pgfpathlineto{\pgfqpoint{0.968369in}{0.733777in}}%
\pgfpathlineto{\pgfqpoint{0.968667in}{0.733776in}}%
\pgfpathlineto{\pgfqpoint{0.968964in}{0.733775in}}%
\pgfpathlineto{\pgfqpoint{0.969262in}{0.733775in}}%
\pgfpathlineto{\pgfqpoint{0.969559in}{0.733774in}}%
\pgfpathlineto{\pgfqpoint{0.969857in}{0.733773in}}%
\pgfpathlineto{\pgfqpoint{0.970154in}{0.733772in}}%
\pgfpathlineto{\pgfqpoint{0.970451in}{0.733772in}}%
\pgfpathlineto{\pgfqpoint{0.970749in}{0.733771in}}%
\pgfpathlineto{\pgfqpoint{0.971046in}{0.733770in}}%
\pgfpathlineto{\pgfqpoint{0.971344in}{0.733769in}}%
\pgfpathlineto{\pgfqpoint{0.971641in}{0.733769in}}%
\pgfpathlineto{\pgfqpoint{0.971939in}{0.733768in}}%
\pgfpathlineto{\pgfqpoint{0.972236in}{0.733767in}}%
\pgfpathlineto{\pgfqpoint{0.972534in}{0.733766in}}%
\pgfpathlineto{\pgfqpoint{0.972831in}{0.733765in}}%
\pgfpathlineto{\pgfqpoint{0.973129in}{0.733765in}}%
\pgfpathlineto{\pgfqpoint{0.973426in}{0.733764in}}%
\pgfpathlineto{\pgfqpoint{0.973724in}{0.733763in}}%
\pgfpathlineto{\pgfqpoint{0.974021in}{0.733762in}}%
\pgfpathlineto{\pgfqpoint{0.974319in}{0.733762in}}%
\pgfpathlineto{\pgfqpoint{0.974616in}{0.733761in}}%
\pgfpathlineto{\pgfqpoint{0.974914in}{0.733760in}}%
\pgfpathlineto{\pgfqpoint{0.975211in}{0.733759in}}%
\pgfpathlineto{\pgfqpoint{0.975509in}{0.733759in}}%
\pgfpathlineto{\pgfqpoint{0.975806in}{0.733758in}}%
\pgfpathlineto{\pgfqpoint{0.976104in}{0.733757in}}%
\pgfpathlineto{\pgfqpoint{0.976401in}{0.733756in}}%
\pgfpathlineto{\pgfqpoint{0.976698in}{0.733756in}}%
\pgfpathlineto{\pgfqpoint{0.976996in}{0.733755in}}%
\pgfpathlineto{\pgfqpoint{0.977293in}{0.733754in}}%
\pgfpathlineto{\pgfqpoint{0.977591in}{0.733753in}}%
\pgfpathlineto{\pgfqpoint{0.977888in}{0.733753in}}%
\pgfpathlineto{\pgfqpoint{0.978186in}{0.733752in}}%
\pgfpathlineto{\pgfqpoint{0.978483in}{0.733751in}}%
\pgfpathlineto{\pgfqpoint{0.978781in}{0.733750in}}%
\pgfpathlineto{\pgfqpoint{0.979078in}{0.733749in}}%
\pgfpathlineto{\pgfqpoint{0.979376in}{0.733749in}}%
\pgfpathlineto{\pgfqpoint{0.979673in}{0.733748in}}%
\pgfpathlineto{\pgfqpoint{0.979971in}{0.733747in}}%
\pgfpathlineto{\pgfqpoint{0.980268in}{0.733746in}}%
\pgfpathlineto{\pgfqpoint{0.980566in}{0.733746in}}%
\pgfpathlineto{\pgfqpoint{0.980863in}{0.733745in}}%
\pgfpathlineto{\pgfqpoint{0.981161in}{0.733744in}}%
\pgfpathlineto{\pgfqpoint{0.981458in}{0.733743in}}%
\pgfpathlineto{\pgfqpoint{0.981756in}{0.733743in}}%
\pgfpathlineto{\pgfqpoint{0.982053in}{0.733742in}}%
\pgfpathlineto{\pgfqpoint{0.982351in}{0.733741in}}%
\pgfpathlineto{\pgfqpoint{0.982648in}{0.733740in}}%
\pgfpathlineto{\pgfqpoint{0.982946in}{0.733740in}}%
\pgfpathlineto{\pgfqpoint{0.983243in}{0.733739in}}%
\pgfpathlineto{\pgfqpoint{0.983540in}{0.733738in}}%
\pgfpathlineto{\pgfqpoint{0.983838in}{0.733737in}}%
\pgfpathlineto{\pgfqpoint{0.984135in}{0.733737in}}%
\pgfpathlineto{\pgfqpoint{0.984433in}{0.733736in}}%
\pgfpathlineto{\pgfqpoint{0.984730in}{0.733735in}}%
\pgfpathlineto{\pgfqpoint{0.985028in}{0.733734in}}%
\pgfpathlineto{\pgfqpoint{0.985325in}{0.733733in}}%
\pgfpathlineto{\pgfqpoint{0.985623in}{0.733733in}}%
\pgfpathlineto{\pgfqpoint{0.985920in}{0.733732in}}%
\pgfpathlineto{\pgfqpoint{0.986218in}{0.733731in}}%
\pgfpathlineto{\pgfqpoint{0.986515in}{0.733730in}}%
\pgfpathlineto{\pgfqpoint{0.986813in}{0.733730in}}%
\pgfpathlineto{\pgfqpoint{0.987110in}{0.733729in}}%
\pgfpathlineto{\pgfqpoint{0.987408in}{0.733728in}}%
\pgfpathlineto{\pgfqpoint{0.987705in}{0.733727in}}%
\pgfpathlineto{\pgfqpoint{0.988003in}{0.733727in}}%
\pgfpathlineto{\pgfqpoint{0.988300in}{0.733726in}}%
\pgfpathlineto{\pgfqpoint{0.988598in}{0.733725in}}%
\pgfpathlineto{\pgfqpoint{0.988895in}{0.733724in}}%
\pgfpathlineto{\pgfqpoint{0.989193in}{0.733724in}}%
\pgfpathlineto{\pgfqpoint{0.989490in}{0.733723in}}%
\pgfpathlineto{\pgfqpoint{0.989788in}{0.733722in}}%
\pgfpathlineto{\pgfqpoint{0.990085in}{0.733721in}}%
\pgfpathlineto{\pgfqpoint{0.990382in}{0.733721in}}%
\pgfpathlineto{\pgfqpoint{0.990680in}{0.733720in}}%
\pgfpathlineto{\pgfqpoint{0.990977in}{0.733719in}}%
\pgfpathlineto{\pgfqpoint{0.991275in}{0.733718in}}%
\pgfpathlineto{\pgfqpoint{0.991572in}{0.733717in}}%
\pgfpathlineto{\pgfqpoint{0.991870in}{0.733717in}}%
\pgfpathlineto{\pgfqpoint{0.992167in}{0.733716in}}%
\pgfpathlineto{\pgfqpoint{0.992465in}{0.733715in}}%
\pgfpathlineto{\pgfqpoint{0.992762in}{0.733714in}}%
\pgfpathlineto{\pgfqpoint{0.993060in}{0.733714in}}%
\pgfpathlineto{\pgfqpoint{0.993357in}{0.733713in}}%
\pgfpathlineto{\pgfqpoint{0.993655in}{0.733712in}}%
\pgfpathlineto{\pgfqpoint{0.993952in}{0.733711in}}%
\pgfpathlineto{\pgfqpoint{0.994250in}{0.733711in}}%
\pgfpathlineto{\pgfqpoint{0.994547in}{0.733710in}}%
\pgfpathlineto{\pgfqpoint{0.994845in}{0.733709in}}%
\pgfpathlineto{\pgfqpoint{0.995142in}{0.733708in}}%
\pgfpathlineto{\pgfqpoint{0.995440in}{0.733708in}}%
\pgfpathlineto{\pgfqpoint{0.995737in}{0.733707in}}%
\pgfpathlineto{\pgfqpoint{0.996035in}{0.733706in}}%
\pgfpathlineto{\pgfqpoint{0.996332in}{0.733705in}}%
\pgfpathlineto{\pgfqpoint{0.996629in}{0.733705in}}%
\pgfpathlineto{\pgfqpoint{0.996927in}{0.733704in}}%
\pgfpathlineto{\pgfqpoint{0.997224in}{0.733703in}}%
\pgfpathlineto{\pgfqpoint{0.997522in}{0.733702in}}%
\pgfpathlineto{\pgfqpoint{0.997819in}{0.733701in}}%
\pgfpathlineto{\pgfqpoint{0.998117in}{0.733701in}}%
\pgfpathlineto{\pgfqpoint{0.998414in}{0.733700in}}%
\pgfpathlineto{\pgfqpoint{0.998712in}{0.733699in}}%
\pgfpathlineto{\pgfqpoint{0.999009in}{0.733698in}}%
\pgfpathlineto{\pgfqpoint{0.999307in}{0.733698in}}%
\pgfpathlineto{\pgfqpoint{0.999604in}{0.733697in}}%
\pgfpathlineto{\pgfqpoint{0.999902in}{0.733696in}}%
\pgfpathlineto{\pgfqpoint{1.000199in}{0.733695in}}%
\pgfpathlineto{\pgfqpoint{1.000497in}{0.733695in}}%
\pgfpathlineto{\pgfqpoint{1.000794in}{0.733694in}}%
\pgfpathlineto{\pgfqpoint{1.001092in}{0.733693in}}%
\pgfpathlineto{\pgfqpoint{1.001389in}{0.733692in}}%
\pgfpathlineto{\pgfqpoint{1.001687in}{0.733692in}}%
\pgfpathlineto{\pgfqpoint{1.001984in}{0.733691in}}%
\pgfpathlineto{\pgfqpoint{1.002282in}{0.733690in}}%
\pgfpathlineto{\pgfqpoint{1.002579in}{0.733689in}}%
\pgfpathlineto{\pgfqpoint{1.002877in}{0.733689in}}%
\pgfpathlineto{\pgfqpoint{1.003174in}{0.733688in}}%
\pgfpathlineto{\pgfqpoint{1.003471in}{0.733687in}}%
\pgfpathlineto{\pgfqpoint{1.003769in}{0.733686in}}%
\pgfpathlineto{\pgfqpoint{1.004066in}{0.733685in}}%
\pgfpathlineto{\pgfqpoint{1.004364in}{0.733685in}}%
\pgfpathlineto{\pgfqpoint{1.004661in}{0.733684in}}%
\pgfpathlineto{\pgfqpoint{1.004959in}{0.733683in}}%
\pgfpathlineto{\pgfqpoint{1.005256in}{0.733682in}}%
\pgfpathlineto{\pgfqpoint{1.005554in}{0.733682in}}%
\pgfpathlineto{\pgfqpoint{1.005851in}{0.733681in}}%
\pgfpathlineto{\pgfqpoint{1.006149in}{0.733660in}}%
\pgfpathlineto{\pgfqpoint{1.006446in}{0.733432in}}%
\pgfpathlineto{\pgfqpoint{1.006744in}{0.733150in}}%
\pgfpathlineto{\pgfqpoint{1.007041in}{0.732867in}}%
\pgfpathlineto{\pgfqpoint{1.007339in}{0.732584in}}%
\pgfpathlineto{\pgfqpoint{1.007636in}{0.732301in}}%
\pgfpathlineto{\pgfqpoint{1.007934in}{0.732018in}}%
\pgfpathlineto{\pgfqpoint{1.008231in}{0.731735in}}%
\pgfpathlineto{\pgfqpoint{1.008529in}{0.731452in}}%
\pgfpathlineto{\pgfqpoint{1.008826in}{0.731169in}}%
\pgfpathlineto{\pgfqpoint{1.009124in}{0.730886in}}%
\pgfpathlineto{\pgfqpoint{1.009421in}{0.730603in}}%
\pgfpathlineto{\pgfqpoint{1.009719in}{0.730320in}}%
\pgfpathlineto{\pgfqpoint{1.010016in}{0.730037in}}%
\pgfpathlineto{\pgfqpoint{1.010313in}{0.729754in}}%
\pgfpathlineto{\pgfqpoint{1.010611in}{0.729472in}}%
\pgfpathlineto{\pgfqpoint{1.010908in}{0.729189in}}%
\pgfpathlineto{\pgfqpoint{1.011206in}{0.728906in}}%
\pgfpathlineto{\pgfqpoint{1.011503in}{0.728623in}}%
\pgfpathlineto{\pgfqpoint{1.011801in}{0.728340in}}%
\pgfpathlineto{\pgfqpoint{1.012098in}{0.728057in}}%
\pgfpathlineto{\pgfqpoint{1.012396in}{0.727774in}}%
\pgfpathlineto{\pgfqpoint{1.012693in}{0.727491in}}%
\pgfpathlineto{\pgfqpoint{1.012991in}{0.727208in}}%
\pgfpathlineto{\pgfqpoint{1.013288in}{0.726925in}}%
\pgfpathlineto{\pgfqpoint{1.013586in}{0.726642in}}%
\pgfpathlineto{\pgfqpoint{1.013883in}{0.726359in}}%
\pgfpathlineto{\pgfqpoint{1.014181in}{0.726208in}}%
\pgfpathlineto{\pgfqpoint{1.014478in}{0.726215in}}%
\pgfpathlineto{\pgfqpoint{1.014776in}{0.726223in}}%
\pgfpathlineto{\pgfqpoint{1.015073in}{0.726230in}}%
\pgfpathlineto{\pgfqpoint{1.015371in}{0.726238in}}%
\pgfpathlineto{\pgfqpoint{1.015668in}{0.726246in}}%
\pgfpathlineto{\pgfqpoint{1.015966in}{0.726253in}}%
\pgfpathlineto{\pgfqpoint{1.016263in}{0.726261in}}%
\pgfpathlineto{\pgfqpoint{1.016560in}{0.726269in}}%
\pgfpathlineto{\pgfqpoint{1.016858in}{0.726276in}}%
\pgfpathlineto{\pgfqpoint{1.017155in}{0.726284in}}%
\pgfpathlineto{\pgfqpoint{1.017453in}{0.726292in}}%
\pgfpathlineto{\pgfqpoint{1.017750in}{0.726299in}}%
\pgfpathlineto{\pgfqpoint{1.018048in}{0.726307in}}%
\pgfpathlineto{\pgfqpoint{1.018345in}{0.726315in}}%
\pgfpathlineto{\pgfqpoint{1.018643in}{0.726322in}}%
\pgfpathlineto{\pgfqpoint{1.018940in}{0.726330in}}%
\pgfpathlineto{\pgfqpoint{1.019238in}{0.726338in}}%
\pgfpathlineto{\pgfqpoint{1.019535in}{0.726345in}}%
\pgfpathlineto{\pgfqpoint{1.019833in}{0.726353in}}%
\pgfpathlineto{\pgfqpoint{1.020130in}{0.726361in}}%
\pgfpathlineto{\pgfqpoint{1.020428in}{0.726369in}}%
\pgfpathlineto{\pgfqpoint{1.020725in}{0.726376in}}%
\pgfpathlineto{\pgfqpoint{1.021023in}{0.726384in}}%
\pgfpathlineto{\pgfqpoint{1.021320in}{0.726392in}}%
\pgfpathlineto{\pgfqpoint{1.021618in}{0.726399in}}%
\pgfpathlineto{\pgfqpoint{1.021915in}{0.726407in}}%
\pgfpathlineto{\pgfqpoint{1.022213in}{0.726415in}}%
\pgfpathlineto{\pgfqpoint{1.022510in}{0.726422in}}%
\pgfpathlineto{\pgfqpoint{1.022808in}{0.726430in}}%
\pgfpathlineto{\pgfqpoint{1.023105in}{0.726438in}}%
\pgfpathlineto{\pgfqpoint{1.023402in}{0.726445in}}%
\pgfpathlineto{\pgfqpoint{1.023700in}{0.726453in}}%
\pgfpathlineto{\pgfqpoint{1.023997in}{0.726461in}}%
\pgfpathlineto{\pgfqpoint{1.024295in}{0.726468in}}%
\pgfpathlineto{\pgfqpoint{1.024592in}{0.726476in}}%
\pgfpathlineto{\pgfqpoint{1.024890in}{0.726484in}}%
\pgfpathlineto{\pgfqpoint{1.025187in}{0.726491in}}%
\pgfpathlineto{\pgfqpoint{1.025485in}{0.726499in}}%
\pgfpathlineto{\pgfqpoint{1.025782in}{0.726507in}}%
\pgfpathlineto{\pgfqpoint{1.026080in}{0.726514in}}%
\pgfpathlineto{\pgfqpoint{1.026377in}{0.726522in}}%
\pgfpathlineto{\pgfqpoint{1.026675in}{0.726530in}}%
\pgfpathlineto{\pgfqpoint{1.026972in}{0.726537in}}%
\pgfpathlineto{\pgfqpoint{1.027270in}{0.726545in}}%
\pgfpathlineto{\pgfqpoint{1.027567in}{0.726553in}}%
\pgfpathlineto{\pgfqpoint{1.027865in}{0.726560in}}%
\pgfpathlineto{\pgfqpoint{1.028162in}{0.726568in}}%
\pgfpathlineto{\pgfqpoint{1.028460in}{0.726576in}}%
\pgfpathlineto{\pgfqpoint{1.028757in}{0.726584in}}%
\pgfpathlineto{\pgfqpoint{1.029055in}{0.726591in}}%
\pgfpathlineto{\pgfqpoint{1.029352in}{0.726599in}}%
\pgfpathlineto{\pgfqpoint{1.029650in}{0.726607in}}%
\pgfpathlineto{\pgfqpoint{1.029947in}{0.726614in}}%
\pgfpathlineto{\pgfqpoint{1.030244in}{0.726622in}}%
\pgfpathlineto{\pgfqpoint{1.030542in}{0.726630in}}%
\pgfpathlineto{\pgfqpoint{1.030839in}{0.726637in}}%
\pgfpathlineto{\pgfqpoint{1.031137in}{0.726645in}}%
\pgfpathlineto{\pgfqpoint{1.031434in}{0.726653in}}%
\pgfpathlineto{\pgfqpoint{1.031732in}{0.726660in}}%
\pgfpathlineto{\pgfqpoint{1.032029in}{0.726668in}}%
\pgfpathlineto{\pgfqpoint{1.032327in}{0.726676in}}%
\pgfpathlineto{\pgfqpoint{1.032624in}{0.726683in}}%
\pgfpathlineto{\pgfqpoint{1.032922in}{0.726691in}}%
\pgfpathlineto{\pgfqpoint{1.033219in}{0.726699in}}%
\pgfpathlineto{\pgfqpoint{1.033517in}{0.726706in}}%
\pgfpathlineto{\pgfqpoint{1.033814in}{0.726714in}}%
\pgfpathlineto{\pgfqpoint{1.034112in}{0.726722in}}%
\pgfpathlineto{\pgfqpoint{1.034409in}{0.726729in}}%
\pgfpathlineto{\pgfqpoint{1.034707in}{0.726737in}}%
\pgfpathlineto{\pgfqpoint{1.035004in}{0.726744in}}%
\pgfpathlineto{\pgfqpoint{1.035302in}{0.726748in}}%
\pgfpathlineto{\pgfqpoint{1.035599in}{0.726752in}}%
\pgfpathlineto{\pgfqpoint{1.035897in}{0.726755in}}%
\pgfpathlineto{\pgfqpoint{1.036194in}{0.726759in}}%
\pgfpathlineto{\pgfqpoint{1.036491in}{0.726763in}}%
\pgfpathlineto{\pgfqpoint{1.036789in}{0.726767in}}%
\pgfpathlineto{\pgfqpoint{1.037086in}{0.726770in}}%
\pgfpathlineto{\pgfqpoint{1.037384in}{0.726774in}}%
\pgfpathlineto{\pgfqpoint{1.037681in}{0.726778in}}%
\pgfpathlineto{\pgfqpoint{1.037979in}{0.726782in}}%
\pgfpathlineto{\pgfqpoint{1.038276in}{0.726786in}}%
\pgfpathlineto{\pgfqpoint{1.038574in}{0.726789in}}%
\pgfpathlineto{\pgfqpoint{1.038871in}{0.726793in}}%
\pgfpathlineto{\pgfqpoint{1.039169in}{0.726797in}}%
\pgfpathlineto{\pgfqpoint{1.039466in}{0.726801in}}%
\pgfpathlineto{\pgfqpoint{1.039764in}{0.726805in}}%
\pgfpathlineto{\pgfqpoint{1.040061in}{0.726808in}}%
\pgfpathlineto{\pgfqpoint{1.040359in}{0.726812in}}%
\pgfpathlineto{\pgfqpoint{1.040656in}{0.726816in}}%
\pgfpathlineto{\pgfqpoint{1.040954in}{0.726820in}}%
\pgfpathlineto{\pgfqpoint{1.041251in}{0.726824in}}%
\pgfpathlineto{\pgfqpoint{1.041549in}{0.726827in}}%
\pgfpathlineto{\pgfqpoint{1.041846in}{0.726831in}}%
\pgfpathlineto{\pgfqpoint{1.042144in}{0.726835in}}%
\pgfpathlineto{\pgfqpoint{1.042441in}{0.726839in}}%
\pgfpathlineto{\pgfqpoint{1.042739in}{0.726842in}}%
\pgfpathlineto{\pgfqpoint{1.043036in}{0.726846in}}%
\pgfpathlineto{\pgfqpoint{1.043333in}{0.726850in}}%
\pgfpathlineto{\pgfqpoint{1.043631in}{0.726854in}}%
\pgfpathlineto{\pgfqpoint{1.043928in}{0.726945in}}%
\pgfpathlineto{\pgfqpoint{1.044226in}{0.727265in}}%
\pgfpathlineto{\pgfqpoint{1.044523in}{0.727607in}}%
\pgfpathlineto{\pgfqpoint{1.044821in}{0.727949in}}%
\pgfpathlineto{\pgfqpoint{1.045118in}{0.728114in}}%
\pgfpathlineto{\pgfqpoint{1.045416in}{0.728108in}}%
\pgfpathlineto{\pgfqpoint{1.045713in}{0.728103in}}%
\pgfpathlineto{\pgfqpoint{1.046011in}{0.728097in}}%
\pgfpathlineto{\pgfqpoint{1.046308in}{0.728091in}}%
\pgfpathlineto{\pgfqpoint{1.046606in}{0.728085in}}%
\pgfpathlineto{\pgfqpoint{1.046903in}{0.728080in}}%
\pgfpathlineto{\pgfqpoint{1.047201in}{0.728074in}}%
\pgfpathlineto{\pgfqpoint{1.047498in}{0.728068in}}%
\pgfpathlineto{\pgfqpoint{1.047796in}{0.728062in}}%
\pgfpathlineto{\pgfqpoint{1.048093in}{0.728057in}}%
\pgfpathlineto{\pgfqpoint{1.048391in}{0.728051in}}%
\pgfpathlineto{\pgfqpoint{1.048688in}{0.728045in}}%
\pgfpathlineto{\pgfqpoint{1.048986in}{0.728039in}}%
\pgfpathlineto{\pgfqpoint{1.049283in}{0.728034in}}%
\pgfpathlineto{\pgfqpoint{1.049581in}{0.728028in}}%
\pgfpathlineto{\pgfqpoint{1.049878in}{0.728026in}}%
\pgfpathlineto{\pgfqpoint{1.050175in}{0.728184in}}%
\pgfpathlineto{\pgfqpoint{1.050473in}{0.728218in}}%
\pgfpathlineto{\pgfqpoint{1.050770in}{0.728213in}}%
\pgfpathlineto{\pgfqpoint{1.051068in}{0.728208in}}%
\pgfpathlineto{\pgfqpoint{1.051365in}{0.728204in}}%
\pgfpathlineto{\pgfqpoint{1.051663in}{0.728199in}}%
\pgfpathlineto{\pgfqpoint{1.051960in}{0.728194in}}%
\pgfpathlineto{\pgfqpoint{1.052258in}{0.728190in}}%
\pgfpathlineto{\pgfqpoint{1.052555in}{0.728185in}}%
\pgfpathlineto{\pgfqpoint{1.052853in}{0.728180in}}%
\pgfpathlineto{\pgfqpoint{1.053150in}{0.728176in}}%
\pgfpathlineto{\pgfqpoint{1.053448in}{0.728171in}}%
\pgfpathlineto{\pgfqpoint{1.053745in}{0.728166in}}%
\pgfpathlineto{\pgfqpoint{1.054043in}{0.728162in}}%
\pgfpathlineto{\pgfqpoint{1.054340in}{0.728157in}}%
\pgfpathlineto{\pgfqpoint{1.054638in}{0.728153in}}%
\pgfpathlineto{\pgfqpoint{1.054935in}{0.728148in}}%
\pgfpathlineto{\pgfqpoint{1.055233in}{0.728143in}}%
\pgfpathlineto{\pgfqpoint{1.055530in}{0.728139in}}%
\pgfpathlineto{\pgfqpoint{1.055828in}{0.728134in}}%
\pgfpathlineto{\pgfqpoint{1.056125in}{0.728129in}}%
\pgfpathlineto{\pgfqpoint{1.056422in}{0.728125in}}%
\pgfpathlineto{\pgfqpoint{1.056720in}{0.728120in}}%
\pgfpathlineto{\pgfqpoint{1.057017in}{0.728115in}}%
\pgfpathlineto{\pgfqpoint{1.057315in}{0.728111in}}%
\pgfpathlineto{\pgfqpoint{1.057612in}{0.728106in}}%
\pgfpathlineto{\pgfqpoint{1.057910in}{0.728101in}}%
\pgfpathlineto{\pgfqpoint{1.058207in}{0.728097in}}%
\pgfpathlineto{\pgfqpoint{1.058505in}{0.728092in}}%
\pgfpathlineto{\pgfqpoint{1.058802in}{0.728087in}}%
\pgfpathlineto{\pgfqpoint{1.059100in}{0.728083in}}%
\pgfpathlineto{\pgfqpoint{1.059397in}{0.728078in}}%
\pgfpathlineto{\pgfqpoint{1.059695in}{0.728073in}}%
\pgfpathlineto{\pgfqpoint{1.059992in}{0.728069in}}%
\pgfpathlineto{\pgfqpoint{1.060290in}{0.728064in}}%
\pgfpathlineto{\pgfqpoint{1.060587in}{0.728059in}}%
\pgfpathlineto{\pgfqpoint{1.060885in}{0.728055in}}%
\pgfpathlineto{\pgfqpoint{1.061182in}{0.728050in}}%
\pgfpathlineto{\pgfqpoint{1.061480in}{0.728046in}}%
\pgfpathlineto{\pgfqpoint{1.061777in}{0.728041in}}%
\pgfpathlineto{\pgfqpoint{1.062075in}{0.728036in}}%
\pgfpathlineto{\pgfqpoint{1.062372in}{0.728032in}}%
\pgfpathlineto{\pgfqpoint{1.062670in}{0.728027in}}%
\pgfpathlineto{\pgfqpoint{1.062967in}{0.728022in}}%
\pgfpathlineto{\pgfqpoint{1.063264in}{0.728018in}}%
\pgfpathlineto{\pgfqpoint{1.063562in}{0.728013in}}%
\pgfpathlineto{\pgfqpoint{1.063859in}{0.728008in}}%
\pgfpathlineto{\pgfqpoint{1.064157in}{0.728004in}}%
\pgfpathlineto{\pgfqpoint{1.064454in}{0.727999in}}%
\pgfpathlineto{\pgfqpoint{1.064752in}{0.727994in}}%
\pgfpathlineto{\pgfqpoint{1.065049in}{0.727990in}}%
\pgfpathlineto{\pgfqpoint{1.065347in}{0.727985in}}%
\pgfpathlineto{\pgfqpoint{1.065644in}{0.727978in}}%
\pgfpathlineto{\pgfqpoint{1.065942in}{0.727964in}}%
\pgfpathlineto{\pgfqpoint{1.066239in}{0.727951in}}%
\pgfpathlineto{\pgfqpoint{1.066537in}{0.727937in}}%
\pgfpathlineto{\pgfqpoint{1.066834in}{0.727924in}}%
\pgfpathlineto{\pgfqpoint{1.067132in}{0.727910in}}%
\pgfpathlineto{\pgfqpoint{1.067429in}{0.727896in}}%
\pgfpathlineto{\pgfqpoint{1.067727in}{0.727883in}}%
\pgfpathlineto{\pgfqpoint{1.068024in}{0.727869in}}%
\pgfpathlineto{\pgfqpoint{1.068322in}{0.727855in}}%
\pgfpathlineto{\pgfqpoint{1.068619in}{0.727842in}}%
\pgfpathlineto{\pgfqpoint{1.068917in}{0.727828in}}%
\pgfpathlineto{\pgfqpoint{1.069214in}{0.727815in}}%
\pgfpathlineto{\pgfqpoint{1.069512in}{0.727801in}}%
\pgfpathlineto{\pgfqpoint{1.069809in}{0.727787in}}%
\pgfpathlineto{\pgfqpoint{1.070106in}{0.727774in}}%
\pgfpathlineto{\pgfqpoint{1.070404in}{0.727760in}}%
\pgfpathlineto{\pgfqpoint{1.070701in}{0.727746in}}%
\pgfpathlineto{\pgfqpoint{1.070999in}{0.727733in}}%
\pgfpathlineto{\pgfqpoint{1.071296in}{0.727719in}}%
\pgfpathlineto{\pgfqpoint{1.071594in}{0.727706in}}%
\pgfpathlineto{\pgfqpoint{1.071891in}{0.727692in}}%
\pgfpathlineto{\pgfqpoint{1.072189in}{0.727678in}}%
\pgfpathlineto{\pgfqpoint{1.072486in}{0.727665in}}%
\pgfpathlineto{\pgfqpoint{1.072784in}{0.727651in}}%
\pgfpathlineto{\pgfqpoint{1.073081in}{0.727638in}}%
\pgfpathlineto{\pgfqpoint{1.073379in}{0.727624in}}%
\pgfpathlineto{\pgfqpoint{1.073676in}{0.727610in}}%
\pgfpathlineto{\pgfqpoint{1.073974in}{0.727597in}}%
\pgfpathlineto{\pgfqpoint{1.074271in}{0.727583in}}%
\pgfpathlineto{\pgfqpoint{1.074569in}{0.727569in}}%
\pgfpathlineto{\pgfqpoint{1.074866in}{0.727556in}}%
\pgfpathlineto{\pgfqpoint{1.075164in}{0.727542in}}%
\pgfpathlineto{\pgfqpoint{1.075461in}{0.727529in}}%
\pgfpathlineto{\pgfqpoint{1.075759in}{0.727515in}}%
\pgfpathlineto{\pgfqpoint{1.076056in}{0.727501in}}%
\pgfpathlineto{\pgfqpoint{1.076353in}{0.727488in}}%
\pgfpathlineto{\pgfqpoint{1.076651in}{0.727474in}}%
\pgfpathlineto{\pgfqpoint{1.076948in}{0.727460in}}%
\pgfpathlineto{\pgfqpoint{1.077246in}{0.727447in}}%
\pgfpathlineto{\pgfqpoint{1.077543in}{0.727433in}}%
\pgfpathlineto{\pgfqpoint{1.077841in}{0.727420in}}%
\pgfpathlineto{\pgfqpoint{1.078138in}{0.727406in}}%
\pgfpathlineto{\pgfqpoint{1.078436in}{0.727392in}}%
\pgfpathlineto{\pgfqpoint{1.078733in}{0.727379in}}%
\pgfpathlineto{\pgfqpoint{1.079031in}{0.727365in}}%
\pgfpathlineto{\pgfqpoint{1.079328in}{0.727351in}}%
\pgfpathlineto{\pgfqpoint{1.079626in}{0.727338in}}%
\pgfpathlineto{\pgfqpoint{1.079923in}{0.727324in}}%
\pgfpathlineto{\pgfqpoint{1.080221in}{0.727311in}}%
\pgfpathlineto{\pgfqpoint{1.080518in}{0.727297in}}%
\pgfpathlineto{\pgfqpoint{1.080816in}{0.727283in}}%
\pgfpathlineto{\pgfqpoint{1.081113in}{0.727270in}}%
\pgfpathlineto{\pgfqpoint{1.081411in}{0.727256in}}%
\pgfpathlineto{\pgfqpoint{1.081708in}{0.727242in}}%
\pgfpathlineto{\pgfqpoint{1.082006in}{0.727229in}}%
\pgfpathlineto{\pgfqpoint{1.082303in}{0.727215in}}%
\pgfpathlineto{\pgfqpoint{1.082601in}{0.727202in}}%
\pgfpathlineto{\pgfqpoint{1.082898in}{0.727188in}}%
\pgfpathlineto{\pgfqpoint{1.083195in}{0.727174in}}%
\pgfpathlineto{\pgfqpoint{1.083493in}{0.727161in}}%
\pgfpathlineto{\pgfqpoint{1.083790in}{0.727147in}}%
\pgfpathlineto{\pgfqpoint{1.084088in}{0.727133in}}%
\pgfpathlineto{\pgfqpoint{1.084385in}{0.727120in}}%
\pgfpathlineto{\pgfqpoint{1.084683in}{0.727106in}}%
\pgfpathlineto{\pgfqpoint{1.084980in}{0.727093in}}%
\pgfpathlineto{\pgfqpoint{1.085278in}{0.727079in}}%
\pgfpathlineto{\pgfqpoint{1.085575in}{0.727065in}}%
\pgfpathlineto{\pgfqpoint{1.085873in}{0.727052in}}%
\pgfpathlineto{\pgfqpoint{1.086170in}{0.727038in}}%
\pgfpathlineto{\pgfqpoint{1.086468in}{0.727024in}}%
\pgfpathlineto{\pgfqpoint{1.086765in}{0.727011in}}%
\pgfpathlineto{\pgfqpoint{1.087063in}{0.726998in}}%
\pgfpathlineto{\pgfqpoint{1.087360in}{0.726985in}}%
\pgfpathlineto{\pgfqpoint{1.087658in}{0.726973in}}%
\pgfpathlineto{\pgfqpoint{1.087955in}{0.726960in}}%
\pgfpathlineto{\pgfqpoint{1.088253in}{0.726948in}}%
\pgfpathlineto{\pgfqpoint{1.088550in}{0.726935in}}%
\pgfpathlineto{\pgfqpoint{1.088848in}{0.726923in}}%
\pgfpathlineto{\pgfqpoint{1.089145in}{0.726910in}}%
\pgfpathlineto{\pgfqpoint{1.089443in}{0.726898in}}%
\pgfpathlineto{\pgfqpoint{1.089740in}{0.726885in}}%
\pgfpathlineto{\pgfqpoint{1.090037in}{0.726872in}}%
\pgfpathlineto{\pgfqpoint{1.090335in}{0.726860in}}%
\pgfpathlineto{\pgfqpoint{1.090632in}{0.726847in}}%
\pgfpathlineto{\pgfqpoint{1.090930in}{0.726835in}}%
\pgfpathlineto{\pgfqpoint{1.091227in}{0.726822in}}%
\pgfpathlineto{\pgfqpoint{1.091525in}{0.726810in}}%
\pgfpathlineto{\pgfqpoint{1.091822in}{0.726797in}}%
\pgfpathlineto{\pgfqpoint{1.092120in}{0.726785in}}%
\pgfpathlineto{\pgfqpoint{1.092417in}{0.726772in}}%
\pgfpathlineto{\pgfqpoint{1.092715in}{0.726760in}}%
\pgfpathlineto{\pgfqpoint{1.093012in}{0.726747in}}%
\pgfpathlineto{\pgfqpoint{1.093310in}{0.726735in}}%
\pgfpathlineto{\pgfqpoint{1.093607in}{0.726722in}}%
\pgfpathlineto{\pgfqpoint{1.093905in}{0.726710in}}%
\pgfpathlineto{\pgfqpoint{1.094202in}{0.726697in}}%
\pgfpathlineto{\pgfqpoint{1.094500in}{0.726691in}}%
\pgfpathlineto{\pgfqpoint{1.094797in}{0.726691in}}%
\pgfpathlineto{\pgfqpoint{1.095095in}{0.726690in}}%
\pgfpathlineto{\pgfqpoint{1.095392in}{0.726689in}}%
\pgfpathlineto{\pgfqpoint{1.095690in}{0.726688in}}%
\pgfpathlineto{\pgfqpoint{1.095987in}{0.726687in}}%
\pgfpathlineto{\pgfqpoint{1.096284in}{0.726686in}}%
\pgfpathlineto{\pgfqpoint{1.096582in}{0.726685in}}%
\pgfpathlineto{\pgfqpoint{1.096879in}{0.726685in}}%
\pgfpathlineto{\pgfqpoint{1.097177in}{0.726684in}}%
\pgfpathlineto{\pgfqpoint{1.097474in}{0.726683in}}%
\pgfpathlineto{\pgfqpoint{1.097772in}{0.726682in}}%
\pgfpathlineto{\pgfqpoint{1.098069in}{0.726681in}}%
\pgfpathlineto{\pgfqpoint{1.098367in}{0.726680in}}%
\pgfpathlineto{\pgfqpoint{1.098664in}{0.726680in}}%
\pgfpathlineto{\pgfqpoint{1.098962in}{0.726679in}}%
\pgfpathlineto{\pgfqpoint{1.099259in}{0.726678in}}%
\pgfpathlineto{\pgfqpoint{1.099557in}{0.726677in}}%
\pgfpathlineto{\pgfqpoint{1.099854in}{0.726676in}}%
\pgfpathlineto{\pgfqpoint{1.100152in}{0.726675in}}%
\pgfpathlineto{\pgfqpoint{1.100449in}{0.726674in}}%
\pgfpathlineto{\pgfqpoint{1.100747in}{0.726674in}}%
\pgfpathlineto{\pgfqpoint{1.101044in}{0.726673in}}%
\pgfpathlineto{\pgfqpoint{1.101342in}{0.726672in}}%
\pgfpathlineto{\pgfqpoint{1.101639in}{0.726671in}}%
\pgfpathlineto{\pgfqpoint{1.101937in}{0.726670in}}%
\pgfpathlineto{\pgfqpoint{1.102234in}{0.726669in}}%
\pgfpathlineto{\pgfqpoint{1.102532in}{0.726669in}}%
\pgfpathlineto{\pgfqpoint{1.102829in}{0.726668in}}%
\pgfpathlineto{\pgfqpoint{1.103126in}{0.726667in}}%
\pgfpathlineto{\pgfqpoint{1.103424in}{0.726666in}}%
\pgfpathlineto{\pgfqpoint{1.103721in}{0.726665in}}%
\pgfpathlineto{\pgfqpoint{1.104019in}{0.726664in}}%
\pgfpathlineto{\pgfqpoint{1.104316in}{0.726663in}}%
\pgfpathlineto{\pgfqpoint{1.104614in}{0.726663in}}%
\pgfpathlineto{\pgfqpoint{1.104911in}{0.726662in}}%
\pgfpathlineto{\pgfqpoint{1.105209in}{0.726661in}}%
\pgfpathlineto{\pgfqpoint{1.105506in}{0.726660in}}%
\pgfpathlineto{\pgfqpoint{1.105804in}{0.726659in}}%
\pgfpathlineto{\pgfqpoint{1.106101in}{0.726658in}}%
\pgfpathlineto{\pgfqpoint{1.106399in}{0.726658in}}%
\pgfpathlineto{\pgfqpoint{1.106696in}{0.726657in}}%
\pgfpathlineto{\pgfqpoint{1.106994in}{0.726656in}}%
\pgfpathlineto{\pgfqpoint{1.107291in}{0.726655in}}%
\pgfpathlineto{\pgfqpoint{1.107589in}{0.726654in}}%
\pgfpathlineto{\pgfqpoint{1.107886in}{0.726653in}}%
\pgfpathlineto{\pgfqpoint{1.108184in}{0.726652in}}%
\pgfpathlineto{\pgfqpoint{1.108481in}{0.726652in}}%
\pgfpathlineto{\pgfqpoint{1.108779in}{0.726651in}}%
\pgfpathlineto{\pgfqpoint{1.109076in}{0.726650in}}%
\pgfpathlineto{\pgfqpoint{1.109374in}{0.726649in}}%
\pgfpathlineto{\pgfqpoint{1.109671in}{0.726648in}}%
\pgfpathlineto{\pgfqpoint{1.109968in}{0.726647in}}%
\pgfpathlineto{\pgfqpoint{1.110266in}{0.726647in}}%
\pgfpathlineto{\pgfqpoint{1.110563in}{0.726646in}}%
\pgfpathlineto{\pgfqpoint{1.110861in}{0.726645in}}%
\pgfpathlineto{\pgfqpoint{1.111158in}{0.726644in}}%
\pgfpathlineto{\pgfqpoint{1.111456in}{0.726643in}}%
\pgfpathlineto{\pgfqpoint{1.111753in}{0.726642in}}%
\pgfpathlineto{\pgfqpoint{1.112051in}{0.726642in}}%
\pgfpathlineto{\pgfqpoint{1.112348in}{0.726641in}}%
\pgfpathlineto{\pgfqpoint{1.112646in}{0.726640in}}%
\pgfpathlineto{\pgfqpoint{1.112943in}{0.726639in}}%
\pgfpathlineto{\pgfqpoint{1.113241in}{0.726638in}}%
\pgfpathlineto{\pgfqpoint{1.113538in}{0.726637in}}%
\pgfpathlineto{\pgfqpoint{1.113836in}{0.726636in}}%
\pgfpathlineto{\pgfqpoint{1.114133in}{0.726636in}}%
\pgfpathlineto{\pgfqpoint{1.114431in}{0.726635in}}%
\pgfpathlineto{\pgfqpoint{1.114728in}{0.726634in}}%
\pgfpathlineto{\pgfqpoint{1.115026in}{0.726633in}}%
\pgfpathlineto{\pgfqpoint{1.115323in}{0.726632in}}%
\pgfpathlineto{\pgfqpoint{1.115621in}{0.726631in}}%
\pgfpathlineto{\pgfqpoint{1.115918in}{0.726631in}}%
\pgfpathlineto{\pgfqpoint{1.116215in}{0.726630in}}%
\pgfpathlineto{\pgfqpoint{1.116513in}{0.726629in}}%
\pgfpathlineto{\pgfqpoint{1.116810in}{0.726628in}}%
\pgfpathlineto{\pgfqpoint{1.117108in}{0.726627in}}%
\pgfpathlineto{\pgfqpoint{1.117405in}{0.726626in}}%
\pgfpathlineto{\pgfqpoint{1.117703in}{0.726625in}}%
\pgfpathlineto{\pgfqpoint{1.118000in}{0.726625in}}%
\pgfpathlineto{\pgfqpoint{1.118298in}{0.726624in}}%
\pgfpathlineto{\pgfqpoint{1.118595in}{0.726623in}}%
\pgfpathlineto{\pgfqpoint{1.118893in}{0.726622in}}%
\pgfpathlineto{\pgfqpoint{1.119190in}{0.726621in}}%
\pgfpathlineto{\pgfqpoint{1.119488in}{0.726620in}}%
\pgfpathlineto{\pgfqpoint{1.119785in}{0.726620in}}%
\pgfpathlineto{\pgfqpoint{1.120083in}{0.726619in}}%
\pgfpathlineto{\pgfqpoint{1.120380in}{0.726618in}}%
\pgfpathlineto{\pgfqpoint{1.120678in}{0.726617in}}%
\pgfpathlineto{\pgfqpoint{1.120975in}{0.726616in}}%
\pgfpathlineto{\pgfqpoint{1.121273in}{0.726615in}}%
\pgfpathlineto{\pgfqpoint{1.121570in}{0.726614in}}%
\pgfpathlineto{\pgfqpoint{1.121868in}{0.726614in}}%
\pgfpathlineto{\pgfqpoint{1.122165in}{0.726613in}}%
\pgfpathlineto{\pgfqpoint{1.122463in}{0.726612in}}%
\pgfpathlineto{\pgfqpoint{1.122760in}{0.726611in}}%
\pgfpathlineto{\pgfqpoint{1.123057in}{0.726610in}}%
\pgfpathlineto{\pgfqpoint{1.123355in}{0.726609in}}%
\pgfpathlineto{\pgfqpoint{1.123652in}{0.726609in}}%
\pgfpathlineto{\pgfqpoint{1.123950in}{0.726608in}}%
\pgfpathlineto{\pgfqpoint{1.124247in}{0.726607in}}%
\pgfpathlineto{\pgfqpoint{1.124545in}{0.726606in}}%
\pgfpathlineto{\pgfqpoint{1.124842in}{0.726605in}}%
\pgfpathlineto{\pgfqpoint{1.125140in}{0.726604in}}%
\pgfpathlineto{\pgfqpoint{1.125437in}{0.726603in}}%
\pgfpathlineto{\pgfqpoint{1.125735in}{0.726603in}}%
\pgfpathlineto{\pgfqpoint{1.126032in}{0.726602in}}%
\pgfpathlineto{\pgfqpoint{1.126330in}{0.726601in}}%
\pgfpathlineto{\pgfqpoint{1.126627in}{0.726600in}}%
\pgfpathlineto{\pgfqpoint{1.126925in}{0.726599in}}%
\pgfpathlineto{\pgfqpoint{1.127222in}{0.726598in}}%
\pgfpathlineto{\pgfqpoint{1.127520in}{0.726598in}}%
\pgfpathlineto{\pgfqpoint{1.127817in}{0.726597in}}%
\pgfpathlineto{\pgfqpoint{1.128115in}{0.726596in}}%
\pgfpathlineto{\pgfqpoint{1.128412in}{0.726595in}}%
\pgfpathlineto{\pgfqpoint{1.128710in}{0.726594in}}%
\pgfpathlineto{\pgfqpoint{1.129007in}{0.726593in}}%
\pgfpathlineto{\pgfqpoint{1.129305in}{0.726592in}}%
\pgfpathlineto{\pgfqpoint{1.129602in}{0.726592in}}%
\pgfpathlineto{\pgfqpoint{1.129899in}{0.726591in}}%
\pgfpathlineto{\pgfqpoint{1.130197in}{0.726590in}}%
\pgfpathlineto{\pgfqpoint{1.130494in}{0.726589in}}%
\pgfpathlineto{\pgfqpoint{1.130792in}{0.726588in}}%
\pgfpathlineto{\pgfqpoint{1.131089in}{0.726587in}}%
\pgfpathlineto{\pgfqpoint{1.131387in}{0.726587in}}%
\pgfpathlineto{\pgfqpoint{1.131684in}{0.726586in}}%
\pgfpathlineto{\pgfqpoint{1.131982in}{0.726585in}}%
\pgfpathlineto{\pgfqpoint{1.132279in}{0.726584in}}%
\pgfpathlineto{\pgfqpoint{1.132577in}{0.726583in}}%
\pgfpathlineto{\pgfqpoint{1.132874in}{0.726582in}}%
\pgfpathlineto{\pgfqpoint{1.133172in}{0.726581in}}%
\pgfpathlineto{\pgfqpoint{1.133469in}{0.726581in}}%
\pgfpathlineto{\pgfqpoint{1.133767in}{0.726580in}}%
\pgfpathlineto{\pgfqpoint{1.134064in}{0.726579in}}%
\pgfpathlineto{\pgfqpoint{1.134362in}{0.726578in}}%
\pgfpathlineto{\pgfqpoint{1.134659in}{0.726577in}}%
\pgfpathlineto{\pgfqpoint{1.134957in}{0.726576in}}%
\pgfpathlineto{\pgfqpoint{1.135254in}{0.726576in}}%
\pgfpathlineto{\pgfqpoint{1.135552in}{0.726575in}}%
\pgfpathlineto{\pgfqpoint{1.135849in}{0.726574in}}%
\pgfpathlineto{\pgfqpoint{1.136146in}{0.726573in}}%
\pgfpathlineto{\pgfqpoint{1.136444in}{0.726572in}}%
\pgfpathlineto{\pgfqpoint{1.136741in}{0.726571in}}%
\pgfpathlineto{\pgfqpoint{1.137039in}{0.726570in}}%
\pgfpathlineto{\pgfqpoint{1.137336in}{0.726570in}}%
\pgfpathlineto{\pgfqpoint{1.137634in}{0.726569in}}%
\pgfpathlineto{\pgfqpoint{1.137931in}{0.726568in}}%
\pgfpathlineto{\pgfqpoint{1.138229in}{0.726567in}}%
\pgfpathlineto{\pgfqpoint{1.138526in}{0.726566in}}%
\pgfpathlineto{\pgfqpoint{1.138824in}{0.726565in}}%
\pgfpathlineto{\pgfqpoint{1.139121in}{0.726565in}}%
\pgfpathlineto{\pgfqpoint{1.139419in}{0.726564in}}%
\pgfpathlineto{\pgfqpoint{1.139716in}{0.726563in}}%
\pgfpathlineto{\pgfqpoint{1.140014in}{0.726562in}}%
\pgfpathlineto{\pgfqpoint{1.140311in}{0.726561in}}%
\pgfpathlineto{\pgfqpoint{1.140609in}{0.726560in}}%
\pgfpathlineto{\pgfqpoint{1.140906in}{0.726560in}}%
\pgfpathlineto{\pgfqpoint{1.141204in}{0.726559in}}%
\pgfpathlineto{\pgfqpoint{1.141501in}{0.726558in}}%
\pgfpathlineto{\pgfqpoint{1.141799in}{0.726557in}}%
\pgfpathlineto{\pgfqpoint{1.142096in}{0.726556in}}%
\pgfpathlineto{\pgfqpoint{1.142394in}{0.726555in}}%
\pgfpathlineto{\pgfqpoint{1.142691in}{0.726554in}}%
\pgfpathlineto{\pgfqpoint{1.142988in}{0.726554in}}%
\pgfpathlineto{\pgfqpoint{1.143286in}{0.726553in}}%
\pgfpathlineto{\pgfqpoint{1.143583in}{0.726552in}}%
\pgfpathlineto{\pgfqpoint{1.143881in}{0.726551in}}%
\pgfpathlineto{\pgfqpoint{1.144178in}{0.726550in}}%
\pgfpathlineto{\pgfqpoint{1.144476in}{0.726549in}}%
\pgfpathlineto{\pgfqpoint{1.144773in}{0.726549in}}%
\pgfpathlineto{\pgfqpoint{1.145071in}{0.726548in}}%
\pgfpathlineto{\pgfqpoint{1.145368in}{0.726547in}}%
\pgfpathlineto{\pgfqpoint{1.145666in}{0.726546in}}%
\pgfpathlineto{\pgfqpoint{1.145963in}{0.726545in}}%
\pgfpathlineto{\pgfqpoint{1.146261in}{0.726544in}}%
\pgfpathlineto{\pgfqpoint{1.146558in}{0.726543in}}%
\pgfpathlineto{\pgfqpoint{1.146856in}{0.726543in}}%
\pgfpathlineto{\pgfqpoint{1.147153in}{0.726542in}}%
\pgfpathlineto{\pgfqpoint{1.147451in}{0.726541in}}%
\pgfpathlineto{\pgfqpoint{1.147748in}{0.726540in}}%
\pgfpathlineto{\pgfqpoint{1.148046in}{0.726539in}}%
\pgfpathlineto{\pgfqpoint{1.148343in}{0.726538in}}%
\pgfpathlineto{\pgfqpoint{1.148641in}{0.726538in}}%
\pgfpathlineto{\pgfqpoint{1.148938in}{0.726537in}}%
\pgfpathlineto{\pgfqpoint{1.149236in}{0.726536in}}%
\pgfpathlineto{\pgfqpoint{1.149533in}{0.726535in}}%
\pgfpathlineto{\pgfqpoint{1.149830in}{0.726534in}}%
\pgfpathlineto{\pgfqpoint{1.150128in}{0.726533in}}%
\pgfpathlineto{\pgfqpoint{1.150425in}{0.726532in}}%
\pgfpathlineto{\pgfqpoint{1.150723in}{0.726532in}}%
\pgfpathlineto{\pgfqpoint{1.151020in}{0.726531in}}%
\pgfpathlineto{\pgfqpoint{1.151318in}{0.726530in}}%
\pgfpathlineto{\pgfqpoint{1.151615in}{0.726529in}}%
\pgfpathlineto{\pgfqpoint{1.151913in}{0.726528in}}%
\pgfpathlineto{\pgfqpoint{1.152210in}{0.726527in}}%
\pgfpathlineto{\pgfqpoint{1.152508in}{0.726527in}}%
\pgfpathlineto{\pgfqpoint{1.152805in}{0.726526in}}%
\pgfpathlineto{\pgfqpoint{1.153103in}{0.726525in}}%
\pgfpathlineto{\pgfqpoint{1.153400in}{0.726524in}}%
\pgfpathlineto{\pgfqpoint{1.153698in}{0.726523in}}%
\pgfpathlineto{\pgfqpoint{1.153995in}{0.726522in}}%
\pgfpathlineto{\pgfqpoint{1.154293in}{0.726521in}}%
\pgfpathlineto{\pgfqpoint{1.154590in}{0.726521in}}%
\pgfpathlineto{\pgfqpoint{1.154888in}{0.726520in}}%
\pgfpathlineto{\pgfqpoint{1.155185in}{0.726519in}}%
\pgfpathlineto{\pgfqpoint{1.155483in}{0.726518in}}%
\pgfpathlineto{\pgfqpoint{1.155780in}{0.726517in}}%
\pgfpathlineto{\pgfqpoint{1.156078in}{0.726516in}}%
\pgfpathlineto{\pgfqpoint{1.156375in}{0.726516in}}%
\pgfpathlineto{\pgfqpoint{1.156672in}{0.726515in}}%
\pgfpathlineto{\pgfqpoint{1.156970in}{0.726514in}}%
\pgfpathlineto{\pgfqpoint{1.157267in}{0.726513in}}%
\pgfpathlineto{\pgfqpoint{1.157565in}{0.726512in}}%
\pgfpathlineto{\pgfqpoint{1.157862in}{0.726511in}}%
\pgfpathlineto{\pgfqpoint{1.158160in}{0.726510in}}%
\pgfpathlineto{\pgfqpoint{1.158457in}{0.726510in}}%
\pgfpathlineto{\pgfqpoint{1.158755in}{0.726509in}}%
\pgfpathlineto{\pgfqpoint{1.159052in}{0.726508in}}%
\pgfpathlineto{\pgfqpoint{1.159350in}{0.726507in}}%
\pgfpathlineto{\pgfqpoint{1.159647in}{0.726506in}}%
\pgfpathlineto{\pgfqpoint{1.159945in}{0.726505in}}%
\pgfpathlineto{\pgfqpoint{1.160242in}{0.726505in}}%
\pgfpathlineto{\pgfqpoint{1.160540in}{0.726504in}}%
\pgfpathlineto{\pgfqpoint{1.160837in}{0.726503in}}%
\pgfpathlineto{\pgfqpoint{1.161135in}{0.726502in}}%
\pgfpathlineto{\pgfqpoint{1.161432in}{0.726501in}}%
\pgfpathlineto{\pgfqpoint{1.161730in}{0.726500in}}%
\pgfpathlineto{\pgfqpoint{1.162027in}{0.726499in}}%
\pgfpathlineto{\pgfqpoint{1.162325in}{0.726499in}}%
\pgfpathlineto{\pgfqpoint{1.162622in}{0.726498in}}%
\pgfpathlineto{\pgfqpoint{1.162919in}{0.726497in}}%
\pgfpathlineto{\pgfqpoint{1.163217in}{0.726496in}}%
\pgfpathlineto{\pgfqpoint{1.163514in}{0.726495in}}%
\pgfpathlineto{\pgfqpoint{1.163812in}{0.726494in}}%
\pgfpathlineto{\pgfqpoint{1.164109in}{0.726494in}}%
\pgfpathlineto{\pgfqpoint{1.164407in}{0.726493in}}%
\pgfpathlineto{\pgfqpoint{1.164704in}{0.726492in}}%
\pgfpathlineto{\pgfqpoint{1.165002in}{0.726491in}}%
\pgfpathlineto{\pgfqpoint{1.165299in}{0.726490in}}%
\pgfpathlineto{\pgfqpoint{1.165597in}{0.726489in}}%
\pgfpathlineto{\pgfqpoint{1.165894in}{0.726488in}}%
\pgfpathlineto{\pgfqpoint{1.166192in}{0.726488in}}%
\pgfpathlineto{\pgfqpoint{1.166489in}{0.726487in}}%
\pgfpathlineto{\pgfqpoint{1.166787in}{0.726486in}}%
\pgfpathlineto{\pgfqpoint{1.167084in}{0.726485in}}%
\pgfpathlineto{\pgfqpoint{1.167382in}{0.726484in}}%
\pgfpathlineto{\pgfqpoint{1.167679in}{0.726483in}}%
\pgfpathlineto{\pgfqpoint{1.167977in}{0.726483in}}%
\pgfpathlineto{\pgfqpoint{1.168274in}{0.726482in}}%
\pgfpathlineto{\pgfqpoint{1.168572in}{0.726481in}}%
\pgfpathlineto{\pgfqpoint{1.168869in}{0.726480in}}%
\pgfpathlineto{\pgfqpoint{1.169167in}{0.726479in}}%
\pgfpathlineto{\pgfqpoint{1.169464in}{0.726478in}}%
\pgfpathlineto{\pgfqpoint{1.169761in}{0.726477in}}%
\pgfpathlineto{\pgfqpoint{1.170059in}{0.726477in}}%
\pgfpathlineto{\pgfqpoint{1.170356in}{0.726476in}}%
\pgfpathlineto{\pgfqpoint{1.170654in}{0.726475in}}%
\pgfpathlineto{\pgfqpoint{1.170951in}{0.726474in}}%
\pgfpathlineto{\pgfqpoint{1.171249in}{0.726473in}}%
\pgfpathlineto{\pgfqpoint{1.171546in}{0.726472in}}%
\pgfpathlineto{\pgfqpoint{1.171844in}{0.726472in}}%
\pgfpathlineto{\pgfqpoint{1.172141in}{0.726471in}}%
\pgfpathlineto{\pgfqpoint{1.172439in}{0.726470in}}%
\pgfpathlineto{\pgfqpoint{1.172736in}{0.726469in}}%
\pgfpathlineto{\pgfqpoint{1.173034in}{0.726468in}}%
\pgfpathlineto{\pgfqpoint{1.173331in}{0.726467in}}%
\pgfpathlineto{\pgfqpoint{1.173629in}{0.726467in}}%
\pgfpathlineto{\pgfqpoint{1.173926in}{0.726466in}}%
\pgfpathlineto{\pgfqpoint{1.174224in}{0.726465in}}%
\pgfpathlineto{\pgfqpoint{1.174521in}{0.726464in}}%
\pgfpathlineto{\pgfqpoint{1.174819in}{0.726463in}}%
\pgfpathlineto{\pgfqpoint{1.175116in}{0.726462in}}%
\pgfpathlineto{\pgfqpoint{1.175414in}{0.726461in}}%
\pgfpathlineto{\pgfqpoint{1.175711in}{0.726461in}}%
\pgfpathlineto{\pgfqpoint{1.176009in}{0.726460in}}%
\pgfpathlineto{\pgfqpoint{1.176306in}{0.726459in}}%
\pgfpathlineto{\pgfqpoint{1.176603in}{0.726458in}}%
\pgfpathlineto{\pgfqpoint{1.176901in}{0.726457in}}%
\pgfpathlineto{\pgfqpoint{1.177198in}{0.726456in}}%
\pgfpathlineto{\pgfqpoint{1.177496in}{0.726456in}}%
\pgfpathlineto{\pgfqpoint{1.177793in}{0.726455in}}%
\pgfpathlineto{\pgfqpoint{1.178091in}{0.726454in}}%
\pgfpathlineto{\pgfqpoint{1.178388in}{0.726453in}}%
\pgfpathlineto{\pgfqpoint{1.178686in}{0.726452in}}%
\pgfpathlineto{\pgfqpoint{1.178983in}{0.726451in}}%
\pgfpathlineto{\pgfqpoint{1.179281in}{0.726450in}}%
\pgfpathlineto{\pgfqpoint{1.179578in}{0.726450in}}%
\pgfpathlineto{\pgfqpoint{1.179876in}{0.726449in}}%
\pgfpathlineto{\pgfqpoint{1.180173in}{0.726448in}}%
\pgfpathlineto{\pgfqpoint{1.180471in}{0.726447in}}%
\pgfpathlineto{\pgfqpoint{1.180768in}{0.726446in}}%
\pgfpathlineto{\pgfqpoint{1.181066in}{0.726445in}}%
\pgfpathlineto{\pgfqpoint{1.181363in}{0.726445in}}%
\pgfpathlineto{\pgfqpoint{1.181661in}{0.726444in}}%
\pgfpathlineto{\pgfqpoint{1.181958in}{0.726443in}}%
\pgfpathlineto{\pgfqpoint{1.182256in}{0.726442in}}%
\pgfpathlineto{\pgfqpoint{1.182553in}{0.726441in}}%
\pgfpathlineto{\pgfqpoint{1.182850in}{0.726440in}}%
\pgfpathlineto{\pgfqpoint{1.183148in}{0.726439in}}%
\pgfpathlineto{\pgfqpoint{1.183445in}{0.726439in}}%
\pgfpathlineto{\pgfqpoint{1.183743in}{0.726438in}}%
\pgfpathlineto{\pgfqpoint{1.184040in}{0.726437in}}%
\pgfpathlineto{\pgfqpoint{1.184338in}{0.726436in}}%
\pgfpathlineto{\pgfqpoint{1.184635in}{0.726435in}}%
\pgfpathlineto{\pgfqpoint{1.184933in}{0.726434in}}%
\pgfpathlineto{\pgfqpoint{1.185230in}{0.726434in}}%
\pgfpathlineto{\pgfqpoint{1.185528in}{0.726433in}}%
\pgfpathlineto{\pgfqpoint{1.185825in}{0.726432in}}%
\pgfpathlineto{\pgfqpoint{1.186123in}{0.726431in}}%
\pgfpathlineto{\pgfqpoint{1.186420in}{0.726430in}}%
\pgfpathlineto{\pgfqpoint{1.186718in}{0.726429in}}%
\pgfpathlineto{\pgfqpoint{1.187015in}{0.726428in}}%
\pgfpathlineto{\pgfqpoint{1.187313in}{0.726428in}}%
\pgfpathlineto{\pgfqpoint{1.187610in}{0.726427in}}%
\pgfpathlineto{\pgfqpoint{1.187908in}{0.726426in}}%
\pgfpathlineto{\pgfqpoint{1.188205in}{0.726425in}}%
\pgfpathlineto{\pgfqpoint{1.188503in}{0.726424in}}%
\pgfpathlineto{\pgfqpoint{1.188800in}{0.726423in}}%
\pgfpathlineto{\pgfqpoint{1.189098in}{0.726423in}}%
\pgfpathlineto{\pgfqpoint{1.189395in}{0.726422in}}%
\pgfpathlineto{\pgfqpoint{1.189692in}{0.726421in}}%
\pgfpathlineto{\pgfqpoint{1.189990in}{0.726420in}}%
\pgfpathlineto{\pgfqpoint{1.190287in}{0.726419in}}%
\pgfpathlineto{\pgfqpoint{1.190585in}{0.726418in}}%
\pgfpathlineto{\pgfqpoint{1.190882in}{0.726417in}}%
\pgfpathlineto{\pgfqpoint{1.191180in}{0.726417in}}%
\pgfpathlineto{\pgfqpoint{1.191477in}{0.726416in}}%
\pgfpathlineto{\pgfqpoint{1.191775in}{0.726415in}}%
\pgfpathlineto{\pgfqpoint{1.192072in}{0.726414in}}%
\pgfpathlineto{\pgfqpoint{1.192370in}{0.726413in}}%
\pgfpathlineto{\pgfqpoint{1.192667in}{0.726412in}}%
\pgfpathlineto{\pgfqpoint{1.192965in}{0.726412in}}%
\pgfpathlineto{\pgfqpoint{1.193262in}{0.726411in}}%
\pgfpathlineto{\pgfqpoint{1.193560in}{0.726410in}}%
\pgfpathlineto{\pgfqpoint{1.193857in}{0.726409in}}%
\pgfpathlineto{\pgfqpoint{1.194155in}{0.726408in}}%
\pgfpathlineto{\pgfqpoint{1.194452in}{0.726407in}}%
\pgfpathlineto{\pgfqpoint{1.194750in}{0.726406in}}%
\pgfpathlineto{\pgfqpoint{1.195047in}{0.726406in}}%
\pgfpathlineto{\pgfqpoint{1.195345in}{0.726405in}}%
\pgfpathlineto{\pgfqpoint{1.195642in}{0.726404in}}%
\pgfpathlineto{\pgfqpoint{1.195940in}{0.726403in}}%
\pgfpathlineto{\pgfqpoint{1.196237in}{0.726402in}}%
\pgfpathlineto{\pgfqpoint{1.196534in}{0.726401in}}%
\pgfpathlineto{\pgfqpoint{1.196832in}{0.726401in}}%
\pgfpathlineto{\pgfqpoint{1.197129in}{0.726400in}}%
\pgfpathlineto{\pgfqpoint{1.197427in}{0.726399in}}%
\pgfpathlineto{\pgfqpoint{1.197724in}{0.726398in}}%
\pgfpathlineto{\pgfqpoint{1.198022in}{0.726397in}}%
\pgfpathlineto{\pgfqpoint{1.198319in}{0.726396in}}%
\pgfpathlineto{\pgfqpoint{1.198617in}{0.726395in}}%
\pgfpathlineto{\pgfqpoint{1.198914in}{0.726395in}}%
\pgfpathlineto{\pgfqpoint{1.199212in}{0.726394in}}%
\pgfpathlineto{\pgfqpoint{1.199509in}{0.726393in}}%
\pgfpathlineto{\pgfqpoint{1.199807in}{0.726392in}}%
\pgfpathlineto{\pgfqpoint{1.200104in}{0.726391in}}%
\pgfpathlineto{\pgfqpoint{1.200402in}{0.726390in}}%
\pgfpathlineto{\pgfqpoint{1.200699in}{0.726390in}}%
\pgfpathlineto{\pgfqpoint{1.200997in}{0.726389in}}%
\pgfpathlineto{\pgfqpoint{1.201294in}{0.726388in}}%
\pgfpathlineto{\pgfqpoint{1.201592in}{0.726387in}}%
\pgfpathlineto{\pgfqpoint{1.201889in}{0.726386in}}%
\pgfpathlineto{\pgfqpoint{1.202187in}{0.726385in}}%
\pgfpathlineto{\pgfqpoint{1.202484in}{0.726385in}}%
\pgfpathlineto{\pgfqpoint{1.202781in}{0.726384in}}%
\pgfpathlineto{\pgfqpoint{1.203079in}{0.726383in}}%
\pgfpathlineto{\pgfqpoint{1.203376in}{0.726382in}}%
\pgfpathlineto{\pgfqpoint{1.203674in}{0.726381in}}%
\pgfpathlineto{\pgfqpoint{1.203971in}{0.726380in}}%
\pgfpathlineto{\pgfqpoint{1.204269in}{0.726379in}}%
\pgfpathlineto{\pgfqpoint{1.204566in}{0.726379in}}%
\pgfpathlineto{\pgfqpoint{1.204864in}{0.726378in}}%
\pgfpathlineto{\pgfqpoint{1.205161in}{0.726377in}}%
\pgfpathlineto{\pgfqpoint{1.205459in}{0.726376in}}%
\pgfpathlineto{\pgfqpoint{1.205756in}{0.726375in}}%
\pgfpathlineto{\pgfqpoint{1.206054in}{0.726374in}}%
\pgfpathlineto{\pgfqpoint{1.206351in}{0.726374in}}%
\pgfpathlineto{\pgfqpoint{1.206649in}{0.726373in}}%
\pgfpathlineto{\pgfqpoint{1.206946in}{0.726372in}}%
\pgfpathlineto{\pgfqpoint{1.207244in}{0.726371in}}%
\pgfpathlineto{\pgfqpoint{1.207541in}{0.726370in}}%
\pgfpathlineto{\pgfqpoint{1.207839in}{0.726369in}}%
\pgfpathlineto{\pgfqpoint{1.208136in}{0.726368in}}%
\pgfpathlineto{\pgfqpoint{1.208434in}{0.726368in}}%
\pgfpathlineto{\pgfqpoint{1.208731in}{0.726367in}}%
\pgfpathlineto{\pgfqpoint{1.209029in}{0.726366in}}%
\pgfpathlineto{\pgfqpoint{1.209326in}{0.726365in}}%
\pgfpathlineto{\pgfqpoint{1.209623in}{0.726364in}}%
\pgfpathlineto{\pgfqpoint{1.209921in}{0.726363in}}%
\pgfpathlineto{\pgfqpoint{1.210218in}{0.726363in}}%
\pgfpathlineto{\pgfqpoint{1.210516in}{0.726362in}}%
\pgfpathlineto{\pgfqpoint{1.210813in}{0.726361in}}%
\pgfpathlineto{\pgfqpoint{1.211111in}{0.726360in}}%
\pgfpathlineto{\pgfqpoint{1.211408in}{0.726359in}}%
\pgfpathlineto{\pgfqpoint{1.211706in}{0.726358in}}%
\pgfpathlineto{\pgfqpoint{1.212003in}{0.726357in}}%
\pgfpathlineto{\pgfqpoint{1.212301in}{0.726357in}}%
\pgfpathlineto{\pgfqpoint{1.212598in}{0.726356in}}%
\pgfpathlineto{\pgfqpoint{1.212896in}{0.726355in}}%
\pgfpathlineto{\pgfqpoint{1.213193in}{0.726354in}}%
\pgfpathlineto{\pgfqpoint{1.213491in}{0.726353in}}%
\pgfpathlineto{\pgfqpoint{1.213788in}{0.726352in}}%
\pgfpathlineto{\pgfqpoint{1.214086in}{0.726352in}}%
\pgfpathlineto{\pgfqpoint{1.214383in}{0.726351in}}%
\pgfpathlineto{\pgfqpoint{1.214681in}{0.726350in}}%
\pgfpathlineto{\pgfqpoint{1.214978in}{0.726349in}}%
\pgfpathlineto{\pgfqpoint{1.215276in}{0.726348in}}%
\pgfpathlineto{\pgfqpoint{1.215573in}{0.726347in}}%
\pgfpathlineto{\pgfqpoint{1.215871in}{0.726346in}}%
\pgfpathlineto{\pgfqpoint{1.216168in}{0.726346in}}%
\pgfpathlineto{\pgfqpoint{1.216465in}{0.726345in}}%
\pgfpathlineto{\pgfqpoint{1.216763in}{0.726344in}}%
\pgfpathlineto{\pgfqpoint{1.217060in}{0.726343in}}%
\pgfpathlineto{\pgfqpoint{1.217358in}{0.726342in}}%
\pgfpathlineto{\pgfqpoint{1.217655in}{0.726341in}}%
\pgfpathlineto{\pgfqpoint{1.217953in}{0.726341in}}%
\pgfpathlineto{\pgfqpoint{1.218250in}{0.726340in}}%
\pgfpathlineto{\pgfqpoint{1.218548in}{0.726339in}}%
\pgfpathlineto{\pgfqpoint{1.218845in}{0.726338in}}%
\pgfpathlineto{\pgfqpoint{1.219143in}{0.726337in}}%
\pgfpathlineto{\pgfqpoint{1.219440in}{0.726336in}}%
\pgfpathlineto{\pgfqpoint{1.219738in}{0.726335in}}%
\pgfpathlineto{\pgfqpoint{1.220035in}{0.726335in}}%
\pgfpathlineto{\pgfqpoint{1.220333in}{0.726334in}}%
\pgfpathlineto{\pgfqpoint{1.220630in}{0.726333in}}%
\pgfpathlineto{\pgfqpoint{1.220928in}{0.726332in}}%
\pgfpathlineto{\pgfqpoint{1.221225in}{0.726331in}}%
\pgfpathlineto{\pgfqpoint{1.221523in}{0.726330in}}%
\pgfpathlineto{\pgfqpoint{1.221820in}{0.726330in}}%
\pgfpathlineto{\pgfqpoint{1.222118in}{0.726329in}}%
\pgfpathlineto{\pgfqpoint{1.222415in}{0.726328in}}%
\pgfpathlineto{\pgfqpoint{1.222712in}{0.726327in}}%
\pgfpathlineto{\pgfqpoint{1.223010in}{0.726326in}}%
\pgfpathlineto{\pgfqpoint{1.223307in}{0.726325in}}%
\pgfpathlineto{\pgfqpoint{1.223605in}{0.726324in}}%
\pgfpathlineto{\pgfqpoint{1.223902in}{0.726324in}}%
\pgfpathlineto{\pgfqpoint{1.224200in}{0.726323in}}%
\pgfpathlineto{\pgfqpoint{1.224497in}{0.726322in}}%
\pgfpathlineto{\pgfqpoint{1.224795in}{0.726321in}}%
\pgfpathlineto{\pgfqpoint{1.225092in}{0.726320in}}%
\pgfpathlineto{\pgfqpoint{1.225390in}{0.726319in}}%
\pgfpathlineto{\pgfqpoint{1.225687in}{0.726319in}}%
\pgfpathlineto{\pgfqpoint{1.225985in}{0.726318in}}%
\pgfpathlineto{\pgfqpoint{1.226282in}{0.726317in}}%
\pgfpathlineto{\pgfqpoint{1.226580in}{0.726316in}}%
\pgfpathlineto{\pgfqpoint{1.226877in}{0.726315in}}%
\pgfpathlineto{\pgfqpoint{1.227175in}{0.726314in}}%
\pgfpathlineto{\pgfqpoint{1.227472in}{0.726313in}}%
\pgfpathlineto{\pgfqpoint{1.227770in}{0.726313in}}%
\pgfpathlineto{\pgfqpoint{1.228067in}{0.726312in}}%
\pgfpathlineto{\pgfqpoint{1.228365in}{0.726311in}}%
\pgfpathlineto{\pgfqpoint{1.228662in}{0.726310in}}%
\pgfpathlineto{\pgfqpoint{1.228960in}{0.726309in}}%
\pgfpathlineto{\pgfqpoint{1.229257in}{0.726308in}}%
\pgfpathlineto{\pgfqpoint{1.229554in}{0.726308in}}%
\pgfpathlineto{\pgfqpoint{1.229852in}{0.726307in}}%
\pgfpathlineto{\pgfqpoint{1.230149in}{0.726306in}}%
\pgfpathlineto{\pgfqpoint{1.230447in}{0.726305in}}%
\pgfpathlineto{\pgfqpoint{1.230744in}{0.726304in}}%
\pgfpathlineto{\pgfqpoint{1.231042in}{0.726303in}}%
\pgfpathlineto{\pgfqpoint{1.231339in}{0.726303in}}%
\pgfpathlineto{\pgfqpoint{1.231637in}{0.726302in}}%
\pgfpathlineto{\pgfqpoint{1.231934in}{0.726301in}}%
\pgfpathlineto{\pgfqpoint{1.232232in}{0.726300in}}%
\pgfpathlineto{\pgfqpoint{1.232529in}{0.726299in}}%
\pgfpathlineto{\pgfqpoint{1.232827in}{0.726298in}}%
\pgfpathlineto{\pgfqpoint{1.233124in}{0.726297in}}%
\pgfpathlineto{\pgfqpoint{1.233422in}{0.726297in}}%
\pgfpathlineto{\pgfqpoint{1.233719in}{0.726296in}}%
\pgfpathlineto{\pgfqpoint{1.234017in}{0.726295in}}%
\pgfpathlineto{\pgfqpoint{1.234314in}{0.726294in}}%
\pgfpathlineto{\pgfqpoint{1.234612in}{0.726293in}}%
\pgfpathlineto{\pgfqpoint{1.234909in}{0.726292in}}%
\pgfpathlineto{\pgfqpoint{1.235207in}{0.726291in}}%
\pgfpathlineto{\pgfqpoint{1.235504in}{0.726290in}}%
\pgfpathlineto{\pgfqpoint{1.235802in}{0.726289in}}%
\pgfpathlineto{\pgfqpoint{1.236099in}{0.726288in}}%
\pgfpathlineto{\pgfqpoint{1.236396in}{0.726287in}}%
\pgfpathlineto{\pgfqpoint{1.236694in}{0.726286in}}%
\pgfpathlineto{\pgfqpoint{1.236991in}{0.726285in}}%
\pgfpathlineto{\pgfqpoint{1.237289in}{0.726284in}}%
\pgfpathlineto{\pgfqpoint{1.237586in}{0.726283in}}%
\pgfpathlineto{\pgfqpoint{1.237884in}{0.726282in}}%
\pgfpathlineto{\pgfqpoint{1.238181in}{0.726281in}}%
\pgfpathlineto{\pgfqpoint{1.238479in}{0.726280in}}%
\pgfpathlineto{\pgfqpoint{1.238776in}{0.726279in}}%
\pgfpathlineto{\pgfqpoint{1.239074in}{0.726278in}}%
\pgfpathlineto{\pgfqpoint{1.239371in}{0.726277in}}%
\pgfpathlineto{\pgfqpoint{1.239669in}{0.726276in}}%
\pgfpathlineto{\pgfqpoint{1.239966in}{0.726275in}}%
\pgfpathlineto{\pgfqpoint{1.240264in}{0.726274in}}%
\pgfpathlineto{\pgfqpoint{1.240561in}{0.726273in}}%
\pgfpathlineto{\pgfqpoint{1.240859in}{0.726272in}}%
\pgfpathlineto{\pgfqpoint{1.241156in}{0.726271in}}%
\pgfpathlineto{\pgfqpoint{1.241454in}{0.726270in}}%
\pgfpathlineto{\pgfqpoint{1.241751in}{0.726269in}}%
\pgfpathlineto{\pgfqpoint{1.242049in}{0.726268in}}%
\pgfpathlineto{\pgfqpoint{1.242346in}{0.726267in}}%
\pgfpathlineto{\pgfqpoint{1.242643in}{0.726266in}}%
\pgfpathlineto{\pgfqpoint{1.242941in}{0.726265in}}%
\pgfpathlineto{\pgfqpoint{1.243238in}{0.726264in}}%
\pgfpathlineto{\pgfqpoint{1.243536in}{0.726263in}}%
\pgfpathlineto{\pgfqpoint{1.243833in}{0.726262in}}%
\pgfpathlineto{\pgfqpoint{1.244131in}{0.726261in}}%
\pgfpathlineto{\pgfqpoint{1.244428in}{0.726260in}}%
\pgfpathlineto{\pgfqpoint{1.244726in}{0.726259in}}%
\pgfpathlineto{\pgfqpoint{1.245023in}{0.726258in}}%
\pgfpathlineto{\pgfqpoint{1.245321in}{0.726257in}}%
\pgfpathlineto{\pgfqpoint{1.245618in}{0.726256in}}%
\pgfpathlineto{\pgfqpoint{1.245916in}{0.726255in}}%
\pgfpathlineto{\pgfqpoint{1.246213in}{0.726254in}}%
\pgfpathlineto{\pgfqpoint{1.246511in}{0.726253in}}%
\pgfpathlineto{\pgfqpoint{1.246808in}{0.726252in}}%
\pgfpathlineto{\pgfqpoint{1.247106in}{0.726251in}}%
\pgfpathlineto{\pgfqpoint{1.247403in}{0.726250in}}%
\pgfpathlineto{\pgfqpoint{1.247701in}{0.726249in}}%
\pgfpathlineto{\pgfqpoint{1.247998in}{0.726248in}}%
\pgfpathlineto{\pgfqpoint{1.248296in}{0.726247in}}%
\pgfpathlineto{\pgfqpoint{1.248593in}{0.726246in}}%
\pgfpathlineto{\pgfqpoint{1.248891in}{0.726245in}}%
\pgfpathlineto{\pgfqpoint{1.249188in}{0.726244in}}%
\pgfpathlineto{\pgfqpoint{1.249485in}{0.726243in}}%
\pgfpathlineto{\pgfqpoint{1.249783in}{0.726242in}}%
\pgfpathlineto{\pgfqpoint{1.250080in}{0.726241in}}%
\pgfpathlineto{\pgfqpoint{1.250378in}{0.726240in}}%
\pgfpathlineto{\pgfqpoint{1.250675in}{0.726239in}}%
\pgfpathlineto{\pgfqpoint{1.250973in}{0.726238in}}%
\pgfpathlineto{\pgfqpoint{1.251270in}{0.726237in}}%
\pgfpathlineto{\pgfqpoint{1.251568in}{0.726235in}}%
\pgfpathlineto{\pgfqpoint{1.251865in}{0.726233in}}%
\pgfpathlineto{\pgfqpoint{1.252163in}{0.726231in}}%
\pgfpathlineto{\pgfqpoint{1.252460in}{0.726229in}}%
\pgfpathlineto{\pgfqpoint{1.252758in}{0.726227in}}%
\pgfpathlineto{\pgfqpoint{1.253055in}{0.726225in}}%
\pgfpathlineto{\pgfqpoint{1.253353in}{0.726223in}}%
\pgfpathlineto{\pgfqpoint{1.253650in}{0.726221in}}%
\pgfpathlineto{\pgfqpoint{1.253948in}{0.726219in}}%
\pgfpathlineto{\pgfqpoint{1.254245in}{0.726217in}}%
\pgfpathlineto{\pgfqpoint{1.254543in}{0.726215in}}%
\pgfpathlineto{\pgfqpoint{1.254840in}{0.726213in}}%
\pgfpathlineto{\pgfqpoint{1.255138in}{0.726211in}}%
\pgfpathlineto{\pgfqpoint{1.255435in}{0.726209in}}%
\pgfpathlineto{\pgfqpoint{1.255733in}{0.726208in}}%
\pgfpathlineto{\pgfqpoint{1.256030in}{0.726206in}}%
\pgfpathlineto{\pgfqpoint{1.256327in}{0.726204in}}%
\pgfpathlineto{\pgfqpoint{1.256625in}{0.726202in}}%
\pgfpathlineto{\pgfqpoint{1.256922in}{0.726200in}}%
\pgfpathlineto{\pgfqpoint{1.257220in}{0.726198in}}%
\pgfpathlineto{\pgfqpoint{1.257517in}{0.726196in}}%
\pgfpathlineto{\pgfqpoint{1.257815in}{0.726194in}}%
\pgfpathlineto{\pgfqpoint{1.258112in}{0.726192in}}%
\pgfpathlineto{\pgfqpoint{1.258410in}{0.726190in}}%
\pgfpathlineto{\pgfqpoint{1.258707in}{0.726188in}}%
\pgfpathlineto{\pgfqpoint{1.259005in}{0.726185in}}%
\pgfpathlineto{\pgfqpoint{1.259302in}{0.726181in}}%
\pgfpathlineto{\pgfqpoint{1.259600in}{0.726177in}}%
\pgfpathlineto{\pgfqpoint{1.259897in}{0.726172in}}%
\pgfpathlineto{\pgfqpoint{1.260195in}{0.726168in}}%
\pgfpathlineto{\pgfqpoint{1.260492in}{0.726164in}}%
\pgfpathlineto{\pgfqpoint{1.260790in}{0.726160in}}%
\pgfpathlineto{\pgfqpoint{1.261087in}{0.726155in}}%
\pgfpathlineto{\pgfqpoint{1.261385in}{0.726151in}}%
\pgfpathlineto{\pgfqpoint{1.261682in}{0.726147in}}%
\pgfpathlineto{\pgfqpoint{1.261980in}{0.726143in}}%
\pgfpathlineto{\pgfqpoint{1.262277in}{0.726138in}}%
\pgfpathlineto{\pgfqpoint{1.262574in}{0.726134in}}%
\pgfpathlineto{\pgfqpoint{1.262872in}{0.726130in}}%
\pgfpathlineto{\pgfqpoint{1.263169in}{0.726126in}}%
\pgfpathlineto{\pgfqpoint{1.263467in}{0.726121in}}%
\pgfpathlineto{\pgfqpoint{1.263764in}{0.726117in}}%
\pgfpathlineto{\pgfqpoint{1.264062in}{0.726113in}}%
\pgfpathlineto{\pgfqpoint{1.264359in}{0.726109in}}%
\pgfpathlineto{\pgfqpoint{1.264657in}{0.726104in}}%
\pgfpathlineto{\pgfqpoint{1.264954in}{0.726100in}}%
\pgfpathlineto{\pgfqpoint{1.265252in}{0.726096in}}%
\pgfpathlineto{\pgfqpoint{1.265549in}{0.726092in}}%
\pgfpathlineto{\pgfqpoint{1.265847in}{0.726088in}}%
\pgfpathlineto{\pgfqpoint{1.266144in}{0.726083in}}%
\pgfpathlineto{\pgfqpoint{1.266442in}{0.726079in}}%
\pgfpathlineto{\pgfqpoint{1.266739in}{0.726075in}}%
\pgfpathlineto{\pgfqpoint{1.267037in}{0.726071in}}%
\pgfpathlineto{\pgfqpoint{1.267334in}{0.726066in}}%
\pgfpathlineto{\pgfqpoint{1.267632in}{0.726062in}}%
\pgfpathlineto{\pgfqpoint{1.267929in}{0.726058in}}%
\pgfpathlineto{\pgfqpoint{1.268227in}{0.726054in}}%
\pgfpathlineto{\pgfqpoint{1.268524in}{0.726049in}}%
\pgfpathlineto{\pgfqpoint{1.268822in}{0.726045in}}%
\pgfpathlineto{\pgfqpoint{1.269119in}{0.726041in}}%
\pgfpathlineto{\pgfqpoint{1.269416in}{0.726037in}}%
\pgfpathlineto{\pgfqpoint{1.269714in}{0.726032in}}%
\pgfpathlineto{\pgfqpoint{1.270011in}{0.726028in}}%
\pgfpathlineto{\pgfqpoint{1.270309in}{0.726024in}}%
\pgfpathlineto{\pgfqpoint{1.270606in}{0.726020in}}%
\pgfpathlineto{\pgfqpoint{1.270904in}{0.726015in}}%
\pgfpathlineto{\pgfqpoint{1.271201in}{0.726011in}}%
\pgfpathlineto{\pgfqpoint{1.271499in}{0.726007in}}%
\pgfpathlineto{\pgfqpoint{1.271796in}{0.726003in}}%
\pgfpathlineto{\pgfqpoint{1.272094in}{0.725999in}}%
\pgfpathlineto{\pgfqpoint{1.272391in}{0.725994in}}%
\pgfpathlineto{\pgfqpoint{1.272689in}{0.725990in}}%
\pgfpathlineto{\pgfqpoint{1.272986in}{0.725986in}}%
\pgfpathlineto{\pgfqpoint{1.273284in}{0.725982in}}%
\pgfpathlineto{\pgfqpoint{1.273581in}{0.725977in}}%
\pgfpathlineto{\pgfqpoint{1.273879in}{0.725973in}}%
\pgfpathlineto{\pgfqpoint{1.274176in}{0.725969in}}%
\pgfpathlineto{\pgfqpoint{1.274474in}{0.725965in}}%
\pgfpathlineto{\pgfqpoint{1.274771in}{0.725960in}}%
\pgfpathlineto{\pgfqpoint{1.275069in}{0.725956in}}%
\pgfpathlineto{\pgfqpoint{1.275366in}{0.725952in}}%
\pgfpathlineto{\pgfqpoint{1.275664in}{0.725948in}}%
\pgfpathlineto{\pgfqpoint{1.275961in}{0.725943in}}%
\pgfpathlineto{\pgfqpoint{1.276258in}{0.725939in}}%
\pgfpathlineto{\pgfqpoint{1.276556in}{0.725935in}}%
\pgfpathlineto{\pgfqpoint{1.276853in}{0.725931in}}%
\pgfpathlineto{\pgfqpoint{1.277151in}{0.725926in}}%
\pgfpathlineto{\pgfqpoint{1.277448in}{0.725922in}}%
\pgfpathlineto{\pgfqpoint{1.277746in}{0.725918in}}%
\pgfpathlineto{\pgfqpoint{1.278043in}{0.725914in}}%
\pgfpathlineto{\pgfqpoint{1.278341in}{0.725909in}}%
\pgfpathlineto{\pgfqpoint{1.278638in}{0.725905in}}%
\pgfpathlineto{\pgfqpoint{1.278936in}{0.725901in}}%
\pgfpathlineto{\pgfqpoint{1.279233in}{0.725897in}}%
\pgfpathlineto{\pgfqpoint{1.279531in}{0.725893in}}%
\pgfpathlineto{\pgfqpoint{1.279828in}{0.725888in}}%
\pgfpathlineto{\pgfqpoint{1.280126in}{0.725884in}}%
\pgfpathlineto{\pgfqpoint{1.280423in}{0.725880in}}%
\pgfpathlineto{\pgfqpoint{1.280721in}{0.725876in}}%
\pgfpathlineto{\pgfqpoint{1.281018in}{0.725871in}}%
\pgfpathlineto{\pgfqpoint{1.281316in}{0.725867in}}%
\pgfpathlineto{\pgfqpoint{1.281613in}{0.725863in}}%
\pgfpathlineto{\pgfqpoint{1.281911in}{0.725859in}}%
\pgfpathlineto{\pgfqpoint{1.282208in}{0.725854in}}%
\pgfpathlineto{\pgfqpoint{1.282505in}{0.725850in}}%
\pgfpathlineto{\pgfqpoint{1.282803in}{0.725846in}}%
\pgfpathlineto{\pgfqpoint{1.283100in}{0.725842in}}%
\pgfpathlineto{\pgfqpoint{1.283398in}{0.725837in}}%
\pgfpathlineto{\pgfqpoint{1.283695in}{0.725833in}}%
\pgfpathlineto{\pgfqpoint{1.283993in}{0.725829in}}%
\pgfpathlineto{\pgfqpoint{1.284290in}{0.725825in}}%
\pgfpathlineto{\pgfqpoint{1.284588in}{0.725820in}}%
\pgfpathlineto{\pgfqpoint{1.284885in}{0.725816in}}%
\pgfpathlineto{\pgfqpoint{1.285183in}{0.725812in}}%
\pgfpathlineto{\pgfqpoint{1.285480in}{0.725808in}}%
\pgfpathlineto{\pgfqpoint{1.285778in}{0.725804in}}%
\pgfpathlineto{\pgfqpoint{1.286075in}{0.725799in}}%
\pgfpathlineto{\pgfqpoint{1.286373in}{0.725795in}}%
\pgfpathlineto{\pgfqpoint{1.286670in}{0.726094in}}%
\pgfpathlineto{\pgfqpoint{1.286968in}{0.727214in}}%
\pgfpathlineto{\pgfqpoint{1.287265in}{0.727208in}}%
\pgfpathlineto{\pgfqpoint{1.287563in}{0.727204in}}%
\pgfpathlineto{\pgfqpoint{1.287860in}{0.727199in}}%
\pgfpathlineto{\pgfqpoint{1.288158in}{0.727195in}}%
\pgfpathlineto{\pgfqpoint{1.288455in}{0.727191in}}%
\pgfpathlineto{\pgfqpoint{1.288753in}{0.727187in}}%
\pgfpathlineto{\pgfqpoint{1.289050in}{0.727182in}}%
\pgfpathlineto{\pgfqpoint{1.289347in}{0.727178in}}%
\pgfpathlineto{\pgfqpoint{1.289645in}{0.727174in}}%
\pgfpathlineto{\pgfqpoint{1.289942in}{0.727170in}}%
\pgfpathlineto{\pgfqpoint{1.290240in}{0.727166in}}%
\pgfpathlineto{\pgfqpoint{1.290537in}{0.727161in}}%
\pgfpathlineto{\pgfqpoint{1.290835in}{0.727157in}}%
\pgfpathlineto{\pgfqpoint{1.291132in}{0.727153in}}%
\pgfpathlineto{\pgfqpoint{1.291430in}{0.727147in}}%
\pgfpathlineto{\pgfqpoint{1.291727in}{0.725289in}}%
\pgfpathlineto{\pgfqpoint{1.292025in}{0.725068in}}%
\pgfpathlineto{\pgfqpoint{1.292322in}{0.725063in}}%
\pgfpathlineto{\pgfqpoint{1.292620in}{0.725059in}}%
\pgfpathlineto{\pgfqpoint{1.292917in}{0.725055in}}%
\pgfpathlineto{\pgfqpoint{1.293215in}{0.725075in}}%
\pgfpathlineto{\pgfqpoint{1.293512in}{0.725157in}}%
\pgfpathlineto{\pgfqpoint{1.293810in}{0.725151in}}%
\pgfpathlineto{\pgfqpoint{1.294107in}{0.725143in}}%
\pgfpathlineto{\pgfqpoint{1.294405in}{0.725134in}}%
\pgfpathlineto{\pgfqpoint{1.294702in}{0.725126in}}%
\pgfpathlineto{\pgfqpoint{1.295000in}{0.725118in}}%
\pgfpathlineto{\pgfqpoint{1.295297in}{0.725110in}}%
\pgfpathlineto{\pgfqpoint{1.295595in}{0.725102in}}%
\pgfpathlineto{\pgfqpoint{1.295892in}{0.725094in}}%
\pgfpathlineto{\pgfqpoint{1.296189in}{0.725086in}}%
\pgfpathlineto{\pgfqpoint{1.296487in}{0.725078in}}%
\pgfpathlineto{\pgfqpoint{1.296784in}{0.725069in}}%
\pgfpathlineto{\pgfqpoint{1.297082in}{0.725061in}}%
\pgfpathlineto{\pgfqpoint{1.297379in}{0.725053in}}%
\pgfpathlineto{\pgfqpoint{1.297677in}{0.725045in}}%
\pgfpathlineto{\pgfqpoint{1.297974in}{0.725037in}}%
\pgfpathlineto{\pgfqpoint{1.298272in}{0.725029in}}%
\pgfpathlineto{\pgfqpoint{1.298569in}{0.725021in}}%
\pgfpathlineto{\pgfqpoint{1.298867in}{0.725013in}}%
\pgfpathlineto{\pgfqpoint{1.299164in}{0.725004in}}%
\pgfpathlineto{\pgfqpoint{1.299462in}{0.724996in}}%
\pgfpathlineto{\pgfqpoint{1.299759in}{0.724988in}}%
\pgfpathlineto{\pgfqpoint{1.300057in}{0.724980in}}%
\pgfpathlineto{\pgfqpoint{1.300354in}{0.724972in}}%
\pgfpathlineto{\pgfqpoint{1.300652in}{0.724965in}}%
\pgfpathlineto{\pgfqpoint{1.300949in}{0.724961in}}%
\pgfpathlineto{\pgfqpoint{1.301247in}{0.724957in}}%
\pgfpathlineto{\pgfqpoint{1.301544in}{0.724952in}}%
\pgfpathlineto{\pgfqpoint{1.301842in}{0.724948in}}%
\pgfpathlineto{\pgfqpoint{1.302139in}{0.724944in}}%
\pgfpathlineto{\pgfqpoint{1.302436in}{0.724940in}}%
\pgfpathlineto{\pgfqpoint{1.302734in}{0.724936in}}%
\pgfpathlineto{\pgfqpoint{1.303031in}{0.724931in}}%
\pgfpathlineto{\pgfqpoint{1.303329in}{0.724927in}}%
\pgfpathlineto{\pgfqpoint{1.303626in}{0.724923in}}%
\pgfpathlineto{\pgfqpoint{1.303924in}{0.724919in}}%
\pgfpathlineto{\pgfqpoint{1.304221in}{0.724914in}}%
\pgfpathlineto{\pgfqpoint{1.304519in}{0.724910in}}%
\pgfpathlineto{\pgfqpoint{1.304816in}{0.724906in}}%
\pgfpathlineto{\pgfqpoint{1.305114in}{0.724902in}}%
\pgfpathlineto{\pgfqpoint{1.305411in}{0.724897in}}%
\pgfpathlineto{\pgfqpoint{1.305709in}{0.724893in}}%
\pgfpathlineto{\pgfqpoint{1.306006in}{0.724889in}}%
\pgfpathlineto{\pgfqpoint{1.306304in}{0.724885in}}%
\pgfpathlineto{\pgfqpoint{1.306601in}{0.724880in}}%
\pgfpathlineto{\pgfqpoint{1.306899in}{0.724876in}}%
\pgfpathlineto{\pgfqpoint{1.307196in}{0.724872in}}%
\pgfpathlineto{\pgfqpoint{1.307494in}{0.724868in}}%
\pgfpathlineto{\pgfqpoint{1.307791in}{0.724863in}}%
\pgfpathlineto{\pgfqpoint{1.308089in}{0.724859in}}%
\pgfpathlineto{\pgfqpoint{1.308386in}{0.724855in}}%
\pgfpathlineto{\pgfqpoint{1.308684in}{0.724851in}}%
\pgfpathlineto{\pgfqpoint{1.308981in}{0.724847in}}%
\pgfpathlineto{\pgfqpoint{1.309278in}{0.724842in}}%
\pgfpathlineto{\pgfqpoint{1.309576in}{0.724838in}}%
\pgfpathlineto{\pgfqpoint{1.309873in}{0.724834in}}%
\pgfpathlineto{\pgfqpoint{1.310171in}{0.724830in}}%
\pgfpathlineto{\pgfqpoint{1.310468in}{0.724825in}}%
\pgfpathlineto{\pgfqpoint{1.310766in}{0.724821in}}%
\pgfpathlineto{\pgfqpoint{1.311063in}{0.724817in}}%
\pgfpathlineto{\pgfqpoint{1.311361in}{0.724813in}}%
\pgfpathlineto{\pgfqpoint{1.311658in}{0.724808in}}%
\pgfpathlineto{\pgfqpoint{1.311956in}{0.724804in}}%
\pgfpathlineto{\pgfqpoint{1.312253in}{0.724800in}}%
\pgfpathlineto{\pgfqpoint{1.312551in}{0.724796in}}%
\pgfpathlineto{\pgfqpoint{1.312848in}{0.724791in}}%
\pgfpathlineto{\pgfqpoint{1.313146in}{0.724787in}}%
\pgfpathlineto{\pgfqpoint{1.313443in}{0.724783in}}%
\pgfpathlineto{\pgfqpoint{1.313741in}{0.724779in}}%
\pgfpathlineto{\pgfqpoint{1.314038in}{0.724774in}}%
\pgfpathlineto{\pgfqpoint{1.314336in}{0.724770in}}%
\pgfpathlineto{\pgfqpoint{1.314633in}{0.724766in}}%
\pgfpathlineto{\pgfqpoint{1.314931in}{0.724762in}}%
\pgfpathlineto{\pgfqpoint{1.315228in}{0.724757in}}%
\pgfpathlineto{\pgfqpoint{1.315526in}{0.724753in}}%
\pgfpathlineto{\pgfqpoint{1.315823in}{0.724749in}}%
\pgfpathlineto{\pgfqpoint{1.316120in}{0.724745in}}%
\pgfpathlineto{\pgfqpoint{1.316418in}{0.724741in}}%
\pgfpathlineto{\pgfqpoint{1.316715in}{0.724736in}}%
\pgfpathlineto{\pgfqpoint{1.317013in}{0.724732in}}%
\pgfpathlineto{\pgfqpoint{1.317310in}{0.724728in}}%
\pgfpathlineto{\pgfqpoint{1.317608in}{0.724724in}}%
\pgfpathlineto{\pgfqpoint{1.317905in}{0.724719in}}%
\pgfpathlineto{\pgfqpoint{1.318203in}{0.724715in}}%
\pgfpathlineto{\pgfqpoint{1.318500in}{0.724711in}}%
\pgfpathlineto{\pgfqpoint{1.318798in}{0.724707in}}%
\pgfpathlineto{\pgfqpoint{1.319095in}{0.724702in}}%
\pgfpathlineto{\pgfqpoint{1.319393in}{0.724698in}}%
\pgfpathlineto{\pgfqpoint{1.319690in}{0.724694in}}%
\pgfpathlineto{\pgfqpoint{1.319988in}{0.724690in}}%
\pgfpathlineto{\pgfqpoint{1.320285in}{0.724685in}}%
\pgfpathlineto{\pgfqpoint{1.320583in}{0.724681in}}%
\pgfpathlineto{\pgfqpoint{1.320880in}{0.724677in}}%
\pgfpathlineto{\pgfqpoint{1.321178in}{0.724673in}}%
\pgfpathlineto{\pgfqpoint{1.321475in}{0.724668in}}%
\pgfpathlineto{\pgfqpoint{1.321773in}{0.724664in}}%
\pgfpathlineto{\pgfqpoint{1.322070in}{0.724660in}}%
\pgfpathlineto{\pgfqpoint{1.322367in}{0.724656in}}%
\pgfpathlineto{\pgfqpoint{1.322665in}{0.724652in}}%
\pgfpathlineto{\pgfqpoint{1.322962in}{0.724647in}}%
\pgfpathlineto{\pgfqpoint{1.323260in}{0.724643in}}%
\pgfpathlineto{\pgfqpoint{1.323557in}{0.724639in}}%
\pgfpathlineto{\pgfqpoint{1.323855in}{0.724635in}}%
\pgfpathlineto{\pgfqpoint{1.324152in}{0.724630in}}%
\pgfpathlineto{\pgfqpoint{1.324450in}{0.724626in}}%
\pgfpathlineto{\pgfqpoint{1.324747in}{0.724622in}}%
\pgfpathlineto{\pgfqpoint{1.325045in}{0.724618in}}%
\pgfpathlineto{\pgfqpoint{1.325342in}{0.724613in}}%
\pgfpathlineto{\pgfqpoint{1.325640in}{0.724609in}}%
\pgfpathlineto{\pgfqpoint{1.325937in}{0.724605in}}%
\pgfpathlineto{\pgfqpoint{1.326235in}{0.724601in}}%
\pgfpathlineto{\pgfqpoint{1.326532in}{0.724596in}}%
\pgfpathlineto{\pgfqpoint{1.326830in}{0.724592in}}%
\pgfpathlineto{\pgfqpoint{1.327127in}{0.724588in}}%
\pgfpathlineto{\pgfqpoint{1.327425in}{0.724584in}}%
\pgfpathlineto{\pgfqpoint{1.327722in}{0.724579in}}%
\pgfpathlineto{\pgfqpoint{1.328020in}{0.724575in}}%
\pgfpathlineto{\pgfqpoint{1.328317in}{0.724571in}}%
\pgfpathlineto{\pgfqpoint{1.328615in}{0.724567in}}%
\pgfpathlineto{\pgfqpoint{1.328912in}{0.724563in}}%
\pgfpathlineto{\pgfqpoint{1.329209in}{0.724558in}}%
\pgfpathlineto{\pgfqpoint{1.329507in}{0.724554in}}%
\pgfpathlineto{\pgfqpoint{1.329804in}{0.724550in}}%
\pgfpathlineto{\pgfqpoint{1.330102in}{0.724546in}}%
\pgfpathlineto{\pgfqpoint{1.330399in}{0.724541in}}%
\pgfpathlineto{\pgfqpoint{1.330697in}{0.724537in}}%
\pgfpathlineto{\pgfqpoint{1.330994in}{0.724533in}}%
\pgfpathlineto{\pgfqpoint{1.331292in}{0.724529in}}%
\pgfpathlineto{\pgfqpoint{1.331589in}{0.724524in}}%
\pgfpathlineto{\pgfqpoint{1.331887in}{0.724520in}}%
\pgfpathlineto{\pgfqpoint{1.332184in}{0.724516in}}%
\pgfpathlineto{\pgfqpoint{1.332482in}{0.724512in}}%
\pgfpathlineto{\pgfqpoint{1.332779in}{0.724507in}}%
\pgfpathlineto{\pgfqpoint{1.333077in}{0.724503in}}%
\pgfpathlineto{\pgfqpoint{1.333374in}{0.724499in}}%
\pgfpathlineto{\pgfqpoint{1.333672in}{0.724495in}}%
\pgfpathlineto{\pgfqpoint{1.333969in}{0.724490in}}%
\pgfpathlineto{\pgfqpoint{1.334267in}{0.724486in}}%
\pgfpathlineto{\pgfqpoint{1.334564in}{0.724478in}}%
\pgfpathlineto{\pgfqpoint{1.334862in}{0.724462in}}%
\pgfpathlineto{\pgfqpoint{1.335159in}{0.724445in}}%
\pgfpathlineto{\pgfqpoint{1.335457in}{0.724428in}}%
\pgfpathlineto{\pgfqpoint{1.335754in}{0.724411in}}%
\pgfpathlineto{\pgfqpoint{1.336051in}{0.724394in}}%
\pgfpathlineto{\pgfqpoint{1.336349in}{0.724377in}}%
\pgfpathlineto{\pgfqpoint{1.336646in}{0.724360in}}%
\pgfpathlineto{\pgfqpoint{1.336944in}{0.724342in}}%
\pgfpathlineto{\pgfqpoint{1.337241in}{0.724325in}}%
\pgfpathlineto{\pgfqpoint{1.337539in}{0.724308in}}%
\pgfpathlineto{\pgfqpoint{1.337836in}{0.724291in}}%
\pgfpathlineto{\pgfqpoint{1.338134in}{0.724274in}}%
\pgfpathlineto{\pgfqpoint{1.338431in}{0.724257in}}%
\pgfpathlineto{\pgfqpoint{1.338729in}{0.724240in}}%
\pgfpathlineto{\pgfqpoint{1.339026in}{0.724223in}}%
\pgfpathlineto{\pgfqpoint{1.339324in}{0.724206in}}%
\pgfpathlineto{\pgfqpoint{1.339621in}{0.724189in}}%
\pgfpathlineto{\pgfqpoint{1.339919in}{0.724172in}}%
\pgfpathlineto{\pgfqpoint{1.340216in}{0.724155in}}%
\pgfpathlineto{\pgfqpoint{1.340514in}{0.724138in}}%
\pgfpathlineto{\pgfqpoint{1.340811in}{0.724121in}}%
\pgfpathlineto{\pgfqpoint{1.341109in}{0.724104in}}%
\pgfpathlineto{\pgfqpoint{1.341406in}{0.725833in}}%
\pgfpathlineto{\pgfqpoint{1.341704in}{0.727293in}}%
\pgfpathlineto{\pgfqpoint{1.342001in}{0.727276in}}%
\pgfpathlineto{\pgfqpoint{1.342298in}{0.727259in}}%
\pgfpathlineto{\pgfqpoint{1.342596in}{0.727242in}}%
\pgfpathlineto{\pgfqpoint{1.342893in}{0.727225in}}%
\pgfpathlineto{\pgfqpoint{1.343191in}{0.727210in}}%
\pgfpathlineto{\pgfqpoint{1.343488in}{0.727205in}}%
\pgfpathlineto{\pgfqpoint{1.343786in}{0.727201in}}%
\pgfpathlineto{\pgfqpoint{1.344083in}{0.727197in}}%
\pgfpathlineto{\pgfqpoint{1.344381in}{0.727193in}}%
\pgfpathlineto{\pgfqpoint{1.344678in}{0.727189in}}%
\pgfpathlineto{\pgfqpoint{1.344976in}{0.727185in}}%
\pgfpathlineto{\pgfqpoint{1.345273in}{0.727181in}}%
\pgfpathlineto{\pgfqpoint{1.345571in}{0.727174in}}%
\pgfpathlineto{\pgfqpoint{1.345868in}{0.727158in}}%
\pgfpathlineto{\pgfqpoint{1.346166in}{0.727141in}}%
\pgfpathlineto{\pgfqpoint{1.346463in}{0.727124in}}%
\pgfpathlineto{\pgfqpoint{1.346761in}{0.727107in}}%
\pgfpathlineto{\pgfqpoint{1.347058in}{0.727090in}}%
\pgfpathlineto{\pgfqpoint{1.347356in}{0.727073in}}%
\pgfpathlineto{\pgfqpoint{1.347653in}{0.727056in}}%
\pgfpathlineto{\pgfqpoint{1.347951in}{0.727039in}}%
\pgfpathlineto{\pgfqpoint{1.348248in}{0.727022in}}%
\pgfpathlineto{\pgfqpoint{1.348546in}{0.727005in}}%
\pgfpathlineto{\pgfqpoint{1.348843in}{0.726988in}}%
\pgfpathlineto{\pgfqpoint{1.349140in}{0.726971in}}%
\pgfpathlineto{\pgfqpoint{1.349438in}{0.726954in}}%
\pgfpathlineto{\pgfqpoint{1.349735in}{0.726936in}}%
\pgfpathlineto{\pgfqpoint{1.350033in}{0.726919in}}%
\pgfpathlineto{\pgfqpoint{1.350330in}{0.726902in}}%
\pgfpathlineto{\pgfqpoint{1.350628in}{0.726958in}}%
\pgfpathlineto{\pgfqpoint{1.350925in}{0.727638in}}%
\pgfpathlineto{\pgfqpoint{1.351223in}{0.726952in}}%
\pgfpathlineto{\pgfqpoint{1.351520in}{0.726058in}}%
\pgfpathlineto{\pgfqpoint{1.351818in}{0.725164in}}%
\pgfpathlineto{\pgfqpoint{1.352115in}{0.724269in}}%
\pgfpathlineto{\pgfqpoint{1.352413in}{0.724940in}}%
\pgfpathlineto{\pgfqpoint{1.352710in}{0.727783in}}%
\pgfpathlineto{\pgfqpoint{1.353008in}{0.727766in}}%
\pgfpathlineto{\pgfqpoint{1.353305in}{0.727747in}}%
\pgfpathlineto{\pgfqpoint{1.353603in}{0.727728in}}%
\pgfpathlineto{\pgfqpoint{1.353900in}{0.727709in}}%
\pgfpathlineto{\pgfqpoint{1.354198in}{0.727690in}}%
\pgfpathlineto{\pgfqpoint{1.354495in}{0.727671in}}%
\pgfpathlineto{\pgfqpoint{1.354793in}{0.727652in}}%
\pgfpathlineto{\pgfqpoint{1.355090in}{0.727633in}}%
\pgfpathlineto{\pgfqpoint{1.355388in}{0.727614in}}%
\pgfpathlineto{\pgfqpoint{1.355685in}{0.727595in}}%
\pgfpathlineto{\pgfqpoint{1.355982in}{0.727576in}}%
\pgfpathlineto{\pgfqpoint{1.356280in}{0.727557in}}%
\pgfpathlineto{\pgfqpoint{1.356577in}{0.727538in}}%
\pgfpathlineto{\pgfqpoint{1.356875in}{0.727519in}}%
\pgfpathlineto{\pgfqpoint{1.357172in}{0.727500in}}%
\pgfpathlineto{\pgfqpoint{1.357470in}{0.727482in}}%
\pgfpathlineto{\pgfqpoint{1.357767in}{0.727468in}}%
\pgfpathlineto{\pgfqpoint{1.358065in}{0.727451in}}%
\pgfpathlineto{\pgfqpoint{1.358362in}{0.727434in}}%
\pgfpathlineto{\pgfqpoint{1.358660in}{0.727417in}}%
\pgfpathlineto{\pgfqpoint{1.358957in}{0.727400in}}%
\pgfpathlineto{\pgfqpoint{1.359255in}{0.727383in}}%
\pgfpathlineto{\pgfqpoint{1.359552in}{0.727365in}}%
\pgfpathlineto{\pgfqpoint{1.359850in}{0.727348in}}%
\pgfpathlineto{\pgfqpoint{1.360147in}{0.727331in}}%
\pgfpathlineto{\pgfqpoint{1.360445in}{0.727314in}}%
\pgfpathlineto{\pgfqpoint{1.360742in}{0.727297in}}%
\pgfpathlineto{\pgfqpoint{1.361040in}{0.727280in}}%
\pgfpathlineto{\pgfqpoint{1.361337in}{0.727263in}}%
\pgfpathlineto{\pgfqpoint{1.361635in}{0.727246in}}%
\pgfpathlineto{\pgfqpoint{1.361932in}{0.727229in}}%
\pgfpathlineto{\pgfqpoint{1.362229in}{0.727212in}}%
\pgfpathlineto{\pgfqpoint{1.362527in}{0.727195in}}%
\pgfpathlineto{\pgfqpoint{1.362824in}{0.727178in}}%
\pgfpathlineto{\pgfqpoint{1.363122in}{0.727161in}}%
\pgfpathlineto{\pgfqpoint{1.363419in}{0.727144in}}%
\pgfpathlineto{\pgfqpoint{1.363717in}{0.727127in}}%
\pgfpathlineto{\pgfqpoint{1.364014in}{0.727110in}}%
\pgfpathlineto{\pgfqpoint{1.364312in}{0.727093in}}%
\pgfpathlineto{\pgfqpoint{1.364609in}{0.727076in}}%
\pgfpathlineto{\pgfqpoint{1.364907in}{0.727059in}}%
\pgfpathlineto{\pgfqpoint{1.365204in}{0.727041in}}%
\pgfpathlineto{\pgfqpoint{1.365502in}{0.727024in}}%
\pgfpathlineto{\pgfqpoint{1.365799in}{0.727007in}}%
\pgfpathlineto{\pgfqpoint{1.366097in}{0.726990in}}%
\pgfpathlineto{\pgfqpoint{1.366394in}{0.726973in}}%
\pgfpathlineto{\pgfqpoint{1.366692in}{0.726956in}}%
\pgfpathlineto{\pgfqpoint{1.366989in}{0.726939in}}%
\pgfpathlineto{\pgfqpoint{1.367287in}{0.726922in}}%
\pgfpathlineto{\pgfqpoint{1.367584in}{0.726905in}}%
\pgfpathlineto{\pgfqpoint{1.367882in}{0.726888in}}%
\pgfpathlineto{\pgfqpoint{1.368179in}{0.726871in}}%
\pgfpathlineto{\pgfqpoint{1.368477in}{0.726854in}}%
\pgfpathlineto{\pgfqpoint{1.368774in}{0.726837in}}%
\pgfpathlineto{\pgfqpoint{1.369071in}{0.726820in}}%
\pgfpathlineto{\pgfqpoint{1.369369in}{0.726803in}}%
\pgfpathlineto{\pgfqpoint{1.369666in}{0.726786in}}%
\pgfpathlineto{\pgfqpoint{1.369964in}{0.726769in}}%
\pgfpathlineto{\pgfqpoint{1.370261in}{0.726752in}}%
\pgfpathlineto{\pgfqpoint{1.370559in}{0.726735in}}%
\pgfpathlineto{\pgfqpoint{1.370856in}{0.726718in}}%
\pgfpathlineto{\pgfqpoint{1.371154in}{0.726700in}}%
\pgfpathlineto{\pgfqpoint{1.371451in}{0.726683in}}%
\pgfpathlineto{\pgfqpoint{1.371749in}{0.726666in}}%
\pgfpathlineto{\pgfqpoint{1.372046in}{0.726649in}}%
\pgfpathlineto{\pgfqpoint{1.372344in}{0.726632in}}%
\pgfpathlineto{\pgfqpoint{1.372641in}{0.726615in}}%
\pgfpathlineto{\pgfqpoint{1.372939in}{0.726598in}}%
\pgfpathlineto{\pgfqpoint{1.373236in}{0.726581in}}%
\pgfpathlineto{\pgfqpoint{1.373534in}{0.726564in}}%
\pgfpathlineto{\pgfqpoint{1.373831in}{0.726547in}}%
\pgfpathlineto{\pgfqpoint{1.374129in}{0.726530in}}%
\pgfpathlineto{\pgfqpoint{1.374426in}{0.726513in}}%
\pgfpathlineto{\pgfqpoint{1.374724in}{0.726496in}}%
\pgfpathlineto{\pgfqpoint{1.375021in}{0.726479in}}%
\pgfpathlineto{\pgfqpoint{1.375319in}{0.726462in}}%
\pgfpathlineto{\pgfqpoint{1.375616in}{0.726445in}}%
\pgfpathlineto{\pgfqpoint{1.375913in}{0.726428in}}%
\pgfpathlineto{\pgfqpoint{1.376211in}{0.726411in}}%
\pgfpathlineto{\pgfqpoint{1.376508in}{0.726394in}}%
\pgfpathlineto{\pgfqpoint{1.376806in}{0.726376in}}%
\pgfpathlineto{\pgfqpoint{1.377103in}{0.726359in}}%
\pgfpathlineto{\pgfqpoint{1.377401in}{0.726342in}}%
\pgfpathlineto{\pgfqpoint{1.377698in}{0.726325in}}%
\pgfpathlineto{\pgfqpoint{1.377996in}{0.726308in}}%
\pgfpathlineto{\pgfqpoint{1.378293in}{0.726291in}}%
\pgfpathlineto{\pgfqpoint{1.378591in}{0.726274in}}%
\pgfpathlineto{\pgfqpoint{1.378888in}{0.726257in}}%
\pgfpathlineto{\pgfqpoint{1.379186in}{0.726240in}}%
\pgfpathlineto{\pgfqpoint{1.379483in}{0.726223in}}%
\pgfpathlineto{\pgfqpoint{1.379781in}{0.726206in}}%
\pgfpathlineto{\pgfqpoint{1.380078in}{0.726189in}}%
\pgfpathlineto{\pgfqpoint{1.380376in}{0.726172in}}%
\pgfpathlineto{\pgfqpoint{1.380673in}{0.726155in}}%
\pgfpathlineto{\pgfqpoint{1.380971in}{0.726138in}}%
\pgfpathlineto{\pgfqpoint{1.381268in}{0.726121in}}%
\pgfpathlineto{\pgfqpoint{1.381566in}{0.726104in}}%
\pgfpathlineto{\pgfqpoint{1.381863in}{0.726087in}}%
\pgfpathlineto{\pgfqpoint{1.382161in}{0.726070in}}%
\pgfpathlineto{\pgfqpoint{1.382458in}{0.726052in}}%
\pgfpathlineto{\pgfqpoint{1.382755in}{0.726035in}}%
\pgfpathlineto{\pgfqpoint{1.383053in}{0.726018in}}%
\pgfpathlineto{\pgfqpoint{1.383350in}{0.726001in}}%
\pgfpathlineto{\pgfqpoint{1.383648in}{0.725984in}}%
\pgfpathlineto{\pgfqpoint{1.383945in}{0.725967in}}%
\pgfpathlineto{\pgfqpoint{1.384243in}{0.725950in}}%
\pgfpathlineto{\pgfqpoint{1.384540in}{0.725933in}}%
\pgfpathlineto{\pgfqpoint{1.384838in}{0.725903in}}%
\pgfpathlineto{\pgfqpoint{1.385135in}{0.725867in}}%
\pgfpathlineto{\pgfqpoint{1.385433in}{0.725831in}}%
\pgfpathlineto{\pgfqpoint{1.385730in}{0.725795in}}%
\pgfpathlineto{\pgfqpoint{1.386028in}{0.725903in}}%
\pgfpathlineto{\pgfqpoint{1.386325in}{0.727283in}}%
\pgfpathlineto{\pgfqpoint{1.386623in}{0.727248in}}%
\pgfpathlineto{\pgfqpoint{1.386920in}{0.727213in}}%
\pgfpathlineto{\pgfqpoint{1.387218in}{0.727177in}}%
\pgfpathlineto{\pgfqpoint{1.387515in}{0.727142in}}%
\pgfpathlineto{\pgfqpoint{1.387813in}{0.727106in}}%
\pgfpathlineto{\pgfqpoint{1.388110in}{0.727071in}}%
\pgfpathlineto{\pgfqpoint{1.388408in}{0.727036in}}%
\pgfpathlineto{\pgfqpoint{1.388705in}{0.727000in}}%
\pgfpathlineto{\pgfqpoint{1.389002in}{0.726965in}}%
\pgfpathlineto{\pgfqpoint{1.389300in}{0.726930in}}%
\pgfpathlineto{\pgfqpoint{1.389597in}{0.726894in}}%
\pgfpathlineto{\pgfqpoint{1.389895in}{0.726859in}}%
\pgfpathlineto{\pgfqpoint{1.390192in}{0.726823in}}%
\pgfpathlineto{\pgfqpoint{1.390490in}{0.726788in}}%
\pgfpathlineto{\pgfqpoint{1.390787in}{0.726753in}}%
\pgfpathlineto{\pgfqpoint{1.391085in}{0.726717in}}%
\pgfpathlineto{\pgfqpoint{1.391382in}{0.726682in}}%
\pgfpathlineto{\pgfqpoint{1.391680in}{0.726646in}}%
\pgfpathlineto{\pgfqpoint{1.391977in}{0.726611in}}%
\pgfpathlineto{\pgfqpoint{1.392275in}{0.726576in}}%
\pgfpathlineto{\pgfqpoint{1.392572in}{0.726540in}}%
\pgfpathlineto{\pgfqpoint{1.392870in}{0.726505in}}%
\pgfpathlineto{\pgfqpoint{1.393167in}{0.726470in}}%
\pgfpathlineto{\pgfqpoint{1.393465in}{0.726434in}}%
\pgfpathlineto{\pgfqpoint{1.393762in}{0.726399in}}%
\pgfpathlineto{\pgfqpoint{1.394060in}{0.726363in}}%
\pgfpathlineto{\pgfqpoint{1.394357in}{0.726328in}}%
\pgfpathlineto{\pgfqpoint{1.394655in}{0.726293in}}%
\pgfpathlineto{\pgfqpoint{1.394952in}{0.726257in}}%
\pgfpathlineto{\pgfqpoint{1.395250in}{0.726222in}}%
\pgfpathlineto{\pgfqpoint{1.395547in}{0.726187in}}%
\pgfpathlineto{\pgfqpoint{1.395844in}{0.726151in}}%
\pgfpathlineto{\pgfqpoint{1.396142in}{0.726116in}}%
\pgfpathlineto{\pgfqpoint{1.396439in}{0.726080in}}%
\pgfpathlineto{\pgfqpoint{1.396737in}{0.726045in}}%
\pgfpathlineto{\pgfqpoint{1.397034in}{0.726010in}}%
\pgfpathlineto{\pgfqpoint{1.397332in}{0.725974in}}%
\pgfpathlineto{\pgfqpoint{1.397629in}{0.725939in}}%
\pgfpathlineto{\pgfqpoint{1.397927in}{0.725904in}}%
\pgfpathlineto{\pgfqpoint{1.398224in}{0.725868in}}%
\pgfpathlineto{\pgfqpoint{1.398522in}{0.725832in}}%
\pgfpathlineto{\pgfqpoint{1.398819in}{0.725718in}}%
\pgfpathlineto{\pgfqpoint{1.399117in}{0.725548in}}%
\pgfpathlineto{\pgfqpoint{1.399414in}{0.725379in}}%
\pgfpathlineto{\pgfqpoint{1.399712in}{0.725209in}}%
\pgfpathlineto{\pgfqpoint{1.400009in}{0.725039in}}%
\pgfpathlineto{\pgfqpoint{1.400307in}{0.724870in}}%
\pgfpathlineto{\pgfqpoint{1.400604in}{0.724700in}}%
\pgfpathlineto{\pgfqpoint{1.400902in}{0.724531in}}%
\pgfpathlineto{\pgfqpoint{1.401199in}{0.724361in}}%
\pgfpathlineto{\pgfqpoint{1.401497in}{0.724191in}}%
\pgfpathlineto{\pgfqpoint{1.401794in}{0.725183in}}%
\pgfpathlineto{\pgfqpoint{1.402092in}{0.727263in}}%
\pgfpathlineto{\pgfqpoint{1.402389in}{0.727229in}}%
\pgfpathlineto{\pgfqpoint{1.402686in}{0.727196in}}%
\pgfpathlineto{\pgfqpoint{1.402984in}{0.727162in}}%
\pgfpathlineto{\pgfqpoint{1.403281in}{0.727128in}}%
\pgfpathlineto{\pgfqpoint{1.403579in}{0.727095in}}%
\pgfpathlineto{\pgfqpoint{1.403876in}{0.727061in}}%
\pgfpathlineto{\pgfqpoint{1.404174in}{0.727027in}}%
\pgfpathlineto{\pgfqpoint{1.404471in}{0.726993in}}%
\pgfpathlineto{\pgfqpoint{1.404769in}{0.726960in}}%
\pgfpathlineto{\pgfqpoint{1.405066in}{0.726926in}}%
\pgfpathlineto{\pgfqpoint{1.405364in}{0.726892in}}%
\pgfpathlineto{\pgfqpoint{1.405661in}{0.726859in}}%
\pgfpathlineto{\pgfqpoint{1.405959in}{0.726825in}}%
\pgfpathlineto{\pgfqpoint{1.406256in}{0.726791in}}%
\pgfpathlineto{\pgfqpoint{1.406554in}{0.726757in}}%
\pgfpathlineto{\pgfqpoint{1.406851in}{0.726724in}}%
\pgfpathlineto{\pgfqpoint{1.407149in}{0.726690in}}%
\pgfpathlineto{\pgfqpoint{1.407446in}{0.726656in}}%
\pgfpathlineto{\pgfqpoint{1.407744in}{0.726621in}}%
\pgfpathlineto{\pgfqpoint{1.408041in}{0.726585in}}%
\pgfpathlineto{\pgfqpoint{1.408339in}{0.726484in}}%
\pgfpathlineto{\pgfqpoint{1.408636in}{0.726343in}}%
\pgfpathlineto{\pgfqpoint{1.408933in}{0.726224in}}%
\pgfpathlineto{\pgfqpoint{1.409231in}{0.726107in}}%
\pgfpathlineto{\pgfqpoint{1.409528in}{0.725990in}}%
\pgfpathlineto{\pgfqpoint{1.409826in}{0.725873in}}%
\pgfpathlineto{\pgfqpoint{1.410123in}{0.725756in}}%
\pgfpathlineto{\pgfqpoint{1.410421in}{0.725639in}}%
\pgfpathlineto{\pgfqpoint{1.410718in}{0.725522in}}%
\pgfpathlineto{\pgfqpoint{1.411016in}{0.725405in}}%
\pgfpathlineto{\pgfqpoint{1.411313in}{0.725287in}}%
\pgfpathlineto{\pgfqpoint{1.411611in}{0.725170in}}%
\pgfpathlineto{\pgfqpoint{1.411908in}{0.725053in}}%
\pgfpathlineto{\pgfqpoint{1.412206in}{0.724936in}}%
\pgfpathlineto{\pgfqpoint{1.412503in}{0.724819in}}%
\pgfpathlineto{\pgfqpoint{1.412801in}{0.724702in}}%
\pgfpathlineto{\pgfqpoint{1.413098in}{0.725409in}}%
\pgfpathlineto{\pgfqpoint{1.413396in}{0.727309in}}%
\pgfpathlineto{\pgfqpoint{1.413693in}{0.726186in}}%
\pgfpathlineto{\pgfqpoint{1.413991in}{0.725181in}}%
\pgfpathlineto{\pgfqpoint{1.414288in}{0.726376in}}%
\pgfpathlineto{\pgfqpoint{1.414586in}{0.727971in}}%
\pgfpathlineto{\pgfqpoint{1.414883in}{0.728422in}}%
\pgfpathlineto{\pgfqpoint{1.415181in}{0.728521in}}%
\pgfpathlineto{\pgfqpoint{1.415478in}{0.728522in}}%
\pgfpathlineto{\pgfqpoint{1.415775in}{0.728522in}}%
\pgfpathlineto{\pgfqpoint{1.416073in}{0.728523in}}%
\pgfpathlineto{\pgfqpoint{1.416370in}{0.728523in}}%
\pgfpathlineto{\pgfqpoint{1.416668in}{0.728524in}}%
\pgfpathlineto{\pgfqpoint{1.416965in}{0.728524in}}%
\pgfpathlineto{\pgfqpoint{1.417263in}{0.728525in}}%
\pgfpathlineto{\pgfqpoint{1.417560in}{0.728525in}}%
\pgfpathlineto{\pgfqpoint{1.417858in}{0.728525in}}%
\pgfpathlineto{\pgfqpoint{1.418155in}{0.728526in}}%
\pgfpathlineto{\pgfqpoint{1.418453in}{0.728526in}}%
\pgfpathlineto{\pgfqpoint{1.418750in}{0.728527in}}%
\pgfpathlineto{\pgfqpoint{1.419048in}{0.728527in}}%
\pgfpathlineto{\pgfqpoint{1.419345in}{0.728528in}}%
\pgfpathlineto{\pgfqpoint{1.419643in}{0.728528in}}%
\pgfpathlineto{\pgfqpoint{1.419940in}{0.728529in}}%
\pgfpathlineto{\pgfqpoint{1.420238in}{0.728529in}}%
\pgfpathlineto{\pgfqpoint{1.420535in}{0.728529in}}%
\pgfpathlineto{\pgfqpoint{1.420833in}{0.728530in}}%
\pgfpathlineto{\pgfqpoint{1.421130in}{0.728530in}}%
\pgfpathlineto{\pgfqpoint{1.421428in}{0.728531in}}%
\pgfpathlineto{\pgfqpoint{1.421725in}{0.728531in}}%
\pgfpathlineto{\pgfqpoint{1.422023in}{0.728532in}}%
\pgfpathlineto{\pgfqpoint{1.422320in}{0.728532in}}%
\pgfpathlineto{\pgfqpoint{1.422617in}{0.728533in}}%
\pgfpathlineto{\pgfqpoint{1.422915in}{0.728533in}}%
\pgfpathlineto{\pgfqpoint{1.423212in}{0.728533in}}%
\pgfpathlineto{\pgfqpoint{1.423510in}{0.728534in}}%
\pgfpathlineto{\pgfqpoint{1.423807in}{0.728534in}}%
\pgfpathlineto{\pgfqpoint{1.424105in}{0.728535in}}%
\pgfpathlineto{\pgfqpoint{1.424402in}{0.728535in}}%
\pgfpathlineto{\pgfqpoint{1.424700in}{0.728536in}}%
\pgfpathlineto{\pgfqpoint{1.424997in}{0.728536in}}%
\pgfpathlineto{\pgfqpoint{1.425295in}{0.728537in}}%
\pgfpathlineto{\pgfqpoint{1.425592in}{0.728537in}}%
\pgfpathlineto{\pgfqpoint{1.425890in}{0.728537in}}%
\pgfpathlineto{\pgfqpoint{1.426187in}{0.728538in}}%
\pgfpathlineto{\pgfqpoint{1.426485in}{0.728538in}}%
\pgfpathlineto{\pgfqpoint{1.426782in}{0.728539in}}%
\pgfpathlineto{\pgfqpoint{1.427080in}{0.728539in}}%
\pgfpathlineto{\pgfqpoint{1.427377in}{0.728540in}}%
\pgfpathlineto{\pgfqpoint{1.427675in}{0.728540in}}%
\pgfpathlineto{\pgfqpoint{1.427972in}{0.728540in}}%
\pgfpathlineto{\pgfqpoint{1.428270in}{0.728541in}}%
\pgfpathlineto{\pgfqpoint{1.428567in}{0.728541in}}%
\pgfpathlineto{\pgfqpoint{1.428864in}{0.728542in}}%
\pgfpathlineto{\pgfqpoint{1.429162in}{0.728542in}}%
\pgfpathlineto{\pgfqpoint{1.429459in}{0.728543in}}%
\pgfpathlineto{\pgfqpoint{1.429757in}{0.728543in}}%
\pgfpathlineto{\pgfqpoint{1.430054in}{0.728544in}}%
\pgfpathlineto{\pgfqpoint{1.430352in}{0.728544in}}%
\pgfpathlineto{\pgfqpoint{1.430649in}{0.728544in}}%
\pgfpathlineto{\pgfqpoint{1.430947in}{0.728545in}}%
\pgfpathlineto{\pgfqpoint{1.431244in}{0.728545in}}%
\pgfpathlineto{\pgfqpoint{1.431542in}{0.728546in}}%
\pgfpathlineto{\pgfqpoint{1.431839in}{0.728546in}}%
\pgfpathlineto{\pgfqpoint{1.432137in}{0.728547in}}%
\pgfpathlineto{\pgfqpoint{1.432434in}{0.728547in}}%
\pgfpathlineto{\pgfqpoint{1.432732in}{0.728548in}}%
\pgfpathlineto{\pgfqpoint{1.433029in}{0.728548in}}%
\pgfpathlineto{\pgfqpoint{1.433327in}{0.728548in}}%
\pgfpathlineto{\pgfqpoint{1.433624in}{0.728549in}}%
\pgfpathlineto{\pgfqpoint{1.433922in}{0.728549in}}%
\pgfpathlineto{\pgfqpoint{1.434219in}{0.728550in}}%
\pgfpathlineto{\pgfqpoint{1.434517in}{0.728550in}}%
\pgfpathlineto{\pgfqpoint{1.434814in}{0.728551in}}%
\pgfpathlineto{\pgfqpoint{1.435112in}{0.728551in}}%
\pgfpathlineto{\pgfqpoint{1.435409in}{0.728552in}}%
\pgfpathlineto{\pgfqpoint{1.435706in}{0.728552in}}%
\pgfpathlineto{\pgfqpoint{1.436004in}{0.728552in}}%
\pgfpathlineto{\pgfqpoint{1.436301in}{0.728553in}}%
\pgfpathlineto{\pgfqpoint{1.436599in}{0.728553in}}%
\pgfpathlineto{\pgfqpoint{1.436896in}{0.728554in}}%
\pgfpathlineto{\pgfqpoint{1.437194in}{0.728554in}}%
\pgfpathlineto{\pgfqpoint{1.437491in}{0.728555in}}%
\pgfpathlineto{\pgfqpoint{1.437789in}{0.728555in}}%
\pgfpathlineto{\pgfqpoint{1.438086in}{0.728556in}}%
\pgfpathlineto{\pgfqpoint{1.438384in}{0.728556in}}%
\pgfpathlineto{\pgfqpoint{1.438681in}{0.728556in}}%
\pgfpathlineto{\pgfqpoint{1.438979in}{0.728557in}}%
\pgfpathlineto{\pgfqpoint{1.439276in}{0.728557in}}%
\pgfpathlineto{\pgfqpoint{1.439574in}{0.728558in}}%
\pgfpathlineto{\pgfqpoint{1.439871in}{0.728558in}}%
\pgfpathlineto{\pgfqpoint{1.440169in}{0.728559in}}%
\pgfpathlineto{\pgfqpoint{1.440466in}{0.728559in}}%
\pgfpathlineto{\pgfqpoint{1.440764in}{0.728560in}}%
\pgfpathlineto{\pgfqpoint{1.441061in}{0.728560in}}%
\pgfpathlineto{\pgfqpoint{1.441359in}{0.728560in}}%
\pgfpathlineto{\pgfqpoint{1.441656in}{0.728561in}}%
\pgfpathlineto{\pgfqpoint{1.441954in}{0.728561in}}%
\pgfpathlineto{\pgfqpoint{1.442251in}{0.728562in}}%
\pgfpathlineto{\pgfqpoint{1.442548in}{0.728562in}}%
\pgfpathlineto{\pgfqpoint{1.442846in}{0.728563in}}%
\pgfpathlineto{\pgfqpoint{1.443143in}{0.728563in}}%
\pgfpathlineto{\pgfqpoint{1.443441in}{0.728564in}}%
\pgfpathlineto{\pgfqpoint{1.443738in}{0.728564in}}%
\pgfpathlineto{\pgfqpoint{1.444036in}{0.728564in}}%
\pgfpathlineto{\pgfqpoint{1.444333in}{0.728565in}}%
\pgfpathlineto{\pgfqpoint{1.444631in}{0.728565in}}%
\pgfpathlineto{\pgfqpoint{1.444928in}{0.728566in}}%
\pgfpathlineto{\pgfqpoint{1.445226in}{0.728566in}}%
\pgfpathlineto{\pgfqpoint{1.445523in}{0.728567in}}%
\pgfpathlineto{\pgfqpoint{1.445821in}{0.728567in}}%
\pgfpathlineto{\pgfqpoint{1.446118in}{0.728568in}}%
\pgfpathlineto{\pgfqpoint{1.446416in}{0.728568in}}%
\pgfpathlineto{\pgfqpoint{1.446713in}{0.728568in}}%
\pgfpathlineto{\pgfqpoint{1.447011in}{0.728569in}}%
\pgfpathlineto{\pgfqpoint{1.447308in}{0.728569in}}%
\pgfpathlineto{\pgfqpoint{1.447606in}{0.728570in}}%
\pgfpathlineto{\pgfqpoint{1.447903in}{0.728570in}}%
\pgfpathlineto{\pgfqpoint{1.448201in}{0.728571in}}%
\pgfpathlineto{\pgfqpoint{1.448498in}{0.728571in}}%
\pgfpathlineto{\pgfqpoint{1.448795in}{0.728572in}}%
\pgfpathlineto{\pgfqpoint{1.449093in}{0.728572in}}%
\pgfpathlineto{\pgfqpoint{1.449390in}{0.728572in}}%
\pgfpathlineto{\pgfqpoint{1.449688in}{0.728573in}}%
\pgfpathlineto{\pgfqpoint{1.449985in}{0.728573in}}%
\pgfpathlineto{\pgfqpoint{1.450283in}{0.728574in}}%
\pgfpathlineto{\pgfqpoint{1.450580in}{0.728574in}}%
\pgfpathlineto{\pgfqpoint{1.450878in}{0.728575in}}%
\pgfpathlineto{\pgfqpoint{1.451175in}{0.728575in}}%
\pgfpathlineto{\pgfqpoint{1.451473in}{0.728575in}}%
\pgfpathlineto{\pgfqpoint{1.451770in}{0.728576in}}%
\pgfpathlineto{\pgfqpoint{1.452068in}{0.728576in}}%
\pgfpathlineto{\pgfqpoint{1.452365in}{0.728577in}}%
\pgfpathlineto{\pgfqpoint{1.452663in}{0.728577in}}%
\pgfpathlineto{\pgfqpoint{1.452960in}{0.728578in}}%
\pgfpathlineto{\pgfqpoint{1.453258in}{0.728578in}}%
\pgfpathlineto{\pgfqpoint{1.453555in}{0.728579in}}%
\pgfpathlineto{\pgfqpoint{1.453853in}{0.728579in}}%
\pgfpathlineto{\pgfqpoint{1.454150in}{0.728579in}}%
\pgfpathlineto{\pgfqpoint{1.454448in}{0.728580in}}%
\pgfpathlineto{\pgfqpoint{1.454745in}{0.728580in}}%
\pgfpathlineto{\pgfqpoint{1.455043in}{0.728581in}}%
\pgfpathlineto{\pgfqpoint{1.455340in}{0.728581in}}%
\pgfpathlineto{\pgfqpoint{1.455637in}{0.728582in}}%
\pgfpathlineto{\pgfqpoint{1.455935in}{0.728582in}}%
\pgfpathlineto{\pgfqpoint{1.456232in}{0.728583in}}%
\pgfpathlineto{\pgfqpoint{1.456530in}{0.728583in}}%
\pgfpathlineto{\pgfqpoint{1.456827in}{0.728583in}}%
\pgfpathlineto{\pgfqpoint{1.457125in}{0.728584in}}%
\pgfpathlineto{\pgfqpoint{1.457422in}{0.728584in}}%
\pgfpathlineto{\pgfqpoint{1.457720in}{0.728585in}}%
\pgfpathlineto{\pgfqpoint{1.458017in}{0.728585in}}%
\pgfpathlineto{\pgfqpoint{1.458315in}{0.728586in}}%
\pgfpathlineto{\pgfqpoint{1.458612in}{0.728586in}}%
\pgfpathlineto{\pgfqpoint{1.458910in}{0.728587in}}%
\pgfpathlineto{\pgfqpoint{1.459207in}{0.728587in}}%
\pgfpathlineto{\pgfqpoint{1.459505in}{0.728587in}}%
\pgfpathlineto{\pgfqpoint{1.459802in}{0.728588in}}%
\pgfpathlineto{\pgfqpoint{1.460100in}{0.728588in}}%
\pgfpathlineto{\pgfqpoint{1.460397in}{0.728589in}}%
\pgfpathlineto{\pgfqpoint{1.460695in}{0.728589in}}%
\pgfpathlineto{\pgfqpoint{1.460992in}{0.728590in}}%
\pgfpathlineto{\pgfqpoint{1.461290in}{0.728590in}}%
\pgfpathlineto{\pgfqpoint{1.461587in}{0.728591in}}%
\pgfpathlineto{\pgfqpoint{1.461885in}{0.728591in}}%
\pgfpathlineto{\pgfqpoint{1.462182in}{0.728591in}}%
\pgfpathlineto{\pgfqpoint{1.462479in}{0.728592in}}%
\pgfpathlineto{\pgfqpoint{1.462777in}{0.728592in}}%
\pgfpathlineto{\pgfqpoint{1.463074in}{0.728593in}}%
\pgfpathlineto{\pgfqpoint{1.463372in}{0.728593in}}%
\pgfpathlineto{\pgfqpoint{1.463669in}{0.728594in}}%
\pgfpathlineto{\pgfqpoint{1.463967in}{0.728594in}}%
\pgfpathlineto{\pgfqpoint{1.464264in}{0.728595in}}%
\pgfpathlineto{\pgfqpoint{1.464562in}{0.728595in}}%
\pgfpathlineto{\pgfqpoint{1.464859in}{0.728595in}}%
\pgfpathlineto{\pgfqpoint{1.465157in}{0.728596in}}%
\pgfpathlineto{\pgfqpoint{1.465454in}{0.728596in}}%
\pgfpathlineto{\pgfqpoint{1.465752in}{0.728597in}}%
\pgfpathlineto{\pgfqpoint{1.466049in}{0.728597in}}%
\pgfpathlineto{\pgfqpoint{1.466347in}{0.728598in}}%
\pgfpathlineto{\pgfqpoint{1.466644in}{0.728598in}}%
\pgfpathlineto{\pgfqpoint{1.466942in}{0.728599in}}%
\pgfpathlineto{\pgfqpoint{1.467239in}{0.728599in}}%
\pgfpathlineto{\pgfqpoint{1.467537in}{0.728599in}}%
\pgfpathlineto{\pgfqpoint{1.467834in}{0.728600in}}%
\pgfpathlineto{\pgfqpoint{1.468132in}{0.728600in}}%
\pgfpathlineto{\pgfqpoint{1.468429in}{0.728601in}}%
\pgfpathlineto{\pgfqpoint{1.468726in}{0.728601in}}%
\pgfpathlineto{\pgfqpoint{1.469024in}{0.728602in}}%
\pgfpathlineto{\pgfqpoint{1.469321in}{0.728602in}}%
\pgfpathlineto{\pgfqpoint{1.469619in}{0.728603in}}%
\pgfpathlineto{\pgfqpoint{1.469916in}{0.728603in}}%
\pgfpathlineto{\pgfqpoint{1.470214in}{0.728603in}}%
\pgfpathlineto{\pgfqpoint{1.470511in}{0.728604in}}%
\pgfpathlineto{\pgfqpoint{1.470809in}{0.728604in}}%
\pgfpathlineto{\pgfqpoint{1.471106in}{0.728605in}}%
\pgfpathlineto{\pgfqpoint{1.471404in}{0.728605in}}%
\pgfpathlineto{\pgfqpoint{1.471701in}{0.728606in}}%
\pgfpathlineto{\pgfqpoint{1.471999in}{0.728606in}}%
\pgfpathlineto{\pgfqpoint{1.472296in}{0.728607in}}%
\pgfpathlineto{\pgfqpoint{1.472594in}{0.728607in}}%
\pgfpathlineto{\pgfqpoint{1.472891in}{0.728607in}}%
\pgfpathlineto{\pgfqpoint{1.473189in}{0.728608in}}%
\pgfpathlineto{\pgfqpoint{1.473486in}{0.728608in}}%
\pgfpathlineto{\pgfqpoint{1.473784in}{0.728609in}}%
\pgfpathlineto{\pgfqpoint{1.474081in}{0.728609in}}%
\pgfpathlineto{\pgfqpoint{1.474379in}{0.728610in}}%
\pgfpathlineto{\pgfqpoint{1.474676in}{0.728610in}}%
\pgfpathlineto{\pgfqpoint{1.474974in}{0.728610in}}%
\pgfpathlineto{\pgfqpoint{1.475271in}{0.728611in}}%
\pgfpathlineto{\pgfqpoint{1.475568in}{0.728611in}}%
\pgfpathlineto{\pgfqpoint{1.475866in}{0.728612in}}%
\pgfpathlineto{\pgfqpoint{1.476163in}{0.728612in}}%
\pgfpathlineto{\pgfqpoint{1.476461in}{0.728613in}}%
\pgfpathlineto{\pgfqpoint{1.476758in}{0.728613in}}%
\pgfpathlineto{\pgfqpoint{1.477056in}{0.728614in}}%
\pgfpathlineto{\pgfqpoint{1.477353in}{0.728614in}}%
\pgfpathlineto{\pgfqpoint{1.477651in}{0.728614in}}%
\pgfpathlineto{\pgfqpoint{1.477948in}{0.728615in}}%
\pgfpathlineto{\pgfqpoint{1.478246in}{0.728615in}}%
\pgfpathlineto{\pgfqpoint{1.478543in}{0.728616in}}%
\pgfpathlineto{\pgfqpoint{1.478841in}{0.728616in}}%
\pgfpathlineto{\pgfqpoint{1.479138in}{0.728617in}}%
\pgfpathlineto{\pgfqpoint{1.479436in}{0.728617in}}%
\pgfpathlineto{\pgfqpoint{1.479733in}{0.728618in}}%
\pgfpathlineto{\pgfqpoint{1.480031in}{0.728618in}}%
\pgfpathlineto{\pgfqpoint{1.480328in}{0.728618in}}%
\pgfpathlineto{\pgfqpoint{1.480626in}{0.728619in}}%
\pgfpathlineto{\pgfqpoint{1.480923in}{0.728619in}}%
\pgfpathlineto{\pgfqpoint{1.481221in}{0.728620in}}%
\pgfpathlineto{\pgfqpoint{1.481518in}{0.728620in}}%
\pgfpathlineto{\pgfqpoint{1.481816in}{0.728621in}}%
\pgfpathlineto{\pgfqpoint{1.482113in}{0.728621in}}%
\pgfpathlineto{\pgfqpoint{1.482410in}{0.728622in}}%
\pgfpathlineto{\pgfqpoint{1.482708in}{0.728622in}}%
\pgfpathlineto{\pgfqpoint{1.483005in}{0.728622in}}%
\pgfpathlineto{\pgfqpoint{1.483303in}{0.728623in}}%
\pgfpathlineto{\pgfqpoint{1.483600in}{0.728623in}}%
\pgfpathlineto{\pgfqpoint{1.483898in}{0.728624in}}%
\pgfpathlineto{\pgfqpoint{1.484195in}{0.728624in}}%
\pgfpathlineto{\pgfqpoint{1.484493in}{0.728625in}}%
\pgfpathlineto{\pgfqpoint{1.484790in}{0.728625in}}%
\pgfpathlineto{\pgfqpoint{1.485088in}{0.728626in}}%
\pgfpathlineto{\pgfqpoint{1.485385in}{0.728626in}}%
\pgfpathlineto{\pgfqpoint{1.485683in}{0.728626in}}%
\pgfpathlineto{\pgfqpoint{1.485980in}{0.728627in}}%
\pgfpathlineto{\pgfqpoint{1.486278in}{0.728627in}}%
\pgfpathlineto{\pgfqpoint{1.486575in}{0.728628in}}%
\pgfpathlineto{\pgfqpoint{1.486873in}{0.728628in}}%
\pgfpathlineto{\pgfqpoint{1.487170in}{0.728629in}}%
\pgfpathlineto{\pgfqpoint{1.487468in}{0.728629in}}%
\pgfpathlineto{\pgfqpoint{1.487765in}{0.728630in}}%
\pgfpathlineto{\pgfqpoint{1.488063in}{0.728630in}}%
\pgfpathlineto{\pgfqpoint{1.488360in}{0.728630in}}%
\pgfpathlineto{\pgfqpoint{1.488657in}{0.728631in}}%
\pgfpathlineto{\pgfqpoint{1.488955in}{0.728631in}}%
\pgfpathlineto{\pgfqpoint{1.489252in}{0.728632in}}%
\pgfpathlineto{\pgfqpoint{1.489550in}{0.728632in}}%
\pgfpathlineto{\pgfqpoint{1.489847in}{0.728633in}}%
\pgfpathlineto{\pgfqpoint{1.490145in}{0.728633in}}%
\pgfpathlineto{\pgfqpoint{1.490442in}{0.728634in}}%
\pgfpathlineto{\pgfqpoint{1.490740in}{0.728634in}}%
\pgfpathlineto{\pgfqpoint{1.491037in}{0.728634in}}%
\pgfpathlineto{\pgfqpoint{1.491335in}{0.728635in}}%
\pgfpathlineto{\pgfqpoint{1.491632in}{0.728635in}}%
\pgfpathlineto{\pgfqpoint{1.491930in}{0.728636in}}%
\pgfpathlineto{\pgfqpoint{1.492227in}{0.728636in}}%
\pgfpathlineto{\pgfqpoint{1.492525in}{0.728635in}}%
\pgfpathlineto{\pgfqpoint{1.492822in}{0.728635in}}%
\pgfpathlineto{\pgfqpoint{1.493120in}{0.728635in}}%
\pgfpathlineto{\pgfqpoint{1.493417in}{0.728634in}}%
\pgfpathlineto{\pgfqpoint{1.493715in}{0.728634in}}%
\pgfpathlineto{\pgfqpoint{1.494012in}{0.728634in}}%
\pgfpathlineto{\pgfqpoint{1.494310in}{0.728633in}}%
\pgfpathlineto{\pgfqpoint{1.494607in}{0.728633in}}%
\pgfpathlineto{\pgfqpoint{1.494905in}{0.728633in}}%
\pgfpathlineto{\pgfqpoint{1.495202in}{0.728632in}}%
\pgfpathlineto{\pgfqpoint{1.495499in}{0.728632in}}%
\pgfpathlineto{\pgfqpoint{1.495797in}{0.728632in}}%
\pgfpathlineto{\pgfqpoint{1.496094in}{0.728631in}}%
\pgfpathlineto{\pgfqpoint{1.496392in}{0.728631in}}%
\pgfpathlineto{\pgfqpoint{1.496689in}{0.728631in}}%
\pgfpathlineto{\pgfqpoint{1.496987in}{0.728630in}}%
\pgfpathlineto{\pgfqpoint{1.497284in}{0.728630in}}%
\pgfpathlineto{\pgfqpoint{1.497582in}{0.728630in}}%
\pgfpathlineto{\pgfqpoint{1.497879in}{0.728629in}}%
\pgfpathlineto{\pgfqpoint{1.498177in}{0.728629in}}%
\pgfpathlineto{\pgfqpoint{1.498474in}{0.728629in}}%
\pgfpathlineto{\pgfqpoint{1.498772in}{0.728628in}}%
\pgfpathlineto{\pgfqpoint{1.499069in}{0.728628in}}%
\pgfpathlineto{\pgfqpoint{1.499367in}{0.728628in}}%
\pgfpathlineto{\pgfqpoint{1.499664in}{0.728628in}}%
\pgfpathlineto{\pgfqpoint{1.499962in}{0.728627in}}%
\pgfpathlineto{\pgfqpoint{1.500259in}{0.728627in}}%
\pgfpathlineto{\pgfqpoint{1.500557in}{0.728627in}}%
\pgfpathlineto{\pgfqpoint{1.500854in}{0.728626in}}%
\pgfpathlineto{\pgfqpoint{1.501152in}{0.728626in}}%
\pgfpathlineto{\pgfqpoint{1.501449in}{0.728626in}}%
\pgfpathlineto{\pgfqpoint{1.501747in}{0.728625in}}%
\pgfpathlineto{\pgfqpoint{1.502044in}{0.728625in}}%
\pgfpathlineto{\pgfqpoint{1.502341in}{0.728625in}}%
\pgfpathlineto{\pgfqpoint{1.502639in}{0.728624in}}%
\pgfpathlineto{\pgfqpoint{1.502936in}{0.728624in}}%
\pgfpathlineto{\pgfqpoint{1.503234in}{0.728624in}}%
\pgfpathlineto{\pgfqpoint{1.503531in}{0.728623in}}%
\pgfpathlineto{\pgfqpoint{1.503829in}{0.728623in}}%
\pgfpathlineto{\pgfqpoint{1.504126in}{0.728623in}}%
\pgfpathlineto{\pgfqpoint{1.504424in}{0.728622in}}%
\pgfpathlineto{\pgfqpoint{1.504721in}{0.728622in}}%
\pgfpathlineto{\pgfqpoint{1.505019in}{0.728622in}}%
\pgfpathlineto{\pgfqpoint{1.505316in}{0.728621in}}%
\pgfpathlineto{\pgfqpoint{1.505614in}{0.728621in}}%
\pgfpathlineto{\pgfqpoint{1.505911in}{0.728621in}}%
\pgfpathlineto{\pgfqpoint{1.506209in}{0.728620in}}%
\pgfpathlineto{\pgfqpoint{1.506506in}{0.728620in}}%
\pgfpathlineto{\pgfqpoint{1.506804in}{0.728620in}}%
\pgfpathlineto{\pgfqpoint{1.507101in}{0.728619in}}%
\pgfpathlineto{\pgfqpoint{1.507399in}{0.728619in}}%
\pgfpathlineto{\pgfqpoint{1.507696in}{0.728619in}}%
\pgfpathlineto{\pgfqpoint{1.507994in}{0.728618in}}%
\pgfpathlineto{\pgfqpoint{1.508291in}{0.728618in}}%
\pgfpathlineto{\pgfqpoint{1.508588in}{0.728618in}}%
\pgfpathlineto{\pgfqpoint{1.508886in}{0.728617in}}%
\pgfpathlineto{\pgfqpoint{1.509183in}{0.728617in}}%
\pgfpathlineto{\pgfqpoint{1.509481in}{0.728617in}}%
\pgfpathlineto{\pgfqpoint{1.509778in}{0.728616in}}%
\pgfpathlineto{\pgfqpoint{1.510076in}{0.728616in}}%
\pgfpathlineto{\pgfqpoint{1.510373in}{0.728616in}}%
\pgfpathlineto{\pgfqpoint{1.510671in}{0.728616in}}%
\pgfpathlineto{\pgfqpoint{1.510968in}{0.728615in}}%
\pgfpathlineto{\pgfqpoint{1.511266in}{0.728615in}}%
\pgfpathlineto{\pgfqpoint{1.511563in}{0.728615in}}%
\pgfpathlineto{\pgfqpoint{1.511861in}{0.728614in}}%
\pgfpathlineto{\pgfqpoint{1.512158in}{0.728614in}}%
\pgfpathlineto{\pgfqpoint{1.512456in}{0.728614in}}%
\pgfpathlineto{\pgfqpoint{1.512753in}{0.728613in}}%
\pgfpathlineto{\pgfqpoint{1.513051in}{0.728613in}}%
\pgfpathlineto{\pgfqpoint{1.513348in}{0.728613in}}%
\pgfpathlineto{\pgfqpoint{1.513646in}{0.728612in}}%
\pgfpathlineto{\pgfqpoint{1.513943in}{0.728612in}}%
\pgfpathlineto{\pgfqpoint{1.514241in}{0.728612in}}%
\pgfpathlineto{\pgfqpoint{1.514538in}{0.728611in}}%
\pgfpathlineto{\pgfqpoint{1.514836in}{0.728611in}}%
\pgfpathlineto{\pgfqpoint{1.515133in}{0.728611in}}%
\pgfpathlineto{\pgfqpoint{1.515430in}{0.728610in}}%
\pgfpathlineto{\pgfqpoint{1.515728in}{0.728610in}}%
\pgfpathlineto{\pgfqpoint{1.516025in}{0.728610in}}%
\pgfpathlineto{\pgfqpoint{1.516323in}{0.728609in}}%
\pgfpathlineto{\pgfqpoint{1.516620in}{0.728609in}}%
\pgfpathlineto{\pgfqpoint{1.516918in}{0.728609in}}%
\pgfpathlineto{\pgfqpoint{1.517215in}{0.728608in}}%
\pgfpathlineto{\pgfqpoint{1.517513in}{0.728608in}}%
\pgfpathlineto{\pgfqpoint{1.517810in}{0.728608in}}%
\pgfpathlineto{\pgfqpoint{1.518108in}{0.728607in}}%
\pgfpathlineto{\pgfqpoint{1.518405in}{0.728607in}}%
\pgfpathlineto{\pgfqpoint{1.518703in}{0.728607in}}%
\pgfpathlineto{\pgfqpoint{1.519000in}{0.728606in}}%
\pgfpathlineto{\pgfqpoint{1.519298in}{0.728606in}}%
\pgfpathlineto{\pgfqpoint{1.519595in}{0.728606in}}%
\pgfpathlineto{\pgfqpoint{1.519893in}{0.728605in}}%
\pgfpathlineto{\pgfqpoint{1.520190in}{0.728605in}}%
\pgfpathlineto{\pgfqpoint{1.520488in}{0.728605in}}%
\pgfpathlineto{\pgfqpoint{1.520785in}{0.728605in}}%
\pgfpathlineto{\pgfqpoint{1.521083in}{0.728604in}}%
\pgfpathlineto{\pgfqpoint{1.521380in}{0.728604in}}%
\pgfpathlineto{\pgfqpoint{1.521678in}{0.728604in}}%
\pgfpathlineto{\pgfqpoint{1.521975in}{0.728603in}}%
\pgfpathlineto{\pgfqpoint{1.522272in}{0.728603in}}%
\pgfpathlineto{\pgfqpoint{1.522570in}{0.728603in}}%
\pgfpathlineto{\pgfqpoint{1.522867in}{0.728602in}}%
\pgfpathlineto{\pgfqpoint{1.523165in}{0.728602in}}%
\pgfpathlineto{\pgfqpoint{1.523462in}{0.728602in}}%
\pgfpathlineto{\pgfqpoint{1.523760in}{0.728601in}}%
\pgfpathlineto{\pgfqpoint{1.524057in}{0.728601in}}%
\pgfpathlineto{\pgfqpoint{1.524355in}{0.728601in}}%
\pgfpathlineto{\pgfqpoint{1.524652in}{0.728600in}}%
\pgfpathlineto{\pgfqpoint{1.524950in}{0.728600in}}%
\pgfpathlineto{\pgfqpoint{1.525247in}{0.728600in}}%
\pgfpathlineto{\pgfqpoint{1.525545in}{0.728599in}}%
\pgfpathlineto{\pgfqpoint{1.525842in}{0.728599in}}%
\pgfpathlineto{\pgfqpoint{1.526140in}{0.728599in}}%
\pgfpathlineto{\pgfqpoint{1.526437in}{0.728598in}}%
\pgfpathlineto{\pgfqpoint{1.526735in}{0.728598in}}%
\pgfpathlineto{\pgfqpoint{1.527032in}{0.728598in}}%
\pgfpathlineto{\pgfqpoint{1.527330in}{0.728597in}}%
\pgfpathlineto{\pgfqpoint{1.527627in}{0.728597in}}%
\pgfpathlineto{\pgfqpoint{1.527925in}{0.728597in}}%
\pgfpathlineto{\pgfqpoint{1.528222in}{0.728596in}}%
\pgfpathlineto{\pgfqpoint{1.528519in}{0.728596in}}%
\pgfpathlineto{\pgfqpoint{1.528817in}{0.728596in}}%
\pgfpathlineto{\pgfqpoint{1.529114in}{0.728595in}}%
\pgfpathlineto{\pgfqpoint{1.529412in}{0.728595in}}%
\pgfpathlineto{\pgfqpoint{1.529709in}{0.728595in}}%
\pgfpathlineto{\pgfqpoint{1.530007in}{0.728594in}}%
\pgfpathlineto{\pgfqpoint{1.530304in}{0.728594in}}%
\pgfpathlineto{\pgfqpoint{1.530602in}{0.728594in}}%
\pgfpathlineto{\pgfqpoint{1.530899in}{0.728593in}}%
\pgfpathlineto{\pgfqpoint{1.531197in}{0.728593in}}%
\pgfpathlineto{\pgfqpoint{1.531494in}{0.728593in}}%
\pgfpathlineto{\pgfqpoint{1.531792in}{0.728593in}}%
\pgfpathlineto{\pgfqpoint{1.532089in}{0.728592in}}%
\pgfpathlineto{\pgfqpoint{1.532387in}{0.728592in}}%
\pgfpathlineto{\pgfqpoint{1.532684in}{0.728592in}}%
\pgfpathlineto{\pgfqpoint{1.532982in}{0.728591in}}%
\pgfpathlineto{\pgfqpoint{1.533279in}{0.728591in}}%
\pgfpathlineto{\pgfqpoint{1.533577in}{0.728591in}}%
\pgfpathlineto{\pgfqpoint{1.533874in}{0.728590in}}%
\pgfpathlineto{\pgfqpoint{1.534172in}{0.728590in}}%
\pgfpathlineto{\pgfqpoint{1.534469in}{0.728590in}}%
\pgfpathlineto{\pgfqpoint{1.534767in}{0.728589in}}%
\pgfpathlineto{\pgfqpoint{1.535064in}{0.728589in}}%
\pgfpathlineto{\pgfqpoint{1.535361in}{0.728589in}}%
\pgfpathlineto{\pgfqpoint{1.535659in}{0.728588in}}%
\pgfpathlineto{\pgfqpoint{1.535956in}{0.728588in}}%
\pgfpathlineto{\pgfqpoint{1.536254in}{0.728588in}}%
\pgfpathlineto{\pgfqpoint{1.536551in}{0.728587in}}%
\pgfpathlineto{\pgfqpoint{1.536849in}{0.728587in}}%
\pgfpathlineto{\pgfqpoint{1.537146in}{0.728587in}}%
\pgfpathlineto{\pgfqpoint{1.537444in}{0.728586in}}%
\pgfpathlineto{\pgfqpoint{1.537741in}{0.728586in}}%
\pgfpathlineto{\pgfqpoint{1.538039in}{0.728586in}}%
\pgfpathlineto{\pgfqpoint{1.538336in}{0.728585in}}%
\pgfpathlineto{\pgfqpoint{1.538634in}{0.728585in}}%
\pgfpathlineto{\pgfqpoint{1.538931in}{0.728585in}}%
\pgfpathlineto{\pgfqpoint{1.539229in}{0.728584in}}%
\pgfpathlineto{\pgfqpoint{1.539526in}{0.728584in}}%
\pgfpathlineto{\pgfqpoint{1.539824in}{0.728584in}}%
\pgfpathlineto{\pgfqpoint{1.540121in}{0.728583in}}%
\pgfpathlineto{\pgfqpoint{1.540419in}{0.728583in}}%
\pgfpathlineto{\pgfqpoint{1.540716in}{0.728583in}}%
\pgfpathlineto{\pgfqpoint{1.541014in}{0.728582in}}%
\pgfpathlineto{\pgfqpoint{1.541311in}{0.728582in}}%
\pgfpathlineto{\pgfqpoint{1.541609in}{0.728582in}}%
\pgfpathlineto{\pgfqpoint{1.541906in}{0.728582in}}%
\pgfpathlineto{\pgfqpoint{1.542203in}{0.728582in}}%
\pgfpathlineto{\pgfqpoint{1.542501in}{0.728584in}}%
\pgfpathlineto{\pgfqpoint{1.542798in}{0.728587in}}%
\pgfpathlineto{\pgfqpoint{1.543096in}{0.728589in}}%
\pgfpathlineto{\pgfqpoint{1.543393in}{0.728592in}}%
\pgfpathlineto{\pgfqpoint{1.543691in}{0.728595in}}%
\pgfpathlineto{\pgfqpoint{1.543988in}{0.728598in}}%
\pgfpathlineto{\pgfqpoint{1.544286in}{0.728600in}}%
\pgfpathlineto{\pgfqpoint{1.544583in}{0.728603in}}%
\pgfpathlineto{\pgfqpoint{1.544881in}{0.728606in}}%
\pgfpathlineto{\pgfqpoint{1.545178in}{0.728609in}}%
\pgfpathlineto{\pgfqpoint{1.545476in}{0.728612in}}%
\pgfpathlineto{\pgfqpoint{1.545773in}{0.728614in}}%
\pgfpathlineto{\pgfqpoint{1.546071in}{0.728617in}}%
\pgfpathlineto{\pgfqpoint{1.546368in}{0.728620in}}%
\pgfpathlineto{\pgfqpoint{1.546666in}{0.728623in}}%
\pgfpathlineto{\pgfqpoint{1.546963in}{0.728625in}}%
\pgfpathlineto{\pgfqpoint{1.547261in}{0.728628in}}%
\pgfpathlineto{\pgfqpoint{1.547558in}{0.728631in}}%
\pgfpathlineto{\pgfqpoint{1.547856in}{0.728634in}}%
\pgfpathlineto{\pgfqpoint{1.548153in}{0.728636in}}%
\pgfpathlineto{\pgfqpoint{1.548450in}{0.728639in}}%
\pgfpathlineto{\pgfqpoint{1.548748in}{0.728642in}}%
\pgfpathlineto{\pgfqpoint{1.549045in}{0.728645in}}%
\pgfpathlineto{\pgfqpoint{1.549343in}{0.728648in}}%
\pgfpathlineto{\pgfqpoint{1.549640in}{0.728650in}}%
\pgfpathlineto{\pgfqpoint{1.549938in}{0.728653in}}%
\pgfpathlineto{\pgfqpoint{1.550235in}{0.728656in}}%
\pgfpathlineto{\pgfqpoint{1.550533in}{0.728659in}}%
\pgfpathlineto{\pgfqpoint{1.550830in}{0.728661in}}%
\pgfpathlineto{\pgfqpoint{1.551128in}{0.728664in}}%
\pgfpathlineto{\pgfqpoint{1.551425in}{0.728667in}}%
\pgfpathlineto{\pgfqpoint{1.551723in}{0.728670in}}%
\pgfpathlineto{\pgfqpoint{1.552020in}{0.728672in}}%
\pgfpathlineto{\pgfqpoint{1.552318in}{0.728675in}}%
\pgfpathlineto{\pgfqpoint{1.552615in}{0.728678in}}%
\pgfpathlineto{\pgfqpoint{1.552913in}{0.728681in}}%
\pgfpathlineto{\pgfqpoint{1.553210in}{0.728684in}}%
\pgfpathlineto{\pgfqpoint{1.553508in}{0.728686in}}%
\pgfpathlineto{\pgfqpoint{1.553805in}{0.728689in}}%
\pgfpathlineto{\pgfqpoint{1.554103in}{0.728692in}}%
\pgfpathlineto{\pgfqpoint{1.554400in}{0.728695in}}%
\pgfpathlineto{\pgfqpoint{1.554698in}{0.728697in}}%
\pgfpathlineto{\pgfqpoint{1.554995in}{0.728700in}}%
\pgfpathlineto{\pgfqpoint{1.555292in}{0.728703in}}%
\pgfpathlineto{\pgfqpoint{1.555590in}{0.728706in}}%
\pgfpathlineto{\pgfqpoint{1.555887in}{0.728708in}}%
\pgfpathlineto{\pgfqpoint{1.556185in}{0.728711in}}%
\pgfpathlineto{\pgfqpoint{1.556482in}{0.728714in}}%
\pgfpathlineto{\pgfqpoint{1.556780in}{0.728717in}}%
\pgfpathlineto{\pgfqpoint{1.557077in}{0.728720in}}%
\pgfpathlineto{\pgfqpoint{1.557375in}{0.728722in}}%
\pgfpathlineto{\pgfqpoint{1.557672in}{0.728725in}}%
\pgfpathlineto{\pgfqpoint{1.557970in}{0.728728in}}%
\pgfpathlineto{\pgfqpoint{1.558267in}{0.728731in}}%
\pgfpathlineto{\pgfqpoint{1.558565in}{0.728733in}}%
\pgfpathlineto{\pgfqpoint{1.558862in}{0.728736in}}%
\pgfpathlineto{\pgfqpoint{1.559160in}{0.728739in}}%
\pgfpathlineto{\pgfqpoint{1.559457in}{0.728742in}}%
\pgfpathlineto{\pgfqpoint{1.559755in}{0.728744in}}%
\pgfpathlineto{\pgfqpoint{1.560052in}{0.728747in}}%
\pgfpathlineto{\pgfqpoint{1.560350in}{0.728750in}}%
\pgfpathlineto{\pgfqpoint{1.560647in}{0.728753in}}%
\pgfpathlineto{\pgfqpoint{1.560945in}{0.728756in}}%
\pgfpathlineto{\pgfqpoint{1.561242in}{0.728758in}}%
\pgfpathlineto{\pgfqpoint{1.561540in}{0.728761in}}%
\pgfpathlineto{\pgfqpoint{1.561837in}{0.728764in}}%
\pgfpathlineto{\pgfqpoint{1.562134in}{0.728767in}}%
\pgfpathlineto{\pgfqpoint{1.562432in}{0.728769in}}%
\pgfpathlineto{\pgfqpoint{1.562729in}{0.728772in}}%
\pgfpathlineto{\pgfqpoint{1.563027in}{0.728775in}}%
\pgfpathlineto{\pgfqpoint{1.563324in}{0.728778in}}%
\pgfpathlineto{\pgfqpoint{1.563622in}{0.728780in}}%
\pgfpathlineto{\pgfqpoint{1.563919in}{0.728783in}}%
\pgfpathlineto{\pgfqpoint{1.564217in}{0.728786in}}%
\pgfpathlineto{\pgfqpoint{1.564514in}{0.728789in}}%
\pgfpathlineto{\pgfqpoint{1.564812in}{0.728792in}}%
\pgfpathlineto{\pgfqpoint{1.565109in}{0.728794in}}%
\pgfpathlineto{\pgfqpoint{1.565407in}{0.728796in}}%
\pgfpathlineto{\pgfqpoint{1.565704in}{0.728796in}}%
\pgfpathlineto{\pgfqpoint{1.566002in}{0.728795in}}%
\pgfpathlineto{\pgfqpoint{1.566299in}{0.728793in}}%
\pgfpathlineto{\pgfqpoint{1.566597in}{0.728792in}}%
\pgfpathlineto{\pgfqpoint{1.566894in}{0.728791in}}%
\pgfpathlineto{\pgfqpoint{1.567192in}{0.728790in}}%
\pgfpathlineto{\pgfqpoint{1.567489in}{0.728789in}}%
\pgfpathlineto{\pgfqpoint{1.567787in}{0.728788in}}%
\pgfpathlineto{\pgfqpoint{1.568084in}{0.728787in}}%
\pgfpathlineto{\pgfqpoint{1.568381in}{0.728786in}}%
\pgfpathlineto{\pgfqpoint{1.568679in}{0.728785in}}%
\pgfpathlineto{\pgfqpoint{1.568976in}{0.728784in}}%
\pgfpathlineto{\pgfqpoint{1.569274in}{0.728783in}}%
\pgfpathlineto{\pgfqpoint{1.569571in}{0.728782in}}%
\pgfpathlineto{\pgfqpoint{1.569869in}{0.728781in}}%
\pgfpathlineto{\pgfqpoint{1.570166in}{0.728780in}}%
\pgfpathlineto{\pgfqpoint{1.570464in}{0.728779in}}%
\pgfpathlineto{\pgfqpoint{1.570761in}{0.728778in}}%
\pgfpathlineto{\pgfqpoint{1.571059in}{0.728777in}}%
\pgfpathlineto{\pgfqpoint{1.571356in}{0.728775in}}%
\pgfpathlineto{\pgfqpoint{1.571654in}{0.728774in}}%
\pgfpathlineto{\pgfqpoint{1.571951in}{0.728773in}}%
\pgfpathlineto{\pgfqpoint{1.572249in}{0.728772in}}%
\pgfpathlineto{\pgfqpoint{1.572546in}{0.728771in}}%
\pgfpathlineto{\pgfqpoint{1.572844in}{0.728770in}}%
\pgfpathlineto{\pgfqpoint{1.573141in}{0.728769in}}%
\pgfpathlineto{\pgfqpoint{1.573439in}{0.728768in}}%
\pgfpathlineto{\pgfqpoint{1.573736in}{0.728767in}}%
\pgfpathlineto{\pgfqpoint{1.574034in}{0.728766in}}%
\pgfpathlineto{\pgfqpoint{1.574331in}{0.728765in}}%
\pgfpathlineto{\pgfqpoint{1.574629in}{0.728764in}}%
\pgfpathlineto{\pgfqpoint{1.574926in}{0.728763in}}%
\pgfpathlineto{\pgfqpoint{1.575223in}{0.728762in}}%
\pgfpathlineto{\pgfqpoint{1.575521in}{0.728761in}}%
\pgfpathlineto{\pgfqpoint{1.575818in}{0.728760in}}%
\pgfpathlineto{\pgfqpoint{1.576116in}{0.728759in}}%
\pgfpathlineto{\pgfqpoint{1.576413in}{0.728758in}}%
\pgfpathlineto{\pgfqpoint{1.576711in}{0.728756in}}%
\pgfpathlineto{\pgfqpoint{1.577008in}{0.728755in}}%
\pgfpathlineto{\pgfqpoint{1.577306in}{0.728754in}}%
\pgfpathlineto{\pgfqpoint{1.577603in}{0.728753in}}%
\pgfpathlineto{\pgfqpoint{1.577901in}{0.728752in}}%
\pgfpathlineto{\pgfqpoint{1.578198in}{0.728751in}}%
\pgfpathlineto{\pgfqpoint{1.578496in}{0.728750in}}%
\pgfpathlineto{\pgfqpoint{1.578793in}{0.728749in}}%
\pgfpathlineto{\pgfqpoint{1.579091in}{0.728748in}}%
\pgfpathlineto{\pgfqpoint{1.579388in}{0.728747in}}%
\pgfpathlineto{\pgfqpoint{1.579686in}{0.728746in}}%
\pgfpathlineto{\pgfqpoint{1.579983in}{0.728745in}}%
\pgfpathlineto{\pgfqpoint{1.580281in}{0.728744in}}%
\pgfpathlineto{\pgfqpoint{1.580578in}{0.728743in}}%
\pgfpathlineto{\pgfqpoint{1.580876in}{0.728742in}}%
\pgfpathlineto{\pgfqpoint{1.581173in}{0.728741in}}%
\pgfpathlineto{\pgfqpoint{1.581471in}{0.728740in}}%
\pgfpathlineto{\pgfqpoint{1.581768in}{0.728738in}}%
\pgfpathlineto{\pgfqpoint{1.582065in}{0.728737in}}%
\pgfpathlineto{\pgfqpoint{1.582363in}{0.728736in}}%
\pgfpathlineto{\pgfqpoint{1.582660in}{0.728735in}}%
\pgfpathlineto{\pgfqpoint{1.582958in}{0.728734in}}%
\pgfpathlineto{\pgfqpoint{1.583255in}{0.728733in}}%
\pgfpathlineto{\pgfqpoint{1.583553in}{0.728732in}}%
\pgfpathlineto{\pgfqpoint{1.583850in}{0.728731in}}%
\pgfpathlineto{\pgfqpoint{1.584148in}{0.728730in}}%
\pgfpathlineto{\pgfqpoint{1.584445in}{0.728729in}}%
\pgfpathlineto{\pgfqpoint{1.584743in}{0.728728in}}%
\pgfpathlineto{\pgfqpoint{1.585040in}{0.728727in}}%
\pgfpathlineto{\pgfqpoint{1.585338in}{0.728726in}}%
\pgfpathlineto{\pgfqpoint{1.585635in}{0.728725in}}%
\pgfpathlineto{\pgfqpoint{1.585933in}{0.728724in}}%
\pgfpathlineto{\pgfqpoint{1.586230in}{0.728723in}}%
\pgfpathlineto{\pgfqpoint{1.586528in}{0.728722in}}%
\pgfpathlineto{\pgfqpoint{1.586825in}{0.728721in}}%
\pgfpathlineto{\pgfqpoint{1.587123in}{0.728719in}}%
\pgfpathlineto{\pgfqpoint{1.587420in}{0.728718in}}%
\pgfpathlineto{\pgfqpoint{1.587718in}{0.728717in}}%
\pgfpathlineto{\pgfqpoint{1.588015in}{0.728716in}}%
\pgfpathlineto{\pgfqpoint{1.588313in}{0.728715in}}%
\pgfpathlineto{\pgfqpoint{1.588610in}{0.728714in}}%
\pgfpathlineto{\pgfqpoint{1.588907in}{0.728713in}}%
\pgfpathlineto{\pgfqpoint{1.589205in}{0.728712in}}%
\pgfpathlineto{\pgfqpoint{1.589502in}{0.728711in}}%
\pgfpathlineto{\pgfqpoint{1.589800in}{0.728710in}}%
\pgfpathlineto{\pgfqpoint{1.590097in}{0.728709in}}%
\pgfpathlineto{\pgfqpoint{1.590395in}{0.728708in}}%
\pgfpathlineto{\pgfqpoint{1.590692in}{0.728707in}}%
\pgfpathlineto{\pgfqpoint{1.590990in}{0.728706in}}%
\pgfpathlineto{\pgfqpoint{1.591287in}{0.728705in}}%
\pgfpathlineto{\pgfqpoint{1.591585in}{0.728704in}}%
\pgfpathlineto{\pgfqpoint{1.591882in}{0.728703in}}%
\pgfpathlineto{\pgfqpoint{1.592180in}{0.728701in}}%
\pgfpathlineto{\pgfqpoint{1.592477in}{0.728700in}}%
\pgfpathlineto{\pgfqpoint{1.592775in}{0.728699in}}%
\pgfpathlineto{\pgfqpoint{1.593072in}{0.728698in}}%
\pgfpathlineto{\pgfqpoint{1.593370in}{0.728697in}}%
\pgfpathlineto{\pgfqpoint{1.593667in}{0.728696in}}%
\pgfpathlineto{\pgfqpoint{1.593965in}{0.728695in}}%
\pgfpathlineto{\pgfqpoint{1.594262in}{0.728694in}}%
\pgfpathlineto{\pgfqpoint{1.594560in}{0.728693in}}%
\pgfpathlineto{\pgfqpoint{1.594857in}{0.728692in}}%
\pgfpathlineto{\pgfqpoint{1.595154in}{0.728691in}}%
\pgfpathlineto{\pgfqpoint{1.595452in}{0.728690in}}%
\pgfpathlineto{\pgfqpoint{1.595749in}{0.728689in}}%
\pgfpathlineto{\pgfqpoint{1.596047in}{0.728688in}}%
\pgfpathlineto{\pgfqpoint{1.596344in}{0.728687in}}%
\pgfpathlineto{\pgfqpoint{1.596642in}{0.728686in}}%
\pgfpathlineto{\pgfqpoint{1.596939in}{0.728685in}}%
\pgfpathlineto{\pgfqpoint{1.597237in}{0.728684in}}%
\pgfpathlineto{\pgfqpoint{1.597534in}{0.728682in}}%
\pgfpathlineto{\pgfqpoint{1.597832in}{0.728681in}}%
\pgfpathlineto{\pgfqpoint{1.598129in}{0.728680in}}%
\pgfpathlineto{\pgfqpoint{1.598427in}{0.728679in}}%
\pgfpathlineto{\pgfqpoint{1.598724in}{0.728678in}}%
\pgfpathlineto{\pgfqpoint{1.599022in}{0.728677in}}%
\pgfpathlineto{\pgfqpoint{1.599319in}{0.728676in}}%
\pgfpathlineto{\pgfqpoint{1.599617in}{0.728675in}}%
\pgfpathlineto{\pgfqpoint{1.599914in}{0.728674in}}%
\pgfpathlineto{\pgfqpoint{1.600212in}{0.728673in}}%
\pgfpathlineto{\pgfqpoint{1.600509in}{0.728672in}}%
\pgfpathlineto{\pgfqpoint{1.600807in}{0.728671in}}%
\pgfpathlineto{\pgfqpoint{1.601104in}{0.728670in}}%
\pgfpathlineto{\pgfqpoint{1.601402in}{0.728669in}}%
\pgfpathlineto{\pgfqpoint{1.601699in}{0.728668in}}%
\pgfpathlineto{\pgfqpoint{1.601996in}{0.728667in}}%
\pgfpathlineto{\pgfqpoint{1.602294in}{0.728666in}}%
\pgfpathlineto{\pgfqpoint{1.602591in}{0.728664in}}%
\pgfpathlineto{\pgfqpoint{1.602889in}{0.728663in}}%
\pgfpathlineto{\pgfqpoint{1.603186in}{0.728662in}}%
\pgfpathlineto{\pgfqpoint{1.603484in}{0.728661in}}%
\pgfpathlineto{\pgfqpoint{1.603781in}{0.728660in}}%
\pgfpathlineto{\pgfqpoint{1.604079in}{0.728659in}}%
\pgfpathlineto{\pgfqpoint{1.604376in}{0.728658in}}%
\pgfpathlineto{\pgfqpoint{1.604674in}{0.728657in}}%
\pgfpathlineto{\pgfqpoint{1.604971in}{0.728656in}}%
\pgfpathlineto{\pgfqpoint{1.605269in}{0.728655in}}%
\pgfpathlineto{\pgfqpoint{1.605566in}{0.728654in}}%
\pgfpathlineto{\pgfqpoint{1.605864in}{0.728653in}}%
\pgfpathlineto{\pgfqpoint{1.606161in}{0.728652in}}%
\pgfpathlineto{\pgfqpoint{1.606459in}{0.728651in}}%
\pgfpathlineto{\pgfqpoint{1.606756in}{0.728650in}}%
\pgfpathlineto{\pgfqpoint{1.607054in}{0.728649in}}%
\pgfpathlineto{\pgfqpoint{1.607351in}{0.728648in}}%
\pgfpathlineto{\pgfqpoint{1.607649in}{0.728647in}}%
\pgfpathlineto{\pgfqpoint{1.607946in}{0.728645in}}%
\pgfpathlineto{\pgfqpoint{1.608244in}{0.728644in}}%
\pgfpathlineto{\pgfqpoint{1.608541in}{0.728643in}}%
\pgfpathlineto{\pgfqpoint{1.608838in}{0.728642in}}%
\pgfpathlineto{\pgfqpoint{1.609136in}{0.728641in}}%
\pgfpathlineto{\pgfqpoint{1.609433in}{0.728640in}}%
\pgfpathlineto{\pgfqpoint{1.609731in}{0.728639in}}%
\pgfpathlineto{\pgfqpoint{1.610028in}{0.728638in}}%
\pgfpathlineto{\pgfqpoint{1.610326in}{0.728637in}}%
\pgfpathlineto{\pgfqpoint{1.610623in}{0.728636in}}%
\pgfpathlineto{\pgfqpoint{1.610921in}{0.728635in}}%
\pgfpathlineto{\pgfqpoint{1.611218in}{0.728634in}}%
\pgfpathlineto{\pgfqpoint{1.611516in}{0.728633in}}%
\pgfpathlineto{\pgfqpoint{1.611813in}{0.728632in}}%
\pgfpathlineto{\pgfqpoint{1.612111in}{0.728631in}}%
\pgfpathlineto{\pgfqpoint{1.612408in}{0.728630in}}%
\pgfpathlineto{\pgfqpoint{1.612706in}{0.728629in}}%
\pgfpathlineto{\pgfqpoint{1.613003in}{0.728627in}}%
\pgfpathlineto{\pgfqpoint{1.613301in}{0.728626in}}%
\pgfpathlineto{\pgfqpoint{1.613598in}{0.728625in}}%
\pgfpathlineto{\pgfqpoint{1.613896in}{0.728624in}}%
\pgfpathlineto{\pgfqpoint{1.614193in}{0.728623in}}%
\pgfpathlineto{\pgfqpoint{1.614491in}{0.728622in}}%
\pgfpathlineto{\pgfqpoint{1.614788in}{0.728621in}}%
\pgfpathlineto{\pgfqpoint{1.615085in}{0.728620in}}%
\pgfpathlineto{\pgfqpoint{1.615383in}{0.728619in}}%
\pgfpathlineto{\pgfqpoint{1.615680in}{0.728618in}}%
\pgfpathlineto{\pgfqpoint{1.615978in}{0.728617in}}%
\pgfpathlineto{\pgfqpoint{1.616275in}{0.728616in}}%
\pgfpathlineto{\pgfqpoint{1.616573in}{0.728615in}}%
\pgfpathlineto{\pgfqpoint{1.616870in}{0.728614in}}%
\pgfpathlineto{\pgfqpoint{1.617168in}{0.728613in}}%
\pgfpathlineto{\pgfqpoint{1.617465in}{0.728612in}}%
\pgfpathlineto{\pgfqpoint{1.617763in}{0.728611in}}%
\pgfpathlineto{\pgfqpoint{1.618060in}{0.728610in}}%
\pgfpathlineto{\pgfqpoint{1.618358in}{0.728608in}}%
\pgfpathlineto{\pgfqpoint{1.618655in}{0.728607in}}%
\pgfpathlineto{\pgfqpoint{1.618953in}{0.728606in}}%
\pgfpathlineto{\pgfqpoint{1.619250in}{0.728605in}}%
\pgfpathlineto{\pgfqpoint{1.619548in}{0.728604in}}%
\pgfpathlineto{\pgfqpoint{1.619845in}{0.728603in}}%
\pgfpathlineto{\pgfqpoint{1.620143in}{0.728602in}}%
\pgfpathlineto{\pgfqpoint{1.620440in}{0.728601in}}%
\pgfpathlineto{\pgfqpoint{1.620738in}{0.728600in}}%
\pgfpathlineto{\pgfqpoint{1.621035in}{0.728599in}}%
\pgfpathlineto{\pgfqpoint{1.621333in}{0.728598in}}%
\pgfpathlineto{\pgfqpoint{1.621630in}{0.728597in}}%
\pgfpathlineto{\pgfqpoint{1.621927in}{0.728596in}}%
\pgfpathlineto{\pgfqpoint{1.622225in}{0.728595in}}%
\pgfpathlineto{\pgfqpoint{1.622522in}{0.728594in}}%
\pgfpathlineto{\pgfqpoint{1.622820in}{0.728593in}}%
\pgfpathlineto{\pgfqpoint{1.623117in}{0.728592in}}%
\pgfpathlineto{\pgfqpoint{1.623415in}{0.728590in}}%
\pgfpathlineto{\pgfqpoint{1.623712in}{0.728589in}}%
\pgfpathlineto{\pgfqpoint{1.624010in}{0.728588in}}%
\pgfpathlineto{\pgfqpoint{1.624307in}{0.728587in}}%
\pgfpathlineto{\pgfqpoint{1.624605in}{0.728586in}}%
\pgfpathlineto{\pgfqpoint{1.624902in}{0.728585in}}%
\pgfpathlineto{\pgfqpoint{1.625200in}{0.728584in}}%
\pgfpathlineto{\pgfqpoint{1.625497in}{0.728583in}}%
\pgfpathlineto{\pgfqpoint{1.625795in}{0.728582in}}%
\pgfpathlineto{\pgfqpoint{1.626092in}{0.728581in}}%
\pgfpathlineto{\pgfqpoint{1.626390in}{0.728580in}}%
\pgfpathlineto{\pgfqpoint{1.626687in}{0.728579in}}%
\pgfpathlineto{\pgfqpoint{1.626985in}{0.728578in}}%
\pgfpathlineto{\pgfqpoint{1.627282in}{0.728577in}}%
\pgfpathlineto{\pgfqpoint{1.627580in}{0.728576in}}%
\pgfpathlineto{\pgfqpoint{1.627877in}{0.728575in}}%
\pgfpathlineto{\pgfqpoint{1.628175in}{0.728574in}}%
\pgfpathlineto{\pgfqpoint{1.628472in}{0.728572in}}%
\pgfpathlineto{\pgfqpoint{1.628769in}{0.728571in}}%
\pgfpathlineto{\pgfqpoint{1.629067in}{0.728570in}}%
\pgfpathlineto{\pgfqpoint{1.629364in}{0.728569in}}%
\pgfpathlineto{\pgfqpoint{1.629662in}{0.728568in}}%
\pgfpathlineto{\pgfqpoint{1.629959in}{0.728567in}}%
\pgfpathlineto{\pgfqpoint{1.630257in}{0.728566in}}%
\pgfpathlineto{\pgfqpoint{1.630554in}{0.728565in}}%
\pgfpathlineto{\pgfqpoint{1.630852in}{0.728564in}}%
\pgfpathlineto{\pgfqpoint{1.631149in}{0.728563in}}%
\pgfpathlineto{\pgfqpoint{1.631447in}{0.728562in}}%
\pgfpathlineto{\pgfqpoint{1.631744in}{0.728561in}}%
\pgfpathlineto{\pgfqpoint{1.632042in}{0.728560in}}%
\pgfpathlineto{\pgfqpoint{1.632339in}{0.728559in}}%
\pgfpathlineto{\pgfqpoint{1.632637in}{0.728558in}}%
\pgfpathlineto{\pgfqpoint{1.632934in}{0.728557in}}%
\pgfpathlineto{\pgfqpoint{1.633232in}{0.728556in}}%
\pgfpathlineto{\pgfqpoint{1.633529in}{0.728555in}}%
\pgfpathlineto{\pgfqpoint{1.633827in}{0.728553in}}%
\pgfpathlineto{\pgfqpoint{1.634124in}{0.728552in}}%
\pgfpathlineto{\pgfqpoint{1.634422in}{0.728551in}}%
\pgfpathlineto{\pgfqpoint{1.634719in}{0.728550in}}%
\pgfpathlineto{\pgfqpoint{1.635016in}{0.728549in}}%
\pgfpathlineto{\pgfqpoint{1.635314in}{0.728548in}}%
\pgfpathlineto{\pgfqpoint{1.635611in}{0.728547in}}%
\pgfpathlineto{\pgfqpoint{1.635909in}{0.728546in}}%
\pgfpathlineto{\pgfqpoint{1.636206in}{0.728545in}}%
\pgfpathlineto{\pgfqpoint{1.636504in}{0.728544in}}%
\pgfpathlineto{\pgfqpoint{1.636801in}{0.728543in}}%
\pgfpathlineto{\pgfqpoint{1.637099in}{0.728542in}}%
\pgfpathlineto{\pgfqpoint{1.637396in}{0.728541in}}%
\pgfpathlineto{\pgfqpoint{1.637694in}{0.728540in}}%
\pgfpathlineto{\pgfqpoint{1.637991in}{0.728539in}}%
\pgfpathlineto{\pgfqpoint{1.638289in}{0.728538in}}%
\pgfpathlineto{\pgfqpoint{1.638586in}{0.728537in}}%
\pgfpathlineto{\pgfqpoint{1.638884in}{0.728535in}}%
\pgfpathlineto{\pgfqpoint{1.639181in}{0.728534in}}%
\pgfpathlineto{\pgfqpoint{1.639479in}{0.728533in}}%
\pgfpathlineto{\pgfqpoint{1.639776in}{0.728532in}}%
\pgfpathlineto{\pgfqpoint{1.640074in}{0.728531in}}%
\pgfpathlineto{\pgfqpoint{1.640371in}{0.728530in}}%
\pgfpathlineto{\pgfqpoint{1.640669in}{0.728529in}}%
\pgfpathlineto{\pgfqpoint{1.640966in}{0.728528in}}%
\pgfpathlineto{\pgfqpoint{1.641264in}{0.728527in}}%
\pgfpathlineto{\pgfqpoint{1.641561in}{0.728526in}}%
\pgfpathlineto{\pgfqpoint{1.641858in}{0.728525in}}%
\pgfpathlineto{\pgfqpoint{1.642156in}{0.728524in}}%
\pgfpathlineto{\pgfqpoint{1.642453in}{0.728523in}}%
\pgfpathlineto{\pgfqpoint{1.642751in}{0.728522in}}%
\pgfpathlineto{\pgfqpoint{1.643048in}{0.728521in}}%
\pgfpathlineto{\pgfqpoint{1.643346in}{0.728520in}}%
\pgfpathlineto{\pgfqpoint{1.643643in}{0.728519in}}%
\pgfpathlineto{\pgfqpoint{1.643941in}{0.728518in}}%
\pgfpathlineto{\pgfqpoint{1.644238in}{0.728516in}}%
\pgfpathlineto{\pgfqpoint{1.644536in}{0.728515in}}%
\pgfpathlineto{\pgfqpoint{1.644833in}{0.728514in}}%
\pgfpathlineto{\pgfqpoint{1.645131in}{0.728513in}}%
\pgfpathlineto{\pgfqpoint{1.645428in}{0.728512in}}%
\pgfpathlineto{\pgfqpoint{1.645726in}{0.728511in}}%
\pgfpathlineto{\pgfqpoint{1.646023in}{0.728510in}}%
\pgfpathlineto{\pgfqpoint{1.646321in}{0.728509in}}%
\pgfpathlineto{\pgfqpoint{1.646618in}{0.728508in}}%
\pgfpathlineto{\pgfqpoint{1.646916in}{0.728507in}}%
\pgfpathlineto{\pgfqpoint{1.647213in}{0.728506in}}%
\pgfpathlineto{\pgfqpoint{1.647511in}{0.728505in}}%
\pgfpathlineto{\pgfqpoint{1.647808in}{0.728504in}}%
\pgfpathlineto{\pgfqpoint{1.648106in}{0.728503in}}%
\pgfpathlineto{\pgfqpoint{1.648403in}{0.728502in}}%
\pgfpathlineto{\pgfqpoint{1.648700in}{0.728501in}}%
\pgfpathlineto{\pgfqpoint{1.648998in}{0.728500in}}%
\pgfpathlineto{\pgfqpoint{1.649295in}{0.728498in}}%
\pgfpathlineto{\pgfqpoint{1.649593in}{0.728497in}}%
\pgfpathlineto{\pgfqpoint{1.649890in}{0.728496in}}%
\pgfpathlineto{\pgfqpoint{1.650188in}{0.728495in}}%
\pgfpathlineto{\pgfqpoint{1.650485in}{0.728494in}}%
\pgfpathlineto{\pgfqpoint{1.650783in}{0.728493in}}%
\pgfpathlineto{\pgfqpoint{1.651080in}{0.728492in}}%
\pgfpathlineto{\pgfqpoint{1.651378in}{0.728491in}}%
\pgfpathlineto{\pgfqpoint{1.651675in}{0.728490in}}%
\pgfpathlineto{\pgfqpoint{1.651973in}{0.728489in}}%
\pgfpathlineto{\pgfqpoint{1.652270in}{0.728488in}}%
\pgfpathlineto{\pgfqpoint{1.652568in}{0.728487in}}%
\pgfpathlineto{\pgfqpoint{1.652865in}{0.728486in}}%
\pgfpathlineto{\pgfqpoint{1.653163in}{0.728485in}}%
\pgfpathlineto{\pgfqpoint{1.653460in}{0.728484in}}%
\pgfpathlineto{\pgfqpoint{1.653758in}{0.728483in}}%
\pgfpathlineto{\pgfqpoint{1.654055in}{0.728482in}}%
\pgfpathlineto{\pgfqpoint{1.654353in}{0.728481in}}%
\pgfpathlineto{\pgfqpoint{1.654650in}{0.728479in}}%
\pgfpathlineto{\pgfqpoint{1.654947in}{0.728478in}}%
\pgfpathlineto{\pgfqpoint{1.655245in}{0.728477in}}%
\pgfpathlineto{\pgfqpoint{1.655542in}{0.728476in}}%
\pgfpathlineto{\pgfqpoint{1.655840in}{0.728475in}}%
\pgfpathlineto{\pgfqpoint{1.656137in}{0.728474in}}%
\pgfpathlineto{\pgfqpoint{1.656435in}{0.728473in}}%
\pgfpathlineto{\pgfqpoint{1.656732in}{0.728472in}}%
\pgfpathlineto{\pgfqpoint{1.657030in}{0.728471in}}%
\pgfpathlineto{\pgfqpoint{1.657327in}{0.728470in}}%
\pgfpathlineto{\pgfqpoint{1.657625in}{0.728469in}}%
\pgfpathlineto{\pgfqpoint{1.657922in}{0.728468in}}%
\pgfpathlineto{\pgfqpoint{1.658220in}{0.728467in}}%
\pgfpathlineto{\pgfqpoint{1.658517in}{0.728466in}}%
\pgfpathlineto{\pgfqpoint{1.658815in}{0.728465in}}%
\pgfpathlineto{\pgfqpoint{1.659112in}{0.728464in}}%
\pgfpathlineto{\pgfqpoint{1.659410in}{0.728463in}}%
\pgfpathlineto{\pgfqpoint{1.659707in}{0.728461in}}%
\pgfpathlineto{\pgfqpoint{1.660005in}{0.728460in}}%
\pgfpathlineto{\pgfqpoint{1.660302in}{0.728459in}}%
\pgfpathlineto{\pgfqpoint{1.660600in}{0.728458in}}%
\pgfpathlineto{\pgfqpoint{1.660897in}{0.728457in}}%
\pgfpathlineto{\pgfqpoint{1.661195in}{0.728456in}}%
\pgfpathlineto{\pgfqpoint{1.661492in}{0.728455in}}%
\pgfpathlineto{\pgfqpoint{1.661789in}{0.728454in}}%
\pgfpathlineto{\pgfqpoint{1.662087in}{0.728453in}}%
\pgfpathlineto{\pgfqpoint{1.662384in}{0.728452in}}%
\pgfpathlineto{\pgfqpoint{1.662682in}{0.728451in}}%
\pgfpathlineto{\pgfqpoint{1.662979in}{0.728450in}}%
\pgfpathlineto{\pgfqpoint{1.663277in}{0.728449in}}%
\pgfpathlineto{\pgfqpoint{1.663574in}{0.728448in}}%
\pgfpathlineto{\pgfqpoint{1.663872in}{0.728447in}}%
\pgfpathlineto{\pgfqpoint{1.664169in}{0.728446in}}%
\pgfpathlineto{\pgfqpoint{1.664467in}{0.728445in}}%
\pgfpathlineto{\pgfqpoint{1.664764in}{0.728444in}}%
\pgfpathlineto{\pgfqpoint{1.665062in}{0.728442in}}%
\pgfpathlineto{\pgfqpoint{1.665359in}{0.728441in}}%
\pgfpathlineto{\pgfqpoint{1.665657in}{0.728440in}}%
\pgfpathlineto{\pgfqpoint{1.665954in}{0.728439in}}%
\pgfpathlineto{\pgfqpoint{1.666252in}{0.728438in}}%
\pgfpathlineto{\pgfqpoint{1.666549in}{0.728437in}}%
\pgfpathlineto{\pgfqpoint{1.666847in}{0.728436in}}%
\pgfpathlineto{\pgfqpoint{1.667144in}{0.728435in}}%
\pgfpathlineto{\pgfqpoint{1.667442in}{0.728434in}}%
\pgfpathlineto{\pgfqpoint{1.667739in}{0.728433in}}%
\pgfpathlineto{\pgfqpoint{1.668037in}{0.728432in}}%
\pgfpathlineto{\pgfqpoint{1.668334in}{0.728431in}}%
\pgfpathlineto{\pgfqpoint{1.668631in}{0.728430in}}%
\pgfpathlineto{\pgfqpoint{1.668929in}{0.728429in}}%
\pgfpathlineto{\pgfqpoint{1.669226in}{0.728428in}}%
\pgfpathlineto{\pgfqpoint{1.669524in}{0.728427in}}%
\pgfpathlineto{\pgfqpoint{1.669821in}{0.728426in}}%
\pgfpathlineto{\pgfqpoint{1.670119in}{0.728424in}}%
\pgfpathlineto{\pgfqpoint{1.670416in}{0.728423in}}%
\pgfpathlineto{\pgfqpoint{1.670714in}{0.728422in}}%
\pgfpathlineto{\pgfqpoint{1.671011in}{0.728421in}}%
\pgfpathlineto{\pgfqpoint{1.671309in}{0.728420in}}%
\pgfpathlineto{\pgfqpoint{1.671606in}{0.728419in}}%
\pgfpathlineto{\pgfqpoint{1.671904in}{0.728418in}}%
\pgfpathlineto{\pgfqpoint{1.672201in}{0.728417in}}%
\pgfpathlineto{\pgfqpoint{1.672499in}{0.728416in}}%
\pgfpathlineto{\pgfqpoint{1.672796in}{0.728415in}}%
\pgfpathlineto{\pgfqpoint{1.673094in}{0.728414in}}%
\pgfpathlineto{\pgfqpoint{1.673391in}{0.728413in}}%
\pgfpathlineto{\pgfqpoint{1.673689in}{0.728412in}}%
\pgfpathlineto{\pgfqpoint{1.673986in}{0.728411in}}%
\pgfpathlineto{\pgfqpoint{1.674284in}{0.728410in}}%
\pgfpathlineto{\pgfqpoint{1.674581in}{0.728409in}}%
\pgfpathlineto{\pgfqpoint{1.674878in}{0.728408in}}%
\pgfpathlineto{\pgfqpoint{1.675176in}{0.728407in}}%
\pgfpathlineto{\pgfqpoint{1.675473in}{0.728405in}}%
\pgfpathlineto{\pgfqpoint{1.675771in}{0.728404in}}%
\pgfpathlineto{\pgfqpoint{1.676068in}{0.728403in}}%
\pgfpathlineto{\pgfqpoint{1.676366in}{0.728402in}}%
\pgfpathlineto{\pgfqpoint{1.676663in}{0.728401in}}%
\pgfpathlineto{\pgfqpoint{1.676961in}{0.728400in}}%
\pgfpathlineto{\pgfqpoint{1.677258in}{0.728399in}}%
\pgfpathlineto{\pgfqpoint{1.677556in}{0.728398in}}%
\pgfpathlineto{\pgfqpoint{1.677853in}{0.728397in}}%
\pgfpathlineto{\pgfqpoint{1.678151in}{0.728396in}}%
\pgfpathlineto{\pgfqpoint{1.678448in}{0.728395in}}%
\pgfpathlineto{\pgfqpoint{1.678746in}{0.728394in}}%
\pgfpathlineto{\pgfqpoint{1.679043in}{0.728393in}}%
\pgfpathlineto{\pgfqpoint{1.679341in}{0.728392in}}%
\pgfpathlineto{\pgfqpoint{1.679638in}{0.728391in}}%
\pgfpathlineto{\pgfqpoint{1.679936in}{0.728390in}}%
\pgfpathlineto{\pgfqpoint{1.680233in}{0.728389in}}%
\pgfpathlineto{\pgfqpoint{1.680531in}{0.728387in}}%
\pgfpathlineto{\pgfqpoint{1.680828in}{0.728386in}}%
\pgfpathlineto{\pgfqpoint{1.681126in}{0.728385in}}%
\pgfpathlineto{\pgfqpoint{1.681423in}{0.728384in}}%
\pgfpathlineto{\pgfqpoint{1.681720in}{0.728383in}}%
\pgfpathlineto{\pgfqpoint{1.682018in}{0.728382in}}%
\pgfpathlineto{\pgfqpoint{1.682315in}{0.728381in}}%
\pgfpathlineto{\pgfqpoint{1.682613in}{0.728380in}}%
\pgfpathlineto{\pgfqpoint{1.682910in}{0.728379in}}%
\pgfpathlineto{\pgfqpoint{1.683208in}{0.728378in}}%
\pgfpathlineto{\pgfqpoint{1.683505in}{0.728377in}}%
\pgfpathlineto{\pgfqpoint{1.683803in}{0.728376in}}%
\pgfpathlineto{\pgfqpoint{1.684100in}{0.728375in}}%
\pgfpathlineto{\pgfqpoint{1.684398in}{0.728374in}}%
\pgfpathlineto{\pgfqpoint{1.684695in}{0.728373in}}%
\pgfpathlineto{\pgfqpoint{1.684993in}{0.728372in}}%
\pgfpathlineto{\pgfqpoint{1.685290in}{0.728371in}}%
\pgfpathlineto{\pgfqpoint{1.685588in}{0.728370in}}%
\pgfpathlineto{\pgfqpoint{1.685885in}{0.728368in}}%
\pgfpathlineto{\pgfqpoint{1.686183in}{0.728367in}}%
\pgfpathlineto{\pgfqpoint{1.686480in}{0.728366in}}%
\pgfpathlineto{\pgfqpoint{1.686778in}{0.728365in}}%
\pgfpathlineto{\pgfqpoint{1.687075in}{0.728364in}}%
\pgfpathlineto{\pgfqpoint{1.687373in}{0.728363in}}%
\pgfpathlineto{\pgfqpoint{1.687670in}{0.728362in}}%
\pgfpathlineto{\pgfqpoint{1.687968in}{0.728361in}}%
\pgfpathlineto{\pgfqpoint{1.688265in}{0.728360in}}%
\pgfpathlineto{\pgfqpoint{1.688562in}{0.728359in}}%
\pgfpathlineto{\pgfqpoint{1.688860in}{0.728358in}}%
\pgfpathlineto{\pgfqpoint{1.689157in}{0.728357in}}%
\pgfpathlineto{\pgfqpoint{1.689455in}{0.728356in}}%
\pgfpathlineto{\pgfqpoint{1.689752in}{0.728355in}}%
\pgfpathlineto{\pgfqpoint{1.690050in}{0.728354in}}%
\pgfpathlineto{\pgfqpoint{1.690347in}{0.728353in}}%
\pgfpathlineto{\pgfqpoint{1.690645in}{0.728352in}}%
\pgfpathlineto{\pgfqpoint{1.690942in}{0.728350in}}%
\pgfpathlineto{\pgfqpoint{1.691240in}{0.728349in}}%
\pgfpathlineto{\pgfqpoint{1.691537in}{0.728348in}}%
\pgfpathlineto{\pgfqpoint{1.691835in}{0.728347in}}%
\pgfpathlineto{\pgfqpoint{1.692132in}{0.728346in}}%
\pgfpathlineto{\pgfqpoint{1.692430in}{0.728345in}}%
\pgfpathlineto{\pgfqpoint{1.692727in}{0.728344in}}%
\pgfpathlineto{\pgfqpoint{1.693025in}{0.728343in}}%
\pgfpathlineto{\pgfqpoint{1.693322in}{0.728342in}}%
\pgfpathlineto{\pgfqpoint{1.693620in}{0.728341in}}%
\pgfpathlineto{\pgfqpoint{1.693917in}{0.728340in}}%
\pgfpathlineto{\pgfqpoint{1.694215in}{0.728339in}}%
\pgfpathlineto{\pgfqpoint{1.694512in}{0.728338in}}%
\pgfpathlineto{\pgfqpoint{1.694809in}{0.728337in}}%
\pgfpathlineto{\pgfqpoint{1.695107in}{0.728336in}}%
\pgfpathlineto{\pgfqpoint{1.695404in}{0.728335in}}%
\pgfpathlineto{\pgfqpoint{1.695702in}{0.728334in}}%
\pgfpathlineto{\pgfqpoint{1.695999in}{0.728333in}}%
\pgfpathlineto{\pgfqpoint{1.696297in}{0.728331in}}%
\pgfpathlineto{\pgfqpoint{1.696594in}{0.728330in}}%
\pgfpathlineto{\pgfqpoint{1.696892in}{0.728329in}}%
\pgfpathlineto{\pgfqpoint{1.697189in}{0.728328in}}%
\pgfpathlineto{\pgfqpoint{1.697487in}{0.728327in}}%
\pgfpathlineto{\pgfqpoint{1.697784in}{0.728326in}}%
\pgfpathlineto{\pgfqpoint{1.698082in}{0.728325in}}%
\pgfpathlineto{\pgfqpoint{1.698379in}{0.728324in}}%
\pgfpathlineto{\pgfqpoint{1.698677in}{0.728323in}}%
\pgfpathlineto{\pgfqpoint{1.698974in}{0.728322in}}%
\pgfpathlineto{\pgfqpoint{1.699272in}{0.728321in}}%
\pgfpathlineto{\pgfqpoint{1.699569in}{0.728320in}}%
\pgfpathlineto{\pgfqpoint{1.699867in}{0.728319in}}%
\pgfpathlineto{\pgfqpoint{1.700164in}{0.728318in}}%
\pgfpathlineto{\pgfqpoint{1.700462in}{0.728317in}}%
\pgfpathlineto{\pgfqpoint{1.700759in}{0.728316in}}%
\pgfpathlineto{\pgfqpoint{1.701057in}{0.728315in}}%
\pgfpathlineto{\pgfqpoint{1.701354in}{0.728313in}}%
\pgfpathlineto{\pgfqpoint{1.701651in}{0.728312in}}%
\pgfpathlineto{\pgfqpoint{1.701949in}{0.728311in}}%
\pgfpathlineto{\pgfqpoint{1.702246in}{0.728310in}}%
\pgfpathlineto{\pgfqpoint{1.702544in}{0.728309in}}%
\pgfpathlineto{\pgfqpoint{1.702841in}{0.728308in}}%
\pgfpathlineto{\pgfqpoint{1.703139in}{0.728307in}}%
\pgfpathlineto{\pgfqpoint{1.703436in}{0.728306in}}%
\pgfpathlineto{\pgfqpoint{1.703734in}{0.728305in}}%
\pgfpathlineto{\pgfqpoint{1.704031in}{0.728304in}}%
\pgfpathlineto{\pgfqpoint{1.704329in}{0.728303in}}%
\pgfpathlineto{\pgfqpoint{1.704626in}{0.728302in}}%
\pgfpathlineto{\pgfqpoint{1.704924in}{0.728301in}}%
\pgfpathlineto{\pgfqpoint{1.705221in}{0.728300in}}%
\pgfpathlineto{\pgfqpoint{1.705519in}{0.728299in}}%
\pgfpathlineto{\pgfqpoint{1.705816in}{0.728298in}}%
\pgfpathlineto{\pgfqpoint{1.706114in}{0.728297in}}%
\pgfpathlineto{\pgfqpoint{1.706411in}{0.728296in}}%
\pgfpathlineto{\pgfqpoint{1.706709in}{0.728294in}}%
\pgfpathlineto{\pgfqpoint{1.707006in}{0.728293in}}%
\pgfpathlineto{\pgfqpoint{1.707304in}{0.728292in}}%
\pgfpathlineto{\pgfqpoint{1.707601in}{0.728291in}}%
\pgfpathlineto{\pgfqpoint{1.707899in}{0.728290in}}%
\pgfpathlineto{\pgfqpoint{1.708196in}{0.728289in}}%
\pgfpathlineto{\pgfqpoint{1.708493in}{0.728288in}}%
\pgfpathlineto{\pgfqpoint{1.708791in}{0.728287in}}%
\pgfpathlineto{\pgfqpoint{1.709088in}{0.728286in}}%
\pgfpathlineto{\pgfqpoint{1.709386in}{0.728285in}}%
\pgfpathlineto{\pgfqpoint{1.709683in}{0.728284in}}%
\pgfpathlineto{\pgfqpoint{1.709981in}{0.728283in}}%
\pgfpathlineto{\pgfqpoint{1.710278in}{0.728282in}}%
\pgfpathlineto{\pgfqpoint{1.710576in}{0.728281in}}%
\pgfpathlineto{\pgfqpoint{1.710873in}{0.728280in}}%
\pgfpathlineto{\pgfqpoint{1.711171in}{0.728279in}}%
\pgfpathlineto{\pgfqpoint{1.711468in}{0.728278in}}%
\pgfpathlineto{\pgfqpoint{1.711766in}{0.728276in}}%
\pgfpathlineto{\pgfqpoint{1.712063in}{0.728275in}}%
\pgfpathlineto{\pgfqpoint{1.712361in}{0.728274in}}%
\pgfpathlineto{\pgfqpoint{1.712658in}{0.728273in}}%
\pgfpathlineto{\pgfqpoint{1.712956in}{0.728272in}}%
\pgfpathlineto{\pgfqpoint{1.713253in}{0.728271in}}%
\pgfpathlineto{\pgfqpoint{1.713551in}{0.728270in}}%
\pgfpathlineto{\pgfqpoint{1.713848in}{0.728269in}}%
\pgfpathlineto{\pgfqpoint{1.714146in}{0.728268in}}%
\pgfpathlineto{\pgfqpoint{1.714443in}{0.728267in}}%
\pgfpathlineto{\pgfqpoint{1.714740in}{0.728266in}}%
\pgfpathlineto{\pgfqpoint{1.715038in}{0.728265in}}%
\pgfpathlineto{\pgfqpoint{1.715335in}{0.728264in}}%
\pgfpathlineto{\pgfqpoint{1.715633in}{0.728263in}}%
\pgfpathlineto{\pgfqpoint{1.715930in}{0.728262in}}%
\pgfpathlineto{\pgfqpoint{1.716228in}{0.728261in}}%
\pgfpathlineto{\pgfqpoint{1.716525in}{0.728260in}}%
\pgfpathlineto{\pgfqpoint{1.716823in}{0.728259in}}%
\pgfpathlineto{\pgfqpoint{1.717120in}{0.728257in}}%
\pgfpathlineto{\pgfqpoint{1.717418in}{0.728256in}}%
\pgfpathlineto{\pgfqpoint{1.717715in}{0.728255in}}%
\pgfpathlineto{\pgfqpoint{1.718013in}{0.728254in}}%
\pgfpathlineto{\pgfqpoint{1.718310in}{0.728253in}}%
\pgfpathlineto{\pgfqpoint{1.718608in}{0.728252in}}%
\pgfpathlineto{\pgfqpoint{1.718905in}{0.728251in}}%
\pgfpathlineto{\pgfqpoint{1.719203in}{0.728250in}}%
\pgfpathlineto{\pgfqpoint{1.719500in}{0.728249in}}%
\pgfpathlineto{\pgfqpoint{1.719798in}{0.728248in}}%
\pgfpathlineto{\pgfqpoint{1.720095in}{0.728247in}}%
\pgfpathlineto{\pgfqpoint{1.720393in}{0.728246in}}%
\pgfpathlineto{\pgfqpoint{1.720690in}{0.728245in}}%
\pgfpathlineto{\pgfqpoint{1.720988in}{0.728244in}}%
\pgfpathlineto{\pgfqpoint{1.721285in}{0.728243in}}%
\pgfpathlineto{\pgfqpoint{1.721582in}{0.728242in}}%
\pgfpathlineto{\pgfqpoint{1.721880in}{0.728241in}}%
\pgfpathlineto{\pgfqpoint{1.722177in}{0.728239in}}%
\pgfpathlineto{\pgfqpoint{1.722475in}{0.728238in}}%
\pgfpathlineto{\pgfqpoint{1.722772in}{0.728237in}}%
\pgfpathlineto{\pgfqpoint{1.723070in}{0.728236in}}%
\pgfpathlineto{\pgfqpoint{1.723367in}{0.728235in}}%
\pgfpathlineto{\pgfqpoint{1.723665in}{0.728234in}}%
\pgfpathlineto{\pgfqpoint{1.723962in}{0.728233in}}%
\pgfpathlineto{\pgfqpoint{1.724260in}{0.728232in}}%
\pgfpathlineto{\pgfqpoint{1.724557in}{0.728231in}}%
\pgfpathlineto{\pgfqpoint{1.724855in}{0.728230in}}%
\pgfpathlineto{\pgfqpoint{1.725152in}{0.728229in}}%
\pgfpathlineto{\pgfqpoint{1.725450in}{0.728228in}}%
\pgfpathlineto{\pgfqpoint{1.725747in}{0.728227in}}%
\pgfpathlineto{\pgfqpoint{1.726045in}{0.728226in}}%
\pgfpathlineto{\pgfqpoint{1.726342in}{0.728225in}}%
\pgfpathlineto{\pgfqpoint{1.726640in}{0.728224in}}%
\pgfpathlineto{\pgfqpoint{1.726937in}{0.728223in}}%
\pgfpathlineto{\pgfqpoint{1.727235in}{0.728222in}}%
\pgfpathlineto{\pgfqpoint{1.727532in}{0.728220in}}%
\pgfpathlineto{\pgfqpoint{1.727830in}{0.728219in}}%
\pgfpathlineto{\pgfqpoint{1.728127in}{0.728218in}}%
\pgfpathlineto{\pgfqpoint{1.728424in}{0.728217in}}%
\pgfpathlineto{\pgfqpoint{1.728722in}{0.728216in}}%
\pgfpathlineto{\pgfqpoint{1.729019in}{0.728215in}}%
\pgfpathlineto{\pgfqpoint{1.729317in}{0.728214in}}%
\pgfpathlineto{\pgfqpoint{1.729614in}{0.728213in}}%
\pgfpathlineto{\pgfqpoint{1.729912in}{0.728212in}}%
\pgfpathlineto{\pgfqpoint{1.730209in}{0.728211in}}%
\pgfpathlineto{\pgfqpoint{1.730507in}{0.728210in}}%
\pgfpathlineto{\pgfqpoint{1.730804in}{0.728209in}}%
\pgfpathlineto{\pgfqpoint{1.731102in}{0.728208in}}%
\pgfpathlineto{\pgfqpoint{1.731399in}{0.728207in}}%
\pgfpathlineto{\pgfqpoint{1.731697in}{0.728206in}}%
\pgfpathlineto{\pgfqpoint{1.731994in}{0.728205in}}%
\pgfpathlineto{\pgfqpoint{1.732292in}{0.728204in}}%
\pgfpathlineto{\pgfqpoint{1.732589in}{0.728202in}}%
\pgfpathlineto{\pgfqpoint{1.732887in}{0.728201in}}%
\pgfpathlineto{\pgfqpoint{1.733184in}{0.728200in}}%
\pgfpathlineto{\pgfqpoint{1.733482in}{0.728199in}}%
\pgfpathlineto{\pgfqpoint{1.733779in}{0.728198in}}%
\pgfpathlineto{\pgfqpoint{1.734077in}{0.728197in}}%
\pgfpathlineto{\pgfqpoint{1.734374in}{0.728196in}}%
\pgfpathlineto{\pgfqpoint{1.734671in}{0.728195in}}%
\pgfpathlineto{\pgfqpoint{1.734969in}{0.728194in}}%
\pgfpathlineto{\pgfqpoint{1.735266in}{0.728193in}}%
\pgfpathlineto{\pgfqpoint{1.735564in}{0.728192in}}%
\pgfpathlineto{\pgfqpoint{1.735861in}{0.728191in}}%
\pgfpathlineto{\pgfqpoint{1.736159in}{0.728190in}}%
\pgfpathlineto{\pgfqpoint{1.736456in}{0.728189in}}%
\pgfpathlineto{\pgfqpoint{1.736754in}{0.728188in}}%
\pgfpathlineto{\pgfqpoint{1.737051in}{0.728187in}}%
\pgfpathlineto{\pgfqpoint{1.737349in}{0.728186in}}%
\pgfpathlineto{\pgfqpoint{1.737646in}{0.728185in}}%
\pgfpathlineto{\pgfqpoint{1.737944in}{0.728183in}}%
\pgfpathlineto{\pgfqpoint{1.738241in}{0.728182in}}%
\pgfpathlineto{\pgfqpoint{1.738539in}{0.728181in}}%
\pgfpathlineto{\pgfqpoint{1.738836in}{0.728180in}}%
\pgfpathlineto{\pgfqpoint{1.739134in}{0.728179in}}%
\pgfpathlineto{\pgfqpoint{1.739431in}{0.728178in}}%
\pgfpathlineto{\pgfqpoint{1.739729in}{0.728177in}}%
\pgfpathlineto{\pgfqpoint{1.740026in}{0.728176in}}%
\pgfpathlineto{\pgfqpoint{1.740324in}{0.728175in}}%
\pgfpathlineto{\pgfqpoint{1.740621in}{0.728174in}}%
\pgfpathlineto{\pgfqpoint{1.740919in}{0.728173in}}%
\pgfpathlineto{\pgfqpoint{1.741216in}{0.728172in}}%
\pgfpathlineto{\pgfqpoint{1.741513in}{0.728171in}}%
\pgfpathlineto{\pgfqpoint{1.741811in}{0.728170in}}%
\pgfpathlineto{\pgfqpoint{1.742108in}{0.728169in}}%
\pgfpathlineto{\pgfqpoint{1.742406in}{0.728168in}}%
\pgfpathlineto{\pgfqpoint{1.742703in}{0.728167in}}%
\pgfpathlineto{\pgfqpoint{1.743001in}{0.728165in}}%
\pgfpathlineto{\pgfqpoint{1.743298in}{0.728164in}}%
\pgfpathlineto{\pgfqpoint{1.743596in}{0.728163in}}%
\pgfpathlineto{\pgfqpoint{1.743893in}{0.728162in}}%
\pgfpathlineto{\pgfqpoint{1.744191in}{0.728161in}}%
\pgfpathlineto{\pgfqpoint{1.744488in}{0.728160in}}%
\pgfpathlineto{\pgfqpoint{1.744786in}{0.728159in}}%
\pgfpathlineto{\pgfqpoint{1.745083in}{0.728158in}}%
\pgfpathlineto{\pgfqpoint{1.745381in}{0.728157in}}%
\pgfpathlineto{\pgfqpoint{1.745678in}{0.728156in}}%
\pgfpathlineto{\pgfqpoint{1.745976in}{0.728155in}}%
\pgfpathlineto{\pgfqpoint{1.746273in}{0.728154in}}%
\pgfpathlineto{\pgfqpoint{1.746571in}{0.728153in}}%
\pgfpathlineto{\pgfqpoint{1.746868in}{0.728152in}}%
\pgfpathlineto{\pgfqpoint{1.747166in}{0.728151in}}%
\pgfpathlineto{\pgfqpoint{1.747463in}{0.728150in}}%
\pgfpathlineto{\pgfqpoint{1.747761in}{0.728149in}}%
\pgfpathlineto{\pgfqpoint{1.748058in}{0.728148in}}%
\pgfpathlineto{\pgfqpoint{1.748355in}{0.728146in}}%
\pgfpathlineto{\pgfqpoint{1.748653in}{0.728145in}}%
\pgfpathlineto{\pgfqpoint{1.748950in}{0.728144in}}%
\pgfpathlineto{\pgfqpoint{1.749248in}{0.728143in}}%
\pgfpathlineto{\pgfqpoint{1.749545in}{0.728142in}}%
\pgfpathlineto{\pgfqpoint{1.749843in}{0.728141in}}%
\pgfpathlineto{\pgfqpoint{1.750140in}{0.728140in}}%
\pgfpathlineto{\pgfqpoint{1.750438in}{0.728139in}}%
\pgfpathlineto{\pgfqpoint{1.750735in}{0.728138in}}%
\pgfpathlineto{\pgfqpoint{1.751033in}{0.728137in}}%
\pgfpathlineto{\pgfqpoint{1.751330in}{0.728136in}}%
\pgfpathlineto{\pgfqpoint{1.751628in}{0.728135in}}%
\pgfpathlineto{\pgfqpoint{1.751925in}{0.728134in}}%
\pgfpathlineto{\pgfqpoint{1.752223in}{0.728133in}}%
\pgfpathlineto{\pgfqpoint{1.752520in}{0.728132in}}%
\pgfpathlineto{\pgfqpoint{1.752818in}{0.728131in}}%
\pgfpathlineto{\pgfqpoint{1.753115in}{0.728130in}}%
\pgfpathlineto{\pgfqpoint{1.753413in}{0.728128in}}%
\pgfpathlineto{\pgfqpoint{1.753710in}{0.728127in}}%
\pgfpathlineto{\pgfqpoint{1.754008in}{0.728126in}}%
\pgfpathlineto{\pgfqpoint{1.754305in}{0.728125in}}%
\pgfpathlineto{\pgfqpoint{1.754602in}{0.728124in}}%
\pgfpathlineto{\pgfqpoint{1.754900in}{0.728123in}}%
\pgfpathlineto{\pgfqpoint{1.755197in}{0.728122in}}%
\pgfpathlineto{\pgfqpoint{1.755495in}{0.728121in}}%
\pgfpathlineto{\pgfqpoint{1.755792in}{0.728120in}}%
\pgfpathlineto{\pgfqpoint{1.756090in}{0.728119in}}%
\pgfpathlineto{\pgfqpoint{1.756387in}{0.728118in}}%
\pgfpathlineto{\pgfqpoint{1.756685in}{0.728117in}}%
\pgfpathlineto{\pgfqpoint{1.756982in}{0.728116in}}%
\pgfpathlineto{\pgfqpoint{1.757280in}{0.728115in}}%
\pgfpathlineto{\pgfqpoint{1.757577in}{0.728114in}}%
\pgfpathlineto{\pgfqpoint{1.757875in}{0.728113in}}%
\pgfpathlineto{\pgfqpoint{1.758172in}{0.728112in}}%
\pgfpathlineto{\pgfqpoint{1.758470in}{0.728110in}}%
\pgfpathlineto{\pgfqpoint{1.758767in}{0.728109in}}%
\pgfpathlineto{\pgfqpoint{1.759065in}{0.728108in}}%
\pgfpathlineto{\pgfqpoint{1.759362in}{0.728107in}}%
\pgfpathlineto{\pgfqpoint{1.759660in}{0.728106in}}%
\pgfpathlineto{\pgfqpoint{1.759957in}{0.728105in}}%
\pgfpathlineto{\pgfqpoint{1.760255in}{0.728104in}}%
\pgfpathlineto{\pgfqpoint{1.760552in}{0.728103in}}%
\pgfpathlineto{\pgfqpoint{1.760850in}{0.728102in}}%
\pgfpathlineto{\pgfqpoint{1.761147in}{0.728101in}}%
\pgfpathlineto{\pgfqpoint{1.761444in}{0.728100in}}%
\pgfpathlineto{\pgfqpoint{1.761742in}{0.728099in}}%
\pgfpathlineto{\pgfqpoint{1.762039in}{0.728098in}}%
\pgfpathlineto{\pgfqpoint{1.762337in}{0.728097in}}%
\pgfpathlineto{\pgfqpoint{1.762634in}{0.728096in}}%
\pgfpathlineto{\pgfqpoint{1.762932in}{0.728095in}}%
\pgfpathlineto{\pgfqpoint{1.763229in}{0.728094in}}%
\pgfpathlineto{\pgfqpoint{1.763527in}{0.728093in}}%
\pgfpathlineto{\pgfqpoint{1.763824in}{0.728091in}}%
\pgfpathlineto{\pgfqpoint{1.764122in}{0.728090in}}%
\pgfpathlineto{\pgfqpoint{1.764419in}{0.728089in}}%
\pgfpathlineto{\pgfqpoint{1.764717in}{0.728088in}}%
\pgfpathlineto{\pgfqpoint{1.765014in}{0.728087in}}%
\pgfpathlineto{\pgfqpoint{1.765312in}{0.728086in}}%
\pgfpathlineto{\pgfqpoint{1.765609in}{0.728085in}}%
\pgfpathlineto{\pgfqpoint{1.765907in}{0.728084in}}%
\pgfpathlineto{\pgfqpoint{1.766204in}{0.728083in}}%
\pgfpathlineto{\pgfqpoint{1.766502in}{0.728082in}}%
\pgfpathlineto{\pgfqpoint{1.766799in}{0.728081in}}%
\pgfpathlineto{\pgfqpoint{1.767097in}{0.728080in}}%
\pgfpathlineto{\pgfqpoint{1.767394in}{0.728079in}}%
\pgfpathlineto{\pgfqpoint{1.767692in}{0.728078in}}%
\pgfpathlineto{\pgfqpoint{1.767989in}{0.728077in}}%
\pgfpathlineto{\pgfqpoint{1.768286in}{0.728076in}}%
\pgfpathlineto{\pgfqpoint{1.768584in}{0.728075in}}%
\pgfpathlineto{\pgfqpoint{1.768881in}{0.728073in}}%
\pgfpathlineto{\pgfqpoint{1.769179in}{0.728072in}}%
\pgfpathlineto{\pgfqpoint{1.769476in}{0.728071in}}%
\pgfpathlineto{\pgfqpoint{1.769774in}{0.728070in}}%
\pgfpathlineto{\pgfqpoint{1.770071in}{0.728069in}}%
\pgfpathlineto{\pgfqpoint{1.770369in}{0.728068in}}%
\pgfpathlineto{\pgfqpoint{1.770666in}{0.728067in}}%
\pgfpathlineto{\pgfqpoint{1.770964in}{0.728066in}}%
\pgfpathlineto{\pgfqpoint{1.771261in}{0.728065in}}%
\pgfpathlineto{\pgfqpoint{1.771559in}{0.728064in}}%
\pgfpathlineto{\pgfqpoint{1.771856in}{0.728063in}}%
\pgfpathlineto{\pgfqpoint{1.772154in}{0.728062in}}%
\pgfpathlineto{\pgfqpoint{1.772451in}{0.728061in}}%
\pgfpathlineto{\pgfqpoint{1.772749in}{0.728060in}}%
\pgfpathlineto{\pgfqpoint{1.773046in}{0.728059in}}%
\pgfpathlineto{\pgfqpoint{1.773344in}{0.728058in}}%
\pgfpathlineto{\pgfqpoint{1.773641in}{0.728057in}}%
\pgfpathlineto{\pgfqpoint{1.773939in}{0.728056in}}%
\pgfpathlineto{\pgfqpoint{1.774236in}{0.728054in}}%
\pgfpathlineto{\pgfqpoint{1.774533in}{0.728053in}}%
\pgfpathlineto{\pgfqpoint{1.774831in}{0.728052in}}%
\pgfpathlineto{\pgfqpoint{1.775128in}{0.728051in}}%
\pgfpathlineto{\pgfqpoint{1.775426in}{0.728050in}}%
\pgfpathlineto{\pgfqpoint{1.775723in}{0.728049in}}%
\pgfpathlineto{\pgfqpoint{1.776021in}{0.728048in}}%
\pgfpathlineto{\pgfqpoint{1.776318in}{0.728047in}}%
\pgfpathlineto{\pgfqpoint{1.776616in}{0.728046in}}%
\pgfpathlineto{\pgfqpoint{1.776913in}{0.728045in}}%
\pgfpathlineto{\pgfqpoint{1.777211in}{0.728044in}}%
\pgfpathlineto{\pgfqpoint{1.777508in}{0.728043in}}%
\pgfpathlineto{\pgfqpoint{1.777806in}{0.728042in}}%
\pgfpathlineto{\pgfqpoint{1.778103in}{0.728041in}}%
\pgfpathlineto{\pgfqpoint{1.778401in}{0.728040in}}%
\pgfpathlineto{\pgfqpoint{1.778698in}{0.728039in}}%
\pgfpathlineto{\pgfqpoint{1.778996in}{0.728038in}}%
\pgfpathlineto{\pgfqpoint{1.779293in}{0.728036in}}%
\pgfpathlineto{\pgfqpoint{1.779591in}{0.728035in}}%
\pgfpathlineto{\pgfqpoint{1.779888in}{0.728034in}}%
\pgfpathlineto{\pgfqpoint{1.780186in}{0.728033in}}%
\pgfpathlineto{\pgfqpoint{1.780483in}{0.728032in}}%
\pgfpathlineto{\pgfqpoint{1.780781in}{0.728031in}}%
\pgfpathlineto{\pgfqpoint{1.781078in}{0.728030in}}%
\pgfpathlineto{\pgfqpoint{1.781375in}{0.728029in}}%
\pgfpathlineto{\pgfqpoint{1.781673in}{0.728028in}}%
\pgfpathlineto{\pgfqpoint{1.781970in}{0.728027in}}%
\pgfpathlineto{\pgfqpoint{1.782268in}{0.728026in}}%
\pgfpathlineto{\pgfqpoint{1.782565in}{0.728025in}}%
\pgfpathlineto{\pgfqpoint{1.782863in}{0.728024in}}%
\pgfpathlineto{\pgfqpoint{1.783160in}{0.728023in}}%
\pgfpathlineto{\pgfqpoint{1.783458in}{0.728022in}}%
\pgfpathlineto{\pgfqpoint{1.783755in}{0.728021in}}%
\pgfpathlineto{\pgfqpoint{1.784053in}{0.728020in}}%
\pgfpathlineto{\pgfqpoint{1.784350in}{0.728019in}}%
\pgfpathlineto{\pgfqpoint{1.784648in}{0.728017in}}%
\pgfpathlineto{\pgfqpoint{1.784945in}{0.728016in}}%
\pgfpathlineto{\pgfqpoint{1.785243in}{0.728015in}}%
\pgfpathlineto{\pgfqpoint{1.785540in}{0.728014in}}%
\pgfpathlineto{\pgfqpoint{1.785838in}{0.728013in}}%
\pgfpathlineto{\pgfqpoint{1.786135in}{0.728012in}}%
\pgfpathlineto{\pgfqpoint{1.786433in}{0.728011in}}%
\pgfpathlineto{\pgfqpoint{1.786730in}{0.728010in}}%
\pgfpathlineto{\pgfqpoint{1.787028in}{0.728009in}}%
\pgfpathlineto{\pgfqpoint{1.787325in}{0.728008in}}%
\pgfpathlineto{\pgfqpoint{1.787623in}{0.728007in}}%
\pgfpathlineto{\pgfqpoint{1.787920in}{0.728006in}}%
\pgfpathlineto{\pgfqpoint{1.788217in}{0.728005in}}%
\pgfpathlineto{\pgfqpoint{1.788515in}{0.728004in}}%
\pgfpathlineto{\pgfqpoint{1.788812in}{0.728003in}}%
\pgfpathlineto{\pgfqpoint{1.789110in}{0.728002in}}%
\pgfpathlineto{\pgfqpoint{1.789407in}{0.728001in}}%
\pgfpathlineto{\pgfqpoint{1.789705in}{0.727999in}}%
\pgfpathlineto{\pgfqpoint{1.790002in}{0.727998in}}%
\pgfpathlineto{\pgfqpoint{1.790300in}{0.727997in}}%
\pgfpathlineto{\pgfqpoint{1.790597in}{0.727996in}}%
\pgfpathlineto{\pgfqpoint{1.790895in}{0.727995in}}%
\pgfpathlineto{\pgfqpoint{1.791192in}{0.727994in}}%
\pgfpathlineto{\pgfqpoint{1.791490in}{0.727993in}}%
\pgfpathlineto{\pgfqpoint{1.791787in}{0.727992in}}%
\pgfpathlineto{\pgfqpoint{1.792085in}{0.727991in}}%
\pgfpathlineto{\pgfqpoint{1.792382in}{0.727990in}}%
\pgfpathlineto{\pgfqpoint{1.792680in}{0.727989in}}%
\pgfpathlineto{\pgfqpoint{1.792977in}{0.727988in}}%
\pgfpathlineto{\pgfqpoint{1.793275in}{0.727987in}}%
\pgfpathlineto{\pgfqpoint{1.793572in}{0.727986in}}%
\pgfpathlineto{\pgfqpoint{1.793870in}{0.727985in}}%
\pgfpathlineto{\pgfqpoint{1.794167in}{0.727984in}}%
\pgfpathlineto{\pgfqpoint{1.794464in}{0.727983in}}%
\pgfpathlineto{\pgfqpoint{1.794762in}{0.727982in}}%
\pgfpathlineto{\pgfqpoint{1.795059in}{0.727980in}}%
\pgfpathlineto{\pgfqpoint{1.795357in}{0.727979in}}%
\pgfpathlineto{\pgfqpoint{1.795654in}{0.727978in}}%
\pgfpathlineto{\pgfqpoint{1.795952in}{0.727977in}}%
\pgfpathlineto{\pgfqpoint{1.796249in}{0.727976in}}%
\pgfpathlineto{\pgfqpoint{1.796547in}{0.727975in}}%
\pgfpathlineto{\pgfqpoint{1.796844in}{0.727974in}}%
\pgfpathlineto{\pgfqpoint{1.797142in}{0.727973in}}%
\pgfpathlineto{\pgfqpoint{1.797439in}{0.727972in}}%
\pgfpathlineto{\pgfqpoint{1.797737in}{0.727971in}}%
\pgfpathlineto{\pgfqpoint{1.798034in}{0.727970in}}%
\pgfpathlineto{\pgfqpoint{1.798332in}{0.727969in}}%
\pgfpathlineto{\pgfqpoint{1.798629in}{0.727968in}}%
\pgfpathlineto{\pgfqpoint{1.798927in}{0.727967in}}%
\pgfpathlineto{\pgfqpoint{1.799224in}{0.727966in}}%
\pgfpathlineto{\pgfqpoint{1.799522in}{0.727965in}}%
\pgfpathlineto{\pgfqpoint{1.799819in}{0.727964in}}%
\pgfpathlineto{\pgfqpoint{1.800117in}{0.727962in}}%
\pgfpathlineto{\pgfqpoint{1.800414in}{0.727961in}}%
\pgfpathlineto{\pgfqpoint{1.800712in}{0.727960in}}%
\pgfpathlineto{\pgfqpoint{1.801009in}{0.727959in}}%
\pgfpathlineto{\pgfqpoint{1.801306in}{0.727958in}}%
\pgfpathlineto{\pgfqpoint{1.801604in}{0.727957in}}%
\pgfpathlineto{\pgfqpoint{1.801901in}{0.727956in}}%
\pgfpathlineto{\pgfqpoint{1.802199in}{0.727955in}}%
\pgfpathlineto{\pgfqpoint{1.802496in}{0.727954in}}%
\pgfpathlineto{\pgfqpoint{1.802794in}{0.727953in}}%
\pgfpathlineto{\pgfqpoint{1.803091in}{0.727952in}}%
\pgfpathlineto{\pgfqpoint{1.803389in}{0.727951in}}%
\pgfpathlineto{\pgfqpoint{1.803686in}{0.727950in}}%
\pgfpathlineto{\pgfqpoint{1.803984in}{0.727949in}}%
\pgfpathlineto{\pgfqpoint{1.804281in}{0.727948in}}%
\pgfpathlineto{\pgfqpoint{1.804579in}{0.727947in}}%
\pgfpathlineto{\pgfqpoint{1.804876in}{0.727946in}}%
\pgfpathlineto{\pgfqpoint{1.805174in}{0.727945in}}%
\pgfpathlineto{\pgfqpoint{1.805471in}{0.727943in}}%
\pgfpathlineto{\pgfqpoint{1.805769in}{0.727942in}}%
\pgfpathlineto{\pgfqpoint{1.806066in}{0.727941in}}%
\pgfpathlineto{\pgfqpoint{1.806364in}{0.727940in}}%
\pgfpathlineto{\pgfqpoint{1.806661in}{0.727939in}}%
\pgfpathlineto{\pgfqpoint{1.806959in}{0.727938in}}%
\pgfpathlineto{\pgfqpoint{1.807256in}{0.727937in}}%
\pgfpathlineto{\pgfqpoint{1.807554in}{0.727936in}}%
\pgfpathlineto{\pgfqpoint{1.807851in}{0.727935in}}%
\pgfpathlineto{\pgfqpoint{1.808148in}{0.727934in}}%
\pgfpathlineto{\pgfqpoint{1.808446in}{0.727933in}}%
\pgfpathlineto{\pgfqpoint{1.808743in}{0.727932in}}%
\pgfpathlineto{\pgfqpoint{1.809041in}{0.727931in}}%
\pgfpathlineto{\pgfqpoint{1.809338in}{0.727930in}}%
\pgfpathlineto{\pgfqpoint{1.809636in}{0.727929in}}%
\pgfpathlineto{\pgfqpoint{1.809933in}{0.727928in}}%
\pgfpathlineto{\pgfqpoint{1.810231in}{0.727927in}}%
\pgfpathlineto{\pgfqpoint{1.810528in}{0.727925in}}%
\pgfpathlineto{\pgfqpoint{1.810826in}{0.727924in}}%
\pgfpathlineto{\pgfqpoint{1.811123in}{0.727923in}}%
\pgfpathlineto{\pgfqpoint{1.811421in}{0.727922in}}%
\pgfpathlineto{\pgfqpoint{1.811718in}{0.727921in}}%
\pgfpathlineto{\pgfqpoint{1.812016in}{0.727920in}}%
\pgfpathlineto{\pgfqpoint{1.812313in}{0.727919in}}%
\pgfpathlineto{\pgfqpoint{1.812611in}{0.727918in}}%
\pgfpathlineto{\pgfqpoint{1.812908in}{0.727917in}}%
\pgfpathlineto{\pgfqpoint{1.813206in}{0.727916in}}%
\pgfpathlineto{\pgfqpoint{1.813503in}{0.727915in}}%
\pgfpathlineto{\pgfqpoint{1.813801in}{0.727914in}}%
\pgfpathlineto{\pgfqpoint{1.814098in}{0.727913in}}%
\pgfpathlineto{\pgfqpoint{1.814396in}{0.727912in}}%
\pgfpathlineto{\pgfqpoint{1.814693in}{0.727911in}}%
\pgfpathlineto{\pgfqpoint{1.814990in}{0.727910in}}%
\pgfpathlineto{\pgfqpoint{1.815288in}{0.727909in}}%
\pgfpathlineto{\pgfqpoint{1.815585in}{0.727908in}}%
\pgfpathlineto{\pgfqpoint{1.815883in}{0.727906in}}%
\pgfpathlineto{\pgfqpoint{1.816180in}{0.727905in}}%
\pgfpathlineto{\pgfqpoint{1.816478in}{0.727904in}}%
\pgfpathlineto{\pgfqpoint{1.816775in}{0.727903in}}%
\pgfpathlineto{\pgfqpoint{1.817073in}{0.727902in}}%
\pgfpathlineto{\pgfqpoint{1.817370in}{0.727901in}}%
\pgfpathlineto{\pgfqpoint{1.817668in}{0.727900in}}%
\pgfpathlineto{\pgfqpoint{1.817965in}{0.727899in}}%
\pgfpathlineto{\pgfqpoint{1.818263in}{0.727898in}}%
\pgfpathlineto{\pgfqpoint{1.818560in}{0.727897in}}%
\pgfpathlineto{\pgfqpoint{1.818858in}{0.727896in}}%
\pgfpathlineto{\pgfqpoint{1.819155in}{0.727895in}}%
\pgfpathlineto{\pgfqpoint{1.819453in}{0.727894in}}%
\pgfpathlineto{\pgfqpoint{1.819750in}{0.727893in}}%
\pgfpathlineto{\pgfqpoint{1.820048in}{0.727892in}}%
\pgfpathlineto{\pgfqpoint{1.820345in}{0.727891in}}%
\pgfpathlineto{\pgfqpoint{1.820643in}{0.727890in}}%
\pgfpathlineto{\pgfqpoint{1.820940in}{0.727888in}}%
\pgfpathlineto{\pgfqpoint{1.821237in}{0.727887in}}%
\pgfpathlineto{\pgfqpoint{1.821535in}{0.727886in}}%
\pgfpathlineto{\pgfqpoint{1.821832in}{0.727885in}}%
\pgfpathlineto{\pgfqpoint{1.822130in}{0.727884in}}%
\pgfpathlineto{\pgfqpoint{1.822427in}{0.727883in}}%
\pgfpathlineto{\pgfqpoint{1.822725in}{0.727882in}}%
\pgfpathlineto{\pgfqpoint{1.823022in}{0.727881in}}%
\pgfpathlineto{\pgfqpoint{1.823320in}{0.727880in}}%
\pgfpathlineto{\pgfqpoint{1.823617in}{0.727879in}}%
\pgfpathlineto{\pgfqpoint{1.823915in}{0.727878in}}%
\pgfpathlineto{\pgfqpoint{1.824212in}{0.727877in}}%
\pgfpathlineto{\pgfqpoint{1.824510in}{0.727876in}}%
\pgfpathlineto{\pgfqpoint{1.824807in}{0.727875in}}%
\pgfpathlineto{\pgfqpoint{1.825105in}{0.727874in}}%
\pgfpathlineto{\pgfqpoint{1.825402in}{0.727873in}}%
\pgfpathlineto{\pgfqpoint{1.825700in}{0.727872in}}%
\pgfpathlineto{\pgfqpoint{1.825997in}{0.727871in}}%
\pgfpathlineto{\pgfqpoint{1.826295in}{0.727869in}}%
\pgfpathlineto{\pgfqpoint{1.826592in}{0.727868in}}%
\pgfpathlineto{\pgfqpoint{1.826890in}{0.727867in}}%
\pgfpathlineto{\pgfqpoint{1.827187in}{0.727866in}}%
\pgfpathlineto{\pgfqpoint{1.827485in}{0.727865in}}%
\pgfpathlineto{\pgfqpoint{1.827782in}{0.727864in}}%
\pgfpathlineto{\pgfqpoint{1.828079in}{0.727863in}}%
\pgfpathlineto{\pgfqpoint{1.828377in}{0.727862in}}%
\pgfpathlineto{\pgfqpoint{1.828674in}{0.727861in}}%
\pgfpathlineto{\pgfqpoint{1.828972in}{0.727860in}}%
\pgfpathlineto{\pgfqpoint{1.829269in}{0.727859in}}%
\pgfpathlineto{\pgfqpoint{1.829567in}{0.727858in}}%
\pgfpathlineto{\pgfqpoint{1.829864in}{0.727857in}}%
\pgfpathlineto{\pgfqpoint{1.830162in}{0.727856in}}%
\pgfpathlineto{\pgfqpoint{1.830459in}{0.727855in}}%
\pgfpathlineto{\pgfqpoint{1.830757in}{0.727854in}}%
\pgfpathlineto{\pgfqpoint{1.831054in}{0.727853in}}%
\pgfpathlineto{\pgfqpoint{1.831352in}{0.727851in}}%
\pgfpathlineto{\pgfqpoint{1.831649in}{0.727850in}}%
\pgfpathlineto{\pgfqpoint{1.831947in}{0.727849in}}%
\pgfpathlineto{\pgfqpoint{1.832244in}{0.727848in}}%
\pgfpathlineto{\pgfqpoint{1.832542in}{0.727847in}}%
\pgfpathlineto{\pgfqpoint{1.832839in}{0.727846in}}%
\pgfpathlineto{\pgfqpoint{1.833137in}{0.727845in}}%
\pgfpathlineto{\pgfqpoint{1.833434in}{0.727844in}}%
\pgfpathlineto{\pgfqpoint{1.833732in}{0.727843in}}%
\pgfpathlineto{\pgfqpoint{1.834029in}{0.727842in}}%
\pgfpathlineto{\pgfqpoint{1.834327in}{0.727841in}}%
\pgfpathlineto{\pgfqpoint{1.834624in}{0.727840in}}%
\pgfpathlineto{\pgfqpoint{1.834921in}{0.727839in}}%
\pgfpathlineto{\pgfqpoint{1.835219in}{0.727838in}}%
\pgfpathlineto{\pgfqpoint{1.835516in}{0.727837in}}%
\pgfpathlineto{\pgfqpoint{1.835814in}{0.727836in}}%
\pgfpathlineto{\pgfqpoint{1.836111in}{0.727835in}}%
\pgfpathlineto{\pgfqpoint{1.836409in}{0.727834in}}%
\pgfpathlineto{\pgfqpoint{1.836706in}{0.727832in}}%
\pgfpathlineto{\pgfqpoint{1.837004in}{0.727831in}}%
\pgfpathlineto{\pgfqpoint{1.837301in}{0.727830in}}%
\pgfpathlineto{\pgfqpoint{1.837599in}{0.727829in}}%
\pgfpathlineto{\pgfqpoint{1.837896in}{0.727828in}}%
\pgfpathlineto{\pgfqpoint{1.838194in}{0.727826in}}%
\pgfpathlineto{\pgfqpoint{1.838491in}{0.727825in}}%
\pgfpathlineto{\pgfqpoint{1.838789in}{0.727823in}}%
\pgfpathlineto{\pgfqpoint{1.839086in}{0.727821in}}%
\pgfpathlineto{\pgfqpoint{1.839384in}{0.727820in}}%
\pgfpathlineto{\pgfqpoint{1.839681in}{0.727818in}}%
\pgfpathlineto{\pgfqpoint{1.839979in}{0.727817in}}%
\pgfpathlineto{\pgfqpoint{1.840276in}{0.727815in}}%
\pgfpathlineto{\pgfqpoint{1.840574in}{0.727814in}}%
\pgfpathlineto{\pgfqpoint{1.840871in}{0.727812in}}%
\pgfpathlineto{\pgfqpoint{1.841168in}{0.727810in}}%
\pgfpathlineto{\pgfqpoint{1.841466in}{0.727809in}}%
\pgfpathlineto{\pgfqpoint{1.841763in}{0.727807in}}%
\pgfpathlineto{\pgfqpoint{1.842061in}{0.727806in}}%
\pgfpathlineto{\pgfqpoint{1.842358in}{0.727804in}}%
\pgfpathlineto{\pgfqpoint{1.842656in}{0.727803in}}%
\pgfpathlineto{\pgfqpoint{1.842953in}{0.727801in}}%
\pgfpathlineto{\pgfqpoint{1.843251in}{0.727799in}}%
\pgfpathlineto{\pgfqpoint{1.843548in}{0.727782in}}%
\pgfpathlineto{\pgfqpoint{1.843846in}{0.727778in}}%
\pgfpathlineto{\pgfqpoint{1.844143in}{0.727778in}}%
\pgfpathlineto{\pgfqpoint{1.844441in}{0.727778in}}%
\pgfpathlineto{\pgfqpoint{1.844738in}{0.727778in}}%
\pgfpathlineto{\pgfqpoint{1.845036in}{0.727777in}}%
\pgfpathlineto{\pgfqpoint{1.845333in}{0.727777in}}%
\pgfpathlineto{\pgfqpoint{1.845631in}{0.727777in}}%
\pgfpathlineto{\pgfqpoint{1.845928in}{0.727776in}}%
\pgfpathlineto{\pgfqpoint{1.846226in}{0.727776in}}%
\pgfpathlineto{\pgfqpoint{1.846523in}{0.727776in}}%
\pgfpathlineto{\pgfqpoint{1.846821in}{0.727776in}}%
\pgfpathlineto{\pgfqpoint{1.847118in}{0.727775in}}%
\pgfpathlineto{\pgfqpoint{1.847416in}{0.727775in}}%
\pgfpathlineto{\pgfqpoint{1.847713in}{0.727775in}}%
\pgfpathlineto{\pgfqpoint{1.848010in}{0.727775in}}%
\pgfpathlineto{\pgfqpoint{1.848308in}{0.727774in}}%
\pgfpathlineto{\pgfqpoint{1.848605in}{0.727774in}}%
\pgfpathlineto{\pgfqpoint{1.848903in}{0.727773in}}%
\pgfpathlineto{\pgfqpoint{1.849200in}{0.727758in}}%
\pgfpathlineto{\pgfqpoint{1.849498in}{0.727757in}}%
\pgfpathlineto{\pgfqpoint{1.849795in}{0.727756in}}%
\pgfpathlineto{\pgfqpoint{1.850093in}{0.727755in}}%
\pgfpathlineto{\pgfqpoint{1.850390in}{0.727754in}}%
\pgfpathlineto{\pgfqpoint{1.850688in}{0.727753in}}%
\pgfpathlineto{\pgfqpoint{1.850985in}{0.727752in}}%
\pgfpathlineto{\pgfqpoint{1.851283in}{0.727751in}}%
\pgfpathlineto{\pgfqpoint{1.851580in}{0.727750in}}%
\pgfpathlineto{\pgfqpoint{1.851878in}{0.727749in}}%
\pgfpathlineto{\pgfqpoint{1.852175in}{0.727748in}}%
\pgfpathlineto{\pgfqpoint{1.852473in}{0.727747in}}%
\pgfpathlineto{\pgfqpoint{1.852770in}{0.727746in}}%
\pgfpathlineto{\pgfqpoint{1.853068in}{0.727745in}}%
\pgfpathlineto{\pgfqpoint{1.853365in}{0.727743in}}%
\pgfpathlineto{\pgfqpoint{1.853663in}{0.727742in}}%
\pgfpathlineto{\pgfqpoint{1.853960in}{0.727741in}}%
\pgfpathlineto{\pgfqpoint{1.854258in}{0.727740in}}%
\pgfpathlineto{\pgfqpoint{1.854555in}{0.727739in}}%
\pgfpathlineto{\pgfqpoint{1.854852in}{0.727738in}}%
\pgfpathlineto{\pgfqpoint{1.855150in}{0.727737in}}%
\pgfpathlineto{\pgfqpoint{1.855447in}{0.727736in}}%
\pgfpathlineto{\pgfqpoint{1.855745in}{0.727735in}}%
\pgfpathlineto{\pgfqpoint{1.856042in}{0.727734in}}%
\pgfpathlineto{\pgfqpoint{1.856340in}{0.727733in}}%
\pgfpathlineto{\pgfqpoint{1.856637in}{0.727732in}}%
\pgfpathlineto{\pgfqpoint{1.856935in}{0.727731in}}%
\pgfpathlineto{\pgfqpoint{1.857232in}{0.727730in}}%
\pgfpathlineto{\pgfqpoint{1.857530in}{0.727729in}}%
\pgfpathlineto{\pgfqpoint{1.857827in}{0.727728in}}%
\pgfpathlineto{\pgfqpoint{1.858125in}{0.727727in}}%
\pgfpathlineto{\pgfqpoint{1.858422in}{0.727725in}}%
\pgfpathlineto{\pgfqpoint{1.858720in}{0.727724in}}%
\pgfpathlineto{\pgfqpoint{1.859017in}{0.727723in}}%
\pgfpathlineto{\pgfqpoint{1.859315in}{0.727722in}}%
\pgfpathlineto{\pgfqpoint{1.859612in}{0.727721in}}%
\pgfpathlineto{\pgfqpoint{1.859910in}{0.727720in}}%
\pgfpathlineto{\pgfqpoint{1.860207in}{0.727719in}}%
\pgfpathlineto{\pgfqpoint{1.860505in}{0.727718in}}%
\pgfpathlineto{\pgfqpoint{1.860802in}{0.727717in}}%
\pgfpathlineto{\pgfqpoint{1.861099in}{0.727716in}}%
\pgfpathlineto{\pgfqpoint{1.861397in}{0.727715in}}%
\pgfpathlineto{\pgfqpoint{1.861694in}{0.727714in}}%
\pgfpathlineto{\pgfqpoint{1.861992in}{0.727713in}}%
\pgfpathlineto{\pgfqpoint{1.862289in}{0.727712in}}%
\pgfpathlineto{\pgfqpoint{1.862587in}{0.727711in}}%
\pgfpathlineto{\pgfqpoint{1.862884in}{0.727710in}}%
\pgfpathlineto{\pgfqpoint{1.863182in}{0.727709in}}%
\pgfpathlineto{\pgfqpoint{1.863479in}{0.727708in}}%
\pgfpathlineto{\pgfqpoint{1.863777in}{0.727698in}}%
\pgfpathlineto{\pgfqpoint{1.864074in}{0.727612in}}%
\pgfpathlineto{\pgfqpoint{1.864372in}{0.727506in}}%
\pgfpathlineto{\pgfqpoint{1.864669in}{0.727400in}}%
\pgfpathlineto{\pgfqpoint{1.864967in}{0.727295in}}%
\pgfpathlineto{\pgfqpoint{1.865264in}{0.727189in}}%
\pgfpathlineto{\pgfqpoint{1.865562in}{0.727083in}}%
\pgfpathlineto{\pgfqpoint{1.865859in}{0.726977in}}%
\pgfpathlineto{\pgfqpoint{1.866157in}{0.726872in}}%
\pgfpathlineto{\pgfqpoint{1.866454in}{0.726766in}}%
\pgfpathlineto{\pgfqpoint{1.866752in}{0.726660in}}%
\pgfpathlineto{\pgfqpoint{1.867049in}{0.726554in}}%
\pgfpathlineto{\pgfqpoint{1.867347in}{0.726449in}}%
\pgfpathlineto{\pgfqpoint{1.867644in}{0.726343in}}%
\pgfpathlineto{\pgfqpoint{1.867941in}{0.726237in}}%
\pgfpathlineto{\pgfqpoint{1.868239in}{0.726131in}}%
\pgfpathlineto{\pgfqpoint{1.868536in}{0.726026in}}%
\pgfpathlineto{\pgfqpoint{1.868834in}{0.725920in}}%
\pgfpathlineto{\pgfqpoint{1.869131in}{0.725814in}}%
\pgfpathlineto{\pgfqpoint{1.869429in}{0.725708in}}%
\pgfpathlineto{\pgfqpoint{1.869726in}{0.725602in}}%
\pgfpathlineto{\pgfqpoint{1.870024in}{0.725497in}}%
\pgfpathlineto{\pgfqpoint{1.870321in}{0.725391in}}%
\pgfpathlineto{\pgfqpoint{1.870619in}{0.725285in}}%
\pgfpathlineto{\pgfqpoint{1.870916in}{0.725179in}}%
\pgfpathlineto{\pgfqpoint{1.871214in}{0.725074in}}%
\pgfpathlineto{\pgfqpoint{1.871511in}{0.724968in}}%
\pgfpathlineto{\pgfqpoint{1.871809in}{0.724862in}}%
\pgfpathlineto{\pgfqpoint{1.872106in}{0.724756in}}%
\pgfpathlineto{\pgfqpoint{1.872404in}{0.724651in}}%
\pgfpathlineto{\pgfqpoint{1.872701in}{0.724545in}}%
\pgfpathlineto{\pgfqpoint{1.872999in}{0.724439in}}%
\pgfpathlineto{\pgfqpoint{1.873296in}{0.724333in}}%
\pgfpathlineto{\pgfqpoint{1.873594in}{0.724227in}}%
\pgfpathlineto{\pgfqpoint{1.873891in}{0.724122in}}%
\pgfpathlineto{\pgfqpoint{1.874189in}{0.724016in}}%
\pgfpathlineto{\pgfqpoint{1.874486in}{0.723910in}}%
\pgfpathlineto{\pgfqpoint{1.874783in}{0.723804in}}%
\pgfpathlineto{\pgfqpoint{1.875081in}{0.723699in}}%
\pgfpathlineto{\pgfqpoint{1.875378in}{0.723593in}}%
\pgfpathlineto{\pgfqpoint{1.875676in}{0.723487in}}%
\pgfpathlineto{\pgfqpoint{1.875973in}{0.723381in}}%
\pgfpathlineto{\pgfqpoint{1.876271in}{0.723276in}}%
\pgfpathlineto{\pgfqpoint{1.876568in}{0.723170in}}%
\pgfpathlineto{\pgfqpoint{1.876866in}{0.723064in}}%
\pgfpathlineto{\pgfqpoint{1.877163in}{0.722958in}}%
\pgfpathlineto{\pgfqpoint{1.877461in}{0.722853in}}%
\pgfpathlineto{\pgfqpoint{1.877758in}{0.722747in}}%
\pgfpathlineto{\pgfqpoint{1.878056in}{0.722641in}}%
\pgfpathlineto{\pgfqpoint{1.878353in}{0.722535in}}%
\pgfpathlineto{\pgfqpoint{1.878651in}{0.722429in}}%
\pgfpathlineto{\pgfqpoint{1.878948in}{0.722324in}}%
\pgfpathlineto{\pgfqpoint{1.879246in}{0.722218in}}%
\pgfpathlineto{\pgfqpoint{1.879543in}{0.722112in}}%
\pgfpathlineto{\pgfqpoint{1.879841in}{0.722006in}}%
\pgfpathlineto{\pgfqpoint{1.880138in}{0.721901in}}%
\pgfpathlineto{\pgfqpoint{1.880436in}{0.721795in}}%
\pgfpathlineto{\pgfqpoint{1.880733in}{0.721689in}}%
\pgfpathlineto{\pgfqpoint{1.881030in}{0.721583in}}%
\pgfpathlineto{\pgfqpoint{1.881328in}{0.721478in}}%
\pgfpathlineto{\pgfqpoint{1.881625in}{0.721372in}}%
\pgfpathlineto{\pgfqpoint{1.881923in}{0.721266in}}%
\pgfpathlineto{\pgfqpoint{1.882220in}{0.721160in}}%
\pgfpathlineto{\pgfqpoint{1.882518in}{0.721054in}}%
\pgfpathlineto{\pgfqpoint{1.882815in}{0.720949in}}%
\pgfpathlineto{\pgfqpoint{1.883113in}{0.720843in}}%
\pgfpathlineto{\pgfqpoint{1.883410in}{0.720737in}}%
\pgfpathlineto{\pgfqpoint{1.883708in}{0.720631in}}%
\pgfpathlineto{\pgfqpoint{1.884005in}{0.720526in}}%
\pgfpathlineto{\pgfqpoint{1.884303in}{0.720420in}}%
\pgfpathlineto{\pgfqpoint{1.884600in}{0.720314in}}%
\pgfpathlineto{\pgfqpoint{1.884898in}{0.720208in}}%
\pgfpathlineto{\pgfqpoint{1.885195in}{0.720103in}}%
\pgfpathlineto{\pgfqpoint{1.885493in}{0.719997in}}%
\pgfpathlineto{\pgfqpoint{1.885790in}{0.719891in}}%
\pgfpathlineto{\pgfqpoint{1.886088in}{0.719785in}}%
\pgfpathlineto{\pgfqpoint{1.886385in}{0.720649in}}%
\pgfpathlineto{\pgfqpoint{1.886683in}{0.722937in}}%
\pgfpathlineto{\pgfqpoint{1.886980in}{0.723158in}}%
\pgfpathlineto{\pgfqpoint{1.887278in}{0.723203in}}%
\pgfpathlineto{\pgfqpoint{1.887575in}{0.723182in}}%
\pgfpathlineto{\pgfqpoint{1.887872in}{0.723161in}}%
\pgfpathlineto{\pgfqpoint{1.888170in}{0.723141in}}%
\pgfpathlineto{\pgfqpoint{1.888467in}{0.723120in}}%
\pgfpathlineto{\pgfqpoint{1.888765in}{0.723099in}}%
\pgfpathlineto{\pgfqpoint{1.889062in}{0.723079in}}%
\pgfpathlineto{\pgfqpoint{1.889360in}{0.723058in}}%
\pgfpathlineto{\pgfqpoint{1.889657in}{0.723037in}}%
\pgfpathlineto{\pgfqpoint{1.889955in}{0.723016in}}%
\pgfpathlineto{\pgfqpoint{1.890252in}{0.722996in}}%
\pgfpathlineto{\pgfqpoint{1.890550in}{0.722975in}}%
\pgfpathlineto{\pgfqpoint{1.890847in}{0.722954in}}%
\pgfpathlineto{\pgfqpoint{1.891145in}{0.722934in}}%
\pgfpathlineto{\pgfqpoint{1.891442in}{0.722913in}}%
\pgfpathlineto{\pgfqpoint{1.891740in}{0.722892in}}%
\pgfpathlineto{\pgfqpoint{1.892037in}{0.722871in}}%
\pgfpathlineto{\pgfqpoint{1.892335in}{0.722851in}}%
\pgfpathlineto{\pgfqpoint{1.892632in}{0.722830in}}%
\pgfpathlineto{\pgfqpoint{1.892930in}{0.722809in}}%
\pgfpathlineto{\pgfqpoint{1.893227in}{0.722789in}}%
\pgfpathlineto{\pgfqpoint{1.893525in}{0.722768in}}%
\pgfpathlineto{\pgfqpoint{1.893822in}{0.722747in}}%
\pgfpathlineto{\pgfqpoint{1.894120in}{0.722726in}}%
\pgfpathlineto{\pgfqpoint{1.894417in}{0.722706in}}%
\pgfpathlineto{\pgfqpoint{1.894714in}{0.722685in}}%
\pgfpathlineto{\pgfqpoint{1.895012in}{0.722664in}}%
\pgfpathlineto{\pgfqpoint{1.895309in}{0.722644in}}%
\pgfpathlineto{\pgfqpoint{1.895607in}{0.722623in}}%
\pgfpathlineto{\pgfqpoint{1.895904in}{0.722602in}}%
\pgfpathlineto{\pgfqpoint{1.896202in}{0.722581in}}%
\pgfpathlineto{\pgfqpoint{1.896499in}{0.722561in}}%
\pgfpathlineto{\pgfqpoint{1.896797in}{0.722540in}}%
\pgfpathlineto{\pgfqpoint{1.897094in}{0.722519in}}%
\pgfpathlineto{\pgfqpoint{1.897392in}{0.722499in}}%
\pgfpathlineto{\pgfqpoint{1.897689in}{0.722478in}}%
\pgfpathlineto{\pgfqpoint{1.897987in}{0.722457in}}%
\pgfpathlineto{\pgfqpoint{1.898284in}{0.722436in}}%
\pgfpathlineto{\pgfqpoint{1.898582in}{0.722416in}}%
\pgfpathlineto{\pgfqpoint{1.898879in}{0.722395in}}%
\pgfpathlineto{\pgfqpoint{1.899177in}{0.722374in}}%
\pgfpathlineto{\pgfqpoint{1.899474in}{0.722354in}}%
\pgfpathlineto{\pgfqpoint{1.899772in}{0.722336in}}%
\pgfpathlineto{\pgfqpoint{1.900069in}{0.722919in}}%
\pgfpathlineto{\pgfqpoint{1.900367in}{0.723060in}}%
\pgfpathlineto{\pgfqpoint{1.900664in}{0.723061in}}%
\pgfpathlineto{\pgfqpoint{1.900961in}{0.723063in}}%
\pgfpathlineto{\pgfqpoint{1.901259in}{0.723065in}}%
\pgfpathlineto{\pgfqpoint{1.901556in}{0.723066in}}%
\pgfpathlineto{\pgfqpoint{1.901854in}{0.723068in}}%
\pgfpathlineto{\pgfqpoint{1.902151in}{0.723070in}}%
\pgfpathlineto{\pgfqpoint{1.902449in}{0.723072in}}%
\pgfpathlineto{\pgfqpoint{1.902746in}{0.723073in}}%
\pgfpathlineto{\pgfqpoint{1.903044in}{0.723075in}}%
\pgfpathlineto{\pgfqpoint{1.903341in}{0.723077in}}%
\pgfpathlineto{\pgfqpoint{1.903639in}{0.723078in}}%
\pgfpathlineto{\pgfqpoint{1.903936in}{0.723080in}}%
\pgfpathlineto{\pgfqpoint{1.904234in}{0.723082in}}%
\pgfpathlineto{\pgfqpoint{1.904531in}{0.723084in}}%
\pgfpathlineto{\pgfqpoint{1.904829in}{0.723085in}}%
\pgfpathlineto{\pgfqpoint{1.905126in}{0.723087in}}%
\pgfpathlineto{\pgfqpoint{1.905424in}{0.723089in}}%
\pgfpathlineto{\pgfqpoint{1.905721in}{0.723090in}}%
\pgfpathlineto{\pgfqpoint{1.906019in}{0.723092in}}%
\pgfpathlineto{\pgfqpoint{1.906316in}{0.723094in}}%
\pgfpathlineto{\pgfqpoint{1.906614in}{0.723096in}}%
\pgfpathlineto{\pgfqpoint{1.906911in}{0.723097in}}%
\pgfpathlineto{\pgfqpoint{1.907209in}{0.723099in}}%
\pgfpathlineto{\pgfqpoint{1.907506in}{0.723101in}}%
\pgfpathlineto{\pgfqpoint{1.907803in}{0.723102in}}%
\pgfpathlineto{\pgfqpoint{1.908101in}{0.723104in}}%
\pgfpathlineto{\pgfqpoint{1.908398in}{0.723106in}}%
\pgfpathlineto{\pgfqpoint{1.908696in}{0.723107in}}%
\pgfpathlineto{\pgfqpoint{1.908993in}{0.723109in}}%
\pgfpathlineto{\pgfqpoint{1.909291in}{0.723111in}}%
\pgfpathlineto{\pgfqpoint{1.909588in}{0.723113in}}%
\pgfpathlineto{\pgfqpoint{1.909886in}{0.723114in}}%
\pgfpathlineto{\pgfqpoint{1.910183in}{0.723116in}}%
\pgfpathlineto{\pgfqpoint{1.910481in}{0.723118in}}%
\pgfpathlineto{\pgfqpoint{1.910778in}{0.723119in}}%
\pgfpathlineto{\pgfqpoint{1.911076in}{0.723121in}}%
\pgfpathlineto{\pgfqpoint{1.911373in}{0.723123in}}%
\pgfpathlineto{\pgfqpoint{1.911671in}{0.723125in}}%
\pgfpathlineto{\pgfqpoint{1.911968in}{0.723126in}}%
\pgfpathlineto{\pgfqpoint{1.912266in}{0.723128in}}%
\pgfpathlineto{\pgfqpoint{1.912563in}{0.723130in}}%
\pgfpathlineto{\pgfqpoint{1.912861in}{0.723131in}}%
\pgfpathlineto{\pgfqpoint{1.913158in}{0.723133in}}%
\pgfpathlineto{\pgfqpoint{1.913456in}{0.723135in}}%
\pgfpathlineto{\pgfqpoint{1.913753in}{0.723137in}}%
\pgfpathlineto{\pgfqpoint{1.914051in}{0.723137in}}%
\pgfpathlineto{\pgfqpoint{1.914348in}{0.723136in}}%
\pgfpathlineto{\pgfqpoint{1.914645in}{0.723135in}}%
\pgfpathlineto{\pgfqpoint{1.914943in}{0.723134in}}%
\pgfpathlineto{\pgfqpoint{1.915240in}{0.723133in}}%
\pgfpathlineto{\pgfqpoint{1.915538in}{0.723132in}}%
\pgfpathlineto{\pgfqpoint{1.915835in}{0.723131in}}%
\pgfpathlineto{\pgfqpoint{1.916133in}{0.723130in}}%
\pgfpathlineto{\pgfqpoint{1.916430in}{0.723129in}}%
\pgfpathlineto{\pgfqpoint{1.916728in}{0.723128in}}%
\pgfpathlineto{\pgfqpoint{1.917025in}{0.723127in}}%
\pgfpathlineto{\pgfqpoint{1.917323in}{0.723126in}}%
\pgfpathlineto{\pgfqpoint{1.917620in}{0.723125in}}%
\pgfpathlineto{\pgfqpoint{1.917918in}{0.723124in}}%
\pgfpathlineto{\pgfqpoint{1.918215in}{0.723123in}}%
\pgfpathlineto{\pgfqpoint{1.918513in}{0.723122in}}%
\pgfpathlineto{\pgfqpoint{1.918810in}{0.723121in}}%
\pgfpathlineto{\pgfqpoint{1.919108in}{0.723120in}}%
\pgfpathlineto{\pgfqpoint{1.919405in}{0.723119in}}%
\pgfpathlineto{\pgfqpoint{1.919703in}{0.723118in}}%
\pgfpathlineto{\pgfqpoint{1.920000in}{0.723118in}}%
\pgfpathlineto{\pgfqpoint{1.920298in}{0.723117in}}%
\pgfpathlineto{\pgfqpoint{1.920595in}{0.723116in}}%
\pgfpathlineto{\pgfqpoint{1.920892in}{0.723115in}}%
\pgfpathlineto{\pgfqpoint{1.921190in}{0.723114in}}%
\pgfpathlineto{\pgfqpoint{1.921487in}{0.723113in}}%
\pgfpathlineto{\pgfqpoint{1.921785in}{0.723112in}}%
\pgfpathlineto{\pgfqpoint{1.922082in}{0.723111in}}%
\pgfpathlineto{\pgfqpoint{1.922380in}{0.723110in}}%
\pgfpathlineto{\pgfqpoint{1.922677in}{0.723109in}}%
\pgfpathlineto{\pgfqpoint{1.922975in}{0.723108in}}%
\pgfpathlineto{\pgfqpoint{1.923272in}{0.723107in}}%
\pgfpathlineto{\pgfqpoint{1.923570in}{0.723106in}}%
\pgfpathlineto{\pgfqpoint{1.923867in}{0.723105in}}%
\pgfpathlineto{\pgfqpoint{1.924165in}{0.723104in}}%
\pgfpathlineto{\pgfqpoint{1.924462in}{0.723103in}}%
\pgfpathlineto{\pgfqpoint{1.924760in}{0.723102in}}%
\pgfpathlineto{\pgfqpoint{1.925057in}{0.723101in}}%
\pgfpathlineto{\pgfqpoint{1.925355in}{0.723100in}}%
\pgfpathlineto{\pgfqpoint{1.925652in}{0.723099in}}%
\pgfpathlineto{\pgfqpoint{1.925950in}{0.723098in}}%
\pgfpathlineto{\pgfqpoint{1.926247in}{0.723097in}}%
\pgfpathlineto{\pgfqpoint{1.926545in}{0.723096in}}%
\pgfpathlineto{\pgfqpoint{1.926842in}{0.723095in}}%
\pgfpathlineto{\pgfqpoint{1.927140in}{0.723094in}}%
\pgfpathlineto{\pgfqpoint{1.927437in}{0.723093in}}%
\pgfpathlineto{\pgfqpoint{1.927734in}{0.723092in}}%
\pgfpathlineto{\pgfqpoint{1.928032in}{0.723091in}}%
\pgfpathlineto{\pgfqpoint{1.928329in}{0.723090in}}%
\pgfpathlineto{\pgfqpoint{1.928627in}{0.723089in}}%
\pgfpathlineto{\pgfqpoint{1.928924in}{0.723088in}}%
\pgfpathlineto{\pgfqpoint{1.929222in}{0.723087in}}%
\pgfpathlineto{\pgfqpoint{1.929519in}{0.723086in}}%
\pgfpathlineto{\pgfqpoint{1.929817in}{0.723085in}}%
\pgfpathlineto{\pgfqpoint{1.930114in}{0.723084in}}%
\pgfpathlineto{\pgfqpoint{1.930412in}{0.723083in}}%
\pgfpathlineto{\pgfqpoint{1.930709in}{0.723082in}}%
\pgfpathlineto{\pgfqpoint{1.931007in}{0.723081in}}%
\pgfpathlineto{\pgfqpoint{1.931304in}{0.723080in}}%
\pgfpathlineto{\pgfqpoint{1.931602in}{0.723079in}}%
\pgfpathlineto{\pgfqpoint{1.931899in}{0.723078in}}%
\pgfpathlineto{\pgfqpoint{1.932197in}{0.723077in}}%
\pgfpathlineto{\pgfqpoint{1.932494in}{0.723076in}}%
\pgfpathlineto{\pgfqpoint{1.932792in}{0.723075in}}%
\pgfpathlineto{\pgfqpoint{1.933089in}{0.723074in}}%
\pgfpathlineto{\pgfqpoint{1.933387in}{0.723073in}}%
\pgfpathlineto{\pgfqpoint{1.933684in}{0.723072in}}%
\pgfpathlineto{\pgfqpoint{1.933982in}{0.723072in}}%
\pgfpathlineto{\pgfqpoint{1.934279in}{0.723071in}}%
\pgfpathlineto{\pgfqpoint{1.934576in}{0.723070in}}%
\pgfpathlineto{\pgfqpoint{1.934874in}{0.723069in}}%
\pgfpathlineto{\pgfqpoint{1.935171in}{0.723068in}}%
\pgfpathlineto{\pgfqpoint{1.935469in}{0.723067in}}%
\pgfpathlineto{\pgfqpoint{1.935766in}{0.723066in}}%
\pgfpathlineto{\pgfqpoint{1.936064in}{0.723065in}}%
\pgfpathlineto{\pgfqpoint{1.936361in}{0.723064in}}%
\pgfpathlineto{\pgfqpoint{1.936659in}{0.723063in}}%
\pgfpathlineto{\pgfqpoint{1.936956in}{0.723062in}}%
\pgfpathlineto{\pgfqpoint{1.937254in}{0.723061in}}%
\pgfpathlineto{\pgfqpoint{1.937551in}{0.723060in}}%
\pgfpathlineto{\pgfqpoint{1.937849in}{0.723059in}}%
\pgfpathlineto{\pgfqpoint{1.938146in}{0.723058in}}%
\pgfpathlineto{\pgfqpoint{1.938444in}{0.723057in}}%
\pgfpathlineto{\pgfqpoint{1.938741in}{0.723056in}}%
\pgfpathlineto{\pgfqpoint{1.939039in}{0.723055in}}%
\pgfpathlineto{\pgfqpoint{1.939336in}{0.723054in}}%
\pgfpathlineto{\pgfqpoint{1.939634in}{0.723053in}}%
\pgfpathlineto{\pgfqpoint{1.939931in}{0.723052in}}%
\pgfpathlineto{\pgfqpoint{1.940229in}{0.723051in}}%
\pgfpathlineto{\pgfqpoint{1.940526in}{0.723050in}}%
\pgfpathlineto{\pgfqpoint{1.940823in}{0.723049in}}%
\pgfpathlineto{\pgfqpoint{1.941121in}{0.723048in}}%
\pgfpathlineto{\pgfqpoint{1.941418in}{0.723047in}}%
\pgfpathlineto{\pgfqpoint{1.941716in}{0.723046in}}%
\pgfpathlineto{\pgfqpoint{1.942013in}{0.723045in}}%
\pgfpathlineto{\pgfqpoint{1.942311in}{0.723041in}}%
\pgfpathlineto{\pgfqpoint{1.942608in}{0.723037in}}%
\pgfpathlineto{\pgfqpoint{1.942906in}{0.723034in}}%
\pgfpathlineto{\pgfqpoint{1.943203in}{0.723030in}}%
\pgfpathlineto{\pgfqpoint{1.943501in}{0.723026in}}%
\pgfpathlineto{\pgfqpoint{1.943798in}{0.723022in}}%
\pgfpathlineto{\pgfqpoint{1.944096in}{0.723019in}}%
\pgfpathlineto{\pgfqpoint{1.944393in}{0.723015in}}%
\pgfpathlineto{\pgfqpoint{1.944691in}{0.723011in}}%
\pgfpathlineto{\pgfqpoint{1.944988in}{0.723007in}}%
\pgfpathlineto{\pgfqpoint{1.945286in}{0.723005in}}%
\pgfpathlineto{\pgfqpoint{1.945583in}{0.723006in}}%
\pgfpathlineto{\pgfqpoint{1.945881in}{0.723007in}}%
\pgfpathlineto{\pgfqpoint{1.946178in}{0.723008in}}%
\pgfpathlineto{\pgfqpoint{1.946476in}{0.723009in}}%
\pgfpathlineto{\pgfqpoint{1.946773in}{0.723010in}}%
\pgfpathlineto{\pgfqpoint{1.947071in}{0.723011in}}%
\pgfpathlineto{\pgfqpoint{1.947368in}{0.723012in}}%
\pgfpathlineto{\pgfqpoint{1.947665in}{0.723013in}}%
\pgfpathlineto{\pgfqpoint{1.947963in}{0.723014in}}%
\pgfpathlineto{\pgfqpoint{1.948260in}{0.723015in}}%
\pgfpathlineto{\pgfqpoint{1.948558in}{0.723016in}}%
\pgfpathlineto{\pgfqpoint{1.948855in}{0.723018in}}%
\pgfpathlineto{\pgfqpoint{1.949153in}{0.723019in}}%
\pgfpathlineto{\pgfqpoint{1.949450in}{0.723020in}}%
\pgfpathlineto{\pgfqpoint{1.949748in}{0.723021in}}%
\pgfpathlineto{\pgfqpoint{1.950045in}{0.723022in}}%
\pgfpathlineto{\pgfqpoint{1.950343in}{0.723023in}}%
\pgfpathlineto{\pgfqpoint{1.950640in}{0.723024in}}%
\pgfpathlineto{\pgfqpoint{1.950938in}{0.723025in}}%
\pgfpathlineto{\pgfqpoint{1.951235in}{0.723026in}}%
\pgfpathlineto{\pgfqpoint{1.951533in}{0.723027in}}%
\pgfpathlineto{\pgfqpoint{1.951830in}{0.723028in}}%
\pgfpathlineto{\pgfqpoint{1.952128in}{0.723029in}}%
\pgfpathlineto{\pgfqpoint{1.952425in}{0.723030in}}%
\pgfpathlineto{\pgfqpoint{1.952723in}{0.723031in}}%
\pgfpathlineto{\pgfqpoint{1.953020in}{0.723032in}}%
\pgfpathlineto{\pgfqpoint{1.953318in}{0.723033in}}%
\pgfpathlineto{\pgfqpoint{1.953615in}{0.723034in}}%
\pgfpathlineto{\pgfqpoint{1.953913in}{0.723035in}}%
\pgfpathlineto{\pgfqpoint{1.954210in}{0.723036in}}%
\pgfpathlineto{\pgfqpoint{1.954507in}{0.723037in}}%
\pgfpathlineto{\pgfqpoint{1.954805in}{0.723038in}}%
\pgfpathlineto{\pgfqpoint{1.955102in}{0.723039in}}%
\pgfpathlineto{\pgfqpoint{1.955400in}{0.723040in}}%
\pgfpathlineto{\pgfqpoint{1.955697in}{0.723041in}}%
\pgfpathlineto{\pgfqpoint{1.955995in}{0.723042in}}%
\pgfpathlineto{\pgfqpoint{1.956292in}{0.723043in}}%
\pgfpathlineto{\pgfqpoint{1.956590in}{0.723044in}}%
\pgfpathlineto{\pgfqpoint{1.956887in}{0.723045in}}%
\pgfpathlineto{\pgfqpoint{1.957185in}{0.723046in}}%
\pgfpathlineto{\pgfqpoint{1.957482in}{0.723047in}}%
\pgfpathlineto{\pgfqpoint{1.957780in}{0.723048in}}%
\pgfpathlineto{\pgfqpoint{1.958077in}{0.723049in}}%
\pgfpathlineto{\pgfqpoint{1.958375in}{0.723050in}}%
\pgfpathlineto{\pgfqpoint{1.958672in}{0.723051in}}%
\pgfpathlineto{\pgfqpoint{1.958970in}{0.723052in}}%
\pgfpathlineto{\pgfqpoint{1.959267in}{0.723054in}}%
\pgfpathlineto{\pgfqpoint{1.959565in}{0.723055in}}%
\pgfpathlineto{\pgfqpoint{1.959862in}{0.723056in}}%
\pgfpathlineto{\pgfqpoint{1.960160in}{0.723057in}}%
\pgfpathlineto{\pgfqpoint{1.960457in}{0.723058in}}%
\pgfpathlineto{\pgfqpoint{1.960754in}{0.723059in}}%
\pgfpathlineto{\pgfqpoint{1.961052in}{0.723060in}}%
\pgfpathlineto{\pgfqpoint{1.961349in}{0.723061in}}%
\pgfpathlineto{\pgfqpoint{1.961647in}{0.723062in}}%
\pgfpathlineto{\pgfqpoint{1.961944in}{0.723063in}}%
\pgfpathlineto{\pgfqpoint{1.962242in}{0.723064in}}%
\pgfpathlineto{\pgfqpoint{1.962539in}{0.723065in}}%
\pgfpathlineto{\pgfqpoint{1.962837in}{0.723066in}}%
\pgfpathlineto{\pgfqpoint{1.963134in}{0.723067in}}%
\pgfpathlineto{\pgfqpoint{1.963432in}{0.723068in}}%
\pgfpathlineto{\pgfqpoint{1.963729in}{0.723069in}}%
\pgfpathlineto{\pgfqpoint{1.964027in}{0.723064in}}%
\pgfpathlineto{\pgfqpoint{1.964324in}{0.723059in}}%
\pgfpathlineto{\pgfqpoint{1.964622in}{0.723054in}}%
\pgfpathlineto{\pgfqpoint{1.964919in}{0.723048in}}%
\pgfpathlineto{\pgfqpoint{1.965217in}{0.723043in}}%
\pgfpathlineto{\pgfqpoint{1.965514in}{0.723038in}}%
\pgfpathlineto{\pgfqpoint{1.965812in}{0.723032in}}%
\pgfpathlineto{\pgfqpoint{1.966109in}{0.723027in}}%
\pgfpathlineto{\pgfqpoint{1.966407in}{0.723022in}}%
\pgfpathlineto{\pgfqpoint{1.966704in}{0.723017in}}%
\pgfpathlineto{\pgfqpoint{1.967002in}{0.723011in}}%
\pgfpathlineto{\pgfqpoint{1.967299in}{0.723006in}}%
\pgfpathlineto{\pgfqpoint{1.967596in}{0.723001in}}%
\pgfpathlineto{\pgfqpoint{1.967894in}{0.722995in}}%
\pgfpathlineto{\pgfqpoint{1.968191in}{0.722990in}}%
\pgfpathlineto{\pgfqpoint{1.968489in}{0.722985in}}%
\pgfpathlineto{\pgfqpoint{1.968786in}{0.722979in}}%
\pgfpathlineto{\pgfqpoint{1.969084in}{0.722974in}}%
\pgfpathlineto{\pgfqpoint{1.969381in}{0.722969in}}%
\pgfpathlineto{\pgfqpoint{1.969679in}{0.722963in}}%
\pgfpathlineto{\pgfqpoint{1.969976in}{0.722958in}}%
\pgfpathlineto{\pgfqpoint{1.970274in}{0.722953in}}%
\pgfpathlineto{\pgfqpoint{1.970571in}{0.722947in}}%
\pgfpathlineto{\pgfqpoint{1.970869in}{0.722942in}}%
\pgfpathlineto{\pgfqpoint{1.971166in}{0.722937in}}%
\pgfpathlineto{\pgfqpoint{1.971464in}{0.722931in}}%
\pgfpathlineto{\pgfqpoint{1.971761in}{0.722926in}}%
\pgfpathlineto{\pgfqpoint{1.972059in}{0.722921in}}%
\pgfpathlineto{\pgfqpoint{1.972356in}{0.722915in}}%
\pgfpathlineto{\pgfqpoint{1.972654in}{0.722910in}}%
\pgfpathlineto{\pgfqpoint{1.972951in}{0.722905in}}%
\pgfpathlineto{\pgfqpoint{1.973249in}{0.722899in}}%
\pgfpathlineto{\pgfqpoint{1.973546in}{0.722894in}}%
\pgfpathlineto{\pgfqpoint{1.973844in}{0.722889in}}%
\pgfpathlineto{\pgfqpoint{1.974141in}{0.722883in}}%
\pgfpathlineto{\pgfqpoint{1.974438in}{0.722878in}}%
\pgfpathlineto{\pgfqpoint{1.974736in}{0.722873in}}%
\pgfpathlineto{\pgfqpoint{1.975033in}{0.722867in}}%
\pgfpathlineto{\pgfqpoint{1.975331in}{0.722862in}}%
\pgfpathlineto{\pgfqpoint{1.975628in}{0.722857in}}%
\pgfpathlineto{\pgfqpoint{1.975926in}{0.722851in}}%
\pgfpathlineto{\pgfqpoint{1.976223in}{0.722846in}}%
\pgfpathlineto{\pgfqpoint{1.976521in}{0.722841in}}%
\pgfpathlineto{\pgfqpoint{1.976818in}{0.722835in}}%
\pgfpathlineto{\pgfqpoint{1.977116in}{0.722830in}}%
\pgfpathlineto{\pgfqpoint{1.977413in}{0.722825in}}%
\pgfpathlineto{\pgfqpoint{1.977711in}{0.722819in}}%
\pgfpathlineto{\pgfqpoint{1.978008in}{0.722814in}}%
\pgfpathlineto{\pgfqpoint{1.978306in}{0.722809in}}%
\pgfpathlineto{\pgfqpoint{1.978603in}{0.722803in}}%
\pgfpathlineto{\pgfqpoint{1.978901in}{0.722798in}}%
\pgfpathlineto{\pgfqpoint{1.979198in}{0.722793in}}%
\pgfpathlineto{\pgfqpoint{1.979496in}{0.722788in}}%
\pgfpathlineto{\pgfqpoint{1.979793in}{0.722782in}}%
\pgfpathlineto{\pgfqpoint{1.980091in}{0.722777in}}%
\pgfpathlineto{\pgfqpoint{1.980388in}{0.722772in}}%
\pgfpathlineto{\pgfqpoint{1.980685in}{0.722766in}}%
\pgfpathlineto{\pgfqpoint{1.980983in}{0.722761in}}%
\pgfpathlineto{\pgfqpoint{1.981280in}{0.722756in}}%
\pgfpathlineto{\pgfqpoint{1.981578in}{0.722750in}}%
\pgfpathlineto{\pgfqpoint{1.981875in}{0.722745in}}%
\pgfpathlineto{\pgfqpoint{1.982173in}{0.722740in}}%
\pgfpathlineto{\pgfqpoint{1.982470in}{0.722734in}}%
\pgfpathlineto{\pgfqpoint{1.982768in}{0.722729in}}%
\pgfpathlineto{\pgfqpoint{1.983065in}{0.722724in}}%
\pgfpathlineto{\pgfqpoint{1.983363in}{0.722718in}}%
\pgfpathlineto{\pgfqpoint{1.983660in}{0.722713in}}%
\pgfpathlineto{\pgfqpoint{1.983958in}{0.722708in}}%
\pgfpathlineto{\pgfqpoint{1.984255in}{0.722702in}}%
\pgfpathlineto{\pgfqpoint{1.984553in}{0.722697in}}%
\pgfpathlineto{\pgfqpoint{1.984850in}{0.722692in}}%
\pgfpathlineto{\pgfqpoint{1.985148in}{0.722686in}}%
\pgfpathlineto{\pgfqpoint{1.985445in}{0.722681in}}%
\pgfpathlineto{\pgfqpoint{1.985743in}{0.722676in}}%
\pgfpathlineto{\pgfqpoint{1.986040in}{0.722670in}}%
\pgfpathlineto{\pgfqpoint{1.986338in}{0.722665in}}%
\pgfpathlineto{\pgfqpoint{1.986635in}{0.722660in}}%
\pgfpathlineto{\pgfqpoint{1.986933in}{0.722654in}}%
\pgfpathlineto{\pgfqpoint{1.987230in}{0.722649in}}%
\pgfpathlineto{\pgfqpoint{1.987527in}{0.722644in}}%
\pgfpathlineto{\pgfqpoint{1.987825in}{0.722638in}}%
\pgfpathlineto{\pgfqpoint{1.988122in}{0.722633in}}%
\pgfpathlineto{\pgfqpoint{1.988420in}{0.722628in}}%
\pgfpathlineto{\pgfqpoint{1.988717in}{0.722622in}}%
\pgfpathlineto{\pgfqpoint{1.989015in}{0.722617in}}%
\pgfpathlineto{\pgfqpoint{1.989312in}{0.722612in}}%
\pgfpathlineto{\pgfqpoint{1.989610in}{0.722606in}}%
\pgfpathlineto{\pgfqpoint{1.989907in}{0.722601in}}%
\pgfpathlineto{\pgfqpoint{1.990205in}{0.722596in}}%
\pgfpathlineto{\pgfqpoint{1.990502in}{0.722590in}}%
\pgfpathlineto{\pgfqpoint{1.990800in}{0.722585in}}%
\pgfpathlineto{\pgfqpoint{1.991097in}{0.722580in}}%
\pgfpathlineto{\pgfqpoint{1.991395in}{0.722574in}}%
\pgfpathlineto{\pgfqpoint{1.991692in}{0.722570in}}%
\pgfpathlineto{\pgfqpoint{1.991990in}{0.722644in}}%
\pgfpathlineto{\pgfqpoint{1.992287in}{0.722774in}}%
\pgfpathlineto{\pgfqpoint{1.992585in}{0.722904in}}%
\pgfpathlineto{\pgfqpoint{1.992882in}{0.723034in}}%
\pgfpathlineto{\pgfqpoint{1.993180in}{0.723164in}}%
\pgfpathlineto{\pgfqpoint{1.993477in}{0.723294in}}%
\pgfpathlineto{\pgfqpoint{1.993775in}{0.723424in}}%
\pgfpathlineto{\pgfqpoint{1.994072in}{0.723553in}}%
\pgfpathlineto{\pgfqpoint{1.994369in}{0.723683in}}%
\pgfpathlineto{\pgfqpoint{1.994667in}{0.723813in}}%
\pgfpathlineto{\pgfqpoint{1.994964in}{0.723943in}}%
\pgfpathlineto{\pgfqpoint{1.995262in}{0.724073in}}%
\pgfpathlineto{\pgfqpoint{1.995559in}{0.724203in}}%
\pgfpathlineto{\pgfqpoint{1.995857in}{0.724333in}}%
\pgfpathlineto{\pgfqpoint{1.996154in}{0.724463in}}%
\pgfpathlineto{\pgfqpoint{1.996452in}{0.724593in}}%
\pgfpathlineto{\pgfqpoint{1.996749in}{0.724723in}}%
\pgfpathlineto{\pgfqpoint{1.997047in}{0.724853in}}%
\pgfpathlineto{\pgfqpoint{1.997344in}{0.724983in}}%
\pgfpathlineto{\pgfqpoint{1.997642in}{0.725113in}}%
\pgfpathlineto{\pgfqpoint{1.997939in}{0.725243in}}%
\pgfpathlineto{\pgfqpoint{1.998237in}{0.725373in}}%
\pgfpathlineto{\pgfqpoint{1.998534in}{0.725503in}}%
\pgfpathlineto{\pgfqpoint{1.998832in}{0.725633in}}%
\pgfpathlineto{\pgfqpoint{1.999129in}{0.725763in}}%
\pgfpathlineto{\pgfqpoint{1.999427in}{0.725892in}}%
\pgfpathlineto{\pgfqpoint{1.999724in}{0.726022in}}%
\pgfpathlineto{\pgfqpoint{2.000022in}{0.726152in}}%
\pgfpathlineto{\pgfqpoint{2.000319in}{0.726279in}}%
\pgfpathlineto{\pgfqpoint{2.000616in}{0.726395in}}%
\pgfpathlineto{\pgfqpoint{2.000914in}{0.726509in}}%
\pgfpathlineto{\pgfqpoint{2.001211in}{0.726623in}}%
\pgfpathlineto{\pgfqpoint{2.001509in}{0.726737in}}%
\pgfpathlineto{\pgfqpoint{2.001806in}{0.734679in}}%
\pgfpathlineto{\pgfqpoint{2.002104in}{0.740686in}}%
\pgfpathlineto{\pgfqpoint{2.002401in}{0.740696in}}%
\pgfpathlineto{\pgfqpoint{2.002699in}{0.740738in}}%
\pgfpathlineto{\pgfqpoint{2.002996in}{0.740734in}}%
\pgfpathlineto{\pgfqpoint{2.003294in}{0.740730in}}%
\pgfpathlineto{\pgfqpoint{2.003591in}{0.740726in}}%
\pgfpathlineto{\pgfqpoint{2.003889in}{0.740723in}}%
\pgfpathlineto{\pgfqpoint{2.004186in}{0.740719in}}%
\pgfpathlineto{\pgfqpoint{2.004484in}{0.740715in}}%
\pgfpathlineto{\pgfqpoint{2.004781in}{0.740711in}}%
\pgfpathlineto{\pgfqpoint{2.005079in}{0.740708in}}%
\pgfpathlineto{\pgfqpoint{2.005376in}{0.740704in}}%
\pgfpathlineto{\pgfqpoint{2.005674in}{0.740735in}}%
\pgfpathlineto{\pgfqpoint{2.005971in}{0.740946in}}%
\pgfpathlineto{\pgfqpoint{2.006269in}{0.741182in}}%
\pgfpathlineto{\pgfqpoint{2.006566in}{0.741288in}}%
\pgfpathlineto{\pgfqpoint{2.006864in}{0.741284in}}%
\pgfpathlineto{\pgfqpoint{2.007161in}{0.741281in}}%
\pgfpathlineto{\pgfqpoint{2.007458in}{0.741277in}}%
\pgfpathlineto{\pgfqpoint{2.007756in}{0.741273in}}%
\pgfpathlineto{\pgfqpoint{2.008053in}{0.741269in}}%
\pgfpathlineto{\pgfqpoint{2.008351in}{0.741266in}}%
\pgfpathlineto{\pgfqpoint{2.008648in}{0.741262in}}%
\pgfpathlineto{\pgfqpoint{2.008946in}{0.741258in}}%
\pgfpathlineto{\pgfqpoint{2.009243in}{0.741254in}}%
\pgfpathlineto{\pgfqpoint{2.009541in}{0.741251in}}%
\pgfpathlineto{\pgfqpoint{2.009838in}{0.741247in}}%
\pgfpathlineto{\pgfqpoint{2.010136in}{0.741243in}}%
\pgfpathlineto{\pgfqpoint{2.010433in}{0.741239in}}%
\pgfpathlineto{\pgfqpoint{2.010731in}{0.741236in}}%
\pgfpathlineto{\pgfqpoint{2.011028in}{0.741232in}}%
\pgfpathlineto{\pgfqpoint{2.011326in}{0.741228in}}%
\pgfpathlineto{\pgfqpoint{2.011623in}{0.741224in}}%
\pgfpathlineto{\pgfqpoint{2.011921in}{0.741221in}}%
\pgfpathlineto{\pgfqpoint{2.012218in}{0.741217in}}%
\pgfpathlineto{\pgfqpoint{2.012516in}{0.741213in}}%
\pgfpathlineto{\pgfqpoint{2.012813in}{0.741210in}}%
\pgfpathlineto{\pgfqpoint{2.013111in}{0.741206in}}%
\pgfpathlineto{\pgfqpoint{2.013408in}{0.741202in}}%
\pgfpathlineto{\pgfqpoint{2.013706in}{0.741198in}}%
\pgfpathlineto{\pgfqpoint{2.014003in}{0.741195in}}%
\pgfpathlineto{\pgfqpoint{2.014300in}{0.741191in}}%
\pgfpathlineto{\pgfqpoint{2.014598in}{0.741187in}}%
\pgfpathlineto{\pgfqpoint{2.014895in}{0.741183in}}%
\pgfpathlineto{\pgfqpoint{2.015193in}{0.741180in}}%
\pgfpathlineto{\pgfqpoint{2.015490in}{0.741176in}}%
\pgfpathlineto{\pgfqpoint{2.015788in}{0.741172in}}%
\pgfpathlineto{\pgfqpoint{2.016085in}{0.741168in}}%
\pgfpathlineto{\pgfqpoint{2.016383in}{0.741165in}}%
\pgfpathlineto{\pgfqpoint{2.016680in}{0.741161in}}%
\pgfpathlineto{\pgfqpoint{2.016978in}{0.741157in}}%
\pgfpathlineto{\pgfqpoint{2.017275in}{0.741153in}}%
\pgfpathlineto{\pgfqpoint{2.017573in}{0.741150in}}%
\pgfpathlineto{\pgfqpoint{2.017870in}{0.741146in}}%
\pgfpathlineto{\pgfqpoint{2.018168in}{0.741142in}}%
\pgfpathlineto{\pgfqpoint{2.018465in}{0.741138in}}%
\pgfpathlineto{\pgfqpoint{2.018763in}{0.741135in}}%
\pgfpathlineto{\pgfqpoint{2.019060in}{0.741131in}}%
\pgfpathlineto{\pgfqpoint{2.019358in}{0.741127in}}%
\pgfpathlineto{\pgfqpoint{2.019655in}{0.741123in}}%
\pgfpathlineto{\pgfqpoint{2.019953in}{0.741120in}}%
\pgfpathlineto{\pgfqpoint{2.020250in}{0.741116in}}%
\pgfpathlineto{\pgfqpoint{2.020548in}{0.741112in}}%
\pgfpathlineto{\pgfqpoint{2.020845in}{0.741108in}}%
\pgfpathlineto{\pgfqpoint{2.021142in}{0.741105in}}%
\pgfpathlineto{\pgfqpoint{2.021440in}{0.741101in}}%
\pgfpathlineto{\pgfqpoint{2.021737in}{0.741097in}}%
\pgfpathlineto{\pgfqpoint{2.022035in}{0.741093in}}%
\pgfpathlineto{\pgfqpoint{2.022332in}{0.741090in}}%
\pgfpathlineto{\pgfqpoint{2.022630in}{0.741086in}}%
\pgfpathlineto{\pgfqpoint{2.022927in}{0.741082in}}%
\pgfpathlineto{\pgfqpoint{2.023225in}{0.741078in}}%
\pgfpathlineto{\pgfqpoint{2.023522in}{0.741075in}}%
\pgfpathlineto{\pgfqpoint{2.023820in}{0.741071in}}%
\pgfpathlineto{\pgfqpoint{2.024117in}{0.741067in}}%
\pgfpathlineto{\pgfqpoint{2.024415in}{0.741063in}}%
\pgfpathlineto{\pgfqpoint{2.024712in}{0.741060in}}%
\pgfpathlineto{\pgfqpoint{2.025010in}{0.741056in}}%
\pgfpathlineto{\pgfqpoint{2.025307in}{0.741052in}}%
\pgfpathlineto{\pgfqpoint{2.025605in}{0.741048in}}%
\pgfpathlineto{\pgfqpoint{2.025902in}{0.741045in}}%
\pgfpathlineto{\pgfqpoint{2.026200in}{0.741041in}}%
\pgfpathlineto{\pgfqpoint{2.026497in}{0.741037in}}%
\pgfpathlineto{\pgfqpoint{2.026795in}{0.741033in}}%
\pgfpathlineto{\pgfqpoint{2.027092in}{0.741030in}}%
\pgfpathlineto{\pgfqpoint{2.027389in}{0.741026in}}%
\pgfpathlineto{\pgfqpoint{2.027687in}{0.741022in}}%
\pgfpathlineto{\pgfqpoint{2.027984in}{0.741019in}}%
\pgfpathlineto{\pgfqpoint{2.028282in}{0.741015in}}%
\pgfpathlineto{\pgfqpoint{2.028579in}{0.741011in}}%
\pgfpathlineto{\pgfqpoint{2.028877in}{0.741007in}}%
\pgfpathlineto{\pgfqpoint{2.029174in}{0.741004in}}%
\pgfpathlineto{\pgfqpoint{2.029472in}{0.741000in}}%
\pgfpathlineto{\pgfqpoint{2.029769in}{0.740996in}}%
\pgfpathlineto{\pgfqpoint{2.030067in}{0.740992in}}%
\pgfpathlineto{\pgfqpoint{2.030364in}{0.740989in}}%
\pgfpathlineto{\pgfqpoint{2.030662in}{0.740985in}}%
\pgfpathlineto{\pgfqpoint{2.030959in}{0.740981in}}%
\pgfpathlineto{\pgfqpoint{2.031257in}{0.740977in}}%
\pgfpathlineto{\pgfqpoint{2.031554in}{0.740974in}}%
\pgfpathlineto{\pgfqpoint{2.031852in}{0.740970in}}%
\pgfpathlineto{\pgfqpoint{2.032149in}{0.740966in}}%
\pgfpathlineto{\pgfqpoint{2.032447in}{0.740962in}}%
\pgfpathlineto{\pgfqpoint{2.032744in}{0.740959in}}%
\pgfpathlineto{\pgfqpoint{2.033042in}{0.740955in}}%
\pgfpathlineto{\pgfqpoint{2.033339in}{0.740951in}}%
\pgfpathlineto{\pgfqpoint{2.033637in}{0.740947in}}%
\pgfpathlineto{\pgfqpoint{2.033934in}{0.740944in}}%
\pgfpathlineto{\pgfqpoint{2.034231in}{0.740940in}}%
\pgfpathlineto{\pgfqpoint{2.034529in}{0.740936in}}%
\pgfpathlineto{\pgfqpoint{2.034826in}{0.740932in}}%
\pgfpathlineto{\pgfqpoint{2.035124in}{0.740929in}}%
\pgfpathlineto{\pgfqpoint{2.035421in}{0.740925in}}%
\pgfpathlineto{\pgfqpoint{2.035719in}{0.740921in}}%
\pgfpathlineto{\pgfqpoint{2.036016in}{0.740917in}}%
\pgfpathlineto{\pgfqpoint{2.036314in}{0.740914in}}%
\pgfpathlineto{\pgfqpoint{2.036611in}{0.740910in}}%
\pgfpathlineto{\pgfqpoint{2.036909in}{0.740906in}}%
\pgfpathlineto{\pgfqpoint{2.037206in}{0.740902in}}%
\pgfpathlineto{\pgfqpoint{2.037504in}{0.740899in}}%
\pgfpathlineto{\pgfqpoint{2.037801in}{0.740895in}}%
\pgfpathlineto{\pgfqpoint{2.038099in}{0.740891in}}%
\pgfpathlineto{\pgfqpoint{2.038396in}{0.740887in}}%
\pgfpathlineto{\pgfqpoint{2.038694in}{0.740884in}}%
\pgfpathlineto{\pgfqpoint{2.038991in}{0.740880in}}%
\pgfpathlineto{\pgfqpoint{2.039289in}{0.740876in}}%
\pgfpathlineto{\pgfqpoint{2.039586in}{0.740872in}}%
\pgfpathlineto{\pgfqpoint{2.039884in}{0.740869in}}%
\pgfpathlineto{\pgfqpoint{2.040181in}{0.740865in}}%
\pgfpathlineto{\pgfqpoint{2.040479in}{0.740861in}}%
\pgfpathlineto{\pgfqpoint{2.040776in}{0.740857in}}%
\pgfpathlineto{\pgfqpoint{2.041073in}{0.740854in}}%
\pgfpathlineto{\pgfqpoint{2.041371in}{0.740850in}}%
\pgfpathlineto{\pgfqpoint{2.041668in}{0.740847in}}%
\pgfpathlineto{\pgfqpoint{2.041966in}{0.740845in}}%
\pgfpathlineto{\pgfqpoint{2.042263in}{0.740843in}}%
\pgfpathlineto{\pgfqpoint{2.042561in}{0.740841in}}%
\pgfpathlineto{\pgfqpoint{2.042858in}{0.740839in}}%
\pgfpathlineto{\pgfqpoint{2.043156in}{0.740837in}}%
\pgfpathlineto{\pgfqpoint{2.043453in}{0.740835in}}%
\pgfpathlineto{\pgfqpoint{2.043751in}{0.740833in}}%
\pgfpathlineto{\pgfqpoint{2.044048in}{0.740831in}}%
\pgfpathlineto{\pgfqpoint{2.044346in}{0.740829in}}%
\pgfpathlineto{\pgfqpoint{2.044643in}{0.740827in}}%
\pgfpathlineto{\pgfqpoint{2.044941in}{0.740825in}}%
\pgfpathlineto{\pgfqpoint{2.045238in}{0.740823in}}%
\pgfpathlineto{\pgfqpoint{2.045536in}{0.740821in}}%
\pgfpathlineto{\pgfqpoint{2.045833in}{0.740819in}}%
\pgfpathlineto{\pgfqpoint{2.046131in}{0.740817in}}%
\pgfpathlineto{\pgfqpoint{2.046428in}{0.740815in}}%
\pgfpathlineto{\pgfqpoint{2.046726in}{0.740813in}}%
\pgfpathlineto{\pgfqpoint{2.047023in}{0.740811in}}%
\pgfpathlineto{\pgfqpoint{2.047320in}{0.740809in}}%
\pgfpathlineto{\pgfqpoint{2.047618in}{0.740807in}}%
\pgfpathlineto{\pgfqpoint{2.047915in}{0.740805in}}%
\pgfpathlineto{\pgfqpoint{2.048213in}{0.740803in}}%
\pgfpathlineto{\pgfqpoint{2.048510in}{0.740801in}}%
\pgfpathlineto{\pgfqpoint{2.048808in}{0.740799in}}%
\pgfpathlineto{\pgfqpoint{2.049105in}{0.740797in}}%
\pgfpathlineto{\pgfqpoint{2.049403in}{0.740794in}}%
\pgfpathlineto{\pgfqpoint{2.049700in}{0.740792in}}%
\pgfpathlineto{\pgfqpoint{2.049998in}{0.740790in}}%
\pgfpathlineto{\pgfqpoint{2.050295in}{0.740788in}}%
\pgfpathlineto{\pgfqpoint{2.050593in}{0.740786in}}%
\pgfpathlineto{\pgfqpoint{2.050890in}{0.740784in}}%
\pgfpathlineto{\pgfqpoint{2.051188in}{0.740782in}}%
\pgfpathlineto{\pgfqpoint{2.051485in}{0.740780in}}%
\pgfpathlineto{\pgfqpoint{2.051783in}{0.740778in}}%
\pgfpathlineto{\pgfqpoint{2.052080in}{0.740776in}}%
\pgfpathlineto{\pgfqpoint{2.052378in}{0.740774in}}%
\pgfpathlineto{\pgfqpoint{2.052675in}{0.740772in}}%
\pgfpathlineto{\pgfqpoint{2.052973in}{0.740770in}}%
\pgfpathlineto{\pgfqpoint{2.053270in}{0.740768in}}%
\pgfpathlineto{\pgfqpoint{2.053568in}{0.740766in}}%
\pgfpathlineto{\pgfqpoint{2.053865in}{0.740764in}}%
\pgfpathlineto{\pgfqpoint{2.054162in}{0.740762in}}%
\pgfpathlineto{\pgfqpoint{2.054460in}{0.740760in}}%
\pgfpathlineto{\pgfqpoint{2.054757in}{0.740758in}}%
\pgfpathlineto{\pgfqpoint{2.055055in}{0.740756in}}%
\pgfpathlineto{\pgfqpoint{2.055352in}{0.740754in}}%
\pgfpathlineto{\pgfqpoint{2.055650in}{0.740752in}}%
\pgfpathlineto{\pgfqpoint{2.055947in}{0.740750in}}%
\pgfpathlineto{\pgfqpoint{2.056245in}{0.740748in}}%
\pgfpathlineto{\pgfqpoint{2.056542in}{0.740720in}}%
\pgfpathlineto{\pgfqpoint{2.056840in}{0.740393in}}%
\pgfpathlineto{\pgfqpoint{2.057137in}{0.739981in}}%
\pgfpathlineto{\pgfqpoint{2.057435in}{0.739570in}}%
\pgfpathlineto{\pgfqpoint{2.057732in}{0.739159in}}%
\pgfpathlineto{\pgfqpoint{2.058030in}{0.738956in}}%
\pgfpathlineto{\pgfqpoint{2.058327in}{0.738954in}}%
\pgfpathlineto{\pgfqpoint{2.058625in}{0.738952in}}%
\pgfpathlineto{\pgfqpoint{2.058922in}{0.738950in}}%
\pgfpathlineto{\pgfqpoint{2.059220in}{0.738949in}}%
\pgfpathlineto{\pgfqpoint{2.059517in}{0.738947in}}%
\pgfpathlineto{\pgfqpoint{2.059815in}{0.738945in}}%
\pgfpathlineto{\pgfqpoint{2.060112in}{0.738943in}}%
\pgfpathlineto{\pgfqpoint{2.060410in}{0.738941in}}%
\pgfpathlineto{\pgfqpoint{2.060707in}{0.738939in}}%
\pgfpathlineto{\pgfqpoint{2.061004in}{0.738937in}}%
\pgfpathlineto{\pgfqpoint{2.061302in}{0.738936in}}%
\pgfpathlineto{\pgfqpoint{2.061599in}{0.738934in}}%
\pgfpathlineto{\pgfqpoint{2.061897in}{0.738932in}}%
\pgfpathlineto{\pgfqpoint{2.062194in}{0.738930in}}%
\pgfpathlineto{\pgfqpoint{2.062492in}{0.738928in}}%
\pgfpathlineto{\pgfqpoint{2.062789in}{0.738926in}}%
\pgfpathlineto{\pgfqpoint{2.063087in}{0.738924in}}%
\pgfpathlineto{\pgfqpoint{2.063384in}{0.738923in}}%
\pgfpathlineto{\pgfqpoint{2.063682in}{0.738921in}}%
\pgfpathlineto{\pgfqpoint{2.063979in}{0.738919in}}%
\pgfpathlineto{\pgfqpoint{2.064277in}{0.738917in}}%
\pgfpathlineto{\pgfqpoint{2.064574in}{0.738915in}}%
\pgfpathlineto{\pgfqpoint{2.064872in}{0.738913in}}%
\pgfpathlineto{\pgfqpoint{2.065169in}{0.738912in}}%
\pgfpathlineto{\pgfqpoint{2.065467in}{0.738910in}}%
\pgfpathlineto{\pgfqpoint{2.065764in}{0.738908in}}%
\pgfpathlineto{\pgfqpoint{2.066062in}{0.738906in}}%
\pgfpathlineto{\pgfqpoint{2.066359in}{0.738904in}}%
\pgfpathlineto{\pgfqpoint{2.066657in}{0.738902in}}%
\pgfpathlineto{\pgfqpoint{2.066954in}{0.738900in}}%
\pgfpathlineto{\pgfqpoint{2.067251in}{0.738899in}}%
\pgfpathlineto{\pgfqpoint{2.067549in}{0.738897in}}%
\pgfpathlineto{\pgfqpoint{2.067846in}{0.738895in}}%
\pgfpathlineto{\pgfqpoint{2.068144in}{0.738893in}}%
\pgfpathlineto{\pgfqpoint{2.068441in}{0.738891in}}%
\pgfpathlineto{\pgfqpoint{2.068739in}{0.738889in}}%
\pgfpathlineto{\pgfqpoint{2.069036in}{0.738887in}}%
\pgfpathlineto{\pgfqpoint{2.069334in}{0.738886in}}%
\pgfpathlineto{\pgfqpoint{2.069631in}{0.738884in}}%
\pgfpathlineto{\pgfqpoint{2.069929in}{0.738882in}}%
\pgfpathlineto{\pgfqpoint{2.070226in}{0.738880in}}%
\pgfpathlineto{\pgfqpoint{2.070524in}{0.738878in}}%
\pgfpathlineto{\pgfqpoint{2.070821in}{0.738876in}}%
\pgfpathlineto{\pgfqpoint{2.071119in}{0.738874in}}%
\pgfpathlineto{\pgfqpoint{2.071416in}{0.738873in}}%
\pgfpathlineto{\pgfqpoint{2.071714in}{0.738871in}}%
\pgfpathlineto{\pgfqpoint{2.072011in}{0.738869in}}%
\pgfpathlineto{\pgfqpoint{2.072309in}{0.738867in}}%
\pgfpathlineto{\pgfqpoint{2.072606in}{0.738865in}}%
\pgfpathlineto{\pgfqpoint{2.072904in}{0.738863in}}%
\pgfpathlineto{\pgfqpoint{2.073201in}{0.738861in}}%
\pgfpathlineto{\pgfqpoint{2.073499in}{0.738860in}}%
\pgfpathlineto{\pgfqpoint{2.073796in}{0.738858in}}%
\pgfpathlineto{\pgfqpoint{2.074093in}{0.738856in}}%
\pgfpathlineto{\pgfqpoint{2.074391in}{0.738854in}}%
\pgfpathlineto{\pgfqpoint{2.074688in}{0.738852in}}%
\pgfpathlineto{\pgfqpoint{2.074986in}{0.738850in}}%
\pgfpathlineto{\pgfqpoint{2.075283in}{0.738849in}}%
\pgfpathlineto{\pgfqpoint{2.075581in}{0.738847in}}%
\pgfpathlineto{\pgfqpoint{2.075878in}{0.738845in}}%
\pgfpathlineto{\pgfqpoint{2.076176in}{0.738843in}}%
\pgfpathlineto{\pgfqpoint{2.076473in}{0.738841in}}%
\pgfpathlineto{\pgfqpoint{2.076771in}{0.738839in}}%
\pgfpathlineto{\pgfqpoint{2.077068in}{0.738837in}}%
\pgfpathlineto{\pgfqpoint{2.077366in}{0.738836in}}%
\pgfpathlineto{\pgfqpoint{2.077663in}{0.738834in}}%
\pgfpathlineto{\pgfqpoint{2.077961in}{0.738832in}}%
\pgfpathlineto{\pgfqpoint{2.078258in}{0.738830in}}%
\pgfpathlineto{\pgfqpoint{2.078556in}{0.738828in}}%
\pgfpathlineto{\pgfqpoint{2.078853in}{0.738826in}}%
\pgfpathlineto{\pgfqpoint{2.079151in}{0.738824in}}%
\pgfpathlineto{\pgfqpoint{2.079448in}{0.738823in}}%
\pgfpathlineto{\pgfqpoint{2.079746in}{0.738821in}}%
\pgfpathlineto{\pgfqpoint{2.080043in}{0.738819in}}%
\pgfpathlineto{\pgfqpoint{2.080341in}{0.738817in}}%
\pgfpathlineto{\pgfqpoint{2.080638in}{0.738815in}}%
\pgfpathlineto{\pgfqpoint{2.080935in}{0.738813in}}%
\pgfpathlineto{\pgfqpoint{2.081233in}{0.738811in}}%
\pgfpathlineto{\pgfqpoint{2.081530in}{0.738810in}}%
\pgfpathlineto{\pgfqpoint{2.081828in}{0.738808in}}%
\pgfpathlineto{\pgfqpoint{2.082125in}{0.738806in}}%
\pgfpathlineto{\pgfqpoint{2.082423in}{0.738804in}}%
\pgfpathlineto{\pgfqpoint{2.082720in}{0.738802in}}%
\pgfpathlineto{\pgfqpoint{2.083018in}{0.738800in}}%
\pgfpathlineto{\pgfqpoint{2.083315in}{0.738798in}}%
\pgfpathlineto{\pgfqpoint{2.083613in}{0.738797in}}%
\pgfpathlineto{\pgfqpoint{2.083910in}{0.738795in}}%
\pgfpathlineto{\pgfqpoint{2.084208in}{0.738793in}}%
\pgfpathlineto{\pgfqpoint{2.084505in}{0.738791in}}%
\pgfpathlineto{\pgfqpoint{2.084803in}{0.738789in}}%
\pgfpathlineto{\pgfqpoint{2.085100in}{0.738787in}}%
\pgfpathlineto{\pgfqpoint{2.085398in}{0.738785in}}%
\pgfpathlineto{\pgfqpoint{2.085695in}{0.738784in}}%
\pgfpathlineto{\pgfqpoint{2.085993in}{0.738782in}}%
\pgfpathlineto{\pgfqpoint{2.086290in}{0.738780in}}%
\pgfpathlineto{\pgfqpoint{2.086588in}{0.738778in}}%
\pgfpathlineto{\pgfqpoint{2.086885in}{0.738776in}}%
\pgfpathlineto{\pgfqpoint{2.087182in}{0.738774in}}%
\pgfpathlineto{\pgfqpoint{2.087480in}{0.738773in}}%
\pgfpathlineto{\pgfqpoint{2.087777in}{0.738771in}}%
\pgfpathlineto{\pgfqpoint{2.088075in}{0.738769in}}%
\pgfpathlineto{\pgfqpoint{2.088372in}{0.738767in}}%
\pgfpathlineto{\pgfqpoint{2.088670in}{0.738765in}}%
\pgfpathlineto{\pgfqpoint{2.088967in}{0.738763in}}%
\pgfpathlineto{\pgfqpoint{2.089265in}{0.738761in}}%
\pgfpathlineto{\pgfqpoint{2.089562in}{0.738760in}}%
\pgfpathlineto{\pgfqpoint{2.089860in}{0.738758in}}%
\pgfpathlineto{\pgfqpoint{2.090157in}{0.738756in}}%
\pgfpathlineto{\pgfqpoint{2.090455in}{0.738754in}}%
\pgfpathlineto{\pgfqpoint{2.090752in}{0.738752in}}%
\pgfpathlineto{\pgfqpoint{2.091050in}{0.738750in}}%
\pgfpathlineto{\pgfqpoint{2.091347in}{0.738748in}}%
\pgfpathlineto{\pgfqpoint{2.091645in}{0.738747in}}%
\pgfpathlineto{\pgfqpoint{2.091942in}{0.738745in}}%
\pgfpathlineto{\pgfqpoint{2.092240in}{0.738743in}}%
\pgfpathlineto{\pgfqpoint{2.092537in}{0.738741in}}%
\pgfpathlineto{\pgfqpoint{2.092835in}{0.738739in}}%
\pgfpathlineto{\pgfqpoint{2.093132in}{0.738737in}}%
\pgfpathlineto{\pgfqpoint{2.093430in}{0.738735in}}%
\pgfpathlineto{\pgfqpoint{2.093727in}{0.738734in}}%
\pgfpathlineto{\pgfqpoint{2.094024in}{0.738732in}}%
\pgfpathlineto{\pgfqpoint{2.094322in}{0.738730in}}%
\pgfpathlineto{\pgfqpoint{2.094619in}{0.738728in}}%
\pgfpathlineto{\pgfqpoint{2.094917in}{0.738726in}}%
\pgfpathlineto{\pgfqpoint{2.095214in}{0.738724in}}%
\pgfpathlineto{\pgfqpoint{2.095512in}{0.738722in}}%
\pgfpathlineto{\pgfqpoint{2.095809in}{0.738721in}}%
\pgfpathlineto{\pgfqpoint{2.096107in}{0.738719in}}%
\pgfpathlineto{\pgfqpoint{2.096404in}{0.738717in}}%
\pgfpathlineto{\pgfqpoint{2.096702in}{0.738715in}}%
\pgfpathlineto{\pgfqpoint{2.096999in}{0.738713in}}%
\pgfpathlineto{\pgfqpoint{2.097297in}{0.738711in}}%
\pgfpathlineto{\pgfqpoint{2.097594in}{0.738710in}}%
\pgfpathlineto{\pgfqpoint{2.097892in}{0.738708in}}%
\pgfpathlineto{\pgfqpoint{2.098189in}{0.738706in}}%
\pgfpathlineto{\pgfqpoint{2.098487in}{0.738704in}}%
\pgfpathlineto{\pgfqpoint{2.098784in}{0.738702in}}%
\pgfpathlineto{\pgfqpoint{2.099082in}{0.738700in}}%
\pgfpathlineto{\pgfqpoint{2.099379in}{0.738698in}}%
\pgfpathlineto{\pgfqpoint{2.099677in}{0.738697in}}%
\pgfpathlineto{\pgfqpoint{2.099974in}{0.738695in}}%
\pgfpathlineto{\pgfqpoint{2.100272in}{0.738693in}}%
\pgfpathlineto{\pgfqpoint{2.100569in}{0.738691in}}%
\pgfpathlineto{\pgfqpoint{2.100866in}{0.738689in}}%
\pgfpathlineto{\pgfqpoint{2.101164in}{0.738687in}}%
\pgfpathlineto{\pgfqpoint{2.101461in}{0.738685in}}%
\pgfpathlineto{\pgfqpoint{2.101759in}{0.738684in}}%
\pgfpathlineto{\pgfqpoint{2.102056in}{0.738682in}}%
\pgfpathlineto{\pgfqpoint{2.102354in}{0.738680in}}%
\pgfpathlineto{\pgfqpoint{2.102651in}{0.738678in}}%
\pgfpathlineto{\pgfqpoint{2.102949in}{0.738676in}}%
\pgfpathlineto{\pgfqpoint{2.103246in}{0.738674in}}%
\pgfpathlineto{\pgfqpoint{2.103544in}{0.738672in}}%
\pgfpathlineto{\pgfqpoint{2.103841in}{0.738671in}}%
\pgfpathlineto{\pgfqpoint{2.104139in}{0.738669in}}%
\pgfpathlineto{\pgfqpoint{2.104436in}{0.738667in}}%
\pgfpathlineto{\pgfqpoint{2.104734in}{0.738665in}}%
\pgfpathlineto{\pgfqpoint{2.105031in}{0.738663in}}%
\pgfpathlineto{\pgfqpoint{2.105329in}{0.738661in}}%
\pgfpathlineto{\pgfqpoint{2.105626in}{0.738659in}}%
\pgfpathlineto{\pgfqpoint{2.105924in}{0.738658in}}%
\pgfpathlineto{\pgfqpoint{2.106221in}{0.738656in}}%
\pgfpathlineto{\pgfqpoint{2.106519in}{0.738654in}}%
\pgfpathlineto{\pgfqpoint{2.106816in}{0.738652in}}%
\pgfpathlineto{\pgfqpoint{2.107113in}{0.738650in}}%
\pgfpathlineto{\pgfqpoint{2.107411in}{0.738648in}}%
\pgfpathlineto{\pgfqpoint{2.107708in}{0.738647in}}%
\pgfpathlineto{\pgfqpoint{2.108006in}{0.738645in}}%
\pgfpathlineto{\pgfqpoint{2.108303in}{0.738643in}}%
\pgfpathlineto{\pgfqpoint{2.108601in}{0.738641in}}%
\pgfpathlineto{\pgfqpoint{2.108898in}{0.738639in}}%
\pgfpathlineto{\pgfqpoint{2.109196in}{0.738637in}}%
\pgfpathlineto{\pgfqpoint{2.109493in}{0.738635in}}%
\pgfpathlineto{\pgfqpoint{2.109791in}{0.738634in}}%
\pgfpathlineto{\pgfqpoint{2.110088in}{0.738632in}}%
\pgfpathlineto{\pgfqpoint{2.110386in}{0.738630in}}%
\pgfpathlineto{\pgfqpoint{2.110683in}{0.738628in}}%
\pgfpathlineto{\pgfqpoint{2.110981in}{0.738626in}}%
\pgfpathlineto{\pgfqpoint{2.111278in}{0.738624in}}%
\pgfpathlineto{\pgfqpoint{2.111576in}{0.738622in}}%
\pgfpathlineto{\pgfqpoint{2.111873in}{0.738621in}}%
\pgfpathlineto{\pgfqpoint{2.112171in}{0.738619in}}%
\pgfpathlineto{\pgfqpoint{2.112468in}{0.738617in}}%
\pgfpathlineto{\pgfqpoint{2.112766in}{0.738615in}}%
\pgfpathlineto{\pgfqpoint{2.113063in}{0.738613in}}%
\pgfpathlineto{\pgfqpoint{2.113361in}{0.738611in}}%
\pgfpathlineto{\pgfqpoint{2.113658in}{0.738609in}}%
\pgfpathlineto{\pgfqpoint{2.113955in}{0.738608in}}%
\pgfpathlineto{\pgfqpoint{2.114253in}{0.738606in}}%
\pgfpathlineto{\pgfqpoint{2.114550in}{0.738604in}}%
\pgfpathlineto{\pgfqpoint{2.114848in}{0.738602in}}%
\pgfpathlineto{\pgfqpoint{2.115145in}{0.738600in}}%
\pgfpathlineto{\pgfqpoint{2.115443in}{0.738598in}}%
\pgfpathlineto{\pgfqpoint{2.115740in}{0.738596in}}%
\pgfpathlineto{\pgfqpoint{2.116038in}{0.738595in}}%
\pgfpathlineto{\pgfqpoint{2.116335in}{0.738593in}}%
\pgfpathlineto{\pgfqpoint{2.116633in}{0.738591in}}%
\pgfpathlineto{\pgfqpoint{2.116930in}{0.738589in}}%
\pgfpathlineto{\pgfqpoint{2.117228in}{0.738587in}}%
\pgfpathlineto{\pgfqpoint{2.117525in}{0.738585in}}%
\pgfpathlineto{\pgfqpoint{2.117823in}{0.738583in}}%
\pgfpathlineto{\pgfqpoint{2.118120in}{0.738582in}}%
\pgfpathlineto{\pgfqpoint{2.118418in}{0.738580in}}%
\pgfpathlineto{\pgfqpoint{2.118715in}{0.738578in}}%
\pgfpathlineto{\pgfqpoint{2.119013in}{0.738576in}}%
\pgfpathlineto{\pgfqpoint{2.119310in}{0.738574in}}%
\pgfpathlineto{\pgfqpoint{2.119608in}{0.738572in}}%
\pgfpathlineto{\pgfqpoint{2.119905in}{0.738571in}}%
\pgfpathlineto{\pgfqpoint{2.120203in}{0.738569in}}%
\pgfpathlineto{\pgfqpoint{2.120500in}{0.738567in}}%
\pgfpathlineto{\pgfqpoint{2.120797in}{0.738565in}}%
\pgfpathlineto{\pgfqpoint{2.121095in}{0.738563in}}%
\pgfpathlineto{\pgfqpoint{2.121392in}{0.738561in}}%
\pgfpathlineto{\pgfqpoint{2.121690in}{0.738559in}}%
\pgfpathlineto{\pgfqpoint{2.121987in}{0.738558in}}%
\pgfpathlineto{\pgfqpoint{2.122285in}{0.738556in}}%
\pgfpathlineto{\pgfqpoint{2.122582in}{0.738554in}}%
\pgfpathlineto{\pgfqpoint{2.122880in}{0.738552in}}%
\pgfpathlineto{\pgfqpoint{2.123177in}{0.738550in}}%
\pgfpathlineto{\pgfqpoint{2.123475in}{0.738548in}}%
\pgfpathlineto{\pgfqpoint{2.123772in}{0.738546in}}%
\pgfpathlineto{\pgfqpoint{2.124070in}{0.738545in}}%
\pgfpathlineto{\pgfqpoint{2.124367in}{0.738543in}}%
\pgfpathlineto{\pgfqpoint{2.124665in}{0.738541in}}%
\pgfpathlineto{\pgfqpoint{2.124962in}{0.738539in}}%
\pgfpathlineto{\pgfqpoint{2.125260in}{0.738537in}}%
\pgfpathlineto{\pgfqpoint{2.125557in}{0.738535in}}%
\pgfpathlineto{\pgfqpoint{2.125855in}{0.738533in}}%
\pgfpathlineto{\pgfqpoint{2.126152in}{0.738532in}}%
\pgfpathlineto{\pgfqpoint{2.126450in}{0.738530in}}%
\pgfpathlineto{\pgfqpoint{2.126747in}{0.738528in}}%
\pgfpathlineto{\pgfqpoint{2.127044in}{0.738526in}}%
\pgfpathlineto{\pgfqpoint{2.127342in}{0.738524in}}%
\pgfpathlineto{\pgfqpoint{2.127639in}{0.738522in}}%
\pgfpathlineto{\pgfqpoint{2.127937in}{0.738520in}}%
\pgfpathlineto{\pgfqpoint{2.128234in}{0.738519in}}%
\pgfpathlineto{\pgfqpoint{2.128532in}{0.738517in}}%
\pgfpathlineto{\pgfqpoint{2.128829in}{0.738515in}}%
\pgfpathlineto{\pgfqpoint{2.129127in}{0.738513in}}%
\pgfpathlineto{\pgfqpoint{2.129424in}{0.738511in}}%
\pgfpathlineto{\pgfqpoint{2.129722in}{0.738509in}}%
\pgfpathlineto{\pgfqpoint{2.130019in}{0.738508in}}%
\pgfpathlineto{\pgfqpoint{2.130317in}{0.738506in}}%
\pgfpathlineto{\pgfqpoint{2.130614in}{0.738504in}}%
\pgfpathlineto{\pgfqpoint{2.130912in}{0.738502in}}%
\pgfpathlineto{\pgfqpoint{2.131209in}{0.738500in}}%
\pgfpathlineto{\pgfqpoint{2.131507in}{0.738498in}}%
\pgfpathlineto{\pgfqpoint{2.131804in}{0.738496in}}%
\pgfpathlineto{\pgfqpoint{2.132102in}{0.738495in}}%
\pgfpathlineto{\pgfqpoint{2.132399in}{0.738493in}}%
\pgfpathlineto{\pgfqpoint{2.132697in}{0.738491in}}%
\pgfpathlineto{\pgfqpoint{2.132994in}{0.738489in}}%
\pgfpathlineto{\pgfqpoint{2.133292in}{0.738487in}}%
\pgfpathlineto{\pgfqpoint{2.133589in}{0.738485in}}%
\pgfpathlineto{\pgfqpoint{2.133886in}{0.738483in}}%
\pgfpathlineto{\pgfqpoint{2.134184in}{0.738482in}}%
\pgfpathlineto{\pgfqpoint{2.134481in}{0.738480in}}%
\pgfpathlineto{\pgfqpoint{2.134779in}{0.738478in}}%
\pgfpathlineto{\pgfqpoint{2.135076in}{0.738476in}}%
\pgfpathlineto{\pgfqpoint{2.135374in}{0.738474in}}%
\pgfpathlineto{\pgfqpoint{2.135671in}{0.738472in}}%
\pgfpathlineto{\pgfqpoint{2.135969in}{0.738470in}}%
\pgfpathlineto{\pgfqpoint{2.136266in}{0.738469in}}%
\pgfpathlineto{\pgfqpoint{2.136564in}{0.738467in}}%
\pgfpathlineto{\pgfqpoint{2.136861in}{0.738465in}}%
\pgfpathlineto{\pgfqpoint{2.137159in}{0.738463in}}%
\pgfpathlineto{\pgfqpoint{2.137456in}{0.738461in}}%
\pgfpathlineto{\pgfqpoint{2.137754in}{0.738459in}}%
\pgfpathlineto{\pgfqpoint{2.138051in}{0.738457in}}%
\pgfpathlineto{\pgfqpoint{2.138349in}{0.738456in}}%
\pgfpathlineto{\pgfqpoint{2.138646in}{0.738454in}}%
\pgfpathlineto{\pgfqpoint{2.138944in}{0.738452in}}%
\pgfpathlineto{\pgfqpoint{2.139241in}{0.738450in}}%
\pgfpathlineto{\pgfqpoint{2.139539in}{0.738448in}}%
\pgfpathlineto{\pgfqpoint{2.139836in}{0.738446in}}%
\pgfpathlineto{\pgfqpoint{2.140134in}{0.738445in}}%
\pgfpathlineto{\pgfqpoint{2.140431in}{0.738443in}}%
\pgfpathlineto{\pgfqpoint{2.140728in}{0.738441in}}%
\pgfpathlineto{\pgfqpoint{2.141026in}{0.738439in}}%
\pgfpathlineto{\pgfqpoint{2.141323in}{0.738437in}}%
\pgfpathlineto{\pgfqpoint{2.141621in}{0.738435in}}%
\pgfpathlineto{\pgfqpoint{2.141918in}{0.738433in}}%
\pgfpathlineto{\pgfqpoint{2.142216in}{0.738432in}}%
\pgfpathlineto{\pgfqpoint{2.142513in}{0.738430in}}%
\pgfpathlineto{\pgfqpoint{2.142811in}{0.738428in}}%
\pgfpathlineto{\pgfqpoint{2.143108in}{0.738426in}}%
\pgfpathlineto{\pgfqpoint{2.143406in}{0.738424in}}%
\pgfpathlineto{\pgfqpoint{2.143703in}{0.738422in}}%
\pgfpathlineto{\pgfqpoint{2.144001in}{0.738420in}}%
\pgfpathlineto{\pgfqpoint{2.144298in}{0.738419in}}%
\pgfpathlineto{\pgfqpoint{2.144596in}{0.738417in}}%
\pgfpathlineto{\pgfqpoint{2.144893in}{0.738415in}}%
\pgfpathlineto{\pgfqpoint{2.145191in}{0.738413in}}%
\pgfpathlineto{\pgfqpoint{2.145488in}{0.738411in}}%
\pgfpathlineto{\pgfqpoint{2.145786in}{0.738409in}}%
\pgfpathlineto{\pgfqpoint{2.146083in}{0.738407in}}%
\pgfpathlineto{\pgfqpoint{2.146381in}{0.738406in}}%
\pgfpathlineto{\pgfqpoint{2.146678in}{0.738404in}}%
\pgfpathlineto{\pgfqpoint{2.146975in}{0.738402in}}%
\pgfpathlineto{\pgfqpoint{2.147273in}{0.738400in}}%
\pgfpathlineto{\pgfqpoint{2.147570in}{0.738398in}}%
\pgfpathlineto{\pgfqpoint{2.147868in}{0.738396in}}%
\pgfpathlineto{\pgfqpoint{2.148165in}{0.738394in}}%
\pgfpathlineto{\pgfqpoint{2.148463in}{0.738393in}}%
\pgfpathlineto{\pgfqpoint{2.148760in}{0.738391in}}%
\pgfpathlineto{\pgfqpoint{2.149058in}{0.738389in}}%
\pgfpathlineto{\pgfqpoint{2.149355in}{0.738387in}}%
\pgfpathlineto{\pgfqpoint{2.149653in}{0.738385in}}%
\pgfpathlineto{\pgfqpoint{2.149950in}{0.738383in}}%
\pgfpathlineto{\pgfqpoint{2.150248in}{0.738381in}}%
\pgfpathlineto{\pgfqpoint{2.150545in}{0.738380in}}%
\pgfpathlineto{\pgfqpoint{2.150843in}{0.738378in}}%
\pgfpathlineto{\pgfqpoint{2.151140in}{0.738376in}}%
\pgfpathlineto{\pgfqpoint{2.151438in}{0.738374in}}%
\pgfpathlineto{\pgfqpoint{2.151735in}{0.738372in}}%
\pgfpathlineto{\pgfqpoint{2.152033in}{0.738370in}}%
\pgfpathlineto{\pgfqpoint{2.152330in}{0.738369in}}%
\pgfpathlineto{\pgfqpoint{2.152628in}{0.738367in}}%
\pgfpathlineto{\pgfqpoint{2.152925in}{0.738365in}}%
\pgfpathlineto{\pgfqpoint{2.153223in}{0.738363in}}%
\pgfpathlineto{\pgfqpoint{2.153520in}{0.738361in}}%
\pgfpathlineto{\pgfqpoint{2.153817in}{0.738359in}}%
\pgfpathlineto{\pgfqpoint{2.154115in}{0.738357in}}%
\pgfpathlineto{\pgfqpoint{2.154412in}{0.738356in}}%
\pgfpathlineto{\pgfqpoint{2.154710in}{0.738354in}}%
\pgfpathlineto{\pgfqpoint{2.155007in}{0.738352in}}%
\pgfpathlineto{\pgfqpoint{2.155305in}{0.738350in}}%
\pgfpathlineto{\pgfqpoint{2.155602in}{0.738348in}}%
\pgfpathlineto{\pgfqpoint{2.155900in}{0.738346in}}%
\pgfpathlineto{\pgfqpoint{2.156197in}{0.738344in}}%
\pgfpathlineto{\pgfqpoint{2.156495in}{0.738343in}}%
\pgfpathlineto{\pgfqpoint{2.156792in}{0.738341in}}%
\pgfpathlineto{\pgfqpoint{2.157090in}{0.738339in}}%
\pgfpathlineto{\pgfqpoint{2.157387in}{0.738337in}}%
\pgfpathlineto{\pgfqpoint{2.157685in}{0.738335in}}%
\pgfpathlineto{\pgfqpoint{2.157982in}{0.738333in}}%
\pgfpathlineto{\pgfqpoint{2.158280in}{0.738331in}}%
\pgfpathlineto{\pgfqpoint{2.158577in}{0.738330in}}%
\pgfpathlineto{\pgfqpoint{2.158875in}{0.738328in}}%
\pgfpathlineto{\pgfqpoint{2.159172in}{0.738326in}}%
\pgfpathlineto{\pgfqpoint{2.159470in}{0.738324in}}%
\pgfpathlineto{\pgfqpoint{2.159767in}{0.738322in}}%
\pgfpathlineto{\pgfqpoint{2.160065in}{0.738320in}}%
\pgfpathlineto{\pgfqpoint{2.160362in}{0.738318in}}%
\pgfpathlineto{\pgfqpoint{2.160659in}{0.738317in}}%
\pgfpathlineto{\pgfqpoint{2.160957in}{0.738315in}}%
\pgfpathlineto{\pgfqpoint{2.161254in}{0.738313in}}%
\pgfpathlineto{\pgfqpoint{2.161552in}{0.738311in}}%
\pgfpathlineto{\pgfqpoint{2.161849in}{0.738309in}}%
\pgfpathlineto{\pgfqpoint{2.162147in}{0.738307in}}%
\pgfpathlineto{\pgfqpoint{2.162444in}{0.738306in}}%
\pgfpathlineto{\pgfqpoint{2.162742in}{0.738304in}}%
\pgfpathlineto{\pgfqpoint{2.163039in}{0.738302in}}%
\pgfpathlineto{\pgfqpoint{2.163337in}{0.738300in}}%
\pgfpathlineto{\pgfqpoint{2.163634in}{0.738298in}}%
\pgfpathlineto{\pgfqpoint{2.163932in}{0.738296in}}%
\pgfpathlineto{\pgfqpoint{2.164229in}{0.738294in}}%
\pgfpathlineto{\pgfqpoint{2.164527in}{0.738293in}}%
\pgfpathlineto{\pgfqpoint{2.164824in}{0.738291in}}%
\pgfpathlineto{\pgfqpoint{2.165122in}{0.738289in}}%
\pgfpathlineto{\pgfqpoint{2.165419in}{0.738287in}}%
\pgfpathlineto{\pgfqpoint{2.165717in}{0.738285in}}%
\pgfpathlineto{\pgfqpoint{2.166014in}{0.738283in}}%
\pgfpathlineto{\pgfqpoint{2.166312in}{0.738281in}}%
\pgfpathlineto{\pgfqpoint{2.166609in}{0.738280in}}%
\pgfpathlineto{\pgfqpoint{2.166906in}{0.738278in}}%
\pgfpathlineto{\pgfqpoint{2.167204in}{0.738276in}}%
\pgfpathlineto{\pgfqpoint{2.167501in}{0.738274in}}%
\pgfpathlineto{\pgfqpoint{2.167799in}{0.738272in}}%
\pgfpathlineto{\pgfqpoint{2.168096in}{0.738270in}}%
\pgfpathlineto{\pgfqpoint{2.168394in}{0.738268in}}%
\pgfpathlineto{\pgfqpoint{2.168691in}{0.738267in}}%
\pgfpathlineto{\pgfqpoint{2.168989in}{0.738265in}}%
\pgfpathlineto{\pgfqpoint{2.169286in}{0.738263in}}%
\pgfpathlineto{\pgfqpoint{2.169584in}{0.738261in}}%
\pgfpathlineto{\pgfqpoint{2.169881in}{0.738259in}}%
\pgfpathlineto{\pgfqpoint{2.170179in}{0.738257in}}%
\pgfpathlineto{\pgfqpoint{2.170476in}{0.738255in}}%
\pgfpathlineto{\pgfqpoint{2.170774in}{0.738254in}}%
\pgfpathlineto{\pgfqpoint{2.171071in}{0.738252in}}%
\pgfpathlineto{\pgfqpoint{2.171369in}{0.738250in}}%
\pgfpathlineto{\pgfqpoint{2.171666in}{0.738248in}}%
\pgfpathlineto{\pgfqpoint{2.171964in}{0.738246in}}%
\pgfpathlineto{\pgfqpoint{2.172261in}{0.738244in}}%
\pgfpathlineto{\pgfqpoint{2.172559in}{0.738242in}}%
\pgfpathlineto{\pgfqpoint{2.172856in}{0.738241in}}%
\pgfpathlineto{\pgfqpoint{2.173154in}{0.738239in}}%
\pgfpathlineto{\pgfqpoint{2.173451in}{0.738237in}}%
\pgfpathlineto{\pgfqpoint{2.173748in}{0.738235in}}%
\pgfpathlineto{\pgfqpoint{2.174046in}{0.738233in}}%
\pgfpathlineto{\pgfqpoint{2.174343in}{0.738231in}}%
\pgfpathlineto{\pgfqpoint{2.174641in}{0.738230in}}%
\pgfpathlineto{\pgfqpoint{2.174938in}{0.738228in}}%
\pgfpathlineto{\pgfqpoint{2.175236in}{0.738226in}}%
\pgfpathlineto{\pgfqpoint{2.175533in}{0.738224in}}%
\pgfpathlineto{\pgfqpoint{2.175831in}{0.738222in}}%
\pgfpathlineto{\pgfqpoint{2.176128in}{0.738220in}}%
\pgfpathlineto{\pgfqpoint{2.176426in}{0.738218in}}%
\pgfpathlineto{\pgfqpoint{2.176723in}{0.738217in}}%
\pgfpathlineto{\pgfqpoint{2.177021in}{0.738215in}}%
\pgfpathlineto{\pgfqpoint{2.177318in}{0.738213in}}%
\pgfpathlineto{\pgfqpoint{2.177616in}{0.738211in}}%
\pgfpathlineto{\pgfqpoint{2.177913in}{0.738209in}}%
\pgfpathlineto{\pgfqpoint{2.178211in}{0.738207in}}%
\pgfpathlineto{\pgfqpoint{2.178508in}{0.738205in}}%
\pgfpathlineto{\pgfqpoint{2.178806in}{0.738204in}}%
\pgfpathlineto{\pgfqpoint{2.179103in}{0.738202in}}%
\pgfpathlineto{\pgfqpoint{2.179401in}{0.738200in}}%
\pgfpathlineto{\pgfqpoint{2.179698in}{0.738198in}}%
\pgfpathlineto{\pgfqpoint{2.179996in}{0.738196in}}%
\pgfpathlineto{\pgfqpoint{2.180293in}{0.738194in}}%
\pgfpathlineto{\pgfqpoint{2.180590in}{0.738192in}}%
\pgfpathlineto{\pgfqpoint{2.180888in}{0.738191in}}%
\pgfpathlineto{\pgfqpoint{2.181185in}{0.738189in}}%
\pgfpathlineto{\pgfqpoint{2.181483in}{0.738187in}}%
\pgfpathlineto{\pgfqpoint{2.181780in}{0.738185in}}%
\pgfpathlineto{\pgfqpoint{2.182078in}{0.738183in}}%
\pgfpathlineto{\pgfqpoint{2.182375in}{0.738181in}}%
\pgfpathlineto{\pgfqpoint{2.182673in}{0.738179in}}%
\pgfpathlineto{\pgfqpoint{2.182970in}{0.738178in}}%
\pgfpathlineto{\pgfqpoint{2.183268in}{0.738176in}}%
\pgfpathlineto{\pgfqpoint{2.183565in}{0.738174in}}%
\pgfpathlineto{\pgfqpoint{2.183863in}{0.738172in}}%
\pgfpathlineto{\pgfqpoint{2.184160in}{0.738170in}}%
\pgfpathlineto{\pgfqpoint{2.184458in}{0.738168in}}%
\pgfpathlineto{\pgfqpoint{2.184755in}{0.738167in}}%
\pgfpathlineto{\pgfqpoint{2.185053in}{0.738165in}}%
\pgfpathlineto{\pgfqpoint{2.185350in}{0.738163in}}%
\pgfpathlineto{\pgfqpoint{2.185648in}{0.738161in}}%
\pgfpathlineto{\pgfqpoint{2.185945in}{0.738159in}}%
\pgfpathlineto{\pgfqpoint{2.186243in}{0.738157in}}%
\pgfpathlineto{\pgfqpoint{2.186540in}{0.738155in}}%
\pgfpathlineto{\pgfqpoint{2.186837in}{0.738154in}}%
\pgfpathlineto{\pgfqpoint{2.187135in}{0.738152in}}%
\pgfpathlineto{\pgfqpoint{2.187432in}{0.738150in}}%
\pgfpathlineto{\pgfqpoint{2.187730in}{0.738148in}}%
\pgfpathlineto{\pgfqpoint{2.188027in}{0.738146in}}%
\pgfpathlineto{\pgfqpoint{2.188325in}{0.738144in}}%
\pgfpathlineto{\pgfqpoint{2.188622in}{0.738142in}}%
\pgfpathlineto{\pgfqpoint{2.188920in}{0.738141in}}%
\pgfpathlineto{\pgfqpoint{2.189217in}{0.738139in}}%
\pgfpathlineto{\pgfqpoint{2.189515in}{0.738137in}}%
\pgfpathlineto{\pgfqpoint{2.189812in}{0.738135in}}%
\pgfpathlineto{\pgfqpoint{2.190110in}{0.738133in}}%
\pgfpathlineto{\pgfqpoint{2.190407in}{0.738131in}}%
\pgfpathlineto{\pgfqpoint{2.190705in}{0.738129in}}%
\pgfpathlineto{\pgfqpoint{2.191002in}{0.738128in}}%
\pgfpathlineto{\pgfqpoint{2.191300in}{0.738126in}}%
\pgfpathlineto{\pgfqpoint{2.191597in}{0.738124in}}%
\pgfpathlineto{\pgfqpoint{2.191895in}{0.738122in}}%
\pgfpathlineto{\pgfqpoint{2.192192in}{0.738120in}}%
\pgfpathlineto{\pgfqpoint{2.192490in}{0.738118in}}%
\pgfpathlineto{\pgfqpoint{2.192787in}{0.738116in}}%
\pgfpathlineto{\pgfqpoint{2.193085in}{0.738115in}}%
\pgfpathlineto{\pgfqpoint{2.193382in}{0.738113in}}%
\pgfpathlineto{\pgfqpoint{2.193679in}{0.738111in}}%
\pgfpathlineto{\pgfqpoint{2.193977in}{0.738109in}}%
\pgfpathlineto{\pgfqpoint{2.194274in}{0.738107in}}%
\pgfpathlineto{\pgfqpoint{2.194572in}{0.738105in}}%
\pgfpathlineto{\pgfqpoint{2.194869in}{0.738104in}}%
\pgfpathlineto{\pgfqpoint{2.195167in}{0.738102in}}%
\pgfpathlineto{\pgfqpoint{2.195464in}{0.738100in}}%
\pgfpathlineto{\pgfqpoint{2.195762in}{0.738098in}}%
\pgfpathlineto{\pgfqpoint{2.196059in}{0.738096in}}%
\pgfpathlineto{\pgfqpoint{2.196357in}{0.738094in}}%
\pgfpathlineto{\pgfqpoint{2.196654in}{0.738092in}}%
\pgfpathlineto{\pgfqpoint{2.196952in}{0.738091in}}%
\pgfpathlineto{\pgfqpoint{2.197249in}{0.738089in}}%
\pgfpathlineto{\pgfqpoint{2.197547in}{0.738087in}}%
\pgfpathlineto{\pgfqpoint{2.197844in}{0.738085in}}%
\pgfpathlineto{\pgfqpoint{2.198142in}{0.738083in}}%
\pgfpathlineto{\pgfqpoint{2.198439in}{0.738081in}}%
\pgfpathlineto{\pgfqpoint{2.198737in}{0.738079in}}%
\pgfpathlineto{\pgfqpoint{2.199034in}{0.738078in}}%
\pgfpathlineto{\pgfqpoint{2.199332in}{0.738076in}}%
\pgfpathlineto{\pgfqpoint{2.199629in}{0.738074in}}%
\pgfpathlineto{\pgfqpoint{2.199927in}{0.738072in}}%
\pgfpathlineto{\pgfqpoint{2.200224in}{0.738070in}}%
\pgfpathlineto{\pgfqpoint{2.200521in}{0.738068in}}%
\pgfpathlineto{\pgfqpoint{2.200819in}{0.738066in}}%
\pgfpathlineto{\pgfqpoint{2.201116in}{0.738065in}}%
\pgfpathlineto{\pgfqpoint{2.201414in}{0.738063in}}%
\pgfpathlineto{\pgfqpoint{2.201711in}{0.738061in}}%
\pgfpathlineto{\pgfqpoint{2.202009in}{0.738059in}}%
\pgfpathlineto{\pgfqpoint{2.202306in}{0.738057in}}%
\pgfpathlineto{\pgfqpoint{2.202604in}{0.738055in}}%
\pgfpathlineto{\pgfqpoint{2.202901in}{0.738053in}}%
\pgfpathlineto{\pgfqpoint{2.203199in}{0.738052in}}%
\pgfpathlineto{\pgfqpoint{2.203496in}{0.738050in}}%
\pgfpathlineto{\pgfqpoint{2.203794in}{0.738048in}}%
\pgfpathlineto{\pgfqpoint{2.204091in}{0.738046in}}%
\pgfpathlineto{\pgfqpoint{2.204389in}{0.738044in}}%
\pgfpathlineto{\pgfqpoint{2.204686in}{0.738042in}}%
\pgfpathlineto{\pgfqpoint{2.204984in}{0.738040in}}%
\pgfpathlineto{\pgfqpoint{2.205281in}{0.738039in}}%
\pgfpathlineto{\pgfqpoint{2.205579in}{0.738037in}}%
\pgfpathlineto{\pgfqpoint{2.205876in}{0.738035in}}%
\pgfpathlineto{\pgfqpoint{2.206174in}{0.738033in}}%
\pgfpathlineto{\pgfqpoint{2.206471in}{0.738031in}}%
\pgfpathlineto{\pgfqpoint{2.206768in}{0.738029in}}%
\pgfpathlineto{\pgfqpoint{2.207066in}{0.738028in}}%
\pgfpathlineto{\pgfqpoint{2.207363in}{0.738026in}}%
\pgfpathlineto{\pgfqpoint{2.207661in}{0.738024in}}%
\pgfpathlineto{\pgfqpoint{2.207958in}{0.738022in}}%
\pgfpathlineto{\pgfqpoint{2.208256in}{0.738020in}}%
\pgfpathlineto{\pgfqpoint{2.208553in}{0.738018in}}%
\pgfpathlineto{\pgfqpoint{2.208851in}{0.738016in}}%
\pgfpathlineto{\pgfqpoint{2.209148in}{0.738015in}}%
\pgfpathlineto{\pgfqpoint{2.209446in}{0.738013in}}%
\pgfpathlineto{\pgfqpoint{2.209743in}{0.738011in}}%
\pgfpathlineto{\pgfqpoint{2.210041in}{0.738009in}}%
\pgfpathlineto{\pgfqpoint{2.210338in}{0.738007in}}%
\pgfpathlineto{\pgfqpoint{2.210636in}{0.738005in}}%
\pgfpathlineto{\pgfqpoint{2.210933in}{0.738003in}}%
\pgfpathlineto{\pgfqpoint{2.211231in}{0.738002in}}%
\pgfpathlineto{\pgfqpoint{2.211528in}{0.738000in}}%
\pgfpathlineto{\pgfqpoint{2.211826in}{0.737998in}}%
\pgfpathlineto{\pgfqpoint{2.212123in}{0.737996in}}%
\pgfpathlineto{\pgfqpoint{2.212421in}{0.737994in}}%
\pgfpathlineto{\pgfqpoint{2.212718in}{0.737992in}}%
\pgfpathlineto{\pgfqpoint{2.213016in}{0.737990in}}%
\pgfpathlineto{\pgfqpoint{2.213313in}{0.737989in}}%
\pgfpathlineto{\pgfqpoint{2.213610in}{0.737987in}}%
\pgfpathlineto{\pgfqpoint{2.213908in}{0.737985in}}%
\pgfpathlineto{\pgfqpoint{2.214205in}{0.737983in}}%
\pgfpathlineto{\pgfqpoint{2.214503in}{0.737981in}}%
\pgfpathlineto{\pgfqpoint{2.214800in}{0.737979in}}%
\pgfpathlineto{\pgfqpoint{2.215098in}{0.737977in}}%
\pgfpathlineto{\pgfqpoint{2.215395in}{0.737976in}}%
\pgfpathlineto{\pgfqpoint{2.215693in}{0.737974in}}%
\pgfpathlineto{\pgfqpoint{2.215990in}{0.737972in}}%
\pgfpathlineto{\pgfqpoint{2.216288in}{0.737970in}}%
\pgfpathlineto{\pgfqpoint{2.216585in}{0.737968in}}%
\pgfpathlineto{\pgfqpoint{2.216883in}{0.737966in}}%
\pgfpathlineto{\pgfqpoint{2.217180in}{0.737964in}}%
\pgfpathlineto{\pgfqpoint{2.217478in}{0.737963in}}%
\pgfpathlineto{\pgfqpoint{2.217775in}{0.737961in}}%
\pgfpathlineto{\pgfqpoint{2.218073in}{0.737959in}}%
\pgfpathlineto{\pgfqpoint{2.218370in}{0.737957in}}%
\pgfpathlineto{\pgfqpoint{2.218668in}{0.737955in}}%
\pgfpathlineto{\pgfqpoint{2.218965in}{0.737953in}}%
\pgfpathlineto{\pgfqpoint{2.219263in}{0.737951in}}%
\pgfpathlineto{\pgfqpoint{2.219560in}{0.737950in}}%
\pgfpathlineto{\pgfqpoint{2.219858in}{0.737948in}}%
\pgfpathlineto{\pgfqpoint{2.220155in}{0.737946in}}%
\pgfpathlineto{\pgfqpoint{2.220452in}{0.737944in}}%
\pgfpathlineto{\pgfqpoint{2.220750in}{0.737942in}}%
\pgfpathlineto{\pgfqpoint{2.221047in}{0.737940in}}%
\pgfpathlineto{\pgfqpoint{2.221345in}{0.737938in}}%
\pgfpathlineto{\pgfqpoint{2.221642in}{0.737937in}}%
\pgfpathlineto{\pgfqpoint{2.221940in}{0.737935in}}%
\pgfpathlineto{\pgfqpoint{2.222237in}{0.737933in}}%
\pgfpathlineto{\pgfqpoint{2.222535in}{0.737931in}}%
\pgfpathlineto{\pgfqpoint{2.222832in}{0.737929in}}%
\pgfpathlineto{\pgfqpoint{2.223130in}{0.737927in}}%
\pgfpathlineto{\pgfqpoint{2.223427in}{0.737925in}}%
\pgfpathlineto{\pgfqpoint{2.223725in}{0.737924in}}%
\pgfpathlineto{\pgfqpoint{2.224022in}{0.737922in}}%
\pgfpathlineto{\pgfqpoint{2.224320in}{0.737920in}}%
\pgfpathlineto{\pgfqpoint{2.224617in}{0.737918in}}%
\pgfpathlineto{\pgfqpoint{2.224915in}{0.737916in}}%
\pgfpathlineto{\pgfqpoint{2.225212in}{0.737914in}}%
\pgfpathlineto{\pgfqpoint{2.225510in}{0.737912in}}%
\pgfpathlineto{\pgfqpoint{2.225807in}{0.737911in}}%
\pgfpathlineto{\pgfqpoint{2.226105in}{0.737909in}}%
\pgfpathlineto{\pgfqpoint{2.226402in}{0.737907in}}%
\pgfpathlineto{\pgfqpoint{2.226699in}{0.737905in}}%
\pgfpathlineto{\pgfqpoint{2.226997in}{0.737903in}}%
\pgfpathlineto{\pgfqpoint{2.227294in}{0.737901in}}%
\pgfpathlineto{\pgfqpoint{2.227592in}{0.737899in}}%
\pgfpathlineto{\pgfqpoint{2.227889in}{0.737898in}}%
\pgfpathlineto{\pgfqpoint{2.228187in}{0.737896in}}%
\pgfpathlineto{\pgfqpoint{2.228484in}{0.737894in}}%
\pgfpathlineto{\pgfqpoint{2.228782in}{0.737892in}}%
\pgfpathlineto{\pgfqpoint{2.229079in}{0.737890in}}%
\pgfpathlineto{\pgfqpoint{2.229377in}{0.737888in}}%
\pgfpathlineto{\pgfqpoint{2.229674in}{0.737886in}}%
\pgfpathlineto{\pgfqpoint{2.229972in}{0.737885in}}%
\pgfpathlineto{\pgfqpoint{2.230269in}{0.737883in}}%
\pgfpathlineto{\pgfqpoint{2.230567in}{0.737881in}}%
\pgfpathlineto{\pgfqpoint{2.230864in}{0.737879in}}%
\pgfpathlineto{\pgfqpoint{2.231162in}{0.737877in}}%
\pgfpathlineto{\pgfqpoint{2.231459in}{0.737875in}}%
\pgfpathlineto{\pgfqpoint{2.231757in}{0.737874in}}%
\pgfpathlineto{\pgfqpoint{2.232054in}{0.737872in}}%
\pgfpathlineto{\pgfqpoint{2.232352in}{0.737870in}}%
\pgfpathlineto{\pgfqpoint{2.232649in}{0.737868in}}%
\pgfpathlineto{\pgfqpoint{2.232947in}{0.737866in}}%
\pgfpathlineto{\pgfqpoint{2.233244in}{0.737864in}}%
\pgfpathlineto{\pgfqpoint{2.233541in}{0.737862in}}%
\pgfpathlineto{\pgfqpoint{2.233839in}{0.737861in}}%
\pgfpathlineto{\pgfqpoint{2.234136in}{0.737859in}}%
\pgfpathlineto{\pgfqpoint{2.234434in}{0.737857in}}%
\pgfpathlineto{\pgfqpoint{2.234731in}{0.737855in}}%
\pgfpathlineto{\pgfqpoint{2.235029in}{0.737853in}}%
\pgfpathlineto{\pgfqpoint{2.235326in}{0.737851in}}%
\pgfpathlineto{\pgfqpoint{2.235624in}{0.737849in}}%
\pgfpathlineto{\pgfqpoint{2.235921in}{0.737848in}}%
\pgfpathlineto{\pgfqpoint{2.236219in}{0.737846in}}%
\pgfpathlineto{\pgfqpoint{2.236516in}{0.737844in}}%
\pgfpathlineto{\pgfqpoint{2.236814in}{0.737842in}}%
\pgfpathlineto{\pgfqpoint{2.237111in}{0.737840in}}%
\pgfpathlineto{\pgfqpoint{2.237409in}{0.737838in}}%
\pgfpathlineto{\pgfqpoint{2.237706in}{0.737836in}}%
\pgfpathlineto{\pgfqpoint{2.238004in}{0.737835in}}%
\pgfpathlineto{\pgfqpoint{2.238301in}{0.737833in}}%
\pgfpathlineto{\pgfqpoint{2.238599in}{0.737831in}}%
\pgfpathlineto{\pgfqpoint{2.238896in}{0.737829in}}%
\pgfpathlineto{\pgfqpoint{2.239194in}{0.737827in}}%
\pgfpathlineto{\pgfqpoint{2.239491in}{0.737825in}}%
\pgfpathlineto{\pgfqpoint{2.239789in}{0.737823in}}%
\pgfpathlineto{\pgfqpoint{2.240086in}{0.737822in}}%
\pgfpathlineto{\pgfqpoint{2.240383in}{0.737820in}}%
\pgfpathlineto{\pgfqpoint{2.240681in}{0.737818in}}%
\pgfpathlineto{\pgfqpoint{2.240978in}{0.737816in}}%
\pgfpathlineto{\pgfqpoint{2.241276in}{0.737814in}}%
\pgfpathlineto{\pgfqpoint{2.241573in}{0.737812in}}%
\pgfpathlineto{\pgfqpoint{2.241871in}{0.737810in}}%
\pgfpathlineto{\pgfqpoint{2.242168in}{0.737811in}}%
\pgfpathlineto{\pgfqpoint{2.242466in}{0.737839in}}%
\pgfpathlineto{\pgfqpoint{2.242763in}{0.737837in}}%
\pgfpathlineto{\pgfqpoint{2.243061in}{0.737834in}}%
\pgfpathlineto{\pgfqpoint{2.243358in}{0.737830in}}%
\pgfpathlineto{\pgfqpoint{2.243656in}{0.737827in}}%
\pgfpathlineto{\pgfqpoint{2.243953in}{0.737824in}}%
\pgfpathlineto{\pgfqpoint{2.244251in}{0.737821in}}%
\pgfpathlineto{\pgfqpoint{2.244548in}{0.737818in}}%
\pgfpathlineto{\pgfqpoint{2.244846in}{0.737815in}}%
\pgfpathlineto{\pgfqpoint{2.245143in}{0.737812in}}%
\pgfpathlineto{\pgfqpoint{2.245441in}{0.737809in}}%
\pgfpathlineto{\pgfqpoint{2.245738in}{0.737806in}}%
\pgfpathlineto{\pgfqpoint{2.246036in}{0.737802in}}%
\pgfpathlineto{\pgfqpoint{2.246333in}{0.737799in}}%
\pgfpathlineto{\pgfqpoint{2.246631in}{0.737796in}}%
\pgfpathlineto{\pgfqpoint{2.246928in}{0.737793in}}%
\pgfpathlineto{\pgfqpoint{2.247225in}{0.737790in}}%
\pgfpathlineto{\pgfqpoint{2.247523in}{0.737787in}}%
\pgfpathlineto{\pgfqpoint{2.247820in}{0.737784in}}%
\pgfpathlineto{\pgfqpoint{2.248118in}{0.737781in}}%
\pgfpathlineto{\pgfqpoint{2.248415in}{0.737778in}}%
\pgfpathlineto{\pgfqpoint{2.248713in}{0.737774in}}%
\pgfpathlineto{\pgfqpoint{2.249010in}{0.737771in}}%
\pgfpathlineto{\pgfqpoint{2.249308in}{0.737768in}}%
\pgfpathlineto{\pgfqpoint{2.249605in}{0.737765in}}%
\pgfpathlineto{\pgfqpoint{2.249903in}{0.737762in}}%
\pgfpathlineto{\pgfqpoint{2.250200in}{0.737759in}}%
\pgfpathlineto{\pgfqpoint{2.250498in}{0.737756in}}%
\pgfpathlineto{\pgfqpoint{2.250795in}{0.737753in}}%
\pgfpathlineto{\pgfqpoint{2.251093in}{0.737749in}}%
\pgfpathlineto{\pgfqpoint{2.251390in}{0.737746in}}%
\pgfpathlineto{\pgfqpoint{2.251688in}{0.737743in}}%
\pgfpathlineto{\pgfqpoint{2.251985in}{0.737740in}}%
\pgfpathlineto{\pgfqpoint{2.252283in}{0.737737in}}%
\pgfpathlineto{\pgfqpoint{2.252580in}{0.737734in}}%
\pgfpathlineto{\pgfqpoint{2.252878in}{0.737731in}}%
\pgfpathlineto{\pgfqpoint{2.253175in}{0.737728in}}%
\pgfpathlineto{\pgfqpoint{2.253472in}{0.737725in}}%
\pgfpathlineto{\pgfqpoint{2.253770in}{0.737721in}}%
\pgfpathlineto{\pgfqpoint{2.254067in}{0.737718in}}%
\pgfpathlineto{\pgfqpoint{2.254365in}{0.737715in}}%
\pgfpathlineto{\pgfqpoint{2.254662in}{0.737712in}}%
\pgfpathlineto{\pgfqpoint{2.254960in}{0.737709in}}%
\pgfpathlineto{\pgfqpoint{2.255257in}{0.737706in}}%
\pgfpathlineto{\pgfqpoint{2.255555in}{0.737703in}}%
\pgfpathlineto{\pgfqpoint{2.255852in}{0.737700in}}%
\pgfpathlineto{\pgfqpoint{2.256150in}{0.737696in}}%
\pgfpathlineto{\pgfqpoint{2.256447in}{0.737693in}}%
\pgfpathlineto{\pgfqpoint{2.256745in}{0.737690in}}%
\pgfpathlineto{\pgfqpoint{2.257042in}{0.737687in}}%
\pgfpathlineto{\pgfqpoint{2.257340in}{0.737684in}}%
\pgfpathlineto{\pgfqpoint{2.257637in}{0.737681in}}%
\pgfpathlineto{\pgfqpoint{2.257935in}{0.737678in}}%
\pgfpathlineto{\pgfqpoint{2.258232in}{0.737675in}}%
\pgfpathlineto{\pgfqpoint{2.258530in}{0.737672in}}%
\pgfpathlineto{\pgfqpoint{2.258827in}{0.737668in}}%
\pgfpathlineto{\pgfqpoint{2.259125in}{0.737665in}}%
\pgfpathlineto{\pgfqpoint{2.259422in}{0.737662in}}%
\pgfpathlineto{\pgfqpoint{2.259720in}{0.737659in}}%
\pgfpathlineto{\pgfqpoint{2.260017in}{0.737656in}}%
\pgfpathlineto{\pgfqpoint{2.260314in}{0.737653in}}%
\pgfpathlineto{\pgfqpoint{2.260612in}{0.737650in}}%
\pgfpathlineto{\pgfqpoint{2.260909in}{0.737647in}}%
\pgfpathlineto{\pgfqpoint{2.261207in}{0.737644in}}%
\pgfpathlineto{\pgfqpoint{2.261504in}{0.737640in}}%
\pgfpathlineto{\pgfqpoint{2.261802in}{0.737637in}}%
\pgfpathlineto{\pgfqpoint{2.262099in}{0.737634in}}%
\pgfpathlineto{\pgfqpoint{2.262397in}{0.737631in}}%
\pgfpathlineto{\pgfqpoint{2.262694in}{0.737628in}}%
\pgfpathlineto{\pgfqpoint{2.262992in}{0.737625in}}%
\pgfpathlineto{\pgfqpoint{2.263289in}{0.737622in}}%
\pgfpathlineto{\pgfqpoint{2.263587in}{0.737619in}}%
\pgfpathlineto{\pgfqpoint{2.263884in}{0.737615in}}%
\pgfpathlineto{\pgfqpoint{2.264182in}{0.737612in}}%
\pgfpathlineto{\pgfqpoint{2.264479in}{0.737609in}}%
\pgfpathlineto{\pgfqpoint{2.264777in}{0.737606in}}%
\pgfpathlineto{\pgfqpoint{2.265074in}{0.737603in}}%
\pgfpathlineto{\pgfqpoint{2.265372in}{0.737600in}}%
\pgfpathlineto{\pgfqpoint{2.265669in}{0.737597in}}%
\pgfpathlineto{\pgfqpoint{2.265967in}{0.737594in}}%
\pgfpathlineto{\pgfqpoint{2.266264in}{0.737591in}}%
\pgfpathlineto{\pgfqpoint{2.266562in}{0.737587in}}%
\pgfpathlineto{\pgfqpoint{2.266859in}{0.737584in}}%
\pgfpathlineto{\pgfqpoint{2.267156in}{0.737581in}}%
\pgfpathlineto{\pgfqpoint{2.267454in}{0.737578in}}%
\pgfpathlineto{\pgfqpoint{2.267751in}{0.737575in}}%
\pgfpathlineto{\pgfqpoint{2.268049in}{0.737572in}}%
\pgfpathlineto{\pgfqpoint{2.268346in}{0.737569in}}%
\pgfpathlineto{\pgfqpoint{2.268644in}{0.737566in}}%
\pgfpathlineto{\pgfqpoint{2.268941in}{0.737563in}}%
\pgfpathlineto{\pgfqpoint{2.269239in}{0.737559in}}%
\pgfpathlineto{\pgfqpoint{2.269536in}{0.737556in}}%
\pgfpathlineto{\pgfqpoint{2.269834in}{0.737553in}}%
\pgfpathlineto{\pgfqpoint{2.270131in}{0.737550in}}%
\pgfpathlineto{\pgfqpoint{2.270429in}{0.737547in}}%
\pgfpathlineto{\pgfqpoint{2.270726in}{0.737544in}}%
\pgfpathlineto{\pgfqpoint{2.271024in}{0.737541in}}%
\pgfpathlineto{\pgfqpoint{2.271321in}{0.737538in}}%
\pgfpathlineto{\pgfqpoint{2.271619in}{0.737534in}}%
\pgfpathlineto{\pgfqpoint{2.271916in}{0.737531in}}%
\pgfpathlineto{\pgfqpoint{2.272214in}{0.737528in}}%
\pgfpathlineto{\pgfqpoint{2.272511in}{0.737525in}}%
\pgfpathlineto{\pgfqpoint{2.272809in}{0.737522in}}%
\pgfpathlineto{\pgfqpoint{2.273106in}{0.737519in}}%
\pgfpathlineto{\pgfqpoint{2.273403in}{0.737516in}}%
\pgfpathlineto{\pgfqpoint{2.273701in}{0.737513in}}%
\pgfpathlineto{\pgfqpoint{2.273998in}{0.737510in}}%
\pgfpathlineto{\pgfqpoint{2.274296in}{0.737506in}}%
\pgfpathlineto{\pgfqpoint{2.274593in}{0.737503in}}%
\pgfpathlineto{\pgfqpoint{2.274891in}{0.737500in}}%
\pgfpathlineto{\pgfqpoint{2.275188in}{0.737497in}}%
\pgfpathlineto{\pgfqpoint{2.275486in}{0.737494in}}%
\pgfpathlineto{\pgfqpoint{2.275783in}{0.737491in}}%
\pgfpathlineto{\pgfqpoint{2.276081in}{0.737488in}}%
\pgfpathlineto{\pgfqpoint{2.276378in}{0.737485in}}%
\pgfpathlineto{\pgfqpoint{2.276676in}{0.737482in}}%
\pgfpathlineto{\pgfqpoint{2.276973in}{0.737478in}}%
\pgfpathlineto{\pgfqpoint{2.277271in}{0.737475in}}%
\pgfpathlineto{\pgfqpoint{2.277568in}{0.737472in}}%
\pgfpathlineto{\pgfqpoint{2.277866in}{0.737469in}}%
\pgfpathlineto{\pgfqpoint{2.278163in}{0.737466in}}%
\pgfpathlineto{\pgfqpoint{2.278461in}{0.737463in}}%
\pgfpathlineto{\pgfqpoint{2.278758in}{0.737460in}}%
\pgfpathlineto{\pgfqpoint{2.279056in}{0.737457in}}%
\pgfpathlineto{\pgfqpoint{2.279353in}{0.737453in}}%
\pgfpathlineto{\pgfqpoint{2.279651in}{0.737450in}}%
\pgfpathlineto{\pgfqpoint{2.279948in}{0.737447in}}%
\pgfpathlineto{\pgfqpoint{2.280245in}{0.737444in}}%
\pgfpathlineto{\pgfqpoint{2.280543in}{0.737441in}}%
\pgfpathlineto{\pgfqpoint{2.280840in}{0.737438in}}%
\pgfpathlineto{\pgfqpoint{2.281138in}{0.737435in}}%
\pgfpathlineto{\pgfqpoint{2.281435in}{0.737432in}}%
\pgfpathlineto{\pgfqpoint{2.281733in}{0.737429in}}%
\pgfpathlineto{\pgfqpoint{2.282030in}{0.737425in}}%
\pgfpathlineto{\pgfqpoint{2.282328in}{0.737422in}}%
\pgfpathlineto{\pgfqpoint{2.282625in}{0.737419in}}%
\pgfpathlineto{\pgfqpoint{2.282923in}{0.737416in}}%
\pgfpathlineto{\pgfqpoint{2.283220in}{0.737413in}}%
\pgfpathlineto{\pgfqpoint{2.283518in}{0.737410in}}%
\pgfpathlineto{\pgfqpoint{2.283815in}{0.737407in}}%
\pgfpathlineto{\pgfqpoint{2.284113in}{0.737404in}}%
\pgfpathlineto{\pgfqpoint{2.284410in}{0.737401in}}%
\pgfpathlineto{\pgfqpoint{2.284708in}{0.737397in}}%
\pgfpathlineto{\pgfqpoint{2.285005in}{0.737394in}}%
\pgfpathlineto{\pgfqpoint{2.285303in}{0.737391in}}%
\pgfpathlineto{\pgfqpoint{2.285600in}{0.737388in}}%
\pgfpathlineto{\pgfqpoint{2.285898in}{0.737385in}}%
\pgfpathlineto{\pgfqpoint{2.286195in}{0.737382in}}%
\pgfpathlineto{\pgfqpoint{2.286493in}{0.737379in}}%
\pgfpathlineto{\pgfqpoint{2.286790in}{0.737376in}}%
\pgfpathlineto{\pgfqpoint{2.287087in}{0.737372in}}%
\pgfpathlineto{\pgfqpoint{2.287385in}{0.737369in}}%
\pgfpathlineto{\pgfqpoint{2.287682in}{0.737366in}}%
\pgfpathlineto{\pgfqpoint{2.287980in}{0.737363in}}%
\pgfpathlineto{\pgfqpoint{2.288277in}{0.737360in}}%
\pgfpathlineto{\pgfqpoint{2.288575in}{0.737357in}}%
\pgfpathlineto{\pgfqpoint{2.288872in}{0.737354in}}%
\pgfpathlineto{\pgfqpoint{2.289170in}{0.737351in}}%
\pgfpathlineto{\pgfqpoint{2.289467in}{0.737348in}}%
\pgfpathlineto{\pgfqpoint{2.289765in}{0.737344in}}%
\pgfpathlineto{\pgfqpoint{2.290062in}{0.737341in}}%
\pgfpathlineto{\pgfqpoint{2.290360in}{0.737338in}}%
\pgfpathlineto{\pgfqpoint{2.290657in}{0.737335in}}%
\pgfpathlineto{\pgfqpoint{2.290955in}{0.737332in}}%
\pgfpathlineto{\pgfqpoint{2.291252in}{0.737329in}}%
\pgfpathlineto{\pgfqpoint{2.291550in}{0.737326in}}%
\pgfpathlineto{\pgfqpoint{2.291847in}{0.737323in}}%
\pgfpathlineto{\pgfqpoint{2.292145in}{0.737320in}}%
\pgfpathlineto{\pgfqpoint{2.292442in}{0.737316in}}%
\pgfpathlineto{\pgfqpoint{2.292740in}{0.737313in}}%
\pgfpathlineto{\pgfqpoint{2.293037in}{0.737310in}}%
\pgfpathlineto{\pgfqpoint{2.293334in}{0.737307in}}%
\pgfpathlineto{\pgfqpoint{2.293632in}{0.737304in}}%
\pgfpathlineto{\pgfqpoint{2.293929in}{0.737301in}}%
\pgfpathlineto{\pgfqpoint{2.294227in}{0.737298in}}%
\pgfpathlineto{\pgfqpoint{2.294524in}{0.737295in}}%
\pgfpathlineto{\pgfqpoint{2.294822in}{0.737291in}}%
\pgfpathlineto{\pgfqpoint{2.295119in}{0.737288in}}%
\pgfpathlineto{\pgfqpoint{2.295417in}{0.737285in}}%
\pgfpathlineto{\pgfqpoint{2.295714in}{0.737282in}}%
\pgfpathlineto{\pgfqpoint{2.296012in}{0.737279in}}%
\pgfpathlineto{\pgfqpoint{2.296309in}{0.737276in}}%
\pgfpathlineto{\pgfqpoint{2.296607in}{0.737273in}}%
\pgfpathlineto{\pgfqpoint{2.296904in}{0.737270in}}%
\pgfpathlineto{\pgfqpoint{2.297202in}{0.737267in}}%
\pgfpathlineto{\pgfqpoint{2.297499in}{0.737263in}}%
\pgfpathlineto{\pgfqpoint{2.297797in}{0.737260in}}%
\pgfpathlineto{\pgfqpoint{2.298094in}{0.737257in}}%
\pgfpathlineto{\pgfqpoint{2.298392in}{0.737254in}}%
\pgfpathlineto{\pgfqpoint{2.298689in}{0.737251in}}%
\pgfpathlineto{\pgfqpoint{2.298987in}{0.737248in}}%
\pgfpathlineto{\pgfqpoint{2.299284in}{0.737245in}}%
\pgfpathlineto{\pgfqpoint{2.299582in}{0.737242in}}%
\pgfpathlineto{\pgfqpoint{2.299879in}{0.737239in}}%
\pgfpathlineto{\pgfqpoint{2.300176in}{0.737235in}}%
\pgfpathlineto{\pgfqpoint{2.300474in}{0.737232in}}%
\pgfpathlineto{\pgfqpoint{2.300771in}{0.737229in}}%
\pgfpathlineto{\pgfqpoint{2.301069in}{0.737226in}}%
\pgfpathlineto{\pgfqpoint{2.301366in}{0.737223in}}%
\pgfpathlineto{\pgfqpoint{2.301664in}{0.737220in}}%
\pgfpathlineto{\pgfqpoint{2.301961in}{0.737217in}}%
\pgfpathlineto{\pgfqpoint{2.302259in}{0.737214in}}%
\pgfpathlineto{\pgfqpoint{2.302556in}{0.737210in}}%
\pgfpathlineto{\pgfqpoint{2.302854in}{0.737207in}}%
\pgfpathlineto{\pgfqpoint{2.303151in}{0.737204in}}%
\pgfpathlineto{\pgfqpoint{2.303449in}{0.737201in}}%
\pgfpathlineto{\pgfqpoint{2.303746in}{0.737198in}}%
\pgfpathlineto{\pgfqpoint{2.304044in}{0.737195in}}%
\pgfpathlineto{\pgfqpoint{2.304341in}{0.737192in}}%
\pgfpathlineto{\pgfqpoint{2.304639in}{0.737189in}}%
\pgfpathlineto{\pgfqpoint{2.304936in}{0.737186in}}%
\pgfpathlineto{\pgfqpoint{2.305234in}{0.737182in}}%
\pgfpathlineto{\pgfqpoint{2.305531in}{0.737179in}}%
\pgfpathlineto{\pgfqpoint{2.305829in}{0.737176in}}%
\pgfpathlineto{\pgfqpoint{2.306126in}{0.737173in}}%
\pgfpathlineto{\pgfqpoint{2.306424in}{0.737170in}}%
\pgfpathlineto{\pgfqpoint{2.306721in}{0.737167in}}%
\pgfpathlineto{\pgfqpoint{2.307018in}{0.737164in}}%
\pgfpathlineto{\pgfqpoint{2.307316in}{0.737161in}}%
\pgfpathlineto{\pgfqpoint{2.307613in}{0.737158in}}%
\pgfpathlineto{\pgfqpoint{2.307911in}{0.737154in}}%
\pgfpathlineto{\pgfqpoint{2.308208in}{0.737151in}}%
\pgfpathlineto{\pgfqpoint{2.308506in}{0.737148in}}%
\pgfpathlineto{\pgfqpoint{2.308803in}{0.737145in}}%
\pgfpathlineto{\pgfqpoint{2.309101in}{0.737142in}}%
\pgfpathlineto{\pgfqpoint{2.309398in}{0.737139in}}%
\pgfpathlineto{\pgfqpoint{2.309696in}{0.737136in}}%
\pgfpathlineto{\pgfqpoint{2.309993in}{0.737133in}}%
\pgfpathlineto{\pgfqpoint{2.310291in}{0.737129in}}%
\pgfpathlineto{\pgfqpoint{2.310588in}{0.737126in}}%
\pgfpathlineto{\pgfqpoint{2.310886in}{0.737123in}}%
\pgfpathlineto{\pgfqpoint{2.311183in}{0.737120in}}%
\pgfpathlineto{\pgfqpoint{2.311481in}{0.737117in}}%
\pgfpathlineto{\pgfqpoint{2.311778in}{0.737114in}}%
\pgfpathlineto{\pgfqpoint{2.312076in}{0.737111in}}%
\pgfpathlineto{\pgfqpoint{2.312373in}{0.737108in}}%
\pgfpathlineto{\pgfqpoint{2.312671in}{0.737105in}}%
\pgfpathlineto{\pgfqpoint{2.312968in}{0.737101in}}%
\pgfpathlineto{\pgfqpoint{2.313265in}{0.737098in}}%
\pgfpathlineto{\pgfqpoint{2.313563in}{0.737094in}}%
\pgfpathlineto{\pgfqpoint{2.313860in}{0.737091in}}%
\pgfpathlineto{\pgfqpoint{2.314158in}{0.737087in}}%
\pgfpathlineto{\pgfqpoint{2.314455in}{0.737084in}}%
\pgfpathlineto{\pgfqpoint{2.314753in}{0.737080in}}%
\pgfpathlineto{\pgfqpoint{2.315050in}{0.737077in}}%
\pgfpathlineto{\pgfqpoint{2.315348in}{0.737073in}}%
\pgfpathlineto{\pgfqpoint{2.315645in}{0.737069in}}%
\pgfpathlineto{\pgfqpoint{2.315943in}{0.737066in}}%
\pgfpathlineto{\pgfqpoint{2.316240in}{0.737062in}}%
\pgfpathlineto{\pgfqpoint{2.316538in}{0.737059in}}%
\pgfpathlineto{\pgfqpoint{2.316835in}{0.737055in}}%
\pgfpathlineto{\pgfqpoint{2.317133in}{0.737052in}}%
\pgfpathlineto{\pgfqpoint{2.317430in}{0.737048in}}%
\pgfpathlineto{\pgfqpoint{2.317728in}{0.737044in}}%
\pgfpathlineto{\pgfqpoint{2.318025in}{0.737041in}}%
\pgfpathlineto{\pgfqpoint{2.318323in}{0.737037in}}%
\pgfpathlineto{\pgfqpoint{2.318620in}{0.737034in}}%
\pgfpathlineto{\pgfqpoint{2.318918in}{0.737030in}}%
\pgfpathlineto{\pgfqpoint{2.319215in}{0.737027in}}%
\pgfpathlineto{\pgfqpoint{2.319513in}{0.737023in}}%
\pgfpathlineto{\pgfqpoint{2.319810in}{0.737019in}}%
\pgfpathlineto{\pgfqpoint{2.320107in}{0.737016in}}%
\pgfpathlineto{\pgfqpoint{2.320405in}{0.737012in}}%
\pgfpathlineto{\pgfqpoint{2.320702in}{0.737009in}}%
\pgfpathlineto{\pgfqpoint{2.321000in}{0.737005in}}%
\pgfpathlineto{\pgfqpoint{2.321297in}{0.737002in}}%
\pgfpathlineto{\pgfqpoint{2.321595in}{0.736998in}}%
\pgfpathlineto{\pgfqpoint{2.321892in}{0.736994in}}%
\pgfpathlineto{\pgfqpoint{2.322190in}{0.736991in}}%
\pgfpathlineto{\pgfqpoint{2.322487in}{0.736987in}}%
\pgfpathlineto{\pgfqpoint{2.322785in}{0.736984in}}%
\pgfpathlineto{\pgfqpoint{2.323082in}{0.736980in}}%
\pgfpathlineto{\pgfqpoint{2.323380in}{0.736977in}}%
\pgfpathlineto{\pgfqpoint{2.323677in}{0.736973in}}%
\pgfpathlineto{\pgfqpoint{2.323975in}{0.736969in}}%
\pgfpathlineto{\pgfqpoint{2.324272in}{0.736966in}}%
\pgfpathlineto{\pgfqpoint{2.324570in}{0.736962in}}%
\pgfpathlineto{\pgfqpoint{2.324867in}{0.736959in}}%
\pgfpathlineto{\pgfqpoint{2.325165in}{0.736955in}}%
\pgfpathlineto{\pgfqpoint{2.325462in}{0.736952in}}%
\pgfpathlineto{\pgfqpoint{2.325760in}{0.736948in}}%
\pgfpathlineto{\pgfqpoint{2.326057in}{0.736944in}}%
\pgfpathlineto{\pgfqpoint{2.326355in}{0.736941in}}%
\pgfpathlineto{\pgfqpoint{2.326652in}{0.736937in}}%
\pgfpathlineto{\pgfqpoint{2.326949in}{0.736934in}}%
\pgfpathlineto{\pgfqpoint{2.327247in}{0.736930in}}%
\pgfpathlineto{\pgfqpoint{2.327544in}{0.736927in}}%
\pgfpathlineto{\pgfqpoint{2.327842in}{0.736923in}}%
\pgfpathlineto{\pgfqpoint{2.328139in}{0.736919in}}%
\pgfpathlineto{\pgfqpoint{2.328437in}{0.736916in}}%
\pgfpathlineto{\pgfqpoint{2.328734in}{0.736912in}}%
\pgfpathlineto{\pgfqpoint{2.329032in}{0.736909in}}%
\pgfpathlineto{\pgfqpoint{2.329329in}{0.736905in}}%
\pgfpathlineto{\pgfqpoint{2.329627in}{0.736902in}}%
\pgfpathlineto{\pgfqpoint{2.329924in}{0.736898in}}%
\pgfpathlineto{\pgfqpoint{2.330222in}{0.736894in}}%
\pgfpathlineto{\pgfqpoint{2.330519in}{0.736891in}}%
\pgfpathlineto{\pgfqpoint{2.330817in}{0.736887in}}%
\pgfpathlineto{\pgfqpoint{2.331114in}{0.736884in}}%
\pgfpathlineto{\pgfqpoint{2.331412in}{0.736880in}}%
\pgfpathlineto{\pgfqpoint{2.331709in}{0.736877in}}%
\pgfpathlineto{\pgfqpoint{2.332007in}{0.736873in}}%
\pgfpathlineto{\pgfqpoint{2.332304in}{0.736869in}}%
\pgfpathlineto{\pgfqpoint{2.332602in}{0.736866in}}%
\pgfpathlineto{\pgfqpoint{2.332899in}{0.736862in}}%
\pgfpathlineto{\pgfqpoint{2.333196in}{0.736859in}}%
\pgfpathlineto{\pgfqpoint{2.333494in}{0.736855in}}%
\pgfpathlineto{\pgfqpoint{2.333791in}{0.736852in}}%
\pgfpathlineto{\pgfqpoint{2.334089in}{0.736848in}}%
\pgfpathlineto{\pgfqpoint{2.334386in}{0.736845in}}%
\pgfpathlineto{\pgfqpoint{2.334684in}{0.736841in}}%
\pgfpathlineto{\pgfqpoint{2.334981in}{0.736837in}}%
\pgfpathlineto{\pgfqpoint{2.335279in}{0.736834in}}%
\pgfpathlineto{\pgfqpoint{2.335576in}{0.736830in}}%
\pgfpathlineto{\pgfqpoint{2.335874in}{0.736827in}}%
\pgfpathlineto{\pgfqpoint{2.336171in}{0.736823in}}%
\pgfpathlineto{\pgfqpoint{2.336469in}{0.736820in}}%
\pgfpathlineto{\pgfqpoint{2.336766in}{0.736816in}}%
\pgfpathlineto{\pgfqpoint{2.337064in}{0.736812in}}%
\pgfpathlineto{\pgfqpoint{2.337361in}{0.736809in}}%
\pgfpathlineto{\pgfqpoint{2.337659in}{0.736805in}}%
\pgfpathlineto{\pgfqpoint{2.337956in}{0.736802in}}%
\pgfpathlineto{\pgfqpoint{2.338254in}{0.736798in}}%
\pgfpathlineto{\pgfqpoint{2.338551in}{0.736795in}}%
\pgfpathlineto{\pgfqpoint{2.338849in}{0.736791in}}%
\pgfpathlineto{\pgfqpoint{2.339146in}{0.736787in}}%
\pgfpathlineto{\pgfqpoint{2.339444in}{0.736784in}}%
\pgfpathlineto{\pgfqpoint{2.339741in}{0.736780in}}%
\pgfpathlineto{\pgfqpoint{2.340038in}{0.736777in}}%
\pgfpathlineto{\pgfqpoint{2.340336in}{0.736773in}}%
\pgfpathlineto{\pgfqpoint{2.340633in}{0.736770in}}%
\pgfpathlineto{\pgfqpoint{2.340931in}{0.736766in}}%
\pgfpathlineto{\pgfqpoint{2.341228in}{0.736762in}}%
\pgfpathlineto{\pgfqpoint{2.341526in}{0.736759in}}%
\pgfpathlineto{\pgfqpoint{2.341823in}{0.736755in}}%
\pgfpathlineto{\pgfqpoint{2.342121in}{0.736752in}}%
\pgfpathlineto{\pgfqpoint{2.342418in}{0.736748in}}%
\pgfpathlineto{\pgfqpoint{2.342716in}{0.736745in}}%
\pgfpathlineto{\pgfqpoint{2.343013in}{0.736741in}}%
\pgfpathlineto{\pgfqpoint{2.343311in}{0.736737in}}%
\pgfpathlineto{\pgfqpoint{2.343608in}{0.736734in}}%
\pgfpathlineto{\pgfqpoint{2.343906in}{0.736730in}}%
\pgfpathlineto{\pgfqpoint{2.344203in}{0.736727in}}%
\pgfpathlineto{\pgfqpoint{2.344501in}{0.736723in}}%
\pgfpathlineto{\pgfqpoint{2.344798in}{0.736720in}}%
\pgfpathlineto{\pgfqpoint{2.345096in}{0.736716in}}%
\pgfpathlineto{\pgfqpoint{2.345393in}{0.736712in}}%
\pgfpathlineto{\pgfqpoint{2.345691in}{0.736709in}}%
\pgfpathlineto{\pgfqpoint{2.345988in}{0.736705in}}%
\pgfpathlineto{\pgfqpoint{2.346286in}{0.736702in}}%
\pgfpathlineto{\pgfqpoint{2.346583in}{0.736698in}}%
\pgfpathlineto{\pgfqpoint{2.346880in}{0.736695in}}%
\pgfpathlineto{\pgfqpoint{2.347178in}{0.736691in}}%
\pgfpathlineto{\pgfqpoint{2.347475in}{0.736687in}}%
\pgfpathlineto{\pgfqpoint{2.347773in}{0.736684in}}%
\pgfpathlineto{\pgfqpoint{2.348070in}{0.736680in}}%
\pgfpathlineto{\pgfqpoint{2.348368in}{0.736677in}}%
\pgfpathlineto{\pgfqpoint{2.348665in}{0.736673in}}%
\pgfpathlineto{\pgfqpoint{2.348963in}{0.736670in}}%
\pgfpathlineto{\pgfqpoint{2.349260in}{0.736666in}}%
\pgfpathlineto{\pgfqpoint{2.349558in}{0.736662in}}%
\pgfpathlineto{\pgfqpoint{2.349855in}{0.736659in}}%
\pgfpathlineto{\pgfqpoint{2.350153in}{0.736655in}}%
\pgfpathlineto{\pgfqpoint{2.350450in}{0.736652in}}%
\pgfpathlineto{\pgfqpoint{2.350748in}{0.736648in}}%
\pgfpathlineto{\pgfqpoint{2.351045in}{0.736645in}}%
\pgfpathlineto{\pgfqpoint{2.351343in}{0.736641in}}%
\pgfpathlineto{\pgfqpoint{2.351640in}{0.736637in}}%
\pgfpathlineto{\pgfqpoint{2.351938in}{0.736634in}}%
\pgfpathlineto{\pgfqpoint{2.352235in}{0.736630in}}%
\pgfpathlineto{\pgfqpoint{2.352533in}{0.736627in}}%
\pgfpathlineto{\pgfqpoint{2.352830in}{0.736623in}}%
\pgfpathlineto{\pgfqpoint{2.353127in}{0.736620in}}%
\pgfpathlineto{\pgfqpoint{2.353425in}{0.736616in}}%
\pgfpathlineto{\pgfqpoint{2.353722in}{0.736612in}}%
\pgfpathlineto{\pgfqpoint{2.354020in}{0.736609in}}%
\pgfpathlineto{\pgfqpoint{2.354317in}{0.736605in}}%
\pgfpathlineto{\pgfqpoint{2.354615in}{0.736602in}}%
\pgfpathlineto{\pgfqpoint{2.354912in}{0.736598in}}%
\pgfpathlineto{\pgfqpoint{2.355210in}{0.736595in}}%
\pgfpathlineto{\pgfqpoint{2.355507in}{0.736591in}}%
\pgfpathlineto{\pgfqpoint{2.355805in}{0.736587in}}%
\pgfpathlineto{\pgfqpoint{2.356102in}{0.736584in}}%
\pgfpathlineto{\pgfqpoint{2.356400in}{0.736580in}}%
\pgfpathlineto{\pgfqpoint{2.356697in}{0.736577in}}%
\pgfpathlineto{\pgfqpoint{2.356995in}{0.736573in}}%
\pgfpathlineto{\pgfqpoint{2.357292in}{0.735445in}}%
\pgfpathlineto{\pgfqpoint{2.357590in}{0.734753in}}%
\pgfpathlineto{\pgfqpoint{2.357887in}{0.734734in}}%
\pgfpathlineto{\pgfqpoint{2.358185in}{0.734714in}}%
\pgfpathlineto{\pgfqpoint{2.358482in}{0.734695in}}%
\pgfpathlineto{\pgfqpoint{2.358780in}{0.734675in}}%
\pgfpathlineto{\pgfqpoint{2.359077in}{0.734656in}}%
\pgfpathlineto{\pgfqpoint{2.359375in}{0.734636in}}%
\pgfpathlineto{\pgfqpoint{2.359672in}{0.734617in}}%
\pgfpathlineto{\pgfqpoint{2.359969in}{0.734597in}}%
\pgfpathlineto{\pgfqpoint{2.360267in}{0.734578in}}%
\pgfpathlineto{\pgfqpoint{2.360564in}{0.734558in}}%
\pgfpathlineto{\pgfqpoint{2.360862in}{0.734539in}}%
\pgfpathlineto{\pgfqpoint{2.361159in}{0.734519in}}%
\pgfpathlineto{\pgfqpoint{2.361457in}{0.734500in}}%
\pgfpathlineto{\pgfqpoint{2.361754in}{0.734480in}}%
\pgfpathlineto{\pgfqpoint{2.362052in}{0.734461in}}%
\pgfpathlineto{\pgfqpoint{2.362349in}{0.734441in}}%
\pgfpathlineto{\pgfqpoint{2.362647in}{0.734422in}}%
\pgfpathlineto{\pgfqpoint{2.362944in}{0.734402in}}%
\pgfpathlineto{\pgfqpoint{2.363242in}{0.734383in}}%
\pgfpathlineto{\pgfqpoint{2.363539in}{0.734363in}}%
\pgfpathlineto{\pgfqpoint{2.363837in}{0.734344in}}%
\pgfpathlineto{\pgfqpoint{2.364134in}{0.734324in}}%
\pgfpathlineto{\pgfqpoint{2.364432in}{0.734305in}}%
\pgfpathlineto{\pgfqpoint{2.364729in}{0.734285in}}%
\pgfpathlineto{\pgfqpoint{2.365027in}{0.734266in}}%
\pgfpathlineto{\pgfqpoint{2.365324in}{0.734246in}}%
\pgfpathlineto{\pgfqpoint{2.365622in}{0.734227in}}%
\pgfpathlineto{\pgfqpoint{2.365919in}{0.734207in}}%
\pgfpathlineto{\pgfqpoint{2.366217in}{0.734188in}}%
\pgfpathlineto{\pgfqpoint{2.366514in}{0.734168in}}%
\pgfpathlineto{\pgfqpoint{2.366811in}{0.734149in}}%
\pgfpathlineto{\pgfqpoint{2.367109in}{0.734129in}}%
\pgfpathlineto{\pgfqpoint{2.367406in}{0.734110in}}%
\pgfpathlineto{\pgfqpoint{2.367704in}{0.734090in}}%
\pgfpathlineto{\pgfqpoint{2.368001in}{0.734071in}}%
\pgfpathlineto{\pgfqpoint{2.368299in}{0.734051in}}%
\pgfpathlineto{\pgfqpoint{2.368596in}{0.734032in}}%
\pgfpathlineto{\pgfqpoint{2.368894in}{0.734012in}}%
\pgfpathlineto{\pgfqpoint{2.369191in}{0.733993in}}%
\pgfpathlineto{\pgfqpoint{2.369489in}{0.733973in}}%
\pgfpathlineto{\pgfqpoint{2.369786in}{0.733954in}}%
\pgfpathlineto{\pgfqpoint{2.370084in}{0.733934in}}%
\pgfpathlineto{\pgfqpoint{2.370381in}{0.733915in}}%
\pgfpathlineto{\pgfqpoint{2.370679in}{0.733895in}}%
\pgfpathlineto{\pgfqpoint{2.370976in}{0.733876in}}%
\pgfpathlineto{\pgfqpoint{2.371274in}{0.733856in}}%
\pgfpathlineto{\pgfqpoint{2.371571in}{0.733837in}}%
\pgfpathlineto{\pgfqpoint{2.371869in}{0.733817in}}%
\pgfpathlineto{\pgfqpoint{2.372166in}{0.733798in}}%
\pgfpathlineto{\pgfqpoint{2.372464in}{0.733778in}}%
\pgfpathlineto{\pgfqpoint{2.372761in}{0.733759in}}%
\pgfpathlineto{\pgfqpoint{2.373058in}{0.733739in}}%
\pgfpathlineto{\pgfqpoint{2.373356in}{0.733720in}}%
\pgfpathlineto{\pgfqpoint{2.373653in}{0.733700in}}%
\pgfpathlineto{\pgfqpoint{2.373951in}{0.733681in}}%
\pgfpathlineto{\pgfqpoint{2.374248in}{0.733661in}}%
\pgfpathlineto{\pgfqpoint{2.374546in}{0.733642in}}%
\pgfpathlineto{\pgfqpoint{2.374843in}{0.733622in}}%
\pgfpathlineto{\pgfqpoint{2.375141in}{0.733603in}}%
\pgfpathlineto{\pgfqpoint{2.375438in}{0.733583in}}%
\pgfpathlineto{\pgfqpoint{2.375736in}{0.733564in}}%
\pgfpathlineto{\pgfqpoint{2.376033in}{0.733544in}}%
\pgfpathlineto{\pgfqpoint{2.376331in}{0.733525in}}%
\pgfpathlineto{\pgfqpoint{2.376628in}{0.733505in}}%
\pgfpathlineto{\pgfqpoint{2.376926in}{0.733486in}}%
\pgfpathlineto{\pgfqpoint{2.377223in}{0.733466in}}%
\pgfpathlineto{\pgfqpoint{2.377521in}{0.733447in}}%
\pgfpathlineto{\pgfqpoint{2.377818in}{0.733427in}}%
\pgfpathlineto{\pgfqpoint{2.378116in}{0.733408in}}%
\pgfpathlineto{\pgfqpoint{2.378413in}{0.733388in}}%
\pgfpathlineto{\pgfqpoint{2.378711in}{0.733369in}}%
\pgfpathlineto{\pgfqpoint{2.379008in}{0.733349in}}%
\pgfpathlineto{\pgfqpoint{2.379306in}{0.733330in}}%
\pgfpathlineto{\pgfqpoint{2.379603in}{0.733310in}}%
\pgfpathlineto{\pgfqpoint{2.379900in}{0.733291in}}%
\pgfpathlineto{\pgfqpoint{2.380198in}{0.733271in}}%
\pgfpathlineto{\pgfqpoint{2.380495in}{0.733252in}}%
\pgfpathlineto{\pgfqpoint{2.380793in}{0.733232in}}%
\pgfpathlineto{\pgfqpoint{2.381090in}{0.733213in}}%
\pgfpathlineto{\pgfqpoint{2.381388in}{0.733193in}}%
\pgfpathlineto{\pgfqpoint{2.381685in}{0.733174in}}%
\pgfpathlineto{\pgfqpoint{2.381983in}{0.733154in}}%
\pgfpathlineto{\pgfqpoint{2.382280in}{0.733135in}}%
\pgfpathlineto{\pgfqpoint{2.382578in}{0.733115in}}%
\pgfpathlineto{\pgfqpoint{2.382875in}{0.733096in}}%
\pgfpathlineto{\pgfqpoint{2.383173in}{0.733076in}}%
\pgfpathlineto{\pgfqpoint{2.383470in}{0.733057in}}%
\pgfpathlineto{\pgfqpoint{2.383768in}{0.733037in}}%
\pgfpathlineto{\pgfqpoint{2.384065in}{0.733018in}}%
\pgfpathlineto{\pgfqpoint{2.384363in}{0.732998in}}%
\pgfpathlineto{\pgfqpoint{2.384660in}{0.732944in}}%
\pgfpathlineto{\pgfqpoint{2.384958in}{0.732641in}}%
\pgfpathlineto{\pgfqpoint{2.385255in}{0.732290in}}%
\pgfpathlineto{\pgfqpoint{2.385553in}{0.731939in}}%
\pgfpathlineto{\pgfqpoint{2.385850in}{0.731587in}}%
\pgfpathlineto{\pgfqpoint{2.386148in}{0.731346in}}%
\pgfpathlineto{\pgfqpoint{2.386445in}{0.731292in}}%
\pgfpathlineto{\pgfqpoint{2.386742in}{0.731241in}}%
\pgfpathlineto{\pgfqpoint{2.387040in}{0.731189in}}%
\pgfpathlineto{\pgfqpoint{2.387337in}{0.731138in}}%
\pgfpathlineto{\pgfqpoint{2.387635in}{0.731086in}}%
\pgfpathlineto{\pgfqpoint{2.387932in}{0.731035in}}%
\pgfpathlineto{\pgfqpoint{2.388230in}{0.730983in}}%
\pgfpathlineto{\pgfqpoint{2.388527in}{0.730932in}}%
\pgfpathlineto{\pgfqpoint{2.388825in}{0.730881in}}%
\pgfpathlineto{\pgfqpoint{2.389122in}{0.730829in}}%
\pgfpathlineto{\pgfqpoint{2.389420in}{0.730778in}}%
\pgfpathlineto{\pgfqpoint{2.389717in}{0.730726in}}%
\pgfpathlineto{\pgfqpoint{2.390015in}{0.730675in}}%
\pgfpathlineto{\pgfqpoint{2.390312in}{0.730623in}}%
\pgfpathlineto{\pgfqpoint{2.390610in}{0.730572in}}%
\pgfpathlineto{\pgfqpoint{2.390907in}{0.730521in}}%
\pgfpathlineto{\pgfqpoint{2.391205in}{0.730469in}}%
\pgfpathlineto{\pgfqpoint{2.391502in}{0.730418in}}%
\pgfpathlineto{\pgfqpoint{2.391800in}{0.730362in}}%
\pgfpathlineto{\pgfqpoint{2.392097in}{0.730273in}}%
\pgfpathlineto{\pgfqpoint{2.392395in}{0.730393in}}%
\pgfpathlineto{\pgfqpoint{2.392692in}{0.730548in}}%
\pgfpathlineto{\pgfqpoint{2.392989in}{0.730704in}}%
\pgfpathlineto{\pgfqpoint{2.393287in}{0.730860in}}%
\pgfpathlineto{\pgfqpoint{2.393584in}{0.731015in}}%
\pgfpathlineto{\pgfqpoint{2.393882in}{0.731171in}}%
\pgfpathlineto{\pgfqpoint{2.394179in}{0.731326in}}%
\pgfpathlineto{\pgfqpoint{2.394477in}{0.731482in}}%
\pgfpathlineto{\pgfqpoint{2.394774in}{0.731637in}}%
\pgfpathlineto{\pgfqpoint{2.395072in}{0.731793in}}%
\pgfpathlineto{\pgfqpoint{2.395369in}{0.731948in}}%
\pgfpathlineto{\pgfqpoint{2.395667in}{0.732104in}}%
\pgfpathlineto{\pgfqpoint{2.395964in}{0.732260in}}%
\pgfpathlineto{\pgfqpoint{2.396262in}{0.732415in}}%
\pgfpathlineto{\pgfqpoint{2.396559in}{0.732571in}}%
\pgfpathlineto{\pgfqpoint{2.396857in}{0.732726in}}%
\pgfpathlineto{\pgfqpoint{2.397154in}{0.732882in}}%
\pgfpathlineto{\pgfqpoint{2.397452in}{0.733037in}}%
\pgfpathlineto{\pgfqpoint{2.397749in}{0.733193in}}%
\pgfpathlineto{\pgfqpoint{2.398047in}{0.733348in}}%
\pgfpathlineto{\pgfqpoint{2.398344in}{0.733504in}}%
\pgfpathlineto{\pgfqpoint{2.398642in}{0.733659in}}%
\pgfpathlineto{\pgfqpoint{2.398939in}{0.733815in}}%
\pgfpathlineto{\pgfqpoint{2.399237in}{0.733971in}}%
\pgfpathlineto{\pgfqpoint{2.399534in}{0.734066in}}%
\pgfpathlineto{\pgfqpoint{2.399831in}{0.734062in}}%
\pgfpathlineto{\pgfqpoint{2.400129in}{0.733992in}}%
\pgfpathlineto{\pgfqpoint{2.400426in}{0.733900in}}%
\pgfpathlineto{\pgfqpoint{2.400724in}{0.733980in}}%
\pgfpathlineto{\pgfqpoint{2.401021in}{0.734032in}}%
\pgfpathlineto{\pgfqpoint{2.401319in}{0.734087in}}%
\pgfpathlineto{\pgfqpoint{2.401616in}{0.734142in}}%
\pgfpathlineto{\pgfqpoint{2.401914in}{0.734197in}}%
\pgfpathlineto{\pgfqpoint{2.402211in}{0.734251in}}%
\pgfpathlineto{\pgfqpoint{2.402509in}{0.734306in}}%
\pgfpathlineto{\pgfqpoint{2.402806in}{0.734361in}}%
\pgfpathlineto{\pgfqpoint{2.403104in}{0.734416in}}%
\pgfpathlineto{\pgfqpoint{2.403401in}{0.734471in}}%
\pgfpathlineto{\pgfqpoint{2.403699in}{0.734526in}}%
\pgfpathlineto{\pgfqpoint{2.403996in}{0.734580in}}%
\pgfpathlineto{\pgfqpoint{2.404294in}{0.734635in}}%
\pgfpathlineto{\pgfqpoint{2.404591in}{0.734690in}}%
\pgfpathlineto{\pgfqpoint{2.404889in}{0.734733in}}%
\pgfpathlineto{\pgfqpoint{2.405186in}{0.734766in}}%
\pgfpathlineto{\pgfqpoint{2.405484in}{0.734989in}}%
\pgfpathlineto{\pgfqpoint{2.405781in}{0.735110in}}%
\pgfpathlineto{\pgfqpoint{2.406079in}{0.735103in}}%
\pgfpathlineto{\pgfqpoint{2.406376in}{0.735096in}}%
\pgfpathlineto{\pgfqpoint{2.406673in}{0.735089in}}%
\pgfpathlineto{\pgfqpoint{2.406971in}{0.735082in}}%
\pgfpathlineto{\pgfqpoint{2.407268in}{0.735075in}}%
\pgfpathlineto{\pgfqpoint{2.407566in}{0.735068in}}%
\pgfpathlineto{\pgfqpoint{2.407863in}{0.735061in}}%
\pgfpathlineto{\pgfqpoint{2.408161in}{0.735054in}}%
\pgfpathlineto{\pgfqpoint{2.408458in}{0.735047in}}%
\pgfpathlineto{\pgfqpoint{2.408756in}{0.735040in}}%
\pgfpathlineto{\pgfqpoint{2.409053in}{0.735033in}}%
\pgfpathlineto{\pgfqpoint{2.409351in}{0.735026in}}%
\pgfpathlineto{\pgfqpoint{2.409648in}{0.735019in}}%
\pgfpathlineto{\pgfqpoint{2.409946in}{0.735012in}}%
\pgfpathlineto{\pgfqpoint{2.410243in}{0.735005in}}%
\pgfpathlineto{\pgfqpoint{2.410541in}{0.734998in}}%
\pgfpathlineto{\pgfqpoint{2.410838in}{0.734992in}}%
\pgfpathlineto{\pgfqpoint{2.411136in}{0.734985in}}%
\pgfpathlineto{\pgfqpoint{2.411433in}{0.734978in}}%
\pgfpathlineto{\pgfqpoint{2.411731in}{0.734971in}}%
\pgfpathlineto{\pgfqpoint{2.412028in}{0.734964in}}%
\pgfpathlineto{\pgfqpoint{2.412326in}{0.734957in}}%
\pgfpathlineto{\pgfqpoint{2.412623in}{0.734950in}}%
\pgfpathlineto{\pgfqpoint{2.412920in}{0.734943in}}%
\pgfpathlineto{\pgfqpoint{2.413218in}{0.734936in}}%
\pgfpathlineto{\pgfqpoint{2.413515in}{0.734929in}}%
\pgfpathlineto{\pgfqpoint{2.413813in}{0.734922in}}%
\pgfpathlineto{\pgfqpoint{2.414110in}{0.734915in}}%
\pgfpathlineto{\pgfqpoint{2.414408in}{0.734908in}}%
\pgfpathlineto{\pgfqpoint{2.414705in}{0.734901in}}%
\pgfpathlineto{\pgfqpoint{2.415003in}{0.734894in}}%
\pgfpathlineto{\pgfqpoint{2.415300in}{0.734887in}}%
\pgfpathlineto{\pgfqpoint{2.415598in}{0.734880in}}%
\pgfpathlineto{\pgfqpoint{2.415895in}{0.734873in}}%
\pgfpathlineto{\pgfqpoint{2.416193in}{0.734866in}}%
\pgfpathlineto{\pgfqpoint{2.416490in}{0.734859in}}%
\pgfpathlineto{\pgfqpoint{2.416788in}{0.734852in}}%
\pgfpathlineto{\pgfqpoint{2.417085in}{0.734845in}}%
\pgfpathlineto{\pgfqpoint{2.417383in}{0.734838in}}%
\pgfpathlineto{\pgfqpoint{2.417680in}{0.734831in}}%
\pgfpathlineto{\pgfqpoint{2.417978in}{0.734824in}}%
\pgfpathlineto{\pgfqpoint{2.418275in}{0.734817in}}%
\pgfpathlineto{\pgfqpoint{2.418573in}{0.734810in}}%
\pgfpathlineto{\pgfqpoint{2.418870in}{0.734803in}}%
\pgfpathlineto{\pgfqpoint{2.419168in}{0.734796in}}%
\pgfpathlineto{\pgfqpoint{2.419465in}{0.734789in}}%
\pgfpathlineto{\pgfqpoint{2.419762in}{0.734782in}}%
\pgfpathlineto{\pgfqpoint{2.420060in}{0.734776in}}%
\pgfpathlineto{\pgfqpoint{2.420357in}{0.734769in}}%
\pgfpathlineto{\pgfqpoint{2.420655in}{0.734762in}}%
\pgfpathlineto{\pgfqpoint{2.420952in}{0.734755in}}%
\pgfpathlineto{\pgfqpoint{2.421250in}{0.734748in}}%
\pgfpathlineto{\pgfqpoint{2.421547in}{0.734741in}}%
\pgfpathlineto{\pgfqpoint{2.421845in}{0.734734in}}%
\pgfpathlineto{\pgfqpoint{2.422142in}{0.734727in}}%
\pgfpathlineto{\pgfqpoint{2.422440in}{0.734720in}}%
\pgfpathlineto{\pgfqpoint{2.422737in}{0.734713in}}%
\pgfpathlineto{\pgfqpoint{2.423035in}{0.734706in}}%
\pgfpathlineto{\pgfqpoint{2.423332in}{0.734699in}}%
\pgfpathlineto{\pgfqpoint{2.423630in}{0.734692in}}%
\pgfpathlineto{\pgfqpoint{2.423927in}{0.734685in}}%
\pgfpathlineto{\pgfqpoint{2.424225in}{0.734678in}}%
\pgfpathlineto{\pgfqpoint{2.424522in}{0.734671in}}%
\pgfpathlineto{\pgfqpoint{2.424820in}{0.734664in}}%
\pgfpathlineto{\pgfqpoint{2.425117in}{0.734657in}}%
\pgfpathlineto{\pgfqpoint{2.425415in}{0.734650in}}%
\pgfpathlineto{\pgfqpoint{2.425712in}{0.734643in}}%
\pgfpathlineto{\pgfqpoint{2.426010in}{0.734636in}}%
\pgfpathlineto{\pgfqpoint{2.426307in}{0.734629in}}%
\pgfpathlineto{\pgfqpoint{2.426604in}{0.734622in}}%
\pgfpathlineto{\pgfqpoint{2.426902in}{0.734615in}}%
\pgfpathlineto{\pgfqpoint{2.427199in}{0.734608in}}%
\pgfpathlineto{\pgfqpoint{2.427497in}{0.734601in}}%
\pgfpathlineto{\pgfqpoint{2.427794in}{0.734594in}}%
\pgfpathlineto{\pgfqpoint{2.428092in}{0.734587in}}%
\pgfpathlineto{\pgfqpoint{2.428389in}{0.734580in}}%
\pgfpathlineto{\pgfqpoint{2.428687in}{0.734573in}}%
\pgfpathlineto{\pgfqpoint{2.428984in}{0.734566in}}%
\pgfpathlineto{\pgfqpoint{2.429282in}{0.734560in}}%
\pgfpathlineto{\pgfqpoint{2.429579in}{0.734553in}}%
\pgfpathlineto{\pgfqpoint{2.429877in}{0.734546in}}%
\pgfpathlineto{\pgfqpoint{2.430174in}{0.734539in}}%
\pgfpathlineto{\pgfqpoint{2.430472in}{0.734532in}}%
\pgfpathlineto{\pgfqpoint{2.430769in}{0.734525in}}%
\pgfpathlineto{\pgfqpoint{2.431067in}{0.734518in}}%
\pgfpathlineto{\pgfqpoint{2.431364in}{0.734511in}}%
\pgfpathlineto{\pgfqpoint{2.431662in}{0.734504in}}%
\pgfpathlineto{\pgfqpoint{2.431959in}{0.734497in}}%
\pgfpathlineto{\pgfqpoint{2.432257in}{0.734490in}}%
\pgfpathlineto{\pgfqpoint{2.432554in}{0.734483in}}%
\pgfpathlineto{\pgfqpoint{2.432851in}{0.734476in}}%
\pgfpathlineto{\pgfqpoint{2.433149in}{0.734469in}}%
\pgfpathlineto{\pgfqpoint{2.433446in}{0.734462in}}%
\pgfpathlineto{\pgfqpoint{2.433744in}{0.734455in}}%
\pgfpathlineto{\pgfqpoint{2.434041in}{0.734448in}}%
\pgfpathlineto{\pgfqpoint{2.434339in}{0.734441in}}%
\pgfpathlineto{\pgfqpoint{2.434636in}{0.734434in}}%
\pgfpathlineto{\pgfqpoint{2.434934in}{0.734427in}}%
\pgfpathlineto{\pgfqpoint{2.435231in}{0.734420in}}%
\pgfpathlineto{\pgfqpoint{2.435529in}{0.734413in}}%
\pgfpathlineto{\pgfqpoint{2.435826in}{0.734406in}}%
\pgfpathlineto{\pgfqpoint{2.436124in}{0.734399in}}%
\pgfpathlineto{\pgfqpoint{2.436421in}{0.734392in}}%
\pgfpathlineto{\pgfqpoint{2.436719in}{0.734385in}}%
\pgfpathlineto{\pgfqpoint{2.437016in}{0.734378in}}%
\pgfpathlineto{\pgfqpoint{2.437314in}{0.734371in}}%
\pgfpathlineto{\pgfqpoint{2.437611in}{0.734364in}}%
\pgfpathlineto{\pgfqpoint{2.437909in}{0.734357in}}%
\pgfpathlineto{\pgfqpoint{2.438206in}{0.734351in}}%
\pgfpathlineto{\pgfqpoint{2.438504in}{0.734344in}}%
\pgfpathlineto{\pgfqpoint{2.438801in}{0.734337in}}%
\pgfpathlineto{\pgfqpoint{2.439099in}{0.734330in}}%
\pgfpathlineto{\pgfqpoint{2.439396in}{0.734323in}}%
\pgfpathlineto{\pgfqpoint{2.439693in}{0.734316in}}%
\pgfpathlineto{\pgfqpoint{2.439991in}{0.734309in}}%
\pgfpathlineto{\pgfqpoint{2.440288in}{0.734302in}}%
\pgfpathlineto{\pgfqpoint{2.440586in}{0.734295in}}%
\pgfpathlineto{\pgfqpoint{2.440883in}{0.734288in}}%
\pgfpathlineto{\pgfqpoint{2.441181in}{0.734281in}}%
\pgfpathlineto{\pgfqpoint{2.441478in}{0.734274in}}%
\pgfpathlineto{\pgfqpoint{2.441776in}{0.734267in}}%
\pgfpathlineto{\pgfqpoint{2.442073in}{0.734260in}}%
\pgfpathlineto{\pgfqpoint{2.442371in}{0.734253in}}%
\pgfpathlineto{\pgfqpoint{2.442668in}{0.734246in}}%
\pgfpathlineto{\pgfqpoint{2.442966in}{0.734239in}}%
\pgfpathlineto{\pgfqpoint{2.443263in}{0.734232in}}%
\pgfpathlineto{\pgfqpoint{2.443561in}{0.734225in}}%
\pgfpathlineto{\pgfqpoint{2.443858in}{0.734218in}}%
\pgfpathlineto{\pgfqpoint{2.444156in}{0.734211in}}%
\pgfpathlineto{\pgfqpoint{2.444453in}{0.734204in}}%
\pgfpathlineto{\pgfqpoint{2.444751in}{0.734197in}}%
\pgfpathlineto{\pgfqpoint{2.445048in}{0.734190in}}%
\pgfpathlineto{\pgfqpoint{2.445346in}{0.734183in}}%
\pgfpathlineto{\pgfqpoint{2.445643in}{0.734176in}}%
\pgfpathlineto{\pgfqpoint{2.445941in}{0.734169in}}%
\pgfpathlineto{\pgfqpoint{2.446238in}{0.734162in}}%
\pgfpathlineto{\pgfqpoint{2.446535in}{0.734155in}}%
\pgfpathlineto{\pgfqpoint{2.446833in}{0.734148in}}%
\pgfpathlineto{\pgfqpoint{2.447130in}{0.734141in}}%
\pgfpathlineto{\pgfqpoint{2.447428in}{0.734135in}}%
\pgfpathlineto{\pgfqpoint{2.447725in}{0.734128in}}%
\pgfpathlineto{\pgfqpoint{2.448023in}{0.734121in}}%
\pgfpathlineto{\pgfqpoint{2.448320in}{0.734114in}}%
\pgfpathlineto{\pgfqpoint{2.448618in}{0.734107in}}%
\pgfpathlineto{\pgfqpoint{2.448915in}{0.734100in}}%
\pgfpathlineto{\pgfqpoint{2.449213in}{0.734093in}}%
\pgfpathlineto{\pgfqpoint{2.449510in}{0.734086in}}%
\pgfpathlineto{\pgfqpoint{2.449808in}{0.734079in}}%
\pgfpathlineto{\pgfqpoint{2.450105in}{0.734072in}}%
\pgfpathlineto{\pgfqpoint{2.450403in}{0.734065in}}%
\pgfpathlineto{\pgfqpoint{2.450700in}{0.734058in}}%
\pgfpathlineto{\pgfqpoint{2.450998in}{0.734051in}}%
\pgfpathlineto{\pgfqpoint{2.451295in}{0.734044in}}%
\pgfpathlineto{\pgfqpoint{2.451593in}{0.734037in}}%
\pgfpathlineto{\pgfqpoint{2.451890in}{0.734030in}}%
\pgfpathlineto{\pgfqpoint{2.452188in}{0.734023in}}%
\pgfpathlineto{\pgfqpoint{2.452485in}{0.734016in}}%
\pgfpathlineto{\pgfqpoint{2.452783in}{0.734009in}}%
\pgfpathlineto{\pgfqpoint{2.453080in}{0.734002in}}%
\pgfpathlineto{\pgfqpoint{2.453377in}{0.733995in}}%
\pgfpathlineto{\pgfqpoint{2.453675in}{0.733988in}}%
\pgfpathlineto{\pgfqpoint{2.453972in}{0.733981in}}%
\pgfpathlineto{\pgfqpoint{2.454270in}{0.733974in}}%
\pgfpathlineto{\pgfqpoint{2.454567in}{0.733967in}}%
\pgfpathlineto{\pgfqpoint{2.454865in}{0.733960in}}%
\pgfpathlineto{\pgfqpoint{2.455162in}{0.733953in}}%
\pgfpathlineto{\pgfqpoint{2.455460in}{0.733946in}}%
\pgfpathlineto{\pgfqpoint{2.455757in}{0.733939in}}%
\pgfpathlineto{\pgfqpoint{2.456055in}{0.733932in}}%
\pgfpathlineto{\pgfqpoint{2.456352in}{0.733925in}}%
\pgfpathlineto{\pgfqpoint{2.456650in}{0.733919in}}%
\pgfpathlineto{\pgfqpoint{2.456947in}{0.733912in}}%
\pgfpathlineto{\pgfqpoint{2.457245in}{0.733905in}}%
\pgfpathlineto{\pgfqpoint{2.457542in}{0.733906in}}%
\pgfpathlineto{\pgfqpoint{2.457840in}{0.733959in}}%
\pgfpathlineto{\pgfqpoint{2.458137in}{0.733967in}}%
\pgfpathlineto{\pgfqpoint{2.458435in}{0.733960in}}%
\pgfpathlineto{\pgfqpoint{2.458732in}{0.733953in}}%
\pgfpathlineto{\pgfqpoint{2.459030in}{0.733946in}}%
\pgfpathlineto{\pgfqpoint{2.459327in}{0.733940in}}%
\pgfpathlineto{\pgfqpoint{2.459624in}{0.733933in}}%
\pgfpathlineto{\pgfqpoint{2.459922in}{0.733926in}}%
\pgfpathlineto{\pgfqpoint{2.460219in}{0.733919in}}%
\pgfpathlineto{\pgfqpoint{2.460517in}{0.733912in}}%
\pgfpathlineto{\pgfqpoint{2.460814in}{0.733905in}}%
\pgfpathlineto{\pgfqpoint{2.461112in}{0.733898in}}%
\pgfpathlineto{\pgfqpoint{2.461409in}{0.733892in}}%
\pgfpathlineto{\pgfqpoint{2.461707in}{0.733885in}}%
\pgfpathlineto{\pgfqpoint{2.462004in}{0.733878in}}%
\pgfpathlineto{\pgfqpoint{2.462302in}{0.733871in}}%
\pgfpathlineto{\pgfqpoint{2.462599in}{0.733864in}}%
\pgfpathlineto{\pgfqpoint{2.462897in}{0.733857in}}%
\pgfpathlineto{\pgfqpoint{2.463194in}{0.733850in}}%
\pgfpathlineto{\pgfqpoint{2.463492in}{0.733844in}}%
\pgfpathlineto{\pgfqpoint{2.463789in}{0.733837in}}%
\pgfpathlineto{\pgfqpoint{2.464087in}{0.733830in}}%
\pgfpathlineto{\pgfqpoint{2.464384in}{0.733823in}}%
\pgfpathlineto{\pgfqpoint{2.464682in}{0.733816in}}%
\pgfpathlineto{\pgfqpoint{2.464979in}{0.733809in}}%
\pgfpathlineto{\pgfqpoint{2.465277in}{0.733803in}}%
\pgfpathlineto{\pgfqpoint{2.465574in}{0.733796in}}%
\pgfpathlineto{\pgfqpoint{2.465872in}{0.733789in}}%
\pgfpathlineto{\pgfqpoint{2.466169in}{0.733782in}}%
\pgfpathlineto{\pgfqpoint{2.466466in}{0.733775in}}%
\pgfpathlineto{\pgfqpoint{2.466764in}{0.733768in}}%
\pgfpathlineto{\pgfqpoint{2.467061in}{0.733761in}}%
\pgfpathlineto{\pgfqpoint{2.467359in}{0.733755in}}%
\pgfpathlineto{\pgfqpoint{2.467656in}{0.733748in}}%
\pgfpathlineto{\pgfqpoint{2.467954in}{0.733741in}}%
\pgfpathlineto{\pgfqpoint{2.468251in}{0.733734in}}%
\pgfpathlineto{\pgfqpoint{2.468549in}{0.733727in}}%
\pgfpathlineto{\pgfqpoint{2.468846in}{0.733720in}}%
\pgfpathlineto{\pgfqpoint{2.469144in}{0.733713in}}%
\pgfpathlineto{\pgfqpoint{2.469441in}{0.733707in}}%
\pgfpathlineto{\pgfqpoint{2.469739in}{0.733700in}}%
\pgfpathlineto{\pgfqpoint{2.470036in}{0.733693in}}%
\pgfpathlineto{\pgfqpoint{2.470334in}{0.733686in}}%
\pgfpathlineto{\pgfqpoint{2.470631in}{0.733679in}}%
\pgfpathlineto{\pgfqpoint{2.470929in}{0.733672in}}%
\pgfpathlineto{\pgfqpoint{2.471226in}{0.733666in}}%
\pgfpathlineto{\pgfqpoint{2.471524in}{0.733659in}}%
\pgfpathlineto{\pgfqpoint{2.471821in}{0.733652in}}%
\pgfpathlineto{\pgfqpoint{2.472119in}{0.733645in}}%
\pgfpathlineto{\pgfqpoint{2.472416in}{0.733638in}}%
\pgfpathlineto{\pgfqpoint{2.472714in}{0.733631in}}%
\pgfpathlineto{\pgfqpoint{2.473011in}{0.733624in}}%
\pgfpathlineto{\pgfqpoint{2.473308in}{0.733618in}}%
\pgfpathlineto{\pgfqpoint{2.473606in}{0.733611in}}%
\pgfpathlineto{\pgfqpoint{2.473903in}{0.733604in}}%
\pgfpathlineto{\pgfqpoint{2.474201in}{0.733597in}}%
\pgfpathlineto{\pgfqpoint{2.474498in}{0.733590in}}%
\pgfpathlineto{\pgfqpoint{2.474796in}{0.733583in}}%
\pgfpathlineto{\pgfqpoint{2.475093in}{0.733576in}}%
\pgfpathlineto{\pgfqpoint{2.475391in}{0.733570in}}%
\pgfpathlineto{\pgfqpoint{2.475688in}{0.733563in}}%
\pgfpathlineto{\pgfqpoint{2.475986in}{0.733556in}}%
\pgfpathlineto{\pgfqpoint{2.476283in}{0.733549in}}%
\pgfpathlineto{\pgfqpoint{2.476581in}{0.733542in}}%
\pgfpathlineto{\pgfqpoint{2.476878in}{0.733535in}}%
\pgfpathlineto{\pgfqpoint{2.477176in}{0.733529in}}%
\pgfpathlineto{\pgfqpoint{2.477473in}{0.733522in}}%
\pgfpathlineto{\pgfqpoint{2.477771in}{0.733519in}}%
\pgfpathlineto{\pgfqpoint{2.478068in}{0.733517in}}%
\pgfpathlineto{\pgfqpoint{2.478366in}{0.733515in}}%
\pgfpathlineto{\pgfqpoint{2.478663in}{0.733513in}}%
\pgfpathlineto{\pgfqpoint{2.478961in}{0.733511in}}%
\pgfpathlineto{\pgfqpoint{2.479258in}{0.733510in}}%
\pgfpathlineto{\pgfqpoint{2.479555in}{0.733508in}}%
\pgfpathlineto{\pgfqpoint{2.479853in}{0.733506in}}%
\pgfpathlineto{\pgfqpoint{2.480150in}{0.733504in}}%
\pgfpathlineto{\pgfqpoint{2.480448in}{0.733502in}}%
\pgfpathlineto{\pgfqpoint{2.480745in}{0.733500in}}%
\pgfpathlineto{\pgfqpoint{2.481043in}{0.733498in}}%
\pgfpathlineto{\pgfqpoint{2.481340in}{0.733497in}}%
\pgfpathlineto{\pgfqpoint{2.481638in}{0.733495in}}%
\pgfpathlineto{\pgfqpoint{2.481935in}{0.733493in}}%
\pgfpathlineto{\pgfqpoint{2.482233in}{0.733491in}}%
\pgfpathlineto{\pgfqpoint{2.482530in}{0.733489in}}%
\pgfpathlineto{\pgfqpoint{2.482828in}{0.733487in}}%
\pgfpathlineto{\pgfqpoint{2.483125in}{0.733486in}}%
\pgfpathlineto{\pgfqpoint{2.483423in}{0.733484in}}%
\pgfpathlineto{\pgfqpoint{2.483720in}{0.733482in}}%
\pgfpathlineto{\pgfqpoint{2.484018in}{0.733480in}}%
\pgfpathlineto{\pgfqpoint{2.484315in}{0.733478in}}%
\pgfpathlineto{\pgfqpoint{2.484613in}{0.733476in}}%
\pgfpathlineto{\pgfqpoint{2.484910in}{0.733474in}}%
\pgfpathlineto{\pgfqpoint{2.485208in}{0.733473in}}%
\pgfpathlineto{\pgfqpoint{2.485505in}{0.733471in}}%
\pgfpathlineto{\pgfqpoint{2.485803in}{0.733469in}}%
\pgfpathlineto{\pgfqpoint{2.486100in}{0.733467in}}%
\pgfpathlineto{\pgfqpoint{2.486397in}{0.733465in}}%
\pgfpathlineto{\pgfqpoint{2.486695in}{0.733463in}}%
\pgfpathlineto{\pgfqpoint{2.486992in}{0.733462in}}%
\pgfpathlineto{\pgfqpoint{2.487290in}{0.733460in}}%
\pgfpathlineto{\pgfqpoint{2.487587in}{0.733458in}}%
\pgfpathlineto{\pgfqpoint{2.487885in}{0.733456in}}%
\pgfpathlineto{\pgfqpoint{2.488182in}{0.733454in}}%
\pgfpathlineto{\pgfqpoint{2.488480in}{0.733452in}}%
\pgfpathlineto{\pgfqpoint{2.488777in}{0.733450in}}%
\pgfpathlineto{\pgfqpoint{2.489075in}{0.733449in}}%
\pgfpathlineto{\pgfqpoint{2.489372in}{0.733447in}}%
\pgfpathlineto{\pgfqpoint{2.489670in}{0.733445in}}%
\pgfpathlineto{\pgfqpoint{2.489967in}{0.733443in}}%
\pgfpathlineto{\pgfqpoint{2.490265in}{0.733441in}}%
\pgfpathlineto{\pgfqpoint{2.490562in}{0.733439in}}%
\pgfpathlineto{\pgfqpoint{2.490860in}{0.733438in}}%
\pgfpathlineto{\pgfqpoint{2.491157in}{0.733436in}}%
\pgfpathlineto{\pgfqpoint{2.491455in}{0.733434in}}%
\pgfpathlineto{\pgfqpoint{2.491752in}{0.733432in}}%
\pgfpathlineto{\pgfqpoint{2.492050in}{0.733430in}}%
\pgfpathlineto{\pgfqpoint{2.492347in}{0.733428in}}%
\pgfpathlineto{\pgfqpoint{2.492645in}{0.733427in}}%
\pgfpathlineto{\pgfqpoint{2.492942in}{0.733425in}}%
\pgfpathlineto{\pgfqpoint{2.493239in}{0.733423in}}%
\pgfpathlineto{\pgfqpoint{2.493537in}{0.733421in}}%
\pgfpathlineto{\pgfqpoint{2.493834in}{0.733419in}}%
\pgfpathlineto{\pgfqpoint{2.494132in}{0.733417in}}%
\pgfpathlineto{\pgfqpoint{2.494429in}{0.733415in}}%
\pgfpathlineto{\pgfqpoint{2.494727in}{0.733414in}}%
\pgfpathlineto{\pgfqpoint{2.495024in}{0.733412in}}%
\pgfpathlineto{\pgfqpoint{2.495322in}{0.733410in}}%
\pgfpathlineto{\pgfqpoint{2.495619in}{0.733408in}}%
\pgfpathlineto{\pgfqpoint{2.495917in}{0.733406in}}%
\pgfpathlineto{\pgfqpoint{2.496214in}{0.733404in}}%
\pgfpathlineto{\pgfqpoint{2.496512in}{0.733403in}}%
\pgfpathlineto{\pgfqpoint{2.496809in}{0.733401in}}%
\pgfpathlineto{\pgfqpoint{2.497107in}{0.733399in}}%
\pgfpathlineto{\pgfqpoint{2.497404in}{0.733397in}}%
\pgfpathlineto{\pgfqpoint{2.497702in}{0.733395in}}%
\pgfpathlineto{\pgfqpoint{2.497999in}{0.733393in}}%
\pgfpathlineto{\pgfqpoint{2.498297in}{0.733391in}}%
\pgfpathlineto{\pgfqpoint{2.498594in}{0.733390in}}%
\pgfpathlineto{\pgfqpoint{2.498892in}{0.733388in}}%
\pgfpathlineto{\pgfqpoint{2.499189in}{0.733386in}}%
\pgfpathlineto{\pgfqpoint{2.499486in}{0.733610in}}%
\pgfpathlineto{\pgfqpoint{2.499784in}{0.734194in}}%
\pgfpathlineto{\pgfqpoint{2.500081in}{0.734782in}}%
\pgfpathlineto{\pgfqpoint{2.500379in}{0.735371in}}%
\pgfpathlineto{\pgfqpoint{2.500676in}{0.735958in}}%
\pgfpathlineto{\pgfqpoint{2.500974in}{0.736201in}}%
\pgfpathlineto{\pgfqpoint{2.501271in}{0.736196in}}%
\pgfpathlineto{\pgfqpoint{2.501569in}{0.736194in}}%
\pgfpathlineto{\pgfqpoint{2.501866in}{0.736192in}}%
\pgfpathlineto{\pgfqpoint{2.502164in}{0.736191in}}%
\pgfpathlineto{\pgfqpoint{2.502461in}{0.736189in}}%
\pgfpathlineto{\pgfqpoint{2.502759in}{0.736187in}}%
\pgfpathlineto{\pgfqpoint{2.503056in}{0.736185in}}%
\pgfpathlineto{\pgfqpoint{2.503354in}{0.736183in}}%
\pgfpathlineto{\pgfqpoint{2.503651in}{0.736181in}}%
\pgfpathlineto{\pgfqpoint{2.503949in}{0.736180in}}%
\pgfpathlineto{\pgfqpoint{2.504246in}{0.736178in}}%
\pgfpathlineto{\pgfqpoint{2.504544in}{0.736176in}}%
\pgfpathlineto{\pgfqpoint{2.504841in}{0.736174in}}%
\pgfpathlineto{\pgfqpoint{2.505139in}{0.736172in}}%
\pgfpathlineto{\pgfqpoint{2.505436in}{0.736170in}}%
\pgfpathlineto{\pgfqpoint{2.505734in}{0.736168in}}%
\pgfpathlineto{\pgfqpoint{2.506031in}{0.736167in}}%
\pgfpathlineto{\pgfqpoint{2.506328in}{0.736165in}}%
\pgfpathlineto{\pgfqpoint{2.506626in}{0.736127in}}%
\pgfpathlineto{\pgfqpoint{2.506923in}{0.734970in}}%
\pgfpathlineto{\pgfqpoint{2.507221in}{0.733288in}}%
\pgfpathlineto{\pgfqpoint{2.507518in}{0.731605in}}%
\pgfpathlineto{\pgfqpoint{2.507816in}{0.729922in}}%
\pgfpathlineto{\pgfqpoint{2.508113in}{0.728239in}}%
\pgfpathlineto{\pgfqpoint{2.508411in}{0.726556in}}%
\pgfpathlineto{\pgfqpoint{2.508708in}{0.724873in}}%
\pgfpathlineto{\pgfqpoint{2.509006in}{0.723191in}}%
\pgfpathlineto{\pgfqpoint{2.509303in}{0.721508in}}%
\pgfpathlineto{\pgfqpoint{2.509601in}{0.719825in}}%
\pgfpathlineto{\pgfqpoint{2.509898in}{0.718142in}}%
\pgfpathlineto{\pgfqpoint{2.510196in}{0.716459in}}%
\pgfpathlineto{\pgfqpoint{2.510493in}{0.714776in}}%
\pgfpathlineto{\pgfqpoint{2.510791in}{0.713093in}}%
\pgfpathlineto{\pgfqpoint{2.511088in}{0.711411in}}%
\pgfpathlineto{\pgfqpoint{2.511386in}{0.709728in}}%
\pgfpathlineto{\pgfqpoint{2.511683in}{0.708045in}}%
\pgfpathlineto{\pgfqpoint{2.511981in}{0.706362in}}%
\pgfpathlineto{\pgfqpoint{2.512278in}{0.704679in}}%
\pgfpathlineto{\pgfqpoint{2.512576in}{0.702996in}}%
\pgfpathlineto{\pgfqpoint{2.512873in}{0.701314in}}%
\pgfpathlineto{\pgfqpoint{2.513170in}{0.699927in}}%
\pgfpathlineto{\pgfqpoint{2.513468in}{0.699825in}}%
\pgfpathlineto{\pgfqpoint{2.513765in}{0.699864in}}%
\pgfpathlineto{\pgfqpoint{2.514063in}{0.699904in}}%
\pgfpathlineto{\pgfqpoint{2.514360in}{0.699919in}}%
\pgfpathlineto{\pgfqpoint{2.514658in}{0.699917in}}%
\pgfpathlineto{\pgfqpoint{2.514955in}{0.699915in}}%
\pgfpathlineto{\pgfqpoint{2.515253in}{0.699913in}}%
\pgfpathlineto{\pgfqpoint{2.515550in}{0.699911in}}%
\pgfpathlineto{\pgfqpoint{2.515848in}{0.699910in}}%
\pgfpathlineto{\pgfqpoint{2.516145in}{0.699908in}}%
\pgfpathlineto{\pgfqpoint{2.516443in}{0.699906in}}%
\pgfpathlineto{\pgfqpoint{2.516740in}{0.699904in}}%
\pgfpathlineto{\pgfqpoint{2.517038in}{0.699902in}}%
\pgfpathlineto{\pgfqpoint{2.517335in}{0.699900in}}%
\pgfpathlineto{\pgfqpoint{2.517633in}{0.699898in}}%
\pgfpathlineto{\pgfqpoint{2.517930in}{0.699897in}}%
\pgfpathlineto{\pgfqpoint{2.518228in}{0.699895in}}%
\pgfpathlineto{\pgfqpoint{2.518525in}{0.699893in}}%
\pgfpathlineto{\pgfqpoint{2.518823in}{0.699891in}}%
\pgfpathlineto{\pgfqpoint{2.519120in}{0.699889in}}%
\pgfpathlineto{\pgfqpoint{2.519417in}{0.699887in}}%
\pgfpathlineto{\pgfqpoint{2.519715in}{0.699886in}}%
\pgfpathlineto{\pgfqpoint{2.520012in}{0.699884in}}%
\pgfpathlineto{\pgfqpoint{2.520310in}{0.699882in}}%
\pgfpathlineto{\pgfqpoint{2.520607in}{0.699880in}}%
\pgfpathlineto{\pgfqpoint{2.520905in}{0.699878in}}%
\pgfpathlineto{\pgfqpoint{2.521202in}{0.699876in}}%
\pgfpathlineto{\pgfqpoint{2.521500in}{0.699875in}}%
\pgfpathlineto{\pgfqpoint{2.521797in}{0.699873in}}%
\pgfpathlineto{\pgfqpoint{2.522095in}{0.699871in}}%
\pgfpathlineto{\pgfqpoint{2.522392in}{0.699869in}}%
\pgfpathlineto{\pgfqpoint{2.522690in}{0.699867in}}%
\pgfpathlineto{\pgfqpoint{2.522987in}{0.699865in}}%
\pgfpathlineto{\pgfqpoint{2.523285in}{0.699863in}}%
\pgfpathlineto{\pgfqpoint{2.523582in}{0.699862in}}%
\pgfpathlineto{\pgfqpoint{2.523880in}{0.699860in}}%
\pgfpathlineto{\pgfqpoint{2.524177in}{0.699858in}}%
\pgfpathlineto{\pgfqpoint{2.524475in}{0.699856in}}%
\pgfpathlineto{\pgfqpoint{2.524772in}{0.699854in}}%
\pgfpathlineto{\pgfqpoint{2.525070in}{0.699852in}}%
\pgfpathlineto{\pgfqpoint{2.525367in}{0.699851in}}%
\pgfpathlineto{\pgfqpoint{2.525665in}{0.699849in}}%
\pgfpathlineto{\pgfqpoint{2.525962in}{0.699847in}}%
\pgfpathlineto{\pgfqpoint{2.526259in}{0.699845in}}%
\pgfpathlineto{\pgfqpoint{2.526557in}{0.699843in}}%
\pgfpathlineto{\pgfqpoint{2.526854in}{0.699841in}}%
\pgfpathlineto{\pgfqpoint{2.527152in}{0.699839in}}%
\pgfpathlineto{\pgfqpoint{2.527449in}{0.699838in}}%
\pgfpathlineto{\pgfqpoint{2.527747in}{0.699836in}}%
\pgfpathlineto{\pgfqpoint{2.528044in}{0.699834in}}%
\pgfpathlineto{\pgfqpoint{2.528342in}{0.699832in}}%
\pgfpathlineto{\pgfqpoint{2.528639in}{0.699830in}}%
\pgfpathlineto{\pgfqpoint{2.528937in}{0.699828in}}%
\pgfpathlineto{\pgfqpoint{2.529234in}{0.699827in}}%
\pgfpathlineto{\pgfqpoint{2.529532in}{0.699825in}}%
\pgfpathlineto{\pgfqpoint{2.529829in}{0.699823in}}%
\pgfpathlineto{\pgfqpoint{2.530127in}{0.699821in}}%
\pgfpathlineto{\pgfqpoint{2.530424in}{0.699819in}}%
\pgfpathlineto{\pgfqpoint{2.530722in}{0.699817in}}%
\pgfpathlineto{\pgfqpoint{2.531019in}{0.699815in}}%
\pgfpathlineto{\pgfqpoint{2.531317in}{0.699814in}}%
\pgfpathlineto{\pgfqpoint{2.531614in}{0.699812in}}%
\pgfpathlineto{\pgfqpoint{2.531912in}{0.699810in}}%
\pgfpathlineto{\pgfqpoint{2.532209in}{0.699808in}}%
\pgfpathlineto{\pgfqpoint{2.532507in}{0.699806in}}%
\pgfpathlineto{\pgfqpoint{2.532804in}{0.699804in}}%
\pgfpathlineto{\pgfqpoint{2.533101in}{0.699803in}}%
\pgfpathlineto{\pgfqpoint{2.533399in}{0.699801in}}%
\pgfpathlineto{\pgfqpoint{2.533696in}{0.699799in}}%
\pgfpathlineto{\pgfqpoint{2.533994in}{0.699797in}}%
\pgfpathlineto{\pgfqpoint{2.534291in}{0.699795in}}%
\pgfpathlineto{\pgfqpoint{2.534589in}{0.699793in}}%
\pgfpathlineto{\pgfqpoint{2.534886in}{0.699791in}}%
\pgfpathlineto{\pgfqpoint{2.535184in}{0.699790in}}%
\pgfpathlineto{\pgfqpoint{2.535481in}{0.699785in}}%
\pgfpathlineto{\pgfqpoint{2.535779in}{0.699772in}}%
\pgfpathlineto{\pgfqpoint{2.536076in}{0.699758in}}%
\pgfpathlineto{\pgfqpoint{2.536374in}{0.699743in}}%
\pgfpathlineto{\pgfqpoint{2.536671in}{0.699729in}}%
\pgfpathlineto{\pgfqpoint{2.536969in}{0.699715in}}%
\pgfpathlineto{\pgfqpoint{2.537266in}{0.699700in}}%
\pgfpathlineto{\pgfqpoint{2.537564in}{0.699686in}}%
\pgfpathlineto{\pgfqpoint{2.537861in}{0.699671in}}%
\pgfpathlineto{\pgfqpoint{2.538159in}{0.699657in}}%
\pgfpathlineto{\pgfqpoint{2.538456in}{0.699643in}}%
\pgfpathlineto{\pgfqpoint{2.538754in}{0.699628in}}%
\pgfpathlineto{\pgfqpoint{2.539051in}{0.699614in}}%
\pgfpathlineto{\pgfqpoint{2.539348in}{0.699599in}}%
\pgfpathlineto{\pgfqpoint{2.539646in}{0.699585in}}%
\pgfpathlineto{\pgfqpoint{2.539943in}{0.699571in}}%
\pgfpathlineto{\pgfqpoint{2.540241in}{0.699556in}}%
\pgfpathlineto{\pgfqpoint{2.540538in}{0.699542in}}%
\pgfpathlineto{\pgfqpoint{2.540836in}{0.699528in}}%
\pgfpathlineto{\pgfqpoint{2.541133in}{0.699513in}}%
\pgfpathlineto{\pgfqpoint{2.541431in}{0.699499in}}%
\pgfpathlineto{\pgfqpoint{2.541728in}{0.699484in}}%
\pgfpathlineto{\pgfqpoint{2.542026in}{0.699470in}}%
\pgfpathlineto{\pgfqpoint{2.542323in}{0.699456in}}%
\pgfpathlineto{\pgfqpoint{2.542621in}{0.699441in}}%
\pgfpathlineto{\pgfqpoint{2.542918in}{0.699427in}}%
\pgfpathlineto{\pgfqpoint{2.543216in}{0.699412in}}%
\pgfpathlineto{\pgfqpoint{2.543513in}{0.699398in}}%
\pgfpathlineto{\pgfqpoint{2.543811in}{0.699384in}}%
\pgfpathlineto{\pgfqpoint{2.544108in}{0.699369in}}%
\pgfpathlineto{\pgfqpoint{2.544406in}{0.699355in}}%
\pgfpathlineto{\pgfqpoint{2.544703in}{0.699341in}}%
\pgfpathlineto{\pgfqpoint{2.545001in}{0.699326in}}%
\pgfpathlineto{\pgfqpoint{2.545298in}{0.699312in}}%
\pgfpathlineto{\pgfqpoint{2.545596in}{0.699297in}}%
\pgfpathlineto{\pgfqpoint{2.545893in}{0.699283in}}%
\pgfpathlineto{\pgfqpoint{2.546190in}{0.699269in}}%
\pgfpathlineto{\pgfqpoint{2.546488in}{0.699254in}}%
\pgfpathlineto{\pgfqpoint{2.546785in}{0.699240in}}%
\pgfpathlineto{\pgfqpoint{2.547083in}{0.699225in}}%
\pgfpathlineto{\pgfqpoint{2.547380in}{0.699211in}}%
\pgfpathlineto{\pgfqpoint{2.547678in}{0.699197in}}%
\pgfpathlineto{\pgfqpoint{2.547975in}{0.699182in}}%
\pgfpathlineto{\pgfqpoint{2.548273in}{0.699168in}}%
\pgfpathlineto{\pgfqpoint{2.548570in}{0.699154in}}%
\pgfpathlineto{\pgfqpoint{2.548868in}{0.699139in}}%
\pgfpathlineto{\pgfqpoint{2.549165in}{0.699125in}}%
\pgfpathlineto{\pgfqpoint{2.549463in}{0.699110in}}%
\pgfpathlineto{\pgfqpoint{2.549760in}{0.699096in}}%
\pgfpathlineto{\pgfqpoint{2.550058in}{0.699082in}}%
\pgfpathlineto{\pgfqpoint{2.550355in}{0.699067in}}%
\pgfpathlineto{\pgfqpoint{2.550653in}{0.699053in}}%
\pgfpathlineto{\pgfqpoint{2.550950in}{0.699038in}}%
\pgfpathlineto{\pgfqpoint{2.551248in}{0.699024in}}%
\pgfpathlineto{\pgfqpoint{2.551545in}{0.699010in}}%
\pgfpathlineto{\pgfqpoint{2.551843in}{0.698995in}}%
\pgfpathlineto{\pgfqpoint{2.552140in}{0.698981in}}%
\pgfpathlineto{\pgfqpoint{2.552438in}{0.698967in}}%
\pgfpathlineto{\pgfqpoint{2.552735in}{0.698952in}}%
\pgfpathlineto{\pgfqpoint{2.553032in}{0.698938in}}%
\pgfpathlineto{\pgfqpoint{2.553330in}{0.698923in}}%
\pgfpathlineto{\pgfqpoint{2.553627in}{0.698909in}}%
\pgfpathlineto{\pgfqpoint{2.553925in}{0.698895in}}%
\pgfpathlineto{\pgfqpoint{2.554222in}{0.698880in}}%
\pgfpathlineto{\pgfqpoint{2.554520in}{0.698866in}}%
\pgfpathlineto{\pgfqpoint{2.554817in}{0.698851in}}%
\pgfpathlineto{\pgfqpoint{2.555115in}{0.698837in}}%
\pgfpathlineto{\pgfqpoint{2.555412in}{0.698823in}}%
\pgfpathlineto{\pgfqpoint{2.555710in}{0.698808in}}%
\pgfpathlineto{\pgfqpoint{2.556007in}{0.698794in}}%
\pgfpathlineto{\pgfqpoint{2.556305in}{0.698780in}}%
\pgfpathlineto{\pgfqpoint{2.556602in}{0.698765in}}%
\pgfpathlineto{\pgfqpoint{2.556900in}{0.698751in}}%
\pgfpathlineto{\pgfqpoint{2.557197in}{0.698736in}}%
\pgfpathlineto{\pgfqpoint{2.557495in}{0.698722in}}%
\pgfpathlineto{\pgfqpoint{2.557792in}{0.698707in}}%
\pgfpathlineto{\pgfqpoint{2.558090in}{0.698689in}}%
\pgfpathlineto{\pgfqpoint{2.558387in}{0.698671in}}%
\pgfpathlineto{\pgfqpoint{2.558685in}{0.698653in}}%
\pgfpathlineto{\pgfqpoint{2.558982in}{0.698635in}}%
\pgfpathlineto{\pgfqpoint{2.559279in}{0.698617in}}%
\pgfpathlineto{\pgfqpoint{2.559577in}{0.698598in}}%
\pgfpathlineto{\pgfqpoint{2.559874in}{0.698580in}}%
\pgfpathlineto{\pgfqpoint{2.560172in}{0.698562in}}%
\pgfpathlineto{\pgfqpoint{2.560469in}{0.698544in}}%
\pgfpathlineto{\pgfqpoint{2.560767in}{0.698526in}}%
\pgfpathlineto{\pgfqpoint{2.561064in}{0.698508in}}%
\pgfpathlineto{\pgfqpoint{2.561362in}{0.698490in}}%
\pgfpathlineto{\pgfqpoint{2.561659in}{0.698472in}}%
\pgfpathlineto{\pgfqpoint{2.561957in}{0.698454in}}%
\pgfpathlineto{\pgfqpoint{2.562254in}{0.698436in}}%
\pgfpathlineto{\pgfqpoint{2.562552in}{0.698418in}}%
\pgfpathlineto{\pgfqpoint{2.562849in}{0.698400in}}%
\pgfpathlineto{\pgfqpoint{2.563147in}{0.698382in}}%
\pgfpathlineto{\pgfqpoint{2.563444in}{0.698363in}}%
\pgfpathlineto{\pgfqpoint{2.563742in}{0.698345in}}%
\pgfpathlineto{\pgfqpoint{2.564039in}{0.698327in}}%
\pgfpathlineto{\pgfqpoint{2.564337in}{0.698309in}}%
\pgfpathlineto{\pgfqpoint{2.564634in}{0.698291in}}%
\pgfpathlineto{\pgfqpoint{2.564932in}{0.698273in}}%
\pgfpathlineto{\pgfqpoint{2.565229in}{0.698255in}}%
\pgfpathlineto{\pgfqpoint{2.565527in}{0.698237in}}%
\pgfpathlineto{\pgfqpoint{2.565824in}{0.698219in}}%
\pgfpathlineto{\pgfqpoint{2.566121in}{0.698201in}}%
\pgfpathlineto{\pgfqpoint{2.566419in}{0.698183in}}%
\pgfpathlineto{\pgfqpoint{2.566716in}{0.698165in}}%
\pgfpathlineto{\pgfqpoint{2.567014in}{0.698147in}}%
\pgfpathlineto{\pgfqpoint{2.567311in}{0.698128in}}%
\pgfpathlineto{\pgfqpoint{2.567609in}{0.698110in}}%
\pgfpathlineto{\pgfqpoint{2.567906in}{0.698092in}}%
\pgfpathlineto{\pgfqpoint{2.568204in}{0.698074in}}%
\pgfpathlineto{\pgfqpoint{2.568501in}{0.698056in}}%
\pgfpathlineto{\pgfqpoint{2.568799in}{0.698038in}}%
\pgfpathlineto{\pgfqpoint{2.569096in}{0.698020in}}%
\pgfpathlineto{\pgfqpoint{2.569394in}{0.698002in}}%
\pgfpathlineto{\pgfqpoint{2.569691in}{0.697984in}}%
\pgfpathlineto{\pgfqpoint{2.569989in}{0.697966in}}%
\pgfpathlineto{\pgfqpoint{2.570286in}{0.697948in}}%
\pgfpathlineto{\pgfqpoint{2.570584in}{0.697930in}}%
\pgfpathlineto{\pgfqpoint{2.570881in}{0.697912in}}%
\pgfpathlineto{\pgfqpoint{2.571179in}{0.697893in}}%
\pgfpathlineto{\pgfqpoint{2.571476in}{0.697875in}}%
\pgfpathlineto{\pgfqpoint{2.571774in}{0.697857in}}%
\pgfpathlineto{\pgfqpoint{2.572071in}{0.697839in}}%
\pgfpathlineto{\pgfqpoint{2.572369in}{0.697821in}}%
\pgfpathlineto{\pgfqpoint{2.572666in}{0.697803in}}%
\pgfpathlineto{\pgfqpoint{2.572963in}{0.697785in}}%
\pgfpathlineto{\pgfqpoint{2.573261in}{0.697767in}}%
\pgfpathlineto{\pgfqpoint{2.573558in}{0.697749in}}%
\pgfpathlineto{\pgfqpoint{2.573856in}{0.697731in}}%
\pgfpathlineto{\pgfqpoint{2.574153in}{0.697713in}}%
\pgfpathlineto{\pgfqpoint{2.574451in}{0.697695in}}%
\pgfpathlineto{\pgfqpoint{2.574748in}{0.697677in}}%
\pgfpathlineto{\pgfqpoint{2.575046in}{0.697658in}}%
\pgfpathlineto{\pgfqpoint{2.575343in}{0.697640in}}%
\pgfpathlineto{\pgfqpoint{2.575641in}{0.697622in}}%
\pgfpathlineto{\pgfqpoint{2.575938in}{0.697604in}}%
\pgfpathlineto{\pgfqpoint{2.576236in}{0.697586in}}%
\pgfpathlineto{\pgfqpoint{2.576533in}{0.697568in}}%
\pgfpathlineto{\pgfqpoint{2.576831in}{0.697550in}}%
\pgfpathlineto{\pgfqpoint{2.577128in}{0.697532in}}%
\pgfpathlineto{\pgfqpoint{2.577426in}{0.697514in}}%
\pgfpathlineto{\pgfqpoint{2.577723in}{0.697496in}}%
\pgfpathlineto{\pgfqpoint{2.578021in}{0.697478in}}%
\pgfpathlineto{\pgfqpoint{2.578318in}{0.697460in}}%
\pgfpathlineto{\pgfqpoint{2.578616in}{0.697442in}}%
\pgfpathlineto{\pgfqpoint{2.578913in}{0.697423in}}%
\pgfpathlineto{\pgfqpoint{2.579210in}{0.697405in}}%
\pgfpathlineto{\pgfqpoint{2.579508in}{0.697387in}}%
\pgfpathlineto{\pgfqpoint{2.579805in}{0.697369in}}%
\pgfpathlineto{\pgfqpoint{2.580103in}{0.697351in}}%
\pgfpathlineto{\pgfqpoint{2.580400in}{0.697333in}}%
\pgfpathlineto{\pgfqpoint{2.580698in}{0.697315in}}%
\pgfpathlineto{\pgfqpoint{2.580995in}{0.697297in}}%
\pgfpathlineto{\pgfqpoint{2.581293in}{0.697279in}}%
\pgfpathlineto{\pgfqpoint{2.581590in}{0.697261in}}%
\pgfpathlineto{\pgfqpoint{2.581888in}{0.697243in}}%
\pgfpathlineto{\pgfqpoint{2.582185in}{0.697225in}}%
\pgfpathlineto{\pgfqpoint{2.582483in}{0.697207in}}%
\pgfpathlineto{\pgfqpoint{2.582780in}{0.697188in}}%
\pgfpathlineto{\pgfqpoint{2.583078in}{0.697170in}}%
\pgfpathlineto{\pgfqpoint{2.583375in}{0.697152in}}%
\pgfpathlineto{\pgfqpoint{2.583673in}{0.697134in}}%
\pgfpathlineto{\pgfqpoint{2.583970in}{0.697116in}}%
\pgfpathlineto{\pgfqpoint{2.584268in}{0.697094in}}%
\pgfpathlineto{\pgfqpoint{2.584565in}{0.697005in}}%
\pgfpathlineto{\pgfqpoint{2.584863in}{0.696891in}}%
\pgfpathlineto{\pgfqpoint{2.585160in}{0.696777in}}%
\pgfpathlineto{\pgfqpoint{2.585458in}{0.696663in}}%
\pgfpathlineto{\pgfqpoint{2.585755in}{0.696549in}}%
\pgfpathlineto{\pgfqpoint{2.586052in}{0.696435in}}%
\pgfpathlineto{\pgfqpoint{2.586350in}{0.696320in}}%
\pgfpathlineto{\pgfqpoint{2.586647in}{0.696206in}}%
\pgfpathlineto{\pgfqpoint{2.586945in}{0.696092in}}%
\pgfpathlineto{\pgfqpoint{2.587242in}{0.695978in}}%
\pgfpathlineto{\pgfqpoint{2.587540in}{0.695864in}}%
\pgfpathlineto{\pgfqpoint{2.587837in}{0.695749in}}%
\pgfpathlineto{\pgfqpoint{2.588135in}{0.695635in}}%
\pgfpathlineto{\pgfqpoint{2.588432in}{0.695521in}}%
\pgfpathlineto{\pgfqpoint{2.588730in}{0.695407in}}%
\pgfpathlineto{\pgfqpoint{2.589027in}{0.695293in}}%
\pgfpathlineto{\pgfqpoint{2.589325in}{0.695178in}}%
\pgfpathlineto{\pgfqpoint{2.589622in}{0.695064in}}%
\pgfpathlineto{\pgfqpoint{2.589920in}{0.694950in}}%
\pgfpathlineto{\pgfqpoint{2.590217in}{0.694836in}}%
\pgfpathlineto{\pgfqpoint{2.590515in}{0.694722in}}%
\pgfpathlineto{\pgfqpoint{2.590812in}{0.694607in}}%
\pgfpathlineto{\pgfqpoint{2.591110in}{0.694493in}}%
\pgfpathlineto{\pgfqpoint{2.591407in}{0.694379in}}%
\pgfpathlineto{\pgfqpoint{2.591705in}{0.694265in}}%
\pgfpathlineto{\pgfqpoint{2.592002in}{0.694151in}}%
\pgfpathlineto{\pgfqpoint{2.592300in}{0.694037in}}%
\pgfpathlineto{\pgfqpoint{2.592597in}{0.693922in}}%
\pgfpathlineto{\pgfqpoint{2.592894in}{0.693808in}}%
\pgfpathlineto{\pgfqpoint{2.593192in}{0.693702in}}%
\pgfpathlineto{\pgfqpoint{2.593489in}{0.693674in}}%
\pgfpathlineto{\pgfqpoint{2.593787in}{0.693665in}}%
\pgfpathlineto{\pgfqpoint{2.594084in}{0.693656in}}%
\pgfpathlineto{\pgfqpoint{2.594382in}{0.693648in}}%
\pgfpathlineto{\pgfqpoint{2.594679in}{0.693639in}}%
\pgfpathlineto{\pgfqpoint{2.594977in}{0.693630in}}%
\pgfpathlineto{\pgfqpoint{2.595274in}{0.693621in}}%
\pgfpathlineto{\pgfqpoint{2.595572in}{0.693613in}}%
\pgfpathlineto{\pgfqpoint{2.595869in}{0.693604in}}%
\pgfpathlineto{\pgfqpoint{2.596167in}{0.693595in}}%
\pgfpathlineto{\pgfqpoint{2.596464in}{0.693587in}}%
\pgfpathlineto{\pgfqpoint{2.596762in}{0.693578in}}%
\pgfpathlineto{\pgfqpoint{2.597059in}{0.693569in}}%
\pgfpathlineto{\pgfqpoint{2.597357in}{0.693560in}}%
\pgfpathlineto{\pgfqpoint{2.597654in}{0.693552in}}%
\pgfpathlineto{\pgfqpoint{2.597952in}{0.693543in}}%
\pgfpathlineto{\pgfqpoint{2.598249in}{0.693534in}}%
\pgfpathlineto{\pgfqpoint{2.598547in}{0.693526in}}%
\pgfpathlineto{\pgfqpoint{2.598844in}{0.693517in}}%
\pgfpathlineto{\pgfqpoint{2.599141in}{0.693508in}}%
\pgfpathlineto{\pgfqpoint{2.599439in}{0.693499in}}%
\pgfpathlineto{\pgfqpoint{2.599736in}{0.693491in}}%
\pgfpathlineto{\pgfqpoint{2.600034in}{0.693482in}}%
\pgfpathlineto{\pgfqpoint{2.600331in}{0.693473in}}%
\pgfpathlineto{\pgfqpoint{2.600629in}{0.693465in}}%
\pgfpathlineto{\pgfqpoint{2.600926in}{0.693456in}}%
\pgfpathlineto{\pgfqpoint{2.601224in}{0.693447in}}%
\pgfpathlineto{\pgfqpoint{2.601521in}{0.693438in}}%
\pgfpathlineto{\pgfqpoint{2.601819in}{0.693430in}}%
\pgfpathlineto{\pgfqpoint{2.602116in}{0.693421in}}%
\pgfpathlineto{\pgfqpoint{2.602414in}{0.693412in}}%
\pgfpathlineto{\pgfqpoint{2.602711in}{0.693403in}}%
\pgfpathlineto{\pgfqpoint{2.603009in}{0.693395in}}%
\pgfpathlineto{\pgfqpoint{2.603306in}{0.693386in}}%
\pgfpathlineto{\pgfqpoint{2.603604in}{0.693377in}}%
\pgfpathlineto{\pgfqpoint{2.603901in}{0.693369in}}%
\pgfpathlineto{\pgfqpoint{2.604199in}{0.693360in}}%
\pgfpathlineto{\pgfqpoint{2.604496in}{0.693351in}}%
\pgfpathlineto{\pgfqpoint{2.604794in}{0.693342in}}%
\pgfpathlineto{\pgfqpoint{2.605091in}{0.693334in}}%
\pgfpathlineto{\pgfqpoint{2.605389in}{0.693325in}}%
\pgfpathlineto{\pgfqpoint{2.605686in}{0.693316in}}%
\pgfpathlineto{\pgfqpoint{2.605983in}{0.693308in}}%
\pgfpathlineto{\pgfqpoint{2.606281in}{0.693299in}}%
\pgfpathlineto{\pgfqpoint{2.606578in}{0.693290in}}%
\pgfpathlineto{\pgfqpoint{2.606876in}{0.693281in}}%
\pgfpathlineto{\pgfqpoint{2.607173in}{0.693273in}}%
\pgfpathlineto{\pgfqpoint{2.607471in}{0.693264in}}%
\pgfpathlineto{\pgfqpoint{2.607768in}{0.693255in}}%
\pgfpathlineto{\pgfqpoint{2.608066in}{0.693247in}}%
\pgfpathlineto{\pgfqpoint{2.608363in}{0.693238in}}%
\pgfpathlineto{\pgfqpoint{2.608661in}{0.693229in}}%
\pgfpathlineto{\pgfqpoint{2.608958in}{0.693220in}}%
\pgfpathlineto{\pgfqpoint{2.609256in}{0.693212in}}%
\pgfpathlineto{\pgfqpoint{2.609553in}{0.693203in}}%
\pgfpathlineto{\pgfqpoint{2.609851in}{0.693194in}}%
\pgfpathlineto{\pgfqpoint{2.610148in}{0.693185in}}%
\pgfpathlineto{\pgfqpoint{2.610446in}{0.693177in}}%
\pgfpathlineto{\pgfqpoint{2.610743in}{0.693168in}}%
\pgfpathlineto{\pgfqpoint{2.611041in}{0.693159in}}%
\pgfpathlineto{\pgfqpoint{2.611338in}{0.693151in}}%
\pgfpathlineto{\pgfqpoint{2.611636in}{0.693142in}}%
\pgfpathlineto{\pgfqpoint{2.611933in}{0.693133in}}%
\pgfpathlineto{\pgfqpoint{2.612231in}{0.693124in}}%
\pgfpathlineto{\pgfqpoint{2.612528in}{0.693116in}}%
\pgfpathlineto{\pgfqpoint{2.612825in}{0.693107in}}%
\pgfpathlineto{\pgfqpoint{2.613123in}{0.693098in}}%
\pgfpathlineto{\pgfqpoint{2.613420in}{0.693090in}}%
\pgfpathlineto{\pgfqpoint{2.613718in}{0.693081in}}%
\pgfpathlineto{\pgfqpoint{2.614015in}{0.693072in}}%
\pgfpathlineto{\pgfqpoint{2.614313in}{0.693063in}}%
\pgfpathlineto{\pgfqpoint{2.614610in}{0.693055in}}%
\pgfpathlineto{\pgfqpoint{2.614908in}{0.693046in}}%
\pgfpathlineto{\pgfqpoint{2.615205in}{0.693037in}}%
\pgfpathlineto{\pgfqpoint{2.615503in}{0.693029in}}%
\pgfpathlineto{\pgfqpoint{2.615800in}{0.693020in}}%
\pgfpathlineto{\pgfqpoint{2.616098in}{0.693011in}}%
\pgfpathlineto{\pgfqpoint{2.616395in}{0.693002in}}%
\pgfpathlineto{\pgfqpoint{2.616693in}{0.692994in}}%
\pgfpathlineto{\pgfqpoint{2.616990in}{0.692985in}}%
\pgfpathlineto{\pgfqpoint{2.617288in}{0.692976in}}%
\pgfpathlineto{\pgfqpoint{2.617585in}{0.692967in}}%
\pgfpathlineto{\pgfqpoint{2.617883in}{0.692959in}}%
\pgfpathlineto{\pgfqpoint{2.618180in}{0.692950in}}%
\pgfpathlineto{\pgfqpoint{2.618478in}{0.692941in}}%
\pgfpathlineto{\pgfqpoint{2.618775in}{0.692933in}}%
\pgfpathlineto{\pgfqpoint{2.619072in}{0.692924in}}%
\pgfpathlineto{\pgfqpoint{2.619370in}{0.692915in}}%
\pgfpathlineto{\pgfqpoint{2.619667in}{0.692906in}}%
\pgfpathlineto{\pgfqpoint{2.619965in}{0.692898in}}%
\pgfpathlineto{\pgfqpoint{2.620262in}{0.692889in}}%
\pgfpathlineto{\pgfqpoint{2.620560in}{0.692880in}}%
\pgfpathlineto{\pgfqpoint{2.620857in}{0.692872in}}%
\pgfpathlineto{\pgfqpoint{2.621155in}{0.692863in}}%
\pgfpathlineto{\pgfqpoint{2.621452in}{0.692854in}}%
\pgfpathlineto{\pgfqpoint{2.621750in}{0.692845in}}%
\pgfpathlineto{\pgfqpoint{2.622047in}{0.692837in}}%
\pgfpathlineto{\pgfqpoint{2.622345in}{0.692828in}}%
\pgfpathlineto{\pgfqpoint{2.622642in}{0.692819in}}%
\pgfpathlineto{\pgfqpoint{2.622940in}{0.692811in}}%
\pgfpathlineto{\pgfqpoint{2.623237in}{0.692802in}}%
\pgfpathlineto{\pgfqpoint{2.623535in}{0.692793in}}%
\pgfpathlineto{\pgfqpoint{2.623832in}{0.692784in}}%
\pgfpathlineto{\pgfqpoint{2.624130in}{0.692776in}}%
\pgfpathlineto{\pgfqpoint{2.624427in}{0.692767in}}%
\pgfpathlineto{\pgfqpoint{2.624725in}{0.692758in}}%
\pgfpathlineto{\pgfqpoint{2.625022in}{0.692749in}}%
\pgfpathlineto{\pgfqpoint{2.625320in}{0.692741in}}%
\pgfpathlineto{\pgfqpoint{2.625617in}{0.692732in}}%
\pgfpathlineto{\pgfqpoint{2.625914in}{0.692723in}}%
\pgfpathlineto{\pgfqpoint{2.626212in}{0.692715in}}%
\pgfpathlineto{\pgfqpoint{2.626509in}{0.692706in}}%
\pgfpathlineto{\pgfqpoint{2.626807in}{0.692697in}}%
\pgfpathlineto{\pgfqpoint{2.627104in}{0.692688in}}%
\pgfpathlineto{\pgfqpoint{2.627402in}{0.692680in}}%
\pgfpathlineto{\pgfqpoint{2.627699in}{0.692671in}}%
\pgfpathlineto{\pgfqpoint{2.627997in}{0.692662in}}%
\pgfpathlineto{\pgfqpoint{2.628294in}{0.692654in}}%
\pgfpathlineto{\pgfqpoint{2.628592in}{0.692645in}}%
\pgfpathlineto{\pgfqpoint{2.628889in}{0.692636in}}%
\pgfpathlineto{\pgfqpoint{2.629187in}{0.692627in}}%
\pgfpathlineto{\pgfqpoint{2.629484in}{0.692619in}}%
\pgfpathlineto{\pgfqpoint{2.629782in}{0.692610in}}%
\pgfpathlineto{\pgfqpoint{2.630079in}{0.692601in}}%
\pgfpathlineto{\pgfqpoint{2.630377in}{0.692593in}}%
\pgfpathlineto{\pgfqpoint{2.630674in}{0.692584in}}%
\pgfpathlineto{\pgfqpoint{2.630972in}{0.692575in}}%
\pgfpathlineto{\pgfqpoint{2.631269in}{0.692566in}}%
\pgfpathlineto{\pgfqpoint{2.631567in}{0.692558in}}%
\pgfpathlineto{\pgfqpoint{2.631864in}{0.692549in}}%
\pgfpathlineto{\pgfqpoint{2.632162in}{0.692540in}}%
\pgfpathlineto{\pgfqpoint{2.632459in}{0.692531in}}%
\pgfpathlineto{\pgfqpoint{2.632756in}{0.692523in}}%
\pgfpathlineto{\pgfqpoint{2.633054in}{0.692514in}}%
\pgfpathlineto{\pgfqpoint{2.633351in}{0.692505in}}%
\pgfpathlineto{\pgfqpoint{2.633649in}{0.692497in}}%
\pgfpathlineto{\pgfqpoint{2.633946in}{0.692488in}}%
\pgfpathlineto{\pgfqpoint{2.634244in}{0.692479in}}%
\pgfpathlineto{\pgfqpoint{2.634541in}{0.692470in}}%
\pgfpathlineto{\pgfqpoint{2.634839in}{0.692462in}}%
\pgfpathlineto{\pgfqpoint{2.635136in}{0.692453in}}%
\pgfpathlineto{\pgfqpoint{2.635434in}{0.692444in}}%
\pgfpathlineto{\pgfqpoint{2.635731in}{0.692436in}}%
\pgfpathlineto{\pgfqpoint{2.636029in}{0.692427in}}%
\pgfpathlineto{\pgfqpoint{2.636326in}{0.692418in}}%
\pgfpathlineto{\pgfqpoint{2.636624in}{0.692409in}}%
\pgfpathlineto{\pgfqpoint{2.636921in}{0.692401in}}%
\pgfpathlineto{\pgfqpoint{2.637219in}{0.692392in}}%
\pgfpathlineto{\pgfqpoint{2.637516in}{0.692383in}}%
\pgfpathlineto{\pgfqpoint{2.637814in}{0.692374in}}%
\pgfpathlineto{\pgfqpoint{2.638111in}{0.692366in}}%
\pgfpathlineto{\pgfqpoint{2.638409in}{0.692357in}}%
\pgfpathlineto{\pgfqpoint{2.638706in}{0.692348in}}%
\pgfpathlineto{\pgfqpoint{2.639003in}{0.692340in}}%
\pgfpathlineto{\pgfqpoint{2.639301in}{0.692331in}}%
\pgfpathlineto{\pgfqpoint{2.639598in}{0.692322in}}%
\pgfpathlineto{\pgfqpoint{2.639896in}{0.692313in}}%
\pgfpathlineto{\pgfqpoint{2.640193in}{0.692305in}}%
\pgfpathlineto{\pgfqpoint{2.640491in}{0.692296in}}%
\pgfpathlineto{\pgfqpoint{2.640788in}{0.692287in}}%
\pgfpathlineto{\pgfqpoint{2.641086in}{0.692279in}}%
\pgfpathlineto{\pgfqpoint{2.641383in}{0.692270in}}%
\pgfpathlineto{\pgfqpoint{2.641681in}{0.692261in}}%
\pgfpathlineto{\pgfqpoint{2.641978in}{0.692252in}}%
\pgfpathlineto{\pgfqpoint{2.642276in}{0.692244in}}%
\pgfpathlineto{\pgfqpoint{2.642573in}{0.692235in}}%
\pgfpathlineto{\pgfqpoint{2.642871in}{0.692226in}}%
\pgfpathlineto{\pgfqpoint{2.643168in}{0.692218in}}%
\pgfpathlineto{\pgfqpoint{2.643466in}{0.692209in}}%
\pgfpathlineto{\pgfqpoint{2.643763in}{0.692200in}}%
\pgfpathlineto{\pgfqpoint{2.644061in}{0.692191in}}%
\pgfpathlineto{\pgfqpoint{2.644358in}{0.692183in}}%
\pgfpathlineto{\pgfqpoint{2.644656in}{0.692174in}}%
\pgfpathlineto{\pgfqpoint{2.644953in}{0.692165in}}%
\pgfpathlineto{\pgfqpoint{2.645251in}{0.692156in}}%
\pgfpathlineto{\pgfqpoint{2.645548in}{0.692148in}}%
\pgfpathlineto{\pgfqpoint{2.645845in}{0.692139in}}%
\pgfpathlineto{\pgfqpoint{2.646143in}{0.692130in}}%
\pgfpathlineto{\pgfqpoint{2.646440in}{0.692122in}}%
\pgfpathlineto{\pgfqpoint{2.646738in}{0.692113in}}%
\pgfpathlineto{\pgfqpoint{2.647035in}{0.692104in}}%
\pgfpathlineto{\pgfqpoint{2.647333in}{0.692095in}}%
\pgfpathlineto{\pgfqpoint{2.647630in}{0.692087in}}%
\pgfpathlineto{\pgfqpoint{2.647928in}{0.692078in}}%
\pgfpathlineto{\pgfqpoint{2.648225in}{0.692069in}}%
\pgfpathlineto{\pgfqpoint{2.648523in}{0.692061in}}%
\pgfpathlineto{\pgfqpoint{2.648820in}{0.692052in}}%
\pgfpathlineto{\pgfqpoint{2.649118in}{0.692043in}}%
\pgfpathlineto{\pgfqpoint{2.649415in}{0.692034in}}%
\pgfpathlineto{\pgfqpoint{2.649713in}{0.692026in}}%
\pgfpathlineto{\pgfqpoint{2.650010in}{0.692017in}}%
\pgfpathlineto{\pgfqpoint{2.650308in}{0.692008in}}%
\pgfpathlineto{\pgfqpoint{2.650605in}{0.692000in}}%
\pgfpathlineto{\pgfqpoint{2.650903in}{0.691991in}}%
\pgfpathlineto{\pgfqpoint{2.651200in}{0.691982in}}%
\pgfpathlineto{\pgfqpoint{2.651498in}{0.691973in}}%
\pgfpathlineto{\pgfqpoint{2.651795in}{0.691965in}}%
\pgfpathlineto{\pgfqpoint{2.652093in}{0.691956in}}%
\pgfpathlineto{\pgfqpoint{2.652390in}{0.691947in}}%
\pgfpathlineto{\pgfqpoint{2.652687in}{0.691938in}}%
\pgfpathlineto{\pgfqpoint{2.652985in}{0.691930in}}%
\pgfpathlineto{\pgfqpoint{2.653282in}{0.691921in}}%
\pgfpathlineto{\pgfqpoint{2.653580in}{0.691912in}}%
\pgfpathlineto{\pgfqpoint{2.653877in}{0.691904in}}%
\pgfpathlineto{\pgfqpoint{2.654175in}{0.691895in}}%
\pgfpathlineto{\pgfqpoint{2.654472in}{0.691886in}}%
\pgfpathlineto{\pgfqpoint{2.654770in}{0.691877in}}%
\pgfpathlineto{\pgfqpoint{2.655067in}{0.691869in}}%
\pgfpathlineto{\pgfqpoint{2.655365in}{0.691860in}}%
\pgfpathlineto{\pgfqpoint{2.655662in}{0.691851in}}%
\pgfpathlineto{\pgfqpoint{2.655960in}{0.691843in}}%
\pgfpathlineto{\pgfqpoint{2.656257in}{0.691834in}}%
\pgfpathlineto{\pgfqpoint{2.656555in}{0.691825in}}%
\pgfpathlineto{\pgfqpoint{2.656852in}{0.691816in}}%
\pgfpathlineto{\pgfqpoint{2.657150in}{0.691808in}}%
\pgfpathlineto{\pgfqpoint{2.657447in}{0.691799in}}%
\pgfpathlineto{\pgfqpoint{2.657745in}{0.691790in}}%
\pgfpathlineto{\pgfqpoint{2.658042in}{0.691782in}}%
\pgfpathlineto{\pgfqpoint{2.658340in}{0.691773in}}%
\pgfpathlineto{\pgfqpoint{2.658637in}{0.691764in}}%
\pgfpathlineto{\pgfqpoint{2.658934in}{0.691755in}}%
\pgfpathlineto{\pgfqpoint{2.659232in}{0.691747in}}%
\pgfpathlineto{\pgfqpoint{2.659529in}{0.691738in}}%
\pgfpathlineto{\pgfqpoint{2.659827in}{0.691729in}}%
\pgfpathlineto{\pgfqpoint{2.660124in}{0.691720in}}%
\pgfpathlineto{\pgfqpoint{2.660422in}{0.691712in}}%
\pgfpathlineto{\pgfqpoint{2.660719in}{0.691703in}}%
\pgfpathlineto{\pgfqpoint{2.661017in}{0.691694in}}%
\pgfpathlineto{\pgfqpoint{2.661314in}{0.691686in}}%
\pgfpathlineto{\pgfqpoint{2.661612in}{0.691677in}}%
\pgfpathlineto{\pgfqpoint{2.661909in}{0.691668in}}%
\pgfpathlineto{\pgfqpoint{2.662207in}{0.691659in}}%
\pgfpathlineto{\pgfqpoint{2.662504in}{0.691651in}}%
\pgfpathlineto{\pgfqpoint{2.662802in}{0.691642in}}%
\pgfpathlineto{\pgfqpoint{2.663099in}{0.691633in}}%
\pgfpathlineto{\pgfqpoint{2.663397in}{0.691625in}}%
\pgfpathlineto{\pgfqpoint{2.663694in}{0.691616in}}%
\pgfpathlineto{\pgfqpoint{2.663992in}{0.691607in}}%
\pgfpathlineto{\pgfqpoint{2.664289in}{0.691598in}}%
\pgfpathlineto{\pgfqpoint{2.664587in}{0.691590in}}%
\pgfpathlineto{\pgfqpoint{2.664884in}{0.691581in}}%
\pgfpathlineto{\pgfqpoint{2.665182in}{0.691572in}}%
\pgfpathlineto{\pgfqpoint{2.665479in}{0.691564in}}%
\pgfpathlineto{\pgfqpoint{2.665776in}{0.691555in}}%
\pgfpathlineto{\pgfqpoint{2.666074in}{0.691546in}}%
\pgfpathlineto{\pgfqpoint{2.666371in}{0.691537in}}%
\pgfpathlineto{\pgfqpoint{2.666669in}{0.691529in}}%
\pgfpathlineto{\pgfqpoint{2.666966in}{0.691520in}}%
\pgfpathlineto{\pgfqpoint{2.667264in}{0.691511in}}%
\pgfpathlineto{\pgfqpoint{2.667561in}{0.691502in}}%
\pgfpathlineto{\pgfqpoint{2.667859in}{0.691494in}}%
\pgfpathlineto{\pgfqpoint{2.668156in}{0.691485in}}%
\pgfpathlineto{\pgfqpoint{2.668454in}{0.691476in}}%
\pgfpathlineto{\pgfqpoint{2.668751in}{0.691468in}}%
\pgfpathlineto{\pgfqpoint{2.669049in}{0.691459in}}%
\pgfpathlineto{\pgfqpoint{2.669346in}{0.691450in}}%
\pgfpathlineto{\pgfqpoint{2.669644in}{0.691441in}}%
\pgfpathlineto{\pgfqpoint{2.669941in}{0.691433in}}%
\pgfpathlineto{\pgfqpoint{2.670239in}{0.691424in}}%
\pgfpathlineto{\pgfqpoint{2.670536in}{0.691415in}}%
\pgfpathlineto{\pgfqpoint{2.670834in}{0.691407in}}%
\pgfpathlineto{\pgfqpoint{2.671131in}{0.691398in}}%
\pgfpathlineto{\pgfqpoint{2.671429in}{0.691389in}}%
\pgfpathlineto{\pgfqpoint{2.671726in}{0.691380in}}%
\pgfpathlineto{\pgfqpoint{2.672024in}{0.691372in}}%
\pgfpathlineto{\pgfqpoint{2.672321in}{0.691363in}}%
\pgfpathlineto{\pgfqpoint{2.672618in}{0.691354in}}%
\pgfpathlineto{\pgfqpoint{2.672916in}{0.691346in}}%
\pgfpathlineto{\pgfqpoint{2.673213in}{0.691337in}}%
\pgfpathlineto{\pgfqpoint{2.673511in}{0.691328in}}%
\pgfpathlineto{\pgfqpoint{2.673808in}{0.691319in}}%
\pgfpathlineto{\pgfqpoint{2.674106in}{0.691311in}}%
\pgfpathlineto{\pgfqpoint{2.674403in}{0.691302in}}%
\pgfpathlineto{\pgfqpoint{2.674701in}{0.691293in}}%
\pgfpathlineto{\pgfqpoint{2.674998in}{0.691284in}}%
\pgfpathlineto{\pgfqpoint{2.675296in}{0.691276in}}%
\pgfpathlineto{\pgfqpoint{2.675593in}{0.691267in}}%
\pgfpathlineto{\pgfqpoint{2.675891in}{0.691258in}}%
\pgfpathlineto{\pgfqpoint{2.676188in}{0.691250in}}%
\pgfpathlineto{\pgfqpoint{2.676486in}{0.691241in}}%
\pgfpathlineto{\pgfqpoint{2.676783in}{0.691232in}}%
\pgfpathlineto{\pgfqpoint{2.677081in}{0.691223in}}%
\pgfpathlineto{\pgfqpoint{2.677378in}{0.691215in}}%
\pgfpathlineto{\pgfqpoint{2.677676in}{0.691206in}}%
\pgfpathlineto{\pgfqpoint{2.677973in}{0.691197in}}%
\pgfpathlineto{\pgfqpoint{2.678271in}{0.691189in}}%
\pgfpathlineto{\pgfqpoint{2.678568in}{0.691180in}}%
\pgfpathlineto{\pgfqpoint{2.678866in}{0.691171in}}%
\pgfpathlineto{\pgfqpoint{2.679163in}{0.691162in}}%
\pgfpathlineto{\pgfqpoint{2.679460in}{0.691154in}}%
\pgfpathlineto{\pgfqpoint{2.679758in}{0.691145in}}%
\pgfpathlineto{\pgfqpoint{2.680055in}{0.691136in}}%
\pgfpathlineto{\pgfqpoint{2.680353in}{0.691128in}}%
\pgfpathlineto{\pgfqpoint{2.680650in}{0.691119in}}%
\pgfpathlineto{\pgfqpoint{2.680948in}{0.691110in}}%
\pgfpathlineto{\pgfqpoint{2.681245in}{0.691101in}}%
\pgfpathlineto{\pgfqpoint{2.681543in}{0.691093in}}%
\pgfpathlineto{\pgfqpoint{2.681840in}{0.691084in}}%
\pgfpathlineto{\pgfqpoint{2.682138in}{0.691075in}}%
\pgfpathlineto{\pgfqpoint{2.682435in}{0.691066in}}%
\pgfpathlineto{\pgfqpoint{2.682733in}{0.691058in}}%
\pgfpathlineto{\pgfqpoint{2.683030in}{0.691049in}}%
\pgfpathlineto{\pgfqpoint{2.683328in}{0.691040in}}%
\pgfpathlineto{\pgfqpoint{2.683625in}{0.691032in}}%
\pgfpathlineto{\pgfqpoint{2.683923in}{0.691023in}}%
\pgfpathlineto{\pgfqpoint{2.684220in}{0.691014in}}%
\pgfpathlineto{\pgfqpoint{2.684518in}{0.691005in}}%
\pgfpathlineto{\pgfqpoint{2.684815in}{0.690997in}}%
\pgfpathlineto{\pgfqpoint{2.685113in}{0.690988in}}%
\pgfpathlineto{\pgfqpoint{2.685410in}{0.690979in}}%
\pgfpathlineto{\pgfqpoint{2.685707in}{0.690971in}}%
\pgfpathlineto{\pgfqpoint{2.686005in}{0.690962in}}%
\pgfpathlineto{\pgfqpoint{2.686302in}{0.690953in}}%
\pgfpathlineto{\pgfqpoint{2.686600in}{0.690944in}}%
\pgfpathlineto{\pgfqpoint{2.686897in}{0.690936in}}%
\pgfpathlineto{\pgfqpoint{2.687195in}{0.690927in}}%
\pgfpathlineto{\pgfqpoint{2.687492in}{0.690918in}}%
\pgfpathlineto{\pgfqpoint{2.687790in}{0.690910in}}%
\pgfpathlineto{\pgfqpoint{2.688087in}{0.690901in}}%
\pgfpathlineto{\pgfqpoint{2.688385in}{0.690892in}}%
\pgfpathlineto{\pgfqpoint{2.688682in}{0.690883in}}%
\pgfpathlineto{\pgfqpoint{2.688980in}{0.690875in}}%
\pgfpathlineto{\pgfqpoint{2.689277in}{0.690866in}}%
\pgfpathlineto{\pgfqpoint{2.689575in}{0.690857in}}%
\pgfpathlineto{\pgfqpoint{2.689872in}{0.690848in}}%
\pgfpathlineto{\pgfqpoint{2.690170in}{0.690840in}}%
\pgfpathlineto{\pgfqpoint{2.690467in}{0.690831in}}%
\pgfpathlineto{\pgfqpoint{2.690765in}{0.690822in}}%
\pgfpathlineto{\pgfqpoint{2.691062in}{0.690814in}}%
\pgfpathlineto{\pgfqpoint{2.691360in}{0.690618in}}%
\pgfpathlineto{\pgfqpoint{2.691657in}{0.690105in}}%
\pgfpathlineto{\pgfqpoint{2.691955in}{0.689742in}}%
\pgfpathlineto{\pgfqpoint{2.692252in}{0.689726in}}%
\pgfpathlineto{\pgfqpoint{2.692549in}{0.689722in}}%
\pgfpathlineto{\pgfqpoint{2.692847in}{0.689718in}}%
\pgfpathlineto{\pgfqpoint{2.693144in}{0.689714in}}%
\pgfpathlineto{\pgfqpoint{2.693442in}{0.689711in}}%
\pgfpathlineto{\pgfqpoint{2.693739in}{0.689707in}}%
\pgfpathlineto{\pgfqpoint{2.694037in}{0.689703in}}%
\pgfpathlineto{\pgfqpoint{2.694334in}{0.689699in}}%
\pgfpathlineto{\pgfqpoint{2.694632in}{0.689695in}}%
\pgfpathlineto{\pgfqpoint{2.694929in}{0.689692in}}%
\pgfpathlineto{\pgfqpoint{2.695227in}{0.689688in}}%
\pgfpathlineto{\pgfqpoint{2.695524in}{0.689684in}}%
\pgfpathlineto{\pgfqpoint{2.695822in}{0.689680in}}%
\pgfpathlineto{\pgfqpoint{2.696119in}{0.689677in}}%
\pgfpathlineto{\pgfqpoint{2.696417in}{0.689673in}}%
\pgfpathlineto{\pgfqpoint{2.696714in}{0.689669in}}%
\pgfpathlineto{\pgfqpoint{2.697012in}{0.689665in}}%
\pgfpathlineto{\pgfqpoint{2.697309in}{0.689662in}}%
\pgfpathlineto{\pgfqpoint{2.697607in}{0.689658in}}%
\pgfpathlineto{\pgfqpoint{2.697904in}{0.689654in}}%
\pgfpathlineto{\pgfqpoint{2.698202in}{0.689650in}}%
\pgfpathlineto{\pgfqpoint{2.698499in}{0.689647in}}%
\pgfpathlineto{\pgfqpoint{2.698797in}{0.689643in}}%
\pgfpathlineto{\pgfqpoint{2.699094in}{0.689639in}}%
\pgfpathlineto{\pgfqpoint{2.699391in}{0.689635in}}%
\pgfpathlineto{\pgfqpoint{2.699689in}{0.689632in}}%
\pgfpathlineto{\pgfqpoint{2.699986in}{0.689628in}}%
\pgfpathlineto{\pgfqpoint{2.700284in}{0.689624in}}%
\pgfpathlineto{\pgfqpoint{2.700581in}{0.689620in}}%
\pgfpathlineto{\pgfqpoint{2.700879in}{0.689616in}}%
\pgfpathlineto{\pgfqpoint{2.701176in}{0.689613in}}%
\pgfpathlineto{\pgfqpoint{2.701474in}{0.689609in}}%
\pgfpathlineto{\pgfqpoint{2.701771in}{0.689605in}}%
\pgfpathlineto{\pgfqpoint{2.702069in}{0.689601in}}%
\pgfpathlineto{\pgfqpoint{2.702366in}{0.689598in}}%
\pgfpathlineto{\pgfqpoint{2.702664in}{0.689594in}}%
\pgfpathlineto{\pgfqpoint{2.702961in}{0.689590in}}%
\pgfpathlineto{\pgfqpoint{2.703259in}{0.689586in}}%
\pgfpathlineto{\pgfqpoint{2.703556in}{0.689583in}}%
\pgfpathlineto{\pgfqpoint{2.703854in}{0.689579in}}%
\pgfpathlineto{\pgfqpoint{2.704151in}{0.689575in}}%
\pgfpathlineto{\pgfqpoint{2.704449in}{0.689571in}}%
\pgfpathlineto{\pgfqpoint{2.704746in}{0.689568in}}%
\pgfpathlineto{\pgfqpoint{2.705044in}{0.689564in}}%
\pgfpathlineto{\pgfqpoint{2.705341in}{0.689560in}}%
\pgfpathlineto{\pgfqpoint{2.705638in}{0.689556in}}%
\pgfpathlineto{\pgfqpoint{2.705936in}{0.689553in}}%
\pgfpathlineto{\pgfqpoint{2.706233in}{0.689549in}}%
\pgfpathlineto{\pgfqpoint{2.706531in}{0.689545in}}%
\pgfpathlineto{\pgfqpoint{2.706828in}{0.689541in}}%
\pgfpathlineto{\pgfqpoint{2.707126in}{0.689537in}}%
\pgfpathlineto{\pgfqpoint{2.707423in}{0.689534in}}%
\pgfpathlineto{\pgfqpoint{2.707721in}{0.689530in}}%
\pgfpathlineto{\pgfqpoint{2.708018in}{0.689526in}}%
\pgfpathlineto{\pgfqpoint{2.708316in}{0.689522in}}%
\pgfpathlineto{\pgfqpoint{2.708613in}{0.689519in}}%
\pgfpathlineto{\pgfqpoint{2.708911in}{0.689515in}}%
\pgfpathlineto{\pgfqpoint{2.709208in}{0.689511in}}%
\pgfpathlineto{\pgfqpoint{2.709506in}{0.689507in}}%
\pgfpathlineto{\pgfqpoint{2.709803in}{0.689504in}}%
\pgfpathlineto{\pgfqpoint{2.710101in}{0.689500in}}%
\pgfpathlineto{\pgfqpoint{2.710398in}{0.689496in}}%
\pgfpathlineto{\pgfqpoint{2.710696in}{0.689492in}}%
\pgfpathlineto{\pgfqpoint{2.710993in}{0.689489in}}%
\pgfpathlineto{\pgfqpoint{2.711291in}{0.689485in}}%
\pgfpathlineto{\pgfqpoint{2.711588in}{0.689481in}}%
\pgfpathlineto{\pgfqpoint{2.711886in}{0.689477in}}%
\pgfpathlineto{\pgfqpoint{2.712183in}{0.689473in}}%
\pgfpathlineto{\pgfqpoint{2.712480in}{0.689470in}}%
\pgfpathlineto{\pgfqpoint{2.712778in}{0.689466in}}%
\pgfpathlineto{\pgfqpoint{2.713075in}{0.689462in}}%
\pgfpathlineto{\pgfqpoint{2.713373in}{0.689458in}}%
\pgfpathlineto{\pgfqpoint{2.713670in}{0.689455in}}%
\pgfpathlineto{\pgfqpoint{2.713968in}{0.689451in}}%
\pgfpathlineto{\pgfqpoint{2.714265in}{0.689447in}}%
\pgfpathlineto{\pgfqpoint{2.714563in}{0.689443in}}%
\pgfpathlineto{\pgfqpoint{2.714860in}{0.689440in}}%
\pgfpathlineto{\pgfqpoint{2.715158in}{0.689436in}}%
\pgfpathlineto{\pgfqpoint{2.715455in}{0.689432in}}%
\pgfpathlineto{\pgfqpoint{2.715753in}{0.689428in}}%
\pgfpathlineto{\pgfqpoint{2.716050in}{0.689425in}}%
\pgfpathlineto{\pgfqpoint{2.716348in}{0.689421in}}%
\pgfpathlineto{\pgfqpoint{2.716645in}{0.689417in}}%
\pgfpathlineto{\pgfqpoint{2.716943in}{0.689413in}}%
\pgfpathlineto{\pgfqpoint{2.717240in}{0.689410in}}%
\pgfpathlineto{\pgfqpoint{2.717538in}{0.689406in}}%
\pgfpathlineto{\pgfqpoint{2.717835in}{0.689402in}}%
\pgfpathlineto{\pgfqpoint{2.718133in}{0.689398in}}%
\pgfpathlineto{\pgfqpoint{2.718430in}{0.689394in}}%
\pgfpathlineto{\pgfqpoint{2.718728in}{0.689391in}}%
\pgfpathlineto{\pgfqpoint{2.719025in}{0.689387in}}%
\pgfpathlineto{\pgfqpoint{2.719322in}{0.689383in}}%
\pgfpathlineto{\pgfqpoint{2.719620in}{0.689379in}}%
\pgfpathlineto{\pgfqpoint{2.719917in}{0.689376in}}%
\pgfpathlineto{\pgfqpoint{2.720215in}{0.689372in}}%
\pgfpathlineto{\pgfqpoint{2.720512in}{0.689368in}}%
\pgfpathlineto{\pgfqpoint{2.720810in}{0.689364in}}%
\pgfpathlineto{\pgfqpoint{2.721107in}{0.689361in}}%
\pgfpathlineto{\pgfqpoint{2.721405in}{0.689357in}}%
\pgfpathlineto{\pgfqpoint{2.721702in}{0.689353in}}%
\pgfpathlineto{\pgfqpoint{2.722000in}{0.689349in}}%
\pgfpathlineto{\pgfqpoint{2.722297in}{0.689346in}}%
\pgfpathlineto{\pgfqpoint{2.722595in}{0.689342in}}%
\pgfpathlineto{\pgfqpoint{2.722892in}{0.689338in}}%
\pgfpathlineto{\pgfqpoint{2.723190in}{0.689334in}}%
\pgfpathlineto{\pgfqpoint{2.723487in}{0.689331in}}%
\pgfpathlineto{\pgfqpoint{2.723785in}{0.689327in}}%
\pgfpathlineto{\pgfqpoint{2.724082in}{0.689323in}}%
\pgfpathlineto{\pgfqpoint{2.724380in}{0.689319in}}%
\pgfpathlineto{\pgfqpoint{2.724677in}{0.689315in}}%
\pgfpathlineto{\pgfqpoint{2.724975in}{0.689312in}}%
\pgfpathlineto{\pgfqpoint{2.725272in}{0.689308in}}%
\pgfpathlineto{\pgfqpoint{2.725569in}{0.689304in}}%
\pgfpathlineto{\pgfqpoint{2.725867in}{0.689300in}}%
\pgfpathlineto{\pgfqpoint{2.726164in}{0.689297in}}%
\pgfpathlineto{\pgfqpoint{2.726462in}{0.689293in}}%
\pgfpathlineto{\pgfqpoint{2.726759in}{0.689289in}}%
\pgfpathlineto{\pgfqpoint{2.727057in}{0.689285in}}%
\pgfpathlineto{\pgfqpoint{2.727354in}{0.689282in}}%
\pgfpathlineto{\pgfqpoint{2.727652in}{0.689278in}}%
\pgfpathlineto{\pgfqpoint{2.727949in}{0.689274in}}%
\pgfpathlineto{\pgfqpoint{2.728247in}{0.689270in}}%
\pgfpathlineto{\pgfqpoint{2.728544in}{0.689267in}}%
\pgfpathlineto{\pgfqpoint{2.728842in}{0.689263in}}%
\pgfpathlineto{\pgfqpoint{2.729139in}{0.689259in}}%
\pgfpathlineto{\pgfqpoint{2.729437in}{0.689255in}}%
\pgfpathlineto{\pgfqpoint{2.729734in}{0.689252in}}%
\pgfpathlineto{\pgfqpoint{2.730032in}{0.689248in}}%
\pgfpathlineto{\pgfqpoint{2.730329in}{0.689244in}}%
\pgfpathlineto{\pgfqpoint{2.730627in}{0.689240in}}%
\pgfpathlineto{\pgfqpoint{2.730924in}{0.689236in}}%
\pgfpathlineto{\pgfqpoint{2.731222in}{0.689233in}}%
\pgfpathlineto{\pgfqpoint{2.731519in}{0.689229in}}%
\pgfpathlineto{\pgfqpoint{2.731817in}{0.689225in}}%
\pgfpathlineto{\pgfqpoint{2.732114in}{0.689221in}}%
\pgfpathlineto{\pgfqpoint{2.732411in}{0.689218in}}%
\pgfpathlineto{\pgfqpoint{2.732709in}{0.689214in}}%
\pgfpathlineto{\pgfqpoint{2.733006in}{0.689210in}}%
\pgfpathlineto{\pgfqpoint{2.733304in}{0.689206in}}%
\pgfpathlineto{\pgfqpoint{2.733601in}{0.689203in}}%
\pgfpathlineto{\pgfqpoint{2.733899in}{0.689199in}}%
\pgfpathlineto{\pgfqpoint{2.734196in}{0.689195in}}%
\pgfpathlineto{\pgfqpoint{2.734494in}{0.689191in}}%
\pgfpathlineto{\pgfqpoint{2.734791in}{0.689188in}}%
\pgfpathlineto{\pgfqpoint{2.735089in}{0.689184in}}%
\pgfpathlineto{\pgfqpoint{2.735386in}{0.689180in}}%
\pgfpathlineto{\pgfqpoint{2.735684in}{0.689176in}}%
\pgfpathlineto{\pgfqpoint{2.735981in}{0.689172in}}%
\pgfpathlineto{\pgfqpoint{2.736279in}{0.689169in}}%
\pgfpathlineto{\pgfqpoint{2.736576in}{0.689165in}}%
\pgfpathlineto{\pgfqpoint{2.736874in}{0.689161in}}%
\pgfpathlineto{\pgfqpoint{2.737171in}{0.689157in}}%
\pgfpathlineto{\pgfqpoint{2.737469in}{0.689154in}}%
\pgfpathlineto{\pgfqpoint{2.737766in}{0.689150in}}%
\pgfpathlineto{\pgfqpoint{2.738064in}{0.689146in}}%
\pgfpathlineto{\pgfqpoint{2.738361in}{0.689142in}}%
\pgfpathlineto{\pgfqpoint{2.738659in}{0.689139in}}%
\pgfpathlineto{\pgfqpoint{2.738956in}{0.689135in}}%
\pgfpathlineto{\pgfqpoint{2.739253in}{0.689131in}}%
\pgfpathlineto{\pgfqpoint{2.739551in}{0.689127in}}%
\pgfpathlineto{\pgfqpoint{2.739848in}{0.689124in}}%
\pgfpathlineto{\pgfqpoint{2.740146in}{0.689120in}}%
\pgfpathlineto{\pgfqpoint{2.740443in}{0.689116in}}%
\pgfpathlineto{\pgfqpoint{2.740741in}{0.689112in}}%
\pgfpathlineto{\pgfqpoint{2.741038in}{0.689109in}}%
\pgfpathlineto{\pgfqpoint{2.741336in}{0.689105in}}%
\pgfpathlineto{\pgfqpoint{2.741633in}{0.689101in}}%
\pgfpathlineto{\pgfqpoint{2.741931in}{0.689097in}}%
\pgfpathlineto{\pgfqpoint{2.742228in}{0.689093in}}%
\pgfpathlineto{\pgfqpoint{2.742526in}{0.689090in}}%
\pgfpathlineto{\pgfqpoint{2.742823in}{0.689086in}}%
\pgfpathlineto{\pgfqpoint{2.743121in}{0.689082in}}%
\pgfpathlineto{\pgfqpoint{2.743418in}{0.689078in}}%
\pgfpathlineto{\pgfqpoint{2.743716in}{0.689075in}}%
\pgfpathlineto{\pgfqpoint{2.744013in}{0.689071in}}%
\pgfpathlineto{\pgfqpoint{2.744311in}{0.689067in}}%
\pgfpathlineto{\pgfqpoint{2.744608in}{0.689063in}}%
\pgfpathlineto{\pgfqpoint{2.744906in}{0.689060in}}%
\pgfpathlineto{\pgfqpoint{2.745203in}{0.689056in}}%
\pgfpathlineto{\pgfqpoint{2.745500in}{0.689052in}}%
\pgfpathlineto{\pgfqpoint{2.745798in}{0.689048in}}%
\pgfpathlineto{\pgfqpoint{2.746095in}{0.689045in}}%
\pgfpathlineto{\pgfqpoint{2.746393in}{0.689041in}}%
\pgfpathlineto{\pgfqpoint{2.746690in}{0.689037in}}%
\pgfpathlineto{\pgfqpoint{2.746988in}{0.689033in}}%
\pgfpathlineto{\pgfqpoint{2.747285in}{0.689030in}}%
\pgfpathlineto{\pgfqpoint{2.747583in}{0.689026in}}%
\pgfpathlineto{\pgfqpoint{2.747880in}{0.689022in}}%
\pgfpathlineto{\pgfqpoint{2.748178in}{0.689018in}}%
\pgfpathlineto{\pgfqpoint{2.748475in}{0.689014in}}%
\pgfpathlineto{\pgfqpoint{2.748773in}{0.689011in}}%
\pgfpathlineto{\pgfqpoint{2.749070in}{0.689007in}}%
\pgfpathlineto{\pgfqpoint{2.749368in}{0.689003in}}%
\pgfpathlineto{\pgfqpoint{2.749665in}{0.688999in}}%
\pgfpathlineto{\pgfqpoint{2.749963in}{0.688996in}}%
\pgfpathlineto{\pgfqpoint{2.750260in}{0.688992in}}%
\pgfpathlineto{\pgfqpoint{2.750558in}{0.688988in}}%
\pgfpathlineto{\pgfqpoint{2.750855in}{0.688984in}}%
\pgfpathlineto{\pgfqpoint{2.751153in}{0.688981in}}%
\pgfpathlineto{\pgfqpoint{2.751450in}{0.688977in}}%
\pgfpathlineto{\pgfqpoint{2.751748in}{0.688973in}}%
\pgfpathlineto{\pgfqpoint{2.752045in}{0.688969in}}%
\pgfpathlineto{\pgfqpoint{2.752342in}{0.688966in}}%
\pgfpathlineto{\pgfqpoint{2.752640in}{0.688962in}}%
\pgfpathlineto{\pgfqpoint{2.752937in}{0.688958in}}%
\pgfpathlineto{\pgfqpoint{2.753235in}{0.688954in}}%
\pgfpathlineto{\pgfqpoint{2.753532in}{0.688950in}}%
\pgfpathlineto{\pgfqpoint{2.753830in}{0.688947in}}%
\pgfpathlineto{\pgfqpoint{2.754127in}{0.688943in}}%
\pgfpathlineto{\pgfqpoint{2.754425in}{0.688939in}}%
\pgfpathlineto{\pgfqpoint{2.754722in}{0.688935in}}%
\pgfpathlineto{\pgfqpoint{2.755020in}{0.688932in}}%
\pgfpathlineto{\pgfqpoint{2.755317in}{0.688928in}}%
\pgfpathlineto{\pgfqpoint{2.755615in}{0.688924in}}%
\pgfpathlineto{\pgfqpoint{2.755912in}{0.688920in}}%
\pgfpathlineto{\pgfqpoint{2.756210in}{0.688917in}}%
\pgfpathlineto{\pgfqpoint{2.756507in}{0.688913in}}%
\pgfpathlineto{\pgfqpoint{2.756805in}{0.688909in}}%
\pgfpathlineto{\pgfqpoint{2.757102in}{0.688905in}}%
\pgfpathlineto{\pgfqpoint{2.757400in}{0.688902in}}%
\pgfpathlineto{\pgfqpoint{2.757697in}{0.688898in}}%
\pgfpathlineto{\pgfqpoint{2.757995in}{0.688894in}}%
\pgfpathlineto{\pgfqpoint{2.758292in}{0.688890in}}%
\pgfpathlineto{\pgfqpoint{2.758590in}{0.688887in}}%
\pgfpathlineto{\pgfqpoint{2.758887in}{0.688883in}}%
\pgfpathlineto{\pgfqpoint{2.759184in}{0.688879in}}%
\pgfpathlineto{\pgfqpoint{2.759482in}{0.688875in}}%
\pgfpathlineto{\pgfqpoint{2.759779in}{0.688871in}}%
\pgfpathlineto{\pgfqpoint{2.760077in}{0.688868in}}%
\pgfpathlineto{\pgfqpoint{2.760374in}{0.688864in}}%
\pgfpathlineto{\pgfqpoint{2.760672in}{0.688860in}}%
\pgfpathlineto{\pgfqpoint{2.760969in}{0.688856in}}%
\pgfpathlineto{\pgfqpoint{2.761267in}{0.688853in}}%
\pgfpathlineto{\pgfqpoint{2.761564in}{0.688849in}}%
\pgfpathlineto{\pgfqpoint{2.761862in}{0.688845in}}%
\pgfpathlineto{\pgfqpoint{2.762159in}{0.688841in}}%
\pgfpathlineto{\pgfqpoint{2.762457in}{0.688838in}}%
\pgfpathlineto{\pgfqpoint{2.762754in}{0.688834in}}%
\pgfpathlineto{\pgfqpoint{2.763052in}{0.688830in}}%
\pgfpathlineto{\pgfqpoint{2.763349in}{0.688826in}}%
\pgfpathlineto{\pgfqpoint{2.763647in}{0.688823in}}%
\pgfpathlineto{\pgfqpoint{2.763944in}{0.688819in}}%
\pgfpathlineto{\pgfqpoint{2.764242in}{0.688815in}}%
\pgfpathlineto{\pgfqpoint{2.764539in}{0.688811in}}%
\pgfpathlineto{\pgfqpoint{2.764837in}{0.688808in}}%
\pgfpathlineto{\pgfqpoint{2.765134in}{0.688804in}}%
\pgfpathlineto{\pgfqpoint{2.765431in}{0.688800in}}%
\pgfpathlineto{\pgfqpoint{2.765729in}{0.688796in}}%
\pgfpathlineto{\pgfqpoint{2.766026in}{0.688792in}}%
\pgfpathlineto{\pgfqpoint{2.766324in}{0.688789in}}%
\pgfpathlineto{\pgfqpoint{2.766621in}{0.688785in}}%
\pgfpathlineto{\pgfqpoint{2.766919in}{0.688781in}}%
\pgfpathlineto{\pgfqpoint{2.767216in}{0.688777in}}%
\pgfpathlineto{\pgfqpoint{2.767514in}{0.688774in}}%
\pgfpathlineto{\pgfqpoint{2.767811in}{0.688770in}}%
\pgfpathlineto{\pgfqpoint{2.768109in}{0.688766in}}%
\pgfpathlineto{\pgfqpoint{2.768406in}{0.688762in}}%
\pgfpathlineto{\pgfqpoint{2.768704in}{0.688759in}}%
\pgfpathlineto{\pgfqpoint{2.769001in}{0.688755in}}%
\pgfpathlineto{\pgfqpoint{2.769299in}{0.688751in}}%
\pgfpathlineto{\pgfqpoint{2.769596in}{0.688747in}}%
\pgfpathlineto{\pgfqpoint{2.769894in}{0.688744in}}%
\pgfpathlineto{\pgfqpoint{2.770191in}{0.688740in}}%
\pgfpathlineto{\pgfqpoint{2.770489in}{0.688736in}}%
\pgfpathlineto{\pgfqpoint{2.770786in}{0.688732in}}%
\pgfpathlineto{\pgfqpoint{2.771084in}{0.688729in}}%
\pgfpathlineto{\pgfqpoint{2.771381in}{0.688725in}}%
\pgfpathlineto{\pgfqpoint{2.771679in}{0.688721in}}%
\pgfpathlineto{\pgfqpoint{2.771976in}{0.688717in}}%
\pgfpathlineto{\pgfqpoint{2.772273in}{0.688713in}}%
\pgfpathlineto{\pgfqpoint{2.772571in}{0.688710in}}%
\pgfpathlineto{\pgfqpoint{2.772868in}{0.688706in}}%
\pgfpathlineto{\pgfqpoint{2.773166in}{0.688702in}}%
\pgfpathlineto{\pgfqpoint{2.773463in}{0.688698in}}%
\pgfpathlineto{\pgfqpoint{2.773761in}{0.688695in}}%
\pgfpathlineto{\pgfqpoint{2.774058in}{0.688691in}}%
\pgfpathlineto{\pgfqpoint{2.774356in}{0.688687in}}%
\pgfpathlineto{\pgfqpoint{2.774653in}{0.688683in}}%
\pgfpathlineto{\pgfqpoint{2.774951in}{0.688680in}}%
\pgfpathlineto{\pgfqpoint{2.775248in}{0.688676in}}%
\pgfpathlineto{\pgfqpoint{2.775546in}{0.688672in}}%
\pgfpathlineto{\pgfqpoint{2.775843in}{0.688668in}}%
\pgfpathlineto{\pgfqpoint{2.776141in}{0.688665in}}%
\pgfpathlineto{\pgfqpoint{2.776438in}{0.688661in}}%
\pgfpathlineto{\pgfqpoint{2.776736in}{0.688657in}}%
\pgfpathlineto{\pgfqpoint{2.777033in}{0.688653in}}%
\pgfpathlineto{\pgfqpoint{2.777331in}{0.688649in}}%
\pgfpathlineto{\pgfqpoint{2.777628in}{0.688646in}}%
\pgfpathlineto{\pgfqpoint{2.777926in}{0.688642in}}%
\pgfpathlineto{\pgfqpoint{2.778223in}{0.688638in}}%
\pgfpathlineto{\pgfqpoint{2.778521in}{0.688634in}}%
\pgfpathlineto{\pgfqpoint{2.778818in}{0.688631in}}%
\pgfpathlineto{\pgfqpoint{2.779115in}{0.688627in}}%
\pgfpathlineto{\pgfqpoint{2.779413in}{0.688623in}}%
\pgfpathlineto{\pgfqpoint{2.779710in}{0.688619in}}%
\pgfpathlineto{\pgfqpoint{2.780008in}{0.688616in}}%
\pgfpathlineto{\pgfqpoint{2.780305in}{0.688612in}}%
\pgfpathlineto{\pgfqpoint{2.780603in}{0.688608in}}%
\pgfpathlineto{\pgfqpoint{2.780900in}{0.688604in}}%
\pgfpathlineto{\pgfqpoint{2.781198in}{0.688601in}}%
\pgfpathlineto{\pgfqpoint{2.781495in}{0.688597in}}%
\pgfpathlineto{\pgfqpoint{2.781793in}{0.688593in}}%
\pgfpathlineto{\pgfqpoint{2.782090in}{0.688589in}}%
\pgfpathlineto{\pgfqpoint{2.782388in}{0.688586in}}%
\pgfpathlineto{\pgfqpoint{2.782685in}{0.688582in}}%
\pgfpathlineto{\pgfqpoint{2.782983in}{0.688578in}}%
\pgfpathlineto{\pgfqpoint{2.783280in}{0.688574in}}%
\pgfpathlineto{\pgfqpoint{2.783578in}{0.688570in}}%
\pgfpathlineto{\pgfqpoint{2.783875in}{0.688567in}}%
\pgfpathlineto{\pgfqpoint{2.784173in}{0.688563in}}%
\pgfpathlineto{\pgfqpoint{2.784470in}{0.688559in}}%
\pgfpathlineto{\pgfqpoint{2.784768in}{0.688555in}}%
\pgfpathlineto{\pgfqpoint{2.785065in}{0.688552in}}%
\pgfpathlineto{\pgfqpoint{2.785362in}{0.688548in}}%
\pgfpathlineto{\pgfqpoint{2.785660in}{0.688544in}}%
\pgfpathlineto{\pgfqpoint{2.785957in}{0.688540in}}%
\pgfpathlineto{\pgfqpoint{2.786255in}{0.688537in}}%
\pgfpathlineto{\pgfqpoint{2.786552in}{0.688533in}}%
\pgfpathlineto{\pgfqpoint{2.786850in}{0.688529in}}%
\pgfpathlineto{\pgfqpoint{2.787147in}{0.688525in}}%
\pgfpathlineto{\pgfqpoint{2.787445in}{0.688522in}}%
\pgfpathlineto{\pgfqpoint{2.787742in}{0.688518in}}%
\pgfpathlineto{\pgfqpoint{2.788040in}{0.688514in}}%
\pgfpathlineto{\pgfqpoint{2.788337in}{0.688510in}}%
\pgfpathlineto{\pgfqpoint{2.788635in}{0.688507in}}%
\pgfpathlineto{\pgfqpoint{2.788932in}{0.688503in}}%
\pgfpathlineto{\pgfqpoint{2.789230in}{0.688499in}}%
\pgfpathlineto{\pgfqpoint{2.789527in}{0.688495in}}%
\pgfpathlineto{\pgfqpoint{2.789825in}{0.688491in}}%
\pgfpathlineto{\pgfqpoint{2.790122in}{0.688488in}}%
\pgfpathlineto{\pgfqpoint{2.790420in}{0.688484in}}%
\pgfpathlineto{\pgfqpoint{2.790717in}{0.688480in}}%
\pgfpathlineto{\pgfqpoint{2.791015in}{0.688476in}}%
\pgfpathlineto{\pgfqpoint{2.791312in}{0.688473in}}%
\pgfpathlineto{\pgfqpoint{2.791610in}{0.688469in}}%
\pgfpathlineto{\pgfqpoint{2.791907in}{0.688465in}}%
\pgfpathlineto{\pgfqpoint{2.792204in}{0.688461in}}%
\pgfpathlineto{\pgfqpoint{2.792502in}{0.688458in}}%
\pgfpathlineto{\pgfqpoint{2.792799in}{0.688454in}}%
\pgfpathlineto{\pgfqpoint{2.793097in}{0.688450in}}%
\pgfpathlineto{\pgfqpoint{2.793394in}{0.688446in}}%
\pgfpathlineto{\pgfqpoint{2.793692in}{0.688443in}}%
\pgfpathlineto{\pgfqpoint{2.793989in}{0.688439in}}%
\pgfpathlineto{\pgfqpoint{2.794287in}{0.688435in}}%
\pgfpathlineto{\pgfqpoint{2.794584in}{0.688431in}}%
\pgfpathlineto{\pgfqpoint{2.794882in}{0.688427in}}%
\pgfpathlineto{\pgfqpoint{2.795179in}{0.688424in}}%
\pgfpathlineto{\pgfqpoint{2.795477in}{0.688420in}}%
\pgfpathlineto{\pgfqpoint{2.795774in}{0.688416in}}%
\pgfpathlineto{\pgfqpoint{2.796072in}{0.688412in}}%
\pgfpathlineto{\pgfqpoint{2.796369in}{0.688409in}}%
\pgfpathlineto{\pgfqpoint{2.796667in}{0.688405in}}%
\pgfpathlineto{\pgfqpoint{2.796964in}{0.688401in}}%
\pgfpathlineto{\pgfqpoint{2.797262in}{0.688397in}}%
\pgfpathlineto{\pgfqpoint{2.797559in}{0.688394in}}%
\pgfpathlineto{\pgfqpoint{2.797857in}{0.688390in}}%
\pgfpathlineto{\pgfqpoint{2.798154in}{0.688386in}}%
\pgfpathlineto{\pgfqpoint{2.798452in}{0.688382in}}%
\pgfpathlineto{\pgfqpoint{2.798749in}{0.688379in}}%
\pgfpathlineto{\pgfqpoint{2.799046in}{0.688375in}}%
\pgfpathlineto{\pgfqpoint{2.799344in}{0.688371in}}%
\pgfpathlineto{\pgfqpoint{2.799641in}{0.688367in}}%
\pgfpathlineto{\pgfqpoint{2.799939in}{0.688364in}}%
\pgfpathlineto{\pgfqpoint{2.800236in}{0.688360in}}%
\pgfpathlineto{\pgfqpoint{2.800534in}{0.688356in}}%
\pgfpathlineto{\pgfqpoint{2.800831in}{0.688352in}}%
\pgfpathlineto{\pgfqpoint{2.801129in}{0.688348in}}%
\pgfpathlineto{\pgfqpoint{2.801426in}{0.688345in}}%
\pgfpathlineto{\pgfqpoint{2.801724in}{0.688341in}}%
\pgfpathlineto{\pgfqpoint{2.802021in}{0.688337in}}%
\pgfpathlineto{\pgfqpoint{2.802319in}{0.688333in}}%
\pgfpathlineto{\pgfqpoint{2.802616in}{0.688330in}}%
\pgfpathlineto{\pgfqpoint{2.802914in}{0.688326in}}%
\pgfpathlineto{\pgfqpoint{2.803211in}{0.688322in}}%
\pgfpathlineto{\pgfqpoint{2.803509in}{0.688318in}}%
\pgfpathlineto{\pgfqpoint{2.803806in}{0.688315in}}%
\pgfpathlineto{\pgfqpoint{2.804104in}{0.688311in}}%
\pgfpathlineto{\pgfqpoint{2.804401in}{0.688307in}}%
\pgfpathlineto{\pgfqpoint{2.804699in}{0.688303in}}%
\pgfpathlineto{\pgfqpoint{2.804996in}{0.688300in}}%
\pgfpathlineto{\pgfqpoint{2.805293in}{0.688296in}}%
\pgfpathlineto{\pgfqpoint{2.805591in}{0.688292in}}%
\pgfpathlineto{\pgfqpoint{2.805888in}{0.688288in}}%
\pgfpathlineto{\pgfqpoint{2.806186in}{0.688285in}}%
\pgfpathlineto{\pgfqpoint{2.806483in}{0.688281in}}%
\pgfpathlineto{\pgfqpoint{2.806781in}{0.688277in}}%
\pgfpathlineto{\pgfqpoint{2.807078in}{0.688273in}}%
\pgfpathlineto{\pgfqpoint{2.807376in}{0.688269in}}%
\pgfpathlineto{\pgfqpoint{2.807673in}{0.688266in}}%
\pgfpathlineto{\pgfqpoint{2.807971in}{0.688262in}}%
\pgfpathlineto{\pgfqpoint{2.808268in}{0.688258in}}%
\pgfpathlineto{\pgfqpoint{2.808566in}{0.688254in}}%
\pgfpathlineto{\pgfqpoint{2.808863in}{0.688251in}}%
\pgfpathlineto{\pgfqpoint{2.809161in}{0.688247in}}%
\pgfpathlineto{\pgfqpoint{2.809458in}{0.688243in}}%
\pgfpathlineto{\pgfqpoint{2.809756in}{0.688239in}}%
\pgfpathlineto{\pgfqpoint{2.810053in}{0.688236in}}%
\pgfpathlineto{\pgfqpoint{2.810351in}{0.688232in}}%
\pgfpathlineto{\pgfqpoint{2.810648in}{0.688228in}}%
\pgfpathlineto{\pgfqpoint{2.810946in}{0.688224in}}%
\pgfpathlineto{\pgfqpoint{2.811243in}{0.688221in}}%
\pgfpathlineto{\pgfqpoint{2.811541in}{0.688217in}}%
\pgfpathlineto{\pgfqpoint{2.811838in}{0.688213in}}%
\pgfpathlineto{\pgfqpoint{2.812135in}{0.688209in}}%
\pgfpathlineto{\pgfqpoint{2.812433in}{0.688206in}}%
\pgfpathlineto{\pgfqpoint{2.812730in}{0.688202in}}%
\pgfpathlineto{\pgfqpoint{2.813028in}{0.688198in}}%
\pgfpathlineto{\pgfqpoint{2.813325in}{0.688194in}}%
\pgfpathlineto{\pgfqpoint{2.813623in}{0.688190in}}%
\pgfpathlineto{\pgfqpoint{2.813920in}{0.688187in}}%
\pgfpathlineto{\pgfqpoint{2.814218in}{0.688183in}}%
\pgfpathlineto{\pgfqpoint{2.814515in}{0.688179in}}%
\pgfpathlineto{\pgfqpoint{2.814813in}{0.688175in}}%
\pgfpathlineto{\pgfqpoint{2.815110in}{0.688172in}}%
\pgfpathlineto{\pgfqpoint{2.815408in}{0.688168in}}%
\pgfpathlineto{\pgfqpoint{2.815705in}{0.688164in}}%
\pgfpathlineto{\pgfqpoint{2.816003in}{0.688160in}}%
\pgfpathlineto{\pgfqpoint{2.816300in}{0.688157in}}%
\pgfpathlineto{\pgfqpoint{2.816598in}{0.688153in}}%
\pgfpathlineto{\pgfqpoint{2.816895in}{0.688149in}}%
\pgfpathlineto{\pgfqpoint{2.817193in}{0.688145in}}%
\pgfpathlineto{\pgfqpoint{2.817490in}{0.688142in}}%
\pgfpathlineto{\pgfqpoint{2.817788in}{0.688138in}}%
\pgfpathlineto{\pgfqpoint{2.818085in}{0.688134in}}%
\pgfpathlineto{\pgfqpoint{2.818383in}{0.688130in}}%
\pgfpathlineto{\pgfqpoint{2.818680in}{0.688126in}}%
\pgfpathlineto{\pgfqpoint{2.818977in}{0.688123in}}%
\pgfpathlineto{\pgfqpoint{2.819275in}{0.688119in}}%
\pgfpathlineto{\pgfqpoint{2.819572in}{0.688115in}}%
\pgfpathlineto{\pgfqpoint{2.819870in}{0.688111in}}%
\pgfpathlineto{\pgfqpoint{2.820167in}{0.688108in}}%
\pgfpathlineto{\pgfqpoint{2.820465in}{0.688104in}}%
\pgfpathlineto{\pgfqpoint{2.820762in}{0.688100in}}%
\pgfpathlineto{\pgfqpoint{2.821060in}{0.688096in}}%
\pgfpathlineto{\pgfqpoint{2.821357in}{0.688093in}}%
\pgfpathlineto{\pgfqpoint{2.821655in}{0.688089in}}%
\pgfpathlineto{\pgfqpoint{2.821952in}{0.688085in}}%
\pgfpathlineto{\pgfqpoint{2.822250in}{0.688081in}}%
\pgfpathlineto{\pgfqpoint{2.822547in}{0.688078in}}%
\pgfpathlineto{\pgfqpoint{2.822845in}{0.688074in}}%
\pgfpathlineto{\pgfqpoint{2.823142in}{0.688070in}}%
\pgfpathlineto{\pgfqpoint{2.823440in}{0.688066in}}%
\pgfpathlineto{\pgfqpoint{2.823737in}{0.688063in}}%
\pgfpathlineto{\pgfqpoint{2.824035in}{0.688059in}}%
\pgfpathlineto{\pgfqpoint{2.824332in}{0.688055in}}%
\pgfpathlineto{\pgfqpoint{2.824630in}{0.688051in}}%
\pgfpathlineto{\pgfqpoint{2.824927in}{0.688047in}}%
\pgfpathlineto{\pgfqpoint{2.825224in}{0.688044in}}%
\pgfpathlineto{\pgfqpoint{2.825522in}{0.688040in}}%
\pgfpathlineto{\pgfqpoint{2.825819in}{0.688036in}}%
\pgfpathlineto{\pgfqpoint{2.826117in}{0.688032in}}%
\pgfpathlineto{\pgfqpoint{2.826414in}{0.688029in}}%
\pgfpathlineto{\pgfqpoint{2.826712in}{0.688025in}}%
\pgfpathlineto{\pgfqpoint{2.827009in}{0.688021in}}%
\pgfpathlineto{\pgfqpoint{2.827307in}{0.688017in}}%
\pgfpathlineto{\pgfqpoint{2.827604in}{0.688014in}}%
\pgfpathlineto{\pgfqpoint{2.827902in}{0.688010in}}%
\pgfpathlineto{\pgfqpoint{2.828199in}{0.688006in}}%
\pgfpathlineto{\pgfqpoint{2.828497in}{0.688002in}}%
\pgfpathlineto{\pgfqpoint{2.828794in}{0.687999in}}%
\pgfpathlineto{\pgfqpoint{2.829092in}{0.687995in}}%
\pgfpathlineto{\pgfqpoint{2.829389in}{0.687991in}}%
\pgfpathlineto{\pgfqpoint{2.829687in}{0.687987in}}%
\pgfpathlineto{\pgfqpoint{2.829984in}{0.687984in}}%
\pgfpathlineto{\pgfqpoint{2.830282in}{0.687980in}}%
\pgfpathlineto{\pgfqpoint{2.830579in}{0.687976in}}%
\pgfpathlineto{\pgfqpoint{2.830877in}{0.687972in}}%
\pgfpathlineto{\pgfqpoint{2.831174in}{0.687968in}}%
\pgfpathlineto{\pgfqpoint{2.831472in}{0.687965in}}%
\pgfpathlineto{\pgfqpoint{2.831769in}{0.687961in}}%
\pgfpathlineto{\pgfqpoint{2.832066in}{0.687957in}}%
\pgfpathlineto{\pgfqpoint{2.832364in}{0.687953in}}%
\pgfpathlineto{\pgfqpoint{2.832661in}{0.687950in}}%
\pgfpathlineto{\pgfqpoint{2.832959in}{0.687946in}}%
\pgfpathlineto{\pgfqpoint{2.833256in}{0.687942in}}%
\pgfpathlineto{\pgfqpoint{2.833554in}{0.687938in}}%
\pgfpathlineto{\pgfqpoint{2.833851in}{0.687931in}}%
\pgfpathlineto{\pgfqpoint{2.834149in}{0.687900in}}%
\pgfpathlineto{\pgfqpoint{2.834446in}{0.687863in}}%
\pgfpathlineto{\pgfqpoint{2.834744in}{0.687826in}}%
\pgfpathlineto{\pgfqpoint{2.835041in}{0.687789in}}%
\pgfpathlineto{\pgfqpoint{2.835339in}{0.687752in}}%
\pgfpathlineto{\pgfqpoint{2.835636in}{0.687715in}}%
\pgfpathlineto{\pgfqpoint{2.835934in}{0.687678in}}%
\pgfpathlineto{\pgfqpoint{2.836231in}{0.687642in}}%
\pgfpathlineto{\pgfqpoint{2.836529in}{0.687605in}}%
\pgfpathlineto{\pgfqpoint{2.836826in}{0.687568in}}%
\pgfpathlineto{\pgfqpoint{2.837124in}{0.687531in}}%
\pgfpathlineto{\pgfqpoint{2.837421in}{0.687494in}}%
\pgfpathlineto{\pgfqpoint{2.837719in}{0.687457in}}%
\pgfpathlineto{\pgfqpoint{2.838016in}{0.687420in}}%
\pgfpathlineto{\pgfqpoint{2.838314in}{0.687383in}}%
\pgfpathlineto{\pgfqpoint{2.838611in}{0.687346in}}%
\pgfpathlineto{\pgfqpoint{2.838908in}{0.687310in}}%
\pgfpathlineto{\pgfqpoint{2.839206in}{0.687273in}}%
\pgfpathlineto{\pgfqpoint{2.839503in}{0.687236in}}%
\pgfpathlineto{\pgfqpoint{2.839801in}{0.687199in}}%
\pgfpathlineto{\pgfqpoint{2.840098in}{0.687162in}}%
\pgfpathlineto{\pgfqpoint{2.840396in}{0.687125in}}%
\pgfpathlineto{\pgfqpoint{2.840693in}{0.687088in}}%
\pgfpathlineto{\pgfqpoint{2.840991in}{0.687051in}}%
\pgfpathlineto{\pgfqpoint{2.841288in}{0.687015in}}%
\pgfpathlineto{\pgfqpoint{2.841586in}{0.686978in}}%
\pgfpathlineto{\pgfqpoint{2.841883in}{0.686941in}}%
\pgfpathlineto{\pgfqpoint{2.842181in}{0.686904in}}%
\pgfpathlineto{\pgfqpoint{2.842478in}{0.686867in}}%
\pgfpathlineto{\pgfqpoint{2.842776in}{0.686830in}}%
\pgfpathlineto{\pgfqpoint{2.843073in}{0.686793in}}%
\pgfpathlineto{\pgfqpoint{2.843371in}{0.686756in}}%
\pgfpathlineto{\pgfqpoint{2.843668in}{0.686720in}}%
\pgfpathlineto{\pgfqpoint{2.843966in}{0.686683in}}%
\pgfpathlineto{\pgfqpoint{2.844263in}{0.686646in}}%
\pgfpathlineto{\pgfqpoint{2.844561in}{0.686609in}}%
\pgfpathlineto{\pgfqpoint{2.844858in}{0.686572in}}%
\pgfpathlineto{\pgfqpoint{2.845155in}{0.686535in}}%
\pgfpathlineto{\pgfqpoint{2.845453in}{0.686498in}}%
\pgfpathlineto{\pgfqpoint{2.845750in}{0.686461in}}%
\pgfpathlineto{\pgfqpoint{2.846048in}{0.686424in}}%
\pgfpathlineto{\pgfqpoint{2.846345in}{0.686388in}}%
\pgfpathlineto{\pgfqpoint{2.846643in}{0.686351in}}%
\pgfpathlineto{\pgfqpoint{2.846940in}{0.686314in}}%
\pgfpathlineto{\pgfqpoint{2.847238in}{0.686277in}}%
\pgfpathlineto{\pgfqpoint{2.847535in}{0.686240in}}%
\pgfpathlineto{\pgfqpoint{2.847833in}{0.686203in}}%
\pgfpathlineto{\pgfqpoint{2.848130in}{0.686166in}}%
\pgfpathlineto{\pgfqpoint{2.848428in}{0.686137in}}%
\pgfpathlineto{\pgfqpoint{2.848725in}{0.686126in}}%
\pgfpathlineto{\pgfqpoint{2.849023in}{0.686115in}}%
\pgfpathlineto{\pgfqpoint{2.849320in}{0.686105in}}%
\pgfpathlineto{\pgfqpoint{2.849618in}{0.686095in}}%
\pgfpathlineto{\pgfqpoint{2.849915in}{0.686084in}}%
\pgfpathlineto{\pgfqpoint{2.850213in}{0.686074in}}%
\pgfpathlineto{\pgfqpoint{2.850510in}{0.686064in}}%
\pgfpathlineto{\pgfqpoint{2.850808in}{0.686054in}}%
\pgfpathlineto{\pgfqpoint{2.851105in}{0.686043in}}%
\pgfpathlineto{\pgfqpoint{2.851403in}{0.686033in}}%
\pgfpathlineto{\pgfqpoint{2.851700in}{0.686023in}}%
\pgfpathlineto{\pgfqpoint{2.851997in}{0.686012in}}%
\pgfpathlineto{\pgfqpoint{2.852295in}{0.686002in}}%
\pgfpathlineto{\pgfqpoint{2.852592in}{0.685992in}}%
\pgfpathlineto{\pgfqpoint{2.852890in}{0.685982in}}%
\pgfpathlineto{\pgfqpoint{2.853187in}{0.685971in}}%
\pgfpathlineto{\pgfqpoint{2.853485in}{0.685961in}}%
\pgfpathlineto{\pgfqpoint{2.853782in}{0.685951in}}%
\pgfpathlineto{\pgfqpoint{2.854080in}{0.685941in}}%
\pgfpathlineto{\pgfqpoint{2.854377in}{0.685930in}}%
\pgfpathlineto{\pgfqpoint{2.854675in}{0.685920in}}%
\pgfpathlineto{\pgfqpoint{2.854972in}{0.685910in}}%
\pgfpathlineto{\pgfqpoint{2.855270in}{0.685899in}}%
\pgfpathlineto{\pgfqpoint{2.855567in}{0.685889in}}%
\pgfpathlineto{\pgfqpoint{2.855865in}{0.685879in}}%
\pgfpathlineto{\pgfqpoint{2.856162in}{0.685869in}}%
\pgfpathlineto{\pgfqpoint{2.856460in}{0.685858in}}%
\pgfpathlineto{\pgfqpoint{2.856757in}{0.685848in}}%
\pgfpathlineto{\pgfqpoint{2.857055in}{0.685838in}}%
\pgfpathlineto{\pgfqpoint{2.857352in}{0.685827in}}%
\pgfpathlineto{\pgfqpoint{2.857650in}{0.685817in}}%
\pgfpathlineto{\pgfqpoint{2.857947in}{0.685807in}}%
\pgfpathlineto{\pgfqpoint{2.858245in}{0.685797in}}%
\pgfpathlineto{\pgfqpoint{2.858542in}{0.685786in}}%
\pgfpathlineto{\pgfqpoint{2.858839in}{0.685776in}}%
\pgfpathlineto{\pgfqpoint{2.859137in}{0.685766in}}%
\pgfpathlineto{\pgfqpoint{2.859434in}{0.685755in}}%
\pgfpathlineto{\pgfqpoint{2.859732in}{0.685745in}}%
\pgfpathlineto{\pgfqpoint{2.860029in}{0.685735in}}%
\pgfpathlineto{\pgfqpoint{2.860327in}{0.685725in}}%
\pgfpathlineto{\pgfqpoint{2.860624in}{0.685714in}}%
\pgfpathlineto{\pgfqpoint{2.860922in}{0.685704in}}%
\pgfpathlineto{\pgfqpoint{2.861219in}{0.685694in}}%
\pgfpathlineto{\pgfqpoint{2.861517in}{0.685683in}}%
\pgfpathlineto{\pgfqpoint{2.861814in}{0.685673in}}%
\pgfpathlineto{\pgfqpoint{2.862112in}{0.685663in}}%
\pgfpathlineto{\pgfqpoint{2.862409in}{0.685653in}}%
\pgfpathlineto{\pgfqpoint{2.862707in}{0.685642in}}%
\pgfpathlineto{\pgfqpoint{2.863004in}{0.685632in}}%
\pgfpathlineto{\pgfqpoint{2.863302in}{0.685622in}}%
\pgfpathlineto{\pgfqpoint{2.863599in}{0.685611in}}%
\pgfpathlineto{\pgfqpoint{2.863897in}{0.685601in}}%
\pgfpathlineto{\pgfqpoint{2.864194in}{0.685591in}}%
\pgfpathlineto{\pgfqpoint{2.864492in}{0.685581in}}%
\pgfpathlineto{\pgfqpoint{2.864789in}{0.685570in}}%
\pgfpathlineto{\pgfqpoint{2.865086in}{0.685560in}}%
\pgfpathlineto{\pgfqpoint{2.865384in}{0.685550in}}%
\pgfpathlineto{\pgfqpoint{2.865681in}{0.685539in}}%
\pgfpathlineto{\pgfqpoint{2.865979in}{0.685529in}}%
\pgfpathlineto{\pgfqpoint{2.866276in}{0.685519in}}%
\pgfpathlineto{\pgfqpoint{2.866574in}{0.685509in}}%
\pgfpathlineto{\pgfqpoint{2.866871in}{0.685498in}}%
\pgfpathlineto{\pgfqpoint{2.867169in}{0.685488in}}%
\pgfpathlineto{\pgfqpoint{2.867466in}{0.685478in}}%
\pgfpathlineto{\pgfqpoint{2.867764in}{0.685467in}}%
\pgfpathlineto{\pgfqpoint{2.868061in}{0.685457in}}%
\pgfpathlineto{\pgfqpoint{2.868359in}{0.685447in}}%
\pgfpathlineto{\pgfqpoint{2.868656in}{0.685437in}}%
\pgfpathlineto{\pgfqpoint{2.868954in}{0.685426in}}%
\pgfpathlineto{\pgfqpoint{2.869251in}{0.685416in}}%
\pgfpathlineto{\pgfqpoint{2.869549in}{0.685406in}}%
\pgfpathlineto{\pgfqpoint{2.869846in}{0.685395in}}%
\pgfpathlineto{\pgfqpoint{2.870144in}{0.685385in}}%
\pgfpathlineto{\pgfqpoint{2.870441in}{0.685375in}}%
\pgfpathlineto{\pgfqpoint{2.870739in}{0.685365in}}%
\pgfpathlineto{\pgfqpoint{2.871036in}{0.685354in}}%
\pgfpathlineto{\pgfqpoint{2.871334in}{0.685344in}}%
\pgfpathlineto{\pgfqpoint{2.871631in}{0.685334in}}%
\pgfpathlineto{\pgfqpoint{2.871928in}{0.685324in}}%
\pgfpathlineto{\pgfqpoint{2.872226in}{0.685313in}}%
\pgfpathlineto{\pgfqpoint{2.872523in}{0.685303in}}%
\pgfpathlineto{\pgfqpoint{2.872821in}{0.685293in}}%
\pgfpathlineto{\pgfqpoint{2.873118in}{0.685282in}}%
\pgfpathlineto{\pgfqpoint{2.873416in}{0.685272in}}%
\pgfpathlineto{\pgfqpoint{2.873713in}{0.685262in}}%
\pgfpathlineto{\pgfqpoint{2.874011in}{0.685252in}}%
\pgfpathlineto{\pgfqpoint{2.874308in}{0.685241in}}%
\pgfpathlineto{\pgfqpoint{2.874606in}{0.685231in}}%
\pgfpathlineto{\pgfqpoint{2.874903in}{0.685221in}}%
\pgfpathlineto{\pgfqpoint{2.875201in}{0.685210in}}%
\pgfpathlineto{\pgfqpoint{2.875498in}{0.685200in}}%
\pgfpathlineto{\pgfqpoint{2.875796in}{0.685190in}}%
\pgfpathlineto{\pgfqpoint{2.876093in}{0.685180in}}%
\pgfpathlineto{\pgfqpoint{2.876391in}{0.685169in}}%
\pgfpathlineto{\pgfqpoint{2.876688in}{0.685159in}}%
\pgfpathlineto{\pgfqpoint{2.876986in}{0.685149in}}%
\pgfpathlineto{\pgfqpoint{2.877283in}{0.685138in}}%
\pgfpathlineto{\pgfqpoint{2.877581in}{0.685128in}}%
\pgfpathlineto{\pgfqpoint{2.877878in}{0.685118in}}%
\pgfpathlineto{\pgfqpoint{2.878176in}{0.685108in}}%
\pgfpathlineto{\pgfqpoint{2.878473in}{0.685097in}}%
\pgfpathlineto{\pgfqpoint{2.878770in}{0.685087in}}%
\pgfpathlineto{\pgfqpoint{2.879068in}{0.685077in}}%
\pgfpathlineto{\pgfqpoint{2.879365in}{0.685066in}}%
\pgfpathlineto{\pgfqpoint{2.879663in}{0.685056in}}%
\pgfpathlineto{\pgfqpoint{2.879960in}{0.685046in}}%
\pgfpathlineto{\pgfqpoint{2.880258in}{0.685036in}}%
\pgfpathlineto{\pgfqpoint{2.880555in}{0.685025in}}%
\pgfpathlineto{\pgfqpoint{2.880853in}{0.685015in}}%
\pgfpathlineto{\pgfqpoint{2.881150in}{0.685005in}}%
\pgfpathlineto{\pgfqpoint{2.881448in}{0.684994in}}%
\pgfpathlineto{\pgfqpoint{2.881745in}{0.684984in}}%
\pgfpathlineto{\pgfqpoint{2.882043in}{0.684974in}}%
\pgfpathlineto{\pgfqpoint{2.882340in}{0.684964in}}%
\pgfpathlineto{\pgfqpoint{2.882638in}{0.684953in}}%
\pgfpathlineto{\pgfqpoint{2.882935in}{0.684943in}}%
\pgfpathlineto{\pgfqpoint{2.883233in}{0.684933in}}%
\pgfpathlineto{\pgfqpoint{2.883530in}{0.684922in}}%
\pgfpathlineto{\pgfqpoint{2.883828in}{0.684912in}}%
\pgfpathlineto{\pgfqpoint{2.884125in}{0.684902in}}%
\pgfpathlineto{\pgfqpoint{2.884423in}{0.684892in}}%
\pgfpathlineto{\pgfqpoint{2.884720in}{0.684881in}}%
\pgfpathlineto{\pgfqpoint{2.885018in}{0.684871in}}%
\pgfpathlineto{\pgfqpoint{2.885315in}{0.684861in}}%
\pgfpathlineto{\pgfqpoint{2.885612in}{0.684850in}}%
\pgfpathlineto{\pgfqpoint{2.885910in}{0.684840in}}%
\pgfpathlineto{\pgfqpoint{2.886207in}{0.684830in}}%
\pgfpathlineto{\pgfqpoint{2.886505in}{0.684820in}}%
\pgfpathlineto{\pgfqpoint{2.886802in}{0.684809in}}%
\pgfpathlineto{\pgfqpoint{2.887100in}{0.684799in}}%
\pgfpathlineto{\pgfqpoint{2.887397in}{0.684789in}}%
\pgfpathlineto{\pgfqpoint{2.887695in}{0.684778in}}%
\pgfpathlineto{\pgfqpoint{2.887992in}{0.684768in}}%
\pgfpathlineto{\pgfqpoint{2.888290in}{0.684758in}}%
\pgfpathlineto{\pgfqpoint{2.888587in}{0.684748in}}%
\pgfpathlineto{\pgfqpoint{2.888885in}{0.684737in}}%
\pgfpathlineto{\pgfqpoint{2.889182in}{0.684727in}}%
\pgfpathlineto{\pgfqpoint{2.889480in}{0.684717in}}%
\pgfpathlineto{\pgfqpoint{2.889777in}{0.684706in}}%
\pgfpathlineto{\pgfqpoint{2.890075in}{0.684696in}}%
\pgfpathlineto{\pgfqpoint{2.890372in}{0.684686in}}%
\pgfpathlineto{\pgfqpoint{2.890670in}{0.684676in}}%
\pgfpathlineto{\pgfqpoint{2.890967in}{0.684665in}}%
\pgfpathlineto{\pgfqpoint{2.891265in}{0.684655in}}%
\pgfpathlineto{\pgfqpoint{2.891562in}{0.684645in}}%
\pgfpathlineto{\pgfqpoint{2.891859in}{0.684635in}}%
\pgfpathlineto{\pgfqpoint{2.892157in}{0.684624in}}%
\pgfpathlineto{\pgfqpoint{2.892454in}{0.684614in}}%
\pgfpathlineto{\pgfqpoint{2.892752in}{0.684604in}}%
\pgfpathlineto{\pgfqpoint{2.893049in}{0.684593in}}%
\pgfpathlineto{\pgfqpoint{2.893347in}{0.684583in}}%
\pgfpathlineto{\pgfqpoint{2.893644in}{0.684573in}}%
\pgfpathlineto{\pgfqpoint{2.893942in}{0.684563in}}%
\pgfpathlineto{\pgfqpoint{2.894239in}{0.684552in}}%
\pgfpathlineto{\pgfqpoint{2.894537in}{0.684542in}}%
\pgfpathlineto{\pgfqpoint{2.894834in}{0.684532in}}%
\pgfpathlineto{\pgfqpoint{2.895132in}{0.684521in}}%
\pgfpathlineto{\pgfqpoint{2.895429in}{0.684511in}}%
\pgfpathlineto{\pgfqpoint{2.895727in}{0.684501in}}%
\pgfpathlineto{\pgfqpoint{2.896024in}{0.684491in}}%
\pgfpathlineto{\pgfqpoint{2.896322in}{0.684480in}}%
\pgfpathlineto{\pgfqpoint{2.896619in}{0.684470in}}%
\pgfpathlineto{\pgfqpoint{2.896917in}{0.684460in}}%
\pgfpathlineto{\pgfqpoint{2.897214in}{0.684449in}}%
\pgfpathlineto{\pgfqpoint{2.897512in}{0.684439in}}%
\pgfpathlineto{\pgfqpoint{2.897809in}{0.684429in}}%
\pgfpathlineto{\pgfqpoint{2.898107in}{0.684419in}}%
\pgfpathlineto{\pgfqpoint{2.898404in}{0.684408in}}%
\pgfpathlineto{\pgfqpoint{2.898701in}{0.684398in}}%
\pgfpathlineto{\pgfqpoint{2.898999in}{0.684388in}}%
\pgfpathlineto{\pgfqpoint{2.899296in}{0.684377in}}%
\pgfpathlineto{\pgfqpoint{2.899594in}{0.684367in}}%
\pgfpathlineto{\pgfqpoint{2.899891in}{0.684357in}}%
\pgfpathlineto{\pgfqpoint{2.900189in}{0.684347in}}%
\pgfpathlineto{\pgfqpoint{2.900486in}{0.684339in}}%
\pgfpathlineto{\pgfqpoint{2.900784in}{0.684333in}}%
\pgfpathlineto{\pgfqpoint{2.901081in}{0.684327in}}%
\pgfpathlineto{\pgfqpoint{2.901379in}{0.684321in}}%
\pgfpathlineto{\pgfqpoint{2.901676in}{0.684315in}}%
\pgfpathlineto{\pgfqpoint{2.901974in}{0.684310in}}%
\pgfpathlineto{\pgfqpoint{2.902271in}{0.684304in}}%
\pgfpathlineto{\pgfqpoint{2.902569in}{0.684298in}}%
\pgfpathlineto{\pgfqpoint{2.902866in}{0.684292in}}%
\pgfpathlineto{\pgfqpoint{2.903164in}{0.684286in}}%
\pgfpathlineto{\pgfqpoint{2.903461in}{0.684280in}}%
\pgfpathlineto{\pgfqpoint{2.903759in}{0.684274in}}%
\pgfpathlineto{\pgfqpoint{2.904056in}{0.684268in}}%
\pgfpathlineto{\pgfqpoint{2.904354in}{0.684263in}}%
\pgfpathlineto{\pgfqpoint{2.904651in}{0.684257in}}%
\pgfpathlineto{\pgfqpoint{2.904949in}{0.684251in}}%
\pgfpathlineto{\pgfqpoint{2.905246in}{0.684245in}}%
\pgfpathlineto{\pgfqpoint{2.905543in}{0.684239in}}%
\pgfpathlineto{\pgfqpoint{2.905841in}{0.684233in}}%
\pgfpathlineto{\pgfqpoint{2.906138in}{0.684227in}}%
\pgfpathlineto{\pgfqpoint{2.906436in}{0.684222in}}%
\pgfpathlineto{\pgfqpoint{2.906733in}{0.684216in}}%
\pgfpathlineto{\pgfqpoint{2.907031in}{0.684210in}}%
\pgfpathlineto{\pgfqpoint{2.907328in}{0.684204in}}%
\pgfpathlineto{\pgfqpoint{2.907626in}{0.684198in}}%
\pgfpathlineto{\pgfqpoint{2.907923in}{0.684192in}}%
\pgfpathlineto{\pgfqpoint{2.908221in}{0.684186in}}%
\pgfpathlineto{\pgfqpoint{2.908518in}{0.684180in}}%
\pgfpathlineto{\pgfqpoint{2.908816in}{0.684175in}}%
\pgfpathlineto{\pgfqpoint{2.909113in}{0.684169in}}%
\pgfpathlineto{\pgfqpoint{2.909411in}{0.684163in}}%
\pgfpathlineto{\pgfqpoint{2.909708in}{0.684157in}}%
\pgfpathlineto{\pgfqpoint{2.910006in}{0.684151in}}%
\pgfpathlineto{\pgfqpoint{2.910303in}{0.684145in}}%
\pgfpathlineto{\pgfqpoint{2.910601in}{0.684139in}}%
\pgfpathlineto{\pgfqpoint{2.910898in}{0.684134in}}%
\pgfpathlineto{\pgfqpoint{2.911196in}{0.684128in}}%
\pgfpathlineto{\pgfqpoint{2.911493in}{0.684122in}}%
\pgfpathlineto{\pgfqpoint{2.911790in}{0.684116in}}%
\pgfpathlineto{\pgfqpoint{2.912088in}{0.684110in}}%
\pgfpathlineto{\pgfqpoint{2.912385in}{0.684104in}}%
\pgfpathlineto{\pgfqpoint{2.912683in}{0.684098in}}%
\pgfpathlineto{\pgfqpoint{2.912980in}{0.684092in}}%
\pgfpathlineto{\pgfqpoint{2.913278in}{0.684087in}}%
\pgfpathlineto{\pgfqpoint{2.913575in}{0.684081in}}%
\pgfpathlineto{\pgfqpoint{2.913873in}{0.684075in}}%
\pgfpathlineto{\pgfqpoint{2.914170in}{0.684069in}}%
\pgfpathlineto{\pgfqpoint{2.914468in}{0.684063in}}%
\pgfpathlineto{\pgfqpoint{2.914765in}{0.684057in}}%
\pgfpathlineto{\pgfqpoint{2.915063in}{0.684051in}}%
\pgfpathlineto{\pgfqpoint{2.915360in}{0.684045in}}%
\pgfpathlineto{\pgfqpoint{2.915658in}{0.684040in}}%
\pgfpathlineto{\pgfqpoint{2.915955in}{0.684034in}}%
\pgfpathlineto{\pgfqpoint{2.916253in}{0.684028in}}%
\pgfpathlineto{\pgfqpoint{2.916550in}{0.684022in}}%
\pgfpathlineto{\pgfqpoint{2.916848in}{0.684016in}}%
\pgfpathlineto{\pgfqpoint{2.917145in}{0.684010in}}%
\pgfpathlineto{\pgfqpoint{2.917443in}{0.684004in}}%
\pgfpathlineto{\pgfqpoint{2.917740in}{0.683999in}}%
\pgfpathlineto{\pgfqpoint{2.918038in}{0.683993in}}%
\pgfpathlineto{\pgfqpoint{2.918335in}{0.683987in}}%
\pgfpathlineto{\pgfqpoint{2.918632in}{0.683981in}}%
\pgfpathlineto{\pgfqpoint{2.918930in}{0.683975in}}%
\pgfpathlineto{\pgfqpoint{2.919227in}{0.683969in}}%
\pgfpathlineto{\pgfqpoint{2.919525in}{0.683963in}}%
\pgfpathlineto{\pgfqpoint{2.919822in}{0.683957in}}%
\pgfpathlineto{\pgfqpoint{2.920120in}{0.683952in}}%
\pgfpathlineto{\pgfqpoint{2.920417in}{0.683946in}}%
\pgfpathlineto{\pgfqpoint{2.920715in}{0.683940in}}%
\pgfpathlineto{\pgfqpoint{2.921012in}{0.683934in}}%
\pgfpathlineto{\pgfqpoint{2.921310in}{0.683928in}}%
\pgfpathlineto{\pgfqpoint{2.921607in}{0.683922in}}%
\pgfpathlineto{\pgfqpoint{2.921905in}{0.683916in}}%
\pgfpathlineto{\pgfqpoint{2.922202in}{0.683911in}}%
\pgfpathlineto{\pgfqpoint{2.922500in}{0.683905in}}%
\pgfpathlineto{\pgfqpoint{2.922797in}{0.683899in}}%
\pgfpathlineto{\pgfqpoint{2.923095in}{0.683893in}}%
\pgfpathlineto{\pgfqpoint{2.923392in}{0.683887in}}%
\pgfpathlineto{\pgfqpoint{2.923690in}{0.683881in}}%
\pgfpathlineto{\pgfqpoint{2.923987in}{0.683875in}}%
\pgfpathlineto{\pgfqpoint{2.924285in}{0.683869in}}%
\pgfpathlineto{\pgfqpoint{2.924582in}{0.683864in}}%
\pgfpathlineto{\pgfqpoint{2.924880in}{0.683858in}}%
\pgfpathlineto{\pgfqpoint{2.925177in}{0.683852in}}%
\pgfpathlineto{\pgfqpoint{2.925474in}{0.683846in}}%
\pgfpathlineto{\pgfqpoint{2.925772in}{0.683840in}}%
\pgfpathlineto{\pgfqpoint{2.926069in}{0.683834in}}%
\pgfpathlineto{\pgfqpoint{2.926367in}{0.683828in}}%
\pgfpathlineto{\pgfqpoint{2.926664in}{0.683822in}}%
\pgfpathlineto{\pgfqpoint{2.926962in}{0.683817in}}%
\pgfpathlineto{\pgfqpoint{2.927259in}{0.683811in}}%
\pgfpathlineto{\pgfqpoint{2.927557in}{0.683805in}}%
\pgfpathlineto{\pgfqpoint{2.927854in}{0.683799in}}%
\pgfpathlineto{\pgfqpoint{2.928152in}{0.683793in}}%
\pgfpathlineto{\pgfqpoint{2.928449in}{0.683787in}}%
\pgfpathlineto{\pgfqpoint{2.928747in}{0.683781in}}%
\pgfpathlineto{\pgfqpoint{2.929044in}{0.683776in}}%
\pgfpathlineto{\pgfqpoint{2.929342in}{0.683770in}}%
\pgfpathlineto{\pgfqpoint{2.929639in}{0.683764in}}%
\pgfpathlineto{\pgfqpoint{2.929937in}{0.683758in}}%
\pgfpathlineto{\pgfqpoint{2.930234in}{0.683752in}}%
\pgfpathlineto{\pgfqpoint{2.930532in}{0.683746in}}%
\pgfpathlineto{\pgfqpoint{2.930829in}{0.683740in}}%
\pgfpathlineto{\pgfqpoint{2.931127in}{0.683734in}}%
\pgfpathlineto{\pgfqpoint{2.931424in}{0.683729in}}%
\pgfpathlineto{\pgfqpoint{2.931721in}{0.683723in}}%
\pgfpathlineto{\pgfqpoint{2.932019in}{0.683717in}}%
\pgfpathlineto{\pgfqpoint{2.932316in}{0.683711in}}%
\pgfpathlineto{\pgfqpoint{2.932614in}{0.683705in}}%
\pgfpathlineto{\pgfqpoint{2.932911in}{0.683699in}}%
\pgfpathlineto{\pgfqpoint{2.933209in}{0.683693in}}%
\pgfpathlineto{\pgfqpoint{2.933506in}{0.683688in}}%
\pgfpathlineto{\pgfqpoint{2.933804in}{0.683682in}}%
\pgfpathlineto{\pgfqpoint{2.934101in}{0.683676in}}%
\pgfpathlineto{\pgfqpoint{2.934399in}{0.683670in}}%
\pgfpathlineto{\pgfqpoint{2.934696in}{0.683664in}}%
\pgfpathlineto{\pgfqpoint{2.934994in}{0.683658in}}%
\pgfpathlineto{\pgfqpoint{2.935291in}{0.683652in}}%
\pgfpathlineto{\pgfqpoint{2.935589in}{0.683646in}}%
\pgfpathlineto{\pgfqpoint{2.935886in}{0.683641in}}%
\pgfpathlineto{\pgfqpoint{2.936184in}{0.683635in}}%
\pgfpathlineto{\pgfqpoint{2.936481in}{0.683629in}}%
\pgfpathlineto{\pgfqpoint{2.936779in}{0.683623in}}%
\pgfpathlineto{\pgfqpoint{2.937076in}{0.683617in}}%
\pgfpathlineto{\pgfqpoint{2.937374in}{0.683611in}}%
\pgfpathlineto{\pgfqpoint{2.937671in}{0.683605in}}%
\pgfpathlineto{\pgfqpoint{2.937969in}{0.683599in}}%
\pgfpathlineto{\pgfqpoint{2.938266in}{0.683594in}}%
\pgfpathlineto{\pgfqpoint{2.938563in}{0.683588in}}%
\pgfpathlineto{\pgfqpoint{2.938861in}{0.683582in}}%
\pgfpathlineto{\pgfqpoint{2.939158in}{0.683576in}}%
\pgfpathlineto{\pgfqpoint{2.939456in}{0.683570in}}%
\pgfpathlineto{\pgfqpoint{2.939753in}{0.683564in}}%
\pgfpathlineto{\pgfqpoint{2.940051in}{0.683558in}}%
\pgfpathlineto{\pgfqpoint{2.940348in}{0.683553in}}%
\pgfpathlineto{\pgfqpoint{2.940646in}{0.683547in}}%
\pgfpathlineto{\pgfqpoint{2.940943in}{0.683541in}}%
\pgfpathlineto{\pgfqpoint{2.941241in}{0.683535in}}%
\pgfpathlineto{\pgfqpoint{2.941538in}{0.683529in}}%
\pgfpathlineto{\pgfqpoint{2.941836in}{0.683523in}}%
\pgfpathlineto{\pgfqpoint{2.942133in}{0.683517in}}%
\pgfpathlineto{\pgfqpoint{2.942431in}{0.683511in}}%
\pgfpathlineto{\pgfqpoint{2.942728in}{0.683506in}}%
\pgfpathlineto{\pgfqpoint{2.943026in}{0.683500in}}%
\pgfpathlineto{\pgfqpoint{2.943323in}{0.683494in}}%
\pgfpathlineto{\pgfqpoint{2.943621in}{0.683488in}}%
\pgfpathlineto{\pgfqpoint{2.943918in}{0.683482in}}%
\pgfpathlineto{\pgfqpoint{2.944216in}{0.683476in}}%
\pgfpathlineto{\pgfqpoint{2.944513in}{0.683470in}}%
\pgfpathlineto{\pgfqpoint{2.944811in}{0.683465in}}%
\pgfpathlineto{\pgfqpoint{2.945108in}{0.683459in}}%
\pgfpathlineto{\pgfqpoint{2.945405in}{0.683453in}}%
\pgfpathlineto{\pgfqpoint{2.945703in}{0.683447in}}%
\pgfpathlineto{\pgfqpoint{2.946000in}{0.683441in}}%
\pgfpathlineto{\pgfqpoint{2.946298in}{0.683435in}}%
\pgfpathlineto{\pgfqpoint{2.946595in}{0.683429in}}%
\pgfpathlineto{\pgfqpoint{2.946893in}{0.683423in}}%
\pgfpathlineto{\pgfqpoint{2.947190in}{0.683418in}}%
\pgfpathlineto{\pgfqpoint{2.947488in}{0.683412in}}%
\pgfpathlineto{\pgfqpoint{2.947785in}{0.683406in}}%
\pgfpathlineto{\pgfqpoint{2.948083in}{0.683400in}}%
\pgfpathlineto{\pgfqpoint{2.948380in}{0.683394in}}%
\pgfpathlineto{\pgfqpoint{2.948678in}{0.683388in}}%
\pgfpathlineto{\pgfqpoint{2.948975in}{0.683382in}}%
\pgfpathlineto{\pgfqpoint{2.949273in}{0.683376in}}%
\pgfpathlineto{\pgfqpoint{2.949570in}{0.683371in}}%
\pgfpathlineto{\pgfqpoint{2.949868in}{0.683365in}}%
\pgfpathlineto{\pgfqpoint{2.950165in}{0.683359in}}%
\pgfpathlineto{\pgfqpoint{2.950463in}{0.683353in}}%
\pgfpathlineto{\pgfqpoint{2.950760in}{0.683347in}}%
\pgfpathlineto{\pgfqpoint{2.951058in}{0.683341in}}%
\pgfpathlineto{\pgfqpoint{2.951355in}{0.683335in}}%
\pgfpathlineto{\pgfqpoint{2.951652in}{0.683330in}}%
\pgfpathlineto{\pgfqpoint{2.951950in}{0.683324in}}%
\pgfpathlineto{\pgfqpoint{2.952247in}{0.683318in}}%
\pgfpathlineto{\pgfqpoint{2.952545in}{0.683312in}}%
\pgfpathlineto{\pgfqpoint{2.952842in}{0.683306in}}%
\pgfpathlineto{\pgfqpoint{2.953140in}{0.683300in}}%
\pgfpathlineto{\pgfqpoint{2.953437in}{0.683294in}}%
\pgfpathlineto{\pgfqpoint{2.953735in}{0.683288in}}%
\pgfpathlineto{\pgfqpoint{2.954032in}{0.683283in}}%
\pgfpathlineto{\pgfqpoint{2.954330in}{0.683277in}}%
\pgfpathlineto{\pgfqpoint{2.954627in}{0.683271in}}%
\pgfpathlineto{\pgfqpoint{2.954925in}{0.683265in}}%
\pgfpathlineto{\pgfqpoint{2.955222in}{0.683259in}}%
\pgfpathlineto{\pgfqpoint{2.955520in}{0.683253in}}%
\pgfpathlineto{\pgfqpoint{2.955817in}{0.683247in}}%
\pgfpathlineto{\pgfqpoint{2.956115in}{0.683242in}}%
\pgfpathlineto{\pgfqpoint{2.956412in}{0.683236in}}%
\pgfpathlineto{\pgfqpoint{2.956710in}{0.683230in}}%
\pgfpathlineto{\pgfqpoint{2.957007in}{0.683224in}}%
\pgfpathlineto{\pgfqpoint{2.957305in}{0.683218in}}%
\pgfpathlineto{\pgfqpoint{2.957602in}{0.683212in}}%
\pgfpathlineto{\pgfqpoint{2.957900in}{0.683206in}}%
\pgfpathlineto{\pgfqpoint{2.958197in}{0.683200in}}%
\pgfpathlineto{\pgfqpoint{2.958494in}{0.683195in}}%
\pgfpathlineto{\pgfqpoint{2.958792in}{0.683189in}}%
\pgfpathlineto{\pgfqpoint{2.959089in}{0.683183in}}%
\pgfpathlineto{\pgfqpoint{2.959387in}{0.683177in}}%
\pgfpathlineto{\pgfqpoint{2.959684in}{0.683171in}}%
\pgfpathlineto{\pgfqpoint{2.959982in}{0.683165in}}%
\pgfpathlineto{\pgfqpoint{2.960279in}{0.683159in}}%
\pgfpathlineto{\pgfqpoint{2.960577in}{0.683154in}}%
\pgfpathlineto{\pgfqpoint{2.960874in}{0.683148in}}%
\pgfpathlineto{\pgfqpoint{2.961172in}{0.683142in}}%
\pgfpathlineto{\pgfqpoint{2.961469in}{0.683136in}}%
\pgfpathlineto{\pgfqpoint{2.961767in}{0.683130in}}%
\pgfpathlineto{\pgfqpoint{2.962064in}{0.683124in}}%
\pgfpathlineto{\pgfqpoint{2.962362in}{0.683118in}}%
\pgfpathlineto{\pgfqpoint{2.962659in}{0.683112in}}%
\pgfpathlineto{\pgfqpoint{2.962957in}{0.683107in}}%
\pgfpathlineto{\pgfqpoint{2.963254in}{0.683101in}}%
\pgfpathlineto{\pgfqpoint{2.963552in}{0.683095in}}%
\pgfpathlineto{\pgfqpoint{2.963849in}{0.683089in}}%
\pgfpathlineto{\pgfqpoint{2.964147in}{0.683083in}}%
\pgfpathlineto{\pgfqpoint{2.964444in}{0.683077in}}%
\pgfpathlineto{\pgfqpoint{2.964742in}{0.683071in}}%
\pgfpathlineto{\pgfqpoint{2.965039in}{0.683065in}}%
\pgfpathlineto{\pgfqpoint{2.965336in}{0.683060in}}%
\pgfpathlineto{\pgfqpoint{2.965634in}{0.683054in}}%
\pgfpathlineto{\pgfqpoint{2.965931in}{0.683048in}}%
\pgfpathlineto{\pgfqpoint{2.966229in}{0.683042in}}%
\pgfpathlineto{\pgfqpoint{2.966526in}{0.683036in}}%
\pgfpathlineto{\pgfqpoint{2.966824in}{0.683030in}}%
\pgfpathlineto{\pgfqpoint{2.967121in}{0.683024in}}%
\pgfpathlineto{\pgfqpoint{2.967419in}{0.683019in}}%
\pgfpathlineto{\pgfqpoint{2.967716in}{0.683013in}}%
\pgfpathlineto{\pgfqpoint{2.968014in}{0.683007in}}%
\pgfpathlineto{\pgfqpoint{2.968311in}{0.683001in}}%
\pgfpathlineto{\pgfqpoint{2.968609in}{0.682995in}}%
\pgfpathlineto{\pgfqpoint{2.968906in}{0.682989in}}%
\pgfpathlineto{\pgfqpoint{2.969204in}{0.682983in}}%
\pgfpathlineto{\pgfqpoint{2.969501in}{0.682977in}}%
\pgfpathlineto{\pgfqpoint{2.969799in}{0.682972in}}%
\pgfpathlineto{\pgfqpoint{2.970096in}{0.682966in}}%
\pgfpathlineto{\pgfqpoint{2.970394in}{0.682960in}}%
\pgfpathlineto{\pgfqpoint{2.970691in}{0.682954in}}%
\pgfpathlineto{\pgfqpoint{2.970989in}{0.682948in}}%
\pgfpathlineto{\pgfqpoint{2.971286in}{0.682942in}}%
\pgfpathlineto{\pgfqpoint{2.971583in}{0.682936in}}%
\pgfpathlineto{\pgfqpoint{2.971881in}{0.682931in}}%
\pgfpathlineto{\pgfqpoint{2.972178in}{0.682925in}}%
\pgfpathlineto{\pgfqpoint{2.972476in}{0.682919in}}%
\pgfpathlineto{\pgfqpoint{2.972773in}{0.682913in}}%
\pgfpathlineto{\pgfqpoint{2.973071in}{0.682907in}}%
\pgfpathlineto{\pgfqpoint{2.973368in}{0.682901in}}%
\pgfpathlineto{\pgfqpoint{2.973666in}{0.682895in}}%
\pgfpathlineto{\pgfqpoint{2.973963in}{0.682889in}}%
\pgfpathlineto{\pgfqpoint{2.974261in}{0.682884in}}%
\pgfpathlineto{\pgfqpoint{2.974558in}{0.682878in}}%
\pgfpathlineto{\pgfqpoint{2.974856in}{0.682872in}}%
\pgfpathlineto{\pgfqpoint{2.975153in}{0.682866in}}%
\pgfpathlineto{\pgfqpoint{2.975451in}{0.682860in}}%
\pgfpathlineto{\pgfqpoint{2.975748in}{0.682854in}}%
\pgfpathlineto{\pgfqpoint{2.976046in}{0.682848in}}%
\pgfpathlineto{\pgfqpoint{2.976343in}{0.682842in}}%
\pgfpathlineto{\pgfqpoint{2.976641in}{0.682837in}}%
\pgfpathlineto{\pgfqpoint{2.976938in}{0.682831in}}%
\pgfpathlineto{\pgfqpoint{2.977236in}{0.682825in}}%
\pgfpathlineto{\pgfqpoint{2.977533in}{0.682819in}}%
\pgfpathlineto{\pgfqpoint{2.977831in}{0.682813in}}%
\pgfpathlineto{\pgfqpoint{2.978128in}{0.682807in}}%
\pgfpathlineto{\pgfqpoint{2.978425in}{0.682801in}}%
\pgfpathlineto{\pgfqpoint{2.978723in}{0.682796in}}%
\pgfpathlineto{\pgfqpoint{2.979020in}{0.682790in}}%
\pgfpathlineto{\pgfqpoint{2.979318in}{0.682784in}}%
\pgfpathlineto{\pgfqpoint{2.979615in}{0.682778in}}%
\pgfpathlineto{\pgfqpoint{2.979913in}{0.682772in}}%
\pgfpathlineto{\pgfqpoint{2.980210in}{0.682766in}}%
\pgfpathlineto{\pgfqpoint{2.980508in}{0.682760in}}%
\pgfpathlineto{\pgfqpoint{2.980805in}{0.682754in}}%
\pgfpathlineto{\pgfqpoint{2.981103in}{0.682749in}}%
\pgfpathlineto{\pgfqpoint{2.981400in}{0.682743in}}%
\pgfpathlineto{\pgfqpoint{2.981698in}{0.682737in}}%
\pgfpathlineto{\pgfqpoint{2.981995in}{0.682731in}}%
\pgfpathlineto{\pgfqpoint{2.982293in}{0.682725in}}%
\pgfpathlineto{\pgfqpoint{2.982590in}{0.682719in}}%
\pgfpathlineto{\pgfqpoint{2.982888in}{0.682713in}}%
\pgfpathlineto{\pgfqpoint{2.983185in}{0.682708in}}%
\pgfpathlineto{\pgfqpoint{2.983483in}{0.682702in}}%
\pgfpathlineto{\pgfqpoint{2.983780in}{0.682696in}}%
\pgfpathlineto{\pgfqpoint{2.984078in}{0.682690in}}%
\pgfpathlineto{\pgfqpoint{2.984375in}{0.682684in}}%
\pgfpathlineto{\pgfqpoint{2.984673in}{0.682678in}}%
\pgfpathlineto{\pgfqpoint{2.984970in}{0.682672in}}%
\pgfpathlineto{\pgfqpoint{2.985267in}{0.682666in}}%
\pgfpathlineto{\pgfqpoint{2.985565in}{0.682661in}}%
\pgfpathlineto{\pgfqpoint{2.985862in}{0.682655in}}%
\pgfpathlineto{\pgfqpoint{2.986160in}{0.682649in}}%
\pgfpathlineto{\pgfqpoint{2.986457in}{0.682643in}}%
\pgfpathlineto{\pgfqpoint{2.986755in}{0.682637in}}%
\pgfpathlineto{\pgfqpoint{2.987052in}{0.682631in}}%
\pgfpathlineto{\pgfqpoint{2.987350in}{0.682625in}}%
\pgfpathlineto{\pgfqpoint{2.987647in}{0.682619in}}%
\pgfpathlineto{\pgfqpoint{2.987945in}{0.682614in}}%
\pgfpathlineto{\pgfqpoint{2.988242in}{0.682608in}}%
\pgfpathlineto{\pgfqpoint{2.988540in}{0.682602in}}%
\pgfpathlineto{\pgfqpoint{2.988837in}{0.682596in}}%
\pgfpathlineto{\pgfqpoint{2.989135in}{0.682590in}}%
\pgfpathlineto{\pgfqpoint{2.989432in}{0.682584in}}%
\pgfpathlineto{\pgfqpoint{2.989730in}{0.682578in}}%
\pgfpathlineto{\pgfqpoint{2.990027in}{0.682573in}}%
\pgfpathlineto{\pgfqpoint{2.990325in}{0.682567in}}%
\pgfpathlineto{\pgfqpoint{2.990622in}{0.682561in}}%
\pgfpathlineto{\pgfqpoint{2.990920in}{0.682555in}}%
\pgfpathlineto{\pgfqpoint{2.991217in}{0.682549in}}%
\pgfpathlineto{\pgfqpoint{2.991514in}{0.682485in}}%
\pgfpathlineto{\pgfqpoint{2.991812in}{0.682371in}}%
\pgfpathlineto{\pgfqpoint{2.992109in}{0.682257in}}%
\pgfpathlineto{\pgfqpoint{2.992407in}{0.682144in}}%
\pgfpathlineto{\pgfqpoint{2.992704in}{0.682030in}}%
\pgfpathlineto{\pgfqpoint{2.993002in}{0.681916in}}%
\pgfpathlineto{\pgfqpoint{2.993299in}{0.681802in}}%
\pgfpathlineto{\pgfqpoint{2.993597in}{0.681689in}}%
\pgfpathlineto{\pgfqpoint{2.993894in}{0.681575in}}%
\pgfpathlineto{\pgfqpoint{2.994192in}{0.681461in}}%
\pgfpathlineto{\pgfqpoint{2.994489in}{0.681347in}}%
\pgfpathlineto{\pgfqpoint{2.994787in}{0.681233in}}%
\pgfpathlineto{\pgfqpoint{2.995084in}{0.681120in}}%
\pgfpathlineto{\pgfqpoint{2.995382in}{0.681006in}}%
\pgfpathlineto{\pgfqpoint{2.995679in}{0.680892in}}%
\pgfpathlineto{\pgfqpoint{2.995977in}{0.680778in}}%
\pgfpathlineto{\pgfqpoint{2.996274in}{0.680665in}}%
\pgfpathlineto{\pgfqpoint{2.996572in}{0.680551in}}%
\pgfpathlineto{\pgfqpoint{2.996869in}{0.680437in}}%
\pgfpathlineto{\pgfqpoint{2.997167in}{0.680323in}}%
\pgfpathlineto{\pgfqpoint{2.997464in}{0.680210in}}%
\pgfpathlineto{\pgfqpoint{2.997762in}{0.680096in}}%
\pgfpathlineto{\pgfqpoint{2.998059in}{0.679982in}}%
\pgfpathlineto{\pgfqpoint{2.998356in}{0.679875in}}%
\pgfpathlineto{\pgfqpoint{2.998654in}{0.679843in}}%
\pgfpathlineto{\pgfqpoint{2.998951in}{0.679832in}}%
\pgfpathlineto{\pgfqpoint{2.999249in}{0.679820in}}%
\pgfpathlineto{\pgfqpoint{2.999546in}{0.679809in}}%
\pgfpathlineto{\pgfqpoint{2.999844in}{0.679798in}}%
\pgfpathlineto{\pgfqpoint{3.000141in}{0.679787in}}%
\pgfpathlineto{\pgfqpoint{3.000439in}{0.679776in}}%
\pgfpathlineto{\pgfqpoint{3.000736in}{0.679764in}}%
\pgfpathlineto{\pgfqpoint{3.001034in}{0.679753in}}%
\pgfpathlineto{\pgfqpoint{3.001331in}{0.679742in}}%
\pgfpathlineto{\pgfqpoint{3.001629in}{0.679731in}}%
\pgfpathlineto{\pgfqpoint{3.001926in}{0.679720in}}%
\pgfpathlineto{\pgfqpoint{3.002224in}{0.679708in}}%
\pgfpathlineto{\pgfqpoint{3.002521in}{0.679697in}}%
\pgfpathlineto{\pgfqpoint{3.002819in}{0.679686in}}%
\pgfpathlineto{\pgfqpoint{3.003116in}{0.679675in}}%
\pgfpathlineto{\pgfqpoint{3.003414in}{0.679664in}}%
\pgfpathlineto{\pgfqpoint{3.003711in}{0.679652in}}%
\pgfpathlineto{\pgfqpoint{3.004009in}{0.679641in}}%
\pgfpathlineto{\pgfqpoint{3.004306in}{0.679630in}}%
\pgfpathlineto{\pgfqpoint{3.004604in}{0.679619in}}%
\pgfpathlineto{\pgfqpoint{3.004901in}{0.679608in}}%
\pgfpathlineto{\pgfqpoint{3.005198in}{0.679596in}}%
\pgfpathlineto{\pgfqpoint{3.005496in}{0.679585in}}%
\pgfpathlineto{\pgfqpoint{3.005793in}{0.679574in}}%
\pgfpathlineto{\pgfqpoint{3.006091in}{0.679563in}}%
\pgfpathlineto{\pgfqpoint{3.006388in}{0.679552in}}%
\pgfpathlineto{\pgfqpoint{3.006686in}{0.679540in}}%
\pgfpathlineto{\pgfqpoint{3.006983in}{0.679529in}}%
\pgfpathlineto{\pgfqpoint{3.007281in}{0.679518in}}%
\pgfpathlineto{\pgfqpoint{3.007578in}{0.679507in}}%
\pgfpathlineto{\pgfqpoint{3.007876in}{0.679496in}}%
\pgfpathlineto{\pgfqpoint{3.008173in}{0.679484in}}%
\pgfpathlineto{\pgfqpoint{3.008471in}{0.679473in}}%
\pgfpathlineto{\pgfqpoint{3.008768in}{0.679462in}}%
\pgfpathlineto{\pgfqpoint{3.009066in}{0.679451in}}%
\pgfpathlineto{\pgfqpoint{3.009363in}{0.679440in}}%
\pgfpathlineto{\pgfqpoint{3.009661in}{0.679428in}}%
\pgfpathlineto{\pgfqpoint{3.009958in}{0.679417in}}%
\pgfpathlineto{\pgfqpoint{3.010256in}{0.679406in}}%
\pgfpathlineto{\pgfqpoint{3.010553in}{0.679395in}}%
\pgfpathlineto{\pgfqpoint{3.010851in}{0.679384in}}%
\pgfpathlineto{\pgfqpoint{3.011148in}{0.679372in}}%
\pgfpathlineto{\pgfqpoint{3.011445in}{0.679361in}}%
\pgfpathlineto{\pgfqpoint{3.011743in}{0.679350in}}%
\pgfpathlineto{\pgfqpoint{3.012040in}{0.679339in}}%
\pgfpathlineto{\pgfqpoint{3.012338in}{0.679328in}}%
\pgfpathlineto{\pgfqpoint{3.012635in}{0.679316in}}%
\pgfpathlineto{\pgfqpoint{3.012933in}{0.679305in}}%
\pgfpathlineto{\pgfqpoint{3.013230in}{0.679294in}}%
\pgfpathlineto{\pgfqpoint{3.013528in}{0.679283in}}%
\pgfpathlineto{\pgfqpoint{3.013825in}{0.679272in}}%
\pgfpathlineto{\pgfqpoint{3.014123in}{0.679260in}}%
\pgfpathlineto{\pgfqpoint{3.014420in}{0.679249in}}%
\pgfpathlineto{\pgfqpoint{3.014718in}{0.679238in}}%
\pgfpathlineto{\pgfqpoint{3.015015in}{0.679227in}}%
\pgfpathlineto{\pgfqpoint{3.015313in}{0.679216in}}%
\pgfpathlineto{\pgfqpoint{3.015610in}{0.679204in}}%
\pgfpathlineto{\pgfqpoint{3.015908in}{0.679193in}}%
\pgfpathlineto{\pgfqpoint{3.016205in}{0.679182in}}%
\pgfpathlineto{\pgfqpoint{3.016503in}{0.679171in}}%
\pgfpathlineto{\pgfqpoint{3.016800in}{0.679160in}}%
\pgfpathlineto{\pgfqpoint{3.017098in}{0.679148in}}%
\pgfpathlineto{\pgfqpoint{3.017395in}{0.679137in}}%
\pgfpathlineto{\pgfqpoint{3.017693in}{0.679126in}}%
\pgfpathlineto{\pgfqpoint{3.017990in}{0.679115in}}%
\pgfpathlineto{\pgfqpoint{3.018287in}{0.679104in}}%
\pgfpathlineto{\pgfqpoint{3.018585in}{0.679092in}}%
\pgfpathlineto{\pgfqpoint{3.018882in}{0.679081in}}%
\pgfpathlineto{\pgfqpoint{3.019180in}{0.679070in}}%
\pgfpathlineto{\pgfqpoint{3.019477in}{0.679059in}}%
\pgfpathlineto{\pgfqpoint{3.019775in}{0.679048in}}%
\pgfpathlineto{\pgfqpoint{3.020072in}{0.679036in}}%
\pgfpathlineto{\pgfqpoint{3.020370in}{0.679025in}}%
\pgfpathlineto{\pgfqpoint{3.020667in}{0.679014in}}%
\pgfpathlineto{\pgfqpoint{3.020965in}{0.679003in}}%
\pgfpathlineto{\pgfqpoint{3.021262in}{0.678992in}}%
\pgfpathlineto{\pgfqpoint{3.021560in}{0.678980in}}%
\pgfpathlineto{\pgfqpoint{3.021857in}{0.678969in}}%
\pgfpathlineto{\pgfqpoint{3.022155in}{0.678958in}}%
\pgfpathlineto{\pgfqpoint{3.022452in}{0.678947in}}%
\pgfpathlineto{\pgfqpoint{3.022750in}{0.678936in}}%
\pgfpathlineto{\pgfqpoint{3.023047in}{0.678924in}}%
\pgfpathlineto{\pgfqpoint{3.023345in}{0.678913in}}%
\pgfpathlineto{\pgfqpoint{3.023642in}{0.678902in}}%
\pgfpathlineto{\pgfqpoint{3.023940in}{0.678891in}}%
\pgfpathlineto{\pgfqpoint{3.024237in}{0.678880in}}%
\pgfpathlineto{\pgfqpoint{3.024535in}{0.678868in}}%
\pgfpathlineto{\pgfqpoint{3.024832in}{0.678857in}}%
\pgfpathlineto{\pgfqpoint{3.025129in}{0.678846in}}%
\pgfpathlineto{\pgfqpoint{3.025427in}{0.678835in}}%
\pgfpathlineto{\pgfqpoint{3.025724in}{0.678824in}}%
\pgfpathlineto{\pgfqpoint{3.026022in}{0.678812in}}%
\pgfpathlineto{\pgfqpoint{3.026319in}{0.678801in}}%
\pgfpathlineto{\pgfqpoint{3.026617in}{0.678790in}}%
\pgfpathlineto{\pgfqpoint{3.026914in}{0.678779in}}%
\pgfpathlineto{\pgfqpoint{3.027212in}{0.678768in}}%
\pgfpathlineto{\pgfqpoint{3.027509in}{0.678756in}}%
\pgfpathlineto{\pgfqpoint{3.027807in}{0.678745in}}%
\pgfpathlineto{\pgfqpoint{3.028104in}{0.678734in}}%
\pgfpathlineto{\pgfqpoint{3.028402in}{0.678723in}}%
\pgfpathlineto{\pgfqpoint{3.028699in}{0.678712in}}%
\pgfpathlineto{\pgfqpoint{3.028997in}{0.678700in}}%
\pgfpathlineto{\pgfqpoint{3.029294in}{0.678689in}}%
\pgfpathlineto{\pgfqpoint{3.029592in}{0.678678in}}%
\pgfpathlineto{\pgfqpoint{3.029889in}{0.678667in}}%
\pgfpathlineto{\pgfqpoint{3.030187in}{0.678656in}}%
\pgfpathlineto{\pgfqpoint{3.030484in}{0.678644in}}%
\pgfpathlineto{\pgfqpoint{3.030782in}{0.678633in}}%
\pgfpathlineto{\pgfqpoint{3.031079in}{0.678622in}}%
\pgfpathlineto{\pgfqpoint{3.031376in}{0.678611in}}%
\pgfpathlineto{\pgfqpoint{3.031674in}{0.678600in}}%
\pgfpathlineto{\pgfqpoint{3.031971in}{0.678588in}}%
\pgfpathlineto{\pgfqpoint{3.032269in}{0.678577in}}%
\pgfpathlineto{\pgfqpoint{3.032566in}{0.678566in}}%
\pgfpathlineto{\pgfqpoint{3.032864in}{0.678555in}}%
\pgfpathlineto{\pgfqpoint{3.033161in}{0.678544in}}%
\pgfpathlineto{\pgfqpoint{3.033459in}{0.678532in}}%
\pgfpathlineto{\pgfqpoint{3.033756in}{0.678521in}}%
\pgfpathlineto{\pgfqpoint{3.034054in}{0.678510in}}%
\pgfpathlineto{\pgfqpoint{3.034351in}{0.678499in}}%
\pgfpathlineto{\pgfqpoint{3.034649in}{0.678488in}}%
\pgfpathlineto{\pgfqpoint{3.034946in}{0.678476in}}%
\pgfpathlineto{\pgfqpoint{3.035244in}{0.678465in}}%
\pgfpathlineto{\pgfqpoint{3.035541in}{0.678454in}}%
\pgfpathlineto{\pgfqpoint{3.035839in}{0.678443in}}%
\pgfpathlineto{\pgfqpoint{3.036136in}{0.678432in}}%
\pgfpathlineto{\pgfqpoint{3.036434in}{0.678420in}}%
\pgfpathlineto{\pgfqpoint{3.036731in}{0.678409in}}%
\pgfpathlineto{\pgfqpoint{3.037029in}{0.678398in}}%
\pgfpathlineto{\pgfqpoint{3.037326in}{0.678387in}}%
\pgfpathlineto{\pgfqpoint{3.037624in}{0.678376in}}%
\pgfpathlineto{\pgfqpoint{3.037921in}{0.678364in}}%
\pgfpathlineto{\pgfqpoint{3.038218in}{0.678353in}}%
\pgfpathlineto{\pgfqpoint{3.038516in}{0.678342in}}%
\pgfpathlineto{\pgfqpoint{3.038813in}{0.678331in}}%
\pgfpathlineto{\pgfqpoint{3.039111in}{0.678320in}}%
\pgfpathlineto{\pgfqpoint{3.039408in}{0.678308in}}%
\pgfpathlineto{\pgfqpoint{3.039706in}{0.678297in}}%
\pgfpathlineto{\pgfqpoint{3.040003in}{0.678286in}}%
\pgfpathlineto{\pgfqpoint{3.040301in}{0.678275in}}%
\pgfpathlineto{\pgfqpoint{3.040598in}{0.678264in}}%
\pgfpathlineto{\pgfqpoint{3.040896in}{0.678252in}}%
\pgfpathlineto{\pgfqpoint{3.041193in}{0.678241in}}%
\pgfpathlineto{\pgfqpoint{3.041491in}{0.678229in}}%
\pgfpathlineto{\pgfqpoint{3.041788in}{0.678217in}}%
\pgfpathlineto{\pgfqpoint{3.042086in}{0.678205in}}%
\pgfpathlineto{\pgfqpoint{3.042383in}{0.678192in}}%
\pgfpathlineto{\pgfqpoint{3.042681in}{0.678180in}}%
\pgfpathlineto{\pgfqpoint{3.042978in}{0.678167in}}%
\pgfpathlineto{\pgfqpoint{3.043276in}{0.678155in}}%
\pgfpathlineto{\pgfqpoint{3.043573in}{0.678143in}}%
\pgfpathlineto{\pgfqpoint{3.043871in}{0.678130in}}%
\pgfpathlineto{\pgfqpoint{3.044168in}{0.678118in}}%
\pgfpathlineto{\pgfqpoint{3.044466in}{0.678105in}}%
\pgfpathlineto{\pgfqpoint{3.044763in}{0.678093in}}%
\pgfpathlineto{\pgfqpoint{3.045060in}{0.678081in}}%
\pgfpathlineto{\pgfqpoint{3.045358in}{0.678068in}}%
\pgfpathlineto{\pgfqpoint{3.045655in}{0.678056in}}%
\pgfpathlineto{\pgfqpoint{3.045953in}{0.678044in}}%
\pgfpathlineto{\pgfqpoint{3.046250in}{0.678031in}}%
\pgfpathlineto{\pgfqpoint{3.046548in}{0.678019in}}%
\pgfpathlineto{\pgfqpoint{3.046845in}{0.678006in}}%
\pgfpathlineto{\pgfqpoint{3.047143in}{0.677994in}}%
\pgfpathlineto{\pgfqpoint{3.047440in}{0.677982in}}%
\pgfpathlineto{\pgfqpoint{3.047738in}{0.677969in}}%
\pgfpathlineto{\pgfqpoint{3.048035in}{0.677957in}}%
\pgfpathlineto{\pgfqpoint{3.048333in}{0.677945in}}%
\pgfpathlineto{\pgfqpoint{3.048630in}{0.677932in}}%
\pgfpathlineto{\pgfqpoint{3.048928in}{0.677920in}}%
\pgfpathlineto{\pgfqpoint{3.049225in}{0.677907in}}%
\pgfpathlineto{\pgfqpoint{3.049523in}{0.677895in}}%
\pgfpathlineto{\pgfqpoint{3.049820in}{0.677883in}}%
\pgfpathlineto{\pgfqpoint{3.050118in}{0.677870in}}%
\pgfpathlineto{\pgfqpoint{3.050415in}{0.677858in}}%
\pgfpathlineto{\pgfqpoint{3.050713in}{0.677846in}}%
\pgfpathlineto{\pgfqpoint{3.051010in}{0.677833in}}%
\pgfpathlineto{\pgfqpoint{3.051307in}{0.677821in}}%
\pgfpathlineto{\pgfqpoint{3.051605in}{0.677808in}}%
\pgfpathlineto{\pgfqpoint{3.051902in}{0.677796in}}%
\pgfpathlineto{\pgfqpoint{3.052200in}{0.677784in}}%
\pgfpathlineto{\pgfqpoint{3.052497in}{0.677771in}}%
\pgfpathlineto{\pgfqpoint{3.052795in}{0.677759in}}%
\pgfpathlineto{\pgfqpoint{3.053092in}{0.677746in}}%
\pgfpathlineto{\pgfqpoint{3.053390in}{0.677734in}}%
\pgfpathlineto{\pgfqpoint{3.053687in}{0.677722in}}%
\pgfpathlineto{\pgfqpoint{3.053985in}{0.677709in}}%
\pgfpathlineto{\pgfqpoint{3.054282in}{0.677697in}}%
\pgfpathlineto{\pgfqpoint{3.054580in}{0.677685in}}%
\pgfpathlineto{\pgfqpoint{3.054877in}{0.677672in}}%
\pgfpathlineto{\pgfqpoint{3.055175in}{0.677660in}}%
\pgfpathlineto{\pgfqpoint{3.055472in}{0.677647in}}%
\pgfpathlineto{\pgfqpoint{3.055770in}{0.677635in}}%
\pgfpathlineto{\pgfqpoint{3.056067in}{0.677623in}}%
\pgfpathlineto{\pgfqpoint{3.056365in}{0.677610in}}%
\pgfpathlineto{\pgfqpoint{3.056662in}{0.677598in}}%
\pgfpathlineto{\pgfqpoint{3.056960in}{0.677586in}}%
\pgfpathlineto{\pgfqpoint{3.057257in}{0.677573in}}%
\pgfpathlineto{\pgfqpoint{3.057555in}{0.677561in}}%
\pgfpathlineto{\pgfqpoint{3.057852in}{0.677548in}}%
\pgfpathlineto{\pgfqpoint{3.058149in}{0.677536in}}%
\pgfpathlineto{\pgfqpoint{3.058447in}{0.677524in}}%
\pgfpathlineto{\pgfqpoint{3.058744in}{0.677511in}}%
\pgfpathlineto{\pgfqpoint{3.059042in}{0.677499in}}%
\pgfpathlineto{\pgfqpoint{3.059339in}{0.677487in}}%
\pgfpathlineto{\pgfqpoint{3.059637in}{0.677474in}}%
\pgfpathlineto{\pgfqpoint{3.059934in}{0.677462in}}%
\pgfpathlineto{\pgfqpoint{3.060232in}{0.677449in}}%
\pgfpathlineto{\pgfqpoint{3.060529in}{0.677437in}}%
\pgfpathlineto{\pgfqpoint{3.060827in}{0.677425in}}%
\pgfpathlineto{\pgfqpoint{3.061124in}{0.677412in}}%
\pgfpathlineto{\pgfqpoint{3.061422in}{0.677400in}}%
\pgfpathlineto{\pgfqpoint{3.061719in}{0.677387in}}%
\pgfpathlineto{\pgfqpoint{3.062017in}{0.677375in}}%
\pgfpathlineto{\pgfqpoint{3.062314in}{0.677363in}}%
\pgfpathlineto{\pgfqpoint{3.062612in}{0.677350in}}%
\pgfpathlineto{\pgfqpoint{3.062909in}{0.677338in}}%
\pgfpathlineto{\pgfqpoint{3.063207in}{0.677326in}}%
\pgfpathlineto{\pgfqpoint{3.063504in}{0.677313in}}%
\pgfpathlineto{\pgfqpoint{3.063802in}{0.677301in}}%
\pgfpathlineto{\pgfqpoint{3.064099in}{0.677288in}}%
\pgfpathlineto{\pgfqpoint{3.064397in}{0.677276in}}%
\pgfpathlineto{\pgfqpoint{3.064694in}{0.677264in}}%
\pgfpathlineto{\pgfqpoint{3.064991in}{0.677251in}}%
\pgfpathlineto{\pgfqpoint{3.065289in}{0.677239in}}%
\pgfpathlineto{\pgfqpoint{3.065586in}{0.677227in}}%
\pgfpathlineto{\pgfqpoint{3.065884in}{0.677214in}}%
\pgfpathlineto{\pgfqpoint{3.066181in}{0.677202in}}%
\pgfpathlineto{\pgfqpoint{3.066479in}{0.677189in}}%
\pgfpathlineto{\pgfqpoint{3.066776in}{0.677177in}}%
\pgfpathlineto{\pgfqpoint{3.067074in}{0.677165in}}%
\pgfpathlineto{\pgfqpoint{3.067371in}{0.677152in}}%
\pgfpathlineto{\pgfqpoint{3.067669in}{0.677140in}}%
\pgfpathlineto{\pgfqpoint{3.067966in}{0.677128in}}%
\pgfpathlineto{\pgfqpoint{3.068264in}{0.677115in}}%
\pgfpathlineto{\pgfqpoint{3.068561in}{0.677103in}}%
\pgfpathlineto{\pgfqpoint{3.068859in}{0.677090in}}%
\pgfpathlineto{\pgfqpoint{3.069156in}{0.677078in}}%
\pgfpathlineto{\pgfqpoint{3.069454in}{0.677066in}}%
\pgfpathlineto{\pgfqpoint{3.069751in}{0.677053in}}%
\pgfpathlineto{\pgfqpoint{3.070049in}{0.677041in}}%
\pgfpathlineto{\pgfqpoint{3.070346in}{0.677028in}}%
\pgfpathlineto{\pgfqpoint{3.070644in}{0.677016in}}%
\pgfpathlineto{\pgfqpoint{3.070941in}{0.677004in}}%
\pgfpathlineto{\pgfqpoint{3.071238in}{0.676991in}}%
\pgfpathlineto{\pgfqpoint{3.071536in}{0.676979in}}%
\pgfpathlineto{\pgfqpoint{3.071833in}{0.676967in}}%
\pgfpathlineto{\pgfqpoint{3.072131in}{0.676954in}}%
\pgfpathlineto{\pgfqpoint{3.072428in}{0.676942in}}%
\pgfpathlineto{\pgfqpoint{3.072726in}{0.676929in}}%
\pgfpathlineto{\pgfqpoint{3.073023in}{0.676917in}}%
\pgfpathlineto{\pgfqpoint{3.073321in}{0.676905in}}%
\pgfpathlineto{\pgfqpoint{3.073618in}{0.676892in}}%
\pgfpathlineto{\pgfqpoint{3.073916in}{0.676880in}}%
\pgfpathlineto{\pgfqpoint{3.074213in}{0.676868in}}%
\pgfpathlineto{\pgfqpoint{3.074511in}{0.676855in}}%
\pgfpathlineto{\pgfqpoint{3.074808in}{0.676843in}}%
\pgfpathlineto{\pgfqpoint{3.075106in}{0.676830in}}%
\pgfpathlineto{\pgfqpoint{3.075403in}{0.676818in}}%
\pgfpathlineto{\pgfqpoint{3.075701in}{0.676806in}}%
\pgfpathlineto{\pgfqpoint{3.075998in}{0.676793in}}%
\pgfpathlineto{\pgfqpoint{3.076296in}{0.676781in}}%
\pgfpathlineto{\pgfqpoint{3.076593in}{0.676769in}}%
\pgfpathlineto{\pgfqpoint{3.076891in}{0.676756in}}%
\pgfpathlineto{\pgfqpoint{3.077188in}{0.676744in}}%
\pgfpathlineto{\pgfqpoint{3.077486in}{0.676731in}}%
\pgfpathlineto{\pgfqpoint{3.077783in}{0.676719in}}%
\pgfpathlineto{\pgfqpoint{3.078080in}{0.676707in}}%
\pgfpathlineto{\pgfqpoint{3.078378in}{0.676694in}}%
\pgfpathlineto{\pgfqpoint{3.078675in}{0.676682in}}%
\pgfpathlineto{\pgfqpoint{3.078973in}{0.676669in}}%
\pgfpathlineto{\pgfqpoint{3.079270in}{0.676657in}}%
\pgfpathlineto{\pgfqpoint{3.079568in}{0.676645in}}%
\pgfpathlineto{\pgfqpoint{3.079865in}{0.676632in}}%
\pgfpathlineto{\pgfqpoint{3.080163in}{0.676620in}}%
\pgfpathlineto{\pgfqpoint{3.080460in}{0.676608in}}%
\pgfpathlineto{\pgfqpoint{3.080758in}{0.676595in}}%
\pgfpathlineto{\pgfqpoint{3.081055in}{0.676583in}}%
\pgfpathlineto{\pgfqpoint{3.081353in}{0.676570in}}%
\pgfpathlineto{\pgfqpoint{3.081650in}{0.676558in}}%
\pgfpathlineto{\pgfqpoint{3.081948in}{0.676546in}}%
\pgfpathlineto{\pgfqpoint{3.082245in}{0.676533in}}%
\pgfpathlineto{\pgfqpoint{3.082543in}{0.676521in}}%
\pgfpathlineto{\pgfqpoint{3.082840in}{0.676509in}}%
\pgfpathlineto{\pgfqpoint{3.083138in}{0.676496in}}%
\pgfpathlineto{\pgfqpoint{3.083435in}{0.676484in}}%
\pgfpathlineto{\pgfqpoint{3.083733in}{0.676471in}}%
\pgfpathlineto{\pgfqpoint{3.084030in}{0.676459in}}%
\pgfpathlineto{\pgfqpoint{3.084328in}{0.676447in}}%
\pgfpathlineto{\pgfqpoint{3.084625in}{0.676429in}}%
\pgfpathlineto{\pgfqpoint{3.084922in}{0.676404in}}%
\pgfpathlineto{\pgfqpoint{3.085220in}{0.676380in}}%
\pgfpathlineto{\pgfqpoint{3.085517in}{0.676355in}}%
\pgfpathlineto{\pgfqpoint{3.085815in}{0.676331in}}%
\pgfpathlineto{\pgfqpoint{3.086112in}{0.676306in}}%
\pgfpathlineto{\pgfqpoint{3.086410in}{0.676282in}}%
\pgfpathlineto{\pgfqpoint{3.086707in}{0.676257in}}%
\pgfpathlineto{\pgfqpoint{3.087005in}{0.676233in}}%
\pgfpathlineto{\pgfqpoint{3.087302in}{0.676208in}}%
\pgfpathlineto{\pgfqpoint{3.087600in}{0.676184in}}%
\pgfpathlineto{\pgfqpoint{3.087897in}{0.676160in}}%
\pgfpathlineto{\pgfqpoint{3.088195in}{0.676135in}}%
\pgfpathlineto{\pgfqpoint{3.088492in}{0.676111in}}%
\pgfpathlineto{\pgfqpoint{3.088790in}{0.676086in}}%
\pgfpathlineto{\pgfqpoint{3.089087in}{0.676062in}}%
\pgfpathlineto{\pgfqpoint{3.089385in}{0.676037in}}%
\pgfpathlineto{\pgfqpoint{3.089682in}{0.676013in}}%
\pgfpathlineto{\pgfqpoint{3.089980in}{0.675988in}}%
\pgfpathlineto{\pgfqpoint{3.090277in}{0.675964in}}%
\pgfpathlineto{\pgfqpoint{3.090575in}{0.675939in}}%
\pgfpathlineto{\pgfqpoint{3.090872in}{0.675915in}}%
\pgfpathlineto{\pgfqpoint{3.091169in}{0.675890in}}%
\pgfpathlineto{\pgfqpoint{3.091467in}{0.675866in}}%
\pgfpathlineto{\pgfqpoint{3.091764in}{0.675841in}}%
\pgfpathlineto{\pgfqpoint{3.092062in}{0.675817in}}%
\pgfpathlineto{\pgfqpoint{3.092359in}{0.675792in}}%
\pgfpathlineto{\pgfqpoint{3.092657in}{0.675768in}}%
\pgfpathlineto{\pgfqpoint{3.092954in}{0.675743in}}%
\pgfpathlineto{\pgfqpoint{3.093252in}{0.675719in}}%
\pgfpathlineto{\pgfqpoint{3.093549in}{0.675694in}}%
\pgfpathlineto{\pgfqpoint{3.093847in}{0.675670in}}%
\pgfpathlineto{\pgfqpoint{3.094144in}{0.675646in}}%
\pgfpathlineto{\pgfqpoint{3.094442in}{0.675621in}}%
\pgfpathlineto{\pgfqpoint{3.094739in}{0.675597in}}%
\pgfpathlineto{\pgfqpoint{3.095037in}{0.675572in}}%
\pgfpathlineto{\pgfqpoint{3.095334in}{0.675548in}}%
\pgfpathlineto{\pgfqpoint{3.095632in}{0.675523in}}%
\pgfpathlineto{\pgfqpoint{3.095929in}{0.675499in}}%
\pgfpathlineto{\pgfqpoint{3.096227in}{0.675474in}}%
\pgfpathlineto{\pgfqpoint{3.096524in}{0.675450in}}%
\pgfpathlineto{\pgfqpoint{3.096822in}{0.675425in}}%
\pgfpathlineto{\pgfqpoint{3.097119in}{0.675401in}}%
\pgfpathlineto{\pgfqpoint{3.097417in}{0.675376in}}%
\pgfpathlineto{\pgfqpoint{3.097714in}{0.675352in}}%
\pgfpathlineto{\pgfqpoint{3.098011in}{0.675327in}}%
\pgfpathlineto{\pgfqpoint{3.098309in}{0.675303in}}%
\pgfpathlineto{\pgfqpoint{3.098606in}{0.675278in}}%
\pgfpathlineto{\pgfqpoint{3.098904in}{0.675254in}}%
\pgfpathlineto{\pgfqpoint{3.099201in}{0.675229in}}%
\pgfpathlineto{\pgfqpoint{3.099499in}{0.675205in}}%
\pgfpathlineto{\pgfqpoint{3.099796in}{0.675180in}}%
\pgfpathlineto{\pgfqpoint{3.100094in}{0.675156in}}%
\pgfpathlineto{\pgfqpoint{3.100391in}{0.675132in}}%
\pgfpathlineto{\pgfqpoint{3.100689in}{0.675107in}}%
\pgfpathlineto{\pgfqpoint{3.100986in}{0.675083in}}%
\pgfpathlineto{\pgfqpoint{3.101284in}{0.675058in}}%
\pgfpathlineto{\pgfqpoint{3.101581in}{0.675034in}}%
\pgfpathlineto{\pgfqpoint{3.101879in}{0.675009in}}%
\pgfpathlineto{\pgfqpoint{3.102176in}{0.674985in}}%
\pgfpathlineto{\pgfqpoint{3.102474in}{0.674960in}}%
\pgfpathlineto{\pgfqpoint{3.102771in}{0.674936in}}%
\pgfpathlineto{\pgfqpoint{3.103069in}{0.674911in}}%
\pgfpathlineto{\pgfqpoint{3.103366in}{0.674887in}}%
\pgfpathlineto{\pgfqpoint{3.103664in}{0.674862in}}%
\pgfpathlineto{\pgfqpoint{3.103961in}{0.674838in}}%
\pgfpathlineto{\pgfqpoint{3.104259in}{0.674813in}}%
\pgfpathlineto{\pgfqpoint{3.104556in}{0.674789in}}%
\pgfpathlineto{\pgfqpoint{3.104853in}{0.674764in}}%
\pgfpathlineto{\pgfqpoint{3.105151in}{0.674740in}}%
\pgfpathlineto{\pgfqpoint{3.105448in}{0.674715in}}%
\pgfpathlineto{\pgfqpoint{3.105746in}{0.674691in}}%
\pgfpathlineto{\pgfqpoint{3.106043in}{0.674666in}}%
\pgfpathlineto{\pgfqpoint{3.106341in}{0.674642in}}%
\pgfpathlineto{\pgfqpoint{3.106638in}{0.674618in}}%
\pgfpathlineto{\pgfqpoint{3.106936in}{0.674593in}}%
\pgfpathlineto{\pgfqpoint{3.107233in}{0.674569in}}%
\pgfpathlineto{\pgfqpoint{3.107531in}{0.674544in}}%
\pgfpathlineto{\pgfqpoint{3.107828in}{0.674520in}}%
\pgfpathlineto{\pgfqpoint{3.108126in}{0.674495in}}%
\pgfpathlineto{\pgfqpoint{3.108423in}{0.674471in}}%
\pgfpathlineto{\pgfqpoint{3.108721in}{0.674446in}}%
\pgfpathlineto{\pgfqpoint{3.109018in}{0.674422in}}%
\pgfpathlineto{\pgfqpoint{3.109316in}{0.674397in}}%
\pgfpathlineto{\pgfqpoint{3.109613in}{0.674373in}}%
\pgfpathlineto{\pgfqpoint{3.109911in}{0.674348in}}%
\pgfpathlineto{\pgfqpoint{3.110208in}{0.674324in}}%
\pgfpathlineto{\pgfqpoint{3.110506in}{0.674299in}}%
\pgfpathlineto{\pgfqpoint{3.110803in}{0.674275in}}%
\pgfpathlineto{\pgfqpoint{3.111101in}{0.674250in}}%
\pgfpathlineto{\pgfqpoint{3.111398in}{0.674226in}}%
\pgfpathlineto{\pgfqpoint{3.111695in}{0.674201in}}%
\pgfpathlineto{\pgfqpoint{3.111993in}{0.674177in}}%
\pgfpathlineto{\pgfqpoint{3.112290in}{0.674152in}}%
\pgfpathlineto{\pgfqpoint{3.112588in}{0.674128in}}%
\pgfpathlineto{\pgfqpoint{3.112885in}{0.674104in}}%
\pgfpathlineto{\pgfqpoint{3.113183in}{0.674079in}}%
\pgfpathlineto{\pgfqpoint{3.113480in}{0.674055in}}%
\pgfpathlineto{\pgfqpoint{3.113778in}{0.674030in}}%
\pgfpathlineto{\pgfqpoint{3.114075in}{0.674006in}}%
\pgfpathlineto{\pgfqpoint{3.114373in}{0.673981in}}%
\pgfpathlineto{\pgfqpoint{3.114670in}{0.673957in}}%
\pgfpathlineto{\pgfqpoint{3.114968in}{0.673932in}}%
\pgfpathlineto{\pgfqpoint{3.115265in}{0.673908in}}%
\pgfpathlineto{\pgfqpoint{3.115563in}{0.673883in}}%
\pgfpathlineto{\pgfqpoint{3.115860in}{0.673859in}}%
\pgfpathlineto{\pgfqpoint{3.116158in}{0.673834in}}%
\pgfpathlineto{\pgfqpoint{3.116455in}{0.673810in}}%
\pgfpathlineto{\pgfqpoint{3.116753in}{0.673785in}}%
\pgfpathlineto{\pgfqpoint{3.117050in}{0.673761in}}%
\pgfpathlineto{\pgfqpoint{3.117348in}{0.673736in}}%
\pgfpathlineto{\pgfqpoint{3.117645in}{0.673712in}}%
\pgfpathlineto{\pgfqpoint{3.117942in}{0.673687in}}%
\pgfpathlineto{\pgfqpoint{3.118240in}{0.673663in}}%
\pgfpathlineto{\pgfqpoint{3.118537in}{0.673638in}}%
\pgfpathlineto{\pgfqpoint{3.118835in}{0.673614in}}%
\pgfpathlineto{\pgfqpoint{3.119132in}{0.673590in}}%
\pgfpathlineto{\pgfqpoint{3.119430in}{0.673565in}}%
\pgfpathlineto{\pgfqpoint{3.119727in}{0.673541in}}%
\pgfpathlineto{\pgfqpoint{3.120025in}{0.673516in}}%
\pgfpathlineto{\pgfqpoint{3.120322in}{0.673492in}}%
\pgfpathlineto{\pgfqpoint{3.120620in}{0.673467in}}%
\pgfpathlineto{\pgfqpoint{3.120917in}{0.673443in}}%
\pgfpathlineto{\pgfqpoint{3.121215in}{0.673418in}}%
\pgfpathlineto{\pgfqpoint{3.121512in}{0.673394in}}%
\pgfpathlineto{\pgfqpoint{3.121810in}{0.673369in}}%
\pgfpathlineto{\pgfqpoint{3.122107in}{0.673345in}}%
\pgfpathlineto{\pgfqpoint{3.122405in}{0.673320in}}%
\pgfpathlineto{\pgfqpoint{3.122702in}{0.673296in}}%
\pgfpathlineto{\pgfqpoint{3.123000in}{0.673271in}}%
\pgfpathlineto{\pgfqpoint{3.123297in}{0.673247in}}%
\pgfpathlineto{\pgfqpoint{3.123595in}{0.673222in}}%
\pgfpathlineto{\pgfqpoint{3.123892in}{0.673198in}}%
\pgfpathlineto{\pgfqpoint{3.124190in}{0.673173in}}%
\pgfpathlineto{\pgfqpoint{3.124487in}{0.673149in}}%
\pgfpathlineto{\pgfqpoint{3.124784in}{0.673124in}}%
\pgfpathlineto{\pgfqpoint{3.125082in}{0.673100in}}%
\pgfpathlineto{\pgfqpoint{3.125379in}{0.673076in}}%
\pgfpathlineto{\pgfqpoint{3.125677in}{0.673051in}}%
\pgfpathlineto{\pgfqpoint{3.125974in}{0.673027in}}%
\pgfpathlineto{\pgfqpoint{3.126272in}{0.673002in}}%
\pgfpathlineto{\pgfqpoint{3.126569in}{0.672978in}}%
\pgfpathlineto{\pgfqpoint{3.126867in}{0.672953in}}%
\pgfpathlineto{\pgfqpoint{3.127164in}{0.672929in}}%
\pgfpathlineto{\pgfqpoint{3.127462in}{0.672904in}}%
\pgfpathlineto{\pgfqpoint{3.127759in}{0.672880in}}%
\pgfpathlineto{\pgfqpoint{3.128057in}{0.672856in}}%
\pgfpathlineto{\pgfqpoint{3.128354in}{0.672831in}}%
\pgfpathlineto{\pgfqpoint{3.128652in}{0.672807in}}%
\pgfpathlineto{\pgfqpoint{3.128949in}{0.672783in}}%
\pgfpathlineto{\pgfqpoint{3.129247in}{0.672759in}}%
\pgfpathlineto{\pgfqpoint{3.129544in}{0.672734in}}%
\pgfpathlineto{\pgfqpoint{3.129842in}{0.672710in}}%
\pgfpathlineto{\pgfqpoint{3.130139in}{0.672686in}}%
\pgfpathlineto{\pgfqpoint{3.130437in}{0.672661in}}%
\pgfpathlineto{\pgfqpoint{3.130734in}{0.672637in}}%
\pgfpathlineto{\pgfqpoint{3.131032in}{0.672613in}}%
\pgfpathlineto{\pgfqpoint{3.131329in}{0.672589in}}%
\pgfpathlineto{\pgfqpoint{3.131626in}{0.672564in}}%
\pgfpathlineto{\pgfqpoint{3.131924in}{0.672540in}}%
\pgfpathlineto{\pgfqpoint{3.132221in}{0.672516in}}%
\pgfpathlineto{\pgfqpoint{3.132519in}{0.672491in}}%
\pgfpathlineto{\pgfqpoint{3.132816in}{0.672467in}}%
\pgfpathlineto{\pgfqpoint{3.133114in}{0.672443in}}%
\pgfpathlineto{\pgfqpoint{3.133411in}{0.672418in}}%
\pgfpathlineto{\pgfqpoint{3.133709in}{0.672394in}}%
\pgfpathlineto{\pgfqpoint{3.134006in}{0.672370in}}%
\pgfpathlineto{\pgfqpoint{3.134304in}{0.672346in}}%
\pgfpathlineto{\pgfqpoint{3.134601in}{0.672321in}}%
\pgfpathlineto{\pgfqpoint{3.134899in}{0.672297in}}%
\pgfpathlineto{\pgfqpoint{3.135196in}{0.672273in}}%
\pgfpathlineto{\pgfqpoint{3.135494in}{0.672248in}}%
\pgfpathlineto{\pgfqpoint{3.135791in}{0.672224in}}%
\pgfpathlineto{\pgfqpoint{3.136089in}{0.672200in}}%
\pgfpathlineto{\pgfqpoint{3.136386in}{0.672175in}}%
\pgfpathlineto{\pgfqpoint{3.136684in}{0.672151in}}%
\pgfpathlineto{\pgfqpoint{3.136981in}{0.672127in}}%
\pgfpathlineto{\pgfqpoint{3.137279in}{0.672103in}}%
\pgfpathlineto{\pgfqpoint{3.137576in}{0.672078in}}%
\pgfpathlineto{\pgfqpoint{3.137873in}{0.672054in}}%
\pgfpathlineto{\pgfqpoint{3.138171in}{0.672030in}}%
\pgfpathlineto{\pgfqpoint{3.138468in}{0.672005in}}%
\pgfpathlineto{\pgfqpoint{3.138766in}{0.671981in}}%
\pgfpathlineto{\pgfqpoint{3.139063in}{0.671957in}}%
\pgfpathlineto{\pgfqpoint{3.139361in}{0.671932in}}%
\pgfpathlineto{\pgfqpoint{3.139658in}{0.671908in}}%
\pgfpathlineto{\pgfqpoint{3.139956in}{0.671884in}}%
\pgfpathlineto{\pgfqpoint{3.140253in}{0.671860in}}%
\pgfpathlineto{\pgfqpoint{3.140551in}{0.671835in}}%
\pgfpathlineto{\pgfqpoint{3.140848in}{0.671811in}}%
\pgfpathlineto{\pgfqpoint{3.141146in}{0.671787in}}%
\pgfpathlineto{\pgfqpoint{3.141443in}{0.671765in}}%
\pgfpathlineto{\pgfqpoint{3.141741in}{0.671745in}}%
\pgfpathlineto{\pgfqpoint{3.142038in}{0.671724in}}%
\pgfpathlineto{\pgfqpoint{3.142336in}{0.671704in}}%
\pgfpathlineto{\pgfqpoint{3.142633in}{0.671684in}}%
\pgfpathlineto{\pgfqpoint{3.142931in}{0.671663in}}%
\pgfpathlineto{\pgfqpoint{3.143228in}{0.671643in}}%
\pgfpathlineto{\pgfqpoint{3.143526in}{0.671623in}}%
\pgfpathlineto{\pgfqpoint{3.143823in}{0.671602in}}%
\pgfpathlineto{\pgfqpoint{3.144121in}{0.671582in}}%
\pgfpathlineto{\pgfqpoint{3.144418in}{0.671562in}}%
\pgfpathlineto{\pgfqpoint{3.144715in}{0.671541in}}%
\pgfpathlineto{\pgfqpoint{3.145013in}{0.671521in}}%
\pgfpathlineto{\pgfqpoint{3.145310in}{0.671501in}}%
\pgfpathlineto{\pgfqpoint{3.145608in}{0.671480in}}%
\pgfpathlineto{\pgfqpoint{3.145905in}{0.671460in}}%
\pgfpathlineto{\pgfqpoint{3.146203in}{0.671440in}}%
\pgfpathlineto{\pgfqpoint{3.146500in}{0.671419in}}%
\pgfpathlineto{\pgfqpoint{3.146798in}{0.671399in}}%
\pgfpathlineto{\pgfqpoint{3.147095in}{0.671378in}}%
\pgfpathlineto{\pgfqpoint{3.147393in}{0.671358in}}%
\pgfpathlineto{\pgfqpoint{3.147690in}{0.671338in}}%
\pgfpathlineto{\pgfqpoint{3.147988in}{0.671317in}}%
\pgfpathlineto{\pgfqpoint{3.148285in}{0.671297in}}%
\pgfpathlineto{\pgfqpoint{3.148583in}{0.671277in}}%
\pgfpathlineto{\pgfqpoint{3.148880in}{0.671256in}}%
\pgfpathlineto{\pgfqpoint{3.149178in}{0.671236in}}%
\pgfpathlineto{\pgfqpoint{3.149475in}{0.671216in}}%
\pgfpathlineto{\pgfqpoint{3.149773in}{0.671195in}}%
\pgfpathlineto{\pgfqpoint{3.150070in}{0.671175in}}%
\pgfpathlineto{\pgfqpoint{3.150368in}{0.671155in}}%
\pgfpathlineto{\pgfqpoint{3.150665in}{0.671134in}}%
\pgfpathlineto{\pgfqpoint{3.150963in}{0.671114in}}%
\pgfpathlineto{\pgfqpoint{3.151260in}{0.671094in}}%
\pgfpathlineto{\pgfqpoint{3.151557in}{0.671073in}}%
\pgfpathlineto{\pgfqpoint{3.151855in}{0.671053in}}%
\pgfpathlineto{\pgfqpoint{3.152152in}{0.671033in}}%
\pgfpathlineto{\pgfqpoint{3.152450in}{0.671012in}}%
\pgfpathlineto{\pgfqpoint{3.152747in}{0.670992in}}%
\pgfpathlineto{\pgfqpoint{3.153045in}{0.670972in}}%
\pgfpathlineto{\pgfqpoint{3.153342in}{0.670951in}}%
\pgfpathlineto{\pgfqpoint{3.153640in}{0.670931in}}%
\pgfpathlineto{\pgfqpoint{3.153937in}{0.670911in}}%
\pgfpathlineto{\pgfqpoint{3.154235in}{0.670890in}}%
\pgfpathlineto{\pgfqpoint{3.154532in}{0.670870in}}%
\pgfpathlineto{\pgfqpoint{3.154830in}{0.670850in}}%
\pgfpathlineto{\pgfqpoint{3.155127in}{0.670829in}}%
\pgfpathlineto{\pgfqpoint{3.155425in}{0.670809in}}%
\pgfpathlineto{\pgfqpoint{3.155722in}{0.670788in}}%
\pgfpathlineto{\pgfqpoint{3.156020in}{0.670768in}}%
\pgfpathlineto{\pgfqpoint{3.156317in}{0.670748in}}%
\pgfpathlineto{\pgfqpoint{3.156615in}{0.670727in}}%
\pgfpathlineto{\pgfqpoint{3.156912in}{0.670707in}}%
\pgfpathlineto{\pgfqpoint{3.157210in}{0.670687in}}%
\pgfpathlineto{\pgfqpoint{3.157507in}{0.670666in}}%
\pgfpathlineto{\pgfqpoint{3.157804in}{0.670646in}}%
\pgfpathlineto{\pgfqpoint{3.158102in}{0.670626in}}%
\pgfpathlineto{\pgfqpoint{3.158399in}{0.670605in}}%
\pgfpathlineto{\pgfqpoint{3.158697in}{0.670585in}}%
\pgfpathlineto{\pgfqpoint{3.158994in}{0.670565in}}%
\pgfpathlineto{\pgfqpoint{3.159292in}{0.670544in}}%
\pgfpathlineto{\pgfqpoint{3.159589in}{0.670524in}}%
\pgfpathlineto{\pgfqpoint{3.159887in}{0.670504in}}%
\pgfpathlineto{\pgfqpoint{3.160184in}{0.670483in}}%
\pgfpathlineto{\pgfqpoint{3.160482in}{0.670463in}}%
\pgfpathlineto{\pgfqpoint{3.160779in}{0.670443in}}%
\pgfpathlineto{\pgfqpoint{3.161077in}{0.670422in}}%
\pgfpathlineto{\pgfqpoint{3.161374in}{0.670402in}}%
\pgfpathlineto{\pgfqpoint{3.161672in}{0.670382in}}%
\pgfpathlineto{\pgfqpoint{3.161969in}{0.670361in}}%
\pgfpathlineto{\pgfqpoint{3.162267in}{0.670341in}}%
\pgfpathlineto{\pgfqpoint{3.162564in}{0.670321in}}%
\pgfpathlineto{\pgfqpoint{3.162862in}{0.670300in}}%
\pgfpathlineto{\pgfqpoint{3.163159in}{0.670280in}}%
\pgfpathlineto{\pgfqpoint{3.163457in}{0.670260in}}%
\pgfpathlineto{\pgfqpoint{3.163754in}{0.670239in}}%
\pgfpathlineto{\pgfqpoint{3.164052in}{0.670219in}}%
\pgfpathlineto{\pgfqpoint{3.164349in}{0.670198in}}%
\pgfpathlineto{\pgfqpoint{3.164646in}{0.670178in}}%
\pgfpathlineto{\pgfqpoint{3.164944in}{0.670158in}}%
\pgfpathlineto{\pgfqpoint{3.165241in}{0.670137in}}%
\pgfpathlineto{\pgfqpoint{3.165539in}{0.670117in}}%
\pgfpathlineto{\pgfqpoint{3.165836in}{0.670097in}}%
\pgfpathlineto{\pgfqpoint{3.166134in}{0.670076in}}%
\pgfpathlineto{\pgfqpoint{3.166431in}{0.670056in}}%
\pgfpathlineto{\pgfqpoint{3.166729in}{0.670036in}}%
\pgfpathlineto{\pgfqpoint{3.167026in}{0.670015in}}%
\pgfpathlineto{\pgfqpoint{3.167324in}{0.669995in}}%
\pgfpathlineto{\pgfqpoint{3.167621in}{0.669975in}}%
\pgfpathlineto{\pgfqpoint{3.167919in}{0.669954in}}%
\pgfpathlineto{\pgfqpoint{3.168216in}{0.669934in}}%
\pgfpathlineto{\pgfqpoint{3.168514in}{0.669914in}}%
\pgfpathlineto{\pgfqpoint{3.168811in}{0.669893in}}%
\pgfpathlineto{\pgfqpoint{3.169109in}{0.669873in}}%
\pgfpathlineto{\pgfqpoint{3.169406in}{0.669853in}}%
\pgfpathlineto{\pgfqpoint{3.169704in}{0.669832in}}%
\pgfpathlineto{\pgfqpoint{3.170001in}{0.669812in}}%
\pgfpathlineto{\pgfqpoint{3.170299in}{0.669792in}}%
\pgfpathlineto{\pgfqpoint{3.170596in}{0.669771in}}%
\pgfpathlineto{\pgfqpoint{3.170894in}{0.669751in}}%
\pgfpathlineto{\pgfqpoint{3.171191in}{0.669731in}}%
\pgfpathlineto{\pgfqpoint{3.171488in}{0.669710in}}%
\pgfpathlineto{\pgfqpoint{3.171786in}{0.669690in}}%
\pgfpathlineto{\pgfqpoint{3.172083in}{0.669670in}}%
\pgfpathlineto{\pgfqpoint{3.172381in}{0.669649in}}%
\pgfpathlineto{\pgfqpoint{3.172678in}{0.669629in}}%
\pgfpathlineto{\pgfqpoint{3.172976in}{0.669608in}}%
\pgfpathlineto{\pgfqpoint{3.173273in}{0.669588in}}%
\pgfpathlineto{\pgfqpoint{3.173571in}{0.669568in}}%
\pgfpathlineto{\pgfqpoint{3.173868in}{0.669547in}}%
\pgfpathlineto{\pgfqpoint{3.174166in}{0.669527in}}%
\pgfpathlineto{\pgfqpoint{3.174463in}{0.669507in}}%
\pgfpathlineto{\pgfqpoint{3.174761in}{0.669486in}}%
\pgfpathlineto{\pgfqpoint{3.175058in}{0.669466in}}%
\pgfpathlineto{\pgfqpoint{3.175356in}{0.669446in}}%
\pgfpathlineto{\pgfqpoint{3.175653in}{0.669425in}}%
\pgfpathlineto{\pgfqpoint{3.175951in}{0.669405in}}%
\pgfpathlineto{\pgfqpoint{3.176248in}{0.669385in}}%
\pgfpathlineto{\pgfqpoint{3.176546in}{0.669364in}}%
\pgfpathlineto{\pgfqpoint{3.176843in}{0.669344in}}%
\pgfpathlineto{\pgfqpoint{3.177141in}{0.669324in}}%
\pgfpathlineto{\pgfqpoint{3.177438in}{0.669303in}}%
\pgfpathlineto{\pgfqpoint{3.177735in}{0.669283in}}%
\pgfpathlineto{\pgfqpoint{3.178033in}{0.669263in}}%
\pgfpathlineto{\pgfqpoint{3.178330in}{0.669242in}}%
\pgfpathlineto{\pgfqpoint{3.178628in}{0.669222in}}%
\pgfpathlineto{\pgfqpoint{3.178925in}{0.669202in}}%
\pgfpathlineto{\pgfqpoint{3.179223in}{0.669181in}}%
\pgfpathlineto{\pgfqpoint{3.179520in}{0.669161in}}%
\pgfpathlineto{\pgfqpoint{3.179818in}{0.669141in}}%
\pgfpathlineto{\pgfqpoint{3.180115in}{0.669120in}}%
\pgfpathlineto{\pgfqpoint{3.180413in}{0.669100in}}%
\pgfpathlineto{\pgfqpoint{3.180710in}{0.669080in}}%
\pgfpathlineto{\pgfqpoint{3.181008in}{0.669059in}}%
\pgfpathlineto{\pgfqpoint{3.181305in}{0.669039in}}%
\pgfpathlineto{\pgfqpoint{3.181603in}{0.669018in}}%
\pgfpathlineto{\pgfqpoint{3.181900in}{0.668998in}}%
\pgfpathlineto{\pgfqpoint{3.182198in}{0.668978in}}%
\pgfpathlineto{\pgfqpoint{3.182495in}{0.668957in}}%
\pgfpathlineto{\pgfqpoint{3.182793in}{0.668937in}}%
\pgfpathlineto{\pgfqpoint{3.183090in}{0.668917in}}%
\pgfpathlineto{\pgfqpoint{3.183388in}{0.668896in}}%
\pgfpathlineto{\pgfqpoint{3.183685in}{0.668876in}}%
\pgfpathlineto{\pgfqpoint{3.183983in}{0.668856in}}%
\pgfpathlineto{\pgfqpoint{3.184280in}{0.668835in}}%
\pgfpathlineto{\pgfqpoint{3.184577in}{0.668815in}}%
\pgfpathlineto{\pgfqpoint{3.184875in}{0.668794in}}%
\pgfpathlineto{\pgfqpoint{3.185172in}{0.668774in}}%
\pgfpathlineto{\pgfqpoint{3.185470in}{0.668753in}}%
\pgfpathlineto{\pgfqpoint{3.185767in}{0.668733in}}%
\pgfpathlineto{\pgfqpoint{3.186065in}{0.668713in}}%
\pgfpathlineto{\pgfqpoint{3.186362in}{0.668692in}}%
\pgfpathlineto{\pgfqpoint{3.186660in}{0.668672in}}%
\pgfpathlineto{\pgfqpoint{3.186957in}{0.668651in}}%
\pgfpathlineto{\pgfqpoint{3.187255in}{0.668631in}}%
\pgfpathlineto{\pgfqpoint{3.187552in}{0.668610in}}%
\pgfpathlineto{\pgfqpoint{3.187850in}{0.668590in}}%
\pgfpathlineto{\pgfqpoint{3.188147in}{0.668569in}}%
\pgfpathlineto{\pgfqpoint{3.188445in}{0.668549in}}%
\pgfpathlineto{\pgfqpoint{3.188742in}{0.668529in}}%
\pgfpathlineto{\pgfqpoint{3.189040in}{0.668508in}}%
\pgfpathlineto{\pgfqpoint{3.189337in}{0.668488in}}%
\pgfpathlineto{\pgfqpoint{3.189635in}{0.668467in}}%
\pgfpathlineto{\pgfqpoint{3.189932in}{0.668447in}}%
\pgfpathlineto{\pgfqpoint{3.190230in}{0.668426in}}%
\pgfpathlineto{\pgfqpoint{3.190527in}{0.668406in}}%
\pgfpathlineto{\pgfqpoint{3.190825in}{0.668385in}}%
\pgfpathlineto{\pgfqpoint{3.191122in}{0.668365in}}%
\pgfpathlineto{\pgfqpoint{3.191419in}{0.668344in}}%
\pgfpathlineto{\pgfqpoint{3.191717in}{0.668324in}}%
\pgfpathlineto{\pgfqpoint{3.192014in}{0.668304in}}%
\pgfpathlineto{\pgfqpoint{3.192312in}{0.668283in}}%
\pgfpathlineto{\pgfqpoint{3.192609in}{0.668263in}}%
\pgfpathlineto{\pgfqpoint{3.192907in}{0.668242in}}%
\pgfpathlineto{\pgfqpoint{3.193204in}{0.668222in}}%
\pgfpathlineto{\pgfqpoint{3.193502in}{0.668201in}}%
\pgfpathlineto{\pgfqpoint{3.193799in}{0.668181in}}%
\pgfpathlineto{\pgfqpoint{3.194097in}{0.668160in}}%
\pgfpathlineto{\pgfqpoint{3.194394in}{0.668140in}}%
\pgfpathlineto{\pgfqpoint{3.194692in}{0.668120in}}%
\pgfpathlineto{\pgfqpoint{3.194989in}{0.668099in}}%
\pgfpathlineto{\pgfqpoint{3.195287in}{0.668079in}}%
\pgfpathlineto{\pgfqpoint{3.195584in}{0.668058in}}%
\pgfpathlineto{\pgfqpoint{3.195882in}{0.668038in}}%
\pgfpathlineto{\pgfqpoint{3.196179in}{0.668017in}}%
\pgfpathlineto{\pgfqpoint{3.196477in}{0.667997in}}%
\pgfpathlineto{\pgfqpoint{3.196774in}{0.667976in}}%
\pgfpathlineto{\pgfqpoint{3.197072in}{0.667956in}}%
\pgfpathlineto{\pgfqpoint{3.197369in}{0.667935in}}%
\pgfpathlineto{\pgfqpoint{3.197666in}{0.667915in}}%
\pgfpathlineto{\pgfqpoint{3.197964in}{0.667895in}}%
\pgfpathlineto{\pgfqpoint{3.198261in}{0.667874in}}%
\pgfpathlineto{\pgfqpoint{3.198559in}{0.667854in}}%
\pgfpathlineto{\pgfqpoint{3.198856in}{0.667833in}}%
\pgfpathlineto{\pgfqpoint{3.199154in}{0.667813in}}%
\pgfpathlineto{\pgfqpoint{3.199451in}{0.667792in}}%
\pgfpathlineto{\pgfqpoint{3.199749in}{0.667772in}}%
\pgfpathlineto{\pgfqpoint{3.200046in}{0.667751in}}%
\pgfpathlineto{\pgfqpoint{3.200344in}{0.667731in}}%
\pgfpathlineto{\pgfqpoint{3.200641in}{0.667711in}}%
\pgfpathlineto{\pgfqpoint{3.200939in}{0.667690in}}%
\pgfpathlineto{\pgfqpoint{3.201236in}{0.667670in}}%
\pgfpathlineto{\pgfqpoint{3.201534in}{0.667649in}}%
\pgfpathlineto{\pgfqpoint{3.201831in}{0.667629in}}%
\pgfpathlineto{\pgfqpoint{3.202129in}{0.667608in}}%
\pgfpathlineto{\pgfqpoint{3.202426in}{0.667588in}}%
\pgfpathlineto{\pgfqpoint{3.202724in}{0.667567in}}%
\pgfpathlineto{\pgfqpoint{3.203021in}{0.667547in}}%
\pgfpathlineto{\pgfqpoint{3.203319in}{0.667526in}}%
\pgfpathlineto{\pgfqpoint{3.203616in}{0.667506in}}%
\pgfpathlineto{\pgfqpoint{3.203914in}{0.667486in}}%
\pgfpathlineto{\pgfqpoint{3.204211in}{0.667465in}}%
\pgfpathlineto{\pgfqpoint{3.204508in}{0.667445in}}%
\pgfpathlineto{\pgfqpoint{3.204806in}{0.667424in}}%
\pgfpathlineto{\pgfqpoint{3.205103in}{0.667404in}}%
\pgfpathlineto{\pgfqpoint{3.205401in}{0.667383in}}%
\pgfpathlineto{\pgfqpoint{3.205698in}{0.667363in}}%
\pgfpathlineto{\pgfqpoint{3.205996in}{0.667342in}}%
\pgfpathlineto{\pgfqpoint{3.206293in}{0.667322in}}%
\pgfpathlineto{\pgfqpoint{3.206591in}{0.667301in}}%
\pgfpathlineto{\pgfqpoint{3.206888in}{0.667281in}}%
\pgfpathlineto{\pgfqpoint{3.207186in}{0.667261in}}%
\pgfpathlineto{\pgfqpoint{3.207483in}{0.667240in}}%
\pgfpathlineto{\pgfqpoint{3.207781in}{0.667220in}}%
\pgfpathlineto{\pgfqpoint{3.208078in}{0.667199in}}%
\pgfpathlineto{\pgfqpoint{3.208376in}{0.667179in}}%
\pgfpathlineto{\pgfqpoint{3.208673in}{0.667158in}}%
\pgfpathlineto{\pgfqpoint{3.208971in}{0.667138in}}%
\pgfpathlineto{\pgfqpoint{3.209268in}{0.667117in}}%
\pgfpathlineto{\pgfqpoint{3.209566in}{0.667097in}}%
\pgfpathlineto{\pgfqpoint{3.209863in}{0.667077in}}%
\pgfpathlineto{\pgfqpoint{3.210161in}{0.667056in}}%
\pgfpathlineto{\pgfqpoint{3.210458in}{0.667036in}}%
\pgfpathlineto{\pgfqpoint{3.210756in}{0.667015in}}%
\pgfpathlineto{\pgfqpoint{3.211053in}{0.666995in}}%
\pgfpathlineto{\pgfqpoint{3.211350in}{0.666974in}}%
\pgfpathlineto{\pgfqpoint{3.211648in}{0.666954in}}%
\pgfpathlineto{\pgfqpoint{3.211945in}{0.666933in}}%
\pgfpathlineto{\pgfqpoint{3.212243in}{0.666913in}}%
\pgfpathlineto{\pgfqpoint{3.212540in}{0.666892in}}%
\pgfpathlineto{\pgfqpoint{3.212838in}{0.666872in}}%
\pgfpathlineto{\pgfqpoint{3.213135in}{0.666852in}}%
\pgfpathlineto{\pgfqpoint{3.213433in}{0.666831in}}%
\pgfpathlineto{\pgfqpoint{3.213730in}{0.666811in}}%
\pgfpathlineto{\pgfqpoint{3.214028in}{0.666790in}}%
\pgfpathlineto{\pgfqpoint{3.214325in}{0.666770in}}%
\pgfpathlineto{\pgfqpoint{3.214623in}{0.666749in}}%
\pgfpathlineto{\pgfqpoint{3.214920in}{0.666729in}}%
\pgfpathlineto{\pgfqpoint{3.215218in}{0.666708in}}%
\pgfpathlineto{\pgfqpoint{3.215515in}{0.666688in}}%
\pgfpathlineto{\pgfqpoint{3.215813in}{0.666668in}}%
\pgfpathlineto{\pgfqpoint{3.216110in}{0.666647in}}%
\pgfpathlineto{\pgfqpoint{3.216408in}{0.666627in}}%
\pgfpathlineto{\pgfqpoint{3.216705in}{0.666606in}}%
\pgfpathlineto{\pgfqpoint{3.217003in}{0.666586in}}%
\pgfpathlineto{\pgfqpoint{3.217300in}{0.666565in}}%
\pgfpathlineto{\pgfqpoint{3.217597in}{0.666545in}}%
\pgfpathlineto{\pgfqpoint{3.217895in}{0.666524in}}%
\pgfpathlineto{\pgfqpoint{3.218192in}{0.666504in}}%
\pgfpathlineto{\pgfqpoint{3.218490in}{0.666483in}}%
\pgfpathlineto{\pgfqpoint{3.218787in}{0.666463in}}%
\pgfpathlineto{\pgfqpoint{3.219085in}{0.666443in}}%
\pgfpathlineto{\pgfqpoint{3.219382in}{0.666422in}}%
\pgfpathlineto{\pgfqpoint{3.219680in}{0.666402in}}%
\pgfpathlineto{\pgfqpoint{3.219977in}{0.666381in}}%
\pgfpathlineto{\pgfqpoint{3.220275in}{0.666361in}}%
\pgfpathlineto{\pgfqpoint{3.220572in}{0.666340in}}%
\pgfpathlineto{\pgfqpoint{3.220870in}{0.666320in}}%
\pgfpathlineto{\pgfqpoint{3.221167in}{0.666299in}}%
\pgfpathlineto{\pgfqpoint{3.221465in}{0.666279in}}%
\pgfpathlineto{\pgfqpoint{3.221762in}{0.666259in}}%
\pgfpathlineto{\pgfqpoint{3.222060in}{0.666238in}}%
\pgfpathlineto{\pgfqpoint{3.222357in}{0.666218in}}%
\pgfpathlineto{\pgfqpoint{3.222655in}{0.666197in}}%
\pgfpathlineto{\pgfqpoint{3.222952in}{0.666177in}}%
\pgfpathlineto{\pgfqpoint{3.223250in}{0.666156in}}%
\pgfpathlineto{\pgfqpoint{3.223547in}{0.666136in}}%
\pgfpathlineto{\pgfqpoint{3.223845in}{0.666115in}}%
\pgfpathlineto{\pgfqpoint{3.224142in}{0.666095in}}%
\pgfpathlineto{\pgfqpoint{3.224439in}{0.666074in}}%
\pgfpathlineto{\pgfqpoint{3.224737in}{0.666054in}}%
\pgfpathlineto{\pgfqpoint{3.225034in}{0.666034in}}%
\pgfpathlineto{\pgfqpoint{3.225332in}{0.666013in}}%
\pgfpathlineto{\pgfqpoint{3.225629in}{0.665993in}}%
\pgfpathlineto{\pgfqpoint{3.225927in}{0.665972in}}%
\pgfpathlineto{\pgfqpoint{3.226224in}{0.665952in}}%
\pgfpathlineto{\pgfqpoint{3.226522in}{0.665931in}}%
\pgfpathlineto{\pgfqpoint{3.226819in}{0.665911in}}%
\pgfpathlineto{\pgfqpoint{3.227117in}{0.665890in}}%
\pgfpathlineto{\pgfqpoint{3.227414in}{0.665870in}}%
\pgfpathlineto{\pgfqpoint{3.227712in}{0.665850in}}%
\pgfpathlineto{\pgfqpoint{3.228009in}{0.665829in}}%
\pgfpathlineto{\pgfqpoint{3.228307in}{0.665809in}}%
\pgfpathlineto{\pgfqpoint{3.228604in}{0.665788in}}%
\pgfpathlineto{\pgfqpoint{3.228902in}{0.665768in}}%
\pgfpathlineto{\pgfqpoint{3.229199in}{0.665747in}}%
\pgfpathlineto{\pgfqpoint{3.229497in}{0.665727in}}%
\pgfpathlineto{\pgfqpoint{3.229794in}{0.665706in}}%
\pgfpathlineto{\pgfqpoint{3.230092in}{0.665686in}}%
\pgfpathlineto{\pgfqpoint{3.230389in}{0.665665in}}%
\pgfpathlineto{\pgfqpoint{3.230687in}{0.665645in}}%
\pgfpathlineto{\pgfqpoint{3.230984in}{0.665625in}}%
\pgfpathlineto{\pgfqpoint{3.231281in}{0.665604in}}%
\pgfpathlineto{\pgfqpoint{3.231579in}{0.665584in}}%
\pgfpathlineto{\pgfqpoint{3.231876in}{0.665563in}}%
\pgfpathlineto{\pgfqpoint{3.232174in}{0.665543in}}%
\pgfpathlineto{\pgfqpoint{3.232471in}{0.665522in}}%
\pgfpathlineto{\pgfqpoint{3.232769in}{0.665502in}}%
\pgfpathlineto{\pgfqpoint{3.233066in}{0.665481in}}%
\pgfpathlineto{\pgfqpoint{3.233364in}{0.665461in}}%
\pgfpathlineto{\pgfqpoint{3.233661in}{0.665441in}}%
\pgfpathlineto{\pgfqpoint{3.233959in}{0.665420in}}%
\pgfpathlineto{\pgfqpoint{3.234256in}{0.665400in}}%
\pgfpathlineto{\pgfqpoint{3.234554in}{0.665379in}}%
\pgfpathlineto{\pgfqpoint{3.234851in}{0.665359in}}%
\pgfpathlineto{\pgfqpoint{3.235149in}{0.665338in}}%
\pgfpathlineto{\pgfqpoint{3.235446in}{0.665318in}}%
\pgfpathlineto{\pgfqpoint{3.235744in}{0.665297in}}%
\pgfpathlineto{\pgfqpoint{3.236041in}{0.665277in}}%
\pgfpathlineto{\pgfqpoint{3.236339in}{0.665256in}}%
\pgfpathlineto{\pgfqpoint{3.236636in}{0.665236in}}%
\pgfpathlineto{\pgfqpoint{3.236934in}{0.665215in}}%
\pgfpathlineto{\pgfqpoint{3.237231in}{0.665195in}}%
\pgfpathlineto{\pgfqpoint{3.237528in}{0.665174in}}%
\pgfpathlineto{\pgfqpoint{3.237826in}{0.665154in}}%
\pgfpathlineto{\pgfqpoint{3.238123in}{0.665133in}}%
\pgfpathlineto{\pgfqpoint{3.238421in}{0.665113in}}%
\pgfpathlineto{\pgfqpoint{3.238718in}{0.665092in}}%
\pgfpathlineto{\pgfqpoint{3.239016in}{0.665072in}}%
\pgfpathlineto{\pgfqpoint{3.239313in}{0.665051in}}%
\pgfpathlineto{\pgfqpoint{3.239611in}{0.665030in}}%
\pgfpathlineto{\pgfqpoint{3.239908in}{0.665010in}}%
\pgfpathlineto{\pgfqpoint{3.240206in}{0.664989in}}%
\pgfpathlineto{\pgfqpoint{3.240503in}{0.664969in}}%
\pgfpathlineto{\pgfqpoint{3.240801in}{0.664948in}}%
\pgfpathlineto{\pgfqpoint{3.241098in}{0.664928in}}%
\pgfpathlineto{\pgfqpoint{3.241396in}{0.664907in}}%
\pgfpathlineto{\pgfqpoint{3.241693in}{0.664887in}}%
\pgfpathlineto{\pgfqpoint{3.241991in}{0.664866in}}%
\pgfpathlineto{\pgfqpoint{3.242288in}{0.664846in}}%
\pgfpathlineto{\pgfqpoint{3.242586in}{0.664825in}}%
\pgfpathlineto{\pgfqpoint{3.242883in}{0.664805in}}%
\pgfpathlineto{\pgfqpoint{3.243181in}{0.664784in}}%
\pgfpathlineto{\pgfqpoint{3.243478in}{0.664764in}}%
\pgfpathlineto{\pgfqpoint{3.243776in}{0.664743in}}%
\pgfpathlineto{\pgfqpoint{3.244073in}{0.664723in}}%
\pgfpathlineto{\pgfqpoint{3.244370in}{0.664702in}}%
\pgfpathlineto{\pgfqpoint{3.244668in}{0.664682in}}%
\pgfpathlineto{\pgfqpoint{3.244965in}{0.664661in}}%
\pgfpathlineto{\pgfqpoint{3.245263in}{0.664641in}}%
\pgfpathlineto{\pgfqpoint{3.245560in}{0.664620in}}%
\pgfpathlineto{\pgfqpoint{3.245858in}{0.664599in}}%
\pgfpathlineto{\pgfqpoint{3.246155in}{0.664579in}}%
\pgfpathlineto{\pgfqpoint{3.246453in}{0.664558in}}%
\pgfpathlineto{\pgfqpoint{3.246750in}{0.664538in}}%
\pgfpathlineto{\pgfqpoint{3.247048in}{0.664517in}}%
\pgfpathlineto{\pgfqpoint{3.247345in}{0.664497in}}%
\pgfpathlineto{\pgfqpoint{3.247643in}{0.664476in}}%
\pgfpathlineto{\pgfqpoint{3.247940in}{0.664456in}}%
\pgfpathlineto{\pgfqpoint{3.248238in}{0.664435in}}%
\pgfpathlineto{\pgfqpoint{3.248535in}{0.664415in}}%
\pgfpathlineto{\pgfqpoint{3.248833in}{0.664394in}}%
\pgfpathlineto{\pgfqpoint{3.249130in}{0.664374in}}%
\pgfpathlineto{\pgfqpoint{3.249428in}{0.664353in}}%
\pgfpathlineto{\pgfqpoint{3.249725in}{0.664333in}}%
\pgfpathlineto{\pgfqpoint{3.250023in}{0.664312in}}%
\pgfpathlineto{\pgfqpoint{3.250320in}{0.664292in}}%
\pgfpathlineto{\pgfqpoint{3.250618in}{0.664271in}}%
\pgfpathlineto{\pgfqpoint{3.250915in}{0.664251in}}%
\pgfpathlineto{\pgfqpoint{3.251212in}{0.664230in}}%
\pgfpathlineto{\pgfqpoint{3.251510in}{0.664210in}}%
\pgfpathlineto{\pgfqpoint{3.251807in}{0.664189in}}%
\pgfpathlineto{\pgfqpoint{3.252105in}{0.664169in}}%
\pgfpathlineto{\pgfqpoint{3.252402in}{0.664148in}}%
\pgfpathlineto{\pgfqpoint{3.252700in}{0.664127in}}%
\pgfpathlineto{\pgfqpoint{3.252997in}{0.664107in}}%
\pgfpathlineto{\pgfqpoint{3.253295in}{0.664086in}}%
\pgfpathlineto{\pgfqpoint{3.253592in}{0.664066in}}%
\pgfpathlineto{\pgfqpoint{3.253890in}{0.664045in}}%
\pgfpathlineto{\pgfqpoint{3.254187in}{0.664025in}}%
\pgfpathlineto{\pgfqpoint{3.254485in}{0.664004in}}%
\pgfpathlineto{\pgfqpoint{3.254782in}{0.663984in}}%
\pgfpathlineto{\pgfqpoint{3.255080in}{0.663963in}}%
\pgfpathlineto{\pgfqpoint{3.255377in}{0.663943in}}%
\pgfpathlineto{\pgfqpoint{3.255675in}{0.663922in}}%
\pgfpathlineto{\pgfqpoint{3.255972in}{0.663902in}}%
\pgfpathlineto{\pgfqpoint{3.256270in}{0.663881in}}%
\pgfpathlineto{\pgfqpoint{3.256567in}{0.663861in}}%
\pgfpathlineto{\pgfqpoint{3.256865in}{0.663840in}}%
\pgfpathlineto{\pgfqpoint{3.257162in}{0.663820in}}%
\pgfpathlineto{\pgfqpoint{3.257459in}{0.663799in}}%
\pgfpathlineto{\pgfqpoint{3.257757in}{0.663779in}}%
\pgfpathlineto{\pgfqpoint{3.258054in}{0.663758in}}%
\pgfpathlineto{\pgfqpoint{3.258352in}{0.663738in}}%
\pgfpathlineto{\pgfqpoint{3.258649in}{0.663717in}}%
\pgfpathlineto{\pgfqpoint{3.258947in}{0.663696in}}%
\pgfpathlineto{\pgfqpoint{3.259244in}{0.663676in}}%
\pgfpathlineto{\pgfqpoint{3.259542in}{0.663655in}}%
\pgfpathlineto{\pgfqpoint{3.259839in}{0.663635in}}%
\pgfpathlineto{\pgfqpoint{3.260137in}{0.663614in}}%
\pgfpathlineto{\pgfqpoint{3.260434in}{0.663594in}}%
\pgfpathlineto{\pgfqpoint{3.260732in}{0.663573in}}%
\pgfpathlineto{\pgfqpoint{3.261029in}{0.663553in}}%
\pgfpathlineto{\pgfqpoint{3.261327in}{0.663532in}}%
\pgfpathlineto{\pgfqpoint{3.261624in}{0.663512in}}%
\pgfpathlineto{\pgfqpoint{3.261922in}{0.663491in}}%
\pgfpathlineto{\pgfqpoint{3.262219in}{0.663471in}}%
\pgfpathlineto{\pgfqpoint{3.262517in}{0.663450in}}%
\pgfpathlineto{\pgfqpoint{3.262814in}{0.663430in}}%
\pgfpathlineto{\pgfqpoint{3.263112in}{0.663409in}}%
\pgfpathlineto{\pgfqpoint{3.263409in}{0.663389in}}%
\pgfpathlineto{\pgfqpoint{3.263707in}{0.663368in}}%
\pgfpathlineto{\pgfqpoint{3.264004in}{0.663348in}}%
\pgfpathlineto{\pgfqpoint{3.264301in}{0.663327in}}%
\pgfpathlineto{\pgfqpoint{3.264599in}{0.663307in}}%
\pgfpathlineto{\pgfqpoint{3.264896in}{0.663286in}}%
\pgfpathlineto{\pgfqpoint{3.265194in}{0.663265in}}%
\pgfpathlineto{\pgfqpoint{3.265491in}{0.663245in}}%
\pgfpathlineto{\pgfqpoint{3.265789in}{0.663224in}}%
\pgfpathlineto{\pgfqpoint{3.266086in}{0.663204in}}%
\pgfpathlineto{\pgfqpoint{3.266384in}{0.663183in}}%
\pgfpathlineto{\pgfqpoint{3.266681in}{0.663163in}}%
\pgfpathlineto{\pgfqpoint{3.266979in}{0.663142in}}%
\pgfpathlineto{\pgfqpoint{3.267276in}{0.663122in}}%
\pgfpathlineto{\pgfqpoint{3.267574in}{0.663101in}}%
\pgfpathlineto{\pgfqpoint{3.267871in}{0.663081in}}%
\pgfpathlineto{\pgfqpoint{3.268169in}{0.663060in}}%
\pgfpathlineto{\pgfqpoint{3.268466in}{0.663040in}}%
\pgfpathlineto{\pgfqpoint{3.268764in}{0.663019in}}%
\pgfpathlineto{\pgfqpoint{3.269061in}{0.662999in}}%
\pgfpathlineto{\pgfqpoint{3.269359in}{0.662978in}}%
\pgfpathlineto{\pgfqpoint{3.269656in}{0.662958in}}%
\pgfpathlineto{\pgfqpoint{3.269954in}{0.662937in}}%
\pgfpathlineto{\pgfqpoint{3.270251in}{0.662917in}}%
\pgfpathlineto{\pgfqpoint{3.270549in}{0.662896in}}%
\pgfpathlineto{\pgfqpoint{3.270846in}{0.662876in}}%
\pgfpathlineto{\pgfqpoint{3.271143in}{0.662855in}}%
\pgfpathlineto{\pgfqpoint{3.271441in}{0.662834in}}%
\pgfpathlineto{\pgfqpoint{3.271738in}{0.662814in}}%
\pgfpathlineto{\pgfqpoint{3.272036in}{0.662793in}}%
\pgfpathlineto{\pgfqpoint{3.272333in}{0.662773in}}%
\pgfpathlineto{\pgfqpoint{3.272631in}{0.662752in}}%
\pgfpathlineto{\pgfqpoint{3.272928in}{0.662732in}}%
\pgfpathlineto{\pgfqpoint{3.273226in}{0.662711in}}%
\pgfpathlineto{\pgfqpoint{3.273523in}{0.662691in}}%
\pgfpathlineto{\pgfqpoint{3.273821in}{0.662670in}}%
\pgfpathlineto{\pgfqpoint{3.274118in}{0.662650in}}%
\pgfpathlineto{\pgfqpoint{3.274416in}{0.662629in}}%
\pgfpathlineto{\pgfqpoint{3.274713in}{0.662609in}}%
\pgfpathlineto{\pgfqpoint{3.275011in}{0.662588in}}%
\pgfpathlineto{\pgfqpoint{3.275308in}{0.662568in}}%
\pgfpathlineto{\pgfqpoint{3.275606in}{0.662547in}}%
\pgfpathlineto{\pgfqpoint{3.275903in}{0.662527in}}%
\pgfpathlineto{\pgfqpoint{3.276201in}{0.662506in}}%
\pgfpathlineto{\pgfqpoint{3.276498in}{0.662486in}}%
\pgfpathlineto{\pgfqpoint{3.276796in}{0.662465in}}%
\pgfpathlineto{\pgfqpoint{3.277093in}{0.662445in}}%
\pgfpathlineto{\pgfqpoint{3.277390in}{0.662424in}}%
\pgfpathlineto{\pgfqpoint{3.277688in}{0.662403in}}%
\pgfpathlineto{\pgfqpoint{3.277985in}{0.662383in}}%
\pgfpathlineto{\pgfqpoint{3.278283in}{0.662362in}}%
\pgfpathlineto{\pgfqpoint{3.278580in}{0.662342in}}%
\pgfpathlineto{\pgfqpoint{3.278878in}{0.662321in}}%
\pgfpathlineto{\pgfqpoint{3.279175in}{0.662301in}}%
\pgfpathlineto{\pgfqpoint{3.279473in}{0.662280in}}%
\pgfpathlineto{\pgfqpoint{3.279770in}{0.662260in}}%
\pgfpathlineto{\pgfqpoint{3.280068in}{0.662239in}}%
\pgfpathlineto{\pgfqpoint{3.280365in}{0.662219in}}%
\pgfpathlineto{\pgfqpoint{3.280663in}{0.662198in}}%
\pgfpathlineto{\pgfqpoint{3.280960in}{0.662178in}}%
\pgfpathlineto{\pgfqpoint{3.281258in}{0.662157in}}%
\pgfpathlineto{\pgfqpoint{3.281555in}{0.662137in}}%
\pgfpathlineto{\pgfqpoint{3.281853in}{0.662116in}}%
\pgfpathlineto{\pgfqpoint{3.282150in}{0.662096in}}%
\pgfpathlineto{\pgfqpoint{3.282448in}{0.662075in}}%
\pgfpathlineto{\pgfqpoint{3.282745in}{0.662055in}}%
\pgfpathlineto{\pgfqpoint{3.283043in}{0.662034in}}%
\pgfpathlineto{\pgfqpoint{3.283340in}{0.662014in}}%
\pgfpathlineto{\pgfqpoint{3.283638in}{0.661993in}}%
\pgfpathlineto{\pgfqpoint{3.283935in}{0.661973in}}%
\pgfpathlineto{\pgfqpoint{3.284232in}{0.661952in}}%
\pgfpathlineto{\pgfqpoint{3.284530in}{0.661931in}}%
\pgfpathlineto{\pgfqpoint{3.284827in}{0.661911in}}%
\pgfpathlineto{\pgfqpoint{3.285125in}{0.661890in}}%
\pgfpathlineto{\pgfqpoint{3.285422in}{0.661870in}}%
\pgfpathlineto{\pgfqpoint{3.285720in}{0.661849in}}%
\pgfpathlineto{\pgfqpoint{3.286017in}{0.661829in}}%
\pgfpathlineto{\pgfqpoint{3.286315in}{0.661808in}}%
\pgfpathlineto{\pgfqpoint{3.286612in}{0.661788in}}%
\pgfpathlineto{\pgfqpoint{3.286910in}{0.661767in}}%
\pgfpathlineto{\pgfqpoint{3.287207in}{0.661747in}}%
\pgfpathlineto{\pgfqpoint{3.287505in}{0.661726in}}%
\pgfpathlineto{\pgfqpoint{3.287802in}{0.661706in}}%
\pgfpathlineto{\pgfqpoint{3.288100in}{0.661685in}}%
\pgfpathlineto{\pgfqpoint{3.288397in}{0.661665in}}%
\pgfpathlineto{\pgfqpoint{3.288695in}{0.661644in}}%
\pgfpathlineto{\pgfqpoint{3.288992in}{0.661624in}}%
\pgfpathlineto{\pgfqpoint{3.289290in}{0.661603in}}%
\pgfpathlineto{\pgfqpoint{3.289587in}{0.661583in}}%
\pgfpathlineto{\pgfqpoint{3.289885in}{0.661562in}}%
\pgfpathlineto{\pgfqpoint{3.290182in}{0.661542in}}%
\pgfpathlineto{\pgfqpoint{3.290480in}{0.661521in}}%
\pgfpathlineto{\pgfqpoint{3.290777in}{0.661500in}}%
\pgfpathlineto{\pgfqpoint{3.291074in}{0.661480in}}%
\pgfpathlineto{\pgfqpoint{3.291372in}{0.661459in}}%
\pgfpathlineto{\pgfqpoint{3.291669in}{0.661439in}}%
\pgfpathlineto{\pgfqpoint{3.291967in}{0.661418in}}%
\pgfpathlineto{\pgfqpoint{3.292264in}{0.661398in}}%
\pgfpathlineto{\pgfqpoint{3.292562in}{0.661377in}}%
\pgfpathlineto{\pgfqpoint{3.292859in}{0.661357in}}%
\pgfpathlineto{\pgfqpoint{3.293157in}{0.661336in}}%
\pgfpathlineto{\pgfqpoint{3.293454in}{0.661316in}}%
\pgfpathlineto{\pgfqpoint{3.293752in}{0.661295in}}%
\pgfpathlineto{\pgfqpoint{3.294049in}{0.661275in}}%
\pgfpathlineto{\pgfqpoint{3.294347in}{0.661254in}}%
\pgfpathlineto{\pgfqpoint{3.294644in}{0.661234in}}%
\pgfpathlineto{\pgfqpoint{3.294942in}{0.661213in}}%
\pgfpathlineto{\pgfqpoint{3.295239in}{0.661193in}}%
\pgfpathlineto{\pgfqpoint{3.295537in}{0.661172in}}%
\pgfpathlineto{\pgfqpoint{3.295834in}{0.661152in}}%
\pgfpathlineto{\pgfqpoint{3.296132in}{0.661131in}}%
\pgfpathlineto{\pgfqpoint{3.296429in}{0.661111in}}%
\pgfpathlineto{\pgfqpoint{3.296727in}{0.661090in}}%
\pgfpathlineto{\pgfqpoint{3.297024in}{0.661069in}}%
\pgfpathlineto{\pgfqpoint{3.297321in}{0.661049in}}%
\pgfpathlineto{\pgfqpoint{3.297619in}{0.661028in}}%
\pgfpathlineto{\pgfqpoint{3.297916in}{0.661008in}}%
\pgfpathlineto{\pgfqpoint{3.298214in}{0.660987in}}%
\pgfpathlineto{\pgfqpoint{3.298511in}{0.660967in}}%
\pgfpathlineto{\pgfqpoint{3.298809in}{0.660946in}}%
\pgfpathlineto{\pgfqpoint{3.299106in}{0.660926in}}%
\pgfpathlineto{\pgfqpoint{3.299404in}{0.660905in}}%
\pgfpathlineto{\pgfqpoint{3.299701in}{0.660885in}}%
\pgfpathlineto{\pgfqpoint{3.299999in}{0.660864in}}%
\pgfpathlineto{\pgfqpoint{3.300296in}{0.660844in}}%
\pgfpathlineto{\pgfqpoint{3.300594in}{0.660823in}}%
\pgfpathlineto{\pgfqpoint{3.300891in}{0.660803in}}%
\pgfpathlineto{\pgfqpoint{3.301189in}{0.660782in}}%
\pgfpathlineto{\pgfqpoint{3.301486in}{0.660762in}}%
\pgfpathlineto{\pgfqpoint{3.301784in}{0.660741in}}%
\pgfpathlineto{\pgfqpoint{3.302081in}{0.660721in}}%
\pgfpathlineto{\pgfqpoint{3.302379in}{0.660700in}}%
\pgfpathlineto{\pgfqpoint{3.302676in}{0.660680in}}%
\pgfpathlineto{\pgfqpoint{3.302974in}{0.660659in}}%
\pgfpathlineto{\pgfqpoint{3.303271in}{0.660638in}}%
\pgfpathlineto{\pgfqpoint{3.303569in}{0.660618in}}%
\pgfpathlineto{\pgfqpoint{3.303866in}{0.660597in}}%
\pgfpathlineto{\pgfqpoint{3.304163in}{0.660577in}}%
\pgfpathlineto{\pgfqpoint{3.304461in}{0.660556in}}%
\pgfpathlineto{\pgfqpoint{3.304758in}{0.660536in}}%
\pgfpathlineto{\pgfqpoint{3.305056in}{0.660515in}}%
\pgfpathlineto{\pgfqpoint{3.305353in}{0.660495in}}%
\pgfpathlineto{\pgfqpoint{3.305651in}{0.660474in}}%
\pgfpathlineto{\pgfqpoint{3.305948in}{0.660454in}}%
\pgfpathlineto{\pgfqpoint{3.306246in}{0.660433in}}%
\pgfpathlineto{\pgfqpoint{3.306543in}{0.660413in}}%
\pgfpathlineto{\pgfqpoint{3.306841in}{0.660392in}}%
\pgfpathlineto{\pgfqpoint{3.307138in}{0.660372in}}%
\pgfpathlineto{\pgfqpoint{3.307436in}{0.660351in}}%
\pgfpathlineto{\pgfqpoint{3.307733in}{0.660331in}}%
\pgfpathlineto{\pgfqpoint{3.308031in}{0.660310in}}%
\pgfpathlineto{\pgfqpoint{3.308328in}{0.660290in}}%
\pgfpathlineto{\pgfqpoint{3.308626in}{0.660269in}}%
\pgfpathlineto{\pgfqpoint{3.308923in}{0.660249in}}%
\pgfpathlineto{\pgfqpoint{3.309221in}{0.660228in}}%
\pgfpathlineto{\pgfqpoint{3.309518in}{0.660208in}}%
\pgfpathlineto{\pgfqpoint{3.309816in}{0.660187in}}%
\pgfpathlineto{\pgfqpoint{3.310113in}{0.660166in}}%
\pgfpathlineto{\pgfqpoint{3.310411in}{0.660146in}}%
\pgfpathlineto{\pgfqpoint{3.310708in}{0.660125in}}%
\pgfpathlineto{\pgfqpoint{3.311005in}{0.660105in}}%
\pgfpathlineto{\pgfqpoint{3.311303in}{0.660084in}}%
\pgfpathlineto{\pgfqpoint{3.311600in}{0.660064in}}%
\pgfpathlineto{\pgfqpoint{3.311898in}{0.660043in}}%
\pgfpathlineto{\pgfqpoint{3.312195in}{0.660023in}}%
\pgfpathlineto{\pgfqpoint{3.312493in}{0.660002in}}%
\pgfpathlineto{\pgfqpoint{3.312790in}{0.659982in}}%
\pgfpathlineto{\pgfqpoint{3.313088in}{0.659961in}}%
\pgfpathlineto{\pgfqpoint{3.313385in}{0.659941in}}%
\pgfpathlineto{\pgfqpoint{3.313683in}{0.659920in}}%
\pgfpathlineto{\pgfqpoint{3.313980in}{0.659900in}}%
\pgfpathlineto{\pgfqpoint{3.314278in}{0.659879in}}%
\pgfpathlineto{\pgfqpoint{3.314575in}{0.659859in}}%
\pgfpathlineto{\pgfqpoint{3.314873in}{0.659838in}}%
\pgfpathlineto{\pgfqpoint{3.315170in}{0.659818in}}%
\pgfpathlineto{\pgfqpoint{3.315468in}{0.659797in}}%
\pgfpathlineto{\pgfqpoint{3.315765in}{0.659777in}}%
\pgfpathlineto{\pgfqpoint{3.316063in}{0.659756in}}%
\pgfpathlineto{\pgfqpoint{3.316360in}{0.659735in}}%
\pgfpathlineto{\pgfqpoint{3.316658in}{0.659715in}}%
\pgfpathlineto{\pgfqpoint{3.316955in}{0.659694in}}%
\pgfpathlineto{\pgfqpoint{3.317253in}{0.659674in}}%
\pgfpathlineto{\pgfqpoint{3.317550in}{0.659653in}}%
\pgfpathlineto{\pgfqpoint{3.317847in}{0.659633in}}%
\pgfpathlineto{\pgfqpoint{3.318145in}{0.659612in}}%
\pgfpathlineto{\pgfqpoint{3.318442in}{0.659592in}}%
\pgfpathlineto{\pgfqpoint{3.318740in}{0.659571in}}%
\pgfpathlineto{\pgfqpoint{3.319037in}{0.659551in}}%
\pgfpathlineto{\pgfqpoint{3.319335in}{0.659530in}}%
\pgfpathlineto{\pgfqpoint{3.319632in}{0.659510in}}%
\pgfpathlineto{\pgfqpoint{3.319930in}{0.659489in}}%
\pgfpathlineto{\pgfqpoint{3.320227in}{0.659469in}}%
\pgfpathlineto{\pgfqpoint{3.320525in}{0.659448in}}%
\pgfpathlineto{\pgfqpoint{3.320822in}{0.659428in}}%
\pgfpathlineto{\pgfqpoint{3.321120in}{0.659407in}}%
\pgfpathlineto{\pgfqpoint{3.321417in}{0.659387in}}%
\pgfpathlineto{\pgfqpoint{3.321715in}{0.659366in}}%
\pgfpathlineto{\pgfqpoint{3.322012in}{0.659346in}}%
\pgfpathlineto{\pgfqpoint{3.322310in}{0.659325in}}%
\pgfpathlineto{\pgfqpoint{3.322607in}{0.659304in}}%
\pgfpathlineto{\pgfqpoint{3.322905in}{0.659284in}}%
\pgfpathlineto{\pgfqpoint{3.323202in}{0.659263in}}%
\pgfpathlineto{\pgfqpoint{3.323500in}{0.659243in}}%
\pgfpathlineto{\pgfqpoint{3.323797in}{0.659222in}}%
\pgfpathlineto{\pgfqpoint{3.324094in}{0.659202in}}%
\pgfpathlineto{\pgfqpoint{3.324392in}{0.659181in}}%
\pgfpathlineto{\pgfqpoint{3.324689in}{0.659161in}}%
\pgfpathlineto{\pgfqpoint{3.324987in}{0.659140in}}%
\pgfpathlineto{\pgfqpoint{3.325284in}{0.659120in}}%
\pgfpathlineto{\pgfqpoint{3.325582in}{0.659099in}}%
\pgfpathlineto{\pgfqpoint{3.325879in}{0.659079in}}%
\pgfpathlineto{\pgfqpoint{3.326177in}{0.659058in}}%
\pgfpathlineto{\pgfqpoint{3.326474in}{0.659038in}}%
\pgfpathlineto{\pgfqpoint{3.326772in}{0.659017in}}%
\pgfpathlineto{\pgfqpoint{3.327069in}{0.658997in}}%
\pgfpathlineto{\pgfqpoint{3.327367in}{0.658976in}}%
\pgfpathlineto{\pgfqpoint{3.327664in}{0.658956in}}%
\pgfpathlineto{\pgfqpoint{3.327962in}{0.658935in}}%
\pgfpathlineto{\pgfqpoint{3.328259in}{0.658915in}}%
\pgfpathlineto{\pgfqpoint{3.328557in}{0.658894in}}%
\pgfpathlineto{\pgfqpoint{3.328854in}{0.658873in}}%
\pgfpathlineto{\pgfqpoint{3.329152in}{0.658853in}}%
\pgfpathlineto{\pgfqpoint{3.329449in}{0.658832in}}%
\pgfpathlineto{\pgfqpoint{3.329747in}{0.658812in}}%
\pgfpathlineto{\pgfqpoint{3.330044in}{0.658791in}}%
\pgfpathlineto{\pgfqpoint{3.330342in}{0.658771in}}%
\pgfpathlineto{\pgfqpoint{3.330639in}{0.658750in}}%
\pgfpathlineto{\pgfqpoint{3.330936in}{0.658730in}}%
\pgfpathlineto{\pgfqpoint{3.331234in}{0.658709in}}%
\pgfpathlineto{\pgfqpoint{3.331531in}{0.658689in}}%
\pgfpathlineto{\pgfqpoint{3.331829in}{0.658668in}}%
\pgfpathlineto{\pgfqpoint{3.332126in}{0.658648in}}%
\pgfpathlineto{\pgfqpoint{3.332424in}{0.658627in}}%
\pgfpathlineto{\pgfqpoint{3.332721in}{0.658607in}}%
\pgfpathlineto{\pgfqpoint{3.333019in}{0.658586in}}%
\pgfpathlineto{\pgfqpoint{3.333316in}{0.658566in}}%
\pgfpathlineto{\pgfqpoint{3.333614in}{0.658545in}}%
\pgfpathlineto{\pgfqpoint{3.333911in}{0.658525in}}%
\pgfpathlineto{\pgfqpoint{3.334209in}{0.658504in}}%
\pgfpathlineto{\pgfqpoint{3.334506in}{0.658484in}}%
\pgfpathlineto{\pgfqpoint{3.334804in}{0.658463in}}%
\pgfpathlineto{\pgfqpoint{3.335101in}{0.658443in}}%
\pgfpathlineto{\pgfqpoint{3.335399in}{0.658422in}}%
\pgfpathlineto{\pgfqpoint{3.335696in}{0.658401in}}%
\pgfpathlineto{\pgfqpoint{3.335994in}{0.658381in}}%
\pgfpathlineto{\pgfqpoint{3.336291in}{0.658360in}}%
\pgfpathlineto{\pgfqpoint{3.336589in}{0.658340in}}%
\pgfpathlineto{\pgfqpoint{3.336886in}{0.658319in}}%
\pgfpathlineto{\pgfqpoint{3.337184in}{0.658299in}}%
\pgfpathlineto{\pgfqpoint{3.337481in}{0.658278in}}%
\pgfpathlineto{\pgfqpoint{3.337778in}{0.658258in}}%
\pgfpathlineto{\pgfqpoint{3.338076in}{0.658237in}}%
\pgfpathlineto{\pgfqpoint{3.338373in}{0.658217in}}%
\pgfpathlineto{\pgfqpoint{3.338671in}{0.658196in}}%
\pgfpathlineto{\pgfqpoint{3.338968in}{0.658176in}}%
\pgfpathlineto{\pgfqpoint{3.339266in}{0.658155in}}%
\pgfpathlineto{\pgfqpoint{3.339563in}{0.658135in}}%
\pgfpathlineto{\pgfqpoint{3.339861in}{0.658114in}}%
\pgfpathlineto{\pgfqpoint{3.340158in}{0.658094in}}%
\pgfpathlineto{\pgfqpoint{3.340456in}{0.658073in}}%
\pgfpathlineto{\pgfqpoint{3.340753in}{0.658053in}}%
\pgfpathlineto{\pgfqpoint{3.341051in}{0.658032in}}%
\pgfpathlineto{\pgfqpoint{3.341348in}{0.658012in}}%
\pgfpathlineto{\pgfqpoint{3.341646in}{0.657991in}}%
\pgfpathlineto{\pgfqpoint{3.341943in}{0.657970in}}%
\pgfpathlineto{\pgfqpoint{3.342241in}{0.657950in}}%
\pgfpathlineto{\pgfqpoint{3.342538in}{0.657929in}}%
\pgfpathlineto{\pgfqpoint{3.342836in}{0.657909in}}%
\pgfpathlineto{\pgfqpoint{3.343133in}{0.657888in}}%
\pgfpathlineto{\pgfqpoint{3.343431in}{0.657868in}}%
\pgfpathlineto{\pgfqpoint{3.343728in}{0.657847in}}%
\pgfpathlineto{\pgfqpoint{3.344025in}{0.657827in}}%
\pgfpathlineto{\pgfqpoint{3.344323in}{0.657806in}}%
\pgfpathlineto{\pgfqpoint{3.344620in}{0.657786in}}%
\pgfpathlineto{\pgfqpoint{3.344918in}{0.657765in}}%
\pgfpathlineto{\pgfqpoint{3.345215in}{0.657745in}}%
\pgfpathlineto{\pgfqpoint{3.345513in}{0.657724in}}%
\pgfpathlineto{\pgfqpoint{3.345810in}{0.657704in}}%
\pgfpathlineto{\pgfqpoint{3.346108in}{0.657683in}}%
\pgfpathlineto{\pgfqpoint{3.346405in}{0.657663in}}%
\pgfpathlineto{\pgfqpoint{3.346703in}{0.657642in}}%
\pgfpathlineto{\pgfqpoint{3.347000in}{0.657622in}}%
\pgfpathlineto{\pgfqpoint{3.347298in}{0.657601in}}%
\pgfpathlineto{\pgfqpoint{3.347595in}{0.657581in}}%
\pgfpathlineto{\pgfqpoint{3.347893in}{0.657560in}}%
\pgfpathlineto{\pgfqpoint{3.348190in}{0.657539in}}%
\pgfpathlineto{\pgfqpoint{3.348488in}{0.657519in}}%
\pgfpathlineto{\pgfqpoint{3.348785in}{0.657498in}}%
\pgfpathlineto{\pgfqpoint{3.349083in}{0.657478in}}%
\pgfpathlineto{\pgfqpoint{3.349380in}{0.657457in}}%
\pgfpathlineto{\pgfqpoint{3.349678in}{0.657440in}}%
\pgfpathlineto{\pgfqpoint{3.349975in}{0.657424in}}%
\pgfpathlineto{\pgfqpoint{3.350273in}{0.657409in}}%
\pgfpathlineto{\pgfqpoint{3.350570in}{0.657393in}}%
\pgfpathlineto{\pgfqpoint{3.350867in}{0.657378in}}%
\pgfpathlineto{\pgfqpoint{3.351165in}{0.657362in}}%
\pgfpathlineto{\pgfqpoint{3.351462in}{0.657347in}}%
\pgfpathlineto{\pgfqpoint{3.351760in}{0.657331in}}%
\pgfpathlineto{\pgfqpoint{3.352057in}{0.657316in}}%
\pgfpathlineto{\pgfqpoint{3.352355in}{0.657300in}}%
\pgfpathlineto{\pgfqpoint{3.352652in}{0.657285in}}%
\pgfpathlineto{\pgfqpoint{3.352950in}{0.657269in}}%
\pgfpathlineto{\pgfqpoint{3.353247in}{0.657254in}}%
\pgfpathlineto{\pgfqpoint{3.353545in}{0.657238in}}%
\pgfpathlineto{\pgfqpoint{3.353842in}{0.657223in}}%
\pgfpathlineto{\pgfqpoint{3.354140in}{0.657207in}}%
\pgfpathlineto{\pgfqpoint{3.354437in}{0.657192in}}%
\pgfpathlineto{\pgfqpoint{3.354735in}{0.657176in}}%
\pgfpathlineto{\pgfqpoint{3.355032in}{0.657161in}}%
\pgfpathlineto{\pgfqpoint{3.355330in}{0.657145in}}%
\pgfpathlineto{\pgfqpoint{3.355627in}{0.657130in}}%
\pgfpathlineto{\pgfqpoint{3.355925in}{0.657114in}}%
\pgfpathlineto{\pgfqpoint{3.356222in}{0.657099in}}%
\pgfpathlineto{\pgfqpoint{3.356520in}{0.657083in}}%
\pgfpathlineto{\pgfqpoint{3.356817in}{0.657068in}}%
\pgfpathlineto{\pgfqpoint{3.357115in}{0.657052in}}%
\pgfpathlineto{\pgfqpoint{3.357412in}{0.657037in}}%
\pgfpathlineto{\pgfqpoint{3.357709in}{0.657021in}}%
\pgfpathlineto{\pgfqpoint{3.358007in}{0.657006in}}%
\pgfpathlineto{\pgfqpoint{3.358304in}{0.656990in}}%
\pgfpathlineto{\pgfqpoint{3.358602in}{0.656975in}}%
\pgfpathlineto{\pgfqpoint{3.358899in}{0.656959in}}%
\pgfpathlineto{\pgfqpoint{3.359197in}{0.656944in}}%
\pgfpathlineto{\pgfqpoint{3.359494in}{0.656928in}}%
\pgfpathlineto{\pgfqpoint{3.359792in}{0.656913in}}%
\pgfpathlineto{\pgfqpoint{3.360089in}{0.656897in}}%
\pgfpathlineto{\pgfqpoint{3.360387in}{0.656882in}}%
\pgfpathlineto{\pgfqpoint{3.360684in}{0.656866in}}%
\pgfpathlineto{\pgfqpoint{3.360982in}{0.656851in}}%
\pgfpathlineto{\pgfqpoint{3.361279in}{0.656835in}}%
\pgfpathlineto{\pgfqpoint{3.361577in}{0.656820in}}%
\pgfpathlineto{\pgfqpoint{3.361874in}{0.656804in}}%
\pgfpathlineto{\pgfqpoint{3.362172in}{0.656789in}}%
\pgfpathlineto{\pgfqpoint{3.362469in}{0.656773in}}%
\pgfpathlineto{\pgfqpoint{3.362767in}{0.656757in}}%
\pgfpathlineto{\pgfqpoint{3.363064in}{0.656742in}}%
\pgfpathlineto{\pgfqpoint{3.363362in}{0.656726in}}%
\pgfpathlineto{\pgfqpoint{3.363659in}{0.656711in}}%
\pgfpathlineto{\pgfqpoint{3.363956in}{0.656695in}}%
\pgfpathlineto{\pgfqpoint{3.364254in}{0.656680in}}%
\pgfpathlineto{\pgfqpoint{3.364551in}{0.656664in}}%
\pgfpathlineto{\pgfqpoint{3.364849in}{0.656649in}}%
\pgfpathlineto{\pgfqpoint{3.365146in}{0.656633in}}%
\pgfpathlineto{\pgfqpoint{3.365444in}{0.656618in}}%
\pgfpathlineto{\pgfqpoint{3.365741in}{0.656602in}}%
\pgfpathlineto{\pgfqpoint{3.366039in}{0.656587in}}%
\pgfpathlineto{\pgfqpoint{3.366336in}{0.656571in}}%
\pgfpathlineto{\pgfqpoint{3.366634in}{0.656556in}}%
\pgfpathlineto{\pgfqpoint{3.366931in}{0.656540in}}%
\pgfpathlineto{\pgfqpoint{3.367229in}{0.656525in}}%
\pgfpathlineto{\pgfqpoint{3.367526in}{0.656509in}}%
\pgfpathlineto{\pgfqpoint{3.367824in}{0.656494in}}%
\pgfpathlineto{\pgfqpoint{3.368121in}{0.656478in}}%
\pgfpathlineto{\pgfqpoint{3.368419in}{0.656463in}}%
\pgfpathlineto{\pgfqpoint{3.368716in}{0.656447in}}%
\pgfpathlineto{\pgfqpoint{3.369014in}{0.656432in}}%
\pgfpathlineto{\pgfqpoint{3.369311in}{0.656416in}}%
\pgfpathlineto{\pgfqpoint{3.369609in}{0.656401in}}%
\pgfpathlineto{\pgfqpoint{3.369906in}{0.656385in}}%
\pgfpathlineto{\pgfqpoint{3.370204in}{0.656370in}}%
\pgfpathlineto{\pgfqpoint{3.370501in}{0.656354in}}%
\pgfpathlineto{\pgfqpoint{3.370798in}{0.656339in}}%
\pgfpathlineto{\pgfqpoint{3.371096in}{0.656323in}}%
\pgfpathlineto{\pgfqpoint{3.371393in}{0.656308in}}%
\pgfpathlineto{\pgfqpoint{3.371691in}{0.656292in}}%
\pgfpathlineto{\pgfqpoint{3.371988in}{0.656277in}}%
\pgfpathlineto{\pgfqpoint{3.372286in}{0.656261in}}%
\pgfpathlineto{\pgfqpoint{3.372583in}{0.656246in}}%
\pgfpathlineto{\pgfqpoint{3.372881in}{0.656230in}}%
\pgfpathlineto{\pgfqpoint{3.373178in}{0.656215in}}%
\pgfpathlineto{\pgfqpoint{3.373476in}{0.656199in}}%
\pgfpathlineto{\pgfqpoint{3.373773in}{0.656184in}}%
\pgfpathlineto{\pgfqpoint{3.374071in}{0.656168in}}%
\pgfpathlineto{\pgfqpoint{3.374368in}{0.656153in}}%
\pgfpathlineto{\pgfqpoint{3.374666in}{0.656137in}}%
\pgfpathlineto{\pgfqpoint{3.374963in}{0.656122in}}%
\pgfpathlineto{\pgfqpoint{3.375261in}{0.656106in}}%
\pgfpathlineto{\pgfqpoint{3.375558in}{0.656091in}}%
\pgfpathlineto{\pgfqpoint{3.375856in}{0.656075in}}%
\pgfpathlineto{\pgfqpoint{3.376153in}{0.656060in}}%
\pgfpathlineto{\pgfqpoint{3.376451in}{0.656044in}}%
\pgfpathlineto{\pgfqpoint{3.376748in}{0.656029in}}%
\pgfpathlineto{\pgfqpoint{3.377046in}{0.656013in}}%
\pgfpathlineto{\pgfqpoint{3.377343in}{0.655998in}}%
\pgfpathlineto{\pgfqpoint{3.377640in}{0.655982in}}%
\pgfpathlineto{\pgfqpoint{3.377938in}{0.655967in}}%
\pgfpathlineto{\pgfqpoint{3.378235in}{0.655951in}}%
\pgfpathlineto{\pgfqpoint{3.378533in}{0.655936in}}%
\pgfpathlineto{\pgfqpoint{3.378830in}{0.655920in}}%
\pgfpathlineto{\pgfqpoint{3.379128in}{0.655905in}}%
\pgfpathlineto{\pgfqpoint{3.379425in}{0.655889in}}%
\pgfpathlineto{\pgfqpoint{3.379723in}{0.655874in}}%
\pgfpathlineto{\pgfqpoint{3.380020in}{0.655858in}}%
\pgfpathlineto{\pgfqpoint{3.380318in}{0.655843in}}%
\pgfpathlineto{\pgfqpoint{3.380615in}{0.655827in}}%
\pgfpathlineto{\pgfqpoint{3.380913in}{0.655812in}}%
\pgfpathlineto{\pgfqpoint{3.381210in}{0.655796in}}%
\pgfpathlineto{\pgfqpoint{3.381508in}{0.655781in}}%
\pgfpathlineto{\pgfqpoint{3.381805in}{0.655765in}}%
\pgfpathlineto{\pgfqpoint{3.382103in}{0.655749in}}%
\pgfpathlineto{\pgfqpoint{3.382400in}{0.655734in}}%
\pgfpathlineto{\pgfqpoint{3.382698in}{0.655718in}}%
\pgfpathlineto{\pgfqpoint{3.382995in}{0.655703in}}%
\pgfpathlineto{\pgfqpoint{3.383293in}{0.655687in}}%
\pgfpathlineto{\pgfqpoint{3.383590in}{0.655672in}}%
\pgfpathlineto{\pgfqpoint{3.383887in}{0.655656in}}%
\pgfpathlineto{\pgfqpoint{3.384185in}{0.655641in}}%
\pgfpathlineto{\pgfqpoint{3.384482in}{0.655625in}}%
\pgfpathlineto{\pgfqpoint{3.384780in}{0.655610in}}%
\pgfpathlineto{\pgfqpoint{3.385077in}{0.655594in}}%
\pgfpathlineto{\pgfqpoint{3.385375in}{0.655579in}}%
\pgfpathlineto{\pgfqpoint{3.385672in}{0.655563in}}%
\pgfpathlineto{\pgfqpoint{3.385970in}{0.655548in}}%
\pgfpathlineto{\pgfqpoint{3.386267in}{0.655532in}}%
\pgfpathlineto{\pgfqpoint{3.386565in}{0.655517in}}%
\pgfpathlineto{\pgfqpoint{3.386862in}{0.655501in}}%
\pgfpathlineto{\pgfqpoint{3.387160in}{0.655486in}}%
\pgfpathlineto{\pgfqpoint{3.387457in}{0.655470in}}%
\pgfpathlineto{\pgfqpoint{3.387755in}{0.655455in}}%
\pgfpathlineto{\pgfqpoint{3.388052in}{0.655439in}}%
\pgfpathlineto{\pgfqpoint{3.388350in}{0.655424in}}%
\pgfpathlineto{\pgfqpoint{3.388647in}{0.655408in}}%
\pgfpathlineto{\pgfqpoint{3.388945in}{0.655393in}}%
\pgfpathlineto{\pgfqpoint{3.389242in}{0.655377in}}%
\pgfpathlineto{\pgfqpoint{3.389540in}{0.655362in}}%
\pgfpathlineto{\pgfqpoint{3.389837in}{0.655346in}}%
\pgfpathlineto{\pgfqpoint{3.390135in}{0.655331in}}%
\pgfpathlineto{\pgfqpoint{3.390432in}{0.655315in}}%
\pgfpathlineto{\pgfqpoint{3.390729in}{0.655300in}}%
\pgfpathlineto{\pgfqpoint{3.391027in}{0.655284in}}%
\pgfpathlineto{\pgfqpoint{3.391324in}{0.655269in}}%
\pgfpathlineto{\pgfqpoint{3.391622in}{0.655253in}}%
\pgfpathlineto{\pgfqpoint{3.391919in}{0.655238in}}%
\pgfpathlineto{\pgfqpoint{3.392217in}{0.655222in}}%
\pgfpathlineto{\pgfqpoint{3.392514in}{0.655207in}}%
\pgfpathlineto{\pgfqpoint{3.392812in}{0.655191in}}%
\pgfpathlineto{\pgfqpoint{3.393109in}{0.655176in}}%
\pgfpathlineto{\pgfqpoint{3.393407in}{0.655160in}}%
\pgfpathlineto{\pgfqpoint{3.393704in}{0.655145in}}%
\pgfpathlineto{\pgfqpoint{3.394002in}{0.655129in}}%
\pgfpathlineto{\pgfqpoint{3.394299in}{0.655114in}}%
\pgfpathlineto{\pgfqpoint{3.394597in}{0.655098in}}%
\pgfpathlineto{\pgfqpoint{3.394894in}{0.655083in}}%
\pgfpathlineto{\pgfqpoint{3.395192in}{0.655067in}}%
\pgfpathlineto{\pgfqpoint{3.395489in}{0.655052in}}%
\pgfpathlineto{\pgfqpoint{3.395787in}{0.655036in}}%
\pgfpathlineto{\pgfqpoint{3.396084in}{0.655021in}}%
\pgfpathlineto{\pgfqpoint{3.396382in}{0.655005in}}%
\pgfpathlineto{\pgfqpoint{3.396679in}{0.654990in}}%
\pgfpathlineto{\pgfqpoint{3.396977in}{0.654974in}}%
\pgfpathlineto{\pgfqpoint{3.397274in}{0.654959in}}%
\pgfpathlineto{\pgfqpoint{3.397571in}{0.654943in}}%
\pgfpathlineto{\pgfqpoint{3.397869in}{0.654928in}}%
\pgfpathlineto{\pgfqpoint{3.398166in}{0.654912in}}%
\pgfpathlineto{\pgfqpoint{3.398464in}{0.654897in}}%
\pgfpathlineto{\pgfqpoint{3.398761in}{0.654881in}}%
\pgfpathlineto{\pgfqpoint{3.399059in}{0.654866in}}%
\pgfpathlineto{\pgfqpoint{3.399356in}{0.654850in}}%
\pgfpathlineto{\pgfqpoint{3.399654in}{0.654835in}}%
\pgfpathlineto{\pgfqpoint{3.399951in}{0.654819in}}%
\pgfpathlineto{\pgfqpoint{3.400249in}{0.654804in}}%
\pgfpathlineto{\pgfqpoint{3.400546in}{0.654788in}}%
\pgfpathlineto{\pgfqpoint{3.400844in}{0.654773in}}%
\pgfpathlineto{\pgfqpoint{3.401141in}{0.654758in}}%
\pgfpathlineto{\pgfqpoint{3.401439in}{0.654742in}}%
\pgfpathlineto{\pgfqpoint{3.401736in}{0.654727in}}%
\pgfpathlineto{\pgfqpoint{3.402034in}{0.654711in}}%
\pgfpathlineto{\pgfqpoint{3.402331in}{0.654696in}}%
\pgfpathlineto{\pgfqpoint{3.402629in}{0.654681in}}%
\pgfpathlineto{\pgfqpoint{3.402926in}{0.654665in}}%
\pgfpathlineto{\pgfqpoint{3.403224in}{0.654650in}}%
\pgfpathlineto{\pgfqpoint{3.403521in}{0.654635in}}%
\pgfpathlineto{\pgfqpoint{3.403818in}{0.654619in}}%
\pgfpathlineto{\pgfqpoint{3.404116in}{0.654604in}}%
\pgfpathlineto{\pgfqpoint{3.404413in}{0.654588in}}%
\pgfpathlineto{\pgfqpoint{3.404711in}{0.654573in}}%
\pgfpathlineto{\pgfqpoint{3.405008in}{0.654558in}}%
\pgfpathlineto{\pgfqpoint{3.405306in}{0.654542in}}%
\pgfpathlineto{\pgfqpoint{3.405603in}{0.654527in}}%
\pgfpathlineto{\pgfqpoint{3.405901in}{0.654511in}}%
\pgfpathlineto{\pgfqpoint{3.406198in}{0.654496in}}%
\pgfpathlineto{\pgfqpoint{3.406496in}{0.654481in}}%
\pgfpathlineto{\pgfqpoint{3.406793in}{0.654465in}}%
\pgfpathlineto{\pgfqpoint{3.407091in}{0.654450in}}%
\pgfpathlineto{\pgfqpoint{3.407388in}{0.654434in}}%
\pgfpathlineto{\pgfqpoint{3.407686in}{0.654419in}}%
\pgfpathlineto{\pgfqpoint{3.407983in}{0.654404in}}%
\pgfpathlineto{\pgfqpoint{3.408281in}{0.654388in}}%
\pgfpathlineto{\pgfqpoint{3.408578in}{0.654373in}}%
\pgfpathlineto{\pgfqpoint{3.408876in}{0.654358in}}%
\pgfpathlineto{\pgfqpoint{3.409173in}{0.654342in}}%
\pgfpathlineto{\pgfqpoint{3.409471in}{0.654327in}}%
\pgfpathlineto{\pgfqpoint{3.409768in}{0.654311in}}%
\pgfpathlineto{\pgfqpoint{3.410066in}{0.654296in}}%
\pgfpathlineto{\pgfqpoint{3.410363in}{0.654281in}}%
\pgfpathlineto{\pgfqpoint{3.410660in}{0.654265in}}%
\pgfpathlineto{\pgfqpoint{3.410958in}{0.654250in}}%
\pgfpathlineto{\pgfqpoint{3.411255in}{0.654234in}}%
\pgfpathlineto{\pgfqpoint{3.411553in}{0.654219in}}%
\pgfpathlineto{\pgfqpoint{3.411850in}{0.654204in}}%
\pgfpathlineto{\pgfqpoint{3.412148in}{0.654188in}}%
\pgfpathlineto{\pgfqpoint{3.412445in}{0.654173in}}%
\pgfpathlineto{\pgfqpoint{3.412743in}{0.654157in}}%
\pgfpathlineto{\pgfqpoint{3.413040in}{0.654142in}}%
\pgfpathlineto{\pgfqpoint{3.413338in}{0.654127in}}%
\pgfpathlineto{\pgfqpoint{3.413635in}{0.654111in}}%
\pgfpathlineto{\pgfqpoint{3.413933in}{0.654096in}}%
\pgfpathlineto{\pgfqpoint{3.414230in}{0.654081in}}%
\pgfpathlineto{\pgfqpoint{3.414528in}{0.654065in}}%
\pgfpathlineto{\pgfqpoint{3.414825in}{0.654050in}}%
\pgfpathlineto{\pgfqpoint{3.415123in}{0.654034in}}%
\pgfpathlineto{\pgfqpoint{3.415420in}{0.654019in}}%
\pgfpathlineto{\pgfqpoint{3.415718in}{0.654004in}}%
\pgfpathlineto{\pgfqpoint{3.416015in}{0.653988in}}%
\pgfpathlineto{\pgfqpoint{3.416313in}{0.653973in}}%
\pgfpathlineto{\pgfqpoint{3.416610in}{0.653957in}}%
\pgfpathlineto{\pgfqpoint{3.416908in}{0.653942in}}%
\pgfpathlineto{\pgfqpoint{3.417205in}{0.653927in}}%
\pgfpathlineto{\pgfqpoint{3.417502in}{0.653911in}}%
\pgfpathlineto{\pgfqpoint{3.417800in}{0.653896in}}%
\pgfpathlineto{\pgfqpoint{3.418097in}{0.653880in}}%
\pgfpathlineto{\pgfqpoint{3.418395in}{0.653865in}}%
\pgfpathlineto{\pgfqpoint{3.418692in}{0.653850in}}%
\pgfpathlineto{\pgfqpoint{3.418990in}{0.653834in}}%
\pgfpathlineto{\pgfqpoint{3.419287in}{0.653819in}}%
\pgfpathlineto{\pgfqpoint{3.419585in}{0.653803in}}%
\pgfpathlineto{\pgfqpoint{3.419882in}{0.653788in}}%
\pgfpathlineto{\pgfqpoint{3.420180in}{0.653773in}}%
\pgfpathlineto{\pgfqpoint{3.420477in}{0.653757in}}%
\pgfpathlineto{\pgfqpoint{3.420775in}{0.653742in}}%
\pgfpathlineto{\pgfqpoint{3.421072in}{0.653727in}}%
\pgfpathlineto{\pgfqpoint{3.421370in}{0.653711in}}%
\pgfpathlineto{\pgfqpoint{3.421667in}{0.653696in}}%
\pgfpathlineto{\pgfqpoint{3.421965in}{0.653680in}}%
\pgfpathlineto{\pgfqpoint{3.422262in}{0.653665in}}%
\pgfpathlineto{\pgfqpoint{3.422560in}{0.653650in}}%
\pgfpathlineto{\pgfqpoint{3.422857in}{0.653634in}}%
\pgfpathlineto{\pgfqpoint{3.423155in}{0.653619in}}%
\pgfpathlineto{\pgfqpoint{3.423452in}{0.653603in}}%
\pgfpathlineto{\pgfqpoint{3.423749in}{0.653588in}}%
\pgfpathlineto{\pgfqpoint{3.424047in}{0.653573in}}%
\pgfpathlineto{\pgfqpoint{3.424344in}{0.653557in}}%
\pgfpathlineto{\pgfqpoint{3.424642in}{0.653542in}}%
\pgfpathlineto{\pgfqpoint{3.424939in}{0.653526in}}%
\pgfpathlineto{\pgfqpoint{3.425237in}{0.653511in}}%
\pgfpathlineto{\pgfqpoint{3.425534in}{0.653496in}}%
\pgfpathlineto{\pgfqpoint{3.425832in}{0.653480in}}%
\pgfpathlineto{\pgfqpoint{3.426129in}{0.653465in}}%
\pgfpathlineto{\pgfqpoint{3.426427in}{0.653450in}}%
\pgfpathlineto{\pgfqpoint{3.426724in}{0.653434in}}%
\pgfpathlineto{\pgfqpoint{3.427022in}{0.653419in}}%
\pgfpathlineto{\pgfqpoint{3.427319in}{0.653403in}}%
\pgfpathlineto{\pgfqpoint{3.427617in}{0.653388in}}%
\pgfpathlineto{\pgfqpoint{3.427914in}{0.653373in}}%
\pgfpathlineto{\pgfqpoint{3.428212in}{0.653357in}}%
\pgfpathlineto{\pgfqpoint{3.428509in}{0.653342in}}%
\pgfpathlineto{\pgfqpoint{3.428807in}{0.653326in}}%
\pgfpathlineto{\pgfqpoint{3.429104in}{0.653311in}}%
\pgfpathlineto{\pgfqpoint{3.429402in}{0.653296in}}%
\pgfpathlineto{\pgfqpoint{3.429699in}{0.653280in}}%
\pgfpathlineto{\pgfqpoint{3.429997in}{0.653265in}}%
\pgfpathlineto{\pgfqpoint{3.430294in}{0.653249in}}%
\pgfpathlineto{\pgfqpoint{3.430591in}{0.653234in}}%
\pgfpathlineto{\pgfqpoint{3.430889in}{0.653219in}}%
\pgfpathlineto{\pgfqpoint{3.431186in}{0.653203in}}%
\pgfpathlineto{\pgfqpoint{3.431484in}{0.653188in}}%
\pgfpathlineto{\pgfqpoint{3.431781in}{0.653173in}}%
\pgfpathlineto{\pgfqpoint{3.432079in}{0.653157in}}%
\pgfpathlineto{\pgfqpoint{3.432376in}{0.653142in}}%
\pgfpathlineto{\pgfqpoint{3.432674in}{0.653126in}}%
\pgfpathlineto{\pgfqpoint{3.432971in}{0.653111in}}%
\pgfpathlineto{\pgfqpoint{3.433269in}{0.653096in}}%
\pgfpathlineto{\pgfqpoint{3.433566in}{0.653080in}}%
\pgfpathlineto{\pgfqpoint{3.433864in}{0.653065in}}%
\pgfpathlineto{\pgfqpoint{3.434161in}{0.653049in}}%
\pgfpathlineto{\pgfqpoint{3.434459in}{0.653034in}}%
\pgfpathlineto{\pgfqpoint{3.434756in}{0.653019in}}%
\pgfpathlineto{\pgfqpoint{3.435054in}{0.653003in}}%
\pgfpathlineto{\pgfqpoint{3.435351in}{0.652988in}}%
\pgfpathlineto{\pgfqpoint{3.435649in}{0.652972in}}%
\pgfpathlineto{\pgfqpoint{3.435946in}{0.652957in}}%
\pgfpathlineto{\pgfqpoint{3.436244in}{0.652942in}}%
\pgfpathlineto{\pgfqpoint{3.436541in}{0.652926in}}%
\pgfpathlineto{\pgfqpoint{3.436839in}{0.652911in}}%
\pgfpathlineto{\pgfqpoint{3.437136in}{0.652896in}}%
\pgfpathlineto{\pgfqpoint{3.437433in}{0.652880in}}%
\pgfpathlineto{\pgfqpoint{3.437731in}{0.652865in}}%
\pgfpathlineto{\pgfqpoint{3.438028in}{0.652849in}}%
\pgfpathlineto{\pgfqpoint{3.438326in}{0.652834in}}%
\pgfpathlineto{\pgfqpoint{3.438623in}{0.652819in}}%
\pgfpathlineto{\pgfqpoint{3.438921in}{0.652803in}}%
\pgfpathlineto{\pgfqpoint{3.439218in}{0.652788in}}%
\pgfpathlineto{\pgfqpoint{3.439516in}{0.652772in}}%
\pgfpathlineto{\pgfqpoint{3.439813in}{0.652757in}}%
\pgfpathlineto{\pgfqpoint{3.440111in}{0.652742in}}%
\pgfpathlineto{\pgfqpoint{3.440408in}{0.652726in}}%
\pgfpathlineto{\pgfqpoint{3.440706in}{0.652711in}}%
\pgfpathlineto{\pgfqpoint{3.441003in}{0.652695in}}%
\pgfpathlineto{\pgfqpoint{3.441301in}{0.652680in}}%
\pgfpathlineto{\pgfqpoint{3.441598in}{0.652665in}}%
\pgfpathlineto{\pgfqpoint{3.441896in}{0.652649in}}%
\pgfpathlineto{\pgfqpoint{3.442193in}{0.652634in}}%
\pgfpathlineto{\pgfqpoint{3.442491in}{0.652619in}}%
\pgfpathlineto{\pgfqpoint{3.442788in}{0.652603in}}%
\pgfpathlineto{\pgfqpoint{3.443086in}{0.652588in}}%
\pgfpathlineto{\pgfqpoint{3.443383in}{0.652572in}}%
\pgfpathlineto{\pgfqpoint{3.443680in}{0.652557in}}%
\pgfpathlineto{\pgfqpoint{3.443978in}{0.652542in}}%
\pgfpathlineto{\pgfqpoint{3.444275in}{0.652526in}}%
\pgfpathlineto{\pgfqpoint{3.444573in}{0.652511in}}%
\pgfpathlineto{\pgfqpoint{3.444870in}{0.652495in}}%
\pgfpathlineto{\pgfqpoint{3.445168in}{0.652480in}}%
\pgfpathlineto{\pgfqpoint{3.445465in}{0.652465in}}%
\pgfpathlineto{\pgfqpoint{3.445763in}{0.652449in}}%
\pgfpathlineto{\pgfqpoint{3.446060in}{0.652434in}}%
\pgfpathlineto{\pgfqpoint{3.446358in}{0.652418in}}%
\pgfpathlineto{\pgfqpoint{3.446655in}{0.652403in}}%
\pgfpathlineto{\pgfqpoint{3.446953in}{0.652388in}}%
\pgfpathlineto{\pgfqpoint{3.447250in}{0.652372in}}%
\pgfpathlineto{\pgfqpoint{3.447548in}{0.652357in}}%
\pgfpathlineto{\pgfqpoint{3.447845in}{0.652341in}}%
\pgfpathlineto{\pgfqpoint{3.448143in}{0.652326in}}%
\pgfpathlineto{\pgfqpoint{3.448440in}{0.652311in}}%
\pgfpathlineto{\pgfqpoint{3.448738in}{0.652295in}}%
\pgfpathlineto{\pgfqpoint{3.449035in}{0.652280in}}%
\pgfpathlineto{\pgfqpoint{3.449333in}{0.652265in}}%
\pgfpathlineto{\pgfqpoint{3.449630in}{0.652249in}}%
\pgfpathlineto{\pgfqpoint{3.449928in}{0.652234in}}%
\pgfpathlineto{\pgfqpoint{3.450225in}{0.652218in}}%
\pgfpathlineto{\pgfqpoint{3.450522in}{0.652203in}}%
\pgfpathlineto{\pgfqpoint{3.450820in}{0.652188in}}%
\pgfpathlineto{\pgfqpoint{3.451117in}{0.652172in}}%
\pgfpathlineto{\pgfqpoint{3.451415in}{0.652157in}}%
\pgfpathlineto{\pgfqpoint{3.451712in}{0.652141in}}%
\pgfpathlineto{\pgfqpoint{3.452010in}{0.652126in}}%
\pgfpathlineto{\pgfqpoint{3.452307in}{0.652111in}}%
\pgfpathlineto{\pgfqpoint{3.452605in}{0.652095in}}%
\pgfpathlineto{\pgfqpoint{3.452902in}{0.652080in}}%
\pgfpathlineto{\pgfqpoint{3.453200in}{0.652064in}}%
\pgfpathlineto{\pgfqpoint{3.453497in}{0.652049in}}%
\pgfpathlineto{\pgfqpoint{3.453795in}{0.652034in}}%
\pgfpathlineto{\pgfqpoint{3.454092in}{0.652018in}}%
\pgfpathlineto{\pgfqpoint{3.454390in}{0.652003in}}%
\pgfpathlineto{\pgfqpoint{3.454687in}{0.651988in}}%
\pgfpathlineto{\pgfqpoint{3.454985in}{0.651972in}}%
\pgfpathlineto{\pgfqpoint{3.455282in}{0.651957in}}%
\pgfpathlineto{\pgfqpoint{3.455580in}{0.651941in}}%
\pgfpathlineto{\pgfqpoint{3.455877in}{0.651926in}}%
\pgfpathlineto{\pgfqpoint{3.456175in}{0.651911in}}%
\pgfpathlineto{\pgfqpoint{3.456472in}{0.651895in}}%
\pgfpathlineto{\pgfqpoint{3.456770in}{0.651880in}}%
\pgfpathlineto{\pgfqpoint{3.457067in}{0.651864in}}%
\pgfpathlineto{\pgfqpoint{3.457364in}{0.651849in}}%
\pgfpathlineto{\pgfqpoint{3.457662in}{0.651834in}}%
\pgfpathlineto{\pgfqpoint{3.457959in}{0.651818in}}%
\pgfpathlineto{\pgfqpoint{3.458257in}{0.651803in}}%
\pgfpathlineto{\pgfqpoint{3.458554in}{0.651787in}}%
\pgfpathlineto{\pgfqpoint{3.458852in}{0.651772in}}%
\pgfpathlineto{\pgfqpoint{3.459149in}{0.651757in}}%
\pgfpathlineto{\pgfqpoint{3.459447in}{0.651741in}}%
\pgfpathlineto{\pgfqpoint{3.459744in}{0.651726in}}%
\pgfpathlineto{\pgfqpoint{3.460042in}{0.651711in}}%
\pgfpathlineto{\pgfqpoint{3.460339in}{0.651695in}}%
\pgfpathlineto{\pgfqpoint{3.460637in}{0.651680in}}%
\pgfpathlineto{\pgfqpoint{3.460934in}{0.651664in}}%
\pgfpathlineto{\pgfqpoint{3.461232in}{0.651649in}}%
\pgfpathlineto{\pgfqpoint{3.461529in}{0.651634in}}%
\pgfpathlineto{\pgfqpoint{3.461827in}{0.651618in}}%
\pgfpathlineto{\pgfqpoint{3.462124in}{0.651603in}}%
\pgfpathlineto{\pgfqpoint{3.462422in}{0.651587in}}%
\pgfpathlineto{\pgfqpoint{3.462719in}{0.651572in}}%
\pgfpathlineto{\pgfqpoint{3.463017in}{0.651557in}}%
\pgfpathlineto{\pgfqpoint{3.463314in}{0.651541in}}%
\pgfpathlineto{\pgfqpoint{3.463611in}{0.651526in}}%
\pgfpathlineto{\pgfqpoint{3.463909in}{0.651510in}}%
\pgfpathlineto{\pgfqpoint{3.464206in}{0.651495in}}%
\pgfpathlineto{\pgfqpoint{3.464504in}{0.651480in}}%
\pgfpathlineto{\pgfqpoint{3.464801in}{0.651464in}}%
\pgfpathlineto{\pgfqpoint{3.465099in}{0.651449in}}%
\pgfpathlineto{\pgfqpoint{3.465396in}{0.651434in}}%
\pgfpathlineto{\pgfqpoint{3.465694in}{0.651418in}}%
\pgfpathlineto{\pgfqpoint{3.465991in}{0.651403in}}%
\pgfpathlineto{\pgfqpoint{3.466289in}{0.651387in}}%
\pgfpathlineto{\pgfqpoint{3.466586in}{0.651372in}}%
\pgfpathlineto{\pgfqpoint{3.466884in}{0.651357in}}%
\pgfpathlineto{\pgfqpoint{3.467181in}{0.651341in}}%
\pgfpathlineto{\pgfqpoint{3.467479in}{0.651326in}}%
\pgfpathlineto{\pgfqpoint{3.467776in}{0.651310in}}%
\pgfpathlineto{\pgfqpoint{3.468074in}{0.651295in}}%
\pgfpathlineto{\pgfqpoint{3.468371in}{0.651280in}}%
\pgfpathlineto{\pgfqpoint{3.468669in}{0.651264in}}%
\pgfpathlineto{\pgfqpoint{3.468966in}{0.651249in}}%
\pgfpathlineto{\pgfqpoint{3.469264in}{0.651233in}}%
\pgfpathlineto{\pgfqpoint{3.469561in}{0.651218in}}%
\pgfpathlineto{\pgfqpoint{3.469859in}{0.651203in}}%
\pgfpathlineto{\pgfqpoint{3.470156in}{0.651187in}}%
\pgfpathlineto{\pgfqpoint{3.470453in}{0.651172in}}%
\pgfpathlineto{\pgfqpoint{3.470751in}{0.651157in}}%
\pgfpathlineto{\pgfqpoint{3.471048in}{0.651141in}}%
\pgfpathlineto{\pgfqpoint{3.471346in}{0.651126in}}%
\pgfpathlineto{\pgfqpoint{3.471643in}{0.651110in}}%
\pgfpathlineto{\pgfqpoint{3.471941in}{0.651095in}}%
\pgfpathlineto{\pgfqpoint{3.472238in}{0.651080in}}%
\pgfpathlineto{\pgfqpoint{3.472536in}{0.651064in}}%
\pgfpathlineto{\pgfqpoint{3.472833in}{0.651049in}}%
\pgfpathlineto{\pgfqpoint{3.473131in}{0.651033in}}%
\pgfpathlineto{\pgfqpoint{3.473428in}{0.651018in}}%
\pgfpathlineto{\pgfqpoint{3.473726in}{0.651003in}}%
\pgfpathlineto{\pgfqpoint{3.474023in}{0.650987in}}%
\pgfpathlineto{\pgfqpoint{3.474321in}{0.650972in}}%
\pgfpathlineto{\pgfqpoint{3.474618in}{0.650956in}}%
\pgfpathlineto{\pgfqpoint{3.474916in}{0.650941in}}%
\pgfpathlineto{\pgfqpoint{3.475213in}{0.650926in}}%
\pgfpathlineto{\pgfqpoint{3.475511in}{0.650910in}}%
\pgfpathlineto{\pgfqpoint{3.475808in}{0.650895in}}%
\pgfpathlineto{\pgfqpoint{3.476106in}{0.650879in}}%
\pgfpathlineto{\pgfqpoint{3.476403in}{0.650864in}}%
\pgfpathlineto{\pgfqpoint{3.476701in}{0.650849in}}%
\pgfpathlineto{\pgfqpoint{3.476998in}{0.650833in}}%
\pgfpathlineto{\pgfqpoint{3.477295in}{0.650818in}}%
\pgfpathlineto{\pgfqpoint{3.477593in}{0.650803in}}%
\pgfpathlineto{\pgfqpoint{3.477890in}{0.650787in}}%
\pgfpathlineto{\pgfqpoint{3.478188in}{0.650772in}}%
\pgfpathlineto{\pgfqpoint{3.478485in}{0.650756in}}%
\pgfpathlineto{\pgfqpoint{3.478783in}{0.650741in}}%
\pgfpathlineto{\pgfqpoint{3.479080in}{0.650726in}}%
\pgfpathlineto{\pgfqpoint{3.479378in}{0.650710in}}%
\pgfpathlineto{\pgfqpoint{3.479675in}{0.650695in}}%
\pgfpathlineto{\pgfqpoint{3.479973in}{0.650679in}}%
\pgfpathlineto{\pgfqpoint{3.480270in}{0.650664in}}%
\pgfpathlineto{\pgfqpoint{3.480568in}{0.650649in}}%
\pgfpathlineto{\pgfqpoint{3.480865in}{0.650633in}}%
\pgfpathlineto{\pgfqpoint{3.481163in}{0.650618in}}%
\pgfpathlineto{\pgfqpoint{3.481460in}{0.650602in}}%
\pgfpathlineto{\pgfqpoint{3.481758in}{0.650587in}}%
\pgfpathlineto{\pgfqpoint{3.482055in}{0.650572in}}%
\pgfpathlineto{\pgfqpoint{3.482353in}{0.650556in}}%
\pgfpathlineto{\pgfqpoint{3.482650in}{0.650541in}}%
\pgfpathlineto{\pgfqpoint{3.482948in}{0.650526in}}%
\pgfpathlineto{\pgfqpoint{3.483245in}{0.650510in}}%
\pgfpathlineto{\pgfqpoint{3.483542in}{0.650495in}}%
\pgfpathlineto{\pgfqpoint{3.483840in}{0.650479in}}%
\pgfpathlineto{\pgfqpoint{3.484137in}{0.650464in}}%
\pgfpathlineto{\pgfqpoint{3.484435in}{0.650449in}}%
\pgfpathlineto{\pgfqpoint{3.484732in}{0.650433in}}%
\pgfpathlineto{\pgfqpoint{3.485030in}{0.650418in}}%
\pgfpathlineto{\pgfqpoint{3.485327in}{0.650402in}}%
\pgfpathlineto{\pgfqpoint{3.485625in}{0.650387in}}%
\pgfpathlineto{\pgfqpoint{3.485922in}{0.650372in}}%
\pgfpathlineto{\pgfqpoint{3.486220in}{0.650356in}}%
\pgfpathlineto{\pgfqpoint{3.486517in}{0.650341in}}%
\pgfpathlineto{\pgfqpoint{3.486815in}{0.650325in}}%
\pgfpathlineto{\pgfqpoint{3.487112in}{0.650310in}}%
\pgfpathlineto{\pgfqpoint{3.487410in}{0.650295in}}%
\pgfpathlineto{\pgfqpoint{3.487707in}{0.650279in}}%
\pgfpathlineto{\pgfqpoint{3.488005in}{0.650264in}}%
\pgfpathlineto{\pgfqpoint{3.488302in}{0.650249in}}%
\pgfpathlineto{\pgfqpoint{3.488600in}{0.650233in}}%
\pgfpathlineto{\pgfqpoint{3.488897in}{0.650218in}}%
\pgfpathlineto{\pgfqpoint{3.489195in}{0.650202in}}%
\pgfpathlineto{\pgfqpoint{3.489492in}{0.650187in}}%
\pgfpathlineto{\pgfqpoint{3.489790in}{0.650172in}}%
\pgfpathlineto{\pgfqpoint{3.490087in}{0.650156in}}%
\pgfpathlineto{\pgfqpoint{3.490384in}{0.650141in}}%
\pgfpathlineto{\pgfqpoint{3.490682in}{0.650125in}}%
\pgfpathlineto{\pgfqpoint{3.490979in}{0.650110in}}%
\pgfpathlineto{\pgfqpoint{3.491277in}{0.650095in}}%
\pgfpathlineto{\pgfqpoint{3.491574in}{0.650079in}}%
\pgfpathlineto{\pgfqpoint{3.491872in}{0.650064in}}%
\pgfpathlineto{\pgfqpoint{3.492169in}{0.650048in}}%
\pgfpathlineto{\pgfqpoint{3.492467in}{0.650033in}}%
\pgfpathlineto{\pgfqpoint{3.492764in}{0.650018in}}%
\pgfpathlineto{\pgfqpoint{3.493062in}{0.650002in}}%
\pgfpathlineto{\pgfqpoint{3.493359in}{0.649987in}}%
\pgfpathlineto{\pgfqpoint{3.493657in}{0.649972in}}%
\pgfpathlineto{\pgfqpoint{3.493954in}{0.649956in}}%
\pgfpathlineto{\pgfqpoint{3.494252in}{0.649941in}}%
\pgfpathlineto{\pgfqpoint{3.494549in}{0.649925in}}%
\pgfpathlineto{\pgfqpoint{3.494847in}{0.649910in}}%
\pgfpathlineto{\pgfqpoint{3.495144in}{0.649895in}}%
\pgfpathlineto{\pgfqpoint{3.495442in}{0.649879in}}%
\pgfpathlineto{\pgfqpoint{3.495739in}{0.649864in}}%
\pgfpathlineto{\pgfqpoint{3.496037in}{0.649848in}}%
\pgfpathlineto{\pgfqpoint{3.496334in}{0.649833in}}%
\pgfpathlineto{\pgfqpoint{3.496632in}{0.649818in}}%
\pgfpathlineto{\pgfqpoint{3.496929in}{0.649802in}}%
\pgfpathlineto{\pgfqpoint{3.497226in}{0.649787in}}%
\pgfpathlineto{\pgfqpoint{3.497524in}{0.649771in}}%
\pgfpathlineto{\pgfqpoint{3.497821in}{0.649756in}}%
\pgfpathlineto{\pgfqpoint{3.498119in}{0.649741in}}%
\pgfpathlineto{\pgfqpoint{3.498416in}{0.649725in}}%
\pgfpathlineto{\pgfqpoint{3.498714in}{0.649710in}}%
\pgfpathlineto{\pgfqpoint{3.499011in}{0.649694in}}%
\pgfpathlineto{\pgfqpoint{3.499309in}{0.649679in}}%
\pgfpathlineto{\pgfqpoint{3.499606in}{0.649664in}}%
\pgfpathlineto{\pgfqpoint{3.499904in}{0.649648in}}%
\pgfpathlineto{\pgfqpoint{3.500201in}{0.649633in}}%
\pgfpathlineto{\pgfqpoint{3.500499in}{0.649618in}}%
\pgfpathlineto{\pgfqpoint{3.500796in}{0.649602in}}%
\pgfpathlineto{\pgfqpoint{3.501094in}{0.649587in}}%
\pgfpathlineto{\pgfqpoint{3.501391in}{0.649571in}}%
\pgfpathlineto{\pgfqpoint{3.501689in}{0.649556in}}%
\pgfpathlineto{\pgfqpoint{3.501986in}{0.649541in}}%
\pgfpathlineto{\pgfqpoint{3.502284in}{0.649525in}}%
\pgfpathlineto{\pgfqpoint{3.502581in}{0.649510in}}%
\pgfpathlineto{\pgfqpoint{3.502879in}{0.649494in}}%
\pgfpathlineto{\pgfqpoint{3.503176in}{0.649479in}}%
\pgfpathlineto{\pgfqpoint{3.503473in}{0.649464in}}%
\pgfpathlineto{\pgfqpoint{3.503771in}{0.649448in}}%
\pgfpathlineto{\pgfqpoint{3.504068in}{0.649433in}}%
\pgfpathlineto{\pgfqpoint{3.504366in}{0.649417in}}%
\pgfpathlineto{\pgfqpoint{3.504663in}{0.649402in}}%
\pgfpathlineto{\pgfqpoint{3.504961in}{0.649387in}}%
\pgfpathlineto{\pgfqpoint{3.505258in}{0.649371in}}%
\pgfpathlineto{\pgfqpoint{3.505556in}{0.649356in}}%
\pgfpathlineto{\pgfqpoint{3.505853in}{0.649341in}}%
\pgfpathlineto{\pgfqpoint{3.506151in}{0.649325in}}%
\pgfpathlineto{\pgfqpoint{3.506448in}{0.649310in}}%
\pgfpathlineto{\pgfqpoint{3.506746in}{0.649294in}}%
\pgfpathlineto{\pgfqpoint{3.507043in}{0.649279in}}%
\pgfpathlineto{\pgfqpoint{3.507341in}{0.649264in}}%
\pgfpathlineto{\pgfqpoint{3.507638in}{0.649248in}}%
\pgfpathlineto{\pgfqpoint{3.507936in}{0.649233in}}%
\pgfpathlineto{\pgfqpoint{3.508233in}{0.649217in}}%
\pgfpathlineto{\pgfqpoint{3.508531in}{0.649202in}}%
\pgfpathlineto{\pgfqpoint{3.508828in}{0.649187in}}%
\pgfpathlineto{\pgfqpoint{3.509126in}{0.649171in}}%
\pgfpathlineto{\pgfqpoint{3.509423in}{0.649156in}}%
\pgfpathlineto{\pgfqpoint{3.509721in}{0.649140in}}%
\pgfpathlineto{\pgfqpoint{3.510018in}{0.649125in}}%
\pgfpathlineto{\pgfqpoint{3.510315in}{0.649110in}}%
\pgfpathlineto{\pgfqpoint{3.510613in}{0.649094in}}%
\pgfpathlineto{\pgfqpoint{3.510910in}{0.649079in}}%
\pgfpathlineto{\pgfqpoint{3.511208in}{0.649064in}}%
\pgfpathlineto{\pgfqpoint{3.511505in}{0.649048in}}%
\pgfpathlineto{\pgfqpoint{3.511803in}{0.649033in}}%
\pgfpathlineto{\pgfqpoint{3.512100in}{0.649017in}}%
\pgfpathlineto{\pgfqpoint{3.512398in}{0.649002in}}%
\pgfpathlineto{\pgfqpoint{3.512695in}{0.648987in}}%
\pgfpathlineto{\pgfqpoint{3.512993in}{0.648971in}}%
\pgfpathlineto{\pgfqpoint{3.513290in}{0.648956in}}%
\pgfpathlineto{\pgfqpoint{3.513588in}{0.648940in}}%
\pgfpathlineto{\pgfqpoint{3.513885in}{0.648925in}}%
\pgfpathlineto{\pgfqpoint{3.514183in}{0.648910in}}%
\pgfpathlineto{\pgfqpoint{3.514480in}{0.648894in}}%
\pgfpathlineto{\pgfqpoint{3.514778in}{0.648879in}}%
\pgfpathlineto{\pgfqpoint{3.515075in}{0.648863in}}%
\pgfpathlineto{\pgfqpoint{3.515373in}{0.648848in}}%
\pgfpathlineto{\pgfqpoint{3.515670in}{0.648833in}}%
\pgfpathlineto{\pgfqpoint{3.515968in}{0.648817in}}%
\pgfpathlineto{\pgfqpoint{3.516265in}{0.648802in}}%
\pgfpathlineto{\pgfqpoint{3.516563in}{0.648787in}}%
\pgfpathlineto{\pgfqpoint{3.516860in}{0.648771in}}%
\pgfpathlineto{\pgfqpoint{3.517157in}{0.648756in}}%
\pgfpathlineto{\pgfqpoint{3.517455in}{0.648740in}}%
\pgfpathlineto{\pgfqpoint{3.517752in}{0.648725in}}%
\pgfpathlineto{\pgfqpoint{3.518050in}{0.648710in}}%
\pgfpathlineto{\pgfqpoint{3.518347in}{0.648694in}}%
\pgfpathlineto{\pgfqpoint{3.518645in}{0.648679in}}%
\pgfpathlineto{\pgfqpoint{3.518942in}{0.648663in}}%
\pgfpathlineto{\pgfqpoint{3.519240in}{0.648648in}}%
\pgfpathlineto{\pgfqpoint{3.519537in}{0.648633in}}%
\pgfpathlineto{\pgfqpoint{3.519835in}{0.648617in}}%
\pgfpathlineto{\pgfqpoint{3.520132in}{0.648602in}}%
\pgfpathlineto{\pgfqpoint{3.520430in}{0.648586in}}%
\pgfpathlineto{\pgfqpoint{3.520727in}{0.648571in}}%
\pgfpathlineto{\pgfqpoint{3.521025in}{0.648556in}}%
\pgfpathlineto{\pgfqpoint{3.521322in}{0.648540in}}%
\pgfpathlineto{\pgfqpoint{3.521620in}{0.648525in}}%
\pgfpathlineto{\pgfqpoint{3.521917in}{0.648510in}}%
\pgfpathlineto{\pgfqpoint{3.522215in}{0.648494in}}%
\pgfpathlineto{\pgfqpoint{3.522512in}{0.648479in}}%
\pgfpathlineto{\pgfqpoint{3.522810in}{0.648463in}}%
\pgfpathlineto{\pgfqpoint{3.523107in}{0.648448in}}%
\pgfpathlineto{\pgfqpoint{3.523404in}{0.648433in}}%
\pgfpathlineto{\pgfqpoint{3.523702in}{0.648417in}}%
\pgfpathlineto{\pgfqpoint{3.523999in}{0.648402in}}%
\pgfpathlineto{\pgfqpoint{3.524297in}{0.648386in}}%
\pgfpathlineto{\pgfqpoint{3.524594in}{0.648371in}}%
\pgfpathlineto{\pgfqpoint{3.524892in}{0.648356in}}%
\pgfpathlineto{\pgfqpoint{3.525189in}{0.648340in}}%
\pgfpathlineto{\pgfqpoint{3.525487in}{0.648325in}}%
\pgfpathlineto{\pgfqpoint{3.525784in}{0.648309in}}%
\pgfpathlineto{\pgfqpoint{3.526082in}{0.648294in}}%
\pgfpathlineto{\pgfqpoint{3.526379in}{0.648279in}}%
\pgfpathlineto{\pgfqpoint{3.526677in}{0.648263in}}%
\pgfpathlineto{\pgfqpoint{3.526974in}{0.648248in}}%
\pgfpathlineto{\pgfqpoint{3.527272in}{0.648232in}}%
\pgfpathlineto{\pgfqpoint{3.527569in}{0.648217in}}%
\pgfpathlineto{\pgfqpoint{3.527867in}{0.648202in}}%
\pgfpathlineto{\pgfqpoint{3.528164in}{0.648186in}}%
\pgfpathlineto{\pgfqpoint{3.528462in}{0.648171in}}%
\pgfpathlineto{\pgfqpoint{3.528759in}{0.648156in}}%
\pgfpathlineto{\pgfqpoint{3.529057in}{0.648140in}}%
\pgfpathlineto{\pgfqpoint{3.529354in}{0.648125in}}%
\pgfpathlineto{\pgfqpoint{3.529652in}{0.648109in}}%
\pgfpathlineto{\pgfqpoint{3.529949in}{0.648094in}}%
\pgfpathlineto{\pgfqpoint{3.530246in}{0.648079in}}%
\pgfpathlineto{\pgfqpoint{3.530544in}{0.648063in}}%
\pgfpathlineto{\pgfqpoint{3.530841in}{0.648048in}}%
\pgfpathlineto{\pgfqpoint{3.531139in}{0.648032in}}%
\pgfpathlineto{\pgfqpoint{3.531436in}{0.648017in}}%
\pgfpathlineto{\pgfqpoint{3.531734in}{0.648002in}}%
\pgfpathlineto{\pgfqpoint{3.532031in}{0.647986in}}%
\pgfpathlineto{\pgfqpoint{3.532329in}{0.647971in}}%
\pgfpathlineto{\pgfqpoint{3.532626in}{0.647955in}}%
\pgfpathlineto{\pgfqpoint{3.532924in}{0.647940in}}%
\pgfpathlineto{\pgfqpoint{3.533221in}{0.647925in}}%
\pgfpathlineto{\pgfqpoint{3.533519in}{0.647909in}}%
\pgfpathlineto{\pgfqpoint{3.533816in}{0.647894in}}%
\pgfpathlineto{\pgfqpoint{3.534114in}{0.647879in}}%
\pgfpathlineto{\pgfqpoint{3.534411in}{0.647863in}}%
\pgfpathlineto{\pgfqpoint{3.534709in}{0.647848in}}%
\pgfpathlineto{\pgfqpoint{3.535006in}{0.647832in}}%
\pgfpathlineto{\pgfqpoint{3.535304in}{0.647817in}}%
\pgfpathlineto{\pgfqpoint{3.535601in}{0.647802in}}%
\pgfpathlineto{\pgfqpoint{3.535899in}{0.647786in}}%
\pgfpathlineto{\pgfqpoint{3.536196in}{0.647771in}}%
\pgfpathlineto{\pgfqpoint{3.536494in}{0.647755in}}%
\pgfpathlineto{\pgfqpoint{3.536791in}{0.647740in}}%
\pgfpathlineto{\pgfqpoint{3.537088in}{0.647725in}}%
\pgfpathlineto{\pgfqpoint{3.537386in}{0.647709in}}%
\pgfpathlineto{\pgfqpoint{3.537683in}{0.647694in}}%
\pgfpathlineto{\pgfqpoint{3.537981in}{0.647678in}}%
\pgfpathlineto{\pgfqpoint{3.538278in}{0.647663in}}%
\pgfpathlineto{\pgfqpoint{3.538576in}{0.647648in}}%
\pgfpathlineto{\pgfqpoint{3.538873in}{0.647632in}}%
\pgfpathlineto{\pgfqpoint{3.539171in}{0.647617in}}%
\pgfpathlineto{\pgfqpoint{3.539468in}{0.647602in}}%
\pgfpathlineto{\pgfqpoint{3.539766in}{0.647586in}}%
\pgfpathlineto{\pgfqpoint{3.540063in}{0.647571in}}%
\pgfpathlineto{\pgfqpoint{3.540361in}{0.647555in}}%
\pgfpathlineto{\pgfqpoint{3.540658in}{0.647540in}}%
\pgfpathlineto{\pgfqpoint{3.540956in}{0.647525in}}%
\pgfpathlineto{\pgfqpoint{3.541253in}{0.647509in}}%
\pgfpathlineto{\pgfqpoint{3.541551in}{0.647494in}}%
\pgfpathlineto{\pgfqpoint{3.541848in}{0.647478in}}%
\pgfpathlineto{\pgfqpoint{3.542146in}{0.647463in}}%
\pgfpathlineto{\pgfqpoint{3.542443in}{0.647448in}}%
\pgfpathlineto{\pgfqpoint{3.542741in}{0.647432in}}%
\pgfpathlineto{\pgfqpoint{3.543038in}{0.647417in}}%
\pgfpathlineto{\pgfqpoint{3.543336in}{0.647401in}}%
\pgfpathlineto{\pgfqpoint{3.543633in}{0.647386in}}%
\pgfpathlineto{\pgfqpoint{3.543930in}{0.647371in}}%
\pgfpathlineto{\pgfqpoint{3.544228in}{0.647355in}}%
\pgfpathlineto{\pgfqpoint{3.544525in}{0.647340in}}%
\pgfpathlineto{\pgfqpoint{3.544823in}{0.647325in}}%
\pgfpathlineto{\pgfqpoint{3.545120in}{0.647309in}}%
\pgfpathlineto{\pgfqpoint{3.545418in}{0.647294in}}%
\pgfpathlineto{\pgfqpoint{3.545715in}{0.647278in}}%
\pgfpathlineto{\pgfqpoint{3.546013in}{0.647263in}}%
\pgfpathlineto{\pgfqpoint{3.546310in}{0.647248in}}%
\pgfpathlineto{\pgfqpoint{3.546608in}{0.647232in}}%
\pgfpathlineto{\pgfqpoint{3.546905in}{0.647217in}}%
\pgfpathlineto{\pgfqpoint{3.547203in}{0.647201in}}%
\pgfpathlineto{\pgfqpoint{3.547500in}{0.647186in}}%
\pgfpathlineto{\pgfqpoint{3.547798in}{0.647171in}}%
\pgfpathlineto{\pgfqpoint{3.548095in}{0.647155in}}%
\pgfpathlineto{\pgfqpoint{3.548393in}{0.647145in}}%
\pgfpathlineto{\pgfqpoint{3.548690in}{0.647146in}}%
\pgfpathlineto{\pgfqpoint{3.548988in}{0.647147in}}%
\pgfpathlineto{\pgfqpoint{3.549285in}{0.647147in}}%
\pgfpathlineto{\pgfqpoint{3.549583in}{0.647148in}}%
\pgfpathlineto{\pgfqpoint{3.549880in}{0.647149in}}%
\pgfpathlineto{\pgfqpoint{3.550177in}{0.647149in}}%
\pgfpathlineto{\pgfqpoint{3.550475in}{0.647137in}}%
\pgfpathlineto{\pgfqpoint{3.550772in}{0.647122in}}%
\pgfpathlineto{\pgfqpoint{3.551070in}{0.647106in}}%
\pgfpathlineto{\pgfqpoint{3.551367in}{0.647091in}}%
\pgfpathlineto{\pgfqpoint{3.551665in}{0.647075in}}%
\pgfpathlineto{\pgfqpoint{3.551962in}{0.647060in}}%
\pgfpathlineto{\pgfqpoint{3.552260in}{0.647044in}}%
\pgfpathlineto{\pgfqpoint{3.552557in}{0.647029in}}%
\pgfpathlineto{\pgfqpoint{3.552855in}{0.647013in}}%
\pgfpathlineto{\pgfqpoint{3.553152in}{0.646998in}}%
\pgfpathlineto{\pgfqpoint{3.553450in}{0.646982in}}%
\pgfpathlineto{\pgfqpoint{3.553747in}{0.646967in}}%
\pgfpathlineto{\pgfqpoint{3.554045in}{0.646951in}}%
\pgfpathlineto{\pgfqpoint{3.554342in}{0.646936in}}%
\pgfpathlineto{\pgfqpoint{3.554640in}{0.646920in}}%
\pgfpathlineto{\pgfqpoint{3.554937in}{0.646905in}}%
\pgfpathlineto{\pgfqpoint{3.555235in}{0.646889in}}%
\pgfpathlineto{\pgfqpoint{3.555532in}{0.646874in}}%
\pgfpathlineto{\pgfqpoint{3.555830in}{0.646858in}}%
\pgfpathlineto{\pgfqpoint{3.556127in}{0.646842in}}%
\pgfpathlineto{\pgfqpoint{3.556425in}{0.646826in}}%
\pgfpathlineto{\pgfqpoint{3.556722in}{0.646809in}}%
\pgfpathlineto{\pgfqpoint{3.557019in}{0.646793in}}%
\pgfpathlineto{\pgfqpoint{3.557317in}{0.646777in}}%
\pgfpathlineto{\pgfqpoint{3.557614in}{0.646760in}}%
\pgfpathlineto{\pgfqpoint{3.557912in}{0.646744in}}%
\pgfpathlineto{\pgfqpoint{3.558209in}{0.646727in}}%
\pgfpathlineto{\pgfqpoint{3.558507in}{0.646711in}}%
\pgfpathlineto{\pgfqpoint{3.558804in}{0.646695in}}%
\pgfpathlineto{\pgfqpoint{3.559102in}{0.646678in}}%
\pgfpathlineto{\pgfqpoint{3.559399in}{0.646662in}}%
\pgfpathlineto{\pgfqpoint{3.559697in}{0.646645in}}%
\pgfpathlineto{\pgfqpoint{3.559994in}{0.646629in}}%
\pgfpathlineto{\pgfqpoint{3.560292in}{0.646612in}}%
\pgfpathlineto{\pgfqpoint{3.560589in}{0.646596in}}%
\pgfpathlineto{\pgfqpoint{3.560887in}{0.646580in}}%
\pgfpathlineto{\pgfqpoint{3.561184in}{0.646563in}}%
\pgfpathlineto{\pgfqpoint{3.561482in}{0.646547in}}%
\pgfpathlineto{\pgfqpoint{3.561779in}{0.646530in}}%
\pgfpathlineto{\pgfqpoint{3.562077in}{0.646514in}}%
\pgfpathlineto{\pgfqpoint{3.562374in}{0.646498in}}%
\pgfpathlineto{\pgfqpoint{3.562672in}{0.646481in}}%
\pgfpathlineto{\pgfqpoint{3.562969in}{0.646465in}}%
\pgfpathlineto{\pgfqpoint{3.563267in}{0.646448in}}%
\pgfpathlineto{\pgfqpoint{3.563564in}{0.646432in}}%
\pgfpathlineto{\pgfqpoint{3.563861in}{0.646415in}}%
\pgfpathlineto{\pgfqpoint{3.564159in}{0.646399in}}%
\pgfpathlineto{\pgfqpoint{3.564456in}{0.646383in}}%
\pgfpathlineto{\pgfqpoint{3.564754in}{0.646366in}}%
\pgfpathlineto{\pgfqpoint{3.565051in}{0.646350in}}%
\pgfpathlineto{\pgfqpoint{3.565349in}{0.646333in}}%
\pgfpathlineto{\pgfqpoint{3.565646in}{0.646317in}}%
\pgfpathlineto{\pgfqpoint{3.565944in}{0.646300in}}%
\pgfpathlineto{\pgfqpoint{3.566241in}{0.646284in}}%
\pgfpathlineto{\pgfqpoint{3.566539in}{0.646268in}}%
\pgfpathlineto{\pgfqpoint{3.566836in}{0.646251in}}%
\pgfpathlineto{\pgfqpoint{3.567134in}{0.646235in}}%
\pgfpathlineto{\pgfqpoint{3.567431in}{0.646218in}}%
\pgfpathlineto{\pgfqpoint{3.567729in}{0.646202in}}%
\pgfpathlineto{\pgfqpoint{3.568026in}{0.646186in}}%
\pgfpathlineto{\pgfqpoint{3.568324in}{0.646169in}}%
\pgfpathlineto{\pgfqpoint{3.568621in}{0.646153in}}%
\pgfpathlineto{\pgfqpoint{3.568919in}{0.646136in}}%
\pgfpathlineto{\pgfqpoint{3.569216in}{0.646120in}}%
\pgfpathlineto{\pgfqpoint{3.569514in}{0.646103in}}%
\pgfpathlineto{\pgfqpoint{3.569811in}{0.646087in}}%
\pgfpathlineto{\pgfqpoint{3.570108in}{0.646071in}}%
\pgfpathlineto{\pgfqpoint{3.570406in}{0.646054in}}%
\pgfpathlineto{\pgfqpoint{3.570703in}{0.646038in}}%
\pgfpathlineto{\pgfqpoint{3.571001in}{0.646021in}}%
\pgfpathlineto{\pgfqpoint{3.571298in}{0.646005in}}%
\pgfpathlineto{\pgfqpoint{3.571596in}{0.645989in}}%
\pgfpathlineto{\pgfqpoint{3.571893in}{0.645972in}}%
\pgfpathlineto{\pgfqpoint{3.572191in}{0.645956in}}%
\pgfpathlineto{\pgfqpoint{3.572488in}{0.645939in}}%
\pgfpathlineto{\pgfqpoint{3.572786in}{0.645923in}}%
\pgfpathlineto{\pgfqpoint{3.573083in}{0.645906in}}%
\pgfpathlineto{\pgfqpoint{3.573381in}{0.645890in}}%
\pgfpathlineto{\pgfqpoint{3.573678in}{0.645874in}}%
\pgfpathlineto{\pgfqpoint{3.573976in}{0.645857in}}%
\pgfpathlineto{\pgfqpoint{3.574273in}{0.645841in}}%
\pgfpathlineto{\pgfqpoint{3.574571in}{0.645824in}}%
\pgfpathlineto{\pgfqpoint{3.574868in}{0.645808in}}%
\pgfpathlineto{\pgfqpoint{3.575166in}{0.645792in}}%
\pgfpathlineto{\pgfqpoint{3.575463in}{0.645775in}}%
\pgfpathlineto{\pgfqpoint{3.575761in}{0.645759in}}%
\pgfpathlineto{\pgfqpoint{3.576058in}{0.645742in}}%
\pgfpathlineto{\pgfqpoint{3.576356in}{0.645726in}}%
\pgfpathlineto{\pgfqpoint{3.576653in}{0.645709in}}%
\pgfpathlineto{\pgfqpoint{3.576950in}{0.645693in}}%
\pgfpathlineto{\pgfqpoint{3.577248in}{0.645677in}}%
\pgfpathlineto{\pgfqpoint{3.577545in}{0.645660in}}%
\pgfpathlineto{\pgfqpoint{3.577843in}{0.645644in}}%
\pgfpathlineto{\pgfqpoint{3.578140in}{0.645627in}}%
\pgfpathlineto{\pgfqpoint{3.578438in}{0.645611in}}%
\pgfpathlineto{\pgfqpoint{3.578735in}{0.645594in}}%
\pgfpathlineto{\pgfqpoint{3.579033in}{0.645578in}}%
\pgfpathlineto{\pgfqpoint{3.579330in}{0.645562in}}%
\pgfpathlineto{\pgfqpoint{3.579628in}{0.645545in}}%
\pgfpathlineto{\pgfqpoint{3.579925in}{0.645529in}}%
\pgfpathlineto{\pgfqpoint{3.580223in}{0.645512in}}%
\pgfpathlineto{\pgfqpoint{3.580520in}{0.645496in}}%
\pgfpathlineto{\pgfqpoint{3.580818in}{0.645480in}}%
\pgfpathlineto{\pgfqpoint{3.581115in}{0.645463in}}%
\pgfpathlineto{\pgfqpoint{3.581413in}{0.645447in}}%
\pgfpathlineto{\pgfqpoint{3.581710in}{0.645430in}}%
\pgfpathlineto{\pgfqpoint{3.582008in}{0.645414in}}%
\pgfpathlineto{\pgfqpoint{3.582305in}{0.645397in}}%
\pgfpathlineto{\pgfqpoint{3.582603in}{0.645381in}}%
\pgfpathlineto{\pgfqpoint{3.582900in}{0.645365in}}%
\pgfpathlineto{\pgfqpoint{3.583198in}{0.645348in}}%
\pgfpathlineto{\pgfqpoint{3.583495in}{0.645332in}}%
\pgfpathlineto{\pgfqpoint{3.583792in}{0.645316in}}%
\pgfpathlineto{\pgfqpoint{3.584090in}{0.645303in}}%
\pgfpathlineto{\pgfqpoint{3.584387in}{0.645290in}}%
\pgfpathlineto{\pgfqpoint{3.584685in}{0.645277in}}%
\pgfpathlineto{\pgfqpoint{3.584982in}{0.645264in}}%
\pgfpathlineto{\pgfqpoint{3.585280in}{0.645251in}}%
\pgfpathlineto{\pgfqpoint{3.585577in}{0.645238in}}%
\pgfpathlineto{\pgfqpoint{3.585875in}{0.645225in}}%
\pgfpathlineto{\pgfqpoint{3.586172in}{0.645213in}}%
\pgfpathlineto{\pgfqpoint{3.586470in}{0.645200in}}%
\pgfpathlineto{\pgfqpoint{3.586767in}{0.645187in}}%
\pgfpathlineto{\pgfqpoint{3.587065in}{0.645174in}}%
\pgfpathlineto{\pgfqpoint{3.587362in}{0.645161in}}%
\pgfpathlineto{\pgfqpoint{3.587660in}{0.645148in}}%
\pgfpathlineto{\pgfqpoint{3.587957in}{0.645135in}}%
\pgfpathlineto{\pgfqpoint{3.588255in}{0.645122in}}%
\pgfpathlineto{\pgfqpoint{3.588552in}{0.645110in}}%
\pgfpathlineto{\pgfqpoint{3.588850in}{0.645097in}}%
\pgfpathlineto{\pgfqpoint{3.589147in}{0.645084in}}%
\pgfpathlineto{\pgfqpoint{3.589445in}{0.645071in}}%
\pgfpathlineto{\pgfqpoint{3.589742in}{0.645058in}}%
\pgfpathlineto{\pgfqpoint{3.590039in}{0.645045in}}%
\pgfpathlineto{\pgfqpoint{3.590337in}{0.645032in}}%
\pgfpathlineto{\pgfqpoint{3.590634in}{0.645020in}}%
\pgfpathlineto{\pgfqpoint{3.590932in}{0.645007in}}%
\pgfpathlineto{\pgfqpoint{3.591229in}{0.644994in}}%
\pgfpathlineto{\pgfqpoint{3.591527in}{0.645031in}}%
\pgfpathlineto{\pgfqpoint{3.591824in}{0.645122in}}%
\pgfpathlineto{\pgfqpoint{3.592122in}{0.645110in}}%
\pgfpathlineto{\pgfqpoint{3.592419in}{0.645098in}}%
\pgfpathlineto{\pgfqpoint{3.592717in}{0.645086in}}%
\pgfpathlineto{\pgfqpoint{3.593014in}{0.645074in}}%
\pgfpathlineto{\pgfqpoint{3.593312in}{0.645062in}}%
\pgfpathlineto{\pgfqpoint{3.593609in}{0.645050in}}%
\pgfpathlineto{\pgfqpoint{3.593907in}{0.645038in}}%
\pgfpathlineto{\pgfqpoint{3.594204in}{0.645026in}}%
\pgfpathlineto{\pgfqpoint{3.594502in}{0.645015in}}%
\pgfpathlineto{\pgfqpoint{3.594799in}{0.645003in}}%
\pgfpathlineto{\pgfqpoint{3.595097in}{0.644991in}}%
\pgfpathlineto{\pgfqpoint{3.595394in}{0.644979in}}%
\pgfpathlineto{\pgfqpoint{3.595692in}{0.644967in}}%
\pgfpathlineto{\pgfqpoint{3.595989in}{0.644955in}}%
\pgfpathlineto{\pgfqpoint{3.596287in}{0.644943in}}%
\pgfpathlineto{\pgfqpoint{3.596584in}{0.644931in}}%
\pgfpathlineto{\pgfqpoint{3.596881in}{0.644919in}}%
\pgfpathlineto{\pgfqpoint{3.597179in}{0.644907in}}%
\pgfpathlineto{\pgfqpoint{3.597476in}{0.644895in}}%
\pgfpathlineto{\pgfqpoint{3.597774in}{0.644883in}}%
\pgfpathlineto{\pgfqpoint{3.598071in}{0.644871in}}%
\pgfpathlineto{\pgfqpoint{3.598369in}{0.644859in}}%
\pgfpathlineto{\pgfqpoint{3.598666in}{0.644847in}}%
\pgfpathlineto{\pgfqpoint{3.598964in}{0.644835in}}%
\pgfpathlineto{\pgfqpoint{3.599261in}{0.644823in}}%
\pgfpathlineto{\pgfqpoint{3.599559in}{0.644811in}}%
\pgfpathlineto{\pgfqpoint{3.599856in}{0.644799in}}%
\pgfpathlineto{\pgfqpoint{3.600154in}{0.644787in}}%
\pgfpathlineto{\pgfqpoint{3.600451in}{0.644775in}}%
\pgfpathlineto{\pgfqpoint{3.600749in}{0.644763in}}%
\pgfpathlineto{\pgfqpoint{3.601046in}{0.644751in}}%
\pgfpathlineto{\pgfqpoint{3.601344in}{0.644739in}}%
\pgfpathlineto{\pgfqpoint{3.601641in}{0.644727in}}%
\pgfpathlineto{\pgfqpoint{3.601939in}{0.644715in}}%
\pgfpathlineto{\pgfqpoint{3.602236in}{0.644704in}}%
\pgfpathlineto{\pgfqpoint{3.602534in}{0.644692in}}%
\pgfpathlineto{\pgfqpoint{3.602831in}{0.644680in}}%
\pgfpathlineto{\pgfqpoint{3.603129in}{0.644668in}}%
\pgfpathlineto{\pgfqpoint{3.603426in}{0.644656in}}%
\pgfpathlineto{\pgfqpoint{3.603723in}{0.644644in}}%
\pgfpathlineto{\pgfqpoint{3.604021in}{0.644632in}}%
\pgfpathlineto{\pgfqpoint{3.604318in}{0.644620in}}%
\pgfpathlineto{\pgfqpoint{3.604616in}{0.644608in}}%
\pgfpathlineto{\pgfqpoint{3.604913in}{0.644596in}}%
\pgfpathlineto{\pgfqpoint{3.605211in}{0.644584in}}%
\pgfpathlineto{\pgfqpoint{3.605508in}{0.644572in}}%
\pgfpathlineto{\pgfqpoint{3.605806in}{0.644560in}}%
\pgfpathlineto{\pgfqpoint{3.606103in}{0.644547in}}%
\pgfpathlineto{\pgfqpoint{3.606401in}{0.644535in}}%
\pgfpathlineto{\pgfqpoint{3.606698in}{0.644523in}}%
\pgfpathlineto{\pgfqpoint{3.606996in}{0.644511in}}%
\pgfpathlineto{\pgfqpoint{3.607293in}{0.644499in}}%
\pgfpathlineto{\pgfqpoint{3.607591in}{0.644486in}}%
\pgfpathlineto{\pgfqpoint{3.607888in}{0.644474in}}%
\pgfpathlineto{\pgfqpoint{3.608186in}{0.644462in}}%
\pgfpathlineto{\pgfqpoint{3.608483in}{0.644450in}}%
\pgfpathlineto{\pgfqpoint{3.608781in}{0.644438in}}%
\pgfpathlineto{\pgfqpoint{3.609078in}{0.644425in}}%
\pgfpathlineto{\pgfqpoint{3.609376in}{0.644413in}}%
\pgfpathlineto{\pgfqpoint{3.609673in}{0.644401in}}%
\pgfpathlineto{\pgfqpoint{3.609970in}{0.644389in}}%
\pgfpathlineto{\pgfqpoint{3.610268in}{0.644377in}}%
\pgfpathlineto{\pgfqpoint{3.610565in}{0.644364in}}%
\pgfpathlineto{\pgfqpoint{3.610863in}{0.644352in}}%
\pgfpathlineto{\pgfqpoint{3.611160in}{0.644340in}}%
\pgfpathlineto{\pgfqpoint{3.611458in}{0.644328in}}%
\pgfpathlineto{\pgfqpoint{3.611755in}{0.644316in}}%
\pgfpathlineto{\pgfqpoint{3.612053in}{0.644304in}}%
\pgfpathlineto{\pgfqpoint{3.612350in}{0.644291in}}%
\pgfpathlineto{\pgfqpoint{3.612648in}{0.644279in}}%
\pgfpathlineto{\pgfqpoint{3.612945in}{0.644267in}}%
\pgfpathlineto{\pgfqpoint{3.613243in}{0.644255in}}%
\pgfpathlineto{\pgfqpoint{3.613540in}{0.644243in}}%
\pgfpathlineto{\pgfqpoint{3.613838in}{0.644230in}}%
\pgfpathlineto{\pgfqpoint{3.614135in}{0.644218in}}%
\pgfpathlineto{\pgfqpoint{3.614433in}{0.644206in}}%
\pgfpathlineto{\pgfqpoint{3.614730in}{0.644194in}}%
\pgfpathlineto{\pgfqpoint{3.615028in}{0.644182in}}%
\pgfpathlineto{\pgfqpoint{3.615325in}{0.644169in}}%
\pgfpathlineto{\pgfqpoint{3.615623in}{0.644157in}}%
\pgfpathlineto{\pgfqpoint{3.615920in}{0.644145in}}%
\pgfpathlineto{\pgfqpoint{3.616218in}{0.644133in}}%
\pgfpathlineto{\pgfqpoint{3.616515in}{0.644121in}}%
\pgfpathlineto{\pgfqpoint{3.616812in}{0.644108in}}%
\pgfpathlineto{\pgfqpoint{3.617110in}{0.644096in}}%
\pgfpathlineto{\pgfqpoint{3.617407in}{0.644084in}}%
\pgfpathlineto{\pgfqpoint{3.617705in}{0.644072in}}%
\pgfpathlineto{\pgfqpoint{3.618002in}{0.644060in}}%
\pgfpathlineto{\pgfqpoint{3.618300in}{0.644047in}}%
\pgfpathlineto{\pgfqpoint{3.618597in}{0.644035in}}%
\pgfpathlineto{\pgfqpoint{3.618895in}{0.644023in}}%
\pgfpathlineto{\pgfqpoint{3.619192in}{0.644011in}}%
\pgfpathlineto{\pgfqpoint{3.619490in}{0.643999in}}%
\pgfpathlineto{\pgfqpoint{3.619787in}{0.643987in}}%
\pgfpathlineto{\pgfqpoint{3.620085in}{0.643974in}}%
\pgfpathlineto{\pgfqpoint{3.620382in}{0.643962in}}%
\pgfpathlineto{\pgfqpoint{3.620680in}{0.643950in}}%
\pgfpathlineto{\pgfqpoint{3.620977in}{0.643938in}}%
\pgfpathlineto{\pgfqpoint{3.621275in}{0.643926in}}%
\pgfpathlineto{\pgfqpoint{3.621572in}{0.643913in}}%
\pgfpathlineto{\pgfqpoint{3.621870in}{0.643901in}}%
\pgfpathlineto{\pgfqpoint{3.622167in}{0.643889in}}%
\pgfpathlineto{\pgfqpoint{3.622465in}{0.643877in}}%
\pgfpathlineto{\pgfqpoint{3.622762in}{0.643865in}}%
\pgfpathlineto{\pgfqpoint{3.623060in}{0.643852in}}%
\pgfpathlineto{\pgfqpoint{3.623357in}{0.643840in}}%
\pgfpathlineto{\pgfqpoint{3.623654in}{0.643828in}}%
\pgfpathlineto{\pgfqpoint{3.623952in}{0.643816in}}%
\pgfpathlineto{\pgfqpoint{3.624249in}{0.643804in}}%
\pgfpathlineto{\pgfqpoint{3.624547in}{0.643791in}}%
\pgfpathlineto{\pgfqpoint{3.624844in}{0.643779in}}%
\pgfpathlineto{\pgfqpoint{3.625142in}{0.643767in}}%
\pgfpathlineto{\pgfqpoint{3.625439in}{0.643755in}}%
\pgfpathlineto{\pgfqpoint{3.625737in}{0.643743in}}%
\pgfpathlineto{\pgfqpoint{3.626034in}{0.643730in}}%
\pgfpathlineto{\pgfqpoint{3.626332in}{0.643718in}}%
\pgfpathlineto{\pgfqpoint{3.626629in}{0.643706in}}%
\pgfpathlineto{\pgfqpoint{3.626927in}{0.643694in}}%
\pgfpathlineto{\pgfqpoint{3.627224in}{0.643682in}}%
\pgfpathlineto{\pgfqpoint{3.627522in}{0.643669in}}%
\pgfpathlineto{\pgfqpoint{3.627819in}{0.643657in}}%
\pgfpathlineto{\pgfqpoint{3.628117in}{0.643645in}}%
\pgfpathlineto{\pgfqpoint{3.628414in}{0.643633in}}%
\pgfpathlineto{\pgfqpoint{3.628712in}{0.643621in}}%
\pgfpathlineto{\pgfqpoint{3.629009in}{0.643609in}}%
\pgfpathlineto{\pgfqpoint{3.629307in}{0.643596in}}%
\pgfpathlineto{\pgfqpoint{3.629604in}{0.643584in}}%
\pgfpathlineto{\pgfqpoint{3.629901in}{0.643572in}}%
\pgfpathlineto{\pgfqpoint{3.630199in}{0.643560in}}%
\pgfpathlineto{\pgfqpoint{3.630496in}{0.643548in}}%
\pgfpathlineto{\pgfqpoint{3.630794in}{0.643535in}}%
\pgfpathlineto{\pgfqpoint{3.631091in}{0.643523in}}%
\pgfpathlineto{\pgfqpoint{3.631389in}{0.643511in}}%
\pgfpathlineto{\pgfqpoint{3.631686in}{0.643499in}}%
\pgfpathlineto{\pgfqpoint{3.631984in}{0.643487in}}%
\pgfpathlineto{\pgfqpoint{3.632281in}{0.643474in}}%
\pgfpathlineto{\pgfqpoint{3.632579in}{0.643462in}}%
\pgfpathlineto{\pgfqpoint{3.632876in}{0.643450in}}%
\pgfpathlineto{\pgfqpoint{3.633174in}{0.643438in}}%
\pgfpathlineto{\pgfqpoint{3.633471in}{0.643426in}}%
\pgfpathlineto{\pgfqpoint{3.633769in}{0.643413in}}%
\pgfpathlineto{\pgfqpoint{3.634066in}{0.643401in}}%
\pgfpathlineto{\pgfqpoint{3.634364in}{0.643389in}}%
\pgfpathlineto{\pgfqpoint{3.634661in}{0.643377in}}%
\pgfpathlineto{\pgfqpoint{3.634959in}{0.643365in}}%
\pgfpathlineto{\pgfqpoint{3.635256in}{0.643352in}}%
\pgfpathlineto{\pgfqpoint{3.635554in}{0.643340in}}%
\pgfpathlineto{\pgfqpoint{3.635851in}{0.643328in}}%
\pgfpathlineto{\pgfqpoint{3.636149in}{0.643316in}}%
\pgfpathlineto{\pgfqpoint{3.636446in}{0.643304in}}%
\pgfpathlineto{\pgfqpoint{3.636743in}{0.643292in}}%
\pgfpathlineto{\pgfqpoint{3.637041in}{0.643279in}}%
\pgfpathlineto{\pgfqpoint{3.637338in}{0.643267in}}%
\pgfpathlineto{\pgfqpoint{3.637636in}{0.643255in}}%
\pgfpathlineto{\pgfqpoint{3.637933in}{0.643243in}}%
\pgfpathlineto{\pgfqpoint{3.638231in}{0.643231in}}%
\pgfpathlineto{\pgfqpoint{3.638528in}{0.643218in}}%
\pgfpathlineto{\pgfqpoint{3.638826in}{0.643206in}}%
\pgfpathlineto{\pgfqpoint{3.639123in}{0.643194in}}%
\pgfpathlineto{\pgfqpoint{3.639421in}{0.643182in}}%
\pgfpathlineto{\pgfqpoint{3.639718in}{0.643170in}}%
\pgfpathlineto{\pgfqpoint{3.640016in}{0.643157in}}%
\pgfpathlineto{\pgfqpoint{3.640313in}{0.643145in}}%
\pgfpathlineto{\pgfqpoint{3.640611in}{0.643133in}}%
\pgfpathlineto{\pgfqpoint{3.640908in}{0.643121in}}%
\pgfpathlineto{\pgfqpoint{3.641206in}{0.643109in}}%
\pgfpathlineto{\pgfqpoint{3.641503in}{0.643096in}}%
\pgfpathlineto{\pgfqpoint{3.641801in}{0.643084in}}%
\pgfpathlineto{\pgfqpoint{3.642098in}{0.643072in}}%
\pgfpathlineto{\pgfqpoint{3.642396in}{0.643060in}}%
\pgfpathlineto{\pgfqpoint{3.642693in}{0.643048in}}%
\pgfpathlineto{\pgfqpoint{3.642991in}{0.643035in}}%
\pgfpathlineto{\pgfqpoint{3.643288in}{0.643023in}}%
\pgfpathlineto{\pgfqpoint{3.643585in}{0.643011in}}%
\pgfpathlineto{\pgfqpoint{3.643883in}{0.642999in}}%
\pgfpathlineto{\pgfqpoint{3.644180in}{0.642987in}}%
\pgfpathlineto{\pgfqpoint{3.644478in}{0.642974in}}%
\pgfpathlineto{\pgfqpoint{3.644775in}{0.642962in}}%
\pgfpathlineto{\pgfqpoint{3.645073in}{0.642950in}}%
\pgfpathlineto{\pgfqpoint{3.645370in}{0.642938in}}%
\pgfpathlineto{\pgfqpoint{3.645668in}{0.642926in}}%
\pgfpathlineto{\pgfqpoint{3.645965in}{0.642914in}}%
\pgfpathlineto{\pgfqpoint{3.646263in}{0.642901in}}%
\pgfpathlineto{\pgfqpoint{3.646560in}{0.642889in}}%
\pgfpathlineto{\pgfqpoint{3.646858in}{0.642877in}}%
\pgfpathlineto{\pgfqpoint{3.647155in}{0.642865in}}%
\pgfpathlineto{\pgfqpoint{3.647453in}{0.642853in}}%
\pgfpathlineto{\pgfqpoint{3.647750in}{0.642840in}}%
\pgfpathlineto{\pgfqpoint{3.648048in}{0.642828in}}%
\pgfpathlineto{\pgfqpoint{3.648345in}{0.642816in}}%
\pgfpathlineto{\pgfqpoint{3.648643in}{0.642804in}}%
\pgfpathlineto{\pgfqpoint{3.648940in}{0.642792in}}%
\pgfpathlineto{\pgfqpoint{3.649238in}{0.642779in}}%
\pgfpathlineto{\pgfqpoint{3.649535in}{0.642767in}}%
\pgfpathlineto{\pgfqpoint{3.649832in}{0.642755in}}%
\pgfpathlineto{\pgfqpoint{3.650130in}{0.642743in}}%
\pgfpathlineto{\pgfqpoint{3.650427in}{0.642731in}}%
\pgfpathlineto{\pgfqpoint{3.650725in}{0.642718in}}%
\pgfpathlineto{\pgfqpoint{3.651022in}{0.642706in}}%
\pgfpathlineto{\pgfqpoint{3.651320in}{0.642694in}}%
\pgfpathlineto{\pgfqpoint{3.651617in}{0.642682in}}%
\pgfpathlineto{\pgfqpoint{3.651915in}{0.642670in}}%
\pgfpathlineto{\pgfqpoint{3.652212in}{0.642657in}}%
\pgfpathlineto{\pgfqpoint{3.652510in}{0.642645in}}%
\pgfpathlineto{\pgfqpoint{3.652807in}{0.642633in}}%
\pgfpathlineto{\pgfqpoint{3.653105in}{0.642621in}}%
\pgfpathlineto{\pgfqpoint{3.653402in}{0.642609in}}%
\pgfpathlineto{\pgfqpoint{3.653700in}{0.642596in}}%
\pgfpathlineto{\pgfqpoint{3.653997in}{0.642584in}}%
\pgfpathlineto{\pgfqpoint{3.654295in}{0.642572in}}%
\pgfpathlineto{\pgfqpoint{3.654592in}{0.642560in}}%
\pgfpathlineto{\pgfqpoint{3.654890in}{0.642548in}}%
\pgfpathlineto{\pgfqpoint{3.655187in}{0.642536in}}%
\pgfpathlineto{\pgfqpoint{3.655485in}{0.642523in}}%
\pgfpathlineto{\pgfqpoint{3.655782in}{0.642511in}}%
\pgfpathlineto{\pgfqpoint{3.656080in}{0.642499in}}%
\pgfpathlineto{\pgfqpoint{3.656377in}{0.642487in}}%
\pgfpathlineto{\pgfqpoint{3.656674in}{0.642475in}}%
\pgfpathlineto{\pgfqpoint{3.656972in}{0.642462in}}%
\pgfpathlineto{\pgfqpoint{3.657269in}{0.642450in}}%
\pgfpathlineto{\pgfqpoint{3.657567in}{0.642438in}}%
\pgfpathlineto{\pgfqpoint{3.657864in}{0.642426in}}%
\pgfpathlineto{\pgfqpoint{3.658162in}{0.642414in}}%
\pgfpathlineto{\pgfqpoint{3.658459in}{0.642401in}}%
\pgfpathlineto{\pgfqpoint{3.658757in}{0.642389in}}%
\pgfpathlineto{\pgfqpoint{3.659054in}{0.642377in}}%
\pgfpathlineto{\pgfqpoint{3.659352in}{0.642365in}}%
\pgfpathlineto{\pgfqpoint{3.659649in}{0.642353in}}%
\pgfpathlineto{\pgfqpoint{3.659947in}{0.642340in}}%
\pgfpathlineto{\pgfqpoint{3.660244in}{0.642328in}}%
\pgfpathlineto{\pgfqpoint{3.660542in}{0.642316in}}%
\pgfpathlineto{\pgfqpoint{3.660839in}{0.642304in}}%
\pgfpathlineto{\pgfqpoint{3.661137in}{0.642292in}}%
\pgfpathlineto{\pgfqpoint{3.661434in}{0.642279in}}%
\pgfpathlineto{\pgfqpoint{3.661732in}{0.642267in}}%
\pgfpathlineto{\pgfqpoint{3.662029in}{0.642255in}}%
\pgfpathlineto{\pgfqpoint{3.662327in}{0.642243in}}%
\pgfpathlineto{\pgfqpoint{3.662624in}{0.642259in}}%
\pgfpathlineto{\pgfqpoint{3.662922in}{0.642262in}}%
\pgfpathlineto{\pgfqpoint{3.663219in}{0.642260in}}%
\pgfpathlineto{\pgfqpoint{3.663516in}{0.642257in}}%
\pgfpathlineto{\pgfqpoint{3.663814in}{0.642255in}}%
\pgfpathlineto{\pgfqpoint{3.664111in}{0.642252in}}%
\pgfpathlineto{\pgfqpoint{3.664409in}{0.642249in}}%
\pgfpathlineto{\pgfqpoint{3.664706in}{0.642246in}}%
\pgfpathlineto{\pgfqpoint{3.665004in}{0.642243in}}%
\pgfpathlineto{\pgfqpoint{3.665301in}{0.642240in}}%
\pgfpathlineto{\pgfqpoint{3.665599in}{0.642237in}}%
\pgfpathlineto{\pgfqpoint{3.665896in}{0.642234in}}%
\pgfpathlineto{\pgfqpoint{3.666194in}{0.642231in}}%
\pgfpathlineto{\pgfqpoint{3.666491in}{0.642228in}}%
\pgfpathlineto{\pgfqpoint{3.666789in}{0.642225in}}%
\pgfpathlineto{\pgfqpoint{3.667086in}{0.642222in}}%
\pgfpathlineto{\pgfqpoint{3.667384in}{0.642219in}}%
\pgfpathlineto{\pgfqpoint{3.667681in}{0.642216in}}%
\pgfpathlineto{\pgfqpoint{3.667979in}{0.642213in}}%
\pgfpathlineto{\pgfqpoint{3.668276in}{0.642210in}}%
\pgfpathlineto{\pgfqpoint{3.668574in}{0.642207in}}%
\pgfpathlineto{\pgfqpoint{3.668871in}{0.642204in}}%
\pgfpathlineto{\pgfqpoint{3.669169in}{0.642202in}}%
\pgfpathlineto{\pgfqpoint{3.669466in}{0.642199in}}%
\pgfpathlineto{\pgfqpoint{3.669763in}{0.642196in}}%
\pgfpathlineto{\pgfqpoint{3.670061in}{0.642193in}}%
\pgfpathlineto{\pgfqpoint{3.670358in}{0.642190in}}%
\pgfpathlineto{\pgfqpoint{3.670656in}{0.642187in}}%
\pgfpathlineto{\pgfqpoint{3.670953in}{0.642184in}}%
\pgfpathlineto{\pgfqpoint{3.671251in}{0.642181in}}%
\pgfpathlineto{\pgfqpoint{3.671548in}{0.642178in}}%
\pgfpathlineto{\pgfqpoint{3.671846in}{0.642175in}}%
\pgfpathlineto{\pgfqpoint{3.672143in}{0.642172in}}%
\pgfpathlineto{\pgfqpoint{3.672441in}{0.642169in}}%
\pgfpathlineto{\pgfqpoint{3.672738in}{0.642166in}}%
\pgfpathlineto{\pgfqpoint{3.673036in}{0.642163in}}%
\pgfpathlineto{\pgfqpoint{3.673333in}{0.642160in}}%
\pgfpathlineto{\pgfqpoint{3.673631in}{0.642157in}}%
\pgfpathlineto{\pgfqpoint{3.673928in}{0.642154in}}%
\pgfpathlineto{\pgfqpoint{3.674226in}{0.642151in}}%
\pgfpathlineto{\pgfqpoint{3.674523in}{0.642148in}}%
\pgfpathlineto{\pgfqpoint{3.674821in}{0.642146in}}%
\pgfpathlineto{\pgfqpoint{3.675118in}{0.642143in}}%
\pgfpathlineto{\pgfqpoint{3.675416in}{0.642140in}}%
\pgfpathlineto{\pgfqpoint{3.675713in}{0.642137in}}%
\pgfpathlineto{\pgfqpoint{3.676011in}{0.642134in}}%
\pgfpathlineto{\pgfqpoint{3.676308in}{0.642131in}}%
\pgfpathlineto{\pgfqpoint{3.676605in}{0.642128in}}%
\pgfpathlineto{\pgfqpoint{3.676903in}{0.642125in}}%
\pgfpathlineto{\pgfqpoint{3.677200in}{0.642122in}}%
\pgfpathlineto{\pgfqpoint{3.677498in}{0.642119in}}%
\pgfpathlineto{\pgfqpoint{3.677795in}{0.642116in}}%
\pgfpathlineto{\pgfqpoint{3.678093in}{0.642113in}}%
\pgfpathlineto{\pgfqpoint{3.678390in}{0.642110in}}%
\pgfpathlineto{\pgfqpoint{3.678688in}{0.642107in}}%
\pgfpathlineto{\pgfqpoint{3.678985in}{0.642104in}}%
\pgfpathlineto{\pgfqpoint{3.679283in}{0.642101in}}%
\pgfpathlineto{\pgfqpoint{3.679580in}{0.642098in}}%
\pgfpathlineto{\pgfqpoint{3.679878in}{0.642095in}}%
\pgfpathlineto{\pgfqpoint{3.680175in}{0.642093in}}%
\pgfpathlineto{\pgfqpoint{3.680473in}{0.642090in}}%
\pgfpathlineto{\pgfqpoint{3.680770in}{0.642087in}}%
\pgfpathlineto{\pgfqpoint{3.681068in}{0.642084in}}%
\pgfpathlineto{\pgfqpoint{3.681365in}{0.642081in}}%
\pgfpathlineto{\pgfqpoint{3.681663in}{0.642078in}}%
\pgfpathlineto{\pgfqpoint{3.681960in}{0.642075in}}%
\pgfpathlineto{\pgfqpoint{3.682258in}{0.642072in}}%
\pgfpathlineto{\pgfqpoint{3.682555in}{0.642069in}}%
\pgfpathlineto{\pgfqpoint{3.682853in}{0.642066in}}%
\pgfpathlineto{\pgfqpoint{3.683150in}{0.642063in}}%
\pgfpathlineto{\pgfqpoint{3.683447in}{0.642060in}}%
\pgfpathlineto{\pgfqpoint{3.683745in}{0.642057in}}%
\pgfpathlineto{\pgfqpoint{3.684042in}{0.642054in}}%
\pgfpathlineto{\pgfqpoint{3.684340in}{0.642051in}}%
\pgfpathlineto{\pgfqpoint{3.684637in}{0.642048in}}%
\pgfpathlineto{\pgfqpoint{3.684935in}{0.642045in}}%
\pgfpathlineto{\pgfqpoint{3.685232in}{0.642042in}}%
\pgfpathlineto{\pgfqpoint{3.685530in}{0.642039in}}%
\pgfpathlineto{\pgfqpoint{3.685827in}{0.642037in}}%
\pgfpathlineto{\pgfqpoint{3.686125in}{0.642034in}}%
\pgfpathlineto{\pgfqpoint{3.686422in}{0.642031in}}%
\pgfpathlineto{\pgfqpoint{3.686720in}{0.642028in}}%
\pgfpathlineto{\pgfqpoint{3.687017in}{0.642025in}}%
\pgfpathlineto{\pgfqpoint{3.687315in}{0.642022in}}%
\pgfpathlineto{\pgfqpoint{3.687612in}{0.642019in}}%
\pgfpathlineto{\pgfqpoint{3.687910in}{0.642016in}}%
\pgfpathlineto{\pgfqpoint{3.688207in}{0.642013in}}%
\pgfpathlineto{\pgfqpoint{3.688505in}{0.642010in}}%
\pgfpathlineto{\pgfqpoint{3.688802in}{0.642007in}}%
\pgfpathlineto{\pgfqpoint{3.689100in}{0.642004in}}%
\pgfpathlineto{\pgfqpoint{3.689397in}{0.642001in}}%
\pgfpathlineto{\pgfqpoint{3.689694in}{0.641998in}}%
\pgfpathlineto{\pgfqpoint{3.689992in}{0.641995in}}%
\pgfpathlineto{\pgfqpoint{3.690289in}{0.641992in}}%
\pgfpathlineto{\pgfqpoint{3.690587in}{0.641989in}}%
\pgfpathlineto{\pgfqpoint{3.690884in}{0.641052in}}%
\pgfpathlineto{\pgfqpoint{3.691182in}{0.636205in}}%
\pgfpathlineto{\pgfqpoint{3.691479in}{0.636153in}}%
\pgfpathlineto{\pgfqpoint{3.691777in}{0.636100in}}%
\pgfpathlineto{\pgfqpoint{3.692074in}{0.636048in}}%
\pgfpathlineto{\pgfqpoint{3.692372in}{0.635996in}}%
\pgfpathlineto{\pgfqpoint{3.692669in}{0.635943in}}%
\pgfpathlineto{\pgfqpoint{3.692967in}{0.635891in}}%
\pgfpathlineto{\pgfqpoint{3.693264in}{0.635838in}}%
\pgfpathlineto{\pgfqpoint{3.693562in}{0.635786in}}%
\pgfpathlineto{\pgfqpoint{3.693859in}{0.635734in}}%
\pgfpathlineto{\pgfqpoint{3.694157in}{0.635681in}}%
\pgfpathlineto{\pgfqpoint{3.694454in}{0.635629in}}%
\pgfpathlineto{\pgfqpoint{3.694752in}{0.635577in}}%
\pgfpathlineto{\pgfqpoint{3.695049in}{0.635524in}}%
\pgfpathlineto{\pgfqpoint{3.695347in}{0.635472in}}%
\pgfpathlineto{\pgfqpoint{3.695644in}{0.635419in}}%
\pgfpathlineto{\pgfqpoint{3.695942in}{0.635367in}}%
\pgfpathlineto{\pgfqpoint{3.696239in}{0.635315in}}%
\pgfpathlineto{\pgfqpoint{3.696536in}{0.635262in}}%
\pgfpathlineto{\pgfqpoint{3.696834in}{0.635210in}}%
\pgfpathlineto{\pgfqpoint{3.697131in}{0.635157in}}%
\pgfpathlineto{\pgfqpoint{3.697429in}{0.635105in}}%
\pgfpathlineto{\pgfqpoint{3.697726in}{0.635053in}}%
\pgfpathlineto{\pgfqpoint{3.698024in}{0.635000in}}%
\pgfpathlineto{\pgfqpoint{3.698321in}{0.634948in}}%
\pgfpathlineto{\pgfqpoint{3.698619in}{0.634896in}}%
\pgfpathlineto{\pgfqpoint{3.698916in}{0.634843in}}%
\pgfpathlineto{\pgfqpoint{3.699214in}{0.634791in}}%
\pgfpathlineto{\pgfqpoint{3.699511in}{0.634738in}}%
\pgfpathlineto{\pgfqpoint{3.699809in}{0.634686in}}%
\pgfpathlineto{\pgfqpoint{3.700106in}{0.634634in}}%
\pgfpathlineto{\pgfqpoint{3.700404in}{0.634581in}}%
\pgfpathlineto{\pgfqpoint{3.700701in}{0.634529in}}%
\pgfpathlineto{\pgfqpoint{3.700999in}{0.634476in}}%
\pgfpathlineto{\pgfqpoint{3.701296in}{0.634424in}}%
\pgfpathlineto{\pgfqpoint{3.701594in}{0.634372in}}%
\pgfpathlineto{\pgfqpoint{3.701891in}{0.634319in}}%
\pgfpathlineto{\pgfqpoint{3.702189in}{0.634267in}}%
\pgfpathlineto{\pgfqpoint{3.702486in}{0.634214in}}%
\pgfpathlineto{\pgfqpoint{3.702784in}{0.634162in}}%
\pgfpathlineto{\pgfqpoint{3.703081in}{0.634110in}}%
\pgfpathlineto{\pgfqpoint{3.703378in}{0.634057in}}%
\pgfpathlineto{\pgfqpoint{3.703676in}{0.634005in}}%
\pgfpathlineto{\pgfqpoint{3.703973in}{0.633953in}}%
\pgfpathlineto{\pgfqpoint{3.704271in}{0.633900in}}%
\pgfpathlineto{\pgfqpoint{3.704568in}{0.633848in}}%
\pgfpathlineto{\pgfqpoint{3.704866in}{0.632864in}}%
\pgfpathlineto{\pgfqpoint{3.705163in}{0.630897in}}%
\pgfpathlineto{\pgfqpoint{3.705461in}{0.630876in}}%
\pgfpathlineto{\pgfqpoint{3.705758in}{0.630855in}}%
\pgfpathlineto{\pgfqpoint{3.706056in}{0.630834in}}%
\pgfpathlineto{\pgfqpoint{3.706353in}{0.630813in}}%
\pgfpathlineto{\pgfqpoint{3.706651in}{0.630792in}}%
\pgfpathlineto{\pgfqpoint{3.706948in}{0.630771in}}%
\pgfpathlineto{\pgfqpoint{3.707246in}{0.630750in}}%
\pgfpathlineto{\pgfqpoint{3.707543in}{0.630729in}}%
\pgfpathlineto{\pgfqpoint{3.707841in}{0.630707in}}%
\pgfpathlineto{\pgfqpoint{3.708138in}{0.630686in}}%
\pgfpathlineto{\pgfqpoint{3.708436in}{0.630665in}}%
\pgfpathlineto{\pgfqpoint{3.708733in}{0.630644in}}%
\pgfpathlineto{\pgfqpoint{3.709031in}{0.630623in}}%
\pgfpathlineto{\pgfqpoint{3.709328in}{0.630602in}}%
\pgfpathlineto{\pgfqpoint{3.709625in}{0.630581in}}%
\pgfpathlineto{\pgfqpoint{3.709923in}{0.630560in}}%
\pgfpathlineto{\pgfqpoint{3.710220in}{0.630539in}}%
\pgfpathlineto{\pgfqpoint{3.710518in}{0.630518in}}%
\pgfpathlineto{\pgfqpoint{3.710815in}{0.630497in}}%
\pgfpathlineto{\pgfqpoint{3.711113in}{0.630476in}}%
\pgfpathlineto{\pgfqpoint{3.711410in}{0.630455in}}%
\pgfpathlineto{\pgfqpoint{3.711708in}{0.630434in}}%
\pgfpathlineto{\pgfqpoint{3.712005in}{0.630413in}}%
\pgfpathlineto{\pgfqpoint{3.712303in}{0.630392in}}%
\pgfpathlineto{\pgfqpoint{3.712600in}{0.630371in}}%
\pgfpathlineto{\pgfqpoint{3.712898in}{0.630350in}}%
\pgfpathlineto{\pgfqpoint{3.713195in}{0.630329in}}%
\pgfpathlineto{\pgfqpoint{3.713493in}{0.630308in}}%
\pgfpathlineto{\pgfqpoint{3.713790in}{0.630287in}}%
\pgfpathlineto{\pgfqpoint{3.714088in}{0.630266in}}%
\pgfpathlineto{\pgfqpoint{3.714385in}{0.630245in}}%
\pgfpathlineto{\pgfqpoint{3.714683in}{0.630224in}}%
\pgfpathlineto{\pgfqpoint{3.714980in}{0.630203in}}%
\pgfpathlineto{\pgfqpoint{3.715278in}{0.630182in}}%
\pgfpathlineto{\pgfqpoint{3.715575in}{0.630161in}}%
\pgfpathlineto{\pgfqpoint{3.715873in}{0.630140in}}%
\pgfpathlineto{\pgfqpoint{3.716170in}{0.630119in}}%
\pgfpathlineto{\pgfqpoint{3.716467in}{0.630098in}}%
\pgfpathlineto{\pgfqpoint{3.716765in}{0.630076in}}%
\pgfpathlineto{\pgfqpoint{3.717062in}{0.630055in}}%
\pgfpathlineto{\pgfqpoint{3.717360in}{0.630034in}}%
\pgfpathlineto{\pgfqpoint{3.717657in}{0.630013in}}%
\pgfpathlineto{\pgfqpoint{3.717955in}{0.629992in}}%
\pgfpathlineto{\pgfqpoint{3.718252in}{0.629971in}}%
\pgfpathlineto{\pgfqpoint{3.718550in}{0.629950in}}%
\pgfpathlineto{\pgfqpoint{3.718847in}{0.629929in}}%
\pgfpathlineto{\pgfqpoint{3.719145in}{0.629908in}}%
\pgfpathlineto{\pgfqpoint{3.719442in}{0.629887in}}%
\pgfpathlineto{\pgfqpoint{3.719740in}{0.629866in}}%
\pgfpathlineto{\pgfqpoint{3.720037in}{0.629845in}}%
\pgfpathlineto{\pgfqpoint{3.720335in}{0.629824in}}%
\pgfpathlineto{\pgfqpoint{3.720632in}{0.629803in}}%
\pgfpathlineto{\pgfqpoint{3.720930in}{0.629782in}}%
\pgfpathlineto{\pgfqpoint{3.721227in}{0.629761in}}%
\pgfpathlineto{\pgfqpoint{3.721525in}{0.629740in}}%
\pgfpathlineto{\pgfqpoint{3.721822in}{0.629719in}}%
\pgfpathlineto{\pgfqpoint{3.722120in}{0.629698in}}%
\pgfpathlineto{\pgfqpoint{3.722417in}{0.629677in}}%
\pgfpathlineto{\pgfqpoint{3.722715in}{0.629656in}}%
\pgfpathlineto{\pgfqpoint{3.723012in}{0.629635in}}%
\pgfpathlineto{\pgfqpoint{3.723309in}{0.629614in}}%
\pgfpathlineto{\pgfqpoint{3.723607in}{0.629593in}}%
\pgfpathlineto{\pgfqpoint{3.723904in}{0.629572in}}%
\pgfpathlineto{\pgfqpoint{3.724202in}{0.629551in}}%
\pgfpathlineto{\pgfqpoint{3.724499in}{0.629530in}}%
\pgfpathlineto{\pgfqpoint{3.724797in}{0.629509in}}%
\pgfpathlineto{\pgfqpoint{3.725094in}{0.629488in}}%
\pgfpathlineto{\pgfqpoint{3.725392in}{0.629467in}}%
\pgfpathlineto{\pgfqpoint{3.725689in}{0.629446in}}%
\pgfpathlineto{\pgfqpoint{3.725987in}{0.629424in}}%
\pgfpathlineto{\pgfqpoint{3.726284in}{0.629403in}}%
\pgfpathlineto{\pgfqpoint{3.726582in}{0.629382in}}%
\pgfpathlineto{\pgfqpoint{3.726879in}{0.629361in}}%
\pgfpathlineto{\pgfqpoint{3.727177in}{0.629340in}}%
\pgfpathlineto{\pgfqpoint{3.727474in}{0.629319in}}%
\pgfpathlineto{\pgfqpoint{3.727772in}{0.629298in}}%
\pgfpathlineto{\pgfqpoint{3.728069in}{0.629277in}}%
\pgfpathlineto{\pgfqpoint{3.728367in}{0.629256in}}%
\pgfpathlineto{\pgfqpoint{3.728664in}{0.629235in}}%
\pgfpathlineto{\pgfqpoint{3.728962in}{0.629214in}}%
\pgfpathlineto{\pgfqpoint{3.729259in}{0.629193in}}%
\pgfpathlineto{\pgfqpoint{3.729556in}{0.629172in}}%
\pgfpathlineto{\pgfqpoint{3.729854in}{0.629151in}}%
\pgfpathlineto{\pgfqpoint{3.730151in}{0.629130in}}%
\pgfpathlineto{\pgfqpoint{3.730449in}{0.629109in}}%
\pgfpathlineto{\pgfqpoint{3.730746in}{0.629088in}}%
\pgfpathlineto{\pgfqpoint{3.731044in}{0.629067in}}%
\pgfpathlineto{\pgfqpoint{3.731341in}{0.629046in}}%
\pgfpathlineto{\pgfqpoint{3.731639in}{0.629025in}}%
\pgfpathlineto{\pgfqpoint{3.731936in}{0.629004in}}%
\pgfpathlineto{\pgfqpoint{3.732234in}{0.628983in}}%
\pgfpathlineto{\pgfqpoint{3.732531in}{0.628962in}}%
\pgfpathlineto{\pgfqpoint{3.732829in}{0.628941in}}%
\pgfpathlineto{\pgfqpoint{3.733126in}{0.628920in}}%
\pgfpathlineto{\pgfqpoint{3.733424in}{0.628886in}}%
\pgfpathlineto{\pgfqpoint{3.733721in}{0.628810in}}%
\pgfpathlineto{\pgfqpoint{3.734019in}{0.628732in}}%
\pgfpathlineto{\pgfqpoint{3.734316in}{0.628653in}}%
\pgfpathlineto{\pgfqpoint{3.734614in}{0.628575in}}%
\pgfpathlineto{\pgfqpoint{3.734911in}{0.628496in}}%
\pgfpathlineto{\pgfqpoint{3.735209in}{0.628418in}}%
\pgfpathlineto{\pgfqpoint{3.735506in}{0.628339in}}%
\pgfpathlineto{\pgfqpoint{3.735804in}{0.628261in}}%
\pgfpathlineto{\pgfqpoint{3.736101in}{0.628183in}}%
\pgfpathlineto{\pgfqpoint{3.736398in}{0.628104in}}%
\pgfpathlineto{\pgfqpoint{3.736696in}{0.628026in}}%
\pgfpathlineto{\pgfqpoint{3.736993in}{0.627947in}}%
\pgfpathlineto{\pgfqpoint{3.737291in}{0.627869in}}%
\pgfpathlineto{\pgfqpoint{3.737588in}{0.627790in}}%
\pgfpathlineto{\pgfqpoint{3.737886in}{0.627712in}}%
\pgfpathlineto{\pgfqpoint{3.738183in}{0.627633in}}%
\pgfpathlineto{\pgfqpoint{3.738481in}{0.627555in}}%
\pgfpathlineto{\pgfqpoint{3.738778in}{0.627477in}}%
\pgfpathlineto{\pgfqpoint{3.739076in}{0.627398in}}%
\pgfpathlineto{\pgfqpoint{3.739373in}{0.627320in}}%
\pgfpathlineto{\pgfqpoint{3.739671in}{0.627241in}}%
\pgfpathlineto{\pgfqpoint{3.739968in}{0.627163in}}%
\pgfpathlineto{\pgfqpoint{3.740266in}{0.627084in}}%
\pgfpathlineto{\pgfqpoint{3.740563in}{0.627006in}}%
\pgfpathlineto{\pgfqpoint{3.740861in}{0.626927in}}%
\pgfpathlineto{\pgfqpoint{3.741158in}{0.626849in}}%
\pgfpathlineto{\pgfqpoint{3.741456in}{0.626771in}}%
\pgfpathlineto{\pgfqpoint{3.741753in}{0.626692in}}%
\pgfpathlineto{\pgfqpoint{3.742051in}{0.626614in}}%
\pgfpathlineto{\pgfqpoint{3.742348in}{0.626535in}}%
\pgfpathlineto{\pgfqpoint{3.742646in}{0.626457in}}%
\pgfpathlineto{\pgfqpoint{3.742943in}{0.626378in}}%
\pgfpathlineto{\pgfqpoint{3.743240in}{0.626300in}}%
\pgfpathlineto{\pgfqpoint{3.743538in}{0.626221in}}%
\pgfpathlineto{\pgfqpoint{3.743835in}{0.626143in}}%
\pgfpathlineto{\pgfqpoint{3.744133in}{0.626065in}}%
\pgfpathlineto{\pgfqpoint{3.744430in}{0.625986in}}%
\pgfpathlineto{\pgfqpoint{3.744728in}{0.625908in}}%
\pgfpathlineto{\pgfqpoint{3.745025in}{0.625829in}}%
\pgfpathlineto{\pgfqpoint{3.745323in}{0.625751in}}%
\pgfpathlineto{\pgfqpoint{3.745620in}{0.625672in}}%
\pgfpathlineto{\pgfqpoint{3.745918in}{0.625594in}}%
\pgfpathlineto{\pgfqpoint{3.746215in}{0.625515in}}%
\pgfpathlineto{\pgfqpoint{3.746513in}{0.625437in}}%
\pgfpathlineto{\pgfqpoint{3.746810in}{0.625358in}}%
\pgfpathlineto{\pgfqpoint{3.747108in}{0.625280in}}%
\pgfpathlineto{\pgfqpoint{3.747405in}{0.625202in}}%
\pgfpathlineto{\pgfqpoint{3.747703in}{0.625123in}}%
\pgfpathlineto{\pgfqpoint{3.748000in}{0.623399in}}%
\pgfpathlineto{\pgfqpoint{3.748298in}{0.620729in}}%
\pgfpathlineto{\pgfqpoint{3.748595in}{0.620732in}}%
\pgfpathlineto{\pgfqpoint{3.748893in}{0.620736in}}%
\pgfpathlineto{\pgfqpoint{3.749190in}{0.620739in}}%
\pgfpathlineto{\pgfqpoint{3.749488in}{0.620742in}}%
\pgfpathlineto{\pgfqpoint{3.749785in}{0.620745in}}%
\pgfpathlineto{\pgfqpoint{3.750082in}{0.620749in}}%
\pgfpathlineto{\pgfqpoint{3.750380in}{0.620752in}}%
\pgfpathlineto{\pgfqpoint{3.750677in}{0.620755in}}%
\pgfpathlineto{\pgfqpoint{3.750975in}{0.620759in}}%
\pgfpathlineto{\pgfqpoint{3.751272in}{0.620762in}}%
\pgfpathlineto{\pgfqpoint{3.751570in}{0.620765in}}%
\pgfpathlineto{\pgfqpoint{3.751867in}{0.620769in}}%
\pgfpathlineto{\pgfqpoint{3.752165in}{0.620772in}}%
\pgfpathlineto{\pgfqpoint{3.752462in}{0.620775in}}%
\pgfpathlineto{\pgfqpoint{3.752760in}{0.620779in}}%
\pgfpathlineto{\pgfqpoint{3.753057in}{0.620782in}}%
\pgfpathlineto{\pgfqpoint{3.753355in}{0.620785in}}%
\pgfpathlineto{\pgfqpoint{3.753652in}{0.620789in}}%
\pgfpathlineto{\pgfqpoint{3.753950in}{0.620792in}}%
\pgfpathlineto{\pgfqpoint{3.754247in}{0.620795in}}%
\pgfpathlineto{\pgfqpoint{3.754545in}{0.620799in}}%
\pgfpathlineto{\pgfqpoint{3.754842in}{0.620802in}}%
\pgfpathlineto{\pgfqpoint{3.755140in}{0.620805in}}%
\pgfpathlineto{\pgfqpoint{3.755437in}{0.620809in}}%
\pgfpathlineto{\pgfqpoint{3.755735in}{0.620812in}}%
\pgfpathlineto{\pgfqpoint{3.756032in}{0.620815in}}%
\pgfpathlineto{\pgfqpoint{3.756329in}{0.620819in}}%
\pgfpathlineto{\pgfqpoint{3.756627in}{0.620822in}}%
\pgfpathlineto{\pgfqpoint{3.756924in}{0.620825in}}%
\pgfpathlineto{\pgfqpoint{3.757222in}{0.620828in}}%
\pgfpathlineto{\pgfqpoint{3.757519in}{0.620832in}}%
\pgfpathlineto{\pgfqpoint{3.757817in}{0.620835in}}%
\pgfpathlineto{\pgfqpoint{3.758114in}{0.620838in}}%
\pgfpathlineto{\pgfqpoint{3.758412in}{0.620842in}}%
\pgfpathlineto{\pgfqpoint{3.758709in}{0.620845in}}%
\pgfpathlineto{\pgfqpoint{3.759007in}{0.620848in}}%
\pgfpathlineto{\pgfqpoint{3.759304in}{0.620852in}}%
\pgfpathlineto{\pgfqpoint{3.759602in}{0.620855in}}%
\pgfpathlineto{\pgfqpoint{3.759899in}{0.620858in}}%
\pgfpathlineto{\pgfqpoint{3.760197in}{0.620862in}}%
\pgfpathlineto{\pgfqpoint{3.760494in}{0.620865in}}%
\pgfpathlineto{\pgfqpoint{3.760792in}{0.620868in}}%
\pgfpathlineto{\pgfqpoint{3.761089in}{0.620872in}}%
\pgfpathlineto{\pgfqpoint{3.761387in}{0.620875in}}%
\pgfpathlineto{\pgfqpoint{3.761684in}{0.620878in}}%
\pgfpathlineto{\pgfqpoint{3.761982in}{0.620882in}}%
\pgfpathlineto{\pgfqpoint{3.762279in}{0.620885in}}%
\pgfpathlineto{\pgfqpoint{3.762577in}{0.620888in}}%
\pgfpathlineto{\pgfqpoint{3.762874in}{0.620892in}}%
\pgfpathlineto{\pgfqpoint{3.763171in}{0.620895in}}%
\pgfpathlineto{\pgfqpoint{3.763469in}{0.620898in}}%
\pgfpathlineto{\pgfqpoint{3.763766in}{0.620902in}}%
\pgfpathlineto{\pgfqpoint{3.764064in}{0.620905in}}%
\pgfpathlineto{\pgfqpoint{3.764361in}{0.620908in}}%
\pgfpathlineto{\pgfqpoint{3.764659in}{0.620912in}}%
\pgfpathlineto{\pgfqpoint{3.764956in}{0.620915in}}%
\pgfpathlineto{\pgfqpoint{3.765254in}{0.620918in}}%
\pgfpathlineto{\pgfqpoint{3.765551in}{0.620921in}}%
\pgfpathlineto{\pgfqpoint{3.765849in}{0.620925in}}%
\pgfpathlineto{\pgfqpoint{3.766146in}{0.620928in}}%
\pgfpathlineto{\pgfqpoint{3.766444in}{0.620931in}}%
\pgfpathlineto{\pgfqpoint{3.766741in}{0.620935in}}%
\pgfpathlineto{\pgfqpoint{3.767039in}{0.620938in}}%
\pgfpathlineto{\pgfqpoint{3.767336in}{0.620941in}}%
\pgfpathlineto{\pgfqpoint{3.767634in}{0.620945in}}%
\pgfpathlineto{\pgfqpoint{3.767931in}{0.620948in}}%
\pgfpathlineto{\pgfqpoint{3.768229in}{0.620951in}}%
\pgfpathlineto{\pgfqpoint{3.768526in}{0.620955in}}%
\pgfpathlineto{\pgfqpoint{3.768824in}{0.620958in}}%
\pgfpathlineto{\pgfqpoint{3.769121in}{0.620961in}}%
\pgfpathlineto{\pgfqpoint{3.769419in}{0.620965in}}%
\pgfpathlineto{\pgfqpoint{3.769716in}{0.620968in}}%
\pgfpathlineto{\pgfqpoint{3.770013in}{0.620971in}}%
\pgfpathlineto{\pgfqpoint{3.770311in}{0.620975in}}%
\pgfpathlineto{\pgfqpoint{3.770608in}{0.620978in}}%
\pgfpathlineto{\pgfqpoint{3.770906in}{0.620981in}}%
\pgfpathlineto{\pgfqpoint{3.771203in}{0.620985in}}%
\pgfpathlineto{\pgfqpoint{3.771501in}{0.620988in}}%
\pgfpathlineto{\pgfqpoint{3.771798in}{0.620991in}}%
\pgfpathlineto{\pgfqpoint{3.772096in}{0.620995in}}%
\pgfpathlineto{\pgfqpoint{3.772393in}{0.620998in}}%
\pgfpathlineto{\pgfqpoint{3.772691in}{0.621001in}}%
\pgfpathlineto{\pgfqpoint{3.772988in}{0.621004in}}%
\pgfpathlineto{\pgfqpoint{3.773286in}{0.621008in}}%
\pgfpathlineto{\pgfqpoint{3.773583in}{0.621011in}}%
\pgfpathlineto{\pgfqpoint{3.773881in}{0.621014in}}%
\pgfpathlineto{\pgfqpoint{3.774178in}{0.621018in}}%
\pgfpathlineto{\pgfqpoint{3.774476in}{0.621021in}}%
\pgfpathlineto{\pgfqpoint{3.774773in}{0.621024in}}%
\pgfpathlineto{\pgfqpoint{3.775071in}{0.621028in}}%
\pgfpathlineto{\pgfqpoint{3.775368in}{0.621031in}}%
\pgfpathlineto{\pgfqpoint{3.775666in}{0.621034in}}%
\pgfpathlineto{\pgfqpoint{3.775963in}{0.621038in}}%
\pgfpathlineto{\pgfqpoint{3.776260in}{0.621041in}}%
\pgfpathlineto{\pgfqpoint{3.776558in}{0.621044in}}%
\pgfpathlineto{\pgfqpoint{3.776855in}{0.621048in}}%
\pgfpathlineto{\pgfqpoint{3.777153in}{0.621051in}}%
\pgfpathlineto{\pgfqpoint{3.777450in}{0.621054in}}%
\pgfpathlineto{\pgfqpoint{3.777748in}{0.621058in}}%
\pgfpathlineto{\pgfqpoint{3.778045in}{0.621061in}}%
\pgfpathlineto{\pgfqpoint{3.778343in}{0.621064in}}%
\pgfpathlineto{\pgfqpoint{3.778640in}{0.621068in}}%
\pgfpathlineto{\pgfqpoint{3.778938in}{0.621071in}}%
\pgfpathlineto{\pgfqpoint{3.779235in}{0.621074in}}%
\pgfpathlineto{\pgfqpoint{3.779533in}{0.621078in}}%
\pgfpathlineto{\pgfqpoint{3.779830in}{0.621081in}}%
\pgfpathlineto{\pgfqpoint{3.780128in}{0.621084in}}%
\pgfpathlineto{\pgfqpoint{3.780425in}{0.621088in}}%
\pgfpathlineto{\pgfqpoint{3.780723in}{0.621091in}}%
\pgfpathlineto{\pgfqpoint{3.781020in}{0.621094in}}%
\pgfpathlineto{\pgfqpoint{3.781318in}{0.621097in}}%
\pgfpathlineto{\pgfqpoint{3.781615in}{0.621101in}}%
\pgfpathlineto{\pgfqpoint{3.781913in}{0.621104in}}%
\pgfpathlineto{\pgfqpoint{3.782210in}{0.621107in}}%
\pgfpathlineto{\pgfqpoint{3.782508in}{0.621111in}}%
\pgfpathlineto{\pgfqpoint{3.782805in}{0.621114in}}%
\pgfpathlineto{\pgfqpoint{3.783102in}{0.621117in}}%
\pgfpathlineto{\pgfqpoint{3.783400in}{0.621121in}}%
\pgfpathlineto{\pgfqpoint{3.783697in}{0.621124in}}%
\pgfpathlineto{\pgfqpoint{3.783995in}{0.621127in}}%
\pgfpathlineto{\pgfqpoint{3.784292in}{0.621131in}}%
\pgfpathlineto{\pgfqpoint{3.784590in}{0.621134in}}%
\pgfpathlineto{\pgfqpoint{3.784887in}{0.621137in}}%
\pgfpathlineto{\pgfqpoint{3.785185in}{0.621141in}}%
\pgfpathlineto{\pgfqpoint{3.785482in}{0.621144in}}%
\pgfpathlineto{\pgfqpoint{3.785780in}{0.621147in}}%
\pgfpathlineto{\pgfqpoint{3.786077in}{0.621151in}}%
\pgfpathlineto{\pgfqpoint{3.786375in}{0.621154in}}%
\pgfpathlineto{\pgfqpoint{3.786672in}{0.621157in}}%
\pgfpathlineto{\pgfqpoint{3.786970in}{0.621161in}}%
\pgfpathlineto{\pgfqpoint{3.787267in}{0.621164in}}%
\pgfpathlineto{\pgfqpoint{3.787565in}{0.621167in}}%
\pgfpathlineto{\pgfqpoint{3.787862in}{0.621171in}}%
\pgfpathlineto{\pgfqpoint{3.788160in}{0.621174in}}%
\pgfpathlineto{\pgfqpoint{3.788457in}{0.621177in}}%
\pgfpathlineto{\pgfqpoint{3.788755in}{0.621181in}}%
\pgfpathlineto{\pgfqpoint{3.789052in}{0.621184in}}%
\pgfpathlineto{\pgfqpoint{3.789350in}{0.621187in}}%
\pgfpathlineto{\pgfqpoint{3.789647in}{0.621190in}}%
\pgfpathlineto{\pgfqpoint{3.789944in}{0.621194in}}%
\pgfpathlineto{\pgfqpoint{3.790242in}{0.621197in}}%
\pgfpathlineto{\pgfqpoint{3.790539in}{0.621200in}}%
\pgfpathlineto{\pgfqpoint{3.790837in}{0.621204in}}%
\pgfpathlineto{\pgfqpoint{3.791134in}{0.621207in}}%
\pgfpathlineto{\pgfqpoint{3.791432in}{0.621210in}}%
\pgfpathlineto{\pgfqpoint{3.791729in}{0.621214in}}%
\pgfpathlineto{\pgfqpoint{3.792027in}{0.621217in}}%
\pgfpathlineto{\pgfqpoint{3.792324in}{0.621220in}}%
\pgfpathlineto{\pgfqpoint{3.792622in}{0.621224in}}%
\pgfpathlineto{\pgfqpoint{3.792919in}{0.621227in}}%
\pgfpathlineto{\pgfqpoint{3.793217in}{0.621230in}}%
\pgfpathlineto{\pgfqpoint{3.793514in}{0.621234in}}%
\pgfpathlineto{\pgfqpoint{3.793812in}{0.621237in}}%
\pgfpathlineto{\pgfqpoint{3.794109in}{0.621240in}}%
\pgfpathlineto{\pgfqpoint{3.794407in}{0.621244in}}%
\pgfpathlineto{\pgfqpoint{3.794704in}{0.621247in}}%
\pgfpathlineto{\pgfqpoint{3.795002in}{0.621250in}}%
\pgfpathlineto{\pgfqpoint{3.795299in}{0.621254in}}%
\pgfpathlineto{\pgfqpoint{3.795597in}{0.621257in}}%
\pgfpathlineto{\pgfqpoint{3.795894in}{0.621260in}}%
\pgfpathlineto{\pgfqpoint{3.796191in}{0.621264in}}%
\pgfpathlineto{\pgfqpoint{3.796489in}{0.621267in}}%
\pgfpathlineto{\pgfqpoint{3.796786in}{0.621270in}}%
\pgfpathlineto{\pgfqpoint{3.797084in}{0.621273in}}%
\pgfpathlineto{\pgfqpoint{3.797381in}{0.621277in}}%
\pgfpathlineto{\pgfqpoint{3.797679in}{0.621280in}}%
\pgfpathlineto{\pgfqpoint{3.797976in}{0.621283in}}%
\pgfpathlineto{\pgfqpoint{3.798274in}{0.621287in}}%
\pgfpathlineto{\pgfqpoint{3.798571in}{0.621290in}}%
\pgfpathlineto{\pgfqpoint{3.798869in}{0.621293in}}%
\pgfpathlineto{\pgfqpoint{3.799166in}{0.621297in}}%
\pgfpathlineto{\pgfqpoint{3.799464in}{0.621300in}}%
\pgfpathlineto{\pgfqpoint{3.799761in}{0.621303in}}%
\pgfpathlineto{\pgfqpoint{3.800059in}{0.621307in}}%
\pgfpathlineto{\pgfqpoint{3.800356in}{0.621310in}}%
\pgfpathlineto{\pgfqpoint{3.800654in}{0.621313in}}%
\pgfpathlineto{\pgfqpoint{3.800951in}{0.621317in}}%
\pgfpathlineto{\pgfqpoint{3.801249in}{0.621320in}}%
\pgfpathlineto{\pgfqpoint{3.801546in}{0.621323in}}%
\pgfpathlineto{\pgfqpoint{3.801844in}{0.621327in}}%
\pgfpathlineto{\pgfqpoint{3.802141in}{0.621330in}}%
\pgfpathlineto{\pgfqpoint{3.802439in}{0.621333in}}%
\pgfpathlineto{\pgfqpoint{3.802736in}{0.621337in}}%
\pgfpathlineto{\pgfqpoint{3.803033in}{0.621340in}}%
\pgfpathlineto{\pgfqpoint{3.803331in}{0.621343in}}%
\pgfpathlineto{\pgfqpoint{3.803628in}{0.621347in}}%
\pgfpathlineto{\pgfqpoint{3.803926in}{0.621350in}}%
\pgfpathlineto{\pgfqpoint{3.804223in}{0.621353in}}%
\pgfpathlineto{\pgfqpoint{3.804521in}{0.621357in}}%
\pgfpathlineto{\pgfqpoint{3.804818in}{0.621360in}}%
\pgfpathlineto{\pgfqpoint{3.805116in}{0.621363in}}%
\pgfpathlineto{\pgfqpoint{3.805413in}{0.621366in}}%
\pgfpathlineto{\pgfqpoint{3.805711in}{0.621370in}}%
\pgfpathlineto{\pgfqpoint{3.806008in}{0.621373in}}%
\pgfpathlineto{\pgfqpoint{3.806306in}{0.621376in}}%
\pgfpathlineto{\pgfqpoint{3.806603in}{0.621380in}}%
\pgfpathlineto{\pgfqpoint{3.806901in}{0.621383in}}%
\pgfpathlineto{\pgfqpoint{3.807198in}{0.621386in}}%
\pgfpathlineto{\pgfqpoint{3.807496in}{0.621390in}}%
\pgfpathlineto{\pgfqpoint{3.807793in}{0.621393in}}%
\pgfpathlineto{\pgfqpoint{3.808091in}{0.621396in}}%
\pgfpathlineto{\pgfqpoint{3.808388in}{0.621400in}}%
\pgfpathlineto{\pgfqpoint{3.808686in}{0.621403in}}%
\pgfpathlineto{\pgfqpoint{3.808983in}{0.621406in}}%
\pgfpathlineto{\pgfqpoint{3.809281in}{0.621410in}}%
\pgfpathlineto{\pgfqpoint{3.809578in}{0.621413in}}%
\pgfpathlineto{\pgfqpoint{3.809875in}{0.621416in}}%
\pgfpathlineto{\pgfqpoint{3.810173in}{0.621420in}}%
\pgfpathlineto{\pgfqpoint{3.810470in}{0.621423in}}%
\pgfpathlineto{\pgfqpoint{3.810768in}{0.621426in}}%
\pgfpathlineto{\pgfqpoint{3.811065in}{0.621430in}}%
\pgfpathlineto{\pgfqpoint{3.811363in}{0.621433in}}%
\pgfpathlineto{\pgfqpoint{3.811660in}{0.621436in}}%
\pgfpathlineto{\pgfqpoint{3.811958in}{0.621440in}}%
\pgfpathlineto{\pgfqpoint{3.812255in}{0.621443in}}%
\pgfpathlineto{\pgfqpoint{3.812553in}{0.621446in}}%
\pgfpathlineto{\pgfqpoint{3.812850in}{0.621449in}}%
\pgfpathlineto{\pgfqpoint{3.813148in}{0.621453in}}%
\pgfpathlineto{\pgfqpoint{3.813445in}{0.621456in}}%
\pgfpathlineto{\pgfqpoint{3.813743in}{0.621459in}}%
\pgfpathlineto{\pgfqpoint{3.814040in}{0.621463in}}%
\pgfpathlineto{\pgfqpoint{3.814338in}{0.621466in}}%
\pgfpathlineto{\pgfqpoint{3.814635in}{0.621469in}}%
\pgfpathlineto{\pgfqpoint{3.814933in}{0.621473in}}%
\pgfpathlineto{\pgfqpoint{3.815230in}{0.621476in}}%
\pgfpathlineto{\pgfqpoint{3.815528in}{0.621479in}}%
\pgfpathlineto{\pgfqpoint{3.815825in}{0.621483in}}%
\pgfpathlineto{\pgfqpoint{3.816122in}{0.621486in}}%
\pgfpathlineto{\pgfqpoint{3.816420in}{0.621489in}}%
\pgfpathlineto{\pgfqpoint{3.816717in}{0.621493in}}%
\pgfpathlineto{\pgfqpoint{3.817015in}{0.621496in}}%
\pgfpathlineto{\pgfqpoint{3.817312in}{0.621499in}}%
\pgfpathlineto{\pgfqpoint{3.817610in}{0.621503in}}%
\pgfpathlineto{\pgfqpoint{3.817907in}{0.621506in}}%
\pgfpathlineto{\pgfqpoint{3.818205in}{0.621509in}}%
\pgfpathlineto{\pgfqpoint{3.818502in}{0.621513in}}%
\pgfpathlineto{\pgfqpoint{3.818800in}{0.621516in}}%
\pgfpathlineto{\pgfqpoint{3.819097in}{0.621519in}}%
\pgfpathlineto{\pgfqpoint{3.819395in}{0.621523in}}%
\pgfpathlineto{\pgfqpoint{3.819692in}{0.621526in}}%
\pgfpathlineto{\pgfqpoint{3.819990in}{0.621529in}}%
\pgfpathlineto{\pgfqpoint{3.820287in}{0.621533in}}%
\pgfpathlineto{\pgfqpoint{3.820585in}{0.621536in}}%
\pgfpathlineto{\pgfqpoint{3.820882in}{0.621539in}}%
\pgfpathlineto{\pgfqpoint{3.821180in}{0.621542in}}%
\pgfpathlineto{\pgfqpoint{3.821477in}{0.621546in}}%
\pgfpathlineto{\pgfqpoint{3.821775in}{0.621549in}}%
\pgfpathlineto{\pgfqpoint{3.822072in}{0.621552in}}%
\pgfpathlineto{\pgfqpoint{3.822370in}{0.621556in}}%
\pgfpathlineto{\pgfqpoint{3.822667in}{0.621559in}}%
\pgfpathlineto{\pgfqpoint{3.822964in}{0.621562in}}%
\pgfpathlineto{\pgfqpoint{3.823262in}{0.621566in}}%
\pgfpathlineto{\pgfqpoint{3.823559in}{0.621569in}}%
\pgfpathlineto{\pgfqpoint{3.823857in}{0.621572in}}%
\pgfpathlineto{\pgfqpoint{3.824154in}{0.621576in}}%
\pgfpathlineto{\pgfqpoint{3.824452in}{0.621579in}}%
\pgfpathlineto{\pgfqpoint{3.824749in}{0.621582in}}%
\pgfpathlineto{\pgfqpoint{3.825047in}{0.621586in}}%
\pgfpathlineto{\pgfqpoint{3.825344in}{0.621589in}}%
\pgfpathlineto{\pgfqpoint{3.825642in}{0.621592in}}%
\pgfpathlineto{\pgfqpoint{3.825939in}{0.621596in}}%
\pgfpathlineto{\pgfqpoint{3.826237in}{0.621599in}}%
\pgfpathlineto{\pgfqpoint{3.826534in}{0.621602in}}%
\pgfpathlineto{\pgfqpoint{3.826832in}{0.621606in}}%
\pgfpathlineto{\pgfqpoint{3.827129in}{0.621609in}}%
\pgfpathlineto{\pgfqpoint{3.827427in}{0.621612in}}%
\pgfpathlineto{\pgfqpoint{3.827724in}{0.621616in}}%
\pgfpathlineto{\pgfqpoint{3.828022in}{0.621619in}}%
\pgfpathlineto{\pgfqpoint{3.828319in}{0.621622in}}%
\pgfpathlineto{\pgfqpoint{3.828617in}{0.621625in}}%
\pgfpathlineto{\pgfqpoint{3.828914in}{0.621629in}}%
\pgfpathlineto{\pgfqpoint{3.829212in}{0.621632in}}%
\pgfpathlineto{\pgfqpoint{3.829509in}{0.621635in}}%
\pgfpathlineto{\pgfqpoint{3.829806in}{0.621639in}}%
\pgfpathlineto{\pgfqpoint{3.830104in}{0.621642in}}%
\pgfpathlineto{\pgfqpoint{3.830401in}{0.621645in}}%
\pgfpathlineto{\pgfqpoint{3.830699in}{0.621649in}}%
\pgfpathlineto{\pgfqpoint{3.830996in}{0.621652in}}%
\pgfpathlineto{\pgfqpoint{3.831294in}{0.621655in}}%
\pgfpathlineto{\pgfqpoint{3.831591in}{0.621659in}}%
\pgfpathlineto{\pgfqpoint{3.831889in}{0.621662in}}%
\pgfpathlineto{\pgfqpoint{3.832186in}{0.621665in}}%
\pgfpathlineto{\pgfqpoint{3.832484in}{0.621669in}}%
\pgfpathlineto{\pgfqpoint{3.832781in}{0.621672in}}%
\pgfpathlineto{\pgfqpoint{3.833079in}{0.621675in}}%
\pgfpathlineto{\pgfqpoint{3.833376in}{0.621679in}}%
\pgfpathlineto{\pgfqpoint{3.833674in}{0.621682in}}%
\pgfpathlineto{\pgfqpoint{3.833971in}{0.621685in}}%
\pgfpathlineto{\pgfqpoint{3.834269in}{0.621689in}}%
\pgfpathlineto{\pgfqpoint{3.834566in}{0.621692in}}%
\pgfpathlineto{\pgfqpoint{3.834864in}{0.621695in}}%
\pgfpathlineto{\pgfqpoint{3.835161in}{0.621699in}}%
\pgfpathlineto{\pgfqpoint{3.835459in}{0.621702in}}%
\pgfpathlineto{\pgfqpoint{3.835756in}{0.621705in}}%
\pgfpathlineto{\pgfqpoint{3.836053in}{0.621709in}}%
\pgfpathlineto{\pgfqpoint{3.836351in}{0.621712in}}%
\pgfpathlineto{\pgfqpoint{3.836648in}{0.621715in}}%
\pgfpathlineto{\pgfqpoint{3.836946in}{0.621718in}}%
\pgfpathlineto{\pgfqpoint{3.837243in}{0.621722in}}%
\pgfpathlineto{\pgfqpoint{3.837541in}{0.621725in}}%
\pgfpathlineto{\pgfqpoint{3.837838in}{0.621728in}}%
\pgfpathlineto{\pgfqpoint{3.838136in}{0.621732in}}%
\pgfpathlineto{\pgfqpoint{3.838433in}{0.621735in}}%
\pgfpathlineto{\pgfqpoint{3.838731in}{0.621738in}}%
\pgfpathlineto{\pgfqpoint{3.839028in}{0.621742in}}%
\pgfpathlineto{\pgfqpoint{3.839326in}{0.621745in}}%
\pgfpathlineto{\pgfqpoint{3.839623in}{0.621748in}}%
\pgfpathlineto{\pgfqpoint{3.839921in}{0.621752in}}%
\pgfpathlineto{\pgfqpoint{3.840218in}{0.621755in}}%
\pgfpathlineto{\pgfqpoint{3.840516in}{0.621758in}}%
\pgfpathlineto{\pgfqpoint{3.840813in}{0.621762in}}%
\pgfpathlineto{\pgfqpoint{3.841111in}{0.621765in}}%
\pgfpathlineto{\pgfqpoint{3.841408in}{0.621768in}}%
\pgfpathlineto{\pgfqpoint{3.841706in}{0.621772in}}%
\pgfpathlineto{\pgfqpoint{3.842003in}{0.621775in}}%
\pgfpathlineto{\pgfqpoint{3.842301in}{0.621778in}}%
\pgfpathlineto{\pgfqpoint{3.842598in}{0.621782in}}%
\pgfpathlineto{\pgfqpoint{3.842895in}{0.621785in}}%
\pgfpathlineto{\pgfqpoint{3.843193in}{0.621788in}}%
\pgfpathlineto{\pgfqpoint{3.843490in}{0.621792in}}%
\pgfpathlineto{\pgfqpoint{3.843788in}{0.621795in}}%
\pgfpathlineto{\pgfqpoint{3.844085in}{0.621798in}}%
\pgfpathlineto{\pgfqpoint{3.844383in}{0.621802in}}%
\pgfpathlineto{\pgfqpoint{3.844680in}{0.621805in}}%
\pgfpathlineto{\pgfqpoint{3.844978in}{0.621808in}}%
\pgfpathlineto{\pgfqpoint{3.845275in}{0.621811in}}%
\pgfpathlineto{\pgfqpoint{3.845573in}{0.621815in}}%
\pgfpathlineto{\pgfqpoint{3.845870in}{0.621818in}}%
\pgfpathlineto{\pgfqpoint{3.846168in}{0.621821in}}%
\pgfpathlineto{\pgfqpoint{3.846465in}{0.621825in}}%
\pgfpathlineto{\pgfqpoint{3.846763in}{0.621828in}}%
\pgfpathlineto{\pgfqpoint{3.847060in}{0.621831in}}%
\pgfpathlineto{\pgfqpoint{3.847358in}{0.621835in}}%
\pgfpathlineto{\pgfqpoint{3.847655in}{0.621816in}}%
\pgfpathlineto{\pgfqpoint{3.847953in}{0.621781in}}%
\pgfpathlineto{\pgfqpoint{3.848250in}{0.621746in}}%
\pgfpathlineto{\pgfqpoint{3.848548in}{0.621711in}}%
\pgfpathlineto{\pgfqpoint{3.848845in}{0.621676in}}%
\pgfpathlineto{\pgfqpoint{3.849143in}{0.621641in}}%
\pgfpathlineto{\pgfqpoint{3.849440in}{0.621605in}}%
\pgfpathlineto{\pgfqpoint{3.849737in}{0.621570in}}%
\pgfpathlineto{\pgfqpoint{3.850035in}{0.621535in}}%
\pgfpathlineto{\pgfqpoint{3.850332in}{0.621500in}}%
\pgfpathlineto{\pgfqpoint{3.850630in}{0.621465in}}%
\pgfpathlineto{\pgfqpoint{3.850927in}{0.621430in}}%
\pgfpathlineto{\pgfqpoint{3.851225in}{0.621395in}}%
\pgfpathlineto{\pgfqpoint{3.851522in}{0.621360in}}%
\pgfpathlineto{\pgfqpoint{3.851820in}{0.621325in}}%
\pgfpathlineto{\pgfqpoint{3.852117in}{0.621290in}}%
\pgfpathlineto{\pgfqpoint{3.852415in}{0.621255in}}%
\pgfpathlineto{\pgfqpoint{3.852712in}{0.621220in}}%
\pgfpathlineto{\pgfqpoint{3.853010in}{0.621184in}}%
\pgfpathlineto{\pgfqpoint{3.853307in}{0.621149in}}%
\pgfpathlineto{\pgfqpoint{3.853605in}{0.621114in}}%
\pgfpathlineto{\pgfqpoint{3.853902in}{0.621079in}}%
\pgfpathlineto{\pgfqpoint{3.854200in}{0.621044in}}%
\pgfpathlineto{\pgfqpoint{3.854497in}{0.621009in}}%
\pgfpathlineto{\pgfqpoint{3.854795in}{0.620974in}}%
\pgfpathlineto{\pgfqpoint{3.855092in}{0.620939in}}%
\pgfpathlineto{\pgfqpoint{3.855390in}{0.620904in}}%
\pgfpathlineto{\pgfqpoint{3.855687in}{0.620869in}}%
\pgfpathlineto{\pgfqpoint{3.855984in}{0.620834in}}%
\pgfpathlineto{\pgfqpoint{3.856282in}{0.620799in}}%
\pgfpathlineto{\pgfqpoint{3.856579in}{0.620763in}}%
\pgfpathlineto{\pgfqpoint{3.856877in}{0.620728in}}%
\pgfpathlineto{\pgfqpoint{3.857174in}{0.620693in}}%
\pgfpathlineto{\pgfqpoint{3.857472in}{0.620658in}}%
\pgfpathlineto{\pgfqpoint{3.857769in}{0.620623in}}%
\pgfpathlineto{\pgfqpoint{3.858067in}{0.620588in}}%
\pgfpathlineto{\pgfqpoint{3.858364in}{0.620553in}}%
\pgfpathlineto{\pgfqpoint{3.858662in}{0.620518in}}%
\pgfpathlineto{\pgfqpoint{3.858959in}{0.620483in}}%
\pgfpathlineto{\pgfqpoint{3.859257in}{0.620448in}}%
\pgfpathlineto{\pgfqpoint{3.859554in}{0.620413in}}%
\pgfpathlineto{\pgfqpoint{3.859852in}{0.620378in}}%
\pgfpathlineto{\pgfqpoint{3.860149in}{0.620342in}}%
\pgfpathlineto{\pgfqpoint{3.860447in}{0.620307in}}%
\pgfpathlineto{\pgfqpoint{3.860744in}{0.620272in}}%
\pgfpathlineto{\pgfqpoint{3.861042in}{0.620237in}}%
\pgfpathlineto{\pgfqpoint{3.861339in}{0.620202in}}%
\pgfpathlineto{\pgfqpoint{3.861637in}{0.620167in}}%
\pgfpathlineto{\pgfqpoint{3.861934in}{0.620132in}}%
\pgfpathlineto{\pgfqpoint{3.862232in}{0.620097in}}%
\pgfpathlineto{\pgfqpoint{3.862529in}{0.620062in}}%
\pgfpathlineto{\pgfqpoint{3.862826in}{0.620027in}}%
\pgfpathlineto{\pgfqpoint{3.863124in}{0.619992in}}%
\pgfpathlineto{\pgfqpoint{3.863421in}{0.619957in}}%
\pgfpathlineto{\pgfqpoint{3.863719in}{0.619922in}}%
\pgfpathlineto{\pgfqpoint{3.864016in}{0.619886in}}%
\pgfpathlineto{\pgfqpoint{3.864314in}{0.619851in}}%
\pgfpathlineto{\pgfqpoint{3.864611in}{0.619816in}}%
\pgfpathlineto{\pgfqpoint{3.864909in}{0.619781in}}%
\pgfpathlineto{\pgfqpoint{3.865206in}{0.619746in}}%
\pgfpathlineto{\pgfqpoint{3.865504in}{0.619711in}}%
\pgfpathlineto{\pgfqpoint{3.865801in}{0.619676in}}%
\pgfpathlineto{\pgfqpoint{3.866099in}{0.619641in}}%
\pgfpathlineto{\pgfqpoint{3.866396in}{0.619606in}}%
\pgfpathlineto{\pgfqpoint{3.866694in}{0.619571in}}%
\pgfpathlineto{\pgfqpoint{3.866991in}{0.619536in}}%
\pgfpathlineto{\pgfqpoint{3.867289in}{0.619501in}}%
\pgfpathlineto{\pgfqpoint{3.867586in}{0.619465in}}%
\pgfpathlineto{\pgfqpoint{3.867884in}{0.619430in}}%
\pgfpathlineto{\pgfqpoint{3.868181in}{0.619395in}}%
\pgfpathlineto{\pgfqpoint{3.868479in}{0.619360in}}%
\pgfpathlineto{\pgfqpoint{3.868776in}{0.619325in}}%
\pgfpathlineto{\pgfqpoint{3.869074in}{0.619290in}}%
\pgfpathlineto{\pgfqpoint{3.869371in}{0.619255in}}%
\pgfpathlineto{\pgfqpoint{3.869668in}{0.619220in}}%
\pgfpathlineto{\pgfqpoint{3.869966in}{0.619185in}}%
\pgfpathlineto{\pgfqpoint{3.870263in}{0.619150in}}%
\pgfpathlineto{\pgfqpoint{3.870561in}{0.619115in}}%
\pgfpathlineto{\pgfqpoint{3.870858in}{0.619080in}}%
\pgfpathlineto{\pgfqpoint{3.871156in}{0.619044in}}%
\pgfpathlineto{\pgfqpoint{3.871453in}{0.619009in}}%
\pgfpathlineto{\pgfqpoint{3.871751in}{0.618974in}}%
\pgfpathlineto{\pgfqpoint{3.872048in}{0.618939in}}%
\pgfpathlineto{\pgfqpoint{3.872346in}{0.618904in}}%
\pgfpathlineto{\pgfqpoint{3.872643in}{0.618869in}}%
\pgfpathlineto{\pgfqpoint{3.872941in}{0.618834in}}%
\pgfpathlineto{\pgfqpoint{3.873238in}{0.618799in}}%
\pgfpathlineto{\pgfqpoint{3.873536in}{0.618764in}}%
\pgfpathlineto{\pgfqpoint{3.873833in}{0.618729in}}%
\pgfpathlineto{\pgfqpoint{3.874131in}{0.618694in}}%
\pgfpathlineto{\pgfqpoint{3.874428in}{0.618659in}}%
\pgfpathlineto{\pgfqpoint{3.874726in}{0.618623in}}%
\pgfpathlineto{\pgfqpoint{3.875023in}{0.618588in}}%
\pgfpathlineto{\pgfqpoint{3.875321in}{0.618553in}}%
\pgfpathlineto{\pgfqpoint{3.875618in}{0.618518in}}%
\pgfpathlineto{\pgfqpoint{3.875915in}{0.618483in}}%
\pgfpathlineto{\pgfqpoint{3.876213in}{0.618448in}}%
\pgfpathlineto{\pgfqpoint{3.876510in}{0.618413in}}%
\pgfpathlineto{\pgfqpoint{3.876808in}{0.618378in}}%
\pgfpathlineto{\pgfqpoint{3.877105in}{0.618343in}}%
\pgfpathlineto{\pgfqpoint{3.877403in}{0.618308in}}%
\pgfpathlineto{\pgfqpoint{3.877700in}{0.618273in}}%
\pgfpathlineto{\pgfqpoint{3.877998in}{0.618238in}}%
\pgfpathlineto{\pgfqpoint{3.878295in}{0.618202in}}%
\pgfpathlineto{\pgfqpoint{3.878593in}{0.618167in}}%
\pgfpathlineto{\pgfqpoint{3.878890in}{0.618132in}}%
\pgfpathlineto{\pgfqpoint{3.879188in}{0.618097in}}%
\pgfpathlineto{\pgfqpoint{3.879485in}{0.618062in}}%
\pgfpathlineto{\pgfqpoint{3.879783in}{0.618027in}}%
\pgfpathlineto{\pgfqpoint{3.880080in}{0.617992in}}%
\pgfpathlineto{\pgfqpoint{3.880378in}{0.617957in}}%
\pgfpathlineto{\pgfqpoint{3.880675in}{0.617922in}}%
\pgfpathlineto{\pgfqpoint{3.880973in}{0.617887in}}%
\pgfpathlineto{\pgfqpoint{3.881270in}{0.617852in}}%
\pgfpathlineto{\pgfqpoint{3.881568in}{0.617817in}}%
\pgfpathlineto{\pgfqpoint{3.881865in}{0.617781in}}%
\pgfpathlineto{\pgfqpoint{3.882163in}{0.617746in}}%
\pgfpathlineto{\pgfqpoint{3.882460in}{0.617711in}}%
\pgfpathlineto{\pgfqpoint{3.882757in}{0.617676in}}%
\pgfpathlineto{\pgfqpoint{3.883055in}{0.617641in}}%
\pgfpathlineto{\pgfqpoint{3.883352in}{0.617606in}}%
\pgfpathlineto{\pgfqpoint{3.883650in}{0.617571in}}%
\pgfpathlineto{\pgfqpoint{3.883947in}{0.617536in}}%
\pgfpathlineto{\pgfqpoint{3.884245in}{0.617501in}}%
\pgfpathlineto{\pgfqpoint{3.884542in}{0.617466in}}%
\pgfpathlineto{\pgfqpoint{3.884840in}{0.617431in}}%
\pgfpathlineto{\pgfqpoint{3.885137in}{0.617396in}}%
\pgfpathlineto{\pgfqpoint{3.885435in}{0.617361in}}%
\pgfpathlineto{\pgfqpoint{3.885732in}{0.617325in}}%
\pgfpathlineto{\pgfqpoint{3.886030in}{0.617290in}}%
\pgfpathlineto{\pgfqpoint{3.886327in}{0.617255in}}%
\pgfpathlineto{\pgfqpoint{3.886625in}{0.617265in}}%
\pgfpathlineto{\pgfqpoint{3.886922in}{0.617423in}}%
\pgfpathlineto{\pgfqpoint{3.887220in}{0.617593in}}%
\pgfpathlineto{\pgfqpoint{3.887517in}{0.617763in}}%
\pgfpathlineto{\pgfqpoint{3.887815in}{0.617933in}}%
\pgfpathlineto{\pgfqpoint{3.888112in}{0.618103in}}%
\pgfpathlineto{\pgfqpoint{3.888410in}{0.618274in}}%
\pgfpathlineto{\pgfqpoint{3.888707in}{0.618444in}}%
\pgfpathlineto{\pgfqpoint{3.889005in}{0.618614in}}%
\pgfpathlineto{\pgfqpoint{3.889302in}{0.618784in}}%
\pgfpathlineto{\pgfqpoint{3.889599in}{0.618954in}}%
\pgfpathlineto{\pgfqpoint{3.889897in}{0.619124in}}%
\pgfpathlineto{\pgfqpoint{3.890194in}{0.619294in}}%
\pgfpathlineto{\pgfqpoint{3.890492in}{0.619465in}}%
\pgfpathlineto{\pgfqpoint{3.890789in}{0.619635in}}%
\pgfpathlineto{\pgfqpoint{3.891087in}{0.619805in}}%
\pgfpathlineto{\pgfqpoint{3.891384in}{0.619975in}}%
\pgfpathlineto{\pgfqpoint{3.891682in}{0.620145in}}%
\pgfpathlineto{\pgfqpoint{3.891979in}{0.620315in}}%
\pgfpathlineto{\pgfqpoint{3.892277in}{0.620486in}}%
\pgfpathlineto{\pgfqpoint{3.892574in}{0.620656in}}%
\pgfpathlineto{\pgfqpoint{3.892872in}{0.620826in}}%
\pgfpathlineto{\pgfqpoint{3.893169in}{0.620996in}}%
\pgfpathlineto{\pgfqpoint{3.893467in}{0.621166in}}%
\pgfpathlineto{\pgfqpoint{3.893764in}{0.621336in}}%
\pgfpathlineto{\pgfqpoint{3.894062in}{0.621506in}}%
\pgfpathlineto{\pgfqpoint{3.894359in}{0.621677in}}%
\pgfpathlineto{\pgfqpoint{3.894657in}{0.621847in}}%
\pgfpathlineto{\pgfqpoint{3.894954in}{0.622017in}}%
\pgfpathlineto{\pgfqpoint{3.895252in}{0.622187in}}%
\pgfpathlineto{\pgfqpoint{3.895549in}{0.622357in}}%
\pgfpathlineto{\pgfqpoint{3.895846in}{0.622527in}}%
\pgfpathlineto{\pgfqpoint{3.896144in}{0.622697in}}%
\pgfpathlineto{\pgfqpoint{3.896441in}{0.622868in}}%
\pgfpathlineto{\pgfqpoint{3.896739in}{0.623038in}}%
\pgfpathlineto{\pgfqpoint{3.897036in}{0.623208in}}%
\pgfpathlineto{\pgfqpoint{3.897334in}{0.623378in}}%
\pgfpathlineto{\pgfqpoint{3.897631in}{0.623548in}}%
\pgfpathlineto{\pgfqpoint{3.897929in}{0.623718in}}%
\pgfpathlineto{\pgfqpoint{3.898226in}{0.623887in}}%
\pgfpathlineto{\pgfqpoint{3.898524in}{0.624054in}}%
\pgfpathlineto{\pgfqpoint{3.898821in}{0.624220in}}%
\pgfpathlineto{\pgfqpoint{3.899119in}{0.624386in}}%
\pgfpathlineto{\pgfqpoint{3.899416in}{0.624552in}}%
\pgfpathlineto{\pgfqpoint{3.899714in}{0.624718in}}%
\pgfpathlineto{\pgfqpoint{3.900011in}{0.624884in}}%
\pgfpathlineto{\pgfqpoint{3.900309in}{0.625050in}}%
\pgfpathlineto{\pgfqpoint{3.900606in}{0.625217in}}%
\pgfpathlineto{\pgfqpoint{3.900904in}{0.625383in}}%
\pgfpathlineto{\pgfqpoint{3.901201in}{0.625549in}}%
\pgfpathlineto{\pgfqpoint{3.901499in}{0.625715in}}%
\pgfpathlineto{\pgfqpoint{3.901796in}{0.625881in}}%
\pgfpathlineto{\pgfqpoint{3.902094in}{0.626047in}}%
\pgfpathlineto{\pgfqpoint{3.902391in}{0.626213in}}%
\pgfpathlineto{\pgfqpoint{3.902688in}{0.626379in}}%
\pgfpathlineto{\pgfqpoint{3.902986in}{0.626545in}}%
\pgfpathlineto{\pgfqpoint{3.903283in}{0.626711in}}%
\pgfpathlineto{\pgfqpoint{3.903581in}{0.626878in}}%
\pgfpathlineto{\pgfqpoint{3.903878in}{0.627044in}}%
\pgfpathlineto{\pgfqpoint{3.904176in}{0.627210in}}%
\pgfpathlineto{\pgfqpoint{3.904473in}{0.627376in}}%
\pgfpathlineto{\pgfqpoint{3.904771in}{0.627542in}}%
\pgfpathlineto{\pgfqpoint{3.905068in}{0.627708in}}%
\pgfpathlineto{\pgfqpoint{3.905366in}{0.627874in}}%
\pgfpathlineto{\pgfqpoint{3.905663in}{0.628040in}}%
\pgfpathlineto{\pgfqpoint{3.905961in}{0.628206in}}%
\pgfpathlineto{\pgfqpoint{3.906258in}{0.628373in}}%
\pgfpathlineto{\pgfqpoint{3.906556in}{0.628539in}}%
\pgfpathlineto{\pgfqpoint{3.906853in}{0.628705in}}%
\pgfpathlineto{\pgfqpoint{3.907151in}{0.628871in}}%
\pgfpathlineto{\pgfqpoint{3.907448in}{0.629037in}}%
\pgfpathlineto{\pgfqpoint{3.907746in}{0.629203in}}%
\pgfpathlineto{\pgfqpoint{3.908043in}{0.629369in}}%
\pgfpathlineto{\pgfqpoint{3.908341in}{0.629535in}}%
\pgfpathlineto{\pgfqpoint{3.908638in}{0.629701in}}%
\pgfpathlineto{\pgfqpoint{3.908936in}{0.629868in}}%
\pgfpathlineto{\pgfqpoint{3.909233in}{0.630034in}}%
\pgfpathlineto{\pgfqpoint{3.909530in}{0.630200in}}%
\pgfpathlineto{\pgfqpoint{3.909828in}{0.630366in}}%
\pgfpathlineto{\pgfqpoint{3.910125in}{0.630532in}}%
\pgfpathlineto{\pgfqpoint{3.910423in}{0.630698in}}%
\pgfpathlineto{\pgfqpoint{3.910720in}{0.630864in}}%
\pgfpathlineto{\pgfqpoint{3.911018in}{0.631030in}}%
\pgfpathlineto{\pgfqpoint{3.911315in}{0.631196in}}%
\pgfpathlineto{\pgfqpoint{3.911613in}{0.631363in}}%
\pgfpathlineto{\pgfqpoint{3.911910in}{0.631529in}}%
\pgfpathlineto{\pgfqpoint{3.912208in}{0.631695in}}%
\pgfpathlineto{\pgfqpoint{3.912505in}{0.631861in}}%
\pgfpathlineto{\pgfqpoint{3.912803in}{0.632027in}}%
\pgfpathlineto{\pgfqpoint{3.913100in}{0.632071in}}%
\pgfpathlineto{\pgfqpoint{3.913398in}{0.631752in}}%
\pgfpathlineto{\pgfqpoint{3.913695in}{0.631408in}}%
\pgfpathlineto{\pgfqpoint{3.913993in}{0.631065in}}%
\pgfpathlineto{\pgfqpoint{3.914290in}{0.630721in}}%
\pgfpathlineto{\pgfqpoint{3.914588in}{0.630378in}}%
\pgfpathlineto{\pgfqpoint{3.914885in}{0.630034in}}%
\pgfpathlineto{\pgfqpoint{3.915183in}{0.629691in}}%
\pgfpathlineto{\pgfqpoint{3.915480in}{0.629347in}}%
\pgfpathlineto{\pgfqpoint{3.915777in}{0.629004in}}%
\pgfpathlineto{\pgfqpoint{3.916075in}{0.628660in}}%
\pgfpathlineto{\pgfqpoint{3.916372in}{0.628316in}}%
\pgfpathlineto{\pgfqpoint{3.916670in}{0.627973in}}%
\pgfpathlineto{\pgfqpoint{3.916967in}{0.627629in}}%
\pgfpathlineto{\pgfqpoint{3.917265in}{0.627286in}}%
\pgfpathlineto{\pgfqpoint{3.917562in}{0.626942in}}%
\pgfpathlineto{\pgfqpoint{3.917860in}{0.626599in}}%
\pgfpathlineto{\pgfqpoint{3.918157in}{0.626255in}}%
\pgfpathlineto{\pgfqpoint{3.918455in}{0.625912in}}%
\pgfpathlineto{\pgfqpoint{3.918752in}{0.625568in}}%
\pgfpathlineto{\pgfqpoint{3.919050in}{0.625225in}}%
\pgfpathlineto{\pgfqpoint{3.919347in}{0.624881in}}%
\pgfpathlineto{\pgfqpoint{3.919645in}{0.624537in}}%
\pgfpathlineto{\pgfqpoint{3.919942in}{0.624194in}}%
\pgfpathlineto{\pgfqpoint{3.920240in}{0.623850in}}%
\pgfpathlineto{\pgfqpoint{3.920537in}{0.623507in}}%
\pgfpathlineto{\pgfqpoint{3.920835in}{0.623163in}}%
\pgfpathlineto{\pgfqpoint{3.921132in}{0.622820in}}%
\pgfpathlineto{\pgfqpoint{3.921430in}{0.622476in}}%
\pgfpathlineto{\pgfqpoint{3.921727in}{0.622133in}}%
\pgfpathlineto{\pgfqpoint{3.922025in}{0.621789in}}%
\pgfpathlineto{\pgfqpoint{3.922322in}{0.621446in}}%
\pgfpathlineto{\pgfqpoint{3.922619in}{0.621102in}}%
\pgfpathlineto{\pgfqpoint{3.922917in}{0.620759in}}%
\pgfpathlineto{\pgfqpoint{3.923214in}{0.620415in}}%
\pgfpathlineto{\pgfqpoint{3.923512in}{0.620071in}}%
\pgfpathlineto{\pgfqpoint{3.923809in}{0.619728in}}%
\pgfpathlineto{\pgfqpoint{3.924107in}{0.619384in}}%
\pgfpathlineto{\pgfqpoint{3.924404in}{0.619041in}}%
\pgfpathlineto{\pgfqpoint{3.924702in}{0.618697in}}%
\pgfpathlineto{\pgfqpoint{3.924999in}{0.618354in}}%
\pgfpathlineto{\pgfqpoint{3.925297in}{0.618010in}}%
\pgfpathlineto{\pgfqpoint{3.925594in}{0.617667in}}%
\pgfpathlineto{\pgfqpoint{3.925892in}{0.617323in}}%
\pgfpathlineto{\pgfqpoint{3.926189in}{0.616980in}}%
\pgfpathlineto{\pgfqpoint{3.926487in}{0.616636in}}%
\pgfpathlineto{\pgfqpoint{3.926784in}{0.616292in}}%
\pgfpathlineto{\pgfqpoint{3.927082in}{0.616063in}}%
\pgfpathlineto{\pgfqpoint{3.927379in}{0.616048in}}%
\pgfpathlineto{\pgfqpoint{3.927677in}{0.616038in}}%
\pgfpathlineto{\pgfqpoint{3.927974in}{0.616028in}}%
\pgfpathlineto{\pgfqpoint{3.928272in}{0.616018in}}%
\pgfpathlineto{\pgfqpoint{3.928569in}{0.616009in}}%
\pgfpathlineto{\pgfqpoint{3.928867in}{0.615999in}}%
\pgfpathlineto{\pgfqpoint{3.929164in}{0.615989in}}%
\pgfpathlineto{\pgfqpoint{3.929461in}{0.615980in}}%
\pgfpathlineto{\pgfqpoint{3.929759in}{0.615970in}}%
\pgfpathlineto{\pgfqpoint{3.930056in}{0.615960in}}%
\pgfpathlineto{\pgfqpoint{3.930354in}{0.615950in}}%
\pgfpathlineto{\pgfqpoint{3.930651in}{0.615941in}}%
\pgfpathlineto{\pgfqpoint{3.930949in}{0.615931in}}%
\pgfpathlineto{\pgfqpoint{3.931246in}{0.615921in}}%
\pgfpathlineto{\pgfqpoint{3.931544in}{0.615911in}}%
\pgfpathlineto{\pgfqpoint{3.931841in}{0.615902in}}%
\pgfpathlineto{\pgfqpoint{3.932139in}{0.615892in}}%
\pgfpathlineto{\pgfqpoint{3.932436in}{0.615882in}}%
\pgfpathlineto{\pgfqpoint{3.932734in}{0.615872in}}%
\pgfpathlineto{\pgfqpoint{3.933031in}{0.615863in}}%
\pgfpathlineto{\pgfqpoint{3.933329in}{0.615853in}}%
\pgfpathlineto{\pgfqpoint{3.933626in}{0.615843in}}%
\pgfpathlineto{\pgfqpoint{3.933924in}{0.615834in}}%
\pgfpathlineto{\pgfqpoint{3.934221in}{0.615824in}}%
\pgfpathlineto{\pgfqpoint{3.934519in}{0.615814in}}%
\pgfpathlineto{\pgfqpoint{3.934816in}{0.615804in}}%
\pgfpathlineto{\pgfqpoint{3.935114in}{0.615795in}}%
\pgfpathlineto{\pgfqpoint{3.935411in}{0.615785in}}%
\pgfpathlineto{\pgfqpoint{3.935708in}{0.615775in}}%
\pgfpathlineto{\pgfqpoint{3.936006in}{0.615765in}}%
\pgfpathlineto{\pgfqpoint{3.936303in}{0.615756in}}%
\pgfpathlineto{\pgfqpoint{3.936601in}{0.615746in}}%
\pgfpathlineto{\pgfqpoint{3.936898in}{0.615736in}}%
\pgfpathlineto{\pgfqpoint{3.937196in}{0.615726in}}%
\pgfpathlineto{\pgfqpoint{3.937493in}{0.615717in}}%
\pgfpathlineto{\pgfqpoint{3.937791in}{0.615707in}}%
\pgfpathlineto{\pgfqpoint{3.938088in}{0.615697in}}%
\pgfpathlineto{\pgfqpoint{3.938386in}{0.615687in}}%
\pgfpathlineto{\pgfqpoint{3.938683in}{0.615678in}}%
\pgfpathlineto{\pgfqpoint{3.938981in}{0.615668in}}%
\pgfpathlineto{\pgfqpoint{3.939278in}{0.615658in}}%
\pgfpathlineto{\pgfqpoint{3.939576in}{0.615649in}}%
\pgfpathlineto{\pgfqpoint{3.939873in}{0.615639in}}%
\pgfpathlineto{\pgfqpoint{3.940171in}{0.615629in}}%
\pgfpathlineto{\pgfqpoint{3.940468in}{0.615619in}}%
\pgfpathlineto{\pgfqpoint{3.940766in}{0.615610in}}%
\pgfpathlineto{\pgfqpoint{3.941063in}{0.615600in}}%
\pgfpathlineto{\pgfqpoint{3.941361in}{0.615590in}}%
\pgfpathlineto{\pgfqpoint{3.941658in}{0.615580in}}%
\pgfpathlineto{\pgfqpoint{3.941956in}{0.615571in}}%
\pgfpathlineto{\pgfqpoint{3.942253in}{0.615561in}}%
\pgfpathlineto{\pgfqpoint{3.942550in}{0.615551in}}%
\pgfpathlineto{\pgfqpoint{3.942848in}{0.615541in}}%
\pgfpathlineto{\pgfqpoint{3.943145in}{0.615532in}}%
\pgfpathlineto{\pgfqpoint{3.943443in}{0.615522in}}%
\pgfpathlineto{\pgfqpoint{3.943740in}{0.615512in}}%
\pgfpathlineto{\pgfqpoint{3.944038in}{0.615502in}}%
\pgfpathlineto{\pgfqpoint{3.944335in}{0.615493in}}%
\pgfpathlineto{\pgfqpoint{3.944633in}{0.615483in}}%
\pgfpathlineto{\pgfqpoint{3.944930in}{0.615473in}}%
\pgfpathlineto{\pgfqpoint{3.945228in}{0.615464in}}%
\pgfpathlineto{\pgfqpoint{3.945525in}{0.615454in}}%
\pgfpathlineto{\pgfqpoint{3.945823in}{0.615444in}}%
\pgfpathlineto{\pgfqpoint{3.946120in}{0.615434in}}%
\pgfpathlineto{\pgfqpoint{3.946418in}{0.615425in}}%
\pgfpathlineto{\pgfqpoint{3.946715in}{0.615415in}}%
\pgfpathlineto{\pgfqpoint{3.947013in}{0.615405in}}%
\pgfpathlineto{\pgfqpoint{3.947310in}{0.615395in}}%
\pgfpathlineto{\pgfqpoint{3.947608in}{0.615386in}}%
\pgfpathlineto{\pgfqpoint{3.947905in}{0.615376in}}%
\pgfpathlineto{\pgfqpoint{3.948203in}{0.615366in}}%
\pgfpathlineto{\pgfqpoint{3.948500in}{0.615356in}}%
\pgfpathlineto{\pgfqpoint{3.948798in}{0.615347in}}%
\pgfpathlineto{\pgfqpoint{3.949095in}{0.615337in}}%
\pgfpathlineto{\pgfqpoint{3.949392in}{0.615327in}}%
\pgfpathlineto{\pgfqpoint{3.949690in}{0.615317in}}%
\pgfpathlineto{\pgfqpoint{3.949987in}{0.615308in}}%
\pgfpathlineto{\pgfqpoint{3.950285in}{0.615298in}}%
\pgfpathlineto{\pgfqpoint{3.950582in}{0.615288in}}%
\pgfpathlineto{\pgfqpoint{3.950880in}{0.615279in}}%
\pgfpathlineto{\pgfqpoint{3.951177in}{0.615269in}}%
\pgfpathlineto{\pgfqpoint{3.951475in}{0.615259in}}%
\pgfpathlineto{\pgfqpoint{3.951772in}{0.615249in}}%
\pgfpathlineto{\pgfqpoint{3.952070in}{0.615240in}}%
\pgfpathlineto{\pgfqpoint{3.952367in}{0.615230in}}%
\pgfpathlineto{\pgfqpoint{3.952665in}{0.615220in}}%
\pgfpathlineto{\pgfqpoint{3.952962in}{0.615210in}}%
\pgfpathlineto{\pgfqpoint{3.953260in}{0.615201in}}%
\pgfpathlineto{\pgfqpoint{3.953557in}{0.615191in}}%
\pgfpathlineto{\pgfqpoint{3.953855in}{0.615181in}}%
\pgfpathlineto{\pgfqpoint{3.954152in}{0.615171in}}%
\pgfpathlineto{\pgfqpoint{3.954450in}{0.615162in}}%
\pgfpathlineto{\pgfqpoint{3.954747in}{0.615152in}}%
\pgfpathlineto{\pgfqpoint{3.955045in}{0.615142in}}%
\pgfpathlineto{\pgfqpoint{3.955342in}{0.615133in}}%
\pgfpathlineto{\pgfqpoint{3.955639in}{0.615123in}}%
\pgfpathlineto{\pgfqpoint{3.955937in}{0.615113in}}%
\pgfpathlineto{\pgfqpoint{3.956234in}{0.615103in}}%
\pgfpathlineto{\pgfqpoint{3.956532in}{0.615094in}}%
\pgfpathlineto{\pgfqpoint{3.956829in}{0.615084in}}%
\pgfpathlineto{\pgfqpoint{3.957127in}{0.615074in}}%
\pgfpathlineto{\pgfqpoint{3.957424in}{0.615064in}}%
\pgfpathlineto{\pgfqpoint{3.957722in}{0.615055in}}%
\pgfpathlineto{\pgfqpoint{3.958019in}{0.615045in}}%
\pgfpathlineto{\pgfqpoint{3.958317in}{0.615035in}}%
\pgfpathlineto{\pgfqpoint{3.958614in}{0.615025in}}%
\pgfpathlineto{\pgfqpoint{3.958912in}{0.615016in}}%
\pgfpathlineto{\pgfqpoint{3.959209in}{0.615006in}}%
\pgfpathlineto{\pgfqpoint{3.959507in}{0.614996in}}%
\pgfpathlineto{\pgfqpoint{3.959804in}{0.614986in}}%
\pgfpathlineto{\pgfqpoint{3.960102in}{0.614977in}}%
\pgfpathlineto{\pgfqpoint{3.960399in}{0.614967in}}%
\pgfpathlineto{\pgfqpoint{3.960697in}{0.614957in}}%
\pgfpathlineto{\pgfqpoint{3.960994in}{0.614948in}}%
\pgfpathlineto{\pgfqpoint{3.961292in}{0.614938in}}%
\pgfpathlineto{\pgfqpoint{3.961589in}{0.614928in}}%
\pgfpathlineto{\pgfqpoint{3.961887in}{0.614918in}}%
\pgfpathlineto{\pgfqpoint{3.962184in}{0.614909in}}%
\pgfpathlineto{\pgfqpoint{3.962481in}{0.614899in}}%
\pgfpathlineto{\pgfqpoint{3.962779in}{0.614889in}}%
\pgfpathlineto{\pgfqpoint{3.963076in}{0.614879in}}%
\pgfpathlineto{\pgfqpoint{3.963374in}{0.614870in}}%
\pgfpathlineto{\pgfqpoint{3.963671in}{0.614860in}}%
\pgfpathlineto{\pgfqpoint{3.963969in}{0.614850in}}%
\pgfpathlineto{\pgfqpoint{3.964266in}{0.614840in}}%
\pgfpathlineto{\pgfqpoint{3.964564in}{0.614831in}}%
\pgfpathlineto{\pgfqpoint{3.964861in}{0.614821in}}%
\pgfpathlineto{\pgfqpoint{3.965159in}{0.614811in}}%
\pgfpathlineto{\pgfqpoint{3.965456in}{0.614801in}}%
\pgfpathlineto{\pgfqpoint{3.965754in}{0.614792in}}%
\pgfpathlineto{\pgfqpoint{3.966051in}{0.614782in}}%
\pgfpathlineto{\pgfqpoint{3.966349in}{0.614772in}}%
\pgfpathlineto{\pgfqpoint{3.966646in}{0.614763in}}%
\pgfpathlineto{\pgfqpoint{3.966944in}{0.614753in}}%
\pgfpathlineto{\pgfqpoint{3.967241in}{0.614743in}}%
\pgfpathlineto{\pgfqpoint{3.967539in}{0.614733in}}%
\pgfpathlineto{\pgfqpoint{3.967836in}{0.614724in}}%
\pgfpathlineto{\pgfqpoint{3.968134in}{0.614714in}}%
\pgfpathlineto{\pgfqpoint{3.968431in}{0.614704in}}%
\pgfpathlineto{\pgfqpoint{3.968729in}{0.614694in}}%
\pgfpathlineto{\pgfqpoint{3.969026in}{0.614685in}}%
\pgfpathlineto{\pgfqpoint{3.969323in}{0.614675in}}%
\pgfpathlineto{\pgfqpoint{3.969621in}{0.614665in}}%
\pgfpathlineto{\pgfqpoint{3.969918in}{0.614655in}}%
\pgfpathlineto{\pgfqpoint{3.970216in}{0.614646in}}%
\pgfpathlineto{\pgfqpoint{3.970513in}{0.614636in}}%
\pgfpathlineto{\pgfqpoint{3.970811in}{0.614626in}}%
\pgfpathlineto{\pgfqpoint{3.971108in}{0.614616in}}%
\pgfpathlineto{\pgfqpoint{3.971406in}{0.614607in}}%
\pgfpathlineto{\pgfqpoint{3.971703in}{0.614597in}}%
\pgfpathlineto{\pgfqpoint{3.972001in}{0.614587in}}%
\pgfpathlineto{\pgfqpoint{3.972298in}{0.614578in}}%
\pgfpathlineto{\pgfqpoint{3.972596in}{0.614568in}}%
\pgfpathlineto{\pgfqpoint{3.972893in}{0.614558in}}%
\pgfpathlineto{\pgfqpoint{3.973191in}{0.614548in}}%
\pgfpathlineto{\pgfqpoint{3.973488in}{0.614539in}}%
\pgfpathlineto{\pgfqpoint{3.973786in}{0.614529in}}%
\pgfpathlineto{\pgfqpoint{3.974083in}{0.614519in}}%
\pgfpathlineto{\pgfqpoint{3.974381in}{0.614509in}}%
\pgfpathlineto{\pgfqpoint{3.974678in}{0.614500in}}%
\pgfpathlineto{\pgfqpoint{3.974976in}{0.614490in}}%
\pgfpathlineto{\pgfqpoint{3.975273in}{0.614480in}}%
\pgfpathlineto{\pgfqpoint{3.975571in}{0.614470in}}%
\pgfpathlineto{\pgfqpoint{3.975868in}{0.614461in}}%
\pgfpathlineto{\pgfqpoint{3.976165in}{0.614451in}}%
\pgfpathlineto{\pgfqpoint{3.976463in}{0.614441in}}%
\pgfpathlineto{\pgfqpoint{3.976760in}{0.614432in}}%
\pgfpathlineto{\pgfqpoint{3.977058in}{0.614422in}}%
\pgfpathlineto{\pgfqpoint{3.977355in}{0.614412in}}%
\pgfpathlineto{\pgfqpoint{3.977653in}{0.614402in}}%
\pgfpathlineto{\pgfqpoint{3.977950in}{0.614393in}}%
\pgfpathlineto{\pgfqpoint{3.978248in}{0.614383in}}%
\pgfpathlineto{\pgfqpoint{3.978545in}{0.614373in}}%
\pgfpathlineto{\pgfqpoint{3.978843in}{0.614363in}}%
\pgfpathlineto{\pgfqpoint{3.979140in}{0.614354in}}%
\pgfpathlineto{\pgfqpoint{3.979438in}{0.614344in}}%
\pgfpathlineto{\pgfqpoint{3.979735in}{0.614334in}}%
\pgfpathlineto{\pgfqpoint{3.980033in}{0.614324in}}%
\pgfpathlineto{\pgfqpoint{3.980330in}{0.614315in}}%
\pgfpathlineto{\pgfqpoint{3.980628in}{0.614305in}}%
\pgfpathlineto{\pgfqpoint{3.980925in}{0.614295in}}%
\pgfpathlineto{\pgfqpoint{3.981223in}{0.614285in}}%
\pgfpathlineto{\pgfqpoint{3.981520in}{0.614276in}}%
\pgfpathlineto{\pgfqpoint{3.981818in}{0.614266in}}%
\pgfpathlineto{\pgfqpoint{3.982115in}{0.614256in}}%
\pgfpathlineto{\pgfqpoint{3.982412in}{0.614247in}}%
\pgfpathlineto{\pgfqpoint{3.982710in}{0.614237in}}%
\pgfpathlineto{\pgfqpoint{3.983007in}{0.614227in}}%
\pgfpathlineto{\pgfqpoint{3.983305in}{0.614217in}}%
\pgfpathlineto{\pgfqpoint{3.983602in}{0.614208in}}%
\pgfpathlineto{\pgfqpoint{3.983900in}{0.614198in}}%
\pgfpathlineto{\pgfqpoint{3.984197in}{0.614188in}}%
\pgfpathlineto{\pgfqpoint{3.984495in}{0.614178in}}%
\pgfpathlineto{\pgfqpoint{3.984792in}{0.614225in}}%
\pgfpathlineto{\pgfqpoint{3.985090in}{0.614380in}}%
\pgfpathlineto{\pgfqpoint{3.985387in}{0.614537in}}%
\pgfpathlineto{\pgfqpoint{3.985685in}{0.614694in}}%
\pgfpathlineto{\pgfqpoint{3.985982in}{0.614851in}}%
\pgfpathlineto{\pgfqpoint{3.986280in}{0.615008in}}%
\pgfpathlineto{\pgfqpoint{3.986577in}{0.615165in}}%
\pgfpathlineto{\pgfqpoint{3.986875in}{0.615322in}}%
\pgfpathlineto{\pgfqpoint{3.987172in}{0.615479in}}%
\pgfpathlineto{\pgfqpoint{3.987470in}{0.615636in}}%
\pgfpathlineto{\pgfqpoint{3.987767in}{0.615793in}}%
\pgfpathlineto{\pgfqpoint{3.988065in}{0.615950in}}%
\pgfpathlineto{\pgfqpoint{3.988362in}{0.616107in}}%
\pgfpathlineto{\pgfqpoint{3.988660in}{0.616264in}}%
\pgfpathlineto{\pgfqpoint{3.988957in}{0.616422in}}%
\pgfpathlineto{\pgfqpoint{3.989254in}{0.616579in}}%
\pgfpathlineto{\pgfqpoint{3.989552in}{0.616736in}}%
\pgfpathlineto{\pgfqpoint{3.989849in}{0.616893in}}%
\pgfpathlineto{\pgfqpoint{3.990147in}{0.617050in}}%
\pgfpathlineto{\pgfqpoint{3.990444in}{0.617207in}}%
\pgfpathlineto{\pgfqpoint{3.990742in}{0.617364in}}%
\pgfpathlineto{\pgfqpoint{3.991039in}{0.617521in}}%
\pgfpathlineto{\pgfqpoint{3.991337in}{0.617678in}}%
\pgfpathlineto{\pgfqpoint{3.991634in}{0.617835in}}%
\pgfpathlineto{\pgfqpoint{3.991932in}{0.617992in}}%
\pgfpathlineto{\pgfqpoint{3.992229in}{0.618149in}}%
\pgfpathlineto{\pgfqpoint{3.992527in}{0.618306in}}%
\pgfpathlineto{\pgfqpoint{3.992824in}{0.618463in}}%
\pgfpathlineto{\pgfqpoint{3.993122in}{0.618620in}}%
\pgfpathlineto{\pgfqpoint{3.993419in}{0.618777in}}%
\pgfpathlineto{\pgfqpoint{3.993717in}{0.618934in}}%
\pgfpathlineto{\pgfqpoint{3.994014in}{0.619091in}}%
\pgfpathlineto{\pgfqpoint{3.994312in}{0.619248in}}%
\pgfpathlineto{\pgfqpoint{3.994609in}{0.619405in}}%
\pgfpathlineto{\pgfqpoint{3.994907in}{0.619562in}}%
\pgfpathlineto{\pgfqpoint{3.995204in}{0.619720in}}%
\pgfpathlineto{\pgfqpoint{3.995502in}{0.619877in}}%
\pgfpathlineto{\pgfqpoint{3.995799in}{0.620034in}}%
\pgfpathlineto{\pgfqpoint{3.996096in}{0.620191in}}%
\pgfpathlineto{\pgfqpoint{3.996394in}{0.620348in}}%
\pgfpathlineto{\pgfqpoint{3.996691in}{0.620505in}}%
\pgfpathlineto{\pgfqpoint{3.996989in}{0.620662in}}%
\pgfpathlineto{\pgfqpoint{3.997286in}{0.620819in}}%
\pgfpathlineto{\pgfqpoint{3.997584in}{0.620976in}}%
\pgfpathlineto{\pgfqpoint{3.997881in}{0.621061in}}%
\pgfpathlineto{\pgfqpoint{3.998179in}{0.621066in}}%
\pgfpathlineto{\pgfqpoint{3.998476in}{0.621070in}}%
\pgfpathlineto{\pgfqpoint{3.998774in}{0.621075in}}%
\pgfpathlineto{\pgfqpoint{3.999071in}{0.621080in}}%
\pgfpathlineto{\pgfqpoint{3.999369in}{0.621085in}}%
\pgfpathlineto{\pgfqpoint{3.999666in}{0.621090in}}%
\pgfpathlineto{\pgfqpoint{3.999964in}{0.621095in}}%
\pgfpathlineto{\pgfqpoint{4.000261in}{0.621100in}}%
\pgfpathlineto{\pgfqpoint{4.000559in}{0.621104in}}%
\pgfpathlineto{\pgfqpoint{4.000856in}{0.621109in}}%
\pgfpathlineto{\pgfqpoint{4.001154in}{0.621114in}}%
\pgfpathlineto{\pgfqpoint{4.001451in}{0.621119in}}%
\pgfpathlineto{\pgfqpoint{4.001749in}{0.621124in}}%
\pgfpathlineto{\pgfqpoint{4.002046in}{0.621129in}}%
\pgfpathlineto{\pgfqpoint{4.002343in}{0.621134in}}%
\pgfpathlineto{\pgfqpoint{4.002641in}{0.621138in}}%
\pgfpathlineto{\pgfqpoint{4.002938in}{0.621143in}}%
\pgfpathlineto{\pgfqpoint{4.003236in}{0.621148in}}%
\pgfpathlineto{\pgfqpoint{4.003533in}{0.621153in}}%
\pgfpathlineto{\pgfqpoint{4.003831in}{0.621158in}}%
\pgfpathlineto{\pgfqpoint{4.004128in}{0.621163in}}%
\pgfpathlineto{\pgfqpoint{4.004426in}{0.621167in}}%
\pgfpathlineto{\pgfqpoint{4.004723in}{0.621172in}}%
\pgfpathlineto{\pgfqpoint{4.005021in}{0.621177in}}%
\pgfpathlineto{\pgfqpoint{4.005318in}{0.621182in}}%
\pgfpathlineto{\pgfqpoint{4.005616in}{0.621187in}}%
\pgfpathlineto{\pgfqpoint{4.005913in}{0.621192in}}%
\pgfpathlineto{\pgfqpoint{4.006211in}{0.621197in}}%
\pgfpathlineto{\pgfqpoint{4.006508in}{0.621201in}}%
\pgfpathlineto{\pgfqpoint{4.006806in}{0.621206in}}%
\pgfpathlineto{\pgfqpoint{4.007103in}{0.621211in}}%
\pgfpathlineto{\pgfqpoint{4.007401in}{0.621216in}}%
\pgfpathlineto{\pgfqpoint{4.007698in}{0.621221in}}%
\pgfpathlineto{\pgfqpoint{4.007996in}{0.621226in}}%
\pgfpathlineto{\pgfqpoint{4.008293in}{0.621231in}}%
\pgfpathlineto{\pgfqpoint{4.008591in}{0.621235in}}%
\pgfpathlineto{\pgfqpoint{4.008888in}{0.621240in}}%
\pgfpathlineto{\pgfqpoint{4.009185in}{0.621245in}}%
\pgfpathlineto{\pgfqpoint{4.009483in}{0.621250in}}%
\pgfpathlineto{\pgfqpoint{4.009780in}{0.621255in}}%
\pgfpathlineto{\pgfqpoint{4.010078in}{0.621260in}}%
\pgfpathlineto{\pgfqpoint{4.010375in}{0.621265in}}%
\pgfpathlineto{\pgfqpoint{4.010673in}{0.621269in}}%
\pgfpathlineto{\pgfqpoint{4.010970in}{0.621274in}}%
\pgfpathlineto{\pgfqpoint{4.011268in}{0.621279in}}%
\pgfpathlineto{\pgfqpoint{4.011565in}{0.621284in}}%
\pgfpathlineto{\pgfqpoint{4.011863in}{0.621289in}}%
\pgfpathlineto{\pgfqpoint{4.012160in}{0.621294in}}%
\pgfpathlineto{\pgfqpoint{4.012458in}{0.621299in}}%
\pgfpathlineto{\pgfqpoint{4.012755in}{0.621303in}}%
\pgfpathlineto{\pgfqpoint{4.013053in}{0.621308in}}%
\pgfpathlineto{\pgfqpoint{4.013350in}{0.621313in}}%
\pgfpathlineto{\pgfqpoint{4.013648in}{0.621318in}}%
\pgfpathlineto{\pgfqpoint{4.013945in}{0.621323in}}%
\pgfpathlineto{\pgfqpoint{4.014243in}{0.621328in}}%
\pgfpathlineto{\pgfqpoint{4.014540in}{0.621333in}}%
\pgfpathlineto{\pgfqpoint{4.014838in}{0.621337in}}%
\pgfpathlineto{\pgfqpoint{4.015135in}{0.621342in}}%
\pgfpathlineto{\pgfqpoint{4.015433in}{0.621347in}}%
\pgfpathlineto{\pgfqpoint{4.015730in}{0.621352in}}%
\pgfpathlineto{\pgfqpoint{4.016027in}{0.621357in}}%
\pgfpathlineto{\pgfqpoint{4.016325in}{0.621362in}}%
\pgfpathlineto{\pgfqpoint{4.016622in}{0.621367in}}%
\pgfpathlineto{\pgfqpoint{4.016920in}{0.621371in}}%
\pgfpathlineto{\pgfqpoint{4.017217in}{0.621376in}}%
\pgfpathlineto{\pgfqpoint{4.017515in}{0.621381in}}%
\pgfpathlineto{\pgfqpoint{4.017812in}{0.621386in}}%
\pgfpathlineto{\pgfqpoint{4.018110in}{0.621391in}}%
\pgfpathlineto{\pgfqpoint{4.018407in}{0.621396in}}%
\pgfpathlineto{\pgfqpoint{4.018705in}{0.621401in}}%
\pgfpathlineto{\pgfqpoint{4.019002in}{0.621405in}}%
\pgfpathlineto{\pgfqpoint{4.019300in}{0.621410in}}%
\pgfpathlineto{\pgfqpoint{4.019597in}{0.621415in}}%
\pgfpathlineto{\pgfqpoint{4.019895in}{0.621420in}}%
\pgfpathlineto{\pgfqpoint{4.020192in}{0.621425in}}%
\pgfpathlineto{\pgfqpoint{4.020490in}{0.621430in}}%
\pgfpathlineto{\pgfqpoint{4.020787in}{0.621435in}}%
\pgfpathlineto{\pgfqpoint{4.021085in}{0.621439in}}%
\pgfpathlineto{\pgfqpoint{4.021382in}{0.621444in}}%
\pgfpathlineto{\pgfqpoint{4.021680in}{0.621449in}}%
\pgfpathlineto{\pgfqpoint{4.021977in}{0.621454in}}%
\pgfpathlineto{\pgfqpoint{4.022274in}{0.621459in}}%
\pgfpathlineto{\pgfqpoint{4.022572in}{0.621464in}}%
\pgfpathlineto{\pgfqpoint{4.022869in}{0.621468in}}%
\pgfpathlineto{\pgfqpoint{4.023167in}{0.621473in}}%
\pgfpathlineto{\pgfqpoint{4.023464in}{0.621478in}}%
\pgfpathlineto{\pgfqpoint{4.023762in}{0.621483in}}%
\pgfpathlineto{\pgfqpoint{4.024059in}{0.621488in}}%
\pgfpathlineto{\pgfqpoint{4.024357in}{0.621493in}}%
\pgfpathlineto{\pgfqpoint{4.024654in}{0.621498in}}%
\pgfpathlineto{\pgfqpoint{4.024952in}{0.621502in}}%
\pgfpathlineto{\pgfqpoint{4.025249in}{0.621507in}}%
\pgfpathlineto{\pgfqpoint{4.025547in}{0.621512in}}%
\pgfpathlineto{\pgfqpoint{4.025844in}{0.621517in}}%
\pgfpathlineto{\pgfqpoint{4.026142in}{0.621436in}}%
\pgfpathlineto{\pgfqpoint{4.026439in}{0.619563in}}%
\pgfpathlineto{\pgfqpoint{4.026737in}{0.616979in}}%
\pgfpathlineto{\pgfqpoint{4.027034in}{0.614395in}}%
\pgfpathlineto{\pgfqpoint{4.027332in}{0.612811in}}%
\pgfpathlineto{\pgfqpoint{4.027629in}{0.612984in}}%
\pgfpathlineto{\pgfqpoint{4.027927in}{0.613234in}}%
\pgfpathlineto{\pgfqpoint{4.028224in}{0.613483in}}%
\pgfpathlineto{\pgfqpoint{4.028522in}{0.613732in}}%
\pgfpathlineto{\pgfqpoint{4.028819in}{0.613981in}}%
\pgfpathlineto{\pgfqpoint{4.029116in}{0.614230in}}%
\pgfpathlineto{\pgfqpoint{4.029414in}{0.614479in}}%
\pgfpathlineto{\pgfqpoint{4.029711in}{0.614728in}}%
\pgfpathlineto{\pgfqpoint{4.030009in}{0.614977in}}%
\pgfpathlineto{\pgfqpoint{4.030306in}{0.615226in}}%
\pgfpathlineto{\pgfqpoint{4.030604in}{0.615475in}}%
\pgfpathlineto{\pgfqpoint{4.030901in}{0.615724in}}%
\pgfpathlineto{\pgfqpoint{4.031199in}{0.615973in}}%
\pgfpathlineto{\pgfqpoint{4.031496in}{0.616222in}}%
\pgfpathlineto{\pgfqpoint{4.031794in}{0.616471in}}%
\pgfpathlineto{\pgfqpoint{4.032091in}{0.616720in}}%
\pgfpathlineto{\pgfqpoint{4.032389in}{0.616969in}}%
\pgfpathlineto{\pgfqpoint{4.032686in}{0.617219in}}%
\pgfpathlineto{\pgfqpoint{4.032984in}{0.617427in}}%
\pgfpathlineto{\pgfqpoint{4.033281in}{0.617877in}}%
\pgfpathlineto{\pgfqpoint{4.033579in}{0.618749in}}%
\pgfpathlineto{\pgfqpoint{4.033876in}{0.619622in}}%
\pgfpathlineto{\pgfqpoint{4.034174in}{0.620495in}}%
\pgfpathlineto{\pgfqpoint{4.034471in}{0.621368in}}%
\pgfpathlineto{\pgfqpoint{4.034769in}{0.622241in}}%
\pgfpathlineto{\pgfqpoint{4.035066in}{0.623113in}}%
\pgfpathlineto{\pgfqpoint{4.035364in}{0.623986in}}%
\pgfpathlineto{\pgfqpoint{4.035661in}{0.624859in}}%
\pgfpathlineto{\pgfqpoint{4.035958in}{0.625732in}}%
\pgfpathlineto{\pgfqpoint{4.036256in}{0.626604in}}%
\pgfpathlineto{\pgfqpoint{4.036553in}{0.627477in}}%
\pgfpathlineto{\pgfqpoint{4.036851in}{0.628350in}}%
\pgfpathlineto{\pgfqpoint{4.037148in}{0.629223in}}%
\pgfpathlineto{\pgfqpoint{4.037446in}{0.630096in}}%
\pgfpathlineto{\pgfqpoint{4.037743in}{0.630968in}}%
\pgfpathlineto{\pgfqpoint{4.038041in}{0.631841in}}%
\pgfpathlineto{\pgfqpoint{4.038338in}{0.632714in}}%
\pgfpathlineto{\pgfqpoint{4.038636in}{0.633587in}}%
\pgfpathlineto{\pgfqpoint{4.038933in}{0.634459in}}%
\pgfpathlineto{\pgfqpoint{4.039231in}{0.635332in}}%
\pgfpathlineto{\pgfqpoint{4.039528in}{0.632761in}}%
\pgfpathlineto{\pgfqpoint{4.039826in}{0.634087in}}%
\pgfpathlineto{\pgfqpoint{4.040123in}{0.634077in}}%
\pgfpathlineto{\pgfqpoint{4.040421in}{0.634067in}}%
\pgfpathlineto{\pgfqpoint{4.040718in}{0.633878in}}%
\pgfpathlineto{\pgfqpoint{4.041016in}{0.632257in}}%
\pgfpathlineto{\pgfqpoint{4.041313in}{0.632256in}}%
\pgfpathlineto{\pgfqpoint{4.041611in}{0.632255in}}%
\pgfpathlineto{\pgfqpoint{4.041908in}{0.632253in}}%
\pgfpathlineto{\pgfqpoint{4.042205in}{0.632252in}}%
\pgfpathlineto{\pgfqpoint{4.042503in}{0.632251in}}%
\pgfpathlineto{\pgfqpoint{4.042800in}{0.632249in}}%
\pgfpathlineto{\pgfqpoint{4.043098in}{0.632248in}}%
\pgfpathlineto{\pgfqpoint{4.043395in}{0.632247in}}%
\pgfpathlineto{\pgfqpoint{4.043693in}{0.632245in}}%
\pgfpathlineto{\pgfqpoint{4.043990in}{0.632244in}}%
\pgfpathlineto{\pgfqpoint{4.044288in}{0.632243in}}%
\pgfpathlineto{\pgfqpoint{4.044585in}{0.632241in}}%
\pgfpathlineto{\pgfqpoint{4.044883in}{0.632240in}}%
\pgfpathlineto{\pgfqpoint{4.045180in}{0.632239in}}%
\pgfpathlineto{\pgfqpoint{4.045478in}{0.632237in}}%
\pgfpathlineto{\pgfqpoint{4.045775in}{0.632236in}}%
\pgfpathlineto{\pgfqpoint{4.046073in}{0.632235in}}%
\pgfpathlineto{\pgfqpoint{4.046370in}{0.632233in}}%
\pgfpathlineto{\pgfqpoint{4.046668in}{0.632232in}}%
\pgfpathlineto{\pgfqpoint{4.046965in}{0.632231in}}%
\pgfpathlineto{\pgfqpoint{4.047263in}{0.632229in}}%
\pgfpathlineto{\pgfqpoint{4.047560in}{0.632228in}}%
\pgfpathlineto{\pgfqpoint{4.047858in}{0.632227in}}%
\pgfpathlineto{\pgfqpoint{4.048155in}{0.632226in}}%
\pgfpathlineto{\pgfqpoint{4.048453in}{0.632224in}}%
\pgfpathlineto{\pgfqpoint{4.048750in}{0.632223in}}%
\pgfpathlineto{\pgfqpoint{4.049047in}{0.632222in}}%
\pgfpathlineto{\pgfqpoint{4.049345in}{0.632220in}}%
\pgfpathlineto{\pgfqpoint{4.049642in}{0.632219in}}%
\pgfpathlineto{\pgfqpoint{4.049940in}{0.632218in}}%
\pgfpathlineto{\pgfqpoint{4.050237in}{0.632216in}}%
\pgfpathlineto{\pgfqpoint{4.050535in}{0.632215in}}%
\pgfpathlineto{\pgfqpoint{4.050832in}{0.632214in}}%
\pgfpathlineto{\pgfqpoint{4.051130in}{0.632212in}}%
\pgfpathlineto{\pgfqpoint{4.051427in}{0.632211in}}%
\pgfpathlineto{\pgfqpoint{4.051725in}{0.632210in}}%
\pgfpathlineto{\pgfqpoint{4.052022in}{0.632208in}}%
\pgfpathlineto{\pgfqpoint{4.052320in}{0.632207in}}%
\pgfpathlineto{\pgfqpoint{4.052617in}{0.632206in}}%
\pgfpathlineto{\pgfqpoint{4.052915in}{0.632204in}}%
\pgfpathlineto{\pgfqpoint{4.053212in}{0.632203in}}%
\pgfpathlineto{\pgfqpoint{4.053510in}{0.632202in}}%
\pgfpathlineto{\pgfqpoint{4.053807in}{0.632200in}}%
\pgfpathlineto{\pgfqpoint{4.054105in}{0.632199in}}%
\pgfpathlineto{\pgfqpoint{4.054402in}{0.632198in}}%
\pgfpathlineto{\pgfqpoint{4.054700in}{0.632197in}}%
\pgfpathlineto{\pgfqpoint{4.054997in}{0.632195in}}%
\pgfpathlineto{\pgfqpoint{4.055295in}{0.632194in}}%
\pgfpathlineto{\pgfqpoint{4.055592in}{0.632193in}}%
\pgfpathlineto{\pgfqpoint{4.055889in}{0.632191in}}%
\pgfpathlineto{\pgfqpoint{4.056187in}{0.632190in}}%
\pgfpathlineto{\pgfqpoint{4.056484in}{0.632189in}}%
\pgfpathlineto{\pgfqpoint{4.056782in}{0.632187in}}%
\pgfpathlineto{\pgfqpoint{4.057079in}{0.632186in}}%
\pgfpathlineto{\pgfqpoint{4.057377in}{0.632185in}}%
\pgfpathlineto{\pgfqpoint{4.057674in}{0.632183in}}%
\pgfpathlineto{\pgfqpoint{4.057972in}{0.632182in}}%
\pgfpathlineto{\pgfqpoint{4.058269in}{0.632181in}}%
\pgfpathlineto{\pgfqpoint{4.058567in}{0.632179in}}%
\pgfpathlineto{\pgfqpoint{4.058864in}{0.632178in}}%
\pgfpathlineto{\pgfqpoint{4.059162in}{0.632177in}}%
\pgfpathlineto{\pgfqpoint{4.059459in}{0.632175in}}%
\pgfpathlineto{\pgfqpoint{4.059757in}{0.632174in}}%
\pgfpathlineto{\pgfqpoint{4.060054in}{0.632173in}}%
\pgfpathlineto{\pgfqpoint{4.060352in}{0.632172in}}%
\pgfpathlineto{\pgfqpoint{4.060649in}{0.632170in}}%
\pgfpathlineto{\pgfqpoint{4.060947in}{0.632169in}}%
\pgfpathlineto{\pgfqpoint{4.061244in}{0.632168in}}%
\pgfpathlineto{\pgfqpoint{4.061542in}{0.632166in}}%
\pgfpathlineto{\pgfqpoint{4.061839in}{0.632165in}}%
\pgfpathlineto{\pgfqpoint{4.062136in}{0.632164in}}%
\pgfpathlineto{\pgfqpoint{4.062434in}{0.632162in}}%
\pgfpathlineto{\pgfqpoint{4.062731in}{0.632161in}}%
\pgfpathlineto{\pgfqpoint{4.063029in}{0.632160in}}%
\pgfpathlineto{\pgfqpoint{4.063326in}{0.632158in}}%
\pgfpathlineto{\pgfqpoint{4.063624in}{0.632157in}}%
\pgfpathlineto{\pgfqpoint{4.063921in}{0.632156in}}%
\pgfpathlineto{\pgfqpoint{4.064219in}{0.632154in}}%
\pgfpathlineto{\pgfqpoint{4.064516in}{0.632153in}}%
\pgfpathlineto{\pgfqpoint{4.064814in}{0.632152in}}%
\pgfpathlineto{\pgfqpoint{4.065111in}{0.632150in}}%
\pgfpathlineto{\pgfqpoint{4.065409in}{0.632149in}}%
\pgfpathlineto{\pgfqpoint{4.065706in}{0.632148in}}%
\pgfpathlineto{\pgfqpoint{4.066004in}{0.632146in}}%
\pgfpathlineto{\pgfqpoint{4.066301in}{0.632145in}}%
\pgfpathlineto{\pgfqpoint{4.066599in}{0.632144in}}%
\pgfpathlineto{\pgfqpoint{4.066896in}{0.632143in}}%
\pgfpathlineto{\pgfqpoint{4.067194in}{0.632141in}}%
\pgfpathlineto{\pgfqpoint{4.067491in}{0.632140in}}%
\pgfpathlineto{\pgfqpoint{4.067789in}{0.632139in}}%
\pgfpathlineto{\pgfqpoint{4.068086in}{0.632137in}}%
\pgfpathlineto{\pgfqpoint{4.068384in}{0.632136in}}%
\pgfpathlineto{\pgfqpoint{4.068681in}{0.632135in}}%
\pgfpathlineto{\pgfqpoint{4.068978in}{0.632133in}}%
\pgfpathlineto{\pgfqpoint{4.069276in}{0.632132in}}%
\pgfpathlineto{\pgfqpoint{4.069573in}{0.632131in}}%
\pgfpathlineto{\pgfqpoint{4.069871in}{0.632129in}}%
\pgfpathlineto{\pgfqpoint{4.070168in}{0.632128in}}%
\pgfpathlineto{\pgfqpoint{4.070466in}{0.632127in}}%
\pgfpathlineto{\pgfqpoint{4.070763in}{0.632125in}}%
\pgfpathlineto{\pgfqpoint{4.071061in}{0.632124in}}%
\pgfpathlineto{\pgfqpoint{4.071358in}{0.632123in}}%
\pgfpathlineto{\pgfqpoint{4.071656in}{0.632121in}}%
\pgfpathlineto{\pgfqpoint{4.071953in}{0.632120in}}%
\pgfpathlineto{\pgfqpoint{4.072251in}{0.632119in}}%
\pgfpathlineto{\pgfqpoint{4.072548in}{0.632118in}}%
\pgfpathlineto{\pgfqpoint{4.072846in}{0.632116in}}%
\pgfpathlineto{\pgfqpoint{4.073143in}{0.632115in}}%
\pgfpathlineto{\pgfqpoint{4.073441in}{0.632114in}}%
\pgfpathlineto{\pgfqpoint{4.073738in}{0.632112in}}%
\pgfpathlineto{\pgfqpoint{4.074036in}{0.632111in}}%
\pgfpathlineto{\pgfqpoint{4.074333in}{0.632110in}}%
\pgfpathlineto{\pgfqpoint{4.074631in}{0.632108in}}%
\pgfpathlineto{\pgfqpoint{4.074928in}{0.632107in}}%
\pgfpathlineto{\pgfqpoint{4.075226in}{0.632106in}}%
\pgfpathlineto{\pgfqpoint{4.075523in}{0.632104in}}%
\pgfpathlineto{\pgfqpoint{4.075820in}{0.632103in}}%
\pgfpathlineto{\pgfqpoint{4.076118in}{0.632106in}}%
\pgfpathlineto{\pgfqpoint{4.076415in}{0.632133in}}%
\pgfpathlineto{\pgfqpoint{4.076713in}{0.632165in}}%
\pgfpathlineto{\pgfqpoint{4.077010in}{0.632196in}}%
\pgfpathlineto{\pgfqpoint{4.077308in}{0.632227in}}%
\pgfpathlineto{\pgfqpoint{4.077605in}{0.632259in}}%
\pgfpathlineto{\pgfqpoint{4.077903in}{0.632290in}}%
\pgfpathlineto{\pgfqpoint{4.078200in}{0.632321in}}%
\pgfpathlineto{\pgfqpoint{4.078498in}{0.632353in}}%
\pgfpathlineto{\pgfqpoint{4.078795in}{0.632384in}}%
\pgfpathlineto{\pgfqpoint{4.079093in}{0.632415in}}%
\pgfpathlineto{\pgfqpoint{4.079390in}{0.632447in}}%
\pgfpathlineto{\pgfqpoint{4.079688in}{0.632478in}}%
\pgfpathlineto{\pgfqpoint{4.079985in}{0.632509in}}%
\pgfpathlineto{\pgfqpoint{4.080283in}{0.632540in}}%
\pgfpathlineto{\pgfqpoint{4.080580in}{0.632546in}}%
\pgfpathlineto{\pgfqpoint{4.080878in}{0.632537in}}%
\pgfpathlineto{\pgfqpoint{4.081175in}{0.632528in}}%
\pgfpathlineto{\pgfqpoint{4.081473in}{0.632518in}}%
\pgfpathlineto{\pgfqpoint{4.081770in}{0.632509in}}%
\pgfpathlineto{\pgfqpoint{4.082067in}{0.632500in}}%
\pgfpathlineto{\pgfqpoint{4.082365in}{0.632491in}}%
\pgfpathlineto{\pgfqpoint{4.082662in}{0.632482in}}%
\pgfpathlineto{\pgfqpoint{4.082960in}{0.632473in}}%
\pgfpathlineto{\pgfqpoint{4.083257in}{0.632463in}}%
\pgfpathlineto{\pgfqpoint{4.083555in}{0.632454in}}%
\pgfpathlineto{\pgfqpoint{4.083852in}{0.632445in}}%
\pgfpathlineto{\pgfqpoint{4.084150in}{0.632436in}}%
\pgfpathlineto{\pgfqpoint{4.084447in}{0.632427in}}%
\pgfpathlineto{\pgfqpoint{4.084745in}{0.632417in}}%
\pgfpathlineto{\pgfqpoint{4.085042in}{0.632408in}}%
\pgfpathlineto{\pgfqpoint{4.085340in}{0.632399in}}%
\pgfpathlineto{\pgfqpoint{4.085637in}{0.632390in}}%
\pgfpathlineto{\pgfqpoint{4.085935in}{0.632381in}}%
\pgfpathlineto{\pgfqpoint{4.086232in}{0.632371in}}%
\pgfpathlineto{\pgfqpoint{4.086530in}{0.632362in}}%
\pgfpathlineto{\pgfqpoint{4.086827in}{0.632353in}}%
\pgfpathlineto{\pgfqpoint{4.087125in}{0.632344in}}%
\pgfpathlineto{\pgfqpoint{4.087422in}{0.632335in}}%
\pgfpathlineto{\pgfqpoint{4.087720in}{0.632325in}}%
\pgfpathlineto{\pgfqpoint{4.088017in}{0.632316in}}%
\pgfpathlineto{\pgfqpoint{4.088315in}{0.632307in}}%
\pgfpathlineto{\pgfqpoint{4.088612in}{0.632298in}}%
\pgfpathlineto{\pgfqpoint{4.088909in}{0.632289in}}%
\pgfpathlineto{\pgfqpoint{4.089207in}{0.632280in}}%
\pgfpathlineto{\pgfqpoint{4.089504in}{0.632270in}}%
\pgfpathlineto{\pgfqpoint{4.089802in}{0.632261in}}%
\pgfpathlineto{\pgfqpoint{4.090099in}{0.632252in}}%
\pgfpathlineto{\pgfqpoint{4.090397in}{0.632243in}}%
\pgfpathlineto{\pgfqpoint{4.090694in}{0.632234in}}%
\pgfpathlineto{\pgfqpoint{4.090992in}{0.632224in}}%
\pgfpathlineto{\pgfqpoint{4.091289in}{0.632215in}}%
\pgfpathlineto{\pgfqpoint{4.091587in}{0.632206in}}%
\pgfpathlineto{\pgfqpoint{4.091884in}{0.632197in}}%
\pgfpathlineto{\pgfqpoint{4.092182in}{0.632188in}}%
\pgfpathlineto{\pgfqpoint{4.092479in}{0.632178in}}%
\pgfpathlineto{\pgfqpoint{4.092777in}{0.632169in}}%
\pgfpathlineto{\pgfqpoint{4.093074in}{0.632160in}}%
\pgfpathlineto{\pgfqpoint{4.093372in}{0.632151in}}%
\pgfpathlineto{\pgfqpoint{4.093669in}{0.632142in}}%
\pgfpathlineto{\pgfqpoint{4.093967in}{0.632133in}}%
\pgfpathlineto{\pgfqpoint{4.094264in}{0.632123in}}%
\pgfpathlineto{\pgfqpoint{4.094562in}{0.632114in}}%
\pgfpathlineto{\pgfqpoint{4.094859in}{0.632105in}}%
\pgfpathlineto{\pgfqpoint{4.095157in}{0.632096in}}%
\pgfpathlineto{\pgfqpoint{4.095454in}{0.632087in}}%
\pgfpathlineto{\pgfqpoint{4.095751in}{0.632077in}}%
\pgfpathlineto{\pgfqpoint{4.096049in}{0.632068in}}%
\pgfpathlineto{\pgfqpoint{4.096346in}{0.632059in}}%
\pgfpathlineto{\pgfqpoint{4.096644in}{0.632050in}}%
\pgfpathlineto{\pgfqpoint{4.096941in}{0.632041in}}%
\pgfpathlineto{\pgfqpoint{4.097239in}{0.632031in}}%
\pgfpathlineto{\pgfqpoint{4.097536in}{0.632022in}}%
\pgfpathlineto{\pgfqpoint{4.097834in}{0.632013in}}%
\pgfpathlineto{\pgfqpoint{4.098131in}{0.632005in}}%
\pgfpathlineto{\pgfqpoint{4.098429in}{0.632003in}}%
\pgfpathlineto{\pgfqpoint{4.098726in}{0.632002in}}%
\pgfpathlineto{\pgfqpoint{4.099024in}{0.632000in}}%
\pgfpathlineto{\pgfqpoint{4.099321in}{0.631999in}}%
\pgfpathlineto{\pgfqpoint{4.099619in}{0.631998in}}%
\pgfpathlineto{\pgfqpoint{4.099916in}{0.631996in}}%
\pgfpathlineto{\pgfqpoint{4.100214in}{0.631995in}}%
\pgfpathlineto{\pgfqpoint{4.100511in}{0.631994in}}%
\pgfpathlineto{\pgfqpoint{4.100809in}{0.631992in}}%
\pgfpathlineto{\pgfqpoint{4.101106in}{0.631991in}}%
\pgfpathlineto{\pgfqpoint{4.101404in}{0.631990in}}%
\pgfpathlineto{\pgfqpoint{4.101701in}{0.631988in}}%
\pgfpathlineto{\pgfqpoint{4.101998in}{0.631987in}}%
\pgfpathlineto{\pgfqpoint{4.102296in}{0.631986in}}%
\pgfpathlineto{\pgfqpoint{4.102593in}{0.631984in}}%
\pgfpathlineto{\pgfqpoint{4.102891in}{0.631983in}}%
\pgfpathlineto{\pgfqpoint{4.103188in}{0.631982in}}%
\pgfpathlineto{\pgfqpoint{4.103486in}{0.631981in}}%
\pgfpathlineto{\pgfqpoint{4.103783in}{0.631979in}}%
\pgfpathlineto{\pgfqpoint{4.104081in}{0.631978in}}%
\pgfpathlineto{\pgfqpoint{4.104378in}{0.631977in}}%
\pgfpathlineto{\pgfqpoint{4.104676in}{0.631975in}}%
\pgfpathlineto{\pgfqpoint{4.104973in}{0.631974in}}%
\pgfpathlineto{\pgfqpoint{4.105271in}{0.631973in}}%
\pgfpathlineto{\pgfqpoint{4.105568in}{0.631971in}}%
\pgfpathlineto{\pgfqpoint{4.105866in}{0.631970in}}%
\pgfpathlineto{\pgfqpoint{4.106163in}{0.631969in}}%
\pgfpathlineto{\pgfqpoint{4.106461in}{0.631967in}}%
\pgfpathlineto{\pgfqpoint{4.106758in}{0.631966in}}%
\pgfpathlineto{\pgfqpoint{4.107056in}{0.631965in}}%
\pgfpathlineto{\pgfqpoint{4.107353in}{0.631963in}}%
\pgfpathlineto{\pgfqpoint{4.107651in}{0.631962in}}%
\pgfpathlineto{\pgfqpoint{4.107948in}{0.631961in}}%
\pgfpathlineto{\pgfqpoint{4.108246in}{0.631959in}}%
\pgfpathlineto{\pgfqpoint{4.108543in}{0.631958in}}%
\pgfpathlineto{\pgfqpoint{4.108840in}{0.631957in}}%
\pgfpathlineto{\pgfqpoint{4.109138in}{0.631955in}}%
\pgfpathlineto{\pgfqpoint{4.109435in}{0.631954in}}%
\pgfpathlineto{\pgfqpoint{4.109733in}{0.631953in}}%
\pgfpathlineto{\pgfqpoint{4.110030in}{0.631952in}}%
\pgfpathlineto{\pgfqpoint{4.110328in}{0.631950in}}%
\pgfpathlineto{\pgfqpoint{4.110625in}{0.631949in}}%
\pgfpathlineto{\pgfqpoint{4.110923in}{0.631948in}}%
\pgfpathlineto{\pgfqpoint{4.111220in}{0.631946in}}%
\pgfpathlineto{\pgfqpoint{4.111518in}{0.631945in}}%
\pgfpathlineto{\pgfqpoint{4.111815in}{0.631944in}}%
\pgfpathlineto{\pgfqpoint{4.112113in}{0.631942in}}%
\pgfpathlineto{\pgfqpoint{4.112410in}{0.631941in}}%
\pgfpathlineto{\pgfqpoint{4.112708in}{0.631940in}}%
\pgfpathlineto{\pgfqpoint{4.113005in}{0.631938in}}%
\pgfpathlineto{\pgfqpoint{4.113303in}{0.631937in}}%
\pgfpathlineto{\pgfqpoint{4.113600in}{0.631936in}}%
\pgfpathlineto{\pgfqpoint{4.113898in}{0.631934in}}%
\pgfpathlineto{\pgfqpoint{4.114195in}{0.631933in}}%
\pgfpathlineto{\pgfqpoint{4.114493in}{0.631932in}}%
\pgfpathlineto{\pgfqpoint{4.114790in}{0.631930in}}%
\pgfpathlineto{\pgfqpoint{4.115088in}{0.631929in}}%
\pgfpathlineto{\pgfqpoint{4.115385in}{0.631928in}}%
\pgfpathlineto{\pgfqpoint{4.115682in}{0.631926in}}%
\pgfpathlineto{\pgfqpoint{4.115980in}{0.631925in}}%
\pgfpathlineto{\pgfqpoint{4.116277in}{0.631924in}}%
\pgfpathlineto{\pgfqpoint{4.116575in}{0.631923in}}%
\pgfpathlineto{\pgfqpoint{4.116872in}{0.631921in}}%
\pgfpathlineto{\pgfqpoint{4.117170in}{0.631920in}}%
\pgfpathlineto{\pgfqpoint{4.117467in}{0.631919in}}%
\pgfpathlineto{\pgfqpoint{4.117765in}{0.631917in}}%
\pgfpathlineto{\pgfqpoint{4.118062in}{0.631916in}}%
\pgfpathlineto{\pgfqpoint{4.118360in}{0.631915in}}%
\pgfpathlineto{\pgfqpoint{4.118657in}{0.631913in}}%
\pgfpathlineto{\pgfqpoint{4.118955in}{0.631912in}}%
\pgfpathlineto{\pgfqpoint{4.119252in}{0.631911in}}%
\pgfpathlineto{\pgfqpoint{4.119550in}{0.631909in}}%
\pgfpathlineto{\pgfqpoint{4.119847in}{0.631908in}}%
\pgfpathlineto{\pgfqpoint{4.120145in}{0.631907in}}%
\pgfpathlineto{\pgfqpoint{4.120442in}{0.631905in}}%
\pgfpathlineto{\pgfqpoint{4.120740in}{0.631904in}}%
\pgfpathlineto{\pgfqpoint{4.121037in}{0.631903in}}%
\pgfpathlineto{\pgfqpoint{4.121335in}{0.631901in}}%
\pgfpathlineto{\pgfqpoint{4.121632in}{0.631900in}}%
\pgfpathlineto{\pgfqpoint{4.121929in}{0.631899in}}%
\pgfpathlineto{\pgfqpoint{4.122227in}{0.631898in}}%
\pgfpathlineto{\pgfqpoint{4.122524in}{0.631896in}}%
\pgfpathlineto{\pgfqpoint{4.122822in}{0.631895in}}%
\pgfpathlineto{\pgfqpoint{4.123119in}{0.631894in}}%
\pgfpathlineto{\pgfqpoint{4.123417in}{0.631892in}}%
\pgfpathlineto{\pgfqpoint{4.123714in}{0.631891in}}%
\pgfpathlineto{\pgfqpoint{4.124012in}{0.631890in}}%
\pgfpathlineto{\pgfqpoint{4.124309in}{0.631888in}}%
\pgfpathlineto{\pgfqpoint{4.124607in}{0.631887in}}%
\pgfpathlineto{\pgfqpoint{4.124904in}{0.631886in}}%
\pgfpathlineto{\pgfqpoint{4.125202in}{0.631884in}}%
\pgfpathlineto{\pgfqpoint{4.125499in}{0.631883in}}%
\pgfpathlineto{\pgfqpoint{4.125797in}{0.631882in}}%
\pgfpathlineto{\pgfqpoint{4.126094in}{0.631880in}}%
\pgfpathlineto{\pgfqpoint{4.126392in}{0.631879in}}%
\pgfpathlineto{\pgfqpoint{4.126689in}{0.631878in}}%
\pgfpathlineto{\pgfqpoint{4.126987in}{0.631887in}}%
\pgfpathlineto{\pgfqpoint{4.127284in}{0.631898in}}%
\pgfpathlineto{\pgfqpoint{4.127582in}{0.631910in}}%
\pgfpathlineto{\pgfqpoint{4.127879in}{0.631921in}}%
\pgfpathlineto{\pgfqpoint{4.128177in}{0.631932in}}%
\pgfpathlineto{\pgfqpoint{4.128474in}{0.631944in}}%
\pgfpathlineto{\pgfqpoint{4.128771in}{0.631955in}}%
\pgfpathlineto{\pgfqpoint{4.129069in}{0.631967in}}%
\pgfpathlineto{\pgfqpoint{4.129366in}{0.631978in}}%
\pgfpathlineto{\pgfqpoint{4.129664in}{0.631990in}}%
\pgfpathlineto{\pgfqpoint{4.129961in}{0.632001in}}%
\pgfpathlineto{\pgfqpoint{4.130259in}{0.632013in}}%
\pgfpathlineto{\pgfqpoint{4.130556in}{0.632024in}}%
\pgfpathlineto{\pgfqpoint{4.130854in}{0.632035in}}%
\pgfpathlineto{\pgfqpoint{4.131151in}{0.632047in}}%
\pgfpathlineto{\pgfqpoint{4.131449in}{0.632058in}}%
\pgfpathlineto{\pgfqpoint{4.131746in}{0.632070in}}%
\pgfpathlineto{\pgfqpoint{4.132044in}{0.632081in}}%
\pgfpathlineto{\pgfqpoint{4.132341in}{0.632093in}}%
\pgfpathlineto{\pgfqpoint{4.132639in}{0.632104in}}%
\pgfpathlineto{\pgfqpoint{4.132936in}{0.632115in}}%
\pgfpathlineto{\pgfqpoint{4.133234in}{0.632127in}}%
\pgfpathlineto{\pgfqpoint{4.133531in}{0.632138in}}%
\pgfpathlineto{\pgfqpoint{4.133829in}{0.632150in}}%
\pgfpathlineto{\pgfqpoint{4.134126in}{0.632161in}}%
\pgfpathlineto{\pgfqpoint{4.134424in}{0.632173in}}%
\pgfpathlineto{\pgfqpoint{4.134721in}{0.632184in}}%
\pgfpathlineto{\pgfqpoint{4.135019in}{0.632196in}}%
\pgfpathlineto{\pgfqpoint{4.135316in}{0.632207in}}%
\pgfpathlineto{\pgfqpoint{4.135613in}{0.632218in}}%
\pgfpathlineto{\pgfqpoint{4.135911in}{0.632230in}}%
\pgfpathlineto{\pgfqpoint{4.136208in}{0.632241in}}%
\pgfpathlineto{\pgfqpoint{4.136506in}{0.632253in}}%
\pgfpathlineto{\pgfqpoint{4.136803in}{0.632264in}}%
\pgfpathlineto{\pgfqpoint{4.137101in}{0.632276in}}%
\pgfpathlineto{\pgfqpoint{4.137398in}{0.632287in}}%
\pgfpathlineto{\pgfqpoint{4.137696in}{0.632298in}}%
\pgfpathlineto{\pgfqpoint{4.137993in}{0.632310in}}%
\pgfpathlineto{\pgfqpoint{4.138291in}{0.632321in}}%
\pgfpathlineto{\pgfqpoint{4.138588in}{0.632333in}}%
\pgfpathlineto{\pgfqpoint{4.138886in}{0.632344in}}%
\pgfpathlineto{\pgfqpoint{4.139183in}{0.632356in}}%
\pgfpathlineto{\pgfqpoint{4.139481in}{0.632367in}}%
\pgfpathlineto{\pgfqpoint{4.139778in}{0.632379in}}%
\pgfpathlineto{\pgfqpoint{4.140076in}{0.632390in}}%
\pgfpathlineto{\pgfqpoint{4.140373in}{0.632401in}}%
\pgfpathlineto{\pgfqpoint{4.140671in}{0.632413in}}%
\pgfpathlineto{\pgfqpoint{4.140968in}{0.632424in}}%
\pgfpathlineto{\pgfqpoint{4.141266in}{0.632329in}}%
\pgfpathlineto{\pgfqpoint{4.141563in}{0.631931in}}%
\pgfpathlineto{\pgfqpoint{4.141860in}{0.631635in}}%
\pgfpathlineto{\pgfqpoint{4.142158in}{0.631735in}}%
\pgfpathlineto{\pgfqpoint{4.142455in}{0.631989in}}%
\pgfpathlineto{\pgfqpoint{4.142753in}{0.632244in}}%
\pgfpathlineto{\pgfqpoint{4.143050in}{0.632498in}}%
\pgfpathlineto{\pgfqpoint{4.143348in}{0.632753in}}%
\pgfpathlineto{\pgfqpoint{4.143645in}{0.633008in}}%
\pgfpathlineto{\pgfqpoint{4.143943in}{0.633262in}}%
\pgfpathlineto{\pgfqpoint{4.144240in}{0.633517in}}%
\pgfpathlineto{\pgfqpoint{4.144538in}{0.633771in}}%
\pgfpathlineto{\pgfqpoint{4.144835in}{0.634026in}}%
\pgfpathlineto{\pgfqpoint{4.145133in}{0.634280in}}%
\pgfpathlineto{\pgfqpoint{4.145430in}{0.634535in}}%
\pgfpathlineto{\pgfqpoint{4.145728in}{0.634526in}}%
\pgfpathlineto{\pgfqpoint{4.146025in}{0.634187in}}%
\pgfpathlineto{\pgfqpoint{4.146323in}{0.633847in}}%
\pgfpathlineto{\pgfqpoint{4.146620in}{0.633506in}}%
\pgfpathlineto{\pgfqpoint{4.146918in}{0.633166in}}%
\pgfpathlineto{\pgfqpoint{4.147215in}{0.632826in}}%
\pgfpathlineto{\pgfqpoint{4.147513in}{0.632486in}}%
\pgfpathlineto{\pgfqpoint{4.147810in}{0.632689in}}%
\pgfpathlineto{\pgfqpoint{4.148108in}{0.634420in}}%
\pgfpathlineto{\pgfqpoint{4.148405in}{0.635279in}}%
\pgfpathlineto{\pgfqpoint{4.148702in}{0.635264in}}%
\pgfpathlineto{\pgfqpoint{4.149000in}{0.635249in}}%
\pgfpathlineto{\pgfqpoint{4.149297in}{0.635234in}}%
\pgfpathlineto{\pgfqpoint{4.149595in}{0.635219in}}%
\pgfpathlineto{\pgfqpoint{4.149892in}{0.635203in}}%
\pgfpathlineto{\pgfqpoint{4.150190in}{0.635188in}}%
\pgfpathlineto{\pgfqpoint{4.150487in}{0.635173in}}%
\pgfpathlineto{\pgfqpoint{4.150785in}{0.635158in}}%
\pgfpathlineto{\pgfqpoint{4.151082in}{0.635143in}}%
\pgfpathlineto{\pgfqpoint{4.151380in}{0.635128in}}%
\pgfpathlineto{\pgfqpoint{4.151677in}{0.635113in}}%
\pgfpathlineto{\pgfqpoint{4.151975in}{0.635097in}}%
\pgfpathlineto{\pgfqpoint{4.152272in}{0.635082in}}%
\pgfpathlineto{\pgfqpoint{4.152570in}{0.635454in}}%
\pgfpathlineto{\pgfqpoint{4.152867in}{0.635579in}}%
\pgfpathlineto{\pgfqpoint{4.153165in}{0.635578in}}%
\pgfpathlineto{\pgfqpoint{4.153462in}{0.635577in}}%
\pgfpathlineto{\pgfqpoint{4.153760in}{0.635575in}}%
\pgfpathlineto{\pgfqpoint{4.154057in}{0.635574in}}%
\pgfpathlineto{\pgfqpoint{4.154355in}{0.635573in}}%
\pgfpathlineto{\pgfqpoint{4.154652in}{0.635571in}}%
\pgfpathlineto{\pgfqpoint{4.154950in}{0.635570in}}%
\pgfpathlineto{\pgfqpoint{4.155247in}{0.635569in}}%
\pgfpathlineto{\pgfqpoint{4.155544in}{0.635567in}}%
\pgfpathlineto{\pgfqpoint{4.155842in}{0.635561in}}%
\pgfpathlineto{\pgfqpoint{4.156139in}{0.634899in}}%
\pgfpathlineto{\pgfqpoint{4.156437in}{0.634799in}}%
\pgfpathlineto{\pgfqpoint{4.156734in}{0.634854in}}%
\pgfpathlineto{\pgfqpoint{4.157032in}{0.634908in}}%
\pgfpathlineto{\pgfqpoint{4.157329in}{0.634963in}}%
\pgfpathlineto{\pgfqpoint{4.157627in}{0.635017in}}%
\pgfpathlineto{\pgfqpoint{4.157924in}{0.635071in}}%
\pgfpathlineto{\pgfqpoint{4.158222in}{0.635126in}}%
\pgfpathlineto{\pgfqpoint{4.158519in}{0.635180in}}%
\pgfpathlineto{\pgfqpoint{4.158817in}{0.635234in}}%
\pgfpathlineto{\pgfqpoint{4.159114in}{0.635289in}}%
\pgfpathlineto{\pgfqpoint{4.159412in}{0.635343in}}%
\pgfpathlineto{\pgfqpoint{4.159709in}{0.635397in}}%
\pgfpathlineto{\pgfqpoint{4.160007in}{0.635452in}}%
\pgfpathlineto{\pgfqpoint{4.160304in}{0.635506in}}%
\pgfpathlineto{\pgfqpoint{4.160602in}{0.635560in}}%
\pgfpathlineto{\pgfqpoint{4.160899in}{0.635615in}}%
\pgfpathlineto{\pgfqpoint{4.161197in}{0.635669in}}%
\pgfpathlineto{\pgfqpoint{4.161494in}{0.635723in}}%
\pgfpathlineto{\pgfqpoint{4.161791in}{0.635778in}}%
\pgfpathlineto{\pgfqpoint{4.162089in}{0.635832in}}%
\pgfpathlineto{\pgfqpoint{4.162386in}{0.635886in}}%
\pgfpathlineto{\pgfqpoint{4.162684in}{0.635941in}}%
\pgfpathlineto{\pgfqpoint{4.162981in}{0.635995in}}%
\pgfpathlineto{\pgfqpoint{4.163279in}{0.636049in}}%
\pgfpathlineto{\pgfqpoint{4.163576in}{0.636104in}}%
\pgfpathlineto{\pgfqpoint{4.163874in}{0.636158in}}%
\pgfpathlineto{\pgfqpoint{4.164171in}{0.636212in}}%
\pgfpathlineto{\pgfqpoint{4.164469in}{0.636267in}}%
\pgfpathlineto{\pgfqpoint{4.164766in}{0.636321in}}%
\pgfpathlineto{\pgfqpoint{4.165064in}{0.636375in}}%
\pgfpathlineto{\pgfqpoint{4.165361in}{0.636430in}}%
\pgfpathlineto{\pgfqpoint{4.165659in}{0.636484in}}%
\pgfpathlineto{\pgfqpoint{4.165956in}{0.636538in}}%
\pgfpathlineto{\pgfqpoint{4.166254in}{0.636593in}}%
\pgfpathlineto{\pgfqpoint{4.166551in}{0.636647in}}%
\pgfpathlineto{\pgfqpoint{4.166849in}{0.636701in}}%
\pgfpathlineto{\pgfqpoint{4.167146in}{0.636756in}}%
\pgfpathlineto{\pgfqpoint{4.167444in}{0.636810in}}%
\pgfpathlineto{\pgfqpoint{4.167741in}{0.636864in}}%
\pgfpathlineto{\pgfqpoint{4.168039in}{0.636919in}}%
\pgfpathlineto{\pgfqpoint{4.168336in}{0.636973in}}%
\pgfpathlineto{\pgfqpoint{4.168633in}{0.637027in}}%
\pgfpathlineto{\pgfqpoint{4.168931in}{0.637082in}}%
\pgfpathlineto{\pgfqpoint{4.169228in}{0.637136in}}%
\pgfpathlineto{\pgfqpoint{4.169526in}{0.637191in}}%
\pgfpathlineto{\pgfqpoint{4.169823in}{0.637245in}}%
\pgfpathlineto{\pgfqpoint{4.170121in}{0.637299in}}%
\pgfpathlineto{\pgfqpoint{4.170418in}{0.637354in}}%
\pgfpathlineto{\pgfqpoint{4.170716in}{0.637408in}}%
\pgfpathlineto{\pgfqpoint{4.171013in}{0.637462in}}%
\pgfpathlineto{\pgfqpoint{4.171311in}{0.637517in}}%
\pgfpathlineto{\pgfqpoint{4.171608in}{0.637571in}}%
\pgfpathlineto{\pgfqpoint{4.171906in}{0.637625in}}%
\pgfpathlineto{\pgfqpoint{4.172203in}{0.637680in}}%
\pgfpathlineto{\pgfqpoint{4.172501in}{0.637734in}}%
\pgfpathlineto{\pgfqpoint{4.172798in}{0.637788in}}%
\pgfpathlineto{\pgfqpoint{4.173096in}{0.637843in}}%
\pgfpathlineto{\pgfqpoint{4.173393in}{0.637897in}}%
\pgfpathlineto{\pgfqpoint{4.173691in}{0.637951in}}%
\pgfpathlineto{\pgfqpoint{4.173988in}{0.638006in}}%
\pgfpathlineto{\pgfqpoint{4.174286in}{0.638060in}}%
\pgfpathlineto{\pgfqpoint{4.174583in}{0.638114in}}%
\pgfpathlineto{\pgfqpoint{4.174881in}{0.638169in}}%
\pgfpathlineto{\pgfqpoint{4.175178in}{0.638223in}}%
\pgfpathlineto{\pgfqpoint{4.175475in}{0.638277in}}%
\pgfpathlineto{\pgfqpoint{4.175773in}{0.638332in}}%
\pgfpathlineto{\pgfqpoint{4.176070in}{0.638386in}}%
\pgfpathlineto{\pgfqpoint{4.176368in}{0.638440in}}%
\pgfpathlineto{\pgfqpoint{4.176665in}{0.638495in}}%
\pgfpathlineto{\pgfqpoint{4.176963in}{0.638549in}}%
\pgfpathlineto{\pgfqpoint{4.177260in}{0.638603in}}%
\pgfpathlineto{\pgfqpoint{4.177558in}{0.638658in}}%
\pgfpathlineto{\pgfqpoint{4.177855in}{0.638712in}}%
\pgfpathlineto{\pgfqpoint{4.178153in}{0.638766in}}%
\pgfpathlineto{\pgfqpoint{4.178450in}{0.638821in}}%
\pgfpathlineto{\pgfqpoint{4.178748in}{0.638875in}}%
\pgfpathlineto{\pgfqpoint{4.179045in}{0.638929in}}%
\pgfpathlineto{\pgfqpoint{4.179343in}{0.638984in}}%
\pgfpathlineto{\pgfqpoint{4.179640in}{0.639038in}}%
\pgfpathlineto{\pgfqpoint{4.179938in}{0.639092in}}%
\pgfpathlineto{\pgfqpoint{4.180235in}{0.639147in}}%
\pgfpathlineto{\pgfqpoint{4.180533in}{0.639201in}}%
\pgfpathlineto{\pgfqpoint{4.180830in}{0.639255in}}%
\pgfpathlineto{\pgfqpoint{4.181128in}{0.639310in}}%
\pgfpathlineto{\pgfqpoint{4.181425in}{0.639364in}}%
\pgfpathlineto{\pgfqpoint{4.181723in}{0.639419in}}%
\pgfpathlineto{\pgfqpoint{4.182020in}{0.639473in}}%
\pgfpathlineto{\pgfqpoint{4.182317in}{0.639527in}}%
\pgfpathlineto{\pgfqpoint{4.182615in}{0.639582in}}%
\pgfpathlineto{\pgfqpoint{4.182912in}{0.639636in}}%
\pgfpathlineto{\pgfqpoint{4.183210in}{0.639690in}}%
\pgfpathlineto{\pgfqpoint{4.183507in}{0.639745in}}%
\pgfpathlineto{\pgfqpoint{4.183805in}{0.639799in}}%
\pgfpathlineto{\pgfqpoint{4.184102in}{0.639853in}}%
\pgfpathlineto{\pgfqpoint{4.184400in}{0.639908in}}%
\pgfpathlineto{\pgfqpoint{4.184697in}{0.639962in}}%
\pgfpathlineto{\pgfqpoint{4.184995in}{0.640016in}}%
\pgfpathlineto{\pgfqpoint{4.185292in}{0.640071in}}%
\pgfpathlineto{\pgfqpoint{4.185590in}{0.640125in}}%
\pgfpathlineto{\pgfqpoint{4.185887in}{0.640179in}}%
\pgfpathlineto{\pgfqpoint{4.186185in}{0.640234in}}%
\pgfpathlineto{\pgfqpoint{4.186482in}{0.640288in}}%
\pgfpathlineto{\pgfqpoint{4.186780in}{0.640342in}}%
\pgfpathlineto{\pgfqpoint{4.187077in}{0.640397in}}%
\pgfpathlineto{\pgfqpoint{4.187375in}{0.640451in}}%
\pgfpathlineto{\pgfqpoint{4.187672in}{0.640505in}}%
\pgfpathlineto{\pgfqpoint{4.187970in}{0.640560in}}%
\pgfpathlineto{\pgfqpoint{4.188267in}{0.640614in}}%
\pgfpathlineto{\pgfqpoint{4.188564in}{0.640668in}}%
\pgfpathlineto{\pgfqpoint{4.188862in}{0.640723in}}%
\pgfpathlineto{\pgfqpoint{4.189159in}{0.640777in}}%
\pgfpathlineto{\pgfqpoint{4.189457in}{0.640831in}}%
\pgfpathlineto{\pgfqpoint{4.189754in}{0.640886in}}%
\pgfpathlineto{\pgfqpoint{4.190052in}{0.640940in}}%
\pgfpathlineto{\pgfqpoint{4.190349in}{0.640994in}}%
\pgfpathlineto{\pgfqpoint{4.190647in}{0.641049in}}%
\pgfpathlineto{\pgfqpoint{4.190944in}{0.641103in}}%
\pgfpathlineto{\pgfqpoint{4.191242in}{0.641066in}}%
\pgfpathlineto{\pgfqpoint{4.191539in}{0.640921in}}%
\pgfpathlineto{\pgfqpoint{4.191837in}{0.640776in}}%
\pgfpathlineto{\pgfqpoint{4.192134in}{0.640631in}}%
\pgfpathlineto{\pgfqpoint{4.192432in}{0.640486in}}%
\pgfpathlineto{\pgfqpoint{4.192729in}{0.640341in}}%
\pgfpathlineto{\pgfqpoint{4.193027in}{0.640196in}}%
\pgfpathlineto{\pgfqpoint{4.193324in}{0.640051in}}%
\pgfpathlineto{\pgfqpoint{4.193622in}{0.639905in}}%
\pgfpathlineto{\pgfqpoint{4.193919in}{0.639760in}}%
\pgfpathlineto{\pgfqpoint{4.194217in}{0.639615in}}%
\pgfpathlineto{\pgfqpoint{4.194514in}{0.639470in}}%
\pgfpathlineto{\pgfqpoint{4.194812in}{0.639325in}}%
\pgfpathlineto{\pgfqpoint{4.195109in}{0.639180in}}%
\pgfpathlineto{\pgfqpoint{4.195406in}{0.639035in}}%
\pgfpathlineto{\pgfqpoint{4.195704in}{0.638890in}}%
\pgfpathlineto{\pgfqpoint{4.196001in}{0.638745in}}%
\pgfpathlineto{\pgfqpoint{4.196299in}{0.638600in}}%
\pgfpathlineto{\pgfqpoint{4.196596in}{0.638455in}}%
\pgfpathlineto{\pgfqpoint{4.196894in}{0.638310in}}%
\pgfpathlineto{\pgfqpoint{4.197191in}{0.638165in}}%
\pgfpathlineto{\pgfqpoint{4.197489in}{0.638019in}}%
\pgfpathlineto{\pgfqpoint{4.197786in}{0.637874in}}%
\pgfpathlineto{\pgfqpoint{4.198084in}{0.637729in}}%
\pgfpathlineto{\pgfqpoint{4.198381in}{0.637584in}}%
\pgfpathlineto{\pgfqpoint{4.198679in}{0.637439in}}%
\pgfpathlineto{\pgfqpoint{4.198976in}{0.637294in}}%
\pgfpathlineto{\pgfqpoint{4.199274in}{0.637149in}}%
\pgfpathlineto{\pgfqpoint{4.199571in}{0.637004in}}%
\pgfpathlineto{\pgfqpoint{4.199869in}{0.636859in}}%
\pgfpathlineto{\pgfqpoint{4.200166in}{0.636714in}}%
\pgfpathlineto{\pgfqpoint{4.200464in}{0.636569in}}%
\pgfpathlineto{\pgfqpoint{4.200761in}{0.636424in}}%
\pgfpathlineto{\pgfqpoint{4.201059in}{0.636279in}}%
\pgfpathlineto{\pgfqpoint{4.201356in}{0.636133in}}%
\pgfpathlineto{\pgfqpoint{4.201654in}{0.635988in}}%
\pgfpathlineto{\pgfqpoint{4.201951in}{0.635843in}}%
\pgfpathlineto{\pgfqpoint{4.202248in}{0.635698in}}%
\pgfpathlineto{\pgfqpoint{4.202546in}{0.635553in}}%
\pgfpathlineto{\pgfqpoint{4.202843in}{0.635408in}}%
\pgfpathlineto{\pgfqpoint{4.203141in}{0.635263in}}%
\pgfpathlineto{\pgfqpoint{4.203438in}{0.635118in}}%
\pgfpathlineto{\pgfqpoint{4.203736in}{0.634973in}}%
\pgfpathlineto{\pgfqpoint{4.204033in}{0.634828in}}%
\pgfpathlineto{\pgfqpoint{4.204331in}{0.634683in}}%
\pgfpathlineto{\pgfqpoint{4.204628in}{0.634538in}}%
\pgfpathlineto{\pgfqpoint{4.204926in}{0.634392in}}%
\pgfpathlineto{\pgfqpoint{4.205223in}{0.634247in}}%
\pgfpathlineto{\pgfqpoint{4.205521in}{0.634102in}}%
\pgfpathlineto{\pgfqpoint{4.205818in}{0.633957in}}%
\pgfpathlineto{\pgfqpoint{4.206116in}{0.633812in}}%
\pgfpathlineto{\pgfqpoint{4.206413in}{0.633721in}}%
\pgfpathlineto{\pgfqpoint{4.206711in}{0.633792in}}%
\pgfpathlineto{\pgfqpoint{4.207008in}{0.633874in}}%
\pgfpathlineto{\pgfqpoint{4.207306in}{0.633956in}}%
\pgfpathlineto{\pgfqpoint{4.207603in}{0.634038in}}%
\pgfpathlineto{\pgfqpoint{4.207901in}{0.634119in}}%
\pgfpathlineto{\pgfqpoint{4.208198in}{0.634201in}}%
\pgfpathlineto{\pgfqpoint{4.208495in}{0.634283in}}%
\pgfpathlineto{\pgfqpoint{4.208793in}{0.634365in}}%
\pgfpathlineto{\pgfqpoint{4.209090in}{0.634447in}}%
\pgfpathlineto{\pgfqpoint{4.209388in}{0.634528in}}%
\pgfpathlineto{\pgfqpoint{4.209685in}{0.634610in}}%
\pgfpathlineto{\pgfqpoint{4.209983in}{0.634692in}}%
\pgfpathlineto{\pgfqpoint{4.210280in}{0.634774in}}%
\pgfpathlineto{\pgfqpoint{4.210578in}{0.634855in}}%
\pgfpathlineto{\pgfqpoint{4.210875in}{0.634937in}}%
\pgfpathlineto{\pgfqpoint{4.211173in}{0.635019in}}%
\pgfpathlineto{\pgfqpoint{4.211470in}{0.635101in}}%
\pgfpathlineto{\pgfqpoint{4.211768in}{0.635183in}}%
\pgfpathlineto{\pgfqpoint{4.212065in}{0.635264in}}%
\pgfpathlineto{\pgfqpoint{4.212363in}{0.635346in}}%
\pgfpathlineto{\pgfqpoint{4.212660in}{0.635428in}}%
\pgfpathlineto{\pgfqpoint{4.212958in}{0.635510in}}%
\pgfpathlineto{\pgfqpoint{4.213255in}{0.635591in}}%
\pgfpathlineto{\pgfqpoint{4.213553in}{0.635673in}}%
\pgfpathlineto{\pgfqpoint{4.213850in}{0.635755in}}%
\pgfpathlineto{\pgfqpoint{4.214148in}{0.635837in}}%
\pgfpathlineto{\pgfqpoint{4.214445in}{0.635918in}}%
\pgfpathlineto{\pgfqpoint{4.214743in}{0.636000in}}%
\pgfpathlineto{\pgfqpoint{4.215040in}{0.636082in}}%
\pgfpathlineto{\pgfqpoint{4.215337in}{0.636164in}}%
\pgfpathlineto{\pgfqpoint{4.215635in}{0.636246in}}%
\pgfpathlineto{\pgfqpoint{4.215932in}{0.636327in}}%
\pgfpathlineto{\pgfqpoint{4.216230in}{0.636409in}}%
\pgfpathlineto{\pgfqpoint{4.216527in}{0.636491in}}%
\pgfpathlineto{\pgfqpoint{4.216825in}{0.636573in}}%
\pgfpathlineto{\pgfqpoint{4.217122in}{0.636654in}}%
\pgfpathlineto{\pgfqpoint{4.217420in}{0.636736in}}%
\pgfpathlineto{\pgfqpoint{4.217717in}{0.636818in}}%
\pgfpathlineto{\pgfqpoint{4.218015in}{0.636900in}}%
\pgfpathlineto{\pgfqpoint{4.218312in}{0.636982in}}%
\pgfpathlineto{\pgfqpoint{4.218610in}{0.637063in}}%
\pgfpathlineto{\pgfqpoint{4.218907in}{0.637145in}}%
\pgfpathlineto{\pgfqpoint{4.219205in}{0.637227in}}%
\pgfpathlineto{\pgfqpoint{4.219502in}{0.637309in}}%
\pgfpathlineto{\pgfqpoint{4.219800in}{0.637390in}}%
\pgfpathlineto{\pgfqpoint{4.220097in}{0.637472in}}%
\pgfpathlineto{\pgfqpoint{4.220395in}{0.637554in}}%
\pgfpathlineto{\pgfqpoint{4.220692in}{0.637636in}}%
\pgfpathlineto{\pgfqpoint{4.220990in}{0.637717in}}%
\pgfpathlineto{\pgfqpoint{4.221287in}{0.637799in}}%
\pgfpathlineto{\pgfqpoint{4.221585in}{0.637881in}}%
\pgfpathlineto{\pgfqpoint{4.221882in}{0.637963in}}%
\pgfpathlineto{\pgfqpoint{4.222179in}{0.638045in}}%
\pgfpathlineto{\pgfqpoint{4.222477in}{0.638126in}}%
\pgfpathlineto{\pgfqpoint{4.222774in}{0.638208in}}%
\pgfpathlineto{\pgfqpoint{4.223072in}{0.638290in}}%
\pgfpathlineto{\pgfqpoint{4.223369in}{0.638372in}}%
\pgfpathlineto{\pgfqpoint{4.223667in}{0.638453in}}%
\pgfpathlineto{\pgfqpoint{4.223964in}{0.638535in}}%
\pgfpathlineto{\pgfqpoint{4.224262in}{0.638617in}}%
\pgfpathlineto{\pgfqpoint{4.224559in}{0.638699in}}%
\pgfpathlineto{\pgfqpoint{4.224857in}{0.638781in}}%
\pgfpathlineto{\pgfqpoint{4.225154in}{0.638862in}}%
\pgfpathlineto{\pgfqpoint{4.225452in}{0.638944in}}%
\pgfpathlineto{\pgfqpoint{4.225749in}{0.639026in}}%
\pgfpathlineto{\pgfqpoint{4.226047in}{0.639108in}}%
\pgfpathlineto{\pgfqpoint{4.226344in}{0.639189in}}%
\pgfpathlineto{\pgfqpoint{4.226642in}{0.639271in}}%
\pgfpathlineto{\pgfqpoint{4.226939in}{0.639353in}}%
\pgfpathlineto{\pgfqpoint{4.227237in}{0.639435in}}%
\pgfpathlineto{\pgfqpoint{4.227534in}{0.639516in}}%
\pgfpathlineto{\pgfqpoint{4.227832in}{0.639598in}}%
\pgfpathlineto{\pgfqpoint{4.228129in}{0.639680in}}%
\pgfpathlineto{\pgfqpoint{4.228426in}{0.639762in}}%
\pgfpathlineto{\pgfqpoint{4.228724in}{0.639844in}}%
\pgfpathlineto{\pgfqpoint{4.229021in}{0.639925in}}%
\pgfpathlineto{\pgfqpoint{4.229319in}{0.640007in}}%
\pgfpathlineto{\pgfqpoint{4.229616in}{0.640089in}}%
\pgfpathlineto{\pgfqpoint{4.229914in}{0.640171in}}%
\pgfpathlineto{\pgfqpoint{4.230211in}{0.640252in}}%
\pgfpathlineto{\pgfqpoint{4.230509in}{0.640334in}}%
\pgfpathlineto{\pgfqpoint{4.230806in}{0.640416in}}%
\pgfpathlineto{\pgfqpoint{4.231104in}{0.640498in}}%
\pgfpathlineto{\pgfqpoint{4.231401in}{0.640580in}}%
\pgfpathlineto{\pgfqpoint{4.231699in}{0.640661in}}%
\pgfpathlineto{\pgfqpoint{4.231996in}{0.640743in}}%
\pgfpathlineto{\pgfqpoint{4.232294in}{0.640825in}}%
\pgfpathlineto{\pgfqpoint{4.232591in}{0.640907in}}%
\pgfpathlineto{\pgfqpoint{4.232889in}{0.640948in}}%
\pgfpathlineto{\pgfqpoint{4.233186in}{0.640946in}}%
\pgfpathlineto{\pgfqpoint{4.233484in}{0.640943in}}%
\pgfpathlineto{\pgfqpoint{4.233781in}{0.640939in}}%
\pgfpathlineto{\pgfqpoint{4.234079in}{0.640934in}}%
\pgfpathlineto{\pgfqpoint{4.234376in}{0.640929in}}%
\pgfpathlineto{\pgfqpoint{4.234674in}{0.640924in}}%
\pgfpathlineto{\pgfqpoint{4.234971in}{0.640920in}}%
\pgfpathlineto{\pgfqpoint{4.235268in}{0.640915in}}%
\pgfpathlineto{\pgfqpoint{4.235566in}{0.640910in}}%
\pgfpathlineto{\pgfqpoint{4.235863in}{0.640905in}}%
\pgfpathlineto{\pgfqpoint{4.236161in}{0.640901in}}%
\pgfpathlineto{\pgfqpoint{4.236458in}{0.640896in}}%
\pgfpathlineto{\pgfqpoint{4.236756in}{0.640891in}}%
\pgfpathlineto{\pgfqpoint{4.237053in}{0.640886in}}%
\pgfpathlineto{\pgfqpoint{4.237351in}{0.640882in}}%
\pgfpathlineto{\pgfqpoint{4.237648in}{0.640877in}}%
\pgfpathlineto{\pgfqpoint{4.237946in}{0.640872in}}%
\pgfpathlineto{\pgfqpoint{4.238243in}{0.640868in}}%
\pgfpathlineto{\pgfqpoint{4.238541in}{0.640863in}}%
\pgfpathlineto{\pgfqpoint{4.238838in}{0.640858in}}%
\pgfpathlineto{\pgfqpoint{4.239136in}{0.640853in}}%
\pgfpathlineto{\pgfqpoint{4.239433in}{0.640849in}}%
\pgfpathlineto{\pgfqpoint{4.239731in}{0.640844in}}%
\pgfpathlineto{\pgfqpoint{4.240028in}{0.640839in}}%
\pgfpathlineto{\pgfqpoint{4.240326in}{0.640834in}}%
\pgfpathlineto{\pgfqpoint{4.240623in}{0.640830in}}%
\pgfpathlineto{\pgfqpoint{4.240921in}{0.640825in}}%
\pgfpathlineto{\pgfqpoint{4.241218in}{0.640820in}}%
\pgfpathlineto{\pgfqpoint{4.241516in}{0.640815in}}%
\pgfpathlineto{\pgfqpoint{4.241813in}{0.640811in}}%
\pgfpathlineto{\pgfqpoint{4.242110in}{0.640806in}}%
\pgfpathlineto{\pgfqpoint{4.242408in}{0.640801in}}%
\pgfpathlineto{\pgfqpoint{4.242705in}{0.640797in}}%
\pgfpathlineto{\pgfqpoint{4.243003in}{0.640792in}}%
\pgfpathlineto{\pgfqpoint{4.243300in}{0.640787in}}%
\pgfpathlineto{\pgfqpoint{4.243598in}{0.640782in}}%
\pgfpathlineto{\pgfqpoint{4.243895in}{0.640778in}}%
\pgfpathlineto{\pgfqpoint{4.244193in}{0.640773in}}%
\pgfpathlineto{\pgfqpoint{4.244490in}{0.640768in}}%
\pgfpathlineto{\pgfqpoint{4.244788in}{0.640763in}}%
\pgfpathlineto{\pgfqpoint{4.245085in}{0.640759in}}%
\pgfpathlineto{\pgfqpoint{4.245383in}{0.640754in}}%
\pgfpathlineto{\pgfqpoint{4.245680in}{0.640749in}}%
\pgfpathlineto{\pgfqpoint{4.245978in}{0.640744in}}%
\pgfpathlineto{\pgfqpoint{4.246275in}{0.640740in}}%
\pgfpathlineto{\pgfqpoint{4.246573in}{0.640735in}}%
\pgfpathlineto{\pgfqpoint{4.246870in}{0.640730in}}%
\pgfpathlineto{\pgfqpoint{4.247168in}{0.640725in}}%
\pgfpathlineto{\pgfqpoint{4.247465in}{0.640721in}}%
\pgfpathlineto{\pgfqpoint{4.247763in}{0.640716in}}%
\pgfpathlineto{\pgfqpoint{4.248060in}{0.640711in}}%
\pgfpathlineto{\pgfqpoint{4.248357in}{0.640707in}}%
\pgfpathlineto{\pgfqpoint{4.248655in}{0.640702in}}%
\pgfpathlineto{\pgfqpoint{4.248952in}{0.640697in}}%
\pgfpathlineto{\pgfqpoint{4.249250in}{0.640692in}}%
\pgfpathlineto{\pgfqpoint{4.249547in}{0.640688in}}%
\pgfpathlineto{\pgfqpoint{4.249845in}{0.640683in}}%
\pgfpathlineto{\pgfqpoint{4.250142in}{0.640678in}}%
\pgfpathlineto{\pgfqpoint{4.250440in}{0.640673in}}%
\pgfpathlineto{\pgfqpoint{4.250737in}{0.640669in}}%
\pgfpathlineto{\pgfqpoint{4.251035in}{0.640664in}}%
\pgfpathlineto{\pgfqpoint{4.251332in}{0.640659in}}%
\pgfpathlineto{\pgfqpoint{4.251630in}{0.640654in}}%
\pgfpathlineto{\pgfqpoint{4.251927in}{0.640650in}}%
\pgfpathlineto{\pgfqpoint{4.252225in}{0.640645in}}%
\pgfpathlineto{\pgfqpoint{4.252522in}{0.640640in}}%
\pgfpathlineto{\pgfqpoint{4.252820in}{0.640635in}}%
\pgfpathlineto{\pgfqpoint{4.253117in}{0.640631in}}%
\pgfpathlineto{\pgfqpoint{4.253415in}{0.640626in}}%
\pgfpathlineto{\pgfqpoint{4.253712in}{0.642926in}}%
\pgfpathlineto{\pgfqpoint{4.254010in}{0.650509in}}%
\pgfpathlineto{\pgfqpoint{4.254307in}{0.650641in}}%
\pgfpathlineto{\pgfqpoint{4.254605in}{0.650741in}}%
\pgfpathlineto{\pgfqpoint{4.254902in}{0.650841in}}%
\pgfpathlineto{\pgfqpoint{4.255199in}{0.650941in}}%
\pgfpathlineto{\pgfqpoint{4.255497in}{0.651041in}}%
\pgfpathlineto{\pgfqpoint{4.255794in}{0.651142in}}%
\pgfpathlineto{\pgfqpoint{4.256092in}{0.651242in}}%
\pgfpathlineto{\pgfqpoint{4.256389in}{0.651342in}}%
\pgfpathlineto{\pgfqpoint{4.256687in}{0.651442in}}%
\pgfpathlineto{\pgfqpoint{4.256984in}{0.651542in}}%
\pgfpathlineto{\pgfqpoint{4.257282in}{0.651643in}}%
\pgfpathlineto{\pgfqpoint{4.257579in}{0.651743in}}%
\pgfpathlineto{\pgfqpoint{4.257877in}{0.651843in}}%
\pgfpathlineto{\pgfqpoint{4.258174in}{0.651943in}}%
\pgfpathlineto{\pgfqpoint{4.258472in}{0.652043in}}%
\pgfpathlineto{\pgfqpoint{4.258769in}{0.652144in}}%
\pgfpathlineto{\pgfqpoint{4.259067in}{0.652244in}}%
\pgfpathlineto{\pgfqpoint{4.259364in}{0.652344in}}%
\pgfpathlineto{\pgfqpoint{4.259662in}{0.652444in}}%
\pgfpathlineto{\pgfqpoint{4.259959in}{0.652544in}}%
\pgfpathlineto{\pgfqpoint{4.260257in}{0.652645in}}%
\pgfpathlineto{\pgfqpoint{4.260554in}{0.652745in}}%
\pgfpathlineto{\pgfqpoint{4.260852in}{0.652845in}}%
\pgfpathlineto{\pgfqpoint{4.261149in}{0.652945in}}%
\pgfpathlineto{\pgfqpoint{4.261447in}{0.653045in}}%
\pgfpathlineto{\pgfqpoint{4.261744in}{0.653146in}}%
\pgfpathlineto{\pgfqpoint{4.262041in}{0.653246in}}%
\pgfpathlineto{\pgfqpoint{4.262339in}{0.653346in}}%
\pgfpathlineto{\pgfqpoint{4.262636in}{0.653446in}}%
\pgfpathlineto{\pgfqpoint{4.262934in}{0.653546in}}%
\pgfpathlineto{\pgfqpoint{4.263231in}{0.653647in}}%
\pgfpathlineto{\pgfqpoint{4.263529in}{0.653747in}}%
\pgfpathlineto{\pgfqpoint{4.263826in}{0.653847in}}%
\pgfpathlineto{\pgfqpoint{4.264124in}{0.653947in}}%
\pgfpathlineto{\pgfqpoint{4.264421in}{0.654047in}}%
\pgfpathlineto{\pgfqpoint{4.264719in}{0.654148in}}%
\pgfpathlineto{\pgfqpoint{4.265016in}{0.654248in}}%
\pgfpathlineto{\pgfqpoint{4.265314in}{0.654348in}}%
\pgfpathlineto{\pgfqpoint{4.265611in}{0.654448in}}%
\pgfpathlineto{\pgfqpoint{4.265909in}{0.654548in}}%
\pgfpathlineto{\pgfqpoint{4.266206in}{0.654649in}}%
\pgfpathlineto{\pgfqpoint{4.266504in}{0.654749in}}%
\pgfpathlineto{\pgfqpoint{4.266801in}{0.654849in}}%
\pgfpathlineto{\pgfqpoint{4.267099in}{0.654949in}}%
\pgfpathlineto{\pgfqpoint{4.267396in}{0.655050in}}%
\pgfpathlineto{\pgfqpoint{4.267694in}{0.655150in}}%
\pgfpathlineto{\pgfqpoint{4.267991in}{0.655250in}}%
\pgfpathlineto{\pgfqpoint{4.268288in}{0.655350in}}%
\pgfpathlineto{\pgfqpoint{4.268586in}{0.655450in}}%
\pgfpathlineto{\pgfqpoint{4.268883in}{0.655551in}}%
\pgfpathlineto{\pgfqpoint{4.269181in}{0.655651in}}%
\pgfpathlineto{\pgfqpoint{4.269478in}{0.655751in}}%
\pgfpathlineto{\pgfqpoint{4.269776in}{0.655851in}}%
\pgfpathlineto{\pgfqpoint{4.270073in}{0.655951in}}%
\pgfpathlineto{\pgfqpoint{4.270371in}{0.656052in}}%
\pgfpathlineto{\pgfqpoint{4.270668in}{0.656152in}}%
\pgfpathlineto{\pgfqpoint{4.270966in}{0.656252in}}%
\pgfpathlineto{\pgfqpoint{4.271263in}{0.656352in}}%
\pgfpathlineto{\pgfqpoint{4.271561in}{0.656452in}}%
\pgfpathlineto{\pgfqpoint{4.271858in}{0.656553in}}%
\pgfpathlineto{\pgfqpoint{4.272156in}{0.656653in}}%
\pgfpathlineto{\pgfqpoint{4.272453in}{0.656753in}}%
\pgfpathlineto{\pgfqpoint{4.272751in}{0.656853in}}%
\pgfpathlineto{\pgfqpoint{4.273048in}{0.656953in}}%
\pgfpathlineto{\pgfqpoint{4.273346in}{0.657054in}}%
\pgfpathlineto{\pgfqpoint{4.273643in}{0.657154in}}%
\pgfpathlineto{\pgfqpoint{4.273941in}{0.657254in}}%
\pgfpathlineto{\pgfqpoint{4.274238in}{0.657354in}}%
\pgfpathlineto{\pgfqpoint{4.274536in}{0.657454in}}%
\pgfpathlineto{\pgfqpoint{4.274833in}{0.657555in}}%
\pgfpathlineto{\pgfqpoint{4.275130in}{0.657655in}}%
\pgfpathlineto{\pgfqpoint{4.275428in}{0.657755in}}%
\pgfpathlineto{\pgfqpoint{4.275725in}{0.657853in}}%
\pgfpathlineto{\pgfqpoint{4.276023in}{0.657907in}}%
\pgfpathlineto{\pgfqpoint{4.276320in}{0.657943in}}%
\pgfpathlineto{\pgfqpoint{4.276618in}{0.657979in}}%
\pgfpathlineto{\pgfqpoint{4.276915in}{0.658015in}}%
\pgfpathlineto{\pgfqpoint{4.277213in}{0.658051in}}%
\pgfpathlineto{\pgfqpoint{4.277510in}{0.658087in}}%
\pgfpathlineto{\pgfqpoint{4.277808in}{0.658124in}}%
\pgfpathlineto{\pgfqpoint{4.278105in}{0.658160in}}%
\pgfpathlineto{\pgfqpoint{4.278403in}{0.658196in}}%
\pgfpathlineto{\pgfqpoint{4.278700in}{0.658232in}}%
\pgfpathlineto{\pgfqpoint{4.278998in}{0.658268in}}%
\pgfpathlineto{\pgfqpoint{4.279295in}{0.658304in}}%
\pgfpathlineto{\pgfqpoint{4.279593in}{0.658340in}}%
\pgfpathlineto{\pgfqpoint{4.279890in}{0.658376in}}%
\pgfpathlineto{\pgfqpoint{4.280188in}{0.658412in}}%
\pgfpathlineto{\pgfqpoint{4.280485in}{0.658449in}}%
\pgfpathlineto{\pgfqpoint{4.280783in}{0.658485in}}%
\pgfpathlineto{\pgfqpoint{4.281080in}{0.658371in}}%
\pgfpathlineto{\pgfqpoint{4.281378in}{0.657511in}}%
\pgfpathlineto{\pgfqpoint{4.281675in}{0.656549in}}%
\pgfpathlineto{\pgfqpoint{4.281972in}{0.655586in}}%
\pgfpathlineto{\pgfqpoint{4.282270in}{0.654623in}}%
\pgfpathlineto{\pgfqpoint{4.282567in}{0.653660in}}%
\pgfpathlineto{\pgfqpoint{4.282865in}{0.652697in}}%
\pgfpathlineto{\pgfqpoint{4.283162in}{0.651735in}}%
\pgfpathlineto{\pgfqpoint{4.283460in}{0.653925in}}%
\pgfpathlineto{\pgfqpoint{4.283757in}{0.658855in}}%
\pgfpathlineto{\pgfqpoint{4.284055in}{0.658854in}}%
\pgfpathlineto{\pgfqpoint{4.284352in}{0.658846in}}%
\pgfpathlineto{\pgfqpoint{4.284650in}{0.658832in}}%
\pgfpathlineto{\pgfqpoint{4.284947in}{0.658819in}}%
\pgfpathlineto{\pgfqpoint{4.285245in}{0.658805in}}%
\pgfpathlineto{\pgfqpoint{4.285542in}{0.658792in}}%
\pgfpathlineto{\pgfqpoint{4.285840in}{0.658779in}}%
\pgfpathlineto{\pgfqpoint{4.286137in}{0.658765in}}%
\pgfpathlineto{\pgfqpoint{4.286435in}{0.658752in}}%
\pgfpathlineto{\pgfqpoint{4.286732in}{0.658739in}}%
\pgfpathlineto{\pgfqpoint{4.287030in}{0.658725in}}%
\pgfpathlineto{\pgfqpoint{4.287327in}{0.658712in}}%
\pgfpathlineto{\pgfqpoint{4.287625in}{0.658698in}}%
\pgfpathlineto{\pgfqpoint{4.287922in}{0.658685in}}%
\pgfpathlineto{\pgfqpoint{4.288219in}{0.658672in}}%
\pgfpathlineto{\pgfqpoint{4.288517in}{0.658658in}}%
\pgfpathlineto{\pgfqpoint{4.288814in}{0.658645in}}%
\pgfpathlineto{\pgfqpoint{4.289112in}{0.658631in}}%
\pgfpathlineto{\pgfqpoint{4.289409in}{0.658618in}}%
\pgfpathlineto{\pgfqpoint{4.289707in}{0.658605in}}%
\pgfpathlineto{\pgfqpoint{4.290004in}{0.658592in}}%
\pgfpathlineto{\pgfqpoint{4.290302in}{0.658592in}}%
\pgfpathlineto{\pgfqpoint{4.290599in}{0.658594in}}%
\pgfpathlineto{\pgfqpoint{4.290897in}{0.658597in}}%
\pgfpathlineto{\pgfqpoint{4.291194in}{0.658599in}}%
\pgfpathlineto{\pgfqpoint{4.291492in}{0.658602in}}%
\pgfpathlineto{\pgfqpoint{4.291789in}{0.658605in}}%
\pgfpathlineto{\pgfqpoint{4.292087in}{0.658607in}}%
\pgfpathlineto{\pgfqpoint{4.292384in}{0.658610in}}%
\pgfpathlineto{\pgfqpoint{4.292682in}{0.658612in}}%
\pgfpathlineto{\pgfqpoint{4.292979in}{0.658615in}}%
\pgfpathlineto{\pgfqpoint{4.293277in}{0.658618in}}%
\pgfpathlineto{\pgfqpoint{4.293574in}{0.658620in}}%
\pgfpathlineto{\pgfqpoint{4.293872in}{0.658623in}}%
\pgfpathlineto{\pgfqpoint{4.294169in}{0.658625in}}%
\pgfpathlineto{\pgfqpoint{4.294467in}{0.658628in}}%
\pgfpathlineto{\pgfqpoint{4.294764in}{0.658631in}}%
\pgfpathlineto{\pgfqpoint{4.295061in}{0.658633in}}%
\pgfpathlineto{\pgfqpoint{4.295359in}{0.658636in}}%
\pgfpathlineto{\pgfqpoint{4.295656in}{0.658638in}}%
\pgfpathlineto{\pgfqpoint{4.295954in}{0.658641in}}%
\pgfpathlineto{\pgfqpoint{4.296251in}{0.658644in}}%
\pgfpathlineto{\pgfqpoint{4.296549in}{0.658646in}}%
\pgfpathlineto{\pgfqpoint{4.296846in}{0.658649in}}%
\pgfpathlineto{\pgfqpoint{4.297144in}{0.658651in}}%
\pgfpathlineto{\pgfqpoint{4.297441in}{0.658654in}}%
\pgfpathlineto{\pgfqpoint{4.297739in}{0.658657in}}%
\pgfpathlineto{\pgfqpoint{4.298036in}{0.658659in}}%
\pgfpathlineto{\pgfqpoint{4.298334in}{0.658662in}}%
\pgfpathlineto{\pgfqpoint{4.298631in}{0.658664in}}%
\pgfpathlineto{\pgfqpoint{4.298929in}{0.658667in}}%
\pgfpathlineto{\pgfqpoint{4.299226in}{0.658670in}}%
\pgfpathlineto{\pgfqpoint{4.299524in}{0.658672in}}%
\pgfpathlineto{\pgfqpoint{4.299821in}{0.658675in}}%
\pgfpathlineto{\pgfqpoint{4.300119in}{0.658677in}}%
\pgfpathlineto{\pgfqpoint{4.300416in}{0.658680in}}%
\pgfpathlineto{\pgfqpoint{4.300714in}{0.658683in}}%
\pgfpathlineto{\pgfqpoint{4.301011in}{0.658685in}}%
\pgfpathlineto{\pgfqpoint{4.301309in}{0.658688in}}%
\pgfpathlineto{\pgfqpoint{4.301606in}{0.658690in}}%
\pgfpathlineto{\pgfqpoint{4.301903in}{0.658693in}}%
\pgfpathlineto{\pgfqpoint{4.302201in}{0.658696in}}%
\pgfpathlineto{\pgfqpoint{4.302498in}{0.658698in}}%
\pgfpathlineto{\pgfqpoint{4.302796in}{0.658701in}}%
\pgfpathlineto{\pgfqpoint{4.303093in}{0.658703in}}%
\pgfpathlineto{\pgfqpoint{4.303391in}{0.658706in}}%
\pgfpathlineto{\pgfqpoint{4.303688in}{0.658709in}}%
\pgfpathlineto{\pgfqpoint{4.303986in}{0.658711in}}%
\pgfpathlineto{\pgfqpoint{4.304283in}{0.658714in}}%
\pgfpathlineto{\pgfqpoint{4.304581in}{0.658713in}}%
\pgfpathlineto{\pgfqpoint{4.304878in}{0.658699in}}%
\pgfpathlineto{\pgfqpoint{4.305176in}{0.658684in}}%
\pgfpathlineto{\pgfqpoint{4.305473in}{0.658670in}}%
\pgfpathlineto{\pgfqpoint{4.305771in}{0.658643in}}%
\pgfpathlineto{\pgfqpoint{4.306068in}{0.658588in}}%
\pgfpathlineto{\pgfqpoint{4.306366in}{0.658570in}}%
\pgfpathlineto{\pgfqpoint{4.306663in}{0.658556in}}%
\pgfpathlineto{\pgfqpoint{4.306961in}{0.658543in}}%
\pgfpathlineto{\pgfqpoint{4.307258in}{0.658530in}}%
\pgfpathlineto{\pgfqpoint{4.307556in}{0.658516in}}%
\pgfpathlineto{\pgfqpoint{4.307853in}{0.658503in}}%
\pgfpathlineto{\pgfqpoint{4.308150in}{0.658489in}}%
\pgfpathlineto{\pgfqpoint{4.308448in}{0.658476in}}%
\pgfpathlineto{\pgfqpoint{4.308745in}{0.658462in}}%
\pgfpathlineto{\pgfqpoint{4.309043in}{0.658449in}}%
\pgfpathlineto{\pgfqpoint{4.309340in}{0.658435in}}%
\pgfpathlineto{\pgfqpoint{4.309638in}{0.658422in}}%
\pgfpathlineto{\pgfqpoint{4.309935in}{0.658408in}}%
\pgfpathlineto{\pgfqpoint{4.310233in}{0.658395in}}%
\pgfpathlineto{\pgfqpoint{4.310530in}{0.658382in}}%
\pgfpathlineto{\pgfqpoint{4.310828in}{0.658368in}}%
\pgfpathlineto{\pgfqpoint{4.311125in}{0.658355in}}%
\pgfpathlineto{\pgfqpoint{4.311423in}{0.658341in}}%
\pgfpathlineto{\pgfqpoint{4.311720in}{0.658328in}}%
\pgfpathlineto{\pgfqpoint{4.312018in}{0.658314in}}%
\pgfpathlineto{\pgfqpoint{4.312315in}{0.658301in}}%
\pgfpathlineto{\pgfqpoint{4.312613in}{0.658287in}}%
\pgfpathlineto{\pgfqpoint{4.312910in}{0.658274in}}%
\pgfpathlineto{\pgfqpoint{4.313208in}{0.658260in}}%
\pgfpathlineto{\pgfqpoint{4.313505in}{0.658247in}}%
\pgfpathlineto{\pgfqpoint{4.313803in}{0.658234in}}%
\pgfpathlineto{\pgfqpoint{4.314100in}{0.658220in}}%
\pgfpathlineto{\pgfqpoint{4.314398in}{0.658207in}}%
\pgfpathlineto{\pgfqpoint{4.314695in}{0.658193in}}%
\pgfpathlineto{\pgfqpoint{4.314992in}{0.658180in}}%
\pgfpathlineto{\pgfqpoint{4.315290in}{0.658166in}}%
\pgfpathlineto{\pgfqpoint{4.315587in}{0.658153in}}%
\pgfpathlineto{\pgfqpoint{4.315885in}{0.658139in}}%
\pgfpathlineto{\pgfqpoint{4.316182in}{0.658126in}}%
\pgfpathlineto{\pgfqpoint{4.316480in}{0.658112in}}%
\pgfpathlineto{\pgfqpoint{4.316777in}{0.658099in}}%
\pgfpathlineto{\pgfqpoint{4.317075in}{0.658086in}}%
\pgfpathlineto{\pgfqpoint{4.317372in}{0.658072in}}%
\pgfpathlineto{\pgfqpoint{4.317670in}{0.658059in}}%
\pgfpathlineto{\pgfqpoint{4.317967in}{0.658045in}}%
\pgfpathlineto{\pgfqpoint{4.318265in}{0.658032in}}%
\pgfpathlineto{\pgfqpoint{4.318562in}{0.658018in}}%
\pgfpathlineto{\pgfqpoint{4.318860in}{0.658005in}}%
\pgfpathlineto{\pgfqpoint{4.319157in}{0.657991in}}%
\pgfpathlineto{\pgfqpoint{4.319455in}{0.657978in}}%
\pgfpathlineto{\pgfqpoint{4.319752in}{0.657965in}}%
\pgfpathlineto{\pgfqpoint{4.320050in}{0.657951in}}%
\pgfpathlineto{\pgfqpoint{4.320347in}{0.657938in}}%
\pgfpathlineto{\pgfqpoint{4.320645in}{0.657924in}}%
\pgfpathlineto{\pgfqpoint{4.320942in}{0.657911in}}%
\pgfpathlineto{\pgfqpoint{4.321240in}{0.657897in}}%
\pgfpathlineto{\pgfqpoint{4.321537in}{0.657884in}}%
\pgfpathlineto{\pgfqpoint{4.321834in}{0.657870in}}%
\pgfpathlineto{\pgfqpoint{4.322132in}{0.657857in}}%
\pgfpathlineto{\pgfqpoint{4.322429in}{0.657843in}}%
\pgfpathlineto{\pgfqpoint{4.322727in}{0.657830in}}%
\pgfpathlineto{\pgfqpoint{4.323024in}{0.657817in}}%
\pgfpathlineto{\pgfqpoint{4.323322in}{0.657803in}}%
\pgfpathlineto{\pgfqpoint{4.323619in}{0.657790in}}%
\pgfpathlineto{\pgfqpoint{4.323917in}{0.657776in}}%
\pgfpathlineto{\pgfqpoint{4.324214in}{0.657763in}}%
\pgfpathlineto{\pgfqpoint{4.324512in}{0.657749in}}%
\pgfpathlineto{\pgfqpoint{4.324809in}{0.657736in}}%
\pgfpathlineto{\pgfqpoint{4.325107in}{0.657722in}}%
\pgfpathlineto{\pgfqpoint{4.325404in}{0.657709in}}%
\pgfpathlineto{\pgfqpoint{4.325702in}{0.657695in}}%
\pgfpathlineto{\pgfqpoint{4.325999in}{0.657682in}}%
\pgfpathlineto{\pgfqpoint{4.326297in}{0.657669in}}%
\pgfpathlineto{\pgfqpoint{4.326594in}{0.657708in}}%
\pgfpathlineto{\pgfqpoint{4.326892in}{0.658215in}}%
\pgfpathlineto{\pgfqpoint{4.327189in}{0.658833in}}%
\pgfpathlineto{\pgfqpoint{4.327487in}{0.659450in}}%
\pgfpathlineto{\pgfqpoint{4.327784in}{0.660068in}}%
\pgfpathlineto{\pgfqpoint{4.328081in}{0.660686in}}%
\pgfpathlineto{\pgfqpoint{4.328379in}{0.661303in}}%
\pgfpathlineto{\pgfqpoint{4.328676in}{0.661921in}}%
\pgfpathlineto{\pgfqpoint{4.328974in}{0.662539in}}%
\pgfpathlineto{\pgfqpoint{4.329271in}{0.663156in}}%
\pgfpathlineto{\pgfqpoint{4.329569in}{0.663774in}}%
\pgfpathlineto{\pgfqpoint{4.329866in}{0.664392in}}%
\pgfpathlineto{\pgfqpoint{4.330164in}{0.665009in}}%
\pgfpathlineto{\pgfqpoint{4.330461in}{0.665627in}}%
\pgfpathlineto{\pgfqpoint{4.330759in}{0.666245in}}%
\pgfpathlineto{\pgfqpoint{4.331056in}{0.666856in}}%
\pgfpathlineto{\pgfqpoint{4.331354in}{0.667292in}}%
\pgfpathlineto{\pgfqpoint{4.331651in}{0.667315in}}%
\pgfpathlineto{\pgfqpoint{4.331949in}{0.667301in}}%
\pgfpathlineto{\pgfqpoint{4.332246in}{0.667288in}}%
\pgfpathlineto{\pgfqpoint{4.332544in}{0.667274in}}%
\pgfpathlineto{\pgfqpoint{4.332841in}{0.667260in}}%
\pgfpathlineto{\pgfqpoint{4.333139in}{0.667246in}}%
\pgfpathlineto{\pgfqpoint{4.333436in}{0.667232in}}%
\pgfpathlineto{\pgfqpoint{4.333734in}{0.667366in}}%
\pgfpathlineto{\pgfqpoint{4.334031in}{0.667696in}}%
\pgfpathlineto{\pgfqpoint{4.334329in}{0.667875in}}%
\pgfpathlineto{\pgfqpoint{4.334626in}{0.667862in}}%
\pgfpathlineto{\pgfqpoint{4.334923in}{0.667848in}}%
\pgfpathlineto{\pgfqpoint{4.335221in}{0.667835in}}%
\pgfpathlineto{\pgfqpoint{4.335518in}{0.667822in}}%
\pgfpathlineto{\pgfqpoint{4.335816in}{0.667808in}}%
\pgfpathlineto{\pgfqpoint{4.336113in}{0.667795in}}%
\pgfpathlineto{\pgfqpoint{4.336411in}{0.667782in}}%
\pgfpathlineto{\pgfqpoint{4.336708in}{0.667768in}}%
\pgfpathlineto{\pgfqpoint{4.337006in}{0.667755in}}%
\pgfpathlineto{\pgfqpoint{4.337303in}{0.667741in}}%
\pgfpathlineto{\pgfqpoint{4.337601in}{0.667728in}}%
\pgfpathlineto{\pgfqpoint{4.337898in}{0.667715in}}%
\pgfpathlineto{\pgfqpoint{4.338196in}{0.667701in}}%
\pgfpathlineto{\pgfqpoint{4.338493in}{0.667688in}}%
\pgfpathlineto{\pgfqpoint{4.338791in}{0.667674in}}%
\pgfpathlineto{\pgfqpoint{4.339088in}{0.667661in}}%
\pgfpathlineto{\pgfqpoint{4.339386in}{0.667648in}}%
\pgfpathlineto{\pgfqpoint{4.339683in}{0.667634in}}%
\pgfpathlineto{\pgfqpoint{4.339981in}{0.667621in}}%
\pgfpathlineto{\pgfqpoint{4.340278in}{0.667607in}}%
\pgfpathlineto{\pgfqpoint{4.340576in}{0.667594in}}%
\pgfpathlineto{\pgfqpoint{4.340873in}{0.667581in}}%
\pgfpathlineto{\pgfqpoint{4.341171in}{0.667567in}}%
\pgfpathlineto{\pgfqpoint{4.341468in}{0.667554in}}%
\pgfpathlineto{\pgfqpoint{4.341765in}{0.667541in}}%
\pgfpathlineto{\pgfqpoint{4.342063in}{0.667527in}}%
\pgfpathlineto{\pgfqpoint{4.342360in}{0.667514in}}%
\pgfpathlineto{\pgfqpoint{4.342658in}{0.667500in}}%
\pgfpathlineto{\pgfqpoint{4.342955in}{0.667487in}}%
\pgfpathlineto{\pgfqpoint{4.343253in}{0.667474in}}%
\pgfpathlineto{\pgfqpoint{4.343550in}{0.667460in}}%
\pgfpathlineto{\pgfqpoint{4.343848in}{0.667447in}}%
\pgfpathlineto{\pgfqpoint{4.344145in}{0.667433in}}%
\pgfpathlineto{\pgfqpoint{4.344443in}{0.667420in}}%
\pgfpathlineto{\pgfqpoint{4.344740in}{0.667407in}}%
\pgfpathlineto{\pgfqpoint{4.345038in}{0.667393in}}%
\pgfpathlineto{\pgfqpoint{4.345335in}{0.667380in}}%
\pgfpathlineto{\pgfqpoint{4.345633in}{0.667367in}}%
\pgfpathlineto{\pgfqpoint{4.345930in}{0.667353in}}%
\pgfpathlineto{\pgfqpoint{4.346228in}{0.667340in}}%
\pgfpathlineto{\pgfqpoint{4.346525in}{0.667326in}}%
\pgfpathlineto{\pgfqpoint{4.346823in}{0.667313in}}%
\pgfpathlineto{\pgfqpoint{4.347120in}{0.667300in}}%
\pgfpathlineto{\pgfqpoint{4.347418in}{0.667286in}}%
\pgfpathlineto{\pgfqpoint{4.347715in}{0.667273in}}%
\pgfpathlineto{\pgfqpoint{4.348012in}{0.667260in}}%
\pgfpathlineto{\pgfqpoint{4.348310in}{0.667258in}}%
\pgfpathlineto{\pgfqpoint{4.348607in}{0.667257in}}%
\pgfpathlineto{\pgfqpoint{4.348905in}{0.667257in}}%
\pgfpathlineto{\pgfqpoint{4.349202in}{0.667257in}}%
\pgfpathlineto{\pgfqpoint{4.349500in}{0.667256in}}%
\pgfpathlineto{\pgfqpoint{4.349797in}{0.667256in}}%
\pgfpathlineto{\pgfqpoint{4.350095in}{0.667256in}}%
\pgfpathlineto{\pgfqpoint{4.350392in}{0.667255in}}%
\pgfpathlineto{\pgfqpoint{4.350690in}{0.667255in}}%
\pgfpathlineto{\pgfqpoint{4.350987in}{0.667255in}}%
\pgfpathlineto{\pgfqpoint{4.351285in}{0.667254in}}%
\pgfpathlineto{\pgfqpoint{4.351582in}{0.667254in}}%
\pgfpathlineto{\pgfqpoint{4.351880in}{0.667254in}}%
\pgfpathlineto{\pgfqpoint{4.352177in}{0.667253in}}%
\pgfpathlineto{\pgfqpoint{4.352475in}{0.667253in}}%
\pgfpathlineto{\pgfqpoint{4.352772in}{0.667253in}}%
\pgfpathlineto{\pgfqpoint{4.353070in}{0.667252in}}%
\pgfpathlineto{\pgfqpoint{4.353367in}{0.667252in}}%
\pgfpathlineto{\pgfqpoint{4.353665in}{0.667252in}}%
\pgfpathlineto{\pgfqpoint{4.353962in}{0.667251in}}%
\pgfpathlineto{\pgfqpoint{4.354260in}{0.667251in}}%
\pgfpathlineto{\pgfqpoint{4.354557in}{0.667251in}}%
\pgfpathlineto{\pgfqpoint{4.354854in}{0.667250in}}%
\pgfpathlineto{\pgfqpoint{4.355152in}{0.667250in}}%
\pgfpathlineto{\pgfqpoint{4.355449in}{0.667250in}}%
\pgfpathlineto{\pgfqpoint{4.355747in}{0.667249in}}%
\pgfpathlineto{\pgfqpoint{4.356044in}{0.667249in}}%
\pgfpathlineto{\pgfqpoint{4.356342in}{0.667249in}}%
\pgfpathlineto{\pgfqpoint{4.356639in}{0.667248in}}%
\pgfpathlineto{\pgfqpoint{4.356937in}{0.667248in}}%
\pgfpathlineto{\pgfqpoint{4.357234in}{0.667248in}}%
\pgfpathlineto{\pgfqpoint{4.357532in}{0.667248in}}%
\pgfpathlineto{\pgfqpoint{4.357829in}{0.667247in}}%
\pgfpathlineto{\pgfqpoint{4.358127in}{0.667247in}}%
\pgfpathlineto{\pgfqpoint{4.358424in}{0.667247in}}%
\pgfpathlineto{\pgfqpoint{4.358722in}{0.667246in}}%
\pgfpathlineto{\pgfqpoint{4.359019in}{0.667246in}}%
\pgfpathlineto{\pgfqpoint{4.359317in}{0.667246in}}%
\pgfpathlineto{\pgfqpoint{4.359614in}{0.667245in}}%
\pgfpathlineto{\pgfqpoint{4.359912in}{0.667245in}}%
\pgfpathlineto{\pgfqpoint{4.360209in}{0.667245in}}%
\pgfpathlineto{\pgfqpoint{4.360507in}{0.667244in}}%
\pgfpathlineto{\pgfqpoint{4.360804in}{0.667244in}}%
\pgfpathlineto{\pgfqpoint{4.361102in}{0.667244in}}%
\pgfpathlineto{\pgfqpoint{4.361399in}{0.667243in}}%
\pgfpathlineto{\pgfqpoint{4.361696in}{0.667243in}}%
\pgfpathlineto{\pgfqpoint{4.361994in}{0.667243in}}%
\pgfpathlineto{\pgfqpoint{4.362291in}{0.667242in}}%
\pgfpathlineto{\pgfqpoint{4.362589in}{0.667242in}}%
\pgfpathlineto{\pgfqpoint{4.362886in}{0.667242in}}%
\pgfpathlineto{\pgfqpoint{4.363184in}{0.667241in}}%
\pgfpathlineto{\pgfqpoint{4.363481in}{0.667241in}}%
\pgfpathlineto{\pgfqpoint{4.363779in}{0.667241in}}%
\pgfpathlineto{\pgfqpoint{4.364076in}{0.667240in}}%
\pgfpathlineto{\pgfqpoint{4.364374in}{0.667240in}}%
\pgfpathlineto{\pgfqpoint{4.364671in}{0.667240in}}%
\pgfpathlineto{\pgfqpoint{4.364969in}{0.667239in}}%
\pgfpathlineto{\pgfqpoint{4.365266in}{0.667239in}}%
\pgfpathlineto{\pgfqpoint{4.365564in}{0.667239in}}%
\pgfpathlineto{\pgfqpoint{4.365861in}{0.667238in}}%
\pgfpathlineto{\pgfqpoint{4.366159in}{0.667238in}}%
\pgfpathlineto{\pgfqpoint{4.366456in}{0.667238in}}%
\pgfpathlineto{\pgfqpoint{4.366754in}{0.667238in}}%
\pgfpathlineto{\pgfqpoint{4.367051in}{0.667237in}}%
\pgfpathlineto{\pgfqpoint{4.367349in}{0.667237in}}%
\pgfpathlineto{\pgfqpoint{4.367646in}{0.667237in}}%
\pgfpathlineto{\pgfqpoint{4.367943in}{0.667236in}}%
\pgfpathlineto{\pgfqpoint{4.368241in}{0.667236in}}%
\pgfpathlineto{\pgfqpoint{4.368538in}{0.667236in}}%
\pgfpathlineto{\pgfqpoint{4.368836in}{0.667235in}}%
\pgfpathlineto{\pgfqpoint{4.369133in}{0.667235in}}%
\pgfpathlineto{\pgfqpoint{4.369431in}{0.667235in}}%
\pgfpathlineto{\pgfqpoint{4.369728in}{0.667234in}}%
\pgfpathlineto{\pgfqpoint{4.370026in}{0.667234in}}%
\pgfpathlineto{\pgfqpoint{4.370323in}{0.667234in}}%
\pgfpathlineto{\pgfqpoint{4.370621in}{0.667233in}}%
\pgfpathlineto{\pgfqpoint{4.370918in}{0.667233in}}%
\pgfpathlineto{\pgfqpoint{4.371216in}{0.667233in}}%
\pgfpathlineto{\pgfqpoint{4.371513in}{0.667232in}}%
\pgfpathlineto{\pgfqpoint{4.371811in}{0.667232in}}%
\pgfpathlineto{\pgfqpoint{4.372108in}{0.667232in}}%
\pgfpathlineto{\pgfqpoint{4.372406in}{0.667231in}}%
\pgfpathlineto{\pgfqpoint{4.372703in}{0.667231in}}%
\pgfpathlineto{\pgfqpoint{4.373001in}{0.667231in}}%
\pgfpathlineto{\pgfqpoint{4.373298in}{0.667230in}}%
\pgfpathlineto{\pgfqpoint{4.373596in}{0.667230in}}%
\pgfpathlineto{\pgfqpoint{4.373893in}{0.667230in}}%
\pgfpathlineto{\pgfqpoint{4.374191in}{0.667229in}}%
\pgfpathlineto{\pgfqpoint{4.374488in}{0.667229in}}%
\pgfpathlineto{\pgfqpoint{4.374785in}{0.667229in}}%
\pgfpathlineto{\pgfqpoint{4.375083in}{0.667228in}}%
\pgfpathlineto{\pgfqpoint{4.375380in}{0.667228in}}%
\pgfpathlineto{\pgfqpoint{4.375678in}{0.667228in}}%
\pgfpathlineto{\pgfqpoint{4.375975in}{0.667228in}}%
\pgfpathlineto{\pgfqpoint{4.376273in}{0.667227in}}%
\pgfpathlineto{\pgfqpoint{4.376570in}{0.667227in}}%
\pgfpathlineto{\pgfqpoint{4.376868in}{0.667227in}}%
\pgfpathlineto{\pgfqpoint{4.377165in}{0.667227in}}%
\pgfpathlineto{\pgfqpoint{4.377463in}{0.667251in}}%
\pgfpathlineto{\pgfqpoint{4.377760in}{0.667288in}}%
\pgfpathlineto{\pgfqpoint{4.378058in}{0.667325in}}%
\pgfpathlineto{\pgfqpoint{4.378355in}{0.667362in}}%
\pgfpathlineto{\pgfqpoint{4.378653in}{0.667400in}}%
\pgfpathlineto{\pgfqpoint{4.378950in}{0.667437in}}%
\pgfpathlineto{\pgfqpoint{4.379248in}{0.667474in}}%
\pgfpathlineto{\pgfqpoint{4.379545in}{0.667511in}}%
\pgfpathlineto{\pgfqpoint{4.379843in}{0.667549in}}%
\pgfpathlineto{\pgfqpoint{4.380140in}{0.667586in}}%
\pgfpathlineto{\pgfqpoint{4.380438in}{0.667623in}}%
\pgfpathlineto{\pgfqpoint{4.380735in}{0.667661in}}%
\pgfpathlineto{\pgfqpoint{4.381033in}{0.667698in}}%
\pgfpathlineto{\pgfqpoint{4.381330in}{0.667735in}}%
\pgfpathlineto{\pgfqpoint{4.381627in}{0.667772in}}%
\pgfpathlineto{\pgfqpoint{4.381925in}{0.667810in}}%
\pgfpathlineto{\pgfqpoint{4.382222in}{0.667847in}}%
\pgfpathlineto{\pgfqpoint{4.382520in}{0.667863in}}%
\pgfpathlineto{\pgfqpoint{4.382817in}{0.667863in}}%
\pgfpathlineto{\pgfqpoint{4.383115in}{0.667889in}}%
\pgfpathlineto{\pgfqpoint{4.383412in}{0.667922in}}%
\pgfpathlineto{\pgfqpoint{4.383710in}{0.667955in}}%
\pgfpathlineto{\pgfqpoint{4.384007in}{0.667989in}}%
\pgfpathlineto{\pgfqpoint{4.384305in}{0.668022in}}%
\pgfpathlineto{\pgfqpoint{4.384602in}{0.668055in}}%
\pgfpathlineto{\pgfqpoint{4.384900in}{0.668089in}}%
\pgfpathlineto{\pgfqpoint{4.385197in}{0.668122in}}%
\pgfpathlineto{\pgfqpoint{4.385495in}{0.668155in}}%
\pgfpathlineto{\pgfqpoint{4.385792in}{0.668189in}}%
\pgfpathlineto{\pgfqpoint{4.386090in}{0.668222in}}%
\pgfpathlineto{\pgfqpoint{4.386387in}{0.668255in}}%
\pgfpathlineto{\pgfqpoint{4.386685in}{0.668289in}}%
\pgfpathlineto{\pgfqpoint{4.386982in}{0.668322in}}%
\pgfpathlineto{\pgfqpoint{4.387280in}{0.668355in}}%
\pgfpathlineto{\pgfqpoint{4.387577in}{0.668389in}}%
\pgfpathlineto{\pgfqpoint{4.387874in}{0.668422in}}%
\pgfpathlineto{\pgfqpoint{4.388172in}{0.668456in}}%
\pgfpathlineto{\pgfqpoint{4.388469in}{0.668489in}}%
\pgfpathlineto{\pgfqpoint{4.388767in}{0.668522in}}%
\pgfpathlineto{\pgfqpoint{4.389064in}{0.668556in}}%
\pgfpathlineto{\pgfqpoint{4.389362in}{0.668589in}}%
\pgfpathlineto{\pgfqpoint{4.389659in}{0.668622in}}%
\pgfpathlineto{\pgfqpoint{4.389957in}{0.668656in}}%
\pgfpathlineto{\pgfqpoint{4.390254in}{0.668689in}}%
\pgfpathlineto{\pgfqpoint{4.390552in}{0.668722in}}%
\pgfpathlineto{\pgfqpoint{4.390849in}{0.668748in}}%
\pgfpathlineto{\pgfqpoint{4.391147in}{0.668750in}}%
\pgfpathlineto{\pgfqpoint{4.391444in}{0.668750in}}%
\pgfpathlineto{\pgfqpoint{4.391742in}{0.668751in}}%
\pgfpathlineto{\pgfqpoint{4.392039in}{0.668752in}}%
\pgfpathlineto{\pgfqpoint{4.392337in}{0.668752in}}%
\pgfpathlineto{\pgfqpoint{4.392634in}{0.668753in}}%
\pgfpathlineto{\pgfqpoint{4.392932in}{0.668753in}}%
\pgfpathlineto{\pgfqpoint{4.393229in}{0.668754in}}%
\pgfpathlineto{\pgfqpoint{4.393527in}{0.668754in}}%
\pgfpathlineto{\pgfqpoint{4.393824in}{0.668755in}}%
\pgfpathlineto{\pgfqpoint{4.394122in}{0.668755in}}%
\pgfpathlineto{\pgfqpoint{4.394419in}{0.668756in}}%
\pgfpathlineto{\pgfqpoint{4.394716in}{0.668757in}}%
\pgfpathlineto{\pgfqpoint{4.395014in}{0.668757in}}%
\pgfpathlineto{\pgfqpoint{4.395311in}{0.668758in}}%
\pgfpathlineto{\pgfqpoint{4.395609in}{0.668758in}}%
\pgfpathlineto{\pgfqpoint{4.395906in}{0.668759in}}%
\pgfpathlineto{\pgfqpoint{4.396204in}{0.668759in}}%
\pgfpathlineto{\pgfqpoint{4.396501in}{0.668760in}}%
\pgfpathlineto{\pgfqpoint{4.396799in}{0.668760in}}%
\pgfpathlineto{\pgfqpoint{4.397096in}{0.668761in}}%
\pgfpathlineto{\pgfqpoint{4.397394in}{0.668762in}}%
\pgfpathlineto{\pgfqpoint{4.397691in}{0.668762in}}%
\pgfpathlineto{\pgfqpoint{4.397989in}{0.668763in}}%
\pgfpathlineto{\pgfqpoint{4.398286in}{0.668763in}}%
\pgfpathlineto{\pgfqpoint{4.398584in}{0.668763in}}%
\pgfpathlineto{\pgfqpoint{4.398881in}{0.668763in}}%
\pgfpathlineto{\pgfqpoint{4.399179in}{0.668764in}}%
\pgfpathlineto{\pgfqpoint{4.399476in}{0.668765in}}%
\pgfpathlineto{\pgfqpoint{4.399774in}{0.668766in}}%
\pgfpathlineto{\pgfqpoint{4.400071in}{0.668767in}}%
\pgfpathlineto{\pgfqpoint{4.400369in}{0.668768in}}%
\pgfpathlineto{\pgfqpoint{4.400666in}{0.668768in}}%
\pgfpathlineto{\pgfqpoint{4.400964in}{0.668769in}}%
\pgfpathlineto{\pgfqpoint{4.401261in}{0.668770in}}%
\pgfpathlineto{\pgfqpoint{4.401558in}{0.668771in}}%
\pgfpathlineto{\pgfqpoint{4.401856in}{0.668772in}}%
\pgfpathlineto{\pgfqpoint{4.402153in}{0.668773in}}%
\pgfpathlineto{\pgfqpoint{4.402451in}{0.668773in}}%
\pgfpathlineto{\pgfqpoint{4.402748in}{0.668774in}}%
\pgfpathlineto{\pgfqpoint{4.403046in}{0.668775in}}%
\pgfpathlineto{\pgfqpoint{4.403343in}{0.668776in}}%
\pgfpathlineto{\pgfqpoint{4.403641in}{0.668777in}}%
\pgfpathlineto{\pgfqpoint{4.403938in}{0.668778in}}%
\pgfpathlineto{\pgfqpoint{4.404236in}{0.668779in}}%
\pgfpathlineto{\pgfqpoint{4.404533in}{0.668779in}}%
\pgfpathlineto{\pgfqpoint{4.404831in}{0.668780in}}%
\pgfpathlineto{\pgfqpoint{4.405128in}{0.668781in}}%
\pgfpathlineto{\pgfqpoint{4.405426in}{0.668782in}}%
\pgfpathlineto{\pgfqpoint{4.405723in}{0.668783in}}%
\pgfpathlineto{\pgfqpoint{4.406021in}{0.668784in}}%
\pgfpathlineto{\pgfqpoint{4.406318in}{0.668785in}}%
\pgfpathlineto{\pgfqpoint{4.406616in}{0.668785in}}%
\pgfpathlineto{\pgfqpoint{4.406913in}{0.668786in}}%
\pgfpathlineto{\pgfqpoint{4.407211in}{0.668787in}}%
\pgfpathlineto{\pgfqpoint{4.407508in}{0.668788in}}%
\pgfpathlineto{\pgfqpoint{4.407806in}{0.668789in}}%
\pgfpathlineto{\pgfqpoint{4.408103in}{0.668790in}}%
\pgfpathlineto{\pgfqpoint{4.408400in}{0.668791in}}%
\pgfpathlineto{\pgfqpoint{4.408698in}{0.668791in}}%
\pgfpathlineto{\pgfqpoint{4.408995in}{0.668792in}}%
\pgfpathlineto{\pgfqpoint{4.409293in}{0.668793in}}%
\pgfpathlineto{\pgfqpoint{4.409590in}{0.668794in}}%
\pgfpathlineto{\pgfqpoint{4.409888in}{0.668795in}}%
\pgfpathlineto{\pgfqpoint{4.410185in}{0.668796in}}%
\pgfpathlineto{\pgfqpoint{4.410483in}{0.668796in}}%
\pgfpathlineto{\pgfqpoint{4.410780in}{0.668797in}}%
\pgfpathlineto{\pgfqpoint{4.411078in}{0.668798in}}%
\pgfpathlineto{\pgfqpoint{4.411375in}{0.668799in}}%
\pgfpathlineto{\pgfqpoint{4.411673in}{0.668800in}}%
\pgfpathlineto{\pgfqpoint{4.411970in}{0.668801in}}%
\pgfpathlineto{\pgfqpoint{4.412268in}{0.668802in}}%
\pgfpathlineto{\pgfqpoint{4.412565in}{0.668802in}}%
\pgfpathlineto{\pgfqpoint{4.412863in}{0.668803in}}%
\pgfpathlineto{\pgfqpoint{4.413160in}{0.668804in}}%
\pgfpathlineto{\pgfqpoint{4.413458in}{0.668805in}}%
\pgfpathlineto{\pgfqpoint{4.413755in}{0.668806in}}%
\pgfpathlineto{\pgfqpoint{4.414053in}{0.668807in}}%
\pgfpathlineto{\pgfqpoint{4.414350in}{0.668808in}}%
\pgfpathlineto{\pgfqpoint{4.414647in}{0.668808in}}%
\pgfpathlineto{\pgfqpoint{4.414945in}{0.668809in}}%
\pgfpathlineto{\pgfqpoint{4.415242in}{0.668810in}}%
\pgfpathlineto{\pgfqpoint{4.415540in}{0.668811in}}%
\pgfpathlineto{\pgfqpoint{4.415837in}{0.668812in}}%
\pgfpathlineto{\pgfqpoint{4.416135in}{0.668813in}}%
\pgfpathlineto{\pgfqpoint{4.416432in}{0.668814in}}%
\pgfpathlineto{\pgfqpoint{4.416730in}{0.668814in}}%
\pgfpathlineto{\pgfqpoint{4.417027in}{0.668815in}}%
\pgfpathlineto{\pgfqpoint{4.417325in}{0.668816in}}%
\pgfpathlineto{\pgfqpoint{4.417622in}{0.668817in}}%
\pgfpathlineto{\pgfqpoint{4.417920in}{0.668818in}}%
\pgfpathlineto{\pgfqpoint{4.418217in}{0.668819in}}%
\pgfpathlineto{\pgfqpoint{4.418515in}{0.668819in}}%
\pgfpathlineto{\pgfqpoint{4.418812in}{0.668820in}}%
\pgfpathlineto{\pgfqpoint{4.419110in}{0.668821in}}%
\pgfpathlineto{\pgfqpoint{4.419407in}{0.668822in}}%
\pgfpathlineto{\pgfqpoint{4.419705in}{0.668823in}}%
\pgfpathlineto{\pgfqpoint{4.420002in}{0.668824in}}%
\pgfpathlineto{\pgfqpoint{4.420300in}{0.668825in}}%
\pgfpathlineto{\pgfqpoint{4.420597in}{0.668825in}}%
\pgfpathlineto{\pgfqpoint{4.420895in}{0.668826in}}%
\pgfpathlineto{\pgfqpoint{4.421192in}{0.668827in}}%
\pgfpathlineto{\pgfqpoint{4.421489in}{0.668828in}}%
\pgfpathlineto{\pgfqpoint{4.421787in}{0.668829in}}%
\pgfpathlineto{\pgfqpoint{4.422084in}{0.668830in}}%
\pgfpathlineto{\pgfqpoint{4.422382in}{0.668831in}}%
\pgfpathlineto{\pgfqpoint{4.422679in}{0.668831in}}%
\pgfpathlineto{\pgfqpoint{4.422977in}{0.668832in}}%
\pgfpathlineto{\pgfqpoint{4.423274in}{0.668833in}}%
\pgfpathlineto{\pgfqpoint{4.423572in}{0.668834in}}%
\pgfpathlineto{\pgfqpoint{4.423869in}{0.668835in}}%
\pgfpathlineto{\pgfqpoint{4.424167in}{0.668836in}}%
\pgfpathlineto{\pgfqpoint{4.424464in}{0.668837in}}%
\pgfpathlineto{\pgfqpoint{4.424762in}{0.668837in}}%
\pgfpathlineto{\pgfqpoint{4.425059in}{0.668838in}}%
\pgfpathlineto{\pgfqpoint{4.425357in}{0.668839in}}%
\pgfpathlineto{\pgfqpoint{4.425654in}{0.668840in}}%
\pgfpathlineto{\pgfqpoint{4.425952in}{0.668841in}}%
\pgfpathlineto{\pgfqpoint{4.426249in}{0.668842in}}%
\pgfpathlineto{\pgfqpoint{4.426547in}{0.668842in}}%
\pgfpathlineto{\pgfqpoint{4.426844in}{0.668843in}}%
\pgfpathlineto{\pgfqpoint{4.427142in}{0.668844in}}%
\pgfpathlineto{\pgfqpoint{4.427439in}{0.668845in}}%
\pgfpathlineto{\pgfqpoint{4.427737in}{0.668846in}}%
\pgfpathlineto{\pgfqpoint{4.428034in}{0.668847in}}%
\pgfpathlineto{\pgfqpoint{4.428331in}{0.668848in}}%
\pgfpathlineto{\pgfqpoint{4.428629in}{0.668848in}}%
\pgfpathlineto{\pgfqpoint{4.428926in}{0.668849in}}%
\pgfpathlineto{\pgfqpoint{4.429224in}{0.668850in}}%
\pgfpathlineto{\pgfqpoint{4.429521in}{0.668851in}}%
\pgfpathlineto{\pgfqpoint{4.429819in}{0.668852in}}%
\pgfpathlineto{\pgfqpoint{4.430116in}{0.668853in}}%
\pgfpathlineto{\pgfqpoint{4.430414in}{0.668854in}}%
\pgfpathlineto{\pgfqpoint{4.430711in}{0.668854in}}%
\pgfpathlineto{\pgfqpoint{4.431009in}{0.668855in}}%
\pgfpathlineto{\pgfqpoint{4.431306in}{0.668856in}}%
\pgfpathlineto{\pgfqpoint{4.431604in}{0.668857in}}%
\pgfpathlineto{\pgfqpoint{4.431901in}{0.668858in}}%
\pgfpathlineto{\pgfqpoint{4.432199in}{0.668859in}}%
\pgfpathlineto{\pgfqpoint{4.432496in}{0.668860in}}%
\pgfpathlineto{\pgfqpoint{4.432794in}{0.668860in}}%
\pgfpathlineto{\pgfqpoint{4.433091in}{0.668861in}}%
\pgfpathlineto{\pgfqpoint{4.433389in}{0.668862in}}%
\pgfpathlineto{\pgfqpoint{4.433686in}{0.668863in}}%
\pgfpathlineto{\pgfqpoint{4.433984in}{0.668864in}}%
\pgfpathlineto{\pgfqpoint{4.434281in}{0.668865in}}%
\pgfpathlineto{\pgfqpoint{4.434578in}{0.668865in}}%
\pgfpathlineto{\pgfqpoint{4.434876in}{0.668866in}}%
\pgfpathlineto{\pgfqpoint{4.435173in}{0.668867in}}%
\pgfpathlineto{\pgfqpoint{4.435471in}{0.668868in}}%
\pgfpathlineto{\pgfqpoint{4.435768in}{0.668869in}}%
\pgfpathlineto{\pgfqpoint{4.436066in}{0.668870in}}%
\pgfpathlineto{\pgfqpoint{4.436363in}{0.668871in}}%
\pgfpathlineto{\pgfqpoint{4.436661in}{0.668871in}}%
\pgfpathlineto{\pgfqpoint{4.436958in}{0.668872in}}%
\pgfpathlineto{\pgfqpoint{4.437256in}{0.668873in}}%
\pgfpathlineto{\pgfqpoint{4.437553in}{0.668874in}}%
\pgfpathlineto{\pgfqpoint{4.437851in}{0.668875in}}%
\pgfpathlineto{\pgfqpoint{4.438148in}{0.668876in}}%
\pgfpathlineto{\pgfqpoint{4.438446in}{0.668877in}}%
\pgfpathlineto{\pgfqpoint{4.438743in}{0.668877in}}%
\pgfpathlineto{\pgfqpoint{4.439041in}{0.668878in}}%
\pgfpathlineto{\pgfqpoint{4.439338in}{0.668879in}}%
\pgfpathlineto{\pgfqpoint{4.439636in}{0.668880in}}%
\pgfpathlineto{\pgfqpoint{4.439933in}{0.668881in}}%
\pgfpathlineto{\pgfqpoint{4.440231in}{0.668882in}}%
\pgfpathlineto{\pgfqpoint{4.440528in}{0.668883in}}%
\pgfpathlineto{\pgfqpoint{4.440826in}{0.668883in}}%
\pgfpathlineto{\pgfqpoint{4.441123in}{0.668884in}}%
\pgfpathlineto{\pgfqpoint{4.441420in}{0.668885in}}%
\pgfpathlineto{\pgfqpoint{4.441718in}{0.668886in}}%
\pgfpathlineto{\pgfqpoint{4.442015in}{0.668887in}}%
\pgfpathlineto{\pgfqpoint{4.442313in}{0.668888in}}%
\pgfpathlineto{\pgfqpoint{4.442610in}{0.668888in}}%
\pgfpathlineto{\pgfqpoint{4.442908in}{0.668889in}}%
\pgfpathlineto{\pgfqpoint{4.443205in}{0.668890in}}%
\pgfpathlineto{\pgfqpoint{4.443503in}{0.668891in}}%
\pgfpathlineto{\pgfqpoint{4.443800in}{0.668892in}}%
\pgfpathlineto{\pgfqpoint{4.444098in}{0.668893in}}%
\pgfpathlineto{\pgfqpoint{4.444395in}{0.668894in}}%
\pgfpathlineto{\pgfqpoint{4.444693in}{0.668894in}}%
\pgfpathlineto{\pgfqpoint{4.444990in}{0.668895in}}%
\pgfpathlineto{\pgfqpoint{4.445288in}{0.668896in}}%
\pgfpathlineto{\pgfqpoint{4.445585in}{0.668897in}}%
\pgfpathlineto{\pgfqpoint{4.445883in}{0.668898in}}%
\pgfpathlineto{\pgfqpoint{4.446180in}{0.668899in}}%
\pgfpathlineto{\pgfqpoint{4.446478in}{0.668900in}}%
\pgfpathlineto{\pgfqpoint{4.446775in}{0.668900in}}%
\pgfpathlineto{\pgfqpoint{4.447073in}{0.668901in}}%
\pgfpathlineto{\pgfqpoint{4.447370in}{0.668902in}}%
\pgfpathlineto{\pgfqpoint{4.447668in}{0.668903in}}%
\pgfpathlineto{\pgfqpoint{4.447965in}{0.668904in}}%
\pgfpathlineto{\pgfqpoint{4.448262in}{0.668905in}}%
\pgfpathlineto{\pgfqpoint{4.448560in}{0.668906in}}%
\pgfpathlineto{\pgfqpoint{4.448857in}{0.668906in}}%
\pgfpathlineto{\pgfqpoint{4.449155in}{0.668907in}}%
\pgfpathlineto{\pgfqpoint{4.449452in}{0.668908in}}%
\pgfpathlineto{\pgfqpoint{4.449750in}{0.668909in}}%
\pgfpathlineto{\pgfqpoint{4.450047in}{0.668910in}}%
\pgfpathlineto{\pgfqpoint{4.450345in}{0.668911in}}%
\pgfpathlineto{\pgfqpoint{4.450642in}{0.668911in}}%
\pgfpathlineto{\pgfqpoint{4.450940in}{0.668912in}}%
\pgfpathlineto{\pgfqpoint{4.451237in}{0.668913in}}%
\pgfpathlineto{\pgfqpoint{4.451535in}{0.668914in}}%
\pgfpathlineto{\pgfqpoint{4.451832in}{0.668915in}}%
\pgfpathlineto{\pgfqpoint{4.452130in}{0.668916in}}%
\pgfpathlineto{\pgfqpoint{4.452427in}{0.668917in}}%
\pgfpathlineto{\pgfqpoint{4.452725in}{0.668917in}}%
\pgfpathlineto{\pgfqpoint{4.453022in}{0.668918in}}%
\pgfpathlineto{\pgfqpoint{4.453320in}{0.668919in}}%
\pgfpathlineto{\pgfqpoint{4.453617in}{0.668920in}}%
\pgfpathlineto{\pgfqpoint{4.453915in}{0.668921in}}%
\pgfpathlineto{\pgfqpoint{4.454212in}{0.668922in}}%
\pgfpathlineto{\pgfqpoint{4.454509in}{0.668923in}}%
\pgfpathlineto{\pgfqpoint{4.454807in}{0.668923in}}%
\pgfpathlineto{\pgfqpoint{4.455104in}{0.668924in}}%
\pgfpathlineto{\pgfqpoint{4.455402in}{0.668925in}}%
\pgfpathlineto{\pgfqpoint{4.455699in}{0.668926in}}%
\pgfpathlineto{\pgfqpoint{4.455997in}{0.668927in}}%
\pgfpathlineto{\pgfqpoint{4.456294in}{0.668928in}}%
\pgfpathlineto{\pgfqpoint{4.456592in}{0.668929in}}%
\pgfpathlineto{\pgfqpoint{4.456889in}{0.668929in}}%
\pgfpathlineto{\pgfqpoint{4.457187in}{0.668930in}}%
\pgfpathlineto{\pgfqpoint{4.457484in}{0.668931in}}%
\pgfpathlineto{\pgfqpoint{4.457782in}{0.668932in}}%
\pgfpathlineto{\pgfqpoint{4.458079in}{0.668933in}}%
\pgfpathlineto{\pgfqpoint{4.458377in}{0.668934in}}%
\pgfpathlineto{\pgfqpoint{4.458674in}{0.668934in}}%
\pgfpathlineto{\pgfqpoint{4.458972in}{0.668935in}}%
\pgfpathlineto{\pgfqpoint{4.459269in}{0.668936in}}%
\pgfpathlineto{\pgfqpoint{4.459567in}{0.668937in}}%
\pgfpathlineto{\pgfqpoint{4.459864in}{0.668938in}}%
\pgfpathlineto{\pgfqpoint{4.460162in}{0.668939in}}%
\pgfpathlineto{\pgfqpoint{4.460459in}{0.668940in}}%
\pgfpathlineto{\pgfqpoint{4.460757in}{0.668940in}}%
\pgfpathlineto{\pgfqpoint{4.461054in}{0.668941in}}%
\pgfpathlineto{\pgfqpoint{4.461351in}{0.668942in}}%
\pgfpathlineto{\pgfqpoint{4.461649in}{0.668943in}}%
\pgfpathlineto{\pgfqpoint{4.461946in}{0.668944in}}%
\pgfpathlineto{\pgfqpoint{4.462244in}{0.668945in}}%
\pgfpathlineto{\pgfqpoint{4.462541in}{0.668946in}}%
\pgfpathlineto{\pgfqpoint{4.462839in}{0.668946in}}%
\pgfpathlineto{\pgfqpoint{4.463136in}{0.668947in}}%
\pgfpathlineto{\pgfqpoint{4.463434in}{0.668948in}}%
\pgfpathlineto{\pgfqpoint{4.463731in}{0.668949in}}%
\pgfpathlineto{\pgfqpoint{4.464029in}{0.668950in}}%
\pgfpathlineto{\pgfqpoint{4.464326in}{0.668951in}}%
\pgfpathlineto{\pgfqpoint{4.464624in}{0.668952in}}%
\pgfpathlineto{\pgfqpoint{4.464921in}{0.668952in}}%
\pgfpathlineto{\pgfqpoint{4.465219in}{0.668953in}}%
\pgfpathlineto{\pgfqpoint{4.465516in}{0.668954in}}%
\pgfpathlineto{\pgfqpoint{4.465814in}{0.668955in}}%
\pgfpathlineto{\pgfqpoint{4.466111in}{0.668956in}}%
\pgfpathlineto{\pgfqpoint{4.466409in}{0.668957in}}%
\pgfpathlineto{\pgfqpoint{4.466706in}{0.668957in}}%
\pgfpathlineto{\pgfqpoint{4.467004in}{0.668958in}}%
\pgfpathlineto{\pgfqpoint{4.467301in}{0.668959in}}%
\pgfpathlineto{\pgfqpoint{4.467599in}{0.668960in}}%
\pgfpathlineto{\pgfqpoint{4.467896in}{0.668961in}}%
\pgfpathlineto{\pgfqpoint{4.468193in}{0.668962in}}%
\pgfpathlineto{\pgfqpoint{4.468491in}{0.668963in}}%
\pgfpathlineto{\pgfqpoint{4.468788in}{0.668963in}}%
\pgfpathlineto{\pgfqpoint{4.469086in}{0.668964in}}%
\pgfpathlineto{\pgfqpoint{4.469383in}{0.668965in}}%
\pgfpathlineto{\pgfqpoint{4.469681in}{0.668966in}}%
\pgfpathlineto{\pgfqpoint{4.469978in}{0.668967in}}%
\pgfpathlineto{\pgfqpoint{4.470276in}{0.668968in}}%
\pgfpathlineto{\pgfqpoint{4.470573in}{0.668969in}}%
\pgfpathlineto{\pgfqpoint{4.470871in}{0.668969in}}%
\pgfpathlineto{\pgfqpoint{4.471168in}{0.668970in}}%
\pgfpathlineto{\pgfqpoint{4.471466in}{0.668971in}}%
\pgfpathlineto{\pgfqpoint{4.471763in}{0.668972in}}%
\pgfpathlineto{\pgfqpoint{4.472061in}{0.668973in}}%
\pgfpathlineto{\pgfqpoint{4.472358in}{0.668974in}}%
\pgfpathlineto{\pgfqpoint{4.472656in}{0.668975in}}%
\pgfpathlineto{\pgfqpoint{4.472953in}{0.668975in}}%
\pgfpathlineto{\pgfqpoint{4.473251in}{0.668976in}}%
\pgfpathlineto{\pgfqpoint{4.473548in}{0.668977in}}%
\pgfpathlineto{\pgfqpoint{4.473846in}{0.668978in}}%
\pgfpathlineto{\pgfqpoint{4.474143in}{0.668979in}}%
\pgfpathlineto{\pgfqpoint{4.474440in}{0.668980in}}%
\pgfpathlineto{\pgfqpoint{4.474738in}{0.668980in}}%
\pgfpathlineto{\pgfqpoint{4.475035in}{0.668981in}}%
\pgfpathlineto{\pgfqpoint{4.475333in}{0.668982in}}%
\pgfpathlineto{\pgfqpoint{4.475630in}{0.668983in}}%
\pgfpathlineto{\pgfqpoint{4.475928in}{0.668984in}}%
\pgfpathlineto{\pgfqpoint{4.476225in}{0.668985in}}%
\pgfpathlineto{\pgfqpoint{4.476523in}{0.668986in}}%
\pgfpathlineto{\pgfqpoint{4.476820in}{0.668986in}}%
\pgfpathlineto{\pgfqpoint{4.477118in}{0.668987in}}%
\pgfpathlineto{\pgfqpoint{4.477415in}{0.668988in}}%
\pgfpathlineto{\pgfqpoint{4.477713in}{0.668989in}}%
\pgfpathlineto{\pgfqpoint{4.478010in}{0.668990in}}%
\pgfpathlineto{\pgfqpoint{4.478308in}{0.668991in}}%
\pgfpathlineto{\pgfqpoint{4.478605in}{0.668992in}}%
\pgfpathlineto{\pgfqpoint{4.478903in}{0.668992in}}%
\pgfpathlineto{\pgfqpoint{4.479200in}{0.668993in}}%
\pgfpathlineto{\pgfqpoint{4.479498in}{0.668994in}}%
\pgfpathlineto{\pgfqpoint{4.479795in}{0.668995in}}%
\pgfpathlineto{\pgfqpoint{4.480093in}{0.668996in}}%
\pgfpathlineto{\pgfqpoint{4.480390in}{0.668997in}}%
\pgfpathlineto{\pgfqpoint{4.480688in}{0.668998in}}%
\pgfpathlineto{\pgfqpoint{4.480985in}{0.668998in}}%
\pgfpathlineto{\pgfqpoint{4.481282in}{0.668999in}}%
\pgfpathlineto{\pgfqpoint{4.481580in}{0.669000in}}%
\pgfpathlineto{\pgfqpoint{4.481877in}{0.669001in}}%
\pgfpathlineto{\pgfqpoint{4.482175in}{0.669002in}}%
\pgfpathlineto{\pgfqpoint{4.482472in}{0.669003in}}%
\pgfpathlineto{\pgfqpoint{4.482770in}{0.669003in}}%
\pgfpathlineto{\pgfqpoint{4.483067in}{0.669004in}}%
\pgfpathlineto{\pgfqpoint{4.483365in}{0.669005in}}%
\pgfpathlineto{\pgfqpoint{4.483662in}{0.669229in}}%
\pgfpathlineto{\pgfqpoint{4.483960in}{0.669877in}}%
\pgfpathlineto{\pgfqpoint{4.484257in}{0.670534in}}%
\pgfpathlineto{\pgfqpoint{4.484555in}{0.671191in}}%
\pgfpathlineto{\pgfqpoint{4.484852in}{0.671848in}}%
\pgfpathlineto{\pgfqpoint{4.485150in}{0.672506in}}%
\pgfpathlineto{\pgfqpoint{4.485447in}{0.673163in}}%
\pgfpathlineto{\pgfqpoint{4.485745in}{0.673820in}}%
\pgfpathlineto{\pgfqpoint{4.486042in}{0.674477in}}%
\pgfpathlineto{\pgfqpoint{4.486340in}{0.675135in}}%
\pgfpathlineto{\pgfqpoint{4.486637in}{0.675792in}}%
\pgfpathlineto{\pgfqpoint{4.486935in}{0.676449in}}%
\pgfpathlineto{\pgfqpoint{4.487232in}{0.677106in}}%
\pgfpathlineto{\pgfqpoint{4.487530in}{0.677764in}}%
\pgfpathlineto{\pgfqpoint{4.487827in}{0.678421in}}%
\pgfpathlineto{\pgfqpoint{4.488124in}{0.679078in}}%
\pgfpathlineto{\pgfqpoint{4.488422in}{0.679735in}}%
\pgfpathlineto{\pgfqpoint{4.488719in}{0.680393in}}%
\pgfpathlineto{\pgfqpoint{4.489017in}{0.681050in}}%
\pgfpathlineto{\pgfqpoint{4.489314in}{0.681707in}}%
\pgfpathlineto{\pgfqpoint{4.489612in}{0.682364in}}%
\pgfpathlineto{\pgfqpoint{4.489909in}{0.683021in}}%
\pgfpathlineto{\pgfqpoint{4.490207in}{0.683679in}}%
\pgfpathlineto{\pgfqpoint{4.490504in}{0.684336in}}%
\pgfpathlineto{\pgfqpoint{4.490802in}{0.684993in}}%
\pgfpathlineto{\pgfqpoint{4.491099in}{0.685554in}}%
\pgfpathlineto{\pgfqpoint{4.491397in}{0.685921in}}%
\pgfpathlineto{\pgfqpoint{4.491694in}{0.686283in}}%
\pgfpathlineto{\pgfqpoint{4.491992in}{0.686644in}}%
\pgfpathlineto{\pgfqpoint{4.492289in}{0.686867in}}%
\pgfpathlineto{\pgfqpoint{4.492587in}{0.686870in}}%
\pgfpathlineto{\pgfqpoint{4.492884in}{0.686869in}}%
\pgfpathlineto{\pgfqpoint{4.493182in}{0.686869in}}%
\pgfpathlineto{\pgfqpoint{4.493479in}{0.686869in}}%
\pgfpathlineto{\pgfqpoint{4.493777in}{0.686868in}}%
\pgfpathlineto{\pgfqpoint{4.494074in}{0.686868in}}%
\pgfpathlineto{\pgfqpoint{4.494371in}{0.686868in}}%
\pgfpathlineto{\pgfqpoint{4.494669in}{0.686867in}}%
\pgfpathlineto{\pgfqpoint{4.494966in}{0.686867in}}%
\pgfpathlineto{\pgfqpoint{4.495264in}{0.686867in}}%
\pgfpathlineto{\pgfqpoint{4.495561in}{0.686866in}}%
\pgfpathlineto{\pgfqpoint{4.495859in}{0.686866in}}%
\pgfpathlineto{\pgfqpoint{4.496156in}{0.686866in}}%
\pgfpathlineto{\pgfqpoint{4.496454in}{0.686865in}}%
\pgfpathlineto{\pgfqpoint{4.496751in}{0.686865in}}%
\pgfpathlineto{\pgfqpoint{4.497049in}{0.686865in}}%
\pgfpathlineto{\pgfqpoint{4.497346in}{0.686865in}}%
\pgfpathlineto{\pgfqpoint{4.497644in}{0.686864in}}%
\pgfpathlineto{\pgfqpoint{4.497941in}{0.686864in}}%
\pgfpathlineto{\pgfqpoint{4.498239in}{0.686864in}}%
\pgfpathlineto{\pgfqpoint{4.498536in}{0.686863in}}%
\pgfpathlineto{\pgfqpoint{4.498834in}{0.686863in}}%
\pgfpathlineto{\pgfqpoint{4.499131in}{0.686863in}}%
\pgfpathlineto{\pgfqpoint{4.499429in}{0.686862in}}%
\pgfpathlineto{\pgfqpoint{4.499726in}{0.686862in}}%
\pgfpathlineto{\pgfqpoint{4.500024in}{0.686862in}}%
\pgfpathlineto{\pgfqpoint{4.500321in}{0.686861in}}%
\pgfpathlineto{\pgfqpoint{4.500619in}{0.686861in}}%
\pgfpathlineto{\pgfqpoint{4.500916in}{0.686861in}}%
\pgfpathlineto{\pgfqpoint{4.501213in}{0.686860in}}%
\pgfpathlineto{\pgfqpoint{4.501511in}{0.686860in}}%
\pgfpathlineto{\pgfqpoint{4.501808in}{0.686860in}}%
\pgfpathlineto{\pgfqpoint{4.502106in}{0.686859in}}%
\pgfpathlineto{\pgfqpoint{4.502403in}{0.686859in}}%
\pgfpathlineto{\pgfqpoint{4.502701in}{0.686859in}}%
\pgfpathlineto{\pgfqpoint{4.502998in}{0.686858in}}%
\pgfpathlineto{\pgfqpoint{4.503296in}{0.686858in}}%
\pgfpathlineto{\pgfqpoint{4.503593in}{0.686858in}}%
\pgfpathlineto{\pgfqpoint{4.503891in}{0.686857in}}%
\pgfpathlineto{\pgfqpoint{4.504188in}{0.686857in}}%
\pgfpathlineto{\pgfqpoint{4.504486in}{0.686857in}}%
\pgfpathlineto{\pgfqpoint{4.504783in}{0.686856in}}%
\pgfpathlineto{\pgfqpoint{4.505081in}{0.686856in}}%
\pgfpathlineto{\pgfqpoint{4.505378in}{0.686856in}}%
\pgfpathlineto{\pgfqpoint{4.505676in}{0.686855in}}%
\pgfpathlineto{\pgfqpoint{4.505973in}{0.686864in}}%
\pgfpathlineto{\pgfqpoint{4.506271in}{0.686885in}}%
\pgfpathlineto{\pgfqpoint{4.506568in}{0.686907in}}%
\pgfpathlineto{\pgfqpoint{4.506866in}{0.686929in}}%
\pgfpathlineto{\pgfqpoint{4.507163in}{0.686950in}}%
\pgfpathlineto{\pgfqpoint{4.507461in}{0.686972in}}%
\pgfpathlineto{\pgfqpoint{4.507758in}{0.686993in}}%
\pgfpathlineto{\pgfqpoint{4.508055in}{0.687015in}}%
\pgfpathlineto{\pgfqpoint{4.508353in}{0.687037in}}%
\pgfpathlineto{\pgfqpoint{4.508650in}{0.687058in}}%
\pgfpathlineto{\pgfqpoint{4.508948in}{0.687080in}}%
\pgfpathlineto{\pgfqpoint{4.509245in}{0.687102in}}%
\pgfpathlineto{\pgfqpoint{4.509543in}{0.687123in}}%
\pgfpathlineto{\pgfqpoint{4.509840in}{0.687145in}}%
\pgfpathlineto{\pgfqpoint{4.510138in}{0.687166in}}%
\pgfpathlineto{\pgfqpoint{4.510435in}{0.687188in}}%
\pgfpathlineto{\pgfqpoint{4.510733in}{0.687210in}}%
\pgfpathlineto{\pgfqpoint{4.511030in}{0.687231in}}%
\pgfpathlineto{\pgfqpoint{4.511328in}{0.687253in}}%
\pgfpathlineto{\pgfqpoint{4.511625in}{0.687275in}}%
\pgfpathlineto{\pgfqpoint{4.511923in}{0.687296in}}%
\pgfpathlineto{\pgfqpoint{4.512220in}{0.687318in}}%
\pgfpathlineto{\pgfqpoint{4.512518in}{0.687339in}}%
\pgfpathlineto{\pgfqpoint{4.512815in}{0.687361in}}%
\pgfpathlineto{\pgfqpoint{4.513113in}{0.687383in}}%
\pgfpathlineto{\pgfqpoint{4.513410in}{0.687404in}}%
\pgfpathlineto{\pgfqpoint{4.513708in}{0.687426in}}%
\pgfpathlineto{\pgfqpoint{4.514005in}{0.687447in}}%
\pgfpathlineto{\pgfqpoint{4.514302in}{0.687469in}}%
\pgfpathlineto{\pgfqpoint{4.514600in}{0.687491in}}%
\pgfpathlineto{\pgfqpoint{4.514897in}{0.687512in}}%
\pgfpathlineto{\pgfqpoint{4.515195in}{0.687534in}}%
\pgfpathlineto{\pgfqpoint{4.515492in}{0.687556in}}%
\pgfpathlineto{\pgfqpoint{4.515790in}{0.687577in}}%
\pgfpathlineto{\pgfqpoint{4.516087in}{0.687599in}}%
\pgfpathlineto{\pgfqpoint{4.516385in}{0.687620in}}%
\pgfpathlineto{\pgfqpoint{4.516682in}{0.687642in}}%
\pgfpathlineto{\pgfqpoint{4.516980in}{0.687664in}}%
\pgfpathlineto{\pgfqpoint{4.517277in}{0.687685in}}%
\pgfpathlineto{\pgfqpoint{4.517575in}{0.687707in}}%
\pgfpathlineto{\pgfqpoint{4.517872in}{0.687729in}}%
\pgfpathlineto{\pgfqpoint{4.518170in}{0.687750in}}%
\pgfpathlineto{\pgfqpoint{4.518467in}{0.687772in}}%
\pgfpathlineto{\pgfqpoint{4.518765in}{0.687793in}}%
\pgfpathlineto{\pgfqpoint{4.519062in}{0.687815in}}%
\pgfpathlineto{\pgfqpoint{4.519360in}{0.687837in}}%
\pgfpathlineto{\pgfqpoint{4.519657in}{0.687858in}}%
\pgfpathlineto{\pgfqpoint{4.519955in}{0.687880in}}%
\pgfpathlineto{\pgfqpoint{4.520252in}{0.687902in}}%
\pgfpathlineto{\pgfqpoint{4.520550in}{0.687923in}}%
\pgfpathlineto{\pgfqpoint{4.520847in}{0.687945in}}%
\pgfpathlineto{\pgfqpoint{4.521144in}{0.687966in}}%
\pgfpathlineto{\pgfqpoint{4.521442in}{0.687988in}}%
\pgfpathlineto{\pgfqpoint{4.521739in}{0.688010in}}%
\pgfpathlineto{\pgfqpoint{4.522037in}{0.688031in}}%
\pgfpathlineto{\pgfqpoint{4.522334in}{0.688053in}}%
\pgfpathlineto{\pgfqpoint{4.522632in}{0.688074in}}%
\pgfpathlineto{\pgfqpoint{4.522929in}{0.688096in}}%
\pgfpathlineto{\pgfqpoint{4.523227in}{0.688118in}}%
\pgfpathlineto{\pgfqpoint{4.523524in}{0.688139in}}%
\pgfpathlineto{\pgfqpoint{4.523822in}{0.688161in}}%
\pgfpathlineto{\pgfqpoint{4.524119in}{0.688183in}}%
\pgfpathlineto{\pgfqpoint{4.524417in}{0.688204in}}%
\pgfpathlineto{\pgfqpoint{4.524714in}{0.688226in}}%
\pgfpathlineto{\pgfqpoint{4.525012in}{0.688247in}}%
\pgfpathlineto{\pgfqpoint{4.525309in}{0.688269in}}%
\pgfpathlineto{\pgfqpoint{4.525607in}{0.688291in}}%
\pgfpathlineto{\pgfqpoint{4.525904in}{0.688312in}}%
\pgfpathlineto{\pgfqpoint{4.526202in}{0.688334in}}%
\pgfpathlineto{\pgfqpoint{4.526499in}{0.688356in}}%
\pgfpathlineto{\pgfqpoint{4.526797in}{0.688377in}}%
\pgfpathlineto{\pgfqpoint{4.527094in}{0.688399in}}%
\pgfpathlineto{\pgfqpoint{4.527392in}{0.688420in}}%
\pgfpathlineto{\pgfqpoint{4.527689in}{0.688442in}}%
\pgfpathlineto{\pgfqpoint{4.527986in}{0.688464in}}%
\pgfpathlineto{\pgfqpoint{4.528284in}{0.688485in}}%
\pgfpathlineto{\pgfqpoint{4.528581in}{0.688507in}}%
\pgfpathlineto{\pgfqpoint{4.528879in}{0.688529in}}%
\pgfpathlineto{\pgfqpoint{4.529176in}{0.688550in}}%
\pgfpathlineto{\pgfqpoint{4.529474in}{0.688572in}}%
\pgfpathlineto{\pgfqpoint{4.529771in}{0.688593in}}%
\pgfpathlineto{\pgfqpoint{4.530069in}{0.688615in}}%
\pgfpathlineto{\pgfqpoint{4.530366in}{0.688637in}}%
\pgfpathlineto{\pgfqpoint{4.530664in}{0.688658in}}%
\pgfpathlineto{\pgfqpoint{4.530961in}{0.688680in}}%
\pgfpathlineto{\pgfqpoint{4.531259in}{0.688701in}}%
\pgfpathlineto{\pgfqpoint{4.531556in}{0.688723in}}%
\pgfpathlineto{\pgfqpoint{4.531854in}{0.688745in}}%
\pgfpathlineto{\pgfqpoint{4.532151in}{0.688766in}}%
\pgfpathlineto{\pgfqpoint{4.532449in}{0.688788in}}%
\pgfpathlineto{\pgfqpoint{4.532746in}{0.688810in}}%
\pgfpathlineto{\pgfqpoint{4.533044in}{0.688831in}}%
\pgfpathlineto{\pgfqpoint{4.533341in}{0.688853in}}%
\pgfpathlineto{\pgfqpoint{4.533639in}{0.688874in}}%
\pgfpathlineto{\pgfqpoint{4.533936in}{0.688896in}}%
\pgfpathlineto{\pgfqpoint{4.534233in}{0.688918in}}%
\pgfpathlineto{\pgfqpoint{4.534531in}{0.688939in}}%
\pgfpathlineto{\pgfqpoint{4.534828in}{0.688961in}}%
\pgfpathlineto{\pgfqpoint{4.535126in}{0.688983in}}%
\pgfpathlineto{\pgfqpoint{4.535423in}{0.689004in}}%
\pgfpathlineto{\pgfqpoint{4.535721in}{0.689026in}}%
\pgfpathlineto{\pgfqpoint{4.536018in}{0.689047in}}%
\pgfpathlineto{\pgfqpoint{4.536316in}{0.689069in}}%
\pgfpathlineto{\pgfqpoint{4.536613in}{0.689091in}}%
\pgfpathlineto{\pgfqpoint{4.536911in}{0.689112in}}%
\pgfpathlineto{\pgfqpoint{4.537208in}{0.689134in}}%
\pgfpathlineto{\pgfqpoint{4.537506in}{0.689155in}}%
\pgfpathlineto{\pgfqpoint{4.537803in}{0.689177in}}%
\pgfpathlineto{\pgfqpoint{4.538101in}{0.689199in}}%
\pgfpathlineto{\pgfqpoint{4.538398in}{0.689220in}}%
\pgfpathlineto{\pgfqpoint{4.538696in}{0.689242in}}%
\pgfpathlineto{\pgfqpoint{4.538993in}{0.689264in}}%
\pgfpathlineto{\pgfqpoint{4.539291in}{0.689285in}}%
\pgfpathlineto{\pgfqpoint{4.539588in}{0.689307in}}%
\pgfpathlineto{\pgfqpoint{4.539886in}{0.689328in}}%
\pgfpathlineto{\pgfqpoint{4.540183in}{0.689350in}}%
\pgfpathlineto{\pgfqpoint{4.540481in}{0.689372in}}%
\pgfpathlineto{\pgfqpoint{4.540778in}{0.689393in}}%
\pgfpathlineto{\pgfqpoint{4.541075in}{0.689415in}}%
\pgfpathlineto{\pgfqpoint{4.541373in}{0.689437in}}%
\pgfpathlineto{\pgfqpoint{4.541670in}{0.689458in}}%
\pgfpathlineto{\pgfqpoint{4.541968in}{0.689473in}}%
\pgfpathlineto{\pgfqpoint{4.542265in}{0.689485in}}%
\pgfpathlineto{\pgfqpoint{4.542563in}{0.689497in}}%
\pgfpathlineto{\pgfqpoint{4.542860in}{0.689508in}}%
\pgfpathlineto{\pgfqpoint{4.543158in}{0.689520in}}%
\pgfpathlineto{\pgfqpoint{4.543455in}{0.689532in}}%
\pgfpathlineto{\pgfqpoint{4.543753in}{0.689544in}}%
\pgfpathlineto{\pgfqpoint{4.544050in}{0.689555in}}%
\pgfpathlineto{\pgfqpoint{4.544348in}{0.689567in}}%
\pgfpathlineto{\pgfqpoint{4.544645in}{0.689579in}}%
\pgfpathlineto{\pgfqpoint{4.544943in}{0.689591in}}%
\pgfpathlineto{\pgfqpoint{4.545240in}{0.689602in}}%
\pgfpathlineto{\pgfqpoint{4.545538in}{0.689614in}}%
\pgfpathlineto{\pgfqpoint{4.545835in}{0.689626in}}%
\pgfpathlineto{\pgfqpoint{4.546133in}{0.689638in}}%
\pgfpathlineto{\pgfqpoint{4.546430in}{0.689649in}}%
\pgfpathlineto{\pgfqpoint{4.546728in}{0.689661in}}%
\pgfpathlineto{\pgfqpoint{4.547025in}{0.689673in}}%
\pgfpathlineto{\pgfqpoint{4.547323in}{0.689685in}}%
\pgfpathlineto{\pgfqpoint{4.547620in}{0.689696in}}%
\pgfpathlineto{\pgfqpoint{4.547917in}{0.689708in}}%
\pgfpathlineto{\pgfqpoint{4.548215in}{0.689720in}}%
\pgfpathlineto{\pgfqpoint{4.548512in}{0.689732in}}%
\pgfpathlineto{\pgfqpoint{4.548810in}{0.689743in}}%
\pgfpathlineto{\pgfqpoint{4.549107in}{0.689755in}}%
\pgfpathlineto{\pgfqpoint{4.549405in}{0.689767in}}%
\pgfpathlineto{\pgfqpoint{4.549702in}{0.689779in}}%
\pgfpathlineto{\pgfqpoint{4.550000in}{0.689790in}}%
\pgfpathlineto{\pgfqpoint{4.550297in}{0.689802in}}%
\pgfpathlineto{\pgfqpoint{4.550595in}{0.689814in}}%
\pgfpathlineto{\pgfqpoint{4.550892in}{0.689826in}}%
\pgfpathlineto{\pgfqpoint{4.551190in}{0.689837in}}%
\pgfpathlineto{\pgfqpoint{4.551487in}{0.689849in}}%
\pgfpathlineto{\pgfqpoint{4.551785in}{0.689861in}}%
\pgfpathlineto{\pgfqpoint{4.552082in}{0.689873in}}%
\pgfpathlineto{\pgfqpoint{4.552380in}{0.689884in}}%
\pgfpathlineto{\pgfqpoint{4.552677in}{0.689896in}}%
\pgfpathlineto{\pgfqpoint{4.552975in}{0.689908in}}%
\pgfpathlineto{\pgfqpoint{4.553272in}{0.689920in}}%
\pgfpathlineto{\pgfqpoint{4.553570in}{0.689931in}}%
\pgfpathlineto{\pgfqpoint{4.553867in}{0.689943in}}%
\pgfpathlineto{\pgfqpoint{4.554164in}{0.689955in}}%
\pgfpathlineto{\pgfqpoint{4.554462in}{0.689967in}}%
\pgfpathlineto{\pgfqpoint{4.554759in}{0.689979in}}%
\pgfpathlineto{\pgfqpoint{4.555057in}{0.689990in}}%
\pgfpathlineto{\pgfqpoint{4.555354in}{0.690002in}}%
\pgfpathlineto{\pgfqpoint{4.555652in}{0.690014in}}%
\pgfpathlineto{\pgfqpoint{4.555949in}{0.690026in}}%
\pgfpathlineto{\pgfqpoint{4.556247in}{0.690037in}}%
\pgfpathlineto{\pgfqpoint{4.556544in}{0.690049in}}%
\pgfpathlineto{\pgfqpoint{4.556842in}{0.690061in}}%
\pgfpathlineto{\pgfqpoint{4.557139in}{0.690073in}}%
\pgfpathlineto{\pgfqpoint{4.557437in}{0.690084in}}%
\pgfpathlineto{\pgfqpoint{4.557734in}{0.690096in}}%
\pgfpathlineto{\pgfqpoint{4.558032in}{0.690108in}}%
\pgfpathlineto{\pgfqpoint{4.558329in}{0.690120in}}%
\pgfpathlineto{\pgfqpoint{4.558627in}{0.690131in}}%
\pgfpathlineto{\pgfqpoint{4.558924in}{0.690143in}}%
\pgfpathlineto{\pgfqpoint{4.559222in}{0.690155in}}%
\pgfpathlineto{\pgfqpoint{4.559519in}{0.690167in}}%
\pgfpathlineto{\pgfqpoint{4.559817in}{0.690178in}}%
\pgfpathlineto{\pgfqpoint{4.560114in}{0.690190in}}%
\pgfpathlineto{\pgfqpoint{4.560412in}{0.690202in}}%
\pgfpathlineto{\pgfqpoint{4.560709in}{0.690214in}}%
\pgfpathlineto{\pgfqpoint{4.561006in}{0.690225in}}%
\pgfpathlineto{\pgfqpoint{4.561304in}{0.690237in}}%
\pgfpathlineto{\pgfqpoint{4.561601in}{0.690249in}}%
\pgfpathlineto{\pgfqpoint{4.561899in}{0.690261in}}%
\pgfpathlineto{\pgfqpoint{4.562196in}{0.690272in}}%
\pgfpathlineto{\pgfqpoint{4.562494in}{0.690284in}}%
\pgfpathlineto{\pgfqpoint{4.562791in}{0.690296in}}%
\pgfpathlineto{\pgfqpoint{4.563089in}{0.690308in}}%
\pgfpathlineto{\pgfqpoint{4.563386in}{0.690319in}}%
\pgfpathlineto{\pgfqpoint{4.563684in}{0.690331in}}%
\pgfpathlineto{\pgfqpoint{4.563981in}{0.690343in}}%
\pgfpathlineto{\pgfqpoint{4.564279in}{0.690355in}}%
\pgfpathlineto{\pgfqpoint{4.564576in}{0.690366in}}%
\pgfpathlineto{\pgfqpoint{4.564874in}{0.690378in}}%
\pgfpathlineto{\pgfqpoint{4.565171in}{0.690390in}}%
\pgfpathlineto{\pgfqpoint{4.565469in}{0.690402in}}%
\pgfpathlineto{\pgfqpoint{4.565766in}{0.690413in}}%
\pgfpathlineto{\pgfqpoint{4.566064in}{0.690425in}}%
\pgfpathlineto{\pgfqpoint{4.566361in}{0.690437in}}%
\pgfpathlineto{\pgfqpoint{4.566659in}{0.690449in}}%
\pgfpathlineto{\pgfqpoint{4.566956in}{0.690460in}}%
\pgfpathlineto{\pgfqpoint{4.567254in}{0.690472in}}%
\pgfpathlineto{\pgfqpoint{4.567551in}{0.690484in}}%
\pgfpathlineto{\pgfqpoint{4.567848in}{0.690496in}}%
\pgfpathlineto{\pgfqpoint{4.568146in}{0.690507in}}%
\pgfpathlineto{\pgfqpoint{4.568443in}{0.690519in}}%
\pgfpathlineto{\pgfqpoint{4.568741in}{0.690531in}}%
\pgfpathlineto{\pgfqpoint{4.569038in}{0.690543in}}%
\pgfpathlineto{\pgfqpoint{4.569336in}{0.690554in}}%
\pgfpathlineto{\pgfqpoint{4.569633in}{0.690566in}}%
\pgfpathlineto{\pgfqpoint{4.569931in}{0.690578in}}%
\pgfpathlineto{\pgfqpoint{4.570228in}{0.690590in}}%
\pgfpathlineto{\pgfqpoint{4.570526in}{0.690602in}}%
\pgfpathlineto{\pgfqpoint{4.570823in}{0.690613in}}%
\pgfpathlineto{\pgfqpoint{4.571121in}{0.690625in}}%
\pgfpathlineto{\pgfqpoint{4.571418in}{0.690637in}}%
\pgfpathlineto{\pgfqpoint{4.571716in}{0.690649in}}%
\pgfpathlineto{\pgfqpoint{4.572013in}{0.690660in}}%
\pgfpathlineto{\pgfqpoint{4.572311in}{0.690672in}}%
\pgfpathlineto{\pgfqpoint{4.572608in}{0.690684in}}%
\pgfpathlineto{\pgfqpoint{4.572906in}{0.690696in}}%
\pgfpathlineto{\pgfqpoint{4.573203in}{0.690707in}}%
\pgfpathlineto{\pgfqpoint{4.573501in}{0.690719in}}%
\pgfpathlineto{\pgfqpoint{4.573798in}{0.690731in}}%
\pgfpathlineto{\pgfqpoint{4.574095in}{0.690743in}}%
\pgfpathlineto{\pgfqpoint{4.574393in}{0.690754in}}%
\pgfpathlineto{\pgfqpoint{4.574690in}{0.690766in}}%
\pgfpathlineto{\pgfqpoint{4.574988in}{0.690778in}}%
\pgfpathlineto{\pgfqpoint{4.575285in}{0.690790in}}%
\pgfpathlineto{\pgfqpoint{4.575583in}{0.690801in}}%
\pgfpathlineto{\pgfqpoint{4.575880in}{0.690813in}}%
\pgfpathlineto{\pgfqpoint{4.576178in}{0.690825in}}%
\pgfpathlineto{\pgfqpoint{4.576475in}{0.690837in}}%
\pgfpathlineto{\pgfqpoint{4.576773in}{0.690848in}}%
\pgfpathlineto{\pgfqpoint{4.577070in}{0.690860in}}%
\pgfpathlineto{\pgfqpoint{4.577368in}{0.690872in}}%
\pgfpathlineto{\pgfqpoint{4.577665in}{0.690884in}}%
\pgfpathlineto{\pgfqpoint{4.577963in}{0.690895in}}%
\pgfpathlineto{\pgfqpoint{4.578260in}{0.690907in}}%
\pgfpathlineto{\pgfqpoint{4.578558in}{0.690919in}}%
\pgfpathlineto{\pgfqpoint{4.578855in}{0.690931in}}%
\pgfpathlineto{\pgfqpoint{4.579153in}{0.690942in}}%
\pgfpathlineto{\pgfqpoint{4.579450in}{0.690954in}}%
\pgfpathlineto{\pgfqpoint{4.579748in}{0.690966in}}%
\pgfpathlineto{\pgfqpoint{4.580045in}{0.690978in}}%
\pgfpathlineto{\pgfqpoint{4.580343in}{0.690989in}}%
\pgfpathlineto{\pgfqpoint{4.580640in}{0.691001in}}%
\pgfpathlineto{\pgfqpoint{4.580937in}{0.691013in}}%
\pgfpathlineto{\pgfqpoint{4.581235in}{0.691025in}}%
\pgfpathlineto{\pgfqpoint{4.581532in}{0.691036in}}%
\pgfpathlineto{\pgfqpoint{4.581830in}{0.691048in}}%
\pgfpathlineto{\pgfqpoint{4.582127in}{0.691060in}}%
\pgfpathlineto{\pgfqpoint{4.582425in}{0.691072in}}%
\pgfpathlineto{\pgfqpoint{4.582722in}{0.691083in}}%
\pgfpathlineto{\pgfqpoint{4.583020in}{0.691095in}}%
\pgfpathlineto{\pgfqpoint{4.583317in}{0.691107in}}%
\pgfpathlineto{\pgfqpoint{4.583615in}{0.691119in}}%
\pgfpathlineto{\pgfqpoint{4.583912in}{0.691133in}}%
\pgfpathlineto{\pgfqpoint{4.584210in}{0.691217in}}%
\pgfpathlineto{\pgfqpoint{4.584507in}{0.691329in}}%
\pgfpathlineto{\pgfqpoint{4.584805in}{0.691391in}}%
\pgfpathlineto{\pgfqpoint{4.585102in}{0.691398in}}%
\pgfpathlineto{\pgfqpoint{4.585400in}{0.691405in}}%
\pgfpathlineto{\pgfqpoint{4.585697in}{0.691413in}}%
\pgfpathlineto{\pgfqpoint{4.585995in}{0.691420in}}%
\pgfpathlineto{\pgfqpoint{4.586292in}{0.691427in}}%
\pgfpathlineto{\pgfqpoint{4.586590in}{0.691434in}}%
\pgfpathlineto{\pgfqpoint{4.586887in}{0.691441in}}%
\pgfpathlineto{\pgfqpoint{4.587185in}{0.691448in}}%
\pgfpathlineto{\pgfqpoint{4.587482in}{0.691455in}}%
\pgfpathlineto{\pgfqpoint{4.587779in}{0.691462in}}%
\pgfpathlineto{\pgfqpoint{4.588077in}{0.691469in}}%
\pgfpathlineto{\pgfqpoint{4.588374in}{0.691477in}}%
\pgfpathlineto{\pgfqpoint{4.588672in}{0.691484in}}%
\pgfpathlineto{\pgfqpoint{4.588969in}{0.691491in}}%
\pgfpathlineto{\pgfqpoint{4.589267in}{0.691498in}}%
\pgfpathlineto{\pgfqpoint{4.589564in}{0.691505in}}%
\pgfpathlineto{\pgfqpoint{4.589862in}{0.691512in}}%
\pgfpathlineto{\pgfqpoint{4.590159in}{0.691519in}}%
\pgfpathlineto{\pgfqpoint{4.590457in}{0.691526in}}%
\pgfpathlineto{\pgfqpoint{4.590754in}{0.691533in}}%
\pgfpathlineto{\pgfqpoint{4.591052in}{0.691541in}}%
\pgfpathlineto{\pgfqpoint{4.591349in}{0.691548in}}%
\pgfpathlineto{\pgfqpoint{4.591647in}{0.691555in}}%
\pgfpathlineto{\pgfqpoint{4.591944in}{0.691562in}}%
\pgfpathlineto{\pgfqpoint{4.592242in}{0.691569in}}%
\pgfpathlineto{\pgfqpoint{4.592539in}{0.691576in}}%
\pgfpathlineto{\pgfqpoint{4.592837in}{0.691583in}}%
\pgfpathlineto{\pgfqpoint{4.593134in}{0.691590in}}%
\pgfpathlineto{\pgfqpoint{4.593432in}{0.691597in}}%
\pgfpathlineto{\pgfqpoint{4.593729in}{0.691605in}}%
\pgfpathlineto{\pgfqpoint{4.594026in}{0.691612in}}%
\pgfpathlineto{\pgfqpoint{4.594324in}{0.691619in}}%
\pgfpathlineto{\pgfqpoint{4.594621in}{0.691626in}}%
\pgfpathlineto{\pgfqpoint{4.594919in}{0.691633in}}%
\pgfpathlineto{\pgfqpoint{4.595216in}{0.691640in}}%
\pgfpathlineto{\pgfqpoint{4.595514in}{0.691647in}}%
\pgfpathlineto{\pgfqpoint{4.595811in}{0.691654in}}%
\pgfpathlineto{\pgfqpoint{4.596109in}{0.691661in}}%
\pgfpathlineto{\pgfqpoint{4.596406in}{0.691669in}}%
\pgfpathlineto{\pgfqpoint{4.596704in}{0.691676in}}%
\pgfpathlineto{\pgfqpoint{4.597001in}{0.691683in}}%
\pgfpathlineto{\pgfqpoint{4.597299in}{0.691690in}}%
\pgfpathlineto{\pgfqpoint{4.597596in}{0.691697in}}%
\pgfpathlineto{\pgfqpoint{4.597894in}{0.691704in}}%
\pgfpathlineto{\pgfqpoint{4.598191in}{0.691710in}}%
\pgfpathlineto{\pgfqpoint{4.598489in}{0.691711in}}%
\pgfpathlineto{\pgfqpoint{4.598786in}{0.691711in}}%
\pgfpathlineto{\pgfqpoint{4.599084in}{0.691711in}}%
\pgfpathlineto{\pgfqpoint{4.599381in}{0.691711in}}%
\pgfpathlineto{\pgfqpoint{4.599679in}{0.691712in}}%
\pgfpathlineto{\pgfqpoint{4.599976in}{0.691712in}}%
\pgfpathlineto{\pgfqpoint{4.600274in}{0.691712in}}%
\pgfpathlineto{\pgfqpoint{4.600571in}{0.691712in}}%
\pgfpathlineto{\pgfqpoint{4.600868in}{0.691713in}}%
\pgfpathlineto{\pgfqpoint{4.601166in}{0.691713in}}%
\pgfpathlineto{\pgfqpoint{4.601463in}{0.691713in}}%
\pgfpathlineto{\pgfqpoint{4.601761in}{0.691713in}}%
\pgfpathlineto{\pgfqpoint{4.602058in}{0.691713in}}%
\pgfpathlineto{\pgfqpoint{4.602356in}{0.691714in}}%
\pgfpathlineto{\pgfqpoint{4.602653in}{0.691714in}}%
\pgfpathlineto{\pgfqpoint{4.602951in}{0.691714in}}%
\pgfpathlineto{\pgfqpoint{4.603248in}{0.691714in}}%
\pgfpathlineto{\pgfqpoint{4.603546in}{0.691714in}}%
\pgfpathlineto{\pgfqpoint{4.603843in}{0.691715in}}%
\pgfpathlineto{\pgfqpoint{4.604141in}{0.691715in}}%
\pgfpathlineto{\pgfqpoint{4.604438in}{0.691715in}}%
\pgfpathlineto{\pgfqpoint{4.604736in}{0.691715in}}%
\pgfpathlineto{\pgfqpoint{4.605033in}{0.691715in}}%
\pgfpathlineto{\pgfqpoint{4.605331in}{0.691716in}}%
\pgfpathlineto{\pgfqpoint{4.605628in}{0.691716in}}%
\pgfpathlineto{\pgfqpoint{4.605926in}{0.691716in}}%
\pgfpathlineto{\pgfqpoint{4.606223in}{0.691716in}}%
\pgfpathlineto{\pgfqpoint{4.606521in}{0.691716in}}%
\pgfpathlineto{\pgfqpoint{4.606818in}{0.691717in}}%
\pgfpathlineto{\pgfqpoint{4.607116in}{0.691717in}}%
\pgfpathlineto{\pgfqpoint{4.607413in}{0.691717in}}%
\pgfpathlineto{\pgfqpoint{4.607710in}{0.691717in}}%
\pgfpathlineto{\pgfqpoint{4.608008in}{0.691717in}}%
\pgfpathlineto{\pgfqpoint{4.608305in}{0.691718in}}%
\pgfpathlineto{\pgfqpoint{4.608603in}{0.691718in}}%
\pgfpathlineto{\pgfqpoint{4.608900in}{0.691718in}}%
\pgfpathlineto{\pgfqpoint{4.609198in}{0.691718in}}%
\pgfpathlineto{\pgfqpoint{4.609495in}{0.691719in}}%
\pgfpathlineto{\pgfqpoint{4.609793in}{0.691719in}}%
\pgfpathlineto{\pgfqpoint{4.610090in}{0.691719in}}%
\pgfpathlineto{\pgfqpoint{4.610388in}{0.691719in}}%
\pgfpathlineto{\pgfqpoint{4.610685in}{0.691719in}}%
\pgfpathlineto{\pgfqpoint{4.610983in}{0.691720in}}%
\pgfpathlineto{\pgfqpoint{4.611280in}{0.691720in}}%
\pgfpathlineto{\pgfqpoint{4.611578in}{0.691720in}}%
\pgfpathlineto{\pgfqpoint{4.611875in}{0.691720in}}%
\pgfpathlineto{\pgfqpoint{4.612173in}{0.691720in}}%
\pgfpathlineto{\pgfqpoint{4.612470in}{0.691721in}}%
\pgfpathlineto{\pgfqpoint{4.612768in}{0.691721in}}%
\pgfpathlineto{\pgfqpoint{4.613065in}{0.691721in}}%
\pgfpathlineto{\pgfqpoint{4.613363in}{0.691721in}}%
\pgfpathlineto{\pgfqpoint{4.613660in}{0.691721in}}%
\pgfpathlineto{\pgfqpoint{4.613958in}{0.691722in}}%
\pgfpathlineto{\pgfqpoint{4.614255in}{0.691722in}}%
\pgfpathlineto{\pgfqpoint{4.614552in}{0.691722in}}%
\pgfpathlineto{\pgfqpoint{4.614850in}{0.691722in}}%
\pgfpathlineto{\pgfqpoint{4.615147in}{0.691722in}}%
\pgfpathlineto{\pgfqpoint{4.615445in}{0.691723in}}%
\pgfpathlineto{\pgfqpoint{4.615742in}{0.691723in}}%
\pgfpathlineto{\pgfqpoint{4.616040in}{0.691723in}}%
\pgfpathlineto{\pgfqpoint{4.616337in}{0.691723in}}%
\pgfpathlineto{\pgfqpoint{4.616635in}{0.691723in}}%
\pgfpathlineto{\pgfqpoint{4.616932in}{0.691724in}}%
\pgfpathlineto{\pgfqpoint{4.617230in}{0.691724in}}%
\pgfpathlineto{\pgfqpoint{4.617527in}{0.691724in}}%
\pgfpathlineto{\pgfqpoint{4.617825in}{0.691724in}}%
\pgfpathlineto{\pgfqpoint{4.618122in}{0.691724in}}%
\pgfpathlineto{\pgfqpoint{4.618420in}{0.691725in}}%
\pgfpathlineto{\pgfqpoint{4.618717in}{0.691725in}}%
\pgfpathlineto{\pgfqpoint{4.619015in}{0.691725in}}%
\pgfpathlineto{\pgfqpoint{4.619312in}{0.691725in}}%
\pgfpathlineto{\pgfqpoint{4.619610in}{0.691726in}}%
\pgfpathlineto{\pgfqpoint{4.619907in}{0.691726in}}%
\pgfpathlineto{\pgfqpoint{4.620205in}{0.691726in}}%
\pgfpathlineto{\pgfqpoint{4.620502in}{0.691726in}}%
\pgfpathlineto{\pgfqpoint{4.620799in}{0.691726in}}%
\pgfpathlineto{\pgfqpoint{4.621097in}{0.691727in}}%
\pgfpathlineto{\pgfqpoint{4.621394in}{0.691727in}}%
\pgfpathlineto{\pgfqpoint{4.621692in}{0.691727in}}%
\pgfpathlineto{\pgfqpoint{4.621989in}{0.691727in}}%
\pgfpathlineto{\pgfqpoint{4.622287in}{0.691727in}}%
\pgfpathlineto{\pgfqpoint{4.622584in}{0.691728in}}%
\pgfpathlineto{\pgfqpoint{4.622882in}{0.691728in}}%
\pgfpathlineto{\pgfqpoint{4.623179in}{0.691728in}}%
\pgfpathlineto{\pgfqpoint{4.623477in}{0.691728in}}%
\pgfpathlineto{\pgfqpoint{4.623774in}{0.691728in}}%
\pgfpathlineto{\pgfqpoint{4.624072in}{0.691729in}}%
\pgfpathlineto{\pgfqpoint{4.624369in}{0.691729in}}%
\pgfpathlineto{\pgfqpoint{4.624667in}{0.691729in}}%
\pgfpathlineto{\pgfqpoint{4.624964in}{0.691729in}}%
\pgfpathlineto{\pgfqpoint{4.625262in}{0.691729in}}%
\pgfpathlineto{\pgfqpoint{4.625559in}{0.691730in}}%
\pgfpathlineto{\pgfqpoint{4.625857in}{0.691730in}}%
\pgfpathlineto{\pgfqpoint{4.626154in}{0.691730in}}%
\pgfpathlineto{\pgfqpoint{4.626452in}{0.691730in}}%
\pgfpathlineto{\pgfqpoint{4.626749in}{0.691730in}}%
\pgfpathlineto{\pgfqpoint{4.627047in}{0.691731in}}%
\pgfpathlineto{\pgfqpoint{4.627344in}{0.691731in}}%
\pgfpathlineto{\pgfqpoint{4.627641in}{0.691731in}}%
\pgfpathlineto{\pgfqpoint{4.627939in}{0.691731in}}%
\pgfpathlineto{\pgfqpoint{4.628236in}{0.691732in}}%
\pgfpathlineto{\pgfqpoint{4.628534in}{0.691732in}}%
\pgfpathlineto{\pgfqpoint{4.628831in}{0.691732in}}%
\pgfpathlineto{\pgfqpoint{4.629129in}{0.691732in}}%
\pgfpathlineto{\pgfqpoint{4.629426in}{0.691732in}}%
\pgfpathlineto{\pgfqpoint{4.629724in}{0.691733in}}%
\pgfpathlineto{\pgfqpoint{4.630021in}{0.691733in}}%
\pgfpathlineto{\pgfqpoint{4.630319in}{0.691733in}}%
\pgfpathlineto{\pgfqpoint{4.630616in}{0.691733in}}%
\pgfpathlineto{\pgfqpoint{4.630914in}{0.691733in}}%
\pgfpathlineto{\pgfqpoint{4.631211in}{0.691734in}}%
\pgfpathlineto{\pgfqpoint{4.631509in}{0.691734in}}%
\pgfpathlineto{\pgfqpoint{4.631806in}{0.691734in}}%
\pgfpathlineto{\pgfqpoint{4.632104in}{0.691734in}}%
\pgfpathlineto{\pgfqpoint{4.632401in}{0.691734in}}%
\pgfpathlineto{\pgfqpoint{4.632699in}{0.691735in}}%
\pgfpathlineto{\pgfqpoint{4.632996in}{0.691735in}}%
\pgfpathlineto{\pgfqpoint{4.633294in}{0.691735in}}%
\pgfpathlineto{\pgfqpoint{4.633591in}{0.691735in}}%
\pgfpathlineto{\pgfqpoint{4.633889in}{0.691735in}}%
\pgfpathlineto{\pgfqpoint{4.634186in}{0.691736in}}%
\pgfpathlineto{\pgfqpoint{4.634483in}{0.691736in}}%
\pgfpathlineto{\pgfqpoint{4.634781in}{0.691736in}}%
\pgfpathlineto{\pgfqpoint{4.635078in}{0.691738in}}%
\pgfpathlineto{\pgfqpoint{4.635376in}{0.691751in}}%
\pgfpathlineto{\pgfqpoint{4.635673in}{0.691766in}}%
\pgfpathlineto{\pgfqpoint{4.635971in}{0.691781in}}%
\pgfpathlineto{\pgfqpoint{4.636268in}{0.691795in}}%
\pgfpathlineto{\pgfqpoint{4.636566in}{0.691810in}}%
\pgfpathlineto{\pgfqpoint{4.636863in}{0.691825in}}%
\pgfpathlineto{\pgfqpoint{4.637161in}{0.691840in}}%
\pgfpathlineto{\pgfqpoint{4.637458in}{0.691855in}}%
\pgfpathlineto{\pgfqpoint{4.637756in}{0.691870in}}%
\pgfpathlineto{\pgfqpoint{4.638053in}{0.691884in}}%
\pgfpathlineto{\pgfqpoint{4.638351in}{0.691899in}}%
\pgfpathlineto{\pgfqpoint{4.638648in}{0.691914in}}%
\pgfpathlineto{\pgfqpoint{4.638946in}{0.691929in}}%
\pgfpathlineto{\pgfqpoint{4.639243in}{0.691944in}}%
\pgfpathlineto{\pgfqpoint{4.639541in}{0.691959in}}%
\pgfpathlineto{\pgfqpoint{4.639838in}{0.691973in}}%
\pgfpathlineto{\pgfqpoint{4.640136in}{0.691988in}}%
\pgfpathlineto{\pgfqpoint{4.640433in}{0.692003in}}%
\pgfpathlineto{\pgfqpoint{4.640730in}{0.692018in}}%
\pgfpathlineto{\pgfqpoint{4.641028in}{0.692033in}}%
\pgfpathlineto{\pgfqpoint{4.641325in}{0.692048in}}%
\pgfpathlineto{\pgfqpoint{4.641623in}{0.692062in}}%
\pgfpathlineto{\pgfqpoint{4.641920in}{0.692077in}}%
\pgfpathlineto{\pgfqpoint{4.642218in}{0.692092in}}%
\pgfpathlineto{\pgfqpoint{4.642515in}{0.692107in}}%
\pgfpathlineto{\pgfqpoint{4.642813in}{0.692122in}}%
\pgfpathlineto{\pgfqpoint{4.643110in}{0.692137in}}%
\pgfpathlineto{\pgfqpoint{4.643408in}{0.692151in}}%
\pgfpathlineto{\pgfqpoint{4.643705in}{0.692166in}}%
\pgfpathlineto{\pgfqpoint{4.644003in}{0.692181in}}%
\pgfpathlineto{\pgfqpoint{4.644300in}{0.692196in}}%
\pgfpathlineto{\pgfqpoint{4.644598in}{0.692211in}}%
\pgfpathlineto{\pgfqpoint{4.644895in}{0.692226in}}%
\pgfpathlineto{\pgfqpoint{4.645193in}{0.692241in}}%
\pgfpathlineto{\pgfqpoint{4.645490in}{0.692255in}}%
\pgfpathlineto{\pgfqpoint{4.645788in}{0.692270in}}%
\pgfpathlineto{\pgfqpoint{4.646085in}{0.692285in}}%
\pgfpathlineto{\pgfqpoint{4.646383in}{0.692300in}}%
\pgfpathlineto{\pgfqpoint{4.646680in}{0.692315in}}%
\pgfpathlineto{\pgfqpoint{4.646978in}{0.692330in}}%
\pgfpathlineto{\pgfqpoint{4.647275in}{0.692090in}}%
\pgfpathlineto{\pgfqpoint{4.647572in}{0.682744in}}%
\pgfpathlineto{\pgfqpoint{4.647870in}{0.670821in}}%
\pgfpathlineto{\pgfqpoint{4.648167in}{0.684235in}}%
\pgfpathlineto{\pgfqpoint{4.648465in}{0.677508in}}%
\pgfpathlineto{\pgfqpoint{4.648762in}{0.669247in}}%
\pgfpathlineto{\pgfqpoint{4.649060in}{0.669134in}}%
\pgfpathlineto{\pgfqpoint{4.649357in}{0.688507in}}%
\pgfpathlineto{\pgfqpoint{4.649655in}{0.691880in}}%
\pgfpathlineto{\pgfqpoint{4.649952in}{0.691915in}}%
\pgfpathlineto{\pgfqpoint{4.650250in}{0.691950in}}%
\pgfpathlineto{\pgfqpoint{4.650547in}{0.691985in}}%
\pgfpathlineto{\pgfqpoint{4.650845in}{0.692020in}}%
\pgfpathlineto{\pgfqpoint{4.651142in}{0.692055in}}%
\pgfpathlineto{\pgfqpoint{4.651440in}{0.692090in}}%
\pgfpathlineto{\pgfqpoint{4.651737in}{0.692126in}}%
\pgfpathlineto{\pgfqpoint{4.652035in}{0.692161in}}%
\pgfpathlineto{\pgfqpoint{4.652332in}{0.692196in}}%
\pgfpathlineto{\pgfqpoint{4.652630in}{0.692231in}}%
\pgfpathlineto{\pgfqpoint{4.652927in}{0.692266in}}%
\pgfpathlineto{\pgfqpoint{4.653225in}{0.692301in}}%
\pgfpathlineto{\pgfqpoint{4.653522in}{0.692336in}}%
\pgfpathlineto{\pgfqpoint{4.653820in}{0.692371in}}%
\pgfpathlineto{\pgfqpoint{4.654117in}{0.692407in}}%
\pgfpathlineto{\pgfqpoint{4.654414in}{0.692442in}}%
\pgfpathlineto{\pgfqpoint{4.654712in}{0.692477in}}%
\pgfpathlineto{\pgfqpoint{4.655009in}{0.692512in}}%
\pgfpathlineto{\pgfqpoint{4.655307in}{0.692547in}}%
\pgfpathlineto{\pgfqpoint{4.655604in}{0.692575in}}%
\pgfpathlineto{\pgfqpoint{4.655902in}{0.692578in}}%
\pgfpathlineto{\pgfqpoint{4.656199in}{0.692578in}}%
\pgfpathlineto{\pgfqpoint{4.656497in}{0.692578in}}%
\pgfpathlineto{\pgfqpoint{4.656794in}{0.692578in}}%
\pgfpathlineto{\pgfqpoint{4.657092in}{0.692579in}}%
\pgfpathlineto{\pgfqpoint{4.657389in}{0.692579in}}%
\pgfpathlineto{\pgfqpoint{4.657687in}{0.692579in}}%
\pgfpathlineto{\pgfqpoint{4.657984in}{0.692579in}}%
\pgfpathlineto{\pgfqpoint{4.658282in}{0.692579in}}%
\pgfpathlineto{\pgfqpoint{4.658579in}{0.692579in}}%
\pgfpathlineto{\pgfqpoint{4.658877in}{0.692579in}}%
\pgfpathlineto{\pgfqpoint{4.659174in}{0.692579in}}%
\pgfpathlineto{\pgfqpoint{4.659472in}{0.692579in}}%
\pgfpathlineto{\pgfqpoint{4.659769in}{0.692580in}}%
\pgfpathlineto{\pgfqpoint{4.660067in}{0.692580in}}%
\pgfpathlineto{\pgfqpoint{4.660364in}{0.692580in}}%
\pgfpathlineto{\pgfqpoint{4.660661in}{0.692580in}}%
\pgfpathlineto{\pgfqpoint{4.660959in}{0.692580in}}%
\pgfpathlineto{\pgfqpoint{4.661256in}{0.692580in}}%
\pgfpathlineto{\pgfqpoint{4.661554in}{0.692580in}}%
\pgfpathlineto{\pgfqpoint{4.661851in}{0.692580in}}%
\pgfpathlineto{\pgfqpoint{4.662149in}{0.692580in}}%
\pgfpathlineto{\pgfqpoint{4.662446in}{0.692581in}}%
\pgfpathlineto{\pgfqpoint{4.662744in}{0.692581in}}%
\pgfpathlineto{\pgfqpoint{4.663041in}{0.692581in}}%
\pgfpathlineto{\pgfqpoint{4.663339in}{0.692581in}}%
\pgfpathlineto{\pgfqpoint{4.663636in}{0.692581in}}%
\pgfpathlineto{\pgfqpoint{4.663934in}{0.692581in}}%
\pgfpathlineto{\pgfqpoint{4.664231in}{0.692581in}}%
\pgfpathlineto{\pgfqpoint{4.664529in}{0.692581in}}%
\pgfpathlineto{\pgfqpoint{4.664826in}{0.692581in}}%
\pgfpathlineto{\pgfqpoint{4.665124in}{0.692582in}}%
\pgfpathlineto{\pgfqpoint{4.665421in}{0.692582in}}%
\pgfpathlineto{\pgfqpoint{4.665719in}{0.692582in}}%
\pgfpathlineto{\pgfqpoint{4.666016in}{0.692582in}}%
\pgfpathlineto{\pgfqpoint{4.666314in}{0.692582in}}%
\pgfpathlineto{\pgfqpoint{4.666611in}{0.692582in}}%
\pgfpathlineto{\pgfqpoint{4.666909in}{0.692582in}}%
\pgfpathlineto{\pgfqpoint{4.667206in}{0.692582in}}%
\pgfpathlineto{\pgfqpoint{4.667503in}{0.692583in}}%
\pgfpathlineto{\pgfqpoint{4.667801in}{0.692583in}}%
\pgfpathlineto{\pgfqpoint{4.668098in}{0.692583in}}%
\pgfpathlineto{\pgfqpoint{4.668396in}{0.692583in}}%
\pgfpathlineto{\pgfqpoint{4.668693in}{0.692583in}}%
\pgfpathlineto{\pgfqpoint{4.668991in}{0.692583in}}%
\pgfpathlineto{\pgfqpoint{4.669288in}{0.692583in}}%
\pgfpathlineto{\pgfqpoint{4.669586in}{0.692583in}}%
\pgfpathlineto{\pgfqpoint{4.669883in}{0.692583in}}%
\pgfpathlineto{\pgfqpoint{4.670181in}{0.692584in}}%
\pgfpathlineto{\pgfqpoint{4.670478in}{0.692584in}}%
\pgfpathlineto{\pgfqpoint{4.670776in}{0.692584in}}%
\pgfpathlineto{\pgfqpoint{4.671073in}{0.692584in}}%
\pgfpathlineto{\pgfqpoint{4.671371in}{0.692584in}}%
\pgfpathlineto{\pgfqpoint{4.671668in}{0.692584in}}%
\pgfpathlineto{\pgfqpoint{4.671966in}{0.692584in}}%
\pgfpathlineto{\pgfqpoint{4.672263in}{0.692584in}}%
\pgfpathlineto{\pgfqpoint{4.672561in}{0.692584in}}%
\pgfpathlineto{\pgfqpoint{4.672858in}{0.692585in}}%
\pgfpathlineto{\pgfqpoint{4.673156in}{0.692585in}}%
\pgfpathlineto{\pgfqpoint{4.673453in}{0.692585in}}%
\pgfpathlineto{\pgfqpoint{4.673751in}{0.692585in}}%
\pgfpathlineto{\pgfqpoint{4.674048in}{0.692585in}}%
\pgfpathlineto{\pgfqpoint{4.674345in}{0.692585in}}%
\pgfpathlineto{\pgfqpoint{4.674643in}{0.692585in}}%
\pgfpathlineto{\pgfqpoint{4.674940in}{0.692585in}}%
\pgfpathlineto{\pgfqpoint{4.675238in}{0.692586in}}%
\pgfpathlineto{\pgfqpoint{4.675535in}{0.692586in}}%
\pgfpathlineto{\pgfqpoint{4.675833in}{0.692586in}}%
\pgfpathlineto{\pgfqpoint{4.676130in}{0.692586in}}%
\pgfpathlineto{\pgfqpoint{4.676428in}{0.692586in}}%
\pgfpathlineto{\pgfqpoint{4.676725in}{0.692586in}}%
\pgfpathlineto{\pgfqpoint{4.677023in}{0.692586in}}%
\pgfpathlineto{\pgfqpoint{4.677320in}{0.692586in}}%
\pgfpathlineto{\pgfqpoint{4.677618in}{0.692586in}}%
\pgfpathlineto{\pgfqpoint{4.677915in}{0.692587in}}%
\pgfpathlineto{\pgfqpoint{4.678213in}{0.692587in}}%
\pgfpathlineto{\pgfqpoint{4.678510in}{0.692587in}}%
\pgfpathlineto{\pgfqpoint{4.678808in}{0.692587in}}%
\pgfpathlineto{\pgfqpoint{4.679105in}{0.692587in}}%
\pgfpathlineto{\pgfqpoint{4.679403in}{0.692587in}}%
\pgfpathlineto{\pgfqpoint{4.679700in}{0.692587in}}%
\pgfpathlineto{\pgfqpoint{4.679998in}{0.692587in}}%
\pgfpathlineto{\pgfqpoint{4.680295in}{0.692587in}}%
\pgfpathlineto{\pgfqpoint{4.680592in}{0.692588in}}%
\pgfpathlineto{\pgfqpoint{4.680890in}{0.692588in}}%
\pgfpathlineto{\pgfqpoint{4.681187in}{0.692588in}}%
\pgfpathlineto{\pgfqpoint{4.681485in}{0.692588in}}%
\pgfpathlineto{\pgfqpoint{4.681782in}{0.692588in}}%
\pgfpathlineto{\pgfqpoint{4.682080in}{0.692588in}}%
\pgfpathlineto{\pgfqpoint{4.682377in}{0.692588in}}%
\pgfpathlineto{\pgfqpoint{4.682675in}{0.692588in}}%
\pgfpathlineto{\pgfqpoint{4.682972in}{0.692588in}}%
\pgfpathlineto{\pgfqpoint{4.683270in}{0.692589in}}%
\pgfpathlineto{\pgfqpoint{4.683567in}{0.692589in}}%
\pgfpathlineto{\pgfqpoint{4.683865in}{0.692589in}}%
\pgfpathlineto{\pgfqpoint{4.684162in}{0.692589in}}%
\pgfpathlineto{\pgfqpoint{4.684460in}{0.692589in}}%
\pgfpathlineto{\pgfqpoint{4.684757in}{0.692589in}}%
\pgfpathlineto{\pgfqpoint{4.685055in}{0.692589in}}%
\pgfpathlineto{\pgfqpoint{4.685352in}{0.692589in}}%
\pgfpathlineto{\pgfqpoint{4.685650in}{0.692590in}}%
\pgfpathlineto{\pgfqpoint{4.685947in}{0.692590in}}%
\pgfpathlineto{\pgfqpoint{4.686245in}{0.692590in}}%
\pgfpathlineto{\pgfqpoint{4.686542in}{0.692590in}}%
\pgfpathlineto{\pgfqpoint{4.686840in}{0.692590in}}%
\pgfpathlineto{\pgfqpoint{4.687137in}{0.692590in}}%
\pgfpathlineto{\pgfqpoint{4.687434in}{0.692590in}}%
\pgfpathlineto{\pgfqpoint{4.687732in}{0.692590in}}%
\pgfpathlineto{\pgfqpoint{4.688029in}{0.692590in}}%
\pgfpathlineto{\pgfqpoint{4.688327in}{0.692591in}}%
\pgfpathlineto{\pgfqpoint{4.688624in}{0.692591in}}%
\pgfpathlineto{\pgfqpoint{4.688922in}{0.692591in}}%
\pgfpathlineto{\pgfqpoint{4.689219in}{0.692591in}}%
\pgfpathlineto{\pgfqpoint{4.689517in}{0.692591in}}%
\pgfpathlineto{\pgfqpoint{4.689814in}{0.692591in}}%
\pgfpathlineto{\pgfqpoint{4.690112in}{0.692591in}}%
\pgfpathlineto{\pgfqpoint{4.690409in}{0.692591in}}%
\pgfpathlineto{\pgfqpoint{4.690707in}{0.692591in}}%
\pgfpathlineto{\pgfqpoint{4.691004in}{0.692592in}}%
\pgfpathlineto{\pgfqpoint{4.691302in}{0.692592in}}%
\pgfpathlineto{\pgfqpoint{4.691599in}{0.692592in}}%
\pgfpathlineto{\pgfqpoint{4.691897in}{0.692592in}}%
\pgfpathlineto{\pgfqpoint{4.692194in}{0.692592in}}%
\pgfpathlineto{\pgfqpoint{4.692492in}{0.692592in}}%
\pgfpathlineto{\pgfqpoint{4.692789in}{0.692592in}}%
\pgfpathlineto{\pgfqpoint{4.693087in}{0.692592in}}%
\pgfpathlineto{\pgfqpoint{4.693384in}{0.692593in}}%
\pgfpathlineto{\pgfqpoint{4.693682in}{0.692593in}}%
\pgfpathlineto{\pgfqpoint{4.693979in}{0.692593in}}%
\pgfpathlineto{\pgfqpoint{4.694276in}{0.692593in}}%
\pgfpathlineto{\pgfqpoint{4.694574in}{0.692593in}}%
\pgfpathlineto{\pgfqpoint{4.694871in}{0.692593in}}%
\pgfpathlineto{\pgfqpoint{4.695169in}{0.692593in}}%
\pgfpathlineto{\pgfqpoint{4.695466in}{0.692593in}}%
\pgfpathlineto{\pgfqpoint{4.695764in}{0.692593in}}%
\pgfpathlineto{\pgfqpoint{4.696061in}{0.692594in}}%
\pgfpathlineto{\pgfqpoint{4.696359in}{0.692594in}}%
\pgfpathlineto{\pgfqpoint{4.696656in}{0.692594in}}%
\pgfpathlineto{\pgfqpoint{4.696954in}{0.692594in}}%
\pgfpathlineto{\pgfqpoint{4.697251in}{0.692594in}}%
\pgfpathlineto{\pgfqpoint{4.697549in}{0.692594in}}%
\pgfpathlineto{\pgfqpoint{4.697846in}{0.692594in}}%
\pgfpathlineto{\pgfqpoint{4.698144in}{0.692594in}}%
\pgfpathlineto{\pgfqpoint{4.698441in}{0.692594in}}%
\pgfpathlineto{\pgfqpoint{4.698739in}{0.692594in}}%
\pgfpathlineto{\pgfqpoint{4.699036in}{0.692594in}}%
\pgfpathlineto{\pgfqpoint{4.699334in}{0.692594in}}%
\pgfpathlineto{\pgfqpoint{4.699631in}{0.692594in}}%
\pgfpathlineto{\pgfqpoint{4.699929in}{0.692594in}}%
\pgfpathlineto{\pgfqpoint{4.700226in}{0.692594in}}%
\pgfpathlineto{\pgfqpoint{4.700523in}{0.692594in}}%
\pgfpathlineto{\pgfqpoint{4.700821in}{0.692594in}}%
\pgfpathlineto{\pgfqpoint{4.701118in}{0.692594in}}%
\pgfpathlineto{\pgfqpoint{4.701416in}{0.692594in}}%
\pgfpathlineto{\pgfqpoint{4.701713in}{0.692594in}}%
\pgfpathlineto{\pgfqpoint{4.702011in}{0.692594in}}%
\pgfpathlineto{\pgfqpoint{4.702308in}{0.692594in}}%
\pgfpathlineto{\pgfqpoint{4.702606in}{0.692594in}}%
\pgfpathlineto{\pgfqpoint{4.702903in}{0.692594in}}%
\pgfpathlineto{\pgfqpoint{4.703201in}{0.692594in}}%
\pgfpathlineto{\pgfqpoint{4.703498in}{0.692594in}}%
\pgfpathlineto{\pgfqpoint{4.703796in}{0.692593in}}%
\pgfpathlineto{\pgfqpoint{4.704093in}{0.692593in}}%
\pgfpathlineto{\pgfqpoint{4.704391in}{0.692593in}}%
\pgfpathlineto{\pgfqpoint{4.704688in}{0.692593in}}%
\pgfpathlineto{\pgfqpoint{4.704986in}{0.692593in}}%
\pgfpathlineto{\pgfqpoint{4.705283in}{0.692593in}}%
\pgfpathlineto{\pgfqpoint{4.705581in}{0.692593in}}%
\pgfpathlineto{\pgfqpoint{4.705878in}{0.692593in}}%
\pgfpathlineto{\pgfqpoint{4.706176in}{0.692593in}}%
\pgfpathlineto{\pgfqpoint{4.706473in}{0.692593in}}%
\pgfpathlineto{\pgfqpoint{4.706771in}{0.692593in}}%
\pgfpathlineto{\pgfqpoint{4.707068in}{0.692593in}}%
\pgfpathlineto{\pgfqpoint{4.707365in}{0.692593in}}%
\pgfpathlineto{\pgfqpoint{4.707663in}{0.692593in}}%
\pgfpathlineto{\pgfqpoint{4.707960in}{0.692593in}}%
\pgfpathlineto{\pgfqpoint{4.708258in}{0.692593in}}%
\pgfpathlineto{\pgfqpoint{4.708555in}{0.692593in}}%
\pgfpathlineto{\pgfqpoint{4.708853in}{0.692592in}}%
\pgfpathlineto{\pgfqpoint{4.709150in}{0.692592in}}%
\pgfpathlineto{\pgfqpoint{4.709448in}{0.692592in}}%
\pgfpathlineto{\pgfqpoint{4.709745in}{0.692592in}}%
\pgfpathlineto{\pgfqpoint{4.710043in}{0.692592in}}%
\pgfpathlineto{\pgfqpoint{4.710340in}{0.692592in}}%
\pgfpathlineto{\pgfqpoint{4.710638in}{0.692592in}}%
\pgfpathlineto{\pgfqpoint{4.710935in}{0.692592in}}%
\pgfpathlineto{\pgfqpoint{4.711233in}{0.692592in}}%
\pgfpathlineto{\pgfqpoint{4.711530in}{0.692592in}}%
\pgfpathlineto{\pgfqpoint{4.711828in}{0.692592in}}%
\pgfpathlineto{\pgfqpoint{4.712125in}{0.692592in}}%
\pgfpathlineto{\pgfqpoint{4.712423in}{0.692592in}}%
\pgfpathlineto{\pgfqpoint{4.712720in}{0.692592in}}%
\pgfpathlineto{\pgfqpoint{4.713018in}{0.692592in}}%
\pgfpathlineto{\pgfqpoint{4.713315in}{0.692592in}}%
\pgfpathlineto{\pgfqpoint{4.713613in}{0.692592in}}%
\pgfpathlineto{\pgfqpoint{4.713910in}{0.692592in}}%
\pgfpathlineto{\pgfqpoint{4.714207in}{0.692591in}}%
\pgfpathlineto{\pgfqpoint{4.714505in}{0.692591in}}%
\pgfpathlineto{\pgfqpoint{4.714802in}{0.692591in}}%
\pgfpathlineto{\pgfqpoint{4.715100in}{0.692591in}}%
\pgfpathlineto{\pgfqpoint{4.715397in}{0.692591in}}%
\pgfpathlineto{\pgfqpoint{4.715695in}{0.692591in}}%
\pgfpathlineto{\pgfqpoint{4.715992in}{0.692591in}}%
\pgfpathlineto{\pgfqpoint{4.716290in}{0.692591in}}%
\pgfpathlineto{\pgfqpoint{4.716587in}{0.692591in}}%
\pgfpathlineto{\pgfqpoint{4.716885in}{0.692591in}}%
\pgfpathlineto{\pgfqpoint{4.717182in}{0.692591in}}%
\pgfpathlineto{\pgfqpoint{4.717480in}{0.692591in}}%
\pgfpathlineto{\pgfqpoint{4.717777in}{0.692591in}}%
\pgfpathlineto{\pgfqpoint{4.718075in}{0.692591in}}%
\pgfpathlineto{\pgfqpoint{4.718372in}{0.692591in}}%
\pgfpathlineto{\pgfqpoint{4.718670in}{0.692591in}}%
\pgfpathlineto{\pgfqpoint{4.718967in}{0.692591in}}%
\pgfpathlineto{\pgfqpoint{4.719265in}{0.692591in}}%
\pgfpathlineto{\pgfqpoint{4.719562in}{0.692590in}}%
\pgfpathlineto{\pgfqpoint{4.719860in}{0.692590in}}%
\pgfpathlineto{\pgfqpoint{4.720157in}{0.692590in}}%
\pgfpathlineto{\pgfqpoint{4.720454in}{0.692590in}}%
\pgfpathlineto{\pgfqpoint{4.720752in}{0.692590in}}%
\pgfpathlineto{\pgfqpoint{4.721049in}{0.692590in}}%
\pgfpathlineto{\pgfqpoint{4.721347in}{0.692590in}}%
\pgfpathlineto{\pgfqpoint{4.721644in}{0.692590in}}%
\pgfpathlineto{\pgfqpoint{4.721942in}{0.692590in}}%
\pgfpathlineto{\pgfqpoint{4.722239in}{0.692590in}}%
\pgfpathlineto{\pgfqpoint{4.722537in}{0.692590in}}%
\pgfpathlineto{\pgfqpoint{4.722834in}{0.692590in}}%
\pgfpathlineto{\pgfqpoint{4.723132in}{0.692590in}}%
\pgfpathlineto{\pgfqpoint{4.723429in}{0.692590in}}%
\pgfpathlineto{\pgfqpoint{4.723727in}{0.692590in}}%
\pgfpathlineto{\pgfqpoint{4.724024in}{0.692590in}}%
\pgfpathlineto{\pgfqpoint{4.724322in}{0.692590in}}%
\pgfpathlineto{\pgfqpoint{4.724619in}{0.692590in}}%
\pgfpathlineto{\pgfqpoint{4.724917in}{0.692589in}}%
\pgfpathlineto{\pgfqpoint{4.725214in}{0.692589in}}%
\pgfpathlineto{\pgfqpoint{4.725512in}{0.692589in}}%
\pgfpathlineto{\pgfqpoint{4.725809in}{0.692589in}}%
\pgfpathlineto{\pgfqpoint{4.726107in}{0.692589in}}%
\pgfpathlineto{\pgfqpoint{4.726404in}{0.692589in}}%
\pgfpathlineto{\pgfqpoint{4.726702in}{0.692589in}}%
\pgfpathlineto{\pgfqpoint{4.726999in}{0.692589in}}%
\pgfpathlineto{\pgfqpoint{4.727296in}{0.692589in}}%
\pgfpathlineto{\pgfqpoint{4.727594in}{0.692589in}}%
\pgfpathlineto{\pgfqpoint{4.727891in}{0.692589in}}%
\pgfpathlineto{\pgfqpoint{4.728189in}{0.692589in}}%
\pgfpathlineto{\pgfqpoint{4.728486in}{0.692589in}}%
\pgfpathlineto{\pgfqpoint{4.728784in}{0.692589in}}%
\pgfpathlineto{\pgfqpoint{4.729081in}{0.692589in}}%
\pgfpathlineto{\pgfqpoint{4.729379in}{0.692589in}}%
\pgfpathlineto{\pgfqpoint{4.729676in}{0.692589in}}%
\pgfpathlineto{\pgfqpoint{4.729974in}{0.692589in}}%
\pgfpathlineto{\pgfqpoint{4.730271in}{0.692588in}}%
\pgfpathlineto{\pgfqpoint{4.730569in}{0.692588in}}%
\pgfpathlineto{\pgfqpoint{4.730866in}{0.692588in}}%
\pgfpathlineto{\pgfqpoint{4.731164in}{0.692588in}}%
\pgfpathlineto{\pgfqpoint{4.731461in}{0.692588in}}%
\pgfpathlineto{\pgfqpoint{4.731759in}{0.692588in}}%
\pgfpathlineto{\pgfqpoint{4.732056in}{0.692588in}}%
\pgfpathlineto{\pgfqpoint{4.732354in}{0.692588in}}%
\pgfpathlineto{\pgfqpoint{4.732651in}{0.692588in}}%
\pgfpathlineto{\pgfqpoint{4.732949in}{0.692588in}}%
\pgfpathlineto{\pgfqpoint{4.733246in}{0.692588in}}%
\pgfpathlineto{\pgfqpoint{4.733544in}{0.692588in}}%
\pgfpathlineto{\pgfqpoint{4.733841in}{0.692588in}}%
\pgfpathlineto{\pgfqpoint{4.734138in}{0.692588in}}%
\pgfpathlineto{\pgfqpoint{4.734436in}{0.692588in}}%
\pgfpathlineto{\pgfqpoint{4.734733in}{0.692588in}}%
\pgfpathlineto{\pgfqpoint{4.735031in}{0.692588in}}%
\pgfpathlineto{\pgfqpoint{4.735328in}{0.692588in}}%
\pgfpathlineto{\pgfqpoint{4.735626in}{0.692587in}}%
\pgfpathlineto{\pgfqpoint{4.735923in}{0.692587in}}%
\pgfpathlineto{\pgfqpoint{4.736221in}{0.692587in}}%
\pgfpathlineto{\pgfqpoint{4.736518in}{0.692587in}}%
\pgfpathlineto{\pgfqpoint{4.736816in}{0.692587in}}%
\pgfpathlineto{\pgfqpoint{4.737113in}{0.692587in}}%
\pgfpathlineto{\pgfqpoint{4.737411in}{0.692587in}}%
\pgfpathlineto{\pgfqpoint{4.737708in}{0.692587in}}%
\pgfpathlineto{\pgfqpoint{4.738006in}{0.692587in}}%
\pgfpathlineto{\pgfqpoint{4.738303in}{0.692587in}}%
\pgfpathlineto{\pgfqpoint{4.738601in}{0.692587in}}%
\pgfpathlineto{\pgfqpoint{4.738898in}{0.692587in}}%
\pgfpathlineto{\pgfqpoint{4.739196in}{0.692587in}}%
\pgfpathlineto{\pgfqpoint{4.739493in}{0.692587in}}%
\pgfpathlineto{\pgfqpoint{4.739791in}{0.692587in}}%
\pgfpathlineto{\pgfqpoint{4.740088in}{0.692587in}}%
\pgfpathlineto{\pgfqpoint{4.740385in}{0.692587in}}%
\pgfpathlineto{\pgfqpoint{4.740683in}{0.692587in}}%
\pgfpathlineto{\pgfqpoint{4.740980in}{0.692586in}}%
\pgfpathlineto{\pgfqpoint{4.741278in}{0.692586in}}%
\pgfpathlineto{\pgfqpoint{4.741575in}{0.692586in}}%
\pgfpathlineto{\pgfqpoint{4.741873in}{0.692586in}}%
\pgfpathlineto{\pgfqpoint{4.742170in}{0.692586in}}%
\pgfpathlineto{\pgfqpoint{4.742468in}{0.692586in}}%
\pgfpathlineto{\pgfqpoint{4.742765in}{0.692586in}}%
\pgfpathlineto{\pgfqpoint{4.743063in}{0.692586in}}%
\pgfpathlineto{\pgfqpoint{4.743360in}{0.692586in}}%
\pgfpathlineto{\pgfqpoint{4.743658in}{0.692586in}}%
\pgfpathlineto{\pgfqpoint{4.743955in}{0.692586in}}%
\pgfpathlineto{\pgfqpoint{4.744253in}{0.692586in}}%
\pgfpathlineto{\pgfqpoint{4.744550in}{0.692586in}}%
\pgfpathlineto{\pgfqpoint{4.744848in}{0.692586in}}%
\pgfpathlineto{\pgfqpoint{4.745145in}{0.692586in}}%
\pgfpathlineto{\pgfqpoint{4.745443in}{0.692586in}}%
\pgfpathlineto{\pgfqpoint{4.745740in}{0.692586in}}%
\pgfpathlineto{\pgfqpoint{4.746038in}{0.692586in}}%
\pgfpathlineto{\pgfqpoint{4.746335in}{0.692585in}}%
\pgfpathlineto{\pgfqpoint{4.746633in}{0.692585in}}%
\pgfpathlineto{\pgfqpoint{4.746930in}{0.692585in}}%
\pgfpathlineto{\pgfqpoint{4.747227in}{0.692585in}}%
\pgfpathlineto{\pgfqpoint{4.747525in}{0.692585in}}%
\pgfpathlineto{\pgfqpoint{4.747822in}{0.692585in}}%
\pgfpathlineto{\pgfqpoint{4.748120in}{0.692585in}}%
\pgfpathlineto{\pgfqpoint{4.748417in}{0.692585in}}%
\pgfpathlineto{\pgfqpoint{4.748715in}{0.692585in}}%
\pgfpathlineto{\pgfqpoint{4.749012in}{0.692585in}}%
\pgfpathlineto{\pgfqpoint{4.749310in}{0.692585in}}%
\pgfpathlineto{\pgfqpoint{4.749607in}{0.692585in}}%
\pgfpathlineto{\pgfqpoint{4.749905in}{0.692585in}}%
\pgfpathlineto{\pgfqpoint{4.750202in}{0.692585in}}%
\pgfpathlineto{\pgfqpoint{4.750500in}{0.692585in}}%
\pgfpathlineto{\pgfqpoint{4.750797in}{0.692585in}}%
\pgfpathlineto{\pgfqpoint{4.751095in}{0.692585in}}%
\pgfpathlineto{\pgfqpoint{4.751392in}{0.692585in}}%
\pgfpathlineto{\pgfqpoint{4.751690in}{0.692584in}}%
\pgfpathlineto{\pgfqpoint{4.751987in}{0.692584in}}%
\pgfpathlineto{\pgfqpoint{4.752285in}{0.692584in}}%
\pgfpathlineto{\pgfqpoint{4.752582in}{0.692584in}}%
\pgfpathlineto{\pgfqpoint{4.752880in}{0.692584in}}%
\pgfpathlineto{\pgfqpoint{4.753177in}{0.692584in}}%
\pgfpathlineto{\pgfqpoint{4.753475in}{0.692584in}}%
\pgfpathlineto{\pgfqpoint{4.753772in}{0.692584in}}%
\pgfpathlineto{\pgfqpoint{4.754069in}{0.692584in}}%
\pgfpathlineto{\pgfqpoint{4.754367in}{0.692584in}}%
\pgfpathlineto{\pgfqpoint{4.754664in}{0.692584in}}%
\pgfpathlineto{\pgfqpoint{4.754962in}{0.692584in}}%
\pgfpathlineto{\pgfqpoint{4.755259in}{0.692584in}}%
\pgfpathlineto{\pgfqpoint{4.755557in}{0.692584in}}%
\pgfpathlineto{\pgfqpoint{4.755854in}{0.692584in}}%
\pgfpathlineto{\pgfqpoint{4.756152in}{0.692584in}}%
\pgfpathlineto{\pgfqpoint{4.756449in}{0.692584in}}%
\pgfpathlineto{\pgfqpoint{4.756747in}{0.692584in}}%
\pgfpathlineto{\pgfqpoint{4.757044in}{0.692583in}}%
\pgfpathlineto{\pgfqpoint{4.757342in}{0.692583in}}%
\pgfpathlineto{\pgfqpoint{4.757639in}{0.692583in}}%
\pgfpathlineto{\pgfqpoint{4.757937in}{0.692583in}}%
\pgfpathlineto{\pgfqpoint{4.758234in}{0.692583in}}%
\pgfpathlineto{\pgfqpoint{4.758532in}{0.692583in}}%
\pgfpathlineto{\pgfqpoint{4.758829in}{0.692583in}}%
\pgfpathlineto{\pgfqpoint{4.759127in}{0.692583in}}%
\pgfpathlineto{\pgfqpoint{4.759424in}{0.692583in}}%
\pgfpathlineto{\pgfqpoint{4.759722in}{0.692583in}}%
\pgfpathlineto{\pgfqpoint{4.760019in}{0.692583in}}%
\pgfpathlineto{\pgfqpoint{4.760316in}{0.692583in}}%
\pgfpathlineto{\pgfqpoint{4.760614in}{0.692583in}}%
\pgfpathlineto{\pgfqpoint{4.760911in}{0.692583in}}%
\pgfpathlineto{\pgfqpoint{4.761209in}{0.692583in}}%
\pgfpathlineto{\pgfqpoint{4.761506in}{0.692583in}}%
\pgfpathlineto{\pgfqpoint{4.761804in}{0.692583in}}%
\pgfpathlineto{\pgfqpoint{4.762101in}{0.692583in}}%
\pgfpathlineto{\pgfqpoint{4.762399in}{0.692582in}}%
\pgfpathlineto{\pgfqpoint{4.762696in}{0.692582in}}%
\pgfpathlineto{\pgfqpoint{4.762994in}{0.692582in}}%
\pgfpathlineto{\pgfqpoint{4.763291in}{0.692582in}}%
\pgfpathlineto{\pgfqpoint{4.763589in}{0.692582in}}%
\pgfpathlineto{\pgfqpoint{4.763886in}{0.692582in}}%
\pgfpathlineto{\pgfqpoint{4.764184in}{0.692582in}}%
\pgfpathlineto{\pgfqpoint{4.764481in}{0.692582in}}%
\pgfpathlineto{\pgfqpoint{4.764779in}{0.692582in}}%
\pgfpathlineto{\pgfqpoint{4.765076in}{0.692582in}}%
\pgfpathlineto{\pgfqpoint{4.765374in}{0.692582in}}%
\pgfpathlineto{\pgfqpoint{4.765671in}{0.692582in}}%
\pgfpathlineto{\pgfqpoint{4.765969in}{0.692582in}}%
\pgfpathlineto{\pgfqpoint{4.766266in}{0.692582in}}%
\pgfpathlineto{\pgfqpoint{4.766564in}{0.692582in}}%
\pgfpathlineto{\pgfqpoint{4.766861in}{0.692582in}}%
\pgfpathlineto{\pgfqpoint{4.767158in}{0.692582in}}%
\pgfpathlineto{\pgfqpoint{4.767456in}{0.692581in}}%
\pgfpathlineto{\pgfqpoint{4.767753in}{0.692581in}}%
\pgfpathlineto{\pgfqpoint{4.768051in}{0.692581in}}%
\pgfpathlineto{\pgfqpoint{4.768348in}{0.692581in}}%
\pgfpathlineto{\pgfqpoint{4.768646in}{0.692581in}}%
\pgfpathlineto{\pgfqpoint{4.768943in}{0.692581in}}%
\pgfpathlineto{\pgfqpoint{4.769241in}{0.692581in}}%
\pgfpathlineto{\pgfqpoint{4.769538in}{0.692581in}}%
\pgfpathlineto{\pgfqpoint{4.769836in}{0.692581in}}%
\pgfpathlineto{\pgfqpoint{4.770133in}{0.692581in}}%
\pgfpathlineto{\pgfqpoint{4.770431in}{0.692581in}}%
\pgfpathlineto{\pgfqpoint{4.770728in}{0.692581in}}%
\pgfpathlineto{\pgfqpoint{4.771026in}{0.692581in}}%
\pgfpathlineto{\pgfqpoint{4.771323in}{0.692581in}}%
\pgfpathlineto{\pgfqpoint{4.771621in}{0.692581in}}%
\pgfpathlineto{\pgfqpoint{4.771918in}{0.692581in}}%
\pgfpathlineto{\pgfqpoint{4.772216in}{0.692581in}}%
\pgfpathlineto{\pgfqpoint{4.772513in}{0.692581in}}%
\pgfpathlineto{\pgfqpoint{4.772811in}{0.692580in}}%
\pgfpathlineto{\pgfqpoint{4.773108in}{0.692580in}}%
\pgfpathlineto{\pgfqpoint{4.773406in}{0.692580in}}%
\pgfpathlineto{\pgfqpoint{4.773703in}{0.692580in}}%
\pgfpathlineto{\pgfqpoint{4.774000in}{0.692580in}}%
\pgfpathlineto{\pgfqpoint{4.774298in}{0.692580in}}%
\pgfpathlineto{\pgfqpoint{4.774595in}{0.692580in}}%
\pgfpathlineto{\pgfqpoint{4.774893in}{0.692580in}}%
\pgfpathlineto{\pgfqpoint{4.775190in}{0.692580in}}%
\pgfpathlineto{\pgfqpoint{4.775488in}{0.692580in}}%
\pgfpathlineto{\pgfqpoint{4.775785in}{0.692580in}}%
\pgfpathlineto{\pgfqpoint{4.776083in}{0.692580in}}%
\pgfpathlineto{\pgfqpoint{4.776380in}{0.692580in}}%
\pgfpathlineto{\pgfqpoint{4.776678in}{0.692580in}}%
\pgfpathlineto{\pgfqpoint{4.776975in}{0.692580in}}%
\pgfpathlineto{\pgfqpoint{4.777273in}{0.692580in}}%
\pgfpathlineto{\pgfqpoint{4.777570in}{0.692580in}}%
\pgfpathlineto{\pgfqpoint{4.777868in}{0.692580in}}%
\pgfpathlineto{\pgfqpoint{4.778165in}{0.692579in}}%
\pgfpathlineto{\pgfqpoint{4.778463in}{0.692579in}}%
\pgfpathlineto{\pgfqpoint{4.778760in}{0.692579in}}%
\pgfpathlineto{\pgfqpoint{4.779058in}{0.692579in}}%
\pgfpathlineto{\pgfqpoint{4.779355in}{0.692579in}}%
\pgfpathlineto{\pgfqpoint{4.779653in}{0.692579in}}%
\pgfpathlineto{\pgfqpoint{4.779950in}{0.692579in}}%
\pgfpathlineto{\pgfqpoint{4.780247in}{0.692579in}}%
\pgfpathlineto{\pgfqpoint{4.780545in}{0.692579in}}%
\pgfpathlineto{\pgfqpoint{4.780842in}{0.692579in}}%
\pgfpathlineto{\pgfqpoint{4.781140in}{0.692579in}}%
\pgfpathlineto{\pgfqpoint{4.781437in}{0.692579in}}%
\pgfpathlineto{\pgfqpoint{4.781735in}{0.692579in}}%
\pgfpathlineto{\pgfqpoint{4.782032in}{0.692579in}}%
\pgfpathlineto{\pgfqpoint{4.782330in}{0.692579in}}%
\pgfpathlineto{\pgfqpoint{4.782627in}{0.692579in}}%
\pgfpathlineto{\pgfqpoint{4.782925in}{0.692579in}}%
\pgfpathlineto{\pgfqpoint{4.783222in}{0.692579in}}%
\pgfpathlineto{\pgfqpoint{4.783520in}{0.692578in}}%
\pgfpathlineto{\pgfqpoint{4.783817in}{0.692578in}}%
\pgfpathlineto{\pgfqpoint{4.784115in}{0.692578in}}%
\pgfpathlineto{\pgfqpoint{4.784412in}{0.692578in}}%
\pgfpathlineto{\pgfqpoint{4.784710in}{0.692578in}}%
\pgfpathlineto{\pgfqpoint{4.785007in}{0.692578in}}%
\pgfpathlineto{\pgfqpoint{4.785305in}{0.692578in}}%
\pgfpathlineto{\pgfqpoint{4.785602in}{0.692578in}}%
\pgfpathlineto{\pgfqpoint{4.785900in}{0.692578in}}%
\pgfpathlineto{\pgfqpoint{4.786197in}{0.692578in}}%
\pgfpathlineto{\pgfqpoint{4.786495in}{0.692578in}}%
\pgfpathlineto{\pgfqpoint{4.786792in}{0.692578in}}%
\pgfpathlineto{\pgfqpoint{4.787089in}{0.692578in}}%
\pgfpathlineto{\pgfqpoint{4.787387in}{0.692578in}}%
\pgfpathlineto{\pgfqpoint{4.787684in}{0.692578in}}%
\pgfpathlineto{\pgfqpoint{4.787982in}{0.692578in}}%
\pgfpathlineto{\pgfqpoint{4.788279in}{0.692578in}}%
\pgfpathlineto{\pgfqpoint{4.788577in}{0.692578in}}%
\pgfpathlineto{\pgfqpoint{4.788874in}{0.692577in}}%
\pgfpathlineto{\pgfqpoint{4.789172in}{0.692577in}}%
\pgfpathlineto{\pgfqpoint{4.789469in}{0.692577in}}%
\pgfpathlineto{\pgfqpoint{4.789767in}{0.692577in}}%
\pgfpathlineto{\pgfqpoint{4.790064in}{0.692577in}}%
\pgfpathlineto{\pgfqpoint{4.790362in}{0.692577in}}%
\pgfpathlineto{\pgfqpoint{4.790659in}{0.692577in}}%
\pgfpathlineto{\pgfqpoint{4.790957in}{0.692577in}}%
\pgfpathlineto{\pgfqpoint{4.791254in}{0.692577in}}%
\pgfpathlineto{\pgfqpoint{4.791552in}{0.692577in}}%
\pgfpathlineto{\pgfqpoint{4.791849in}{0.692577in}}%
\pgfpathlineto{\pgfqpoint{4.792147in}{0.692577in}}%
\pgfpathlineto{\pgfqpoint{4.792444in}{0.692577in}}%
\pgfpathlineto{\pgfqpoint{4.792742in}{0.692577in}}%
\pgfpathlineto{\pgfqpoint{4.793039in}{0.692577in}}%
\pgfpathlineto{\pgfqpoint{4.793337in}{0.692577in}}%
\pgfpathlineto{\pgfqpoint{4.793634in}{0.692577in}}%
\pgfpathlineto{\pgfqpoint{4.793931in}{0.692577in}}%
\pgfpathlineto{\pgfqpoint{4.794229in}{0.692576in}}%
\pgfpathlineto{\pgfqpoint{4.794526in}{0.692576in}}%
\pgfpathlineto{\pgfqpoint{4.794824in}{0.692576in}}%
\pgfpathlineto{\pgfqpoint{4.795121in}{0.692576in}}%
\pgfpathlineto{\pgfqpoint{4.795419in}{0.692576in}}%
\pgfpathlineto{\pgfqpoint{4.795716in}{0.692576in}}%
\pgfpathlineto{\pgfqpoint{4.796014in}{0.692576in}}%
\pgfpathlineto{\pgfqpoint{4.796311in}{0.692576in}}%
\pgfpathlineto{\pgfqpoint{4.796609in}{0.692576in}}%
\pgfpathlineto{\pgfqpoint{4.796906in}{0.692576in}}%
\pgfpathlineto{\pgfqpoint{4.797204in}{0.692576in}}%
\pgfpathlineto{\pgfqpoint{4.797501in}{0.692576in}}%
\pgfpathlineto{\pgfqpoint{4.797799in}{0.692576in}}%
\pgfpathlineto{\pgfqpoint{4.798096in}{0.692576in}}%
\pgfpathlineto{\pgfqpoint{4.798394in}{0.692576in}}%
\pgfpathlineto{\pgfqpoint{4.798691in}{0.692576in}}%
\pgfpathlineto{\pgfqpoint{4.798989in}{0.692576in}}%
\pgfpathlineto{\pgfqpoint{4.799286in}{0.692576in}}%
\pgfpathlineto{\pgfqpoint{4.799584in}{0.692575in}}%
\pgfpathlineto{\pgfqpoint{4.799881in}{0.692575in}}%
\pgfpathlineto{\pgfqpoint{4.800178in}{0.692575in}}%
\pgfpathlineto{\pgfqpoint{4.800476in}{0.692575in}}%
\pgfpathlineto{\pgfqpoint{4.800773in}{0.692575in}}%
\pgfpathlineto{\pgfqpoint{4.801071in}{0.692575in}}%
\pgfpathlineto{\pgfqpoint{4.801368in}{0.692575in}}%
\pgfpathlineto{\pgfqpoint{4.801666in}{0.692575in}}%
\pgfpathlineto{\pgfqpoint{4.801963in}{0.692575in}}%
\pgfpathlineto{\pgfqpoint{4.802261in}{0.692575in}}%
\pgfpathlineto{\pgfqpoint{4.802558in}{0.692575in}}%
\pgfpathlineto{\pgfqpoint{4.802856in}{0.692575in}}%
\pgfpathlineto{\pgfqpoint{4.803153in}{0.692575in}}%
\pgfpathlineto{\pgfqpoint{4.803451in}{0.692575in}}%
\pgfpathlineto{\pgfqpoint{4.803748in}{0.692575in}}%
\pgfpathlineto{\pgfqpoint{4.804046in}{0.692575in}}%
\pgfpathlineto{\pgfqpoint{4.804343in}{0.692575in}}%
\pgfpathlineto{\pgfqpoint{4.804641in}{0.692575in}}%
\pgfpathlineto{\pgfqpoint{4.804938in}{0.692574in}}%
\pgfpathlineto{\pgfqpoint{4.805236in}{0.692574in}}%
\pgfpathlineto{\pgfqpoint{4.805533in}{0.692574in}}%
\pgfpathlineto{\pgfqpoint{4.805831in}{0.692574in}}%
\pgfpathlineto{\pgfqpoint{4.806128in}{0.692573in}}%
\pgfpathlineto{\pgfqpoint{4.806426in}{0.692573in}}%
\pgfpathlineto{\pgfqpoint{4.806723in}{0.692573in}}%
\pgfpathlineto{\pgfqpoint{4.807020in}{0.692572in}}%
\pgfpathlineto{\pgfqpoint{4.807318in}{0.692572in}}%
\pgfpathlineto{\pgfqpoint{4.807615in}{0.692572in}}%
\pgfpathlineto{\pgfqpoint{4.807913in}{0.692571in}}%
\pgfpathlineto{\pgfqpoint{4.808210in}{0.692571in}}%
\pgfpathlineto{\pgfqpoint{4.808508in}{0.692571in}}%
\pgfpathlineto{\pgfqpoint{4.808805in}{0.692570in}}%
\pgfpathlineto{\pgfqpoint{4.809103in}{0.692570in}}%
\pgfpathlineto{\pgfqpoint{4.809400in}{0.692570in}}%
\pgfpathlineto{\pgfqpoint{4.809698in}{0.692569in}}%
\pgfpathlineto{\pgfqpoint{4.809995in}{0.692569in}}%
\pgfpathlineto{\pgfqpoint{4.810293in}{0.692569in}}%
\pgfpathlineto{\pgfqpoint{4.810590in}{0.692569in}}%
\pgfpathlineto{\pgfqpoint{4.810888in}{0.692568in}}%
\pgfpathlineto{\pgfqpoint{4.811185in}{0.692568in}}%
\pgfpathlineto{\pgfqpoint{4.811483in}{0.692568in}}%
\pgfpathlineto{\pgfqpoint{4.811780in}{0.692567in}}%
\pgfpathlineto{\pgfqpoint{4.812078in}{0.692567in}}%
\pgfpathlineto{\pgfqpoint{4.812375in}{0.692567in}}%
\pgfpathlineto{\pgfqpoint{4.812673in}{0.692566in}}%
\pgfpathlineto{\pgfqpoint{4.812970in}{0.692566in}}%
\pgfpathlineto{\pgfqpoint{4.813268in}{0.692566in}}%
\pgfpathlineto{\pgfqpoint{4.813565in}{0.692565in}}%
\pgfpathlineto{\pgfqpoint{4.813862in}{0.692565in}}%
\pgfpathlineto{\pgfqpoint{4.814160in}{0.692565in}}%
\pgfpathlineto{\pgfqpoint{4.814457in}{0.692564in}}%
\pgfpathlineto{\pgfqpoint{4.814755in}{0.692564in}}%
\pgfpathlineto{\pgfqpoint{4.815052in}{0.692564in}}%
\pgfpathlineto{\pgfqpoint{4.815350in}{0.692563in}}%
\pgfpathlineto{\pgfqpoint{4.815647in}{0.692563in}}%
\pgfpathlineto{\pgfqpoint{4.815945in}{0.692563in}}%
\pgfpathlineto{\pgfqpoint{4.816242in}{0.692562in}}%
\pgfpathlineto{\pgfqpoint{4.816540in}{0.692562in}}%
\pgfpathlineto{\pgfqpoint{4.816837in}{0.692562in}}%
\pgfpathlineto{\pgfqpoint{4.817135in}{0.692561in}}%
\pgfpathlineto{\pgfqpoint{4.817432in}{0.692561in}}%
\pgfpathlineto{\pgfqpoint{4.817730in}{0.692561in}}%
\pgfpathlineto{\pgfqpoint{4.818027in}{0.692560in}}%
\pgfpathlineto{\pgfqpoint{4.818325in}{0.692560in}}%
\pgfpathlineto{\pgfqpoint{4.818622in}{0.692560in}}%
\pgfpathlineto{\pgfqpoint{4.818920in}{0.692559in}}%
\pgfpathlineto{\pgfqpoint{4.819217in}{0.692559in}}%
\pgfpathlineto{\pgfqpoint{4.819515in}{0.692559in}}%
\pgfpathlineto{\pgfqpoint{4.819812in}{0.692559in}}%
\pgfpathlineto{\pgfqpoint{4.820109in}{0.692558in}}%
\pgfpathlineto{\pgfqpoint{4.820407in}{0.692558in}}%
\pgfpathlineto{\pgfqpoint{4.820704in}{0.692558in}}%
\pgfpathlineto{\pgfqpoint{4.821002in}{0.692557in}}%
\pgfpathlineto{\pgfqpoint{4.821299in}{0.692557in}}%
\pgfpathlineto{\pgfqpoint{4.821597in}{0.692557in}}%
\pgfpathlineto{\pgfqpoint{4.821894in}{0.692556in}}%
\pgfpathlineto{\pgfqpoint{4.822192in}{0.692556in}}%
\pgfpathlineto{\pgfqpoint{4.822489in}{0.692556in}}%
\pgfpathlineto{\pgfqpoint{4.822787in}{0.692555in}}%
\pgfpathlineto{\pgfqpoint{4.823084in}{0.692555in}}%
\pgfpathlineto{\pgfqpoint{4.823382in}{0.692555in}}%
\pgfpathlineto{\pgfqpoint{4.823679in}{0.692554in}}%
\pgfpathlineto{\pgfqpoint{4.823977in}{0.692554in}}%
\pgfpathlineto{\pgfqpoint{4.824274in}{0.692554in}}%
\pgfpathlineto{\pgfqpoint{4.824572in}{0.692553in}}%
\pgfpathlineto{\pgfqpoint{4.824869in}{0.692553in}}%
\pgfpathlineto{\pgfqpoint{4.825167in}{0.692553in}}%
\pgfpathlineto{\pgfqpoint{4.825464in}{0.692552in}}%
\pgfpathlineto{\pgfqpoint{4.825762in}{0.692552in}}%
\pgfpathlineto{\pgfqpoint{4.826059in}{0.692552in}}%
\pgfpathlineto{\pgfqpoint{4.826357in}{0.692551in}}%
\pgfpathlineto{\pgfqpoint{4.826654in}{0.692551in}}%
\pgfpathlineto{\pgfqpoint{4.826951in}{0.692551in}}%
\pgfpathlineto{\pgfqpoint{4.827249in}{0.692550in}}%
\pgfpathlineto{\pgfqpoint{4.827546in}{0.692550in}}%
\pgfpathlineto{\pgfqpoint{4.827844in}{0.692550in}}%
\pgfpathlineto{\pgfqpoint{4.828141in}{0.692549in}}%
\pgfpathlineto{\pgfqpoint{4.828439in}{0.692549in}}%
\pgfpathlineto{\pgfqpoint{4.828736in}{0.692549in}}%
\pgfpathlineto{\pgfqpoint{4.829034in}{0.692549in}}%
\pgfpathlineto{\pgfqpoint{4.829331in}{0.692548in}}%
\pgfpathlineto{\pgfqpoint{4.829629in}{0.692548in}}%
\pgfpathlineto{\pgfqpoint{4.829926in}{0.692548in}}%
\pgfpathlineto{\pgfqpoint{4.830224in}{0.692547in}}%
\pgfpathlineto{\pgfqpoint{4.830521in}{0.692547in}}%
\pgfpathlineto{\pgfqpoint{4.830819in}{0.692547in}}%
\pgfpathlineto{\pgfqpoint{4.831116in}{0.692546in}}%
\pgfpathlineto{\pgfqpoint{4.831414in}{0.692546in}}%
\pgfpathlineto{\pgfqpoint{4.831711in}{0.692546in}}%
\pgfpathlineto{\pgfqpoint{4.832009in}{0.692545in}}%
\pgfpathlineto{\pgfqpoint{4.832306in}{0.692545in}}%
\pgfpathlineto{\pgfqpoint{4.832604in}{0.692545in}}%
\pgfpathlineto{\pgfqpoint{4.832901in}{0.692544in}}%
\pgfpathlineto{\pgfqpoint{4.833199in}{0.692544in}}%
\pgfpathlineto{\pgfqpoint{4.833496in}{0.692544in}}%
\pgfpathlineto{\pgfqpoint{4.833793in}{0.692543in}}%
\pgfpathlineto{\pgfqpoint{4.834091in}{0.692543in}}%
\pgfpathlineto{\pgfqpoint{4.834388in}{0.692543in}}%
\pgfpathlineto{\pgfqpoint{4.834686in}{0.692542in}}%
\pgfpathlineto{\pgfqpoint{4.834983in}{0.692542in}}%
\pgfpathlineto{\pgfqpoint{4.835281in}{0.692542in}}%
\pgfpathlineto{\pgfqpoint{4.835578in}{0.692541in}}%
\pgfpathlineto{\pgfqpoint{4.835876in}{0.692541in}}%
\pgfpathlineto{\pgfqpoint{4.836173in}{0.692541in}}%
\pgfpathlineto{\pgfqpoint{4.836471in}{0.692540in}}%
\pgfpathlineto{\pgfqpoint{4.836768in}{0.692540in}}%
\pgfpathlineto{\pgfqpoint{4.837066in}{0.692540in}}%
\pgfpathlineto{\pgfqpoint{4.837363in}{0.692539in}}%
\pgfpathlineto{\pgfqpoint{4.837661in}{0.692539in}}%
\pgfpathlineto{\pgfqpoint{4.837958in}{0.692539in}}%
\pgfpathlineto{\pgfqpoint{4.838256in}{0.692539in}}%
\pgfpathlineto{\pgfqpoint{4.838553in}{0.692538in}}%
\pgfpathlineto{\pgfqpoint{4.838851in}{0.692538in}}%
\pgfpathlineto{\pgfqpoint{4.839148in}{0.692538in}}%
\pgfpathlineto{\pgfqpoint{4.839446in}{0.692537in}}%
\pgfpathlineto{\pgfqpoint{4.839743in}{0.692537in}}%
\pgfpathlineto{\pgfqpoint{4.840041in}{0.692537in}}%
\pgfpathlineto{\pgfqpoint{4.840338in}{0.692536in}}%
\pgfpathlineto{\pgfqpoint{4.840635in}{0.692536in}}%
\pgfpathlineto{\pgfqpoint{4.840933in}{0.692536in}}%
\pgfpathlineto{\pgfqpoint{4.841230in}{0.692535in}}%
\pgfpathlineto{\pgfqpoint{4.841528in}{0.692535in}}%
\pgfpathlineto{\pgfqpoint{4.841825in}{0.692535in}}%
\pgfpathlineto{\pgfqpoint{4.842123in}{0.692534in}}%
\pgfpathlineto{\pgfqpoint{4.842420in}{0.692534in}}%
\pgfpathlineto{\pgfqpoint{4.842718in}{0.692534in}}%
\pgfpathlineto{\pgfqpoint{4.843015in}{0.692533in}}%
\pgfpathlineto{\pgfqpoint{4.843313in}{0.692533in}}%
\pgfpathlineto{\pgfqpoint{4.843610in}{0.692533in}}%
\pgfpathlineto{\pgfqpoint{4.843908in}{0.692532in}}%
\pgfpathlineto{\pgfqpoint{4.844205in}{0.692532in}}%
\pgfpathlineto{\pgfqpoint{4.844503in}{0.692532in}}%
\pgfpathlineto{\pgfqpoint{4.844800in}{0.692531in}}%
\pgfpathlineto{\pgfqpoint{4.845098in}{0.692531in}}%
\pgfpathlineto{\pgfqpoint{4.845395in}{0.692531in}}%
\pgfpathlineto{\pgfqpoint{4.845693in}{0.692530in}}%
\pgfpathlineto{\pgfqpoint{4.845990in}{0.692530in}}%
\pgfpathlineto{\pgfqpoint{4.846288in}{0.692530in}}%
\pgfpathlineto{\pgfqpoint{4.846585in}{0.692529in}}%
\pgfpathlineto{\pgfqpoint{4.846882in}{0.692529in}}%
\pgfpathlineto{\pgfqpoint{4.847180in}{0.692529in}}%
\pgfpathlineto{\pgfqpoint{4.847477in}{0.692529in}}%
\pgfpathlineto{\pgfqpoint{4.847775in}{0.692528in}}%
\pgfpathlineto{\pgfqpoint{4.848072in}{0.692528in}}%
\pgfpathlineto{\pgfqpoint{4.848370in}{0.692528in}}%
\pgfpathlineto{\pgfqpoint{4.848667in}{0.692527in}}%
\pgfpathlineto{\pgfqpoint{4.848965in}{0.692527in}}%
\pgfpathlineto{\pgfqpoint{4.849262in}{0.692527in}}%
\pgfpathlineto{\pgfqpoint{4.849560in}{0.692526in}}%
\pgfpathlineto{\pgfqpoint{4.849857in}{0.692526in}}%
\pgfpathlineto{\pgfqpoint{4.850155in}{0.692526in}}%
\pgfpathlineto{\pgfqpoint{4.850452in}{0.692525in}}%
\pgfpathlineto{\pgfqpoint{4.850750in}{0.692525in}}%
\pgfpathlineto{\pgfqpoint{4.851047in}{0.692525in}}%
\pgfpathlineto{\pgfqpoint{4.851345in}{0.692524in}}%
\pgfpathlineto{\pgfqpoint{4.851642in}{0.692524in}}%
\pgfpathlineto{\pgfqpoint{4.851940in}{0.692524in}}%
\pgfpathlineto{\pgfqpoint{4.852237in}{0.692523in}}%
\pgfpathlineto{\pgfqpoint{4.852535in}{0.692523in}}%
\pgfpathlineto{\pgfqpoint{4.852832in}{0.692523in}}%
\pgfpathlineto{\pgfqpoint{4.853130in}{0.692522in}}%
\pgfpathlineto{\pgfqpoint{4.853427in}{0.692522in}}%
\pgfpathlineto{\pgfqpoint{4.853724in}{0.692522in}}%
\pgfpathlineto{\pgfqpoint{4.854022in}{0.692521in}}%
\pgfpathlineto{\pgfqpoint{4.854319in}{0.692521in}}%
\pgfpathlineto{\pgfqpoint{4.854617in}{0.692521in}}%
\pgfpathlineto{\pgfqpoint{4.854914in}{0.692522in}}%
\pgfpathlineto{\pgfqpoint{4.855212in}{0.692523in}}%
\pgfpathlineto{\pgfqpoint{4.855509in}{0.692524in}}%
\pgfpathlineto{\pgfqpoint{4.855807in}{0.692524in}}%
\pgfpathlineto{\pgfqpoint{4.856104in}{0.692525in}}%
\pgfpathlineto{\pgfqpoint{4.856402in}{0.692526in}}%
\pgfpathlineto{\pgfqpoint{4.856699in}{0.692526in}}%
\pgfpathlineto{\pgfqpoint{4.856997in}{0.692527in}}%
\pgfpathlineto{\pgfqpoint{4.857294in}{0.692528in}}%
\pgfpathlineto{\pgfqpoint{4.857592in}{0.692528in}}%
\pgfpathlineto{\pgfqpoint{4.857889in}{0.692529in}}%
\pgfpathlineto{\pgfqpoint{4.858187in}{0.692530in}}%
\pgfpathlineto{\pgfqpoint{4.858484in}{0.692531in}}%
\pgfpathlineto{\pgfqpoint{4.858782in}{0.692531in}}%
\pgfpathlineto{\pgfqpoint{4.859079in}{0.692532in}}%
\pgfpathlineto{\pgfqpoint{4.859377in}{0.692533in}}%
\pgfpathlineto{\pgfqpoint{4.859674in}{0.692533in}}%
\pgfpathlineto{\pgfqpoint{4.859972in}{0.692534in}}%
\pgfpathlineto{\pgfqpoint{4.860269in}{0.692535in}}%
\pgfpathlineto{\pgfqpoint{4.860566in}{0.692536in}}%
\pgfpathlineto{\pgfqpoint{4.860864in}{0.692536in}}%
\pgfpathlineto{\pgfqpoint{4.861161in}{0.692537in}}%
\pgfpathlineto{\pgfqpoint{4.861459in}{0.692538in}}%
\pgfpathlineto{\pgfqpoint{4.861756in}{0.692538in}}%
\pgfpathlineto{\pgfqpoint{4.862054in}{0.692539in}}%
\pgfpathlineto{\pgfqpoint{4.862351in}{0.692540in}}%
\pgfpathlineto{\pgfqpoint{4.862649in}{0.692541in}}%
\pgfpathlineto{\pgfqpoint{4.862946in}{0.692541in}}%
\pgfpathlineto{\pgfqpoint{4.863244in}{0.692542in}}%
\pgfpathlineto{\pgfqpoint{4.863541in}{0.692543in}}%
\pgfpathlineto{\pgfqpoint{4.863839in}{0.692543in}}%
\pgfpathlineto{\pgfqpoint{4.864136in}{0.692544in}}%
\pgfpathlineto{\pgfqpoint{4.864434in}{0.692545in}}%
\pgfpathlineto{\pgfqpoint{4.864731in}{0.692546in}}%
\pgfpathlineto{\pgfqpoint{4.865029in}{0.692546in}}%
\pgfpathlineto{\pgfqpoint{4.865326in}{0.692547in}}%
\pgfpathlineto{\pgfqpoint{4.865624in}{0.692548in}}%
\pgfpathlineto{\pgfqpoint{4.865921in}{0.692548in}}%
\pgfpathlineto{\pgfqpoint{4.866219in}{0.692549in}}%
\pgfpathlineto{\pgfqpoint{4.866516in}{0.692550in}}%
\pgfpathlineto{\pgfqpoint{4.866813in}{0.692551in}}%
\pgfpathlineto{\pgfqpoint{4.867111in}{0.692551in}}%
\pgfpathlineto{\pgfqpoint{4.867408in}{0.692552in}}%
\pgfpathlineto{\pgfqpoint{4.867706in}{0.692553in}}%
\pgfpathlineto{\pgfqpoint{4.868003in}{0.692553in}}%
\pgfpathlineto{\pgfqpoint{4.868301in}{0.692554in}}%
\pgfpathlineto{\pgfqpoint{4.868598in}{0.692555in}}%
\pgfpathlineto{\pgfqpoint{4.868896in}{0.692556in}}%
\pgfpathlineto{\pgfqpoint{4.869193in}{0.692556in}}%
\pgfpathlineto{\pgfqpoint{4.869491in}{0.692557in}}%
\pgfpathlineto{\pgfqpoint{4.869788in}{0.692558in}}%
\pgfpathlineto{\pgfqpoint{4.870086in}{0.692558in}}%
\pgfpathlineto{\pgfqpoint{4.870383in}{0.692559in}}%
\pgfpathlineto{\pgfqpoint{4.870681in}{0.692560in}}%
\pgfpathlineto{\pgfqpoint{4.870978in}{0.692561in}}%
\pgfpathlineto{\pgfqpoint{4.871276in}{0.692561in}}%
\pgfpathlineto{\pgfqpoint{4.871573in}{0.692562in}}%
\pgfpathlineto{\pgfqpoint{4.871871in}{0.692563in}}%
\pgfpathlineto{\pgfqpoint{4.872168in}{0.692563in}}%
\pgfpathlineto{\pgfqpoint{4.872466in}{0.692564in}}%
\pgfpathlineto{\pgfqpoint{4.872763in}{0.692565in}}%
\pgfpathlineto{\pgfqpoint{4.873061in}{0.692565in}}%
\pgfpathlineto{\pgfqpoint{4.873358in}{0.692566in}}%
\pgfpathlineto{\pgfqpoint{4.873655in}{0.692567in}}%
\pgfpathlineto{\pgfqpoint{4.873953in}{0.692568in}}%
\pgfpathlineto{\pgfqpoint{4.874250in}{0.692568in}}%
\pgfpathlineto{\pgfqpoint{4.874548in}{0.692569in}}%
\pgfpathlineto{\pgfqpoint{4.874845in}{0.692570in}}%
\pgfpathlineto{\pgfqpoint{4.875143in}{0.692570in}}%
\pgfpathlineto{\pgfqpoint{4.875440in}{0.692571in}}%
\pgfpathlineto{\pgfqpoint{4.875738in}{0.692572in}}%
\pgfpathlineto{\pgfqpoint{4.876035in}{0.692573in}}%
\pgfpathlineto{\pgfqpoint{4.876333in}{0.692573in}}%
\pgfpathlineto{\pgfqpoint{4.876630in}{0.692574in}}%
\pgfpathlineto{\pgfqpoint{4.876928in}{0.692575in}}%
\pgfpathlineto{\pgfqpoint{4.877225in}{0.692575in}}%
\pgfpathlineto{\pgfqpoint{4.877523in}{0.692576in}}%
\pgfpathlineto{\pgfqpoint{4.877820in}{0.692577in}}%
\pgfpathlineto{\pgfqpoint{4.878118in}{0.692578in}}%
\pgfpathlineto{\pgfqpoint{4.878415in}{0.692578in}}%
\pgfpathlineto{\pgfqpoint{4.878713in}{0.692579in}}%
\pgfpathlineto{\pgfqpoint{4.879010in}{0.692580in}}%
\pgfpathlineto{\pgfqpoint{4.879308in}{0.692580in}}%
\pgfpathlineto{\pgfqpoint{4.879605in}{0.692581in}}%
\pgfpathlineto{\pgfqpoint{4.879903in}{0.692582in}}%
\pgfpathlineto{\pgfqpoint{4.880200in}{0.692583in}}%
\pgfpathlineto{\pgfqpoint{4.880497in}{0.692583in}}%
\pgfpathlineto{\pgfqpoint{4.880795in}{0.692584in}}%
\pgfpathlineto{\pgfqpoint{4.881092in}{0.692585in}}%
\pgfpathlineto{\pgfqpoint{4.881390in}{0.692585in}}%
\pgfpathlineto{\pgfqpoint{4.881687in}{0.692586in}}%
\pgfpathlineto{\pgfqpoint{4.881985in}{0.692587in}}%
\pgfpathlineto{\pgfqpoint{4.882282in}{0.692588in}}%
\pgfpathlineto{\pgfqpoint{4.882580in}{0.692588in}}%
\pgfpathlineto{\pgfqpoint{4.882877in}{0.692589in}}%
\pgfpathlineto{\pgfqpoint{4.883175in}{0.692590in}}%
\pgfpathlineto{\pgfqpoint{4.883472in}{0.692590in}}%
\pgfpathlineto{\pgfqpoint{4.883770in}{0.692591in}}%
\pgfpathlineto{\pgfqpoint{4.884067in}{0.692592in}}%
\pgfpathlineto{\pgfqpoint{4.884365in}{0.692594in}}%
\pgfpathlineto{\pgfqpoint{4.884662in}{0.692598in}}%
\pgfpathlineto{\pgfqpoint{4.884960in}{0.692602in}}%
\pgfpathlineto{\pgfqpoint{4.885257in}{0.692606in}}%
\pgfpathlineto{\pgfqpoint{4.885555in}{0.692610in}}%
\pgfpathlineto{\pgfqpoint{4.885852in}{0.692614in}}%
\pgfpathlineto{\pgfqpoint{4.886150in}{0.692615in}}%
\pgfpathlineto{\pgfqpoint{4.886447in}{0.692616in}}%
\pgfpathlineto{\pgfqpoint{4.886744in}{0.692617in}}%
\pgfpathlineto{\pgfqpoint{4.887042in}{0.692618in}}%
\pgfpathlineto{\pgfqpoint{4.887339in}{0.692620in}}%
\pgfpathlineto{\pgfqpoint{4.887637in}{0.692621in}}%
\pgfpathlineto{\pgfqpoint{4.887934in}{0.692622in}}%
\pgfpathlineto{\pgfqpoint{4.888232in}{0.692623in}}%
\pgfpathlineto{\pgfqpoint{4.888529in}{0.692624in}}%
\pgfpathlineto{\pgfqpoint{4.888827in}{0.692625in}}%
\pgfpathlineto{\pgfqpoint{4.889124in}{0.692627in}}%
\pgfpathlineto{\pgfqpoint{4.889422in}{0.692628in}}%
\pgfpathlineto{\pgfqpoint{4.889719in}{0.692629in}}%
\pgfpathlineto{\pgfqpoint{4.890017in}{0.692630in}}%
\pgfpathlineto{\pgfqpoint{4.890314in}{0.692631in}}%
\pgfpathlineto{\pgfqpoint{4.890612in}{0.692632in}}%
\pgfpathlineto{\pgfqpoint{4.890909in}{0.692634in}}%
\pgfpathlineto{\pgfqpoint{4.891207in}{0.692635in}}%
\pgfpathlineto{\pgfqpoint{4.891504in}{0.692636in}}%
\pgfpathlineto{\pgfqpoint{4.891802in}{0.692637in}}%
\pgfpathlineto{\pgfqpoint{4.892099in}{0.692638in}}%
\pgfpathlineto{\pgfqpoint{4.892397in}{0.692639in}}%
\pgfpathlineto{\pgfqpoint{4.892694in}{0.692641in}}%
\pgfpathlineto{\pgfqpoint{4.892992in}{0.692642in}}%
\pgfpathlineto{\pgfqpoint{4.893289in}{0.692643in}}%
\pgfpathlineto{\pgfqpoint{4.893586in}{0.692644in}}%
\pgfpathlineto{\pgfqpoint{4.893884in}{0.692645in}}%
\pgfpathlineto{\pgfqpoint{4.894181in}{0.692646in}}%
\pgfpathlineto{\pgfqpoint{4.894479in}{0.692648in}}%
\pgfpathlineto{\pgfqpoint{4.894776in}{0.692649in}}%
\pgfpathlineto{\pgfqpoint{4.895074in}{0.692650in}}%
\pgfpathlineto{\pgfqpoint{4.895371in}{0.692651in}}%
\pgfpathlineto{\pgfqpoint{4.895669in}{0.692652in}}%
\pgfpathlineto{\pgfqpoint{4.895966in}{0.692653in}}%
\pgfpathlineto{\pgfqpoint{4.896264in}{0.692655in}}%
\pgfpathlineto{\pgfqpoint{4.896561in}{0.692656in}}%
\pgfpathlineto{\pgfqpoint{4.896859in}{0.692657in}}%
\pgfpathlineto{\pgfqpoint{4.897156in}{0.692658in}}%
\pgfpathlineto{\pgfqpoint{4.897454in}{0.692659in}}%
\pgfpathlineto{\pgfqpoint{4.897751in}{0.692660in}}%
\pgfpathlineto{\pgfqpoint{4.898049in}{0.692662in}}%
\pgfpathlineto{\pgfqpoint{4.898346in}{0.692663in}}%
\pgfpathlineto{\pgfqpoint{4.898644in}{0.692664in}}%
\pgfpathlineto{\pgfqpoint{4.898941in}{0.692665in}}%
\pgfpathlineto{\pgfqpoint{4.899239in}{0.692666in}}%
\pgfpathlineto{\pgfqpoint{4.899536in}{0.692667in}}%
\pgfpathlineto{\pgfqpoint{4.899834in}{0.692669in}}%
\pgfpathlineto{\pgfqpoint{4.900131in}{0.692670in}}%
\pgfpathlineto{\pgfqpoint{4.900428in}{0.692671in}}%
\pgfpathlineto{\pgfqpoint{4.900726in}{0.692672in}}%
\pgfpathlineto{\pgfqpoint{4.901023in}{0.692673in}}%
\pgfpathlineto{\pgfqpoint{4.901321in}{0.692674in}}%
\pgfpathlineto{\pgfqpoint{4.901618in}{0.692676in}}%
\pgfpathlineto{\pgfqpoint{4.901916in}{0.692677in}}%
\pgfpathlineto{\pgfqpoint{4.902213in}{0.692678in}}%
\pgfpathlineto{\pgfqpoint{4.902511in}{0.692679in}}%
\pgfpathlineto{\pgfqpoint{4.902808in}{0.692680in}}%
\pgfpathlineto{\pgfqpoint{4.903106in}{0.692681in}}%
\pgfpathlineto{\pgfqpoint{4.903403in}{0.692683in}}%
\pgfpathlineto{\pgfqpoint{4.903701in}{0.692684in}}%
\pgfpathlineto{\pgfqpoint{4.903998in}{0.692685in}}%
\pgfpathlineto{\pgfqpoint{4.904296in}{0.692686in}}%
\pgfpathlineto{\pgfqpoint{4.904593in}{0.692687in}}%
\pgfpathlineto{\pgfqpoint{4.904891in}{0.692689in}}%
\pgfpathlineto{\pgfqpoint{4.905188in}{0.692690in}}%
\pgfpathlineto{\pgfqpoint{4.905486in}{0.692691in}}%
\pgfpathlineto{\pgfqpoint{4.905783in}{0.692692in}}%
\pgfpathlineto{\pgfqpoint{4.906081in}{0.692693in}}%
\pgfpathlineto{\pgfqpoint{4.906378in}{0.692694in}}%
\pgfpathlineto{\pgfqpoint{4.906675in}{0.692696in}}%
\pgfpathlineto{\pgfqpoint{4.906973in}{0.692697in}}%
\pgfpathlineto{\pgfqpoint{4.907270in}{0.692698in}}%
\pgfpathlineto{\pgfqpoint{4.907568in}{0.692699in}}%
\pgfpathlineto{\pgfqpoint{4.907865in}{0.692700in}}%
\pgfpathlineto{\pgfqpoint{4.908163in}{0.692701in}}%
\pgfpathlineto{\pgfqpoint{4.908460in}{0.692703in}}%
\pgfpathlineto{\pgfqpoint{4.908758in}{0.692704in}}%
\pgfpathlineto{\pgfqpoint{4.909055in}{0.692705in}}%
\pgfpathlineto{\pgfqpoint{4.909353in}{0.692706in}}%
\pgfpathlineto{\pgfqpoint{4.909650in}{0.692707in}}%
\pgfpathlineto{\pgfqpoint{4.909948in}{0.692708in}}%
\pgfpathlineto{\pgfqpoint{4.910245in}{0.692710in}}%
\pgfpathlineto{\pgfqpoint{4.910543in}{0.692711in}}%
\pgfpathlineto{\pgfqpoint{4.910840in}{0.692712in}}%
\pgfpathlineto{\pgfqpoint{4.911138in}{0.692713in}}%
\pgfpathlineto{\pgfqpoint{4.911435in}{0.692714in}}%
\pgfpathlineto{\pgfqpoint{4.911733in}{0.692715in}}%
\pgfpathlineto{\pgfqpoint{4.912030in}{0.692717in}}%
\pgfpathlineto{\pgfqpoint{4.912328in}{0.692718in}}%
\pgfpathlineto{\pgfqpoint{4.912625in}{0.692719in}}%
\pgfpathlineto{\pgfqpoint{4.912923in}{0.692720in}}%
\pgfpathlineto{\pgfqpoint{4.913220in}{0.692721in}}%
\pgfpathlineto{\pgfqpoint{4.913517in}{0.692722in}}%
\pgfpathlineto{\pgfqpoint{4.913815in}{0.692724in}}%
\pgfpathlineto{\pgfqpoint{4.914112in}{0.692725in}}%
\pgfpathlineto{\pgfqpoint{4.914410in}{0.692726in}}%
\pgfpathlineto{\pgfqpoint{4.914707in}{0.692727in}}%
\pgfpathlineto{\pgfqpoint{4.915005in}{0.692728in}}%
\pgfpathlineto{\pgfqpoint{4.915302in}{0.692729in}}%
\pgfpathlineto{\pgfqpoint{4.915600in}{0.692731in}}%
\pgfpathlineto{\pgfqpoint{4.915897in}{0.692732in}}%
\pgfpathlineto{\pgfqpoint{4.916195in}{0.692733in}}%
\pgfpathlineto{\pgfqpoint{4.916492in}{0.692734in}}%
\pgfpathlineto{\pgfqpoint{4.916790in}{0.692735in}}%
\pgfpathlineto{\pgfqpoint{4.917087in}{0.692736in}}%
\pgfpathlineto{\pgfqpoint{4.917385in}{0.692738in}}%
\pgfpathlineto{\pgfqpoint{4.917682in}{0.692739in}}%
\pgfpathlineto{\pgfqpoint{4.917980in}{0.692740in}}%
\pgfpathlineto{\pgfqpoint{4.918277in}{0.692741in}}%
\pgfpathlineto{\pgfqpoint{4.918575in}{0.692742in}}%
\pgfpathlineto{\pgfqpoint{4.918872in}{0.692743in}}%
\pgfpathlineto{\pgfqpoint{4.919170in}{0.692745in}}%
\pgfpathlineto{\pgfqpoint{4.919467in}{0.692746in}}%
\pgfpathlineto{\pgfqpoint{4.919765in}{0.692747in}}%
\pgfpathlineto{\pgfqpoint{4.920062in}{0.692748in}}%
\pgfpathlineto{\pgfqpoint{4.920359in}{0.692749in}}%
\pgfpathlineto{\pgfqpoint{4.920657in}{0.692750in}}%
\pgfpathlineto{\pgfqpoint{4.920954in}{0.692752in}}%
\pgfpathlineto{\pgfqpoint{4.921252in}{0.692753in}}%
\pgfpathlineto{\pgfqpoint{4.921549in}{0.692754in}}%
\pgfpathlineto{\pgfqpoint{4.921847in}{0.692755in}}%
\pgfpathlineto{\pgfqpoint{4.922144in}{0.692756in}}%
\pgfpathlineto{\pgfqpoint{4.922442in}{0.692757in}}%
\pgfpathlineto{\pgfqpoint{4.922739in}{0.692759in}}%
\pgfpathlineto{\pgfqpoint{4.923037in}{0.692760in}}%
\pgfpathlineto{\pgfqpoint{4.923334in}{0.692761in}}%
\pgfpathlineto{\pgfqpoint{4.923632in}{0.692762in}}%
\pgfpathlineto{\pgfqpoint{4.923929in}{0.692763in}}%
\pgfpathlineto{\pgfqpoint{4.924227in}{0.692764in}}%
\pgfpathlineto{\pgfqpoint{4.924524in}{0.692766in}}%
\pgfpathlineto{\pgfqpoint{4.924822in}{0.692767in}}%
\pgfpathlineto{\pgfqpoint{4.925119in}{0.692768in}}%
\pgfpathlineto{\pgfqpoint{4.925417in}{0.692769in}}%
\pgfpathlineto{\pgfqpoint{4.925714in}{0.692770in}}%
\pgfpathlineto{\pgfqpoint{4.926012in}{0.692772in}}%
\pgfpathlineto{\pgfqpoint{4.926309in}{0.692773in}}%
\pgfpathlineto{\pgfqpoint{4.926606in}{0.692774in}}%
\pgfpathlineto{\pgfqpoint{4.926904in}{0.692775in}}%
\pgfpathlineto{\pgfqpoint{4.927201in}{0.692776in}}%
\pgfpathlineto{\pgfqpoint{4.927499in}{0.692777in}}%
\pgfpathlineto{\pgfqpoint{4.927796in}{0.692779in}}%
\pgfpathlineto{\pgfqpoint{4.928094in}{0.692780in}}%
\pgfpathlineto{\pgfqpoint{4.928391in}{0.692781in}}%
\pgfpathlineto{\pgfqpoint{4.928689in}{0.692782in}}%
\pgfpathlineto{\pgfqpoint{4.928986in}{0.692783in}}%
\pgfpathlineto{\pgfqpoint{4.929284in}{0.692784in}}%
\pgfpathlineto{\pgfqpoint{4.929581in}{0.692786in}}%
\pgfpathlineto{\pgfqpoint{4.929879in}{0.692787in}}%
\pgfpathlineto{\pgfqpoint{4.930176in}{0.692788in}}%
\pgfpathlineto{\pgfqpoint{4.930474in}{0.692789in}}%
\pgfpathlineto{\pgfqpoint{4.930771in}{0.692790in}}%
\pgfpathlineto{\pgfqpoint{4.931069in}{0.692791in}}%
\pgfpathlineto{\pgfqpoint{4.931366in}{0.692793in}}%
\pgfpathlineto{\pgfqpoint{4.931664in}{0.692794in}}%
\pgfpathlineto{\pgfqpoint{4.931961in}{0.692795in}}%
\pgfpathlineto{\pgfqpoint{4.932259in}{0.692796in}}%
\pgfpathlineto{\pgfqpoint{4.932556in}{0.692797in}}%
\pgfpathlineto{\pgfqpoint{4.932854in}{0.692798in}}%
\pgfpathlineto{\pgfqpoint{4.933151in}{0.692800in}}%
\pgfpathlineto{\pgfqpoint{4.933448in}{0.692801in}}%
\pgfpathlineto{\pgfqpoint{4.933746in}{0.692802in}}%
\pgfpathlineto{\pgfqpoint{4.934043in}{0.692803in}}%
\pgfpathlineto{\pgfqpoint{4.934341in}{0.692804in}}%
\pgfpathlineto{\pgfqpoint{4.934638in}{0.692805in}}%
\pgfpathlineto{\pgfqpoint{4.934936in}{0.692807in}}%
\pgfpathlineto{\pgfqpoint{4.935233in}{0.692808in}}%
\pgfpathlineto{\pgfqpoint{4.935531in}{0.692809in}}%
\pgfpathlineto{\pgfqpoint{4.935828in}{0.692810in}}%
\pgfpathlineto{\pgfqpoint{4.936126in}{0.692811in}}%
\pgfpathlineto{\pgfqpoint{4.936423in}{0.692812in}}%
\pgfpathlineto{\pgfqpoint{4.936721in}{0.692814in}}%
\pgfpathlineto{\pgfqpoint{4.937018in}{0.692815in}}%
\pgfpathlineto{\pgfqpoint{4.937316in}{0.692816in}}%
\pgfpathlineto{\pgfqpoint{4.937613in}{0.692817in}}%
\pgfpathlineto{\pgfqpoint{4.937911in}{0.692818in}}%
\pgfpathlineto{\pgfqpoint{4.938208in}{0.692819in}}%
\pgfpathlineto{\pgfqpoint{4.938506in}{0.692821in}}%
\pgfpathlineto{\pgfqpoint{4.938803in}{0.692822in}}%
\pgfpathlineto{\pgfqpoint{4.939101in}{0.692823in}}%
\pgfpathlineto{\pgfqpoint{4.939398in}{0.692824in}}%
\pgfpathlineto{\pgfqpoint{4.939696in}{0.692825in}}%
\pgfpathlineto{\pgfqpoint{4.939993in}{0.692826in}}%
\pgfpathlineto{\pgfqpoint{4.940290in}{0.692828in}}%
\pgfpathlineto{\pgfqpoint{4.940588in}{0.692829in}}%
\pgfpathlineto{\pgfqpoint{4.940885in}{0.692830in}}%
\pgfpathlineto{\pgfqpoint{4.941183in}{0.692831in}}%
\pgfpathlineto{\pgfqpoint{4.941480in}{0.692832in}}%
\pgfpathlineto{\pgfqpoint{4.941778in}{0.692833in}}%
\pgfpathlineto{\pgfqpoint{4.942075in}{0.692835in}}%
\pgfpathlineto{\pgfqpoint{4.942373in}{0.692836in}}%
\pgfpathlineto{\pgfqpoint{4.942670in}{0.692837in}}%
\pgfpathlineto{\pgfqpoint{4.942968in}{0.692838in}}%
\pgfpathlineto{\pgfqpoint{4.943265in}{0.692839in}}%
\pgfpathlineto{\pgfqpoint{4.943563in}{0.692840in}}%
\pgfpathlineto{\pgfqpoint{4.943860in}{0.692842in}}%
\pgfpathlineto{\pgfqpoint{4.944158in}{0.692843in}}%
\pgfpathlineto{\pgfqpoint{4.944455in}{0.692844in}}%
\pgfpathlineto{\pgfqpoint{4.944753in}{0.692845in}}%
\pgfpathlineto{\pgfqpoint{4.945050in}{0.692846in}}%
\pgfpathlineto{\pgfqpoint{4.945348in}{0.692847in}}%
\pgfpathlineto{\pgfqpoint{4.945645in}{0.692849in}}%
\pgfpathlineto{\pgfqpoint{4.945943in}{0.692850in}}%
\pgfpathlineto{\pgfqpoint{4.946240in}{0.692851in}}%
\pgfpathlineto{\pgfqpoint{4.946537in}{0.692852in}}%
\pgfpathlineto{\pgfqpoint{4.946835in}{0.692853in}}%
\pgfpathlineto{\pgfqpoint{4.947132in}{0.692855in}}%
\pgfpathlineto{\pgfqpoint{4.947430in}{0.692856in}}%
\pgfpathlineto{\pgfqpoint{4.947727in}{0.692857in}}%
\pgfpathlineto{\pgfqpoint{4.948025in}{0.692858in}}%
\pgfpathlineto{\pgfqpoint{4.948322in}{0.692859in}}%
\pgfpathlineto{\pgfqpoint{4.948620in}{0.692860in}}%
\pgfpathlineto{\pgfqpoint{4.948917in}{0.692862in}}%
\pgfpathlineto{\pgfqpoint{4.949215in}{0.692863in}}%
\pgfpathlineto{\pgfqpoint{4.949512in}{0.692864in}}%
\pgfpathlineto{\pgfqpoint{4.949810in}{0.692865in}}%
\pgfpathlineto{\pgfqpoint{4.950107in}{0.692866in}}%
\pgfpathlineto{\pgfqpoint{4.950405in}{0.692867in}}%
\pgfpathlineto{\pgfqpoint{4.950702in}{0.692869in}}%
\pgfpathlineto{\pgfqpoint{4.951000in}{0.692870in}}%
\pgfpathlineto{\pgfqpoint{4.951297in}{0.692871in}}%
\pgfpathlineto{\pgfqpoint{4.951595in}{0.692872in}}%
\pgfpathlineto{\pgfqpoint{4.951892in}{0.692873in}}%
\pgfpathlineto{\pgfqpoint{4.952190in}{0.692874in}}%
\pgfpathlineto{\pgfqpoint{4.952487in}{0.692876in}}%
\pgfpathlineto{\pgfqpoint{4.952785in}{0.692877in}}%
\pgfpathlineto{\pgfqpoint{4.953082in}{0.692878in}}%
\pgfpathlineto{\pgfqpoint{4.953379in}{0.692879in}}%
\pgfpathlineto{\pgfqpoint{4.953677in}{0.692880in}}%
\pgfpathlineto{\pgfqpoint{4.953974in}{0.692881in}}%
\pgfpathlineto{\pgfqpoint{4.954272in}{0.692883in}}%
\pgfpathlineto{\pgfqpoint{4.954569in}{0.692884in}}%
\pgfpathlineto{\pgfqpoint{4.954867in}{0.692885in}}%
\pgfpathlineto{\pgfqpoint{4.955164in}{0.692886in}}%
\pgfpathlineto{\pgfqpoint{4.955462in}{0.692887in}}%
\pgfpathlineto{\pgfqpoint{4.955759in}{0.692888in}}%
\pgfpathlineto{\pgfqpoint{4.956057in}{0.692890in}}%
\pgfpathlineto{\pgfqpoint{4.956354in}{0.692891in}}%
\pgfpathlineto{\pgfqpoint{4.956652in}{0.692892in}}%
\pgfpathlineto{\pgfqpoint{4.956949in}{0.692893in}}%
\pgfpathlineto{\pgfqpoint{4.957247in}{0.692894in}}%
\pgfpathlineto{\pgfqpoint{4.957544in}{0.692895in}}%
\pgfpathlineto{\pgfqpoint{4.957842in}{0.692897in}}%
\pgfpathlineto{\pgfqpoint{4.958139in}{0.692898in}}%
\pgfpathlineto{\pgfqpoint{4.958437in}{0.692899in}}%
\pgfpathlineto{\pgfqpoint{4.958734in}{0.692900in}}%
\pgfpathlineto{\pgfqpoint{4.959032in}{0.692901in}}%
\pgfpathlineto{\pgfqpoint{4.959329in}{0.692902in}}%
\pgfpathlineto{\pgfqpoint{4.959627in}{0.692904in}}%
\pgfpathlineto{\pgfqpoint{4.959924in}{0.692905in}}%
\pgfpathlineto{\pgfqpoint{4.960221in}{0.692906in}}%
\pgfpathlineto{\pgfqpoint{4.960519in}{0.692907in}}%
\pgfpathlineto{\pgfqpoint{4.960816in}{0.692908in}}%
\pgfpathlineto{\pgfqpoint{4.961114in}{0.692909in}}%
\pgfpathlineto{\pgfqpoint{4.961411in}{0.692911in}}%
\pgfpathlineto{\pgfqpoint{4.961709in}{0.692912in}}%
\pgfpathlineto{\pgfqpoint{4.962006in}{0.692913in}}%
\pgfpathlineto{\pgfqpoint{4.962304in}{0.692914in}}%
\pgfpathlineto{\pgfqpoint{4.962601in}{0.692915in}}%
\pgfpathlineto{\pgfqpoint{4.962899in}{0.692916in}}%
\pgfpathlineto{\pgfqpoint{4.963196in}{0.692918in}}%
\pgfpathlineto{\pgfqpoint{4.963494in}{0.692919in}}%
\pgfpathlineto{\pgfqpoint{4.963791in}{0.692920in}}%
\pgfpathlineto{\pgfqpoint{4.964089in}{0.692921in}}%
\pgfpathlineto{\pgfqpoint{4.964386in}{0.692922in}}%
\pgfpathlineto{\pgfqpoint{4.964684in}{0.692923in}}%
\pgfpathlineto{\pgfqpoint{4.964981in}{0.692925in}}%
\pgfpathlineto{\pgfqpoint{4.965279in}{0.692926in}}%
\pgfpathlineto{\pgfqpoint{4.965576in}{0.692927in}}%
\pgfpathlineto{\pgfqpoint{4.965874in}{0.692928in}}%
\pgfpathlineto{\pgfqpoint{4.966171in}{0.692929in}}%
\pgfpathlineto{\pgfqpoint{4.966468in}{0.692931in}}%
\pgfpathlineto{\pgfqpoint{4.966766in}{0.692932in}}%
\pgfpathlineto{\pgfqpoint{4.967063in}{0.692933in}}%
\pgfpathlineto{\pgfqpoint{4.967361in}{0.692934in}}%
\pgfpathlineto{\pgfqpoint{4.967658in}{0.692935in}}%
\pgfpathlineto{\pgfqpoint{4.967956in}{0.692936in}}%
\pgfpathlineto{\pgfqpoint{4.968253in}{0.692938in}}%
\pgfpathlineto{\pgfqpoint{4.968551in}{0.692939in}}%
\pgfpathlineto{\pgfqpoint{4.968848in}{0.692940in}}%
\pgfpathlineto{\pgfqpoint{4.969146in}{0.692941in}}%
\pgfpathlineto{\pgfqpoint{4.969443in}{0.692942in}}%
\pgfpathlineto{\pgfqpoint{4.969741in}{0.692943in}}%
\pgfpathlineto{\pgfqpoint{4.970038in}{0.692945in}}%
\pgfpathlineto{\pgfqpoint{4.970336in}{0.692946in}}%
\pgfpathlineto{\pgfqpoint{4.970633in}{0.692947in}}%
\pgfpathlineto{\pgfqpoint{4.970931in}{0.692948in}}%
\pgfpathlineto{\pgfqpoint{4.971228in}{0.692949in}}%
\pgfpathlineto{\pgfqpoint{4.971526in}{0.692950in}}%
\pgfpathlineto{\pgfqpoint{4.971823in}{0.692952in}}%
\pgfpathlineto{\pgfqpoint{4.972121in}{0.692953in}}%
\pgfpathlineto{\pgfqpoint{4.972418in}{0.692954in}}%
\pgfpathlineto{\pgfqpoint{4.972716in}{0.692955in}}%
\pgfpathlineto{\pgfqpoint{4.973013in}{0.692956in}}%
\pgfpathlineto{\pgfqpoint{4.973310in}{0.692957in}}%
\pgfpathlineto{\pgfqpoint{4.973608in}{0.692959in}}%
\pgfpathlineto{\pgfqpoint{4.973905in}{0.692960in}}%
\pgfpathlineto{\pgfqpoint{4.974203in}{0.692961in}}%
\pgfpathlineto{\pgfqpoint{4.974500in}{0.692962in}}%
\pgfpathlineto{\pgfqpoint{4.974798in}{0.692963in}}%
\pgfpathlineto{\pgfqpoint{4.975095in}{0.692964in}}%
\pgfpathlineto{\pgfqpoint{4.975393in}{0.692966in}}%
\pgfpathlineto{\pgfqpoint{4.975690in}{0.692967in}}%
\pgfpathlineto{\pgfqpoint{4.975988in}{0.692968in}}%
\pgfpathlineto{\pgfqpoint{4.976285in}{0.692969in}}%
\pgfpathlineto{\pgfqpoint{4.976583in}{0.692970in}}%
\pgfpathlineto{\pgfqpoint{4.976880in}{0.692971in}}%
\pgfpathlineto{\pgfqpoint{4.977178in}{0.692973in}}%
\pgfpathlineto{\pgfqpoint{4.977475in}{0.692974in}}%
\pgfpathlineto{\pgfqpoint{4.977773in}{0.692975in}}%
\pgfpathlineto{\pgfqpoint{4.978070in}{0.692976in}}%
\pgfpathlineto{\pgfqpoint{4.978368in}{0.692977in}}%
\pgfpathlineto{\pgfqpoint{4.978665in}{0.692978in}}%
\pgfpathlineto{\pgfqpoint{4.978963in}{0.692980in}}%
\pgfpathlineto{\pgfqpoint{4.979260in}{0.692981in}}%
\pgfpathlineto{\pgfqpoint{4.979558in}{0.692982in}}%
\pgfpathlineto{\pgfqpoint{4.979855in}{0.692983in}}%
\pgfpathlineto{\pgfqpoint{4.980152in}{0.692984in}}%
\pgfpathlineto{\pgfqpoint{4.980450in}{0.692985in}}%
\pgfpathlineto{\pgfqpoint{4.980747in}{0.692987in}}%
\pgfpathlineto{\pgfqpoint{4.981045in}{0.692988in}}%
\pgfpathlineto{\pgfqpoint{4.981342in}{0.692989in}}%
\pgfpathlineto{\pgfqpoint{4.981640in}{0.692990in}}%
\pgfpathlineto{\pgfqpoint{4.981937in}{0.692991in}}%
\pgfpathlineto{\pgfqpoint{4.982235in}{0.692992in}}%
\pgfpathlineto{\pgfqpoint{4.982532in}{0.692994in}}%
\pgfpathlineto{\pgfqpoint{4.982830in}{0.692995in}}%
\pgfpathlineto{\pgfqpoint{4.983127in}{0.692996in}}%
\pgfpathlineto{\pgfqpoint{4.983425in}{0.692997in}}%
\pgfpathlineto{\pgfqpoint{4.983722in}{0.692998in}}%
\pgfpathlineto{\pgfqpoint{4.984020in}{0.692999in}}%
\pgfpathlineto{\pgfqpoint{4.984317in}{0.693001in}}%
\pgfpathlineto{\pgfqpoint{4.984615in}{0.693002in}}%
\pgfpathlineto{\pgfqpoint{4.984912in}{0.693003in}}%
\pgfpathlineto{\pgfqpoint{4.985210in}{0.693004in}}%
\pgfpathlineto{\pgfqpoint{4.985507in}{0.693005in}}%
\pgfpathlineto{\pgfqpoint{4.985805in}{0.693006in}}%
\pgfpathlineto{\pgfqpoint{4.986102in}{0.693008in}}%
\pgfpathlineto{\pgfqpoint{4.986399in}{0.693009in}}%
\pgfpathlineto{\pgfqpoint{4.986697in}{0.693010in}}%
\pgfpathlineto{\pgfqpoint{4.986994in}{0.693011in}}%
\pgfpathlineto{\pgfqpoint{4.987292in}{0.693012in}}%
\pgfpathlineto{\pgfqpoint{4.987589in}{0.693014in}}%
\pgfpathlineto{\pgfqpoint{4.987887in}{0.693015in}}%
\pgfpathlineto{\pgfqpoint{4.988184in}{0.693016in}}%
\pgfpathlineto{\pgfqpoint{4.988482in}{0.693017in}}%
\pgfpathlineto{\pgfqpoint{4.988779in}{0.693018in}}%
\pgfpathlineto{\pgfqpoint{4.989077in}{0.693019in}}%
\pgfpathlineto{\pgfqpoint{4.989374in}{0.693021in}}%
\pgfpathlineto{\pgfqpoint{4.989672in}{0.693022in}}%
\pgfpathlineto{\pgfqpoint{4.989969in}{0.693023in}}%
\pgfpathlineto{\pgfqpoint{4.990267in}{0.693024in}}%
\pgfpathlineto{\pgfqpoint{4.990564in}{0.693025in}}%
\pgfpathlineto{\pgfqpoint{4.990862in}{0.693026in}}%
\pgfpathlineto{\pgfqpoint{4.991159in}{0.693028in}}%
\pgfpathlineto{\pgfqpoint{4.991457in}{0.693029in}}%
\pgfpathlineto{\pgfqpoint{4.991754in}{0.693030in}}%
\pgfpathlineto{\pgfqpoint{4.992052in}{0.693031in}}%
\pgfpathlineto{\pgfqpoint{4.992349in}{0.693032in}}%
\pgfpathlineto{\pgfqpoint{4.992647in}{0.693033in}}%
\pgfpathlineto{\pgfqpoint{4.992944in}{0.693035in}}%
\pgfpathlineto{\pgfqpoint{4.993241in}{0.693036in}}%
\pgfpathlineto{\pgfqpoint{4.993539in}{0.693037in}}%
\pgfpathlineto{\pgfqpoint{4.993836in}{0.693038in}}%
\pgfpathlineto{\pgfqpoint{4.994134in}{0.693039in}}%
\pgfpathlineto{\pgfqpoint{4.994431in}{0.693040in}}%
\pgfpathlineto{\pgfqpoint{4.994729in}{0.693042in}}%
\pgfpathlineto{\pgfqpoint{4.995026in}{0.693043in}}%
\pgfpathlineto{\pgfqpoint{4.995324in}{0.693044in}}%
\pgfpathlineto{\pgfqpoint{4.995621in}{0.693045in}}%
\pgfpathlineto{\pgfqpoint{4.995919in}{0.693046in}}%
\pgfpathlineto{\pgfqpoint{4.996216in}{0.693047in}}%
\pgfpathlineto{\pgfqpoint{4.996514in}{0.693049in}}%
\pgfpathlineto{\pgfqpoint{4.996811in}{0.693050in}}%
\pgfpathlineto{\pgfqpoint{4.997109in}{0.693051in}}%
\pgfpathlineto{\pgfqpoint{4.997406in}{0.693052in}}%
\pgfpathlineto{\pgfqpoint{4.997704in}{0.693053in}}%
\pgfpathlineto{\pgfqpoint{4.998001in}{0.693054in}}%
\pgfpathlineto{\pgfqpoint{4.998299in}{0.693056in}}%
\pgfpathlineto{\pgfqpoint{4.998596in}{0.693057in}}%
\pgfpathlineto{\pgfqpoint{4.998894in}{0.693058in}}%
\pgfpathlineto{\pgfqpoint{4.999191in}{0.693059in}}%
\pgfpathlineto{\pgfqpoint{4.999489in}{0.693060in}}%
\pgfpathlineto{\pgfqpoint{4.999786in}{0.693061in}}%
\pgfpathlineto{\pgfqpoint{5.000083in}{0.693063in}}%
\pgfpathlineto{\pgfqpoint{5.000381in}{0.693064in}}%
\pgfpathlineto{\pgfqpoint{5.000678in}{0.693065in}}%
\pgfpathlineto{\pgfqpoint{5.000976in}{0.693066in}}%
\pgfpathlineto{\pgfqpoint{5.001273in}{0.693067in}}%
\pgfpathlineto{\pgfqpoint{5.001571in}{0.693068in}}%
\pgfpathlineto{\pgfqpoint{5.001868in}{0.693070in}}%
\pgfpathlineto{\pgfqpoint{5.002166in}{0.693071in}}%
\pgfpathlineto{\pgfqpoint{5.002463in}{0.693072in}}%
\pgfpathlineto{\pgfqpoint{5.002761in}{0.693073in}}%
\pgfpathlineto{\pgfqpoint{5.003058in}{0.693074in}}%
\pgfpathlineto{\pgfqpoint{5.003356in}{0.693075in}}%
\pgfpathlineto{\pgfqpoint{5.003653in}{0.693077in}}%
\pgfpathlineto{\pgfqpoint{5.003951in}{0.693078in}}%
\pgfpathlineto{\pgfqpoint{5.004248in}{0.693079in}}%
\pgfpathlineto{\pgfqpoint{5.004546in}{0.693080in}}%
\pgfpathlineto{\pgfqpoint{5.004843in}{0.693081in}}%
\pgfpathlineto{\pgfqpoint{5.005141in}{0.693082in}}%
\pgfpathlineto{\pgfqpoint{5.005438in}{0.693084in}}%
\pgfpathlineto{\pgfqpoint{5.005736in}{0.693085in}}%
\pgfpathlineto{\pgfqpoint{5.006033in}{0.693086in}}%
\pgfpathlineto{\pgfqpoint{5.006330in}{0.693087in}}%
\pgfpathlineto{\pgfqpoint{5.006628in}{0.693088in}}%
\pgfpathlineto{\pgfqpoint{5.006925in}{0.693089in}}%
\pgfpathlineto{\pgfqpoint{5.007223in}{0.693091in}}%
\pgfpathlineto{\pgfqpoint{5.007520in}{0.693092in}}%
\pgfpathlineto{\pgfqpoint{5.007818in}{0.693093in}}%
\pgfpathlineto{\pgfqpoint{5.008115in}{0.693094in}}%
\pgfpathlineto{\pgfqpoint{5.008413in}{0.693095in}}%
\pgfpathlineto{\pgfqpoint{5.008710in}{0.693097in}}%
\pgfpathlineto{\pgfqpoint{5.009008in}{0.693098in}}%
\pgfpathlineto{\pgfqpoint{5.009305in}{0.693099in}}%
\pgfpathlineto{\pgfqpoint{5.009603in}{0.693100in}}%
\pgfpathlineto{\pgfqpoint{5.009900in}{0.693101in}}%
\pgfpathlineto{\pgfqpoint{5.010198in}{0.693102in}}%
\pgfpathlineto{\pgfqpoint{5.010495in}{0.693104in}}%
\pgfpathlineto{\pgfqpoint{5.010793in}{0.693105in}}%
\pgfpathlineto{\pgfqpoint{5.011090in}{0.693106in}}%
\pgfpathlineto{\pgfqpoint{5.011388in}{0.693107in}}%
\pgfpathlineto{\pgfqpoint{5.011685in}{0.693108in}}%
\pgfpathlineto{\pgfqpoint{5.011983in}{0.693109in}}%
\pgfpathlineto{\pgfqpoint{5.012280in}{0.693111in}}%
\pgfpathlineto{\pgfqpoint{5.012578in}{0.693112in}}%
\pgfpathlineto{\pgfqpoint{5.012875in}{0.693113in}}%
\pgfpathlineto{\pgfqpoint{5.013172in}{0.693114in}}%
\pgfpathlineto{\pgfqpoint{5.013470in}{0.693115in}}%
\pgfpathlineto{\pgfqpoint{5.013767in}{0.693116in}}%
\pgfpathlineto{\pgfqpoint{5.014065in}{0.693118in}}%
\pgfpathlineto{\pgfqpoint{5.014362in}{0.693119in}}%
\pgfpathlineto{\pgfqpoint{5.014660in}{0.693120in}}%
\pgfpathlineto{\pgfqpoint{5.014957in}{0.693121in}}%
\pgfpathlineto{\pgfqpoint{5.015255in}{0.693122in}}%
\pgfpathlineto{\pgfqpoint{5.015552in}{0.693123in}}%
\pgfpathlineto{\pgfqpoint{5.015850in}{0.693125in}}%
\pgfpathlineto{\pgfqpoint{5.016147in}{0.693126in}}%
\pgfpathlineto{\pgfqpoint{5.016445in}{0.693127in}}%
\pgfpathlineto{\pgfqpoint{5.016742in}{0.693128in}}%
\pgfpathlineto{\pgfqpoint{5.017040in}{0.693129in}}%
\pgfpathlineto{\pgfqpoint{5.017337in}{0.693130in}}%
\pgfpathlineto{\pgfqpoint{5.017635in}{0.693132in}}%
\pgfpathlineto{\pgfqpoint{5.017932in}{0.693133in}}%
\pgfpathlineto{\pgfqpoint{5.018230in}{0.693134in}}%
\pgfpathlineto{\pgfqpoint{5.018527in}{0.693135in}}%
\pgfpathlineto{\pgfqpoint{5.018825in}{0.693136in}}%
\pgfpathlineto{\pgfqpoint{5.019122in}{0.693137in}}%
\pgfpathlineto{\pgfqpoint{5.019420in}{0.693139in}}%
\pgfpathlineto{\pgfqpoint{5.019717in}{0.693140in}}%
\pgfpathlineto{\pgfqpoint{5.020014in}{0.693141in}}%
\pgfpathlineto{\pgfqpoint{5.020312in}{0.693142in}}%
\pgfpathlineto{\pgfqpoint{5.020609in}{0.693143in}}%
\pgfpathlineto{\pgfqpoint{5.020907in}{0.693144in}}%
\pgfpathlineto{\pgfqpoint{5.021204in}{0.693146in}}%
\pgfpathlineto{\pgfqpoint{5.021502in}{0.693147in}}%
\pgfpathlineto{\pgfqpoint{5.021799in}{0.693148in}}%
\pgfpathlineto{\pgfqpoint{5.022097in}{0.693149in}}%
\pgfpathlineto{\pgfqpoint{5.022394in}{0.693150in}}%
\pgfpathlineto{\pgfqpoint{5.022692in}{0.693151in}}%
\pgfpathlineto{\pgfqpoint{5.022989in}{0.693153in}}%
\pgfpathlineto{\pgfqpoint{5.023287in}{0.693154in}}%
\pgfpathlineto{\pgfqpoint{5.023584in}{0.693155in}}%
\pgfpathlineto{\pgfqpoint{5.023882in}{0.693156in}}%
\pgfpathlineto{\pgfqpoint{5.024179in}{0.693157in}}%
\pgfpathlineto{\pgfqpoint{5.024477in}{0.693158in}}%
\pgfpathlineto{\pgfqpoint{5.024774in}{0.693160in}}%
\pgfpathlineto{\pgfqpoint{5.025072in}{0.693161in}}%
\pgfpathlineto{\pgfqpoint{5.025369in}{0.693162in}}%
\pgfpathlineto{\pgfqpoint{5.025667in}{0.693163in}}%
\pgfpathlineto{\pgfqpoint{5.025964in}{0.693164in}}%
\pgfpathlineto{\pgfqpoint{5.026261in}{0.693165in}}%
\pgfpathlineto{\pgfqpoint{5.026559in}{0.693167in}}%
\pgfpathlineto{\pgfqpoint{5.026856in}{0.693168in}}%
\pgfpathlineto{\pgfqpoint{5.027154in}{0.693169in}}%
\pgfpathlineto{\pgfqpoint{5.027451in}{0.693170in}}%
\pgfpathlineto{\pgfqpoint{5.027749in}{0.693171in}}%
\pgfpathlineto{\pgfqpoint{5.028046in}{0.693173in}}%
\pgfpathlineto{\pgfqpoint{5.028344in}{0.693174in}}%
\pgfpathlineto{\pgfqpoint{5.028641in}{0.693175in}}%
\pgfpathlineto{\pgfqpoint{5.028939in}{0.693176in}}%
\pgfpathlineto{\pgfqpoint{5.029236in}{0.693177in}}%
\pgfpathlineto{\pgfqpoint{5.029534in}{0.693178in}}%
\pgfpathlineto{\pgfqpoint{5.029831in}{0.693180in}}%
\pgfpathlineto{\pgfqpoint{5.030129in}{0.693181in}}%
\pgfpathlineto{\pgfqpoint{5.030426in}{0.693182in}}%
\pgfpathlineto{\pgfqpoint{5.030724in}{0.693183in}}%
\pgfpathlineto{\pgfqpoint{5.031021in}{0.693184in}}%
\pgfpathlineto{\pgfqpoint{5.031319in}{0.693185in}}%
\pgfpathlineto{\pgfqpoint{5.031616in}{0.693187in}}%
\pgfpathlineto{\pgfqpoint{5.031914in}{0.693188in}}%
\pgfpathlineto{\pgfqpoint{5.032211in}{0.693189in}}%
\pgfpathlineto{\pgfqpoint{5.032509in}{0.693190in}}%
\pgfpathlineto{\pgfqpoint{5.032806in}{0.693191in}}%
\pgfpathlineto{\pgfqpoint{5.033103in}{0.693192in}}%
\pgfpathlineto{\pgfqpoint{5.033401in}{0.693194in}}%
\pgfpathlineto{\pgfqpoint{5.033698in}{0.693195in}}%
\pgfpathlineto{\pgfqpoint{5.033996in}{0.693196in}}%
\pgfpathlineto{\pgfqpoint{5.034293in}{0.693197in}}%
\pgfpathlineto{\pgfqpoint{5.034591in}{0.693198in}}%
\pgfpathlineto{\pgfqpoint{5.034888in}{0.693199in}}%
\pgfpathlineto{\pgfqpoint{5.035186in}{0.693201in}}%
\pgfpathlineto{\pgfqpoint{5.035483in}{0.693202in}}%
\pgfpathlineto{\pgfqpoint{5.035781in}{0.693203in}}%
\pgfpathlineto{\pgfqpoint{5.036078in}{0.693204in}}%
\pgfpathlineto{\pgfqpoint{5.036376in}{0.693205in}}%
\pgfpathlineto{\pgfqpoint{5.036673in}{0.693206in}}%
\pgfpathlineto{\pgfqpoint{5.036971in}{0.693208in}}%
\pgfpathlineto{\pgfqpoint{5.037268in}{0.693209in}}%
\pgfpathlineto{\pgfqpoint{5.037566in}{0.693210in}}%
\pgfpathlineto{\pgfqpoint{5.037863in}{0.693211in}}%
\pgfpathlineto{\pgfqpoint{5.038161in}{0.693212in}}%
\pgfpathlineto{\pgfqpoint{5.038458in}{0.693213in}}%
\pgfpathlineto{\pgfqpoint{5.038756in}{0.693215in}}%
\pgfpathlineto{\pgfqpoint{5.039053in}{0.693216in}}%
\pgfpathlineto{\pgfqpoint{5.039351in}{0.693217in}}%
\pgfpathlineto{\pgfqpoint{5.039648in}{0.693218in}}%
\pgfpathlineto{\pgfqpoint{5.039945in}{0.693219in}}%
\pgfpathlineto{\pgfqpoint{5.040243in}{0.693224in}}%
\pgfpathlineto{\pgfqpoint{5.040540in}{0.693231in}}%
\pgfpathlineto{\pgfqpoint{5.040838in}{0.693238in}}%
\pgfpathlineto{\pgfqpoint{5.041135in}{0.693246in}}%
\pgfpathlineto{\pgfqpoint{5.041433in}{0.693253in}}%
\pgfpathlineto{\pgfqpoint{5.041730in}{0.693260in}}%
\pgfpathlineto{\pgfqpoint{5.042028in}{0.693268in}}%
\pgfpathlineto{\pgfqpoint{5.042325in}{0.693275in}}%
\pgfpathlineto{\pgfqpoint{5.042623in}{0.693282in}}%
\pgfpathlineto{\pgfqpoint{5.042920in}{0.693290in}}%
\pgfpathlineto{\pgfqpoint{5.043218in}{0.693297in}}%
\pgfpathlineto{\pgfqpoint{5.043515in}{0.693304in}}%
\pgfpathlineto{\pgfqpoint{5.043813in}{0.693312in}}%
\pgfpathlineto{\pgfqpoint{5.044110in}{0.693319in}}%
\pgfpathlineto{\pgfqpoint{5.044408in}{0.693326in}}%
\pgfpathlineto{\pgfqpoint{5.044705in}{0.693334in}}%
\pgfpathlineto{\pgfqpoint{5.045003in}{0.693341in}}%
\pgfpathlineto{\pgfqpoint{5.045300in}{0.693348in}}%
\pgfpathlineto{\pgfqpoint{5.045598in}{0.693356in}}%
\pgfpathlineto{\pgfqpoint{5.045895in}{0.693363in}}%
\pgfpathlineto{\pgfqpoint{5.046193in}{0.693370in}}%
\pgfpathlineto{\pgfqpoint{5.046490in}{0.693378in}}%
\pgfpathlineto{\pgfqpoint{5.046787in}{0.693385in}}%
\pgfpathlineto{\pgfqpoint{5.047085in}{0.693392in}}%
\pgfpathlineto{\pgfqpoint{5.047382in}{0.693400in}}%
\pgfpathlineto{\pgfqpoint{5.047680in}{0.693407in}}%
\pgfpathlineto{\pgfqpoint{5.047977in}{0.693414in}}%
\pgfpathlineto{\pgfqpoint{5.048275in}{0.693422in}}%
\pgfpathlineto{\pgfqpoint{5.048572in}{0.693429in}}%
\pgfpathlineto{\pgfqpoint{5.048870in}{0.693436in}}%
\pgfpathlineto{\pgfqpoint{5.049167in}{0.693444in}}%
\pgfpathlineto{\pgfqpoint{5.049465in}{0.693451in}}%
\pgfpathlineto{\pgfqpoint{5.049762in}{0.693458in}}%
\pgfpathlineto{\pgfqpoint{5.050060in}{0.693466in}}%
\pgfpathlineto{\pgfqpoint{5.050357in}{0.693473in}}%
\pgfpathlineto{\pgfqpoint{5.050655in}{0.693481in}}%
\pgfpathlineto{\pgfqpoint{5.050952in}{0.693488in}}%
\pgfpathlineto{\pgfqpoint{5.051250in}{0.693495in}}%
\pgfpathlineto{\pgfqpoint{5.051547in}{0.693503in}}%
\pgfpathlineto{\pgfqpoint{5.051845in}{0.693510in}}%
\pgfpathlineto{\pgfqpoint{5.052142in}{0.693517in}}%
\pgfpathlineto{\pgfqpoint{5.052440in}{0.693525in}}%
\pgfpathlineto{\pgfqpoint{5.052737in}{0.693532in}}%
\pgfpathlineto{\pgfqpoint{5.053034in}{0.693539in}}%
\pgfpathlineto{\pgfqpoint{5.053332in}{0.693547in}}%
\pgfpathlineto{\pgfqpoint{5.053629in}{0.693554in}}%
\pgfpathlineto{\pgfqpoint{5.053927in}{0.693561in}}%
\pgfpathlineto{\pgfqpoint{5.054224in}{0.693569in}}%
\pgfpathlineto{\pgfqpoint{5.054522in}{0.693576in}}%
\pgfpathlineto{\pgfqpoint{5.054819in}{0.693583in}}%
\pgfpathlineto{\pgfqpoint{5.055117in}{0.693591in}}%
\pgfpathlineto{\pgfqpoint{5.055414in}{0.693598in}}%
\pgfpathlineto{\pgfqpoint{5.055712in}{0.693605in}}%
\pgfpathlineto{\pgfqpoint{5.056009in}{0.693613in}}%
\pgfpathlineto{\pgfqpoint{5.056307in}{0.693620in}}%
\pgfpathlineto{\pgfqpoint{5.056604in}{0.693627in}}%
\pgfpathlineto{\pgfqpoint{5.056902in}{0.693635in}}%
\pgfpathlineto{\pgfqpoint{5.057199in}{0.693642in}}%
\pgfpathlineto{\pgfqpoint{5.057497in}{0.693649in}}%
\pgfpathlineto{\pgfqpoint{5.057794in}{0.693657in}}%
\pgfpathlineto{\pgfqpoint{5.058092in}{0.693664in}}%
\pgfpathlineto{\pgfqpoint{5.058389in}{0.693671in}}%
\pgfpathlineto{\pgfqpoint{5.058687in}{0.693679in}}%
\pgfpathlineto{\pgfqpoint{5.058984in}{0.693686in}}%
\pgfpathlineto{\pgfqpoint{5.059282in}{0.693693in}}%
\pgfpathlineto{\pgfqpoint{5.059579in}{0.693701in}}%
\pgfpathlineto{\pgfqpoint{5.059876in}{0.693708in}}%
\pgfpathlineto{\pgfqpoint{5.060174in}{0.693715in}}%
\pgfpathlineto{\pgfqpoint{5.060471in}{0.693723in}}%
\pgfpathlineto{\pgfqpoint{5.060769in}{0.693730in}}%
\pgfpathlineto{\pgfqpoint{5.061066in}{0.693737in}}%
\pgfpathlineto{\pgfqpoint{5.061364in}{0.693745in}}%
\pgfpathlineto{\pgfqpoint{5.061661in}{0.693752in}}%
\pgfpathlineto{\pgfqpoint{5.061959in}{0.693759in}}%
\pgfpathlineto{\pgfqpoint{5.062256in}{0.693767in}}%
\pgfpathlineto{\pgfqpoint{5.062554in}{0.693774in}}%
\pgfpathlineto{\pgfqpoint{5.062851in}{0.693781in}}%
\pgfpathlineto{\pgfqpoint{5.063149in}{0.693789in}}%
\pgfpathlineto{\pgfqpoint{5.063446in}{0.693796in}}%
\pgfpathlineto{\pgfqpoint{5.063744in}{0.693803in}}%
\pgfpathlineto{\pgfqpoint{5.064041in}{0.693811in}}%
\pgfpathlineto{\pgfqpoint{5.064339in}{0.693818in}}%
\pgfpathlineto{\pgfqpoint{5.064636in}{0.693825in}}%
\pgfpathlineto{\pgfqpoint{5.064934in}{0.693833in}}%
\pgfpathlineto{\pgfqpoint{5.065231in}{0.693840in}}%
\pgfpathlineto{\pgfqpoint{5.065529in}{0.693847in}}%
\pgfpathlineto{\pgfqpoint{5.065826in}{0.693855in}}%
\pgfpathlineto{\pgfqpoint{5.066124in}{0.693862in}}%
\pgfpathlineto{\pgfqpoint{5.066421in}{0.693869in}}%
\pgfpathlineto{\pgfqpoint{5.066718in}{0.693877in}}%
\pgfpathlineto{\pgfqpoint{5.067016in}{0.693884in}}%
\pgfpathlineto{\pgfqpoint{5.067313in}{0.693891in}}%
\pgfpathlineto{\pgfqpoint{5.067611in}{0.693899in}}%
\pgfpathlineto{\pgfqpoint{5.067908in}{0.693906in}}%
\pgfpathlineto{\pgfqpoint{5.068206in}{0.693913in}}%
\pgfpathlineto{\pgfqpoint{5.068503in}{0.693921in}}%
\pgfpathlineto{\pgfqpoint{5.068801in}{0.693928in}}%
\pgfpathlineto{\pgfqpoint{5.069098in}{0.693935in}}%
\pgfpathlineto{\pgfqpoint{5.069396in}{0.693943in}}%
\pgfpathlineto{\pgfqpoint{5.069693in}{0.693950in}}%
\pgfpathlineto{\pgfqpoint{5.069991in}{0.693957in}}%
\pgfpathlineto{\pgfqpoint{5.070288in}{0.693965in}}%
\pgfpathlineto{\pgfqpoint{5.070586in}{0.693972in}}%
\pgfpathlineto{\pgfqpoint{5.070883in}{0.693979in}}%
\pgfpathlineto{\pgfqpoint{5.071181in}{0.693987in}}%
\pgfpathlineto{\pgfqpoint{5.071478in}{0.693994in}}%
\pgfpathlineto{\pgfqpoint{5.071776in}{0.694001in}}%
\pgfpathlineto{\pgfqpoint{5.072073in}{0.694009in}}%
\pgfpathlineto{\pgfqpoint{5.072371in}{0.694016in}}%
\pgfpathlineto{\pgfqpoint{5.072668in}{0.694024in}}%
\pgfpathlineto{\pgfqpoint{5.072965in}{0.694031in}}%
\pgfpathlineto{\pgfqpoint{5.073263in}{0.694038in}}%
\pgfpathlineto{\pgfqpoint{5.073560in}{0.694046in}}%
\pgfpathlineto{\pgfqpoint{5.073858in}{0.694053in}}%
\pgfpathlineto{\pgfqpoint{5.074155in}{0.694060in}}%
\pgfpathlineto{\pgfqpoint{5.074453in}{0.694068in}}%
\pgfpathlineto{\pgfqpoint{5.074750in}{0.694075in}}%
\pgfpathlineto{\pgfqpoint{5.075048in}{0.694082in}}%
\pgfpathlineto{\pgfqpoint{5.075345in}{0.694090in}}%
\pgfpathlineto{\pgfqpoint{5.075643in}{0.694097in}}%
\pgfpathlineto{\pgfqpoint{5.075940in}{0.694104in}}%
\pgfpathlineto{\pgfqpoint{5.076238in}{0.694112in}}%
\pgfpathlineto{\pgfqpoint{5.076535in}{0.694119in}}%
\pgfpathlineto{\pgfqpoint{5.076833in}{0.694126in}}%
\pgfpathlineto{\pgfqpoint{5.077130in}{0.694134in}}%
\pgfpathlineto{\pgfqpoint{5.077428in}{0.694141in}}%
\pgfpathlineto{\pgfqpoint{5.077725in}{0.694148in}}%
\pgfpathlineto{\pgfqpoint{5.078023in}{0.694156in}}%
\pgfpathlineto{\pgfqpoint{5.078320in}{0.694163in}}%
\pgfpathlineto{\pgfqpoint{5.078618in}{0.694170in}}%
\pgfpathlineto{\pgfqpoint{5.078915in}{0.694178in}}%
\pgfpathlineto{\pgfqpoint{5.079213in}{0.694185in}}%
\pgfpathlineto{\pgfqpoint{5.079510in}{0.694192in}}%
\pgfpathlineto{\pgfqpoint{5.079807in}{0.694200in}}%
\pgfpathlineto{\pgfqpoint{5.080105in}{0.694207in}}%
\pgfpathlineto{\pgfqpoint{5.080402in}{0.694214in}}%
\pgfpathlineto{\pgfqpoint{5.080700in}{0.694222in}}%
\pgfpathlineto{\pgfqpoint{5.080997in}{0.694229in}}%
\pgfpathlineto{\pgfqpoint{5.081295in}{0.694236in}}%
\pgfpathlineto{\pgfqpoint{5.081592in}{0.694244in}}%
\pgfpathlineto{\pgfqpoint{5.081890in}{0.694251in}}%
\pgfpathlineto{\pgfqpoint{5.082187in}{0.694258in}}%
\pgfpathlineto{\pgfqpoint{5.082485in}{0.694266in}}%
\pgfpathlineto{\pgfqpoint{5.082782in}{0.694273in}}%
\pgfpathlineto{\pgfqpoint{5.083080in}{0.694280in}}%
\pgfpathlineto{\pgfqpoint{5.083377in}{0.694288in}}%
\pgfpathlineto{\pgfqpoint{5.083675in}{0.694295in}}%
\pgfpathlineto{\pgfqpoint{5.083972in}{0.694302in}}%
\pgfpathlineto{\pgfqpoint{5.084270in}{0.694310in}}%
\pgfpathlineto{\pgfqpoint{5.084567in}{0.694317in}}%
\pgfpathlineto{\pgfqpoint{5.084865in}{0.694324in}}%
\pgfpathlineto{\pgfqpoint{5.085162in}{0.694332in}}%
\pgfpathlineto{\pgfqpoint{5.085460in}{0.694339in}}%
\pgfpathlineto{\pgfqpoint{5.085757in}{0.694346in}}%
\pgfpathlineto{\pgfqpoint{5.086055in}{0.694354in}}%
\pgfpathlineto{\pgfqpoint{5.086352in}{0.694361in}}%
\pgfpathlineto{\pgfqpoint{5.086649in}{0.694368in}}%
\pgfpathlineto{\pgfqpoint{5.086947in}{0.694376in}}%
\pgfpathlineto{\pgfqpoint{5.087244in}{0.694383in}}%
\pgfpathlineto{\pgfqpoint{5.087542in}{0.694390in}}%
\pgfpathlineto{\pgfqpoint{5.087839in}{0.694398in}}%
\pgfpathlineto{\pgfqpoint{5.088137in}{0.694405in}}%
\pgfpathlineto{\pgfqpoint{5.088434in}{0.694412in}}%
\pgfpathlineto{\pgfqpoint{5.088732in}{0.694420in}}%
\pgfpathlineto{\pgfqpoint{5.089029in}{0.694427in}}%
\pgfpathlineto{\pgfqpoint{5.089327in}{0.694434in}}%
\pgfpathlineto{\pgfqpoint{5.089624in}{0.694442in}}%
\pgfpathlineto{\pgfqpoint{5.089922in}{0.694449in}}%
\pgfpathlineto{\pgfqpoint{5.090219in}{0.694456in}}%
\pgfpathlineto{\pgfqpoint{5.090517in}{0.694464in}}%
\pgfpathlineto{\pgfqpoint{5.090814in}{0.694471in}}%
\pgfpathlineto{\pgfqpoint{5.091112in}{0.694478in}}%
\pgfpathlineto{\pgfqpoint{5.091409in}{0.694486in}}%
\pgfpathlineto{\pgfqpoint{5.091707in}{0.694493in}}%
\pgfpathlineto{\pgfqpoint{5.092004in}{0.694500in}}%
\pgfpathlineto{\pgfqpoint{5.092302in}{0.694508in}}%
\pgfpathlineto{\pgfqpoint{5.092599in}{0.694515in}}%
\pgfpathlineto{\pgfqpoint{5.092896in}{0.694522in}}%
\pgfpathlineto{\pgfqpoint{5.093194in}{0.694530in}}%
\pgfpathlineto{\pgfqpoint{5.093491in}{0.694537in}}%
\pgfpathlineto{\pgfqpoint{5.093789in}{0.694545in}}%
\pgfpathlineto{\pgfqpoint{5.094086in}{0.694552in}}%
\pgfpathlineto{\pgfqpoint{5.094384in}{0.694559in}}%
\pgfpathlineto{\pgfqpoint{5.094681in}{0.694567in}}%
\pgfpathlineto{\pgfqpoint{5.094979in}{0.694574in}}%
\pgfpathlineto{\pgfqpoint{5.095276in}{0.694581in}}%
\pgfpathlineto{\pgfqpoint{5.095574in}{0.694589in}}%
\pgfpathlineto{\pgfqpoint{5.095871in}{0.694596in}}%
\pgfpathlineto{\pgfqpoint{5.096169in}{0.694603in}}%
\pgfpathlineto{\pgfqpoint{5.096466in}{0.694611in}}%
\pgfpathlineto{\pgfqpoint{5.096764in}{0.694618in}}%
\pgfpathlineto{\pgfqpoint{5.097061in}{0.694625in}}%
\pgfpathlineto{\pgfqpoint{5.097359in}{0.694633in}}%
\pgfpathlineto{\pgfqpoint{5.097656in}{0.694640in}}%
\pgfpathlineto{\pgfqpoint{5.097954in}{0.694647in}}%
\pgfpathlineto{\pgfqpoint{5.098251in}{0.694655in}}%
\pgfpathlineto{\pgfqpoint{5.098549in}{0.694662in}}%
\pgfpathlineto{\pgfqpoint{5.098846in}{0.694669in}}%
\pgfpathlineto{\pgfqpoint{5.099144in}{0.694677in}}%
\pgfpathlineto{\pgfqpoint{5.099441in}{0.694684in}}%
\pgfpathlineto{\pgfqpoint{5.099738in}{0.694691in}}%
\pgfpathlineto{\pgfqpoint{5.100036in}{0.694699in}}%
\pgfpathlineto{\pgfqpoint{5.100333in}{0.694706in}}%
\pgfpathlineto{\pgfqpoint{5.100631in}{0.694713in}}%
\pgfpathlineto{\pgfqpoint{5.100928in}{0.694721in}}%
\pgfpathlineto{\pgfqpoint{5.101226in}{0.694728in}}%
\pgfpathlineto{\pgfqpoint{5.101523in}{0.694735in}}%
\pgfpathlineto{\pgfqpoint{5.101821in}{0.694743in}}%
\pgfpathlineto{\pgfqpoint{5.102118in}{0.694750in}}%
\pgfpathlineto{\pgfqpoint{5.102416in}{0.694757in}}%
\pgfpathlineto{\pgfqpoint{5.102713in}{0.694765in}}%
\pgfpathlineto{\pgfqpoint{5.103011in}{0.694772in}}%
\pgfpathlineto{\pgfqpoint{5.103308in}{0.694779in}}%
\pgfpathlineto{\pgfqpoint{5.103606in}{0.694787in}}%
\pgfpathlineto{\pgfqpoint{5.103903in}{0.694794in}}%
\pgfpathlineto{\pgfqpoint{5.104201in}{0.694801in}}%
\pgfpathlineto{\pgfqpoint{5.104498in}{0.694809in}}%
\pgfpathlineto{\pgfqpoint{5.104796in}{0.694816in}}%
\pgfpathlineto{\pgfqpoint{5.105093in}{0.694823in}}%
\pgfpathlineto{\pgfqpoint{5.105391in}{0.694831in}}%
\pgfpathlineto{\pgfqpoint{5.105688in}{0.694838in}}%
\pgfpathlineto{\pgfqpoint{5.105986in}{0.694845in}}%
\pgfpathlineto{\pgfqpoint{5.106283in}{0.694853in}}%
\pgfpathlineto{\pgfqpoint{5.106580in}{0.694860in}}%
\pgfpathlineto{\pgfqpoint{5.106878in}{0.694867in}}%
\pgfpathlineto{\pgfqpoint{5.107175in}{0.694875in}}%
\pgfpathlineto{\pgfqpoint{5.107473in}{0.694882in}}%
\pgfpathlineto{\pgfqpoint{5.107770in}{0.694889in}}%
\pgfpathlineto{\pgfqpoint{5.108068in}{0.694897in}}%
\pgfpathlineto{\pgfqpoint{5.108365in}{0.694904in}}%
\pgfpathlineto{\pgfqpoint{5.108663in}{0.694911in}}%
\pgfpathlineto{\pgfqpoint{5.108960in}{0.694919in}}%
\pgfpathlineto{\pgfqpoint{5.109258in}{0.694926in}}%
\pgfpathlineto{\pgfqpoint{5.109555in}{0.694933in}}%
\pgfpathlineto{\pgfqpoint{5.109853in}{0.694941in}}%
\pgfpathlineto{\pgfqpoint{5.110150in}{0.694948in}}%
\pgfpathlineto{\pgfqpoint{5.110448in}{0.694955in}}%
\pgfpathlineto{\pgfqpoint{5.110745in}{0.694963in}}%
\pgfpathlineto{\pgfqpoint{5.111043in}{0.694970in}}%
\pgfpathlineto{\pgfqpoint{5.111340in}{0.694977in}}%
\pgfpathlineto{\pgfqpoint{5.111638in}{0.694985in}}%
\pgfpathlineto{\pgfqpoint{5.111935in}{0.694992in}}%
\pgfpathlineto{\pgfqpoint{5.112233in}{0.694999in}}%
\pgfpathlineto{\pgfqpoint{5.112530in}{0.695007in}}%
\pgfpathlineto{\pgfqpoint{5.112827in}{0.695014in}}%
\pgfpathlineto{\pgfqpoint{5.113125in}{0.695021in}}%
\pgfpathlineto{\pgfqpoint{5.113422in}{0.695029in}}%
\pgfpathlineto{\pgfqpoint{5.113720in}{0.695036in}}%
\pgfpathlineto{\pgfqpoint{5.114017in}{0.695043in}}%
\pgfpathlineto{\pgfqpoint{5.114315in}{0.695051in}}%
\pgfpathlineto{\pgfqpoint{5.114612in}{0.695058in}}%
\pgfpathlineto{\pgfqpoint{5.114910in}{0.695066in}}%
\pgfpathlineto{\pgfqpoint{5.115207in}{0.695073in}}%
\pgfpathlineto{\pgfqpoint{5.115505in}{0.695080in}}%
\pgfpathlineto{\pgfqpoint{5.115802in}{0.695088in}}%
\pgfpathlineto{\pgfqpoint{5.116100in}{0.695095in}}%
\pgfpathlineto{\pgfqpoint{5.116397in}{0.695102in}}%
\pgfpathlineto{\pgfqpoint{5.116695in}{0.695110in}}%
\pgfpathlineto{\pgfqpoint{5.116992in}{0.695117in}}%
\pgfpathlineto{\pgfqpoint{5.117290in}{0.695124in}}%
\pgfpathlineto{\pgfqpoint{5.117587in}{0.695132in}}%
\pgfpathlineto{\pgfqpoint{5.117885in}{0.695139in}}%
\pgfpathlineto{\pgfqpoint{5.118182in}{0.695146in}}%
\pgfpathlineto{\pgfqpoint{5.118480in}{0.695154in}}%
\pgfpathlineto{\pgfqpoint{5.118777in}{0.695161in}}%
\pgfpathlineto{\pgfqpoint{5.119075in}{0.695168in}}%
\pgfpathlineto{\pgfqpoint{5.119372in}{0.695176in}}%
\pgfpathlineto{\pgfqpoint{5.119669in}{0.695183in}}%
\pgfpathlineto{\pgfqpoint{5.119967in}{0.695190in}}%
\pgfpathlineto{\pgfqpoint{5.120264in}{0.695198in}}%
\pgfpathlineto{\pgfqpoint{5.120562in}{0.695205in}}%
\pgfpathlineto{\pgfqpoint{5.120859in}{0.695212in}}%
\pgfpathlineto{\pgfqpoint{5.121157in}{0.695220in}}%
\pgfpathlineto{\pgfqpoint{5.121454in}{0.695227in}}%
\pgfpathlineto{\pgfqpoint{5.121752in}{0.695234in}}%
\pgfpathlineto{\pgfqpoint{5.122049in}{0.695242in}}%
\pgfpathlineto{\pgfqpoint{5.122347in}{0.695249in}}%
\pgfpathlineto{\pgfqpoint{5.122644in}{0.695256in}}%
\pgfpathlineto{\pgfqpoint{5.122942in}{0.695264in}}%
\pgfpathlineto{\pgfqpoint{5.123239in}{0.695271in}}%
\pgfpathlineto{\pgfqpoint{5.123537in}{0.695278in}}%
\pgfpathlineto{\pgfqpoint{5.123834in}{0.695286in}}%
\pgfpathlineto{\pgfqpoint{5.124132in}{0.695293in}}%
\pgfpathlineto{\pgfqpoint{5.124429in}{0.695300in}}%
\pgfpathlineto{\pgfqpoint{5.124727in}{0.695308in}}%
\pgfpathlineto{\pgfqpoint{5.125024in}{0.695315in}}%
\pgfpathlineto{\pgfqpoint{5.125322in}{0.695322in}}%
\pgfpathlineto{\pgfqpoint{5.125619in}{0.695330in}}%
\pgfpathlineto{\pgfqpoint{5.125917in}{0.695337in}}%
\pgfpathlineto{\pgfqpoint{5.126214in}{0.695344in}}%
\pgfpathlineto{\pgfqpoint{5.126511in}{0.695352in}}%
\pgfpathlineto{\pgfqpoint{5.126809in}{0.695359in}}%
\pgfpathlineto{\pgfqpoint{5.127106in}{0.695366in}}%
\pgfpathlineto{\pgfqpoint{5.127404in}{0.695371in}}%
\pgfpathlineto{\pgfqpoint{5.127701in}{0.695371in}}%
\pgfpathlineto{\pgfqpoint{5.127999in}{0.695371in}}%
\pgfpathlineto{\pgfqpoint{5.128296in}{0.695371in}}%
\pgfpathlineto{\pgfqpoint{5.128594in}{0.695370in}}%
\pgfpathlineto{\pgfqpoint{5.128891in}{0.695370in}}%
\pgfpathlineto{\pgfqpoint{5.129189in}{0.695370in}}%
\pgfpathlineto{\pgfqpoint{5.129486in}{0.695369in}}%
\pgfpathlineto{\pgfqpoint{5.129784in}{0.695369in}}%
\pgfpathlineto{\pgfqpoint{5.130081in}{0.695369in}}%
\pgfpathlineto{\pgfqpoint{5.130379in}{0.695368in}}%
\pgfpathlineto{\pgfqpoint{5.130676in}{0.695368in}}%
\pgfpathlineto{\pgfqpoint{5.130974in}{0.695368in}}%
\pgfpathlineto{\pgfqpoint{5.131271in}{0.695368in}}%
\pgfpathlineto{\pgfqpoint{5.131569in}{0.695367in}}%
\pgfpathlineto{\pgfqpoint{5.131866in}{0.695367in}}%
\pgfpathlineto{\pgfqpoint{5.132164in}{0.695367in}}%
\pgfpathlineto{\pgfqpoint{5.132461in}{0.695366in}}%
\pgfpathlineto{\pgfqpoint{5.132758in}{0.695366in}}%
\pgfpathlineto{\pgfqpoint{5.133056in}{0.695366in}}%
\pgfpathlineto{\pgfqpoint{5.133353in}{0.695365in}}%
\pgfpathlineto{\pgfqpoint{5.133651in}{0.695365in}}%
\pgfpathlineto{\pgfqpoint{5.133948in}{0.695365in}}%
\pgfpathlineto{\pgfqpoint{5.134246in}{0.695364in}}%
\pgfpathlineto{\pgfqpoint{5.134543in}{0.695364in}}%
\pgfpathlineto{\pgfqpoint{5.134841in}{0.695364in}}%
\pgfpathlineto{\pgfqpoint{5.135138in}{0.695363in}}%
\pgfpathlineto{\pgfqpoint{5.135436in}{0.695363in}}%
\pgfpathlineto{\pgfqpoint{5.135733in}{0.695363in}}%
\pgfpathlineto{\pgfqpoint{5.136031in}{0.695362in}}%
\pgfpathlineto{\pgfqpoint{5.136328in}{0.695362in}}%
\pgfpathlineto{\pgfqpoint{5.136626in}{0.695362in}}%
\pgfpathlineto{\pgfqpoint{5.136923in}{0.695361in}}%
\pgfpathlineto{\pgfqpoint{5.137221in}{0.695361in}}%
\pgfpathlineto{\pgfqpoint{5.137518in}{0.695361in}}%
\pgfpathlineto{\pgfqpoint{5.137816in}{0.695360in}}%
\pgfpathlineto{\pgfqpoint{5.138113in}{0.695360in}}%
\pgfpathlineto{\pgfqpoint{5.138411in}{0.695360in}}%
\pgfpathlineto{\pgfqpoint{5.138708in}{0.695359in}}%
\pgfpathlineto{\pgfqpoint{5.139006in}{0.695359in}}%
\pgfpathlineto{\pgfqpoint{5.139303in}{0.695359in}}%
\pgfpathlineto{\pgfqpoint{5.139600in}{0.695358in}}%
\pgfpathlineto{\pgfqpoint{5.139898in}{0.695358in}}%
\pgfpathlineto{\pgfqpoint{5.140195in}{0.695358in}}%
\pgfpathlineto{\pgfqpoint{5.140493in}{0.695358in}}%
\pgfpathlineto{\pgfqpoint{5.140790in}{0.695357in}}%
\pgfpathlineto{\pgfqpoint{5.141088in}{0.695357in}}%
\pgfpathlineto{\pgfqpoint{5.141385in}{0.695357in}}%
\pgfpathlineto{\pgfqpoint{5.141683in}{0.695356in}}%
\pgfpathlineto{\pgfqpoint{5.141980in}{0.695356in}}%
\pgfpathlineto{\pgfqpoint{5.142278in}{0.695356in}}%
\pgfpathlineto{\pgfqpoint{5.142575in}{0.695355in}}%
\pgfpathlineto{\pgfqpoint{5.142873in}{0.695355in}}%
\pgfpathlineto{\pgfqpoint{5.143170in}{0.695355in}}%
\pgfpathlineto{\pgfqpoint{5.143468in}{0.695354in}}%
\pgfpathlineto{\pgfqpoint{5.143765in}{0.695354in}}%
\pgfpathlineto{\pgfqpoint{5.144063in}{0.695354in}}%
\pgfpathlineto{\pgfqpoint{5.144360in}{0.695353in}}%
\pgfpathlineto{\pgfqpoint{5.144658in}{0.695353in}}%
\pgfpathlineto{\pgfqpoint{5.144955in}{0.695353in}}%
\pgfpathlineto{\pgfqpoint{5.145253in}{0.695352in}}%
\pgfpathlineto{\pgfqpoint{5.145550in}{0.695352in}}%
\pgfpathlineto{\pgfqpoint{5.145848in}{0.695352in}}%
\pgfpathlineto{\pgfqpoint{5.146145in}{0.695351in}}%
\pgfpathlineto{\pgfqpoint{5.146442in}{0.695351in}}%
\pgfpathlineto{\pgfqpoint{5.146740in}{0.695351in}}%
\pgfpathlineto{\pgfqpoint{5.147037in}{0.695350in}}%
\pgfpathlineto{\pgfqpoint{5.147335in}{0.695350in}}%
\pgfpathlineto{\pgfqpoint{5.147632in}{0.695350in}}%
\pgfpathlineto{\pgfqpoint{5.147930in}{0.695349in}}%
\pgfpathlineto{\pgfqpoint{5.148227in}{0.695349in}}%
\pgfpathlineto{\pgfqpoint{5.148525in}{0.695349in}}%
\pgfpathlineto{\pgfqpoint{5.148822in}{0.695348in}}%
\pgfpathlineto{\pgfqpoint{5.149120in}{0.695348in}}%
\pgfpathlineto{\pgfqpoint{5.149417in}{0.695348in}}%
\pgfpathlineto{\pgfqpoint{5.149715in}{0.695348in}}%
\pgfpathlineto{\pgfqpoint{5.150012in}{0.695347in}}%
\pgfpathlineto{\pgfqpoint{5.150310in}{0.695347in}}%
\pgfpathlineto{\pgfqpoint{5.150607in}{0.695347in}}%
\pgfpathlineto{\pgfqpoint{5.150905in}{0.695346in}}%
\pgfpathlineto{\pgfqpoint{5.151202in}{0.695346in}}%
\pgfpathlineto{\pgfqpoint{5.151500in}{0.695346in}}%
\pgfpathlineto{\pgfqpoint{5.151797in}{0.695345in}}%
\pgfpathlineto{\pgfqpoint{5.152095in}{0.695345in}}%
\pgfpathlineto{\pgfqpoint{5.152392in}{0.695345in}}%
\pgfpathlineto{\pgfqpoint{5.152689in}{0.695344in}}%
\pgfpathlineto{\pgfqpoint{5.152987in}{0.695344in}}%
\pgfpathlineto{\pgfqpoint{5.153284in}{0.695344in}}%
\pgfpathlineto{\pgfqpoint{5.153582in}{0.695343in}}%
\pgfpathlineto{\pgfqpoint{5.153879in}{0.695343in}}%
\pgfpathlineto{\pgfqpoint{5.154177in}{0.695343in}}%
\pgfpathlineto{\pgfqpoint{5.154474in}{0.695342in}}%
\pgfpathlineto{\pgfqpoint{5.154772in}{0.695342in}}%
\pgfpathlineto{\pgfqpoint{5.155069in}{0.695342in}}%
\pgfpathlineto{\pgfqpoint{5.155367in}{0.695341in}}%
\pgfpathlineto{\pgfqpoint{5.155664in}{0.695341in}}%
\pgfpathlineto{\pgfqpoint{5.155962in}{0.695341in}}%
\pgfpathlineto{\pgfqpoint{5.156259in}{0.695340in}}%
\pgfpathlineto{\pgfqpoint{5.156557in}{0.695340in}}%
\pgfpathlineto{\pgfqpoint{5.156854in}{0.695340in}}%
\pgfpathlineto{\pgfqpoint{5.157152in}{0.695339in}}%
\pgfpathlineto{\pgfqpoint{5.157449in}{0.695339in}}%
\pgfpathlineto{\pgfqpoint{5.157747in}{0.695339in}}%
\pgfpathlineto{\pgfqpoint{5.158044in}{0.695338in}}%
\pgfpathlineto{\pgfqpoint{5.158342in}{0.695338in}}%
\pgfpathlineto{\pgfqpoint{5.158639in}{0.695338in}}%
\pgfpathlineto{\pgfqpoint{5.158937in}{0.695338in}}%
\pgfpathlineto{\pgfqpoint{5.159234in}{0.695337in}}%
\pgfpathlineto{\pgfqpoint{5.159531in}{0.695337in}}%
\pgfpathlineto{\pgfqpoint{5.159829in}{0.695337in}}%
\pgfpathlineto{\pgfqpoint{5.160126in}{0.695336in}}%
\pgfpathlineto{\pgfqpoint{5.160424in}{0.695336in}}%
\pgfpathlineto{\pgfqpoint{5.160721in}{0.695336in}}%
\pgfpathlineto{\pgfqpoint{5.161019in}{0.695335in}}%
\pgfpathlineto{\pgfqpoint{5.161316in}{0.695335in}}%
\pgfpathlineto{\pgfqpoint{5.161614in}{0.695335in}}%
\pgfpathlineto{\pgfqpoint{5.161911in}{0.695334in}}%
\pgfpathlineto{\pgfqpoint{5.162209in}{0.695334in}}%
\pgfpathlineto{\pgfqpoint{5.162506in}{0.695334in}}%
\pgfpathlineto{\pgfqpoint{5.162804in}{0.695333in}}%
\pgfpathlineto{\pgfqpoint{5.163101in}{0.695333in}}%
\pgfpathlineto{\pgfqpoint{5.163399in}{0.695333in}}%
\pgfpathlineto{\pgfqpoint{5.163696in}{0.695332in}}%
\pgfpathlineto{\pgfqpoint{5.163994in}{0.695332in}}%
\pgfpathlineto{\pgfqpoint{5.164291in}{0.695332in}}%
\pgfpathlineto{\pgfqpoint{5.164589in}{0.695331in}}%
\pgfpathlineto{\pgfqpoint{5.164886in}{0.695331in}}%
\pgfpathlineto{\pgfqpoint{5.165184in}{0.695331in}}%
\pgfpathlineto{\pgfqpoint{5.165481in}{0.695330in}}%
\pgfpathlineto{\pgfqpoint{5.165779in}{0.695330in}}%
\pgfpathlineto{\pgfqpoint{5.166076in}{0.695330in}}%
\pgfpathlineto{\pgfqpoint{5.166373in}{0.695329in}}%
\pgfpathlineto{\pgfqpoint{5.166671in}{0.695329in}}%
\pgfpathlineto{\pgfqpoint{5.166968in}{0.695329in}}%
\pgfpathlineto{\pgfqpoint{5.167266in}{0.695328in}}%
\pgfpathlineto{\pgfqpoint{5.167563in}{0.695328in}}%
\pgfpathlineto{\pgfqpoint{5.167861in}{0.695328in}}%
\pgfpathlineto{\pgfqpoint{5.168158in}{0.695328in}}%
\pgfpathlineto{\pgfqpoint{5.168456in}{0.695327in}}%
\pgfpathlineto{\pgfqpoint{5.168753in}{0.695327in}}%
\pgfpathlineto{\pgfqpoint{5.169051in}{0.695327in}}%
\pgfpathlineto{\pgfqpoint{5.169348in}{0.695326in}}%
\pgfpathlineto{\pgfqpoint{5.169646in}{0.695326in}}%
\pgfpathlineto{\pgfqpoint{5.169943in}{0.695326in}}%
\pgfpathlineto{\pgfqpoint{5.170241in}{0.695325in}}%
\pgfpathlineto{\pgfqpoint{5.170538in}{0.695325in}}%
\pgfpathlineto{\pgfqpoint{5.170836in}{0.695325in}}%
\pgfpathlineto{\pgfqpoint{5.171133in}{0.695324in}}%
\pgfpathlineto{\pgfqpoint{5.171431in}{0.695324in}}%
\pgfpathlineto{\pgfqpoint{5.171728in}{0.695324in}}%
\pgfpathlineto{\pgfqpoint{5.172026in}{0.695323in}}%
\pgfpathlineto{\pgfqpoint{5.172323in}{0.695323in}}%
\pgfpathlineto{\pgfqpoint{5.172620in}{0.695323in}}%
\pgfpathlineto{\pgfqpoint{5.172918in}{0.695322in}}%
\pgfpathlineto{\pgfqpoint{5.173215in}{0.695322in}}%
\pgfpathlineto{\pgfqpoint{5.173513in}{0.695322in}}%
\pgfpathlineto{\pgfqpoint{5.173810in}{0.695321in}}%
\pgfpathlineto{\pgfqpoint{5.174108in}{0.695321in}}%
\pgfpathlineto{\pgfqpoint{5.174405in}{0.695321in}}%
\pgfpathlineto{\pgfqpoint{5.174703in}{0.695320in}}%
\pgfpathlineto{\pgfqpoint{5.175000in}{0.695320in}}%
\pgfpathlineto{\pgfqpoint{5.175298in}{0.695320in}}%
\pgfpathlineto{\pgfqpoint{5.175595in}{0.695319in}}%
\pgfpathlineto{\pgfqpoint{5.175893in}{0.695319in}}%
\pgfpathlineto{\pgfqpoint{5.176190in}{0.695319in}}%
\pgfpathlineto{\pgfqpoint{5.176488in}{0.695318in}}%
\pgfpathlineto{\pgfqpoint{5.176785in}{0.695318in}}%
\pgfpathlineto{\pgfqpoint{5.177083in}{0.695318in}}%
\pgfpathlineto{\pgfqpoint{5.177380in}{0.695318in}}%
\pgfpathlineto{\pgfqpoint{5.177678in}{0.695317in}}%
\pgfpathlineto{\pgfqpoint{5.177975in}{0.695317in}}%
\pgfpathlineto{\pgfqpoint{5.178273in}{0.695317in}}%
\pgfpathlineto{\pgfqpoint{5.178570in}{0.695316in}}%
\pgfpathlineto{\pgfqpoint{5.178868in}{0.695316in}}%
\pgfpathlineto{\pgfqpoint{5.179165in}{0.695316in}}%
\pgfpathlineto{\pgfqpoint{5.179462in}{0.695315in}}%
\pgfpathlineto{\pgfqpoint{5.179760in}{0.695315in}}%
\pgfpathlineto{\pgfqpoint{5.180057in}{0.695315in}}%
\pgfpathlineto{\pgfqpoint{5.180355in}{0.695314in}}%
\pgfpathlineto{\pgfqpoint{5.180652in}{0.695314in}}%
\pgfpathlineto{\pgfqpoint{5.180950in}{0.695314in}}%
\pgfpathlineto{\pgfqpoint{5.181247in}{0.695313in}}%
\pgfpathlineto{\pgfqpoint{5.181545in}{0.695313in}}%
\pgfpathlineto{\pgfqpoint{5.181842in}{0.695313in}}%
\pgfpathlineto{\pgfqpoint{5.182140in}{0.695312in}}%
\pgfpathlineto{\pgfqpoint{5.182437in}{0.695312in}}%
\pgfpathlineto{\pgfqpoint{5.182735in}{0.695312in}}%
\pgfpathlineto{\pgfqpoint{5.183032in}{0.695311in}}%
\pgfpathlineto{\pgfqpoint{5.183330in}{0.695311in}}%
\pgfpathlineto{\pgfqpoint{5.183627in}{0.695311in}}%
\pgfpathlineto{\pgfqpoint{5.183925in}{0.695310in}}%
\pgfpathlineto{\pgfqpoint{5.184222in}{0.695310in}}%
\pgfpathlineto{\pgfqpoint{5.184520in}{0.695310in}}%
\pgfpathlineto{\pgfqpoint{5.184817in}{0.695309in}}%
\pgfpathlineto{\pgfqpoint{5.185115in}{0.695309in}}%
\pgfpathlineto{\pgfqpoint{5.185412in}{0.695309in}}%
\pgfpathlineto{\pgfqpoint{5.185710in}{0.695308in}}%
\pgfpathlineto{\pgfqpoint{5.186007in}{0.695308in}}%
\pgfpathlineto{\pgfqpoint{5.186304in}{0.695308in}}%
\pgfpathlineto{\pgfqpoint{5.186602in}{0.695308in}}%
\pgfpathlineto{\pgfqpoint{5.186899in}{0.695307in}}%
\pgfpathlineto{\pgfqpoint{5.187197in}{0.695307in}}%
\pgfpathlineto{\pgfqpoint{5.187494in}{0.695307in}}%
\pgfpathlineto{\pgfqpoint{5.187792in}{0.695306in}}%
\pgfpathlineto{\pgfqpoint{5.188089in}{0.695306in}}%
\pgfpathlineto{\pgfqpoint{5.188387in}{0.695306in}}%
\pgfpathlineto{\pgfqpoint{5.188684in}{0.695305in}}%
\pgfpathlineto{\pgfqpoint{5.188982in}{0.695305in}}%
\pgfpathlineto{\pgfqpoint{5.189279in}{0.695305in}}%
\pgfpathlineto{\pgfqpoint{5.189577in}{0.695304in}}%
\pgfpathlineto{\pgfqpoint{5.189874in}{0.695304in}}%
\pgfpathlineto{\pgfqpoint{5.190172in}{0.695304in}}%
\pgfpathlineto{\pgfqpoint{5.190469in}{0.695303in}}%
\pgfpathlineto{\pgfqpoint{5.190767in}{0.695303in}}%
\pgfpathlineto{\pgfqpoint{5.191064in}{0.695303in}}%
\pgfpathlineto{\pgfqpoint{5.191362in}{0.695302in}}%
\pgfpathlineto{\pgfqpoint{5.191659in}{0.695302in}}%
\pgfpathlineto{\pgfqpoint{5.191957in}{0.695302in}}%
\pgfpathlineto{\pgfqpoint{5.192254in}{0.695301in}}%
\pgfpathlineto{\pgfqpoint{5.192551in}{0.695301in}}%
\pgfpathlineto{\pgfqpoint{5.192849in}{0.695301in}}%
\pgfpathlineto{\pgfqpoint{5.193146in}{0.695300in}}%
\pgfpathlineto{\pgfqpoint{5.193444in}{0.695300in}}%
\pgfpathlineto{\pgfqpoint{5.193741in}{0.695300in}}%
\pgfpathlineto{\pgfqpoint{5.194039in}{0.695299in}}%
\pgfpathlineto{\pgfqpoint{5.194336in}{0.695299in}}%
\pgfpathlineto{\pgfqpoint{5.194634in}{0.695299in}}%
\pgfpathlineto{\pgfqpoint{5.194931in}{0.695298in}}%
\pgfpathlineto{\pgfqpoint{5.195229in}{0.695298in}}%
\pgfpathlineto{\pgfqpoint{5.195526in}{0.695298in}}%
\pgfpathlineto{\pgfqpoint{5.195824in}{0.695298in}}%
\pgfpathlineto{\pgfqpoint{5.196121in}{0.695297in}}%
\pgfpathlineto{\pgfqpoint{5.196419in}{0.695297in}}%
\pgfpathlineto{\pgfqpoint{5.196716in}{0.695297in}}%
\pgfpathlineto{\pgfqpoint{5.197014in}{0.695296in}}%
\pgfpathlineto{\pgfqpoint{5.197311in}{0.695296in}}%
\pgfpathlineto{\pgfqpoint{5.197609in}{0.695296in}}%
\pgfpathlineto{\pgfqpoint{5.197906in}{0.695295in}}%
\pgfpathlineto{\pgfqpoint{5.198204in}{0.695295in}}%
\pgfpathlineto{\pgfqpoint{5.198501in}{0.695295in}}%
\pgfpathlineto{\pgfqpoint{5.198799in}{0.695294in}}%
\pgfpathlineto{\pgfqpoint{5.199096in}{0.695294in}}%
\pgfpathlineto{\pgfqpoint{5.199393in}{0.695294in}}%
\pgfpathlineto{\pgfqpoint{5.199691in}{0.695293in}}%
\pgfpathlineto{\pgfqpoint{5.199988in}{0.695293in}}%
\pgfpathlineto{\pgfqpoint{5.200286in}{0.695293in}}%
\pgfpathlineto{\pgfqpoint{5.200583in}{0.695292in}}%
\pgfpathlineto{\pgfqpoint{5.200881in}{0.695292in}}%
\pgfpathlineto{\pgfqpoint{5.201178in}{0.695292in}}%
\pgfpathlineto{\pgfqpoint{5.201476in}{0.695291in}}%
\pgfpathlineto{\pgfqpoint{5.201773in}{0.695291in}}%
\pgfpathlineto{\pgfqpoint{5.202071in}{0.695291in}}%
\pgfpathlineto{\pgfqpoint{5.202368in}{0.695290in}}%
\pgfpathlineto{\pgfqpoint{5.202666in}{0.695290in}}%
\pgfpathlineto{\pgfqpoint{5.202963in}{0.695290in}}%
\pgfpathlineto{\pgfqpoint{5.203261in}{0.695289in}}%
\pgfpathlineto{\pgfqpoint{5.203558in}{0.695289in}}%
\pgfpathlineto{\pgfqpoint{5.203856in}{0.695289in}}%
\pgfpathlineto{\pgfqpoint{5.204153in}{0.695289in}}%
\pgfpathlineto{\pgfqpoint{5.204451in}{0.695288in}}%
\pgfpathlineto{\pgfqpoint{5.204748in}{0.695288in}}%
\pgfpathlineto{\pgfqpoint{5.205046in}{0.695288in}}%
\pgfpathlineto{\pgfqpoint{5.205343in}{0.695287in}}%
\pgfpathlineto{\pgfqpoint{5.205641in}{0.695287in}}%
\pgfpathlineto{\pgfqpoint{5.205938in}{0.695287in}}%
\pgfpathlineto{\pgfqpoint{5.206235in}{0.695286in}}%
\pgfpathlineto{\pgfqpoint{5.206533in}{0.695286in}}%
\pgfpathlineto{\pgfqpoint{5.206830in}{0.695286in}}%
\pgfpathlineto{\pgfqpoint{5.207128in}{0.695285in}}%
\pgfpathlineto{\pgfqpoint{5.207425in}{0.695285in}}%
\pgfpathlineto{\pgfqpoint{5.207723in}{0.695285in}}%
\pgfpathlineto{\pgfqpoint{5.208020in}{0.695284in}}%
\pgfpathlineto{\pgfqpoint{5.208318in}{0.695284in}}%
\pgfpathlineto{\pgfqpoint{5.208615in}{0.695284in}}%
\pgfpathlineto{\pgfqpoint{5.208913in}{0.695283in}}%
\pgfpathlineto{\pgfqpoint{5.209210in}{0.695283in}}%
\pgfpathlineto{\pgfqpoint{5.209508in}{0.695283in}}%
\pgfpathlineto{\pgfqpoint{5.209805in}{0.695282in}}%
\pgfpathlineto{\pgfqpoint{5.210103in}{0.695282in}}%
\pgfpathlineto{\pgfqpoint{5.210400in}{0.695282in}}%
\pgfpathlineto{\pgfqpoint{5.210698in}{0.695281in}}%
\pgfpathlineto{\pgfqpoint{5.210995in}{0.695281in}}%
\pgfpathlineto{\pgfqpoint{5.211293in}{0.695281in}}%
\pgfpathlineto{\pgfqpoint{5.211590in}{0.695280in}}%
\pgfpathlineto{\pgfqpoint{5.211888in}{0.695280in}}%
\pgfpathlineto{\pgfqpoint{5.212185in}{0.695280in}}%
\pgfpathlineto{\pgfqpoint{5.212482in}{0.695279in}}%
\pgfpathlineto{\pgfqpoint{5.212780in}{0.695279in}}%
\pgfpathlineto{\pgfqpoint{5.213077in}{0.695279in}}%
\pgfpathlineto{\pgfqpoint{5.213375in}{0.695279in}}%
\pgfpathlineto{\pgfqpoint{5.213672in}{0.695278in}}%
\pgfpathlineto{\pgfqpoint{5.213970in}{0.695278in}}%
\pgfpathlineto{\pgfqpoint{5.214267in}{0.695278in}}%
\pgfpathlineto{\pgfqpoint{5.214565in}{0.695277in}}%
\pgfpathlineto{\pgfqpoint{5.214862in}{0.695277in}}%
\pgfpathlineto{\pgfqpoint{5.215160in}{0.695277in}}%
\pgfpathlineto{\pgfqpoint{5.215457in}{0.695276in}}%
\pgfpathlineto{\pgfqpoint{5.215755in}{0.695276in}}%
\pgfpathlineto{\pgfqpoint{5.216052in}{0.695276in}}%
\pgfpathlineto{\pgfqpoint{5.216350in}{0.695275in}}%
\pgfpathlineto{\pgfqpoint{5.216647in}{0.695275in}}%
\pgfpathlineto{\pgfqpoint{5.216945in}{0.695275in}}%
\pgfpathlineto{\pgfqpoint{5.217242in}{0.695274in}}%
\pgfpathlineto{\pgfqpoint{5.217540in}{0.695274in}}%
\pgfpathlineto{\pgfqpoint{5.217837in}{0.695274in}}%
\pgfpathlineto{\pgfqpoint{5.218135in}{0.695273in}}%
\pgfpathlineto{\pgfqpoint{5.218432in}{0.695273in}}%
\pgfpathlineto{\pgfqpoint{5.218730in}{0.695273in}}%
\pgfpathlineto{\pgfqpoint{5.219027in}{0.695272in}}%
\pgfpathlineto{\pgfqpoint{5.219324in}{0.695272in}}%
\pgfpathlineto{\pgfqpoint{5.219622in}{0.695272in}}%
\pgfpathlineto{\pgfqpoint{5.219919in}{0.695271in}}%
\pgfpathlineto{\pgfqpoint{5.220217in}{0.695271in}}%
\pgfpathlineto{\pgfqpoint{5.220514in}{0.695271in}}%
\pgfpathlineto{\pgfqpoint{5.220812in}{0.695270in}}%
\pgfpathlineto{\pgfqpoint{5.221109in}{0.695270in}}%
\pgfpathlineto{\pgfqpoint{5.221407in}{0.695270in}}%
\pgfpathlineto{\pgfqpoint{5.221704in}{0.695269in}}%
\pgfpathlineto{\pgfqpoint{5.222002in}{0.695269in}}%
\pgfpathlineto{\pgfqpoint{5.222299in}{0.695269in}}%
\pgfpathlineto{\pgfqpoint{5.222597in}{0.695269in}}%
\pgfpathlineto{\pgfqpoint{5.222894in}{0.695268in}}%
\pgfpathlineto{\pgfqpoint{5.223192in}{0.695268in}}%
\pgfpathlineto{\pgfqpoint{5.223489in}{0.695268in}}%
\pgfpathlineto{\pgfqpoint{5.223787in}{0.695267in}}%
\pgfpathlineto{\pgfqpoint{5.224084in}{0.695267in}}%
\pgfpathlineto{\pgfqpoint{5.224382in}{0.695267in}}%
\pgfpathlineto{\pgfqpoint{5.224679in}{0.695266in}}%
\pgfpathlineto{\pgfqpoint{5.224977in}{0.695266in}}%
\pgfpathlineto{\pgfqpoint{5.225274in}{0.695266in}}%
\pgfpathlineto{\pgfqpoint{5.225572in}{0.695265in}}%
\pgfpathlineto{\pgfqpoint{5.225869in}{0.695265in}}%
\pgfpathlineto{\pgfqpoint{5.226166in}{0.695265in}}%
\pgfpathlineto{\pgfqpoint{5.226464in}{0.695264in}}%
\pgfpathlineto{\pgfqpoint{5.226761in}{0.695264in}}%
\pgfpathlineto{\pgfqpoint{5.227059in}{0.695264in}}%
\pgfpathlineto{\pgfqpoint{5.227356in}{0.695263in}}%
\pgfpathlineto{\pgfqpoint{5.227654in}{0.695263in}}%
\pgfpathlineto{\pgfqpoint{5.227951in}{0.695263in}}%
\pgfpathlineto{\pgfqpoint{5.228249in}{0.695262in}}%
\pgfpathlineto{\pgfqpoint{5.228546in}{0.695262in}}%
\pgfpathlineto{\pgfqpoint{5.228844in}{0.695262in}}%
\pgfpathlineto{\pgfqpoint{5.229141in}{0.695261in}}%
\pgfpathlineto{\pgfqpoint{5.229439in}{0.695261in}}%
\pgfpathlineto{\pgfqpoint{5.229736in}{0.695261in}}%
\pgfpathlineto{\pgfqpoint{5.230034in}{0.695260in}}%
\pgfpathlineto{\pgfqpoint{5.230331in}{0.695260in}}%
\pgfpathlineto{\pgfqpoint{5.230629in}{0.695260in}}%
\pgfpathlineto{\pgfqpoint{5.230926in}{0.695259in}}%
\pgfpathlineto{\pgfqpoint{5.231224in}{0.695259in}}%
\pgfpathlineto{\pgfqpoint{5.231521in}{0.695259in}}%
\pgfpathlineto{\pgfqpoint{5.231819in}{0.695259in}}%
\pgfpathlineto{\pgfqpoint{5.232116in}{0.695258in}}%
\pgfpathlineto{\pgfqpoint{5.232413in}{0.695258in}}%
\pgfpathlineto{\pgfqpoint{5.232711in}{0.695258in}}%
\pgfpathlineto{\pgfqpoint{5.233008in}{0.695257in}}%
\pgfpathlineto{\pgfqpoint{5.233306in}{0.695257in}}%
\pgfpathlineto{\pgfqpoint{5.233603in}{0.695257in}}%
\pgfpathlineto{\pgfqpoint{5.233901in}{0.695256in}}%
\pgfpathlineto{\pgfqpoint{5.234198in}{0.695256in}}%
\pgfpathlineto{\pgfqpoint{5.234496in}{0.695256in}}%
\pgfpathlineto{\pgfqpoint{5.234793in}{0.695255in}}%
\pgfpathlineto{\pgfqpoint{5.235091in}{0.695255in}}%
\pgfpathlineto{\pgfqpoint{5.235388in}{0.695255in}}%
\pgfpathlineto{\pgfqpoint{5.235686in}{0.695254in}}%
\pgfpathlineto{\pgfqpoint{5.235983in}{0.695254in}}%
\pgfpathlineto{\pgfqpoint{5.236281in}{0.695254in}}%
\pgfpathlineto{\pgfqpoint{5.236578in}{0.695253in}}%
\pgfpathlineto{\pgfqpoint{5.236876in}{0.695253in}}%
\pgfpathlineto{\pgfqpoint{5.237173in}{0.695253in}}%
\pgfpathlineto{\pgfqpoint{5.237471in}{0.695252in}}%
\pgfpathlineto{\pgfqpoint{5.237768in}{0.695252in}}%
\pgfpathlineto{\pgfqpoint{5.238066in}{0.695252in}}%
\pgfpathlineto{\pgfqpoint{5.238363in}{0.695251in}}%
\pgfpathlineto{\pgfqpoint{5.238661in}{0.695251in}}%
\pgfpathlineto{\pgfqpoint{5.238958in}{0.695251in}}%
\pgfpathlineto{\pgfqpoint{5.239255in}{0.695250in}}%
\pgfpathlineto{\pgfqpoint{5.239553in}{0.695250in}}%
\pgfpathlineto{\pgfqpoint{5.239850in}{0.695250in}}%
\pgfpathlineto{\pgfqpoint{5.240148in}{0.695249in}}%
\pgfpathlineto{\pgfqpoint{5.240445in}{0.695249in}}%
\pgfpathlineto{\pgfqpoint{5.240743in}{0.695249in}}%
\pgfpathlineto{\pgfqpoint{5.241040in}{0.695249in}}%
\pgfpathlineto{\pgfqpoint{5.241338in}{0.695248in}}%
\pgfpathlineto{\pgfqpoint{5.241635in}{0.695248in}}%
\pgfpathlineto{\pgfqpoint{5.241933in}{0.695248in}}%
\pgfpathlineto{\pgfqpoint{5.242230in}{0.695247in}}%
\pgfpathlineto{\pgfqpoint{5.242528in}{0.695247in}}%
\pgfpathlineto{\pgfqpoint{5.242825in}{0.695247in}}%
\pgfpathlineto{\pgfqpoint{5.243123in}{0.695246in}}%
\pgfpathlineto{\pgfqpoint{5.243420in}{0.695246in}}%
\pgfpathlineto{\pgfqpoint{5.243718in}{0.695246in}}%
\pgfpathlineto{\pgfqpoint{5.244015in}{0.695245in}}%
\pgfpathlineto{\pgfqpoint{5.244313in}{0.695245in}}%
\pgfpathlineto{\pgfqpoint{5.244610in}{0.695245in}}%
\pgfpathlineto{\pgfqpoint{5.244908in}{0.695244in}}%
\pgfpathlineto{\pgfqpoint{5.245205in}{0.695244in}}%
\pgfpathlineto{\pgfqpoint{5.245503in}{0.695244in}}%
\pgfpathlineto{\pgfqpoint{5.245800in}{0.695243in}}%
\pgfpathlineto{\pgfqpoint{5.246097in}{0.695243in}}%
\pgfpathlineto{\pgfqpoint{5.246395in}{0.695243in}}%
\pgfpathlineto{\pgfqpoint{5.246692in}{0.695242in}}%
\pgfpathlineto{\pgfqpoint{5.246990in}{0.695242in}}%
\pgfpathlineto{\pgfqpoint{5.247287in}{0.695242in}}%
\pgfpathlineto{\pgfqpoint{5.247585in}{0.695241in}}%
\pgfpathlineto{\pgfqpoint{5.247882in}{0.695241in}}%
\pgfpathlineto{\pgfqpoint{5.248180in}{0.695241in}}%
\pgfpathlineto{\pgfqpoint{5.248477in}{0.695240in}}%
\pgfpathlineto{\pgfqpoint{5.248775in}{0.695240in}}%
\pgfpathlineto{\pgfqpoint{5.249072in}{0.695240in}}%
\pgfpathlineto{\pgfqpoint{5.249370in}{0.695239in}}%
\pgfpathlineto{\pgfqpoint{5.249667in}{0.695239in}}%
\pgfpathlineto{\pgfqpoint{5.249965in}{0.695239in}}%
\pgfpathlineto{\pgfqpoint{5.250262in}{0.695239in}}%
\pgfpathlineto{\pgfqpoint{5.250560in}{0.695238in}}%
\pgfpathlineto{\pgfqpoint{5.250857in}{0.695238in}}%
\pgfpathlineto{\pgfqpoint{5.251155in}{0.695238in}}%
\pgfpathlineto{\pgfqpoint{5.251452in}{0.695237in}}%
\pgfpathlineto{\pgfqpoint{5.251750in}{0.695237in}}%
\pgfpathlineto{\pgfqpoint{5.252047in}{0.695237in}}%
\pgfpathlineto{\pgfqpoint{5.252344in}{0.695236in}}%
\pgfpathlineto{\pgfqpoint{5.252642in}{0.695236in}}%
\pgfpathlineto{\pgfqpoint{5.252939in}{0.695236in}}%
\pgfpathlineto{\pgfqpoint{5.253237in}{0.695235in}}%
\pgfpathlineto{\pgfqpoint{5.253534in}{0.695235in}}%
\pgfpathlineto{\pgfqpoint{5.253832in}{0.695235in}}%
\pgfpathlineto{\pgfqpoint{5.254129in}{0.695234in}}%
\pgfpathlineto{\pgfqpoint{5.254427in}{0.695234in}}%
\pgfpathlineto{\pgfqpoint{5.254724in}{0.695234in}}%
\pgfpathlineto{\pgfqpoint{5.255022in}{0.695233in}}%
\pgfpathlineto{\pgfqpoint{5.255319in}{0.695233in}}%
\pgfpathlineto{\pgfqpoint{5.255617in}{0.695233in}}%
\pgfpathlineto{\pgfqpoint{5.255914in}{0.695232in}}%
\pgfpathlineto{\pgfqpoint{5.256212in}{0.695232in}}%
\pgfpathlineto{\pgfqpoint{5.256509in}{0.695232in}}%
\pgfpathlineto{\pgfqpoint{5.256807in}{0.695231in}}%
\pgfpathlineto{\pgfqpoint{5.257104in}{0.695231in}}%
\pgfpathlineto{\pgfqpoint{5.257402in}{0.695231in}}%
\pgfpathlineto{\pgfqpoint{5.257699in}{0.695230in}}%
\pgfpathlineto{\pgfqpoint{5.257997in}{0.695230in}}%
\pgfpathlineto{\pgfqpoint{5.258294in}{0.695230in}}%
\pgfpathlineto{\pgfqpoint{5.258592in}{0.695229in}}%
\pgfpathlineto{\pgfqpoint{5.258889in}{0.695229in}}%
\pgfpathlineto{\pgfqpoint{5.259186in}{0.695229in}}%
\pgfpathlineto{\pgfqpoint{5.259484in}{0.695229in}}%
\pgfpathlineto{\pgfqpoint{5.259781in}{0.695228in}}%
\pgfpathlineto{\pgfqpoint{5.260079in}{0.695228in}}%
\pgfpathlineto{\pgfqpoint{5.260376in}{0.695228in}}%
\pgfpathlineto{\pgfqpoint{5.260674in}{0.695227in}}%
\pgfpathlineto{\pgfqpoint{5.260971in}{0.695227in}}%
\pgfpathlineto{\pgfqpoint{5.261269in}{0.695227in}}%
\pgfpathlineto{\pgfqpoint{5.261566in}{0.695226in}}%
\pgfpathlineto{\pgfqpoint{5.261864in}{0.695226in}}%
\pgfpathlineto{\pgfqpoint{5.262161in}{0.695226in}}%
\pgfpathlineto{\pgfqpoint{5.262459in}{0.695225in}}%
\pgfpathlineto{\pgfqpoint{5.262756in}{0.695225in}}%
\pgfpathlineto{\pgfqpoint{5.263054in}{0.695225in}}%
\pgfpathlineto{\pgfqpoint{5.263351in}{0.695224in}}%
\pgfpathlineto{\pgfqpoint{5.263649in}{0.695224in}}%
\pgfpathlineto{\pgfqpoint{5.263946in}{0.695224in}}%
\pgfpathlineto{\pgfqpoint{5.264244in}{0.695223in}}%
\pgfpathlineto{\pgfqpoint{5.264541in}{0.695223in}}%
\pgfpathlineto{\pgfqpoint{5.264839in}{0.695223in}}%
\pgfpathlineto{\pgfqpoint{5.265136in}{0.695222in}}%
\pgfpathlineto{\pgfqpoint{5.265434in}{0.695222in}}%
\pgfpathlineto{\pgfqpoint{5.265731in}{0.695222in}}%
\pgfpathlineto{\pgfqpoint{5.266028in}{0.695221in}}%
\pgfpathlineto{\pgfqpoint{5.266326in}{0.695221in}}%
\pgfpathlineto{\pgfqpoint{5.266623in}{0.695221in}}%
\pgfpathlineto{\pgfqpoint{5.266921in}{0.695220in}}%
\pgfpathlineto{\pgfqpoint{5.267218in}{0.695220in}}%
\pgfpathlineto{\pgfqpoint{5.267516in}{0.695220in}}%
\pgfpathlineto{\pgfqpoint{5.267813in}{0.695219in}}%
\pgfpathlineto{\pgfqpoint{5.268111in}{0.695219in}}%
\pgfpathlineto{\pgfqpoint{5.268408in}{0.695219in}}%
\pgfpathlineto{\pgfqpoint{5.268706in}{0.695219in}}%
\pgfpathlineto{\pgfqpoint{5.269003in}{0.695218in}}%
\pgfpathlineto{\pgfqpoint{5.269301in}{0.695218in}}%
\pgfpathlineto{\pgfqpoint{5.269598in}{0.695218in}}%
\pgfpathlineto{\pgfqpoint{5.269896in}{0.695217in}}%
\pgfpathlineto{\pgfqpoint{5.270193in}{0.695217in}}%
\pgfpathlineto{\pgfqpoint{5.270491in}{0.695217in}}%
\pgfpathlineto{\pgfqpoint{5.270788in}{0.695216in}}%
\pgfpathlineto{\pgfqpoint{5.271086in}{0.695216in}}%
\pgfpathlineto{\pgfqpoint{5.271383in}{0.695216in}}%
\pgfpathlineto{\pgfqpoint{5.271681in}{0.695215in}}%
\pgfpathlineto{\pgfqpoint{5.271978in}{0.695215in}}%
\pgfpathlineto{\pgfqpoint{5.272276in}{0.695215in}}%
\pgfpathlineto{\pgfqpoint{5.272573in}{0.695214in}}%
\pgfpathlineto{\pgfqpoint{5.272870in}{0.695214in}}%
\pgfpathlineto{\pgfqpoint{5.273168in}{0.695214in}}%
\pgfpathlineto{\pgfqpoint{5.273465in}{0.695213in}}%
\pgfpathlineto{\pgfqpoint{5.273763in}{0.695213in}}%
\pgfpathlineto{\pgfqpoint{5.274060in}{0.695213in}}%
\pgfpathlineto{\pgfqpoint{5.274358in}{0.695212in}}%
\pgfpathlineto{\pgfqpoint{5.274655in}{0.695212in}}%
\pgfpathlineto{\pgfqpoint{5.274953in}{0.695212in}}%
\pgfpathlineto{\pgfqpoint{5.275250in}{0.695211in}}%
\pgfpathlineto{\pgfqpoint{5.275548in}{0.695211in}}%
\pgfpathlineto{\pgfqpoint{5.275845in}{0.695211in}}%
\pgfpathlineto{\pgfqpoint{5.276143in}{0.695210in}}%
\pgfpathlineto{\pgfqpoint{5.276440in}{0.695210in}}%
\pgfpathlineto{\pgfqpoint{5.276738in}{0.695210in}}%
\pgfpathlineto{\pgfqpoint{5.277035in}{0.695209in}}%
\pgfpathlineto{\pgfqpoint{5.277333in}{0.695209in}}%
\pgfpathlineto{\pgfqpoint{5.277630in}{0.695209in}}%
\pgfpathlineto{\pgfqpoint{5.277928in}{0.695209in}}%
\pgfpathlineto{\pgfqpoint{5.278225in}{0.695208in}}%
\pgfpathlineto{\pgfqpoint{5.278523in}{0.695208in}}%
\pgfpathlineto{\pgfqpoint{5.278820in}{0.695208in}}%
\pgfpathlineto{\pgfqpoint{5.279117in}{0.695207in}}%
\pgfpathlineto{\pgfqpoint{5.279415in}{0.695207in}}%
\pgfpathlineto{\pgfqpoint{5.279712in}{0.695207in}}%
\pgfpathlineto{\pgfqpoint{5.280010in}{0.695206in}}%
\pgfpathlineto{\pgfqpoint{5.280307in}{0.695206in}}%
\pgfpathlineto{\pgfqpoint{5.280605in}{0.695206in}}%
\pgfpathlineto{\pgfqpoint{5.280902in}{0.695205in}}%
\pgfpathlineto{\pgfqpoint{5.281200in}{0.695205in}}%
\pgfpathlineto{\pgfqpoint{5.281497in}{0.695205in}}%
\pgfpathlineto{\pgfqpoint{5.281795in}{0.695204in}}%
\pgfpathlineto{\pgfqpoint{5.282092in}{0.695204in}}%
\pgfpathlineto{\pgfqpoint{5.282390in}{0.695204in}}%
\pgfpathlineto{\pgfqpoint{5.282687in}{0.695203in}}%
\pgfpathlineto{\pgfqpoint{5.282985in}{0.695203in}}%
\pgfpathlineto{\pgfqpoint{5.283282in}{0.695203in}}%
\pgfpathlineto{\pgfqpoint{5.283580in}{0.695202in}}%
\pgfpathlineto{\pgfqpoint{5.283877in}{0.695202in}}%
\pgfpathlineto{\pgfqpoint{5.284175in}{0.695202in}}%
\pgfpathlineto{\pgfqpoint{5.284472in}{0.695201in}}%
\pgfpathlineto{\pgfqpoint{5.284770in}{0.695201in}}%
\pgfpathlineto{\pgfqpoint{5.285067in}{0.695201in}}%
\pgfpathlineto{\pgfqpoint{5.285365in}{0.695200in}}%
\pgfpathlineto{\pgfqpoint{5.285662in}{0.695200in}}%
\pgfpathlineto{\pgfqpoint{5.285959in}{0.695200in}}%
\pgfpathlineto{\pgfqpoint{5.286257in}{0.695199in}}%
\pgfpathlineto{\pgfqpoint{5.286554in}{0.695199in}}%
\pgfpathlineto{\pgfqpoint{5.286852in}{0.695199in}}%
\pgfpathlineto{\pgfqpoint{5.287149in}{0.695199in}}%
\pgfpathlineto{\pgfqpoint{5.287447in}{0.695198in}}%
\pgfpathlineto{\pgfqpoint{5.287744in}{0.695198in}}%
\pgfpathlineto{\pgfqpoint{5.288042in}{0.695198in}}%
\pgfpathlineto{\pgfqpoint{5.288339in}{0.695197in}}%
\pgfpathlineto{\pgfqpoint{5.288637in}{0.695197in}}%
\pgfpathlineto{\pgfqpoint{5.288934in}{0.695197in}}%
\pgfpathlineto{\pgfqpoint{5.289232in}{0.695196in}}%
\pgfpathlineto{\pgfqpoint{5.289529in}{0.695196in}}%
\pgfpathlineto{\pgfqpoint{5.289827in}{0.695196in}}%
\pgfpathlineto{\pgfqpoint{5.290124in}{0.695195in}}%
\pgfpathlineto{\pgfqpoint{5.290422in}{0.695195in}}%
\pgfpathlineto{\pgfqpoint{5.290719in}{0.695195in}}%
\pgfpathlineto{\pgfqpoint{5.291017in}{0.695194in}}%
\pgfpathlineto{\pgfqpoint{5.291314in}{0.695194in}}%
\pgfpathlineto{\pgfqpoint{5.291612in}{0.695194in}}%
\pgfpathlineto{\pgfqpoint{5.291909in}{0.695193in}}%
\pgfpathlineto{\pgfqpoint{5.292207in}{0.695193in}}%
\pgfpathlineto{\pgfqpoint{5.292504in}{0.695193in}}%
\pgfpathlineto{\pgfqpoint{5.292801in}{0.695192in}}%
\pgfpathlineto{\pgfqpoint{5.293099in}{0.695192in}}%
\pgfpathlineto{\pgfqpoint{5.293396in}{0.695192in}}%
\pgfpathlineto{\pgfqpoint{5.293694in}{0.695191in}}%
\pgfpathlineto{\pgfqpoint{5.293991in}{0.695191in}}%
\pgfpathlineto{\pgfqpoint{5.294289in}{0.695191in}}%
\pgfpathlineto{\pgfqpoint{5.294586in}{0.695190in}}%
\pgfpathlineto{\pgfqpoint{5.294884in}{0.695190in}}%
\pgfpathlineto{\pgfqpoint{5.295181in}{0.695190in}}%
\pgfpathlineto{\pgfqpoint{5.295479in}{0.695189in}}%
\pgfpathlineto{\pgfqpoint{5.295776in}{0.695189in}}%
\pgfpathlineto{\pgfqpoint{5.296074in}{0.695189in}}%
\pgfpathlineto{\pgfqpoint{5.296371in}{0.695189in}}%
\pgfpathlineto{\pgfqpoint{5.296669in}{0.695188in}}%
\pgfpathlineto{\pgfqpoint{5.296966in}{0.695188in}}%
\pgfpathlineto{\pgfqpoint{5.297264in}{0.695188in}}%
\pgfpathlineto{\pgfqpoint{5.297561in}{0.695187in}}%
\pgfpathlineto{\pgfqpoint{5.297859in}{0.695187in}}%
\pgfpathlineto{\pgfqpoint{5.298156in}{0.695187in}}%
\pgfpathlineto{\pgfqpoint{5.298454in}{0.695186in}}%
\pgfpathlineto{\pgfqpoint{5.298751in}{0.695186in}}%
\pgfpathlineto{\pgfqpoint{5.299048in}{0.695186in}}%
\pgfpathlineto{\pgfqpoint{5.299346in}{0.695185in}}%
\pgfpathlineto{\pgfqpoint{5.299643in}{0.695185in}}%
\pgfpathlineto{\pgfqpoint{5.299941in}{0.695185in}}%
\pgfpathlineto{\pgfqpoint{5.300238in}{0.695184in}}%
\pgfpathlineto{\pgfqpoint{5.300536in}{0.695184in}}%
\pgfpathlineto{\pgfqpoint{5.300833in}{0.695184in}}%
\pgfpathlineto{\pgfqpoint{5.301131in}{0.695183in}}%
\pgfpathlineto{\pgfqpoint{5.301428in}{0.695183in}}%
\pgfpathlineto{\pgfqpoint{5.301726in}{0.695183in}}%
\pgfpathlineto{\pgfqpoint{5.302023in}{0.695182in}}%
\pgfpathlineto{\pgfqpoint{5.302321in}{0.695182in}}%
\pgfpathlineto{\pgfqpoint{5.302618in}{0.695182in}}%
\pgfpathlineto{\pgfqpoint{5.302916in}{0.695181in}}%
\pgfpathlineto{\pgfqpoint{5.303213in}{0.695181in}}%
\pgfpathlineto{\pgfqpoint{5.303511in}{0.695181in}}%
\pgfpathlineto{\pgfqpoint{5.303808in}{0.695180in}}%
\pgfpathlineto{\pgfqpoint{5.304106in}{0.695180in}}%
\pgfpathlineto{\pgfqpoint{5.304403in}{0.695180in}}%
\pgfpathlineto{\pgfqpoint{5.304701in}{0.695179in}}%
\pgfpathlineto{\pgfqpoint{5.304998in}{0.695179in}}%
\pgfpathlineto{\pgfqpoint{5.305296in}{0.695179in}}%
\pgfpathlineto{\pgfqpoint{5.305593in}{0.695179in}}%
\pgfpathlineto{\pgfqpoint{5.305890in}{0.695178in}}%
\pgfpathlineto{\pgfqpoint{5.306188in}{0.695178in}}%
\pgfpathlineto{\pgfqpoint{5.306485in}{0.695178in}}%
\pgfpathlineto{\pgfqpoint{5.306783in}{0.695177in}}%
\pgfpathlineto{\pgfqpoint{5.307080in}{0.695177in}}%
\pgfpathlineto{\pgfqpoint{5.307378in}{0.695177in}}%
\pgfpathlineto{\pgfqpoint{5.307675in}{0.695176in}}%
\pgfpathlineto{\pgfqpoint{5.307973in}{0.695176in}}%
\pgfpathlineto{\pgfqpoint{5.308270in}{0.695176in}}%
\pgfpathlineto{\pgfqpoint{5.308568in}{0.695175in}}%
\pgfpathlineto{\pgfqpoint{5.308865in}{0.695175in}}%
\pgfpathlineto{\pgfqpoint{5.309163in}{0.695175in}}%
\pgfpathlineto{\pgfqpoint{5.309460in}{0.695174in}}%
\pgfpathlineto{\pgfqpoint{5.309758in}{0.695174in}}%
\pgfpathlineto{\pgfqpoint{5.310055in}{0.695174in}}%
\pgfpathlineto{\pgfqpoint{5.310353in}{0.695173in}}%
\pgfpathlineto{\pgfqpoint{5.310650in}{0.695173in}}%
\pgfpathlineto{\pgfqpoint{5.310948in}{0.695173in}}%
\pgfpathlineto{\pgfqpoint{5.311245in}{0.695172in}}%
\pgfpathlineto{\pgfqpoint{5.311543in}{0.695172in}}%
\pgfpathlineto{\pgfqpoint{5.311840in}{0.695172in}}%
\pgfpathlineto{\pgfqpoint{5.312138in}{0.695171in}}%
\pgfpathlineto{\pgfqpoint{5.312435in}{0.695171in}}%
\pgfpathlineto{\pgfqpoint{5.312732in}{0.695171in}}%
\pgfpathlineto{\pgfqpoint{5.313030in}{0.695170in}}%
\pgfpathlineto{\pgfqpoint{5.313327in}{0.695170in}}%
\pgfpathlineto{\pgfqpoint{5.313625in}{0.695170in}}%
\pgfpathlineto{\pgfqpoint{5.313922in}{0.695169in}}%
\pgfpathlineto{\pgfqpoint{5.314220in}{0.695169in}}%
\pgfpathlineto{\pgfqpoint{5.314517in}{0.695169in}}%
\pgfpathlineto{\pgfqpoint{5.314815in}{0.695169in}}%
\pgfpathlineto{\pgfqpoint{5.315112in}{0.695168in}}%
\pgfpathlineto{\pgfqpoint{5.315410in}{0.695168in}}%
\pgfpathlineto{\pgfqpoint{5.315707in}{0.695168in}}%
\pgfpathlineto{\pgfqpoint{5.316005in}{0.695167in}}%
\pgfpathlineto{\pgfqpoint{5.316302in}{0.695167in}}%
\pgfpathlineto{\pgfqpoint{5.316600in}{0.695167in}}%
\pgfpathlineto{\pgfqpoint{5.316897in}{0.695166in}}%
\pgfpathlineto{\pgfqpoint{5.317195in}{0.695166in}}%
\pgfpathlineto{\pgfqpoint{5.317492in}{0.695166in}}%
\pgfpathlineto{\pgfqpoint{5.317790in}{0.695165in}}%
\pgfpathlineto{\pgfqpoint{5.318087in}{0.695165in}}%
\pgfpathlineto{\pgfqpoint{5.318385in}{0.695165in}}%
\pgfpathlineto{\pgfqpoint{5.318682in}{0.695164in}}%
\pgfpathlineto{\pgfqpoint{5.318979in}{0.695164in}}%
\pgfpathlineto{\pgfqpoint{5.319277in}{0.695164in}}%
\pgfpathlineto{\pgfqpoint{5.319574in}{0.695163in}}%
\pgfpathlineto{\pgfqpoint{5.319872in}{0.695163in}}%
\pgfpathlineto{\pgfqpoint{5.320169in}{0.695163in}}%
\pgfpathlineto{\pgfqpoint{5.320467in}{0.695162in}}%
\pgfpathlineto{\pgfqpoint{5.320764in}{0.695162in}}%
\pgfpathlineto{\pgfqpoint{5.321062in}{0.695162in}}%
\pgfpathlineto{\pgfqpoint{5.321359in}{0.695161in}}%
\pgfpathlineto{\pgfqpoint{5.321657in}{0.695161in}}%
\pgfpathlineto{\pgfqpoint{5.321954in}{0.695161in}}%
\pgfpathlineto{\pgfqpoint{5.322252in}{0.695160in}}%
\pgfpathlineto{\pgfqpoint{5.322549in}{0.695160in}}%
\pgfpathlineto{\pgfqpoint{5.322847in}{0.695160in}}%
\pgfpathlineto{\pgfqpoint{5.323144in}{0.695159in}}%
\pgfpathlineto{\pgfqpoint{5.323442in}{0.695159in}}%
\pgfpathlineto{\pgfqpoint{5.323739in}{0.695159in}}%
\pgfpathlineto{\pgfqpoint{5.324037in}{0.695159in}}%
\pgfpathlineto{\pgfqpoint{5.324334in}{0.695158in}}%
\pgfpathlineto{\pgfqpoint{5.324632in}{0.695158in}}%
\pgfpathlineto{\pgfqpoint{5.324929in}{0.695158in}}%
\pgfpathlineto{\pgfqpoint{5.325227in}{0.695157in}}%
\pgfpathlineto{\pgfqpoint{5.325524in}{0.695157in}}%
\pgfpathlineto{\pgfqpoint{5.325821in}{0.695157in}}%
\pgfpathlineto{\pgfqpoint{5.326119in}{0.695156in}}%
\pgfpathlineto{\pgfqpoint{5.326416in}{0.695156in}}%
\pgfpathlineto{\pgfqpoint{5.326714in}{0.695156in}}%
\pgfpathlineto{\pgfqpoint{5.327011in}{0.695155in}}%
\pgfpathlineto{\pgfqpoint{5.327309in}{0.695155in}}%
\pgfpathlineto{\pgfqpoint{5.327606in}{0.695155in}}%
\pgfpathlineto{\pgfqpoint{5.327904in}{0.695154in}}%
\pgfpathlineto{\pgfqpoint{5.328201in}{0.695154in}}%
\pgfpathlineto{\pgfqpoint{5.328499in}{0.695154in}}%
\pgfpathlineto{\pgfqpoint{5.328796in}{0.695153in}}%
\pgfpathlineto{\pgfqpoint{5.329094in}{0.695153in}}%
\pgfpathlineto{\pgfqpoint{5.329391in}{0.695153in}}%
\pgfpathlineto{\pgfqpoint{5.329689in}{0.695152in}}%
\pgfpathlineto{\pgfqpoint{5.329986in}{0.695152in}}%
\pgfpathlineto{\pgfqpoint{5.330284in}{0.695152in}}%
\pgfpathlineto{\pgfqpoint{5.330581in}{0.695151in}}%
\pgfpathlineto{\pgfqpoint{5.330879in}{0.695151in}}%
\pgfpathlineto{\pgfqpoint{5.331176in}{0.695151in}}%
\pgfpathlineto{\pgfqpoint{5.331474in}{0.695150in}}%
\pgfpathlineto{\pgfqpoint{5.331771in}{0.695150in}}%
\pgfpathlineto{\pgfqpoint{5.332069in}{0.695150in}}%
\pgfpathlineto{\pgfqpoint{5.332366in}{0.695149in}}%
\pgfpathlineto{\pgfqpoint{5.332663in}{0.695149in}}%
\pgfpathlineto{\pgfqpoint{5.332961in}{0.695149in}}%
\pgfpathlineto{\pgfqpoint{5.333258in}{0.695149in}}%
\pgfpathlineto{\pgfqpoint{5.333556in}{0.695148in}}%
\pgfpathlineto{\pgfqpoint{5.333853in}{0.695148in}}%
\pgfpathlineto{\pgfqpoint{5.334151in}{0.695148in}}%
\pgfpathlineto{\pgfqpoint{5.334448in}{0.695147in}}%
\pgfpathlineto{\pgfqpoint{5.334746in}{0.695147in}}%
\pgfpathlineto{\pgfqpoint{5.335043in}{0.695147in}}%
\pgfpathlineto{\pgfqpoint{5.335341in}{0.695146in}}%
\pgfpathlineto{\pgfqpoint{5.335638in}{0.695146in}}%
\pgfpathlineto{\pgfqpoint{5.335936in}{0.695146in}}%
\pgfpathlineto{\pgfqpoint{5.336233in}{0.695145in}}%
\pgfpathlineto{\pgfqpoint{5.336531in}{0.695145in}}%
\pgfpathlineto{\pgfqpoint{5.336828in}{0.695145in}}%
\pgfpathlineto{\pgfqpoint{5.337126in}{0.695144in}}%
\pgfpathlineto{\pgfqpoint{5.337423in}{0.695144in}}%
\pgfpathlineto{\pgfqpoint{5.337721in}{0.695144in}}%
\pgfpathlineto{\pgfqpoint{5.338018in}{0.695143in}}%
\pgfpathlineto{\pgfqpoint{5.338316in}{0.695143in}}%
\pgfpathlineto{\pgfqpoint{5.338613in}{0.695143in}}%
\pgfpathlineto{\pgfqpoint{5.338910in}{0.695142in}}%
\pgfpathlineto{\pgfqpoint{5.339208in}{0.695142in}}%
\pgfpathlineto{\pgfqpoint{5.339505in}{0.695142in}}%
\pgfpathlineto{\pgfqpoint{5.339803in}{0.695141in}}%
\pgfpathlineto{\pgfqpoint{5.340100in}{0.695141in}}%
\pgfpathlineto{\pgfqpoint{5.340398in}{0.695141in}}%
\pgfpathlineto{\pgfqpoint{5.340695in}{0.695140in}}%
\pgfpathlineto{\pgfqpoint{5.340993in}{0.695140in}}%
\pgfpathlineto{\pgfqpoint{5.341290in}{0.695140in}}%
\pgfpathlineto{\pgfqpoint{5.341588in}{0.695139in}}%
\pgfpathlineto{\pgfqpoint{5.341885in}{0.695139in}}%
\pgfpathlineto{\pgfqpoint{5.342183in}{0.695139in}}%
\pgfpathlineto{\pgfqpoint{5.342480in}{0.695139in}}%
\pgfpathlineto{\pgfqpoint{5.342778in}{0.695138in}}%
\pgfpathlineto{\pgfqpoint{5.343075in}{0.695138in}}%
\pgfpathlineto{\pgfqpoint{5.343373in}{0.695138in}}%
\pgfpathlineto{\pgfqpoint{5.343670in}{0.695137in}}%
\pgfpathlineto{\pgfqpoint{5.343968in}{0.695137in}}%
\pgfpathlineto{\pgfqpoint{5.344265in}{0.695137in}}%
\pgfpathlineto{\pgfqpoint{5.344563in}{0.695136in}}%
\pgfpathlineto{\pgfqpoint{5.344860in}{0.695136in}}%
\pgfpathlineto{\pgfqpoint{5.345158in}{0.695136in}}%
\pgfpathlineto{\pgfqpoint{5.345455in}{0.695135in}}%
\pgfpathlineto{\pgfqpoint{5.345752in}{0.695135in}}%
\pgfpathlineto{\pgfqpoint{5.346050in}{0.695135in}}%
\pgfpathlineto{\pgfqpoint{5.346347in}{0.695134in}}%
\pgfpathlineto{\pgfqpoint{5.346645in}{0.695134in}}%
\pgfpathlineto{\pgfqpoint{5.346942in}{0.695134in}}%
\pgfpathlineto{\pgfqpoint{5.347240in}{0.695133in}}%
\pgfpathlineto{\pgfqpoint{5.347537in}{0.695133in}}%
\pgfpathlineto{\pgfqpoint{5.347835in}{0.695133in}}%
\pgfpathlineto{\pgfqpoint{5.348132in}{0.695132in}}%
\pgfpathlineto{\pgfqpoint{5.348430in}{0.695132in}}%
\pgfpathlineto{\pgfqpoint{5.348727in}{0.695132in}}%
\pgfpathlineto{\pgfqpoint{5.349025in}{0.695131in}}%
\pgfpathlineto{\pgfqpoint{5.349322in}{0.695131in}}%
\pgfpathlineto{\pgfqpoint{5.349620in}{0.695131in}}%
\pgfpathlineto{\pgfqpoint{5.349917in}{0.695130in}}%
\pgfpathlineto{\pgfqpoint{5.350215in}{0.695130in}}%
\pgfpathlineto{\pgfqpoint{5.350512in}{0.695130in}}%
\pgfpathlineto{\pgfqpoint{5.350810in}{0.695129in}}%
\pgfpathlineto{\pgfqpoint{5.351107in}{0.695129in}}%
\pgfpathlineto{\pgfqpoint{5.351405in}{0.695129in}}%
\pgfpathlineto{\pgfqpoint{5.351702in}{0.695129in}}%
\pgfpathlineto{\pgfqpoint{5.352000in}{0.695128in}}%
\pgfpathlineto{\pgfqpoint{5.352297in}{0.695128in}}%
\pgfpathlineto{\pgfqpoint{5.352594in}{0.695128in}}%
\pgfpathlineto{\pgfqpoint{5.352892in}{0.695127in}}%
\pgfpathlineto{\pgfqpoint{5.353189in}{0.695127in}}%
\pgfpathlineto{\pgfqpoint{5.353487in}{0.695127in}}%
\pgfpathlineto{\pgfqpoint{5.353784in}{0.695126in}}%
\pgfpathlineto{\pgfqpoint{5.354082in}{0.695126in}}%
\pgfpathlineto{\pgfqpoint{5.354379in}{0.695126in}}%
\pgfpathlineto{\pgfqpoint{5.354677in}{0.695125in}}%
\pgfpathlineto{\pgfqpoint{5.354974in}{0.695125in}}%
\pgfpathlineto{\pgfqpoint{5.355272in}{0.695125in}}%
\pgfpathlineto{\pgfqpoint{5.355569in}{0.695124in}}%
\pgfpathlineto{\pgfqpoint{5.355867in}{0.695124in}}%
\pgfpathlineto{\pgfqpoint{5.356164in}{0.695124in}}%
\pgfpathlineto{\pgfqpoint{5.356462in}{0.695123in}}%
\pgfpathlineto{\pgfqpoint{5.356759in}{0.695123in}}%
\pgfpathlineto{\pgfqpoint{5.357057in}{0.695123in}}%
\pgfpathlineto{\pgfqpoint{5.357354in}{0.695122in}}%
\pgfpathlineto{\pgfqpoint{5.357652in}{0.695122in}}%
\pgfpathlineto{\pgfqpoint{5.357949in}{0.695122in}}%
\pgfpathlineto{\pgfqpoint{5.358247in}{0.695121in}}%
\pgfpathlineto{\pgfqpoint{5.358544in}{0.695121in}}%
\pgfpathlineto{\pgfqpoint{5.358841in}{0.695121in}}%
\pgfpathlineto{\pgfqpoint{5.359139in}{0.695120in}}%
\pgfpathlineto{\pgfqpoint{5.359436in}{0.695120in}}%
\pgfpathlineto{\pgfqpoint{5.359734in}{0.695120in}}%
\pgfpathlineto{\pgfqpoint{5.360031in}{0.695119in}}%
\pgfpathlineto{\pgfqpoint{5.360329in}{0.695119in}}%
\pgfpathlineto{\pgfqpoint{5.360626in}{0.695119in}}%
\pgfpathlineto{\pgfqpoint{5.360924in}{0.695119in}}%
\pgfpathlineto{\pgfqpoint{5.361221in}{0.695118in}}%
\pgfpathlineto{\pgfqpoint{5.361519in}{0.695118in}}%
\pgfpathlineto{\pgfqpoint{5.361816in}{0.695118in}}%
\pgfpathlineto{\pgfqpoint{5.362114in}{0.695117in}}%
\pgfpathlineto{\pgfqpoint{5.362411in}{0.695117in}}%
\pgfpathlineto{\pgfqpoint{5.362709in}{0.695117in}}%
\pgfpathlineto{\pgfqpoint{5.363006in}{0.695116in}}%
\pgfpathlineto{\pgfqpoint{5.363304in}{0.695116in}}%
\pgfpathlineto{\pgfqpoint{5.363601in}{0.695116in}}%
\pgfpathlineto{\pgfqpoint{5.363899in}{0.695115in}}%
\pgfpathlineto{\pgfqpoint{5.364196in}{0.695115in}}%
\pgfpathlineto{\pgfqpoint{5.364494in}{0.695115in}}%
\pgfpathlineto{\pgfqpoint{5.364791in}{0.695114in}}%
\pgfpathlineto{\pgfqpoint{5.365089in}{0.695114in}}%
\pgfpathlineto{\pgfqpoint{5.365386in}{0.695114in}}%
\pgfpathlineto{\pgfqpoint{5.365683in}{0.695113in}}%
\pgfpathlineto{\pgfqpoint{5.365981in}{0.695113in}}%
\pgfpathlineto{\pgfqpoint{5.366278in}{0.695113in}}%
\pgfpathlineto{\pgfqpoint{5.366576in}{0.695112in}}%
\pgfpathlineto{\pgfqpoint{5.366873in}{0.695112in}}%
\pgfpathlineto{\pgfqpoint{5.367171in}{0.695112in}}%
\pgfpathlineto{\pgfqpoint{5.367468in}{0.695111in}}%
\pgfpathlineto{\pgfqpoint{5.367766in}{0.695111in}}%
\pgfpathlineto{\pgfqpoint{5.368063in}{0.695111in}}%
\pgfpathlineto{\pgfqpoint{5.368361in}{0.695110in}}%
\pgfpathlineto{\pgfqpoint{5.368658in}{0.695110in}}%
\pgfpathlineto{\pgfqpoint{5.368956in}{0.695110in}}%
\pgfpathlineto{\pgfqpoint{5.369253in}{0.695109in}}%
\pgfpathlineto{\pgfqpoint{5.369551in}{0.695109in}}%
\pgfpathlineto{\pgfqpoint{5.369848in}{0.695109in}}%
\pgfpathlineto{\pgfqpoint{5.370146in}{0.695109in}}%
\pgfpathlineto{\pgfqpoint{5.370443in}{0.695108in}}%
\pgfpathlineto{\pgfqpoint{5.370741in}{0.695108in}}%
\pgfpathlineto{\pgfqpoint{5.371038in}{0.695108in}}%
\pgfpathlineto{\pgfqpoint{5.371336in}{0.695107in}}%
\pgfpathlineto{\pgfqpoint{5.371633in}{0.695107in}}%
\pgfpathlineto{\pgfqpoint{5.371931in}{0.695107in}}%
\pgfpathlineto{\pgfqpoint{5.372228in}{0.695106in}}%
\pgfpathlineto{\pgfqpoint{5.372525in}{0.695106in}}%
\pgfpathlineto{\pgfqpoint{5.372823in}{0.695106in}}%
\pgfpathlineto{\pgfqpoint{5.373120in}{0.695105in}}%
\pgfpathlineto{\pgfqpoint{5.373418in}{0.695105in}}%
\pgfpathlineto{\pgfqpoint{5.373715in}{0.695105in}}%
\pgfpathlineto{\pgfqpoint{5.374013in}{0.695104in}}%
\pgfpathlineto{\pgfqpoint{5.374310in}{0.695104in}}%
\pgfpathlineto{\pgfqpoint{5.374608in}{0.695104in}}%
\pgfpathlineto{\pgfqpoint{5.374905in}{0.695103in}}%
\pgfpathlineto{\pgfqpoint{5.375203in}{0.695103in}}%
\pgfpathlineto{\pgfqpoint{5.375500in}{0.695103in}}%
\pgfpathlineto{\pgfqpoint{5.375798in}{0.695102in}}%
\pgfpathlineto{\pgfqpoint{5.376095in}{0.695102in}}%
\pgfpathlineto{\pgfqpoint{5.376393in}{0.695102in}}%
\pgfpathlineto{\pgfqpoint{5.376690in}{0.695101in}}%
\pgfpathlineto{\pgfqpoint{5.376988in}{0.695101in}}%
\pgfpathlineto{\pgfqpoint{5.377285in}{0.695101in}}%
\pgfpathlineto{\pgfqpoint{5.377583in}{0.695100in}}%
\pgfpathlineto{\pgfqpoint{5.377880in}{0.695100in}}%
\pgfpathlineto{\pgfqpoint{5.378178in}{0.695100in}}%
\pgfpathlineto{\pgfqpoint{5.378475in}{0.695099in}}%
\pgfpathlineto{\pgfqpoint{5.378772in}{0.695099in}}%
\pgfpathlineto{\pgfqpoint{5.379070in}{0.695099in}}%
\pgfpathlineto{\pgfqpoint{5.379367in}{0.695099in}}%
\pgfpathlineto{\pgfqpoint{5.379665in}{0.695098in}}%
\pgfpathlineto{\pgfqpoint{5.379962in}{0.695098in}}%
\pgfpathlineto{\pgfqpoint{5.380260in}{0.695098in}}%
\pgfpathlineto{\pgfqpoint{5.380557in}{0.695097in}}%
\pgfpathlineto{\pgfqpoint{5.380855in}{0.695097in}}%
\pgfpathlineto{\pgfqpoint{5.381152in}{0.695097in}}%
\pgfpathlineto{\pgfqpoint{5.381450in}{0.695096in}}%
\pgfpathlineto{\pgfqpoint{5.381747in}{0.695096in}}%
\pgfpathlineto{\pgfqpoint{5.382045in}{0.695096in}}%
\pgfpathlineto{\pgfqpoint{5.382342in}{0.695095in}}%
\pgfpathlineto{\pgfqpoint{5.382640in}{0.695095in}}%
\pgfpathlineto{\pgfqpoint{5.382937in}{0.695095in}}%
\pgfpathlineto{\pgfqpoint{5.383235in}{0.695094in}}%
\pgfpathlineto{\pgfqpoint{5.383532in}{0.695094in}}%
\pgfpathlineto{\pgfqpoint{5.383830in}{0.695094in}}%
\pgfpathlineto{\pgfqpoint{5.384127in}{0.695093in}}%
\pgfpathlineto{\pgfqpoint{5.384425in}{0.695093in}}%
\pgfpathlineto{\pgfqpoint{5.384722in}{0.695093in}}%
\pgfpathlineto{\pgfqpoint{5.385020in}{0.695092in}}%
\pgfpathlineto{\pgfqpoint{5.385317in}{0.695092in}}%
\pgfpathlineto{\pgfqpoint{5.385614in}{0.695092in}}%
\pgfpathlineto{\pgfqpoint{5.385912in}{0.695091in}}%
\pgfpathlineto{\pgfqpoint{5.386209in}{0.695091in}}%
\pgfpathlineto{\pgfqpoint{5.386507in}{0.695091in}}%
\pgfpathlineto{\pgfqpoint{5.386804in}{0.695090in}}%
\pgfpathlineto{\pgfqpoint{5.387102in}{0.695090in}}%
\pgfpathlineto{\pgfqpoint{5.387399in}{0.695090in}}%
\pgfpathlineto{\pgfqpoint{5.387697in}{0.695089in}}%
\pgfpathlineto{\pgfqpoint{5.387994in}{0.695089in}}%
\pgfpathlineto{\pgfqpoint{5.388292in}{0.695089in}}%
\pgfpathlineto{\pgfqpoint{5.388589in}{0.695089in}}%
\pgfpathlineto{\pgfqpoint{5.388887in}{0.695088in}}%
\pgfpathlineto{\pgfqpoint{5.389184in}{0.695088in}}%
\pgfpathlineto{\pgfqpoint{5.389482in}{0.695088in}}%
\pgfpathlineto{\pgfqpoint{5.389779in}{0.695087in}}%
\pgfpathlineto{\pgfqpoint{5.390077in}{0.695087in}}%
\pgfpathlineto{\pgfqpoint{5.390374in}{0.695087in}}%
\pgfpathlineto{\pgfqpoint{5.390672in}{0.695086in}}%
\pgfpathlineto{\pgfqpoint{5.390969in}{0.695086in}}%
\pgfpathlineto{\pgfqpoint{5.391267in}{0.695086in}}%
\pgfpathlineto{\pgfqpoint{5.391564in}{0.695085in}}%
\pgfpathlineto{\pgfqpoint{5.391862in}{0.695085in}}%
\pgfpathlineto{\pgfqpoint{5.392159in}{0.695085in}}%
\pgfpathlineto{\pgfqpoint{5.392456in}{0.695084in}}%
\pgfpathlineto{\pgfqpoint{5.392754in}{0.695084in}}%
\pgfpathlineto{\pgfqpoint{5.393051in}{0.695084in}}%
\pgfpathlineto{\pgfqpoint{5.393349in}{0.695083in}}%
\pgfpathlineto{\pgfqpoint{5.393646in}{0.695083in}}%
\pgfpathlineto{\pgfqpoint{5.393944in}{0.695083in}}%
\pgfpathlineto{\pgfqpoint{5.394241in}{0.695082in}}%
\pgfpathlineto{\pgfqpoint{5.394539in}{0.695082in}}%
\pgfpathlineto{\pgfqpoint{5.394836in}{0.695082in}}%
\pgfpathlineto{\pgfqpoint{5.395134in}{0.695081in}}%
\pgfpathlineto{\pgfqpoint{5.395431in}{0.695081in}}%
\pgfpathlineto{\pgfqpoint{5.395729in}{0.695081in}}%
\pgfpathlineto{\pgfqpoint{5.396026in}{0.695080in}}%
\pgfpathlineto{\pgfqpoint{5.396324in}{0.695080in}}%
\pgfpathlineto{\pgfqpoint{5.396621in}{0.695080in}}%
\pgfpathlineto{\pgfqpoint{5.396919in}{0.695079in}}%
\pgfpathlineto{\pgfqpoint{5.397216in}{0.695079in}}%
\pgfpathlineto{\pgfqpoint{5.397514in}{0.695079in}}%
\pgfpathlineto{\pgfqpoint{5.397811in}{0.695079in}}%
\pgfpathlineto{\pgfqpoint{5.398109in}{0.695079in}}%
\pgfpathlineto{\pgfqpoint{5.398406in}{0.695079in}}%
\pgfpathlineto{\pgfqpoint{5.398703in}{0.695079in}}%
\pgfpathlineto{\pgfqpoint{5.399001in}{0.695079in}}%
\pgfpathlineto{\pgfqpoint{5.399298in}{0.695080in}}%
\pgfpathlineto{\pgfqpoint{5.399596in}{0.695080in}}%
\pgfpathlineto{\pgfqpoint{5.399893in}{0.695080in}}%
\pgfpathlineto{\pgfqpoint{5.400191in}{0.695080in}}%
\pgfpathlineto{\pgfqpoint{5.400488in}{0.695080in}}%
\pgfpathlineto{\pgfqpoint{5.400786in}{0.695080in}}%
\pgfpathlineto{\pgfqpoint{5.401083in}{0.695080in}}%
\pgfpathlineto{\pgfqpoint{5.401381in}{0.695081in}}%
\pgfpathlineto{\pgfqpoint{5.401678in}{0.695081in}}%
\pgfpathlineto{\pgfqpoint{5.401976in}{0.695081in}}%
\pgfpathlineto{\pgfqpoint{5.402273in}{0.695081in}}%
\pgfpathlineto{\pgfqpoint{5.402571in}{0.695081in}}%
\pgfpathlineto{\pgfqpoint{5.402868in}{0.695081in}}%
\pgfpathlineto{\pgfqpoint{5.403166in}{0.695081in}}%
\pgfpathlineto{\pgfqpoint{5.403463in}{0.695081in}}%
\pgfpathlineto{\pgfqpoint{5.403761in}{0.695082in}}%
\pgfpathlineto{\pgfqpoint{5.404058in}{0.695082in}}%
\pgfpathlineto{\pgfqpoint{5.404356in}{0.695082in}}%
\pgfpathlineto{\pgfqpoint{5.404653in}{0.695082in}}%
\pgfpathlineto{\pgfqpoint{5.404951in}{0.695082in}}%
\pgfpathlineto{\pgfqpoint{5.405248in}{0.695082in}}%
\pgfpathlineto{\pgfqpoint{5.405545in}{0.695082in}}%
\pgfpathlineto{\pgfqpoint{5.405843in}{0.695083in}}%
\pgfpathlineto{\pgfqpoint{5.406140in}{0.695083in}}%
\pgfpathlineto{\pgfqpoint{5.406438in}{0.695083in}}%
\pgfpathlineto{\pgfqpoint{5.406735in}{0.695083in}}%
\pgfpathlineto{\pgfqpoint{5.407033in}{0.695083in}}%
\pgfpathlineto{\pgfqpoint{5.407330in}{0.695083in}}%
\pgfpathlineto{\pgfqpoint{5.407628in}{0.695083in}}%
\pgfpathlineto{\pgfqpoint{5.407925in}{0.695083in}}%
\pgfpathlineto{\pgfqpoint{5.408223in}{0.695084in}}%
\pgfpathlineto{\pgfqpoint{5.408520in}{0.695084in}}%
\pgfpathlineto{\pgfqpoint{5.408818in}{0.695084in}}%
\pgfpathlineto{\pgfqpoint{5.409115in}{0.695084in}}%
\pgfpathlineto{\pgfqpoint{5.409413in}{0.695084in}}%
\pgfpathlineto{\pgfqpoint{5.409710in}{0.695084in}}%
\pgfpathlineto{\pgfqpoint{5.410008in}{0.695084in}}%
\pgfpathlineto{\pgfqpoint{5.410305in}{0.695084in}}%
\pgfpathlineto{\pgfqpoint{5.410603in}{0.695085in}}%
\pgfpathlineto{\pgfqpoint{5.410900in}{0.695085in}}%
\pgfpathlineto{\pgfqpoint{5.411198in}{0.695085in}}%
\pgfpathlineto{\pgfqpoint{5.411495in}{0.695085in}}%
\pgfpathlineto{\pgfqpoint{5.411793in}{0.695085in}}%
\pgfpathlineto{\pgfqpoint{5.412090in}{0.695085in}}%
\pgfpathlineto{\pgfqpoint{5.412387in}{0.695085in}}%
\pgfpathlineto{\pgfqpoint{5.412685in}{0.695086in}}%
\pgfpathlineto{\pgfqpoint{5.412982in}{0.695086in}}%
\pgfpathlineto{\pgfqpoint{5.413280in}{0.695086in}}%
\pgfpathlineto{\pgfqpoint{5.413577in}{0.695086in}}%
\pgfpathlineto{\pgfqpoint{5.413875in}{0.695086in}}%
\pgfpathlineto{\pgfqpoint{5.414172in}{0.695086in}}%
\pgfpathlineto{\pgfqpoint{5.414470in}{0.695086in}}%
\pgfpathlineto{\pgfqpoint{5.414767in}{0.695086in}}%
\pgfpathlineto{\pgfqpoint{5.415065in}{0.695087in}}%
\pgfpathlineto{\pgfqpoint{5.415362in}{0.695087in}}%
\pgfpathlineto{\pgfqpoint{5.415660in}{0.695087in}}%
\pgfpathlineto{\pgfqpoint{5.415957in}{0.695087in}}%
\pgfpathlineto{\pgfqpoint{5.416255in}{0.695087in}}%
\pgfpathlineto{\pgfqpoint{5.416552in}{0.695087in}}%
\pgfpathlineto{\pgfqpoint{5.416850in}{0.695087in}}%
\pgfpathlineto{\pgfqpoint{5.417147in}{0.695088in}}%
\pgfpathlineto{\pgfqpoint{5.417445in}{0.695088in}}%
\pgfpathlineto{\pgfqpoint{5.417742in}{0.695088in}}%
\pgfpathlineto{\pgfqpoint{5.418040in}{0.695088in}}%
\pgfpathlineto{\pgfqpoint{5.418337in}{0.695088in}}%
\pgfpathlineto{\pgfqpoint{5.418634in}{0.695088in}}%
\pgfpathlineto{\pgfqpoint{5.418932in}{0.695088in}}%
\pgfpathlineto{\pgfqpoint{5.419229in}{0.695088in}}%
\pgfpathlineto{\pgfqpoint{5.419527in}{0.695089in}}%
\pgfpathlineto{\pgfqpoint{5.419824in}{0.695089in}}%
\pgfpathlineto{\pgfqpoint{5.420122in}{0.695089in}}%
\pgfpathlineto{\pgfqpoint{5.420419in}{0.695089in}}%
\pgfpathlineto{\pgfqpoint{5.420717in}{0.695089in}}%
\pgfpathlineto{\pgfqpoint{5.421014in}{0.695089in}}%
\pgfpathlineto{\pgfqpoint{5.421312in}{0.695089in}}%
\pgfpathlineto{\pgfqpoint{5.421609in}{0.695090in}}%
\pgfpathlineto{\pgfqpoint{5.421907in}{0.695090in}}%
\pgfpathlineto{\pgfqpoint{5.422204in}{0.695090in}}%
\pgfpathlineto{\pgfqpoint{5.422502in}{0.695090in}}%
\pgfpathlineto{\pgfqpoint{5.422799in}{0.695090in}}%
\pgfpathlineto{\pgfqpoint{5.423097in}{0.695090in}}%
\pgfpathlineto{\pgfqpoint{5.423394in}{0.695090in}}%
\pgfpathlineto{\pgfqpoint{5.423692in}{0.695090in}}%
\pgfpathlineto{\pgfqpoint{5.423989in}{0.695091in}}%
\pgfpathlineto{\pgfqpoint{5.424287in}{0.695091in}}%
\pgfpathlineto{\pgfqpoint{5.424584in}{0.695091in}}%
\pgfpathlineto{\pgfqpoint{5.424882in}{0.695091in}}%
\pgfpathlineto{\pgfqpoint{5.425179in}{0.695091in}}%
\pgfpathlineto{\pgfqpoint{5.425476in}{0.695091in}}%
\pgfpathlineto{\pgfqpoint{5.425774in}{0.695091in}}%
\pgfpathlineto{\pgfqpoint{5.426071in}{0.695092in}}%
\pgfpathlineto{\pgfqpoint{5.426369in}{0.695092in}}%
\pgfpathlineto{\pgfqpoint{5.426666in}{0.695092in}}%
\pgfpathlineto{\pgfqpoint{5.426964in}{0.695092in}}%
\pgfpathlineto{\pgfqpoint{5.427261in}{0.695092in}}%
\pgfpathlineto{\pgfqpoint{5.427559in}{0.695092in}}%
\pgfpathlineto{\pgfqpoint{5.427856in}{0.695092in}}%
\pgfpathlineto{\pgfqpoint{5.428154in}{0.695092in}}%
\pgfpathlineto{\pgfqpoint{5.428451in}{0.695093in}}%
\pgfpathlineto{\pgfqpoint{5.428749in}{0.695093in}}%
\pgfpathlineto{\pgfqpoint{5.429046in}{0.695093in}}%
\pgfpathlineto{\pgfqpoint{5.429344in}{0.695093in}}%
\pgfpathlineto{\pgfqpoint{5.429641in}{0.695093in}}%
\pgfpathlineto{\pgfqpoint{5.429939in}{0.695093in}}%
\pgfpathlineto{\pgfqpoint{5.430236in}{0.695093in}}%
\pgfpathlineto{\pgfqpoint{5.430534in}{0.695094in}}%
\pgfpathlineto{\pgfqpoint{5.430831in}{0.695094in}}%
\pgfpathlineto{\pgfqpoint{5.431129in}{0.695094in}}%
\pgfpathlineto{\pgfqpoint{5.431426in}{0.695094in}}%
\pgfpathlineto{\pgfqpoint{5.431724in}{0.695094in}}%
\pgfpathlineto{\pgfqpoint{5.432021in}{0.695094in}}%
\pgfpathlineto{\pgfqpoint{5.432318in}{0.695094in}}%
\pgfpathlineto{\pgfqpoint{5.432616in}{0.695094in}}%
\pgfpathlineto{\pgfqpoint{5.432913in}{0.695095in}}%
\pgfpathlineto{\pgfqpoint{5.433211in}{0.695095in}}%
\pgfpathlineto{\pgfqpoint{5.433508in}{0.695095in}}%
\pgfpathlineto{\pgfqpoint{5.433806in}{0.695095in}}%
\pgfpathlineto{\pgfqpoint{5.434103in}{0.695095in}}%
\pgfpathlineto{\pgfqpoint{5.434401in}{0.695095in}}%
\pgfpathlineto{\pgfqpoint{5.434698in}{0.695095in}}%
\pgfpathlineto{\pgfqpoint{5.434996in}{0.695096in}}%
\pgfpathlineto{\pgfqpoint{5.435293in}{0.695096in}}%
\pgfpathlineto{\pgfqpoint{5.435591in}{0.695096in}}%
\pgfpathlineto{\pgfqpoint{5.435888in}{0.695096in}}%
\pgfpathlineto{\pgfqpoint{5.436186in}{0.695096in}}%
\pgfpathlineto{\pgfqpoint{5.436483in}{0.695096in}}%
\pgfpathlineto{\pgfqpoint{5.436781in}{0.695096in}}%
\pgfpathlineto{\pgfqpoint{5.437078in}{0.695096in}}%
\pgfpathlineto{\pgfqpoint{5.437376in}{0.695097in}}%
\pgfpathlineto{\pgfqpoint{5.437673in}{0.695097in}}%
\pgfpathlineto{\pgfqpoint{5.437971in}{0.695097in}}%
\pgfpathlineto{\pgfqpoint{5.438268in}{0.695097in}}%
\pgfpathlineto{\pgfqpoint{5.438565in}{0.695097in}}%
\pgfpathlineto{\pgfqpoint{5.438863in}{0.695097in}}%
\pgfpathlineto{\pgfqpoint{5.439160in}{0.695097in}}%
\pgfpathlineto{\pgfqpoint{5.439458in}{0.695098in}}%
\pgfpathlineto{\pgfqpoint{5.439755in}{0.695098in}}%
\pgfpathlineto{\pgfqpoint{5.440053in}{0.695098in}}%
\pgfpathlineto{\pgfqpoint{5.440350in}{0.695098in}}%
\pgfpathlineto{\pgfqpoint{5.440648in}{0.695098in}}%
\pgfpathlineto{\pgfqpoint{5.440945in}{0.695098in}}%
\pgfpathlineto{\pgfqpoint{5.441243in}{0.695098in}}%
\pgfpathlineto{\pgfqpoint{5.441540in}{0.695098in}}%
\pgfpathlineto{\pgfqpoint{5.441838in}{0.695099in}}%
\pgfpathlineto{\pgfqpoint{5.442135in}{0.695099in}}%
\pgfpathlineto{\pgfqpoint{5.442433in}{0.695099in}}%
\pgfpathlineto{\pgfqpoint{5.442730in}{0.695099in}}%
\pgfpathlineto{\pgfqpoint{5.443028in}{0.695099in}}%
\pgfpathlineto{\pgfqpoint{5.443325in}{0.695099in}}%
\pgfpathlineto{\pgfqpoint{5.443623in}{0.695099in}}%
\pgfpathlineto{\pgfqpoint{5.443920in}{0.695099in}}%
\pgfpathlineto{\pgfqpoint{5.444218in}{0.695100in}}%
\pgfpathlineto{\pgfqpoint{5.444515in}{0.695100in}}%
\pgfpathlineto{\pgfqpoint{5.444813in}{0.695100in}}%
\pgfpathlineto{\pgfqpoint{5.445110in}{0.695100in}}%
\pgfpathlineto{\pgfqpoint{5.445407in}{0.695100in}}%
\pgfpathlineto{\pgfqpoint{5.445705in}{0.695100in}}%
\pgfpathlineto{\pgfqpoint{5.446002in}{0.695100in}}%
\pgfpathlineto{\pgfqpoint{5.446300in}{0.695101in}}%
\pgfpathlineto{\pgfqpoint{5.446597in}{0.695101in}}%
\pgfpathlineto{\pgfqpoint{5.446895in}{0.695101in}}%
\pgfpathlineto{\pgfqpoint{5.447192in}{0.695101in}}%
\pgfpathlineto{\pgfqpoint{5.447490in}{0.695101in}}%
\pgfpathlineto{\pgfqpoint{5.447787in}{0.695131in}}%
\pgfpathlineto{\pgfqpoint{5.448085in}{0.695188in}}%
\pgfpathlineto{\pgfqpoint{5.448382in}{0.695223in}}%
\pgfpathlineto{\pgfqpoint{5.448680in}{0.695223in}}%
\pgfpathlineto{\pgfqpoint{5.448977in}{0.695221in}}%
\pgfpathlineto{\pgfqpoint{5.449275in}{0.695220in}}%
\pgfpathlineto{\pgfqpoint{5.449572in}{0.695219in}}%
\pgfpathlineto{\pgfqpoint{5.449870in}{0.695218in}}%
\pgfpathlineto{\pgfqpoint{5.450167in}{0.695217in}}%
\pgfpathlineto{\pgfqpoint{5.450465in}{0.695216in}}%
\pgfpathlineto{\pgfqpoint{5.450762in}{0.695215in}}%
\pgfpathlineto{\pgfqpoint{5.451060in}{0.695213in}}%
\pgfpathlineto{\pgfqpoint{5.451357in}{0.695212in}}%
\pgfpathlineto{\pgfqpoint{5.451655in}{0.695211in}}%
\pgfpathlineto{\pgfqpoint{5.451952in}{0.695210in}}%
\pgfpathlineto{\pgfqpoint{5.452249in}{0.695209in}}%
\pgfpathlineto{\pgfqpoint{5.452547in}{0.695208in}}%
\pgfpathlineto{\pgfqpoint{5.452844in}{0.695206in}}%
\pgfpathlineto{\pgfqpoint{5.453142in}{0.695205in}}%
\pgfpathlineto{\pgfqpoint{5.453439in}{0.695204in}}%
\pgfpathlineto{\pgfqpoint{5.453737in}{0.695203in}}%
\pgfpathlineto{\pgfqpoint{5.454034in}{0.695202in}}%
\pgfpathlineto{\pgfqpoint{5.454332in}{0.695201in}}%
\pgfpathlineto{\pgfqpoint{5.454629in}{0.695199in}}%
\pgfpathlineto{\pgfqpoint{5.454927in}{0.695198in}}%
\pgfpathlineto{\pgfqpoint{5.455224in}{0.695197in}}%
\pgfpathlineto{\pgfqpoint{5.455522in}{0.695196in}}%
\pgfpathlineto{\pgfqpoint{5.455819in}{0.695195in}}%
\pgfpathlineto{\pgfqpoint{5.456117in}{0.695194in}}%
\pgfpathlineto{\pgfqpoint{5.456414in}{0.695193in}}%
\pgfpathlineto{\pgfqpoint{5.456712in}{0.695191in}}%
\pgfpathlineto{\pgfqpoint{5.457009in}{0.695190in}}%
\pgfpathlineto{\pgfqpoint{5.457307in}{0.695189in}}%
\pgfpathlineto{\pgfqpoint{5.457604in}{0.695188in}}%
\pgfpathlineto{\pgfqpoint{5.457902in}{0.695187in}}%
\pgfpathlineto{\pgfqpoint{5.458199in}{0.695186in}}%
\pgfpathlineto{\pgfqpoint{5.458496in}{0.695184in}}%
\pgfpathlineto{\pgfqpoint{5.458794in}{0.695183in}}%
\pgfpathlineto{\pgfqpoint{5.459091in}{0.695182in}}%
\pgfpathlineto{\pgfqpoint{5.459389in}{0.695181in}}%
\pgfpathlineto{\pgfqpoint{5.459686in}{0.695180in}}%
\pgfpathlineto{\pgfqpoint{5.459984in}{0.695179in}}%
\pgfpathlineto{\pgfqpoint{5.460281in}{0.695177in}}%
\pgfpathlineto{\pgfqpoint{5.460579in}{0.695176in}}%
\pgfpathlineto{\pgfqpoint{5.460876in}{0.695175in}}%
\pgfpathlineto{\pgfqpoint{5.461174in}{0.695174in}}%
\pgfpathlineto{\pgfqpoint{5.461471in}{0.695173in}}%
\pgfpathlineto{\pgfqpoint{5.461769in}{0.695172in}}%
\pgfpathlineto{\pgfqpoint{5.462066in}{0.695171in}}%
\pgfpathlineto{\pgfqpoint{5.462364in}{0.695169in}}%
\pgfpathlineto{\pgfqpoint{5.462661in}{0.695168in}}%
\pgfpathlineto{\pgfqpoint{5.462959in}{0.695167in}}%
\pgfpathlineto{\pgfqpoint{5.463256in}{0.695166in}}%
\pgfpathlineto{\pgfqpoint{5.463554in}{0.695165in}}%
\pgfpathlineto{\pgfqpoint{5.463851in}{0.695164in}}%
\pgfpathlineto{\pgfqpoint{5.464149in}{0.695162in}}%
\pgfpathlineto{\pgfqpoint{5.464446in}{0.695161in}}%
\pgfpathlineto{\pgfqpoint{5.464744in}{0.695160in}}%
\pgfpathlineto{\pgfqpoint{5.465041in}{0.695159in}}%
\pgfpathlineto{\pgfqpoint{5.465338in}{0.695158in}}%
\pgfpathlineto{\pgfqpoint{5.465636in}{0.695157in}}%
\pgfpathlineto{\pgfqpoint{5.465933in}{0.695155in}}%
\pgfpathlineto{\pgfqpoint{5.466231in}{0.695154in}}%
\pgfpathlineto{\pgfqpoint{5.466528in}{0.695153in}}%
\pgfpathlineto{\pgfqpoint{5.466826in}{0.695152in}}%
\pgfpathlineto{\pgfqpoint{5.467123in}{0.695151in}}%
\pgfpathlineto{\pgfqpoint{5.467421in}{0.695150in}}%
\pgfpathlineto{\pgfqpoint{5.467718in}{0.695149in}}%
\pgfpathlineto{\pgfqpoint{5.468016in}{0.695147in}}%
\pgfpathlineto{\pgfqpoint{5.468313in}{0.695146in}}%
\pgfpathlineto{\pgfqpoint{5.468611in}{0.695145in}}%
\pgfpathlineto{\pgfqpoint{5.468908in}{0.695144in}}%
\pgfpathlineto{\pgfqpoint{5.469206in}{0.695143in}}%
\pgfpathlineto{\pgfqpoint{5.469503in}{0.695142in}}%
\pgfpathlineto{\pgfqpoint{5.469801in}{0.695140in}}%
\pgfpathlineto{\pgfqpoint{5.470098in}{0.695139in}}%
\pgfpathlineto{\pgfqpoint{5.470396in}{0.695138in}}%
\pgfpathlineto{\pgfqpoint{5.470693in}{0.695137in}}%
\pgfpathlineto{\pgfqpoint{5.470991in}{0.695136in}}%
\pgfpathlineto{\pgfqpoint{5.471288in}{0.695135in}}%
\pgfpathlineto{\pgfqpoint{5.471586in}{0.695133in}}%
\pgfpathlineto{\pgfqpoint{5.471883in}{0.695132in}}%
\pgfpathlineto{\pgfqpoint{5.472180in}{0.695131in}}%
\pgfpathlineto{\pgfqpoint{5.472478in}{0.695130in}}%
\pgfpathlineto{\pgfqpoint{5.472775in}{0.695129in}}%
\pgfpathlineto{\pgfqpoint{5.473073in}{0.695128in}}%
\pgfpathlineto{\pgfqpoint{5.473370in}{0.695127in}}%
\pgfpathlineto{\pgfqpoint{5.473668in}{0.695125in}}%
\pgfpathlineto{\pgfqpoint{5.473965in}{0.695124in}}%
\pgfpathlineto{\pgfqpoint{5.474263in}{0.695123in}}%
\pgfpathlineto{\pgfqpoint{5.474560in}{0.695122in}}%
\pgfpathlineto{\pgfqpoint{5.474858in}{0.695121in}}%
\pgfpathlineto{\pgfqpoint{5.475155in}{0.695120in}}%
\pgfpathlineto{\pgfqpoint{5.475453in}{0.695124in}}%
\pgfpathlineto{\pgfqpoint{5.475750in}{0.695478in}}%
\pgfpathlineto{\pgfqpoint{5.476048in}{0.696039in}}%
\pgfpathlineto{\pgfqpoint{5.476345in}{0.696600in}}%
\pgfpathlineto{\pgfqpoint{5.476643in}{0.697037in}}%
\pgfpathlineto{\pgfqpoint{5.476940in}{0.697368in}}%
\pgfpathlineto{\pgfqpoint{5.477238in}{0.697699in}}%
\pgfpathlineto{\pgfqpoint{5.477535in}{0.698030in}}%
\pgfpathlineto{\pgfqpoint{5.477833in}{0.698361in}}%
\pgfpathlineto{\pgfqpoint{5.478130in}{0.698692in}}%
\pgfpathlineto{\pgfqpoint{5.478428in}{0.699022in}}%
\pgfpathlineto{\pgfqpoint{5.478725in}{0.699353in}}%
\pgfpathlineto{\pgfqpoint{5.479022in}{0.699684in}}%
\pgfpathlineto{\pgfqpoint{5.479320in}{0.700015in}}%
\pgfpathlineto{\pgfqpoint{5.479617in}{0.700346in}}%
\pgfpathlineto{\pgfqpoint{5.479915in}{0.700677in}}%
\pgfpathlineto{\pgfqpoint{5.480212in}{0.701008in}}%
\pgfpathlineto{\pgfqpoint{5.480510in}{0.701339in}}%
\pgfpathlineto{\pgfqpoint{5.480807in}{0.701670in}}%
\pgfpathlineto{\pgfqpoint{5.481105in}{0.702001in}}%
\pgfpathlineto{\pgfqpoint{5.481402in}{0.702331in}}%
\pgfpathlineto{\pgfqpoint{5.481700in}{0.702662in}}%
\pgfpathlineto{\pgfqpoint{5.481997in}{0.702993in}}%
\pgfpathlineto{\pgfqpoint{5.482295in}{0.703324in}}%
\pgfpathlineto{\pgfqpoint{5.482592in}{0.703655in}}%
\pgfpathlineto{\pgfqpoint{5.482890in}{0.703986in}}%
\pgfpathlineto{\pgfqpoint{5.483187in}{0.704317in}}%
\pgfpathlineto{\pgfqpoint{5.483485in}{0.704648in}}%
\pgfpathlineto{\pgfqpoint{5.483782in}{0.704979in}}%
\pgfpathlineto{\pgfqpoint{5.484080in}{0.705210in}}%
\pgfpathlineto{\pgfqpoint{5.484377in}{0.705324in}}%
\pgfpathlineto{\pgfqpoint{5.484675in}{0.705437in}}%
\pgfpathlineto{\pgfqpoint{5.484972in}{0.705551in}}%
\pgfpathlineto{\pgfqpoint{5.485269in}{0.705664in}}%
\pgfpathlineto{\pgfqpoint{5.485567in}{0.705778in}}%
\pgfpathlineto{\pgfqpoint{5.485864in}{0.705892in}}%
\pgfpathlineto{\pgfqpoint{5.486162in}{0.706005in}}%
\pgfpathlineto{\pgfqpoint{5.486459in}{0.706119in}}%
\pgfpathlineto{\pgfqpoint{5.486757in}{0.706233in}}%
\pgfpathlineto{\pgfqpoint{5.487054in}{0.706346in}}%
\pgfpathlineto{\pgfqpoint{5.487352in}{0.706460in}}%
\pgfpathlineto{\pgfqpoint{5.487649in}{0.706574in}}%
\pgfpathlineto{\pgfqpoint{5.487947in}{0.706687in}}%
\pgfpathlineto{\pgfqpoint{5.488244in}{0.706801in}}%
\pgfpathlineto{\pgfqpoint{5.488542in}{0.706914in}}%
\pgfpathlineto{\pgfqpoint{5.488839in}{0.707028in}}%
\pgfpathlineto{\pgfqpoint{5.489137in}{0.707142in}}%
\pgfpathlineto{\pgfqpoint{5.489434in}{0.707255in}}%
\pgfpathlineto{\pgfqpoint{5.489732in}{0.707369in}}%
\pgfpathlineto{\pgfqpoint{5.490029in}{0.707483in}}%
\pgfpathlineto{\pgfqpoint{5.490327in}{0.707596in}}%
\pgfpathlineto{\pgfqpoint{5.490624in}{0.707710in}}%
\pgfpathlineto{\pgfqpoint{5.490922in}{0.707824in}}%
\pgfpathlineto{\pgfqpoint{5.491219in}{0.707937in}}%
\pgfpathlineto{\pgfqpoint{5.491517in}{0.708035in}}%
\pgfpathlineto{\pgfqpoint{5.491814in}{0.707874in}}%
\pgfpathlineto{\pgfqpoint{5.492111in}{0.707622in}}%
\pgfpathlineto{\pgfqpoint{5.492409in}{0.707369in}}%
\pgfpathlineto{\pgfqpoint{5.492706in}{0.707116in}}%
\pgfpathlineto{\pgfqpoint{5.493004in}{0.706863in}}%
\pgfpathlineto{\pgfqpoint{5.493301in}{0.706611in}}%
\pgfpathlineto{\pgfqpoint{5.493599in}{0.706358in}}%
\pgfpathlineto{\pgfqpoint{5.493896in}{0.706105in}}%
\pgfpathlineto{\pgfqpoint{5.494194in}{0.705852in}}%
\pgfpathlineto{\pgfqpoint{5.494491in}{0.705600in}}%
\pgfpathlineto{\pgfqpoint{5.494789in}{0.705347in}}%
\pgfpathlineto{\pgfqpoint{5.495086in}{0.705094in}}%
\pgfpathlineto{\pgfqpoint{5.495384in}{0.704841in}}%
\pgfpathlineto{\pgfqpoint{5.495681in}{0.704589in}}%
\pgfpathlineto{\pgfqpoint{5.495979in}{0.704336in}}%
\pgfpathlineto{\pgfqpoint{5.496276in}{0.704083in}}%
\pgfpathlineto{\pgfqpoint{5.496574in}{0.703830in}}%
\pgfpathlineto{\pgfqpoint{5.496871in}{0.703578in}}%
\pgfpathlineto{\pgfqpoint{5.497169in}{0.703325in}}%
\pgfpathlineto{\pgfqpoint{5.497466in}{0.703072in}}%
\pgfpathlineto{\pgfqpoint{5.497764in}{0.702819in}}%
\pgfpathlineto{\pgfqpoint{5.498061in}{0.702567in}}%
\pgfpathlineto{\pgfqpoint{5.498359in}{0.702314in}}%
\pgfpathlineto{\pgfqpoint{5.498656in}{0.702061in}}%
\pgfpathlineto{\pgfqpoint{5.498953in}{0.701808in}}%
\pgfpathlineto{\pgfqpoint{5.499251in}{0.701556in}}%
\pgfpathlineto{\pgfqpoint{5.499548in}{0.701303in}}%
\pgfpathlineto{\pgfqpoint{5.499846in}{0.701050in}}%
\pgfpathlineto{\pgfqpoint{5.500143in}{0.700797in}}%
\pgfpathlineto{\pgfqpoint{5.500441in}{0.700545in}}%
\pgfpathlineto{\pgfqpoint{5.500738in}{0.700292in}}%
\pgfpathlineto{\pgfqpoint{5.501036in}{0.700039in}}%
\pgfpathlineto{\pgfqpoint{5.501333in}{0.699786in}}%
\pgfpathlineto{\pgfqpoint{5.501631in}{0.699534in}}%
\pgfpathlineto{\pgfqpoint{5.501928in}{0.699281in}}%
\pgfpathlineto{\pgfqpoint{5.502226in}{0.699028in}}%
\pgfpathlineto{\pgfqpoint{5.502523in}{0.698775in}}%
\pgfpathlineto{\pgfqpoint{5.502821in}{0.698523in}}%
\pgfpathlineto{\pgfqpoint{5.503118in}{0.698270in}}%
\pgfpathlineto{\pgfqpoint{5.503416in}{0.698017in}}%
\pgfpathlineto{\pgfqpoint{5.503713in}{0.697764in}}%
\pgfpathlineto{\pgfqpoint{5.504011in}{0.697512in}}%
\pgfpathlineto{\pgfqpoint{5.504308in}{0.697497in}}%
\pgfpathlineto{\pgfqpoint{5.504606in}{0.702414in}}%
\pgfpathlineto{\pgfqpoint{5.504903in}{0.708129in}}%
\pgfpathlineto{\pgfqpoint{5.505200in}{0.708741in}}%
\pgfpathlineto{\pgfqpoint{5.505498in}{0.708742in}}%
\pgfpathlineto{\pgfqpoint{5.505795in}{0.708742in}}%
\pgfpathlineto{\pgfqpoint{5.506093in}{0.708743in}}%
\pgfpathlineto{\pgfqpoint{5.506390in}{0.708743in}}%
\pgfpathlineto{\pgfqpoint{5.506688in}{0.708744in}}%
\pgfpathlineto{\pgfqpoint{5.506985in}{0.708744in}}%
\pgfpathlineto{\pgfqpoint{5.507283in}{0.708745in}}%
\pgfpathlineto{\pgfqpoint{5.507580in}{0.708746in}}%
\pgfpathlineto{\pgfqpoint{5.507878in}{0.708746in}}%
\pgfpathlineto{\pgfqpoint{5.508175in}{0.708747in}}%
\pgfpathlineto{\pgfqpoint{5.508473in}{0.708747in}}%
\pgfpathlineto{\pgfqpoint{5.508770in}{0.708748in}}%
\pgfpathlineto{\pgfqpoint{5.509068in}{0.708749in}}%
\pgfpathlineto{\pgfqpoint{5.509365in}{0.708749in}}%
\pgfpathlineto{\pgfqpoint{5.509663in}{0.708750in}}%
\pgfpathlineto{\pgfqpoint{5.509960in}{0.708750in}}%
\pgfpathlineto{\pgfqpoint{5.510258in}{0.708751in}}%
\pgfpathlineto{\pgfqpoint{5.510555in}{0.708751in}}%
\pgfpathlineto{\pgfqpoint{5.510853in}{0.708752in}}%
\pgfpathlineto{\pgfqpoint{5.511150in}{0.708753in}}%
\pgfpathlineto{\pgfqpoint{5.511448in}{0.708753in}}%
\pgfpathlineto{\pgfqpoint{5.511745in}{0.708754in}}%
\pgfpathlineto{\pgfqpoint{5.512042in}{0.708754in}}%
\pgfpathlineto{\pgfqpoint{5.512340in}{0.708755in}}%
\pgfpathlineto{\pgfqpoint{5.512637in}{0.708755in}}%
\pgfpathlineto{\pgfqpoint{5.512935in}{0.708756in}}%
\pgfpathlineto{\pgfqpoint{5.513232in}{0.708757in}}%
\pgfpathlineto{\pgfqpoint{5.513530in}{0.708757in}}%
\pgfpathlineto{\pgfqpoint{5.513827in}{0.708758in}}%
\pgfpathlineto{\pgfqpoint{5.514125in}{0.708758in}}%
\pgfpathlineto{\pgfqpoint{5.514422in}{0.708759in}}%
\pgfpathlineto{\pgfqpoint{5.514720in}{0.708759in}}%
\pgfpathlineto{\pgfqpoint{5.515017in}{0.708760in}}%
\pgfpathlineto{\pgfqpoint{5.515315in}{0.708761in}}%
\pgfpathlineto{\pgfqpoint{5.515612in}{0.708761in}}%
\pgfpathlineto{\pgfqpoint{5.515910in}{0.708762in}}%
\pgfpathlineto{\pgfqpoint{5.516207in}{0.708762in}}%
\pgfpathlineto{\pgfqpoint{5.516505in}{0.708763in}}%
\pgfpathlineto{\pgfqpoint{5.516802in}{0.708764in}}%
\pgfpathlineto{\pgfqpoint{5.517100in}{0.708764in}}%
\pgfpathlineto{\pgfqpoint{5.517397in}{0.708765in}}%
\pgfpathlineto{\pgfqpoint{5.517695in}{0.708765in}}%
\pgfpathlineto{\pgfqpoint{5.517992in}{0.708766in}}%
\pgfpathlineto{\pgfqpoint{5.518290in}{0.708766in}}%
\pgfpathlineto{\pgfqpoint{5.518587in}{0.708767in}}%
\pgfpathlineto{\pgfqpoint{5.518884in}{0.708768in}}%
\pgfpathlineto{\pgfqpoint{5.519182in}{0.708768in}}%
\pgfpathlineto{\pgfqpoint{5.519479in}{0.708769in}}%
\pgfpathlineto{\pgfqpoint{5.519777in}{0.708769in}}%
\pgfpathlineto{\pgfqpoint{5.520074in}{0.708770in}}%
\pgfpathlineto{\pgfqpoint{5.520372in}{0.708770in}}%
\pgfpathlineto{\pgfqpoint{5.520669in}{0.708771in}}%
\pgfpathlineto{\pgfqpoint{5.520967in}{0.708772in}}%
\pgfpathlineto{\pgfqpoint{5.521264in}{0.708772in}}%
\pgfpathlineto{\pgfqpoint{5.521562in}{0.708773in}}%
\pgfpathlineto{\pgfqpoint{5.521859in}{0.708773in}}%
\pgfpathlineto{\pgfqpoint{5.522157in}{0.708774in}}%
\pgfpathlineto{\pgfqpoint{5.522454in}{0.708774in}}%
\pgfpathlineto{\pgfqpoint{5.522752in}{0.708775in}}%
\pgfpathlineto{\pgfqpoint{5.523049in}{0.708776in}}%
\pgfpathlineto{\pgfqpoint{5.523347in}{0.708776in}}%
\pgfpathlineto{\pgfqpoint{5.523644in}{0.708777in}}%
\pgfpathlineto{\pgfqpoint{5.523942in}{0.708777in}}%
\pgfpathlineto{\pgfqpoint{5.524239in}{0.708778in}}%
\pgfpathlineto{\pgfqpoint{5.524537in}{0.708779in}}%
\pgfpathlineto{\pgfqpoint{5.524834in}{0.708779in}}%
\pgfpathlineto{\pgfqpoint{5.525131in}{0.708780in}}%
\pgfpathlineto{\pgfqpoint{5.525429in}{0.708780in}}%
\pgfpathlineto{\pgfqpoint{5.525726in}{0.708781in}}%
\pgfpathlineto{\pgfqpoint{5.526024in}{0.708781in}}%
\pgfpathlineto{\pgfqpoint{5.526321in}{0.708782in}}%
\pgfpathlineto{\pgfqpoint{5.526619in}{0.708783in}}%
\pgfpathlineto{\pgfqpoint{5.526916in}{0.708783in}}%
\pgfpathlineto{\pgfqpoint{5.527214in}{0.708784in}}%
\pgfpathlineto{\pgfqpoint{5.527511in}{0.708789in}}%
\pgfpathlineto{\pgfqpoint{5.527809in}{0.708796in}}%
\pgfpathlineto{\pgfqpoint{5.528106in}{0.708803in}}%
\pgfpathlineto{\pgfqpoint{5.528404in}{0.708810in}}%
\pgfpathlineto{\pgfqpoint{5.528701in}{0.708817in}}%
\pgfpathlineto{\pgfqpoint{5.528999in}{0.708824in}}%
\pgfpathlineto{\pgfqpoint{5.529296in}{0.708831in}}%
\pgfpathlineto{\pgfqpoint{5.529594in}{0.708838in}}%
\pgfpathlineto{\pgfqpoint{5.529891in}{0.708845in}}%
\pgfpathlineto{\pgfqpoint{5.530189in}{0.708852in}}%
\pgfpathlineto{\pgfqpoint{5.530486in}{0.708859in}}%
\pgfpathlineto{\pgfqpoint{5.530784in}{0.708866in}}%
\pgfpathlineto{\pgfqpoint{5.531081in}{0.708874in}}%
\pgfpathlineto{\pgfqpoint{5.531379in}{0.708881in}}%
\pgfpathlineto{\pgfqpoint{5.531676in}{0.708888in}}%
\pgfpathlineto{\pgfqpoint{5.531973in}{0.708895in}}%
\pgfpathlineto{\pgfqpoint{5.532271in}{0.708902in}}%
\pgfpathlineto{\pgfqpoint{5.532568in}{0.708909in}}%
\pgfpathlineto{\pgfqpoint{5.532866in}{0.708916in}}%
\pgfpathlineto{\pgfqpoint{5.533163in}{0.708923in}}%
\pgfpathlineto{\pgfqpoint{5.533461in}{0.708932in}}%
\pgfpathlineto{\pgfqpoint{5.533758in}{0.708943in}}%
\pgfpathlineto{\pgfqpoint{5.534056in}{0.708955in}}%
\pgfpathlineto{\pgfqpoint{5.534353in}{0.708966in}}%
\pgfpathlineto{\pgfqpoint{5.534651in}{0.708978in}}%
\pgfpathlineto{\pgfqpoint{5.534948in}{0.708989in}}%
\pgfpathlineto{\pgfqpoint{5.535246in}{0.709001in}}%
\pgfpathlineto{\pgfqpoint{5.535543in}{0.709012in}}%
\pgfpathlineto{\pgfqpoint{5.535841in}{0.709024in}}%
\pgfpathlineto{\pgfqpoint{5.536138in}{0.709035in}}%
\pgfpathlineto{\pgfqpoint{5.536436in}{0.709047in}}%
\pgfpathlineto{\pgfqpoint{5.536733in}{0.709059in}}%
\pgfpathlineto{\pgfqpoint{5.537031in}{0.709070in}}%
\pgfpathlineto{\pgfqpoint{5.537328in}{0.709082in}}%
\pgfpathlineto{\pgfqpoint{5.537626in}{0.709093in}}%
\pgfpathlineto{\pgfqpoint{5.537923in}{0.709105in}}%
\pgfpathlineto{\pgfqpoint{5.538221in}{0.709116in}}%
\pgfpathlineto{\pgfqpoint{5.538518in}{0.709128in}}%
\pgfpathlineto{\pgfqpoint{5.538815in}{0.709139in}}%
\pgfpathlineto{\pgfqpoint{5.539113in}{0.709151in}}%
\pgfpathlineto{\pgfqpoint{5.539410in}{0.709163in}}%
\pgfpathlineto{\pgfqpoint{5.539708in}{0.709174in}}%
\pgfpathlineto{\pgfqpoint{5.540005in}{0.709186in}}%
\pgfpathlineto{\pgfqpoint{5.540303in}{0.709197in}}%
\pgfpathlineto{\pgfqpoint{5.540600in}{0.709209in}}%
\pgfpathlineto{\pgfqpoint{5.540898in}{0.709220in}}%
\pgfpathlineto{\pgfqpoint{5.541195in}{0.709232in}}%
\pgfpathlineto{\pgfqpoint{5.541493in}{0.709243in}}%
\pgfpathlineto{\pgfqpoint{5.541790in}{0.709255in}}%
\pgfpathlineto{\pgfqpoint{5.542088in}{0.709267in}}%
\pgfpathlineto{\pgfqpoint{5.542385in}{0.709278in}}%
\pgfpathlineto{\pgfqpoint{5.542683in}{0.709290in}}%
\pgfpathlineto{\pgfqpoint{5.542980in}{0.709301in}}%
\pgfpathlineto{\pgfqpoint{5.543278in}{0.709313in}}%
\pgfpathlineto{\pgfqpoint{5.543575in}{0.709324in}}%
\pgfpathlineto{\pgfqpoint{5.543873in}{0.709336in}}%
\pgfpathlineto{\pgfqpoint{5.544170in}{0.709347in}}%
\pgfpathlineto{\pgfqpoint{5.544468in}{0.709359in}}%
\pgfpathlineto{\pgfqpoint{5.544765in}{0.709371in}}%
\pgfpathlineto{\pgfqpoint{5.545062in}{0.709382in}}%
\pgfpathlineto{\pgfqpoint{5.545360in}{0.709394in}}%
\pgfpathlineto{\pgfqpoint{5.545657in}{0.709405in}}%
\pgfpathlineto{\pgfqpoint{5.545955in}{0.709417in}}%
\pgfpathlineto{\pgfqpoint{5.546252in}{0.709428in}}%
\pgfpathlineto{\pgfqpoint{5.546550in}{0.709440in}}%
\pgfpathlineto{\pgfqpoint{5.546847in}{0.709451in}}%
\pgfpathlineto{\pgfqpoint{5.547145in}{0.709463in}}%
\pgfpathlineto{\pgfqpoint{5.547442in}{0.709475in}}%
\pgfpathlineto{\pgfqpoint{5.547740in}{0.709485in}}%
\pgfpathlineto{\pgfqpoint{5.548037in}{0.709493in}}%
\pgfpathlineto{\pgfqpoint{5.548335in}{0.709501in}}%
\pgfpathlineto{\pgfqpoint{5.548632in}{0.709509in}}%
\pgfpathlineto{\pgfqpoint{5.548930in}{0.709518in}}%
\pgfpathlineto{\pgfqpoint{5.549227in}{0.709526in}}%
\pgfpathlineto{\pgfqpoint{5.549525in}{0.709534in}}%
\pgfpathlineto{\pgfqpoint{5.549822in}{0.709542in}}%
\pgfpathlineto{\pgfqpoint{5.550120in}{0.709550in}}%
\pgfpathlineto{\pgfqpoint{5.550417in}{0.709558in}}%
\pgfpathlineto{\pgfqpoint{5.550715in}{0.709567in}}%
\pgfpathlineto{\pgfqpoint{5.551012in}{0.709575in}}%
\pgfpathlineto{\pgfqpoint{5.551310in}{0.709583in}}%
\pgfpathlineto{\pgfqpoint{5.551607in}{0.709591in}}%
\pgfpathlineto{\pgfqpoint{5.551904in}{0.709599in}}%
\pgfpathlineto{\pgfqpoint{5.552202in}{0.709607in}}%
\pgfpathlineto{\pgfqpoint{5.552499in}{0.709615in}}%
\pgfpathlineto{\pgfqpoint{5.552797in}{0.709624in}}%
\pgfpathlineto{\pgfqpoint{5.553094in}{0.709632in}}%
\pgfpathlineto{\pgfqpoint{5.553392in}{0.709640in}}%
\pgfpathlineto{\pgfqpoint{5.553689in}{0.709648in}}%
\pgfpathlineto{\pgfqpoint{5.553987in}{0.709656in}}%
\pgfpathlineto{\pgfqpoint{5.554284in}{0.709664in}}%
\pgfpathlineto{\pgfqpoint{5.554582in}{0.709673in}}%
\pgfpathlineto{\pgfqpoint{5.554879in}{0.709681in}}%
\pgfpathlineto{\pgfqpoint{5.555177in}{0.709689in}}%
\pgfpathlineto{\pgfqpoint{5.555474in}{0.709697in}}%
\pgfpathlineto{\pgfqpoint{5.555772in}{0.709705in}}%
\pgfpathlineto{\pgfqpoint{5.556069in}{0.709713in}}%
\pgfpathlineto{\pgfqpoint{5.556367in}{0.709722in}}%
\pgfpathlineto{\pgfqpoint{5.556664in}{0.709730in}}%
\pgfpathlineto{\pgfqpoint{5.556962in}{0.709738in}}%
\pgfpathlineto{\pgfqpoint{5.557259in}{0.709746in}}%
\pgfpathlineto{\pgfqpoint{5.557557in}{0.709754in}}%
\pgfpathlineto{\pgfqpoint{5.557854in}{0.709762in}}%
\pgfpathlineto{\pgfqpoint{5.558152in}{0.709771in}}%
\pgfpathlineto{\pgfqpoint{5.558449in}{0.709779in}}%
\pgfpathlineto{\pgfqpoint{5.558746in}{0.709787in}}%
\pgfpathlineto{\pgfqpoint{5.559044in}{0.709795in}}%
\pgfpathlineto{\pgfqpoint{5.559341in}{0.709803in}}%
\pgfpathlineto{\pgfqpoint{5.559639in}{0.709811in}}%
\pgfpathlineto{\pgfqpoint{5.559936in}{0.709819in}}%
\pgfpathlineto{\pgfqpoint{5.560234in}{0.709828in}}%
\pgfpathlineto{\pgfqpoint{5.560531in}{0.709836in}}%
\pgfpathlineto{\pgfqpoint{5.560829in}{0.709844in}}%
\pgfpathlineto{\pgfqpoint{5.561126in}{0.709852in}}%
\pgfpathlineto{\pgfqpoint{5.561424in}{0.709860in}}%
\pgfpathlineto{\pgfqpoint{5.561721in}{0.709868in}}%
\pgfpathlineto{\pgfqpoint{5.562019in}{0.709877in}}%
\pgfpathlineto{\pgfqpoint{5.562316in}{0.709885in}}%
\pgfpathlineto{\pgfqpoint{5.562614in}{0.709893in}}%
\pgfpathlineto{\pgfqpoint{5.562911in}{0.709901in}}%
\pgfpathlineto{\pgfqpoint{5.563209in}{0.709909in}}%
\pgfpathlineto{\pgfqpoint{5.563506in}{0.709917in}}%
\pgfpathlineto{\pgfqpoint{5.563804in}{0.709926in}}%
\pgfpathlineto{\pgfqpoint{5.564101in}{0.709934in}}%
\pgfpathlineto{\pgfqpoint{5.564399in}{0.709942in}}%
\pgfpathlineto{\pgfqpoint{5.564696in}{0.709950in}}%
\pgfpathlineto{\pgfqpoint{5.564993in}{0.709958in}}%
\pgfpathlineto{\pgfqpoint{5.565291in}{0.709966in}}%
\pgfpathlineto{\pgfqpoint{5.565588in}{0.709975in}}%
\pgfpathlineto{\pgfqpoint{5.565886in}{0.709983in}}%
\pgfpathlineto{\pgfqpoint{5.566183in}{0.709991in}}%
\pgfpathlineto{\pgfqpoint{5.566481in}{0.709999in}}%
\pgfpathlineto{\pgfqpoint{5.566778in}{0.710007in}}%
\pgfpathlineto{\pgfqpoint{5.567076in}{0.710015in}}%
\pgfpathlineto{\pgfqpoint{5.567373in}{0.710023in}}%
\pgfpathlineto{\pgfqpoint{5.567671in}{0.710032in}}%
\pgfpathlineto{\pgfqpoint{5.567968in}{0.710040in}}%
\pgfpathlineto{\pgfqpoint{5.568266in}{0.710048in}}%
\pgfpathlineto{\pgfqpoint{5.568563in}{0.710056in}}%
\pgfpathlineto{\pgfqpoint{5.568861in}{0.710064in}}%
\pgfpathlineto{\pgfqpoint{5.569158in}{0.710072in}}%
\pgfpathlineto{\pgfqpoint{5.569456in}{0.710081in}}%
\pgfpathlineto{\pgfqpoint{5.569753in}{0.710089in}}%
\pgfpathlineto{\pgfqpoint{5.570051in}{0.710097in}}%
\pgfpathlineto{\pgfqpoint{5.570348in}{0.710105in}}%
\pgfpathlineto{\pgfqpoint{5.570646in}{0.710113in}}%
\pgfpathlineto{\pgfqpoint{5.570943in}{0.710121in}}%
\pgfpathlineto{\pgfqpoint{5.571241in}{0.710130in}}%
\pgfpathlineto{\pgfqpoint{5.571538in}{0.710138in}}%
\pgfpathlineto{\pgfqpoint{5.571835in}{0.710146in}}%
\pgfpathlineto{\pgfqpoint{5.572133in}{0.710154in}}%
\pgfpathlineto{\pgfqpoint{5.572430in}{0.710162in}}%
\pgfpathlineto{\pgfqpoint{5.572728in}{0.710170in}}%
\pgfpathlineto{\pgfqpoint{5.573025in}{0.710178in}}%
\pgfpathlineto{\pgfqpoint{5.573323in}{0.710187in}}%
\pgfpathlineto{\pgfqpoint{5.573620in}{0.710195in}}%
\pgfpathlineto{\pgfqpoint{5.573918in}{0.710203in}}%
\pgfpathlineto{\pgfqpoint{5.574215in}{0.710211in}}%
\pgfpathlineto{\pgfqpoint{5.574513in}{0.710219in}}%
\pgfpathlineto{\pgfqpoint{5.574810in}{0.710227in}}%
\pgfpathlineto{\pgfqpoint{5.575108in}{0.710236in}}%
\pgfpathlineto{\pgfqpoint{5.575405in}{0.710244in}}%
\pgfpathlineto{\pgfqpoint{5.575703in}{0.710252in}}%
\pgfpathlineto{\pgfqpoint{5.576000in}{0.710260in}}%
\pgfpathlineto{\pgfqpoint{5.576298in}{0.710268in}}%
\pgfpathlineto{\pgfqpoint{5.576595in}{0.710276in}}%
\pgfpathlineto{\pgfqpoint{5.576893in}{0.710285in}}%
\pgfpathlineto{\pgfqpoint{5.577190in}{0.710293in}}%
\pgfpathlineto{\pgfqpoint{5.577488in}{0.710301in}}%
\pgfpathlineto{\pgfqpoint{5.577785in}{0.710309in}}%
\pgfpathlineto{\pgfqpoint{5.578083in}{0.710317in}}%
\pgfpathlineto{\pgfqpoint{5.578380in}{0.710325in}}%
\pgfpathlineto{\pgfqpoint{5.578677in}{0.710334in}}%
\pgfpathlineto{\pgfqpoint{5.578975in}{0.710342in}}%
\pgfpathlineto{\pgfqpoint{5.579272in}{0.710350in}}%
\pgfpathlineto{\pgfqpoint{5.579570in}{0.710358in}}%
\pgfpathlineto{\pgfqpoint{5.579867in}{0.710366in}}%
\pgfpathlineto{\pgfqpoint{5.580165in}{0.710374in}}%
\pgfpathlineto{\pgfqpoint{5.580462in}{0.710382in}}%
\pgfpathlineto{\pgfqpoint{5.580760in}{0.710391in}}%
\pgfpathlineto{\pgfqpoint{5.581057in}{0.710399in}}%
\pgfpathlineto{\pgfqpoint{5.581355in}{0.710407in}}%
\pgfpathlineto{\pgfqpoint{5.581652in}{0.710415in}}%
\pgfpathlineto{\pgfqpoint{5.581950in}{0.710423in}}%
\pgfpathlineto{\pgfqpoint{5.582247in}{0.710431in}}%
\pgfpathlineto{\pgfqpoint{5.582545in}{0.710440in}}%
\pgfpathlineto{\pgfqpoint{5.582842in}{0.710448in}}%
\pgfpathlineto{\pgfqpoint{5.583140in}{0.710456in}}%
\pgfpathlineto{\pgfqpoint{5.583437in}{0.710464in}}%
\pgfpathlineto{\pgfqpoint{5.583735in}{0.710472in}}%
\pgfpathlineto{\pgfqpoint{5.584032in}{0.710480in}}%
\pgfpathlineto{\pgfqpoint{5.584330in}{0.710489in}}%
\pgfpathlineto{\pgfqpoint{5.584627in}{0.710497in}}%
\pgfpathlineto{\pgfqpoint{5.584924in}{0.710505in}}%
\pgfpathlineto{\pgfqpoint{5.585222in}{0.710513in}}%
\pgfpathlineto{\pgfqpoint{5.585519in}{0.710521in}}%
\pgfpathlineto{\pgfqpoint{5.585817in}{0.710529in}}%
\pgfpathlineto{\pgfqpoint{5.586114in}{0.710538in}}%
\pgfpathlineto{\pgfqpoint{5.586412in}{0.710546in}}%
\pgfpathlineto{\pgfqpoint{5.586709in}{0.710554in}}%
\pgfpathlineto{\pgfqpoint{5.587007in}{0.710562in}}%
\pgfpathlineto{\pgfqpoint{5.587304in}{0.710570in}}%
\pgfpathlineto{\pgfqpoint{5.587602in}{0.710578in}}%
\pgfpathlineto{\pgfqpoint{5.587899in}{0.710586in}}%
\pgfpathlineto{\pgfqpoint{5.588197in}{0.710595in}}%
\pgfpathlineto{\pgfqpoint{5.588494in}{0.710603in}}%
\pgfpathlineto{\pgfqpoint{5.588792in}{0.710611in}}%
\pgfpathlineto{\pgfqpoint{5.589089in}{0.710619in}}%
\pgfpathlineto{\pgfqpoint{5.589387in}{0.710627in}}%
\pgfpathlineto{\pgfqpoint{5.589684in}{0.710635in}}%
\pgfpathlineto{\pgfqpoint{5.589982in}{0.710644in}}%
\pgfpathlineto{\pgfqpoint{5.590279in}{0.710652in}}%
\pgfpathlineto{\pgfqpoint{5.590577in}{0.710660in}}%
\pgfpathlineto{\pgfqpoint{5.590874in}{0.710668in}}%
\pgfpathlineto{\pgfqpoint{5.591172in}{0.710676in}}%
\pgfpathlineto{\pgfqpoint{5.591469in}{0.710684in}}%
\pgfpathlineto{\pgfqpoint{5.591766in}{0.710693in}}%
\pgfpathlineto{\pgfqpoint{5.592064in}{0.710701in}}%
\pgfpathlineto{\pgfqpoint{5.592361in}{0.710709in}}%
\pgfpathlineto{\pgfqpoint{5.592659in}{0.710717in}}%
\pgfpathlineto{\pgfqpoint{5.592956in}{0.710725in}}%
\pgfpathlineto{\pgfqpoint{5.593254in}{0.710733in}}%
\pgfpathlineto{\pgfqpoint{5.593551in}{0.710741in}}%
\pgfpathlineto{\pgfqpoint{5.593849in}{0.710750in}}%
\pgfpathlineto{\pgfqpoint{5.594146in}{0.710758in}}%
\pgfpathlineto{\pgfqpoint{5.594444in}{0.710766in}}%
\pgfpathlineto{\pgfqpoint{5.594741in}{0.710774in}}%
\pgfpathlineto{\pgfqpoint{5.595039in}{0.710782in}}%
\pgfpathlineto{\pgfqpoint{5.595336in}{0.710790in}}%
\pgfpathlineto{\pgfqpoint{5.595634in}{0.710799in}}%
\pgfpathlineto{\pgfqpoint{5.595931in}{0.710807in}}%
\pgfpathlineto{\pgfqpoint{5.596229in}{0.710815in}}%
\pgfpathlineto{\pgfqpoint{5.596526in}{0.710823in}}%
\pgfpathlineto{\pgfqpoint{5.596824in}{0.710831in}}%
\pgfpathlineto{\pgfqpoint{5.597121in}{0.710839in}}%
\pgfpathlineto{\pgfqpoint{5.597419in}{0.710848in}}%
\pgfpathlineto{\pgfqpoint{5.597716in}{0.710856in}}%
\pgfpathlineto{\pgfqpoint{5.598014in}{0.710864in}}%
\pgfpathlineto{\pgfqpoint{5.598311in}{0.710872in}}%
\pgfpathlineto{\pgfqpoint{5.598608in}{0.710933in}}%
\pgfpathlineto{\pgfqpoint{5.598906in}{0.716708in}}%
\pgfpathlineto{\pgfqpoint{5.599203in}{0.718233in}}%
\pgfpathlineto{\pgfqpoint{5.599501in}{0.718238in}}%
\pgfpathlineto{\pgfqpoint{5.599798in}{0.718243in}}%
\pgfpathlineto{\pgfqpoint{5.600096in}{0.718248in}}%
\pgfpathlineto{\pgfqpoint{5.600393in}{0.718252in}}%
\pgfpathlineto{\pgfqpoint{5.600691in}{0.718257in}}%
\pgfpathlineto{\pgfqpoint{5.600988in}{0.718262in}}%
\pgfpathlineto{\pgfqpoint{5.601286in}{0.718267in}}%
\pgfpathlineto{\pgfqpoint{5.601583in}{0.718272in}}%
\pgfpathlineto{\pgfqpoint{5.601881in}{0.718276in}}%
\pgfpathlineto{\pgfqpoint{5.602178in}{0.718281in}}%
\pgfpathlineto{\pgfqpoint{5.602476in}{0.718286in}}%
\pgfpathlineto{\pgfqpoint{5.602773in}{0.718291in}}%
\pgfpathlineto{\pgfqpoint{5.603071in}{0.718295in}}%
\pgfpathlineto{\pgfqpoint{5.603368in}{0.718300in}}%
\pgfpathlineto{\pgfqpoint{5.603666in}{0.718305in}}%
\pgfpathlineto{\pgfqpoint{5.603963in}{0.718310in}}%
\pgfpathlineto{\pgfqpoint{5.604261in}{0.718314in}}%
\pgfpathlineto{\pgfqpoint{5.604558in}{0.718309in}}%
\pgfpathlineto{\pgfqpoint{5.604855in}{0.716231in}}%
\pgfpathlineto{\pgfqpoint{5.605153in}{0.711381in}}%
\pgfpathlineto{\pgfqpoint{5.605450in}{0.710882in}}%
\pgfpathlineto{\pgfqpoint{5.605748in}{0.710882in}}%
\pgfpathlineto{\pgfqpoint{5.606045in}{0.710882in}}%
\pgfpathlineto{\pgfqpoint{5.606343in}{0.710882in}}%
\pgfpathlineto{\pgfqpoint{5.606640in}{0.710882in}}%
\pgfpathlineto{\pgfqpoint{5.606938in}{0.710882in}}%
\pgfpathlineto{\pgfqpoint{5.607235in}{0.710882in}}%
\pgfpathlineto{\pgfqpoint{5.607533in}{0.710882in}}%
\pgfpathlineto{\pgfqpoint{5.607830in}{0.710882in}}%
\pgfpathlineto{\pgfqpoint{5.608128in}{0.710882in}}%
\pgfpathlineto{\pgfqpoint{5.608425in}{0.710882in}}%
\pgfpathlineto{\pgfqpoint{5.608723in}{0.710882in}}%
\pgfpathlineto{\pgfqpoint{5.609020in}{0.710882in}}%
\pgfpathlineto{\pgfqpoint{5.609318in}{0.710882in}}%
\pgfpathlineto{\pgfqpoint{5.609615in}{0.710881in}}%
\pgfpathlineto{\pgfqpoint{5.609913in}{0.710881in}}%
\pgfpathlineto{\pgfqpoint{5.610210in}{0.710881in}}%
\pgfpathlineto{\pgfqpoint{5.610508in}{0.710881in}}%
\pgfpathlineto{\pgfqpoint{5.610805in}{0.710881in}}%
\pgfpathlineto{\pgfqpoint{5.611103in}{0.710881in}}%
\pgfpathlineto{\pgfqpoint{5.611400in}{0.710881in}}%
\pgfpathlineto{\pgfqpoint{5.611697in}{0.710881in}}%
\pgfpathlineto{\pgfqpoint{5.611995in}{0.710881in}}%
\pgfpathlineto{\pgfqpoint{5.612292in}{0.710881in}}%
\pgfpathlineto{\pgfqpoint{5.612590in}{0.710881in}}%
\pgfpathlineto{\pgfqpoint{5.612887in}{0.710881in}}%
\pgfpathlineto{\pgfqpoint{5.613185in}{0.710881in}}%
\pgfpathlineto{\pgfqpoint{5.613482in}{0.710881in}}%
\pgfpathlineto{\pgfqpoint{5.613780in}{0.710881in}}%
\pgfpathlineto{\pgfqpoint{5.614077in}{0.710881in}}%
\pgfpathlineto{\pgfqpoint{5.614375in}{0.710881in}}%
\pgfpathlineto{\pgfqpoint{5.614672in}{0.710880in}}%
\pgfpathlineto{\pgfqpoint{5.614970in}{0.710880in}}%
\pgfpathlineto{\pgfqpoint{5.615267in}{0.710880in}}%
\pgfpathlineto{\pgfqpoint{5.615565in}{0.710879in}}%
\pgfpathlineto{\pgfqpoint{5.615862in}{0.710878in}}%
\pgfpathlineto{\pgfqpoint{5.616160in}{0.710878in}}%
\pgfpathlineto{\pgfqpoint{5.616457in}{0.710878in}}%
\pgfpathlineto{\pgfqpoint{5.616755in}{0.710877in}}%
\pgfpathlineto{\pgfqpoint{5.617052in}{0.710877in}}%
\pgfpathlineto{\pgfqpoint{5.617350in}{0.710877in}}%
\pgfpathlineto{\pgfqpoint{5.617647in}{0.710877in}}%
\pgfpathlineto{\pgfqpoint{5.617945in}{0.710876in}}%
\pgfpathlineto{\pgfqpoint{5.618242in}{0.710876in}}%
\pgfpathlineto{\pgfqpoint{5.618539in}{0.710876in}}%
\pgfpathlineto{\pgfqpoint{5.618837in}{0.710876in}}%
\pgfpathlineto{\pgfqpoint{5.619134in}{0.710875in}}%
\pgfpathlineto{\pgfqpoint{5.619432in}{0.710875in}}%
\pgfpathlineto{\pgfqpoint{5.619729in}{0.710875in}}%
\pgfpathlineto{\pgfqpoint{5.620027in}{0.710874in}}%
\pgfpathlineto{\pgfqpoint{5.620324in}{0.710874in}}%
\pgfpathlineto{\pgfqpoint{5.620622in}{0.710874in}}%
\pgfpathlineto{\pgfqpoint{5.620919in}{0.710874in}}%
\pgfpathlineto{\pgfqpoint{5.621217in}{0.710873in}}%
\pgfpathlineto{\pgfqpoint{5.621514in}{0.710873in}}%
\pgfpathlineto{\pgfqpoint{5.621812in}{0.710873in}}%
\pgfpathlineto{\pgfqpoint{5.622109in}{0.710872in}}%
\pgfpathlineto{\pgfqpoint{5.622407in}{0.710872in}}%
\pgfpathlineto{\pgfqpoint{5.622704in}{0.710872in}}%
\pgfpathlineto{\pgfqpoint{5.623002in}{0.710872in}}%
\pgfpathlineto{\pgfqpoint{5.623299in}{0.710871in}}%
\pgfpathlineto{\pgfqpoint{5.623597in}{0.710871in}}%
\pgfpathlineto{\pgfqpoint{5.623894in}{0.710871in}}%
\pgfpathlineto{\pgfqpoint{5.624192in}{0.710871in}}%
\pgfpathlineto{\pgfqpoint{5.624489in}{0.710870in}}%
\pgfpathlineto{\pgfqpoint{5.624786in}{0.710870in}}%
\pgfpathlineto{\pgfqpoint{5.625084in}{0.710870in}}%
\pgfpathlineto{\pgfqpoint{5.625381in}{0.710869in}}%
\pgfpathlineto{\pgfqpoint{5.625679in}{0.710869in}}%
\pgfpathlineto{\pgfqpoint{5.625976in}{0.710874in}}%
\pgfpathlineto{\pgfqpoint{5.626274in}{0.710891in}}%
\pgfpathlineto{\pgfqpoint{5.626571in}{0.710908in}}%
\pgfpathlineto{\pgfqpoint{5.626869in}{0.710795in}}%
\pgfpathlineto{\pgfqpoint{5.627166in}{0.710918in}}%
\pgfpathlineto{\pgfqpoint{5.627464in}{0.711090in}}%
\pgfpathlineto{\pgfqpoint{5.627761in}{0.711262in}}%
\pgfpathlineto{\pgfqpoint{5.628059in}{0.711434in}}%
\pgfpathlineto{\pgfqpoint{5.628356in}{0.711606in}}%
\pgfpathlineto{\pgfqpoint{5.628654in}{0.711778in}}%
\pgfpathlineto{\pgfqpoint{5.628951in}{0.711950in}}%
\pgfpathlineto{\pgfqpoint{5.629249in}{0.712122in}}%
\pgfpathlineto{\pgfqpoint{5.629546in}{0.712294in}}%
\pgfpathlineto{\pgfqpoint{5.629844in}{0.712466in}}%
\pgfpathlineto{\pgfqpoint{5.630141in}{0.712638in}}%
\pgfpathlineto{\pgfqpoint{5.630439in}{0.712810in}}%
\pgfpathlineto{\pgfqpoint{5.630736in}{0.712982in}}%
\pgfpathlineto{\pgfqpoint{5.631034in}{0.713153in}}%
\pgfpathlineto{\pgfqpoint{5.631331in}{0.713325in}}%
\pgfpathlineto{\pgfqpoint{5.631628in}{0.713497in}}%
\pgfpathlineto{\pgfqpoint{5.631926in}{0.713669in}}%
\pgfpathlineto{\pgfqpoint{5.632223in}{0.713841in}}%
\pgfpathlineto{\pgfqpoint{5.632521in}{0.714013in}}%
\pgfpathlineto{\pgfqpoint{5.632818in}{0.714185in}}%
\pgfpathlineto{\pgfqpoint{5.633116in}{0.714357in}}%
\pgfpathlineto{\pgfqpoint{5.633413in}{0.714529in}}%
\pgfpathlineto{\pgfqpoint{5.633711in}{0.714701in}}%
\pgfpathlineto{\pgfqpoint{5.634008in}{0.714873in}}%
\pgfpathlineto{\pgfqpoint{5.634306in}{0.715045in}}%
\pgfpathlineto{\pgfqpoint{5.634603in}{0.715217in}}%
\pgfpathlineto{\pgfqpoint{5.634901in}{0.715389in}}%
\pgfpathlineto{\pgfqpoint{5.635198in}{0.715561in}}%
\pgfpathlineto{\pgfqpoint{5.635496in}{0.715733in}}%
\pgfpathlineto{\pgfqpoint{5.635793in}{0.715904in}}%
\pgfpathlineto{\pgfqpoint{5.636091in}{0.716076in}}%
\pgfpathlineto{\pgfqpoint{5.636388in}{0.716248in}}%
\pgfpathlineto{\pgfqpoint{5.636686in}{0.716420in}}%
\pgfpathlineto{\pgfqpoint{5.636983in}{0.716592in}}%
\pgfpathlineto{\pgfqpoint{5.637281in}{0.716764in}}%
\pgfpathlineto{\pgfqpoint{5.637578in}{0.716936in}}%
\pgfpathlineto{\pgfqpoint{5.637876in}{0.717108in}}%
\pgfpathlineto{\pgfqpoint{5.638173in}{0.717280in}}%
\pgfpathlineto{\pgfqpoint{5.638470in}{0.717452in}}%
\pgfpathlineto{\pgfqpoint{5.638768in}{0.717624in}}%
\pgfpathlineto{\pgfqpoint{5.639065in}{0.717796in}}%
\pgfpathlineto{\pgfqpoint{5.639363in}{0.717968in}}%
\pgfpathlineto{\pgfqpoint{5.639660in}{0.718140in}}%
\pgfpathlineto{\pgfqpoint{5.639958in}{0.718956in}}%
\pgfpathlineto{\pgfqpoint{5.640255in}{0.718400in}}%
\pgfpathlineto{\pgfqpoint{5.640553in}{0.714117in}}%
\pgfpathlineto{\pgfqpoint{5.640850in}{0.713343in}}%
\pgfpathlineto{\pgfqpoint{5.641148in}{0.713632in}}%
\pgfpathlineto{\pgfqpoint{5.641445in}{0.713921in}}%
\pgfpathlineto{\pgfqpoint{5.641743in}{0.714209in}}%
\pgfpathlineto{\pgfqpoint{5.642040in}{0.714498in}}%
\pgfpathlineto{\pgfqpoint{5.642338in}{0.714787in}}%
\pgfpathlineto{\pgfqpoint{5.642635in}{0.715076in}}%
\pgfpathlineto{\pgfqpoint{5.642933in}{0.715365in}}%
\pgfpathlineto{\pgfqpoint{5.643230in}{0.715654in}}%
\pgfpathlineto{\pgfqpoint{5.643528in}{0.715942in}}%
\pgfpathlineto{\pgfqpoint{5.643825in}{0.716231in}}%
\pgfpathlineto{\pgfqpoint{5.644123in}{0.716520in}}%
\pgfpathlineto{\pgfqpoint{5.644420in}{0.716809in}}%
\pgfpathlineto{\pgfqpoint{5.644717in}{0.717098in}}%
\pgfpathlineto{\pgfqpoint{5.645015in}{0.717386in}}%
\pgfpathlineto{\pgfqpoint{5.645312in}{0.717675in}}%
\pgfpathlineto{\pgfqpoint{5.645610in}{0.717964in}}%
\pgfpathlineto{\pgfqpoint{5.645907in}{0.718253in}}%
\pgfpathlineto{\pgfqpoint{5.646205in}{0.718542in}}%
\pgfpathlineto{\pgfqpoint{5.646502in}{0.718831in}}%
\pgfpathlineto{\pgfqpoint{5.646800in}{0.719119in}}%
\pgfpathlineto{\pgfqpoint{5.647097in}{0.719408in}}%
\pgfpathlineto{\pgfqpoint{5.647395in}{0.719697in}}%
\pgfpathlineto{\pgfqpoint{5.647692in}{0.719986in}}%
\pgfpathlineto{\pgfqpoint{5.647990in}{0.720200in}}%
\pgfpathlineto{\pgfqpoint{5.648287in}{0.720264in}}%
\pgfpathlineto{\pgfqpoint{5.648585in}{0.720323in}}%
\pgfpathlineto{\pgfqpoint{5.648882in}{0.720383in}}%
\pgfpathlineto{\pgfqpoint{5.649180in}{0.720442in}}%
\pgfpathlineto{\pgfqpoint{5.649477in}{0.720502in}}%
\pgfpathlineto{\pgfqpoint{5.649775in}{0.720561in}}%
\pgfpathlineto{\pgfqpoint{5.650072in}{0.720621in}}%
\pgfpathlineto{\pgfqpoint{5.650370in}{0.720680in}}%
\pgfpathlineto{\pgfqpoint{5.650667in}{0.720740in}}%
\pgfpathlineto{\pgfqpoint{5.650965in}{0.720799in}}%
\pgfpathlineto{\pgfqpoint{5.651262in}{0.720859in}}%
\pgfpathlineto{\pgfqpoint{5.651559in}{0.720918in}}%
\pgfpathlineto{\pgfqpoint{5.651857in}{0.720978in}}%
\pgfpathlineto{\pgfqpoint{5.652154in}{0.721037in}}%
\pgfpathlineto{\pgfqpoint{5.652452in}{0.721097in}}%
\pgfpathlineto{\pgfqpoint{5.652749in}{0.721156in}}%
\pgfpathlineto{\pgfqpoint{5.653047in}{0.721216in}}%
\pgfpathlineto{\pgfqpoint{5.653344in}{0.721275in}}%
\pgfpathlineto{\pgfqpoint{5.653642in}{0.721335in}}%
\pgfpathlineto{\pgfqpoint{5.653939in}{0.721394in}}%
\pgfpathlineto{\pgfqpoint{5.654237in}{0.721454in}}%
\pgfpathlineto{\pgfqpoint{5.654534in}{0.721513in}}%
\pgfpathlineto{\pgfqpoint{5.654832in}{0.721564in}}%
\pgfpathlineto{\pgfqpoint{5.655129in}{0.721570in}}%
\pgfpathlineto{\pgfqpoint{5.655427in}{0.721569in}}%
\pgfpathlineto{\pgfqpoint{5.655724in}{0.721569in}}%
\pgfpathlineto{\pgfqpoint{5.656022in}{0.721569in}}%
\pgfpathlineto{\pgfqpoint{5.656319in}{0.721568in}}%
\pgfpathlineto{\pgfqpoint{5.656617in}{0.721568in}}%
\pgfpathlineto{\pgfqpoint{5.656914in}{0.721566in}}%
\pgfpathlineto{\pgfqpoint{5.657212in}{0.721563in}}%
\pgfpathlineto{\pgfqpoint{5.657509in}{0.721561in}}%
\pgfpathlineto{\pgfqpoint{5.657807in}{0.721558in}}%
\pgfpathlineto{\pgfqpoint{5.658104in}{0.721555in}}%
\pgfpathlineto{\pgfqpoint{5.658401in}{0.721552in}}%
\pgfpathlineto{\pgfqpoint{5.658699in}{0.721550in}}%
\pgfpathlineto{\pgfqpoint{5.658996in}{0.721547in}}%
\pgfpathlineto{\pgfqpoint{5.659294in}{0.721544in}}%
\pgfpathlineto{\pgfqpoint{5.659591in}{0.721542in}}%
\pgfpathlineto{\pgfqpoint{5.659889in}{0.721539in}}%
\pgfpathlineto{\pgfqpoint{5.660186in}{0.721536in}}%
\pgfpathlineto{\pgfqpoint{5.660484in}{0.721534in}}%
\pgfpathlineto{\pgfqpoint{5.660781in}{0.721531in}}%
\pgfpathlineto{\pgfqpoint{5.661079in}{0.721528in}}%
\pgfpathlineto{\pgfqpoint{5.661376in}{0.721525in}}%
\pgfpathlineto{\pgfqpoint{5.661674in}{0.721523in}}%
\pgfpathlineto{\pgfqpoint{5.661971in}{0.721520in}}%
\pgfpathlineto{\pgfqpoint{5.662269in}{0.721517in}}%
\pgfpathlineto{\pgfqpoint{5.662566in}{0.721515in}}%
\pgfpathlineto{\pgfqpoint{5.662864in}{0.721512in}}%
\pgfpathlineto{\pgfqpoint{5.663161in}{0.721509in}}%
\pgfpathlineto{\pgfqpoint{5.663459in}{0.721507in}}%
\pgfpathlineto{\pgfqpoint{5.663756in}{0.721504in}}%
\pgfpathlineto{\pgfqpoint{5.664054in}{0.721501in}}%
\pgfpathlineto{\pgfqpoint{5.664351in}{0.721498in}}%
\pgfpathlineto{\pgfqpoint{5.664648in}{0.721496in}}%
\pgfpathlineto{\pgfqpoint{5.664946in}{0.721493in}}%
\pgfpathlineto{\pgfqpoint{5.665243in}{0.721490in}}%
\pgfpathlineto{\pgfqpoint{5.665541in}{0.721488in}}%
\pgfpathlineto{\pgfqpoint{5.665838in}{0.721485in}}%
\pgfpathlineto{\pgfqpoint{5.666136in}{0.721482in}}%
\pgfpathlineto{\pgfqpoint{5.666433in}{0.721480in}}%
\pgfpathlineto{\pgfqpoint{5.666731in}{0.721477in}}%
\pgfpathlineto{\pgfqpoint{5.667028in}{0.721474in}}%
\pgfpathlineto{\pgfqpoint{5.667326in}{0.721471in}}%
\pgfpathlineto{\pgfqpoint{5.667623in}{0.721469in}}%
\pgfpathlineto{\pgfqpoint{5.667921in}{0.721466in}}%
\pgfpathlineto{\pgfqpoint{5.668218in}{0.721463in}}%
\pgfpathlineto{\pgfqpoint{5.668516in}{0.721461in}}%
\pgfpathlineto{\pgfqpoint{5.668813in}{0.721458in}}%
\pgfpathlineto{\pgfqpoint{5.669111in}{0.721455in}}%
\pgfpathlineto{\pgfqpoint{5.669408in}{0.721453in}}%
\pgfpathlineto{\pgfqpoint{5.669706in}{0.721450in}}%
\pgfpathlineto{\pgfqpoint{5.670003in}{0.721447in}}%
\pgfpathlineto{\pgfqpoint{5.670301in}{0.721444in}}%
\pgfpathlineto{\pgfqpoint{5.670598in}{0.721442in}}%
\pgfpathlineto{\pgfqpoint{5.670896in}{0.721439in}}%
\pgfpathlineto{\pgfqpoint{5.671193in}{0.721436in}}%
\pgfpathlineto{\pgfqpoint{5.671490in}{0.721434in}}%
\pgfpathlineto{\pgfqpoint{5.671788in}{0.721431in}}%
\pgfpathlineto{\pgfqpoint{5.672085in}{0.721428in}}%
\pgfpathlineto{\pgfqpoint{5.672383in}{0.721426in}}%
\pgfpathlineto{\pgfqpoint{5.672680in}{0.721423in}}%
\pgfpathlineto{\pgfqpoint{5.672978in}{0.721420in}}%
\pgfpathlineto{\pgfqpoint{5.673275in}{0.721417in}}%
\pgfpathlineto{\pgfqpoint{5.673573in}{0.721415in}}%
\pgfpathlineto{\pgfqpoint{5.673870in}{0.721412in}}%
\pgfpathlineto{\pgfqpoint{5.674168in}{0.721409in}}%
\pgfpathlineto{\pgfqpoint{5.674465in}{0.721407in}}%
\pgfpathlineto{\pgfqpoint{5.674763in}{0.721404in}}%
\pgfpathlineto{\pgfqpoint{5.675060in}{0.721401in}}%
\pgfpathlineto{\pgfqpoint{5.675358in}{0.721399in}}%
\pgfpathlineto{\pgfqpoint{5.675655in}{0.721396in}}%
\pgfpathlineto{\pgfqpoint{5.675953in}{0.721393in}}%
\pgfpathlineto{\pgfqpoint{5.676250in}{0.721390in}}%
\pgfpathlineto{\pgfqpoint{5.676548in}{0.721388in}}%
\pgfpathlineto{\pgfqpoint{5.676845in}{0.721385in}}%
\pgfpathlineto{\pgfqpoint{5.677143in}{0.721382in}}%
\pgfpathlineto{\pgfqpoint{5.677440in}{0.721380in}}%
\pgfpathlineto{\pgfqpoint{5.677738in}{0.721377in}}%
\pgfpathlineto{\pgfqpoint{5.678035in}{0.721374in}}%
\pgfpathlineto{\pgfqpoint{5.678332in}{0.721372in}}%
\pgfpathlineto{\pgfqpoint{5.678630in}{0.721369in}}%
\pgfpathlineto{\pgfqpoint{5.678927in}{0.721366in}}%
\pgfpathlineto{\pgfqpoint{5.679225in}{0.721363in}}%
\pgfpathlineto{\pgfqpoint{5.679522in}{0.721361in}}%
\pgfpathlineto{\pgfqpoint{5.679820in}{0.721358in}}%
\pgfpathlineto{\pgfqpoint{5.680117in}{0.721355in}}%
\pgfpathlineto{\pgfqpoint{5.680415in}{0.721353in}}%
\pgfpathlineto{\pgfqpoint{5.680712in}{0.721350in}}%
\pgfpathlineto{\pgfqpoint{5.681010in}{0.721347in}}%
\pgfpathlineto{\pgfqpoint{5.681307in}{0.721345in}}%
\pgfpathlineto{\pgfqpoint{5.681605in}{0.721342in}}%
\pgfpathlineto{\pgfqpoint{5.681902in}{0.721339in}}%
\pgfpathlineto{\pgfqpoint{5.682200in}{0.721336in}}%
\pgfpathlineto{\pgfqpoint{5.682497in}{0.721334in}}%
\pgfpathlineto{\pgfqpoint{5.682795in}{0.721331in}}%
\pgfpathlineto{\pgfqpoint{5.683092in}{0.721328in}}%
\pgfpathlineto{\pgfqpoint{5.683390in}{0.721326in}}%
\pgfpathlineto{\pgfqpoint{5.683687in}{0.721323in}}%
\pgfpathlineto{\pgfqpoint{5.683985in}{0.721320in}}%
\pgfpathlineto{\pgfqpoint{5.684282in}{0.721318in}}%
\pgfpathlineto{\pgfqpoint{5.684579in}{0.721315in}}%
\pgfpathlineto{\pgfqpoint{5.684877in}{0.721312in}}%
\pgfpathlineto{\pgfqpoint{5.685174in}{0.721309in}}%
\pgfpathlineto{\pgfqpoint{5.685472in}{0.721307in}}%
\pgfpathlineto{\pgfqpoint{5.685769in}{0.721304in}}%
\pgfpathlineto{\pgfqpoint{5.686067in}{0.721301in}}%
\pgfpathlineto{\pgfqpoint{5.686364in}{0.721299in}}%
\pgfpathlineto{\pgfqpoint{5.686662in}{0.721296in}}%
\pgfpathlineto{\pgfqpoint{5.686959in}{0.721293in}}%
\pgfpathlineto{\pgfqpoint{5.687257in}{0.721291in}}%
\pgfpathlineto{\pgfqpoint{5.687554in}{0.721288in}}%
\pgfpathlineto{\pgfqpoint{5.687852in}{0.721285in}}%
\pgfpathlineto{\pgfqpoint{5.688149in}{0.721282in}}%
\pgfpathlineto{\pgfqpoint{5.688447in}{0.721280in}}%
\pgfpathlineto{\pgfqpoint{5.688744in}{0.721277in}}%
\pgfpathlineto{\pgfqpoint{5.689042in}{0.721274in}}%
\pgfpathlineto{\pgfqpoint{5.689339in}{0.721272in}}%
\pgfpathlineto{\pgfqpoint{5.689637in}{0.721269in}}%
\pgfpathlineto{\pgfqpoint{5.689934in}{0.721266in}}%
\pgfpathlineto{\pgfqpoint{5.690232in}{0.721264in}}%
\pgfpathlineto{\pgfqpoint{5.690529in}{0.721261in}}%
\pgfpathlineto{\pgfqpoint{5.690827in}{0.721258in}}%
\pgfpathlineto{\pgfqpoint{5.691124in}{0.721255in}}%
\pgfpathlineto{\pgfqpoint{5.691421in}{0.721253in}}%
\pgfpathlineto{\pgfqpoint{5.691719in}{0.721250in}}%
\pgfpathlineto{\pgfqpoint{5.692016in}{0.721247in}}%
\pgfpathlineto{\pgfqpoint{5.692314in}{0.721245in}}%
\pgfpathlineto{\pgfqpoint{5.692611in}{0.721242in}}%
\pgfpathlineto{\pgfqpoint{5.692909in}{0.721239in}}%
\pgfpathlineto{\pgfqpoint{5.693206in}{0.721237in}}%
\pgfpathlineto{\pgfqpoint{5.693504in}{0.721234in}}%
\pgfpathlineto{\pgfqpoint{5.693801in}{0.721231in}}%
\pgfpathlineto{\pgfqpoint{5.694099in}{0.721228in}}%
\pgfpathlineto{\pgfqpoint{5.694396in}{0.721226in}}%
\pgfpathlineto{\pgfqpoint{5.694694in}{0.721223in}}%
\pgfpathlineto{\pgfqpoint{5.694991in}{0.721220in}}%
\pgfpathlineto{\pgfqpoint{5.695289in}{0.721218in}}%
\pgfpathlineto{\pgfqpoint{5.695586in}{0.721215in}}%
\pgfpathlineto{\pgfqpoint{5.695884in}{0.721212in}}%
\pgfpathlineto{\pgfqpoint{5.696181in}{0.721210in}}%
\pgfpathlineto{\pgfqpoint{5.696479in}{0.721207in}}%
\pgfpathlineto{\pgfqpoint{5.696776in}{0.721204in}}%
\pgfpathlineto{\pgfqpoint{5.697074in}{0.721201in}}%
\pgfpathlineto{\pgfqpoint{5.697371in}{0.721199in}}%
\pgfpathlineto{\pgfqpoint{5.697669in}{0.721196in}}%
\pgfpathlineto{\pgfqpoint{5.697966in}{0.721193in}}%
\pgfpathlineto{\pgfqpoint{5.698263in}{0.721191in}}%
\pgfpathlineto{\pgfqpoint{5.698561in}{0.721188in}}%
\pgfpathlineto{\pgfqpoint{5.698858in}{0.721185in}}%
\pgfpathlineto{\pgfqpoint{5.699156in}{0.721183in}}%
\pgfpathlineto{\pgfqpoint{5.699453in}{0.721180in}}%
\pgfpathlineto{\pgfqpoint{5.699751in}{0.721177in}}%
\pgfpathlineto{\pgfqpoint{5.700048in}{0.721174in}}%
\pgfpathlineto{\pgfqpoint{5.700346in}{0.721172in}}%
\pgfpathlineto{\pgfqpoint{5.700643in}{0.721169in}}%
\pgfpathlineto{\pgfqpoint{5.700941in}{0.721166in}}%
\pgfpathlineto{\pgfqpoint{5.701238in}{0.721164in}}%
\pgfpathlineto{\pgfqpoint{5.701536in}{0.721161in}}%
\pgfpathlineto{\pgfqpoint{5.701833in}{0.721158in}}%
\pgfpathlineto{\pgfqpoint{5.702131in}{0.721156in}}%
\pgfpathlineto{\pgfqpoint{5.702428in}{0.721153in}}%
\pgfpathlineto{\pgfqpoint{5.702726in}{0.721150in}}%
\pgfpathlineto{\pgfqpoint{5.703023in}{0.721147in}}%
\pgfpathlineto{\pgfqpoint{5.703321in}{0.721145in}}%
\pgfpathlineto{\pgfqpoint{5.703618in}{0.721142in}}%
\pgfpathlineto{\pgfqpoint{5.703916in}{0.721139in}}%
\pgfpathlineto{\pgfqpoint{5.704213in}{0.721137in}}%
\pgfpathlineto{\pgfqpoint{5.704511in}{0.721134in}}%
\pgfpathlineto{\pgfqpoint{5.704808in}{0.721131in}}%
\pgfpathlineto{\pgfqpoint{5.705105in}{0.721129in}}%
\pgfpathlineto{\pgfqpoint{5.705403in}{0.721126in}}%
\pgfpathlineto{\pgfqpoint{5.705700in}{0.721123in}}%
\pgfpathlineto{\pgfqpoint{5.705998in}{0.721120in}}%
\pgfpathlineto{\pgfqpoint{5.706295in}{0.721118in}}%
\pgfpathlineto{\pgfqpoint{5.706593in}{0.721115in}}%
\pgfpathlineto{\pgfqpoint{5.706890in}{0.721112in}}%
\pgfpathlineto{\pgfqpoint{5.707188in}{0.721110in}}%
\pgfpathlineto{\pgfqpoint{5.707485in}{0.721107in}}%
\pgfpathlineto{\pgfqpoint{5.707783in}{0.721104in}}%
\pgfpathlineto{\pgfqpoint{5.708080in}{0.721102in}}%
\pgfpathlineto{\pgfqpoint{5.708378in}{0.721099in}}%
\pgfpathlineto{\pgfqpoint{5.708675in}{0.721096in}}%
\pgfpathlineto{\pgfqpoint{5.708973in}{0.721093in}}%
\pgfpathlineto{\pgfqpoint{5.709270in}{0.721091in}}%
\pgfpathlineto{\pgfqpoint{5.709568in}{0.721088in}}%
\pgfpathlineto{\pgfqpoint{5.709865in}{0.721085in}}%
\pgfpathlineto{\pgfqpoint{5.710163in}{0.721083in}}%
\pgfpathlineto{\pgfqpoint{5.710460in}{0.721080in}}%
\pgfpathlineto{\pgfqpoint{5.710758in}{0.721077in}}%
\pgfpathlineto{\pgfqpoint{5.711055in}{0.721075in}}%
\pgfpathlineto{\pgfqpoint{5.711352in}{0.721072in}}%
\pgfpathlineto{\pgfqpoint{5.711650in}{0.721069in}}%
\pgfpathlineto{\pgfqpoint{5.711947in}{0.721066in}}%
\pgfpathlineto{\pgfqpoint{5.712245in}{0.721064in}}%
\pgfpathlineto{\pgfqpoint{5.712542in}{0.721061in}}%
\pgfpathlineto{\pgfqpoint{5.712840in}{0.721058in}}%
\pgfpathlineto{\pgfqpoint{5.713137in}{0.721056in}}%
\pgfpathlineto{\pgfqpoint{5.713435in}{0.721053in}}%
\pgfpathlineto{\pgfqpoint{5.713732in}{0.721050in}}%
\pgfpathlineto{\pgfqpoint{5.714030in}{0.721048in}}%
\pgfpathlineto{\pgfqpoint{5.714327in}{0.721045in}}%
\pgfpathlineto{\pgfqpoint{5.714625in}{0.721042in}}%
\pgfpathlineto{\pgfqpoint{5.714922in}{0.721039in}}%
\pgfpathlineto{\pgfqpoint{5.715220in}{0.721037in}}%
\pgfpathlineto{\pgfqpoint{5.715517in}{0.721034in}}%
\pgfpathlineto{\pgfqpoint{5.715815in}{0.721031in}}%
\pgfpathlineto{\pgfqpoint{5.716112in}{0.721029in}}%
\pgfpathlineto{\pgfqpoint{5.716410in}{0.721026in}}%
\pgfpathlineto{\pgfqpoint{5.716707in}{0.721023in}}%
\pgfpathlineto{\pgfqpoint{5.717005in}{0.721021in}}%
\pgfpathlineto{\pgfqpoint{5.717302in}{0.721018in}}%
\pgfpathlineto{\pgfqpoint{5.717600in}{0.721015in}}%
\pgfpathlineto{\pgfqpoint{5.717897in}{0.721012in}}%
\pgfpathlineto{\pgfqpoint{5.718194in}{0.721010in}}%
\pgfpathlineto{\pgfqpoint{5.718492in}{0.721007in}}%
\pgfpathlineto{\pgfqpoint{5.718789in}{0.721004in}}%
\pgfpathlineto{\pgfqpoint{5.719087in}{0.721002in}}%
\pgfpathlineto{\pgfqpoint{5.719384in}{0.720999in}}%
\pgfpathlineto{\pgfqpoint{5.719682in}{0.720996in}}%
\pgfpathlineto{\pgfqpoint{5.719979in}{0.720994in}}%
\pgfpathlineto{\pgfqpoint{5.720277in}{0.720991in}}%
\pgfpathlineto{\pgfqpoint{5.720574in}{0.720988in}}%
\pgfpathlineto{\pgfqpoint{5.720872in}{0.720985in}}%
\pgfpathlineto{\pgfqpoint{5.721169in}{0.720983in}}%
\pgfpathlineto{\pgfqpoint{5.721467in}{0.720980in}}%
\pgfpathlineto{\pgfqpoint{5.721764in}{0.720977in}}%
\pgfpathlineto{\pgfqpoint{5.722062in}{0.720975in}}%
\pgfpathlineto{\pgfqpoint{5.722359in}{0.720972in}}%
\pgfpathlineto{\pgfqpoint{5.722657in}{0.720969in}}%
\pgfpathlineto{\pgfqpoint{5.722954in}{0.720967in}}%
\pgfpathlineto{\pgfqpoint{5.723252in}{0.720964in}}%
\pgfpathlineto{\pgfqpoint{5.723549in}{0.720961in}}%
\pgfpathlineto{\pgfqpoint{5.723847in}{0.720958in}}%
\pgfpathlineto{\pgfqpoint{5.724144in}{0.720956in}}%
\pgfpathlineto{\pgfqpoint{5.724442in}{0.720953in}}%
\pgfpathlineto{\pgfqpoint{5.724739in}{0.720950in}}%
\pgfpathlineto{\pgfqpoint{5.725036in}{0.720948in}}%
\pgfpathlineto{\pgfqpoint{5.725334in}{0.720945in}}%
\pgfpathlineto{\pgfqpoint{5.725631in}{0.720942in}}%
\pgfpathlineto{\pgfqpoint{5.725929in}{0.720940in}}%
\pgfpathlineto{\pgfqpoint{5.726226in}{0.720937in}}%
\pgfpathlineto{\pgfqpoint{5.726524in}{0.720934in}}%
\pgfpathlineto{\pgfqpoint{5.726821in}{0.720931in}}%
\pgfpathlineto{\pgfqpoint{5.727119in}{0.720929in}}%
\pgfpathlineto{\pgfqpoint{5.727416in}{0.720926in}}%
\pgfpathlineto{\pgfqpoint{5.727714in}{0.720923in}}%
\pgfpathlineto{\pgfqpoint{5.728011in}{0.720921in}}%
\pgfpathlineto{\pgfqpoint{5.728309in}{0.720918in}}%
\pgfpathlineto{\pgfqpoint{5.728606in}{0.720915in}}%
\pgfpathlineto{\pgfqpoint{5.728904in}{0.720913in}}%
\pgfpathlineto{\pgfqpoint{5.729201in}{0.720910in}}%
\pgfpathlineto{\pgfqpoint{5.729499in}{0.720907in}}%
\pgfpathlineto{\pgfqpoint{5.729796in}{0.720904in}}%
\pgfpathlineto{\pgfqpoint{5.730094in}{0.720902in}}%
\pgfpathlineto{\pgfqpoint{5.730391in}{0.720899in}}%
\pgfpathlineto{\pgfqpoint{5.730689in}{0.720896in}}%
\pgfpathlineto{\pgfqpoint{5.730986in}{0.720894in}}%
\pgfpathlineto{\pgfqpoint{5.731283in}{0.720891in}}%
\pgfpathlineto{\pgfqpoint{5.731581in}{0.720888in}}%
\pgfpathlineto{\pgfqpoint{5.731878in}{0.720886in}}%
\pgfpathlineto{\pgfqpoint{5.732176in}{0.720883in}}%
\pgfpathlineto{\pgfqpoint{5.732473in}{0.720880in}}%
\pgfpathlineto{\pgfqpoint{5.732771in}{0.720877in}}%
\pgfpathlineto{\pgfqpoint{5.733068in}{0.720875in}}%
\pgfpathlineto{\pgfqpoint{5.733366in}{0.720872in}}%
\pgfpathlineto{\pgfqpoint{5.733663in}{0.720869in}}%
\pgfpathlineto{\pgfqpoint{5.733961in}{0.720867in}}%
\pgfpathlineto{\pgfqpoint{5.734258in}{0.720864in}}%
\pgfpathlineto{\pgfqpoint{5.734556in}{0.720861in}}%
\pgfpathlineto{\pgfqpoint{5.734853in}{0.720859in}}%
\pgfpathlineto{\pgfqpoint{5.735151in}{0.720856in}}%
\pgfpathlineto{\pgfqpoint{5.735448in}{0.720853in}}%
\pgfpathlineto{\pgfqpoint{5.735746in}{0.720850in}}%
\pgfpathlineto{\pgfqpoint{5.736043in}{0.720848in}}%
\pgfpathlineto{\pgfqpoint{5.736341in}{0.720845in}}%
\pgfpathlineto{\pgfqpoint{5.736638in}{0.720842in}}%
\pgfpathlineto{\pgfqpoint{5.736936in}{0.720840in}}%
\pgfpathlineto{\pgfqpoint{5.737233in}{0.720837in}}%
\pgfpathlineto{\pgfqpoint{5.737531in}{0.720834in}}%
\pgfpathlineto{\pgfqpoint{5.737828in}{0.720832in}}%
\pgfpathlineto{\pgfqpoint{5.738125in}{0.720829in}}%
\pgfpathlineto{\pgfqpoint{5.738423in}{0.720826in}}%
\pgfpathlineto{\pgfqpoint{5.738720in}{0.720823in}}%
\pgfpathlineto{\pgfqpoint{5.739018in}{0.720821in}}%
\pgfpathlineto{\pgfqpoint{5.739315in}{0.720818in}}%
\pgfpathlineto{\pgfqpoint{5.739613in}{0.720815in}}%
\pgfpathlineto{\pgfqpoint{5.739910in}{0.720813in}}%
\pgfpathlineto{\pgfqpoint{5.740208in}{0.720810in}}%
\pgfpathlineto{\pgfqpoint{5.740505in}{0.720807in}}%
\pgfpathlineto{\pgfqpoint{5.740803in}{0.720805in}}%
\pgfpathlineto{\pgfqpoint{5.741100in}{0.720802in}}%
\pgfpathlineto{\pgfqpoint{5.741398in}{0.720799in}}%
\pgfpathlineto{\pgfqpoint{5.741695in}{0.720796in}}%
\pgfpathlineto{\pgfqpoint{5.741993in}{0.720794in}}%
\pgfpathlineto{\pgfqpoint{5.742290in}{0.720791in}}%
\pgfpathlineto{\pgfqpoint{5.742588in}{0.720788in}}%
\pgfpathlineto{\pgfqpoint{5.742885in}{0.720786in}}%
\pgfpathlineto{\pgfqpoint{5.743183in}{0.720783in}}%
\pgfpathlineto{\pgfqpoint{5.743480in}{0.720780in}}%
\pgfpathlineto{\pgfqpoint{5.743778in}{0.720778in}}%
\pgfpathlineto{\pgfqpoint{5.744075in}{0.720775in}}%
\pgfpathlineto{\pgfqpoint{5.744373in}{0.720772in}}%
\pgfpathlineto{\pgfqpoint{5.744670in}{0.720769in}}%
\pgfpathlineto{\pgfqpoint{5.744967in}{0.720767in}}%
\pgfpathlineto{\pgfqpoint{5.745265in}{0.720764in}}%
\pgfpathlineto{\pgfqpoint{5.745562in}{0.720761in}}%
\pgfpathlineto{\pgfqpoint{5.745860in}{0.720759in}}%
\pgfpathlineto{\pgfqpoint{5.746157in}{0.720756in}}%
\pgfpathlineto{\pgfqpoint{5.746455in}{0.720753in}}%
\pgfpathlineto{\pgfqpoint{5.746752in}{0.720751in}}%
\pgfpathlineto{\pgfqpoint{5.747050in}{0.720748in}}%
\pgfpathlineto{\pgfqpoint{5.747347in}{0.720745in}}%
\pgfpathlineto{\pgfqpoint{5.747645in}{0.720742in}}%
\pgfpathlineto{\pgfqpoint{5.747942in}{0.720740in}}%
\pgfpathlineto{\pgfqpoint{5.748240in}{0.720737in}}%
\pgfpathlineto{\pgfqpoint{5.748537in}{0.720734in}}%
\pgfpathlineto{\pgfqpoint{5.748835in}{0.720732in}}%
\pgfpathlineto{\pgfqpoint{5.749132in}{0.720729in}}%
\pgfpathlineto{\pgfqpoint{5.749430in}{0.720726in}}%
\pgfpathlineto{\pgfqpoint{5.749727in}{0.720724in}}%
\pgfpathlineto{\pgfqpoint{5.750025in}{0.720721in}}%
\pgfpathlineto{\pgfqpoint{5.750322in}{0.720718in}}%
\pgfpathlineto{\pgfqpoint{5.750620in}{0.720715in}}%
\pgfpathlineto{\pgfqpoint{5.750917in}{0.720713in}}%
\pgfpathlineto{\pgfqpoint{5.751214in}{0.720710in}}%
\pgfpathlineto{\pgfqpoint{5.751512in}{0.720707in}}%
\pgfpathlineto{\pgfqpoint{5.751809in}{0.720705in}}%
\pgfpathlineto{\pgfqpoint{5.752107in}{0.720702in}}%
\pgfpathlineto{\pgfqpoint{5.752404in}{0.720699in}}%
\pgfpathlineto{\pgfqpoint{5.752702in}{0.720697in}}%
\pgfpathlineto{\pgfqpoint{5.752999in}{0.720694in}}%
\pgfpathlineto{\pgfqpoint{5.753297in}{0.720691in}}%
\pgfpathlineto{\pgfqpoint{5.753594in}{0.720688in}}%
\pgfpathlineto{\pgfqpoint{5.753892in}{0.720686in}}%
\pgfpathlineto{\pgfqpoint{5.754189in}{0.720683in}}%
\pgfpathlineto{\pgfqpoint{5.754487in}{0.720680in}}%
\pgfpathlineto{\pgfqpoint{5.754784in}{0.720678in}}%
\pgfpathlineto{\pgfqpoint{5.755082in}{0.720675in}}%
\pgfpathlineto{\pgfqpoint{5.755379in}{0.720672in}}%
\pgfpathlineto{\pgfqpoint{5.755677in}{0.720670in}}%
\pgfpathlineto{\pgfqpoint{5.755974in}{0.720667in}}%
\pgfpathlineto{\pgfqpoint{5.756272in}{0.720664in}}%
\pgfpathlineto{\pgfqpoint{5.756569in}{0.720661in}}%
\pgfpathlineto{\pgfqpoint{5.756867in}{0.720659in}}%
\pgfpathlineto{\pgfqpoint{5.757164in}{0.720656in}}%
\pgfpathlineto{\pgfqpoint{5.757462in}{0.720653in}}%
\pgfpathlineto{\pgfqpoint{5.757759in}{0.720651in}}%
\pgfpathlineto{\pgfqpoint{5.758056in}{0.720648in}}%
\pgfpathlineto{\pgfqpoint{5.758354in}{0.720645in}}%
\pgfpathlineto{\pgfqpoint{5.758651in}{0.720643in}}%
\pgfpathlineto{\pgfqpoint{5.758949in}{0.720640in}}%
\pgfpathlineto{\pgfqpoint{5.759246in}{0.720637in}}%
\pgfpathlineto{\pgfqpoint{5.759544in}{0.720634in}}%
\pgfpathlineto{\pgfqpoint{5.759841in}{0.720632in}}%
\pgfpathlineto{\pgfqpoint{5.760139in}{0.720629in}}%
\pgfpathlineto{\pgfqpoint{5.760436in}{0.720626in}}%
\pgfpathlineto{\pgfqpoint{5.760734in}{0.720624in}}%
\pgfpathlineto{\pgfqpoint{5.761031in}{0.720621in}}%
\pgfpathlineto{\pgfqpoint{5.761329in}{0.720618in}}%
\pgfpathlineto{\pgfqpoint{5.761626in}{0.720616in}}%
\pgfpathlineto{\pgfqpoint{5.761924in}{0.720613in}}%
\pgfpathlineto{\pgfqpoint{5.762221in}{0.720610in}}%
\pgfpathlineto{\pgfqpoint{5.762519in}{0.720607in}}%
\pgfpathlineto{\pgfqpoint{5.762816in}{0.720605in}}%
\pgfpathlineto{\pgfqpoint{5.763114in}{0.720602in}}%
\pgfpathlineto{\pgfqpoint{5.763411in}{0.720599in}}%
\pgfpathlineto{\pgfqpoint{5.763709in}{0.720597in}}%
\pgfpathlineto{\pgfqpoint{5.764006in}{0.720594in}}%
\pgfpathlineto{\pgfqpoint{5.764304in}{0.720591in}}%
\pgfpathlineto{\pgfqpoint{5.764601in}{0.720589in}}%
\pgfpathlineto{\pgfqpoint{5.764898in}{0.720586in}}%
\pgfpathlineto{\pgfqpoint{5.765196in}{0.720583in}}%
\pgfpathlineto{\pgfqpoint{5.765493in}{0.720580in}}%
\pgfpathlineto{\pgfqpoint{5.765791in}{0.720578in}}%
\pgfpathlineto{\pgfqpoint{5.766088in}{0.720575in}}%
\pgfpathlineto{\pgfqpoint{5.766386in}{0.720572in}}%
\pgfpathlineto{\pgfqpoint{5.766683in}{0.720570in}}%
\pgfpathlineto{\pgfqpoint{5.766981in}{0.720567in}}%
\pgfpathlineto{\pgfqpoint{5.767278in}{0.720564in}}%
\pgfpathlineto{\pgfqpoint{5.767576in}{0.720562in}}%
\pgfpathlineto{\pgfqpoint{5.767873in}{0.720559in}}%
\pgfpathlineto{\pgfqpoint{5.768171in}{0.720556in}}%
\pgfpathlineto{\pgfqpoint{5.768468in}{0.720553in}}%
\pgfpathlineto{\pgfqpoint{5.768766in}{0.720551in}}%
\pgfpathlineto{\pgfqpoint{5.769063in}{0.720548in}}%
\pgfpathlineto{\pgfqpoint{5.769361in}{0.720545in}}%
\pgfpathlineto{\pgfqpoint{5.769658in}{0.720543in}}%
\pgfpathlineto{\pgfqpoint{5.769956in}{0.720540in}}%
\pgfpathlineto{\pgfqpoint{5.770253in}{0.720537in}}%
\pgfpathlineto{\pgfqpoint{5.770551in}{0.720535in}}%
\pgfpathlineto{\pgfqpoint{5.770848in}{0.720532in}}%
\pgfpathlineto{\pgfqpoint{5.771145in}{0.720529in}}%
\pgfpathlineto{\pgfqpoint{5.771443in}{0.720526in}}%
\pgfpathlineto{\pgfqpoint{5.771740in}{0.720524in}}%
\pgfpathlineto{\pgfqpoint{5.772038in}{0.720521in}}%
\pgfpathlineto{\pgfqpoint{5.772335in}{0.720518in}}%
\pgfpathlineto{\pgfqpoint{5.772633in}{0.720516in}}%
\pgfpathlineto{\pgfqpoint{5.772930in}{0.720513in}}%
\pgfpathlineto{\pgfqpoint{5.773228in}{0.720510in}}%
\pgfpathlineto{\pgfqpoint{5.773525in}{0.720508in}}%
\pgfpathlineto{\pgfqpoint{5.773823in}{0.720505in}}%
\pgfpathlineto{\pgfqpoint{5.774120in}{0.720502in}}%
\pgfpathlineto{\pgfqpoint{5.774418in}{0.720499in}}%
\pgfpathlineto{\pgfqpoint{5.774715in}{0.720497in}}%
\pgfpathlineto{\pgfqpoint{5.775013in}{0.720494in}}%
\pgfpathlineto{\pgfqpoint{5.775310in}{0.720491in}}%
\pgfpathlineto{\pgfqpoint{5.775608in}{0.720489in}}%
\pgfpathlineto{\pgfqpoint{5.775905in}{0.720486in}}%
\pgfpathlineto{\pgfqpoint{5.776203in}{0.720483in}}%
\pgfpathlineto{\pgfqpoint{5.776500in}{0.720481in}}%
\pgfpathlineto{\pgfqpoint{5.776798in}{0.720478in}}%
\pgfpathlineto{\pgfqpoint{5.777095in}{0.720475in}}%
\pgfpathlineto{\pgfqpoint{5.777393in}{0.720472in}}%
\pgfpathlineto{\pgfqpoint{5.777690in}{0.720470in}}%
\pgfpathlineto{\pgfqpoint{5.777987in}{0.720467in}}%
\pgfpathlineto{\pgfqpoint{5.778285in}{0.720464in}}%
\pgfpathlineto{\pgfqpoint{5.778582in}{0.720462in}}%
\pgfpathlineto{\pgfqpoint{5.778880in}{0.720459in}}%
\pgfpathlineto{\pgfqpoint{5.779177in}{0.720456in}}%
\pgfpathlineto{\pgfqpoint{5.779475in}{0.720454in}}%
\pgfpathlineto{\pgfqpoint{5.779772in}{0.720451in}}%
\pgfpathlineto{\pgfqpoint{5.780070in}{0.720448in}}%
\pgfpathlineto{\pgfqpoint{5.780367in}{0.720445in}}%
\pgfpathlineto{\pgfqpoint{5.780665in}{0.720443in}}%
\pgfpathlineto{\pgfqpoint{5.780962in}{0.720440in}}%
\pgfpathlineto{\pgfqpoint{5.781260in}{0.720437in}}%
\pgfpathlineto{\pgfqpoint{5.781557in}{0.720435in}}%
\pgfpathlineto{\pgfqpoint{5.781855in}{0.720432in}}%
\pgfpathlineto{\pgfqpoint{5.782152in}{0.720429in}}%
\pgfpathlineto{\pgfqpoint{5.782450in}{0.720427in}}%
\pgfpathlineto{\pgfqpoint{5.782747in}{0.720424in}}%
\pgfpathlineto{\pgfqpoint{5.783045in}{0.720421in}}%
\pgfpathlineto{\pgfqpoint{5.783342in}{0.720418in}}%
\pgfpathlineto{\pgfqpoint{5.783640in}{0.720416in}}%
\pgfpathlineto{\pgfqpoint{5.783937in}{0.720413in}}%
\pgfpathlineto{\pgfqpoint{5.784235in}{0.720410in}}%
\pgfpathlineto{\pgfqpoint{5.784532in}{0.720408in}}%
\pgfpathlineto{\pgfqpoint{5.784829in}{0.720405in}}%
\pgfpathlineto{\pgfqpoint{5.785127in}{0.720402in}}%
\pgfpathlineto{\pgfqpoint{5.785424in}{0.720400in}}%
\pgfpathlineto{\pgfqpoint{5.785722in}{0.720397in}}%
\pgfpathlineto{\pgfqpoint{5.786019in}{0.720394in}}%
\pgfpathlineto{\pgfqpoint{5.786317in}{0.720391in}}%
\pgfpathlineto{\pgfqpoint{5.786614in}{0.720389in}}%
\pgfpathlineto{\pgfqpoint{5.786912in}{0.720386in}}%
\pgfpathlineto{\pgfqpoint{5.787209in}{0.720383in}}%
\pgfpathlineto{\pgfqpoint{5.787507in}{0.720381in}}%
\pgfpathlineto{\pgfqpoint{5.787804in}{0.720378in}}%
\pgfpathlineto{\pgfqpoint{5.788102in}{0.720375in}}%
\pgfpathlineto{\pgfqpoint{5.788399in}{0.720373in}}%
\pgfpathlineto{\pgfqpoint{5.788697in}{0.720370in}}%
\pgfpathlineto{\pgfqpoint{5.788994in}{0.720367in}}%
\pgfpathlineto{\pgfqpoint{5.789292in}{0.720364in}}%
\pgfpathlineto{\pgfqpoint{5.789589in}{0.720362in}}%
\pgfpathlineto{\pgfqpoint{5.789887in}{0.720359in}}%
\pgfpathlineto{\pgfqpoint{5.790184in}{0.720356in}}%
\pgfpathlineto{\pgfqpoint{5.790482in}{0.720354in}}%
\pgfpathlineto{\pgfqpoint{5.790779in}{0.720351in}}%
\pgfpathlineto{\pgfqpoint{5.791076in}{0.720348in}}%
\pgfpathlineto{\pgfqpoint{5.791374in}{0.720346in}}%
\pgfpathlineto{\pgfqpoint{5.791671in}{0.720343in}}%
\pgfpathlineto{\pgfqpoint{5.791969in}{0.720340in}}%
\pgfpathlineto{\pgfqpoint{5.792266in}{0.720337in}}%
\pgfpathlineto{\pgfqpoint{5.792564in}{0.720335in}}%
\pgfpathlineto{\pgfqpoint{5.792861in}{0.720332in}}%
\pgfpathlineto{\pgfqpoint{5.793159in}{0.720329in}}%
\pgfpathlineto{\pgfqpoint{5.793456in}{0.720327in}}%
\pgfpathlineto{\pgfqpoint{5.793754in}{0.720324in}}%
\pgfpathlineto{\pgfqpoint{5.794051in}{0.720321in}}%
\pgfpathlineto{\pgfqpoint{5.794349in}{0.720319in}}%
\pgfpathlineto{\pgfqpoint{5.794646in}{0.720316in}}%
\pgfpathlineto{\pgfqpoint{5.794944in}{0.720313in}}%
\pgfpathlineto{\pgfqpoint{5.795241in}{0.720310in}}%
\pgfpathlineto{\pgfqpoint{5.795539in}{0.720308in}}%
\pgfpathlineto{\pgfqpoint{5.795836in}{0.720305in}}%
\pgfpathlineto{\pgfqpoint{5.796134in}{0.720302in}}%
\pgfpathlineto{\pgfqpoint{5.796431in}{0.720300in}}%
\pgfpathlineto{\pgfqpoint{5.796729in}{0.720297in}}%
\pgfpathlineto{\pgfqpoint{5.797026in}{0.720294in}}%
\pgfpathlineto{\pgfqpoint{5.797324in}{0.720292in}}%
\pgfpathlineto{\pgfqpoint{5.797621in}{0.720289in}}%
\pgfpathlineto{\pgfqpoint{5.797918in}{0.720287in}}%
\pgfpathlineto{\pgfqpoint{5.798216in}{0.720295in}}%
\pgfpathlineto{\pgfqpoint{5.798513in}{0.720306in}}%
\pgfpathlineto{\pgfqpoint{5.798811in}{0.720318in}}%
\pgfpathlineto{\pgfqpoint{5.799108in}{0.720330in}}%
\pgfpathlineto{\pgfqpoint{5.799406in}{0.720341in}}%
\pgfpathlineto{\pgfqpoint{5.799703in}{0.720353in}}%
\pgfpathlineto{\pgfqpoint{5.800001in}{0.720364in}}%
\pgfpathlineto{\pgfqpoint{5.800298in}{0.720376in}}%
\pgfpathlineto{\pgfqpoint{5.800596in}{0.720388in}}%
\pgfpathlineto{\pgfqpoint{5.800893in}{0.720399in}}%
\pgfpathlineto{\pgfqpoint{5.801191in}{0.720411in}}%
\pgfpathlineto{\pgfqpoint{5.801488in}{0.720422in}}%
\pgfpathlineto{\pgfqpoint{5.801786in}{0.720434in}}%
\pgfpathlineto{\pgfqpoint{5.802083in}{0.720446in}}%
\pgfpathlineto{\pgfqpoint{5.802381in}{0.720457in}}%
\pgfpathlineto{\pgfqpoint{5.802678in}{0.720469in}}%
\pgfpathlineto{\pgfqpoint{5.802976in}{0.720480in}}%
\pgfpathlineto{\pgfqpoint{5.803273in}{0.720492in}}%
\pgfpathlineto{\pgfqpoint{5.803571in}{0.720504in}}%
\pgfpathlineto{\pgfqpoint{5.803868in}{0.720515in}}%
\pgfpathlineto{\pgfqpoint{5.804166in}{0.720527in}}%
\pgfpathlineto{\pgfqpoint{5.804463in}{0.720538in}}%
\pgfpathlineto{\pgfqpoint{5.804760in}{0.720550in}}%
\pgfpathlineto{\pgfqpoint{5.805058in}{0.720562in}}%
\pgfpathlineto{\pgfqpoint{5.805355in}{0.720573in}}%
\pgfpathlineto{\pgfqpoint{5.805653in}{0.720585in}}%
\pgfpathlineto{\pgfqpoint{5.805950in}{0.720596in}}%
\pgfpathlineto{\pgfqpoint{5.806248in}{0.720608in}}%
\pgfpathlineto{\pgfqpoint{5.806545in}{0.720620in}}%
\pgfpathlineto{\pgfqpoint{5.806843in}{0.720631in}}%
\pgfpathlineto{\pgfqpoint{5.807140in}{0.720643in}}%
\pgfpathlineto{\pgfqpoint{5.807438in}{0.720654in}}%
\pgfpathlineto{\pgfqpoint{5.807735in}{0.720666in}}%
\pgfpathlineto{\pgfqpoint{5.808033in}{0.720678in}}%
\pgfpathlineto{\pgfqpoint{5.808330in}{0.720689in}}%
\pgfpathlineto{\pgfqpoint{5.808628in}{0.720701in}}%
\pgfpathlineto{\pgfqpoint{5.808925in}{0.720712in}}%
\pgfpathlineto{\pgfqpoint{5.809223in}{0.720724in}}%
\pgfpathlineto{\pgfqpoint{5.809520in}{0.720736in}}%
\pgfpathlineto{\pgfqpoint{5.809818in}{0.720747in}}%
\pgfpathlineto{\pgfqpoint{5.810115in}{0.720759in}}%
\pgfpathlineto{\pgfqpoint{5.810413in}{0.720770in}}%
\pgfpathlineto{\pgfqpoint{5.810710in}{0.720782in}}%
\pgfpathlineto{\pgfqpoint{5.811007in}{0.720794in}}%
\pgfpathlineto{\pgfqpoint{5.811305in}{0.720805in}}%
\pgfpathlineto{\pgfqpoint{5.811602in}{0.720817in}}%
\pgfpathlineto{\pgfqpoint{5.811900in}{0.720828in}}%
\pgfpathlineto{\pgfqpoint{5.812197in}{0.720840in}}%
\pgfpathlineto{\pgfqpoint{5.812495in}{0.720852in}}%
\pgfpathlineto{\pgfqpoint{5.812792in}{0.720863in}}%
\pgfpathlineto{\pgfqpoint{5.813090in}{0.720875in}}%
\pgfpathlineto{\pgfqpoint{5.813387in}{0.720886in}}%
\pgfpathlineto{\pgfqpoint{5.813685in}{0.720898in}}%
\pgfpathlineto{\pgfqpoint{5.813982in}{0.720910in}}%
\pgfpathlineto{\pgfqpoint{5.814280in}{0.720921in}}%
\pgfpathlineto{\pgfqpoint{5.814577in}{0.720933in}}%
\pgfpathlineto{\pgfqpoint{5.814875in}{0.720944in}}%
\pgfpathlineto{\pgfqpoint{5.815172in}{0.720956in}}%
\pgfpathlineto{\pgfqpoint{5.815470in}{0.720968in}}%
\pgfpathlineto{\pgfqpoint{5.815767in}{0.720979in}}%
\pgfpathlineto{\pgfqpoint{5.816065in}{0.720991in}}%
\pgfpathlineto{\pgfqpoint{5.816362in}{0.721002in}}%
\pgfpathlineto{\pgfqpoint{5.816660in}{0.721014in}}%
\pgfpathlineto{\pgfqpoint{5.816957in}{0.721026in}}%
\pgfpathlineto{\pgfqpoint{5.817255in}{0.721037in}}%
\pgfpathlineto{\pgfqpoint{5.817552in}{0.721049in}}%
\pgfpathlineto{\pgfqpoint{5.817849in}{0.721060in}}%
\pgfpathlineto{\pgfqpoint{5.818147in}{0.721072in}}%
\pgfpathlineto{\pgfqpoint{5.818444in}{0.721084in}}%
\pgfpathlineto{\pgfqpoint{5.818742in}{0.721095in}}%
\pgfpathlineto{\pgfqpoint{5.819039in}{0.721107in}}%
\pgfpathlineto{\pgfqpoint{5.819337in}{0.721118in}}%
\pgfpathlineto{\pgfqpoint{5.819634in}{0.721130in}}%
\pgfpathlineto{\pgfqpoint{5.819932in}{0.721142in}}%
\pgfpathlineto{\pgfqpoint{5.820229in}{0.721153in}}%
\pgfpathlineto{\pgfqpoint{5.820527in}{0.721165in}}%
\pgfpathlineto{\pgfqpoint{5.820824in}{0.721176in}}%
\pgfpathlineto{\pgfqpoint{5.821122in}{0.721188in}}%
\pgfpathlineto{\pgfqpoint{5.821419in}{0.721200in}}%
\pgfpathlineto{\pgfqpoint{5.821717in}{0.721211in}}%
\pgfpathlineto{\pgfqpoint{5.822014in}{0.721223in}}%
\pgfpathlineto{\pgfqpoint{5.822312in}{0.721234in}}%
\pgfpathlineto{\pgfqpoint{5.822609in}{0.721246in}}%
\pgfpathlineto{\pgfqpoint{5.822907in}{0.721258in}}%
\pgfpathlineto{\pgfqpoint{5.823204in}{0.721269in}}%
\pgfpathlineto{\pgfqpoint{5.823502in}{0.721281in}}%
\pgfpathlineto{\pgfqpoint{5.823799in}{0.721292in}}%
\pgfpathlineto{\pgfqpoint{5.824097in}{0.721304in}}%
\pgfpathlineto{\pgfqpoint{5.824394in}{0.721316in}}%
\pgfpathlineto{\pgfqpoint{5.824691in}{0.721327in}}%
\pgfpathlineto{\pgfqpoint{5.824989in}{0.721339in}}%
\pgfpathlineto{\pgfqpoint{5.825286in}{0.721350in}}%
\pgfpathlineto{\pgfqpoint{5.825584in}{0.721362in}}%
\pgfpathlineto{\pgfqpoint{5.825881in}{0.721374in}}%
\pgfpathlineto{\pgfqpoint{5.826179in}{0.721495in}}%
\pgfpathlineto{\pgfqpoint{5.826476in}{0.722109in}}%
\pgfpathlineto{\pgfqpoint{5.826774in}{0.722782in}}%
\pgfpathlineto{\pgfqpoint{5.827071in}{0.723456in}}%
\pgfpathlineto{\pgfqpoint{5.827369in}{0.724130in}}%
\pgfpathlineto{\pgfqpoint{5.827666in}{0.724803in}}%
\pgfpathlineto{\pgfqpoint{5.827964in}{0.725477in}}%
\pgfpathlineto{\pgfqpoint{5.828261in}{0.726151in}}%
\pgfpathlineto{\pgfqpoint{5.828559in}{0.726825in}}%
\pgfpathlineto{\pgfqpoint{5.828856in}{0.727498in}}%
\pgfpathlineto{\pgfqpoint{5.829154in}{0.728172in}}%
\pgfpathlineto{\pgfqpoint{5.829451in}{0.728846in}}%
\pgfpathlineto{\pgfqpoint{5.829749in}{0.729519in}}%
\pgfpathlineto{\pgfqpoint{5.830046in}{0.730193in}}%
\pgfpathlineto{\pgfqpoint{5.830344in}{0.730867in}}%
\pgfpathlineto{\pgfqpoint{5.830641in}{0.731541in}}%
\pgfpathlineto{\pgfqpoint{5.830938in}{0.732214in}}%
\pgfpathlineto{\pgfqpoint{5.831236in}{0.732888in}}%
\pgfpathlineto{\pgfqpoint{5.831533in}{0.733562in}}%
\pgfpathlineto{\pgfqpoint{5.831831in}{0.734236in}}%
\pgfpathlineto{\pgfqpoint{5.832128in}{0.734909in}}%
\pgfpathlineto{\pgfqpoint{5.832426in}{0.735583in}}%
\pgfpathlineto{\pgfqpoint{5.832723in}{0.736257in}}%
\pgfpathlineto{\pgfqpoint{5.833021in}{0.736930in}}%
\pgfpathlineto{\pgfqpoint{5.833318in}{0.736536in}}%
\pgfpathlineto{\pgfqpoint{5.833616in}{0.734847in}}%
\pgfpathlineto{\pgfqpoint{5.833913in}{0.733177in}}%
\pgfpathlineto{\pgfqpoint{5.834211in}{0.731506in}}%
\pgfpathlineto{\pgfqpoint{5.834508in}{0.729836in}}%
\pgfpathlineto{\pgfqpoint{5.834806in}{0.728165in}}%
\pgfpathlineto{\pgfqpoint{5.835103in}{0.726495in}}%
\pgfpathlineto{\pgfqpoint{5.835401in}{0.724824in}}%
\pgfpathlineto{\pgfqpoint{5.835698in}{0.723154in}}%
\pgfpathlineto{\pgfqpoint{5.835996in}{0.721775in}}%
\pgfpathlineto{\pgfqpoint{5.836293in}{0.721636in}}%
\pgfpathlineto{\pgfqpoint{5.836591in}{0.721636in}}%
\pgfpathlineto{\pgfqpoint{5.836888in}{0.721635in}}%
\pgfpathlineto{\pgfqpoint{5.837186in}{0.721635in}}%
\pgfpathlineto{\pgfqpoint{5.837483in}{0.721635in}}%
\pgfpathlineto{\pgfqpoint{5.837780in}{0.721634in}}%
\pgfpathlineto{\pgfqpoint{5.838078in}{0.721634in}}%
\pgfpathlineto{\pgfqpoint{5.838375in}{0.721634in}}%
\pgfpathlineto{\pgfqpoint{5.838673in}{0.721633in}}%
\pgfpathlineto{\pgfqpoint{5.838970in}{0.721633in}}%
\pgfpathlineto{\pgfqpoint{5.839268in}{0.721633in}}%
\pgfpathlineto{\pgfqpoint{5.839565in}{0.721632in}}%
\pgfpathlineto{\pgfqpoint{5.839863in}{0.721632in}}%
\pgfpathlineto{\pgfqpoint{5.840160in}{0.721556in}}%
\pgfpathlineto{\pgfqpoint{5.840458in}{0.721095in}}%
\pgfpathlineto{\pgfqpoint{5.840755in}{0.720580in}}%
\pgfpathlineto{\pgfqpoint{5.841053in}{0.720064in}}%
\pgfpathlineto{\pgfqpoint{5.841350in}{0.719549in}}%
\pgfpathlineto{\pgfqpoint{5.841648in}{0.719034in}}%
\pgfpathlineto{\pgfqpoint{5.841945in}{0.718519in}}%
\pgfpathlineto{\pgfqpoint{5.842243in}{0.718003in}}%
\pgfpathlineto{\pgfqpoint{5.842540in}{0.717488in}}%
\pgfpathlineto{\pgfqpoint{5.842838in}{0.716973in}}%
\pgfpathlineto{\pgfqpoint{5.843135in}{0.716458in}}%
\pgfpathlineto{\pgfqpoint{5.843433in}{0.715942in}}%
\pgfpathlineto{\pgfqpoint{5.843730in}{0.715427in}}%
\pgfpathlineto{\pgfqpoint{5.844028in}{0.714912in}}%
\pgfpathlineto{\pgfqpoint{5.844325in}{0.714397in}}%
\pgfpathlineto{\pgfqpoint{5.844622in}{0.713881in}}%
\pgfpathlineto{\pgfqpoint{5.844920in}{0.713366in}}%
\pgfpathlineto{\pgfqpoint{5.845217in}{0.712851in}}%
\pgfpathlineto{\pgfqpoint{5.845515in}{0.712336in}}%
\pgfpathlineto{\pgfqpoint{5.845812in}{0.711820in}}%
\pgfpathlineto{\pgfqpoint{5.846110in}{0.711305in}}%
\pgfpathlineto{\pgfqpoint{5.846407in}{0.710790in}}%
\pgfpathlineto{\pgfqpoint{5.846705in}{0.710275in}}%
\pgfpathlineto{\pgfqpoint{5.847002in}{0.709759in}}%
\pgfpathlineto{\pgfqpoint{5.847300in}{0.709244in}}%
\pgfpathlineto{\pgfqpoint{5.847597in}{0.708729in}}%
\pgfpathlineto{\pgfqpoint{5.847895in}{0.712097in}}%
\pgfpathlineto{\pgfqpoint{5.848192in}{0.721711in}}%
\pgfpathlineto{\pgfqpoint{5.848490in}{0.721716in}}%
\pgfpathlineto{\pgfqpoint{5.848787in}{0.721721in}}%
\pgfpathlineto{\pgfqpoint{5.849085in}{0.721726in}}%
\pgfpathlineto{\pgfqpoint{5.849382in}{0.721730in}}%
\pgfpathlineto{\pgfqpoint{5.849680in}{0.721735in}}%
\pgfpathlineto{\pgfqpoint{5.849977in}{0.721740in}}%
\pgfpathlineto{\pgfqpoint{5.850275in}{0.721745in}}%
\pgfpathlineto{\pgfqpoint{5.850572in}{0.721749in}}%
\pgfpathlineto{\pgfqpoint{5.850869in}{0.721754in}}%
\pgfpathlineto{\pgfqpoint{5.851167in}{0.721759in}}%
\pgfpathlineto{\pgfqpoint{5.851464in}{0.721764in}}%
\pgfpathlineto{\pgfqpoint{5.851762in}{0.721768in}}%
\pgfpathlineto{\pgfqpoint{5.852059in}{0.721773in}}%
\pgfpathlineto{\pgfqpoint{5.852357in}{0.721778in}}%
\pgfpathlineto{\pgfqpoint{5.852654in}{0.721783in}}%
\pgfpathlineto{\pgfqpoint{5.852952in}{0.721787in}}%
\pgfpathlineto{\pgfqpoint{5.853249in}{0.721792in}}%
\pgfpathlineto{\pgfqpoint{5.853547in}{0.721797in}}%
\pgfpathlineto{\pgfqpoint{5.853844in}{0.721802in}}%
\pgfpathlineto{\pgfqpoint{5.854142in}{0.721820in}}%
\pgfpathlineto{\pgfqpoint{5.854439in}{0.721843in}}%
\pgfpathlineto{\pgfqpoint{5.854737in}{0.721866in}}%
\pgfpathlineto{\pgfqpoint{5.855034in}{0.721883in}}%
\pgfpathlineto{\pgfqpoint{5.855332in}{0.721884in}}%
\pgfpathlineto{\pgfqpoint{5.855629in}{0.721884in}}%
\pgfpathlineto{\pgfqpoint{5.855927in}{0.721884in}}%
\pgfpathlineto{\pgfqpoint{5.856224in}{0.721884in}}%
\pgfpathlineto{\pgfqpoint{5.856522in}{0.721884in}}%
\pgfpathlineto{\pgfqpoint{5.856819in}{0.721884in}}%
\pgfpathlineto{\pgfqpoint{5.857117in}{0.721884in}}%
\pgfpathlineto{\pgfqpoint{5.857414in}{0.721884in}}%
\pgfpathlineto{\pgfqpoint{5.857711in}{0.721883in}}%
\pgfpathlineto{\pgfqpoint{5.858009in}{0.721883in}}%
\pgfpathlineto{\pgfqpoint{5.858306in}{0.721883in}}%
\pgfpathlineto{\pgfqpoint{5.858604in}{0.721883in}}%
\pgfpathlineto{\pgfqpoint{5.858901in}{0.721883in}}%
\pgfpathlineto{\pgfqpoint{5.859199in}{0.721883in}}%
\pgfpathlineto{\pgfqpoint{5.859496in}{0.721883in}}%
\pgfpathlineto{\pgfqpoint{5.859794in}{0.721883in}}%
\pgfpathlineto{\pgfqpoint{5.860091in}{0.721883in}}%
\pgfpathlineto{\pgfqpoint{5.860389in}{0.721882in}}%
\pgfpathlineto{\pgfqpoint{5.860686in}{0.721882in}}%
\pgfpathlineto{\pgfqpoint{5.860984in}{0.721882in}}%
\pgfpathlineto{\pgfqpoint{5.861281in}{0.721882in}}%
\pgfpathlineto{\pgfqpoint{5.861579in}{0.721882in}}%
\pgfpathlineto{\pgfqpoint{5.861876in}{0.721882in}}%
\pgfpathlineto{\pgfqpoint{5.862174in}{0.721882in}}%
\pgfpathlineto{\pgfqpoint{5.862471in}{0.721882in}}%
\pgfpathlineto{\pgfqpoint{5.862769in}{0.721882in}}%
\pgfpathlineto{\pgfqpoint{5.863066in}{0.721882in}}%
\pgfpathlineto{\pgfqpoint{5.863364in}{0.721881in}}%
\pgfpathlineto{\pgfqpoint{5.863661in}{0.721881in}}%
\pgfpathlineto{\pgfqpoint{5.863959in}{0.721881in}}%
\pgfpathlineto{\pgfqpoint{5.864256in}{0.721881in}}%
\pgfpathlineto{\pgfqpoint{5.864553in}{0.721881in}}%
\pgfpathlineto{\pgfqpoint{5.864851in}{0.721881in}}%
\pgfpathlineto{\pgfqpoint{5.865148in}{0.721881in}}%
\pgfpathlineto{\pgfqpoint{5.865446in}{0.721881in}}%
\pgfpathlineto{\pgfqpoint{5.865743in}{0.721881in}}%
\pgfpathlineto{\pgfqpoint{5.866041in}{0.721881in}}%
\pgfpathlineto{\pgfqpoint{5.866338in}{0.721880in}}%
\pgfpathlineto{\pgfqpoint{5.866636in}{0.721880in}}%
\pgfpathlineto{\pgfqpoint{5.866933in}{0.721880in}}%
\pgfpathlineto{\pgfqpoint{5.867231in}{0.721880in}}%
\pgfpathlineto{\pgfqpoint{5.867528in}{0.721880in}}%
\pgfpathlineto{\pgfqpoint{5.867826in}{0.721880in}}%
\pgfpathlineto{\pgfqpoint{5.868123in}{0.721880in}}%
\pgfpathlineto{\pgfqpoint{5.868421in}{0.721880in}}%
\pgfpathlineto{\pgfqpoint{5.868718in}{0.721880in}}%
\pgfpathlineto{\pgfqpoint{5.869016in}{0.721880in}}%
\pgfpathlineto{\pgfqpoint{5.869313in}{0.721879in}}%
\pgfpathlineto{\pgfqpoint{5.869611in}{0.721879in}}%
\pgfpathlineto{\pgfqpoint{5.869908in}{0.721879in}}%
\pgfpathlineto{\pgfqpoint{5.870206in}{0.721879in}}%
\pgfpathlineto{\pgfqpoint{5.870503in}{0.721879in}}%
\pgfpathlineto{\pgfqpoint{5.870800in}{0.721879in}}%
\pgfpathlineto{\pgfqpoint{5.871098in}{0.721879in}}%
\pgfpathlineto{\pgfqpoint{5.871395in}{0.721879in}}%
\pgfpathlineto{\pgfqpoint{5.871693in}{0.721879in}}%
\pgfpathlineto{\pgfqpoint{5.871990in}{0.721879in}}%
\pgfpathlineto{\pgfqpoint{5.872288in}{0.721878in}}%
\pgfpathlineto{\pgfqpoint{5.872585in}{0.721878in}}%
\pgfpathlineto{\pgfqpoint{5.872883in}{0.721878in}}%
\pgfpathlineto{\pgfqpoint{5.873180in}{0.721878in}}%
\pgfpathlineto{\pgfqpoint{5.873478in}{0.721878in}}%
\pgfpathlineto{\pgfqpoint{5.873775in}{0.721878in}}%
\pgfpathlineto{\pgfqpoint{5.874073in}{0.721878in}}%
\pgfpathlineto{\pgfqpoint{5.874370in}{0.721878in}}%
\pgfpathlineto{\pgfqpoint{5.874668in}{0.721878in}}%
\pgfpathlineto{\pgfqpoint{5.874965in}{0.721877in}}%
\pgfpathlineto{\pgfqpoint{5.875263in}{0.721877in}}%
\pgfpathlineto{\pgfqpoint{5.875560in}{0.721877in}}%
\pgfpathlineto{\pgfqpoint{5.875858in}{0.721905in}}%
\pgfpathlineto{\pgfqpoint{5.876155in}{0.721955in}}%
\pgfpathlineto{\pgfqpoint{5.876453in}{0.722004in}}%
\pgfpathlineto{\pgfqpoint{5.876750in}{0.722053in}}%
\pgfpathlineto{\pgfqpoint{5.877048in}{0.722102in}}%
\pgfpathlineto{\pgfqpoint{5.877345in}{0.722151in}}%
\pgfpathlineto{\pgfqpoint{5.877642in}{0.722200in}}%
\pgfpathlineto{\pgfqpoint{5.877940in}{0.722249in}}%
\pgfpathlineto{\pgfqpoint{5.878237in}{0.722298in}}%
\pgfpathlineto{\pgfqpoint{5.878535in}{0.722347in}}%
\pgfpathlineto{\pgfqpoint{5.878832in}{0.722396in}}%
\pgfpathlineto{\pgfqpoint{5.879130in}{0.722446in}}%
\pgfpathlineto{\pgfqpoint{5.879427in}{0.722495in}}%
\pgfpathlineto{\pgfqpoint{5.879725in}{0.722544in}}%
\pgfpathlineto{\pgfqpoint{5.880022in}{0.722593in}}%
\pgfpathlineto{\pgfqpoint{5.880320in}{0.722642in}}%
\pgfpathlineto{\pgfqpoint{5.880617in}{0.722691in}}%
\pgfpathlineto{\pgfqpoint{5.880915in}{0.722740in}}%
\pgfpathlineto{\pgfqpoint{5.881212in}{0.722789in}}%
\pgfpathlineto{\pgfqpoint{5.881510in}{0.722838in}}%
\pgfpathlineto{\pgfqpoint{5.881807in}{0.722887in}}%
\pgfpathlineto{\pgfqpoint{5.882105in}{0.722936in}}%
\pgfpathlineto{\pgfqpoint{5.882402in}{0.722986in}}%
\pgfpathlineto{\pgfqpoint{5.882700in}{0.723035in}}%
\pgfpathlineto{\pgfqpoint{5.882997in}{0.723084in}}%
\pgfpathlineto{\pgfqpoint{5.883295in}{0.723133in}}%
\pgfpathlineto{\pgfqpoint{5.883592in}{0.723182in}}%
\pgfpathlineto{\pgfqpoint{5.883890in}{0.723231in}}%
\pgfpathlineto{\pgfqpoint{5.884187in}{0.723280in}}%
\pgfpathlineto{\pgfqpoint{5.884484in}{0.723329in}}%
\pgfpathlineto{\pgfqpoint{5.884782in}{0.723378in}}%
\pgfpathlineto{\pgfqpoint{5.885079in}{0.723427in}}%
\pgfpathlineto{\pgfqpoint{5.885377in}{0.723476in}}%
\pgfpathlineto{\pgfqpoint{5.885674in}{0.723526in}}%
\pgfpathlineto{\pgfqpoint{5.885972in}{0.723575in}}%
\pgfpathlineto{\pgfqpoint{5.886269in}{0.723624in}}%
\pgfpathlineto{\pgfqpoint{5.886567in}{0.723673in}}%
\pgfpathlineto{\pgfqpoint{5.886864in}{0.723722in}}%
\pgfpathlineto{\pgfqpoint{5.887162in}{0.723771in}}%
\pgfpathlineto{\pgfqpoint{5.887459in}{0.723820in}}%
\pgfpathlineto{\pgfqpoint{5.887757in}{0.723869in}}%
\pgfpathlineto{\pgfqpoint{5.888054in}{0.723918in}}%
\pgfpathlineto{\pgfqpoint{5.888352in}{0.723967in}}%
\pgfpathlineto{\pgfqpoint{5.888649in}{0.724017in}}%
\pgfpathlineto{\pgfqpoint{5.888947in}{0.724066in}}%
\pgfpathlineto{\pgfqpoint{5.889244in}{0.724115in}}%
\pgfpathlineto{\pgfqpoint{5.889542in}{0.724164in}}%
\pgfpathlineto{\pgfqpoint{5.889839in}{0.724213in}}%
\pgfpathlineto{\pgfqpoint{5.890137in}{0.724262in}}%
\pgfpathlineto{\pgfqpoint{5.890434in}{0.724311in}}%
\pgfpathlineto{\pgfqpoint{5.890731in}{0.724360in}}%
\pgfpathlineto{\pgfqpoint{5.891029in}{0.724409in}}%
\pgfpathlineto{\pgfqpoint{5.891326in}{0.724458in}}%
\pgfpathlineto{\pgfqpoint{5.891624in}{0.724507in}}%
\pgfpathlineto{\pgfqpoint{5.891921in}{0.724557in}}%
\pgfpathlineto{\pgfqpoint{5.892219in}{0.724606in}}%
\pgfpathlineto{\pgfqpoint{5.892516in}{0.724655in}}%
\pgfpathlineto{\pgfqpoint{5.892814in}{0.724704in}}%
\pgfpathlineto{\pgfqpoint{5.893111in}{0.724753in}}%
\pgfpathlineto{\pgfqpoint{5.893409in}{0.724802in}}%
\pgfpathlineto{\pgfqpoint{5.893706in}{0.724851in}}%
\pgfpathlineto{\pgfqpoint{5.894004in}{0.724900in}}%
\pgfpathlineto{\pgfqpoint{5.894301in}{0.724949in}}%
\pgfpathlineto{\pgfqpoint{5.894599in}{0.724998in}}%
\pgfpathlineto{\pgfqpoint{5.894896in}{0.725048in}}%
\pgfpathlineto{\pgfqpoint{5.895194in}{0.725097in}}%
\pgfpathlineto{\pgfqpoint{5.895491in}{0.725146in}}%
\pgfpathlineto{\pgfqpoint{5.895789in}{0.725195in}}%
\pgfpathlineto{\pgfqpoint{5.896086in}{0.725244in}}%
\pgfpathlineto{\pgfqpoint{5.896384in}{0.725293in}}%
\pgfpathlineto{\pgfqpoint{5.896681in}{0.725342in}}%
\pgfpathlineto{\pgfqpoint{5.896979in}{0.725391in}}%
\pgfpathlineto{\pgfqpoint{5.897276in}{0.725440in}}%
\pgfpathlineto{\pgfqpoint{5.897573in}{0.725489in}}%
\pgfpathlineto{\pgfqpoint{5.897871in}{0.725538in}}%
\pgfpathlineto{\pgfqpoint{5.898168in}{0.725588in}}%
\pgfpathlineto{\pgfqpoint{5.898466in}{0.725637in}}%
\pgfpathlineto{\pgfqpoint{5.898763in}{0.725686in}}%
\pgfpathlineto{\pgfqpoint{5.899061in}{0.725735in}}%
\pgfpathlineto{\pgfqpoint{5.899358in}{0.725784in}}%
\pgfpathlineto{\pgfqpoint{5.899656in}{0.725833in}}%
\pgfpathlineto{\pgfqpoint{5.899953in}{0.725882in}}%
\pgfpathlineto{\pgfqpoint{5.900251in}{0.725931in}}%
\pgfpathlineto{\pgfqpoint{5.900548in}{0.725980in}}%
\pgfpathlineto{\pgfqpoint{5.900846in}{0.726029in}}%
\pgfpathlineto{\pgfqpoint{5.901143in}{0.726079in}}%
\pgfpathlineto{\pgfqpoint{5.901441in}{0.726128in}}%
\pgfpathlineto{\pgfqpoint{5.901738in}{0.726177in}}%
\pgfpathlineto{\pgfqpoint{5.902036in}{0.726226in}}%
\pgfpathlineto{\pgfqpoint{5.902333in}{0.726275in}}%
\pgfpathlineto{\pgfqpoint{5.902631in}{0.726324in}}%
\pgfpathlineto{\pgfqpoint{5.902928in}{0.726373in}}%
\pgfpathlineto{\pgfqpoint{5.903226in}{0.726422in}}%
\pgfpathlineto{\pgfqpoint{5.903523in}{0.726471in}}%
\pgfpathlineto{\pgfqpoint{5.903821in}{0.726520in}}%
\pgfpathlineto{\pgfqpoint{5.904118in}{0.726569in}}%
\pgfpathlineto{\pgfqpoint{5.904415in}{0.726619in}}%
\pgfpathlineto{\pgfqpoint{5.904713in}{0.726668in}}%
\pgfpathlineto{\pgfqpoint{5.905010in}{0.726717in}}%
\pgfpathlineto{\pgfqpoint{5.905308in}{0.726766in}}%
\pgfpathlineto{\pgfqpoint{5.905605in}{0.726815in}}%
\pgfpathlineto{\pgfqpoint{5.905903in}{0.726864in}}%
\pgfpathlineto{\pgfqpoint{5.906200in}{0.726913in}}%
\pgfpathlineto{\pgfqpoint{5.906498in}{0.726962in}}%
\pgfpathlineto{\pgfqpoint{5.906795in}{0.727011in}}%
\pgfpathlineto{\pgfqpoint{5.907093in}{0.727060in}}%
\pgfpathlineto{\pgfqpoint{5.907390in}{0.727110in}}%
\pgfpathlineto{\pgfqpoint{5.907688in}{0.727159in}}%
\pgfpathlineto{\pgfqpoint{5.907985in}{0.727208in}}%
\pgfpathlineto{\pgfqpoint{5.908283in}{0.727257in}}%
\pgfpathlineto{\pgfqpoint{5.908580in}{0.727306in}}%
\pgfpathlineto{\pgfqpoint{5.908878in}{0.727355in}}%
\pgfpathlineto{\pgfqpoint{5.909175in}{0.727404in}}%
\pgfpathlineto{\pgfqpoint{5.909473in}{0.727453in}}%
\pgfpathlineto{\pgfqpoint{5.909770in}{0.727502in}}%
\pgfpathlineto{\pgfqpoint{5.910068in}{0.727551in}}%
\pgfpathlineto{\pgfqpoint{5.910365in}{0.727600in}}%
\pgfpathlineto{\pgfqpoint{5.910663in}{0.727650in}}%
\pgfpathlineto{\pgfqpoint{5.910960in}{0.727699in}}%
\pgfpathlineto{\pgfqpoint{5.911257in}{0.727748in}}%
\pgfpathlineto{\pgfqpoint{5.911555in}{0.727797in}}%
\pgfpathlineto{\pgfqpoint{5.911852in}{0.727846in}}%
\pgfpathlineto{\pgfqpoint{5.912150in}{0.727895in}}%
\pgfpathlineto{\pgfqpoint{5.912447in}{0.727944in}}%
\pgfpathlineto{\pgfqpoint{5.912745in}{0.727993in}}%
\pgfpathlineto{\pgfqpoint{5.913042in}{0.728042in}}%
\pgfpathlineto{\pgfqpoint{5.913340in}{0.728091in}}%
\pgfpathlineto{\pgfqpoint{5.913637in}{0.728140in}}%
\pgfpathlineto{\pgfqpoint{5.913935in}{0.728190in}}%
\pgfpathlineto{\pgfqpoint{5.914232in}{0.728239in}}%
\pgfpathlineto{\pgfqpoint{5.914530in}{0.728288in}}%
\pgfpathlineto{\pgfqpoint{5.914827in}{0.728337in}}%
\pgfpathlineto{\pgfqpoint{5.915125in}{0.728386in}}%
\pgfpathlineto{\pgfqpoint{5.915422in}{0.728435in}}%
\pgfpathlineto{\pgfqpoint{5.915720in}{0.728484in}}%
\pgfpathlineto{\pgfqpoint{5.916017in}{0.728533in}}%
\pgfpathlineto{\pgfqpoint{5.916315in}{0.728582in}}%
\pgfpathlineto{\pgfqpoint{5.916612in}{0.728631in}}%
\pgfpathlineto{\pgfqpoint{5.916910in}{0.728681in}}%
\pgfpathlineto{\pgfqpoint{5.917207in}{0.728730in}}%
\pgfpathlineto{\pgfqpoint{5.917504in}{0.728779in}}%
\pgfpathlineto{\pgfqpoint{5.917802in}{0.728828in}}%
\pgfpathlineto{\pgfqpoint{5.918099in}{0.728877in}}%
\pgfpathlineto{\pgfqpoint{5.918397in}{0.728926in}}%
\pgfpathlineto{\pgfqpoint{5.918694in}{0.728975in}}%
\pgfpathlineto{\pgfqpoint{5.918992in}{0.729024in}}%
\pgfpathlineto{\pgfqpoint{5.919289in}{0.729073in}}%
\pgfpathlineto{\pgfqpoint{5.919587in}{0.729122in}}%
\pgfpathlineto{\pgfqpoint{5.919884in}{0.729171in}}%
\pgfpathlineto{\pgfqpoint{5.920182in}{0.729221in}}%
\pgfpathlineto{\pgfqpoint{5.920479in}{0.729270in}}%
\pgfpathlineto{\pgfqpoint{5.920777in}{0.729319in}}%
\pgfpathlineto{\pgfqpoint{5.921074in}{0.729368in}}%
\pgfpathlineto{\pgfqpoint{5.921372in}{0.729417in}}%
\pgfpathlineto{\pgfqpoint{5.921669in}{0.729466in}}%
\pgfpathlineto{\pgfqpoint{5.921967in}{0.729515in}}%
\pgfpathlineto{\pgfqpoint{5.922264in}{0.729564in}}%
\pgfpathlineto{\pgfqpoint{5.922562in}{0.729613in}}%
\pgfpathlineto{\pgfqpoint{5.922859in}{0.729662in}}%
\pgfpathlineto{\pgfqpoint{5.923157in}{0.729712in}}%
\pgfpathlineto{\pgfqpoint{5.923454in}{0.729761in}}%
\pgfpathlineto{\pgfqpoint{5.923752in}{0.729810in}}%
\pgfpathlineto{\pgfqpoint{5.924049in}{0.729859in}}%
\pgfpathlineto{\pgfqpoint{5.924346in}{0.729908in}}%
\pgfpathlineto{\pgfqpoint{5.924644in}{0.729957in}}%
\pgfpathlineto{\pgfqpoint{5.924941in}{0.730006in}}%
\pgfpathlineto{\pgfqpoint{5.925239in}{0.730055in}}%
\pgfpathlineto{\pgfqpoint{5.925536in}{0.730104in}}%
\pgfpathlineto{\pgfqpoint{5.925834in}{0.730153in}}%
\pgfpathlineto{\pgfqpoint{5.926131in}{0.730202in}}%
\pgfpathlineto{\pgfqpoint{5.926429in}{0.730252in}}%
\pgfpathlineto{\pgfqpoint{5.926726in}{0.730301in}}%
\pgfpathlineto{\pgfqpoint{5.927024in}{0.730350in}}%
\pgfpathlineto{\pgfqpoint{5.927321in}{0.730399in}}%
\pgfpathlineto{\pgfqpoint{5.927619in}{0.730448in}}%
\pgfpathlineto{\pgfqpoint{5.927916in}{0.730497in}}%
\pgfpathlineto{\pgfqpoint{5.928214in}{0.730546in}}%
\pgfpathlineto{\pgfqpoint{5.928511in}{0.730595in}}%
\pgfpathlineto{\pgfqpoint{5.928809in}{0.730644in}}%
\pgfpathlineto{\pgfqpoint{5.929106in}{0.730693in}}%
\pgfpathlineto{\pgfqpoint{5.929404in}{0.730743in}}%
\pgfpathlineto{\pgfqpoint{5.929701in}{0.730792in}}%
\pgfpathlineto{\pgfqpoint{5.929999in}{0.730841in}}%
\pgfpathlineto{\pgfqpoint{5.930296in}{0.730890in}}%
\pgfpathlineto{\pgfqpoint{5.930594in}{0.730939in}}%
\pgfpathlineto{\pgfqpoint{5.930891in}{0.730988in}}%
\pgfpathlineto{\pgfqpoint{5.931188in}{0.731037in}}%
\pgfpathlineto{\pgfqpoint{5.931486in}{0.731086in}}%
\pgfpathlineto{\pgfqpoint{5.931783in}{0.731135in}}%
\pgfpathlineto{\pgfqpoint{5.932081in}{0.731184in}}%
\pgfpathlineto{\pgfqpoint{5.932378in}{0.731233in}}%
\pgfpathlineto{\pgfqpoint{5.932676in}{0.731283in}}%
\pgfpathlineto{\pgfqpoint{5.932973in}{0.731332in}}%
\pgfpathlineto{\pgfqpoint{5.933271in}{0.731381in}}%
\pgfpathlineto{\pgfqpoint{5.933568in}{0.731430in}}%
\pgfpathlineto{\pgfqpoint{5.933866in}{0.731479in}}%
\pgfpathlineto{\pgfqpoint{5.934163in}{0.731528in}}%
\pgfpathlineto{\pgfqpoint{5.934461in}{0.731577in}}%
\pgfpathlineto{\pgfqpoint{5.934758in}{0.731626in}}%
\pgfpathlineto{\pgfqpoint{5.935056in}{0.731675in}}%
\pgfpathlineto{\pgfqpoint{5.935353in}{0.731724in}}%
\pgfpathlineto{\pgfqpoint{5.935651in}{0.731773in}}%
\pgfpathlineto{\pgfqpoint{5.935948in}{0.731823in}}%
\pgfpathlineto{\pgfqpoint{5.936246in}{0.731872in}}%
\pgfpathlineto{\pgfqpoint{5.936543in}{0.731921in}}%
\pgfpathlineto{\pgfqpoint{5.936841in}{0.731970in}}%
\pgfpathlineto{\pgfqpoint{5.937138in}{0.732019in}}%
\pgfpathlineto{\pgfqpoint{5.937435in}{0.732068in}}%
\pgfpathlineto{\pgfqpoint{5.937733in}{0.732117in}}%
\pgfpathlineto{\pgfqpoint{5.938030in}{0.732166in}}%
\pgfpathlineto{\pgfqpoint{5.938328in}{0.732215in}}%
\pgfpathlineto{\pgfqpoint{5.938625in}{0.732264in}}%
\pgfpathlineto{\pgfqpoint{5.938923in}{0.732314in}}%
\pgfpathlineto{\pgfqpoint{5.939220in}{0.732363in}}%
\pgfpathlineto{\pgfqpoint{5.939518in}{0.732412in}}%
\pgfpathlineto{\pgfqpoint{5.939815in}{0.732461in}}%
\pgfpathlineto{\pgfqpoint{5.940113in}{0.732510in}}%
\pgfpathlineto{\pgfqpoint{5.940410in}{0.732559in}}%
\pgfpathlineto{\pgfqpoint{5.940708in}{0.732608in}}%
\pgfpathlineto{\pgfqpoint{5.941005in}{0.732657in}}%
\pgfpathlineto{\pgfqpoint{5.941303in}{0.732706in}}%
\pgfpathlineto{\pgfqpoint{5.941600in}{0.732755in}}%
\pgfpathlineto{\pgfqpoint{5.941898in}{0.732804in}}%
\pgfpathlineto{\pgfqpoint{5.942195in}{0.732854in}}%
\pgfpathlineto{\pgfqpoint{5.942493in}{0.732903in}}%
\pgfpathlineto{\pgfqpoint{5.942790in}{0.732952in}}%
\pgfpathlineto{\pgfqpoint{5.943088in}{0.733001in}}%
\pgfpathlineto{\pgfqpoint{5.943385in}{0.733050in}}%
\pgfpathlineto{\pgfqpoint{5.943683in}{0.733099in}}%
\pgfpathlineto{\pgfqpoint{5.943980in}{0.733148in}}%
\pgfpathlineto{\pgfqpoint{5.944277in}{0.733197in}}%
\pgfpathlineto{\pgfqpoint{5.944575in}{0.733246in}}%
\pgfpathlineto{\pgfqpoint{5.944872in}{0.733295in}}%
\pgfpathlineto{\pgfqpoint{5.945170in}{0.733345in}}%
\pgfpathlineto{\pgfqpoint{5.945467in}{0.733394in}}%
\pgfpathlineto{\pgfqpoint{5.945765in}{0.733443in}}%
\pgfpathlineto{\pgfqpoint{5.946062in}{0.733492in}}%
\pgfpathlineto{\pgfqpoint{5.946360in}{0.733541in}}%
\pgfpathlineto{\pgfqpoint{5.946657in}{0.733590in}}%
\pgfpathlineto{\pgfqpoint{5.946955in}{0.733639in}}%
\pgfpathlineto{\pgfqpoint{5.947252in}{0.733688in}}%
\pgfpathlineto{\pgfqpoint{5.947550in}{0.733737in}}%
\pgfpathlineto{\pgfqpoint{5.947847in}{0.733786in}}%
\pgfpathlineto{\pgfqpoint{5.948145in}{0.733835in}}%
\pgfpathlineto{\pgfqpoint{5.948442in}{0.733885in}}%
\pgfpathlineto{\pgfqpoint{5.948740in}{0.733934in}}%
\pgfpathlineto{\pgfqpoint{5.949037in}{0.733983in}}%
\pgfpathlineto{\pgfqpoint{5.949335in}{0.734032in}}%
\pgfpathlineto{\pgfqpoint{5.949632in}{0.734081in}}%
\pgfpathlineto{\pgfqpoint{5.949930in}{0.734130in}}%
\pgfpathlineto{\pgfqpoint{5.950227in}{0.734179in}}%
\pgfpathlineto{\pgfqpoint{5.950525in}{0.734228in}}%
\pgfpathlineto{\pgfqpoint{5.950822in}{0.734277in}}%
\pgfpathlineto{\pgfqpoint{5.951119in}{0.734326in}}%
\pgfpathlineto{\pgfqpoint{5.951417in}{0.734376in}}%
\pgfpathlineto{\pgfqpoint{5.951714in}{0.734425in}}%
\pgfpathlineto{\pgfqpoint{5.952012in}{0.734474in}}%
\pgfpathlineto{\pgfqpoint{5.952309in}{0.734523in}}%
\pgfpathlineto{\pgfqpoint{5.952607in}{0.734572in}}%
\pgfpathlineto{\pgfqpoint{5.952904in}{0.734621in}}%
\pgfpathlineto{\pgfqpoint{5.953202in}{0.734670in}}%
\pgfpathlineto{\pgfqpoint{5.953499in}{0.734719in}}%
\pgfpathlineto{\pgfqpoint{5.953797in}{0.734768in}}%
\pgfpathlineto{\pgfqpoint{5.954094in}{0.734817in}}%
\pgfpathlineto{\pgfqpoint{5.954392in}{0.734866in}}%
\pgfpathlineto{\pgfqpoint{5.954689in}{0.734916in}}%
\pgfpathlineto{\pgfqpoint{5.954987in}{0.734965in}}%
\pgfpathlineto{\pgfqpoint{5.955284in}{0.735014in}}%
\pgfpathlineto{\pgfqpoint{5.955582in}{0.735063in}}%
\pgfpathlineto{\pgfqpoint{5.955879in}{0.735112in}}%
\pgfpathlineto{\pgfqpoint{5.956177in}{0.735161in}}%
\pgfpathlineto{\pgfqpoint{5.956474in}{0.735210in}}%
\pgfpathlineto{\pgfqpoint{5.956772in}{0.735259in}}%
\pgfpathlineto{\pgfqpoint{5.957069in}{0.735308in}}%
\pgfpathlineto{\pgfqpoint{5.957366in}{0.735357in}}%
\pgfpathlineto{\pgfqpoint{5.957664in}{0.735407in}}%
\pgfpathlineto{\pgfqpoint{5.957961in}{0.735456in}}%
\pgfpathlineto{\pgfqpoint{5.958259in}{0.735505in}}%
\pgfpathlineto{\pgfqpoint{5.958556in}{0.735554in}}%
\pgfpathlineto{\pgfqpoint{5.958854in}{0.735603in}}%
\pgfpathlineto{\pgfqpoint{5.959151in}{0.735652in}}%
\pgfpathlineto{\pgfqpoint{5.959449in}{0.735701in}}%
\pgfpathlineto{\pgfqpoint{5.959746in}{0.735750in}}%
\pgfpathlineto{\pgfqpoint{5.960044in}{0.735799in}}%
\pgfpathlineto{\pgfqpoint{5.960341in}{0.735848in}}%
\pgfpathlineto{\pgfqpoint{5.960639in}{0.735897in}}%
\pgfpathlineto{\pgfqpoint{5.960936in}{0.735947in}}%
\pgfpathlineto{\pgfqpoint{5.961234in}{0.735996in}}%
\pgfpathlineto{\pgfqpoint{5.961531in}{0.736045in}}%
\pgfpathlineto{\pgfqpoint{5.961829in}{0.736094in}}%
\pgfpathlineto{\pgfqpoint{5.962126in}{0.736143in}}%
\pgfpathlineto{\pgfqpoint{5.962424in}{0.736192in}}%
\pgfpathlineto{\pgfqpoint{5.962721in}{0.736241in}}%
\pgfpathlineto{\pgfqpoint{5.963019in}{0.736290in}}%
\pgfpathlineto{\pgfqpoint{5.963316in}{0.736339in}}%
\pgfpathlineto{\pgfqpoint{5.963614in}{0.736388in}}%
\pgfpathlineto{\pgfqpoint{5.963911in}{0.736437in}}%
\pgfpathlineto{\pgfqpoint{5.964208in}{0.736487in}}%
\pgfpathlineto{\pgfqpoint{5.964506in}{0.736536in}}%
\pgfpathlineto{\pgfqpoint{5.964803in}{0.736585in}}%
\pgfpathlineto{\pgfqpoint{5.965101in}{0.736634in}}%
\pgfpathlineto{\pgfqpoint{5.965398in}{0.736683in}}%
\pgfpathlineto{\pgfqpoint{5.965696in}{0.736732in}}%
\pgfpathlineto{\pgfqpoint{5.965993in}{0.736781in}}%
\pgfpathlineto{\pgfqpoint{5.966291in}{0.736830in}}%
\pgfpathlineto{\pgfqpoint{5.966588in}{0.736879in}}%
\pgfpathlineto{\pgfqpoint{5.966886in}{0.736928in}}%
\pgfpathlineto{\pgfqpoint{5.967183in}{0.736978in}}%
\pgfpathlineto{\pgfqpoint{5.967481in}{0.737027in}}%
\pgfpathlineto{\pgfqpoint{5.967778in}{0.737076in}}%
\pgfpathlineto{\pgfqpoint{5.968076in}{0.737125in}}%
\pgfpathlineto{\pgfqpoint{5.968373in}{0.737174in}}%
\pgfpathlineto{\pgfqpoint{5.968671in}{0.737223in}}%
\pgfpathlineto{\pgfqpoint{5.968968in}{0.737272in}}%
\pgfpathlineto{\pgfqpoint{5.969266in}{0.737321in}}%
\pgfpathlineto{\pgfqpoint{5.969563in}{0.737370in}}%
\pgfpathlineto{\pgfqpoint{5.969861in}{0.737419in}}%
\pgfpathlineto{\pgfqpoint{5.970158in}{0.737468in}}%
\pgfpathlineto{\pgfqpoint{5.970456in}{0.737518in}}%
\pgfpathlineto{\pgfqpoint{5.970753in}{0.737567in}}%
\pgfpathlineto{\pgfqpoint{5.971050in}{0.737616in}}%
\pgfpathlineto{\pgfqpoint{5.971348in}{0.737665in}}%
\pgfpathlineto{\pgfqpoint{5.971645in}{0.737714in}}%
\pgfpathlineto{\pgfqpoint{5.971943in}{0.737763in}}%
\pgfpathlineto{\pgfqpoint{5.972240in}{0.737812in}}%
\pgfpathlineto{\pgfqpoint{5.972538in}{0.737861in}}%
\pgfpathlineto{\pgfqpoint{5.972835in}{0.737910in}}%
\pgfpathlineto{\pgfqpoint{5.973133in}{0.737959in}}%
\pgfpathlineto{\pgfqpoint{5.973430in}{0.738009in}}%
\pgfpathlineto{\pgfqpoint{5.973728in}{0.738058in}}%
\pgfpathlineto{\pgfqpoint{5.974025in}{0.738107in}}%
\pgfpathlineto{\pgfqpoint{5.974323in}{0.738156in}}%
\pgfpathlineto{\pgfqpoint{5.974620in}{0.738205in}}%
\pgfpathlineto{\pgfqpoint{5.974918in}{0.738254in}}%
\pgfpathlineto{\pgfqpoint{5.975215in}{0.738303in}}%
\pgfpathlineto{\pgfqpoint{5.975513in}{0.738352in}}%
\pgfpathlineto{\pgfqpoint{5.975810in}{0.738401in}}%
\pgfpathlineto{\pgfqpoint{5.976108in}{0.738450in}}%
\pgfpathlineto{\pgfqpoint{5.976405in}{0.738405in}}%
\pgfpathlineto{\pgfqpoint{5.976703in}{0.737926in}}%
\pgfpathlineto{\pgfqpoint{5.977000in}{0.737259in}}%
\pgfpathlineto{\pgfqpoint{5.977297in}{0.736593in}}%
\pgfpathlineto{\pgfqpoint{5.977595in}{0.735926in}}%
\pgfpathlineto{\pgfqpoint{5.977892in}{0.735260in}}%
\pgfpathlineto{\pgfqpoint{5.978190in}{0.734593in}}%
\pgfpathlineto{\pgfqpoint{5.978487in}{0.733926in}}%
\pgfpathlineto{\pgfqpoint{5.978785in}{0.733260in}}%
\pgfpathlineto{\pgfqpoint{5.979082in}{0.732593in}}%
\pgfpathlineto{\pgfqpoint{5.979380in}{0.731927in}}%
\pgfpathlineto{\pgfqpoint{5.979677in}{0.731260in}}%
\pgfpathlineto{\pgfqpoint{5.979975in}{0.730594in}}%
\pgfpathlineto{\pgfqpoint{5.980272in}{0.729927in}}%
\pgfpathlineto{\pgfqpoint{5.980570in}{0.729261in}}%
\pgfpathlineto{\pgfqpoint{5.980867in}{0.728594in}}%
\pgfpathlineto{\pgfqpoint{5.981165in}{0.727927in}}%
\pgfpathlineto{\pgfqpoint{5.981462in}{0.727261in}}%
\pgfpathlineto{\pgfqpoint{5.981760in}{0.726594in}}%
\pgfpathlineto{\pgfqpoint{5.982057in}{0.725928in}}%
\pgfpathlineto{\pgfqpoint{5.982355in}{0.725261in}}%
\pgfpathlineto{\pgfqpoint{5.982652in}{0.724595in}}%
\pgfpathlineto{\pgfqpoint{5.982950in}{0.723928in}}%
\pgfpathlineto{\pgfqpoint{5.983247in}{0.727136in}}%
\pgfpathlineto{\pgfqpoint{5.983545in}{0.739610in}}%
\pgfpathlineto{\pgfqpoint{5.983842in}{0.739609in}}%
\pgfpathlineto{\pgfqpoint{5.984139in}{0.739608in}}%
\pgfpathlineto{\pgfqpoint{5.984437in}{0.739607in}}%
\pgfpathlineto{\pgfqpoint{5.984734in}{0.739606in}}%
\pgfpathlineto{\pgfqpoint{5.985032in}{0.739605in}}%
\pgfpathlineto{\pgfqpoint{5.985329in}{0.739604in}}%
\pgfpathlineto{\pgfqpoint{5.985627in}{0.739603in}}%
\pgfpathlineto{\pgfqpoint{5.985924in}{0.739602in}}%
\pgfpathlineto{\pgfqpoint{5.986222in}{0.739601in}}%
\pgfpathlineto{\pgfqpoint{5.986519in}{0.739600in}}%
\pgfpathlineto{\pgfqpoint{5.986817in}{0.739599in}}%
\pgfpathlineto{\pgfqpoint{5.987114in}{0.739598in}}%
\pgfpathlineto{\pgfqpoint{5.987412in}{0.739597in}}%
\pgfpathlineto{\pgfqpoint{5.987709in}{0.739596in}}%
\pgfpathlineto{\pgfqpoint{5.988007in}{0.739595in}}%
\pgfpathlineto{\pgfqpoint{5.988304in}{0.739594in}}%
\pgfpathlineto{\pgfqpoint{5.988602in}{0.739593in}}%
\pgfpathlineto{\pgfqpoint{5.988899in}{0.739592in}}%
\pgfpathlineto{\pgfqpoint{5.989197in}{0.739591in}}%
\pgfpathlineto{\pgfqpoint{5.989494in}{0.739590in}}%
\pgfpathlineto{\pgfqpoint{5.989792in}{0.739593in}}%
\pgfpathlineto{\pgfqpoint{5.990089in}{0.739597in}}%
\pgfpathlineto{\pgfqpoint{5.990387in}{0.739599in}}%
\pgfpathlineto{\pgfqpoint{5.990684in}{0.739600in}}%
\pgfpathlineto{\pgfqpoint{5.990981in}{0.739601in}}%
\pgfpathlineto{\pgfqpoint{5.991279in}{0.739602in}}%
\pgfpathlineto{\pgfqpoint{5.991576in}{0.739604in}}%
\pgfpathlineto{\pgfqpoint{5.991874in}{0.739605in}}%
\pgfpathlineto{\pgfqpoint{5.992171in}{0.739606in}}%
\pgfpathlineto{\pgfqpoint{5.992469in}{0.739607in}}%
\pgfpathlineto{\pgfqpoint{5.992766in}{0.739608in}}%
\pgfpathlineto{\pgfqpoint{5.993064in}{0.739610in}}%
\pgfpathlineto{\pgfqpoint{5.993361in}{0.739611in}}%
\pgfpathlineto{\pgfqpoint{5.993659in}{0.739612in}}%
\pgfpathlineto{\pgfqpoint{5.993956in}{0.739613in}}%
\pgfpathlineto{\pgfqpoint{5.994254in}{0.739615in}}%
\pgfpathlineto{\pgfqpoint{5.994551in}{0.739616in}}%
\pgfpathlineto{\pgfqpoint{5.994849in}{0.739617in}}%
\pgfpathlineto{\pgfqpoint{5.995146in}{0.739618in}}%
\pgfpathlineto{\pgfqpoint{5.995444in}{0.739619in}}%
\pgfpathlineto{\pgfqpoint{5.995741in}{0.739621in}}%
\pgfpathlineto{\pgfqpoint{5.996039in}{0.739622in}}%
\pgfpathlineto{\pgfqpoint{5.996336in}{0.739623in}}%
\pgfpathlineto{\pgfqpoint{5.996634in}{0.739624in}}%
\pgfpathlineto{\pgfqpoint{5.996931in}{0.739626in}}%
\pgfpathlineto{\pgfqpoint{5.997228in}{0.739627in}}%
\pgfpathlineto{\pgfqpoint{5.997526in}{0.739628in}}%
\pgfpathlineto{\pgfqpoint{5.997823in}{0.739629in}}%
\pgfpathlineto{\pgfqpoint{5.998121in}{0.739630in}}%
\pgfpathlineto{\pgfqpoint{5.998418in}{0.739632in}}%
\pgfpathlineto{\pgfqpoint{5.998716in}{0.739633in}}%
\pgfpathlineto{\pgfqpoint{5.999013in}{0.739634in}}%
\pgfpathlineto{\pgfqpoint{5.999311in}{0.739635in}}%
\pgfpathlineto{\pgfqpoint{5.999608in}{0.739637in}}%
\pgfpathlineto{\pgfqpoint{5.999906in}{0.739638in}}%
\pgfpathlineto{\pgfqpoint{6.000203in}{0.739639in}}%
\pgfpathlineto{\pgfqpoint{6.000501in}{0.739640in}}%
\pgfpathlineto{\pgfqpoint{6.000798in}{0.739641in}}%
\pgfpathlineto{\pgfqpoint{6.001096in}{0.739643in}}%
\pgfpathlineto{\pgfqpoint{6.001393in}{0.739644in}}%
\pgfpathlineto{\pgfqpoint{6.001691in}{0.739645in}}%
\pgfpathlineto{\pgfqpoint{6.001988in}{0.739646in}}%
\pgfpathlineto{\pgfqpoint{6.002286in}{0.739648in}}%
\pgfpathlineto{\pgfqpoint{6.002583in}{0.739649in}}%
\pgfpathlineto{\pgfqpoint{6.002881in}{0.739650in}}%
\pgfpathlineto{\pgfqpoint{6.003178in}{0.739651in}}%
\pgfpathlineto{\pgfqpoint{6.003476in}{0.739652in}}%
\pgfpathlineto{\pgfqpoint{6.003773in}{0.739654in}}%
\pgfpathlineto{\pgfqpoint{6.004070in}{0.739655in}}%
\pgfpathlineto{\pgfqpoint{6.004368in}{0.739656in}}%
\pgfpathlineto{\pgfqpoint{6.004368in}{0.739656in}}%
\pgfpathlineto{\pgfqpoint{6.004368in}{0.739656in}}%
\pgfpathlineto{\pgfqpoint{6.004070in}{0.739656in}}%
\pgfpathlineto{\pgfqpoint{6.003773in}{0.739656in}}%
\pgfpathlineto{\pgfqpoint{6.003476in}{0.739656in}}%
\pgfpathlineto{\pgfqpoint{6.003178in}{0.739656in}}%
\pgfpathlineto{\pgfqpoint{6.002881in}{0.739656in}}%
\pgfpathlineto{\pgfqpoint{6.002583in}{0.739656in}}%
\pgfpathlineto{\pgfqpoint{6.002286in}{0.739656in}}%
\pgfpathlineto{\pgfqpoint{6.001988in}{0.739656in}}%
\pgfpathlineto{\pgfqpoint{6.001691in}{0.739656in}}%
\pgfpathlineto{\pgfqpoint{6.001393in}{0.739656in}}%
\pgfpathlineto{\pgfqpoint{6.001096in}{0.739656in}}%
\pgfpathlineto{\pgfqpoint{6.000798in}{0.739656in}}%
\pgfpathlineto{\pgfqpoint{6.000501in}{0.739656in}}%
\pgfpathlineto{\pgfqpoint{6.000203in}{0.739656in}}%
\pgfpathlineto{\pgfqpoint{5.999906in}{0.739656in}}%
\pgfpathlineto{\pgfqpoint{5.999608in}{0.739656in}}%
\pgfpathlineto{\pgfqpoint{5.999311in}{0.739656in}}%
\pgfpathlineto{\pgfqpoint{5.999013in}{0.739656in}}%
\pgfpathlineto{\pgfqpoint{5.998716in}{0.739656in}}%
\pgfpathlineto{\pgfqpoint{5.998418in}{0.739656in}}%
\pgfpathlineto{\pgfqpoint{5.998121in}{0.739656in}}%
\pgfpathlineto{\pgfqpoint{5.997823in}{0.739656in}}%
\pgfpathlineto{\pgfqpoint{5.997526in}{0.739656in}}%
\pgfpathlineto{\pgfqpoint{5.997228in}{0.739656in}}%
\pgfpathlineto{\pgfqpoint{5.996931in}{0.739656in}}%
\pgfpathlineto{\pgfqpoint{5.996634in}{0.739656in}}%
\pgfpathlineto{\pgfqpoint{5.996336in}{0.739656in}}%
\pgfpathlineto{\pgfqpoint{5.996039in}{0.739656in}}%
\pgfpathlineto{\pgfqpoint{5.995741in}{0.739656in}}%
\pgfpathlineto{\pgfqpoint{5.995444in}{0.739656in}}%
\pgfpathlineto{\pgfqpoint{5.995146in}{0.739656in}}%
\pgfpathlineto{\pgfqpoint{5.994849in}{0.739656in}}%
\pgfpathlineto{\pgfqpoint{5.994551in}{0.739656in}}%
\pgfpathlineto{\pgfqpoint{5.994254in}{0.739656in}}%
\pgfpathlineto{\pgfqpoint{5.993956in}{0.739656in}}%
\pgfpathlineto{\pgfqpoint{5.993659in}{0.739656in}}%
\pgfpathlineto{\pgfqpoint{5.993361in}{0.739656in}}%
\pgfpathlineto{\pgfqpoint{5.993064in}{0.739656in}}%
\pgfpathlineto{\pgfqpoint{5.992766in}{0.739656in}}%
\pgfpathlineto{\pgfqpoint{5.992469in}{0.739656in}}%
\pgfpathlineto{\pgfqpoint{5.992171in}{0.739656in}}%
\pgfpathlineto{\pgfqpoint{5.991874in}{0.739656in}}%
\pgfpathlineto{\pgfqpoint{5.991576in}{0.739656in}}%
\pgfpathlineto{\pgfqpoint{5.991279in}{0.739656in}}%
\pgfpathlineto{\pgfqpoint{5.990981in}{0.739656in}}%
\pgfpathlineto{\pgfqpoint{5.990684in}{0.739656in}}%
\pgfpathlineto{\pgfqpoint{5.990387in}{0.739656in}}%
\pgfpathlineto{\pgfqpoint{5.990089in}{0.739656in}}%
\pgfpathlineto{\pgfqpoint{5.989792in}{0.739656in}}%
\pgfpathlineto{\pgfqpoint{5.989494in}{0.739656in}}%
\pgfpathlineto{\pgfqpoint{5.989197in}{0.739656in}}%
\pgfpathlineto{\pgfqpoint{5.988899in}{0.739656in}}%
\pgfpathlineto{\pgfqpoint{5.988602in}{0.739656in}}%
\pgfpathlineto{\pgfqpoint{5.988304in}{0.739656in}}%
\pgfpathlineto{\pgfqpoint{5.988007in}{0.739656in}}%
\pgfpathlineto{\pgfqpoint{5.987709in}{0.739656in}}%
\pgfpathlineto{\pgfqpoint{5.987412in}{0.739656in}}%
\pgfpathlineto{\pgfqpoint{5.987114in}{0.739656in}}%
\pgfpathlineto{\pgfqpoint{5.986817in}{0.739656in}}%
\pgfpathlineto{\pgfqpoint{5.986519in}{0.739656in}}%
\pgfpathlineto{\pgfqpoint{5.986222in}{0.739656in}}%
\pgfpathlineto{\pgfqpoint{5.985924in}{0.739656in}}%
\pgfpathlineto{\pgfqpoint{5.985627in}{0.739656in}}%
\pgfpathlineto{\pgfqpoint{5.985329in}{0.739656in}}%
\pgfpathlineto{\pgfqpoint{5.985032in}{0.739656in}}%
\pgfpathlineto{\pgfqpoint{5.984734in}{0.739656in}}%
\pgfpathlineto{\pgfqpoint{5.984437in}{0.739656in}}%
\pgfpathlineto{\pgfqpoint{5.984139in}{0.739656in}}%
\pgfpathlineto{\pgfqpoint{5.983842in}{0.739656in}}%
\pgfpathlineto{\pgfqpoint{5.983545in}{0.739656in}}%
\pgfpathlineto{\pgfqpoint{5.983247in}{0.739656in}}%
\pgfpathlineto{\pgfqpoint{5.982950in}{0.739656in}}%
\pgfpathlineto{\pgfqpoint{5.982652in}{0.739656in}}%
\pgfpathlineto{\pgfqpoint{5.982355in}{0.739656in}}%
\pgfpathlineto{\pgfqpoint{5.982057in}{0.739656in}}%
\pgfpathlineto{\pgfqpoint{5.981760in}{0.739656in}}%
\pgfpathlineto{\pgfqpoint{5.981462in}{0.739656in}}%
\pgfpathlineto{\pgfqpoint{5.981165in}{0.739656in}}%
\pgfpathlineto{\pgfqpoint{5.980867in}{0.739656in}}%
\pgfpathlineto{\pgfqpoint{5.980570in}{0.739656in}}%
\pgfpathlineto{\pgfqpoint{5.980272in}{0.739656in}}%
\pgfpathlineto{\pgfqpoint{5.979975in}{0.739656in}}%
\pgfpathlineto{\pgfqpoint{5.979677in}{0.739656in}}%
\pgfpathlineto{\pgfqpoint{5.979380in}{0.739656in}}%
\pgfpathlineto{\pgfqpoint{5.979082in}{0.739656in}}%
\pgfpathlineto{\pgfqpoint{5.978785in}{0.739656in}}%
\pgfpathlineto{\pgfqpoint{5.978487in}{0.739656in}}%
\pgfpathlineto{\pgfqpoint{5.978190in}{0.739656in}}%
\pgfpathlineto{\pgfqpoint{5.977892in}{0.739656in}}%
\pgfpathlineto{\pgfqpoint{5.977595in}{0.739656in}}%
\pgfpathlineto{\pgfqpoint{5.977297in}{0.739656in}}%
\pgfpathlineto{\pgfqpoint{5.977000in}{0.739656in}}%
\pgfpathlineto{\pgfqpoint{5.976703in}{0.739656in}}%
\pgfpathlineto{\pgfqpoint{5.976405in}{0.739656in}}%
\pgfpathlineto{\pgfqpoint{5.976108in}{0.739656in}}%
\pgfpathlineto{\pgfqpoint{5.975810in}{0.739656in}}%
\pgfpathlineto{\pgfqpoint{5.975513in}{0.739656in}}%
\pgfpathlineto{\pgfqpoint{5.975215in}{0.739656in}}%
\pgfpathlineto{\pgfqpoint{5.974918in}{0.739656in}}%
\pgfpathlineto{\pgfqpoint{5.974620in}{0.739656in}}%
\pgfpathlineto{\pgfqpoint{5.974323in}{0.739656in}}%
\pgfpathlineto{\pgfqpoint{5.974025in}{0.739656in}}%
\pgfpathlineto{\pgfqpoint{5.973728in}{0.739656in}}%
\pgfpathlineto{\pgfqpoint{5.973430in}{0.739656in}}%
\pgfpathlineto{\pgfqpoint{5.973133in}{0.739656in}}%
\pgfpathlineto{\pgfqpoint{5.972835in}{0.739656in}}%
\pgfpathlineto{\pgfqpoint{5.972538in}{0.739656in}}%
\pgfpathlineto{\pgfqpoint{5.972240in}{0.739656in}}%
\pgfpathlineto{\pgfqpoint{5.971943in}{0.739656in}}%
\pgfpathlineto{\pgfqpoint{5.971645in}{0.739656in}}%
\pgfpathlineto{\pgfqpoint{5.971348in}{0.739656in}}%
\pgfpathlineto{\pgfqpoint{5.971050in}{0.739656in}}%
\pgfpathlineto{\pgfqpoint{5.970753in}{0.739656in}}%
\pgfpathlineto{\pgfqpoint{5.970456in}{0.739656in}}%
\pgfpathlineto{\pgfqpoint{5.970158in}{0.739656in}}%
\pgfpathlineto{\pgfqpoint{5.969861in}{0.739656in}}%
\pgfpathlineto{\pgfqpoint{5.969563in}{0.739656in}}%
\pgfpathlineto{\pgfqpoint{5.969266in}{0.739656in}}%
\pgfpathlineto{\pgfqpoint{5.968968in}{0.739656in}}%
\pgfpathlineto{\pgfqpoint{5.968671in}{0.739656in}}%
\pgfpathlineto{\pgfqpoint{5.968373in}{0.739656in}}%
\pgfpathlineto{\pgfqpoint{5.968076in}{0.739656in}}%
\pgfpathlineto{\pgfqpoint{5.967778in}{0.739656in}}%
\pgfpathlineto{\pgfqpoint{5.967481in}{0.739656in}}%
\pgfpathlineto{\pgfqpoint{5.967183in}{0.739656in}}%
\pgfpathlineto{\pgfqpoint{5.966886in}{0.739656in}}%
\pgfpathlineto{\pgfqpoint{5.966588in}{0.739656in}}%
\pgfpathlineto{\pgfqpoint{5.966291in}{0.739656in}}%
\pgfpathlineto{\pgfqpoint{5.965993in}{0.739656in}}%
\pgfpathlineto{\pgfqpoint{5.965696in}{0.739656in}}%
\pgfpathlineto{\pgfqpoint{5.965398in}{0.739656in}}%
\pgfpathlineto{\pgfqpoint{5.965101in}{0.739656in}}%
\pgfpathlineto{\pgfqpoint{5.964803in}{0.739656in}}%
\pgfpathlineto{\pgfqpoint{5.964506in}{0.739656in}}%
\pgfpathlineto{\pgfqpoint{5.964208in}{0.739656in}}%
\pgfpathlineto{\pgfqpoint{5.963911in}{0.739656in}}%
\pgfpathlineto{\pgfqpoint{5.963614in}{0.739656in}}%
\pgfpathlineto{\pgfqpoint{5.963316in}{0.739656in}}%
\pgfpathlineto{\pgfqpoint{5.963019in}{0.739656in}}%
\pgfpathlineto{\pgfqpoint{5.962721in}{0.739656in}}%
\pgfpathlineto{\pgfqpoint{5.962424in}{0.739656in}}%
\pgfpathlineto{\pgfqpoint{5.962126in}{0.739656in}}%
\pgfpathlineto{\pgfqpoint{5.961829in}{0.739656in}}%
\pgfpathlineto{\pgfqpoint{5.961531in}{0.739656in}}%
\pgfpathlineto{\pgfqpoint{5.961234in}{0.739656in}}%
\pgfpathlineto{\pgfqpoint{5.960936in}{0.739656in}}%
\pgfpathlineto{\pgfqpoint{5.960639in}{0.739656in}}%
\pgfpathlineto{\pgfqpoint{5.960341in}{0.739656in}}%
\pgfpathlineto{\pgfqpoint{5.960044in}{0.739656in}}%
\pgfpathlineto{\pgfqpoint{5.959746in}{0.739656in}}%
\pgfpathlineto{\pgfqpoint{5.959449in}{0.739656in}}%
\pgfpathlineto{\pgfqpoint{5.959151in}{0.739656in}}%
\pgfpathlineto{\pgfqpoint{5.958854in}{0.739656in}}%
\pgfpathlineto{\pgfqpoint{5.958556in}{0.739656in}}%
\pgfpathlineto{\pgfqpoint{5.958259in}{0.739656in}}%
\pgfpathlineto{\pgfqpoint{5.957961in}{0.739656in}}%
\pgfpathlineto{\pgfqpoint{5.957664in}{0.739656in}}%
\pgfpathlineto{\pgfqpoint{5.957366in}{0.739656in}}%
\pgfpathlineto{\pgfqpoint{5.957069in}{0.739656in}}%
\pgfpathlineto{\pgfqpoint{5.956772in}{0.739656in}}%
\pgfpathlineto{\pgfqpoint{5.956474in}{0.739656in}}%
\pgfpathlineto{\pgfqpoint{5.956177in}{0.739656in}}%
\pgfpathlineto{\pgfqpoint{5.955879in}{0.739656in}}%
\pgfpathlineto{\pgfqpoint{5.955582in}{0.739656in}}%
\pgfpathlineto{\pgfqpoint{5.955284in}{0.739656in}}%
\pgfpathlineto{\pgfqpoint{5.954987in}{0.739656in}}%
\pgfpathlineto{\pgfqpoint{5.954689in}{0.739656in}}%
\pgfpathlineto{\pgfqpoint{5.954392in}{0.739656in}}%
\pgfpathlineto{\pgfqpoint{5.954094in}{0.739656in}}%
\pgfpathlineto{\pgfqpoint{5.953797in}{0.739656in}}%
\pgfpathlineto{\pgfqpoint{5.953499in}{0.739656in}}%
\pgfpathlineto{\pgfqpoint{5.953202in}{0.739656in}}%
\pgfpathlineto{\pgfqpoint{5.952904in}{0.739656in}}%
\pgfpathlineto{\pgfqpoint{5.952607in}{0.739656in}}%
\pgfpathlineto{\pgfqpoint{5.952309in}{0.739656in}}%
\pgfpathlineto{\pgfqpoint{5.952012in}{0.739656in}}%
\pgfpathlineto{\pgfqpoint{5.951714in}{0.739656in}}%
\pgfpathlineto{\pgfqpoint{5.951417in}{0.739656in}}%
\pgfpathlineto{\pgfqpoint{5.951119in}{0.739656in}}%
\pgfpathlineto{\pgfqpoint{5.950822in}{0.739656in}}%
\pgfpathlineto{\pgfqpoint{5.950525in}{0.739656in}}%
\pgfpathlineto{\pgfqpoint{5.950227in}{0.739656in}}%
\pgfpathlineto{\pgfqpoint{5.949930in}{0.739656in}}%
\pgfpathlineto{\pgfqpoint{5.949632in}{0.739656in}}%
\pgfpathlineto{\pgfqpoint{5.949335in}{0.739656in}}%
\pgfpathlineto{\pgfqpoint{5.949037in}{0.739656in}}%
\pgfpathlineto{\pgfqpoint{5.948740in}{0.739656in}}%
\pgfpathlineto{\pgfqpoint{5.948442in}{0.739656in}}%
\pgfpathlineto{\pgfqpoint{5.948145in}{0.739656in}}%
\pgfpathlineto{\pgfqpoint{5.947847in}{0.739656in}}%
\pgfpathlineto{\pgfqpoint{5.947550in}{0.739656in}}%
\pgfpathlineto{\pgfqpoint{5.947252in}{0.739656in}}%
\pgfpathlineto{\pgfqpoint{5.946955in}{0.739656in}}%
\pgfpathlineto{\pgfqpoint{5.946657in}{0.739656in}}%
\pgfpathlineto{\pgfqpoint{5.946360in}{0.739656in}}%
\pgfpathlineto{\pgfqpoint{5.946062in}{0.739656in}}%
\pgfpathlineto{\pgfqpoint{5.945765in}{0.739656in}}%
\pgfpathlineto{\pgfqpoint{5.945467in}{0.739656in}}%
\pgfpathlineto{\pgfqpoint{5.945170in}{0.739656in}}%
\pgfpathlineto{\pgfqpoint{5.944872in}{0.739656in}}%
\pgfpathlineto{\pgfqpoint{5.944575in}{0.739656in}}%
\pgfpathlineto{\pgfqpoint{5.944277in}{0.739656in}}%
\pgfpathlineto{\pgfqpoint{5.943980in}{0.739656in}}%
\pgfpathlineto{\pgfqpoint{5.943683in}{0.739656in}}%
\pgfpathlineto{\pgfqpoint{5.943385in}{0.739656in}}%
\pgfpathlineto{\pgfqpoint{5.943088in}{0.739656in}}%
\pgfpathlineto{\pgfqpoint{5.942790in}{0.739656in}}%
\pgfpathlineto{\pgfqpoint{5.942493in}{0.739656in}}%
\pgfpathlineto{\pgfqpoint{5.942195in}{0.739656in}}%
\pgfpathlineto{\pgfqpoint{5.941898in}{0.739656in}}%
\pgfpathlineto{\pgfqpoint{5.941600in}{0.739656in}}%
\pgfpathlineto{\pgfqpoint{5.941303in}{0.739656in}}%
\pgfpathlineto{\pgfqpoint{5.941005in}{0.739656in}}%
\pgfpathlineto{\pgfqpoint{5.940708in}{0.739656in}}%
\pgfpathlineto{\pgfqpoint{5.940410in}{0.739656in}}%
\pgfpathlineto{\pgfqpoint{5.940113in}{0.739656in}}%
\pgfpathlineto{\pgfqpoint{5.939815in}{0.739656in}}%
\pgfpathlineto{\pgfqpoint{5.939518in}{0.739656in}}%
\pgfpathlineto{\pgfqpoint{5.939220in}{0.739656in}}%
\pgfpathlineto{\pgfqpoint{5.938923in}{0.739656in}}%
\pgfpathlineto{\pgfqpoint{5.938625in}{0.739656in}}%
\pgfpathlineto{\pgfqpoint{5.938328in}{0.739656in}}%
\pgfpathlineto{\pgfqpoint{5.938030in}{0.739656in}}%
\pgfpathlineto{\pgfqpoint{5.937733in}{0.739656in}}%
\pgfpathlineto{\pgfqpoint{5.937435in}{0.739656in}}%
\pgfpathlineto{\pgfqpoint{5.937138in}{0.739656in}}%
\pgfpathlineto{\pgfqpoint{5.936841in}{0.739656in}}%
\pgfpathlineto{\pgfqpoint{5.936543in}{0.739656in}}%
\pgfpathlineto{\pgfqpoint{5.936246in}{0.739656in}}%
\pgfpathlineto{\pgfqpoint{5.935948in}{0.739656in}}%
\pgfpathlineto{\pgfqpoint{5.935651in}{0.739656in}}%
\pgfpathlineto{\pgfqpoint{5.935353in}{0.739656in}}%
\pgfpathlineto{\pgfqpoint{5.935056in}{0.739656in}}%
\pgfpathlineto{\pgfqpoint{5.934758in}{0.739656in}}%
\pgfpathlineto{\pgfqpoint{5.934461in}{0.739656in}}%
\pgfpathlineto{\pgfqpoint{5.934163in}{0.739656in}}%
\pgfpathlineto{\pgfqpoint{5.933866in}{0.739656in}}%
\pgfpathlineto{\pgfqpoint{5.933568in}{0.739656in}}%
\pgfpathlineto{\pgfqpoint{5.933271in}{0.739656in}}%
\pgfpathlineto{\pgfqpoint{5.932973in}{0.739656in}}%
\pgfpathlineto{\pgfqpoint{5.932676in}{0.739656in}}%
\pgfpathlineto{\pgfqpoint{5.932378in}{0.739656in}}%
\pgfpathlineto{\pgfqpoint{5.932081in}{0.739656in}}%
\pgfpathlineto{\pgfqpoint{5.931783in}{0.739656in}}%
\pgfpathlineto{\pgfqpoint{5.931486in}{0.739656in}}%
\pgfpathlineto{\pgfqpoint{5.931188in}{0.739656in}}%
\pgfpathlineto{\pgfqpoint{5.930891in}{0.739656in}}%
\pgfpathlineto{\pgfqpoint{5.930594in}{0.739656in}}%
\pgfpathlineto{\pgfqpoint{5.930296in}{0.739656in}}%
\pgfpathlineto{\pgfqpoint{5.929999in}{0.739656in}}%
\pgfpathlineto{\pgfqpoint{5.929701in}{0.739656in}}%
\pgfpathlineto{\pgfqpoint{5.929404in}{0.739656in}}%
\pgfpathlineto{\pgfqpoint{5.929106in}{0.739656in}}%
\pgfpathlineto{\pgfqpoint{5.928809in}{0.739656in}}%
\pgfpathlineto{\pgfqpoint{5.928511in}{0.739656in}}%
\pgfpathlineto{\pgfqpoint{5.928214in}{0.739656in}}%
\pgfpathlineto{\pgfqpoint{5.927916in}{0.739656in}}%
\pgfpathlineto{\pgfqpoint{5.927619in}{0.739656in}}%
\pgfpathlineto{\pgfqpoint{5.927321in}{0.739656in}}%
\pgfpathlineto{\pgfqpoint{5.927024in}{0.739656in}}%
\pgfpathlineto{\pgfqpoint{5.926726in}{0.739656in}}%
\pgfpathlineto{\pgfqpoint{5.926429in}{0.739656in}}%
\pgfpathlineto{\pgfqpoint{5.926131in}{0.739656in}}%
\pgfpathlineto{\pgfqpoint{5.925834in}{0.739656in}}%
\pgfpathlineto{\pgfqpoint{5.925536in}{0.739656in}}%
\pgfpathlineto{\pgfqpoint{5.925239in}{0.739656in}}%
\pgfpathlineto{\pgfqpoint{5.924941in}{0.739656in}}%
\pgfpathlineto{\pgfqpoint{5.924644in}{0.739656in}}%
\pgfpathlineto{\pgfqpoint{5.924346in}{0.739656in}}%
\pgfpathlineto{\pgfqpoint{5.924049in}{0.739656in}}%
\pgfpathlineto{\pgfqpoint{5.923752in}{0.739656in}}%
\pgfpathlineto{\pgfqpoint{5.923454in}{0.739656in}}%
\pgfpathlineto{\pgfqpoint{5.923157in}{0.739656in}}%
\pgfpathlineto{\pgfqpoint{5.922859in}{0.739656in}}%
\pgfpathlineto{\pgfqpoint{5.922562in}{0.739656in}}%
\pgfpathlineto{\pgfqpoint{5.922264in}{0.739656in}}%
\pgfpathlineto{\pgfqpoint{5.921967in}{0.739656in}}%
\pgfpathlineto{\pgfqpoint{5.921669in}{0.739656in}}%
\pgfpathlineto{\pgfqpoint{5.921372in}{0.739656in}}%
\pgfpathlineto{\pgfqpoint{5.921074in}{0.739656in}}%
\pgfpathlineto{\pgfqpoint{5.920777in}{0.739656in}}%
\pgfpathlineto{\pgfqpoint{5.920479in}{0.739656in}}%
\pgfpathlineto{\pgfqpoint{5.920182in}{0.739656in}}%
\pgfpathlineto{\pgfqpoint{5.919884in}{0.739656in}}%
\pgfpathlineto{\pgfqpoint{5.919587in}{0.739656in}}%
\pgfpathlineto{\pgfqpoint{5.919289in}{0.739656in}}%
\pgfpathlineto{\pgfqpoint{5.918992in}{0.739656in}}%
\pgfpathlineto{\pgfqpoint{5.918694in}{0.739656in}}%
\pgfpathlineto{\pgfqpoint{5.918397in}{0.739656in}}%
\pgfpathlineto{\pgfqpoint{5.918099in}{0.739656in}}%
\pgfpathlineto{\pgfqpoint{5.917802in}{0.739656in}}%
\pgfpathlineto{\pgfqpoint{5.917504in}{0.739656in}}%
\pgfpathlineto{\pgfqpoint{5.917207in}{0.739656in}}%
\pgfpathlineto{\pgfqpoint{5.916910in}{0.739656in}}%
\pgfpathlineto{\pgfqpoint{5.916612in}{0.739656in}}%
\pgfpathlineto{\pgfqpoint{5.916315in}{0.739656in}}%
\pgfpathlineto{\pgfqpoint{5.916017in}{0.739656in}}%
\pgfpathlineto{\pgfqpoint{5.915720in}{0.739656in}}%
\pgfpathlineto{\pgfqpoint{5.915422in}{0.739656in}}%
\pgfpathlineto{\pgfqpoint{5.915125in}{0.739656in}}%
\pgfpathlineto{\pgfqpoint{5.914827in}{0.739656in}}%
\pgfpathlineto{\pgfqpoint{5.914530in}{0.739656in}}%
\pgfpathlineto{\pgfqpoint{5.914232in}{0.739656in}}%
\pgfpathlineto{\pgfqpoint{5.913935in}{0.739656in}}%
\pgfpathlineto{\pgfqpoint{5.913637in}{0.739656in}}%
\pgfpathlineto{\pgfqpoint{5.913340in}{0.739656in}}%
\pgfpathlineto{\pgfqpoint{5.913042in}{0.739656in}}%
\pgfpathlineto{\pgfqpoint{5.912745in}{0.739656in}}%
\pgfpathlineto{\pgfqpoint{5.912447in}{0.739656in}}%
\pgfpathlineto{\pgfqpoint{5.912150in}{0.739656in}}%
\pgfpathlineto{\pgfqpoint{5.911852in}{0.739656in}}%
\pgfpathlineto{\pgfqpoint{5.911555in}{0.739656in}}%
\pgfpathlineto{\pgfqpoint{5.911257in}{0.739656in}}%
\pgfpathlineto{\pgfqpoint{5.910960in}{0.739656in}}%
\pgfpathlineto{\pgfqpoint{5.910663in}{0.739656in}}%
\pgfpathlineto{\pgfqpoint{5.910365in}{0.739656in}}%
\pgfpathlineto{\pgfqpoint{5.910068in}{0.739656in}}%
\pgfpathlineto{\pgfqpoint{5.909770in}{0.739656in}}%
\pgfpathlineto{\pgfqpoint{5.909473in}{0.739656in}}%
\pgfpathlineto{\pgfqpoint{5.909175in}{0.739656in}}%
\pgfpathlineto{\pgfqpoint{5.908878in}{0.739656in}}%
\pgfpathlineto{\pgfqpoint{5.908580in}{0.739656in}}%
\pgfpathlineto{\pgfqpoint{5.908283in}{0.739656in}}%
\pgfpathlineto{\pgfqpoint{5.907985in}{0.739656in}}%
\pgfpathlineto{\pgfqpoint{5.907688in}{0.739656in}}%
\pgfpathlineto{\pgfqpoint{5.907390in}{0.739656in}}%
\pgfpathlineto{\pgfqpoint{5.907093in}{0.739656in}}%
\pgfpathlineto{\pgfqpoint{5.906795in}{0.739656in}}%
\pgfpathlineto{\pgfqpoint{5.906498in}{0.739656in}}%
\pgfpathlineto{\pgfqpoint{5.906200in}{0.739656in}}%
\pgfpathlineto{\pgfqpoint{5.905903in}{0.739656in}}%
\pgfpathlineto{\pgfqpoint{5.905605in}{0.739656in}}%
\pgfpathlineto{\pgfqpoint{5.905308in}{0.739656in}}%
\pgfpathlineto{\pgfqpoint{5.905010in}{0.739656in}}%
\pgfpathlineto{\pgfqpoint{5.904713in}{0.739656in}}%
\pgfpathlineto{\pgfqpoint{5.904415in}{0.739656in}}%
\pgfpathlineto{\pgfqpoint{5.904118in}{0.739656in}}%
\pgfpathlineto{\pgfqpoint{5.903821in}{0.739656in}}%
\pgfpathlineto{\pgfqpoint{5.903523in}{0.739656in}}%
\pgfpathlineto{\pgfqpoint{5.903226in}{0.739656in}}%
\pgfpathlineto{\pgfqpoint{5.902928in}{0.739656in}}%
\pgfpathlineto{\pgfqpoint{5.902631in}{0.739656in}}%
\pgfpathlineto{\pgfqpoint{5.902333in}{0.739656in}}%
\pgfpathlineto{\pgfqpoint{5.902036in}{0.739656in}}%
\pgfpathlineto{\pgfqpoint{5.901738in}{0.739656in}}%
\pgfpathlineto{\pgfqpoint{5.901441in}{0.739656in}}%
\pgfpathlineto{\pgfqpoint{5.901143in}{0.739656in}}%
\pgfpathlineto{\pgfqpoint{5.900846in}{0.739656in}}%
\pgfpathlineto{\pgfqpoint{5.900548in}{0.739656in}}%
\pgfpathlineto{\pgfqpoint{5.900251in}{0.739656in}}%
\pgfpathlineto{\pgfqpoint{5.899953in}{0.739656in}}%
\pgfpathlineto{\pgfqpoint{5.899656in}{0.739656in}}%
\pgfpathlineto{\pgfqpoint{5.899358in}{0.739656in}}%
\pgfpathlineto{\pgfqpoint{5.899061in}{0.739656in}}%
\pgfpathlineto{\pgfqpoint{5.898763in}{0.739656in}}%
\pgfpathlineto{\pgfqpoint{5.898466in}{0.739656in}}%
\pgfpathlineto{\pgfqpoint{5.898168in}{0.739656in}}%
\pgfpathlineto{\pgfqpoint{5.897871in}{0.739656in}}%
\pgfpathlineto{\pgfqpoint{5.897573in}{0.739656in}}%
\pgfpathlineto{\pgfqpoint{5.897276in}{0.739656in}}%
\pgfpathlineto{\pgfqpoint{5.896979in}{0.739656in}}%
\pgfpathlineto{\pgfqpoint{5.896681in}{0.739656in}}%
\pgfpathlineto{\pgfqpoint{5.896384in}{0.739656in}}%
\pgfpathlineto{\pgfqpoint{5.896086in}{0.739656in}}%
\pgfpathlineto{\pgfqpoint{5.895789in}{0.739656in}}%
\pgfpathlineto{\pgfqpoint{5.895491in}{0.739656in}}%
\pgfpathlineto{\pgfqpoint{5.895194in}{0.739656in}}%
\pgfpathlineto{\pgfqpoint{5.894896in}{0.739656in}}%
\pgfpathlineto{\pgfqpoint{5.894599in}{0.739656in}}%
\pgfpathlineto{\pgfqpoint{5.894301in}{0.739656in}}%
\pgfpathlineto{\pgfqpoint{5.894004in}{0.739656in}}%
\pgfpathlineto{\pgfqpoint{5.893706in}{0.739656in}}%
\pgfpathlineto{\pgfqpoint{5.893409in}{0.739656in}}%
\pgfpathlineto{\pgfqpoint{5.893111in}{0.739656in}}%
\pgfpathlineto{\pgfqpoint{5.892814in}{0.739656in}}%
\pgfpathlineto{\pgfqpoint{5.892516in}{0.739656in}}%
\pgfpathlineto{\pgfqpoint{5.892219in}{0.739656in}}%
\pgfpathlineto{\pgfqpoint{5.891921in}{0.739656in}}%
\pgfpathlineto{\pgfqpoint{5.891624in}{0.739656in}}%
\pgfpathlineto{\pgfqpoint{5.891326in}{0.739656in}}%
\pgfpathlineto{\pgfqpoint{5.891029in}{0.739656in}}%
\pgfpathlineto{\pgfqpoint{5.890731in}{0.739656in}}%
\pgfpathlineto{\pgfqpoint{5.890434in}{0.739656in}}%
\pgfpathlineto{\pgfqpoint{5.890137in}{0.739656in}}%
\pgfpathlineto{\pgfqpoint{5.889839in}{0.739656in}}%
\pgfpathlineto{\pgfqpoint{5.889542in}{0.739656in}}%
\pgfpathlineto{\pgfqpoint{5.889244in}{0.739656in}}%
\pgfpathlineto{\pgfqpoint{5.888947in}{0.739656in}}%
\pgfpathlineto{\pgfqpoint{5.888649in}{0.739656in}}%
\pgfpathlineto{\pgfqpoint{5.888352in}{0.739656in}}%
\pgfpathlineto{\pgfqpoint{5.888054in}{0.739656in}}%
\pgfpathlineto{\pgfqpoint{5.887757in}{0.739656in}}%
\pgfpathlineto{\pgfqpoint{5.887459in}{0.739656in}}%
\pgfpathlineto{\pgfqpoint{5.887162in}{0.739656in}}%
\pgfpathlineto{\pgfqpoint{5.886864in}{0.739656in}}%
\pgfpathlineto{\pgfqpoint{5.886567in}{0.739656in}}%
\pgfpathlineto{\pgfqpoint{5.886269in}{0.739656in}}%
\pgfpathlineto{\pgfqpoint{5.885972in}{0.739656in}}%
\pgfpathlineto{\pgfqpoint{5.885674in}{0.739656in}}%
\pgfpathlineto{\pgfqpoint{5.885377in}{0.739656in}}%
\pgfpathlineto{\pgfqpoint{5.885079in}{0.739656in}}%
\pgfpathlineto{\pgfqpoint{5.884782in}{0.739656in}}%
\pgfpathlineto{\pgfqpoint{5.884484in}{0.739656in}}%
\pgfpathlineto{\pgfqpoint{5.884187in}{0.739656in}}%
\pgfpathlineto{\pgfqpoint{5.883890in}{0.739656in}}%
\pgfpathlineto{\pgfqpoint{5.883592in}{0.739656in}}%
\pgfpathlineto{\pgfqpoint{5.883295in}{0.739656in}}%
\pgfpathlineto{\pgfqpoint{5.882997in}{0.739656in}}%
\pgfpathlineto{\pgfqpoint{5.882700in}{0.739656in}}%
\pgfpathlineto{\pgfqpoint{5.882402in}{0.739656in}}%
\pgfpathlineto{\pgfqpoint{5.882105in}{0.739656in}}%
\pgfpathlineto{\pgfqpoint{5.881807in}{0.739656in}}%
\pgfpathlineto{\pgfqpoint{5.881510in}{0.739656in}}%
\pgfpathlineto{\pgfqpoint{5.881212in}{0.739656in}}%
\pgfpathlineto{\pgfqpoint{5.880915in}{0.739656in}}%
\pgfpathlineto{\pgfqpoint{5.880617in}{0.739656in}}%
\pgfpathlineto{\pgfqpoint{5.880320in}{0.739656in}}%
\pgfpathlineto{\pgfqpoint{5.880022in}{0.739656in}}%
\pgfpathlineto{\pgfqpoint{5.879725in}{0.739656in}}%
\pgfpathlineto{\pgfqpoint{5.879427in}{0.739656in}}%
\pgfpathlineto{\pgfqpoint{5.879130in}{0.739656in}}%
\pgfpathlineto{\pgfqpoint{5.878832in}{0.739656in}}%
\pgfpathlineto{\pgfqpoint{5.878535in}{0.739656in}}%
\pgfpathlineto{\pgfqpoint{5.878237in}{0.739656in}}%
\pgfpathlineto{\pgfqpoint{5.877940in}{0.739656in}}%
\pgfpathlineto{\pgfqpoint{5.877642in}{0.739656in}}%
\pgfpathlineto{\pgfqpoint{5.877345in}{0.739656in}}%
\pgfpathlineto{\pgfqpoint{5.877048in}{0.739656in}}%
\pgfpathlineto{\pgfqpoint{5.876750in}{0.739656in}}%
\pgfpathlineto{\pgfqpoint{5.876453in}{0.739656in}}%
\pgfpathlineto{\pgfqpoint{5.876155in}{0.739656in}}%
\pgfpathlineto{\pgfqpoint{5.875858in}{0.739656in}}%
\pgfpathlineto{\pgfqpoint{5.875560in}{0.739656in}}%
\pgfpathlineto{\pgfqpoint{5.875263in}{0.739656in}}%
\pgfpathlineto{\pgfqpoint{5.874965in}{0.739656in}}%
\pgfpathlineto{\pgfqpoint{5.874668in}{0.739656in}}%
\pgfpathlineto{\pgfqpoint{5.874370in}{0.739656in}}%
\pgfpathlineto{\pgfqpoint{5.874073in}{0.739656in}}%
\pgfpathlineto{\pgfqpoint{5.873775in}{0.739656in}}%
\pgfpathlineto{\pgfqpoint{5.873478in}{0.739656in}}%
\pgfpathlineto{\pgfqpoint{5.873180in}{0.739656in}}%
\pgfpathlineto{\pgfqpoint{5.872883in}{0.739656in}}%
\pgfpathlineto{\pgfqpoint{5.872585in}{0.739656in}}%
\pgfpathlineto{\pgfqpoint{5.872288in}{0.739656in}}%
\pgfpathlineto{\pgfqpoint{5.871990in}{0.739656in}}%
\pgfpathlineto{\pgfqpoint{5.871693in}{0.739656in}}%
\pgfpathlineto{\pgfqpoint{5.871395in}{0.739656in}}%
\pgfpathlineto{\pgfqpoint{5.871098in}{0.739656in}}%
\pgfpathlineto{\pgfqpoint{5.870800in}{0.739656in}}%
\pgfpathlineto{\pgfqpoint{5.870503in}{0.739656in}}%
\pgfpathlineto{\pgfqpoint{5.870206in}{0.739656in}}%
\pgfpathlineto{\pgfqpoint{5.869908in}{0.739656in}}%
\pgfpathlineto{\pgfqpoint{5.869611in}{0.739656in}}%
\pgfpathlineto{\pgfqpoint{5.869313in}{0.739656in}}%
\pgfpathlineto{\pgfqpoint{5.869016in}{0.739656in}}%
\pgfpathlineto{\pgfqpoint{5.868718in}{0.739656in}}%
\pgfpathlineto{\pgfqpoint{5.868421in}{0.739656in}}%
\pgfpathlineto{\pgfqpoint{5.868123in}{0.739656in}}%
\pgfpathlineto{\pgfqpoint{5.867826in}{0.739656in}}%
\pgfpathlineto{\pgfqpoint{5.867528in}{0.739656in}}%
\pgfpathlineto{\pgfqpoint{5.867231in}{0.739656in}}%
\pgfpathlineto{\pgfqpoint{5.866933in}{0.739656in}}%
\pgfpathlineto{\pgfqpoint{5.866636in}{0.739656in}}%
\pgfpathlineto{\pgfqpoint{5.866338in}{0.739656in}}%
\pgfpathlineto{\pgfqpoint{5.866041in}{0.739656in}}%
\pgfpathlineto{\pgfqpoint{5.865743in}{0.739656in}}%
\pgfpathlineto{\pgfqpoint{5.865446in}{0.739656in}}%
\pgfpathlineto{\pgfqpoint{5.865148in}{0.739656in}}%
\pgfpathlineto{\pgfqpoint{5.864851in}{0.739656in}}%
\pgfpathlineto{\pgfqpoint{5.864553in}{0.739656in}}%
\pgfpathlineto{\pgfqpoint{5.864256in}{0.739656in}}%
\pgfpathlineto{\pgfqpoint{5.863959in}{0.739656in}}%
\pgfpathlineto{\pgfqpoint{5.863661in}{0.739656in}}%
\pgfpathlineto{\pgfqpoint{5.863364in}{0.739656in}}%
\pgfpathlineto{\pgfqpoint{5.863066in}{0.739656in}}%
\pgfpathlineto{\pgfqpoint{5.862769in}{0.739656in}}%
\pgfpathlineto{\pgfqpoint{5.862471in}{0.739656in}}%
\pgfpathlineto{\pgfqpoint{5.862174in}{0.739656in}}%
\pgfpathlineto{\pgfqpoint{5.861876in}{0.739656in}}%
\pgfpathlineto{\pgfqpoint{5.861579in}{0.739656in}}%
\pgfpathlineto{\pgfqpoint{5.861281in}{0.739656in}}%
\pgfpathlineto{\pgfqpoint{5.860984in}{0.739656in}}%
\pgfpathlineto{\pgfqpoint{5.860686in}{0.739656in}}%
\pgfpathlineto{\pgfqpoint{5.860389in}{0.739656in}}%
\pgfpathlineto{\pgfqpoint{5.860091in}{0.739656in}}%
\pgfpathlineto{\pgfqpoint{5.859794in}{0.739656in}}%
\pgfpathlineto{\pgfqpoint{5.859496in}{0.739656in}}%
\pgfpathlineto{\pgfqpoint{5.859199in}{0.739656in}}%
\pgfpathlineto{\pgfqpoint{5.858901in}{0.739656in}}%
\pgfpathlineto{\pgfqpoint{5.858604in}{0.739656in}}%
\pgfpathlineto{\pgfqpoint{5.858306in}{0.739656in}}%
\pgfpathlineto{\pgfqpoint{5.858009in}{0.739656in}}%
\pgfpathlineto{\pgfqpoint{5.857711in}{0.739656in}}%
\pgfpathlineto{\pgfqpoint{5.857414in}{0.739656in}}%
\pgfpathlineto{\pgfqpoint{5.857117in}{0.739656in}}%
\pgfpathlineto{\pgfqpoint{5.856819in}{0.739656in}}%
\pgfpathlineto{\pgfqpoint{5.856522in}{0.739656in}}%
\pgfpathlineto{\pgfqpoint{5.856224in}{0.739656in}}%
\pgfpathlineto{\pgfqpoint{5.855927in}{0.739656in}}%
\pgfpathlineto{\pgfqpoint{5.855629in}{0.739656in}}%
\pgfpathlineto{\pgfqpoint{5.855332in}{0.739656in}}%
\pgfpathlineto{\pgfqpoint{5.855034in}{0.739656in}}%
\pgfpathlineto{\pgfqpoint{5.854737in}{0.739656in}}%
\pgfpathlineto{\pgfqpoint{5.854439in}{0.739656in}}%
\pgfpathlineto{\pgfqpoint{5.854142in}{0.739656in}}%
\pgfpathlineto{\pgfqpoint{5.853844in}{0.739656in}}%
\pgfpathlineto{\pgfqpoint{5.853547in}{0.739656in}}%
\pgfpathlineto{\pgfqpoint{5.853249in}{0.739656in}}%
\pgfpathlineto{\pgfqpoint{5.852952in}{0.739656in}}%
\pgfpathlineto{\pgfqpoint{5.852654in}{0.739656in}}%
\pgfpathlineto{\pgfqpoint{5.852357in}{0.739656in}}%
\pgfpathlineto{\pgfqpoint{5.852059in}{0.739656in}}%
\pgfpathlineto{\pgfqpoint{5.851762in}{0.739656in}}%
\pgfpathlineto{\pgfqpoint{5.851464in}{0.739656in}}%
\pgfpathlineto{\pgfqpoint{5.851167in}{0.739656in}}%
\pgfpathlineto{\pgfqpoint{5.850869in}{0.739656in}}%
\pgfpathlineto{\pgfqpoint{5.850572in}{0.739656in}}%
\pgfpathlineto{\pgfqpoint{5.850275in}{0.739656in}}%
\pgfpathlineto{\pgfqpoint{5.849977in}{0.739656in}}%
\pgfpathlineto{\pgfqpoint{5.849680in}{0.739656in}}%
\pgfpathlineto{\pgfqpoint{5.849382in}{0.739656in}}%
\pgfpathlineto{\pgfqpoint{5.849085in}{0.739656in}}%
\pgfpathlineto{\pgfqpoint{5.848787in}{0.739656in}}%
\pgfpathlineto{\pgfqpoint{5.848490in}{0.739656in}}%
\pgfpathlineto{\pgfqpoint{5.848192in}{0.739656in}}%
\pgfpathlineto{\pgfqpoint{5.847895in}{0.739656in}}%
\pgfpathlineto{\pgfqpoint{5.847597in}{0.739656in}}%
\pgfpathlineto{\pgfqpoint{5.847300in}{0.739656in}}%
\pgfpathlineto{\pgfqpoint{5.847002in}{0.739656in}}%
\pgfpathlineto{\pgfqpoint{5.846705in}{0.739656in}}%
\pgfpathlineto{\pgfqpoint{5.846407in}{0.739656in}}%
\pgfpathlineto{\pgfqpoint{5.846110in}{0.739656in}}%
\pgfpathlineto{\pgfqpoint{5.845812in}{0.739656in}}%
\pgfpathlineto{\pgfqpoint{5.845515in}{0.739656in}}%
\pgfpathlineto{\pgfqpoint{5.845217in}{0.739656in}}%
\pgfpathlineto{\pgfqpoint{5.844920in}{0.739656in}}%
\pgfpathlineto{\pgfqpoint{5.844622in}{0.739656in}}%
\pgfpathlineto{\pgfqpoint{5.844325in}{0.739656in}}%
\pgfpathlineto{\pgfqpoint{5.844028in}{0.739656in}}%
\pgfpathlineto{\pgfqpoint{5.843730in}{0.739656in}}%
\pgfpathlineto{\pgfqpoint{5.843433in}{0.739656in}}%
\pgfpathlineto{\pgfqpoint{5.843135in}{0.739656in}}%
\pgfpathlineto{\pgfqpoint{5.842838in}{0.739656in}}%
\pgfpathlineto{\pgfqpoint{5.842540in}{0.739656in}}%
\pgfpathlineto{\pgfqpoint{5.842243in}{0.739656in}}%
\pgfpathlineto{\pgfqpoint{5.841945in}{0.739656in}}%
\pgfpathlineto{\pgfqpoint{5.841648in}{0.739656in}}%
\pgfpathlineto{\pgfqpoint{5.841350in}{0.739656in}}%
\pgfpathlineto{\pgfqpoint{5.841053in}{0.739656in}}%
\pgfpathlineto{\pgfqpoint{5.840755in}{0.739656in}}%
\pgfpathlineto{\pgfqpoint{5.840458in}{0.739656in}}%
\pgfpathlineto{\pgfqpoint{5.840160in}{0.739656in}}%
\pgfpathlineto{\pgfqpoint{5.839863in}{0.739656in}}%
\pgfpathlineto{\pgfqpoint{5.839565in}{0.739656in}}%
\pgfpathlineto{\pgfqpoint{5.839268in}{0.739656in}}%
\pgfpathlineto{\pgfqpoint{5.838970in}{0.739656in}}%
\pgfpathlineto{\pgfqpoint{5.838673in}{0.739656in}}%
\pgfpathlineto{\pgfqpoint{5.838375in}{0.739656in}}%
\pgfpathlineto{\pgfqpoint{5.838078in}{0.739656in}}%
\pgfpathlineto{\pgfqpoint{5.837780in}{0.739656in}}%
\pgfpathlineto{\pgfqpoint{5.837483in}{0.739656in}}%
\pgfpathlineto{\pgfqpoint{5.837186in}{0.739656in}}%
\pgfpathlineto{\pgfqpoint{5.836888in}{0.739656in}}%
\pgfpathlineto{\pgfqpoint{5.836591in}{0.739656in}}%
\pgfpathlineto{\pgfqpoint{5.836293in}{0.739656in}}%
\pgfpathlineto{\pgfqpoint{5.835996in}{0.739656in}}%
\pgfpathlineto{\pgfqpoint{5.835698in}{0.739656in}}%
\pgfpathlineto{\pgfqpoint{5.835401in}{0.739656in}}%
\pgfpathlineto{\pgfqpoint{5.835103in}{0.739656in}}%
\pgfpathlineto{\pgfqpoint{5.834806in}{0.739656in}}%
\pgfpathlineto{\pgfqpoint{5.834508in}{0.739656in}}%
\pgfpathlineto{\pgfqpoint{5.834211in}{0.739656in}}%
\pgfpathlineto{\pgfqpoint{5.833913in}{0.739656in}}%
\pgfpathlineto{\pgfqpoint{5.833616in}{0.739656in}}%
\pgfpathlineto{\pgfqpoint{5.833318in}{0.739656in}}%
\pgfpathlineto{\pgfqpoint{5.833021in}{0.739656in}}%
\pgfpathlineto{\pgfqpoint{5.832723in}{0.739656in}}%
\pgfpathlineto{\pgfqpoint{5.832426in}{0.739656in}}%
\pgfpathlineto{\pgfqpoint{5.832128in}{0.739656in}}%
\pgfpathlineto{\pgfqpoint{5.831831in}{0.739656in}}%
\pgfpathlineto{\pgfqpoint{5.831533in}{0.739656in}}%
\pgfpathlineto{\pgfqpoint{5.831236in}{0.739656in}}%
\pgfpathlineto{\pgfqpoint{5.830938in}{0.739656in}}%
\pgfpathlineto{\pgfqpoint{5.830641in}{0.739656in}}%
\pgfpathlineto{\pgfqpoint{5.830344in}{0.739656in}}%
\pgfpathlineto{\pgfqpoint{5.830046in}{0.739656in}}%
\pgfpathlineto{\pgfqpoint{5.829749in}{0.739656in}}%
\pgfpathlineto{\pgfqpoint{5.829451in}{0.739656in}}%
\pgfpathlineto{\pgfqpoint{5.829154in}{0.739656in}}%
\pgfpathlineto{\pgfqpoint{5.828856in}{0.739656in}}%
\pgfpathlineto{\pgfqpoint{5.828559in}{0.739656in}}%
\pgfpathlineto{\pgfqpoint{5.828261in}{0.739656in}}%
\pgfpathlineto{\pgfqpoint{5.827964in}{0.739656in}}%
\pgfpathlineto{\pgfqpoint{5.827666in}{0.739656in}}%
\pgfpathlineto{\pgfqpoint{5.827369in}{0.739656in}}%
\pgfpathlineto{\pgfqpoint{5.827071in}{0.739656in}}%
\pgfpathlineto{\pgfqpoint{5.826774in}{0.739656in}}%
\pgfpathlineto{\pgfqpoint{5.826476in}{0.739656in}}%
\pgfpathlineto{\pgfqpoint{5.826179in}{0.739656in}}%
\pgfpathlineto{\pgfqpoint{5.825881in}{0.739656in}}%
\pgfpathlineto{\pgfqpoint{5.825584in}{0.739656in}}%
\pgfpathlineto{\pgfqpoint{5.825286in}{0.739656in}}%
\pgfpathlineto{\pgfqpoint{5.824989in}{0.739656in}}%
\pgfpathlineto{\pgfqpoint{5.824691in}{0.739656in}}%
\pgfpathlineto{\pgfqpoint{5.824394in}{0.739656in}}%
\pgfpathlineto{\pgfqpoint{5.824097in}{0.739656in}}%
\pgfpathlineto{\pgfqpoint{5.823799in}{0.739656in}}%
\pgfpathlineto{\pgfqpoint{5.823502in}{0.739656in}}%
\pgfpathlineto{\pgfqpoint{5.823204in}{0.739656in}}%
\pgfpathlineto{\pgfqpoint{5.822907in}{0.739656in}}%
\pgfpathlineto{\pgfqpoint{5.822609in}{0.739656in}}%
\pgfpathlineto{\pgfqpoint{5.822312in}{0.739656in}}%
\pgfpathlineto{\pgfqpoint{5.822014in}{0.739656in}}%
\pgfpathlineto{\pgfqpoint{5.821717in}{0.739656in}}%
\pgfpathlineto{\pgfqpoint{5.821419in}{0.739656in}}%
\pgfpathlineto{\pgfqpoint{5.821122in}{0.739656in}}%
\pgfpathlineto{\pgfqpoint{5.820824in}{0.739656in}}%
\pgfpathlineto{\pgfqpoint{5.820527in}{0.739656in}}%
\pgfpathlineto{\pgfqpoint{5.820229in}{0.739656in}}%
\pgfpathlineto{\pgfqpoint{5.819932in}{0.739656in}}%
\pgfpathlineto{\pgfqpoint{5.819634in}{0.739656in}}%
\pgfpathlineto{\pgfqpoint{5.819337in}{0.739656in}}%
\pgfpathlineto{\pgfqpoint{5.819039in}{0.739656in}}%
\pgfpathlineto{\pgfqpoint{5.818742in}{0.739656in}}%
\pgfpathlineto{\pgfqpoint{5.818444in}{0.739656in}}%
\pgfpathlineto{\pgfqpoint{5.818147in}{0.739656in}}%
\pgfpathlineto{\pgfqpoint{5.817849in}{0.739656in}}%
\pgfpathlineto{\pgfqpoint{5.817552in}{0.739656in}}%
\pgfpathlineto{\pgfqpoint{5.817255in}{0.739656in}}%
\pgfpathlineto{\pgfqpoint{5.816957in}{0.739656in}}%
\pgfpathlineto{\pgfqpoint{5.816660in}{0.739656in}}%
\pgfpathlineto{\pgfqpoint{5.816362in}{0.739656in}}%
\pgfpathlineto{\pgfqpoint{5.816065in}{0.739656in}}%
\pgfpathlineto{\pgfqpoint{5.815767in}{0.739656in}}%
\pgfpathlineto{\pgfqpoint{5.815470in}{0.739656in}}%
\pgfpathlineto{\pgfqpoint{5.815172in}{0.739656in}}%
\pgfpathlineto{\pgfqpoint{5.814875in}{0.739656in}}%
\pgfpathlineto{\pgfqpoint{5.814577in}{0.739656in}}%
\pgfpathlineto{\pgfqpoint{5.814280in}{0.739656in}}%
\pgfpathlineto{\pgfqpoint{5.813982in}{0.739656in}}%
\pgfpathlineto{\pgfqpoint{5.813685in}{0.739656in}}%
\pgfpathlineto{\pgfqpoint{5.813387in}{0.739656in}}%
\pgfpathlineto{\pgfqpoint{5.813090in}{0.739656in}}%
\pgfpathlineto{\pgfqpoint{5.812792in}{0.739656in}}%
\pgfpathlineto{\pgfqpoint{5.812495in}{0.739656in}}%
\pgfpathlineto{\pgfqpoint{5.812197in}{0.739656in}}%
\pgfpathlineto{\pgfqpoint{5.811900in}{0.739656in}}%
\pgfpathlineto{\pgfqpoint{5.811602in}{0.739656in}}%
\pgfpathlineto{\pgfqpoint{5.811305in}{0.739656in}}%
\pgfpathlineto{\pgfqpoint{5.811007in}{0.739656in}}%
\pgfpathlineto{\pgfqpoint{5.810710in}{0.739656in}}%
\pgfpathlineto{\pgfqpoint{5.810413in}{0.739656in}}%
\pgfpathlineto{\pgfqpoint{5.810115in}{0.739656in}}%
\pgfpathlineto{\pgfqpoint{5.809818in}{0.739656in}}%
\pgfpathlineto{\pgfqpoint{5.809520in}{0.739656in}}%
\pgfpathlineto{\pgfqpoint{5.809223in}{0.739656in}}%
\pgfpathlineto{\pgfqpoint{5.808925in}{0.739656in}}%
\pgfpathlineto{\pgfqpoint{5.808628in}{0.739656in}}%
\pgfpathlineto{\pgfqpoint{5.808330in}{0.739656in}}%
\pgfpathlineto{\pgfqpoint{5.808033in}{0.739656in}}%
\pgfpathlineto{\pgfqpoint{5.807735in}{0.739656in}}%
\pgfpathlineto{\pgfqpoint{5.807438in}{0.739656in}}%
\pgfpathlineto{\pgfqpoint{5.807140in}{0.739656in}}%
\pgfpathlineto{\pgfqpoint{5.806843in}{0.739656in}}%
\pgfpathlineto{\pgfqpoint{5.806545in}{0.739656in}}%
\pgfpathlineto{\pgfqpoint{5.806248in}{0.739656in}}%
\pgfpathlineto{\pgfqpoint{5.805950in}{0.739656in}}%
\pgfpathlineto{\pgfqpoint{5.805653in}{0.739656in}}%
\pgfpathlineto{\pgfqpoint{5.805355in}{0.739656in}}%
\pgfpathlineto{\pgfqpoint{5.805058in}{0.739656in}}%
\pgfpathlineto{\pgfqpoint{5.804760in}{0.739656in}}%
\pgfpathlineto{\pgfqpoint{5.804463in}{0.739656in}}%
\pgfpathlineto{\pgfqpoint{5.804166in}{0.739656in}}%
\pgfpathlineto{\pgfqpoint{5.803868in}{0.739656in}}%
\pgfpathlineto{\pgfqpoint{5.803571in}{0.739656in}}%
\pgfpathlineto{\pgfqpoint{5.803273in}{0.739656in}}%
\pgfpathlineto{\pgfqpoint{5.802976in}{0.739656in}}%
\pgfpathlineto{\pgfqpoint{5.802678in}{0.739656in}}%
\pgfpathlineto{\pgfqpoint{5.802381in}{0.739656in}}%
\pgfpathlineto{\pgfqpoint{5.802083in}{0.739656in}}%
\pgfpathlineto{\pgfqpoint{5.801786in}{0.739656in}}%
\pgfpathlineto{\pgfqpoint{5.801488in}{0.739656in}}%
\pgfpathlineto{\pgfqpoint{5.801191in}{0.739656in}}%
\pgfpathlineto{\pgfqpoint{5.800893in}{0.739656in}}%
\pgfpathlineto{\pgfqpoint{5.800596in}{0.739656in}}%
\pgfpathlineto{\pgfqpoint{5.800298in}{0.739656in}}%
\pgfpathlineto{\pgfqpoint{5.800001in}{0.739656in}}%
\pgfpathlineto{\pgfqpoint{5.799703in}{0.739656in}}%
\pgfpathlineto{\pgfqpoint{5.799406in}{0.739656in}}%
\pgfpathlineto{\pgfqpoint{5.799108in}{0.739656in}}%
\pgfpathlineto{\pgfqpoint{5.798811in}{0.739656in}}%
\pgfpathlineto{\pgfqpoint{5.798513in}{0.739656in}}%
\pgfpathlineto{\pgfqpoint{5.798216in}{0.739656in}}%
\pgfpathlineto{\pgfqpoint{5.797918in}{0.739656in}}%
\pgfpathlineto{\pgfqpoint{5.797621in}{0.739656in}}%
\pgfpathlineto{\pgfqpoint{5.797324in}{0.739656in}}%
\pgfpathlineto{\pgfqpoint{5.797026in}{0.739656in}}%
\pgfpathlineto{\pgfqpoint{5.796729in}{0.739656in}}%
\pgfpathlineto{\pgfqpoint{5.796431in}{0.739656in}}%
\pgfpathlineto{\pgfqpoint{5.796134in}{0.739656in}}%
\pgfpathlineto{\pgfqpoint{5.795836in}{0.739656in}}%
\pgfpathlineto{\pgfqpoint{5.795539in}{0.739656in}}%
\pgfpathlineto{\pgfqpoint{5.795241in}{0.739656in}}%
\pgfpathlineto{\pgfqpoint{5.794944in}{0.739656in}}%
\pgfpathlineto{\pgfqpoint{5.794646in}{0.739656in}}%
\pgfpathlineto{\pgfqpoint{5.794349in}{0.739656in}}%
\pgfpathlineto{\pgfqpoint{5.794051in}{0.739656in}}%
\pgfpathlineto{\pgfqpoint{5.793754in}{0.739656in}}%
\pgfpathlineto{\pgfqpoint{5.793456in}{0.739656in}}%
\pgfpathlineto{\pgfqpoint{5.793159in}{0.739656in}}%
\pgfpathlineto{\pgfqpoint{5.792861in}{0.739656in}}%
\pgfpathlineto{\pgfqpoint{5.792564in}{0.739656in}}%
\pgfpathlineto{\pgfqpoint{5.792266in}{0.739656in}}%
\pgfpathlineto{\pgfqpoint{5.791969in}{0.739656in}}%
\pgfpathlineto{\pgfqpoint{5.791671in}{0.739656in}}%
\pgfpathlineto{\pgfqpoint{5.791374in}{0.739656in}}%
\pgfpathlineto{\pgfqpoint{5.791076in}{0.739656in}}%
\pgfpathlineto{\pgfqpoint{5.790779in}{0.739656in}}%
\pgfpathlineto{\pgfqpoint{5.790482in}{0.739656in}}%
\pgfpathlineto{\pgfqpoint{5.790184in}{0.739656in}}%
\pgfpathlineto{\pgfqpoint{5.789887in}{0.739656in}}%
\pgfpathlineto{\pgfqpoint{5.789589in}{0.739656in}}%
\pgfpathlineto{\pgfqpoint{5.789292in}{0.739656in}}%
\pgfpathlineto{\pgfqpoint{5.788994in}{0.739656in}}%
\pgfpathlineto{\pgfqpoint{5.788697in}{0.739656in}}%
\pgfpathlineto{\pgfqpoint{5.788399in}{0.739656in}}%
\pgfpathlineto{\pgfqpoint{5.788102in}{0.739656in}}%
\pgfpathlineto{\pgfqpoint{5.787804in}{0.739656in}}%
\pgfpathlineto{\pgfqpoint{5.787507in}{0.739656in}}%
\pgfpathlineto{\pgfqpoint{5.787209in}{0.739656in}}%
\pgfpathlineto{\pgfqpoint{5.786912in}{0.739656in}}%
\pgfpathlineto{\pgfqpoint{5.786614in}{0.739656in}}%
\pgfpathlineto{\pgfqpoint{5.786317in}{0.739656in}}%
\pgfpathlineto{\pgfqpoint{5.786019in}{0.739656in}}%
\pgfpathlineto{\pgfqpoint{5.785722in}{0.739656in}}%
\pgfpathlineto{\pgfqpoint{5.785424in}{0.739656in}}%
\pgfpathlineto{\pgfqpoint{5.785127in}{0.739656in}}%
\pgfpathlineto{\pgfqpoint{5.784829in}{0.739656in}}%
\pgfpathlineto{\pgfqpoint{5.784532in}{0.739656in}}%
\pgfpathlineto{\pgfqpoint{5.784235in}{0.739656in}}%
\pgfpathlineto{\pgfqpoint{5.783937in}{0.739656in}}%
\pgfpathlineto{\pgfqpoint{5.783640in}{0.739656in}}%
\pgfpathlineto{\pgfqpoint{5.783342in}{0.739656in}}%
\pgfpathlineto{\pgfqpoint{5.783045in}{0.739656in}}%
\pgfpathlineto{\pgfqpoint{5.782747in}{0.739656in}}%
\pgfpathlineto{\pgfqpoint{5.782450in}{0.739656in}}%
\pgfpathlineto{\pgfqpoint{5.782152in}{0.739656in}}%
\pgfpathlineto{\pgfqpoint{5.781855in}{0.739656in}}%
\pgfpathlineto{\pgfqpoint{5.781557in}{0.739656in}}%
\pgfpathlineto{\pgfqpoint{5.781260in}{0.739656in}}%
\pgfpathlineto{\pgfqpoint{5.780962in}{0.739656in}}%
\pgfpathlineto{\pgfqpoint{5.780665in}{0.739656in}}%
\pgfpathlineto{\pgfqpoint{5.780367in}{0.739656in}}%
\pgfpathlineto{\pgfqpoint{5.780070in}{0.739656in}}%
\pgfpathlineto{\pgfqpoint{5.779772in}{0.739656in}}%
\pgfpathlineto{\pgfqpoint{5.779475in}{0.739656in}}%
\pgfpathlineto{\pgfqpoint{5.779177in}{0.739656in}}%
\pgfpathlineto{\pgfqpoint{5.778880in}{0.739656in}}%
\pgfpathlineto{\pgfqpoint{5.778582in}{0.739656in}}%
\pgfpathlineto{\pgfqpoint{5.778285in}{0.739656in}}%
\pgfpathlineto{\pgfqpoint{5.777987in}{0.739656in}}%
\pgfpathlineto{\pgfqpoint{5.777690in}{0.739656in}}%
\pgfpathlineto{\pgfqpoint{5.777393in}{0.739656in}}%
\pgfpathlineto{\pgfqpoint{5.777095in}{0.739656in}}%
\pgfpathlineto{\pgfqpoint{5.776798in}{0.739656in}}%
\pgfpathlineto{\pgfqpoint{5.776500in}{0.739656in}}%
\pgfpathlineto{\pgfqpoint{5.776203in}{0.739656in}}%
\pgfpathlineto{\pgfqpoint{5.775905in}{0.739656in}}%
\pgfpathlineto{\pgfqpoint{5.775608in}{0.739656in}}%
\pgfpathlineto{\pgfqpoint{5.775310in}{0.739656in}}%
\pgfpathlineto{\pgfqpoint{5.775013in}{0.739656in}}%
\pgfpathlineto{\pgfqpoint{5.774715in}{0.739656in}}%
\pgfpathlineto{\pgfqpoint{5.774418in}{0.739656in}}%
\pgfpathlineto{\pgfqpoint{5.774120in}{0.739656in}}%
\pgfpathlineto{\pgfqpoint{5.773823in}{0.739656in}}%
\pgfpathlineto{\pgfqpoint{5.773525in}{0.739656in}}%
\pgfpathlineto{\pgfqpoint{5.773228in}{0.739656in}}%
\pgfpathlineto{\pgfqpoint{5.772930in}{0.739656in}}%
\pgfpathlineto{\pgfqpoint{5.772633in}{0.739656in}}%
\pgfpathlineto{\pgfqpoint{5.772335in}{0.739656in}}%
\pgfpathlineto{\pgfqpoint{5.772038in}{0.739656in}}%
\pgfpathlineto{\pgfqpoint{5.771740in}{0.739656in}}%
\pgfpathlineto{\pgfqpoint{5.771443in}{0.739656in}}%
\pgfpathlineto{\pgfqpoint{5.771145in}{0.739656in}}%
\pgfpathlineto{\pgfqpoint{5.770848in}{0.739656in}}%
\pgfpathlineto{\pgfqpoint{5.770551in}{0.739656in}}%
\pgfpathlineto{\pgfqpoint{5.770253in}{0.739656in}}%
\pgfpathlineto{\pgfqpoint{5.769956in}{0.739656in}}%
\pgfpathlineto{\pgfqpoint{5.769658in}{0.739656in}}%
\pgfpathlineto{\pgfqpoint{5.769361in}{0.739656in}}%
\pgfpathlineto{\pgfqpoint{5.769063in}{0.739656in}}%
\pgfpathlineto{\pgfqpoint{5.768766in}{0.739656in}}%
\pgfpathlineto{\pgfqpoint{5.768468in}{0.739656in}}%
\pgfpathlineto{\pgfqpoint{5.768171in}{0.739656in}}%
\pgfpathlineto{\pgfqpoint{5.767873in}{0.739656in}}%
\pgfpathlineto{\pgfqpoint{5.767576in}{0.739656in}}%
\pgfpathlineto{\pgfqpoint{5.767278in}{0.739656in}}%
\pgfpathlineto{\pgfqpoint{5.766981in}{0.739656in}}%
\pgfpathlineto{\pgfqpoint{5.766683in}{0.739656in}}%
\pgfpathlineto{\pgfqpoint{5.766386in}{0.739656in}}%
\pgfpathlineto{\pgfqpoint{5.766088in}{0.739656in}}%
\pgfpathlineto{\pgfqpoint{5.765791in}{0.739656in}}%
\pgfpathlineto{\pgfqpoint{5.765493in}{0.739656in}}%
\pgfpathlineto{\pgfqpoint{5.765196in}{0.739656in}}%
\pgfpathlineto{\pgfqpoint{5.764898in}{0.739656in}}%
\pgfpathlineto{\pgfqpoint{5.764601in}{0.739656in}}%
\pgfpathlineto{\pgfqpoint{5.764304in}{0.739656in}}%
\pgfpathlineto{\pgfqpoint{5.764006in}{0.739656in}}%
\pgfpathlineto{\pgfqpoint{5.763709in}{0.739656in}}%
\pgfpathlineto{\pgfqpoint{5.763411in}{0.739656in}}%
\pgfpathlineto{\pgfqpoint{5.763114in}{0.739656in}}%
\pgfpathlineto{\pgfqpoint{5.762816in}{0.739656in}}%
\pgfpathlineto{\pgfqpoint{5.762519in}{0.739656in}}%
\pgfpathlineto{\pgfqpoint{5.762221in}{0.739656in}}%
\pgfpathlineto{\pgfqpoint{5.761924in}{0.739656in}}%
\pgfpathlineto{\pgfqpoint{5.761626in}{0.739656in}}%
\pgfpathlineto{\pgfqpoint{5.761329in}{0.739656in}}%
\pgfpathlineto{\pgfqpoint{5.761031in}{0.739656in}}%
\pgfpathlineto{\pgfqpoint{5.760734in}{0.739656in}}%
\pgfpathlineto{\pgfqpoint{5.760436in}{0.739656in}}%
\pgfpathlineto{\pgfqpoint{5.760139in}{0.739656in}}%
\pgfpathlineto{\pgfqpoint{5.759841in}{0.739656in}}%
\pgfpathlineto{\pgfqpoint{5.759544in}{0.739656in}}%
\pgfpathlineto{\pgfqpoint{5.759246in}{0.739656in}}%
\pgfpathlineto{\pgfqpoint{5.758949in}{0.739656in}}%
\pgfpathlineto{\pgfqpoint{5.758651in}{0.739656in}}%
\pgfpathlineto{\pgfqpoint{5.758354in}{0.739656in}}%
\pgfpathlineto{\pgfqpoint{5.758056in}{0.739656in}}%
\pgfpathlineto{\pgfqpoint{5.757759in}{0.739656in}}%
\pgfpathlineto{\pgfqpoint{5.757462in}{0.739656in}}%
\pgfpathlineto{\pgfqpoint{5.757164in}{0.739656in}}%
\pgfpathlineto{\pgfqpoint{5.756867in}{0.739656in}}%
\pgfpathlineto{\pgfqpoint{5.756569in}{0.739656in}}%
\pgfpathlineto{\pgfqpoint{5.756272in}{0.739656in}}%
\pgfpathlineto{\pgfqpoint{5.755974in}{0.739656in}}%
\pgfpathlineto{\pgfqpoint{5.755677in}{0.739656in}}%
\pgfpathlineto{\pgfqpoint{5.755379in}{0.739656in}}%
\pgfpathlineto{\pgfqpoint{5.755082in}{0.739656in}}%
\pgfpathlineto{\pgfqpoint{5.754784in}{0.739656in}}%
\pgfpathlineto{\pgfqpoint{5.754487in}{0.739656in}}%
\pgfpathlineto{\pgfqpoint{5.754189in}{0.739656in}}%
\pgfpathlineto{\pgfqpoint{5.753892in}{0.739656in}}%
\pgfpathlineto{\pgfqpoint{5.753594in}{0.739656in}}%
\pgfpathlineto{\pgfqpoint{5.753297in}{0.739656in}}%
\pgfpathlineto{\pgfqpoint{5.752999in}{0.739656in}}%
\pgfpathlineto{\pgfqpoint{5.752702in}{0.739656in}}%
\pgfpathlineto{\pgfqpoint{5.752404in}{0.739656in}}%
\pgfpathlineto{\pgfqpoint{5.752107in}{0.739656in}}%
\pgfpathlineto{\pgfqpoint{5.751809in}{0.739656in}}%
\pgfpathlineto{\pgfqpoint{5.751512in}{0.739656in}}%
\pgfpathlineto{\pgfqpoint{5.751214in}{0.739656in}}%
\pgfpathlineto{\pgfqpoint{5.750917in}{0.739656in}}%
\pgfpathlineto{\pgfqpoint{5.750620in}{0.739656in}}%
\pgfpathlineto{\pgfqpoint{5.750322in}{0.739656in}}%
\pgfpathlineto{\pgfqpoint{5.750025in}{0.739656in}}%
\pgfpathlineto{\pgfqpoint{5.749727in}{0.739656in}}%
\pgfpathlineto{\pgfqpoint{5.749430in}{0.739656in}}%
\pgfpathlineto{\pgfqpoint{5.749132in}{0.739656in}}%
\pgfpathlineto{\pgfqpoint{5.748835in}{0.739656in}}%
\pgfpathlineto{\pgfqpoint{5.748537in}{0.739656in}}%
\pgfpathlineto{\pgfqpoint{5.748240in}{0.739656in}}%
\pgfpathlineto{\pgfqpoint{5.747942in}{0.739656in}}%
\pgfpathlineto{\pgfqpoint{5.747645in}{0.739656in}}%
\pgfpathlineto{\pgfqpoint{5.747347in}{0.739656in}}%
\pgfpathlineto{\pgfqpoint{5.747050in}{0.739656in}}%
\pgfpathlineto{\pgfqpoint{5.746752in}{0.739656in}}%
\pgfpathlineto{\pgfqpoint{5.746455in}{0.739656in}}%
\pgfpathlineto{\pgfqpoint{5.746157in}{0.739656in}}%
\pgfpathlineto{\pgfqpoint{5.745860in}{0.739656in}}%
\pgfpathlineto{\pgfqpoint{5.745562in}{0.739656in}}%
\pgfpathlineto{\pgfqpoint{5.745265in}{0.739656in}}%
\pgfpathlineto{\pgfqpoint{5.744967in}{0.739656in}}%
\pgfpathlineto{\pgfqpoint{5.744670in}{0.739656in}}%
\pgfpathlineto{\pgfqpoint{5.744373in}{0.739656in}}%
\pgfpathlineto{\pgfqpoint{5.744075in}{0.739656in}}%
\pgfpathlineto{\pgfqpoint{5.743778in}{0.739656in}}%
\pgfpathlineto{\pgfqpoint{5.743480in}{0.739656in}}%
\pgfpathlineto{\pgfqpoint{5.743183in}{0.739656in}}%
\pgfpathlineto{\pgfqpoint{5.742885in}{0.739656in}}%
\pgfpathlineto{\pgfqpoint{5.742588in}{0.739656in}}%
\pgfpathlineto{\pgfqpoint{5.742290in}{0.739656in}}%
\pgfpathlineto{\pgfqpoint{5.741993in}{0.739656in}}%
\pgfpathlineto{\pgfqpoint{5.741695in}{0.739656in}}%
\pgfpathlineto{\pgfqpoint{5.741398in}{0.739656in}}%
\pgfpathlineto{\pgfqpoint{5.741100in}{0.739656in}}%
\pgfpathlineto{\pgfqpoint{5.740803in}{0.739656in}}%
\pgfpathlineto{\pgfqpoint{5.740505in}{0.739656in}}%
\pgfpathlineto{\pgfqpoint{5.740208in}{0.739656in}}%
\pgfpathlineto{\pgfqpoint{5.739910in}{0.739656in}}%
\pgfpathlineto{\pgfqpoint{5.739613in}{0.739656in}}%
\pgfpathlineto{\pgfqpoint{5.739315in}{0.739656in}}%
\pgfpathlineto{\pgfqpoint{5.739018in}{0.739656in}}%
\pgfpathlineto{\pgfqpoint{5.738720in}{0.739656in}}%
\pgfpathlineto{\pgfqpoint{5.738423in}{0.739656in}}%
\pgfpathlineto{\pgfqpoint{5.738125in}{0.739656in}}%
\pgfpathlineto{\pgfqpoint{5.737828in}{0.739656in}}%
\pgfpathlineto{\pgfqpoint{5.737531in}{0.739656in}}%
\pgfpathlineto{\pgfqpoint{5.737233in}{0.739656in}}%
\pgfpathlineto{\pgfqpoint{5.736936in}{0.739656in}}%
\pgfpathlineto{\pgfqpoint{5.736638in}{0.739656in}}%
\pgfpathlineto{\pgfqpoint{5.736341in}{0.739656in}}%
\pgfpathlineto{\pgfqpoint{5.736043in}{0.739656in}}%
\pgfpathlineto{\pgfqpoint{5.735746in}{0.739656in}}%
\pgfpathlineto{\pgfqpoint{5.735448in}{0.739656in}}%
\pgfpathlineto{\pgfqpoint{5.735151in}{0.739656in}}%
\pgfpathlineto{\pgfqpoint{5.734853in}{0.739656in}}%
\pgfpathlineto{\pgfqpoint{5.734556in}{0.739656in}}%
\pgfpathlineto{\pgfqpoint{5.734258in}{0.739656in}}%
\pgfpathlineto{\pgfqpoint{5.733961in}{0.739656in}}%
\pgfpathlineto{\pgfqpoint{5.733663in}{0.739656in}}%
\pgfpathlineto{\pgfqpoint{5.733366in}{0.739656in}}%
\pgfpathlineto{\pgfqpoint{5.733068in}{0.739656in}}%
\pgfpathlineto{\pgfqpoint{5.732771in}{0.739656in}}%
\pgfpathlineto{\pgfqpoint{5.732473in}{0.739656in}}%
\pgfpathlineto{\pgfqpoint{5.732176in}{0.739656in}}%
\pgfpathlineto{\pgfqpoint{5.731878in}{0.739656in}}%
\pgfpathlineto{\pgfqpoint{5.731581in}{0.739656in}}%
\pgfpathlineto{\pgfqpoint{5.731283in}{0.739656in}}%
\pgfpathlineto{\pgfqpoint{5.730986in}{0.739656in}}%
\pgfpathlineto{\pgfqpoint{5.730689in}{0.739656in}}%
\pgfpathlineto{\pgfqpoint{5.730391in}{0.739656in}}%
\pgfpathlineto{\pgfqpoint{5.730094in}{0.739656in}}%
\pgfpathlineto{\pgfqpoint{5.729796in}{0.739656in}}%
\pgfpathlineto{\pgfqpoint{5.729499in}{0.739656in}}%
\pgfpathlineto{\pgfqpoint{5.729201in}{0.739656in}}%
\pgfpathlineto{\pgfqpoint{5.728904in}{0.739656in}}%
\pgfpathlineto{\pgfqpoint{5.728606in}{0.739656in}}%
\pgfpathlineto{\pgfqpoint{5.728309in}{0.739656in}}%
\pgfpathlineto{\pgfqpoint{5.728011in}{0.739656in}}%
\pgfpathlineto{\pgfqpoint{5.727714in}{0.739656in}}%
\pgfpathlineto{\pgfqpoint{5.727416in}{0.739656in}}%
\pgfpathlineto{\pgfqpoint{5.727119in}{0.739656in}}%
\pgfpathlineto{\pgfqpoint{5.726821in}{0.739656in}}%
\pgfpathlineto{\pgfqpoint{5.726524in}{0.739656in}}%
\pgfpathlineto{\pgfqpoint{5.726226in}{0.739656in}}%
\pgfpathlineto{\pgfqpoint{5.725929in}{0.739656in}}%
\pgfpathlineto{\pgfqpoint{5.725631in}{0.739656in}}%
\pgfpathlineto{\pgfqpoint{5.725334in}{0.739656in}}%
\pgfpathlineto{\pgfqpoint{5.725036in}{0.739656in}}%
\pgfpathlineto{\pgfqpoint{5.724739in}{0.739656in}}%
\pgfpathlineto{\pgfqpoint{5.724442in}{0.739656in}}%
\pgfpathlineto{\pgfqpoint{5.724144in}{0.739656in}}%
\pgfpathlineto{\pgfqpoint{5.723847in}{0.739656in}}%
\pgfpathlineto{\pgfqpoint{5.723549in}{0.739656in}}%
\pgfpathlineto{\pgfqpoint{5.723252in}{0.739656in}}%
\pgfpathlineto{\pgfqpoint{5.722954in}{0.739656in}}%
\pgfpathlineto{\pgfqpoint{5.722657in}{0.739656in}}%
\pgfpathlineto{\pgfqpoint{5.722359in}{0.739656in}}%
\pgfpathlineto{\pgfqpoint{5.722062in}{0.739656in}}%
\pgfpathlineto{\pgfqpoint{5.721764in}{0.739656in}}%
\pgfpathlineto{\pgfqpoint{5.721467in}{0.739656in}}%
\pgfpathlineto{\pgfqpoint{5.721169in}{0.739656in}}%
\pgfpathlineto{\pgfqpoint{5.720872in}{0.739656in}}%
\pgfpathlineto{\pgfqpoint{5.720574in}{0.739656in}}%
\pgfpathlineto{\pgfqpoint{5.720277in}{0.739656in}}%
\pgfpathlineto{\pgfqpoint{5.719979in}{0.739656in}}%
\pgfpathlineto{\pgfqpoint{5.719682in}{0.739656in}}%
\pgfpathlineto{\pgfqpoint{5.719384in}{0.739656in}}%
\pgfpathlineto{\pgfqpoint{5.719087in}{0.739656in}}%
\pgfpathlineto{\pgfqpoint{5.718789in}{0.739656in}}%
\pgfpathlineto{\pgfqpoint{5.718492in}{0.739656in}}%
\pgfpathlineto{\pgfqpoint{5.718194in}{0.739656in}}%
\pgfpathlineto{\pgfqpoint{5.717897in}{0.739656in}}%
\pgfpathlineto{\pgfqpoint{5.717600in}{0.739656in}}%
\pgfpathlineto{\pgfqpoint{5.717302in}{0.739656in}}%
\pgfpathlineto{\pgfqpoint{5.717005in}{0.739656in}}%
\pgfpathlineto{\pgfqpoint{5.716707in}{0.739656in}}%
\pgfpathlineto{\pgfqpoint{5.716410in}{0.739656in}}%
\pgfpathlineto{\pgfqpoint{5.716112in}{0.739656in}}%
\pgfpathlineto{\pgfqpoint{5.715815in}{0.739656in}}%
\pgfpathlineto{\pgfqpoint{5.715517in}{0.739656in}}%
\pgfpathlineto{\pgfqpoint{5.715220in}{0.739656in}}%
\pgfpathlineto{\pgfqpoint{5.714922in}{0.739656in}}%
\pgfpathlineto{\pgfqpoint{5.714625in}{0.739656in}}%
\pgfpathlineto{\pgfqpoint{5.714327in}{0.739656in}}%
\pgfpathlineto{\pgfqpoint{5.714030in}{0.739656in}}%
\pgfpathlineto{\pgfqpoint{5.713732in}{0.739656in}}%
\pgfpathlineto{\pgfqpoint{5.713435in}{0.739656in}}%
\pgfpathlineto{\pgfqpoint{5.713137in}{0.739656in}}%
\pgfpathlineto{\pgfqpoint{5.712840in}{0.739656in}}%
\pgfpathlineto{\pgfqpoint{5.712542in}{0.739656in}}%
\pgfpathlineto{\pgfqpoint{5.712245in}{0.739656in}}%
\pgfpathlineto{\pgfqpoint{5.711947in}{0.739656in}}%
\pgfpathlineto{\pgfqpoint{5.711650in}{0.739656in}}%
\pgfpathlineto{\pgfqpoint{5.711352in}{0.739656in}}%
\pgfpathlineto{\pgfqpoint{5.711055in}{0.739656in}}%
\pgfpathlineto{\pgfqpoint{5.710758in}{0.739656in}}%
\pgfpathlineto{\pgfqpoint{5.710460in}{0.739656in}}%
\pgfpathlineto{\pgfqpoint{5.710163in}{0.739656in}}%
\pgfpathlineto{\pgfqpoint{5.709865in}{0.739656in}}%
\pgfpathlineto{\pgfqpoint{5.709568in}{0.739656in}}%
\pgfpathlineto{\pgfqpoint{5.709270in}{0.739656in}}%
\pgfpathlineto{\pgfqpoint{5.708973in}{0.739656in}}%
\pgfpathlineto{\pgfqpoint{5.708675in}{0.739656in}}%
\pgfpathlineto{\pgfqpoint{5.708378in}{0.739656in}}%
\pgfpathlineto{\pgfqpoint{5.708080in}{0.739656in}}%
\pgfpathlineto{\pgfqpoint{5.707783in}{0.739656in}}%
\pgfpathlineto{\pgfqpoint{5.707485in}{0.739656in}}%
\pgfpathlineto{\pgfqpoint{5.707188in}{0.739656in}}%
\pgfpathlineto{\pgfqpoint{5.706890in}{0.739656in}}%
\pgfpathlineto{\pgfqpoint{5.706593in}{0.739656in}}%
\pgfpathlineto{\pgfqpoint{5.706295in}{0.739656in}}%
\pgfpathlineto{\pgfqpoint{5.705998in}{0.739656in}}%
\pgfpathlineto{\pgfqpoint{5.705700in}{0.739656in}}%
\pgfpathlineto{\pgfqpoint{5.705403in}{0.739656in}}%
\pgfpathlineto{\pgfqpoint{5.705105in}{0.739656in}}%
\pgfpathlineto{\pgfqpoint{5.704808in}{0.739656in}}%
\pgfpathlineto{\pgfqpoint{5.704511in}{0.739656in}}%
\pgfpathlineto{\pgfqpoint{5.704213in}{0.739656in}}%
\pgfpathlineto{\pgfqpoint{5.703916in}{0.739656in}}%
\pgfpathlineto{\pgfqpoint{5.703618in}{0.739656in}}%
\pgfpathlineto{\pgfqpoint{5.703321in}{0.739656in}}%
\pgfpathlineto{\pgfqpoint{5.703023in}{0.739656in}}%
\pgfpathlineto{\pgfqpoint{5.702726in}{0.739656in}}%
\pgfpathlineto{\pgfqpoint{5.702428in}{0.739656in}}%
\pgfpathlineto{\pgfqpoint{5.702131in}{0.739656in}}%
\pgfpathlineto{\pgfqpoint{5.701833in}{0.739656in}}%
\pgfpathlineto{\pgfqpoint{5.701536in}{0.739656in}}%
\pgfpathlineto{\pgfqpoint{5.701238in}{0.739656in}}%
\pgfpathlineto{\pgfqpoint{5.700941in}{0.739656in}}%
\pgfpathlineto{\pgfqpoint{5.700643in}{0.739656in}}%
\pgfpathlineto{\pgfqpoint{5.700346in}{0.739656in}}%
\pgfpathlineto{\pgfqpoint{5.700048in}{0.739656in}}%
\pgfpathlineto{\pgfqpoint{5.699751in}{0.739656in}}%
\pgfpathlineto{\pgfqpoint{5.699453in}{0.739656in}}%
\pgfpathlineto{\pgfqpoint{5.699156in}{0.739656in}}%
\pgfpathlineto{\pgfqpoint{5.698858in}{0.739656in}}%
\pgfpathlineto{\pgfqpoint{5.698561in}{0.739656in}}%
\pgfpathlineto{\pgfqpoint{5.698263in}{0.739656in}}%
\pgfpathlineto{\pgfqpoint{5.697966in}{0.739656in}}%
\pgfpathlineto{\pgfqpoint{5.697669in}{0.739656in}}%
\pgfpathlineto{\pgfqpoint{5.697371in}{0.739656in}}%
\pgfpathlineto{\pgfqpoint{5.697074in}{0.739656in}}%
\pgfpathlineto{\pgfqpoint{5.696776in}{0.739656in}}%
\pgfpathlineto{\pgfqpoint{5.696479in}{0.739656in}}%
\pgfpathlineto{\pgfqpoint{5.696181in}{0.739656in}}%
\pgfpathlineto{\pgfqpoint{5.695884in}{0.739656in}}%
\pgfpathlineto{\pgfqpoint{5.695586in}{0.739656in}}%
\pgfpathlineto{\pgfqpoint{5.695289in}{0.739656in}}%
\pgfpathlineto{\pgfqpoint{5.694991in}{0.739656in}}%
\pgfpathlineto{\pgfqpoint{5.694694in}{0.739656in}}%
\pgfpathlineto{\pgfqpoint{5.694396in}{0.739656in}}%
\pgfpathlineto{\pgfqpoint{5.694099in}{0.739656in}}%
\pgfpathlineto{\pgfqpoint{5.693801in}{0.739656in}}%
\pgfpathlineto{\pgfqpoint{5.693504in}{0.739656in}}%
\pgfpathlineto{\pgfqpoint{5.693206in}{0.739656in}}%
\pgfpathlineto{\pgfqpoint{5.692909in}{0.739656in}}%
\pgfpathlineto{\pgfqpoint{5.692611in}{0.739656in}}%
\pgfpathlineto{\pgfqpoint{5.692314in}{0.739656in}}%
\pgfpathlineto{\pgfqpoint{5.692016in}{0.739656in}}%
\pgfpathlineto{\pgfqpoint{5.691719in}{0.739656in}}%
\pgfpathlineto{\pgfqpoint{5.691421in}{0.739656in}}%
\pgfpathlineto{\pgfqpoint{5.691124in}{0.739656in}}%
\pgfpathlineto{\pgfqpoint{5.690827in}{0.739656in}}%
\pgfpathlineto{\pgfqpoint{5.690529in}{0.739656in}}%
\pgfpathlineto{\pgfqpoint{5.690232in}{0.739656in}}%
\pgfpathlineto{\pgfqpoint{5.689934in}{0.739656in}}%
\pgfpathlineto{\pgfqpoint{5.689637in}{0.739656in}}%
\pgfpathlineto{\pgfqpoint{5.689339in}{0.739656in}}%
\pgfpathlineto{\pgfqpoint{5.689042in}{0.739656in}}%
\pgfpathlineto{\pgfqpoint{5.688744in}{0.739656in}}%
\pgfpathlineto{\pgfqpoint{5.688447in}{0.739656in}}%
\pgfpathlineto{\pgfqpoint{5.688149in}{0.739656in}}%
\pgfpathlineto{\pgfqpoint{5.687852in}{0.739656in}}%
\pgfpathlineto{\pgfqpoint{5.687554in}{0.739656in}}%
\pgfpathlineto{\pgfqpoint{5.687257in}{0.739656in}}%
\pgfpathlineto{\pgfqpoint{5.686959in}{0.739656in}}%
\pgfpathlineto{\pgfqpoint{5.686662in}{0.739656in}}%
\pgfpathlineto{\pgfqpoint{5.686364in}{0.739656in}}%
\pgfpathlineto{\pgfqpoint{5.686067in}{0.739656in}}%
\pgfpathlineto{\pgfqpoint{5.685769in}{0.739656in}}%
\pgfpathlineto{\pgfqpoint{5.685472in}{0.739656in}}%
\pgfpathlineto{\pgfqpoint{5.685174in}{0.739656in}}%
\pgfpathlineto{\pgfqpoint{5.684877in}{0.739656in}}%
\pgfpathlineto{\pgfqpoint{5.684579in}{0.739656in}}%
\pgfpathlineto{\pgfqpoint{5.684282in}{0.739656in}}%
\pgfpathlineto{\pgfqpoint{5.683985in}{0.739656in}}%
\pgfpathlineto{\pgfqpoint{5.683687in}{0.739656in}}%
\pgfpathlineto{\pgfqpoint{5.683390in}{0.739656in}}%
\pgfpathlineto{\pgfqpoint{5.683092in}{0.739656in}}%
\pgfpathlineto{\pgfqpoint{5.682795in}{0.739656in}}%
\pgfpathlineto{\pgfqpoint{5.682497in}{0.739656in}}%
\pgfpathlineto{\pgfqpoint{5.682200in}{0.739656in}}%
\pgfpathlineto{\pgfqpoint{5.681902in}{0.739656in}}%
\pgfpathlineto{\pgfqpoint{5.681605in}{0.739656in}}%
\pgfpathlineto{\pgfqpoint{5.681307in}{0.739656in}}%
\pgfpathlineto{\pgfqpoint{5.681010in}{0.739656in}}%
\pgfpathlineto{\pgfqpoint{5.680712in}{0.739656in}}%
\pgfpathlineto{\pgfqpoint{5.680415in}{0.739656in}}%
\pgfpathlineto{\pgfqpoint{5.680117in}{0.739656in}}%
\pgfpathlineto{\pgfqpoint{5.679820in}{0.739656in}}%
\pgfpathlineto{\pgfqpoint{5.679522in}{0.739656in}}%
\pgfpathlineto{\pgfqpoint{5.679225in}{0.739656in}}%
\pgfpathlineto{\pgfqpoint{5.678927in}{0.739656in}}%
\pgfpathlineto{\pgfqpoint{5.678630in}{0.739656in}}%
\pgfpathlineto{\pgfqpoint{5.678332in}{0.739656in}}%
\pgfpathlineto{\pgfqpoint{5.678035in}{0.739656in}}%
\pgfpathlineto{\pgfqpoint{5.677738in}{0.739656in}}%
\pgfpathlineto{\pgfqpoint{5.677440in}{0.739656in}}%
\pgfpathlineto{\pgfqpoint{5.677143in}{0.739656in}}%
\pgfpathlineto{\pgfqpoint{5.676845in}{0.739656in}}%
\pgfpathlineto{\pgfqpoint{5.676548in}{0.739656in}}%
\pgfpathlineto{\pgfqpoint{5.676250in}{0.739656in}}%
\pgfpathlineto{\pgfqpoint{5.675953in}{0.739656in}}%
\pgfpathlineto{\pgfqpoint{5.675655in}{0.739656in}}%
\pgfpathlineto{\pgfqpoint{5.675358in}{0.739656in}}%
\pgfpathlineto{\pgfqpoint{5.675060in}{0.739656in}}%
\pgfpathlineto{\pgfqpoint{5.674763in}{0.739656in}}%
\pgfpathlineto{\pgfqpoint{5.674465in}{0.739656in}}%
\pgfpathlineto{\pgfqpoint{5.674168in}{0.739656in}}%
\pgfpathlineto{\pgfqpoint{5.673870in}{0.739656in}}%
\pgfpathlineto{\pgfqpoint{5.673573in}{0.739656in}}%
\pgfpathlineto{\pgfqpoint{5.673275in}{0.739656in}}%
\pgfpathlineto{\pgfqpoint{5.672978in}{0.739656in}}%
\pgfpathlineto{\pgfqpoint{5.672680in}{0.739656in}}%
\pgfpathlineto{\pgfqpoint{5.672383in}{0.739656in}}%
\pgfpathlineto{\pgfqpoint{5.672085in}{0.739656in}}%
\pgfpathlineto{\pgfqpoint{5.671788in}{0.739656in}}%
\pgfpathlineto{\pgfqpoint{5.671490in}{0.739656in}}%
\pgfpathlineto{\pgfqpoint{5.671193in}{0.739656in}}%
\pgfpathlineto{\pgfqpoint{5.670896in}{0.739656in}}%
\pgfpathlineto{\pgfqpoint{5.670598in}{0.739656in}}%
\pgfpathlineto{\pgfqpoint{5.670301in}{0.739656in}}%
\pgfpathlineto{\pgfqpoint{5.670003in}{0.739656in}}%
\pgfpathlineto{\pgfqpoint{5.669706in}{0.739656in}}%
\pgfpathlineto{\pgfqpoint{5.669408in}{0.739656in}}%
\pgfpathlineto{\pgfqpoint{5.669111in}{0.739656in}}%
\pgfpathlineto{\pgfqpoint{5.668813in}{0.739656in}}%
\pgfpathlineto{\pgfqpoint{5.668516in}{0.739656in}}%
\pgfpathlineto{\pgfqpoint{5.668218in}{0.739656in}}%
\pgfpathlineto{\pgfqpoint{5.667921in}{0.739656in}}%
\pgfpathlineto{\pgfqpoint{5.667623in}{0.739656in}}%
\pgfpathlineto{\pgfqpoint{5.667326in}{0.739656in}}%
\pgfpathlineto{\pgfqpoint{5.667028in}{0.739656in}}%
\pgfpathlineto{\pgfqpoint{5.666731in}{0.739656in}}%
\pgfpathlineto{\pgfqpoint{5.666433in}{0.739656in}}%
\pgfpathlineto{\pgfqpoint{5.666136in}{0.739656in}}%
\pgfpathlineto{\pgfqpoint{5.665838in}{0.739656in}}%
\pgfpathlineto{\pgfqpoint{5.665541in}{0.739656in}}%
\pgfpathlineto{\pgfqpoint{5.665243in}{0.739656in}}%
\pgfpathlineto{\pgfqpoint{5.664946in}{0.739656in}}%
\pgfpathlineto{\pgfqpoint{5.664648in}{0.739656in}}%
\pgfpathlineto{\pgfqpoint{5.664351in}{0.739656in}}%
\pgfpathlineto{\pgfqpoint{5.664054in}{0.739656in}}%
\pgfpathlineto{\pgfqpoint{5.663756in}{0.739656in}}%
\pgfpathlineto{\pgfqpoint{5.663459in}{0.739656in}}%
\pgfpathlineto{\pgfqpoint{5.663161in}{0.739656in}}%
\pgfpathlineto{\pgfqpoint{5.662864in}{0.739656in}}%
\pgfpathlineto{\pgfqpoint{5.662566in}{0.739656in}}%
\pgfpathlineto{\pgfqpoint{5.662269in}{0.739656in}}%
\pgfpathlineto{\pgfqpoint{5.661971in}{0.739656in}}%
\pgfpathlineto{\pgfqpoint{5.661674in}{0.739656in}}%
\pgfpathlineto{\pgfqpoint{5.661376in}{0.739656in}}%
\pgfpathlineto{\pgfqpoint{5.661079in}{0.739656in}}%
\pgfpathlineto{\pgfqpoint{5.660781in}{0.739656in}}%
\pgfpathlineto{\pgfqpoint{5.660484in}{0.739656in}}%
\pgfpathlineto{\pgfqpoint{5.660186in}{0.739656in}}%
\pgfpathlineto{\pgfqpoint{5.659889in}{0.739656in}}%
\pgfpathlineto{\pgfqpoint{5.659591in}{0.739656in}}%
\pgfpathlineto{\pgfqpoint{5.659294in}{0.739656in}}%
\pgfpathlineto{\pgfqpoint{5.658996in}{0.739656in}}%
\pgfpathlineto{\pgfqpoint{5.658699in}{0.739656in}}%
\pgfpathlineto{\pgfqpoint{5.658401in}{0.739656in}}%
\pgfpathlineto{\pgfqpoint{5.658104in}{0.739656in}}%
\pgfpathlineto{\pgfqpoint{5.657807in}{0.739656in}}%
\pgfpathlineto{\pgfqpoint{5.657509in}{0.739656in}}%
\pgfpathlineto{\pgfqpoint{5.657212in}{0.739656in}}%
\pgfpathlineto{\pgfqpoint{5.656914in}{0.739656in}}%
\pgfpathlineto{\pgfqpoint{5.656617in}{0.739656in}}%
\pgfpathlineto{\pgfqpoint{5.656319in}{0.739656in}}%
\pgfpathlineto{\pgfqpoint{5.656022in}{0.739656in}}%
\pgfpathlineto{\pgfqpoint{5.655724in}{0.739656in}}%
\pgfpathlineto{\pgfqpoint{5.655427in}{0.739656in}}%
\pgfpathlineto{\pgfqpoint{5.655129in}{0.739656in}}%
\pgfpathlineto{\pgfqpoint{5.654832in}{0.739656in}}%
\pgfpathlineto{\pgfqpoint{5.654534in}{0.739656in}}%
\pgfpathlineto{\pgfqpoint{5.654237in}{0.739656in}}%
\pgfpathlineto{\pgfqpoint{5.653939in}{0.739656in}}%
\pgfpathlineto{\pgfqpoint{5.653642in}{0.739656in}}%
\pgfpathlineto{\pgfqpoint{5.653344in}{0.739656in}}%
\pgfpathlineto{\pgfqpoint{5.653047in}{0.739656in}}%
\pgfpathlineto{\pgfqpoint{5.652749in}{0.739656in}}%
\pgfpathlineto{\pgfqpoint{5.652452in}{0.739656in}}%
\pgfpathlineto{\pgfqpoint{5.652154in}{0.739656in}}%
\pgfpathlineto{\pgfqpoint{5.651857in}{0.739656in}}%
\pgfpathlineto{\pgfqpoint{5.651559in}{0.739656in}}%
\pgfpathlineto{\pgfqpoint{5.651262in}{0.739656in}}%
\pgfpathlineto{\pgfqpoint{5.650965in}{0.739656in}}%
\pgfpathlineto{\pgfqpoint{5.650667in}{0.739656in}}%
\pgfpathlineto{\pgfqpoint{5.650370in}{0.739656in}}%
\pgfpathlineto{\pgfqpoint{5.650072in}{0.739656in}}%
\pgfpathlineto{\pgfqpoint{5.649775in}{0.739656in}}%
\pgfpathlineto{\pgfqpoint{5.649477in}{0.739656in}}%
\pgfpathlineto{\pgfqpoint{5.649180in}{0.739656in}}%
\pgfpathlineto{\pgfqpoint{5.648882in}{0.739656in}}%
\pgfpathlineto{\pgfqpoint{5.648585in}{0.739656in}}%
\pgfpathlineto{\pgfqpoint{5.648287in}{0.739656in}}%
\pgfpathlineto{\pgfqpoint{5.647990in}{0.739656in}}%
\pgfpathlineto{\pgfqpoint{5.647692in}{0.739656in}}%
\pgfpathlineto{\pgfqpoint{5.647395in}{0.739656in}}%
\pgfpathlineto{\pgfqpoint{5.647097in}{0.739656in}}%
\pgfpathlineto{\pgfqpoint{5.646800in}{0.739656in}}%
\pgfpathlineto{\pgfqpoint{5.646502in}{0.739656in}}%
\pgfpathlineto{\pgfqpoint{5.646205in}{0.739656in}}%
\pgfpathlineto{\pgfqpoint{5.645907in}{0.739656in}}%
\pgfpathlineto{\pgfqpoint{5.645610in}{0.739656in}}%
\pgfpathlineto{\pgfqpoint{5.645312in}{0.739656in}}%
\pgfpathlineto{\pgfqpoint{5.645015in}{0.739656in}}%
\pgfpathlineto{\pgfqpoint{5.644717in}{0.739656in}}%
\pgfpathlineto{\pgfqpoint{5.644420in}{0.739656in}}%
\pgfpathlineto{\pgfqpoint{5.644123in}{0.739656in}}%
\pgfpathlineto{\pgfqpoint{5.643825in}{0.739656in}}%
\pgfpathlineto{\pgfqpoint{5.643528in}{0.739656in}}%
\pgfpathlineto{\pgfqpoint{5.643230in}{0.739656in}}%
\pgfpathlineto{\pgfqpoint{5.642933in}{0.739656in}}%
\pgfpathlineto{\pgfqpoint{5.642635in}{0.739656in}}%
\pgfpathlineto{\pgfqpoint{5.642338in}{0.739656in}}%
\pgfpathlineto{\pgfqpoint{5.642040in}{0.739656in}}%
\pgfpathlineto{\pgfqpoint{5.641743in}{0.739656in}}%
\pgfpathlineto{\pgfqpoint{5.641445in}{0.739656in}}%
\pgfpathlineto{\pgfqpoint{5.641148in}{0.739656in}}%
\pgfpathlineto{\pgfqpoint{5.640850in}{0.739656in}}%
\pgfpathlineto{\pgfqpoint{5.640553in}{0.739656in}}%
\pgfpathlineto{\pgfqpoint{5.640255in}{0.739656in}}%
\pgfpathlineto{\pgfqpoint{5.639958in}{0.739656in}}%
\pgfpathlineto{\pgfqpoint{5.639660in}{0.739656in}}%
\pgfpathlineto{\pgfqpoint{5.639363in}{0.739656in}}%
\pgfpathlineto{\pgfqpoint{5.639065in}{0.739656in}}%
\pgfpathlineto{\pgfqpoint{5.638768in}{0.739656in}}%
\pgfpathlineto{\pgfqpoint{5.638470in}{0.739656in}}%
\pgfpathlineto{\pgfqpoint{5.638173in}{0.739656in}}%
\pgfpathlineto{\pgfqpoint{5.637876in}{0.739656in}}%
\pgfpathlineto{\pgfqpoint{5.637578in}{0.739656in}}%
\pgfpathlineto{\pgfqpoint{5.637281in}{0.739656in}}%
\pgfpathlineto{\pgfqpoint{5.636983in}{0.739656in}}%
\pgfpathlineto{\pgfqpoint{5.636686in}{0.739656in}}%
\pgfpathlineto{\pgfqpoint{5.636388in}{0.739656in}}%
\pgfpathlineto{\pgfqpoint{5.636091in}{0.739656in}}%
\pgfpathlineto{\pgfqpoint{5.635793in}{0.739656in}}%
\pgfpathlineto{\pgfqpoint{5.635496in}{0.739656in}}%
\pgfpathlineto{\pgfqpoint{5.635198in}{0.739656in}}%
\pgfpathlineto{\pgfqpoint{5.634901in}{0.739656in}}%
\pgfpathlineto{\pgfqpoint{5.634603in}{0.739656in}}%
\pgfpathlineto{\pgfqpoint{5.634306in}{0.739656in}}%
\pgfpathlineto{\pgfqpoint{5.634008in}{0.739656in}}%
\pgfpathlineto{\pgfqpoint{5.633711in}{0.739656in}}%
\pgfpathlineto{\pgfqpoint{5.633413in}{0.739656in}}%
\pgfpathlineto{\pgfqpoint{5.633116in}{0.739656in}}%
\pgfpathlineto{\pgfqpoint{5.632818in}{0.739656in}}%
\pgfpathlineto{\pgfqpoint{5.632521in}{0.739656in}}%
\pgfpathlineto{\pgfqpoint{5.632223in}{0.739656in}}%
\pgfpathlineto{\pgfqpoint{5.631926in}{0.739656in}}%
\pgfpathlineto{\pgfqpoint{5.631628in}{0.739656in}}%
\pgfpathlineto{\pgfqpoint{5.631331in}{0.739656in}}%
\pgfpathlineto{\pgfqpoint{5.631034in}{0.739656in}}%
\pgfpathlineto{\pgfqpoint{5.630736in}{0.739656in}}%
\pgfpathlineto{\pgfqpoint{5.630439in}{0.739656in}}%
\pgfpathlineto{\pgfqpoint{5.630141in}{0.739656in}}%
\pgfpathlineto{\pgfqpoint{5.629844in}{0.739656in}}%
\pgfpathlineto{\pgfqpoint{5.629546in}{0.739656in}}%
\pgfpathlineto{\pgfqpoint{5.629249in}{0.739656in}}%
\pgfpathlineto{\pgfqpoint{5.628951in}{0.739656in}}%
\pgfpathlineto{\pgfqpoint{5.628654in}{0.739656in}}%
\pgfpathlineto{\pgfqpoint{5.628356in}{0.739656in}}%
\pgfpathlineto{\pgfqpoint{5.628059in}{0.739656in}}%
\pgfpathlineto{\pgfqpoint{5.627761in}{0.739656in}}%
\pgfpathlineto{\pgfqpoint{5.627464in}{0.739656in}}%
\pgfpathlineto{\pgfqpoint{5.627166in}{0.739656in}}%
\pgfpathlineto{\pgfqpoint{5.626869in}{0.739656in}}%
\pgfpathlineto{\pgfqpoint{5.626571in}{0.739656in}}%
\pgfpathlineto{\pgfqpoint{5.626274in}{0.739656in}}%
\pgfpathlineto{\pgfqpoint{5.625976in}{0.739656in}}%
\pgfpathlineto{\pgfqpoint{5.625679in}{0.739656in}}%
\pgfpathlineto{\pgfqpoint{5.625381in}{0.739656in}}%
\pgfpathlineto{\pgfqpoint{5.625084in}{0.739656in}}%
\pgfpathlineto{\pgfqpoint{5.624786in}{0.739656in}}%
\pgfpathlineto{\pgfqpoint{5.624489in}{0.739656in}}%
\pgfpathlineto{\pgfqpoint{5.624192in}{0.739656in}}%
\pgfpathlineto{\pgfqpoint{5.623894in}{0.739656in}}%
\pgfpathlineto{\pgfqpoint{5.623597in}{0.739656in}}%
\pgfpathlineto{\pgfqpoint{5.623299in}{0.739656in}}%
\pgfpathlineto{\pgfqpoint{5.623002in}{0.739656in}}%
\pgfpathlineto{\pgfqpoint{5.622704in}{0.739656in}}%
\pgfpathlineto{\pgfqpoint{5.622407in}{0.739656in}}%
\pgfpathlineto{\pgfqpoint{5.622109in}{0.739656in}}%
\pgfpathlineto{\pgfqpoint{5.621812in}{0.739656in}}%
\pgfpathlineto{\pgfqpoint{5.621514in}{0.739656in}}%
\pgfpathlineto{\pgfqpoint{5.621217in}{0.739656in}}%
\pgfpathlineto{\pgfqpoint{5.620919in}{0.739656in}}%
\pgfpathlineto{\pgfqpoint{5.620622in}{0.739656in}}%
\pgfpathlineto{\pgfqpoint{5.620324in}{0.739656in}}%
\pgfpathlineto{\pgfqpoint{5.620027in}{0.739656in}}%
\pgfpathlineto{\pgfqpoint{5.619729in}{0.739656in}}%
\pgfpathlineto{\pgfqpoint{5.619432in}{0.739656in}}%
\pgfpathlineto{\pgfqpoint{5.619134in}{0.739656in}}%
\pgfpathlineto{\pgfqpoint{5.618837in}{0.739656in}}%
\pgfpathlineto{\pgfqpoint{5.618539in}{0.739656in}}%
\pgfpathlineto{\pgfqpoint{5.618242in}{0.739656in}}%
\pgfpathlineto{\pgfqpoint{5.617945in}{0.739656in}}%
\pgfpathlineto{\pgfqpoint{5.617647in}{0.739656in}}%
\pgfpathlineto{\pgfqpoint{5.617350in}{0.739656in}}%
\pgfpathlineto{\pgfqpoint{5.617052in}{0.739656in}}%
\pgfpathlineto{\pgfqpoint{5.616755in}{0.739656in}}%
\pgfpathlineto{\pgfqpoint{5.616457in}{0.739656in}}%
\pgfpathlineto{\pgfqpoint{5.616160in}{0.739656in}}%
\pgfpathlineto{\pgfqpoint{5.615862in}{0.739656in}}%
\pgfpathlineto{\pgfqpoint{5.615565in}{0.739656in}}%
\pgfpathlineto{\pgfqpoint{5.615267in}{0.739656in}}%
\pgfpathlineto{\pgfqpoint{5.614970in}{0.739656in}}%
\pgfpathlineto{\pgfqpoint{5.614672in}{0.739656in}}%
\pgfpathlineto{\pgfqpoint{5.614375in}{0.739656in}}%
\pgfpathlineto{\pgfqpoint{5.614077in}{0.739656in}}%
\pgfpathlineto{\pgfqpoint{5.613780in}{0.739656in}}%
\pgfpathlineto{\pgfqpoint{5.613482in}{0.739656in}}%
\pgfpathlineto{\pgfqpoint{5.613185in}{0.739656in}}%
\pgfpathlineto{\pgfqpoint{5.612887in}{0.739656in}}%
\pgfpathlineto{\pgfqpoint{5.612590in}{0.739656in}}%
\pgfpathlineto{\pgfqpoint{5.612292in}{0.739656in}}%
\pgfpathlineto{\pgfqpoint{5.611995in}{0.739656in}}%
\pgfpathlineto{\pgfqpoint{5.611697in}{0.739656in}}%
\pgfpathlineto{\pgfqpoint{5.611400in}{0.739656in}}%
\pgfpathlineto{\pgfqpoint{5.611103in}{0.739656in}}%
\pgfpathlineto{\pgfqpoint{5.610805in}{0.739656in}}%
\pgfpathlineto{\pgfqpoint{5.610508in}{0.739656in}}%
\pgfpathlineto{\pgfqpoint{5.610210in}{0.739656in}}%
\pgfpathlineto{\pgfqpoint{5.609913in}{0.739656in}}%
\pgfpathlineto{\pgfqpoint{5.609615in}{0.739656in}}%
\pgfpathlineto{\pgfqpoint{5.609318in}{0.739656in}}%
\pgfpathlineto{\pgfqpoint{5.609020in}{0.739656in}}%
\pgfpathlineto{\pgfqpoint{5.608723in}{0.739656in}}%
\pgfpathlineto{\pgfqpoint{5.608425in}{0.739656in}}%
\pgfpathlineto{\pgfqpoint{5.608128in}{0.739656in}}%
\pgfpathlineto{\pgfqpoint{5.607830in}{0.739656in}}%
\pgfpathlineto{\pgfqpoint{5.607533in}{0.739656in}}%
\pgfpathlineto{\pgfqpoint{5.607235in}{0.739656in}}%
\pgfpathlineto{\pgfqpoint{5.606938in}{0.739656in}}%
\pgfpathlineto{\pgfqpoint{5.606640in}{0.739656in}}%
\pgfpathlineto{\pgfqpoint{5.606343in}{0.739656in}}%
\pgfpathlineto{\pgfqpoint{5.606045in}{0.739656in}}%
\pgfpathlineto{\pgfqpoint{5.605748in}{0.739656in}}%
\pgfpathlineto{\pgfqpoint{5.605450in}{0.739656in}}%
\pgfpathlineto{\pgfqpoint{5.605153in}{0.739656in}}%
\pgfpathlineto{\pgfqpoint{5.604855in}{0.739656in}}%
\pgfpathlineto{\pgfqpoint{5.604558in}{0.739656in}}%
\pgfpathlineto{\pgfqpoint{5.604261in}{0.739656in}}%
\pgfpathlineto{\pgfqpoint{5.603963in}{0.739656in}}%
\pgfpathlineto{\pgfqpoint{5.603666in}{0.739656in}}%
\pgfpathlineto{\pgfqpoint{5.603368in}{0.739656in}}%
\pgfpathlineto{\pgfqpoint{5.603071in}{0.739656in}}%
\pgfpathlineto{\pgfqpoint{5.602773in}{0.739656in}}%
\pgfpathlineto{\pgfqpoint{5.602476in}{0.739656in}}%
\pgfpathlineto{\pgfqpoint{5.602178in}{0.739656in}}%
\pgfpathlineto{\pgfqpoint{5.601881in}{0.739656in}}%
\pgfpathlineto{\pgfqpoint{5.601583in}{0.739656in}}%
\pgfpathlineto{\pgfqpoint{5.601286in}{0.739656in}}%
\pgfpathlineto{\pgfqpoint{5.600988in}{0.739656in}}%
\pgfpathlineto{\pgfqpoint{5.600691in}{0.739656in}}%
\pgfpathlineto{\pgfqpoint{5.600393in}{0.739656in}}%
\pgfpathlineto{\pgfqpoint{5.600096in}{0.739656in}}%
\pgfpathlineto{\pgfqpoint{5.599798in}{0.739656in}}%
\pgfpathlineto{\pgfqpoint{5.599501in}{0.739656in}}%
\pgfpathlineto{\pgfqpoint{5.599203in}{0.739656in}}%
\pgfpathlineto{\pgfqpoint{5.598906in}{0.739656in}}%
\pgfpathlineto{\pgfqpoint{5.598608in}{0.739656in}}%
\pgfpathlineto{\pgfqpoint{5.598311in}{0.739656in}}%
\pgfpathlineto{\pgfqpoint{5.598014in}{0.739656in}}%
\pgfpathlineto{\pgfqpoint{5.597716in}{0.739656in}}%
\pgfpathlineto{\pgfqpoint{5.597419in}{0.739656in}}%
\pgfpathlineto{\pgfqpoint{5.597121in}{0.739656in}}%
\pgfpathlineto{\pgfqpoint{5.596824in}{0.739656in}}%
\pgfpathlineto{\pgfqpoint{5.596526in}{0.739656in}}%
\pgfpathlineto{\pgfqpoint{5.596229in}{0.739656in}}%
\pgfpathlineto{\pgfqpoint{5.595931in}{0.739656in}}%
\pgfpathlineto{\pgfqpoint{5.595634in}{0.739656in}}%
\pgfpathlineto{\pgfqpoint{5.595336in}{0.739656in}}%
\pgfpathlineto{\pgfqpoint{5.595039in}{0.739656in}}%
\pgfpathlineto{\pgfqpoint{5.594741in}{0.739656in}}%
\pgfpathlineto{\pgfqpoint{5.594444in}{0.739656in}}%
\pgfpathlineto{\pgfqpoint{5.594146in}{0.739656in}}%
\pgfpathlineto{\pgfqpoint{5.593849in}{0.739656in}}%
\pgfpathlineto{\pgfqpoint{5.593551in}{0.739656in}}%
\pgfpathlineto{\pgfqpoint{5.593254in}{0.739656in}}%
\pgfpathlineto{\pgfqpoint{5.592956in}{0.739656in}}%
\pgfpathlineto{\pgfqpoint{5.592659in}{0.739656in}}%
\pgfpathlineto{\pgfqpoint{5.592361in}{0.739656in}}%
\pgfpathlineto{\pgfqpoint{5.592064in}{0.739656in}}%
\pgfpathlineto{\pgfqpoint{5.591766in}{0.739656in}}%
\pgfpathlineto{\pgfqpoint{5.591469in}{0.739656in}}%
\pgfpathlineto{\pgfqpoint{5.591172in}{0.739656in}}%
\pgfpathlineto{\pgfqpoint{5.590874in}{0.739656in}}%
\pgfpathlineto{\pgfqpoint{5.590577in}{0.739656in}}%
\pgfpathlineto{\pgfqpoint{5.590279in}{0.739656in}}%
\pgfpathlineto{\pgfqpoint{5.589982in}{0.739656in}}%
\pgfpathlineto{\pgfqpoint{5.589684in}{0.739656in}}%
\pgfpathlineto{\pgfqpoint{5.589387in}{0.739656in}}%
\pgfpathlineto{\pgfqpoint{5.589089in}{0.739656in}}%
\pgfpathlineto{\pgfqpoint{5.588792in}{0.739656in}}%
\pgfpathlineto{\pgfqpoint{5.588494in}{0.739656in}}%
\pgfpathlineto{\pgfqpoint{5.588197in}{0.739656in}}%
\pgfpathlineto{\pgfqpoint{5.587899in}{0.739656in}}%
\pgfpathlineto{\pgfqpoint{5.587602in}{0.739656in}}%
\pgfpathlineto{\pgfqpoint{5.587304in}{0.739656in}}%
\pgfpathlineto{\pgfqpoint{5.587007in}{0.739656in}}%
\pgfpathlineto{\pgfqpoint{5.586709in}{0.739656in}}%
\pgfpathlineto{\pgfqpoint{5.586412in}{0.739656in}}%
\pgfpathlineto{\pgfqpoint{5.586114in}{0.739656in}}%
\pgfpathlineto{\pgfqpoint{5.585817in}{0.739656in}}%
\pgfpathlineto{\pgfqpoint{5.585519in}{0.739656in}}%
\pgfpathlineto{\pgfqpoint{5.585222in}{0.739656in}}%
\pgfpathlineto{\pgfqpoint{5.584924in}{0.739656in}}%
\pgfpathlineto{\pgfqpoint{5.584627in}{0.739656in}}%
\pgfpathlineto{\pgfqpoint{5.584330in}{0.739656in}}%
\pgfpathlineto{\pgfqpoint{5.584032in}{0.739656in}}%
\pgfpathlineto{\pgfqpoint{5.583735in}{0.739656in}}%
\pgfpathlineto{\pgfqpoint{5.583437in}{0.739656in}}%
\pgfpathlineto{\pgfqpoint{5.583140in}{0.739656in}}%
\pgfpathlineto{\pgfqpoint{5.582842in}{0.739656in}}%
\pgfpathlineto{\pgfqpoint{5.582545in}{0.739656in}}%
\pgfpathlineto{\pgfqpoint{5.582247in}{0.739656in}}%
\pgfpathlineto{\pgfqpoint{5.581950in}{0.739656in}}%
\pgfpathlineto{\pgfqpoint{5.581652in}{0.739656in}}%
\pgfpathlineto{\pgfqpoint{5.581355in}{0.739656in}}%
\pgfpathlineto{\pgfqpoint{5.581057in}{0.739656in}}%
\pgfpathlineto{\pgfqpoint{5.580760in}{0.739656in}}%
\pgfpathlineto{\pgfqpoint{5.580462in}{0.739656in}}%
\pgfpathlineto{\pgfqpoint{5.580165in}{0.739656in}}%
\pgfpathlineto{\pgfqpoint{5.579867in}{0.739656in}}%
\pgfpathlineto{\pgfqpoint{5.579570in}{0.739656in}}%
\pgfpathlineto{\pgfqpoint{5.579272in}{0.739656in}}%
\pgfpathlineto{\pgfqpoint{5.578975in}{0.739656in}}%
\pgfpathlineto{\pgfqpoint{5.578677in}{0.739656in}}%
\pgfpathlineto{\pgfqpoint{5.578380in}{0.739656in}}%
\pgfpathlineto{\pgfqpoint{5.578083in}{0.739656in}}%
\pgfpathlineto{\pgfqpoint{5.577785in}{0.739656in}}%
\pgfpathlineto{\pgfqpoint{5.577488in}{0.739656in}}%
\pgfpathlineto{\pgfqpoint{5.577190in}{0.739656in}}%
\pgfpathlineto{\pgfqpoint{5.576893in}{0.739656in}}%
\pgfpathlineto{\pgfqpoint{5.576595in}{0.739656in}}%
\pgfpathlineto{\pgfqpoint{5.576298in}{0.739656in}}%
\pgfpathlineto{\pgfqpoint{5.576000in}{0.739656in}}%
\pgfpathlineto{\pgfqpoint{5.575703in}{0.739656in}}%
\pgfpathlineto{\pgfqpoint{5.575405in}{0.739656in}}%
\pgfpathlineto{\pgfqpoint{5.575108in}{0.739656in}}%
\pgfpathlineto{\pgfqpoint{5.574810in}{0.739656in}}%
\pgfpathlineto{\pgfqpoint{5.574513in}{0.739656in}}%
\pgfpathlineto{\pgfqpoint{5.574215in}{0.739656in}}%
\pgfpathlineto{\pgfqpoint{5.573918in}{0.739656in}}%
\pgfpathlineto{\pgfqpoint{5.573620in}{0.739656in}}%
\pgfpathlineto{\pgfqpoint{5.573323in}{0.739656in}}%
\pgfpathlineto{\pgfqpoint{5.573025in}{0.739656in}}%
\pgfpathlineto{\pgfqpoint{5.572728in}{0.739656in}}%
\pgfpathlineto{\pgfqpoint{5.572430in}{0.739656in}}%
\pgfpathlineto{\pgfqpoint{5.572133in}{0.739656in}}%
\pgfpathlineto{\pgfqpoint{5.571835in}{0.739656in}}%
\pgfpathlineto{\pgfqpoint{5.571538in}{0.739656in}}%
\pgfpathlineto{\pgfqpoint{5.571241in}{0.739656in}}%
\pgfpathlineto{\pgfqpoint{5.570943in}{0.739656in}}%
\pgfpathlineto{\pgfqpoint{5.570646in}{0.739656in}}%
\pgfpathlineto{\pgfqpoint{5.570348in}{0.739656in}}%
\pgfpathlineto{\pgfqpoint{5.570051in}{0.739656in}}%
\pgfpathlineto{\pgfqpoint{5.569753in}{0.739656in}}%
\pgfpathlineto{\pgfqpoint{5.569456in}{0.739656in}}%
\pgfpathlineto{\pgfqpoint{5.569158in}{0.739656in}}%
\pgfpathlineto{\pgfqpoint{5.568861in}{0.739656in}}%
\pgfpathlineto{\pgfqpoint{5.568563in}{0.739656in}}%
\pgfpathlineto{\pgfqpoint{5.568266in}{0.739656in}}%
\pgfpathlineto{\pgfqpoint{5.567968in}{0.739656in}}%
\pgfpathlineto{\pgfqpoint{5.567671in}{0.739656in}}%
\pgfpathlineto{\pgfqpoint{5.567373in}{0.739656in}}%
\pgfpathlineto{\pgfqpoint{5.567076in}{0.739656in}}%
\pgfpathlineto{\pgfqpoint{5.566778in}{0.739656in}}%
\pgfpathlineto{\pgfqpoint{5.566481in}{0.739656in}}%
\pgfpathlineto{\pgfqpoint{5.566183in}{0.739656in}}%
\pgfpathlineto{\pgfqpoint{5.565886in}{0.739656in}}%
\pgfpathlineto{\pgfqpoint{5.565588in}{0.739656in}}%
\pgfpathlineto{\pgfqpoint{5.565291in}{0.739656in}}%
\pgfpathlineto{\pgfqpoint{5.564993in}{0.739656in}}%
\pgfpathlineto{\pgfqpoint{5.564696in}{0.739656in}}%
\pgfpathlineto{\pgfqpoint{5.564399in}{0.739656in}}%
\pgfpathlineto{\pgfqpoint{5.564101in}{0.739656in}}%
\pgfpathlineto{\pgfqpoint{5.563804in}{0.739656in}}%
\pgfpathlineto{\pgfqpoint{5.563506in}{0.739656in}}%
\pgfpathlineto{\pgfqpoint{5.563209in}{0.739656in}}%
\pgfpathlineto{\pgfqpoint{5.562911in}{0.739656in}}%
\pgfpathlineto{\pgfqpoint{5.562614in}{0.739656in}}%
\pgfpathlineto{\pgfqpoint{5.562316in}{0.739656in}}%
\pgfpathlineto{\pgfqpoint{5.562019in}{0.739656in}}%
\pgfpathlineto{\pgfqpoint{5.561721in}{0.739656in}}%
\pgfpathlineto{\pgfqpoint{5.561424in}{0.739656in}}%
\pgfpathlineto{\pgfqpoint{5.561126in}{0.739656in}}%
\pgfpathlineto{\pgfqpoint{5.560829in}{0.739656in}}%
\pgfpathlineto{\pgfqpoint{5.560531in}{0.739656in}}%
\pgfpathlineto{\pgfqpoint{5.560234in}{0.739656in}}%
\pgfpathlineto{\pgfqpoint{5.559936in}{0.739656in}}%
\pgfpathlineto{\pgfqpoint{5.559639in}{0.739656in}}%
\pgfpathlineto{\pgfqpoint{5.559341in}{0.739656in}}%
\pgfpathlineto{\pgfqpoint{5.559044in}{0.739656in}}%
\pgfpathlineto{\pgfqpoint{5.558746in}{0.739656in}}%
\pgfpathlineto{\pgfqpoint{5.558449in}{0.739656in}}%
\pgfpathlineto{\pgfqpoint{5.558152in}{0.739656in}}%
\pgfpathlineto{\pgfqpoint{5.557854in}{0.739656in}}%
\pgfpathlineto{\pgfqpoint{5.557557in}{0.739656in}}%
\pgfpathlineto{\pgfqpoint{5.557259in}{0.739656in}}%
\pgfpathlineto{\pgfqpoint{5.556962in}{0.739656in}}%
\pgfpathlineto{\pgfqpoint{5.556664in}{0.739656in}}%
\pgfpathlineto{\pgfqpoint{5.556367in}{0.739656in}}%
\pgfpathlineto{\pgfqpoint{5.556069in}{0.739656in}}%
\pgfpathlineto{\pgfqpoint{5.555772in}{0.739656in}}%
\pgfpathlineto{\pgfqpoint{5.555474in}{0.739656in}}%
\pgfpathlineto{\pgfqpoint{5.555177in}{0.739656in}}%
\pgfpathlineto{\pgfqpoint{5.554879in}{0.739656in}}%
\pgfpathlineto{\pgfqpoint{5.554582in}{0.739656in}}%
\pgfpathlineto{\pgfqpoint{5.554284in}{0.739656in}}%
\pgfpathlineto{\pgfqpoint{5.553987in}{0.739656in}}%
\pgfpathlineto{\pgfqpoint{5.553689in}{0.739656in}}%
\pgfpathlineto{\pgfqpoint{5.553392in}{0.739656in}}%
\pgfpathlineto{\pgfqpoint{5.553094in}{0.739656in}}%
\pgfpathlineto{\pgfqpoint{5.552797in}{0.739656in}}%
\pgfpathlineto{\pgfqpoint{5.552499in}{0.739656in}}%
\pgfpathlineto{\pgfqpoint{5.552202in}{0.739656in}}%
\pgfpathlineto{\pgfqpoint{5.551904in}{0.739656in}}%
\pgfpathlineto{\pgfqpoint{5.551607in}{0.739656in}}%
\pgfpathlineto{\pgfqpoint{5.551310in}{0.739656in}}%
\pgfpathlineto{\pgfqpoint{5.551012in}{0.739656in}}%
\pgfpathlineto{\pgfqpoint{5.550715in}{0.739656in}}%
\pgfpathlineto{\pgfqpoint{5.550417in}{0.739656in}}%
\pgfpathlineto{\pgfqpoint{5.550120in}{0.739656in}}%
\pgfpathlineto{\pgfqpoint{5.549822in}{0.739656in}}%
\pgfpathlineto{\pgfqpoint{5.549525in}{0.739656in}}%
\pgfpathlineto{\pgfqpoint{5.549227in}{0.739656in}}%
\pgfpathlineto{\pgfqpoint{5.548930in}{0.739656in}}%
\pgfpathlineto{\pgfqpoint{5.548632in}{0.739656in}}%
\pgfpathlineto{\pgfqpoint{5.548335in}{0.739656in}}%
\pgfpathlineto{\pgfqpoint{5.548037in}{0.739656in}}%
\pgfpathlineto{\pgfqpoint{5.547740in}{0.739656in}}%
\pgfpathlineto{\pgfqpoint{5.547442in}{0.739656in}}%
\pgfpathlineto{\pgfqpoint{5.547145in}{0.739656in}}%
\pgfpathlineto{\pgfqpoint{5.546847in}{0.739656in}}%
\pgfpathlineto{\pgfqpoint{5.546550in}{0.739656in}}%
\pgfpathlineto{\pgfqpoint{5.546252in}{0.739656in}}%
\pgfpathlineto{\pgfqpoint{5.545955in}{0.739656in}}%
\pgfpathlineto{\pgfqpoint{5.545657in}{0.739656in}}%
\pgfpathlineto{\pgfqpoint{5.545360in}{0.739656in}}%
\pgfpathlineto{\pgfqpoint{5.545062in}{0.739656in}}%
\pgfpathlineto{\pgfqpoint{5.544765in}{0.739656in}}%
\pgfpathlineto{\pgfqpoint{5.544468in}{0.739656in}}%
\pgfpathlineto{\pgfqpoint{5.544170in}{0.739656in}}%
\pgfpathlineto{\pgfqpoint{5.543873in}{0.739656in}}%
\pgfpathlineto{\pgfqpoint{5.543575in}{0.739656in}}%
\pgfpathlineto{\pgfqpoint{5.543278in}{0.739656in}}%
\pgfpathlineto{\pgfqpoint{5.542980in}{0.739656in}}%
\pgfpathlineto{\pgfqpoint{5.542683in}{0.739656in}}%
\pgfpathlineto{\pgfqpoint{5.542385in}{0.739656in}}%
\pgfpathlineto{\pgfqpoint{5.542088in}{0.739656in}}%
\pgfpathlineto{\pgfqpoint{5.541790in}{0.739656in}}%
\pgfpathlineto{\pgfqpoint{5.541493in}{0.739656in}}%
\pgfpathlineto{\pgfqpoint{5.541195in}{0.739656in}}%
\pgfpathlineto{\pgfqpoint{5.540898in}{0.739656in}}%
\pgfpathlineto{\pgfqpoint{5.540600in}{0.739656in}}%
\pgfpathlineto{\pgfqpoint{5.540303in}{0.739656in}}%
\pgfpathlineto{\pgfqpoint{5.540005in}{0.739656in}}%
\pgfpathlineto{\pgfqpoint{5.539708in}{0.739656in}}%
\pgfpathlineto{\pgfqpoint{5.539410in}{0.739656in}}%
\pgfpathlineto{\pgfqpoint{5.539113in}{0.739656in}}%
\pgfpathlineto{\pgfqpoint{5.538815in}{0.739656in}}%
\pgfpathlineto{\pgfqpoint{5.538518in}{0.739656in}}%
\pgfpathlineto{\pgfqpoint{5.538221in}{0.739656in}}%
\pgfpathlineto{\pgfqpoint{5.537923in}{0.739656in}}%
\pgfpathlineto{\pgfqpoint{5.537626in}{0.739656in}}%
\pgfpathlineto{\pgfqpoint{5.537328in}{0.739656in}}%
\pgfpathlineto{\pgfqpoint{5.537031in}{0.739656in}}%
\pgfpathlineto{\pgfqpoint{5.536733in}{0.739656in}}%
\pgfpathlineto{\pgfqpoint{5.536436in}{0.739656in}}%
\pgfpathlineto{\pgfqpoint{5.536138in}{0.739656in}}%
\pgfpathlineto{\pgfqpoint{5.535841in}{0.739656in}}%
\pgfpathlineto{\pgfqpoint{5.535543in}{0.739656in}}%
\pgfpathlineto{\pgfqpoint{5.535246in}{0.739656in}}%
\pgfpathlineto{\pgfqpoint{5.534948in}{0.739656in}}%
\pgfpathlineto{\pgfqpoint{5.534651in}{0.739656in}}%
\pgfpathlineto{\pgfqpoint{5.534353in}{0.739656in}}%
\pgfpathlineto{\pgfqpoint{5.534056in}{0.739656in}}%
\pgfpathlineto{\pgfqpoint{5.533758in}{0.739656in}}%
\pgfpathlineto{\pgfqpoint{5.533461in}{0.739656in}}%
\pgfpathlineto{\pgfqpoint{5.533163in}{0.739656in}}%
\pgfpathlineto{\pgfqpoint{5.532866in}{0.739656in}}%
\pgfpathlineto{\pgfqpoint{5.532568in}{0.739656in}}%
\pgfpathlineto{\pgfqpoint{5.532271in}{0.739656in}}%
\pgfpathlineto{\pgfqpoint{5.531973in}{0.739656in}}%
\pgfpathlineto{\pgfqpoint{5.531676in}{0.739656in}}%
\pgfpathlineto{\pgfqpoint{5.531379in}{0.739656in}}%
\pgfpathlineto{\pgfqpoint{5.531081in}{0.739656in}}%
\pgfpathlineto{\pgfqpoint{5.530784in}{0.739656in}}%
\pgfpathlineto{\pgfqpoint{5.530486in}{0.739656in}}%
\pgfpathlineto{\pgfqpoint{5.530189in}{0.739656in}}%
\pgfpathlineto{\pgfqpoint{5.529891in}{0.739656in}}%
\pgfpathlineto{\pgfqpoint{5.529594in}{0.739656in}}%
\pgfpathlineto{\pgfqpoint{5.529296in}{0.739656in}}%
\pgfpathlineto{\pgfqpoint{5.528999in}{0.739656in}}%
\pgfpathlineto{\pgfqpoint{5.528701in}{0.739656in}}%
\pgfpathlineto{\pgfqpoint{5.528404in}{0.739656in}}%
\pgfpathlineto{\pgfqpoint{5.528106in}{0.739656in}}%
\pgfpathlineto{\pgfqpoint{5.527809in}{0.739656in}}%
\pgfpathlineto{\pgfqpoint{5.527511in}{0.739656in}}%
\pgfpathlineto{\pgfqpoint{5.527214in}{0.739656in}}%
\pgfpathlineto{\pgfqpoint{5.526916in}{0.739656in}}%
\pgfpathlineto{\pgfqpoint{5.526619in}{0.739656in}}%
\pgfpathlineto{\pgfqpoint{5.526321in}{0.739656in}}%
\pgfpathlineto{\pgfqpoint{5.526024in}{0.739656in}}%
\pgfpathlineto{\pgfqpoint{5.525726in}{0.739656in}}%
\pgfpathlineto{\pgfqpoint{5.525429in}{0.739656in}}%
\pgfpathlineto{\pgfqpoint{5.525131in}{0.739656in}}%
\pgfpathlineto{\pgfqpoint{5.524834in}{0.739656in}}%
\pgfpathlineto{\pgfqpoint{5.524537in}{0.739656in}}%
\pgfpathlineto{\pgfqpoint{5.524239in}{0.739656in}}%
\pgfpathlineto{\pgfqpoint{5.523942in}{0.739656in}}%
\pgfpathlineto{\pgfqpoint{5.523644in}{0.739656in}}%
\pgfpathlineto{\pgfqpoint{5.523347in}{0.739656in}}%
\pgfpathlineto{\pgfqpoint{5.523049in}{0.739656in}}%
\pgfpathlineto{\pgfqpoint{5.522752in}{0.739656in}}%
\pgfpathlineto{\pgfqpoint{5.522454in}{0.739656in}}%
\pgfpathlineto{\pgfqpoint{5.522157in}{0.739656in}}%
\pgfpathlineto{\pgfqpoint{5.521859in}{0.739656in}}%
\pgfpathlineto{\pgfqpoint{5.521562in}{0.739656in}}%
\pgfpathlineto{\pgfqpoint{5.521264in}{0.739656in}}%
\pgfpathlineto{\pgfqpoint{5.520967in}{0.739656in}}%
\pgfpathlineto{\pgfqpoint{5.520669in}{0.739656in}}%
\pgfpathlineto{\pgfqpoint{5.520372in}{0.739656in}}%
\pgfpathlineto{\pgfqpoint{5.520074in}{0.739656in}}%
\pgfpathlineto{\pgfqpoint{5.519777in}{0.739656in}}%
\pgfpathlineto{\pgfqpoint{5.519479in}{0.739656in}}%
\pgfpathlineto{\pgfqpoint{5.519182in}{0.739656in}}%
\pgfpathlineto{\pgfqpoint{5.518884in}{0.739656in}}%
\pgfpathlineto{\pgfqpoint{5.518587in}{0.739656in}}%
\pgfpathlineto{\pgfqpoint{5.518290in}{0.739656in}}%
\pgfpathlineto{\pgfqpoint{5.517992in}{0.739656in}}%
\pgfpathlineto{\pgfqpoint{5.517695in}{0.739656in}}%
\pgfpathlineto{\pgfqpoint{5.517397in}{0.739656in}}%
\pgfpathlineto{\pgfqpoint{5.517100in}{0.739656in}}%
\pgfpathlineto{\pgfqpoint{5.516802in}{0.739656in}}%
\pgfpathlineto{\pgfqpoint{5.516505in}{0.739656in}}%
\pgfpathlineto{\pgfqpoint{5.516207in}{0.739656in}}%
\pgfpathlineto{\pgfqpoint{5.515910in}{0.739656in}}%
\pgfpathlineto{\pgfqpoint{5.515612in}{0.739656in}}%
\pgfpathlineto{\pgfqpoint{5.515315in}{0.739656in}}%
\pgfpathlineto{\pgfqpoint{5.515017in}{0.739656in}}%
\pgfpathlineto{\pgfqpoint{5.514720in}{0.739656in}}%
\pgfpathlineto{\pgfqpoint{5.514422in}{0.739656in}}%
\pgfpathlineto{\pgfqpoint{5.514125in}{0.739656in}}%
\pgfpathlineto{\pgfqpoint{5.513827in}{0.739656in}}%
\pgfpathlineto{\pgfqpoint{5.513530in}{0.739656in}}%
\pgfpathlineto{\pgfqpoint{5.513232in}{0.739656in}}%
\pgfpathlineto{\pgfqpoint{5.512935in}{0.739656in}}%
\pgfpathlineto{\pgfqpoint{5.512637in}{0.739656in}}%
\pgfpathlineto{\pgfqpoint{5.512340in}{0.739656in}}%
\pgfpathlineto{\pgfqpoint{5.512042in}{0.739656in}}%
\pgfpathlineto{\pgfqpoint{5.511745in}{0.739656in}}%
\pgfpathlineto{\pgfqpoint{5.511448in}{0.739656in}}%
\pgfpathlineto{\pgfqpoint{5.511150in}{0.739656in}}%
\pgfpathlineto{\pgfqpoint{5.510853in}{0.739656in}}%
\pgfpathlineto{\pgfqpoint{5.510555in}{0.739656in}}%
\pgfpathlineto{\pgfqpoint{5.510258in}{0.739656in}}%
\pgfpathlineto{\pgfqpoint{5.509960in}{0.739656in}}%
\pgfpathlineto{\pgfqpoint{5.509663in}{0.739656in}}%
\pgfpathlineto{\pgfqpoint{5.509365in}{0.739656in}}%
\pgfpathlineto{\pgfqpoint{5.509068in}{0.739656in}}%
\pgfpathlineto{\pgfqpoint{5.508770in}{0.739656in}}%
\pgfpathlineto{\pgfqpoint{5.508473in}{0.739656in}}%
\pgfpathlineto{\pgfqpoint{5.508175in}{0.739656in}}%
\pgfpathlineto{\pgfqpoint{5.507878in}{0.739656in}}%
\pgfpathlineto{\pgfqpoint{5.507580in}{0.739656in}}%
\pgfpathlineto{\pgfqpoint{5.507283in}{0.739656in}}%
\pgfpathlineto{\pgfqpoint{5.506985in}{0.739656in}}%
\pgfpathlineto{\pgfqpoint{5.506688in}{0.739656in}}%
\pgfpathlineto{\pgfqpoint{5.506390in}{0.739656in}}%
\pgfpathlineto{\pgfqpoint{5.506093in}{0.739656in}}%
\pgfpathlineto{\pgfqpoint{5.505795in}{0.739656in}}%
\pgfpathlineto{\pgfqpoint{5.505498in}{0.739656in}}%
\pgfpathlineto{\pgfqpoint{5.505200in}{0.739656in}}%
\pgfpathlineto{\pgfqpoint{5.504903in}{0.739656in}}%
\pgfpathlineto{\pgfqpoint{5.504606in}{0.739656in}}%
\pgfpathlineto{\pgfqpoint{5.504308in}{0.739656in}}%
\pgfpathlineto{\pgfqpoint{5.504011in}{0.739656in}}%
\pgfpathlineto{\pgfqpoint{5.503713in}{0.739656in}}%
\pgfpathlineto{\pgfqpoint{5.503416in}{0.739656in}}%
\pgfpathlineto{\pgfqpoint{5.503118in}{0.739656in}}%
\pgfpathlineto{\pgfqpoint{5.502821in}{0.739656in}}%
\pgfpathlineto{\pgfqpoint{5.502523in}{0.739656in}}%
\pgfpathlineto{\pgfqpoint{5.502226in}{0.739656in}}%
\pgfpathlineto{\pgfqpoint{5.501928in}{0.739656in}}%
\pgfpathlineto{\pgfqpoint{5.501631in}{0.739656in}}%
\pgfpathlineto{\pgfqpoint{5.501333in}{0.739656in}}%
\pgfpathlineto{\pgfqpoint{5.501036in}{0.739656in}}%
\pgfpathlineto{\pgfqpoint{5.500738in}{0.739656in}}%
\pgfpathlineto{\pgfqpoint{5.500441in}{0.739656in}}%
\pgfpathlineto{\pgfqpoint{5.500143in}{0.739656in}}%
\pgfpathlineto{\pgfqpoint{5.499846in}{0.739656in}}%
\pgfpathlineto{\pgfqpoint{5.499548in}{0.739656in}}%
\pgfpathlineto{\pgfqpoint{5.499251in}{0.739656in}}%
\pgfpathlineto{\pgfqpoint{5.498953in}{0.739656in}}%
\pgfpathlineto{\pgfqpoint{5.498656in}{0.739656in}}%
\pgfpathlineto{\pgfqpoint{5.498359in}{0.739656in}}%
\pgfpathlineto{\pgfqpoint{5.498061in}{0.739656in}}%
\pgfpathlineto{\pgfqpoint{5.497764in}{0.739656in}}%
\pgfpathlineto{\pgfqpoint{5.497466in}{0.739656in}}%
\pgfpathlineto{\pgfqpoint{5.497169in}{0.739656in}}%
\pgfpathlineto{\pgfqpoint{5.496871in}{0.739656in}}%
\pgfpathlineto{\pgfqpoint{5.496574in}{0.739656in}}%
\pgfpathlineto{\pgfqpoint{5.496276in}{0.739656in}}%
\pgfpathlineto{\pgfqpoint{5.495979in}{0.739656in}}%
\pgfpathlineto{\pgfqpoint{5.495681in}{0.739656in}}%
\pgfpathlineto{\pgfqpoint{5.495384in}{0.739656in}}%
\pgfpathlineto{\pgfqpoint{5.495086in}{0.739656in}}%
\pgfpathlineto{\pgfqpoint{5.494789in}{0.739656in}}%
\pgfpathlineto{\pgfqpoint{5.494491in}{0.739656in}}%
\pgfpathlineto{\pgfqpoint{5.494194in}{0.739656in}}%
\pgfpathlineto{\pgfqpoint{5.493896in}{0.739656in}}%
\pgfpathlineto{\pgfqpoint{5.493599in}{0.739656in}}%
\pgfpathlineto{\pgfqpoint{5.493301in}{0.739656in}}%
\pgfpathlineto{\pgfqpoint{5.493004in}{0.739656in}}%
\pgfpathlineto{\pgfqpoint{5.492706in}{0.739656in}}%
\pgfpathlineto{\pgfqpoint{5.492409in}{0.739656in}}%
\pgfpathlineto{\pgfqpoint{5.492111in}{0.739656in}}%
\pgfpathlineto{\pgfqpoint{5.491814in}{0.739656in}}%
\pgfpathlineto{\pgfqpoint{5.491517in}{0.739656in}}%
\pgfpathlineto{\pgfqpoint{5.491219in}{0.739656in}}%
\pgfpathlineto{\pgfqpoint{5.490922in}{0.739656in}}%
\pgfpathlineto{\pgfqpoint{5.490624in}{0.739656in}}%
\pgfpathlineto{\pgfqpoint{5.490327in}{0.739656in}}%
\pgfpathlineto{\pgfqpoint{5.490029in}{0.739656in}}%
\pgfpathlineto{\pgfqpoint{5.489732in}{0.739656in}}%
\pgfpathlineto{\pgfqpoint{5.489434in}{0.739656in}}%
\pgfpathlineto{\pgfqpoint{5.489137in}{0.739656in}}%
\pgfpathlineto{\pgfqpoint{5.488839in}{0.739656in}}%
\pgfpathlineto{\pgfqpoint{5.488542in}{0.739656in}}%
\pgfpathlineto{\pgfqpoint{5.488244in}{0.739656in}}%
\pgfpathlineto{\pgfqpoint{5.487947in}{0.739656in}}%
\pgfpathlineto{\pgfqpoint{5.487649in}{0.739656in}}%
\pgfpathlineto{\pgfqpoint{5.487352in}{0.739656in}}%
\pgfpathlineto{\pgfqpoint{5.487054in}{0.739656in}}%
\pgfpathlineto{\pgfqpoint{5.486757in}{0.739656in}}%
\pgfpathlineto{\pgfqpoint{5.486459in}{0.739656in}}%
\pgfpathlineto{\pgfqpoint{5.486162in}{0.739656in}}%
\pgfpathlineto{\pgfqpoint{5.485864in}{0.739656in}}%
\pgfpathlineto{\pgfqpoint{5.485567in}{0.739656in}}%
\pgfpathlineto{\pgfqpoint{5.485269in}{0.739656in}}%
\pgfpathlineto{\pgfqpoint{5.484972in}{0.739656in}}%
\pgfpathlineto{\pgfqpoint{5.484675in}{0.739656in}}%
\pgfpathlineto{\pgfqpoint{5.484377in}{0.739656in}}%
\pgfpathlineto{\pgfqpoint{5.484080in}{0.739656in}}%
\pgfpathlineto{\pgfqpoint{5.483782in}{0.739656in}}%
\pgfpathlineto{\pgfqpoint{5.483485in}{0.739656in}}%
\pgfpathlineto{\pgfqpoint{5.483187in}{0.739656in}}%
\pgfpathlineto{\pgfqpoint{5.482890in}{0.739656in}}%
\pgfpathlineto{\pgfqpoint{5.482592in}{0.739656in}}%
\pgfpathlineto{\pgfqpoint{5.482295in}{0.739656in}}%
\pgfpathlineto{\pgfqpoint{5.481997in}{0.739656in}}%
\pgfpathlineto{\pgfqpoint{5.481700in}{0.739656in}}%
\pgfpathlineto{\pgfqpoint{5.481402in}{0.739656in}}%
\pgfpathlineto{\pgfqpoint{5.481105in}{0.739656in}}%
\pgfpathlineto{\pgfqpoint{5.480807in}{0.739656in}}%
\pgfpathlineto{\pgfqpoint{5.480510in}{0.739656in}}%
\pgfpathlineto{\pgfqpoint{5.480212in}{0.739656in}}%
\pgfpathlineto{\pgfqpoint{5.479915in}{0.739656in}}%
\pgfpathlineto{\pgfqpoint{5.479617in}{0.739656in}}%
\pgfpathlineto{\pgfqpoint{5.479320in}{0.739656in}}%
\pgfpathlineto{\pgfqpoint{5.479022in}{0.739656in}}%
\pgfpathlineto{\pgfqpoint{5.478725in}{0.739656in}}%
\pgfpathlineto{\pgfqpoint{5.478428in}{0.739656in}}%
\pgfpathlineto{\pgfqpoint{5.478130in}{0.739656in}}%
\pgfpathlineto{\pgfqpoint{5.477833in}{0.739656in}}%
\pgfpathlineto{\pgfqpoint{5.477535in}{0.739656in}}%
\pgfpathlineto{\pgfqpoint{5.477238in}{0.739656in}}%
\pgfpathlineto{\pgfqpoint{5.476940in}{0.739656in}}%
\pgfpathlineto{\pgfqpoint{5.476643in}{0.739656in}}%
\pgfpathlineto{\pgfqpoint{5.476345in}{0.739656in}}%
\pgfpathlineto{\pgfqpoint{5.476048in}{0.739656in}}%
\pgfpathlineto{\pgfqpoint{5.475750in}{0.739656in}}%
\pgfpathlineto{\pgfqpoint{5.475453in}{0.739656in}}%
\pgfpathlineto{\pgfqpoint{5.475155in}{0.739656in}}%
\pgfpathlineto{\pgfqpoint{5.474858in}{0.739656in}}%
\pgfpathlineto{\pgfqpoint{5.474560in}{0.739656in}}%
\pgfpathlineto{\pgfqpoint{5.474263in}{0.739656in}}%
\pgfpathlineto{\pgfqpoint{5.473965in}{0.739656in}}%
\pgfpathlineto{\pgfqpoint{5.473668in}{0.739656in}}%
\pgfpathlineto{\pgfqpoint{5.473370in}{0.739656in}}%
\pgfpathlineto{\pgfqpoint{5.473073in}{0.739656in}}%
\pgfpathlineto{\pgfqpoint{5.472775in}{0.739656in}}%
\pgfpathlineto{\pgfqpoint{5.472478in}{0.739656in}}%
\pgfpathlineto{\pgfqpoint{5.472180in}{0.739656in}}%
\pgfpathlineto{\pgfqpoint{5.471883in}{0.739656in}}%
\pgfpathlineto{\pgfqpoint{5.471586in}{0.739656in}}%
\pgfpathlineto{\pgfqpoint{5.471288in}{0.739656in}}%
\pgfpathlineto{\pgfqpoint{5.470991in}{0.739656in}}%
\pgfpathlineto{\pgfqpoint{5.470693in}{0.739656in}}%
\pgfpathlineto{\pgfqpoint{5.470396in}{0.739656in}}%
\pgfpathlineto{\pgfqpoint{5.470098in}{0.739656in}}%
\pgfpathlineto{\pgfqpoint{5.469801in}{0.739656in}}%
\pgfpathlineto{\pgfqpoint{5.469503in}{0.739656in}}%
\pgfpathlineto{\pgfqpoint{5.469206in}{0.739656in}}%
\pgfpathlineto{\pgfqpoint{5.468908in}{0.739656in}}%
\pgfpathlineto{\pgfqpoint{5.468611in}{0.739656in}}%
\pgfpathlineto{\pgfqpoint{5.468313in}{0.739656in}}%
\pgfpathlineto{\pgfqpoint{5.468016in}{0.739656in}}%
\pgfpathlineto{\pgfqpoint{5.467718in}{0.739656in}}%
\pgfpathlineto{\pgfqpoint{5.467421in}{0.739656in}}%
\pgfpathlineto{\pgfqpoint{5.467123in}{0.739656in}}%
\pgfpathlineto{\pgfqpoint{5.466826in}{0.739656in}}%
\pgfpathlineto{\pgfqpoint{5.466528in}{0.739656in}}%
\pgfpathlineto{\pgfqpoint{5.466231in}{0.739656in}}%
\pgfpathlineto{\pgfqpoint{5.465933in}{0.739656in}}%
\pgfpathlineto{\pgfqpoint{5.465636in}{0.739656in}}%
\pgfpathlineto{\pgfqpoint{5.465338in}{0.739656in}}%
\pgfpathlineto{\pgfqpoint{5.465041in}{0.739656in}}%
\pgfpathlineto{\pgfqpoint{5.464744in}{0.739656in}}%
\pgfpathlineto{\pgfqpoint{5.464446in}{0.739656in}}%
\pgfpathlineto{\pgfqpoint{5.464149in}{0.739656in}}%
\pgfpathlineto{\pgfqpoint{5.463851in}{0.739656in}}%
\pgfpathlineto{\pgfqpoint{5.463554in}{0.739656in}}%
\pgfpathlineto{\pgfqpoint{5.463256in}{0.739656in}}%
\pgfpathlineto{\pgfqpoint{5.462959in}{0.739656in}}%
\pgfpathlineto{\pgfqpoint{5.462661in}{0.739656in}}%
\pgfpathlineto{\pgfqpoint{5.462364in}{0.739656in}}%
\pgfpathlineto{\pgfqpoint{5.462066in}{0.739656in}}%
\pgfpathlineto{\pgfqpoint{5.461769in}{0.739656in}}%
\pgfpathlineto{\pgfqpoint{5.461471in}{0.739656in}}%
\pgfpathlineto{\pgfqpoint{5.461174in}{0.739656in}}%
\pgfpathlineto{\pgfqpoint{5.460876in}{0.739656in}}%
\pgfpathlineto{\pgfqpoint{5.460579in}{0.739656in}}%
\pgfpathlineto{\pgfqpoint{5.460281in}{0.739656in}}%
\pgfpathlineto{\pgfqpoint{5.459984in}{0.739656in}}%
\pgfpathlineto{\pgfqpoint{5.459686in}{0.739656in}}%
\pgfpathlineto{\pgfqpoint{5.459389in}{0.739656in}}%
\pgfpathlineto{\pgfqpoint{5.459091in}{0.739656in}}%
\pgfpathlineto{\pgfqpoint{5.458794in}{0.739656in}}%
\pgfpathlineto{\pgfqpoint{5.458496in}{0.739656in}}%
\pgfpathlineto{\pgfqpoint{5.458199in}{0.739656in}}%
\pgfpathlineto{\pgfqpoint{5.457902in}{0.739656in}}%
\pgfpathlineto{\pgfqpoint{5.457604in}{0.739656in}}%
\pgfpathlineto{\pgfqpoint{5.457307in}{0.739656in}}%
\pgfpathlineto{\pgfqpoint{5.457009in}{0.739656in}}%
\pgfpathlineto{\pgfqpoint{5.456712in}{0.739656in}}%
\pgfpathlineto{\pgfqpoint{5.456414in}{0.739656in}}%
\pgfpathlineto{\pgfqpoint{5.456117in}{0.739656in}}%
\pgfpathlineto{\pgfqpoint{5.455819in}{0.739656in}}%
\pgfpathlineto{\pgfqpoint{5.455522in}{0.739656in}}%
\pgfpathlineto{\pgfqpoint{5.455224in}{0.739656in}}%
\pgfpathlineto{\pgfqpoint{5.454927in}{0.739656in}}%
\pgfpathlineto{\pgfqpoint{5.454629in}{0.739656in}}%
\pgfpathlineto{\pgfqpoint{5.454332in}{0.739656in}}%
\pgfpathlineto{\pgfqpoint{5.454034in}{0.739656in}}%
\pgfpathlineto{\pgfqpoint{5.453737in}{0.739656in}}%
\pgfpathlineto{\pgfqpoint{5.453439in}{0.739656in}}%
\pgfpathlineto{\pgfqpoint{5.453142in}{0.739656in}}%
\pgfpathlineto{\pgfqpoint{5.452844in}{0.739656in}}%
\pgfpathlineto{\pgfqpoint{5.452547in}{0.739656in}}%
\pgfpathlineto{\pgfqpoint{5.452249in}{0.739656in}}%
\pgfpathlineto{\pgfqpoint{5.451952in}{0.739656in}}%
\pgfpathlineto{\pgfqpoint{5.451655in}{0.739656in}}%
\pgfpathlineto{\pgfqpoint{5.451357in}{0.739656in}}%
\pgfpathlineto{\pgfqpoint{5.451060in}{0.739656in}}%
\pgfpathlineto{\pgfqpoint{5.450762in}{0.739656in}}%
\pgfpathlineto{\pgfqpoint{5.450465in}{0.739656in}}%
\pgfpathlineto{\pgfqpoint{5.450167in}{0.739656in}}%
\pgfpathlineto{\pgfqpoint{5.449870in}{0.739656in}}%
\pgfpathlineto{\pgfqpoint{5.449572in}{0.739656in}}%
\pgfpathlineto{\pgfqpoint{5.449275in}{0.739656in}}%
\pgfpathlineto{\pgfqpoint{5.448977in}{0.739656in}}%
\pgfpathlineto{\pgfqpoint{5.448680in}{0.739656in}}%
\pgfpathlineto{\pgfqpoint{5.448382in}{0.739656in}}%
\pgfpathlineto{\pgfqpoint{5.448085in}{0.739656in}}%
\pgfpathlineto{\pgfqpoint{5.447787in}{0.739656in}}%
\pgfpathlineto{\pgfqpoint{5.447490in}{0.739656in}}%
\pgfpathlineto{\pgfqpoint{5.447192in}{0.739656in}}%
\pgfpathlineto{\pgfqpoint{5.446895in}{0.739656in}}%
\pgfpathlineto{\pgfqpoint{5.446597in}{0.739656in}}%
\pgfpathlineto{\pgfqpoint{5.446300in}{0.739656in}}%
\pgfpathlineto{\pgfqpoint{5.446002in}{0.739656in}}%
\pgfpathlineto{\pgfqpoint{5.445705in}{0.739656in}}%
\pgfpathlineto{\pgfqpoint{5.445407in}{0.739656in}}%
\pgfpathlineto{\pgfqpoint{5.445110in}{0.739656in}}%
\pgfpathlineto{\pgfqpoint{5.444813in}{0.739656in}}%
\pgfpathlineto{\pgfqpoint{5.444515in}{0.739656in}}%
\pgfpathlineto{\pgfqpoint{5.444218in}{0.739656in}}%
\pgfpathlineto{\pgfqpoint{5.443920in}{0.739656in}}%
\pgfpathlineto{\pgfqpoint{5.443623in}{0.739656in}}%
\pgfpathlineto{\pgfqpoint{5.443325in}{0.739656in}}%
\pgfpathlineto{\pgfqpoint{5.443028in}{0.739656in}}%
\pgfpathlineto{\pgfqpoint{5.442730in}{0.739656in}}%
\pgfpathlineto{\pgfqpoint{5.442433in}{0.739656in}}%
\pgfpathlineto{\pgfqpoint{5.442135in}{0.739656in}}%
\pgfpathlineto{\pgfqpoint{5.441838in}{0.739656in}}%
\pgfpathlineto{\pgfqpoint{5.441540in}{0.739656in}}%
\pgfpathlineto{\pgfqpoint{5.441243in}{0.739656in}}%
\pgfpathlineto{\pgfqpoint{5.440945in}{0.739656in}}%
\pgfpathlineto{\pgfqpoint{5.440648in}{0.739656in}}%
\pgfpathlineto{\pgfqpoint{5.440350in}{0.739656in}}%
\pgfpathlineto{\pgfqpoint{5.440053in}{0.739656in}}%
\pgfpathlineto{\pgfqpoint{5.439755in}{0.739656in}}%
\pgfpathlineto{\pgfqpoint{5.439458in}{0.739656in}}%
\pgfpathlineto{\pgfqpoint{5.439160in}{0.739656in}}%
\pgfpathlineto{\pgfqpoint{5.438863in}{0.739656in}}%
\pgfpathlineto{\pgfqpoint{5.438565in}{0.739656in}}%
\pgfpathlineto{\pgfqpoint{5.438268in}{0.739656in}}%
\pgfpathlineto{\pgfqpoint{5.437971in}{0.739656in}}%
\pgfpathlineto{\pgfqpoint{5.437673in}{0.739656in}}%
\pgfpathlineto{\pgfqpoint{5.437376in}{0.739656in}}%
\pgfpathlineto{\pgfqpoint{5.437078in}{0.739656in}}%
\pgfpathlineto{\pgfqpoint{5.436781in}{0.739656in}}%
\pgfpathlineto{\pgfqpoint{5.436483in}{0.739656in}}%
\pgfpathlineto{\pgfqpoint{5.436186in}{0.739656in}}%
\pgfpathlineto{\pgfqpoint{5.435888in}{0.739656in}}%
\pgfpathlineto{\pgfqpoint{5.435591in}{0.739656in}}%
\pgfpathlineto{\pgfqpoint{5.435293in}{0.739656in}}%
\pgfpathlineto{\pgfqpoint{5.434996in}{0.739656in}}%
\pgfpathlineto{\pgfqpoint{5.434698in}{0.739656in}}%
\pgfpathlineto{\pgfqpoint{5.434401in}{0.739656in}}%
\pgfpathlineto{\pgfqpoint{5.434103in}{0.739656in}}%
\pgfpathlineto{\pgfqpoint{5.433806in}{0.739656in}}%
\pgfpathlineto{\pgfqpoint{5.433508in}{0.739656in}}%
\pgfpathlineto{\pgfqpoint{5.433211in}{0.739656in}}%
\pgfpathlineto{\pgfqpoint{5.432913in}{0.739656in}}%
\pgfpathlineto{\pgfqpoint{5.432616in}{0.739656in}}%
\pgfpathlineto{\pgfqpoint{5.432318in}{0.739656in}}%
\pgfpathlineto{\pgfqpoint{5.432021in}{0.739656in}}%
\pgfpathlineto{\pgfqpoint{5.431724in}{0.739656in}}%
\pgfpathlineto{\pgfqpoint{5.431426in}{0.739656in}}%
\pgfpathlineto{\pgfqpoint{5.431129in}{0.739656in}}%
\pgfpathlineto{\pgfqpoint{5.430831in}{0.739656in}}%
\pgfpathlineto{\pgfqpoint{5.430534in}{0.739656in}}%
\pgfpathlineto{\pgfqpoint{5.430236in}{0.739656in}}%
\pgfpathlineto{\pgfqpoint{5.429939in}{0.739656in}}%
\pgfpathlineto{\pgfqpoint{5.429641in}{0.739656in}}%
\pgfpathlineto{\pgfqpoint{5.429344in}{0.739656in}}%
\pgfpathlineto{\pgfqpoint{5.429046in}{0.739656in}}%
\pgfpathlineto{\pgfqpoint{5.428749in}{0.739656in}}%
\pgfpathlineto{\pgfqpoint{5.428451in}{0.739656in}}%
\pgfpathlineto{\pgfqpoint{5.428154in}{0.739656in}}%
\pgfpathlineto{\pgfqpoint{5.427856in}{0.739656in}}%
\pgfpathlineto{\pgfqpoint{5.427559in}{0.739656in}}%
\pgfpathlineto{\pgfqpoint{5.427261in}{0.739656in}}%
\pgfpathlineto{\pgfqpoint{5.426964in}{0.739656in}}%
\pgfpathlineto{\pgfqpoint{5.426666in}{0.739656in}}%
\pgfpathlineto{\pgfqpoint{5.426369in}{0.739656in}}%
\pgfpathlineto{\pgfqpoint{5.426071in}{0.739656in}}%
\pgfpathlineto{\pgfqpoint{5.425774in}{0.739656in}}%
\pgfpathlineto{\pgfqpoint{5.425476in}{0.739656in}}%
\pgfpathlineto{\pgfqpoint{5.425179in}{0.739656in}}%
\pgfpathlineto{\pgfqpoint{5.424882in}{0.739656in}}%
\pgfpathlineto{\pgfqpoint{5.424584in}{0.739656in}}%
\pgfpathlineto{\pgfqpoint{5.424287in}{0.739656in}}%
\pgfpathlineto{\pgfqpoint{5.423989in}{0.739656in}}%
\pgfpathlineto{\pgfqpoint{5.423692in}{0.739656in}}%
\pgfpathlineto{\pgfqpoint{5.423394in}{0.739656in}}%
\pgfpathlineto{\pgfqpoint{5.423097in}{0.739656in}}%
\pgfpathlineto{\pgfqpoint{5.422799in}{0.739656in}}%
\pgfpathlineto{\pgfqpoint{5.422502in}{0.739656in}}%
\pgfpathlineto{\pgfqpoint{5.422204in}{0.739656in}}%
\pgfpathlineto{\pgfqpoint{5.421907in}{0.739656in}}%
\pgfpathlineto{\pgfqpoint{5.421609in}{0.739656in}}%
\pgfpathlineto{\pgfqpoint{5.421312in}{0.739656in}}%
\pgfpathlineto{\pgfqpoint{5.421014in}{0.739656in}}%
\pgfpathlineto{\pgfqpoint{5.420717in}{0.739656in}}%
\pgfpathlineto{\pgfqpoint{5.420419in}{0.739656in}}%
\pgfpathlineto{\pgfqpoint{5.420122in}{0.739656in}}%
\pgfpathlineto{\pgfqpoint{5.419824in}{0.739656in}}%
\pgfpathlineto{\pgfqpoint{5.419527in}{0.739656in}}%
\pgfpathlineto{\pgfqpoint{5.419229in}{0.739656in}}%
\pgfpathlineto{\pgfqpoint{5.418932in}{0.739656in}}%
\pgfpathlineto{\pgfqpoint{5.418634in}{0.739656in}}%
\pgfpathlineto{\pgfqpoint{5.418337in}{0.739656in}}%
\pgfpathlineto{\pgfqpoint{5.418040in}{0.739656in}}%
\pgfpathlineto{\pgfqpoint{5.417742in}{0.739656in}}%
\pgfpathlineto{\pgfqpoint{5.417445in}{0.739656in}}%
\pgfpathlineto{\pgfqpoint{5.417147in}{0.739656in}}%
\pgfpathlineto{\pgfqpoint{5.416850in}{0.739656in}}%
\pgfpathlineto{\pgfqpoint{5.416552in}{0.739656in}}%
\pgfpathlineto{\pgfqpoint{5.416255in}{0.739656in}}%
\pgfpathlineto{\pgfqpoint{5.415957in}{0.739656in}}%
\pgfpathlineto{\pgfqpoint{5.415660in}{0.739656in}}%
\pgfpathlineto{\pgfqpoint{5.415362in}{0.739656in}}%
\pgfpathlineto{\pgfqpoint{5.415065in}{0.739656in}}%
\pgfpathlineto{\pgfqpoint{5.414767in}{0.739656in}}%
\pgfpathlineto{\pgfqpoint{5.414470in}{0.739656in}}%
\pgfpathlineto{\pgfqpoint{5.414172in}{0.739656in}}%
\pgfpathlineto{\pgfqpoint{5.413875in}{0.739656in}}%
\pgfpathlineto{\pgfqpoint{5.413577in}{0.739656in}}%
\pgfpathlineto{\pgfqpoint{5.413280in}{0.739656in}}%
\pgfpathlineto{\pgfqpoint{5.412982in}{0.739656in}}%
\pgfpathlineto{\pgfqpoint{5.412685in}{0.739656in}}%
\pgfpathlineto{\pgfqpoint{5.412387in}{0.739656in}}%
\pgfpathlineto{\pgfqpoint{5.412090in}{0.739656in}}%
\pgfpathlineto{\pgfqpoint{5.411793in}{0.739656in}}%
\pgfpathlineto{\pgfqpoint{5.411495in}{0.739656in}}%
\pgfpathlineto{\pgfqpoint{5.411198in}{0.739656in}}%
\pgfpathlineto{\pgfqpoint{5.410900in}{0.739656in}}%
\pgfpathlineto{\pgfqpoint{5.410603in}{0.739656in}}%
\pgfpathlineto{\pgfqpoint{5.410305in}{0.739656in}}%
\pgfpathlineto{\pgfqpoint{5.410008in}{0.739656in}}%
\pgfpathlineto{\pgfqpoint{5.409710in}{0.739656in}}%
\pgfpathlineto{\pgfqpoint{5.409413in}{0.739656in}}%
\pgfpathlineto{\pgfqpoint{5.409115in}{0.739656in}}%
\pgfpathlineto{\pgfqpoint{5.408818in}{0.739656in}}%
\pgfpathlineto{\pgfqpoint{5.408520in}{0.739656in}}%
\pgfpathlineto{\pgfqpoint{5.408223in}{0.739656in}}%
\pgfpathlineto{\pgfqpoint{5.407925in}{0.739656in}}%
\pgfpathlineto{\pgfqpoint{5.407628in}{0.739656in}}%
\pgfpathlineto{\pgfqpoint{5.407330in}{0.739656in}}%
\pgfpathlineto{\pgfqpoint{5.407033in}{0.739656in}}%
\pgfpathlineto{\pgfqpoint{5.406735in}{0.739656in}}%
\pgfpathlineto{\pgfqpoint{5.406438in}{0.739656in}}%
\pgfpathlineto{\pgfqpoint{5.406140in}{0.739656in}}%
\pgfpathlineto{\pgfqpoint{5.405843in}{0.739656in}}%
\pgfpathlineto{\pgfqpoint{5.405545in}{0.739656in}}%
\pgfpathlineto{\pgfqpoint{5.405248in}{0.739656in}}%
\pgfpathlineto{\pgfqpoint{5.404951in}{0.739656in}}%
\pgfpathlineto{\pgfqpoint{5.404653in}{0.739656in}}%
\pgfpathlineto{\pgfqpoint{5.404356in}{0.739656in}}%
\pgfpathlineto{\pgfqpoint{5.404058in}{0.739656in}}%
\pgfpathlineto{\pgfqpoint{5.403761in}{0.739656in}}%
\pgfpathlineto{\pgfqpoint{5.403463in}{0.739656in}}%
\pgfpathlineto{\pgfqpoint{5.403166in}{0.739656in}}%
\pgfpathlineto{\pgfqpoint{5.402868in}{0.739656in}}%
\pgfpathlineto{\pgfqpoint{5.402571in}{0.739656in}}%
\pgfpathlineto{\pgfqpoint{5.402273in}{0.739656in}}%
\pgfpathlineto{\pgfqpoint{5.401976in}{0.739656in}}%
\pgfpathlineto{\pgfqpoint{5.401678in}{0.739656in}}%
\pgfpathlineto{\pgfqpoint{5.401381in}{0.739656in}}%
\pgfpathlineto{\pgfqpoint{5.401083in}{0.739656in}}%
\pgfpathlineto{\pgfqpoint{5.400786in}{0.739656in}}%
\pgfpathlineto{\pgfqpoint{5.400488in}{0.739656in}}%
\pgfpathlineto{\pgfqpoint{5.400191in}{0.739656in}}%
\pgfpathlineto{\pgfqpoint{5.399893in}{0.739656in}}%
\pgfpathlineto{\pgfqpoint{5.399596in}{0.739656in}}%
\pgfpathlineto{\pgfqpoint{5.399298in}{0.739656in}}%
\pgfpathlineto{\pgfqpoint{5.399001in}{0.739656in}}%
\pgfpathlineto{\pgfqpoint{5.398703in}{0.739656in}}%
\pgfpathlineto{\pgfqpoint{5.398406in}{0.739656in}}%
\pgfpathlineto{\pgfqpoint{5.398109in}{0.739656in}}%
\pgfpathlineto{\pgfqpoint{5.397811in}{0.739656in}}%
\pgfpathlineto{\pgfqpoint{5.397514in}{0.739656in}}%
\pgfpathlineto{\pgfqpoint{5.397216in}{0.739656in}}%
\pgfpathlineto{\pgfqpoint{5.396919in}{0.739656in}}%
\pgfpathlineto{\pgfqpoint{5.396621in}{0.739656in}}%
\pgfpathlineto{\pgfqpoint{5.396324in}{0.739656in}}%
\pgfpathlineto{\pgfqpoint{5.396026in}{0.739656in}}%
\pgfpathlineto{\pgfqpoint{5.395729in}{0.739656in}}%
\pgfpathlineto{\pgfqpoint{5.395431in}{0.739656in}}%
\pgfpathlineto{\pgfqpoint{5.395134in}{0.739656in}}%
\pgfpathlineto{\pgfqpoint{5.394836in}{0.739656in}}%
\pgfpathlineto{\pgfqpoint{5.394539in}{0.739656in}}%
\pgfpathlineto{\pgfqpoint{5.394241in}{0.739656in}}%
\pgfpathlineto{\pgfqpoint{5.393944in}{0.739656in}}%
\pgfpathlineto{\pgfqpoint{5.393646in}{0.739656in}}%
\pgfpathlineto{\pgfqpoint{5.393349in}{0.739656in}}%
\pgfpathlineto{\pgfqpoint{5.393051in}{0.739656in}}%
\pgfpathlineto{\pgfqpoint{5.392754in}{0.739656in}}%
\pgfpathlineto{\pgfqpoint{5.392456in}{0.739656in}}%
\pgfpathlineto{\pgfqpoint{5.392159in}{0.739656in}}%
\pgfpathlineto{\pgfqpoint{5.391862in}{0.739656in}}%
\pgfpathlineto{\pgfqpoint{5.391564in}{0.739656in}}%
\pgfpathlineto{\pgfqpoint{5.391267in}{0.739656in}}%
\pgfpathlineto{\pgfqpoint{5.390969in}{0.739656in}}%
\pgfpathlineto{\pgfqpoint{5.390672in}{0.739656in}}%
\pgfpathlineto{\pgfqpoint{5.390374in}{0.739656in}}%
\pgfpathlineto{\pgfqpoint{5.390077in}{0.739656in}}%
\pgfpathlineto{\pgfqpoint{5.389779in}{0.739656in}}%
\pgfpathlineto{\pgfqpoint{5.389482in}{0.739656in}}%
\pgfpathlineto{\pgfqpoint{5.389184in}{0.739656in}}%
\pgfpathlineto{\pgfqpoint{5.388887in}{0.739656in}}%
\pgfpathlineto{\pgfqpoint{5.388589in}{0.739656in}}%
\pgfpathlineto{\pgfqpoint{5.388292in}{0.739656in}}%
\pgfpathlineto{\pgfqpoint{5.387994in}{0.739656in}}%
\pgfpathlineto{\pgfqpoint{5.387697in}{0.739656in}}%
\pgfpathlineto{\pgfqpoint{5.387399in}{0.739656in}}%
\pgfpathlineto{\pgfqpoint{5.387102in}{0.739656in}}%
\pgfpathlineto{\pgfqpoint{5.386804in}{0.739656in}}%
\pgfpathlineto{\pgfqpoint{5.386507in}{0.739656in}}%
\pgfpathlineto{\pgfqpoint{5.386209in}{0.739656in}}%
\pgfpathlineto{\pgfqpoint{5.385912in}{0.739656in}}%
\pgfpathlineto{\pgfqpoint{5.385614in}{0.739656in}}%
\pgfpathlineto{\pgfqpoint{5.385317in}{0.739656in}}%
\pgfpathlineto{\pgfqpoint{5.385020in}{0.739656in}}%
\pgfpathlineto{\pgfqpoint{5.384722in}{0.739656in}}%
\pgfpathlineto{\pgfqpoint{5.384425in}{0.739656in}}%
\pgfpathlineto{\pgfqpoint{5.384127in}{0.739656in}}%
\pgfpathlineto{\pgfqpoint{5.383830in}{0.739656in}}%
\pgfpathlineto{\pgfqpoint{5.383532in}{0.739656in}}%
\pgfpathlineto{\pgfqpoint{5.383235in}{0.739656in}}%
\pgfpathlineto{\pgfqpoint{5.382937in}{0.739656in}}%
\pgfpathlineto{\pgfqpoint{5.382640in}{0.739656in}}%
\pgfpathlineto{\pgfqpoint{5.382342in}{0.739656in}}%
\pgfpathlineto{\pgfqpoint{5.382045in}{0.739656in}}%
\pgfpathlineto{\pgfqpoint{5.381747in}{0.739656in}}%
\pgfpathlineto{\pgfqpoint{5.381450in}{0.739656in}}%
\pgfpathlineto{\pgfqpoint{5.381152in}{0.739656in}}%
\pgfpathlineto{\pgfqpoint{5.380855in}{0.739656in}}%
\pgfpathlineto{\pgfqpoint{5.380557in}{0.739656in}}%
\pgfpathlineto{\pgfqpoint{5.380260in}{0.739656in}}%
\pgfpathlineto{\pgfqpoint{5.379962in}{0.739656in}}%
\pgfpathlineto{\pgfqpoint{5.379665in}{0.739656in}}%
\pgfpathlineto{\pgfqpoint{5.379367in}{0.739656in}}%
\pgfpathlineto{\pgfqpoint{5.379070in}{0.739656in}}%
\pgfpathlineto{\pgfqpoint{5.378772in}{0.739656in}}%
\pgfpathlineto{\pgfqpoint{5.378475in}{0.739656in}}%
\pgfpathlineto{\pgfqpoint{5.378178in}{0.739656in}}%
\pgfpathlineto{\pgfqpoint{5.377880in}{0.739656in}}%
\pgfpathlineto{\pgfqpoint{5.377583in}{0.739656in}}%
\pgfpathlineto{\pgfqpoint{5.377285in}{0.739656in}}%
\pgfpathlineto{\pgfqpoint{5.376988in}{0.739656in}}%
\pgfpathlineto{\pgfqpoint{5.376690in}{0.739656in}}%
\pgfpathlineto{\pgfqpoint{5.376393in}{0.739656in}}%
\pgfpathlineto{\pgfqpoint{5.376095in}{0.739656in}}%
\pgfpathlineto{\pgfqpoint{5.375798in}{0.739656in}}%
\pgfpathlineto{\pgfqpoint{5.375500in}{0.739656in}}%
\pgfpathlineto{\pgfqpoint{5.375203in}{0.739656in}}%
\pgfpathlineto{\pgfqpoint{5.374905in}{0.739656in}}%
\pgfpathlineto{\pgfqpoint{5.374608in}{0.739656in}}%
\pgfpathlineto{\pgfqpoint{5.374310in}{0.739656in}}%
\pgfpathlineto{\pgfqpoint{5.374013in}{0.739656in}}%
\pgfpathlineto{\pgfqpoint{5.373715in}{0.739656in}}%
\pgfpathlineto{\pgfqpoint{5.373418in}{0.739656in}}%
\pgfpathlineto{\pgfqpoint{5.373120in}{0.739656in}}%
\pgfpathlineto{\pgfqpoint{5.372823in}{0.739656in}}%
\pgfpathlineto{\pgfqpoint{5.372525in}{0.739656in}}%
\pgfpathlineto{\pgfqpoint{5.372228in}{0.739656in}}%
\pgfpathlineto{\pgfqpoint{5.371931in}{0.739656in}}%
\pgfpathlineto{\pgfqpoint{5.371633in}{0.739656in}}%
\pgfpathlineto{\pgfqpoint{5.371336in}{0.739656in}}%
\pgfpathlineto{\pgfqpoint{5.371038in}{0.739656in}}%
\pgfpathlineto{\pgfqpoint{5.370741in}{0.739656in}}%
\pgfpathlineto{\pgfqpoint{5.370443in}{0.739656in}}%
\pgfpathlineto{\pgfqpoint{5.370146in}{0.739656in}}%
\pgfpathlineto{\pgfqpoint{5.369848in}{0.739656in}}%
\pgfpathlineto{\pgfqpoint{5.369551in}{0.739656in}}%
\pgfpathlineto{\pgfqpoint{5.369253in}{0.739656in}}%
\pgfpathlineto{\pgfqpoint{5.368956in}{0.739656in}}%
\pgfpathlineto{\pgfqpoint{5.368658in}{0.739656in}}%
\pgfpathlineto{\pgfqpoint{5.368361in}{0.739656in}}%
\pgfpathlineto{\pgfqpoint{5.368063in}{0.739656in}}%
\pgfpathlineto{\pgfqpoint{5.367766in}{0.739656in}}%
\pgfpathlineto{\pgfqpoint{5.367468in}{0.739656in}}%
\pgfpathlineto{\pgfqpoint{5.367171in}{0.739656in}}%
\pgfpathlineto{\pgfqpoint{5.366873in}{0.739656in}}%
\pgfpathlineto{\pgfqpoint{5.366576in}{0.739656in}}%
\pgfpathlineto{\pgfqpoint{5.366278in}{0.739656in}}%
\pgfpathlineto{\pgfqpoint{5.365981in}{0.739656in}}%
\pgfpathlineto{\pgfqpoint{5.365683in}{0.739656in}}%
\pgfpathlineto{\pgfqpoint{5.365386in}{0.739656in}}%
\pgfpathlineto{\pgfqpoint{5.365089in}{0.739656in}}%
\pgfpathlineto{\pgfqpoint{5.364791in}{0.739656in}}%
\pgfpathlineto{\pgfqpoint{5.364494in}{0.739656in}}%
\pgfpathlineto{\pgfqpoint{5.364196in}{0.739656in}}%
\pgfpathlineto{\pgfqpoint{5.363899in}{0.739656in}}%
\pgfpathlineto{\pgfqpoint{5.363601in}{0.739656in}}%
\pgfpathlineto{\pgfqpoint{5.363304in}{0.739656in}}%
\pgfpathlineto{\pgfqpoint{5.363006in}{0.739656in}}%
\pgfpathlineto{\pgfqpoint{5.362709in}{0.739656in}}%
\pgfpathlineto{\pgfqpoint{5.362411in}{0.739656in}}%
\pgfpathlineto{\pgfqpoint{5.362114in}{0.739656in}}%
\pgfpathlineto{\pgfqpoint{5.361816in}{0.739656in}}%
\pgfpathlineto{\pgfqpoint{5.361519in}{0.739656in}}%
\pgfpathlineto{\pgfqpoint{5.361221in}{0.739656in}}%
\pgfpathlineto{\pgfqpoint{5.360924in}{0.739656in}}%
\pgfpathlineto{\pgfqpoint{5.360626in}{0.739656in}}%
\pgfpathlineto{\pgfqpoint{5.360329in}{0.739656in}}%
\pgfpathlineto{\pgfqpoint{5.360031in}{0.739656in}}%
\pgfpathlineto{\pgfqpoint{5.359734in}{0.739656in}}%
\pgfpathlineto{\pgfqpoint{5.359436in}{0.739656in}}%
\pgfpathlineto{\pgfqpoint{5.359139in}{0.739656in}}%
\pgfpathlineto{\pgfqpoint{5.358841in}{0.739656in}}%
\pgfpathlineto{\pgfqpoint{5.358544in}{0.739656in}}%
\pgfpathlineto{\pgfqpoint{5.358247in}{0.739656in}}%
\pgfpathlineto{\pgfqpoint{5.357949in}{0.739656in}}%
\pgfpathlineto{\pgfqpoint{5.357652in}{0.739656in}}%
\pgfpathlineto{\pgfqpoint{5.357354in}{0.739656in}}%
\pgfpathlineto{\pgfqpoint{5.357057in}{0.739656in}}%
\pgfpathlineto{\pgfqpoint{5.356759in}{0.739656in}}%
\pgfpathlineto{\pgfqpoint{5.356462in}{0.739656in}}%
\pgfpathlineto{\pgfqpoint{5.356164in}{0.739656in}}%
\pgfpathlineto{\pgfqpoint{5.355867in}{0.739656in}}%
\pgfpathlineto{\pgfqpoint{5.355569in}{0.739656in}}%
\pgfpathlineto{\pgfqpoint{5.355272in}{0.739656in}}%
\pgfpathlineto{\pgfqpoint{5.354974in}{0.739656in}}%
\pgfpathlineto{\pgfqpoint{5.354677in}{0.739656in}}%
\pgfpathlineto{\pgfqpoint{5.354379in}{0.739656in}}%
\pgfpathlineto{\pgfqpoint{5.354082in}{0.739656in}}%
\pgfpathlineto{\pgfqpoint{5.353784in}{0.739656in}}%
\pgfpathlineto{\pgfqpoint{5.353487in}{0.739656in}}%
\pgfpathlineto{\pgfqpoint{5.353189in}{0.739656in}}%
\pgfpathlineto{\pgfqpoint{5.352892in}{0.739656in}}%
\pgfpathlineto{\pgfqpoint{5.352594in}{0.739656in}}%
\pgfpathlineto{\pgfqpoint{5.352297in}{0.739656in}}%
\pgfpathlineto{\pgfqpoint{5.352000in}{0.739656in}}%
\pgfpathlineto{\pgfqpoint{5.351702in}{0.739656in}}%
\pgfpathlineto{\pgfqpoint{5.351405in}{0.739656in}}%
\pgfpathlineto{\pgfqpoint{5.351107in}{0.739656in}}%
\pgfpathlineto{\pgfqpoint{5.350810in}{0.739656in}}%
\pgfpathlineto{\pgfqpoint{5.350512in}{0.739656in}}%
\pgfpathlineto{\pgfqpoint{5.350215in}{0.739656in}}%
\pgfpathlineto{\pgfqpoint{5.349917in}{0.739656in}}%
\pgfpathlineto{\pgfqpoint{5.349620in}{0.739656in}}%
\pgfpathlineto{\pgfqpoint{5.349322in}{0.739656in}}%
\pgfpathlineto{\pgfqpoint{5.349025in}{0.739656in}}%
\pgfpathlineto{\pgfqpoint{5.348727in}{0.739656in}}%
\pgfpathlineto{\pgfqpoint{5.348430in}{0.739656in}}%
\pgfpathlineto{\pgfqpoint{5.348132in}{0.739656in}}%
\pgfpathlineto{\pgfqpoint{5.347835in}{0.739656in}}%
\pgfpathlineto{\pgfqpoint{5.347537in}{0.739656in}}%
\pgfpathlineto{\pgfqpoint{5.347240in}{0.739656in}}%
\pgfpathlineto{\pgfqpoint{5.346942in}{0.739656in}}%
\pgfpathlineto{\pgfqpoint{5.346645in}{0.739656in}}%
\pgfpathlineto{\pgfqpoint{5.346347in}{0.739656in}}%
\pgfpathlineto{\pgfqpoint{5.346050in}{0.739656in}}%
\pgfpathlineto{\pgfqpoint{5.345752in}{0.739656in}}%
\pgfpathlineto{\pgfqpoint{5.345455in}{0.739656in}}%
\pgfpathlineto{\pgfqpoint{5.345158in}{0.739656in}}%
\pgfpathlineto{\pgfqpoint{5.344860in}{0.739656in}}%
\pgfpathlineto{\pgfqpoint{5.344563in}{0.739656in}}%
\pgfpathlineto{\pgfqpoint{5.344265in}{0.739656in}}%
\pgfpathlineto{\pgfqpoint{5.343968in}{0.739656in}}%
\pgfpathlineto{\pgfqpoint{5.343670in}{0.739656in}}%
\pgfpathlineto{\pgfqpoint{5.343373in}{0.739656in}}%
\pgfpathlineto{\pgfqpoint{5.343075in}{0.739656in}}%
\pgfpathlineto{\pgfqpoint{5.342778in}{0.739656in}}%
\pgfpathlineto{\pgfqpoint{5.342480in}{0.739656in}}%
\pgfpathlineto{\pgfqpoint{5.342183in}{0.739656in}}%
\pgfpathlineto{\pgfqpoint{5.341885in}{0.739656in}}%
\pgfpathlineto{\pgfqpoint{5.341588in}{0.739656in}}%
\pgfpathlineto{\pgfqpoint{5.341290in}{0.739656in}}%
\pgfpathlineto{\pgfqpoint{5.340993in}{0.739656in}}%
\pgfpathlineto{\pgfqpoint{5.340695in}{0.739656in}}%
\pgfpathlineto{\pgfqpoint{5.340398in}{0.739656in}}%
\pgfpathlineto{\pgfqpoint{5.340100in}{0.739656in}}%
\pgfpathlineto{\pgfqpoint{5.339803in}{0.739656in}}%
\pgfpathlineto{\pgfqpoint{5.339505in}{0.739656in}}%
\pgfpathlineto{\pgfqpoint{5.339208in}{0.739656in}}%
\pgfpathlineto{\pgfqpoint{5.338910in}{0.739656in}}%
\pgfpathlineto{\pgfqpoint{5.338613in}{0.739656in}}%
\pgfpathlineto{\pgfqpoint{5.338316in}{0.739656in}}%
\pgfpathlineto{\pgfqpoint{5.338018in}{0.739656in}}%
\pgfpathlineto{\pgfqpoint{5.337721in}{0.739656in}}%
\pgfpathlineto{\pgfqpoint{5.337423in}{0.739656in}}%
\pgfpathlineto{\pgfqpoint{5.337126in}{0.739656in}}%
\pgfpathlineto{\pgfqpoint{5.336828in}{0.739656in}}%
\pgfpathlineto{\pgfqpoint{5.336531in}{0.739656in}}%
\pgfpathlineto{\pgfqpoint{5.336233in}{0.739656in}}%
\pgfpathlineto{\pgfqpoint{5.335936in}{0.739656in}}%
\pgfpathlineto{\pgfqpoint{5.335638in}{0.739656in}}%
\pgfpathlineto{\pgfqpoint{5.335341in}{0.739656in}}%
\pgfpathlineto{\pgfqpoint{5.335043in}{0.739656in}}%
\pgfpathlineto{\pgfqpoint{5.334746in}{0.739656in}}%
\pgfpathlineto{\pgfqpoint{5.334448in}{0.739656in}}%
\pgfpathlineto{\pgfqpoint{5.334151in}{0.739656in}}%
\pgfpathlineto{\pgfqpoint{5.333853in}{0.739656in}}%
\pgfpathlineto{\pgfqpoint{5.333556in}{0.739656in}}%
\pgfpathlineto{\pgfqpoint{5.333258in}{0.739656in}}%
\pgfpathlineto{\pgfqpoint{5.332961in}{0.739656in}}%
\pgfpathlineto{\pgfqpoint{5.332663in}{0.739656in}}%
\pgfpathlineto{\pgfqpoint{5.332366in}{0.739656in}}%
\pgfpathlineto{\pgfqpoint{5.332069in}{0.739656in}}%
\pgfpathlineto{\pgfqpoint{5.331771in}{0.739656in}}%
\pgfpathlineto{\pgfqpoint{5.331474in}{0.739656in}}%
\pgfpathlineto{\pgfqpoint{5.331176in}{0.739656in}}%
\pgfpathlineto{\pgfqpoint{5.330879in}{0.739656in}}%
\pgfpathlineto{\pgfqpoint{5.330581in}{0.739656in}}%
\pgfpathlineto{\pgfqpoint{5.330284in}{0.739656in}}%
\pgfpathlineto{\pgfqpoint{5.329986in}{0.739656in}}%
\pgfpathlineto{\pgfqpoint{5.329689in}{0.739656in}}%
\pgfpathlineto{\pgfqpoint{5.329391in}{0.739656in}}%
\pgfpathlineto{\pgfqpoint{5.329094in}{0.739656in}}%
\pgfpathlineto{\pgfqpoint{5.328796in}{0.739656in}}%
\pgfpathlineto{\pgfqpoint{5.328499in}{0.739656in}}%
\pgfpathlineto{\pgfqpoint{5.328201in}{0.739656in}}%
\pgfpathlineto{\pgfqpoint{5.327904in}{0.739656in}}%
\pgfpathlineto{\pgfqpoint{5.327606in}{0.739656in}}%
\pgfpathlineto{\pgfqpoint{5.327309in}{0.739656in}}%
\pgfpathlineto{\pgfqpoint{5.327011in}{0.739656in}}%
\pgfpathlineto{\pgfqpoint{5.326714in}{0.739656in}}%
\pgfpathlineto{\pgfqpoint{5.326416in}{0.739656in}}%
\pgfpathlineto{\pgfqpoint{5.326119in}{0.739656in}}%
\pgfpathlineto{\pgfqpoint{5.325821in}{0.739656in}}%
\pgfpathlineto{\pgfqpoint{5.325524in}{0.739656in}}%
\pgfpathlineto{\pgfqpoint{5.325227in}{0.739656in}}%
\pgfpathlineto{\pgfqpoint{5.324929in}{0.739656in}}%
\pgfpathlineto{\pgfqpoint{5.324632in}{0.739656in}}%
\pgfpathlineto{\pgfqpoint{5.324334in}{0.739656in}}%
\pgfpathlineto{\pgfqpoint{5.324037in}{0.739656in}}%
\pgfpathlineto{\pgfqpoint{5.323739in}{0.739656in}}%
\pgfpathlineto{\pgfqpoint{5.323442in}{0.739656in}}%
\pgfpathlineto{\pgfqpoint{5.323144in}{0.739656in}}%
\pgfpathlineto{\pgfqpoint{5.322847in}{0.739656in}}%
\pgfpathlineto{\pgfqpoint{5.322549in}{0.739656in}}%
\pgfpathlineto{\pgfqpoint{5.322252in}{0.739656in}}%
\pgfpathlineto{\pgfqpoint{5.321954in}{0.739656in}}%
\pgfpathlineto{\pgfqpoint{5.321657in}{0.739656in}}%
\pgfpathlineto{\pgfqpoint{5.321359in}{0.739656in}}%
\pgfpathlineto{\pgfqpoint{5.321062in}{0.739656in}}%
\pgfpathlineto{\pgfqpoint{5.320764in}{0.739656in}}%
\pgfpathlineto{\pgfqpoint{5.320467in}{0.739656in}}%
\pgfpathlineto{\pgfqpoint{5.320169in}{0.739656in}}%
\pgfpathlineto{\pgfqpoint{5.319872in}{0.739656in}}%
\pgfpathlineto{\pgfqpoint{5.319574in}{0.739656in}}%
\pgfpathlineto{\pgfqpoint{5.319277in}{0.739656in}}%
\pgfpathlineto{\pgfqpoint{5.318979in}{0.739656in}}%
\pgfpathlineto{\pgfqpoint{5.318682in}{0.739656in}}%
\pgfpathlineto{\pgfqpoint{5.318385in}{0.739656in}}%
\pgfpathlineto{\pgfqpoint{5.318087in}{0.739656in}}%
\pgfpathlineto{\pgfqpoint{5.317790in}{0.739656in}}%
\pgfpathlineto{\pgfqpoint{5.317492in}{0.739656in}}%
\pgfpathlineto{\pgfqpoint{5.317195in}{0.739656in}}%
\pgfpathlineto{\pgfqpoint{5.316897in}{0.739656in}}%
\pgfpathlineto{\pgfqpoint{5.316600in}{0.739656in}}%
\pgfpathlineto{\pgfqpoint{5.316302in}{0.739656in}}%
\pgfpathlineto{\pgfqpoint{5.316005in}{0.739656in}}%
\pgfpathlineto{\pgfqpoint{5.315707in}{0.739656in}}%
\pgfpathlineto{\pgfqpoint{5.315410in}{0.739656in}}%
\pgfpathlineto{\pgfqpoint{5.315112in}{0.739656in}}%
\pgfpathlineto{\pgfqpoint{5.314815in}{0.739656in}}%
\pgfpathlineto{\pgfqpoint{5.314517in}{0.739656in}}%
\pgfpathlineto{\pgfqpoint{5.314220in}{0.739656in}}%
\pgfpathlineto{\pgfqpoint{5.313922in}{0.739656in}}%
\pgfpathlineto{\pgfqpoint{5.313625in}{0.739656in}}%
\pgfpathlineto{\pgfqpoint{5.313327in}{0.739656in}}%
\pgfpathlineto{\pgfqpoint{5.313030in}{0.739656in}}%
\pgfpathlineto{\pgfqpoint{5.312732in}{0.739656in}}%
\pgfpathlineto{\pgfqpoint{5.312435in}{0.739656in}}%
\pgfpathlineto{\pgfqpoint{5.312138in}{0.739656in}}%
\pgfpathlineto{\pgfqpoint{5.311840in}{0.739656in}}%
\pgfpathlineto{\pgfqpoint{5.311543in}{0.739656in}}%
\pgfpathlineto{\pgfqpoint{5.311245in}{0.739656in}}%
\pgfpathlineto{\pgfqpoint{5.310948in}{0.739656in}}%
\pgfpathlineto{\pgfqpoint{5.310650in}{0.739656in}}%
\pgfpathlineto{\pgfqpoint{5.310353in}{0.739656in}}%
\pgfpathlineto{\pgfqpoint{5.310055in}{0.739656in}}%
\pgfpathlineto{\pgfqpoint{5.309758in}{0.739656in}}%
\pgfpathlineto{\pgfqpoint{5.309460in}{0.739656in}}%
\pgfpathlineto{\pgfqpoint{5.309163in}{0.739656in}}%
\pgfpathlineto{\pgfqpoint{5.308865in}{0.739656in}}%
\pgfpathlineto{\pgfqpoint{5.308568in}{0.739656in}}%
\pgfpathlineto{\pgfqpoint{5.308270in}{0.739656in}}%
\pgfpathlineto{\pgfqpoint{5.307973in}{0.739656in}}%
\pgfpathlineto{\pgfqpoint{5.307675in}{0.739656in}}%
\pgfpathlineto{\pgfqpoint{5.307378in}{0.739656in}}%
\pgfpathlineto{\pgfqpoint{5.307080in}{0.739656in}}%
\pgfpathlineto{\pgfqpoint{5.306783in}{0.739656in}}%
\pgfpathlineto{\pgfqpoint{5.306485in}{0.739656in}}%
\pgfpathlineto{\pgfqpoint{5.306188in}{0.739656in}}%
\pgfpathlineto{\pgfqpoint{5.305890in}{0.739656in}}%
\pgfpathlineto{\pgfqpoint{5.305593in}{0.739656in}}%
\pgfpathlineto{\pgfqpoint{5.305296in}{0.739656in}}%
\pgfpathlineto{\pgfqpoint{5.304998in}{0.739656in}}%
\pgfpathlineto{\pgfqpoint{5.304701in}{0.739656in}}%
\pgfpathlineto{\pgfqpoint{5.304403in}{0.739656in}}%
\pgfpathlineto{\pgfqpoint{5.304106in}{0.739656in}}%
\pgfpathlineto{\pgfqpoint{5.303808in}{0.739656in}}%
\pgfpathlineto{\pgfqpoint{5.303511in}{0.739656in}}%
\pgfpathlineto{\pgfqpoint{5.303213in}{0.739656in}}%
\pgfpathlineto{\pgfqpoint{5.302916in}{0.739656in}}%
\pgfpathlineto{\pgfqpoint{5.302618in}{0.739656in}}%
\pgfpathlineto{\pgfqpoint{5.302321in}{0.739656in}}%
\pgfpathlineto{\pgfqpoint{5.302023in}{0.739656in}}%
\pgfpathlineto{\pgfqpoint{5.301726in}{0.739656in}}%
\pgfpathlineto{\pgfqpoint{5.301428in}{0.739656in}}%
\pgfpathlineto{\pgfqpoint{5.301131in}{0.739656in}}%
\pgfpathlineto{\pgfqpoint{5.300833in}{0.739656in}}%
\pgfpathlineto{\pgfqpoint{5.300536in}{0.739656in}}%
\pgfpathlineto{\pgfqpoint{5.300238in}{0.739656in}}%
\pgfpathlineto{\pgfqpoint{5.299941in}{0.739656in}}%
\pgfpathlineto{\pgfqpoint{5.299643in}{0.739656in}}%
\pgfpathlineto{\pgfqpoint{5.299346in}{0.739656in}}%
\pgfpathlineto{\pgfqpoint{5.299048in}{0.739656in}}%
\pgfpathlineto{\pgfqpoint{5.298751in}{0.739656in}}%
\pgfpathlineto{\pgfqpoint{5.298454in}{0.739656in}}%
\pgfpathlineto{\pgfqpoint{5.298156in}{0.739656in}}%
\pgfpathlineto{\pgfqpoint{5.297859in}{0.739656in}}%
\pgfpathlineto{\pgfqpoint{5.297561in}{0.739656in}}%
\pgfpathlineto{\pgfqpoint{5.297264in}{0.739656in}}%
\pgfpathlineto{\pgfqpoint{5.296966in}{0.739656in}}%
\pgfpathlineto{\pgfqpoint{5.296669in}{0.739656in}}%
\pgfpathlineto{\pgfqpoint{5.296371in}{0.739656in}}%
\pgfpathlineto{\pgfqpoint{5.296074in}{0.739656in}}%
\pgfpathlineto{\pgfqpoint{5.295776in}{0.739656in}}%
\pgfpathlineto{\pgfqpoint{5.295479in}{0.739656in}}%
\pgfpathlineto{\pgfqpoint{5.295181in}{0.739656in}}%
\pgfpathlineto{\pgfqpoint{5.294884in}{0.739656in}}%
\pgfpathlineto{\pgfqpoint{5.294586in}{0.739656in}}%
\pgfpathlineto{\pgfqpoint{5.294289in}{0.739656in}}%
\pgfpathlineto{\pgfqpoint{5.293991in}{0.739656in}}%
\pgfpathlineto{\pgfqpoint{5.293694in}{0.739656in}}%
\pgfpathlineto{\pgfqpoint{5.293396in}{0.739656in}}%
\pgfpathlineto{\pgfqpoint{5.293099in}{0.739656in}}%
\pgfpathlineto{\pgfqpoint{5.292801in}{0.739656in}}%
\pgfpathlineto{\pgfqpoint{5.292504in}{0.739656in}}%
\pgfpathlineto{\pgfqpoint{5.292207in}{0.739656in}}%
\pgfpathlineto{\pgfqpoint{5.291909in}{0.739656in}}%
\pgfpathlineto{\pgfqpoint{5.291612in}{0.739656in}}%
\pgfpathlineto{\pgfqpoint{5.291314in}{0.739656in}}%
\pgfpathlineto{\pgfqpoint{5.291017in}{0.739656in}}%
\pgfpathlineto{\pgfqpoint{5.290719in}{0.739656in}}%
\pgfpathlineto{\pgfqpoint{5.290422in}{0.739656in}}%
\pgfpathlineto{\pgfqpoint{5.290124in}{0.739656in}}%
\pgfpathlineto{\pgfqpoint{5.289827in}{0.739656in}}%
\pgfpathlineto{\pgfqpoint{5.289529in}{0.739656in}}%
\pgfpathlineto{\pgfqpoint{5.289232in}{0.739656in}}%
\pgfpathlineto{\pgfqpoint{5.288934in}{0.739656in}}%
\pgfpathlineto{\pgfqpoint{5.288637in}{0.739656in}}%
\pgfpathlineto{\pgfqpoint{5.288339in}{0.739656in}}%
\pgfpathlineto{\pgfqpoint{5.288042in}{0.739656in}}%
\pgfpathlineto{\pgfqpoint{5.287744in}{0.739656in}}%
\pgfpathlineto{\pgfqpoint{5.287447in}{0.739656in}}%
\pgfpathlineto{\pgfqpoint{5.287149in}{0.739656in}}%
\pgfpathlineto{\pgfqpoint{5.286852in}{0.739656in}}%
\pgfpathlineto{\pgfqpoint{5.286554in}{0.739656in}}%
\pgfpathlineto{\pgfqpoint{5.286257in}{0.739656in}}%
\pgfpathlineto{\pgfqpoint{5.285959in}{0.739656in}}%
\pgfpathlineto{\pgfqpoint{5.285662in}{0.739656in}}%
\pgfpathlineto{\pgfqpoint{5.285365in}{0.739656in}}%
\pgfpathlineto{\pgfqpoint{5.285067in}{0.739656in}}%
\pgfpathlineto{\pgfqpoint{5.284770in}{0.739656in}}%
\pgfpathlineto{\pgfqpoint{5.284472in}{0.739656in}}%
\pgfpathlineto{\pgfqpoint{5.284175in}{0.739656in}}%
\pgfpathlineto{\pgfqpoint{5.283877in}{0.739656in}}%
\pgfpathlineto{\pgfqpoint{5.283580in}{0.739656in}}%
\pgfpathlineto{\pgfqpoint{5.283282in}{0.739656in}}%
\pgfpathlineto{\pgfqpoint{5.282985in}{0.739656in}}%
\pgfpathlineto{\pgfqpoint{5.282687in}{0.739656in}}%
\pgfpathlineto{\pgfqpoint{5.282390in}{0.739656in}}%
\pgfpathlineto{\pgfqpoint{5.282092in}{0.739656in}}%
\pgfpathlineto{\pgfqpoint{5.281795in}{0.739656in}}%
\pgfpathlineto{\pgfqpoint{5.281497in}{0.739656in}}%
\pgfpathlineto{\pgfqpoint{5.281200in}{0.739656in}}%
\pgfpathlineto{\pgfqpoint{5.280902in}{0.739656in}}%
\pgfpathlineto{\pgfqpoint{5.280605in}{0.739656in}}%
\pgfpathlineto{\pgfqpoint{5.280307in}{0.739656in}}%
\pgfpathlineto{\pgfqpoint{5.280010in}{0.739656in}}%
\pgfpathlineto{\pgfqpoint{5.279712in}{0.739656in}}%
\pgfpathlineto{\pgfqpoint{5.279415in}{0.739656in}}%
\pgfpathlineto{\pgfqpoint{5.279117in}{0.739656in}}%
\pgfpathlineto{\pgfqpoint{5.278820in}{0.739656in}}%
\pgfpathlineto{\pgfqpoint{5.278523in}{0.739656in}}%
\pgfpathlineto{\pgfqpoint{5.278225in}{0.739656in}}%
\pgfpathlineto{\pgfqpoint{5.277928in}{0.739656in}}%
\pgfpathlineto{\pgfqpoint{5.277630in}{0.739656in}}%
\pgfpathlineto{\pgfqpoint{5.277333in}{0.739656in}}%
\pgfpathlineto{\pgfqpoint{5.277035in}{0.739656in}}%
\pgfpathlineto{\pgfqpoint{5.276738in}{0.739656in}}%
\pgfpathlineto{\pgfqpoint{5.276440in}{0.739656in}}%
\pgfpathlineto{\pgfqpoint{5.276143in}{0.739656in}}%
\pgfpathlineto{\pgfqpoint{5.275845in}{0.739656in}}%
\pgfpathlineto{\pgfqpoint{5.275548in}{0.739656in}}%
\pgfpathlineto{\pgfqpoint{5.275250in}{0.739656in}}%
\pgfpathlineto{\pgfqpoint{5.274953in}{0.739656in}}%
\pgfpathlineto{\pgfqpoint{5.274655in}{0.739656in}}%
\pgfpathlineto{\pgfqpoint{5.274358in}{0.739656in}}%
\pgfpathlineto{\pgfqpoint{5.274060in}{0.739656in}}%
\pgfpathlineto{\pgfqpoint{5.273763in}{0.739656in}}%
\pgfpathlineto{\pgfqpoint{5.273465in}{0.739656in}}%
\pgfpathlineto{\pgfqpoint{5.273168in}{0.739656in}}%
\pgfpathlineto{\pgfqpoint{5.272870in}{0.739656in}}%
\pgfpathlineto{\pgfqpoint{5.272573in}{0.739656in}}%
\pgfpathlineto{\pgfqpoint{5.272276in}{0.739656in}}%
\pgfpathlineto{\pgfqpoint{5.271978in}{0.739656in}}%
\pgfpathlineto{\pgfqpoint{5.271681in}{0.739656in}}%
\pgfpathlineto{\pgfqpoint{5.271383in}{0.739656in}}%
\pgfpathlineto{\pgfqpoint{5.271086in}{0.739656in}}%
\pgfpathlineto{\pgfqpoint{5.270788in}{0.739656in}}%
\pgfpathlineto{\pgfqpoint{5.270491in}{0.739656in}}%
\pgfpathlineto{\pgfqpoint{5.270193in}{0.739656in}}%
\pgfpathlineto{\pgfqpoint{5.269896in}{0.739656in}}%
\pgfpathlineto{\pgfqpoint{5.269598in}{0.739656in}}%
\pgfpathlineto{\pgfqpoint{5.269301in}{0.739656in}}%
\pgfpathlineto{\pgfqpoint{5.269003in}{0.739656in}}%
\pgfpathlineto{\pgfqpoint{5.268706in}{0.739656in}}%
\pgfpathlineto{\pgfqpoint{5.268408in}{0.739656in}}%
\pgfpathlineto{\pgfqpoint{5.268111in}{0.739656in}}%
\pgfpathlineto{\pgfqpoint{5.267813in}{0.739656in}}%
\pgfpathlineto{\pgfqpoint{5.267516in}{0.739656in}}%
\pgfpathlineto{\pgfqpoint{5.267218in}{0.739656in}}%
\pgfpathlineto{\pgfqpoint{5.266921in}{0.739656in}}%
\pgfpathlineto{\pgfqpoint{5.266623in}{0.739656in}}%
\pgfpathlineto{\pgfqpoint{5.266326in}{0.739656in}}%
\pgfpathlineto{\pgfqpoint{5.266028in}{0.739656in}}%
\pgfpathlineto{\pgfqpoint{5.265731in}{0.739656in}}%
\pgfpathlineto{\pgfqpoint{5.265434in}{0.739656in}}%
\pgfpathlineto{\pgfqpoint{5.265136in}{0.739656in}}%
\pgfpathlineto{\pgfqpoint{5.264839in}{0.739656in}}%
\pgfpathlineto{\pgfqpoint{5.264541in}{0.739656in}}%
\pgfpathlineto{\pgfqpoint{5.264244in}{0.739656in}}%
\pgfpathlineto{\pgfqpoint{5.263946in}{0.739656in}}%
\pgfpathlineto{\pgfqpoint{5.263649in}{0.739656in}}%
\pgfpathlineto{\pgfqpoint{5.263351in}{0.739656in}}%
\pgfpathlineto{\pgfqpoint{5.263054in}{0.739656in}}%
\pgfpathlineto{\pgfqpoint{5.262756in}{0.739656in}}%
\pgfpathlineto{\pgfqpoint{5.262459in}{0.739656in}}%
\pgfpathlineto{\pgfqpoint{5.262161in}{0.739656in}}%
\pgfpathlineto{\pgfqpoint{5.261864in}{0.739656in}}%
\pgfpathlineto{\pgfqpoint{5.261566in}{0.739656in}}%
\pgfpathlineto{\pgfqpoint{5.261269in}{0.739656in}}%
\pgfpathlineto{\pgfqpoint{5.260971in}{0.739656in}}%
\pgfpathlineto{\pgfqpoint{5.260674in}{0.739656in}}%
\pgfpathlineto{\pgfqpoint{5.260376in}{0.739656in}}%
\pgfpathlineto{\pgfqpoint{5.260079in}{0.739656in}}%
\pgfpathlineto{\pgfqpoint{5.259781in}{0.739656in}}%
\pgfpathlineto{\pgfqpoint{5.259484in}{0.739656in}}%
\pgfpathlineto{\pgfqpoint{5.259186in}{0.739656in}}%
\pgfpathlineto{\pgfqpoint{5.258889in}{0.739656in}}%
\pgfpathlineto{\pgfqpoint{5.258592in}{0.739656in}}%
\pgfpathlineto{\pgfqpoint{5.258294in}{0.739656in}}%
\pgfpathlineto{\pgfqpoint{5.257997in}{0.739656in}}%
\pgfpathlineto{\pgfqpoint{5.257699in}{0.739656in}}%
\pgfpathlineto{\pgfqpoint{5.257402in}{0.739656in}}%
\pgfpathlineto{\pgfqpoint{5.257104in}{0.739656in}}%
\pgfpathlineto{\pgfqpoint{5.256807in}{0.739656in}}%
\pgfpathlineto{\pgfqpoint{5.256509in}{0.739656in}}%
\pgfpathlineto{\pgfqpoint{5.256212in}{0.739656in}}%
\pgfpathlineto{\pgfqpoint{5.255914in}{0.739656in}}%
\pgfpathlineto{\pgfqpoint{5.255617in}{0.739656in}}%
\pgfpathlineto{\pgfqpoint{5.255319in}{0.739656in}}%
\pgfpathlineto{\pgfqpoint{5.255022in}{0.739656in}}%
\pgfpathlineto{\pgfqpoint{5.254724in}{0.739656in}}%
\pgfpathlineto{\pgfqpoint{5.254427in}{0.739656in}}%
\pgfpathlineto{\pgfqpoint{5.254129in}{0.739656in}}%
\pgfpathlineto{\pgfqpoint{5.253832in}{0.739656in}}%
\pgfpathlineto{\pgfqpoint{5.253534in}{0.739656in}}%
\pgfpathlineto{\pgfqpoint{5.253237in}{0.739656in}}%
\pgfpathlineto{\pgfqpoint{5.252939in}{0.739656in}}%
\pgfpathlineto{\pgfqpoint{5.252642in}{0.739656in}}%
\pgfpathlineto{\pgfqpoint{5.252344in}{0.739656in}}%
\pgfpathlineto{\pgfqpoint{5.252047in}{0.739656in}}%
\pgfpathlineto{\pgfqpoint{5.251750in}{0.739656in}}%
\pgfpathlineto{\pgfqpoint{5.251452in}{0.739656in}}%
\pgfpathlineto{\pgfqpoint{5.251155in}{0.739656in}}%
\pgfpathlineto{\pgfqpoint{5.250857in}{0.739656in}}%
\pgfpathlineto{\pgfqpoint{5.250560in}{0.739656in}}%
\pgfpathlineto{\pgfqpoint{5.250262in}{0.739656in}}%
\pgfpathlineto{\pgfqpoint{5.249965in}{0.739656in}}%
\pgfpathlineto{\pgfqpoint{5.249667in}{0.739656in}}%
\pgfpathlineto{\pgfqpoint{5.249370in}{0.739656in}}%
\pgfpathlineto{\pgfqpoint{5.249072in}{0.739656in}}%
\pgfpathlineto{\pgfqpoint{5.248775in}{0.739656in}}%
\pgfpathlineto{\pgfqpoint{5.248477in}{0.739656in}}%
\pgfpathlineto{\pgfqpoint{5.248180in}{0.739656in}}%
\pgfpathlineto{\pgfqpoint{5.247882in}{0.739656in}}%
\pgfpathlineto{\pgfqpoint{5.247585in}{0.739656in}}%
\pgfpathlineto{\pgfqpoint{5.247287in}{0.739656in}}%
\pgfpathlineto{\pgfqpoint{5.246990in}{0.739656in}}%
\pgfpathlineto{\pgfqpoint{5.246692in}{0.739656in}}%
\pgfpathlineto{\pgfqpoint{5.246395in}{0.739656in}}%
\pgfpathlineto{\pgfqpoint{5.246097in}{0.739656in}}%
\pgfpathlineto{\pgfqpoint{5.245800in}{0.739656in}}%
\pgfpathlineto{\pgfqpoint{5.245503in}{0.739656in}}%
\pgfpathlineto{\pgfqpoint{5.245205in}{0.739656in}}%
\pgfpathlineto{\pgfqpoint{5.244908in}{0.739656in}}%
\pgfpathlineto{\pgfqpoint{5.244610in}{0.739656in}}%
\pgfpathlineto{\pgfqpoint{5.244313in}{0.739656in}}%
\pgfpathlineto{\pgfqpoint{5.244015in}{0.739656in}}%
\pgfpathlineto{\pgfqpoint{5.243718in}{0.739656in}}%
\pgfpathlineto{\pgfqpoint{5.243420in}{0.739656in}}%
\pgfpathlineto{\pgfqpoint{5.243123in}{0.739656in}}%
\pgfpathlineto{\pgfqpoint{5.242825in}{0.739656in}}%
\pgfpathlineto{\pgfqpoint{5.242528in}{0.739656in}}%
\pgfpathlineto{\pgfqpoint{5.242230in}{0.739656in}}%
\pgfpathlineto{\pgfqpoint{5.241933in}{0.739656in}}%
\pgfpathlineto{\pgfqpoint{5.241635in}{0.739656in}}%
\pgfpathlineto{\pgfqpoint{5.241338in}{0.739656in}}%
\pgfpathlineto{\pgfqpoint{5.241040in}{0.739656in}}%
\pgfpathlineto{\pgfqpoint{5.240743in}{0.739656in}}%
\pgfpathlineto{\pgfqpoint{5.240445in}{0.739656in}}%
\pgfpathlineto{\pgfqpoint{5.240148in}{0.739656in}}%
\pgfpathlineto{\pgfqpoint{5.239850in}{0.739656in}}%
\pgfpathlineto{\pgfqpoint{5.239553in}{0.739656in}}%
\pgfpathlineto{\pgfqpoint{5.239255in}{0.739656in}}%
\pgfpathlineto{\pgfqpoint{5.238958in}{0.739656in}}%
\pgfpathlineto{\pgfqpoint{5.238661in}{0.739656in}}%
\pgfpathlineto{\pgfqpoint{5.238363in}{0.739656in}}%
\pgfpathlineto{\pgfqpoint{5.238066in}{0.739656in}}%
\pgfpathlineto{\pgfqpoint{5.237768in}{0.739656in}}%
\pgfpathlineto{\pgfqpoint{5.237471in}{0.739656in}}%
\pgfpathlineto{\pgfqpoint{5.237173in}{0.739656in}}%
\pgfpathlineto{\pgfqpoint{5.236876in}{0.739656in}}%
\pgfpathlineto{\pgfqpoint{5.236578in}{0.739656in}}%
\pgfpathlineto{\pgfqpoint{5.236281in}{0.739656in}}%
\pgfpathlineto{\pgfqpoint{5.235983in}{0.739656in}}%
\pgfpathlineto{\pgfqpoint{5.235686in}{0.739656in}}%
\pgfpathlineto{\pgfqpoint{5.235388in}{0.739656in}}%
\pgfpathlineto{\pgfqpoint{5.235091in}{0.739656in}}%
\pgfpathlineto{\pgfqpoint{5.234793in}{0.739656in}}%
\pgfpathlineto{\pgfqpoint{5.234496in}{0.739656in}}%
\pgfpathlineto{\pgfqpoint{5.234198in}{0.739656in}}%
\pgfpathlineto{\pgfqpoint{5.233901in}{0.739656in}}%
\pgfpathlineto{\pgfqpoint{5.233603in}{0.739656in}}%
\pgfpathlineto{\pgfqpoint{5.233306in}{0.739656in}}%
\pgfpathlineto{\pgfqpoint{5.233008in}{0.739656in}}%
\pgfpathlineto{\pgfqpoint{5.232711in}{0.739656in}}%
\pgfpathlineto{\pgfqpoint{5.232413in}{0.739656in}}%
\pgfpathlineto{\pgfqpoint{5.232116in}{0.739656in}}%
\pgfpathlineto{\pgfqpoint{5.231819in}{0.739656in}}%
\pgfpathlineto{\pgfqpoint{5.231521in}{0.739656in}}%
\pgfpathlineto{\pgfqpoint{5.231224in}{0.739656in}}%
\pgfpathlineto{\pgfqpoint{5.230926in}{0.739656in}}%
\pgfpathlineto{\pgfqpoint{5.230629in}{0.739656in}}%
\pgfpathlineto{\pgfqpoint{5.230331in}{0.739656in}}%
\pgfpathlineto{\pgfqpoint{5.230034in}{0.739656in}}%
\pgfpathlineto{\pgfqpoint{5.229736in}{0.739656in}}%
\pgfpathlineto{\pgfqpoint{5.229439in}{0.739656in}}%
\pgfpathlineto{\pgfqpoint{5.229141in}{0.739656in}}%
\pgfpathlineto{\pgfqpoint{5.228844in}{0.739656in}}%
\pgfpathlineto{\pgfqpoint{5.228546in}{0.739656in}}%
\pgfpathlineto{\pgfqpoint{5.228249in}{0.739656in}}%
\pgfpathlineto{\pgfqpoint{5.227951in}{0.739656in}}%
\pgfpathlineto{\pgfqpoint{5.227654in}{0.739656in}}%
\pgfpathlineto{\pgfqpoint{5.227356in}{0.739656in}}%
\pgfpathlineto{\pgfqpoint{5.227059in}{0.739656in}}%
\pgfpathlineto{\pgfqpoint{5.226761in}{0.739656in}}%
\pgfpathlineto{\pgfqpoint{5.226464in}{0.739656in}}%
\pgfpathlineto{\pgfqpoint{5.226166in}{0.739656in}}%
\pgfpathlineto{\pgfqpoint{5.225869in}{0.739656in}}%
\pgfpathlineto{\pgfqpoint{5.225572in}{0.739656in}}%
\pgfpathlineto{\pgfqpoint{5.225274in}{0.739656in}}%
\pgfpathlineto{\pgfqpoint{5.224977in}{0.739656in}}%
\pgfpathlineto{\pgfqpoint{5.224679in}{0.739656in}}%
\pgfpathlineto{\pgfqpoint{5.224382in}{0.739656in}}%
\pgfpathlineto{\pgfqpoint{5.224084in}{0.739656in}}%
\pgfpathlineto{\pgfqpoint{5.223787in}{0.739656in}}%
\pgfpathlineto{\pgfqpoint{5.223489in}{0.739656in}}%
\pgfpathlineto{\pgfqpoint{5.223192in}{0.739656in}}%
\pgfpathlineto{\pgfqpoint{5.222894in}{0.739656in}}%
\pgfpathlineto{\pgfqpoint{5.222597in}{0.739656in}}%
\pgfpathlineto{\pgfqpoint{5.222299in}{0.739656in}}%
\pgfpathlineto{\pgfqpoint{5.222002in}{0.739656in}}%
\pgfpathlineto{\pgfqpoint{5.221704in}{0.739656in}}%
\pgfpathlineto{\pgfqpoint{5.221407in}{0.739656in}}%
\pgfpathlineto{\pgfqpoint{5.221109in}{0.739656in}}%
\pgfpathlineto{\pgfqpoint{5.220812in}{0.739656in}}%
\pgfpathlineto{\pgfqpoint{5.220514in}{0.739656in}}%
\pgfpathlineto{\pgfqpoint{5.220217in}{0.739656in}}%
\pgfpathlineto{\pgfqpoint{5.219919in}{0.739656in}}%
\pgfpathlineto{\pgfqpoint{5.219622in}{0.739656in}}%
\pgfpathlineto{\pgfqpoint{5.219324in}{0.739656in}}%
\pgfpathlineto{\pgfqpoint{5.219027in}{0.739656in}}%
\pgfpathlineto{\pgfqpoint{5.218730in}{0.739656in}}%
\pgfpathlineto{\pgfqpoint{5.218432in}{0.739656in}}%
\pgfpathlineto{\pgfqpoint{5.218135in}{0.739656in}}%
\pgfpathlineto{\pgfqpoint{5.217837in}{0.739656in}}%
\pgfpathlineto{\pgfqpoint{5.217540in}{0.739656in}}%
\pgfpathlineto{\pgfqpoint{5.217242in}{0.739656in}}%
\pgfpathlineto{\pgfqpoint{5.216945in}{0.739656in}}%
\pgfpathlineto{\pgfqpoint{5.216647in}{0.739656in}}%
\pgfpathlineto{\pgfqpoint{5.216350in}{0.739656in}}%
\pgfpathlineto{\pgfqpoint{5.216052in}{0.739656in}}%
\pgfpathlineto{\pgfqpoint{5.215755in}{0.739656in}}%
\pgfpathlineto{\pgfqpoint{5.215457in}{0.739656in}}%
\pgfpathlineto{\pgfqpoint{5.215160in}{0.739656in}}%
\pgfpathlineto{\pgfqpoint{5.214862in}{0.739656in}}%
\pgfpathlineto{\pgfqpoint{5.214565in}{0.739656in}}%
\pgfpathlineto{\pgfqpoint{5.214267in}{0.739656in}}%
\pgfpathlineto{\pgfqpoint{5.213970in}{0.739656in}}%
\pgfpathlineto{\pgfqpoint{5.213672in}{0.739656in}}%
\pgfpathlineto{\pgfqpoint{5.213375in}{0.739656in}}%
\pgfpathlineto{\pgfqpoint{5.213077in}{0.739656in}}%
\pgfpathlineto{\pgfqpoint{5.212780in}{0.739656in}}%
\pgfpathlineto{\pgfqpoint{5.212482in}{0.739656in}}%
\pgfpathlineto{\pgfqpoint{5.212185in}{0.739656in}}%
\pgfpathlineto{\pgfqpoint{5.211888in}{0.739656in}}%
\pgfpathlineto{\pgfqpoint{5.211590in}{0.739656in}}%
\pgfpathlineto{\pgfqpoint{5.211293in}{0.739656in}}%
\pgfpathlineto{\pgfqpoint{5.210995in}{0.739656in}}%
\pgfpathlineto{\pgfqpoint{5.210698in}{0.739656in}}%
\pgfpathlineto{\pgfqpoint{5.210400in}{0.739656in}}%
\pgfpathlineto{\pgfqpoint{5.210103in}{0.739656in}}%
\pgfpathlineto{\pgfqpoint{5.209805in}{0.739656in}}%
\pgfpathlineto{\pgfqpoint{5.209508in}{0.739656in}}%
\pgfpathlineto{\pgfqpoint{5.209210in}{0.739656in}}%
\pgfpathlineto{\pgfqpoint{5.208913in}{0.739656in}}%
\pgfpathlineto{\pgfqpoint{5.208615in}{0.739656in}}%
\pgfpathlineto{\pgfqpoint{5.208318in}{0.739656in}}%
\pgfpathlineto{\pgfqpoint{5.208020in}{0.739656in}}%
\pgfpathlineto{\pgfqpoint{5.207723in}{0.739656in}}%
\pgfpathlineto{\pgfqpoint{5.207425in}{0.739656in}}%
\pgfpathlineto{\pgfqpoint{5.207128in}{0.739656in}}%
\pgfpathlineto{\pgfqpoint{5.206830in}{0.739656in}}%
\pgfpathlineto{\pgfqpoint{5.206533in}{0.739656in}}%
\pgfpathlineto{\pgfqpoint{5.206235in}{0.739656in}}%
\pgfpathlineto{\pgfqpoint{5.205938in}{0.739656in}}%
\pgfpathlineto{\pgfqpoint{5.205641in}{0.739656in}}%
\pgfpathlineto{\pgfqpoint{5.205343in}{0.739656in}}%
\pgfpathlineto{\pgfqpoint{5.205046in}{0.739656in}}%
\pgfpathlineto{\pgfqpoint{5.204748in}{0.739656in}}%
\pgfpathlineto{\pgfqpoint{5.204451in}{0.739656in}}%
\pgfpathlineto{\pgfqpoint{5.204153in}{0.739656in}}%
\pgfpathlineto{\pgfqpoint{5.203856in}{0.739656in}}%
\pgfpathlineto{\pgfqpoint{5.203558in}{0.739656in}}%
\pgfpathlineto{\pgfqpoint{5.203261in}{0.739656in}}%
\pgfpathlineto{\pgfqpoint{5.202963in}{0.739656in}}%
\pgfpathlineto{\pgfqpoint{5.202666in}{0.739656in}}%
\pgfpathlineto{\pgfqpoint{5.202368in}{0.739656in}}%
\pgfpathlineto{\pgfqpoint{5.202071in}{0.739656in}}%
\pgfpathlineto{\pgfqpoint{5.201773in}{0.739656in}}%
\pgfpathlineto{\pgfqpoint{5.201476in}{0.739656in}}%
\pgfpathlineto{\pgfqpoint{5.201178in}{0.739656in}}%
\pgfpathlineto{\pgfqpoint{5.200881in}{0.739656in}}%
\pgfpathlineto{\pgfqpoint{5.200583in}{0.739656in}}%
\pgfpathlineto{\pgfqpoint{5.200286in}{0.739656in}}%
\pgfpathlineto{\pgfqpoint{5.199988in}{0.739656in}}%
\pgfpathlineto{\pgfqpoint{5.199691in}{0.739656in}}%
\pgfpathlineto{\pgfqpoint{5.199393in}{0.739656in}}%
\pgfpathlineto{\pgfqpoint{5.199096in}{0.739656in}}%
\pgfpathlineto{\pgfqpoint{5.198799in}{0.739656in}}%
\pgfpathlineto{\pgfqpoint{5.198501in}{0.739656in}}%
\pgfpathlineto{\pgfqpoint{5.198204in}{0.739656in}}%
\pgfpathlineto{\pgfqpoint{5.197906in}{0.739656in}}%
\pgfpathlineto{\pgfqpoint{5.197609in}{0.739656in}}%
\pgfpathlineto{\pgfqpoint{5.197311in}{0.739656in}}%
\pgfpathlineto{\pgfqpoint{5.197014in}{0.739656in}}%
\pgfpathlineto{\pgfqpoint{5.196716in}{0.739656in}}%
\pgfpathlineto{\pgfqpoint{5.196419in}{0.739656in}}%
\pgfpathlineto{\pgfqpoint{5.196121in}{0.739656in}}%
\pgfpathlineto{\pgfqpoint{5.195824in}{0.739656in}}%
\pgfpathlineto{\pgfqpoint{5.195526in}{0.739656in}}%
\pgfpathlineto{\pgfqpoint{5.195229in}{0.739656in}}%
\pgfpathlineto{\pgfqpoint{5.194931in}{0.739656in}}%
\pgfpathlineto{\pgfqpoint{5.194634in}{0.739656in}}%
\pgfpathlineto{\pgfqpoint{5.194336in}{0.739656in}}%
\pgfpathlineto{\pgfqpoint{5.194039in}{0.739656in}}%
\pgfpathlineto{\pgfqpoint{5.193741in}{0.739656in}}%
\pgfpathlineto{\pgfqpoint{5.193444in}{0.739656in}}%
\pgfpathlineto{\pgfqpoint{5.193146in}{0.739656in}}%
\pgfpathlineto{\pgfqpoint{5.192849in}{0.739656in}}%
\pgfpathlineto{\pgfqpoint{5.192551in}{0.739656in}}%
\pgfpathlineto{\pgfqpoint{5.192254in}{0.739656in}}%
\pgfpathlineto{\pgfqpoint{5.191957in}{0.739656in}}%
\pgfpathlineto{\pgfqpoint{5.191659in}{0.739656in}}%
\pgfpathlineto{\pgfqpoint{5.191362in}{0.739656in}}%
\pgfpathlineto{\pgfqpoint{5.191064in}{0.739656in}}%
\pgfpathlineto{\pgfqpoint{5.190767in}{0.739656in}}%
\pgfpathlineto{\pgfqpoint{5.190469in}{0.739656in}}%
\pgfpathlineto{\pgfqpoint{5.190172in}{0.739656in}}%
\pgfpathlineto{\pgfqpoint{5.189874in}{0.739656in}}%
\pgfpathlineto{\pgfqpoint{5.189577in}{0.739656in}}%
\pgfpathlineto{\pgfqpoint{5.189279in}{0.739656in}}%
\pgfpathlineto{\pgfqpoint{5.188982in}{0.739656in}}%
\pgfpathlineto{\pgfqpoint{5.188684in}{0.739656in}}%
\pgfpathlineto{\pgfqpoint{5.188387in}{0.739656in}}%
\pgfpathlineto{\pgfqpoint{5.188089in}{0.739656in}}%
\pgfpathlineto{\pgfqpoint{5.187792in}{0.739656in}}%
\pgfpathlineto{\pgfqpoint{5.187494in}{0.739656in}}%
\pgfpathlineto{\pgfqpoint{5.187197in}{0.739656in}}%
\pgfpathlineto{\pgfqpoint{5.186899in}{0.739656in}}%
\pgfpathlineto{\pgfqpoint{5.186602in}{0.739656in}}%
\pgfpathlineto{\pgfqpoint{5.186304in}{0.739656in}}%
\pgfpathlineto{\pgfqpoint{5.186007in}{0.739656in}}%
\pgfpathlineto{\pgfqpoint{5.185710in}{0.739656in}}%
\pgfpathlineto{\pgfqpoint{5.185412in}{0.739656in}}%
\pgfpathlineto{\pgfqpoint{5.185115in}{0.739656in}}%
\pgfpathlineto{\pgfqpoint{5.184817in}{0.739656in}}%
\pgfpathlineto{\pgfqpoint{5.184520in}{0.739656in}}%
\pgfpathlineto{\pgfqpoint{5.184222in}{0.739656in}}%
\pgfpathlineto{\pgfqpoint{5.183925in}{0.739656in}}%
\pgfpathlineto{\pgfqpoint{5.183627in}{0.739656in}}%
\pgfpathlineto{\pgfqpoint{5.183330in}{0.739656in}}%
\pgfpathlineto{\pgfqpoint{5.183032in}{0.739656in}}%
\pgfpathlineto{\pgfqpoint{5.182735in}{0.739656in}}%
\pgfpathlineto{\pgfqpoint{5.182437in}{0.739656in}}%
\pgfpathlineto{\pgfqpoint{5.182140in}{0.739656in}}%
\pgfpathlineto{\pgfqpoint{5.181842in}{0.739656in}}%
\pgfpathlineto{\pgfqpoint{5.181545in}{0.739656in}}%
\pgfpathlineto{\pgfqpoint{5.181247in}{0.739656in}}%
\pgfpathlineto{\pgfqpoint{5.180950in}{0.739656in}}%
\pgfpathlineto{\pgfqpoint{5.180652in}{0.739656in}}%
\pgfpathlineto{\pgfqpoint{5.180355in}{0.739656in}}%
\pgfpathlineto{\pgfqpoint{5.180057in}{0.739656in}}%
\pgfpathlineto{\pgfqpoint{5.179760in}{0.739656in}}%
\pgfpathlineto{\pgfqpoint{5.179462in}{0.739656in}}%
\pgfpathlineto{\pgfqpoint{5.179165in}{0.739656in}}%
\pgfpathlineto{\pgfqpoint{5.178868in}{0.739656in}}%
\pgfpathlineto{\pgfqpoint{5.178570in}{0.739656in}}%
\pgfpathlineto{\pgfqpoint{5.178273in}{0.739656in}}%
\pgfpathlineto{\pgfqpoint{5.177975in}{0.739656in}}%
\pgfpathlineto{\pgfqpoint{5.177678in}{0.739656in}}%
\pgfpathlineto{\pgfqpoint{5.177380in}{0.739656in}}%
\pgfpathlineto{\pgfqpoint{5.177083in}{0.739656in}}%
\pgfpathlineto{\pgfqpoint{5.176785in}{0.739656in}}%
\pgfpathlineto{\pgfqpoint{5.176488in}{0.739656in}}%
\pgfpathlineto{\pgfqpoint{5.176190in}{0.739656in}}%
\pgfpathlineto{\pgfqpoint{5.175893in}{0.739656in}}%
\pgfpathlineto{\pgfqpoint{5.175595in}{0.739656in}}%
\pgfpathlineto{\pgfqpoint{5.175298in}{0.739656in}}%
\pgfpathlineto{\pgfqpoint{5.175000in}{0.739656in}}%
\pgfpathlineto{\pgfqpoint{5.174703in}{0.739656in}}%
\pgfpathlineto{\pgfqpoint{5.174405in}{0.739656in}}%
\pgfpathlineto{\pgfqpoint{5.174108in}{0.739656in}}%
\pgfpathlineto{\pgfqpoint{5.173810in}{0.739656in}}%
\pgfpathlineto{\pgfqpoint{5.173513in}{0.739656in}}%
\pgfpathlineto{\pgfqpoint{5.173215in}{0.739656in}}%
\pgfpathlineto{\pgfqpoint{5.172918in}{0.739656in}}%
\pgfpathlineto{\pgfqpoint{5.172620in}{0.739656in}}%
\pgfpathlineto{\pgfqpoint{5.172323in}{0.739656in}}%
\pgfpathlineto{\pgfqpoint{5.172026in}{0.739656in}}%
\pgfpathlineto{\pgfqpoint{5.171728in}{0.739656in}}%
\pgfpathlineto{\pgfqpoint{5.171431in}{0.739656in}}%
\pgfpathlineto{\pgfqpoint{5.171133in}{0.739656in}}%
\pgfpathlineto{\pgfqpoint{5.170836in}{0.739656in}}%
\pgfpathlineto{\pgfqpoint{5.170538in}{0.739656in}}%
\pgfpathlineto{\pgfqpoint{5.170241in}{0.739656in}}%
\pgfpathlineto{\pgfqpoint{5.169943in}{0.739656in}}%
\pgfpathlineto{\pgfqpoint{5.169646in}{0.739656in}}%
\pgfpathlineto{\pgfqpoint{5.169348in}{0.739656in}}%
\pgfpathlineto{\pgfqpoint{5.169051in}{0.739656in}}%
\pgfpathlineto{\pgfqpoint{5.168753in}{0.739656in}}%
\pgfpathlineto{\pgfqpoint{5.168456in}{0.739656in}}%
\pgfpathlineto{\pgfqpoint{5.168158in}{0.739656in}}%
\pgfpathlineto{\pgfqpoint{5.167861in}{0.739656in}}%
\pgfpathlineto{\pgfqpoint{5.167563in}{0.739656in}}%
\pgfpathlineto{\pgfqpoint{5.167266in}{0.739656in}}%
\pgfpathlineto{\pgfqpoint{5.166968in}{0.739656in}}%
\pgfpathlineto{\pgfqpoint{5.166671in}{0.739656in}}%
\pgfpathlineto{\pgfqpoint{5.166373in}{0.739656in}}%
\pgfpathlineto{\pgfqpoint{5.166076in}{0.739656in}}%
\pgfpathlineto{\pgfqpoint{5.165779in}{0.739656in}}%
\pgfpathlineto{\pgfqpoint{5.165481in}{0.739656in}}%
\pgfpathlineto{\pgfqpoint{5.165184in}{0.739656in}}%
\pgfpathlineto{\pgfqpoint{5.164886in}{0.739656in}}%
\pgfpathlineto{\pgfqpoint{5.164589in}{0.739656in}}%
\pgfpathlineto{\pgfqpoint{5.164291in}{0.739656in}}%
\pgfpathlineto{\pgfqpoint{5.163994in}{0.739656in}}%
\pgfpathlineto{\pgfqpoint{5.163696in}{0.739656in}}%
\pgfpathlineto{\pgfqpoint{5.163399in}{0.739656in}}%
\pgfpathlineto{\pgfqpoint{5.163101in}{0.739656in}}%
\pgfpathlineto{\pgfqpoint{5.162804in}{0.739656in}}%
\pgfpathlineto{\pgfqpoint{5.162506in}{0.739656in}}%
\pgfpathlineto{\pgfqpoint{5.162209in}{0.739656in}}%
\pgfpathlineto{\pgfqpoint{5.161911in}{0.739656in}}%
\pgfpathlineto{\pgfqpoint{5.161614in}{0.739656in}}%
\pgfpathlineto{\pgfqpoint{5.161316in}{0.739656in}}%
\pgfpathlineto{\pgfqpoint{5.161019in}{0.739656in}}%
\pgfpathlineto{\pgfqpoint{5.160721in}{0.739656in}}%
\pgfpathlineto{\pgfqpoint{5.160424in}{0.739656in}}%
\pgfpathlineto{\pgfqpoint{5.160126in}{0.739656in}}%
\pgfpathlineto{\pgfqpoint{5.159829in}{0.739656in}}%
\pgfpathlineto{\pgfqpoint{5.159531in}{0.739656in}}%
\pgfpathlineto{\pgfqpoint{5.159234in}{0.739656in}}%
\pgfpathlineto{\pgfqpoint{5.158937in}{0.739656in}}%
\pgfpathlineto{\pgfqpoint{5.158639in}{0.739656in}}%
\pgfpathlineto{\pgfqpoint{5.158342in}{0.739656in}}%
\pgfpathlineto{\pgfqpoint{5.158044in}{0.739656in}}%
\pgfpathlineto{\pgfqpoint{5.157747in}{0.739656in}}%
\pgfpathlineto{\pgfqpoint{5.157449in}{0.739656in}}%
\pgfpathlineto{\pgfqpoint{5.157152in}{0.739656in}}%
\pgfpathlineto{\pgfqpoint{5.156854in}{0.739656in}}%
\pgfpathlineto{\pgfqpoint{5.156557in}{0.739656in}}%
\pgfpathlineto{\pgfqpoint{5.156259in}{0.739656in}}%
\pgfpathlineto{\pgfqpoint{5.155962in}{0.739656in}}%
\pgfpathlineto{\pgfqpoint{5.155664in}{0.739656in}}%
\pgfpathlineto{\pgfqpoint{5.155367in}{0.739656in}}%
\pgfpathlineto{\pgfqpoint{5.155069in}{0.739656in}}%
\pgfpathlineto{\pgfqpoint{5.154772in}{0.739656in}}%
\pgfpathlineto{\pgfqpoint{5.154474in}{0.739656in}}%
\pgfpathlineto{\pgfqpoint{5.154177in}{0.739656in}}%
\pgfpathlineto{\pgfqpoint{5.153879in}{0.739656in}}%
\pgfpathlineto{\pgfqpoint{5.153582in}{0.739656in}}%
\pgfpathlineto{\pgfqpoint{5.153284in}{0.739656in}}%
\pgfpathlineto{\pgfqpoint{5.152987in}{0.739656in}}%
\pgfpathlineto{\pgfqpoint{5.152689in}{0.739656in}}%
\pgfpathlineto{\pgfqpoint{5.152392in}{0.739656in}}%
\pgfpathlineto{\pgfqpoint{5.152095in}{0.739656in}}%
\pgfpathlineto{\pgfqpoint{5.151797in}{0.739656in}}%
\pgfpathlineto{\pgfqpoint{5.151500in}{0.739656in}}%
\pgfpathlineto{\pgfqpoint{5.151202in}{0.739656in}}%
\pgfpathlineto{\pgfqpoint{5.150905in}{0.739656in}}%
\pgfpathlineto{\pgfqpoint{5.150607in}{0.739656in}}%
\pgfpathlineto{\pgfqpoint{5.150310in}{0.739656in}}%
\pgfpathlineto{\pgfqpoint{5.150012in}{0.739656in}}%
\pgfpathlineto{\pgfqpoint{5.149715in}{0.739656in}}%
\pgfpathlineto{\pgfqpoint{5.149417in}{0.739656in}}%
\pgfpathlineto{\pgfqpoint{5.149120in}{0.739656in}}%
\pgfpathlineto{\pgfqpoint{5.148822in}{0.739656in}}%
\pgfpathlineto{\pgfqpoint{5.148525in}{0.739656in}}%
\pgfpathlineto{\pgfqpoint{5.148227in}{0.739656in}}%
\pgfpathlineto{\pgfqpoint{5.147930in}{0.739656in}}%
\pgfpathlineto{\pgfqpoint{5.147632in}{0.739656in}}%
\pgfpathlineto{\pgfqpoint{5.147335in}{0.739656in}}%
\pgfpathlineto{\pgfqpoint{5.147037in}{0.739656in}}%
\pgfpathlineto{\pgfqpoint{5.146740in}{0.739656in}}%
\pgfpathlineto{\pgfqpoint{5.146442in}{0.739656in}}%
\pgfpathlineto{\pgfqpoint{5.146145in}{0.739656in}}%
\pgfpathlineto{\pgfqpoint{5.145848in}{0.739656in}}%
\pgfpathlineto{\pgfqpoint{5.145550in}{0.739656in}}%
\pgfpathlineto{\pgfqpoint{5.145253in}{0.739656in}}%
\pgfpathlineto{\pgfqpoint{5.144955in}{0.739656in}}%
\pgfpathlineto{\pgfqpoint{5.144658in}{0.739656in}}%
\pgfpathlineto{\pgfqpoint{5.144360in}{0.739656in}}%
\pgfpathlineto{\pgfqpoint{5.144063in}{0.739656in}}%
\pgfpathlineto{\pgfqpoint{5.143765in}{0.739656in}}%
\pgfpathlineto{\pgfqpoint{5.143468in}{0.739656in}}%
\pgfpathlineto{\pgfqpoint{5.143170in}{0.739656in}}%
\pgfpathlineto{\pgfqpoint{5.142873in}{0.739656in}}%
\pgfpathlineto{\pgfqpoint{5.142575in}{0.739656in}}%
\pgfpathlineto{\pgfqpoint{5.142278in}{0.739656in}}%
\pgfpathlineto{\pgfqpoint{5.141980in}{0.739656in}}%
\pgfpathlineto{\pgfqpoint{5.141683in}{0.739656in}}%
\pgfpathlineto{\pgfqpoint{5.141385in}{0.739656in}}%
\pgfpathlineto{\pgfqpoint{5.141088in}{0.739656in}}%
\pgfpathlineto{\pgfqpoint{5.140790in}{0.739656in}}%
\pgfpathlineto{\pgfqpoint{5.140493in}{0.739656in}}%
\pgfpathlineto{\pgfqpoint{5.140195in}{0.739656in}}%
\pgfpathlineto{\pgfqpoint{5.139898in}{0.739656in}}%
\pgfpathlineto{\pgfqpoint{5.139600in}{0.739656in}}%
\pgfpathlineto{\pgfqpoint{5.139303in}{0.739656in}}%
\pgfpathlineto{\pgfqpoint{5.139006in}{0.739656in}}%
\pgfpathlineto{\pgfqpoint{5.138708in}{0.739656in}}%
\pgfpathlineto{\pgfqpoint{5.138411in}{0.739656in}}%
\pgfpathlineto{\pgfqpoint{5.138113in}{0.739656in}}%
\pgfpathlineto{\pgfqpoint{5.137816in}{0.739656in}}%
\pgfpathlineto{\pgfqpoint{5.137518in}{0.739656in}}%
\pgfpathlineto{\pgfqpoint{5.137221in}{0.739656in}}%
\pgfpathlineto{\pgfqpoint{5.136923in}{0.739656in}}%
\pgfpathlineto{\pgfqpoint{5.136626in}{0.739656in}}%
\pgfpathlineto{\pgfqpoint{5.136328in}{0.739656in}}%
\pgfpathlineto{\pgfqpoint{5.136031in}{0.739656in}}%
\pgfpathlineto{\pgfqpoint{5.135733in}{0.739656in}}%
\pgfpathlineto{\pgfqpoint{5.135436in}{0.739656in}}%
\pgfpathlineto{\pgfqpoint{5.135138in}{0.739656in}}%
\pgfpathlineto{\pgfqpoint{5.134841in}{0.739656in}}%
\pgfpathlineto{\pgfqpoint{5.134543in}{0.739656in}}%
\pgfpathlineto{\pgfqpoint{5.134246in}{0.739656in}}%
\pgfpathlineto{\pgfqpoint{5.133948in}{0.739656in}}%
\pgfpathlineto{\pgfqpoint{5.133651in}{0.739656in}}%
\pgfpathlineto{\pgfqpoint{5.133353in}{0.739656in}}%
\pgfpathlineto{\pgfqpoint{5.133056in}{0.739656in}}%
\pgfpathlineto{\pgfqpoint{5.132758in}{0.739656in}}%
\pgfpathlineto{\pgfqpoint{5.132461in}{0.739656in}}%
\pgfpathlineto{\pgfqpoint{5.132164in}{0.739656in}}%
\pgfpathlineto{\pgfqpoint{5.131866in}{0.739656in}}%
\pgfpathlineto{\pgfqpoint{5.131569in}{0.739656in}}%
\pgfpathlineto{\pgfqpoint{5.131271in}{0.739656in}}%
\pgfpathlineto{\pgfqpoint{5.130974in}{0.739656in}}%
\pgfpathlineto{\pgfqpoint{5.130676in}{0.739656in}}%
\pgfpathlineto{\pgfqpoint{5.130379in}{0.739656in}}%
\pgfpathlineto{\pgfqpoint{5.130081in}{0.739656in}}%
\pgfpathlineto{\pgfqpoint{5.129784in}{0.739656in}}%
\pgfpathlineto{\pgfqpoint{5.129486in}{0.739656in}}%
\pgfpathlineto{\pgfqpoint{5.129189in}{0.739656in}}%
\pgfpathlineto{\pgfqpoint{5.128891in}{0.739656in}}%
\pgfpathlineto{\pgfqpoint{5.128594in}{0.739656in}}%
\pgfpathlineto{\pgfqpoint{5.128296in}{0.739656in}}%
\pgfpathlineto{\pgfqpoint{5.127999in}{0.739656in}}%
\pgfpathlineto{\pgfqpoint{5.127701in}{0.739656in}}%
\pgfpathlineto{\pgfqpoint{5.127404in}{0.739656in}}%
\pgfpathlineto{\pgfqpoint{5.127106in}{0.739656in}}%
\pgfpathlineto{\pgfqpoint{5.126809in}{0.739656in}}%
\pgfpathlineto{\pgfqpoint{5.126511in}{0.739656in}}%
\pgfpathlineto{\pgfqpoint{5.126214in}{0.739656in}}%
\pgfpathlineto{\pgfqpoint{5.125917in}{0.739656in}}%
\pgfpathlineto{\pgfqpoint{5.125619in}{0.739656in}}%
\pgfpathlineto{\pgfqpoint{5.125322in}{0.739656in}}%
\pgfpathlineto{\pgfqpoint{5.125024in}{0.739656in}}%
\pgfpathlineto{\pgfqpoint{5.124727in}{0.739656in}}%
\pgfpathlineto{\pgfqpoint{5.124429in}{0.739656in}}%
\pgfpathlineto{\pgfqpoint{5.124132in}{0.739656in}}%
\pgfpathlineto{\pgfqpoint{5.123834in}{0.739656in}}%
\pgfpathlineto{\pgfqpoint{5.123537in}{0.739656in}}%
\pgfpathlineto{\pgfqpoint{5.123239in}{0.739656in}}%
\pgfpathlineto{\pgfqpoint{5.122942in}{0.739656in}}%
\pgfpathlineto{\pgfqpoint{5.122644in}{0.739656in}}%
\pgfpathlineto{\pgfqpoint{5.122347in}{0.739656in}}%
\pgfpathlineto{\pgfqpoint{5.122049in}{0.739656in}}%
\pgfpathlineto{\pgfqpoint{5.121752in}{0.739656in}}%
\pgfpathlineto{\pgfqpoint{5.121454in}{0.739656in}}%
\pgfpathlineto{\pgfqpoint{5.121157in}{0.739656in}}%
\pgfpathlineto{\pgfqpoint{5.120859in}{0.739656in}}%
\pgfpathlineto{\pgfqpoint{5.120562in}{0.739656in}}%
\pgfpathlineto{\pgfqpoint{5.120264in}{0.739656in}}%
\pgfpathlineto{\pgfqpoint{5.119967in}{0.739656in}}%
\pgfpathlineto{\pgfqpoint{5.119669in}{0.739656in}}%
\pgfpathlineto{\pgfqpoint{5.119372in}{0.739656in}}%
\pgfpathlineto{\pgfqpoint{5.119075in}{0.739656in}}%
\pgfpathlineto{\pgfqpoint{5.118777in}{0.739656in}}%
\pgfpathlineto{\pgfqpoint{5.118480in}{0.739656in}}%
\pgfpathlineto{\pgfqpoint{5.118182in}{0.739656in}}%
\pgfpathlineto{\pgfqpoint{5.117885in}{0.739656in}}%
\pgfpathlineto{\pgfqpoint{5.117587in}{0.739656in}}%
\pgfpathlineto{\pgfqpoint{5.117290in}{0.739656in}}%
\pgfpathlineto{\pgfqpoint{5.116992in}{0.739656in}}%
\pgfpathlineto{\pgfqpoint{5.116695in}{0.739656in}}%
\pgfpathlineto{\pgfqpoint{5.116397in}{0.739656in}}%
\pgfpathlineto{\pgfqpoint{5.116100in}{0.739656in}}%
\pgfpathlineto{\pgfqpoint{5.115802in}{0.739656in}}%
\pgfpathlineto{\pgfqpoint{5.115505in}{0.739656in}}%
\pgfpathlineto{\pgfqpoint{5.115207in}{0.739656in}}%
\pgfpathlineto{\pgfqpoint{5.114910in}{0.739656in}}%
\pgfpathlineto{\pgfqpoint{5.114612in}{0.739656in}}%
\pgfpathlineto{\pgfqpoint{5.114315in}{0.739656in}}%
\pgfpathlineto{\pgfqpoint{5.114017in}{0.739656in}}%
\pgfpathlineto{\pgfqpoint{5.113720in}{0.739656in}}%
\pgfpathlineto{\pgfqpoint{5.113422in}{0.739656in}}%
\pgfpathlineto{\pgfqpoint{5.113125in}{0.739656in}}%
\pgfpathlineto{\pgfqpoint{5.112827in}{0.739656in}}%
\pgfpathlineto{\pgfqpoint{5.112530in}{0.739656in}}%
\pgfpathlineto{\pgfqpoint{5.112233in}{0.739656in}}%
\pgfpathlineto{\pgfqpoint{5.111935in}{0.739656in}}%
\pgfpathlineto{\pgfqpoint{5.111638in}{0.739656in}}%
\pgfpathlineto{\pgfqpoint{5.111340in}{0.739656in}}%
\pgfpathlineto{\pgfqpoint{5.111043in}{0.739656in}}%
\pgfpathlineto{\pgfqpoint{5.110745in}{0.739656in}}%
\pgfpathlineto{\pgfqpoint{5.110448in}{0.739656in}}%
\pgfpathlineto{\pgfqpoint{5.110150in}{0.739656in}}%
\pgfpathlineto{\pgfqpoint{5.109853in}{0.739656in}}%
\pgfpathlineto{\pgfqpoint{5.109555in}{0.739656in}}%
\pgfpathlineto{\pgfqpoint{5.109258in}{0.739656in}}%
\pgfpathlineto{\pgfqpoint{5.108960in}{0.739656in}}%
\pgfpathlineto{\pgfqpoint{5.108663in}{0.739656in}}%
\pgfpathlineto{\pgfqpoint{5.108365in}{0.739656in}}%
\pgfpathlineto{\pgfqpoint{5.108068in}{0.739656in}}%
\pgfpathlineto{\pgfqpoint{5.107770in}{0.739656in}}%
\pgfpathlineto{\pgfqpoint{5.107473in}{0.739656in}}%
\pgfpathlineto{\pgfqpoint{5.107175in}{0.739656in}}%
\pgfpathlineto{\pgfqpoint{5.106878in}{0.739656in}}%
\pgfpathlineto{\pgfqpoint{5.106580in}{0.739656in}}%
\pgfpathlineto{\pgfqpoint{5.106283in}{0.739656in}}%
\pgfpathlineto{\pgfqpoint{5.105986in}{0.739656in}}%
\pgfpathlineto{\pgfqpoint{5.105688in}{0.739656in}}%
\pgfpathlineto{\pgfqpoint{5.105391in}{0.739656in}}%
\pgfpathlineto{\pgfqpoint{5.105093in}{0.739656in}}%
\pgfpathlineto{\pgfqpoint{5.104796in}{0.739656in}}%
\pgfpathlineto{\pgfqpoint{5.104498in}{0.739656in}}%
\pgfpathlineto{\pgfqpoint{5.104201in}{0.739656in}}%
\pgfpathlineto{\pgfqpoint{5.103903in}{0.739656in}}%
\pgfpathlineto{\pgfqpoint{5.103606in}{0.739656in}}%
\pgfpathlineto{\pgfqpoint{5.103308in}{0.739656in}}%
\pgfpathlineto{\pgfqpoint{5.103011in}{0.739656in}}%
\pgfpathlineto{\pgfqpoint{5.102713in}{0.739656in}}%
\pgfpathlineto{\pgfqpoint{5.102416in}{0.739656in}}%
\pgfpathlineto{\pgfqpoint{5.102118in}{0.739656in}}%
\pgfpathlineto{\pgfqpoint{5.101821in}{0.739656in}}%
\pgfpathlineto{\pgfqpoint{5.101523in}{0.739656in}}%
\pgfpathlineto{\pgfqpoint{5.101226in}{0.739656in}}%
\pgfpathlineto{\pgfqpoint{5.100928in}{0.739656in}}%
\pgfpathlineto{\pgfqpoint{5.100631in}{0.739656in}}%
\pgfpathlineto{\pgfqpoint{5.100333in}{0.739656in}}%
\pgfpathlineto{\pgfqpoint{5.100036in}{0.739656in}}%
\pgfpathlineto{\pgfqpoint{5.099738in}{0.739656in}}%
\pgfpathlineto{\pgfqpoint{5.099441in}{0.739656in}}%
\pgfpathlineto{\pgfqpoint{5.099144in}{0.739656in}}%
\pgfpathlineto{\pgfqpoint{5.098846in}{0.739656in}}%
\pgfpathlineto{\pgfqpoint{5.098549in}{0.739656in}}%
\pgfpathlineto{\pgfqpoint{5.098251in}{0.739656in}}%
\pgfpathlineto{\pgfqpoint{5.097954in}{0.739656in}}%
\pgfpathlineto{\pgfqpoint{5.097656in}{0.739656in}}%
\pgfpathlineto{\pgfqpoint{5.097359in}{0.739656in}}%
\pgfpathlineto{\pgfqpoint{5.097061in}{0.739656in}}%
\pgfpathlineto{\pgfqpoint{5.096764in}{0.739656in}}%
\pgfpathlineto{\pgfqpoint{5.096466in}{0.739656in}}%
\pgfpathlineto{\pgfqpoint{5.096169in}{0.739656in}}%
\pgfpathlineto{\pgfqpoint{5.095871in}{0.739656in}}%
\pgfpathlineto{\pgfqpoint{5.095574in}{0.739656in}}%
\pgfpathlineto{\pgfqpoint{5.095276in}{0.739656in}}%
\pgfpathlineto{\pgfqpoint{5.094979in}{0.739656in}}%
\pgfpathlineto{\pgfqpoint{5.094681in}{0.739656in}}%
\pgfpathlineto{\pgfqpoint{5.094384in}{0.739656in}}%
\pgfpathlineto{\pgfqpoint{5.094086in}{0.739656in}}%
\pgfpathlineto{\pgfqpoint{5.093789in}{0.739656in}}%
\pgfpathlineto{\pgfqpoint{5.093491in}{0.739656in}}%
\pgfpathlineto{\pgfqpoint{5.093194in}{0.739656in}}%
\pgfpathlineto{\pgfqpoint{5.092896in}{0.739656in}}%
\pgfpathlineto{\pgfqpoint{5.092599in}{0.739656in}}%
\pgfpathlineto{\pgfqpoint{5.092302in}{0.739656in}}%
\pgfpathlineto{\pgfqpoint{5.092004in}{0.739656in}}%
\pgfpathlineto{\pgfqpoint{5.091707in}{0.739656in}}%
\pgfpathlineto{\pgfqpoint{5.091409in}{0.739656in}}%
\pgfpathlineto{\pgfqpoint{5.091112in}{0.739656in}}%
\pgfpathlineto{\pgfqpoint{5.090814in}{0.739656in}}%
\pgfpathlineto{\pgfqpoint{5.090517in}{0.739656in}}%
\pgfpathlineto{\pgfqpoint{5.090219in}{0.739656in}}%
\pgfpathlineto{\pgfqpoint{5.089922in}{0.739656in}}%
\pgfpathlineto{\pgfqpoint{5.089624in}{0.739656in}}%
\pgfpathlineto{\pgfqpoint{5.089327in}{0.739656in}}%
\pgfpathlineto{\pgfqpoint{5.089029in}{0.739656in}}%
\pgfpathlineto{\pgfqpoint{5.088732in}{0.739656in}}%
\pgfpathlineto{\pgfqpoint{5.088434in}{0.739656in}}%
\pgfpathlineto{\pgfqpoint{5.088137in}{0.739656in}}%
\pgfpathlineto{\pgfqpoint{5.087839in}{0.739656in}}%
\pgfpathlineto{\pgfqpoint{5.087542in}{0.739656in}}%
\pgfpathlineto{\pgfqpoint{5.087244in}{0.739656in}}%
\pgfpathlineto{\pgfqpoint{5.086947in}{0.739656in}}%
\pgfpathlineto{\pgfqpoint{5.086649in}{0.739656in}}%
\pgfpathlineto{\pgfqpoint{5.086352in}{0.739656in}}%
\pgfpathlineto{\pgfqpoint{5.086055in}{0.739656in}}%
\pgfpathlineto{\pgfqpoint{5.085757in}{0.739656in}}%
\pgfpathlineto{\pgfqpoint{5.085460in}{0.739656in}}%
\pgfpathlineto{\pgfqpoint{5.085162in}{0.739656in}}%
\pgfpathlineto{\pgfqpoint{5.084865in}{0.739656in}}%
\pgfpathlineto{\pgfqpoint{5.084567in}{0.739656in}}%
\pgfpathlineto{\pgfqpoint{5.084270in}{0.739656in}}%
\pgfpathlineto{\pgfqpoint{5.083972in}{0.739656in}}%
\pgfpathlineto{\pgfqpoint{5.083675in}{0.739656in}}%
\pgfpathlineto{\pgfqpoint{5.083377in}{0.739656in}}%
\pgfpathlineto{\pgfqpoint{5.083080in}{0.739656in}}%
\pgfpathlineto{\pgfqpoint{5.082782in}{0.739656in}}%
\pgfpathlineto{\pgfqpoint{5.082485in}{0.739656in}}%
\pgfpathlineto{\pgfqpoint{5.082187in}{0.739656in}}%
\pgfpathlineto{\pgfqpoint{5.081890in}{0.739656in}}%
\pgfpathlineto{\pgfqpoint{5.081592in}{0.739656in}}%
\pgfpathlineto{\pgfqpoint{5.081295in}{0.739656in}}%
\pgfpathlineto{\pgfqpoint{5.080997in}{0.739656in}}%
\pgfpathlineto{\pgfqpoint{5.080700in}{0.739656in}}%
\pgfpathlineto{\pgfqpoint{5.080402in}{0.739656in}}%
\pgfpathlineto{\pgfqpoint{5.080105in}{0.739656in}}%
\pgfpathlineto{\pgfqpoint{5.079807in}{0.739656in}}%
\pgfpathlineto{\pgfqpoint{5.079510in}{0.739656in}}%
\pgfpathlineto{\pgfqpoint{5.079213in}{0.739656in}}%
\pgfpathlineto{\pgfqpoint{5.078915in}{0.739656in}}%
\pgfpathlineto{\pgfqpoint{5.078618in}{0.739656in}}%
\pgfpathlineto{\pgfqpoint{5.078320in}{0.739656in}}%
\pgfpathlineto{\pgfqpoint{5.078023in}{0.739656in}}%
\pgfpathlineto{\pgfqpoint{5.077725in}{0.739656in}}%
\pgfpathlineto{\pgfqpoint{5.077428in}{0.739656in}}%
\pgfpathlineto{\pgfqpoint{5.077130in}{0.739656in}}%
\pgfpathlineto{\pgfqpoint{5.076833in}{0.739656in}}%
\pgfpathlineto{\pgfqpoint{5.076535in}{0.739656in}}%
\pgfpathlineto{\pgfqpoint{5.076238in}{0.739656in}}%
\pgfpathlineto{\pgfqpoint{5.075940in}{0.739656in}}%
\pgfpathlineto{\pgfqpoint{5.075643in}{0.739656in}}%
\pgfpathlineto{\pgfqpoint{5.075345in}{0.739656in}}%
\pgfpathlineto{\pgfqpoint{5.075048in}{0.739656in}}%
\pgfpathlineto{\pgfqpoint{5.074750in}{0.739656in}}%
\pgfpathlineto{\pgfqpoint{5.074453in}{0.739656in}}%
\pgfpathlineto{\pgfqpoint{5.074155in}{0.739656in}}%
\pgfpathlineto{\pgfqpoint{5.073858in}{0.739656in}}%
\pgfpathlineto{\pgfqpoint{5.073560in}{0.739656in}}%
\pgfpathlineto{\pgfqpoint{5.073263in}{0.739656in}}%
\pgfpathlineto{\pgfqpoint{5.072965in}{0.739656in}}%
\pgfpathlineto{\pgfqpoint{5.072668in}{0.739656in}}%
\pgfpathlineto{\pgfqpoint{5.072371in}{0.739656in}}%
\pgfpathlineto{\pgfqpoint{5.072073in}{0.739656in}}%
\pgfpathlineto{\pgfqpoint{5.071776in}{0.739656in}}%
\pgfpathlineto{\pgfqpoint{5.071478in}{0.739656in}}%
\pgfpathlineto{\pgfqpoint{5.071181in}{0.739656in}}%
\pgfpathlineto{\pgfqpoint{5.070883in}{0.739656in}}%
\pgfpathlineto{\pgfqpoint{5.070586in}{0.739656in}}%
\pgfpathlineto{\pgfqpoint{5.070288in}{0.739656in}}%
\pgfpathlineto{\pgfqpoint{5.069991in}{0.739656in}}%
\pgfpathlineto{\pgfqpoint{5.069693in}{0.739656in}}%
\pgfpathlineto{\pgfqpoint{5.069396in}{0.739656in}}%
\pgfpathlineto{\pgfqpoint{5.069098in}{0.739656in}}%
\pgfpathlineto{\pgfqpoint{5.068801in}{0.739656in}}%
\pgfpathlineto{\pgfqpoint{5.068503in}{0.739656in}}%
\pgfpathlineto{\pgfqpoint{5.068206in}{0.739656in}}%
\pgfpathlineto{\pgfqpoint{5.067908in}{0.739656in}}%
\pgfpathlineto{\pgfqpoint{5.067611in}{0.739656in}}%
\pgfpathlineto{\pgfqpoint{5.067313in}{0.739656in}}%
\pgfpathlineto{\pgfqpoint{5.067016in}{0.739656in}}%
\pgfpathlineto{\pgfqpoint{5.066718in}{0.739656in}}%
\pgfpathlineto{\pgfqpoint{5.066421in}{0.739656in}}%
\pgfpathlineto{\pgfqpoint{5.066124in}{0.739656in}}%
\pgfpathlineto{\pgfqpoint{5.065826in}{0.739656in}}%
\pgfpathlineto{\pgfqpoint{5.065529in}{0.739656in}}%
\pgfpathlineto{\pgfqpoint{5.065231in}{0.739656in}}%
\pgfpathlineto{\pgfqpoint{5.064934in}{0.739656in}}%
\pgfpathlineto{\pgfqpoint{5.064636in}{0.739656in}}%
\pgfpathlineto{\pgfqpoint{5.064339in}{0.739656in}}%
\pgfpathlineto{\pgfqpoint{5.064041in}{0.739656in}}%
\pgfpathlineto{\pgfqpoint{5.063744in}{0.739656in}}%
\pgfpathlineto{\pgfqpoint{5.063446in}{0.739656in}}%
\pgfpathlineto{\pgfqpoint{5.063149in}{0.739656in}}%
\pgfpathlineto{\pgfqpoint{5.062851in}{0.739656in}}%
\pgfpathlineto{\pgfqpoint{5.062554in}{0.739656in}}%
\pgfpathlineto{\pgfqpoint{5.062256in}{0.739656in}}%
\pgfpathlineto{\pgfqpoint{5.061959in}{0.739656in}}%
\pgfpathlineto{\pgfqpoint{5.061661in}{0.739656in}}%
\pgfpathlineto{\pgfqpoint{5.061364in}{0.739656in}}%
\pgfpathlineto{\pgfqpoint{5.061066in}{0.739656in}}%
\pgfpathlineto{\pgfqpoint{5.060769in}{0.739656in}}%
\pgfpathlineto{\pgfqpoint{5.060471in}{0.739656in}}%
\pgfpathlineto{\pgfqpoint{5.060174in}{0.739656in}}%
\pgfpathlineto{\pgfqpoint{5.059876in}{0.739656in}}%
\pgfpathlineto{\pgfqpoint{5.059579in}{0.739656in}}%
\pgfpathlineto{\pgfqpoint{5.059282in}{0.739656in}}%
\pgfpathlineto{\pgfqpoint{5.058984in}{0.739656in}}%
\pgfpathlineto{\pgfqpoint{5.058687in}{0.739656in}}%
\pgfpathlineto{\pgfqpoint{5.058389in}{0.739656in}}%
\pgfpathlineto{\pgfqpoint{5.058092in}{0.739656in}}%
\pgfpathlineto{\pgfqpoint{5.057794in}{0.739656in}}%
\pgfpathlineto{\pgfqpoint{5.057497in}{0.739656in}}%
\pgfpathlineto{\pgfqpoint{5.057199in}{0.739656in}}%
\pgfpathlineto{\pgfqpoint{5.056902in}{0.739656in}}%
\pgfpathlineto{\pgfqpoint{5.056604in}{0.739656in}}%
\pgfpathlineto{\pgfqpoint{5.056307in}{0.739656in}}%
\pgfpathlineto{\pgfqpoint{5.056009in}{0.739656in}}%
\pgfpathlineto{\pgfqpoint{5.055712in}{0.739656in}}%
\pgfpathlineto{\pgfqpoint{5.055414in}{0.739656in}}%
\pgfpathlineto{\pgfqpoint{5.055117in}{0.739656in}}%
\pgfpathlineto{\pgfqpoint{5.054819in}{0.739656in}}%
\pgfpathlineto{\pgfqpoint{5.054522in}{0.739656in}}%
\pgfpathlineto{\pgfqpoint{5.054224in}{0.739656in}}%
\pgfpathlineto{\pgfqpoint{5.053927in}{0.739656in}}%
\pgfpathlineto{\pgfqpoint{5.053629in}{0.739656in}}%
\pgfpathlineto{\pgfqpoint{5.053332in}{0.739656in}}%
\pgfpathlineto{\pgfqpoint{5.053034in}{0.739656in}}%
\pgfpathlineto{\pgfqpoint{5.052737in}{0.739656in}}%
\pgfpathlineto{\pgfqpoint{5.052440in}{0.739656in}}%
\pgfpathlineto{\pgfqpoint{5.052142in}{0.739656in}}%
\pgfpathlineto{\pgfqpoint{5.051845in}{0.739656in}}%
\pgfpathlineto{\pgfqpoint{5.051547in}{0.739656in}}%
\pgfpathlineto{\pgfqpoint{5.051250in}{0.739656in}}%
\pgfpathlineto{\pgfqpoint{5.050952in}{0.739656in}}%
\pgfpathlineto{\pgfqpoint{5.050655in}{0.739656in}}%
\pgfpathlineto{\pgfqpoint{5.050357in}{0.739656in}}%
\pgfpathlineto{\pgfqpoint{5.050060in}{0.739656in}}%
\pgfpathlineto{\pgfqpoint{5.049762in}{0.739656in}}%
\pgfpathlineto{\pgfqpoint{5.049465in}{0.739656in}}%
\pgfpathlineto{\pgfqpoint{5.049167in}{0.739656in}}%
\pgfpathlineto{\pgfqpoint{5.048870in}{0.739656in}}%
\pgfpathlineto{\pgfqpoint{5.048572in}{0.739656in}}%
\pgfpathlineto{\pgfqpoint{5.048275in}{0.739656in}}%
\pgfpathlineto{\pgfqpoint{5.047977in}{0.739656in}}%
\pgfpathlineto{\pgfqpoint{5.047680in}{0.739656in}}%
\pgfpathlineto{\pgfqpoint{5.047382in}{0.739656in}}%
\pgfpathlineto{\pgfqpoint{5.047085in}{0.739656in}}%
\pgfpathlineto{\pgfqpoint{5.046787in}{0.739656in}}%
\pgfpathlineto{\pgfqpoint{5.046490in}{0.739656in}}%
\pgfpathlineto{\pgfqpoint{5.046193in}{0.739656in}}%
\pgfpathlineto{\pgfqpoint{5.045895in}{0.739656in}}%
\pgfpathlineto{\pgfqpoint{5.045598in}{0.739656in}}%
\pgfpathlineto{\pgfqpoint{5.045300in}{0.739656in}}%
\pgfpathlineto{\pgfqpoint{5.045003in}{0.739656in}}%
\pgfpathlineto{\pgfqpoint{5.044705in}{0.739656in}}%
\pgfpathlineto{\pgfqpoint{5.044408in}{0.739656in}}%
\pgfpathlineto{\pgfqpoint{5.044110in}{0.739656in}}%
\pgfpathlineto{\pgfqpoint{5.043813in}{0.739656in}}%
\pgfpathlineto{\pgfqpoint{5.043515in}{0.739656in}}%
\pgfpathlineto{\pgfqpoint{5.043218in}{0.739656in}}%
\pgfpathlineto{\pgfqpoint{5.042920in}{0.739656in}}%
\pgfpathlineto{\pgfqpoint{5.042623in}{0.739656in}}%
\pgfpathlineto{\pgfqpoint{5.042325in}{0.739656in}}%
\pgfpathlineto{\pgfqpoint{5.042028in}{0.739656in}}%
\pgfpathlineto{\pgfqpoint{5.041730in}{0.739656in}}%
\pgfpathlineto{\pgfqpoint{5.041433in}{0.739656in}}%
\pgfpathlineto{\pgfqpoint{5.041135in}{0.739656in}}%
\pgfpathlineto{\pgfqpoint{5.040838in}{0.739656in}}%
\pgfpathlineto{\pgfqpoint{5.040540in}{0.739656in}}%
\pgfpathlineto{\pgfqpoint{5.040243in}{0.739656in}}%
\pgfpathlineto{\pgfqpoint{5.039945in}{0.739656in}}%
\pgfpathlineto{\pgfqpoint{5.039648in}{0.739656in}}%
\pgfpathlineto{\pgfqpoint{5.039351in}{0.739656in}}%
\pgfpathlineto{\pgfqpoint{5.039053in}{0.739656in}}%
\pgfpathlineto{\pgfqpoint{5.038756in}{0.739656in}}%
\pgfpathlineto{\pgfqpoint{5.038458in}{0.739656in}}%
\pgfpathlineto{\pgfqpoint{5.038161in}{0.739656in}}%
\pgfpathlineto{\pgfqpoint{5.037863in}{0.739656in}}%
\pgfpathlineto{\pgfqpoint{5.037566in}{0.739656in}}%
\pgfpathlineto{\pgfqpoint{5.037268in}{0.739656in}}%
\pgfpathlineto{\pgfqpoint{5.036971in}{0.739656in}}%
\pgfpathlineto{\pgfqpoint{5.036673in}{0.739656in}}%
\pgfpathlineto{\pgfqpoint{5.036376in}{0.739656in}}%
\pgfpathlineto{\pgfqpoint{5.036078in}{0.739656in}}%
\pgfpathlineto{\pgfqpoint{5.035781in}{0.739656in}}%
\pgfpathlineto{\pgfqpoint{5.035483in}{0.739656in}}%
\pgfpathlineto{\pgfqpoint{5.035186in}{0.739656in}}%
\pgfpathlineto{\pgfqpoint{5.034888in}{0.739656in}}%
\pgfpathlineto{\pgfqpoint{5.034591in}{0.739656in}}%
\pgfpathlineto{\pgfqpoint{5.034293in}{0.739656in}}%
\pgfpathlineto{\pgfqpoint{5.033996in}{0.739656in}}%
\pgfpathlineto{\pgfqpoint{5.033698in}{0.739656in}}%
\pgfpathlineto{\pgfqpoint{5.033401in}{0.739656in}}%
\pgfpathlineto{\pgfqpoint{5.033103in}{0.739656in}}%
\pgfpathlineto{\pgfqpoint{5.032806in}{0.739656in}}%
\pgfpathlineto{\pgfqpoint{5.032509in}{0.739656in}}%
\pgfpathlineto{\pgfqpoint{5.032211in}{0.739656in}}%
\pgfpathlineto{\pgfqpoint{5.031914in}{0.739656in}}%
\pgfpathlineto{\pgfqpoint{5.031616in}{0.739656in}}%
\pgfpathlineto{\pgfqpoint{5.031319in}{0.739656in}}%
\pgfpathlineto{\pgfqpoint{5.031021in}{0.739656in}}%
\pgfpathlineto{\pgfqpoint{5.030724in}{0.739656in}}%
\pgfpathlineto{\pgfqpoint{5.030426in}{0.739656in}}%
\pgfpathlineto{\pgfqpoint{5.030129in}{0.739656in}}%
\pgfpathlineto{\pgfqpoint{5.029831in}{0.739656in}}%
\pgfpathlineto{\pgfqpoint{5.029534in}{0.739656in}}%
\pgfpathlineto{\pgfqpoint{5.029236in}{0.739656in}}%
\pgfpathlineto{\pgfqpoint{5.028939in}{0.739656in}}%
\pgfpathlineto{\pgfqpoint{5.028641in}{0.739656in}}%
\pgfpathlineto{\pgfqpoint{5.028344in}{0.739656in}}%
\pgfpathlineto{\pgfqpoint{5.028046in}{0.739656in}}%
\pgfpathlineto{\pgfqpoint{5.027749in}{0.739656in}}%
\pgfpathlineto{\pgfqpoint{5.027451in}{0.739656in}}%
\pgfpathlineto{\pgfqpoint{5.027154in}{0.739656in}}%
\pgfpathlineto{\pgfqpoint{5.026856in}{0.739656in}}%
\pgfpathlineto{\pgfqpoint{5.026559in}{0.739656in}}%
\pgfpathlineto{\pgfqpoint{5.026261in}{0.739656in}}%
\pgfpathlineto{\pgfqpoint{5.025964in}{0.739656in}}%
\pgfpathlineto{\pgfqpoint{5.025667in}{0.739656in}}%
\pgfpathlineto{\pgfqpoint{5.025369in}{0.739656in}}%
\pgfpathlineto{\pgfqpoint{5.025072in}{0.739656in}}%
\pgfpathlineto{\pgfqpoint{5.024774in}{0.739656in}}%
\pgfpathlineto{\pgfqpoint{5.024477in}{0.739656in}}%
\pgfpathlineto{\pgfqpoint{5.024179in}{0.739656in}}%
\pgfpathlineto{\pgfqpoint{5.023882in}{0.739656in}}%
\pgfpathlineto{\pgfqpoint{5.023584in}{0.739656in}}%
\pgfpathlineto{\pgfqpoint{5.023287in}{0.739656in}}%
\pgfpathlineto{\pgfqpoint{5.022989in}{0.739656in}}%
\pgfpathlineto{\pgfqpoint{5.022692in}{0.739656in}}%
\pgfpathlineto{\pgfqpoint{5.022394in}{0.739656in}}%
\pgfpathlineto{\pgfqpoint{5.022097in}{0.739656in}}%
\pgfpathlineto{\pgfqpoint{5.021799in}{0.739656in}}%
\pgfpathlineto{\pgfqpoint{5.021502in}{0.739656in}}%
\pgfpathlineto{\pgfqpoint{5.021204in}{0.739656in}}%
\pgfpathlineto{\pgfqpoint{5.020907in}{0.739656in}}%
\pgfpathlineto{\pgfqpoint{5.020609in}{0.739656in}}%
\pgfpathlineto{\pgfqpoint{5.020312in}{0.739656in}}%
\pgfpathlineto{\pgfqpoint{5.020014in}{0.739656in}}%
\pgfpathlineto{\pgfqpoint{5.019717in}{0.739656in}}%
\pgfpathlineto{\pgfqpoint{5.019420in}{0.739656in}}%
\pgfpathlineto{\pgfqpoint{5.019122in}{0.739656in}}%
\pgfpathlineto{\pgfqpoint{5.018825in}{0.739656in}}%
\pgfpathlineto{\pgfqpoint{5.018527in}{0.739656in}}%
\pgfpathlineto{\pgfqpoint{5.018230in}{0.739656in}}%
\pgfpathlineto{\pgfqpoint{5.017932in}{0.739656in}}%
\pgfpathlineto{\pgfqpoint{5.017635in}{0.739656in}}%
\pgfpathlineto{\pgfqpoint{5.017337in}{0.739656in}}%
\pgfpathlineto{\pgfqpoint{5.017040in}{0.739656in}}%
\pgfpathlineto{\pgfqpoint{5.016742in}{0.739656in}}%
\pgfpathlineto{\pgfqpoint{5.016445in}{0.739656in}}%
\pgfpathlineto{\pgfqpoint{5.016147in}{0.739656in}}%
\pgfpathlineto{\pgfqpoint{5.015850in}{0.739656in}}%
\pgfpathlineto{\pgfqpoint{5.015552in}{0.739656in}}%
\pgfpathlineto{\pgfqpoint{5.015255in}{0.739656in}}%
\pgfpathlineto{\pgfqpoint{5.014957in}{0.739656in}}%
\pgfpathlineto{\pgfqpoint{5.014660in}{0.739656in}}%
\pgfpathlineto{\pgfqpoint{5.014362in}{0.739656in}}%
\pgfpathlineto{\pgfqpoint{5.014065in}{0.739656in}}%
\pgfpathlineto{\pgfqpoint{5.013767in}{0.739656in}}%
\pgfpathlineto{\pgfqpoint{5.013470in}{0.739656in}}%
\pgfpathlineto{\pgfqpoint{5.013172in}{0.739656in}}%
\pgfpathlineto{\pgfqpoint{5.012875in}{0.739656in}}%
\pgfpathlineto{\pgfqpoint{5.012578in}{0.739656in}}%
\pgfpathlineto{\pgfqpoint{5.012280in}{0.739656in}}%
\pgfpathlineto{\pgfqpoint{5.011983in}{0.739656in}}%
\pgfpathlineto{\pgfqpoint{5.011685in}{0.739656in}}%
\pgfpathlineto{\pgfqpoint{5.011388in}{0.739656in}}%
\pgfpathlineto{\pgfqpoint{5.011090in}{0.739656in}}%
\pgfpathlineto{\pgfqpoint{5.010793in}{0.739656in}}%
\pgfpathlineto{\pgfqpoint{5.010495in}{0.739656in}}%
\pgfpathlineto{\pgfqpoint{5.010198in}{0.739656in}}%
\pgfpathlineto{\pgfqpoint{5.009900in}{0.739656in}}%
\pgfpathlineto{\pgfqpoint{5.009603in}{0.739656in}}%
\pgfpathlineto{\pgfqpoint{5.009305in}{0.739656in}}%
\pgfpathlineto{\pgfqpoint{5.009008in}{0.739656in}}%
\pgfpathlineto{\pgfqpoint{5.008710in}{0.739656in}}%
\pgfpathlineto{\pgfqpoint{5.008413in}{0.739656in}}%
\pgfpathlineto{\pgfqpoint{5.008115in}{0.739656in}}%
\pgfpathlineto{\pgfqpoint{5.007818in}{0.739656in}}%
\pgfpathlineto{\pgfqpoint{5.007520in}{0.739656in}}%
\pgfpathlineto{\pgfqpoint{5.007223in}{0.739656in}}%
\pgfpathlineto{\pgfqpoint{5.006925in}{0.739656in}}%
\pgfpathlineto{\pgfqpoint{5.006628in}{0.739656in}}%
\pgfpathlineto{\pgfqpoint{5.006330in}{0.739656in}}%
\pgfpathlineto{\pgfqpoint{5.006033in}{0.739656in}}%
\pgfpathlineto{\pgfqpoint{5.005736in}{0.739656in}}%
\pgfpathlineto{\pgfqpoint{5.005438in}{0.739656in}}%
\pgfpathlineto{\pgfqpoint{5.005141in}{0.739656in}}%
\pgfpathlineto{\pgfqpoint{5.004843in}{0.739656in}}%
\pgfpathlineto{\pgfqpoint{5.004546in}{0.739656in}}%
\pgfpathlineto{\pgfqpoint{5.004248in}{0.739656in}}%
\pgfpathlineto{\pgfqpoint{5.003951in}{0.739656in}}%
\pgfpathlineto{\pgfqpoint{5.003653in}{0.739656in}}%
\pgfpathlineto{\pgfqpoint{5.003356in}{0.739656in}}%
\pgfpathlineto{\pgfqpoint{5.003058in}{0.739656in}}%
\pgfpathlineto{\pgfqpoint{5.002761in}{0.739656in}}%
\pgfpathlineto{\pgfqpoint{5.002463in}{0.739656in}}%
\pgfpathlineto{\pgfqpoint{5.002166in}{0.739656in}}%
\pgfpathlineto{\pgfqpoint{5.001868in}{0.739656in}}%
\pgfpathlineto{\pgfqpoint{5.001571in}{0.739656in}}%
\pgfpathlineto{\pgfqpoint{5.001273in}{0.739656in}}%
\pgfpathlineto{\pgfqpoint{5.000976in}{0.739656in}}%
\pgfpathlineto{\pgfqpoint{5.000678in}{0.739656in}}%
\pgfpathlineto{\pgfqpoint{5.000381in}{0.739656in}}%
\pgfpathlineto{\pgfqpoint{5.000083in}{0.739656in}}%
\pgfpathlineto{\pgfqpoint{4.999786in}{0.739656in}}%
\pgfpathlineto{\pgfqpoint{4.999489in}{0.739656in}}%
\pgfpathlineto{\pgfqpoint{4.999191in}{0.739656in}}%
\pgfpathlineto{\pgfqpoint{4.998894in}{0.739656in}}%
\pgfpathlineto{\pgfqpoint{4.998596in}{0.739656in}}%
\pgfpathlineto{\pgfqpoint{4.998299in}{0.739656in}}%
\pgfpathlineto{\pgfqpoint{4.998001in}{0.739656in}}%
\pgfpathlineto{\pgfqpoint{4.997704in}{0.739656in}}%
\pgfpathlineto{\pgfqpoint{4.997406in}{0.739656in}}%
\pgfpathlineto{\pgfqpoint{4.997109in}{0.739656in}}%
\pgfpathlineto{\pgfqpoint{4.996811in}{0.739656in}}%
\pgfpathlineto{\pgfqpoint{4.996514in}{0.739656in}}%
\pgfpathlineto{\pgfqpoint{4.996216in}{0.739656in}}%
\pgfpathlineto{\pgfqpoint{4.995919in}{0.739656in}}%
\pgfpathlineto{\pgfqpoint{4.995621in}{0.739656in}}%
\pgfpathlineto{\pgfqpoint{4.995324in}{0.739656in}}%
\pgfpathlineto{\pgfqpoint{4.995026in}{0.739656in}}%
\pgfpathlineto{\pgfqpoint{4.994729in}{0.739656in}}%
\pgfpathlineto{\pgfqpoint{4.994431in}{0.739656in}}%
\pgfpathlineto{\pgfqpoint{4.994134in}{0.739656in}}%
\pgfpathlineto{\pgfqpoint{4.993836in}{0.739656in}}%
\pgfpathlineto{\pgfqpoint{4.993539in}{0.739656in}}%
\pgfpathlineto{\pgfqpoint{4.993241in}{0.739656in}}%
\pgfpathlineto{\pgfqpoint{4.992944in}{0.739656in}}%
\pgfpathlineto{\pgfqpoint{4.992647in}{0.739656in}}%
\pgfpathlineto{\pgfqpoint{4.992349in}{0.739656in}}%
\pgfpathlineto{\pgfqpoint{4.992052in}{0.739656in}}%
\pgfpathlineto{\pgfqpoint{4.991754in}{0.739656in}}%
\pgfpathlineto{\pgfqpoint{4.991457in}{0.739656in}}%
\pgfpathlineto{\pgfqpoint{4.991159in}{0.739656in}}%
\pgfpathlineto{\pgfqpoint{4.990862in}{0.739656in}}%
\pgfpathlineto{\pgfqpoint{4.990564in}{0.739656in}}%
\pgfpathlineto{\pgfqpoint{4.990267in}{0.739656in}}%
\pgfpathlineto{\pgfqpoint{4.989969in}{0.739656in}}%
\pgfpathlineto{\pgfqpoint{4.989672in}{0.739656in}}%
\pgfpathlineto{\pgfqpoint{4.989374in}{0.739656in}}%
\pgfpathlineto{\pgfqpoint{4.989077in}{0.739656in}}%
\pgfpathlineto{\pgfqpoint{4.988779in}{0.739656in}}%
\pgfpathlineto{\pgfqpoint{4.988482in}{0.739656in}}%
\pgfpathlineto{\pgfqpoint{4.988184in}{0.739656in}}%
\pgfpathlineto{\pgfqpoint{4.987887in}{0.739656in}}%
\pgfpathlineto{\pgfqpoint{4.987589in}{0.739656in}}%
\pgfpathlineto{\pgfqpoint{4.987292in}{0.739656in}}%
\pgfpathlineto{\pgfqpoint{4.986994in}{0.739656in}}%
\pgfpathlineto{\pgfqpoint{4.986697in}{0.739656in}}%
\pgfpathlineto{\pgfqpoint{4.986399in}{0.739656in}}%
\pgfpathlineto{\pgfqpoint{4.986102in}{0.739656in}}%
\pgfpathlineto{\pgfqpoint{4.985805in}{0.739656in}}%
\pgfpathlineto{\pgfqpoint{4.985507in}{0.739656in}}%
\pgfpathlineto{\pgfqpoint{4.985210in}{0.739656in}}%
\pgfpathlineto{\pgfqpoint{4.984912in}{0.739656in}}%
\pgfpathlineto{\pgfqpoint{4.984615in}{0.739656in}}%
\pgfpathlineto{\pgfqpoint{4.984317in}{0.739656in}}%
\pgfpathlineto{\pgfqpoint{4.984020in}{0.739656in}}%
\pgfpathlineto{\pgfqpoint{4.983722in}{0.739656in}}%
\pgfpathlineto{\pgfqpoint{4.983425in}{0.739656in}}%
\pgfpathlineto{\pgfqpoint{4.983127in}{0.739656in}}%
\pgfpathlineto{\pgfqpoint{4.982830in}{0.739656in}}%
\pgfpathlineto{\pgfqpoint{4.982532in}{0.739656in}}%
\pgfpathlineto{\pgfqpoint{4.982235in}{0.739656in}}%
\pgfpathlineto{\pgfqpoint{4.981937in}{0.739656in}}%
\pgfpathlineto{\pgfqpoint{4.981640in}{0.739656in}}%
\pgfpathlineto{\pgfqpoint{4.981342in}{0.739656in}}%
\pgfpathlineto{\pgfqpoint{4.981045in}{0.739656in}}%
\pgfpathlineto{\pgfqpoint{4.980747in}{0.739656in}}%
\pgfpathlineto{\pgfqpoint{4.980450in}{0.739656in}}%
\pgfpathlineto{\pgfqpoint{4.980152in}{0.739656in}}%
\pgfpathlineto{\pgfqpoint{4.979855in}{0.739656in}}%
\pgfpathlineto{\pgfqpoint{4.979558in}{0.739656in}}%
\pgfpathlineto{\pgfqpoint{4.979260in}{0.739656in}}%
\pgfpathlineto{\pgfqpoint{4.978963in}{0.739656in}}%
\pgfpathlineto{\pgfqpoint{4.978665in}{0.739656in}}%
\pgfpathlineto{\pgfqpoint{4.978368in}{0.739656in}}%
\pgfpathlineto{\pgfqpoint{4.978070in}{0.739656in}}%
\pgfpathlineto{\pgfqpoint{4.977773in}{0.739656in}}%
\pgfpathlineto{\pgfqpoint{4.977475in}{0.739656in}}%
\pgfpathlineto{\pgfqpoint{4.977178in}{0.739656in}}%
\pgfpathlineto{\pgfqpoint{4.976880in}{0.739656in}}%
\pgfpathlineto{\pgfqpoint{4.976583in}{0.739656in}}%
\pgfpathlineto{\pgfqpoint{4.976285in}{0.739656in}}%
\pgfpathlineto{\pgfqpoint{4.975988in}{0.739656in}}%
\pgfpathlineto{\pgfqpoint{4.975690in}{0.739656in}}%
\pgfpathlineto{\pgfqpoint{4.975393in}{0.739656in}}%
\pgfpathlineto{\pgfqpoint{4.975095in}{0.739656in}}%
\pgfpathlineto{\pgfqpoint{4.974798in}{0.739656in}}%
\pgfpathlineto{\pgfqpoint{4.974500in}{0.739656in}}%
\pgfpathlineto{\pgfqpoint{4.974203in}{0.739656in}}%
\pgfpathlineto{\pgfqpoint{4.973905in}{0.739656in}}%
\pgfpathlineto{\pgfqpoint{4.973608in}{0.739656in}}%
\pgfpathlineto{\pgfqpoint{4.973310in}{0.739656in}}%
\pgfpathlineto{\pgfqpoint{4.973013in}{0.739656in}}%
\pgfpathlineto{\pgfqpoint{4.972716in}{0.739656in}}%
\pgfpathlineto{\pgfqpoint{4.972418in}{0.739656in}}%
\pgfpathlineto{\pgfqpoint{4.972121in}{0.739656in}}%
\pgfpathlineto{\pgfqpoint{4.971823in}{0.739656in}}%
\pgfpathlineto{\pgfqpoint{4.971526in}{0.739656in}}%
\pgfpathlineto{\pgfqpoint{4.971228in}{0.739656in}}%
\pgfpathlineto{\pgfqpoint{4.970931in}{0.739656in}}%
\pgfpathlineto{\pgfqpoint{4.970633in}{0.739656in}}%
\pgfpathlineto{\pgfqpoint{4.970336in}{0.739656in}}%
\pgfpathlineto{\pgfqpoint{4.970038in}{0.739656in}}%
\pgfpathlineto{\pgfqpoint{4.969741in}{0.739656in}}%
\pgfpathlineto{\pgfqpoint{4.969443in}{0.739656in}}%
\pgfpathlineto{\pgfqpoint{4.969146in}{0.739656in}}%
\pgfpathlineto{\pgfqpoint{4.968848in}{0.739656in}}%
\pgfpathlineto{\pgfqpoint{4.968551in}{0.739656in}}%
\pgfpathlineto{\pgfqpoint{4.968253in}{0.739656in}}%
\pgfpathlineto{\pgfqpoint{4.967956in}{0.739656in}}%
\pgfpathlineto{\pgfqpoint{4.967658in}{0.739656in}}%
\pgfpathlineto{\pgfqpoint{4.967361in}{0.739656in}}%
\pgfpathlineto{\pgfqpoint{4.967063in}{0.739656in}}%
\pgfpathlineto{\pgfqpoint{4.966766in}{0.739656in}}%
\pgfpathlineto{\pgfqpoint{4.966468in}{0.739656in}}%
\pgfpathlineto{\pgfqpoint{4.966171in}{0.739656in}}%
\pgfpathlineto{\pgfqpoint{4.965874in}{0.739656in}}%
\pgfpathlineto{\pgfqpoint{4.965576in}{0.739656in}}%
\pgfpathlineto{\pgfqpoint{4.965279in}{0.739656in}}%
\pgfpathlineto{\pgfqpoint{4.964981in}{0.739656in}}%
\pgfpathlineto{\pgfqpoint{4.964684in}{0.739656in}}%
\pgfpathlineto{\pgfqpoint{4.964386in}{0.739656in}}%
\pgfpathlineto{\pgfqpoint{4.964089in}{0.739656in}}%
\pgfpathlineto{\pgfqpoint{4.963791in}{0.739656in}}%
\pgfpathlineto{\pgfqpoint{4.963494in}{0.739656in}}%
\pgfpathlineto{\pgfqpoint{4.963196in}{0.739656in}}%
\pgfpathlineto{\pgfqpoint{4.962899in}{0.739656in}}%
\pgfpathlineto{\pgfqpoint{4.962601in}{0.739656in}}%
\pgfpathlineto{\pgfqpoint{4.962304in}{0.739656in}}%
\pgfpathlineto{\pgfqpoint{4.962006in}{0.739656in}}%
\pgfpathlineto{\pgfqpoint{4.961709in}{0.739656in}}%
\pgfpathlineto{\pgfqpoint{4.961411in}{0.739656in}}%
\pgfpathlineto{\pgfqpoint{4.961114in}{0.739656in}}%
\pgfpathlineto{\pgfqpoint{4.960816in}{0.739656in}}%
\pgfpathlineto{\pgfqpoint{4.960519in}{0.739656in}}%
\pgfpathlineto{\pgfqpoint{4.960221in}{0.739656in}}%
\pgfpathlineto{\pgfqpoint{4.959924in}{0.739656in}}%
\pgfpathlineto{\pgfqpoint{4.959627in}{0.739656in}}%
\pgfpathlineto{\pgfqpoint{4.959329in}{0.739656in}}%
\pgfpathlineto{\pgfqpoint{4.959032in}{0.739656in}}%
\pgfpathlineto{\pgfqpoint{4.958734in}{0.739656in}}%
\pgfpathlineto{\pgfqpoint{4.958437in}{0.739656in}}%
\pgfpathlineto{\pgfqpoint{4.958139in}{0.739656in}}%
\pgfpathlineto{\pgfqpoint{4.957842in}{0.739656in}}%
\pgfpathlineto{\pgfqpoint{4.957544in}{0.739656in}}%
\pgfpathlineto{\pgfqpoint{4.957247in}{0.739656in}}%
\pgfpathlineto{\pgfqpoint{4.956949in}{0.739656in}}%
\pgfpathlineto{\pgfqpoint{4.956652in}{0.739656in}}%
\pgfpathlineto{\pgfqpoint{4.956354in}{0.739656in}}%
\pgfpathlineto{\pgfqpoint{4.956057in}{0.739656in}}%
\pgfpathlineto{\pgfqpoint{4.955759in}{0.739656in}}%
\pgfpathlineto{\pgfqpoint{4.955462in}{0.739656in}}%
\pgfpathlineto{\pgfqpoint{4.955164in}{0.739656in}}%
\pgfpathlineto{\pgfqpoint{4.954867in}{0.739656in}}%
\pgfpathlineto{\pgfqpoint{4.954569in}{0.739656in}}%
\pgfpathlineto{\pgfqpoint{4.954272in}{0.739656in}}%
\pgfpathlineto{\pgfqpoint{4.953974in}{0.739656in}}%
\pgfpathlineto{\pgfqpoint{4.953677in}{0.739656in}}%
\pgfpathlineto{\pgfqpoint{4.953379in}{0.739656in}}%
\pgfpathlineto{\pgfqpoint{4.953082in}{0.739656in}}%
\pgfpathlineto{\pgfqpoint{4.952785in}{0.739656in}}%
\pgfpathlineto{\pgfqpoint{4.952487in}{0.739656in}}%
\pgfpathlineto{\pgfqpoint{4.952190in}{0.739656in}}%
\pgfpathlineto{\pgfqpoint{4.951892in}{0.739656in}}%
\pgfpathlineto{\pgfqpoint{4.951595in}{0.739656in}}%
\pgfpathlineto{\pgfqpoint{4.951297in}{0.739656in}}%
\pgfpathlineto{\pgfqpoint{4.951000in}{0.739656in}}%
\pgfpathlineto{\pgfqpoint{4.950702in}{0.739656in}}%
\pgfpathlineto{\pgfqpoint{4.950405in}{0.739656in}}%
\pgfpathlineto{\pgfqpoint{4.950107in}{0.739656in}}%
\pgfpathlineto{\pgfqpoint{4.949810in}{0.739656in}}%
\pgfpathlineto{\pgfqpoint{4.949512in}{0.739656in}}%
\pgfpathlineto{\pgfqpoint{4.949215in}{0.739656in}}%
\pgfpathlineto{\pgfqpoint{4.948917in}{0.739656in}}%
\pgfpathlineto{\pgfqpoint{4.948620in}{0.739656in}}%
\pgfpathlineto{\pgfqpoint{4.948322in}{0.739656in}}%
\pgfpathlineto{\pgfqpoint{4.948025in}{0.739656in}}%
\pgfpathlineto{\pgfqpoint{4.947727in}{0.739656in}}%
\pgfpathlineto{\pgfqpoint{4.947430in}{0.739656in}}%
\pgfpathlineto{\pgfqpoint{4.947132in}{0.739656in}}%
\pgfpathlineto{\pgfqpoint{4.946835in}{0.739656in}}%
\pgfpathlineto{\pgfqpoint{4.946537in}{0.739656in}}%
\pgfpathlineto{\pgfqpoint{4.946240in}{0.739656in}}%
\pgfpathlineto{\pgfqpoint{4.945943in}{0.739656in}}%
\pgfpathlineto{\pgfqpoint{4.945645in}{0.739656in}}%
\pgfpathlineto{\pgfqpoint{4.945348in}{0.739656in}}%
\pgfpathlineto{\pgfqpoint{4.945050in}{0.739656in}}%
\pgfpathlineto{\pgfqpoint{4.944753in}{0.739656in}}%
\pgfpathlineto{\pgfqpoint{4.944455in}{0.739656in}}%
\pgfpathlineto{\pgfqpoint{4.944158in}{0.739656in}}%
\pgfpathlineto{\pgfqpoint{4.943860in}{0.739656in}}%
\pgfpathlineto{\pgfqpoint{4.943563in}{0.739656in}}%
\pgfpathlineto{\pgfqpoint{4.943265in}{0.739656in}}%
\pgfpathlineto{\pgfqpoint{4.942968in}{0.739656in}}%
\pgfpathlineto{\pgfqpoint{4.942670in}{0.739656in}}%
\pgfpathlineto{\pgfqpoint{4.942373in}{0.739656in}}%
\pgfpathlineto{\pgfqpoint{4.942075in}{0.739656in}}%
\pgfpathlineto{\pgfqpoint{4.941778in}{0.739656in}}%
\pgfpathlineto{\pgfqpoint{4.941480in}{0.739656in}}%
\pgfpathlineto{\pgfqpoint{4.941183in}{0.739656in}}%
\pgfpathlineto{\pgfqpoint{4.940885in}{0.739656in}}%
\pgfpathlineto{\pgfqpoint{4.940588in}{0.739656in}}%
\pgfpathlineto{\pgfqpoint{4.940290in}{0.739656in}}%
\pgfpathlineto{\pgfqpoint{4.939993in}{0.739656in}}%
\pgfpathlineto{\pgfqpoint{4.939696in}{0.739656in}}%
\pgfpathlineto{\pgfqpoint{4.939398in}{0.739656in}}%
\pgfpathlineto{\pgfqpoint{4.939101in}{0.739656in}}%
\pgfpathlineto{\pgfqpoint{4.938803in}{0.739656in}}%
\pgfpathlineto{\pgfqpoint{4.938506in}{0.739656in}}%
\pgfpathlineto{\pgfqpoint{4.938208in}{0.739656in}}%
\pgfpathlineto{\pgfqpoint{4.937911in}{0.739656in}}%
\pgfpathlineto{\pgfqpoint{4.937613in}{0.739656in}}%
\pgfpathlineto{\pgfqpoint{4.937316in}{0.739656in}}%
\pgfpathlineto{\pgfqpoint{4.937018in}{0.739656in}}%
\pgfpathlineto{\pgfqpoint{4.936721in}{0.739656in}}%
\pgfpathlineto{\pgfqpoint{4.936423in}{0.739656in}}%
\pgfpathlineto{\pgfqpoint{4.936126in}{0.739656in}}%
\pgfpathlineto{\pgfqpoint{4.935828in}{0.739656in}}%
\pgfpathlineto{\pgfqpoint{4.935531in}{0.739656in}}%
\pgfpathlineto{\pgfqpoint{4.935233in}{0.739656in}}%
\pgfpathlineto{\pgfqpoint{4.934936in}{0.739656in}}%
\pgfpathlineto{\pgfqpoint{4.934638in}{0.739656in}}%
\pgfpathlineto{\pgfqpoint{4.934341in}{0.739656in}}%
\pgfpathlineto{\pgfqpoint{4.934043in}{0.739656in}}%
\pgfpathlineto{\pgfqpoint{4.933746in}{0.739656in}}%
\pgfpathlineto{\pgfqpoint{4.933448in}{0.739656in}}%
\pgfpathlineto{\pgfqpoint{4.933151in}{0.739656in}}%
\pgfpathlineto{\pgfqpoint{4.932854in}{0.739656in}}%
\pgfpathlineto{\pgfqpoint{4.932556in}{0.739656in}}%
\pgfpathlineto{\pgfqpoint{4.932259in}{0.739656in}}%
\pgfpathlineto{\pgfqpoint{4.931961in}{0.739656in}}%
\pgfpathlineto{\pgfqpoint{4.931664in}{0.739656in}}%
\pgfpathlineto{\pgfqpoint{4.931366in}{0.739656in}}%
\pgfpathlineto{\pgfqpoint{4.931069in}{0.739656in}}%
\pgfpathlineto{\pgfqpoint{4.930771in}{0.739656in}}%
\pgfpathlineto{\pgfqpoint{4.930474in}{0.739656in}}%
\pgfpathlineto{\pgfqpoint{4.930176in}{0.739656in}}%
\pgfpathlineto{\pgfqpoint{4.929879in}{0.739656in}}%
\pgfpathlineto{\pgfqpoint{4.929581in}{0.739656in}}%
\pgfpathlineto{\pgfqpoint{4.929284in}{0.739656in}}%
\pgfpathlineto{\pgfqpoint{4.928986in}{0.739656in}}%
\pgfpathlineto{\pgfqpoint{4.928689in}{0.739656in}}%
\pgfpathlineto{\pgfqpoint{4.928391in}{0.739656in}}%
\pgfpathlineto{\pgfqpoint{4.928094in}{0.739656in}}%
\pgfpathlineto{\pgfqpoint{4.927796in}{0.739656in}}%
\pgfpathlineto{\pgfqpoint{4.927499in}{0.739656in}}%
\pgfpathlineto{\pgfqpoint{4.927201in}{0.739656in}}%
\pgfpathlineto{\pgfqpoint{4.926904in}{0.739656in}}%
\pgfpathlineto{\pgfqpoint{4.926606in}{0.739656in}}%
\pgfpathlineto{\pgfqpoint{4.926309in}{0.739656in}}%
\pgfpathlineto{\pgfqpoint{4.926012in}{0.739656in}}%
\pgfpathlineto{\pgfqpoint{4.925714in}{0.739656in}}%
\pgfpathlineto{\pgfqpoint{4.925417in}{0.739656in}}%
\pgfpathlineto{\pgfqpoint{4.925119in}{0.739656in}}%
\pgfpathlineto{\pgfqpoint{4.924822in}{0.739656in}}%
\pgfpathlineto{\pgfqpoint{4.924524in}{0.739656in}}%
\pgfpathlineto{\pgfqpoint{4.924227in}{0.739656in}}%
\pgfpathlineto{\pgfqpoint{4.923929in}{0.739656in}}%
\pgfpathlineto{\pgfqpoint{4.923632in}{0.739656in}}%
\pgfpathlineto{\pgfqpoint{4.923334in}{0.739656in}}%
\pgfpathlineto{\pgfqpoint{4.923037in}{0.739656in}}%
\pgfpathlineto{\pgfqpoint{4.922739in}{0.739656in}}%
\pgfpathlineto{\pgfqpoint{4.922442in}{0.739656in}}%
\pgfpathlineto{\pgfqpoint{4.922144in}{0.739656in}}%
\pgfpathlineto{\pgfqpoint{4.921847in}{0.739656in}}%
\pgfpathlineto{\pgfqpoint{4.921549in}{0.739656in}}%
\pgfpathlineto{\pgfqpoint{4.921252in}{0.739656in}}%
\pgfpathlineto{\pgfqpoint{4.920954in}{0.739656in}}%
\pgfpathlineto{\pgfqpoint{4.920657in}{0.739656in}}%
\pgfpathlineto{\pgfqpoint{4.920359in}{0.739656in}}%
\pgfpathlineto{\pgfqpoint{4.920062in}{0.739656in}}%
\pgfpathlineto{\pgfqpoint{4.919765in}{0.739656in}}%
\pgfpathlineto{\pgfqpoint{4.919467in}{0.739656in}}%
\pgfpathlineto{\pgfqpoint{4.919170in}{0.739656in}}%
\pgfpathlineto{\pgfqpoint{4.918872in}{0.739656in}}%
\pgfpathlineto{\pgfqpoint{4.918575in}{0.739656in}}%
\pgfpathlineto{\pgfqpoint{4.918277in}{0.739656in}}%
\pgfpathlineto{\pgfqpoint{4.917980in}{0.739656in}}%
\pgfpathlineto{\pgfqpoint{4.917682in}{0.739656in}}%
\pgfpathlineto{\pgfqpoint{4.917385in}{0.739656in}}%
\pgfpathlineto{\pgfqpoint{4.917087in}{0.739656in}}%
\pgfpathlineto{\pgfqpoint{4.916790in}{0.739656in}}%
\pgfpathlineto{\pgfqpoint{4.916492in}{0.739656in}}%
\pgfpathlineto{\pgfqpoint{4.916195in}{0.739656in}}%
\pgfpathlineto{\pgfqpoint{4.915897in}{0.739656in}}%
\pgfpathlineto{\pgfqpoint{4.915600in}{0.739656in}}%
\pgfpathlineto{\pgfqpoint{4.915302in}{0.739656in}}%
\pgfpathlineto{\pgfqpoint{4.915005in}{0.739656in}}%
\pgfpathlineto{\pgfqpoint{4.914707in}{0.739656in}}%
\pgfpathlineto{\pgfqpoint{4.914410in}{0.739656in}}%
\pgfpathlineto{\pgfqpoint{4.914112in}{0.739656in}}%
\pgfpathlineto{\pgfqpoint{4.913815in}{0.739656in}}%
\pgfpathlineto{\pgfqpoint{4.913517in}{0.739656in}}%
\pgfpathlineto{\pgfqpoint{4.913220in}{0.739656in}}%
\pgfpathlineto{\pgfqpoint{4.912923in}{0.739656in}}%
\pgfpathlineto{\pgfqpoint{4.912625in}{0.739656in}}%
\pgfpathlineto{\pgfqpoint{4.912328in}{0.739656in}}%
\pgfpathlineto{\pgfqpoint{4.912030in}{0.739656in}}%
\pgfpathlineto{\pgfqpoint{4.911733in}{0.739656in}}%
\pgfpathlineto{\pgfqpoint{4.911435in}{0.739656in}}%
\pgfpathlineto{\pgfqpoint{4.911138in}{0.739656in}}%
\pgfpathlineto{\pgfqpoint{4.910840in}{0.739656in}}%
\pgfpathlineto{\pgfqpoint{4.910543in}{0.739656in}}%
\pgfpathlineto{\pgfqpoint{4.910245in}{0.739656in}}%
\pgfpathlineto{\pgfqpoint{4.909948in}{0.739656in}}%
\pgfpathlineto{\pgfqpoint{4.909650in}{0.739656in}}%
\pgfpathlineto{\pgfqpoint{4.909353in}{0.739656in}}%
\pgfpathlineto{\pgfqpoint{4.909055in}{0.739656in}}%
\pgfpathlineto{\pgfqpoint{4.908758in}{0.739656in}}%
\pgfpathlineto{\pgfqpoint{4.908460in}{0.739656in}}%
\pgfpathlineto{\pgfqpoint{4.908163in}{0.739656in}}%
\pgfpathlineto{\pgfqpoint{4.907865in}{0.739656in}}%
\pgfpathlineto{\pgfqpoint{4.907568in}{0.739656in}}%
\pgfpathlineto{\pgfqpoint{4.907270in}{0.739656in}}%
\pgfpathlineto{\pgfqpoint{4.906973in}{0.739656in}}%
\pgfpathlineto{\pgfqpoint{4.906675in}{0.739656in}}%
\pgfpathlineto{\pgfqpoint{4.906378in}{0.739656in}}%
\pgfpathlineto{\pgfqpoint{4.906081in}{0.739656in}}%
\pgfpathlineto{\pgfqpoint{4.905783in}{0.739656in}}%
\pgfpathlineto{\pgfqpoint{4.905486in}{0.739656in}}%
\pgfpathlineto{\pgfqpoint{4.905188in}{0.739656in}}%
\pgfpathlineto{\pgfqpoint{4.904891in}{0.739656in}}%
\pgfpathlineto{\pgfqpoint{4.904593in}{0.739656in}}%
\pgfpathlineto{\pgfqpoint{4.904296in}{0.739656in}}%
\pgfpathlineto{\pgfqpoint{4.903998in}{0.739656in}}%
\pgfpathlineto{\pgfqpoint{4.903701in}{0.739656in}}%
\pgfpathlineto{\pgfqpoint{4.903403in}{0.739656in}}%
\pgfpathlineto{\pgfqpoint{4.903106in}{0.739656in}}%
\pgfpathlineto{\pgfqpoint{4.902808in}{0.739656in}}%
\pgfpathlineto{\pgfqpoint{4.902511in}{0.739656in}}%
\pgfpathlineto{\pgfqpoint{4.902213in}{0.739656in}}%
\pgfpathlineto{\pgfqpoint{4.901916in}{0.739656in}}%
\pgfpathlineto{\pgfqpoint{4.901618in}{0.739656in}}%
\pgfpathlineto{\pgfqpoint{4.901321in}{0.739656in}}%
\pgfpathlineto{\pgfqpoint{4.901023in}{0.739656in}}%
\pgfpathlineto{\pgfqpoint{4.900726in}{0.739656in}}%
\pgfpathlineto{\pgfqpoint{4.900428in}{0.739656in}}%
\pgfpathlineto{\pgfqpoint{4.900131in}{0.739656in}}%
\pgfpathlineto{\pgfqpoint{4.899834in}{0.739656in}}%
\pgfpathlineto{\pgfqpoint{4.899536in}{0.739656in}}%
\pgfpathlineto{\pgfqpoint{4.899239in}{0.739656in}}%
\pgfpathlineto{\pgfqpoint{4.898941in}{0.739656in}}%
\pgfpathlineto{\pgfqpoint{4.898644in}{0.739656in}}%
\pgfpathlineto{\pgfqpoint{4.898346in}{0.739656in}}%
\pgfpathlineto{\pgfqpoint{4.898049in}{0.739656in}}%
\pgfpathlineto{\pgfqpoint{4.897751in}{0.739656in}}%
\pgfpathlineto{\pgfqpoint{4.897454in}{0.739656in}}%
\pgfpathlineto{\pgfqpoint{4.897156in}{0.739656in}}%
\pgfpathlineto{\pgfqpoint{4.896859in}{0.739656in}}%
\pgfpathlineto{\pgfqpoint{4.896561in}{0.739656in}}%
\pgfpathlineto{\pgfqpoint{4.896264in}{0.739656in}}%
\pgfpathlineto{\pgfqpoint{4.895966in}{0.739656in}}%
\pgfpathlineto{\pgfqpoint{4.895669in}{0.739656in}}%
\pgfpathlineto{\pgfqpoint{4.895371in}{0.739656in}}%
\pgfpathlineto{\pgfqpoint{4.895074in}{0.739656in}}%
\pgfpathlineto{\pgfqpoint{4.894776in}{0.739656in}}%
\pgfpathlineto{\pgfqpoint{4.894479in}{0.739656in}}%
\pgfpathlineto{\pgfqpoint{4.894181in}{0.739656in}}%
\pgfpathlineto{\pgfqpoint{4.893884in}{0.739656in}}%
\pgfpathlineto{\pgfqpoint{4.893586in}{0.739656in}}%
\pgfpathlineto{\pgfqpoint{4.893289in}{0.739656in}}%
\pgfpathlineto{\pgfqpoint{4.892992in}{0.739656in}}%
\pgfpathlineto{\pgfqpoint{4.892694in}{0.739656in}}%
\pgfpathlineto{\pgfqpoint{4.892397in}{0.739656in}}%
\pgfpathlineto{\pgfqpoint{4.892099in}{0.739656in}}%
\pgfpathlineto{\pgfqpoint{4.891802in}{0.739656in}}%
\pgfpathlineto{\pgfqpoint{4.891504in}{0.739656in}}%
\pgfpathlineto{\pgfqpoint{4.891207in}{0.739656in}}%
\pgfpathlineto{\pgfqpoint{4.890909in}{0.739656in}}%
\pgfpathlineto{\pgfqpoint{4.890612in}{0.739656in}}%
\pgfpathlineto{\pgfqpoint{4.890314in}{0.739656in}}%
\pgfpathlineto{\pgfqpoint{4.890017in}{0.739656in}}%
\pgfpathlineto{\pgfqpoint{4.889719in}{0.739656in}}%
\pgfpathlineto{\pgfqpoint{4.889422in}{0.739656in}}%
\pgfpathlineto{\pgfqpoint{4.889124in}{0.739656in}}%
\pgfpathlineto{\pgfqpoint{4.888827in}{0.739656in}}%
\pgfpathlineto{\pgfqpoint{4.888529in}{0.739656in}}%
\pgfpathlineto{\pgfqpoint{4.888232in}{0.739656in}}%
\pgfpathlineto{\pgfqpoint{4.887934in}{0.739656in}}%
\pgfpathlineto{\pgfqpoint{4.887637in}{0.739656in}}%
\pgfpathlineto{\pgfqpoint{4.887339in}{0.739656in}}%
\pgfpathlineto{\pgfqpoint{4.887042in}{0.739656in}}%
\pgfpathlineto{\pgfqpoint{4.886744in}{0.739656in}}%
\pgfpathlineto{\pgfqpoint{4.886447in}{0.739656in}}%
\pgfpathlineto{\pgfqpoint{4.886150in}{0.739656in}}%
\pgfpathlineto{\pgfqpoint{4.885852in}{0.739656in}}%
\pgfpathlineto{\pgfqpoint{4.885555in}{0.739656in}}%
\pgfpathlineto{\pgfqpoint{4.885257in}{0.739656in}}%
\pgfpathlineto{\pgfqpoint{4.884960in}{0.739656in}}%
\pgfpathlineto{\pgfqpoint{4.884662in}{0.739656in}}%
\pgfpathlineto{\pgfqpoint{4.884365in}{0.739656in}}%
\pgfpathlineto{\pgfqpoint{4.884067in}{0.739656in}}%
\pgfpathlineto{\pgfqpoint{4.883770in}{0.739656in}}%
\pgfpathlineto{\pgfqpoint{4.883472in}{0.739656in}}%
\pgfpathlineto{\pgfqpoint{4.883175in}{0.739656in}}%
\pgfpathlineto{\pgfqpoint{4.882877in}{0.739656in}}%
\pgfpathlineto{\pgfqpoint{4.882580in}{0.739656in}}%
\pgfpathlineto{\pgfqpoint{4.882282in}{0.739656in}}%
\pgfpathlineto{\pgfqpoint{4.881985in}{0.739656in}}%
\pgfpathlineto{\pgfqpoint{4.881687in}{0.739656in}}%
\pgfpathlineto{\pgfqpoint{4.881390in}{0.739656in}}%
\pgfpathlineto{\pgfqpoint{4.881092in}{0.739656in}}%
\pgfpathlineto{\pgfqpoint{4.880795in}{0.739656in}}%
\pgfpathlineto{\pgfqpoint{4.880497in}{0.739656in}}%
\pgfpathlineto{\pgfqpoint{4.880200in}{0.739656in}}%
\pgfpathlineto{\pgfqpoint{4.879903in}{0.739656in}}%
\pgfpathlineto{\pgfqpoint{4.879605in}{0.739656in}}%
\pgfpathlineto{\pgfqpoint{4.879308in}{0.739656in}}%
\pgfpathlineto{\pgfqpoint{4.879010in}{0.739656in}}%
\pgfpathlineto{\pgfqpoint{4.878713in}{0.739656in}}%
\pgfpathlineto{\pgfqpoint{4.878415in}{0.739656in}}%
\pgfpathlineto{\pgfqpoint{4.878118in}{0.739656in}}%
\pgfpathlineto{\pgfqpoint{4.877820in}{0.739656in}}%
\pgfpathlineto{\pgfqpoint{4.877523in}{0.739656in}}%
\pgfpathlineto{\pgfqpoint{4.877225in}{0.739656in}}%
\pgfpathlineto{\pgfqpoint{4.876928in}{0.739656in}}%
\pgfpathlineto{\pgfqpoint{4.876630in}{0.739656in}}%
\pgfpathlineto{\pgfqpoint{4.876333in}{0.739656in}}%
\pgfpathlineto{\pgfqpoint{4.876035in}{0.739656in}}%
\pgfpathlineto{\pgfqpoint{4.875738in}{0.739656in}}%
\pgfpathlineto{\pgfqpoint{4.875440in}{0.739656in}}%
\pgfpathlineto{\pgfqpoint{4.875143in}{0.739656in}}%
\pgfpathlineto{\pgfqpoint{4.874845in}{0.739656in}}%
\pgfpathlineto{\pgfqpoint{4.874548in}{0.739656in}}%
\pgfpathlineto{\pgfqpoint{4.874250in}{0.739656in}}%
\pgfpathlineto{\pgfqpoint{4.873953in}{0.739656in}}%
\pgfpathlineto{\pgfqpoint{4.873655in}{0.739656in}}%
\pgfpathlineto{\pgfqpoint{4.873358in}{0.739656in}}%
\pgfpathlineto{\pgfqpoint{4.873061in}{0.739656in}}%
\pgfpathlineto{\pgfqpoint{4.872763in}{0.739656in}}%
\pgfpathlineto{\pgfqpoint{4.872466in}{0.739656in}}%
\pgfpathlineto{\pgfqpoint{4.872168in}{0.739656in}}%
\pgfpathlineto{\pgfqpoint{4.871871in}{0.739656in}}%
\pgfpathlineto{\pgfqpoint{4.871573in}{0.739656in}}%
\pgfpathlineto{\pgfqpoint{4.871276in}{0.739656in}}%
\pgfpathlineto{\pgfqpoint{4.870978in}{0.739656in}}%
\pgfpathlineto{\pgfqpoint{4.870681in}{0.739656in}}%
\pgfpathlineto{\pgfqpoint{4.870383in}{0.739656in}}%
\pgfpathlineto{\pgfqpoint{4.870086in}{0.739656in}}%
\pgfpathlineto{\pgfqpoint{4.869788in}{0.739656in}}%
\pgfpathlineto{\pgfqpoint{4.869491in}{0.739656in}}%
\pgfpathlineto{\pgfqpoint{4.869193in}{0.739656in}}%
\pgfpathlineto{\pgfqpoint{4.868896in}{0.739656in}}%
\pgfpathlineto{\pgfqpoint{4.868598in}{0.739656in}}%
\pgfpathlineto{\pgfqpoint{4.868301in}{0.739656in}}%
\pgfpathlineto{\pgfqpoint{4.868003in}{0.739656in}}%
\pgfpathlineto{\pgfqpoint{4.867706in}{0.739656in}}%
\pgfpathlineto{\pgfqpoint{4.867408in}{0.739656in}}%
\pgfpathlineto{\pgfqpoint{4.867111in}{0.739656in}}%
\pgfpathlineto{\pgfqpoint{4.866813in}{0.739656in}}%
\pgfpathlineto{\pgfqpoint{4.866516in}{0.739656in}}%
\pgfpathlineto{\pgfqpoint{4.866219in}{0.739656in}}%
\pgfpathlineto{\pgfqpoint{4.865921in}{0.739656in}}%
\pgfpathlineto{\pgfqpoint{4.865624in}{0.739656in}}%
\pgfpathlineto{\pgfqpoint{4.865326in}{0.739656in}}%
\pgfpathlineto{\pgfqpoint{4.865029in}{0.739656in}}%
\pgfpathlineto{\pgfqpoint{4.864731in}{0.739656in}}%
\pgfpathlineto{\pgfqpoint{4.864434in}{0.739656in}}%
\pgfpathlineto{\pgfqpoint{4.864136in}{0.739656in}}%
\pgfpathlineto{\pgfqpoint{4.863839in}{0.739656in}}%
\pgfpathlineto{\pgfqpoint{4.863541in}{0.739656in}}%
\pgfpathlineto{\pgfqpoint{4.863244in}{0.739656in}}%
\pgfpathlineto{\pgfqpoint{4.862946in}{0.739656in}}%
\pgfpathlineto{\pgfqpoint{4.862649in}{0.739656in}}%
\pgfpathlineto{\pgfqpoint{4.862351in}{0.739656in}}%
\pgfpathlineto{\pgfqpoint{4.862054in}{0.739656in}}%
\pgfpathlineto{\pgfqpoint{4.861756in}{0.739656in}}%
\pgfpathlineto{\pgfqpoint{4.861459in}{0.739656in}}%
\pgfpathlineto{\pgfqpoint{4.861161in}{0.739656in}}%
\pgfpathlineto{\pgfqpoint{4.860864in}{0.739656in}}%
\pgfpathlineto{\pgfqpoint{4.860566in}{0.739656in}}%
\pgfpathlineto{\pgfqpoint{4.860269in}{0.739656in}}%
\pgfpathlineto{\pgfqpoint{4.859972in}{0.739656in}}%
\pgfpathlineto{\pgfqpoint{4.859674in}{0.739656in}}%
\pgfpathlineto{\pgfqpoint{4.859377in}{0.739656in}}%
\pgfpathlineto{\pgfqpoint{4.859079in}{0.739656in}}%
\pgfpathlineto{\pgfqpoint{4.858782in}{0.739656in}}%
\pgfpathlineto{\pgfqpoint{4.858484in}{0.739656in}}%
\pgfpathlineto{\pgfqpoint{4.858187in}{0.739656in}}%
\pgfpathlineto{\pgfqpoint{4.857889in}{0.739656in}}%
\pgfpathlineto{\pgfqpoint{4.857592in}{0.739656in}}%
\pgfpathlineto{\pgfqpoint{4.857294in}{0.739656in}}%
\pgfpathlineto{\pgfqpoint{4.856997in}{0.739656in}}%
\pgfpathlineto{\pgfqpoint{4.856699in}{0.739656in}}%
\pgfpathlineto{\pgfqpoint{4.856402in}{0.739656in}}%
\pgfpathlineto{\pgfqpoint{4.856104in}{0.739656in}}%
\pgfpathlineto{\pgfqpoint{4.855807in}{0.739656in}}%
\pgfpathlineto{\pgfqpoint{4.855509in}{0.739656in}}%
\pgfpathlineto{\pgfqpoint{4.855212in}{0.739656in}}%
\pgfpathlineto{\pgfqpoint{4.854914in}{0.739656in}}%
\pgfpathlineto{\pgfqpoint{4.854617in}{0.739656in}}%
\pgfpathlineto{\pgfqpoint{4.854319in}{0.739656in}}%
\pgfpathlineto{\pgfqpoint{4.854022in}{0.739656in}}%
\pgfpathlineto{\pgfqpoint{4.853724in}{0.739656in}}%
\pgfpathlineto{\pgfqpoint{4.853427in}{0.739656in}}%
\pgfpathlineto{\pgfqpoint{4.853130in}{0.739656in}}%
\pgfpathlineto{\pgfqpoint{4.852832in}{0.739656in}}%
\pgfpathlineto{\pgfqpoint{4.852535in}{0.739656in}}%
\pgfpathlineto{\pgfqpoint{4.852237in}{0.739656in}}%
\pgfpathlineto{\pgfqpoint{4.851940in}{0.739656in}}%
\pgfpathlineto{\pgfqpoint{4.851642in}{0.739656in}}%
\pgfpathlineto{\pgfqpoint{4.851345in}{0.739656in}}%
\pgfpathlineto{\pgfqpoint{4.851047in}{0.739656in}}%
\pgfpathlineto{\pgfqpoint{4.850750in}{0.739656in}}%
\pgfpathlineto{\pgfqpoint{4.850452in}{0.739656in}}%
\pgfpathlineto{\pgfqpoint{4.850155in}{0.739656in}}%
\pgfpathlineto{\pgfqpoint{4.849857in}{0.739656in}}%
\pgfpathlineto{\pgfqpoint{4.849560in}{0.739656in}}%
\pgfpathlineto{\pgfqpoint{4.849262in}{0.739656in}}%
\pgfpathlineto{\pgfqpoint{4.848965in}{0.739656in}}%
\pgfpathlineto{\pgfqpoint{4.848667in}{0.739656in}}%
\pgfpathlineto{\pgfqpoint{4.848370in}{0.739656in}}%
\pgfpathlineto{\pgfqpoint{4.848072in}{0.739656in}}%
\pgfpathlineto{\pgfqpoint{4.847775in}{0.739656in}}%
\pgfpathlineto{\pgfqpoint{4.847477in}{0.739656in}}%
\pgfpathlineto{\pgfqpoint{4.847180in}{0.739656in}}%
\pgfpathlineto{\pgfqpoint{4.846882in}{0.739656in}}%
\pgfpathlineto{\pgfqpoint{4.846585in}{0.739656in}}%
\pgfpathlineto{\pgfqpoint{4.846288in}{0.739656in}}%
\pgfpathlineto{\pgfqpoint{4.845990in}{0.739656in}}%
\pgfpathlineto{\pgfqpoint{4.845693in}{0.739656in}}%
\pgfpathlineto{\pgfqpoint{4.845395in}{0.739656in}}%
\pgfpathlineto{\pgfqpoint{4.845098in}{0.739656in}}%
\pgfpathlineto{\pgfqpoint{4.844800in}{0.739656in}}%
\pgfpathlineto{\pgfqpoint{4.844503in}{0.739656in}}%
\pgfpathlineto{\pgfqpoint{4.844205in}{0.739656in}}%
\pgfpathlineto{\pgfqpoint{4.843908in}{0.739656in}}%
\pgfpathlineto{\pgfqpoint{4.843610in}{0.739656in}}%
\pgfpathlineto{\pgfqpoint{4.843313in}{0.739656in}}%
\pgfpathlineto{\pgfqpoint{4.843015in}{0.739656in}}%
\pgfpathlineto{\pgfqpoint{4.842718in}{0.739656in}}%
\pgfpathlineto{\pgfqpoint{4.842420in}{0.739656in}}%
\pgfpathlineto{\pgfqpoint{4.842123in}{0.739656in}}%
\pgfpathlineto{\pgfqpoint{4.841825in}{0.739656in}}%
\pgfpathlineto{\pgfqpoint{4.841528in}{0.739656in}}%
\pgfpathlineto{\pgfqpoint{4.841230in}{0.739656in}}%
\pgfpathlineto{\pgfqpoint{4.840933in}{0.739656in}}%
\pgfpathlineto{\pgfqpoint{4.840635in}{0.739656in}}%
\pgfpathlineto{\pgfqpoint{4.840338in}{0.739656in}}%
\pgfpathlineto{\pgfqpoint{4.840041in}{0.739656in}}%
\pgfpathlineto{\pgfqpoint{4.839743in}{0.739656in}}%
\pgfpathlineto{\pgfqpoint{4.839446in}{0.739656in}}%
\pgfpathlineto{\pgfqpoint{4.839148in}{0.739656in}}%
\pgfpathlineto{\pgfqpoint{4.838851in}{0.739656in}}%
\pgfpathlineto{\pgfqpoint{4.838553in}{0.739656in}}%
\pgfpathlineto{\pgfqpoint{4.838256in}{0.739656in}}%
\pgfpathlineto{\pgfqpoint{4.837958in}{0.739656in}}%
\pgfpathlineto{\pgfqpoint{4.837661in}{0.739656in}}%
\pgfpathlineto{\pgfqpoint{4.837363in}{0.739656in}}%
\pgfpathlineto{\pgfqpoint{4.837066in}{0.739656in}}%
\pgfpathlineto{\pgfqpoint{4.836768in}{0.739656in}}%
\pgfpathlineto{\pgfqpoint{4.836471in}{0.739656in}}%
\pgfpathlineto{\pgfqpoint{4.836173in}{0.739656in}}%
\pgfpathlineto{\pgfqpoint{4.835876in}{0.739656in}}%
\pgfpathlineto{\pgfqpoint{4.835578in}{0.739656in}}%
\pgfpathlineto{\pgfqpoint{4.835281in}{0.739656in}}%
\pgfpathlineto{\pgfqpoint{4.834983in}{0.739656in}}%
\pgfpathlineto{\pgfqpoint{4.834686in}{0.739656in}}%
\pgfpathlineto{\pgfqpoint{4.834388in}{0.739656in}}%
\pgfpathlineto{\pgfqpoint{4.834091in}{0.739656in}}%
\pgfpathlineto{\pgfqpoint{4.833793in}{0.739656in}}%
\pgfpathlineto{\pgfqpoint{4.833496in}{0.739656in}}%
\pgfpathlineto{\pgfqpoint{4.833199in}{0.739656in}}%
\pgfpathlineto{\pgfqpoint{4.832901in}{0.739656in}}%
\pgfpathlineto{\pgfqpoint{4.832604in}{0.739656in}}%
\pgfpathlineto{\pgfqpoint{4.832306in}{0.739656in}}%
\pgfpathlineto{\pgfqpoint{4.832009in}{0.739656in}}%
\pgfpathlineto{\pgfqpoint{4.831711in}{0.739656in}}%
\pgfpathlineto{\pgfqpoint{4.831414in}{0.739656in}}%
\pgfpathlineto{\pgfqpoint{4.831116in}{0.739656in}}%
\pgfpathlineto{\pgfqpoint{4.830819in}{0.739656in}}%
\pgfpathlineto{\pgfqpoint{4.830521in}{0.739656in}}%
\pgfpathlineto{\pgfqpoint{4.830224in}{0.739656in}}%
\pgfpathlineto{\pgfqpoint{4.829926in}{0.739656in}}%
\pgfpathlineto{\pgfqpoint{4.829629in}{0.739656in}}%
\pgfpathlineto{\pgfqpoint{4.829331in}{0.739656in}}%
\pgfpathlineto{\pgfqpoint{4.829034in}{0.739656in}}%
\pgfpathlineto{\pgfqpoint{4.828736in}{0.739656in}}%
\pgfpathlineto{\pgfqpoint{4.828439in}{0.739656in}}%
\pgfpathlineto{\pgfqpoint{4.828141in}{0.739656in}}%
\pgfpathlineto{\pgfqpoint{4.827844in}{0.739656in}}%
\pgfpathlineto{\pgfqpoint{4.827546in}{0.739656in}}%
\pgfpathlineto{\pgfqpoint{4.827249in}{0.739656in}}%
\pgfpathlineto{\pgfqpoint{4.826951in}{0.739656in}}%
\pgfpathlineto{\pgfqpoint{4.826654in}{0.739656in}}%
\pgfpathlineto{\pgfqpoint{4.826357in}{0.739656in}}%
\pgfpathlineto{\pgfqpoint{4.826059in}{0.739656in}}%
\pgfpathlineto{\pgfqpoint{4.825762in}{0.739656in}}%
\pgfpathlineto{\pgfqpoint{4.825464in}{0.739656in}}%
\pgfpathlineto{\pgfqpoint{4.825167in}{0.739656in}}%
\pgfpathlineto{\pgfqpoint{4.824869in}{0.739656in}}%
\pgfpathlineto{\pgfqpoint{4.824572in}{0.739656in}}%
\pgfpathlineto{\pgfqpoint{4.824274in}{0.739656in}}%
\pgfpathlineto{\pgfqpoint{4.823977in}{0.739656in}}%
\pgfpathlineto{\pgfqpoint{4.823679in}{0.739656in}}%
\pgfpathlineto{\pgfqpoint{4.823382in}{0.739656in}}%
\pgfpathlineto{\pgfqpoint{4.823084in}{0.739656in}}%
\pgfpathlineto{\pgfqpoint{4.822787in}{0.739656in}}%
\pgfpathlineto{\pgfqpoint{4.822489in}{0.739656in}}%
\pgfpathlineto{\pgfqpoint{4.822192in}{0.739656in}}%
\pgfpathlineto{\pgfqpoint{4.821894in}{0.739656in}}%
\pgfpathlineto{\pgfqpoint{4.821597in}{0.739656in}}%
\pgfpathlineto{\pgfqpoint{4.821299in}{0.739656in}}%
\pgfpathlineto{\pgfqpoint{4.821002in}{0.739656in}}%
\pgfpathlineto{\pgfqpoint{4.820704in}{0.739656in}}%
\pgfpathlineto{\pgfqpoint{4.820407in}{0.739656in}}%
\pgfpathlineto{\pgfqpoint{4.820109in}{0.739656in}}%
\pgfpathlineto{\pgfqpoint{4.819812in}{0.739656in}}%
\pgfpathlineto{\pgfqpoint{4.819515in}{0.739656in}}%
\pgfpathlineto{\pgfqpoint{4.819217in}{0.739656in}}%
\pgfpathlineto{\pgfqpoint{4.818920in}{0.739656in}}%
\pgfpathlineto{\pgfqpoint{4.818622in}{0.739656in}}%
\pgfpathlineto{\pgfqpoint{4.818325in}{0.739656in}}%
\pgfpathlineto{\pgfqpoint{4.818027in}{0.739656in}}%
\pgfpathlineto{\pgfqpoint{4.817730in}{0.739656in}}%
\pgfpathlineto{\pgfqpoint{4.817432in}{0.739656in}}%
\pgfpathlineto{\pgfqpoint{4.817135in}{0.739656in}}%
\pgfpathlineto{\pgfqpoint{4.816837in}{0.739656in}}%
\pgfpathlineto{\pgfqpoint{4.816540in}{0.739656in}}%
\pgfpathlineto{\pgfqpoint{4.816242in}{0.739656in}}%
\pgfpathlineto{\pgfqpoint{4.815945in}{0.739656in}}%
\pgfpathlineto{\pgfqpoint{4.815647in}{0.739656in}}%
\pgfpathlineto{\pgfqpoint{4.815350in}{0.739656in}}%
\pgfpathlineto{\pgfqpoint{4.815052in}{0.739656in}}%
\pgfpathlineto{\pgfqpoint{4.814755in}{0.739656in}}%
\pgfpathlineto{\pgfqpoint{4.814457in}{0.739656in}}%
\pgfpathlineto{\pgfqpoint{4.814160in}{0.739656in}}%
\pgfpathlineto{\pgfqpoint{4.813862in}{0.739656in}}%
\pgfpathlineto{\pgfqpoint{4.813565in}{0.739656in}}%
\pgfpathlineto{\pgfqpoint{4.813268in}{0.739656in}}%
\pgfpathlineto{\pgfqpoint{4.812970in}{0.739656in}}%
\pgfpathlineto{\pgfqpoint{4.812673in}{0.739656in}}%
\pgfpathlineto{\pgfqpoint{4.812375in}{0.739656in}}%
\pgfpathlineto{\pgfqpoint{4.812078in}{0.739656in}}%
\pgfpathlineto{\pgfqpoint{4.811780in}{0.739656in}}%
\pgfpathlineto{\pgfqpoint{4.811483in}{0.739656in}}%
\pgfpathlineto{\pgfqpoint{4.811185in}{0.739656in}}%
\pgfpathlineto{\pgfqpoint{4.810888in}{0.739656in}}%
\pgfpathlineto{\pgfqpoint{4.810590in}{0.739656in}}%
\pgfpathlineto{\pgfqpoint{4.810293in}{0.739656in}}%
\pgfpathlineto{\pgfqpoint{4.809995in}{0.739656in}}%
\pgfpathlineto{\pgfqpoint{4.809698in}{0.739656in}}%
\pgfpathlineto{\pgfqpoint{4.809400in}{0.739656in}}%
\pgfpathlineto{\pgfqpoint{4.809103in}{0.739656in}}%
\pgfpathlineto{\pgfqpoint{4.808805in}{0.739656in}}%
\pgfpathlineto{\pgfqpoint{4.808508in}{0.739656in}}%
\pgfpathlineto{\pgfqpoint{4.808210in}{0.739656in}}%
\pgfpathlineto{\pgfqpoint{4.807913in}{0.739656in}}%
\pgfpathlineto{\pgfqpoint{4.807615in}{0.739656in}}%
\pgfpathlineto{\pgfqpoint{4.807318in}{0.739656in}}%
\pgfpathlineto{\pgfqpoint{4.807020in}{0.739656in}}%
\pgfpathlineto{\pgfqpoint{4.806723in}{0.739656in}}%
\pgfpathlineto{\pgfqpoint{4.806426in}{0.739656in}}%
\pgfpathlineto{\pgfqpoint{4.806128in}{0.739656in}}%
\pgfpathlineto{\pgfqpoint{4.805831in}{0.739656in}}%
\pgfpathlineto{\pgfqpoint{4.805533in}{0.739656in}}%
\pgfpathlineto{\pgfqpoint{4.805236in}{0.739656in}}%
\pgfpathlineto{\pgfqpoint{4.804938in}{0.739656in}}%
\pgfpathlineto{\pgfqpoint{4.804641in}{0.739656in}}%
\pgfpathlineto{\pgfqpoint{4.804343in}{0.739656in}}%
\pgfpathlineto{\pgfqpoint{4.804046in}{0.739656in}}%
\pgfpathlineto{\pgfqpoint{4.803748in}{0.739656in}}%
\pgfpathlineto{\pgfqpoint{4.803451in}{0.739656in}}%
\pgfpathlineto{\pgfqpoint{4.803153in}{0.739656in}}%
\pgfpathlineto{\pgfqpoint{4.802856in}{0.739656in}}%
\pgfpathlineto{\pgfqpoint{4.802558in}{0.739656in}}%
\pgfpathlineto{\pgfqpoint{4.802261in}{0.739656in}}%
\pgfpathlineto{\pgfqpoint{4.801963in}{0.739656in}}%
\pgfpathlineto{\pgfqpoint{4.801666in}{0.739656in}}%
\pgfpathlineto{\pgfqpoint{4.801368in}{0.739656in}}%
\pgfpathlineto{\pgfqpoint{4.801071in}{0.739656in}}%
\pgfpathlineto{\pgfqpoint{4.800773in}{0.739656in}}%
\pgfpathlineto{\pgfqpoint{4.800476in}{0.739656in}}%
\pgfpathlineto{\pgfqpoint{4.800178in}{0.739656in}}%
\pgfpathlineto{\pgfqpoint{4.799881in}{0.739656in}}%
\pgfpathlineto{\pgfqpoint{4.799584in}{0.739656in}}%
\pgfpathlineto{\pgfqpoint{4.799286in}{0.739656in}}%
\pgfpathlineto{\pgfqpoint{4.798989in}{0.739656in}}%
\pgfpathlineto{\pgfqpoint{4.798691in}{0.739656in}}%
\pgfpathlineto{\pgfqpoint{4.798394in}{0.739656in}}%
\pgfpathlineto{\pgfqpoint{4.798096in}{0.739656in}}%
\pgfpathlineto{\pgfqpoint{4.797799in}{0.739656in}}%
\pgfpathlineto{\pgfqpoint{4.797501in}{0.739656in}}%
\pgfpathlineto{\pgfqpoint{4.797204in}{0.739656in}}%
\pgfpathlineto{\pgfqpoint{4.796906in}{0.739656in}}%
\pgfpathlineto{\pgfqpoint{4.796609in}{0.739656in}}%
\pgfpathlineto{\pgfqpoint{4.796311in}{0.739656in}}%
\pgfpathlineto{\pgfqpoint{4.796014in}{0.739656in}}%
\pgfpathlineto{\pgfqpoint{4.795716in}{0.739656in}}%
\pgfpathlineto{\pgfqpoint{4.795419in}{0.739656in}}%
\pgfpathlineto{\pgfqpoint{4.795121in}{0.739656in}}%
\pgfpathlineto{\pgfqpoint{4.794824in}{0.739656in}}%
\pgfpathlineto{\pgfqpoint{4.794526in}{0.739656in}}%
\pgfpathlineto{\pgfqpoint{4.794229in}{0.739656in}}%
\pgfpathlineto{\pgfqpoint{4.793931in}{0.739656in}}%
\pgfpathlineto{\pgfqpoint{4.793634in}{0.739656in}}%
\pgfpathlineto{\pgfqpoint{4.793337in}{0.739656in}}%
\pgfpathlineto{\pgfqpoint{4.793039in}{0.739656in}}%
\pgfpathlineto{\pgfqpoint{4.792742in}{0.739656in}}%
\pgfpathlineto{\pgfqpoint{4.792444in}{0.739656in}}%
\pgfpathlineto{\pgfqpoint{4.792147in}{0.739656in}}%
\pgfpathlineto{\pgfqpoint{4.791849in}{0.739656in}}%
\pgfpathlineto{\pgfqpoint{4.791552in}{0.739656in}}%
\pgfpathlineto{\pgfqpoint{4.791254in}{0.739656in}}%
\pgfpathlineto{\pgfqpoint{4.790957in}{0.739656in}}%
\pgfpathlineto{\pgfqpoint{4.790659in}{0.739656in}}%
\pgfpathlineto{\pgfqpoint{4.790362in}{0.739656in}}%
\pgfpathlineto{\pgfqpoint{4.790064in}{0.739656in}}%
\pgfpathlineto{\pgfqpoint{4.789767in}{0.739656in}}%
\pgfpathlineto{\pgfqpoint{4.789469in}{0.739656in}}%
\pgfpathlineto{\pgfqpoint{4.789172in}{0.739656in}}%
\pgfpathlineto{\pgfqpoint{4.788874in}{0.739656in}}%
\pgfpathlineto{\pgfqpoint{4.788577in}{0.739656in}}%
\pgfpathlineto{\pgfqpoint{4.788279in}{0.739656in}}%
\pgfpathlineto{\pgfqpoint{4.787982in}{0.739656in}}%
\pgfpathlineto{\pgfqpoint{4.787684in}{0.739656in}}%
\pgfpathlineto{\pgfqpoint{4.787387in}{0.739656in}}%
\pgfpathlineto{\pgfqpoint{4.787089in}{0.739656in}}%
\pgfpathlineto{\pgfqpoint{4.786792in}{0.739656in}}%
\pgfpathlineto{\pgfqpoint{4.786495in}{0.739656in}}%
\pgfpathlineto{\pgfqpoint{4.786197in}{0.739656in}}%
\pgfpathlineto{\pgfqpoint{4.785900in}{0.739656in}}%
\pgfpathlineto{\pgfqpoint{4.785602in}{0.739656in}}%
\pgfpathlineto{\pgfqpoint{4.785305in}{0.739656in}}%
\pgfpathlineto{\pgfqpoint{4.785007in}{0.739656in}}%
\pgfpathlineto{\pgfqpoint{4.784710in}{0.739656in}}%
\pgfpathlineto{\pgfqpoint{4.784412in}{0.739656in}}%
\pgfpathlineto{\pgfqpoint{4.784115in}{0.739656in}}%
\pgfpathlineto{\pgfqpoint{4.783817in}{0.739656in}}%
\pgfpathlineto{\pgfqpoint{4.783520in}{0.739656in}}%
\pgfpathlineto{\pgfqpoint{4.783222in}{0.739656in}}%
\pgfpathlineto{\pgfqpoint{4.782925in}{0.739656in}}%
\pgfpathlineto{\pgfqpoint{4.782627in}{0.739656in}}%
\pgfpathlineto{\pgfqpoint{4.782330in}{0.739656in}}%
\pgfpathlineto{\pgfqpoint{4.782032in}{0.739656in}}%
\pgfpathlineto{\pgfqpoint{4.781735in}{0.739656in}}%
\pgfpathlineto{\pgfqpoint{4.781437in}{0.739656in}}%
\pgfpathlineto{\pgfqpoint{4.781140in}{0.739656in}}%
\pgfpathlineto{\pgfqpoint{4.780842in}{0.739656in}}%
\pgfpathlineto{\pgfqpoint{4.780545in}{0.739656in}}%
\pgfpathlineto{\pgfqpoint{4.780247in}{0.739656in}}%
\pgfpathlineto{\pgfqpoint{4.779950in}{0.739656in}}%
\pgfpathlineto{\pgfqpoint{4.779653in}{0.739656in}}%
\pgfpathlineto{\pgfqpoint{4.779355in}{0.739656in}}%
\pgfpathlineto{\pgfqpoint{4.779058in}{0.739656in}}%
\pgfpathlineto{\pgfqpoint{4.778760in}{0.739656in}}%
\pgfpathlineto{\pgfqpoint{4.778463in}{0.739656in}}%
\pgfpathlineto{\pgfqpoint{4.778165in}{0.739656in}}%
\pgfpathlineto{\pgfqpoint{4.777868in}{0.739656in}}%
\pgfpathlineto{\pgfqpoint{4.777570in}{0.739656in}}%
\pgfpathlineto{\pgfqpoint{4.777273in}{0.739656in}}%
\pgfpathlineto{\pgfqpoint{4.776975in}{0.739656in}}%
\pgfpathlineto{\pgfqpoint{4.776678in}{0.739656in}}%
\pgfpathlineto{\pgfqpoint{4.776380in}{0.739656in}}%
\pgfpathlineto{\pgfqpoint{4.776083in}{0.739656in}}%
\pgfpathlineto{\pgfqpoint{4.775785in}{0.739656in}}%
\pgfpathlineto{\pgfqpoint{4.775488in}{0.739656in}}%
\pgfpathlineto{\pgfqpoint{4.775190in}{0.739656in}}%
\pgfpathlineto{\pgfqpoint{4.774893in}{0.739656in}}%
\pgfpathlineto{\pgfqpoint{4.774595in}{0.739656in}}%
\pgfpathlineto{\pgfqpoint{4.774298in}{0.739656in}}%
\pgfpathlineto{\pgfqpoint{4.774000in}{0.739656in}}%
\pgfpathlineto{\pgfqpoint{4.773703in}{0.739656in}}%
\pgfpathlineto{\pgfqpoint{4.773406in}{0.739656in}}%
\pgfpathlineto{\pgfqpoint{4.773108in}{0.739656in}}%
\pgfpathlineto{\pgfqpoint{4.772811in}{0.739656in}}%
\pgfpathlineto{\pgfqpoint{4.772513in}{0.739656in}}%
\pgfpathlineto{\pgfqpoint{4.772216in}{0.739656in}}%
\pgfpathlineto{\pgfqpoint{4.771918in}{0.739656in}}%
\pgfpathlineto{\pgfqpoint{4.771621in}{0.739656in}}%
\pgfpathlineto{\pgfqpoint{4.771323in}{0.739656in}}%
\pgfpathlineto{\pgfqpoint{4.771026in}{0.739656in}}%
\pgfpathlineto{\pgfqpoint{4.770728in}{0.739656in}}%
\pgfpathlineto{\pgfqpoint{4.770431in}{0.739656in}}%
\pgfpathlineto{\pgfqpoint{4.770133in}{0.739656in}}%
\pgfpathlineto{\pgfqpoint{4.769836in}{0.739656in}}%
\pgfpathlineto{\pgfqpoint{4.769538in}{0.739656in}}%
\pgfpathlineto{\pgfqpoint{4.769241in}{0.739656in}}%
\pgfpathlineto{\pgfqpoint{4.768943in}{0.739656in}}%
\pgfpathlineto{\pgfqpoint{4.768646in}{0.739656in}}%
\pgfpathlineto{\pgfqpoint{4.768348in}{0.739656in}}%
\pgfpathlineto{\pgfqpoint{4.768051in}{0.739656in}}%
\pgfpathlineto{\pgfqpoint{4.767753in}{0.739656in}}%
\pgfpathlineto{\pgfqpoint{4.767456in}{0.739656in}}%
\pgfpathlineto{\pgfqpoint{4.767158in}{0.739656in}}%
\pgfpathlineto{\pgfqpoint{4.766861in}{0.739656in}}%
\pgfpathlineto{\pgfqpoint{4.766564in}{0.739656in}}%
\pgfpathlineto{\pgfqpoint{4.766266in}{0.739656in}}%
\pgfpathlineto{\pgfqpoint{4.765969in}{0.739656in}}%
\pgfpathlineto{\pgfqpoint{4.765671in}{0.739656in}}%
\pgfpathlineto{\pgfqpoint{4.765374in}{0.739656in}}%
\pgfpathlineto{\pgfqpoint{4.765076in}{0.739656in}}%
\pgfpathlineto{\pgfqpoint{4.764779in}{0.739656in}}%
\pgfpathlineto{\pgfqpoint{4.764481in}{0.739656in}}%
\pgfpathlineto{\pgfqpoint{4.764184in}{0.739656in}}%
\pgfpathlineto{\pgfqpoint{4.763886in}{0.739656in}}%
\pgfpathlineto{\pgfqpoint{4.763589in}{0.739656in}}%
\pgfpathlineto{\pgfqpoint{4.763291in}{0.739656in}}%
\pgfpathlineto{\pgfqpoint{4.762994in}{0.739656in}}%
\pgfpathlineto{\pgfqpoint{4.762696in}{0.739656in}}%
\pgfpathlineto{\pgfqpoint{4.762399in}{0.739656in}}%
\pgfpathlineto{\pgfqpoint{4.762101in}{0.739656in}}%
\pgfpathlineto{\pgfqpoint{4.761804in}{0.739656in}}%
\pgfpathlineto{\pgfqpoint{4.761506in}{0.739656in}}%
\pgfpathlineto{\pgfqpoint{4.761209in}{0.739656in}}%
\pgfpathlineto{\pgfqpoint{4.760911in}{0.739656in}}%
\pgfpathlineto{\pgfqpoint{4.760614in}{0.739656in}}%
\pgfpathlineto{\pgfqpoint{4.760316in}{0.739656in}}%
\pgfpathlineto{\pgfqpoint{4.760019in}{0.739656in}}%
\pgfpathlineto{\pgfqpoint{4.759722in}{0.739656in}}%
\pgfpathlineto{\pgfqpoint{4.759424in}{0.739656in}}%
\pgfpathlineto{\pgfqpoint{4.759127in}{0.739656in}}%
\pgfpathlineto{\pgfqpoint{4.758829in}{0.739656in}}%
\pgfpathlineto{\pgfqpoint{4.758532in}{0.739656in}}%
\pgfpathlineto{\pgfqpoint{4.758234in}{0.739656in}}%
\pgfpathlineto{\pgfqpoint{4.757937in}{0.739656in}}%
\pgfpathlineto{\pgfqpoint{4.757639in}{0.739656in}}%
\pgfpathlineto{\pgfqpoint{4.757342in}{0.739656in}}%
\pgfpathlineto{\pgfqpoint{4.757044in}{0.739656in}}%
\pgfpathlineto{\pgfqpoint{4.756747in}{0.739656in}}%
\pgfpathlineto{\pgfqpoint{4.756449in}{0.739656in}}%
\pgfpathlineto{\pgfqpoint{4.756152in}{0.739656in}}%
\pgfpathlineto{\pgfqpoint{4.755854in}{0.739656in}}%
\pgfpathlineto{\pgfqpoint{4.755557in}{0.739656in}}%
\pgfpathlineto{\pgfqpoint{4.755259in}{0.739656in}}%
\pgfpathlineto{\pgfqpoint{4.754962in}{0.739656in}}%
\pgfpathlineto{\pgfqpoint{4.754664in}{0.739656in}}%
\pgfpathlineto{\pgfqpoint{4.754367in}{0.739656in}}%
\pgfpathlineto{\pgfqpoint{4.754069in}{0.739656in}}%
\pgfpathlineto{\pgfqpoint{4.753772in}{0.739656in}}%
\pgfpathlineto{\pgfqpoint{4.753475in}{0.739656in}}%
\pgfpathlineto{\pgfqpoint{4.753177in}{0.739656in}}%
\pgfpathlineto{\pgfqpoint{4.752880in}{0.739656in}}%
\pgfpathlineto{\pgfqpoint{4.752582in}{0.739656in}}%
\pgfpathlineto{\pgfqpoint{4.752285in}{0.739656in}}%
\pgfpathlineto{\pgfqpoint{4.751987in}{0.739656in}}%
\pgfpathlineto{\pgfqpoint{4.751690in}{0.739656in}}%
\pgfpathlineto{\pgfqpoint{4.751392in}{0.739656in}}%
\pgfpathlineto{\pgfqpoint{4.751095in}{0.739656in}}%
\pgfpathlineto{\pgfqpoint{4.750797in}{0.739656in}}%
\pgfpathlineto{\pgfqpoint{4.750500in}{0.739656in}}%
\pgfpathlineto{\pgfqpoint{4.750202in}{0.739656in}}%
\pgfpathlineto{\pgfqpoint{4.749905in}{0.739656in}}%
\pgfpathlineto{\pgfqpoint{4.749607in}{0.739656in}}%
\pgfpathlineto{\pgfqpoint{4.749310in}{0.739656in}}%
\pgfpathlineto{\pgfqpoint{4.749012in}{0.739656in}}%
\pgfpathlineto{\pgfqpoint{4.748715in}{0.739656in}}%
\pgfpathlineto{\pgfqpoint{4.748417in}{0.739656in}}%
\pgfpathlineto{\pgfqpoint{4.748120in}{0.739656in}}%
\pgfpathlineto{\pgfqpoint{4.747822in}{0.739656in}}%
\pgfpathlineto{\pgfqpoint{4.747525in}{0.739656in}}%
\pgfpathlineto{\pgfqpoint{4.747227in}{0.739656in}}%
\pgfpathlineto{\pgfqpoint{4.746930in}{0.739656in}}%
\pgfpathlineto{\pgfqpoint{4.746633in}{0.739656in}}%
\pgfpathlineto{\pgfqpoint{4.746335in}{0.739656in}}%
\pgfpathlineto{\pgfqpoint{4.746038in}{0.739656in}}%
\pgfpathlineto{\pgfqpoint{4.745740in}{0.739656in}}%
\pgfpathlineto{\pgfqpoint{4.745443in}{0.739656in}}%
\pgfpathlineto{\pgfqpoint{4.745145in}{0.739656in}}%
\pgfpathlineto{\pgfqpoint{4.744848in}{0.739656in}}%
\pgfpathlineto{\pgfqpoint{4.744550in}{0.739656in}}%
\pgfpathlineto{\pgfqpoint{4.744253in}{0.739656in}}%
\pgfpathlineto{\pgfqpoint{4.743955in}{0.739656in}}%
\pgfpathlineto{\pgfqpoint{4.743658in}{0.739656in}}%
\pgfpathlineto{\pgfqpoint{4.743360in}{0.739656in}}%
\pgfpathlineto{\pgfqpoint{4.743063in}{0.739656in}}%
\pgfpathlineto{\pgfqpoint{4.742765in}{0.739656in}}%
\pgfpathlineto{\pgfqpoint{4.742468in}{0.739656in}}%
\pgfpathlineto{\pgfqpoint{4.742170in}{0.739656in}}%
\pgfpathlineto{\pgfqpoint{4.741873in}{0.739656in}}%
\pgfpathlineto{\pgfqpoint{4.741575in}{0.739656in}}%
\pgfpathlineto{\pgfqpoint{4.741278in}{0.739656in}}%
\pgfpathlineto{\pgfqpoint{4.740980in}{0.739656in}}%
\pgfpathlineto{\pgfqpoint{4.740683in}{0.739656in}}%
\pgfpathlineto{\pgfqpoint{4.740385in}{0.739656in}}%
\pgfpathlineto{\pgfqpoint{4.740088in}{0.739656in}}%
\pgfpathlineto{\pgfqpoint{4.739791in}{0.739656in}}%
\pgfpathlineto{\pgfqpoint{4.739493in}{0.739656in}}%
\pgfpathlineto{\pgfqpoint{4.739196in}{0.739656in}}%
\pgfpathlineto{\pgfqpoint{4.738898in}{0.739656in}}%
\pgfpathlineto{\pgfqpoint{4.738601in}{0.739656in}}%
\pgfpathlineto{\pgfqpoint{4.738303in}{0.739656in}}%
\pgfpathlineto{\pgfqpoint{4.738006in}{0.739656in}}%
\pgfpathlineto{\pgfqpoint{4.737708in}{0.739656in}}%
\pgfpathlineto{\pgfqpoint{4.737411in}{0.739656in}}%
\pgfpathlineto{\pgfqpoint{4.737113in}{0.739656in}}%
\pgfpathlineto{\pgfqpoint{4.736816in}{0.739656in}}%
\pgfpathlineto{\pgfqpoint{4.736518in}{0.739656in}}%
\pgfpathlineto{\pgfqpoint{4.736221in}{0.739656in}}%
\pgfpathlineto{\pgfqpoint{4.735923in}{0.739656in}}%
\pgfpathlineto{\pgfqpoint{4.735626in}{0.739656in}}%
\pgfpathlineto{\pgfqpoint{4.735328in}{0.739656in}}%
\pgfpathlineto{\pgfqpoint{4.735031in}{0.739656in}}%
\pgfpathlineto{\pgfqpoint{4.734733in}{0.739656in}}%
\pgfpathlineto{\pgfqpoint{4.734436in}{0.739656in}}%
\pgfpathlineto{\pgfqpoint{4.734138in}{0.739656in}}%
\pgfpathlineto{\pgfqpoint{4.733841in}{0.739656in}}%
\pgfpathlineto{\pgfqpoint{4.733544in}{0.739656in}}%
\pgfpathlineto{\pgfqpoint{4.733246in}{0.739656in}}%
\pgfpathlineto{\pgfqpoint{4.732949in}{0.739656in}}%
\pgfpathlineto{\pgfqpoint{4.732651in}{0.739656in}}%
\pgfpathlineto{\pgfqpoint{4.732354in}{0.739656in}}%
\pgfpathlineto{\pgfqpoint{4.732056in}{0.739656in}}%
\pgfpathlineto{\pgfqpoint{4.731759in}{0.739656in}}%
\pgfpathlineto{\pgfqpoint{4.731461in}{0.739656in}}%
\pgfpathlineto{\pgfqpoint{4.731164in}{0.739656in}}%
\pgfpathlineto{\pgfqpoint{4.730866in}{0.739656in}}%
\pgfpathlineto{\pgfqpoint{4.730569in}{0.739656in}}%
\pgfpathlineto{\pgfqpoint{4.730271in}{0.739656in}}%
\pgfpathlineto{\pgfqpoint{4.729974in}{0.739656in}}%
\pgfpathlineto{\pgfqpoint{4.729676in}{0.739656in}}%
\pgfpathlineto{\pgfqpoint{4.729379in}{0.739656in}}%
\pgfpathlineto{\pgfqpoint{4.729081in}{0.739656in}}%
\pgfpathlineto{\pgfqpoint{4.728784in}{0.739656in}}%
\pgfpathlineto{\pgfqpoint{4.728486in}{0.739656in}}%
\pgfpathlineto{\pgfqpoint{4.728189in}{0.739656in}}%
\pgfpathlineto{\pgfqpoint{4.727891in}{0.739656in}}%
\pgfpathlineto{\pgfqpoint{4.727594in}{0.739656in}}%
\pgfpathlineto{\pgfqpoint{4.727296in}{0.739656in}}%
\pgfpathlineto{\pgfqpoint{4.726999in}{0.739656in}}%
\pgfpathlineto{\pgfqpoint{4.726702in}{0.739656in}}%
\pgfpathlineto{\pgfqpoint{4.726404in}{0.739656in}}%
\pgfpathlineto{\pgfqpoint{4.726107in}{0.739656in}}%
\pgfpathlineto{\pgfqpoint{4.725809in}{0.739656in}}%
\pgfpathlineto{\pgfqpoint{4.725512in}{0.739656in}}%
\pgfpathlineto{\pgfqpoint{4.725214in}{0.739656in}}%
\pgfpathlineto{\pgfqpoint{4.724917in}{0.739656in}}%
\pgfpathlineto{\pgfqpoint{4.724619in}{0.739656in}}%
\pgfpathlineto{\pgfqpoint{4.724322in}{0.739656in}}%
\pgfpathlineto{\pgfqpoint{4.724024in}{0.739656in}}%
\pgfpathlineto{\pgfqpoint{4.723727in}{0.739656in}}%
\pgfpathlineto{\pgfqpoint{4.723429in}{0.739656in}}%
\pgfpathlineto{\pgfqpoint{4.723132in}{0.739656in}}%
\pgfpathlineto{\pgfqpoint{4.722834in}{0.739656in}}%
\pgfpathlineto{\pgfqpoint{4.722537in}{0.739656in}}%
\pgfpathlineto{\pgfqpoint{4.722239in}{0.739656in}}%
\pgfpathlineto{\pgfqpoint{4.721942in}{0.739656in}}%
\pgfpathlineto{\pgfqpoint{4.721644in}{0.739656in}}%
\pgfpathlineto{\pgfqpoint{4.721347in}{0.739656in}}%
\pgfpathlineto{\pgfqpoint{4.721049in}{0.739656in}}%
\pgfpathlineto{\pgfqpoint{4.720752in}{0.739656in}}%
\pgfpathlineto{\pgfqpoint{4.720454in}{0.739656in}}%
\pgfpathlineto{\pgfqpoint{4.720157in}{0.739656in}}%
\pgfpathlineto{\pgfqpoint{4.719860in}{0.739656in}}%
\pgfpathlineto{\pgfqpoint{4.719562in}{0.739656in}}%
\pgfpathlineto{\pgfqpoint{4.719265in}{0.739656in}}%
\pgfpathlineto{\pgfqpoint{4.718967in}{0.739656in}}%
\pgfpathlineto{\pgfqpoint{4.718670in}{0.739656in}}%
\pgfpathlineto{\pgfqpoint{4.718372in}{0.739656in}}%
\pgfpathlineto{\pgfqpoint{4.718075in}{0.739656in}}%
\pgfpathlineto{\pgfqpoint{4.717777in}{0.739656in}}%
\pgfpathlineto{\pgfqpoint{4.717480in}{0.739656in}}%
\pgfpathlineto{\pgfqpoint{4.717182in}{0.739656in}}%
\pgfpathlineto{\pgfqpoint{4.716885in}{0.739656in}}%
\pgfpathlineto{\pgfqpoint{4.716587in}{0.739656in}}%
\pgfpathlineto{\pgfqpoint{4.716290in}{0.739656in}}%
\pgfpathlineto{\pgfqpoint{4.715992in}{0.739656in}}%
\pgfpathlineto{\pgfqpoint{4.715695in}{0.739656in}}%
\pgfpathlineto{\pgfqpoint{4.715397in}{0.739656in}}%
\pgfpathlineto{\pgfqpoint{4.715100in}{0.739656in}}%
\pgfpathlineto{\pgfqpoint{4.714802in}{0.739656in}}%
\pgfpathlineto{\pgfqpoint{4.714505in}{0.739656in}}%
\pgfpathlineto{\pgfqpoint{4.714207in}{0.739656in}}%
\pgfpathlineto{\pgfqpoint{4.713910in}{0.739656in}}%
\pgfpathlineto{\pgfqpoint{4.713613in}{0.739656in}}%
\pgfpathlineto{\pgfqpoint{4.713315in}{0.739656in}}%
\pgfpathlineto{\pgfqpoint{4.713018in}{0.739656in}}%
\pgfpathlineto{\pgfqpoint{4.712720in}{0.739656in}}%
\pgfpathlineto{\pgfqpoint{4.712423in}{0.739656in}}%
\pgfpathlineto{\pgfqpoint{4.712125in}{0.739656in}}%
\pgfpathlineto{\pgfqpoint{4.711828in}{0.739656in}}%
\pgfpathlineto{\pgfqpoint{4.711530in}{0.739656in}}%
\pgfpathlineto{\pgfqpoint{4.711233in}{0.739656in}}%
\pgfpathlineto{\pgfqpoint{4.710935in}{0.739656in}}%
\pgfpathlineto{\pgfqpoint{4.710638in}{0.739656in}}%
\pgfpathlineto{\pgfqpoint{4.710340in}{0.739656in}}%
\pgfpathlineto{\pgfqpoint{4.710043in}{0.739656in}}%
\pgfpathlineto{\pgfqpoint{4.709745in}{0.739656in}}%
\pgfpathlineto{\pgfqpoint{4.709448in}{0.739656in}}%
\pgfpathlineto{\pgfqpoint{4.709150in}{0.739656in}}%
\pgfpathlineto{\pgfqpoint{4.708853in}{0.739656in}}%
\pgfpathlineto{\pgfqpoint{4.708555in}{0.739656in}}%
\pgfpathlineto{\pgfqpoint{4.708258in}{0.739656in}}%
\pgfpathlineto{\pgfqpoint{4.707960in}{0.739656in}}%
\pgfpathlineto{\pgfqpoint{4.707663in}{0.739656in}}%
\pgfpathlineto{\pgfqpoint{4.707365in}{0.739656in}}%
\pgfpathlineto{\pgfqpoint{4.707068in}{0.739656in}}%
\pgfpathlineto{\pgfqpoint{4.706771in}{0.739656in}}%
\pgfpathlineto{\pgfqpoint{4.706473in}{0.739656in}}%
\pgfpathlineto{\pgfqpoint{4.706176in}{0.739656in}}%
\pgfpathlineto{\pgfqpoint{4.705878in}{0.739656in}}%
\pgfpathlineto{\pgfqpoint{4.705581in}{0.739656in}}%
\pgfpathlineto{\pgfqpoint{4.705283in}{0.739656in}}%
\pgfpathlineto{\pgfqpoint{4.704986in}{0.739656in}}%
\pgfpathlineto{\pgfqpoint{4.704688in}{0.739656in}}%
\pgfpathlineto{\pgfqpoint{4.704391in}{0.739656in}}%
\pgfpathlineto{\pgfqpoint{4.704093in}{0.739656in}}%
\pgfpathlineto{\pgfqpoint{4.703796in}{0.739656in}}%
\pgfpathlineto{\pgfqpoint{4.703498in}{0.739656in}}%
\pgfpathlineto{\pgfqpoint{4.703201in}{0.739656in}}%
\pgfpathlineto{\pgfqpoint{4.702903in}{0.739656in}}%
\pgfpathlineto{\pgfqpoint{4.702606in}{0.739656in}}%
\pgfpathlineto{\pgfqpoint{4.702308in}{0.739656in}}%
\pgfpathlineto{\pgfqpoint{4.702011in}{0.739656in}}%
\pgfpathlineto{\pgfqpoint{4.701713in}{0.739656in}}%
\pgfpathlineto{\pgfqpoint{4.701416in}{0.739656in}}%
\pgfpathlineto{\pgfqpoint{4.701118in}{0.739656in}}%
\pgfpathlineto{\pgfqpoint{4.700821in}{0.739656in}}%
\pgfpathlineto{\pgfqpoint{4.700523in}{0.739656in}}%
\pgfpathlineto{\pgfqpoint{4.700226in}{0.739656in}}%
\pgfpathlineto{\pgfqpoint{4.699929in}{0.739656in}}%
\pgfpathlineto{\pgfqpoint{4.699631in}{0.739656in}}%
\pgfpathlineto{\pgfqpoint{4.699334in}{0.739656in}}%
\pgfpathlineto{\pgfqpoint{4.699036in}{0.739656in}}%
\pgfpathlineto{\pgfqpoint{4.698739in}{0.739656in}}%
\pgfpathlineto{\pgfqpoint{4.698441in}{0.739656in}}%
\pgfpathlineto{\pgfqpoint{4.698144in}{0.739656in}}%
\pgfpathlineto{\pgfqpoint{4.697846in}{0.739656in}}%
\pgfpathlineto{\pgfqpoint{4.697549in}{0.739656in}}%
\pgfpathlineto{\pgfqpoint{4.697251in}{0.739656in}}%
\pgfpathlineto{\pgfqpoint{4.696954in}{0.739656in}}%
\pgfpathlineto{\pgfqpoint{4.696656in}{0.739656in}}%
\pgfpathlineto{\pgfqpoint{4.696359in}{0.739656in}}%
\pgfpathlineto{\pgfqpoint{4.696061in}{0.739656in}}%
\pgfpathlineto{\pgfqpoint{4.695764in}{0.739656in}}%
\pgfpathlineto{\pgfqpoint{4.695466in}{0.739656in}}%
\pgfpathlineto{\pgfqpoint{4.695169in}{0.739656in}}%
\pgfpathlineto{\pgfqpoint{4.694871in}{0.739656in}}%
\pgfpathlineto{\pgfqpoint{4.694574in}{0.739656in}}%
\pgfpathlineto{\pgfqpoint{4.694276in}{0.739656in}}%
\pgfpathlineto{\pgfqpoint{4.693979in}{0.739656in}}%
\pgfpathlineto{\pgfqpoint{4.693682in}{0.739656in}}%
\pgfpathlineto{\pgfqpoint{4.693384in}{0.739656in}}%
\pgfpathlineto{\pgfqpoint{4.693087in}{0.739656in}}%
\pgfpathlineto{\pgfqpoint{4.692789in}{0.739656in}}%
\pgfpathlineto{\pgfqpoint{4.692492in}{0.739656in}}%
\pgfpathlineto{\pgfqpoint{4.692194in}{0.739656in}}%
\pgfpathlineto{\pgfqpoint{4.691897in}{0.739656in}}%
\pgfpathlineto{\pgfqpoint{4.691599in}{0.739656in}}%
\pgfpathlineto{\pgfqpoint{4.691302in}{0.739656in}}%
\pgfpathlineto{\pgfqpoint{4.691004in}{0.739656in}}%
\pgfpathlineto{\pgfqpoint{4.690707in}{0.739656in}}%
\pgfpathlineto{\pgfqpoint{4.690409in}{0.739656in}}%
\pgfpathlineto{\pgfqpoint{4.690112in}{0.739656in}}%
\pgfpathlineto{\pgfqpoint{4.689814in}{0.739656in}}%
\pgfpathlineto{\pgfqpoint{4.689517in}{0.739656in}}%
\pgfpathlineto{\pgfqpoint{4.689219in}{0.739656in}}%
\pgfpathlineto{\pgfqpoint{4.688922in}{0.739656in}}%
\pgfpathlineto{\pgfqpoint{4.688624in}{0.739656in}}%
\pgfpathlineto{\pgfqpoint{4.688327in}{0.739656in}}%
\pgfpathlineto{\pgfqpoint{4.688029in}{0.739656in}}%
\pgfpathlineto{\pgfqpoint{4.687732in}{0.739656in}}%
\pgfpathlineto{\pgfqpoint{4.687434in}{0.739656in}}%
\pgfpathlineto{\pgfqpoint{4.687137in}{0.739656in}}%
\pgfpathlineto{\pgfqpoint{4.686840in}{0.739656in}}%
\pgfpathlineto{\pgfqpoint{4.686542in}{0.739656in}}%
\pgfpathlineto{\pgfqpoint{4.686245in}{0.739656in}}%
\pgfpathlineto{\pgfqpoint{4.685947in}{0.739656in}}%
\pgfpathlineto{\pgfqpoint{4.685650in}{0.739656in}}%
\pgfpathlineto{\pgfqpoint{4.685352in}{0.739656in}}%
\pgfpathlineto{\pgfqpoint{4.685055in}{0.739656in}}%
\pgfpathlineto{\pgfqpoint{4.684757in}{0.739656in}}%
\pgfpathlineto{\pgfqpoint{4.684460in}{0.739656in}}%
\pgfpathlineto{\pgfqpoint{4.684162in}{0.739656in}}%
\pgfpathlineto{\pgfqpoint{4.683865in}{0.739656in}}%
\pgfpathlineto{\pgfqpoint{4.683567in}{0.739656in}}%
\pgfpathlineto{\pgfqpoint{4.683270in}{0.739656in}}%
\pgfpathlineto{\pgfqpoint{4.682972in}{0.739656in}}%
\pgfpathlineto{\pgfqpoint{4.682675in}{0.739656in}}%
\pgfpathlineto{\pgfqpoint{4.682377in}{0.739656in}}%
\pgfpathlineto{\pgfqpoint{4.682080in}{0.739656in}}%
\pgfpathlineto{\pgfqpoint{4.681782in}{0.739656in}}%
\pgfpathlineto{\pgfqpoint{4.681485in}{0.739656in}}%
\pgfpathlineto{\pgfqpoint{4.681187in}{0.739656in}}%
\pgfpathlineto{\pgfqpoint{4.680890in}{0.739656in}}%
\pgfpathlineto{\pgfqpoint{4.680592in}{0.739656in}}%
\pgfpathlineto{\pgfqpoint{4.680295in}{0.739656in}}%
\pgfpathlineto{\pgfqpoint{4.679998in}{0.739656in}}%
\pgfpathlineto{\pgfqpoint{4.679700in}{0.739656in}}%
\pgfpathlineto{\pgfqpoint{4.679403in}{0.739656in}}%
\pgfpathlineto{\pgfqpoint{4.679105in}{0.739656in}}%
\pgfpathlineto{\pgfqpoint{4.678808in}{0.739656in}}%
\pgfpathlineto{\pgfqpoint{4.678510in}{0.739656in}}%
\pgfpathlineto{\pgfqpoint{4.678213in}{0.739656in}}%
\pgfpathlineto{\pgfqpoint{4.677915in}{0.739656in}}%
\pgfpathlineto{\pgfqpoint{4.677618in}{0.739656in}}%
\pgfpathlineto{\pgfqpoint{4.677320in}{0.739656in}}%
\pgfpathlineto{\pgfqpoint{4.677023in}{0.739656in}}%
\pgfpathlineto{\pgfqpoint{4.676725in}{0.739656in}}%
\pgfpathlineto{\pgfqpoint{4.676428in}{0.739656in}}%
\pgfpathlineto{\pgfqpoint{4.676130in}{0.739656in}}%
\pgfpathlineto{\pgfqpoint{4.675833in}{0.739656in}}%
\pgfpathlineto{\pgfqpoint{4.675535in}{0.739656in}}%
\pgfpathlineto{\pgfqpoint{4.675238in}{0.739656in}}%
\pgfpathlineto{\pgfqpoint{4.674940in}{0.739656in}}%
\pgfpathlineto{\pgfqpoint{4.674643in}{0.739656in}}%
\pgfpathlineto{\pgfqpoint{4.674345in}{0.739656in}}%
\pgfpathlineto{\pgfqpoint{4.674048in}{0.739656in}}%
\pgfpathlineto{\pgfqpoint{4.673751in}{0.739656in}}%
\pgfpathlineto{\pgfqpoint{4.673453in}{0.739656in}}%
\pgfpathlineto{\pgfqpoint{4.673156in}{0.739656in}}%
\pgfpathlineto{\pgfqpoint{4.672858in}{0.739656in}}%
\pgfpathlineto{\pgfqpoint{4.672561in}{0.739656in}}%
\pgfpathlineto{\pgfqpoint{4.672263in}{0.739656in}}%
\pgfpathlineto{\pgfqpoint{4.671966in}{0.739656in}}%
\pgfpathlineto{\pgfqpoint{4.671668in}{0.739656in}}%
\pgfpathlineto{\pgfqpoint{4.671371in}{0.739656in}}%
\pgfpathlineto{\pgfqpoint{4.671073in}{0.739656in}}%
\pgfpathlineto{\pgfqpoint{4.670776in}{0.739656in}}%
\pgfpathlineto{\pgfqpoint{4.670478in}{0.739656in}}%
\pgfpathlineto{\pgfqpoint{4.670181in}{0.739656in}}%
\pgfpathlineto{\pgfqpoint{4.669883in}{0.739656in}}%
\pgfpathlineto{\pgfqpoint{4.669586in}{0.739656in}}%
\pgfpathlineto{\pgfqpoint{4.669288in}{0.739656in}}%
\pgfpathlineto{\pgfqpoint{4.668991in}{0.739656in}}%
\pgfpathlineto{\pgfqpoint{4.668693in}{0.739656in}}%
\pgfpathlineto{\pgfqpoint{4.668396in}{0.739656in}}%
\pgfpathlineto{\pgfqpoint{4.668098in}{0.739656in}}%
\pgfpathlineto{\pgfqpoint{4.667801in}{0.739656in}}%
\pgfpathlineto{\pgfqpoint{4.667503in}{0.739656in}}%
\pgfpathlineto{\pgfqpoint{4.667206in}{0.739656in}}%
\pgfpathlineto{\pgfqpoint{4.666909in}{0.739656in}}%
\pgfpathlineto{\pgfqpoint{4.666611in}{0.739656in}}%
\pgfpathlineto{\pgfqpoint{4.666314in}{0.739656in}}%
\pgfpathlineto{\pgfqpoint{4.666016in}{0.739656in}}%
\pgfpathlineto{\pgfqpoint{4.665719in}{0.739656in}}%
\pgfpathlineto{\pgfqpoint{4.665421in}{0.739656in}}%
\pgfpathlineto{\pgfqpoint{4.665124in}{0.739656in}}%
\pgfpathlineto{\pgfqpoint{4.664826in}{0.739656in}}%
\pgfpathlineto{\pgfqpoint{4.664529in}{0.739656in}}%
\pgfpathlineto{\pgfqpoint{4.664231in}{0.739656in}}%
\pgfpathlineto{\pgfqpoint{4.663934in}{0.739656in}}%
\pgfpathlineto{\pgfqpoint{4.663636in}{0.739656in}}%
\pgfpathlineto{\pgfqpoint{4.663339in}{0.739656in}}%
\pgfpathlineto{\pgfqpoint{4.663041in}{0.739656in}}%
\pgfpathlineto{\pgfqpoint{4.662744in}{0.739656in}}%
\pgfpathlineto{\pgfqpoint{4.662446in}{0.739656in}}%
\pgfpathlineto{\pgfqpoint{4.662149in}{0.739656in}}%
\pgfpathlineto{\pgfqpoint{4.661851in}{0.739656in}}%
\pgfpathlineto{\pgfqpoint{4.661554in}{0.739656in}}%
\pgfpathlineto{\pgfqpoint{4.661256in}{0.739656in}}%
\pgfpathlineto{\pgfqpoint{4.660959in}{0.739656in}}%
\pgfpathlineto{\pgfqpoint{4.660661in}{0.739656in}}%
\pgfpathlineto{\pgfqpoint{4.660364in}{0.739656in}}%
\pgfpathlineto{\pgfqpoint{4.660067in}{0.739656in}}%
\pgfpathlineto{\pgfqpoint{4.659769in}{0.739656in}}%
\pgfpathlineto{\pgfqpoint{4.659472in}{0.739656in}}%
\pgfpathlineto{\pgfqpoint{4.659174in}{0.739656in}}%
\pgfpathlineto{\pgfqpoint{4.658877in}{0.739656in}}%
\pgfpathlineto{\pgfqpoint{4.658579in}{0.739656in}}%
\pgfpathlineto{\pgfqpoint{4.658282in}{0.739656in}}%
\pgfpathlineto{\pgfqpoint{4.657984in}{0.739656in}}%
\pgfpathlineto{\pgfqpoint{4.657687in}{0.739656in}}%
\pgfpathlineto{\pgfqpoint{4.657389in}{0.739656in}}%
\pgfpathlineto{\pgfqpoint{4.657092in}{0.739656in}}%
\pgfpathlineto{\pgfqpoint{4.656794in}{0.739656in}}%
\pgfpathlineto{\pgfqpoint{4.656497in}{0.739656in}}%
\pgfpathlineto{\pgfqpoint{4.656199in}{0.739656in}}%
\pgfpathlineto{\pgfqpoint{4.655902in}{0.739656in}}%
\pgfpathlineto{\pgfqpoint{4.655604in}{0.739656in}}%
\pgfpathlineto{\pgfqpoint{4.655307in}{0.739656in}}%
\pgfpathlineto{\pgfqpoint{4.655009in}{0.739656in}}%
\pgfpathlineto{\pgfqpoint{4.654712in}{0.739656in}}%
\pgfpathlineto{\pgfqpoint{4.654414in}{0.739656in}}%
\pgfpathlineto{\pgfqpoint{4.654117in}{0.739656in}}%
\pgfpathlineto{\pgfqpoint{4.653820in}{0.739656in}}%
\pgfpathlineto{\pgfqpoint{4.653522in}{0.739656in}}%
\pgfpathlineto{\pgfqpoint{4.653225in}{0.739656in}}%
\pgfpathlineto{\pgfqpoint{4.652927in}{0.739656in}}%
\pgfpathlineto{\pgfqpoint{4.652630in}{0.739656in}}%
\pgfpathlineto{\pgfqpoint{4.652332in}{0.739656in}}%
\pgfpathlineto{\pgfqpoint{4.652035in}{0.739656in}}%
\pgfpathlineto{\pgfqpoint{4.651737in}{0.739656in}}%
\pgfpathlineto{\pgfqpoint{4.651440in}{0.739656in}}%
\pgfpathlineto{\pgfqpoint{4.651142in}{0.739656in}}%
\pgfpathlineto{\pgfqpoint{4.650845in}{0.739656in}}%
\pgfpathlineto{\pgfqpoint{4.650547in}{0.739656in}}%
\pgfpathlineto{\pgfqpoint{4.650250in}{0.739656in}}%
\pgfpathlineto{\pgfqpoint{4.649952in}{0.739656in}}%
\pgfpathlineto{\pgfqpoint{4.649655in}{0.739656in}}%
\pgfpathlineto{\pgfqpoint{4.649357in}{0.739656in}}%
\pgfpathlineto{\pgfqpoint{4.649060in}{0.739656in}}%
\pgfpathlineto{\pgfqpoint{4.648762in}{0.739656in}}%
\pgfpathlineto{\pgfqpoint{4.648465in}{0.739656in}}%
\pgfpathlineto{\pgfqpoint{4.648167in}{0.739656in}}%
\pgfpathlineto{\pgfqpoint{4.647870in}{0.739656in}}%
\pgfpathlineto{\pgfqpoint{4.647572in}{0.739656in}}%
\pgfpathlineto{\pgfqpoint{4.647275in}{0.739656in}}%
\pgfpathlineto{\pgfqpoint{4.646978in}{0.739656in}}%
\pgfpathlineto{\pgfqpoint{4.646680in}{0.739656in}}%
\pgfpathlineto{\pgfqpoint{4.646383in}{0.739656in}}%
\pgfpathlineto{\pgfqpoint{4.646085in}{0.739656in}}%
\pgfpathlineto{\pgfqpoint{4.645788in}{0.739656in}}%
\pgfpathlineto{\pgfqpoint{4.645490in}{0.739656in}}%
\pgfpathlineto{\pgfqpoint{4.645193in}{0.739656in}}%
\pgfpathlineto{\pgfqpoint{4.644895in}{0.739656in}}%
\pgfpathlineto{\pgfqpoint{4.644598in}{0.739656in}}%
\pgfpathlineto{\pgfqpoint{4.644300in}{0.739656in}}%
\pgfpathlineto{\pgfqpoint{4.644003in}{0.739656in}}%
\pgfpathlineto{\pgfqpoint{4.643705in}{0.739656in}}%
\pgfpathlineto{\pgfqpoint{4.643408in}{0.739656in}}%
\pgfpathlineto{\pgfqpoint{4.643110in}{0.739656in}}%
\pgfpathlineto{\pgfqpoint{4.642813in}{0.739656in}}%
\pgfpathlineto{\pgfqpoint{4.642515in}{0.739656in}}%
\pgfpathlineto{\pgfqpoint{4.642218in}{0.739656in}}%
\pgfpathlineto{\pgfqpoint{4.641920in}{0.739656in}}%
\pgfpathlineto{\pgfqpoint{4.641623in}{0.739656in}}%
\pgfpathlineto{\pgfqpoint{4.641325in}{0.739656in}}%
\pgfpathlineto{\pgfqpoint{4.641028in}{0.739656in}}%
\pgfpathlineto{\pgfqpoint{4.640730in}{0.739656in}}%
\pgfpathlineto{\pgfqpoint{4.640433in}{0.739656in}}%
\pgfpathlineto{\pgfqpoint{4.640136in}{0.739656in}}%
\pgfpathlineto{\pgfqpoint{4.639838in}{0.739656in}}%
\pgfpathlineto{\pgfqpoint{4.639541in}{0.739656in}}%
\pgfpathlineto{\pgfqpoint{4.639243in}{0.739656in}}%
\pgfpathlineto{\pgfqpoint{4.638946in}{0.739656in}}%
\pgfpathlineto{\pgfqpoint{4.638648in}{0.739656in}}%
\pgfpathlineto{\pgfqpoint{4.638351in}{0.739656in}}%
\pgfpathlineto{\pgfqpoint{4.638053in}{0.739656in}}%
\pgfpathlineto{\pgfqpoint{4.637756in}{0.739656in}}%
\pgfpathlineto{\pgfqpoint{4.637458in}{0.739656in}}%
\pgfpathlineto{\pgfqpoint{4.637161in}{0.739656in}}%
\pgfpathlineto{\pgfqpoint{4.636863in}{0.739656in}}%
\pgfpathlineto{\pgfqpoint{4.636566in}{0.739656in}}%
\pgfpathlineto{\pgfqpoint{4.636268in}{0.739656in}}%
\pgfpathlineto{\pgfqpoint{4.635971in}{0.739656in}}%
\pgfpathlineto{\pgfqpoint{4.635673in}{0.739656in}}%
\pgfpathlineto{\pgfqpoint{4.635376in}{0.739656in}}%
\pgfpathlineto{\pgfqpoint{4.635078in}{0.739656in}}%
\pgfpathlineto{\pgfqpoint{4.634781in}{0.739656in}}%
\pgfpathlineto{\pgfqpoint{4.634483in}{0.739656in}}%
\pgfpathlineto{\pgfqpoint{4.634186in}{0.739656in}}%
\pgfpathlineto{\pgfqpoint{4.633889in}{0.739656in}}%
\pgfpathlineto{\pgfqpoint{4.633591in}{0.739656in}}%
\pgfpathlineto{\pgfqpoint{4.633294in}{0.739656in}}%
\pgfpathlineto{\pgfqpoint{4.632996in}{0.739656in}}%
\pgfpathlineto{\pgfqpoint{4.632699in}{0.739656in}}%
\pgfpathlineto{\pgfqpoint{4.632401in}{0.739656in}}%
\pgfpathlineto{\pgfqpoint{4.632104in}{0.739656in}}%
\pgfpathlineto{\pgfqpoint{4.631806in}{0.739656in}}%
\pgfpathlineto{\pgfqpoint{4.631509in}{0.739656in}}%
\pgfpathlineto{\pgfqpoint{4.631211in}{0.739656in}}%
\pgfpathlineto{\pgfqpoint{4.630914in}{0.739656in}}%
\pgfpathlineto{\pgfqpoint{4.630616in}{0.739656in}}%
\pgfpathlineto{\pgfqpoint{4.630319in}{0.739656in}}%
\pgfpathlineto{\pgfqpoint{4.630021in}{0.739656in}}%
\pgfpathlineto{\pgfqpoint{4.629724in}{0.739656in}}%
\pgfpathlineto{\pgfqpoint{4.629426in}{0.739656in}}%
\pgfpathlineto{\pgfqpoint{4.629129in}{0.739656in}}%
\pgfpathlineto{\pgfqpoint{4.628831in}{0.739656in}}%
\pgfpathlineto{\pgfqpoint{4.628534in}{0.739656in}}%
\pgfpathlineto{\pgfqpoint{4.628236in}{0.739656in}}%
\pgfpathlineto{\pgfqpoint{4.627939in}{0.739656in}}%
\pgfpathlineto{\pgfqpoint{4.627641in}{0.739656in}}%
\pgfpathlineto{\pgfqpoint{4.627344in}{0.739656in}}%
\pgfpathlineto{\pgfqpoint{4.627047in}{0.739656in}}%
\pgfpathlineto{\pgfqpoint{4.626749in}{0.739656in}}%
\pgfpathlineto{\pgfqpoint{4.626452in}{0.739656in}}%
\pgfpathlineto{\pgfqpoint{4.626154in}{0.739656in}}%
\pgfpathlineto{\pgfqpoint{4.625857in}{0.739656in}}%
\pgfpathlineto{\pgfqpoint{4.625559in}{0.739656in}}%
\pgfpathlineto{\pgfqpoint{4.625262in}{0.739656in}}%
\pgfpathlineto{\pgfqpoint{4.624964in}{0.739656in}}%
\pgfpathlineto{\pgfqpoint{4.624667in}{0.739656in}}%
\pgfpathlineto{\pgfqpoint{4.624369in}{0.739656in}}%
\pgfpathlineto{\pgfqpoint{4.624072in}{0.739656in}}%
\pgfpathlineto{\pgfqpoint{4.623774in}{0.739656in}}%
\pgfpathlineto{\pgfqpoint{4.623477in}{0.739656in}}%
\pgfpathlineto{\pgfqpoint{4.623179in}{0.739656in}}%
\pgfpathlineto{\pgfqpoint{4.622882in}{0.739656in}}%
\pgfpathlineto{\pgfqpoint{4.622584in}{0.739656in}}%
\pgfpathlineto{\pgfqpoint{4.622287in}{0.739656in}}%
\pgfpathlineto{\pgfqpoint{4.621989in}{0.739656in}}%
\pgfpathlineto{\pgfqpoint{4.621692in}{0.739656in}}%
\pgfpathlineto{\pgfqpoint{4.621394in}{0.739656in}}%
\pgfpathlineto{\pgfqpoint{4.621097in}{0.739656in}}%
\pgfpathlineto{\pgfqpoint{4.620799in}{0.739656in}}%
\pgfpathlineto{\pgfqpoint{4.620502in}{0.739656in}}%
\pgfpathlineto{\pgfqpoint{4.620205in}{0.739656in}}%
\pgfpathlineto{\pgfqpoint{4.619907in}{0.739656in}}%
\pgfpathlineto{\pgfqpoint{4.619610in}{0.739656in}}%
\pgfpathlineto{\pgfqpoint{4.619312in}{0.739656in}}%
\pgfpathlineto{\pgfqpoint{4.619015in}{0.739656in}}%
\pgfpathlineto{\pgfqpoint{4.618717in}{0.739656in}}%
\pgfpathlineto{\pgfqpoint{4.618420in}{0.739656in}}%
\pgfpathlineto{\pgfqpoint{4.618122in}{0.739656in}}%
\pgfpathlineto{\pgfqpoint{4.617825in}{0.739656in}}%
\pgfpathlineto{\pgfqpoint{4.617527in}{0.739656in}}%
\pgfpathlineto{\pgfqpoint{4.617230in}{0.739656in}}%
\pgfpathlineto{\pgfqpoint{4.616932in}{0.739656in}}%
\pgfpathlineto{\pgfqpoint{4.616635in}{0.739656in}}%
\pgfpathlineto{\pgfqpoint{4.616337in}{0.739656in}}%
\pgfpathlineto{\pgfqpoint{4.616040in}{0.739656in}}%
\pgfpathlineto{\pgfqpoint{4.615742in}{0.739656in}}%
\pgfpathlineto{\pgfqpoint{4.615445in}{0.739656in}}%
\pgfpathlineto{\pgfqpoint{4.615147in}{0.739656in}}%
\pgfpathlineto{\pgfqpoint{4.614850in}{0.739656in}}%
\pgfpathlineto{\pgfqpoint{4.614552in}{0.739656in}}%
\pgfpathlineto{\pgfqpoint{4.614255in}{0.739656in}}%
\pgfpathlineto{\pgfqpoint{4.613958in}{0.739656in}}%
\pgfpathlineto{\pgfqpoint{4.613660in}{0.739656in}}%
\pgfpathlineto{\pgfqpoint{4.613363in}{0.739656in}}%
\pgfpathlineto{\pgfqpoint{4.613065in}{0.739656in}}%
\pgfpathlineto{\pgfqpoint{4.612768in}{0.739656in}}%
\pgfpathlineto{\pgfqpoint{4.612470in}{0.739656in}}%
\pgfpathlineto{\pgfqpoint{4.612173in}{0.739656in}}%
\pgfpathlineto{\pgfqpoint{4.611875in}{0.739656in}}%
\pgfpathlineto{\pgfqpoint{4.611578in}{0.739656in}}%
\pgfpathlineto{\pgfqpoint{4.611280in}{0.739656in}}%
\pgfpathlineto{\pgfqpoint{4.610983in}{0.739656in}}%
\pgfpathlineto{\pgfqpoint{4.610685in}{0.739656in}}%
\pgfpathlineto{\pgfqpoint{4.610388in}{0.739656in}}%
\pgfpathlineto{\pgfqpoint{4.610090in}{0.739656in}}%
\pgfpathlineto{\pgfqpoint{4.609793in}{0.739656in}}%
\pgfpathlineto{\pgfqpoint{4.609495in}{0.739656in}}%
\pgfpathlineto{\pgfqpoint{4.609198in}{0.739656in}}%
\pgfpathlineto{\pgfqpoint{4.608900in}{0.739656in}}%
\pgfpathlineto{\pgfqpoint{4.608603in}{0.739656in}}%
\pgfpathlineto{\pgfqpoint{4.608305in}{0.739656in}}%
\pgfpathlineto{\pgfqpoint{4.608008in}{0.739656in}}%
\pgfpathlineto{\pgfqpoint{4.607710in}{0.739656in}}%
\pgfpathlineto{\pgfqpoint{4.607413in}{0.739656in}}%
\pgfpathlineto{\pgfqpoint{4.607116in}{0.739656in}}%
\pgfpathlineto{\pgfqpoint{4.606818in}{0.739656in}}%
\pgfpathlineto{\pgfqpoint{4.606521in}{0.739656in}}%
\pgfpathlineto{\pgfqpoint{4.606223in}{0.739656in}}%
\pgfpathlineto{\pgfqpoint{4.605926in}{0.739656in}}%
\pgfpathlineto{\pgfqpoint{4.605628in}{0.739656in}}%
\pgfpathlineto{\pgfqpoint{4.605331in}{0.739656in}}%
\pgfpathlineto{\pgfqpoint{4.605033in}{0.739656in}}%
\pgfpathlineto{\pgfqpoint{4.604736in}{0.739656in}}%
\pgfpathlineto{\pgfqpoint{4.604438in}{0.739656in}}%
\pgfpathlineto{\pgfqpoint{4.604141in}{0.739656in}}%
\pgfpathlineto{\pgfqpoint{4.603843in}{0.739656in}}%
\pgfpathlineto{\pgfqpoint{4.603546in}{0.739656in}}%
\pgfpathlineto{\pgfqpoint{4.603248in}{0.739656in}}%
\pgfpathlineto{\pgfqpoint{4.602951in}{0.739656in}}%
\pgfpathlineto{\pgfqpoint{4.602653in}{0.739656in}}%
\pgfpathlineto{\pgfqpoint{4.602356in}{0.739656in}}%
\pgfpathlineto{\pgfqpoint{4.602058in}{0.739656in}}%
\pgfpathlineto{\pgfqpoint{4.601761in}{0.739656in}}%
\pgfpathlineto{\pgfqpoint{4.601463in}{0.739656in}}%
\pgfpathlineto{\pgfqpoint{4.601166in}{0.739656in}}%
\pgfpathlineto{\pgfqpoint{4.600868in}{0.739656in}}%
\pgfpathlineto{\pgfqpoint{4.600571in}{0.739656in}}%
\pgfpathlineto{\pgfqpoint{4.600274in}{0.739656in}}%
\pgfpathlineto{\pgfqpoint{4.599976in}{0.739656in}}%
\pgfpathlineto{\pgfqpoint{4.599679in}{0.739656in}}%
\pgfpathlineto{\pgfqpoint{4.599381in}{0.739656in}}%
\pgfpathlineto{\pgfqpoint{4.599084in}{0.739656in}}%
\pgfpathlineto{\pgfqpoint{4.598786in}{0.739656in}}%
\pgfpathlineto{\pgfqpoint{4.598489in}{0.739656in}}%
\pgfpathlineto{\pgfqpoint{4.598191in}{0.739656in}}%
\pgfpathlineto{\pgfqpoint{4.597894in}{0.739656in}}%
\pgfpathlineto{\pgfqpoint{4.597596in}{0.739656in}}%
\pgfpathlineto{\pgfqpoint{4.597299in}{0.739656in}}%
\pgfpathlineto{\pgfqpoint{4.597001in}{0.739656in}}%
\pgfpathlineto{\pgfqpoint{4.596704in}{0.739656in}}%
\pgfpathlineto{\pgfqpoint{4.596406in}{0.739656in}}%
\pgfpathlineto{\pgfqpoint{4.596109in}{0.739656in}}%
\pgfpathlineto{\pgfqpoint{4.595811in}{0.739656in}}%
\pgfpathlineto{\pgfqpoint{4.595514in}{0.739656in}}%
\pgfpathlineto{\pgfqpoint{4.595216in}{0.739656in}}%
\pgfpathlineto{\pgfqpoint{4.594919in}{0.739656in}}%
\pgfpathlineto{\pgfqpoint{4.594621in}{0.739656in}}%
\pgfpathlineto{\pgfqpoint{4.594324in}{0.739656in}}%
\pgfpathlineto{\pgfqpoint{4.594026in}{0.739656in}}%
\pgfpathlineto{\pgfqpoint{4.593729in}{0.739656in}}%
\pgfpathlineto{\pgfqpoint{4.593432in}{0.739656in}}%
\pgfpathlineto{\pgfqpoint{4.593134in}{0.739656in}}%
\pgfpathlineto{\pgfqpoint{4.592837in}{0.739656in}}%
\pgfpathlineto{\pgfqpoint{4.592539in}{0.739656in}}%
\pgfpathlineto{\pgfqpoint{4.592242in}{0.739656in}}%
\pgfpathlineto{\pgfqpoint{4.591944in}{0.739656in}}%
\pgfpathlineto{\pgfqpoint{4.591647in}{0.739656in}}%
\pgfpathlineto{\pgfqpoint{4.591349in}{0.739656in}}%
\pgfpathlineto{\pgfqpoint{4.591052in}{0.739656in}}%
\pgfpathlineto{\pgfqpoint{4.590754in}{0.739656in}}%
\pgfpathlineto{\pgfqpoint{4.590457in}{0.739656in}}%
\pgfpathlineto{\pgfqpoint{4.590159in}{0.739656in}}%
\pgfpathlineto{\pgfqpoint{4.589862in}{0.739656in}}%
\pgfpathlineto{\pgfqpoint{4.589564in}{0.739656in}}%
\pgfpathlineto{\pgfqpoint{4.589267in}{0.739656in}}%
\pgfpathlineto{\pgfqpoint{4.588969in}{0.739656in}}%
\pgfpathlineto{\pgfqpoint{4.588672in}{0.739656in}}%
\pgfpathlineto{\pgfqpoint{4.588374in}{0.739656in}}%
\pgfpathlineto{\pgfqpoint{4.588077in}{0.739656in}}%
\pgfpathlineto{\pgfqpoint{4.587779in}{0.739656in}}%
\pgfpathlineto{\pgfqpoint{4.587482in}{0.739656in}}%
\pgfpathlineto{\pgfqpoint{4.587185in}{0.739656in}}%
\pgfpathlineto{\pgfqpoint{4.586887in}{0.739656in}}%
\pgfpathlineto{\pgfqpoint{4.586590in}{0.739656in}}%
\pgfpathlineto{\pgfqpoint{4.586292in}{0.739656in}}%
\pgfpathlineto{\pgfqpoint{4.585995in}{0.739656in}}%
\pgfpathlineto{\pgfqpoint{4.585697in}{0.739656in}}%
\pgfpathlineto{\pgfqpoint{4.585400in}{0.739656in}}%
\pgfpathlineto{\pgfqpoint{4.585102in}{0.739656in}}%
\pgfpathlineto{\pgfqpoint{4.584805in}{0.739656in}}%
\pgfpathlineto{\pgfqpoint{4.584507in}{0.739656in}}%
\pgfpathlineto{\pgfqpoint{4.584210in}{0.739656in}}%
\pgfpathlineto{\pgfqpoint{4.583912in}{0.739656in}}%
\pgfpathlineto{\pgfqpoint{4.583615in}{0.739656in}}%
\pgfpathlineto{\pgfqpoint{4.583317in}{0.739656in}}%
\pgfpathlineto{\pgfqpoint{4.583020in}{0.739656in}}%
\pgfpathlineto{\pgfqpoint{4.582722in}{0.739656in}}%
\pgfpathlineto{\pgfqpoint{4.582425in}{0.739656in}}%
\pgfpathlineto{\pgfqpoint{4.582127in}{0.739656in}}%
\pgfpathlineto{\pgfqpoint{4.581830in}{0.739656in}}%
\pgfpathlineto{\pgfqpoint{4.581532in}{0.739656in}}%
\pgfpathlineto{\pgfqpoint{4.581235in}{0.739656in}}%
\pgfpathlineto{\pgfqpoint{4.580937in}{0.739656in}}%
\pgfpathlineto{\pgfqpoint{4.580640in}{0.739656in}}%
\pgfpathlineto{\pgfqpoint{4.580343in}{0.739656in}}%
\pgfpathlineto{\pgfqpoint{4.580045in}{0.739656in}}%
\pgfpathlineto{\pgfqpoint{4.579748in}{0.739656in}}%
\pgfpathlineto{\pgfqpoint{4.579450in}{0.739656in}}%
\pgfpathlineto{\pgfqpoint{4.579153in}{0.739656in}}%
\pgfpathlineto{\pgfqpoint{4.578855in}{0.739656in}}%
\pgfpathlineto{\pgfqpoint{4.578558in}{0.739656in}}%
\pgfpathlineto{\pgfqpoint{4.578260in}{0.739656in}}%
\pgfpathlineto{\pgfqpoint{4.577963in}{0.739656in}}%
\pgfpathlineto{\pgfqpoint{4.577665in}{0.739656in}}%
\pgfpathlineto{\pgfqpoint{4.577368in}{0.739656in}}%
\pgfpathlineto{\pgfqpoint{4.577070in}{0.739656in}}%
\pgfpathlineto{\pgfqpoint{4.576773in}{0.739656in}}%
\pgfpathlineto{\pgfqpoint{4.576475in}{0.739656in}}%
\pgfpathlineto{\pgfqpoint{4.576178in}{0.739656in}}%
\pgfpathlineto{\pgfqpoint{4.575880in}{0.739656in}}%
\pgfpathlineto{\pgfqpoint{4.575583in}{0.739656in}}%
\pgfpathlineto{\pgfqpoint{4.575285in}{0.739656in}}%
\pgfpathlineto{\pgfqpoint{4.574988in}{0.739656in}}%
\pgfpathlineto{\pgfqpoint{4.574690in}{0.739656in}}%
\pgfpathlineto{\pgfqpoint{4.574393in}{0.739656in}}%
\pgfpathlineto{\pgfqpoint{4.574095in}{0.739656in}}%
\pgfpathlineto{\pgfqpoint{4.573798in}{0.739656in}}%
\pgfpathlineto{\pgfqpoint{4.573501in}{0.739656in}}%
\pgfpathlineto{\pgfqpoint{4.573203in}{0.739656in}}%
\pgfpathlineto{\pgfqpoint{4.572906in}{0.739656in}}%
\pgfpathlineto{\pgfqpoint{4.572608in}{0.739656in}}%
\pgfpathlineto{\pgfqpoint{4.572311in}{0.739656in}}%
\pgfpathlineto{\pgfqpoint{4.572013in}{0.739656in}}%
\pgfpathlineto{\pgfqpoint{4.571716in}{0.739656in}}%
\pgfpathlineto{\pgfqpoint{4.571418in}{0.739656in}}%
\pgfpathlineto{\pgfqpoint{4.571121in}{0.739656in}}%
\pgfpathlineto{\pgfqpoint{4.570823in}{0.739656in}}%
\pgfpathlineto{\pgfqpoint{4.570526in}{0.739656in}}%
\pgfpathlineto{\pgfqpoint{4.570228in}{0.739656in}}%
\pgfpathlineto{\pgfqpoint{4.569931in}{0.739656in}}%
\pgfpathlineto{\pgfqpoint{4.569633in}{0.739656in}}%
\pgfpathlineto{\pgfqpoint{4.569336in}{0.739656in}}%
\pgfpathlineto{\pgfqpoint{4.569038in}{0.739656in}}%
\pgfpathlineto{\pgfqpoint{4.568741in}{0.739656in}}%
\pgfpathlineto{\pgfqpoint{4.568443in}{0.739656in}}%
\pgfpathlineto{\pgfqpoint{4.568146in}{0.739656in}}%
\pgfpathlineto{\pgfqpoint{4.567848in}{0.739656in}}%
\pgfpathlineto{\pgfqpoint{4.567551in}{0.739656in}}%
\pgfpathlineto{\pgfqpoint{4.567254in}{0.739656in}}%
\pgfpathlineto{\pgfqpoint{4.566956in}{0.739656in}}%
\pgfpathlineto{\pgfqpoint{4.566659in}{0.739656in}}%
\pgfpathlineto{\pgfqpoint{4.566361in}{0.739656in}}%
\pgfpathlineto{\pgfqpoint{4.566064in}{0.739656in}}%
\pgfpathlineto{\pgfqpoint{4.565766in}{0.739656in}}%
\pgfpathlineto{\pgfqpoint{4.565469in}{0.739656in}}%
\pgfpathlineto{\pgfqpoint{4.565171in}{0.739656in}}%
\pgfpathlineto{\pgfqpoint{4.564874in}{0.739656in}}%
\pgfpathlineto{\pgfqpoint{4.564576in}{0.739656in}}%
\pgfpathlineto{\pgfqpoint{4.564279in}{0.739656in}}%
\pgfpathlineto{\pgfqpoint{4.563981in}{0.739656in}}%
\pgfpathlineto{\pgfqpoint{4.563684in}{0.739656in}}%
\pgfpathlineto{\pgfqpoint{4.563386in}{0.739656in}}%
\pgfpathlineto{\pgfqpoint{4.563089in}{0.739656in}}%
\pgfpathlineto{\pgfqpoint{4.562791in}{0.739656in}}%
\pgfpathlineto{\pgfqpoint{4.562494in}{0.739656in}}%
\pgfpathlineto{\pgfqpoint{4.562196in}{0.739656in}}%
\pgfpathlineto{\pgfqpoint{4.561899in}{0.739656in}}%
\pgfpathlineto{\pgfqpoint{4.561601in}{0.739656in}}%
\pgfpathlineto{\pgfqpoint{4.561304in}{0.739656in}}%
\pgfpathlineto{\pgfqpoint{4.561006in}{0.739656in}}%
\pgfpathlineto{\pgfqpoint{4.560709in}{0.739656in}}%
\pgfpathlineto{\pgfqpoint{4.560412in}{0.739656in}}%
\pgfpathlineto{\pgfqpoint{4.560114in}{0.739656in}}%
\pgfpathlineto{\pgfqpoint{4.559817in}{0.739656in}}%
\pgfpathlineto{\pgfqpoint{4.559519in}{0.739656in}}%
\pgfpathlineto{\pgfqpoint{4.559222in}{0.739656in}}%
\pgfpathlineto{\pgfqpoint{4.558924in}{0.739656in}}%
\pgfpathlineto{\pgfqpoint{4.558627in}{0.739656in}}%
\pgfpathlineto{\pgfqpoint{4.558329in}{0.739656in}}%
\pgfpathlineto{\pgfqpoint{4.558032in}{0.739656in}}%
\pgfpathlineto{\pgfqpoint{4.557734in}{0.739656in}}%
\pgfpathlineto{\pgfqpoint{4.557437in}{0.739656in}}%
\pgfpathlineto{\pgfqpoint{4.557139in}{0.739656in}}%
\pgfpathlineto{\pgfqpoint{4.556842in}{0.739656in}}%
\pgfpathlineto{\pgfqpoint{4.556544in}{0.739656in}}%
\pgfpathlineto{\pgfqpoint{4.556247in}{0.739656in}}%
\pgfpathlineto{\pgfqpoint{4.555949in}{0.739656in}}%
\pgfpathlineto{\pgfqpoint{4.555652in}{0.739656in}}%
\pgfpathlineto{\pgfqpoint{4.555354in}{0.739656in}}%
\pgfpathlineto{\pgfqpoint{4.555057in}{0.739656in}}%
\pgfpathlineto{\pgfqpoint{4.554759in}{0.739656in}}%
\pgfpathlineto{\pgfqpoint{4.554462in}{0.739656in}}%
\pgfpathlineto{\pgfqpoint{4.554164in}{0.739656in}}%
\pgfpathlineto{\pgfqpoint{4.553867in}{0.739656in}}%
\pgfpathlineto{\pgfqpoint{4.553570in}{0.739656in}}%
\pgfpathlineto{\pgfqpoint{4.553272in}{0.739656in}}%
\pgfpathlineto{\pgfqpoint{4.552975in}{0.739656in}}%
\pgfpathlineto{\pgfqpoint{4.552677in}{0.739656in}}%
\pgfpathlineto{\pgfqpoint{4.552380in}{0.739656in}}%
\pgfpathlineto{\pgfqpoint{4.552082in}{0.739656in}}%
\pgfpathlineto{\pgfqpoint{4.551785in}{0.739656in}}%
\pgfpathlineto{\pgfqpoint{4.551487in}{0.739656in}}%
\pgfpathlineto{\pgfqpoint{4.551190in}{0.739656in}}%
\pgfpathlineto{\pgfqpoint{4.550892in}{0.739656in}}%
\pgfpathlineto{\pgfqpoint{4.550595in}{0.739656in}}%
\pgfpathlineto{\pgfqpoint{4.550297in}{0.739656in}}%
\pgfpathlineto{\pgfqpoint{4.550000in}{0.739656in}}%
\pgfpathlineto{\pgfqpoint{4.549702in}{0.739656in}}%
\pgfpathlineto{\pgfqpoint{4.549405in}{0.739656in}}%
\pgfpathlineto{\pgfqpoint{4.549107in}{0.739656in}}%
\pgfpathlineto{\pgfqpoint{4.548810in}{0.739656in}}%
\pgfpathlineto{\pgfqpoint{4.548512in}{0.739656in}}%
\pgfpathlineto{\pgfqpoint{4.548215in}{0.739656in}}%
\pgfpathlineto{\pgfqpoint{4.547917in}{0.739656in}}%
\pgfpathlineto{\pgfqpoint{4.547620in}{0.739656in}}%
\pgfpathlineto{\pgfqpoint{4.547323in}{0.739656in}}%
\pgfpathlineto{\pgfqpoint{4.547025in}{0.739656in}}%
\pgfpathlineto{\pgfqpoint{4.546728in}{0.739656in}}%
\pgfpathlineto{\pgfqpoint{4.546430in}{0.739656in}}%
\pgfpathlineto{\pgfqpoint{4.546133in}{0.739656in}}%
\pgfpathlineto{\pgfqpoint{4.545835in}{0.739656in}}%
\pgfpathlineto{\pgfqpoint{4.545538in}{0.739656in}}%
\pgfpathlineto{\pgfqpoint{4.545240in}{0.739656in}}%
\pgfpathlineto{\pgfqpoint{4.544943in}{0.739656in}}%
\pgfpathlineto{\pgfqpoint{4.544645in}{0.739656in}}%
\pgfpathlineto{\pgfqpoint{4.544348in}{0.739656in}}%
\pgfpathlineto{\pgfqpoint{4.544050in}{0.739656in}}%
\pgfpathlineto{\pgfqpoint{4.543753in}{0.739656in}}%
\pgfpathlineto{\pgfqpoint{4.543455in}{0.739656in}}%
\pgfpathlineto{\pgfqpoint{4.543158in}{0.739656in}}%
\pgfpathlineto{\pgfqpoint{4.542860in}{0.739656in}}%
\pgfpathlineto{\pgfqpoint{4.542563in}{0.739656in}}%
\pgfpathlineto{\pgfqpoint{4.542265in}{0.739656in}}%
\pgfpathlineto{\pgfqpoint{4.541968in}{0.739656in}}%
\pgfpathlineto{\pgfqpoint{4.541670in}{0.739656in}}%
\pgfpathlineto{\pgfqpoint{4.541373in}{0.739656in}}%
\pgfpathlineto{\pgfqpoint{4.541075in}{0.739656in}}%
\pgfpathlineto{\pgfqpoint{4.540778in}{0.739656in}}%
\pgfpathlineto{\pgfqpoint{4.540481in}{0.739656in}}%
\pgfpathlineto{\pgfqpoint{4.540183in}{0.739656in}}%
\pgfpathlineto{\pgfqpoint{4.539886in}{0.739656in}}%
\pgfpathlineto{\pgfqpoint{4.539588in}{0.739656in}}%
\pgfpathlineto{\pgfqpoint{4.539291in}{0.739656in}}%
\pgfpathlineto{\pgfqpoint{4.538993in}{0.739656in}}%
\pgfpathlineto{\pgfqpoint{4.538696in}{0.739656in}}%
\pgfpathlineto{\pgfqpoint{4.538398in}{0.739656in}}%
\pgfpathlineto{\pgfqpoint{4.538101in}{0.739656in}}%
\pgfpathlineto{\pgfqpoint{4.537803in}{0.739656in}}%
\pgfpathlineto{\pgfqpoint{4.537506in}{0.739656in}}%
\pgfpathlineto{\pgfqpoint{4.537208in}{0.739656in}}%
\pgfpathlineto{\pgfqpoint{4.536911in}{0.739656in}}%
\pgfpathlineto{\pgfqpoint{4.536613in}{0.739656in}}%
\pgfpathlineto{\pgfqpoint{4.536316in}{0.739656in}}%
\pgfpathlineto{\pgfqpoint{4.536018in}{0.739656in}}%
\pgfpathlineto{\pgfqpoint{4.535721in}{0.739656in}}%
\pgfpathlineto{\pgfqpoint{4.535423in}{0.739656in}}%
\pgfpathlineto{\pgfqpoint{4.535126in}{0.739656in}}%
\pgfpathlineto{\pgfqpoint{4.534828in}{0.739656in}}%
\pgfpathlineto{\pgfqpoint{4.534531in}{0.739656in}}%
\pgfpathlineto{\pgfqpoint{4.534233in}{0.739656in}}%
\pgfpathlineto{\pgfqpoint{4.533936in}{0.739656in}}%
\pgfpathlineto{\pgfqpoint{4.533639in}{0.739656in}}%
\pgfpathlineto{\pgfqpoint{4.533341in}{0.739656in}}%
\pgfpathlineto{\pgfqpoint{4.533044in}{0.739656in}}%
\pgfpathlineto{\pgfqpoint{4.532746in}{0.739656in}}%
\pgfpathlineto{\pgfqpoint{4.532449in}{0.739656in}}%
\pgfpathlineto{\pgfqpoint{4.532151in}{0.739656in}}%
\pgfpathlineto{\pgfqpoint{4.531854in}{0.739656in}}%
\pgfpathlineto{\pgfqpoint{4.531556in}{0.739656in}}%
\pgfpathlineto{\pgfqpoint{4.531259in}{0.739656in}}%
\pgfpathlineto{\pgfqpoint{4.530961in}{0.739656in}}%
\pgfpathlineto{\pgfqpoint{4.530664in}{0.739656in}}%
\pgfpathlineto{\pgfqpoint{4.530366in}{0.739656in}}%
\pgfpathlineto{\pgfqpoint{4.530069in}{0.739656in}}%
\pgfpathlineto{\pgfqpoint{4.529771in}{0.739656in}}%
\pgfpathlineto{\pgfqpoint{4.529474in}{0.739656in}}%
\pgfpathlineto{\pgfqpoint{4.529176in}{0.739656in}}%
\pgfpathlineto{\pgfqpoint{4.528879in}{0.739656in}}%
\pgfpathlineto{\pgfqpoint{4.528581in}{0.739656in}}%
\pgfpathlineto{\pgfqpoint{4.528284in}{0.739656in}}%
\pgfpathlineto{\pgfqpoint{4.527986in}{0.739656in}}%
\pgfpathlineto{\pgfqpoint{4.527689in}{0.739656in}}%
\pgfpathlineto{\pgfqpoint{4.527392in}{0.739656in}}%
\pgfpathlineto{\pgfqpoint{4.527094in}{0.739656in}}%
\pgfpathlineto{\pgfqpoint{4.526797in}{0.739656in}}%
\pgfpathlineto{\pgfqpoint{4.526499in}{0.739656in}}%
\pgfpathlineto{\pgfqpoint{4.526202in}{0.739656in}}%
\pgfpathlineto{\pgfqpoint{4.525904in}{0.739656in}}%
\pgfpathlineto{\pgfqpoint{4.525607in}{0.739656in}}%
\pgfpathlineto{\pgfqpoint{4.525309in}{0.739656in}}%
\pgfpathlineto{\pgfqpoint{4.525012in}{0.739656in}}%
\pgfpathlineto{\pgfqpoint{4.524714in}{0.739656in}}%
\pgfpathlineto{\pgfqpoint{4.524417in}{0.739656in}}%
\pgfpathlineto{\pgfqpoint{4.524119in}{0.739656in}}%
\pgfpathlineto{\pgfqpoint{4.523822in}{0.739656in}}%
\pgfpathlineto{\pgfqpoint{4.523524in}{0.739656in}}%
\pgfpathlineto{\pgfqpoint{4.523227in}{0.739656in}}%
\pgfpathlineto{\pgfqpoint{4.522929in}{0.739656in}}%
\pgfpathlineto{\pgfqpoint{4.522632in}{0.739656in}}%
\pgfpathlineto{\pgfqpoint{4.522334in}{0.739656in}}%
\pgfpathlineto{\pgfqpoint{4.522037in}{0.739656in}}%
\pgfpathlineto{\pgfqpoint{4.521739in}{0.739656in}}%
\pgfpathlineto{\pgfqpoint{4.521442in}{0.739656in}}%
\pgfpathlineto{\pgfqpoint{4.521144in}{0.739656in}}%
\pgfpathlineto{\pgfqpoint{4.520847in}{0.739656in}}%
\pgfpathlineto{\pgfqpoint{4.520550in}{0.739656in}}%
\pgfpathlineto{\pgfqpoint{4.520252in}{0.739656in}}%
\pgfpathlineto{\pgfqpoint{4.519955in}{0.739656in}}%
\pgfpathlineto{\pgfqpoint{4.519657in}{0.739656in}}%
\pgfpathlineto{\pgfqpoint{4.519360in}{0.739656in}}%
\pgfpathlineto{\pgfqpoint{4.519062in}{0.739656in}}%
\pgfpathlineto{\pgfqpoint{4.518765in}{0.739656in}}%
\pgfpathlineto{\pgfqpoint{4.518467in}{0.739656in}}%
\pgfpathlineto{\pgfqpoint{4.518170in}{0.739656in}}%
\pgfpathlineto{\pgfqpoint{4.517872in}{0.739656in}}%
\pgfpathlineto{\pgfqpoint{4.517575in}{0.739656in}}%
\pgfpathlineto{\pgfqpoint{4.517277in}{0.739656in}}%
\pgfpathlineto{\pgfqpoint{4.516980in}{0.739656in}}%
\pgfpathlineto{\pgfqpoint{4.516682in}{0.739656in}}%
\pgfpathlineto{\pgfqpoint{4.516385in}{0.739656in}}%
\pgfpathlineto{\pgfqpoint{4.516087in}{0.739656in}}%
\pgfpathlineto{\pgfqpoint{4.515790in}{0.739656in}}%
\pgfpathlineto{\pgfqpoint{4.515492in}{0.739656in}}%
\pgfpathlineto{\pgfqpoint{4.515195in}{0.739656in}}%
\pgfpathlineto{\pgfqpoint{4.514897in}{0.739656in}}%
\pgfpathlineto{\pgfqpoint{4.514600in}{0.739656in}}%
\pgfpathlineto{\pgfqpoint{4.514302in}{0.739656in}}%
\pgfpathlineto{\pgfqpoint{4.514005in}{0.739656in}}%
\pgfpathlineto{\pgfqpoint{4.513708in}{0.739656in}}%
\pgfpathlineto{\pgfqpoint{4.513410in}{0.739656in}}%
\pgfpathlineto{\pgfqpoint{4.513113in}{0.739656in}}%
\pgfpathlineto{\pgfqpoint{4.512815in}{0.739656in}}%
\pgfpathlineto{\pgfqpoint{4.512518in}{0.739656in}}%
\pgfpathlineto{\pgfqpoint{4.512220in}{0.739656in}}%
\pgfpathlineto{\pgfqpoint{4.511923in}{0.739656in}}%
\pgfpathlineto{\pgfqpoint{4.511625in}{0.739656in}}%
\pgfpathlineto{\pgfqpoint{4.511328in}{0.739656in}}%
\pgfpathlineto{\pgfqpoint{4.511030in}{0.739656in}}%
\pgfpathlineto{\pgfqpoint{4.510733in}{0.739656in}}%
\pgfpathlineto{\pgfqpoint{4.510435in}{0.739656in}}%
\pgfpathlineto{\pgfqpoint{4.510138in}{0.739656in}}%
\pgfpathlineto{\pgfqpoint{4.509840in}{0.739656in}}%
\pgfpathlineto{\pgfqpoint{4.509543in}{0.739656in}}%
\pgfpathlineto{\pgfqpoint{4.509245in}{0.739656in}}%
\pgfpathlineto{\pgfqpoint{4.508948in}{0.739656in}}%
\pgfpathlineto{\pgfqpoint{4.508650in}{0.739656in}}%
\pgfpathlineto{\pgfqpoint{4.508353in}{0.739656in}}%
\pgfpathlineto{\pgfqpoint{4.508055in}{0.739656in}}%
\pgfpathlineto{\pgfqpoint{4.507758in}{0.739656in}}%
\pgfpathlineto{\pgfqpoint{4.507461in}{0.739656in}}%
\pgfpathlineto{\pgfqpoint{4.507163in}{0.739656in}}%
\pgfpathlineto{\pgfqpoint{4.506866in}{0.739656in}}%
\pgfpathlineto{\pgfqpoint{4.506568in}{0.739656in}}%
\pgfpathlineto{\pgfqpoint{4.506271in}{0.739656in}}%
\pgfpathlineto{\pgfqpoint{4.505973in}{0.739656in}}%
\pgfpathlineto{\pgfqpoint{4.505676in}{0.739656in}}%
\pgfpathlineto{\pgfqpoint{4.505378in}{0.739656in}}%
\pgfpathlineto{\pgfqpoint{4.505081in}{0.739656in}}%
\pgfpathlineto{\pgfqpoint{4.504783in}{0.739656in}}%
\pgfpathlineto{\pgfqpoint{4.504486in}{0.739656in}}%
\pgfpathlineto{\pgfqpoint{4.504188in}{0.739656in}}%
\pgfpathlineto{\pgfqpoint{4.503891in}{0.739656in}}%
\pgfpathlineto{\pgfqpoint{4.503593in}{0.739656in}}%
\pgfpathlineto{\pgfqpoint{4.503296in}{0.739656in}}%
\pgfpathlineto{\pgfqpoint{4.502998in}{0.739656in}}%
\pgfpathlineto{\pgfqpoint{4.502701in}{0.739656in}}%
\pgfpathlineto{\pgfqpoint{4.502403in}{0.739656in}}%
\pgfpathlineto{\pgfqpoint{4.502106in}{0.739656in}}%
\pgfpathlineto{\pgfqpoint{4.501808in}{0.739656in}}%
\pgfpathlineto{\pgfqpoint{4.501511in}{0.739656in}}%
\pgfpathlineto{\pgfqpoint{4.501213in}{0.739656in}}%
\pgfpathlineto{\pgfqpoint{4.500916in}{0.739656in}}%
\pgfpathlineto{\pgfqpoint{4.500619in}{0.739656in}}%
\pgfpathlineto{\pgfqpoint{4.500321in}{0.739656in}}%
\pgfpathlineto{\pgfqpoint{4.500024in}{0.739656in}}%
\pgfpathlineto{\pgfqpoint{4.499726in}{0.739656in}}%
\pgfpathlineto{\pgfqpoint{4.499429in}{0.739656in}}%
\pgfpathlineto{\pgfqpoint{4.499131in}{0.739656in}}%
\pgfpathlineto{\pgfqpoint{4.498834in}{0.739656in}}%
\pgfpathlineto{\pgfqpoint{4.498536in}{0.739656in}}%
\pgfpathlineto{\pgfqpoint{4.498239in}{0.739656in}}%
\pgfpathlineto{\pgfqpoint{4.497941in}{0.739656in}}%
\pgfpathlineto{\pgfqpoint{4.497644in}{0.739656in}}%
\pgfpathlineto{\pgfqpoint{4.497346in}{0.739656in}}%
\pgfpathlineto{\pgfqpoint{4.497049in}{0.739656in}}%
\pgfpathlineto{\pgfqpoint{4.496751in}{0.739656in}}%
\pgfpathlineto{\pgfqpoint{4.496454in}{0.739656in}}%
\pgfpathlineto{\pgfqpoint{4.496156in}{0.739656in}}%
\pgfpathlineto{\pgfqpoint{4.495859in}{0.739656in}}%
\pgfpathlineto{\pgfqpoint{4.495561in}{0.739656in}}%
\pgfpathlineto{\pgfqpoint{4.495264in}{0.739656in}}%
\pgfpathlineto{\pgfqpoint{4.494966in}{0.739656in}}%
\pgfpathlineto{\pgfqpoint{4.494669in}{0.739656in}}%
\pgfpathlineto{\pgfqpoint{4.494371in}{0.739656in}}%
\pgfpathlineto{\pgfqpoint{4.494074in}{0.739656in}}%
\pgfpathlineto{\pgfqpoint{4.493777in}{0.739656in}}%
\pgfpathlineto{\pgfqpoint{4.493479in}{0.739656in}}%
\pgfpathlineto{\pgfqpoint{4.493182in}{0.739656in}}%
\pgfpathlineto{\pgfqpoint{4.492884in}{0.739656in}}%
\pgfpathlineto{\pgfqpoint{4.492587in}{0.739656in}}%
\pgfpathlineto{\pgfqpoint{4.492289in}{0.739656in}}%
\pgfpathlineto{\pgfqpoint{4.491992in}{0.739656in}}%
\pgfpathlineto{\pgfqpoint{4.491694in}{0.739656in}}%
\pgfpathlineto{\pgfqpoint{4.491397in}{0.739656in}}%
\pgfpathlineto{\pgfqpoint{4.491099in}{0.739656in}}%
\pgfpathlineto{\pgfqpoint{4.490802in}{0.739656in}}%
\pgfpathlineto{\pgfqpoint{4.490504in}{0.739656in}}%
\pgfpathlineto{\pgfqpoint{4.490207in}{0.739656in}}%
\pgfpathlineto{\pgfqpoint{4.489909in}{0.739656in}}%
\pgfpathlineto{\pgfqpoint{4.489612in}{0.739656in}}%
\pgfpathlineto{\pgfqpoint{4.489314in}{0.739656in}}%
\pgfpathlineto{\pgfqpoint{4.489017in}{0.739656in}}%
\pgfpathlineto{\pgfqpoint{4.488719in}{0.739656in}}%
\pgfpathlineto{\pgfqpoint{4.488422in}{0.739656in}}%
\pgfpathlineto{\pgfqpoint{4.488124in}{0.739656in}}%
\pgfpathlineto{\pgfqpoint{4.487827in}{0.739656in}}%
\pgfpathlineto{\pgfqpoint{4.487530in}{0.739656in}}%
\pgfpathlineto{\pgfqpoint{4.487232in}{0.739656in}}%
\pgfpathlineto{\pgfqpoint{4.486935in}{0.739656in}}%
\pgfpathlineto{\pgfqpoint{4.486637in}{0.739656in}}%
\pgfpathlineto{\pgfqpoint{4.486340in}{0.739656in}}%
\pgfpathlineto{\pgfqpoint{4.486042in}{0.739656in}}%
\pgfpathlineto{\pgfqpoint{4.485745in}{0.739656in}}%
\pgfpathlineto{\pgfqpoint{4.485447in}{0.739656in}}%
\pgfpathlineto{\pgfqpoint{4.485150in}{0.739656in}}%
\pgfpathlineto{\pgfqpoint{4.484852in}{0.739656in}}%
\pgfpathlineto{\pgfqpoint{4.484555in}{0.739656in}}%
\pgfpathlineto{\pgfqpoint{4.484257in}{0.739656in}}%
\pgfpathlineto{\pgfqpoint{4.483960in}{0.739656in}}%
\pgfpathlineto{\pgfqpoint{4.483662in}{0.739656in}}%
\pgfpathlineto{\pgfqpoint{4.483365in}{0.739656in}}%
\pgfpathlineto{\pgfqpoint{4.483067in}{0.739656in}}%
\pgfpathlineto{\pgfqpoint{4.482770in}{0.739656in}}%
\pgfpathlineto{\pgfqpoint{4.482472in}{0.739656in}}%
\pgfpathlineto{\pgfqpoint{4.482175in}{0.739656in}}%
\pgfpathlineto{\pgfqpoint{4.481877in}{0.739656in}}%
\pgfpathlineto{\pgfqpoint{4.481580in}{0.739656in}}%
\pgfpathlineto{\pgfqpoint{4.481282in}{0.739656in}}%
\pgfpathlineto{\pgfqpoint{4.480985in}{0.739656in}}%
\pgfpathlineto{\pgfqpoint{4.480688in}{0.739656in}}%
\pgfpathlineto{\pgfqpoint{4.480390in}{0.739656in}}%
\pgfpathlineto{\pgfqpoint{4.480093in}{0.739656in}}%
\pgfpathlineto{\pgfqpoint{4.479795in}{0.739656in}}%
\pgfpathlineto{\pgfqpoint{4.479498in}{0.739656in}}%
\pgfpathlineto{\pgfqpoint{4.479200in}{0.739656in}}%
\pgfpathlineto{\pgfqpoint{4.478903in}{0.739656in}}%
\pgfpathlineto{\pgfqpoint{4.478605in}{0.739656in}}%
\pgfpathlineto{\pgfqpoint{4.478308in}{0.739656in}}%
\pgfpathlineto{\pgfqpoint{4.478010in}{0.739656in}}%
\pgfpathlineto{\pgfqpoint{4.477713in}{0.739656in}}%
\pgfpathlineto{\pgfqpoint{4.477415in}{0.739656in}}%
\pgfpathlineto{\pgfqpoint{4.477118in}{0.739656in}}%
\pgfpathlineto{\pgfqpoint{4.476820in}{0.739656in}}%
\pgfpathlineto{\pgfqpoint{4.476523in}{0.739656in}}%
\pgfpathlineto{\pgfqpoint{4.476225in}{0.739656in}}%
\pgfpathlineto{\pgfqpoint{4.475928in}{0.739656in}}%
\pgfpathlineto{\pgfqpoint{4.475630in}{0.739656in}}%
\pgfpathlineto{\pgfqpoint{4.475333in}{0.739656in}}%
\pgfpathlineto{\pgfqpoint{4.475035in}{0.739656in}}%
\pgfpathlineto{\pgfqpoint{4.474738in}{0.739656in}}%
\pgfpathlineto{\pgfqpoint{4.474440in}{0.739656in}}%
\pgfpathlineto{\pgfqpoint{4.474143in}{0.739656in}}%
\pgfpathlineto{\pgfqpoint{4.473846in}{0.739656in}}%
\pgfpathlineto{\pgfqpoint{4.473548in}{0.739656in}}%
\pgfpathlineto{\pgfqpoint{4.473251in}{0.739656in}}%
\pgfpathlineto{\pgfqpoint{4.472953in}{0.739656in}}%
\pgfpathlineto{\pgfqpoint{4.472656in}{0.739656in}}%
\pgfpathlineto{\pgfqpoint{4.472358in}{0.739656in}}%
\pgfpathlineto{\pgfqpoint{4.472061in}{0.739656in}}%
\pgfpathlineto{\pgfqpoint{4.471763in}{0.739656in}}%
\pgfpathlineto{\pgfqpoint{4.471466in}{0.739656in}}%
\pgfpathlineto{\pgfqpoint{4.471168in}{0.739656in}}%
\pgfpathlineto{\pgfqpoint{4.470871in}{0.739656in}}%
\pgfpathlineto{\pgfqpoint{4.470573in}{0.739656in}}%
\pgfpathlineto{\pgfqpoint{4.470276in}{0.739656in}}%
\pgfpathlineto{\pgfqpoint{4.469978in}{0.739656in}}%
\pgfpathlineto{\pgfqpoint{4.469681in}{0.739656in}}%
\pgfpathlineto{\pgfqpoint{4.469383in}{0.739656in}}%
\pgfpathlineto{\pgfqpoint{4.469086in}{0.739656in}}%
\pgfpathlineto{\pgfqpoint{4.468788in}{0.739656in}}%
\pgfpathlineto{\pgfqpoint{4.468491in}{0.739656in}}%
\pgfpathlineto{\pgfqpoint{4.468193in}{0.739656in}}%
\pgfpathlineto{\pgfqpoint{4.467896in}{0.739656in}}%
\pgfpathlineto{\pgfqpoint{4.467599in}{0.739656in}}%
\pgfpathlineto{\pgfqpoint{4.467301in}{0.739656in}}%
\pgfpathlineto{\pgfqpoint{4.467004in}{0.739656in}}%
\pgfpathlineto{\pgfqpoint{4.466706in}{0.739656in}}%
\pgfpathlineto{\pgfqpoint{4.466409in}{0.739656in}}%
\pgfpathlineto{\pgfqpoint{4.466111in}{0.739656in}}%
\pgfpathlineto{\pgfqpoint{4.465814in}{0.739656in}}%
\pgfpathlineto{\pgfqpoint{4.465516in}{0.739656in}}%
\pgfpathlineto{\pgfqpoint{4.465219in}{0.739656in}}%
\pgfpathlineto{\pgfqpoint{4.464921in}{0.739656in}}%
\pgfpathlineto{\pgfqpoint{4.464624in}{0.739656in}}%
\pgfpathlineto{\pgfqpoint{4.464326in}{0.739656in}}%
\pgfpathlineto{\pgfqpoint{4.464029in}{0.739656in}}%
\pgfpathlineto{\pgfqpoint{4.463731in}{0.739656in}}%
\pgfpathlineto{\pgfqpoint{4.463434in}{0.739656in}}%
\pgfpathlineto{\pgfqpoint{4.463136in}{0.739656in}}%
\pgfpathlineto{\pgfqpoint{4.462839in}{0.739656in}}%
\pgfpathlineto{\pgfqpoint{4.462541in}{0.739656in}}%
\pgfpathlineto{\pgfqpoint{4.462244in}{0.739656in}}%
\pgfpathlineto{\pgfqpoint{4.461946in}{0.739656in}}%
\pgfpathlineto{\pgfqpoint{4.461649in}{0.739656in}}%
\pgfpathlineto{\pgfqpoint{4.461351in}{0.739656in}}%
\pgfpathlineto{\pgfqpoint{4.461054in}{0.739656in}}%
\pgfpathlineto{\pgfqpoint{4.460757in}{0.739656in}}%
\pgfpathlineto{\pgfqpoint{4.460459in}{0.739656in}}%
\pgfpathlineto{\pgfqpoint{4.460162in}{0.739656in}}%
\pgfpathlineto{\pgfqpoint{4.459864in}{0.739656in}}%
\pgfpathlineto{\pgfqpoint{4.459567in}{0.739656in}}%
\pgfpathlineto{\pgfqpoint{4.459269in}{0.739656in}}%
\pgfpathlineto{\pgfqpoint{4.458972in}{0.739656in}}%
\pgfpathlineto{\pgfqpoint{4.458674in}{0.739656in}}%
\pgfpathlineto{\pgfqpoint{4.458377in}{0.739656in}}%
\pgfpathlineto{\pgfqpoint{4.458079in}{0.739656in}}%
\pgfpathlineto{\pgfqpoint{4.457782in}{0.739656in}}%
\pgfpathlineto{\pgfqpoint{4.457484in}{0.739656in}}%
\pgfpathlineto{\pgfqpoint{4.457187in}{0.739656in}}%
\pgfpathlineto{\pgfqpoint{4.456889in}{0.739656in}}%
\pgfpathlineto{\pgfqpoint{4.456592in}{0.739656in}}%
\pgfpathlineto{\pgfqpoint{4.456294in}{0.739656in}}%
\pgfpathlineto{\pgfqpoint{4.455997in}{0.739656in}}%
\pgfpathlineto{\pgfqpoint{4.455699in}{0.739656in}}%
\pgfpathlineto{\pgfqpoint{4.455402in}{0.739656in}}%
\pgfpathlineto{\pgfqpoint{4.455104in}{0.739656in}}%
\pgfpathlineto{\pgfqpoint{4.454807in}{0.739656in}}%
\pgfpathlineto{\pgfqpoint{4.454509in}{0.739656in}}%
\pgfpathlineto{\pgfqpoint{4.454212in}{0.739656in}}%
\pgfpathlineto{\pgfqpoint{4.453915in}{0.739656in}}%
\pgfpathlineto{\pgfqpoint{4.453617in}{0.739656in}}%
\pgfpathlineto{\pgfqpoint{4.453320in}{0.739656in}}%
\pgfpathlineto{\pgfqpoint{4.453022in}{0.739656in}}%
\pgfpathlineto{\pgfqpoint{4.452725in}{0.739656in}}%
\pgfpathlineto{\pgfqpoint{4.452427in}{0.739656in}}%
\pgfpathlineto{\pgfqpoint{4.452130in}{0.739656in}}%
\pgfpathlineto{\pgfqpoint{4.451832in}{0.739656in}}%
\pgfpathlineto{\pgfqpoint{4.451535in}{0.739656in}}%
\pgfpathlineto{\pgfqpoint{4.451237in}{0.739656in}}%
\pgfpathlineto{\pgfqpoint{4.450940in}{0.739656in}}%
\pgfpathlineto{\pgfqpoint{4.450642in}{0.739656in}}%
\pgfpathlineto{\pgfqpoint{4.450345in}{0.739656in}}%
\pgfpathlineto{\pgfqpoint{4.450047in}{0.739656in}}%
\pgfpathlineto{\pgfqpoint{4.449750in}{0.739656in}}%
\pgfpathlineto{\pgfqpoint{4.449452in}{0.739656in}}%
\pgfpathlineto{\pgfqpoint{4.449155in}{0.739656in}}%
\pgfpathlineto{\pgfqpoint{4.448857in}{0.739656in}}%
\pgfpathlineto{\pgfqpoint{4.448560in}{0.739656in}}%
\pgfpathlineto{\pgfqpoint{4.448262in}{0.739656in}}%
\pgfpathlineto{\pgfqpoint{4.447965in}{0.739656in}}%
\pgfpathlineto{\pgfqpoint{4.447668in}{0.739656in}}%
\pgfpathlineto{\pgfqpoint{4.447370in}{0.739656in}}%
\pgfpathlineto{\pgfqpoint{4.447073in}{0.739656in}}%
\pgfpathlineto{\pgfqpoint{4.446775in}{0.739656in}}%
\pgfpathlineto{\pgfqpoint{4.446478in}{0.739656in}}%
\pgfpathlineto{\pgfqpoint{4.446180in}{0.739656in}}%
\pgfpathlineto{\pgfqpoint{4.445883in}{0.739656in}}%
\pgfpathlineto{\pgfqpoint{4.445585in}{0.739656in}}%
\pgfpathlineto{\pgfqpoint{4.445288in}{0.739656in}}%
\pgfpathlineto{\pgfqpoint{4.444990in}{0.739656in}}%
\pgfpathlineto{\pgfqpoint{4.444693in}{0.739656in}}%
\pgfpathlineto{\pgfqpoint{4.444395in}{0.739656in}}%
\pgfpathlineto{\pgfqpoint{4.444098in}{0.739656in}}%
\pgfpathlineto{\pgfqpoint{4.443800in}{0.739656in}}%
\pgfpathlineto{\pgfqpoint{4.443503in}{0.739656in}}%
\pgfpathlineto{\pgfqpoint{4.443205in}{0.739656in}}%
\pgfpathlineto{\pgfqpoint{4.442908in}{0.739656in}}%
\pgfpathlineto{\pgfqpoint{4.442610in}{0.739656in}}%
\pgfpathlineto{\pgfqpoint{4.442313in}{0.739656in}}%
\pgfpathlineto{\pgfqpoint{4.442015in}{0.739656in}}%
\pgfpathlineto{\pgfqpoint{4.441718in}{0.739656in}}%
\pgfpathlineto{\pgfqpoint{4.441420in}{0.739656in}}%
\pgfpathlineto{\pgfqpoint{4.441123in}{0.739656in}}%
\pgfpathlineto{\pgfqpoint{4.440826in}{0.739656in}}%
\pgfpathlineto{\pgfqpoint{4.440528in}{0.739656in}}%
\pgfpathlineto{\pgfqpoint{4.440231in}{0.739656in}}%
\pgfpathlineto{\pgfqpoint{4.439933in}{0.739656in}}%
\pgfpathlineto{\pgfqpoint{4.439636in}{0.739656in}}%
\pgfpathlineto{\pgfqpoint{4.439338in}{0.739656in}}%
\pgfpathlineto{\pgfqpoint{4.439041in}{0.739656in}}%
\pgfpathlineto{\pgfqpoint{4.438743in}{0.739656in}}%
\pgfpathlineto{\pgfqpoint{4.438446in}{0.739656in}}%
\pgfpathlineto{\pgfqpoint{4.438148in}{0.739656in}}%
\pgfpathlineto{\pgfqpoint{4.437851in}{0.739656in}}%
\pgfpathlineto{\pgfqpoint{4.437553in}{0.739656in}}%
\pgfpathlineto{\pgfqpoint{4.437256in}{0.739656in}}%
\pgfpathlineto{\pgfqpoint{4.436958in}{0.739656in}}%
\pgfpathlineto{\pgfqpoint{4.436661in}{0.739656in}}%
\pgfpathlineto{\pgfqpoint{4.436363in}{0.739656in}}%
\pgfpathlineto{\pgfqpoint{4.436066in}{0.739656in}}%
\pgfpathlineto{\pgfqpoint{4.435768in}{0.739656in}}%
\pgfpathlineto{\pgfqpoint{4.435471in}{0.739656in}}%
\pgfpathlineto{\pgfqpoint{4.435173in}{0.739656in}}%
\pgfpathlineto{\pgfqpoint{4.434876in}{0.739656in}}%
\pgfpathlineto{\pgfqpoint{4.434578in}{0.739656in}}%
\pgfpathlineto{\pgfqpoint{4.434281in}{0.739656in}}%
\pgfpathlineto{\pgfqpoint{4.433984in}{0.739656in}}%
\pgfpathlineto{\pgfqpoint{4.433686in}{0.739656in}}%
\pgfpathlineto{\pgfqpoint{4.433389in}{0.739656in}}%
\pgfpathlineto{\pgfqpoint{4.433091in}{0.739656in}}%
\pgfpathlineto{\pgfqpoint{4.432794in}{0.739656in}}%
\pgfpathlineto{\pgfqpoint{4.432496in}{0.739656in}}%
\pgfpathlineto{\pgfqpoint{4.432199in}{0.739656in}}%
\pgfpathlineto{\pgfqpoint{4.431901in}{0.739656in}}%
\pgfpathlineto{\pgfqpoint{4.431604in}{0.739656in}}%
\pgfpathlineto{\pgfqpoint{4.431306in}{0.739656in}}%
\pgfpathlineto{\pgfqpoint{4.431009in}{0.739656in}}%
\pgfpathlineto{\pgfqpoint{4.430711in}{0.739656in}}%
\pgfpathlineto{\pgfqpoint{4.430414in}{0.739656in}}%
\pgfpathlineto{\pgfqpoint{4.430116in}{0.739656in}}%
\pgfpathlineto{\pgfqpoint{4.429819in}{0.739656in}}%
\pgfpathlineto{\pgfqpoint{4.429521in}{0.739656in}}%
\pgfpathlineto{\pgfqpoint{4.429224in}{0.739656in}}%
\pgfpathlineto{\pgfqpoint{4.428926in}{0.739656in}}%
\pgfpathlineto{\pgfqpoint{4.428629in}{0.739656in}}%
\pgfpathlineto{\pgfqpoint{4.428331in}{0.739656in}}%
\pgfpathlineto{\pgfqpoint{4.428034in}{0.739656in}}%
\pgfpathlineto{\pgfqpoint{4.427737in}{0.739656in}}%
\pgfpathlineto{\pgfqpoint{4.427439in}{0.739656in}}%
\pgfpathlineto{\pgfqpoint{4.427142in}{0.739656in}}%
\pgfpathlineto{\pgfqpoint{4.426844in}{0.739656in}}%
\pgfpathlineto{\pgfqpoint{4.426547in}{0.739656in}}%
\pgfpathlineto{\pgfqpoint{4.426249in}{0.739656in}}%
\pgfpathlineto{\pgfqpoint{4.425952in}{0.739656in}}%
\pgfpathlineto{\pgfqpoint{4.425654in}{0.739656in}}%
\pgfpathlineto{\pgfqpoint{4.425357in}{0.739656in}}%
\pgfpathlineto{\pgfqpoint{4.425059in}{0.739656in}}%
\pgfpathlineto{\pgfqpoint{4.424762in}{0.739656in}}%
\pgfpathlineto{\pgfqpoint{4.424464in}{0.739656in}}%
\pgfpathlineto{\pgfqpoint{4.424167in}{0.739656in}}%
\pgfpathlineto{\pgfqpoint{4.423869in}{0.739656in}}%
\pgfpathlineto{\pgfqpoint{4.423572in}{0.739656in}}%
\pgfpathlineto{\pgfqpoint{4.423274in}{0.739656in}}%
\pgfpathlineto{\pgfqpoint{4.422977in}{0.739656in}}%
\pgfpathlineto{\pgfqpoint{4.422679in}{0.739656in}}%
\pgfpathlineto{\pgfqpoint{4.422382in}{0.739656in}}%
\pgfpathlineto{\pgfqpoint{4.422084in}{0.739656in}}%
\pgfpathlineto{\pgfqpoint{4.421787in}{0.739656in}}%
\pgfpathlineto{\pgfqpoint{4.421489in}{0.739656in}}%
\pgfpathlineto{\pgfqpoint{4.421192in}{0.739656in}}%
\pgfpathlineto{\pgfqpoint{4.420895in}{0.739656in}}%
\pgfpathlineto{\pgfqpoint{4.420597in}{0.739656in}}%
\pgfpathlineto{\pgfqpoint{4.420300in}{0.739656in}}%
\pgfpathlineto{\pgfqpoint{4.420002in}{0.739656in}}%
\pgfpathlineto{\pgfqpoint{4.419705in}{0.739656in}}%
\pgfpathlineto{\pgfqpoint{4.419407in}{0.739656in}}%
\pgfpathlineto{\pgfqpoint{4.419110in}{0.739656in}}%
\pgfpathlineto{\pgfqpoint{4.418812in}{0.739656in}}%
\pgfpathlineto{\pgfqpoint{4.418515in}{0.739656in}}%
\pgfpathlineto{\pgfqpoint{4.418217in}{0.739656in}}%
\pgfpathlineto{\pgfqpoint{4.417920in}{0.739656in}}%
\pgfpathlineto{\pgfqpoint{4.417622in}{0.739656in}}%
\pgfpathlineto{\pgfqpoint{4.417325in}{0.739656in}}%
\pgfpathlineto{\pgfqpoint{4.417027in}{0.739656in}}%
\pgfpathlineto{\pgfqpoint{4.416730in}{0.739656in}}%
\pgfpathlineto{\pgfqpoint{4.416432in}{0.739656in}}%
\pgfpathlineto{\pgfqpoint{4.416135in}{0.739656in}}%
\pgfpathlineto{\pgfqpoint{4.415837in}{0.739656in}}%
\pgfpathlineto{\pgfqpoint{4.415540in}{0.739656in}}%
\pgfpathlineto{\pgfqpoint{4.415242in}{0.739656in}}%
\pgfpathlineto{\pgfqpoint{4.414945in}{0.739656in}}%
\pgfpathlineto{\pgfqpoint{4.414647in}{0.739656in}}%
\pgfpathlineto{\pgfqpoint{4.414350in}{0.739656in}}%
\pgfpathlineto{\pgfqpoint{4.414053in}{0.739656in}}%
\pgfpathlineto{\pgfqpoint{4.413755in}{0.739656in}}%
\pgfpathlineto{\pgfqpoint{4.413458in}{0.739656in}}%
\pgfpathlineto{\pgfqpoint{4.413160in}{0.739656in}}%
\pgfpathlineto{\pgfqpoint{4.412863in}{0.739656in}}%
\pgfpathlineto{\pgfqpoint{4.412565in}{0.739656in}}%
\pgfpathlineto{\pgfqpoint{4.412268in}{0.739656in}}%
\pgfpathlineto{\pgfqpoint{4.411970in}{0.739656in}}%
\pgfpathlineto{\pgfqpoint{4.411673in}{0.739656in}}%
\pgfpathlineto{\pgfqpoint{4.411375in}{0.739656in}}%
\pgfpathlineto{\pgfqpoint{4.411078in}{0.739656in}}%
\pgfpathlineto{\pgfqpoint{4.410780in}{0.739656in}}%
\pgfpathlineto{\pgfqpoint{4.410483in}{0.739656in}}%
\pgfpathlineto{\pgfqpoint{4.410185in}{0.739656in}}%
\pgfpathlineto{\pgfqpoint{4.409888in}{0.739656in}}%
\pgfpathlineto{\pgfqpoint{4.409590in}{0.739656in}}%
\pgfpathlineto{\pgfqpoint{4.409293in}{0.739656in}}%
\pgfpathlineto{\pgfqpoint{4.408995in}{0.739656in}}%
\pgfpathlineto{\pgfqpoint{4.408698in}{0.739656in}}%
\pgfpathlineto{\pgfqpoint{4.408400in}{0.739656in}}%
\pgfpathlineto{\pgfqpoint{4.408103in}{0.739656in}}%
\pgfpathlineto{\pgfqpoint{4.407806in}{0.739656in}}%
\pgfpathlineto{\pgfqpoint{4.407508in}{0.739656in}}%
\pgfpathlineto{\pgfqpoint{4.407211in}{0.739656in}}%
\pgfpathlineto{\pgfqpoint{4.406913in}{0.739656in}}%
\pgfpathlineto{\pgfqpoint{4.406616in}{0.739656in}}%
\pgfpathlineto{\pgfqpoint{4.406318in}{0.739656in}}%
\pgfpathlineto{\pgfqpoint{4.406021in}{0.739656in}}%
\pgfpathlineto{\pgfqpoint{4.405723in}{0.739656in}}%
\pgfpathlineto{\pgfqpoint{4.405426in}{0.739656in}}%
\pgfpathlineto{\pgfqpoint{4.405128in}{0.739656in}}%
\pgfpathlineto{\pgfqpoint{4.404831in}{0.739656in}}%
\pgfpathlineto{\pgfqpoint{4.404533in}{0.739656in}}%
\pgfpathlineto{\pgfqpoint{4.404236in}{0.739656in}}%
\pgfpathlineto{\pgfqpoint{4.403938in}{0.739656in}}%
\pgfpathlineto{\pgfqpoint{4.403641in}{0.739656in}}%
\pgfpathlineto{\pgfqpoint{4.403343in}{0.739656in}}%
\pgfpathlineto{\pgfqpoint{4.403046in}{0.739656in}}%
\pgfpathlineto{\pgfqpoint{4.402748in}{0.739656in}}%
\pgfpathlineto{\pgfqpoint{4.402451in}{0.739656in}}%
\pgfpathlineto{\pgfqpoint{4.402153in}{0.739656in}}%
\pgfpathlineto{\pgfqpoint{4.401856in}{0.739656in}}%
\pgfpathlineto{\pgfqpoint{4.401558in}{0.739656in}}%
\pgfpathlineto{\pgfqpoint{4.401261in}{0.739656in}}%
\pgfpathlineto{\pgfqpoint{4.400964in}{0.739656in}}%
\pgfpathlineto{\pgfqpoint{4.400666in}{0.739656in}}%
\pgfpathlineto{\pgfqpoint{4.400369in}{0.739656in}}%
\pgfpathlineto{\pgfqpoint{4.400071in}{0.739656in}}%
\pgfpathlineto{\pgfqpoint{4.399774in}{0.739656in}}%
\pgfpathlineto{\pgfqpoint{4.399476in}{0.739656in}}%
\pgfpathlineto{\pgfqpoint{4.399179in}{0.739656in}}%
\pgfpathlineto{\pgfqpoint{4.398881in}{0.739656in}}%
\pgfpathlineto{\pgfqpoint{4.398584in}{0.739656in}}%
\pgfpathlineto{\pgfqpoint{4.398286in}{0.739656in}}%
\pgfpathlineto{\pgfqpoint{4.397989in}{0.739656in}}%
\pgfpathlineto{\pgfqpoint{4.397691in}{0.739656in}}%
\pgfpathlineto{\pgfqpoint{4.397394in}{0.739656in}}%
\pgfpathlineto{\pgfqpoint{4.397096in}{0.739656in}}%
\pgfpathlineto{\pgfqpoint{4.396799in}{0.739656in}}%
\pgfpathlineto{\pgfqpoint{4.396501in}{0.739656in}}%
\pgfpathlineto{\pgfqpoint{4.396204in}{0.739656in}}%
\pgfpathlineto{\pgfqpoint{4.395906in}{0.739656in}}%
\pgfpathlineto{\pgfqpoint{4.395609in}{0.739656in}}%
\pgfpathlineto{\pgfqpoint{4.395311in}{0.739656in}}%
\pgfpathlineto{\pgfqpoint{4.395014in}{0.739656in}}%
\pgfpathlineto{\pgfqpoint{4.394716in}{0.739656in}}%
\pgfpathlineto{\pgfqpoint{4.394419in}{0.739656in}}%
\pgfpathlineto{\pgfqpoint{4.394122in}{0.739656in}}%
\pgfpathlineto{\pgfqpoint{4.393824in}{0.739656in}}%
\pgfpathlineto{\pgfqpoint{4.393527in}{0.739656in}}%
\pgfpathlineto{\pgfqpoint{4.393229in}{0.739656in}}%
\pgfpathlineto{\pgfqpoint{4.392932in}{0.739656in}}%
\pgfpathlineto{\pgfqpoint{4.392634in}{0.739656in}}%
\pgfpathlineto{\pgfqpoint{4.392337in}{0.739656in}}%
\pgfpathlineto{\pgfqpoint{4.392039in}{0.739656in}}%
\pgfpathlineto{\pgfqpoint{4.391742in}{0.739656in}}%
\pgfpathlineto{\pgfqpoint{4.391444in}{0.739656in}}%
\pgfpathlineto{\pgfqpoint{4.391147in}{0.739656in}}%
\pgfpathlineto{\pgfqpoint{4.390849in}{0.739656in}}%
\pgfpathlineto{\pgfqpoint{4.390552in}{0.739656in}}%
\pgfpathlineto{\pgfqpoint{4.390254in}{0.739656in}}%
\pgfpathlineto{\pgfqpoint{4.389957in}{0.739656in}}%
\pgfpathlineto{\pgfqpoint{4.389659in}{0.739656in}}%
\pgfpathlineto{\pgfqpoint{4.389362in}{0.739656in}}%
\pgfpathlineto{\pgfqpoint{4.389064in}{0.739656in}}%
\pgfpathlineto{\pgfqpoint{4.388767in}{0.739656in}}%
\pgfpathlineto{\pgfqpoint{4.388469in}{0.739656in}}%
\pgfpathlineto{\pgfqpoint{4.388172in}{0.739656in}}%
\pgfpathlineto{\pgfqpoint{4.387874in}{0.739656in}}%
\pgfpathlineto{\pgfqpoint{4.387577in}{0.739656in}}%
\pgfpathlineto{\pgfqpoint{4.387280in}{0.739656in}}%
\pgfpathlineto{\pgfqpoint{4.386982in}{0.739656in}}%
\pgfpathlineto{\pgfqpoint{4.386685in}{0.739656in}}%
\pgfpathlineto{\pgfqpoint{4.386387in}{0.739656in}}%
\pgfpathlineto{\pgfqpoint{4.386090in}{0.739656in}}%
\pgfpathlineto{\pgfqpoint{4.385792in}{0.739656in}}%
\pgfpathlineto{\pgfqpoint{4.385495in}{0.739656in}}%
\pgfpathlineto{\pgfqpoint{4.385197in}{0.739656in}}%
\pgfpathlineto{\pgfqpoint{4.384900in}{0.739656in}}%
\pgfpathlineto{\pgfqpoint{4.384602in}{0.739656in}}%
\pgfpathlineto{\pgfqpoint{4.384305in}{0.739656in}}%
\pgfpathlineto{\pgfqpoint{4.384007in}{0.739656in}}%
\pgfpathlineto{\pgfqpoint{4.383710in}{0.739656in}}%
\pgfpathlineto{\pgfqpoint{4.383412in}{0.739656in}}%
\pgfpathlineto{\pgfqpoint{4.383115in}{0.739656in}}%
\pgfpathlineto{\pgfqpoint{4.382817in}{0.739656in}}%
\pgfpathlineto{\pgfqpoint{4.382520in}{0.739656in}}%
\pgfpathlineto{\pgfqpoint{4.382222in}{0.739656in}}%
\pgfpathlineto{\pgfqpoint{4.381925in}{0.739656in}}%
\pgfpathlineto{\pgfqpoint{4.381627in}{0.739656in}}%
\pgfpathlineto{\pgfqpoint{4.381330in}{0.739656in}}%
\pgfpathlineto{\pgfqpoint{4.381033in}{0.739656in}}%
\pgfpathlineto{\pgfqpoint{4.380735in}{0.739656in}}%
\pgfpathlineto{\pgfqpoint{4.380438in}{0.739656in}}%
\pgfpathlineto{\pgfqpoint{4.380140in}{0.739656in}}%
\pgfpathlineto{\pgfqpoint{4.379843in}{0.739656in}}%
\pgfpathlineto{\pgfqpoint{4.379545in}{0.739656in}}%
\pgfpathlineto{\pgfqpoint{4.379248in}{0.739656in}}%
\pgfpathlineto{\pgfqpoint{4.378950in}{0.739656in}}%
\pgfpathlineto{\pgfqpoint{4.378653in}{0.739656in}}%
\pgfpathlineto{\pgfqpoint{4.378355in}{0.739656in}}%
\pgfpathlineto{\pgfqpoint{4.378058in}{0.739656in}}%
\pgfpathlineto{\pgfqpoint{4.377760in}{0.739656in}}%
\pgfpathlineto{\pgfqpoint{4.377463in}{0.739656in}}%
\pgfpathlineto{\pgfqpoint{4.377165in}{0.739656in}}%
\pgfpathlineto{\pgfqpoint{4.376868in}{0.739656in}}%
\pgfpathlineto{\pgfqpoint{4.376570in}{0.739656in}}%
\pgfpathlineto{\pgfqpoint{4.376273in}{0.739656in}}%
\pgfpathlineto{\pgfqpoint{4.375975in}{0.739656in}}%
\pgfpathlineto{\pgfqpoint{4.375678in}{0.739656in}}%
\pgfpathlineto{\pgfqpoint{4.375380in}{0.739656in}}%
\pgfpathlineto{\pgfqpoint{4.375083in}{0.739656in}}%
\pgfpathlineto{\pgfqpoint{4.374785in}{0.739656in}}%
\pgfpathlineto{\pgfqpoint{4.374488in}{0.739656in}}%
\pgfpathlineto{\pgfqpoint{4.374191in}{0.739656in}}%
\pgfpathlineto{\pgfqpoint{4.373893in}{0.739656in}}%
\pgfpathlineto{\pgfqpoint{4.373596in}{0.739656in}}%
\pgfpathlineto{\pgfqpoint{4.373298in}{0.739656in}}%
\pgfpathlineto{\pgfqpoint{4.373001in}{0.739656in}}%
\pgfpathlineto{\pgfqpoint{4.372703in}{0.739656in}}%
\pgfpathlineto{\pgfqpoint{4.372406in}{0.739656in}}%
\pgfpathlineto{\pgfqpoint{4.372108in}{0.739656in}}%
\pgfpathlineto{\pgfqpoint{4.371811in}{0.739656in}}%
\pgfpathlineto{\pgfqpoint{4.371513in}{0.739656in}}%
\pgfpathlineto{\pgfqpoint{4.371216in}{0.739656in}}%
\pgfpathlineto{\pgfqpoint{4.370918in}{0.739656in}}%
\pgfpathlineto{\pgfqpoint{4.370621in}{0.739656in}}%
\pgfpathlineto{\pgfqpoint{4.370323in}{0.739656in}}%
\pgfpathlineto{\pgfqpoint{4.370026in}{0.739656in}}%
\pgfpathlineto{\pgfqpoint{4.369728in}{0.739656in}}%
\pgfpathlineto{\pgfqpoint{4.369431in}{0.739656in}}%
\pgfpathlineto{\pgfqpoint{4.369133in}{0.739656in}}%
\pgfpathlineto{\pgfqpoint{4.368836in}{0.739656in}}%
\pgfpathlineto{\pgfqpoint{4.368538in}{0.739656in}}%
\pgfpathlineto{\pgfqpoint{4.368241in}{0.739656in}}%
\pgfpathlineto{\pgfqpoint{4.367943in}{0.739656in}}%
\pgfpathlineto{\pgfqpoint{4.367646in}{0.739656in}}%
\pgfpathlineto{\pgfqpoint{4.367349in}{0.739656in}}%
\pgfpathlineto{\pgfqpoint{4.367051in}{0.739656in}}%
\pgfpathlineto{\pgfqpoint{4.366754in}{0.739656in}}%
\pgfpathlineto{\pgfqpoint{4.366456in}{0.739656in}}%
\pgfpathlineto{\pgfqpoint{4.366159in}{0.739656in}}%
\pgfpathlineto{\pgfqpoint{4.365861in}{0.739656in}}%
\pgfpathlineto{\pgfqpoint{4.365564in}{0.739656in}}%
\pgfpathlineto{\pgfqpoint{4.365266in}{0.739656in}}%
\pgfpathlineto{\pgfqpoint{4.364969in}{0.739656in}}%
\pgfpathlineto{\pgfqpoint{4.364671in}{0.739656in}}%
\pgfpathlineto{\pgfqpoint{4.364374in}{0.739656in}}%
\pgfpathlineto{\pgfqpoint{4.364076in}{0.739656in}}%
\pgfpathlineto{\pgfqpoint{4.363779in}{0.739656in}}%
\pgfpathlineto{\pgfqpoint{4.363481in}{0.739656in}}%
\pgfpathlineto{\pgfqpoint{4.363184in}{0.739656in}}%
\pgfpathlineto{\pgfqpoint{4.362886in}{0.739656in}}%
\pgfpathlineto{\pgfqpoint{4.362589in}{0.739656in}}%
\pgfpathlineto{\pgfqpoint{4.362291in}{0.739656in}}%
\pgfpathlineto{\pgfqpoint{4.361994in}{0.739656in}}%
\pgfpathlineto{\pgfqpoint{4.361696in}{0.739656in}}%
\pgfpathlineto{\pgfqpoint{4.361399in}{0.739656in}}%
\pgfpathlineto{\pgfqpoint{4.361102in}{0.739656in}}%
\pgfpathlineto{\pgfqpoint{4.360804in}{0.739656in}}%
\pgfpathlineto{\pgfqpoint{4.360507in}{0.739656in}}%
\pgfpathlineto{\pgfqpoint{4.360209in}{0.739656in}}%
\pgfpathlineto{\pgfqpoint{4.359912in}{0.739656in}}%
\pgfpathlineto{\pgfqpoint{4.359614in}{0.739656in}}%
\pgfpathlineto{\pgfqpoint{4.359317in}{0.739656in}}%
\pgfpathlineto{\pgfqpoint{4.359019in}{0.739656in}}%
\pgfpathlineto{\pgfqpoint{4.358722in}{0.739656in}}%
\pgfpathlineto{\pgfqpoint{4.358424in}{0.739656in}}%
\pgfpathlineto{\pgfqpoint{4.358127in}{0.739656in}}%
\pgfpathlineto{\pgfqpoint{4.357829in}{0.739656in}}%
\pgfpathlineto{\pgfqpoint{4.357532in}{0.739656in}}%
\pgfpathlineto{\pgfqpoint{4.357234in}{0.739656in}}%
\pgfpathlineto{\pgfqpoint{4.356937in}{0.739656in}}%
\pgfpathlineto{\pgfqpoint{4.356639in}{0.739656in}}%
\pgfpathlineto{\pgfqpoint{4.356342in}{0.739656in}}%
\pgfpathlineto{\pgfqpoint{4.356044in}{0.739656in}}%
\pgfpathlineto{\pgfqpoint{4.355747in}{0.739656in}}%
\pgfpathlineto{\pgfqpoint{4.355449in}{0.739656in}}%
\pgfpathlineto{\pgfqpoint{4.355152in}{0.739656in}}%
\pgfpathlineto{\pgfqpoint{4.354854in}{0.739656in}}%
\pgfpathlineto{\pgfqpoint{4.354557in}{0.739656in}}%
\pgfpathlineto{\pgfqpoint{4.354260in}{0.739656in}}%
\pgfpathlineto{\pgfqpoint{4.353962in}{0.739656in}}%
\pgfpathlineto{\pgfqpoint{4.353665in}{0.739656in}}%
\pgfpathlineto{\pgfqpoint{4.353367in}{0.739656in}}%
\pgfpathlineto{\pgfqpoint{4.353070in}{0.739656in}}%
\pgfpathlineto{\pgfqpoint{4.352772in}{0.739656in}}%
\pgfpathlineto{\pgfqpoint{4.352475in}{0.739656in}}%
\pgfpathlineto{\pgfqpoint{4.352177in}{0.739656in}}%
\pgfpathlineto{\pgfqpoint{4.351880in}{0.739656in}}%
\pgfpathlineto{\pgfqpoint{4.351582in}{0.739656in}}%
\pgfpathlineto{\pgfqpoint{4.351285in}{0.739656in}}%
\pgfpathlineto{\pgfqpoint{4.350987in}{0.739656in}}%
\pgfpathlineto{\pgfqpoint{4.350690in}{0.739656in}}%
\pgfpathlineto{\pgfqpoint{4.350392in}{0.739656in}}%
\pgfpathlineto{\pgfqpoint{4.350095in}{0.739656in}}%
\pgfpathlineto{\pgfqpoint{4.349797in}{0.739656in}}%
\pgfpathlineto{\pgfqpoint{4.349500in}{0.739656in}}%
\pgfpathlineto{\pgfqpoint{4.349202in}{0.739656in}}%
\pgfpathlineto{\pgfqpoint{4.348905in}{0.739656in}}%
\pgfpathlineto{\pgfqpoint{4.348607in}{0.739656in}}%
\pgfpathlineto{\pgfqpoint{4.348310in}{0.739656in}}%
\pgfpathlineto{\pgfqpoint{4.348012in}{0.739656in}}%
\pgfpathlineto{\pgfqpoint{4.347715in}{0.739656in}}%
\pgfpathlineto{\pgfqpoint{4.347418in}{0.739656in}}%
\pgfpathlineto{\pgfqpoint{4.347120in}{0.739656in}}%
\pgfpathlineto{\pgfqpoint{4.346823in}{0.739656in}}%
\pgfpathlineto{\pgfqpoint{4.346525in}{0.739656in}}%
\pgfpathlineto{\pgfqpoint{4.346228in}{0.739656in}}%
\pgfpathlineto{\pgfqpoint{4.345930in}{0.739656in}}%
\pgfpathlineto{\pgfqpoint{4.345633in}{0.739656in}}%
\pgfpathlineto{\pgfqpoint{4.345335in}{0.739656in}}%
\pgfpathlineto{\pgfqpoint{4.345038in}{0.739656in}}%
\pgfpathlineto{\pgfqpoint{4.344740in}{0.739656in}}%
\pgfpathlineto{\pgfqpoint{4.344443in}{0.739656in}}%
\pgfpathlineto{\pgfqpoint{4.344145in}{0.739656in}}%
\pgfpathlineto{\pgfqpoint{4.343848in}{0.739656in}}%
\pgfpathlineto{\pgfqpoint{4.343550in}{0.739656in}}%
\pgfpathlineto{\pgfqpoint{4.343253in}{0.739656in}}%
\pgfpathlineto{\pgfqpoint{4.342955in}{0.739656in}}%
\pgfpathlineto{\pgfqpoint{4.342658in}{0.739656in}}%
\pgfpathlineto{\pgfqpoint{4.342360in}{0.739656in}}%
\pgfpathlineto{\pgfqpoint{4.342063in}{0.739656in}}%
\pgfpathlineto{\pgfqpoint{4.341765in}{0.739656in}}%
\pgfpathlineto{\pgfqpoint{4.341468in}{0.739656in}}%
\pgfpathlineto{\pgfqpoint{4.341171in}{0.739656in}}%
\pgfpathlineto{\pgfqpoint{4.340873in}{0.739656in}}%
\pgfpathlineto{\pgfqpoint{4.340576in}{0.739656in}}%
\pgfpathlineto{\pgfqpoint{4.340278in}{0.739656in}}%
\pgfpathlineto{\pgfqpoint{4.339981in}{0.739656in}}%
\pgfpathlineto{\pgfqpoint{4.339683in}{0.739656in}}%
\pgfpathlineto{\pgfqpoint{4.339386in}{0.739656in}}%
\pgfpathlineto{\pgfqpoint{4.339088in}{0.739656in}}%
\pgfpathlineto{\pgfqpoint{4.338791in}{0.739656in}}%
\pgfpathlineto{\pgfqpoint{4.338493in}{0.739656in}}%
\pgfpathlineto{\pgfqpoint{4.338196in}{0.739656in}}%
\pgfpathlineto{\pgfqpoint{4.337898in}{0.739656in}}%
\pgfpathlineto{\pgfqpoint{4.337601in}{0.739656in}}%
\pgfpathlineto{\pgfqpoint{4.337303in}{0.739656in}}%
\pgfpathlineto{\pgfqpoint{4.337006in}{0.739656in}}%
\pgfpathlineto{\pgfqpoint{4.336708in}{0.739656in}}%
\pgfpathlineto{\pgfqpoint{4.336411in}{0.739656in}}%
\pgfpathlineto{\pgfqpoint{4.336113in}{0.739656in}}%
\pgfpathlineto{\pgfqpoint{4.335816in}{0.739656in}}%
\pgfpathlineto{\pgfqpoint{4.335518in}{0.739656in}}%
\pgfpathlineto{\pgfqpoint{4.335221in}{0.739656in}}%
\pgfpathlineto{\pgfqpoint{4.334923in}{0.739656in}}%
\pgfpathlineto{\pgfqpoint{4.334626in}{0.739656in}}%
\pgfpathlineto{\pgfqpoint{4.334329in}{0.739656in}}%
\pgfpathlineto{\pgfqpoint{4.334031in}{0.739656in}}%
\pgfpathlineto{\pgfqpoint{4.333734in}{0.739656in}}%
\pgfpathlineto{\pgfqpoint{4.333436in}{0.739656in}}%
\pgfpathlineto{\pgfqpoint{4.333139in}{0.739656in}}%
\pgfpathlineto{\pgfqpoint{4.332841in}{0.739656in}}%
\pgfpathlineto{\pgfqpoint{4.332544in}{0.739656in}}%
\pgfpathlineto{\pgfqpoint{4.332246in}{0.739656in}}%
\pgfpathlineto{\pgfqpoint{4.331949in}{0.739656in}}%
\pgfpathlineto{\pgfqpoint{4.331651in}{0.739656in}}%
\pgfpathlineto{\pgfqpoint{4.331354in}{0.739656in}}%
\pgfpathlineto{\pgfqpoint{4.331056in}{0.739656in}}%
\pgfpathlineto{\pgfqpoint{4.330759in}{0.739656in}}%
\pgfpathlineto{\pgfqpoint{4.330461in}{0.739656in}}%
\pgfpathlineto{\pgfqpoint{4.330164in}{0.739656in}}%
\pgfpathlineto{\pgfqpoint{4.329866in}{0.739656in}}%
\pgfpathlineto{\pgfqpoint{4.329569in}{0.739656in}}%
\pgfpathlineto{\pgfqpoint{4.329271in}{0.739656in}}%
\pgfpathlineto{\pgfqpoint{4.328974in}{0.739656in}}%
\pgfpathlineto{\pgfqpoint{4.328676in}{0.739656in}}%
\pgfpathlineto{\pgfqpoint{4.328379in}{0.739656in}}%
\pgfpathlineto{\pgfqpoint{4.328081in}{0.739656in}}%
\pgfpathlineto{\pgfqpoint{4.327784in}{0.739656in}}%
\pgfpathlineto{\pgfqpoint{4.327487in}{0.739656in}}%
\pgfpathlineto{\pgfqpoint{4.327189in}{0.739656in}}%
\pgfpathlineto{\pgfqpoint{4.326892in}{0.739656in}}%
\pgfpathlineto{\pgfqpoint{4.326594in}{0.739656in}}%
\pgfpathlineto{\pgfqpoint{4.326297in}{0.739656in}}%
\pgfpathlineto{\pgfqpoint{4.325999in}{0.739656in}}%
\pgfpathlineto{\pgfqpoint{4.325702in}{0.739656in}}%
\pgfpathlineto{\pgfqpoint{4.325404in}{0.739656in}}%
\pgfpathlineto{\pgfqpoint{4.325107in}{0.739656in}}%
\pgfpathlineto{\pgfqpoint{4.324809in}{0.739656in}}%
\pgfpathlineto{\pgfqpoint{4.324512in}{0.739656in}}%
\pgfpathlineto{\pgfqpoint{4.324214in}{0.739656in}}%
\pgfpathlineto{\pgfqpoint{4.323917in}{0.739656in}}%
\pgfpathlineto{\pgfqpoint{4.323619in}{0.739656in}}%
\pgfpathlineto{\pgfqpoint{4.323322in}{0.739656in}}%
\pgfpathlineto{\pgfqpoint{4.323024in}{0.739656in}}%
\pgfpathlineto{\pgfqpoint{4.322727in}{0.739656in}}%
\pgfpathlineto{\pgfqpoint{4.322429in}{0.739656in}}%
\pgfpathlineto{\pgfqpoint{4.322132in}{0.739656in}}%
\pgfpathlineto{\pgfqpoint{4.321834in}{0.739656in}}%
\pgfpathlineto{\pgfqpoint{4.321537in}{0.739656in}}%
\pgfpathlineto{\pgfqpoint{4.321240in}{0.739656in}}%
\pgfpathlineto{\pgfqpoint{4.320942in}{0.739656in}}%
\pgfpathlineto{\pgfqpoint{4.320645in}{0.739656in}}%
\pgfpathlineto{\pgfqpoint{4.320347in}{0.739656in}}%
\pgfpathlineto{\pgfqpoint{4.320050in}{0.739656in}}%
\pgfpathlineto{\pgfqpoint{4.319752in}{0.739656in}}%
\pgfpathlineto{\pgfqpoint{4.319455in}{0.739656in}}%
\pgfpathlineto{\pgfqpoint{4.319157in}{0.739656in}}%
\pgfpathlineto{\pgfqpoint{4.318860in}{0.739656in}}%
\pgfpathlineto{\pgfqpoint{4.318562in}{0.739656in}}%
\pgfpathlineto{\pgfqpoint{4.318265in}{0.739656in}}%
\pgfpathlineto{\pgfqpoint{4.317967in}{0.739656in}}%
\pgfpathlineto{\pgfqpoint{4.317670in}{0.739656in}}%
\pgfpathlineto{\pgfqpoint{4.317372in}{0.739656in}}%
\pgfpathlineto{\pgfqpoint{4.317075in}{0.739656in}}%
\pgfpathlineto{\pgfqpoint{4.316777in}{0.739656in}}%
\pgfpathlineto{\pgfqpoint{4.316480in}{0.739656in}}%
\pgfpathlineto{\pgfqpoint{4.316182in}{0.739656in}}%
\pgfpathlineto{\pgfqpoint{4.315885in}{0.739656in}}%
\pgfpathlineto{\pgfqpoint{4.315587in}{0.739656in}}%
\pgfpathlineto{\pgfqpoint{4.315290in}{0.739656in}}%
\pgfpathlineto{\pgfqpoint{4.314992in}{0.739656in}}%
\pgfpathlineto{\pgfqpoint{4.314695in}{0.739656in}}%
\pgfpathlineto{\pgfqpoint{4.314398in}{0.739656in}}%
\pgfpathlineto{\pgfqpoint{4.314100in}{0.739656in}}%
\pgfpathlineto{\pgfqpoint{4.313803in}{0.739656in}}%
\pgfpathlineto{\pgfqpoint{4.313505in}{0.739656in}}%
\pgfpathlineto{\pgfqpoint{4.313208in}{0.739656in}}%
\pgfpathlineto{\pgfqpoint{4.312910in}{0.739656in}}%
\pgfpathlineto{\pgfqpoint{4.312613in}{0.739656in}}%
\pgfpathlineto{\pgfqpoint{4.312315in}{0.739656in}}%
\pgfpathlineto{\pgfqpoint{4.312018in}{0.739656in}}%
\pgfpathlineto{\pgfqpoint{4.311720in}{0.739656in}}%
\pgfpathlineto{\pgfqpoint{4.311423in}{0.739656in}}%
\pgfpathlineto{\pgfqpoint{4.311125in}{0.739656in}}%
\pgfpathlineto{\pgfqpoint{4.310828in}{0.739656in}}%
\pgfpathlineto{\pgfqpoint{4.310530in}{0.739656in}}%
\pgfpathlineto{\pgfqpoint{4.310233in}{0.739656in}}%
\pgfpathlineto{\pgfqpoint{4.309935in}{0.739656in}}%
\pgfpathlineto{\pgfqpoint{4.309638in}{0.739656in}}%
\pgfpathlineto{\pgfqpoint{4.309340in}{0.739656in}}%
\pgfpathlineto{\pgfqpoint{4.309043in}{0.739656in}}%
\pgfpathlineto{\pgfqpoint{4.308745in}{0.739656in}}%
\pgfpathlineto{\pgfqpoint{4.308448in}{0.739656in}}%
\pgfpathlineto{\pgfqpoint{4.308150in}{0.739656in}}%
\pgfpathlineto{\pgfqpoint{4.307853in}{0.739656in}}%
\pgfpathlineto{\pgfqpoint{4.307556in}{0.739656in}}%
\pgfpathlineto{\pgfqpoint{4.307258in}{0.739656in}}%
\pgfpathlineto{\pgfqpoint{4.306961in}{0.739656in}}%
\pgfpathlineto{\pgfqpoint{4.306663in}{0.739656in}}%
\pgfpathlineto{\pgfqpoint{4.306366in}{0.739656in}}%
\pgfpathlineto{\pgfqpoint{4.306068in}{0.739656in}}%
\pgfpathlineto{\pgfqpoint{4.305771in}{0.739656in}}%
\pgfpathlineto{\pgfqpoint{4.305473in}{0.739656in}}%
\pgfpathlineto{\pgfqpoint{4.305176in}{0.739656in}}%
\pgfpathlineto{\pgfqpoint{4.304878in}{0.739656in}}%
\pgfpathlineto{\pgfqpoint{4.304581in}{0.739656in}}%
\pgfpathlineto{\pgfqpoint{4.304283in}{0.739656in}}%
\pgfpathlineto{\pgfqpoint{4.303986in}{0.739656in}}%
\pgfpathlineto{\pgfqpoint{4.303688in}{0.739656in}}%
\pgfpathlineto{\pgfqpoint{4.303391in}{0.739656in}}%
\pgfpathlineto{\pgfqpoint{4.303093in}{0.739656in}}%
\pgfpathlineto{\pgfqpoint{4.302796in}{0.739656in}}%
\pgfpathlineto{\pgfqpoint{4.302498in}{0.739656in}}%
\pgfpathlineto{\pgfqpoint{4.302201in}{0.739656in}}%
\pgfpathlineto{\pgfqpoint{4.301903in}{0.739656in}}%
\pgfpathlineto{\pgfqpoint{4.301606in}{0.739656in}}%
\pgfpathlineto{\pgfqpoint{4.301309in}{0.739656in}}%
\pgfpathlineto{\pgfqpoint{4.301011in}{0.739656in}}%
\pgfpathlineto{\pgfqpoint{4.300714in}{0.739656in}}%
\pgfpathlineto{\pgfqpoint{4.300416in}{0.739656in}}%
\pgfpathlineto{\pgfqpoint{4.300119in}{0.739656in}}%
\pgfpathlineto{\pgfqpoint{4.299821in}{0.739656in}}%
\pgfpathlineto{\pgfqpoint{4.299524in}{0.739656in}}%
\pgfpathlineto{\pgfqpoint{4.299226in}{0.739656in}}%
\pgfpathlineto{\pgfqpoint{4.298929in}{0.739656in}}%
\pgfpathlineto{\pgfqpoint{4.298631in}{0.739656in}}%
\pgfpathlineto{\pgfqpoint{4.298334in}{0.739656in}}%
\pgfpathlineto{\pgfqpoint{4.298036in}{0.739656in}}%
\pgfpathlineto{\pgfqpoint{4.297739in}{0.739656in}}%
\pgfpathlineto{\pgfqpoint{4.297441in}{0.739656in}}%
\pgfpathlineto{\pgfqpoint{4.297144in}{0.739656in}}%
\pgfpathlineto{\pgfqpoint{4.296846in}{0.739656in}}%
\pgfpathlineto{\pgfqpoint{4.296549in}{0.739656in}}%
\pgfpathlineto{\pgfqpoint{4.296251in}{0.739656in}}%
\pgfpathlineto{\pgfqpoint{4.295954in}{0.739656in}}%
\pgfpathlineto{\pgfqpoint{4.295656in}{0.739656in}}%
\pgfpathlineto{\pgfqpoint{4.295359in}{0.739656in}}%
\pgfpathlineto{\pgfqpoint{4.295061in}{0.739656in}}%
\pgfpathlineto{\pgfqpoint{4.294764in}{0.739656in}}%
\pgfpathlineto{\pgfqpoint{4.294467in}{0.739656in}}%
\pgfpathlineto{\pgfqpoint{4.294169in}{0.739656in}}%
\pgfpathlineto{\pgfqpoint{4.293872in}{0.739656in}}%
\pgfpathlineto{\pgfqpoint{4.293574in}{0.739656in}}%
\pgfpathlineto{\pgfqpoint{4.293277in}{0.739656in}}%
\pgfpathlineto{\pgfqpoint{4.292979in}{0.739656in}}%
\pgfpathlineto{\pgfqpoint{4.292682in}{0.739656in}}%
\pgfpathlineto{\pgfqpoint{4.292384in}{0.739656in}}%
\pgfpathlineto{\pgfqpoint{4.292087in}{0.739656in}}%
\pgfpathlineto{\pgfqpoint{4.291789in}{0.739656in}}%
\pgfpathlineto{\pgfqpoint{4.291492in}{0.739656in}}%
\pgfpathlineto{\pgfqpoint{4.291194in}{0.739656in}}%
\pgfpathlineto{\pgfqpoint{4.290897in}{0.739656in}}%
\pgfpathlineto{\pgfqpoint{4.290599in}{0.739656in}}%
\pgfpathlineto{\pgfqpoint{4.290302in}{0.739656in}}%
\pgfpathlineto{\pgfqpoint{4.290004in}{0.739656in}}%
\pgfpathlineto{\pgfqpoint{4.289707in}{0.739656in}}%
\pgfpathlineto{\pgfqpoint{4.289409in}{0.739656in}}%
\pgfpathlineto{\pgfqpoint{4.289112in}{0.739656in}}%
\pgfpathlineto{\pgfqpoint{4.288814in}{0.739656in}}%
\pgfpathlineto{\pgfqpoint{4.288517in}{0.739656in}}%
\pgfpathlineto{\pgfqpoint{4.288219in}{0.739656in}}%
\pgfpathlineto{\pgfqpoint{4.287922in}{0.739656in}}%
\pgfpathlineto{\pgfqpoint{4.287625in}{0.739656in}}%
\pgfpathlineto{\pgfqpoint{4.287327in}{0.739656in}}%
\pgfpathlineto{\pgfqpoint{4.287030in}{0.739656in}}%
\pgfpathlineto{\pgfqpoint{4.286732in}{0.739656in}}%
\pgfpathlineto{\pgfqpoint{4.286435in}{0.739656in}}%
\pgfpathlineto{\pgfqpoint{4.286137in}{0.739656in}}%
\pgfpathlineto{\pgfqpoint{4.285840in}{0.739656in}}%
\pgfpathlineto{\pgfqpoint{4.285542in}{0.739656in}}%
\pgfpathlineto{\pgfqpoint{4.285245in}{0.739656in}}%
\pgfpathlineto{\pgfqpoint{4.284947in}{0.739656in}}%
\pgfpathlineto{\pgfqpoint{4.284650in}{0.739656in}}%
\pgfpathlineto{\pgfqpoint{4.284352in}{0.739656in}}%
\pgfpathlineto{\pgfqpoint{4.284055in}{0.739656in}}%
\pgfpathlineto{\pgfqpoint{4.283757in}{0.739656in}}%
\pgfpathlineto{\pgfqpoint{4.283460in}{0.739656in}}%
\pgfpathlineto{\pgfqpoint{4.283162in}{0.739656in}}%
\pgfpathlineto{\pgfqpoint{4.282865in}{0.739656in}}%
\pgfpathlineto{\pgfqpoint{4.282567in}{0.739656in}}%
\pgfpathlineto{\pgfqpoint{4.282270in}{0.739656in}}%
\pgfpathlineto{\pgfqpoint{4.281972in}{0.739656in}}%
\pgfpathlineto{\pgfqpoint{4.281675in}{0.739656in}}%
\pgfpathlineto{\pgfqpoint{4.281378in}{0.739656in}}%
\pgfpathlineto{\pgfqpoint{4.281080in}{0.739656in}}%
\pgfpathlineto{\pgfqpoint{4.280783in}{0.739656in}}%
\pgfpathlineto{\pgfqpoint{4.280485in}{0.739656in}}%
\pgfpathlineto{\pgfqpoint{4.280188in}{0.739656in}}%
\pgfpathlineto{\pgfqpoint{4.279890in}{0.739656in}}%
\pgfpathlineto{\pgfqpoint{4.279593in}{0.739656in}}%
\pgfpathlineto{\pgfqpoint{4.279295in}{0.739656in}}%
\pgfpathlineto{\pgfqpoint{4.278998in}{0.739656in}}%
\pgfpathlineto{\pgfqpoint{4.278700in}{0.739656in}}%
\pgfpathlineto{\pgfqpoint{4.278403in}{0.739656in}}%
\pgfpathlineto{\pgfqpoint{4.278105in}{0.739656in}}%
\pgfpathlineto{\pgfqpoint{4.277808in}{0.739656in}}%
\pgfpathlineto{\pgfqpoint{4.277510in}{0.739656in}}%
\pgfpathlineto{\pgfqpoint{4.277213in}{0.739656in}}%
\pgfpathlineto{\pgfqpoint{4.276915in}{0.739656in}}%
\pgfpathlineto{\pgfqpoint{4.276618in}{0.739656in}}%
\pgfpathlineto{\pgfqpoint{4.276320in}{0.739656in}}%
\pgfpathlineto{\pgfqpoint{4.276023in}{0.739656in}}%
\pgfpathlineto{\pgfqpoint{4.275725in}{0.739656in}}%
\pgfpathlineto{\pgfqpoint{4.275428in}{0.739656in}}%
\pgfpathlineto{\pgfqpoint{4.275130in}{0.739656in}}%
\pgfpathlineto{\pgfqpoint{4.274833in}{0.739656in}}%
\pgfpathlineto{\pgfqpoint{4.274536in}{0.739656in}}%
\pgfpathlineto{\pgfqpoint{4.274238in}{0.739656in}}%
\pgfpathlineto{\pgfqpoint{4.273941in}{0.739656in}}%
\pgfpathlineto{\pgfqpoint{4.273643in}{0.739656in}}%
\pgfpathlineto{\pgfqpoint{4.273346in}{0.739656in}}%
\pgfpathlineto{\pgfqpoint{4.273048in}{0.739656in}}%
\pgfpathlineto{\pgfqpoint{4.272751in}{0.739656in}}%
\pgfpathlineto{\pgfqpoint{4.272453in}{0.739656in}}%
\pgfpathlineto{\pgfqpoint{4.272156in}{0.739656in}}%
\pgfpathlineto{\pgfqpoint{4.271858in}{0.739656in}}%
\pgfpathlineto{\pgfqpoint{4.271561in}{0.739656in}}%
\pgfpathlineto{\pgfqpoint{4.271263in}{0.739656in}}%
\pgfpathlineto{\pgfqpoint{4.270966in}{0.739656in}}%
\pgfpathlineto{\pgfqpoint{4.270668in}{0.739656in}}%
\pgfpathlineto{\pgfqpoint{4.270371in}{0.739656in}}%
\pgfpathlineto{\pgfqpoint{4.270073in}{0.739656in}}%
\pgfpathlineto{\pgfqpoint{4.269776in}{0.739656in}}%
\pgfpathlineto{\pgfqpoint{4.269478in}{0.739656in}}%
\pgfpathlineto{\pgfqpoint{4.269181in}{0.739656in}}%
\pgfpathlineto{\pgfqpoint{4.268883in}{0.739656in}}%
\pgfpathlineto{\pgfqpoint{4.268586in}{0.739656in}}%
\pgfpathlineto{\pgfqpoint{4.268288in}{0.739656in}}%
\pgfpathlineto{\pgfqpoint{4.267991in}{0.739656in}}%
\pgfpathlineto{\pgfqpoint{4.267694in}{0.739656in}}%
\pgfpathlineto{\pgfqpoint{4.267396in}{0.739656in}}%
\pgfpathlineto{\pgfqpoint{4.267099in}{0.739656in}}%
\pgfpathlineto{\pgfqpoint{4.266801in}{0.739656in}}%
\pgfpathlineto{\pgfqpoint{4.266504in}{0.739656in}}%
\pgfpathlineto{\pgfqpoint{4.266206in}{0.739656in}}%
\pgfpathlineto{\pgfqpoint{4.265909in}{0.739656in}}%
\pgfpathlineto{\pgfqpoint{4.265611in}{0.739656in}}%
\pgfpathlineto{\pgfqpoint{4.265314in}{0.739656in}}%
\pgfpathlineto{\pgfqpoint{4.265016in}{0.739656in}}%
\pgfpathlineto{\pgfqpoint{4.264719in}{0.739656in}}%
\pgfpathlineto{\pgfqpoint{4.264421in}{0.739656in}}%
\pgfpathlineto{\pgfqpoint{4.264124in}{0.739656in}}%
\pgfpathlineto{\pgfqpoint{4.263826in}{0.739656in}}%
\pgfpathlineto{\pgfqpoint{4.263529in}{0.739656in}}%
\pgfpathlineto{\pgfqpoint{4.263231in}{0.739656in}}%
\pgfpathlineto{\pgfqpoint{4.262934in}{0.739656in}}%
\pgfpathlineto{\pgfqpoint{4.262636in}{0.739656in}}%
\pgfpathlineto{\pgfqpoint{4.262339in}{0.739656in}}%
\pgfpathlineto{\pgfqpoint{4.262041in}{0.739656in}}%
\pgfpathlineto{\pgfqpoint{4.261744in}{0.739656in}}%
\pgfpathlineto{\pgfqpoint{4.261447in}{0.739656in}}%
\pgfpathlineto{\pgfqpoint{4.261149in}{0.739656in}}%
\pgfpathlineto{\pgfqpoint{4.260852in}{0.739656in}}%
\pgfpathlineto{\pgfqpoint{4.260554in}{0.739656in}}%
\pgfpathlineto{\pgfqpoint{4.260257in}{0.739656in}}%
\pgfpathlineto{\pgfqpoint{4.259959in}{0.739656in}}%
\pgfpathlineto{\pgfqpoint{4.259662in}{0.739656in}}%
\pgfpathlineto{\pgfqpoint{4.259364in}{0.739656in}}%
\pgfpathlineto{\pgfqpoint{4.259067in}{0.739656in}}%
\pgfpathlineto{\pgfqpoint{4.258769in}{0.739656in}}%
\pgfpathlineto{\pgfqpoint{4.258472in}{0.739656in}}%
\pgfpathlineto{\pgfqpoint{4.258174in}{0.739656in}}%
\pgfpathlineto{\pgfqpoint{4.257877in}{0.739656in}}%
\pgfpathlineto{\pgfqpoint{4.257579in}{0.739656in}}%
\pgfpathlineto{\pgfqpoint{4.257282in}{0.739656in}}%
\pgfpathlineto{\pgfqpoint{4.256984in}{0.739656in}}%
\pgfpathlineto{\pgfqpoint{4.256687in}{0.739656in}}%
\pgfpathlineto{\pgfqpoint{4.256389in}{0.739656in}}%
\pgfpathlineto{\pgfqpoint{4.256092in}{0.739656in}}%
\pgfpathlineto{\pgfqpoint{4.255794in}{0.739656in}}%
\pgfpathlineto{\pgfqpoint{4.255497in}{0.739656in}}%
\pgfpathlineto{\pgfqpoint{4.255199in}{0.739656in}}%
\pgfpathlineto{\pgfqpoint{4.254902in}{0.739656in}}%
\pgfpathlineto{\pgfqpoint{4.254605in}{0.739656in}}%
\pgfpathlineto{\pgfqpoint{4.254307in}{0.739656in}}%
\pgfpathlineto{\pgfqpoint{4.254010in}{0.739656in}}%
\pgfpathlineto{\pgfqpoint{4.253712in}{0.739656in}}%
\pgfpathlineto{\pgfqpoint{4.253415in}{0.739656in}}%
\pgfpathlineto{\pgfqpoint{4.253117in}{0.739656in}}%
\pgfpathlineto{\pgfqpoint{4.252820in}{0.739656in}}%
\pgfpathlineto{\pgfqpoint{4.252522in}{0.739656in}}%
\pgfpathlineto{\pgfqpoint{4.252225in}{0.739656in}}%
\pgfpathlineto{\pgfqpoint{4.251927in}{0.739656in}}%
\pgfpathlineto{\pgfqpoint{4.251630in}{0.739656in}}%
\pgfpathlineto{\pgfqpoint{4.251332in}{0.739656in}}%
\pgfpathlineto{\pgfqpoint{4.251035in}{0.739656in}}%
\pgfpathlineto{\pgfqpoint{4.250737in}{0.739656in}}%
\pgfpathlineto{\pgfqpoint{4.250440in}{0.739656in}}%
\pgfpathlineto{\pgfqpoint{4.250142in}{0.739656in}}%
\pgfpathlineto{\pgfqpoint{4.249845in}{0.739656in}}%
\pgfpathlineto{\pgfqpoint{4.249547in}{0.739656in}}%
\pgfpathlineto{\pgfqpoint{4.249250in}{0.739656in}}%
\pgfpathlineto{\pgfqpoint{4.248952in}{0.739656in}}%
\pgfpathlineto{\pgfqpoint{4.248655in}{0.739656in}}%
\pgfpathlineto{\pgfqpoint{4.248357in}{0.739656in}}%
\pgfpathlineto{\pgfqpoint{4.248060in}{0.739656in}}%
\pgfpathlineto{\pgfqpoint{4.247763in}{0.739656in}}%
\pgfpathlineto{\pgfqpoint{4.247465in}{0.739656in}}%
\pgfpathlineto{\pgfqpoint{4.247168in}{0.739656in}}%
\pgfpathlineto{\pgfqpoint{4.246870in}{0.739656in}}%
\pgfpathlineto{\pgfqpoint{4.246573in}{0.739656in}}%
\pgfpathlineto{\pgfqpoint{4.246275in}{0.739656in}}%
\pgfpathlineto{\pgfqpoint{4.245978in}{0.739656in}}%
\pgfpathlineto{\pgfqpoint{4.245680in}{0.739656in}}%
\pgfpathlineto{\pgfqpoint{4.245383in}{0.739656in}}%
\pgfpathlineto{\pgfqpoint{4.245085in}{0.739656in}}%
\pgfpathlineto{\pgfqpoint{4.244788in}{0.739656in}}%
\pgfpathlineto{\pgfqpoint{4.244490in}{0.739656in}}%
\pgfpathlineto{\pgfqpoint{4.244193in}{0.739656in}}%
\pgfpathlineto{\pgfqpoint{4.243895in}{0.739656in}}%
\pgfpathlineto{\pgfqpoint{4.243598in}{0.739656in}}%
\pgfpathlineto{\pgfqpoint{4.243300in}{0.739656in}}%
\pgfpathlineto{\pgfqpoint{4.243003in}{0.739656in}}%
\pgfpathlineto{\pgfqpoint{4.242705in}{0.739656in}}%
\pgfpathlineto{\pgfqpoint{4.242408in}{0.739656in}}%
\pgfpathlineto{\pgfqpoint{4.242110in}{0.739656in}}%
\pgfpathlineto{\pgfqpoint{4.241813in}{0.739656in}}%
\pgfpathlineto{\pgfqpoint{4.241516in}{0.739656in}}%
\pgfpathlineto{\pgfqpoint{4.241218in}{0.739656in}}%
\pgfpathlineto{\pgfqpoint{4.240921in}{0.739656in}}%
\pgfpathlineto{\pgfqpoint{4.240623in}{0.739656in}}%
\pgfpathlineto{\pgfqpoint{4.240326in}{0.739656in}}%
\pgfpathlineto{\pgfqpoint{4.240028in}{0.739656in}}%
\pgfpathlineto{\pgfqpoint{4.239731in}{0.739656in}}%
\pgfpathlineto{\pgfqpoint{4.239433in}{0.739656in}}%
\pgfpathlineto{\pgfqpoint{4.239136in}{0.739656in}}%
\pgfpathlineto{\pgfqpoint{4.238838in}{0.739656in}}%
\pgfpathlineto{\pgfqpoint{4.238541in}{0.739656in}}%
\pgfpathlineto{\pgfqpoint{4.238243in}{0.739656in}}%
\pgfpathlineto{\pgfqpoint{4.237946in}{0.739656in}}%
\pgfpathlineto{\pgfqpoint{4.237648in}{0.739656in}}%
\pgfpathlineto{\pgfqpoint{4.237351in}{0.739656in}}%
\pgfpathlineto{\pgfqpoint{4.237053in}{0.739656in}}%
\pgfpathlineto{\pgfqpoint{4.236756in}{0.739656in}}%
\pgfpathlineto{\pgfqpoint{4.236458in}{0.739656in}}%
\pgfpathlineto{\pgfqpoint{4.236161in}{0.739656in}}%
\pgfpathlineto{\pgfqpoint{4.235863in}{0.739656in}}%
\pgfpathlineto{\pgfqpoint{4.235566in}{0.739656in}}%
\pgfpathlineto{\pgfqpoint{4.235268in}{0.739656in}}%
\pgfpathlineto{\pgfqpoint{4.234971in}{0.739656in}}%
\pgfpathlineto{\pgfqpoint{4.234674in}{0.739656in}}%
\pgfpathlineto{\pgfqpoint{4.234376in}{0.739656in}}%
\pgfpathlineto{\pgfqpoint{4.234079in}{0.739656in}}%
\pgfpathlineto{\pgfqpoint{4.233781in}{0.739656in}}%
\pgfpathlineto{\pgfqpoint{4.233484in}{0.739656in}}%
\pgfpathlineto{\pgfqpoint{4.233186in}{0.739656in}}%
\pgfpathlineto{\pgfqpoint{4.232889in}{0.739656in}}%
\pgfpathlineto{\pgfqpoint{4.232591in}{0.739656in}}%
\pgfpathlineto{\pgfqpoint{4.232294in}{0.739656in}}%
\pgfpathlineto{\pgfqpoint{4.231996in}{0.739656in}}%
\pgfpathlineto{\pgfqpoint{4.231699in}{0.739656in}}%
\pgfpathlineto{\pgfqpoint{4.231401in}{0.739656in}}%
\pgfpathlineto{\pgfqpoint{4.231104in}{0.739656in}}%
\pgfpathlineto{\pgfqpoint{4.230806in}{0.739656in}}%
\pgfpathlineto{\pgfqpoint{4.230509in}{0.739656in}}%
\pgfpathlineto{\pgfqpoint{4.230211in}{0.739656in}}%
\pgfpathlineto{\pgfqpoint{4.229914in}{0.739656in}}%
\pgfpathlineto{\pgfqpoint{4.229616in}{0.739656in}}%
\pgfpathlineto{\pgfqpoint{4.229319in}{0.739656in}}%
\pgfpathlineto{\pgfqpoint{4.229021in}{0.739656in}}%
\pgfpathlineto{\pgfqpoint{4.228724in}{0.739656in}}%
\pgfpathlineto{\pgfqpoint{4.228426in}{0.739656in}}%
\pgfpathlineto{\pgfqpoint{4.228129in}{0.739656in}}%
\pgfpathlineto{\pgfqpoint{4.227832in}{0.739656in}}%
\pgfpathlineto{\pgfqpoint{4.227534in}{0.739656in}}%
\pgfpathlineto{\pgfqpoint{4.227237in}{0.739656in}}%
\pgfpathlineto{\pgfqpoint{4.226939in}{0.739656in}}%
\pgfpathlineto{\pgfqpoint{4.226642in}{0.739656in}}%
\pgfpathlineto{\pgfqpoint{4.226344in}{0.739656in}}%
\pgfpathlineto{\pgfqpoint{4.226047in}{0.739656in}}%
\pgfpathlineto{\pgfqpoint{4.225749in}{0.739656in}}%
\pgfpathlineto{\pgfqpoint{4.225452in}{0.739656in}}%
\pgfpathlineto{\pgfqpoint{4.225154in}{0.739656in}}%
\pgfpathlineto{\pgfqpoint{4.224857in}{0.739656in}}%
\pgfpathlineto{\pgfqpoint{4.224559in}{0.739656in}}%
\pgfpathlineto{\pgfqpoint{4.224262in}{0.739656in}}%
\pgfpathlineto{\pgfqpoint{4.223964in}{0.739656in}}%
\pgfpathlineto{\pgfqpoint{4.223667in}{0.739656in}}%
\pgfpathlineto{\pgfqpoint{4.223369in}{0.739656in}}%
\pgfpathlineto{\pgfqpoint{4.223072in}{0.739656in}}%
\pgfpathlineto{\pgfqpoint{4.222774in}{0.739656in}}%
\pgfpathlineto{\pgfqpoint{4.222477in}{0.739656in}}%
\pgfpathlineto{\pgfqpoint{4.222179in}{0.739656in}}%
\pgfpathlineto{\pgfqpoint{4.221882in}{0.739656in}}%
\pgfpathlineto{\pgfqpoint{4.221585in}{0.739656in}}%
\pgfpathlineto{\pgfqpoint{4.221287in}{0.739656in}}%
\pgfpathlineto{\pgfqpoint{4.220990in}{0.739656in}}%
\pgfpathlineto{\pgfqpoint{4.220692in}{0.739656in}}%
\pgfpathlineto{\pgfqpoint{4.220395in}{0.739656in}}%
\pgfpathlineto{\pgfqpoint{4.220097in}{0.739656in}}%
\pgfpathlineto{\pgfqpoint{4.219800in}{0.739656in}}%
\pgfpathlineto{\pgfqpoint{4.219502in}{0.739656in}}%
\pgfpathlineto{\pgfqpoint{4.219205in}{0.739656in}}%
\pgfpathlineto{\pgfqpoint{4.218907in}{0.739656in}}%
\pgfpathlineto{\pgfqpoint{4.218610in}{0.739656in}}%
\pgfpathlineto{\pgfqpoint{4.218312in}{0.739656in}}%
\pgfpathlineto{\pgfqpoint{4.218015in}{0.739656in}}%
\pgfpathlineto{\pgfqpoint{4.217717in}{0.739656in}}%
\pgfpathlineto{\pgfqpoint{4.217420in}{0.739656in}}%
\pgfpathlineto{\pgfqpoint{4.217122in}{0.739656in}}%
\pgfpathlineto{\pgfqpoint{4.216825in}{0.739656in}}%
\pgfpathlineto{\pgfqpoint{4.216527in}{0.739656in}}%
\pgfpathlineto{\pgfqpoint{4.216230in}{0.739656in}}%
\pgfpathlineto{\pgfqpoint{4.215932in}{0.739656in}}%
\pgfpathlineto{\pgfqpoint{4.215635in}{0.739656in}}%
\pgfpathlineto{\pgfqpoint{4.215337in}{0.739656in}}%
\pgfpathlineto{\pgfqpoint{4.215040in}{0.739656in}}%
\pgfpathlineto{\pgfqpoint{4.214743in}{0.739656in}}%
\pgfpathlineto{\pgfqpoint{4.214445in}{0.739656in}}%
\pgfpathlineto{\pgfqpoint{4.214148in}{0.739656in}}%
\pgfpathlineto{\pgfqpoint{4.213850in}{0.739656in}}%
\pgfpathlineto{\pgfqpoint{4.213553in}{0.739656in}}%
\pgfpathlineto{\pgfqpoint{4.213255in}{0.739656in}}%
\pgfpathlineto{\pgfqpoint{4.212958in}{0.739656in}}%
\pgfpathlineto{\pgfqpoint{4.212660in}{0.739656in}}%
\pgfpathlineto{\pgfqpoint{4.212363in}{0.739656in}}%
\pgfpathlineto{\pgfqpoint{4.212065in}{0.739656in}}%
\pgfpathlineto{\pgfqpoint{4.211768in}{0.739656in}}%
\pgfpathlineto{\pgfqpoint{4.211470in}{0.739656in}}%
\pgfpathlineto{\pgfqpoint{4.211173in}{0.739656in}}%
\pgfpathlineto{\pgfqpoint{4.210875in}{0.739656in}}%
\pgfpathlineto{\pgfqpoint{4.210578in}{0.739656in}}%
\pgfpathlineto{\pgfqpoint{4.210280in}{0.739656in}}%
\pgfpathlineto{\pgfqpoint{4.209983in}{0.739656in}}%
\pgfpathlineto{\pgfqpoint{4.209685in}{0.739656in}}%
\pgfpathlineto{\pgfqpoint{4.209388in}{0.739656in}}%
\pgfpathlineto{\pgfqpoint{4.209090in}{0.739656in}}%
\pgfpathlineto{\pgfqpoint{4.208793in}{0.739656in}}%
\pgfpathlineto{\pgfqpoint{4.208495in}{0.739656in}}%
\pgfpathlineto{\pgfqpoint{4.208198in}{0.739656in}}%
\pgfpathlineto{\pgfqpoint{4.207901in}{0.739656in}}%
\pgfpathlineto{\pgfqpoint{4.207603in}{0.739656in}}%
\pgfpathlineto{\pgfqpoint{4.207306in}{0.739656in}}%
\pgfpathlineto{\pgfqpoint{4.207008in}{0.739656in}}%
\pgfpathlineto{\pgfqpoint{4.206711in}{0.739656in}}%
\pgfpathlineto{\pgfqpoint{4.206413in}{0.739656in}}%
\pgfpathlineto{\pgfqpoint{4.206116in}{0.739656in}}%
\pgfpathlineto{\pgfqpoint{4.205818in}{0.739656in}}%
\pgfpathlineto{\pgfqpoint{4.205521in}{0.739656in}}%
\pgfpathlineto{\pgfqpoint{4.205223in}{0.739656in}}%
\pgfpathlineto{\pgfqpoint{4.204926in}{0.739656in}}%
\pgfpathlineto{\pgfqpoint{4.204628in}{0.739656in}}%
\pgfpathlineto{\pgfqpoint{4.204331in}{0.739656in}}%
\pgfpathlineto{\pgfqpoint{4.204033in}{0.739656in}}%
\pgfpathlineto{\pgfqpoint{4.203736in}{0.739656in}}%
\pgfpathlineto{\pgfqpoint{4.203438in}{0.739656in}}%
\pgfpathlineto{\pgfqpoint{4.203141in}{0.739656in}}%
\pgfpathlineto{\pgfqpoint{4.202843in}{0.739656in}}%
\pgfpathlineto{\pgfqpoint{4.202546in}{0.739656in}}%
\pgfpathlineto{\pgfqpoint{4.202248in}{0.739656in}}%
\pgfpathlineto{\pgfqpoint{4.201951in}{0.739656in}}%
\pgfpathlineto{\pgfqpoint{4.201654in}{0.739656in}}%
\pgfpathlineto{\pgfqpoint{4.201356in}{0.739656in}}%
\pgfpathlineto{\pgfqpoint{4.201059in}{0.739656in}}%
\pgfpathlineto{\pgfqpoint{4.200761in}{0.739656in}}%
\pgfpathlineto{\pgfqpoint{4.200464in}{0.739656in}}%
\pgfpathlineto{\pgfqpoint{4.200166in}{0.739656in}}%
\pgfpathlineto{\pgfqpoint{4.199869in}{0.739656in}}%
\pgfpathlineto{\pgfqpoint{4.199571in}{0.739656in}}%
\pgfpathlineto{\pgfqpoint{4.199274in}{0.739656in}}%
\pgfpathlineto{\pgfqpoint{4.198976in}{0.739656in}}%
\pgfpathlineto{\pgfqpoint{4.198679in}{0.739656in}}%
\pgfpathlineto{\pgfqpoint{4.198381in}{0.739656in}}%
\pgfpathlineto{\pgfqpoint{4.198084in}{0.739656in}}%
\pgfpathlineto{\pgfqpoint{4.197786in}{0.739656in}}%
\pgfpathlineto{\pgfqpoint{4.197489in}{0.739656in}}%
\pgfpathlineto{\pgfqpoint{4.197191in}{0.739656in}}%
\pgfpathlineto{\pgfqpoint{4.196894in}{0.739656in}}%
\pgfpathlineto{\pgfqpoint{4.196596in}{0.739656in}}%
\pgfpathlineto{\pgfqpoint{4.196299in}{0.739656in}}%
\pgfpathlineto{\pgfqpoint{4.196001in}{0.739656in}}%
\pgfpathlineto{\pgfqpoint{4.195704in}{0.739656in}}%
\pgfpathlineto{\pgfqpoint{4.195406in}{0.739656in}}%
\pgfpathlineto{\pgfqpoint{4.195109in}{0.739656in}}%
\pgfpathlineto{\pgfqpoint{4.194812in}{0.739656in}}%
\pgfpathlineto{\pgfqpoint{4.194514in}{0.739656in}}%
\pgfpathlineto{\pgfqpoint{4.194217in}{0.739656in}}%
\pgfpathlineto{\pgfqpoint{4.193919in}{0.739656in}}%
\pgfpathlineto{\pgfqpoint{4.193622in}{0.739656in}}%
\pgfpathlineto{\pgfqpoint{4.193324in}{0.739656in}}%
\pgfpathlineto{\pgfqpoint{4.193027in}{0.739656in}}%
\pgfpathlineto{\pgfqpoint{4.192729in}{0.739656in}}%
\pgfpathlineto{\pgfqpoint{4.192432in}{0.739656in}}%
\pgfpathlineto{\pgfqpoint{4.192134in}{0.739656in}}%
\pgfpathlineto{\pgfqpoint{4.191837in}{0.739656in}}%
\pgfpathlineto{\pgfqpoint{4.191539in}{0.739656in}}%
\pgfpathlineto{\pgfqpoint{4.191242in}{0.739656in}}%
\pgfpathlineto{\pgfqpoint{4.190944in}{0.739656in}}%
\pgfpathlineto{\pgfqpoint{4.190647in}{0.739656in}}%
\pgfpathlineto{\pgfqpoint{4.190349in}{0.739656in}}%
\pgfpathlineto{\pgfqpoint{4.190052in}{0.739656in}}%
\pgfpathlineto{\pgfqpoint{4.189754in}{0.739656in}}%
\pgfpathlineto{\pgfqpoint{4.189457in}{0.739656in}}%
\pgfpathlineto{\pgfqpoint{4.189159in}{0.739656in}}%
\pgfpathlineto{\pgfqpoint{4.188862in}{0.739656in}}%
\pgfpathlineto{\pgfqpoint{4.188564in}{0.739656in}}%
\pgfpathlineto{\pgfqpoint{4.188267in}{0.739656in}}%
\pgfpathlineto{\pgfqpoint{4.187970in}{0.739656in}}%
\pgfpathlineto{\pgfqpoint{4.187672in}{0.739656in}}%
\pgfpathlineto{\pgfqpoint{4.187375in}{0.739656in}}%
\pgfpathlineto{\pgfqpoint{4.187077in}{0.739656in}}%
\pgfpathlineto{\pgfqpoint{4.186780in}{0.739656in}}%
\pgfpathlineto{\pgfqpoint{4.186482in}{0.739656in}}%
\pgfpathlineto{\pgfqpoint{4.186185in}{0.739656in}}%
\pgfpathlineto{\pgfqpoint{4.185887in}{0.739656in}}%
\pgfpathlineto{\pgfqpoint{4.185590in}{0.739656in}}%
\pgfpathlineto{\pgfqpoint{4.185292in}{0.739656in}}%
\pgfpathlineto{\pgfqpoint{4.184995in}{0.739656in}}%
\pgfpathlineto{\pgfqpoint{4.184697in}{0.739656in}}%
\pgfpathlineto{\pgfqpoint{4.184400in}{0.739656in}}%
\pgfpathlineto{\pgfqpoint{4.184102in}{0.739656in}}%
\pgfpathlineto{\pgfqpoint{4.183805in}{0.739656in}}%
\pgfpathlineto{\pgfqpoint{4.183507in}{0.739656in}}%
\pgfpathlineto{\pgfqpoint{4.183210in}{0.739656in}}%
\pgfpathlineto{\pgfqpoint{4.182912in}{0.739656in}}%
\pgfpathlineto{\pgfqpoint{4.182615in}{0.739656in}}%
\pgfpathlineto{\pgfqpoint{4.182317in}{0.739656in}}%
\pgfpathlineto{\pgfqpoint{4.182020in}{0.739656in}}%
\pgfpathlineto{\pgfqpoint{4.181723in}{0.739656in}}%
\pgfpathlineto{\pgfqpoint{4.181425in}{0.739656in}}%
\pgfpathlineto{\pgfqpoint{4.181128in}{0.739656in}}%
\pgfpathlineto{\pgfqpoint{4.180830in}{0.739656in}}%
\pgfpathlineto{\pgfqpoint{4.180533in}{0.739656in}}%
\pgfpathlineto{\pgfqpoint{4.180235in}{0.739656in}}%
\pgfpathlineto{\pgfqpoint{4.179938in}{0.739656in}}%
\pgfpathlineto{\pgfqpoint{4.179640in}{0.739656in}}%
\pgfpathlineto{\pgfqpoint{4.179343in}{0.739656in}}%
\pgfpathlineto{\pgfqpoint{4.179045in}{0.739656in}}%
\pgfpathlineto{\pgfqpoint{4.178748in}{0.739656in}}%
\pgfpathlineto{\pgfqpoint{4.178450in}{0.739656in}}%
\pgfpathlineto{\pgfqpoint{4.178153in}{0.739656in}}%
\pgfpathlineto{\pgfqpoint{4.177855in}{0.739656in}}%
\pgfpathlineto{\pgfqpoint{4.177558in}{0.739656in}}%
\pgfpathlineto{\pgfqpoint{4.177260in}{0.739656in}}%
\pgfpathlineto{\pgfqpoint{4.176963in}{0.739656in}}%
\pgfpathlineto{\pgfqpoint{4.176665in}{0.739656in}}%
\pgfpathlineto{\pgfqpoint{4.176368in}{0.739656in}}%
\pgfpathlineto{\pgfqpoint{4.176070in}{0.739656in}}%
\pgfpathlineto{\pgfqpoint{4.175773in}{0.739656in}}%
\pgfpathlineto{\pgfqpoint{4.175475in}{0.739656in}}%
\pgfpathlineto{\pgfqpoint{4.175178in}{0.739656in}}%
\pgfpathlineto{\pgfqpoint{4.174881in}{0.739656in}}%
\pgfpathlineto{\pgfqpoint{4.174583in}{0.739656in}}%
\pgfpathlineto{\pgfqpoint{4.174286in}{0.739656in}}%
\pgfpathlineto{\pgfqpoint{4.173988in}{0.739656in}}%
\pgfpathlineto{\pgfqpoint{4.173691in}{0.739656in}}%
\pgfpathlineto{\pgfqpoint{4.173393in}{0.739656in}}%
\pgfpathlineto{\pgfqpoint{4.173096in}{0.739656in}}%
\pgfpathlineto{\pgfqpoint{4.172798in}{0.739656in}}%
\pgfpathlineto{\pgfqpoint{4.172501in}{0.739656in}}%
\pgfpathlineto{\pgfqpoint{4.172203in}{0.739656in}}%
\pgfpathlineto{\pgfqpoint{4.171906in}{0.739656in}}%
\pgfpathlineto{\pgfqpoint{4.171608in}{0.739656in}}%
\pgfpathlineto{\pgfqpoint{4.171311in}{0.739656in}}%
\pgfpathlineto{\pgfqpoint{4.171013in}{0.739656in}}%
\pgfpathlineto{\pgfqpoint{4.170716in}{0.739656in}}%
\pgfpathlineto{\pgfqpoint{4.170418in}{0.739656in}}%
\pgfpathlineto{\pgfqpoint{4.170121in}{0.739656in}}%
\pgfpathlineto{\pgfqpoint{4.169823in}{0.739656in}}%
\pgfpathlineto{\pgfqpoint{4.169526in}{0.739656in}}%
\pgfpathlineto{\pgfqpoint{4.169228in}{0.739656in}}%
\pgfpathlineto{\pgfqpoint{4.168931in}{0.739656in}}%
\pgfpathlineto{\pgfqpoint{4.168633in}{0.739656in}}%
\pgfpathlineto{\pgfqpoint{4.168336in}{0.739656in}}%
\pgfpathlineto{\pgfqpoint{4.168039in}{0.739656in}}%
\pgfpathlineto{\pgfqpoint{4.167741in}{0.739656in}}%
\pgfpathlineto{\pgfqpoint{4.167444in}{0.739656in}}%
\pgfpathlineto{\pgfqpoint{4.167146in}{0.739656in}}%
\pgfpathlineto{\pgfqpoint{4.166849in}{0.739656in}}%
\pgfpathlineto{\pgfqpoint{4.166551in}{0.739656in}}%
\pgfpathlineto{\pgfqpoint{4.166254in}{0.739656in}}%
\pgfpathlineto{\pgfqpoint{4.165956in}{0.739656in}}%
\pgfpathlineto{\pgfqpoint{4.165659in}{0.739656in}}%
\pgfpathlineto{\pgfqpoint{4.165361in}{0.739656in}}%
\pgfpathlineto{\pgfqpoint{4.165064in}{0.739656in}}%
\pgfpathlineto{\pgfqpoint{4.164766in}{0.739656in}}%
\pgfpathlineto{\pgfqpoint{4.164469in}{0.739656in}}%
\pgfpathlineto{\pgfqpoint{4.164171in}{0.739656in}}%
\pgfpathlineto{\pgfqpoint{4.163874in}{0.739656in}}%
\pgfpathlineto{\pgfqpoint{4.163576in}{0.739656in}}%
\pgfpathlineto{\pgfqpoint{4.163279in}{0.739656in}}%
\pgfpathlineto{\pgfqpoint{4.162981in}{0.739656in}}%
\pgfpathlineto{\pgfqpoint{4.162684in}{0.739656in}}%
\pgfpathlineto{\pgfqpoint{4.162386in}{0.739656in}}%
\pgfpathlineto{\pgfqpoint{4.162089in}{0.739656in}}%
\pgfpathlineto{\pgfqpoint{4.161791in}{0.739656in}}%
\pgfpathlineto{\pgfqpoint{4.161494in}{0.739656in}}%
\pgfpathlineto{\pgfqpoint{4.161197in}{0.739656in}}%
\pgfpathlineto{\pgfqpoint{4.160899in}{0.739656in}}%
\pgfpathlineto{\pgfqpoint{4.160602in}{0.739656in}}%
\pgfpathlineto{\pgfqpoint{4.160304in}{0.739656in}}%
\pgfpathlineto{\pgfqpoint{4.160007in}{0.739656in}}%
\pgfpathlineto{\pgfqpoint{4.159709in}{0.739656in}}%
\pgfpathlineto{\pgfqpoint{4.159412in}{0.739656in}}%
\pgfpathlineto{\pgfqpoint{4.159114in}{0.739656in}}%
\pgfpathlineto{\pgfqpoint{4.158817in}{0.739656in}}%
\pgfpathlineto{\pgfqpoint{4.158519in}{0.739656in}}%
\pgfpathlineto{\pgfqpoint{4.158222in}{0.739656in}}%
\pgfpathlineto{\pgfqpoint{4.157924in}{0.739656in}}%
\pgfpathlineto{\pgfqpoint{4.157627in}{0.739656in}}%
\pgfpathlineto{\pgfqpoint{4.157329in}{0.739656in}}%
\pgfpathlineto{\pgfqpoint{4.157032in}{0.739656in}}%
\pgfpathlineto{\pgfqpoint{4.156734in}{0.739656in}}%
\pgfpathlineto{\pgfqpoint{4.156437in}{0.739656in}}%
\pgfpathlineto{\pgfqpoint{4.156139in}{0.739656in}}%
\pgfpathlineto{\pgfqpoint{4.155842in}{0.739656in}}%
\pgfpathlineto{\pgfqpoint{4.155544in}{0.739656in}}%
\pgfpathlineto{\pgfqpoint{4.155247in}{0.739656in}}%
\pgfpathlineto{\pgfqpoint{4.154950in}{0.739656in}}%
\pgfpathlineto{\pgfqpoint{4.154652in}{0.739656in}}%
\pgfpathlineto{\pgfqpoint{4.154355in}{0.739656in}}%
\pgfpathlineto{\pgfqpoint{4.154057in}{0.739656in}}%
\pgfpathlineto{\pgfqpoint{4.153760in}{0.739656in}}%
\pgfpathlineto{\pgfqpoint{4.153462in}{0.739656in}}%
\pgfpathlineto{\pgfqpoint{4.153165in}{0.739656in}}%
\pgfpathlineto{\pgfqpoint{4.152867in}{0.739656in}}%
\pgfpathlineto{\pgfqpoint{4.152570in}{0.739656in}}%
\pgfpathlineto{\pgfqpoint{4.152272in}{0.739656in}}%
\pgfpathlineto{\pgfqpoint{4.151975in}{0.739656in}}%
\pgfpathlineto{\pgfqpoint{4.151677in}{0.739656in}}%
\pgfpathlineto{\pgfqpoint{4.151380in}{0.739656in}}%
\pgfpathlineto{\pgfqpoint{4.151082in}{0.739656in}}%
\pgfpathlineto{\pgfqpoint{4.150785in}{0.739656in}}%
\pgfpathlineto{\pgfqpoint{4.150487in}{0.739656in}}%
\pgfpathlineto{\pgfqpoint{4.150190in}{0.739656in}}%
\pgfpathlineto{\pgfqpoint{4.149892in}{0.739656in}}%
\pgfpathlineto{\pgfqpoint{4.149595in}{0.739656in}}%
\pgfpathlineto{\pgfqpoint{4.149297in}{0.739656in}}%
\pgfpathlineto{\pgfqpoint{4.149000in}{0.739656in}}%
\pgfpathlineto{\pgfqpoint{4.148702in}{0.739656in}}%
\pgfpathlineto{\pgfqpoint{4.148405in}{0.739656in}}%
\pgfpathlineto{\pgfqpoint{4.148108in}{0.739656in}}%
\pgfpathlineto{\pgfqpoint{4.147810in}{0.739656in}}%
\pgfpathlineto{\pgfqpoint{4.147513in}{0.739656in}}%
\pgfpathlineto{\pgfqpoint{4.147215in}{0.739656in}}%
\pgfpathlineto{\pgfqpoint{4.146918in}{0.739656in}}%
\pgfpathlineto{\pgfqpoint{4.146620in}{0.739656in}}%
\pgfpathlineto{\pgfqpoint{4.146323in}{0.739656in}}%
\pgfpathlineto{\pgfqpoint{4.146025in}{0.739656in}}%
\pgfpathlineto{\pgfqpoint{4.145728in}{0.739656in}}%
\pgfpathlineto{\pgfqpoint{4.145430in}{0.739656in}}%
\pgfpathlineto{\pgfqpoint{4.145133in}{0.739656in}}%
\pgfpathlineto{\pgfqpoint{4.144835in}{0.739656in}}%
\pgfpathlineto{\pgfqpoint{4.144538in}{0.739656in}}%
\pgfpathlineto{\pgfqpoint{4.144240in}{0.739656in}}%
\pgfpathlineto{\pgfqpoint{4.143943in}{0.739656in}}%
\pgfpathlineto{\pgfqpoint{4.143645in}{0.739656in}}%
\pgfpathlineto{\pgfqpoint{4.143348in}{0.739656in}}%
\pgfpathlineto{\pgfqpoint{4.143050in}{0.739656in}}%
\pgfpathlineto{\pgfqpoint{4.142753in}{0.739656in}}%
\pgfpathlineto{\pgfqpoint{4.142455in}{0.739656in}}%
\pgfpathlineto{\pgfqpoint{4.142158in}{0.739656in}}%
\pgfpathlineto{\pgfqpoint{4.141860in}{0.739656in}}%
\pgfpathlineto{\pgfqpoint{4.141563in}{0.739656in}}%
\pgfpathlineto{\pgfqpoint{4.141266in}{0.739656in}}%
\pgfpathlineto{\pgfqpoint{4.140968in}{0.739656in}}%
\pgfpathlineto{\pgfqpoint{4.140671in}{0.739656in}}%
\pgfpathlineto{\pgfqpoint{4.140373in}{0.739656in}}%
\pgfpathlineto{\pgfqpoint{4.140076in}{0.739656in}}%
\pgfpathlineto{\pgfqpoint{4.139778in}{0.739656in}}%
\pgfpathlineto{\pgfqpoint{4.139481in}{0.739656in}}%
\pgfpathlineto{\pgfqpoint{4.139183in}{0.739656in}}%
\pgfpathlineto{\pgfqpoint{4.138886in}{0.739656in}}%
\pgfpathlineto{\pgfqpoint{4.138588in}{0.739656in}}%
\pgfpathlineto{\pgfqpoint{4.138291in}{0.739656in}}%
\pgfpathlineto{\pgfqpoint{4.137993in}{0.739656in}}%
\pgfpathlineto{\pgfqpoint{4.137696in}{0.739656in}}%
\pgfpathlineto{\pgfqpoint{4.137398in}{0.739656in}}%
\pgfpathlineto{\pgfqpoint{4.137101in}{0.739656in}}%
\pgfpathlineto{\pgfqpoint{4.136803in}{0.739656in}}%
\pgfpathlineto{\pgfqpoint{4.136506in}{0.739656in}}%
\pgfpathlineto{\pgfqpoint{4.136208in}{0.739656in}}%
\pgfpathlineto{\pgfqpoint{4.135911in}{0.739656in}}%
\pgfpathlineto{\pgfqpoint{4.135613in}{0.739656in}}%
\pgfpathlineto{\pgfqpoint{4.135316in}{0.739656in}}%
\pgfpathlineto{\pgfqpoint{4.135019in}{0.739656in}}%
\pgfpathlineto{\pgfqpoint{4.134721in}{0.739656in}}%
\pgfpathlineto{\pgfqpoint{4.134424in}{0.739656in}}%
\pgfpathlineto{\pgfqpoint{4.134126in}{0.739656in}}%
\pgfpathlineto{\pgfqpoint{4.133829in}{0.739656in}}%
\pgfpathlineto{\pgfqpoint{4.133531in}{0.739656in}}%
\pgfpathlineto{\pgfqpoint{4.133234in}{0.739656in}}%
\pgfpathlineto{\pgfqpoint{4.132936in}{0.739656in}}%
\pgfpathlineto{\pgfqpoint{4.132639in}{0.739656in}}%
\pgfpathlineto{\pgfqpoint{4.132341in}{0.739656in}}%
\pgfpathlineto{\pgfqpoint{4.132044in}{0.739656in}}%
\pgfpathlineto{\pgfqpoint{4.131746in}{0.739656in}}%
\pgfpathlineto{\pgfqpoint{4.131449in}{0.739656in}}%
\pgfpathlineto{\pgfqpoint{4.131151in}{0.739656in}}%
\pgfpathlineto{\pgfqpoint{4.130854in}{0.739656in}}%
\pgfpathlineto{\pgfqpoint{4.130556in}{0.739656in}}%
\pgfpathlineto{\pgfqpoint{4.130259in}{0.739656in}}%
\pgfpathlineto{\pgfqpoint{4.129961in}{0.739656in}}%
\pgfpathlineto{\pgfqpoint{4.129664in}{0.739656in}}%
\pgfpathlineto{\pgfqpoint{4.129366in}{0.739656in}}%
\pgfpathlineto{\pgfqpoint{4.129069in}{0.739656in}}%
\pgfpathlineto{\pgfqpoint{4.128771in}{0.739656in}}%
\pgfpathlineto{\pgfqpoint{4.128474in}{0.739656in}}%
\pgfpathlineto{\pgfqpoint{4.128177in}{0.739656in}}%
\pgfpathlineto{\pgfqpoint{4.127879in}{0.739656in}}%
\pgfpathlineto{\pgfqpoint{4.127582in}{0.739656in}}%
\pgfpathlineto{\pgfqpoint{4.127284in}{0.739656in}}%
\pgfpathlineto{\pgfqpoint{4.126987in}{0.739656in}}%
\pgfpathlineto{\pgfqpoint{4.126689in}{0.739656in}}%
\pgfpathlineto{\pgfqpoint{4.126392in}{0.739656in}}%
\pgfpathlineto{\pgfqpoint{4.126094in}{0.739656in}}%
\pgfpathlineto{\pgfqpoint{4.125797in}{0.739656in}}%
\pgfpathlineto{\pgfqpoint{4.125499in}{0.739656in}}%
\pgfpathlineto{\pgfqpoint{4.125202in}{0.739656in}}%
\pgfpathlineto{\pgfqpoint{4.124904in}{0.739656in}}%
\pgfpathlineto{\pgfqpoint{4.124607in}{0.739656in}}%
\pgfpathlineto{\pgfqpoint{4.124309in}{0.739656in}}%
\pgfpathlineto{\pgfqpoint{4.124012in}{0.739656in}}%
\pgfpathlineto{\pgfqpoint{4.123714in}{0.739656in}}%
\pgfpathlineto{\pgfqpoint{4.123417in}{0.739656in}}%
\pgfpathlineto{\pgfqpoint{4.123119in}{0.739656in}}%
\pgfpathlineto{\pgfqpoint{4.122822in}{0.739656in}}%
\pgfpathlineto{\pgfqpoint{4.122524in}{0.739656in}}%
\pgfpathlineto{\pgfqpoint{4.122227in}{0.739656in}}%
\pgfpathlineto{\pgfqpoint{4.121929in}{0.739656in}}%
\pgfpathlineto{\pgfqpoint{4.121632in}{0.739656in}}%
\pgfpathlineto{\pgfqpoint{4.121335in}{0.739656in}}%
\pgfpathlineto{\pgfqpoint{4.121037in}{0.739656in}}%
\pgfpathlineto{\pgfqpoint{4.120740in}{0.739656in}}%
\pgfpathlineto{\pgfqpoint{4.120442in}{0.739656in}}%
\pgfpathlineto{\pgfqpoint{4.120145in}{0.739656in}}%
\pgfpathlineto{\pgfqpoint{4.119847in}{0.739656in}}%
\pgfpathlineto{\pgfqpoint{4.119550in}{0.739656in}}%
\pgfpathlineto{\pgfqpoint{4.119252in}{0.739656in}}%
\pgfpathlineto{\pgfqpoint{4.118955in}{0.739656in}}%
\pgfpathlineto{\pgfqpoint{4.118657in}{0.739656in}}%
\pgfpathlineto{\pgfqpoint{4.118360in}{0.739656in}}%
\pgfpathlineto{\pgfqpoint{4.118062in}{0.739656in}}%
\pgfpathlineto{\pgfqpoint{4.117765in}{0.739656in}}%
\pgfpathlineto{\pgfqpoint{4.117467in}{0.739656in}}%
\pgfpathlineto{\pgfqpoint{4.117170in}{0.739656in}}%
\pgfpathlineto{\pgfqpoint{4.116872in}{0.739656in}}%
\pgfpathlineto{\pgfqpoint{4.116575in}{0.739656in}}%
\pgfpathlineto{\pgfqpoint{4.116277in}{0.739656in}}%
\pgfpathlineto{\pgfqpoint{4.115980in}{0.739656in}}%
\pgfpathlineto{\pgfqpoint{4.115682in}{0.739656in}}%
\pgfpathlineto{\pgfqpoint{4.115385in}{0.739656in}}%
\pgfpathlineto{\pgfqpoint{4.115088in}{0.739656in}}%
\pgfpathlineto{\pgfqpoint{4.114790in}{0.739656in}}%
\pgfpathlineto{\pgfqpoint{4.114493in}{0.739656in}}%
\pgfpathlineto{\pgfqpoint{4.114195in}{0.739656in}}%
\pgfpathlineto{\pgfqpoint{4.113898in}{0.739656in}}%
\pgfpathlineto{\pgfqpoint{4.113600in}{0.739656in}}%
\pgfpathlineto{\pgfqpoint{4.113303in}{0.739656in}}%
\pgfpathlineto{\pgfqpoint{4.113005in}{0.739656in}}%
\pgfpathlineto{\pgfqpoint{4.112708in}{0.739656in}}%
\pgfpathlineto{\pgfqpoint{4.112410in}{0.739656in}}%
\pgfpathlineto{\pgfqpoint{4.112113in}{0.739656in}}%
\pgfpathlineto{\pgfqpoint{4.111815in}{0.739656in}}%
\pgfpathlineto{\pgfqpoint{4.111518in}{0.739656in}}%
\pgfpathlineto{\pgfqpoint{4.111220in}{0.739656in}}%
\pgfpathlineto{\pgfqpoint{4.110923in}{0.739656in}}%
\pgfpathlineto{\pgfqpoint{4.110625in}{0.739656in}}%
\pgfpathlineto{\pgfqpoint{4.110328in}{0.739656in}}%
\pgfpathlineto{\pgfqpoint{4.110030in}{0.739656in}}%
\pgfpathlineto{\pgfqpoint{4.109733in}{0.739656in}}%
\pgfpathlineto{\pgfqpoint{4.109435in}{0.739656in}}%
\pgfpathlineto{\pgfqpoint{4.109138in}{0.739656in}}%
\pgfpathlineto{\pgfqpoint{4.108840in}{0.739656in}}%
\pgfpathlineto{\pgfqpoint{4.108543in}{0.739656in}}%
\pgfpathlineto{\pgfqpoint{4.108246in}{0.739656in}}%
\pgfpathlineto{\pgfqpoint{4.107948in}{0.739656in}}%
\pgfpathlineto{\pgfqpoint{4.107651in}{0.739656in}}%
\pgfpathlineto{\pgfqpoint{4.107353in}{0.739656in}}%
\pgfpathlineto{\pgfqpoint{4.107056in}{0.739656in}}%
\pgfpathlineto{\pgfqpoint{4.106758in}{0.739656in}}%
\pgfpathlineto{\pgfqpoint{4.106461in}{0.739656in}}%
\pgfpathlineto{\pgfqpoint{4.106163in}{0.739656in}}%
\pgfpathlineto{\pgfqpoint{4.105866in}{0.739656in}}%
\pgfpathlineto{\pgfqpoint{4.105568in}{0.739656in}}%
\pgfpathlineto{\pgfqpoint{4.105271in}{0.739656in}}%
\pgfpathlineto{\pgfqpoint{4.104973in}{0.739656in}}%
\pgfpathlineto{\pgfqpoint{4.104676in}{0.739656in}}%
\pgfpathlineto{\pgfqpoint{4.104378in}{0.739656in}}%
\pgfpathlineto{\pgfqpoint{4.104081in}{0.739656in}}%
\pgfpathlineto{\pgfqpoint{4.103783in}{0.739656in}}%
\pgfpathlineto{\pgfqpoint{4.103486in}{0.739656in}}%
\pgfpathlineto{\pgfqpoint{4.103188in}{0.739656in}}%
\pgfpathlineto{\pgfqpoint{4.102891in}{0.739656in}}%
\pgfpathlineto{\pgfqpoint{4.102593in}{0.739656in}}%
\pgfpathlineto{\pgfqpoint{4.102296in}{0.739656in}}%
\pgfpathlineto{\pgfqpoint{4.101998in}{0.739656in}}%
\pgfpathlineto{\pgfqpoint{4.101701in}{0.739656in}}%
\pgfpathlineto{\pgfqpoint{4.101404in}{0.739656in}}%
\pgfpathlineto{\pgfqpoint{4.101106in}{0.739656in}}%
\pgfpathlineto{\pgfqpoint{4.100809in}{0.739656in}}%
\pgfpathlineto{\pgfqpoint{4.100511in}{0.739656in}}%
\pgfpathlineto{\pgfqpoint{4.100214in}{0.739656in}}%
\pgfpathlineto{\pgfqpoint{4.099916in}{0.739656in}}%
\pgfpathlineto{\pgfqpoint{4.099619in}{0.739656in}}%
\pgfpathlineto{\pgfqpoint{4.099321in}{0.739656in}}%
\pgfpathlineto{\pgfqpoint{4.099024in}{0.739656in}}%
\pgfpathlineto{\pgfqpoint{4.098726in}{0.739656in}}%
\pgfpathlineto{\pgfqpoint{4.098429in}{0.739656in}}%
\pgfpathlineto{\pgfqpoint{4.098131in}{0.739656in}}%
\pgfpathlineto{\pgfqpoint{4.097834in}{0.739656in}}%
\pgfpathlineto{\pgfqpoint{4.097536in}{0.739656in}}%
\pgfpathlineto{\pgfqpoint{4.097239in}{0.739656in}}%
\pgfpathlineto{\pgfqpoint{4.096941in}{0.739656in}}%
\pgfpathlineto{\pgfqpoint{4.096644in}{0.739656in}}%
\pgfpathlineto{\pgfqpoint{4.096346in}{0.739656in}}%
\pgfpathlineto{\pgfqpoint{4.096049in}{0.739656in}}%
\pgfpathlineto{\pgfqpoint{4.095751in}{0.739656in}}%
\pgfpathlineto{\pgfqpoint{4.095454in}{0.739656in}}%
\pgfpathlineto{\pgfqpoint{4.095157in}{0.739656in}}%
\pgfpathlineto{\pgfqpoint{4.094859in}{0.739656in}}%
\pgfpathlineto{\pgfqpoint{4.094562in}{0.739656in}}%
\pgfpathlineto{\pgfqpoint{4.094264in}{0.739656in}}%
\pgfpathlineto{\pgfqpoint{4.093967in}{0.739656in}}%
\pgfpathlineto{\pgfqpoint{4.093669in}{0.739656in}}%
\pgfpathlineto{\pgfqpoint{4.093372in}{0.739656in}}%
\pgfpathlineto{\pgfqpoint{4.093074in}{0.739656in}}%
\pgfpathlineto{\pgfqpoint{4.092777in}{0.739656in}}%
\pgfpathlineto{\pgfqpoint{4.092479in}{0.739656in}}%
\pgfpathlineto{\pgfqpoint{4.092182in}{0.739656in}}%
\pgfpathlineto{\pgfqpoint{4.091884in}{0.739656in}}%
\pgfpathlineto{\pgfqpoint{4.091587in}{0.739656in}}%
\pgfpathlineto{\pgfqpoint{4.091289in}{0.739656in}}%
\pgfpathlineto{\pgfqpoint{4.090992in}{0.739656in}}%
\pgfpathlineto{\pgfqpoint{4.090694in}{0.739656in}}%
\pgfpathlineto{\pgfqpoint{4.090397in}{0.739656in}}%
\pgfpathlineto{\pgfqpoint{4.090099in}{0.739656in}}%
\pgfpathlineto{\pgfqpoint{4.089802in}{0.739656in}}%
\pgfpathlineto{\pgfqpoint{4.089504in}{0.739656in}}%
\pgfpathlineto{\pgfqpoint{4.089207in}{0.739656in}}%
\pgfpathlineto{\pgfqpoint{4.088909in}{0.739656in}}%
\pgfpathlineto{\pgfqpoint{4.088612in}{0.739656in}}%
\pgfpathlineto{\pgfqpoint{4.088315in}{0.739656in}}%
\pgfpathlineto{\pgfqpoint{4.088017in}{0.739656in}}%
\pgfpathlineto{\pgfqpoint{4.087720in}{0.739656in}}%
\pgfpathlineto{\pgfqpoint{4.087422in}{0.739656in}}%
\pgfpathlineto{\pgfqpoint{4.087125in}{0.739656in}}%
\pgfpathlineto{\pgfqpoint{4.086827in}{0.739656in}}%
\pgfpathlineto{\pgfqpoint{4.086530in}{0.739656in}}%
\pgfpathlineto{\pgfqpoint{4.086232in}{0.739656in}}%
\pgfpathlineto{\pgfqpoint{4.085935in}{0.739656in}}%
\pgfpathlineto{\pgfqpoint{4.085637in}{0.739656in}}%
\pgfpathlineto{\pgfqpoint{4.085340in}{0.739656in}}%
\pgfpathlineto{\pgfqpoint{4.085042in}{0.739656in}}%
\pgfpathlineto{\pgfqpoint{4.084745in}{0.739656in}}%
\pgfpathlineto{\pgfqpoint{4.084447in}{0.739656in}}%
\pgfpathlineto{\pgfqpoint{4.084150in}{0.739656in}}%
\pgfpathlineto{\pgfqpoint{4.083852in}{0.739656in}}%
\pgfpathlineto{\pgfqpoint{4.083555in}{0.739656in}}%
\pgfpathlineto{\pgfqpoint{4.083257in}{0.739656in}}%
\pgfpathlineto{\pgfqpoint{4.082960in}{0.739656in}}%
\pgfpathlineto{\pgfqpoint{4.082662in}{0.739656in}}%
\pgfpathlineto{\pgfqpoint{4.082365in}{0.739656in}}%
\pgfpathlineto{\pgfqpoint{4.082067in}{0.739656in}}%
\pgfpathlineto{\pgfqpoint{4.081770in}{0.739656in}}%
\pgfpathlineto{\pgfqpoint{4.081473in}{0.739656in}}%
\pgfpathlineto{\pgfqpoint{4.081175in}{0.739656in}}%
\pgfpathlineto{\pgfqpoint{4.080878in}{0.739656in}}%
\pgfpathlineto{\pgfqpoint{4.080580in}{0.739656in}}%
\pgfpathlineto{\pgfqpoint{4.080283in}{0.739656in}}%
\pgfpathlineto{\pgfqpoint{4.079985in}{0.739656in}}%
\pgfpathlineto{\pgfqpoint{4.079688in}{0.739656in}}%
\pgfpathlineto{\pgfqpoint{4.079390in}{0.739656in}}%
\pgfpathlineto{\pgfqpoint{4.079093in}{0.739656in}}%
\pgfpathlineto{\pgfqpoint{4.078795in}{0.739656in}}%
\pgfpathlineto{\pgfqpoint{4.078498in}{0.739656in}}%
\pgfpathlineto{\pgfqpoint{4.078200in}{0.739656in}}%
\pgfpathlineto{\pgfqpoint{4.077903in}{0.739656in}}%
\pgfpathlineto{\pgfqpoint{4.077605in}{0.739656in}}%
\pgfpathlineto{\pgfqpoint{4.077308in}{0.739656in}}%
\pgfpathlineto{\pgfqpoint{4.077010in}{0.739656in}}%
\pgfpathlineto{\pgfqpoint{4.076713in}{0.739656in}}%
\pgfpathlineto{\pgfqpoint{4.076415in}{0.739656in}}%
\pgfpathlineto{\pgfqpoint{4.076118in}{0.739656in}}%
\pgfpathlineto{\pgfqpoint{4.075820in}{0.739656in}}%
\pgfpathlineto{\pgfqpoint{4.075523in}{0.739656in}}%
\pgfpathlineto{\pgfqpoint{4.075226in}{0.739656in}}%
\pgfpathlineto{\pgfqpoint{4.074928in}{0.739656in}}%
\pgfpathlineto{\pgfqpoint{4.074631in}{0.739656in}}%
\pgfpathlineto{\pgfqpoint{4.074333in}{0.739656in}}%
\pgfpathlineto{\pgfqpoint{4.074036in}{0.739656in}}%
\pgfpathlineto{\pgfqpoint{4.073738in}{0.739656in}}%
\pgfpathlineto{\pgfqpoint{4.073441in}{0.739656in}}%
\pgfpathlineto{\pgfqpoint{4.073143in}{0.739656in}}%
\pgfpathlineto{\pgfqpoint{4.072846in}{0.739656in}}%
\pgfpathlineto{\pgfqpoint{4.072548in}{0.739656in}}%
\pgfpathlineto{\pgfqpoint{4.072251in}{0.739656in}}%
\pgfpathlineto{\pgfqpoint{4.071953in}{0.739656in}}%
\pgfpathlineto{\pgfqpoint{4.071656in}{0.739656in}}%
\pgfpathlineto{\pgfqpoint{4.071358in}{0.739656in}}%
\pgfpathlineto{\pgfqpoint{4.071061in}{0.739656in}}%
\pgfpathlineto{\pgfqpoint{4.070763in}{0.739656in}}%
\pgfpathlineto{\pgfqpoint{4.070466in}{0.739656in}}%
\pgfpathlineto{\pgfqpoint{4.070168in}{0.739656in}}%
\pgfpathlineto{\pgfqpoint{4.069871in}{0.739656in}}%
\pgfpathlineto{\pgfqpoint{4.069573in}{0.739656in}}%
\pgfpathlineto{\pgfqpoint{4.069276in}{0.739656in}}%
\pgfpathlineto{\pgfqpoint{4.068978in}{0.739656in}}%
\pgfpathlineto{\pgfqpoint{4.068681in}{0.739656in}}%
\pgfpathlineto{\pgfqpoint{4.068384in}{0.739656in}}%
\pgfpathlineto{\pgfqpoint{4.068086in}{0.739656in}}%
\pgfpathlineto{\pgfqpoint{4.067789in}{0.739656in}}%
\pgfpathlineto{\pgfqpoint{4.067491in}{0.739656in}}%
\pgfpathlineto{\pgfqpoint{4.067194in}{0.739656in}}%
\pgfpathlineto{\pgfqpoint{4.066896in}{0.739656in}}%
\pgfpathlineto{\pgfqpoint{4.066599in}{0.739656in}}%
\pgfpathlineto{\pgfqpoint{4.066301in}{0.739656in}}%
\pgfpathlineto{\pgfqpoint{4.066004in}{0.739656in}}%
\pgfpathlineto{\pgfqpoint{4.065706in}{0.739656in}}%
\pgfpathlineto{\pgfqpoint{4.065409in}{0.739656in}}%
\pgfpathlineto{\pgfqpoint{4.065111in}{0.739656in}}%
\pgfpathlineto{\pgfqpoint{4.064814in}{0.739656in}}%
\pgfpathlineto{\pgfqpoint{4.064516in}{0.739656in}}%
\pgfpathlineto{\pgfqpoint{4.064219in}{0.739656in}}%
\pgfpathlineto{\pgfqpoint{4.063921in}{0.739656in}}%
\pgfpathlineto{\pgfqpoint{4.063624in}{0.739656in}}%
\pgfpathlineto{\pgfqpoint{4.063326in}{0.739656in}}%
\pgfpathlineto{\pgfqpoint{4.063029in}{0.739656in}}%
\pgfpathlineto{\pgfqpoint{4.062731in}{0.739656in}}%
\pgfpathlineto{\pgfqpoint{4.062434in}{0.739656in}}%
\pgfpathlineto{\pgfqpoint{4.062136in}{0.739656in}}%
\pgfpathlineto{\pgfqpoint{4.061839in}{0.739656in}}%
\pgfpathlineto{\pgfqpoint{4.061542in}{0.739656in}}%
\pgfpathlineto{\pgfqpoint{4.061244in}{0.739656in}}%
\pgfpathlineto{\pgfqpoint{4.060947in}{0.739656in}}%
\pgfpathlineto{\pgfqpoint{4.060649in}{0.739656in}}%
\pgfpathlineto{\pgfqpoint{4.060352in}{0.739656in}}%
\pgfpathlineto{\pgfqpoint{4.060054in}{0.739656in}}%
\pgfpathlineto{\pgfqpoint{4.059757in}{0.739656in}}%
\pgfpathlineto{\pgfqpoint{4.059459in}{0.739656in}}%
\pgfpathlineto{\pgfqpoint{4.059162in}{0.739656in}}%
\pgfpathlineto{\pgfqpoint{4.058864in}{0.739656in}}%
\pgfpathlineto{\pgfqpoint{4.058567in}{0.739656in}}%
\pgfpathlineto{\pgfqpoint{4.058269in}{0.739656in}}%
\pgfpathlineto{\pgfqpoint{4.057972in}{0.739656in}}%
\pgfpathlineto{\pgfqpoint{4.057674in}{0.739656in}}%
\pgfpathlineto{\pgfqpoint{4.057377in}{0.739656in}}%
\pgfpathlineto{\pgfqpoint{4.057079in}{0.739656in}}%
\pgfpathlineto{\pgfqpoint{4.056782in}{0.739656in}}%
\pgfpathlineto{\pgfqpoint{4.056484in}{0.739656in}}%
\pgfpathlineto{\pgfqpoint{4.056187in}{0.739656in}}%
\pgfpathlineto{\pgfqpoint{4.055889in}{0.739656in}}%
\pgfpathlineto{\pgfqpoint{4.055592in}{0.739656in}}%
\pgfpathlineto{\pgfqpoint{4.055295in}{0.739656in}}%
\pgfpathlineto{\pgfqpoint{4.054997in}{0.739656in}}%
\pgfpathlineto{\pgfqpoint{4.054700in}{0.739656in}}%
\pgfpathlineto{\pgfqpoint{4.054402in}{0.739656in}}%
\pgfpathlineto{\pgfqpoint{4.054105in}{0.739656in}}%
\pgfpathlineto{\pgfqpoint{4.053807in}{0.739656in}}%
\pgfpathlineto{\pgfqpoint{4.053510in}{0.739656in}}%
\pgfpathlineto{\pgfqpoint{4.053212in}{0.739656in}}%
\pgfpathlineto{\pgfqpoint{4.052915in}{0.739656in}}%
\pgfpathlineto{\pgfqpoint{4.052617in}{0.739656in}}%
\pgfpathlineto{\pgfqpoint{4.052320in}{0.739656in}}%
\pgfpathlineto{\pgfqpoint{4.052022in}{0.739656in}}%
\pgfpathlineto{\pgfqpoint{4.051725in}{0.739656in}}%
\pgfpathlineto{\pgfqpoint{4.051427in}{0.739656in}}%
\pgfpathlineto{\pgfqpoint{4.051130in}{0.739656in}}%
\pgfpathlineto{\pgfqpoint{4.050832in}{0.739656in}}%
\pgfpathlineto{\pgfqpoint{4.050535in}{0.739656in}}%
\pgfpathlineto{\pgfqpoint{4.050237in}{0.739656in}}%
\pgfpathlineto{\pgfqpoint{4.049940in}{0.739656in}}%
\pgfpathlineto{\pgfqpoint{4.049642in}{0.739656in}}%
\pgfpathlineto{\pgfqpoint{4.049345in}{0.739656in}}%
\pgfpathlineto{\pgfqpoint{4.049047in}{0.739656in}}%
\pgfpathlineto{\pgfqpoint{4.048750in}{0.739656in}}%
\pgfpathlineto{\pgfqpoint{4.048453in}{0.739656in}}%
\pgfpathlineto{\pgfqpoint{4.048155in}{0.739656in}}%
\pgfpathlineto{\pgfqpoint{4.047858in}{0.739656in}}%
\pgfpathlineto{\pgfqpoint{4.047560in}{0.739656in}}%
\pgfpathlineto{\pgfqpoint{4.047263in}{0.739656in}}%
\pgfpathlineto{\pgfqpoint{4.046965in}{0.739656in}}%
\pgfpathlineto{\pgfqpoint{4.046668in}{0.739656in}}%
\pgfpathlineto{\pgfqpoint{4.046370in}{0.739656in}}%
\pgfpathlineto{\pgfqpoint{4.046073in}{0.739656in}}%
\pgfpathlineto{\pgfqpoint{4.045775in}{0.739656in}}%
\pgfpathlineto{\pgfqpoint{4.045478in}{0.739656in}}%
\pgfpathlineto{\pgfqpoint{4.045180in}{0.739656in}}%
\pgfpathlineto{\pgfqpoint{4.044883in}{0.739656in}}%
\pgfpathlineto{\pgfqpoint{4.044585in}{0.739656in}}%
\pgfpathlineto{\pgfqpoint{4.044288in}{0.739656in}}%
\pgfpathlineto{\pgfqpoint{4.043990in}{0.739656in}}%
\pgfpathlineto{\pgfqpoint{4.043693in}{0.739656in}}%
\pgfpathlineto{\pgfqpoint{4.043395in}{0.739656in}}%
\pgfpathlineto{\pgfqpoint{4.043098in}{0.739656in}}%
\pgfpathlineto{\pgfqpoint{4.042800in}{0.739656in}}%
\pgfpathlineto{\pgfqpoint{4.042503in}{0.739656in}}%
\pgfpathlineto{\pgfqpoint{4.042205in}{0.739656in}}%
\pgfpathlineto{\pgfqpoint{4.041908in}{0.739656in}}%
\pgfpathlineto{\pgfqpoint{4.041611in}{0.739656in}}%
\pgfpathlineto{\pgfqpoint{4.041313in}{0.739656in}}%
\pgfpathlineto{\pgfqpoint{4.041016in}{0.739656in}}%
\pgfpathlineto{\pgfqpoint{4.040718in}{0.739656in}}%
\pgfpathlineto{\pgfqpoint{4.040421in}{0.739656in}}%
\pgfpathlineto{\pgfqpoint{4.040123in}{0.739656in}}%
\pgfpathlineto{\pgfqpoint{4.039826in}{0.739656in}}%
\pgfpathlineto{\pgfqpoint{4.039528in}{0.739656in}}%
\pgfpathlineto{\pgfqpoint{4.039231in}{0.739656in}}%
\pgfpathlineto{\pgfqpoint{4.038933in}{0.739656in}}%
\pgfpathlineto{\pgfqpoint{4.038636in}{0.739656in}}%
\pgfpathlineto{\pgfqpoint{4.038338in}{0.739656in}}%
\pgfpathlineto{\pgfqpoint{4.038041in}{0.739656in}}%
\pgfpathlineto{\pgfqpoint{4.037743in}{0.739656in}}%
\pgfpathlineto{\pgfqpoint{4.037446in}{0.739656in}}%
\pgfpathlineto{\pgfqpoint{4.037148in}{0.739656in}}%
\pgfpathlineto{\pgfqpoint{4.036851in}{0.739656in}}%
\pgfpathlineto{\pgfqpoint{4.036553in}{0.739656in}}%
\pgfpathlineto{\pgfqpoint{4.036256in}{0.739656in}}%
\pgfpathlineto{\pgfqpoint{4.035958in}{0.739656in}}%
\pgfpathlineto{\pgfqpoint{4.035661in}{0.739656in}}%
\pgfpathlineto{\pgfqpoint{4.035364in}{0.739656in}}%
\pgfpathlineto{\pgfqpoint{4.035066in}{0.739656in}}%
\pgfpathlineto{\pgfqpoint{4.034769in}{0.739656in}}%
\pgfpathlineto{\pgfqpoint{4.034471in}{0.739656in}}%
\pgfpathlineto{\pgfqpoint{4.034174in}{0.739656in}}%
\pgfpathlineto{\pgfqpoint{4.033876in}{0.739656in}}%
\pgfpathlineto{\pgfqpoint{4.033579in}{0.739656in}}%
\pgfpathlineto{\pgfqpoint{4.033281in}{0.739656in}}%
\pgfpathlineto{\pgfqpoint{4.032984in}{0.739656in}}%
\pgfpathlineto{\pgfqpoint{4.032686in}{0.739656in}}%
\pgfpathlineto{\pgfqpoint{4.032389in}{0.739656in}}%
\pgfpathlineto{\pgfqpoint{4.032091in}{0.739656in}}%
\pgfpathlineto{\pgfqpoint{4.031794in}{0.739656in}}%
\pgfpathlineto{\pgfqpoint{4.031496in}{0.739656in}}%
\pgfpathlineto{\pgfqpoint{4.031199in}{0.739656in}}%
\pgfpathlineto{\pgfqpoint{4.030901in}{0.739656in}}%
\pgfpathlineto{\pgfqpoint{4.030604in}{0.739656in}}%
\pgfpathlineto{\pgfqpoint{4.030306in}{0.739656in}}%
\pgfpathlineto{\pgfqpoint{4.030009in}{0.739656in}}%
\pgfpathlineto{\pgfqpoint{4.029711in}{0.739656in}}%
\pgfpathlineto{\pgfqpoint{4.029414in}{0.739656in}}%
\pgfpathlineto{\pgfqpoint{4.029116in}{0.739656in}}%
\pgfpathlineto{\pgfqpoint{4.028819in}{0.739656in}}%
\pgfpathlineto{\pgfqpoint{4.028522in}{0.739656in}}%
\pgfpathlineto{\pgfqpoint{4.028224in}{0.739656in}}%
\pgfpathlineto{\pgfqpoint{4.027927in}{0.739656in}}%
\pgfpathlineto{\pgfqpoint{4.027629in}{0.739656in}}%
\pgfpathlineto{\pgfqpoint{4.027332in}{0.739656in}}%
\pgfpathlineto{\pgfqpoint{4.027034in}{0.739656in}}%
\pgfpathlineto{\pgfqpoint{4.026737in}{0.739656in}}%
\pgfpathlineto{\pgfqpoint{4.026439in}{0.739656in}}%
\pgfpathlineto{\pgfqpoint{4.026142in}{0.739656in}}%
\pgfpathlineto{\pgfqpoint{4.025844in}{0.739656in}}%
\pgfpathlineto{\pgfqpoint{4.025547in}{0.739656in}}%
\pgfpathlineto{\pgfqpoint{4.025249in}{0.739656in}}%
\pgfpathlineto{\pgfqpoint{4.024952in}{0.739656in}}%
\pgfpathlineto{\pgfqpoint{4.024654in}{0.739656in}}%
\pgfpathlineto{\pgfqpoint{4.024357in}{0.739656in}}%
\pgfpathlineto{\pgfqpoint{4.024059in}{0.739656in}}%
\pgfpathlineto{\pgfqpoint{4.023762in}{0.739656in}}%
\pgfpathlineto{\pgfqpoint{4.023464in}{0.739656in}}%
\pgfpathlineto{\pgfqpoint{4.023167in}{0.739656in}}%
\pgfpathlineto{\pgfqpoint{4.022869in}{0.739656in}}%
\pgfpathlineto{\pgfqpoint{4.022572in}{0.739656in}}%
\pgfpathlineto{\pgfqpoint{4.022274in}{0.739656in}}%
\pgfpathlineto{\pgfqpoint{4.021977in}{0.739656in}}%
\pgfpathlineto{\pgfqpoint{4.021680in}{0.739656in}}%
\pgfpathlineto{\pgfqpoint{4.021382in}{0.739656in}}%
\pgfpathlineto{\pgfqpoint{4.021085in}{0.739656in}}%
\pgfpathlineto{\pgfqpoint{4.020787in}{0.739656in}}%
\pgfpathlineto{\pgfqpoint{4.020490in}{0.739656in}}%
\pgfpathlineto{\pgfqpoint{4.020192in}{0.739656in}}%
\pgfpathlineto{\pgfqpoint{4.019895in}{0.739656in}}%
\pgfpathlineto{\pgfqpoint{4.019597in}{0.739656in}}%
\pgfpathlineto{\pgfqpoint{4.019300in}{0.739656in}}%
\pgfpathlineto{\pgfqpoint{4.019002in}{0.739656in}}%
\pgfpathlineto{\pgfqpoint{4.018705in}{0.739656in}}%
\pgfpathlineto{\pgfqpoint{4.018407in}{0.739656in}}%
\pgfpathlineto{\pgfqpoint{4.018110in}{0.739656in}}%
\pgfpathlineto{\pgfqpoint{4.017812in}{0.739656in}}%
\pgfpathlineto{\pgfqpoint{4.017515in}{0.739656in}}%
\pgfpathlineto{\pgfqpoint{4.017217in}{0.739656in}}%
\pgfpathlineto{\pgfqpoint{4.016920in}{0.739656in}}%
\pgfpathlineto{\pgfqpoint{4.016622in}{0.739656in}}%
\pgfpathlineto{\pgfqpoint{4.016325in}{0.739656in}}%
\pgfpathlineto{\pgfqpoint{4.016027in}{0.739656in}}%
\pgfpathlineto{\pgfqpoint{4.015730in}{0.739656in}}%
\pgfpathlineto{\pgfqpoint{4.015433in}{0.739656in}}%
\pgfpathlineto{\pgfqpoint{4.015135in}{0.739656in}}%
\pgfpathlineto{\pgfqpoint{4.014838in}{0.739656in}}%
\pgfpathlineto{\pgfqpoint{4.014540in}{0.739656in}}%
\pgfpathlineto{\pgfqpoint{4.014243in}{0.739656in}}%
\pgfpathlineto{\pgfqpoint{4.013945in}{0.739656in}}%
\pgfpathlineto{\pgfqpoint{4.013648in}{0.739656in}}%
\pgfpathlineto{\pgfqpoint{4.013350in}{0.739656in}}%
\pgfpathlineto{\pgfqpoint{4.013053in}{0.739656in}}%
\pgfpathlineto{\pgfqpoint{4.012755in}{0.739656in}}%
\pgfpathlineto{\pgfqpoint{4.012458in}{0.739656in}}%
\pgfpathlineto{\pgfqpoint{4.012160in}{0.739656in}}%
\pgfpathlineto{\pgfqpoint{4.011863in}{0.739656in}}%
\pgfpathlineto{\pgfqpoint{4.011565in}{0.739656in}}%
\pgfpathlineto{\pgfqpoint{4.011268in}{0.739656in}}%
\pgfpathlineto{\pgfqpoint{4.010970in}{0.739656in}}%
\pgfpathlineto{\pgfqpoint{4.010673in}{0.739656in}}%
\pgfpathlineto{\pgfqpoint{4.010375in}{0.739656in}}%
\pgfpathlineto{\pgfqpoint{4.010078in}{0.739656in}}%
\pgfpathlineto{\pgfqpoint{4.009780in}{0.739656in}}%
\pgfpathlineto{\pgfqpoint{4.009483in}{0.739656in}}%
\pgfpathlineto{\pgfqpoint{4.009185in}{0.739656in}}%
\pgfpathlineto{\pgfqpoint{4.008888in}{0.739656in}}%
\pgfpathlineto{\pgfqpoint{4.008591in}{0.739656in}}%
\pgfpathlineto{\pgfqpoint{4.008293in}{0.739656in}}%
\pgfpathlineto{\pgfqpoint{4.007996in}{0.739656in}}%
\pgfpathlineto{\pgfqpoint{4.007698in}{0.739656in}}%
\pgfpathlineto{\pgfqpoint{4.007401in}{0.739656in}}%
\pgfpathlineto{\pgfqpoint{4.007103in}{0.739656in}}%
\pgfpathlineto{\pgfqpoint{4.006806in}{0.739656in}}%
\pgfpathlineto{\pgfqpoint{4.006508in}{0.739656in}}%
\pgfpathlineto{\pgfqpoint{4.006211in}{0.739656in}}%
\pgfpathlineto{\pgfqpoint{4.005913in}{0.739656in}}%
\pgfpathlineto{\pgfqpoint{4.005616in}{0.739656in}}%
\pgfpathlineto{\pgfqpoint{4.005318in}{0.739656in}}%
\pgfpathlineto{\pgfqpoint{4.005021in}{0.739656in}}%
\pgfpathlineto{\pgfqpoint{4.004723in}{0.739656in}}%
\pgfpathlineto{\pgfqpoint{4.004426in}{0.739656in}}%
\pgfpathlineto{\pgfqpoint{4.004128in}{0.739656in}}%
\pgfpathlineto{\pgfqpoint{4.003831in}{0.739656in}}%
\pgfpathlineto{\pgfqpoint{4.003533in}{0.739656in}}%
\pgfpathlineto{\pgfqpoint{4.003236in}{0.739656in}}%
\pgfpathlineto{\pgfqpoint{4.002938in}{0.739656in}}%
\pgfpathlineto{\pgfqpoint{4.002641in}{0.739656in}}%
\pgfpathlineto{\pgfqpoint{4.002343in}{0.739656in}}%
\pgfpathlineto{\pgfqpoint{4.002046in}{0.739656in}}%
\pgfpathlineto{\pgfqpoint{4.001749in}{0.739656in}}%
\pgfpathlineto{\pgfqpoint{4.001451in}{0.739656in}}%
\pgfpathlineto{\pgfqpoint{4.001154in}{0.739656in}}%
\pgfpathlineto{\pgfqpoint{4.000856in}{0.739656in}}%
\pgfpathlineto{\pgfqpoint{4.000559in}{0.739656in}}%
\pgfpathlineto{\pgfqpoint{4.000261in}{0.739656in}}%
\pgfpathlineto{\pgfqpoint{3.999964in}{0.739656in}}%
\pgfpathlineto{\pgfqpoint{3.999666in}{0.739656in}}%
\pgfpathlineto{\pgfqpoint{3.999369in}{0.739656in}}%
\pgfpathlineto{\pgfqpoint{3.999071in}{0.739656in}}%
\pgfpathlineto{\pgfqpoint{3.998774in}{0.739656in}}%
\pgfpathlineto{\pgfqpoint{3.998476in}{0.739656in}}%
\pgfpathlineto{\pgfqpoint{3.998179in}{0.739656in}}%
\pgfpathlineto{\pgfqpoint{3.997881in}{0.739656in}}%
\pgfpathlineto{\pgfqpoint{3.997584in}{0.739656in}}%
\pgfpathlineto{\pgfqpoint{3.997286in}{0.739656in}}%
\pgfpathlineto{\pgfqpoint{3.996989in}{0.739656in}}%
\pgfpathlineto{\pgfqpoint{3.996691in}{0.739656in}}%
\pgfpathlineto{\pgfqpoint{3.996394in}{0.739656in}}%
\pgfpathlineto{\pgfqpoint{3.996096in}{0.739656in}}%
\pgfpathlineto{\pgfqpoint{3.995799in}{0.739656in}}%
\pgfpathlineto{\pgfqpoint{3.995502in}{0.739656in}}%
\pgfpathlineto{\pgfqpoint{3.995204in}{0.739656in}}%
\pgfpathlineto{\pgfqpoint{3.994907in}{0.739656in}}%
\pgfpathlineto{\pgfqpoint{3.994609in}{0.739656in}}%
\pgfpathlineto{\pgfqpoint{3.994312in}{0.739656in}}%
\pgfpathlineto{\pgfqpoint{3.994014in}{0.739656in}}%
\pgfpathlineto{\pgfqpoint{3.993717in}{0.739656in}}%
\pgfpathlineto{\pgfqpoint{3.993419in}{0.739656in}}%
\pgfpathlineto{\pgfqpoint{3.993122in}{0.739656in}}%
\pgfpathlineto{\pgfqpoint{3.992824in}{0.739656in}}%
\pgfpathlineto{\pgfqpoint{3.992527in}{0.739656in}}%
\pgfpathlineto{\pgfqpoint{3.992229in}{0.739656in}}%
\pgfpathlineto{\pgfqpoint{3.991932in}{0.739656in}}%
\pgfpathlineto{\pgfqpoint{3.991634in}{0.739656in}}%
\pgfpathlineto{\pgfqpoint{3.991337in}{0.739656in}}%
\pgfpathlineto{\pgfqpoint{3.991039in}{0.739656in}}%
\pgfpathlineto{\pgfqpoint{3.990742in}{0.739656in}}%
\pgfpathlineto{\pgfqpoint{3.990444in}{0.739656in}}%
\pgfpathlineto{\pgfqpoint{3.990147in}{0.739656in}}%
\pgfpathlineto{\pgfqpoint{3.989849in}{0.739656in}}%
\pgfpathlineto{\pgfqpoint{3.989552in}{0.739656in}}%
\pgfpathlineto{\pgfqpoint{3.989254in}{0.739656in}}%
\pgfpathlineto{\pgfqpoint{3.988957in}{0.739656in}}%
\pgfpathlineto{\pgfqpoint{3.988660in}{0.739656in}}%
\pgfpathlineto{\pgfqpoint{3.988362in}{0.739656in}}%
\pgfpathlineto{\pgfqpoint{3.988065in}{0.739656in}}%
\pgfpathlineto{\pgfqpoint{3.987767in}{0.739656in}}%
\pgfpathlineto{\pgfqpoint{3.987470in}{0.739656in}}%
\pgfpathlineto{\pgfqpoint{3.987172in}{0.739656in}}%
\pgfpathlineto{\pgfqpoint{3.986875in}{0.739656in}}%
\pgfpathlineto{\pgfqpoint{3.986577in}{0.739656in}}%
\pgfpathlineto{\pgfqpoint{3.986280in}{0.739656in}}%
\pgfpathlineto{\pgfqpoint{3.985982in}{0.739656in}}%
\pgfpathlineto{\pgfqpoint{3.985685in}{0.739656in}}%
\pgfpathlineto{\pgfqpoint{3.985387in}{0.739656in}}%
\pgfpathlineto{\pgfqpoint{3.985090in}{0.739656in}}%
\pgfpathlineto{\pgfqpoint{3.984792in}{0.739656in}}%
\pgfpathlineto{\pgfqpoint{3.984495in}{0.739656in}}%
\pgfpathlineto{\pgfqpoint{3.984197in}{0.739656in}}%
\pgfpathlineto{\pgfqpoint{3.983900in}{0.739656in}}%
\pgfpathlineto{\pgfqpoint{3.983602in}{0.739656in}}%
\pgfpathlineto{\pgfqpoint{3.983305in}{0.739656in}}%
\pgfpathlineto{\pgfqpoint{3.983007in}{0.739656in}}%
\pgfpathlineto{\pgfqpoint{3.982710in}{0.739656in}}%
\pgfpathlineto{\pgfqpoint{3.982412in}{0.739656in}}%
\pgfpathlineto{\pgfqpoint{3.982115in}{0.739656in}}%
\pgfpathlineto{\pgfqpoint{3.981818in}{0.739656in}}%
\pgfpathlineto{\pgfqpoint{3.981520in}{0.739656in}}%
\pgfpathlineto{\pgfqpoint{3.981223in}{0.739656in}}%
\pgfpathlineto{\pgfqpoint{3.980925in}{0.739656in}}%
\pgfpathlineto{\pgfqpoint{3.980628in}{0.739656in}}%
\pgfpathlineto{\pgfqpoint{3.980330in}{0.739656in}}%
\pgfpathlineto{\pgfqpoint{3.980033in}{0.739656in}}%
\pgfpathlineto{\pgfqpoint{3.979735in}{0.739656in}}%
\pgfpathlineto{\pgfqpoint{3.979438in}{0.739656in}}%
\pgfpathlineto{\pgfqpoint{3.979140in}{0.739656in}}%
\pgfpathlineto{\pgfqpoint{3.978843in}{0.739656in}}%
\pgfpathlineto{\pgfqpoint{3.978545in}{0.739656in}}%
\pgfpathlineto{\pgfqpoint{3.978248in}{0.739656in}}%
\pgfpathlineto{\pgfqpoint{3.977950in}{0.739656in}}%
\pgfpathlineto{\pgfqpoint{3.977653in}{0.739656in}}%
\pgfpathlineto{\pgfqpoint{3.977355in}{0.739656in}}%
\pgfpathlineto{\pgfqpoint{3.977058in}{0.739656in}}%
\pgfpathlineto{\pgfqpoint{3.976760in}{0.739656in}}%
\pgfpathlineto{\pgfqpoint{3.976463in}{0.739656in}}%
\pgfpathlineto{\pgfqpoint{3.976165in}{0.739656in}}%
\pgfpathlineto{\pgfqpoint{3.975868in}{0.739656in}}%
\pgfpathlineto{\pgfqpoint{3.975571in}{0.739656in}}%
\pgfpathlineto{\pgfqpoint{3.975273in}{0.739656in}}%
\pgfpathlineto{\pgfqpoint{3.974976in}{0.739656in}}%
\pgfpathlineto{\pgfqpoint{3.974678in}{0.739656in}}%
\pgfpathlineto{\pgfqpoint{3.974381in}{0.739656in}}%
\pgfpathlineto{\pgfqpoint{3.974083in}{0.739656in}}%
\pgfpathlineto{\pgfqpoint{3.973786in}{0.739656in}}%
\pgfpathlineto{\pgfqpoint{3.973488in}{0.739656in}}%
\pgfpathlineto{\pgfqpoint{3.973191in}{0.739656in}}%
\pgfpathlineto{\pgfqpoint{3.972893in}{0.739656in}}%
\pgfpathlineto{\pgfqpoint{3.972596in}{0.739656in}}%
\pgfpathlineto{\pgfqpoint{3.972298in}{0.739656in}}%
\pgfpathlineto{\pgfqpoint{3.972001in}{0.739656in}}%
\pgfpathlineto{\pgfqpoint{3.971703in}{0.739656in}}%
\pgfpathlineto{\pgfqpoint{3.971406in}{0.739656in}}%
\pgfpathlineto{\pgfqpoint{3.971108in}{0.739656in}}%
\pgfpathlineto{\pgfqpoint{3.970811in}{0.739656in}}%
\pgfpathlineto{\pgfqpoint{3.970513in}{0.739656in}}%
\pgfpathlineto{\pgfqpoint{3.970216in}{0.739656in}}%
\pgfpathlineto{\pgfqpoint{3.969918in}{0.739656in}}%
\pgfpathlineto{\pgfqpoint{3.969621in}{0.739656in}}%
\pgfpathlineto{\pgfqpoint{3.969323in}{0.739656in}}%
\pgfpathlineto{\pgfqpoint{3.969026in}{0.739656in}}%
\pgfpathlineto{\pgfqpoint{3.968729in}{0.739656in}}%
\pgfpathlineto{\pgfqpoint{3.968431in}{0.739656in}}%
\pgfpathlineto{\pgfqpoint{3.968134in}{0.739656in}}%
\pgfpathlineto{\pgfqpoint{3.967836in}{0.739656in}}%
\pgfpathlineto{\pgfqpoint{3.967539in}{0.739656in}}%
\pgfpathlineto{\pgfqpoint{3.967241in}{0.739656in}}%
\pgfpathlineto{\pgfqpoint{3.966944in}{0.739656in}}%
\pgfpathlineto{\pgfqpoint{3.966646in}{0.739656in}}%
\pgfpathlineto{\pgfqpoint{3.966349in}{0.739656in}}%
\pgfpathlineto{\pgfqpoint{3.966051in}{0.739656in}}%
\pgfpathlineto{\pgfqpoint{3.965754in}{0.739656in}}%
\pgfpathlineto{\pgfqpoint{3.965456in}{0.739656in}}%
\pgfpathlineto{\pgfqpoint{3.965159in}{0.739656in}}%
\pgfpathlineto{\pgfqpoint{3.964861in}{0.739656in}}%
\pgfpathlineto{\pgfqpoint{3.964564in}{0.739656in}}%
\pgfpathlineto{\pgfqpoint{3.964266in}{0.739656in}}%
\pgfpathlineto{\pgfqpoint{3.963969in}{0.739656in}}%
\pgfpathlineto{\pgfqpoint{3.963671in}{0.739656in}}%
\pgfpathlineto{\pgfqpoint{3.963374in}{0.739656in}}%
\pgfpathlineto{\pgfqpoint{3.963076in}{0.739656in}}%
\pgfpathlineto{\pgfqpoint{3.962779in}{0.739656in}}%
\pgfpathlineto{\pgfqpoint{3.962481in}{0.739656in}}%
\pgfpathlineto{\pgfqpoint{3.962184in}{0.739656in}}%
\pgfpathlineto{\pgfqpoint{3.961887in}{0.739656in}}%
\pgfpathlineto{\pgfqpoint{3.961589in}{0.739656in}}%
\pgfpathlineto{\pgfqpoint{3.961292in}{0.739656in}}%
\pgfpathlineto{\pgfqpoint{3.960994in}{0.739656in}}%
\pgfpathlineto{\pgfqpoint{3.960697in}{0.739656in}}%
\pgfpathlineto{\pgfqpoint{3.960399in}{0.739656in}}%
\pgfpathlineto{\pgfqpoint{3.960102in}{0.739656in}}%
\pgfpathlineto{\pgfqpoint{3.959804in}{0.739656in}}%
\pgfpathlineto{\pgfqpoint{3.959507in}{0.739656in}}%
\pgfpathlineto{\pgfqpoint{3.959209in}{0.739656in}}%
\pgfpathlineto{\pgfqpoint{3.958912in}{0.739656in}}%
\pgfpathlineto{\pgfqpoint{3.958614in}{0.739656in}}%
\pgfpathlineto{\pgfqpoint{3.958317in}{0.739656in}}%
\pgfpathlineto{\pgfqpoint{3.958019in}{0.739656in}}%
\pgfpathlineto{\pgfqpoint{3.957722in}{0.739656in}}%
\pgfpathlineto{\pgfqpoint{3.957424in}{0.739656in}}%
\pgfpathlineto{\pgfqpoint{3.957127in}{0.739656in}}%
\pgfpathlineto{\pgfqpoint{3.956829in}{0.739656in}}%
\pgfpathlineto{\pgfqpoint{3.956532in}{0.739656in}}%
\pgfpathlineto{\pgfqpoint{3.956234in}{0.739656in}}%
\pgfpathlineto{\pgfqpoint{3.955937in}{0.739656in}}%
\pgfpathlineto{\pgfqpoint{3.955639in}{0.739656in}}%
\pgfpathlineto{\pgfqpoint{3.955342in}{0.739656in}}%
\pgfpathlineto{\pgfqpoint{3.955045in}{0.739656in}}%
\pgfpathlineto{\pgfqpoint{3.954747in}{0.739656in}}%
\pgfpathlineto{\pgfqpoint{3.954450in}{0.739656in}}%
\pgfpathlineto{\pgfqpoint{3.954152in}{0.739656in}}%
\pgfpathlineto{\pgfqpoint{3.953855in}{0.739656in}}%
\pgfpathlineto{\pgfqpoint{3.953557in}{0.739656in}}%
\pgfpathlineto{\pgfqpoint{3.953260in}{0.739656in}}%
\pgfpathlineto{\pgfqpoint{3.952962in}{0.739656in}}%
\pgfpathlineto{\pgfqpoint{3.952665in}{0.739656in}}%
\pgfpathlineto{\pgfqpoint{3.952367in}{0.739656in}}%
\pgfpathlineto{\pgfqpoint{3.952070in}{0.739656in}}%
\pgfpathlineto{\pgfqpoint{3.951772in}{0.739656in}}%
\pgfpathlineto{\pgfqpoint{3.951475in}{0.739656in}}%
\pgfpathlineto{\pgfqpoint{3.951177in}{0.739656in}}%
\pgfpathlineto{\pgfqpoint{3.950880in}{0.739656in}}%
\pgfpathlineto{\pgfqpoint{3.950582in}{0.739656in}}%
\pgfpathlineto{\pgfqpoint{3.950285in}{0.739656in}}%
\pgfpathlineto{\pgfqpoint{3.949987in}{0.739656in}}%
\pgfpathlineto{\pgfqpoint{3.949690in}{0.739656in}}%
\pgfpathlineto{\pgfqpoint{3.949392in}{0.739656in}}%
\pgfpathlineto{\pgfqpoint{3.949095in}{0.739656in}}%
\pgfpathlineto{\pgfqpoint{3.948798in}{0.739656in}}%
\pgfpathlineto{\pgfqpoint{3.948500in}{0.739656in}}%
\pgfpathlineto{\pgfqpoint{3.948203in}{0.739656in}}%
\pgfpathlineto{\pgfqpoint{3.947905in}{0.739656in}}%
\pgfpathlineto{\pgfqpoint{3.947608in}{0.739656in}}%
\pgfpathlineto{\pgfqpoint{3.947310in}{0.739656in}}%
\pgfpathlineto{\pgfqpoint{3.947013in}{0.739656in}}%
\pgfpathlineto{\pgfqpoint{3.946715in}{0.739656in}}%
\pgfpathlineto{\pgfqpoint{3.946418in}{0.739656in}}%
\pgfpathlineto{\pgfqpoint{3.946120in}{0.739656in}}%
\pgfpathlineto{\pgfqpoint{3.945823in}{0.739656in}}%
\pgfpathlineto{\pgfqpoint{3.945525in}{0.739656in}}%
\pgfpathlineto{\pgfqpoint{3.945228in}{0.739656in}}%
\pgfpathlineto{\pgfqpoint{3.944930in}{0.739656in}}%
\pgfpathlineto{\pgfqpoint{3.944633in}{0.739656in}}%
\pgfpathlineto{\pgfqpoint{3.944335in}{0.739656in}}%
\pgfpathlineto{\pgfqpoint{3.944038in}{0.739656in}}%
\pgfpathlineto{\pgfqpoint{3.943740in}{0.739656in}}%
\pgfpathlineto{\pgfqpoint{3.943443in}{0.739656in}}%
\pgfpathlineto{\pgfqpoint{3.943145in}{0.739656in}}%
\pgfpathlineto{\pgfqpoint{3.942848in}{0.739656in}}%
\pgfpathlineto{\pgfqpoint{3.942550in}{0.739656in}}%
\pgfpathlineto{\pgfqpoint{3.942253in}{0.739656in}}%
\pgfpathlineto{\pgfqpoint{3.941956in}{0.739656in}}%
\pgfpathlineto{\pgfqpoint{3.941658in}{0.739656in}}%
\pgfpathlineto{\pgfqpoint{3.941361in}{0.739656in}}%
\pgfpathlineto{\pgfqpoint{3.941063in}{0.739656in}}%
\pgfpathlineto{\pgfqpoint{3.940766in}{0.739656in}}%
\pgfpathlineto{\pgfqpoint{3.940468in}{0.739656in}}%
\pgfpathlineto{\pgfqpoint{3.940171in}{0.739656in}}%
\pgfpathlineto{\pgfqpoint{3.939873in}{0.739656in}}%
\pgfpathlineto{\pgfqpoint{3.939576in}{0.739656in}}%
\pgfpathlineto{\pgfqpoint{3.939278in}{0.739656in}}%
\pgfpathlineto{\pgfqpoint{3.938981in}{0.739656in}}%
\pgfpathlineto{\pgfqpoint{3.938683in}{0.739656in}}%
\pgfpathlineto{\pgfqpoint{3.938386in}{0.739656in}}%
\pgfpathlineto{\pgfqpoint{3.938088in}{0.739656in}}%
\pgfpathlineto{\pgfqpoint{3.937791in}{0.739656in}}%
\pgfpathlineto{\pgfqpoint{3.937493in}{0.739656in}}%
\pgfpathlineto{\pgfqpoint{3.937196in}{0.739656in}}%
\pgfpathlineto{\pgfqpoint{3.936898in}{0.739656in}}%
\pgfpathlineto{\pgfqpoint{3.936601in}{0.739656in}}%
\pgfpathlineto{\pgfqpoint{3.936303in}{0.739656in}}%
\pgfpathlineto{\pgfqpoint{3.936006in}{0.739656in}}%
\pgfpathlineto{\pgfqpoint{3.935708in}{0.739656in}}%
\pgfpathlineto{\pgfqpoint{3.935411in}{0.739656in}}%
\pgfpathlineto{\pgfqpoint{3.935114in}{0.739656in}}%
\pgfpathlineto{\pgfqpoint{3.934816in}{0.739656in}}%
\pgfpathlineto{\pgfqpoint{3.934519in}{0.739656in}}%
\pgfpathlineto{\pgfqpoint{3.934221in}{0.739656in}}%
\pgfpathlineto{\pgfqpoint{3.933924in}{0.739656in}}%
\pgfpathlineto{\pgfqpoint{3.933626in}{0.739656in}}%
\pgfpathlineto{\pgfqpoint{3.933329in}{0.739656in}}%
\pgfpathlineto{\pgfqpoint{3.933031in}{0.739656in}}%
\pgfpathlineto{\pgfqpoint{3.932734in}{0.739656in}}%
\pgfpathlineto{\pgfqpoint{3.932436in}{0.739656in}}%
\pgfpathlineto{\pgfqpoint{3.932139in}{0.739656in}}%
\pgfpathlineto{\pgfqpoint{3.931841in}{0.739656in}}%
\pgfpathlineto{\pgfqpoint{3.931544in}{0.739656in}}%
\pgfpathlineto{\pgfqpoint{3.931246in}{0.739656in}}%
\pgfpathlineto{\pgfqpoint{3.930949in}{0.739656in}}%
\pgfpathlineto{\pgfqpoint{3.930651in}{0.739656in}}%
\pgfpathlineto{\pgfqpoint{3.930354in}{0.739656in}}%
\pgfpathlineto{\pgfqpoint{3.930056in}{0.739656in}}%
\pgfpathlineto{\pgfqpoint{3.929759in}{0.739656in}}%
\pgfpathlineto{\pgfqpoint{3.929461in}{0.739656in}}%
\pgfpathlineto{\pgfqpoint{3.929164in}{0.739656in}}%
\pgfpathlineto{\pgfqpoint{3.928867in}{0.739656in}}%
\pgfpathlineto{\pgfqpoint{3.928569in}{0.739656in}}%
\pgfpathlineto{\pgfqpoint{3.928272in}{0.739656in}}%
\pgfpathlineto{\pgfqpoint{3.927974in}{0.739656in}}%
\pgfpathlineto{\pgfqpoint{3.927677in}{0.739656in}}%
\pgfpathlineto{\pgfqpoint{3.927379in}{0.739656in}}%
\pgfpathlineto{\pgfqpoint{3.927082in}{0.739656in}}%
\pgfpathlineto{\pgfqpoint{3.926784in}{0.739656in}}%
\pgfpathlineto{\pgfqpoint{3.926487in}{0.739656in}}%
\pgfpathlineto{\pgfqpoint{3.926189in}{0.739656in}}%
\pgfpathlineto{\pgfqpoint{3.925892in}{0.739656in}}%
\pgfpathlineto{\pgfqpoint{3.925594in}{0.739656in}}%
\pgfpathlineto{\pgfqpoint{3.925297in}{0.739656in}}%
\pgfpathlineto{\pgfqpoint{3.924999in}{0.739656in}}%
\pgfpathlineto{\pgfqpoint{3.924702in}{0.739656in}}%
\pgfpathlineto{\pgfqpoint{3.924404in}{0.739656in}}%
\pgfpathlineto{\pgfqpoint{3.924107in}{0.739656in}}%
\pgfpathlineto{\pgfqpoint{3.923809in}{0.739656in}}%
\pgfpathlineto{\pgfqpoint{3.923512in}{0.739656in}}%
\pgfpathlineto{\pgfqpoint{3.923214in}{0.739656in}}%
\pgfpathlineto{\pgfqpoint{3.922917in}{0.739656in}}%
\pgfpathlineto{\pgfqpoint{3.922619in}{0.739656in}}%
\pgfpathlineto{\pgfqpoint{3.922322in}{0.739656in}}%
\pgfpathlineto{\pgfqpoint{3.922025in}{0.739656in}}%
\pgfpathlineto{\pgfqpoint{3.921727in}{0.739656in}}%
\pgfpathlineto{\pgfqpoint{3.921430in}{0.739656in}}%
\pgfpathlineto{\pgfqpoint{3.921132in}{0.739656in}}%
\pgfpathlineto{\pgfqpoint{3.920835in}{0.739656in}}%
\pgfpathlineto{\pgfqpoint{3.920537in}{0.739656in}}%
\pgfpathlineto{\pgfqpoint{3.920240in}{0.739656in}}%
\pgfpathlineto{\pgfqpoint{3.919942in}{0.739656in}}%
\pgfpathlineto{\pgfqpoint{3.919645in}{0.739656in}}%
\pgfpathlineto{\pgfqpoint{3.919347in}{0.739656in}}%
\pgfpathlineto{\pgfqpoint{3.919050in}{0.739656in}}%
\pgfpathlineto{\pgfqpoint{3.918752in}{0.739656in}}%
\pgfpathlineto{\pgfqpoint{3.918455in}{0.739656in}}%
\pgfpathlineto{\pgfqpoint{3.918157in}{0.739656in}}%
\pgfpathlineto{\pgfqpoint{3.917860in}{0.739656in}}%
\pgfpathlineto{\pgfqpoint{3.917562in}{0.739656in}}%
\pgfpathlineto{\pgfqpoint{3.917265in}{0.739656in}}%
\pgfpathlineto{\pgfqpoint{3.916967in}{0.739656in}}%
\pgfpathlineto{\pgfqpoint{3.916670in}{0.739656in}}%
\pgfpathlineto{\pgfqpoint{3.916372in}{0.739656in}}%
\pgfpathlineto{\pgfqpoint{3.916075in}{0.739656in}}%
\pgfpathlineto{\pgfqpoint{3.915777in}{0.739656in}}%
\pgfpathlineto{\pgfqpoint{3.915480in}{0.739656in}}%
\pgfpathlineto{\pgfqpoint{3.915183in}{0.739656in}}%
\pgfpathlineto{\pgfqpoint{3.914885in}{0.739656in}}%
\pgfpathlineto{\pgfqpoint{3.914588in}{0.739656in}}%
\pgfpathlineto{\pgfqpoint{3.914290in}{0.739656in}}%
\pgfpathlineto{\pgfqpoint{3.913993in}{0.739656in}}%
\pgfpathlineto{\pgfqpoint{3.913695in}{0.739656in}}%
\pgfpathlineto{\pgfqpoint{3.913398in}{0.739656in}}%
\pgfpathlineto{\pgfqpoint{3.913100in}{0.739656in}}%
\pgfpathlineto{\pgfqpoint{3.912803in}{0.739656in}}%
\pgfpathlineto{\pgfqpoint{3.912505in}{0.739656in}}%
\pgfpathlineto{\pgfqpoint{3.912208in}{0.739656in}}%
\pgfpathlineto{\pgfqpoint{3.911910in}{0.739656in}}%
\pgfpathlineto{\pgfqpoint{3.911613in}{0.739656in}}%
\pgfpathlineto{\pgfqpoint{3.911315in}{0.739656in}}%
\pgfpathlineto{\pgfqpoint{3.911018in}{0.739656in}}%
\pgfpathlineto{\pgfqpoint{3.910720in}{0.739656in}}%
\pgfpathlineto{\pgfqpoint{3.910423in}{0.739656in}}%
\pgfpathlineto{\pgfqpoint{3.910125in}{0.739656in}}%
\pgfpathlineto{\pgfqpoint{3.909828in}{0.739656in}}%
\pgfpathlineto{\pgfqpoint{3.909530in}{0.739656in}}%
\pgfpathlineto{\pgfqpoint{3.909233in}{0.739656in}}%
\pgfpathlineto{\pgfqpoint{3.908936in}{0.739656in}}%
\pgfpathlineto{\pgfqpoint{3.908638in}{0.739656in}}%
\pgfpathlineto{\pgfqpoint{3.908341in}{0.739656in}}%
\pgfpathlineto{\pgfqpoint{3.908043in}{0.739656in}}%
\pgfpathlineto{\pgfqpoint{3.907746in}{0.739656in}}%
\pgfpathlineto{\pgfqpoint{3.907448in}{0.739656in}}%
\pgfpathlineto{\pgfqpoint{3.907151in}{0.739656in}}%
\pgfpathlineto{\pgfqpoint{3.906853in}{0.739656in}}%
\pgfpathlineto{\pgfqpoint{3.906556in}{0.739656in}}%
\pgfpathlineto{\pgfqpoint{3.906258in}{0.739656in}}%
\pgfpathlineto{\pgfqpoint{3.905961in}{0.739656in}}%
\pgfpathlineto{\pgfqpoint{3.905663in}{0.739656in}}%
\pgfpathlineto{\pgfqpoint{3.905366in}{0.739656in}}%
\pgfpathlineto{\pgfqpoint{3.905068in}{0.739656in}}%
\pgfpathlineto{\pgfqpoint{3.904771in}{0.739656in}}%
\pgfpathlineto{\pgfqpoint{3.904473in}{0.739656in}}%
\pgfpathlineto{\pgfqpoint{3.904176in}{0.739656in}}%
\pgfpathlineto{\pgfqpoint{3.903878in}{0.739656in}}%
\pgfpathlineto{\pgfqpoint{3.903581in}{0.739656in}}%
\pgfpathlineto{\pgfqpoint{3.903283in}{0.739656in}}%
\pgfpathlineto{\pgfqpoint{3.902986in}{0.739656in}}%
\pgfpathlineto{\pgfqpoint{3.902688in}{0.739656in}}%
\pgfpathlineto{\pgfqpoint{3.902391in}{0.739656in}}%
\pgfpathlineto{\pgfqpoint{3.902094in}{0.739656in}}%
\pgfpathlineto{\pgfqpoint{3.901796in}{0.739656in}}%
\pgfpathlineto{\pgfqpoint{3.901499in}{0.739656in}}%
\pgfpathlineto{\pgfqpoint{3.901201in}{0.739656in}}%
\pgfpathlineto{\pgfqpoint{3.900904in}{0.739656in}}%
\pgfpathlineto{\pgfqpoint{3.900606in}{0.739656in}}%
\pgfpathlineto{\pgfqpoint{3.900309in}{0.739656in}}%
\pgfpathlineto{\pgfqpoint{3.900011in}{0.739656in}}%
\pgfpathlineto{\pgfqpoint{3.899714in}{0.739656in}}%
\pgfpathlineto{\pgfqpoint{3.899416in}{0.739656in}}%
\pgfpathlineto{\pgfqpoint{3.899119in}{0.739656in}}%
\pgfpathlineto{\pgfqpoint{3.898821in}{0.739656in}}%
\pgfpathlineto{\pgfqpoint{3.898524in}{0.739656in}}%
\pgfpathlineto{\pgfqpoint{3.898226in}{0.739656in}}%
\pgfpathlineto{\pgfqpoint{3.897929in}{0.739656in}}%
\pgfpathlineto{\pgfqpoint{3.897631in}{0.739656in}}%
\pgfpathlineto{\pgfqpoint{3.897334in}{0.739656in}}%
\pgfpathlineto{\pgfqpoint{3.897036in}{0.739656in}}%
\pgfpathlineto{\pgfqpoint{3.896739in}{0.739656in}}%
\pgfpathlineto{\pgfqpoint{3.896441in}{0.739656in}}%
\pgfpathlineto{\pgfqpoint{3.896144in}{0.739656in}}%
\pgfpathlineto{\pgfqpoint{3.895846in}{0.739656in}}%
\pgfpathlineto{\pgfqpoint{3.895549in}{0.739656in}}%
\pgfpathlineto{\pgfqpoint{3.895252in}{0.739656in}}%
\pgfpathlineto{\pgfqpoint{3.894954in}{0.739656in}}%
\pgfpathlineto{\pgfqpoint{3.894657in}{0.739656in}}%
\pgfpathlineto{\pgfqpoint{3.894359in}{0.739656in}}%
\pgfpathlineto{\pgfqpoint{3.894062in}{0.739656in}}%
\pgfpathlineto{\pgfqpoint{3.893764in}{0.739656in}}%
\pgfpathlineto{\pgfqpoint{3.893467in}{0.739656in}}%
\pgfpathlineto{\pgfqpoint{3.893169in}{0.739656in}}%
\pgfpathlineto{\pgfqpoint{3.892872in}{0.739656in}}%
\pgfpathlineto{\pgfqpoint{3.892574in}{0.739656in}}%
\pgfpathlineto{\pgfqpoint{3.892277in}{0.739656in}}%
\pgfpathlineto{\pgfqpoint{3.891979in}{0.739656in}}%
\pgfpathlineto{\pgfqpoint{3.891682in}{0.739656in}}%
\pgfpathlineto{\pgfqpoint{3.891384in}{0.739656in}}%
\pgfpathlineto{\pgfqpoint{3.891087in}{0.739656in}}%
\pgfpathlineto{\pgfqpoint{3.890789in}{0.739656in}}%
\pgfpathlineto{\pgfqpoint{3.890492in}{0.739656in}}%
\pgfpathlineto{\pgfqpoint{3.890194in}{0.739656in}}%
\pgfpathlineto{\pgfqpoint{3.889897in}{0.739656in}}%
\pgfpathlineto{\pgfqpoint{3.889599in}{0.739656in}}%
\pgfpathlineto{\pgfqpoint{3.889302in}{0.739656in}}%
\pgfpathlineto{\pgfqpoint{3.889005in}{0.739656in}}%
\pgfpathlineto{\pgfqpoint{3.888707in}{0.739656in}}%
\pgfpathlineto{\pgfqpoint{3.888410in}{0.739656in}}%
\pgfpathlineto{\pgfqpoint{3.888112in}{0.739656in}}%
\pgfpathlineto{\pgfqpoint{3.887815in}{0.739656in}}%
\pgfpathlineto{\pgfqpoint{3.887517in}{0.739656in}}%
\pgfpathlineto{\pgfqpoint{3.887220in}{0.739656in}}%
\pgfpathlineto{\pgfqpoint{3.886922in}{0.739656in}}%
\pgfpathlineto{\pgfqpoint{3.886625in}{0.739656in}}%
\pgfpathlineto{\pgfqpoint{3.886327in}{0.739656in}}%
\pgfpathlineto{\pgfqpoint{3.886030in}{0.739656in}}%
\pgfpathlineto{\pgfqpoint{3.885732in}{0.739656in}}%
\pgfpathlineto{\pgfqpoint{3.885435in}{0.739656in}}%
\pgfpathlineto{\pgfqpoint{3.885137in}{0.739656in}}%
\pgfpathlineto{\pgfqpoint{3.884840in}{0.739656in}}%
\pgfpathlineto{\pgfqpoint{3.884542in}{0.739656in}}%
\pgfpathlineto{\pgfqpoint{3.884245in}{0.739656in}}%
\pgfpathlineto{\pgfqpoint{3.883947in}{0.739656in}}%
\pgfpathlineto{\pgfqpoint{3.883650in}{0.739656in}}%
\pgfpathlineto{\pgfqpoint{3.883352in}{0.739656in}}%
\pgfpathlineto{\pgfqpoint{3.883055in}{0.739656in}}%
\pgfpathlineto{\pgfqpoint{3.882757in}{0.739656in}}%
\pgfpathlineto{\pgfqpoint{3.882460in}{0.739656in}}%
\pgfpathlineto{\pgfqpoint{3.882163in}{0.739656in}}%
\pgfpathlineto{\pgfqpoint{3.881865in}{0.739656in}}%
\pgfpathlineto{\pgfqpoint{3.881568in}{0.739656in}}%
\pgfpathlineto{\pgfqpoint{3.881270in}{0.739656in}}%
\pgfpathlineto{\pgfqpoint{3.880973in}{0.739656in}}%
\pgfpathlineto{\pgfqpoint{3.880675in}{0.739656in}}%
\pgfpathlineto{\pgfqpoint{3.880378in}{0.739656in}}%
\pgfpathlineto{\pgfqpoint{3.880080in}{0.739656in}}%
\pgfpathlineto{\pgfqpoint{3.879783in}{0.739656in}}%
\pgfpathlineto{\pgfqpoint{3.879485in}{0.739656in}}%
\pgfpathlineto{\pgfqpoint{3.879188in}{0.739656in}}%
\pgfpathlineto{\pgfqpoint{3.878890in}{0.739656in}}%
\pgfpathlineto{\pgfqpoint{3.878593in}{0.739656in}}%
\pgfpathlineto{\pgfqpoint{3.878295in}{0.739656in}}%
\pgfpathlineto{\pgfqpoint{3.877998in}{0.739656in}}%
\pgfpathlineto{\pgfqpoint{3.877700in}{0.739656in}}%
\pgfpathlineto{\pgfqpoint{3.877403in}{0.739656in}}%
\pgfpathlineto{\pgfqpoint{3.877105in}{0.739656in}}%
\pgfpathlineto{\pgfqpoint{3.876808in}{0.739656in}}%
\pgfpathlineto{\pgfqpoint{3.876510in}{0.739656in}}%
\pgfpathlineto{\pgfqpoint{3.876213in}{0.739656in}}%
\pgfpathlineto{\pgfqpoint{3.875915in}{0.739656in}}%
\pgfpathlineto{\pgfqpoint{3.875618in}{0.739656in}}%
\pgfpathlineto{\pgfqpoint{3.875321in}{0.739656in}}%
\pgfpathlineto{\pgfqpoint{3.875023in}{0.739656in}}%
\pgfpathlineto{\pgfqpoint{3.874726in}{0.739656in}}%
\pgfpathlineto{\pgfqpoint{3.874428in}{0.739656in}}%
\pgfpathlineto{\pgfqpoint{3.874131in}{0.739656in}}%
\pgfpathlineto{\pgfqpoint{3.873833in}{0.739656in}}%
\pgfpathlineto{\pgfqpoint{3.873536in}{0.739656in}}%
\pgfpathlineto{\pgfqpoint{3.873238in}{0.739656in}}%
\pgfpathlineto{\pgfqpoint{3.872941in}{0.739656in}}%
\pgfpathlineto{\pgfqpoint{3.872643in}{0.739656in}}%
\pgfpathlineto{\pgfqpoint{3.872346in}{0.739656in}}%
\pgfpathlineto{\pgfqpoint{3.872048in}{0.739656in}}%
\pgfpathlineto{\pgfqpoint{3.871751in}{0.739656in}}%
\pgfpathlineto{\pgfqpoint{3.871453in}{0.739656in}}%
\pgfpathlineto{\pgfqpoint{3.871156in}{0.739656in}}%
\pgfpathlineto{\pgfqpoint{3.870858in}{0.739656in}}%
\pgfpathlineto{\pgfqpoint{3.870561in}{0.739656in}}%
\pgfpathlineto{\pgfqpoint{3.870263in}{0.739656in}}%
\pgfpathlineto{\pgfqpoint{3.869966in}{0.739656in}}%
\pgfpathlineto{\pgfqpoint{3.869668in}{0.739656in}}%
\pgfpathlineto{\pgfqpoint{3.869371in}{0.739656in}}%
\pgfpathlineto{\pgfqpoint{3.869074in}{0.739656in}}%
\pgfpathlineto{\pgfqpoint{3.868776in}{0.739656in}}%
\pgfpathlineto{\pgfqpoint{3.868479in}{0.739656in}}%
\pgfpathlineto{\pgfqpoint{3.868181in}{0.739656in}}%
\pgfpathlineto{\pgfqpoint{3.867884in}{0.739656in}}%
\pgfpathlineto{\pgfqpoint{3.867586in}{0.739656in}}%
\pgfpathlineto{\pgfqpoint{3.867289in}{0.739656in}}%
\pgfpathlineto{\pgfqpoint{3.866991in}{0.739656in}}%
\pgfpathlineto{\pgfqpoint{3.866694in}{0.739656in}}%
\pgfpathlineto{\pgfqpoint{3.866396in}{0.739656in}}%
\pgfpathlineto{\pgfqpoint{3.866099in}{0.739656in}}%
\pgfpathlineto{\pgfqpoint{3.865801in}{0.739656in}}%
\pgfpathlineto{\pgfqpoint{3.865504in}{0.739656in}}%
\pgfpathlineto{\pgfqpoint{3.865206in}{0.739656in}}%
\pgfpathlineto{\pgfqpoint{3.864909in}{0.739656in}}%
\pgfpathlineto{\pgfqpoint{3.864611in}{0.739656in}}%
\pgfpathlineto{\pgfqpoint{3.864314in}{0.739656in}}%
\pgfpathlineto{\pgfqpoint{3.864016in}{0.739656in}}%
\pgfpathlineto{\pgfqpoint{3.863719in}{0.739656in}}%
\pgfpathlineto{\pgfqpoint{3.863421in}{0.739656in}}%
\pgfpathlineto{\pgfqpoint{3.863124in}{0.739656in}}%
\pgfpathlineto{\pgfqpoint{3.862826in}{0.739656in}}%
\pgfpathlineto{\pgfqpoint{3.862529in}{0.739656in}}%
\pgfpathlineto{\pgfqpoint{3.862232in}{0.739656in}}%
\pgfpathlineto{\pgfqpoint{3.861934in}{0.739656in}}%
\pgfpathlineto{\pgfqpoint{3.861637in}{0.739656in}}%
\pgfpathlineto{\pgfqpoint{3.861339in}{0.739656in}}%
\pgfpathlineto{\pgfqpoint{3.861042in}{0.739656in}}%
\pgfpathlineto{\pgfqpoint{3.860744in}{0.739656in}}%
\pgfpathlineto{\pgfqpoint{3.860447in}{0.739656in}}%
\pgfpathlineto{\pgfqpoint{3.860149in}{0.739656in}}%
\pgfpathlineto{\pgfqpoint{3.859852in}{0.739656in}}%
\pgfpathlineto{\pgfqpoint{3.859554in}{0.739656in}}%
\pgfpathlineto{\pgfqpoint{3.859257in}{0.739656in}}%
\pgfpathlineto{\pgfqpoint{3.858959in}{0.739656in}}%
\pgfpathlineto{\pgfqpoint{3.858662in}{0.739656in}}%
\pgfpathlineto{\pgfqpoint{3.858364in}{0.739656in}}%
\pgfpathlineto{\pgfqpoint{3.858067in}{0.739656in}}%
\pgfpathlineto{\pgfqpoint{3.857769in}{0.739656in}}%
\pgfpathlineto{\pgfqpoint{3.857472in}{0.739656in}}%
\pgfpathlineto{\pgfqpoint{3.857174in}{0.739656in}}%
\pgfpathlineto{\pgfqpoint{3.856877in}{0.739656in}}%
\pgfpathlineto{\pgfqpoint{3.856579in}{0.739656in}}%
\pgfpathlineto{\pgfqpoint{3.856282in}{0.739656in}}%
\pgfpathlineto{\pgfqpoint{3.855984in}{0.739656in}}%
\pgfpathlineto{\pgfqpoint{3.855687in}{0.739656in}}%
\pgfpathlineto{\pgfqpoint{3.855390in}{0.739656in}}%
\pgfpathlineto{\pgfqpoint{3.855092in}{0.739656in}}%
\pgfpathlineto{\pgfqpoint{3.854795in}{0.739656in}}%
\pgfpathlineto{\pgfqpoint{3.854497in}{0.739656in}}%
\pgfpathlineto{\pgfqpoint{3.854200in}{0.739656in}}%
\pgfpathlineto{\pgfqpoint{3.853902in}{0.739656in}}%
\pgfpathlineto{\pgfqpoint{3.853605in}{0.739656in}}%
\pgfpathlineto{\pgfqpoint{3.853307in}{0.739656in}}%
\pgfpathlineto{\pgfqpoint{3.853010in}{0.739656in}}%
\pgfpathlineto{\pgfqpoint{3.852712in}{0.739656in}}%
\pgfpathlineto{\pgfqpoint{3.852415in}{0.739656in}}%
\pgfpathlineto{\pgfqpoint{3.852117in}{0.739656in}}%
\pgfpathlineto{\pgfqpoint{3.851820in}{0.739656in}}%
\pgfpathlineto{\pgfqpoint{3.851522in}{0.739656in}}%
\pgfpathlineto{\pgfqpoint{3.851225in}{0.739656in}}%
\pgfpathlineto{\pgfqpoint{3.850927in}{0.739656in}}%
\pgfpathlineto{\pgfqpoint{3.850630in}{0.739656in}}%
\pgfpathlineto{\pgfqpoint{3.850332in}{0.739656in}}%
\pgfpathlineto{\pgfqpoint{3.850035in}{0.739656in}}%
\pgfpathlineto{\pgfqpoint{3.849737in}{0.739656in}}%
\pgfpathlineto{\pgfqpoint{3.849440in}{0.739656in}}%
\pgfpathlineto{\pgfqpoint{3.849143in}{0.739656in}}%
\pgfpathlineto{\pgfqpoint{3.848845in}{0.739656in}}%
\pgfpathlineto{\pgfqpoint{3.848548in}{0.739656in}}%
\pgfpathlineto{\pgfqpoint{3.848250in}{0.739656in}}%
\pgfpathlineto{\pgfqpoint{3.847953in}{0.739656in}}%
\pgfpathlineto{\pgfqpoint{3.847655in}{0.739656in}}%
\pgfpathlineto{\pgfqpoint{3.847358in}{0.739656in}}%
\pgfpathlineto{\pgfqpoint{3.847060in}{0.739656in}}%
\pgfpathlineto{\pgfqpoint{3.846763in}{0.739656in}}%
\pgfpathlineto{\pgfqpoint{3.846465in}{0.739656in}}%
\pgfpathlineto{\pgfqpoint{3.846168in}{0.739656in}}%
\pgfpathlineto{\pgfqpoint{3.845870in}{0.739656in}}%
\pgfpathlineto{\pgfqpoint{3.845573in}{0.739656in}}%
\pgfpathlineto{\pgfqpoint{3.845275in}{0.739656in}}%
\pgfpathlineto{\pgfqpoint{3.844978in}{0.739656in}}%
\pgfpathlineto{\pgfqpoint{3.844680in}{0.739656in}}%
\pgfpathlineto{\pgfqpoint{3.844383in}{0.739656in}}%
\pgfpathlineto{\pgfqpoint{3.844085in}{0.739656in}}%
\pgfpathlineto{\pgfqpoint{3.843788in}{0.739656in}}%
\pgfpathlineto{\pgfqpoint{3.843490in}{0.739656in}}%
\pgfpathlineto{\pgfqpoint{3.843193in}{0.739656in}}%
\pgfpathlineto{\pgfqpoint{3.842895in}{0.739656in}}%
\pgfpathlineto{\pgfqpoint{3.842598in}{0.739656in}}%
\pgfpathlineto{\pgfqpoint{3.842301in}{0.739656in}}%
\pgfpathlineto{\pgfqpoint{3.842003in}{0.739656in}}%
\pgfpathlineto{\pgfqpoint{3.841706in}{0.739656in}}%
\pgfpathlineto{\pgfqpoint{3.841408in}{0.739656in}}%
\pgfpathlineto{\pgfqpoint{3.841111in}{0.739656in}}%
\pgfpathlineto{\pgfqpoint{3.840813in}{0.739656in}}%
\pgfpathlineto{\pgfqpoint{3.840516in}{0.739656in}}%
\pgfpathlineto{\pgfqpoint{3.840218in}{0.739656in}}%
\pgfpathlineto{\pgfqpoint{3.839921in}{0.739656in}}%
\pgfpathlineto{\pgfqpoint{3.839623in}{0.739656in}}%
\pgfpathlineto{\pgfqpoint{3.839326in}{0.739656in}}%
\pgfpathlineto{\pgfqpoint{3.839028in}{0.739656in}}%
\pgfpathlineto{\pgfqpoint{3.838731in}{0.739656in}}%
\pgfpathlineto{\pgfqpoint{3.838433in}{0.739656in}}%
\pgfpathlineto{\pgfqpoint{3.838136in}{0.739656in}}%
\pgfpathlineto{\pgfqpoint{3.837838in}{0.739656in}}%
\pgfpathlineto{\pgfqpoint{3.837541in}{0.739656in}}%
\pgfpathlineto{\pgfqpoint{3.837243in}{0.739656in}}%
\pgfpathlineto{\pgfqpoint{3.836946in}{0.739656in}}%
\pgfpathlineto{\pgfqpoint{3.836648in}{0.739656in}}%
\pgfpathlineto{\pgfqpoint{3.836351in}{0.739656in}}%
\pgfpathlineto{\pgfqpoint{3.836053in}{0.739656in}}%
\pgfpathlineto{\pgfqpoint{3.835756in}{0.739656in}}%
\pgfpathlineto{\pgfqpoint{3.835459in}{0.739656in}}%
\pgfpathlineto{\pgfqpoint{3.835161in}{0.739656in}}%
\pgfpathlineto{\pgfqpoint{3.834864in}{0.739656in}}%
\pgfpathlineto{\pgfqpoint{3.834566in}{0.739656in}}%
\pgfpathlineto{\pgfqpoint{3.834269in}{0.739656in}}%
\pgfpathlineto{\pgfqpoint{3.833971in}{0.739656in}}%
\pgfpathlineto{\pgfqpoint{3.833674in}{0.739656in}}%
\pgfpathlineto{\pgfqpoint{3.833376in}{0.739656in}}%
\pgfpathlineto{\pgfqpoint{3.833079in}{0.739656in}}%
\pgfpathlineto{\pgfqpoint{3.832781in}{0.739656in}}%
\pgfpathlineto{\pgfqpoint{3.832484in}{0.739656in}}%
\pgfpathlineto{\pgfqpoint{3.832186in}{0.739656in}}%
\pgfpathlineto{\pgfqpoint{3.831889in}{0.739656in}}%
\pgfpathlineto{\pgfqpoint{3.831591in}{0.739656in}}%
\pgfpathlineto{\pgfqpoint{3.831294in}{0.739656in}}%
\pgfpathlineto{\pgfqpoint{3.830996in}{0.739656in}}%
\pgfpathlineto{\pgfqpoint{3.830699in}{0.739656in}}%
\pgfpathlineto{\pgfqpoint{3.830401in}{0.739656in}}%
\pgfpathlineto{\pgfqpoint{3.830104in}{0.739656in}}%
\pgfpathlineto{\pgfqpoint{3.829806in}{0.739656in}}%
\pgfpathlineto{\pgfqpoint{3.829509in}{0.739656in}}%
\pgfpathlineto{\pgfqpoint{3.829212in}{0.739656in}}%
\pgfpathlineto{\pgfqpoint{3.828914in}{0.739656in}}%
\pgfpathlineto{\pgfqpoint{3.828617in}{0.739656in}}%
\pgfpathlineto{\pgfqpoint{3.828319in}{0.739656in}}%
\pgfpathlineto{\pgfqpoint{3.828022in}{0.739656in}}%
\pgfpathlineto{\pgfqpoint{3.827724in}{0.739656in}}%
\pgfpathlineto{\pgfqpoint{3.827427in}{0.739656in}}%
\pgfpathlineto{\pgfqpoint{3.827129in}{0.739656in}}%
\pgfpathlineto{\pgfqpoint{3.826832in}{0.739656in}}%
\pgfpathlineto{\pgfqpoint{3.826534in}{0.739656in}}%
\pgfpathlineto{\pgfqpoint{3.826237in}{0.739656in}}%
\pgfpathlineto{\pgfqpoint{3.825939in}{0.739656in}}%
\pgfpathlineto{\pgfqpoint{3.825642in}{0.739656in}}%
\pgfpathlineto{\pgfqpoint{3.825344in}{0.739656in}}%
\pgfpathlineto{\pgfqpoint{3.825047in}{0.739656in}}%
\pgfpathlineto{\pgfqpoint{3.824749in}{0.739656in}}%
\pgfpathlineto{\pgfqpoint{3.824452in}{0.739656in}}%
\pgfpathlineto{\pgfqpoint{3.824154in}{0.739656in}}%
\pgfpathlineto{\pgfqpoint{3.823857in}{0.739656in}}%
\pgfpathlineto{\pgfqpoint{3.823559in}{0.739656in}}%
\pgfpathlineto{\pgfqpoint{3.823262in}{0.739656in}}%
\pgfpathlineto{\pgfqpoint{3.822964in}{0.739656in}}%
\pgfpathlineto{\pgfqpoint{3.822667in}{0.739656in}}%
\pgfpathlineto{\pgfqpoint{3.822370in}{0.739656in}}%
\pgfpathlineto{\pgfqpoint{3.822072in}{0.739656in}}%
\pgfpathlineto{\pgfqpoint{3.821775in}{0.739656in}}%
\pgfpathlineto{\pgfqpoint{3.821477in}{0.739656in}}%
\pgfpathlineto{\pgfqpoint{3.821180in}{0.739656in}}%
\pgfpathlineto{\pgfqpoint{3.820882in}{0.739656in}}%
\pgfpathlineto{\pgfqpoint{3.820585in}{0.739656in}}%
\pgfpathlineto{\pgfqpoint{3.820287in}{0.739656in}}%
\pgfpathlineto{\pgfqpoint{3.819990in}{0.739656in}}%
\pgfpathlineto{\pgfqpoint{3.819692in}{0.739656in}}%
\pgfpathlineto{\pgfqpoint{3.819395in}{0.739656in}}%
\pgfpathlineto{\pgfqpoint{3.819097in}{0.739656in}}%
\pgfpathlineto{\pgfqpoint{3.818800in}{0.739656in}}%
\pgfpathlineto{\pgfqpoint{3.818502in}{0.739656in}}%
\pgfpathlineto{\pgfqpoint{3.818205in}{0.739656in}}%
\pgfpathlineto{\pgfqpoint{3.817907in}{0.739656in}}%
\pgfpathlineto{\pgfqpoint{3.817610in}{0.739656in}}%
\pgfpathlineto{\pgfqpoint{3.817312in}{0.739656in}}%
\pgfpathlineto{\pgfqpoint{3.817015in}{0.739656in}}%
\pgfpathlineto{\pgfqpoint{3.816717in}{0.739656in}}%
\pgfpathlineto{\pgfqpoint{3.816420in}{0.739656in}}%
\pgfpathlineto{\pgfqpoint{3.816122in}{0.739656in}}%
\pgfpathlineto{\pgfqpoint{3.815825in}{0.739656in}}%
\pgfpathlineto{\pgfqpoint{3.815528in}{0.739656in}}%
\pgfpathlineto{\pgfqpoint{3.815230in}{0.739656in}}%
\pgfpathlineto{\pgfqpoint{3.814933in}{0.739656in}}%
\pgfpathlineto{\pgfqpoint{3.814635in}{0.739656in}}%
\pgfpathlineto{\pgfqpoint{3.814338in}{0.739656in}}%
\pgfpathlineto{\pgfqpoint{3.814040in}{0.739656in}}%
\pgfpathlineto{\pgfqpoint{3.813743in}{0.739656in}}%
\pgfpathlineto{\pgfqpoint{3.813445in}{0.739656in}}%
\pgfpathlineto{\pgfqpoint{3.813148in}{0.739656in}}%
\pgfpathlineto{\pgfqpoint{3.812850in}{0.739656in}}%
\pgfpathlineto{\pgfqpoint{3.812553in}{0.739656in}}%
\pgfpathlineto{\pgfqpoint{3.812255in}{0.739656in}}%
\pgfpathlineto{\pgfqpoint{3.811958in}{0.739656in}}%
\pgfpathlineto{\pgfqpoint{3.811660in}{0.739656in}}%
\pgfpathlineto{\pgfqpoint{3.811363in}{0.739656in}}%
\pgfpathlineto{\pgfqpoint{3.811065in}{0.739656in}}%
\pgfpathlineto{\pgfqpoint{3.810768in}{0.739656in}}%
\pgfpathlineto{\pgfqpoint{3.810470in}{0.739656in}}%
\pgfpathlineto{\pgfqpoint{3.810173in}{0.739656in}}%
\pgfpathlineto{\pgfqpoint{3.809875in}{0.739656in}}%
\pgfpathlineto{\pgfqpoint{3.809578in}{0.739656in}}%
\pgfpathlineto{\pgfqpoint{3.809281in}{0.739656in}}%
\pgfpathlineto{\pgfqpoint{3.808983in}{0.739656in}}%
\pgfpathlineto{\pgfqpoint{3.808686in}{0.739656in}}%
\pgfpathlineto{\pgfqpoint{3.808388in}{0.739656in}}%
\pgfpathlineto{\pgfqpoint{3.808091in}{0.739656in}}%
\pgfpathlineto{\pgfqpoint{3.807793in}{0.739656in}}%
\pgfpathlineto{\pgfqpoint{3.807496in}{0.739656in}}%
\pgfpathlineto{\pgfqpoint{3.807198in}{0.739656in}}%
\pgfpathlineto{\pgfqpoint{3.806901in}{0.739656in}}%
\pgfpathlineto{\pgfqpoint{3.806603in}{0.739656in}}%
\pgfpathlineto{\pgfqpoint{3.806306in}{0.739656in}}%
\pgfpathlineto{\pgfqpoint{3.806008in}{0.739656in}}%
\pgfpathlineto{\pgfqpoint{3.805711in}{0.739656in}}%
\pgfpathlineto{\pgfqpoint{3.805413in}{0.739656in}}%
\pgfpathlineto{\pgfqpoint{3.805116in}{0.739656in}}%
\pgfpathlineto{\pgfqpoint{3.804818in}{0.739656in}}%
\pgfpathlineto{\pgfqpoint{3.804521in}{0.739656in}}%
\pgfpathlineto{\pgfqpoint{3.804223in}{0.739656in}}%
\pgfpathlineto{\pgfqpoint{3.803926in}{0.739656in}}%
\pgfpathlineto{\pgfqpoint{3.803628in}{0.739656in}}%
\pgfpathlineto{\pgfqpoint{3.803331in}{0.739656in}}%
\pgfpathlineto{\pgfqpoint{3.803033in}{0.739656in}}%
\pgfpathlineto{\pgfqpoint{3.802736in}{0.739656in}}%
\pgfpathlineto{\pgfqpoint{3.802439in}{0.739656in}}%
\pgfpathlineto{\pgfqpoint{3.802141in}{0.739656in}}%
\pgfpathlineto{\pgfqpoint{3.801844in}{0.739656in}}%
\pgfpathlineto{\pgfqpoint{3.801546in}{0.739656in}}%
\pgfpathlineto{\pgfqpoint{3.801249in}{0.739656in}}%
\pgfpathlineto{\pgfqpoint{3.800951in}{0.739656in}}%
\pgfpathlineto{\pgfqpoint{3.800654in}{0.739656in}}%
\pgfpathlineto{\pgfqpoint{3.800356in}{0.739656in}}%
\pgfpathlineto{\pgfqpoint{3.800059in}{0.739656in}}%
\pgfpathlineto{\pgfqpoint{3.799761in}{0.739656in}}%
\pgfpathlineto{\pgfqpoint{3.799464in}{0.739656in}}%
\pgfpathlineto{\pgfqpoint{3.799166in}{0.739656in}}%
\pgfpathlineto{\pgfqpoint{3.798869in}{0.739656in}}%
\pgfpathlineto{\pgfqpoint{3.798571in}{0.739656in}}%
\pgfpathlineto{\pgfqpoint{3.798274in}{0.739656in}}%
\pgfpathlineto{\pgfqpoint{3.797976in}{0.739656in}}%
\pgfpathlineto{\pgfqpoint{3.797679in}{0.739656in}}%
\pgfpathlineto{\pgfqpoint{3.797381in}{0.739656in}}%
\pgfpathlineto{\pgfqpoint{3.797084in}{0.739656in}}%
\pgfpathlineto{\pgfqpoint{3.796786in}{0.739656in}}%
\pgfpathlineto{\pgfqpoint{3.796489in}{0.739656in}}%
\pgfpathlineto{\pgfqpoint{3.796191in}{0.739656in}}%
\pgfpathlineto{\pgfqpoint{3.795894in}{0.739656in}}%
\pgfpathlineto{\pgfqpoint{3.795597in}{0.739656in}}%
\pgfpathlineto{\pgfqpoint{3.795299in}{0.739656in}}%
\pgfpathlineto{\pgfqpoint{3.795002in}{0.739656in}}%
\pgfpathlineto{\pgfqpoint{3.794704in}{0.739656in}}%
\pgfpathlineto{\pgfqpoint{3.794407in}{0.739656in}}%
\pgfpathlineto{\pgfqpoint{3.794109in}{0.739656in}}%
\pgfpathlineto{\pgfqpoint{3.793812in}{0.739656in}}%
\pgfpathlineto{\pgfqpoint{3.793514in}{0.739656in}}%
\pgfpathlineto{\pgfqpoint{3.793217in}{0.739656in}}%
\pgfpathlineto{\pgfqpoint{3.792919in}{0.739656in}}%
\pgfpathlineto{\pgfqpoint{3.792622in}{0.739656in}}%
\pgfpathlineto{\pgfqpoint{3.792324in}{0.739656in}}%
\pgfpathlineto{\pgfqpoint{3.792027in}{0.739656in}}%
\pgfpathlineto{\pgfqpoint{3.791729in}{0.739656in}}%
\pgfpathlineto{\pgfqpoint{3.791432in}{0.739656in}}%
\pgfpathlineto{\pgfqpoint{3.791134in}{0.739656in}}%
\pgfpathlineto{\pgfqpoint{3.790837in}{0.739656in}}%
\pgfpathlineto{\pgfqpoint{3.790539in}{0.739656in}}%
\pgfpathlineto{\pgfqpoint{3.790242in}{0.739656in}}%
\pgfpathlineto{\pgfqpoint{3.789944in}{0.739656in}}%
\pgfpathlineto{\pgfqpoint{3.789647in}{0.739656in}}%
\pgfpathlineto{\pgfqpoint{3.789350in}{0.739656in}}%
\pgfpathlineto{\pgfqpoint{3.789052in}{0.739656in}}%
\pgfpathlineto{\pgfqpoint{3.788755in}{0.739656in}}%
\pgfpathlineto{\pgfqpoint{3.788457in}{0.739656in}}%
\pgfpathlineto{\pgfqpoint{3.788160in}{0.739656in}}%
\pgfpathlineto{\pgfqpoint{3.787862in}{0.739656in}}%
\pgfpathlineto{\pgfqpoint{3.787565in}{0.739656in}}%
\pgfpathlineto{\pgfqpoint{3.787267in}{0.739656in}}%
\pgfpathlineto{\pgfqpoint{3.786970in}{0.739656in}}%
\pgfpathlineto{\pgfqpoint{3.786672in}{0.739656in}}%
\pgfpathlineto{\pgfqpoint{3.786375in}{0.739656in}}%
\pgfpathlineto{\pgfqpoint{3.786077in}{0.739656in}}%
\pgfpathlineto{\pgfqpoint{3.785780in}{0.739656in}}%
\pgfpathlineto{\pgfqpoint{3.785482in}{0.739656in}}%
\pgfpathlineto{\pgfqpoint{3.785185in}{0.739656in}}%
\pgfpathlineto{\pgfqpoint{3.784887in}{0.739656in}}%
\pgfpathlineto{\pgfqpoint{3.784590in}{0.739656in}}%
\pgfpathlineto{\pgfqpoint{3.784292in}{0.739656in}}%
\pgfpathlineto{\pgfqpoint{3.783995in}{0.739656in}}%
\pgfpathlineto{\pgfqpoint{3.783697in}{0.739656in}}%
\pgfpathlineto{\pgfqpoint{3.783400in}{0.739656in}}%
\pgfpathlineto{\pgfqpoint{3.783102in}{0.739656in}}%
\pgfpathlineto{\pgfqpoint{3.782805in}{0.739656in}}%
\pgfpathlineto{\pgfqpoint{3.782508in}{0.739656in}}%
\pgfpathlineto{\pgfqpoint{3.782210in}{0.739656in}}%
\pgfpathlineto{\pgfqpoint{3.781913in}{0.739656in}}%
\pgfpathlineto{\pgfqpoint{3.781615in}{0.739656in}}%
\pgfpathlineto{\pgfqpoint{3.781318in}{0.739656in}}%
\pgfpathlineto{\pgfqpoint{3.781020in}{0.739656in}}%
\pgfpathlineto{\pgfqpoint{3.780723in}{0.739656in}}%
\pgfpathlineto{\pgfqpoint{3.780425in}{0.739656in}}%
\pgfpathlineto{\pgfqpoint{3.780128in}{0.739656in}}%
\pgfpathlineto{\pgfqpoint{3.779830in}{0.739656in}}%
\pgfpathlineto{\pgfqpoint{3.779533in}{0.739656in}}%
\pgfpathlineto{\pgfqpoint{3.779235in}{0.739656in}}%
\pgfpathlineto{\pgfqpoint{3.778938in}{0.739656in}}%
\pgfpathlineto{\pgfqpoint{3.778640in}{0.739656in}}%
\pgfpathlineto{\pgfqpoint{3.778343in}{0.739656in}}%
\pgfpathlineto{\pgfqpoint{3.778045in}{0.739656in}}%
\pgfpathlineto{\pgfqpoint{3.777748in}{0.739656in}}%
\pgfpathlineto{\pgfqpoint{3.777450in}{0.739656in}}%
\pgfpathlineto{\pgfqpoint{3.777153in}{0.739656in}}%
\pgfpathlineto{\pgfqpoint{3.776855in}{0.739656in}}%
\pgfpathlineto{\pgfqpoint{3.776558in}{0.739656in}}%
\pgfpathlineto{\pgfqpoint{3.776260in}{0.739656in}}%
\pgfpathlineto{\pgfqpoint{3.775963in}{0.739656in}}%
\pgfpathlineto{\pgfqpoint{3.775666in}{0.739656in}}%
\pgfpathlineto{\pgfqpoint{3.775368in}{0.739656in}}%
\pgfpathlineto{\pgfqpoint{3.775071in}{0.739656in}}%
\pgfpathlineto{\pgfqpoint{3.774773in}{0.739656in}}%
\pgfpathlineto{\pgfqpoint{3.774476in}{0.739656in}}%
\pgfpathlineto{\pgfqpoint{3.774178in}{0.739656in}}%
\pgfpathlineto{\pgfqpoint{3.773881in}{0.739656in}}%
\pgfpathlineto{\pgfqpoint{3.773583in}{0.739656in}}%
\pgfpathlineto{\pgfqpoint{3.773286in}{0.739656in}}%
\pgfpathlineto{\pgfqpoint{3.772988in}{0.739656in}}%
\pgfpathlineto{\pgfqpoint{3.772691in}{0.739656in}}%
\pgfpathlineto{\pgfqpoint{3.772393in}{0.739656in}}%
\pgfpathlineto{\pgfqpoint{3.772096in}{0.739656in}}%
\pgfpathlineto{\pgfqpoint{3.771798in}{0.739656in}}%
\pgfpathlineto{\pgfqpoint{3.771501in}{0.739656in}}%
\pgfpathlineto{\pgfqpoint{3.771203in}{0.739656in}}%
\pgfpathlineto{\pgfqpoint{3.770906in}{0.739656in}}%
\pgfpathlineto{\pgfqpoint{3.770608in}{0.739656in}}%
\pgfpathlineto{\pgfqpoint{3.770311in}{0.739656in}}%
\pgfpathlineto{\pgfqpoint{3.770013in}{0.739656in}}%
\pgfpathlineto{\pgfqpoint{3.769716in}{0.739656in}}%
\pgfpathlineto{\pgfqpoint{3.769419in}{0.739656in}}%
\pgfpathlineto{\pgfqpoint{3.769121in}{0.739656in}}%
\pgfpathlineto{\pgfqpoint{3.768824in}{0.739656in}}%
\pgfpathlineto{\pgfqpoint{3.768526in}{0.739656in}}%
\pgfpathlineto{\pgfqpoint{3.768229in}{0.739656in}}%
\pgfpathlineto{\pgfqpoint{3.767931in}{0.739656in}}%
\pgfpathlineto{\pgfqpoint{3.767634in}{0.739656in}}%
\pgfpathlineto{\pgfqpoint{3.767336in}{0.739656in}}%
\pgfpathlineto{\pgfqpoint{3.767039in}{0.739656in}}%
\pgfpathlineto{\pgfqpoint{3.766741in}{0.739656in}}%
\pgfpathlineto{\pgfqpoint{3.766444in}{0.739656in}}%
\pgfpathlineto{\pgfqpoint{3.766146in}{0.739656in}}%
\pgfpathlineto{\pgfqpoint{3.765849in}{0.739656in}}%
\pgfpathlineto{\pgfqpoint{3.765551in}{0.739656in}}%
\pgfpathlineto{\pgfqpoint{3.765254in}{0.739656in}}%
\pgfpathlineto{\pgfqpoint{3.764956in}{0.739656in}}%
\pgfpathlineto{\pgfqpoint{3.764659in}{0.739656in}}%
\pgfpathlineto{\pgfqpoint{3.764361in}{0.739656in}}%
\pgfpathlineto{\pgfqpoint{3.764064in}{0.739656in}}%
\pgfpathlineto{\pgfqpoint{3.763766in}{0.739656in}}%
\pgfpathlineto{\pgfqpoint{3.763469in}{0.739656in}}%
\pgfpathlineto{\pgfqpoint{3.763171in}{0.739656in}}%
\pgfpathlineto{\pgfqpoint{3.762874in}{0.739656in}}%
\pgfpathlineto{\pgfqpoint{3.762577in}{0.739656in}}%
\pgfpathlineto{\pgfqpoint{3.762279in}{0.739656in}}%
\pgfpathlineto{\pgfqpoint{3.761982in}{0.739656in}}%
\pgfpathlineto{\pgfqpoint{3.761684in}{0.739656in}}%
\pgfpathlineto{\pgfqpoint{3.761387in}{0.739656in}}%
\pgfpathlineto{\pgfqpoint{3.761089in}{0.739656in}}%
\pgfpathlineto{\pgfqpoint{3.760792in}{0.739656in}}%
\pgfpathlineto{\pgfqpoint{3.760494in}{0.739656in}}%
\pgfpathlineto{\pgfqpoint{3.760197in}{0.739656in}}%
\pgfpathlineto{\pgfqpoint{3.759899in}{0.739656in}}%
\pgfpathlineto{\pgfqpoint{3.759602in}{0.739656in}}%
\pgfpathlineto{\pgfqpoint{3.759304in}{0.739656in}}%
\pgfpathlineto{\pgfqpoint{3.759007in}{0.739656in}}%
\pgfpathlineto{\pgfqpoint{3.758709in}{0.739656in}}%
\pgfpathlineto{\pgfqpoint{3.758412in}{0.739656in}}%
\pgfpathlineto{\pgfqpoint{3.758114in}{0.739656in}}%
\pgfpathlineto{\pgfqpoint{3.757817in}{0.739656in}}%
\pgfpathlineto{\pgfqpoint{3.757519in}{0.739656in}}%
\pgfpathlineto{\pgfqpoint{3.757222in}{0.739656in}}%
\pgfpathlineto{\pgfqpoint{3.756924in}{0.739656in}}%
\pgfpathlineto{\pgfqpoint{3.756627in}{0.739656in}}%
\pgfpathlineto{\pgfqpoint{3.756329in}{0.739656in}}%
\pgfpathlineto{\pgfqpoint{3.756032in}{0.739656in}}%
\pgfpathlineto{\pgfqpoint{3.755735in}{0.739656in}}%
\pgfpathlineto{\pgfqpoint{3.755437in}{0.739656in}}%
\pgfpathlineto{\pgfqpoint{3.755140in}{0.739656in}}%
\pgfpathlineto{\pgfqpoint{3.754842in}{0.739656in}}%
\pgfpathlineto{\pgfqpoint{3.754545in}{0.739656in}}%
\pgfpathlineto{\pgfqpoint{3.754247in}{0.739656in}}%
\pgfpathlineto{\pgfqpoint{3.753950in}{0.739656in}}%
\pgfpathlineto{\pgfqpoint{3.753652in}{0.739656in}}%
\pgfpathlineto{\pgfqpoint{3.753355in}{0.739656in}}%
\pgfpathlineto{\pgfqpoint{3.753057in}{0.739656in}}%
\pgfpathlineto{\pgfqpoint{3.752760in}{0.739656in}}%
\pgfpathlineto{\pgfqpoint{3.752462in}{0.739656in}}%
\pgfpathlineto{\pgfqpoint{3.752165in}{0.739656in}}%
\pgfpathlineto{\pgfqpoint{3.751867in}{0.739656in}}%
\pgfpathlineto{\pgfqpoint{3.751570in}{0.739656in}}%
\pgfpathlineto{\pgfqpoint{3.751272in}{0.739656in}}%
\pgfpathlineto{\pgfqpoint{3.750975in}{0.739656in}}%
\pgfpathlineto{\pgfqpoint{3.750677in}{0.739656in}}%
\pgfpathlineto{\pgfqpoint{3.750380in}{0.739656in}}%
\pgfpathlineto{\pgfqpoint{3.750082in}{0.739656in}}%
\pgfpathlineto{\pgfqpoint{3.749785in}{0.739656in}}%
\pgfpathlineto{\pgfqpoint{3.749488in}{0.739656in}}%
\pgfpathlineto{\pgfqpoint{3.749190in}{0.739656in}}%
\pgfpathlineto{\pgfqpoint{3.748893in}{0.739656in}}%
\pgfpathlineto{\pgfqpoint{3.748595in}{0.739656in}}%
\pgfpathlineto{\pgfqpoint{3.748298in}{0.739656in}}%
\pgfpathlineto{\pgfqpoint{3.748000in}{0.739656in}}%
\pgfpathlineto{\pgfqpoint{3.747703in}{0.739656in}}%
\pgfpathlineto{\pgfqpoint{3.747405in}{0.739656in}}%
\pgfpathlineto{\pgfqpoint{3.747108in}{0.739656in}}%
\pgfpathlineto{\pgfqpoint{3.746810in}{0.739656in}}%
\pgfpathlineto{\pgfqpoint{3.746513in}{0.739656in}}%
\pgfpathlineto{\pgfqpoint{3.746215in}{0.739656in}}%
\pgfpathlineto{\pgfqpoint{3.745918in}{0.739656in}}%
\pgfpathlineto{\pgfqpoint{3.745620in}{0.739656in}}%
\pgfpathlineto{\pgfqpoint{3.745323in}{0.739656in}}%
\pgfpathlineto{\pgfqpoint{3.745025in}{0.739656in}}%
\pgfpathlineto{\pgfqpoint{3.744728in}{0.739656in}}%
\pgfpathlineto{\pgfqpoint{3.744430in}{0.739656in}}%
\pgfpathlineto{\pgfqpoint{3.744133in}{0.739656in}}%
\pgfpathlineto{\pgfqpoint{3.743835in}{0.739656in}}%
\pgfpathlineto{\pgfqpoint{3.743538in}{0.739656in}}%
\pgfpathlineto{\pgfqpoint{3.743240in}{0.739656in}}%
\pgfpathlineto{\pgfqpoint{3.742943in}{0.739656in}}%
\pgfpathlineto{\pgfqpoint{3.742646in}{0.739656in}}%
\pgfpathlineto{\pgfqpoint{3.742348in}{0.739656in}}%
\pgfpathlineto{\pgfqpoint{3.742051in}{0.739656in}}%
\pgfpathlineto{\pgfqpoint{3.741753in}{0.739656in}}%
\pgfpathlineto{\pgfqpoint{3.741456in}{0.739656in}}%
\pgfpathlineto{\pgfqpoint{3.741158in}{0.739656in}}%
\pgfpathlineto{\pgfqpoint{3.740861in}{0.739656in}}%
\pgfpathlineto{\pgfqpoint{3.740563in}{0.739656in}}%
\pgfpathlineto{\pgfqpoint{3.740266in}{0.739656in}}%
\pgfpathlineto{\pgfqpoint{3.739968in}{0.739656in}}%
\pgfpathlineto{\pgfqpoint{3.739671in}{0.739656in}}%
\pgfpathlineto{\pgfqpoint{3.739373in}{0.739656in}}%
\pgfpathlineto{\pgfqpoint{3.739076in}{0.739656in}}%
\pgfpathlineto{\pgfqpoint{3.738778in}{0.739656in}}%
\pgfpathlineto{\pgfqpoint{3.738481in}{0.739656in}}%
\pgfpathlineto{\pgfqpoint{3.738183in}{0.739656in}}%
\pgfpathlineto{\pgfqpoint{3.737886in}{0.739656in}}%
\pgfpathlineto{\pgfqpoint{3.737588in}{0.739656in}}%
\pgfpathlineto{\pgfqpoint{3.737291in}{0.739656in}}%
\pgfpathlineto{\pgfqpoint{3.736993in}{0.739656in}}%
\pgfpathlineto{\pgfqpoint{3.736696in}{0.739656in}}%
\pgfpathlineto{\pgfqpoint{3.736398in}{0.739656in}}%
\pgfpathlineto{\pgfqpoint{3.736101in}{0.739656in}}%
\pgfpathlineto{\pgfqpoint{3.735804in}{0.739656in}}%
\pgfpathlineto{\pgfqpoint{3.735506in}{0.739656in}}%
\pgfpathlineto{\pgfqpoint{3.735209in}{0.739656in}}%
\pgfpathlineto{\pgfqpoint{3.734911in}{0.739656in}}%
\pgfpathlineto{\pgfqpoint{3.734614in}{0.739656in}}%
\pgfpathlineto{\pgfqpoint{3.734316in}{0.739656in}}%
\pgfpathlineto{\pgfqpoint{3.734019in}{0.739656in}}%
\pgfpathlineto{\pgfqpoint{3.733721in}{0.739656in}}%
\pgfpathlineto{\pgfqpoint{3.733424in}{0.739656in}}%
\pgfpathlineto{\pgfqpoint{3.733126in}{0.739656in}}%
\pgfpathlineto{\pgfqpoint{3.732829in}{0.739656in}}%
\pgfpathlineto{\pgfqpoint{3.732531in}{0.739656in}}%
\pgfpathlineto{\pgfqpoint{3.732234in}{0.739656in}}%
\pgfpathlineto{\pgfqpoint{3.731936in}{0.739656in}}%
\pgfpathlineto{\pgfqpoint{3.731639in}{0.739656in}}%
\pgfpathlineto{\pgfqpoint{3.731341in}{0.739656in}}%
\pgfpathlineto{\pgfqpoint{3.731044in}{0.739656in}}%
\pgfpathlineto{\pgfqpoint{3.730746in}{0.739656in}}%
\pgfpathlineto{\pgfqpoint{3.730449in}{0.739656in}}%
\pgfpathlineto{\pgfqpoint{3.730151in}{0.739656in}}%
\pgfpathlineto{\pgfqpoint{3.729854in}{0.739656in}}%
\pgfpathlineto{\pgfqpoint{3.729556in}{0.739656in}}%
\pgfpathlineto{\pgfqpoint{3.729259in}{0.739656in}}%
\pgfpathlineto{\pgfqpoint{3.728962in}{0.739656in}}%
\pgfpathlineto{\pgfqpoint{3.728664in}{0.739656in}}%
\pgfpathlineto{\pgfqpoint{3.728367in}{0.739656in}}%
\pgfpathlineto{\pgfqpoint{3.728069in}{0.739656in}}%
\pgfpathlineto{\pgfqpoint{3.727772in}{0.739656in}}%
\pgfpathlineto{\pgfqpoint{3.727474in}{0.739656in}}%
\pgfpathlineto{\pgfqpoint{3.727177in}{0.739656in}}%
\pgfpathlineto{\pgfqpoint{3.726879in}{0.739656in}}%
\pgfpathlineto{\pgfqpoint{3.726582in}{0.739656in}}%
\pgfpathlineto{\pgfqpoint{3.726284in}{0.739656in}}%
\pgfpathlineto{\pgfqpoint{3.725987in}{0.739656in}}%
\pgfpathlineto{\pgfqpoint{3.725689in}{0.739656in}}%
\pgfpathlineto{\pgfqpoint{3.725392in}{0.739656in}}%
\pgfpathlineto{\pgfqpoint{3.725094in}{0.739656in}}%
\pgfpathlineto{\pgfqpoint{3.724797in}{0.739656in}}%
\pgfpathlineto{\pgfqpoint{3.724499in}{0.739656in}}%
\pgfpathlineto{\pgfqpoint{3.724202in}{0.739656in}}%
\pgfpathlineto{\pgfqpoint{3.723904in}{0.739656in}}%
\pgfpathlineto{\pgfqpoint{3.723607in}{0.739656in}}%
\pgfpathlineto{\pgfqpoint{3.723309in}{0.739656in}}%
\pgfpathlineto{\pgfqpoint{3.723012in}{0.739656in}}%
\pgfpathlineto{\pgfqpoint{3.722715in}{0.739656in}}%
\pgfpathlineto{\pgfqpoint{3.722417in}{0.739656in}}%
\pgfpathlineto{\pgfqpoint{3.722120in}{0.739656in}}%
\pgfpathlineto{\pgfqpoint{3.721822in}{0.739656in}}%
\pgfpathlineto{\pgfqpoint{3.721525in}{0.739656in}}%
\pgfpathlineto{\pgfqpoint{3.721227in}{0.739656in}}%
\pgfpathlineto{\pgfqpoint{3.720930in}{0.739656in}}%
\pgfpathlineto{\pgfqpoint{3.720632in}{0.739656in}}%
\pgfpathlineto{\pgfqpoint{3.720335in}{0.739656in}}%
\pgfpathlineto{\pgfqpoint{3.720037in}{0.739656in}}%
\pgfpathlineto{\pgfqpoint{3.719740in}{0.739656in}}%
\pgfpathlineto{\pgfqpoint{3.719442in}{0.739656in}}%
\pgfpathlineto{\pgfqpoint{3.719145in}{0.739656in}}%
\pgfpathlineto{\pgfqpoint{3.718847in}{0.739656in}}%
\pgfpathlineto{\pgfqpoint{3.718550in}{0.739656in}}%
\pgfpathlineto{\pgfqpoint{3.718252in}{0.739656in}}%
\pgfpathlineto{\pgfqpoint{3.717955in}{0.739656in}}%
\pgfpathlineto{\pgfqpoint{3.717657in}{0.739656in}}%
\pgfpathlineto{\pgfqpoint{3.717360in}{0.739656in}}%
\pgfpathlineto{\pgfqpoint{3.717062in}{0.739656in}}%
\pgfpathlineto{\pgfqpoint{3.716765in}{0.739656in}}%
\pgfpathlineto{\pgfqpoint{3.716467in}{0.739656in}}%
\pgfpathlineto{\pgfqpoint{3.716170in}{0.739656in}}%
\pgfpathlineto{\pgfqpoint{3.715873in}{0.739656in}}%
\pgfpathlineto{\pgfqpoint{3.715575in}{0.739656in}}%
\pgfpathlineto{\pgfqpoint{3.715278in}{0.739656in}}%
\pgfpathlineto{\pgfqpoint{3.714980in}{0.739656in}}%
\pgfpathlineto{\pgfqpoint{3.714683in}{0.739656in}}%
\pgfpathlineto{\pgfqpoint{3.714385in}{0.739656in}}%
\pgfpathlineto{\pgfqpoint{3.714088in}{0.739656in}}%
\pgfpathlineto{\pgfqpoint{3.713790in}{0.739656in}}%
\pgfpathlineto{\pgfqpoint{3.713493in}{0.739656in}}%
\pgfpathlineto{\pgfqpoint{3.713195in}{0.739656in}}%
\pgfpathlineto{\pgfqpoint{3.712898in}{0.739656in}}%
\pgfpathlineto{\pgfqpoint{3.712600in}{0.739656in}}%
\pgfpathlineto{\pgfqpoint{3.712303in}{0.739656in}}%
\pgfpathlineto{\pgfqpoint{3.712005in}{0.739656in}}%
\pgfpathlineto{\pgfqpoint{3.711708in}{0.739656in}}%
\pgfpathlineto{\pgfqpoint{3.711410in}{0.739656in}}%
\pgfpathlineto{\pgfqpoint{3.711113in}{0.739656in}}%
\pgfpathlineto{\pgfqpoint{3.710815in}{0.739656in}}%
\pgfpathlineto{\pgfqpoint{3.710518in}{0.739656in}}%
\pgfpathlineto{\pgfqpoint{3.710220in}{0.739656in}}%
\pgfpathlineto{\pgfqpoint{3.709923in}{0.739656in}}%
\pgfpathlineto{\pgfqpoint{3.709625in}{0.739656in}}%
\pgfpathlineto{\pgfqpoint{3.709328in}{0.739656in}}%
\pgfpathlineto{\pgfqpoint{3.709031in}{0.739656in}}%
\pgfpathlineto{\pgfqpoint{3.708733in}{0.739656in}}%
\pgfpathlineto{\pgfqpoint{3.708436in}{0.739656in}}%
\pgfpathlineto{\pgfqpoint{3.708138in}{0.739656in}}%
\pgfpathlineto{\pgfqpoint{3.707841in}{0.739656in}}%
\pgfpathlineto{\pgfqpoint{3.707543in}{0.739656in}}%
\pgfpathlineto{\pgfqpoint{3.707246in}{0.739656in}}%
\pgfpathlineto{\pgfqpoint{3.706948in}{0.739656in}}%
\pgfpathlineto{\pgfqpoint{3.706651in}{0.739656in}}%
\pgfpathlineto{\pgfqpoint{3.706353in}{0.739656in}}%
\pgfpathlineto{\pgfqpoint{3.706056in}{0.739656in}}%
\pgfpathlineto{\pgfqpoint{3.705758in}{0.739656in}}%
\pgfpathlineto{\pgfqpoint{3.705461in}{0.739656in}}%
\pgfpathlineto{\pgfqpoint{3.705163in}{0.739656in}}%
\pgfpathlineto{\pgfqpoint{3.704866in}{0.739656in}}%
\pgfpathlineto{\pgfqpoint{3.704568in}{0.739656in}}%
\pgfpathlineto{\pgfqpoint{3.704271in}{0.739656in}}%
\pgfpathlineto{\pgfqpoint{3.703973in}{0.739656in}}%
\pgfpathlineto{\pgfqpoint{3.703676in}{0.739656in}}%
\pgfpathlineto{\pgfqpoint{3.703378in}{0.739656in}}%
\pgfpathlineto{\pgfqpoint{3.703081in}{0.739656in}}%
\pgfpathlineto{\pgfqpoint{3.702784in}{0.739656in}}%
\pgfpathlineto{\pgfqpoint{3.702486in}{0.739656in}}%
\pgfpathlineto{\pgfqpoint{3.702189in}{0.739656in}}%
\pgfpathlineto{\pgfqpoint{3.701891in}{0.739656in}}%
\pgfpathlineto{\pgfqpoint{3.701594in}{0.739656in}}%
\pgfpathlineto{\pgfqpoint{3.701296in}{0.739656in}}%
\pgfpathlineto{\pgfqpoint{3.700999in}{0.739656in}}%
\pgfpathlineto{\pgfqpoint{3.700701in}{0.739656in}}%
\pgfpathlineto{\pgfqpoint{3.700404in}{0.739656in}}%
\pgfpathlineto{\pgfqpoint{3.700106in}{0.739656in}}%
\pgfpathlineto{\pgfqpoint{3.699809in}{0.739656in}}%
\pgfpathlineto{\pgfqpoint{3.699511in}{0.739656in}}%
\pgfpathlineto{\pgfqpoint{3.699214in}{0.739656in}}%
\pgfpathlineto{\pgfqpoint{3.698916in}{0.739656in}}%
\pgfpathlineto{\pgfqpoint{3.698619in}{0.739656in}}%
\pgfpathlineto{\pgfqpoint{3.698321in}{0.739656in}}%
\pgfpathlineto{\pgfqpoint{3.698024in}{0.739656in}}%
\pgfpathlineto{\pgfqpoint{3.697726in}{0.739656in}}%
\pgfpathlineto{\pgfqpoint{3.697429in}{0.739656in}}%
\pgfpathlineto{\pgfqpoint{3.697131in}{0.739656in}}%
\pgfpathlineto{\pgfqpoint{3.696834in}{0.739656in}}%
\pgfpathlineto{\pgfqpoint{3.696536in}{0.739656in}}%
\pgfpathlineto{\pgfqpoint{3.696239in}{0.739656in}}%
\pgfpathlineto{\pgfqpoint{3.695942in}{0.739656in}}%
\pgfpathlineto{\pgfqpoint{3.695644in}{0.739656in}}%
\pgfpathlineto{\pgfqpoint{3.695347in}{0.739656in}}%
\pgfpathlineto{\pgfqpoint{3.695049in}{0.739656in}}%
\pgfpathlineto{\pgfqpoint{3.694752in}{0.739656in}}%
\pgfpathlineto{\pgfqpoint{3.694454in}{0.739656in}}%
\pgfpathlineto{\pgfqpoint{3.694157in}{0.739656in}}%
\pgfpathlineto{\pgfqpoint{3.693859in}{0.739656in}}%
\pgfpathlineto{\pgfqpoint{3.693562in}{0.739656in}}%
\pgfpathlineto{\pgfqpoint{3.693264in}{0.739656in}}%
\pgfpathlineto{\pgfqpoint{3.692967in}{0.739656in}}%
\pgfpathlineto{\pgfqpoint{3.692669in}{0.739656in}}%
\pgfpathlineto{\pgfqpoint{3.692372in}{0.739656in}}%
\pgfpathlineto{\pgfqpoint{3.692074in}{0.739656in}}%
\pgfpathlineto{\pgfqpoint{3.691777in}{0.739656in}}%
\pgfpathlineto{\pgfqpoint{3.691479in}{0.739656in}}%
\pgfpathlineto{\pgfqpoint{3.691182in}{0.739656in}}%
\pgfpathlineto{\pgfqpoint{3.690884in}{0.739656in}}%
\pgfpathlineto{\pgfqpoint{3.690587in}{0.739656in}}%
\pgfpathlineto{\pgfqpoint{3.690289in}{0.739656in}}%
\pgfpathlineto{\pgfqpoint{3.689992in}{0.739656in}}%
\pgfpathlineto{\pgfqpoint{3.689694in}{0.739656in}}%
\pgfpathlineto{\pgfqpoint{3.689397in}{0.739656in}}%
\pgfpathlineto{\pgfqpoint{3.689100in}{0.739656in}}%
\pgfpathlineto{\pgfqpoint{3.688802in}{0.739656in}}%
\pgfpathlineto{\pgfqpoint{3.688505in}{0.739656in}}%
\pgfpathlineto{\pgfqpoint{3.688207in}{0.739656in}}%
\pgfpathlineto{\pgfqpoint{3.687910in}{0.739656in}}%
\pgfpathlineto{\pgfqpoint{3.687612in}{0.739656in}}%
\pgfpathlineto{\pgfqpoint{3.687315in}{0.739656in}}%
\pgfpathlineto{\pgfqpoint{3.687017in}{0.739656in}}%
\pgfpathlineto{\pgfqpoint{3.686720in}{0.739656in}}%
\pgfpathlineto{\pgfqpoint{3.686422in}{0.739656in}}%
\pgfpathlineto{\pgfqpoint{3.686125in}{0.739656in}}%
\pgfpathlineto{\pgfqpoint{3.685827in}{0.739656in}}%
\pgfpathlineto{\pgfqpoint{3.685530in}{0.739656in}}%
\pgfpathlineto{\pgfqpoint{3.685232in}{0.739656in}}%
\pgfpathlineto{\pgfqpoint{3.684935in}{0.739656in}}%
\pgfpathlineto{\pgfqpoint{3.684637in}{0.739656in}}%
\pgfpathlineto{\pgfqpoint{3.684340in}{0.739656in}}%
\pgfpathlineto{\pgfqpoint{3.684042in}{0.739656in}}%
\pgfpathlineto{\pgfqpoint{3.683745in}{0.739656in}}%
\pgfpathlineto{\pgfqpoint{3.683447in}{0.739656in}}%
\pgfpathlineto{\pgfqpoint{3.683150in}{0.739656in}}%
\pgfpathlineto{\pgfqpoint{3.682853in}{0.739656in}}%
\pgfpathlineto{\pgfqpoint{3.682555in}{0.739656in}}%
\pgfpathlineto{\pgfqpoint{3.682258in}{0.739656in}}%
\pgfpathlineto{\pgfqpoint{3.681960in}{0.739656in}}%
\pgfpathlineto{\pgfqpoint{3.681663in}{0.739656in}}%
\pgfpathlineto{\pgfqpoint{3.681365in}{0.739656in}}%
\pgfpathlineto{\pgfqpoint{3.681068in}{0.739656in}}%
\pgfpathlineto{\pgfqpoint{3.680770in}{0.739656in}}%
\pgfpathlineto{\pgfqpoint{3.680473in}{0.739656in}}%
\pgfpathlineto{\pgfqpoint{3.680175in}{0.739656in}}%
\pgfpathlineto{\pgfqpoint{3.679878in}{0.739656in}}%
\pgfpathlineto{\pgfqpoint{3.679580in}{0.739656in}}%
\pgfpathlineto{\pgfqpoint{3.679283in}{0.739656in}}%
\pgfpathlineto{\pgfqpoint{3.678985in}{0.739656in}}%
\pgfpathlineto{\pgfqpoint{3.678688in}{0.739656in}}%
\pgfpathlineto{\pgfqpoint{3.678390in}{0.739656in}}%
\pgfpathlineto{\pgfqpoint{3.678093in}{0.739656in}}%
\pgfpathlineto{\pgfqpoint{3.677795in}{0.739656in}}%
\pgfpathlineto{\pgfqpoint{3.677498in}{0.739656in}}%
\pgfpathlineto{\pgfqpoint{3.677200in}{0.739656in}}%
\pgfpathlineto{\pgfqpoint{3.676903in}{0.739656in}}%
\pgfpathlineto{\pgfqpoint{3.676605in}{0.739656in}}%
\pgfpathlineto{\pgfqpoint{3.676308in}{0.739656in}}%
\pgfpathlineto{\pgfqpoint{3.676011in}{0.739656in}}%
\pgfpathlineto{\pgfqpoint{3.675713in}{0.739656in}}%
\pgfpathlineto{\pgfqpoint{3.675416in}{0.739656in}}%
\pgfpathlineto{\pgfqpoint{3.675118in}{0.739656in}}%
\pgfpathlineto{\pgfqpoint{3.674821in}{0.739656in}}%
\pgfpathlineto{\pgfqpoint{3.674523in}{0.739656in}}%
\pgfpathlineto{\pgfqpoint{3.674226in}{0.739656in}}%
\pgfpathlineto{\pgfqpoint{3.673928in}{0.739656in}}%
\pgfpathlineto{\pgfqpoint{3.673631in}{0.739656in}}%
\pgfpathlineto{\pgfqpoint{3.673333in}{0.739656in}}%
\pgfpathlineto{\pgfqpoint{3.673036in}{0.739656in}}%
\pgfpathlineto{\pgfqpoint{3.672738in}{0.739656in}}%
\pgfpathlineto{\pgfqpoint{3.672441in}{0.739656in}}%
\pgfpathlineto{\pgfqpoint{3.672143in}{0.739656in}}%
\pgfpathlineto{\pgfqpoint{3.671846in}{0.739656in}}%
\pgfpathlineto{\pgfqpoint{3.671548in}{0.739656in}}%
\pgfpathlineto{\pgfqpoint{3.671251in}{0.739656in}}%
\pgfpathlineto{\pgfqpoint{3.670953in}{0.739656in}}%
\pgfpathlineto{\pgfqpoint{3.670656in}{0.739656in}}%
\pgfpathlineto{\pgfqpoint{3.670358in}{0.739656in}}%
\pgfpathlineto{\pgfqpoint{3.670061in}{0.739656in}}%
\pgfpathlineto{\pgfqpoint{3.669763in}{0.739656in}}%
\pgfpathlineto{\pgfqpoint{3.669466in}{0.739656in}}%
\pgfpathlineto{\pgfqpoint{3.669169in}{0.739656in}}%
\pgfpathlineto{\pgfqpoint{3.668871in}{0.739656in}}%
\pgfpathlineto{\pgfqpoint{3.668574in}{0.739656in}}%
\pgfpathlineto{\pgfqpoint{3.668276in}{0.739656in}}%
\pgfpathlineto{\pgfqpoint{3.667979in}{0.739656in}}%
\pgfpathlineto{\pgfqpoint{3.667681in}{0.739656in}}%
\pgfpathlineto{\pgfqpoint{3.667384in}{0.739656in}}%
\pgfpathlineto{\pgfqpoint{3.667086in}{0.739656in}}%
\pgfpathlineto{\pgfqpoint{3.666789in}{0.739656in}}%
\pgfpathlineto{\pgfqpoint{3.666491in}{0.739656in}}%
\pgfpathlineto{\pgfqpoint{3.666194in}{0.739656in}}%
\pgfpathlineto{\pgfqpoint{3.665896in}{0.739656in}}%
\pgfpathlineto{\pgfqpoint{3.665599in}{0.739656in}}%
\pgfpathlineto{\pgfqpoint{3.665301in}{0.739656in}}%
\pgfpathlineto{\pgfqpoint{3.665004in}{0.739656in}}%
\pgfpathlineto{\pgfqpoint{3.664706in}{0.739656in}}%
\pgfpathlineto{\pgfqpoint{3.664409in}{0.739656in}}%
\pgfpathlineto{\pgfqpoint{3.664111in}{0.739656in}}%
\pgfpathlineto{\pgfqpoint{3.663814in}{0.739656in}}%
\pgfpathlineto{\pgfqpoint{3.663516in}{0.739656in}}%
\pgfpathlineto{\pgfqpoint{3.663219in}{0.739656in}}%
\pgfpathlineto{\pgfqpoint{3.662922in}{0.739656in}}%
\pgfpathlineto{\pgfqpoint{3.662624in}{0.739656in}}%
\pgfpathlineto{\pgfqpoint{3.662327in}{0.739656in}}%
\pgfpathlineto{\pgfqpoint{3.662029in}{0.739656in}}%
\pgfpathlineto{\pgfqpoint{3.661732in}{0.739656in}}%
\pgfpathlineto{\pgfqpoint{3.661434in}{0.739656in}}%
\pgfpathlineto{\pgfqpoint{3.661137in}{0.739656in}}%
\pgfpathlineto{\pgfqpoint{3.660839in}{0.739656in}}%
\pgfpathlineto{\pgfqpoint{3.660542in}{0.739656in}}%
\pgfpathlineto{\pgfqpoint{3.660244in}{0.739656in}}%
\pgfpathlineto{\pgfqpoint{3.659947in}{0.739656in}}%
\pgfpathlineto{\pgfqpoint{3.659649in}{0.739656in}}%
\pgfpathlineto{\pgfqpoint{3.659352in}{0.739656in}}%
\pgfpathlineto{\pgfqpoint{3.659054in}{0.739656in}}%
\pgfpathlineto{\pgfqpoint{3.658757in}{0.739656in}}%
\pgfpathlineto{\pgfqpoint{3.658459in}{0.739656in}}%
\pgfpathlineto{\pgfqpoint{3.658162in}{0.739656in}}%
\pgfpathlineto{\pgfqpoint{3.657864in}{0.739656in}}%
\pgfpathlineto{\pgfqpoint{3.657567in}{0.739656in}}%
\pgfpathlineto{\pgfqpoint{3.657269in}{0.739656in}}%
\pgfpathlineto{\pgfqpoint{3.656972in}{0.739656in}}%
\pgfpathlineto{\pgfqpoint{3.656674in}{0.739656in}}%
\pgfpathlineto{\pgfqpoint{3.656377in}{0.739656in}}%
\pgfpathlineto{\pgfqpoint{3.656080in}{0.739656in}}%
\pgfpathlineto{\pgfqpoint{3.655782in}{0.739656in}}%
\pgfpathlineto{\pgfqpoint{3.655485in}{0.739656in}}%
\pgfpathlineto{\pgfqpoint{3.655187in}{0.739656in}}%
\pgfpathlineto{\pgfqpoint{3.654890in}{0.739656in}}%
\pgfpathlineto{\pgfqpoint{3.654592in}{0.739656in}}%
\pgfpathlineto{\pgfqpoint{3.654295in}{0.739656in}}%
\pgfpathlineto{\pgfqpoint{3.653997in}{0.739656in}}%
\pgfpathlineto{\pgfqpoint{3.653700in}{0.739656in}}%
\pgfpathlineto{\pgfqpoint{3.653402in}{0.739656in}}%
\pgfpathlineto{\pgfqpoint{3.653105in}{0.739656in}}%
\pgfpathlineto{\pgfqpoint{3.652807in}{0.739656in}}%
\pgfpathlineto{\pgfqpoint{3.652510in}{0.739656in}}%
\pgfpathlineto{\pgfqpoint{3.652212in}{0.739656in}}%
\pgfpathlineto{\pgfqpoint{3.651915in}{0.739656in}}%
\pgfpathlineto{\pgfqpoint{3.651617in}{0.739656in}}%
\pgfpathlineto{\pgfqpoint{3.651320in}{0.739656in}}%
\pgfpathlineto{\pgfqpoint{3.651022in}{0.739656in}}%
\pgfpathlineto{\pgfqpoint{3.650725in}{0.739656in}}%
\pgfpathlineto{\pgfqpoint{3.650427in}{0.739656in}}%
\pgfpathlineto{\pgfqpoint{3.650130in}{0.739656in}}%
\pgfpathlineto{\pgfqpoint{3.649832in}{0.739656in}}%
\pgfpathlineto{\pgfqpoint{3.649535in}{0.739656in}}%
\pgfpathlineto{\pgfqpoint{3.649238in}{0.739656in}}%
\pgfpathlineto{\pgfqpoint{3.648940in}{0.739656in}}%
\pgfpathlineto{\pgfqpoint{3.648643in}{0.739656in}}%
\pgfpathlineto{\pgfqpoint{3.648345in}{0.739656in}}%
\pgfpathlineto{\pgfqpoint{3.648048in}{0.739656in}}%
\pgfpathlineto{\pgfqpoint{3.647750in}{0.739656in}}%
\pgfpathlineto{\pgfqpoint{3.647453in}{0.739656in}}%
\pgfpathlineto{\pgfqpoint{3.647155in}{0.739656in}}%
\pgfpathlineto{\pgfqpoint{3.646858in}{0.739656in}}%
\pgfpathlineto{\pgfqpoint{3.646560in}{0.739656in}}%
\pgfpathlineto{\pgfqpoint{3.646263in}{0.739656in}}%
\pgfpathlineto{\pgfqpoint{3.645965in}{0.739656in}}%
\pgfpathlineto{\pgfqpoint{3.645668in}{0.739656in}}%
\pgfpathlineto{\pgfqpoint{3.645370in}{0.739656in}}%
\pgfpathlineto{\pgfqpoint{3.645073in}{0.739656in}}%
\pgfpathlineto{\pgfqpoint{3.644775in}{0.739656in}}%
\pgfpathlineto{\pgfqpoint{3.644478in}{0.739656in}}%
\pgfpathlineto{\pgfqpoint{3.644180in}{0.739656in}}%
\pgfpathlineto{\pgfqpoint{3.643883in}{0.739656in}}%
\pgfpathlineto{\pgfqpoint{3.643585in}{0.739656in}}%
\pgfpathlineto{\pgfqpoint{3.643288in}{0.739656in}}%
\pgfpathlineto{\pgfqpoint{3.642991in}{0.739656in}}%
\pgfpathlineto{\pgfqpoint{3.642693in}{0.739656in}}%
\pgfpathlineto{\pgfqpoint{3.642396in}{0.739656in}}%
\pgfpathlineto{\pgfqpoint{3.642098in}{0.739656in}}%
\pgfpathlineto{\pgfqpoint{3.641801in}{0.739656in}}%
\pgfpathlineto{\pgfqpoint{3.641503in}{0.739656in}}%
\pgfpathlineto{\pgfqpoint{3.641206in}{0.739656in}}%
\pgfpathlineto{\pgfqpoint{3.640908in}{0.739656in}}%
\pgfpathlineto{\pgfqpoint{3.640611in}{0.739656in}}%
\pgfpathlineto{\pgfqpoint{3.640313in}{0.739656in}}%
\pgfpathlineto{\pgfqpoint{3.640016in}{0.739656in}}%
\pgfpathlineto{\pgfqpoint{3.639718in}{0.739656in}}%
\pgfpathlineto{\pgfqpoint{3.639421in}{0.739656in}}%
\pgfpathlineto{\pgfqpoint{3.639123in}{0.739656in}}%
\pgfpathlineto{\pgfqpoint{3.638826in}{0.739656in}}%
\pgfpathlineto{\pgfqpoint{3.638528in}{0.739656in}}%
\pgfpathlineto{\pgfqpoint{3.638231in}{0.739656in}}%
\pgfpathlineto{\pgfqpoint{3.637933in}{0.739656in}}%
\pgfpathlineto{\pgfqpoint{3.637636in}{0.739656in}}%
\pgfpathlineto{\pgfqpoint{3.637338in}{0.739656in}}%
\pgfpathlineto{\pgfqpoint{3.637041in}{0.739656in}}%
\pgfpathlineto{\pgfqpoint{3.636743in}{0.739656in}}%
\pgfpathlineto{\pgfqpoint{3.636446in}{0.739656in}}%
\pgfpathlineto{\pgfqpoint{3.636149in}{0.739656in}}%
\pgfpathlineto{\pgfqpoint{3.635851in}{0.739656in}}%
\pgfpathlineto{\pgfqpoint{3.635554in}{0.739656in}}%
\pgfpathlineto{\pgfqpoint{3.635256in}{0.739656in}}%
\pgfpathlineto{\pgfqpoint{3.634959in}{0.739656in}}%
\pgfpathlineto{\pgfqpoint{3.634661in}{0.739656in}}%
\pgfpathlineto{\pgfqpoint{3.634364in}{0.739656in}}%
\pgfpathlineto{\pgfqpoint{3.634066in}{0.739656in}}%
\pgfpathlineto{\pgfqpoint{3.633769in}{0.739656in}}%
\pgfpathlineto{\pgfqpoint{3.633471in}{0.739656in}}%
\pgfpathlineto{\pgfqpoint{3.633174in}{0.739656in}}%
\pgfpathlineto{\pgfqpoint{3.632876in}{0.739656in}}%
\pgfpathlineto{\pgfqpoint{3.632579in}{0.739656in}}%
\pgfpathlineto{\pgfqpoint{3.632281in}{0.739656in}}%
\pgfpathlineto{\pgfqpoint{3.631984in}{0.739656in}}%
\pgfpathlineto{\pgfqpoint{3.631686in}{0.739656in}}%
\pgfpathlineto{\pgfqpoint{3.631389in}{0.739656in}}%
\pgfpathlineto{\pgfqpoint{3.631091in}{0.739656in}}%
\pgfpathlineto{\pgfqpoint{3.630794in}{0.739656in}}%
\pgfpathlineto{\pgfqpoint{3.630496in}{0.739656in}}%
\pgfpathlineto{\pgfqpoint{3.630199in}{0.739656in}}%
\pgfpathlineto{\pgfqpoint{3.629901in}{0.739656in}}%
\pgfpathlineto{\pgfqpoint{3.629604in}{0.739656in}}%
\pgfpathlineto{\pgfqpoint{3.629307in}{0.739656in}}%
\pgfpathlineto{\pgfqpoint{3.629009in}{0.739656in}}%
\pgfpathlineto{\pgfqpoint{3.628712in}{0.739656in}}%
\pgfpathlineto{\pgfqpoint{3.628414in}{0.739656in}}%
\pgfpathlineto{\pgfqpoint{3.628117in}{0.739656in}}%
\pgfpathlineto{\pgfqpoint{3.627819in}{0.739656in}}%
\pgfpathlineto{\pgfqpoint{3.627522in}{0.739656in}}%
\pgfpathlineto{\pgfqpoint{3.627224in}{0.739656in}}%
\pgfpathlineto{\pgfqpoint{3.626927in}{0.739656in}}%
\pgfpathlineto{\pgfqpoint{3.626629in}{0.739656in}}%
\pgfpathlineto{\pgfqpoint{3.626332in}{0.739656in}}%
\pgfpathlineto{\pgfqpoint{3.626034in}{0.739656in}}%
\pgfpathlineto{\pgfqpoint{3.625737in}{0.739656in}}%
\pgfpathlineto{\pgfqpoint{3.625439in}{0.739656in}}%
\pgfpathlineto{\pgfqpoint{3.625142in}{0.739656in}}%
\pgfpathlineto{\pgfqpoint{3.624844in}{0.739656in}}%
\pgfpathlineto{\pgfqpoint{3.624547in}{0.739656in}}%
\pgfpathlineto{\pgfqpoint{3.624249in}{0.739656in}}%
\pgfpathlineto{\pgfqpoint{3.623952in}{0.739656in}}%
\pgfpathlineto{\pgfqpoint{3.623654in}{0.739656in}}%
\pgfpathlineto{\pgfqpoint{3.623357in}{0.739656in}}%
\pgfpathlineto{\pgfqpoint{3.623060in}{0.739656in}}%
\pgfpathlineto{\pgfqpoint{3.622762in}{0.739656in}}%
\pgfpathlineto{\pgfqpoint{3.622465in}{0.739656in}}%
\pgfpathlineto{\pgfqpoint{3.622167in}{0.739656in}}%
\pgfpathlineto{\pgfqpoint{3.621870in}{0.739656in}}%
\pgfpathlineto{\pgfqpoint{3.621572in}{0.739656in}}%
\pgfpathlineto{\pgfqpoint{3.621275in}{0.739656in}}%
\pgfpathlineto{\pgfqpoint{3.620977in}{0.739656in}}%
\pgfpathlineto{\pgfqpoint{3.620680in}{0.739656in}}%
\pgfpathlineto{\pgfqpoint{3.620382in}{0.739656in}}%
\pgfpathlineto{\pgfqpoint{3.620085in}{0.739656in}}%
\pgfpathlineto{\pgfqpoint{3.619787in}{0.739656in}}%
\pgfpathlineto{\pgfqpoint{3.619490in}{0.739656in}}%
\pgfpathlineto{\pgfqpoint{3.619192in}{0.739656in}}%
\pgfpathlineto{\pgfqpoint{3.618895in}{0.739656in}}%
\pgfpathlineto{\pgfqpoint{3.618597in}{0.739656in}}%
\pgfpathlineto{\pgfqpoint{3.618300in}{0.739656in}}%
\pgfpathlineto{\pgfqpoint{3.618002in}{0.739656in}}%
\pgfpathlineto{\pgfqpoint{3.617705in}{0.739656in}}%
\pgfpathlineto{\pgfqpoint{3.617407in}{0.739656in}}%
\pgfpathlineto{\pgfqpoint{3.617110in}{0.739656in}}%
\pgfpathlineto{\pgfqpoint{3.616812in}{0.739656in}}%
\pgfpathlineto{\pgfqpoint{3.616515in}{0.739656in}}%
\pgfpathlineto{\pgfqpoint{3.616218in}{0.739656in}}%
\pgfpathlineto{\pgfqpoint{3.615920in}{0.739656in}}%
\pgfpathlineto{\pgfqpoint{3.615623in}{0.739656in}}%
\pgfpathlineto{\pgfqpoint{3.615325in}{0.739656in}}%
\pgfpathlineto{\pgfqpoint{3.615028in}{0.739656in}}%
\pgfpathlineto{\pgfqpoint{3.614730in}{0.739656in}}%
\pgfpathlineto{\pgfqpoint{3.614433in}{0.739656in}}%
\pgfpathlineto{\pgfqpoint{3.614135in}{0.739656in}}%
\pgfpathlineto{\pgfqpoint{3.613838in}{0.739656in}}%
\pgfpathlineto{\pgfqpoint{3.613540in}{0.739656in}}%
\pgfpathlineto{\pgfqpoint{3.613243in}{0.739656in}}%
\pgfpathlineto{\pgfqpoint{3.612945in}{0.739656in}}%
\pgfpathlineto{\pgfqpoint{3.612648in}{0.739656in}}%
\pgfpathlineto{\pgfqpoint{3.612350in}{0.739656in}}%
\pgfpathlineto{\pgfqpoint{3.612053in}{0.739656in}}%
\pgfpathlineto{\pgfqpoint{3.611755in}{0.739656in}}%
\pgfpathlineto{\pgfqpoint{3.611458in}{0.739656in}}%
\pgfpathlineto{\pgfqpoint{3.611160in}{0.739656in}}%
\pgfpathlineto{\pgfqpoint{3.610863in}{0.739656in}}%
\pgfpathlineto{\pgfqpoint{3.610565in}{0.739656in}}%
\pgfpathlineto{\pgfqpoint{3.610268in}{0.739656in}}%
\pgfpathlineto{\pgfqpoint{3.609970in}{0.739656in}}%
\pgfpathlineto{\pgfqpoint{3.609673in}{0.739656in}}%
\pgfpathlineto{\pgfqpoint{3.609376in}{0.739656in}}%
\pgfpathlineto{\pgfqpoint{3.609078in}{0.739656in}}%
\pgfpathlineto{\pgfqpoint{3.608781in}{0.739656in}}%
\pgfpathlineto{\pgfqpoint{3.608483in}{0.739656in}}%
\pgfpathlineto{\pgfqpoint{3.608186in}{0.739656in}}%
\pgfpathlineto{\pgfqpoint{3.607888in}{0.739656in}}%
\pgfpathlineto{\pgfqpoint{3.607591in}{0.739656in}}%
\pgfpathlineto{\pgfqpoint{3.607293in}{0.739656in}}%
\pgfpathlineto{\pgfqpoint{3.606996in}{0.739656in}}%
\pgfpathlineto{\pgfqpoint{3.606698in}{0.739656in}}%
\pgfpathlineto{\pgfqpoint{3.606401in}{0.739656in}}%
\pgfpathlineto{\pgfqpoint{3.606103in}{0.739656in}}%
\pgfpathlineto{\pgfqpoint{3.605806in}{0.739656in}}%
\pgfpathlineto{\pgfqpoint{3.605508in}{0.739656in}}%
\pgfpathlineto{\pgfqpoint{3.605211in}{0.739656in}}%
\pgfpathlineto{\pgfqpoint{3.604913in}{0.739656in}}%
\pgfpathlineto{\pgfqpoint{3.604616in}{0.739656in}}%
\pgfpathlineto{\pgfqpoint{3.604318in}{0.739656in}}%
\pgfpathlineto{\pgfqpoint{3.604021in}{0.739656in}}%
\pgfpathlineto{\pgfqpoint{3.603723in}{0.739656in}}%
\pgfpathlineto{\pgfqpoint{3.603426in}{0.739656in}}%
\pgfpathlineto{\pgfqpoint{3.603129in}{0.739656in}}%
\pgfpathlineto{\pgfqpoint{3.602831in}{0.739656in}}%
\pgfpathlineto{\pgfqpoint{3.602534in}{0.739656in}}%
\pgfpathlineto{\pgfqpoint{3.602236in}{0.739656in}}%
\pgfpathlineto{\pgfqpoint{3.601939in}{0.739656in}}%
\pgfpathlineto{\pgfqpoint{3.601641in}{0.739656in}}%
\pgfpathlineto{\pgfqpoint{3.601344in}{0.739656in}}%
\pgfpathlineto{\pgfqpoint{3.601046in}{0.739656in}}%
\pgfpathlineto{\pgfqpoint{3.600749in}{0.739656in}}%
\pgfpathlineto{\pgfqpoint{3.600451in}{0.739656in}}%
\pgfpathlineto{\pgfqpoint{3.600154in}{0.739656in}}%
\pgfpathlineto{\pgfqpoint{3.599856in}{0.739656in}}%
\pgfpathlineto{\pgfqpoint{3.599559in}{0.739656in}}%
\pgfpathlineto{\pgfqpoint{3.599261in}{0.739656in}}%
\pgfpathlineto{\pgfqpoint{3.598964in}{0.739656in}}%
\pgfpathlineto{\pgfqpoint{3.598666in}{0.739656in}}%
\pgfpathlineto{\pgfqpoint{3.598369in}{0.739656in}}%
\pgfpathlineto{\pgfqpoint{3.598071in}{0.739656in}}%
\pgfpathlineto{\pgfqpoint{3.597774in}{0.739656in}}%
\pgfpathlineto{\pgfqpoint{3.597476in}{0.739656in}}%
\pgfpathlineto{\pgfqpoint{3.597179in}{0.739656in}}%
\pgfpathlineto{\pgfqpoint{3.596881in}{0.739656in}}%
\pgfpathlineto{\pgfqpoint{3.596584in}{0.739656in}}%
\pgfpathlineto{\pgfqpoint{3.596287in}{0.739656in}}%
\pgfpathlineto{\pgfqpoint{3.595989in}{0.739656in}}%
\pgfpathlineto{\pgfqpoint{3.595692in}{0.739656in}}%
\pgfpathlineto{\pgfqpoint{3.595394in}{0.739656in}}%
\pgfpathlineto{\pgfqpoint{3.595097in}{0.739656in}}%
\pgfpathlineto{\pgfqpoint{3.594799in}{0.739656in}}%
\pgfpathlineto{\pgfqpoint{3.594502in}{0.739656in}}%
\pgfpathlineto{\pgfqpoint{3.594204in}{0.739656in}}%
\pgfpathlineto{\pgfqpoint{3.593907in}{0.739656in}}%
\pgfpathlineto{\pgfqpoint{3.593609in}{0.739656in}}%
\pgfpathlineto{\pgfqpoint{3.593312in}{0.739656in}}%
\pgfpathlineto{\pgfqpoint{3.593014in}{0.739656in}}%
\pgfpathlineto{\pgfqpoint{3.592717in}{0.739656in}}%
\pgfpathlineto{\pgfqpoint{3.592419in}{0.739656in}}%
\pgfpathlineto{\pgfqpoint{3.592122in}{0.739656in}}%
\pgfpathlineto{\pgfqpoint{3.591824in}{0.739656in}}%
\pgfpathlineto{\pgfqpoint{3.591527in}{0.739656in}}%
\pgfpathlineto{\pgfqpoint{3.591229in}{0.739656in}}%
\pgfpathlineto{\pgfqpoint{3.590932in}{0.739656in}}%
\pgfpathlineto{\pgfqpoint{3.590634in}{0.739656in}}%
\pgfpathlineto{\pgfqpoint{3.590337in}{0.739656in}}%
\pgfpathlineto{\pgfqpoint{3.590039in}{0.739656in}}%
\pgfpathlineto{\pgfqpoint{3.589742in}{0.739656in}}%
\pgfpathlineto{\pgfqpoint{3.589445in}{0.739656in}}%
\pgfpathlineto{\pgfqpoint{3.589147in}{0.739656in}}%
\pgfpathlineto{\pgfqpoint{3.588850in}{0.739656in}}%
\pgfpathlineto{\pgfqpoint{3.588552in}{0.739656in}}%
\pgfpathlineto{\pgfqpoint{3.588255in}{0.739656in}}%
\pgfpathlineto{\pgfqpoint{3.587957in}{0.739656in}}%
\pgfpathlineto{\pgfqpoint{3.587660in}{0.739656in}}%
\pgfpathlineto{\pgfqpoint{3.587362in}{0.739656in}}%
\pgfpathlineto{\pgfqpoint{3.587065in}{0.739656in}}%
\pgfpathlineto{\pgfqpoint{3.586767in}{0.739656in}}%
\pgfpathlineto{\pgfqpoint{3.586470in}{0.739656in}}%
\pgfpathlineto{\pgfqpoint{3.586172in}{0.739656in}}%
\pgfpathlineto{\pgfqpoint{3.585875in}{0.739656in}}%
\pgfpathlineto{\pgfqpoint{3.585577in}{0.739656in}}%
\pgfpathlineto{\pgfqpoint{3.585280in}{0.739656in}}%
\pgfpathlineto{\pgfqpoint{3.584982in}{0.739656in}}%
\pgfpathlineto{\pgfqpoint{3.584685in}{0.739656in}}%
\pgfpathlineto{\pgfqpoint{3.584387in}{0.739656in}}%
\pgfpathlineto{\pgfqpoint{3.584090in}{0.739656in}}%
\pgfpathlineto{\pgfqpoint{3.583792in}{0.739656in}}%
\pgfpathlineto{\pgfqpoint{3.583495in}{0.739656in}}%
\pgfpathlineto{\pgfqpoint{3.583198in}{0.739656in}}%
\pgfpathlineto{\pgfqpoint{3.582900in}{0.739656in}}%
\pgfpathlineto{\pgfqpoint{3.582603in}{0.739656in}}%
\pgfpathlineto{\pgfqpoint{3.582305in}{0.739656in}}%
\pgfpathlineto{\pgfqpoint{3.582008in}{0.739656in}}%
\pgfpathlineto{\pgfqpoint{3.581710in}{0.739656in}}%
\pgfpathlineto{\pgfqpoint{3.581413in}{0.739656in}}%
\pgfpathlineto{\pgfqpoint{3.581115in}{0.739656in}}%
\pgfpathlineto{\pgfqpoint{3.580818in}{0.739656in}}%
\pgfpathlineto{\pgfqpoint{3.580520in}{0.739656in}}%
\pgfpathlineto{\pgfqpoint{3.580223in}{0.739656in}}%
\pgfpathlineto{\pgfqpoint{3.579925in}{0.739656in}}%
\pgfpathlineto{\pgfqpoint{3.579628in}{0.739656in}}%
\pgfpathlineto{\pgfqpoint{3.579330in}{0.739656in}}%
\pgfpathlineto{\pgfqpoint{3.579033in}{0.739656in}}%
\pgfpathlineto{\pgfqpoint{3.578735in}{0.739656in}}%
\pgfpathlineto{\pgfqpoint{3.578438in}{0.739656in}}%
\pgfpathlineto{\pgfqpoint{3.578140in}{0.739656in}}%
\pgfpathlineto{\pgfqpoint{3.577843in}{0.739656in}}%
\pgfpathlineto{\pgfqpoint{3.577545in}{0.739656in}}%
\pgfpathlineto{\pgfqpoint{3.577248in}{0.739656in}}%
\pgfpathlineto{\pgfqpoint{3.576950in}{0.739656in}}%
\pgfpathlineto{\pgfqpoint{3.576653in}{0.739656in}}%
\pgfpathlineto{\pgfqpoint{3.576356in}{0.739656in}}%
\pgfpathlineto{\pgfqpoint{3.576058in}{0.739656in}}%
\pgfpathlineto{\pgfqpoint{3.575761in}{0.739656in}}%
\pgfpathlineto{\pgfqpoint{3.575463in}{0.739656in}}%
\pgfpathlineto{\pgfqpoint{3.575166in}{0.739656in}}%
\pgfpathlineto{\pgfqpoint{3.574868in}{0.739656in}}%
\pgfpathlineto{\pgfqpoint{3.574571in}{0.739656in}}%
\pgfpathlineto{\pgfqpoint{3.574273in}{0.739656in}}%
\pgfpathlineto{\pgfqpoint{3.573976in}{0.739656in}}%
\pgfpathlineto{\pgfqpoint{3.573678in}{0.739656in}}%
\pgfpathlineto{\pgfqpoint{3.573381in}{0.739656in}}%
\pgfpathlineto{\pgfqpoint{3.573083in}{0.739656in}}%
\pgfpathlineto{\pgfqpoint{3.572786in}{0.739656in}}%
\pgfpathlineto{\pgfqpoint{3.572488in}{0.739656in}}%
\pgfpathlineto{\pgfqpoint{3.572191in}{0.739656in}}%
\pgfpathlineto{\pgfqpoint{3.571893in}{0.739656in}}%
\pgfpathlineto{\pgfqpoint{3.571596in}{0.739656in}}%
\pgfpathlineto{\pgfqpoint{3.571298in}{0.739656in}}%
\pgfpathlineto{\pgfqpoint{3.571001in}{0.739656in}}%
\pgfpathlineto{\pgfqpoint{3.570703in}{0.739656in}}%
\pgfpathlineto{\pgfqpoint{3.570406in}{0.739656in}}%
\pgfpathlineto{\pgfqpoint{3.570108in}{0.739656in}}%
\pgfpathlineto{\pgfqpoint{3.569811in}{0.739656in}}%
\pgfpathlineto{\pgfqpoint{3.569514in}{0.739656in}}%
\pgfpathlineto{\pgfqpoint{3.569216in}{0.739656in}}%
\pgfpathlineto{\pgfqpoint{3.568919in}{0.739656in}}%
\pgfpathlineto{\pgfqpoint{3.568621in}{0.739656in}}%
\pgfpathlineto{\pgfqpoint{3.568324in}{0.739656in}}%
\pgfpathlineto{\pgfqpoint{3.568026in}{0.739656in}}%
\pgfpathlineto{\pgfqpoint{3.567729in}{0.739656in}}%
\pgfpathlineto{\pgfqpoint{3.567431in}{0.739656in}}%
\pgfpathlineto{\pgfqpoint{3.567134in}{0.739656in}}%
\pgfpathlineto{\pgfqpoint{3.566836in}{0.739656in}}%
\pgfpathlineto{\pgfqpoint{3.566539in}{0.739656in}}%
\pgfpathlineto{\pgfqpoint{3.566241in}{0.739656in}}%
\pgfpathlineto{\pgfqpoint{3.565944in}{0.739656in}}%
\pgfpathlineto{\pgfqpoint{3.565646in}{0.739656in}}%
\pgfpathlineto{\pgfqpoint{3.565349in}{0.739656in}}%
\pgfpathlineto{\pgfqpoint{3.565051in}{0.739656in}}%
\pgfpathlineto{\pgfqpoint{3.564754in}{0.739656in}}%
\pgfpathlineto{\pgfqpoint{3.564456in}{0.739656in}}%
\pgfpathlineto{\pgfqpoint{3.564159in}{0.739656in}}%
\pgfpathlineto{\pgfqpoint{3.563861in}{0.739656in}}%
\pgfpathlineto{\pgfqpoint{3.563564in}{0.739656in}}%
\pgfpathlineto{\pgfqpoint{3.563267in}{0.739656in}}%
\pgfpathlineto{\pgfqpoint{3.562969in}{0.739656in}}%
\pgfpathlineto{\pgfqpoint{3.562672in}{0.739656in}}%
\pgfpathlineto{\pgfqpoint{3.562374in}{0.739656in}}%
\pgfpathlineto{\pgfqpoint{3.562077in}{0.739656in}}%
\pgfpathlineto{\pgfqpoint{3.561779in}{0.739656in}}%
\pgfpathlineto{\pgfqpoint{3.561482in}{0.739656in}}%
\pgfpathlineto{\pgfqpoint{3.561184in}{0.739656in}}%
\pgfpathlineto{\pgfqpoint{3.560887in}{0.739656in}}%
\pgfpathlineto{\pgfqpoint{3.560589in}{0.739656in}}%
\pgfpathlineto{\pgfqpoint{3.560292in}{0.739656in}}%
\pgfpathlineto{\pgfqpoint{3.559994in}{0.739656in}}%
\pgfpathlineto{\pgfqpoint{3.559697in}{0.739656in}}%
\pgfpathlineto{\pgfqpoint{3.559399in}{0.739656in}}%
\pgfpathlineto{\pgfqpoint{3.559102in}{0.739656in}}%
\pgfpathlineto{\pgfqpoint{3.558804in}{0.739656in}}%
\pgfpathlineto{\pgfqpoint{3.558507in}{0.739656in}}%
\pgfpathlineto{\pgfqpoint{3.558209in}{0.739656in}}%
\pgfpathlineto{\pgfqpoint{3.557912in}{0.739656in}}%
\pgfpathlineto{\pgfqpoint{3.557614in}{0.739656in}}%
\pgfpathlineto{\pgfqpoint{3.557317in}{0.739656in}}%
\pgfpathlineto{\pgfqpoint{3.557019in}{0.739656in}}%
\pgfpathlineto{\pgfqpoint{3.556722in}{0.739656in}}%
\pgfpathlineto{\pgfqpoint{3.556425in}{0.739656in}}%
\pgfpathlineto{\pgfqpoint{3.556127in}{0.739656in}}%
\pgfpathlineto{\pgfqpoint{3.555830in}{0.739656in}}%
\pgfpathlineto{\pgfqpoint{3.555532in}{0.739656in}}%
\pgfpathlineto{\pgfqpoint{3.555235in}{0.739656in}}%
\pgfpathlineto{\pgfqpoint{3.554937in}{0.739656in}}%
\pgfpathlineto{\pgfqpoint{3.554640in}{0.739656in}}%
\pgfpathlineto{\pgfqpoint{3.554342in}{0.739656in}}%
\pgfpathlineto{\pgfqpoint{3.554045in}{0.739656in}}%
\pgfpathlineto{\pgfqpoint{3.553747in}{0.739656in}}%
\pgfpathlineto{\pgfqpoint{3.553450in}{0.739656in}}%
\pgfpathlineto{\pgfqpoint{3.553152in}{0.739656in}}%
\pgfpathlineto{\pgfqpoint{3.552855in}{0.739656in}}%
\pgfpathlineto{\pgfqpoint{3.552557in}{0.739656in}}%
\pgfpathlineto{\pgfqpoint{3.552260in}{0.739656in}}%
\pgfpathlineto{\pgfqpoint{3.551962in}{0.739656in}}%
\pgfpathlineto{\pgfqpoint{3.551665in}{0.739656in}}%
\pgfpathlineto{\pgfqpoint{3.551367in}{0.739656in}}%
\pgfpathlineto{\pgfqpoint{3.551070in}{0.739656in}}%
\pgfpathlineto{\pgfqpoint{3.550772in}{0.739656in}}%
\pgfpathlineto{\pgfqpoint{3.550475in}{0.739656in}}%
\pgfpathlineto{\pgfqpoint{3.550177in}{0.739656in}}%
\pgfpathlineto{\pgfqpoint{3.549880in}{0.739656in}}%
\pgfpathlineto{\pgfqpoint{3.549583in}{0.739656in}}%
\pgfpathlineto{\pgfqpoint{3.549285in}{0.739656in}}%
\pgfpathlineto{\pgfqpoint{3.548988in}{0.739656in}}%
\pgfpathlineto{\pgfqpoint{3.548690in}{0.739656in}}%
\pgfpathlineto{\pgfqpoint{3.548393in}{0.739656in}}%
\pgfpathlineto{\pgfqpoint{3.548095in}{0.739656in}}%
\pgfpathlineto{\pgfqpoint{3.547798in}{0.739656in}}%
\pgfpathlineto{\pgfqpoint{3.547500in}{0.739656in}}%
\pgfpathlineto{\pgfqpoint{3.547203in}{0.739656in}}%
\pgfpathlineto{\pgfqpoint{3.546905in}{0.739656in}}%
\pgfpathlineto{\pgfqpoint{3.546608in}{0.739656in}}%
\pgfpathlineto{\pgfqpoint{3.546310in}{0.739656in}}%
\pgfpathlineto{\pgfqpoint{3.546013in}{0.739656in}}%
\pgfpathlineto{\pgfqpoint{3.545715in}{0.739656in}}%
\pgfpathlineto{\pgfqpoint{3.545418in}{0.739656in}}%
\pgfpathlineto{\pgfqpoint{3.545120in}{0.739656in}}%
\pgfpathlineto{\pgfqpoint{3.544823in}{0.739656in}}%
\pgfpathlineto{\pgfqpoint{3.544525in}{0.739656in}}%
\pgfpathlineto{\pgfqpoint{3.544228in}{0.739656in}}%
\pgfpathlineto{\pgfqpoint{3.543930in}{0.739656in}}%
\pgfpathlineto{\pgfqpoint{3.543633in}{0.739656in}}%
\pgfpathlineto{\pgfqpoint{3.543336in}{0.739656in}}%
\pgfpathlineto{\pgfqpoint{3.543038in}{0.739656in}}%
\pgfpathlineto{\pgfqpoint{3.542741in}{0.739656in}}%
\pgfpathlineto{\pgfqpoint{3.542443in}{0.739656in}}%
\pgfpathlineto{\pgfqpoint{3.542146in}{0.739656in}}%
\pgfpathlineto{\pgfqpoint{3.541848in}{0.739656in}}%
\pgfpathlineto{\pgfqpoint{3.541551in}{0.739656in}}%
\pgfpathlineto{\pgfqpoint{3.541253in}{0.739656in}}%
\pgfpathlineto{\pgfqpoint{3.540956in}{0.739656in}}%
\pgfpathlineto{\pgfqpoint{3.540658in}{0.739656in}}%
\pgfpathlineto{\pgfqpoint{3.540361in}{0.739656in}}%
\pgfpathlineto{\pgfqpoint{3.540063in}{0.739656in}}%
\pgfpathlineto{\pgfqpoint{3.539766in}{0.739656in}}%
\pgfpathlineto{\pgfqpoint{3.539468in}{0.739656in}}%
\pgfpathlineto{\pgfqpoint{3.539171in}{0.739656in}}%
\pgfpathlineto{\pgfqpoint{3.538873in}{0.739656in}}%
\pgfpathlineto{\pgfqpoint{3.538576in}{0.739656in}}%
\pgfpathlineto{\pgfqpoint{3.538278in}{0.739656in}}%
\pgfpathlineto{\pgfqpoint{3.537981in}{0.739656in}}%
\pgfpathlineto{\pgfqpoint{3.537683in}{0.739656in}}%
\pgfpathlineto{\pgfqpoint{3.537386in}{0.739656in}}%
\pgfpathlineto{\pgfqpoint{3.537088in}{0.739656in}}%
\pgfpathlineto{\pgfqpoint{3.536791in}{0.739656in}}%
\pgfpathlineto{\pgfqpoint{3.536494in}{0.739656in}}%
\pgfpathlineto{\pgfqpoint{3.536196in}{0.739656in}}%
\pgfpathlineto{\pgfqpoint{3.535899in}{0.739656in}}%
\pgfpathlineto{\pgfqpoint{3.535601in}{0.739656in}}%
\pgfpathlineto{\pgfqpoint{3.535304in}{0.739656in}}%
\pgfpathlineto{\pgfqpoint{3.535006in}{0.739656in}}%
\pgfpathlineto{\pgfqpoint{3.534709in}{0.739656in}}%
\pgfpathlineto{\pgfqpoint{3.534411in}{0.739656in}}%
\pgfpathlineto{\pgfqpoint{3.534114in}{0.739656in}}%
\pgfpathlineto{\pgfqpoint{3.533816in}{0.739656in}}%
\pgfpathlineto{\pgfqpoint{3.533519in}{0.739656in}}%
\pgfpathlineto{\pgfqpoint{3.533221in}{0.739656in}}%
\pgfpathlineto{\pgfqpoint{3.532924in}{0.739656in}}%
\pgfpathlineto{\pgfqpoint{3.532626in}{0.739656in}}%
\pgfpathlineto{\pgfqpoint{3.532329in}{0.739656in}}%
\pgfpathlineto{\pgfqpoint{3.532031in}{0.739656in}}%
\pgfpathlineto{\pgfqpoint{3.531734in}{0.739656in}}%
\pgfpathlineto{\pgfqpoint{3.531436in}{0.739656in}}%
\pgfpathlineto{\pgfqpoint{3.531139in}{0.739656in}}%
\pgfpathlineto{\pgfqpoint{3.530841in}{0.739656in}}%
\pgfpathlineto{\pgfqpoint{3.530544in}{0.739656in}}%
\pgfpathlineto{\pgfqpoint{3.530246in}{0.739656in}}%
\pgfpathlineto{\pgfqpoint{3.529949in}{0.739656in}}%
\pgfpathlineto{\pgfqpoint{3.529652in}{0.739656in}}%
\pgfpathlineto{\pgfqpoint{3.529354in}{0.739656in}}%
\pgfpathlineto{\pgfqpoint{3.529057in}{0.739656in}}%
\pgfpathlineto{\pgfqpoint{3.528759in}{0.739656in}}%
\pgfpathlineto{\pgfqpoint{3.528462in}{0.739656in}}%
\pgfpathlineto{\pgfqpoint{3.528164in}{0.739656in}}%
\pgfpathlineto{\pgfqpoint{3.527867in}{0.739656in}}%
\pgfpathlineto{\pgfqpoint{3.527569in}{0.739656in}}%
\pgfpathlineto{\pgfqpoint{3.527272in}{0.739656in}}%
\pgfpathlineto{\pgfqpoint{3.526974in}{0.739656in}}%
\pgfpathlineto{\pgfqpoint{3.526677in}{0.739656in}}%
\pgfpathlineto{\pgfqpoint{3.526379in}{0.739656in}}%
\pgfpathlineto{\pgfqpoint{3.526082in}{0.739656in}}%
\pgfpathlineto{\pgfqpoint{3.525784in}{0.739656in}}%
\pgfpathlineto{\pgfqpoint{3.525487in}{0.739656in}}%
\pgfpathlineto{\pgfqpoint{3.525189in}{0.739656in}}%
\pgfpathlineto{\pgfqpoint{3.524892in}{0.739656in}}%
\pgfpathlineto{\pgfqpoint{3.524594in}{0.739656in}}%
\pgfpathlineto{\pgfqpoint{3.524297in}{0.739656in}}%
\pgfpathlineto{\pgfqpoint{3.523999in}{0.739656in}}%
\pgfpathlineto{\pgfqpoint{3.523702in}{0.739656in}}%
\pgfpathlineto{\pgfqpoint{3.523404in}{0.739656in}}%
\pgfpathlineto{\pgfqpoint{3.523107in}{0.739656in}}%
\pgfpathlineto{\pgfqpoint{3.522810in}{0.739656in}}%
\pgfpathlineto{\pgfqpoint{3.522512in}{0.739656in}}%
\pgfpathlineto{\pgfqpoint{3.522215in}{0.739656in}}%
\pgfpathlineto{\pgfqpoint{3.521917in}{0.739656in}}%
\pgfpathlineto{\pgfqpoint{3.521620in}{0.739656in}}%
\pgfpathlineto{\pgfqpoint{3.521322in}{0.739656in}}%
\pgfpathlineto{\pgfqpoint{3.521025in}{0.739656in}}%
\pgfpathlineto{\pgfqpoint{3.520727in}{0.739656in}}%
\pgfpathlineto{\pgfqpoint{3.520430in}{0.739656in}}%
\pgfpathlineto{\pgfqpoint{3.520132in}{0.739656in}}%
\pgfpathlineto{\pgfqpoint{3.519835in}{0.739656in}}%
\pgfpathlineto{\pgfqpoint{3.519537in}{0.739656in}}%
\pgfpathlineto{\pgfqpoint{3.519240in}{0.739656in}}%
\pgfpathlineto{\pgfqpoint{3.518942in}{0.739656in}}%
\pgfpathlineto{\pgfqpoint{3.518645in}{0.739656in}}%
\pgfpathlineto{\pgfqpoint{3.518347in}{0.739656in}}%
\pgfpathlineto{\pgfqpoint{3.518050in}{0.739656in}}%
\pgfpathlineto{\pgfqpoint{3.517752in}{0.739656in}}%
\pgfpathlineto{\pgfqpoint{3.517455in}{0.739656in}}%
\pgfpathlineto{\pgfqpoint{3.517157in}{0.739656in}}%
\pgfpathlineto{\pgfqpoint{3.516860in}{0.739656in}}%
\pgfpathlineto{\pgfqpoint{3.516563in}{0.739656in}}%
\pgfpathlineto{\pgfqpoint{3.516265in}{0.739656in}}%
\pgfpathlineto{\pgfqpoint{3.515968in}{0.739656in}}%
\pgfpathlineto{\pgfqpoint{3.515670in}{0.739656in}}%
\pgfpathlineto{\pgfqpoint{3.515373in}{0.739656in}}%
\pgfpathlineto{\pgfqpoint{3.515075in}{0.739656in}}%
\pgfpathlineto{\pgfqpoint{3.514778in}{0.739656in}}%
\pgfpathlineto{\pgfqpoint{3.514480in}{0.739656in}}%
\pgfpathlineto{\pgfqpoint{3.514183in}{0.739656in}}%
\pgfpathlineto{\pgfqpoint{3.513885in}{0.739656in}}%
\pgfpathlineto{\pgfqpoint{3.513588in}{0.739656in}}%
\pgfpathlineto{\pgfqpoint{3.513290in}{0.739656in}}%
\pgfpathlineto{\pgfqpoint{3.512993in}{0.739656in}}%
\pgfpathlineto{\pgfqpoint{3.512695in}{0.739656in}}%
\pgfpathlineto{\pgfqpoint{3.512398in}{0.739656in}}%
\pgfpathlineto{\pgfqpoint{3.512100in}{0.739656in}}%
\pgfpathlineto{\pgfqpoint{3.511803in}{0.739656in}}%
\pgfpathlineto{\pgfqpoint{3.511505in}{0.739656in}}%
\pgfpathlineto{\pgfqpoint{3.511208in}{0.739656in}}%
\pgfpathlineto{\pgfqpoint{3.510910in}{0.739656in}}%
\pgfpathlineto{\pgfqpoint{3.510613in}{0.739656in}}%
\pgfpathlineto{\pgfqpoint{3.510315in}{0.739656in}}%
\pgfpathlineto{\pgfqpoint{3.510018in}{0.739656in}}%
\pgfpathlineto{\pgfqpoint{3.509721in}{0.739656in}}%
\pgfpathlineto{\pgfqpoint{3.509423in}{0.739656in}}%
\pgfpathlineto{\pgfqpoint{3.509126in}{0.739656in}}%
\pgfpathlineto{\pgfqpoint{3.508828in}{0.739656in}}%
\pgfpathlineto{\pgfqpoint{3.508531in}{0.739656in}}%
\pgfpathlineto{\pgfqpoint{3.508233in}{0.739656in}}%
\pgfpathlineto{\pgfqpoint{3.507936in}{0.739656in}}%
\pgfpathlineto{\pgfqpoint{3.507638in}{0.739656in}}%
\pgfpathlineto{\pgfqpoint{3.507341in}{0.739656in}}%
\pgfpathlineto{\pgfqpoint{3.507043in}{0.739656in}}%
\pgfpathlineto{\pgfqpoint{3.506746in}{0.739656in}}%
\pgfpathlineto{\pgfqpoint{3.506448in}{0.739656in}}%
\pgfpathlineto{\pgfqpoint{3.506151in}{0.739656in}}%
\pgfpathlineto{\pgfqpoint{3.505853in}{0.739656in}}%
\pgfpathlineto{\pgfqpoint{3.505556in}{0.739656in}}%
\pgfpathlineto{\pgfqpoint{3.505258in}{0.739656in}}%
\pgfpathlineto{\pgfqpoint{3.504961in}{0.739656in}}%
\pgfpathlineto{\pgfqpoint{3.504663in}{0.739656in}}%
\pgfpathlineto{\pgfqpoint{3.504366in}{0.739656in}}%
\pgfpathlineto{\pgfqpoint{3.504068in}{0.739656in}}%
\pgfpathlineto{\pgfqpoint{3.503771in}{0.739656in}}%
\pgfpathlineto{\pgfqpoint{3.503473in}{0.739656in}}%
\pgfpathlineto{\pgfqpoint{3.503176in}{0.739656in}}%
\pgfpathlineto{\pgfqpoint{3.502879in}{0.739656in}}%
\pgfpathlineto{\pgfqpoint{3.502581in}{0.739656in}}%
\pgfpathlineto{\pgfqpoint{3.502284in}{0.739656in}}%
\pgfpathlineto{\pgfqpoint{3.501986in}{0.739656in}}%
\pgfpathlineto{\pgfqpoint{3.501689in}{0.739656in}}%
\pgfpathlineto{\pgfqpoint{3.501391in}{0.739656in}}%
\pgfpathlineto{\pgfqpoint{3.501094in}{0.739656in}}%
\pgfpathlineto{\pgfqpoint{3.500796in}{0.739656in}}%
\pgfpathlineto{\pgfqpoint{3.500499in}{0.739656in}}%
\pgfpathlineto{\pgfqpoint{3.500201in}{0.739656in}}%
\pgfpathlineto{\pgfqpoint{3.499904in}{0.739656in}}%
\pgfpathlineto{\pgfqpoint{3.499606in}{0.739656in}}%
\pgfpathlineto{\pgfqpoint{3.499309in}{0.739656in}}%
\pgfpathlineto{\pgfqpoint{3.499011in}{0.739656in}}%
\pgfpathlineto{\pgfqpoint{3.498714in}{0.739656in}}%
\pgfpathlineto{\pgfqpoint{3.498416in}{0.739656in}}%
\pgfpathlineto{\pgfqpoint{3.498119in}{0.739656in}}%
\pgfpathlineto{\pgfqpoint{3.497821in}{0.739656in}}%
\pgfpathlineto{\pgfqpoint{3.497524in}{0.739656in}}%
\pgfpathlineto{\pgfqpoint{3.497226in}{0.739656in}}%
\pgfpathlineto{\pgfqpoint{3.496929in}{0.739656in}}%
\pgfpathlineto{\pgfqpoint{3.496632in}{0.739656in}}%
\pgfpathlineto{\pgfqpoint{3.496334in}{0.739656in}}%
\pgfpathlineto{\pgfqpoint{3.496037in}{0.739656in}}%
\pgfpathlineto{\pgfqpoint{3.495739in}{0.739656in}}%
\pgfpathlineto{\pgfqpoint{3.495442in}{0.739656in}}%
\pgfpathlineto{\pgfqpoint{3.495144in}{0.739656in}}%
\pgfpathlineto{\pgfqpoint{3.494847in}{0.739656in}}%
\pgfpathlineto{\pgfqpoint{3.494549in}{0.739656in}}%
\pgfpathlineto{\pgfqpoint{3.494252in}{0.739656in}}%
\pgfpathlineto{\pgfqpoint{3.493954in}{0.739656in}}%
\pgfpathlineto{\pgfqpoint{3.493657in}{0.739656in}}%
\pgfpathlineto{\pgfqpoint{3.493359in}{0.739656in}}%
\pgfpathlineto{\pgfqpoint{3.493062in}{0.739656in}}%
\pgfpathlineto{\pgfqpoint{3.492764in}{0.739656in}}%
\pgfpathlineto{\pgfqpoint{3.492467in}{0.739656in}}%
\pgfpathlineto{\pgfqpoint{3.492169in}{0.739656in}}%
\pgfpathlineto{\pgfqpoint{3.491872in}{0.739656in}}%
\pgfpathlineto{\pgfqpoint{3.491574in}{0.739656in}}%
\pgfpathlineto{\pgfqpoint{3.491277in}{0.739656in}}%
\pgfpathlineto{\pgfqpoint{3.490979in}{0.739656in}}%
\pgfpathlineto{\pgfqpoint{3.490682in}{0.739656in}}%
\pgfpathlineto{\pgfqpoint{3.490384in}{0.739656in}}%
\pgfpathlineto{\pgfqpoint{3.490087in}{0.739656in}}%
\pgfpathlineto{\pgfqpoint{3.489790in}{0.739656in}}%
\pgfpathlineto{\pgfqpoint{3.489492in}{0.739656in}}%
\pgfpathlineto{\pgfqpoint{3.489195in}{0.739656in}}%
\pgfpathlineto{\pgfqpoint{3.488897in}{0.739656in}}%
\pgfpathlineto{\pgfqpoint{3.488600in}{0.739656in}}%
\pgfpathlineto{\pgfqpoint{3.488302in}{0.739656in}}%
\pgfpathlineto{\pgfqpoint{3.488005in}{0.739656in}}%
\pgfpathlineto{\pgfqpoint{3.487707in}{0.739656in}}%
\pgfpathlineto{\pgfqpoint{3.487410in}{0.739656in}}%
\pgfpathlineto{\pgfqpoint{3.487112in}{0.739656in}}%
\pgfpathlineto{\pgfqpoint{3.486815in}{0.739656in}}%
\pgfpathlineto{\pgfqpoint{3.486517in}{0.739656in}}%
\pgfpathlineto{\pgfqpoint{3.486220in}{0.739656in}}%
\pgfpathlineto{\pgfqpoint{3.485922in}{0.739656in}}%
\pgfpathlineto{\pgfqpoint{3.485625in}{0.739656in}}%
\pgfpathlineto{\pgfqpoint{3.485327in}{0.739656in}}%
\pgfpathlineto{\pgfqpoint{3.485030in}{0.739656in}}%
\pgfpathlineto{\pgfqpoint{3.484732in}{0.739656in}}%
\pgfpathlineto{\pgfqpoint{3.484435in}{0.739656in}}%
\pgfpathlineto{\pgfqpoint{3.484137in}{0.739656in}}%
\pgfpathlineto{\pgfqpoint{3.483840in}{0.739656in}}%
\pgfpathlineto{\pgfqpoint{3.483542in}{0.739656in}}%
\pgfpathlineto{\pgfqpoint{3.483245in}{0.739656in}}%
\pgfpathlineto{\pgfqpoint{3.482948in}{0.739656in}}%
\pgfpathlineto{\pgfqpoint{3.482650in}{0.739656in}}%
\pgfpathlineto{\pgfqpoint{3.482353in}{0.739656in}}%
\pgfpathlineto{\pgfqpoint{3.482055in}{0.739656in}}%
\pgfpathlineto{\pgfqpoint{3.481758in}{0.739656in}}%
\pgfpathlineto{\pgfqpoint{3.481460in}{0.739656in}}%
\pgfpathlineto{\pgfqpoint{3.481163in}{0.739656in}}%
\pgfpathlineto{\pgfqpoint{3.480865in}{0.739656in}}%
\pgfpathlineto{\pgfqpoint{3.480568in}{0.739656in}}%
\pgfpathlineto{\pgfqpoint{3.480270in}{0.739656in}}%
\pgfpathlineto{\pgfqpoint{3.479973in}{0.739656in}}%
\pgfpathlineto{\pgfqpoint{3.479675in}{0.739656in}}%
\pgfpathlineto{\pgfqpoint{3.479378in}{0.739656in}}%
\pgfpathlineto{\pgfqpoint{3.479080in}{0.739656in}}%
\pgfpathlineto{\pgfqpoint{3.478783in}{0.739656in}}%
\pgfpathlineto{\pgfqpoint{3.478485in}{0.739656in}}%
\pgfpathlineto{\pgfqpoint{3.478188in}{0.739656in}}%
\pgfpathlineto{\pgfqpoint{3.477890in}{0.739656in}}%
\pgfpathlineto{\pgfqpoint{3.477593in}{0.739656in}}%
\pgfpathlineto{\pgfqpoint{3.477295in}{0.739656in}}%
\pgfpathlineto{\pgfqpoint{3.476998in}{0.739656in}}%
\pgfpathlineto{\pgfqpoint{3.476701in}{0.739656in}}%
\pgfpathlineto{\pgfqpoint{3.476403in}{0.739656in}}%
\pgfpathlineto{\pgfqpoint{3.476106in}{0.739656in}}%
\pgfpathlineto{\pgfqpoint{3.475808in}{0.739656in}}%
\pgfpathlineto{\pgfqpoint{3.475511in}{0.739656in}}%
\pgfpathlineto{\pgfqpoint{3.475213in}{0.739656in}}%
\pgfpathlineto{\pgfqpoint{3.474916in}{0.739656in}}%
\pgfpathlineto{\pgfqpoint{3.474618in}{0.739656in}}%
\pgfpathlineto{\pgfqpoint{3.474321in}{0.739656in}}%
\pgfpathlineto{\pgfqpoint{3.474023in}{0.739656in}}%
\pgfpathlineto{\pgfqpoint{3.473726in}{0.739656in}}%
\pgfpathlineto{\pgfqpoint{3.473428in}{0.739656in}}%
\pgfpathlineto{\pgfqpoint{3.473131in}{0.739656in}}%
\pgfpathlineto{\pgfqpoint{3.472833in}{0.739656in}}%
\pgfpathlineto{\pgfqpoint{3.472536in}{0.739656in}}%
\pgfpathlineto{\pgfqpoint{3.472238in}{0.739656in}}%
\pgfpathlineto{\pgfqpoint{3.471941in}{0.739656in}}%
\pgfpathlineto{\pgfqpoint{3.471643in}{0.739656in}}%
\pgfpathlineto{\pgfqpoint{3.471346in}{0.739656in}}%
\pgfpathlineto{\pgfqpoint{3.471048in}{0.739656in}}%
\pgfpathlineto{\pgfqpoint{3.470751in}{0.739656in}}%
\pgfpathlineto{\pgfqpoint{3.470453in}{0.739656in}}%
\pgfpathlineto{\pgfqpoint{3.470156in}{0.739656in}}%
\pgfpathlineto{\pgfqpoint{3.469859in}{0.739656in}}%
\pgfpathlineto{\pgfqpoint{3.469561in}{0.739656in}}%
\pgfpathlineto{\pgfqpoint{3.469264in}{0.739656in}}%
\pgfpathlineto{\pgfqpoint{3.468966in}{0.739656in}}%
\pgfpathlineto{\pgfqpoint{3.468669in}{0.739656in}}%
\pgfpathlineto{\pgfqpoint{3.468371in}{0.739656in}}%
\pgfpathlineto{\pgfqpoint{3.468074in}{0.739656in}}%
\pgfpathlineto{\pgfqpoint{3.467776in}{0.739656in}}%
\pgfpathlineto{\pgfqpoint{3.467479in}{0.739656in}}%
\pgfpathlineto{\pgfqpoint{3.467181in}{0.739656in}}%
\pgfpathlineto{\pgfqpoint{3.466884in}{0.739656in}}%
\pgfpathlineto{\pgfqpoint{3.466586in}{0.739656in}}%
\pgfpathlineto{\pgfqpoint{3.466289in}{0.739656in}}%
\pgfpathlineto{\pgfqpoint{3.465991in}{0.739656in}}%
\pgfpathlineto{\pgfqpoint{3.465694in}{0.739656in}}%
\pgfpathlineto{\pgfqpoint{3.465396in}{0.739656in}}%
\pgfpathlineto{\pgfqpoint{3.465099in}{0.739656in}}%
\pgfpathlineto{\pgfqpoint{3.464801in}{0.739656in}}%
\pgfpathlineto{\pgfqpoint{3.464504in}{0.739656in}}%
\pgfpathlineto{\pgfqpoint{3.464206in}{0.739656in}}%
\pgfpathlineto{\pgfqpoint{3.463909in}{0.739656in}}%
\pgfpathlineto{\pgfqpoint{3.463611in}{0.739656in}}%
\pgfpathlineto{\pgfqpoint{3.463314in}{0.739656in}}%
\pgfpathlineto{\pgfqpoint{3.463017in}{0.739656in}}%
\pgfpathlineto{\pgfqpoint{3.462719in}{0.739656in}}%
\pgfpathlineto{\pgfqpoint{3.462422in}{0.739656in}}%
\pgfpathlineto{\pgfqpoint{3.462124in}{0.739656in}}%
\pgfpathlineto{\pgfqpoint{3.461827in}{0.739656in}}%
\pgfpathlineto{\pgfqpoint{3.461529in}{0.739656in}}%
\pgfpathlineto{\pgfqpoint{3.461232in}{0.739656in}}%
\pgfpathlineto{\pgfqpoint{3.460934in}{0.739656in}}%
\pgfpathlineto{\pgfqpoint{3.460637in}{0.739656in}}%
\pgfpathlineto{\pgfqpoint{3.460339in}{0.739656in}}%
\pgfpathlineto{\pgfqpoint{3.460042in}{0.739656in}}%
\pgfpathlineto{\pgfqpoint{3.459744in}{0.739656in}}%
\pgfpathlineto{\pgfqpoint{3.459447in}{0.739656in}}%
\pgfpathlineto{\pgfqpoint{3.459149in}{0.739656in}}%
\pgfpathlineto{\pgfqpoint{3.458852in}{0.739656in}}%
\pgfpathlineto{\pgfqpoint{3.458554in}{0.739656in}}%
\pgfpathlineto{\pgfqpoint{3.458257in}{0.739656in}}%
\pgfpathlineto{\pgfqpoint{3.457959in}{0.739656in}}%
\pgfpathlineto{\pgfqpoint{3.457662in}{0.739656in}}%
\pgfpathlineto{\pgfqpoint{3.457364in}{0.739656in}}%
\pgfpathlineto{\pgfqpoint{3.457067in}{0.739656in}}%
\pgfpathlineto{\pgfqpoint{3.456770in}{0.739656in}}%
\pgfpathlineto{\pgfqpoint{3.456472in}{0.739656in}}%
\pgfpathlineto{\pgfqpoint{3.456175in}{0.739656in}}%
\pgfpathlineto{\pgfqpoint{3.455877in}{0.739656in}}%
\pgfpathlineto{\pgfqpoint{3.455580in}{0.739656in}}%
\pgfpathlineto{\pgfqpoint{3.455282in}{0.739656in}}%
\pgfpathlineto{\pgfqpoint{3.454985in}{0.739656in}}%
\pgfpathlineto{\pgfqpoint{3.454687in}{0.739656in}}%
\pgfpathlineto{\pgfqpoint{3.454390in}{0.739656in}}%
\pgfpathlineto{\pgfqpoint{3.454092in}{0.739656in}}%
\pgfpathlineto{\pgfqpoint{3.453795in}{0.739656in}}%
\pgfpathlineto{\pgfqpoint{3.453497in}{0.739656in}}%
\pgfpathlineto{\pgfqpoint{3.453200in}{0.739656in}}%
\pgfpathlineto{\pgfqpoint{3.452902in}{0.739656in}}%
\pgfpathlineto{\pgfqpoint{3.452605in}{0.739656in}}%
\pgfpathlineto{\pgfqpoint{3.452307in}{0.739656in}}%
\pgfpathlineto{\pgfqpoint{3.452010in}{0.739656in}}%
\pgfpathlineto{\pgfqpoint{3.451712in}{0.739656in}}%
\pgfpathlineto{\pgfqpoint{3.451415in}{0.739656in}}%
\pgfpathlineto{\pgfqpoint{3.451117in}{0.739656in}}%
\pgfpathlineto{\pgfqpoint{3.450820in}{0.739656in}}%
\pgfpathlineto{\pgfqpoint{3.450522in}{0.739656in}}%
\pgfpathlineto{\pgfqpoint{3.450225in}{0.739656in}}%
\pgfpathlineto{\pgfqpoint{3.449928in}{0.739656in}}%
\pgfpathlineto{\pgfqpoint{3.449630in}{0.739656in}}%
\pgfpathlineto{\pgfqpoint{3.449333in}{0.739656in}}%
\pgfpathlineto{\pgfqpoint{3.449035in}{0.739656in}}%
\pgfpathlineto{\pgfqpoint{3.448738in}{0.739656in}}%
\pgfpathlineto{\pgfqpoint{3.448440in}{0.739656in}}%
\pgfpathlineto{\pgfqpoint{3.448143in}{0.739656in}}%
\pgfpathlineto{\pgfqpoint{3.447845in}{0.739656in}}%
\pgfpathlineto{\pgfqpoint{3.447548in}{0.739656in}}%
\pgfpathlineto{\pgfqpoint{3.447250in}{0.739656in}}%
\pgfpathlineto{\pgfqpoint{3.446953in}{0.739656in}}%
\pgfpathlineto{\pgfqpoint{3.446655in}{0.739656in}}%
\pgfpathlineto{\pgfqpoint{3.446358in}{0.739656in}}%
\pgfpathlineto{\pgfqpoint{3.446060in}{0.739656in}}%
\pgfpathlineto{\pgfqpoint{3.445763in}{0.739656in}}%
\pgfpathlineto{\pgfqpoint{3.445465in}{0.739656in}}%
\pgfpathlineto{\pgfqpoint{3.445168in}{0.739656in}}%
\pgfpathlineto{\pgfqpoint{3.444870in}{0.739656in}}%
\pgfpathlineto{\pgfqpoint{3.444573in}{0.739656in}}%
\pgfpathlineto{\pgfqpoint{3.444275in}{0.739656in}}%
\pgfpathlineto{\pgfqpoint{3.443978in}{0.739656in}}%
\pgfpathlineto{\pgfqpoint{3.443680in}{0.739656in}}%
\pgfpathlineto{\pgfqpoint{3.443383in}{0.739656in}}%
\pgfpathlineto{\pgfqpoint{3.443086in}{0.739656in}}%
\pgfpathlineto{\pgfqpoint{3.442788in}{0.739656in}}%
\pgfpathlineto{\pgfqpoint{3.442491in}{0.739656in}}%
\pgfpathlineto{\pgfqpoint{3.442193in}{0.739656in}}%
\pgfpathlineto{\pgfqpoint{3.441896in}{0.739656in}}%
\pgfpathlineto{\pgfqpoint{3.441598in}{0.739656in}}%
\pgfpathlineto{\pgfqpoint{3.441301in}{0.739656in}}%
\pgfpathlineto{\pgfqpoint{3.441003in}{0.739656in}}%
\pgfpathlineto{\pgfqpoint{3.440706in}{0.739656in}}%
\pgfpathlineto{\pgfqpoint{3.440408in}{0.739656in}}%
\pgfpathlineto{\pgfqpoint{3.440111in}{0.739656in}}%
\pgfpathlineto{\pgfqpoint{3.439813in}{0.739656in}}%
\pgfpathlineto{\pgfqpoint{3.439516in}{0.739656in}}%
\pgfpathlineto{\pgfqpoint{3.439218in}{0.739656in}}%
\pgfpathlineto{\pgfqpoint{3.438921in}{0.739656in}}%
\pgfpathlineto{\pgfqpoint{3.438623in}{0.739656in}}%
\pgfpathlineto{\pgfqpoint{3.438326in}{0.739656in}}%
\pgfpathlineto{\pgfqpoint{3.438028in}{0.739656in}}%
\pgfpathlineto{\pgfqpoint{3.437731in}{0.739656in}}%
\pgfpathlineto{\pgfqpoint{3.437433in}{0.739656in}}%
\pgfpathlineto{\pgfqpoint{3.437136in}{0.739656in}}%
\pgfpathlineto{\pgfqpoint{3.436839in}{0.739656in}}%
\pgfpathlineto{\pgfqpoint{3.436541in}{0.739656in}}%
\pgfpathlineto{\pgfqpoint{3.436244in}{0.739656in}}%
\pgfpathlineto{\pgfqpoint{3.435946in}{0.739656in}}%
\pgfpathlineto{\pgfqpoint{3.435649in}{0.739656in}}%
\pgfpathlineto{\pgfqpoint{3.435351in}{0.739656in}}%
\pgfpathlineto{\pgfqpoint{3.435054in}{0.739656in}}%
\pgfpathlineto{\pgfqpoint{3.434756in}{0.739656in}}%
\pgfpathlineto{\pgfqpoint{3.434459in}{0.739656in}}%
\pgfpathlineto{\pgfqpoint{3.434161in}{0.739656in}}%
\pgfpathlineto{\pgfqpoint{3.433864in}{0.739656in}}%
\pgfpathlineto{\pgfqpoint{3.433566in}{0.739656in}}%
\pgfpathlineto{\pgfqpoint{3.433269in}{0.739656in}}%
\pgfpathlineto{\pgfqpoint{3.432971in}{0.739656in}}%
\pgfpathlineto{\pgfqpoint{3.432674in}{0.739656in}}%
\pgfpathlineto{\pgfqpoint{3.432376in}{0.739656in}}%
\pgfpathlineto{\pgfqpoint{3.432079in}{0.739656in}}%
\pgfpathlineto{\pgfqpoint{3.431781in}{0.739656in}}%
\pgfpathlineto{\pgfqpoint{3.431484in}{0.739656in}}%
\pgfpathlineto{\pgfqpoint{3.431186in}{0.739656in}}%
\pgfpathlineto{\pgfqpoint{3.430889in}{0.739656in}}%
\pgfpathlineto{\pgfqpoint{3.430591in}{0.739656in}}%
\pgfpathlineto{\pgfqpoint{3.430294in}{0.739656in}}%
\pgfpathlineto{\pgfqpoint{3.429997in}{0.739656in}}%
\pgfpathlineto{\pgfqpoint{3.429699in}{0.739656in}}%
\pgfpathlineto{\pgfqpoint{3.429402in}{0.739656in}}%
\pgfpathlineto{\pgfqpoint{3.429104in}{0.739656in}}%
\pgfpathlineto{\pgfqpoint{3.428807in}{0.739656in}}%
\pgfpathlineto{\pgfqpoint{3.428509in}{0.739656in}}%
\pgfpathlineto{\pgfqpoint{3.428212in}{0.739656in}}%
\pgfpathlineto{\pgfqpoint{3.427914in}{0.739656in}}%
\pgfpathlineto{\pgfqpoint{3.427617in}{0.739656in}}%
\pgfpathlineto{\pgfqpoint{3.427319in}{0.739656in}}%
\pgfpathlineto{\pgfqpoint{3.427022in}{0.739656in}}%
\pgfpathlineto{\pgfqpoint{3.426724in}{0.739656in}}%
\pgfpathlineto{\pgfqpoint{3.426427in}{0.739656in}}%
\pgfpathlineto{\pgfqpoint{3.426129in}{0.739656in}}%
\pgfpathlineto{\pgfqpoint{3.425832in}{0.739656in}}%
\pgfpathlineto{\pgfqpoint{3.425534in}{0.739656in}}%
\pgfpathlineto{\pgfqpoint{3.425237in}{0.739656in}}%
\pgfpathlineto{\pgfqpoint{3.424939in}{0.739656in}}%
\pgfpathlineto{\pgfqpoint{3.424642in}{0.739656in}}%
\pgfpathlineto{\pgfqpoint{3.424344in}{0.739656in}}%
\pgfpathlineto{\pgfqpoint{3.424047in}{0.739656in}}%
\pgfpathlineto{\pgfqpoint{3.423749in}{0.739656in}}%
\pgfpathlineto{\pgfqpoint{3.423452in}{0.739656in}}%
\pgfpathlineto{\pgfqpoint{3.423155in}{0.739656in}}%
\pgfpathlineto{\pgfqpoint{3.422857in}{0.739656in}}%
\pgfpathlineto{\pgfqpoint{3.422560in}{0.739656in}}%
\pgfpathlineto{\pgfqpoint{3.422262in}{0.739656in}}%
\pgfpathlineto{\pgfqpoint{3.421965in}{0.739656in}}%
\pgfpathlineto{\pgfqpoint{3.421667in}{0.739656in}}%
\pgfpathlineto{\pgfqpoint{3.421370in}{0.739656in}}%
\pgfpathlineto{\pgfqpoint{3.421072in}{0.739656in}}%
\pgfpathlineto{\pgfqpoint{3.420775in}{0.739656in}}%
\pgfpathlineto{\pgfqpoint{3.420477in}{0.739656in}}%
\pgfpathlineto{\pgfqpoint{3.420180in}{0.739656in}}%
\pgfpathlineto{\pgfqpoint{3.419882in}{0.739656in}}%
\pgfpathlineto{\pgfqpoint{3.419585in}{0.739656in}}%
\pgfpathlineto{\pgfqpoint{3.419287in}{0.739656in}}%
\pgfpathlineto{\pgfqpoint{3.418990in}{0.739656in}}%
\pgfpathlineto{\pgfqpoint{3.418692in}{0.739656in}}%
\pgfpathlineto{\pgfqpoint{3.418395in}{0.739656in}}%
\pgfpathlineto{\pgfqpoint{3.418097in}{0.739656in}}%
\pgfpathlineto{\pgfqpoint{3.417800in}{0.739656in}}%
\pgfpathlineto{\pgfqpoint{3.417502in}{0.739656in}}%
\pgfpathlineto{\pgfqpoint{3.417205in}{0.739656in}}%
\pgfpathlineto{\pgfqpoint{3.416908in}{0.739656in}}%
\pgfpathlineto{\pgfqpoint{3.416610in}{0.739656in}}%
\pgfpathlineto{\pgfqpoint{3.416313in}{0.739656in}}%
\pgfpathlineto{\pgfqpoint{3.416015in}{0.739656in}}%
\pgfpathlineto{\pgfqpoint{3.415718in}{0.739656in}}%
\pgfpathlineto{\pgfqpoint{3.415420in}{0.739656in}}%
\pgfpathlineto{\pgfqpoint{3.415123in}{0.739656in}}%
\pgfpathlineto{\pgfqpoint{3.414825in}{0.739656in}}%
\pgfpathlineto{\pgfqpoint{3.414528in}{0.739656in}}%
\pgfpathlineto{\pgfqpoint{3.414230in}{0.739656in}}%
\pgfpathlineto{\pgfqpoint{3.413933in}{0.739656in}}%
\pgfpathlineto{\pgfqpoint{3.413635in}{0.739656in}}%
\pgfpathlineto{\pgfqpoint{3.413338in}{0.739656in}}%
\pgfpathlineto{\pgfqpoint{3.413040in}{0.739656in}}%
\pgfpathlineto{\pgfqpoint{3.412743in}{0.739656in}}%
\pgfpathlineto{\pgfqpoint{3.412445in}{0.739656in}}%
\pgfpathlineto{\pgfqpoint{3.412148in}{0.739656in}}%
\pgfpathlineto{\pgfqpoint{3.411850in}{0.739656in}}%
\pgfpathlineto{\pgfqpoint{3.411553in}{0.739656in}}%
\pgfpathlineto{\pgfqpoint{3.411255in}{0.739656in}}%
\pgfpathlineto{\pgfqpoint{3.410958in}{0.739656in}}%
\pgfpathlineto{\pgfqpoint{3.410660in}{0.739656in}}%
\pgfpathlineto{\pgfqpoint{3.410363in}{0.739656in}}%
\pgfpathlineto{\pgfqpoint{3.410066in}{0.739656in}}%
\pgfpathlineto{\pgfqpoint{3.409768in}{0.739656in}}%
\pgfpathlineto{\pgfqpoint{3.409471in}{0.739656in}}%
\pgfpathlineto{\pgfqpoint{3.409173in}{0.739656in}}%
\pgfpathlineto{\pgfqpoint{3.408876in}{0.739656in}}%
\pgfpathlineto{\pgfqpoint{3.408578in}{0.739656in}}%
\pgfpathlineto{\pgfqpoint{3.408281in}{0.739656in}}%
\pgfpathlineto{\pgfqpoint{3.407983in}{0.739656in}}%
\pgfpathlineto{\pgfqpoint{3.407686in}{0.739656in}}%
\pgfpathlineto{\pgfqpoint{3.407388in}{0.739656in}}%
\pgfpathlineto{\pgfqpoint{3.407091in}{0.739656in}}%
\pgfpathlineto{\pgfqpoint{3.406793in}{0.739656in}}%
\pgfpathlineto{\pgfqpoint{3.406496in}{0.739656in}}%
\pgfpathlineto{\pgfqpoint{3.406198in}{0.739656in}}%
\pgfpathlineto{\pgfqpoint{3.405901in}{0.739656in}}%
\pgfpathlineto{\pgfqpoint{3.405603in}{0.739656in}}%
\pgfpathlineto{\pgfqpoint{3.405306in}{0.739656in}}%
\pgfpathlineto{\pgfqpoint{3.405008in}{0.739656in}}%
\pgfpathlineto{\pgfqpoint{3.404711in}{0.739656in}}%
\pgfpathlineto{\pgfqpoint{3.404413in}{0.739656in}}%
\pgfpathlineto{\pgfqpoint{3.404116in}{0.739656in}}%
\pgfpathlineto{\pgfqpoint{3.403818in}{0.739656in}}%
\pgfpathlineto{\pgfqpoint{3.403521in}{0.739656in}}%
\pgfpathlineto{\pgfqpoint{3.403224in}{0.739656in}}%
\pgfpathlineto{\pgfqpoint{3.402926in}{0.739656in}}%
\pgfpathlineto{\pgfqpoint{3.402629in}{0.739656in}}%
\pgfpathlineto{\pgfqpoint{3.402331in}{0.739656in}}%
\pgfpathlineto{\pgfqpoint{3.402034in}{0.739656in}}%
\pgfpathlineto{\pgfqpoint{3.401736in}{0.739656in}}%
\pgfpathlineto{\pgfqpoint{3.401439in}{0.739656in}}%
\pgfpathlineto{\pgfqpoint{3.401141in}{0.739656in}}%
\pgfpathlineto{\pgfqpoint{3.400844in}{0.739656in}}%
\pgfpathlineto{\pgfqpoint{3.400546in}{0.739656in}}%
\pgfpathlineto{\pgfqpoint{3.400249in}{0.739656in}}%
\pgfpathlineto{\pgfqpoint{3.399951in}{0.739656in}}%
\pgfpathlineto{\pgfqpoint{3.399654in}{0.739656in}}%
\pgfpathlineto{\pgfqpoint{3.399356in}{0.739656in}}%
\pgfpathlineto{\pgfqpoint{3.399059in}{0.739656in}}%
\pgfpathlineto{\pgfqpoint{3.398761in}{0.739656in}}%
\pgfpathlineto{\pgfqpoint{3.398464in}{0.739656in}}%
\pgfpathlineto{\pgfqpoint{3.398166in}{0.739656in}}%
\pgfpathlineto{\pgfqpoint{3.397869in}{0.739656in}}%
\pgfpathlineto{\pgfqpoint{3.397571in}{0.739656in}}%
\pgfpathlineto{\pgfqpoint{3.397274in}{0.739656in}}%
\pgfpathlineto{\pgfqpoint{3.396977in}{0.739656in}}%
\pgfpathlineto{\pgfqpoint{3.396679in}{0.739656in}}%
\pgfpathlineto{\pgfqpoint{3.396382in}{0.739656in}}%
\pgfpathlineto{\pgfqpoint{3.396084in}{0.739656in}}%
\pgfpathlineto{\pgfqpoint{3.395787in}{0.739656in}}%
\pgfpathlineto{\pgfqpoint{3.395489in}{0.739656in}}%
\pgfpathlineto{\pgfqpoint{3.395192in}{0.739656in}}%
\pgfpathlineto{\pgfqpoint{3.394894in}{0.739656in}}%
\pgfpathlineto{\pgfqpoint{3.394597in}{0.739656in}}%
\pgfpathlineto{\pgfqpoint{3.394299in}{0.739656in}}%
\pgfpathlineto{\pgfqpoint{3.394002in}{0.739656in}}%
\pgfpathlineto{\pgfqpoint{3.393704in}{0.739656in}}%
\pgfpathlineto{\pgfqpoint{3.393407in}{0.739656in}}%
\pgfpathlineto{\pgfqpoint{3.393109in}{0.739656in}}%
\pgfpathlineto{\pgfqpoint{3.392812in}{0.739656in}}%
\pgfpathlineto{\pgfqpoint{3.392514in}{0.739656in}}%
\pgfpathlineto{\pgfqpoint{3.392217in}{0.739656in}}%
\pgfpathlineto{\pgfqpoint{3.391919in}{0.739656in}}%
\pgfpathlineto{\pgfqpoint{3.391622in}{0.739656in}}%
\pgfpathlineto{\pgfqpoint{3.391324in}{0.739656in}}%
\pgfpathlineto{\pgfqpoint{3.391027in}{0.739656in}}%
\pgfpathlineto{\pgfqpoint{3.390729in}{0.739656in}}%
\pgfpathlineto{\pgfqpoint{3.390432in}{0.739656in}}%
\pgfpathlineto{\pgfqpoint{3.390135in}{0.739656in}}%
\pgfpathlineto{\pgfqpoint{3.389837in}{0.739656in}}%
\pgfpathlineto{\pgfqpoint{3.389540in}{0.739656in}}%
\pgfpathlineto{\pgfqpoint{3.389242in}{0.739656in}}%
\pgfpathlineto{\pgfqpoint{3.388945in}{0.739656in}}%
\pgfpathlineto{\pgfqpoint{3.388647in}{0.739656in}}%
\pgfpathlineto{\pgfqpoint{3.388350in}{0.739656in}}%
\pgfpathlineto{\pgfqpoint{3.388052in}{0.739656in}}%
\pgfpathlineto{\pgfqpoint{3.387755in}{0.739656in}}%
\pgfpathlineto{\pgfqpoint{3.387457in}{0.739656in}}%
\pgfpathlineto{\pgfqpoint{3.387160in}{0.739656in}}%
\pgfpathlineto{\pgfqpoint{3.386862in}{0.739656in}}%
\pgfpathlineto{\pgfqpoint{3.386565in}{0.739656in}}%
\pgfpathlineto{\pgfqpoint{3.386267in}{0.739656in}}%
\pgfpathlineto{\pgfqpoint{3.385970in}{0.739656in}}%
\pgfpathlineto{\pgfqpoint{3.385672in}{0.739656in}}%
\pgfpathlineto{\pgfqpoint{3.385375in}{0.739656in}}%
\pgfpathlineto{\pgfqpoint{3.385077in}{0.739656in}}%
\pgfpathlineto{\pgfqpoint{3.384780in}{0.739656in}}%
\pgfpathlineto{\pgfqpoint{3.384482in}{0.739656in}}%
\pgfpathlineto{\pgfqpoint{3.384185in}{0.739656in}}%
\pgfpathlineto{\pgfqpoint{3.383887in}{0.739656in}}%
\pgfpathlineto{\pgfqpoint{3.383590in}{0.739656in}}%
\pgfpathlineto{\pgfqpoint{3.383293in}{0.739656in}}%
\pgfpathlineto{\pgfqpoint{3.382995in}{0.739656in}}%
\pgfpathlineto{\pgfqpoint{3.382698in}{0.739656in}}%
\pgfpathlineto{\pgfqpoint{3.382400in}{0.739656in}}%
\pgfpathlineto{\pgfqpoint{3.382103in}{0.739656in}}%
\pgfpathlineto{\pgfqpoint{3.381805in}{0.739656in}}%
\pgfpathlineto{\pgfqpoint{3.381508in}{0.739656in}}%
\pgfpathlineto{\pgfqpoint{3.381210in}{0.739656in}}%
\pgfpathlineto{\pgfqpoint{3.380913in}{0.739656in}}%
\pgfpathlineto{\pgfqpoint{3.380615in}{0.739656in}}%
\pgfpathlineto{\pgfqpoint{3.380318in}{0.739656in}}%
\pgfpathlineto{\pgfqpoint{3.380020in}{0.739656in}}%
\pgfpathlineto{\pgfqpoint{3.379723in}{0.739656in}}%
\pgfpathlineto{\pgfqpoint{3.379425in}{0.739656in}}%
\pgfpathlineto{\pgfqpoint{3.379128in}{0.739656in}}%
\pgfpathlineto{\pgfqpoint{3.378830in}{0.739656in}}%
\pgfpathlineto{\pgfqpoint{3.378533in}{0.739656in}}%
\pgfpathlineto{\pgfqpoint{3.378235in}{0.739656in}}%
\pgfpathlineto{\pgfqpoint{3.377938in}{0.739656in}}%
\pgfpathlineto{\pgfqpoint{3.377640in}{0.739656in}}%
\pgfpathlineto{\pgfqpoint{3.377343in}{0.739656in}}%
\pgfpathlineto{\pgfqpoint{3.377046in}{0.739656in}}%
\pgfpathlineto{\pgfqpoint{3.376748in}{0.739656in}}%
\pgfpathlineto{\pgfqpoint{3.376451in}{0.739656in}}%
\pgfpathlineto{\pgfqpoint{3.376153in}{0.739656in}}%
\pgfpathlineto{\pgfqpoint{3.375856in}{0.739656in}}%
\pgfpathlineto{\pgfqpoint{3.375558in}{0.739656in}}%
\pgfpathlineto{\pgfqpoint{3.375261in}{0.739656in}}%
\pgfpathlineto{\pgfqpoint{3.374963in}{0.739656in}}%
\pgfpathlineto{\pgfqpoint{3.374666in}{0.739656in}}%
\pgfpathlineto{\pgfqpoint{3.374368in}{0.739656in}}%
\pgfpathlineto{\pgfqpoint{3.374071in}{0.739656in}}%
\pgfpathlineto{\pgfqpoint{3.373773in}{0.739656in}}%
\pgfpathlineto{\pgfqpoint{3.373476in}{0.739656in}}%
\pgfpathlineto{\pgfqpoint{3.373178in}{0.739656in}}%
\pgfpathlineto{\pgfqpoint{3.372881in}{0.739656in}}%
\pgfpathlineto{\pgfqpoint{3.372583in}{0.739656in}}%
\pgfpathlineto{\pgfqpoint{3.372286in}{0.739656in}}%
\pgfpathlineto{\pgfqpoint{3.371988in}{0.739656in}}%
\pgfpathlineto{\pgfqpoint{3.371691in}{0.739656in}}%
\pgfpathlineto{\pgfqpoint{3.371393in}{0.739656in}}%
\pgfpathlineto{\pgfqpoint{3.371096in}{0.739656in}}%
\pgfpathlineto{\pgfqpoint{3.370798in}{0.739656in}}%
\pgfpathlineto{\pgfqpoint{3.370501in}{0.739656in}}%
\pgfpathlineto{\pgfqpoint{3.370204in}{0.739656in}}%
\pgfpathlineto{\pgfqpoint{3.369906in}{0.739656in}}%
\pgfpathlineto{\pgfqpoint{3.369609in}{0.739656in}}%
\pgfpathlineto{\pgfqpoint{3.369311in}{0.739656in}}%
\pgfpathlineto{\pgfqpoint{3.369014in}{0.739656in}}%
\pgfpathlineto{\pgfqpoint{3.368716in}{0.739656in}}%
\pgfpathlineto{\pgfqpoint{3.368419in}{0.739656in}}%
\pgfpathlineto{\pgfqpoint{3.368121in}{0.739656in}}%
\pgfpathlineto{\pgfqpoint{3.367824in}{0.739656in}}%
\pgfpathlineto{\pgfqpoint{3.367526in}{0.739656in}}%
\pgfpathlineto{\pgfqpoint{3.367229in}{0.739656in}}%
\pgfpathlineto{\pgfqpoint{3.366931in}{0.739656in}}%
\pgfpathlineto{\pgfqpoint{3.366634in}{0.739656in}}%
\pgfpathlineto{\pgfqpoint{3.366336in}{0.739656in}}%
\pgfpathlineto{\pgfqpoint{3.366039in}{0.739656in}}%
\pgfpathlineto{\pgfqpoint{3.365741in}{0.739656in}}%
\pgfpathlineto{\pgfqpoint{3.365444in}{0.739656in}}%
\pgfpathlineto{\pgfqpoint{3.365146in}{0.739656in}}%
\pgfpathlineto{\pgfqpoint{3.364849in}{0.739656in}}%
\pgfpathlineto{\pgfqpoint{3.364551in}{0.739656in}}%
\pgfpathlineto{\pgfqpoint{3.364254in}{0.739656in}}%
\pgfpathlineto{\pgfqpoint{3.363956in}{0.739656in}}%
\pgfpathlineto{\pgfqpoint{3.363659in}{0.739656in}}%
\pgfpathlineto{\pgfqpoint{3.363362in}{0.739656in}}%
\pgfpathlineto{\pgfqpoint{3.363064in}{0.739656in}}%
\pgfpathlineto{\pgfqpoint{3.362767in}{0.739656in}}%
\pgfpathlineto{\pgfqpoint{3.362469in}{0.739656in}}%
\pgfpathlineto{\pgfqpoint{3.362172in}{0.739656in}}%
\pgfpathlineto{\pgfqpoint{3.361874in}{0.739656in}}%
\pgfpathlineto{\pgfqpoint{3.361577in}{0.739656in}}%
\pgfpathlineto{\pgfqpoint{3.361279in}{0.739656in}}%
\pgfpathlineto{\pgfqpoint{3.360982in}{0.739656in}}%
\pgfpathlineto{\pgfqpoint{3.360684in}{0.739656in}}%
\pgfpathlineto{\pgfqpoint{3.360387in}{0.739656in}}%
\pgfpathlineto{\pgfqpoint{3.360089in}{0.739656in}}%
\pgfpathlineto{\pgfqpoint{3.359792in}{0.739656in}}%
\pgfpathlineto{\pgfqpoint{3.359494in}{0.739656in}}%
\pgfpathlineto{\pgfqpoint{3.359197in}{0.739656in}}%
\pgfpathlineto{\pgfqpoint{3.358899in}{0.739656in}}%
\pgfpathlineto{\pgfqpoint{3.358602in}{0.739656in}}%
\pgfpathlineto{\pgfqpoint{3.358304in}{0.739656in}}%
\pgfpathlineto{\pgfqpoint{3.358007in}{0.739656in}}%
\pgfpathlineto{\pgfqpoint{3.357709in}{0.739656in}}%
\pgfpathlineto{\pgfqpoint{3.357412in}{0.739656in}}%
\pgfpathlineto{\pgfqpoint{3.357115in}{0.739656in}}%
\pgfpathlineto{\pgfqpoint{3.356817in}{0.739656in}}%
\pgfpathlineto{\pgfqpoint{3.356520in}{0.739656in}}%
\pgfpathlineto{\pgfqpoint{3.356222in}{0.739656in}}%
\pgfpathlineto{\pgfqpoint{3.355925in}{0.739656in}}%
\pgfpathlineto{\pgfqpoint{3.355627in}{0.739656in}}%
\pgfpathlineto{\pgfqpoint{3.355330in}{0.739656in}}%
\pgfpathlineto{\pgfqpoint{3.355032in}{0.739656in}}%
\pgfpathlineto{\pgfqpoint{3.354735in}{0.739656in}}%
\pgfpathlineto{\pgfqpoint{3.354437in}{0.739656in}}%
\pgfpathlineto{\pgfqpoint{3.354140in}{0.739656in}}%
\pgfpathlineto{\pgfqpoint{3.353842in}{0.739656in}}%
\pgfpathlineto{\pgfqpoint{3.353545in}{0.739656in}}%
\pgfpathlineto{\pgfqpoint{3.353247in}{0.739656in}}%
\pgfpathlineto{\pgfqpoint{3.352950in}{0.739656in}}%
\pgfpathlineto{\pgfqpoint{3.352652in}{0.739656in}}%
\pgfpathlineto{\pgfqpoint{3.352355in}{0.739656in}}%
\pgfpathlineto{\pgfqpoint{3.352057in}{0.739656in}}%
\pgfpathlineto{\pgfqpoint{3.351760in}{0.739656in}}%
\pgfpathlineto{\pgfqpoint{3.351462in}{0.739656in}}%
\pgfpathlineto{\pgfqpoint{3.351165in}{0.739656in}}%
\pgfpathlineto{\pgfqpoint{3.350867in}{0.739656in}}%
\pgfpathlineto{\pgfqpoint{3.350570in}{0.739656in}}%
\pgfpathlineto{\pgfqpoint{3.350273in}{0.739656in}}%
\pgfpathlineto{\pgfqpoint{3.349975in}{0.739656in}}%
\pgfpathlineto{\pgfqpoint{3.349678in}{0.739656in}}%
\pgfpathlineto{\pgfqpoint{3.349380in}{0.739656in}}%
\pgfpathlineto{\pgfqpoint{3.349083in}{0.739656in}}%
\pgfpathlineto{\pgfqpoint{3.348785in}{0.739656in}}%
\pgfpathlineto{\pgfqpoint{3.348488in}{0.739656in}}%
\pgfpathlineto{\pgfqpoint{3.348190in}{0.739656in}}%
\pgfpathlineto{\pgfqpoint{3.347893in}{0.739656in}}%
\pgfpathlineto{\pgfqpoint{3.347595in}{0.739656in}}%
\pgfpathlineto{\pgfqpoint{3.347298in}{0.739656in}}%
\pgfpathlineto{\pgfqpoint{3.347000in}{0.739656in}}%
\pgfpathlineto{\pgfqpoint{3.346703in}{0.739656in}}%
\pgfpathlineto{\pgfqpoint{3.346405in}{0.739656in}}%
\pgfpathlineto{\pgfqpoint{3.346108in}{0.739656in}}%
\pgfpathlineto{\pgfqpoint{3.345810in}{0.739656in}}%
\pgfpathlineto{\pgfqpoint{3.345513in}{0.739656in}}%
\pgfpathlineto{\pgfqpoint{3.345215in}{0.739656in}}%
\pgfpathlineto{\pgfqpoint{3.344918in}{0.739656in}}%
\pgfpathlineto{\pgfqpoint{3.344620in}{0.739656in}}%
\pgfpathlineto{\pgfqpoint{3.344323in}{0.739656in}}%
\pgfpathlineto{\pgfqpoint{3.344025in}{0.739656in}}%
\pgfpathlineto{\pgfqpoint{3.343728in}{0.739656in}}%
\pgfpathlineto{\pgfqpoint{3.343431in}{0.739656in}}%
\pgfpathlineto{\pgfqpoint{3.343133in}{0.739656in}}%
\pgfpathlineto{\pgfqpoint{3.342836in}{0.739656in}}%
\pgfpathlineto{\pgfqpoint{3.342538in}{0.739656in}}%
\pgfpathlineto{\pgfqpoint{3.342241in}{0.739656in}}%
\pgfpathlineto{\pgfqpoint{3.341943in}{0.739656in}}%
\pgfpathlineto{\pgfqpoint{3.341646in}{0.739656in}}%
\pgfpathlineto{\pgfqpoint{3.341348in}{0.739656in}}%
\pgfpathlineto{\pgfqpoint{3.341051in}{0.739656in}}%
\pgfpathlineto{\pgfqpoint{3.340753in}{0.739656in}}%
\pgfpathlineto{\pgfqpoint{3.340456in}{0.739656in}}%
\pgfpathlineto{\pgfqpoint{3.340158in}{0.739656in}}%
\pgfpathlineto{\pgfqpoint{3.339861in}{0.739656in}}%
\pgfpathlineto{\pgfqpoint{3.339563in}{0.739656in}}%
\pgfpathlineto{\pgfqpoint{3.339266in}{0.739656in}}%
\pgfpathlineto{\pgfqpoint{3.338968in}{0.739656in}}%
\pgfpathlineto{\pgfqpoint{3.338671in}{0.739656in}}%
\pgfpathlineto{\pgfqpoint{3.338373in}{0.739656in}}%
\pgfpathlineto{\pgfqpoint{3.338076in}{0.739656in}}%
\pgfpathlineto{\pgfqpoint{3.337778in}{0.739656in}}%
\pgfpathlineto{\pgfqpoint{3.337481in}{0.739656in}}%
\pgfpathlineto{\pgfqpoint{3.337184in}{0.739656in}}%
\pgfpathlineto{\pgfqpoint{3.336886in}{0.739656in}}%
\pgfpathlineto{\pgfqpoint{3.336589in}{0.739656in}}%
\pgfpathlineto{\pgfqpoint{3.336291in}{0.739656in}}%
\pgfpathlineto{\pgfqpoint{3.335994in}{0.739656in}}%
\pgfpathlineto{\pgfqpoint{3.335696in}{0.739656in}}%
\pgfpathlineto{\pgfqpoint{3.335399in}{0.739656in}}%
\pgfpathlineto{\pgfqpoint{3.335101in}{0.739656in}}%
\pgfpathlineto{\pgfqpoint{3.334804in}{0.739656in}}%
\pgfpathlineto{\pgfqpoint{3.334506in}{0.739656in}}%
\pgfpathlineto{\pgfqpoint{3.334209in}{0.739656in}}%
\pgfpathlineto{\pgfqpoint{3.333911in}{0.739656in}}%
\pgfpathlineto{\pgfqpoint{3.333614in}{0.739656in}}%
\pgfpathlineto{\pgfqpoint{3.333316in}{0.739656in}}%
\pgfpathlineto{\pgfqpoint{3.333019in}{0.739656in}}%
\pgfpathlineto{\pgfqpoint{3.332721in}{0.739656in}}%
\pgfpathlineto{\pgfqpoint{3.332424in}{0.739656in}}%
\pgfpathlineto{\pgfqpoint{3.332126in}{0.739656in}}%
\pgfpathlineto{\pgfqpoint{3.331829in}{0.739656in}}%
\pgfpathlineto{\pgfqpoint{3.331531in}{0.739656in}}%
\pgfpathlineto{\pgfqpoint{3.331234in}{0.739656in}}%
\pgfpathlineto{\pgfqpoint{3.330936in}{0.739656in}}%
\pgfpathlineto{\pgfqpoint{3.330639in}{0.739656in}}%
\pgfpathlineto{\pgfqpoint{3.330342in}{0.739656in}}%
\pgfpathlineto{\pgfqpoint{3.330044in}{0.739656in}}%
\pgfpathlineto{\pgfqpoint{3.329747in}{0.739656in}}%
\pgfpathlineto{\pgfqpoint{3.329449in}{0.739656in}}%
\pgfpathlineto{\pgfqpoint{3.329152in}{0.739656in}}%
\pgfpathlineto{\pgfqpoint{3.328854in}{0.739656in}}%
\pgfpathlineto{\pgfqpoint{3.328557in}{0.739656in}}%
\pgfpathlineto{\pgfqpoint{3.328259in}{0.739656in}}%
\pgfpathlineto{\pgfqpoint{3.327962in}{0.739656in}}%
\pgfpathlineto{\pgfqpoint{3.327664in}{0.739656in}}%
\pgfpathlineto{\pgfqpoint{3.327367in}{0.739656in}}%
\pgfpathlineto{\pgfqpoint{3.327069in}{0.739656in}}%
\pgfpathlineto{\pgfqpoint{3.326772in}{0.739656in}}%
\pgfpathlineto{\pgfqpoint{3.326474in}{0.739656in}}%
\pgfpathlineto{\pgfqpoint{3.326177in}{0.739656in}}%
\pgfpathlineto{\pgfqpoint{3.325879in}{0.739656in}}%
\pgfpathlineto{\pgfqpoint{3.325582in}{0.739656in}}%
\pgfpathlineto{\pgfqpoint{3.325284in}{0.739656in}}%
\pgfpathlineto{\pgfqpoint{3.324987in}{0.739656in}}%
\pgfpathlineto{\pgfqpoint{3.324689in}{0.739656in}}%
\pgfpathlineto{\pgfqpoint{3.324392in}{0.739656in}}%
\pgfpathlineto{\pgfqpoint{3.324094in}{0.739656in}}%
\pgfpathlineto{\pgfqpoint{3.323797in}{0.739656in}}%
\pgfpathlineto{\pgfqpoint{3.323500in}{0.739656in}}%
\pgfpathlineto{\pgfqpoint{3.323202in}{0.739656in}}%
\pgfpathlineto{\pgfqpoint{3.322905in}{0.739656in}}%
\pgfpathlineto{\pgfqpoint{3.322607in}{0.739656in}}%
\pgfpathlineto{\pgfqpoint{3.322310in}{0.739656in}}%
\pgfpathlineto{\pgfqpoint{3.322012in}{0.739656in}}%
\pgfpathlineto{\pgfqpoint{3.321715in}{0.739656in}}%
\pgfpathlineto{\pgfqpoint{3.321417in}{0.739656in}}%
\pgfpathlineto{\pgfqpoint{3.321120in}{0.739656in}}%
\pgfpathlineto{\pgfqpoint{3.320822in}{0.739656in}}%
\pgfpathlineto{\pgfqpoint{3.320525in}{0.739656in}}%
\pgfpathlineto{\pgfqpoint{3.320227in}{0.739656in}}%
\pgfpathlineto{\pgfqpoint{3.319930in}{0.739656in}}%
\pgfpathlineto{\pgfqpoint{3.319632in}{0.739656in}}%
\pgfpathlineto{\pgfqpoint{3.319335in}{0.739656in}}%
\pgfpathlineto{\pgfqpoint{3.319037in}{0.739656in}}%
\pgfpathlineto{\pgfqpoint{3.318740in}{0.739656in}}%
\pgfpathlineto{\pgfqpoint{3.318442in}{0.739656in}}%
\pgfpathlineto{\pgfqpoint{3.318145in}{0.739656in}}%
\pgfpathlineto{\pgfqpoint{3.317847in}{0.739656in}}%
\pgfpathlineto{\pgfqpoint{3.317550in}{0.739656in}}%
\pgfpathlineto{\pgfqpoint{3.317253in}{0.739656in}}%
\pgfpathlineto{\pgfqpoint{3.316955in}{0.739656in}}%
\pgfpathlineto{\pgfqpoint{3.316658in}{0.739656in}}%
\pgfpathlineto{\pgfqpoint{3.316360in}{0.739656in}}%
\pgfpathlineto{\pgfqpoint{3.316063in}{0.739656in}}%
\pgfpathlineto{\pgfqpoint{3.315765in}{0.739656in}}%
\pgfpathlineto{\pgfqpoint{3.315468in}{0.739656in}}%
\pgfpathlineto{\pgfqpoint{3.315170in}{0.739656in}}%
\pgfpathlineto{\pgfqpoint{3.314873in}{0.739656in}}%
\pgfpathlineto{\pgfqpoint{3.314575in}{0.739656in}}%
\pgfpathlineto{\pgfqpoint{3.314278in}{0.739656in}}%
\pgfpathlineto{\pgfqpoint{3.313980in}{0.739656in}}%
\pgfpathlineto{\pgfqpoint{3.313683in}{0.739656in}}%
\pgfpathlineto{\pgfqpoint{3.313385in}{0.739656in}}%
\pgfpathlineto{\pgfqpoint{3.313088in}{0.739656in}}%
\pgfpathlineto{\pgfqpoint{3.312790in}{0.739656in}}%
\pgfpathlineto{\pgfqpoint{3.312493in}{0.739656in}}%
\pgfpathlineto{\pgfqpoint{3.312195in}{0.739656in}}%
\pgfpathlineto{\pgfqpoint{3.311898in}{0.739656in}}%
\pgfpathlineto{\pgfqpoint{3.311600in}{0.739656in}}%
\pgfpathlineto{\pgfqpoint{3.311303in}{0.739656in}}%
\pgfpathlineto{\pgfqpoint{3.311005in}{0.739656in}}%
\pgfpathlineto{\pgfqpoint{3.310708in}{0.739656in}}%
\pgfpathlineto{\pgfqpoint{3.310411in}{0.739656in}}%
\pgfpathlineto{\pgfqpoint{3.310113in}{0.739656in}}%
\pgfpathlineto{\pgfqpoint{3.309816in}{0.739656in}}%
\pgfpathlineto{\pgfqpoint{3.309518in}{0.739656in}}%
\pgfpathlineto{\pgfqpoint{3.309221in}{0.739656in}}%
\pgfpathlineto{\pgfqpoint{3.308923in}{0.739656in}}%
\pgfpathlineto{\pgfqpoint{3.308626in}{0.739656in}}%
\pgfpathlineto{\pgfqpoint{3.308328in}{0.739656in}}%
\pgfpathlineto{\pgfqpoint{3.308031in}{0.739656in}}%
\pgfpathlineto{\pgfqpoint{3.307733in}{0.739656in}}%
\pgfpathlineto{\pgfqpoint{3.307436in}{0.739656in}}%
\pgfpathlineto{\pgfqpoint{3.307138in}{0.739656in}}%
\pgfpathlineto{\pgfqpoint{3.306841in}{0.739656in}}%
\pgfpathlineto{\pgfqpoint{3.306543in}{0.739656in}}%
\pgfpathlineto{\pgfqpoint{3.306246in}{0.739656in}}%
\pgfpathlineto{\pgfqpoint{3.305948in}{0.739656in}}%
\pgfpathlineto{\pgfqpoint{3.305651in}{0.739656in}}%
\pgfpathlineto{\pgfqpoint{3.305353in}{0.739656in}}%
\pgfpathlineto{\pgfqpoint{3.305056in}{0.739656in}}%
\pgfpathlineto{\pgfqpoint{3.304758in}{0.739656in}}%
\pgfpathlineto{\pgfqpoint{3.304461in}{0.739656in}}%
\pgfpathlineto{\pgfqpoint{3.304163in}{0.739656in}}%
\pgfpathlineto{\pgfqpoint{3.303866in}{0.739656in}}%
\pgfpathlineto{\pgfqpoint{3.303569in}{0.739656in}}%
\pgfpathlineto{\pgfqpoint{3.303271in}{0.739656in}}%
\pgfpathlineto{\pgfqpoint{3.302974in}{0.739656in}}%
\pgfpathlineto{\pgfqpoint{3.302676in}{0.739656in}}%
\pgfpathlineto{\pgfqpoint{3.302379in}{0.739656in}}%
\pgfpathlineto{\pgfqpoint{3.302081in}{0.739656in}}%
\pgfpathlineto{\pgfqpoint{3.301784in}{0.739656in}}%
\pgfpathlineto{\pgfqpoint{3.301486in}{0.739656in}}%
\pgfpathlineto{\pgfqpoint{3.301189in}{0.739656in}}%
\pgfpathlineto{\pgfqpoint{3.300891in}{0.739656in}}%
\pgfpathlineto{\pgfqpoint{3.300594in}{0.739656in}}%
\pgfpathlineto{\pgfqpoint{3.300296in}{0.739656in}}%
\pgfpathlineto{\pgfqpoint{3.299999in}{0.739656in}}%
\pgfpathlineto{\pgfqpoint{3.299701in}{0.739656in}}%
\pgfpathlineto{\pgfqpoint{3.299404in}{0.739656in}}%
\pgfpathlineto{\pgfqpoint{3.299106in}{0.739656in}}%
\pgfpathlineto{\pgfqpoint{3.298809in}{0.739656in}}%
\pgfpathlineto{\pgfqpoint{3.298511in}{0.739656in}}%
\pgfpathlineto{\pgfqpoint{3.298214in}{0.739656in}}%
\pgfpathlineto{\pgfqpoint{3.297916in}{0.739656in}}%
\pgfpathlineto{\pgfqpoint{3.297619in}{0.739656in}}%
\pgfpathlineto{\pgfqpoint{3.297321in}{0.739656in}}%
\pgfpathlineto{\pgfqpoint{3.297024in}{0.739656in}}%
\pgfpathlineto{\pgfqpoint{3.296727in}{0.739656in}}%
\pgfpathlineto{\pgfqpoint{3.296429in}{0.739656in}}%
\pgfpathlineto{\pgfqpoint{3.296132in}{0.739656in}}%
\pgfpathlineto{\pgfqpoint{3.295834in}{0.739656in}}%
\pgfpathlineto{\pgfqpoint{3.295537in}{0.739656in}}%
\pgfpathlineto{\pgfqpoint{3.295239in}{0.739656in}}%
\pgfpathlineto{\pgfqpoint{3.294942in}{0.739656in}}%
\pgfpathlineto{\pgfqpoint{3.294644in}{0.739656in}}%
\pgfpathlineto{\pgfqpoint{3.294347in}{0.739656in}}%
\pgfpathlineto{\pgfqpoint{3.294049in}{0.739656in}}%
\pgfpathlineto{\pgfqpoint{3.293752in}{0.739656in}}%
\pgfpathlineto{\pgfqpoint{3.293454in}{0.739656in}}%
\pgfpathlineto{\pgfqpoint{3.293157in}{0.739656in}}%
\pgfpathlineto{\pgfqpoint{3.292859in}{0.739656in}}%
\pgfpathlineto{\pgfqpoint{3.292562in}{0.739656in}}%
\pgfpathlineto{\pgfqpoint{3.292264in}{0.739656in}}%
\pgfpathlineto{\pgfqpoint{3.291967in}{0.739656in}}%
\pgfpathlineto{\pgfqpoint{3.291669in}{0.739656in}}%
\pgfpathlineto{\pgfqpoint{3.291372in}{0.739656in}}%
\pgfpathlineto{\pgfqpoint{3.291074in}{0.739656in}}%
\pgfpathlineto{\pgfqpoint{3.290777in}{0.739656in}}%
\pgfpathlineto{\pgfqpoint{3.290480in}{0.739656in}}%
\pgfpathlineto{\pgfqpoint{3.290182in}{0.739656in}}%
\pgfpathlineto{\pgfqpoint{3.289885in}{0.739656in}}%
\pgfpathlineto{\pgfqpoint{3.289587in}{0.739656in}}%
\pgfpathlineto{\pgfqpoint{3.289290in}{0.739656in}}%
\pgfpathlineto{\pgfqpoint{3.288992in}{0.739656in}}%
\pgfpathlineto{\pgfqpoint{3.288695in}{0.739656in}}%
\pgfpathlineto{\pgfqpoint{3.288397in}{0.739656in}}%
\pgfpathlineto{\pgfqpoint{3.288100in}{0.739656in}}%
\pgfpathlineto{\pgfqpoint{3.287802in}{0.739656in}}%
\pgfpathlineto{\pgfqpoint{3.287505in}{0.739656in}}%
\pgfpathlineto{\pgfqpoint{3.287207in}{0.739656in}}%
\pgfpathlineto{\pgfqpoint{3.286910in}{0.739656in}}%
\pgfpathlineto{\pgfqpoint{3.286612in}{0.739656in}}%
\pgfpathlineto{\pgfqpoint{3.286315in}{0.739656in}}%
\pgfpathlineto{\pgfqpoint{3.286017in}{0.739656in}}%
\pgfpathlineto{\pgfqpoint{3.285720in}{0.739656in}}%
\pgfpathlineto{\pgfqpoint{3.285422in}{0.739656in}}%
\pgfpathlineto{\pgfqpoint{3.285125in}{0.739656in}}%
\pgfpathlineto{\pgfqpoint{3.284827in}{0.739656in}}%
\pgfpathlineto{\pgfqpoint{3.284530in}{0.739656in}}%
\pgfpathlineto{\pgfqpoint{3.284232in}{0.739656in}}%
\pgfpathlineto{\pgfqpoint{3.283935in}{0.739656in}}%
\pgfpathlineto{\pgfqpoint{3.283638in}{0.739656in}}%
\pgfpathlineto{\pgfqpoint{3.283340in}{0.739656in}}%
\pgfpathlineto{\pgfqpoint{3.283043in}{0.739656in}}%
\pgfpathlineto{\pgfqpoint{3.282745in}{0.739656in}}%
\pgfpathlineto{\pgfqpoint{3.282448in}{0.739656in}}%
\pgfpathlineto{\pgfqpoint{3.282150in}{0.739656in}}%
\pgfpathlineto{\pgfqpoint{3.281853in}{0.739656in}}%
\pgfpathlineto{\pgfqpoint{3.281555in}{0.739656in}}%
\pgfpathlineto{\pgfqpoint{3.281258in}{0.739656in}}%
\pgfpathlineto{\pgfqpoint{3.280960in}{0.739656in}}%
\pgfpathlineto{\pgfqpoint{3.280663in}{0.739656in}}%
\pgfpathlineto{\pgfqpoint{3.280365in}{0.739656in}}%
\pgfpathlineto{\pgfqpoint{3.280068in}{0.739656in}}%
\pgfpathlineto{\pgfqpoint{3.279770in}{0.739656in}}%
\pgfpathlineto{\pgfqpoint{3.279473in}{0.739656in}}%
\pgfpathlineto{\pgfqpoint{3.279175in}{0.739656in}}%
\pgfpathlineto{\pgfqpoint{3.278878in}{0.739656in}}%
\pgfpathlineto{\pgfqpoint{3.278580in}{0.739656in}}%
\pgfpathlineto{\pgfqpoint{3.278283in}{0.739656in}}%
\pgfpathlineto{\pgfqpoint{3.277985in}{0.739656in}}%
\pgfpathlineto{\pgfqpoint{3.277688in}{0.739656in}}%
\pgfpathlineto{\pgfqpoint{3.277390in}{0.739656in}}%
\pgfpathlineto{\pgfqpoint{3.277093in}{0.739656in}}%
\pgfpathlineto{\pgfqpoint{3.276796in}{0.739656in}}%
\pgfpathlineto{\pgfqpoint{3.276498in}{0.739656in}}%
\pgfpathlineto{\pgfqpoint{3.276201in}{0.739656in}}%
\pgfpathlineto{\pgfqpoint{3.275903in}{0.739656in}}%
\pgfpathlineto{\pgfqpoint{3.275606in}{0.739656in}}%
\pgfpathlineto{\pgfqpoint{3.275308in}{0.739656in}}%
\pgfpathlineto{\pgfqpoint{3.275011in}{0.739656in}}%
\pgfpathlineto{\pgfqpoint{3.274713in}{0.739656in}}%
\pgfpathlineto{\pgfqpoint{3.274416in}{0.739656in}}%
\pgfpathlineto{\pgfqpoint{3.274118in}{0.739656in}}%
\pgfpathlineto{\pgfqpoint{3.273821in}{0.739656in}}%
\pgfpathlineto{\pgfqpoint{3.273523in}{0.739656in}}%
\pgfpathlineto{\pgfqpoint{3.273226in}{0.739656in}}%
\pgfpathlineto{\pgfqpoint{3.272928in}{0.739656in}}%
\pgfpathlineto{\pgfqpoint{3.272631in}{0.739656in}}%
\pgfpathlineto{\pgfqpoint{3.272333in}{0.739656in}}%
\pgfpathlineto{\pgfqpoint{3.272036in}{0.739656in}}%
\pgfpathlineto{\pgfqpoint{3.271738in}{0.739656in}}%
\pgfpathlineto{\pgfqpoint{3.271441in}{0.739656in}}%
\pgfpathlineto{\pgfqpoint{3.271143in}{0.739656in}}%
\pgfpathlineto{\pgfqpoint{3.270846in}{0.739656in}}%
\pgfpathlineto{\pgfqpoint{3.270549in}{0.739656in}}%
\pgfpathlineto{\pgfqpoint{3.270251in}{0.739656in}}%
\pgfpathlineto{\pgfqpoint{3.269954in}{0.739656in}}%
\pgfpathlineto{\pgfqpoint{3.269656in}{0.739656in}}%
\pgfpathlineto{\pgfqpoint{3.269359in}{0.739656in}}%
\pgfpathlineto{\pgfqpoint{3.269061in}{0.739656in}}%
\pgfpathlineto{\pgfqpoint{3.268764in}{0.739656in}}%
\pgfpathlineto{\pgfqpoint{3.268466in}{0.739656in}}%
\pgfpathlineto{\pgfqpoint{3.268169in}{0.739656in}}%
\pgfpathlineto{\pgfqpoint{3.267871in}{0.739656in}}%
\pgfpathlineto{\pgfqpoint{3.267574in}{0.739656in}}%
\pgfpathlineto{\pgfqpoint{3.267276in}{0.739656in}}%
\pgfpathlineto{\pgfqpoint{3.266979in}{0.739656in}}%
\pgfpathlineto{\pgfqpoint{3.266681in}{0.739656in}}%
\pgfpathlineto{\pgfqpoint{3.266384in}{0.739656in}}%
\pgfpathlineto{\pgfqpoint{3.266086in}{0.739656in}}%
\pgfpathlineto{\pgfqpoint{3.265789in}{0.739656in}}%
\pgfpathlineto{\pgfqpoint{3.265491in}{0.739656in}}%
\pgfpathlineto{\pgfqpoint{3.265194in}{0.739656in}}%
\pgfpathlineto{\pgfqpoint{3.264896in}{0.739656in}}%
\pgfpathlineto{\pgfqpoint{3.264599in}{0.739656in}}%
\pgfpathlineto{\pgfqpoint{3.264301in}{0.739656in}}%
\pgfpathlineto{\pgfqpoint{3.264004in}{0.739656in}}%
\pgfpathlineto{\pgfqpoint{3.263707in}{0.739656in}}%
\pgfpathlineto{\pgfqpoint{3.263409in}{0.739656in}}%
\pgfpathlineto{\pgfqpoint{3.263112in}{0.739656in}}%
\pgfpathlineto{\pgfqpoint{3.262814in}{0.739656in}}%
\pgfpathlineto{\pgfqpoint{3.262517in}{0.739656in}}%
\pgfpathlineto{\pgfqpoint{3.262219in}{0.739656in}}%
\pgfpathlineto{\pgfqpoint{3.261922in}{0.739656in}}%
\pgfpathlineto{\pgfqpoint{3.261624in}{0.739656in}}%
\pgfpathlineto{\pgfqpoint{3.261327in}{0.739656in}}%
\pgfpathlineto{\pgfqpoint{3.261029in}{0.739656in}}%
\pgfpathlineto{\pgfqpoint{3.260732in}{0.739656in}}%
\pgfpathlineto{\pgfqpoint{3.260434in}{0.739656in}}%
\pgfpathlineto{\pgfqpoint{3.260137in}{0.739656in}}%
\pgfpathlineto{\pgfqpoint{3.259839in}{0.739656in}}%
\pgfpathlineto{\pgfqpoint{3.259542in}{0.739656in}}%
\pgfpathlineto{\pgfqpoint{3.259244in}{0.739656in}}%
\pgfpathlineto{\pgfqpoint{3.258947in}{0.739656in}}%
\pgfpathlineto{\pgfqpoint{3.258649in}{0.739656in}}%
\pgfpathlineto{\pgfqpoint{3.258352in}{0.739656in}}%
\pgfpathlineto{\pgfqpoint{3.258054in}{0.739656in}}%
\pgfpathlineto{\pgfqpoint{3.257757in}{0.739656in}}%
\pgfpathlineto{\pgfqpoint{3.257459in}{0.739656in}}%
\pgfpathlineto{\pgfqpoint{3.257162in}{0.739656in}}%
\pgfpathlineto{\pgfqpoint{3.256865in}{0.739656in}}%
\pgfpathlineto{\pgfqpoint{3.256567in}{0.739656in}}%
\pgfpathlineto{\pgfqpoint{3.256270in}{0.739656in}}%
\pgfpathlineto{\pgfqpoint{3.255972in}{0.739656in}}%
\pgfpathlineto{\pgfqpoint{3.255675in}{0.739656in}}%
\pgfpathlineto{\pgfqpoint{3.255377in}{0.739656in}}%
\pgfpathlineto{\pgfqpoint{3.255080in}{0.739656in}}%
\pgfpathlineto{\pgfqpoint{3.254782in}{0.739656in}}%
\pgfpathlineto{\pgfqpoint{3.254485in}{0.739656in}}%
\pgfpathlineto{\pgfqpoint{3.254187in}{0.739656in}}%
\pgfpathlineto{\pgfqpoint{3.253890in}{0.739656in}}%
\pgfpathlineto{\pgfqpoint{3.253592in}{0.739656in}}%
\pgfpathlineto{\pgfqpoint{3.253295in}{0.739656in}}%
\pgfpathlineto{\pgfqpoint{3.252997in}{0.739656in}}%
\pgfpathlineto{\pgfqpoint{3.252700in}{0.739656in}}%
\pgfpathlineto{\pgfqpoint{3.252402in}{0.739656in}}%
\pgfpathlineto{\pgfqpoint{3.252105in}{0.739656in}}%
\pgfpathlineto{\pgfqpoint{3.251807in}{0.739656in}}%
\pgfpathlineto{\pgfqpoint{3.251510in}{0.739656in}}%
\pgfpathlineto{\pgfqpoint{3.251212in}{0.739656in}}%
\pgfpathlineto{\pgfqpoint{3.250915in}{0.739656in}}%
\pgfpathlineto{\pgfqpoint{3.250618in}{0.739656in}}%
\pgfpathlineto{\pgfqpoint{3.250320in}{0.739656in}}%
\pgfpathlineto{\pgfqpoint{3.250023in}{0.739656in}}%
\pgfpathlineto{\pgfqpoint{3.249725in}{0.739656in}}%
\pgfpathlineto{\pgfqpoint{3.249428in}{0.739656in}}%
\pgfpathlineto{\pgfqpoint{3.249130in}{0.739656in}}%
\pgfpathlineto{\pgfqpoint{3.248833in}{0.739656in}}%
\pgfpathlineto{\pgfqpoint{3.248535in}{0.739656in}}%
\pgfpathlineto{\pgfqpoint{3.248238in}{0.739656in}}%
\pgfpathlineto{\pgfqpoint{3.247940in}{0.739656in}}%
\pgfpathlineto{\pgfqpoint{3.247643in}{0.739656in}}%
\pgfpathlineto{\pgfqpoint{3.247345in}{0.739656in}}%
\pgfpathlineto{\pgfqpoint{3.247048in}{0.739656in}}%
\pgfpathlineto{\pgfqpoint{3.246750in}{0.739656in}}%
\pgfpathlineto{\pgfqpoint{3.246453in}{0.739656in}}%
\pgfpathlineto{\pgfqpoint{3.246155in}{0.739656in}}%
\pgfpathlineto{\pgfqpoint{3.245858in}{0.739656in}}%
\pgfpathlineto{\pgfqpoint{3.245560in}{0.739656in}}%
\pgfpathlineto{\pgfqpoint{3.245263in}{0.739656in}}%
\pgfpathlineto{\pgfqpoint{3.244965in}{0.739656in}}%
\pgfpathlineto{\pgfqpoint{3.244668in}{0.739656in}}%
\pgfpathlineto{\pgfqpoint{3.244370in}{0.739656in}}%
\pgfpathlineto{\pgfqpoint{3.244073in}{0.739656in}}%
\pgfpathlineto{\pgfqpoint{3.243776in}{0.739656in}}%
\pgfpathlineto{\pgfqpoint{3.243478in}{0.739656in}}%
\pgfpathlineto{\pgfqpoint{3.243181in}{0.739656in}}%
\pgfpathlineto{\pgfqpoint{3.242883in}{0.739656in}}%
\pgfpathlineto{\pgfqpoint{3.242586in}{0.739656in}}%
\pgfpathlineto{\pgfqpoint{3.242288in}{0.739656in}}%
\pgfpathlineto{\pgfqpoint{3.241991in}{0.739656in}}%
\pgfpathlineto{\pgfqpoint{3.241693in}{0.739656in}}%
\pgfpathlineto{\pgfqpoint{3.241396in}{0.739656in}}%
\pgfpathlineto{\pgfqpoint{3.241098in}{0.739656in}}%
\pgfpathlineto{\pgfqpoint{3.240801in}{0.739656in}}%
\pgfpathlineto{\pgfqpoint{3.240503in}{0.739656in}}%
\pgfpathlineto{\pgfqpoint{3.240206in}{0.739656in}}%
\pgfpathlineto{\pgfqpoint{3.239908in}{0.739656in}}%
\pgfpathlineto{\pgfqpoint{3.239611in}{0.739656in}}%
\pgfpathlineto{\pgfqpoint{3.239313in}{0.739656in}}%
\pgfpathlineto{\pgfqpoint{3.239016in}{0.739656in}}%
\pgfpathlineto{\pgfqpoint{3.238718in}{0.739656in}}%
\pgfpathlineto{\pgfqpoint{3.238421in}{0.739656in}}%
\pgfpathlineto{\pgfqpoint{3.238123in}{0.739656in}}%
\pgfpathlineto{\pgfqpoint{3.237826in}{0.739656in}}%
\pgfpathlineto{\pgfqpoint{3.237528in}{0.739656in}}%
\pgfpathlineto{\pgfqpoint{3.237231in}{0.739656in}}%
\pgfpathlineto{\pgfqpoint{3.236934in}{0.739656in}}%
\pgfpathlineto{\pgfqpoint{3.236636in}{0.739656in}}%
\pgfpathlineto{\pgfqpoint{3.236339in}{0.739656in}}%
\pgfpathlineto{\pgfqpoint{3.236041in}{0.739656in}}%
\pgfpathlineto{\pgfqpoint{3.235744in}{0.739656in}}%
\pgfpathlineto{\pgfqpoint{3.235446in}{0.739656in}}%
\pgfpathlineto{\pgfqpoint{3.235149in}{0.739656in}}%
\pgfpathlineto{\pgfqpoint{3.234851in}{0.739656in}}%
\pgfpathlineto{\pgfqpoint{3.234554in}{0.739656in}}%
\pgfpathlineto{\pgfqpoint{3.234256in}{0.739656in}}%
\pgfpathlineto{\pgfqpoint{3.233959in}{0.739656in}}%
\pgfpathlineto{\pgfqpoint{3.233661in}{0.739656in}}%
\pgfpathlineto{\pgfqpoint{3.233364in}{0.739656in}}%
\pgfpathlineto{\pgfqpoint{3.233066in}{0.739656in}}%
\pgfpathlineto{\pgfqpoint{3.232769in}{0.739656in}}%
\pgfpathlineto{\pgfqpoint{3.232471in}{0.739656in}}%
\pgfpathlineto{\pgfqpoint{3.232174in}{0.739656in}}%
\pgfpathlineto{\pgfqpoint{3.231876in}{0.739656in}}%
\pgfpathlineto{\pgfqpoint{3.231579in}{0.739656in}}%
\pgfpathlineto{\pgfqpoint{3.231281in}{0.739656in}}%
\pgfpathlineto{\pgfqpoint{3.230984in}{0.739656in}}%
\pgfpathlineto{\pgfqpoint{3.230687in}{0.739656in}}%
\pgfpathlineto{\pgfqpoint{3.230389in}{0.739656in}}%
\pgfpathlineto{\pgfqpoint{3.230092in}{0.739656in}}%
\pgfpathlineto{\pgfqpoint{3.229794in}{0.739656in}}%
\pgfpathlineto{\pgfqpoint{3.229497in}{0.739656in}}%
\pgfpathlineto{\pgfqpoint{3.229199in}{0.739656in}}%
\pgfpathlineto{\pgfqpoint{3.228902in}{0.739656in}}%
\pgfpathlineto{\pgfqpoint{3.228604in}{0.739656in}}%
\pgfpathlineto{\pgfqpoint{3.228307in}{0.739656in}}%
\pgfpathlineto{\pgfqpoint{3.228009in}{0.739656in}}%
\pgfpathlineto{\pgfqpoint{3.227712in}{0.739656in}}%
\pgfpathlineto{\pgfqpoint{3.227414in}{0.739656in}}%
\pgfpathlineto{\pgfqpoint{3.227117in}{0.739656in}}%
\pgfpathlineto{\pgfqpoint{3.226819in}{0.739656in}}%
\pgfpathlineto{\pgfqpoint{3.226522in}{0.739656in}}%
\pgfpathlineto{\pgfqpoint{3.226224in}{0.739656in}}%
\pgfpathlineto{\pgfqpoint{3.225927in}{0.739656in}}%
\pgfpathlineto{\pgfqpoint{3.225629in}{0.739656in}}%
\pgfpathlineto{\pgfqpoint{3.225332in}{0.739656in}}%
\pgfpathlineto{\pgfqpoint{3.225034in}{0.739656in}}%
\pgfpathlineto{\pgfqpoint{3.224737in}{0.739656in}}%
\pgfpathlineto{\pgfqpoint{3.224439in}{0.739656in}}%
\pgfpathlineto{\pgfqpoint{3.224142in}{0.739656in}}%
\pgfpathlineto{\pgfqpoint{3.223845in}{0.739656in}}%
\pgfpathlineto{\pgfqpoint{3.223547in}{0.739656in}}%
\pgfpathlineto{\pgfqpoint{3.223250in}{0.739656in}}%
\pgfpathlineto{\pgfqpoint{3.222952in}{0.739656in}}%
\pgfpathlineto{\pgfqpoint{3.222655in}{0.739656in}}%
\pgfpathlineto{\pgfqpoint{3.222357in}{0.739656in}}%
\pgfpathlineto{\pgfqpoint{3.222060in}{0.739656in}}%
\pgfpathlineto{\pgfqpoint{3.221762in}{0.739656in}}%
\pgfpathlineto{\pgfqpoint{3.221465in}{0.739656in}}%
\pgfpathlineto{\pgfqpoint{3.221167in}{0.739656in}}%
\pgfpathlineto{\pgfqpoint{3.220870in}{0.739656in}}%
\pgfpathlineto{\pgfqpoint{3.220572in}{0.739656in}}%
\pgfpathlineto{\pgfqpoint{3.220275in}{0.739656in}}%
\pgfpathlineto{\pgfqpoint{3.219977in}{0.739656in}}%
\pgfpathlineto{\pgfqpoint{3.219680in}{0.739656in}}%
\pgfpathlineto{\pgfqpoint{3.219382in}{0.739656in}}%
\pgfpathlineto{\pgfqpoint{3.219085in}{0.739656in}}%
\pgfpathlineto{\pgfqpoint{3.218787in}{0.739656in}}%
\pgfpathlineto{\pgfqpoint{3.218490in}{0.739656in}}%
\pgfpathlineto{\pgfqpoint{3.218192in}{0.739656in}}%
\pgfpathlineto{\pgfqpoint{3.217895in}{0.739656in}}%
\pgfpathlineto{\pgfqpoint{3.217597in}{0.739656in}}%
\pgfpathlineto{\pgfqpoint{3.217300in}{0.739656in}}%
\pgfpathlineto{\pgfqpoint{3.217003in}{0.739656in}}%
\pgfpathlineto{\pgfqpoint{3.216705in}{0.739656in}}%
\pgfpathlineto{\pgfqpoint{3.216408in}{0.739656in}}%
\pgfpathlineto{\pgfqpoint{3.216110in}{0.739656in}}%
\pgfpathlineto{\pgfqpoint{3.215813in}{0.739656in}}%
\pgfpathlineto{\pgfqpoint{3.215515in}{0.739656in}}%
\pgfpathlineto{\pgfqpoint{3.215218in}{0.739656in}}%
\pgfpathlineto{\pgfqpoint{3.214920in}{0.739656in}}%
\pgfpathlineto{\pgfqpoint{3.214623in}{0.739656in}}%
\pgfpathlineto{\pgfqpoint{3.214325in}{0.739656in}}%
\pgfpathlineto{\pgfqpoint{3.214028in}{0.739656in}}%
\pgfpathlineto{\pgfqpoint{3.213730in}{0.739656in}}%
\pgfpathlineto{\pgfqpoint{3.213433in}{0.739656in}}%
\pgfpathlineto{\pgfqpoint{3.213135in}{0.739656in}}%
\pgfpathlineto{\pgfqpoint{3.212838in}{0.739656in}}%
\pgfpathlineto{\pgfqpoint{3.212540in}{0.739656in}}%
\pgfpathlineto{\pgfqpoint{3.212243in}{0.739656in}}%
\pgfpathlineto{\pgfqpoint{3.211945in}{0.739656in}}%
\pgfpathlineto{\pgfqpoint{3.211648in}{0.739656in}}%
\pgfpathlineto{\pgfqpoint{3.211350in}{0.739656in}}%
\pgfpathlineto{\pgfqpoint{3.211053in}{0.739656in}}%
\pgfpathlineto{\pgfqpoint{3.210756in}{0.739656in}}%
\pgfpathlineto{\pgfqpoint{3.210458in}{0.739656in}}%
\pgfpathlineto{\pgfqpoint{3.210161in}{0.739656in}}%
\pgfpathlineto{\pgfqpoint{3.209863in}{0.739656in}}%
\pgfpathlineto{\pgfqpoint{3.209566in}{0.739656in}}%
\pgfpathlineto{\pgfqpoint{3.209268in}{0.739656in}}%
\pgfpathlineto{\pgfqpoint{3.208971in}{0.739656in}}%
\pgfpathlineto{\pgfqpoint{3.208673in}{0.739656in}}%
\pgfpathlineto{\pgfqpoint{3.208376in}{0.739656in}}%
\pgfpathlineto{\pgfqpoint{3.208078in}{0.739656in}}%
\pgfpathlineto{\pgfqpoint{3.207781in}{0.739656in}}%
\pgfpathlineto{\pgfqpoint{3.207483in}{0.739656in}}%
\pgfpathlineto{\pgfqpoint{3.207186in}{0.739656in}}%
\pgfpathlineto{\pgfqpoint{3.206888in}{0.739656in}}%
\pgfpathlineto{\pgfqpoint{3.206591in}{0.739656in}}%
\pgfpathlineto{\pgfqpoint{3.206293in}{0.739656in}}%
\pgfpathlineto{\pgfqpoint{3.205996in}{0.739656in}}%
\pgfpathlineto{\pgfqpoint{3.205698in}{0.739656in}}%
\pgfpathlineto{\pgfqpoint{3.205401in}{0.739656in}}%
\pgfpathlineto{\pgfqpoint{3.205103in}{0.739656in}}%
\pgfpathlineto{\pgfqpoint{3.204806in}{0.739656in}}%
\pgfpathlineto{\pgfqpoint{3.204508in}{0.739656in}}%
\pgfpathlineto{\pgfqpoint{3.204211in}{0.739656in}}%
\pgfpathlineto{\pgfqpoint{3.203914in}{0.739656in}}%
\pgfpathlineto{\pgfqpoint{3.203616in}{0.739656in}}%
\pgfpathlineto{\pgfqpoint{3.203319in}{0.739656in}}%
\pgfpathlineto{\pgfqpoint{3.203021in}{0.739656in}}%
\pgfpathlineto{\pgfqpoint{3.202724in}{0.739656in}}%
\pgfpathlineto{\pgfqpoint{3.202426in}{0.739656in}}%
\pgfpathlineto{\pgfqpoint{3.202129in}{0.739656in}}%
\pgfpathlineto{\pgfqpoint{3.201831in}{0.739656in}}%
\pgfpathlineto{\pgfqpoint{3.201534in}{0.739656in}}%
\pgfpathlineto{\pgfqpoint{3.201236in}{0.739656in}}%
\pgfpathlineto{\pgfqpoint{3.200939in}{0.739656in}}%
\pgfpathlineto{\pgfqpoint{3.200641in}{0.739656in}}%
\pgfpathlineto{\pgfqpoint{3.200344in}{0.739656in}}%
\pgfpathlineto{\pgfqpoint{3.200046in}{0.739656in}}%
\pgfpathlineto{\pgfqpoint{3.199749in}{0.739656in}}%
\pgfpathlineto{\pgfqpoint{3.199451in}{0.739656in}}%
\pgfpathlineto{\pgfqpoint{3.199154in}{0.739656in}}%
\pgfpathlineto{\pgfqpoint{3.198856in}{0.739656in}}%
\pgfpathlineto{\pgfqpoint{3.198559in}{0.739656in}}%
\pgfpathlineto{\pgfqpoint{3.198261in}{0.739656in}}%
\pgfpathlineto{\pgfqpoint{3.197964in}{0.739656in}}%
\pgfpathlineto{\pgfqpoint{3.197666in}{0.739656in}}%
\pgfpathlineto{\pgfqpoint{3.197369in}{0.739656in}}%
\pgfpathlineto{\pgfqpoint{3.197072in}{0.739656in}}%
\pgfpathlineto{\pgfqpoint{3.196774in}{0.739656in}}%
\pgfpathlineto{\pgfqpoint{3.196477in}{0.739656in}}%
\pgfpathlineto{\pgfqpoint{3.196179in}{0.739656in}}%
\pgfpathlineto{\pgfqpoint{3.195882in}{0.739656in}}%
\pgfpathlineto{\pgfqpoint{3.195584in}{0.739656in}}%
\pgfpathlineto{\pgfqpoint{3.195287in}{0.739656in}}%
\pgfpathlineto{\pgfqpoint{3.194989in}{0.739656in}}%
\pgfpathlineto{\pgfqpoint{3.194692in}{0.739656in}}%
\pgfpathlineto{\pgfqpoint{3.194394in}{0.739656in}}%
\pgfpathlineto{\pgfqpoint{3.194097in}{0.739656in}}%
\pgfpathlineto{\pgfqpoint{3.193799in}{0.739656in}}%
\pgfpathlineto{\pgfqpoint{3.193502in}{0.739656in}}%
\pgfpathlineto{\pgfqpoint{3.193204in}{0.739656in}}%
\pgfpathlineto{\pgfqpoint{3.192907in}{0.739656in}}%
\pgfpathlineto{\pgfqpoint{3.192609in}{0.739656in}}%
\pgfpathlineto{\pgfqpoint{3.192312in}{0.739656in}}%
\pgfpathlineto{\pgfqpoint{3.192014in}{0.739656in}}%
\pgfpathlineto{\pgfqpoint{3.191717in}{0.739656in}}%
\pgfpathlineto{\pgfqpoint{3.191419in}{0.739656in}}%
\pgfpathlineto{\pgfqpoint{3.191122in}{0.739656in}}%
\pgfpathlineto{\pgfqpoint{3.190825in}{0.739656in}}%
\pgfpathlineto{\pgfqpoint{3.190527in}{0.739656in}}%
\pgfpathlineto{\pgfqpoint{3.190230in}{0.739656in}}%
\pgfpathlineto{\pgfqpoint{3.189932in}{0.739656in}}%
\pgfpathlineto{\pgfqpoint{3.189635in}{0.739656in}}%
\pgfpathlineto{\pgfqpoint{3.189337in}{0.739656in}}%
\pgfpathlineto{\pgfqpoint{3.189040in}{0.739656in}}%
\pgfpathlineto{\pgfqpoint{3.188742in}{0.739656in}}%
\pgfpathlineto{\pgfqpoint{3.188445in}{0.739656in}}%
\pgfpathlineto{\pgfqpoint{3.188147in}{0.739656in}}%
\pgfpathlineto{\pgfqpoint{3.187850in}{0.739656in}}%
\pgfpathlineto{\pgfqpoint{3.187552in}{0.739656in}}%
\pgfpathlineto{\pgfqpoint{3.187255in}{0.739656in}}%
\pgfpathlineto{\pgfqpoint{3.186957in}{0.739656in}}%
\pgfpathlineto{\pgfqpoint{3.186660in}{0.739656in}}%
\pgfpathlineto{\pgfqpoint{3.186362in}{0.739656in}}%
\pgfpathlineto{\pgfqpoint{3.186065in}{0.739656in}}%
\pgfpathlineto{\pgfqpoint{3.185767in}{0.739656in}}%
\pgfpathlineto{\pgfqpoint{3.185470in}{0.739656in}}%
\pgfpathlineto{\pgfqpoint{3.185172in}{0.739656in}}%
\pgfpathlineto{\pgfqpoint{3.184875in}{0.739656in}}%
\pgfpathlineto{\pgfqpoint{3.184577in}{0.739656in}}%
\pgfpathlineto{\pgfqpoint{3.184280in}{0.739656in}}%
\pgfpathlineto{\pgfqpoint{3.183983in}{0.739656in}}%
\pgfpathlineto{\pgfqpoint{3.183685in}{0.739656in}}%
\pgfpathlineto{\pgfqpoint{3.183388in}{0.739656in}}%
\pgfpathlineto{\pgfqpoint{3.183090in}{0.739656in}}%
\pgfpathlineto{\pgfqpoint{3.182793in}{0.739656in}}%
\pgfpathlineto{\pgfqpoint{3.182495in}{0.739656in}}%
\pgfpathlineto{\pgfqpoint{3.182198in}{0.739656in}}%
\pgfpathlineto{\pgfqpoint{3.181900in}{0.739656in}}%
\pgfpathlineto{\pgfqpoint{3.181603in}{0.739656in}}%
\pgfpathlineto{\pgfqpoint{3.181305in}{0.739656in}}%
\pgfpathlineto{\pgfqpoint{3.181008in}{0.739656in}}%
\pgfpathlineto{\pgfqpoint{3.180710in}{0.739656in}}%
\pgfpathlineto{\pgfqpoint{3.180413in}{0.739656in}}%
\pgfpathlineto{\pgfqpoint{3.180115in}{0.739656in}}%
\pgfpathlineto{\pgfqpoint{3.179818in}{0.739656in}}%
\pgfpathlineto{\pgfqpoint{3.179520in}{0.739656in}}%
\pgfpathlineto{\pgfqpoint{3.179223in}{0.739656in}}%
\pgfpathlineto{\pgfqpoint{3.178925in}{0.739656in}}%
\pgfpathlineto{\pgfqpoint{3.178628in}{0.739656in}}%
\pgfpathlineto{\pgfqpoint{3.178330in}{0.739656in}}%
\pgfpathlineto{\pgfqpoint{3.178033in}{0.739656in}}%
\pgfpathlineto{\pgfqpoint{3.177735in}{0.739656in}}%
\pgfpathlineto{\pgfqpoint{3.177438in}{0.739656in}}%
\pgfpathlineto{\pgfqpoint{3.177141in}{0.739656in}}%
\pgfpathlineto{\pgfqpoint{3.176843in}{0.739656in}}%
\pgfpathlineto{\pgfqpoint{3.176546in}{0.739656in}}%
\pgfpathlineto{\pgfqpoint{3.176248in}{0.739656in}}%
\pgfpathlineto{\pgfqpoint{3.175951in}{0.739656in}}%
\pgfpathlineto{\pgfqpoint{3.175653in}{0.739656in}}%
\pgfpathlineto{\pgfqpoint{3.175356in}{0.739656in}}%
\pgfpathlineto{\pgfqpoint{3.175058in}{0.739656in}}%
\pgfpathlineto{\pgfqpoint{3.174761in}{0.739656in}}%
\pgfpathlineto{\pgfqpoint{3.174463in}{0.739656in}}%
\pgfpathlineto{\pgfqpoint{3.174166in}{0.739656in}}%
\pgfpathlineto{\pgfqpoint{3.173868in}{0.739656in}}%
\pgfpathlineto{\pgfqpoint{3.173571in}{0.739656in}}%
\pgfpathlineto{\pgfqpoint{3.173273in}{0.739656in}}%
\pgfpathlineto{\pgfqpoint{3.172976in}{0.739656in}}%
\pgfpathlineto{\pgfqpoint{3.172678in}{0.739656in}}%
\pgfpathlineto{\pgfqpoint{3.172381in}{0.739656in}}%
\pgfpathlineto{\pgfqpoint{3.172083in}{0.739656in}}%
\pgfpathlineto{\pgfqpoint{3.171786in}{0.739656in}}%
\pgfpathlineto{\pgfqpoint{3.171488in}{0.739656in}}%
\pgfpathlineto{\pgfqpoint{3.171191in}{0.739656in}}%
\pgfpathlineto{\pgfqpoint{3.170894in}{0.739656in}}%
\pgfpathlineto{\pgfqpoint{3.170596in}{0.739656in}}%
\pgfpathlineto{\pgfqpoint{3.170299in}{0.739656in}}%
\pgfpathlineto{\pgfqpoint{3.170001in}{0.739656in}}%
\pgfpathlineto{\pgfqpoint{3.169704in}{0.739656in}}%
\pgfpathlineto{\pgfqpoint{3.169406in}{0.739656in}}%
\pgfpathlineto{\pgfqpoint{3.169109in}{0.739656in}}%
\pgfpathlineto{\pgfqpoint{3.168811in}{0.739656in}}%
\pgfpathlineto{\pgfqpoint{3.168514in}{0.739656in}}%
\pgfpathlineto{\pgfqpoint{3.168216in}{0.739656in}}%
\pgfpathlineto{\pgfqpoint{3.167919in}{0.739656in}}%
\pgfpathlineto{\pgfqpoint{3.167621in}{0.739656in}}%
\pgfpathlineto{\pgfqpoint{3.167324in}{0.739656in}}%
\pgfpathlineto{\pgfqpoint{3.167026in}{0.739656in}}%
\pgfpathlineto{\pgfqpoint{3.166729in}{0.739656in}}%
\pgfpathlineto{\pgfqpoint{3.166431in}{0.739656in}}%
\pgfpathlineto{\pgfqpoint{3.166134in}{0.739656in}}%
\pgfpathlineto{\pgfqpoint{3.165836in}{0.739656in}}%
\pgfpathlineto{\pgfqpoint{3.165539in}{0.739656in}}%
\pgfpathlineto{\pgfqpoint{3.165241in}{0.739656in}}%
\pgfpathlineto{\pgfqpoint{3.164944in}{0.739656in}}%
\pgfpathlineto{\pgfqpoint{3.164646in}{0.739656in}}%
\pgfpathlineto{\pgfqpoint{3.164349in}{0.739656in}}%
\pgfpathlineto{\pgfqpoint{3.164052in}{0.739656in}}%
\pgfpathlineto{\pgfqpoint{3.163754in}{0.739656in}}%
\pgfpathlineto{\pgfqpoint{3.163457in}{0.739656in}}%
\pgfpathlineto{\pgfqpoint{3.163159in}{0.739656in}}%
\pgfpathlineto{\pgfqpoint{3.162862in}{0.739656in}}%
\pgfpathlineto{\pgfqpoint{3.162564in}{0.739656in}}%
\pgfpathlineto{\pgfqpoint{3.162267in}{0.739656in}}%
\pgfpathlineto{\pgfqpoint{3.161969in}{0.739656in}}%
\pgfpathlineto{\pgfqpoint{3.161672in}{0.739656in}}%
\pgfpathlineto{\pgfqpoint{3.161374in}{0.739656in}}%
\pgfpathlineto{\pgfqpoint{3.161077in}{0.739656in}}%
\pgfpathlineto{\pgfqpoint{3.160779in}{0.739656in}}%
\pgfpathlineto{\pgfqpoint{3.160482in}{0.739656in}}%
\pgfpathlineto{\pgfqpoint{3.160184in}{0.739656in}}%
\pgfpathlineto{\pgfqpoint{3.159887in}{0.739656in}}%
\pgfpathlineto{\pgfqpoint{3.159589in}{0.739656in}}%
\pgfpathlineto{\pgfqpoint{3.159292in}{0.739656in}}%
\pgfpathlineto{\pgfqpoint{3.158994in}{0.739656in}}%
\pgfpathlineto{\pgfqpoint{3.158697in}{0.739656in}}%
\pgfpathlineto{\pgfqpoint{3.158399in}{0.739656in}}%
\pgfpathlineto{\pgfqpoint{3.158102in}{0.739656in}}%
\pgfpathlineto{\pgfqpoint{3.157804in}{0.739656in}}%
\pgfpathlineto{\pgfqpoint{3.157507in}{0.739656in}}%
\pgfpathlineto{\pgfqpoint{3.157210in}{0.739656in}}%
\pgfpathlineto{\pgfqpoint{3.156912in}{0.739656in}}%
\pgfpathlineto{\pgfqpoint{3.156615in}{0.739656in}}%
\pgfpathlineto{\pgfqpoint{3.156317in}{0.739656in}}%
\pgfpathlineto{\pgfqpoint{3.156020in}{0.739656in}}%
\pgfpathlineto{\pgfqpoint{3.155722in}{0.739656in}}%
\pgfpathlineto{\pgfqpoint{3.155425in}{0.739656in}}%
\pgfpathlineto{\pgfqpoint{3.155127in}{0.739656in}}%
\pgfpathlineto{\pgfqpoint{3.154830in}{0.739656in}}%
\pgfpathlineto{\pgfqpoint{3.154532in}{0.739656in}}%
\pgfpathlineto{\pgfqpoint{3.154235in}{0.739656in}}%
\pgfpathlineto{\pgfqpoint{3.153937in}{0.739656in}}%
\pgfpathlineto{\pgfqpoint{3.153640in}{0.739656in}}%
\pgfpathlineto{\pgfqpoint{3.153342in}{0.739656in}}%
\pgfpathlineto{\pgfqpoint{3.153045in}{0.739656in}}%
\pgfpathlineto{\pgfqpoint{3.152747in}{0.739656in}}%
\pgfpathlineto{\pgfqpoint{3.152450in}{0.739656in}}%
\pgfpathlineto{\pgfqpoint{3.152152in}{0.739656in}}%
\pgfpathlineto{\pgfqpoint{3.151855in}{0.739656in}}%
\pgfpathlineto{\pgfqpoint{3.151557in}{0.739656in}}%
\pgfpathlineto{\pgfqpoint{3.151260in}{0.739656in}}%
\pgfpathlineto{\pgfqpoint{3.150963in}{0.739656in}}%
\pgfpathlineto{\pgfqpoint{3.150665in}{0.739656in}}%
\pgfpathlineto{\pgfqpoint{3.150368in}{0.739656in}}%
\pgfpathlineto{\pgfqpoint{3.150070in}{0.739656in}}%
\pgfpathlineto{\pgfqpoint{3.149773in}{0.739656in}}%
\pgfpathlineto{\pgfqpoint{3.149475in}{0.739656in}}%
\pgfpathlineto{\pgfqpoint{3.149178in}{0.739656in}}%
\pgfpathlineto{\pgfqpoint{3.148880in}{0.739656in}}%
\pgfpathlineto{\pgfqpoint{3.148583in}{0.739656in}}%
\pgfpathlineto{\pgfqpoint{3.148285in}{0.739656in}}%
\pgfpathlineto{\pgfqpoint{3.147988in}{0.739656in}}%
\pgfpathlineto{\pgfqpoint{3.147690in}{0.739656in}}%
\pgfpathlineto{\pgfqpoint{3.147393in}{0.739656in}}%
\pgfpathlineto{\pgfqpoint{3.147095in}{0.739656in}}%
\pgfpathlineto{\pgfqpoint{3.146798in}{0.739656in}}%
\pgfpathlineto{\pgfqpoint{3.146500in}{0.739656in}}%
\pgfpathlineto{\pgfqpoint{3.146203in}{0.739656in}}%
\pgfpathlineto{\pgfqpoint{3.145905in}{0.739656in}}%
\pgfpathlineto{\pgfqpoint{3.145608in}{0.739656in}}%
\pgfpathlineto{\pgfqpoint{3.145310in}{0.739656in}}%
\pgfpathlineto{\pgfqpoint{3.145013in}{0.739656in}}%
\pgfpathlineto{\pgfqpoint{3.144715in}{0.739656in}}%
\pgfpathlineto{\pgfqpoint{3.144418in}{0.739656in}}%
\pgfpathlineto{\pgfqpoint{3.144121in}{0.739656in}}%
\pgfpathlineto{\pgfqpoint{3.143823in}{0.739656in}}%
\pgfpathlineto{\pgfqpoint{3.143526in}{0.739656in}}%
\pgfpathlineto{\pgfqpoint{3.143228in}{0.739656in}}%
\pgfpathlineto{\pgfqpoint{3.142931in}{0.739656in}}%
\pgfpathlineto{\pgfqpoint{3.142633in}{0.739656in}}%
\pgfpathlineto{\pgfqpoint{3.142336in}{0.739656in}}%
\pgfpathlineto{\pgfqpoint{3.142038in}{0.739656in}}%
\pgfpathlineto{\pgfqpoint{3.141741in}{0.739656in}}%
\pgfpathlineto{\pgfqpoint{3.141443in}{0.739656in}}%
\pgfpathlineto{\pgfqpoint{3.141146in}{0.739656in}}%
\pgfpathlineto{\pgfqpoint{3.140848in}{0.739656in}}%
\pgfpathlineto{\pgfqpoint{3.140551in}{0.739656in}}%
\pgfpathlineto{\pgfqpoint{3.140253in}{0.739656in}}%
\pgfpathlineto{\pgfqpoint{3.139956in}{0.739656in}}%
\pgfpathlineto{\pgfqpoint{3.139658in}{0.739656in}}%
\pgfpathlineto{\pgfqpoint{3.139361in}{0.739656in}}%
\pgfpathlineto{\pgfqpoint{3.139063in}{0.739656in}}%
\pgfpathlineto{\pgfqpoint{3.138766in}{0.739656in}}%
\pgfpathlineto{\pgfqpoint{3.138468in}{0.739656in}}%
\pgfpathlineto{\pgfqpoint{3.138171in}{0.739656in}}%
\pgfpathlineto{\pgfqpoint{3.137873in}{0.739656in}}%
\pgfpathlineto{\pgfqpoint{3.137576in}{0.739656in}}%
\pgfpathlineto{\pgfqpoint{3.137279in}{0.739656in}}%
\pgfpathlineto{\pgfqpoint{3.136981in}{0.739656in}}%
\pgfpathlineto{\pgfqpoint{3.136684in}{0.739656in}}%
\pgfpathlineto{\pgfqpoint{3.136386in}{0.739656in}}%
\pgfpathlineto{\pgfqpoint{3.136089in}{0.739656in}}%
\pgfpathlineto{\pgfqpoint{3.135791in}{0.739656in}}%
\pgfpathlineto{\pgfqpoint{3.135494in}{0.739656in}}%
\pgfpathlineto{\pgfqpoint{3.135196in}{0.739656in}}%
\pgfpathlineto{\pgfqpoint{3.134899in}{0.739656in}}%
\pgfpathlineto{\pgfqpoint{3.134601in}{0.739656in}}%
\pgfpathlineto{\pgfqpoint{3.134304in}{0.739656in}}%
\pgfpathlineto{\pgfqpoint{3.134006in}{0.739656in}}%
\pgfpathlineto{\pgfqpoint{3.133709in}{0.739656in}}%
\pgfpathlineto{\pgfqpoint{3.133411in}{0.739656in}}%
\pgfpathlineto{\pgfqpoint{3.133114in}{0.739656in}}%
\pgfpathlineto{\pgfqpoint{3.132816in}{0.739656in}}%
\pgfpathlineto{\pgfqpoint{3.132519in}{0.739656in}}%
\pgfpathlineto{\pgfqpoint{3.132221in}{0.739656in}}%
\pgfpathlineto{\pgfqpoint{3.131924in}{0.739656in}}%
\pgfpathlineto{\pgfqpoint{3.131626in}{0.739656in}}%
\pgfpathlineto{\pgfqpoint{3.131329in}{0.739656in}}%
\pgfpathlineto{\pgfqpoint{3.131032in}{0.739656in}}%
\pgfpathlineto{\pgfqpoint{3.130734in}{0.739656in}}%
\pgfpathlineto{\pgfqpoint{3.130437in}{0.739656in}}%
\pgfpathlineto{\pgfqpoint{3.130139in}{0.739656in}}%
\pgfpathlineto{\pgfqpoint{3.129842in}{0.739656in}}%
\pgfpathlineto{\pgfqpoint{3.129544in}{0.739656in}}%
\pgfpathlineto{\pgfqpoint{3.129247in}{0.739656in}}%
\pgfpathlineto{\pgfqpoint{3.128949in}{0.739656in}}%
\pgfpathlineto{\pgfqpoint{3.128652in}{0.739656in}}%
\pgfpathlineto{\pgfqpoint{3.128354in}{0.739656in}}%
\pgfpathlineto{\pgfqpoint{3.128057in}{0.739656in}}%
\pgfpathlineto{\pgfqpoint{3.127759in}{0.739656in}}%
\pgfpathlineto{\pgfqpoint{3.127462in}{0.739656in}}%
\pgfpathlineto{\pgfqpoint{3.127164in}{0.739656in}}%
\pgfpathlineto{\pgfqpoint{3.126867in}{0.739656in}}%
\pgfpathlineto{\pgfqpoint{3.126569in}{0.739656in}}%
\pgfpathlineto{\pgfqpoint{3.126272in}{0.739656in}}%
\pgfpathlineto{\pgfqpoint{3.125974in}{0.739656in}}%
\pgfpathlineto{\pgfqpoint{3.125677in}{0.739656in}}%
\pgfpathlineto{\pgfqpoint{3.125379in}{0.739656in}}%
\pgfpathlineto{\pgfqpoint{3.125082in}{0.739656in}}%
\pgfpathlineto{\pgfqpoint{3.124784in}{0.739656in}}%
\pgfpathlineto{\pgfqpoint{3.124487in}{0.739656in}}%
\pgfpathlineto{\pgfqpoint{3.124190in}{0.739656in}}%
\pgfpathlineto{\pgfqpoint{3.123892in}{0.739656in}}%
\pgfpathlineto{\pgfqpoint{3.123595in}{0.739656in}}%
\pgfpathlineto{\pgfqpoint{3.123297in}{0.739656in}}%
\pgfpathlineto{\pgfqpoint{3.123000in}{0.739656in}}%
\pgfpathlineto{\pgfqpoint{3.122702in}{0.739656in}}%
\pgfpathlineto{\pgfqpoint{3.122405in}{0.739656in}}%
\pgfpathlineto{\pgfqpoint{3.122107in}{0.739656in}}%
\pgfpathlineto{\pgfqpoint{3.121810in}{0.739656in}}%
\pgfpathlineto{\pgfqpoint{3.121512in}{0.739656in}}%
\pgfpathlineto{\pgfqpoint{3.121215in}{0.739656in}}%
\pgfpathlineto{\pgfqpoint{3.120917in}{0.739656in}}%
\pgfpathlineto{\pgfqpoint{3.120620in}{0.739656in}}%
\pgfpathlineto{\pgfqpoint{3.120322in}{0.739656in}}%
\pgfpathlineto{\pgfqpoint{3.120025in}{0.739656in}}%
\pgfpathlineto{\pgfqpoint{3.119727in}{0.739656in}}%
\pgfpathlineto{\pgfqpoint{3.119430in}{0.739656in}}%
\pgfpathlineto{\pgfqpoint{3.119132in}{0.739656in}}%
\pgfpathlineto{\pgfqpoint{3.118835in}{0.739656in}}%
\pgfpathlineto{\pgfqpoint{3.118537in}{0.739656in}}%
\pgfpathlineto{\pgfqpoint{3.118240in}{0.739656in}}%
\pgfpathlineto{\pgfqpoint{3.117942in}{0.739656in}}%
\pgfpathlineto{\pgfqpoint{3.117645in}{0.739656in}}%
\pgfpathlineto{\pgfqpoint{3.117348in}{0.739656in}}%
\pgfpathlineto{\pgfqpoint{3.117050in}{0.739656in}}%
\pgfpathlineto{\pgfqpoint{3.116753in}{0.739656in}}%
\pgfpathlineto{\pgfqpoint{3.116455in}{0.739656in}}%
\pgfpathlineto{\pgfqpoint{3.116158in}{0.739656in}}%
\pgfpathlineto{\pgfqpoint{3.115860in}{0.739656in}}%
\pgfpathlineto{\pgfqpoint{3.115563in}{0.739656in}}%
\pgfpathlineto{\pgfqpoint{3.115265in}{0.739656in}}%
\pgfpathlineto{\pgfqpoint{3.114968in}{0.739656in}}%
\pgfpathlineto{\pgfqpoint{3.114670in}{0.739656in}}%
\pgfpathlineto{\pgfqpoint{3.114373in}{0.739656in}}%
\pgfpathlineto{\pgfqpoint{3.114075in}{0.739656in}}%
\pgfpathlineto{\pgfqpoint{3.113778in}{0.739656in}}%
\pgfpathlineto{\pgfqpoint{3.113480in}{0.739656in}}%
\pgfpathlineto{\pgfqpoint{3.113183in}{0.739656in}}%
\pgfpathlineto{\pgfqpoint{3.112885in}{0.739656in}}%
\pgfpathlineto{\pgfqpoint{3.112588in}{0.739656in}}%
\pgfpathlineto{\pgfqpoint{3.112290in}{0.739656in}}%
\pgfpathlineto{\pgfqpoint{3.111993in}{0.739656in}}%
\pgfpathlineto{\pgfqpoint{3.111695in}{0.739656in}}%
\pgfpathlineto{\pgfqpoint{3.111398in}{0.739656in}}%
\pgfpathlineto{\pgfqpoint{3.111101in}{0.739656in}}%
\pgfpathlineto{\pgfqpoint{3.110803in}{0.739656in}}%
\pgfpathlineto{\pgfqpoint{3.110506in}{0.739656in}}%
\pgfpathlineto{\pgfqpoint{3.110208in}{0.739656in}}%
\pgfpathlineto{\pgfqpoint{3.109911in}{0.739656in}}%
\pgfpathlineto{\pgfqpoint{3.109613in}{0.739656in}}%
\pgfpathlineto{\pgfqpoint{3.109316in}{0.739656in}}%
\pgfpathlineto{\pgfqpoint{3.109018in}{0.739656in}}%
\pgfpathlineto{\pgfqpoint{3.108721in}{0.739656in}}%
\pgfpathlineto{\pgfqpoint{3.108423in}{0.739656in}}%
\pgfpathlineto{\pgfqpoint{3.108126in}{0.739656in}}%
\pgfpathlineto{\pgfqpoint{3.107828in}{0.739656in}}%
\pgfpathlineto{\pgfqpoint{3.107531in}{0.739656in}}%
\pgfpathlineto{\pgfqpoint{3.107233in}{0.739656in}}%
\pgfpathlineto{\pgfqpoint{3.106936in}{0.739656in}}%
\pgfpathlineto{\pgfqpoint{3.106638in}{0.739656in}}%
\pgfpathlineto{\pgfqpoint{3.106341in}{0.739656in}}%
\pgfpathlineto{\pgfqpoint{3.106043in}{0.739656in}}%
\pgfpathlineto{\pgfqpoint{3.105746in}{0.739656in}}%
\pgfpathlineto{\pgfqpoint{3.105448in}{0.739656in}}%
\pgfpathlineto{\pgfqpoint{3.105151in}{0.739656in}}%
\pgfpathlineto{\pgfqpoint{3.104853in}{0.739656in}}%
\pgfpathlineto{\pgfqpoint{3.104556in}{0.739656in}}%
\pgfpathlineto{\pgfqpoint{3.104259in}{0.739656in}}%
\pgfpathlineto{\pgfqpoint{3.103961in}{0.739656in}}%
\pgfpathlineto{\pgfqpoint{3.103664in}{0.739656in}}%
\pgfpathlineto{\pgfqpoint{3.103366in}{0.739656in}}%
\pgfpathlineto{\pgfqpoint{3.103069in}{0.739656in}}%
\pgfpathlineto{\pgfqpoint{3.102771in}{0.739656in}}%
\pgfpathlineto{\pgfqpoint{3.102474in}{0.739656in}}%
\pgfpathlineto{\pgfqpoint{3.102176in}{0.739656in}}%
\pgfpathlineto{\pgfqpoint{3.101879in}{0.739656in}}%
\pgfpathlineto{\pgfqpoint{3.101581in}{0.739656in}}%
\pgfpathlineto{\pgfqpoint{3.101284in}{0.739656in}}%
\pgfpathlineto{\pgfqpoint{3.100986in}{0.739656in}}%
\pgfpathlineto{\pgfqpoint{3.100689in}{0.739656in}}%
\pgfpathlineto{\pgfqpoint{3.100391in}{0.739656in}}%
\pgfpathlineto{\pgfqpoint{3.100094in}{0.739656in}}%
\pgfpathlineto{\pgfqpoint{3.099796in}{0.739656in}}%
\pgfpathlineto{\pgfqpoint{3.099499in}{0.739656in}}%
\pgfpathlineto{\pgfqpoint{3.099201in}{0.739656in}}%
\pgfpathlineto{\pgfqpoint{3.098904in}{0.739656in}}%
\pgfpathlineto{\pgfqpoint{3.098606in}{0.739656in}}%
\pgfpathlineto{\pgfqpoint{3.098309in}{0.739656in}}%
\pgfpathlineto{\pgfqpoint{3.098011in}{0.739656in}}%
\pgfpathlineto{\pgfqpoint{3.097714in}{0.739656in}}%
\pgfpathlineto{\pgfqpoint{3.097417in}{0.739656in}}%
\pgfpathlineto{\pgfqpoint{3.097119in}{0.739656in}}%
\pgfpathlineto{\pgfqpoint{3.096822in}{0.739656in}}%
\pgfpathlineto{\pgfqpoint{3.096524in}{0.739656in}}%
\pgfpathlineto{\pgfqpoint{3.096227in}{0.739656in}}%
\pgfpathlineto{\pgfqpoint{3.095929in}{0.739656in}}%
\pgfpathlineto{\pgfqpoint{3.095632in}{0.739656in}}%
\pgfpathlineto{\pgfqpoint{3.095334in}{0.739656in}}%
\pgfpathlineto{\pgfqpoint{3.095037in}{0.739656in}}%
\pgfpathlineto{\pgfqpoint{3.094739in}{0.739656in}}%
\pgfpathlineto{\pgfqpoint{3.094442in}{0.739656in}}%
\pgfpathlineto{\pgfqpoint{3.094144in}{0.739656in}}%
\pgfpathlineto{\pgfqpoint{3.093847in}{0.739656in}}%
\pgfpathlineto{\pgfqpoint{3.093549in}{0.739656in}}%
\pgfpathlineto{\pgfqpoint{3.093252in}{0.739656in}}%
\pgfpathlineto{\pgfqpoint{3.092954in}{0.739656in}}%
\pgfpathlineto{\pgfqpoint{3.092657in}{0.739656in}}%
\pgfpathlineto{\pgfqpoint{3.092359in}{0.739656in}}%
\pgfpathlineto{\pgfqpoint{3.092062in}{0.739656in}}%
\pgfpathlineto{\pgfqpoint{3.091764in}{0.739656in}}%
\pgfpathlineto{\pgfqpoint{3.091467in}{0.739656in}}%
\pgfpathlineto{\pgfqpoint{3.091169in}{0.739656in}}%
\pgfpathlineto{\pgfqpoint{3.090872in}{0.739656in}}%
\pgfpathlineto{\pgfqpoint{3.090575in}{0.739656in}}%
\pgfpathlineto{\pgfqpoint{3.090277in}{0.739656in}}%
\pgfpathlineto{\pgfqpoint{3.089980in}{0.739656in}}%
\pgfpathlineto{\pgfqpoint{3.089682in}{0.739656in}}%
\pgfpathlineto{\pgfqpoint{3.089385in}{0.739656in}}%
\pgfpathlineto{\pgfqpoint{3.089087in}{0.739656in}}%
\pgfpathlineto{\pgfqpoint{3.088790in}{0.739656in}}%
\pgfpathlineto{\pgfqpoint{3.088492in}{0.739656in}}%
\pgfpathlineto{\pgfqpoint{3.088195in}{0.739656in}}%
\pgfpathlineto{\pgfqpoint{3.087897in}{0.739656in}}%
\pgfpathlineto{\pgfqpoint{3.087600in}{0.739656in}}%
\pgfpathlineto{\pgfqpoint{3.087302in}{0.739656in}}%
\pgfpathlineto{\pgfqpoint{3.087005in}{0.739656in}}%
\pgfpathlineto{\pgfqpoint{3.086707in}{0.739656in}}%
\pgfpathlineto{\pgfqpoint{3.086410in}{0.739656in}}%
\pgfpathlineto{\pgfqpoint{3.086112in}{0.739656in}}%
\pgfpathlineto{\pgfqpoint{3.085815in}{0.739656in}}%
\pgfpathlineto{\pgfqpoint{3.085517in}{0.739656in}}%
\pgfpathlineto{\pgfqpoint{3.085220in}{0.739656in}}%
\pgfpathlineto{\pgfqpoint{3.084922in}{0.739656in}}%
\pgfpathlineto{\pgfqpoint{3.084625in}{0.739656in}}%
\pgfpathlineto{\pgfqpoint{3.084328in}{0.739656in}}%
\pgfpathlineto{\pgfqpoint{3.084030in}{0.739656in}}%
\pgfpathlineto{\pgfqpoint{3.083733in}{0.739656in}}%
\pgfpathlineto{\pgfqpoint{3.083435in}{0.739656in}}%
\pgfpathlineto{\pgfqpoint{3.083138in}{0.739656in}}%
\pgfpathlineto{\pgfqpoint{3.082840in}{0.739656in}}%
\pgfpathlineto{\pgfqpoint{3.082543in}{0.739656in}}%
\pgfpathlineto{\pgfqpoint{3.082245in}{0.739656in}}%
\pgfpathlineto{\pgfqpoint{3.081948in}{0.739656in}}%
\pgfpathlineto{\pgfqpoint{3.081650in}{0.739656in}}%
\pgfpathlineto{\pgfqpoint{3.081353in}{0.739656in}}%
\pgfpathlineto{\pgfqpoint{3.081055in}{0.739656in}}%
\pgfpathlineto{\pgfqpoint{3.080758in}{0.739656in}}%
\pgfpathlineto{\pgfqpoint{3.080460in}{0.739656in}}%
\pgfpathlineto{\pgfqpoint{3.080163in}{0.739656in}}%
\pgfpathlineto{\pgfqpoint{3.079865in}{0.739656in}}%
\pgfpathlineto{\pgfqpoint{3.079568in}{0.739656in}}%
\pgfpathlineto{\pgfqpoint{3.079270in}{0.739656in}}%
\pgfpathlineto{\pgfqpoint{3.078973in}{0.739656in}}%
\pgfpathlineto{\pgfqpoint{3.078675in}{0.739656in}}%
\pgfpathlineto{\pgfqpoint{3.078378in}{0.739656in}}%
\pgfpathlineto{\pgfqpoint{3.078080in}{0.739656in}}%
\pgfpathlineto{\pgfqpoint{3.077783in}{0.739656in}}%
\pgfpathlineto{\pgfqpoint{3.077486in}{0.739656in}}%
\pgfpathlineto{\pgfqpoint{3.077188in}{0.739656in}}%
\pgfpathlineto{\pgfqpoint{3.076891in}{0.739656in}}%
\pgfpathlineto{\pgfqpoint{3.076593in}{0.739656in}}%
\pgfpathlineto{\pgfqpoint{3.076296in}{0.739656in}}%
\pgfpathlineto{\pgfqpoint{3.075998in}{0.739656in}}%
\pgfpathlineto{\pgfqpoint{3.075701in}{0.739656in}}%
\pgfpathlineto{\pgfqpoint{3.075403in}{0.739656in}}%
\pgfpathlineto{\pgfqpoint{3.075106in}{0.739656in}}%
\pgfpathlineto{\pgfqpoint{3.074808in}{0.739656in}}%
\pgfpathlineto{\pgfqpoint{3.074511in}{0.739656in}}%
\pgfpathlineto{\pgfqpoint{3.074213in}{0.739656in}}%
\pgfpathlineto{\pgfqpoint{3.073916in}{0.739656in}}%
\pgfpathlineto{\pgfqpoint{3.073618in}{0.739656in}}%
\pgfpathlineto{\pgfqpoint{3.073321in}{0.739656in}}%
\pgfpathlineto{\pgfqpoint{3.073023in}{0.739656in}}%
\pgfpathlineto{\pgfqpoint{3.072726in}{0.739656in}}%
\pgfpathlineto{\pgfqpoint{3.072428in}{0.739656in}}%
\pgfpathlineto{\pgfqpoint{3.072131in}{0.739656in}}%
\pgfpathlineto{\pgfqpoint{3.071833in}{0.739656in}}%
\pgfpathlineto{\pgfqpoint{3.071536in}{0.739656in}}%
\pgfpathlineto{\pgfqpoint{3.071238in}{0.739656in}}%
\pgfpathlineto{\pgfqpoint{3.070941in}{0.739656in}}%
\pgfpathlineto{\pgfqpoint{3.070644in}{0.739656in}}%
\pgfpathlineto{\pgfqpoint{3.070346in}{0.739656in}}%
\pgfpathlineto{\pgfqpoint{3.070049in}{0.739656in}}%
\pgfpathlineto{\pgfqpoint{3.069751in}{0.739656in}}%
\pgfpathlineto{\pgfqpoint{3.069454in}{0.739656in}}%
\pgfpathlineto{\pgfqpoint{3.069156in}{0.739656in}}%
\pgfpathlineto{\pgfqpoint{3.068859in}{0.739656in}}%
\pgfpathlineto{\pgfqpoint{3.068561in}{0.739656in}}%
\pgfpathlineto{\pgfqpoint{3.068264in}{0.739656in}}%
\pgfpathlineto{\pgfqpoint{3.067966in}{0.739656in}}%
\pgfpathlineto{\pgfqpoint{3.067669in}{0.739656in}}%
\pgfpathlineto{\pgfqpoint{3.067371in}{0.739656in}}%
\pgfpathlineto{\pgfqpoint{3.067074in}{0.739656in}}%
\pgfpathlineto{\pgfqpoint{3.066776in}{0.739656in}}%
\pgfpathlineto{\pgfqpoint{3.066479in}{0.739656in}}%
\pgfpathlineto{\pgfqpoint{3.066181in}{0.739656in}}%
\pgfpathlineto{\pgfqpoint{3.065884in}{0.739656in}}%
\pgfpathlineto{\pgfqpoint{3.065586in}{0.739656in}}%
\pgfpathlineto{\pgfqpoint{3.065289in}{0.739656in}}%
\pgfpathlineto{\pgfqpoint{3.064991in}{0.739656in}}%
\pgfpathlineto{\pgfqpoint{3.064694in}{0.739656in}}%
\pgfpathlineto{\pgfqpoint{3.064397in}{0.739656in}}%
\pgfpathlineto{\pgfqpoint{3.064099in}{0.739656in}}%
\pgfpathlineto{\pgfqpoint{3.063802in}{0.739656in}}%
\pgfpathlineto{\pgfqpoint{3.063504in}{0.739656in}}%
\pgfpathlineto{\pgfqpoint{3.063207in}{0.739656in}}%
\pgfpathlineto{\pgfqpoint{3.062909in}{0.739656in}}%
\pgfpathlineto{\pgfqpoint{3.062612in}{0.739656in}}%
\pgfpathlineto{\pgfqpoint{3.062314in}{0.739656in}}%
\pgfpathlineto{\pgfqpoint{3.062017in}{0.739656in}}%
\pgfpathlineto{\pgfqpoint{3.061719in}{0.739656in}}%
\pgfpathlineto{\pgfqpoint{3.061422in}{0.739656in}}%
\pgfpathlineto{\pgfqpoint{3.061124in}{0.739656in}}%
\pgfpathlineto{\pgfqpoint{3.060827in}{0.739656in}}%
\pgfpathlineto{\pgfqpoint{3.060529in}{0.739656in}}%
\pgfpathlineto{\pgfqpoint{3.060232in}{0.739656in}}%
\pgfpathlineto{\pgfqpoint{3.059934in}{0.739656in}}%
\pgfpathlineto{\pgfqpoint{3.059637in}{0.739656in}}%
\pgfpathlineto{\pgfqpoint{3.059339in}{0.739656in}}%
\pgfpathlineto{\pgfqpoint{3.059042in}{0.739656in}}%
\pgfpathlineto{\pgfqpoint{3.058744in}{0.739656in}}%
\pgfpathlineto{\pgfqpoint{3.058447in}{0.739656in}}%
\pgfpathlineto{\pgfqpoint{3.058149in}{0.739656in}}%
\pgfpathlineto{\pgfqpoint{3.057852in}{0.739656in}}%
\pgfpathlineto{\pgfqpoint{3.057555in}{0.739656in}}%
\pgfpathlineto{\pgfqpoint{3.057257in}{0.739656in}}%
\pgfpathlineto{\pgfqpoint{3.056960in}{0.739656in}}%
\pgfpathlineto{\pgfqpoint{3.056662in}{0.739656in}}%
\pgfpathlineto{\pgfqpoint{3.056365in}{0.739656in}}%
\pgfpathlineto{\pgfqpoint{3.056067in}{0.739656in}}%
\pgfpathlineto{\pgfqpoint{3.055770in}{0.739656in}}%
\pgfpathlineto{\pgfqpoint{3.055472in}{0.739656in}}%
\pgfpathlineto{\pgfqpoint{3.055175in}{0.739656in}}%
\pgfpathlineto{\pgfqpoint{3.054877in}{0.739656in}}%
\pgfpathlineto{\pgfqpoint{3.054580in}{0.739656in}}%
\pgfpathlineto{\pgfqpoint{3.054282in}{0.739656in}}%
\pgfpathlineto{\pgfqpoint{3.053985in}{0.739656in}}%
\pgfpathlineto{\pgfqpoint{3.053687in}{0.739656in}}%
\pgfpathlineto{\pgfqpoint{3.053390in}{0.739656in}}%
\pgfpathlineto{\pgfqpoint{3.053092in}{0.739656in}}%
\pgfpathlineto{\pgfqpoint{3.052795in}{0.739656in}}%
\pgfpathlineto{\pgfqpoint{3.052497in}{0.739656in}}%
\pgfpathlineto{\pgfqpoint{3.052200in}{0.739656in}}%
\pgfpathlineto{\pgfqpoint{3.051902in}{0.739656in}}%
\pgfpathlineto{\pgfqpoint{3.051605in}{0.739656in}}%
\pgfpathlineto{\pgfqpoint{3.051307in}{0.739656in}}%
\pgfpathlineto{\pgfqpoint{3.051010in}{0.739656in}}%
\pgfpathlineto{\pgfqpoint{3.050713in}{0.739656in}}%
\pgfpathlineto{\pgfqpoint{3.050415in}{0.739656in}}%
\pgfpathlineto{\pgfqpoint{3.050118in}{0.739656in}}%
\pgfpathlineto{\pgfqpoint{3.049820in}{0.739656in}}%
\pgfpathlineto{\pgfqpoint{3.049523in}{0.739656in}}%
\pgfpathlineto{\pgfqpoint{3.049225in}{0.739656in}}%
\pgfpathlineto{\pgfqpoint{3.048928in}{0.739656in}}%
\pgfpathlineto{\pgfqpoint{3.048630in}{0.739656in}}%
\pgfpathlineto{\pgfqpoint{3.048333in}{0.739656in}}%
\pgfpathlineto{\pgfqpoint{3.048035in}{0.739656in}}%
\pgfpathlineto{\pgfqpoint{3.047738in}{0.739656in}}%
\pgfpathlineto{\pgfqpoint{3.047440in}{0.739656in}}%
\pgfpathlineto{\pgfqpoint{3.047143in}{0.739656in}}%
\pgfpathlineto{\pgfqpoint{3.046845in}{0.739656in}}%
\pgfpathlineto{\pgfqpoint{3.046548in}{0.739656in}}%
\pgfpathlineto{\pgfqpoint{3.046250in}{0.739656in}}%
\pgfpathlineto{\pgfqpoint{3.045953in}{0.739656in}}%
\pgfpathlineto{\pgfqpoint{3.045655in}{0.739656in}}%
\pgfpathlineto{\pgfqpoint{3.045358in}{0.739656in}}%
\pgfpathlineto{\pgfqpoint{3.045060in}{0.739656in}}%
\pgfpathlineto{\pgfqpoint{3.044763in}{0.739656in}}%
\pgfpathlineto{\pgfqpoint{3.044466in}{0.739656in}}%
\pgfpathlineto{\pgfqpoint{3.044168in}{0.739656in}}%
\pgfpathlineto{\pgfqpoint{3.043871in}{0.739656in}}%
\pgfpathlineto{\pgfqpoint{3.043573in}{0.739656in}}%
\pgfpathlineto{\pgfqpoint{3.043276in}{0.739656in}}%
\pgfpathlineto{\pgfqpoint{3.042978in}{0.739656in}}%
\pgfpathlineto{\pgfqpoint{3.042681in}{0.739656in}}%
\pgfpathlineto{\pgfqpoint{3.042383in}{0.739656in}}%
\pgfpathlineto{\pgfqpoint{3.042086in}{0.739656in}}%
\pgfpathlineto{\pgfqpoint{3.041788in}{0.739656in}}%
\pgfpathlineto{\pgfqpoint{3.041491in}{0.739656in}}%
\pgfpathlineto{\pgfqpoint{3.041193in}{0.739656in}}%
\pgfpathlineto{\pgfqpoint{3.040896in}{0.739656in}}%
\pgfpathlineto{\pgfqpoint{3.040598in}{0.739656in}}%
\pgfpathlineto{\pgfqpoint{3.040301in}{0.739656in}}%
\pgfpathlineto{\pgfqpoint{3.040003in}{0.739656in}}%
\pgfpathlineto{\pgfqpoint{3.039706in}{0.739656in}}%
\pgfpathlineto{\pgfqpoint{3.039408in}{0.739656in}}%
\pgfpathlineto{\pgfqpoint{3.039111in}{0.739656in}}%
\pgfpathlineto{\pgfqpoint{3.038813in}{0.739656in}}%
\pgfpathlineto{\pgfqpoint{3.038516in}{0.739656in}}%
\pgfpathlineto{\pgfqpoint{3.038218in}{0.739656in}}%
\pgfpathlineto{\pgfqpoint{3.037921in}{0.739656in}}%
\pgfpathlineto{\pgfqpoint{3.037624in}{0.739656in}}%
\pgfpathlineto{\pgfqpoint{3.037326in}{0.739656in}}%
\pgfpathlineto{\pgfqpoint{3.037029in}{0.739656in}}%
\pgfpathlineto{\pgfqpoint{3.036731in}{0.739656in}}%
\pgfpathlineto{\pgfqpoint{3.036434in}{0.739656in}}%
\pgfpathlineto{\pgfqpoint{3.036136in}{0.739656in}}%
\pgfpathlineto{\pgfqpoint{3.035839in}{0.739656in}}%
\pgfpathlineto{\pgfqpoint{3.035541in}{0.739656in}}%
\pgfpathlineto{\pgfqpoint{3.035244in}{0.739656in}}%
\pgfpathlineto{\pgfqpoint{3.034946in}{0.739656in}}%
\pgfpathlineto{\pgfqpoint{3.034649in}{0.739656in}}%
\pgfpathlineto{\pgfqpoint{3.034351in}{0.739656in}}%
\pgfpathlineto{\pgfqpoint{3.034054in}{0.739656in}}%
\pgfpathlineto{\pgfqpoint{3.033756in}{0.739656in}}%
\pgfpathlineto{\pgfqpoint{3.033459in}{0.739656in}}%
\pgfpathlineto{\pgfqpoint{3.033161in}{0.739656in}}%
\pgfpathlineto{\pgfqpoint{3.032864in}{0.739656in}}%
\pgfpathlineto{\pgfqpoint{3.032566in}{0.739656in}}%
\pgfpathlineto{\pgfqpoint{3.032269in}{0.739656in}}%
\pgfpathlineto{\pgfqpoint{3.031971in}{0.739656in}}%
\pgfpathlineto{\pgfqpoint{3.031674in}{0.739656in}}%
\pgfpathlineto{\pgfqpoint{3.031376in}{0.739656in}}%
\pgfpathlineto{\pgfqpoint{3.031079in}{0.739656in}}%
\pgfpathlineto{\pgfqpoint{3.030782in}{0.739656in}}%
\pgfpathlineto{\pgfqpoint{3.030484in}{0.739656in}}%
\pgfpathlineto{\pgfqpoint{3.030187in}{0.739656in}}%
\pgfpathlineto{\pgfqpoint{3.029889in}{0.739656in}}%
\pgfpathlineto{\pgfqpoint{3.029592in}{0.739656in}}%
\pgfpathlineto{\pgfqpoint{3.029294in}{0.739656in}}%
\pgfpathlineto{\pgfqpoint{3.028997in}{0.739656in}}%
\pgfpathlineto{\pgfqpoint{3.028699in}{0.739656in}}%
\pgfpathlineto{\pgfqpoint{3.028402in}{0.739656in}}%
\pgfpathlineto{\pgfqpoint{3.028104in}{0.739656in}}%
\pgfpathlineto{\pgfqpoint{3.027807in}{0.739656in}}%
\pgfpathlineto{\pgfqpoint{3.027509in}{0.739656in}}%
\pgfpathlineto{\pgfqpoint{3.027212in}{0.739656in}}%
\pgfpathlineto{\pgfqpoint{3.026914in}{0.739656in}}%
\pgfpathlineto{\pgfqpoint{3.026617in}{0.739656in}}%
\pgfpathlineto{\pgfqpoint{3.026319in}{0.739656in}}%
\pgfpathlineto{\pgfqpoint{3.026022in}{0.739656in}}%
\pgfpathlineto{\pgfqpoint{3.025724in}{0.739656in}}%
\pgfpathlineto{\pgfqpoint{3.025427in}{0.739656in}}%
\pgfpathlineto{\pgfqpoint{3.025129in}{0.739656in}}%
\pgfpathlineto{\pgfqpoint{3.024832in}{0.739656in}}%
\pgfpathlineto{\pgfqpoint{3.024535in}{0.739656in}}%
\pgfpathlineto{\pgfqpoint{3.024237in}{0.739656in}}%
\pgfpathlineto{\pgfqpoint{3.023940in}{0.739656in}}%
\pgfpathlineto{\pgfqpoint{3.023642in}{0.739656in}}%
\pgfpathlineto{\pgfqpoint{3.023345in}{0.739656in}}%
\pgfpathlineto{\pgfqpoint{3.023047in}{0.739656in}}%
\pgfpathlineto{\pgfqpoint{3.022750in}{0.739656in}}%
\pgfpathlineto{\pgfqpoint{3.022452in}{0.739656in}}%
\pgfpathlineto{\pgfqpoint{3.022155in}{0.739656in}}%
\pgfpathlineto{\pgfqpoint{3.021857in}{0.739656in}}%
\pgfpathlineto{\pgfqpoint{3.021560in}{0.739656in}}%
\pgfpathlineto{\pgfqpoint{3.021262in}{0.739656in}}%
\pgfpathlineto{\pgfqpoint{3.020965in}{0.739656in}}%
\pgfpathlineto{\pgfqpoint{3.020667in}{0.739656in}}%
\pgfpathlineto{\pgfqpoint{3.020370in}{0.739656in}}%
\pgfpathlineto{\pgfqpoint{3.020072in}{0.739656in}}%
\pgfpathlineto{\pgfqpoint{3.019775in}{0.739656in}}%
\pgfpathlineto{\pgfqpoint{3.019477in}{0.739656in}}%
\pgfpathlineto{\pgfqpoint{3.019180in}{0.739656in}}%
\pgfpathlineto{\pgfqpoint{3.018882in}{0.739656in}}%
\pgfpathlineto{\pgfqpoint{3.018585in}{0.739656in}}%
\pgfpathlineto{\pgfqpoint{3.018287in}{0.739656in}}%
\pgfpathlineto{\pgfqpoint{3.017990in}{0.739656in}}%
\pgfpathlineto{\pgfqpoint{3.017693in}{0.739656in}}%
\pgfpathlineto{\pgfqpoint{3.017395in}{0.739656in}}%
\pgfpathlineto{\pgfqpoint{3.017098in}{0.739656in}}%
\pgfpathlineto{\pgfqpoint{3.016800in}{0.739656in}}%
\pgfpathlineto{\pgfqpoint{3.016503in}{0.739656in}}%
\pgfpathlineto{\pgfqpoint{3.016205in}{0.739656in}}%
\pgfpathlineto{\pgfqpoint{3.015908in}{0.739656in}}%
\pgfpathlineto{\pgfqpoint{3.015610in}{0.739656in}}%
\pgfpathlineto{\pgfqpoint{3.015313in}{0.739656in}}%
\pgfpathlineto{\pgfqpoint{3.015015in}{0.739656in}}%
\pgfpathlineto{\pgfqpoint{3.014718in}{0.739656in}}%
\pgfpathlineto{\pgfqpoint{3.014420in}{0.739656in}}%
\pgfpathlineto{\pgfqpoint{3.014123in}{0.739656in}}%
\pgfpathlineto{\pgfqpoint{3.013825in}{0.739656in}}%
\pgfpathlineto{\pgfqpoint{3.013528in}{0.739656in}}%
\pgfpathlineto{\pgfqpoint{3.013230in}{0.739656in}}%
\pgfpathlineto{\pgfqpoint{3.012933in}{0.739656in}}%
\pgfpathlineto{\pgfqpoint{3.012635in}{0.739656in}}%
\pgfpathlineto{\pgfqpoint{3.012338in}{0.739656in}}%
\pgfpathlineto{\pgfqpoint{3.012040in}{0.739656in}}%
\pgfpathlineto{\pgfqpoint{3.011743in}{0.739656in}}%
\pgfpathlineto{\pgfqpoint{3.011445in}{0.739656in}}%
\pgfpathlineto{\pgfqpoint{3.011148in}{0.739656in}}%
\pgfpathlineto{\pgfqpoint{3.010851in}{0.739656in}}%
\pgfpathlineto{\pgfqpoint{3.010553in}{0.739656in}}%
\pgfpathlineto{\pgfqpoint{3.010256in}{0.739656in}}%
\pgfpathlineto{\pgfqpoint{3.009958in}{0.739656in}}%
\pgfpathlineto{\pgfqpoint{3.009661in}{0.739656in}}%
\pgfpathlineto{\pgfqpoint{3.009363in}{0.739656in}}%
\pgfpathlineto{\pgfqpoint{3.009066in}{0.739656in}}%
\pgfpathlineto{\pgfqpoint{3.008768in}{0.739656in}}%
\pgfpathlineto{\pgfqpoint{3.008471in}{0.739656in}}%
\pgfpathlineto{\pgfqpoint{3.008173in}{0.739656in}}%
\pgfpathlineto{\pgfqpoint{3.007876in}{0.739656in}}%
\pgfpathlineto{\pgfqpoint{3.007578in}{0.739656in}}%
\pgfpathlineto{\pgfqpoint{3.007281in}{0.739656in}}%
\pgfpathlineto{\pgfqpoint{3.006983in}{0.739656in}}%
\pgfpathlineto{\pgfqpoint{3.006686in}{0.739656in}}%
\pgfpathlineto{\pgfqpoint{3.006388in}{0.739656in}}%
\pgfpathlineto{\pgfqpoint{3.006091in}{0.739656in}}%
\pgfpathlineto{\pgfqpoint{3.005793in}{0.739656in}}%
\pgfpathlineto{\pgfqpoint{3.005496in}{0.739656in}}%
\pgfpathlineto{\pgfqpoint{3.005198in}{0.739656in}}%
\pgfpathlineto{\pgfqpoint{3.004901in}{0.739656in}}%
\pgfpathlineto{\pgfqpoint{3.004604in}{0.739656in}}%
\pgfpathlineto{\pgfqpoint{3.004306in}{0.739656in}}%
\pgfpathlineto{\pgfqpoint{3.004009in}{0.739656in}}%
\pgfpathlineto{\pgfqpoint{3.003711in}{0.739656in}}%
\pgfpathlineto{\pgfqpoint{3.003414in}{0.739656in}}%
\pgfpathlineto{\pgfqpoint{3.003116in}{0.739656in}}%
\pgfpathlineto{\pgfqpoint{3.002819in}{0.739656in}}%
\pgfpathlineto{\pgfqpoint{3.002521in}{0.739656in}}%
\pgfpathlineto{\pgfqpoint{3.002224in}{0.739656in}}%
\pgfpathlineto{\pgfqpoint{3.001926in}{0.739656in}}%
\pgfpathlineto{\pgfqpoint{3.001629in}{0.739656in}}%
\pgfpathlineto{\pgfqpoint{3.001331in}{0.739656in}}%
\pgfpathlineto{\pgfqpoint{3.001034in}{0.739656in}}%
\pgfpathlineto{\pgfqpoint{3.000736in}{0.739656in}}%
\pgfpathlineto{\pgfqpoint{3.000439in}{0.739656in}}%
\pgfpathlineto{\pgfqpoint{3.000141in}{0.739656in}}%
\pgfpathlineto{\pgfqpoint{2.999844in}{0.739656in}}%
\pgfpathlineto{\pgfqpoint{2.999546in}{0.739656in}}%
\pgfpathlineto{\pgfqpoint{2.999249in}{0.739656in}}%
\pgfpathlineto{\pgfqpoint{2.998951in}{0.739656in}}%
\pgfpathlineto{\pgfqpoint{2.998654in}{0.739656in}}%
\pgfpathlineto{\pgfqpoint{2.998356in}{0.739656in}}%
\pgfpathlineto{\pgfqpoint{2.998059in}{0.739656in}}%
\pgfpathlineto{\pgfqpoint{2.997762in}{0.739656in}}%
\pgfpathlineto{\pgfqpoint{2.997464in}{0.739656in}}%
\pgfpathlineto{\pgfqpoint{2.997167in}{0.739656in}}%
\pgfpathlineto{\pgfqpoint{2.996869in}{0.739656in}}%
\pgfpathlineto{\pgfqpoint{2.996572in}{0.739656in}}%
\pgfpathlineto{\pgfqpoint{2.996274in}{0.739656in}}%
\pgfpathlineto{\pgfqpoint{2.995977in}{0.739656in}}%
\pgfpathlineto{\pgfqpoint{2.995679in}{0.739656in}}%
\pgfpathlineto{\pgfqpoint{2.995382in}{0.739656in}}%
\pgfpathlineto{\pgfqpoint{2.995084in}{0.739656in}}%
\pgfpathlineto{\pgfqpoint{2.994787in}{0.739656in}}%
\pgfpathlineto{\pgfqpoint{2.994489in}{0.739656in}}%
\pgfpathlineto{\pgfqpoint{2.994192in}{0.739656in}}%
\pgfpathlineto{\pgfqpoint{2.993894in}{0.739656in}}%
\pgfpathlineto{\pgfqpoint{2.993597in}{0.739656in}}%
\pgfpathlineto{\pgfqpoint{2.993299in}{0.739656in}}%
\pgfpathlineto{\pgfqpoint{2.993002in}{0.739656in}}%
\pgfpathlineto{\pgfqpoint{2.992704in}{0.739656in}}%
\pgfpathlineto{\pgfqpoint{2.992407in}{0.739656in}}%
\pgfpathlineto{\pgfqpoint{2.992109in}{0.739656in}}%
\pgfpathlineto{\pgfqpoint{2.991812in}{0.739656in}}%
\pgfpathlineto{\pgfqpoint{2.991514in}{0.739656in}}%
\pgfpathlineto{\pgfqpoint{2.991217in}{0.739656in}}%
\pgfpathlineto{\pgfqpoint{2.990920in}{0.739656in}}%
\pgfpathlineto{\pgfqpoint{2.990622in}{0.739656in}}%
\pgfpathlineto{\pgfqpoint{2.990325in}{0.739656in}}%
\pgfpathlineto{\pgfqpoint{2.990027in}{0.739656in}}%
\pgfpathlineto{\pgfqpoint{2.989730in}{0.739656in}}%
\pgfpathlineto{\pgfqpoint{2.989432in}{0.739656in}}%
\pgfpathlineto{\pgfqpoint{2.989135in}{0.739656in}}%
\pgfpathlineto{\pgfqpoint{2.988837in}{0.739656in}}%
\pgfpathlineto{\pgfqpoint{2.988540in}{0.739656in}}%
\pgfpathlineto{\pgfqpoint{2.988242in}{0.739656in}}%
\pgfpathlineto{\pgfqpoint{2.987945in}{0.739656in}}%
\pgfpathlineto{\pgfqpoint{2.987647in}{0.739656in}}%
\pgfpathlineto{\pgfqpoint{2.987350in}{0.739656in}}%
\pgfpathlineto{\pgfqpoint{2.987052in}{0.739656in}}%
\pgfpathlineto{\pgfqpoint{2.986755in}{0.739656in}}%
\pgfpathlineto{\pgfqpoint{2.986457in}{0.739656in}}%
\pgfpathlineto{\pgfqpoint{2.986160in}{0.739656in}}%
\pgfpathlineto{\pgfqpoint{2.985862in}{0.739656in}}%
\pgfpathlineto{\pgfqpoint{2.985565in}{0.739656in}}%
\pgfpathlineto{\pgfqpoint{2.985267in}{0.739656in}}%
\pgfpathlineto{\pgfqpoint{2.984970in}{0.739656in}}%
\pgfpathlineto{\pgfqpoint{2.984673in}{0.739656in}}%
\pgfpathlineto{\pgfqpoint{2.984375in}{0.739656in}}%
\pgfpathlineto{\pgfqpoint{2.984078in}{0.739656in}}%
\pgfpathlineto{\pgfqpoint{2.983780in}{0.739656in}}%
\pgfpathlineto{\pgfqpoint{2.983483in}{0.739656in}}%
\pgfpathlineto{\pgfqpoint{2.983185in}{0.739656in}}%
\pgfpathlineto{\pgfqpoint{2.982888in}{0.739656in}}%
\pgfpathlineto{\pgfqpoint{2.982590in}{0.739656in}}%
\pgfpathlineto{\pgfqpoint{2.982293in}{0.739656in}}%
\pgfpathlineto{\pgfqpoint{2.981995in}{0.739656in}}%
\pgfpathlineto{\pgfqpoint{2.981698in}{0.739656in}}%
\pgfpathlineto{\pgfqpoint{2.981400in}{0.739656in}}%
\pgfpathlineto{\pgfqpoint{2.981103in}{0.739656in}}%
\pgfpathlineto{\pgfqpoint{2.980805in}{0.739656in}}%
\pgfpathlineto{\pgfqpoint{2.980508in}{0.739656in}}%
\pgfpathlineto{\pgfqpoint{2.980210in}{0.739656in}}%
\pgfpathlineto{\pgfqpoint{2.979913in}{0.739656in}}%
\pgfpathlineto{\pgfqpoint{2.979615in}{0.739656in}}%
\pgfpathlineto{\pgfqpoint{2.979318in}{0.739656in}}%
\pgfpathlineto{\pgfqpoint{2.979020in}{0.739656in}}%
\pgfpathlineto{\pgfqpoint{2.978723in}{0.739656in}}%
\pgfpathlineto{\pgfqpoint{2.978425in}{0.739656in}}%
\pgfpathlineto{\pgfqpoint{2.978128in}{0.739656in}}%
\pgfpathlineto{\pgfqpoint{2.977831in}{0.739656in}}%
\pgfpathlineto{\pgfqpoint{2.977533in}{0.739656in}}%
\pgfpathlineto{\pgfqpoint{2.977236in}{0.739656in}}%
\pgfpathlineto{\pgfqpoint{2.976938in}{0.739656in}}%
\pgfpathlineto{\pgfqpoint{2.976641in}{0.739656in}}%
\pgfpathlineto{\pgfqpoint{2.976343in}{0.739656in}}%
\pgfpathlineto{\pgfqpoint{2.976046in}{0.739656in}}%
\pgfpathlineto{\pgfqpoint{2.975748in}{0.739656in}}%
\pgfpathlineto{\pgfqpoint{2.975451in}{0.739656in}}%
\pgfpathlineto{\pgfqpoint{2.975153in}{0.739656in}}%
\pgfpathlineto{\pgfqpoint{2.974856in}{0.739656in}}%
\pgfpathlineto{\pgfqpoint{2.974558in}{0.739656in}}%
\pgfpathlineto{\pgfqpoint{2.974261in}{0.739656in}}%
\pgfpathlineto{\pgfqpoint{2.973963in}{0.739656in}}%
\pgfpathlineto{\pgfqpoint{2.973666in}{0.739656in}}%
\pgfpathlineto{\pgfqpoint{2.973368in}{0.739656in}}%
\pgfpathlineto{\pgfqpoint{2.973071in}{0.739656in}}%
\pgfpathlineto{\pgfqpoint{2.972773in}{0.739656in}}%
\pgfpathlineto{\pgfqpoint{2.972476in}{0.739656in}}%
\pgfpathlineto{\pgfqpoint{2.972178in}{0.739656in}}%
\pgfpathlineto{\pgfqpoint{2.971881in}{0.739656in}}%
\pgfpathlineto{\pgfqpoint{2.971583in}{0.739656in}}%
\pgfpathlineto{\pgfqpoint{2.971286in}{0.739656in}}%
\pgfpathlineto{\pgfqpoint{2.970989in}{0.739656in}}%
\pgfpathlineto{\pgfqpoint{2.970691in}{0.739656in}}%
\pgfpathlineto{\pgfqpoint{2.970394in}{0.739656in}}%
\pgfpathlineto{\pgfqpoint{2.970096in}{0.739656in}}%
\pgfpathlineto{\pgfqpoint{2.969799in}{0.739656in}}%
\pgfpathlineto{\pgfqpoint{2.969501in}{0.739656in}}%
\pgfpathlineto{\pgfqpoint{2.969204in}{0.739656in}}%
\pgfpathlineto{\pgfqpoint{2.968906in}{0.739656in}}%
\pgfpathlineto{\pgfqpoint{2.968609in}{0.739656in}}%
\pgfpathlineto{\pgfqpoint{2.968311in}{0.739656in}}%
\pgfpathlineto{\pgfqpoint{2.968014in}{0.739656in}}%
\pgfpathlineto{\pgfqpoint{2.967716in}{0.739656in}}%
\pgfpathlineto{\pgfqpoint{2.967419in}{0.739656in}}%
\pgfpathlineto{\pgfqpoint{2.967121in}{0.739656in}}%
\pgfpathlineto{\pgfqpoint{2.966824in}{0.739656in}}%
\pgfpathlineto{\pgfqpoint{2.966526in}{0.739656in}}%
\pgfpathlineto{\pgfqpoint{2.966229in}{0.739656in}}%
\pgfpathlineto{\pgfqpoint{2.965931in}{0.739656in}}%
\pgfpathlineto{\pgfqpoint{2.965634in}{0.739656in}}%
\pgfpathlineto{\pgfqpoint{2.965336in}{0.739656in}}%
\pgfpathlineto{\pgfqpoint{2.965039in}{0.739656in}}%
\pgfpathlineto{\pgfqpoint{2.964742in}{0.739656in}}%
\pgfpathlineto{\pgfqpoint{2.964444in}{0.739656in}}%
\pgfpathlineto{\pgfqpoint{2.964147in}{0.739656in}}%
\pgfpathlineto{\pgfqpoint{2.963849in}{0.739656in}}%
\pgfpathlineto{\pgfqpoint{2.963552in}{0.739656in}}%
\pgfpathlineto{\pgfqpoint{2.963254in}{0.739656in}}%
\pgfpathlineto{\pgfqpoint{2.962957in}{0.739656in}}%
\pgfpathlineto{\pgfqpoint{2.962659in}{0.739656in}}%
\pgfpathlineto{\pgfqpoint{2.962362in}{0.739656in}}%
\pgfpathlineto{\pgfqpoint{2.962064in}{0.739656in}}%
\pgfpathlineto{\pgfqpoint{2.961767in}{0.739656in}}%
\pgfpathlineto{\pgfqpoint{2.961469in}{0.739656in}}%
\pgfpathlineto{\pgfqpoint{2.961172in}{0.739656in}}%
\pgfpathlineto{\pgfqpoint{2.960874in}{0.739656in}}%
\pgfpathlineto{\pgfqpoint{2.960577in}{0.739656in}}%
\pgfpathlineto{\pgfqpoint{2.960279in}{0.739656in}}%
\pgfpathlineto{\pgfqpoint{2.959982in}{0.739656in}}%
\pgfpathlineto{\pgfqpoint{2.959684in}{0.739656in}}%
\pgfpathlineto{\pgfqpoint{2.959387in}{0.739656in}}%
\pgfpathlineto{\pgfqpoint{2.959089in}{0.739656in}}%
\pgfpathlineto{\pgfqpoint{2.958792in}{0.739656in}}%
\pgfpathlineto{\pgfqpoint{2.958494in}{0.739656in}}%
\pgfpathlineto{\pgfqpoint{2.958197in}{0.739656in}}%
\pgfpathlineto{\pgfqpoint{2.957900in}{0.739656in}}%
\pgfpathlineto{\pgfqpoint{2.957602in}{0.739656in}}%
\pgfpathlineto{\pgfqpoint{2.957305in}{0.739656in}}%
\pgfpathlineto{\pgfqpoint{2.957007in}{0.739656in}}%
\pgfpathlineto{\pgfqpoint{2.956710in}{0.739656in}}%
\pgfpathlineto{\pgfqpoint{2.956412in}{0.739656in}}%
\pgfpathlineto{\pgfqpoint{2.956115in}{0.739656in}}%
\pgfpathlineto{\pgfqpoint{2.955817in}{0.739656in}}%
\pgfpathlineto{\pgfqpoint{2.955520in}{0.739656in}}%
\pgfpathlineto{\pgfqpoint{2.955222in}{0.739656in}}%
\pgfpathlineto{\pgfqpoint{2.954925in}{0.739656in}}%
\pgfpathlineto{\pgfqpoint{2.954627in}{0.739656in}}%
\pgfpathlineto{\pgfqpoint{2.954330in}{0.739656in}}%
\pgfpathlineto{\pgfqpoint{2.954032in}{0.739656in}}%
\pgfpathlineto{\pgfqpoint{2.953735in}{0.739656in}}%
\pgfpathlineto{\pgfqpoint{2.953437in}{0.739656in}}%
\pgfpathlineto{\pgfqpoint{2.953140in}{0.739656in}}%
\pgfpathlineto{\pgfqpoint{2.952842in}{0.739656in}}%
\pgfpathlineto{\pgfqpoint{2.952545in}{0.739656in}}%
\pgfpathlineto{\pgfqpoint{2.952247in}{0.739656in}}%
\pgfpathlineto{\pgfqpoint{2.951950in}{0.739656in}}%
\pgfpathlineto{\pgfqpoint{2.951652in}{0.739656in}}%
\pgfpathlineto{\pgfqpoint{2.951355in}{0.739656in}}%
\pgfpathlineto{\pgfqpoint{2.951058in}{0.739656in}}%
\pgfpathlineto{\pgfqpoint{2.950760in}{0.739656in}}%
\pgfpathlineto{\pgfqpoint{2.950463in}{0.739656in}}%
\pgfpathlineto{\pgfqpoint{2.950165in}{0.739656in}}%
\pgfpathlineto{\pgfqpoint{2.949868in}{0.739656in}}%
\pgfpathlineto{\pgfqpoint{2.949570in}{0.739656in}}%
\pgfpathlineto{\pgfqpoint{2.949273in}{0.739656in}}%
\pgfpathlineto{\pgfqpoint{2.948975in}{0.739656in}}%
\pgfpathlineto{\pgfqpoint{2.948678in}{0.739656in}}%
\pgfpathlineto{\pgfqpoint{2.948380in}{0.739656in}}%
\pgfpathlineto{\pgfqpoint{2.948083in}{0.739656in}}%
\pgfpathlineto{\pgfqpoint{2.947785in}{0.739656in}}%
\pgfpathlineto{\pgfqpoint{2.947488in}{0.739656in}}%
\pgfpathlineto{\pgfqpoint{2.947190in}{0.739656in}}%
\pgfpathlineto{\pgfqpoint{2.946893in}{0.739656in}}%
\pgfpathlineto{\pgfqpoint{2.946595in}{0.739656in}}%
\pgfpathlineto{\pgfqpoint{2.946298in}{0.739656in}}%
\pgfpathlineto{\pgfqpoint{2.946000in}{0.739656in}}%
\pgfpathlineto{\pgfqpoint{2.945703in}{0.739656in}}%
\pgfpathlineto{\pgfqpoint{2.945405in}{0.739656in}}%
\pgfpathlineto{\pgfqpoint{2.945108in}{0.739656in}}%
\pgfpathlineto{\pgfqpoint{2.944811in}{0.739656in}}%
\pgfpathlineto{\pgfqpoint{2.944513in}{0.739656in}}%
\pgfpathlineto{\pgfqpoint{2.944216in}{0.739656in}}%
\pgfpathlineto{\pgfqpoint{2.943918in}{0.739656in}}%
\pgfpathlineto{\pgfqpoint{2.943621in}{0.739656in}}%
\pgfpathlineto{\pgfqpoint{2.943323in}{0.739656in}}%
\pgfpathlineto{\pgfqpoint{2.943026in}{0.739656in}}%
\pgfpathlineto{\pgfqpoint{2.942728in}{0.739656in}}%
\pgfpathlineto{\pgfqpoint{2.942431in}{0.739656in}}%
\pgfpathlineto{\pgfqpoint{2.942133in}{0.739656in}}%
\pgfpathlineto{\pgfqpoint{2.941836in}{0.739656in}}%
\pgfpathlineto{\pgfqpoint{2.941538in}{0.739656in}}%
\pgfpathlineto{\pgfqpoint{2.941241in}{0.739656in}}%
\pgfpathlineto{\pgfqpoint{2.940943in}{0.739656in}}%
\pgfpathlineto{\pgfqpoint{2.940646in}{0.739656in}}%
\pgfpathlineto{\pgfqpoint{2.940348in}{0.739656in}}%
\pgfpathlineto{\pgfqpoint{2.940051in}{0.739656in}}%
\pgfpathlineto{\pgfqpoint{2.939753in}{0.739656in}}%
\pgfpathlineto{\pgfqpoint{2.939456in}{0.739656in}}%
\pgfpathlineto{\pgfqpoint{2.939158in}{0.739656in}}%
\pgfpathlineto{\pgfqpoint{2.938861in}{0.739656in}}%
\pgfpathlineto{\pgfqpoint{2.938563in}{0.739656in}}%
\pgfpathlineto{\pgfqpoint{2.938266in}{0.739656in}}%
\pgfpathlineto{\pgfqpoint{2.937969in}{0.739656in}}%
\pgfpathlineto{\pgfqpoint{2.937671in}{0.739656in}}%
\pgfpathlineto{\pgfqpoint{2.937374in}{0.739656in}}%
\pgfpathlineto{\pgfqpoint{2.937076in}{0.739656in}}%
\pgfpathlineto{\pgfqpoint{2.936779in}{0.739656in}}%
\pgfpathlineto{\pgfqpoint{2.936481in}{0.739656in}}%
\pgfpathlineto{\pgfqpoint{2.936184in}{0.739656in}}%
\pgfpathlineto{\pgfqpoint{2.935886in}{0.739656in}}%
\pgfpathlineto{\pgfqpoint{2.935589in}{0.739656in}}%
\pgfpathlineto{\pgfqpoint{2.935291in}{0.739656in}}%
\pgfpathlineto{\pgfqpoint{2.934994in}{0.739656in}}%
\pgfpathlineto{\pgfqpoint{2.934696in}{0.739656in}}%
\pgfpathlineto{\pgfqpoint{2.934399in}{0.739656in}}%
\pgfpathlineto{\pgfqpoint{2.934101in}{0.739656in}}%
\pgfpathlineto{\pgfqpoint{2.933804in}{0.739656in}}%
\pgfpathlineto{\pgfqpoint{2.933506in}{0.739656in}}%
\pgfpathlineto{\pgfqpoint{2.933209in}{0.739656in}}%
\pgfpathlineto{\pgfqpoint{2.932911in}{0.739656in}}%
\pgfpathlineto{\pgfqpoint{2.932614in}{0.739656in}}%
\pgfpathlineto{\pgfqpoint{2.932316in}{0.739656in}}%
\pgfpathlineto{\pgfqpoint{2.932019in}{0.739656in}}%
\pgfpathlineto{\pgfqpoint{2.931721in}{0.739656in}}%
\pgfpathlineto{\pgfqpoint{2.931424in}{0.739656in}}%
\pgfpathlineto{\pgfqpoint{2.931127in}{0.739656in}}%
\pgfpathlineto{\pgfqpoint{2.930829in}{0.739656in}}%
\pgfpathlineto{\pgfqpoint{2.930532in}{0.739656in}}%
\pgfpathlineto{\pgfqpoint{2.930234in}{0.739656in}}%
\pgfpathlineto{\pgfqpoint{2.929937in}{0.739656in}}%
\pgfpathlineto{\pgfqpoint{2.929639in}{0.739656in}}%
\pgfpathlineto{\pgfqpoint{2.929342in}{0.739656in}}%
\pgfpathlineto{\pgfqpoint{2.929044in}{0.739656in}}%
\pgfpathlineto{\pgfqpoint{2.928747in}{0.739656in}}%
\pgfpathlineto{\pgfqpoint{2.928449in}{0.739656in}}%
\pgfpathlineto{\pgfqpoint{2.928152in}{0.739656in}}%
\pgfpathlineto{\pgfqpoint{2.927854in}{0.739656in}}%
\pgfpathlineto{\pgfqpoint{2.927557in}{0.739656in}}%
\pgfpathlineto{\pgfqpoint{2.927259in}{0.739656in}}%
\pgfpathlineto{\pgfqpoint{2.926962in}{0.739656in}}%
\pgfpathlineto{\pgfqpoint{2.926664in}{0.739656in}}%
\pgfpathlineto{\pgfqpoint{2.926367in}{0.739656in}}%
\pgfpathlineto{\pgfqpoint{2.926069in}{0.739656in}}%
\pgfpathlineto{\pgfqpoint{2.925772in}{0.739656in}}%
\pgfpathlineto{\pgfqpoint{2.925474in}{0.739656in}}%
\pgfpathlineto{\pgfqpoint{2.925177in}{0.739656in}}%
\pgfpathlineto{\pgfqpoint{2.924880in}{0.739656in}}%
\pgfpathlineto{\pgfqpoint{2.924582in}{0.739656in}}%
\pgfpathlineto{\pgfqpoint{2.924285in}{0.739656in}}%
\pgfpathlineto{\pgfqpoint{2.923987in}{0.739656in}}%
\pgfpathlineto{\pgfqpoint{2.923690in}{0.739656in}}%
\pgfpathlineto{\pgfqpoint{2.923392in}{0.739656in}}%
\pgfpathlineto{\pgfqpoint{2.923095in}{0.739656in}}%
\pgfpathlineto{\pgfqpoint{2.922797in}{0.739656in}}%
\pgfpathlineto{\pgfqpoint{2.922500in}{0.739656in}}%
\pgfpathlineto{\pgfqpoint{2.922202in}{0.739656in}}%
\pgfpathlineto{\pgfqpoint{2.921905in}{0.739656in}}%
\pgfpathlineto{\pgfqpoint{2.921607in}{0.739656in}}%
\pgfpathlineto{\pgfqpoint{2.921310in}{0.739656in}}%
\pgfpathlineto{\pgfqpoint{2.921012in}{0.739656in}}%
\pgfpathlineto{\pgfqpoint{2.920715in}{0.739656in}}%
\pgfpathlineto{\pgfqpoint{2.920417in}{0.739656in}}%
\pgfpathlineto{\pgfqpoint{2.920120in}{0.739656in}}%
\pgfpathlineto{\pgfqpoint{2.919822in}{0.739656in}}%
\pgfpathlineto{\pgfqpoint{2.919525in}{0.739656in}}%
\pgfpathlineto{\pgfqpoint{2.919227in}{0.739656in}}%
\pgfpathlineto{\pgfqpoint{2.918930in}{0.739656in}}%
\pgfpathlineto{\pgfqpoint{2.918632in}{0.739656in}}%
\pgfpathlineto{\pgfqpoint{2.918335in}{0.739656in}}%
\pgfpathlineto{\pgfqpoint{2.918038in}{0.739656in}}%
\pgfpathlineto{\pgfqpoint{2.917740in}{0.739656in}}%
\pgfpathlineto{\pgfqpoint{2.917443in}{0.739656in}}%
\pgfpathlineto{\pgfqpoint{2.917145in}{0.739656in}}%
\pgfpathlineto{\pgfqpoint{2.916848in}{0.739656in}}%
\pgfpathlineto{\pgfqpoint{2.916550in}{0.739656in}}%
\pgfpathlineto{\pgfqpoint{2.916253in}{0.739656in}}%
\pgfpathlineto{\pgfqpoint{2.915955in}{0.739656in}}%
\pgfpathlineto{\pgfqpoint{2.915658in}{0.739656in}}%
\pgfpathlineto{\pgfqpoint{2.915360in}{0.739656in}}%
\pgfpathlineto{\pgfqpoint{2.915063in}{0.739656in}}%
\pgfpathlineto{\pgfqpoint{2.914765in}{0.739656in}}%
\pgfpathlineto{\pgfqpoint{2.914468in}{0.739656in}}%
\pgfpathlineto{\pgfqpoint{2.914170in}{0.739656in}}%
\pgfpathlineto{\pgfqpoint{2.913873in}{0.739656in}}%
\pgfpathlineto{\pgfqpoint{2.913575in}{0.739656in}}%
\pgfpathlineto{\pgfqpoint{2.913278in}{0.739656in}}%
\pgfpathlineto{\pgfqpoint{2.912980in}{0.739656in}}%
\pgfpathlineto{\pgfqpoint{2.912683in}{0.739656in}}%
\pgfpathlineto{\pgfqpoint{2.912385in}{0.739656in}}%
\pgfpathlineto{\pgfqpoint{2.912088in}{0.739656in}}%
\pgfpathlineto{\pgfqpoint{2.911790in}{0.739656in}}%
\pgfpathlineto{\pgfqpoint{2.911493in}{0.739656in}}%
\pgfpathlineto{\pgfqpoint{2.911196in}{0.739656in}}%
\pgfpathlineto{\pgfqpoint{2.910898in}{0.739656in}}%
\pgfpathlineto{\pgfqpoint{2.910601in}{0.739656in}}%
\pgfpathlineto{\pgfqpoint{2.910303in}{0.739656in}}%
\pgfpathlineto{\pgfqpoint{2.910006in}{0.739656in}}%
\pgfpathlineto{\pgfqpoint{2.909708in}{0.739656in}}%
\pgfpathlineto{\pgfqpoint{2.909411in}{0.739656in}}%
\pgfpathlineto{\pgfqpoint{2.909113in}{0.739656in}}%
\pgfpathlineto{\pgfqpoint{2.908816in}{0.739656in}}%
\pgfpathlineto{\pgfqpoint{2.908518in}{0.739656in}}%
\pgfpathlineto{\pgfqpoint{2.908221in}{0.739656in}}%
\pgfpathlineto{\pgfqpoint{2.907923in}{0.739656in}}%
\pgfpathlineto{\pgfqpoint{2.907626in}{0.739656in}}%
\pgfpathlineto{\pgfqpoint{2.907328in}{0.739656in}}%
\pgfpathlineto{\pgfqpoint{2.907031in}{0.739656in}}%
\pgfpathlineto{\pgfqpoint{2.906733in}{0.739656in}}%
\pgfpathlineto{\pgfqpoint{2.906436in}{0.739656in}}%
\pgfpathlineto{\pgfqpoint{2.906138in}{0.739656in}}%
\pgfpathlineto{\pgfqpoint{2.905841in}{0.739656in}}%
\pgfpathlineto{\pgfqpoint{2.905543in}{0.739656in}}%
\pgfpathlineto{\pgfqpoint{2.905246in}{0.739656in}}%
\pgfpathlineto{\pgfqpoint{2.904949in}{0.739656in}}%
\pgfpathlineto{\pgfqpoint{2.904651in}{0.739656in}}%
\pgfpathlineto{\pgfqpoint{2.904354in}{0.739656in}}%
\pgfpathlineto{\pgfqpoint{2.904056in}{0.739656in}}%
\pgfpathlineto{\pgfqpoint{2.903759in}{0.739656in}}%
\pgfpathlineto{\pgfqpoint{2.903461in}{0.739656in}}%
\pgfpathlineto{\pgfqpoint{2.903164in}{0.739656in}}%
\pgfpathlineto{\pgfqpoint{2.902866in}{0.739656in}}%
\pgfpathlineto{\pgfqpoint{2.902569in}{0.739656in}}%
\pgfpathlineto{\pgfqpoint{2.902271in}{0.739656in}}%
\pgfpathlineto{\pgfqpoint{2.901974in}{0.739656in}}%
\pgfpathlineto{\pgfqpoint{2.901676in}{0.739656in}}%
\pgfpathlineto{\pgfqpoint{2.901379in}{0.739656in}}%
\pgfpathlineto{\pgfqpoint{2.901081in}{0.739656in}}%
\pgfpathlineto{\pgfqpoint{2.900784in}{0.739656in}}%
\pgfpathlineto{\pgfqpoint{2.900486in}{0.739656in}}%
\pgfpathlineto{\pgfqpoint{2.900189in}{0.739656in}}%
\pgfpathlineto{\pgfqpoint{2.899891in}{0.739656in}}%
\pgfpathlineto{\pgfqpoint{2.899594in}{0.739656in}}%
\pgfpathlineto{\pgfqpoint{2.899296in}{0.739656in}}%
\pgfpathlineto{\pgfqpoint{2.898999in}{0.739656in}}%
\pgfpathlineto{\pgfqpoint{2.898701in}{0.739656in}}%
\pgfpathlineto{\pgfqpoint{2.898404in}{0.739656in}}%
\pgfpathlineto{\pgfqpoint{2.898107in}{0.739656in}}%
\pgfpathlineto{\pgfqpoint{2.897809in}{0.739656in}}%
\pgfpathlineto{\pgfqpoint{2.897512in}{0.739656in}}%
\pgfpathlineto{\pgfqpoint{2.897214in}{0.739656in}}%
\pgfpathlineto{\pgfqpoint{2.896917in}{0.739656in}}%
\pgfpathlineto{\pgfqpoint{2.896619in}{0.739656in}}%
\pgfpathlineto{\pgfqpoint{2.896322in}{0.739656in}}%
\pgfpathlineto{\pgfqpoint{2.896024in}{0.739656in}}%
\pgfpathlineto{\pgfqpoint{2.895727in}{0.739656in}}%
\pgfpathlineto{\pgfqpoint{2.895429in}{0.739656in}}%
\pgfpathlineto{\pgfqpoint{2.895132in}{0.739656in}}%
\pgfpathlineto{\pgfqpoint{2.894834in}{0.739656in}}%
\pgfpathlineto{\pgfqpoint{2.894537in}{0.739656in}}%
\pgfpathlineto{\pgfqpoint{2.894239in}{0.739656in}}%
\pgfpathlineto{\pgfqpoint{2.893942in}{0.739656in}}%
\pgfpathlineto{\pgfqpoint{2.893644in}{0.739656in}}%
\pgfpathlineto{\pgfqpoint{2.893347in}{0.739656in}}%
\pgfpathlineto{\pgfqpoint{2.893049in}{0.739656in}}%
\pgfpathlineto{\pgfqpoint{2.892752in}{0.739656in}}%
\pgfpathlineto{\pgfqpoint{2.892454in}{0.739656in}}%
\pgfpathlineto{\pgfqpoint{2.892157in}{0.739656in}}%
\pgfpathlineto{\pgfqpoint{2.891859in}{0.739656in}}%
\pgfpathlineto{\pgfqpoint{2.891562in}{0.739656in}}%
\pgfpathlineto{\pgfqpoint{2.891265in}{0.739656in}}%
\pgfpathlineto{\pgfqpoint{2.890967in}{0.739656in}}%
\pgfpathlineto{\pgfqpoint{2.890670in}{0.739656in}}%
\pgfpathlineto{\pgfqpoint{2.890372in}{0.739656in}}%
\pgfpathlineto{\pgfqpoint{2.890075in}{0.739656in}}%
\pgfpathlineto{\pgfqpoint{2.889777in}{0.739656in}}%
\pgfpathlineto{\pgfqpoint{2.889480in}{0.739656in}}%
\pgfpathlineto{\pgfqpoint{2.889182in}{0.739656in}}%
\pgfpathlineto{\pgfqpoint{2.888885in}{0.739656in}}%
\pgfpathlineto{\pgfqpoint{2.888587in}{0.739656in}}%
\pgfpathlineto{\pgfqpoint{2.888290in}{0.739656in}}%
\pgfpathlineto{\pgfqpoint{2.887992in}{0.739656in}}%
\pgfpathlineto{\pgfqpoint{2.887695in}{0.739656in}}%
\pgfpathlineto{\pgfqpoint{2.887397in}{0.739656in}}%
\pgfpathlineto{\pgfqpoint{2.887100in}{0.739656in}}%
\pgfpathlineto{\pgfqpoint{2.886802in}{0.739656in}}%
\pgfpathlineto{\pgfqpoint{2.886505in}{0.739656in}}%
\pgfpathlineto{\pgfqpoint{2.886207in}{0.739656in}}%
\pgfpathlineto{\pgfqpoint{2.885910in}{0.739656in}}%
\pgfpathlineto{\pgfqpoint{2.885612in}{0.739656in}}%
\pgfpathlineto{\pgfqpoint{2.885315in}{0.739656in}}%
\pgfpathlineto{\pgfqpoint{2.885018in}{0.739656in}}%
\pgfpathlineto{\pgfqpoint{2.884720in}{0.739656in}}%
\pgfpathlineto{\pgfqpoint{2.884423in}{0.739656in}}%
\pgfpathlineto{\pgfqpoint{2.884125in}{0.739656in}}%
\pgfpathlineto{\pgfqpoint{2.883828in}{0.739656in}}%
\pgfpathlineto{\pgfqpoint{2.883530in}{0.739656in}}%
\pgfpathlineto{\pgfqpoint{2.883233in}{0.739656in}}%
\pgfpathlineto{\pgfqpoint{2.882935in}{0.739656in}}%
\pgfpathlineto{\pgfqpoint{2.882638in}{0.739656in}}%
\pgfpathlineto{\pgfqpoint{2.882340in}{0.739656in}}%
\pgfpathlineto{\pgfqpoint{2.882043in}{0.739656in}}%
\pgfpathlineto{\pgfqpoint{2.881745in}{0.739656in}}%
\pgfpathlineto{\pgfqpoint{2.881448in}{0.739656in}}%
\pgfpathlineto{\pgfqpoint{2.881150in}{0.739656in}}%
\pgfpathlineto{\pgfqpoint{2.880853in}{0.739656in}}%
\pgfpathlineto{\pgfqpoint{2.880555in}{0.739656in}}%
\pgfpathlineto{\pgfqpoint{2.880258in}{0.739656in}}%
\pgfpathlineto{\pgfqpoint{2.879960in}{0.739656in}}%
\pgfpathlineto{\pgfqpoint{2.879663in}{0.739656in}}%
\pgfpathlineto{\pgfqpoint{2.879365in}{0.739656in}}%
\pgfpathlineto{\pgfqpoint{2.879068in}{0.739656in}}%
\pgfpathlineto{\pgfqpoint{2.878770in}{0.739656in}}%
\pgfpathlineto{\pgfqpoint{2.878473in}{0.739656in}}%
\pgfpathlineto{\pgfqpoint{2.878176in}{0.739656in}}%
\pgfpathlineto{\pgfqpoint{2.877878in}{0.739656in}}%
\pgfpathlineto{\pgfqpoint{2.877581in}{0.739656in}}%
\pgfpathlineto{\pgfqpoint{2.877283in}{0.739656in}}%
\pgfpathlineto{\pgfqpoint{2.876986in}{0.739656in}}%
\pgfpathlineto{\pgfqpoint{2.876688in}{0.739656in}}%
\pgfpathlineto{\pgfqpoint{2.876391in}{0.739656in}}%
\pgfpathlineto{\pgfqpoint{2.876093in}{0.739656in}}%
\pgfpathlineto{\pgfqpoint{2.875796in}{0.739656in}}%
\pgfpathlineto{\pgfqpoint{2.875498in}{0.739656in}}%
\pgfpathlineto{\pgfqpoint{2.875201in}{0.739656in}}%
\pgfpathlineto{\pgfqpoint{2.874903in}{0.739656in}}%
\pgfpathlineto{\pgfqpoint{2.874606in}{0.739656in}}%
\pgfpathlineto{\pgfqpoint{2.874308in}{0.739656in}}%
\pgfpathlineto{\pgfqpoint{2.874011in}{0.739656in}}%
\pgfpathlineto{\pgfqpoint{2.873713in}{0.739656in}}%
\pgfpathlineto{\pgfqpoint{2.873416in}{0.739656in}}%
\pgfpathlineto{\pgfqpoint{2.873118in}{0.739656in}}%
\pgfpathlineto{\pgfqpoint{2.872821in}{0.739656in}}%
\pgfpathlineto{\pgfqpoint{2.872523in}{0.739656in}}%
\pgfpathlineto{\pgfqpoint{2.872226in}{0.739656in}}%
\pgfpathlineto{\pgfqpoint{2.871928in}{0.739656in}}%
\pgfpathlineto{\pgfqpoint{2.871631in}{0.739656in}}%
\pgfpathlineto{\pgfqpoint{2.871334in}{0.739656in}}%
\pgfpathlineto{\pgfqpoint{2.871036in}{0.739656in}}%
\pgfpathlineto{\pgfqpoint{2.870739in}{0.739656in}}%
\pgfpathlineto{\pgfqpoint{2.870441in}{0.739656in}}%
\pgfpathlineto{\pgfqpoint{2.870144in}{0.739656in}}%
\pgfpathlineto{\pgfqpoint{2.869846in}{0.739656in}}%
\pgfpathlineto{\pgfqpoint{2.869549in}{0.739656in}}%
\pgfpathlineto{\pgfqpoint{2.869251in}{0.739656in}}%
\pgfpathlineto{\pgfqpoint{2.868954in}{0.739656in}}%
\pgfpathlineto{\pgfqpoint{2.868656in}{0.739656in}}%
\pgfpathlineto{\pgfqpoint{2.868359in}{0.739656in}}%
\pgfpathlineto{\pgfqpoint{2.868061in}{0.739656in}}%
\pgfpathlineto{\pgfqpoint{2.867764in}{0.739656in}}%
\pgfpathlineto{\pgfqpoint{2.867466in}{0.739656in}}%
\pgfpathlineto{\pgfqpoint{2.867169in}{0.739656in}}%
\pgfpathlineto{\pgfqpoint{2.866871in}{0.739656in}}%
\pgfpathlineto{\pgfqpoint{2.866574in}{0.739656in}}%
\pgfpathlineto{\pgfqpoint{2.866276in}{0.739656in}}%
\pgfpathlineto{\pgfqpoint{2.865979in}{0.739656in}}%
\pgfpathlineto{\pgfqpoint{2.865681in}{0.739656in}}%
\pgfpathlineto{\pgfqpoint{2.865384in}{0.739656in}}%
\pgfpathlineto{\pgfqpoint{2.865086in}{0.739656in}}%
\pgfpathlineto{\pgfqpoint{2.864789in}{0.739656in}}%
\pgfpathlineto{\pgfqpoint{2.864492in}{0.739656in}}%
\pgfpathlineto{\pgfqpoint{2.864194in}{0.739656in}}%
\pgfpathlineto{\pgfqpoint{2.863897in}{0.739656in}}%
\pgfpathlineto{\pgfqpoint{2.863599in}{0.739656in}}%
\pgfpathlineto{\pgfqpoint{2.863302in}{0.739656in}}%
\pgfpathlineto{\pgfqpoint{2.863004in}{0.739656in}}%
\pgfpathlineto{\pgfqpoint{2.862707in}{0.739656in}}%
\pgfpathlineto{\pgfqpoint{2.862409in}{0.739656in}}%
\pgfpathlineto{\pgfqpoint{2.862112in}{0.739656in}}%
\pgfpathlineto{\pgfqpoint{2.861814in}{0.739656in}}%
\pgfpathlineto{\pgfqpoint{2.861517in}{0.739656in}}%
\pgfpathlineto{\pgfqpoint{2.861219in}{0.739656in}}%
\pgfpathlineto{\pgfqpoint{2.860922in}{0.739656in}}%
\pgfpathlineto{\pgfqpoint{2.860624in}{0.739656in}}%
\pgfpathlineto{\pgfqpoint{2.860327in}{0.739656in}}%
\pgfpathlineto{\pgfqpoint{2.860029in}{0.739656in}}%
\pgfpathlineto{\pgfqpoint{2.859732in}{0.739656in}}%
\pgfpathlineto{\pgfqpoint{2.859434in}{0.739656in}}%
\pgfpathlineto{\pgfqpoint{2.859137in}{0.739656in}}%
\pgfpathlineto{\pgfqpoint{2.858839in}{0.739656in}}%
\pgfpathlineto{\pgfqpoint{2.858542in}{0.739656in}}%
\pgfpathlineto{\pgfqpoint{2.858245in}{0.739656in}}%
\pgfpathlineto{\pgfqpoint{2.857947in}{0.739656in}}%
\pgfpathlineto{\pgfqpoint{2.857650in}{0.739656in}}%
\pgfpathlineto{\pgfqpoint{2.857352in}{0.739656in}}%
\pgfpathlineto{\pgfqpoint{2.857055in}{0.739656in}}%
\pgfpathlineto{\pgfqpoint{2.856757in}{0.739656in}}%
\pgfpathlineto{\pgfqpoint{2.856460in}{0.739656in}}%
\pgfpathlineto{\pgfqpoint{2.856162in}{0.739656in}}%
\pgfpathlineto{\pgfqpoint{2.855865in}{0.739656in}}%
\pgfpathlineto{\pgfqpoint{2.855567in}{0.739656in}}%
\pgfpathlineto{\pgfqpoint{2.855270in}{0.739656in}}%
\pgfpathlineto{\pgfqpoint{2.854972in}{0.739656in}}%
\pgfpathlineto{\pgfqpoint{2.854675in}{0.739656in}}%
\pgfpathlineto{\pgfqpoint{2.854377in}{0.739656in}}%
\pgfpathlineto{\pgfqpoint{2.854080in}{0.739656in}}%
\pgfpathlineto{\pgfqpoint{2.853782in}{0.739656in}}%
\pgfpathlineto{\pgfqpoint{2.853485in}{0.739656in}}%
\pgfpathlineto{\pgfqpoint{2.853187in}{0.739656in}}%
\pgfpathlineto{\pgfqpoint{2.852890in}{0.739656in}}%
\pgfpathlineto{\pgfqpoint{2.852592in}{0.739656in}}%
\pgfpathlineto{\pgfqpoint{2.852295in}{0.739656in}}%
\pgfpathlineto{\pgfqpoint{2.851997in}{0.739656in}}%
\pgfpathlineto{\pgfqpoint{2.851700in}{0.739656in}}%
\pgfpathlineto{\pgfqpoint{2.851403in}{0.739656in}}%
\pgfpathlineto{\pgfqpoint{2.851105in}{0.739656in}}%
\pgfpathlineto{\pgfqpoint{2.850808in}{0.739656in}}%
\pgfpathlineto{\pgfqpoint{2.850510in}{0.739656in}}%
\pgfpathlineto{\pgfqpoint{2.850213in}{0.739656in}}%
\pgfpathlineto{\pgfqpoint{2.849915in}{0.739656in}}%
\pgfpathlineto{\pgfqpoint{2.849618in}{0.739656in}}%
\pgfpathlineto{\pgfqpoint{2.849320in}{0.739656in}}%
\pgfpathlineto{\pgfqpoint{2.849023in}{0.739656in}}%
\pgfpathlineto{\pgfqpoint{2.848725in}{0.739656in}}%
\pgfpathlineto{\pgfqpoint{2.848428in}{0.739656in}}%
\pgfpathlineto{\pgfqpoint{2.848130in}{0.739656in}}%
\pgfpathlineto{\pgfqpoint{2.847833in}{0.739656in}}%
\pgfpathlineto{\pgfqpoint{2.847535in}{0.739656in}}%
\pgfpathlineto{\pgfqpoint{2.847238in}{0.739656in}}%
\pgfpathlineto{\pgfqpoint{2.846940in}{0.739656in}}%
\pgfpathlineto{\pgfqpoint{2.846643in}{0.739656in}}%
\pgfpathlineto{\pgfqpoint{2.846345in}{0.739656in}}%
\pgfpathlineto{\pgfqpoint{2.846048in}{0.739656in}}%
\pgfpathlineto{\pgfqpoint{2.845750in}{0.739656in}}%
\pgfpathlineto{\pgfqpoint{2.845453in}{0.739656in}}%
\pgfpathlineto{\pgfqpoint{2.845155in}{0.739656in}}%
\pgfpathlineto{\pgfqpoint{2.844858in}{0.739656in}}%
\pgfpathlineto{\pgfqpoint{2.844561in}{0.739656in}}%
\pgfpathlineto{\pgfqpoint{2.844263in}{0.739656in}}%
\pgfpathlineto{\pgfqpoint{2.843966in}{0.739656in}}%
\pgfpathlineto{\pgfqpoint{2.843668in}{0.739656in}}%
\pgfpathlineto{\pgfqpoint{2.843371in}{0.739656in}}%
\pgfpathlineto{\pgfqpoint{2.843073in}{0.739656in}}%
\pgfpathlineto{\pgfqpoint{2.842776in}{0.739656in}}%
\pgfpathlineto{\pgfqpoint{2.842478in}{0.739656in}}%
\pgfpathlineto{\pgfqpoint{2.842181in}{0.739656in}}%
\pgfpathlineto{\pgfqpoint{2.841883in}{0.739656in}}%
\pgfpathlineto{\pgfqpoint{2.841586in}{0.739656in}}%
\pgfpathlineto{\pgfqpoint{2.841288in}{0.739656in}}%
\pgfpathlineto{\pgfqpoint{2.840991in}{0.739656in}}%
\pgfpathlineto{\pgfqpoint{2.840693in}{0.739656in}}%
\pgfpathlineto{\pgfqpoint{2.840396in}{0.739656in}}%
\pgfpathlineto{\pgfqpoint{2.840098in}{0.739656in}}%
\pgfpathlineto{\pgfqpoint{2.839801in}{0.739656in}}%
\pgfpathlineto{\pgfqpoint{2.839503in}{0.739656in}}%
\pgfpathlineto{\pgfqpoint{2.839206in}{0.739656in}}%
\pgfpathlineto{\pgfqpoint{2.838908in}{0.739656in}}%
\pgfpathlineto{\pgfqpoint{2.838611in}{0.739656in}}%
\pgfpathlineto{\pgfqpoint{2.838314in}{0.739656in}}%
\pgfpathlineto{\pgfqpoint{2.838016in}{0.739656in}}%
\pgfpathlineto{\pgfqpoint{2.837719in}{0.739656in}}%
\pgfpathlineto{\pgfqpoint{2.837421in}{0.739656in}}%
\pgfpathlineto{\pgfqpoint{2.837124in}{0.739656in}}%
\pgfpathlineto{\pgfqpoint{2.836826in}{0.739656in}}%
\pgfpathlineto{\pgfqpoint{2.836529in}{0.739656in}}%
\pgfpathlineto{\pgfqpoint{2.836231in}{0.739656in}}%
\pgfpathlineto{\pgfqpoint{2.835934in}{0.739656in}}%
\pgfpathlineto{\pgfqpoint{2.835636in}{0.739656in}}%
\pgfpathlineto{\pgfqpoint{2.835339in}{0.739656in}}%
\pgfpathlineto{\pgfqpoint{2.835041in}{0.739656in}}%
\pgfpathlineto{\pgfqpoint{2.834744in}{0.739656in}}%
\pgfpathlineto{\pgfqpoint{2.834446in}{0.739656in}}%
\pgfpathlineto{\pgfqpoint{2.834149in}{0.739656in}}%
\pgfpathlineto{\pgfqpoint{2.833851in}{0.739656in}}%
\pgfpathlineto{\pgfqpoint{2.833554in}{0.739656in}}%
\pgfpathlineto{\pgfqpoint{2.833256in}{0.739656in}}%
\pgfpathlineto{\pgfqpoint{2.832959in}{0.739656in}}%
\pgfpathlineto{\pgfqpoint{2.832661in}{0.739656in}}%
\pgfpathlineto{\pgfqpoint{2.832364in}{0.739656in}}%
\pgfpathlineto{\pgfqpoint{2.832066in}{0.739656in}}%
\pgfpathlineto{\pgfqpoint{2.831769in}{0.739656in}}%
\pgfpathlineto{\pgfqpoint{2.831472in}{0.739656in}}%
\pgfpathlineto{\pgfqpoint{2.831174in}{0.739656in}}%
\pgfpathlineto{\pgfqpoint{2.830877in}{0.739656in}}%
\pgfpathlineto{\pgfqpoint{2.830579in}{0.739656in}}%
\pgfpathlineto{\pgfqpoint{2.830282in}{0.739656in}}%
\pgfpathlineto{\pgfqpoint{2.829984in}{0.739656in}}%
\pgfpathlineto{\pgfqpoint{2.829687in}{0.739656in}}%
\pgfpathlineto{\pgfqpoint{2.829389in}{0.739656in}}%
\pgfpathlineto{\pgfqpoint{2.829092in}{0.739656in}}%
\pgfpathlineto{\pgfqpoint{2.828794in}{0.739656in}}%
\pgfpathlineto{\pgfqpoint{2.828497in}{0.739656in}}%
\pgfpathlineto{\pgfqpoint{2.828199in}{0.739656in}}%
\pgfpathlineto{\pgfqpoint{2.827902in}{0.739656in}}%
\pgfpathlineto{\pgfqpoint{2.827604in}{0.739656in}}%
\pgfpathlineto{\pgfqpoint{2.827307in}{0.739656in}}%
\pgfpathlineto{\pgfqpoint{2.827009in}{0.739656in}}%
\pgfpathlineto{\pgfqpoint{2.826712in}{0.739656in}}%
\pgfpathlineto{\pgfqpoint{2.826414in}{0.739656in}}%
\pgfpathlineto{\pgfqpoint{2.826117in}{0.739656in}}%
\pgfpathlineto{\pgfqpoint{2.825819in}{0.739656in}}%
\pgfpathlineto{\pgfqpoint{2.825522in}{0.739656in}}%
\pgfpathlineto{\pgfqpoint{2.825224in}{0.739656in}}%
\pgfpathlineto{\pgfqpoint{2.824927in}{0.739656in}}%
\pgfpathlineto{\pgfqpoint{2.824630in}{0.739656in}}%
\pgfpathlineto{\pgfqpoint{2.824332in}{0.739656in}}%
\pgfpathlineto{\pgfqpoint{2.824035in}{0.739656in}}%
\pgfpathlineto{\pgfqpoint{2.823737in}{0.739656in}}%
\pgfpathlineto{\pgfqpoint{2.823440in}{0.739656in}}%
\pgfpathlineto{\pgfqpoint{2.823142in}{0.739656in}}%
\pgfpathlineto{\pgfqpoint{2.822845in}{0.739656in}}%
\pgfpathlineto{\pgfqpoint{2.822547in}{0.739656in}}%
\pgfpathlineto{\pgfqpoint{2.822250in}{0.739656in}}%
\pgfpathlineto{\pgfqpoint{2.821952in}{0.739656in}}%
\pgfpathlineto{\pgfqpoint{2.821655in}{0.739656in}}%
\pgfpathlineto{\pgfqpoint{2.821357in}{0.739656in}}%
\pgfpathlineto{\pgfqpoint{2.821060in}{0.739656in}}%
\pgfpathlineto{\pgfqpoint{2.820762in}{0.739656in}}%
\pgfpathlineto{\pgfqpoint{2.820465in}{0.739656in}}%
\pgfpathlineto{\pgfqpoint{2.820167in}{0.739656in}}%
\pgfpathlineto{\pgfqpoint{2.819870in}{0.739656in}}%
\pgfpathlineto{\pgfqpoint{2.819572in}{0.739656in}}%
\pgfpathlineto{\pgfqpoint{2.819275in}{0.739656in}}%
\pgfpathlineto{\pgfqpoint{2.818977in}{0.739656in}}%
\pgfpathlineto{\pgfqpoint{2.818680in}{0.739656in}}%
\pgfpathlineto{\pgfqpoint{2.818383in}{0.739656in}}%
\pgfpathlineto{\pgfqpoint{2.818085in}{0.739656in}}%
\pgfpathlineto{\pgfqpoint{2.817788in}{0.739656in}}%
\pgfpathlineto{\pgfqpoint{2.817490in}{0.739656in}}%
\pgfpathlineto{\pgfqpoint{2.817193in}{0.739656in}}%
\pgfpathlineto{\pgfqpoint{2.816895in}{0.739656in}}%
\pgfpathlineto{\pgfqpoint{2.816598in}{0.739656in}}%
\pgfpathlineto{\pgfqpoint{2.816300in}{0.739656in}}%
\pgfpathlineto{\pgfqpoint{2.816003in}{0.739656in}}%
\pgfpathlineto{\pgfqpoint{2.815705in}{0.739656in}}%
\pgfpathlineto{\pgfqpoint{2.815408in}{0.739656in}}%
\pgfpathlineto{\pgfqpoint{2.815110in}{0.739656in}}%
\pgfpathlineto{\pgfqpoint{2.814813in}{0.739656in}}%
\pgfpathlineto{\pgfqpoint{2.814515in}{0.739656in}}%
\pgfpathlineto{\pgfqpoint{2.814218in}{0.739656in}}%
\pgfpathlineto{\pgfqpoint{2.813920in}{0.739656in}}%
\pgfpathlineto{\pgfqpoint{2.813623in}{0.739656in}}%
\pgfpathlineto{\pgfqpoint{2.813325in}{0.739656in}}%
\pgfpathlineto{\pgfqpoint{2.813028in}{0.739656in}}%
\pgfpathlineto{\pgfqpoint{2.812730in}{0.739656in}}%
\pgfpathlineto{\pgfqpoint{2.812433in}{0.739656in}}%
\pgfpathlineto{\pgfqpoint{2.812135in}{0.739656in}}%
\pgfpathlineto{\pgfqpoint{2.811838in}{0.739656in}}%
\pgfpathlineto{\pgfqpoint{2.811541in}{0.739656in}}%
\pgfpathlineto{\pgfqpoint{2.811243in}{0.739656in}}%
\pgfpathlineto{\pgfqpoint{2.810946in}{0.739656in}}%
\pgfpathlineto{\pgfqpoint{2.810648in}{0.739656in}}%
\pgfpathlineto{\pgfqpoint{2.810351in}{0.739656in}}%
\pgfpathlineto{\pgfqpoint{2.810053in}{0.739656in}}%
\pgfpathlineto{\pgfqpoint{2.809756in}{0.739656in}}%
\pgfpathlineto{\pgfqpoint{2.809458in}{0.739656in}}%
\pgfpathlineto{\pgfqpoint{2.809161in}{0.739656in}}%
\pgfpathlineto{\pgfqpoint{2.808863in}{0.739656in}}%
\pgfpathlineto{\pgfqpoint{2.808566in}{0.739656in}}%
\pgfpathlineto{\pgfqpoint{2.808268in}{0.739656in}}%
\pgfpathlineto{\pgfqpoint{2.807971in}{0.739656in}}%
\pgfpathlineto{\pgfqpoint{2.807673in}{0.739656in}}%
\pgfpathlineto{\pgfqpoint{2.807376in}{0.739656in}}%
\pgfpathlineto{\pgfqpoint{2.807078in}{0.739656in}}%
\pgfpathlineto{\pgfqpoint{2.806781in}{0.739656in}}%
\pgfpathlineto{\pgfqpoint{2.806483in}{0.739656in}}%
\pgfpathlineto{\pgfqpoint{2.806186in}{0.739656in}}%
\pgfpathlineto{\pgfqpoint{2.805888in}{0.739656in}}%
\pgfpathlineto{\pgfqpoint{2.805591in}{0.739656in}}%
\pgfpathlineto{\pgfqpoint{2.805293in}{0.739656in}}%
\pgfpathlineto{\pgfqpoint{2.804996in}{0.739656in}}%
\pgfpathlineto{\pgfqpoint{2.804699in}{0.739656in}}%
\pgfpathlineto{\pgfqpoint{2.804401in}{0.739656in}}%
\pgfpathlineto{\pgfqpoint{2.804104in}{0.739656in}}%
\pgfpathlineto{\pgfqpoint{2.803806in}{0.739656in}}%
\pgfpathlineto{\pgfqpoint{2.803509in}{0.739656in}}%
\pgfpathlineto{\pgfqpoint{2.803211in}{0.739656in}}%
\pgfpathlineto{\pgfqpoint{2.802914in}{0.739656in}}%
\pgfpathlineto{\pgfqpoint{2.802616in}{0.739656in}}%
\pgfpathlineto{\pgfqpoint{2.802319in}{0.739656in}}%
\pgfpathlineto{\pgfqpoint{2.802021in}{0.739656in}}%
\pgfpathlineto{\pgfqpoint{2.801724in}{0.739656in}}%
\pgfpathlineto{\pgfqpoint{2.801426in}{0.739656in}}%
\pgfpathlineto{\pgfqpoint{2.801129in}{0.739656in}}%
\pgfpathlineto{\pgfqpoint{2.800831in}{0.739656in}}%
\pgfpathlineto{\pgfqpoint{2.800534in}{0.739656in}}%
\pgfpathlineto{\pgfqpoint{2.800236in}{0.739656in}}%
\pgfpathlineto{\pgfqpoint{2.799939in}{0.739656in}}%
\pgfpathlineto{\pgfqpoint{2.799641in}{0.739656in}}%
\pgfpathlineto{\pgfqpoint{2.799344in}{0.739656in}}%
\pgfpathlineto{\pgfqpoint{2.799046in}{0.739656in}}%
\pgfpathlineto{\pgfqpoint{2.798749in}{0.739656in}}%
\pgfpathlineto{\pgfqpoint{2.798452in}{0.739656in}}%
\pgfpathlineto{\pgfqpoint{2.798154in}{0.739656in}}%
\pgfpathlineto{\pgfqpoint{2.797857in}{0.739656in}}%
\pgfpathlineto{\pgfqpoint{2.797559in}{0.739656in}}%
\pgfpathlineto{\pgfqpoint{2.797262in}{0.739656in}}%
\pgfpathlineto{\pgfqpoint{2.796964in}{0.739656in}}%
\pgfpathlineto{\pgfqpoint{2.796667in}{0.739656in}}%
\pgfpathlineto{\pgfqpoint{2.796369in}{0.739656in}}%
\pgfpathlineto{\pgfqpoint{2.796072in}{0.739656in}}%
\pgfpathlineto{\pgfqpoint{2.795774in}{0.739656in}}%
\pgfpathlineto{\pgfqpoint{2.795477in}{0.739656in}}%
\pgfpathlineto{\pgfqpoint{2.795179in}{0.739656in}}%
\pgfpathlineto{\pgfqpoint{2.794882in}{0.739656in}}%
\pgfpathlineto{\pgfqpoint{2.794584in}{0.739656in}}%
\pgfpathlineto{\pgfqpoint{2.794287in}{0.739656in}}%
\pgfpathlineto{\pgfqpoint{2.793989in}{0.739656in}}%
\pgfpathlineto{\pgfqpoint{2.793692in}{0.739656in}}%
\pgfpathlineto{\pgfqpoint{2.793394in}{0.739656in}}%
\pgfpathlineto{\pgfqpoint{2.793097in}{0.739656in}}%
\pgfpathlineto{\pgfqpoint{2.792799in}{0.739656in}}%
\pgfpathlineto{\pgfqpoint{2.792502in}{0.739656in}}%
\pgfpathlineto{\pgfqpoint{2.792204in}{0.739656in}}%
\pgfpathlineto{\pgfqpoint{2.791907in}{0.739656in}}%
\pgfpathlineto{\pgfqpoint{2.791610in}{0.739656in}}%
\pgfpathlineto{\pgfqpoint{2.791312in}{0.739656in}}%
\pgfpathlineto{\pgfqpoint{2.791015in}{0.739656in}}%
\pgfpathlineto{\pgfqpoint{2.790717in}{0.739656in}}%
\pgfpathlineto{\pgfqpoint{2.790420in}{0.739656in}}%
\pgfpathlineto{\pgfqpoint{2.790122in}{0.739656in}}%
\pgfpathlineto{\pgfqpoint{2.789825in}{0.739656in}}%
\pgfpathlineto{\pgfqpoint{2.789527in}{0.739656in}}%
\pgfpathlineto{\pgfqpoint{2.789230in}{0.739656in}}%
\pgfpathlineto{\pgfqpoint{2.788932in}{0.739656in}}%
\pgfpathlineto{\pgfqpoint{2.788635in}{0.739656in}}%
\pgfpathlineto{\pgfqpoint{2.788337in}{0.739656in}}%
\pgfpathlineto{\pgfqpoint{2.788040in}{0.739656in}}%
\pgfpathlineto{\pgfqpoint{2.787742in}{0.739656in}}%
\pgfpathlineto{\pgfqpoint{2.787445in}{0.739656in}}%
\pgfpathlineto{\pgfqpoint{2.787147in}{0.739656in}}%
\pgfpathlineto{\pgfqpoint{2.786850in}{0.739656in}}%
\pgfpathlineto{\pgfqpoint{2.786552in}{0.739656in}}%
\pgfpathlineto{\pgfqpoint{2.786255in}{0.739656in}}%
\pgfpathlineto{\pgfqpoint{2.785957in}{0.739656in}}%
\pgfpathlineto{\pgfqpoint{2.785660in}{0.739656in}}%
\pgfpathlineto{\pgfqpoint{2.785362in}{0.739656in}}%
\pgfpathlineto{\pgfqpoint{2.785065in}{0.739656in}}%
\pgfpathlineto{\pgfqpoint{2.784768in}{0.739656in}}%
\pgfpathlineto{\pgfqpoint{2.784470in}{0.739656in}}%
\pgfpathlineto{\pgfqpoint{2.784173in}{0.739656in}}%
\pgfpathlineto{\pgfqpoint{2.783875in}{0.739656in}}%
\pgfpathlineto{\pgfqpoint{2.783578in}{0.739656in}}%
\pgfpathlineto{\pgfqpoint{2.783280in}{0.739656in}}%
\pgfpathlineto{\pgfqpoint{2.782983in}{0.739656in}}%
\pgfpathlineto{\pgfqpoint{2.782685in}{0.739656in}}%
\pgfpathlineto{\pgfqpoint{2.782388in}{0.739656in}}%
\pgfpathlineto{\pgfqpoint{2.782090in}{0.739656in}}%
\pgfpathlineto{\pgfqpoint{2.781793in}{0.739656in}}%
\pgfpathlineto{\pgfqpoint{2.781495in}{0.739656in}}%
\pgfpathlineto{\pgfqpoint{2.781198in}{0.739656in}}%
\pgfpathlineto{\pgfqpoint{2.780900in}{0.739656in}}%
\pgfpathlineto{\pgfqpoint{2.780603in}{0.739656in}}%
\pgfpathlineto{\pgfqpoint{2.780305in}{0.739656in}}%
\pgfpathlineto{\pgfqpoint{2.780008in}{0.739656in}}%
\pgfpathlineto{\pgfqpoint{2.779710in}{0.739656in}}%
\pgfpathlineto{\pgfqpoint{2.779413in}{0.739656in}}%
\pgfpathlineto{\pgfqpoint{2.779115in}{0.739656in}}%
\pgfpathlineto{\pgfqpoint{2.778818in}{0.739656in}}%
\pgfpathlineto{\pgfqpoint{2.778521in}{0.739656in}}%
\pgfpathlineto{\pgfqpoint{2.778223in}{0.739656in}}%
\pgfpathlineto{\pgfqpoint{2.777926in}{0.739656in}}%
\pgfpathlineto{\pgfqpoint{2.777628in}{0.739656in}}%
\pgfpathlineto{\pgfqpoint{2.777331in}{0.739656in}}%
\pgfpathlineto{\pgfqpoint{2.777033in}{0.739656in}}%
\pgfpathlineto{\pgfqpoint{2.776736in}{0.739656in}}%
\pgfpathlineto{\pgfqpoint{2.776438in}{0.739656in}}%
\pgfpathlineto{\pgfqpoint{2.776141in}{0.739656in}}%
\pgfpathlineto{\pgfqpoint{2.775843in}{0.739656in}}%
\pgfpathlineto{\pgfqpoint{2.775546in}{0.739656in}}%
\pgfpathlineto{\pgfqpoint{2.775248in}{0.739656in}}%
\pgfpathlineto{\pgfqpoint{2.774951in}{0.739656in}}%
\pgfpathlineto{\pgfqpoint{2.774653in}{0.739656in}}%
\pgfpathlineto{\pgfqpoint{2.774356in}{0.739656in}}%
\pgfpathlineto{\pgfqpoint{2.774058in}{0.739656in}}%
\pgfpathlineto{\pgfqpoint{2.773761in}{0.739656in}}%
\pgfpathlineto{\pgfqpoint{2.773463in}{0.739656in}}%
\pgfpathlineto{\pgfqpoint{2.773166in}{0.739656in}}%
\pgfpathlineto{\pgfqpoint{2.772868in}{0.739656in}}%
\pgfpathlineto{\pgfqpoint{2.772571in}{0.739656in}}%
\pgfpathlineto{\pgfqpoint{2.772273in}{0.739656in}}%
\pgfpathlineto{\pgfqpoint{2.771976in}{0.739656in}}%
\pgfpathlineto{\pgfqpoint{2.771679in}{0.739656in}}%
\pgfpathlineto{\pgfqpoint{2.771381in}{0.739656in}}%
\pgfpathlineto{\pgfqpoint{2.771084in}{0.739656in}}%
\pgfpathlineto{\pgfqpoint{2.770786in}{0.739656in}}%
\pgfpathlineto{\pgfqpoint{2.770489in}{0.739656in}}%
\pgfpathlineto{\pgfqpoint{2.770191in}{0.739656in}}%
\pgfpathlineto{\pgfqpoint{2.769894in}{0.739656in}}%
\pgfpathlineto{\pgfqpoint{2.769596in}{0.739656in}}%
\pgfpathlineto{\pgfqpoint{2.769299in}{0.739656in}}%
\pgfpathlineto{\pgfqpoint{2.769001in}{0.739656in}}%
\pgfpathlineto{\pgfqpoint{2.768704in}{0.739656in}}%
\pgfpathlineto{\pgfqpoint{2.768406in}{0.739656in}}%
\pgfpathlineto{\pgfqpoint{2.768109in}{0.739656in}}%
\pgfpathlineto{\pgfqpoint{2.767811in}{0.739656in}}%
\pgfpathlineto{\pgfqpoint{2.767514in}{0.739656in}}%
\pgfpathlineto{\pgfqpoint{2.767216in}{0.739656in}}%
\pgfpathlineto{\pgfqpoint{2.766919in}{0.739656in}}%
\pgfpathlineto{\pgfqpoint{2.766621in}{0.739656in}}%
\pgfpathlineto{\pgfqpoint{2.766324in}{0.739656in}}%
\pgfpathlineto{\pgfqpoint{2.766026in}{0.739656in}}%
\pgfpathlineto{\pgfqpoint{2.765729in}{0.739656in}}%
\pgfpathlineto{\pgfqpoint{2.765431in}{0.739656in}}%
\pgfpathlineto{\pgfqpoint{2.765134in}{0.739656in}}%
\pgfpathlineto{\pgfqpoint{2.764837in}{0.739656in}}%
\pgfpathlineto{\pgfqpoint{2.764539in}{0.739656in}}%
\pgfpathlineto{\pgfqpoint{2.764242in}{0.739656in}}%
\pgfpathlineto{\pgfqpoint{2.763944in}{0.739656in}}%
\pgfpathlineto{\pgfqpoint{2.763647in}{0.739656in}}%
\pgfpathlineto{\pgfqpoint{2.763349in}{0.739656in}}%
\pgfpathlineto{\pgfqpoint{2.763052in}{0.739656in}}%
\pgfpathlineto{\pgfqpoint{2.762754in}{0.739656in}}%
\pgfpathlineto{\pgfqpoint{2.762457in}{0.739656in}}%
\pgfpathlineto{\pgfqpoint{2.762159in}{0.739656in}}%
\pgfpathlineto{\pgfqpoint{2.761862in}{0.739656in}}%
\pgfpathlineto{\pgfqpoint{2.761564in}{0.739656in}}%
\pgfpathlineto{\pgfqpoint{2.761267in}{0.739656in}}%
\pgfpathlineto{\pgfqpoint{2.760969in}{0.739656in}}%
\pgfpathlineto{\pgfqpoint{2.760672in}{0.739656in}}%
\pgfpathlineto{\pgfqpoint{2.760374in}{0.739656in}}%
\pgfpathlineto{\pgfqpoint{2.760077in}{0.739656in}}%
\pgfpathlineto{\pgfqpoint{2.759779in}{0.739656in}}%
\pgfpathlineto{\pgfqpoint{2.759482in}{0.739656in}}%
\pgfpathlineto{\pgfqpoint{2.759184in}{0.739656in}}%
\pgfpathlineto{\pgfqpoint{2.758887in}{0.739656in}}%
\pgfpathlineto{\pgfqpoint{2.758590in}{0.739656in}}%
\pgfpathlineto{\pgfqpoint{2.758292in}{0.739656in}}%
\pgfpathlineto{\pgfqpoint{2.757995in}{0.739656in}}%
\pgfpathlineto{\pgfqpoint{2.757697in}{0.739656in}}%
\pgfpathlineto{\pgfqpoint{2.757400in}{0.739656in}}%
\pgfpathlineto{\pgfqpoint{2.757102in}{0.739656in}}%
\pgfpathlineto{\pgfqpoint{2.756805in}{0.739656in}}%
\pgfpathlineto{\pgfqpoint{2.756507in}{0.739656in}}%
\pgfpathlineto{\pgfqpoint{2.756210in}{0.739656in}}%
\pgfpathlineto{\pgfqpoint{2.755912in}{0.739656in}}%
\pgfpathlineto{\pgfqpoint{2.755615in}{0.739656in}}%
\pgfpathlineto{\pgfqpoint{2.755317in}{0.739656in}}%
\pgfpathlineto{\pgfqpoint{2.755020in}{0.739656in}}%
\pgfpathlineto{\pgfqpoint{2.754722in}{0.739656in}}%
\pgfpathlineto{\pgfqpoint{2.754425in}{0.739656in}}%
\pgfpathlineto{\pgfqpoint{2.754127in}{0.739656in}}%
\pgfpathlineto{\pgfqpoint{2.753830in}{0.739656in}}%
\pgfpathlineto{\pgfqpoint{2.753532in}{0.739656in}}%
\pgfpathlineto{\pgfqpoint{2.753235in}{0.739656in}}%
\pgfpathlineto{\pgfqpoint{2.752937in}{0.739656in}}%
\pgfpathlineto{\pgfqpoint{2.752640in}{0.739656in}}%
\pgfpathlineto{\pgfqpoint{2.752342in}{0.739656in}}%
\pgfpathlineto{\pgfqpoint{2.752045in}{0.739656in}}%
\pgfpathlineto{\pgfqpoint{2.751748in}{0.739656in}}%
\pgfpathlineto{\pgfqpoint{2.751450in}{0.739656in}}%
\pgfpathlineto{\pgfqpoint{2.751153in}{0.739656in}}%
\pgfpathlineto{\pgfqpoint{2.750855in}{0.739656in}}%
\pgfpathlineto{\pgfqpoint{2.750558in}{0.739656in}}%
\pgfpathlineto{\pgfqpoint{2.750260in}{0.739656in}}%
\pgfpathlineto{\pgfqpoint{2.749963in}{0.739656in}}%
\pgfpathlineto{\pgfqpoint{2.749665in}{0.739656in}}%
\pgfpathlineto{\pgfqpoint{2.749368in}{0.739656in}}%
\pgfpathlineto{\pgfqpoint{2.749070in}{0.739656in}}%
\pgfpathlineto{\pgfqpoint{2.748773in}{0.739656in}}%
\pgfpathlineto{\pgfqpoint{2.748475in}{0.739656in}}%
\pgfpathlineto{\pgfqpoint{2.748178in}{0.739656in}}%
\pgfpathlineto{\pgfqpoint{2.747880in}{0.739656in}}%
\pgfpathlineto{\pgfqpoint{2.747583in}{0.739656in}}%
\pgfpathlineto{\pgfqpoint{2.747285in}{0.739656in}}%
\pgfpathlineto{\pgfqpoint{2.746988in}{0.739656in}}%
\pgfpathlineto{\pgfqpoint{2.746690in}{0.739656in}}%
\pgfpathlineto{\pgfqpoint{2.746393in}{0.739656in}}%
\pgfpathlineto{\pgfqpoint{2.746095in}{0.739656in}}%
\pgfpathlineto{\pgfqpoint{2.745798in}{0.739656in}}%
\pgfpathlineto{\pgfqpoint{2.745500in}{0.739656in}}%
\pgfpathlineto{\pgfqpoint{2.745203in}{0.739656in}}%
\pgfpathlineto{\pgfqpoint{2.744906in}{0.739656in}}%
\pgfpathlineto{\pgfqpoint{2.744608in}{0.739656in}}%
\pgfpathlineto{\pgfqpoint{2.744311in}{0.739656in}}%
\pgfpathlineto{\pgfqpoint{2.744013in}{0.739656in}}%
\pgfpathlineto{\pgfqpoint{2.743716in}{0.739656in}}%
\pgfpathlineto{\pgfqpoint{2.743418in}{0.739656in}}%
\pgfpathlineto{\pgfqpoint{2.743121in}{0.739656in}}%
\pgfpathlineto{\pgfqpoint{2.742823in}{0.739656in}}%
\pgfpathlineto{\pgfqpoint{2.742526in}{0.739656in}}%
\pgfpathlineto{\pgfqpoint{2.742228in}{0.739656in}}%
\pgfpathlineto{\pgfqpoint{2.741931in}{0.739656in}}%
\pgfpathlineto{\pgfqpoint{2.741633in}{0.739656in}}%
\pgfpathlineto{\pgfqpoint{2.741336in}{0.739656in}}%
\pgfpathlineto{\pgfqpoint{2.741038in}{0.739656in}}%
\pgfpathlineto{\pgfqpoint{2.740741in}{0.739656in}}%
\pgfpathlineto{\pgfqpoint{2.740443in}{0.739656in}}%
\pgfpathlineto{\pgfqpoint{2.740146in}{0.739656in}}%
\pgfpathlineto{\pgfqpoint{2.739848in}{0.739656in}}%
\pgfpathlineto{\pgfqpoint{2.739551in}{0.739656in}}%
\pgfpathlineto{\pgfqpoint{2.739253in}{0.739656in}}%
\pgfpathlineto{\pgfqpoint{2.738956in}{0.739656in}}%
\pgfpathlineto{\pgfqpoint{2.738659in}{0.739656in}}%
\pgfpathlineto{\pgfqpoint{2.738361in}{0.739656in}}%
\pgfpathlineto{\pgfqpoint{2.738064in}{0.739656in}}%
\pgfpathlineto{\pgfqpoint{2.737766in}{0.739656in}}%
\pgfpathlineto{\pgfqpoint{2.737469in}{0.739656in}}%
\pgfpathlineto{\pgfqpoint{2.737171in}{0.739656in}}%
\pgfpathlineto{\pgfqpoint{2.736874in}{0.739656in}}%
\pgfpathlineto{\pgfqpoint{2.736576in}{0.739656in}}%
\pgfpathlineto{\pgfqpoint{2.736279in}{0.739656in}}%
\pgfpathlineto{\pgfqpoint{2.735981in}{0.739656in}}%
\pgfpathlineto{\pgfqpoint{2.735684in}{0.739656in}}%
\pgfpathlineto{\pgfqpoint{2.735386in}{0.739656in}}%
\pgfpathlineto{\pgfqpoint{2.735089in}{0.739656in}}%
\pgfpathlineto{\pgfqpoint{2.734791in}{0.739656in}}%
\pgfpathlineto{\pgfqpoint{2.734494in}{0.739656in}}%
\pgfpathlineto{\pgfqpoint{2.734196in}{0.739656in}}%
\pgfpathlineto{\pgfqpoint{2.733899in}{0.739656in}}%
\pgfpathlineto{\pgfqpoint{2.733601in}{0.739656in}}%
\pgfpathlineto{\pgfqpoint{2.733304in}{0.739656in}}%
\pgfpathlineto{\pgfqpoint{2.733006in}{0.739656in}}%
\pgfpathlineto{\pgfqpoint{2.732709in}{0.739656in}}%
\pgfpathlineto{\pgfqpoint{2.732411in}{0.739656in}}%
\pgfpathlineto{\pgfqpoint{2.732114in}{0.739656in}}%
\pgfpathlineto{\pgfqpoint{2.731817in}{0.739656in}}%
\pgfpathlineto{\pgfqpoint{2.731519in}{0.739656in}}%
\pgfpathlineto{\pgfqpoint{2.731222in}{0.739656in}}%
\pgfpathlineto{\pgfqpoint{2.730924in}{0.739656in}}%
\pgfpathlineto{\pgfqpoint{2.730627in}{0.739656in}}%
\pgfpathlineto{\pgfqpoint{2.730329in}{0.739656in}}%
\pgfpathlineto{\pgfqpoint{2.730032in}{0.739656in}}%
\pgfpathlineto{\pgfqpoint{2.729734in}{0.739656in}}%
\pgfpathlineto{\pgfqpoint{2.729437in}{0.739656in}}%
\pgfpathlineto{\pgfqpoint{2.729139in}{0.739656in}}%
\pgfpathlineto{\pgfqpoint{2.728842in}{0.739656in}}%
\pgfpathlineto{\pgfqpoint{2.728544in}{0.739656in}}%
\pgfpathlineto{\pgfqpoint{2.728247in}{0.739656in}}%
\pgfpathlineto{\pgfqpoint{2.727949in}{0.739656in}}%
\pgfpathlineto{\pgfqpoint{2.727652in}{0.739656in}}%
\pgfpathlineto{\pgfqpoint{2.727354in}{0.739656in}}%
\pgfpathlineto{\pgfqpoint{2.727057in}{0.739656in}}%
\pgfpathlineto{\pgfqpoint{2.726759in}{0.739656in}}%
\pgfpathlineto{\pgfqpoint{2.726462in}{0.739656in}}%
\pgfpathlineto{\pgfqpoint{2.726164in}{0.739656in}}%
\pgfpathlineto{\pgfqpoint{2.725867in}{0.739656in}}%
\pgfpathlineto{\pgfqpoint{2.725569in}{0.739656in}}%
\pgfpathlineto{\pgfqpoint{2.725272in}{0.739656in}}%
\pgfpathlineto{\pgfqpoint{2.724975in}{0.739656in}}%
\pgfpathlineto{\pgfqpoint{2.724677in}{0.739656in}}%
\pgfpathlineto{\pgfqpoint{2.724380in}{0.739656in}}%
\pgfpathlineto{\pgfqpoint{2.724082in}{0.739656in}}%
\pgfpathlineto{\pgfqpoint{2.723785in}{0.739656in}}%
\pgfpathlineto{\pgfqpoint{2.723487in}{0.739656in}}%
\pgfpathlineto{\pgfqpoint{2.723190in}{0.739656in}}%
\pgfpathlineto{\pgfqpoint{2.722892in}{0.739656in}}%
\pgfpathlineto{\pgfqpoint{2.722595in}{0.739656in}}%
\pgfpathlineto{\pgfqpoint{2.722297in}{0.739656in}}%
\pgfpathlineto{\pgfqpoint{2.722000in}{0.739656in}}%
\pgfpathlineto{\pgfqpoint{2.721702in}{0.739656in}}%
\pgfpathlineto{\pgfqpoint{2.721405in}{0.739656in}}%
\pgfpathlineto{\pgfqpoint{2.721107in}{0.739656in}}%
\pgfpathlineto{\pgfqpoint{2.720810in}{0.739656in}}%
\pgfpathlineto{\pgfqpoint{2.720512in}{0.739656in}}%
\pgfpathlineto{\pgfqpoint{2.720215in}{0.739656in}}%
\pgfpathlineto{\pgfqpoint{2.719917in}{0.739656in}}%
\pgfpathlineto{\pgfqpoint{2.719620in}{0.739656in}}%
\pgfpathlineto{\pgfqpoint{2.719322in}{0.739656in}}%
\pgfpathlineto{\pgfqpoint{2.719025in}{0.739656in}}%
\pgfpathlineto{\pgfqpoint{2.718728in}{0.739656in}}%
\pgfpathlineto{\pgfqpoint{2.718430in}{0.739656in}}%
\pgfpathlineto{\pgfqpoint{2.718133in}{0.739656in}}%
\pgfpathlineto{\pgfqpoint{2.717835in}{0.739656in}}%
\pgfpathlineto{\pgfqpoint{2.717538in}{0.739656in}}%
\pgfpathlineto{\pgfqpoint{2.717240in}{0.739656in}}%
\pgfpathlineto{\pgfqpoint{2.716943in}{0.739656in}}%
\pgfpathlineto{\pgfqpoint{2.716645in}{0.739656in}}%
\pgfpathlineto{\pgfqpoint{2.716348in}{0.739656in}}%
\pgfpathlineto{\pgfqpoint{2.716050in}{0.739656in}}%
\pgfpathlineto{\pgfqpoint{2.715753in}{0.739656in}}%
\pgfpathlineto{\pgfqpoint{2.715455in}{0.739656in}}%
\pgfpathlineto{\pgfqpoint{2.715158in}{0.739656in}}%
\pgfpathlineto{\pgfqpoint{2.714860in}{0.739656in}}%
\pgfpathlineto{\pgfqpoint{2.714563in}{0.739656in}}%
\pgfpathlineto{\pgfqpoint{2.714265in}{0.739656in}}%
\pgfpathlineto{\pgfqpoint{2.713968in}{0.739656in}}%
\pgfpathlineto{\pgfqpoint{2.713670in}{0.739656in}}%
\pgfpathlineto{\pgfqpoint{2.713373in}{0.739656in}}%
\pgfpathlineto{\pgfqpoint{2.713075in}{0.739656in}}%
\pgfpathlineto{\pgfqpoint{2.712778in}{0.739656in}}%
\pgfpathlineto{\pgfqpoint{2.712480in}{0.739656in}}%
\pgfpathlineto{\pgfqpoint{2.712183in}{0.739656in}}%
\pgfpathlineto{\pgfqpoint{2.711886in}{0.739656in}}%
\pgfpathlineto{\pgfqpoint{2.711588in}{0.739656in}}%
\pgfpathlineto{\pgfqpoint{2.711291in}{0.739656in}}%
\pgfpathlineto{\pgfqpoint{2.710993in}{0.739656in}}%
\pgfpathlineto{\pgfqpoint{2.710696in}{0.739656in}}%
\pgfpathlineto{\pgfqpoint{2.710398in}{0.739656in}}%
\pgfpathlineto{\pgfqpoint{2.710101in}{0.739656in}}%
\pgfpathlineto{\pgfqpoint{2.709803in}{0.739656in}}%
\pgfpathlineto{\pgfqpoint{2.709506in}{0.739656in}}%
\pgfpathlineto{\pgfqpoint{2.709208in}{0.739656in}}%
\pgfpathlineto{\pgfqpoint{2.708911in}{0.739656in}}%
\pgfpathlineto{\pgfqpoint{2.708613in}{0.739656in}}%
\pgfpathlineto{\pgfqpoint{2.708316in}{0.739656in}}%
\pgfpathlineto{\pgfqpoint{2.708018in}{0.739656in}}%
\pgfpathlineto{\pgfqpoint{2.707721in}{0.739656in}}%
\pgfpathlineto{\pgfqpoint{2.707423in}{0.739656in}}%
\pgfpathlineto{\pgfqpoint{2.707126in}{0.739656in}}%
\pgfpathlineto{\pgfqpoint{2.706828in}{0.739656in}}%
\pgfpathlineto{\pgfqpoint{2.706531in}{0.739656in}}%
\pgfpathlineto{\pgfqpoint{2.706233in}{0.739656in}}%
\pgfpathlineto{\pgfqpoint{2.705936in}{0.739656in}}%
\pgfpathlineto{\pgfqpoint{2.705638in}{0.739656in}}%
\pgfpathlineto{\pgfqpoint{2.705341in}{0.739656in}}%
\pgfpathlineto{\pgfqpoint{2.705044in}{0.739656in}}%
\pgfpathlineto{\pgfqpoint{2.704746in}{0.739656in}}%
\pgfpathlineto{\pgfqpoint{2.704449in}{0.739656in}}%
\pgfpathlineto{\pgfqpoint{2.704151in}{0.739656in}}%
\pgfpathlineto{\pgfqpoint{2.703854in}{0.739656in}}%
\pgfpathlineto{\pgfqpoint{2.703556in}{0.739656in}}%
\pgfpathlineto{\pgfqpoint{2.703259in}{0.739656in}}%
\pgfpathlineto{\pgfqpoint{2.702961in}{0.739656in}}%
\pgfpathlineto{\pgfqpoint{2.702664in}{0.739656in}}%
\pgfpathlineto{\pgfqpoint{2.702366in}{0.739656in}}%
\pgfpathlineto{\pgfqpoint{2.702069in}{0.739656in}}%
\pgfpathlineto{\pgfqpoint{2.701771in}{0.739656in}}%
\pgfpathlineto{\pgfqpoint{2.701474in}{0.739656in}}%
\pgfpathlineto{\pgfqpoint{2.701176in}{0.739656in}}%
\pgfpathlineto{\pgfqpoint{2.700879in}{0.739656in}}%
\pgfpathlineto{\pgfqpoint{2.700581in}{0.739656in}}%
\pgfpathlineto{\pgfqpoint{2.700284in}{0.739656in}}%
\pgfpathlineto{\pgfqpoint{2.699986in}{0.739656in}}%
\pgfpathlineto{\pgfqpoint{2.699689in}{0.739656in}}%
\pgfpathlineto{\pgfqpoint{2.699391in}{0.739656in}}%
\pgfpathlineto{\pgfqpoint{2.699094in}{0.739656in}}%
\pgfpathlineto{\pgfqpoint{2.698797in}{0.739656in}}%
\pgfpathlineto{\pgfqpoint{2.698499in}{0.739656in}}%
\pgfpathlineto{\pgfqpoint{2.698202in}{0.739656in}}%
\pgfpathlineto{\pgfqpoint{2.697904in}{0.739656in}}%
\pgfpathlineto{\pgfqpoint{2.697607in}{0.739656in}}%
\pgfpathlineto{\pgfqpoint{2.697309in}{0.739656in}}%
\pgfpathlineto{\pgfqpoint{2.697012in}{0.739656in}}%
\pgfpathlineto{\pgfqpoint{2.696714in}{0.739656in}}%
\pgfpathlineto{\pgfqpoint{2.696417in}{0.739656in}}%
\pgfpathlineto{\pgfqpoint{2.696119in}{0.739656in}}%
\pgfpathlineto{\pgfqpoint{2.695822in}{0.739656in}}%
\pgfpathlineto{\pgfqpoint{2.695524in}{0.739656in}}%
\pgfpathlineto{\pgfqpoint{2.695227in}{0.739656in}}%
\pgfpathlineto{\pgfqpoint{2.694929in}{0.739656in}}%
\pgfpathlineto{\pgfqpoint{2.694632in}{0.739656in}}%
\pgfpathlineto{\pgfqpoint{2.694334in}{0.739656in}}%
\pgfpathlineto{\pgfqpoint{2.694037in}{0.739656in}}%
\pgfpathlineto{\pgfqpoint{2.693739in}{0.739656in}}%
\pgfpathlineto{\pgfqpoint{2.693442in}{0.739656in}}%
\pgfpathlineto{\pgfqpoint{2.693144in}{0.739656in}}%
\pgfpathlineto{\pgfqpoint{2.692847in}{0.739656in}}%
\pgfpathlineto{\pgfqpoint{2.692549in}{0.739656in}}%
\pgfpathlineto{\pgfqpoint{2.692252in}{0.739656in}}%
\pgfpathlineto{\pgfqpoint{2.691955in}{0.739656in}}%
\pgfpathlineto{\pgfqpoint{2.691657in}{0.739656in}}%
\pgfpathlineto{\pgfqpoint{2.691360in}{0.739656in}}%
\pgfpathlineto{\pgfqpoint{2.691062in}{0.739656in}}%
\pgfpathlineto{\pgfqpoint{2.690765in}{0.739656in}}%
\pgfpathlineto{\pgfqpoint{2.690467in}{0.739656in}}%
\pgfpathlineto{\pgfqpoint{2.690170in}{0.739656in}}%
\pgfpathlineto{\pgfqpoint{2.689872in}{0.739656in}}%
\pgfpathlineto{\pgfqpoint{2.689575in}{0.739656in}}%
\pgfpathlineto{\pgfqpoint{2.689277in}{0.739656in}}%
\pgfpathlineto{\pgfqpoint{2.688980in}{0.739656in}}%
\pgfpathlineto{\pgfqpoint{2.688682in}{0.739656in}}%
\pgfpathlineto{\pgfqpoint{2.688385in}{0.739656in}}%
\pgfpathlineto{\pgfqpoint{2.688087in}{0.739656in}}%
\pgfpathlineto{\pgfqpoint{2.687790in}{0.739656in}}%
\pgfpathlineto{\pgfqpoint{2.687492in}{0.739656in}}%
\pgfpathlineto{\pgfqpoint{2.687195in}{0.739656in}}%
\pgfpathlineto{\pgfqpoint{2.686897in}{0.739656in}}%
\pgfpathlineto{\pgfqpoint{2.686600in}{0.739656in}}%
\pgfpathlineto{\pgfqpoint{2.686302in}{0.739656in}}%
\pgfpathlineto{\pgfqpoint{2.686005in}{0.739656in}}%
\pgfpathlineto{\pgfqpoint{2.685707in}{0.739656in}}%
\pgfpathlineto{\pgfqpoint{2.685410in}{0.739656in}}%
\pgfpathlineto{\pgfqpoint{2.685113in}{0.739656in}}%
\pgfpathlineto{\pgfqpoint{2.684815in}{0.739656in}}%
\pgfpathlineto{\pgfqpoint{2.684518in}{0.739656in}}%
\pgfpathlineto{\pgfqpoint{2.684220in}{0.739656in}}%
\pgfpathlineto{\pgfqpoint{2.683923in}{0.739656in}}%
\pgfpathlineto{\pgfqpoint{2.683625in}{0.739656in}}%
\pgfpathlineto{\pgfqpoint{2.683328in}{0.739656in}}%
\pgfpathlineto{\pgfqpoint{2.683030in}{0.739656in}}%
\pgfpathlineto{\pgfqpoint{2.682733in}{0.739656in}}%
\pgfpathlineto{\pgfqpoint{2.682435in}{0.739656in}}%
\pgfpathlineto{\pgfqpoint{2.682138in}{0.739656in}}%
\pgfpathlineto{\pgfqpoint{2.681840in}{0.739656in}}%
\pgfpathlineto{\pgfqpoint{2.681543in}{0.739656in}}%
\pgfpathlineto{\pgfqpoint{2.681245in}{0.739656in}}%
\pgfpathlineto{\pgfqpoint{2.680948in}{0.739656in}}%
\pgfpathlineto{\pgfqpoint{2.680650in}{0.739656in}}%
\pgfpathlineto{\pgfqpoint{2.680353in}{0.739656in}}%
\pgfpathlineto{\pgfqpoint{2.680055in}{0.739656in}}%
\pgfpathlineto{\pgfqpoint{2.679758in}{0.739656in}}%
\pgfpathlineto{\pgfqpoint{2.679460in}{0.739656in}}%
\pgfpathlineto{\pgfqpoint{2.679163in}{0.739656in}}%
\pgfpathlineto{\pgfqpoint{2.678866in}{0.739656in}}%
\pgfpathlineto{\pgfqpoint{2.678568in}{0.739656in}}%
\pgfpathlineto{\pgfqpoint{2.678271in}{0.739656in}}%
\pgfpathlineto{\pgfqpoint{2.677973in}{0.739656in}}%
\pgfpathlineto{\pgfqpoint{2.677676in}{0.739656in}}%
\pgfpathlineto{\pgfqpoint{2.677378in}{0.739656in}}%
\pgfpathlineto{\pgfqpoint{2.677081in}{0.739656in}}%
\pgfpathlineto{\pgfqpoint{2.676783in}{0.739656in}}%
\pgfpathlineto{\pgfqpoint{2.676486in}{0.739656in}}%
\pgfpathlineto{\pgfqpoint{2.676188in}{0.739656in}}%
\pgfpathlineto{\pgfqpoint{2.675891in}{0.739656in}}%
\pgfpathlineto{\pgfqpoint{2.675593in}{0.739656in}}%
\pgfpathlineto{\pgfqpoint{2.675296in}{0.739656in}}%
\pgfpathlineto{\pgfqpoint{2.674998in}{0.739656in}}%
\pgfpathlineto{\pgfqpoint{2.674701in}{0.739656in}}%
\pgfpathlineto{\pgfqpoint{2.674403in}{0.739656in}}%
\pgfpathlineto{\pgfqpoint{2.674106in}{0.739656in}}%
\pgfpathlineto{\pgfqpoint{2.673808in}{0.739656in}}%
\pgfpathlineto{\pgfqpoint{2.673511in}{0.739656in}}%
\pgfpathlineto{\pgfqpoint{2.673213in}{0.739656in}}%
\pgfpathlineto{\pgfqpoint{2.672916in}{0.739656in}}%
\pgfpathlineto{\pgfqpoint{2.672618in}{0.739656in}}%
\pgfpathlineto{\pgfqpoint{2.672321in}{0.739656in}}%
\pgfpathlineto{\pgfqpoint{2.672024in}{0.739656in}}%
\pgfpathlineto{\pgfqpoint{2.671726in}{0.739656in}}%
\pgfpathlineto{\pgfqpoint{2.671429in}{0.739656in}}%
\pgfpathlineto{\pgfqpoint{2.671131in}{0.739656in}}%
\pgfpathlineto{\pgfqpoint{2.670834in}{0.739656in}}%
\pgfpathlineto{\pgfqpoint{2.670536in}{0.739656in}}%
\pgfpathlineto{\pgfqpoint{2.670239in}{0.739656in}}%
\pgfpathlineto{\pgfqpoint{2.669941in}{0.739656in}}%
\pgfpathlineto{\pgfqpoint{2.669644in}{0.739656in}}%
\pgfpathlineto{\pgfqpoint{2.669346in}{0.739656in}}%
\pgfpathlineto{\pgfqpoint{2.669049in}{0.739656in}}%
\pgfpathlineto{\pgfqpoint{2.668751in}{0.739656in}}%
\pgfpathlineto{\pgfqpoint{2.668454in}{0.739656in}}%
\pgfpathlineto{\pgfqpoint{2.668156in}{0.739656in}}%
\pgfpathlineto{\pgfqpoint{2.667859in}{0.739656in}}%
\pgfpathlineto{\pgfqpoint{2.667561in}{0.739656in}}%
\pgfpathlineto{\pgfqpoint{2.667264in}{0.739656in}}%
\pgfpathlineto{\pgfqpoint{2.666966in}{0.739656in}}%
\pgfpathlineto{\pgfqpoint{2.666669in}{0.739656in}}%
\pgfpathlineto{\pgfqpoint{2.666371in}{0.739656in}}%
\pgfpathlineto{\pgfqpoint{2.666074in}{0.739656in}}%
\pgfpathlineto{\pgfqpoint{2.665776in}{0.739656in}}%
\pgfpathlineto{\pgfqpoint{2.665479in}{0.739656in}}%
\pgfpathlineto{\pgfqpoint{2.665182in}{0.739656in}}%
\pgfpathlineto{\pgfqpoint{2.664884in}{0.739656in}}%
\pgfpathlineto{\pgfqpoint{2.664587in}{0.739656in}}%
\pgfpathlineto{\pgfqpoint{2.664289in}{0.739656in}}%
\pgfpathlineto{\pgfqpoint{2.663992in}{0.739656in}}%
\pgfpathlineto{\pgfqpoint{2.663694in}{0.739656in}}%
\pgfpathlineto{\pgfqpoint{2.663397in}{0.739656in}}%
\pgfpathlineto{\pgfqpoint{2.663099in}{0.739656in}}%
\pgfpathlineto{\pgfqpoint{2.662802in}{0.739656in}}%
\pgfpathlineto{\pgfqpoint{2.662504in}{0.739656in}}%
\pgfpathlineto{\pgfqpoint{2.662207in}{0.739656in}}%
\pgfpathlineto{\pgfqpoint{2.661909in}{0.739656in}}%
\pgfpathlineto{\pgfqpoint{2.661612in}{0.739656in}}%
\pgfpathlineto{\pgfqpoint{2.661314in}{0.739656in}}%
\pgfpathlineto{\pgfqpoint{2.661017in}{0.739656in}}%
\pgfpathlineto{\pgfqpoint{2.660719in}{0.739656in}}%
\pgfpathlineto{\pgfqpoint{2.660422in}{0.739656in}}%
\pgfpathlineto{\pgfqpoint{2.660124in}{0.739656in}}%
\pgfpathlineto{\pgfqpoint{2.659827in}{0.739656in}}%
\pgfpathlineto{\pgfqpoint{2.659529in}{0.739656in}}%
\pgfpathlineto{\pgfqpoint{2.659232in}{0.739656in}}%
\pgfpathlineto{\pgfqpoint{2.658934in}{0.739656in}}%
\pgfpathlineto{\pgfqpoint{2.658637in}{0.739656in}}%
\pgfpathlineto{\pgfqpoint{2.658340in}{0.739656in}}%
\pgfpathlineto{\pgfqpoint{2.658042in}{0.739656in}}%
\pgfpathlineto{\pgfqpoint{2.657745in}{0.739656in}}%
\pgfpathlineto{\pgfqpoint{2.657447in}{0.739656in}}%
\pgfpathlineto{\pgfqpoint{2.657150in}{0.739656in}}%
\pgfpathlineto{\pgfqpoint{2.656852in}{0.739656in}}%
\pgfpathlineto{\pgfqpoint{2.656555in}{0.739656in}}%
\pgfpathlineto{\pgfqpoint{2.656257in}{0.739656in}}%
\pgfpathlineto{\pgfqpoint{2.655960in}{0.739656in}}%
\pgfpathlineto{\pgfqpoint{2.655662in}{0.739656in}}%
\pgfpathlineto{\pgfqpoint{2.655365in}{0.739656in}}%
\pgfpathlineto{\pgfqpoint{2.655067in}{0.739656in}}%
\pgfpathlineto{\pgfqpoint{2.654770in}{0.739656in}}%
\pgfpathlineto{\pgfqpoint{2.654472in}{0.739656in}}%
\pgfpathlineto{\pgfqpoint{2.654175in}{0.739656in}}%
\pgfpathlineto{\pgfqpoint{2.653877in}{0.739656in}}%
\pgfpathlineto{\pgfqpoint{2.653580in}{0.739656in}}%
\pgfpathlineto{\pgfqpoint{2.653282in}{0.739656in}}%
\pgfpathlineto{\pgfqpoint{2.652985in}{0.739656in}}%
\pgfpathlineto{\pgfqpoint{2.652687in}{0.739656in}}%
\pgfpathlineto{\pgfqpoint{2.652390in}{0.739656in}}%
\pgfpathlineto{\pgfqpoint{2.652093in}{0.739656in}}%
\pgfpathlineto{\pgfqpoint{2.651795in}{0.739656in}}%
\pgfpathlineto{\pgfqpoint{2.651498in}{0.739656in}}%
\pgfpathlineto{\pgfqpoint{2.651200in}{0.739656in}}%
\pgfpathlineto{\pgfqpoint{2.650903in}{0.739656in}}%
\pgfpathlineto{\pgfqpoint{2.650605in}{0.739656in}}%
\pgfpathlineto{\pgfqpoint{2.650308in}{0.739656in}}%
\pgfpathlineto{\pgfqpoint{2.650010in}{0.739656in}}%
\pgfpathlineto{\pgfqpoint{2.649713in}{0.739656in}}%
\pgfpathlineto{\pgfqpoint{2.649415in}{0.739656in}}%
\pgfpathlineto{\pgfqpoint{2.649118in}{0.739656in}}%
\pgfpathlineto{\pgfqpoint{2.648820in}{0.739656in}}%
\pgfpathlineto{\pgfqpoint{2.648523in}{0.739656in}}%
\pgfpathlineto{\pgfqpoint{2.648225in}{0.739656in}}%
\pgfpathlineto{\pgfqpoint{2.647928in}{0.739656in}}%
\pgfpathlineto{\pgfqpoint{2.647630in}{0.739656in}}%
\pgfpathlineto{\pgfqpoint{2.647333in}{0.739656in}}%
\pgfpathlineto{\pgfqpoint{2.647035in}{0.739656in}}%
\pgfpathlineto{\pgfqpoint{2.646738in}{0.739656in}}%
\pgfpathlineto{\pgfqpoint{2.646440in}{0.739656in}}%
\pgfpathlineto{\pgfqpoint{2.646143in}{0.739656in}}%
\pgfpathlineto{\pgfqpoint{2.645845in}{0.739656in}}%
\pgfpathlineto{\pgfqpoint{2.645548in}{0.739656in}}%
\pgfpathlineto{\pgfqpoint{2.645251in}{0.739656in}}%
\pgfpathlineto{\pgfqpoint{2.644953in}{0.739656in}}%
\pgfpathlineto{\pgfqpoint{2.644656in}{0.739656in}}%
\pgfpathlineto{\pgfqpoint{2.644358in}{0.739656in}}%
\pgfpathlineto{\pgfqpoint{2.644061in}{0.739656in}}%
\pgfpathlineto{\pgfqpoint{2.643763in}{0.739656in}}%
\pgfpathlineto{\pgfqpoint{2.643466in}{0.739656in}}%
\pgfpathlineto{\pgfqpoint{2.643168in}{0.739656in}}%
\pgfpathlineto{\pgfqpoint{2.642871in}{0.739656in}}%
\pgfpathlineto{\pgfqpoint{2.642573in}{0.739656in}}%
\pgfpathlineto{\pgfqpoint{2.642276in}{0.739656in}}%
\pgfpathlineto{\pgfqpoint{2.641978in}{0.739656in}}%
\pgfpathlineto{\pgfqpoint{2.641681in}{0.739656in}}%
\pgfpathlineto{\pgfqpoint{2.641383in}{0.739656in}}%
\pgfpathlineto{\pgfqpoint{2.641086in}{0.739656in}}%
\pgfpathlineto{\pgfqpoint{2.640788in}{0.739656in}}%
\pgfpathlineto{\pgfqpoint{2.640491in}{0.739656in}}%
\pgfpathlineto{\pgfqpoint{2.640193in}{0.739656in}}%
\pgfpathlineto{\pgfqpoint{2.639896in}{0.739656in}}%
\pgfpathlineto{\pgfqpoint{2.639598in}{0.739656in}}%
\pgfpathlineto{\pgfqpoint{2.639301in}{0.739656in}}%
\pgfpathlineto{\pgfqpoint{2.639003in}{0.739656in}}%
\pgfpathlineto{\pgfqpoint{2.638706in}{0.739656in}}%
\pgfpathlineto{\pgfqpoint{2.638409in}{0.739656in}}%
\pgfpathlineto{\pgfqpoint{2.638111in}{0.739656in}}%
\pgfpathlineto{\pgfqpoint{2.637814in}{0.739656in}}%
\pgfpathlineto{\pgfqpoint{2.637516in}{0.739656in}}%
\pgfpathlineto{\pgfqpoint{2.637219in}{0.739656in}}%
\pgfpathlineto{\pgfqpoint{2.636921in}{0.739656in}}%
\pgfpathlineto{\pgfqpoint{2.636624in}{0.739656in}}%
\pgfpathlineto{\pgfqpoint{2.636326in}{0.739656in}}%
\pgfpathlineto{\pgfqpoint{2.636029in}{0.739656in}}%
\pgfpathlineto{\pgfqpoint{2.635731in}{0.739656in}}%
\pgfpathlineto{\pgfqpoint{2.635434in}{0.739656in}}%
\pgfpathlineto{\pgfqpoint{2.635136in}{0.739656in}}%
\pgfpathlineto{\pgfqpoint{2.634839in}{0.739656in}}%
\pgfpathlineto{\pgfqpoint{2.634541in}{0.739656in}}%
\pgfpathlineto{\pgfqpoint{2.634244in}{0.739656in}}%
\pgfpathlineto{\pgfqpoint{2.633946in}{0.739656in}}%
\pgfpathlineto{\pgfqpoint{2.633649in}{0.739656in}}%
\pgfpathlineto{\pgfqpoint{2.633351in}{0.739656in}}%
\pgfpathlineto{\pgfqpoint{2.633054in}{0.739656in}}%
\pgfpathlineto{\pgfqpoint{2.632756in}{0.739656in}}%
\pgfpathlineto{\pgfqpoint{2.632459in}{0.739656in}}%
\pgfpathlineto{\pgfqpoint{2.632162in}{0.739656in}}%
\pgfpathlineto{\pgfqpoint{2.631864in}{0.739656in}}%
\pgfpathlineto{\pgfqpoint{2.631567in}{0.739656in}}%
\pgfpathlineto{\pgfqpoint{2.631269in}{0.739656in}}%
\pgfpathlineto{\pgfqpoint{2.630972in}{0.739656in}}%
\pgfpathlineto{\pgfqpoint{2.630674in}{0.739656in}}%
\pgfpathlineto{\pgfqpoint{2.630377in}{0.739656in}}%
\pgfpathlineto{\pgfqpoint{2.630079in}{0.739656in}}%
\pgfpathlineto{\pgfqpoint{2.629782in}{0.739656in}}%
\pgfpathlineto{\pgfqpoint{2.629484in}{0.739656in}}%
\pgfpathlineto{\pgfqpoint{2.629187in}{0.739656in}}%
\pgfpathlineto{\pgfqpoint{2.628889in}{0.739656in}}%
\pgfpathlineto{\pgfqpoint{2.628592in}{0.739656in}}%
\pgfpathlineto{\pgfqpoint{2.628294in}{0.739656in}}%
\pgfpathlineto{\pgfqpoint{2.627997in}{0.739656in}}%
\pgfpathlineto{\pgfqpoint{2.627699in}{0.739656in}}%
\pgfpathlineto{\pgfqpoint{2.627402in}{0.739656in}}%
\pgfpathlineto{\pgfqpoint{2.627104in}{0.739656in}}%
\pgfpathlineto{\pgfqpoint{2.626807in}{0.739656in}}%
\pgfpathlineto{\pgfqpoint{2.626509in}{0.739656in}}%
\pgfpathlineto{\pgfqpoint{2.626212in}{0.739656in}}%
\pgfpathlineto{\pgfqpoint{2.625914in}{0.739656in}}%
\pgfpathlineto{\pgfqpoint{2.625617in}{0.739656in}}%
\pgfpathlineto{\pgfqpoint{2.625320in}{0.739656in}}%
\pgfpathlineto{\pgfqpoint{2.625022in}{0.739656in}}%
\pgfpathlineto{\pgfqpoint{2.624725in}{0.739656in}}%
\pgfpathlineto{\pgfqpoint{2.624427in}{0.739656in}}%
\pgfpathlineto{\pgfqpoint{2.624130in}{0.739656in}}%
\pgfpathlineto{\pgfqpoint{2.623832in}{0.739656in}}%
\pgfpathlineto{\pgfqpoint{2.623535in}{0.739656in}}%
\pgfpathlineto{\pgfqpoint{2.623237in}{0.739656in}}%
\pgfpathlineto{\pgfqpoint{2.622940in}{0.739656in}}%
\pgfpathlineto{\pgfqpoint{2.622642in}{0.739656in}}%
\pgfpathlineto{\pgfqpoint{2.622345in}{0.739656in}}%
\pgfpathlineto{\pgfqpoint{2.622047in}{0.739656in}}%
\pgfpathlineto{\pgfqpoint{2.621750in}{0.739656in}}%
\pgfpathlineto{\pgfqpoint{2.621452in}{0.739656in}}%
\pgfpathlineto{\pgfqpoint{2.621155in}{0.739656in}}%
\pgfpathlineto{\pgfqpoint{2.620857in}{0.739656in}}%
\pgfpathlineto{\pgfqpoint{2.620560in}{0.739656in}}%
\pgfpathlineto{\pgfqpoint{2.620262in}{0.739656in}}%
\pgfpathlineto{\pgfqpoint{2.619965in}{0.739656in}}%
\pgfpathlineto{\pgfqpoint{2.619667in}{0.739656in}}%
\pgfpathlineto{\pgfqpoint{2.619370in}{0.739656in}}%
\pgfpathlineto{\pgfqpoint{2.619072in}{0.739656in}}%
\pgfpathlineto{\pgfqpoint{2.618775in}{0.739656in}}%
\pgfpathlineto{\pgfqpoint{2.618478in}{0.739656in}}%
\pgfpathlineto{\pgfqpoint{2.618180in}{0.739656in}}%
\pgfpathlineto{\pgfqpoint{2.617883in}{0.739656in}}%
\pgfpathlineto{\pgfqpoint{2.617585in}{0.739656in}}%
\pgfpathlineto{\pgfqpoint{2.617288in}{0.739656in}}%
\pgfpathlineto{\pgfqpoint{2.616990in}{0.739656in}}%
\pgfpathlineto{\pgfqpoint{2.616693in}{0.739656in}}%
\pgfpathlineto{\pgfqpoint{2.616395in}{0.739656in}}%
\pgfpathlineto{\pgfqpoint{2.616098in}{0.739656in}}%
\pgfpathlineto{\pgfqpoint{2.615800in}{0.739656in}}%
\pgfpathlineto{\pgfqpoint{2.615503in}{0.739656in}}%
\pgfpathlineto{\pgfqpoint{2.615205in}{0.739656in}}%
\pgfpathlineto{\pgfqpoint{2.614908in}{0.739656in}}%
\pgfpathlineto{\pgfqpoint{2.614610in}{0.739656in}}%
\pgfpathlineto{\pgfqpoint{2.614313in}{0.739656in}}%
\pgfpathlineto{\pgfqpoint{2.614015in}{0.739656in}}%
\pgfpathlineto{\pgfqpoint{2.613718in}{0.739656in}}%
\pgfpathlineto{\pgfqpoint{2.613420in}{0.739656in}}%
\pgfpathlineto{\pgfqpoint{2.613123in}{0.739656in}}%
\pgfpathlineto{\pgfqpoint{2.612825in}{0.739656in}}%
\pgfpathlineto{\pgfqpoint{2.612528in}{0.739656in}}%
\pgfpathlineto{\pgfqpoint{2.612231in}{0.739656in}}%
\pgfpathlineto{\pgfqpoint{2.611933in}{0.739656in}}%
\pgfpathlineto{\pgfqpoint{2.611636in}{0.739656in}}%
\pgfpathlineto{\pgfqpoint{2.611338in}{0.739656in}}%
\pgfpathlineto{\pgfqpoint{2.611041in}{0.739656in}}%
\pgfpathlineto{\pgfqpoint{2.610743in}{0.739656in}}%
\pgfpathlineto{\pgfqpoint{2.610446in}{0.739656in}}%
\pgfpathlineto{\pgfqpoint{2.610148in}{0.739656in}}%
\pgfpathlineto{\pgfqpoint{2.609851in}{0.739656in}}%
\pgfpathlineto{\pgfqpoint{2.609553in}{0.739656in}}%
\pgfpathlineto{\pgfqpoint{2.609256in}{0.739656in}}%
\pgfpathlineto{\pgfqpoint{2.608958in}{0.739656in}}%
\pgfpathlineto{\pgfqpoint{2.608661in}{0.739656in}}%
\pgfpathlineto{\pgfqpoint{2.608363in}{0.739656in}}%
\pgfpathlineto{\pgfqpoint{2.608066in}{0.739656in}}%
\pgfpathlineto{\pgfqpoint{2.607768in}{0.739656in}}%
\pgfpathlineto{\pgfqpoint{2.607471in}{0.739656in}}%
\pgfpathlineto{\pgfqpoint{2.607173in}{0.739656in}}%
\pgfpathlineto{\pgfqpoint{2.606876in}{0.739656in}}%
\pgfpathlineto{\pgfqpoint{2.606578in}{0.739656in}}%
\pgfpathlineto{\pgfqpoint{2.606281in}{0.739656in}}%
\pgfpathlineto{\pgfqpoint{2.605983in}{0.739656in}}%
\pgfpathlineto{\pgfqpoint{2.605686in}{0.739656in}}%
\pgfpathlineto{\pgfqpoint{2.605389in}{0.739656in}}%
\pgfpathlineto{\pgfqpoint{2.605091in}{0.739656in}}%
\pgfpathlineto{\pgfqpoint{2.604794in}{0.739656in}}%
\pgfpathlineto{\pgfqpoint{2.604496in}{0.739656in}}%
\pgfpathlineto{\pgfqpoint{2.604199in}{0.739656in}}%
\pgfpathlineto{\pgfqpoint{2.603901in}{0.739656in}}%
\pgfpathlineto{\pgfqpoint{2.603604in}{0.739656in}}%
\pgfpathlineto{\pgfqpoint{2.603306in}{0.739656in}}%
\pgfpathlineto{\pgfqpoint{2.603009in}{0.739656in}}%
\pgfpathlineto{\pgfqpoint{2.602711in}{0.739656in}}%
\pgfpathlineto{\pgfqpoint{2.602414in}{0.739656in}}%
\pgfpathlineto{\pgfqpoint{2.602116in}{0.739656in}}%
\pgfpathlineto{\pgfqpoint{2.601819in}{0.739656in}}%
\pgfpathlineto{\pgfqpoint{2.601521in}{0.739656in}}%
\pgfpathlineto{\pgfqpoint{2.601224in}{0.739656in}}%
\pgfpathlineto{\pgfqpoint{2.600926in}{0.739656in}}%
\pgfpathlineto{\pgfqpoint{2.600629in}{0.739656in}}%
\pgfpathlineto{\pgfqpoint{2.600331in}{0.739656in}}%
\pgfpathlineto{\pgfqpoint{2.600034in}{0.739656in}}%
\pgfpathlineto{\pgfqpoint{2.599736in}{0.739656in}}%
\pgfpathlineto{\pgfqpoint{2.599439in}{0.739656in}}%
\pgfpathlineto{\pgfqpoint{2.599141in}{0.739656in}}%
\pgfpathlineto{\pgfqpoint{2.598844in}{0.739656in}}%
\pgfpathlineto{\pgfqpoint{2.598547in}{0.739656in}}%
\pgfpathlineto{\pgfqpoint{2.598249in}{0.739656in}}%
\pgfpathlineto{\pgfqpoint{2.597952in}{0.739656in}}%
\pgfpathlineto{\pgfqpoint{2.597654in}{0.739656in}}%
\pgfpathlineto{\pgfqpoint{2.597357in}{0.739656in}}%
\pgfpathlineto{\pgfqpoint{2.597059in}{0.739656in}}%
\pgfpathlineto{\pgfqpoint{2.596762in}{0.739656in}}%
\pgfpathlineto{\pgfqpoint{2.596464in}{0.739656in}}%
\pgfpathlineto{\pgfqpoint{2.596167in}{0.739656in}}%
\pgfpathlineto{\pgfqpoint{2.595869in}{0.739656in}}%
\pgfpathlineto{\pgfqpoint{2.595572in}{0.739656in}}%
\pgfpathlineto{\pgfqpoint{2.595274in}{0.739656in}}%
\pgfpathlineto{\pgfqpoint{2.594977in}{0.739656in}}%
\pgfpathlineto{\pgfqpoint{2.594679in}{0.739656in}}%
\pgfpathlineto{\pgfqpoint{2.594382in}{0.739656in}}%
\pgfpathlineto{\pgfqpoint{2.594084in}{0.739656in}}%
\pgfpathlineto{\pgfqpoint{2.593787in}{0.739656in}}%
\pgfpathlineto{\pgfqpoint{2.593489in}{0.739656in}}%
\pgfpathlineto{\pgfqpoint{2.593192in}{0.739656in}}%
\pgfpathlineto{\pgfqpoint{2.592894in}{0.739656in}}%
\pgfpathlineto{\pgfqpoint{2.592597in}{0.739656in}}%
\pgfpathlineto{\pgfqpoint{2.592300in}{0.739656in}}%
\pgfpathlineto{\pgfqpoint{2.592002in}{0.739656in}}%
\pgfpathlineto{\pgfqpoint{2.591705in}{0.739656in}}%
\pgfpathlineto{\pgfqpoint{2.591407in}{0.739656in}}%
\pgfpathlineto{\pgfqpoint{2.591110in}{0.739656in}}%
\pgfpathlineto{\pgfqpoint{2.590812in}{0.739656in}}%
\pgfpathlineto{\pgfqpoint{2.590515in}{0.739656in}}%
\pgfpathlineto{\pgfqpoint{2.590217in}{0.739656in}}%
\pgfpathlineto{\pgfqpoint{2.589920in}{0.739656in}}%
\pgfpathlineto{\pgfqpoint{2.589622in}{0.739656in}}%
\pgfpathlineto{\pgfqpoint{2.589325in}{0.739656in}}%
\pgfpathlineto{\pgfqpoint{2.589027in}{0.739656in}}%
\pgfpathlineto{\pgfqpoint{2.588730in}{0.739656in}}%
\pgfpathlineto{\pgfqpoint{2.588432in}{0.739656in}}%
\pgfpathlineto{\pgfqpoint{2.588135in}{0.739656in}}%
\pgfpathlineto{\pgfqpoint{2.587837in}{0.739656in}}%
\pgfpathlineto{\pgfqpoint{2.587540in}{0.739656in}}%
\pgfpathlineto{\pgfqpoint{2.587242in}{0.739656in}}%
\pgfpathlineto{\pgfqpoint{2.586945in}{0.739656in}}%
\pgfpathlineto{\pgfqpoint{2.586647in}{0.739656in}}%
\pgfpathlineto{\pgfqpoint{2.586350in}{0.739656in}}%
\pgfpathlineto{\pgfqpoint{2.586052in}{0.739656in}}%
\pgfpathlineto{\pgfqpoint{2.585755in}{0.739656in}}%
\pgfpathlineto{\pgfqpoint{2.585458in}{0.739656in}}%
\pgfpathlineto{\pgfqpoint{2.585160in}{0.739656in}}%
\pgfpathlineto{\pgfqpoint{2.584863in}{0.739656in}}%
\pgfpathlineto{\pgfqpoint{2.584565in}{0.739656in}}%
\pgfpathlineto{\pgfqpoint{2.584268in}{0.739656in}}%
\pgfpathlineto{\pgfqpoint{2.583970in}{0.739656in}}%
\pgfpathlineto{\pgfqpoint{2.583673in}{0.739656in}}%
\pgfpathlineto{\pgfqpoint{2.583375in}{0.739656in}}%
\pgfpathlineto{\pgfqpoint{2.583078in}{0.739656in}}%
\pgfpathlineto{\pgfqpoint{2.582780in}{0.739656in}}%
\pgfpathlineto{\pgfqpoint{2.582483in}{0.739656in}}%
\pgfpathlineto{\pgfqpoint{2.582185in}{0.739656in}}%
\pgfpathlineto{\pgfqpoint{2.581888in}{0.739656in}}%
\pgfpathlineto{\pgfqpoint{2.581590in}{0.739656in}}%
\pgfpathlineto{\pgfqpoint{2.581293in}{0.739656in}}%
\pgfpathlineto{\pgfqpoint{2.580995in}{0.739656in}}%
\pgfpathlineto{\pgfqpoint{2.580698in}{0.739656in}}%
\pgfpathlineto{\pgfqpoint{2.580400in}{0.739656in}}%
\pgfpathlineto{\pgfqpoint{2.580103in}{0.739656in}}%
\pgfpathlineto{\pgfqpoint{2.579805in}{0.739656in}}%
\pgfpathlineto{\pgfqpoint{2.579508in}{0.739656in}}%
\pgfpathlineto{\pgfqpoint{2.579210in}{0.739656in}}%
\pgfpathlineto{\pgfqpoint{2.578913in}{0.739656in}}%
\pgfpathlineto{\pgfqpoint{2.578616in}{0.739656in}}%
\pgfpathlineto{\pgfqpoint{2.578318in}{0.739656in}}%
\pgfpathlineto{\pgfqpoint{2.578021in}{0.739656in}}%
\pgfpathlineto{\pgfqpoint{2.577723in}{0.739656in}}%
\pgfpathlineto{\pgfqpoint{2.577426in}{0.739656in}}%
\pgfpathlineto{\pgfqpoint{2.577128in}{0.739656in}}%
\pgfpathlineto{\pgfqpoint{2.576831in}{0.739656in}}%
\pgfpathlineto{\pgfqpoint{2.576533in}{0.739656in}}%
\pgfpathlineto{\pgfqpoint{2.576236in}{0.739656in}}%
\pgfpathlineto{\pgfqpoint{2.575938in}{0.739656in}}%
\pgfpathlineto{\pgfqpoint{2.575641in}{0.739656in}}%
\pgfpathlineto{\pgfqpoint{2.575343in}{0.739656in}}%
\pgfpathlineto{\pgfqpoint{2.575046in}{0.739656in}}%
\pgfpathlineto{\pgfqpoint{2.574748in}{0.739656in}}%
\pgfpathlineto{\pgfqpoint{2.574451in}{0.739656in}}%
\pgfpathlineto{\pgfqpoint{2.574153in}{0.739656in}}%
\pgfpathlineto{\pgfqpoint{2.573856in}{0.739656in}}%
\pgfpathlineto{\pgfqpoint{2.573558in}{0.739656in}}%
\pgfpathlineto{\pgfqpoint{2.573261in}{0.739656in}}%
\pgfpathlineto{\pgfqpoint{2.572963in}{0.739656in}}%
\pgfpathlineto{\pgfqpoint{2.572666in}{0.739656in}}%
\pgfpathlineto{\pgfqpoint{2.572369in}{0.739656in}}%
\pgfpathlineto{\pgfqpoint{2.572071in}{0.739656in}}%
\pgfpathlineto{\pgfqpoint{2.571774in}{0.739656in}}%
\pgfpathlineto{\pgfqpoint{2.571476in}{0.739656in}}%
\pgfpathlineto{\pgfqpoint{2.571179in}{0.739656in}}%
\pgfpathlineto{\pgfqpoint{2.570881in}{0.739656in}}%
\pgfpathlineto{\pgfqpoint{2.570584in}{0.739656in}}%
\pgfpathlineto{\pgfqpoint{2.570286in}{0.739656in}}%
\pgfpathlineto{\pgfqpoint{2.569989in}{0.739656in}}%
\pgfpathlineto{\pgfqpoint{2.569691in}{0.739656in}}%
\pgfpathlineto{\pgfqpoint{2.569394in}{0.739656in}}%
\pgfpathlineto{\pgfqpoint{2.569096in}{0.739656in}}%
\pgfpathlineto{\pgfqpoint{2.568799in}{0.739656in}}%
\pgfpathlineto{\pgfqpoint{2.568501in}{0.739656in}}%
\pgfpathlineto{\pgfqpoint{2.568204in}{0.739656in}}%
\pgfpathlineto{\pgfqpoint{2.567906in}{0.739656in}}%
\pgfpathlineto{\pgfqpoint{2.567609in}{0.739656in}}%
\pgfpathlineto{\pgfqpoint{2.567311in}{0.739656in}}%
\pgfpathlineto{\pgfqpoint{2.567014in}{0.739656in}}%
\pgfpathlineto{\pgfqpoint{2.566716in}{0.739656in}}%
\pgfpathlineto{\pgfqpoint{2.566419in}{0.739656in}}%
\pgfpathlineto{\pgfqpoint{2.566121in}{0.739656in}}%
\pgfpathlineto{\pgfqpoint{2.565824in}{0.739656in}}%
\pgfpathlineto{\pgfqpoint{2.565527in}{0.739656in}}%
\pgfpathlineto{\pgfqpoint{2.565229in}{0.739656in}}%
\pgfpathlineto{\pgfqpoint{2.564932in}{0.739656in}}%
\pgfpathlineto{\pgfqpoint{2.564634in}{0.739656in}}%
\pgfpathlineto{\pgfqpoint{2.564337in}{0.739656in}}%
\pgfpathlineto{\pgfqpoint{2.564039in}{0.739656in}}%
\pgfpathlineto{\pgfqpoint{2.563742in}{0.739656in}}%
\pgfpathlineto{\pgfqpoint{2.563444in}{0.739656in}}%
\pgfpathlineto{\pgfqpoint{2.563147in}{0.739656in}}%
\pgfpathlineto{\pgfqpoint{2.562849in}{0.739656in}}%
\pgfpathlineto{\pgfqpoint{2.562552in}{0.739656in}}%
\pgfpathlineto{\pgfqpoint{2.562254in}{0.739656in}}%
\pgfpathlineto{\pgfqpoint{2.561957in}{0.739656in}}%
\pgfpathlineto{\pgfqpoint{2.561659in}{0.739656in}}%
\pgfpathlineto{\pgfqpoint{2.561362in}{0.739656in}}%
\pgfpathlineto{\pgfqpoint{2.561064in}{0.739656in}}%
\pgfpathlineto{\pgfqpoint{2.560767in}{0.739656in}}%
\pgfpathlineto{\pgfqpoint{2.560469in}{0.739656in}}%
\pgfpathlineto{\pgfqpoint{2.560172in}{0.739656in}}%
\pgfpathlineto{\pgfqpoint{2.559874in}{0.739656in}}%
\pgfpathlineto{\pgfqpoint{2.559577in}{0.739656in}}%
\pgfpathlineto{\pgfqpoint{2.559279in}{0.739656in}}%
\pgfpathlineto{\pgfqpoint{2.558982in}{0.739656in}}%
\pgfpathlineto{\pgfqpoint{2.558685in}{0.739656in}}%
\pgfpathlineto{\pgfqpoint{2.558387in}{0.739656in}}%
\pgfpathlineto{\pgfqpoint{2.558090in}{0.739656in}}%
\pgfpathlineto{\pgfqpoint{2.557792in}{0.739656in}}%
\pgfpathlineto{\pgfqpoint{2.557495in}{0.739656in}}%
\pgfpathlineto{\pgfqpoint{2.557197in}{0.739656in}}%
\pgfpathlineto{\pgfqpoint{2.556900in}{0.739656in}}%
\pgfpathlineto{\pgfqpoint{2.556602in}{0.739656in}}%
\pgfpathlineto{\pgfqpoint{2.556305in}{0.739656in}}%
\pgfpathlineto{\pgfqpoint{2.556007in}{0.739656in}}%
\pgfpathlineto{\pgfqpoint{2.555710in}{0.739656in}}%
\pgfpathlineto{\pgfqpoint{2.555412in}{0.739656in}}%
\pgfpathlineto{\pgfqpoint{2.555115in}{0.739656in}}%
\pgfpathlineto{\pgfqpoint{2.554817in}{0.739656in}}%
\pgfpathlineto{\pgfqpoint{2.554520in}{0.739656in}}%
\pgfpathlineto{\pgfqpoint{2.554222in}{0.739656in}}%
\pgfpathlineto{\pgfqpoint{2.553925in}{0.739656in}}%
\pgfpathlineto{\pgfqpoint{2.553627in}{0.739656in}}%
\pgfpathlineto{\pgfqpoint{2.553330in}{0.739656in}}%
\pgfpathlineto{\pgfqpoint{2.553032in}{0.739656in}}%
\pgfpathlineto{\pgfqpoint{2.552735in}{0.739656in}}%
\pgfpathlineto{\pgfqpoint{2.552438in}{0.739656in}}%
\pgfpathlineto{\pgfqpoint{2.552140in}{0.739656in}}%
\pgfpathlineto{\pgfqpoint{2.551843in}{0.739656in}}%
\pgfpathlineto{\pgfqpoint{2.551545in}{0.739656in}}%
\pgfpathlineto{\pgfqpoint{2.551248in}{0.739656in}}%
\pgfpathlineto{\pgfqpoint{2.550950in}{0.739656in}}%
\pgfpathlineto{\pgfqpoint{2.550653in}{0.739656in}}%
\pgfpathlineto{\pgfqpoint{2.550355in}{0.739656in}}%
\pgfpathlineto{\pgfqpoint{2.550058in}{0.739656in}}%
\pgfpathlineto{\pgfqpoint{2.549760in}{0.739656in}}%
\pgfpathlineto{\pgfqpoint{2.549463in}{0.739656in}}%
\pgfpathlineto{\pgfqpoint{2.549165in}{0.739656in}}%
\pgfpathlineto{\pgfqpoint{2.548868in}{0.739656in}}%
\pgfpathlineto{\pgfqpoint{2.548570in}{0.739656in}}%
\pgfpathlineto{\pgfqpoint{2.548273in}{0.739656in}}%
\pgfpathlineto{\pgfqpoint{2.547975in}{0.739656in}}%
\pgfpathlineto{\pgfqpoint{2.547678in}{0.739656in}}%
\pgfpathlineto{\pgfqpoint{2.547380in}{0.739656in}}%
\pgfpathlineto{\pgfqpoint{2.547083in}{0.739656in}}%
\pgfpathlineto{\pgfqpoint{2.546785in}{0.739656in}}%
\pgfpathlineto{\pgfqpoint{2.546488in}{0.739656in}}%
\pgfpathlineto{\pgfqpoint{2.546190in}{0.739656in}}%
\pgfpathlineto{\pgfqpoint{2.545893in}{0.739656in}}%
\pgfpathlineto{\pgfqpoint{2.545596in}{0.739656in}}%
\pgfpathlineto{\pgfqpoint{2.545298in}{0.739656in}}%
\pgfpathlineto{\pgfqpoint{2.545001in}{0.739656in}}%
\pgfpathlineto{\pgfqpoint{2.544703in}{0.739656in}}%
\pgfpathlineto{\pgfqpoint{2.544406in}{0.739656in}}%
\pgfpathlineto{\pgfqpoint{2.544108in}{0.739656in}}%
\pgfpathlineto{\pgfqpoint{2.543811in}{0.739656in}}%
\pgfpathlineto{\pgfqpoint{2.543513in}{0.739656in}}%
\pgfpathlineto{\pgfqpoint{2.543216in}{0.739656in}}%
\pgfpathlineto{\pgfqpoint{2.542918in}{0.739656in}}%
\pgfpathlineto{\pgfqpoint{2.542621in}{0.739656in}}%
\pgfpathlineto{\pgfqpoint{2.542323in}{0.739656in}}%
\pgfpathlineto{\pgfqpoint{2.542026in}{0.739656in}}%
\pgfpathlineto{\pgfqpoint{2.541728in}{0.739656in}}%
\pgfpathlineto{\pgfqpoint{2.541431in}{0.739656in}}%
\pgfpathlineto{\pgfqpoint{2.541133in}{0.739656in}}%
\pgfpathlineto{\pgfqpoint{2.540836in}{0.739656in}}%
\pgfpathlineto{\pgfqpoint{2.540538in}{0.739656in}}%
\pgfpathlineto{\pgfqpoint{2.540241in}{0.739656in}}%
\pgfpathlineto{\pgfqpoint{2.539943in}{0.739656in}}%
\pgfpathlineto{\pgfqpoint{2.539646in}{0.739656in}}%
\pgfpathlineto{\pgfqpoint{2.539348in}{0.739656in}}%
\pgfpathlineto{\pgfqpoint{2.539051in}{0.739656in}}%
\pgfpathlineto{\pgfqpoint{2.538754in}{0.739656in}}%
\pgfpathlineto{\pgfqpoint{2.538456in}{0.739656in}}%
\pgfpathlineto{\pgfqpoint{2.538159in}{0.739656in}}%
\pgfpathlineto{\pgfqpoint{2.537861in}{0.739656in}}%
\pgfpathlineto{\pgfqpoint{2.537564in}{0.739656in}}%
\pgfpathlineto{\pgfqpoint{2.537266in}{0.739656in}}%
\pgfpathlineto{\pgfqpoint{2.536969in}{0.739656in}}%
\pgfpathlineto{\pgfqpoint{2.536671in}{0.739656in}}%
\pgfpathlineto{\pgfqpoint{2.536374in}{0.739656in}}%
\pgfpathlineto{\pgfqpoint{2.536076in}{0.739656in}}%
\pgfpathlineto{\pgfqpoint{2.535779in}{0.739656in}}%
\pgfpathlineto{\pgfqpoint{2.535481in}{0.739656in}}%
\pgfpathlineto{\pgfqpoint{2.535184in}{0.739656in}}%
\pgfpathlineto{\pgfqpoint{2.534886in}{0.739656in}}%
\pgfpathlineto{\pgfqpoint{2.534589in}{0.739656in}}%
\pgfpathlineto{\pgfqpoint{2.534291in}{0.739656in}}%
\pgfpathlineto{\pgfqpoint{2.533994in}{0.739656in}}%
\pgfpathlineto{\pgfqpoint{2.533696in}{0.739656in}}%
\pgfpathlineto{\pgfqpoint{2.533399in}{0.739656in}}%
\pgfpathlineto{\pgfqpoint{2.533101in}{0.739656in}}%
\pgfpathlineto{\pgfqpoint{2.532804in}{0.739656in}}%
\pgfpathlineto{\pgfqpoint{2.532507in}{0.739656in}}%
\pgfpathlineto{\pgfqpoint{2.532209in}{0.739656in}}%
\pgfpathlineto{\pgfqpoint{2.531912in}{0.739656in}}%
\pgfpathlineto{\pgfqpoint{2.531614in}{0.739656in}}%
\pgfpathlineto{\pgfqpoint{2.531317in}{0.739656in}}%
\pgfpathlineto{\pgfqpoint{2.531019in}{0.739656in}}%
\pgfpathlineto{\pgfqpoint{2.530722in}{0.739656in}}%
\pgfpathlineto{\pgfqpoint{2.530424in}{0.739656in}}%
\pgfpathlineto{\pgfqpoint{2.530127in}{0.739656in}}%
\pgfpathlineto{\pgfqpoint{2.529829in}{0.739656in}}%
\pgfpathlineto{\pgfqpoint{2.529532in}{0.739656in}}%
\pgfpathlineto{\pgfqpoint{2.529234in}{0.739656in}}%
\pgfpathlineto{\pgfqpoint{2.528937in}{0.739656in}}%
\pgfpathlineto{\pgfqpoint{2.528639in}{0.739656in}}%
\pgfpathlineto{\pgfqpoint{2.528342in}{0.739656in}}%
\pgfpathlineto{\pgfqpoint{2.528044in}{0.739656in}}%
\pgfpathlineto{\pgfqpoint{2.527747in}{0.739656in}}%
\pgfpathlineto{\pgfqpoint{2.527449in}{0.739656in}}%
\pgfpathlineto{\pgfqpoint{2.527152in}{0.739656in}}%
\pgfpathlineto{\pgfqpoint{2.526854in}{0.739656in}}%
\pgfpathlineto{\pgfqpoint{2.526557in}{0.739656in}}%
\pgfpathlineto{\pgfqpoint{2.526259in}{0.739656in}}%
\pgfpathlineto{\pgfqpoint{2.525962in}{0.739656in}}%
\pgfpathlineto{\pgfqpoint{2.525665in}{0.739656in}}%
\pgfpathlineto{\pgfqpoint{2.525367in}{0.739656in}}%
\pgfpathlineto{\pgfqpoint{2.525070in}{0.739656in}}%
\pgfpathlineto{\pgfqpoint{2.524772in}{0.739656in}}%
\pgfpathlineto{\pgfqpoint{2.524475in}{0.739656in}}%
\pgfpathlineto{\pgfqpoint{2.524177in}{0.739656in}}%
\pgfpathlineto{\pgfqpoint{2.523880in}{0.739656in}}%
\pgfpathlineto{\pgfqpoint{2.523582in}{0.739656in}}%
\pgfpathlineto{\pgfqpoint{2.523285in}{0.739656in}}%
\pgfpathlineto{\pgfqpoint{2.522987in}{0.739656in}}%
\pgfpathlineto{\pgfqpoint{2.522690in}{0.739656in}}%
\pgfpathlineto{\pgfqpoint{2.522392in}{0.739656in}}%
\pgfpathlineto{\pgfqpoint{2.522095in}{0.739656in}}%
\pgfpathlineto{\pgfqpoint{2.521797in}{0.739656in}}%
\pgfpathlineto{\pgfqpoint{2.521500in}{0.739656in}}%
\pgfpathlineto{\pgfqpoint{2.521202in}{0.739656in}}%
\pgfpathlineto{\pgfqpoint{2.520905in}{0.739656in}}%
\pgfpathlineto{\pgfqpoint{2.520607in}{0.739656in}}%
\pgfpathlineto{\pgfqpoint{2.520310in}{0.739656in}}%
\pgfpathlineto{\pgfqpoint{2.520012in}{0.739656in}}%
\pgfpathlineto{\pgfqpoint{2.519715in}{0.739656in}}%
\pgfpathlineto{\pgfqpoint{2.519417in}{0.739656in}}%
\pgfpathlineto{\pgfqpoint{2.519120in}{0.739656in}}%
\pgfpathlineto{\pgfqpoint{2.518823in}{0.739656in}}%
\pgfpathlineto{\pgfqpoint{2.518525in}{0.739656in}}%
\pgfpathlineto{\pgfqpoint{2.518228in}{0.739656in}}%
\pgfpathlineto{\pgfqpoint{2.517930in}{0.739656in}}%
\pgfpathlineto{\pgfqpoint{2.517633in}{0.739656in}}%
\pgfpathlineto{\pgfqpoint{2.517335in}{0.739656in}}%
\pgfpathlineto{\pgfqpoint{2.517038in}{0.739656in}}%
\pgfpathlineto{\pgfqpoint{2.516740in}{0.739656in}}%
\pgfpathlineto{\pgfqpoint{2.516443in}{0.739656in}}%
\pgfpathlineto{\pgfqpoint{2.516145in}{0.739656in}}%
\pgfpathlineto{\pgfqpoint{2.515848in}{0.739656in}}%
\pgfpathlineto{\pgfqpoint{2.515550in}{0.739656in}}%
\pgfpathlineto{\pgfqpoint{2.515253in}{0.739656in}}%
\pgfpathlineto{\pgfqpoint{2.514955in}{0.739656in}}%
\pgfpathlineto{\pgfqpoint{2.514658in}{0.739656in}}%
\pgfpathlineto{\pgfqpoint{2.514360in}{0.739656in}}%
\pgfpathlineto{\pgfqpoint{2.514063in}{0.739656in}}%
\pgfpathlineto{\pgfqpoint{2.513765in}{0.739656in}}%
\pgfpathlineto{\pgfqpoint{2.513468in}{0.739656in}}%
\pgfpathlineto{\pgfqpoint{2.513170in}{0.739656in}}%
\pgfpathlineto{\pgfqpoint{2.512873in}{0.739656in}}%
\pgfpathlineto{\pgfqpoint{2.512576in}{0.739656in}}%
\pgfpathlineto{\pgfqpoint{2.512278in}{0.739656in}}%
\pgfpathlineto{\pgfqpoint{2.511981in}{0.739656in}}%
\pgfpathlineto{\pgfqpoint{2.511683in}{0.739656in}}%
\pgfpathlineto{\pgfqpoint{2.511386in}{0.739656in}}%
\pgfpathlineto{\pgfqpoint{2.511088in}{0.739656in}}%
\pgfpathlineto{\pgfqpoint{2.510791in}{0.739656in}}%
\pgfpathlineto{\pgfqpoint{2.510493in}{0.739656in}}%
\pgfpathlineto{\pgfqpoint{2.510196in}{0.739656in}}%
\pgfpathlineto{\pgfqpoint{2.509898in}{0.739656in}}%
\pgfpathlineto{\pgfqpoint{2.509601in}{0.739656in}}%
\pgfpathlineto{\pgfqpoint{2.509303in}{0.739656in}}%
\pgfpathlineto{\pgfqpoint{2.509006in}{0.739656in}}%
\pgfpathlineto{\pgfqpoint{2.508708in}{0.739656in}}%
\pgfpathlineto{\pgfqpoint{2.508411in}{0.739656in}}%
\pgfpathlineto{\pgfqpoint{2.508113in}{0.739656in}}%
\pgfpathlineto{\pgfqpoint{2.507816in}{0.739656in}}%
\pgfpathlineto{\pgfqpoint{2.507518in}{0.739656in}}%
\pgfpathlineto{\pgfqpoint{2.507221in}{0.739656in}}%
\pgfpathlineto{\pgfqpoint{2.506923in}{0.739656in}}%
\pgfpathlineto{\pgfqpoint{2.506626in}{0.739656in}}%
\pgfpathlineto{\pgfqpoint{2.506328in}{0.739656in}}%
\pgfpathlineto{\pgfqpoint{2.506031in}{0.739656in}}%
\pgfpathlineto{\pgfqpoint{2.505734in}{0.739656in}}%
\pgfpathlineto{\pgfqpoint{2.505436in}{0.739656in}}%
\pgfpathlineto{\pgfqpoint{2.505139in}{0.739656in}}%
\pgfpathlineto{\pgfqpoint{2.504841in}{0.739656in}}%
\pgfpathlineto{\pgfqpoint{2.504544in}{0.739656in}}%
\pgfpathlineto{\pgfqpoint{2.504246in}{0.739656in}}%
\pgfpathlineto{\pgfqpoint{2.503949in}{0.739656in}}%
\pgfpathlineto{\pgfqpoint{2.503651in}{0.739656in}}%
\pgfpathlineto{\pgfqpoint{2.503354in}{0.739656in}}%
\pgfpathlineto{\pgfqpoint{2.503056in}{0.739656in}}%
\pgfpathlineto{\pgfqpoint{2.502759in}{0.739656in}}%
\pgfpathlineto{\pgfqpoint{2.502461in}{0.739656in}}%
\pgfpathlineto{\pgfqpoint{2.502164in}{0.739656in}}%
\pgfpathlineto{\pgfqpoint{2.501866in}{0.739656in}}%
\pgfpathlineto{\pgfqpoint{2.501569in}{0.739656in}}%
\pgfpathlineto{\pgfqpoint{2.501271in}{0.739656in}}%
\pgfpathlineto{\pgfqpoint{2.500974in}{0.739656in}}%
\pgfpathlineto{\pgfqpoint{2.500676in}{0.739656in}}%
\pgfpathlineto{\pgfqpoint{2.500379in}{0.739656in}}%
\pgfpathlineto{\pgfqpoint{2.500081in}{0.739656in}}%
\pgfpathlineto{\pgfqpoint{2.499784in}{0.739656in}}%
\pgfpathlineto{\pgfqpoint{2.499486in}{0.739656in}}%
\pgfpathlineto{\pgfqpoint{2.499189in}{0.739656in}}%
\pgfpathlineto{\pgfqpoint{2.498892in}{0.739656in}}%
\pgfpathlineto{\pgfqpoint{2.498594in}{0.739656in}}%
\pgfpathlineto{\pgfqpoint{2.498297in}{0.739656in}}%
\pgfpathlineto{\pgfqpoint{2.497999in}{0.739656in}}%
\pgfpathlineto{\pgfqpoint{2.497702in}{0.739656in}}%
\pgfpathlineto{\pgfqpoint{2.497404in}{0.739656in}}%
\pgfpathlineto{\pgfqpoint{2.497107in}{0.739656in}}%
\pgfpathlineto{\pgfqpoint{2.496809in}{0.739656in}}%
\pgfpathlineto{\pgfqpoint{2.496512in}{0.739656in}}%
\pgfpathlineto{\pgfqpoint{2.496214in}{0.739656in}}%
\pgfpathlineto{\pgfqpoint{2.495917in}{0.739656in}}%
\pgfpathlineto{\pgfqpoint{2.495619in}{0.739656in}}%
\pgfpathlineto{\pgfqpoint{2.495322in}{0.739656in}}%
\pgfpathlineto{\pgfqpoint{2.495024in}{0.739656in}}%
\pgfpathlineto{\pgfqpoint{2.494727in}{0.739656in}}%
\pgfpathlineto{\pgfqpoint{2.494429in}{0.739656in}}%
\pgfpathlineto{\pgfqpoint{2.494132in}{0.739656in}}%
\pgfpathlineto{\pgfqpoint{2.493834in}{0.739656in}}%
\pgfpathlineto{\pgfqpoint{2.493537in}{0.739656in}}%
\pgfpathlineto{\pgfqpoint{2.493239in}{0.739656in}}%
\pgfpathlineto{\pgfqpoint{2.492942in}{0.739656in}}%
\pgfpathlineto{\pgfqpoint{2.492645in}{0.739656in}}%
\pgfpathlineto{\pgfqpoint{2.492347in}{0.739656in}}%
\pgfpathlineto{\pgfqpoint{2.492050in}{0.739656in}}%
\pgfpathlineto{\pgfqpoint{2.491752in}{0.739656in}}%
\pgfpathlineto{\pgfqpoint{2.491455in}{0.739656in}}%
\pgfpathlineto{\pgfqpoint{2.491157in}{0.739656in}}%
\pgfpathlineto{\pgfqpoint{2.490860in}{0.739656in}}%
\pgfpathlineto{\pgfqpoint{2.490562in}{0.739656in}}%
\pgfpathlineto{\pgfqpoint{2.490265in}{0.739656in}}%
\pgfpathlineto{\pgfqpoint{2.489967in}{0.739656in}}%
\pgfpathlineto{\pgfqpoint{2.489670in}{0.739656in}}%
\pgfpathlineto{\pgfqpoint{2.489372in}{0.739656in}}%
\pgfpathlineto{\pgfqpoint{2.489075in}{0.739656in}}%
\pgfpathlineto{\pgfqpoint{2.488777in}{0.739656in}}%
\pgfpathlineto{\pgfqpoint{2.488480in}{0.739656in}}%
\pgfpathlineto{\pgfqpoint{2.488182in}{0.739656in}}%
\pgfpathlineto{\pgfqpoint{2.487885in}{0.739656in}}%
\pgfpathlineto{\pgfqpoint{2.487587in}{0.739656in}}%
\pgfpathlineto{\pgfqpoint{2.487290in}{0.739656in}}%
\pgfpathlineto{\pgfqpoint{2.486992in}{0.739656in}}%
\pgfpathlineto{\pgfqpoint{2.486695in}{0.739656in}}%
\pgfpathlineto{\pgfqpoint{2.486397in}{0.739656in}}%
\pgfpathlineto{\pgfqpoint{2.486100in}{0.739656in}}%
\pgfpathlineto{\pgfqpoint{2.485803in}{0.739656in}}%
\pgfpathlineto{\pgfqpoint{2.485505in}{0.739656in}}%
\pgfpathlineto{\pgfqpoint{2.485208in}{0.739656in}}%
\pgfpathlineto{\pgfqpoint{2.484910in}{0.739656in}}%
\pgfpathlineto{\pgfqpoint{2.484613in}{0.739656in}}%
\pgfpathlineto{\pgfqpoint{2.484315in}{0.739656in}}%
\pgfpathlineto{\pgfqpoint{2.484018in}{0.739656in}}%
\pgfpathlineto{\pgfqpoint{2.483720in}{0.739656in}}%
\pgfpathlineto{\pgfqpoint{2.483423in}{0.739656in}}%
\pgfpathlineto{\pgfqpoint{2.483125in}{0.739656in}}%
\pgfpathlineto{\pgfqpoint{2.482828in}{0.739656in}}%
\pgfpathlineto{\pgfqpoint{2.482530in}{0.739656in}}%
\pgfpathlineto{\pgfqpoint{2.482233in}{0.739656in}}%
\pgfpathlineto{\pgfqpoint{2.481935in}{0.739656in}}%
\pgfpathlineto{\pgfqpoint{2.481638in}{0.739656in}}%
\pgfpathlineto{\pgfqpoint{2.481340in}{0.739656in}}%
\pgfpathlineto{\pgfqpoint{2.481043in}{0.739656in}}%
\pgfpathlineto{\pgfqpoint{2.480745in}{0.739656in}}%
\pgfpathlineto{\pgfqpoint{2.480448in}{0.739656in}}%
\pgfpathlineto{\pgfqpoint{2.480150in}{0.739656in}}%
\pgfpathlineto{\pgfqpoint{2.479853in}{0.739656in}}%
\pgfpathlineto{\pgfqpoint{2.479555in}{0.739656in}}%
\pgfpathlineto{\pgfqpoint{2.479258in}{0.739656in}}%
\pgfpathlineto{\pgfqpoint{2.478961in}{0.739656in}}%
\pgfpathlineto{\pgfqpoint{2.478663in}{0.739656in}}%
\pgfpathlineto{\pgfqpoint{2.478366in}{0.739656in}}%
\pgfpathlineto{\pgfqpoint{2.478068in}{0.739656in}}%
\pgfpathlineto{\pgfqpoint{2.477771in}{0.739656in}}%
\pgfpathlineto{\pgfqpoint{2.477473in}{0.739656in}}%
\pgfpathlineto{\pgfqpoint{2.477176in}{0.739656in}}%
\pgfpathlineto{\pgfqpoint{2.476878in}{0.739656in}}%
\pgfpathlineto{\pgfqpoint{2.476581in}{0.739656in}}%
\pgfpathlineto{\pgfqpoint{2.476283in}{0.739656in}}%
\pgfpathlineto{\pgfqpoint{2.475986in}{0.739656in}}%
\pgfpathlineto{\pgfqpoint{2.475688in}{0.739656in}}%
\pgfpathlineto{\pgfqpoint{2.475391in}{0.739656in}}%
\pgfpathlineto{\pgfqpoint{2.475093in}{0.739656in}}%
\pgfpathlineto{\pgfqpoint{2.474796in}{0.739656in}}%
\pgfpathlineto{\pgfqpoint{2.474498in}{0.739656in}}%
\pgfpathlineto{\pgfqpoint{2.474201in}{0.739656in}}%
\pgfpathlineto{\pgfqpoint{2.473903in}{0.739656in}}%
\pgfpathlineto{\pgfqpoint{2.473606in}{0.739656in}}%
\pgfpathlineto{\pgfqpoint{2.473308in}{0.739656in}}%
\pgfpathlineto{\pgfqpoint{2.473011in}{0.739656in}}%
\pgfpathlineto{\pgfqpoint{2.472714in}{0.739656in}}%
\pgfpathlineto{\pgfqpoint{2.472416in}{0.739656in}}%
\pgfpathlineto{\pgfqpoint{2.472119in}{0.739656in}}%
\pgfpathlineto{\pgfqpoint{2.471821in}{0.739656in}}%
\pgfpathlineto{\pgfqpoint{2.471524in}{0.739656in}}%
\pgfpathlineto{\pgfqpoint{2.471226in}{0.739656in}}%
\pgfpathlineto{\pgfqpoint{2.470929in}{0.739656in}}%
\pgfpathlineto{\pgfqpoint{2.470631in}{0.739656in}}%
\pgfpathlineto{\pgfqpoint{2.470334in}{0.739656in}}%
\pgfpathlineto{\pgfqpoint{2.470036in}{0.739656in}}%
\pgfpathlineto{\pgfqpoint{2.469739in}{0.739656in}}%
\pgfpathlineto{\pgfqpoint{2.469441in}{0.739656in}}%
\pgfpathlineto{\pgfqpoint{2.469144in}{0.739656in}}%
\pgfpathlineto{\pgfqpoint{2.468846in}{0.739656in}}%
\pgfpathlineto{\pgfqpoint{2.468549in}{0.739656in}}%
\pgfpathlineto{\pgfqpoint{2.468251in}{0.739656in}}%
\pgfpathlineto{\pgfqpoint{2.467954in}{0.739656in}}%
\pgfpathlineto{\pgfqpoint{2.467656in}{0.739656in}}%
\pgfpathlineto{\pgfqpoint{2.467359in}{0.739656in}}%
\pgfpathlineto{\pgfqpoint{2.467061in}{0.739656in}}%
\pgfpathlineto{\pgfqpoint{2.466764in}{0.739656in}}%
\pgfpathlineto{\pgfqpoint{2.466466in}{0.739656in}}%
\pgfpathlineto{\pgfqpoint{2.466169in}{0.739656in}}%
\pgfpathlineto{\pgfqpoint{2.465872in}{0.739656in}}%
\pgfpathlineto{\pgfqpoint{2.465574in}{0.739656in}}%
\pgfpathlineto{\pgfqpoint{2.465277in}{0.739656in}}%
\pgfpathlineto{\pgfqpoint{2.464979in}{0.739656in}}%
\pgfpathlineto{\pgfqpoint{2.464682in}{0.739656in}}%
\pgfpathlineto{\pgfqpoint{2.464384in}{0.739656in}}%
\pgfpathlineto{\pgfqpoint{2.464087in}{0.739656in}}%
\pgfpathlineto{\pgfqpoint{2.463789in}{0.739656in}}%
\pgfpathlineto{\pgfqpoint{2.463492in}{0.739656in}}%
\pgfpathlineto{\pgfqpoint{2.463194in}{0.739656in}}%
\pgfpathlineto{\pgfqpoint{2.462897in}{0.739656in}}%
\pgfpathlineto{\pgfqpoint{2.462599in}{0.739656in}}%
\pgfpathlineto{\pgfqpoint{2.462302in}{0.739656in}}%
\pgfpathlineto{\pgfqpoint{2.462004in}{0.739656in}}%
\pgfpathlineto{\pgfqpoint{2.461707in}{0.739656in}}%
\pgfpathlineto{\pgfqpoint{2.461409in}{0.739656in}}%
\pgfpathlineto{\pgfqpoint{2.461112in}{0.739656in}}%
\pgfpathlineto{\pgfqpoint{2.460814in}{0.739656in}}%
\pgfpathlineto{\pgfqpoint{2.460517in}{0.739656in}}%
\pgfpathlineto{\pgfqpoint{2.460219in}{0.739656in}}%
\pgfpathlineto{\pgfqpoint{2.459922in}{0.739656in}}%
\pgfpathlineto{\pgfqpoint{2.459624in}{0.739656in}}%
\pgfpathlineto{\pgfqpoint{2.459327in}{0.739656in}}%
\pgfpathlineto{\pgfqpoint{2.459030in}{0.739656in}}%
\pgfpathlineto{\pgfqpoint{2.458732in}{0.739656in}}%
\pgfpathlineto{\pgfqpoint{2.458435in}{0.739656in}}%
\pgfpathlineto{\pgfqpoint{2.458137in}{0.739656in}}%
\pgfpathlineto{\pgfqpoint{2.457840in}{0.739656in}}%
\pgfpathlineto{\pgfqpoint{2.457542in}{0.739656in}}%
\pgfpathlineto{\pgfqpoint{2.457245in}{0.739656in}}%
\pgfpathlineto{\pgfqpoint{2.456947in}{0.739656in}}%
\pgfpathlineto{\pgfqpoint{2.456650in}{0.739656in}}%
\pgfpathlineto{\pgfqpoint{2.456352in}{0.739656in}}%
\pgfpathlineto{\pgfqpoint{2.456055in}{0.739656in}}%
\pgfpathlineto{\pgfqpoint{2.455757in}{0.739656in}}%
\pgfpathlineto{\pgfqpoint{2.455460in}{0.739656in}}%
\pgfpathlineto{\pgfqpoint{2.455162in}{0.739656in}}%
\pgfpathlineto{\pgfqpoint{2.454865in}{0.739656in}}%
\pgfpathlineto{\pgfqpoint{2.454567in}{0.739656in}}%
\pgfpathlineto{\pgfqpoint{2.454270in}{0.739656in}}%
\pgfpathlineto{\pgfqpoint{2.453972in}{0.739656in}}%
\pgfpathlineto{\pgfqpoint{2.453675in}{0.739656in}}%
\pgfpathlineto{\pgfqpoint{2.453377in}{0.739656in}}%
\pgfpathlineto{\pgfqpoint{2.453080in}{0.739656in}}%
\pgfpathlineto{\pgfqpoint{2.452783in}{0.739656in}}%
\pgfpathlineto{\pgfqpoint{2.452485in}{0.739656in}}%
\pgfpathlineto{\pgfqpoint{2.452188in}{0.739656in}}%
\pgfpathlineto{\pgfqpoint{2.451890in}{0.739656in}}%
\pgfpathlineto{\pgfqpoint{2.451593in}{0.739656in}}%
\pgfpathlineto{\pgfqpoint{2.451295in}{0.739656in}}%
\pgfpathlineto{\pgfqpoint{2.450998in}{0.739656in}}%
\pgfpathlineto{\pgfqpoint{2.450700in}{0.739656in}}%
\pgfpathlineto{\pgfqpoint{2.450403in}{0.739656in}}%
\pgfpathlineto{\pgfqpoint{2.450105in}{0.739656in}}%
\pgfpathlineto{\pgfqpoint{2.449808in}{0.739656in}}%
\pgfpathlineto{\pgfqpoint{2.449510in}{0.739656in}}%
\pgfpathlineto{\pgfqpoint{2.449213in}{0.739656in}}%
\pgfpathlineto{\pgfqpoint{2.448915in}{0.739656in}}%
\pgfpathlineto{\pgfqpoint{2.448618in}{0.739656in}}%
\pgfpathlineto{\pgfqpoint{2.448320in}{0.739656in}}%
\pgfpathlineto{\pgfqpoint{2.448023in}{0.739656in}}%
\pgfpathlineto{\pgfqpoint{2.447725in}{0.739656in}}%
\pgfpathlineto{\pgfqpoint{2.447428in}{0.739656in}}%
\pgfpathlineto{\pgfqpoint{2.447130in}{0.739656in}}%
\pgfpathlineto{\pgfqpoint{2.446833in}{0.739656in}}%
\pgfpathlineto{\pgfqpoint{2.446535in}{0.739656in}}%
\pgfpathlineto{\pgfqpoint{2.446238in}{0.739656in}}%
\pgfpathlineto{\pgfqpoint{2.445941in}{0.739656in}}%
\pgfpathlineto{\pgfqpoint{2.445643in}{0.739656in}}%
\pgfpathlineto{\pgfqpoint{2.445346in}{0.739656in}}%
\pgfpathlineto{\pgfqpoint{2.445048in}{0.739656in}}%
\pgfpathlineto{\pgfqpoint{2.444751in}{0.739656in}}%
\pgfpathlineto{\pgfqpoint{2.444453in}{0.739656in}}%
\pgfpathlineto{\pgfqpoint{2.444156in}{0.739656in}}%
\pgfpathlineto{\pgfqpoint{2.443858in}{0.739656in}}%
\pgfpathlineto{\pgfqpoint{2.443561in}{0.739656in}}%
\pgfpathlineto{\pgfqpoint{2.443263in}{0.739656in}}%
\pgfpathlineto{\pgfqpoint{2.442966in}{0.739656in}}%
\pgfpathlineto{\pgfqpoint{2.442668in}{0.739656in}}%
\pgfpathlineto{\pgfqpoint{2.442371in}{0.739656in}}%
\pgfpathlineto{\pgfqpoint{2.442073in}{0.739656in}}%
\pgfpathlineto{\pgfqpoint{2.441776in}{0.739656in}}%
\pgfpathlineto{\pgfqpoint{2.441478in}{0.739656in}}%
\pgfpathlineto{\pgfqpoint{2.441181in}{0.739656in}}%
\pgfpathlineto{\pgfqpoint{2.440883in}{0.739656in}}%
\pgfpathlineto{\pgfqpoint{2.440586in}{0.739656in}}%
\pgfpathlineto{\pgfqpoint{2.440288in}{0.739656in}}%
\pgfpathlineto{\pgfqpoint{2.439991in}{0.739656in}}%
\pgfpathlineto{\pgfqpoint{2.439693in}{0.739656in}}%
\pgfpathlineto{\pgfqpoint{2.439396in}{0.739656in}}%
\pgfpathlineto{\pgfqpoint{2.439099in}{0.739656in}}%
\pgfpathlineto{\pgfqpoint{2.438801in}{0.739656in}}%
\pgfpathlineto{\pgfqpoint{2.438504in}{0.739656in}}%
\pgfpathlineto{\pgfqpoint{2.438206in}{0.739656in}}%
\pgfpathlineto{\pgfqpoint{2.437909in}{0.739656in}}%
\pgfpathlineto{\pgfqpoint{2.437611in}{0.739656in}}%
\pgfpathlineto{\pgfqpoint{2.437314in}{0.739656in}}%
\pgfpathlineto{\pgfqpoint{2.437016in}{0.739656in}}%
\pgfpathlineto{\pgfqpoint{2.436719in}{0.739656in}}%
\pgfpathlineto{\pgfqpoint{2.436421in}{0.739656in}}%
\pgfpathlineto{\pgfqpoint{2.436124in}{0.739656in}}%
\pgfpathlineto{\pgfqpoint{2.435826in}{0.739656in}}%
\pgfpathlineto{\pgfqpoint{2.435529in}{0.739656in}}%
\pgfpathlineto{\pgfqpoint{2.435231in}{0.739656in}}%
\pgfpathlineto{\pgfqpoint{2.434934in}{0.739656in}}%
\pgfpathlineto{\pgfqpoint{2.434636in}{0.739656in}}%
\pgfpathlineto{\pgfqpoint{2.434339in}{0.739656in}}%
\pgfpathlineto{\pgfqpoint{2.434041in}{0.739656in}}%
\pgfpathlineto{\pgfqpoint{2.433744in}{0.739656in}}%
\pgfpathlineto{\pgfqpoint{2.433446in}{0.739656in}}%
\pgfpathlineto{\pgfqpoint{2.433149in}{0.739656in}}%
\pgfpathlineto{\pgfqpoint{2.432851in}{0.739656in}}%
\pgfpathlineto{\pgfqpoint{2.432554in}{0.739656in}}%
\pgfpathlineto{\pgfqpoint{2.432257in}{0.739656in}}%
\pgfpathlineto{\pgfqpoint{2.431959in}{0.739656in}}%
\pgfpathlineto{\pgfqpoint{2.431662in}{0.739656in}}%
\pgfpathlineto{\pgfqpoint{2.431364in}{0.739656in}}%
\pgfpathlineto{\pgfqpoint{2.431067in}{0.739656in}}%
\pgfpathlineto{\pgfqpoint{2.430769in}{0.739656in}}%
\pgfpathlineto{\pgfqpoint{2.430472in}{0.739656in}}%
\pgfpathlineto{\pgfqpoint{2.430174in}{0.739656in}}%
\pgfpathlineto{\pgfqpoint{2.429877in}{0.739656in}}%
\pgfpathlineto{\pgfqpoint{2.429579in}{0.739656in}}%
\pgfpathlineto{\pgfqpoint{2.429282in}{0.739656in}}%
\pgfpathlineto{\pgfqpoint{2.428984in}{0.739656in}}%
\pgfpathlineto{\pgfqpoint{2.428687in}{0.739656in}}%
\pgfpathlineto{\pgfqpoint{2.428389in}{0.739656in}}%
\pgfpathlineto{\pgfqpoint{2.428092in}{0.739656in}}%
\pgfpathlineto{\pgfqpoint{2.427794in}{0.739656in}}%
\pgfpathlineto{\pgfqpoint{2.427497in}{0.739656in}}%
\pgfpathlineto{\pgfqpoint{2.427199in}{0.739656in}}%
\pgfpathlineto{\pgfqpoint{2.426902in}{0.739656in}}%
\pgfpathlineto{\pgfqpoint{2.426604in}{0.739656in}}%
\pgfpathlineto{\pgfqpoint{2.426307in}{0.739656in}}%
\pgfpathlineto{\pgfqpoint{2.426010in}{0.739656in}}%
\pgfpathlineto{\pgfqpoint{2.425712in}{0.739656in}}%
\pgfpathlineto{\pgfqpoint{2.425415in}{0.739656in}}%
\pgfpathlineto{\pgfqpoint{2.425117in}{0.739656in}}%
\pgfpathlineto{\pgfqpoint{2.424820in}{0.739656in}}%
\pgfpathlineto{\pgfqpoint{2.424522in}{0.739656in}}%
\pgfpathlineto{\pgfqpoint{2.424225in}{0.739656in}}%
\pgfpathlineto{\pgfqpoint{2.423927in}{0.739656in}}%
\pgfpathlineto{\pgfqpoint{2.423630in}{0.739656in}}%
\pgfpathlineto{\pgfqpoint{2.423332in}{0.739656in}}%
\pgfpathlineto{\pgfqpoint{2.423035in}{0.739656in}}%
\pgfpathlineto{\pgfqpoint{2.422737in}{0.739656in}}%
\pgfpathlineto{\pgfqpoint{2.422440in}{0.739656in}}%
\pgfpathlineto{\pgfqpoint{2.422142in}{0.739656in}}%
\pgfpathlineto{\pgfqpoint{2.421845in}{0.739656in}}%
\pgfpathlineto{\pgfqpoint{2.421547in}{0.739656in}}%
\pgfpathlineto{\pgfqpoint{2.421250in}{0.739656in}}%
\pgfpathlineto{\pgfqpoint{2.420952in}{0.739656in}}%
\pgfpathlineto{\pgfqpoint{2.420655in}{0.739656in}}%
\pgfpathlineto{\pgfqpoint{2.420357in}{0.739656in}}%
\pgfpathlineto{\pgfqpoint{2.420060in}{0.739656in}}%
\pgfpathlineto{\pgfqpoint{2.419762in}{0.739656in}}%
\pgfpathlineto{\pgfqpoint{2.419465in}{0.739656in}}%
\pgfpathlineto{\pgfqpoint{2.419168in}{0.739656in}}%
\pgfpathlineto{\pgfqpoint{2.418870in}{0.739656in}}%
\pgfpathlineto{\pgfqpoint{2.418573in}{0.739656in}}%
\pgfpathlineto{\pgfqpoint{2.418275in}{0.739656in}}%
\pgfpathlineto{\pgfqpoint{2.417978in}{0.739656in}}%
\pgfpathlineto{\pgfqpoint{2.417680in}{0.739656in}}%
\pgfpathlineto{\pgfqpoint{2.417383in}{0.739656in}}%
\pgfpathlineto{\pgfqpoint{2.417085in}{0.739656in}}%
\pgfpathlineto{\pgfqpoint{2.416788in}{0.739656in}}%
\pgfpathlineto{\pgfqpoint{2.416490in}{0.739656in}}%
\pgfpathlineto{\pgfqpoint{2.416193in}{0.739656in}}%
\pgfpathlineto{\pgfqpoint{2.415895in}{0.739656in}}%
\pgfpathlineto{\pgfqpoint{2.415598in}{0.739656in}}%
\pgfpathlineto{\pgfqpoint{2.415300in}{0.739656in}}%
\pgfpathlineto{\pgfqpoint{2.415003in}{0.739656in}}%
\pgfpathlineto{\pgfqpoint{2.414705in}{0.739656in}}%
\pgfpathlineto{\pgfqpoint{2.414408in}{0.739656in}}%
\pgfpathlineto{\pgfqpoint{2.414110in}{0.739656in}}%
\pgfpathlineto{\pgfqpoint{2.413813in}{0.739656in}}%
\pgfpathlineto{\pgfqpoint{2.413515in}{0.739656in}}%
\pgfpathlineto{\pgfqpoint{2.413218in}{0.739656in}}%
\pgfpathlineto{\pgfqpoint{2.412920in}{0.739656in}}%
\pgfpathlineto{\pgfqpoint{2.412623in}{0.739656in}}%
\pgfpathlineto{\pgfqpoint{2.412326in}{0.739656in}}%
\pgfpathlineto{\pgfqpoint{2.412028in}{0.739656in}}%
\pgfpathlineto{\pgfqpoint{2.411731in}{0.739656in}}%
\pgfpathlineto{\pgfqpoint{2.411433in}{0.739656in}}%
\pgfpathlineto{\pgfqpoint{2.411136in}{0.739656in}}%
\pgfpathlineto{\pgfqpoint{2.410838in}{0.739656in}}%
\pgfpathlineto{\pgfqpoint{2.410541in}{0.739656in}}%
\pgfpathlineto{\pgfqpoint{2.410243in}{0.739656in}}%
\pgfpathlineto{\pgfqpoint{2.409946in}{0.739656in}}%
\pgfpathlineto{\pgfqpoint{2.409648in}{0.739656in}}%
\pgfpathlineto{\pgfqpoint{2.409351in}{0.739656in}}%
\pgfpathlineto{\pgfqpoint{2.409053in}{0.739656in}}%
\pgfpathlineto{\pgfqpoint{2.408756in}{0.739656in}}%
\pgfpathlineto{\pgfqpoint{2.408458in}{0.739656in}}%
\pgfpathlineto{\pgfqpoint{2.408161in}{0.739656in}}%
\pgfpathlineto{\pgfqpoint{2.407863in}{0.739656in}}%
\pgfpathlineto{\pgfqpoint{2.407566in}{0.739656in}}%
\pgfpathlineto{\pgfqpoint{2.407268in}{0.739656in}}%
\pgfpathlineto{\pgfqpoint{2.406971in}{0.739656in}}%
\pgfpathlineto{\pgfqpoint{2.406673in}{0.739656in}}%
\pgfpathlineto{\pgfqpoint{2.406376in}{0.739656in}}%
\pgfpathlineto{\pgfqpoint{2.406079in}{0.739656in}}%
\pgfpathlineto{\pgfqpoint{2.405781in}{0.739656in}}%
\pgfpathlineto{\pgfqpoint{2.405484in}{0.739656in}}%
\pgfpathlineto{\pgfqpoint{2.405186in}{0.739656in}}%
\pgfpathlineto{\pgfqpoint{2.404889in}{0.739656in}}%
\pgfpathlineto{\pgfqpoint{2.404591in}{0.739656in}}%
\pgfpathlineto{\pgfqpoint{2.404294in}{0.739656in}}%
\pgfpathlineto{\pgfqpoint{2.403996in}{0.739656in}}%
\pgfpathlineto{\pgfqpoint{2.403699in}{0.739656in}}%
\pgfpathlineto{\pgfqpoint{2.403401in}{0.739656in}}%
\pgfpathlineto{\pgfqpoint{2.403104in}{0.739656in}}%
\pgfpathlineto{\pgfqpoint{2.402806in}{0.739656in}}%
\pgfpathlineto{\pgfqpoint{2.402509in}{0.739656in}}%
\pgfpathlineto{\pgfqpoint{2.402211in}{0.739656in}}%
\pgfpathlineto{\pgfqpoint{2.401914in}{0.739656in}}%
\pgfpathlineto{\pgfqpoint{2.401616in}{0.739656in}}%
\pgfpathlineto{\pgfqpoint{2.401319in}{0.739656in}}%
\pgfpathlineto{\pgfqpoint{2.401021in}{0.739656in}}%
\pgfpathlineto{\pgfqpoint{2.400724in}{0.739656in}}%
\pgfpathlineto{\pgfqpoint{2.400426in}{0.739656in}}%
\pgfpathlineto{\pgfqpoint{2.400129in}{0.739656in}}%
\pgfpathlineto{\pgfqpoint{2.399831in}{0.739656in}}%
\pgfpathlineto{\pgfqpoint{2.399534in}{0.739656in}}%
\pgfpathlineto{\pgfqpoint{2.399237in}{0.739656in}}%
\pgfpathlineto{\pgfqpoint{2.398939in}{0.739656in}}%
\pgfpathlineto{\pgfqpoint{2.398642in}{0.739656in}}%
\pgfpathlineto{\pgfqpoint{2.398344in}{0.739656in}}%
\pgfpathlineto{\pgfqpoint{2.398047in}{0.739656in}}%
\pgfpathlineto{\pgfqpoint{2.397749in}{0.739656in}}%
\pgfpathlineto{\pgfqpoint{2.397452in}{0.739656in}}%
\pgfpathlineto{\pgfqpoint{2.397154in}{0.739656in}}%
\pgfpathlineto{\pgfqpoint{2.396857in}{0.739656in}}%
\pgfpathlineto{\pgfqpoint{2.396559in}{0.739656in}}%
\pgfpathlineto{\pgfqpoint{2.396262in}{0.739656in}}%
\pgfpathlineto{\pgfqpoint{2.395964in}{0.739656in}}%
\pgfpathlineto{\pgfqpoint{2.395667in}{0.739656in}}%
\pgfpathlineto{\pgfqpoint{2.395369in}{0.739656in}}%
\pgfpathlineto{\pgfqpoint{2.395072in}{0.739656in}}%
\pgfpathlineto{\pgfqpoint{2.394774in}{0.739656in}}%
\pgfpathlineto{\pgfqpoint{2.394477in}{0.739656in}}%
\pgfpathlineto{\pgfqpoint{2.394179in}{0.739656in}}%
\pgfpathlineto{\pgfqpoint{2.393882in}{0.739656in}}%
\pgfpathlineto{\pgfqpoint{2.393584in}{0.739656in}}%
\pgfpathlineto{\pgfqpoint{2.393287in}{0.739656in}}%
\pgfpathlineto{\pgfqpoint{2.392989in}{0.739656in}}%
\pgfpathlineto{\pgfqpoint{2.392692in}{0.739656in}}%
\pgfpathlineto{\pgfqpoint{2.392395in}{0.739656in}}%
\pgfpathlineto{\pgfqpoint{2.392097in}{0.739656in}}%
\pgfpathlineto{\pgfqpoint{2.391800in}{0.739656in}}%
\pgfpathlineto{\pgfqpoint{2.391502in}{0.739656in}}%
\pgfpathlineto{\pgfqpoint{2.391205in}{0.739656in}}%
\pgfpathlineto{\pgfqpoint{2.390907in}{0.739656in}}%
\pgfpathlineto{\pgfqpoint{2.390610in}{0.739656in}}%
\pgfpathlineto{\pgfqpoint{2.390312in}{0.739656in}}%
\pgfpathlineto{\pgfqpoint{2.390015in}{0.739656in}}%
\pgfpathlineto{\pgfqpoint{2.389717in}{0.739656in}}%
\pgfpathlineto{\pgfqpoint{2.389420in}{0.739656in}}%
\pgfpathlineto{\pgfqpoint{2.389122in}{0.739656in}}%
\pgfpathlineto{\pgfqpoint{2.388825in}{0.739656in}}%
\pgfpathlineto{\pgfqpoint{2.388527in}{0.739656in}}%
\pgfpathlineto{\pgfqpoint{2.388230in}{0.739656in}}%
\pgfpathlineto{\pgfqpoint{2.387932in}{0.739656in}}%
\pgfpathlineto{\pgfqpoint{2.387635in}{0.739656in}}%
\pgfpathlineto{\pgfqpoint{2.387337in}{0.739656in}}%
\pgfpathlineto{\pgfqpoint{2.387040in}{0.739656in}}%
\pgfpathlineto{\pgfqpoint{2.386742in}{0.739656in}}%
\pgfpathlineto{\pgfqpoint{2.386445in}{0.739656in}}%
\pgfpathlineto{\pgfqpoint{2.386148in}{0.739656in}}%
\pgfpathlineto{\pgfqpoint{2.385850in}{0.739656in}}%
\pgfpathlineto{\pgfqpoint{2.385553in}{0.739656in}}%
\pgfpathlineto{\pgfqpoint{2.385255in}{0.739656in}}%
\pgfpathlineto{\pgfqpoint{2.384958in}{0.739656in}}%
\pgfpathlineto{\pgfqpoint{2.384660in}{0.739656in}}%
\pgfpathlineto{\pgfqpoint{2.384363in}{0.739656in}}%
\pgfpathlineto{\pgfqpoint{2.384065in}{0.739656in}}%
\pgfpathlineto{\pgfqpoint{2.383768in}{0.739656in}}%
\pgfpathlineto{\pgfqpoint{2.383470in}{0.739656in}}%
\pgfpathlineto{\pgfqpoint{2.383173in}{0.739656in}}%
\pgfpathlineto{\pgfqpoint{2.382875in}{0.739656in}}%
\pgfpathlineto{\pgfqpoint{2.382578in}{0.739656in}}%
\pgfpathlineto{\pgfqpoint{2.382280in}{0.739656in}}%
\pgfpathlineto{\pgfqpoint{2.381983in}{0.739656in}}%
\pgfpathlineto{\pgfqpoint{2.381685in}{0.739656in}}%
\pgfpathlineto{\pgfqpoint{2.381388in}{0.739656in}}%
\pgfpathlineto{\pgfqpoint{2.381090in}{0.739656in}}%
\pgfpathlineto{\pgfqpoint{2.380793in}{0.739656in}}%
\pgfpathlineto{\pgfqpoint{2.380495in}{0.739656in}}%
\pgfpathlineto{\pgfqpoint{2.380198in}{0.739656in}}%
\pgfpathlineto{\pgfqpoint{2.379900in}{0.739656in}}%
\pgfpathlineto{\pgfqpoint{2.379603in}{0.739656in}}%
\pgfpathlineto{\pgfqpoint{2.379306in}{0.739656in}}%
\pgfpathlineto{\pgfqpoint{2.379008in}{0.739656in}}%
\pgfpathlineto{\pgfqpoint{2.378711in}{0.739656in}}%
\pgfpathlineto{\pgfqpoint{2.378413in}{0.739656in}}%
\pgfpathlineto{\pgfqpoint{2.378116in}{0.739656in}}%
\pgfpathlineto{\pgfqpoint{2.377818in}{0.739656in}}%
\pgfpathlineto{\pgfqpoint{2.377521in}{0.739656in}}%
\pgfpathlineto{\pgfqpoint{2.377223in}{0.739656in}}%
\pgfpathlineto{\pgfqpoint{2.376926in}{0.739656in}}%
\pgfpathlineto{\pgfqpoint{2.376628in}{0.739656in}}%
\pgfpathlineto{\pgfqpoint{2.376331in}{0.739656in}}%
\pgfpathlineto{\pgfqpoint{2.376033in}{0.739656in}}%
\pgfpathlineto{\pgfqpoint{2.375736in}{0.739656in}}%
\pgfpathlineto{\pgfqpoint{2.375438in}{0.739656in}}%
\pgfpathlineto{\pgfqpoint{2.375141in}{0.739656in}}%
\pgfpathlineto{\pgfqpoint{2.374843in}{0.739656in}}%
\pgfpathlineto{\pgfqpoint{2.374546in}{0.739656in}}%
\pgfpathlineto{\pgfqpoint{2.374248in}{0.739656in}}%
\pgfpathlineto{\pgfqpoint{2.373951in}{0.739656in}}%
\pgfpathlineto{\pgfqpoint{2.373653in}{0.739656in}}%
\pgfpathlineto{\pgfqpoint{2.373356in}{0.739656in}}%
\pgfpathlineto{\pgfqpoint{2.373058in}{0.739656in}}%
\pgfpathlineto{\pgfqpoint{2.372761in}{0.739656in}}%
\pgfpathlineto{\pgfqpoint{2.372464in}{0.739656in}}%
\pgfpathlineto{\pgfqpoint{2.372166in}{0.739656in}}%
\pgfpathlineto{\pgfqpoint{2.371869in}{0.739656in}}%
\pgfpathlineto{\pgfqpoint{2.371571in}{0.739656in}}%
\pgfpathlineto{\pgfqpoint{2.371274in}{0.739656in}}%
\pgfpathlineto{\pgfqpoint{2.370976in}{0.739656in}}%
\pgfpathlineto{\pgfqpoint{2.370679in}{0.739656in}}%
\pgfpathlineto{\pgfqpoint{2.370381in}{0.739656in}}%
\pgfpathlineto{\pgfqpoint{2.370084in}{0.739656in}}%
\pgfpathlineto{\pgfqpoint{2.369786in}{0.739656in}}%
\pgfpathlineto{\pgfqpoint{2.369489in}{0.739656in}}%
\pgfpathlineto{\pgfqpoint{2.369191in}{0.739656in}}%
\pgfpathlineto{\pgfqpoint{2.368894in}{0.739656in}}%
\pgfpathlineto{\pgfqpoint{2.368596in}{0.739656in}}%
\pgfpathlineto{\pgfqpoint{2.368299in}{0.739656in}}%
\pgfpathlineto{\pgfqpoint{2.368001in}{0.739656in}}%
\pgfpathlineto{\pgfqpoint{2.367704in}{0.739656in}}%
\pgfpathlineto{\pgfqpoint{2.367406in}{0.739656in}}%
\pgfpathlineto{\pgfqpoint{2.367109in}{0.739656in}}%
\pgfpathlineto{\pgfqpoint{2.366811in}{0.739656in}}%
\pgfpathlineto{\pgfqpoint{2.366514in}{0.739656in}}%
\pgfpathlineto{\pgfqpoint{2.366217in}{0.739656in}}%
\pgfpathlineto{\pgfqpoint{2.365919in}{0.739656in}}%
\pgfpathlineto{\pgfqpoint{2.365622in}{0.739656in}}%
\pgfpathlineto{\pgfqpoint{2.365324in}{0.739656in}}%
\pgfpathlineto{\pgfqpoint{2.365027in}{0.739656in}}%
\pgfpathlineto{\pgfqpoint{2.364729in}{0.739656in}}%
\pgfpathlineto{\pgfqpoint{2.364432in}{0.739656in}}%
\pgfpathlineto{\pgfqpoint{2.364134in}{0.739656in}}%
\pgfpathlineto{\pgfqpoint{2.363837in}{0.739656in}}%
\pgfpathlineto{\pgfqpoint{2.363539in}{0.739656in}}%
\pgfpathlineto{\pgfqpoint{2.363242in}{0.739656in}}%
\pgfpathlineto{\pgfqpoint{2.362944in}{0.739656in}}%
\pgfpathlineto{\pgfqpoint{2.362647in}{0.739656in}}%
\pgfpathlineto{\pgfqpoint{2.362349in}{0.739656in}}%
\pgfpathlineto{\pgfqpoint{2.362052in}{0.739656in}}%
\pgfpathlineto{\pgfqpoint{2.361754in}{0.739656in}}%
\pgfpathlineto{\pgfqpoint{2.361457in}{0.739656in}}%
\pgfpathlineto{\pgfqpoint{2.361159in}{0.739656in}}%
\pgfpathlineto{\pgfqpoint{2.360862in}{0.739656in}}%
\pgfpathlineto{\pgfqpoint{2.360564in}{0.739656in}}%
\pgfpathlineto{\pgfqpoint{2.360267in}{0.739656in}}%
\pgfpathlineto{\pgfqpoint{2.359969in}{0.739656in}}%
\pgfpathlineto{\pgfqpoint{2.359672in}{0.739656in}}%
\pgfpathlineto{\pgfqpoint{2.359375in}{0.739656in}}%
\pgfpathlineto{\pgfqpoint{2.359077in}{0.739656in}}%
\pgfpathlineto{\pgfqpoint{2.358780in}{0.739656in}}%
\pgfpathlineto{\pgfqpoint{2.358482in}{0.739656in}}%
\pgfpathlineto{\pgfqpoint{2.358185in}{0.739656in}}%
\pgfpathlineto{\pgfqpoint{2.357887in}{0.739656in}}%
\pgfpathlineto{\pgfqpoint{2.357590in}{0.739656in}}%
\pgfpathlineto{\pgfqpoint{2.357292in}{0.739656in}}%
\pgfpathlineto{\pgfqpoint{2.356995in}{0.739656in}}%
\pgfpathlineto{\pgfqpoint{2.356697in}{0.739656in}}%
\pgfpathlineto{\pgfqpoint{2.356400in}{0.739656in}}%
\pgfpathlineto{\pgfqpoint{2.356102in}{0.739656in}}%
\pgfpathlineto{\pgfqpoint{2.355805in}{0.739656in}}%
\pgfpathlineto{\pgfqpoint{2.355507in}{0.739656in}}%
\pgfpathlineto{\pgfqpoint{2.355210in}{0.739656in}}%
\pgfpathlineto{\pgfqpoint{2.354912in}{0.739656in}}%
\pgfpathlineto{\pgfqpoint{2.354615in}{0.739656in}}%
\pgfpathlineto{\pgfqpoint{2.354317in}{0.739656in}}%
\pgfpathlineto{\pgfqpoint{2.354020in}{0.739656in}}%
\pgfpathlineto{\pgfqpoint{2.353722in}{0.739656in}}%
\pgfpathlineto{\pgfqpoint{2.353425in}{0.739656in}}%
\pgfpathlineto{\pgfqpoint{2.353127in}{0.739656in}}%
\pgfpathlineto{\pgfqpoint{2.352830in}{0.739656in}}%
\pgfpathlineto{\pgfqpoint{2.352533in}{0.739656in}}%
\pgfpathlineto{\pgfqpoint{2.352235in}{0.739656in}}%
\pgfpathlineto{\pgfqpoint{2.351938in}{0.739656in}}%
\pgfpathlineto{\pgfqpoint{2.351640in}{0.739656in}}%
\pgfpathlineto{\pgfqpoint{2.351343in}{0.739656in}}%
\pgfpathlineto{\pgfqpoint{2.351045in}{0.739656in}}%
\pgfpathlineto{\pgfqpoint{2.350748in}{0.739656in}}%
\pgfpathlineto{\pgfqpoint{2.350450in}{0.739656in}}%
\pgfpathlineto{\pgfqpoint{2.350153in}{0.739656in}}%
\pgfpathlineto{\pgfqpoint{2.349855in}{0.739656in}}%
\pgfpathlineto{\pgfqpoint{2.349558in}{0.739656in}}%
\pgfpathlineto{\pgfqpoint{2.349260in}{0.739656in}}%
\pgfpathlineto{\pgfqpoint{2.348963in}{0.739656in}}%
\pgfpathlineto{\pgfqpoint{2.348665in}{0.739656in}}%
\pgfpathlineto{\pgfqpoint{2.348368in}{0.739656in}}%
\pgfpathlineto{\pgfqpoint{2.348070in}{0.739656in}}%
\pgfpathlineto{\pgfqpoint{2.347773in}{0.739656in}}%
\pgfpathlineto{\pgfqpoint{2.347475in}{0.739656in}}%
\pgfpathlineto{\pgfqpoint{2.347178in}{0.739656in}}%
\pgfpathlineto{\pgfqpoint{2.346880in}{0.739656in}}%
\pgfpathlineto{\pgfqpoint{2.346583in}{0.739656in}}%
\pgfpathlineto{\pgfqpoint{2.346286in}{0.739656in}}%
\pgfpathlineto{\pgfqpoint{2.345988in}{0.739656in}}%
\pgfpathlineto{\pgfqpoint{2.345691in}{0.739656in}}%
\pgfpathlineto{\pgfqpoint{2.345393in}{0.739656in}}%
\pgfpathlineto{\pgfqpoint{2.345096in}{0.739656in}}%
\pgfpathlineto{\pgfqpoint{2.344798in}{0.739656in}}%
\pgfpathlineto{\pgfqpoint{2.344501in}{0.739656in}}%
\pgfpathlineto{\pgfqpoint{2.344203in}{0.739656in}}%
\pgfpathlineto{\pgfqpoint{2.343906in}{0.739656in}}%
\pgfpathlineto{\pgfqpoint{2.343608in}{0.739656in}}%
\pgfpathlineto{\pgfqpoint{2.343311in}{0.739656in}}%
\pgfpathlineto{\pgfqpoint{2.343013in}{0.739656in}}%
\pgfpathlineto{\pgfqpoint{2.342716in}{0.739656in}}%
\pgfpathlineto{\pgfqpoint{2.342418in}{0.739656in}}%
\pgfpathlineto{\pgfqpoint{2.342121in}{0.739656in}}%
\pgfpathlineto{\pgfqpoint{2.341823in}{0.739656in}}%
\pgfpathlineto{\pgfqpoint{2.341526in}{0.739656in}}%
\pgfpathlineto{\pgfqpoint{2.341228in}{0.739656in}}%
\pgfpathlineto{\pgfqpoint{2.340931in}{0.739656in}}%
\pgfpathlineto{\pgfqpoint{2.340633in}{0.739656in}}%
\pgfpathlineto{\pgfqpoint{2.340336in}{0.739656in}}%
\pgfpathlineto{\pgfqpoint{2.340038in}{0.739656in}}%
\pgfpathlineto{\pgfqpoint{2.339741in}{0.739656in}}%
\pgfpathlineto{\pgfqpoint{2.339444in}{0.739656in}}%
\pgfpathlineto{\pgfqpoint{2.339146in}{0.739656in}}%
\pgfpathlineto{\pgfqpoint{2.338849in}{0.739656in}}%
\pgfpathlineto{\pgfqpoint{2.338551in}{0.739656in}}%
\pgfpathlineto{\pgfqpoint{2.338254in}{0.739656in}}%
\pgfpathlineto{\pgfqpoint{2.337956in}{0.739656in}}%
\pgfpathlineto{\pgfqpoint{2.337659in}{0.739656in}}%
\pgfpathlineto{\pgfqpoint{2.337361in}{0.739656in}}%
\pgfpathlineto{\pgfqpoint{2.337064in}{0.739656in}}%
\pgfpathlineto{\pgfqpoint{2.336766in}{0.739656in}}%
\pgfpathlineto{\pgfqpoint{2.336469in}{0.739656in}}%
\pgfpathlineto{\pgfqpoint{2.336171in}{0.739656in}}%
\pgfpathlineto{\pgfqpoint{2.335874in}{0.739656in}}%
\pgfpathlineto{\pgfqpoint{2.335576in}{0.739656in}}%
\pgfpathlineto{\pgfqpoint{2.335279in}{0.739656in}}%
\pgfpathlineto{\pgfqpoint{2.334981in}{0.739656in}}%
\pgfpathlineto{\pgfqpoint{2.334684in}{0.739656in}}%
\pgfpathlineto{\pgfqpoint{2.334386in}{0.739656in}}%
\pgfpathlineto{\pgfqpoint{2.334089in}{0.739656in}}%
\pgfpathlineto{\pgfqpoint{2.333791in}{0.739656in}}%
\pgfpathlineto{\pgfqpoint{2.333494in}{0.739656in}}%
\pgfpathlineto{\pgfqpoint{2.333196in}{0.739656in}}%
\pgfpathlineto{\pgfqpoint{2.332899in}{0.739656in}}%
\pgfpathlineto{\pgfqpoint{2.332602in}{0.739656in}}%
\pgfpathlineto{\pgfqpoint{2.332304in}{0.739656in}}%
\pgfpathlineto{\pgfqpoint{2.332007in}{0.739656in}}%
\pgfpathlineto{\pgfqpoint{2.331709in}{0.739656in}}%
\pgfpathlineto{\pgfqpoint{2.331412in}{0.739656in}}%
\pgfpathlineto{\pgfqpoint{2.331114in}{0.739656in}}%
\pgfpathlineto{\pgfqpoint{2.330817in}{0.739656in}}%
\pgfpathlineto{\pgfqpoint{2.330519in}{0.739656in}}%
\pgfpathlineto{\pgfqpoint{2.330222in}{0.739656in}}%
\pgfpathlineto{\pgfqpoint{2.329924in}{0.739656in}}%
\pgfpathlineto{\pgfqpoint{2.329627in}{0.739656in}}%
\pgfpathlineto{\pgfqpoint{2.329329in}{0.739656in}}%
\pgfpathlineto{\pgfqpoint{2.329032in}{0.739656in}}%
\pgfpathlineto{\pgfqpoint{2.328734in}{0.739656in}}%
\pgfpathlineto{\pgfqpoint{2.328437in}{0.739656in}}%
\pgfpathlineto{\pgfqpoint{2.328139in}{0.739656in}}%
\pgfpathlineto{\pgfqpoint{2.327842in}{0.739656in}}%
\pgfpathlineto{\pgfqpoint{2.327544in}{0.739656in}}%
\pgfpathlineto{\pgfqpoint{2.327247in}{0.739656in}}%
\pgfpathlineto{\pgfqpoint{2.326949in}{0.739656in}}%
\pgfpathlineto{\pgfqpoint{2.326652in}{0.739656in}}%
\pgfpathlineto{\pgfqpoint{2.326355in}{0.739656in}}%
\pgfpathlineto{\pgfqpoint{2.326057in}{0.739656in}}%
\pgfpathlineto{\pgfqpoint{2.325760in}{0.739656in}}%
\pgfpathlineto{\pgfqpoint{2.325462in}{0.739656in}}%
\pgfpathlineto{\pgfqpoint{2.325165in}{0.739656in}}%
\pgfpathlineto{\pgfqpoint{2.324867in}{0.739656in}}%
\pgfpathlineto{\pgfqpoint{2.324570in}{0.739656in}}%
\pgfpathlineto{\pgfqpoint{2.324272in}{0.739656in}}%
\pgfpathlineto{\pgfqpoint{2.323975in}{0.739656in}}%
\pgfpathlineto{\pgfqpoint{2.323677in}{0.739656in}}%
\pgfpathlineto{\pgfqpoint{2.323380in}{0.739656in}}%
\pgfpathlineto{\pgfqpoint{2.323082in}{0.739656in}}%
\pgfpathlineto{\pgfqpoint{2.322785in}{0.739656in}}%
\pgfpathlineto{\pgfqpoint{2.322487in}{0.739656in}}%
\pgfpathlineto{\pgfqpoint{2.322190in}{0.739656in}}%
\pgfpathlineto{\pgfqpoint{2.321892in}{0.739656in}}%
\pgfpathlineto{\pgfqpoint{2.321595in}{0.739656in}}%
\pgfpathlineto{\pgfqpoint{2.321297in}{0.739656in}}%
\pgfpathlineto{\pgfqpoint{2.321000in}{0.739656in}}%
\pgfpathlineto{\pgfqpoint{2.320702in}{0.739656in}}%
\pgfpathlineto{\pgfqpoint{2.320405in}{0.739656in}}%
\pgfpathlineto{\pgfqpoint{2.320107in}{0.739656in}}%
\pgfpathlineto{\pgfqpoint{2.319810in}{0.739656in}}%
\pgfpathlineto{\pgfqpoint{2.319513in}{0.739656in}}%
\pgfpathlineto{\pgfqpoint{2.319215in}{0.739656in}}%
\pgfpathlineto{\pgfqpoint{2.318918in}{0.739656in}}%
\pgfpathlineto{\pgfqpoint{2.318620in}{0.739656in}}%
\pgfpathlineto{\pgfqpoint{2.318323in}{0.739656in}}%
\pgfpathlineto{\pgfqpoint{2.318025in}{0.739656in}}%
\pgfpathlineto{\pgfqpoint{2.317728in}{0.739656in}}%
\pgfpathlineto{\pgfqpoint{2.317430in}{0.739656in}}%
\pgfpathlineto{\pgfqpoint{2.317133in}{0.739656in}}%
\pgfpathlineto{\pgfqpoint{2.316835in}{0.739656in}}%
\pgfpathlineto{\pgfqpoint{2.316538in}{0.739656in}}%
\pgfpathlineto{\pgfqpoint{2.316240in}{0.739656in}}%
\pgfpathlineto{\pgfqpoint{2.315943in}{0.739656in}}%
\pgfpathlineto{\pgfqpoint{2.315645in}{0.739656in}}%
\pgfpathlineto{\pgfqpoint{2.315348in}{0.739656in}}%
\pgfpathlineto{\pgfqpoint{2.315050in}{0.739656in}}%
\pgfpathlineto{\pgfqpoint{2.314753in}{0.739656in}}%
\pgfpathlineto{\pgfqpoint{2.314455in}{0.739656in}}%
\pgfpathlineto{\pgfqpoint{2.314158in}{0.739656in}}%
\pgfpathlineto{\pgfqpoint{2.313860in}{0.739656in}}%
\pgfpathlineto{\pgfqpoint{2.313563in}{0.739656in}}%
\pgfpathlineto{\pgfqpoint{2.313265in}{0.739656in}}%
\pgfpathlineto{\pgfqpoint{2.312968in}{0.739656in}}%
\pgfpathlineto{\pgfqpoint{2.312671in}{0.739656in}}%
\pgfpathlineto{\pgfqpoint{2.312373in}{0.739656in}}%
\pgfpathlineto{\pgfqpoint{2.312076in}{0.739656in}}%
\pgfpathlineto{\pgfqpoint{2.311778in}{0.739656in}}%
\pgfpathlineto{\pgfqpoint{2.311481in}{0.739656in}}%
\pgfpathlineto{\pgfqpoint{2.311183in}{0.739656in}}%
\pgfpathlineto{\pgfqpoint{2.310886in}{0.739656in}}%
\pgfpathlineto{\pgfqpoint{2.310588in}{0.739656in}}%
\pgfpathlineto{\pgfqpoint{2.310291in}{0.739656in}}%
\pgfpathlineto{\pgfqpoint{2.309993in}{0.739656in}}%
\pgfpathlineto{\pgfqpoint{2.309696in}{0.739656in}}%
\pgfpathlineto{\pgfqpoint{2.309398in}{0.739656in}}%
\pgfpathlineto{\pgfqpoint{2.309101in}{0.739656in}}%
\pgfpathlineto{\pgfqpoint{2.308803in}{0.739656in}}%
\pgfpathlineto{\pgfqpoint{2.308506in}{0.739656in}}%
\pgfpathlineto{\pgfqpoint{2.308208in}{0.739656in}}%
\pgfpathlineto{\pgfqpoint{2.307911in}{0.739656in}}%
\pgfpathlineto{\pgfqpoint{2.307613in}{0.739656in}}%
\pgfpathlineto{\pgfqpoint{2.307316in}{0.739656in}}%
\pgfpathlineto{\pgfqpoint{2.307018in}{0.739656in}}%
\pgfpathlineto{\pgfqpoint{2.306721in}{0.739656in}}%
\pgfpathlineto{\pgfqpoint{2.306424in}{0.739656in}}%
\pgfpathlineto{\pgfqpoint{2.306126in}{0.739656in}}%
\pgfpathlineto{\pgfqpoint{2.305829in}{0.739656in}}%
\pgfpathlineto{\pgfqpoint{2.305531in}{0.739656in}}%
\pgfpathlineto{\pgfqpoint{2.305234in}{0.739656in}}%
\pgfpathlineto{\pgfqpoint{2.304936in}{0.739656in}}%
\pgfpathlineto{\pgfqpoint{2.304639in}{0.739656in}}%
\pgfpathlineto{\pgfqpoint{2.304341in}{0.739656in}}%
\pgfpathlineto{\pgfqpoint{2.304044in}{0.739656in}}%
\pgfpathlineto{\pgfqpoint{2.303746in}{0.739656in}}%
\pgfpathlineto{\pgfqpoint{2.303449in}{0.739656in}}%
\pgfpathlineto{\pgfqpoint{2.303151in}{0.739656in}}%
\pgfpathlineto{\pgfqpoint{2.302854in}{0.739656in}}%
\pgfpathlineto{\pgfqpoint{2.302556in}{0.739656in}}%
\pgfpathlineto{\pgfqpoint{2.302259in}{0.739656in}}%
\pgfpathlineto{\pgfqpoint{2.301961in}{0.739656in}}%
\pgfpathlineto{\pgfqpoint{2.301664in}{0.739656in}}%
\pgfpathlineto{\pgfqpoint{2.301366in}{0.739656in}}%
\pgfpathlineto{\pgfqpoint{2.301069in}{0.739656in}}%
\pgfpathlineto{\pgfqpoint{2.300771in}{0.739656in}}%
\pgfpathlineto{\pgfqpoint{2.300474in}{0.739656in}}%
\pgfpathlineto{\pgfqpoint{2.300176in}{0.739656in}}%
\pgfpathlineto{\pgfqpoint{2.299879in}{0.739656in}}%
\pgfpathlineto{\pgfqpoint{2.299582in}{0.739656in}}%
\pgfpathlineto{\pgfqpoint{2.299284in}{0.739656in}}%
\pgfpathlineto{\pgfqpoint{2.298987in}{0.739656in}}%
\pgfpathlineto{\pgfqpoint{2.298689in}{0.739656in}}%
\pgfpathlineto{\pgfqpoint{2.298392in}{0.739656in}}%
\pgfpathlineto{\pgfqpoint{2.298094in}{0.739656in}}%
\pgfpathlineto{\pgfqpoint{2.297797in}{0.739656in}}%
\pgfpathlineto{\pgfqpoint{2.297499in}{0.739656in}}%
\pgfpathlineto{\pgfqpoint{2.297202in}{0.739656in}}%
\pgfpathlineto{\pgfqpoint{2.296904in}{0.739656in}}%
\pgfpathlineto{\pgfqpoint{2.296607in}{0.739656in}}%
\pgfpathlineto{\pgfqpoint{2.296309in}{0.739656in}}%
\pgfpathlineto{\pgfqpoint{2.296012in}{0.739656in}}%
\pgfpathlineto{\pgfqpoint{2.295714in}{0.739656in}}%
\pgfpathlineto{\pgfqpoint{2.295417in}{0.739656in}}%
\pgfpathlineto{\pgfqpoint{2.295119in}{0.739656in}}%
\pgfpathlineto{\pgfqpoint{2.294822in}{0.739656in}}%
\pgfpathlineto{\pgfqpoint{2.294524in}{0.739656in}}%
\pgfpathlineto{\pgfqpoint{2.294227in}{0.739656in}}%
\pgfpathlineto{\pgfqpoint{2.293929in}{0.739656in}}%
\pgfpathlineto{\pgfqpoint{2.293632in}{0.739656in}}%
\pgfpathlineto{\pgfqpoint{2.293334in}{0.739656in}}%
\pgfpathlineto{\pgfqpoint{2.293037in}{0.739656in}}%
\pgfpathlineto{\pgfqpoint{2.292740in}{0.739656in}}%
\pgfpathlineto{\pgfqpoint{2.292442in}{0.739656in}}%
\pgfpathlineto{\pgfqpoint{2.292145in}{0.739656in}}%
\pgfpathlineto{\pgfqpoint{2.291847in}{0.739656in}}%
\pgfpathlineto{\pgfqpoint{2.291550in}{0.739656in}}%
\pgfpathlineto{\pgfqpoint{2.291252in}{0.739656in}}%
\pgfpathlineto{\pgfqpoint{2.290955in}{0.739656in}}%
\pgfpathlineto{\pgfqpoint{2.290657in}{0.739656in}}%
\pgfpathlineto{\pgfqpoint{2.290360in}{0.739656in}}%
\pgfpathlineto{\pgfqpoint{2.290062in}{0.739656in}}%
\pgfpathlineto{\pgfqpoint{2.289765in}{0.739656in}}%
\pgfpathlineto{\pgfqpoint{2.289467in}{0.739656in}}%
\pgfpathlineto{\pgfqpoint{2.289170in}{0.739656in}}%
\pgfpathlineto{\pgfqpoint{2.288872in}{0.739656in}}%
\pgfpathlineto{\pgfqpoint{2.288575in}{0.739656in}}%
\pgfpathlineto{\pgfqpoint{2.288277in}{0.739656in}}%
\pgfpathlineto{\pgfqpoint{2.287980in}{0.739656in}}%
\pgfpathlineto{\pgfqpoint{2.287682in}{0.739656in}}%
\pgfpathlineto{\pgfqpoint{2.287385in}{0.739656in}}%
\pgfpathlineto{\pgfqpoint{2.287087in}{0.739656in}}%
\pgfpathlineto{\pgfqpoint{2.286790in}{0.739656in}}%
\pgfpathlineto{\pgfqpoint{2.286493in}{0.739656in}}%
\pgfpathlineto{\pgfqpoint{2.286195in}{0.739656in}}%
\pgfpathlineto{\pgfqpoint{2.285898in}{0.739656in}}%
\pgfpathlineto{\pgfqpoint{2.285600in}{0.739656in}}%
\pgfpathlineto{\pgfqpoint{2.285303in}{0.739656in}}%
\pgfpathlineto{\pgfqpoint{2.285005in}{0.739656in}}%
\pgfpathlineto{\pgfqpoint{2.284708in}{0.739656in}}%
\pgfpathlineto{\pgfqpoint{2.284410in}{0.739656in}}%
\pgfpathlineto{\pgfqpoint{2.284113in}{0.739656in}}%
\pgfpathlineto{\pgfqpoint{2.283815in}{0.739656in}}%
\pgfpathlineto{\pgfqpoint{2.283518in}{0.739656in}}%
\pgfpathlineto{\pgfqpoint{2.283220in}{0.739656in}}%
\pgfpathlineto{\pgfqpoint{2.282923in}{0.739656in}}%
\pgfpathlineto{\pgfqpoint{2.282625in}{0.739656in}}%
\pgfpathlineto{\pgfqpoint{2.282328in}{0.739656in}}%
\pgfpathlineto{\pgfqpoint{2.282030in}{0.739656in}}%
\pgfpathlineto{\pgfqpoint{2.281733in}{0.739656in}}%
\pgfpathlineto{\pgfqpoint{2.281435in}{0.739656in}}%
\pgfpathlineto{\pgfqpoint{2.281138in}{0.739656in}}%
\pgfpathlineto{\pgfqpoint{2.280840in}{0.739656in}}%
\pgfpathlineto{\pgfqpoint{2.280543in}{0.739656in}}%
\pgfpathlineto{\pgfqpoint{2.280245in}{0.739656in}}%
\pgfpathlineto{\pgfqpoint{2.279948in}{0.739656in}}%
\pgfpathlineto{\pgfqpoint{2.279651in}{0.739656in}}%
\pgfpathlineto{\pgfqpoint{2.279353in}{0.739656in}}%
\pgfpathlineto{\pgfqpoint{2.279056in}{0.739656in}}%
\pgfpathlineto{\pgfqpoint{2.278758in}{0.739656in}}%
\pgfpathlineto{\pgfqpoint{2.278461in}{0.739656in}}%
\pgfpathlineto{\pgfqpoint{2.278163in}{0.739656in}}%
\pgfpathlineto{\pgfqpoint{2.277866in}{0.739656in}}%
\pgfpathlineto{\pgfqpoint{2.277568in}{0.739656in}}%
\pgfpathlineto{\pgfqpoint{2.277271in}{0.739656in}}%
\pgfpathlineto{\pgfqpoint{2.276973in}{0.739656in}}%
\pgfpathlineto{\pgfqpoint{2.276676in}{0.739656in}}%
\pgfpathlineto{\pgfqpoint{2.276378in}{0.739656in}}%
\pgfpathlineto{\pgfqpoint{2.276081in}{0.739656in}}%
\pgfpathlineto{\pgfqpoint{2.275783in}{0.739656in}}%
\pgfpathlineto{\pgfqpoint{2.275486in}{0.739656in}}%
\pgfpathlineto{\pgfqpoint{2.275188in}{0.739656in}}%
\pgfpathlineto{\pgfqpoint{2.274891in}{0.739656in}}%
\pgfpathlineto{\pgfqpoint{2.274593in}{0.739656in}}%
\pgfpathlineto{\pgfqpoint{2.274296in}{0.739656in}}%
\pgfpathlineto{\pgfqpoint{2.273998in}{0.739656in}}%
\pgfpathlineto{\pgfqpoint{2.273701in}{0.739656in}}%
\pgfpathlineto{\pgfqpoint{2.273403in}{0.739656in}}%
\pgfpathlineto{\pgfqpoint{2.273106in}{0.739656in}}%
\pgfpathlineto{\pgfqpoint{2.272809in}{0.739656in}}%
\pgfpathlineto{\pgfqpoint{2.272511in}{0.739656in}}%
\pgfpathlineto{\pgfqpoint{2.272214in}{0.739656in}}%
\pgfpathlineto{\pgfqpoint{2.271916in}{0.739656in}}%
\pgfpathlineto{\pgfqpoint{2.271619in}{0.739656in}}%
\pgfpathlineto{\pgfqpoint{2.271321in}{0.739656in}}%
\pgfpathlineto{\pgfqpoint{2.271024in}{0.739656in}}%
\pgfpathlineto{\pgfqpoint{2.270726in}{0.739656in}}%
\pgfpathlineto{\pgfqpoint{2.270429in}{0.739656in}}%
\pgfpathlineto{\pgfqpoint{2.270131in}{0.739656in}}%
\pgfpathlineto{\pgfqpoint{2.269834in}{0.739656in}}%
\pgfpathlineto{\pgfqpoint{2.269536in}{0.739656in}}%
\pgfpathlineto{\pgfqpoint{2.269239in}{0.739656in}}%
\pgfpathlineto{\pgfqpoint{2.268941in}{0.739656in}}%
\pgfpathlineto{\pgfqpoint{2.268644in}{0.739656in}}%
\pgfpathlineto{\pgfqpoint{2.268346in}{0.739656in}}%
\pgfpathlineto{\pgfqpoint{2.268049in}{0.739656in}}%
\pgfpathlineto{\pgfqpoint{2.267751in}{0.739656in}}%
\pgfpathlineto{\pgfqpoint{2.267454in}{0.739656in}}%
\pgfpathlineto{\pgfqpoint{2.267156in}{0.739656in}}%
\pgfpathlineto{\pgfqpoint{2.266859in}{0.739656in}}%
\pgfpathlineto{\pgfqpoint{2.266562in}{0.739656in}}%
\pgfpathlineto{\pgfqpoint{2.266264in}{0.739656in}}%
\pgfpathlineto{\pgfqpoint{2.265967in}{0.739656in}}%
\pgfpathlineto{\pgfqpoint{2.265669in}{0.739656in}}%
\pgfpathlineto{\pgfqpoint{2.265372in}{0.739656in}}%
\pgfpathlineto{\pgfqpoint{2.265074in}{0.739656in}}%
\pgfpathlineto{\pgfqpoint{2.264777in}{0.739656in}}%
\pgfpathlineto{\pgfqpoint{2.264479in}{0.739656in}}%
\pgfpathlineto{\pgfqpoint{2.264182in}{0.739656in}}%
\pgfpathlineto{\pgfqpoint{2.263884in}{0.739656in}}%
\pgfpathlineto{\pgfqpoint{2.263587in}{0.739656in}}%
\pgfpathlineto{\pgfqpoint{2.263289in}{0.739656in}}%
\pgfpathlineto{\pgfqpoint{2.262992in}{0.739656in}}%
\pgfpathlineto{\pgfqpoint{2.262694in}{0.739656in}}%
\pgfpathlineto{\pgfqpoint{2.262397in}{0.739656in}}%
\pgfpathlineto{\pgfqpoint{2.262099in}{0.739656in}}%
\pgfpathlineto{\pgfqpoint{2.261802in}{0.739656in}}%
\pgfpathlineto{\pgfqpoint{2.261504in}{0.739656in}}%
\pgfpathlineto{\pgfqpoint{2.261207in}{0.739656in}}%
\pgfpathlineto{\pgfqpoint{2.260909in}{0.739656in}}%
\pgfpathlineto{\pgfqpoint{2.260612in}{0.739656in}}%
\pgfpathlineto{\pgfqpoint{2.260314in}{0.739656in}}%
\pgfpathlineto{\pgfqpoint{2.260017in}{0.739656in}}%
\pgfpathlineto{\pgfqpoint{2.259720in}{0.739656in}}%
\pgfpathlineto{\pgfqpoint{2.259422in}{0.739656in}}%
\pgfpathlineto{\pgfqpoint{2.259125in}{0.739656in}}%
\pgfpathlineto{\pgfqpoint{2.258827in}{0.739656in}}%
\pgfpathlineto{\pgfqpoint{2.258530in}{0.739656in}}%
\pgfpathlineto{\pgfqpoint{2.258232in}{0.739656in}}%
\pgfpathlineto{\pgfqpoint{2.257935in}{0.739656in}}%
\pgfpathlineto{\pgfqpoint{2.257637in}{0.739656in}}%
\pgfpathlineto{\pgfqpoint{2.257340in}{0.739656in}}%
\pgfpathlineto{\pgfqpoint{2.257042in}{0.739656in}}%
\pgfpathlineto{\pgfqpoint{2.256745in}{0.739656in}}%
\pgfpathlineto{\pgfqpoint{2.256447in}{0.739656in}}%
\pgfpathlineto{\pgfqpoint{2.256150in}{0.739656in}}%
\pgfpathlineto{\pgfqpoint{2.255852in}{0.739656in}}%
\pgfpathlineto{\pgfqpoint{2.255555in}{0.739656in}}%
\pgfpathlineto{\pgfqpoint{2.255257in}{0.739656in}}%
\pgfpathlineto{\pgfqpoint{2.254960in}{0.739656in}}%
\pgfpathlineto{\pgfqpoint{2.254662in}{0.739656in}}%
\pgfpathlineto{\pgfqpoint{2.254365in}{0.739656in}}%
\pgfpathlineto{\pgfqpoint{2.254067in}{0.739656in}}%
\pgfpathlineto{\pgfqpoint{2.253770in}{0.739656in}}%
\pgfpathlineto{\pgfqpoint{2.253472in}{0.739656in}}%
\pgfpathlineto{\pgfqpoint{2.253175in}{0.739656in}}%
\pgfpathlineto{\pgfqpoint{2.252878in}{0.739656in}}%
\pgfpathlineto{\pgfqpoint{2.252580in}{0.739656in}}%
\pgfpathlineto{\pgfqpoint{2.252283in}{0.739656in}}%
\pgfpathlineto{\pgfqpoint{2.251985in}{0.739656in}}%
\pgfpathlineto{\pgfqpoint{2.251688in}{0.739656in}}%
\pgfpathlineto{\pgfqpoint{2.251390in}{0.739656in}}%
\pgfpathlineto{\pgfqpoint{2.251093in}{0.739656in}}%
\pgfpathlineto{\pgfqpoint{2.250795in}{0.739656in}}%
\pgfpathlineto{\pgfqpoint{2.250498in}{0.739656in}}%
\pgfpathlineto{\pgfqpoint{2.250200in}{0.739656in}}%
\pgfpathlineto{\pgfqpoint{2.249903in}{0.739656in}}%
\pgfpathlineto{\pgfqpoint{2.249605in}{0.739656in}}%
\pgfpathlineto{\pgfqpoint{2.249308in}{0.739656in}}%
\pgfpathlineto{\pgfqpoint{2.249010in}{0.739656in}}%
\pgfpathlineto{\pgfqpoint{2.248713in}{0.739656in}}%
\pgfpathlineto{\pgfqpoint{2.248415in}{0.739656in}}%
\pgfpathlineto{\pgfqpoint{2.248118in}{0.739656in}}%
\pgfpathlineto{\pgfqpoint{2.247820in}{0.739656in}}%
\pgfpathlineto{\pgfqpoint{2.247523in}{0.739656in}}%
\pgfpathlineto{\pgfqpoint{2.247225in}{0.739656in}}%
\pgfpathlineto{\pgfqpoint{2.246928in}{0.739656in}}%
\pgfpathlineto{\pgfqpoint{2.246631in}{0.739656in}}%
\pgfpathlineto{\pgfqpoint{2.246333in}{0.739656in}}%
\pgfpathlineto{\pgfqpoint{2.246036in}{0.739656in}}%
\pgfpathlineto{\pgfqpoint{2.245738in}{0.739656in}}%
\pgfpathlineto{\pgfqpoint{2.245441in}{0.739656in}}%
\pgfpathlineto{\pgfqpoint{2.245143in}{0.739656in}}%
\pgfpathlineto{\pgfqpoint{2.244846in}{0.739656in}}%
\pgfpathlineto{\pgfqpoint{2.244548in}{0.739656in}}%
\pgfpathlineto{\pgfqpoint{2.244251in}{0.739656in}}%
\pgfpathlineto{\pgfqpoint{2.243953in}{0.739656in}}%
\pgfpathlineto{\pgfqpoint{2.243656in}{0.739656in}}%
\pgfpathlineto{\pgfqpoint{2.243358in}{0.739656in}}%
\pgfpathlineto{\pgfqpoint{2.243061in}{0.739656in}}%
\pgfpathlineto{\pgfqpoint{2.242763in}{0.739656in}}%
\pgfpathlineto{\pgfqpoint{2.242466in}{0.739656in}}%
\pgfpathlineto{\pgfqpoint{2.242168in}{0.739656in}}%
\pgfpathlineto{\pgfqpoint{2.241871in}{0.739656in}}%
\pgfpathlineto{\pgfqpoint{2.241573in}{0.739656in}}%
\pgfpathlineto{\pgfqpoint{2.241276in}{0.739656in}}%
\pgfpathlineto{\pgfqpoint{2.240978in}{0.739656in}}%
\pgfpathlineto{\pgfqpoint{2.240681in}{0.739656in}}%
\pgfpathlineto{\pgfqpoint{2.240383in}{0.739656in}}%
\pgfpathlineto{\pgfqpoint{2.240086in}{0.739656in}}%
\pgfpathlineto{\pgfqpoint{2.239789in}{0.739656in}}%
\pgfpathlineto{\pgfqpoint{2.239491in}{0.739656in}}%
\pgfpathlineto{\pgfqpoint{2.239194in}{0.739656in}}%
\pgfpathlineto{\pgfqpoint{2.238896in}{0.739656in}}%
\pgfpathlineto{\pgfqpoint{2.238599in}{0.739656in}}%
\pgfpathlineto{\pgfqpoint{2.238301in}{0.739656in}}%
\pgfpathlineto{\pgfqpoint{2.238004in}{0.739656in}}%
\pgfpathlineto{\pgfqpoint{2.237706in}{0.739656in}}%
\pgfpathlineto{\pgfqpoint{2.237409in}{0.739656in}}%
\pgfpathlineto{\pgfqpoint{2.237111in}{0.739656in}}%
\pgfpathlineto{\pgfqpoint{2.236814in}{0.739656in}}%
\pgfpathlineto{\pgfqpoint{2.236516in}{0.739656in}}%
\pgfpathlineto{\pgfqpoint{2.236219in}{0.739656in}}%
\pgfpathlineto{\pgfqpoint{2.235921in}{0.739656in}}%
\pgfpathlineto{\pgfqpoint{2.235624in}{0.739656in}}%
\pgfpathlineto{\pgfqpoint{2.235326in}{0.739656in}}%
\pgfpathlineto{\pgfqpoint{2.235029in}{0.739656in}}%
\pgfpathlineto{\pgfqpoint{2.234731in}{0.739656in}}%
\pgfpathlineto{\pgfqpoint{2.234434in}{0.739656in}}%
\pgfpathlineto{\pgfqpoint{2.234136in}{0.739656in}}%
\pgfpathlineto{\pgfqpoint{2.233839in}{0.739656in}}%
\pgfpathlineto{\pgfqpoint{2.233541in}{0.739656in}}%
\pgfpathlineto{\pgfqpoint{2.233244in}{0.739656in}}%
\pgfpathlineto{\pgfqpoint{2.232947in}{0.739656in}}%
\pgfpathlineto{\pgfqpoint{2.232649in}{0.739656in}}%
\pgfpathlineto{\pgfqpoint{2.232352in}{0.739656in}}%
\pgfpathlineto{\pgfqpoint{2.232054in}{0.739656in}}%
\pgfpathlineto{\pgfqpoint{2.231757in}{0.739656in}}%
\pgfpathlineto{\pgfqpoint{2.231459in}{0.739656in}}%
\pgfpathlineto{\pgfqpoint{2.231162in}{0.739656in}}%
\pgfpathlineto{\pgfqpoint{2.230864in}{0.739656in}}%
\pgfpathlineto{\pgfqpoint{2.230567in}{0.739656in}}%
\pgfpathlineto{\pgfqpoint{2.230269in}{0.739656in}}%
\pgfpathlineto{\pgfqpoint{2.229972in}{0.739656in}}%
\pgfpathlineto{\pgfqpoint{2.229674in}{0.739656in}}%
\pgfpathlineto{\pgfqpoint{2.229377in}{0.739656in}}%
\pgfpathlineto{\pgfqpoint{2.229079in}{0.739656in}}%
\pgfpathlineto{\pgfqpoint{2.228782in}{0.739656in}}%
\pgfpathlineto{\pgfqpoint{2.228484in}{0.739656in}}%
\pgfpathlineto{\pgfqpoint{2.228187in}{0.739656in}}%
\pgfpathlineto{\pgfqpoint{2.227889in}{0.739656in}}%
\pgfpathlineto{\pgfqpoint{2.227592in}{0.739656in}}%
\pgfpathlineto{\pgfqpoint{2.227294in}{0.739656in}}%
\pgfpathlineto{\pgfqpoint{2.226997in}{0.739656in}}%
\pgfpathlineto{\pgfqpoint{2.226699in}{0.739656in}}%
\pgfpathlineto{\pgfqpoint{2.226402in}{0.739656in}}%
\pgfpathlineto{\pgfqpoint{2.226105in}{0.739656in}}%
\pgfpathlineto{\pgfqpoint{2.225807in}{0.739656in}}%
\pgfpathlineto{\pgfqpoint{2.225510in}{0.739656in}}%
\pgfpathlineto{\pgfqpoint{2.225212in}{0.739656in}}%
\pgfpathlineto{\pgfqpoint{2.224915in}{0.739656in}}%
\pgfpathlineto{\pgfqpoint{2.224617in}{0.739656in}}%
\pgfpathlineto{\pgfqpoint{2.224320in}{0.739656in}}%
\pgfpathlineto{\pgfqpoint{2.224022in}{0.739656in}}%
\pgfpathlineto{\pgfqpoint{2.223725in}{0.739656in}}%
\pgfpathlineto{\pgfqpoint{2.223427in}{0.739656in}}%
\pgfpathlineto{\pgfqpoint{2.223130in}{0.739656in}}%
\pgfpathlineto{\pgfqpoint{2.222832in}{0.739656in}}%
\pgfpathlineto{\pgfqpoint{2.222535in}{0.739656in}}%
\pgfpathlineto{\pgfqpoint{2.222237in}{0.739656in}}%
\pgfpathlineto{\pgfqpoint{2.221940in}{0.739656in}}%
\pgfpathlineto{\pgfqpoint{2.221642in}{0.739656in}}%
\pgfpathlineto{\pgfqpoint{2.221345in}{0.739656in}}%
\pgfpathlineto{\pgfqpoint{2.221047in}{0.739656in}}%
\pgfpathlineto{\pgfqpoint{2.220750in}{0.739656in}}%
\pgfpathlineto{\pgfqpoint{2.220452in}{0.739656in}}%
\pgfpathlineto{\pgfqpoint{2.220155in}{0.739656in}}%
\pgfpathlineto{\pgfqpoint{2.219858in}{0.739656in}}%
\pgfpathlineto{\pgfqpoint{2.219560in}{0.739656in}}%
\pgfpathlineto{\pgfqpoint{2.219263in}{0.739656in}}%
\pgfpathlineto{\pgfqpoint{2.218965in}{0.739656in}}%
\pgfpathlineto{\pgfqpoint{2.218668in}{0.739656in}}%
\pgfpathlineto{\pgfqpoint{2.218370in}{0.739656in}}%
\pgfpathlineto{\pgfqpoint{2.218073in}{0.739656in}}%
\pgfpathlineto{\pgfqpoint{2.217775in}{0.739656in}}%
\pgfpathlineto{\pgfqpoint{2.217478in}{0.739656in}}%
\pgfpathlineto{\pgfqpoint{2.217180in}{0.739656in}}%
\pgfpathlineto{\pgfqpoint{2.216883in}{0.739656in}}%
\pgfpathlineto{\pgfqpoint{2.216585in}{0.739656in}}%
\pgfpathlineto{\pgfqpoint{2.216288in}{0.739656in}}%
\pgfpathlineto{\pgfqpoint{2.215990in}{0.739656in}}%
\pgfpathlineto{\pgfqpoint{2.215693in}{0.739656in}}%
\pgfpathlineto{\pgfqpoint{2.215395in}{0.739656in}}%
\pgfpathlineto{\pgfqpoint{2.215098in}{0.739656in}}%
\pgfpathlineto{\pgfqpoint{2.214800in}{0.739656in}}%
\pgfpathlineto{\pgfqpoint{2.214503in}{0.739656in}}%
\pgfpathlineto{\pgfqpoint{2.214205in}{0.739656in}}%
\pgfpathlineto{\pgfqpoint{2.213908in}{0.739656in}}%
\pgfpathlineto{\pgfqpoint{2.213610in}{0.739656in}}%
\pgfpathlineto{\pgfqpoint{2.213313in}{0.739656in}}%
\pgfpathlineto{\pgfqpoint{2.213016in}{0.739656in}}%
\pgfpathlineto{\pgfqpoint{2.212718in}{0.739656in}}%
\pgfpathlineto{\pgfqpoint{2.212421in}{0.739656in}}%
\pgfpathlineto{\pgfqpoint{2.212123in}{0.739656in}}%
\pgfpathlineto{\pgfqpoint{2.211826in}{0.739656in}}%
\pgfpathlineto{\pgfqpoint{2.211528in}{0.739656in}}%
\pgfpathlineto{\pgfqpoint{2.211231in}{0.739656in}}%
\pgfpathlineto{\pgfqpoint{2.210933in}{0.739656in}}%
\pgfpathlineto{\pgfqpoint{2.210636in}{0.739656in}}%
\pgfpathlineto{\pgfqpoint{2.210338in}{0.739656in}}%
\pgfpathlineto{\pgfqpoint{2.210041in}{0.739656in}}%
\pgfpathlineto{\pgfqpoint{2.209743in}{0.739656in}}%
\pgfpathlineto{\pgfqpoint{2.209446in}{0.739656in}}%
\pgfpathlineto{\pgfqpoint{2.209148in}{0.739656in}}%
\pgfpathlineto{\pgfqpoint{2.208851in}{0.739656in}}%
\pgfpathlineto{\pgfqpoint{2.208553in}{0.739656in}}%
\pgfpathlineto{\pgfqpoint{2.208256in}{0.739656in}}%
\pgfpathlineto{\pgfqpoint{2.207958in}{0.739656in}}%
\pgfpathlineto{\pgfqpoint{2.207661in}{0.739656in}}%
\pgfpathlineto{\pgfqpoint{2.207363in}{0.739656in}}%
\pgfpathlineto{\pgfqpoint{2.207066in}{0.739656in}}%
\pgfpathlineto{\pgfqpoint{2.206768in}{0.739656in}}%
\pgfpathlineto{\pgfqpoint{2.206471in}{0.739656in}}%
\pgfpathlineto{\pgfqpoint{2.206174in}{0.739656in}}%
\pgfpathlineto{\pgfqpoint{2.205876in}{0.739656in}}%
\pgfpathlineto{\pgfqpoint{2.205579in}{0.739656in}}%
\pgfpathlineto{\pgfqpoint{2.205281in}{0.739656in}}%
\pgfpathlineto{\pgfqpoint{2.204984in}{0.739656in}}%
\pgfpathlineto{\pgfqpoint{2.204686in}{0.739656in}}%
\pgfpathlineto{\pgfqpoint{2.204389in}{0.739656in}}%
\pgfpathlineto{\pgfqpoint{2.204091in}{0.739656in}}%
\pgfpathlineto{\pgfqpoint{2.203794in}{0.739656in}}%
\pgfpathlineto{\pgfqpoint{2.203496in}{0.739656in}}%
\pgfpathlineto{\pgfqpoint{2.203199in}{0.739656in}}%
\pgfpathlineto{\pgfqpoint{2.202901in}{0.739656in}}%
\pgfpathlineto{\pgfqpoint{2.202604in}{0.739656in}}%
\pgfpathlineto{\pgfqpoint{2.202306in}{0.739656in}}%
\pgfpathlineto{\pgfqpoint{2.202009in}{0.739656in}}%
\pgfpathlineto{\pgfqpoint{2.201711in}{0.739656in}}%
\pgfpathlineto{\pgfqpoint{2.201414in}{0.739656in}}%
\pgfpathlineto{\pgfqpoint{2.201116in}{0.739656in}}%
\pgfpathlineto{\pgfqpoint{2.200819in}{0.739656in}}%
\pgfpathlineto{\pgfqpoint{2.200521in}{0.739656in}}%
\pgfpathlineto{\pgfqpoint{2.200224in}{0.739656in}}%
\pgfpathlineto{\pgfqpoint{2.199927in}{0.739656in}}%
\pgfpathlineto{\pgfqpoint{2.199629in}{0.739656in}}%
\pgfpathlineto{\pgfqpoint{2.199332in}{0.739656in}}%
\pgfpathlineto{\pgfqpoint{2.199034in}{0.739656in}}%
\pgfpathlineto{\pgfqpoint{2.198737in}{0.739656in}}%
\pgfpathlineto{\pgfqpoint{2.198439in}{0.739656in}}%
\pgfpathlineto{\pgfqpoint{2.198142in}{0.739656in}}%
\pgfpathlineto{\pgfqpoint{2.197844in}{0.739656in}}%
\pgfpathlineto{\pgfqpoint{2.197547in}{0.739656in}}%
\pgfpathlineto{\pgfqpoint{2.197249in}{0.739656in}}%
\pgfpathlineto{\pgfqpoint{2.196952in}{0.739656in}}%
\pgfpathlineto{\pgfqpoint{2.196654in}{0.739656in}}%
\pgfpathlineto{\pgfqpoint{2.196357in}{0.739656in}}%
\pgfpathlineto{\pgfqpoint{2.196059in}{0.739656in}}%
\pgfpathlineto{\pgfqpoint{2.195762in}{0.739656in}}%
\pgfpathlineto{\pgfqpoint{2.195464in}{0.739656in}}%
\pgfpathlineto{\pgfqpoint{2.195167in}{0.739656in}}%
\pgfpathlineto{\pgfqpoint{2.194869in}{0.739656in}}%
\pgfpathlineto{\pgfqpoint{2.194572in}{0.739656in}}%
\pgfpathlineto{\pgfqpoint{2.194274in}{0.739656in}}%
\pgfpathlineto{\pgfqpoint{2.193977in}{0.739656in}}%
\pgfpathlineto{\pgfqpoint{2.193679in}{0.739656in}}%
\pgfpathlineto{\pgfqpoint{2.193382in}{0.739656in}}%
\pgfpathlineto{\pgfqpoint{2.193085in}{0.739656in}}%
\pgfpathlineto{\pgfqpoint{2.192787in}{0.739656in}}%
\pgfpathlineto{\pgfqpoint{2.192490in}{0.739656in}}%
\pgfpathlineto{\pgfqpoint{2.192192in}{0.739656in}}%
\pgfpathlineto{\pgfqpoint{2.191895in}{0.739656in}}%
\pgfpathlineto{\pgfqpoint{2.191597in}{0.739656in}}%
\pgfpathlineto{\pgfqpoint{2.191300in}{0.739656in}}%
\pgfpathlineto{\pgfqpoint{2.191002in}{0.739656in}}%
\pgfpathlineto{\pgfqpoint{2.190705in}{0.739656in}}%
\pgfpathlineto{\pgfqpoint{2.190407in}{0.739656in}}%
\pgfpathlineto{\pgfqpoint{2.190110in}{0.739656in}}%
\pgfpathlineto{\pgfqpoint{2.189812in}{0.739656in}}%
\pgfpathlineto{\pgfqpoint{2.189515in}{0.739656in}}%
\pgfpathlineto{\pgfqpoint{2.189217in}{0.739656in}}%
\pgfpathlineto{\pgfqpoint{2.188920in}{0.739656in}}%
\pgfpathlineto{\pgfqpoint{2.188622in}{0.739656in}}%
\pgfpathlineto{\pgfqpoint{2.188325in}{0.739656in}}%
\pgfpathlineto{\pgfqpoint{2.188027in}{0.739656in}}%
\pgfpathlineto{\pgfqpoint{2.187730in}{0.739656in}}%
\pgfpathlineto{\pgfqpoint{2.187432in}{0.739656in}}%
\pgfpathlineto{\pgfqpoint{2.187135in}{0.739656in}}%
\pgfpathlineto{\pgfqpoint{2.186837in}{0.739656in}}%
\pgfpathlineto{\pgfqpoint{2.186540in}{0.739656in}}%
\pgfpathlineto{\pgfqpoint{2.186243in}{0.739656in}}%
\pgfpathlineto{\pgfqpoint{2.185945in}{0.739656in}}%
\pgfpathlineto{\pgfqpoint{2.185648in}{0.739656in}}%
\pgfpathlineto{\pgfqpoint{2.185350in}{0.739656in}}%
\pgfpathlineto{\pgfqpoint{2.185053in}{0.739656in}}%
\pgfpathlineto{\pgfqpoint{2.184755in}{0.739656in}}%
\pgfpathlineto{\pgfqpoint{2.184458in}{0.739656in}}%
\pgfpathlineto{\pgfqpoint{2.184160in}{0.739656in}}%
\pgfpathlineto{\pgfqpoint{2.183863in}{0.739656in}}%
\pgfpathlineto{\pgfqpoint{2.183565in}{0.739656in}}%
\pgfpathlineto{\pgfqpoint{2.183268in}{0.739656in}}%
\pgfpathlineto{\pgfqpoint{2.182970in}{0.739656in}}%
\pgfpathlineto{\pgfqpoint{2.182673in}{0.739656in}}%
\pgfpathlineto{\pgfqpoint{2.182375in}{0.739656in}}%
\pgfpathlineto{\pgfqpoint{2.182078in}{0.739656in}}%
\pgfpathlineto{\pgfqpoint{2.181780in}{0.739656in}}%
\pgfpathlineto{\pgfqpoint{2.181483in}{0.739656in}}%
\pgfpathlineto{\pgfqpoint{2.181185in}{0.739656in}}%
\pgfpathlineto{\pgfqpoint{2.180888in}{0.739656in}}%
\pgfpathlineto{\pgfqpoint{2.180590in}{0.739656in}}%
\pgfpathlineto{\pgfqpoint{2.180293in}{0.739656in}}%
\pgfpathlineto{\pgfqpoint{2.179996in}{0.739656in}}%
\pgfpathlineto{\pgfqpoint{2.179698in}{0.739656in}}%
\pgfpathlineto{\pgfqpoint{2.179401in}{0.739656in}}%
\pgfpathlineto{\pgfqpoint{2.179103in}{0.739656in}}%
\pgfpathlineto{\pgfqpoint{2.178806in}{0.739656in}}%
\pgfpathlineto{\pgfqpoint{2.178508in}{0.739656in}}%
\pgfpathlineto{\pgfqpoint{2.178211in}{0.739656in}}%
\pgfpathlineto{\pgfqpoint{2.177913in}{0.739656in}}%
\pgfpathlineto{\pgfqpoint{2.177616in}{0.739656in}}%
\pgfpathlineto{\pgfqpoint{2.177318in}{0.739656in}}%
\pgfpathlineto{\pgfqpoint{2.177021in}{0.739656in}}%
\pgfpathlineto{\pgfqpoint{2.176723in}{0.739656in}}%
\pgfpathlineto{\pgfqpoint{2.176426in}{0.739656in}}%
\pgfpathlineto{\pgfqpoint{2.176128in}{0.739656in}}%
\pgfpathlineto{\pgfqpoint{2.175831in}{0.739656in}}%
\pgfpathlineto{\pgfqpoint{2.175533in}{0.739656in}}%
\pgfpathlineto{\pgfqpoint{2.175236in}{0.739656in}}%
\pgfpathlineto{\pgfqpoint{2.174938in}{0.739656in}}%
\pgfpathlineto{\pgfqpoint{2.174641in}{0.739656in}}%
\pgfpathlineto{\pgfqpoint{2.174343in}{0.739656in}}%
\pgfpathlineto{\pgfqpoint{2.174046in}{0.739656in}}%
\pgfpathlineto{\pgfqpoint{2.173748in}{0.739656in}}%
\pgfpathlineto{\pgfqpoint{2.173451in}{0.739656in}}%
\pgfpathlineto{\pgfqpoint{2.173154in}{0.739656in}}%
\pgfpathlineto{\pgfqpoint{2.172856in}{0.739656in}}%
\pgfpathlineto{\pgfqpoint{2.172559in}{0.739656in}}%
\pgfpathlineto{\pgfqpoint{2.172261in}{0.739656in}}%
\pgfpathlineto{\pgfqpoint{2.171964in}{0.739656in}}%
\pgfpathlineto{\pgfqpoint{2.171666in}{0.739656in}}%
\pgfpathlineto{\pgfqpoint{2.171369in}{0.739656in}}%
\pgfpathlineto{\pgfqpoint{2.171071in}{0.739656in}}%
\pgfpathlineto{\pgfqpoint{2.170774in}{0.739656in}}%
\pgfpathlineto{\pgfqpoint{2.170476in}{0.739656in}}%
\pgfpathlineto{\pgfqpoint{2.170179in}{0.739656in}}%
\pgfpathlineto{\pgfqpoint{2.169881in}{0.739656in}}%
\pgfpathlineto{\pgfqpoint{2.169584in}{0.739656in}}%
\pgfpathlineto{\pgfqpoint{2.169286in}{0.739656in}}%
\pgfpathlineto{\pgfqpoint{2.168989in}{0.739656in}}%
\pgfpathlineto{\pgfqpoint{2.168691in}{0.739656in}}%
\pgfpathlineto{\pgfqpoint{2.168394in}{0.739656in}}%
\pgfpathlineto{\pgfqpoint{2.168096in}{0.739656in}}%
\pgfpathlineto{\pgfqpoint{2.167799in}{0.739656in}}%
\pgfpathlineto{\pgfqpoint{2.167501in}{0.739656in}}%
\pgfpathlineto{\pgfqpoint{2.167204in}{0.739656in}}%
\pgfpathlineto{\pgfqpoint{2.166906in}{0.739656in}}%
\pgfpathlineto{\pgfqpoint{2.166609in}{0.739656in}}%
\pgfpathlineto{\pgfqpoint{2.166312in}{0.739656in}}%
\pgfpathlineto{\pgfqpoint{2.166014in}{0.739656in}}%
\pgfpathlineto{\pgfqpoint{2.165717in}{0.739656in}}%
\pgfpathlineto{\pgfqpoint{2.165419in}{0.739656in}}%
\pgfpathlineto{\pgfqpoint{2.165122in}{0.739656in}}%
\pgfpathlineto{\pgfqpoint{2.164824in}{0.739656in}}%
\pgfpathlineto{\pgfqpoint{2.164527in}{0.739656in}}%
\pgfpathlineto{\pgfqpoint{2.164229in}{0.739656in}}%
\pgfpathlineto{\pgfqpoint{2.163932in}{0.739656in}}%
\pgfpathlineto{\pgfqpoint{2.163634in}{0.739656in}}%
\pgfpathlineto{\pgfqpoint{2.163337in}{0.739656in}}%
\pgfpathlineto{\pgfqpoint{2.163039in}{0.739656in}}%
\pgfpathlineto{\pgfqpoint{2.162742in}{0.739656in}}%
\pgfpathlineto{\pgfqpoint{2.162444in}{0.739656in}}%
\pgfpathlineto{\pgfqpoint{2.162147in}{0.739656in}}%
\pgfpathlineto{\pgfqpoint{2.161849in}{0.739656in}}%
\pgfpathlineto{\pgfqpoint{2.161552in}{0.739656in}}%
\pgfpathlineto{\pgfqpoint{2.161254in}{0.739656in}}%
\pgfpathlineto{\pgfqpoint{2.160957in}{0.739656in}}%
\pgfpathlineto{\pgfqpoint{2.160659in}{0.739656in}}%
\pgfpathlineto{\pgfqpoint{2.160362in}{0.739656in}}%
\pgfpathlineto{\pgfqpoint{2.160065in}{0.739656in}}%
\pgfpathlineto{\pgfqpoint{2.159767in}{0.739656in}}%
\pgfpathlineto{\pgfqpoint{2.159470in}{0.739656in}}%
\pgfpathlineto{\pgfqpoint{2.159172in}{0.739656in}}%
\pgfpathlineto{\pgfqpoint{2.158875in}{0.739656in}}%
\pgfpathlineto{\pgfqpoint{2.158577in}{0.739656in}}%
\pgfpathlineto{\pgfqpoint{2.158280in}{0.739656in}}%
\pgfpathlineto{\pgfqpoint{2.157982in}{0.739656in}}%
\pgfpathlineto{\pgfqpoint{2.157685in}{0.739656in}}%
\pgfpathlineto{\pgfqpoint{2.157387in}{0.739656in}}%
\pgfpathlineto{\pgfqpoint{2.157090in}{0.739656in}}%
\pgfpathlineto{\pgfqpoint{2.156792in}{0.739656in}}%
\pgfpathlineto{\pgfqpoint{2.156495in}{0.739656in}}%
\pgfpathlineto{\pgfqpoint{2.156197in}{0.739656in}}%
\pgfpathlineto{\pgfqpoint{2.155900in}{0.739656in}}%
\pgfpathlineto{\pgfqpoint{2.155602in}{0.739656in}}%
\pgfpathlineto{\pgfqpoint{2.155305in}{0.739656in}}%
\pgfpathlineto{\pgfqpoint{2.155007in}{0.739656in}}%
\pgfpathlineto{\pgfqpoint{2.154710in}{0.739656in}}%
\pgfpathlineto{\pgfqpoint{2.154412in}{0.739656in}}%
\pgfpathlineto{\pgfqpoint{2.154115in}{0.739656in}}%
\pgfpathlineto{\pgfqpoint{2.153817in}{0.739656in}}%
\pgfpathlineto{\pgfqpoint{2.153520in}{0.739656in}}%
\pgfpathlineto{\pgfqpoint{2.153223in}{0.739656in}}%
\pgfpathlineto{\pgfqpoint{2.152925in}{0.739656in}}%
\pgfpathlineto{\pgfqpoint{2.152628in}{0.739656in}}%
\pgfpathlineto{\pgfqpoint{2.152330in}{0.739656in}}%
\pgfpathlineto{\pgfqpoint{2.152033in}{0.739656in}}%
\pgfpathlineto{\pgfqpoint{2.151735in}{0.739656in}}%
\pgfpathlineto{\pgfqpoint{2.151438in}{0.739656in}}%
\pgfpathlineto{\pgfqpoint{2.151140in}{0.739656in}}%
\pgfpathlineto{\pgfqpoint{2.150843in}{0.739656in}}%
\pgfpathlineto{\pgfqpoint{2.150545in}{0.739656in}}%
\pgfpathlineto{\pgfqpoint{2.150248in}{0.739656in}}%
\pgfpathlineto{\pgfqpoint{2.149950in}{0.739656in}}%
\pgfpathlineto{\pgfqpoint{2.149653in}{0.739656in}}%
\pgfpathlineto{\pgfqpoint{2.149355in}{0.739656in}}%
\pgfpathlineto{\pgfqpoint{2.149058in}{0.739656in}}%
\pgfpathlineto{\pgfqpoint{2.148760in}{0.739656in}}%
\pgfpathlineto{\pgfqpoint{2.148463in}{0.739656in}}%
\pgfpathlineto{\pgfqpoint{2.148165in}{0.739656in}}%
\pgfpathlineto{\pgfqpoint{2.147868in}{0.739656in}}%
\pgfpathlineto{\pgfqpoint{2.147570in}{0.739656in}}%
\pgfpathlineto{\pgfqpoint{2.147273in}{0.739656in}}%
\pgfpathlineto{\pgfqpoint{2.146975in}{0.739656in}}%
\pgfpathlineto{\pgfqpoint{2.146678in}{0.739656in}}%
\pgfpathlineto{\pgfqpoint{2.146381in}{0.739656in}}%
\pgfpathlineto{\pgfqpoint{2.146083in}{0.739656in}}%
\pgfpathlineto{\pgfqpoint{2.145786in}{0.739656in}}%
\pgfpathlineto{\pgfqpoint{2.145488in}{0.739656in}}%
\pgfpathlineto{\pgfqpoint{2.145191in}{0.739656in}}%
\pgfpathlineto{\pgfqpoint{2.144893in}{0.739656in}}%
\pgfpathlineto{\pgfqpoint{2.144596in}{0.739656in}}%
\pgfpathlineto{\pgfqpoint{2.144298in}{0.739656in}}%
\pgfpathlineto{\pgfqpoint{2.144001in}{0.739656in}}%
\pgfpathlineto{\pgfqpoint{2.143703in}{0.739656in}}%
\pgfpathlineto{\pgfqpoint{2.143406in}{0.739656in}}%
\pgfpathlineto{\pgfqpoint{2.143108in}{0.739656in}}%
\pgfpathlineto{\pgfqpoint{2.142811in}{0.739656in}}%
\pgfpathlineto{\pgfqpoint{2.142513in}{0.739656in}}%
\pgfpathlineto{\pgfqpoint{2.142216in}{0.739656in}}%
\pgfpathlineto{\pgfqpoint{2.141918in}{0.739656in}}%
\pgfpathlineto{\pgfqpoint{2.141621in}{0.739656in}}%
\pgfpathlineto{\pgfqpoint{2.141323in}{0.739656in}}%
\pgfpathlineto{\pgfqpoint{2.141026in}{0.739656in}}%
\pgfpathlineto{\pgfqpoint{2.140728in}{0.739656in}}%
\pgfpathlineto{\pgfqpoint{2.140431in}{0.739656in}}%
\pgfpathlineto{\pgfqpoint{2.140134in}{0.739656in}}%
\pgfpathlineto{\pgfqpoint{2.139836in}{0.739656in}}%
\pgfpathlineto{\pgfqpoint{2.139539in}{0.739656in}}%
\pgfpathlineto{\pgfqpoint{2.139241in}{0.739656in}}%
\pgfpathlineto{\pgfqpoint{2.138944in}{0.739656in}}%
\pgfpathlineto{\pgfqpoint{2.138646in}{0.739656in}}%
\pgfpathlineto{\pgfqpoint{2.138349in}{0.739656in}}%
\pgfpathlineto{\pgfqpoint{2.138051in}{0.739656in}}%
\pgfpathlineto{\pgfqpoint{2.137754in}{0.739656in}}%
\pgfpathlineto{\pgfqpoint{2.137456in}{0.739656in}}%
\pgfpathlineto{\pgfqpoint{2.137159in}{0.739656in}}%
\pgfpathlineto{\pgfqpoint{2.136861in}{0.739656in}}%
\pgfpathlineto{\pgfqpoint{2.136564in}{0.739656in}}%
\pgfpathlineto{\pgfqpoint{2.136266in}{0.739656in}}%
\pgfpathlineto{\pgfqpoint{2.135969in}{0.739656in}}%
\pgfpathlineto{\pgfqpoint{2.135671in}{0.739656in}}%
\pgfpathlineto{\pgfqpoint{2.135374in}{0.739656in}}%
\pgfpathlineto{\pgfqpoint{2.135076in}{0.739656in}}%
\pgfpathlineto{\pgfqpoint{2.134779in}{0.739656in}}%
\pgfpathlineto{\pgfqpoint{2.134481in}{0.739656in}}%
\pgfpathlineto{\pgfqpoint{2.134184in}{0.739656in}}%
\pgfpathlineto{\pgfqpoint{2.133886in}{0.739656in}}%
\pgfpathlineto{\pgfqpoint{2.133589in}{0.739656in}}%
\pgfpathlineto{\pgfqpoint{2.133292in}{0.739656in}}%
\pgfpathlineto{\pgfqpoint{2.132994in}{0.739656in}}%
\pgfpathlineto{\pgfqpoint{2.132697in}{0.739656in}}%
\pgfpathlineto{\pgfqpoint{2.132399in}{0.739656in}}%
\pgfpathlineto{\pgfqpoint{2.132102in}{0.739656in}}%
\pgfpathlineto{\pgfqpoint{2.131804in}{0.739656in}}%
\pgfpathlineto{\pgfqpoint{2.131507in}{0.739656in}}%
\pgfpathlineto{\pgfqpoint{2.131209in}{0.739656in}}%
\pgfpathlineto{\pgfqpoint{2.130912in}{0.739656in}}%
\pgfpathlineto{\pgfqpoint{2.130614in}{0.739656in}}%
\pgfpathlineto{\pgfqpoint{2.130317in}{0.739656in}}%
\pgfpathlineto{\pgfqpoint{2.130019in}{0.739656in}}%
\pgfpathlineto{\pgfqpoint{2.129722in}{0.739656in}}%
\pgfpathlineto{\pgfqpoint{2.129424in}{0.739656in}}%
\pgfpathlineto{\pgfqpoint{2.129127in}{0.739656in}}%
\pgfpathlineto{\pgfqpoint{2.128829in}{0.739656in}}%
\pgfpathlineto{\pgfqpoint{2.128532in}{0.739656in}}%
\pgfpathlineto{\pgfqpoint{2.128234in}{0.739656in}}%
\pgfpathlineto{\pgfqpoint{2.127937in}{0.739656in}}%
\pgfpathlineto{\pgfqpoint{2.127639in}{0.739656in}}%
\pgfpathlineto{\pgfqpoint{2.127342in}{0.739656in}}%
\pgfpathlineto{\pgfqpoint{2.127044in}{0.739656in}}%
\pgfpathlineto{\pgfqpoint{2.126747in}{0.739656in}}%
\pgfpathlineto{\pgfqpoint{2.126450in}{0.739656in}}%
\pgfpathlineto{\pgfqpoint{2.126152in}{0.739656in}}%
\pgfpathlineto{\pgfqpoint{2.125855in}{0.739656in}}%
\pgfpathlineto{\pgfqpoint{2.125557in}{0.739656in}}%
\pgfpathlineto{\pgfqpoint{2.125260in}{0.739656in}}%
\pgfpathlineto{\pgfqpoint{2.124962in}{0.739656in}}%
\pgfpathlineto{\pgfqpoint{2.124665in}{0.739656in}}%
\pgfpathlineto{\pgfqpoint{2.124367in}{0.739656in}}%
\pgfpathlineto{\pgfqpoint{2.124070in}{0.739656in}}%
\pgfpathlineto{\pgfqpoint{2.123772in}{0.739656in}}%
\pgfpathlineto{\pgfqpoint{2.123475in}{0.739656in}}%
\pgfpathlineto{\pgfqpoint{2.123177in}{0.739656in}}%
\pgfpathlineto{\pgfqpoint{2.122880in}{0.739656in}}%
\pgfpathlineto{\pgfqpoint{2.122582in}{0.739656in}}%
\pgfpathlineto{\pgfqpoint{2.122285in}{0.739656in}}%
\pgfpathlineto{\pgfqpoint{2.121987in}{0.739656in}}%
\pgfpathlineto{\pgfqpoint{2.121690in}{0.739656in}}%
\pgfpathlineto{\pgfqpoint{2.121392in}{0.739656in}}%
\pgfpathlineto{\pgfqpoint{2.121095in}{0.739656in}}%
\pgfpathlineto{\pgfqpoint{2.120797in}{0.739656in}}%
\pgfpathlineto{\pgfqpoint{2.120500in}{0.739656in}}%
\pgfpathlineto{\pgfqpoint{2.120203in}{0.739656in}}%
\pgfpathlineto{\pgfqpoint{2.119905in}{0.739656in}}%
\pgfpathlineto{\pgfqpoint{2.119608in}{0.739656in}}%
\pgfpathlineto{\pgfqpoint{2.119310in}{0.739656in}}%
\pgfpathlineto{\pgfqpoint{2.119013in}{0.739656in}}%
\pgfpathlineto{\pgfqpoint{2.118715in}{0.739656in}}%
\pgfpathlineto{\pgfqpoint{2.118418in}{0.739656in}}%
\pgfpathlineto{\pgfqpoint{2.118120in}{0.739656in}}%
\pgfpathlineto{\pgfqpoint{2.117823in}{0.739656in}}%
\pgfpathlineto{\pgfqpoint{2.117525in}{0.739656in}}%
\pgfpathlineto{\pgfqpoint{2.117228in}{0.739656in}}%
\pgfpathlineto{\pgfqpoint{2.116930in}{0.739656in}}%
\pgfpathlineto{\pgfqpoint{2.116633in}{0.739656in}}%
\pgfpathlineto{\pgfqpoint{2.116335in}{0.739656in}}%
\pgfpathlineto{\pgfqpoint{2.116038in}{0.739656in}}%
\pgfpathlineto{\pgfqpoint{2.115740in}{0.739656in}}%
\pgfpathlineto{\pgfqpoint{2.115443in}{0.739656in}}%
\pgfpathlineto{\pgfqpoint{2.115145in}{0.739656in}}%
\pgfpathlineto{\pgfqpoint{2.114848in}{0.739656in}}%
\pgfpathlineto{\pgfqpoint{2.114550in}{0.739656in}}%
\pgfpathlineto{\pgfqpoint{2.114253in}{0.739656in}}%
\pgfpathlineto{\pgfqpoint{2.113955in}{0.739656in}}%
\pgfpathlineto{\pgfqpoint{2.113658in}{0.739656in}}%
\pgfpathlineto{\pgfqpoint{2.113361in}{0.739656in}}%
\pgfpathlineto{\pgfqpoint{2.113063in}{0.739656in}}%
\pgfpathlineto{\pgfqpoint{2.112766in}{0.739656in}}%
\pgfpathlineto{\pgfqpoint{2.112468in}{0.739656in}}%
\pgfpathlineto{\pgfqpoint{2.112171in}{0.739656in}}%
\pgfpathlineto{\pgfqpoint{2.111873in}{0.739656in}}%
\pgfpathlineto{\pgfqpoint{2.111576in}{0.739656in}}%
\pgfpathlineto{\pgfqpoint{2.111278in}{0.739656in}}%
\pgfpathlineto{\pgfqpoint{2.110981in}{0.739656in}}%
\pgfpathlineto{\pgfqpoint{2.110683in}{0.739656in}}%
\pgfpathlineto{\pgfqpoint{2.110386in}{0.739656in}}%
\pgfpathlineto{\pgfqpoint{2.110088in}{0.739656in}}%
\pgfpathlineto{\pgfqpoint{2.109791in}{0.739656in}}%
\pgfpathlineto{\pgfqpoint{2.109493in}{0.739656in}}%
\pgfpathlineto{\pgfqpoint{2.109196in}{0.739656in}}%
\pgfpathlineto{\pgfqpoint{2.108898in}{0.739656in}}%
\pgfpathlineto{\pgfqpoint{2.108601in}{0.739656in}}%
\pgfpathlineto{\pgfqpoint{2.108303in}{0.739656in}}%
\pgfpathlineto{\pgfqpoint{2.108006in}{0.739656in}}%
\pgfpathlineto{\pgfqpoint{2.107708in}{0.739656in}}%
\pgfpathlineto{\pgfqpoint{2.107411in}{0.739656in}}%
\pgfpathlineto{\pgfqpoint{2.107113in}{0.739656in}}%
\pgfpathlineto{\pgfqpoint{2.106816in}{0.739656in}}%
\pgfpathlineto{\pgfqpoint{2.106519in}{0.739656in}}%
\pgfpathlineto{\pgfqpoint{2.106221in}{0.739656in}}%
\pgfpathlineto{\pgfqpoint{2.105924in}{0.739656in}}%
\pgfpathlineto{\pgfqpoint{2.105626in}{0.739656in}}%
\pgfpathlineto{\pgfqpoint{2.105329in}{0.739656in}}%
\pgfpathlineto{\pgfqpoint{2.105031in}{0.739656in}}%
\pgfpathlineto{\pgfqpoint{2.104734in}{0.739656in}}%
\pgfpathlineto{\pgfqpoint{2.104436in}{0.739656in}}%
\pgfpathlineto{\pgfqpoint{2.104139in}{0.739656in}}%
\pgfpathlineto{\pgfqpoint{2.103841in}{0.739656in}}%
\pgfpathlineto{\pgfqpoint{2.103544in}{0.739656in}}%
\pgfpathlineto{\pgfqpoint{2.103246in}{0.739656in}}%
\pgfpathlineto{\pgfqpoint{2.102949in}{0.739656in}}%
\pgfpathlineto{\pgfqpoint{2.102651in}{0.739656in}}%
\pgfpathlineto{\pgfqpoint{2.102354in}{0.739656in}}%
\pgfpathlineto{\pgfqpoint{2.102056in}{0.739656in}}%
\pgfpathlineto{\pgfqpoint{2.101759in}{0.739656in}}%
\pgfpathlineto{\pgfqpoint{2.101461in}{0.739656in}}%
\pgfpathlineto{\pgfqpoint{2.101164in}{0.739656in}}%
\pgfpathlineto{\pgfqpoint{2.100866in}{0.739656in}}%
\pgfpathlineto{\pgfqpoint{2.100569in}{0.739656in}}%
\pgfpathlineto{\pgfqpoint{2.100272in}{0.739656in}}%
\pgfpathlineto{\pgfqpoint{2.099974in}{0.739656in}}%
\pgfpathlineto{\pgfqpoint{2.099677in}{0.739656in}}%
\pgfpathlineto{\pgfqpoint{2.099379in}{0.739656in}}%
\pgfpathlineto{\pgfqpoint{2.099082in}{0.739656in}}%
\pgfpathlineto{\pgfqpoint{2.098784in}{0.739656in}}%
\pgfpathlineto{\pgfqpoint{2.098487in}{0.739656in}}%
\pgfpathlineto{\pgfqpoint{2.098189in}{0.739656in}}%
\pgfpathlineto{\pgfqpoint{2.097892in}{0.739656in}}%
\pgfpathlineto{\pgfqpoint{2.097594in}{0.739656in}}%
\pgfpathlineto{\pgfqpoint{2.097297in}{0.739656in}}%
\pgfpathlineto{\pgfqpoint{2.096999in}{0.739656in}}%
\pgfpathlineto{\pgfqpoint{2.096702in}{0.739656in}}%
\pgfpathlineto{\pgfqpoint{2.096404in}{0.739656in}}%
\pgfpathlineto{\pgfqpoint{2.096107in}{0.739656in}}%
\pgfpathlineto{\pgfqpoint{2.095809in}{0.739656in}}%
\pgfpathlineto{\pgfqpoint{2.095512in}{0.739656in}}%
\pgfpathlineto{\pgfqpoint{2.095214in}{0.739656in}}%
\pgfpathlineto{\pgfqpoint{2.094917in}{0.739656in}}%
\pgfpathlineto{\pgfqpoint{2.094619in}{0.739656in}}%
\pgfpathlineto{\pgfqpoint{2.094322in}{0.739656in}}%
\pgfpathlineto{\pgfqpoint{2.094024in}{0.739656in}}%
\pgfpathlineto{\pgfqpoint{2.093727in}{0.739656in}}%
\pgfpathlineto{\pgfqpoint{2.093430in}{0.739656in}}%
\pgfpathlineto{\pgfqpoint{2.093132in}{0.739656in}}%
\pgfpathlineto{\pgfqpoint{2.092835in}{0.739656in}}%
\pgfpathlineto{\pgfqpoint{2.092537in}{0.739656in}}%
\pgfpathlineto{\pgfqpoint{2.092240in}{0.739656in}}%
\pgfpathlineto{\pgfqpoint{2.091942in}{0.739656in}}%
\pgfpathlineto{\pgfqpoint{2.091645in}{0.739656in}}%
\pgfpathlineto{\pgfqpoint{2.091347in}{0.739656in}}%
\pgfpathlineto{\pgfqpoint{2.091050in}{0.739656in}}%
\pgfpathlineto{\pgfqpoint{2.090752in}{0.739656in}}%
\pgfpathlineto{\pgfqpoint{2.090455in}{0.739656in}}%
\pgfpathlineto{\pgfqpoint{2.090157in}{0.739656in}}%
\pgfpathlineto{\pgfqpoint{2.089860in}{0.739656in}}%
\pgfpathlineto{\pgfqpoint{2.089562in}{0.739656in}}%
\pgfpathlineto{\pgfqpoint{2.089265in}{0.739656in}}%
\pgfpathlineto{\pgfqpoint{2.088967in}{0.739656in}}%
\pgfpathlineto{\pgfqpoint{2.088670in}{0.739656in}}%
\pgfpathlineto{\pgfqpoint{2.088372in}{0.739656in}}%
\pgfpathlineto{\pgfqpoint{2.088075in}{0.739656in}}%
\pgfpathlineto{\pgfqpoint{2.087777in}{0.739656in}}%
\pgfpathlineto{\pgfqpoint{2.087480in}{0.739656in}}%
\pgfpathlineto{\pgfqpoint{2.087182in}{0.739656in}}%
\pgfpathlineto{\pgfqpoint{2.086885in}{0.739656in}}%
\pgfpathlineto{\pgfqpoint{2.086588in}{0.739656in}}%
\pgfpathlineto{\pgfqpoint{2.086290in}{0.739656in}}%
\pgfpathlineto{\pgfqpoint{2.085993in}{0.739656in}}%
\pgfpathlineto{\pgfqpoint{2.085695in}{0.739656in}}%
\pgfpathlineto{\pgfqpoint{2.085398in}{0.739656in}}%
\pgfpathlineto{\pgfqpoint{2.085100in}{0.739656in}}%
\pgfpathlineto{\pgfqpoint{2.084803in}{0.739656in}}%
\pgfpathlineto{\pgfqpoint{2.084505in}{0.739656in}}%
\pgfpathlineto{\pgfqpoint{2.084208in}{0.739656in}}%
\pgfpathlineto{\pgfqpoint{2.083910in}{0.739656in}}%
\pgfpathlineto{\pgfqpoint{2.083613in}{0.739656in}}%
\pgfpathlineto{\pgfqpoint{2.083315in}{0.739656in}}%
\pgfpathlineto{\pgfqpoint{2.083018in}{0.739656in}}%
\pgfpathlineto{\pgfqpoint{2.082720in}{0.739656in}}%
\pgfpathlineto{\pgfqpoint{2.082423in}{0.739656in}}%
\pgfpathlineto{\pgfqpoint{2.082125in}{0.739656in}}%
\pgfpathlineto{\pgfqpoint{2.081828in}{0.739656in}}%
\pgfpathlineto{\pgfqpoint{2.081530in}{0.739656in}}%
\pgfpathlineto{\pgfqpoint{2.081233in}{0.739656in}}%
\pgfpathlineto{\pgfqpoint{2.080935in}{0.739656in}}%
\pgfpathlineto{\pgfqpoint{2.080638in}{0.739656in}}%
\pgfpathlineto{\pgfqpoint{2.080341in}{0.739656in}}%
\pgfpathlineto{\pgfqpoint{2.080043in}{0.739656in}}%
\pgfpathlineto{\pgfqpoint{2.079746in}{0.739656in}}%
\pgfpathlineto{\pgfqpoint{2.079448in}{0.739656in}}%
\pgfpathlineto{\pgfqpoint{2.079151in}{0.739656in}}%
\pgfpathlineto{\pgfqpoint{2.078853in}{0.739656in}}%
\pgfpathlineto{\pgfqpoint{2.078556in}{0.739656in}}%
\pgfpathlineto{\pgfqpoint{2.078258in}{0.739656in}}%
\pgfpathlineto{\pgfqpoint{2.077961in}{0.739656in}}%
\pgfpathlineto{\pgfqpoint{2.077663in}{0.739656in}}%
\pgfpathlineto{\pgfqpoint{2.077366in}{0.739656in}}%
\pgfpathlineto{\pgfqpoint{2.077068in}{0.739656in}}%
\pgfpathlineto{\pgfqpoint{2.076771in}{0.739656in}}%
\pgfpathlineto{\pgfqpoint{2.076473in}{0.739656in}}%
\pgfpathlineto{\pgfqpoint{2.076176in}{0.739656in}}%
\pgfpathlineto{\pgfqpoint{2.075878in}{0.739656in}}%
\pgfpathlineto{\pgfqpoint{2.075581in}{0.739656in}}%
\pgfpathlineto{\pgfqpoint{2.075283in}{0.739656in}}%
\pgfpathlineto{\pgfqpoint{2.074986in}{0.739656in}}%
\pgfpathlineto{\pgfqpoint{2.074688in}{0.739656in}}%
\pgfpathlineto{\pgfqpoint{2.074391in}{0.739656in}}%
\pgfpathlineto{\pgfqpoint{2.074093in}{0.739656in}}%
\pgfpathlineto{\pgfqpoint{2.073796in}{0.739656in}}%
\pgfpathlineto{\pgfqpoint{2.073499in}{0.739656in}}%
\pgfpathlineto{\pgfqpoint{2.073201in}{0.739656in}}%
\pgfpathlineto{\pgfqpoint{2.072904in}{0.739656in}}%
\pgfpathlineto{\pgfqpoint{2.072606in}{0.739656in}}%
\pgfpathlineto{\pgfqpoint{2.072309in}{0.739656in}}%
\pgfpathlineto{\pgfqpoint{2.072011in}{0.739656in}}%
\pgfpathlineto{\pgfqpoint{2.071714in}{0.739656in}}%
\pgfpathlineto{\pgfqpoint{2.071416in}{0.739656in}}%
\pgfpathlineto{\pgfqpoint{2.071119in}{0.739656in}}%
\pgfpathlineto{\pgfqpoint{2.070821in}{0.739656in}}%
\pgfpathlineto{\pgfqpoint{2.070524in}{0.739656in}}%
\pgfpathlineto{\pgfqpoint{2.070226in}{0.739656in}}%
\pgfpathlineto{\pgfqpoint{2.069929in}{0.739656in}}%
\pgfpathlineto{\pgfqpoint{2.069631in}{0.739656in}}%
\pgfpathlineto{\pgfqpoint{2.069334in}{0.739656in}}%
\pgfpathlineto{\pgfqpoint{2.069036in}{0.739656in}}%
\pgfpathlineto{\pgfqpoint{2.068739in}{0.739656in}}%
\pgfpathlineto{\pgfqpoint{2.068441in}{0.739656in}}%
\pgfpathlineto{\pgfqpoint{2.068144in}{0.739656in}}%
\pgfpathlineto{\pgfqpoint{2.067846in}{0.739656in}}%
\pgfpathlineto{\pgfqpoint{2.067549in}{0.739656in}}%
\pgfpathlineto{\pgfqpoint{2.067251in}{0.739656in}}%
\pgfpathlineto{\pgfqpoint{2.066954in}{0.739656in}}%
\pgfpathlineto{\pgfqpoint{2.066657in}{0.739656in}}%
\pgfpathlineto{\pgfqpoint{2.066359in}{0.739656in}}%
\pgfpathlineto{\pgfqpoint{2.066062in}{0.739656in}}%
\pgfpathlineto{\pgfqpoint{2.065764in}{0.739656in}}%
\pgfpathlineto{\pgfqpoint{2.065467in}{0.739656in}}%
\pgfpathlineto{\pgfqpoint{2.065169in}{0.739656in}}%
\pgfpathlineto{\pgfqpoint{2.064872in}{0.739656in}}%
\pgfpathlineto{\pgfqpoint{2.064574in}{0.739656in}}%
\pgfpathlineto{\pgfqpoint{2.064277in}{0.739656in}}%
\pgfpathlineto{\pgfqpoint{2.063979in}{0.739656in}}%
\pgfpathlineto{\pgfqpoint{2.063682in}{0.739656in}}%
\pgfpathlineto{\pgfqpoint{2.063384in}{0.739656in}}%
\pgfpathlineto{\pgfqpoint{2.063087in}{0.739656in}}%
\pgfpathlineto{\pgfqpoint{2.062789in}{0.739656in}}%
\pgfpathlineto{\pgfqpoint{2.062492in}{0.739656in}}%
\pgfpathlineto{\pgfqpoint{2.062194in}{0.739656in}}%
\pgfpathlineto{\pgfqpoint{2.061897in}{0.739656in}}%
\pgfpathlineto{\pgfqpoint{2.061599in}{0.739656in}}%
\pgfpathlineto{\pgfqpoint{2.061302in}{0.739656in}}%
\pgfpathlineto{\pgfqpoint{2.061004in}{0.739656in}}%
\pgfpathlineto{\pgfqpoint{2.060707in}{0.739656in}}%
\pgfpathlineto{\pgfqpoint{2.060410in}{0.739656in}}%
\pgfpathlineto{\pgfqpoint{2.060112in}{0.739656in}}%
\pgfpathlineto{\pgfqpoint{2.059815in}{0.739656in}}%
\pgfpathlineto{\pgfqpoint{2.059517in}{0.739656in}}%
\pgfpathlineto{\pgfqpoint{2.059220in}{0.739656in}}%
\pgfpathlineto{\pgfqpoint{2.058922in}{0.739656in}}%
\pgfpathlineto{\pgfqpoint{2.058625in}{0.739656in}}%
\pgfpathlineto{\pgfqpoint{2.058327in}{0.739656in}}%
\pgfpathlineto{\pgfqpoint{2.058030in}{0.739656in}}%
\pgfpathlineto{\pgfqpoint{2.057732in}{0.739656in}}%
\pgfpathlineto{\pgfqpoint{2.057435in}{0.739656in}}%
\pgfpathlineto{\pgfqpoint{2.057137in}{0.739656in}}%
\pgfpathlineto{\pgfqpoint{2.056840in}{0.739656in}}%
\pgfpathlineto{\pgfqpoint{2.056542in}{0.739656in}}%
\pgfpathlineto{\pgfqpoint{2.056245in}{0.739656in}}%
\pgfpathlineto{\pgfqpoint{2.055947in}{0.739656in}}%
\pgfpathlineto{\pgfqpoint{2.055650in}{0.739656in}}%
\pgfpathlineto{\pgfqpoint{2.055352in}{0.739656in}}%
\pgfpathlineto{\pgfqpoint{2.055055in}{0.739656in}}%
\pgfpathlineto{\pgfqpoint{2.054757in}{0.739656in}}%
\pgfpathlineto{\pgfqpoint{2.054460in}{0.739656in}}%
\pgfpathlineto{\pgfqpoint{2.054162in}{0.739656in}}%
\pgfpathlineto{\pgfqpoint{2.053865in}{0.739656in}}%
\pgfpathlineto{\pgfqpoint{2.053568in}{0.739656in}}%
\pgfpathlineto{\pgfqpoint{2.053270in}{0.739656in}}%
\pgfpathlineto{\pgfqpoint{2.052973in}{0.739656in}}%
\pgfpathlineto{\pgfqpoint{2.052675in}{0.739656in}}%
\pgfpathlineto{\pgfqpoint{2.052378in}{0.739656in}}%
\pgfpathlineto{\pgfqpoint{2.052080in}{0.739656in}}%
\pgfpathlineto{\pgfqpoint{2.051783in}{0.739656in}}%
\pgfpathlineto{\pgfqpoint{2.051485in}{0.739656in}}%
\pgfpathlineto{\pgfqpoint{2.051188in}{0.739656in}}%
\pgfpathlineto{\pgfqpoint{2.050890in}{0.739656in}}%
\pgfpathlineto{\pgfqpoint{2.050593in}{0.739656in}}%
\pgfpathlineto{\pgfqpoint{2.050295in}{0.739656in}}%
\pgfpathlineto{\pgfqpoint{2.049998in}{0.739656in}}%
\pgfpathlineto{\pgfqpoint{2.049700in}{0.739656in}}%
\pgfpathlineto{\pgfqpoint{2.049403in}{0.739656in}}%
\pgfpathlineto{\pgfqpoint{2.049105in}{0.739656in}}%
\pgfpathlineto{\pgfqpoint{2.048808in}{0.739656in}}%
\pgfpathlineto{\pgfqpoint{2.048510in}{0.739656in}}%
\pgfpathlineto{\pgfqpoint{2.048213in}{0.739656in}}%
\pgfpathlineto{\pgfqpoint{2.047915in}{0.739656in}}%
\pgfpathlineto{\pgfqpoint{2.047618in}{0.739656in}}%
\pgfpathlineto{\pgfqpoint{2.047320in}{0.739656in}}%
\pgfpathlineto{\pgfqpoint{2.047023in}{0.739656in}}%
\pgfpathlineto{\pgfqpoint{2.046726in}{0.739656in}}%
\pgfpathlineto{\pgfqpoint{2.046428in}{0.739656in}}%
\pgfpathlineto{\pgfqpoint{2.046131in}{0.739656in}}%
\pgfpathlineto{\pgfqpoint{2.045833in}{0.739656in}}%
\pgfpathlineto{\pgfqpoint{2.045536in}{0.739656in}}%
\pgfpathlineto{\pgfqpoint{2.045238in}{0.739656in}}%
\pgfpathlineto{\pgfqpoint{2.044941in}{0.739656in}}%
\pgfpathlineto{\pgfqpoint{2.044643in}{0.739656in}}%
\pgfpathlineto{\pgfqpoint{2.044346in}{0.739656in}}%
\pgfpathlineto{\pgfqpoint{2.044048in}{0.739656in}}%
\pgfpathlineto{\pgfqpoint{2.043751in}{0.739656in}}%
\pgfpathlineto{\pgfqpoint{2.043453in}{0.739656in}}%
\pgfpathlineto{\pgfqpoint{2.043156in}{0.739656in}}%
\pgfpathlineto{\pgfqpoint{2.042858in}{0.739656in}}%
\pgfpathlineto{\pgfqpoint{2.042561in}{0.739656in}}%
\pgfpathlineto{\pgfqpoint{2.042263in}{0.739656in}}%
\pgfpathlineto{\pgfqpoint{2.041966in}{0.739656in}}%
\pgfpathlineto{\pgfqpoint{2.041668in}{0.739656in}}%
\pgfpathlineto{\pgfqpoint{2.041371in}{0.739656in}}%
\pgfpathlineto{\pgfqpoint{2.041073in}{0.739656in}}%
\pgfpathlineto{\pgfqpoint{2.040776in}{0.739656in}}%
\pgfpathlineto{\pgfqpoint{2.040479in}{0.739656in}}%
\pgfpathlineto{\pgfqpoint{2.040181in}{0.739656in}}%
\pgfpathlineto{\pgfqpoint{2.039884in}{0.739656in}}%
\pgfpathlineto{\pgfqpoint{2.039586in}{0.739656in}}%
\pgfpathlineto{\pgfqpoint{2.039289in}{0.739656in}}%
\pgfpathlineto{\pgfqpoint{2.038991in}{0.739656in}}%
\pgfpathlineto{\pgfqpoint{2.038694in}{0.739656in}}%
\pgfpathlineto{\pgfqpoint{2.038396in}{0.739656in}}%
\pgfpathlineto{\pgfqpoint{2.038099in}{0.739656in}}%
\pgfpathlineto{\pgfqpoint{2.037801in}{0.739656in}}%
\pgfpathlineto{\pgfqpoint{2.037504in}{0.739656in}}%
\pgfpathlineto{\pgfqpoint{2.037206in}{0.739656in}}%
\pgfpathlineto{\pgfqpoint{2.036909in}{0.739656in}}%
\pgfpathlineto{\pgfqpoint{2.036611in}{0.739656in}}%
\pgfpathlineto{\pgfqpoint{2.036314in}{0.739656in}}%
\pgfpathlineto{\pgfqpoint{2.036016in}{0.739656in}}%
\pgfpathlineto{\pgfqpoint{2.035719in}{0.739656in}}%
\pgfpathlineto{\pgfqpoint{2.035421in}{0.739656in}}%
\pgfpathlineto{\pgfqpoint{2.035124in}{0.739656in}}%
\pgfpathlineto{\pgfqpoint{2.034826in}{0.739656in}}%
\pgfpathlineto{\pgfqpoint{2.034529in}{0.739656in}}%
\pgfpathlineto{\pgfqpoint{2.034231in}{0.739656in}}%
\pgfpathlineto{\pgfqpoint{2.033934in}{0.739656in}}%
\pgfpathlineto{\pgfqpoint{2.033637in}{0.739656in}}%
\pgfpathlineto{\pgfqpoint{2.033339in}{0.739656in}}%
\pgfpathlineto{\pgfqpoint{2.033042in}{0.739656in}}%
\pgfpathlineto{\pgfqpoint{2.032744in}{0.739656in}}%
\pgfpathlineto{\pgfqpoint{2.032447in}{0.739656in}}%
\pgfpathlineto{\pgfqpoint{2.032149in}{0.739656in}}%
\pgfpathlineto{\pgfqpoint{2.031852in}{0.739656in}}%
\pgfpathlineto{\pgfqpoint{2.031554in}{0.739656in}}%
\pgfpathlineto{\pgfqpoint{2.031257in}{0.739656in}}%
\pgfpathlineto{\pgfqpoint{2.030959in}{0.739656in}}%
\pgfpathlineto{\pgfqpoint{2.030662in}{0.739656in}}%
\pgfpathlineto{\pgfqpoint{2.030364in}{0.739656in}}%
\pgfpathlineto{\pgfqpoint{2.030067in}{0.739656in}}%
\pgfpathlineto{\pgfqpoint{2.029769in}{0.739656in}}%
\pgfpathlineto{\pgfqpoint{2.029472in}{0.739656in}}%
\pgfpathlineto{\pgfqpoint{2.029174in}{0.739656in}}%
\pgfpathlineto{\pgfqpoint{2.028877in}{0.739656in}}%
\pgfpathlineto{\pgfqpoint{2.028579in}{0.739656in}}%
\pgfpathlineto{\pgfqpoint{2.028282in}{0.739656in}}%
\pgfpathlineto{\pgfqpoint{2.027984in}{0.739656in}}%
\pgfpathlineto{\pgfqpoint{2.027687in}{0.739656in}}%
\pgfpathlineto{\pgfqpoint{2.027389in}{0.739656in}}%
\pgfpathlineto{\pgfqpoint{2.027092in}{0.739656in}}%
\pgfpathlineto{\pgfqpoint{2.026795in}{0.739656in}}%
\pgfpathlineto{\pgfqpoint{2.026497in}{0.739656in}}%
\pgfpathlineto{\pgfqpoint{2.026200in}{0.739656in}}%
\pgfpathlineto{\pgfqpoint{2.025902in}{0.739656in}}%
\pgfpathlineto{\pgfqpoint{2.025605in}{0.739656in}}%
\pgfpathlineto{\pgfqpoint{2.025307in}{0.739656in}}%
\pgfpathlineto{\pgfqpoint{2.025010in}{0.739656in}}%
\pgfpathlineto{\pgfqpoint{2.024712in}{0.739656in}}%
\pgfpathlineto{\pgfqpoint{2.024415in}{0.739656in}}%
\pgfpathlineto{\pgfqpoint{2.024117in}{0.739656in}}%
\pgfpathlineto{\pgfqpoint{2.023820in}{0.739656in}}%
\pgfpathlineto{\pgfqpoint{2.023522in}{0.739656in}}%
\pgfpathlineto{\pgfqpoint{2.023225in}{0.739656in}}%
\pgfpathlineto{\pgfqpoint{2.022927in}{0.739656in}}%
\pgfpathlineto{\pgfqpoint{2.022630in}{0.739656in}}%
\pgfpathlineto{\pgfqpoint{2.022332in}{0.739656in}}%
\pgfpathlineto{\pgfqpoint{2.022035in}{0.739656in}}%
\pgfpathlineto{\pgfqpoint{2.021737in}{0.739656in}}%
\pgfpathlineto{\pgfqpoint{2.021440in}{0.739656in}}%
\pgfpathlineto{\pgfqpoint{2.021142in}{0.739656in}}%
\pgfpathlineto{\pgfqpoint{2.020845in}{0.739656in}}%
\pgfpathlineto{\pgfqpoint{2.020548in}{0.739656in}}%
\pgfpathlineto{\pgfqpoint{2.020250in}{0.739656in}}%
\pgfpathlineto{\pgfqpoint{2.019953in}{0.739656in}}%
\pgfpathlineto{\pgfqpoint{2.019655in}{0.739656in}}%
\pgfpathlineto{\pgfqpoint{2.019358in}{0.739656in}}%
\pgfpathlineto{\pgfqpoint{2.019060in}{0.739656in}}%
\pgfpathlineto{\pgfqpoint{2.018763in}{0.739656in}}%
\pgfpathlineto{\pgfqpoint{2.018465in}{0.739656in}}%
\pgfpathlineto{\pgfqpoint{2.018168in}{0.739656in}}%
\pgfpathlineto{\pgfqpoint{2.017870in}{0.739656in}}%
\pgfpathlineto{\pgfqpoint{2.017573in}{0.739656in}}%
\pgfpathlineto{\pgfqpoint{2.017275in}{0.739656in}}%
\pgfpathlineto{\pgfqpoint{2.016978in}{0.739656in}}%
\pgfpathlineto{\pgfqpoint{2.016680in}{0.739656in}}%
\pgfpathlineto{\pgfqpoint{2.016383in}{0.739656in}}%
\pgfpathlineto{\pgfqpoint{2.016085in}{0.739656in}}%
\pgfpathlineto{\pgfqpoint{2.015788in}{0.739656in}}%
\pgfpathlineto{\pgfqpoint{2.015490in}{0.739656in}}%
\pgfpathlineto{\pgfqpoint{2.015193in}{0.739656in}}%
\pgfpathlineto{\pgfqpoint{2.014895in}{0.739656in}}%
\pgfpathlineto{\pgfqpoint{2.014598in}{0.739656in}}%
\pgfpathlineto{\pgfqpoint{2.014300in}{0.739656in}}%
\pgfpathlineto{\pgfqpoint{2.014003in}{0.739656in}}%
\pgfpathlineto{\pgfqpoint{2.013706in}{0.739656in}}%
\pgfpathlineto{\pgfqpoint{2.013408in}{0.739656in}}%
\pgfpathlineto{\pgfqpoint{2.013111in}{0.739656in}}%
\pgfpathlineto{\pgfqpoint{2.012813in}{0.739656in}}%
\pgfpathlineto{\pgfqpoint{2.012516in}{0.739656in}}%
\pgfpathlineto{\pgfqpoint{2.012218in}{0.739656in}}%
\pgfpathlineto{\pgfqpoint{2.011921in}{0.739656in}}%
\pgfpathlineto{\pgfqpoint{2.011623in}{0.739656in}}%
\pgfpathlineto{\pgfqpoint{2.011326in}{0.739656in}}%
\pgfpathlineto{\pgfqpoint{2.011028in}{0.739656in}}%
\pgfpathlineto{\pgfqpoint{2.010731in}{0.739656in}}%
\pgfpathlineto{\pgfqpoint{2.010433in}{0.739656in}}%
\pgfpathlineto{\pgfqpoint{2.010136in}{0.739656in}}%
\pgfpathlineto{\pgfqpoint{2.009838in}{0.739656in}}%
\pgfpathlineto{\pgfqpoint{2.009541in}{0.739656in}}%
\pgfpathlineto{\pgfqpoint{2.009243in}{0.739656in}}%
\pgfpathlineto{\pgfqpoint{2.008946in}{0.739656in}}%
\pgfpathlineto{\pgfqpoint{2.008648in}{0.739656in}}%
\pgfpathlineto{\pgfqpoint{2.008351in}{0.739656in}}%
\pgfpathlineto{\pgfqpoint{2.008053in}{0.739656in}}%
\pgfpathlineto{\pgfqpoint{2.007756in}{0.739656in}}%
\pgfpathlineto{\pgfqpoint{2.007458in}{0.739656in}}%
\pgfpathlineto{\pgfqpoint{2.007161in}{0.739656in}}%
\pgfpathlineto{\pgfqpoint{2.006864in}{0.739656in}}%
\pgfpathlineto{\pgfqpoint{2.006566in}{0.739656in}}%
\pgfpathlineto{\pgfqpoint{2.006269in}{0.739656in}}%
\pgfpathlineto{\pgfqpoint{2.005971in}{0.739656in}}%
\pgfpathlineto{\pgfqpoint{2.005674in}{0.739656in}}%
\pgfpathlineto{\pgfqpoint{2.005376in}{0.739656in}}%
\pgfpathlineto{\pgfqpoint{2.005079in}{0.739656in}}%
\pgfpathlineto{\pgfqpoint{2.004781in}{0.739656in}}%
\pgfpathlineto{\pgfqpoint{2.004484in}{0.739656in}}%
\pgfpathlineto{\pgfqpoint{2.004186in}{0.739656in}}%
\pgfpathlineto{\pgfqpoint{2.003889in}{0.739656in}}%
\pgfpathlineto{\pgfqpoint{2.003591in}{0.739656in}}%
\pgfpathlineto{\pgfqpoint{2.003294in}{0.739656in}}%
\pgfpathlineto{\pgfqpoint{2.002996in}{0.739656in}}%
\pgfpathlineto{\pgfqpoint{2.002699in}{0.739656in}}%
\pgfpathlineto{\pgfqpoint{2.002401in}{0.739656in}}%
\pgfpathlineto{\pgfqpoint{2.002104in}{0.739656in}}%
\pgfpathlineto{\pgfqpoint{2.001806in}{0.739656in}}%
\pgfpathlineto{\pgfqpoint{2.001509in}{0.739656in}}%
\pgfpathlineto{\pgfqpoint{2.001211in}{0.739656in}}%
\pgfpathlineto{\pgfqpoint{2.000914in}{0.739656in}}%
\pgfpathlineto{\pgfqpoint{2.000616in}{0.739656in}}%
\pgfpathlineto{\pgfqpoint{2.000319in}{0.739656in}}%
\pgfpathlineto{\pgfqpoint{2.000022in}{0.739656in}}%
\pgfpathlineto{\pgfqpoint{1.999724in}{0.739656in}}%
\pgfpathlineto{\pgfqpoint{1.999427in}{0.739656in}}%
\pgfpathlineto{\pgfqpoint{1.999129in}{0.739656in}}%
\pgfpathlineto{\pgfqpoint{1.998832in}{0.739656in}}%
\pgfpathlineto{\pgfqpoint{1.998534in}{0.739656in}}%
\pgfpathlineto{\pgfqpoint{1.998237in}{0.739656in}}%
\pgfpathlineto{\pgfqpoint{1.997939in}{0.739656in}}%
\pgfpathlineto{\pgfqpoint{1.997642in}{0.739656in}}%
\pgfpathlineto{\pgfqpoint{1.997344in}{0.739656in}}%
\pgfpathlineto{\pgfqpoint{1.997047in}{0.739656in}}%
\pgfpathlineto{\pgfqpoint{1.996749in}{0.739656in}}%
\pgfpathlineto{\pgfqpoint{1.996452in}{0.739656in}}%
\pgfpathlineto{\pgfqpoint{1.996154in}{0.739656in}}%
\pgfpathlineto{\pgfqpoint{1.995857in}{0.739656in}}%
\pgfpathlineto{\pgfqpoint{1.995559in}{0.739656in}}%
\pgfpathlineto{\pgfqpoint{1.995262in}{0.739656in}}%
\pgfpathlineto{\pgfqpoint{1.994964in}{0.739656in}}%
\pgfpathlineto{\pgfqpoint{1.994667in}{0.739656in}}%
\pgfpathlineto{\pgfqpoint{1.994369in}{0.739656in}}%
\pgfpathlineto{\pgfqpoint{1.994072in}{0.739656in}}%
\pgfpathlineto{\pgfqpoint{1.993775in}{0.739656in}}%
\pgfpathlineto{\pgfqpoint{1.993477in}{0.739656in}}%
\pgfpathlineto{\pgfqpoint{1.993180in}{0.739656in}}%
\pgfpathlineto{\pgfqpoint{1.992882in}{0.739656in}}%
\pgfpathlineto{\pgfqpoint{1.992585in}{0.739656in}}%
\pgfpathlineto{\pgfqpoint{1.992287in}{0.739656in}}%
\pgfpathlineto{\pgfqpoint{1.991990in}{0.739656in}}%
\pgfpathlineto{\pgfqpoint{1.991692in}{0.739656in}}%
\pgfpathlineto{\pgfqpoint{1.991395in}{0.739656in}}%
\pgfpathlineto{\pgfqpoint{1.991097in}{0.739656in}}%
\pgfpathlineto{\pgfqpoint{1.990800in}{0.739656in}}%
\pgfpathlineto{\pgfqpoint{1.990502in}{0.739656in}}%
\pgfpathlineto{\pgfqpoint{1.990205in}{0.739656in}}%
\pgfpathlineto{\pgfqpoint{1.989907in}{0.739656in}}%
\pgfpathlineto{\pgfqpoint{1.989610in}{0.739656in}}%
\pgfpathlineto{\pgfqpoint{1.989312in}{0.739656in}}%
\pgfpathlineto{\pgfqpoint{1.989015in}{0.739656in}}%
\pgfpathlineto{\pgfqpoint{1.988717in}{0.739656in}}%
\pgfpathlineto{\pgfqpoint{1.988420in}{0.739656in}}%
\pgfpathlineto{\pgfqpoint{1.988122in}{0.739656in}}%
\pgfpathlineto{\pgfqpoint{1.987825in}{0.739656in}}%
\pgfpathlineto{\pgfqpoint{1.987527in}{0.739656in}}%
\pgfpathlineto{\pgfqpoint{1.987230in}{0.739656in}}%
\pgfpathlineto{\pgfqpoint{1.986933in}{0.739656in}}%
\pgfpathlineto{\pgfqpoint{1.986635in}{0.739656in}}%
\pgfpathlineto{\pgfqpoint{1.986338in}{0.739656in}}%
\pgfpathlineto{\pgfqpoint{1.986040in}{0.739656in}}%
\pgfpathlineto{\pgfqpoint{1.985743in}{0.739656in}}%
\pgfpathlineto{\pgfqpoint{1.985445in}{0.739656in}}%
\pgfpathlineto{\pgfqpoint{1.985148in}{0.739656in}}%
\pgfpathlineto{\pgfqpoint{1.984850in}{0.739656in}}%
\pgfpathlineto{\pgfqpoint{1.984553in}{0.739656in}}%
\pgfpathlineto{\pgfqpoint{1.984255in}{0.739656in}}%
\pgfpathlineto{\pgfqpoint{1.983958in}{0.739656in}}%
\pgfpathlineto{\pgfqpoint{1.983660in}{0.739656in}}%
\pgfpathlineto{\pgfqpoint{1.983363in}{0.739656in}}%
\pgfpathlineto{\pgfqpoint{1.983065in}{0.739656in}}%
\pgfpathlineto{\pgfqpoint{1.982768in}{0.739656in}}%
\pgfpathlineto{\pgfqpoint{1.982470in}{0.739656in}}%
\pgfpathlineto{\pgfqpoint{1.982173in}{0.739656in}}%
\pgfpathlineto{\pgfqpoint{1.981875in}{0.739656in}}%
\pgfpathlineto{\pgfqpoint{1.981578in}{0.739656in}}%
\pgfpathlineto{\pgfqpoint{1.981280in}{0.739656in}}%
\pgfpathlineto{\pgfqpoint{1.980983in}{0.739656in}}%
\pgfpathlineto{\pgfqpoint{1.980685in}{0.739656in}}%
\pgfpathlineto{\pgfqpoint{1.980388in}{0.739656in}}%
\pgfpathlineto{\pgfqpoint{1.980091in}{0.739656in}}%
\pgfpathlineto{\pgfqpoint{1.979793in}{0.739656in}}%
\pgfpathlineto{\pgfqpoint{1.979496in}{0.739656in}}%
\pgfpathlineto{\pgfqpoint{1.979198in}{0.739656in}}%
\pgfpathlineto{\pgfqpoint{1.978901in}{0.739656in}}%
\pgfpathlineto{\pgfqpoint{1.978603in}{0.739656in}}%
\pgfpathlineto{\pgfqpoint{1.978306in}{0.739656in}}%
\pgfpathlineto{\pgfqpoint{1.978008in}{0.739656in}}%
\pgfpathlineto{\pgfqpoint{1.977711in}{0.739656in}}%
\pgfpathlineto{\pgfqpoint{1.977413in}{0.739656in}}%
\pgfpathlineto{\pgfqpoint{1.977116in}{0.739656in}}%
\pgfpathlineto{\pgfqpoint{1.976818in}{0.739656in}}%
\pgfpathlineto{\pgfqpoint{1.976521in}{0.739656in}}%
\pgfpathlineto{\pgfqpoint{1.976223in}{0.739656in}}%
\pgfpathlineto{\pgfqpoint{1.975926in}{0.739656in}}%
\pgfpathlineto{\pgfqpoint{1.975628in}{0.739656in}}%
\pgfpathlineto{\pgfqpoint{1.975331in}{0.739656in}}%
\pgfpathlineto{\pgfqpoint{1.975033in}{0.739656in}}%
\pgfpathlineto{\pgfqpoint{1.974736in}{0.739656in}}%
\pgfpathlineto{\pgfqpoint{1.974438in}{0.739656in}}%
\pgfpathlineto{\pgfqpoint{1.974141in}{0.739656in}}%
\pgfpathlineto{\pgfqpoint{1.973844in}{0.739656in}}%
\pgfpathlineto{\pgfqpoint{1.973546in}{0.739656in}}%
\pgfpathlineto{\pgfqpoint{1.973249in}{0.739656in}}%
\pgfpathlineto{\pgfqpoint{1.972951in}{0.739656in}}%
\pgfpathlineto{\pgfqpoint{1.972654in}{0.739656in}}%
\pgfpathlineto{\pgfqpoint{1.972356in}{0.739656in}}%
\pgfpathlineto{\pgfqpoint{1.972059in}{0.739656in}}%
\pgfpathlineto{\pgfqpoint{1.971761in}{0.739656in}}%
\pgfpathlineto{\pgfqpoint{1.971464in}{0.739656in}}%
\pgfpathlineto{\pgfqpoint{1.971166in}{0.739656in}}%
\pgfpathlineto{\pgfqpoint{1.970869in}{0.739656in}}%
\pgfpathlineto{\pgfqpoint{1.970571in}{0.739656in}}%
\pgfpathlineto{\pgfqpoint{1.970274in}{0.739656in}}%
\pgfpathlineto{\pgfqpoint{1.969976in}{0.739656in}}%
\pgfpathlineto{\pgfqpoint{1.969679in}{0.739656in}}%
\pgfpathlineto{\pgfqpoint{1.969381in}{0.739656in}}%
\pgfpathlineto{\pgfqpoint{1.969084in}{0.739656in}}%
\pgfpathlineto{\pgfqpoint{1.968786in}{0.739656in}}%
\pgfpathlineto{\pgfqpoint{1.968489in}{0.739656in}}%
\pgfpathlineto{\pgfqpoint{1.968191in}{0.739656in}}%
\pgfpathlineto{\pgfqpoint{1.967894in}{0.739656in}}%
\pgfpathlineto{\pgfqpoint{1.967596in}{0.739656in}}%
\pgfpathlineto{\pgfqpoint{1.967299in}{0.739656in}}%
\pgfpathlineto{\pgfqpoint{1.967002in}{0.739656in}}%
\pgfpathlineto{\pgfqpoint{1.966704in}{0.739656in}}%
\pgfpathlineto{\pgfqpoint{1.966407in}{0.739656in}}%
\pgfpathlineto{\pgfqpoint{1.966109in}{0.739656in}}%
\pgfpathlineto{\pgfqpoint{1.965812in}{0.739656in}}%
\pgfpathlineto{\pgfqpoint{1.965514in}{0.739656in}}%
\pgfpathlineto{\pgfqpoint{1.965217in}{0.739656in}}%
\pgfpathlineto{\pgfqpoint{1.964919in}{0.739656in}}%
\pgfpathlineto{\pgfqpoint{1.964622in}{0.739656in}}%
\pgfpathlineto{\pgfqpoint{1.964324in}{0.739656in}}%
\pgfpathlineto{\pgfqpoint{1.964027in}{0.739656in}}%
\pgfpathlineto{\pgfqpoint{1.963729in}{0.739656in}}%
\pgfpathlineto{\pgfqpoint{1.963432in}{0.739656in}}%
\pgfpathlineto{\pgfqpoint{1.963134in}{0.739656in}}%
\pgfpathlineto{\pgfqpoint{1.962837in}{0.739656in}}%
\pgfpathlineto{\pgfqpoint{1.962539in}{0.739656in}}%
\pgfpathlineto{\pgfqpoint{1.962242in}{0.739656in}}%
\pgfpathlineto{\pgfqpoint{1.961944in}{0.739656in}}%
\pgfpathlineto{\pgfqpoint{1.961647in}{0.739656in}}%
\pgfpathlineto{\pgfqpoint{1.961349in}{0.739656in}}%
\pgfpathlineto{\pgfqpoint{1.961052in}{0.739656in}}%
\pgfpathlineto{\pgfqpoint{1.960754in}{0.739656in}}%
\pgfpathlineto{\pgfqpoint{1.960457in}{0.739656in}}%
\pgfpathlineto{\pgfqpoint{1.960160in}{0.739656in}}%
\pgfpathlineto{\pgfqpoint{1.959862in}{0.739656in}}%
\pgfpathlineto{\pgfqpoint{1.959565in}{0.739656in}}%
\pgfpathlineto{\pgfqpoint{1.959267in}{0.739656in}}%
\pgfpathlineto{\pgfqpoint{1.958970in}{0.739656in}}%
\pgfpathlineto{\pgfqpoint{1.958672in}{0.739656in}}%
\pgfpathlineto{\pgfqpoint{1.958375in}{0.739656in}}%
\pgfpathlineto{\pgfqpoint{1.958077in}{0.739656in}}%
\pgfpathlineto{\pgfqpoint{1.957780in}{0.739656in}}%
\pgfpathlineto{\pgfqpoint{1.957482in}{0.739656in}}%
\pgfpathlineto{\pgfqpoint{1.957185in}{0.739656in}}%
\pgfpathlineto{\pgfqpoint{1.956887in}{0.739656in}}%
\pgfpathlineto{\pgfqpoint{1.956590in}{0.739656in}}%
\pgfpathlineto{\pgfqpoint{1.956292in}{0.739656in}}%
\pgfpathlineto{\pgfqpoint{1.955995in}{0.739656in}}%
\pgfpathlineto{\pgfqpoint{1.955697in}{0.739656in}}%
\pgfpathlineto{\pgfqpoint{1.955400in}{0.739656in}}%
\pgfpathlineto{\pgfqpoint{1.955102in}{0.739656in}}%
\pgfpathlineto{\pgfqpoint{1.954805in}{0.739656in}}%
\pgfpathlineto{\pgfqpoint{1.954507in}{0.739656in}}%
\pgfpathlineto{\pgfqpoint{1.954210in}{0.739656in}}%
\pgfpathlineto{\pgfqpoint{1.953913in}{0.739656in}}%
\pgfpathlineto{\pgfqpoint{1.953615in}{0.739656in}}%
\pgfpathlineto{\pgfqpoint{1.953318in}{0.739656in}}%
\pgfpathlineto{\pgfqpoint{1.953020in}{0.739656in}}%
\pgfpathlineto{\pgfqpoint{1.952723in}{0.739656in}}%
\pgfpathlineto{\pgfqpoint{1.952425in}{0.739656in}}%
\pgfpathlineto{\pgfqpoint{1.952128in}{0.739656in}}%
\pgfpathlineto{\pgfqpoint{1.951830in}{0.739656in}}%
\pgfpathlineto{\pgfqpoint{1.951533in}{0.739656in}}%
\pgfpathlineto{\pgfqpoint{1.951235in}{0.739656in}}%
\pgfpathlineto{\pgfqpoint{1.950938in}{0.739656in}}%
\pgfpathlineto{\pgfqpoint{1.950640in}{0.739656in}}%
\pgfpathlineto{\pgfqpoint{1.950343in}{0.739656in}}%
\pgfpathlineto{\pgfqpoint{1.950045in}{0.739656in}}%
\pgfpathlineto{\pgfqpoint{1.949748in}{0.739656in}}%
\pgfpathlineto{\pgfqpoint{1.949450in}{0.739656in}}%
\pgfpathlineto{\pgfqpoint{1.949153in}{0.739656in}}%
\pgfpathlineto{\pgfqpoint{1.948855in}{0.739656in}}%
\pgfpathlineto{\pgfqpoint{1.948558in}{0.739656in}}%
\pgfpathlineto{\pgfqpoint{1.948260in}{0.739656in}}%
\pgfpathlineto{\pgfqpoint{1.947963in}{0.739656in}}%
\pgfpathlineto{\pgfqpoint{1.947665in}{0.739656in}}%
\pgfpathlineto{\pgfqpoint{1.947368in}{0.739656in}}%
\pgfpathlineto{\pgfqpoint{1.947071in}{0.739656in}}%
\pgfpathlineto{\pgfqpoint{1.946773in}{0.739656in}}%
\pgfpathlineto{\pgfqpoint{1.946476in}{0.739656in}}%
\pgfpathlineto{\pgfqpoint{1.946178in}{0.739656in}}%
\pgfpathlineto{\pgfqpoint{1.945881in}{0.739656in}}%
\pgfpathlineto{\pgfqpoint{1.945583in}{0.739656in}}%
\pgfpathlineto{\pgfqpoint{1.945286in}{0.739656in}}%
\pgfpathlineto{\pgfqpoint{1.944988in}{0.739656in}}%
\pgfpathlineto{\pgfqpoint{1.944691in}{0.739656in}}%
\pgfpathlineto{\pgfqpoint{1.944393in}{0.739656in}}%
\pgfpathlineto{\pgfqpoint{1.944096in}{0.739656in}}%
\pgfpathlineto{\pgfqpoint{1.943798in}{0.739656in}}%
\pgfpathlineto{\pgfqpoint{1.943501in}{0.739656in}}%
\pgfpathlineto{\pgfqpoint{1.943203in}{0.739656in}}%
\pgfpathlineto{\pgfqpoint{1.942906in}{0.739656in}}%
\pgfpathlineto{\pgfqpoint{1.942608in}{0.739656in}}%
\pgfpathlineto{\pgfqpoint{1.942311in}{0.739656in}}%
\pgfpathlineto{\pgfqpoint{1.942013in}{0.739656in}}%
\pgfpathlineto{\pgfqpoint{1.941716in}{0.739656in}}%
\pgfpathlineto{\pgfqpoint{1.941418in}{0.739656in}}%
\pgfpathlineto{\pgfqpoint{1.941121in}{0.739656in}}%
\pgfpathlineto{\pgfqpoint{1.940823in}{0.739656in}}%
\pgfpathlineto{\pgfqpoint{1.940526in}{0.739656in}}%
\pgfpathlineto{\pgfqpoint{1.940229in}{0.739656in}}%
\pgfpathlineto{\pgfqpoint{1.939931in}{0.739656in}}%
\pgfpathlineto{\pgfqpoint{1.939634in}{0.739656in}}%
\pgfpathlineto{\pgfqpoint{1.939336in}{0.739656in}}%
\pgfpathlineto{\pgfqpoint{1.939039in}{0.739656in}}%
\pgfpathlineto{\pgfqpoint{1.938741in}{0.739656in}}%
\pgfpathlineto{\pgfqpoint{1.938444in}{0.739656in}}%
\pgfpathlineto{\pgfqpoint{1.938146in}{0.739656in}}%
\pgfpathlineto{\pgfqpoint{1.937849in}{0.739656in}}%
\pgfpathlineto{\pgfqpoint{1.937551in}{0.739656in}}%
\pgfpathlineto{\pgfqpoint{1.937254in}{0.739656in}}%
\pgfpathlineto{\pgfqpoint{1.936956in}{0.739656in}}%
\pgfpathlineto{\pgfqpoint{1.936659in}{0.739656in}}%
\pgfpathlineto{\pgfqpoint{1.936361in}{0.739656in}}%
\pgfpathlineto{\pgfqpoint{1.936064in}{0.739656in}}%
\pgfpathlineto{\pgfqpoint{1.935766in}{0.739656in}}%
\pgfpathlineto{\pgfqpoint{1.935469in}{0.739656in}}%
\pgfpathlineto{\pgfqpoint{1.935171in}{0.739656in}}%
\pgfpathlineto{\pgfqpoint{1.934874in}{0.739656in}}%
\pgfpathlineto{\pgfqpoint{1.934576in}{0.739656in}}%
\pgfpathlineto{\pgfqpoint{1.934279in}{0.739656in}}%
\pgfpathlineto{\pgfqpoint{1.933982in}{0.739656in}}%
\pgfpathlineto{\pgfqpoint{1.933684in}{0.739656in}}%
\pgfpathlineto{\pgfqpoint{1.933387in}{0.739656in}}%
\pgfpathlineto{\pgfqpoint{1.933089in}{0.739656in}}%
\pgfpathlineto{\pgfqpoint{1.932792in}{0.739656in}}%
\pgfpathlineto{\pgfqpoint{1.932494in}{0.739656in}}%
\pgfpathlineto{\pgfqpoint{1.932197in}{0.739656in}}%
\pgfpathlineto{\pgfqpoint{1.931899in}{0.739656in}}%
\pgfpathlineto{\pgfqpoint{1.931602in}{0.739656in}}%
\pgfpathlineto{\pgfqpoint{1.931304in}{0.739656in}}%
\pgfpathlineto{\pgfqpoint{1.931007in}{0.739656in}}%
\pgfpathlineto{\pgfqpoint{1.930709in}{0.739656in}}%
\pgfpathlineto{\pgfqpoint{1.930412in}{0.739656in}}%
\pgfpathlineto{\pgfqpoint{1.930114in}{0.739656in}}%
\pgfpathlineto{\pgfqpoint{1.929817in}{0.739656in}}%
\pgfpathlineto{\pgfqpoint{1.929519in}{0.739656in}}%
\pgfpathlineto{\pgfqpoint{1.929222in}{0.739656in}}%
\pgfpathlineto{\pgfqpoint{1.928924in}{0.739656in}}%
\pgfpathlineto{\pgfqpoint{1.928627in}{0.739656in}}%
\pgfpathlineto{\pgfqpoint{1.928329in}{0.739656in}}%
\pgfpathlineto{\pgfqpoint{1.928032in}{0.739656in}}%
\pgfpathlineto{\pgfqpoint{1.927734in}{0.739656in}}%
\pgfpathlineto{\pgfqpoint{1.927437in}{0.739656in}}%
\pgfpathlineto{\pgfqpoint{1.927140in}{0.739656in}}%
\pgfpathlineto{\pgfqpoint{1.926842in}{0.739656in}}%
\pgfpathlineto{\pgfqpoint{1.926545in}{0.739656in}}%
\pgfpathlineto{\pgfqpoint{1.926247in}{0.739656in}}%
\pgfpathlineto{\pgfqpoint{1.925950in}{0.739656in}}%
\pgfpathlineto{\pgfqpoint{1.925652in}{0.739656in}}%
\pgfpathlineto{\pgfqpoint{1.925355in}{0.739656in}}%
\pgfpathlineto{\pgfqpoint{1.925057in}{0.739656in}}%
\pgfpathlineto{\pgfqpoint{1.924760in}{0.739656in}}%
\pgfpathlineto{\pgfqpoint{1.924462in}{0.739656in}}%
\pgfpathlineto{\pgfqpoint{1.924165in}{0.739656in}}%
\pgfpathlineto{\pgfqpoint{1.923867in}{0.739656in}}%
\pgfpathlineto{\pgfqpoint{1.923570in}{0.739656in}}%
\pgfpathlineto{\pgfqpoint{1.923272in}{0.739656in}}%
\pgfpathlineto{\pgfqpoint{1.922975in}{0.739656in}}%
\pgfpathlineto{\pgfqpoint{1.922677in}{0.739656in}}%
\pgfpathlineto{\pgfqpoint{1.922380in}{0.739656in}}%
\pgfpathlineto{\pgfqpoint{1.922082in}{0.739656in}}%
\pgfpathlineto{\pgfqpoint{1.921785in}{0.739656in}}%
\pgfpathlineto{\pgfqpoint{1.921487in}{0.739656in}}%
\pgfpathlineto{\pgfqpoint{1.921190in}{0.739656in}}%
\pgfpathlineto{\pgfqpoint{1.920892in}{0.739656in}}%
\pgfpathlineto{\pgfqpoint{1.920595in}{0.739656in}}%
\pgfpathlineto{\pgfqpoint{1.920298in}{0.739656in}}%
\pgfpathlineto{\pgfqpoint{1.920000in}{0.739656in}}%
\pgfpathlineto{\pgfqpoint{1.919703in}{0.739656in}}%
\pgfpathlineto{\pgfqpoint{1.919405in}{0.739656in}}%
\pgfpathlineto{\pgfqpoint{1.919108in}{0.739656in}}%
\pgfpathlineto{\pgfqpoint{1.918810in}{0.739656in}}%
\pgfpathlineto{\pgfqpoint{1.918513in}{0.739656in}}%
\pgfpathlineto{\pgfqpoint{1.918215in}{0.739656in}}%
\pgfpathlineto{\pgfqpoint{1.917918in}{0.739656in}}%
\pgfpathlineto{\pgfqpoint{1.917620in}{0.739656in}}%
\pgfpathlineto{\pgfqpoint{1.917323in}{0.739656in}}%
\pgfpathlineto{\pgfqpoint{1.917025in}{0.739656in}}%
\pgfpathlineto{\pgfqpoint{1.916728in}{0.739656in}}%
\pgfpathlineto{\pgfqpoint{1.916430in}{0.739656in}}%
\pgfpathlineto{\pgfqpoint{1.916133in}{0.739656in}}%
\pgfpathlineto{\pgfqpoint{1.915835in}{0.739656in}}%
\pgfpathlineto{\pgfqpoint{1.915538in}{0.739656in}}%
\pgfpathlineto{\pgfqpoint{1.915240in}{0.739656in}}%
\pgfpathlineto{\pgfqpoint{1.914943in}{0.739656in}}%
\pgfpathlineto{\pgfqpoint{1.914645in}{0.739656in}}%
\pgfpathlineto{\pgfqpoint{1.914348in}{0.739656in}}%
\pgfpathlineto{\pgfqpoint{1.914051in}{0.739656in}}%
\pgfpathlineto{\pgfqpoint{1.913753in}{0.739656in}}%
\pgfpathlineto{\pgfqpoint{1.913456in}{0.739656in}}%
\pgfpathlineto{\pgfqpoint{1.913158in}{0.739656in}}%
\pgfpathlineto{\pgfqpoint{1.912861in}{0.739656in}}%
\pgfpathlineto{\pgfqpoint{1.912563in}{0.739656in}}%
\pgfpathlineto{\pgfqpoint{1.912266in}{0.739656in}}%
\pgfpathlineto{\pgfqpoint{1.911968in}{0.739656in}}%
\pgfpathlineto{\pgfqpoint{1.911671in}{0.739656in}}%
\pgfpathlineto{\pgfqpoint{1.911373in}{0.739656in}}%
\pgfpathlineto{\pgfqpoint{1.911076in}{0.739656in}}%
\pgfpathlineto{\pgfqpoint{1.910778in}{0.739656in}}%
\pgfpathlineto{\pgfqpoint{1.910481in}{0.739656in}}%
\pgfpathlineto{\pgfqpoint{1.910183in}{0.739656in}}%
\pgfpathlineto{\pgfqpoint{1.909886in}{0.739656in}}%
\pgfpathlineto{\pgfqpoint{1.909588in}{0.739656in}}%
\pgfpathlineto{\pgfqpoint{1.909291in}{0.739656in}}%
\pgfpathlineto{\pgfqpoint{1.908993in}{0.739656in}}%
\pgfpathlineto{\pgfqpoint{1.908696in}{0.739656in}}%
\pgfpathlineto{\pgfqpoint{1.908398in}{0.739656in}}%
\pgfpathlineto{\pgfqpoint{1.908101in}{0.739656in}}%
\pgfpathlineto{\pgfqpoint{1.907803in}{0.739656in}}%
\pgfpathlineto{\pgfqpoint{1.907506in}{0.739656in}}%
\pgfpathlineto{\pgfqpoint{1.907209in}{0.739656in}}%
\pgfpathlineto{\pgfqpoint{1.906911in}{0.739656in}}%
\pgfpathlineto{\pgfqpoint{1.906614in}{0.739656in}}%
\pgfpathlineto{\pgfqpoint{1.906316in}{0.739656in}}%
\pgfpathlineto{\pgfqpoint{1.906019in}{0.739656in}}%
\pgfpathlineto{\pgfqpoint{1.905721in}{0.739656in}}%
\pgfpathlineto{\pgfqpoint{1.905424in}{0.739656in}}%
\pgfpathlineto{\pgfqpoint{1.905126in}{0.739656in}}%
\pgfpathlineto{\pgfqpoint{1.904829in}{0.739656in}}%
\pgfpathlineto{\pgfqpoint{1.904531in}{0.739656in}}%
\pgfpathlineto{\pgfqpoint{1.904234in}{0.739656in}}%
\pgfpathlineto{\pgfqpoint{1.903936in}{0.739656in}}%
\pgfpathlineto{\pgfqpoint{1.903639in}{0.739656in}}%
\pgfpathlineto{\pgfqpoint{1.903341in}{0.739656in}}%
\pgfpathlineto{\pgfqpoint{1.903044in}{0.739656in}}%
\pgfpathlineto{\pgfqpoint{1.902746in}{0.739656in}}%
\pgfpathlineto{\pgfqpoint{1.902449in}{0.739656in}}%
\pgfpathlineto{\pgfqpoint{1.902151in}{0.739656in}}%
\pgfpathlineto{\pgfqpoint{1.901854in}{0.739656in}}%
\pgfpathlineto{\pgfqpoint{1.901556in}{0.739656in}}%
\pgfpathlineto{\pgfqpoint{1.901259in}{0.739656in}}%
\pgfpathlineto{\pgfqpoint{1.900961in}{0.739656in}}%
\pgfpathlineto{\pgfqpoint{1.900664in}{0.739656in}}%
\pgfpathlineto{\pgfqpoint{1.900367in}{0.739656in}}%
\pgfpathlineto{\pgfqpoint{1.900069in}{0.739656in}}%
\pgfpathlineto{\pgfqpoint{1.899772in}{0.739656in}}%
\pgfpathlineto{\pgfqpoint{1.899474in}{0.739656in}}%
\pgfpathlineto{\pgfqpoint{1.899177in}{0.739656in}}%
\pgfpathlineto{\pgfqpoint{1.898879in}{0.739656in}}%
\pgfpathlineto{\pgfqpoint{1.898582in}{0.739656in}}%
\pgfpathlineto{\pgfqpoint{1.898284in}{0.739656in}}%
\pgfpathlineto{\pgfqpoint{1.897987in}{0.739656in}}%
\pgfpathlineto{\pgfqpoint{1.897689in}{0.739656in}}%
\pgfpathlineto{\pgfqpoint{1.897392in}{0.739656in}}%
\pgfpathlineto{\pgfqpoint{1.897094in}{0.739656in}}%
\pgfpathlineto{\pgfqpoint{1.896797in}{0.739656in}}%
\pgfpathlineto{\pgfqpoint{1.896499in}{0.739656in}}%
\pgfpathlineto{\pgfqpoint{1.896202in}{0.739656in}}%
\pgfpathlineto{\pgfqpoint{1.895904in}{0.739656in}}%
\pgfpathlineto{\pgfqpoint{1.895607in}{0.739656in}}%
\pgfpathlineto{\pgfqpoint{1.895309in}{0.739656in}}%
\pgfpathlineto{\pgfqpoint{1.895012in}{0.739656in}}%
\pgfpathlineto{\pgfqpoint{1.894714in}{0.739656in}}%
\pgfpathlineto{\pgfqpoint{1.894417in}{0.739656in}}%
\pgfpathlineto{\pgfqpoint{1.894120in}{0.739656in}}%
\pgfpathlineto{\pgfqpoint{1.893822in}{0.739656in}}%
\pgfpathlineto{\pgfqpoint{1.893525in}{0.739656in}}%
\pgfpathlineto{\pgfqpoint{1.893227in}{0.739656in}}%
\pgfpathlineto{\pgfqpoint{1.892930in}{0.739656in}}%
\pgfpathlineto{\pgfqpoint{1.892632in}{0.739656in}}%
\pgfpathlineto{\pgfqpoint{1.892335in}{0.739656in}}%
\pgfpathlineto{\pgfqpoint{1.892037in}{0.739656in}}%
\pgfpathlineto{\pgfqpoint{1.891740in}{0.739656in}}%
\pgfpathlineto{\pgfqpoint{1.891442in}{0.739656in}}%
\pgfpathlineto{\pgfqpoint{1.891145in}{0.739656in}}%
\pgfpathlineto{\pgfqpoint{1.890847in}{0.739656in}}%
\pgfpathlineto{\pgfqpoint{1.890550in}{0.739656in}}%
\pgfpathlineto{\pgfqpoint{1.890252in}{0.739656in}}%
\pgfpathlineto{\pgfqpoint{1.889955in}{0.739656in}}%
\pgfpathlineto{\pgfqpoint{1.889657in}{0.739656in}}%
\pgfpathlineto{\pgfqpoint{1.889360in}{0.739656in}}%
\pgfpathlineto{\pgfqpoint{1.889062in}{0.739656in}}%
\pgfpathlineto{\pgfqpoint{1.888765in}{0.739656in}}%
\pgfpathlineto{\pgfqpoint{1.888467in}{0.739656in}}%
\pgfpathlineto{\pgfqpoint{1.888170in}{0.739656in}}%
\pgfpathlineto{\pgfqpoint{1.887872in}{0.739656in}}%
\pgfpathlineto{\pgfqpoint{1.887575in}{0.739656in}}%
\pgfpathlineto{\pgfqpoint{1.887278in}{0.739656in}}%
\pgfpathlineto{\pgfqpoint{1.886980in}{0.739656in}}%
\pgfpathlineto{\pgfqpoint{1.886683in}{0.739656in}}%
\pgfpathlineto{\pgfqpoint{1.886385in}{0.739656in}}%
\pgfpathlineto{\pgfqpoint{1.886088in}{0.739656in}}%
\pgfpathlineto{\pgfqpoint{1.885790in}{0.739656in}}%
\pgfpathlineto{\pgfqpoint{1.885493in}{0.739656in}}%
\pgfpathlineto{\pgfqpoint{1.885195in}{0.739656in}}%
\pgfpathlineto{\pgfqpoint{1.884898in}{0.739656in}}%
\pgfpathlineto{\pgfqpoint{1.884600in}{0.739656in}}%
\pgfpathlineto{\pgfqpoint{1.884303in}{0.739656in}}%
\pgfpathlineto{\pgfqpoint{1.884005in}{0.739656in}}%
\pgfpathlineto{\pgfqpoint{1.883708in}{0.739656in}}%
\pgfpathlineto{\pgfqpoint{1.883410in}{0.739656in}}%
\pgfpathlineto{\pgfqpoint{1.883113in}{0.739656in}}%
\pgfpathlineto{\pgfqpoint{1.882815in}{0.739656in}}%
\pgfpathlineto{\pgfqpoint{1.882518in}{0.739656in}}%
\pgfpathlineto{\pgfqpoint{1.882220in}{0.739656in}}%
\pgfpathlineto{\pgfqpoint{1.881923in}{0.739656in}}%
\pgfpathlineto{\pgfqpoint{1.881625in}{0.739656in}}%
\pgfpathlineto{\pgfqpoint{1.881328in}{0.739656in}}%
\pgfpathlineto{\pgfqpoint{1.881030in}{0.739656in}}%
\pgfpathlineto{\pgfqpoint{1.880733in}{0.739656in}}%
\pgfpathlineto{\pgfqpoint{1.880436in}{0.739656in}}%
\pgfpathlineto{\pgfqpoint{1.880138in}{0.739656in}}%
\pgfpathlineto{\pgfqpoint{1.879841in}{0.739656in}}%
\pgfpathlineto{\pgfqpoint{1.879543in}{0.739656in}}%
\pgfpathlineto{\pgfqpoint{1.879246in}{0.739656in}}%
\pgfpathlineto{\pgfqpoint{1.878948in}{0.739656in}}%
\pgfpathlineto{\pgfqpoint{1.878651in}{0.739656in}}%
\pgfpathlineto{\pgfqpoint{1.878353in}{0.739656in}}%
\pgfpathlineto{\pgfqpoint{1.878056in}{0.739656in}}%
\pgfpathlineto{\pgfqpoint{1.877758in}{0.739656in}}%
\pgfpathlineto{\pgfqpoint{1.877461in}{0.739656in}}%
\pgfpathlineto{\pgfqpoint{1.877163in}{0.739656in}}%
\pgfpathlineto{\pgfqpoint{1.876866in}{0.739656in}}%
\pgfpathlineto{\pgfqpoint{1.876568in}{0.739656in}}%
\pgfpathlineto{\pgfqpoint{1.876271in}{0.739656in}}%
\pgfpathlineto{\pgfqpoint{1.875973in}{0.739656in}}%
\pgfpathlineto{\pgfqpoint{1.875676in}{0.739656in}}%
\pgfpathlineto{\pgfqpoint{1.875378in}{0.739656in}}%
\pgfpathlineto{\pgfqpoint{1.875081in}{0.739656in}}%
\pgfpathlineto{\pgfqpoint{1.874783in}{0.739656in}}%
\pgfpathlineto{\pgfqpoint{1.874486in}{0.739656in}}%
\pgfpathlineto{\pgfqpoint{1.874189in}{0.739656in}}%
\pgfpathlineto{\pgfqpoint{1.873891in}{0.739656in}}%
\pgfpathlineto{\pgfqpoint{1.873594in}{0.739656in}}%
\pgfpathlineto{\pgfqpoint{1.873296in}{0.739656in}}%
\pgfpathlineto{\pgfqpoint{1.872999in}{0.739656in}}%
\pgfpathlineto{\pgfqpoint{1.872701in}{0.739656in}}%
\pgfpathlineto{\pgfqpoint{1.872404in}{0.739656in}}%
\pgfpathlineto{\pgfqpoint{1.872106in}{0.739656in}}%
\pgfpathlineto{\pgfqpoint{1.871809in}{0.739656in}}%
\pgfpathlineto{\pgfqpoint{1.871511in}{0.739656in}}%
\pgfpathlineto{\pgfqpoint{1.871214in}{0.739656in}}%
\pgfpathlineto{\pgfqpoint{1.870916in}{0.739656in}}%
\pgfpathlineto{\pgfqpoint{1.870619in}{0.739656in}}%
\pgfpathlineto{\pgfqpoint{1.870321in}{0.739656in}}%
\pgfpathlineto{\pgfqpoint{1.870024in}{0.739656in}}%
\pgfpathlineto{\pgfqpoint{1.869726in}{0.739656in}}%
\pgfpathlineto{\pgfqpoint{1.869429in}{0.739656in}}%
\pgfpathlineto{\pgfqpoint{1.869131in}{0.739656in}}%
\pgfpathlineto{\pgfqpoint{1.868834in}{0.739656in}}%
\pgfpathlineto{\pgfqpoint{1.868536in}{0.739656in}}%
\pgfpathlineto{\pgfqpoint{1.868239in}{0.739656in}}%
\pgfpathlineto{\pgfqpoint{1.867941in}{0.739656in}}%
\pgfpathlineto{\pgfqpoint{1.867644in}{0.739656in}}%
\pgfpathlineto{\pgfqpoint{1.867347in}{0.739656in}}%
\pgfpathlineto{\pgfqpoint{1.867049in}{0.739656in}}%
\pgfpathlineto{\pgfqpoint{1.866752in}{0.739656in}}%
\pgfpathlineto{\pgfqpoint{1.866454in}{0.739656in}}%
\pgfpathlineto{\pgfqpoint{1.866157in}{0.739656in}}%
\pgfpathlineto{\pgfqpoint{1.865859in}{0.739656in}}%
\pgfpathlineto{\pgfqpoint{1.865562in}{0.739656in}}%
\pgfpathlineto{\pgfqpoint{1.865264in}{0.739656in}}%
\pgfpathlineto{\pgfqpoint{1.864967in}{0.739656in}}%
\pgfpathlineto{\pgfqpoint{1.864669in}{0.739656in}}%
\pgfpathlineto{\pgfqpoint{1.864372in}{0.739656in}}%
\pgfpathlineto{\pgfqpoint{1.864074in}{0.739656in}}%
\pgfpathlineto{\pgfqpoint{1.863777in}{0.739656in}}%
\pgfpathlineto{\pgfqpoint{1.863479in}{0.739656in}}%
\pgfpathlineto{\pgfqpoint{1.863182in}{0.739656in}}%
\pgfpathlineto{\pgfqpoint{1.862884in}{0.739656in}}%
\pgfpathlineto{\pgfqpoint{1.862587in}{0.739656in}}%
\pgfpathlineto{\pgfqpoint{1.862289in}{0.739656in}}%
\pgfpathlineto{\pgfqpoint{1.861992in}{0.739656in}}%
\pgfpathlineto{\pgfqpoint{1.861694in}{0.739656in}}%
\pgfpathlineto{\pgfqpoint{1.861397in}{0.739656in}}%
\pgfpathlineto{\pgfqpoint{1.861099in}{0.739656in}}%
\pgfpathlineto{\pgfqpoint{1.860802in}{0.739656in}}%
\pgfpathlineto{\pgfqpoint{1.860505in}{0.739656in}}%
\pgfpathlineto{\pgfqpoint{1.860207in}{0.739656in}}%
\pgfpathlineto{\pgfqpoint{1.859910in}{0.739656in}}%
\pgfpathlineto{\pgfqpoint{1.859612in}{0.739656in}}%
\pgfpathlineto{\pgfqpoint{1.859315in}{0.739656in}}%
\pgfpathlineto{\pgfqpoint{1.859017in}{0.739656in}}%
\pgfpathlineto{\pgfqpoint{1.858720in}{0.739656in}}%
\pgfpathlineto{\pgfqpoint{1.858422in}{0.739656in}}%
\pgfpathlineto{\pgfqpoint{1.858125in}{0.739656in}}%
\pgfpathlineto{\pgfqpoint{1.857827in}{0.739656in}}%
\pgfpathlineto{\pgfqpoint{1.857530in}{0.739656in}}%
\pgfpathlineto{\pgfqpoint{1.857232in}{0.739656in}}%
\pgfpathlineto{\pgfqpoint{1.856935in}{0.739656in}}%
\pgfpathlineto{\pgfqpoint{1.856637in}{0.739656in}}%
\pgfpathlineto{\pgfqpoint{1.856340in}{0.739656in}}%
\pgfpathlineto{\pgfqpoint{1.856042in}{0.739656in}}%
\pgfpathlineto{\pgfqpoint{1.855745in}{0.739656in}}%
\pgfpathlineto{\pgfqpoint{1.855447in}{0.739656in}}%
\pgfpathlineto{\pgfqpoint{1.855150in}{0.739656in}}%
\pgfpathlineto{\pgfqpoint{1.854852in}{0.739656in}}%
\pgfpathlineto{\pgfqpoint{1.854555in}{0.739656in}}%
\pgfpathlineto{\pgfqpoint{1.854258in}{0.739656in}}%
\pgfpathlineto{\pgfqpoint{1.853960in}{0.739656in}}%
\pgfpathlineto{\pgfqpoint{1.853663in}{0.739656in}}%
\pgfpathlineto{\pgfqpoint{1.853365in}{0.739656in}}%
\pgfpathlineto{\pgfqpoint{1.853068in}{0.739656in}}%
\pgfpathlineto{\pgfqpoint{1.852770in}{0.739656in}}%
\pgfpathlineto{\pgfqpoint{1.852473in}{0.739656in}}%
\pgfpathlineto{\pgfqpoint{1.852175in}{0.739656in}}%
\pgfpathlineto{\pgfqpoint{1.851878in}{0.739656in}}%
\pgfpathlineto{\pgfqpoint{1.851580in}{0.739656in}}%
\pgfpathlineto{\pgfqpoint{1.851283in}{0.739656in}}%
\pgfpathlineto{\pgfqpoint{1.850985in}{0.739656in}}%
\pgfpathlineto{\pgfqpoint{1.850688in}{0.739656in}}%
\pgfpathlineto{\pgfqpoint{1.850390in}{0.739656in}}%
\pgfpathlineto{\pgfqpoint{1.850093in}{0.739656in}}%
\pgfpathlineto{\pgfqpoint{1.849795in}{0.739656in}}%
\pgfpathlineto{\pgfqpoint{1.849498in}{0.739656in}}%
\pgfpathlineto{\pgfqpoint{1.849200in}{0.739656in}}%
\pgfpathlineto{\pgfqpoint{1.848903in}{0.739656in}}%
\pgfpathlineto{\pgfqpoint{1.848605in}{0.739656in}}%
\pgfpathlineto{\pgfqpoint{1.848308in}{0.739656in}}%
\pgfpathlineto{\pgfqpoint{1.848010in}{0.739656in}}%
\pgfpathlineto{\pgfqpoint{1.847713in}{0.739656in}}%
\pgfpathlineto{\pgfqpoint{1.847416in}{0.739656in}}%
\pgfpathlineto{\pgfqpoint{1.847118in}{0.739656in}}%
\pgfpathlineto{\pgfqpoint{1.846821in}{0.739656in}}%
\pgfpathlineto{\pgfqpoint{1.846523in}{0.739656in}}%
\pgfpathlineto{\pgfqpoint{1.846226in}{0.739656in}}%
\pgfpathlineto{\pgfqpoint{1.845928in}{0.739656in}}%
\pgfpathlineto{\pgfqpoint{1.845631in}{0.739656in}}%
\pgfpathlineto{\pgfqpoint{1.845333in}{0.739656in}}%
\pgfpathlineto{\pgfqpoint{1.845036in}{0.739656in}}%
\pgfpathlineto{\pgfqpoint{1.844738in}{0.739656in}}%
\pgfpathlineto{\pgfqpoint{1.844441in}{0.739656in}}%
\pgfpathlineto{\pgfqpoint{1.844143in}{0.739656in}}%
\pgfpathlineto{\pgfqpoint{1.843846in}{0.739656in}}%
\pgfpathlineto{\pgfqpoint{1.843548in}{0.739656in}}%
\pgfpathlineto{\pgfqpoint{1.843251in}{0.739656in}}%
\pgfpathlineto{\pgfqpoint{1.842953in}{0.739656in}}%
\pgfpathlineto{\pgfqpoint{1.842656in}{0.739656in}}%
\pgfpathlineto{\pgfqpoint{1.842358in}{0.739656in}}%
\pgfpathlineto{\pgfqpoint{1.842061in}{0.739656in}}%
\pgfpathlineto{\pgfqpoint{1.841763in}{0.739656in}}%
\pgfpathlineto{\pgfqpoint{1.841466in}{0.739656in}}%
\pgfpathlineto{\pgfqpoint{1.841168in}{0.739656in}}%
\pgfpathlineto{\pgfqpoint{1.840871in}{0.739656in}}%
\pgfpathlineto{\pgfqpoint{1.840574in}{0.739656in}}%
\pgfpathlineto{\pgfqpoint{1.840276in}{0.739656in}}%
\pgfpathlineto{\pgfqpoint{1.839979in}{0.739656in}}%
\pgfpathlineto{\pgfqpoint{1.839681in}{0.739656in}}%
\pgfpathlineto{\pgfqpoint{1.839384in}{0.739656in}}%
\pgfpathlineto{\pgfqpoint{1.839086in}{0.739656in}}%
\pgfpathlineto{\pgfqpoint{1.838789in}{0.739656in}}%
\pgfpathlineto{\pgfqpoint{1.838491in}{0.739656in}}%
\pgfpathlineto{\pgfqpoint{1.838194in}{0.739656in}}%
\pgfpathlineto{\pgfqpoint{1.837896in}{0.739656in}}%
\pgfpathlineto{\pgfqpoint{1.837599in}{0.739656in}}%
\pgfpathlineto{\pgfqpoint{1.837301in}{0.739656in}}%
\pgfpathlineto{\pgfqpoint{1.837004in}{0.739656in}}%
\pgfpathlineto{\pgfqpoint{1.836706in}{0.739656in}}%
\pgfpathlineto{\pgfqpoint{1.836409in}{0.739656in}}%
\pgfpathlineto{\pgfqpoint{1.836111in}{0.739656in}}%
\pgfpathlineto{\pgfqpoint{1.835814in}{0.739656in}}%
\pgfpathlineto{\pgfqpoint{1.835516in}{0.739656in}}%
\pgfpathlineto{\pgfqpoint{1.835219in}{0.739656in}}%
\pgfpathlineto{\pgfqpoint{1.834921in}{0.739656in}}%
\pgfpathlineto{\pgfqpoint{1.834624in}{0.739656in}}%
\pgfpathlineto{\pgfqpoint{1.834327in}{0.739656in}}%
\pgfpathlineto{\pgfqpoint{1.834029in}{0.739656in}}%
\pgfpathlineto{\pgfqpoint{1.833732in}{0.739656in}}%
\pgfpathlineto{\pgfqpoint{1.833434in}{0.739656in}}%
\pgfpathlineto{\pgfqpoint{1.833137in}{0.739656in}}%
\pgfpathlineto{\pgfqpoint{1.832839in}{0.739656in}}%
\pgfpathlineto{\pgfqpoint{1.832542in}{0.739656in}}%
\pgfpathlineto{\pgfqpoint{1.832244in}{0.739656in}}%
\pgfpathlineto{\pgfqpoint{1.831947in}{0.739656in}}%
\pgfpathlineto{\pgfqpoint{1.831649in}{0.739656in}}%
\pgfpathlineto{\pgfqpoint{1.831352in}{0.739656in}}%
\pgfpathlineto{\pgfqpoint{1.831054in}{0.739656in}}%
\pgfpathlineto{\pgfqpoint{1.830757in}{0.739656in}}%
\pgfpathlineto{\pgfqpoint{1.830459in}{0.739656in}}%
\pgfpathlineto{\pgfqpoint{1.830162in}{0.739656in}}%
\pgfpathlineto{\pgfqpoint{1.829864in}{0.739656in}}%
\pgfpathlineto{\pgfqpoint{1.829567in}{0.739656in}}%
\pgfpathlineto{\pgfqpoint{1.829269in}{0.739656in}}%
\pgfpathlineto{\pgfqpoint{1.828972in}{0.739656in}}%
\pgfpathlineto{\pgfqpoint{1.828674in}{0.739656in}}%
\pgfpathlineto{\pgfqpoint{1.828377in}{0.739656in}}%
\pgfpathlineto{\pgfqpoint{1.828079in}{0.739656in}}%
\pgfpathlineto{\pgfqpoint{1.827782in}{0.739656in}}%
\pgfpathlineto{\pgfqpoint{1.827485in}{0.739656in}}%
\pgfpathlineto{\pgfqpoint{1.827187in}{0.739656in}}%
\pgfpathlineto{\pgfqpoint{1.826890in}{0.739656in}}%
\pgfpathlineto{\pgfqpoint{1.826592in}{0.739656in}}%
\pgfpathlineto{\pgfqpoint{1.826295in}{0.739656in}}%
\pgfpathlineto{\pgfqpoint{1.825997in}{0.739656in}}%
\pgfpathlineto{\pgfqpoint{1.825700in}{0.739656in}}%
\pgfpathlineto{\pgfqpoint{1.825402in}{0.739656in}}%
\pgfpathlineto{\pgfqpoint{1.825105in}{0.739656in}}%
\pgfpathlineto{\pgfqpoint{1.824807in}{0.739656in}}%
\pgfpathlineto{\pgfqpoint{1.824510in}{0.739656in}}%
\pgfpathlineto{\pgfqpoint{1.824212in}{0.739656in}}%
\pgfpathlineto{\pgfqpoint{1.823915in}{0.739656in}}%
\pgfpathlineto{\pgfqpoint{1.823617in}{0.739656in}}%
\pgfpathlineto{\pgfqpoint{1.823320in}{0.739656in}}%
\pgfpathlineto{\pgfqpoint{1.823022in}{0.739656in}}%
\pgfpathlineto{\pgfqpoint{1.822725in}{0.739656in}}%
\pgfpathlineto{\pgfqpoint{1.822427in}{0.739656in}}%
\pgfpathlineto{\pgfqpoint{1.822130in}{0.739656in}}%
\pgfpathlineto{\pgfqpoint{1.821832in}{0.739656in}}%
\pgfpathlineto{\pgfqpoint{1.821535in}{0.739656in}}%
\pgfpathlineto{\pgfqpoint{1.821237in}{0.739656in}}%
\pgfpathlineto{\pgfqpoint{1.820940in}{0.739656in}}%
\pgfpathlineto{\pgfqpoint{1.820643in}{0.739656in}}%
\pgfpathlineto{\pgfqpoint{1.820345in}{0.739656in}}%
\pgfpathlineto{\pgfqpoint{1.820048in}{0.739656in}}%
\pgfpathlineto{\pgfqpoint{1.819750in}{0.739656in}}%
\pgfpathlineto{\pgfqpoint{1.819453in}{0.739656in}}%
\pgfpathlineto{\pgfqpoint{1.819155in}{0.739656in}}%
\pgfpathlineto{\pgfqpoint{1.818858in}{0.739656in}}%
\pgfpathlineto{\pgfqpoint{1.818560in}{0.739656in}}%
\pgfpathlineto{\pgfqpoint{1.818263in}{0.739656in}}%
\pgfpathlineto{\pgfqpoint{1.817965in}{0.739656in}}%
\pgfpathlineto{\pgfqpoint{1.817668in}{0.739656in}}%
\pgfpathlineto{\pgfqpoint{1.817370in}{0.739656in}}%
\pgfpathlineto{\pgfqpoint{1.817073in}{0.739656in}}%
\pgfpathlineto{\pgfqpoint{1.816775in}{0.739656in}}%
\pgfpathlineto{\pgfqpoint{1.816478in}{0.739656in}}%
\pgfpathlineto{\pgfqpoint{1.816180in}{0.739656in}}%
\pgfpathlineto{\pgfqpoint{1.815883in}{0.739656in}}%
\pgfpathlineto{\pgfqpoint{1.815585in}{0.739656in}}%
\pgfpathlineto{\pgfqpoint{1.815288in}{0.739656in}}%
\pgfpathlineto{\pgfqpoint{1.814990in}{0.739656in}}%
\pgfpathlineto{\pgfqpoint{1.814693in}{0.739656in}}%
\pgfpathlineto{\pgfqpoint{1.814396in}{0.739656in}}%
\pgfpathlineto{\pgfqpoint{1.814098in}{0.739656in}}%
\pgfpathlineto{\pgfqpoint{1.813801in}{0.739656in}}%
\pgfpathlineto{\pgfqpoint{1.813503in}{0.739656in}}%
\pgfpathlineto{\pgfqpoint{1.813206in}{0.739656in}}%
\pgfpathlineto{\pgfqpoint{1.812908in}{0.739656in}}%
\pgfpathlineto{\pgfqpoint{1.812611in}{0.739656in}}%
\pgfpathlineto{\pgfqpoint{1.812313in}{0.739656in}}%
\pgfpathlineto{\pgfqpoint{1.812016in}{0.739656in}}%
\pgfpathlineto{\pgfqpoint{1.811718in}{0.739656in}}%
\pgfpathlineto{\pgfqpoint{1.811421in}{0.739656in}}%
\pgfpathlineto{\pgfqpoint{1.811123in}{0.739656in}}%
\pgfpathlineto{\pgfqpoint{1.810826in}{0.739656in}}%
\pgfpathlineto{\pgfqpoint{1.810528in}{0.739656in}}%
\pgfpathlineto{\pgfqpoint{1.810231in}{0.739656in}}%
\pgfpathlineto{\pgfqpoint{1.809933in}{0.739656in}}%
\pgfpathlineto{\pgfqpoint{1.809636in}{0.739656in}}%
\pgfpathlineto{\pgfqpoint{1.809338in}{0.739656in}}%
\pgfpathlineto{\pgfqpoint{1.809041in}{0.739656in}}%
\pgfpathlineto{\pgfqpoint{1.808743in}{0.739656in}}%
\pgfpathlineto{\pgfqpoint{1.808446in}{0.739656in}}%
\pgfpathlineto{\pgfqpoint{1.808148in}{0.739656in}}%
\pgfpathlineto{\pgfqpoint{1.807851in}{0.739656in}}%
\pgfpathlineto{\pgfqpoint{1.807554in}{0.739656in}}%
\pgfpathlineto{\pgfqpoint{1.807256in}{0.739656in}}%
\pgfpathlineto{\pgfqpoint{1.806959in}{0.739656in}}%
\pgfpathlineto{\pgfqpoint{1.806661in}{0.739656in}}%
\pgfpathlineto{\pgfqpoint{1.806364in}{0.739656in}}%
\pgfpathlineto{\pgfqpoint{1.806066in}{0.739656in}}%
\pgfpathlineto{\pgfqpoint{1.805769in}{0.739656in}}%
\pgfpathlineto{\pgfqpoint{1.805471in}{0.739656in}}%
\pgfpathlineto{\pgfqpoint{1.805174in}{0.739656in}}%
\pgfpathlineto{\pgfqpoint{1.804876in}{0.739656in}}%
\pgfpathlineto{\pgfqpoint{1.804579in}{0.739656in}}%
\pgfpathlineto{\pgfqpoint{1.804281in}{0.739656in}}%
\pgfpathlineto{\pgfqpoint{1.803984in}{0.739656in}}%
\pgfpathlineto{\pgfqpoint{1.803686in}{0.739656in}}%
\pgfpathlineto{\pgfqpoint{1.803389in}{0.739656in}}%
\pgfpathlineto{\pgfqpoint{1.803091in}{0.739656in}}%
\pgfpathlineto{\pgfqpoint{1.802794in}{0.739656in}}%
\pgfpathlineto{\pgfqpoint{1.802496in}{0.739656in}}%
\pgfpathlineto{\pgfqpoint{1.802199in}{0.739656in}}%
\pgfpathlineto{\pgfqpoint{1.801901in}{0.739656in}}%
\pgfpathlineto{\pgfqpoint{1.801604in}{0.739656in}}%
\pgfpathlineto{\pgfqpoint{1.801306in}{0.739656in}}%
\pgfpathlineto{\pgfqpoint{1.801009in}{0.739656in}}%
\pgfpathlineto{\pgfqpoint{1.800712in}{0.739656in}}%
\pgfpathlineto{\pgfqpoint{1.800414in}{0.739656in}}%
\pgfpathlineto{\pgfqpoint{1.800117in}{0.739656in}}%
\pgfpathlineto{\pgfqpoint{1.799819in}{0.739656in}}%
\pgfpathlineto{\pgfqpoint{1.799522in}{0.739656in}}%
\pgfpathlineto{\pgfqpoint{1.799224in}{0.739656in}}%
\pgfpathlineto{\pgfqpoint{1.798927in}{0.739656in}}%
\pgfpathlineto{\pgfqpoint{1.798629in}{0.739656in}}%
\pgfpathlineto{\pgfqpoint{1.798332in}{0.739656in}}%
\pgfpathlineto{\pgfqpoint{1.798034in}{0.739656in}}%
\pgfpathlineto{\pgfqpoint{1.797737in}{0.739656in}}%
\pgfpathlineto{\pgfqpoint{1.797439in}{0.739656in}}%
\pgfpathlineto{\pgfqpoint{1.797142in}{0.739656in}}%
\pgfpathlineto{\pgfqpoint{1.796844in}{0.739656in}}%
\pgfpathlineto{\pgfqpoint{1.796547in}{0.739656in}}%
\pgfpathlineto{\pgfqpoint{1.796249in}{0.739656in}}%
\pgfpathlineto{\pgfqpoint{1.795952in}{0.739656in}}%
\pgfpathlineto{\pgfqpoint{1.795654in}{0.739656in}}%
\pgfpathlineto{\pgfqpoint{1.795357in}{0.739656in}}%
\pgfpathlineto{\pgfqpoint{1.795059in}{0.739656in}}%
\pgfpathlineto{\pgfqpoint{1.794762in}{0.739656in}}%
\pgfpathlineto{\pgfqpoint{1.794464in}{0.739656in}}%
\pgfpathlineto{\pgfqpoint{1.794167in}{0.739656in}}%
\pgfpathlineto{\pgfqpoint{1.793870in}{0.739656in}}%
\pgfpathlineto{\pgfqpoint{1.793572in}{0.739656in}}%
\pgfpathlineto{\pgfqpoint{1.793275in}{0.739656in}}%
\pgfpathlineto{\pgfqpoint{1.792977in}{0.739656in}}%
\pgfpathlineto{\pgfqpoint{1.792680in}{0.739656in}}%
\pgfpathlineto{\pgfqpoint{1.792382in}{0.739656in}}%
\pgfpathlineto{\pgfqpoint{1.792085in}{0.739656in}}%
\pgfpathlineto{\pgfqpoint{1.791787in}{0.739656in}}%
\pgfpathlineto{\pgfqpoint{1.791490in}{0.739656in}}%
\pgfpathlineto{\pgfqpoint{1.791192in}{0.739656in}}%
\pgfpathlineto{\pgfqpoint{1.790895in}{0.739656in}}%
\pgfpathlineto{\pgfqpoint{1.790597in}{0.739656in}}%
\pgfpathlineto{\pgfqpoint{1.790300in}{0.739656in}}%
\pgfpathlineto{\pgfqpoint{1.790002in}{0.739656in}}%
\pgfpathlineto{\pgfqpoint{1.789705in}{0.739656in}}%
\pgfpathlineto{\pgfqpoint{1.789407in}{0.739656in}}%
\pgfpathlineto{\pgfqpoint{1.789110in}{0.739656in}}%
\pgfpathlineto{\pgfqpoint{1.788812in}{0.739656in}}%
\pgfpathlineto{\pgfqpoint{1.788515in}{0.739656in}}%
\pgfpathlineto{\pgfqpoint{1.788217in}{0.739656in}}%
\pgfpathlineto{\pgfqpoint{1.787920in}{0.739656in}}%
\pgfpathlineto{\pgfqpoint{1.787623in}{0.739656in}}%
\pgfpathlineto{\pgfqpoint{1.787325in}{0.739656in}}%
\pgfpathlineto{\pgfqpoint{1.787028in}{0.739656in}}%
\pgfpathlineto{\pgfqpoint{1.786730in}{0.739656in}}%
\pgfpathlineto{\pgfqpoint{1.786433in}{0.739656in}}%
\pgfpathlineto{\pgfqpoint{1.786135in}{0.739656in}}%
\pgfpathlineto{\pgfqpoint{1.785838in}{0.739656in}}%
\pgfpathlineto{\pgfqpoint{1.785540in}{0.739656in}}%
\pgfpathlineto{\pgfqpoint{1.785243in}{0.739656in}}%
\pgfpathlineto{\pgfqpoint{1.784945in}{0.739656in}}%
\pgfpathlineto{\pgfqpoint{1.784648in}{0.739656in}}%
\pgfpathlineto{\pgfqpoint{1.784350in}{0.739656in}}%
\pgfpathlineto{\pgfqpoint{1.784053in}{0.739656in}}%
\pgfpathlineto{\pgfqpoint{1.783755in}{0.739656in}}%
\pgfpathlineto{\pgfqpoint{1.783458in}{0.739656in}}%
\pgfpathlineto{\pgfqpoint{1.783160in}{0.739656in}}%
\pgfpathlineto{\pgfqpoint{1.782863in}{0.739656in}}%
\pgfpathlineto{\pgfqpoint{1.782565in}{0.739656in}}%
\pgfpathlineto{\pgfqpoint{1.782268in}{0.739656in}}%
\pgfpathlineto{\pgfqpoint{1.781970in}{0.739656in}}%
\pgfpathlineto{\pgfqpoint{1.781673in}{0.739656in}}%
\pgfpathlineto{\pgfqpoint{1.781375in}{0.739656in}}%
\pgfpathlineto{\pgfqpoint{1.781078in}{0.739656in}}%
\pgfpathlineto{\pgfqpoint{1.780781in}{0.739656in}}%
\pgfpathlineto{\pgfqpoint{1.780483in}{0.739656in}}%
\pgfpathlineto{\pgfqpoint{1.780186in}{0.739656in}}%
\pgfpathlineto{\pgfqpoint{1.779888in}{0.739656in}}%
\pgfpathlineto{\pgfqpoint{1.779591in}{0.739656in}}%
\pgfpathlineto{\pgfqpoint{1.779293in}{0.739656in}}%
\pgfpathlineto{\pgfqpoint{1.778996in}{0.739656in}}%
\pgfpathlineto{\pgfqpoint{1.778698in}{0.739656in}}%
\pgfpathlineto{\pgfqpoint{1.778401in}{0.739656in}}%
\pgfpathlineto{\pgfqpoint{1.778103in}{0.739656in}}%
\pgfpathlineto{\pgfqpoint{1.777806in}{0.739656in}}%
\pgfpathlineto{\pgfqpoint{1.777508in}{0.739656in}}%
\pgfpathlineto{\pgfqpoint{1.777211in}{0.739656in}}%
\pgfpathlineto{\pgfqpoint{1.776913in}{0.739656in}}%
\pgfpathlineto{\pgfqpoint{1.776616in}{0.739656in}}%
\pgfpathlineto{\pgfqpoint{1.776318in}{0.739656in}}%
\pgfpathlineto{\pgfqpoint{1.776021in}{0.739656in}}%
\pgfpathlineto{\pgfqpoint{1.775723in}{0.739656in}}%
\pgfpathlineto{\pgfqpoint{1.775426in}{0.739656in}}%
\pgfpathlineto{\pgfqpoint{1.775128in}{0.739656in}}%
\pgfpathlineto{\pgfqpoint{1.774831in}{0.739656in}}%
\pgfpathlineto{\pgfqpoint{1.774533in}{0.739656in}}%
\pgfpathlineto{\pgfqpoint{1.774236in}{0.739656in}}%
\pgfpathlineto{\pgfqpoint{1.773939in}{0.739656in}}%
\pgfpathlineto{\pgfqpoint{1.773641in}{0.739656in}}%
\pgfpathlineto{\pgfqpoint{1.773344in}{0.739656in}}%
\pgfpathlineto{\pgfqpoint{1.773046in}{0.739656in}}%
\pgfpathlineto{\pgfqpoint{1.772749in}{0.739656in}}%
\pgfpathlineto{\pgfqpoint{1.772451in}{0.739656in}}%
\pgfpathlineto{\pgfqpoint{1.772154in}{0.739656in}}%
\pgfpathlineto{\pgfqpoint{1.771856in}{0.739656in}}%
\pgfpathlineto{\pgfqpoint{1.771559in}{0.739656in}}%
\pgfpathlineto{\pgfqpoint{1.771261in}{0.739656in}}%
\pgfpathlineto{\pgfqpoint{1.770964in}{0.739656in}}%
\pgfpathlineto{\pgfqpoint{1.770666in}{0.739656in}}%
\pgfpathlineto{\pgfqpoint{1.770369in}{0.739656in}}%
\pgfpathlineto{\pgfqpoint{1.770071in}{0.739656in}}%
\pgfpathlineto{\pgfqpoint{1.769774in}{0.739656in}}%
\pgfpathlineto{\pgfqpoint{1.769476in}{0.739656in}}%
\pgfpathlineto{\pgfqpoint{1.769179in}{0.739656in}}%
\pgfpathlineto{\pgfqpoint{1.768881in}{0.739656in}}%
\pgfpathlineto{\pgfqpoint{1.768584in}{0.739656in}}%
\pgfpathlineto{\pgfqpoint{1.768286in}{0.739656in}}%
\pgfpathlineto{\pgfqpoint{1.767989in}{0.739656in}}%
\pgfpathlineto{\pgfqpoint{1.767692in}{0.739656in}}%
\pgfpathlineto{\pgfqpoint{1.767394in}{0.739656in}}%
\pgfpathlineto{\pgfqpoint{1.767097in}{0.739656in}}%
\pgfpathlineto{\pgfqpoint{1.766799in}{0.739656in}}%
\pgfpathlineto{\pgfqpoint{1.766502in}{0.739656in}}%
\pgfpathlineto{\pgfqpoint{1.766204in}{0.739656in}}%
\pgfpathlineto{\pgfqpoint{1.765907in}{0.739656in}}%
\pgfpathlineto{\pgfqpoint{1.765609in}{0.739656in}}%
\pgfpathlineto{\pgfqpoint{1.765312in}{0.739656in}}%
\pgfpathlineto{\pgfqpoint{1.765014in}{0.739656in}}%
\pgfpathlineto{\pgfqpoint{1.764717in}{0.739656in}}%
\pgfpathlineto{\pgfqpoint{1.764419in}{0.739656in}}%
\pgfpathlineto{\pgfqpoint{1.764122in}{0.739656in}}%
\pgfpathlineto{\pgfqpoint{1.763824in}{0.739656in}}%
\pgfpathlineto{\pgfqpoint{1.763527in}{0.739656in}}%
\pgfpathlineto{\pgfqpoint{1.763229in}{0.739656in}}%
\pgfpathlineto{\pgfqpoint{1.762932in}{0.739656in}}%
\pgfpathlineto{\pgfqpoint{1.762634in}{0.739656in}}%
\pgfpathlineto{\pgfqpoint{1.762337in}{0.739656in}}%
\pgfpathlineto{\pgfqpoint{1.762039in}{0.739656in}}%
\pgfpathlineto{\pgfqpoint{1.761742in}{0.739656in}}%
\pgfpathlineto{\pgfqpoint{1.761444in}{0.739656in}}%
\pgfpathlineto{\pgfqpoint{1.761147in}{0.739656in}}%
\pgfpathlineto{\pgfqpoint{1.760850in}{0.739656in}}%
\pgfpathlineto{\pgfqpoint{1.760552in}{0.739656in}}%
\pgfpathlineto{\pgfqpoint{1.760255in}{0.739656in}}%
\pgfpathlineto{\pgfqpoint{1.759957in}{0.739656in}}%
\pgfpathlineto{\pgfqpoint{1.759660in}{0.739656in}}%
\pgfpathlineto{\pgfqpoint{1.759362in}{0.739656in}}%
\pgfpathlineto{\pgfqpoint{1.759065in}{0.739656in}}%
\pgfpathlineto{\pgfqpoint{1.758767in}{0.739656in}}%
\pgfpathlineto{\pgfqpoint{1.758470in}{0.739656in}}%
\pgfpathlineto{\pgfqpoint{1.758172in}{0.739656in}}%
\pgfpathlineto{\pgfqpoint{1.757875in}{0.739656in}}%
\pgfpathlineto{\pgfqpoint{1.757577in}{0.739656in}}%
\pgfpathlineto{\pgfqpoint{1.757280in}{0.739656in}}%
\pgfpathlineto{\pgfqpoint{1.756982in}{0.739656in}}%
\pgfpathlineto{\pgfqpoint{1.756685in}{0.739656in}}%
\pgfpathlineto{\pgfqpoint{1.756387in}{0.739656in}}%
\pgfpathlineto{\pgfqpoint{1.756090in}{0.739656in}}%
\pgfpathlineto{\pgfqpoint{1.755792in}{0.739656in}}%
\pgfpathlineto{\pgfqpoint{1.755495in}{0.739656in}}%
\pgfpathlineto{\pgfqpoint{1.755197in}{0.739656in}}%
\pgfpathlineto{\pgfqpoint{1.754900in}{0.739656in}}%
\pgfpathlineto{\pgfqpoint{1.754602in}{0.739656in}}%
\pgfpathlineto{\pgfqpoint{1.754305in}{0.739656in}}%
\pgfpathlineto{\pgfqpoint{1.754008in}{0.739656in}}%
\pgfpathlineto{\pgfqpoint{1.753710in}{0.739656in}}%
\pgfpathlineto{\pgfqpoint{1.753413in}{0.739656in}}%
\pgfpathlineto{\pgfqpoint{1.753115in}{0.739656in}}%
\pgfpathlineto{\pgfqpoint{1.752818in}{0.739656in}}%
\pgfpathlineto{\pgfqpoint{1.752520in}{0.739656in}}%
\pgfpathlineto{\pgfqpoint{1.752223in}{0.739656in}}%
\pgfpathlineto{\pgfqpoint{1.751925in}{0.739656in}}%
\pgfpathlineto{\pgfqpoint{1.751628in}{0.739656in}}%
\pgfpathlineto{\pgfqpoint{1.751330in}{0.739656in}}%
\pgfpathlineto{\pgfqpoint{1.751033in}{0.739656in}}%
\pgfpathlineto{\pgfqpoint{1.750735in}{0.739656in}}%
\pgfpathlineto{\pgfqpoint{1.750438in}{0.739656in}}%
\pgfpathlineto{\pgfqpoint{1.750140in}{0.739656in}}%
\pgfpathlineto{\pgfqpoint{1.749843in}{0.739656in}}%
\pgfpathlineto{\pgfqpoint{1.749545in}{0.739656in}}%
\pgfpathlineto{\pgfqpoint{1.749248in}{0.739656in}}%
\pgfpathlineto{\pgfqpoint{1.748950in}{0.739656in}}%
\pgfpathlineto{\pgfqpoint{1.748653in}{0.739656in}}%
\pgfpathlineto{\pgfqpoint{1.748355in}{0.739656in}}%
\pgfpathlineto{\pgfqpoint{1.748058in}{0.739656in}}%
\pgfpathlineto{\pgfqpoint{1.747761in}{0.739656in}}%
\pgfpathlineto{\pgfqpoint{1.747463in}{0.739656in}}%
\pgfpathlineto{\pgfqpoint{1.747166in}{0.739656in}}%
\pgfpathlineto{\pgfqpoint{1.746868in}{0.739656in}}%
\pgfpathlineto{\pgfqpoint{1.746571in}{0.739656in}}%
\pgfpathlineto{\pgfqpoint{1.746273in}{0.739656in}}%
\pgfpathlineto{\pgfqpoint{1.745976in}{0.739656in}}%
\pgfpathlineto{\pgfqpoint{1.745678in}{0.739656in}}%
\pgfpathlineto{\pgfqpoint{1.745381in}{0.739656in}}%
\pgfpathlineto{\pgfqpoint{1.745083in}{0.739656in}}%
\pgfpathlineto{\pgfqpoint{1.744786in}{0.739656in}}%
\pgfpathlineto{\pgfqpoint{1.744488in}{0.739656in}}%
\pgfpathlineto{\pgfqpoint{1.744191in}{0.739656in}}%
\pgfpathlineto{\pgfqpoint{1.743893in}{0.739656in}}%
\pgfpathlineto{\pgfqpoint{1.743596in}{0.739656in}}%
\pgfpathlineto{\pgfqpoint{1.743298in}{0.739656in}}%
\pgfpathlineto{\pgfqpoint{1.743001in}{0.739656in}}%
\pgfpathlineto{\pgfqpoint{1.742703in}{0.739656in}}%
\pgfpathlineto{\pgfqpoint{1.742406in}{0.739656in}}%
\pgfpathlineto{\pgfqpoint{1.742108in}{0.739656in}}%
\pgfpathlineto{\pgfqpoint{1.741811in}{0.739656in}}%
\pgfpathlineto{\pgfqpoint{1.741513in}{0.739656in}}%
\pgfpathlineto{\pgfqpoint{1.741216in}{0.739656in}}%
\pgfpathlineto{\pgfqpoint{1.740919in}{0.739656in}}%
\pgfpathlineto{\pgfqpoint{1.740621in}{0.739656in}}%
\pgfpathlineto{\pgfqpoint{1.740324in}{0.739656in}}%
\pgfpathlineto{\pgfqpoint{1.740026in}{0.739656in}}%
\pgfpathlineto{\pgfqpoint{1.739729in}{0.739656in}}%
\pgfpathlineto{\pgfqpoint{1.739431in}{0.739656in}}%
\pgfpathlineto{\pgfqpoint{1.739134in}{0.739656in}}%
\pgfpathlineto{\pgfqpoint{1.738836in}{0.739656in}}%
\pgfpathlineto{\pgfqpoint{1.738539in}{0.739656in}}%
\pgfpathlineto{\pgfqpoint{1.738241in}{0.739656in}}%
\pgfpathlineto{\pgfqpoint{1.737944in}{0.739656in}}%
\pgfpathlineto{\pgfqpoint{1.737646in}{0.739656in}}%
\pgfpathlineto{\pgfqpoint{1.737349in}{0.739656in}}%
\pgfpathlineto{\pgfqpoint{1.737051in}{0.739656in}}%
\pgfpathlineto{\pgfqpoint{1.736754in}{0.739656in}}%
\pgfpathlineto{\pgfqpoint{1.736456in}{0.739656in}}%
\pgfpathlineto{\pgfqpoint{1.736159in}{0.739656in}}%
\pgfpathlineto{\pgfqpoint{1.735861in}{0.739656in}}%
\pgfpathlineto{\pgfqpoint{1.735564in}{0.739656in}}%
\pgfpathlineto{\pgfqpoint{1.735266in}{0.739656in}}%
\pgfpathlineto{\pgfqpoint{1.734969in}{0.739656in}}%
\pgfpathlineto{\pgfqpoint{1.734671in}{0.739656in}}%
\pgfpathlineto{\pgfqpoint{1.734374in}{0.739656in}}%
\pgfpathlineto{\pgfqpoint{1.734077in}{0.739656in}}%
\pgfpathlineto{\pgfqpoint{1.733779in}{0.739656in}}%
\pgfpathlineto{\pgfqpoint{1.733482in}{0.739656in}}%
\pgfpathlineto{\pgfqpoint{1.733184in}{0.739656in}}%
\pgfpathlineto{\pgfqpoint{1.732887in}{0.739656in}}%
\pgfpathlineto{\pgfqpoint{1.732589in}{0.739656in}}%
\pgfpathlineto{\pgfqpoint{1.732292in}{0.739656in}}%
\pgfpathlineto{\pgfqpoint{1.731994in}{0.739656in}}%
\pgfpathlineto{\pgfqpoint{1.731697in}{0.739656in}}%
\pgfpathlineto{\pgfqpoint{1.731399in}{0.739656in}}%
\pgfpathlineto{\pgfqpoint{1.731102in}{0.739656in}}%
\pgfpathlineto{\pgfqpoint{1.730804in}{0.739656in}}%
\pgfpathlineto{\pgfqpoint{1.730507in}{0.739656in}}%
\pgfpathlineto{\pgfqpoint{1.730209in}{0.739656in}}%
\pgfpathlineto{\pgfqpoint{1.729912in}{0.739656in}}%
\pgfpathlineto{\pgfqpoint{1.729614in}{0.739656in}}%
\pgfpathlineto{\pgfqpoint{1.729317in}{0.739656in}}%
\pgfpathlineto{\pgfqpoint{1.729019in}{0.739656in}}%
\pgfpathlineto{\pgfqpoint{1.728722in}{0.739656in}}%
\pgfpathlineto{\pgfqpoint{1.728424in}{0.739656in}}%
\pgfpathlineto{\pgfqpoint{1.728127in}{0.739656in}}%
\pgfpathlineto{\pgfqpoint{1.727830in}{0.739656in}}%
\pgfpathlineto{\pgfqpoint{1.727532in}{0.739656in}}%
\pgfpathlineto{\pgfqpoint{1.727235in}{0.739656in}}%
\pgfpathlineto{\pgfqpoint{1.726937in}{0.739656in}}%
\pgfpathlineto{\pgfqpoint{1.726640in}{0.739656in}}%
\pgfpathlineto{\pgfqpoint{1.726342in}{0.739656in}}%
\pgfpathlineto{\pgfqpoint{1.726045in}{0.739656in}}%
\pgfpathlineto{\pgfqpoint{1.725747in}{0.739656in}}%
\pgfpathlineto{\pgfqpoint{1.725450in}{0.739656in}}%
\pgfpathlineto{\pgfqpoint{1.725152in}{0.739656in}}%
\pgfpathlineto{\pgfqpoint{1.724855in}{0.739656in}}%
\pgfpathlineto{\pgfqpoint{1.724557in}{0.739656in}}%
\pgfpathlineto{\pgfqpoint{1.724260in}{0.739656in}}%
\pgfpathlineto{\pgfqpoint{1.723962in}{0.739656in}}%
\pgfpathlineto{\pgfqpoint{1.723665in}{0.739656in}}%
\pgfpathlineto{\pgfqpoint{1.723367in}{0.739656in}}%
\pgfpathlineto{\pgfqpoint{1.723070in}{0.739656in}}%
\pgfpathlineto{\pgfqpoint{1.722772in}{0.739656in}}%
\pgfpathlineto{\pgfqpoint{1.722475in}{0.739656in}}%
\pgfpathlineto{\pgfqpoint{1.722177in}{0.739656in}}%
\pgfpathlineto{\pgfqpoint{1.721880in}{0.739656in}}%
\pgfpathlineto{\pgfqpoint{1.721582in}{0.739656in}}%
\pgfpathlineto{\pgfqpoint{1.721285in}{0.739656in}}%
\pgfpathlineto{\pgfqpoint{1.720988in}{0.739656in}}%
\pgfpathlineto{\pgfqpoint{1.720690in}{0.739656in}}%
\pgfpathlineto{\pgfqpoint{1.720393in}{0.739656in}}%
\pgfpathlineto{\pgfqpoint{1.720095in}{0.739656in}}%
\pgfpathlineto{\pgfqpoint{1.719798in}{0.739656in}}%
\pgfpathlineto{\pgfqpoint{1.719500in}{0.739656in}}%
\pgfpathlineto{\pgfqpoint{1.719203in}{0.739656in}}%
\pgfpathlineto{\pgfqpoint{1.718905in}{0.739656in}}%
\pgfpathlineto{\pgfqpoint{1.718608in}{0.739656in}}%
\pgfpathlineto{\pgfqpoint{1.718310in}{0.739656in}}%
\pgfpathlineto{\pgfqpoint{1.718013in}{0.739656in}}%
\pgfpathlineto{\pgfqpoint{1.717715in}{0.739656in}}%
\pgfpathlineto{\pgfqpoint{1.717418in}{0.739656in}}%
\pgfpathlineto{\pgfqpoint{1.717120in}{0.739656in}}%
\pgfpathlineto{\pgfqpoint{1.716823in}{0.739656in}}%
\pgfpathlineto{\pgfqpoint{1.716525in}{0.739656in}}%
\pgfpathlineto{\pgfqpoint{1.716228in}{0.739656in}}%
\pgfpathlineto{\pgfqpoint{1.715930in}{0.739656in}}%
\pgfpathlineto{\pgfqpoint{1.715633in}{0.739656in}}%
\pgfpathlineto{\pgfqpoint{1.715335in}{0.739656in}}%
\pgfpathlineto{\pgfqpoint{1.715038in}{0.739656in}}%
\pgfpathlineto{\pgfqpoint{1.714740in}{0.739656in}}%
\pgfpathlineto{\pgfqpoint{1.714443in}{0.739656in}}%
\pgfpathlineto{\pgfqpoint{1.714146in}{0.739656in}}%
\pgfpathlineto{\pgfqpoint{1.713848in}{0.739656in}}%
\pgfpathlineto{\pgfqpoint{1.713551in}{0.739656in}}%
\pgfpathlineto{\pgfqpoint{1.713253in}{0.739656in}}%
\pgfpathlineto{\pgfqpoint{1.712956in}{0.739656in}}%
\pgfpathlineto{\pgfqpoint{1.712658in}{0.739656in}}%
\pgfpathlineto{\pgfqpoint{1.712361in}{0.739656in}}%
\pgfpathlineto{\pgfqpoint{1.712063in}{0.739656in}}%
\pgfpathlineto{\pgfqpoint{1.711766in}{0.739656in}}%
\pgfpathlineto{\pgfqpoint{1.711468in}{0.739656in}}%
\pgfpathlineto{\pgfqpoint{1.711171in}{0.739656in}}%
\pgfpathlineto{\pgfqpoint{1.710873in}{0.739656in}}%
\pgfpathlineto{\pgfqpoint{1.710576in}{0.739656in}}%
\pgfpathlineto{\pgfqpoint{1.710278in}{0.739656in}}%
\pgfpathlineto{\pgfqpoint{1.709981in}{0.739656in}}%
\pgfpathlineto{\pgfqpoint{1.709683in}{0.739656in}}%
\pgfpathlineto{\pgfqpoint{1.709386in}{0.739656in}}%
\pgfpathlineto{\pgfqpoint{1.709088in}{0.739656in}}%
\pgfpathlineto{\pgfqpoint{1.708791in}{0.739656in}}%
\pgfpathlineto{\pgfqpoint{1.708493in}{0.739656in}}%
\pgfpathlineto{\pgfqpoint{1.708196in}{0.739656in}}%
\pgfpathlineto{\pgfqpoint{1.707899in}{0.739656in}}%
\pgfpathlineto{\pgfqpoint{1.707601in}{0.739656in}}%
\pgfpathlineto{\pgfqpoint{1.707304in}{0.739656in}}%
\pgfpathlineto{\pgfqpoint{1.707006in}{0.739656in}}%
\pgfpathlineto{\pgfqpoint{1.706709in}{0.739656in}}%
\pgfpathlineto{\pgfqpoint{1.706411in}{0.739656in}}%
\pgfpathlineto{\pgfqpoint{1.706114in}{0.739656in}}%
\pgfpathlineto{\pgfqpoint{1.705816in}{0.739656in}}%
\pgfpathlineto{\pgfqpoint{1.705519in}{0.739656in}}%
\pgfpathlineto{\pgfqpoint{1.705221in}{0.739656in}}%
\pgfpathlineto{\pgfqpoint{1.704924in}{0.739656in}}%
\pgfpathlineto{\pgfqpoint{1.704626in}{0.739656in}}%
\pgfpathlineto{\pgfqpoint{1.704329in}{0.739656in}}%
\pgfpathlineto{\pgfqpoint{1.704031in}{0.739656in}}%
\pgfpathlineto{\pgfqpoint{1.703734in}{0.739656in}}%
\pgfpathlineto{\pgfqpoint{1.703436in}{0.739656in}}%
\pgfpathlineto{\pgfqpoint{1.703139in}{0.739656in}}%
\pgfpathlineto{\pgfqpoint{1.702841in}{0.739656in}}%
\pgfpathlineto{\pgfqpoint{1.702544in}{0.739656in}}%
\pgfpathlineto{\pgfqpoint{1.702246in}{0.739656in}}%
\pgfpathlineto{\pgfqpoint{1.701949in}{0.739656in}}%
\pgfpathlineto{\pgfqpoint{1.701651in}{0.739656in}}%
\pgfpathlineto{\pgfqpoint{1.701354in}{0.739656in}}%
\pgfpathlineto{\pgfqpoint{1.701057in}{0.739656in}}%
\pgfpathlineto{\pgfqpoint{1.700759in}{0.739656in}}%
\pgfpathlineto{\pgfqpoint{1.700462in}{0.739656in}}%
\pgfpathlineto{\pgfqpoint{1.700164in}{0.739656in}}%
\pgfpathlineto{\pgfqpoint{1.699867in}{0.739656in}}%
\pgfpathlineto{\pgfqpoint{1.699569in}{0.739656in}}%
\pgfpathlineto{\pgfqpoint{1.699272in}{0.739656in}}%
\pgfpathlineto{\pgfqpoint{1.698974in}{0.739656in}}%
\pgfpathlineto{\pgfqpoint{1.698677in}{0.739656in}}%
\pgfpathlineto{\pgfqpoint{1.698379in}{0.739656in}}%
\pgfpathlineto{\pgfqpoint{1.698082in}{0.739656in}}%
\pgfpathlineto{\pgfqpoint{1.697784in}{0.739656in}}%
\pgfpathlineto{\pgfqpoint{1.697487in}{0.739656in}}%
\pgfpathlineto{\pgfqpoint{1.697189in}{0.739656in}}%
\pgfpathlineto{\pgfqpoint{1.696892in}{0.739656in}}%
\pgfpathlineto{\pgfqpoint{1.696594in}{0.739656in}}%
\pgfpathlineto{\pgfqpoint{1.696297in}{0.739656in}}%
\pgfpathlineto{\pgfqpoint{1.695999in}{0.739656in}}%
\pgfpathlineto{\pgfqpoint{1.695702in}{0.739656in}}%
\pgfpathlineto{\pgfqpoint{1.695404in}{0.739656in}}%
\pgfpathlineto{\pgfqpoint{1.695107in}{0.739656in}}%
\pgfpathlineto{\pgfqpoint{1.694809in}{0.739656in}}%
\pgfpathlineto{\pgfqpoint{1.694512in}{0.739656in}}%
\pgfpathlineto{\pgfqpoint{1.694215in}{0.739656in}}%
\pgfpathlineto{\pgfqpoint{1.693917in}{0.739656in}}%
\pgfpathlineto{\pgfqpoint{1.693620in}{0.739656in}}%
\pgfpathlineto{\pgfqpoint{1.693322in}{0.739656in}}%
\pgfpathlineto{\pgfqpoint{1.693025in}{0.739656in}}%
\pgfpathlineto{\pgfqpoint{1.692727in}{0.739656in}}%
\pgfpathlineto{\pgfqpoint{1.692430in}{0.739656in}}%
\pgfpathlineto{\pgfqpoint{1.692132in}{0.739656in}}%
\pgfpathlineto{\pgfqpoint{1.691835in}{0.739656in}}%
\pgfpathlineto{\pgfqpoint{1.691537in}{0.739656in}}%
\pgfpathlineto{\pgfqpoint{1.691240in}{0.739656in}}%
\pgfpathlineto{\pgfqpoint{1.690942in}{0.739656in}}%
\pgfpathlineto{\pgfqpoint{1.690645in}{0.739656in}}%
\pgfpathlineto{\pgfqpoint{1.690347in}{0.739656in}}%
\pgfpathlineto{\pgfqpoint{1.690050in}{0.739656in}}%
\pgfpathlineto{\pgfqpoint{1.689752in}{0.739656in}}%
\pgfpathlineto{\pgfqpoint{1.689455in}{0.739656in}}%
\pgfpathlineto{\pgfqpoint{1.689157in}{0.739656in}}%
\pgfpathlineto{\pgfqpoint{1.688860in}{0.739656in}}%
\pgfpathlineto{\pgfqpoint{1.688562in}{0.739656in}}%
\pgfpathlineto{\pgfqpoint{1.688265in}{0.739656in}}%
\pgfpathlineto{\pgfqpoint{1.687968in}{0.739656in}}%
\pgfpathlineto{\pgfqpoint{1.687670in}{0.739656in}}%
\pgfpathlineto{\pgfqpoint{1.687373in}{0.739656in}}%
\pgfpathlineto{\pgfqpoint{1.687075in}{0.739656in}}%
\pgfpathlineto{\pgfqpoint{1.686778in}{0.739656in}}%
\pgfpathlineto{\pgfqpoint{1.686480in}{0.739656in}}%
\pgfpathlineto{\pgfqpoint{1.686183in}{0.739656in}}%
\pgfpathlineto{\pgfqpoint{1.685885in}{0.739656in}}%
\pgfpathlineto{\pgfqpoint{1.685588in}{0.739656in}}%
\pgfpathlineto{\pgfqpoint{1.685290in}{0.739656in}}%
\pgfpathlineto{\pgfqpoint{1.684993in}{0.739656in}}%
\pgfpathlineto{\pgfqpoint{1.684695in}{0.739656in}}%
\pgfpathlineto{\pgfqpoint{1.684398in}{0.739656in}}%
\pgfpathlineto{\pgfqpoint{1.684100in}{0.739656in}}%
\pgfpathlineto{\pgfqpoint{1.683803in}{0.739656in}}%
\pgfpathlineto{\pgfqpoint{1.683505in}{0.739656in}}%
\pgfpathlineto{\pgfqpoint{1.683208in}{0.739656in}}%
\pgfpathlineto{\pgfqpoint{1.682910in}{0.739656in}}%
\pgfpathlineto{\pgfqpoint{1.682613in}{0.739656in}}%
\pgfpathlineto{\pgfqpoint{1.682315in}{0.739656in}}%
\pgfpathlineto{\pgfqpoint{1.682018in}{0.739656in}}%
\pgfpathlineto{\pgfqpoint{1.681720in}{0.739656in}}%
\pgfpathlineto{\pgfqpoint{1.681423in}{0.739656in}}%
\pgfpathlineto{\pgfqpoint{1.681126in}{0.739656in}}%
\pgfpathlineto{\pgfqpoint{1.680828in}{0.739656in}}%
\pgfpathlineto{\pgfqpoint{1.680531in}{0.739656in}}%
\pgfpathlineto{\pgfqpoint{1.680233in}{0.739656in}}%
\pgfpathlineto{\pgfqpoint{1.679936in}{0.739656in}}%
\pgfpathlineto{\pgfqpoint{1.679638in}{0.739656in}}%
\pgfpathlineto{\pgfqpoint{1.679341in}{0.739656in}}%
\pgfpathlineto{\pgfqpoint{1.679043in}{0.739656in}}%
\pgfpathlineto{\pgfqpoint{1.678746in}{0.739656in}}%
\pgfpathlineto{\pgfqpoint{1.678448in}{0.739656in}}%
\pgfpathlineto{\pgfqpoint{1.678151in}{0.739656in}}%
\pgfpathlineto{\pgfqpoint{1.677853in}{0.739656in}}%
\pgfpathlineto{\pgfqpoint{1.677556in}{0.739656in}}%
\pgfpathlineto{\pgfqpoint{1.677258in}{0.739656in}}%
\pgfpathlineto{\pgfqpoint{1.676961in}{0.739656in}}%
\pgfpathlineto{\pgfqpoint{1.676663in}{0.739656in}}%
\pgfpathlineto{\pgfqpoint{1.676366in}{0.739656in}}%
\pgfpathlineto{\pgfqpoint{1.676068in}{0.739656in}}%
\pgfpathlineto{\pgfqpoint{1.675771in}{0.739656in}}%
\pgfpathlineto{\pgfqpoint{1.675473in}{0.739656in}}%
\pgfpathlineto{\pgfqpoint{1.675176in}{0.739656in}}%
\pgfpathlineto{\pgfqpoint{1.674878in}{0.739656in}}%
\pgfpathlineto{\pgfqpoint{1.674581in}{0.739656in}}%
\pgfpathlineto{\pgfqpoint{1.674284in}{0.739656in}}%
\pgfpathlineto{\pgfqpoint{1.673986in}{0.739656in}}%
\pgfpathlineto{\pgfqpoint{1.673689in}{0.739656in}}%
\pgfpathlineto{\pgfqpoint{1.673391in}{0.739656in}}%
\pgfpathlineto{\pgfqpoint{1.673094in}{0.739656in}}%
\pgfpathlineto{\pgfqpoint{1.672796in}{0.739656in}}%
\pgfpathlineto{\pgfqpoint{1.672499in}{0.739656in}}%
\pgfpathlineto{\pgfqpoint{1.672201in}{0.739656in}}%
\pgfpathlineto{\pgfqpoint{1.671904in}{0.739656in}}%
\pgfpathlineto{\pgfqpoint{1.671606in}{0.739656in}}%
\pgfpathlineto{\pgfqpoint{1.671309in}{0.739656in}}%
\pgfpathlineto{\pgfqpoint{1.671011in}{0.739656in}}%
\pgfpathlineto{\pgfqpoint{1.670714in}{0.739656in}}%
\pgfpathlineto{\pgfqpoint{1.670416in}{0.739656in}}%
\pgfpathlineto{\pgfqpoint{1.670119in}{0.739656in}}%
\pgfpathlineto{\pgfqpoint{1.669821in}{0.739656in}}%
\pgfpathlineto{\pgfqpoint{1.669524in}{0.739656in}}%
\pgfpathlineto{\pgfqpoint{1.669226in}{0.739656in}}%
\pgfpathlineto{\pgfqpoint{1.668929in}{0.739656in}}%
\pgfpathlineto{\pgfqpoint{1.668631in}{0.739656in}}%
\pgfpathlineto{\pgfqpoint{1.668334in}{0.739656in}}%
\pgfpathlineto{\pgfqpoint{1.668037in}{0.739656in}}%
\pgfpathlineto{\pgfqpoint{1.667739in}{0.739656in}}%
\pgfpathlineto{\pgfqpoint{1.667442in}{0.739656in}}%
\pgfpathlineto{\pgfqpoint{1.667144in}{0.739656in}}%
\pgfpathlineto{\pgfqpoint{1.666847in}{0.739656in}}%
\pgfpathlineto{\pgfqpoint{1.666549in}{0.739656in}}%
\pgfpathlineto{\pgfqpoint{1.666252in}{0.739656in}}%
\pgfpathlineto{\pgfqpoint{1.665954in}{0.739656in}}%
\pgfpathlineto{\pgfqpoint{1.665657in}{0.739656in}}%
\pgfpathlineto{\pgfqpoint{1.665359in}{0.739656in}}%
\pgfpathlineto{\pgfqpoint{1.665062in}{0.739656in}}%
\pgfpathlineto{\pgfqpoint{1.664764in}{0.739656in}}%
\pgfpathlineto{\pgfqpoint{1.664467in}{0.739656in}}%
\pgfpathlineto{\pgfqpoint{1.664169in}{0.739656in}}%
\pgfpathlineto{\pgfqpoint{1.663872in}{0.739656in}}%
\pgfpathlineto{\pgfqpoint{1.663574in}{0.739656in}}%
\pgfpathlineto{\pgfqpoint{1.663277in}{0.739656in}}%
\pgfpathlineto{\pgfqpoint{1.662979in}{0.739656in}}%
\pgfpathlineto{\pgfqpoint{1.662682in}{0.739656in}}%
\pgfpathlineto{\pgfqpoint{1.662384in}{0.739656in}}%
\pgfpathlineto{\pgfqpoint{1.662087in}{0.739656in}}%
\pgfpathlineto{\pgfqpoint{1.661789in}{0.739656in}}%
\pgfpathlineto{\pgfqpoint{1.661492in}{0.739656in}}%
\pgfpathlineto{\pgfqpoint{1.661195in}{0.739656in}}%
\pgfpathlineto{\pgfqpoint{1.660897in}{0.739656in}}%
\pgfpathlineto{\pgfqpoint{1.660600in}{0.739656in}}%
\pgfpathlineto{\pgfqpoint{1.660302in}{0.739656in}}%
\pgfpathlineto{\pgfqpoint{1.660005in}{0.739656in}}%
\pgfpathlineto{\pgfqpoint{1.659707in}{0.739656in}}%
\pgfpathlineto{\pgfqpoint{1.659410in}{0.739656in}}%
\pgfpathlineto{\pgfqpoint{1.659112in}{0.739656in}}%
\pgfpathlineto{\pgfqpoint{1.658815in}{0.739656in}}%
\pgfpathlineto{\pgfqpoint{1.658517in}{0.739656in}}%
\pgfpathlineto{\pgfqpoint{1.658220in}{0.739656in}}%
\pgfpathlineto{\pgfqpoint{1.657922in}{0.739656in}}%
\pgfpathlineto{\pgfqpoint{1.657625in}{0.739656in}}%
\pgfpathlineto{\pgfqpoint{1.657327in}{0.739656in}}%
\pgfpathlineto{\pgfqpoint{1.657030in}{0.739656in}}%
\pgfpathlineto{\pgfqpoint{1.656732in}{0.739656in}}%
\pgfpathlineto{\pgfqpoint{1.656435in}{0.739656in}}%
\pgfpathlineto{\pgfqpoint{1.656137in}{0.739656in}}%
\pgfpathlineto{\pgfqpoint{1.655840in}{0.739656in}}%
\pgfpathlineto{\pgfqpoint{1.655542in}{0.739656in}}%
\pgfpathlineto{\pgfqpoint{1.655245in}{0.739656in}}%
\pgfpathlineto{\pgfqpoint{1.654947in}{0.739656in}}%
\pgfpathlineto{\pgfqpoint{1.654650in}{0.739656in}}%
\pgfpathlineto{\pgfqpoint{1.654353in}{0.739656in}}%
\pgfpathlineto{\pgfqpoint{1.654055in}{0.739656in}}%
\pgfpathlineto{\pgfqpoint{1.653758in}{0.739656in}}%
\pgfpathlineto{\pgfqpoint{1.653460in}{0.739656in}}%
\pgfpathlineto{\pgfqpoint{1.653163in}{0.739656in}}%
\pgfpathlineto{\pgfqpoint{1.652865in}{0.739656in}}%
\pgfpathlineto{\pgfqpoint{1.652568in}{0.739656in}}%
\pgfpathlineto{\pgfqpoint{1.652270in}{0.739656in}}%
\pgfpathlineto{\pgfqpoint{1.651973in}{0.739656in}}%
\pgfpathlineto{\pgfqpoint{1.651675in}{0.739656in}}%
\pgfpathlineto{\pgfqpoint{1.651378in}{0.739656in}}%
\pgfpathlineto{\pgfqpoint{1.651080in}{0.739656in}}%
\pgfpathlineto{\pgfqpoint{1.650783in}{0.739656in}}%
\pgfpathlineto{\pgfqpoint{1.650485in}{0.739656in}}%
\pgfpathlineto{\pgfqpoint{1.650188in}{0.739656in}}%
\pgfpathlineto{\pgfqpoint{1.649890in}{0.739656in}}%
\pgfpathlineto{\pgfqpoint{1.649593in}{0.739656in}}%
\pgfpathlineto{\pgfqpoint{1.649295in}{0.739656in}}%
\pgfpathlineto{\pgfqpoint{1.648998in}{0.739656in}}%
\pgfpathlineto{\pgfqpoint{1.648700in}{0.739656in}}%
\pgfpathlineto{\pgfqpoint{1.648403in}{0.739656in}}%
\pgfpathlineto{\pgfqpoint{1.648106in}{0.739656in}}%
\pgfpathlineto{\pgfqpoint{1.647808in}{0.739656in}}%
\pgfpathlineto{\pgfqpoint{1.647511in}{0.739656in}}%
\pgfpathlineto{\pgfqpoint{1.647213in}{0.739656in}}%
\pgfpathlineto{\pgfqpoint{1.646916in}{0.739656in}}%
\pgfpathlineto{\pgfqpoint{1.646618in}{0.739656in}}%
\pgfpathlineto{\pgfqpoint{1.646321in}{0.739656in}}%
\pgfpathlineto{\pgfqpoint{1.646023in}{0.739656in}}%
\pgfpathlineto{\pgfqpoint{1.645726in}{0.739656in}}%
\pgfpathlineto{\pgfqpoint{1.645428in}{0.739656in}}%
\pgfpathlineto{\pgfqpoint{1.645131in}{0.739656in}}%
\pgfpathlineto{\pgfqpoint{1.644833in}{0.739656in}}%
\pgfpathlineto{\pgfqpoint{1.644536in}{0.739656in}}%
\pgfpathlineto{\pgfqpoint{1.644238in}{0.739656in}}%
\pgfpathlineto{\pgfqpoint{1.643941in}{0.739656in}}%
\pgfpathlineto{\pgfqpoint{1.643643in}{0.739656in}}%
\pgfpathlineto{\pgfqpoint{1.643346in}{0.739656in}}%
\pgfpathlineto{\pgfqpoint{1.643048in}{0.739656in}}%
\pgfpathlineto{\pgfqpoint{1.642751in}{0.739656in}}%
\pgfpathlineto{\pgfqpoint{1.642453in}{0.739656in}}%
\pgfpathlineto{\pgfqpoint{1.642156in}{0.739656in}}%
\pgfpathlineto{\pgfqpoint{1.641858in}{0.739656in}}%
\pgfpathlineto{\pgfqpoint{1.641561in}{0.739656in}}%
\pgfpathlineto{\pgfqpoint{1.641264in}{0.739656in}}%
\pgfpathlineto{\pgfqpoint{1.640966in}{0.739656in}}%
\pgfpathlineto{\pgfqpoint{1.640669in}{0.739656in}}%
\pgfpathlineto{\pgfqpoint{1.640371in}{0.739656in}}%
\pgfpathlineto{\pgfqpoint{1.640074in}{0.739656in}}%
\pgfpathlineto{\pgfqpoint{1.639776in}{0.739656in}}%
\pgfpathlineto{\pgfqpoint{1.639479in}{0.739656in}}%
\pgfpathlineto{\pgfqpoint{1.639181in}{0.739656in}}%
\pgfpathlineto{\pgfqpoint{1.638884in}{0.739656in}}%
\pgfpathlineto{\pgfqpoint{1.638586in}{0.739656in}}%
\pgfpathlineto{\pgfqpoint{1.638289in}{0.739656in}}%
\pgfpathlineto{\pgfqpoint{1.637991in}{0.739656in}}%
\pgfpathlineto{\pgfqpoint{1.637694in}{0.739656in}}%
\pgfpathlineto{\pgfqpoint{1.637396in}{0.739656in}}%
\pgfpathlineto{\pgfqpoint{1.637099in}{0.739656in}}%
\pgfpathlineto{\pgfqpoint{1.636801in}{0.739656in}}%
\pgfpathlineto{\pgfqpoint{1.636504in}{0.739656in}}%
\pgfpathlineto{\pgfqpoint{1.636206in}{0.739656in}}%
\pgfpathlineto{\pgfqpoint{1.635909in}{0.739656in}}%
\pgfpathlineto{\pgfqpoint{1.635611in}{0.739656in}}%
\pgfpathlineto{\pgfqpoint{1.635314in}{0.739656in}}%
\pgfpathlineto{\pgfqpoint{1.635016in}{0.739656in}}%
\pgfpathlineto{\pgfqpoint{1.634719in}{0.739656in}}%
\pgfpathlineto{\pgfqpoint{1.634422in}{0.739656in}}%
\pgfpathlineto{\pgfqpoint{1.634124in}{0.739656in}}%
\pgfpathlineto{\pgfqpoint{1.633827in}{0.739656in}}%
\pgfpathlineto{\pgfqpoint{1.633529in}{0.739656in}}%
\pgfpathlineto{\pgfqpoint{1.633232in}{0.739656in}}%
\pgfpathlineto{\pgfqpoint{1.632934in}{0.739656in}}%
\pgfpathlineto{\pgfqpoint{1.632637in}{0.739656in}}%
\pgfpathlineto{\pgfqpoint{1.632339in}{0.739656in}}%
\pgfpathlineto{\pgfqpoint{1.632042in}{0.739656in}}%
\pgfpathlineto{\pgfqpoint{1.631744in}{0.739656in}}%
\pgfpathlineto{\pgfqpoint{1.631447in}{0.739656in}}%
\pgfpathlineto{\pgfqpoint{1.631149in}{0.739656in}}%
\pgfpathlineto{\pgfqpoint{1.630852in}{0.739656in}}%
\pgfpathlineto{\pgfqpoint{1.630554in}{0.739656in}}%
\pgfpathlineto{\pgfqpoint{1.630257in}{0.739656in}}%
\pgfpathlineto{\pgfqpoint{1.629959in}{0.739656in}}%
\pgfpathlineto{\pgfqpoint{1.629662in}{0.739656in}}%
\pgfpathlineto{\pgfqpoint{1.629364in}{0.739656in}}%
\pgfpathlineto{\pgfqpoint{1.629067in}{0.739656in}}%
\pgfpathlineto{\pgfqpoint{1.628769in}{0.739656in}}%
\pgfpathlineto{\pgfqpoint{1.628472in}{0.739656in}}%
\pgfpathlineto{\pgfqpoint{1.628175in}{0.739656in}}%
\pgfpathlineto{\pgfqpoint{1.627877in}{0.739656in}}%
\pgfpathlineto{\pgfqpoint{1.627580in}{0.739656in}}%
\pgfpathlineto{\pgfqpoint{1.627282in}{0.739656in}}%
\pgfpathlineto{\pgfqpoint{1.626985in}{0.739656in}}%
\pgfpathlineto{\pgfqpoint{1.626687in}{0.739656in}}%
\pgfpathlineto{\pgfqpoint{1.626390in}{0.739656in}}%
\pgfpathlineto{\pgfqpoint{1.626092in}{0.739656in}}%
\pgfpathlineto{\pgfqpoint{1.625795in}{0.739656in}}%
\pgfpathlineto{\pgfqpoint{1.625497in}{0.739656in}}%
\pgfpathlineto{\pgfqpoint{1.625200in}{0.739656in}}%
\pgfpathlineto{\pgfqpoint{1.624902in}{0.739656in}}%
\pgfpathlineto{\pgfqpoint{1.624605in}{0.739656in}}%
\pgfpathlineto{\pgfqpoint{1.624307in}{0.739656in}}%
\pgfpathlineto{\pgfqpoint{1.624010in}{0.739656in}}%
\pgfpathlineto{\pgfqpoint{1.623712in}{0.739656in}}%
\pgfpathlineto{\pgfqpoint{1.623415in}{0.739656in}}%
\pgfpathlineto{\pgfqpoint{1.623117in}{0.739656in}}%
\pgfpathlineto{\pgfqpoint{1.622820in}{0.739656in}}%
\pgfpathlineto{\pgfqpoint{1.622522in}{0.739656in}}%
\pgfpathlineto{\pgfqpoint{1.622225in}{0.739656in}}%
\pgfpathlineto{\pgfqpoint{1.621927in}{0.739656in}}%
\pgfpathlineto{\pgfqpoint{1.621630in}{0.739656in}}%
\pgfpathlineto{\pgfqpoint{1.621333in}{0.739656in}}%
\pgfpathlineto{\pgfqpoint{1.621035in}{0.739656in}}%
\pgfpathlineto{\pgfqpoint{1.620738in}{0.739656in}}%
\pgfpathlineto{\pgfqpoint{1.620440in}{0.739656in}}%
\pgfpathlineto{\pgfqpoint{1.620143in}{0.739656in}}%
\pgfpathlineto{\pgfqpoint{1.619845in}{0.739656in}}%
\pgfpathlineto{\pgfqpoint{1.619548in}{0.739656in}}%
\pgfpathlineto{\pgfqpoint{1.619250in}{0.739656in}}%
\pgfpathlineto{\pgfqpoint{1.618953in}{0.739656in}}%
\pgfpathlineto{\pgfqpoint{1.618655in}{0.739656in}}%
\pgfpathlineto{\pgfqpoint{1.618358in}{0.739656in}}%
\pgfpathlineto{\pgfqpoint{1.618060in}{0.739656in}}%
\pgfpathlineto{\pgfqpoint{1.617763in}{0.739656in}}%
\pgfpathlineto{\pgfqpoint{1.617465in}{0.739656in}}%
\pgfpathlineto{\pgfqpoint{1.617168in}{0.739656in}}%
\pgfpathlineto{\pgfqpoint{1.616870in}{0.739656in}}%
\pgfpathlineto{\pgfqpoint{1.616573in}{0.739656in}}%
\pgfpathlineto{\pgfqpoint{1.616275in}{0.739656in}}%
\pgfpathlineto{\pgfqpoint{1.615978in}{0.739656in}}%
\pgfpathlineto{\pgfqpoint{1.615680in}{0.739656in}}%
\pgfpathlineto{\pgfqpoint{1.615383in}{0.739656in}}%
\pgfpathlineto{\pgfqpoint{1.615085in}{0.739656in}}%
\pgfpathlineto{\pgfqpoint{1.614788in}{0.739656in}}%
\pgfpathlineto{\pgfqpoint{1.614491in}{0.739656in}}%
\pgfpathlineto{\pgfqpoint{1.614193in}{0.739656in}}%
\pgfpathlineto{\pgfqpoint{1.613896in}{0.739656in}}%
\pgfpathlineto{\pgfqpoint{1.613598in}{0.739656in}}%
\pgfpathlineto{\pgfqpoint{1.613301in}{0.739656in}}%
\pgfpathlineto{\pgfqpoint{1.613003in}{0.739656in}}%
\pgfpathlineto{\pgfqpoint{1.612706in}{0.739656in}}%
\pgfpathlineto{\pgfqpoint{1.612408in}{0.739656in}}%
\pgfpathlineto{\pgfqpoint{1.612111in}{0.739656in}}%
\pgfpathlineto{\pgfqpoint{1.611813in}{0.739656in}}%
\pgfpathlineto{\pgfqpoint{1.611516in}{0.739656in}}%
\pgfpathlineto{\pgfqpoint{1.611218in}{0.739656in}}%
\pgfpathlineto{\pgfqpoint{1.610921in}{0.739656in}}%
\pgfpathlineto{\pgfqpoint{1.610623in}{0.739656in}}%
\pgfpathlineto{\pgfqpoint{1.610326in}{0.739656in}}%
\pgfpathlineto{\pgfqpoint{1.610028in}{0.739656in}}%
\pgfpathlineto{\pgfqpoint{1.609731in}{0.739656in}}%
\pgfpathlineto{\pgfqpoint{1.609433in}{0.739656in}}%
\pgfpathlineto{\pgfqpoint{1.609136in}{0.739656in}}%
\pgfpathlineto{\pgfqpoint{1.608838in}{0.739656in}}%
\pgfpathlineto{\pgfqpoint{1.608541in}{0.739656in}}%
\pgfpathlineto{\pgfqpoint{1.608244in}{0.739656in}}%
\pgfpathlineto{\pgfqpoint{1.607946in}{0.739656in}}%
\pgfpathlineto{\pgfqpoint{1.607649in}{0.739656in}}%
\pgfpathlineto{\pgfqpoint{1.607351in}{0.739656in}}%
\pgfpathlineto{\pgfqpoint{1.607054in}{0.739656in}}%
\pgfpathlineto{\pgfqpoint{1.606756in}{0.739656in}}%
\pgfpathlineto{\pgfqpoint{1.606459in}{0.739656in}}%
\pgfpathlineto{\pgfqpoint{1.606161in}{0.739656in}}%
\pgfpathlineto{\pgfqpoint{1.605864in}{0.739656in}}%
\pgfpathlineto{\pgfqpoint{1.605566in}{0.739656in}}%
\pgfpathlineto{\pgfqpoint{1.605269in}{0.739656in}}%
\pgfpathlineto{\pgfqpoint{1.604971in}{0.739656in}}%
\pgfpathlineto{\pgfqpoint{1.604674in}{0.739656in}}%
\pgfpathlineto{\pgfqpoint{1.604376in}{0.739656in}}%
\pgfpathlineto{\pgfqpoint{1.604079in}{0.739656in}}%
\pgfpathlineto{\pgfqpoint{1.603781in}{0.739656in}}%
\pgfpathlineto{\pgfqpoint{1.603484in}{0.739656in}}%
\pgfpathlineto{\pgfqpoint{1.603186in}{0.739656in}}%
\pgfpathlineto{\pgfqpoint{1.602889in}{0.739656in}}%
\pgfpathlineto{\pgfqpoint{1.602591in}{0.739656in}}%
\pgfpathlineto{\pgfqpoint{1.602294in}{0.739656in}}%
\pgfpathlineto{\pgfqpoint{1.601996in}{0.739656in}}%
\pgfpathlineto{\pgfqpoint{1.601699in}{0.739656in}}%
\pgfpathlineto{\pgfqpoint{1.601402in}{0.739656in}}%
\pgfpathlineto{\pgfqpoint{1.601104in}{0.739656in}}%
\pgfpathlineto{\pgfqpoint{1.600807in}{0.739656in}}%
\pgfpathlineto{\pgfqpoint{1.600509in}{0.739656in}}%
\pgfpathlineto{\pgfqpoint{1.600212in}{0.739656in}}%
\pgfpathlineto{\pgfqpoint{1.599914in}{0.739656in}}%
\pgfpathlineto{\pgfqpoint{1.599617in}{0.739656in}}%
\pgfpathlineto{\pgfqpoint{1.599319in}{0.739656in}}%
\pgfpathlineto{\pgfqpoint{1.599022in}{0.739656in}}%
\pgfpathlineto{\pgfqpoint{1.598724in}{0.739656in}}%
\pgfpathlineto{\pgfqpoint{1.598427in}{0.739656in}}%
\pgfpathlineto{\pgfqpoint{1.598129in}{0.739656in}}%
\pgfpathlineto{\pgfqpoint{1.597832in}{0.739656in}}%
\pgfpathlineto{\pgfqpoint{1.597534in}{0.739656in}}%
\pgfpathlineto{\pgfqpoint{1.597237in}{0.739656in}}%
\pgfpathlineto{\pgfqpoint{1.596939in}{0.739656in}}%
\pgfpathlineto{\pgfqpoint{1.596642in}{0.739656in}}%
\pgfpathlineto{\pgfqpoint{1.596344in}{0.739656in}}%
\pgfpathlineto{\pgfqpoint{1.596047in}{0.739656in}}%
\pgfpathlineto{\pgfqpoint{1.595749in}{0.739656in}}%
\pgfpathlineto{\pgfqpoint{1.595452in}{0.739656in}}%
\pgfpathlineto{\pgfqpoint{1.595154in}{0.739656in}}%
\pgfpathlineto{\pgfqpoint{1.594857in}{0.739656in}}%
\pgfpathlineto{\pgfqpoint{1.594560in}{0.739656in}}%
\pgfpathlineto{\pgfqpoint{1.594262in}{0.739656in}}%
\pgfpathlineto{\pgfqpoint{1.593965in}{0.739656in}}%
\pgfpathlineto{\pgfqpoint{1.593667in}{0.739656in}}%
\pgfpathlineto{\pgfqpoint{1.593370in}{0.739656in}}%
\pgfpathlineto{\pgfqpoint{1.593072in}{0.739656in}}%
\pgfpathlineto{\pgfqpoint{1.592775in}{0.739656in}}%
\pgfpathlineto{\pgfqpoint{1.592477in}{0.739656in}}%
\pgfpathlineto{\pgfqpoint{1.592180in}{0.739656in}}%
\pgfpathlineto{\pgfqpoint{1.591882in}{0.739656in}}%
\pgfpathlineto{\pgfqpoint{1.591585in}{0.739656in}}%
\pgfpathlineto{\pgfqpoint{1.591287in}{0.739656in}}%
\pgfpathlineto{\pgfqpoint{1.590990in}{0.739656in}}%
\pgfpathlineto{\pgfqpoint{1.590692in}{0.739656in}}%
\pgfpathlineto{\pgfqpoint{1.590395in}{0.739656in}}%
\pgfpathlineto{\pgfqpoint{1.590097in}{0.739656in}}%
\pgfpathlineto{\pgfqpoint{1.589800in}{0.739656in}}%
\pgfpathlineto{\pgfqpoint{1.589502in}{0.739656in}}%
\pgfpathlineto{\pgfqpoint{1.589205in}{0.739656in}}%
\pgfpathlineto{\pgfqpoint{1.588907in}{0.739656in}}%
\pgfpathlineto{\pgfqpoint{1.588610in}{0.739656in}}%
\pgfpathlineto{\pgfqpoint{1.588313in}{0.739656in}}%
\pgfpathlineto{\pgfqpoint{1.588015in}{0.739656in}}%
\pgfpathlineto{\pgfqpoint{1.587718in}{0.739656in}}%
\pgfpathlineto{\pgfqpoint{1.587420in}{0.739656in}}%
\pgfpathlineto{\pgfqpoint{1.587123in}{0.739656in}}%
\pgfpathlineto{\pgfqpoint{1.586825in}{0.739656in}}%
\pgfpathlineto{\pgfqpoint{1.586528in}{0.739656in}}%
\pgfpathlineto{\pgfqpoint{1.586230in}{0.739656in}}%
\pgfpathlineto{\pgfqpoint{1.585933in}{0.739656in}}%
\pgfpathlineto{\pgfqpoint{1.585635in}{0.739656in}}%
\pgfpathlineto{\pgfqpoint{1.585338in}{0.739656in}}%
\pgfpathlineto{\pgfqpoint{1.585040in}{0.739656in}}%
\pgfpathlineto{\pgfqpoint{1.584743in}{0.739656in}}%
\pgfpathlineto{\pgfqpoint{1.584445in}{0.739656in}}%
\pgfpathlineto{\pgfqpoint{1.584148in}{0.739656in}}%
\pgfpathlineto{\pgfqpoint{1.583850in}{0.739656in}}%
\pgfpathlineto{\pgfqpoint{1.583553in}{0.739656in}}%
\pgfpathlineto{\pgfqpoint{1.583255in}{0.739656in}}%
\pgfpathlineto{\pgfqpoint{1.582958in}{0.739656in}}%
\pgfpathlineto{\pgfqpoint{1.582660in}{0.739656in}}%
\pgfpathlineto{\pgfqpoint{1.582363in}{0.739656in}}%
\pgfpathlineto{\pgfqpoint{1.582065in}{0.739656in}}%
\pgfpathlineto{\pgfqpoint{1.581768in}{0.739656in}}%
\pgfpathlineto{\pgfqpoint{1.581471in}{0.739656in}}%
\pgfpathlineto{\pgfqpoint{1.581173in}{0.739656in}}%
\pgfpathlineto{\pgfqpoint{1.580876in}{0.739656in}}%
\pgfpathlineto{\pgfqpoint{1.580578in}{0.739656in}}%
\pgfpathlineto{\pgfqpoint{1.580281in}{0.739656in}}%
\pgfpathlineto{\pgfqpoint{1.579983in}{0.739656in}}%
\pgfpathlineto{\pgfqpoint{1.579686in}{0.739656in}}%
\pgfpathlineto{\pgfqpoint{1.579388in}{0.739656in}}%
\pgfpathlineto{\pgfqpoint{1.579091in}{0.739656in}}%
\pgfpathlineto{\pgfqpoint{1.578793in}{0.739656in}}%
\pgfpathlineto{\pgfqpoint{1.578496in}{0.739656in}}%
\pgfpathlineto{\pgfqpoint{1.578198in}{0.739656in}}%
\pgfpathlineto{\pgfqpoint{1.577901in}{0.739656in}}%
\pgfpathlineto{\pgfqpoint{1.577603in}{0.739656in}}%
\pgfpathlineto{\pgfqpoint{1.577306in}{0.739656in}}%
\pgfpathlineto{\pgfqpoint{1.577008in}{0.739656in}}%
\pgfpathlineto{\pgfqpoint{1.576711in}{0.739656in}}%
\pgfpathlineto{\pgfqpoint{1.576413in}{0.739656in}}%
\pgfpathlineto{\pgfqpoint{1.576116in}{0.739656in}}%
\pgfpathlineto{\pgfqpoint{1.575818in}{0.739656in}}%
\pgfpathlineto{\pgfqpoint{1.575521in}{0.739656in}}%
\pgfpathlineto{\pgfqpoint{1.575223in}{0.739656in}}%
\pgfpathlineto{\pgfqpoint{1.574926in}{0.739656in}}%
\pgfpathlineto{\pgfqpoint{1.574629in}{0.739656in}}%
\pgfpathlineto{\pgfqpoint{1.574331in}{0.739656in}}%
\pgfpathlineto{\pgfqpoint{1.574034in}{0.739656in}}%
\pgfpathlineto{\pgfqpoint{1.573736in}{0.739656in}}%
\pgfpathlineto{\pgfqpoint{1.573439in}{0.739656in}}%
\pgfpathlineto{\pgfqpoint{1.573141in}{0.739656in}}%
\pgfpathlineto{\pgfqpoint{1.572844in}{0.739656in}}%
\pgfpathlineto{\pgfqpoint{1.572546in}{0.739656in}}%
\pgfpathlineto{\pgfqpoint{1.572249in}{0.739656in}}%
\pgfpathlineto{\pgfqpoint{1.571951in}{0.739656in}}%
\pgfpathlineto{\pgfqpoint{1.571654in}{0.739656in}}%
\pgfpathlineto{\pgfqpoint{1.571356in}{0.739656in}}%
\pgfpathlineto{\pgfqpoint{1.571059in}{0.739656in}}%
\pgfpathlineto{\pgfqpoint{1.570761in}{0.739656in}}%
\pgfpathlineto{\pgfqpoint{1.570464in}{0.739656in}}%
\pgfpathlineto{\pgfqpoint{1.570166in}{0.739656in}}%
\pgfpathlineto{\pgfqpoint{1.569869in}{0.739656in}}%
\pgfpathlineto{\pgfqpoint{1.569571in}{0.739656in}}%
\pgfpathlineto{\pgfqpoint{1.569274in}{0.739656in}}%
\pgfpathlineto{\pgfqpoint{1.568976in}{0.739656in}}%
\pgfpathlineto{\pgfqpoint{1.568679in}{0.739656in}}%
\pgfpathlineto{\pgfqpoint{1.568381in}{0.739656in}}%
\pgfpathlineto{\pgfqpoint{1.568084in}{0.739656in}}%
\pgfpathlineto{\pgfqpoint{1.567787in}{0.739656in}}%
\pgfpathlineto{\pgfqpoint{1.567489in}{0.739656in}}%
\pgfpathlineto{\pgfqpoint{1.567192in}{0.739656in}}%
\pgfpathlineto{\pgfqpoint{1.566894in}{0.739656in}}%
\pgfpathlineto{\pgfqpoint{1.566597in}{0.739656in}}%
\pgfpathlineto{\pgfqpoint{1.566299in}{0.739656in}}%
\pgfpathlineto{\pgfqpoint{1.566002in}{0.739656in}}%
\pgfpathlineto{\pgfqpoint{1.565704in}{0.739656in}}%
\pgfpathlineto{\pgfqpoint{1.565407in}{0.739656in}}%
\pgfpathlineto{\pgfqpoint{1.565109in}{0.739656in}}%
\pgfpathlineto{\pgfqpoint{1.564812in}{0.739656in}}%
\pgfpathlineto{\pgfqpoint{1.564514in}{0.739656in}}%
\pgfpathlineto{\pgfqpoint{1.564217in}{0.739656in}}%
\pgfpathlineto{\pgfqpoint{1.563919in}{0.739656in}}%
\pgfpathlineto{\pgfqpoint{1.563622in}{0.739656in}}%
\pgfpathlineto{\pgfqpoint{1.563324in}{0.739656in}}%
\pgfpathlineto{\pgfqpoint{1.563027in}{0.739656in}}%
\pgfpathlineto{\pgfqpoint{1.562729in}{0.739656in}}%
\pgfpathlineto{\pgfqpoint{1.562432in}{0.739656in}}%
\pgfpathlineto{\pgfqpoint{1.562134in}{0.739656in}}%
\pgfpathlineto{\pgfqpoint{1.561837in}{0.739656in}}%
\pgfpathlineto{\pgfqpoint{1.561540in}{0.739656in}}%
\pgfpathlineto{\pgfqpoint{1.561242in}{0.739656in}}%
\pgfpathlineto{\pgfqpoint{1.560945in}{0.739656in}}%
\pgfpathlineto{\pgfqpoint{1.560647in}{0.739656in}}%
\pgfpathlineto{\pgfqpoint{1.560350in}{0.739656in}}%
\pgfpathlineto{\pgfqpoint{1.560052in}{0.739656in}}%
\pgfpathlineto{\pgfqpoint{1.559755in}{0.739656in}}%
\pgfpathlineto{\pgfqpoint{1.559457in}{0.739656in}}%
\pgfpathlineto{\pgfqpoint{1.559160in}{0.739656in}}%
\pgfpathlineto{\pgfqpoint{1.558862in}{0.739656in}}%
\pgfpathlineto{\pgfqpoint{1.558565in}{0.739656in}}%
\pgfpathlineto{\pgfqpoint{1.558267in}{0.739656in}}%
\pgfpathlineto{\pgfqpoint{1.557970in}{0.739656in}}%
\pgfpathlineto{\pgfqpoint{1.557672in}{0.739656in}}%
\pgfpathlineto{\pgfqpoint{1.557375in}{0.739656in}}%
\pgfpathlineto{\pgfqpoint{1.557077in}{0.739656in}}%
\pgfpathlineto{\pgfqpoint{1.556780in}{0.739656in}}%
\pgfpathlineto{\pgfqpoint{1.556482in}{0.739656in}}%
\pgfpathlineto{\pgfqpoint{1.556185in}{0.739656in}}%
\pgfpathlineto{\pgfqpoint{1.555887in}{0.739656in}}%
\pgfpathlineto{\pgfqpoint{1.555590in}{0.739656in}}%
\pgfpathlineto{\pgfqpoint{1.555292in}{0.739656in}}%
\pgfpathlineto{\pgfqpoint{1.554995in}{0.739656in}}%
\pgfpathlineto{\pgfqpoint{1.554698in}{0.739656in}}%
\pgfpathlineto{\pgfqpoint{1.554400in}{0.739656in}}%
\pgfpathlineto{\pgfqpoint{1.554103in}{0.739656in}}%
\pgfpathlineto{\pgfqpoint{1.553805in}{0.739656in}}%
\pgfpathlineto{\pgfqpoint{1.553508in}{0.739656in}}%
\pgfpathlineto{\pgfqpoint{1.553210in}{0.739656in}}%
\pgfpathlineto{\pgfqpoint{1.552913in}{0.739656in}}%
\pgfpathlineto{\pgfqpoint{1.552615in}{0.739656in}}%
\pgfpathlineto{\pgfqpoint{1.552318in}{0.739656in}}%
\pgfpathlineto{\pgfqpoint{1.552020in}{0.739656in}}%
\pgfpathlineto{\pgfqpoint{1.551723in}{0.739656in}}%
\pgfpathlineto{\pgfqpoint{1.551425in}{0.739656in}}%
\pgfpathlineto{\pgfqpoint{1.551128in}{0.739656in}}%
\pgfpathlineto{\pgfqpoint{1.550830in}{0.739656in}}%
\pgfpathlineto{\pgfqpoint{1.550533in}{0.739656in}}%
\pgfpathlineto{\pgfqpoint{1.550235in}{0.739656in}}%
\pgfpathlineto{\pgfqpoint{1.549938in}{0.739656in}}%
\pgfpathlineto{\pgfqpoint{1.549640in}{0.739656in}}%
\pgfpathlineto{\pgfqpoint{1.549343in}{0.739656in}}%
\pgfpathlineto{\pgfqpoint{1.549045in}{0.739656in}}%
\pgfpathlineto{\pgfqpoint{1.548748in}{0.739656in}}%
\pgfpathlineto{\pgfqpoint{1.548450in}{0.739656in}}%
\pgfpathlineto{\pgfqpoint{1.548153in}{0.739656in}}%
\pgfpathlineto{\pgfqpoint{1.547856in}{0.739656in}}%
\pgfpathlineto{\pgfqpoint{1.547558in}{0.739656in}}%
\pgfpathlineto{\pgfqpoint{1.547261in}{0.739656in}}%
\pgfpathlineto{\pgfqpoint{1.546963in}{0.739656in}}%
\pgfpathlineto{\pgfqpoint{1.546666in}{0.739656in}}%
\pgfpathlineto{\pgfqpoint{1.546368in}{0.739656in}}%
\pgfpathlineto{\pgfqpoint{1.546071in}{0.739656in}}%
\pgfpathlineto{\pgfqpoint{1.545773in}{0.739656in}}%
\pgfpathlineto{\pgfqpoint{1.545476in}{0.739656in}}%
\pgfpathlineto{\pgfqpoint{1.545178in}{0.739656in}}%
\pgfpathlineto{\pgfqpoint{1.544881in}{0.739656in}}%
\pgfpathlineto{\pgfqpoint{1.544583in}{0.739656in}}%
\pgfpathlineto{\pgfqpoint{1.544286in}{0.739656in}}%
\pgfpathlineto{\pgfqpoint{1.543988in}{0.739656in}}%
\pgfpathlineto{\pgfqpoint{1.543691in}{0.739656in}}%
\pgfpathlineto{\pgfqpoint{1.543393in}{0.739656in}}%
\pgfpathlineto{\pgfqpoint{1.543096in}{0.739656in}}%
\pgfpathlineto{\pgfqpoint{1.542798in}{0.739656in}}%
\pgfpathlineto{\pgfqpoint{1.542501in}{0.739656in}}%
\pgfpathlineto{\pgfqpoint{1.542203in}{0.739656in}}%
\pgfpathlineto{\pgfqpoint{1.541906in}{0.739656in}}%
\pgfpathlineto{\pgfqpoint{1.541609in}{0.739656in}}%
\pgfpathlineto{\pgfqpoint{1.541311in}{0.739656in}}%
\pgfpathlineto{\pgfqpoint{1.541014in}{0.739656in}}%
\pgfpathlineto{\pgfqpoint{1.540716in}{0.739656in}}%
\pgfpathlineto{\pgfqpoint{1.540419in}{0.739656in}}%
\pgfpathlineto{\pgfqpoint{1.540121in}{0.739656in}}%
\pgfpathlineto{\pgfqpoint{1.539824in}{0.739656in}}%
\pgfpathlineto{\pgfqpoint{1.539526in}{0.739656in}}%
\pgfpathlineto{\pgfqpoint{1.539229in}{0.739656in}}%
\pgfpathlineto{\pgfqpoint{1.538931in}{0.739656in}}%
\pgfpathlineto{\pgfqpoint{1.538634in}{0.739656in}}%
\pgfpathlineto{\pgfqpoint{1.538336in}{0.739656in}}%
\pgfpathlineto{\pgfqpoint{1.538039in}{0.739656in}}%
\pgfpathlineto{\pgfqpoint{1.537741in}{0.739656in}}%
\pgfpathlineto{\pgfqpoint{1.537444in}{0.739656in}}%
\pgfpathlineto{\pgfqpoint{1.537146in}{0.739656in}}%
\pgfpathlineto{\pgfqpoint{1.536849in}{0.739656in}}%
\pgfpathlineto{\pgfqpoint{1.536551in}{0.739656in}}%
\pgfpathlineto{\pgfqpoint{1.536254in}{0.739656in}}%
\pgfpathlineto{\pgfqpoint{1.535956in}{0.739656in}}%
\pgfpathlineto{\pgfqpoint{1.535659in}{0.739656in}}%
\pgfpathlineto{\pgfqpoint{1.535361in}{0.739656in}}%
\pgfpathlineto{\pgfqpoint{1.535064in}{0.739656in}}%
\pgfpathlineto{\pgfqpoint{1.534767in}{0.739656in}}%
\pgfpathlineto{\pgfqpoint{1.534469in}{0.739656in}}%
\pgfpathlineto{\pgfqpoint{1.534172in}{0.739656in}}%
\pgfpathlineto{\pgfqpoint{1.533874in}{0.739656in}}%
\pgfpathlineto{\pgfqpoint{1.533577in}{0.739656in}}%
\pgfpathlineto{\pgfqpoint{1.533279in}{0.739656in}}%
\pgfpathlineto{\pgfqpoint{1.532982in}{0.739656in}}%
\pgfpathlineto{\pgfqpoint{1.532684in}{0.739656in}}%
\pgfpathlineto{\pgfqpoint{1.532387in}{0.739656in}}%
\pgfpathlineto{\pgfqpoint{1.532089in}{0.739656in}}%
\pgfpathlineto{\pgfqpoint{1.531792in}{0.739656in}}%
\pgfpathlineto{\pgfqpoint{1.531494in}{0.739656in}}%
\pgfpathlineto{\pgfqpoint{1.531197in}{0.739656in}}%
\pgfpathlineto{\pgfqpoint{1.530899in}{0.739656in}}%
\pgfpathlineto{\pgfqpoint{1.530602in}{0.739656in}}%
\pgfpathlineto{\pgfqpoint{1.530304in}{0.739656in}}%
\pgfpathlineto{\pgfqpoint{1.530007in}{0.739656in}}%
\pgfpathlineto{\pgfqpoint{1.529709in}{0.739656in}}%
\pgfpathlineto{\pgfqpoint{1.529412in}{0.739656in}}%
\pgfpathlineto{\pgfqpoint{1.529114in}{0.739656in}}%
\pgfpathlineto{\pgfqpoint{1.528817in}{0.739656in}}%
\pgfpathlineto{\pgfqpoint{1.528519in}{0.739656in}}%
\pgfpathlineto{\pgfqpoint{1.528222in}{0.739656in}}%
\pgfpathlineto{\pgfqpoint{1.527925in}{0.739656in}}%
\pgfpathlineto{\pgfqpoint{1.527627in}{0.739656in}}%
\pgfpathlineto{\pgfqpoint{1.527330in}{0.739656in}}%
\pgfpathlineto{\pgfqpoint{1.527032in}{0.739656in}}%
\pgfpathlineto{\pgfqpoint{1.526735in}{0.739656in}}%
\pgfpathlineto{\pgfqpoint{1.526437in}{0.739656in}}%
\pgfpathlineto{\pgfqpoint{1.526140in}{0.739656in}}%
\pgfpathlineto{\pgfqpoint{1.525842in}{0.739656in}}%
\pgfpathlineto{\pgfqpoint{1.525545in}{0.739656in}}%
\pgfpathlineto{\pgfqpoint{1.525247in}{0.739656in}}%
\pgfpathlineto{\pgfqpoint{1.524950in}{0.739656in}}%
\pgfpathlineto{\pgfqpoint{1.524652in}{0.739656in}}%
\pgfpathlineto{\pgfqpoint{1.524355in}{0.739656in}}%
\pgfpathlineto{\pgfqpoint{1.524057in}{0.739656in}}%
\pgfpathlineto{\pgfqpoint{1.523760in}{0.739656in}}%
\pgfpathlineto{\pgfqpoint{1.523462in}{0.739656in}}%
\pgfpathlineto{\pgfqpoint{1.523165in}{0.739656in}}%
\pgfpathlineto{\pgfqpoint{1.522867in}{0.739656in}}%
\pgfpathlineto{\pgfqpoint{1.522570in}{0.739656in}}%
\pgfpathlineto{\pgfqpoint{1.522272in}{0.739656in}}%
\pgfpathlineto{\pgfqpoint{1.521975in}{0.739656in}}%
\pgfpathlineto{\pgfqpoint{1.521678in}{0.739656in}}%
\pgfpathlineto{\pgfqpoint{1.521380in}{0.739656in}}%
\pgfpathlineto{\pgfqpoint{1.521083in}{0.739656in}}%
\pgfpathlineto{\pgfqpoint{1.520785in}{0.739656in}}%
\pgfpathlineto{\pgfqpoint{1.520488in}{0.739656in}}%
\pgfpathlineto{\pgfqpoint{1.520190in}{0.739656in}}%
\pgfpathlineto{\pgfqpoint{1.519893in}{0.739656in}}%
\pgfpathlineto{\pgfqpoint{1.519595in}{0.739656in}}%
\pgfpathlineto{\pgfqpoint{1.519298in}{0.739656in}}%
\pgfpathlineto{\pgfqpoint{1.519000in}{0.739656in}}%
\pgfpathlineto{\pgfqpoint{1.518703in}{0.739656in}}%
\pgfpathlineto{\pgfqpoint{1.518405in}{0.739656in}}%
\pgfpathlineto{\pgfqpoint{1.518108in}{0.739656in}}%
\pgfpathlineto{\pgfqpoint{1.517810in}{0.739656in}}%
\pgfpathlineto{\pgfqpoint{1.517513in}{0.739656in}}%
\pgfpathlineto{\pgfqpoint{1.517215in}{0.739656in}}%
\pgfpathlineto{\pgfqpoint{1.516918in}{0.739656in}}%
\pgfpathlineto{\pgfqpoint{1.516620in}{0.739656in}}%
\pgfpathlineto{\pgfqpoint{1.516323in}{0.739656in}}%
\pgfpathlineto{\pgfqpoint{1.516025in}{0.739656in}}%
\pgfpathlineto{\pgfqpoint{1.515728in}{0.739656in}}%
\pgfpathlineto{\pgfqpoint{1.515430in}{0.739656in}}%
\pgfpathlineto{\pgfqpoint{1.515133in}{0.739656in}}%
\pgfpathlineto{\pgfqpoint{1.514836in}{0.739656in}}%
\pgfpathlineto{\pgfqpoint{1.514538in}{0.739656in}}%
\pgfpathlineto{\pgfqpoint{1.514241in}{0.739656in}}%
\pgfpathlineto{\pgfqpoint{1.513943in}{0.739656in}}%
\pgfpathlineto{\pgfqpoint{1.513646in}{0.739656in}}%
\pgfpathlineto{\pgfqpoint{1.513348in}{0.739656in}}%
\pgfpathlineto{\pgfqpoint{1.513051in}{0.739656in}}%
\pgfpathlineto{\pgfqpoint{1.512753in}{0.739656in}}%
\pgfpathlineto{\pgfqpoint{1.512456in}{0.739656in}}%
\pgfpathlineto{\pgfqpoint{1.512158in}{0.739656in}}%
\pgfpathlineto{\pgfqpoint{1.511861in}{0.739656in}}%
\pgfpathlineto{\pgfqpoint{1.511563in}{0.739656in}}%
\pgfpathlineto{\pgfqpoint{1.511266in}{0.739656in}}%
\pgfpathlineto{\pgfqpoint{1.510968in}{0.739656in}}%
\pgfpathlineto{\pgfqpoint{1.510671in}{0.739656in}}%
\pgfpathlineto{\pgfqpoint{1.510373in}{0.739656in}}%
\pgfpathlineto{\pgfqpoint{1.510076in}{0.739656in}}%
\pgfpathlineto{\pgfqpoint{1.509778in}{0.739656in}}%
\pgfpathlineto{\pgfqpoint{1.509481in}{0.739656in}}%
\pgfpathlineto{\pgfqpoint{1.509183in}{0.739656in}}%
\pgfpathlineto{\pgfqpoint{1.508886in}{0.739656in}}%
\pgfpathlineto{\pgfqpoint{1.508588in}{0.739656in}}%
\pgfpathlineto{\pgfqpoint{1.508291in}{0.739656in}}%
\pgfpathlineto{\pgfqpoint{1.507994in}{0.739656in}}%
\pgfpathlineto{\pgfqpoint{1.507696in}{0.739656in}}%
\pgfpathlineto{\pgfqpoint{1.507399in}{0.739656in}}%
\pgfpathlineto{\pgfqpoint{1.507101in}{0.739656in}}%
\pgfpathlineto{\pgfqpoint{1.506804in}{0.739656in}}%
\pgfpathlineto{\pgfqpoint{1.506506in}{0.739656in}}%
\pgfpathlineto{\pgfqpoint{1.506209in}{0.739656in}}%
\pgfpathlineto{\pgfqpoint{1.505911in}{0.739656in}}%
\pgfpathlineto{\pgfqpoint{1.505614in}{0.739656in}}%
\pgfpathlineto{\pgfqpoint{1.505316in}{0.739656in}}%
\pgfpathlineto{\pgfqpoint{1.505019in}{0.739656in}}%
\pgfpathlineto{\pgfqpoint{1.504721in}{0.739656in}}%
\pgfpathlineto{\pgfqpoint{1.504424in}{0.739656in}}%
\pgfpathlineto{\pgfqpoint{1.504126in}{0.739656in}}%
\pgfpathlineto{\pgfqpoint{1.503829in}{0.739656in}}%
\pgfpathlineto{\pgfqpoint{1.503531in}{0.739656in}}%
\pgfpathlineto{\pgfqpoint{1.503234in}{0.739656in}}%
\pgfpathlineto{\pgfqpoint{1.502936in}{0.739656in}}%
\pgfpathlineto{\pgfqpoint{1.502639in}{0.739656in}}%
\pgfpathlineto{\pgfqpoint{1.502341in}{0.739656in}}%
\pgfpathlineto{\pgfqpoint{1.502044in}{0.739656in}}%
\pgfpathlineto{\pgfqpoint{1.501747in}{0.739656in}}%
\pgfpathlineto{\pgfqpoint{1.501449in}{0.739656in}}%
\pgfpathlineto{\pgfqpoint{1.501152in}{0.739656in}}%
\pgfpathlineto{\pgfqpoint{1.500854in}{0.739656in}}%
\pgfpathlineto{\pgfqpoint{1.500557in}{0.739656in}}%
\pgfpathlineto{\pgfqpoint{1.500259in}{0.739656in}}%
\pgfpathlineto{\pgfqpoint{1.499962in}{0.739656in}}%
\pgfpathlineto{\pgfqpoint{1.499664in}{0.739656in}}%
\pgfpathlineto{\pgfqpoint{1.499367in}{0.739656in}}%
\pgfpathlineto{\pgfqpoint{1.499069in}{0.739656in}}%
\pgfpathlineto{\pgfqpoint{1.498772in}{0.739656in}}%
\pgfpathlineto{\pgfqpoint{1.498474in}{0.739656in}}%
\pgfpathlineto{\pgfqpoint{1.498177in}{0.739656in}}%
\pgfpathlineto{\pgfqpoint{1.497879in}{0.739656in}}%
\pgfpathlineto{\pgfqpoint{1.497582in}{0.739656in}}%
\pgfpathlineto{\pgfqpoint{1.497284in}{0.739656in}}%
\pgfpathlineto{\pgfqpoint{1.496987in}{0.739656in}}%
\pgfpathlineto{\pgfqpoint{1.496689in}{0.739656in}}%
\pgfpathlineto{\pgfqpoint{1.496392in}{0.739656in}}%
\pgfpathlineto{\pgfqpoint{1.496094in}{0.739656in}}%
\pgfpathlineto{\pgfqpoint{1.495797in}{0.739656in}}%
\pgfpathlineto{\pgfqpoint{1.495499in}{0.739656in}}%
\pgfpathlineto{\pgfqpoint{1.495202in}{0.739656in}}%
\pgfpathlineto{\pgfqpoint{1.494905in}{0.739656in}}%
\pgfpathlineto{\pgfqpoint{1.494607in}{0.739656in}}%
\pgfpathlineto{\pgfqpoint{1.494310in}{0.739656in}}%
\pgfpathlineto{\pgfqpoint{1.494012in}{0.739656in}}%
\pgfpathlineto{\pgfqpoint{1.493715in}{0.739656in}}%
\pgfpathlineto{\pgfqpoint{1.493417in}{0.739656in}}%
\pgfpathlineto{\pgfqpoint{1.493120in}{0.739656in}}%
\pgfpathlineto{\pgfqpoint{1.492822in}{0.739656in}}%
\pgfpathlineto{\pgfqpoint{1.492525in}{0.739656in}}%
\pgfpathlineto{\pgfqpoint{1.492227in}{0.739656in}}%
\pgfpathlineto{\pgfqpoint{1.491930in}{0.739656in}}%
\pgfpathlineto{\pgfqpoint{1.491632in}{0.739656in}}%
\pgfpathlineto{\pgfqpoint{1.491335in}{0.739656in}}%
\pgfpathlineto{\pgfqpoint{1.491037in}{0.739656in}}%
\pgfpathlineto{\pgfqpoint{1.490740in}{0.739656in}}%
\pgfpathlineto{\pgfqpoint{1.490442in}{0.739656in}}%
\pgfpathlineto{\pgfqpoint{1.490145in}{0.739656in}}%
\pgfpathlineto{\pgfqpoint{1.489847in}{0.739656in}}%
\pgfpathlineto{\pgfqpoint{1.489550in}{0.739656in}}%
\pgfpathlineto{\pgfqpoint{1.489252in}{0.739656in}}%
\pgfpathlineto{\pgfqpoint{1.488955in}{0.739656in}}%
\pgfpathlineto{\pgfqpoint{1.488657in}{0.739656in}}%
\pgfpathlineto{\pgfqpoint{1.488360in}{0.739656in}}%
\pgfpathlineto{\pgfqpoint{1.488063in}{0.739656in}}%
\pgfpathlineto{\pgfqpoint{1.487765in}{0.739656in}}%
\pgfpathlineto{\pgfqpoint{1.487468in}{0.739656in}}%
\pgfpathlineto{\pgfqpoint{1.487170in}{0.739656in}}%
\pgfpathlineto{\pgfqpoint{1.486873in}{0.739656in}}%
\pgfpathlineto{\pgfqpoint{1.486575in}{0.739656in}}%
\pgfpathlineto{\pgfqpoint{1.486278in}{0.739656in}}%
\pgfpathlineto{\pgfqpoint{1.485980in}{0.739656in}}%
\pgfpathlineto{\pgfqpoint{1.485683in}{0.739656in}}%
\pgfpathlineto{\pgfqpoint{1.485385in}{0.739656in}}%
\pgfpathlineto{\pgfqpoint{1.485088in}{0.739656in}}%
\pgfpathlineto{\pgfqpoint{1.484790in}{0.739656in}}%
\pgfpathlineto{\pgfqpoint{1.484493in}{0.739656in}}%
\pgfpathlineto{\pgfqpoint{1.484195in}{0.739656in}}%
\pgfpathlineto{\pgfqpoint{1.483898in}{0.739656in}}%
\pgfpathlineto{\pgfqpoint{1.483600in}{0.739656in}}%
\pgfpathlineto{\pgfqpoint{1.483303in}{0.739656in}}%
\pgfpathlineto{\pgfqpoint{1.483005in}{0.739656in}}%
\pgfpathlineto{\pgfqpoint{1.482708in}{0.739656in}}%
\pgfpathlineto{\pgfqpoint{1.482410in}{0.739656in}}%
\pgfpathlineto{\pgfqpoint{1.482113in}{0.739656in}}%
\pgfpathlineto{\pgfqpoint{1.481816in}{0.739656in}}%
\pgfpathlineto{\pgfqpoint{1.481518in}{0.739656in}}%
\pgfpathlineto{\pgfqpoint{1.481221in}{0.739656in}}%
\pgfpathlineto{\pgfqpoint{1.480923in}{0.739656in}}%
\pgfpathlineto{\pgfqpoint{1.480626in}{0.739656in}}%
\pgfpathlineto{\pgfqpoint{1.480328in}{0.739656in}}%
\pgfpathlineto{\pgfqpoint{1.480031in}{0.739656in}}%
\pgfpathlineto{\pgfqpoint{1.479733in}{0.739656in}}%
\pgfpathlineto{\pgfqpoint{1.479436in}{0.739656in}}%
\pgfpathlineto{\pgfqpoint{1.479138in}{0.739656in}}%
\pgfpathlineto{\pgfqpoint{1.478841in}{0.739656in}}%
\pgfpathlineto{\pgfqpoint{1.478543in}{0.739656in}}%
\pgfpathlineto{\pgfqpoint{1.478246in}{0.739656in}}%
\pgfpathlineto{\pgfqpoint{1.477948in}{0.739656in}}%
\pgfpathlineto{\pgfqpoint{1.477651in}{0.739656in}}%
\pgfpathlineto{\pgfqpoint{1.477353in}{0.739656in}}%
\pgfpathlineto{\pgfqpoint{1.477056in}{0.739656in}}%
\pgfpathlineto{\pgfqpoint{1.476758in}{0.739656in}}%
\pgfpathlineto{\pgfqpoint{1.476461in}{0.739656in}}%
\pgfpathlineto{\pgfqpoint{1.476163in}{0.739656in}}%
\pgfpathlineto{\pgfqpoint{1.475866in}{0.739656in}}%
\pgfpathlineto{\pgfqpoint{1.475568in}{0.739656in}}%
\pgfpathlineto{\pgfqpoint{1.475271in}{0.739656in}}%
\pgfpathlineto{\pgfqpoint{1.474974in}{0.739656in}}%
\pgfpathlineto{\pgfqpoint{1.474676in}{0.739656in}}%
\pgfpathlineto{\pgfqpoint{1.474379in}{0.739656in}}%
\pgfpathlineto{\pgfqpoint{1.474081in}{0.739656in}}%
\pgfpathlineto{\pgfqpoint{1.473784in}{0.739656in}}%
\pgfpathlineto{\pgfqpoint{1.473486in}{0.739656in}}%
\pgfpathlineto{\pgfqpoint{1.473189in}{0.739656in}}%
\pgfpathlineto{\pgfqpoint{1.472891in}{0.739656in}}%
\pgfpathlineto{\pgfqpoint{1.472594in}{0.739656in}}%
\pgfpathlineto{\pgfqpoint{1.472296in}{0.739656in}}%
\pgfpathlineto{\pgfqpoint{1.471999in}{0.739656in}}%
\pgfpathlineto{\pgfqpoint{1.471701in}{0.739656in}}%
\pgfpathlineto{\pgfqpoint{1.471404in}{0.739656in}}%
\pgfpathlineto{\pgfqpoint{1.471106in}{0.739656in}}%
\pgfpathlineto{\pgfqpoint{1.470809in}{0.739656in}}%
\pgfpathlineto{\pgfqpoint{1.470511in}{0.739656in}}%
\pgfpathlineto{\pgfqpoint{1.470214in}{0.739656in}}%
\pgfpathlineto{\pgfqpoint{1.469916in}{0.739656in}}%
\pgfpathlineto{\pgfqpoint{1.469619in}{0.739656in}}%
\pgfpathlineto{\pgfqpoint{1.469321in}{0.739656in}}%
\pgfpathlineto{\pgfqpoint{1.469024in}{0.739656in}}%
\pgfpathlineto{\pgfqpoint{1.468726in}{0.739656in}}%
\pgfpathlineto{\pgfqpoint{1.468429in}{0.739656in}}%
\pgfpathlineto{\pgfqpoint{1.468132in}{0.739656in}}%
\pgfpathlineto{\pgfqpoint{1.467834in}{0.739656in}}%
\pgfpathlineto{\pgfqpoint{1.467537in}{0.739656in}}%
\pgfpathlineto{\pgfqpoint{1.467239in}{0.739656in}}%
\pgfpathlineto{\pgfqpoint{1.466942in}{0.739656in}}%
\pgfpathlineto{\pgfqpoint{1.466644in}{0.739656in}}%
\pgfpathlineto{\pgfqpoint{1.466347in}{0.739656in}}%
\pgfpathlineto{\pgfqpoint{1.466049in}{0.739656in}}%
\pgfpathlineto{\pgfqpoint{1.465752in}{0.739656in}}%
\pgfpathlineto{\pgfqpoint{1.465454in}{0.739656in}}%
\pgfpathlineto{\pgfqpoint{1.465157in}{0.739656in}}%
\pgfpathlineto{\pgfqpoint{1.464859in}{0.739656in}}%
\pgfpathlineto{\pgfqpoint{1.464562in}{0.739656in}}%
\pgfpathlineto{\pgfqpoint{1.464264in}{0.739656in}}%
\pgfpathlineto{\pgfqpoint{1.463967in}{0.739656in}}%
\pgfpathlineto{\pgfqpoint{1.463669in}{0.739656in}}%
\pgfpathlineto{\pgfqpoint{1.463372in}{0.739656in}}%
\pgfpathlineto{\pgfqpoint{1.463074in}{0.739656in}}%
\pgfpathlineto{\pgfqpoint{1.462777in}{0.739656in}}%
\pgfpathlineto{\pgfqpoint{1.462479in}{0.739656in}}%
\pgfpathlineto{\pgfqpoint{1.462182in}{0.739656in}}%
\pgfpathlineto{\pgfqpoint{1.461885in}{0.739656in}}%
\pgfpathlineto{\pgfqpoint{1.461587in}{0.739656in}}%
\pgfpathlineto{\pgfqpoint{1.461290in}{0.739656in}}%
\pgfpathlineto{\pgfqpoint{1.460992in}{0.739656in}}%
\pgfpathlineto{\pgfqpoint{1.460695in}{0.739656in}}%
\pgfpathlineto{\pgfqpoint{1.460397in}{0.739656in}}%
\pgfpathlineto{\pgfqpoint{1.460100in}{0.739656in}}%
\pgfpathlineto{\pgfqpoint{1.459802in}{0.739656in}}%
\pgfpathlineto{\pgfqpoint{1.459505in}{0.739656in}}%
\pgfpathlineto{\pgfqpoint{1.459207in}{0.739656in}}%
\pgfpathlineto{\pgfqpoint{1.458910in}{0.739656in}}%
\pgfpathlineto{\pgfqpoint{1.458612in}{0.739656in}}%
\pgfpathlineto{\pgfqpoint{1.458315in}{0.739656in}}%
\pgfpathlineto{\pgfqpoint{1.458017in}{0.739656in}}%
\pgfpathlineto{\pgfqpoint{1.457720in}{0.739656in}}%
\pgfpathlineto{\pgfqpoint{1.457422in}{0.739656in}}%
\pgfpathlineto{\pgfqpoint{1.457125in}{0.739656in}}%
\pgfpathlineto{\pgfqpoint{1.456827in}{0.739656in}}%
\pgfpathlineto{\pgfqpoint{1.456530in}{0.739656in}}%
\pgfpathlineto{\pgfqpoint{1.456232in}{0.739656in}}%
\pgfpathlineto{\pgfqpoint{1.455935in}{0.739656in}}%
\pgfpathlineto{\pgfqpoint{1.455637in}{0.739656in}}%
\pgfpathlineto{\pgfqpoint{1.455340in}{0.739656in}}%
\pgfpathlineto{\pgfqpoint{1.455043in}{0.739656in}}%
\pgfpathlineto{\pgfqpoint{1.454745in}{0.739656in}}%
\pgfpathlineto{\pgfqpoint{1.454448in}{0.739656in}}%
\pgfpathlineto{\pgfqpoint{1.454150in}{0.739656in}}%
\pgfpathlineto{\pgfqpoint{1.453853in}{0.739656in}}%
\pgfpathlineto{\pgfqpoint{1.453555in}{0.739656in}}%
\pgfpathlineto{\pgfqpoint{1.453258in}{0.739656in}}%
\pgfpathlineto{\pgfqpoint{1.452960in}{0.739656in}}%
\pgfpathlineto{\pgfqpoint{1.452663in}{0.739656in}}%
\pgfpathlineto{\pgfqpoint{1.452365in}{0.739656in}}%
\pgfpathlineto{\pgfqpoint{1.452068in}{0.739656in}}%
\pgfpathlineto{\pgfqpoint{1.451770in}{0.739656in}}%
\pgfpathlineto{\pgfqpoint{1.451473in}{0.739656in}}%
\pgfpathlineto{\pgfqpoint{1.451175in}{0.739656in}}%
\pgfpathlineto{\pgfqpoint{1.450878in}{0.739656in}}%
\pgfpathlineto{\pgfqpoint{1.450580in}{0.739656in}}%
\pgfpathlineto{\pgfqpoint{1.450283in}{0.739656in}}%
\pgfpathlineto{\pgfqpoint{1.449985in}{0.739656in}}%
\pgfpathlineto{\pgfqpoint{1.449688in}{0.739656in}}%
\pgfpathlineto{\pgfqpoint{1.449390in}{0.739656in}}%
\pgfpathlineto{\pgfqpoint{1.449093in}{0.739656in}}%
\pgfpathlineto{\pgfqpoint{1.448795in}{0.739656in}}%
\pgfpathlineto{\pgfqpoint{1.448498in}{0.739656in}}%
\pgfpathlineto{\pgfqpoint{1.448201in}{0.739656in}}%
\pgfpathlineto{\pgfqpoint{1.447903in}{0.739656in}}%
\pgfpathlineto{\pgfqpoint{1.447606in}{0.739656in}}%
\pgfpathlineto{\pgfqpoint{1.447308in}{0.739656in}}%
\pgfpathlineto{\pgfqpoint{1.447011in}{0.739656in}}%
\pgfpathlineto{\pgfqpoint{1.446713in}{0.739656in}}%
\pgfpathlineto{\pgfqpoint{1.446416in}{0.739656in}}%
\pgfpathlineto{\pgfqpoint{1.446118in}{0.739656in}}%
\pgfpathlineto{\pgfqpoint{1.445821in}{0.739656in}}%
\pgfpathlineto{\pgfqpoint{1.445523in}{0.739656in}}%
\pgfpathlineto{\pgfqpoint{1.445226in}{0.739656in}}%
\pgfpathlineto{\pgfqpoint{1.444928in}{0.739656in}}%
\pgfpathlineto{\pgfqpoint{1.444631in}{0.739656in}}%
\pgfpathlineto{\pgfqpoint{1.444333in}{0.739656in}}%
\pgfpathlineto{\pgfqpoint{1.444036in}{0.739656in}}%
\pgfpathlineto{\pgfqpoint{1.443738in}{0.739656in}}%
\pgfpathlineto{\pgfqpoint{1.443441in}{0.739656in}}%
\pgfpathlineto{\pgfqpoint{1.443143in}{0.739656in}}%
\pgfpathlineto{\pgfqpoint{1.442846in}{0.739656in}}%
\pgfpathlineto{\pgfqpoint{1.442548in}{0.739656in}}%
\pgfpathlineto{\pgfqpoint{1.442251in}{0.739656in}}%
\pgfpathlineto{\pgfqpoint{1.441954in}{0.739656in}}%
\pgfpathlineto{\pgfqpoint{1.441656in}{0.739656in}}%
\pgfpathlineto{\pgfqpoint{1.441359in}{0.739656in}}%
\pgfpathlineto{\pgfqpoint{1.441061in}{0.739656in}}%
\pgfpathlineto{\pgfqpoint{1.440764in}{0.739656in}}%
\pgfpathlineto{\pgfqpoint{1.440466in}{0.739656in}}%
\pgfpathlineto{\pgfqpoint{1.440169in}{0.739656in}}%
\pgfpathlineto{\pgfqpoint{1.439871in}{0.739656in}}%
\pgfpathlineto{\pgfqpoint{1.439574in}{0.739656in}}%
\pgfpathlineto{\pgfqpoint{1.439276in}{0.739656in}}%
\pgfpathlineto{\pgfqpoint{1.438979in}{0.739656in}}%
\pgfpathlineto{\pgfqpoint{1.438681in}{0.739656in}}%
\pgfpathlineto{\pgfqpoint{1.438384in}{0.739656in}}%
\pgfpathlineto{\pgfqpoint{1.438086in}{0.739656in}}%
\pgfpathlineto{\pgfqpoint{1.437789in}{0.739656in}}%
\pgfpathlineto{\pgfqpoint{1.437491in}{0.739656in}}%
\pgfpathlineto{\pgfqpoint{1.437194in}{0.739656in}}%
\pgfpathlineto{\pgfqpoint{1.436896in}{0.739656in}}%
\pgfpathlineto{\pgfqpoint{1.436599in}{0.739656in}}%
\pgfpathlineto{\pgfqpoint{1.436301in}{0.739656in}}%
\pgfpathlineto{\pgfqpoint{1.436004in}{0.739656in}}%
\pgfpathlineto{\pgfqpoint{1.435706in}{0.739656in}}%
\pgfpathlineto{\pgfqpoint{1.435409in}{0.739656in}}%
\pgfpathlineto{\pgfqpoint{1.435112in}{0.739656in}}%
\pgfpathlineto{\pgfqpoint{1.434814in}{0.739656in}}%
\pgfpathlineto{\pgfqpoint{1.434517in}{0.739656in}}%
\pgfpathlineto{\pgfqpoint{1.434219in}{0.739656in}}%
\pgfpathlineto{\pgfqpoint{1.433922in}{0.739656in}}%
\pgfpathlineto{\pgfqpoint{1.433624in}{0.739656in}}%
\pgfpathlineto{\pgfqpoint{1.433327in}{0.739656in}}%
\pgfpathlineto{\pgfqpoint{1.433029in}{0.739656in}}%
\pgfpathlineto{\pgfqpoint{1.432732in}{0.739656in}}%
\pgfpathlineto{\pgfqpoint{1.432434in}{0.739656in}}%
\pgfpathlineto{\pgfqpoint{1.432137in}{0.739656in}}%
\pgfpathlineto{\pgfqpoint{1.431839in}{0.739656in}}%
\pgfpathlineto{\pgfqpoint{1.431542in}{0.739656in}}%
\pgfpathlineto{\pgfqpoint{1.431244in}{0.739656in}}%
\pgfpathlineto{\pgfqpoint{1.430947in}{0.739656in}}%
\pgfpathlineto{\pgfqpoint{1.430649in}{0.739656in}}%
\pgfpathlineto{\pgfqpoint{1.430352in}{0.739656in}}%
\pgfpathlineto{\pgfqpoint{1.430054in}{0.739656in}}%
\pgfpathlineto{\pgfqpoint{1.429757in}{0.739656in}}%
\pgfpathlineto{\pgfqpoint{1.429459in}{0.739656in}}%
\pgfpathlineto{\pgfqpoint{1.429162in}{0.739656in}}%
\pgfpathlineto{\pgfqpoint{1.428864in}{0.739656in}}%
\pgfpathlineto{\pgfqpoint{1.428567in}{0.739656in}}%
\pgfpathlineto{\pgfqpoint{1.428270in}{0.739656in}}%
\pgfpathlineto{\pgfqpoint{1.427972in}{0.739656in}}%
\pgfpathlineto{\pgfqpoint{1.427675in}{0.739656in}}%
\pgfpathlineto{\pgfqpoint{1.427377in}{0.739656in}}%
\pgfpathlineto{\pgfqpoint{1.427080in}{0.739656in}}%
\pgfpathlineto{\pgfqpoint{1.426782in}{0.739656in}}%
\pgfpathlineto{\pgfqpoint{1.426485in}{0.739656in}}%
\pgfpathlineto{\pgfqpoint{1.426187in}{0.739656in}}%
\pgfpathlineto{\pgfqpoint{1.425890in}{0.739656in}}%
\pgfpathlineto{\pgfqpoint{1.425592in}{0.739656in}}%
\pgfpathlineto{\pgfqpoint{1.425295in}{0.739656in}}%
\pgfpathlineto{\pgfqpoint{1.424997in}{0.739656in}}%
\pgfpathlineto{\pgfqpoint{1.424700in}{0.739656in}}%
\pgfpathlineto{\pgfqpoint{1.424402in}{0.739656in}}%
\pgfpathlineto{\pgfqpoint{1.424105in}{0.739656in}}%
\pgfpathlineto{\pgfqpoint{1.423807in}{0.739656in}}%
\pgfpathlineto{\pgfqpoint{1.423510in}{0.739656in}}%
\pgfpathlineto{\pgfqpoint{1.423212in}{0.739656in}}%
\pgfpathlineto{\pgfqpoint{1.422915in}{0.739656in}}%
\pgfpathlineto{\pgfqpoint{1.422617in}{0.739656in}}%
\pgfpathlineto{\pgfqpoint{1.422320in}{0.739656in}}%
\pgfpathlineto{\pgfqpoint{1.422023in}{0.739656in}}%
\pgfpathlineto{\pgfqpoint{1.421725in}{0.739656in}}%
\pgfpathlineto{\pgfqpoint{1.421428in}{0.739656in}}%
\pgfpathlineto{\pgfqpoint{1.421130in}{0.739656in}}%
\pgfpathlineto{\pgfqpoint{1.420833in}{0.739656in}}%
\pgfpathlineto{\pgfqpoint{1.420535in}{0.739656in}}%
\pgfpathlineto{\pgfqpoint{1.420238in}{0.739656in}}%
\pgfpathlineto{\pgfqpoint{1.419940in}{0.739656in}}%
\pgfpathlineto{\pgfqpoint{1.419643in}{0.739656in}}%
\pgfpathlineto{\pgfqpoint{1.419345in}{0.739656in}}%
\pgfpathlineto{\pgfqpoint{1.419048in}{0.739656in}}%
\pgfpathlineto{\pgfqpoint{1.418750in}{0.739656in}}%
\pgfpathlineto{\pgfqpoint{1.418453in}{0.739656in}}%
\pgfpathlineto{\pgfqpoint{1.418155in}{0.739656in}}%
\pgfpathlineto{\pgfqpoint{1.417858in}{0.739656in}}%
\pgfpathlineto{\pgfqpoint{1.417560in}{0.739656in}}%
\pgfpathlineto{\pgfqpoint{1.417263in}{0.739656in}}%
\pgfpathlineto{\pgfqpoint{1.416965in}{0.739656in}}%
\pgfpathlineto{\pgfqpoint{1.416668in}{0.739656in}}%
\pgfpathlineto{\pgfqpoint{1.416370in}{0.739656in}}%
\pgfpathlineto{\pgfqpoint{1.416073in}{0.739656in}}%
\pgfpathlineto{\pgfqpoint{1.415775in}{0.739656in}}%
\pgfpathlineto{\pgfqpoint{1.415478in}{0.739656in}}%
\pgfpathlineto{\pgfqpoint{1.415181in}{0.739656in}}%
\pgfpathlineto{\pgfqpoint{1.414883in}{0.739656in}}%
\pgfpathlineto{\pgfqpoint{1.414586in}{0.739656in}}%
\pgfpathlineto{\pgfqpoint{1.414288in}{0.739656in}}%
\pgfpathlineto{\pgfqpoint{1.413991in}{0.739656in}}%
\pgfpathlineto{\pgfqpoint{1.413693in}{0.739656in}}%
\pgfpathlineto{\pgfqpoint{1.413396in}{0.739656in}}%
\pgfpathlineto{\pgfqpoint{1.413098in}{0.739656in}}%
\pgfpathlineto{\pgfqpoint{1.412801in}{0.739656in}}%
\pgfpathlineto{\pgfqpoint{1.412503in}{0.739656in}}%
\pgfpathlineto{\pgfqpoint{1.412206in}{0.739656in}}%
\pgfpathlineto{\pgfqpoint{1.411908in}{0.739656in}}%
\pgfpathlineto{\pgfqpoint{1.411611in}{0.739656in}}%
\pgfpathlineto{\pgfqpoint{1.411313in}{0.739656in}}%
\pgfpathlineto{\pgfqpoint{1.411016in}{0.739656in}}%
\pgfpathlineto{\pgfqpoint{1.410718in}{0.739656in}}%
\pgfpathlineto{\pgfqpoint{1.410421in}{0.739656in}}%
\pgfpathlineto{\pgfqpoint{1.410123in}{0.739656in}}%
\pgfpathlineto{\pgfqpoint{1.409826in}{0.739656in}}%
\pgfpathlineto{\pgfqpoint{1.409528in}{0.739656in}}%
\pgfpathlineto{\pgfqpoint{1.409231in}{0.739656in}}%
\pgfpathlineto{\pgfqpoint{1.408933in}{0.739656in}}%
\pgfpathlineto{\pgfqpoint{1.408636in}{0.739656in}}%
\pgfpathlineto{\pgfqpoint{1.408339in}{0.739656in}}%
\pgfpathlineto{\pgfqpoint{1.408041in}{0.739656in}}%
\pgfpathlineto{\pgfqpoint{1.407744in}{0.739656in}}%
\pgfpathlineto{\pgfqpoint{1.407446in}{0.739656in}}%
\pgfpathlineto{\pgfqpoint{1.407149in}{0.739656in}}%
\pgfpathlineto{\pgfqpoint{1.406851in}{0.739656in}}%
\pgfpathlineto{\pgfqpoint{1.406554in}{0.739656in}}%
\pgfpathlineto{\pgfqpoint{1.406256in}{0.739656in}}%
\pgfpathlineto{\pgfqpoint{1.405959in}{0.739656in}}%
\pgfpathlineto{\pgfqpoint{1.405661in}{0.739656in}}%
\pgfpathlineto{\pgfqpoint{1.405364in}{0.739656in}}%
\pgfpathlineto{\pgfqpoint{1.405066in}{0.739656in}}%
\pgfpathlineto{\pgfqpoint{1.404769in}{0.739656in}}%
\pgfpathlineto{\pgfqpoint{1.404471in}{0.739656in}}%
\pgfpathlineto{\pgfqpoint{1.404174in}{0.739656in}}%
\pgfpathlineto{\pgfqpoint{1.403876in}{0.739656in}}%
\pgfpathlineto{\pgfqpoint{1.403579in}{0.739656in}}%
\pgfpathlineto{\pgfqpoint{1.403281in}{0.739656in}}%
\pgfpathlineto{\pgfqpoint{1.402984in}{0.739656in}}%
\pgfpathlineto{\pgfqpoint{1.402686in}{0.739656in}}%
\pgfpathlineto{\pgfqpoint{1.402389in}{0.739656in}}%
\pgfpathlineto{\pgfqpoint{1.402092in}{0.739656in}}%
\pgfpathlineto{\pgfqpoint{1.401794in}{0.739656in}}%
\pgfpathlineto{\pgfqpoint{1.401497in}{0.739656in}}%
\pgfpathlineto{\pgfqpoint{1.401199in}{0.739656in}}%
\pgfpathlineto{\pgfqpoint{1.400902in}{0.739656in}}%
\pgfpathlineto{\pgfqpoint{1.400604in}{0.739656in}}%
\pgfpathlineto{\pgfqpoint{1.400307in}{0.739656in}}%
\pgfpathlineto{\pgfqpoint{1.400009in}{0.739656in}}%
\pgfpathlineto{\pgfqpoint{1.399712in}{0.739656in}}%
\pgfpathlineto{\pgfqpoint{1.399414in}{0.739656in}}%
\pgfpathlineto{\pgfqpoint{1.399117in}{0.739656in}}%
\pgfpathlineto{\pgfqpoint{1.398819in}{0.739656in}}%
\pgfpathlineto{\pgfqpoint{1.398522in}{0.739656in}}%
\pgfpathlineto{\pgfqpoint{1.398224in}{0.739656in}}%
\pgfpathlineto{\pgfqpoint{1.397927in}{0.739656in}}%
\pgfpathlineto{\pgfqpoint{1.397629in}{0.739656in}}%
\pgfpathlineto{\pgfqpoint{1.397332in}{0.739656in}}%
\pgfpathlineto{\pgfqpoint{1.397034in}{0.739656in}}%
\pgfpathlineto{\pgfqpoint{1.396737in}{0.739656in}}%
\pgfpathlineto{\pgfqpoint{1.396439in}{0.739656in}}%
\pgfpathlineto{\pgfqpoint{1.396142in}{0.739656in}}%
\pgfpathlineto{\pgfqpoint{1.395844in}{0.739656in}}%
\pgfpathlineto{\pgfqpoint{1.395547in}{0.739656in}}%
\pgfpathlineto{\pgfqpoint{1.395250in}{0.739656in}}%
\pgfpathlineto{\pgfqpoint{1.394952in}{0.739656in}}%
\pgfpathlineto{\pgfqpoint{1.394655in}{0.739656in}}%
\pgfpathlineto{\pgfqpoint{1.394357in}{0.739656in}}%
\pgfpathlineto{\pgfqpoint{1.394060in}{0.739656in}}%
\pgfpathlineto{\pgfqpoint{1.393762in}{0.739656in}}%
\pgfpathlineto{\pgfqpoint{1.393465in}{0.739656in}}%
\pgfpathlineto{\pgfqpoint{1.393167in}{0.739656in}}%
\pgfpathlineto{\pgfqpoint{1.392870in}{0.739656in}}%
\pgfpathlineto{\pgfqpoint{1.392572in}{0.739656in}}%
\pgfpathlineto{\pgfqpoint{1.392275in}{0.739656in}}%
\pgfpathlineto{\pgfqpoint{1.391977in}{0.739656in}}%
\pgfpathlineto{\pgfqpoint{1.391680in}{0.739656in}}%
\pgfpathlineto{\pgfqpoint{1.391382in}{0.739656in}}%
\pgfpathlineto{\pgfqpoint{1.391085in}{0.739656in}}%
\pgfpathlineto{\pgfqpoint{1.390787in}{0.739656in}}%
\pgfpathlineto{\pgfqpoint{1.390490in}{0.739656in}}%
\pgfpathlineto{\pgfqpoint{1.390192in}{0.739656in}}%
\pgfpathlineto{\pgfqpoint{1.389895in}{0.739656in}}%
\pgfpathlineto{\pgfqpoint{1.389597in}{0.739656in}}%
\pgfpathlineto{\pgfqpoint{1.389300in}{0.739656in}}%
\pgfpathlineto{\pgfqpoint{1.389002in}{0.739656in}}%
\pgfpathlineto{\pgfqpoint{1.388705in}{0.739656in}}%
\pgfpathlineto{\pgfqpoint{1.388408in}{0.739656in}}%
\pgfpathlineto{\pgfqpoint{1.388110in}{0.739656in}}%
\pgfpathlineto{\pgfqpoint{1.387813in}{0.739656in}}%
\pgfpathlineto{\pgfqpoint{1.387515in}{0.739656in}}%
\pgfpathlineto{\pgfqpoint{1.387218in}{0.739656in}}%
\pgfpathlineto{\pgfqpoint{1.386920in}{0.739656in}}%
\pgfpathlineto{\pgfqpoint{1.386623in}{0.739656in}}%
\pgfpathlineto{\pgfqpoint{1.386325in}{0.739656in}}%
\pgfpathlineto{\pgfqpoint{1.386028in}{0.739656in}}%
\pgfpathlineto{\pgfqpoint{1.385730in}{0.739656in}}%
\pgfpathlineto{\pgfqpoint{1.385433in}{0.739656in}}%
\pgfpathlineto{\pgfqpoint{1.385135in}{0.739656in}}%
\pgfpathlineto{\pgfqpoint{1.384838in}{0.739656in}}%
\pgfpathlineto{\pgfqpoint{1.384540in}{0.739656in}}%
\pgfpathlineto{\pgfqpoint{1.384243in}{0.739656in}}%
\pgfpathlineto{\pgfqpoint{1.383945in}{0.739656in}}%
\pgfpathlineto{\pgfqpoint{1.383648in}{0.739656in}}%
\pgfpathlineto{\pgfqpoint{1.383350in}{0.739656in}}%
\pgfpathlineto{\pgfqpoint{1.383053in}{0.739656in}}%
\pgfpathlineto{\pgfqpoint{1.382755in}{0.739656in}}%
\pgfpathlineto{\pgfqpoint{1.382458in}{0.739656in}}%
\pgfpathlineto{\pgfqpoint{1.382161in}{0.739656in}}%
\pgfpathlineto{\pgfqpoint{1.381863in}{0.739656in}}%
\pgfpathlineto{\pgfqpoint{1.381566in}{0.739656in}}%
\pgfpathlineto{\pgfqpoint{1.381268in}{0.739656in}}%
\pgfpathlineto{\pgfqpoint{1.380971in}{0.739656in}}%
\pgfpathlineto{\pgfqpoint{1.380673in}{0.739656in}}%
\pgfpathlineto{\pgfqpoint{1.380376in}{0.739656in}}%
\pgfpathlineto{\pgfqpoint{1.380078in}{0.739656in}}%
\pgfpathlineto{\pgfqpoint{1.379781in}{0.739656in}}%
\pgfpathlineto{\pgfqpoint{1.379483in}{0.739656in}}%
\pgfpathlineto{\pgfqpoint{1.379186in}{0.739656in}}%
\pgfpathlineto{\pgfqpoint{1.378888in}{0.739656in}}%
\pgfpathlineto{\pgfqpoint{1.378591in}{0.739656in}}%
\pgfpathlineto{\pgfqpoint{1.378293in}{0.739656in}}%
\pgfpathlineto{\pgfqpoint{1.377996in}{0.739656in}}%
\pgfpathlineto{\pgfqpoint{1.377698in}{0.739656in}}%
\pgfpathlineto{\pgfqpoint{1.377401in}{0.739656in}}%
\pgfpathlineto{\pgfqpoint{1.377103in}{0.739656in}}%
\pgfpathlineto{\pgfqpoint{1.376806in}{0.739656in}}%
\pgfpathlineto{\pgfqpoint{1.376508in}{0.739656in}}%
\pgfpathlineto{\pgfqpoint{1.376211in}{0.739656in}}%
\pgfpathlineto{\pgfqpoint{1.375913in}{0.739656in}}%
\pgfpathlineto{\pgfqpoint{1.375616in}{0.739656in}}%
\pgfpathlineto{\pgfqpoint{1.375319in}{0.739656in}}%
\pgfpathlineto{\pgfqpoint{1.375021in}{0.739656in}}%
\pgfpathlineto{\pgfqpoint{1.374724in}{0.739656in}}%
\pgfpathlineto{\pgfqpoint{1.374426in}{0.739656in}}%
\pgfpathlineto{\pgfqpoint{1.374129in}{0.739656in}}%
\pgfpathlineto{\pgfqpoint{1.373831in}{0.739656in}}%
\pgfpathlineto{\pgfqpoint{1.373534in}{0.739656in}}%
\pgfpathlineto{\pgfqpoint{1.373236in}{0.739656in}}%
\pgfpathlineto{\pgfqpoint{1.372939in}{0.739656in}}%
\pgfpathlineto{\pgfqpoint{1.372641in}{0.739656in}}%
\pgfpathlineto{\pgfqpoint{1.372344in}{0.739656in}}%
\pgfpathlineto{\pgfqpoint{1.372046in}{0.739656in}}%
\pgfpathlineto{\pgfqpoint{1.371749in}{0.739656in}}%
\pgfpathlineto{\pgfqpoint{1.371451in}{0.739656in}}%
\pgfpathlineto{\pgfqpoint{1.371154in}{0.739656in}}%
\pgfpathlineto{\pgfqpoint{1.370856in}{0.739656in}}%
\pgfpathlineto{\pgfqpoint{1.370559in}{0.739656in}}%
\pgfpathlineto{\pgfqpoint{1.370261in}{0.739656in}}%
\pgfpathlineto{\pgfqpoint{1.369964in}{0.739656in}}%
\pgfpathlineto{\pgfqpoint{1.369666in}{0.739656in}}%
\pgfpathlineto{\pgfqpoint{1.369369in}{0.739656in}}%
\pgfpathlineto{\pgfqpoint{1.369071in}{0.739656in}}%
\pgfpathlineto{\pgfqpoint{1.368774in}{0.739656in}}%
\pgfpathlineto{\pgfqpoint{1.368477in}{0.739656in}}%
\pgfpathlineto{\pgfqpoint{1.368179in}{0.739656in}}%
\pgfpathlineto{\pgfqpoint{1.367882in}{0.739656in}}%
\pgfpathlineto{\pgfqpoint{1.367584in}{0.739656in}}%
\pgfpathlineto{\pgfqpoint{1.367287in}{0.739656in}}%
\pgfpathlineto{\pgfqpoint{1.366989in}{0.739656in}}%
\pgfpathlineto{\pgfqpoint{1.366692in}{0.739656in}}%
\pgfpathlineto{\pgfqpoint{1.366394in}{0.739656in}}%
\pgfpathlineto{\pgfqpoint{1.366097in}{0.739656in}}%
\pgfpathlineto{\pgfqpoint{1.365799in}{0.739656in}}%
\pgfpathlineto{\pgfqpoint{1.365502in}{0.739656in}}%
\pgfpathlineto{\pgfqpoint{1.365204in}{0.739656in}}%
\pgfpathlineto{\pgfqpoint{1.364907in}{0.739656in}}%
\pgfpathlineto{\pgfqpoint{1.364609in}{0.739656in}}%
\pgfpathlineto{\pgfqpoint{1.364312in}{0.739656in}}%
\pgfpathlineto{\pgfqpoint{1.364014in}{0.739656in}}%
\pgfpathlineto{\pgfqpoint{1.363717in}{0.739656in}}%
\pgfpathlineto{\pgfqpoint{1.363419in}{0.739656in}}%
\pgfpathlineto{\pgfqpoint{1.363122in}{0.739656in}}%
\pgfpathlineto{\pgfqpoint{1.362824in}{0.739656in}}%
\pgfpathlineto{\pgfqpoint{1.362527in}{0.739656in}}%
\pgfpathlineto{\pgfqpoint{1.362229in}{0.739656in}}%
\pgfpathlineto{\pgfqpoint{1.361932in}{0.739656in}}%
\pgfpathlineto{\pgfqpoint{1.361635in}{0.739656in}}%
\pgfpathlineto{\pgfqpoint{1.361337in}{0.739656in}}%
\pgfpathlineto{\pgfqpoint{1.361040in}{0.739656in}}%
\pgfpathlineto{\pgfqpoint{1.360742in}{0.739656in}}%
\pgfpathlineto{\pgfqpoint{1.360445in}{0.739656in}}%
\pgfpathlineto{\pgfqpoint{1.360147in}{0.739656in}}%
\pgfpathlineto{\pgfqpoint{1.359850in}{0.739656in}}%
\pgfpathlineto{\pgfqpoint{1.359552in}{0.739656in}}%
\pgfpathlineto{\pgfqpoint{1.359255in}{0.739656in}}%
\pgfpathlineto{\pgfqpoint{1.358957in}{0.739656in}}%
\pgfpathlineto{\pgfqpoint{1.358660in}{0.739656in}}%
\pgfpathlineto{\pgfqpoint{1.358362in}{0.739656in}}%
\pgfpathlineto{\pgfqpoint{1.358065in}{0.739656in}}%
\pgfpathlineto{\pgfqpoint{1.357767in}{0.739656in}}%
\pgfpathlineto{\pgfqpoint{1.357470in}{0.739656in}}%
\pgfpathlineto{\pgfqpoint{1.357172in}{0.739656in}}%
\pgfpathlineto{\pgfqpoint{1.356875in}{0.739656in}}%
\pgfpathlineto{\pgfqpoint{1.356577in}{0.739656in}}%
\pgfpathlineto{\pgfqpoint{1.356280in}{0.739656in}}%
\pgfpathlineto{\pgfqpoint{1.355982in}{0.739656in}}%
\pgfpathlineto{\pgfqpoint{1.355685in}{0.739656in}}%
\pgfpathlineto{\pgfqpoint{1.355388in}{0.739656in}}%
\pgfpathlineto{\pgfqpoint{1.355090in}{0.739656in}}%
\pgfpathlineto{\pgfqpoint{1.354793in}{0.739656in}}%
\pgfpathlineto{\pgfqpoint{1.354495in}{0.739656in}}%
\pgfpathlineto{\pgfqpoint{1.354198in}{0.739656in}}%
\pgfpathlineto{\pgfqpoint{1.353900in}{0.739656in}}%
\pgfpathlineto{\pgfqpoint{1.353603in}{0.739656in}}%
\pgfpathlineto{\pgfqpoint{1.353305in}{0.739656in}}%
\pgfpathlineto{\pgfqpoint{1.353008in}{0.739656in}}%
\pgfpathlineto{\pgfqpoint{1.352710in}{0.739656in}}%
\pgfpathlineto{\pgfqpoint{1.352413in}{0.739656in}}%
\pgfpathlineto{\pgfqpoint{1.352115in}{0.739656in}}%
\pgfpathlineto{\pgfqpoint{1.351818in}{0.739656in}}%
\pgfpathlineto{\pgfqpoint{1.351520in}{0.739656in}}%
\pgfpathlineto{\pgfqpoint{1.351223in}{0.739656in}}%
\pgfpathlineto{\pgfqpoint{1.350925in}{0.739656in}}%
\pgfpathlineto{\pgfqpoint{1.350628in}{0.739656in}}%
\pgfpathlineto{\pgfqpoint{1.350330in}{0.739656in}}%
\pgfpathlineto{\pgfqpoint{1.350033in}{0.739656in}}%
\pgfpathlineto{\pgfqpoint{1.349735in}{0.739656in}}%
\pgfpathlineto{\pgfqpoint{1.349438in}{0.739656in}}%
\pgfpathlineto{\pgfqpoint{1.349140in}{0.739656in}}%
\pgfpathlineto{\pgfqpoint{1.348843in}{0.739656in}}%
\pgfpathlineto{\pgfqpoint{1.348546in}{0.739656in}}%
\pgfpathlineto{\pgfqpoint{1.348248in}{0.739656in}}%
\pgfpathlineto{\pgfqpoint{1.347951in}{0.739656in}}%
\pgfpathlineto{\pgfqpoint{1.347653in}{0.739656in}}%
\pgfpathlineto{\pgfqpoint{1.347356in}{0.739656in}}%
\pgfpathlineto{\pgfqpoint{1.347058in}{0.739656in}}%
\pgfpathlineto{\pgfqpoint{1.346761in}{0.739656in}}%
\pgfpathlineto{\pgfqpoint{1.346463in}{0.739656in}}%
\pgfpathlineto{\pgfqpoint{1.346166in}{0.739656in}}%
\pgfpathlineto{\pgfqpoint{1.345868in}{0.739656in}}%
\pgfpathlineto{\pgfqpoint{1.345571in}{0.739656in}}%
\pgfpathlineto{\pgfqpoint{1.345273in}{0.739656in}}%
\pgfpathlineto{\pgfqpoint{1.344976in}{0.739656in}}%
\pgfpathlineto{\pgfqpoint{1.344678in}{0.739656in}}%
\pgfpathlineto{\pgfqpoint{1.344381in}{0.739656in}}%
\pgfpathlineto{\pgfqpoint{1.344083in}{0.739656in}}%
\pgfpathlineto{\pgfqpoint{1.343786in}{0.739656in}}%
\pgfpathlineto{\pgfqpoint{1.343488in}{0.739656in}}%
\pgfpathlineto{\pgfqpoint{1.343191in}{0.739656in}}%
\pgfpathlineto{\pgfqpoint{1.342893in}{0.739656in}}%
\pgfpathlineto{\pgfqpoint{1.342596in}{0.739656in}}%
\pgfpathlineto{\pgfqpoint{1.342298in}{0.739656in}}%
\pgfpathlineto{\pgfqpoint{1.342001in}{0.739656in}}%
\pgfpathlineto{\pgfqpoint{1.341704in}{0.739656in}}%
\pgfpathlineto{\pgfqpoint{1.341406in}{0.739656in}}%
\pgfpathlineto{\pgfqpoint{1.341109in}{0.739656in}}%
\pgfpathlineto{\pgfqpoint{1.340811in}{0.739656in}}%
\pgfpathlineto{\pgfqpoint{1.340514in}{0.739656in}}%
\pgfpathlineto{\pgfqpoint{1.340216in}{0.739656in}}%
\pgfpathlineto{\pgfqpoint{1.339919in}{0.739656in}}%
\pgfpathlineto{\pgfqpoint{1.339621in}{0.739656in}}%
\pgfpathlineto{\pgfqpoint{1.339324in}{0.739656in}}%
\pgfpathlineto{\pgfqpoint{1.339026in}{0.739656in}}%
\pgfpathlineto{\pgfqpoint{1.338729in}{0.739656in}}%
\pgfpathlineto{\pgfqpoint{1.338431in}{0.739656in}}%
\pgfpathlineto{\pgfqpoint{1.338134in}{0.739656in}}%
\pgfpathlineto{\pgfqpoint{1.337836in}{0.739656in}}%
\pgfpathlineto{\pgfqpoint{1.337539in}{0.739656in}}%
\pgfpathlineto{\pgfqpoint{1.337241in}{0.739656in}}%
\pgfpathlineto{\pgfqpoint{1.336944in}{0.739656in}}%
\pgfpathlineto{\pgfqpoint{1.336646in}{0.739656in}}%
\pgfpathlineto{\pgfqpoint{1.336349in}{0.739656in}}%
\pgfpathlineto{\pgfqpoint{1.336051in}{0.739656in}}%
\pgfpathlineto{\pgfqpoint{1.335754in}{0.739656in}}%
\pgfpathlineto{\pgfqpoint{1.335457in}{0.739656in}}%
\pgfpathlineto{\pgfqpoint{1.335159in}{0.739656in}}%
\pgfpathlineto{\pgfqpoint{1.334862in}{0.739656in}}%
\pgfpathlineto{\pgfqpoint{1.334564in}{0.739656in}}%
\pgfpathlineto{\pgfqpoint{1.334267in}{0.739656in}}%
\pgfpathlineto{\pgfqpoint{1.333969in}{0.739656in}}%
\pgfpathlineto{\pgfqpoint{1.333672in}{0.739656in}}%
\pgfpathlineto{\pgfqpoint{1.333374in}{0.739656in}}%
\pgfpathlineto{\pgfqpoint{1.333077in}{0.739656in}}%
\pgfpathlineto{\pgfqpoint{1.332779in}{0.739656in}}%
\pgfpathlineto{\pgfqpoint{1.332482in}{0.739656in}}%
\pgfpathlineto{\pgfqpoint{1.332184in}{0.739656in}}%
\pgfpathlineto{\pgfqpoint{1.331887in}{0.739656in}}%
\pgfpathlineto{\pgfqpoint{1.331589in}{0.739656in}}%
\pgfpathlineto{\pgfqpoint{1.331292in}{0.739656in}}%
\pgfpathlineto{\pgfqpoint{1.330994in}{0.739656in}}%
\pgfpathlineto{\pgfqpoint{1.330697in}{0.739656in}}%
\pgfpathlineto{\pgfqpoint{1.330399in}{0.739656in}}%
\pgfpathlineto{\pgfqpoint{1.330102in}{0.739656in}}%
\pgfpathlineto{\pgfqpoint{1.329804in}{0.739656in}}%
\pgfpathlineto{\pgfqpoint{1.329507in}{0.739656in}}%
\pgfpathlineto{\pgfqpoint{1.329209in}{0.739656in}}%
\pgfpathlineto{\pgfqpoint{1.328912in}{0.739656in}}%
\pgfpathlineto{\pgfqpoint{1.328615in}{0.739656in}}%
\pgfpathlineto{\pgfqpoint{1.328317in}{0.739656in}}%
\pgfpathlineto{\pgfqpoint{1.328020in}{0.739656in}}%
\pgfpathlineto{\pgfqpoint{1.327722in}{0.739656in}}%
\pgfpathlineto{\pgfqpoint{1.327425in}{0.739656in}}%
\pgfpathlineto{\pgfqpoint{1.327127in}{0.739656in}}%
\pgfpathlineto{\pgfqpoint{1.326830in}{0.739656in}}%
\pgfpathlineto{\pgfqpoint{1.326532in}{0.739656in}}%
\pgfpathlineto{\pgfqpoint{1.326235in}{0.739656in}}%
\pgfpathlineto{\pgfqpoint{1.325937in}{0.739656in}}%
\pgfpathlineto{\pgfqpoint{1.325640in}{0.739656in}}%
\pgfpathlineto{\pgfqpoint{1.325342in}{0.739656in}}%
\pgfpathlineto{\pgfqpoint{1.325045in}{0.739656in}}%
\pgfpathlineto{\pgfqpoint{1.324747in}{0.739656in}}%
\pgfpathlineto{\pgfqpoint{1.324450in}{0.739656in}}%
\pgfpathlineto{\pgfqpoint{1.324152in}{0.739656in}}%
\pgfpathlineto{\pgfqpoint{1.323855in}{0.739656in}}%
\pgfpathlineto{\pgfqpoint{1.323557in}{0.739656in}}%
\pgfpathlineto{\pgfqpoint{1.323260in}{0.739656in}}%
\pgfpathlineto{\pgfqpoint{1.322962in}{0.739656in}}%
\pgfpathlineto{\pgfqpoint{1.322665in}{0.739656in}}%
\pgfpathlineto{\pgfqpoint{1.322367in}{0.739656in}}%
\pgfpathlineto{\pgfqpoint{1.322070in}{0.739656in}}%
\pgfpathlineto{\pgfqpoint{1.321773in}{0.739656in}}%
\pgfpathlineto{\pgfqpoint{1.321475in}{0.739656in}}%
\pgfpathlineto{\pgfqpoint{1.321178in}{0.739656in}}%
\pgfpathlineto{\pgfqpoint{1.320880in}{0.739656in}}%
\pgfpathlineto{\pgfqpoint{1.320583in}{0.739656in}}%
\pgfpathlineto{\pgfqpoint{1.320285in}{0.739656in}}%
\pgfpathlineto{\pgfqpoint{1.319988in}{0.739656in}}%
\pgfpathlineto{\pgfqpoint{1.319690in}{0.739656in}}%
\pgfpathlineto{\pgfqpoint{1.319393in}{0.739656in}}%
\pgfpathlineto{\pgfqpoint{1.319095in}{0.739656in}}%
\pgfpathlineto{\pgfqpoint{1.318798in}{0.739656in}}%
\pgfpathlineto{\pgfqpoint{1.318500in}{0.739656in}}%
\pgfpathlineto{\pgfqpoint{1.318203in}{0.739656in}}%
\pgfpathlineto{\pgfqpoint{1.317905in}{0.739656in}}%
\pgfpathlineto{\pgfqpoint{1.317608in}{0.739656in}}%
\pgfpathlineto{\pgfqpoint{1.317310in}{0.739656in}}%
\pgfpathlineto{\pgfqpoint{1.317013in}{0.739656in}}%
\pgfpathlineto{\pgfqpoint{1.316715in}{0.739656in}}%
\pgfpathlineto{\pgfqpoint{1.316418in}{0.739656in}}%
\pgfpathlineto{\pgfqpoint{1.316120in}{0.739656in}}%
\pgfpathlineto{\pgfqpoint{1.315823in}{0.739656in}}%
\pgfpathlineto{\pgfqpoint{1.315526in}{0.739656in}}%
\pgfpathlineto{\pgfqpoint{1.315228in}{0.739656in}}%
\pgfpathlineto{\pgfqpoint{1.314931in}{0.739656in}}%
\pgfpathlineto{\pgfqpoint{1.314633in}{0.739656in}}%
\pgfpathlineto{\pgfqpoint{1.314336in}{0.739656in}}%
\pgfpathlineto{\pgfqpoint{1.314038in}{0.739656in}}%
\pgfpathlineto{\pgfqpoint{1.313741in}{0.739656in}}%
\pgfpathlineto{\pgfqpoint{1.313443in}{0.739656in}}%
\pgfpathlineto{\pgfqpoint{1.313146in}{0.739656in}}%
\pgfpathlineto{\pgfqpoint{1.312848in}{0.739656in}}%
\pgfpathlineto{\pgfqpoint{1.312551in}{0.739656in}}%
\pgfpathlineto{\pgfqpoint{1.312253in}{0.739656in}}%
\pgfpathlineto{\pgfqpoint{1.311956in}{0.739656in}}%
\pgfpathlineto{\pgfqpoint{1.311658in}{0.739656in}}%
\pgfpathlineto{\pgfqpoint{1.311361in}{0.739656in}}%
\pgfpathlineto{\pgfqpoint{1.311063in}{0.739656in}}%
\pgfpathlineto{\pgfqpoint{1.310766in}{0.739656in}}%
\pgfpathlineto{\pgfqpoint{1.310468in}{0.739656in}}%
\pgfpathlineto{\pgfqpoint{1.310171in}{0.739656in}}%
\pgfpathlineto{\pgfqpoint{1.309873in}{0.739656in}}%
\pgfpathlineto{\pgfqpoint{1.309576in}{0.739656in}}%
\pgfpathlineto{\pgfqpoint{1.309278in}{0.739656in}}%
\pgfpathlineto{\pgfqpoint{1.308981in}{0.739656in}}%
\pgfpathlineto{\pgfqpoint{1.308684in}{0.739656in}}%
\pgfpathlineto{\pgfqpoint{1.308386in}{0.739656in}}%
\pgfpathlineto{\pgfqpoint{1.308089in}{0.739656in}}%
\pgfpathlineto{\pgfqpoint{1.307791in}{0.739656in}}%
\pgfpathlineto{\pgfqpoint{1.307494in}{0.739656in}}%
\pgfpathlineto{\pgfqpoint{1.307196in}{0.739656in}}%
\pgfpathlineto{\pgfqpoint{1.306899in}{0.739656in}}%
\pgfpathlineto{\pgfqpoint{1.306601in}{0.739656in}}%
\pgfpathlineto{\pgfqpoint{1.306304in}{0.739656in}}%
\pgfpathlineto{\pgfqpoint{1.306006in}{0.739656in}}%
\pgfpathlineto{\pgfqpoint{1.305709in}{0.739656in}}%
\pgfpathlineto{\pgfqpoint{1.305411in}{0.739656in}}%
\pgfpathlineto{\pgfqpoint{1.305114in}{0.739656in}}%
\pgfpathlineto{\pgfqpoint{1.304816in}{0.739656in}}%
\pgfpathlineto{\pgfqpoint{1.304519in}{0.739656in}}%
\pgfpathlineto{\pgfqpoint{1.304221in}{0.739656in}}%
\pgfpathlineto{\pgfqpoint{1.303924in}{0.739656in}}%
\pgfpathlineto{\pgfqpoint{1.303626in}{0.739656in}}%
\pgfpathlineto{\pgfqpoint{1.303329in}{0.739656in}}%
\pgfpathlineto{\pgfqpoint{1.303031in}{0.739656in}}%
\pgfpathlineto{\pgfqpoint{1.302734in}{0.739656in}}%
\pgfpathlineto{\pgfqpoint{1.302436in}{0.739656in}}%
\pgfpathlineto{\pgfqpoint{1.302139in}{0.739656in}}%
\pgfpathlineto{\pgfqpoint{1.301842in}{0.739656in}}%
\pgfpathlineto{\pgfqpoint{1.301544in}{0.739656in}}%
\pgfpathlineto{\pgfqpoint{1.301247in}{0.739656in}}%
\pgfpathlineto{\pgfqpoint{1.300949in}{0.739656in}}%
\pgfpathlineto{\pgfqpoint{1.300652in}{0.739656in}}%
\pgfpathlineto{\pgfqpoint{1.300354in}{0.739656in}}%
\pgfpathlineto{\pgfqpoint{1.300057in}{0.739656in}}%
\pgfpathlineto{\pgfqpoint{1.299759in}{0.739656in}}%
\pgfpathlineto{\pgfqpoint{1.299462in}{0.739656in}}%
\pgfpathlineto{\pgfqpoint{1.299164in}{0.739656in}}%
\pgfpathlineto{\pgfqpoint{1.298867in}{0.739656in}}%
\pgfpathlineto{\pgfqpoint{1.298569in}{0.739656in}}%
\pgfpathlineto{\pgfqpoint{1.298272in}{0.739656in}}%
\pgfpathlineto{\pgfqpoint{1.297974in}{0.739656in}}%
\pgfpathlineto{\pgfqpoint{1.297677in}{0.739656in}}%
\pgfpathlineto{\pgfqpoint{1.297379in}{0.739656in}}%
\pgfpathlineto{\pgfqpoint{1.297082in}{0.739656in}}%
\pgfpathlineto{\pgfqpoint{1.296784in}{0.739656in}}%
\pgfpathlineto{\pgfqpoint{1.296487in}{0.739656in}}%
\pgfpathlineto{\pgfqpoint{1.296189in}{0.739656in}}%
\pgfpathlineto{\pgfqpoint{1.295892in}{0.739656in}}%
\pgfpathlineto{\pgfqpoint{1.295595in}{0.739656in}}%
\pgfpathlineto{\pgfqpoint{1.295297in}{0.739656in}}%
\pgfpathlineto{\pgfqpoint{1.295000in}{0.739656in}}%
\pgfpathlineto{\pgfqpoint{1.294702in}{0.739656in}}%
\pgfpathlineto{\pgfqpoint{1.294405in}{0.739656in}}%
\pgfpathlineto{\pgfqpoint{1.294107in}{0.739656in}}%
\pgfpathlineto{\pgfqpoint{1.293810in}{0.739656in}}%
\pgfpathlineto{\pgfqpoint{1.293512in}{0.739656in}}%
\pgfpathlineto{\pgfqpoint{1.293215in}{0.739656in}}%
\pgfpathlineto{\pgfqpoint{1.292917in}{0.739656in}}%
\pgfpathlineto{\pgfqpoint{1.292620in}{0.739656in}}%
\pgfpathlineto{\pgfqpoint{1.292322in}{0.739656in}}%
\pgfpathlineto{\pgfqpoint{1.292025in}{0.739656in}}%
\pgfpathlineto{\pgfqpoint{1.291727in}{0.739656in}}%
\pgfpathlineto{\pgfqpoint{1.291430in}{0.739656in}}%
\pgfpathlineto{\pgfqpoint{1.291132in}{0.739656in}}%
\pgfpathlineto{\pgfqpoint{1.290835in}{0.739656in}}%
\pgfpathlineto{\pgfqpoint{1.290537in}{0.739656in}}%
\pgfpathlineto{\pgfqpoint{1.290240in}{0.739656in}}%
\pgfpathlineto{\pgfqpoint{1.289942in}{0.739656in}}%
\pgfpathlineto{\pgfqpoint{1.289645in}{0.739656in}}%
\pgfpathlineto{\pgfqpoint{1.289347in}{0.739656in}}%
\pgfpathlineto{\pgfqpoint{1.289050in}{0.739656in}}%
\pgfpathlineto{\pgfqpoint{1.288753in}{0.739656in}}%
\pgfpathlineto{\pgfqpoint{1.288455in}{0.739656in}}%
\pgfpathlineto{\pgfqpoint{1.288158in}{0.739656in}}%
\pgfpathlineto{\pgfqpoint{1.287860in}{0.739656in}}%
\pgfpathlineto{\pgfqpoint{1.287563in}{0.739656in}}%
\pgfpathlineto{\pgfqpoint{1.287265in}{0.739656in}}%
\pgfpathlineto{\pgfqpoint{1.286968in}{0.739656in}}%
\pgfpathlineto{\pgfqpoint{1.286670in}{0.739656in}}%
\pgfpathlineto{\pgfqpoint{1.286373in}{0.739656in}}%
\pgfpathlineto{\pgfqpoint{1.286075in}{0.739656in}}%
\pgfpathlineto{\pgfqpoint{1.285778in}{0.739656in}}%
\pgfpathlineto{\pgfqpoint{1.285480in}{0.739656in}}%
\pgfpathlineto{\pgfqpoint{1.285183in}{0.739656in}}%
\pgfpathlineto{\pgfqpoint{1.284885in}{0.739656in}}%
\pgfpathlineto{\pgfqpoint{1.284588in}{0.739656in}}%
\pgfpathlineto{\pgfqpoint{1.284290in}{0.739656in}}%
\pgfpathlineto{\pgfqpoint{1.283993in}{0.739656in}}%
\pgfpathlineto{\pgfqpoint{1.283695in}{0.739656in}}%
\pgfpathlineto{\pgfqpoint{1.283398in}{0.739656in}}%
\pgfpathlineto{\pgfqpoint{1.283100in}{0.739656in}}%
\pgfpathlineto{\pgfqpoint{1.282803in}{0.739656in}}%
\pgfpathlineto{\pgfqpoint{1.282505in}{0.739656in}}%
\pgfpathlineto{\pgfqpoint{1.282208in}{0.739656in}}%
\pgfpathlineto{\pgfqpoint{1.281911in}{0.739656in}}%
\pgfpathlineto{\pgfqpoint{1.281613in}{0.739656in}}%
\pgfpathlineto{\pgfqpoint{1.281316in}{0.739656in}}%
\pgfpathlineto{\pgfqpoint{1.281018in}{0.739656in}}%
\pgfpathlineto{\pgfqpoint{1.280721in}{0.739656in}}%
\pgfpathlineto{\pgfqpoint{1.280423in}{0.739656in}}%
\pgfpathlineto{\pgfqpoint{1.280126in}{0.739656in}}%
\pgfpathlineto{\pgfqpoint{1.279828in}{0.739656in}}%
\pgfpathlineto{\pgfqpoint{1.279531in}{0.739656in}}%
\pgfpathlineto{\pgfqpoint{1.279233in}{0.739656in}}%
\pgfpathlineto{\pgfqpoint{1.278936in}{0.739656in}}%
\pgfpathlineto{\pgfqpoint{1.278638in}{0.739656in}}%
\pgfpathlineto{\pgfqpoint{1.278341in}{0.739656in}}%
\pgfpathlineto{\pgfqpoint{1.278043in}{0.739656in}}%
\pgfpathlineto{\pgfqpoint{1.277746in}{0.739656in}}%
\pgfpathlineto{\pgfqpoint{1.277448in}{0.739656in}}%
\pgfpathlineto{\pgfqpoint{1.277151in}{0.739656in}}%
\pgfpathlineto{\pgfqpoint{1.276853in}{0.739656in}}%
\pgfpathlineto{\pgfqpoint{1.276556in}{0.739656in}}%
\pgfpathlineto{\pgfqpoint{1.276258in}{0.739656in}}%
\pgfpathlineto{\pgfqpoint{1.275961in}{0.739656in}}%
\pgfpathlineto{\pgfqpoint{1.275664in}{0.739656in}}%
\pgfpathlineto{\pgfqpoint{1.275366in}{0.739656in}}%
\pgfpathlineto{\pgfqpoint{1.275069in}{0.739656in}}%
\pgfpathlineto{\pgfqpoint{1.274771in}{0.739656in}}%
\pgfpathlineto{\pgfqpoint{1.274474in}{0.739656in}}%
\pgfpathlineto{\pgfqpoint{1.274176in}{0.739656in}}%
\pgfpathlineto{\pgfqpoint{1.273879in}{0.739656in}}%
\pgfpathlineto{\pgfqpoint{1.273581in}{0.739656in}}%
\pgfpathlineto{\pgfqpoint{1.273284in}{0.739656in}}%
\pgfpathlineto{\pgfqpoint{1.272986in}{0.739656in}}%
\pgfpathlineto{\pgfqpoint{1.272689in}{0.739656in}}%
\pgfpathlineto{\pgfqpoint{1.272391in}{0.739656in}}%
\pgfpathlineto{\pgfqpoint{1.272094in}{0.739656in}}%
\pgfpathlineto{\pgfqpoint{1.271796in}{0.739656in}}%
\pgfpathlineto{\pgfqpoint{1.271499in}{0.739656in}}%
\pgfpathlineto{\pgfqpoint{1.271201in}{0.739656in}}%
\pgfpathlineto{\pgfqpoint{1.270904in}{0.739656in}}%
\pgfpathlineto{\pgfqpoint{1.270606in}{0.739656in}}%
\pgfpathlineto{\pgfqpoint{1.270309in}{0.739656in}}%
\pgfpathlineto{\pgfqpoint{1.270011in}{0.739656in}}%
\pgfpathlineto{\pgfqpoint{1.269714in}{0.739656in}}%
\pgfpathlineto{\pgfqpoint{1.269416in}{0.739656in}}%
\pgfpathlineto{\pgfqpoint{1.269119in}{0.739656in}}%
\pgfpathlineto{\pgfqpoint{1.268822in}{0.739656in}}%
\pgfpathlineto{\pgfqpoint{1.268524in}{0.739656in}}%
\pgfpathlineto{\pgfqpoint{1.268227in}{0.739656in}}%
\pgfpathlineto{\pgfqpoint{1.267929in}{0.739656in}}%
\pgfpathlineto{\pgfqpoint{1.267632in}{0.739656in}}%
\pgfpathlineto{\pgfqpoint{1.267334in}{0.739656in}}%
\pgfpathlineto{\pgfqpoint{1.267037in}{0.739656in}}%
\pgfpathlineto{\pgfqpoint{1.266739in}{0.739656in}}%
\pgfpathlineto{\pgfqpoint{1.266442in}{0.739656in}}%
\pgfpathlineto{\pgfqpoint{1.266144in}{0.739656in}}%
\pgfpathlineto{\pgfqpoint{1.265847in}{0.739656in}}%
\pgfpathlineto{\pgfqpoint{1.265549in}{0.739656in}}%
\pgfpathlineto{\pgfqpoint{1.265252in}{0.739656in}}%
\pgfpathlineto{\pgfqpoint{1.264954in}{0.739656in}}%
\pgfpathlineto{\pgfqpoint{1.264657in}{0.739656in}}%
\pgfpathlineto{\pgfqpoint{1.264359in}{0.739656in}}%
\pgfpathlineto{\pgfqpoint{1.264062in}{0.739656in}}%
\pgfpathlineto{\pgfqpoint{1.263764in}{0.739656in}}%
\pgfpathlineto{\pgfqpoint{1.263467in}{0.739656in}}%
\pgfpathlineto{\pgfqpoint{1.263169in}{0.739656in}}%
\pgfpathlineto{\pgfqpoint{1.262872in}{0.739656in}}%
\pgfpathlineto{\pgfqpoint{1.262574in}{0.739656in}}%
\pgfpathlineto{\pgfqpoint{1.262277in}{0.739656in}}%
\pgfpathlineto{\pgfqpoint{1.261980in}{0.739656in}}%
\pgfpathlineto{\pgfqpoint{1.261682in}{0.739656in}}%
\pgfpathlineto{\pgfqpoint{1.261385in}{0.739656in}}%
\pgfpathlineto{\pgfqpoint{1.261087in}{0.739656in}}%
\pgfpathlineto{\pgfqpoint{1.260790in}{0.739656in}}%
\pgfpathlineto{\pgfqpoint{1.260492in}{0.739656in}}%
\pgfpathlineto{\pgfqpoint{1.260195in}{0.739656in}}%
\pgfpathlineto{\pgfqpoint{1.259897in}{0.739656in}}%
\pgfpathlineto{\pgfqpoint{1.259600in}{0.739656in}}%
\pgfpathlineto{\pgfqpoint{1.259302in}{0.739656in}}%
\pgfpathlineto{\pgfqpoint{1.259005in}{0.739656in}}%
\pgfpathlineto{\pgfqpoint{1.258707in}{0.739656in}}%
\pgfpathlineto{\pgfqpoint{1.258410in}{0.739656in}}%
\pgfpathlineto{\pgfqpoint{1.258112in}{0.739656in}}%
\pgfpathlineto{\pgfqpoint{1.257815in}{0.739656in}}%
\pgfpathlineto{\pgfqpoint{1.257517in}{0.739656in}}%
\pgfpathlineto{\pgfqpoint{1.257220in}{0.739656in}}%
\pgfpathlineto{\pgfqpoint{1.256922in}{0.739656in}}%
\pgfpathlineto{\pgfqpoint{1.256625in}{0.739656in}}%
\pgfpathlineto{\pgfqpoint{1.256327in}{0.739656in}}%
\pgfpathlineto{\pgfqpoint{1.256030in}{0.739656in}}%
\pgfpathlineto{\pgfqpoint{1.255733in}{0.739656in}}%
\pgfpathlineto{\pgfqpoint{1.255435in}{0.739656in}}%
\pgfpathlineto{\pgfqpoint{1.255138in}{0.739656in}}%
\pgfpathlineto{\pgfqpoint{1.254840in}{0.739656in}}%
\pgfpathlineto{\pgfqpoint{1.254543in}{0.739656in}}%
\pgfpathlineto{\pgfqpoint{1.254245in}{0.739656in}}%
\pgfpathlineto{\pgfqpoint{1.253948in}{0.739656in}}%
\pgfpathlineto{\pgfqpoint{1.253650in}{0.739656in}}%
\pgfpathlineto{\pgfqpoint{1.253353in}{0.739656in}}%
\pgfpathlineto{\pgfqpoint{1.253055in}{0.739656in}}%
\pgfpathlineto{\pgfqpoint{1.252758in}{0.739656in}}%
\pgfpathlineto{\pgfqpoint{1.252460in}{0.739656in}}%
\pgfpathlineto{\pgfqpoint{1.252163in}{0.739656in}}%
\pgfpathlineto{\pgfqpoint{1.251865in}{0.739656in}}%
\pgfpathlineto{\pgfqpoint{1.251568in}{0.739656in}}%
\pgfpathlineto{\pgfqpoint{1.251270in}{0.739656in}}%
\pgfpathlineto{\pgfqpoint{1.250973in}{0.739656in}}%
\pgfpathlineto{\pgfqpoint{1.250675in}{0.739656in}}%
\pgfpathlineto{\pgfqpoint{1.250378in}{0.739656in}}%
\pgfpathlineto{\pgfqpoint{1.250080in}{0.739656in}}%
\pgfpathlineto{\pgfqpoint{1.249783in}{0.739656in}}%
\pgfpathlineto{\pgfqpoint{1.249485in}{0.739656in}}%
\pgfpathlineto{\pgfqpoint{1.249188in}{0.739656in}}%
\pgfpathlineto{\pgfqpoint{1.248891in}{0.739656in}}%
\pgfpathlineto{\pgfqpoint{1.248593in}{0.739656in}}%
\pgfpathlineto{\pgfqpoint{1.248296in}{0.739656in}}%
\pgfpathlineto{\pgfqpoint{1.247998in}{0.739656in}}%
\pgfpathlineto{\pgfqpoint{1.247701in}{0.739656in}}%
\pgfpathlineto{\pgfqpoint{1.247403in}{0.739656in}}%
\pgfpathlineto{\pgfqpoint{1.247106in}{0.739656in}}%
\pgfpathlineto{\pgfqpoint{1.246808in}{0.739656in}}%
\pgfpathlineto{\pgfqpoint{1.246511in}{0.739656in}}%
\pgfpathlineto{\pgfqpoint{1.246213in}{0.739656in}}%
\pgfpathlineto{\pgfqpoint{1.245916in}{0.739656in}}%
\pgfpathlineto{\pgfqpoint{1.245618in}{0.739656in}}%
\pgfpathlineto{\pgfqpoint{1.245321in}{0.739656in}}%
\pgfpathlineto{\pgfqpoint{1.245023in}{0.739656in}}%
\pgfpathlineto{\pgfqpoint{1.244726in}{0.739656in}}%
\pgfpathlineto{\pgfqpoint{1.244428in}{0.739656in}}%
\pgfpathlineto{\pgfqpoint{1.244131in}{0.739656in}}%
\pgfpathlineto{\pgfqpoint{1.243833in}{0.739656in}}%
\pgfpathlineto{\pgfqpoint{1.243536in}{0.739656in}}%
\pgfpathlineto{\pgfqpoint{1.243238in}{0.739656in}}%
\pgfpathlineto{\pgfqpoint{1.242941in}{0.739656in}}%
\pgfpathlineto{\pgfqpoint{1.242643in}{0.739656in}}%
\pgfpathlineto{\pgfqpoint{1.242346in}{0.739656in}}%
\pgfpathlineto{\pgfqpoint{1.242049in}{0.739656in}}%
\pgfpathlineto{\pgfqpoint{1.241751in}{0.739656in}}%
\pgfpathlineto{\pgfqpoint{1.241454in}{0.739656in}}%
\pgfpathlineto{\pgfqpoint{1.241156in}{0.739656in}}%
\pgfpathlineto{\pgfqpoint{1.240859in}{0.739656in}}%
\pgfpathlineto{\pgfqpoint{1.240561in}{0.739656in}}%
\pgfpathlineto{\pgfqpoint{1.240264in}{0.739656in}}%
\pgfpathlineto{\pgfqpoint{1.239966in}{0.739656in}}%
\pgfpathlineto{\pgfqpoint{1.239669in}{0.739656in}}%
\pgfpathlineto{\pgfqpoint{1.239371in}{0.739656in}}%
\pgfpathlineto{\pgfqpoint{1.239074in}{0.739656in}}%
\pgfpathlineto{\pgfqpoint{1.238776in}{0.739656in}}%
\pgfpathlineto{\pgfqpoint{1.238479in}{0.739656in}}%
\pgfpathlineto{\pgfqpoint{1.238181in}{0.739656in}}%
\pgfpathlineto{\pgfqpoint{1.237884in}{0.739656in}}%
\pgfpathlineto{\pgfqpoint{1.237586in}{0.739656in}}%
\pgfpathlineto{\pgfqpoint{1.237289in}{0.739656in}}%
\pgfpathlineto{\pgfqpoint{1.236991in}{0.739656in}}%
\pgfpathlineto{\pgfqpoint{1.236694in}{0.739656in}}%
\pgfpathlineto{\pgfqpoint{1.236396in}{0.739656in}}%
\pgfpathlineto{\pgfqpoint{1.236099in}{0.739656in}}%
\pgfpathlineto{\pgfqpoint{1.235802in}{0.739656in}}%
\pgfpathlineto{\pgfqpoint{1.235504in}{0.739656in}}%
\pgfpathlineto{\pgfqpoint{1.235207in}{0.739656in}}%
\pgfpathlineto{\pgfqpoint{1.234909in}{0.739656in}}%
\pgfpathlineto{\pgfqpoint{1.234612in}{0.739656in}}%
\pgfpathlineto{\pgfqpoint{1.234314in}{0.739656in}}%
\pgfpathlineto{\pgfqpoint{1.234017in}{0.739656in}}%
\pgfpathlineto{\pgfqpoint{1.233719in}{0.739656in}}%
\pgfpathlineto{\pgfqpoint{1.233422in}{0.739656in}}%
\pgfpathlineto{\pgfqpoint{1.233124in}{0.739656in}}%
\pgfpathlineto{\pgfqpoint{1.232827in}{0.739656in}}%
\pgfpathlineto{\pgfqpoint{1.232529in}{0.739656in}}%
\pgfpathlineto{\pgfqpoint{1.232232in}{0.739656in}}%
\pgfpathlineto{\pgfqpoint{1.231934in}{0.739656in}}%
\pgfpathlineto{\pgfqpoint{1.231637in}{0.739656in}}%
\pgfpathlineto{\pgfqpoint{1.231339in}{0.739656in}}%
\pgfpathlineto{\pgfqpoint{1.231042in}{0.739656in}}%
\pgfpathlineto{\pgfqpoint{1.230744in}{0.739656in}}%
\pgfpathlineto{\pgfqpoint{1.230447in}{0.739656in}}%
\pgfpathlineto{\pgfqpoint{1.230149in}{0.739656in}}%
\pgfpathlineto{\pgfqpoint{1.229852in}{0.739656in}}%
\pgfpathlineto{\pgfqpoint{1.229554in}{0.739656in}}%
\pgfpathlineto{\pgfqpoint{1.229257in}{0.739656in}}%
\pgfpathlineto{\pgfqpoint{1.228960in}{0.739656in}}%
\pgfpathlineto{\pgfqpoint{1.228662in}{0.739656in}}%
\pgfpathlineto{\pgfqpoint{1.228365in}{0.739656in}}%
\pgfpathlineto{\pgfqpoint{1.228067in}{0.739656in}}%
\pgfpathlineto{\pgfqpoint{1.227770in}{0.739656in}}%
\pgfpathlineto{\pgfqpoint{1.227472in}{0.739656in}}%
\pgfpathlineto{\pgfqpoint{1.227175in}{0.739656in}}%
\pgfpathlineto{\pgfqpoint{1.226877in}{0.739656in}}%
\pgfpathlineto{\pgfqpoint{1.226580in}{0.739656in}}%
\pgfpathlineto{\pgfqpoint{1.226282in}{0.739656in}}%
\pgfpathlineto{\pgfqpoint{1.225985in}{0.739656in}}%
\pgfpathlineto{\pgfqpoint{1.225687in}{0.739656in}}%
\pgfpathlineto{\pgfqpoint{1.225390in}{0.739656in}}%
\pgfpathlineto{\pgfqpoint{1.225092in}{0.739656in}}%
\pgfpathlineto{\pgfqpoint{1.224795in}{0.739656in}}%
\pgfpathlineto{\pgfqpoint{1.224497in}{0.739656in}}%
\pgfpathlineto{\pgfqpoint{1.224200in}{0.739656in}}%
\pgfpathlineto{\pgfqpoint{1.223902in}{0.739656in}}%
\pgfpathlineto{\pgfqpoint{1.223605in}{0.739656in}}%
\pgfpathlineto{\pgfqpoint{1.223307in}{0.739656in}}%
\pgfpathlineto{\pgfqpoint{1.223010in}{0.739656in}}%
\pgfpathlineto{\pgfqpoint{1.222712in}{0.739656in}}%
\pgfpathlineto{\pgfqpoint{1.222415in}{0.739656in}}%
\pgfpathlineto{\pgfqpoint{1.222118in}{0.739656in}}%
\pgfpathlineto{\pgfqpoint{1.221820in}{0.739656in}}%
\pgfpathlineto{\pgfqpoint{1.221523in}{0.739656in}}%
\pgfpathlineto{\pgfqpoint{1.221225in}{0.739656in}}%
\pgfpathlineto{\pgfqpoint{1.220928in}{0.739656in}}%
\pgfpathlineto{\pgfqpoint{1.220630in}{0.739656in}}%
\pgfpathlineto{\pgfqpoint{1.220333in}{0.739656in}}%
\pgfpathlineto{\pgfqpoint{1.220035in}{0.739656in}}%
\pgfpathlineto{\pgfqpoint{1.219738in}{0.739656in}}%
\pgfpathlineto{\pgfqpoint{1.219440in}{0.739656in}}%
\pgfpathlineto{\pgfqpoint{1.219143in}{0.739656in}}%
\pgfpathlineto{\pgfqpoint{1.218845in}{0.739656in}}%
\pgfpathlineto{\pgfqpoint{1.218548in}{0.739656in}}%
\pgfpathlineto{\pgfqpoint{1.218250in}{0.739656in}}%
\pgfpathlineto{\pgfqpoint{1.217953in}{0.739656in}}%
\pgfpathlineto{\pgfqpoint{1.217655in}{0.739656in}}%
\pgfpathlineto{\pgfqpoint{1.217358in}{0.739656in}}%
\pgfpathlineto{\pgfqpoint{1.217060in}{0.739656in}}%
\pgfpathlineto{\pgfqpoint{1.216763in}{0.739656in}}%
\pgfpathlineto{\pgfqpoint{1.216465in}{0.739656in}}%
\pgfpathlineto{\pgfqpoint{1.216168in}{0.739656in}}%
\pgfpathlineto{\pgfqpoint{1.215871in}{0.739656in}}%
\pgfpathlineto{\pgfqpoint{1.215573in}{0.739656in}}%
\pgfpathlineto{\pgfqpoint{1.215276in}{0.739656in}}%
\pgfpathlineto{\pgfqpoint{1.214978in}{0.739656in}}%
\pgfpathlineto{\pgfqpoint{1.214681in}{0.739656in}}%
\pgfpathlineto{\pgfqpoint{1.214383in}{0.739656in}}%
\pgfpathlineto{\pgfqpoint{1.214086in}{0.739656in}}%
\pgfpathlineto{\pgfqpoint{1.213788in}{0.739656in}}%
\pgfpathlineto{\pgfqpoint{1.213491in}{0.739656in}}%
\pgfpathlineto{\pgfqpoint{1.213193in}{0.739656in}}%
\pgfpathlineto{\pgfqpoint{1.212896in}{0.739656in}}%
\pgfpathlineto{\pgfqpoint{1.212598in}{0.739656in}}%
\pgfpathlineto{\pgfqpoint{1.212301in}{0.739656in}}%
\pgfpathlineto{\pgfqpoint{1.212003in}{0.739656in}}%
\pgfpathlineto{\pgfqpoint{1.211706in}{0.739656in}}%
\pgfpathlineto{\pgfqpoint{1.211408in}{0.739656in}}%
\pgfpathlineto{\pgfqpoint{1.211111in}{0.739656in}}%
\pgfpathlineto{\pgfqpoint{1.210813in}{0.739656in}}%
\pgfpathlineto{\pgfqpoint{1.210516in}{0.739656in}}%
\pgfpathlineto{\pgfqpoint{1.210218in}{0.739656in}}%
\pgfpathlineto{\pgfqpoint{1.209921in}{0.739656in}}%
\pgfpathlineto{\pgfqpoint{1.209623in}{0.739656in}}%
\pgfpathlineto{\pgfqpoint{1.209326in}{0.739656in}}%
\pgfpathlineto{\pgfqpoint{1.209029in}{0.739656in}}%
\pgfpathlineto{\pgfqpoint{1.208731in}{0.739656in}}%
\pgfpathlineto{\pgfqpoint{1.208434in}{0.739656in}}%
\pgfpathlineto{\pgfqpoint{1.208136in}{0.739656in}}%
\pgfpathlineto{\pgfqpoint{1.207839in}{0.739656in}}%
\pgfpathlineto{\pgfqpoint{1.207541in}{0.739656in}}%
\pgfpathlineto{\pgfqpoint{1.207244in}{0.739656in}}%
\pgfpathlineto{\pgfqpoint{1.206946in}{0.739656in}}%
\pgfpathlineto{\pgfqpoint{1.206649in}{0.739656in}}%
\pgfpathlineto{\pgfqpoint{1.206351in}{0.739656in}}%
\pgfpathlineto{\pgfqpoint{1.206054in}{0.739656in}}%
\pgfpathlineto{\pgfqpoint{1.205756in}{0.739656in}}%
\pgfpathlineto{\pgfqpoint{1.205459in}{0.739656in}}%
\pgfpathlineto{\pgfqpoint{1.205161in}{0.739656in}}%
\pgfpathlineto{\pgfqpoint{1.204864in}{0.739656in}}%
\pgfpathlineto{\pgfqpoint{1.204566in}{0.739656in}}%
\pgfpathlineto{\pgfqpoint{1.204269in}{0.739656in}}%
\pgfpathlineto{\pgfqpoint{1.203971in}{0.739656in}}%
\pgfpathlineto{\pgfqpoint{1.203674in}{0.739656in}}%
\pgfpathlineto{\pgfqpoint{1.203376in}{0.739656in}}%
\pgfpathlineto{\pgfqpoint{1.203079in}{0.739656in}}%
\pgfpathlineto{\pgfqpoint{1.202781in}{0.739656in}}%
\pgfpathlineto{\pgfqpoint{1.202484in}{0.739656in}}%
\pgfpathlineto{\pgfqpoint{1.202187in}{0.739656in}}%
\pgfpathlineto{\pgfqpoint{1.201889in}{0.739656in}}%
\pgfpathlineto{\pgfqpoint{1.201592in}{0.739656in}}%
\pgfpathlineto{\pgfqpoint{1.201294in}{0.739656in}}%
\pgfpathlineto{\pgfqpoint{1.200997in}{0.739656in}}%
\pgfpathlineto{\pgfqpoint{1.200699in}{0.739656in}}%
\pgfpathlineto{\pgfqpoint{1.200402in}{0.739656in}}%
\pgfpathlineto{\pgfqpoint{1.200104in}{0.739656in}}%
\pgfpathlineto{\pgfqpoint{1.199807in}{0.739656in}}%
\pgfpathlineto{\pgfqpoint{1.199509in}{0.739656in}}%
\pgfpathlineto{\pgfqpoint{1.199212in}{0.739656in}}%
\pgfpathlineto{\pgfqpoint{1.198914in}{0.739656in}}%
\pgfpathlineto{\pgfqpoint{1.198617in}{0.739656in}}%
\pgfpathlineto{\pgfqpoint{1.198319in}{0.739656in}}%
\pgfpathlineto{\pgfqpoint{1.198022in}{0.739656in}}%
\pgfpathlineto{\pgfqpoint{1.197724in}{0.739656in}}%
\pgfpathlineto{\pgfqpoint{1.197427in}{0.739656in}}%
\pgfpathlineto{\pgfqpoint{1.197129in}{0.739656in}}%
\pgfpathlineto{\pgfqpoint{1.196832in}{0.739656in}}%
\pgfpathlineto{\pgfqpoint{1.196534in}{0.739656in}}%
\pgfpathlineto{\pgfqpoint{1.196237in}{0.739656in}}%
\pgfpathlineto{\pgfqpoint{1.195940in}{0.739656in}}%
\pgfpathlineto{\pgfqpoint{1.195642in}{0.739656in}}%
\pgfpathlineto{\pgfqpoint{1.195345in}{0.739656in}}%
\pgfpathlineto{\pgfqpoint{1.195047in}{0.739656in}}%
\pgfpathlineto{\pgfqpoint{1.194750in}{0.739656in}}%
\pgfpathlineto{\pgfqpoint{1.194452in}{0.739656in}}%
\pgfpathlineto{\pgfqpoint{1.194155in}{0.739656in}}%
\pgfpathlineto{\pgfqpoint{1.193857in}{0.739656in}}%
\pgfpathlineto{\pgfqpoint{1.193560in}{0.739656in}}%
\pgfpathlineto{\pgfqpoint{1.193262in}{0.739656in}}%
\pgfpathlineto{\pgfqpoint{1.192965in}{0.739656in}}%
\pgfpathlineto{\pgfqpoint{1.192667in}{0.739656in}}%
\pgfpathlineto{\pgfqpoint{1.192370in}{0.739656in}}%
\pgfpathlineto{\pgfqpoint{1.192072in}{0.739656in}}%
\pgfpathlineto{\pgfqpoint{1.191775in}{0.739656in}}%
\pgfpathlineto{\pgfqpoint{1.191477in}{0.739656in}}%
\pgfpathlineto{\pgfqpoint{1.191180in}{0.739656in}}%
\pgfpathlineto{\pgfqpoint{1.190882in}{0.739656in}}%
\pgfpathlineto{\pgfqpoint{1.190585in}{0.739656in}}%
\pgfpathlineto{\pgfqpoint{1.190287in}{0.739656in}}%
\pgfpathlineto{\pgfqpoint{1.189990in}{0.739656in}}%
\pgfpathlineto{\pgfqpoint{1.189692in}{0.739656in}}%
\pgfpathlineto{\pgfqpoint{1.189395in}{0.739656in}}%
\pgfpathlineto{\pgfqpoint{1.189098in}{0.739656in}}%
\pgfpathlineto{\pgfqpoint{1.188800in}{0.739656in}}%
\pgfpathlineto{\pgfqpoint{1.188503in}{0.739656in}}%
\pgfpathlineto{\pgfqpoint{1.188205in}{0.739656in}}%
\pgfpathlineto{\pgfqpoint{1.187908in}{0.739656in}}%
\pgfpathlineto{\pgfqpoint{1.187610in}{0.739656in}}%
\pgfpathlineto{\pgfqpoint{1.187313in}{0.739656in}}%
\pgfpathlineto{\pgfqpoint{1.187015in}{0.739656in}}%
\pgfpathlineto{\pgfqpoint{1.186718in}{0.739656in}}%
\pgfpathlineto{\pgfqpoint{1.186420in}{0.739656in}}%
\pgfpathlineto{\pgfqpoint{1.186123in}{0.739656in}}%
\pgfpathlineto{\pgfqpoint{1.185825in}{0.739656in}}%
\pgfpathlineto{\pgfqpoint{1.185528in}{0.739656in}}%
\pgfpathlineto{\pgfqpoint{1.185230in}{0.739656in}}%
\pgfpathlineto{\pgfqpoint{1.184933in}{0.739656in}}%
\pgfpathlineto{\pgfqpoint{1.184635in}{0.739656in}}%
\pgfpathlineto{\pgfqpoint{1.184338in}{0.739656in}}%
\pgfpathlineto{\pgfqpoint{1.184040in}{0.739656in}}%
\pgfpathlineto{\pgfqpoint{1.183743in}{0.739656in}}%
\pgfpathlineto{\pgfqpoint{1.183445in}{0.739656in}}%
\pgfpathlineto{\pgfqpoint{1.183148in}{0.739656in}}%
\pgfpathlineto{\pgfqpoint{1.182850in}{0.739656in}}%
\pgfpathlineto{\pgfqpoint{1.182553in}{0.739656in}}%
\pgfpathlineto{\pgfqpoint{1.182256in}{0.739656in}}%
\pgfpathlineto{\pgfqpoint{1.181958in}{0.739656in}}%
\pgfpathlineto{\pgfqpoint{1.181661in}{0.739656in}}%
\pgfpathlineto{\pgfqpoint{1.181363in}{0.739656in}}%
\pgfpathlineto{\pgfqpoint{1.181066in}{0.739656in}}%
\pgfpathlineto{\pgfqpoint{1.180768in}{0.739656in}}%
\pgfpathlineto{\pgfqpoint{1.180471in}{0.739656in}}%
\pgfpathlineto{\pgfqpoint{1.180173in}{0.739656in}}%
\pgfpathlineto{\pgfqpoint{1.179876in}{0.739656in}}%
\pgfpathlineto{\pgfqpoint{1.179578in}{0.739656in}}%
\pgfpathlineto{\pgfqpoint{1.179281in}{0.739656in}}%
\pgfpathlineto{\pgfqpoint{1.178983in}{0.739656in}}%
\pgfpathlineto{\pgfqpoint{1.178686in}{0.739656in}}%
\pgfpathlineto{\pgfqpoint{1.178388in}{0.739656in}}%
\pgfpathlineto{\pgfqpoint{1.178091in}{0.739656in}}%
\pgfpathlineto{\pgfqpoint{1.177793in}{0.739656in}}%
\pgfpathlineto{\pgfqpoint{1.177496in}{0.739656in}}%
\pgfpathlineto{\pgfqpoint{1.177198in}{0.739656in}}%
\pgfpathlineto{\pgfqpoint{1.176901in}{0.739656in}}%
\pgfpathlineto{\pgfqpoint{1.176603in}{0.739656in}}%
\pgfpathlineto{\pgfqpoint{1.176306in}{0.739656in}}%
\pgfpathlineto{\pgfqpoint{1.176009in}{0.739656in}}%
\pgfpathlineto{\pgfqpoint{1.175711in}{0.739656in}}%
\pgfpathlineto{\pgfqpoint{1.175414in}{0.739656in}}%
\pgfpathlineto{\pgfqpoint{1.175116in}{0.739656in}}%
\pgfpathlineto{\pgfqpoint{1.174819in}{0.739656in}}%
\pgfpathlineto{\pgfqpoint{1.174521in}{0.739656in}}%
\pgfpathlineto{\pgfqpoint{1.174224in}{0.739656in}}%
\pgfpathlineto{\pgfqpoint{1.173926in}{0.739656in}}%
\pgfpathlineto{\pgfqpoint{1.173629in}{0.739656in}}%
\pgfpathlineto{\pgfqpoint{1.173331in}{0.739656in}}%
\pgfpathlineto{\pgfqpoint{1.173034in}{0.739656in}}%
\pgfpathlineto{\pgfqpoint{1.172736in}{0.739656in}}%
\pgfpathlineto{\pgfqpoint{1.172439in}{0.739656in}}%
\pgfpathlineto{\pgfqpoint{1.172141in}{0.739656in}}%
\pgfpathlineto{\pgfqpoint{1.171844in}{0.739656in}}%
\pgfpathlineto{\pgfqpoint{1.171546in}{0.739656in}}%
\pgfpathlineto{\pgfqpoint{1.171249in}{0.739656in}}%
\pgfpathlineto{\pgfqpoint{1.170951in}{0.739656in}}%
\pgfpathlineto{\pgfqpoint{1.170654in}{0.739656in}}%
\pgfpathlineto{\pgfqpoint{1.170356in}{0.739656in}}%
\pgfpathlineto{\pgfqpoint{1.170059in}{0.739656in}}%
\pgfpathlineto{\pgfqpoint{1.169761in}{0.739656in}}%
\pgfpathlineto{\pgfqpoint{1.169464in}{0.739656in}}%
\pgfpathlineto{\pgfqpoint{1.169167in}{0.739656in}}%
\pgfpathlineto{\pgfqpoint{1.168869in}{0.739656in}}%
\pgfpathlineto{\pgfqpoint{1.168572in}{0.739656in}}%
\pgfpathlineto{\pgfqpoint{1.168274in}{0.739656in}}%
\pgfpathlineto{\pgfqpoint{1.167977in}{0.739656in}}%
\pgfpathlineto{\pgfqpoint{1.167679in}{0.739656in}}%
\pgfpathlineto{\pgfqpoint{1.167382in}{0.739656in}}%
\pgfpathlineto{\pgfqpoint{1.167084in}{0.739656in}}%
\pgfpathlineto{\pgfqpoint{1.166787in}{0.739656in}}%
\pgfpathlineto{\pgfqpoint{1.166489in}{0.739656in}}%
\pgfpathlineto{\pgfqpoint{1.166192in}{0.739656in}}%
\pgfpathlineto{\pgfqpoint{1.165894in}{0.739656in}}%
\pgfpathlineto{\pgfqpoint{1.165597in}{0.739656in}}%
\pgfpathlineto{\pgfqpoint{1.165299in}{0.739656in}}%
\pgfpathlineto{\pgfqpoint{1.165002in}{0.739656in}}%
\pgfpathlineto{\pgfqpoint{1.164704in}{0.739656in}}%
\pgfpathlineto{\pgfqpoint{1.164407in}{0.739656in}}%
\pgfpathlineto{\pgfqpoint{1.164109in}{0.739656in}}%
\pgfpathlineto{\pgfqpoint{1.163812in}{0.739656in}}%
\pgfpathlineto{\pgfqpoint{1.163514in}{0.739656in}}%
\pgfpathlineto{\pgfqpoint{1.163217in}{0.739656in}}%
\pgfpathlineto{\pgfqpoint{1.162919in}{0.739656in}}%
\pgfpathlineto{\pgfqpoint{1.162622in}{0.739656in}}%
\pgfpathlineto{\pgfqpoint{1.162325in}{0.739656in}}%
\pgfpathlineto{\pgfqpoint{1.162027in}{0.739656in}}%
\pgfpathlineto{\pgfqpoint{1.161730in}{0.739656in}}%
\pgfpathlineto{\pgfqpoint{1.161432in}{0.739656in}}%
\pgfpathlineto{\pgfqpoint{1.161135in}{0.739656in}}%
\pgfpathlineto{\pgfqpoint{1.160837in}{0.739656in}}%
\pgfpathlineto{\pgfqpoint{1.160540in}{0.739656in}}%
\pgfpathlineto{\pgfqpoint{1.160242in}{0.739656in}}%
\pgfpathlineto{\pgfqpoint{1.159945in}{0.739656in}}%
\pgfpathlineto{\pgfqpoint{1.159647in}{0.739656in}}%
\pgfpathlineto{\pgfqpoint{1.159350in}{0.739656in}}%
\pgfpathlineto{\pgfqpoint{1.159052in}{0.739656in}}%
\pgfpathlineto{\pgfqpoint{1.158755in}{0.739656in}}%
\pgfpathlineto{\pgfqpoint{1.158457in}{0.739656in}}%
\pgfpathlineto{\pgfqpoint{1.158160in}{0.739656in}}%
\pgfpathlineto{\pgfqpoint{1.157862in}{0.739656in}}%
\pgfpathlineto{\pgfqpoint{1.157565in}{0.739656in}}%
\pgfpathlineto{\pgfqpoint{1.157267in}{0.739656in}}%
\pgfpathlineto{\pgfqpoint{1.156970in}{0.739656in}}%
\pgfpathlineto{\pgfqpoint{1.156672in}{0.739656in}}%
\pgfpathlineto{\pgfqpoint{1.156375in}{0.739656in}}%
\pgfpathlineto{\pgfqpoint{1.156078in}{0.739656in}}%
\pgfpathlineto{\pgfqpoint{1.155780in}{0.739656in}}%
\pgfpathlineto{\pgfqpoint{1.155483in}{0.739656in}}%
\pgfpathlineto{\pgfqpoint{1.155185in}{0.739656in}}%
\pgfpathlineto{\pgfqpoint{1.154888in}{0.739656in}}%
\pgfpathlineto{\pgfqpoint{1.154590in}{0.739656in}}%
\pgfpathlineto{\pgfqpoint{1.154293in}{0.739656in}}%
\pgfpathlineto{\pgfqpoint{1.153995in}{0.739656in}}%
\pgfpathlineto{\pgfqpoint{1.153698in}{0.739656in}}%
\pgfpathlineto{\pgfqpoint{1.153400in}{0.739656in}}%
\pgfpathlineto{\pgfqpoint{1.153103in}{0.739656in}}%
\pgfpathlineto{\pgfqpoint{1.152805in}{0.739656in}}%
\pgfpathlineto{\pgfqpoint{1.152508in}{0.739656in}}%
\pgfpathlineto{\pgfqpoint{1.152210in}{0.739656in}}%
\pgfpathlineto{\pgfqpoint{1.151913in}{0.739656in}}%
\pgfpathlineto{\pgfqpoint{1.151615in}{0.739656in}}%
\pgfpathlineto{\pgfqpoint{1.151318in}{0.739656in}}%
\pgfpathlineto{\pgfqpoint{1.151020in}{0.739656in}}%
\pgfpathlineto{\pgfqpoint{1.150723in}{0.739656in}}%
\pgfpathlineto{\pgfqpoint{1.150425in}{0.739656in}}%
\pgfpathlineto{\pgfqpoint{1.150128in}{0.739656in}}%
\pgfpathlineto{\pgfqpoint{1.149830in}{0.739656in}}%
\pgfpathlineto{\pgfqpoint{1.149533in}{0.739656in}}%
\pgfpathlineto{\pgfqpoint{1.149236in}{0.739656in}}%
\pgfpathlineto{\pgfqpoint{1.148938in}{0.739656in}}%
\pgfpathlineto{\pgfqpoint{1.148641in}{0.739656in}}%
\pgfpathlineto{\pgfqpoint{1.148343in}{0.739656in}}%
\pgfpathlineto{\pgfqpoint{1.148046in}{0.739656in}}%
\pgfpathlineto{\pgfqpoint{1.147748in}{0.739656in}}%
\pgfpathlineto{\pgfqpoint{1.147451in}{0.739656in}}%
\pgfpathlineto{\pgfqpoint{1.147153in}{0.739656in}}%
\pgfpathlineto{\pgfqpoint{1.146856in}{0.739656in}}%
\pgfpathlineto{\pgfqpoint{1.146558in}{0.739656in}}%
\pgfpathlineto{\pgfqpoint{1.146261in}{0.739656in}}%
\pgfpathlineto{\pgfqpoint{1.145963in}{0.739656in}}%
\pgfpathlineto{\pgfqpoint{1.145666in}{0.739656in}}%
\pgfpathlineto{\pgfqpoint{1.145368in}{0.739656in}}%
\pgfpathlineto{\pgfqpoint{1.145071in}{0.739656in}}%
\pgfpathlineto{\pgfqpoint{1.144773in}{0.739656in}}%
\pgfpathlineto{\pgfqpoint{1.144476in}{0.739656in}}%
\pgfpathlineto{\pgfqpoint{1.144178in}{0.739656in}}%
\pgfpathlineto{\pgfqpoint{1.143881in}{0.739656in}}%
\pgfpathlineto{\pgfqpoint{1.143583in}{0.739656in}}%
\pgfpathlineto{\pgfqpoint{1.143286in}{0.739656in}}%
\pgfpathlineto{\pgfqpoint{1.142988in}{0.739656in}}%
\pgfpathlineto{\pgfqpoint{1.142691in}{0.739656in}}%
\pgfpathlineto{\pgfqpoint{1.142394in}{0.739656in}}%
\pgfpathlineto{\pgfqpoint{1.142096in}{0.739656in}}%
\pgfpathlineto{\pgfqpoint{1.141799in}{0.739656in}}%
\pgfpathlineto{\pgfqpoint{1.141501in}{0.739656in}}%
\pgfpathlineto{\pgfqpoint{1.141204in}{0.739656in}}%
\pgfpathlineto{\pgfqpoint{1.140906in}{0.739656in}}%
\pgfpathlineto{\pgfqpoint{1.140609in}{0.739656in}}%
\pgfpathlineto{\pgfqpoint{1.140311in}{0.739656in}}%
\pgfpathlineto{\pgfqpoint{1.140014in}{0.739656in}}%
\pgfpathlineto{\pgfqpoint{1.139716in}{0.739656in}}%
\pgfpathlineto{\pgfqpoint{1.139419in}{0.739656in}}%
\pgfpathlineto{\pgfqpoint{1.139121in}{0.739656in}}%
\pgfpathlineto{\pgfqpoint{1.138824in}{0.739656in}}%
\pgfpathlineto{\pgfqpoint{1.138526in}{0.739656in}}%
\pgfpathlineto{\pgfqpoint{1.138229in}{0.739656in}}%
\pgfpathlineto{\pgfqpoint{1.137931in}{0.739656in}}%
\pgfpathlineto{\pgfqpoint{1.137634in}{0.739656in}}%
\pgfpathlineto{\pgfqpoint{1.137336in}{0.739656in}}%
\pgfpathlineto{\pgfqpoint{1.137039in}{0.739656in}}%
\pgfpathlineto{\pgfqpoint{1.136741in}{0.739656in}}%
\pgfpathlineto{\pgfqpoint{1.136444in}{0.739656in}}%
\pgfpathlineto{\pgfqpoint{1.136146in}{0.739656in}}%
\pgfpathlineto{\pgfqpoint{1.135849in}{0.739656in}}%
\pgfpathlineto{\pgfqpoint{1.135552in}{0.739656in}}%
\pgfpathlineto{\pgfqpoint{1.135254in}{0.739656in}}%
\pgfpathlineto{\pgfqpoint{1.134957in}{0.739656in}}%
\pgfpathlineto{\pgfqpoint{1.134659in}{0.739656in}}%
\pgfpathlineto{\pgfqpoint{1.134362in}{0.739656in}}%
\pgfpathlineto{\pgfqpoint{1.134064in}{0.739656in}}%
\pgfpathlineto{\pgfqpoint{1.133767in}{0.739656in}}%
\pgfpathlineto{\pgfqpoint{1.133469in}{0.739656in}}%
\pgfpathlineto{\pgfqpoint{1.133172in}{0.739656in}}%
\pgfpathlineto{\pgfqpoint{1.132874in}{0.739656in}}%
\pgfpathlineto{\pgfqpoint{1.132577in}{0.739656in}}%
\pgfpathlineto{\pgfqpoint{1.132279in}{0.739656in}}%
\pgfpathlineto{\pgfqpoint{1.131982in}{0.739656in}}%
\pgfpathlineto{\pgfqpoint{1.131684in}{0.739656in}}%
\pgfpathlineto{\pgfqpoint{1.131387in}{0.739656in}}%
\pgfpathlineto{\pgfqpoint{1.131089in}{0.739656in}}%
\pgfpathlineto{\pgfqpoint{1.130792in}{0.739656in}}%
\pgfpathlineto{\pgfqpoint{1.130494in}{0.739656in}}%
\pgfpathlineto{\pgfqpoint{1.130197in}{0.739656in}}%
\pgfpathlineto{\pgfqpoint{1.129899in}{0.739656in}}%
\pgfpathlineto{\pgfqpoint{1.129602in}{0.739656in}}%
\pgfpathlineto{\pgfqpoint{1.129305in}{0.739656in}}%
\pgfpathlineto{\pgfqpoint{1.129007in}{0.739656in}}%
\pgfpathlineto{\pgfqpoint{1.128710in}{0.739656in}}%
\pgfpathlineto{\pgfqpoint{1.128412in}{0.739656in}}%
\pgfpathlineto{\pgfqpoint{1.128115in}{0.739656in}}%
\pgfpathlineto{\pgfqpoint{1.127817in}{0.739656in}}%
\pgfpathlineto{\pgfqpoint{1.127520in}{0.739656in}}%
\pgfpathlineto{\pgfqpoint{1.127222in}{0.739656in}}%
\pgfpathlineto{\pgfqpoint{1.126925in}{0.739656in}}%
\pgfpathlineto{\pgfqpoint{1.126627in}{0.739656in}}%
\pgfpathlineto{\pgfqpoint{1.126330in}{0.739656in}}%
\pgfpathlineto{\pgfqpoint{1.126032in}{0.739656in}}%
\pgfpathlineto{\pgfqpoint{1.125735in}{0.739656in}}%
\pgfpathlineto{\pgfqpoint{1.125437in}{0.739656in}}%
\pgfpathlineto{\pgfqpoint{1.125140in}{0.739656in}}%
\pgfpathlineto{\pgfqpoint{1.124842in}{0.739656in}}%
\pgfpathlineto{\pgfqpoint{1.124545in}{0.739656in}}%
\pgfpathlineto{\pgfqpoint{1.124247in}{0.739656in}}%
\pgfpathlineto{\pgfqpoint{1.123950in}{0.739656in}}%
\pgfpathlineto{\pgfqpoint{1.123652in}{0.739656in}}%
\pgfpathlineto{\pgfqpoint{1.123355in}{0.739656in}}%
\pgfpathlineto{\pgfqpoint{1.123057in}{0.739656in}}%
\pgfpathlineto{\pgfqpoint{1.122760in}{0.739656in}}%
\pgfpathlineto{\pgfqpoint{1.122463in}{0.739656in}}%
\pgfpathlineto{\pgfqpoint{1.122165in}{0.739656in}}%
\pgfpathlineto{\pgfqpoint{1.121868in}{0.739656in}}%
\pgfpathlineto{\pgfqpoint{1.121570in}{0.739656in}}%
\pgfpathlineto{\pgfqpoint{1.121273in}{0.739656in}}%
\pgfpathlineto{\pgfqpoint{1.120975in}{0.739656in}}%
\pgfpathlineto{\pgfqpoint{1.120678in}{0.739656in}}%
\pgfpathlineto{\pgfqpoint{1.120380in}{0.739656in}}%
\pgfpathlineto{\pgfqpoint{1.120083in}{0.739656in}}%
\pgfpathlineto{\pgfqpoint{1.119785in}{0.739656in}}%
\pgfpathlineto{\pgfqpoint{1.119488in}{0.739656in}}%
\pgfpathlineto{\pgfqpoint{1.119190in}{0.739656in}}%
\pgfpathlineto{\pgfqpoint{1.118893in}{0.739656in}}%
\pgfpathlineto{\pgfqpoint{1.118595in}{0.739656in}}%
\pgfpathlineto{\pgfqpoint{1.118298in}{0.739656in}}%
\pgfpathlineto{\pgfqpoint{1.118000in}{0.739656in}}%
\pgfpathlineto{\pgfqpoint{1.117703in}{0.739656in}}%
\pgfpathlineto{\pgfqpoint{1.117405in}{0.739656in}}%
\pgfpathlineto{\pgfqpoint{1.117108in}{0.739656in}}%
\pgfpathlineto{\pgfqpoint{1.116810in}{0.739656in}}%
\pgfpathlineto{\pgfqpoint{1.116513in}{0.739656in}}%
\pgfpathlineto{\pgfqpoint{1.116215in}{0.739656in}}%
\pgfpathlineto{\pgfqpoint{1.115918in}{0.739656in}}%
\pgfpathlineto{\pgfqpoint{1.115621in}{0.739656in}}%
\pgfpathlineto{\pgfqpoint{1.115323in}{0.739656in}}%
\pgfpathlineto{\pgfqpoint{1.115026in}{0.739656in}}%
\pgfpathlineto{\pgfqpoint{1.114728in}{0.739656in}}%
\pgfpathlineto{\pgfqpoint{1.114431in}{0.739656in}}%
\pgfpathlineto{\pgfqpoint{1.114133in}{0.739656in}}%
\pgfpathlineto{\pgfqpoint{1.113836in}{0.739656in}}%
\pgfpathlineto{\pgfqpoint{1.113538in}{0.739656in}}%
\pgfpathlineto{\pgfqpoint{1.113241in}{0.739656in}}%
\pgfpathlineto{\pgfqpoint{1.112943in}{0.739656in}}%
\pgfpathlineto{\pgfqpoint{1.112646in}{0.739656in}}%
\pgfpathlineto{\pgfqpoint{1.112348in}{0.739656in}}%
\pgfpathlineto{\pgfqpoint{1.112051in}{0.739656in}}%
\pgfpathlineto{\pgfqpoint{1.111753in}{0.739656in}}%
\pgfpathlineto{\pgfqpoint{1.111456in}{0.739656in}}%
\pgfpathlineto{\pgfqpoint{1.111158in}{0.739656in}}%
\pgfpathlineto{\pgfqpoint{1.110861in}{0.739656in}}%
\pgfpathlineto{\pgfqpoint{1.110563in}{0.739656in}}%
\pgfpathlineto{\pgfqpoint{1.110266in}{0.739656in}}%
\pgfpathlineto{\pgfqpoint{1.109968in}{0.739656in}}%
\pgfpathlineto{\pgfqpoint{1.109671in}{0.739656in}}%
\pgfpathlineto{\pgfqpoint{1.109374in}{0.739656in}}%
\pgfpathlineto{\pgfqpoint{1.109076in}{0.739656in}}%
\pgfpathlineto{\pgfqpoint{1.108779in}{0.739656in}}%
\pgfpathlineto{\pgfqpoint{1.108481in}{0.739656in}}%
\pgfpathlineto{\pgfqpoint{1.108184in}{0.739656in}}%
\pgfpathlineto{\pgfqpoint{1.107886in}{0.739656in}}%
\pgfpathlineto{\pgfqpoint{1.107589in}{0.739656in}}%
\pgfpathlineto{\pgfqpoint{1.107291in}{0.739656in}}%
\pgfpathlineto{\pgfqpoint{1.106994in}{0.739656in}}%
\pgfpathlineto{\pgfqpoint{1.106696in}{0.739656in}}%
\pgfpathlineto{\pgfqpoint{1.106399in}{0.739656in}}%
\pgfpathlineto{\pgfqpoint{1.106101in}{0.739656in}}%
\pgfpathlineto{\pgfqpoint{1.105804in}{0.739656in}}%
\pgfpathlineto{\pgfqpoint{1.105506in}{0.739656in}}%
\pgfpathlineto{\pgfqpoint{1.105209in}{0.739656in}}%
\pgfpathlineto{\pgfqpoint{1.104911in}{0.739656in}}%
\pgfpathlineto{\pgfqpoint{1.104614in}{0.739656in}}%
\pgfpathlineto{\pgfqpoint{1.104316in}{0.739656in}}%
\pgfpathlineto{\pgfqpoint{1.104019in}{0.739656in}}%
\pgfpathlineto{\pgfqpoint{1.103721in}{0.739656in}}%
\pgfpathlineto{\pgfqpoint{1.103424in}{0.739656in}}%
\pgfpathlineto{\pgfqpoint{1.103126in}{0.739656in}}%
\pgfpathlineto{\pgfqpoint{1.102829in}{0.739656in}}%
\pgfpathlineto{\pgfqpoint{1.102532in}{0.739656in}}%
\pgfpathlineto{\pgfqpoint{1.102234in}{0.739656in}}%
\pgfpathlineto{\pgfqpoint{1.101937in}{0.739656in}}%
\pgfpathlineto{\pgfqpoint{1.101639in}{0.739656in}}%
\pgfpathlineto{\pgfqpoint{1.101342in}{0.739656in}}%
\pgfpathlineto{\pgfqpoint{1.101044in}{0.739656in}}%
\pgfpathlineto{\pgfqpoint{1.100747in}{0.739656in}}%
\pgfpathlineto{\pgfqpoint{1.100449in}{0.739656in}}%
\pgfpathlineto{\pgfqpoint{1.100152in}{0.739656in}}%
\pgfpathlineto{\pgfqpoint{1.099854in}{0.739656in}}%
\pgfpathlineto{\pgfqpoint{1.099557in}{0.739656in}}%
\pgfpathlineto{\pgfqpoint{1.099259in}{0.739656in}}%
\pgfpathlineto{\pgfqpoint{1.098962in}{0.739656in}}%
\pgfpathlineto{\pgfqpoint{1.098664in}{0.739656in}}%
\pgfpathlineto{\pgfqpoint{1.098367in}{0.739656in}}%
\pgfpathlineto{\pgfqpoint{1.098069in}{0.739656in}}%
\pgfpathlineto{\pgfqpoint{1.097772in}{0.739656in}}%
\pgfpathlineto{\pgfqpoint{1.097474in}{0.739656in}}%
\pgfpathlineto{\pgfqpoint{1.097177in}{0.739656in}}%
\pgfpathlineto{\pgfqpoint{1.096879in}{0.739656in}}%
\pgfpathlineto{\pgfqpoint{1.096582in}{0.739656in}}%
\pgfpathlineto{\pgfqpoint{1.096284in}{0.739656in}}%
\pgfpathlineto{\pgfqpoint{1.095987in}{0.739656in}}%
\pgfpathlineto{\pgfqpoint{1.095690in}{0.739656in}}%
\pgfpathlineto{\pgfqpoint{1.095392in}{0.739656in}}%
\pgfpathlineto{\pgfqpoint{1.095095in}{0.739656in}}%
\pgfpathlineto{\pgfqpoint{1.094797in}{0.739656in}}%
\pgfpathlineto{\pgfqpoint{1.094500in}{0.739656in}}%
\pgfpathlineto{\pgfqpoint{1.094202in}{0.739656in}}%
\pgfpathlineto{\pgfqpoint{1.093905in}{0.739656in}}%
\pgfpathlineto{\pgfqpoint{1.093607in}{0.739656in}}%
\pgfpathlineto{\pgfqpoint{1.093310in}{0.739656in}}%
\pgfpathlineto{\pgfqpoint{1.093012in}{0.739656in}}%
\pgfpathlineto{\pgfqpoint{1.092715in}{0.739656in}}%
\pgfpathlineto{\pgfqpoint{1.092417in}{0.739656in}}%
\pgfpathlineto{\pgfqpoint{1.092120in}{0.739656in}}%
\pgfpathlineto{\pgfqpoint{1.091822in}{0.739656in}}%
\pgfpathlineto{\pgfqpoint{1.091525in}{0.739656in}}%
\pgfpathlineto{\pgfqpoint{1.091227in}{0.739656in}}%
\pgfpathlineto{\pgfqpoint{1.090930in}{0.739656in}}%
\pgfpathlineto{\pgfqpoint{1.090632in}{0.739656in}}%
\pgfpathlineto{\pgfqpoint{1.090335in}{0.739656in}}%
\pgfpathlineto{\pgfqpoint{1.090037in}{0.739656in}}%
\pgfpathlineto{\pgfqpoint{1.089740in}{0.739656in}}%
\pgfpathlineto{\pgfqpoint{1.089443in}{0.739656in}}%
\pgfpathlineto{\pgfqpoint{1.089145in}{0.739656in}}%
\pgfpathlineto{\pgfqpoint{1.088848in}{0.739656in}}%
\pgfpathlineto{\pgfqpoint{1.088550in}{0.739656in}}%
\pgfpathlineto{\pgfqpoint{1.088253in}{0.739656in}}%
\pgfpathlineto{\pgfqpoint{1.087955in}{0.739656in}}%
\pgfpathlineto{\pgfqpoint{1.087658in}{0.739656in}}%
\pgfpathlineto{\pgfqpoint{1.087360in}{0.739656in}}%
\pgfpathlineto{\pgfqpoint{1.087063in}{0.739656in}}%
\pgfpathlineto{\pgfqpoint{1.086765in}{0.739656in}}%
\pgfpathlineto{\pgfqpoint{1.086468in}{0.739656in}}%
\pgfpathlineto{\pgfqpoint{1.086170in}{0.739656in}}%
\pgfpathlineto{\pgfqpoint{1.085873in}{0.739656in}}%
\pgfpathlineto{\pgfqpoint{1.085575in}{0.739656in}}%
\pgfpathlineto{\pgfqpoint{1.085278in}{0.739656in}}%
\pgfpathlineto{\pgfqpoint{1.084980in}{0.739656in}}%
\pgfpathlineto{\pgfqpoint{1.084683in}{0.739656in}}%
\pgfpathlineto{\pgfqpoint{1.084385in}{0.739656in}}%
\pgfpathlineto{\pgfqpoint{1.084088in}{0.739656in}}%
\pgfpathlineto{\pgfqpoint{1.083790in}{0.739656in}}%
\pgfpathlineto{\pgfqpoint{1.083493in}{0.739656in}}%
\pgfpathlineto{\pgfqpoint{1.083195in}{0.739656in}}%
\pgfpathlineto{\pgfqpoint{1.082898in}{0.739656in}}%
\pgfpathlineto{\pgfqpoint{1.082601in}{0.739656in}}%
\pgfpathlineto{\pgfqpoint{1.082303in}{0.739656in}}%
\pgfpathlineto{\pgfqpoint{1.082006in}{0.739656in}}%
\pgfpathlineto{\pgfqpoint{1.081708in}{0.739656in}}%
\pgfpathlineto{\pgfqpoint{1.081411in}{0.739656in}}%
\pgfpathlineto{\pgfqpoint{1.081113in}{0.739656in}}%
\pgfpathlineto{\pgfqpoint{1.080816in}{0.739656in}}%
\pgfpathlineto{\pgfqpoint{1.080518in}{0.739656in}}%
\pgfpathlineto{\pgfqpoint{1.080221in}{0.739656in}}%
\pgfpathlineto{\pgfqpoint{1.079923in}{0.739656in}}%
\pgfpathlineto{\pgfqpoint{1.079626in}{0.739656in}}%
\pgfpathlineto{\pgfqpoint{1.079328in}{0.739656in}}%
\pgfpathlineto{\pgfqpoint{1.079031in}{0.739656in}}%
\pgfpathlineto{\pgfqpoint{1.078733in}{0.739656in}}%
\pgfpathlineto{\pgfqpoint{1.078436in}{0.739656in}}%
\pgfpathlineto{\pgfqpoint{1.078138in}{0.739656in}}%
\pgfpathlineto{\pgfqpoint{1.077841in}{0.739656in}}%
\pgfpathlineto{\pgfqpoint{1.077543in}{0.739656in}}%
\pgfpathlineto{\pgfqpoint{1.077246in}{0.739656in}}%
\pgfpathlineto{\pgfqpoint{1.076948in}{0.739656in}}%
\pgfpathlineto{\pgfqpoint{1.076651in}{0.739656in}}%
\pgfpathlineto{\pgfqpoint{1.076353in}{0.739656in}}%
\pgfpathlineto{\pgfqpoint{1.076056in}{0.739656in}}%
\pgfpathlineto{\pgfqpoint{1.075759in}{0.739656in}}%
\pgfpathlineto{\pgfqpoint{1.075461in}{0.739656in}}%
\pgfpathlineto{\pgfqpoint{1.075164in}{0.739656in}}%
\pgfpathlineto{\pgfqpoint{1.074866in}{0.739656in}}%
\pgfpathlineto{\pgfqpoint{1.074569in}{0.739656in}}%
\pgfpathlineto{\pgfqpoint{1.074271in}{0.739656in}}%
\pgfpathlineto{\pgfqpoint{1.073974in}{0.739656in}}%
\pgfpathlineto{\pgfqpoint{1.073676in}{0.739656in}}%
\pgfpathlineto{\pgfqpoint{1.073379in}{0.739656in}}%
\pgfpathlineto{\pgfqpoint{1.073081in}{0.739656in}}%
\pgfpathlineto{\pgfqpoint{1.072784in}{0.739656in}}%
\pgfpathlineto{\pgfqpoint{1.072486in}{0.739656in}}%
\pgfpathlineto{\pgfqpoint{1.072189in}{0.739656in}}%
\pgfpathlineto{\pgfqpoint{1.071891in}{0.739656in}}%
\pgfpathlineto{\pgfqpoint{1.071594in}{0.739656in}}%
\pgfpathlineto{\pgfqpoint{1.071296in}{0.739656in}}%
\pgfpathlineto{\pgfqpoint{1.070999in}{0.739656in}}%
\pgfpathlineto{\pgfqpoint{1.070701in}{0.739656in}}%
\pgfpathlineto{\pgfqpoint{1.070404in}{0.739656in}}%
\pgfpathlineto{\pgfqpoint{1.070106in}{0.739656in}}%
\pgfpathlineto{\pgfqpoint{1.069809in}{0.739656in}}%
\pgfpathlineto{\pgfqpoint{1.069512in}{0.739656in}}%
\pgfpathlineto{\pgfqpoint{1.069214in}{0.739656in}}%
\pgfpathlineto{\pgfqpoint{1.068917in}{0.739656in}}%
\pgfpathlineto{\pgfqpoint{1.068619in}{0.739656in}}%
\pgfpathlineto{\pgfqpoint{1.068322in}{0.739656in}}%
\pgfpathlineto{\pgfqpoint{1.068024in}{0.739656in}}%
\pgfpathlineto{\pgfqpoint{1.067727in}{0.739656in}}%
\pgfpathlineto{\pgfqpoint{1.067429in}{0.739656in}}%
\pgfpathlineto{\pgfqpoint{1.067132in}{0.739656in}}%
\pgfpathlineto{\pgfqpoint{1.066834in}{0.739656in}}%
\pgfpathlineto{\pgfqpoint{1.066537in}{0.739656in}}%
\pgfpathlineto{\pgfqpoint{1.066239in}{0.739656in}}%
\pgfpathlineto{\pgfqpoint{1.065942in}{0.739656in}}%
\pgfpathlineto{\pgfqpoint{1.065644in}{0.739656in}}%
\pgfpathlineto{\pgfqpoint{1.065347in}{0.739656in}}%
\pgfpathlineto{\pgfqpoint{1.065049in}{0.739656in}}%
\pgfpathlineto{\pgfqpoint{1.064752in}{0.739656in}}%
\pgfpathlineto{\pgfqpoint{1.064454in}{0.739656in}}%
\pgfpathlineto{\pgfqpoint{1.064157in}{0.739656in}}%
\pgfpathlineto{\pgfqpoint{1.063859in}{0.739656in}}%
\pgfpathlineto{\pgfqpoint{1.063562in}{0.739656in}}%
\pgfpathlineto{\pgfqpoint{1.063264in}{0.739656in}}%
\pgfpathlineto{\pgfqpoint{1.062967in}{0.739656in}}%
\pgfpathlineto{\pgfqpoint{1.062670in}{0.739656in}}%
\pgfpathlineto{\pgfqpoint{1.062372in}{0.739656in}}%
\pgfpathlineto{\pgfqpoint{1.062075in}{0.739656in}}%
\pgfpathlineto{\pgfqpoint{1.061777in}{0.739656in}}%
\pgfpathlineto{\pgfqpoint{1.061480in}{0.739656in}}%
\pgfpathlineto{\pgfqpoint{1.061182in}{0.739656in}}%
\pgfpathlineto{\pgfqpoint{1.060885in}{0.739656in}}%
\pgfpathlineto{\pgfqpoint{1.060587in}{0.739656in}}%
\pgfpathlineto{\pgfqpoint{1.060290in}{0.739656in}}%
\pgfpathlineto{\pgfqpoint{1.059992in}{0.739656in}}%
\pgfpathlineto{\pgfqpoint{1.059695in}{0.739656in}}%
\pgfpathlineto{\pgfqpoint{1.059397in}{0.739656in}}%
\pgfpathlineto{\pgfqpoint{1.059100in}{0.739656in}}%
\pgfpathlineto{\pgfqpoint{1.058802in}{0.739656in}}%
\pgfpathlineto{\pgfqpoint{1.058505in}{0.739656in}}%
\pgfpathlineto{\pgfqpoint{1.058207in}{0.739656in}}%
\pgfpathlineto{\pgfqpoint{1.057910in}{0.739656in}}%
\pgfpathlineto{\pgfqpoint{1.057612in}{0.739656in}}%
\pgfpathlineto{\pgfqpoint{1.057315in}{0.739656in}}%
\pgfpathlineto{\pgfqpoint{1.057017in}{0.739656in}}%
\pgfpathlineto{\pgfqpoint{1.056720in}{0.739656in}}%
\pgfpathlineto{\pgfqpoint{1.056422in}{0.739656in}}%
\pgfpathlineto{\pgfqpoint{1.056125in}{0.739656in}}%
\pgfpathlineto{\pgfqpoint{1.055828in}{0.739656in}}%
\pgfpathlineto{\pgfqpoint{1.055530in}{0.739656in}}%
\pgfpathlineto{\pgfqpoint{1.055233in}{0.739656in}}%
\pgfpathlineto{\pgfqpoint{1.054935in}{0.739656in}}%
\pgfpathlineto{\pgfqpoint{1.054638in}{0.739656in}}%
\pgfpathlineto{\pgfqpoint{1.054340in}{0.739656in}}%
\pgfpathlineto{\pgfqpoint{1.054043in}{0.739656in}}%
\pgfpathlineto{\pgfqpoint{1.053745in}{0.739656in}}%
\pgfpathlineto{\pgfqpoint{1.053448in}{0.739656in}}%
\pgfpathlineto{\pgfqpoint{1.053150in}{0.739656in}}%
\pgfpathlineto{\pgfqpoint{1.052853in}{0.739656in}}%
\pgfpathlineto{\pgfqpoint{1.052555in}{0.739656in}}%
\pgfpathlineto{\pgfqpoint{1.052258in}{0.739656in}}%
\pgfpathlineto{\pgfqpoint{1.051960in}{0.739656in}}%
\pgfpathlineto{\pgfqpoint{1.051663in}{0.739656in}}%
\pgfpathlineto{\pgfqpoint{1.051365in}{0.739656in}}%
\pgfpathlineto{\pgfqpoint{1.051068in}{0.739656in}}%
\pgfpathlineto{\pgfqpoint{1.050770in}{0.739656in}}%
\pgfpathlineto{\pgfqpoint{1.050473in}{0.739656in}}%
\pgfpathlineto{\pgfqpoint{1.050175in}{0.739656in}}%
\pgfpathlineto{\pgfqpoint{1.049878in}{0.739656in}}%
\pgfpathlineto{\pgfqpoint{1.049581in}{0.739656in}}%
\pgfpathlineto{\pgfqpoint{1.049283in}{0.739656in}}%
\pgfpathlineto{\pgfqpoint{1.048986in}{0.739656in}}%
\pgfpathlineto{\pgfqpoint{1.048688in}{0.739656in}}%
\pgfpathlineto{\pgfqpoint{1.048391in}{0.739656in}}%
\pgfpathlineto{\pgfqpoint{1.048093in}{0.739656in}}%
\pgfpathlineto{\pgfqpoint{1.047796in}{0.739656in}}%
\pgfpathlineto{\pgfqpoint{1.047498in}{0.739656in}}%
\pgfpathlineto{\pgfqpoint{1.047201in}{0.739656in}}%
\pgfpathlineto{\pgfqpoint{1.046903in}{0.739656in}}%
\pgfpathlineto{\pgfqpoint{1.046606in}{0.739656in}}%
\pgfpathlineto{\pgfqpoint{1.046308in}{0.739656in}}%
\pgfpathlineto{\pgfqpoint{1.046011in}{0.739656in}}%
\pgfpathlineto{\pgfqpoint{1.045713in}{0.739656in}}%
\pgfpathlineto{\pgfqpoint{1.045416in}{0.739656in}}%
\pgfpathlineto{\pgfqpoint{1.045118in}{0.739656in}}%
\pgfpathlineto{\pgfqpoint{1.044821in}{0.739656in}}%
\pgfpathlineto{\pgfqpoint{1.044523in}{0.739656in}}%
\pgfpathlineto{\pgfqpoint{1.044226in}{0.739656in}}%
\pgfpathlineto{\pgfqpoint{1.043928in}{0.739656in}}%
\pgfpathlineto{\pgfqpoint{1.043631in}{0.739656in}}%
\pgfpathlineto{\pgfqpoint{1.043333in}{0.739656in}}%
\pgfpathlineto{\pgfqpoint{1.043036in}{0.739656in}}%
\pgfpathlineto{\pgfqpoint{1.042739in}{0.739656in}}%
\pgfpathlineto{\pgfqpoint{1.042441in}{0.739656in}}%
\pgfpathlineto{\pgfqpoint{1.042144in}{0.739656in}}%
\pgfpathlineto{\pgfqpoint{1.041846in}{0.739656in}}%
\pgfpathlineto{\pgfqpoint{1.041549in}{0.739656in}}%
\pgfpathlineto{\pgfqpoint{1.041251in}{0.739656in}}%
\pgfpathlineto{\pgfqpoint{1.040954in}{0.739656in}}%
\pgfpathlineto{\pgfqpoint{1.040656in}{0.739656in}}%
\pgfpathlineto{\pgfqpoint{1.040359in}{0.739656in}}%
\pgfpathlineto{\pgfqpoint{1.040061in}{0.739656in}}%
\pgfpathlineto{\pgfqpoint{1.039764in}{0.739656in}}%
\pgfpathlineto{\pgfqpoint{1.039466in}{0.739656in}}%
\pgfpathlineto{\pgfqpoint{1.039169in}{0.739656in}}%
\pgfpathlineto{\pgfqpoint{1.038871in}{0.739656in}}%
\pgfpathlineto{\pgfqpoint{1.038574in}{0.739656in}}%
\pgfpathlineto{\pgfqpoint{1.038276in}{0.739656in}}%
\pgfpathlineto{\pgfqpoint{1.037979in}{0.739656in}}%
\pgfpathlineto{\pgfqpoint{1.037681in}{0.739656in}}%
\pgfpathlineto{\pgfqpoint{1.037384in}{0.739656in}}%
\pgfpathlineto{\pgfqpoint{1.037086in}{0.739656in}}%
\pgfpathlineto{\pgfqpoint{1.036789in}{0.739656in}}%
\pgfpathlineto{\pgfqpoint{1.036491in}{0.739656in}}%
\pgfpathlineto{\pgfqpoint{1.036194in}{0.739656in}}%
\pgfpathlineto{\pgfqpoint{1.035897in}{0.739656in}}%
\pgfpathlineto{\pgfqpoint{1.035599in}{0.739656in}}%
\pgfpathlineto{\pgfqpoint{1.035302in}{0.739656in}}%
\pgfpathlineto{\pgfqpoint{1.035004in}{0.739656in}}%
\pgfpathlineto{\pgfqpoint{1.034707in}{0.739656in}}%
\pgfpathlineto{\pgfqpoint{1.034409in}{0.739656in}}%
\pgfpathlineto{\pgfqpoint{1.034112in}{0.739656in}}%
\pgfpathlineto{\pgfqpoint{1.033814in}{0.739656in}}%
\pgfpathlineto{\pgfqpoint{1.033517in}{0.739656in}}%
\pgfpathlineto{\pgfqpoint{1.033219in}{0.739656in}}%
\pgfpathlineto{\pgfqpoint{1.032922in}{0.739656in}}%
\pgfpathlineto{\pgfqpoint{1.032624in}{0.739656in}}%
\pgfpathlineto{\pgfqpoint{1.032327in}{0.739656in}}%
\pgfpathlineto{\pgfqpoint{1.032029in}{0.739656in}}%
\pgfpathlineto{\pgfqpoint{1.031732in}{0.739656in}}%
\pgfpathlineto{\pgfqpoint{1.031434in}{0.739656in}}%
\pgfpathlineto{\pgfqpoint{1.031137in}{0.739656in}}%
\pgfpathlineto{\pgfqpoint{1.030839in}{0.739656in}}%
\pgfpathlineto{\pgfqpoint{1.030542in}{0.739656in}}%
\pgfpathlineto{\pgfqpoint{1.030244in}{0.739656in}}%
\pgfpathlineto{\pgfqpoint{1.029947in}{0.739656in}}%
\pgfpathlineto{\pgfqpoint{1.029650in}{0.739656in}}%
\pgfpathlineto{\pgfqpoint{1.029352in}{0.739656in}}%
\pgfpathlineto{\pgfqpoint{1.029055in}{0.739656in}}%
\pgfpathlineto{\pgfqpoint{1.028757in}{0.739656in}}%
\pgfpathlineto{\pgfqpoint{1.028460in}{0.739656in}}%
\pgfpathlineto{\pgfqpoint{1.028162in}{0.739656in}}%
\pgfpathlineto{\pgfqpoint{1.027865in}{0.739656in}}%
\pgfpathlineto{\pgfqpoint{1.027567in}{0.739656in}}%
\pgfpathlineto{\pgfqpoint{1.027270in}{0.739656in}}%
\pgfpathlineto{\pgfqpoint{1.026972in}{0.739656in}}%
\pgfpathlineto{\pgfqpoint{1.026675in}{0.739656in}}%
\pgfpathlineto{\pgfqpoint{1.026377in}{0.739656in}}%
\pgfpathlineto{\pgfqpoint{1.026080in}{0.739656in}}%
\pgfpathlineto{\pgfqpoint{1.025782in}{0.739656in}}%
\pgfpathlineto{\pgfqpoint{1.025485in}{0.739656in}}%
\pgfpathlineto{\pgfqpoint{1.025187in}{0.739656in}}%
\pgfpathlineto{\pgfqpoint{1.024890in}{0.739656in}}%
\pgfpathlineto{\pgfqpoint{1.024592in}{0.739656in}}%
\pgfpathlineto{\pgfqpoint{1.024295in}{0.739656in}}%
\pgfpathlineto{\pgfqpoint{1.023997in}{0.739656in}}%
\pgfpathlineto{\pgfqpoint{1.023700in}{0.739656in}}%
\pgfpathlineto{\pgfqpoint{1.023402in}{0.739656in}}%
\pgfpathlineto{\pgfqpoint{1.023105in}{0.739656in}}%
\pgfpathlineto{\pgfqpoint{1.022808in}{0.739656in}}%
\pgfpathlineto{\pgfqpoint{1.022510in}{0.739656in}}%
\pgfpathlineto{\pgfqpoint{1.022213in}{0.739656in}}%
\pgfpathlineto{\pgfqpoint{1.021915in}{0.739656in}}%
\pgfpathlineto{\pgfqpoint{1.021618in}{0.739656in}}%
\pgfpathlineto{\pgfqpoint{1.021320in}{0.739656in}}%
\pgfpathlineto{\pgfqpoint{1.021023in}{0.739656in}}%
\pgfpathlineto{\pgfqpoint{1.020725in}{0.739656in}}%
\pgfpathlineto{\pgfqpoint{1.020428in}{0.739656in}}%
\pgfpathlineto{\pgfqpoint{1.020130in}{0.739656in}}%
\pgfpathlineto{\pgfqpoint{1.019833in}{0.739656in}}%
\pgfpathlineto{\pgfqpoint{1.019535in}{0.739656in}}%
\pgfpathlineto{\pgfqpoint{1.019238in}{0.739656in}}%
\pgfpathlineto{\pgfqpoint{1.018940in}{0.739656in}}%
\pgfpathlineto{\pgfqpoint{1.018643in}{0.739656in}}%
\pgfpathlineto{\pgfqpoint{1.018345in}{0.739656in}}%
\pgfpathlineto{\pgfqpoint{1.018048in}{0.739656in}}%
\pgfpathlineto{\pgfqpoint{1.017750in}{0.739656in}}%
\pgfpathlineto{\pgfqpoint{1.017453in}{0.739656in}}%
\pgfpathlineto{\pgfqpoint{1.017155in}{0.739656in}}%
\pgfpathlineto{\pgfqpoint{1.016858in}{0.739656in}}%
\pgfpathlineto{\pgfqpoint{1.016560in}{0.739656in}}%
\pgfpathlineto{\pgfqpoint{1.016263in}{0.739656in}}%
\pgfpathlineto{\pgfqpoint{1.015966in}{0.739656in}}%
\pgfpathlineto{\pgfqpoint{1.015668in}{0.739656in}}%
\pgfpathlineto{\pgfqpoint{1.015371in}{0.739656in}}%
\pgfpathlineto{\pgfqpoint{1.015073in}{0.739656in}}%
\pgfpathlineto{\pgfqpoint{1.014776in}{0.739656in}}%
\pgfpathlineto{\pgfqpoint{1.014478in}{0.739656in}}%
\pgfpathlineto{\pgfqpoint{1.014181in}{0.739656in}}%
\pgfpathlineto{\pgfqpoint{1.013883in}{0.739656in}}%
\pgfpathlineto{\pgfqpoint{1.013586in}{0.739656in}}%
\pgfpathlineto{\pgfqpoint{1.013288in}{0.739656in}}%
\pgfpathlineto{\pgfqpoint{1.012991in}{0.739656in}}%
\pgfpathlineto{\pgfqpoint{1.012693in}{0.739656in}}%
\pgfpathlineto{\pgfqpoint{1.012396in}{0.739656in}}%
\pgfpathlineto{\pgfqpoint{1.012098in}{0.739656in}}%
\pgfpathlineto{\pgfqpoint{1.011801in}{0.739656in}}%
\pgfpathlineto{\pgfqpoint{1.011503in}{0.739656in}}%
\pgfpathlineto{\pgfqpoint{1.011206in}{0.739656in}}%
\pgfpathlineto{\pgfqpoint{1.010908in}{0.739656in}}%
\pgfpathlineto{\pgfqpoint{1.010611in}{0.739656in}}%
\pgfpathlineto{\pgfqpoint{1.010313in}{0.739656in}}%
\pgfpathlineto{\pgfqpoint{1.010016in}{0.739656in}}%
\pgfpathlineto{\pgfqpoint{1.009719in}{0.739656in}}%
\pgfpathlineto{\pgfqpoint{1.009421in}{0.739656in}}%
\pgfpathlineto{\pgfqpoint{1.009124in}{0.739656in}}%
\pgfpathlineto{\pgfqpoint{1.008826in}{0.739656in}}%
\pgfpathlineto{\pgfqpoint{1.008529in}{0.739656in}}%
\pgfpathlineto{\pgfqpoint{1.008231in}{0.739656in}}%
\pgfpathlineto{\pgfqpoint{1.007934in}{0.739656in}}%
\pgfpathlineto{\pgfqpoint{1.007636in}{0.739656in}}%
\pgfpathlineto{\pgfqpoint{1.007339in}{0.739656in}}%
\pgfpathlineto{\pgfqpoint{1.007041in}{0.739656in}}%
\pgfpathlineto{\pgfqpoint{1.006744in}{0.739656in}}%
\pgfpathlineto{\pgfqpoint{1.006446in}{0.739656in}}%
\pgfpathlineto{\pgfqpoint{1.006149in}{0.739656in}}%
\pgfpathlineto{\pgfqpoint{1.005851in}{0.739656in}}%
\pgfpathlineto{\pgfqpoint{1.005554in}{0.739656in}}%
\pgfpathlineto{\pgfqpoint{1.005256in}{0.739656in}}%
\pgfpathlineto{\pgfqpoint{1.004959in}{0.739656in}}%
\pgfpathlineto{\pgfqpoint{1.004661in}{0.739656in}}%
\pgfpathlineto{\pgfqpoint{1.004364in}{0.739656in}}%
\pgfpathlineto{\pgfqpoint{1.004066in}{0.739656in}}%
\pgfpathlineto{\pgfqpoint{1.003769in}{0.739656in}}%
\pgfpathlineto{\pgfqpoint{1.003471in}{0.739656in}}%
\pgfpathlineto{\pgfqpoint{1.003174in}{0.739656in}}%
\pgfpathlineto{\pgfqpoint{1.002877in}{0.739656in}}%
\pgfpathlineto{\pgfqpoint{1.002579in}{0.739656in}}%
\pgfpathlineto{\pgfqpoint{1.002282in}{0.739656in}}%
\pgfpathlineto{\pgfqpoint{1.001984in}{0.739656in}}%
\pgfpathlineto{\pgfqpoint{1.001687in}{0.739656in}}%
\pgfpathlineto{\pgfqpoint{1.001389in}{0.739656in}}%
\pgfpathlineto{\pgfqpoint{1.001092in}{0.739656in}}%
\pgfpathlineto{\pgfqpoint{1.000794in}{0.739656in}}%
\pgfpathlineto{\pgfqpoint{1.000497in}{0.739656in}}%
\pgfpathlineto{\pgfqpoint{1.000199in}{0.739656in}}%
\pgfpathlineto{\pgfqpoint{0.999902in}{0.739656in}}%
\pgfpathlineto{\pgfqpoint{0.999604in}{0.739656in}}%
\pgfpathlineto{\pgfqpoint{0.999307in}{0.739656in}}%
\pgfpathlineto{\pgfqpoint{0.999009in}{0.739656in}}%
\pgfpathlineto{\pgfqpoint{0.998712in}{0.739656in}}%
\pgfpathlineto{\pgfqpoint{0.998414in}{0.739656in}}%
\pgfpathlineto{\pgfqpoint{0.998117in}{0.739656in}}%
\pgfpathlineto{\pgfqpoint{0.997819in}{0.739656in}}%
\pgfpathlineto{\pgfqpoint{0.997522in}{0.739656in}}%
\pgfpathlineto{\pgfqpoint{0.997224in}{0.739656in}}%
\pgfpathlineto{\pgfqpoint{0.996927in}{0.739656in}}%
\pgfpathlineto{\pgfqpoint{0.996629in}{0.739656in}}%
\pgfpathlineto{\pgfqpoint{0.996332in}{0.739656in}}%
\pgfpathlineto{\pgfqpoint{0.996035in}{0.739656in}}%
\pgfpathlineto{\pgfqpoint{0.995737in}{0.739656in}}%
\pgfpathlineto{\pgfqpoint{0.995440in}{0.739656in}}%
\pgfpathlineto{\pgfqpoint{0.995142in}{0.739656in}}%
\pgfpathlineto{\pgfqpoint{0.994845in}{0.739656in}}%
\pgfpathlineto{\pgfqpoint{0.994547in}{0.739656in}}%
\pgfpathlineto{\pgfqpoint{0.994250in}{0.739656in}}%
\pgfpathlineto{\pgfqpoint{0.993952in}{0.739656in}}%
\pgfpathlineto{\pgfqpoint{0.993655in}{0.739656in}}%
\pgfpathlineto{\pgfqpoint{0.993357in}{0.739656in}}%
\pgfpathlineto{\pgfqpoint{0.993060in}{0.739656in}}%
\pgfpathlineto{\pgfqpoint{0.992762in}{0.739656in}}%
\pgfpathlineto{\pgfqpoint{0.992465in}{0.739656in}}%
\pgfpathlineto{\pgfqpoint{0.992167in}{0.739656in}}%
\pgfpathlineto{\pgfqpoint{0.991870in}{0.739656in}}%
\pgfpathlineto{\pgfqpoint{0.991572in}{0.739656in}}%
\pgfpathlineto{\pgfqpoint{0.991275in}{0.739656in}}%
\pgfpathlineto{\pgfqpoint{0.990977in}{0.739656in}}%
\pgfpathlineto{\pgfqpoint{0.990680in}{0.739656in}}%
\pgfpathlineto{\pgfqpoint{0.990382in}{0.739656in}}%
\pgfpathlineto{\pgfqpoint{0.990085in}{0.739656in}}%
\pgfpathlineto{\pgfqpoint{0.989788in}{0.739656in}}%
\pgfpathlineto{\pgfqpoint{0.989490in}{0.739656in}}%
\pgfpathlineto{\pgfqpoint{0.989193in}{0.739656in}}%
\pgfpathlineto{\pgfqpoint{0.988895in}{0.739656in}}%
\pgfpathlineto{\pgfqpoint{0.988598in}{0.739656in}}%
\pgfpathlineto{\pgfqpoint{0.988300in}{0.739656in}}%
\pgfpathlineto{\pgfqpoint{0.988003in}{0.739656in}}%
\pgfpathlineto{\pgfqpoint{0.987705in}{0.739656in}}%
\pgfpathlineto{\pgfqpoint{0.987408in}{0.739656in}}%
\pgfpathlineto{\pgfqpoint{0.987110in}{0.739656in}}%
\pgfpathlineto{\pgfqpoint{0.986813in}{0.739656in}}%
\pgfpathlineto{\pgfqpoint{0.986515in}{0.739656in}}%
\pgfpathlineto{\pgfqpoint{0.986218in}{0.739656in}}%
\pgfpathlineto{\pgfqpoint{0.985920in}{0.739656in}}%
\pgfpathlineto{\pgfqpoint{0.985623in}{0.739656in}}%
\pgfpathlineto{\pgfqpoint{0.985325in}{0.739656in}}%
\pgfpathlineto{\pgfqpoint{0.985028in}{0.739656in}}%
\pgfpathlineto{\pgfqpoint{0.984730in}{0.739656in}}%
\pgfpathlineto{\pgfqpoint{0.984433in}{0.739656in}}%
\pgfpathlineto{\pgfqpoint{0.984135in}{0.739656in}}%
\pgfpathlineto{\pgfqpoint{0.983838in}{0.739656in}}%
\pgfpathlineto{\pgfqpoint{0.983540in}{0.739656in}}%
\pgfpathlineto{\pgfqpoint{0.983243in}{0.739656in}}%
\pgfpathlineto{\pgfqpoint{0.982946in}{0.739656in}}%
\pgfpathlineto{\pgfqpoint{0.982648in}{0.739656in}}%
\pgfpathlineto{\pgfqpoint{0.982351in}{0.739656in}}%
\pgfpathlineto{\pgfqpoint{0.982053in}{0.739656in}}%
\pgfpathlineto{\pgfqpoint{0.981756in}{0.739656in}}%
\pgfpathlineto{\pgfqpoint{0.981458in}{0.739656in}}%
\pgfpathlineto{\pgfqpoint{0.981161in}{0.739656in}}%
\pgfpathlineto{\pgfqpoint{0.980863in}{0.739656in}}%
\pgfpathlineto{\pgfqpoint{0.980566in}{0.739656in}}%
\pgfpathlineto{\pgfqpoint{0.980268in}{0.739656in}}%
\pgfpathlineto{\pgfqpoint{0.979971in}{0.739656in}}%
\pgfpathlineto{\pgfqpoint{0.979673in}{0.739656in}}%
\pgfpathlineto{\pgfqpoint{0.979376in}{0.739656in}}%
\pgfpathlineto{\pgfqpoint{0.979078in}{0.739656in}}%
\pgfpathlineto{\pgfqpoint{0.978781in}{0.739656in}}%
\pgfpathlineto{\pgfqpoint{0.978483in}{0.739656in}}%
\pgfpathlineto{\pgfqpoint{0.978186in}{0.739656in}}%
\pgfpathlineto{\pgfqpoint{0.977888in}{0.739656in}}%
\pgfpathlineto{\pgfqpoint{0.977591in}{0.739656in}}%
\pgfpathlineto{\pgfqpoint{0.977293in}{0.739656in}}%
\pgfpathlineto{\pgfqpoint{0.976996in}{0.739656in}}%
\pgfpathlineto{\pgfqpoint{0.976698in}{0.739656in}}%
\pgfpathlineto{\pgfqpoint{0.976401in}{0.739656in}}%
\pgfpathlineto{\pgfqpoint{0.976104in}{0.739656in}}%
\pgfpathlineto{\pgfqpoint{0.975806in}{0.739656in}}%
\pgfpathlineto{\pgfqpoint{0.975509in}{0.739656in}}%
\pgfpathlineto{\pgfqpoint{0.975211in}{0.739656in}}%
\pgfpathlineto{\pgfqpoint{0.974914in}{0.739656in}}%
\pgfpathlineto{\pgfqpoint{0.974616in}{0.739656in}}%
\pgfpathlineto{\pgfqpoint{0.974319in}{0.739656in}}%
\pgfpathlineto{\pgfqpoint{0.974021in}{0.739656in}}%
\pgfpathlineto{\pgfqpoint{0.973724in}{0.739656in}}%
\pgfpathlineto{\pgfqpoint{0.973426in}{0.739656in}}%
\pgfpathlineto{\pgfqpoint{0.973129in}{0.739656in}}%
\pgfpathlineto{\pgfqpoint{0.972831in}{0.739656in}}%
\pgfpathlineto{\pgfqpoint{0.972534in}{0.739656in}}%
\pgfpathlineto{\pgfqpoint{0.972236in}{0.739656in}}%
\pgfpathlineto{\pgfqpoint{0.971939in}{0.739656in}}%
\pgfpathlineto{\pgfqpoint{0.971641in}{0.739656in}}%
\pgfpathlineto{\pgfqpoint{0.971344in}{0.739656in}}%
\pgfpathlineto{\pgfqpoint{0.971046in}{0.739656in}}%
\pgfpathlineto{\pgfqpoint{0.970749in}{0.739656in}}%
\pgfpathlineto{\pgfqpoint{0.970451in}{0.739656in}}%
\pgfpathlineto{\pgfqpoint{0.970154in}{0.739656in}}%
\pgfpathlineto{\pgfqpoint{0.969857in}{0.739656in}}%
\pgfpathlineto{\pgfqpoint{0.969559in}{0.739656in}}%
\pgfpathlineto{\pgfqpoint{0.969262in}{0.739656in}}%
\pgfpathlineto{\pgfqpoint{0.968964in}{0.739656in}}%
\pgfpathlineto{\pgfqpoint{0.968667in}{0.739656in}}%
\pgfpathlineto{\pgfqpoint{0.968369in}{0.739656in}}%
\pgfpathlineto{\pgfqpoint{0.968072in}{0.739656in}}%
\pgfpathlineto{\pgfqpoint{0.967774in}{0.739656in}}%
\pgfpathlineto{\pgfqpoint{0.967477in}{0.739656in}}%
\pgfpathlineto{\pgfqpoint{0.967179in}{0.739656in}}%
\pgfpathlineto{\pgfqpoint{0.966882in}{0.739656in}}%
\pgfpathlineto{\pgfqpoint{0.966584in}{0.739656in}}%
\pgfpathlineto{\pgfqpoint{0.966287in}{0.739656in}}%
\pgfpathlineto{\pgfqpoint{0.965989in}{0.739656in}}%
\pgfpathlineto{\pgfqpoint{0.965692in}{0.739656in}}%
\pgfpathlineto{\pgfqpoint{0.965394in}{0.739656in}}%
\pgfpathlineto{\pgfqpoint{0.965097in}{0.739656in}}%
\pgfpathlineto{\pgfqpoint{0.964799in}{0.739656in}}%
\pgfpathlineto{\pgfqpoint{0.964502in}{0.739656in}}%
\pgfpathlineto{\pgfqpoint{0.964204in}{0.739656in}}%
\pgfpathlineto{\pgfqpoint{0.963907in}{0.739656in}}%
\pgfpathlineto{\pgfqpoint{0.963609in}{0.739656in}}%
\pgfpathlineto{\pgfqpoint{0.963312in}{0.739656in}}%
\pgfpathlineto{\pgfqpoint{0.963015in}{0.739656in}}%
\pgfpathlineto{\pgfqpoint{0.962717in}{0.739656in}}%
\pgfpathlineto{\pgfqpoint{0.962420in}{0.739656in}}%
\pgfpathlineto{\pgfqpoint{0.962122in}{0.739656in}}%
\pgfpathlineto{\pgfqpoint{0.961825in}{0.739656in}}%
\pgfpathlineto{\pgfqpoint{0.961527in}{0.739656in}}%
\pgfpathlineto{\pgfqpoint{0.961230in}{0.739656in}}%
\pgfpathlineto{\pgfqpoint{0.960932in}{0.739656in}}%
\pgfpathlineto{\pgfqpoint{0.960635in}{0.739656in}}%
\pgfpathlineto{\pgfqpoint{0.960337in}{0.739656in}}%
\pgfpathlineto{\pgfqpoint{0.960040in}{0.739656in}}%
\pgfpathlineto{\pgfqpoint{0.959742in}{0.739656in}}%
\pgfpathlineto{\pgfqpoint{0.959445in}{0.739656in}}%
\pgfpathlineto{\pgfqpoint{0.959147in}{0.739656in}}%
\pgfpathlineto{\pgfqpoint{0.958850in}{0.739656in}}%
\pgfpathlineto{\pgfqpoint{0.958552in}{0.739656in}}%
\pgfpathlineto{\pgfqpoint{0.958255in}{0.739656in}}%
\pgfpathlineto{\pgfqpoint{0.957957in}{0.739656in}}%
\pgfpathlineto{\pgfqpoint{0.957660in}{0.739656in}}%
\pgfpathlineto{\pgfqpoint{0.957362in}{0.739656in}}%
\pgfpathlineto{\pgfqpoint{0.957065in}{0.739656in}}%
\pgfpathlineto{\pgfqpoint{0.956767in}{0.739656in}}%
\pgfpathlineto{\pgfqpoint{0.956470in}{0.739656in}}%
\pgfpathlineto{\pgfqpoint{0.956173in}{0.739656in}}%
\pgfpathlineto{\pgfqpoint{0.955875in}{0.739656in}}%
\pgfpathlineto{\pgfqpoint{0.955578in}{0.739656in}}%
\pgfpathlineto{\pgfqpoint{0.955280in}{0.739656in}}%
\pgfpathlineto{\pgfqpoint{0.954983in}{0.739656in}}%
\pgfpathlineto{\pgfqpoint{0.954685in}{0.739656in}}%
\pgfpathlineto{\pgfqpoint{0.954388in}{0.739656in}}%
\pgfpathlineto{\pgfqpoint{0.954090in}{0.739656in}}%
\pgfpathlineto{\pgfqpoint{0.953793in}{0.739656in}}%
\pgfpathlineto{\pgfqpoint{0.953495in}{0.739656in}}%
\pgfpathlineto{\pgfqpoint{0.953198in}{0.739656in}}%
\pgfpathlineto{\pgfqpoint{0.952900in}{0.739656in}}%
\pgfpathlineto{\pgfqpoint{0.952603in}{0.739656in}}%
\pgfpathlineto{\pgfqpoint{0.952305in}{0.739656in}}%
\pgfpathlineto{\pgfqpoint{0.952008in}{0.739656in}}%
\pgfpathlineto{\pgfqpoint{0.951710in}{0.739656in}}%
\pgfpathlineto{\pgfqpoint{0.951413in}{0.739656in}}%
\pgfpathlineto{\pgfqpoint{0.951115in}{0.739656in}}%
\pgfpathlineto{\pgfqpoint{0.950818in}{0.739656in}}%
\pgfpathlineto{\pgfqpoint{0.950520in}{0.739656in}}%
\pgfpathlineto{\pgfqpoint{0.950223in}{0.739656in}}%
\pgfpathlineto{\pgfqpoint{0.949926in}{0.739656in}}%
\pgfpathlineto{\pgfqpoint{0.949628in}{0.739656in}}%
\pgfpathlineto{\pgfqpoint{0.949331in}{0.739656in}}%
\pgfpathlineto{\pgfqpoint{0.949033in}{0.739656in}}%
\pgfpathlineto{\pgfqpoint{0.948736in}{0.739656in}}%
\pgfpathlineto{\pgfqpoint{0.948438in}{0.739656in}}%
\pgfpathlineto{\pgfqpoint{0.948141in}{0.739656in}}%
\pgfpathlineto{\pgfqpoint{0.947843in}{0.739656in}}%
\pgfpathlineto{\pgfqpoint{0.947546in}{0.739656in}}%
\pgfpathlineto{\pgfqpoint{0.947248in}{0.739656in}}%
\pgfpathlineto{\pgfqpoint{0.946951in}{0.739656in}}%
\pgfpathlineto{\pgfqpoint{0.946653in}{0.739656in}}%
\pgfpathlineto{\pgfqpoint{0.946356in}{0.739656in}}%
\pgfpathlineto{\pgfqpoint{0.946058in}{0.739656in}}%
\pgfpathlineto{\pgfqpoint{0.945761in}{0.739656in}}%
\pgfpathlineto{\pgfqpoint{0.945463in}{0.739656in}}%
\pgfpathlineto{\pgfqpoint{0.945166in}{0.739656in}}%
\pgfpathlineto{\pgfqpoint{0.944868in}{0.739656in}}%
\pgfpathlineto{\pgfqpoint{0.944571in}{0.739656in}}%
\pgfpathlineto{\pgfqpoint{0.944273in}{0.739656in}}%
\pgfpathlineto{\pgfqpoint{0.943976in}{0.739656in}}%
\pgfpathlineto{\pgfqpoint{0.943678in}{0.739656in}}%
\pgfpathlineto{\pgfqpoint{0.943381in}{0.739656in}}%
\pgfpathlineto{\pgfqpoint{0.943084in}{0.739656in}}%
\pgfpathlineto{\pgfqpoint{0.942786in}{0.739656in}}%
\pgfpathlineto{\pgfqpoint{0.942489in}{0.739656in}}%
\pgfpathlineto{\pgfqpoint{0.942191in}{0.739656in}}%
\pgfpathlineto{\pgfqpoint{0.941894in}{0.739656in}}%
\pgfpathlineto{\pgfqpoint{0.941596in}{0.739656in}}%
\pgfpathlineto{\pgfqpoint{0.941299in}{0.739656in}}%
\pgfpathlineto{\pgfqpoint{0.941001in}{0.739656in}}%
\pgfpathlineto{\pgfqpoint{0.940704in}{0.739656in}}%
\pgfpathlineto{\pgfqpoint{0.940406in}{0.739656in}}%
\pgfpathlineto{\pgfqpoint{0.940109in}{0.739656in}}%
\pgfpathlineto{\pgfqpoint{0.939811in}{0.739656in}}%
\pgfpathlineto{\pgfqpoint{0.939514in}{0.739656in}}%
\pgfpathlineto{\pgfqpoint{0.939216in}{0.739656in}}%
\pgfpathlineto{\pgfqpoint{0.938919in}{0.739656in}}%
\pgfpathlineto{\pgfqpoint{0.938621in}{0.739656in}}%
\pgfpathlineto{\pgfqpoint{0.938324in}{0.739656in}}%
\pgfpathlineto{\pgfqpoint{0.938026in}{0.739656in}}%
\pgfpathlineto{\pgfqpoint{0.937729in}{0.739656in}}%
\pgfpathlineto{\pgfqpoint{0.937431in}{0.739656in}}%
\pgfpathlineto{\pgfqpoint{0.937134in}{0.739656in}}%
\pgfpathlineto{\pgfqpoint{0.936836in}{0.739656in}}%
\pgfpathlineto{\pgfqpoint{0.936539in}{0.739656in}}%
\pgfpathlineto{\pgfqpoint{0.936242in}{0.739656in}}%
\pgfpathlineto{\pgfqpoint{0.935944in}{0.739656in}}%
\pgfpathlineto{\pgfqpoint{0.935647in}{0.739656in}}%
\pgfpathlineto{\pgfqpoint{0.935349in}{0.739656in}}%
\pgfpathlineto{\pgfqpoint{0.935052in}{0.739656in}}%
\pgfpathlineto{\pgfqpoint{0.934754in}{0.739656in}}%
\pgfpathlineto{\pgfqpoint{0.934457in}{0.739656in}}%
\pgfpathlineto{\pgfqpoint{0.934159in}{0.739656in}}%
\pgfpathlineto{\pgfqpoint{0.933862in}{0.739656in}}%
\pgfpathlineto{\pgfqpoint{0.933564in}{0.739656in}}%
\pgfpathlineto{\pgfqpoint{0.933267in}{0.739656in}}%
\pgfpathlineto{\pgfqpoint{0.932969in}{0.739656in}}%
\pgfpathlineto{\pgfqpoint{0.932672in}{0.739656in}}%
\pgfpathlineto{\pgfqpoint{0.932374in}{0.739656in}}%
\pgfpathlineto{\pgfqpoint{0.932077in}{0.739656in}}%
\pgfpathlineto{\pgfqpoint{0.931779in}{0.739656in}}%
\pgfpathlineto{\pgfqpoint{0.931482in}{0.739656in}}%
\pgfpathlineto{\pgfqpoint{0.931184in}{0.739656in}}%
\pgfpathlineto{\pgfqpoint{0.930887in}{0.739656in}}%
\pgfpathlineto{\pgfqpoint{0.930589in}{0.739656in}}%
\pgfpathlineto{\pgfqpoint{0.930292in}{0.739656in}}%
\pgfpathlineto{\pgfqpoint{0.929994in}{0.739656in}}%
\pgfpathlineto{\pgfqpoint{0.929697in}{0.739656in}}%
\pgfpathlineto{\pgfqpoint{0.929400in}{0.739656in}}%
\pgfpathlineto{\pgfqpoint{0.929102in}{0.739656in}}%
\pgfpathlineto{\pgfqpoint{0.928805in}{0.739656in}}%
\pgfpathlineto{\pgfqpoint{0.928507in}{0.739656in}}%
\pgfpathlineto{\pgfqpoint{0.928210in}{0.739656in}}%
\pgfpathlineto{\pgfqpoint{0.927912in}{0.739656in}}%
\pgfpathlineto{\pgfqpoint{0.927615in}{0.739656in}}%
\pgfpathlineto{\pgfqpoint{0.927317in}{0.739656in}}%
\pgfpathlineto{\pgfqpoint{0.927020in}{0.739656in}}%
\pgfpathlineto{\pgfqpoint{0.926722in}{0.739656in}}%
\pgfpathlineto{\pgfqpoint{0.926425in}{0.739656in}}%
\pgfpathlineto{\pgfqpoint{0.926127in}{0.739656in}}%
\pgfpathlineto{\pgfqpoint{0.925830in}{0.739656in}}%
\pgfpathlineto{\pgfqpoint{0.925532in}{0.739656in}}%
\pgfpathlineto{\pgfqpoint{0.925235in}{0.739656in}}%
\pgfpathlineto{\pgfqpoint{0.924937in}{0.739656in}}%
\pgfpathlineto{\pgfqpoint{0.924640in}{0.739656in}}%
\pgfpathlineto{\pgfqpoint{0.924342in}{0.739656in}}%
\pgfpathlineto{\pgfqpoint{0.924045in}{0.739656in}}%
\pgfpathlineto{\pgfqpoint{0.923747in}{0.739656in}}%
\pgfpathlineto{\pgfqpoint{0.923450in}{0.739656in}}%
\pgfpathlineto{\pgfqpoint{0.923153in}{0.739656in}}%
\pgfpathlineto{\pgfqpoint{0.922855in}{0.739656in}}%
\pgfpathlineto{\pgfqpoint{0.922558in}{0.739656in}}%
\pgfpathlineto{\pgfqpoint{0.922260in}{0.739656in}}%
\pgfpathlineto{\pgfqpoint{0.921963in}{0.739656in}}%
\pgfpathlineto{\pgfqpoint{0.921665in}{0.739656in}}%
\pgfpathlineto{\pgfqpoint{0.921368in}{0.739656in}}%
\pgfpathlineto{\pgfqpoint{0.921070in}{0.739656in}}%
\pgfpathlineto{\pgfqpoint{0.920773in}{0.739656in}}%
\pgfpathlineto{\pgfqpoint{0.920475in}{0.739656in}}%
\pgfpathlineto{\pgfqpoint{0.920178in}{0.739656in}}%
\pgfpathlineto{\pgfqpoint{0.919880in}{0.739656in}}%
\pgfpathlineto{\pgfqpoint{0.919583in}{0.739656in}}%
\pgfpathlineto{\pgfqpoint{0.919285in}{0.739656in}}%
\pgfpathlineto{\pgfqpoint{0.918988in}{0.739656in}}%
\pgfpathlineto{\pgfqpoint{0.918690in}{0.739656in}}%
\pgfpathlineto{\pgfqpoint{0.918393in}{0.739656in}}%
\pgfpathlineto{\pgfqpoint{0.918095in}{0.739656in}}%
\pgfpathlineto{\pgfqpoint{0.917798in}{0.739656in}}%
\pgfpathlineto{\pgfqpoint{0.917500in}{0.739656in}}%
\pgfpathlineto{\pgfqpoint{0.917203in}{0.739656in}}%
\pgfpathlineto{\pgfqpoint{0.916905in}{0.739656in}}%
\pgfpathlineto{\pgfqpoint{0.916608in}{0.739656in}}%
\pgfpathlineto{\pgfqpoint{0.916311in}{0.739656in}}%
\pgfpathlineto{\pgfqpoint{0.916013in}{0.739656in}}%
\pgfpathlineto{\pgfqpoint{0.915716in}{0.739656in}}%
\pgfpathlineto{\pgfqpoint{0.915418in}{0.739656in}}%
\pgfpathlineto{\pgfqpoint{0.915121in}{0.739656in}}%
\pgfpathlineto{\pgfqpoint{0.914823in}{0.739656in}}%
\pgfpathlineto{\pgfqpoint{0.914526in}{0.739656in}}%
\pgfpathlineto{\pgfqpoint{0.914228in}{0.739656in}}%
\pgfpathlineto{\pgfqpoint{0.913931in}{0.739656in}}%
\pgfpathlineto{\pgfqpoint{0.913633in}{0.739656in}}%
\pgfpathlineto{\pgfqpoint{0.913336in}{0.739656in}}%
\pgfpathlineto{\pgfqpoint{0.913038in}{0.739656in}}%
\pgfpathlineto{\pgfqpoint{0.912741in}{0.739656in}}%
\pgfpathlineto{\pgfqpoint{0.912443in}{0.739656in}}%
\pgfpathlineto{\pgfqpoint{0.912146in}{0.739656in}}%
\pgfpathlineto{\pgfqpoint{0.911848in}{0.739656in}}%
\pgfpathlineto{\pgfqpoint{0.911551in}{0.739656in}}%
\pgfpathlineto{\pgfqpoint{0.911253in}{0.739656in}}%
\pgfpathlineto{\pgfqpoint{0.910956in}{0.739656in}}%
\pgfpathlineto{\pgfqpoint{0.910658in}{0.739656in}}%
\pgfpathlineto{\pgfqpoint{0.910361in}{0.739656in}}%
\pgfpathlineto{\pgfqpoint{0.910063in}{0.739656in}}%
\pgfpathlineto{\pgfqpoint{0.909766in}{0.739656in}}%
\pgfpathlineto{\pgfqpoint{0.909469in}{0.739656in}}%
\pgfpathlineto{\pgfqpoint{0.909171in}{0.739656in}}%
\pgfpathlineto{\pgfqpoint{0.908874in}{0.739656in}}%
\pgfpathlineto{\pgfqpoint{0.908576in}{0.739656in}}%
\pgfpathlineto{\pgfqpoint{0.908279in}{0.739656in}}%
\pgfpathlineto{\pgfqpoint{0.907981in}{0.739656in}}%
\pgfpathlineto{\pgfqpoint{0.907684in}{0.739656in}}%
\pgfpathlineto{\pgfqpoint{0.907386in}{0.739656in}}%
\pgfpathlineto{\pgfqpoint{0.907089in}{0.739656in}}%
\pgfpathlineto{\pgfqpoint{0.906791in}{0.739656in}}%
\pgfpathlineto{\pgfqpoint{0.906494in}{0.739656in}}%
\pgfpathlineto{\pgfqpoint{0.906196in}{0.739656in}}%
\pgfpathlineto{\pgfqpoint{0.905899in}{0.739656in}}%
\pgfpathlineto{\pgfqpoint{0.905601in}{0.739656in}}%
\pgfpathlineto{\pgfqpoint{0.905304in}{0.739656in}}%
\pgfpathlineto{\pgfqpoint{0.905006in}{0.739656in}}%
\pgfpathlineto{\pgfqpoint{0.904709in}{0.739656in}}%
\pgfpathlineto{\pgfqpoint{0.904411in}{0.739656in}}%
\pgfpathlineto{\pgfqpoint{0.904114in}{0.739656in}}%
\pgfpathlineto{\pgfqpoint{0.903816in}{0.739656in}}%
\pgfpathlineto{\pgfqpoint{0.903519in}{0.739656in}}%
\pgfpathlineto{\pgfqpoint{0.903222in}{0.739656in}}%
\pgfpathlineto{\pgfqpoint{0.902924in}{0.739656in}}%
\pgfpathlineto{\pgfqpoint{0.902627in}{0.739656in}}%
\pgfpathlineto{\pgfqpoint{0.902329in}{0.739656in}}%
\pgfpathlineto{\pgfqpoint{0.902032in}{0.739656in}}%
\pgfpathlineto{\pgfqpoint{0.901734in}{0.739656in}}%
\pgfpathlineto{\pgfqpoint{0.901437in}{0.739656in}}%
\pgfpathlineto{\pgfqpoint{0.901139in}{0.739656in}}%
\pgfpathlineto{\pgfqpoint{0.900842in}{0.739656in}}%
\pgfpathlineto{\pgfqpoint{0.900544in}{0.739656in}}%
\pgfpathlineto{\pgfqpoint{0.900247in}{0.739656in}}%
\pgfpathlineto{\pgfqpoint{0.899949in}{0.739656in}}%
\pgfpathlineto{\pgfqpoint{0.899652in}{0.739656in}}%
\pgfpathlineto{\pgfqpoint{0.899354in}{0.739656in}}%
\pgfpathlineto{\pgfqpoint{0.899057in}{0.739656in}}%
\pgfpathlineto{\pgfqpoint{0.898759in}{0.739656in}}%
\pgfpathlineto{\pgfqpoint{0.898462in}{0.739656in}}%
\pgfpathlineto{\pgfqpoint{0.898164in}{0.739656in}}%
\pgfpathlineto{\pgfqpoint{0.897867in}{0.739656in}}%
\pgfpathlineto{\pgfqpoint{0.897569in}{0.739656in}}%
\pgfpathlineto{\pgfqpoint{0.897272in}{0.739656in}}%
\pgfpathlineto{\pgfqpoint{0.896974in}{0.739656in}}%
\pgfpathlineto{\pgfqpoint{0.896677in}{0.739656in}}%
\pgfpathlineto{\pgfqpoint{0.896380in}{0.739656in}}%
\pgfpathlineto{\pgfqpoint{0.896082in}{0.739656in}}%
\pgfpathlineto{\pgfqpoint{0.895785in}{0.739656in}}%
\pgfpathlineto{\pgfqpoint{0.895487in}{0.739656in}}%
\pgfpathlineto{\pgfqpoint{0.895190in}{0.739656in}}%
\pgfpathlineto{\pgfqpoint{0.894892in}{0.739656in}}%
\pgfpathlineto{\pgfqpoint{0.894595in}{0.739656in}}%
\pgfpathlineto{\pgfqpoint{0.894297in}{0.739656in}}%
\pgfpathlineto{\pgfqpoint{0.894000in}{0.739656in}}%
\pgfpathlineto{\pgfqpoint{0.893702in}{0.739656in}}%
\pgfpathlineto{\pgfqpoint{0.893405in}{0.739656in}}%
\pgfpathlineto{\pgfqpoint{0.893107in}{0.739656in}}%
\pgfpathlineto{\pgfqpoint{0.892810in}{0.739656in}}%
\pgfpathlineto{\pgfqpoint{0.892512in}{0.739656in}}%
\pgfpathlineto{\pgfqpoint{0.892215in}{0.739656in}}%
\pgfpathlineto{\pgfqpoint{0.891917in}{0.739656in}}%
\pgfpathlineto{\pgfqpoint{0.891620in}{0.739656in}}%
\pgfpathlineto{\pgfqpoint{0.891322in}{0.739656in}}%
\pgfpathlineto{\pgfqpoint{0.891025in}{0.739656in}}%
\pgfpathlineto{\pgfqpoint{0.890727in}{0.739656in}}%
\pgfpathlineto{\pgfqpoint{0.890430in}{0.739656in}}%
\pgfpathlineto{\pgfqpoint{0.890132in}{0.739656in}}%
\pgfpathlineto{\pgfqpoint{0.889835in}{0.739656in}}%
\pgfpathlineto{\pgfqpoint{0.889538in}{0.739656in}}%
\pgfpathlineto{\pgfqpoint{0.889240in}{0.739656in}}%
\pgfpathlineto{\pgfqpoint{0.888943in}{0.739656in}}%
\pgfpathlineto{\pgfqpoint{0.888645in}{0.739656in}}%
\pgfpathlineto{\pgfqpoint{0.888348in}{0.739656in}}%
\pgfpathlineto{\pgfqpoint{0.888050in}{0.739656in}}%
\pgfpathlineto{\pgfqpoint{0.887753in}{0.739656in}}%
\pgfpathlineto{\pgfqpoint{0.887455in}{0.739656in}}%
\pgfpathlineto{\pgfqpoint{0.887158in}{0.739656in}}%
\pgfpathlineto{\pgfqpoint{0.886860in}{0.739656in}}%
\pgfpathlineto{\pgfqpoint{0.886563in}{0.739656in}}%
\pgfpathlineto{\pgfqpoint{0.886265in}{0.739656in}}%
\pgfpathlineto{\pgfqpoint{0.885968in}{0.739656in}}%
\pgfpathlineto{\pgfqpoint{0.885670in}{0.739656in}}%
\pgfpathlineto{\pgfqpoint{0.885373in}{0.739656in}}%
\pgfpathlineto{\pgfqpoint{0.885075in}{0.739656in}}%
\pgfpathlineto{\pgfqpoint{0.884778in}{0.739656in}}%
\pgfpathlineto{\pgfqpoint{0.884480in}{0.739656in}}%
\pgfpathlineto{\pgfqpoint{0.884183in}{0.739656in}}%
\pgfpathlineto{\pgfqpoint{0.883885in}{0.739656in}}%
\pgfpathlineto{\pgfqpoint{0.883588in}{0.739656in}}%
\pgfpathlineto{\pgfqpoint{0.883291in}{0.739656in}}%
\pgfpathlineto{\pgfqpoint{0.882993in}{0.739656in}}%
\pgfpathlineto{\pgfqpoint{0.882696in}{0.739656in}}%
\pgfpathlineto{\pgfqpoint{0.882398in}{0.739656in}}%
\pgfpathlineto{\pgfqpoint{0.882101in}{0.739656in}}%
\pgfpathlineto{\pgfqpoint{0.881803in}{0.739656in}}%
\pgfpathlineto{\pgfqpoint{0.881506in}{0.739656in}}%
\pgfpathlineto{\pgfqpoint{0.881208in}{0.739656in}}%
\pgfpathlineto{\pgfqpoint{0.880911in}{0.739656in}}%
\pgfpathlineto{\pgfqpoint{0.880613in}{0.739656in}}%
\pgfpathlineto{\pgfqpoint{0.880316in}{0.739656in}}%
\pgfpathlineto{\pgfqpoint{0.880018in}{0.739656in}}%
\pgfpathlineto{\pgfqpoint{0.879721in}{0.739656in}}%
\pgfpathlineto{\pgfqpoint{0.879423in}{0.739656in}}%
\pgfpathlineto{\pgfqpoint{0.879126in}{0.739656in}}%
\pgfpathlineto{\pgfqpoint{0.878828in}{0.739656in}}%
\pgfpathlineto{\pgfqpoint{0.878531in}{0.739656in}}%
\pgfpathlineto{\pgfqpoint{0.878233in}{0.739656in}}%
\pgfpathlineto{\pgfqpoint{0.877936in}{0.739656in}}%
\pgfpathlineto{\pgfqpoint{0.877638in}{0.739656in}}%
\pgfpathlineto{\pgfqpoint{0.877341in}{0.739656in}}%
\pgfpathlineto{\pgfqpoint{0.877043in}{0.739656in}}%
\pgfpathlineto{\pgfqpoint{0.876746in}{0.739656in}}%
\pgfpathlineto{\pgfqpoint{0.876449in}{0.739656in}}%
\pgfpathlineto{\pgfqpoint{0.876151in}{0.739656in}}%
\pgfpathlineto{\pgfqpoint{0.875854in}{0.739656in}}%
\pgfpathlineto{\pgfqpoint{0.875556in}{0.739656in}}%
\pgfpathlineto{\pgfqpoint{0.875259in}{0.739656in}}%
\pgfpathlineto{\pgfqpoint{0.874961in}{0.739656in}}%
\pgfpathlineto{\pgfqpoint{0.874664in}{0.739656in}}%
\pgfpathlineto{\pgfqpoint{0.874366in}{0.739656in}}%
\pgfpathlineto{\pgfqpoint{0.874069in}{0.739656in}}%
\pgfpathlineto{\pgfqpoint{0.873771in}{0.739656in}}%
\pgfpathlineto{\pgfqpoint{0.873474in}{0.739656in}}%
\pgfpathlineto{\pgfqpoint{0.873176in}{0.739656in}}%
\pgfpathlineto{\pgfqpoint{0.872879in}{0.739656in}}%
\pgfpathlineto{\pgfqpoint{0.872581in}{0.739656in}}%
\pgfpathlineto{\pgfqpoint{0.872284in}{0.739656in}}%
\pgfpathlineto{\pgfqpoint{0.871986in}{0.739656in}}%
\pgfpathlineto{\pgfqpoint{0.871689in}{0.739656in}}%
\pgfpathlineto{\pgfqpoint{0.871391in}{0.739656in}}%
\pgfpathlineto{\pgfqpoint{0.871094in}{0.739656in}}%
\pgfpathlineto{\pgfqpoint{0.870796in}{0.739656in}}%
\pgfpathlineto{\pgfqpoint{0.870499in}{0.739656in}}%
\pgfpathlineto{\pgfqpoint{0.870201in}{0.739656in}}%
\pgfpathlineto{\pgfqpoint{0.869904in}{0.739656in}}%
\pgfpathlineto{\pgfqpoint{0.869607in}{0.739656in}}%
\pgfpathlineto{\pgfqpoint{0.869309in}{0.739656in}}%
\pgfpathlineto{\pgfqpoint{0.869012in}{0.739656in}}%
\pgfpathlineto{\pgfqpoint{0.868714in}{0.739656in}}%
\pgfpathlineto{\pgfqpoint{0.868417in}{0.739656in}}%
\pgfpathlineto{\pgfqpoint{0.868119in}{0.739656in}}%
\pgfpathlineto{\pgfqpoint{0.867822in}{0.739656in}}%
\pgfpathlineto{\pgfqpoint{0.867524in}{0.739656in}}%
\pgfpathlineto{\pgfqpoint{0.867227in}{0.739656in}}%
\pgfpathlineto{\pgfqpoint{0.866929in}{0.739656in}}%
\pgfpathlineto{\pgfqpoint{0.866632in}{0.739656in}}%
\pgfpathlineto{\pgfqpoint{0.866334in}{0.739656in}}%
\pgfpathlineto{\pgfqpoint{0.866037in}{0.739656in}}%
\pgfpathlineto{\pgfqpoint{0.865739in}{0.739656in}}%
\pgfpathlineto{\pgfqpoint{0.865442in}{0.739656in}}%
\pgfpathlineto{\pgfqpoint{0.865144in}{0.739656in}}%
\pgfpathlineto{\pgfqpoint{0.864847in}{0.739656in}}%
\pgfpathlineto{\pgfqpoint{0.864549in}{0.739656in}}%
\pgfpathlineto{\pgfqpoint{0.864252in}{0.739656in}}%
\pgfpathlineto{\pgfqpoint{0.863954in}{0.739656in}}%
\pgfpathlineto{\pgfqpoint{0.863657in}{0.739656in}}%
\pgfpathlineto{\pgfqpoint{0.863360in}{0.739656in}}%
\pgfpathlineto{\pgfqpoint{0.863062in}{0.739656in}}%
\pgfpathlineto{\pgfqpoint{0.862765in}{0.739656in}}%
\pgfpathlineto{\pgfqpoint{0.862467in}{0.739656in}}%
\pgfpathlineto{\pgfqpoint{0.862170in}{0.739656in}}%
\pgfpathlineto{\pgfqpoint{0.861872in}{0.739656in}}%
\pgfpathlineto{\pgfqpoint{0.861575in}{0.739656in}}%
\pgfpathlineto{\pgfqpoint{0.861277in}{0.739656in}}%
\pgfpathlineto{\pgfqpoint{0.860980in}{0.739656in}}%
\pgfpathlineto{\pgfqpoint{0.860682in}{0.739656in}}%
\pgfpathlineto{\pgfqpoint{0.860385in}{0.739656in}}%
\pgfpathlineto{\pgfqpoint{0.860087in}{0.739656in}}%
\pgfpathlineto{\pgfqpoint{0.859790in}{0.739656in}}%
\pgfpathlineto{\pgfqpoint{0.859492in}{0.739656in}}%
\pgfpathlineto{\pgfqpoint{0.859195in}{0.739656in}}%
\pgfpathlineto{\pgfqpoint{0.858897in}{0.739656in}}%
\pgfpathlineto{\pgfqpoint{0.858600in}{0.739656in}}%
\pgfpathlineto{\pgfqpoint{0.858302in}{0.739656in}}%
\pgfpathlineto{\pgfqpoint{0.858005in}{0.739656in}}%
\pgfpathlineto{\pgfqpoint{0.857707in}{0.739656in}}%
\pgfpathlineto{\pgfqpoint{0.857410in}{0.739656in}}%
\pgfpathlineto{\pgfqpoint{0.857112in}{0.739656in}}%
\pgfpathlineto{\pgfqpoint{0.856815in}{0.739656in}}%
\pgfpathlineto{\pgfqpoint{0.856518in}{0.739656in}}%
\pgfpathlineto{\pgfqpoint{0.856220in}{0.739656in}}%
\pgfpathlineto{\pgfqpoint{0.855923in}{0.739656in}}%
\pgfpathlineto{\pgfqpoint{0.855625in}{0.739656in}}%
\pgfpathlineto{\pgfqpoint{0.855328in}{0.739656in}}%
\pgfpathlineto{\pgfqpoint{0.855030in}{0.739656in}}%
\pgfpathlineto{\pgfqpoint{0.854733in}{0.739656in}}%
\pgfpathlineto{\pgfqpoint{0.854435in}{0.739656in}}%
\pgfpathlineto{\pgfqpoint{0.854138in}{0.739656in}}%
\pgfpathlineto{\pgfqpoint{0.853840in}{0.739656in}}%
\pgfpathlineto{\pgfqpoint{0.853543in}{0.739656in}}%
\pgfpathlineto{\pgfqpoint{0.853245in}{0.739656in}}%
\pgfpathlineto{\pgfqpoint{0.852948in}{0.739656in}}%
\pgfpathlineto{\pgfqpoint{0.852650in}{0.739656in}}%
\pgfpathlineto{\pgfqpoint{0.852353in}{0.739656in}}%
\pgfpathlineto{\pgfqpoint{0.852055in}{0.739656in}}%
\pgfpathlineto{\pgfqpoint{0.851758in}{0.739656in}}%
\pgfpathlineto{\pgfqpoint{0.851460in}{0.739656in}}%
\pgfpathlineto{\pgfqpoint{0.851163in}{0.739656in}}%
\pgfpathlineto{\pgfqpoint{0.850865in}{0.739656in}}%
\pgfpathlineto{\pgfqpoint{0.850568in}{0.739656in}}%
\pgfpathlineto{\pgfqpoint{0.850270in}{0.739656in}}%
\pgfpathlineto{\pgfqpoint{0.849973in}{0.739656in}}%
\pgfpathlineto{\pgfqpoint{0.849676in}{0.739656in}}%
\pgfpathlineto{\pgfqpoint{0.849378in}{0.739656in}}%
\pgfpathlineto{\pgfqpoint{0.849081in}{0.739656in}}%
\pgfpathlineto{\pgfqpoint{0.848783in}{0.739656in}}%
\pgfpathlineto{\pgfqpoint{0.848486in}{0.739656in}}%
\pgfpathlineto{\pgfqpoint{0.848188in}{0.739656in}}%
\pgfpathlineto{\pgfqpoint{0.847891in}{0.739656in}}%
\pgfpathlineto{\pgfqpoint{0.847593in}{0.739656in}}%
\pgfpathlineto{\pgfqpoint{0.847296in}{0.739656in}}%
\pgfpathlineto{\pgfqpoint{0.846998in}{0.739656in}}%
\pgfpathlineto{\pgfqpoint{0.846701in}{0.739656in}}%
\pgfpathlineto{\pgfqpoint{0.846403in}{0.739656in}}%
\pgfpathlineto{\pgfqpoint{0.846106in}{0.739656in}}%
\pgfpathlineto{\pgfqpoint{0.845808in}{0.739656in}}%
\pgfpathlineto{\pgfqpoint{0.845511in}{0.739656in}}%
\pgfpathlineto{\pgfqpoint{0.845213in}{0.739656in}}%
\pgfpathlineto{\pgfqpoint{0.844916in}{0.739656in}}%
\pgfpathlineto{\pgfqpoint{0.844618in}{0.739656in}}%
\pgfpathlineto{\pgfqpoint{0.844321in}{0.739656in}}%
\pgfpathlineto{\pgfqpoint{0.844023in}{0.739656in}}%
\pgfpathlineto{\pgfqpoint{0.843726in}{0.739656in}}%
\pgfpathlineto{\pgfqpoint{0.843429in}{0.739656in}}%
\pgfpathlineto{\pgfqpoint{0.843131in}{0.739656in}}%
\pgfpathlineto{\pgfqpoint{0.842834in}{0.739656in}}%
\pgfpathlineto{\pgfqpoint{0.842536in}{0.739656in}}%
\pgfpathlineto{\pgfqpoint{0.842239in}{0.739656in}}%
\pgfpathlineto{\pgfqpoint{0.841941in}{0.739656in}}%
\pgfpathlineto{\pgfqpoint{0.841644in}{0.739656in}}%
\pgfpathlineto{\pgfqpoint{0.841346in}{0.739656in}}%
\pgfpathlineto{\pgfqpoint{0.841049in}{0.739656in}}%
\pgfpathlineto{\pgfqpoint{0.840751in}{0.739656in}}%
\pgfpathlineto{\pgfqpoint{0.840454in}{0.739656in}}%
\pgfpathlineto{\pgfqpoint{0.840156in}{0.739656in}}%
\pgfpathlineto{\pgfqpoint{0.839859in}{0.739656in}}%
\pgfpathlineto{\pgfqpoint{0.839561in}{0.739656in}}%
\pgfpathlineto{\pgfqpoint{0.839264in}{0.739656in}}%
\pgfpathlineto{\pgfqpoint{0.838966in}{0.739656in}}%
\pgfpathlineto{\pgfqpoint{0.838669in}{0.739656in}}%
\pgfpathlineto{\pgfqpoint{0.838371in}{0.739656in}}%
\pgfpathlineto{\pgfqpoint{0.838074in}{0.739656in}}%
\pgfpathlineto{\pgfqpoint{0.837776in}{0.739656in}}%
\pgfpathlineto{\pgfqpoint{0.837479in}{0.739656in}}%
\pgfpathlineto{\pgfqpoint{0.837181in}{0.739656in}}%
\pgfpathlineto{\pgfqpoint{0.836884in}{0.739656in}}%
\pgfpathlineto{\pgfqpoint{0.836587in}{0.739656in}}%
\pgfpathlineto{\pgfqpoint{0.836289in}{0.739656in}}%
\pgfpathlineto{\pgfqpoint{0.835992in}{0.739656in}}%
\pgfpathlineto{\pgfqpoint{0.835694in}{0.739656in}}%
\pgfpathlineto{\pgfqpoint{0.835397in}{0.739656in}}%
\pgfpathlineto{\pgfqpoint{0.835099in}{0.739656in}}%
\pgfpathlineto{\pgfqpoint{0.834802in}{0.739656in}}%
\pgfpathlineto{\pgfqpoint{0.834504in}{0.739656in}}%
\pgfpathlineto{\pgfqpoint{0.834207in}{0.739656in}}%
\pgfpathlineto{\pgfqpoint{0.833909in}{0.739656in}}%
\pgfpathlineto{\pgfqpoint{0.833612in}{0.739656in}}%
\pgfpathlineto{\pgfqpoint{0.833314in}{0.739656in}}%
\pgfpathlineto{\pgfqpoint{0.833017in}{0.739656in}}%
\pgfpathlineto{\pgfqpoint{0.832719in}{0.739656in}}%
\pgfpathlineto{\pgfqpoint{0.832422in}{0.739656in}}%
\pgfpathlineto{\pgfqpoint{0.832124in}{0.739656in}}%
\pgfpathlineto{\pgfqpoint{0.831827in}{0.739656in}}%
\pgfpathlineto{\pgfqpoint{0.831529in}{0.739656in}}%
\pgfpathlineto{\pgfqpoint{0.831232in}{0.739656in}}%
\pgfpathlineto{\pgfqpoint{0.830934in}{0.739656in}}%
\pgfpathlineto{\pgfqpoint{0.830637in}{0.739656in}}%
\pgfpathlineto{\pgfqpoint{0.830339in}{0.739656in}}%
\pgfpathlineto{\pgfqpoint{0.830042in}{0.739656in}}%
\pgfpathlineto{\pgfqpoint{0.829745in}{0.739656in}}%
\pgfpathlineto{\pgfqpoint{0.829447in}{0.739656in}}%
\pgfpathlineto{\pgfqpoint{0.829150in}{0.739656in}}%
\pgfpathlineto{\pgfqpoint{0.828852in}{0.739656in}}%
\pgfpathlineto{\pgfqpoint{0.828555in}{0.739656in}}%
\pgfpathlineto{\pgfqpoint{0.828257in}{0.739656in}}%
\pgfpathlineto{\pgfqpoint{0.827960in}{0.739656in}}%
\pgfpathlineto{\pgfqpoint{0.827662in}{0.739656in}}%
\pgfpathlineto{\pgfqpoint{0.827365in}{0.739656in}}%
\pgfpathlineto{\pgfqpoint{0.827067in}{0.739656in}}%
\pgfpathlineto{\pgfqpoint{0.826770in}{0.739656in}}%
\pgfpathlineto{\pgfqpoint{0.826472in}{0.739656in}}%
\pgfpathlineto{\pgfqpoint{0.826175in}{0.739656in}}%
\pgfpathlineto{\pgfqpoint{0.825877in}{0.739656in}}%
\pgfpathlineto{\pgfqpoint{0.825580in}{0.739656in}}%
\pgfpathlineto{\pgfqpoint{0.825282in}{0.739656in}}%
\pgfpathlineto{\pgfqpoint{0.824985in}{0.739656in}}%
\pgfpathlineto{\pgfqpoint{0.824687in}{0.739656in}}%
\pgfpathlineto{\pgfqpoint{0.824390in}{0.739656in}}%
\pgfpathlineto{\pgfqpoint{0.824092in}{0.739656in}}%
\pgfpathlineto{\pgfqpoint{0.823795in}{0.739656in}}%
\pgfpathlineto{\pgfqpoint{0.823498in}{0.739656in}}%
\pgfpathlineto{\pgfqpoint{0.823200in}{0.739656in}}%
\pgfpathlineto{\pgfqpoint{0.822903in}{0.739656in}}%
\pgfpathlineto{\pgfqpoint{0.822605in}{0.739656in}}%
\pgfpathlineto{\pgfqpoint{0.822308in}{0.739656in}}%
\pgfpathlineto{\pgfqpoint{0.822010in}{0.739656in}}%
\pgfpathlineto{\pgfqpoint{0.821713in}{0.739656in}}%
\pgfpathlineto{\pgfqpoint{0.821415in}{0.739656in}}%
\pgfpathlineto{\pgfqpoint{0.821118in}{0.739656in}}%
\pgfpathlineto{\pgfqpoint{0.820820in}{0.739656in}}%
\pgfpathlineto{\pgfqpoint{0.820523in}{0.739656in}}%
\pgfpathlineto{\pgfqpoint{0.820225in}{0.739656in}}%
\pgfpathlineto{\pgfqpoint{0.819928in}{0.739656in}}%
\pgfpathlineto{\pgfqpoint{0.819630in}{0.739656in}}%
\pgfpathlineto{\pgfqpoint{0.819333in}{0.739656in}}%
\pgfpathlineto{\pgfqpoint{0.819035in}{0.739656in}}%
\pgfpathlineto{\pgfqpoint{0.818738in}{0.739656in}}%
\pgfpathlineto{\pgfqpoint{0.818440in}{0.739656in}}%
\pgfpathlineto{\pgfqpoint{0.818143in}{0.739656in}}%
\pgfpathlineto{\pgfqpoint{0.817845in}{0.739656in}}%
\pgfpathlineto{\pgfqpoint{0.817548in}{0.739656in}}%
\pgfpathlineto{\pgfqpoint{0.817250in}{0.739656in}}%
\pgfpathlineto{\pgfqpoint{0.816953in}{0.739656in}}%
\pgfpathlineto{\pgfqpoint{0.816656in}{0.739656in}}%
\pgfpathlineto{\pgfqpoint{0.816358in}{0.739656in}}%
\pgfpathlineto{\pgfqpoint{0.816061in}{0.739656in}}%
\pgfpathlineto{\pgfqpoint{0.815763in}{0.739656in}}%
\pgfpathlineto{\pgfqpoint{0.815466in}{0.739656in}}%
\pgfpathlineto{\pgfqpoint{0.815168in}{0.739656in}}%
\pgfpathlineto{\pgfqpoint{0.814871in}{0.739656in}}%
\pgfpathlineto{\pgfqpoint{0.814573in}{0.739656in}}%
\pgfpathlineto{\pgfqpoint{0.814276in}{0.739656in}}%
\pgfpathlineto{\pgfqpoint{0.813978in}{0.739656in}}%
\pgfpathlineto{\pgfqpoint{0.813681in}{0.739656in}}%
\pgfpathlineto{\pgfqpoint{0.813383in}{0.739656in}}%
\pgfpathlineto{\pgfqpoint{0.813086in}{0.739656in}}%
\pgfpathlineto{\pgfqpoint{0.812788in}{0.739656in}}%
\pgfpathlineto{\pgfqpoint{0.812491in}{0.739656in}}%
\pgfpathlineto{\pgfqpoint{0.812193in}{0.739656in}}%
\pgfpathlineto{\pgfqpoint{0.811896in}{0.739656in}}%
\pgfpathlineto{\pgfqpoint{0.811598in}{0.739656in}}%
\pgfpathlineto{\pgfqpoint{0.811301in}{0.739656in}}%
\pgfpathlineto{\pgfqpoint{0.811003in}{0.739656in}}%
\pgfpathlineto{\pgfqpoint{0.810706in}{0.739656in}}%
\pgfpathlineto{\pgfqpoint{0.810408in}{0.739656in}}%
\pgfpathlineto{\pgfqpoint{0.810111in}{0.739656in}}%
\pgfpathlineto{\pgfqpoint{0.809814in}{0.739656in}}%
\pgfpathlineto{\pgfqpoint{0.809516in}{0.739656in}}%
\pgfpathlineto{\pgfqpoint{0.809219in}{0.739656in}}%
\pgfpathlineto{\pgfqpoint{0.808921in}{0.739656in}}%
\pgfpathlineto{\pgfqpoint{0.808624in}{0.739656in}}%
\pgfpathlineto{\pgfqpoint{0.808326in}{0.739656in}}%
\pgfpathlineto{\pgfqpoint{0.808029in}{0.739656in}}%
\pgfpathlineto{\pgfqpoint{0.807731in}{0.739656in}}%
\pgfpathlineto{\pgfqpoint{0.807434in}{0.739656in}}%
\pgfpathlineto{\pgfqpoint{0.807136in}{0.739656in}}%
\pgfpathlineto{\pgfqpoint{0.806839in}{0.739656in}}%
\pgfpathlineto{\pgfqpoint{0.806541in}{0.739656in}}%
\pgfpathlineto{\pgfqpoint{0.806244in}{0.739656in}}%
\pgfpathlineto{\pgfqpoint{0.805946in}{0.739656in}}%
\pgfpathlineto{\pgfqpoint{0.805649in}{0.739656in}}%
\pgfpathlineto{\pgfqpoint{0.805351in}{0.739656in}}%
\pgfpathlineto{\pgfqpoint{0.805054in}{0.739656in}}%
\pgfpathlineto{\pgfqpoint{0.804756in}{0.739656in}}%
\pgfpathlineto{\pgfqpoint{0.804459in}{0.739656in}}%
\pgfpathlineto{\pgfqpoint{0.804161in}{0.739656in}}%
\pgfpathlineto{\pgfqpoint{0.803864in}{0.739656in}}%
\pgfpathlineto{\pgfqpoint{0.803567in}{0.739656in}}%
\pgfpathlineto{\pgfqpoint{0.803269in}{0.739656in}}%
\pgfpathlineto{\pgfqpoint{0.802972in}{0.739656in}}%
\pgfpathlineto{\pgfqpoint{0.802674in}{0.739656in}}%
\pgfpathlineto{\pgfqpoint{0.802377in}{0.739656in}}%
\pgfpathlineto{\pgfqpoint{0.802079in}{0.739656in}}%
\pgfpathlineto{\pgfqpoint{0.801782in}{0.739656in}}%
\pgfpathlineto{\pgfqpoint{0.801484in}{0.739656in}}%
\pgfpathlineto{\pgfqpoint{0.801187in}{0.739656in}}%
\pgfpathlineto{\pgfqpoint{0.800889in}{0.739656in}}%
\pgfpathlineto{\pgfqpoint{0.800592in}{0.739656in}}%
\pgfpathlineto{\pgfqpoint{0.800294in}{0.739656in}}%
\pgfpathlineto{\pgfqpoint{0.799997in}{0.739656in}}%
\pgfpathlineto{\pgfqpoint{0.799699in}{0.739656in}}%
\pgfpathlineto{\pgfqpoint{0.799402in}{0.739656in}}%
\pgfpathlineto{\pgfqpoint{0.799104in}{0.739656in}}%
\pgfpathlineto{\pgfqpoint{0.798807in}{0.739656in}}%
\pgfpathlineto{\pgfqpoint{0.798509in}{0.739656in}}%
\pgfpathlineto{\pgfqpoint{0.798212in}{0.739656in}}%
\pgfpathlineto{\pgfqpoint{0.797914in}{0.739656in}}%
\pgfpathlineto{\pgfqpoint{0.797617in}{0.739656in}}%
\pgfpathlineto{\pgfqpoint{0.797319in}{0.739656in}}%
\pgfpathlineto{\pgfqpoint{0.797022in}{0.739656in}}%
\pgfpathlineto{\pgfqpoint{0.796725in}{0.739656in}}%
\pgfpathlineto{\pgfqpoint{0.796427in}{0.739656in}}%
\pgfpathlineto{\pgfqpoint{0.796130in}{0.739656in}}%
\pgfpathlineto{\pgfqpoint{0.795832in}{0.739656in}}%
\pgfpathlineto{\pgfqpoint{0.795535in}{0.739656in}}%
\pgfpathlineto{\pgfqpoint{0.795237in}{0.739656in}}%
\pgfpathlineto{\pgfqpoint{0.794940in}{0.739656in}}%
\pgfpathlineto{\pgfqpoint{0.794642in}{0.739656in}}%
\pgfpathlineto{\pgfqpoint{0.794345in}{0.739656in}}%
\pgfpathlineto{\pgfqpoint{0.794047in}{0.739656in}}%
\pgfpathlineto{\pgfqpoint{0.793750in}{0.739656in}}%
\pgfpathlineto{\pgfqpoint{0.793452in}{0.739656in}}%
\pgfpathlineto{\pgfqpoint{0.793155in}{0.739656in}}%
\pgfpathlineto{\pgfqpoint{0.792857in}{0.739656in}}%
\pgfpathlineto{\pgfqpoint{0.792560in}{0.739656in}}%
\pgfpathlineto{\pgfqpoint{0.792262in}{0.739656in}}%
\pgfpathlineto{\pgfqpoint{0.791965in}{0.739656in}}%
\pgfpathlineto{\pgfqpoint{0.791667in}{0.739656in}}%
\pgfpathlineto{\pgfqpoint{0.791370in}{0.739656in}}%
\pgfpathlineto{\pgfqpoint{0.791072in}{0.739656in}}%
\pgfpathlineto{\pgfqpoint{0.790775in}{0.739656in}}%
\pgfpathlineto{\pgfqpoint{0.790477in}{0.739656in}}%
\pgfpathlineto{\pgfqpoint{0.790180in}{0.739656in}}%
\pgfpathlineto{\pgfqpoint{0.789883in}{0.739656in}}%
\pgfpathlineto{\pgfqpoint{0.789585in}{0.739656in}}%
\pgfpathlineto{\pgfqpoint{0.789288in}{0.739656in}}%
\pgfpathlineto{\pgfqpoint{0.788990in}{0.739656in}}%
\pgfpathlineto{\pgfqpoint{0.788693in}{0.739656in}}%
\pgfpathlineto{\pgfqpoint{0.788395in}{0.739656in}}%
\pgfpathlineto{\pgfqpoint{0.788098in}{0.739656in}}%
\pgfpathlineto{\pgfqpoint{0.787800in}{0.739656in}}%
\pgfpathlineto{\pgfqpoint{0.787503in}{0.739656in}}%
\pgfpathlineto{\pgfqpoint{0.787205in}{0.739656in}}%
\pgfpathlineto{\pgfqpoint{0.786908in}{0.739656in}}%
\pgfpathlineto{\pgfqpoint{0.786610in}{0.739656in}}%
\pgfpathlineto{\pgfqpoint{0.786313in}{0.739656in}}%
\pgfpathlineto{\pgfqpoint{0.786015in}{0.739656in}}%
\pgfpathlineto{\pgfqpoint{0.785718in}{0.739656in}}%
\pgfpathlineto{\pgfqpoint{0.785420in}{0.739656in}}%
\pgfpathlineto{\pgfqpoint{0.785123in}{0.739656in}}%
\pgfpathlineto{\pgfqpoint{0.784825in}{0.739656in}}%
\pgfpathlineto{\pgfqpoint{0.784528in}{0.739656in}}%
\pgfpathlineto{\pgfqpoint{0.784230in}{0.739656in}}%
\pgfpathlineto{\pgfqpoint{0.783933in}{0.739656in}}%
\pgfpathlineto{\pgfqpoint{0.783636in}{0.739656in}}%
\pgfpathlineto{\pgfqpoint{0.783338in}{0.739656in}}%
\pgfpathlineto{\pgfqpoint{0.783041in}{0.739656in}}%
\pgfpathlineto{\pgfqpoint{0.782743in}{0.739656in}}%
\pgfpathlineto{\pgfqpoint{0.782446in}{0.739656in}}%
\pgfpathlineto{\pgfqpoint{0.782148in}{0.739656in}}%
\pgfpathlineto{\pgfqpoint{0.781851in}{0.739656in}}%
\pgfpathlineto{\pgfqpoint{0.781553in}{0.739656in}}%
\pgfpathlineto{\pgfqpoint{0.781256in}{0.739656in}}%
\pgfpathlineto{\pgfqpoint{0.780958in}{0.739656in}}%
\pgfpathlineto{\pgfqpoint{0.780661in}{0.739656in}}%
\pgfpathlineto{\pgfqpoint{0.780363in}{0.739656in}}%
\pgfpathlineto{\pgfqpoint{0.780066in}{0.739656in}}%
\pgfpathlineto{\pgfqpoint{0.779768in}{0.739656in}}%
\pgfpathlineto{\pgfqpoint{0.779471in}{0.739656in}}%
\pgfpathlineto{\pgfqpoint{0.779173in}{0.739656in}}%
\pgfpathlineto{\pgfqpoint{0.778876in}{0.739656in}}%
\pgfpathlineto{\pgfqpoint{0.778578in}{0.739656in}}%
\pgfpathlineto{\pgfqpoint{0.778281in}{0.739656in}}%
\pgfpathlineto{\pgfqpoint{0.777983in}{0.739656in}}%
\pgfpathlineto{\pgfqpoint{0.777686in}{0.739656in}}%
\pgfpathlineto{\pgfqpoint{0.777388in}{0.739656in}}%
\pgfpathlineto{\pgfqpoint{0.777091in}{0.739656in}}%
\pgfpathlineto{\pgfqpoint{0.776794in}{0.739656in}}%
\pgfpathlineto{\pgfqpoint{0.776496in}{0.739656in}}%
\pgfpathlineto{\pgfqpoint{0.776199in}{0.739656in}}%
\pgfpathlineto{\pgfqpoint{0.775901in}{0.739656in}}%
\pgfpathlineto{\pgfqpoint{0.775604in}{0.739656in}}%
\pgfpathlineto{\pgfqpoint{0.775306in}{0.739656in}}%
\pgfpathlineto{\pgfqpoint{0.775009in}{0.739656in}}%
\pgfpathlineto{\pgfqpoint{0.774711in}{0.739656in}}%
\pgfpathlineto{\pgfqpoint{0.774414in}{0.739656in}}%
\pgfpathlineto{\pgfqpoint{0.774116in}{0.739656in}}%
\pgfpathlineto{\pgfqpoint{0.773819in}{0.739656in}}%
\pgfpathlineto{\pgfqpoint{0.773521in}{0.739656in}}%
\pgfpathlineto{\pgfqpoint{0.773224in}{0.739656in}}%
\pgfpathlineto{\pgfqpoint{0.772926in}{0.739656in}}%
\pgfpathlineto{\pgfqpoint{0.772629in}{0.739656in}}%
\pgfpathlineto{\pgfqpoint{0.772331in}{0.739656in}}%
\pgfpathlineto{\pgfqpoint{0.772034in}{0.739656in}}%
\pgfpathlineto{\pgfqpoint{0.771736in}{0.739656in}}%
\pgfpathlineto{\pgfqpoint{0.771439in}{0.739656in}}%
\pgfpathlineto{\pgfqpoint{0.771141in}{0.739656in}}%
\pgfpathlineto{\pgfqpoint{0.770844in}{0.739656in}}%
\pgfpathlineto{\pgfqpoint{0.770546in}{0.739656in}}%
\pgfpathlineto{\pgfqpoint{0.770249in}{0.739656in}}%
\pgfpathlineto{\pgfqpoint{0.769952in}{0.739656in}}%
\pgfpathlineto{\pgfqpoint{0.769654in}{0.739656in}}%
\pgfpathlineto{\pgfqpoint{0.769357in}{0.739656in}}%
\pgfpathlineto{\pgfqpoint{0.769059in}{0.739656in}}%
\pgfpathlineto{\pgfqpoint{0.768762in}{0.739656in}}%
\pgfpathlineto{\pgfqpoint{0.768464in}{0.739656in}}%
\pgfpathlineto{\pgfqpoint{0.768167in}{0.739656in}}%
\pgfpathlineto{\pgfqpoint{0.767869in}{0.739656in}}%
\pgfpathlineto{\pgfqpoint{0.767572in}{0.739656in}}%
\pgfpathlineto{\pgfqpoint{0.767274in}{0.739656in}}%
\pgfpathlineto{\pgfqpoint{0.766977in}{0.739656in}}%
\pgfpathlineto{\pgfqpoint{0.766679in}{0.739656in}}%
\pgfpathlineto{\pgfqpoint{0.766382in}{0.739656in}}%
\pgfpathlineto{\pgfqpoint{0.766084in}{0.739656in}}%
\pgfpathlineto{\pgfqpoint{0.765787in}{0.739656in}}%
\pgfpathlineto{\pgfqpoint{0.765489in}{0.739656in}}%
\pgfpathlineto{\pgfqpoint{0.765192in}{0.739656in}}%
\pgfpathlineto{\pgfqpoint{0.764894in}{0.739656in}}%
\pgfpathlineto{\pgfqpoint{0.764597in}{0.739656in}}%
\pgfpathlineto{\pgfqpoint{0.764299in}{0.739656in}}%
\pgfpathlineto{\pgfqpoint{0.764002in}{0.739656in}}%
\pgfpathlineto{\pgfqpoint{0.763705in}{0.739656in}}%
\pgfpathlineto{\pgfqpoint{0.763407in}{0.739656in}}%
\pgfpathlineto{\pgfqpoint{0.763110in}{0.739656in}}%
\pgfpathlineto{\pgfqpoint{0.762812in}{0.739656in}}%
\pgfpathlineto{\pgfqpoint{0.762515in}{0.739656in}}%
\pgfpathlineto{\pgfqpoint{0.762217in}{0.739656in}}%
\pgfpathlineto{\pgfqpoint{0.761920in}{0.739656in}}%
\pgfpathlineto{\pgfqpoint{0.761622in}{0.739656in}}%
\pgfpathlineto{\pgfqpoint{0.761325in}{0.739656in}}%
\pgfpathlineto{\pgfqpoint{0.761027in}{0.739656in}}%
\pgfpathlineto{\pgfqpoint{0.760730in}{0.739656in}}%
\pgfpathlineto{\pgfqpoint{0.760432in}{0.739656in}}%
\pgfpathlineto{\pgfqpoint{0.760135in}{0.739656in}}%
\pgfpathlineto{\pgfqpoint{0.759837in}{0.739656in}}%
\pgfpathlineto{\pgfqpoint{0.759540in}{0.739656in}}%
\pgfpathlineto{\pgfqpoint{0.759242in}{0.739656in}}%
\pgfpathlineto{\pgfqpoint{0.758945in}{0.739656in}}%
\pgfpathlineto{\pgfqpoint{0.758647in}{0.739656in}}%
\pgfpathlineto{\pgfqpoint{0.758350in}{0.739656in}}%
\pgfpathlineto{\pgfqpoint{0.758052in}{0.739656in}}%
\pgfpathlineto{\pgfqpoint{0.757755in}{0.739656in}}%
\pgfpathlineto{\pgfqpoint{0.757457in}{0.739656in}}%
\pgfpathlineto{\pgfqpoint{0.757160in}{0.739656in}}%
\pgfpathlineto{\pgfqpoint{0.756863in}{0.739656in}}%
\pgfpathlineto{\pgfqpoint{0.756565in}{0.739656in}}%
\pgfpathlineto{\pgfqpoint{0.756268in}{0.739656in}}%
\pgfpathlineto{\pgfqpoint{0.755970in}{0.739656in}}%
\pgfpathlineto{\pgfqpoint{0.755673in}{0.739656in}}%
\pgfpathlineto{\pgfqpoint{0.755375in}{0.739656in}}%
\pgfpathlineto{\pgfqpoint{0.755078in}{0.739656in}}%
\pgfpathlineto{\pgfqpoint{0.754780in}{0.739656in}}%
\pgfpathlineto{\pgfqpoint{0.754483in}{0.739656in}}%
\pgfpathlineto{\pgfqpoint{0.754185in}{0.739656in}}%
\pgfpathlineto{\pgfqpoint{0.753888in}{0.739656in}}%
\pgfpathlineto{\pgfqpoint{0.753590in}{0.739656in}}%
\pgfpathlineto{\pgfqpoint{0.753293in}{0.739656in}}%
\pgfpathlineto{\pgfqpoint{0.752995in}{0.739656in}}%
\pgfpathlineto{\pgfqpoint{0.752698in}{0.739656in}}%
\pgfpathlineto{\pgfqpoint{0.752400in}{0.739656in}}%
\pgfpathlineto{\pgfqpoint{0.752103in}{0.739656in}}%
\pgfpathlineto{\pgfqpoint{0.751805in}{0.739656in}}%
\pgfpathlineto{\pgfqpoint{0.751508in}{0.739656in}}%
\pgfpathlineto{\pgfqpoint{0.751210in}{0.739656in}}%
\pgfpathlineto{\pgfqpoint{0.750913in}{0.739656in}}%
\pgfpathlineto{\pgfqpoint{0.750615in}{0.739656in}}%
\pgfpathlineto{\pgfqpoint{0.750318in}{0.739656in}}%
\pgfpathlineto{\pgfqpoint{0.750021in}{0.739656in}}%
\pgfpathlineto{\pgfqpoint{0.749723in}{0.739656in}}%
\pgfpathlineto{\pgfqpoint{0.749426in}{0.739656in}}%
\pgfpathlineto{\pgfqpoint{0.749128in}{0.739656in}}%
\pgfpathlineto{\pgfqpoint{0.748831in}{0.739656in}}%
\pgfpathlineto{\pgfqpoint{0.748533in}{0.739656in}}%
\pgfpathlineto{\pgfqpoint{0.748236in}{0.739656in}}%
\pgfpathlineto{\pgfqpoint{0.747938in}{0.739656in}}%
\pgfpathlineto{\pgfqpoint{0.747641in}{0.739656in}}%
\pgfpathlineto{\pgfqpoint{0.747343in}{0.739656in}}%
\pgfpathlineto{\pgfqpoint{0.747046in}{0.739656in}}%
\pgfpathlineto{\pgfqpoint{0.746748in}{0.739656in}}%
\pgfpathlineto{\pgfqpoint{0.746451in}{0.739656in}}%
\pgfpathlineto{\pgfqpoint{0.746153in}{0.739656in}}%
\pgfpathlineto{\pgfqpoint{0.745856in}{0.739656in}}%
\pgfpathlineto{\pgfqpoint{0.745558in}{0.739656in}}%
\pgfpathlineto{\pgfqpoint{0.745261in}{0.739656in}}%
\pgfpathlineto{\pgfqpoint{0.744963in}{0.739656in}}%
\pgfpathlineto{\pgfqpoint{0.744666in}{0.739656in}}%
\pgfpathlineto{\pgfqpoint{0.744666in}{0.739656in}}%
\pgfpathclose%
\pgfusepath{fill}%
\end{pgfscope}%
\begin{pgfscope}%
\pgfsetbuttcap%
\pgfsetroundjoin%
\definecolor{currentfill}{rgb}{0.000000,0.000000,0.000000}%
\pgfsetfillcolor{currentfill}%
\pgfsetlinewidth{0.501875pt}%
\definecolor{currentstroke}{rgb}{0.000000,0.000000,0.000000}%
\pgfsetstrokecolor{currentstroke}%
\pgfsetdash{}{0pt}%
\pgfsys@defobject{currentmarker}{\pgfqpoint{0.000000in}{0.000000in}}{\pgfqpoint{0.000000in}{0.041667in}}{%
\pgfpathmoveto{\pgfqpoint{0.000000in}{0.000000in}}%
\pgfpathlineto{\pgfqpoint{0.000000in}{0.041667in}}%
\pgfusepath{stroke,fill}%
}%
\begin{pgfscope}%
\pgfsys@transformshift{0.689633in}{0.586309in}%
\pgfsys@useobject{currentmarker}{}%
\end{pgfscope}%
\end{pgfscope}%
\begin{pgfscope}%
\pgfsetbuttcap%
\pgfsetroundjoin%
\definecolor{currentfill}{rgb}{0.000000,0.000000,0.000000}%
\pgfsetfillcolor{currentfill}%
\pgfsetlinewidth{0.501875pt}%
\definecolor{currentstroke}{rgb}{0.000000,0.000000,0.000000}%
\pgfsetstrokecolor{currentstroke}%
\pgfsetdash{}{0pt}%
\pgfsys@defobject{currentmarker}{\pgfqpoint{0.000000in}{-0.041667in}}{\pgfqpoint{0.000000in}{0.000000in}}{%
\pgfpathmoveto{\pgfqpoint{0.000000in}{0.000000in}}%
\pgfpathlineto{\pgfqpoint{0.000000in}{-0.041667in}}%
\pgfusepath{stroke,fill}%
}%
\begin{pgfscope}%
\pgfsys@transformshift{0.689633in}{0.893003in}%
\pgfsys@useobject{currentmarker}{}%
\end{pgfscope}%
\end{pgfscope}%
\begin{pgfscope}%
\definecolor{textcolor}{rgb}{0.000000,0.000000,0.000000}%
\pgfsetstrokecolor{textcolor}%
\pgfsetfillcolor{textcolor}%
\pgftext[x=0.232106in, y=0.220556in, left, base,rotate=30.000000]{\color{textcolor}\rmfamily\fontsize{7.000000}{8.400000}\selectfont 2019-10-22}%
\end{pgfscope}%
\begin{pgfscope}%
\pgfsetbuttcap%
\pgfsetroundjoin%
\definecolor{currentfill}{rgb}{0.000000,0.000000,0.000000}%
\pgfsetfillcolor{currentfill}%
\pgfsetlinewidth{0.501875pt}%
\definecolor{currentstroke}{rgb}{0.000000,0.000000,0.000000}%
\pgfsetstrokecolor{currentstroke}%
\pgfsetdash{}{0pt}%
\pgfsys@defobject{currentmarker}{\pgfqpoint{0.000000in}{0.000000in}}{\pgfqpoint{0.000000in}{0.041667in}}{%
\pgfpathmoveto{\pgfqpoint{0.000000in}{0.000000in}}%
\pgfpathlineto{\pgfqpoint{0.000000in}{0.041667in}}%
\pgfusepath{stroke,fill}%
}%
\begin{pgfscope}%
\pgfsys@transformshift{1.118000in}{0.586309in}%
\pgfsys@useobject{currentmarker}{}%
\end{pgfscope}%
\end{pgfscope}%
\begin{pgfscope}%
\pgfsetbuttcap%
\pgfsetroundjoin%
\definecolor{currentfill}{rgb}{0.000000,0.000000,0.000000}%
\pgfsetfillcolor{currentfill}%
\pgfsetlinewidth{0.501875pt}%
\definecolor{currentstroke}{rgb}{0.000000,0.000000,0.000000}%
\pgfsetstrokecolor{currentstroke}%
\pgfsetdash{}{0pt}%
\pgfsys@defobject{currentmarker}{\pgfqpoint{0.000000in}{-0.041667in}}{\pgfqpoint{0.000000in}{0.000000in}}{%
\pgfpathmoveto{\pgfqpoint{0.000000in}{0.000000in}}%
\pgfpathlineto{\pgfqpoint{0.000000in}{-0.041667in}}%
\pgfusepath{stroke,fill}%
}%
\begin{pgfscope}%
\pgfsys@transformshift{1.118000in}{0.893003in}%
\pgfsys@useobject{currentmarker}{}%
\end{pgfscope}%
\end{pgfscope}%
\begin{pgfscope}%
\definecolor{textcolor}{rgb}{0.000000,0.000000,0.000000}%
\pgfsetstrokecolor{textcolor}%
\pgfsetfillcolor{textcolor}%
\pgftext[x=0.660473in, y=0.220556in, left, base,rotate=30.000000]{\color{textcolor}\rmfamily\fontsize{7.000000}{8.400000}\selectfont 2019-12-21}%
\end{pgfscope}%
\begin{pgfscope}%
\pgfsetbuttcap%
\pgfsetroundjoin%
\definecolor{currentfill}{rgb}{0.000000,0.000000,0.000000}%
\pgfsetfillcolor{currentfill}%
\pgfsetlinewidth{0.501875pt}%
\definecolor{currentstroke}{rgb}{0.000000,0.000000,0.000000}%
\pgfsetstrokecolor{currentstroke}%
\pgfsetdash{}{0pt}%
\pgfsys@defobject{currentmarker}{\pgfqpoint{0.000000in}{0.000000in}}{\pgfqpoint{0.000000in}{0.041667in}}{%
\pgfpathmoveto{\pgfqpoint{0.000000in}{0.000000in}}%
\pgfpathlineto{\pgfqpoint{0.000000in}{0.041667in}}%
\pgfusepath{stroke,fill}%
}%
\begin{pgfscope}%
\pgfsys@transformshift{1.546368in}{0.586309in}%
\pgfsys@useobject{currentmarker}{}%
\end{pgfscope}%
\end{pgfscope}%
\begin{pgfscope}%
\pgfsetbuttcap%
\pgfsetroundjoin%
\definecolor{currentfill}{rgb}{0.000000,0.000000,0.000000}%
\pgfsetfillcolor{currentfill}%
\pgfsetlinewidth{0.501875pt}%
\definecolor{currentstroke}{rgb}{0.000000,0.000000,0.000000}%
\pgfsetstrokecolor{currentstroke}%
\pgfsetdash{}{0pt}%
\pgfsys@defobject{currentmarker}{\pgfqpoint{0.000000in}{-0.041667in}}{\pgfqpoint{0.000000in}{0.000000in}}{%
\pgfpathmoveto{\pgfqpoint{0.000000in}{0.000000in}}%
\pgfpathlineto{\pgfqpoint{0.000000in}{-0.041667in}}%
\pgfusepath{stroke,fill}%
}%
\begin{pgfscope}%
\pgfsys@transformshift{1.546368in}{0.893003in}%
\pgfsys@useobject{currentmarker}{}%
\end{pgfscope}%
\end{pgfscope}%
\begin{pgfscope}%
\definecolor{textcolor}{rgb}{0.000000,0.000000,0.000000}%
\pgfsetstrokecolor{textcolor}%
\pgfsetfillcolor{textcolor}%
\pgftext[x=1.088841in, y=0.220556in, left, base,rotate=30.000000]{\color{textcolor}\rmfamily\fontsize{7.000000}{8.400000}\selectfont 2020-02-19}%
\end{pgfscope}%
\begin{pgfscope}%
\pgfsetbuttcap%
\pgfsetroundjoin%
\definecolor{currentfill}{rgb}{0.000000,0.000000,0.000000}%
\pgfsetfillcolor{currentfill}%
\pgfsetlinewidth{0.501875pt}%
\definecolor{currentstroke}{rgb}{0.000000,0.000000,0.000000}%
\pgfsetstrokecolor{currentstroke}%
\pgfsetdash{}{0pt}%
\pgfsys@defobject{currentmarker}{\pgfqpoint{0.000000in}{0.000000in}}{\pgfqpoint{0.000000in}{0.041667in}}{%
\pgfpathmoveto{\pgfqpoint{0.000000in}{0.000000in}}%
\pgfpathlineto{\pgfqpoint{0.000000in}{0.041667in}}%
\pgfusepath{stroke,fill}%
}%
\begin{pgfscope}%
\pgfsys@transformshift{1.974736in}{0.586309in}%
\pgfsys@useobject{currentmarker}{}%
\end{pgfscope}%
\end{pgfscope}%
\begin{pgfscope}%
\pgfsetbuttcap%
\pgfsetroundjoin%
\definecolor{currentfill}{rgb}{0.000000,0.000000,0.000000}%
\pgfsetfillcolor{currentfill}%
\pgfsetlinewidth{0.501875pt}%
\definecolor{currentstroke}{rgb}{0.000000,0.000000,0.000000}%
\pgfsetstrokecolor{currentstroke}%
\pgfsetdash{}{0pt}%
\pgfsys@defobject{currentmarker}{\pgfqpoint{0.000000in}{-0.041667in}}{\pgfqpoint{0.000000in}{0.000000in}}{%
\pgfpathmoveto{\pgfqpoint{0.000000in}{0.000000in}}%
\pgfpathlineto{\pgfqpoint{0.000000in}{-0.041667in}}%
\pgfusepath{stroke,fill}%
}%
\begin{pgfscope}%
\pgfsys@transformshift{1.974736in}{0.893003in}%
\pgfsys@useobject{currentmarker}{}%
\end{pgfscope}%
\end{pgfscope}%
\begin{pgfscope}%
\definecolor{textcolor}{rgb}{0.000000,0.000000,0.000000}%
\pgfsetstrokecolor{textcolor}%
\pgfsetfillcolor{textcolor}%
\pgftext[x=1.517209in, y=0.220556in, left, base,rotate=30.000000]{\color{textcolor}\rmfamily\fontsize{7.000000}{8.400000}\selectfont 2020-04-19}%
\end{pgfscope}%
\begin{pgfscope}%
\pgfsetbuttcap%
\pgfsetroundjoin%
\definecolor{currentfill}{rgb}{0.000000,0.000000,0.000000}%
\pgfsetfillcolor{currentfill}%
\pgfsetlinewidth{0.501875pt}%
\definecolor{currentstroke}{rgb}{0.000000,0.000000,0.000000}%
\pgfsetstrokecolor{currentstroke}%
\pgfsetdash{}{0pt}%
\pgfsys@defobject{currentmarker}{\pgfqpoint{0.000000in}{0.000000in}}{\pgfqpoint{0.000000in}{0.041667in}}{%
\pgfpathmoveto{\pgfqpoint{0.000000in}{0.000000in}}%
\pgfpathlineto{\pgfqpoint{0.000000in}{0.041667in}}%
\pgfusepath{stroke,fill}%
}%
\begin{pgfscope}%
\pgfsys@transformshift{2.403104in}{0.586309in}%
\pgfsys@useobject{currentmarker}{}%
\end{pgfscope}%
\end{pgfscope}%
\begin{pgfscope}%
\pgfsetbuttcap%
\pgfsetroundjoin%
\definecolor{currentfill}{rgb}{0.000000,0.000000,0.000000}%
\pgfsetfillcolor{currentfill}%
\pgfsetlinewidth{0.501875pt}%
\definecolor{currentstroke}{rgb}{0.000000,0.000000,0.000000}%
\pgfsetstrokecolor{currentstroke}%
\pgfsetdash{}{0pt}%
\pgfsys@defobject{currentmarker}{\pgfqpoint{0.000000in}{-0.041667in}}{\pgfqpoint{0.000000in}{0.000000in}}{%
\pgfpathmoveto{\pgfqpoint{0.000000in}{0.000000in}}%
\pgfpathlineto{\pgfqpoint{0.000000in}{-0.041667in}}%
\pgfusepath{stroke,fill}%
}%
\begin{pgfscope}%
\pgfsys@transformshift{2.403104in}{0.893003in}%
\pgfsys@useobject{currentmarker}{}%
\end{pgfscope}%
\end{pgfscope}%
\begin{pgfscope}%
\definecolor{textcolor}{rgb}{0.000000,0.000000,0.000000}%
\pgfsetstrokecolor{textcolor}%
\pgfsetfillcolor{textcolor}%
\pgftext[x=1.945577in, y=0.220556in, left, base,rotate=30.000000]{\color{textcolor}\rmfamily\fontsize{7.000000}{8.400000}\selectfont 2020-06-18}%
\end{pgfscope}%
\begin{pgfscope}%
\pgfsetbuttcap%
\pgfsetroundjoin%
\definecolor{currentfill}{rgb}{0.000000,0.000000,0.000000}%
\pgfsetfillcolor{currentfill}%
\pgfsetlinewidth{0.501875pt}%
\definecolor{currentstroke}{rgb}{0.000000,0.000000,0.000000}%
\pgfsetstrokecolor{currentstroke}%
\pgfsetdash{}{0pt}%
\pgfsys@defobject{currentmarker}{\pgfqpoint{0.000000in}{0.000000in}}{\pgfqpoint{0.000000in}{0.041667in}}{%
\pgfpathmoveto{\pgfqpoint{0.000000in}{0.000000in}}%
\pgfpathlineto{\pgfqpoint{0.000000in}{0.041667in}}%
\pgfusepath{stroke,fill}%
}%
\begin{pgfscope}%
\pgfsys@transformshift{2.831472in}{0.586309in}%
\pgfsys@useobject{currentmarker}{}%
\end{pgfscope}%
\end{pgfscope}%
\begin{pgfscope}%
\pgfsetbuttcap%
\pgfsetroundjoin%
\definecolor{currentfill}{rgb}{0.000000,0.000000,0.000000}%
\pgfsetfillcolor{currentfill}%
\pgfsetlinewidth{0.501875pt}%
\definecolor{currentstroke}{rgb}{0.000000,0.000000,0.000000}%
\pgfsetstrokecolor{currentstroke}%
\pgfsetdash{}{0pt}%
\pgfsys@defobject{currentmarker}{\pgfqpoint{0.000000in}{-0.041667in}}{\pgfqpoint{0.000000in}{0.000000in}}{%
\pgfpathmoveto{\pgfqpoint{0.000000in}{0.000000in}}%
\pgfpathlineto{\pgfqpoint{0.000000in}{-0.041667in}}%
\pgfusepath{stroke,fill}%
}%
\begin{pgfscope}%
\pgfsys@transformshift{2.831472in}{0.893003in}%
\pgfsys@useobject{currentmarker}{}%
\end{pgfscope}%
\end{pgfscope}%
\begin{pgfscope}%
\definecolor{textcolor}{rgb}{0.000000,0.000000,0.000000}%
\pgfsetstrokecolor{textcolor}%
\pgfsetfillcolor{textcolor}%
\pgftext[x=2.373945in, y=0.220556in, left, base,rotate=30.000000]{\color{textcolor}\rmfamily\fontsize{7.000000}{8.400000}\selectfont 2020-08-17}%
\end{pgfscope}%
\begin{pgfscope}%
\pgfsetbuttcap%
\pgfsetroundjoin%
\definecolor{currentfill}{rgb}{0.000000,0.000000,0.000000}%
\pgfsetfillcolor{currentfill}%
\pgfsetlinewidth{0.501875pt}%
\definecolor{currentstroke}{rgb}{0.000000,0.000000,0.000000}%
\pgfsetstrokecolor{currentstroke}%
\pgfsetdash{}{0pt}%
\pgfsys@defobject{currentmarker}{\pgfqpoint{0.000000in}{0.000000in}}{\pgfqpoint{0.000000in}{0.041667in}}{%
\pgfpathmoveto{\pgfqpoint{0.000000in}{0.000000in}}%
\pgfpathlineto{\pgfqpoint{0.000000in}{0.041667in}}%
\pgfusepath{stroke,fill}%
}%
\begin{pgfscope}%
\pgfsys@transformshift{3.259839in}{0.586309in}%
\pgfsys@useobject{currentmarker}{}%
\end{pgfscope}%
\end{pgfscope}%
\begin{pgfscope}%
\pgfsetbuttcap%
\pgfsetroundjoin%
\definecolor{currentfill}{rgb}{0.000000,0.000000,0.000000}%
\pgfsetfillcolor{currentfill}%
\pgfsetlinewidth{0.501875pt}%
\definecolor{currentstroke}{rgb}{0.000000,0.000000,0.000000}%
\pgfsetstrokecolor{currentstroke}%
\pgfsetdash{}{0pt}%
\pgfsys@defobject{currentmarker}{\pgfqpoint{0.000000in}{-0.041667in}}{\pgfqpoint{0.000000in}{0.000000in}}{%
\pgfpathmoveto{\pgfqpoint{0.000000in}{0.000000in}}%
\pgfpathlineto{\pgfqpoint{0.000000in}{-0.041667in}}%
\pgfusepath{stroke,fill}%
}%
\begin{pgfscope}%
\pgfsys@transformshift{3.259839in}{0.893003in}%
\pgfsys@useobject{currentmarker}{}%
\end{pgfscope}%
\end{pgfscope}%
\begin{pgfscope}%
\definecolor{textcolor}{rgb}{0.000000,0.000000,0.000000}%
\pgfsetstrokecolor{textcolor}%
\pgfsetfillcolor{textcolor}%
\pgftext[x=2.802312in, y=0.220556in, left, base,rotate=30.000000]{\color{textcolor}\rmfamily\fontsize{7.000000}{8.400000}\selectfont 2020-10-16}%
\end{pgfscope}%
\begin{pgfscope}%
\pgfsetbuttcap%
\pgfsetroundjoin%
\definecolor{currentfill}{rgb}{0.000000,0.000000,0.000000}%
\pgfsetfillcolor{currentfill}%
\pgfsetlinewidth{0.501875pt}%
\definecolor{currentstroke}{rgb}{0.000000,0.000000,0.000000}%
\pgfsetstrokecolor{currentstroke}%
\pgfsetdash{}{0pt}%
\pgfsys@defobject{currentmarker}{\pgfqpoint{0.000000in}{0.000000in}}{\pgfqpoint{0.000000in}{0.041667in}}{%
\pgfpathmoveto{\pgfqpoint{0.000000in}{0.000000in}}%
\pgfpathlineto{\pgfqpoint{0.000000in}{0.041667in}}%
\pgfusepath{stroke,fill}%
}%
\begin{pgfscope}%
\pgfsys@transformshift{3.688207in}{0.586309in}%
\pgfsys@useobject{currentmarker}{}%
\end{pgfscope}%
\end{pgfscope}%
\begin{pgfscope}%
\pgfsetbuttcap%
\pgfsetroundjoin%
\definecolor{currentfill}{rgb}{0.000000,0.000000,0.000000}%
\pgfsetfillcolor{currentfill}%
\pgfsetlinewidth{0.501875pt}%
\definecolor{currentstroke}{rgb}{0.000000,0.000000,0.000000}%
\pgfsetstrokecolor{currentstroke}%
\pgfsetdash{}{0pt}%
\pgfsys@defobject{currentmarker}{\pgfqpoint{0.000000in}{-0.041667in}}{\pgfqpoint{0.000000in}{0.000000in}}{%
\pgfpathmoveto{\pgfqpoint{0.000000in}{0.000000in}}%
\pgfpathlineto{\pgfqpoint{0.000000in}{-0.041667in}}%
\pgfusepath{stroke,fill}%
}%
\begin{pgfscope}%
\pgfsys@transformshift{3.688207in}{0.893003in}%
\pgfsys@useobject{currentmarker}{}%
\end{pgfscope}%
\end{pgfscope}%
\begin{pgfscope}%
\definecolor{textcolor}{rgb}{0.000000,0.000000,0.000000}%
\pgfsetstrokecolor{textcolor}%
\pgfsetfillcolor{textcolor}%
\pgftext[x=3.230680in, y=0.220556in, left, base,rotate=30.000000]{\color{textcolor}\rmfamily\fontsize{7.000000}{8.400000}\selectfont 2020-12-15}%
\end{pgfscope}%
\begin{pgfscope}%
\pgfsetbuttcap%
\pgfsetroundjoin%
\definecolor{currentfill}{rgb}{0.000000,0.000000,0.000000}%
\pgfsetfillcolor{currentfill}%
\pgfsetlinewidth{0.501875pt}%
\definecolor{currentstroke}{rgb}{0.000000,0.000000,0.000000}%
\pgfsetstrokecolor{currentstroke}%
\pgfsetdash{}{0pt}%
\pgfsys@defobject{currentmarker}{\pgfqpoint{0.000000in}{0.000000in}}{\pgfqpoint{0.000000in}{0.041667in}}{%
\pgfpathmoveto{\pgfqpoint{0.000000in}{0.000000in}}%
\pgfpathlineto{\pgfqpoint{0.000000in}{0.041667in}}%
\pgfusepath{stroke,fill}%
}%
\begin{pgfscope}%
\pgfsys@transformshift{4.116575in}{0.586309in}%
\pgfsys@useobject{currentmarker}{}%
\end{pgfscope}%
\end{pgfscope}%
\begin{pgfscope}%
\pgfsetbuttcap%
\pgfsetroundjoin%
\definecolor{currentfill}{rgb}{0.000000,0.000000,0.000000}%
\pgfsetfillcolor{currentfill}%
\pgfsetlinewidth{0.501875pt}%
\definecolor{currentstroke}{rgb}{0.000000,0.000000,0.000000}%
\pgfsetstrokecolor{currentstroke}%
\pgfsetdash{}{0pt}%
\pgfsys@defobject{currentmarker}{\pgfqpoint{0.000000in}{-0.041667in}}{\pgfqpoint{0.000000in}{0.000000in}}{%
\pgfpathmoveto{\pgfqpoint{0.000000in}{0.000000in}}%
\pgfpathlineto{\pgfqpoint{0.000000in}{-0.041667in}}%
\pgfusepath{stroke,fill}%
}%
\begin{pgfscope}%
\pgfsys@transformshift{4.116575in}{0.893003in}%
\pgfsys@useobject{currentmarker}{}%
\end{pgfscope}%
\end{pgfscope}%
\begin{pgfscope}%
\definecolor{textcolor}{rgb}{0.000000,0.000000,0.000000}%
\pgfsetstrokecolor{textcolor}%
\pgfsetfillcolor{textcolor}%
\pgftext[x=3.659048in, y=0.220556in, left, base,rotate=30.000000]{\color{textcolor}\rmfamily\fontsize{7.000000}{8.400000}\selectfont 2021-02-13}%
\end{pgfscope}%
\begin{pgfscope}%
\pgfsetbuttcap%
\pgfsetroundjoin%
\definecolor{currentfill}{rgb}{0.000000,0.000000,0.000000}%
\pgfsetfillcolor{currentfill}%
\pgfsetlinewidth{0.501875pt}%
\definecolor{currentstroke}{rgb}{0.000000,0.000000,0.000000}%
\pgfsetstrokecolor{currentstroke}%
\pgfsetdash{}{0pt}%
\pgfsys@defobject{currentmarker}{\pgfqpoint{0.000000in}{0.000000in}}{\pgfqpoint{0.000000in}{0.041667in}}{%
\pgfpathmoveto{\pgfqpoint{0.000000in}{0.000000in}}%
\pgfpathlineto{\pgfqpoint{0.000000in}{0.041667in}}%
\pgfusepath{stroke,fill}%
}%
\begin{pgfscope}%
\pgfsys@transformshift{4.544943in}{0.586309in}%
\pgfsys@useobject{currentmarker}{}%
\end{pgfscope}%
\end{pgfscope}%
\begin{pgfscope}%
\pgfsetbuttcap%
\pgfsetroundjoin%
\definecolor{currentfill}{rgb}{0.000000,0.000000,0.000000}%
\pgfsetfillcolor{currentfill}%
\pgfsetlinewidth{0.501875pt}%
\definecolor{currentstroke}{rgb}{0.000000,0.000000,0.000000}%
\pgfsetstrokecolor{currentstroke}%
\pgfsetdash{}{0pt}%
\pgfsys@defobject{currentmarker}{\pgfqpoint{0.000000in}{-0.041667in}}{\pgfqpoint{0.000000in}{0.000000in}}{%
\pgfpathmoveto{\pgfqpoint{0.000000in}{0.000000in}}%
\pgfpathlineto{\pgfqpoint{0.000000in}{-0.041667in}}%
\pgfusepath{stroke,fill}%
}%
\begin{pgfscope}%
\pgfsys@transformshift{4.544943in}{0.893003in}%
\pgfsys@useobject{currentmarker}{}%
\end{pgfscope}%
\end{pgfscope}%
\begin{pgfscope}%
\definecolor{textcolor}{rgb}{0.000000,0.000000,0.000000}%
\pgfsetstrokecolor{textcolor}%
\pgfsetfillcolor{textcolor}%
\pgftext[x=4.087416in, y=0.220556in, left, base,rotate=30.000000]{\color{textcolor}\rmfamily\fontsize{7.000000}{8.400000}\selectfont 2021-04-14}%
\end{pgfscope}%
\begin{pgfscope}%
\pgfsetbuttcap%
\pgfsetroundjoin%
\definecolor{currentfill}{rgb}{0.000000,0.000000,0.000000}%
\pgfsetfillcolor{currentfill}%
\pgfsetlinewidth{0.501875pt}%
\definecolor{currentstroke}{rgb}{0.000000,0.000000,0.000000}%
\pgfsetstrokecolor{currentstroke}%
\pgfsetdash{}{0pt}%
\pgfsys@defobject{currentmarker}{\pgfqpoint{0.000000in}{0.000000in}}{\pgfqpoint{0.000000in}{0.041667in}}{%
\pgfpathmoveto{\pgfqpoint{0.000000in}{0.000000in}}%
\pgfpathlineto{\pgfqpoint{0.000000in}{0.041667in}}%
\pgfusepath{stroke,fill}%
}%
\begin{pgfscope}%
\pgfsys@transformshift{4.973310in}{0.586309in}%
\pgfsys@useobject{currentmarker}{}%
\end{pgfscope}%
\end{pgfscope}%
\begin{pgfscope}%
\pgfsetbuttcap%
\pgfsetroundjoin%
\definecolor{currentfill}{rgb}{0.000000,0.000000,0.000000}%
\pgfsetfillcolor{currentfill}%
\pgfsetlinewidth{0.501875pt}%
\definecolor{currentstroke}{rgb}{0.000000,0.000000,0.000000}%
\pgfsetstrokecolor{currentstroke}%
\pgfsetdash{}{0pt}%
\pgfsys@defobject{currentmarker}{\pgfqpoint{0.000000in}{-0.041667in}}{\pgfqpoint{0.000000in}{0.000000in}}{%
\pgfpathmoveto{\pgfqpoint{0.000000in}{0.000000in}}%
\pgfpathlineto{\pgfqpoint{0.000000in}{-0.041667in}}%
\pgfusepath{stroke,fill}%
}%
\begin{pgfscope}%
\pgfsys@transformshift{4.973310in}{0.893003in}%
\pgfsys@useobject{currentmarker}{}%
\end{pgfscope}%
\end{pgfscope}%
\begin{pgfscope}%
\definecolor{textcolor}{rgb}{0.000000,0.000000,0.000000}%
\pgfsetstrokecolor{textcolor}%
\pgfsetfillcolor{textcolor}%
\pgftext[x=4.515784in, y=0.220556in, left, base,rotate=30.000000]{\color{textcolor}\rmfamily\fontsize{7.000000}{8.400000}\selectfont 2021-06-13}%
\end{pgfscope}%
\begin{pgfscope}%
\pgfsetbuttcap%
\pgfsetroundjoin%
\definecolor{currentfill}{rgb}{0.000000,0.000000,0.000000}%
\pgfsetfillcolor{currentfill}%
\pgfsetlinewidth{0.501875pt}%
\definecolor{currentstroke}{rgb}{0.000000,0.000000,0.000000}%
\pgfsetstrokecolor{currentstroke}%
\pgfsetdash{}{0pt}%
\pgfsys@defobject{currentmarker}{\pgfqpoint{0.000000in}{0.000000in}}{\pgfqpoint{0.000000in}{0.041667in}}{%
\pgfpathmoveto{\pgfqpoint{0.000000in}{0.000000in}}%
\pgfpathlineto{\pgfqpoint{0.000000in}{0.041667in}}%
\pgfusepath{stroke,fill}%
}%
\begin{pgfscope}%
\pgfsys@transformshift{5.401678in}{0.586309in}%
\pgfsys@useobject{currentmarker}{}%
\end{pgfscope}%
\end{pgfscope}%
\begin{pgfscope}%
\pgfsetbuttcap%
\pgfsetroundjoin%
\definecolor{currentfill}{rgb}{0.000000,0.000000,0.000000}%
\pgfsetfillcolor{currentfill}%
\pgfsetlinewidth{0.501875pt}%
\definecolor{currentstroke}{rgb}{0.000000,0.000000,0.000000}%
\pgfsetstrokecolor{currentstroke}%
\pgfsetdash{}{0pt}%
\pgfsys@defobject{currentmarker}{\pgfqpoint{0.000000in}{-0.041667in}}{\pgfqpoint{0.000000in}{0.000000in}}{%
\pgfpathmoveto{\pgfqpoint{0.000000in}{0.000000in}}%
\pgfpathlineto{\pgfqpoint{0.000000in}{-0.041667in}}%
\pgfusepath{stroke,fill}%
}%
\begin{pgfscope}%
\pgfsys@transformshift{5.401678in}{0.893003in}%
\pgfsys@useobject{currentmarker}{}%
\end{pgfscope}%
\end{pgfscope}%
\begin{pgfscope}%
\definecolor{textcolor}{rgb}{0.000000,0.000000,0.000000}%
\pgfsetstrokecolor{textcolor}%
\pgfsetfillcolor{textcolor}%
\pgftext[x=4.944151in, y=0.220556in, left, base,rotate=30.000000]{\color{textcolor}\rmfamily\fontsize{7.000000}{8.400000}\selectfont 2021-08-12}%
\end{pgfscope}%
\begin{pgfscope}%
\pgfsetbuttcap%
\pgfsetroundjoin%
\definecolor{currentfill}{rgb}{0.000000,0.000000,0.000000}%
\pgfsetfillcolor{currentfill}%
\pgfsetlinewidth{0.501875pt}%
\definecolor{currentstroke}{rgb}{0.000000,0.000000,0.000000}%
\pgfsetstrokecolor{currentstroke}%
\pgfsetdash{}{0pt}%
\pgfsys@defobject{currentmarker}{\pgfqpoint{0.000000in}{0.000000in}}{\pgfqpoint{0.000000in}{0.041667in}}{%
\pgfpathmoveto{\pgfqpoint{0.000000in}{0.000000in}}%
\pgfpathlineto{\pgfqpoint{0.000000in}{0.041667in}}%
\pgfusepath{stroke,fill}%
}%
\begin{pgfscope}%
\pgfsys@transformshift{5.830046in}{0.586309in}%
\pgfsys@useobject{currentmarker}{}%
\end{pgfscope}%
\end{pgfscope}%
\begin{pgfscope}%
\pgfsetbuttcap%
\pgfsetroundjoin%
\definecolor{currentfill}{rgb}{0.000000,0.000000,0.000000}%
\pgfsetfillcolor{currentfill}%
\pgfsetlinewidth{0.501875pt}%
\definecolor{currentstroke}{rgb}{0.000000,0.000000,0.000000}%
\pgfsetstrokecolor{currentstroke}%
\pgfsetdash{}{0pt}%
\pgfsys@defobject{currentmarker}{\pgfqpoint{0.000000in}{-0.041667in}}{\pgfqpoint{0.000000in}{0.000000in}}{%
\pgfpathmoveto{\pgfqpoint{0.000000in}{0.000000in}}%
\pgfpathlineto{\pgfqpoint{0.000000in}{-0.041667in}}%
\pgfusepath{stroke,fill}%
}%
\begin{pgfscope}%
\pgfsys@transformshift{5.830046in}{0.893003in}%
\pgfsys@useobject{currentmarker}{}%
\end{pgfscope}%
\end{pgfscope}%
\begin{pgfscope}%
\definecolor{textcolor}{rgb}{0.000000,0.000000,0.000000}%
\pgfsetstrokecolor{textcolor}%
\pgfsetfillcolor{textcolor}%
\pgftext[x=5.372519in, y=0.220556in, left, base,rotate=30.000000]{\color{textcolor}\rmfamily\fontsize{7.000000}{8.400000}\selectfont 2021-10-11}%
\end{pgfscope}%
\begin{pgfscope}%
\pgfsetbuttcap%
\pgfsetroundjoin%
\definecolor{currentfill}{rgb}{0.000000,0.000000,0.000000}%
\pgfsetfillcolor{currentfill}%
\pgfsetlinewidth{0.501875pt}%
\definecolor{currentstroke}{rgb}{0.000000,0.000000,0.000000}%
\pgfsetstrokecolor{currentstroke}%
\pgfsetdash{}{0pt}%
\pgfsys@defobject{currentmarker}{\pgfqpoint{0.000000in}{0.000000in}}{\pgfqpoint{0.000000in}{0.041667in}}{%
\pgfpathmoveto{\pgfqpoint{0.000000in}{0.000000in}}%
\pgfpathlineto{\pgfqpoint{0.000000in}{0.041667in}}%
\pgfusepath{stroke,fill}%
}%
\begin{pgfscope}%
\pgfsys@transformshift{6.258414in}{0.586309in}%
\pgfsys@useobject{currentmarker}{}%
\end{pgfscope}%
\end{pgfscope}%
\begin{pgfscope}%
\pgfsetbuttcap%
\pgfsetroundjoin%
\definecolor{currentfill}{rgb}{0.000000,0.000000,0.000000}%
\pgfsetfillcolor{currentfill}%
\pgfsetlinewidth{0.501875pt}%
\definecolor{currentstroke}{rgb}{0.000000,0.000000,0.000000}%
\pgfsetstrokecolor{currentstroke}%
\pgfsetdash{}{0pt}%
\pgfsys@defobject{currentmarker}{\pgfqpoint{0.000000in}{-0.041667in}}{\pgfqpoint{0.000000in}{0.000000in}}{%
\pgfpathmoveto{\pgfqpoint{0.000000in}{0.000000in}}%
\pgfpathlineto{\pgfqpoint{0.000000in}{-0.041667in}}%
\pgfusepath{stroke,fill}%
}%
\begin{pgfscope}%
\pgfsys@transformshift{6.258414in}{0.893003in}%
\pgfsys@useobject{currentmarker}{}%
\end{pgfscope}%
\end{pgfscope}%
\begin{pgfscope}%
\definecolor{textcolor}{rgb}{0.000000,0.000000,0.000000}%
\pgfsetstrokecolor{textcolor}%
\pgfsetfillcolor{textcolor}%
\pgftext[x=5.800887in, y=0.220556in, left, base,rotate=30.000000]{\color{textcolor}\rmfamily\fontsize{7.000000}{8.400000}\selectfont 2021-12-10}%
\end{pgfscope}%
\begin{pgfscope}%
\pgfsetbuttcap%
\pgfsetroundjoin%
\definecolor{currentfill}{rgb}{0.000000,0.000000,0.000000}%
\pgfsetfillcolor{currentfill}%
\pgfsetlinewidth{0.501875pt}%
\definecolor{currentstroke}{rgb}{0.000000,0.000000,0.000000}%
\pgfsetstrokecolor{currentstroke}%
\pgfsetdash{}{0pt}%
\pgfsys@defobject{currentmarker}{\pgfqpoint{0.000000in}{0.000000in}}{\pgfqpoint{0.000000in}{0.020833in}}{%
\pgfpathmoveto{\pgfqpoint{0.000000in}{0.000000in}}%
\pgfpathlineto{\pgfqpoint{0.000000in}{0.020833in}}%
\pgfusepath{stroke,fill}%
}%
\begin{pgfscope}%
\pgfsys@transformshift{0.511146in}{0.586309in}%
\pgfsys@useobject{currentmarker}{}%
\end{pgfscope}%
\end{pgfscope}%
\begin{pgfscope}%
\pgfsetbuttcap%
\pgfsetroundjoin%
\definecolor{currentfill}{rgb}{0.000000,0.000000,0.000000}%
\pgfsetfillcolor{currentfill}%
\pgfsetlinewidth{0.501875pt}%
\definecolor{currentstroke}{rgb}{0.000000,0.000000,0.000000}%
\pgfsetstrokecolor{currentstroke}%
\pgfsetdash{}{0pt}%
\pgfsys@defobject{currentmarker}{\pgfqpoint{0.000000in}{-0.020833in}}{\pgfqpoint{0.000000in}{0.000000in}}{%
\pgfpathmoveto{\pgfqpoint{0.000000in}{0.000000in}}%
\pgfpathlineto{\pgfqpoint{0.000000in}{-0.020833in}}%
\pgfusepath{stroke,fill}%
}%
\begin{pgfscope}%
\pgfsys@transformshift{0.511146in}{0.893003in}%
\pgfsys@useobject{currentmarker}{}%
\end{pgfscope}%
\end{pgfscope}%
\begin{pgfscope}%
\pgfsetbuttcap%
\pgfsetroundjoin%
\definecolor{currentfill}{rgb}{0.000000,0.000000,0.000000}%
\pgfsetfillcolor{currentfill}%
\pgfsetlinewidth{0.501875pt}%
\definecolor{currentstroke}{rgb}{0.000000,0.000000,0.000000}%
\pgfsetstrokecolor{currentstroke}%
\pgfsetdash{}{0pt}%
\pgfsys@defobject{currentmarker}{\pgfqpoint{0.000000in}{0.000000in}}{\pgfqpoint{0.000000in}{0.020833in}}{%
\pgfpathmoveto{\pgfqpoint{0.000000in}{0.000000in}}%
\pgfpathlineto{\pgfqpoint{0.000000in}{0.020833in}}%
\pgfusepath{stroke,fill}%
}%
\begin{pgfscope}%
\pgfsys@transformshift{0.546843in}{0.586309in}%
\pgfsys@useobject{currentmarker}{}%
\end{pgfscope}%
\end{pgfscope}%
\begin{pgfscope}%
\pgfsetbuttcap%
\pgfsetroundjoin%
\definecolor{currentfill}{rgb}{0.000000,0.000000,0.000000}%
\pgfsetfillcolor{currentfill}%
\pgfsetlinewidth{0.501875pt}%
\definecolor{currentstroke}{rgb}{0.000000,0.000000,0.000000}%
\pgfsetstrokecolor{currentstroke}%
\pgfsetdash{}{0pt}%
\pgfsys@defobject{currentmarker}{\pgfqpoint{0.000000in}{-0.020833in}}{\pgfqpoint{0.000000in}{0.000000in}}{%
\pgfpathmoveto{\pgfqpoint{0.000000in}{0.000000in}}%
\pgfpathlineto{\pgfqpoint{0.000000in}{-0.020833in}}%
\pgfusepath{stroke,fill}%
}%
\begin{pgfscope}%
\pgfsys@transformshift{0.546843in}{0.893003in}%
\pgfsys@useobject{currentmarker}{}%
\end{pgfscope}%
\end{pgfscope}%
\begin{pgfscope}%
\pgfsetbuttcap%
\pgfsetroundjoin%
\definecolor{currentfill}{rgb}{0.000000,0.000000,0.000000}%
\pgfsetfillcolor{currentfill}%
\pgfsetlinewidth{0.501875pt}%
\definecolor{currentstroke}{rgb}{0.000000,0.000000,0.000000}%
\pgfsetstrokecolor{currentstroke}%
\pgfsetdash{}{0pt}%
\pgfsys@defobject{currentmarker}{\pgfqpoint{0.000000in}{0.000000in}}{\pgfqpoint{0.000000in}{0.020833in}}{%
\pgfpathmoveto{\pgfqpoint{0.000000in}{0.000000in}}%
\pgfpathlineto{\pgfqpoint{0.000000in}{0.020833in}}%
\pgfusepath{stroke,fill}%
}%
\begin{pgfscope}%
\pgfsys@transformshift{0.582541in}{0.586309in}%
\pgfsys@useobject{currentmarker}{}%
\end{pgfscope}%
\end{pgfscope}%
\begin{pgfscope}%
\pgfsetbuttcap%
\pgfsetroundjoin%
\definecolor{currentfill}{rgb}{0.000000,0.000000,0.000000}%
\pgfsetfillcolor{currentfill}%
\pgfsetlinewidth{0.501875pt}%
\definecolor{currentstroke}{rgb}{0.000000,0.000000,0.000000}%
\pgfsetstrokecolor{currentstroke}%
\pgfsetdash{}{0pt}%
\pgfsys@defobject{currentmarker}{\pgfqpoint{0.000000in}{-0.020833in}}{\pgfqpoint{0.000000in}{0.000000in}}{%
\pgfpathmoveto{\pgfqpoint{0.000000in}{0.000000in}}%
\pgfpathlineto{\pgfqpoint{0.000000in}{-0.020833in}}%
\pgfusepath{stroke,fill}%
}%
\begin{pgfscope}%
\pgfsys@transformshift{0.582541in}{0.893003in}%
\pgfsys@useobject{currentmarker}{}%
\end{pgfscope}%
\end{pgfscope}%
\begin{pgfscope}%
\pgfsetbuttcap%
\pgfsetroundjoin%
\definecolor{currentfill}{rgb}{0.000000,0.000000,0.000000}%
\pgfsetfillcolor{currentfill}%
\pgfsetlinewidth{0.501875pt}%
\definecolor{currentstroke}{rgb}{0.000000,0.000000,0.000000}%
\pgfsetstrokecolor{currentstroke}%
\pgfsetdash{}{0pt}%
\pgfsys@defobject{currentmarker}{\pgfqpoint{0.000000in}{0.000000in}}{\pgfqpoint{0.000000in}{0.020833in}}{%
\pgfpathmoveto{\pgfqpoint{0.000000in}{0.000000in}}%
\pgfpathlineto{\pgfqpoint{0.000000in}{0.020833in}}%
\pgfusepath{stroke,fill}%
}%
\begin{pgfscope}%
\pgfsys@transformshift{0.618238in}{0.586309in}%
\pgfsys@useobject{currentmarker}{}%
\end{pgfscope}%
\end{pgfscope}%
\begin{pgfscope}%
\pgfsetbuttcap%
\pgfsetroundjoin%
\definecolor{currentfill}{rgb}{0.000000,0.000000,0.000000}%
\pgfsetfillcolor{currentfill}%
\pgfsetlinewidth{0.501875pt}%
\definecolor{currentstroke}{rgb}{0.000000,0.000000,0.000000}%
\pgfsetstrokecolor{currentstroke}%
\pgfsetdash{}{0pt}%
\pgfsys@defobject{currentmarker}{\pgfqpoint{0.000000in}{-0.020833in}}{\pgfqpoint{0.000000in}{0.000000in}}{%
\pgfpathmoveto{\pgfqpoint{0.000000in}{0.000000in}}%
\pgfpathlineto{\pgfqpoint{0.000000in}{-0.020833in}}%
\pgfusepath{stroke,fill}%
}%
\begin{pgfscope}%
\pgfsys@transformshift{0.618238in}{0.893003in}%
\pgfsys@useobject{currentmarker}{}%
\end{pgfscope}%
\end{pgfscope}%
\begin{pgfscope}%
\pgfsetbuttcap%
\pgfsetroundjoin%
\definecolor{currentfill}{rgb}{0.000000,0.000000,0.000000}%
\pgfsetfillcolor{currentfill}%
\pgfsetlinewidth{0.501875pt}%
\definecolor{currentstroke}{rgb}{0.000000,0.000000,0.000000}%
\pgfsetstrokecolor{currentstroke}%
\pgfsetdash{}{0pt}%
\pgfsys@defobject{currentmarker}{\pgfqpoint{0.000000in}{0.000000in}}{\pgfqpoint{0.000000in}{0.020833in}}{%
\pgfpathmoveto{\pgfqpoint{0.000000in}{0.000000in}}%
\pgfpathlineto{\pgfqpoint{0.000000in}{0.020833in}}%
\pgfusepath{stroke,fill}%
}%
\begin{pgfscope}%
\pgfsys@transformshift{0.653935in}{0.586309in}%
\pgfsys@useobject{currentmarker}{}%
\end{pgfscope}%
\end{pgfscope}%
\begin{pgfscope}%
\pgfsetbuttcap%
\pgfsetroundjoin%
\definecolor{currentfill}{rgb}{0.000000,0.000000,0.000000}%
\pgfsetfillcolor{currentfill}%
\pgfsetlinewidth{0.501875pt}%
\definecolor{currentstroke}{rgb}{0.000000,0.000000,0.000000}%
\pgfsetstrokecolor{currentstroke}%
\pgfsetdash{}{0pt}%
\pgfsys@defobject{currentmarker}{\pgfqpoint{0.000000in}{-0.020833in}}{\pgfqpoint{0.000000in}{0.000000in}}{%
\pgfpathmoveto{\pgfqpoint{0.000000in}{0.000000in}}%
\pgfpathlineto{\pgfqpoint{0.000000in}{-0.020833in}}%
\pgfusepath{stroke,fill}%
}%
\begin{pgfscope}%
\pgfsys@transformshift{0.653935in}{0.893003in}%
\pgfsys@useobject{currentmarker}{}%
\end{pgfscope}%
\end{pgfscope}%
\begin{pgfscope}%
\pgfsetbuttcap%
\pgfsetroundjoin%
\definecolor{currentfill}{rgb}{0.000000,0.000000,0.000000}%
\pgfsetfillcolor{currentfill}%
\pgfsetlinewidth{0.501875pt}%
\definecolor{currentstroke}{rgb}{0.000000,0.000000,0.000000}%
\pgfsetstrokecolor{currentstroke}%
\pgfsetdash{}{0pt}%
\pgfsys@defobject{currentmarker}{\pgfqpoint{0.000000in}{0.000000in}}{\pgfqpoint{0.000000in}{0.020833in}}{%
\pgfpathmoveto{\pgfqpoint{0.000000in}{0.000000in}}%
\pgfpathlineto{\pgfqpoint{0.000000in}{0.020833in}}%
\pgfusepath{stroke,fill}%
}%
\begin{pgfscope}%
\pgfsys@transformshift{0.725330in}{0.586309in}%
\pgfsys@useobject{currentmarker}{}%
\end{pgfscope}%
\end{pgfscope}%
\begin{pgfscope}%
\pgfsetbuttcap%
\pgfsetroundjoin%
\definecolor{currentfill}{rgb}{0.000000,0.000000,0.000000}%
\pgfsetfillcolor{currentfill}%
\pgfsetlinewidth{0.501875pt}%
\definecolor{currentstroke}{rgb}{0.000000,0.000000,0.000000}%
\pgfsetstrokecolor{currentstroke}%
\pgfsetdash{}{0pt}%
\pgfsys@defobject{currentmarker}{\pgfqpoint{0.000000in}{-0.020833in}}{\pgfqpoint{0.000000in}{0.000000in}}{%
\pgfpathmoveto{\pgfqpoint{0.000000in}{0.000000in}}%
\pgfpathlineto{\pgfqpoint{0.000000in}{-0.020833in}}%
\pgfusepath{stroke,fill}%
}%
\begin{pgfscope}%
\pgfsys@transformshift{0.725330in}{0.893003in}%
\pgfsys@useobject{currentmarker}{}%
\end{pgfscope}%
\end{pgfscope}%
\begin{pgfscope}%
\pgfsetbuttcap%
\pgfsetroundjoin%
\definecolor{currentfill}{rgb}{0.000000,0.000000,0.000000}%
\pgfsetfillcolor{currentfill}%
\pgfsetlinewidth{0.501875pt}%
\definecolor{currentstroke}{rgb}{0.000000,0.000000,0.000000}%
\pgfsetstrokecolor{currentstroke}%
\pgfsetdash{}{0pt}%
\pgfsys@defobject{currentmarker}{\pgfqpoint{0.000000in}{0.000000in}}{\pgfqpoint{0.000000in}{0.020833in}}{%
\pgfpathmoveto{\pgfqpoint{0.000000in}{0.000000in}}%
\pgfpathlineto{\pgfqpoint{0.000000in}{0.020833in}}%
\pgfusepath{stroke,fill}%
}%
\begin{pgfscope}%
\pgfsys@transformshift{0.761027in}{0.586309in}%
\pgfsys@useobject{currentmarker}{}%
\end{pgfscope}%
\end{pgfscope}%
\begin{pgfscope}%
\pgfsetbuttcap%
\pgfsetroundjoin%
\definecolor{currentfill}{rgb}{0.000000,0.000000,0.000000}%
\pgfsetfillcolor{currentfill}%
\pgfsetlinewidth{0.501875pt}%
\definecolor{currentstroke}{rgb}{0.000000,0.000000,0.000000}%
\pgfsetstrokecolor{currentstroke}%
\pgfsetdash{}{0pt}%
\pgfsys@defobject{currentmarker}{\pgfqpoint{0.000000in}{-0.020833in}}{\pgfqpoint{0.000000in}{0.000000in}}{%
\pgfpathmoveto{\pgfqpoint{0.000000in}{0.000000in}}%
\pgfpathlineto{\pgfqpoint{0.000000in}{-0.020833in}}%
\pgfusepath{stroke,fill}%
}%
\begin{pgfscope}%
\pgfsys@transformshift{0.761027in}{0.893003in}%
\pgfsys@useobject{currentmarker}{}%
\end{pgfscope}%
\end{pgfscope}%
\begin{pgfscope}%
\pgfsetbuttcap%
\pgfsetroundjoin%
\definecolor{currentfill}{rgb}{0.000000,0.000000,0.000000}%
\pgfsetfillcolor{currentfill}%
\pgfsetlinewidth{0.501875pt}%
\definecolor{currentstroke}{rgb}{0.000000,0.000000,0.000000}%
\pgfsetstrokecolor{currentstroke}%
\pgfsetdash{}{0pt}%
\pgfsys@defobject{currentmarker}{\pgfqpoint{0.000000in}{0.000000in}}{\pgfqpoint{0.000000in}{0.020833in}}{%
\pgfpathmoveto{\pgfqpoint{0.000000in}{0.000000in}}%
\pgfpathlineto{\pgfqpoint{0.000000in}{0.020833in}}%
\pgfusepath{stroke,fill}%
}%
\begin{pgfscope}%
\pgfsys@transformshift{0.796725in}{0.586309in}%
\pgfsys@useobject{currentmarker}{}%
\end{pgfscope}%
\end{pgfscope}%
\begin{pgfscope}%
\pgfsetbuttcap%
\pgfsetroundjoin%
\definecolor{currentfill}{rgb}{0.000000,0.000000,0.000000}%
\pgfsetfillcolor{currentfill}%
\pgfsetlinewidth{0.501875pt}%
\definecolor{currentstroke}{rgb}{0.000000,0.000000,0.000000}%
\pgfsetstrokecolor{currentstroke}%
\pgfsetdash{}{0pt}%
\pgfsys@defobject{currentmarker}{\pgfqpoint{0.000000in}{-0.020833in}}{\pgfqpoint{0.000000in}{0.000000in}}{%
\pgfpathmoveto{\pgfqpoint{0.000000in}{0.000000in}}%
\pgfpathlineto{\pgfqpoint{0.000000in}{-0.020833in}}%
\pgfusepath{stroke,fill}%
}%
\begin{pgfscope}%
\pgfsys@transformshift{0.796725in}{0.893003in}%
\pgfsys@useobject{currentmarker}{}%
\end{pgfscope}%
\end{pgfscope}%
\begin{pgfscope}%
\pgfsetbuttcap%
\pgfsetroundjoin%
\definecolor{currentfill}{rgb}{0.000000,0.000000,0.000000}%
\pgfsetfillcolor{currentfill}%
\pgfsetlinewidth{0.501875pt}%
\definecolor{currentstroke}{rgb}{0.000000,0.000000,0.000000}%
\pgfsetstrokecolor{currentstroke}%
\pgfsetdash{}{0pt}%
\pgfsys@defobject{currentmarker}{\pgfqpoint{0.000000in}{0.000000in}}{\pgfqpoint{0.000000in}{0.020833in}}{%
\pgfpathmoveto{\pgfqpoint{0.000000in}{0.000000in}}%
\pgfpathlineto{\pgfqpoint{0.000000in}{0.020833in}}%
\pgfusepath{stroke,fill}%
}%
\begin{pgfscope}%
\pgfsys@transformshift{0.832422in}{0.586309in}%
\pgfsys@useobject{currentmarker}{}%
\end{pgfscope}%
\end{pgfscope}%
\begin{pgfscope}%
\pgfsetbuttcap%
\pgfsetroundjoin%
\definecolor{currentfill}{rgb}{0.000000,0.000000,0.000000}%
\pgfsetfillcolor{currentfill}%
\pgfsetlinewidth{0.501875pt}%
\definecolor{currentstroke}{rgb}{0.000000,0.000000,0.000000}%
\pgfsetstrokecolor{currentstroke}%
\pgfsetdash{}{0pt}%
\pgfsys@defobject{currentmarker}{\pgfqpoint{0.000000in}{-0.020833in}}{\pgfqpoint{0.000000in}{0.000000in}}{%
\pgfpathmoveto{\pgfqpoint{0.000000in}{0.000000in}}%
\pgfpathlineto{\pgfqpoint{0.000000in}{-0.020833in}}%
\pgfusepath{stroke,fill}%
}%
\begin{pgfscope}%
\pgfsys@transformshift{0.832422in}{0.893003in}%
\pgfsys@useobject{currentmarker}{}%
\end{pgfscope}%
\end{pgfscope}%
\begin{pgfscope}%
\pgfsetbuttcap%
\pgfsetroundjoin%
\definecolor{currentfill}{rgb}{0.000000,0.000000,0.000000}%
\pgfsetfillcolor{currentfill}%
\pgfsetlinewidth{0.501875pt}%
\definecolor{currentstroke}{rgb}{0.000000,0.000000,0.000000}%
\pgfsetstrokecolor{currentstroke}%
\pgfsetdash{}{0pt}%
\pgfsys@defobject{currentmarker}{\pgfqpoint{0.000000in}{0.000000in}}{\pgfqpoint{0.000000in}{0.020833in}}{%
\pgfpathmoveto{\pgfqpoint{0.000000in}{0.000000in}}%
\pgfpathlineto{\pgfqpoint{0.000000in}{0.020833in}}%
\pgfusepath{stroke,fill}%
}%
\begin{pgfscope}%
\pgfsys@transformshift{0.868119in}{0.586309in}%
\pgfsys@useobject{currentmarker}{}%
\end{pgfscope}%
\end{pgfscope}%
\begin{pgfscope}%
\pgfsetbuttcap%
\pgfsetroundjoin%
\definecolor{currentfill}{rgb}{0.000000,0.000000,0.000000}%
\pgfsetfillcolor{currentfill}%
\pgfsetlinewidth{0.501875pt}%
\definecolor{currentstroke}{rgb}{0.000000,0.000000,0.000000}%
\pgfsetstrokecolor{currentstroke}%
\pgfsetdash{}{0pt}%
\pgfsys@defobject{currentmarker}{\pgfqpoint{0.000000in}{-0.020833in}}{\pgfqpoint{0.000000in}{0.000000in}}{%
\pgfpathmoveto{\pgfqpoint{0.000000in}{0.000000in}}%
\pgfpathlineto{\pgfqpoint{0.000000in}{-0.020833in}}%
\pgfusepath{stroke,fill}%
}%
\begin{pgfscope}%
\pgfsys@transformshift{0.868119in}{0.893003in}%
\pgfsys@useobject{currentmarker}{}%
\end{pgfscope}%
\end{pgfscope}%
\begin{pgfscope}%
\pgfsetbuttcap%
\pgfsetroundjoin%
\definecolor{currentfill}{rgb}{0.000000,0.000000,0.000000}%
\pgfsetfillcolor{currentfill}%
\pgfsetlinewidth{0.501875pt}%
\definecolor{currentstroke}{rgb}{0.000000,0.000000,0.000000}%
\pgfsetstrokecolor{currentstroke}%
\pgfsetdash{}{0pt}%
\pgfsys@defobject{currentmarker}{\pgfqpoint{0.000000in}{0.000000in}}{\pgfqpoint{0.000000in}{0.020833in}}{%
\pgfpathmoveto{\pgfqpoint{0.000000in}{0.000000in}}%
\pgfpathlineto{\pgfqpoint{0.000000in}{0.020833in}}%
\pgfusepath{stroke,fill}%
}%
\begin{pgfscope}%
\pgfsys@transformshift{0.903816in}{0.586309in}%
\pgfsys@useobject{currentmarker}{}%
\end{pgfscope}%
\end{pgfscope}%
\begin{pgfscope}%
\pgfsetbuttcap%
\pgfsetroundjoin%
\definecolor{currentfill}{rgb}{0.000000,0.000000,0.000000}%
\pgfsetfillcolor{currentfill}%
\pgfsetlinewidth{0.501875pt}%
\definecolor{currentstroke}{rgb}{0.000000,0.000000,0.000000}%
\pgfsetstrokecolor{currentstroke}%
\pgfsetdash{}{0pt}%
\pgfsys@defobject{currentmarker}{\pgfqpoint{0.000000in}{-0.020833in}}{\pgfqpoint{0.000000in}{0.000000in}}{%
\pgfpathmoveto{\pgfqpoint{0.000000in}{0.000000in}}%
\pgfpathlineto{\pgfqpoint{0.000000in}{-0.020833in}}%
\pgfusepath{stroke,fill}%
}%
\begin{pgfscope}%
\pgfsys@transformshift{0.903816in}{0.893003in}%
\pgfsys@useobject{currentmarker}{}%
\end{pgfscope}%
\end{pgfscope}%
\begin{pgfscope}%
\pgfsetbuttcap%
\pgfsetroundjoin%
\definecolor{currentfill}{rgb}{0.000000,0.000000,0.000000}%
\pgfsetfillcolor{currentfill}%
\pgfsetlinewidth{0.501875pt}%
\definecolor{currentstroke}{rgb}{0.000000,0.000000,0.000000}%
\pgfsetstrokecolor{currentstroke}%
\pgfsetdash{}{0pt}%
\pgfsys@defobject{currentmarker}{\pgfqpoint{0.000000in}{0.000000in}}{\pgfqpoint{0.000000in}{0.020833in}}{%
\pgfpathmoveto{\pgfqpoint{0.000000in}{0.000000in}}%
\pgfpathlineto{\pgfqpoint{0.000000in}{0.020833in}}%
\pgfusepath{stroke,fill}%
}%
\begin{pgfscope}%
\pgfsys@transformshift{0.939514in}{0.586309in}%
\pgfsys@useobject{currentmarker}{}%
\end{pgfscope}%
\end{pgfscope}%
\begin{pgfscope}%
\pgfsetbuttcap%
\pgfsetroundjoin%
\definecolor{currentfill}{rgb}{0.000000,0.000000,0.000000}%
\pgfsetfillcolor{currentfill}%
\pgfsetlinewidth{0.501875pt}%
\definecolor{currentstroke}{rgb}{0.000000,0.000000,0.000000}%
\pgfsetstrokecolor{currentstroke}%
\pgfsetdash{}{0pt}%
\pgfsys@defobject{currentmarker}{\pgfqpoint{0.000000in}{-0.020833in}}{\pgfqpoint{0.000000in}{0.000000in}}{%
\pgfpathmoveto{\pgfqpoint{0.000000in}{0.000000in}}%
\pgfpathlineto{\pgfqpoint{0.000000in}{-0.020833in}}%
\pgfusepath{stroke,fill}%
}%
\begin{pgfscope}%
\pgfsys@transformshift{0.939514in}{0.893003in}%
\pgfsys@useobject{currentmarker}{}%
\end{pgfscope}%
\end{pgfscope}%
\begin{pgfscope}%
\pgfsetbuttcap%
\pgfsetroundjoin%
\definecolor{currentfill}{rgb}{0.000000,0.000000,0.000000}%
\pgfsetfillcolor{currentfill}%
\pgfsetlinewidth{0.501875pt}%
\definecolor{currentstroke}{rgb}{0.000000,0.000000,0.000000}%
\pgfsetstrokecolor{currentstroke}%
\pgfsetdash{}{0pt}%
\pgfsys@defobject{currentmarker}{\pgfqpoint{0.000000in}{0.000000in}}{\pgfqpoint{0.000000in}{0.020833in}}{%
\pgfpathmoveto{\pgfqpoint{0.000000in}{0.000000in}}%
\pgfpathlineto{\pgfqpoint{0.000000in}{0.020833in}}%
\pgfusepath{stroke,fill}%
}%
\begin{pgfscope}%
\pgfsys@transformshift{0.975211in}{0.586309in}%
\pgfsys@useobject{currentmarker}{}%
\end{pgfscope}%
\end{pgfscope}%
\begin{pgfscope}%
\pgfsetbuttcap%
\pgfsetroundjoin%
\definecolor{currentfill}{rgb}{0.000000,0.000000,0.000000}%
\pgfsetfillcolor{currentfill}%
\pgfsetlinewidth{0.501875pt}%
\definecolor{currentstroke}{rgb}{0.000000,0.000000,0.000000}%
\pgfsetstrokecolor{currentstroke}%
\pgfsetdash{}{0pt}%
\pgfsys@defobject{currentmarker}{\pgfqpoint{0.000000in}{-0.020833in}}{\pgfqpoint{0.000000in}{0.000000in}}{%
\pgfpathmoveto{\pgfqpoint{0.000000in}{0.000000in}}%
\pgfpathlineto{\pgfqpoint{0.000000in}{-0.020833in}}%
\pgfusepath{stroke,fill}%
}%
\begin{pgfscope}%
\pgfsys@transformshift{0.975211in}{0.893003in}%
\pgfsys@useobject{currentmarker}{}%
\end{pgfscope}%
\end{pgfscope}%
\begin{pgfscope}%
\pgfsetbuttcap%
\pgfsetroundjoin%
\definecolor{currentfill}{rgb}{0.000000,0.000000,0.000000}%
\pgfsetfillcolor{currentfill}%
\pgfsetlinewidth{0.501875pt}%
\definecolor{currentstroke}{rgb}{0.000000,0.000000,0.000000}%
\pgfsetstrokecolor{currentstroke}%
\pgfsetdash{}{0pt}%
\pgfsys@defobject{currentmarker}{\pgfqpoint{0.000000in}{0.000000in}}{\pgfqpoint{0.000000in}{0.020833in}}{%
\pgfpathmoveto{\pgfqpoint{0.000000in}{0.000000in}}%
\pgfpathlineto{\pgfqpoint{0.000000in}{0.020833in}}%
\pgfusepath{stroke,fill}%
}%
\begin{pgfscope}%
\pgfsys@transformshift{1.010908in}{0.586309in}%
\pgfsys@useobject{currentmarker}{}%
\end{pgfscope}%
\end{pgfscope}%
\begin{pgfscope}%
\pgfsetbuttcap%
\pgfsetroundjoin%
\definecolor{currentfill}{rgb}{0.000000,0.000000,0.000000}%
\pgfsetfillcolor{currentfill}%
\pgfsetlinewidth{0.501875pt}%
\definecolor{currentstroke}{rgb}{0.000000,0.000000,0.000000}%
\pgfsetstrokecolor{currentstroke}%
\pgfsetdash{}{0pt}%
\pgfsys@defobject{currentmarker}{\pgfqpoint{0.000000in}{-0.020833in}}{\pgfqpoint{0.000000in}{0.000000in}}{%
\pgfpathmoveto{\pgfqpoint{0.000000in}{0.000000in}}%
\pgfpathlineto{\pgfqpoint{0.000000in}{-0.020833in}}%
\pgfusepath{stroke,fill}%
}%
\begin{pgfscope}%
\pgfsys@transformshift{1.010908in}{0.893003in}%
\pgfsys@useobject{currentmarker}{}%
\end{pgfscope}%
\end{pgfscope}%
\begin{pgfscope}%
\pgfsetbuttcap%
\pgfsetroundjoin%
\definecolor{currentfill}{rgb}{0.000000,0.000000,0.000000}%
\pgfsetfillcolor{currentfill}%
\pgfsetlinewidth{0.501875pt}%
\definecolor{currentstroke}{rgb}{0.000000,0.000000,0.000000}%
\pgfsetstrokecolor{currentstroke}%
\pgfsetdash{}{0pt}%
\pgfsys@defobject{currentmarker}{\pgfqpoint{0.000000in}{0.000000in}}{\pgfqpoint{0.000000in}{0.020833in}}{%
\pgfpathmoveto{\pgfqpoint{0.000000in}{0.000000in}}%
\pgfpathlineto{\pgfqpoint{0.000000in}{0.020833in}}%
\pgfusepath{stroke,fill}%
}%
\begin{pgfscope}%
\pgfsys@transformshift{1.046606in}{0.586309in}%
\pgfsys@useobject{currentmarker}{}%
\end{pgfscope}%
\end{pgfscope}%
\begin{pgfscope}%
\pgfsetbuttcap%
\pgfsetroundjoin%
\definecolor{currentfill}{rgb}{0.000000,0.000000,0.000000}%
\pgfsetfillcolor{currentfill}%
\pgfsetlinewidth{0.501875pt}%
\definecolor{currentstroke}{rgb}{0.000000,0.000000,0.000000}%
\pgfsetstrokecolor{currentstroke}%
\pgfsetdash{}{0pt}%
\pgfsys@defobject{currentmarker}{\pgfqpoint{0.000000in}{-0.020833in}}{\pgfqpoint{0.000000in}{0.000000in}}{%
\pgfpathmoveto{\pgfqpoint{0.000000in}{0.000000in}}%
\pgfpathlineto{\pgfqpoint{0.000000in}{-0.020833in}}%
\pgfusepath{stroke,fill}%
}%
\begin{pgfscope}%
\pgfsys@transformshift{1.046606in}{0.893003in}%
\pgfsys@useobject{currentmarker}{}%
\end{pgfscope}%
\end{pgfscope}%
\begin{pgfscope}%
\pgfsetbuttcap%
\pgfsetroundjoin%
\definecolor{currentfill}{rgb}{0.000000,0.000000,0.000000}%
\pgfsetfillcolor{currentfill}%
\pgfsetlinewidth{0.501875pt}%
\definecolor{currentstroke}{rgb}{0.000000,0.000000,0.000000}%
\pgfsetstrokecolor{currentstroke}%
\pgfsetdash{}{0pt}%
\pgfsys@defobject{currentmarker}{\pgfqpoint{0.000000in}{0.000000in}}{\pgfqpoint{0.000000in}{0.020833in}}{%
\pgfpathmoveto{\pgfqpoint{0.000000in}{0.000000in}}%
\pgfpathlineto{\pgfqpoint{0.000000in}{0.020833in}}%
\pgfusepath{stroke,fill}%
}%
\begin{pgfscope}%
\pgfsys@transformshift{1.082303in}{0.586309in}%
\pgfsys@useobject{currentmarker}{}%
\end{pgfscope}%
\end{pgfscope}%
\begin{pgfscope}%
\pgfsetbuttcap%
\pgfsetroundjoin%
\definecolor{currentfill}{rgb}{0.000000,0.000000,0.000000}%
\pgfsetfillcolor{currentfill}%
\pgfsetlinewidth{0.501875pt}%
\definecolor{currentstroke}{rgb}{0.000000,0.000000,0.000000}%
\pgfsetstrokecolor{currentstroke}%
\pgfsetdash{}{0pt}%
\pgfsys@defobject{currentmarker}{\pgfqpoint{0.000000in}{-0.020833in}}{\pgfqpoint{0.000000in}{0.000000in}}{%
\pgfpathmoveto{\pgfqpoint{0.000000in}{0.000000in}}%
\pgfpathlineto{\pgfqpoint{0.000000in}{-0.020833in}}%
\pgfusepath{stroke,fill}%
}%
\begin{pgfscope}%
\pgfsys@transformshift{1.082303in}{0.893003in}%
\pgfsys@useobject{currentmarker}{}%
\end{pgfscope}%
\end{pgfscope}%
\begin{pgfscope}%
\pgfsetbuttcap%
\pgfsetroundjoin%
\definecolor{currentfill}{rgb}{0.000000,0.000000,0.000000}%
\pgfsetfillcolor{currentfill}%
\pgfsetlinewidth{0.501875pt}%
\definecolor{currentstroke}{rgb}{0.000000,0.000000,0.000000}%
\pgfsetstrokecolor{currentstroke}%
\pgfsetdash{}{0pt}%
\pgfsys@defobject{currentmarker}{\pgfqpoint{0.000000in}{0.000000in}}{\pgfqpoint{0.000000in}{0.020833in}}{%
\pgfpathmoveto{\pgfqpoint{0.000000in}{0.000000in}}%
\pgfpathlineto{\pgfqpoint{0.000000in}{0.020833in}}%
\pgfusepath{stroke,fill}%
}%
\begin{pgfscope}%
\pgfsys@transformshift{1.153698in}{0.586309in}%
\pgfsys@useobject{currentmarker}{}%
\end{pgfscope}%
\end{pgfscope}%
\begin{pgfscope}%
\pgfsetbuttcap%
\pgfsetroundjoin%
\definecolor{currentfill}{rgb}{0.000000,0.000000,0.000000}%
\pgfsetfillcolor{currentfill}%
\pgfsetlinewidth{0.501875pt}%
\definecolor{currentstroke}{rgb}{0.000000,0.000000,0.000000}%
\pgfsetstrokecolor{currentstroke}%
\pgfsetdash{}{0pt}%
\pgfsys@defobject{currentmarker}{\pgfqpoint{0.000000in}{-0.020833in}}{\pgfqpoint{0.000000in}{0.000000in}}{%
\pgfpathmoveto{\pgfqpoint{0.000000in}{0.000000in}}%
\pgfpathlineto{\pgfqpoint{0.000000in}{-0.020833in}}%
\pgfusepath{stroke,fill}%
}%
\begin{pgfscope}%
\pgfsys@transformshift{1.153698in}{0.893003in}%
\pgfsys@useobject{currentmarker}{}%
\end{pgfscope}%
\end{pgfscope}%
\begin{pgfscope}%
\pgfsetbuttcap%
\pgfsetroundjoin%
\definecolor{currentfill}{rgb}{0.000000,0.000000,0.000000}%
\pgfsetfillcolor{currentfill}%
\pgfsetlinewidth{0.501875pt}%
\definecolor{currentstroke}{rgb}{0.000000,0.000000,0.000000}%
\pgfsetstrokecolor{currentstroke}%
\pgfsetdash{}{0pt}%
\pgfsys@defobject{currentmarker}{\pgfqpoint{0.000000in}{0.000000in}}{\pgfqpoint{0.000000in}{0.020833in}}{%
\pgfpathmoveto{\pgfqpoint{0.000000in}{0.000000in}}%
\pgfpathlineto{\pgfqpoint{0.000000in}{0.020833in}}%
\pgfusepath{stroke,fill}%
}%
\begin{pgfscope}%
\pgfsys@transformshift{1.189395in}{0.586309in}%
\pgfsys@useobject{currentmarker}{}%
\end{pgfscope}%
\end{pgfscope}%
\begin{pgfscope}%
\pgfsetbuttcap%
\pgfsetroundjoin%
\definecolor{currentfill}{rgb}{0.000000,0.000000,0.000000}%
\pgfsetfillcolor{currentfill}%
\pgfsetlinewidth{0.501875pt}%
\definecolor{currentstroke}{rgb}{0.000000,0.000000,0.000000}%
\pgfsetstrokecolor{currentstroke}%
\pgfsetdash{}{0pt}%
\pgfsys@defobject{currentmarker}{\pgfqpoint{0.000000in}{-0.020833in}}{\pgfqpoint{0.000000in}{0.000000in}}{%
\pgfpathmoveto{\pgfqpoint{0.000000in}{0.000000in}}%
\pgfpathlineto{\pgfqpoint{0.000000in}{-0.020833in}}%
\pgfusepath{stroke,fill}%
}%
\begin{pgfscope}%
\pgfsys@transformshift{1.189395in}{0.893003in}%
\pgfsys@useobject{currentmarker}{}%
\end{pgfscope}%
\end{pgfscope}%
\begin{pgfscope}%
\pgfsetbuttcap%
\pgfsetroundjoin%
\definecolor{currentfill}{rgb}{0.000000,0.000000,0.000000}%
\pgfsetfillcolor{currentfill}%
\pgfsetlinewidth{0.501875pt}%
\definecolor{currentstroke}{rgb}{0.000000,0.000000,0.000000}%
\pgfsetstrokecolor{currentstroke}%
\pgfsetdash{}{0pt}%
\pgfsys@defobject{currentmarker}{\pgfqpoint{0.000000in}{0.000000in}}{\pgfqpoint{0.000000in}{0.020833in}}{%
\pgfpathmoveto{\pgfqpoint{0.000000in}{0.000000in}}%
\pgfpathlineto{\pgfqpoint{0.000000in}{0.020833in}}%
\pgfusepath{stroke,fill}%
}%
\begin{pgfscope}%
\pgfsys@transformshift{1.225092in}{0.586309in}%
\pgfsys@useobject{currentmarker}{}%
\end{pgfscope}%
\end{pgfscope}%
\begin{pgfscope}%
\pgfsetbuttcap%
\pgfsetroundjoin%
\definecolor{currentfill}{rgb}{0.000000,0.000000,0.000000}%
\pgfsetfillcolor{currentfill}%
\pgfsetlinewidth{0.501875pt}%
\definecolor{currentstroke}{rgb}{0.000000,0.000000,0.000000}%
\pgfsetstrokecolor{currentstroke}%
\pgfsetdash{}{0pt}%
\pgfsys@defobject{currentmarker}{\pgfqpoint{0.000000in}{-0.020833in}}{\pgfqpoint{0.000000in}{0.000000in}}{%
\pgfpathmoveto{\pgfqpoint{0.000000in}{0.000000in}}%
\pgfpathlineto{\pgfqpoint{0.000000in}{-0.020833in}}%
\pgfusepath{stroke,fill}%
}%
\begin{pgfscope}%
\pgfsys@transformshift{1.225092in}{0.893003in}%
\pgfsys@useobject{currentmarker}{}%
\end{pgfscope}%
\end{pgfscope}%
\begin{pgfscope}%
\pgfsetbuttcap%
\pgfsetroundjoin%
\definecolor{currentfill}{rgb}{0.000000,0.000000,0.000000}%
\pgfsetfillcolor{currentfill}%
\pgfsetlinewidth{0.501875pt}%
\definecolor{currentstroke}{rgb}{0.000000,0.000000,0.000000}%
\pgfsetstrokecolor{currentstroke}%
\pgfsetdash{}{0pt}%
\pgfsys@defobject{currentmarker}{\pgfqpoint{0.000000in}{0.000000in}}{\pgfqpoint{0.000000in}{0.020833in}}{%
\pgfpathmoveto{\pgfqpoint{0.000000in}{0.000000in}}%
\pgfpathlineto{\pgfqpoint{0.000000in}{0.020833in}}%
\pgfusepath{stroke,fill}%
}%
\begin{pgfscope}%
\pgfsys@transformshift{1.260790in}{0.586309in}%
\pgfsys@useobject{currentmarker}{}%
\end{pgfscope}%
\end{pgfscope}%
\begin{pgfscope}%
\pgfsetbuttcap%
\pgfsetroundjoin%
\definecolor{currentfill}{rgb}{0.000000,0.000000,0.000000}%
\pgfsetfillcolor{currentfill}%
\pgfsetlinewidth{0.501875pt}%
\definecolor{currentstroke}{rgb}{0.000000,0.000000,0.000000}%
\pgfsetstrokecolor{currentstroke}%
\pgfsetdash{}{0pt}%
\pgfsys@defobject{currentmarker}{\pgfqpoint{0.000000in}{-0.020833in}}{\pgfqpoint{0.000000in}{0.000000in}}{%
\pgfpathmoveto{\pgfqpoint{0.000000in}{0.000000in}}%
\pgfpathlineto{\pgfqpoint{0.000000in}{-0.020833in}}%
\pgfusepath{stroke,fill}%
}%
\begin{pgfscope}%
\pgfsys@transformshift{1.260790in}{0.893003in}%
\pgfsys@useobject{currentmarker}{}%
\end{pgfscope}%
\end{pgfscope}%
\begin{pgfscope}%
\pgfsetbuttcap%
\pgfsetroundjoin%
\definecolor{currentfill}{rgb}{0.000000,0.000000,0.000000}%
\pgfsetfillcolor{currentfill}%
\pgfsetlinewidth{0.501875pt}%
\definecolor{currentstroke}{rgb}{0.000000,0.000000,0.000000}%
\pgfsetstrokecolor{currentstroke}%
\pgfsetdash{}{0pt}%
\pgfsys@defobject{currentmarker}{\pgfqpoint{0.000000in}{0.000000in}}{\pgfqpoint{0.000000in}{0.020833in}}{%
\pgfpathmoveto{\pgfqpoint{0.000000in}{0.000000in}}%
\pgfpathlineto{\pgfqpoint{0.000000in}{0.020833in}}%
\pgfusepath{stroke,fill}%
}%
\begin{pgfscope}%
\pgfsys@transformshift{1.296487in}{0.586309in}%
\pgfsys@useobject{currentmarker}{}%
\end{pgfscope}%
\end{pgfscope}%
\begin{pgfscope}%
\pgfsetbuttcap%
\pgfsetroundjoin%
\definecolor{currentfill}{rgb}{0.000000,0.000000,0.000000}%
\pgfsetfillcolor{currentfill}%
\pgfsetlinewidth{0.501875pt}%
\definecolor{currentstroke}{rgb}{0.000000,0.000000,0.000000}%
\pgfsetstrokecolor{currentstroke}%
\pgfsetdash{}{0pt}%
\pgfsys@defobject{currentmarker}{\pgfqpoint{0.000000in}{-0.020833in}}{\pgfqpoint{0.000000in}{0.000000in}}{%
\pgfpathmoveto{\pgfqpoint{0.000000in}{0.000000in}}%
\pgfpathlineto{\pgfqpoint{0.000000in}{-0.020833in}}%
\pgfusepath{stroke,fill}%
}%
\begin{pgfscope}%
\pgfsys@transformshift{1.296487in}{0.893003in}%
\pgfsys@useobject{currentmarker}{}%
\end{pgfscope}%
\end{pgfscope}%
\begin{pgfscope}%
\pgfsetbuttcap%
\pgfsetroundjoin%
\definecolor{currentfill}{rgb}{0.000000,0.000000,0.000000}%
\pgfsetfillcolor{currentfill}%
\pgfsetlinewidth{0.501875pt}%
\definecolor{currentstroke}{rgb}{0.000000,0.000000,0.000000}%
\pgfsetstrokecolor{currentstroke}%
\pgfsetdash{}{0pt}%
\pgfsys@defobject{currentmarker}{\pgfqpoint{0.000000in}{0.000000in}}{\pgfqpoint{0.000000in}{0.020833in}}{%
\pgfpathmoveto{\pgfqpoint{0.000000in}{0.000000in}}%
\pgfpathlineto{\pgfqpoint{0.000000in}{0.020833in}}%
\pgfusepath{stroke,fill}%
}%
\begin{pgfscope}%
\pgfsys@transformshift{1.332184in}{0.586309in}%
\pgfsys@useobject{currentmarker}{}%
\end{pgfscope}%
\end{pgfscope}%
\begin{pgfscope}%
\pgfsetbuttcap%
\pgfsetroundjoin%
\definecolor{currentfill}{rgb}{0.000000,0.000000,0.000000}%
\pgfsetfillcolor{currentfill}%
\pgfsetlinewidth{0.501875pt}%
\definecolor{currentstroke}{rgb}{0.000000,0.000000,0.000000}%
\pgfsetstrokecolor{currentstroke}%
\pgfsetdash{}{0pt}%
\pgfsys@defobject{currentmarker}{\pgfqpoint{0.000000in}{-0.020833in}}{\pgfqpoint{0.000000in}{0.000000in}}{%
\pgfpathmoveto{\pgfqpoint{0.000000in}{0.000000in}}%
\pgfpathlineto{\pgfqpoint{0.000000in}{-0.020833in}}%
\pgfusepath{stroke,fill}%
}%
\begin{pgfscope}%
\pgfsys@transformshift{1.332184in}{0.893003in}%
\pgfsys@useobject{currentmarker}{}%
\end{pgfscope}%
\end{pgfscope}%
\begin{pgfscope}%
\pgfsetbuttcap%
\pgfsetroundjoin%
\definecolor{currentfill}{rgb}{0.000000,0.000000,0.000000}%
\pgfsetfillcolor{currentfill}%
\pgfsetlinewidth{0.501875pt}%
\definecolor{currentstroke}{rgb}{0.000000,0.000000,0.000000}%
\pgfsetstrokecolor{currentstroke}%
\pgfsetdash{}{0pt}%
\pgfsys@defobject{currentmarker}{\pgfqpoint{0.000000in}{0.000000in}}{\pgfqpoint{0.000000in}{0.020833in}}{%
\pgfpathmoveto{\pgfqpoint{0.000000in}{0.000000in}}%
\pgfpathlineto{\pgfqpoint{0.000000in}{0.020833in}}%
\pgfusepath{stroke,fill}%
}%
\begin{pgfscope}%
\pgfsys@transformshift{1.367882in}{0.586309in}%
\pgfsys@useobject{currentmarker}{}%
\end{pgfscope}%
\end{pgfscope}%
\begin{pgfscope}%
\pgfsetbuttcap%
\pgfsetroundjoin%
\definecolor{currentfill}{rgb}{0.000000,0.000000,0.000000}%
\pgfsetfillcolor{currentfill}%
\pgfsetlinewidth{0.501875pt}%
\definecolor{currentstroke}{rgb}{0.000000,0.000000,0.000000}%
\pgfsetstrokecolor{currentstroke}%
\pgfsetdash{}{0pt}%
\pgfsys@defobject{currentmarker}{\pgfqpoint{0.000000in}{-0.020833in}}{\pgfqpoint{0.000000in}{0.000000in}}{%
\pgfpathmoveto{\pgfqpoint{0.000000in}{0.000000in}}%
\pgfpathlineto{\pgfqpoint{0.000000in}{-0.020833in}}%
\pgfusepath{stroke,fill}%
}%
\begin{pgfscope}%
\pgfsys@transformshift{1.367882in}{0.893003in}%
\pgfsys@useobject{currentmarker}{}%
\end{pgfscope}%
\end{pgfscope}%
\begin{pgfscope}%
\pgfsetbuttcap%
\pgfsetroundjoin%
\definecolor{currentfill}{rgb}{0.000000,0.000000,0.000000}%
\pgfsetfillcolor{currentfill}%
\pgfsetlinewidth{0.501875pt}%
\definecolor{currentstroke}{rgb}{0.000000,0.000000,0.000000}%
\pgfsetstrokecolor{currentstroke}%
\pgfsetdash{}{0pt}%
\pgfsys@defobject{currentmarker}{\pgfqpoint{0.000000in}{0.000000in}}{\pgfqpoint{0.000000in}{0.020833in}}{%
\pgfpathmoveto{\pgfqpoint{0.000000in}{0.000000in}}%
\pgfpathlineto{\pgfqpoint{0.000000in}{0.020833in}}%
\pgfusepath{stroke,fill}%
}%
\begin{pgfscope}%
\pgfsys@transformshift{1.403579in}{0.586309in}%
\pgfsys@useobject{currentmarker}{}%
\end{pgfscope}%
\end{pgfscope}%
\begin{pgfscope}%
\pgfsetbuttcap%
\pgfsetroundjoin%
\definecolor{currentfill}{rgb}{0.000000,0.000000,0.000000}%
\pgfsetfillcolor{currentfill}%
\pgfsetlinewidth{0.501875pt}%
\definecolor{currentstroke}{rgb}{0.000000,0.000000,0.000000}%
\pgfsetstrokecolor{currentstroke}%
\pgfsetdash{}{0pt}%
\pgfsys@defobject{currentmarker}{\pgfqpoint{0.000000in}{-0.020833in}}{\pgfqpoint{0.000000in}{0.000000in}}{%
\pgfpathmoveto{\pgfqpoint{0.000000in}{0.000000in}}%
\pgfpathlineto{\pgfqpoint{0.000000in}{-0.020833in}}%
\pgfusepath{stroke,fill}%
}%
\begin{pgfscope}%
\pgfsys@transformshift{1.403579in}{0.893003in}%
\pgfsys@useobject{currentmarker}{}%
\end{pgfscope}%
\end{pgfscope}%
\begin{pgfscope}%
\pgfsetbuttcap%
\pgfsetroundjoin%
\definecolor{currentfill}{rgb}{0.000000,0.000000,0.000000}%
\pgfsetfillcolor{currentfill}%
\pgfsetlinewidth{0.501875pt}%
\definecolor{currentstroke}{rgb}{0.000000,0.000000,0.000000}%
\pgfsetstrokecolor{currentstroke}%
\pgfsetdash{}{0pt}%
\pgfsys@defobject{currentmarker}{\pgfqpoint{0.000000in}{0.000000in}}{\pgfqpoint{0.000000in}{0.020833in}}{%
\pgfpathmoveto{\pgfqpoint{0.000000in}{0.000000in}}%
\pgfpathlineto{\pgfqpoint{0.000000in}{0.020833in}}%
\pgfusepath{stroke,fill}%
}%
\begin{pgfscope}%
\pgfsys@transformshift{1.439276in}{0.586309in}%
\pgfsys@useobject{currentmarker}{}%
\end{pgfscope}%
\end{pgfscope}%
\begin{pgfscope}%
\pgfsetbuttcap%
\pgfsetroundjoin%
\definecolor{currentfill}{rgb}{0.000000,0.000000,0.000000}%
\pgfsetfillcolor{currentfill}%
\pgfsetlinewidth{0.501875pt}%
\definecolor{currentstroke}{rgb}{0.000000,0.000000,0.000000}%
\pgfsetstrokecolor{currentstroke}%
\pgfsetdash{}{0pt}%
\pgfsys@defobject{currentmarker}{\pgfqpoint{0.000000in}{-0.020833in}}{\pgfqpoint{0.000000in}{0.000000in}}{%
\pgfpathmoveto{\pgfqpoint{0.000000in}{0.000000in}}%
\pgfpathlineto{\pgfqpoint{0.000000in}{-0.020833in}}%
\pgfusepath{stroke,fill}%
}%
\begin{pgfscope}%
\pgfsys@transformshift{1.439276in}{0.893003in}%
\pgfsys@useobject{currentmarker}{}%
\end{pgfscope}%
\end{pgfscope}%
\begin{pgfscope}%
\pgfsetbuttcap%
\pgfsetroundjoin%
\definecolor{currentfill}{rgb}{0.000000,0.000000,0.000000}%
\pgfsetfillcolor{currentfill}%
\pgfsetlinewidth{0.501875pt}%
\definecolor{currentstroke}{rgb}{0.000000,0.000000,0.000000}%
\pgfsetstrokecolor{currentstroke}%
\pgfsetdash{}{0pt}%
\pgfsys@defobject{currentmarker}{\pgfqpoint{0.000000in}{0.000000in}}{\pgfqpoint{0.000000in}{0.020833in}}{%
\pgfpathmoveto{\pgfqpoint{0.000000in}{0.000000in}}%
\pgfpathlineto{\pgfqpoint{0.000000in}{0.020833in}}%
\pgfusepath{stroke,fill}%
}%
\begin{pgfscope}%
\pgfsys@transformshift{1.474974in}{0.586309in}%
\pgfsys@useobject{currentmarker}{}%
\end{pgfscope}%
\end{pgfscope}%
\begin{pgfscope}%
\pgfsetbuttcap%
\pgfsetroundjoin%
\definecolor{currentfill}{rgb}{0.000000,0.000000,0.000000}%
\pgfsetfillcolor{currentfill}%
\pgfsetlinewidth{0.501875pt}%
\definecolor{currentstroke}{rgb}{0.000000,0.000000,0.000000}%
\pgfsetstrokecolor{currentstroke}%
\pgfsetdash{}{0pt}%
\pgfsys@defobject{currentmarker}{\pgfqpoint{0.000000in}{-0.020833in}}{\pgfqpoint{0.000000in}{0.000000in}}{%
\pgfpathmoveto{\pgfqpoint{0.000000in}{0.000000in}}%
\pgfpathlineto{\pgfqpoint{0.000000in}{-0.020833in}}%
\pgfusepath{stroke,fill}%
}%
\begin{pgfscope}%
\pgfsys@transformshift{1.474974in}{0.893003in}%
\pgfsys@useobject{currentmarker}{}%
\end{pgfscope}%
\end{pgfscope}%
\begin{pgfscope}%
\pgfsetbuttcap%
\pgfsetroundjoin%
\definecolor{currentfill}{rgb}{0.000000,0.000000,0.000000}%
\pgfsetfillcolor{currentfill}%
\pgfsetlinewidth{0.501875pt}%
\definecolor{currentstroke}{rgb}{0.000000,0.000000,0.000000}%
\pgfsetstrokecolor{currentstroke}%
\pgfsetdash{}{0pt}%
\pgfsys@defobject{currentmarker}{\pgfqpoint{0.000000in}{0.000000in}}{\pgfqpoint{0.000000in}{0.020833in}}{%
\pgfpathmoveto{\pgfqpoint{0.000000in}{0.000000in}}%
\pgfpathlineto{\pgfqpoint{0.000000in}{0.020833in}}%
\pgfusepath{stroke,fill}%
}%
\begin{pgfscope}%
\pgfsys@transformshift{1.510671in}{0.586309in}%
\pgfsys@useobject{currentmarker}{}%
\end{pgfscope}%
\end{pgfscope}%
\begin{pgfscope}%
\pgfsetbuttcap%
\pgfsetroundjoin%
\definecolor{currentfill}{rgb}{0.000000,0.000000,0.000000}%
\pgfsetfillcolor{currentfill}%
\pgfsetlinewidth{0.501875pt}%
\definecolor{currentstroke}{rgb}{0.000000,0.000000,0.000000}%
\pgfsetstrokecolor{currentstroke}%
\pgfsetdash{}{0pt}%
\pgfsys@defobject{currentmarker}{\pgfqpoint{0.000000in}{-0.020833in}}{\pgfqpoint{0.000000in}{0.000000in}}{%
\pgfpathmoveto{\pgfqpoint{0.000000in}{0.000000in}}%
\pgfpathlineto{\pgfqpoint{0.000000in}{-0.020833in}}%
\pgfusepath{stroke,fill}%
}%
\begin{pgfscope}%
\pgfsys@transformshift{1.510671in}{0.893003in}%
\pgfsys@useobject{currentmarker}{}%
\end{pgfscope}%
\end{pgfscope}%
\begin{pgfscope}%
\pgfsetbuttcap%
\pgfsetroundjoin%
\definecolor{currentfill}{rgb}{0.000000,0.000000,0.000000}%
\pgfsetfillcolor{currentfill}%
\pgfsetlinewidth{0.501875pt}%
\definecolor{currentstroke}{rgb}{0.000000,0.000000,0.000000}%
\pgfsetstrokecolor{currentstroke}%
\pgfsetdash{}{0pt}%
\pgfsys@defobject{currentmarker}{\pgfqpoint{0.000000in}{0.000000in}}{\pgfqpoint{0.000000in}{0.020833in}}{%
\pgfpathmoveto{\pgfqpoint{0.000000in}{0.000000in}}%
\pgfpathlineto{\pgfqpoint{0.000000in}{0.020833in}}%
\pgfusepath{stroke,fill}%
}%
\begin{pgfscope}%
\pgfsys@transformshift{1.582065in}{0.586309in}%
\pgfsys@useobject{currentmarker}{}%
\end{pgfscope}%
\end{pgfscope}%
\begin{pgfscope}%
\pgfsetbuttcap%
\pgfsetroundjoin%
\definecolor{currentfill}{rgb}{0.000000,0.000000,0.000000}%
\pgfsetfillcolor{currentfill}%
\pgfsetlinewidth{0.501875pt}%
\definecolor{currentstroke}{rgb}{0.000000,0.000000,0.000000}%
\pgfsetstrokecolor{currentstroke}%
\pgfsetdash{}{0pt}%
\pgfsys@defobject{currentmarker}{\pgfqpoint{0.000000in}{-0.020833in}}{\pgfqpoint{0.000000in}{0.000000in}}{%
\pgfpathmoveto{\pgfqpoint{0.000000in}{0.000000in}}%
\pgfpathlineto{\pgfqpoint{0.000000in}{-0.020833in}}%
\pgfusepath{stroke,fill}%
}%
\begin{pgfscope}%
\pgfsys@transformshift{1.582065in}{0.893003in}%
\pgfsys@useobject{currentmarker}{}%
\end{pgfscope}%
\end{pgfscope}%
\begin{pgfscope}%
\pgfsetbuttcap%
\pgfsetroundjoin%
\definecolor{currentfill}{rgb}{0.000000,0.000000,0.000000}%
\pgfsetfillcolor{currentfill}%
\pgfsetlinewidth{0.501875pt}%
\definecolor{currentstroke}{rgb}{0.000000,0.000000,0.000000}%
\pgfsetstrokecolor{currentstroke}%
\pgfsetdash{}{0pt}%
\pgfsys@defobject{currentmarker}{\pgfqpoint{0.000000in}{0.000000in}}{\pgfqpoint{0.000000in}{0.020833in}}{%
\pgfpathmoveto{\pgfqpoint{0.000000in}{0.000000in}}%
\pgfpathlineto{\pgfqpoint{0.000000in}{0.020833in}}%
\pgfusepath{stroke,fill}%
}%
\begin{pgfscope}%
\pgfsys@transformshift{1.617763in}{0.586309in}%
\pgfsys@useobject{currentmarker}{}%
\end{pgfscope}%
\end{pgfscope}%
\begin{pgfscope}%
\pgfsetbuttcap%
\pgfsetroundjoin%
\definecolor{currentfill}{rgb}{0.000000,0.000000,0.000000}%
\pgfsetfillcolor{currentfill}%
\pgfsetlinewidth{0.501875pt}%
\definecolor{currentstroke}{rgb}{0.000000,0.000000,0.000000}%
\pgfsetstrokecolor{currentstroke}%
\pgfsetdash{}{0pt}%
\pgfsys@defobject{currentmarker}{\pgfqpoint{0.000000in}{-0.020833in}}{\pgfqpoint{0.000000in}{0.000000in}}{%
\pgfpathmoveto{\pgfqpoint{0.000000in}{0.000000in}}%
\pgfpathlineto{\pgfqpoint{0.000000in}{-0.020833in}}%
\pgfusepath{stroke,fill}%
}%
\begin{pgfscope}%
\pgfsys@transformshift{1.617763in}{0.893003in}%
\pgfsys@useobject{currentmarker}{}%
\end{pgfscope}%
\end{pgfscope}%
\begin{pgfscope}%
\pgfsetbuttcap%
\pgfsetroundjoin%
\definecolor{currentfill}{rgb}{0.000000,0.000000,0.000000}%
\pgfsetfillcolor{currentfill}%
\pgfsetlinewidth{0.501875pt}%
\definecolor{currentstroke}{rgb}{0.000000,0.000000,0.000000}%
\pgfsetstrokecolor{currentstroke}%
\pgfsetdash{}{0pt}%
\pgfsys@defobject{currentmarker}{\pgfqpoint{0.000000in}{0.000000in}}{\pgfqpoint{0.000000in}{0.020833in}}{%
\pgfpathmoveto{\pgfqpoint{0.000000in}{0.000000in}}%
\pgfpathlineto{\pgfqpoint{0.000000in}{0.020833in}}%
\pgfusepath{stroke,fill}%
}%
\begin{pgfscope}%
\pgfsys@transformshift{1.653460in}{0.586309in}%
\pgfsys@useobject{currentmarker}{}%
\end{pgfscope}%
\end{pgfscope}%
\begin{pgfscope}%
\pgfsetbuttcap%
\pgfsetroundjoin%
\definecolor{currentfill}{rgb}{0.000000,0.000000,0.000000}%
\pgfsetfillcolor{currentfill}%
\pgfsetlinewidth{0.501875pt}%
\definecolor{currentstroke}{rgb}{0.000000,0.000000,0.000000}%
\pgfsetstrokecolor{currentstroke}%
\pgfsetdash{}{0pt}%
\pgfsys@defobject{currentmarker}{\pgfqpoint{0.000000in}{-0.020833in}}{\pgfqpoint{0.000000in}{0.000000in}}{%
\pgfpathmoveto{\pgfqpoint{0.000000in}{0.000000in}}%
\pgfpathlineto{\pgfqpoint{0.000000in}{-0.020833in}}%
\pgfusepath{stroke,fill}%
}%
\begin{pgfscope}%
\pgfsys@transformshift{1.653460in}{0.893003in}%
\pgfsys@useobject{currentmarker}{}%
\end{pgfscope}%
\end{pgfscope}%
\begin{pgfscope}%
\pgfsetbuttcap%
\pgfsetroundjoin%
\definecolor{currentfill}{rgb}{0.000000,0.000000,0.000000}%
\pgfsetfillcolor{currentfill}%
\pgfsetlinewidth{0.501875pt}%
\definecolor{currentstroke}{rgb}{0.000000,0.000000,0.000000}%
\pgfsetstrokecolor{currentstroke}%
\pgfsetdash{}{0pt}%
\pgfsys@defobject{currentmarker}{\pgfqpoint{0.000000in}{0.000000in}}{\pgfqpoint{0.000000in}{0.020833in}}{%
\pgfpathmoveto{\pgfqpoint{0.000000in}{0.000000in}}%
\pgfpathlineto{\pgfqpoint{0.000000in}{0.020833in}}%
\pgfusepath{stroke,fill}%
}%
\begin{pgfscope}%
\pgfsys@transformshift{1.689157in}{0.586309in}%
\pgfsys@useobject{currentmarker}{}%
\end{pgfscope}%
\end{pgfscope}%
\begin{pgfscope}%
\pgfsetbuttcap%
\pgfsetroundjoin%
\definecolor{currentfill}{rgb}{0.000000,0.000000,0.000000}%
\pgfsetfillcolor{currentfill}%
\pgfsetlinewidth{0.501875pt}%
\definecolor{currentstroke}{rgb}{0.000000,0.000000,0.000000}%
\pgfsetstrokecolor{currentstroke}%
\pgfsetdash{}{0pt}%
\pgfsys@defobject{currentmarker}{\pgfqpoint{0.000000in}{-0.020833in}}{\pgfqpoint{0.000000in}{0.000000in}}{%
\pgfpathmoveto{\pgfqpoint{0.000000in}{0.000000in}}%
\pgfpathlineto{\pgfqpoint{0.000000in}{-0.020833in}}%
\pgfusepath{stroke,fill}%
}%
\begin{pgfscope}%
\pgfsys@transformshift{1.689157in}{0.893003in}%
\pgfsys@useobject{currentmarker}{}%
\end{pgfscope}%
\end{pgfscope}%
\begin{pgfscope}%
\pgfsetbuttcap%
\pgfsetroundjoin%
\definecolor{currentfill}{rgb}{0.000000,0.000000,0.000000}%
\pgfsetfillcolor{currentfill}%
\pgfsetlinewidth{0.501875pt}%
\definecolor{currentstroke}{rgb}{0.000000,0.000000,0.000000}%
\pgfsetstrokecolor{currentstroke}%
\pgfsetdash{}{0pt}%
\pgfsys@defobject{currentmarker}{\pgfqpoint{0.000000in}{0.000000in}}{\pgfqpoint{0.000000in}{0.020833in}}{%
\pgfpathmoveto{\pgfqpoint{0.000000in}{0.000000in}}%
\pgfpathlineto{\pgfqpoint{0.000000in}{0.020833in}}%
\pgfusepath{stroke,fill}%
}%
\begin{pgfscope}%
\pgfsys@transformshift{1.724855in}{0.586309in}%
\pgfsys@useobject{currentmarker}{}%
\end{pgfscope}%
\end{pgfscope}%
\begin{pgfscope}%
\pgfsetbuttcap%
\pgfsetroundjoin%
\definecolor{currentfill}{rgb}{0.000000,0.000000,0.000000}%
\pgfsetfillcolor{currentfill}%
\pgfsetlinewidth{0.501875pt}%
\definecolor{currentstroke}{rgb}{0.000000,0.000000,0.000000}%
\pgfsetstrokecolor{currentstroke}%
\pgfsetdash{}{0pt}%
\pgfsys@defobject{currentmarker}{\pgfqpoint{0.000000in}{-0.020833in}}{\pgfqpoint{0.000000in}{0.000000in}}{%
\pgfpathmoveto{\pgfqpoint{0.000000in}{0.000000in}}%
\pgfpathlineto{\pgfqpoint{0.000000in}{-0.020833in}}%
\pgfusepath{stroke,fill}%
}%
\begin{pgfscope}%
\pgfsys@transformshift{1.724855in}{0.893003in}%
\pgfsys@useobject{currentmarker}{}%
\end{pgfscope}%
\end{pgfscope}%
\begin{pgfscope}%
\pgfsetbuttcap%
\pgfsetroundjoin%
\definecolor{currentfill}{rgb}{0.000000,0.000000,0.000000}%
\pgfsetfillcolor{currentfill}%
\pgfsetlinewidth{0.501875pt}%
\definecolor{currentstroke}{rgb}{0.000000,0.000000,0.000000}%
\pgfsetstrokecolor{currentstroke}%
\pgfsetdash{}{0pt}%
\pgfsys@defobject{currentmarker}{\pgfqpoint{0.000000in}{0.000000in}}{\pgfqpoint{0.000000in}{0.020833in}}{%
\pgfpathmoveto{\pgfqpoint{0.000000in}{0.000000in}}%
\pgfpathlineto{\pgfqpoint{0.000000in}{0.020833in}}%
\pgfusepath{stroke,fill}%
}%
\begin{pgfscope}%
\pgfsys@transformshift{1.760552in}{0.586309in}%
\pgfsys@useobject{currentmarker}{}%
\end{pgfscope}%
\end{pgfscope}%
\begin{pgfscope}%
\pgfsetbuttcap%
\pgfsetroundjoin%
\definecolor{currentfill}{rgb}{0.000000,0.000000,0.000000}%
\pgfsetfillcolor{currentfill}%
\pgfsetlinewidth{0.501875pt}%
\definecolor{currentstroke}{rgb}{0.000000,0.000000,0.000000}%
\pgfsetstrokecolor{currentstroke}%
\pgfsetdash{}{0pt}%
\pgfsys@defobject{currentmarker}{\pgfqpoint{0.000000in}{-0.020833in}}{\pgfqpoint{0.000000in}{0.000000in}}{%
\pgfpathmoveto{\pgfqpoint{0.000000in}{0.000000in}}%
\pgfpathlineto{\pgfqpoint{0.000000in}{-0.020833in}}%
\pgfusepath{stroke,fill}%
}%
\begin{pgfscope}%
\pgfsys@transformshift{1.760552in}{0.893003in}%
\pgfsys@useobject{currentmarker}{}%
\end{pgfscope}%
\end{pgfscope}%
\begin{pgfscope}%
\pgfsetbuttcap%
\pgfsetroundjoin%
\definecolor{currentfill}{rgb}{0.000000,0.000000,0.000000}%
\pgfsetfillcolor{currentfill}%
\pgfsetlinewidth{0.501875pt}%
\definecolor{currentstroke}{rgb}{0.000000,0.000000,0.000000}%
\pgfsetstrokecolor{currentstroke}%
\pgfsetdash{}{0pt}%
\pgfsys@defobject{currentmarker}{\pgfqpoint{0.000000in}{0.000000in}}{\pgfqpoint{0.000000in}{0.020833in}}{%
\pgfpathmoveto{\pgfqpoint{0.000000in}{0.000000in}}%
\pgfpathlineto{\pgfqpoint{0.000000in}{0.020833in}}%
\pgfusepath{stroke,fill}%
}%
\begin{pgfscope}%
\pgfsys@transformshift{1.796249in}{0.586309in}%
\pgfsys@useobject{currentmarker}{}%
\end{pgfscope}%
\end{pgfscope}%
\begin{pgfscope}%
\pgfsetbuttcap%
\pgfsetroundjoin%
\definecolor{currentfill}{rgb}{0.000000,0.000000,0.000000}%
\pgfsetfillcolor{currentfill}%
\pgfsetlinewidth{0.501875pt}%
\definecolor{currentstroke}{rgb}{0.000000,0.000000,0.000000}%
\pgfsetstrokecolor{currentstroke}%
\pgfsetdash{}{0pt}%
\pgfsys@defobject{currentmarker}{\pgfqpoint{0.000000in}{-0.020833in}}{\pgfqpoint{0.000000in}{0.000000in}}{%
\pgfpathmoveto{\pgfqpoint{0.000000in}{0.000000in}}%
\pgfpathlineto{\pgfqpoint{0.000000in}{-0.020833in}}%
\pgfusepath{stroke,fill}%
}%
\begin{pgfscope}%
\pgfsys@transformshift{1.796249in}{0.893003in}%
\pgfsys@useobject{currentmarker}{}%
\end{pgfscope}%
\end{pgfscope}%
\begin{pgfscope}%
\pgfsetbuttcap%
\pgfsetroundjoin%
\definecolor{currentfill}{rgb}{0.000000,0.000000,0.000000}%
\pgfsetfillcolor{currentfill}%
\pgfsetlinewidth{0.501875pt}%
\definecolor{currentstroke}{rgb}{0.000000,0.000000,0.000000}%
\pgfsetstrokecolor{currentstroke}%
\pgfsetdash{}{0pt}%
\pgfsys@defobject{currentmarker}{\pgfqpoint{0.000000in}{0.000000in}}{\pgfqpoint{0.000000in}{0.020833in}}{%
\pgfpathmoveto{\pgfqpoint{0.000000in}{0.000000in}}%
\pgfpathlineto{\pgfqpoint{0.000000in}{0.020833in}}%
\pgfusepath{stroke,fill}%
}%
\begin{pgfscope}%
\pgfsys@transformshift{1.831947in}{0.586309in}%
\pgfsys@useobject{currentmarker}{}%
\end{pgfscope}%
\end{pgfscope}%
\begin{pgfscope}%
\pgfsetbuttcap%
\pgfsetroundjoin%
\definecolor{currentfill}{rgb}{0.000000,0.000000,0.000000}%
\pgfsetfillcolor{currentfill}%
\pgfsetlinewidth{0.501875pt}%
\definecolor{currentstroke}{rgb}{0.000000,0.000000,0.000000}%
\pgfsetstrokecolor{currentstroke}%
\pgfsetdash{}{0pt}%
\pgfsys@defobject{currentmarker}{\pgfqpoint{0.000000in}{-0.020833in}}{\pgfqpoint{0.000000in}{0.000000in}}{%
\pgfpathmoveto{\pgfqpoint{0.000000in}{0.000000in}}%
\pgfpathlineto{\pgfqpoint{0.000000in}{-0.020833in}}%
\pgfusepath{stroke,fill}%
}%
\begin{pgfscope}%
\pgfsys@transformshift{1.831947in}{0.893003in}%
\pgfsys@useobject{currentmarker}{}%
\end{pgfscope}%
\end{pgfscope}%
\begin{pgfscope}%
\pgfsetbuttcap%
\pgfsetroundjoin%
\definecolor{currentfill}{rgb}{0.000000,0.000000,0.000000}%
\pgfsetfillcolor{currentfill}%
\pgfsetlinewidth{0.501875pt}%
\definecolor{currentstroke}{rgb}{0.000000,0.000000,0.000000}%
\pgfsetstrokecolor{currentstroke}%
\pgfsetdash{}{0pt}%
\pgfsys@defobject{currentmarker}{\pgfqpoint{0.000000in}{0.000000in}}{\pgfqpoint{0.000000in}{0.020833in}}{%
\pgfpathmoveto{\pgfqpoint{0.000000in}{0.000000in}}%
\pgfpathlineto{\pgfqpoint{0.000000in}{0.020833in}}%
\pgfusepath{stroke,fill}%
}%
\begin{pgfscope}%
\pgfsys@transformshift{1.867644in}{0.586309in}%
\pgfsys@useobject{currentmarker}{}%
\end{pgfscope}%
\end{pgfscope}%
\begin{pgfscope}%
\pgfsetbuttcap%
\pgfsetroundjoin%
\definecolor{currentfill}{rgb}{0.000000,0.000000,0.000000}%
\pgfsetfillcolor{currentfill}%
\pgfsetlinewidth{0.501875pt}%
\definecolor{currentstroke}{rgb}{0.000000,0.000000,0.000000}%
\pgfsetstrokecolor{currentstroke}%
\pgfsetdash{}{0pt}%
\pgfsys@defobject{currentmarker}{\pgfqpoint{0.000000in}{-0.020833in}}{\pgfqpoint{0.000000in}{0.000000in}}{%
\pgfpathmoveto{\pgfqpoint{0.000000in}{0.000000in}}%
\pgfpathlineto{\pgfqpoint{0.000000in}{-0.020833in}}%
\pgfusepath{stroke,fill}%
}%
\begin{pgfscope}%
\pgfsys@transformshift{1.867644in}{0.893003in}%
\pgfsys@useobject{currentmarker}{}%
\end{pgfscope}%
\end{pgfscope}%
\begin{pgfscope}%
\pgfsetbuttcap%
\pgfsetroundjoin%
\definecolor{currentfill}{rgb}{0.000000,0.000000,0.000000}%
\pgfsetfillcolor{currentfill}%
\pgfsetlinewidth{0.501875pt}%
\definecolor{currentstroke}{rgb}{0.000000,0.000000,0.000000}%
\pgfsetstrokecolor{currentstroke}%
\pgfsetdash{}{0pt}%
\pgfsys@defobject{currentmarker}{\pgfqpoint{0.000000in}{0.000000in}}{\pgfqpoint{0.000000in}{0.020833in}}{%
\pgfpathmoveto{\pgfqpoint{0.000000in}{0.000000in}}%
\pgfpathlineto{\pgfqpoint{0.000000in}{0.020833in}}%
\pgfusepath{stroke,fill}%
}%
\begin{pgfscope}%
\pgfsys@transformshift{1.903341in}{0.586309in}%
\pgfsys@useobject{currentmarker}{}%
\end{pgfscope}%
\end{pgfscope}%
\begin{pgfscope}%
\pgfsetbuttcap%
\pgfsetroundjoin%
\definecolor{currentfill}{rgb}{0.000000,0.000000,0.000000}%
\pgfsetfillcolor{currentfill}%
\pgfsetlinewidth{0.501875pt}%
\definecolor{currentstroke}{rgb}{0.000000,0.000000,0.000000}%
\pgfsetstrokecolor{currentstroke}%
\pgfsetdash{}{0pt}%
\pgfsys@defobject{currentmarker}{\pgfqpoint{0.000000in}{-0.020833in}}{\pgfqpoint{0.000000in}{0.000000in}}{%
\pgfpathmoveto{\pgfqpoint{0.000000in}{0.000000in}}%
\pgfpathlineto{\pgfqpoint{0.000000in}{-0.020833in}}%
\pgfusepath{stroke,fill}%
}%
\begin{pgfscope}%
\pgfsys@transformshift{1.903341in}{0.893003in}%
\pgfsys@useobject{currentmarker}{}%
\end{pgfscope}%
\end{pgfscope}%
\begin{pgfscope}%
\pgfsetbuttcap%
\pgfsetroundjoin%
\definecolor{currentfill}{rgb}{0.000000,0.000000,0.000000}%
\pgfsetfillcolor{currentfill}%
\pgfsetlinewidth{0.501875pt}%
\definecolor{currentstroke}{rgb}{0.000000,0.000000,0.000000}%
\pgfsetstrokecolor{currentstroke}%
\pgfsetdash{}{0pt}%
\pgfsys@defobject{currentmarker}{\pgfqpoint{0.000000in}{0.000000in}}{\pgfqpoint{0.000000in}{0.020833in}}{%
\pgfpathmoveto{\pgfqpoint{0.000000in}{0.000000in}}%
\pgfpathlineto{\pgfqpoint{0.000000in}{0.020833in}}%
\pgfusepath{stroke,fill}%
}%
\begin{pgfscope}%
\pgfsys@transformshift{1.939039in}{0.586309in}%
\pgfsys@useobject{currentmarker}{}%
\end{pgfscope}%
\end{pgfscope}%
\begin{pgfscope}%
\pgfsetbuttcap%
\pgfsetroundjoin%
\definecolor{currentfill}{rgb}{0.000000,0.000000,0.000000}%
\pgfsetfillcolor{currentfill}%
\pgfsetlinewidth{0.501875pt}%
\definecolor{currentstroke}{rgb}{0.000000,0.000000,0.000000}%
\pgfsetstrokecolor{currentstroke}%
\pgfsetdash{}{0pt}%
\pgfsys@defobject{currentmarker}{\pgfqpoint{0.000000in}{-0.020833in}}{\pgfqpoint{0.000000in}{0.000000in}}{%
\pgfpathmoveto{\pgfqpoint{0.000000in}{0.000000in}}%
\pgfpathlineto{\pgfqpoint{0.000000in}{-0.020833in}}%
\pgfusepath{stroke,fill}%
}%
\begin{pgfscope}%
\pgfsys@transformshift{1.939039in}{0.893003in}%
\pgfsys@useobject{currentmarker}{}%
\end{pgfscope}%
\end{pgfscope}%
\begin{pgfscope}%
\pgfsetbuttcap%
\pgfsetroundjoin%
\definecolor{currentfill}{rgb}{0.000000,0.000000,0.000000}%
\pgfsetfillcolor{currentfill}%
\pgfsetlinewidth{0.501875pt}%
\definecolor{currentstroke}{rgb}{0.000000,0.000000,0.000000}%
\pgfsetstrokecolor{currentstroke}%
\pgfsetdash{}{0pt}%
\pgfsys@defobject{currentmarker}{\pgfqpoint{0.000000in}{0.000000in}}{\pgfqpoint{0.000000in}{0.020833in}}{%
\pgfpathmoveto{\pgfqpoint{0.000000in}{0.000000in}}%
\pgfpathlineto{\pgfqpoint{0.000000in}{0.020833in}}%
\pgfusepath{stroke,fill}%
}%
\begin{pgfscope}%
\pgfsys@transformshift{2.010433in}{0.586309in}%
\pgfsys@useobject{currentmarker}{}%
\end{pgfscope}%
\end{pgfscope}%
\begin{pgfscope}%
\pgfsetbuttcap%
\pgfsetroundjoin%
\definecolor{currentfill}{rgb}{0.000000,0.000000,0.000000}%
\pgfsetfillcolor{currentfill}%
\pgfsetlinewidth{0.501875pt}%
\definecolor{currentstroke}{rgb}{0.000000,0.000000,0.000000}%
\pgfsetstrokecolor{currentstroke}%
\pgfsetdash{}{0pt}%
\pgfsys@defobject{currentmarker}{\pgfqpoint{0.000000in}{-0.020833in}}{\pgfqpoint{0.000000in}{0.000000in}}{%
\pgfpathmoveto{\pgfqpoint{0.000000in}{0.000000in}}%
\pgfpathlineto{\pgfqpoint{0.000000in}{-0.020833in}}%
\pgfusepath{stroke,fill}%
}%
\begin{pgfscope}%
\pgfsys@transformshift{2.010433in}{0.893003in}%
\pgfsys@useobject{currentmarker}{}%
\end{pgfscope}%
\end{pgfscope}%
\begin{pgfscope}%
\pgfsetbuttcap%
\pgfsetroundjoin%
\definecolor{currentfill}{rgb}{0.000000,0.000000,0.000000}%
\pgfsetfillcolor{currentfill}%
\pgfsetlinewidth{0.501875pt}%
\definecolor{currentstroke}{rgb}{0.000000,0.000000,0.000000}%
\pgfsetstrokecolor{currentstroke}%
\pgfsetdash{}{0pt}%
\pgfsys@defobject{currentmarker}{\pgfqpoint{0.000000in}{0.000000in}}{\pgfqpoint{0.000000in}{0.020833in}}{%
\pgfpathmoveto{\pgfqpoint{0.000000in}{0.000000in}}%
\pgfpathlineto{\pgfqpoint{0.000000in}{0.020833in}}%
\pgfusepath{stroke,fill}%
}%
\begin{pgfscope}%
\pgfsys@transformshift{2.046131in}{0.586309in}%
\pgfsys@useobject{currentmarker}{}%
\end{pgfscope}%
\end{pgfscope}%
\begin{pgfscope}%
\pgfsetbuttcap%
\pgfsetroundjoin%
\definecolor{currentfill}{rgb}{0.000000,0.000000,0.000000}%
\pgfsetfillcolor{currentfill}%
\pgfsetlinewidth{0.501875pt}%
\definecolor{currentstroke}{rgb}{0.000000,0.000000,0.000000}%
\pgfsetstrokecolor{currentstroke}%
\pgfsetdash{}{0pt}%
\pgfsys@defobject{currentmarker}{\pgfqpoint{0.000000in}{-0.020833in}}{\pgfqpoint{0.000000in}{0.000000in}}{%
\pgfpathmoveto{\pgfqpoint{0.000000in}{0.000000in}}%
\pgfpathlineto{\pgfqpoint{0.000000in}{-0.020833in}}%
\pgfusepath{stroke,fill}%
}%
\begin{pgfscope}%
\pgfsys@transformshift{2.046131in}{0.893003in}%
\pgfsys@useobject{currentmarker}{}%
\end{pgfscope}%
\end{pgfscope}%
\begin{pgfscope}%
\pgfsetbuttcap%
\pgfsetroundjoin%
\definecolor{currentfill}{rgb}{0.000000,0.000000,0.000000}%
\pgfsetfillcolor{currentfill}%
\pgfsetlinewidth{0.501875pt}%
\definecolor{currentstroke}{rgb}{0.000000,0.000000,0.000000}%
\pgfsetstrokecolor{currentstroke}%
\pgfsetdash{}{0pt}%
\pgfsys@defobject{currentmarker}{\pgfqpoint{0.000000in}{0.000000in}}{\pgfqpoint{0.000000in}{0.020833in}}{%
\pgfpathmoveto{\pgfqpoint{0.000000in}{0.000000in}}%
\pgfpathlineto{\pgfqpoint{0.000000in}{0.020833in}}%
\pgfusepath{stroke,fill}%
}%
\begin{pgfscope}%
\pgfsys@transformshift{2.081828in}{0.586309in}%
\pgfsys@useobject{currentmarker}{}%
\end{pgfscope}%
\end{pgfscope}%
\begin{pgfscope}%
\pgfsetbuttcap%
\pgfsetroundjoin%
\definecolor{currentfill}{rgb}{0.000000,0.000000,0.000000}%
\pgfsetfillcolor{currentfill}%
\pgfsetlinewidth{0.501875pt}%
\definecolor{currentstroke}{rgb}{0.000000,0.000000,0.000000}%
\pgfsetstrokecolor{currentstroke}%
\pgfsetdash{}{0pt}%
\pgfsys@defobject{currentmarker}{\pgfqpoint{0.000000in}{-0.020833in}}{\pgfqpoint{0.000000in}{0.000000in}}{%
\pgfpathmoveto{\pgfqpoint{0.000000in}{0.000000in}}%
\pgfpathlineto{\pgfqpoint{0.000000in}{-0.020833in}}%
\pgfusepath{stroke,fill}%
}%
\begin{pgfscope}%
\pgfsys@transformshift{2.081828in}{0.893003in}%
\pgfsys@useobject{currentmarker}{}%
\end{pgfscope}%
\end{pgfscope}%
\begin{pgfscope}%
\pgfsetbuttcap%
\pgfsetroundjoin%
\definecolor{currentfill}{rgb}{0.000000,0.000000,0.000000}%
\pgfsetfillcolor{currentfill}%
\pgfsetlinewidth{0.501875pt}%
\definecolor{currentstroke}{rgb}{0.000000,0.000000,0.000000}%
\pgfsetstrokecolor{currentstroke}%
\pgfsetdash{}{0pt}%
\pgfsys@defobject{currentmarker}{\pgfqpoint{0.000000in}{0.000000in}}{\pgfqpoint{0.000000in}{0.020833in}}{%
\pgfpathmoveto{\pgfqpoint{0.000000in}{0.000000in}}%
\pgfpathlineto{\pgfqpoint{0.000000in}{0.020833in}}%
\pgfusepath{stroke,fill}%
}%
\begin{pgfscope}%
\pgfsys@transformshift{2.117525in}{0.586309in}%
\pgfsys@useobject{currentmarker}{}%
\end{pgfscope}%
\end{pgfscope}%
\begin{pgfscope}%
\pgfsetbuttcap%
\pgfsetroundjoin%
\definecolor{currentfill}{rgb}{0.000000,0.000000,0.000000}%
\pgfsetfillcolor{currentfill}%
\pgfsetlinewidth{0.501875pt}%
\definecolor{currentstroke}{rgb}{0.000000,0.000000,0.000000}%
\pgfsetstrokecolor{currentstroke}%
\pgfsetdash{}{0pt}%
\pgfsys@defobject{currentmarker}{\pgfqpoint{0.000000in}{-0.020833in}}{\pgfqpoint{0.000000in}{0.000000in}}{%
\pgfpathmoveto{\pgfqpoint{0.000000in}{0.000000in}}%
\pgfpathlineto{\pgfqpoint{0.000000in}{-0.020833in}}%
\pgfusepath{stroke,fill}%
}%
\begin{pgfscope}%
\pgfsys@transformshift{2.117525in}{0.893003in}%
\pgfsys@useobject{currentmarker}{}%
\end{pgfscope}%
\end{pgfscope}%
\begin{pgfscope}%
\pgfsetbuttcap%
\pgfsetroundjoin%
\definecolor{currentfill}{rgb}{0.000000,0.000000,0.000000}%
\pgfsetfillcolor{currentfill}%
\pgfsetlinewidth{0.501875pt}%
\definecolor{currentstroke}{rgb}{0.000000,0.000000,0.000000}%
\pgfsetstrokecolor{currentstroke}%
\pgfsetdash{}{0pt}%
\pgfsys@defobject{currentmarker}{\pgfqpoint{0.000000in}{0.000000in}}{\pgfqpoint{0.000000in}{0.020833in}}{%
\pgfpathmoveto{\pgfqpoint{0.000000in}{0.000000in}}%
\pgfpathlineto{\pgfqpoint{0.000000in}{0.020833in}}%
\pgfusepath{stroke,fill}%
}%
\begin{pgfscope}%
\pgfsys@transformshift{2.153223in}{0.586309in}%
\pgfsys@useobject{currentmarker}{}%
\end{pgfscope}%
\end{pgfscope}%
\begin{pgfscope}%
\pgfsetbuttcap%
\pgfsetroundjoin%
\definecolor{currentfill}{rgb}{0.000000,0.000000,0.000000}%
\pgfsetfillcolor{currentfill}%
\pgfsetlinewidth{0.501875pt}%
\definecolor{currentstroke}{rgb}{0.000000,0.000000,0.000000}%
\pgfsetstrokecolor{currentstroke}%
\pgfsetdash{}{0pt}%
\pgfsys@defobject{currentmarker}{\pgfqpoint{0.000000in}{-0.020833in}}{\pgfqpoint{0.000000in}{0.000000in}}{%
\pgfpathmoveto{\pgfqpoint{0.000000in}{0.000000in}}%
\pgfpathlineto{\pgfqpoint{0.000000in}{-0.020833in}}%
\pgfusepath{stroke,fill}%
}%
\begin{pgfscope}%
\pgfsys@transformshift{2.153223in}{0.893003in}%
\pgfsys@useobject{currentmarker}{}%
\end{pgfscope}%
\end{pgfscope}%
\begin{pgfscope}%
\pgfsetbuttcap%
\pgfsetroundjoin%
\definecolor{currentfill}{rgb}{0.000000,0.000000,0.000000}%
\pgfsetfillcolor{currentfill}%
\pgfsetlinewidth{0.501875pt}%
\definecolor{currentstroke}{rgb}{0.000000,0.000000,0.000000}%
\pgfsetstrokecolor{currentstroke}%
\pgfsetdash{}{0pt}%
\pgfsys@defobject{currentmarker}{\pgfqpoint{0.000000in}{0.000000in}}{\pgfqpoint{0.000000in}{0.020833in}}{%
\pgfpathmoveto{\pgfqpoint{0.000000in}{0.000000in}}%
\pgfpathlineto{\pgfqpoint{0.000000in}{0.020833in}}%
\pgfusepath{stroke,fill}%
}%
\begin{pgfscope}%
\pgfsys@transformshift{2.188920in}{0.586309in}%
\pgfsys@useobject{currentmarker}{}%
\end{pgfscope}%
\end{pgfscope}%
\begin{pgfscope}%
\pgfsetbuttcap%
\pgfsetroundjoin%
\definecolor{currentfill}{rgb}{0.000000,0.000000,0.000000}%
\pgfsetfillcolor{currentfill}%
\pgfsetlinewidth{0.501875pt}%
\definecolor{currentstroke}{rgb}{0.000000,0.000000,0.000000}%
\pgfsetstrokecolor{currentstroke}%
\pgfsetdash{}{0pt}%
\pgfsys@defobject{currentmarker}{\pgfqpoint{0.000000in}{-0.020833in}}{\pgfqpoint{0.000000in}{0.000000in}}{%
\pgfpathmoveto{\pgfqpoint{0.000000in}{0.000000in}}%
\pgfpathlineto{\pgfqpoint{0.000000in}{-0.020833in}}%
\pgfusepath{stroke,fill}%
}%
\begin{pgfscope}%
\pgfsys@transformshift{2.188920in}{0.893003in}%
\pgfsys@useobject{currentmarker}{}%
\end{pgfscope}%
\end{pgfscope}%
\begin{pgfscope}%
\pgfsetbuttcap%
\pgfsetroundjoin%
\definecolor{currentfill}{rgb}{0.000000,0.000000,0.000000}%
\pgfsetfillcolor{currentfill}%
\pgfsetlinewidth{0.501875pt}%
\definecolor{currentstroke}{rgb}{0.000000,0.000000,0.000000}%
\pgfsetstrokecolor{currentstroke}%
\pgfsetdash{}{0pt}%
\pgfsys@defobject{currentmarker}{\pgfqpoint{0.000000in}{0.000000in}}{\pgfqpoint{0.000000in}{0.020833in}}{%
\pgfpathmoveto{\pgfqpoint{0.000000in}{0.000000in}}%
\pgfpathlineto{\pgfqpoint{0.000000in}{0.020833in}}%
\pgfusepath{stroke,fill}%
}%
\begin{pgfscope}%
\pgfsys@transformshift{2.224617in}{0.586309in}%
\pgfsys@useobject{currentmarker}{}%
\end{pgfscope}%
\end{pgfscope}%
\begin{pgfscope}%
\pgfsetbuttcap%
\pgfsetroundjoin%
\definecolor{currentfill}{rgb}{0.000000,0.000000,0.000000}%
\pgfsetfillcolor{currentfill}%
\pgfsetlinewidth{0.501875pt}%
\definecolor{currentstroke}{rgb}{0.000000,0.000000,0.000000}%
\pgfsetstrokecolor{currentstroke}%
\pgfsetdash{}{0pt}%
\pgfsys@defobject{currentmarker}{\pgfqpoint{0.000000in}{-0.020833in}}{\pgfqpoint{0.000000in}{0.000000in}}{%
\pgfpathmoveto{\pgfqpoint{0.000000in}{0.000000in}}%
\pgfpathlineto{\pgfqpoint{0.000000in}{-0.020833in}}%
\pgfusepath{stroke,fill}%
}%
\begin{pgfscope}%
\pgfsys@transformshift{2.224617in}{0.893003in}%
\pgfsys@useobject{currentmarker}{}%
\end{pgfscope}%
\end{pgfscope}%
\begin{pgfscope}%
\pgfsetbuttcap%
\pgfsetroundjoin%
\definecolor{currentfill}{rgb}{0.000000,0.000000,0.000000}%
\pgfsetfillcolor{currentfill}%
\pgfsetlinewidth{0.501875pt}%
\definecolor{currentstroke}{rgb}{0.000000,0.000000,0.000000}%
\pgfsetstrokecolor{currentstroke}%
\pgfsetdash{}{0pt}%
\pgfsys@defobject{currentmarker}{\pgfqpoint{0.000000in}{0.000000in}}{\pgfqpoint{0.000000in}{0.020833in}}{%
\pgfpathmoveto{\pgfqpoint{0.000000in}{0.000000in}}%
\pgfpathlineto{\pgfqpoint{0.000000in}{0.020833in}}%
\pgfusepath{stroke,fill}%
}%
\begin{pgfscope}%
\pgfsys@transformshift{2.260314in}{0.586309in}%
\pgfsys@useobject{currentmarker}{}%
\end{pgfscope}%
\end{pgfscope}%
\begin{pgfscope}%
\pgfsetbuttcap%
\pgfsetroundjoin%
\definecolor{currentfill}{rgb}{0.000000,0.000000,0.000000}%
\pgfsetfillcolor{currentfill}%
\pgfsetlinewidth{0.501875pt}%
\definecolor{currentstroke}{rgb}{0.000000,0.000000,0.000000}%
\pgfsetstrokecolor{currentstroke}%
\pgfsetdash{}{0pt}%
\pgfsys@defobject{currentmarker}{\pgfqpoint{0.000000in}{-0.020833in}}{\pgfqpoint{0.000000in}{0.000000in}}{%
\pgfpathmoveto{\pgfqpoint{0.000000in}{0.000000in}}%
\pgfpathlineto{\pgfqpoint{0.000000in}{-0.020833in}}%
\pgfusepath{stroke,fill}%
}%
\begin{pgfscope}%
\pgfsys@transformshift{2.260314in}{0.893003in}%
\pgfsys@useobject{currentmarker}{}%
\end{pgfscope}%
\end{pgfscope}%
\begin{pgfscope}%
\pgfsetbuttcap%
\pgfsetroundjoin%
\definecolor{currentfill}{rgb}{0.000000,0.000000,0.000000}%
\pgfsetfillcolor{currentfill}%
\pgfsetlinewidth{0.501875pt}%
\definecolor{currentstroke}{rgb}{0.000000,0.000000,0.000000}%
\pgfsetstrokecolor{currentstroke}%
\pgfsetdash{}{0pt}%
\pgfsys@defobject{currentmarker}{\pgfqpoint{0.000000in}{0.000000in}}{\pgfqpoint{0.000000in}{0.020833in}}{%
\pgfpathmoveto{\pgfqpoint{0.000000in}{0.000000in}}%
\pgfpathlineto{\pgfqpoint{0.000000in}{0.020833in}}%
\pgfusepath{stroke,fill}%
}%
\begin{pgfscope}%
\pgfsys@transformshift{2.296012in}{0.586309in}%
\pgfsys@useobject{currentmarker}{}%
\end{pgfscope}%
\end{pgfscope}%
\begin{pgfscope}%
\pgfsetbuttcap%
\pgfsetroundjoin%
\definecolor{currentfill}{rgb}{0.000000,0.000000,0.000000}%
\pgfsetfillcolor{currentfill}%
\pgfsetlinewidth{0.501875pt}%
\definecolor{currentstroke}{rgb}{0.000000,0.000000,0.000000}%
\pgfsetstrokecolor{currentstroke}%
\pgfsetdash{}{0pt}%
\pgfsys@defobject{currentmarker}{\pgfqpoint{0.000000in}{-0.020833in}}{\pgfqpoint{0.000000in}{0.000000in}}{%
\pgfpathmoveto{\pgfqpoint{0.000000in}{0.000000in}}%
\pgfpathlineto{\pgfqpoint{0.000000in}{-0.020833in}}%
\pgfusepath{stroke,fill}%
}%
\begin{pgfscope}%
\pgfsys@transformshift{2.296012in}{0.893003in}%
\pgfsys@useobject{currentmarker}{}%
\end{pgfscope}%
\end{pgfscope}%
\begin{pgfscope}%
\pgfsetbuttcap%
\pgfsetroundjoin%
\definecolor{currentfill}{rgb}{0.000000,0.000000,0.000000}%
\pgfsetfillcolor{currentfill}%
\pgfsetlinewidth{0.501875pt}%
\definecolor{currentstroke}{rgb}{0.000000,0.000000,0.000000}%
\pgfsetstrokecolor{currentstroke}%
\pgfsetdash{}{0pt}%
\pgfsys@defobject{currentmarker}{\pgfqpoint{0.000000in}{0.000000in}}{\pgfqpoint{0.000000in}{0.020833in}}{%
\pgfpathmoveto{\pgfqpoint{0.000000in}{0.000000in}}%
\pgfpathlineto{\pgfqpoint{0.000000in}{0.020833in}}%
\pgfusepath{stroke,fill}%
}%
\begin{pgfscope}%
\pgfsys@transformshift{2.331709in}{0.586309in}%
\pgfsys@useobject{currentmarker}{}%
\end{pgfscope}%
\end{pgfscope}%
\begin{pgfscope}%
\pgfsetbuttcap%
\pgfsetroundjoin%
\definecolor{currentfill}{rgb}{0.000000,0.000000,0.000000}%
\pgfsetfillcolor{currentfill}%
\pgfsetlinewidth{0.501875pt}%
\definecolor{currentstroke}{rgb}{0.000000,0.000000,0.000000}%
\pgfsetstrokecolor{currentstroke}%
\pgfsetdash{}{0pt}%
\pgfsys@defobject{currentmarker}{\pgfqpoint{0.000000in}{-0.020833in}}{\pgfqpoint{0.000000in}{0.000000in}}{%
\pgfpathmoveto{\pgfqpoint{0.000000in}{0.000000in}}%
\pgfpathlineto{\pgfqpoint{0.000000in}{-0.020833in}}%
\pgfusepath{stroke,fill}%
}%
\begin{pgfscope}%
\pgfsys@transformshift{2.331709in}{0.893003in}%
\pgfsys@useobject{currentmarker}{}%
\end{pgfscope}%
\end{pgfscope}%
\begin{pgfscope}%
\pgfsetbuttcap%
\pgfsetroundjoin%
\definecolor{currentfill}{rgb}{0.000000,0.000000,0.000000}%
\pgfsetfillcolor{currentfill}%
\pgfsetlinewidth{0.501875pt}%
\definecolor{currentstroke}{rgb}{0.000000,0.000000,0.000000}%
\pgfsetstrokecolor{currentstroke}%
\pgfsetdash{}{0pt}%
\pgfsys@defobject{currentmarker}{\pgfqpoint{0.000000in}{0.000000in}}{\pgfqpoint{0.000000in}{0.020833in}}{%
\pgfpathmoveto{\pgfqpoint{0.000000in}{0.000000in}}%
\pgfpathlineto{\pgfqpoint{0.000000in}{0.020833in}}%
\pgfusepath{stroke,fill}%
}%
\begin{pgfscope}%
\pgfsys@transformshift{2.367406in}{0.586309in}%
\pgfsys@useobject{currentmarker}{}%
\end{pgfscope}%
\end{pgfscope}%
\begin{pgfscope}%
\pgfsetbuttcap%
\pgfsetroundjoin%
\definecolor{currentfill}{rgb}{0.000000,0.000000,0.000000}%
\pgfsetfillcolor{currentfill}%
\pgfsetlinewidth{0.501875pt}%
\definecolor{currentstroke}{rgb}{0.000000,0.000000,0.000000}%
\pgfsetstrokecolor{currentstroke}%
\pgfsetdash{}{0pt}%
\pgfsys@defobject{currentmarker}{\pgfqpoint{0.000000in}{-0.020833in}}{\pgfqpoint{0.000000in}{0.000000in}}{%
\pgfpathmoveto{\pgfqpoint{0.000000in}{0.000000in}}%
\pgfpathlineto{\pgfqpoint{0.000000in}{-0.020833in}}%
\pgfusepath{stroke,fill}%
}%
\begin{pgfscope}%
\pgfsys@transformshift{2.367406in}{0.893003in}%
\pgfsys@useobject{currentmarker}{}%
\end{pgfscope}%
\end{pgfscope}%
\begin{pgfscope}%
\pgfsetbuttcap%
\pgfsetroundjoin%
\definecolor{currentfill}{rgb}{0.000000,0.000000,0.000000}%
\pgfsetfillcolor{currentfill}%
\pgfsetlinewidth{0.501875pt}%
\definecolor{currentstroke}{rgb}{0.000000,0.000000,0.000000}%
\pgfsetstrokecolor{currentstroke}%
\pgfsetdash{}{0pt}%
\pgfsys@defobject{currentmarker}{\pgfqpoint{0.000000in}{0.000000in}}{\pgfqpoint{0.000000in}{0.020833in}}{%
\pgfpathmoveto{\pgfqpoint{0.000000in}{0.000000in}}%
\pgfpathlineto{\pgfqpoint{0.000000in}{0.020833in}}%
\pgfusepath{stroke,fill}%
}%
\begin{pgfscope}%
\pgfsys@transformshift{2.438801in}{0.586309in}%
\pgfsys@useobject{currentmarker}{}%
\end{pgfscope}%
\end{pgfscope}%
\begin{pgfscope}%
\pgfsetbuttcap%
\pgfsetroundjoin%
\definecolor{currentfill}{rgb}{0.000000,0.000000,0.000000}%
\pgfsetfillcolor{currentfill}%
\pgfsetlinewidth{0.501875pt}%
\definecolor{currentstroke}{rgb}{0.000000,0.000000,0.000000}%
\pgfsetstrokecolor{currentstroke}%
\pgfsetdash{}{0pt}%
\pgfsys@defobject{currentmarker}{\pgfqpoint{0.000000in}{-0.020833in}}{\pgfqpoint{0.000000in}{0.000000in}}{%
\pgfpathmoveto{\pgfqpoint{0.000000in}{0.000000in}}%
\pgfpathlineto{\pgfqpoint{0.000000in}{-0.020833in}}%
\pgfusepath{stroke,fill}%
}%
\begin{pgfscope}%
\pgfsys@transformshift{2.438801in}{0.893003in}%
\pgfsys@useobject{currentmarker}{}%
\end{pgfscope}%
\end{pgfscope}%
\begin{pgfscope}%
\pgfsetbuttcap%
\pgfsetroundjoin%
\definecolor{currentfill}{rgb}{0.000000,0.000000,0.000000}%
\pgfsetfillcolor{currentfill}%
\pgfsetlinewidth{0.501875pt}%
\definecolor{currentstroke}{rgb}{0.000000,0.000000,0.000000}%
\pgfsetstrokecolor{currentstroke}%
\pgfsetdash{}{0pt}%
\pgfsys@defobject{currentmarker}{\pgfqpoint{0.000000in}{0.000000in}}{\pgfqpoint{0.000000in}{0.020833in}}{%
\pgfpathmoveto{\pgfqpoint{0.000000in}{0.000000in}}%
\pgfpathlineto{\pgfqpoint{0.000000in}{0.020833in}}%
\pgfusepath{stroke,fill}%
}%
\begin{pgfscope}%
\pgfsys@transformshift{2.474498in}{0.586309in}%
\pgfsys@useobject{currentmarker}{}%
\end{pgfscope}%
\end{pgfscope}%
\begin{pgfscope}%
\pgfsetbuttcap%
\pgfsetroundjoin%
\definecolor{currentfill}{rgb}{0.000000,0.000000,0.000000}%
\pgfsetfillcolor{currentfill}%
\pgfsetlinewidth{0.501875pt}%
\definecolor{currentstroke}{rgb}{0.000000,0.000000,0.000000}%
\pgfsetstrokecolor{currentstroke}%
\pgfsetdash{}{0pt}%
\pgfsys@defobject{currentmarker}{\pgfqpoint{0.000000in}{-0.020833in}}{\pgfqpoint{0.000000in}{0.000000in}}{%
\pgfpathmoveto{\pgfqpoint{0.000000in}{0.000000in}}%
\pgfpathlineto{\pgfqpoint{0.000000in}{-0.020833in}}%
\pgfusepath{stroke,fill}%
}%
\begin{pgfscope}%
\pgfsys@transformshift{2.474498in}{0.893003in}%
\pgfsys@useobject{currentmarker}{}%
\end{pgfscope}%
\end{pgfscope}%
\begin{pgfscope}%
\pgfsetbuttcap%
\pgfsetroundjoin%
\definecolor{currentfill}{rgb}{0.000000,0.000000,0.000000}%
\pgfsetfillcolor{currentfill}%
\pgfsetlinewidth{0.501875pt}%
\definecolor{currentstroke}{rgb}{0.000000,0.000000,0.000000}%
\pgfsetstrokecolor{currentstroke}%
\pgfsetdash{}{0pt}%
\pgfsys@defobject{currentmarker}{\pgfqpoint{0.000000in}{0.000000in}}{\pgfqpoint{0.000000in}{0.020833in}}{%
\pgfpathmoveto{\pgfqpoint{0.000000in}{0.000000in}}%
\pgfpathlineto{\pgfqpoint{0.000000in}{0.020833in}}%
\pgfusepath{stroke,fill}%
}%
\begin{pgfscope}%
\pgfsys@transformshift{2.510196in}{0.586309in}%
\pgfsys@useobject{currentmarker}{}%
\end{pgfscope}%
\end{pgfscope}%
\begin{pgfscope}%
\pgfsetbuttcap%
\pgfsetroundjoin%
\definecolor{currentfill}{rgb}{0.000000,0.000000,0.000000}%
\pgfsetfillcolor{currentfill}%
\pgfsetlinewidth{0.501875pt}%
\definecolor{currentstroke}{rgb}{0.000000,0.000000,0.000000}%
\pgfsetstrokecolor{currentstroke}%
\pgfsetdash{}{0pt}%
\pgfsys@defobject{currentmarker}{\pgfqpoint{0.000000in}{-0.020833in}}{\pgfqpoint{0.000000in}{0.000000in}}{%
\pgfpathmoveto{\pgfqpoint{0.000000in}{0.000000in}}%
\pgfpathlineto{\pgfqpoint{0.000000in}{-0.020833in}}%
\pgfusepath{stroke,fill}%
}%
\begin{pgfscope}%
\pgfsys@transformshift{2.510196in}{0.893003in}%
\pgfsys@useobject{currentmarker}{}%
\end{pgfscope}%
\end{pgfscope}%
\begin{pgfscope}%
\pgfsetbuttcap%
\pgfsetroundjoin%
\definecolor{currentfill}{rgb}{0.000000,0.000000,0.000000}%
\pgfsetfillcolor{currentfill}%
\pgfsetlinewidth{0.501875pt}%
\definecolor{currentstroke}{rgb}{0.000000,0.000000,0.000000}%
\pgfsetstrokecolor{currentstroke}%
\pgfsetdash{}{0pt}%
\pgfsys@defobject{currentmarker}{\pgfqpoint{0.000000in}{0.000000in}}{\pgfqpoint{0.000000in}{0.020833in}}{%
\pgfpathmoveto{\pgfqpoint{0.000000in}{0.000000in}}%
\pgfpathlineto{\pgfqpoint{0.000000in}{0.020833in}}%
\pgfusepath{stroke,fill}%
}%
\begin{pgfscope}%
\pgfsys@transformshift{2.545893in}{0.586309in}%
\pgfsys@useobject{currentmarker}{}%
\end{pgfscope}%
\end{pgfscope}%
\begin{pgfscope}%
\pgfsetbuttcap%
\pgfsetroundjoin%
\definecolor{currentfill}{rgb}{0.000000,0.000000,0.000000}%
\pgfsetfillcolor{currentfill}%
\pgfsetlinewidth{0.501875pt}%
\definecolor{currentstroke}{rgb}{0.000000,0.000000,0.000000}%
\pgfsetstrokecolor{currentstroke}%
\pgfsetdash{}{0pt}%
\pgfsys@defobject{currentmarker}{\pgfqpoint{0.000000in}{-0.020833in}}{\pgfqpoint{0.000000in}{0.000000in}}{%
\pgfpathmoveto{\pgfqpoint{0.000000in}{0.000000in}}%
\pgfpathlineto{\pgfqpoint{0.000000in}{-0.020833in}}%
\pgfusepath{stroke,fill}%
}%
\begin{pgfscope}%
\pgfsys@transformshift{2.545893in}{0.893003in}%
\pgfsys@useobject{currentmarker}{}%
\end{pgfscope}%
\end{pgfscope}%
\begin{pgfscope}%
\pgfsetbuttcap%
\pgfsetroundjoin%
\definecolor{currentfill}{rgb}{0.000000,0.000000,0.000000}%
\pgfsetfillcolor{currentfill}%
\pgfsetlinewidth{0.501875pt}%
\definecolor{currentstroke}{rgb}{0.000000,0.000000,0.000000}%
\pgfsetstrokecolor{currentstroke}%
\pgfsetdash{}{0pt}%
\pgfsys@defobject{currentmarker}{\pgfqpoint{0.000000in}{0.000000in}}{\pgfqpoint{0.000000in}{0.020833in}}{%
\pgfpathmoveto{\pgfqpoint{0.000000in}{0.000000in}}%
\pgfpathlineto{\pgfqpoint{0.000000in}{0.020833in}}%
\pgfusepath{stroke,fill}%
}%
\begin{pgfscope}%
\pgfsys@transformshift{2.581590in}{0.586309in}%
\pgfsys@useobject{currentmarker}{}%
\end{pgfscope}%
\end{pgfscope}%
\begin{pgfscope}%
\pgfsetbuttcap%
\pgfsetroundjoin%
\definecolor{currentfill}{rgb}{0.000000,0.000000,0.000000}%
\pgfsetfillcolor{currentfill}%
\pgfsetlinewidth{0.501875pt}%
\definecolor{currentstroke}{rgb}{0.000000,0.000000,0.000000}%
\pgfsetstrokecolor{currentstroke}%
\pgfsetdash{}{0pt}%
\pgfsys@defobject{currentmarker}{\pgfqpoint{0.000000in}{-0.020833in}}{\pgfqpoint{0.000000in}{0.000000in}}{%
\pgfpathmoveto{\pgfqpoint{0.000000in}{0.000000in}}%
\pgfpathlineto{\pgfqpoint{0.000000in}{-0.020833in}}%
\pgfusepath{stroke,fill}%
}%
\begin{pgfscope}%
\pgfsys@transformshift{2.581590in}{0.893003in}%
\pgfsys@useobject{currentmarker}{}%
\end{pgfscope}%
\end{pgfscope}%
\begin{pgfscope}%
\pgfsetbuttcap%
\pgfsetroundjoin%
\definecolor{currentfill}{rgb}{0.000000,0.000000,0.000000}%
\pgfsetfillcolor{currentfill}%
\pgfsetlinewidth{0.501875pt}%
\definecolor{currentstroke}{rgb}{0.000000,0.000000,0.000000}%
\pgfsetstrokecolor{currentstroke}%
\pgfsetdash{}{0pt}%
\pgfsys@defobject{currentmarker}{\pgfqpoint{0.000000in}{0.000000in}}{\pgfqpoint{0.000000in}{0.020833in}}{%
\pgfpathmoveto{\pgfqpoint{0.000000in}{0.000000in}}%
\pgfpathlineto{\pgfqpoint{0.000000in}{0.020833in}}%
\pgfusepath{stroke,fill}%
}%
\begin{pgfscope}%
\pgfsys@transformshift{2.617288in}{0.586309in}%
\pgfsys@useobject{currentmarker}{}%
\end{pgfscope}%
\end{pgfscope}%
\begin{pgfscope}%
\pgfsetbuttcap%
\pgfsetroundjoin%
\definecolor{currentfill}{rgb}{0.000000,0.000000,0.000000}%
\pgfsetfillcolor{currentfill}%
\pgfsetlinewidth{0.501875pt}%
\definecolor{currentstroke}{rgb}{0.000000,0.000000,0.000000}%
\pgfsetstrokecolor{currentstroke}%
\pgfsetdash{}{0pt}%
\pgfsys@defobject{currentmarker}{\pgfqpoint{0.000000in}{-0.020833in}}{\pgfqpoint{0.000000in}{0.000000in}}{%
\pgfpathmoveto{\pgfqpoint{0.000000in}{0.000000in}}%
\pgfpathlineto{\pgfqpoint{0.000000in}{-0.020833in}}%
\pgfusepath{stroke,fill}%
}%
\begin{pgfscope}%
\pgfsys@transformshift{2.617288in}{0.893003in}%
\pgfsys@useobject{currentmarker}{}%
\end{pgfscope}%
\end{pgfscope}%
\begin{pgfscope}%
\pgfsetbuttcap%
\pgfsetroundjoin%
\definecolor{currentfill}{rgb}{0.000000,0.000000,0.000000}%
\pgfsetfillcolor{currentfill}%
\pgfsetlinewidth{0.501875pt}%
\definecolor{currentstroke}{rgb}{0.000000,0.000000,0.000000}%
\pgfsetstrokecolor{currentstroke}%
\pgfsetdash{}{0pt}%
\pgfsys@defobject{currentmarker}{\pgfqpoint{0.000000in}{0.000000in}}{\pgfqpoint{0.000000in}{0.020833in}}{%
\pgfpathmoveto{\pgfqpoint{0.000000in}{0.000000in}}%
\pgfpathlineto{\pgfqpoint{0.000000in}{0.020833in}}%
\pgfusepath{stroke,fill}%
}%
\begin{pgfscope}%
\pgfsys@transformshift{2.652985in}{0.586309in}%
\pgfsys@useobject{currentmarker}{}%
\end{pgfscope}%
\end{pgfscope}%
\begin{pgfscope}%
\pgfsetbuttcap%
\pgfsetroundjoin%
\definecolor{currentfill}{rgb}{0.000000,0.000000,0.000000}%
\pgfsetfillcolor{currentfill}%
\pgfsetlinewidth{0.501875pt}%
\definecolor{currentstroke}{rgb}{0.000000,0.000000,0.000000}%
\pgfsetstrokecolor{currentstroke}%
\pgfsetdash{}{0pt}%
\pgfsys@defobject{currentmarker}{\pgfqpoint{0.000000in}{-0.020833in}}{\pgfqpoint{0.000000in}{0.000000in}}{%
\pgfpathmoveto{\pgfqpoint{0.000000in}{0.000000in}}%
\pgfpathlineto{\pgfqpoint{0.000000in}{-0.020833in}}%
\pgfusepath{stroke,fill}%
}%
\begin{pgfscope}%
\pgfsys@transformshift{2.652985in}{0.893003in}%
\pgfsys@useobject{currentmarker}{}%
\end{pgfscope}%
\end{pgfscope}%
\begin{pgfscope}%
\pgfsetbuttcap%
\pgfsetroundjoin%
\definecolor{currentfill}{rgb}{0.000000,0.000000,0.000000}%
\pgfsetfillcolor{currentfill}%
\pgfsetlinewidth{0.501875pt}%
\definecolor{currentstroke}{rgb}{0.000000,0.000000,0.000000}%
\pgfsetstrokecolor{currentstroke}%
\pgfsetdash{}{0pt}%
\pgfsys@defobject{currentmarker}{\pgfqpoint{0.000000in}{0.000000in}}{\pgfqpoint{0.000000in}{0.020833in}}{%
\pgfpathmoveto{\pgfqpoint{0.000000in}{0.000000in}}%
\pgfpathlineto{\pgfqpoint{0.000000in}{0.020833in}}%
\pgfusepath{stroke,fill}%
}%
\begin{pgfscope}%
\pgfsys@transformshift{2.688682in}{0.586309in}%
\pgfsys@useobject{currentmarker}{}%
\end{pgfscope}%
\end{pgfscope}%
\begin{pgfscope}%
\pgfsetbuttcap%
\pgfsetroundjoin%
\definecolor{currentfill}{rgb}{0.000000,0.000000,0.000000}%
\pgfsetfillcolor{currentfill}%
\pgfsetlinewidth{0.501875pt}%
\definecolor{currentstroke}{rgb}{0.000000,0.000000,0.000000}%
\pgfsetstrokecolor{currentstroke}%
\pgfsetdash{}{0pt}%
\pgfsys@defobject{currentmarker}{\pgfqpoint{0.000000in}{-0.020833in}}{\pgfqpoint{0.000000in}{0.000000in}}{%
\pgfpathmoveto{\pgfqpoint{0.000000in}{0.000000in}}%
\pgfpathlineto{\pgfqpoint{0.000000in}{-0.020833in}}%
\pgfusepath{stroke,fill}%
}%
\begin{pgfscope}%
\pgfsys@transformshift{2.688682in}{0.893003in}%
\pgfsys@useobject{currentmarker}{}%
\end{pgfscope}%
\end{pgfscope}%
\begin{pgfscope}%
\pgfsetbuttcap%
\pgfsetroundjoin%
\definecolor{currentfill}{rgb}{0.000000,0.000000,0.000000}%
\pgfsetfillcolor{currentfill}%
\pgfsetlinewidth{0.501875pt}%
\definecolor{currentstroke}{rgb}{0.000000,0.000000,0.000000}%
\pgfsetstrokecolor{currentstroke}%
\pgfsetdash{}{0pt}%
\pgfsys@defobject{currentmarker}{\pgfqpoint{0.000000in}{0.000000in}}{\pgfqpoint{0.000000in}{0.020833in}}{%
\pgfpathmoveto{\pgfqpoint{0.000000in}{0.000000in}}%
\pgfpathlineto{\pgfqpoint{0.000000in}{0.020833in}}%
\pgfusepath{stroke,fill}%
}%
\begin{pgfscope}%
\pgfsys@transformshift{2.724380in}{0.586309in}%
\pgfsys@useobject{currentmarker}{}%
\end{pgfscope}%
\end{pgfscope}%
\begin{pgfscope}%
\pgfsetbuttcap%
\pgfsetroundjoin%
\definecolor{currentfill}{rgb}{0.000000,0.000000,0.000000}%
\pgfsetfillcolor{currentfill}%
\pgfsetlinewidth{0.501875pt}%
\definecolor{currentstroke}{rgb}{0.000000,0.000000,0.000000}%
\pgfsetstrokecolor{currentstroke}%
\pgfsetdash{}{0pt}%
\pgfsys@defobject{currentmarker}{\pgfqpoint{0.000000in}{-0.020833in}}{\pgfqpoint{0.000000in}{0.000000in}}{%
\pgfpathmoveto{\pgfqpoint{0.000000in}{0.000000in}}%
\pgfpathlineto{\pgfqpoint{0.000000in}{-0.020833in}}%
\pgfusepath{stroke,fill}%
}%
\begin{pgfscope}%
\pgfsys@transformshift{2.724380in}{0.893003in}%
\pgfsys@useobject{currentmarker}{}%
\end{pgfscope}%
\end{pgfscope}%
\begin{pgfscope}%
\pgfsetbuttcap%
\pgfsetroundjoin%
\definecolor{currentfill}{rgb}{0.000000,0.000000,0.000000}%
\pgfsetfillcolor{currentfill}%
\pgfsetlinewidth{0.501875pt}%
\definecolor{currentstroke}{rgb}{0.000000,0.000000,0.000000}%
\pgfsetstrokecolor{currentstroke}%
\pgfsetdash{}{0pt}%
\pgfsys@defobject{currentmarker}{\pgfqpoint{0.000000in}{0.000000in}}{\pgfqpoint{0.000000in}{0.020833in}}{%
\pgfpathmoveto{\pgfqpoint{0.000000in}{0.000000in}}%
\pgfpathlineto{\pgfqpoint{0.000000in}{0.020833in}}%
\pgfusepath{stroke,fill}%
}%
\begin{pgfscope}%
\pgfsys@transformshift{2.760077in}{0.586309in}%
\pgfsys@useobject{currentmarker}{}%
\end{pgfscope}%
\end{pgfscope}%
\begin{pgfscope}%
\pgfsetbuttcap%
\pgfsetroundjoin%
\definecolor{currentfill}{rgb}{0.000000,0.000000,0.000000}%
\pgfsetfillcolor{currentfill}%
\pgfsetlinewidth{0.501875pt}%
\definecolor{currentstroke}{rgb}{0.000000,0.000000,0.000000}%
\pgfsetstrokecolor{currentstroke}%
\pgfsetdash{}{0pt}%
\pgfsys@defobject{currentmarker}{\pgfqpoint{0.000000in}{-0.020833in}}{\pgfqpoint{0.000000in}{0.000000in}}{%
\pgfpathmoveto{\pgfqpoint{0.000000in}{0.000000in}}%
\pgfpathlineto{\pgfqpoint{0.000000in}{-0.020833in}}%
\pgfusepath{stroke,fill}%
}%
\begin{pgfscope}%
\pgfsys@transformshift{2.760077in}{0.893003in}%
\pgfsys@useobject{currentmarker}{}%
\end{pgfscope}%
\end{pgfscope}%
\begin{pgfscope}%
\pgfsetbuttcap%
\pgfsetroundjoin%
\definecolor{currentfill}{rgb}{0.000000,0.000000,0.000000}%
\pgfsetfillcolor{currentfill}%
\pgfsetlinewidth{0.501875pt}%
\definecolor{currentstroke}{rgb}{0.000000,0.000000,0.000000}%
\pgfsetstrokecolor{currentstroke}%
\pgfsetdash{}{0pt}%
\pgfsys@defobject{currentmarker}{\pgfqpoint{0.000000in}{0.000000in}}{\pgfqpoint{0.000000in}{0.020833in}}{%
\pgfpathmoveto{\pgfqpoint{0.000000in}{0.000000in}}%
\pgfpathlineto{\pgfqpoint{0.000000in}{0.020833in}}%
\pgfusepath{stroke,fill}%
}%
\begin{pgfscope}%
\pgfsys@transformshift{2.795774in}{0.586309in}%
\pgfsys@useobject{currentmarker}{}%
\end{pgfscope}%
\end{pgfscope}%
\begin{pgfscope}%
\pgfsetbuttcap%
\pgfsetroundjoin%
\definecolor{currentfill}{rgb}{0.000000,0.000000,0.000000}%
\pgfsetfillcolor{currentfill}%
\pgfsetlinewidth{0.501875pt}%
\definecolor{currentstroke}{rgb}{0.000000,0.000000,0.000000}%
\pgfsetstrokecolor{currentstroke}%
\pgfsetdash{}{0pt}%
\pgfsys@defobject{currentmarker}{\pgfqpoint{0.000000in}{-0.020833in}}{\pgfqpoint{0.000000in}{0.000000in}}{%
\pgfpathmoveto{\pgfqpoint{0.000000in}{0.000000in}}%
\pgfpathlineto{\pgfqpoint{0.000000in}{-0.020833in}}%
\pgfusepath{stroke,fill}%
}%
\begin{pgfscope}%
\pgfsys@transformshift{2.795774in}{0.893003in}%
\pgfsys@useobject{currentmarker}{}%
\end{pgfscope}%
\end{pgfscope}%
\begin{pgfscope}%
\pgfsetbuttcap%
\pgfsetroundjoin%
\definecolor{currentfill}{rgb}{0.000000,0.000000,0.000000}%
\pgfsetfillcolor{currentfill}%
\pgfsetlinewidth{0.501875pt}%
\definecolor{currentstroke}{rgb}{0.000000,0.000000,0.000000}%
\pgfsetstrokecolor{currentstroke}%
\pgfsetdash{}{0pt}%
\pgfsys@defobject{currentmarker}{\pgfqpoint{0.000000in}{0.000000in}}{\pgfqpoint{0.000000in}{0.020833in}}{%
\pgfpathmoveto{\pgfqpoint{0.000000in}{0.000000in}}%
\pgfpathlineto{\pgfqpoint{0.000000in}{0.020833in}}%
\pgfusepath{stroke,fill}%
}%
\begin{pgfscope}%
\pgfsys@transformshift{2.867169in}{0.586309in}%
\pgfsys@useobject{currentmarker}{}%
\end{pgfscope}%
\end{pgfscope}%
\begin{pgfscope}%
\pgfsetbuttcap%
\pgfsetroundjoin%
\definecolor{currentfill}{rgb}{0.000000,0.000000,0.000000}%
\pgfsetfillcolor{currentfill}%
\pgfsetlinewidth{0.501875pt}%
\definecolor{currentstroke}{rgb}{0.000000,0.000000,0.000000}%
\pgfsetstrokecolor{currentstroke}%
\pgfsetdash{}{0pt}%
\pgfsys@defobject{currentmarker}{\pgfqpoint{0.000000in}{-0.020833in}}{\pgfqpoint{0.000000in}{0.000000in}}{%
\pgfpathmoveto{\pgfqpoint{0.000000in}{0.000000in}}%
\pgfpathlineto{\pgfqpoint{0.000000in}{-0.020833in}}%
\pgfusepath{stroke,fill}%
}%
\begin{pgfscope}%
\pgfsys@transformshift{2.867169in}{0.893003in}%
\pgfsys@useobject{currentmarker}{}%
\end{pgfscope}%
\end{pgfscope}%
\begin{pgfscope}%
\pgfsetbuttcap%
\pgfsetroundjoin%
\definecolor{currentfill}{rgb}{0.000000,0.000000,0.000000}%
\pgfsetfillcolor{currentfill}%
\pgfsetlinewidth{0.501875pt}%
\definecolor{currentstroke}{rgb}{0.000000,0.000000,0.000000}%
\pgfsetstrokecolor{currentstroke}%
\pgfsetdash{}{0pt}%
\pgfsys@defobject{currentmarker}{\pgfqpoint{0.000000in}{0.000000in}}{\pgfqpoint{0.000000in}{0.020833in}}{%
\pgfpathmoveto{\pgfqpoint{0.000000in}{0.000000in}}%
\pgfpathlineto{\pgfqpoint{0.000000in}{0.020833in}}%
\pgfusepath{stroke,fill}%
}%
\begin{pgfscope}%
\pgfsys@transformshift{2.902866in}{0.586309in}%
\pgfsys@useobject{currentmarker}{}%
\end{pgfscope}%
\end{pgfscope}%
\begin{pgfscope}%
\pgfsetbuttcap%
\pgfsetroundjoin%
\definecolor{currentfill}{rgb}{0.000000,0.000000,0.000000}%
\pgfsetfillcolor{currentfill}%
\pgfsetlinewidth{0.501875pt}%
\definecolor{currentstroke}{rgb}{0.000000,0.000000,0.000000}%
\pgfsetstrokecolor{currentstroke}%
\pgfsetdash{}{0pt}%
\pgfsys@defobject{currentmarker}{\pgfqpoint{0.000000in}{-0.020833in}}{\pgfqpoint{0.000000in}{0.000000in}}{%
\pgfpathmoveto{\pgfqpoint{0.000000in}{0.000000in}}%
\pgfpathlineto{\pgfqpoint{0.000000in}{-0.020833in}}%
\pgfusepath{stroke,fill}%
}%
\begin{pgfscope}%
\pgfsys@transformshift{2.902866in}{0.893003in}%
\pgfsys@useobject{currentmarker}{}%
\end{pgfscope}%
\end{pgfscope}%
\begin{pgfscope}%
\pgfsetbuttcap%
\pgfsetroundjoin%
\definecolor{currentfill}{rgb}{0.000000,0.000000,0.000000}%
\pgfsetfillcolor{currentfill}%
\pgfsetlinewidth{0.501875pt}%
\definecolor{currentstroke}{rgb}{0.000000,0.000000,0.000000}%
\pgfsetstrokecolor{currentstroke}%
\pgfsetdash{}{0pt}%
\pgfsys@defobject{currentmarker}{\pgfqpoint{0.000000in}{0.000000in}}{\pgfqpoint{0.000000in}{0.020833in}}{%
\pgfpathmoveto{\pgfqpoint{0.000000in}{0.000000in}}%
\pgfpathlineto{\pgfqpoint{0.000000in}{0.020833in}}%
\pgfusepath{stroke,fill}%
}%
\begin{pgfscope}%
\pgfsys@transformshift{2.938563in}{0.586309in}%
\pgfsys@useobject{currentmarker}{}%
\end{pgfscope}%
\end{pgfscope}%
\begin{pgfscope}%
\pgfsetbuttcap%
\pgfsetroundjoin%
\definecolor{currentfill}{rgb}{0.000000,0.000000,0.000000}%
\pgfsetfillcolor{currentfill}%
\pgfsetlinewidth{0.501875pt}%
\definecolor{currentstroke}{rgb}{0.000000,0.000000,0.000000}%
\pgfsetstrokecolor{currentstroke}%
\pgfsetdash{}{0pt}%
\pgfsys@defobject{currentmarker}{\pgfqpoint{0.000000in}{-0.020833in}}{\pgfqpoint{0.000000in}{0.000000in}}{%
\pgfpathmoveto{\pgfqpoint{0.000000in}{0.000000in}}%
\pgfpathlineto{\pgfqpoint{0.000000in}{-0.020833in}}%
\pgfusepath{stroke,fill}%
}%
\begin{pgfscope}%
\pgfsys@transformshift{2.938563in}{0.893003in}%
\pgfsys@useobject{currentmarker}{}%
\end{pgfscope}%
\end{pgfscope}%
\begin{pgfscope}%
\pgfsetbuttcap%
\pgfsetroundjoin%
\definecolor{currentfill}{rgb}{0.000000,0.000000,0.000000}%
\pgfsetfillcolor{currentfill}%
\pgfsetlinewidth{0.501875pt}%
\definecolor{currentstroke}{rgb}{0.000000,0.000000,0.000000}%
\pgfsetstrokecolor{currentstroke}%
\pgfsetdash{}{0pt}%
\pgfsys@defobject{currentmarker}{\pgfqpoint{0.000000in}{0.000000in}}{\pgfqpoint{0.000000in}{0.020833in}}{%
\pgfpathmoveto{\pgfqpoint{0.000000in}{0.000000in}}%
\pgfpathlineto{\pgfqpoint{0.000000in}{0.020833in}}%
\pgfusepath{stroke,fill}%
}%
\begin{pgfscope}%
\pgfsys@transformshift{2.974261in}{0.586309in}%
\pgfsys@useobject{currentmarker}{}%
\end{pgfscope}%
\end{pgfscope}%
\begin{pgfscope}%
\pgfsetbuttcap%
\pgfsetroundjoin%
\definecolor{currentfill}{rgb}{0.000000,0.000000,0.000000}%
\pgfsetfillcolor{currentfill}%
\pgfsetlinewidth{0.501875pt}%
\definecolor{currentstroke}{rgb}{0.000000,0.000000,0.000000}%
\pgfsetstrokecolor{currentstroke}%
\pgfsetdash{}{0pt}%
\pgfsys@defobject{currentmarker}{\pgfqpoint{0.000000in}{-0.020833in}}{\pgfqpoint{0.000000in}{0.000000in}}{%
\pgfpathmoveto{\pgfqpoint{0.000000in}{0.000000in}}%
\pgfpathlineto{\pgfqpoint{0.000000in}{-0.020833in}}%
\pgfusepath{stroke,fill}%
}%
\begin{pgfscope}%
\pgfsys@transformshift{2.974261in}{0.893003in}%
\pgfsys@useobject{currentmarker}{}%
\end{pgfscope}%
\end{pgfscope}%
\begin{pgfscope}%
\pgfsetbuttcap%
\pgfsetroundjoin%
\definecolor{currentfill}{rgb}{0.000000,0.000000,0.000000}%
\pgfsetfillcolor{currentfill}%
\pgfsetlinewidth{0.501875pt}%
\definecolor{currentstroke}{rgb}{0.000000,0.000000,0.000000}%
\pgfsetstrokecolor{currentstroke}%
\pgfsetdash{}{0pt}%
\pgfsys@defobject{currentmarker}{\pgfqpoint{0.000000in}{0.000000in}}{\pgfqpoint{0.000000in}{0.020833in}}{%
\pgfpathmoveto{\pgfqpoint{0.000000in}{0.000000in}}%
\pgfpathlineto{\pgfqpoint{0.000000in}{0.020833in}}%
\pgfusepath{stroke,fill}%
}%
\begin{pgfscope}%
\pgfsys@transformshift{3.009958in}{0.586309in}%
\pgfsys@useobject{currentmarker}{}%
\end{pgfscope}%
\end{pgfscope}%
\begin{pgfscope}%
\pgfsetbuttcap%
\pgfsetroundjoin%
\definecolor{currentfill}{rgb}{0.000000,0.000000,0.000000}%
\pgfsetfillcolor{currentfill}%
\pgfsetlinewidth{0.501875pt}%
\definecolor{currentstroke}{rgb}{0.000000,0.000000,0.000000}%
\pgfsetstrokecolor{currentstroke}%
\pgfsetdash{}{0pt}%
\pgfsys@defobject{currentmarker}{\pgfqpoint{0.000000in}{-0.020833in}}{\pgfqpoint{0.000000in}{0.000000in}}{%
\pgfpathmoveto{\pgfqpoint{0.000000in}{0.000000in}}%
\pgfpathlineto{\pgfqpoint{0.000000in}{-0.020833in}}%
\pgfusepath{stroke,fill}%
}%
\begin{pgfscope}%
\pgfsys@transformshift{3.009958in}{0.893003in}%
\pgfsys@useobject{currentmarker}{}%
\end{pgfscope}%
\end{pgfscope}%
\begin{pgfscope}%
\pgfsetbuttcap%
\pgfsetroundjoin%
\definecolor{currentfill}{rgb}{0.000000,0.000000,0.000000}%
\pgfsetfillcolor{currentfill}%
\pgfsetlinewidth{0.501875pt}%
\definecolor{currentstroke}{rgb}{0.000000,0.000000,0.000000}%
\pgfsetstrokecolor{currentstroke}%
\pgfsetdash{}{0pt}%
\pgfsys@defobject{currentmarker}{\pgfqpoint{0.000000in}{0.000000in}}{\pgfqpoint{0.000000in}{0.020833in}}{%
\pgfpathmoveto{\pgfqpoint{0.000000in}{0.000000in}}%
\pgfpathlineto{\pgfqpoint{0.000000in}{0.020833in}}%
\pgfusepath{stroke,fill}%
}%
\begin{pgfscope}%
\pgfsys@transformshift{3.045655in}{0.586309in}%
\pgfsys@useobject{currentmarker}{}%
\end{pgfscope}%
\end{pgfscope}%
\begin{pgfscope}%
\pgfsetbuttcap%
\pgfsetroundjoin%
\definecolor{currentfill}{rgb}{0.000000,0.000000,0.000000}%
\pgfsetfillcolor{currentfill}%
\pgfsetlinewidth{0.501875pt}%
\definecolor{currentstroke}{rgb}{0.000000,0.000000,0.000000}%
\pgfsetstrokecolor{currentstroke}%
\pgfsetdash{}{0pt}%
\pgfsys@defobject{currentmarker}{\pgfqpoint{0.000000in}{-0.020833in}}{\pgfqpoint{0.000000in}{0.000000in}}{%
\pgfpathmoveto{\pgfqpoint{0.000000in}{0.000000in}}%
\pgfpathlineto{\pgfqpoint{0.000000in}{-0.020833in}}%
\pgfusepath{stroke,fill}%
}%
\begin{pgfscope}%
\pgfsys@transformshift{3.045655in}{0.893003in}%
\pgfsys@useobject{currentmarker}{}%
\end{pgfscope}%
\end{pgfscope}%
\begin{pgfscope}%
\pgfsetbuttcap%
\pgfsetroundjoin%
\definecolor{currentfill}{rgb}{0.000000,0.000000,0.000000}%
\pgfsetfillcolor{currentfill}%
\pgfsetlinewidth{0.501875pt}%
\definecolor{currentstroke}{rgb}{0.000000,0.000000,0.000000}%
\pgfsetstrokecolor{currentstroke}%
\pgfsetdash{}{0pt}%
\pgfsys@defobject{currentmarker}{\pgfqpoint{0.000000in}{0.000000in}}{\pgfqpoint{0.000000in}{0.020833in}}{%
\pgfpathmoveto{\pgfqpoint{0.000000in}{0.000000in}}%
\pgfpathlineto{\pgfqpoint{0.000000in}{0.020833in}}%
\pgfusepath{stroke,fill}%
}%
\begin{pgfscope}%
\pgfsys@transformshift{3.081353in}{0.586309in}%
\pgfsys@useobject{currentmarker}{}%
\end{pgfscope}%
\end{pgfscope}%
\begin{pgfscope}%
\pgfsetbuttcap%
\pgfsetroundjoin%
\definecolor{currentfill}{rgb}{0.000000,0.000000,0.000000}%
\pgfsetfillcolor{currentfill}%
\pgfsetlinewidth{0.501875pt}%
\definecolor{currentstroke}{rgb}{0.000000,0.000000,0.000000}%
\pgfsetstrokecolor{currentstroke}%
\pgfsetdash{}{0pt}%
\pgfsys@defobject{currentmarker}{\pgfqpoint{0.000000in}{-0.020833in}}{\pgfqpoint{0.000000in}{0.000000in}}{%
\pgfpathmoveto{\pgfqpoint{0.000000in}{0.000000in}}%
\pgfpathlineto{\pgfqpoint{0.000000in}{-0.020833in}}%
\pgfusepath{stroke,fill}%
}%
\begin{pgfscope}%
\pgfsys@transformshift{3.081353in}{0.893003in}%
\pgfsys@useobject{currentmarker}{}%
\end{pgfscope}%
\end{pgfscope}%
\begin{pgfscope}%
\pgfsetbuttcap%
\pgfsetroundjoin%
\definecolor{currentfill}{rgb}{0.000000,0.000000,0.000000}%
\pgfsetfillcolor{currentfill}%
\pgfsetlinewidth{0.501875pt}%
\definecolor{currentstroke}{rgb}{0.000000,0.000000,0.000000}%
\pgfsetstrokecolor{currentstroke}%
\pgfsetdash{}{0pt}%
\pgfsys@defobject{currentmarker}{\pgfqpoint{0.000000in}{0.000000in}}{\pgfqpoint{0.000000in}{0.020833in}}{%
\pgfpathmoveto{\pgfqpoint{0.000000in}{0.000000in}}%
\pgfpathlineto{\pgfqpoint{0.000000in}{0.020833in}}%
\pgfusepath{stroke,fill}%
}%
\begin{pgfscope}%
\pgfsys@transformshift{3.117050in}{0.586309in}%
\pgfsys@useobject{currentmarker}{}%
\end{pgfscope}%
\end{pgfscope}%
\begin{pgfscope}%
\pgfsetbuttcap%
\pgfsetroundjoin%
\definecolor{currentfill}{rgb}{0.000000,0.000000,0.000000}%
\pgfsetfillcolor{currentfill}%
\pgfsetlinewidth{0.501875pt}%
\definecolor{currentstroke}{rgb}{0.000000,0.000000,0.000000}%
\pgfsetstrokecolor{currentstroke}%
\pgfsetdash{}{0pt}%
\pgfsys@defobject{currentmarker}{\pgfqpoint{0.000000in}{-0.020833in}}{\pgfqpoint{0.000000in}{0.000000in}}{%
\pgfpathmoveto{\pgfqpoint{0.000000in}{0.000000in}}%
\pgfpathlineto{\pgfqpoint{0.000000in}{-0.020833in}}%
\pgfusepath{stroke,fill}%
}%
\begin{pgfscope}%
\pgfsys@transformshift{3.117050in}{0.893003in}%
\pgfsys@useobject{currentmarker}{}%
\end{pgfscope}%
\end{pgfscope}%
\begin{pgfscope}%
\pgfsetbuttcap%
\pgfsetroundjoin%
\definecolor{currentfill}{rgb}{0.000000,0.000000,0.000000}%
\pgfsetfillcolor{currentfill}%
\pgfsetlinewidth{0.501875pt}%
\definecolor{currentstroke}{rgb}{0.000000,0.000000,0.000000}%
\pgfsetstrokecolor{currentstroke}%
\pgfsetdash{}{0pt}%
\pgfsys@defobject{currentmarker}{\pgfqpoint{0.000000in}{0.000000in}}{\pgfqpoint{0.000000in}{0.020833in}}{%
\pgfpathmoveto{\pgfqpoint{0.000000in}{0.000000in}}%
\pgfpathlineto{\pgfqpoint{0.000000in}{0.020833in}}%
\pgfusepath{stroke,fill}%
}%
\begin{pgfscope}%
\pgfsys@transformshift{3.152747in}{0.586309in}%
\pgfsys@useobject{currentmarker}{}%
\end{pgfscope}%
\end{pgfscope}%
\begin{pgfscope}%
\pgfsetbuttcap%
\pgfsetroundjoin%
\definecolor{currentfill}{rgb}{0.000000,0.000000,0.000000}%
\pgfsetfillcolor{currentfill}%
\pgfsetlinewidth{0.501875pt}%
\definecolor{currentstroke}{rgb}{0.000000,0.000000,0.000000}%
\pgfsetstrokecolor{currentstroke}%
\pgfsetdash{}{0pt}%
\pgfsys@defobject{currentmarker}{\pgfqpoint{0.000000in}{-0.020833in}}{\pgfqpoint{0.000000in}{0.000000in}}{%
\pgfpathmoveto{\pgfqpoint{0.000000in}{0.000000in}}%
\pgfpathlineto{\pgfqpoint{0.000000in}{-0.020833in}}%
\pgfusepath{stroke,fill}%
}%
\begin{pgfscope}%
\pgfsys@transformshift{3.152747in}{0.893003in}%
\pgfsys@useobject{currentmarker}{}%
\end{pgfscope}%
\end{pgfscope}%
\begin{pgfscope}%
\pgfsetbuttcap%
\pgfsetroundjoin%
\definecolor{currentfill}{rgb}{0.000000,0.000000,0.000000}%
\pgfsetfillcolor{currentfill}%
\pgfsetlinewidth{0.501875pt}%
\definecolor{currentstroke}{rgb}{0.000000,0.000000,0.000000}%
\pgfsetstrokecolor{currentstroke}%
\pgfsetdash{}{0pt}%
\pgfsys@defobject{currentmarker}{\pgfqpoint{0.000000in}{0.000000in}}{\pgfqpoint{0.000000in}{0.020833in}}{%
\pgfpathmoveto{\pgfqpoint{0.000000in}{0.000000in}}%
\pgfpathlineto{\pgfqpoint{0.000000in}{0.020833in}}%
\pgfusepath{stroke,fill}%
}%
\begin{pgfscope}%
\pgfsys@transformshift{3.188445in}{0.586309in}%
\pgfsys@useobject{currentmarker}{}%
\end{pgfscope}%
\end{pgfscope}%
\begin{pgfscope}%
\pgfsetbuttcap%
\pgfsetroundjoin%
\definecolor{currentfill}{rgb}{0.000000,0.000000,0.000000}%
\pgfsetfillcolor{currentfill}%
\pgfsetlinewidth{0.501875pt}%
\definecolor{currentstroke}{rgb}{0.000000,0.000000,0.000000}%
\pgfsetstrokecolor{currentstroke}%
\pgfsetdash{}{0pt}%
\pgfsys@defobject{currentmarker}{\pgfqpoint{0.000000in}{-0.020833in}}{\pgfqpoint{0.000000in}{0.000000in}}{%
\pgfpathmoveto{\pgfqpoint{0.000000in}{0.000000in}}%
\pgfpathlineto{\pgfqpoint{0.000000in}{-0.020833in}}%
\pgfusepath{stroke,fill}%
}%
\begin{pgfscope}%
\pgfsys@transformshift{3.188445in}{0.893003in}%
\pgfsys@useobject{currentmarker}{}%
\end{pgfscope}%
\end{pgfscope}%
\begin{pgfscope}%
\pgfsetbuttcap%
\pgfsetroundjoin%
\definecolor{currentfill}{rgb}{0.000000,0.000000,0.000000}%
\pgfsetfillcolor{currentfill}%
\pgfsetlinewidth{0.501875pt}%
\definecolor{currentstroke}{rgb}{0.000000,0.000000,0.000000}%
\pgfsetstrokecolor{currentstroke}%
\pgfsetdash{}{0pt}%
\pgfsys@defobject{currentmarker}{\pgfqpoint{0.000000in}{0.000000in}}{\pgfqpoint{0.000000in}{0.020833in}}{%
\pgfpathmoveto{\pgfqpoint{0.000000in}{0.000000in}}%
\pgfpathlineto{\pgfqpoint{0.000000in}{0.020833in}}%
\pgfusepath{stroke,fill}%
}%
\begin{pgfscope}%
\pgfsys@transformshift{3.224142in}{0.586309in}%
\pgfsys@useobject{currentmarker}{}%
\end{pgfscope}%
\end{pgfscope}%
\begin{pgfscope}%
\pgfsetbuttcap%
\pgfsetroundjoin%
\definecolor{currentfill}{rgb}{0.000000,0.000000,0.000000}%
\pgfsetfillcolor{currentfill}%
\pgfsetlinewidth{0.501875pt}%
\definecolor{currentstroke}{rgb}{0.000000,0.000000,0.000000}%
\pgfsetstrokecolor{currentstroke}%
\pgfsetdash{}{0pt}%
\pgfsys@defobject{currentmarker}{\pgfqpoint{0.000000in}{-0.020833in}}{\pgfqpoint{0.000000in}{0.000000in}}{%
\pgfpathmoveto{\pgfqpoint{0.000000in}{0.000000in}}%
\pgfpathlineto{\pgfqpoint{0.000000in}{-0.020833in}}%
\pgfusepath{stroke,fill}%
}%
\begin{pgfscope}%
\pgfsys@transformshift{3.224142in}{0.893003in}%
\pgfsys@useobject{currentmarker}{}%
\end{pgfscope}%
\end{pgfscope}%
\begin{pgfscope}%
\pgfsetbuttcap%
\pgfsetroundjoin%
\definecolor{currentfill}{rgb}{0.000000,0.000000,0.000000}%
\pgfsetfillcolor{currentfill}%
\pgfsetlinewidth{0.501875pt}%
\definecolor{currentstroke}{rgb}{0.000000,0.000000,0.000000}%
\pgfsetstrokecolor{currentstroke}%
\pgfsetdash{}{0pt}%
\pgfsys@defobject{currentmarker}{\pgfqpoint{0.000000in}{0.000000in}}{\pgfqpoint{0.000000in}{0.020833in}}{%
\pgfpathmoveto{\pgfqpoint{0.000000in}{0.000000in}}%
\pgfpathlineto{\pgfqpoint{0.000000in}{0.020833in}}%
\pgfusepath{stroke,fill}%
}%
\begin{pgfscope}%
\pgfsys@transformshift{3.295537in}{0.586309in}%
\pgfsys@useobject{currentmarker}{}%
\end{pgfscope}%
\end{pgfscope}%
\begin{pgfscope}%
\pgfsetbuttcap%
\pgfsetroundjoin%
\definecolor{currentfill}{rgb}{0.000000,0.000000,0.000000}%
\pgfsetfillcolor{currentfill}%
\pgfsetlinewidth{0.501875pt}%
\definecolor{currentstroke}{rgb}{0.000000,0.000000,0.000000}%
\pgfsetstrokecolor{currentstroke}%
\pgfsetdash{}{0pt}%
\pgfsys@defobject{currentmarker}{\pgfqpoint{0.000000in}{-0.020833in}}{\pgfqpoint{0.000000in}{0.000000in}}{%
\pgfpathmoveto{\pgfqpoint{0.000000in}{0.000000in}}%
\pgfpathlineto{\pgfqpoint{0.000000in}{-0.020833in}}%
\pgfusepath{stroke,fill}%
}%
\begin{pgfscope}%
\pgfsys@transformshift{3.295537in}{0.893003in}%
\pgfsys@useobject{currentmarker}{}%
\end{pgfscope}%
\end{pgfscope}%
\begin{pgfscope}%
\pgfsetbuttcap%
\pgfsetroundjoin%
\definecolor{currentfill}{rgb}{0.000000,0.000000,0.000000}%
\pgfsetfillcolor{currentfill}%
\pgfsetlinewidth{0.501875pt}%
\definecolor{currentstroke}{rgb}{0.000000,0.000000,0.000000}%
\pgfsetstrokecolor{currentstroke}%
\pgfsetdash{}{0pt}%
\pgfsys@defobject{currentmarker}{\pgfqpoint{0.000000in}{0.000000in}}{\pgfqpoint{0.000000in}{0.020833in}}{%
\pgfpathmoveto{\pgfqpoint{0.000000in}{0.000000in}}%
\pgfpathlineto{\pgfqpoint{0.000000in}{0.020833in}}%
\pgfusepath{stroke,fill}%
}%
\begin{pgfscope}%
\pgfsys@transformshift{3.331234in}{0.586309in}%
\pgfsys@useobject{currentmarker}{}%
\end{pgfscope}%
\end{pgfscope}%
\begin{pgfscope}%
\pgfsetbuttcap%
\pgfsetroundjoin%
\definecolor{currentfill}{rgb}{0.000000,0.000000,0.000000}%
\pgfsetfillcolor{currentfill}%
\pgfsetlinewidth{0.501875pt}%
\definecolor{currentstroke}{rgb}{0.000000,0.000000,0.000000}%
\pgfsetstrokecolor{currentstroke}%
\pgfsetdash{}{0pt}%
\pgfsys@defobject{currentmarker}{\pgfqpoint{0.000000in}{-0.020833in}}{\pgfqpoint{0.000000in}{0.000000in}}{%
\pgfpathmoveto{\pgfqpoint{0.000000in}{0.000000in}}%
\pgfpathlineto{\pgfqpoint{0.000000in}{-0.020833in}}%
\pgfusepath{stroke,fill}%
}%
\begin{pgfscope}%
\pgfsys@transformshift{3.331234in}{0.893003in}%
\pgfsys@useobject{currentmarker}{}%
\end{pgfscope}%
\end{pgfscope}%
\begin{pgfscope}%
\pgfsetbuttcap%
\pgfsetroundjoin%
\definecolor{currentfill}{rgb}{0.000000,0.000000,0.000000}%
\pgfsetfillcolor{currentfill}%
\pgfsetlinewidth{0.501875pt}%
\definecolor{currentstroke}{rgb}{0.000000,0.000000,0.000000}%
\pgfsetstrokecolor{currentstroke}%
\pgfsetdash{}{0pt}%
\pgfsys@defobject{currentmarker}{\pgfqpoint{0.000000in}{0.000000in}}{\pgfqpoint{0.000000in}{0.020833in}}{%
\pgfpathmoveto{\pgfqpoint{0.000000in}{0.000000in}}%
\pgfpathlineto{\pgfqpoint{0.000000in}{0.020833in}}%
\pgfusepath{stroke,fill}%
}%
\begin{pgfscope}%
\pgfsys@transformshift{3.366931in}{0.586309in}%
\pgfsys@useobject{currentmarker}{}%
\end{pgfscope}%
\end{pgfscope}%
\begin{pgfscope}%
\pgfsetbuttcap%
\pgfsetroundjoin%
\definecolor{currentfill}{rgb}{0.000000,0.000000,0.000000}%
\pgfsetfillcolor{currentfill}%
\pgfsetlinewidth{0.501875pt}%
\definecolor{currentstroke}{rgb}{0.000000,0.000000,0.000000}%
\pgfsetstrokecolor{currentstroke}%
\pgfsetdash{}{0pt}%
\pgfsys@defobject{currentmarker}{\pgfqpoint{0.000000in}{-0.020833in}}{\pgfqpoint{0.000000in}{0.000000in}}{%
\pgfpathmoveto{\pgfqpoint{0.000000in}{0.000000in}}%
\pgfpathlineto{\pgfqpoint{0.000000in}{-0.020833in}}%
\pgfusepath{stroke,fill}%
}%
\begin{pgfscope}%
\pgfsys@transformshift{3.366931in}{0.893003in}%
\pgfsys@useobject{currentmarker}{}%
\end{pgfscope}%
\end{pgfscope}%
\begin{pgfscope}%
\pgfsetbuttcap%
\pgfsetroundjoin%
\definecolor{currentfill}{rgb}{0.000000,0.000000,0.000000}%
\pgfsetfillcolor{currentfill}%
\pgfsetlinewidth{0.501875pt}%
\definecolor{currentstroke}{rgb}{0.000000,0.000000,0.000000}%
\pgfsetstrokecolor{currentstroke}%
\pgfsetdash{}{0pt}%
\pgfsys@defobject{currentmarker}{\pgfqpoint{0.000000in}{0.000000in}}{\pgfqpoint{0.000000in}{0.020833in}}{%
\pgfpathmoveto{\pgfqpoint{0.000000in}{0.000000in}}%
\pgfpathlineto{\pgfqpoint{0.000000in}{0.020833in}}%
\pgfusepath{stroke,fill}%
}%
\begin{pgfscope}%
\pgfsys@transformshift{3.402629in}{0.586309in}%
\pgfsys@useobject{currentmarker}{}%
\end{pgfscope}%
\end{pgfscope}%
\begin{pgfscope}%
\pgfsetbuttcap%
\pgfsetroundjoin%
\definecolor{currentfill}{rgb}{0.000000,0.000000,0.000000}%
\pgfsetfillcolor{currentfill}%
\pgfsetlinewidth{0.501875pt}%
\definecolor{currentstroke}{rgb}{0.000000,0.000000,0.000000}%
\pgfsetstrokecolor{currentstroke}%
\pgfsetdash{}{0pt}%
\pgfsys@defobject{currentmarker}{\pgfqpoint{0.000000in}{-0.020833in}}{\pgfqpoint{0.000000in}{0.000000in}}{%
\pgfpathmoveto{\pgfqpoint{0.000000in}{0.000000in}}%
\pgfpathlineto{\pgfqpoint{0.000000in}{-0.020833in}}%
\pgfusepath{stroke,fill}%
}%
\begin{pgfscope}%
\pgfsys@transformshift{3.402629in}{0.893003in}%
\pgfsys@useobject{currentmarker}{}%
\end{pgfscope}%
\end{pgfscope}%
\begin{pgfscope}%
\pgfsetbuttcap%
\pgfsetroundjoin%
\definecolor{currentfill}{rgb}{0.000000,0.000000,0.000000}%
\pgfsetfillcolor{currentfill}%
\pgfsetlinewidth{0.501875pt}%
\definecolor{currentstroke}{rgb}{0.000000,0.000000,0.000000}%
\pgfsetstrokecolor{currentstroke}%
\pgfsetdash{}{0pt}%
\pgfsys@defobject{currentmarker}{\pgfqpoint{0.000000in}{0.000000in}}{\pgfqpoint{0.000000in}{0.020833in}}{%
\pgfpathmoveto{\pgfqpoint{0.000000in}{0.000000in}}%
\pgfpathlineto{\pgfqpoint{0.000000in}{0.020833in}}%
\pgfusepath{stroke,fill}%
}%
\begin{pgfscope}%
\pgfsys@transformshift{3.438326in}{0.586309in}%
\pgfsys@useobject{currentmarker}{}%
\end{pgfscope}%
\end{pgfscope}%
\begin{pgfscope}%
\pgfsetbuttcap%
\pgfsetroundjoin%
\definecolor{currentfill}{rgb}{0.000000,0.000000,0.000000}%
\pgfsetfillcolor{currentfill}%
\pgfsetlinewidth{0.501875pt}%
\definecolor{currentstroke}{rgb}{0.000000,0.000000,0.000000}%
\pgfsetstrokecolor{currentstroke}%
\pgfsetdash{}{0pt}%
\pgfsys@defobject{currentmarker}{\pgfqpoint{0.000000in}{-0.020833in}}{\pgfqpoint{0.000000in}{0.000000in}}{%
\pgfpathmoveto{\pgfqpoint{0.000000in}{0.000000in}}%
\pgfpathlineto{\pgfqpoint{0.000000in}{-0.020833in}}%
\pgfusepath{stroke,fill}%
}%
\begin{pgfscope}%
\pgfsys@transformshift{3.438326in}{0.893003in}%
\pgfsys@useobject{currentmarker}{}%
\end{pgfscope}%
\end{pgfscope}%
\begin{pgfscope}%
\pgfsetbuttcap%
\pgfsetroundjoin%
\definecolor{currentfill}{rgb}{0.000000,0.000000,0.000000}%
\pgfsetfillcolor{currentfill}%
\pgfsetlinewidth{0.501875pt}%
\definecolor{currentstroke}{rgb}{0.000000,0.000000,0.000000}%
\pgfsetstrokecolor{currentstroke}%
\pgfsetdash{}{0pt}%
\pgfsys@defobject{currentmarker}{\pgfqpoint{0.000000in}{0.000000in}}{\pgfqpoint{0.000000in}{0.020833in}}{%
\pgfpathmoveto{\pgfqpoint{0.000000in}{0.000000in}}%
\pgfpathlineto{\pgfqpoint{0.000000in}{0.020833in}}%
\pgfusepath{stroke,fill}%
}%
\begin{pgfscope}%
\pgfsys@transformshift{3.474023in}{0.586309in}%
\pgfsys@useobject{currentmarker}{}%
\end{pgfscope}%
\end{pgfscope}%
\begin{pgfscope}%
\pgfsetbuttcap%
\pgfsetroundjoin%
\definecolor{currentfill}{rgb}{0.000000,0.000000,0.000000}%
\pgfsetfillcolor{currentfill}%
\pgfsetlinewidth{0.501875pt}%
\definecolor{currentstroke}{rgb}{0.000000,0.000000,0.000000}%
\pgfsetstrokecolor{currentstroke}%
\pgfsetdash{}{0pt}%
\pgfsys@defobject{currentmarker}{\pgfqpoint{0.000000in}{-0.020833in}}{\pgfqpoint{0.000000in}{0.000000in}}{%
\pgfpathmoveto{\pgfqpoint{0.000000in}{0.000000in}}%
\pgfpathlineto{\pgfqpoint{0.000000in}{-0.020833in}}%
\pgfusepath{stroke,fill}%
}%
\begin{pgfscope}%
\pgfsys@transformshift{3.474023in}{0.893003in}%
\pgfsys@useobject{currentmarker}{}%
\end{pgfscope}%
\end{pgfscope}%
\begin{pgfscope}%
\pgfsetbuttcap%
\pgfsetroundjoin%
\definecolor{currentfill}{rgb}{0.000000,0.000000,0.000000}%
\pgfsetfillcolor{currentfill}%
\pgfsetlinewidth{0.501875pt}%
\definecolor{currentstroke}{rgb}{0.000000,0.000000,0.000000}%
\pgfsetstrokecolor{currentstroke}%
\pgfsetdash{}{0pt}%
\pgfsys@defobject{currentmarker}{\pgfqpoint{0.000000in}{0.000000in}}{\pgfqpoint{0.000000in}{0.020833in}}{%
\pgfpathmoveto{\pgfqpoint{0.000000in}{0.000000in}}%
\pgfpathlineto{\pgfqpoint{0.000000in}{0.020833in}}%
\pgfusepath{stroke,fill}%
}%
\begin{pgfscope}%
\pgfsys@transformshift{3.509721in}{0.586309in}%
\pgfsys@useobject{currentmarker}{}%
\end{pgfscope}%
\end{pgfscope}%
\begin{pgfscope}%
\pgfsetbuttcap%
\pgfsetroundjoin%
\definecolor{currentfill}{rgb}{0.000000,0.000000,0.000000}%
\pgfsetfillcolor{currentfill}%
\pgfsetlinewidth{0.501875pt}%
\definecolor{currentstroke}{rgb}{0.000000,0.000000,0.000000}%
\pgfsetstrokecolor{currentstroke}%
\pgfsetdash{}{0pt}%
\pgfsys@defobject{currentmarker}{\pgfqpoint{0.000000in}{-0.020833in}}{\pgfqpoint{0.000000in}{0.000000in}}{%
\pgfpathmoveto{\pgfqpoint{0.000000in}{0.000000in}}%
\pgfpathlineto{\pgfqpoint{0.000000in}{-0.020833in}}%
\pgfusepath{stroke,fill}%
}%
\begin{pgfscope}%
\pgfsys@transformshift{3.509721in}{0.893003in}%
\pgfsys@useobject{currentmarker}{}%
\end{pgfscope}%
\end{pgfscope}%
\begin{pgfscope}%
\pgfsetbuttcap%
\pgfsetroundjoin%
\definecolor{currentfill}{rgb}{0.000000,0.000000,0.000000}%
\pgfsetfillcolor{currentfill}%
\pgfsetlinewidth{0.501875pt}%
\definecolor{currentstroke}{rgb}{0.000000,0.000000,0.000000}%
\pgfsetstrokecolor{currentstroke}%
\pgfsetdash{}{0pt}%
\pgfsys@defobject{currentmarker}{\pgfqpoint{0.000000in}{0.000000in}}{\pgfqpoint{0.000000in}{0.020833in}}{%
\pgfpathmoveto{\pgfqpoint{0.000000in}{0.000000in}}%
\pgfpathlineto{\pgfqpoint{0.000000in}{0.020833in}}%
\pgfusepath{stroke,fill}%
}%
\begin{pgfscope}%
\pgfsys@transformshift{3.545418in}{0.586309in}%
\pgfsys@useobject{currentmarker}{}%
\end{pgfscope}%
\end{pgfscope}%
\begin{pgfscope}%
\pgfsetbuttcap%
\pgfsetroundjoin%
\definecolor{currentfill}{rgb}{0.000000,0.000000,0.000000}%
\pgfsetfillcolor{currentfill}%
\pgfsetlinewidth{0.501875pt}%
\definecolor{currentstroke}{rgb}{0.000000,0.000000,0.000000}%
\pgfsetstrokecolor{currentstroke}%
\pgfsetdash{}{0pt}%
\pgfsys@defobject{currentmarker}{\pgfqpoint{0.000000in}{-0.020833in}}{\pgfqpoint{0.000000in}{0.000000in}}{%
\pgfpathmoveto{\pgfqpoint{0.000000in}{0.000000in}}%
\pgfpathlineto{\pgfqpoint{0.000000in}{-0.020833in}}%
\pgfusepath{stroke,fill}%
}%
\begin{pgfscope}%
\pgfsys@transformshift{3.545418in}{0.893003in}%
\pgfsys@useobject{currentmarker}{}%
\end{pgfscope}%
\end{pgfscope}%
\begin{pgfscope}%
\pgfsetbuttcap%
\pgfsetroundjoin%
\definecolor{currentfill}{rgb}{0.000000,0.000000,0.000000}%
\pgfsetfillcolor{currentfill}%
\pgfsetlinewidth{0.501875pt}%
\definecolor{currentstroke}{rgb}{0.000000,0.000000,0.000000}%
\pgfsetstrokecolor{currentstroke}%
\pgfsetdash{}{0pt}%
\pgfsys@defobject{currentmarker}{\pgfqpoint{0.000000in}{0.000000in}}{\pgfqpoint{0.000000in}{0.020833in}}{%
\pgfpathmoveto{\pgfqpoint{0.000000in}{0.000000in}}%
\pgfpathlineto{\pgfqpoint{0.000000in}{0.020833in}}%
\pgfusepath{stroke,fill}%
}%
\begin{pgfscope}%
\pgfsys@transformshift{3.581115in}{0.586309in}%
\pgfsys@useobject{currentmarker}{}%
\end{pgfscope}%
\end{pgfscope}%
\begin{pgfscope}%
\pgfsetbuttcap%
\pgfsetroundjoin%
\definecolor{currentfill}{rgb}{0.000000,0.000000,0.000000}%
\pgfsetfillcolor{currentfill}%
\pgfsetlinewidth{0.501875pt}%
\definecolor{currentstroke}{rgb}{0.000000,0.000000,0.000000}%
\pgfsetstrokecolor{currentstroke}%
\pgfsetdash{}{0pt}%
\pgfsys@defobject{currentmarker}{\pgfqpoint{0.000000in}{-0.020833in}}{\pgfqpoint{0.000000in}{0.000000in}}{%
\pgfpathmoveto{\pgfqpoint{0.000000in}{0.000000in}}%
\pgfpathlineto{\pgfqpoint{0.000000in}{-0.020833in}}%
\pgfusepath{stroke,fill}%
}%
\begin{pgfscope}%
\pgfsys@transformshift{3.581115in}{0.893003in}%
\pgfsys@useobject{currentmarker}{}%
\end{pgfscope}%
\end{pgfscope}%
\begin{pgfscope}%
\pgfsetbuttcap%
\pgfsetroundjoin%
\definecolor{currentfill}{rgb}{0.000000,0.000000,0.000000}%
\pgfsetfillcolor{currentfill}%
\pgfsetlinewidth{0.501875pt}%
\definecolor{currentstroke}{rgb}{0.000000,0.000000,0.000000}%
\pgfsetstrokecolor{currentstroke}%
\pgfsetdash{}{0pt}%
\pgfsys@defobject{currentmarker}{\pgfqpoint{0.000000in}{0.000000in}}{\pgfqpoint{0.000000in}{0.020833in}}{%
\pgfpathmoveto{\pgfqpoint{0.000000in}{0.000000in}}%
\pgfpathlineto{\pgfqpoint{0.000000in}{0.020833in}}%
\pgfusepath{stroke,fill}%
}%
\begin{pgfscope}%
\pgfsys@transformshift{3.616812in}{0.586309in}%
\pgfsys@useobject{currentmarker}{}%
\end{pgfscope}%
\end{pgfscope}%
\begin{pgfscope}%
\pgfsetbuttcap%
\pgfsetroundjoin%
\definecolor{currentfill}{rgb}{0.000000,0.000000,0.000000}%
\pgfsetfillcolor{currentfill}%
\pgfsetlinewidth{0.501875pt}%
\definecolor{currentstroke}{rgb}{0.000000,0.000000,0.000000}%
\pgfsetstrokecolor{currentstroke}%
\pgfsetdash{}{0pt}%
\pgfsys@defobject{currentmarker}{\pgfqpoint{0.000000in}{-0.020833in}}{\pgfqpoint{0.000000in}{0.000000in}}{%
\pgfpathmoveto{\pgfqpoint{0.000000in}{0.000000in}}%
\pgfpathlineto{\pgfqpoint{0.000000in}{-0.020833in}}%
\pgfusepath{stroke,fill}%
}%
\begin{pgfscope}%
\pgfsys@transformshift{3.616812in}{0.893003in}%
\pgfsys@useobject{currentmarker}{}%
\end{pgfscope}%
\end{pgfscope}%
\begin{pgfscope}%
\pgfsetbuttcap%
\pgfsetroundjoin%
\definecolor{currentfill}{rgb}{0.000000,0.000000,0.000000}%
\pgfsetfillcolor{currentfill}%
\pgfsetlinewidth{0.501875pt}%
\definecolor{currentstroke}{rgb}{0.000000,0.000000,0.000000}%
\pgfsetstrokecolor{currentstroke}%
\pgfsetdash{}{0pt}%
\pgfsys@defobject{currentmarker}{\pgfqpoint{0.000000in}{0.000000in}}{\pgfqpoint{0.000000in}{0.020833in}}{%
\pgfpathmoveto{\pgfqpoint{0.000000in}{0.000000in}}%
\pgfpathlineto{\pgfqpoint{0.000000in}{0.020833in}}%
\pgfusepath{stroke,fill}%
}%
\begin{pgfscope}%
\pgfsys@transformshift{3.652510in}{0.586309in}%
\pgfsys@useobject{currentmarker}{}%
\end{pgfscope}%
\end{pgfscope}%
\begin{pgfscope}%
\pgfsetbuttcap%
\pgfsetroundjoin%
\definecolor{currentfill}{rgb}{0.000000,0.000000,0.000000}%
\pgfsetfillcolor{currentfill}%
\pgfsetlinewidth{0.501875pt}%
\definecolor{currentstroke}{rgb}{0.000000,0.000000,0.000000}%
\pgfsetstrokecolor{currentstroke}%
\pgfsetdash{}{0pt}%
\pgfsys@defobject{currentmarker}{\pgfqpoint{0.000000in}{-0.020833in}}{\pgfqpoint{0.000000in}{0.000000in}}{%
\pgfpathmoveto{\pgfqpoint{0.000000in}{0.000000in}}%
\pgfpathlineto{\pgfqpoint{0.000000in}{-0.020833in}}%
\pgfusepath{stroke,fill}%
}%
\begin{pgfscope}%
\pgfsys@transformshift{3.652510in}{0.893003in}%
\pgfsys@useobject{currentmarker}{}%
\end{pgfscope}%
\end{pgfscope}%
\begin{pgfscope}%
\pgfsetbuttcap%
\pgfsetroundjoin%
\definecolor{currentfill}{rgb}{0.000000,0.000000,0.000000}%
\pgfsetfillcolor{currentfill}%
\pgfsetlinewidth{0.501875pt}%
\definecolor{currentstroke}{rgb}{0.000000,0.000000,0.000000}%
\pgfsetstrokecolor{currentstroke}%
\pgfsetdash{}{0pt}%
\pgfsys@defobject{currentmarker}{\pgfqpoint{0.000000in}{0.000000in}}{\pgfqpoint{0.000000in}{0.020833in}}{%
\pgfpathmoveto{\pgfqpoint{0.000000in}{0.000000in}}%
\pgfpathlineto{\pgfqpoint{0.000000in}{0.020833in}}%
\pgfusepath{stroke,fill}%
}%
\begin{pgfscope}%
\pgfsys@transformshift{3.723904in}{0.586309in}%
\pgfsys@useobject{currentmarker}{}%
\end{pgfscope}%
\end{pgfscope}%
\begin{pgfscope}%
\pgfsetbuttcap%
\pgfsetroundjoin%
\definecolor{currentfill}{rgb}{0.000000,0.000000,0.000000}%
\pgfsetfillcolor{currentfill}%
\pgfsetlinewidth{0.501875pt}%
\definecolor{currentstroke}{rgb}{0.000000,0.000000,0.000000}%
\pgfsetstrokecolor{currentstroke}%
\pgfsetdash{}{0pt}%
\pgfsys@defobject{currentmarker}{\pgfqpoint{0.000000in}{-0.020833in}}{\pgfqpoint{0.000000in}{0.000000in}}{%
\pgfpathmoveto{\pgfqpoint{0.000000in}{0.000000in}}%
\pgfpathlineto{\pgfqpoint{0.000000in}{-0.020833in}}%
\pgfusepath{stroke,fill}%
}%
\begin{pgfscope}%
\pgfsys@transformshift{3.723904in}{0.893003in}%
\pgfsys@useobject{currentmarker}{}%
\end{pgfscope}%
\end{pgfscope}%
\begin{pgfscope}%
\pgfsetbuttcap%
\pgfsetroundjoin%
\definecolor{currentfill}{rgb}{0.000000,0.000000,0.000000}%
\pgfsetfillcolor{currentfill}%
\pgfsetlinewidth{0.501875pt}%
\definecolor{currentstroke}{rgb}{0.000000,0.000000,0.000000}%
\pgfsetstrokecolor{currentstroke}%
\pgfsetdash{}{0pt}%
\pgfsys@defobject{currentmarker}{\pgfqpoint{0.000000in}{0.000000in}}{\pgfqpoint{0.000000in}{0.020833in}}{%
\pgfpathmoveto{\pgfqpoint{0.000000in}{0.000000in}}%
\pgfpathlineto{\pgfqpoint{0.000000in}{0.020833in}}%
\pgfusepath{stroke,fill}%
}%
\begin{pgfscope}%
\pgfsys@transformshift{3.759602in}{0.586309in}%
\pgfsys@useobject{currentmarker}{}%
\end{pgfscope}%
\end{pgfscope}%
\begin{pgfscope}%
\pgfsetbuttcap%
\pgfsetroundjoin%
\definecolor{currentfill}{rgb}{0.000000,0.000000,0.000000}%
\pgfsetfillcolor{currentfill}%
\pgfsetlinewidth{0.501875pt}%
\definecolor{currentstroke}{rgb}{0.000000,0.000000,0.000000}%
\pgfsetstrokecolor{currentstroke}%
\pgfsetdash{}{0pt}%
\pgfsys@defobject{currentmarker}{\pgfqpoint{0.000000in}{-0.020833in}}{\pgfqpoint{0.000000in}{0.000000in}}{%
\pgfpathmoveto{\pgfqpoint{0.000000in}{0.000000in}}%
\pgfpathlineto{\pgfqpoint{0.000000in}{-0.020833in}}%
\pgfusepath{stroke,fill}%
}%
\begin{pgfscope}%
\pgfsys@transformshift{3.759602in}{0.893003in}%
\pgfsys@useobject{currentmarker}{}%
\end{pgfscope}%
\end{pgfscope}%
\begin{pgfscope}%
\pgfsetbuttcap%
\pgfsetroundjoin%
\definecolor{currentfill}{rgb}{0.000000,0.000000,0.000000}%
\pgfsetfillcolor{currentfill}%
\pgfsetlinewidth{0.501875pt}%
\definecolor{currentstroke}{rgb}{0.000000,0.000000,0.000000}%
\pgfsetstrokecolor{currentstroke}%
\pgfsetdash{}{0pt}%
\pgfsys@defobject{currentmarker}{\pgfqpoint{0.000000in}{0.000000in}}{\pgfqpoint{0.000000in}{0.020833in}}{%
\pgfpathmoveto{\pgfqpoint{0.000000in}{0.000000in}}%
\pgfpathlineto{\pgfqpoint{0.000000in}{0.020833in}}%
\pgfusepath{stroke,fill}%
}%
\begin{pgfscope}%
\pgfsys@transformshift{3.795299in}{0.586309in}%
\pgfsys@useobject{currentmarker}{}%
\end{pgfscope}%
\end{pgfscope}%
\begin{pgfscope}%
\pgfsetbuttcap%
\pgfsetroundjoin%
\definecolor{currentfill}{rgb}{0.000000,0.000000,0.000000}%
\pgfsetfillcolor{currentfill}%
\pgfsetlinewidth{0.501875pt}%
\definecolor{currentstroke}{rgb}{0.000000,0.000000,0.000000}%
\pgfsetstrokecolor{currentstroke}%
\pgfsetdash{}{0pt}%
\pgfsys@defobject{currentmarker}{\pgfqpoint{0.000000in}{-0.020833in}}{\pgfqpoint{0.000000in}{0.000000in}}{%
\pgfpathmoveto{\pgfqpoint{0.000000in}{0.000000in}}%
\pgfpathlineto{\pgfqpoint{0.000000in}{-0.020833in}}%
\pgfusepath{stroke,fill}%
}%
\begin{pgfscope}%
\pgfsys@transformshift{3.795299in}{0.893003in}%
\pgfsys@useobject{currentmarker}{}%
\end{pgfscope}%
\end{pgfscope}%
\begin{pgfscope}%
\pgfsetbuttcap%
\pgfsetroundjoin%
\definecolor{currentfill}{rgb}{0.000000,0.000000,0.000000}%
\pgfsetfillcolor{currentfill}%
\pgfsetlinewidth{0.501875pt}%
\definecolor{currentstroke}{rgb}{0.000000,0.000000,0.000000}%
\pgfsetstrokecolor{currentstroke}%
\pgfsetdash{}{0pt}%
\pgfsys@defobject{currentmarker}{\pgfqpoint{0.000000in}{0.000000in}}{\pgfqpoint{0.000000in}{0.020833in}}{%
\pgfpathmoveto{\pgfqpoint{0.000000in}{0.000000in}}%
\pgfpathlineto{\pgfqpoint{0.000000in}{0.020833in}}%
\pgfusepath{stroke,fill}%
}%
\begin{pgfscope}%
\pgfsys@transformshift{3.830996in}{0.586309in}%
\pgfsys@useobject{currentmarker}{}%
\end{pgfscope}%
\end{pgfscope}%
\begin{pgfscope}%
\pgfsetbuttcap%
\pgfsetroundjoin%
\definecolor{currentfill}{rgb}{0.000000,0.000000,0.000000}%
\pgfsetfillcolor{currentfill}%
\pgfsetlinewidth{0.501875pt}%
\definecolor{currentstroke}{rgb}{0.000000,0.000000,0.000000}%
\pgfsetstrokecolor{currentstroke}%
\pgfsetdash{}{0pt}%
\pgfsys@defobject{currentmarker}{\pgfqpoint{0.000000in}{-0.020833in}}{\pgfqpoint{0.000000in}{0.000000in}}{%
\pgfpathmoveto{\pgfqpoint{0.000000in}{0.000000in}}%
\pgfpathlineto{\pgfqpoint{0.000000in}{-0.020833in}}%
\pgfusepath{stroke,fill}%
}%
\begin{pgfscope}%
\pgfsys@transformshift{3.830996in}{0.893003in}%
\pgfsys@useobject{currentmarker}{}%
\end{pgfscope}%
\end{pgfscope}%
\begin{pgfscope}%
\pgfsetbuttcap%
\pgfsetroundjoin%
\definecolor{currentfill}{rgb}{0.000000,0.000000,0.000000}%
\pgfsetfillcolor{currentfill}%
\pgfsetlinewidth{0.501875pt}%
\definecolor{currentstroke}{rgb}{0.000000,0.000000,0.000000}%
\pgfsetstrokecolor{currentstroke}%
\pgfsetdash{}{0pt}%
\pgfsys@defobject{currentmarker}{\pgfqpoint{0.000000in}{0.000000in}}{\pgfqpoint{0.000000in}{0.020833in}}{%
\pgfpathmoveto{\pgfqpoint{0.000000in}{0.000000in}}%
\pgfpathlineto{\pgfqpoint{0.000000in}{0.020833in}}%
\pgfusepath{stroke,fill}%
}%
\begin{pgfscope}%
\pgfsys@transformshift{3.866694in}{0.586309in}%
\pgfsys@useobject{currentmarker}{}%
\end{pgfscope}%
\end{pgfscope}%
\begin{pgfscope}%
\pgfsetbuttcap%
\pgfsetroundjoin%
\definecolor{currentfill}{rgb}{0.000000,0.000000,0.000000}%
\pgfsetfillcolor{currentfill}%
\pgfsetlinewidth{0.501875pt}%
\definecolor{currentstroke}{rgb}{0.000000,0.000000,0.000000}%
\pgfsetstrokecolor{currentstroke}%
\pgfsetdash{}{0pt}%
\pgfsys@defobject{currentmarker}{\pgfqpoint{0.000000in}{-0.020833in}}{\pgfqpoint{0.000000in}{0.000000in}}{%
\pgfpathmoveto{\pgfqpoint{0.000000in}{0.000000in}}%
\pgfpathlineto{\pgfqpoint{0.000000in}{-0.020833in}}%
\pgfusepath{stroke,fill}%
}%
\begin{pgfscope}%
\pgfsys@transformshift{3.866694in}{0.893003in}%
\pgfsys@useobject{currentmarker}{}%
\end{pgfscope}%
\end{pgfscope}%
\begin{pgfscope}%
\pgfsetbuttcap%
\pgfsetroundjoin%
\definecolor{currentfill}{rgb}{0.000000,0.000000,0.000000}%
\pgfsetfillcolor{currentfill}%
\pgfsetlinewidth{0.501875pt}%
\definecolor{currentstroke}{rgb}{0.000000,0.000000,0.000000}%
\pgfsetstrokecolor{currentstroke}%
\pgfsetdash{}{0pt}%
\pgfsys@defobject{currentmarker}{\pgfqpoint{0.000000in}{0.000000in}}{\pgfqpoint{0.000000in}{0.020833in}}{%
\pgfpathmoveto{\pgfqpoint{0.000000in}{0.000000in}}%
\pgfpathlineto{\pgfqpoint{0.000000in}{0.020833in}}%
\pgfusepath{stroke,fill}%
}%
\begin{pgfscope}%
\pgfsys@transformshift{3.902391in}{0.586309in}%
\pgfsys@useobject{currentmarker}{}%
\end{pgfscope}%
\end{pgfscope}%
\begin{pgfscope}%
\pgfsetbuttcap%
\pgfsetroundjoin%
\definecolor{currentfill}{rgb}{0.000000,0.000000,0.000000}%
\pgfsetfillcolor{currentfill}%
\pgfsetlinewidth{0.501875pt}%
\definecolor{currentstroke}{rgb}{0.000000,0.000000,0.000000}%
\pgfsetstrokecolor{currentstroke}%
\pgfsetdash{}{0pt}%
\pgfsys@defobject{currentmarker}{\pgfqpoint{0.000000in}{-0.020833in}}{\pgfqpoint{0.000000in}{0.000000in}}{%
\pgfpathmoveto{\pgfqpoint{0.000000in}{0.000000in}}%
\pgfpathlineto{\pgfqpoint{0.000000in}{-0.020833in}}%
\pgfusepath{stroke,fill}%
}%
\begin{pgfscope}%
\pgfsys@transformshift{3.902391in}{0.893003in}%
\pgfsys@useobject{currentmarker}{}%
\end{pgfscope}%
\end{pgfscope}%
\begin{pgfscope}%
\pgfsetbuttcap%
\pgfsetroundjoin%
\definecolor{currentfill}{rgb}{0.000000,0.000000,0.000000}%
\pgfsetfillcolor{currentfill}%
\pgfsetlinewidth{0.501875pt}%
\definecolor{currentstroke}{rgb}{0.000000,0.000000,0.000000}%
\pgfsetstrokecolor{currentstroke}%
\pgfsetdash{}{0pt}%
\pgfsys@defobject{currentmarker}{\pgfqpoint{0.000000in}{0.000000in}}{\pgfqpoint{0.000000in}{0.020833in}}{%
\pgfpathmoveto{\pgfqpoint{0.000000in}{0.000000in}}%
\pgfpathlineto{\pgfqpoint{0.000000in}{0.020833in}}%
\pgfusepath{stroke,fill}%
}%
\begin{pgfscope}%
\pgfsys@transformshift{3.938088in}{0.586309in}%
\pgfsys@useobject{currentmarker}{}%
\end{pgfscope}%
\end{pgfscope}%
\begin{pgfscope}%
\pgfsetbuttcap%
\pgfsetroundjoin%
\definecolor{currentfill}{rgb}{0.000000,0.000000,0.000000}%
\pgfsetfillcolor{currentfill}%
\pgfsetlinewidth{0.501875pt}%
\definecolor{currentstroke}{rgb}{0.000000,0.000000,0.000000}%
\pgfsetstrokecolor{currentstroke}%
\pgfsetdash{}{0pt}%
\pgfsys@defobject{currentmarker}{\pgfqpoint{0.000000in}{-0.020833in}}{\pgfqpoint{0.000000in}{0.000000in}}{%
\pgfpathmoveto{\pgfqpoint{0.000000in}{0.000000in}}%
\pgfpathlineto{\pgfqpoint{0.000000in}{-0.020833in}}%
\pgfusepath{stroke,fill}%
}%
\begin{pgfscope}%
\pgfsys@transformshift{3.938088in}{0.893003in}%
\pgfsys@useobject{currentmarker}{}%
\end{pgfscope}%
\end{pgfscope}%
\begin{pgfscope}%
\pgfsetbuttcap%
\pgfsetroundjoin%
\definecolor{currentfill}{rgb}{0.000000,0.000000,0.000000}%
\pgfsetfillcolor{currentfill}%
\pgfsetlinewidth{0.501875pt}%
\definecolor{currentstroke}{rgb}{0.000000,0.000000,0.000000}%
\pgfsetstrokecolor{currentstroke}%
\pgfsetdash{}{0pt}%
\pgfsys@defobject{currentmarker}{\pgfqpoint{0.000000in}{0.000000in}}{\pgfqpoint{0.000000in}{0.020833in}}{%
\pgfpathmoveto{\pgfqpoint{0.000000in}{0.000000in}}%
\pgfpathlineto{\pgfqpoint{0.000000in}{0.020833in}}%
\pgfusepath{stroke,fill}%
}%
\begin{pgfscope}%
\pgfsys@transformshift{3.973786in}{0.586309in}%
\pgfsys@useobject{currentmarker}{}%
\end{pgfscope}%
\end{pgfscope}%
\begin{pgfscope}%
\pgfsetbuttcap%
\pgfsetroundjoin%
\definecolor{currentfill}{rgb}{0.000000,0.000000,0.000000}%
\pgfsetfillcolor{currentfill}%
\pgfsetlinewidth{0.501875pt}%
\definecolor{currentstroke}{rgb}{0.000000,0.000000,0.000000}%
\pgfsetstrokecolor{currentstroke}%
\pgfsetdash{}{0pt}%
\pgfsys@defobject{currentmarker}{\pgfqpoint{0.000000in}{-0.020833in}}{\pgfqpoint{0.000000in}{0.000000in}}{%
\pgfpathmoveto{\pgfqpoint{0.000000in}{0.000000in}}%
\pgfpathlineto{\pgfqpoint{0.000000in}{-0.020833in}}%
\pgfusepath{stroke,fill}%
}%
\begin{pgfscope}%
\pgfsys@transformshift{3.973786in}{0.893003in}%
\pgfsys@useobject{currentmarker}{}%
\end{pgfscope}%
\end{pgfscope}%
\begin{pgfscope}%
\pgfsetbuttcap%
\pgfsetroundjoin%
\definecolor{currentfill}{rgb}{0.000000,0.000000,0.000000}%
\pgfsetfillcolor{currentfill}%
\pgfsetlinewidth{0.501875pt}%
\definecolor{currentstroke}{rgb}{0.000000,0.000000,0.000000}%
\pgfsetstrokecolor{currentstroke}%
\pgfsetdash{}{0pt}%
\pgfsys@defobject{currentmarker}{\pgfqpoint{0.000000in}{0.000000in}}{\pgfqpoint{0.000000in}{0.020833in}}{%
\pgfpathmoveto{\pgfqpoint{0.000000in}{0.000000in}}%
\pgfpathlineto{\pgfqpoint{0.000000in}{0.020833in}}%
\pgfusepath{stroke,fill}%
}%
\begin{pgfscope}%
\pgfsys@transformshift{4.009483in}{0.586309in}%
\pgfsys@useobject{currentmarker}{}%
\end{pgfscope}%
\end{pgfscope}%
\begin{pgfscope}%
\pgfsetbuttcap%
\pgfsetroundjoin%
\definecolor{currentfill}{rgb}{0.000000,0.000000,0.000000}%
\pgfsetfillcolor{currentfill}%
\pgfsetlinewidth{0.501875pt}%
\definecolor{currentstroke}{rgb}{0.000000,0.000000,0.000000}%
\pgfsetstrokecolor{currentstroke}%
\pgfsetdash{}{0pt}%
\pgfsys@defobject{currentmarker}{\pgfqpoint{0.000000in}{-0.020833in}}{\pgfqpoint{0.000000in}{0.000000in}}{%
\pgfpathmoveto{\pgfqpoint{0.000000in}{0.000000in}}%
\pgfpathlineto{\pgfqpoint{0.000000in}{-0.020833in}}%
\pgfusepath{stroke,fill}%
}%
\begin{pgfscope}%
\pgfsys@transformshift{4.009483in}{0.893003in}%
\pgfsys@useobject{currentmarker}{}%
\end{pgfscope}%
\end{pgfscope}%
\begin{pgfscope}%
\pgfsetbuttcap%
\pgfsetroundjoin%
\definecolor{currentfill}{rgb}{0.000000,0.000000,0.000000}%
\pgfsetfillcolor{currentfill}%
\pgfsetlinewidth{0.501875pt}%
\definecolor{currentstroke}{rgb}{0.000000,0.000000,0.000000}%
\pgfsetstrokecolor{currentstroke}%
\pgfsetdash{}{0pt}%
\pgfsys@defobject{currentmarker}{\pgfqpoint{0.000000in}{0.000000in}}{\pgfqpoint{0.000000in}{0.020833in}}{%
\pgfpathmoveto{\pgfqpoint{0.000000in}{0.000000in}}%
\pgfpathlineto{\pgfqpoint{0.000000in}{0.020833in}}%
\pgfusepath{stroke,fill}%
}%
\begin{pgfscope}%
\pgfsys@transformshift{4.045180in}{0.586309in}%
\pgfsys@useobject{currentmarker}{}%
\end{pgfscope}%
\end{pgfscope}%
\begin{pgfscope}%
\pgfsetbuttcap%
\pgfsetroundjoin%
\definecolor{currentfill}{rgb}{0.000000,0.000000,0.000000}%
\pgfsetfillcolor{currentfill}%
\pgfsetlinewidth{0.501875pt}%
\definecolor{currentstroke}{rgb}{0.000000,0.000000,0.000000}%
\pgfsetstrokecolor{currentstroke}%
\pgfsetdash{}{0pt}%
\pgfsys@defobject{currentmarker}{\pgfqpoint{0.000000in}{-0.020833in}}{\pgfqpoint{0.000000in}{0.000000in}}{%
\pgfpathmoveto{\pgfqpoint{0.000000in}{0.000000in}}%
\pgfpathlineto{\pgfqpoint{0.000000in}{-0.020833in}}%
\pgfusepath{stroke,fill}%
}%
\begin{pgfscope}%
\pgfsys@transformshift{4.045180in}{0.893003in}%
\pgfsys@useobject{currentmarker}{}%
\end{pgfscope}%
\end{pgfscope}%
\begin{pgfscope}%
\pgfsetbuttcap%
\pgfsetroundjoin%
\definecolor{currentfill}{rgb}{0.000000,0.000000,0.000000}%
\pgfsetfillcolor{currentfill}%
\pgfsetlinewidth{0.501875pt}%
\definecolor{currentstroke}{rgb}{0.000000,0.000000,0.000000}%
\pgfsetstrokecolor{currentstroke}%
\pgfsetdash{}{0pt}%
\pgfsys@defobject{currentmarker}{\pgfqpoint{0.000000in}{0.000000in}}{\pgfqpoint{0.000000in}{0.020833in}}{%
\pgfpathmoveto{\pgfqpoint{0.000000in}{0.000000in}}%
\pgfpathlineto{\pgfqpoint{0.000000in}{0.020833in}}%
\pgfusepath{stroke,fill}%
}%
\begin{pgfscope}%
\pgfsys@transformshift{4.080878in}{0.586309in}%
\pgfsys@useobject{currentmarker}{}%
\end{pgfscope}%
\end{pgfscope}%
\begin{pgfscope}%
\pgfsetbuttcap%
\pgfsetroundjoin%
\definecolor{currentfill}{rgb}{0.000000,0.000000,0.000000}%
\pgfsetfillcolor{currentfill}%
\pgfsetlinewidth{0.501875pt}%
\definecolor{currentstroke}{rgb}{0.000000,0.000000,0.000000}%
\pgfsetstrokecolor{currentstroke}%
\pgfsetdash{}{0pt}%
\pgfsys@defobject{currentmarker}{\pgfqpoint{0.000000in}{-0.020833in}}{\pgfqpoint{0.000000in}{0.000000in}}{%
\pgfpathmoveto{\pgfqpoint{0.000000in}{0.000000in}}%
\pgfpathlineto{\pgfqpoint{0.000000in}{-0.020833in}}%
\pgfusepath{stroke,fill}%
}%
\begin{pgfscope}%
\pgfsys@transformshift{4.080878in}{0.893003in}%
\pgfsys@useobject{currentmarker}{}%
\end{pgfscope}%
\end{pgfscope}%
\begin{pgfscope}%
\pgfsetbuttcap%
\pgfsetroundjoin%
\definecolor{currentfill}{rgb}{0.000000,0.000000,0.000000}%
\pgfsetfillcolor{currentfill}%
\pgfsetlinewidth{0.501875pt}%
\definecolor{currentstroke}{rgb}{0.000000,0.000000,0.000000}%
\pgfsetstrokecolor{currentstroke}%
\pgfsetdash{}{0pt}%
\pgfsys@defobject{currentmarker}{\pgfqpoint{0.000000in}{0.000000in}}{\pgfqpoint{0.000000in}{0.020833in}}{%
\pgfpathmoveto{\pgfqpoint{0.000000in}{0.000000in}}%
\pgfpathlineto{\pgfqpoint{0.000000in}{0.020833in}}%
\pgfusepath{stroke,fill}%
}%
\begin{pgfscope}%
\pgfsys@transformshift{4.152272in}{0.586309in}%
\pgfsys@useobject{currentmarker}{}%
\end{pgfscope}%
\end{pgfscope}%
\begin{pgfscope}%
\pgfsetbuttcap%
\pgfsetroundjoin%
\definecolor{currentfill}{rgb}{0.000000,0.000000,0.000000}%
\pgfsetfillcolor{currentfill}%
\pgfsetlinewidth{0.501875pt}%
\definecolor{currentstroke}{rgb}{0.000000,0.000000,0.000000}%
\pgfsetstrokecolor{currentstroke}%
\pgfsetdash{}{0pt}%
\pgfsys@defobject{currentmarker}{\pgfqpoint{0.000000in}{-0.020833in}}{\pgfqpoint{0.000000in}{0.000000in}}{%
\pgfpathmoveto{\pgfqpoint{0.000000in}{0.000000in}}%
\pgfpathlineto{\pgfqpoint{0.000000in}{-0.020833in}}%
\pgfusepath{stroke,fill}%
}%
\begin{pgfscope}%
\pgfsys@transformshift{4.152272in}{0.893003in}%
\pgfsys@useobject{currentmarker}{}%
\end{pgfscope}%
\end{pgfscope}%
\begin{pgfscope}%
\pgfsetbuttcap%
\pgfsetroundjoin%
\definecolor{currentfill}{rgb}{0.000000,0.000000,0.000000}%
\pgfsetfillcolor{currentfill}%
\pgfsetlinewidth{0.501875pt}%
\definecolor{currentstroke}{rgb}{0.000000,0.000000,0.000000}%
\pgfsetstrokecolor{currentstroke}%
\pgfsetdash{}{0pt}%
\pgfsys@defobject{currentmarker}{\pgfqpoint{0.000000in}{0.000000in}}{\pgfqpoint{0.000000in}{0.020833in}}{%
\pgfpathmoveto{\pgfqpoint{0.000000in}{0.000000in}}%
\pgfpathlineto{\pgfqpoint{0.000000in}{0.020833in}}%
\pgfusepath{stroke,fill}%
}%
\begin{pgfscope}%
\pgfsys@transformshift{4.187970in}{0.586309in}%
\pgfsys@useobject{currentmarker}{}%
\end{pgfscope}%
\end{pgfscope}%
\begin{pgfscope}%
\pgfsetbuttcap%
\pgfsetroundjoin%
\definecolor{currentfill}{rgb}{0.000000,0.000000,0.000000}%
\pgfsetfillcolor{currentfill}%
\pgfsetlinewidth{0.501875pt}%
\definecolor{currentstroke}{rgb}{0.000000,0.000000,0.000000}%
\pgfsetstrokecolor{currentstroke}%
\pgfsetdash{}{0pt}%
\pgfsys@defobject{currentmarker}{\pgfqpoint{0.000000in}{-0.020833in}}{\pgfqpoint{0.000000in}{0.000000in}}{%
\pgfpathmoveto{\pgfqpoint{0.000000in}{0.000000in}}%
\pgfpathlineto{\pgfqpoint{0.000000in}{-0.020833in}}%
\pgfusepath{stroke,fill}%
}%
\begin{pgfscope}%
\pgfsys@transformshift{4.187970in}{0.893003in}%
\pgfsys@useobject{currentmarker}{}%
\end{pgfscope}%
\end{pgfscope}%
\begin{pgfscope}%
\pgfsetbuttcap%
\pgfsetroundjoin%
\definecolor{currentfill}{rgb}{0.000000,0.000000,0.000000}%
\pgfsetfillcolor{currentfill}%
\pgfsetlinewidth{0.501875pt}%
\definecolor{currentstroke}{rgb}{0.000000,0.000000,0.000000}%
\pgfsetstrokecolor{currentstroke}%
\pgfsetdash{}{0pt}%
\pgfsys@defobject{currentmarker}{\pgfqpoint{0.000000in}{0.000000in}}{\pgfqpoint{0.000000in}{0.020833in}}{%
\pgfpathmoveto{\pgfqpoint{0.000000in}{0.000000in}}%
\pgfpathlineto{\pgfqpoint{0.000000in}{0.020833in}}%
\pgfusepath{stroke,fill}%
}%
\begin{pgfscope}%
\pgfsys@transformshift{4.223667in}{0.586309in}%
\pgfsys@useobject{currentmarker}{}%
\end{pgfscope}%
\end{pgfscope}%
\begin{pgfscope}%
\pgfsetbuttcap%
\pgfsetroundjoin%
\definecolor{currentfill}{rgb}{0.000000,0.000000,0.000000}%
\pgfsetfillcolor{currentfill}%
\pgfsetlinewidth{0.501875pt}%
\definecolor{currentstroke}{rgb}{0.000000,0.000000,0.000000}%
\pgfsetstrokecolor{currentstroke}%
\pgfsetdash{}{0pt}%
\pgfsys@defobject{currentmarker}{\pgfqpoint{0.000000in}{-0.020833in}}{\pgfqpoint{0.000000in}{0.000000in}}{%
\pgfpathmoveto{\pgfqpoint{0.000000in}{0.000000in}}%
\pgfpathlineto{\pgfqpoint{0.000000in}{-0.020833in}}%
\pgfusepath{stroke,fill}%
}%
\begin{pgfscope}%
\pgfsys@transformshift{4.223667in}{0.893003in}%
\pgfsys@useobject{currentmarker}{}%
\end{pgfscope}%
\end{pgfscope}%
\begin{pgfscope}%
\pgfsetbuttcap%
\pgfsetroundjoin%
\definecolor{currentfill}{rgb}{0.000000,0.000000,0.000000}%
\pgfsetfillcolor{currentfill}%
\pgfsetlinewidth{0.501875pt}%
\definecolor{currentstroke}{rgb}{0.000000,0.000000,0.000000}%
\pgfsetstrokecolor{currentstroke}%
\pgfsetdash{}{0pt}%
\pgfsys@defobject{currentmarker}{\pgfqpoint{0.000000in}{0.000000in}}{\pgfqpoint{0.000000in}{0.020833in}}{%
\pgfpathmoveto{\pgfqpoint{0.000000in}{0.000000in}}%
\pgfpathlineto{\pgfqpoint{0.000000in}{0.020833in}}%
\pgfusepath{stroke,fill}%
}%
\begin{pgfscope}%
\pgfsys@transformshift{4.259364in}{0.586309in}%
\pgfsys@useobject{currentmarker}{}%
\end{pgfscope}%
\end{pgfscope}%
\begin{pgfscope}%
\pgfsetbuttcap%
\pgfsetroundjoin%
\definecolor{currentfill}{rgb}{0.000000,0.000000,0.000000}%
\pgfsetfillcolor{currentfill}%
\pgfsetlinewidth{0.501875pt}%
\definecolor{currentstroke}{rgb}{0.000000,0.000000,0.000000}%
\pgfsetstrokecolor{currentstroke}%
\pgfsetdash{}{0pt}%
\pgfsys@defobject{currentmarker}{\pgfqpoint{0.000000in}{-0.020833in}}{\pgfqpoint{0.000000in}{0.000000in}}{%
\pgfpathmoveto{\pgfqpoint{0.000000in}{0.000000in}}%
\pgfpathlineto{\pgfqpoint{0.000000in}{-0.020833in}}%
\pgfusepath{stroke,fill}%
}%
\begin{pgfscope}%
\pgfsys@transformshift{4.259364in}{0.893003in}%
\pgfsys@useobject{currentmarker}{}%
\end{pgfscope}%
\end{pgfscope}%
\begin{pgfscope}%
\pgfsetbuttcap%
\pgfsetroundjoin%
\definecolor{currentfill}{rgb}{0.000000,0.000000,0.000000}%
\pgfsetfillcolor{currentfill}%
\pgfsetlinewidth{0.501875pt}%
\definecolor{currentstroke}{rgb}{0.000000,0.000000,0.000000}%
\pgfsetstrokecolor{currentstroke}%
\pgfsetdash{}{0pt}%
\pgfsys@defobject{currentmarker}{\pgfqpoint{0.000000in}{0.000000in}}{\pgfqpoint{0.000000in}{0.020833in}}{%
\pgfpathmoveto{\pgfqpoint{0.000000in}{0.000000in}}%
\pgfpathlineto{\pgfqpoint{0.000000in}{0.020833in}}%
\pgfusepath{stroke,fill}%
}%
\begin{pgfscope}%
\pgfsys@transformshift{4.295061in}{0.586309in}%
\pgfsys@useobject{currentmarker}{}%
\end{pgfscope}%
\end{pgfscope}%
\begin{pgfscope}%
\pgfsetbuttcap%
\pgfsetroundjoin%
\definecolor{currentfill}{rgb}{0.000000,0.000000,0.000000}%
\pgfsetfillcolor{currentfill}%
\pgfsetlinewidth{0.501875pt}%
\definecolor{currentstroke}{rgb}{0.000000,0.000000,0.000000}%
\pgfsetstrokecolor{currentstroke}%
\pgfsetdash{}{0pt}%
\pgfsys@defobject{currentmarker}{\pgfqpoint{0.000000in}{-0.020833in}}{\pgfqpoint{0.000000in}{0.000000in}}{%
\pgfpathmoveto{\pgfqpoint{0.000000in}{0.000000in}}%
\pgfpathlineto{\pgfqpoint{0.000000in}{-0.020833in}}%
\pgfusepath{stroke,fill}%
}%
\begin{pgfscope}%
\pgfsys@transformshift{4.295061in}{0.893003in}%
\pgfsys@useobject{currentmarker}{}%
\end{pgfscope}%
\end{pgfscope}%
\begin{pgfscope}%
\pgfsetbuttcap%
\pgfsetroundjoin%
\definecolor{currentfill}{rgb}{0.000000,0.000000,0.000000}%
\pgfsetfillcolor{currentfill}%
\pgfsetlinewidth{0.501875pt}%
\definecolor{currentstroke}{rgb}{0.000000,0.000000,0.000000}%
\pgfsetstrokecolor{currentstroke}%
\pgfsetdash{}{0pt}%
\pgfsys@defobject{currentmarker}{\pgfqpoint{0.000000in}{0.000000in}}{\pgfqpoint{0.000000in}{0.020833in}}{%
\pgfpathmoveto{\pgfqpoint{0.000000in}{0.000000in}}%
\pgfpathlineto{\pgfqpoint{0.000000in}{0.020833in}}%
\pgfusepath{stroke,fill}%
}%
\begin{pgfscope}%
\pgfsys@transformshift{4.330759in}{0.586309in}%
\pgfsys@useobject{currentmarker}{}%
\end{pgfscope}%
\end{pgfscope}%
\begin{pgfscope}%
\pgfsetbuttcap%
\pgfsetroundjoin%
\definecolor{currentfill}{rgb}{0.000000,0.000000,0.000000}%
\pgfsetfillcolor{currentfill}%
\pgfsetlinewidth{0.501875pt}%
\definecolor{currentstroke}{rgb}{0.000000,0.000000,0.000000}%
\pgfsetstrokecolor{currentstroke}%
\pgfsetdash{}{0pt}%
\pgfsys@defobject{currentmarker}{\pgfqpoint{0.000000in}{-0.020833in}}{\pgfqpoint{0.000000in}{0.000000in}}{%
\pgfpathmoveto{\pgfqpoint{0.000000in}{0.000000in}}%
\pgfpathlineto{\pgfqpoint{0.000000in}{-0.020833in}}%
\pgfusepath{stroke,fill}%
}%
\begin{pgfscope}%
\pgfsys@transformshift{4.330759in}{0.893003in}%
\pgfsys@useobject{currentmarker}{}%
\end{pgfscope}%
\end{pgfscope}%
\begin{pgfscope}%
\pgfsetbuttcap%
\pgfsetroundjoin%
\definecolor{currentfill}{rgb}{0.000000,0.000000,0.000000}%
\pgfsetfillcolor{currentfill}%
\pgfsetlinewidth{0.501875pt}%
\definecolor{currentstroke}{rgb}{0.000000,0.000000,0.000000}%
\pgfsetstrokecolor{currentstroke}%
\pgfsetdash{}{0pt}%
\pgfsys@defobject{currentmarker}{\pgfqpoint{0.000000in}{0.000000in}}{\pgfqpoint{0.000000in}{0.020833in}}{%
\pgfpathmoveto{\pgfqpoint{0.000000in}{0.000000in}}%
\pgfpathlineto{\pgfqpoint{0.000000in}{0.020833in}}%
\pgfusepath{stroke,fill}%
}%
\begin{pgfscope}%
\pgfsys@transformshift{4.366456in}{0.586309in}%
\pgfsys@useobject{currentmarker}{}%
\end{pgfscope}%
\end{pgfscope}%
\begin{pgfscope}%
\pgfsetbuttcap%
\pgfsetroundjoin%
\definecolor{currentfill}{rgb}{0.000000,0.000000,0.000000}%
\pgfsetfillcolor{currentfill}%
\pgfsetlinewidth{0.501875pt}%
\definecolor{currentstroke}{rgb}{0.000000,0.000000,0.000000}%
\pgfsetstrokecolor{currentstroke}%
\pgfsetdash{}{0pt}%
\pgfsys@defobject{currentmarker}{\pgfqpoint{0.000000in}{-0.020833in}}{\pgfqpoint{0.000000in}{0.000000in}}{%
\pgfpathmoveto{\pgfqpoint{0.000000in}{0.000000in}}%
\pgfpathlineto{\pgfqpoint{0.000000in}{-0.020833in}}%
\pgfusepath{stroke,fill}%
}%
\begin{pgfscope}%
\pgfsys@transformshift{4.366456in}{0.893003in}%
\pgfsys@useobject{currentmarker}{}%
\end{pgfscope}%
\end{pgfscope}%
\begin{pgfscope}%
\pgfsetbuttcap%
\pgfsetroundjoin%
\definecolor{currentfill}{rgb}{0.000000,0.000000,0.000000}%
\pgfsetfillcolor{currentfill}%
\pgfsetlinewidth{0.501875pt}%
\definecolor{currentstroke}{rgb}{0.000000,0.000000,0.000000}%
\pgfsetstrokecolor{currentstroke}%
\pgfsetdash{}{0pt}%
\pgfsys@defobject{currentmarker}{\pgfqpoint{0.000000in}{0.000000in}}{\pgfqpoint{0.000000in}{0.020833in}}{%
\pgfpathmoveto{\pgfqpoint{0.000000in}{0.000000in}}%
\pgfpathlineto{\pgfqpoint{0.000000in}{0.020833in}}%
\pgfusepath{stroke,fill}%
}%
\begin{pgfscope}%
\pgfsys@transformshift{4.402153in}{0.586309in}%
\pgfsys@useobject{currentmarker}{}%
\end{pgfscope}%
\end{pgfscope}%
\begin{pgfscope}%
\pgfsetbuttcap%
\pgfsetroundjoin%
\definecolor{currentfill}{rgb}{0.000000,0.000000,0.000000}%
\pgfsetfillcolor{currentfill}%
\pgfsetlinewidth{0.501875pt}%
\definecolor{currentstroke}{rgb}{0.000000,0.000000,0.000000}%
\pgfsetstrokecolor{currentstroke}%
\pgfsetdash{}{0pt}%
\pgfsys@defobject{currentmarker}{\pgfqpoint{0.000000in}{-0.020833in}}{\pgfqpoint{0.000000in}{0.000000in}}{%
\pgfpathmoveto{\pgfqpoint{0.000000in}{0.000000in}}%
\pgfpathlineto{\pgfqpoint{0.000000in}{-0.020833in}}%
\pgfusepath{stroke,fill}%
}%
\begin{pgfscope}%
\pgfsys@transformshift{4.402153in}{0.893003in}%
\pgfsys@useobject{currentmarker}{}%
\end{pgfscope}%
\end{pgfscope}%
\begin{pgfscope}%
\pgfsetbuttcap%
\pgfsetroundjoin%
\definecolor{currentfill}{rgb}{0.000000,0.000000,0.000000}%
\pgfsetfillcolor{currentfill}%
\pgfsetlinewidth{0.501875pt}%
\definecolor{currentstroke}{rgb}{0.000000,0.000000,0.000000}%
\pgfsetstrokecolor{currentstroke}%
\pgfsetdash{}{0pt}%
\pgfsys@defobject{currentmarker}{\pgfqpoint{0.000000in}{0.000000in}}{\pgfqpoint{0.000000in}{0.020833in}}{%
\pgfpathmoveto{\pgfqpoint{0.000000in}{0.000000in}}%
\pgfpathlineto{\pgfqpoint{0.000000in}{0.020833in}}%
\pgfusepath{stroke,fill}%
}%
\begin{pgfscope}%
\pgfsys@transformshift{4.437851in}{0.586309in}%
\pgfsys@useobject{currentmarker}{}%
\end{pgfscope}%
\end{pgfscope}%
\begin{pgfscope}%
\pgfsetbuttcap%
\pgfsetroundjoin%
\definecolor{currentfill}{rgb}{0.000000,0.000000,0.000000}%
\pgfsetfillcolor{currentfill}%
\pgfsetlinewidth{0.501875pt}%
\definecolor{currentstroke}{rgb}{0.000000,0.000000,0.000000}%
\pgfsetstrokecolor{currentstroke}%
\pgfsetdash{}{0pt}%
\pgfsys@defobject{currentmarker}{\pgfqpoint{0.000000in}{-0.020833in}}{\pgfqpoint{0.000000in}{0.000000in}}{%
\pgfpathmoveto{\pgfqpoint{0.000000in}{0.000000in}}%
\pgfpathlineto{\pgfqpoint{0.000000in}{-0.020833in}}%
\pgfusepath{stroke,fill}%
}%
\begin{pgfscope}%
\pgfsys@transformshift{4.437851in}{0.893003in}%
\pgfsys@useobject{currentmarker}{}%
\end{pgfscope}%
\end{pgfscope}%
\begin{pgfscope}%
\pgfsetbuttcap%
\pgfsetroundjoin%
\definecolor{currentfill}{rgb}{0.000000,0.000000,0.000000}%
\pgfsetfillcolor{currentfill}%
\pgfsetlinewidth{0.501875pt}%
\definecolor{currentstroke}{rgb}{0.000000,0.000000,0.000000}%
\pgfsetstrokecolor{currentstroke}%
\pgfsetdash{}{0pt}%
\pgfsys@defobject{currentmarker}{\pgfqpoint{0.000000in}{0.000000in}}{\pgfqpoint{0.000000in}{0.020833in}}{%
\pgfpathmoveto{\pgfqpoint{0.000000in}{0.000000in}}%
\pgfpathlineto{\pgfqpoint{0.000000in}{0.020833in}}%
\pgfusepath{stroke,fill}%
}%
\begin{pgfscope}%
\pgfsys@transformshift{4.473548in}{0.586309in}%
\pgfsys@useobject{currentmarker}{}%
\end{pgfscope}%
\end{pgfscope}%
\begin{pgfscope}%
\pgfsetbuttcap%
\pgfsetroundjoin%
\definecolor{currentfill}{rgb}{0.000000,0.000000,0.000000}%
\pgfsetfillcolor{currentfill}%
\pgfsetlinewidth{0.501875pt}%
\definecolor{currentstroke}{rgb}{0.000000,0.000000,0.000000}%
\pgfsetstrokecolor{currentstroke}%
\pgfsetdash{}{0pt}%
\pgfsys@defobject{currentmarker}{\pgfqpoint{0.000000in}{-0.020833in}}{\pgfqpoint{0.000000in}{0.000000in}}{%
\pgfpathmoveto{\pgfqpoint{0.000000in}{0.000000in}}%
\pgfpathlineto{\pgfqpoint{0.000000in}{-0.020833in}}%
\pgfusepath{stroke,fill}%
}%
\begin{pgfscope}%
\pgfsys@transformshift{4.473548in}{0.893003in}%
\pgfsys@useobject{currentmarker}{}%
\end{pgfscope}%
\end{pgfscope}%
\begin{pgfscope}%
\pgfsetbuttcap%
\pgfsetroundjoin%
\definecolor{currentfill}{rgb}{0.000000,0.000000,0.000000}%
\pgfsetfillcolor{currentfill}%
\pgfsetlinewidth{0.501875pt}%
\definecolor{currentstroke}{rgb}{0.000000,0.000000,0.000000}%
\pgfsetstrokecolor{currentstroke}%
\pgfsetdash{}{0pt}%
\pgfsys@defobject{currentmarker}{\pgfqpoint{0.000000in}{0.000000in}}{\pgfqpoint{0.000000in}{0.020833in}}{%
\pgfpathmoveto{\pgfqpoint{0.000000in}{0.000000in}}%
\pgfpathlineto{\pgfqpoint{0.000000in}{0.020833in}}%
\pgfusepath{stroke,fill}%
}%
\begin{pgfscope}%
\pgfsys@transformshift{4.509245in}{0.586309in}%
\pgfsys@useobject{currentmarker}{}%
\end{pgfscope}%
\end{pgfscope}%
\begin{pgfscope}%
\pgfsetbuttcap%
\pgfsetroundjoin%
\definecolor{currentfill}{rgb}{0.000000,0.000000,0.000000}%
\pgfsetfillcolor{currentfill}%
\pgfsetlinewidth{0.501875pt}%
\definecolor{currentstroke}{rgb}{0.000000,0.000000,0.000000}%
\pgfsetstrokecolor{currentstroke}%
\pgfsetdash{}{0pt}%
\pgfsys@defobject{currentmarker}{\pgfqpoint{0.000000in}{-0.020833in}}{\pgfqpoint{0.000000in}{0.000000in}}{%
\pgfpathmoveto{\pgfqpoint{0.000000in}{0.000000in}}%
\pgfpathlineto{\pgfqpoint{0.000000in}{-0.020833in}}%
\pgfusepath{stroke,fill}%
}%
\begin{pgfscope}%
\pgfsys@transformshift{4.509245in}{0.893003in}%
\pgfsys@useobject{currentmarker}{}%
\end{pgfscope}%
\end{pgfscope}%
\begin{pgfscope}%
\pgfsetbuttcap%
\pgfsetroundjoin%
\definecolor{currentfill}{rgb}{0.000000,0.000000,0.000000}%
\pgfsetfillcolor{currentfill}%
\pgfsetlinewidth{0.501875pt}%
\definecolor{currentstroke}{rgb}{0.000000,0.000000,0.000000}%
\pgfsetstrokecolor{currentstroke}%
\pgfsetdash{}{0pt}%
\pgfsys@defobject{currentmarker}{\pgfqpoint{0.000000in}{0.000000in}}{\pgfqpoint{0.000000in}{0.020833in}}{%
\pgfpathmoveto{\pgfqpoint{0.000000in}{0.000000in}}%
\pgfpathlineto{\pgfqpoint{0.000000in}{0.020833in}}%
\pgfusepath{stroke,fill}%
}%
\begin{pgfscope}%
\pgfsys@transformshift{4.580640in}{0.586309in}%
\pgfsys@useobject{currentmarker}{}%
\end{pgfscope}%
\end{pgfscope}%
\begin{pgfscope}%
\pgfsetbuttcap%
\pgfsetroundjoin%
\definecolor{currentfill}{rgb}{0.000000,0.000000,0.000000}%
\pgfsetfillcolor{currentfill}%
\pgfsetlinewidth{0.501875pt}%
\definecolor{currentstroke}{rgb}{0.000000,0.000000,0.000000}%
\pgfsetstrokecolor{currentstroke}%
\pgfsetdash{}{0pt}%
\pgfsys@defobject{currentmarker}{\pgfqpoint{0.000000in}{-0.020833in}}{\pgfqpoint{0.000000in}{0.000000in}}{%
\pgfpathmoveto{\pgfqpoint{0.000000in}{0.000000in}}%
\pgfpathlineto{\pgfqpoint{0.000000in}{-0.020833in}}%
\pgfusepath{stroke,fill}%
}%
\begin{pgfscope}%
\pgfsys@transformshift{4.580640in}{0.893003in}%
\pgfsys@useobject{currentmarker}{}%
\end{pgfscope}%
\end{pgfscope}%
\begin{pgfscope}%
\pgfsetbuttcap%
\pgfsetroundjoin%
\definecolor{currentfill}{rgb}{0.000000,0.000000,0.000000}%
\pgfsetfillcolor{currentfill}%
\pgfsetlinewidth{0.501875pt}%
\definecolor{currentstroke}{rgb}{0.000000,0.000000,0.000000}%
\pgfsetstrokecolor{currentstroke}%
\pgfsetdash{}{0pt}%
\pgfsys@defobject{currentmarker}{\pgfqpoint{0.000000in}{0.000000in}}{\pgfqpoint{0.000000in}{0.020833in}}{%
\pgfpathmoveto{\pgfqpoint{0.000000in}{0.000000in}}%
\pgfpathlineto{\pgfqpoint{0.000000in}{0.020833in}}%
\pgfusepath{stroke,fill}%
}%
\begin{pgfscope}%
\pgfsys@transformshift{4.616337in}{0.586309in}%
\pgfsys@useobject{currentmarker}{}%
\end{pgfscope}%
\end{pgfscope}%
\begin{pgfscope}%
\pgfsetbuttcap%
\pgfsetroundjoin%
\definecolor{currentfill}{rgb}{0.000000,0.000000,0.000000}%
\pgfsetfillcolor{currentfill}%
\pgfsetlinewidth{0.501875pt}%
\definecolor{currentstroke}{rgb}{0.000000,0.000000,0.000000}%
\pgfsetstrokecolor{currentstroke}%
\pgfsetdash{}{0pt}%
\pgfsys@defobject{currentmarker}{\pgfqpoint{0.000000in}{-0.020833in}}{\pgfqpoint{0.000000in}{0.000000in}}{%
\pgfpathmoveto{\pgfqpoint{0.000000in}{0.000000in}}%
\pgfpathlineto{\pgfqpoint{0.000000in}{-0.020833in}}%
\pgfusepath{stroke,fill}%
}%
\begin{pgfscope}%
\pgfsys@transformshift{4.616337in}{0.893003in}%
\pgfsys@useobject{currentmarker}{}%
\end{pgfscope}%
\end{pgfscope}%
\begin{pgfscope}%
\pgfsetbuttcap%
\pgfsetroundjoin%
\definecolor{currentfill}{rgb}{0.000000,0.000000,0.000000}%
\pgfsetfillcolor{currentfill}%
\pgfsetlinewidth{0.501875pt}%
\definecolor{currentstroke}{rgb}{0.000000,0.000000,0.000000}%
\pgfsetstrokecolor{currentstroke}%
\pgfsetdash{}{0pt}%
\pgfsys@defobject{currentmarker}{\pgfqpoint{0.000000in}{0.000000in}}{\pgfqpoint{0.000000in}{0.020833in}}{%
\pgfpathmoveto{\pgfqpoint{0.000000in}{0.000000in}}%
\pgfpathlineto{\pgfqpoint{0.000000in}{0.020833in}}%
\pgfusepath{stroke,fill}%
}%
\begin{pgfscope}%
\pgfsys@transformshift{4.652035in}{0.586309in}%
\pgfsys@useobject{currentmarker}{}%
\end{pgfscope}%
\end{pgfscope}%
\begin{pgfscope}%
\pgfsetbuttcap%
\pgfsetroundjoin%
\definecolor{currentfill}{rgb}{0.000000,0.000000,0.000000}%
\pgfsetfillcolor{currentfill}%
\pgfsetlinewidth{0.501875pt}%
\definecolor{currentstroke}{rgb}{0.000000,0.000000,0.000000}%
\pgfsetstrokecolor{currentstroke}%
\pgfsetdash{}{0pt}%
\pgfsys@defobject{currentmarker}{\pgfqpoint{0.000000in}{-0.020833in}}{\pgfqpoint{0.000000in}{0.000000in}}{%
\pgfpathmoveto{\pgfqpoint{0.000000in}{0.000000in}}%
\pgfpathlineto{\pgfqpoint{0.000000in}{-0.020833in}}%
\pgfusepath{stroke,fill}%
}%
\begin{pgfscope}%
\pgfsys@transformshift{4.652035in}{0.893003in}%
\pgfsys@useobject{currentmarker}{}%
\end{pgfscope}%
\end{pgfscope}%
\begin{pgfscope}%
\pgfsetbuttcap%
\pgfsetroundjoin%
\definecolor{currentfill}{rgb}{0.000000,0.000000,0.000000}%
\pgfsetfillcolor{currentfill}%
\pgfsetlinewidth{0.501875pt}%
\definecolor{currentstroke}{rgb}{0.000000,0.000000,0.000000}%
\pgfsetstrokecolor{currentstroke}%
\pgfsetdash{}{0pt}%
\pgfsys@defobject{currentmarker}{\pgfqpoint{0.000000in}{0.000000in}}{\pgfqpoint{0.000000in}{0.020833in}}{%
\pgfpathmoveto{\pgfqpoint{0.000000in}{0.000000in}}%
\pgfpathlineto{\pgfqpoint{0.000000in}{0.020833in}}%
\pgfusepath{stroke,fill}%
}%
\begin{pgfscope}%
\pgfsys@transformshift{4.687732in}{0.586309in}%
\pgfsys@useobject{currentmarker}{}%
\end{pgfscope}%
\end{pgfscope}%
\begin{pgfscope}%
\pgfsetbuttcap%
\pgfsetroundjoin%
\definecolor{currentfill}{rgb}{0.000000,0.000000,0.000000}%
\pgfsetfillcolor{currentfill}%
\pgfsetlinewidth{0.501875pt}%
\definecolor{currentstroke}{rgb}{0.000000,0.000000,0.000000}%
\pgfsetstrokecolor{currentstroke}%
\pgfsetdash{}{0pt}%
\pgfsys@defobject{currentmarker}{\pgfqpoint{0.000000in}{-0.020833in}}{\pgfqpoint{0.000000in}{0.000000in}}{%
\pgfpathmoveto{\pgfqpoint{0.000000in}{0.000000in}}%
\pgfpathlineto{\pgfqpoint{0.000000in}{-0.020833in}}%
\pgfusepath{stroke,fill}%
}%
\begin{pgfscope}%
\pgfsys@transformshift{4.687732in}{0.893003in}%
\pgfsys@useobject{currentmarker}{}%
\end{pgfscope}%
\end{pgfscope}%
\begin{pgfscope}%
\pgfsetbuttcap%
\pgfsetroundjoin%
\definecolor{currentfill}{rgb}{0.000000,0.000000,0.000000}%
\pgfsetfillcolor{currentfill}%
\pgfsetlinewidth{0.501875pt}%
\definecolor{currentstroke}{rgb}{0.000000,0.000000,0.000000}%
\pgfsetstrokecolor{currentstroke}%
\pgfsetdash{}{0pt}%
\pgfsys@defobject{currentmarker}{\pgfqpoint{0.000000in}{0.000000in}}{\pgfqpoint{0.000000in}{0.020833in}}{%
\pgfpathmoveto{\pgfqpoint{0.000000in}{0.000000in}}%
\pgfpathlineto{\pgfqpoint{0.000000in}{0.020833in}}%
\pgfusepath{stroke,fill}%
}%
\begin{pgfscope}%
\pgfsys@transformshift{4.723429in}{0.586309in}%
\pgfsys@useobject{currentmarker}{}%
\end{pgfscope}%
\end{pgfscope}%
\begin{pgfscope}%
\pgfsetbuttcap%
\pgfsetroundjoin%
\definecolor{currentfill}{rgb}{0.000000,0.000000,0.000000}%
\pgfsetfillcolor{currentfill}%
\pgfsetlinewidth{0.501875pt}%
\definecolor{currentstroke}{rgb}{0.000000,0.000000,0.000000}%
\pgfsetstrokecolor{currentstroke}%
\pgfsetdash{}{0pt}%
\pgfsys@defobject{currentmarker}{\pgfqpoint{0.000000in}{-0.020833in}}{\pgfqpoint{0.000000in}{0.000000in}}{%
\pgfpathmoveto{\pgfqpoint{0.000000in}{0.000000in}}%
\pgfpathlineto{\pgfqpoint{0.000000in}{-0.020833in}}%
\pgfusepath{stroke,fill}%
}%
\begin{pgfscope}%
\pgfsys@transformshift{4.723429in}{0.893003in}%
\pgfsys@useobject{currentmarker}{}%
\end{pgfscope}%
\end{pgfscope}%
\begin{pgfscope}%
\pgfsetbuttcap%
\pgfsetroundjoin%
\definecolor{currentfill}{rgb}{0.000000,0.000000,0.000000}%
\pgfsetfillcolor{currentfill}%
\pgfsetlinewidth{0.501875pt}%
\definecolor{currentstroke}{rgb}{0.000000,0.000000,0.000000}%
\pgfsetstrokecolor{currentstroke}%
\pgfsetdash{}{0pt}%
\pgfsys@defobject{currentmarker}{\pgfqpoint{0.000000in}{0.000000in}}{\pgfqpoint{0.000000in}{0.020833in}}{%
\pgfpathmoveto{\pgfqpoint{0.000000in}{0.000000in}}%
\pgfpathlineto{\pgfqpoint{0.000000in}{0.020833in}}%
\pgfusepath{stroke,fill}%
}%
\begin{pgfscope}%
\pgfsys@transformshift{4.759127in}{0.586309in}%
\pgfsys@useobject{currentmarker}{}%
\end{pgfscope}%
\end{pgfscope}%
\begin{pgfscope}%
\pgfsetbuttcap%
\pgfsetroundjoin%
\definecolor{currentfill}{rgb}{0.000000,0.000000,0.000000}%
\pgfsetfillcolor{currentfill}%
\pgfsetlinewidth{0.501875pt}%
\definecolor{currentstroke}{rgb}{0.000000,0.000000,0.000000}%
\pgfsetstrokecolor{currentstroke}%
\pgfsetdash{}{0pt}%
\pgfsys@defobject{currentmarker}{\pgfqpoint{0.000000in}{-0.020833in}}{\pgfqpoint{0.000000in}{0.000000in}}{%
\pgfpathmoveto{\pgfqpoint{0.000000in}{0.000000in}}%
\pgfpathlineto{\pgfqpoint{0.000000in}{-0.020833in}}%
\pgfusepath{stroke,fill}%
}%
\begin{pgfscope}%
\pgfsys@transformshift{4.759127in}{0.893003in}%
\pgfsys@useobject{currentmarker}{}%
\end{pgfscope}%
\end{pgfscope}%
\begin{pgfscope}%
\pgfsetbuttcap%
\pgfsetroundjoin%
\definecolor{currentfill}{rgb}{0.000000,0.000000,0.000000}%
\pgfsetfillcolor{currentfill}%
\pgfsetlinewidth{0.501875pt}%
\definecolor{currentstroke}{rgb}{0.000000,0.000000,0.000000}%
\pgfsetstrokecolor{currentstroke}%
\pgfsetdash{}{0pt}%
\pgfsys@defobject{currentmarker}{\pgfqpoint{0.000000in}{0.000000in}}{\pgfqpoint{0.000000in}{0.020833in}}{%
\pgfpathmoveto{\pgfqpoint{0.000000in}{0.000000in}}%
\pgfpathlineto{\pgfqpoint{0.000000in}{0.020833in}}%
\pgfusepath{stroke,fill}%
}%
\begin{pgfscope}%
\pgfsys@transformshift{4.794824in}{0.586309in}%
\pgfsys@useobject{currentmarker}{}%
\end{pgfscope}%
\end{pgfscope}%
\begin{pgfscope}%
\pgfsetbuttcap%
\pgfsetroundjoin%
\definecolor{currentfill}{rgb}{0.000000,0.000000,0.000000}%
\pgfsetfillcolor{currentfill}%
\pgfsetlinewidth{0.501875pt}%
\definecolor{currentstroke}{rgb}{0.000000,0.000000,0.000000}%
\pgfsetstrokecolor{currentstroke}%
\pgfsetdash{}{0pt}%
\pgfsys@defobject{currentmarker}{\pgfqpoint{0.000000in}{-0.020833in}}{\pgfqpoint{0.000000in}{0.000000in}}{%
\pgfpathmoveto{\pgfqpoint{0.000000in}{0.000000in}}%
\pgfpathlineto{\pgfqpoint{0.000000in}{-0.020833in}}%
\pgfusepath{stroke,fill}%
}%
\begin{pgfscope}%
\pgfsys@transformshift{4.794824in}{0.893003in}%
\pgfsys@useobject{currentmarker}{}%
\end{pgfscope}%
\end{pgfscope}%
\begin{pgfscope}%
\pgfsetbuttcap%
\pgfsetroundjoin%
\definecolor{currentfill}{rgb}{0.000000,0.000000,0.000000}%
\pgfsetfillcolor{currentfill}%
\pgfsetlinewidth{0.501875pt}%
\definecolor{currentstroke}{rgb}{0.000000,0.000000,0.000000}%
\pgfsetstrokecolor{currentstroke}%
\pgfsetdash{}{0pt}%
\pgfsys@defobject{currentmarker}{\pgfqpoint{0.000000in}{0.000000in}}{\pgfqpoint{0.000000in}{0.020833in}}{%
\pgfpathmoveto{\pgfqpoint{0.000000in}{0.000000in}}%
\pgfpathlineto{\pgfqpoint{0.000000in}{0.020833in}}%
\pgfusepath{stroke,fill}%
}%
\begin{pgfscope}%
\pgfsys@transformshift{4.830521in}{0.586309in}%
\pgfsys@useobject{currentmarker}{}%
\end{pgfscope}%
\end{pgfscope}%
\begin{pgfscope}%
\pgfsetbuttcap%
\pgfsetroundjoin%
\definecolor{currentfill}{rgb}{0.000000,0.000000,0.000000}%
\pgfsetfillcolor{currentfill}%
\pgfsetlinewidth{0.501875pt}%
\definecolor{currentstroke}{rgb}{0.000000,0.000000,0.000000}%
\pgfsetstrokecolor{currentstroke}%
\pgfsetdash{}{0pt}%
\pgfsys@defobject{currentmarker}{\pgfqpoint{0.000000in}{-0.020833in}}{\pgfqpoint{0.000000in}{0.000000in}}{%
\pgfpathmoveto{\pgfqpoint{0.000000in}{0.000000in}}%
\pgfpathlineto{\pgfqpoint{0.000000in}{-0.020833in}}%
\pgfusepath{stroke,fill}%
}%
\begin{pgfscope}%
\pgfsys@transformshift{4.830521in}{0.893003in}%
\pgfsys@useobject{currentmarker}{}%
\end{pgfscope}%
\end{pgfscope}%
\begin{pgfscope}%
\pgfsetbuttcap%
\pgfsetroundjoin%
\definecolor{currentfill}{rgb}{0.000000,0.000000,0.000000}%
\pgfsetfillcolor{currentfill}%
\pgfsetlinewidth{0.501875pt}%
\definecolor{currentstroke}{rgb}{0.000000,0.000000,0.000000}%
\pgfsetstrokecolor{currentstroke}%
\pgfsetdash{}{0pt}%
\pgfsys@defobject{currentmarker}{\pgfqpoint{0.000000in}{0.000000in}}{\pgfqpoint{0.000000in}{0.020833in}}{%
\pgfpathmoveto{\pgfqpoint{0.000000in}{0.000000in}}%
\pgfpathlineto{\pgfqpoint{0.000000in}{0.020833in}}%
\pgfusepath{stroke,fill}%
}%
\begin{pgfscope}%
\pgfsys@transformshift{4.866219in}{0.586309in}%
\pgfsys@useobject{currentmarker}{}%
\end{pgfscope}%
\end{pgfscope}%
\begin{pgfscope}%
\pgfsetbuttcap%
\pgfsetroundjoin%
\definecolor{currentfill}{rgb}{0.000000,0.000000,0.000000}%
\pgfsetfillcolor{currentfill}%
\pgfsetlinewidth{0.501875pt}%
\definecolor{currentstroke}{rgb}{0.000000,0.000000,0.000000}%
\pgfsetstrokecolor{currentstroke}%
\pgfsetdash{}{0pt}%
\pgfsys@defobject{currentmarker}{\pgfqpoint{0.000000in}{-0.020833in}}{\pgfqpoint{0.000000in}{0.000000in}}{%
\pgfpathmoveto{\pgfqpoint{0.000000in}{0.000000in}}%
\pgfpathlineto{\pgfqpoint{0.000000in}{-0.020833in}}%
\pgfusepath{stroke,fill}%
}%
\begin{pgfscope}%
\pgfsys@transformshift{4.866219in}{0.893003in}%
\pgfsys@useobject{currentmarker}{}%
\end{pgfscope}%
\end{pgfscope}%
\begin{pgfscope}%
\pgfsetbuttcap%
\pgfsetroundjoin%
\definecolor{currentfill}{rgb}{0.000000,0.000000,0.000000}%
\pgfsetfillcolor{currentfill}%
\pgfsetlinewidth{0.501875pt}%
\definecolor{currentstroke}{rgb}{0.000000,0.000000,0.000000}%
\pgfsetstrokecolor{currentstroke}%
\pgfsetdash{}{0pt}%
\pgfsys@defobject{currentmarker}{\pgfqpoint{0.000000in}{0.000000in}}{\pgfqpoint{0.000000in}{0.020833in}}{%
\pgfpathmoveto{\pgfqpoint{0.000000in}{0.000000in}}%
\pgfpathlineto{\pgfqpoint{0.000000in}{0.020833in}}%
\pgfusepath{stroke,fill}%
}%
\begin{pgfscope}%
\pgfsys@transformshift{4.901916in}{0.586309in}%
\pgfsys@useobject{currentmarker}{}%
\end{pgfscope}%
\end{pgfscope}%
\begin{pgfscope}%
\pgfsetbuttcap%
\pgfsetroundjoin%
\definecolor{currentfill}{rgb}{0.000000,0.000000,0.000000}%
\pgfsetfillcolor{currentfill}%
\pgfsetlinewidth{0.501875pt}%
\definecolor{currentstroke}{rgb}{0.000000,0.000000,0.000000}%
\pgfsetstrokecolor{currentstroke}%
\pgfsetdash{}{0pt}%
\pgfsys@defobject{currentmarker}{\pgfqpoint{0.000000in}{-0.020833in}}{\pgfqpoint{0.000000in}{0.000000in}}{%
\pgfpathmoveto{\pgfqpoint{0.000000in}{0.000000in}}%
\pgfpathlineto{\pgfqpoint{0.000000in}{-0.020833in}}%
\pgfusepath{stroke,fill}%
}%
\begin{pgfscope}%
\pgfsys@transformshift{4.901916in}{0.893003in}%
\pgfsys@useobject{currentmarker}{}%
\end{pgfscope}%
\end{pgfscope}%
\begin{pgfscope}%
\pgfsetbuttcap%
\pgfsetroundjoin%
\definecolor{currentfill}{rgb}{0.000000,0.000000,0.000000}%
\pgfsetfillcolor{currentfill}%
\pgfsetlinewidth{0.501875pt}%
\definecolor{currentstroke}{rgb}{0.000000,0.000000,0.000000}%
\pgfsetstrokecolor{currentstroke}%
\pgfsetdash{}{0pt}%
\pgfsys@defobject{currentmarker}{\pgfqpoint{0.000000in}{0.000000in}}{\pgfqpoint{0.000000in}{0.020833in}}{%
\pgfpathmoveto{\pgfqpoint{0.000000in}{0.000000in}}%
\pgfpathlineto{\pgfqpoint{0.000000in}{0.020833in}}%
\pgfusepath{stroke,fill}%
}%
\begin{pgfscope}%
\pgfsys@transformshift{4.937613in}{0.586309in}%
\pgfsys@useobject{currentmarker}{}%
\end{pgfscope}%
\end{pgfscope}%
\begin{pgfscope}%
\pgfsetbuttcap%
\pgfsetroundjoin%
\definecolor{currentfill}{rgb}{0.000000,0.000000,0.000000}%
\pgfsetfillcolor{currentfill}%
\pgfsetlinewidth{0.501875pt}%
\definecolor{currentstroke}{rgb}{0.000000,0.000000,0.000000}%
\pgfsetstrokecolor{currentstroke}%
\pgfsetdash{}{0pt}%
\pgfsys@defobject{currentmarker}{\pgfqpoint{0.000000in}{-0.020833in}}{\pgfqpoint{0.000000in}{0.000000in}}{%
\pgfpathmoveto{\pgfqpoint{0.000000in}{0.000000in}}%
\pgfpathlineto{\pgfqpoint{0.000000in}{-0.020833in}}%
\pgfusepath{stroke,fill}%
}%
\begin{pgfscope}%
\pgfsys@transformshift{4.937613in}{0.893003in}%
\pgfsys@useobject{currentmarker}{}%
\end{pgfscope}%
\end{pgfscope}%
\begin{pgfscope}%
\pgfsetbuttcap%
\pgfsetroundjoin%
\definecolor{currentfill}{rgb}{0.000000,0.000000,0.000000}%
\pgfsetfillcolor{currentfill}%
\pgfsetlinewidth{0.501875pt}%
\definecolor{currentstroke}{rgb}{0.000000,0.000000,0.000000}%
\pgfsetstrokecolor{currentstroke}%
\pgfsetdash{}{0pt}%
\pgfsys@defobject{currentmarker}{\pgfqpoint{0.000000in}{0.000000in}}{\pgfqpoint{0.000000in}{0.020833in}}{%
\pgfpathmoveto{\pgfqpoint{0.000000in}{0.000000in}}%
\pgfpathlineto{\pgfqpoint{0.000000in}{0.020833in}}%
\pgfusepath{stroke,fill}%
}%
\begin{pgfscope}%
\pgfsys@transformshift{5.009008in}{0.586309in}%
\pgfsys@useobject{currentmarker}{}%
\end{pgfscope}%
\end{pgfscope}%
\begin{pgfscope}%
\pgfsetbuttcap%
\pgfsetroundjoin%
\definecolor{currentfill}{rgb}{0.000000,0.000000,0.000000}%
\pgfsetfillcolor{currentfill}%
\pgfsetlinewidth{0.501875pt}%
\definecolor{currentstroke}{rgb}{0.000000,0.000000,0.000000}%
\pgfsetstrokecolor{currentstroke}%
\pgfsetdash{}{0pt}%
\pgfsys@defobject{currentmarker}{\pgfqpoint{0.000000in}{-0.020833in}}{\pgfqpoint{0.000000in}{0.000000in}}{%
\pgfpathmoveto{\pgfqpoint{0.000000in}{0.000000in}}%
\pgfpathlineto{\pgfqpoint{0.000000in}{-0.020833in}}%
\pgfusepath{stroke,fill}%
}%
\begin{pgfscope}%
\pgfsys@transformshift{5.009008in}{0.893003in}%
\pgfsys@useobject{currentmarker}{}%
\end{pgfscope}%
\end{pgfscope}%
\begin{pgfscope}%
\pgfsetbuttcap%
\pgfsetroundjoin%
\definecolor{currentfill}{rgb}{0.000000,0.000000,0.000000}%
\pgfsetfillcolor{currentfill}%
\pgfsetlinewidth{0.501875pt}%
\definecolor{currentstroke}{rgb}{0.000000,0.000000,0.000000}%
\pgfsetstrokecolor{currentstroke}%
\pgfsetdash{}{0pt}%
\pgfsys@defobject{currentmarker}{\pgfqpoint{0.000000in}{0.000000in}}{\pgfqpoint{0.000000in}{0.020833in}}{%
\pgfpathmoveto{\pgfqpoint{0.000000in}{0.000000in}}%
\pgfpathlineto{\pgfqpoint{0.000000in}{0.020833in}}%
\pgfusepath{stroke,fill}%
}%
\begin{pgfscope}%
\pgfsys@transformshift{5.044705in}{0.586309in}%
\pgfsys@useobject{currentmarker}{}%
\end{pgfscope}%
\end{pgfscope}%
\begin{pgfscope}%
\pgfsetbuttcap%
\pgfsetroundjoin%
\definecolor{currentfill}{rgb}{0.000000,0.000000,0.000000}%
\pgfsetfillcolor{currentfill}%
\pgfsetlinewidth{0.501875pt}%
\definecolor{currentstroke}{rgb}{0.000000,0.000000,0.000000}%
\pgfsetstrokecolor{currentstroke}%
\pgfsetdash{}{0pt}%
\pgfsys@defobject{currentmarker}{\pgfqpoint{0.000000in}{-0.020833in}}{\pgfqpoint{0.000000in}{0.000000in}}{%
\pgfpathmoveto{\pgfqpoint{0.000000in}{0.000000in}}%
\pgfpathlineto{\pgfqpoint{0.000000in}{-0.020833in}}%
\pgfusepath{stroke,fill}%
}%
\begin{pgfscope}%
\pgfsys@transformshift{5.044705in}{0.893003in}%
\pgfsys@useobject{currentmarker}{}%
\end{pgfscope}%
\end{pgfscope}%
\begin{pgfscope}%
\pgfsetbuttcap%
\pgfsetroundjoin%
\definecolor{currentfill}{rgb}{0.000000,0.000000,0.000000}%
\pgfsetfillcolor{currentfill}%
\pgfsetlinewidth{0.501875pt}%
\definecolor{currentstroke}{rgb}{0.000000,0.000000,0.000000}%
\pgfsetstrokecolor{currentstroke}%
\pgfsetdash{}{0pt}%
\pgfsys@defobject{currentmarker}{\pgfqpoint{0.000000in}{0.000000in}}{\pgfqpoint{0.000000in}{0.020833in}}{%
\pgfpathmoveto{\pgfqpoint{0.000000in}{0.000000in}}%
\pgfpathlineto{\pgfqpoint{0.000000in}{0.020833in}}%
\pgfusepath{stroke,fill}%
}%
\begin{pgfscope}%
\pgfsys@transformshift{5.080402in}{0.586309in}%
\pgfsys@useobject{currentmarker}{}%
\end{pgfscope}%
\end{pgfscope}%
\begin{pgfscope}%
\pgfsetbuttcap%
\pgfsetroundjoin%
\definecolor{currentfill}{rgb}{0.000000,0.000000,0.000000}%
\pgfsetfillcolor{currentfill}%
\pgfsetlinewidth{0.501875pt}%
\definecolor{currentstroke}{rgb}{0.000000,0.000000,0.000000}%
\pgfsetstrokecolor{currentstroke}%
\pgfsetdash{}{0pt}%
\pgfsys@defobject{currentmarker}{\pgfqpoint{0.000000in}{-0.020833in}}{\pgfqpoint{0.000000in}{0.000000in}}{%
\pgfpathmoveto{\pgfqpoint{0.000000in}{0.000000in}}%
\pgfpathlineto{\pgfqpoint{0.000000in}{-0.020833in}}%
\pgfusepath{stroke,fill}%
}%
\begin{pgfscope}%
\pgfsys@transformshift{5.080402in}{0.893003in}%
\pgfsys@useobject{currentmarker}{}%
\end{pgfscope}%
\end{pgfscope}%
\begin{pgfscope}%
\pgfsetbuttcap%
\pgfsetroundjoin%
\definecolor{currentfill}{rgb}{0.000000,0.000000,0.000000}%
\pgfsetfillcolor{currentfill}%
\pgfsetlinewidth{0.501875pt}%
\definecolor{currentstroke}{rgb}{0.000000,0.000000,0.000000}%
\pgfsetstrokecolor{currentstroke}%
\pgfsetdash{}{0pt}%
\pgfsys@defobject{currentmarker}{\pgfqpoint{0.000000in}{0.000000in}}{\pgfqpoint{0.000000in}{0.020833in}}{%
\pgfpathmoveto{\pgfqpoint{0.000000in}{0.000000in}}%
\pgfpathlineto{\pgfqpoint{0.000000in}{0.020833in}}%
\pgfusepath{stroke,fill}%
}%
\begin{pgfscope}%
\pgfsys@transformshift{5.116100in}{0.586309in}%
\pgfsys@useobject{currentmarker}{}%
\end{pgfscope}%
\end{pgfscope}%
\begin{pgfscope}%
\pgfsetbuttcap%
\pgfsetroundjoin%
\definecolor{currentfill}{rgb}{0.000000,0.000000,0.000000}%
\pgfsetfillcolor{currentfill}%
\pgfsetlinewidth{0.501875pt}%
\definecolor{currentstroke}{rgb}{0.000000,0.000000,0.000000}%
\pgfsetstrokecolor{currentstroke}%
\pgfsetdash{}{0pt}%
\pgfsys@defobject{currentmarker}{\pgfqpoint{0.000000in}{-0.020833in}}{\pgfqpoint{0.000000in}{0.000000in}}{%
\pgfpathmoveto{\pgfqpoint{0.000000in}{0.000000in}}%
\pgfpathlineto{\pgfqpoint{0.000000in}{-0.020833in}}%
\pgfusepath{stroke,fill}%
}%
\begin{pgfscope}%
\pgfsys@transformshift{5.116100in}{0.893003in}%
\pgfsys@useobject{currentmarker}{}%
\end{pgfscope}%
\end{pgfscope}%
\begin{pgfscope}%
\pgfsetbuttcap%
\pgfsetroundjoin%
\definecolor{currentfill}{rgb}{0.000000,0.000000,0.000000}%
\pgfsetfillcolor{currentfill}%
\pgfsetlinewidth{0.501875pt}%
\definecolor{currentstroke}{rgb}{0.000000,0.000000,0.000000}%
\pgfsetstrokecolor{currentstroke}%
\pgfsetdash{}{0pt}%
\pgfsys@defobject{currentmarker}{\pgfqpoint{0.000000in}{0.000000in}}{\pgfqpoint{0.000000in}{0.020833in}}{%
\pgfpathmoveto{\pgfqpoint{0.000000in}{0.000000in}}%
\pgfpathlineto{\pgfqpoint{0.000000in}{0.020833in}}%
\pgfusepath{stroke,fill}%
}%
\begin{pgfscope}%
\pgfsys@transformshift{5.151797in}{0.586309in}%
\pgfsys@useobject{currentmarker}{}%
\end{pgfscope}%
\end{pgfscope}%
\begin{pgfscope}%
\pgfsetbuttcap%
\pgfsetroundjoin%
\definecolor{currentfill}{rgb}{0.000000,0.000000,0.000000}%
\pgfsetfillcolor{currentfill}%
\pgfsetlinewidth{0.501875pt}%
\definecolor{currentstroke}{rgb}{0.000000,0.000000,0.000000}%
\pgfsetstrokecolor{currentstroke}%
\pgfsetdash{}{0pt}%
\pgfsys@defobject{currentmarker}{\pgfqpoint{0.000000in}{-0.020833in}}{\pgfqpoint{0.000000in}{0.000000in}}{%
\pgfpathmoveto{\pgfqpoint{0.000000in}{0.000000in}}%
\pgfpathlineto{\pgfqpoint{0.000000in}{-0.020833in}}%
\pgfusepath{stroke,fill}%
}%
\begin{pgfscope}%
\pgfsys@transformshift{5.151797in}{0.893003in}%
\pgfsys@useobject{currentmarker}{}%
\end{pgfscope}%
\end{pgfscope}%
\begin{pgfscope}%
\pgfsetbuttcap%
\pgfsetroundjoin%
\definecolor{currentfill}{rgb}{0.000000,0.000000,0.000000}%
\pgfsetfillcolor{currentfill}%
\pgfsetlinewidth{0.501875pt}%
\definecolor{currentstroke}{rgb}{0.000000,0.000000,0.000000}%
\pgfsetstrokecolor{currentstroke}%
\pgfsetdash{}{0pt}%
\pgfsys@defobject{currentmarker}{\pgfqpoint{0.000000in}{0.000000in}}{\pgfqpoint{0.000000in}{0.020833in}}{%
\pgfpathmoveto{\pgfqpoint{0.000000in}{0.000000in}}%
\pgfpathlineto{\pgfqpoint{0.000000in}{0.020833in}}%
\pgfusepath{stroke,fill}%
}%
\begin{pgfscope}%
\pgfsys@transformshift{5.187494in}{0.586309in}%
\pgfsys@useobject{currentmarker}{}%
\end{pgfscope}%
\end{pgfscope}%
\begin{pgfscope}%
\pgfsetbuttcap%
\pgfsetroundjoin%
\definecolor{currentfill}{rgb}{0.000000,0.000000,0.000000}%
\pgfsetfillcolor{currentfill}%
\pgfsetlinewidth{0.501875pt}%
\definecolor{currentstroke}{rgb}{0.000000,0.000000,0.000000}%
\pgfsetstrokecolor{currentstroke}%
\pgfsetdash{}{0pt}%
\pgfsys@defobject{currentmarker}{\pgfqpoint{0.000000in}{-0.020833in}}{\pgfqpoint{0.000000in}{0.000000in}}{%
\pgfpathmoveto{\pgfqpoint{0.000000in}{0.000000in}}%
\pgfpathlineto{\pgfqpoint{0.000000in}{-0.020833in}}%
\pgfusepath{stroke,fill}%
}%
\begin{pgfscope}%
\pgfsys@transformshift{5.187494in}{0.893003in}%
\pgfsys@useobject{currentmarker}{}%
\end{pgfscope}%
\end{pgfscope}%
\begin{pgfscope}%
\pgfsetbuttcap%
\pgfsetroundjoin%
\definecolor{currentfill}{rgb}{0.000000,0.000000,0.000000}%
\pgfsetfillcolor{currentfill}%
\pgfsetlinewidth{0.501875pt}%
\definecolor{currentstroke}{rgb}{0.000000,0.000000,0.000000}%
\pgfsetstrokecolor{currentstroke}%
\pgfsetdash{}{0pt}%
\pgfsys@defobject{currentmarker}{\pgfqpoint{0.000000in}{0.000000in}}{\pgfqpoint{0.000000in}{0.020833in}}{%
\pgfpathmoveto{\pgfqpoint{0.000000in}{0.000000in}}%
\pgfpathlineto{\pgfqpoint{0.000000in}{0.020833in}}%
\pgfusepath{stroke,fill}%
}%
\begin{pgfscope}%
\pgfsys@transformshift{5.223192in}{0.586309in}%
\pgfsys@useobject{currentmarker}{}%
\end{pgfscope}%
\end{pgfscope}%
\begin{pgfscope}%
\pgfsetbuttcap%
\pgfsetroundjoin%
\definecolor{currentfill}{rgb}{0.000000,0.000000,0.000000}%
\pgfsetfillcolor{currentfill}%
\pgfsetlinewidth{0.501875pt}%
\definecolor{currentstroke}{rgb}{0.000000,0.000000,0.000000}%
\pgfsetstrokecolor{currentstroke}%
\pgfsetdash{}{0pt}%
\pgfsys@defobject{currentmarker}{\pgfqpoint{0.000000in}{-0.020833in}}{\pgfqpoint{0.000000in}{0.000000in}}{%
\pgfpathmoveto{\pgfqpoint{0.000000in}{0.000000in}}%
\pgfpathlineto{\pgfqpoint{0.000000in}{-0.020833in}}%
\pgfusepath{stroke,fill}%
}%
\begin{pgfscope}%
\pgfsys@transformshift{5.223192in}{0.893003in}%
\pgfsys@useobject{currentmarker}{}%
\end{pgfscope}%
\end{pgfscope}%
\begin{pgfscope}%
\pgfsetbuttcap%
\pgfsetroundjoin%
\definecolor{currentfill}{rgb}{0.000000,0.000000,0.000000}%
\pgfsetfillcolor{currentfill}%
\pgfsetlinewidth{0.501875pt}%
\definecolor{currentstroke}{rgb}{0.000000,0.000000,0.000000}%
\pgfsetstrokecolor{currentstroke}%
\pgfsetdash{}{0pt}%
\pgfsys@defobject{currentmarker}{\pgfqpoint{0.000000in}{0.000000in}}{\pgfqpoint{0.000000in}{0.020833in}}{%
\pgfpathmoveto{\pgfqpoint{0.000000in}{0.000000in}}%
\pgfpathlineto{\pgfqpoint{0.000000in}{0.020833in}}%
\pgfusepath{stroke,fill}%
}%
\begin{pgfscope}%
\pgfsys@transformshift{5.258889in}{0.586309in}%
\pgfsys@useobject{currentmarker}{}%
\end{pgfscope}%
\end{pgfscope}%
\begin{pgfscope}%
\pgfsetbuttcap%
\pgfsetroundjoin%
\definecolor{currentfill}{rgb}{0.000000,0.000000,0.000000}%
\pgfsetfillcolor{currentfill}%
\pgfsetlinewidth{0.501875pt}%
\definecolor{currentstroke}{rgb}{0.000000,0.000000,0.000000}%
\pgfsetstrokecolor{currentstroke}%
\pgfsetdash{}{0pt}%
\pgfsys@defobject{currentmarker}{\pgfqpoint{0.000000in}{-0.020833in}}{\pgfqpoint{0.000000in}{0.000000in}}{%
\pgfpathmoveto{\pgfqpoint{0.000000in}{0.000000in}}%
\pgfpathlineto{\pgfqpoint{0.000000in}{-0.020833in}}%
\pgfusepath{stroke,fill}%
}%
\begin{pgfscope}%
\pgfsys@transformshift{5.258889in}{0.893003in}%
\pgfsys@useobject{currentmarker}{}%
\end{pgfscope}%
\end{pgfscope}%
\begin{pgfscope}%
\pgfsetbuttcap%
\pgfsetroundjoin%
\definecolor{currentfill}{rgb}{0.000000,0.000000,0.000000}%
\pgfsetfillcolor{currentfill}%
\pgfsetlinewidth{0.501875pt}%
\definecolor{currentstroke}{rgb}{0.000000,0.000000,0.000000}%
\pgfsetstrokecolor{currentstroke}%
\pgfsetdash{}{0pt}%
\pgfsys@defobject{currentmarker}{\pgfqpoint{0.000000in}{0.000000in}}{\pgfqpoint{0.000000in}{0.020833in}}{%
\pgfpathmoveto{\pgfqpoint{0.000000in}{0.000000in}}%
\pgfpathlineto{\pgfqpoint{0.000000in}{0.020833in}}%
\pgfusepath{stroke,fill}%
}%
\begin{pgfscope}%
\pgfsys@transformshift{5.294586in}{0.586309in}%
\pgfsys@useobject{currentmarker}{}%
\end{pgfscope}%
\end{pgfscope}%
\begin{pgfscope}%
\pgfsetbuttcap%
\pgfsetroundjoin%
\definecolor{currentfill}{rgb}{0.000000,0.000000,0.000000}%
\pgfsetfillcolor{currentfill}%
\pgfsetlinewidth{0.501875pt}%
\definecolor{currentstroke}{rgb}{0.000000,0.000000,0.000000}%
\pgfsetstrokecolor{currentstroke}%
\pgfsetdash{}{0pt}%
\pgfsys@defobject{currentmarker}{\pgfqpoint{0.000000in}{-0.020833in}}{\pgfqpoint{0.000000in}{0.000000in}}{%
\pgfpathmoveto{\pgfqpoint{0.000000in}{0.000000in}}%
\pgfpathlineto{\pgfqpoint{0.000000in}{-0.020833in}}%
\pgfusepath{stroke,fill}%
}%
\begin{pgfscope}%
\pgfsys@transformshift{5.294586in}{0.893003in}%
\pgfsys@useobject{currentmarker}{}%
\end{pgfscope}%
\end{pgfscope}%
\begin{pgfscope}%
\pgfsetbuttcap%
\pgfsetroundjoin%
\definecolor{currentfill}{rgb}{0.000000,0.000000,0.000000}%
\pgfsetfillcolor{currentfill}%
\pgfsetlinewidth{0.501875pt}%
\definecolor{currentstroke}{rgb}{0.000000,0.000000,0.000000}%
\pgfsetstrokecolor{currentstroke}%
\pgfsetdash{}{0pt}%
\pgfsys@defobject{currentmarker}{\pgfqpoint{0.000000in}{0.000000in}}{\pgfqpoint{0.000000in}{0.020833in}}{%
\pgfpathmoveto{\pgfqpoint{0.000000in}{0.000000in}}%
\pgfpathlineto{\pgfqpoint{0.000000in}{0.020833in}}%
\pgfusepath{stroke,fill}%
}%
\begin{pgfscope}%
\pgfsys@transformshift{5.330284in}{0.586309in}%
\pgfsys@useobject{currentmarker}{}%
\end{pgfscope}%
\end{pgfscope}%
\begin{pgfscope}%
\pgfsetbuttcap%
\pgfsetroundjoin%
\definecolor{currentfill}{rgb}{0.000000,0.000000,0.000000}%
\pgfsetfillcolor{currentfill}%
\pgfsetlinewidth{0.501875pt}%
\definecolor{currentstroke}{rgb}{0.000000,0.000000,0.000000}%
\pgfsetstrokecolor{currentstroke}%
\pgfsetdash{}{0pt}%
\pgfsys@defobject{currentmarker}{\pgfqpoint{0.000000in}{-0.020833in}}{\pgfqpoint{0.000000in}{0.000000in}}{%
\pgfpathmoveto{\pgfqpoint{0.000000in}{0.000000in}}%
\pgfpathlineto{\pgfqpoint{0.000000in}{-0.020833in}}%
\pgfusepath{stroke,fill}%
}%
\begin{pgfscope}%
\pgfsys@transformshift{5.330284in}{0.893003in}%
\pgfsys@useobject{currentmarker}{}%
\end{pgfscope}%
\end{pgfscope}%
\begin{pgfscope}%
\pgfsetbuttcap%
\pgfsetroundjoin%
\definecolor{currentfill}{rgb}{0.000000,0.000000,0.000000}%
\pgfsetfillcolor{currentfill}%
\pgfsetlinewidth{0.501875pt}%
\definecolor{currentstroke}{rgb}{0.000000,0.000000,0.000000}%
\pgfsetstrokecolor{currentstroke}%
\pgfsetdash{}{0pt}%
\pgfsys@defobject{currentmarker}{\pgfqpoint{0.000000in}{0.000000in}}{\pgfqpoint{0.000000in}{0.020833in}}{%
\pgfpathmoveto{\pgfqpoint{0.000000in}{0.000000in}}%
\pgfpathlineto{\pgfqpoint{0.000000in}{0.020833in}}%
\pgfusepath{stroke,fill}%
}%
\begin{pgfscope}%
\pgfsys@transformshift{5.365981in}{0.586309in}%
\pgfsys@useobject{currentmarker}{}%
\end{pgfscope}%
\end{pgfscope}%
\begin{pgfscope}%
\pgfsetbuttcap%
\pgfsetroundjoin%
\definecolor{currentfill}{rgb}{0.000000,0.000000,0.000000}%
\pgfsetfillcolor{currentfill}%
\pgfsetlinewidth{0.501875pt}%
\definecolor{currentstroke}{rgb}{0.000000,0.000000,0.000000}%
\pgfsetstrokecolor{currentstroke}%
\pgfsetdash{}{0pt}%
\pgfsys@defobject{currentmarker}{\pgfqpoint{0.000000in}{-0.020833in}}{\pgfqpoint{0.000000in}{0.000000in}}{%
\pgfpathmoveto{\pgfqpoint{0.000000in}{0.000000in}}%
\pgfpathlineto{\pgfqpoint{0.000000in}{-0.020833in}}%
\pgfusepath{stroke,fill}%
}%
\begin{pgfscope}%
\pgfsys@transformshift{5.365981in}{0.893003in}%
\pgfsys@useobject{currentmarker}{}%
\end{pgfscope}%
\end{pgfscope}%
\begin{pgfscope}%
\pgfsetbuttcap%
\pgfsetroundjoin%
\definecolor{currentfill}{rgb}{0.000000,0.000000,0.000000}%
\pgfsetfillcolor{currentfill}%
\pgfsetlinewidth{0.501875pt}%
\definecolor{currentstroke}{rgb}{0.000000,0.000000,0.000000}%
\pgfsetstrokecolor{currentstroke}%
\pgfsetdash{}{0pt}%
\pgfsys@defobject{currentmarker}{\pgfqpoint{0.000000in}{0.000000in}}{\pgfqpoint{0.000000in}{0.020833in}}{%
\pgfpathmoveto{\pgfqpoint{0.000000in}{0.000000in}}%
\pgfpathlineto{\pgfqpoint{0.000000in}{0.020833in}}%
\pgfusepath{stroke,fill}%
}%
\begin{pgfscope}%
\pgfsys@transformshift{5.437376in}{0.586309in}%
\pgfsys@useobject{currentmarker}{}%
\end{pgfscope}%
\end{pgfscope}%
\begin{pgfscope}%
\pgfsetbuttcap%
\pgfsetroundjoin%
\definecolor{currentfill}{rgb}{0.000000,0.000000,0.000000}%
\pgfsetfillcolor{currentfill}%
\pgfsetlinewidth{0.501875pt}%
\definecolor{currentstroke}{rgb}{0.000000,0.000000,0.000000}%
\pgfsetstrokecolor{currentstroke}%
\pgfsetdash{}{0pt}%
\pgfsys@defobject{currentmarker}{\pgfqpoint{0.000000in}{-0.020833in}}{\pgfqpoint{0.000000in}{0.000000in}}{%
\pgfpathmoveto{\pgfqpoint{0.000000in}{0.000000in}}%
\pgfpathlineto{\pgfqpoint{0.000000in}{-0.020833in}}%
\pgfusepath{stroke,fill}%
}%
\begin{pgfscope}%
\pgfsys@transformshift{5.437376in}{0.893003in}%
\pgfsys@useobject{currentmarker}{}%
\end{pgfscope}%
\end{pgfscope}%
\begin{pgfscope}%
\pgfsetbuttcap%
\pgfsetroundjoin%
\definecolor{currentfill}{rgb}{0.000000,0.000000,0.000000}%
\pgfsetfillcolor{currentfill}%
\pgfsetlinewidth{0.501875pt}%
\definecolor{currentstroke}{rgb}{0.000000,0.000000,0.000000}%
\pgfsetstrokecolor{currentstroke}%
\pgfsetdash{}{0pt}%
\pgfsys@defobject{currentmarker}{\pgfqpoint{0.000000in}{0.000000in}}{\pgfqpoint{0.000000in}{0.020833in}}{%
\pgfpathmoveto{\pgfqpoint{0.000000in}{0.000000in}}%
\pgfpathlineto{\pgfqpoint{0.000000in}{0.020833in}}%
\pgfusepath{stroke,fill}%
}%
\begin{pgfscope}%
\pgfsys@transformshift{5.473073in}{0.586309in}%
\pgfsys@useobject{currentmarker}{}%
\end{pgfscope}%
\end{pgfscope}%
\begin{pgfscope}%
\pgfsetbuttcap%
\pgfsetroundjoin%
\definecolor{currentfill}{rgb}{0.000000,0.000000,0.000000}%
\pgfsetfillcolor{currentfill}%
\pgfsetlinewidth{0.501875pt}%
\definecolor{currentstroke}{rgb}{0.000000,0.000000,0.000000}%
\pgfsetstrokecolor{currentstroke}%
\pgfsetdash{}{0pt}%
\pgfsys@defobject{currentmarker}{\pgfqpoint{0.000000in}{-0.020833in}}{\pgfqpoint{0.000000in}{0.000000in}}{%
\pgfpathmoveto{\pgfqpoint{0.000000in}{0.000000in}}%
\pgfpathlineto{\pgfqpoint{0.000000in}{-0.020833in}}%
\pgfusepath{stroke,fill}%
}%
\begin{pgfscope}%
\pgfsys@transformshift{5.473073in}{0.893003in}%
\pgfsys@useobject{currentmarker}{}%
\end{pgfscope}%
\end{pgfscope}%
\begin{pgfscope}%
\pgfsetbuttcap%
\pgfsetroundjoin%
\definecolor{currentfill}{rgb}{0.000000,0.000000,0.000000}%
\pgfsetfillcolor{currentfill}%
\pgfsetlinewidth{0.501875pt}%
\definecolor{currentstroke}{rgb}{0.000000,0.000000,0.000000}%
\pgfsetstrokecolor{currentstroke}%
\pgfsetdash{}{0pt}%
\pgfsys@defobject{currentmarker}{\pgfqpoint{0.000000in}{0.000000in}}{\pgfqpoint{0.000000in}{0.020833in}}{%
\pgfpathmoveto{\pgfqpoint{0.000000in}{0.000000in}}%
\pgfpathlineto{\pgfqpoint{0.000000in}{0.020833in}}%
\pgfusepath{stroke,fill}%
}%
\begin{pgfscope}%
\pgfsys@transformshift{5.508770in}{0.586309in}%
\pgfsys@useobject{currentmarker}{}%
\end{pgfscope}%
\end{pgfscope}%
\begin{pgfscope}%
\pgfsetbuttcap%
\pgfsetroundjoin%
\definecolor{currentfill}{rgb}{0.000000,0.000000,0.000000}%
\pgfsetfillcolor{currentfill}%
\pgfsetlinewidth{0.501875pt}%
\definecolor{currentstroke}{rgb}{0.000000,0.000000,0.000000}%
\pgfsetstrokecolor{currentstroke}%
\pgfsetdash{}{0pt}%
\pgfsys@defobject{currentmarker}{\pgfqpoint{0.000000in}{-0.020833in}}{\pgfqpoint{0.000000in}{0.000000in}}{%
\pgfpathmoveto{\pgfqpoint{0.000000in}{0.000000in}}%
\pgfpathlineto{\pgfqpoint{0.000000in}{-0.020833in}}%
\pgfusepath{stroke,fill}%
}%
\begin{pgfscope}%
\pgfsys@transformshift{5.508770in}{0.893003in}%
\pgfsys@useobject{currentmarker}{}%
\end{pgfscope}%
\end{pgfscope}%
\begin{pgfscope}%
\pgfsetbuttcap%
\pgfsetroundjoin%
\definecolor{currentfill}{rgb}{0.000000,0.000000,0.000000}%
\pgfsetfillcolor{currentfill}%
\pgfsetlinewidth{0.501875pt}%
\definecolor{currentstroke}{rgb}{0.000000,0.000000,0.000000}%
\pgfsetstrokecolor{currentstroke}%
\pgfsetdash{}{0pt}%
\pgfsys@defobject{currentmarker}{\pgfqpoint{0.000000in}{0.000000in}}{\pgfqpoint{0.000000in}{0.020833in}}{%
\pgfpathmoveto{\pgfqpoint{0.000000in}{0.000000in}}%
\pgfpathlineto{\pgfqpoint{0.000000in}{0.020833in}}%
\pgfusepath{stroke,fill}%
}%
\begin{pgfscope}%
\pgfsys@transformshift{5.544468in}{0.586309in}%
\pgfsys@useobject{currentmarker}{}%
\end{pgfscope}%
\end{pgfscope}%
\begin{pgfscope}%
\pgfsetbuttcap%
\pgfsetroundjoin%
\definecolor{currentfill}{rgb}{0.000000,0.000000,0.000000}%
\pgfsetfillcolor{currentfill}%
\pgfsetlinewidth{0.501875pt}%
\definecolor{currentstroke}{rgb}{0.000000,0.000000,0.000000}%
\pgfsetstrokecolor{currentstroke}%
\pgfsetdash{}{0pt}%
\pgfsys@defobject{currentmarker}{\pgfqpoint{0.000000in}{-0.020833in}}{\pgfqpoint{0.000000in}{0.000000in}}{%
\pgfpathmoveto{\pgfqpoint{0.000000in}{0.000000in}}%
\pgfpathlineto{\pgfqpoint{0.000000in}{-0.020833in}}%
\pgfusepath{stroke,fill}%
}%
\begin{pgfscope}%
\pgfsys@transformshift{5.544468in}{0.893003in}%
\pgfsys@useobject{currentmarker}{}%
\end{pgfscope}%
\end{pgfscope}%
\begin{pgfscope}%
\pgfsetbuttcap%
\pgfsetroundjoin%
\definecolor{currentfill}{rgb}{0.000000,0.000000,0.000000}%
\pgfsetfillcolor{currentfill}%
\pgfsetlinewidth{0.501875pt}%
\definecolor{currentstroke}{rgb}{0.000000,0.000000,0.000000}%
\pgfsetstrokecolor{currentstroke}%
\pgfsetdash{}{0pt}%
\pgfsys@defobject{currentmarker}{\pgfqpoint{0.000000in}{0.000000in}}{\pgfqpoint{0.000000in}{0.020833in}}{%
\pgfpathmoveto{\pgfqpoint{0.000000in}{0.000000in}}%
\pgfpathlineto{\pgfqpoint{0.000000in}{0.020833in}}%
\pgfusepath{stroke,fill}%
}%
\begin{pgfscope}%
\pgfsys@transformshift{5.580165in}{0.586309in}%
\pgfsys@useobject{currentmarker}{}%
\end{pgfscope}%
\end{pgfscope}%
\begin{pgfscope}%
\pgfsetbuttcap%
\pgfsetroundjoin%
\definecolor{currentfill}{rgb}{0.000000,0.000000,0.000000}%
\pgfsetfillcolor{currentfill}%
\pgfsetlinewidth{0.501875pt}%
\definecolor{currentstroke}{rgb}{0.000000,0.000000,0.000000}%
\pgfsetstrokecolor{currentstroke}%
\pgfsetdash{}{0pt}%
\pgfsys@defobject{currentmarker}{\pgfqpoint{0.000000in}{-0.020833in}}{\pgfqpoint{0.000000in}{0.000000in}}{%
\pgfpathmoveto{\pgfqpoint{0.000000in}{0.000000in}}%
\pgfpathlineto{\pgfqpoint{0.000000in}{-0.020833in}}%
\pgfusepath{stroke,fill}%
}%
\begin{pgfscope}%
\pgfsys@transformshift{5.580165in}{0.893003in}%
\pgfsys@useobject{currentmarker}{}%
\end{pgfscope}%
\end{pgfscope}%
\begin{pgfscope}%
\pgfsetbuttcap%
\pgfsetroundjoin%
\definecolor{currentfill}{rgb}{0.000000,0.000000,0.000000}%
\pgfsetfillcolor{currentfill}%
\pgfsetlinewidth{0.501875pt}%
\definecolor{currentstroke}{rgb}{0.000000,0.000000,0.000000}%
\pgfsetstrokecolor{currentstroke}%
\pgfsetdash{}{0pt}%
\pgfsys@defobject{currentmarker}{\pgfqpoint{0.000000in}{0.000000in}}{\pgfqpoint{0.000000in}{0.020833in}}{%
\pgfpathmoveto{\pgfqpoint{0.000000in}{0.000000in}}%
\pgfpathlineto{\pgfqpoint{0.000000in}{0.020833in}}%
\pgfusepath{stroke,fill}%
}%
\begin{pgfscope}%
\pgfsys@transformshift{5.615862in}{0.586309in}%
\pgfsys@useobject{currentmarker}{}%
\end{pgfscope}%
\end{pgfscope}%
\begin{pgfscope}%
\pgfsetbuttcap%
\pgfsetroundjoin%
\definecolor{currentfill}{rgb}{0.000000,0.000000,0.000000}%
\pgfsetfillcolor{currentfill}%
\pgfsetlinewidth{0.501875pt}%
\definecolor{currentstroke}{rgb}{0.000000,0.000000,0.000000}%
\pgfsetstrokecolor{currentstroke}%
\pgfsetdash{}{0pt}%
\pgfsys@defobject{currentmarker}{\pgfqpoint{0.000000in}{-0.020833in}}{\pgfqpoint{0.000000in}{0.000000in}}{%
\pgfpathmoveto{\pgfqpoint{0.000000in}{0.000000in}}%
\pgfpathlineto{\pgfqpoint{0.000000in}{-0.020833in}}%
\pgfusepath{stroke,fill}%
}%
\begin{pgfscope}%
\pgfsys@transformshift{5.615862in}{0.893003in}%
\pgfsys@useobject{currentmarker}{}%
\end{pgfscope}%
\end{pgfscope}%
\begin{pgfscope}%
\pgfsetbuttcap%
\pgfsetroundjoin%
\definecolor{currentfill}{rgb}{0.000000,0.000000,0.000000}%
\pgfsetfillcolor{currentfill}%
\pgfsetlinewidth{0.501875pt}%
\definecolor{currentstroke}{rgb}{0.000000,0.000000,0.000000}%
\pgfsetstrokecolor{currentstroke}%
\pgfsetdash{}{0pt}%
\pgfsys@defobject{currentmarker}{\pgfqpoint{0.000000in}{0.000000in}}{\pgfqpoint{0.000000in}{0.020833in}}{%
\pgfpathmoveto{\pgfqpoint{0.000000in}{0.000000in}}%
\pgfpathlineto{\pgfqpoint{0.000000in}{0.020833in}}%
\pgfusepath{stroke,fill}%
}%
\begin{pgfscope}%
\pgfsys@transformshift{5.651559in}{0.586309in}%
\pgfsys@useobject{currentmarker}{}%
\end{pgfscope}%
\end{pgfscope}%
\begin{pgfscope}%
\pgfsetbuttcap%
\pgfsetroundjoin%
\definecolor{currentfill}{rgb}{0.000000,0.000000,0.000000}%
\pgfsetfillcolor{currentfill}%
\pgfsetlinewidth{0.501875pt}%
\definecolor{currentstroke}{rgb}{0.000000,0.000000,0.000000}%
\pgfsetstrokecolor{currentstroke}%
\pgfsetdash{}{0pt}%
\pgfsys@defobject{currentmarker}{\pgfqpoint{0.000000in}{-0.020833in}}{\pgfqpoint{0.000000in}{0.000000in}}{%
\pgfpathmoveto{\pgfqpoint{0.000000in}{0.000000in}}%
\pgfpathlineto{\pgfqpoint{0.000000in}{-0.020833in}}%
\pgfusepath{stroke,fill}%
}%
\begin{pgfscope}%
\pgfsys@transformshift{5.651559in}{0.893003in}%
\pgfsys@useobject{currentmarker}{}%
\end{pgfscope}%
\end{pgfscope}%
\begin{pgfscope}%
\pgfsetbuttcap%
\pgfsetroundjoin%
\definecolor{currentfill}{rgb}{0.000000,0.000000,0.000000}%
\pgfsetfillcolor{currentfill}%
\pgfsetlinewidth{0.501875pt}%
\definecolor{currentstroke}{rgb}{0.000000,0.000000,0.000000}%
\pgfsetstrokecolor{currentstroke}%
\pgfsetdash{}{0pt}%
\pgfsys@defobject{currentmarker}{\pgfqpoint{0.000000in}{0.000000in}}{\pgfqpoint{0.000000in}{0.020833in}}{%
\pgfpathmoveto{\pgfqpoint{0.000000in}{0.000000in}}%
\pgfpathlineto{\pgfqpoint{0.000000in}{0.020833in}}%
\pgfusepath{stroke,fill}%
}%
\begin{pgfscope}%
\pgfsys@transformshift{5.687257in}{0.586309in}%
\pgfsys@useobject{currentmarker}{}%
\end{pgfscope}%
\end{pgfscope}%
\begin{pgfscope}%
\pgfsetbuttcap%
\pgfsetroundjoin%
\definecolor{currentfill}{rgb}{0.000000,0.000000,0.000000}%
\pgfsetfillcolor{currentfill}%
\pgfsetlinewidth{0.501875pt}%
\definecolor{currentstroke}{rgb}{0.000000,0.000000,0.000000}%
\pgfsetstrokecolor{currentstroke}%
\pgfsetdash{}{0pt}%
\pgfsys@defobject{currentmarker}{\pgfqpoint{0.000000in}{-0.020833in}}{\pgfqpoint{0.000000in}{0.000000in}}{%
\pgfpathmoveto{\pgfqpoint{0.000000in}{0.000000in}}%
\pgfpathlineto{\pgfqpoint{0.000000in}{-0.020833in}}%
\pgfusepath{stroke,fill}%
}%
\begin{pgfscope}%
\pgfsys@transformshift{5.687257in}{0.893003in}%
\pgfsys@useobject{currentmarker}{}%
\end{pgfscope}%
\end{pgfscope}%
\begin{pgfscope}%
\pgfsetbuttcap%
\pgfsetroundjoin%
\definecolor{currentfill}{rgb}{0.000000,0.000000,0.000000}%
\pgfsetfillcolor{currentfill}%
\pgfsetlinewidth{0.501875pt}%
\definecolor{currentstroke}{rgb}{0.000000,0.000000,0.000000}%
\pgfsetstrokecolor{currentstroke}%
\pgfsetdash{}{0pt}%
\pgfsys@defobject{currentmarker}{\pgfqpoint{0.000000in}{0.000000in}}{\pgfqpoint{0.000000in}{0.020833in}}{%
\pgfpathmoveto{\pgfqpoint{0.000000in}{0.000000in}}%
\pgfpathlineto{\pgfqpoint{0.000000in}{0.020833in}}%
\pgfusepath{stroke,fill}%
}%
\begin{pgfscope}%
\pgfsys@transformshift{5.722954in}{0.586309in}%
\pgfsys@useobject{currentmarker}{}%
\end{pgfscope}%
\end{pgfscope}%
\begin{pgfscope}%
\pgfsetbuttcap%
\pgfsetroundjoin%
\definecolor{currentfill}{rgb}{0.000000,0.000000,0.000000}%
\pgfsetfillcolor{currentfill}%
\pgfsetlinewidth{0.501875pt}%
\definecolor{currentstroke}{rgb}{0.000000,0.000000,0.000000}%
\pgfsetstrokecolor{currentstroke}%
\pgfsetdash{}{0pt}%
\pgfsys@defobject{currentmarker}{\pgfqpoint{0.000000in}{-0.020833in}}{\pgfqpoint{0.000000in}{0.000000in}}{%
\pgfpathmoveto{\pgfqpoint{0.000000in}{0.000000in}}%
\pgfpathlineto{\pgfqpoint{0.000000in}{-0.020833in}}%
\pgfusepath{stroke,fill}%
}%
\begin{pgfscope}%
\pgfsys@transformshift{5.722954in}{0.893003in}%
\pgfsys@useobject{currentmarker}{}%
\end{pgfscope}%
\end{pgfscope}%
\begin{pgfscope}%
\pgfsetbuttcap%
\pgfsetroundjoin%
\definecolor{currentfill}{rgb}{0.000000,0.000000,0.000000}%
\pgfsetfillcolor{currentfill}%
\pgfsetlinewidth{0.501875pt}%
\definecolor{currentstroke}{rgb}{0.000000,0.000000,0.000000}%
\pgfsetstrokecolor{currentstroke}%
\pgfsetdash{}{0pt}%
\pgfsys@defobject{currentmarker}{\pgfqpoint{0.000000in}{0.000000in}}{\pgfqpoint{0.000000in}{0.020833in}}{%
\pgfpathmoveto{\pgfqpoint{0.000000in}{0.000000in}}%
\pgfpathlineto{\pgfqpoint{0.000000in}{0.020833in}}%
\pgfusepath{stroke,fill}%
}%
\begin{pgfscope}%
\pgfsys@transformshift{5.758651in}{0.586309in}%
\pgfsys@useobject{currentmarker}{}%
\end{pgfscope}%
\end{pgfscope}%
\begin{pgfscope}%
\pgfsetbuttcap%
\pgfsetroundjoin%
\definecolor{currentfill}{rgb}{0.000000,0.000000,0.000000}%
\pgfsetfillcolor{currentfill}%
\pgfsetlinewidth{0.501875pt}%
\definecolor{currentstroke}{rgb}{0.000000,0.000000,0.000000}%
\pgfsetstrokecolor{currentstroke}%
\pgfsetdash{}{0pt}%
\pgfsys@defobject{currentmarker}{\pgfqpoint{0.000000in}{-0.020833in}}{\pgfqpoint{0.000000in}{0.000000in}}{%
\pgfpathmoveto{\pgfqpoint{0.000000in}{0.000000in}}%
\pgfpathlineto{\pgfqpoint{0.000000in}{-0.020833in}}%
\pgfusepath{stroke,fill}%
}%
\begin{pgfscope}%
\pgfsys@transformshift{5.758651in}{0.893003in}%
\pgfsys@useobject{currentmarker}{}%
\end{pgfscope}%
\end{pgfscope}%
\begin{pgfscope}%
\pgfsetbuttcap%
\pgfsetroundjoin%
\definecolor{currentfill}{rgb}{0.000000,0.000000,0.000000}%
\pgfsetfillcolor{currentfill}%
\pgfsetlinewidth{0.501875pt}%
\definecolor{currentstroke}{rgb}{0.000000,0.000000,0.000000}%
\pgfsetstrokecolor{currentstroke}%
\pgfsetdash{}{0pt}%
\pgfsys@defobject{currentmarker}{\pgfqpoint{0.000000in}{0.000000in}}{\pgfqpoint{0.000000in}{0.020833in}}{%
\pgfpathmoveto{\pgfqpoint{0.000000in}{0.000000in}}%
\pgfpathlineto{\pgfqpoint{0.000000in}{0.020833in}}%
\pgfusepath{stroke,fill}%
}%
\begin{pgfscope}%
\pgfsys@transformshift{5.794349in}{0.586309in}%
\pgfsys@useobject{currentmarker}{}%
\end{pgfscope}%
\end{pgfscope}%
\begin{pgfscope}%
\pgfsetbuttcap%
\pgfsetroundjoin%
\definecolor{currentfill}{rgb}{0.000000,0.000000,0.000000}%
\pgfsetfillcolor{currentfill}%
\pgfsetlinewidth{0.501875pt}%
\definecolor{currentstroke}{rgb}{0.000000,0.000000,0.000000}%
\pgfsetstrokecolor{currentstroke}%
\pgfsetdash{}{0pt}%
\pgfsys@defobject{currentmarker}{\pgfqpoint{0.000000in}{-0.020833in}}{\pgfqpoint{0.000000in}{0.000000in}}{%
\pgfpathmoveto{\pgfqpoint{0.000000in}{0.000000in}}%
\pgfpathlineto{\pgfqpoint{0.000000in}{-0.020833in}}%
\pgfusepath{stroke,fill}%
}%
\begin{pgfscope}%
\pgfsys@transformshift{5.794349in}{0.893003in}%
\pgfsys@useobject{currentmarker}{}%
\end{pgfscope}%
\end{pgfscope}%
\begin{pgfscope}%
\pgfsetbuttcap%
\pgfsetroundjoin%
\definecolor{currentfill}{rgb}{0.000000,0.000000,0.000000}%
\pgfsetfillcolor{currentfill}%
\pgfsetlinewidth{0.501875pt}%
\definecolor{currentstroke}{rgb}{0.000000,0.000000,0.000000}%
\pgfsetstrokecolor{currentstroke}%
\pgfsetdash{}{0pt}%
\pgfsys@defobject{currentmarker}{\pgfqpoint{0.000000in}{0.000000in}}{\pgfqpoint{0.000000in}{0.020833in}}{%
\pgfpathmoveto{\pgfqpoint{0.000000in}{0.000000in}}%
\pgfpathlineto{\pgfqpoint{0.000000in}{0.020833in}}%
\pgfusepath{stroke,fill}%
}%
\begin{pgfscope}%
\pgfsys@transformshift{5.865743in}{0.586309in}%
\pgfsys@useobject{currentmarker}{}%
\end{pgfscope}%
\end{pgfscope}%
\begin{pgfscope}%
\pgfsetbuttcap%
\pgfsetroundjoin%
\definecolor{currentfill}{rgb}{0.000000,0.000000,0.000000}%
\pgfsetfillcolor{currentfill}%
\pgfsetlinewidth{0.501875pt}%
\definecolor{currentstroke}{rgb}{0.000000,0.000000,0.000000}%
\pgfsetstrokecolor{currentstroke}%
\pgfsetdash{}{0pt}%
\pgfsys@defobject{currentmarker}{\pgfqpoint{0.000000in}{-0.020833in}}{\pgfqpoint{0.000000in}{0.000000in}}{%
\pgfpathmoveto{\pgfqpoint{0.000000in}{0.000000in}}%
\pgfpathlineto{\pgfqpoint{0.000000in}{-0.020833in}}%
\pgfusepath{stroke,fill}%
}%
\begin{pgfscope}%
\pgfsys@transformshift{5.865743in}{0.893003in}%
\pgfsys@useobject{currentmarker}{}%
\end{pgfscope}%
\end{pgfscope}%
\begin{pgfscope}%
\pgfsetbuttcap%
\pgfsetroundjoin%
\definecolor{currentfill}{rgb}{0.000000,0.000000,0.000000}%
\pgfsetfillcolor{currentfill}%
\pgfsetlinewidth{0.501875pt}%
\definecolor{currentstroke}{rgb}{0.000000,0.000000,0.000000}%
\pgfsetstrokecolor{currentstroke}%
\pgfsetdash{}{0pt}%
\pgfsys@defobject{currentmarker}{\pgfqpoint{0.000000in}{0.000000in}}{\pgfqpoint{0.000000in}{0.020833in}}{%
\pgfpathmoveto{\pgfqpoint{0.000000in}{0.000000in}}%
\pgfpathlineto{\pgfqpoint{0.000000in}{0.020833in}}%
\pgfusepath{stroke,fill}%
}%
\begin{pgfscope}%
\pgfsys@transformshift{5.901441in}{0.586309in}%
\pgfsys@useobject{currentmarker}{}%
\end{pgfscope}%
\end{pgfscope}%
\begin{pgfscope}%
\pgfsetbuttcap%
\pgfsetroundjoin%
\definecolor{currentfill}{rgb}{0.000000,0.000000,0.000000}%
\pgfsetfillcolor{currentfill}%
\pgfsetlinewidth{0.501875pt}%
\definecolor{currentstroke}{rgb}{0.000000,0.000000,0.000000}%
\pgfsetstrokecolor{currentstroke}%
\pgfsetdash{}{0pt}%
\pgfsys@defobject{currentmarker}{\pgfqpoint{0.000000in}{-0.020833in}}{\pgfqpoint{0.000000in}{0.000000in}}{%
\pgfpathmoveto{\pgfqpoint{0.000000in}{0.000000in}}%
\pgfpathlineto{\pgfqpoint{0.000000in}{-0.020833in}}%
\pgfusepath{stroke,fill}%
}%
\begin{pgfscope}%
\pgfsys@transformshift{5.901441in}{0.893003in}%
\pgfsys@useobject{currentmarker}{}%
\end{pgfscope}%
\end{pgfscope}%
\begin{pgfscope}%
\pgfsetbuttcap%
\pgfsetroundjoin%
\definecolor{currentfill}{rgb}{0.000000,0.000000,0.000000}%
\pgfsetfillcolor{currentfill}%
\pgfsetlinewidth{0.501875pt}%
\definecolor{currentstroke}{rgb}{0.000000,0.000000,0.000000}%
\pgfsetstrokecolor{currentstroke}%
\pgfsetdash{}{0pt}%
\pgfsys@defobject{currentmarker}{\pgfqpoint{0.000000in}{0.000000in}}{\pgfqpoint{0.000000in}{0.020833in}}{%
\pgfpathmoveto{\pgfqpoint{0.000000in}{0.000000in}}%
\pgfpathlineto{\pgfqpoint{0.000000in}{0.020833in}}%
\pgfusepath{stroke,fill}%
}%
\begin{pgfscope}%
\pgfsys@transformshift{5.937138in}{0.586309in}%
\pgfsys@useobject{currentmarker}{}%
\end{pgfscope}%
\end{pgfscope}%
\begin{pgfscope}%
\pgfsetbuttcap%
\pgfsetroundjoin%
\definecolor{currentfill}{rgb}{0.000000,0.000000,0.000000}%
\pgfsetfillcolor{currentfill}%
\pgfsetlinewidth{0.501875pt}%
\definecolor{currentstroke}{rgb}{0.000000,0.000000,0.000000}%
\pgfsetstrokecolor{currentstroke}%
\pgfsetdash{}{0pt}%
\pgfsys@defobject{currentmarker}{\pgfqpoint{0.000000in}{-0.020833in}}{\pgfqpoint{0.000000in}{0.000000in}}{%
\pgfpathmoveto{\pgfqpoint{0.000000in}{0.000000in}}%
\pgfpathlineto{\pgfqpoint{0.000000in}{-0.020833in}}%
\pgfusepath{stroke,fill}%
}%
\begin{pgfscope}%
\pgfsys@transformshift{5.937138in}{0.893003in}%
\pgfsys@useobject{currentmarker}{}%
\end{pgfscope}%
\end{pgfscope}%
\begin{pgfscope}%
\pgfsetbuttcap%
\pgfsetroundjoin%
\definecolor{currentfill}{rgb}{0.000000,0.000000,0.000000}%
\pgfsetfillcolor{currentfill}%
\pgfsetlinewidth{0.501875pt}%
\definecolor{currentstroke}{rgb}{0.000000,0.000000,0.000000}%
\pgfsetstrokecolor{currentstroke}%
\pgfsetdash{}{0pt}%
\pgfsys@defobject{currentmarker}{\pgfqpoint{0.000000in}{0.000000in}}{\pgfqpoint{0.000000in}{0.020833in}}{%
\pgfpathmoveto{\pgfqpoint{0.000000in}{0.000000in}}%
\pgfpathlineto{\pgfqpoint{0.000000in}{0.020833in}}%
\pgfusepath{stroke,fill}%
}%
\begin{pgfscope}%
\pgfsys@transformshift{5.972835in}{0.586309in}%
\pgfsys@useobject{currentmarker}{}%
\end{pgfscope}%
\end{pgfscope}%
\begin{pgfscope}%
\pgfsetbuttcap%
\pgfsetroundjoin%
\definecolor{currentfill}{rgb}{0.000000,0.000000,0.000000}%
\pgfsetfillcolor{currentfill}%
\pgfsetlinewidth{0.501875pt}%
\definecolor{currentstroke}{rgb}{0.000000,0.000000,0.000000}%
\pgfsetstrokecolor{currentstroke}%
\pgfsetdash{}{0pt}%
\pgfsys@defobject{currentmarker}{\pgfqpoint{0.000000in}{-0.020833in}}{\pgfqpoint{0.000000in}{0.000000in}}{%
\pgfpathmoveto{\pgfqpoint{0.000000in}{0.000000in}}%
\pgfpathlineto{\pgfqpoint{0.000000in}{-0.020833in}}%
\pgfusepath{stroke,fill}%
}%
\begin{pgfscope}%
\pgfsys@transformshift{5.972835in}{0.893003in}%
\pgfsys@useobject{currentmarker}{}%
\end{pgfscope}%
\end{pgfscope}%
\begin{pgfscope}%
\pgfsetbuttcap%
\pgfsetroundjoin%
\definecolor{currentfill}{rgb}{0.000000,0.000000,0.000000}%
\pgfsetfillcolor{currentfill}%
\pgfsetlinewidth{0.501875pt}%
\definecolor{currentstroke}{rgb}{0.000000,0.000000,0.000000}%
\pgfsetstrokecolor{currentstroke}%
\pgfsetdash{}{0pt}%
\pgfsys@defobject{currentmarker}{\pgfqpoint{0.000000in}{0.000000in}}{\pgfqpoint{0.000000in}{0.020833in}}{%
\pgfpathmoveto{\pgfqpoint{0.000000in}{0.000000in}}%
\pgfpathlineto{\pgfqpoint{0.000000in}{0.020833in}}%
\pgfusepath{stroke,fill}%
}%
\begin{pgfscope}%
\pgfsys@transformshift{6.008533in}{0.586309in}%
\pgfsys@useobject{currentmarker}{}%
\end{pgfscope}%
\end{pgfscope}%
\begin{pgfscope}%
\pgfsetbuttcap%
\pgfsetroundjoin%
\definecolor{currentfill}{rgb}{0.000000,0.000000,0.000000}%
\pgfsetfillcolor{currentfill}%
\pgfsetlinewidth{0.501875pt}%
\definecolor{currentstroke}{rgb}{0.000000,0.000000,0.000000}%
\pgfsetstrokecolor{currentstroke}%
\pgfsetdash{}{0pt}%
\pgfsys@defobject{currentmarker}{\pgfqpoint{0.000000in}{-0.020833in}}{\pgfqpoint{0.000000in}{0.000000in}}{%
\pgfpathmoveto{\pgfqpoint{0.000000in}{0.000000in}}%
\pgfpathlineto{\pgfqpoint{0.000000in}{-0.020833in}}%
\pgfusepath{stroke,fill}%
}%
\begin{pgfscope}%
\pgfsys@transformshift{6.008533in}{0.893003in}%
\pgfsys@useobject{currentmarker}{}%
\end{pgfscope}%
\end{pgfscope}%
\begin{pgfscope}%
\pgfsetbuttcap%
\pgfsetroundjoin%
\definecolor{currentfill}{rgb}{0.000000,0.000000,0.000000}%
\pgfsetfillcolor{currentfill}%
\pgfsetlinewidth{0.501875pt}%
\definecolor{currentstroke}{rgb}{0.000000,0.000000,0.000000}%
\pgfsetstrokecolor{currentstroke}%
\pgfsetdash{}{0pt}%
\pgfsys@defobject{currentmarker}{\pgfqpoint{0.000000in}{0.000000in}}{\pgfqpoint{0.000000in}{0.020833in}}{%
\pgfpathmoveto{\pgfqpoint{0.000000in}{0.000000in}}%
\pgfpathlineto{\pgfqpoint{0.000000in}{0.020833in}}%
\pgfusepath{stroke,fill}%
}%
\begin{pgfscope}%
\pgfsys@transformshift{6.044230in}{0.586309in}%
\pgfsys@useobject{currentmarker}{}%
\end{pgfscope}%
\end{pgfscope}%
\begin{pgfscope}%
\pgfsetbuttcap%
\pgfsetroundjoin%
\definecolor{currentfill}{rgb}{0.000000,0.000000,0.000000}%
\pgfsetfillcolor{currentfill}%
\pgfsetlinewidth{0.501875pt}%
\definecolor{currentstroke}{rgb}{0.000000,0.000000,0.000000}%
\pgfsetstrokecolor{currentstroke}%
\pgfsetdash{}{0pt}%
\pgfsys@defobject{currentmarker}{\pgfqpoint{0.000000in}{-0.020833in}}{\pgfqpoint{0.000000in}{0.000000in}}{%
\pgfpathmoveto{\pgfqpoint{0.000000in}{0.000000in}}%
\pgfpathlineto{\pgfqpoint{0.000000in}{-0.020833in}}%
\pgfusepath{stroke,fill}%
}%
\begin{pgfscope}%
\pgfsys@transformshift{6.044230in}{0.893003in}%
\pgfsys@useobject{currentmarker}{}%
\end{pgfscope}%
\end{pgfscope}%
\begin{pgfscope}%
\pgfsetbuttcap%
\pgfsetroundjoin%
\definecolor{currentfill}{rgb}{0.000000,0.000000,0.000000}%
\pgfsetfillcolor{currentfill}%
\pgfsetlinewidth{0.501875pt}%
\definecolor{currentstroke}{rgb}{0.000000,0.000000,0.000000}%
\pgfsetstrokecolor{currentstroke}%
\pgfsetdash{}{0pt}%
\pgfsys@defobject{currentmarker}{\pgfqpoint{0.000000in}{0.000000in}}{\pgfqpoint{0.000000in}{0.020833in}}{%
\pgfpathmoveto{\pgfqpoint{0.000000in}{0.000000in}}%
\pgfpathlineto{\pgfqpoint{0.000000in}{0.020833in}}%
\pgfusepath{stroke,fill}%
}%
\begin{pgfscope}%
\pgfsys@transformshift{6.079927in}{0.586309in}%
\pgfsys@useobject{currentmarker}{}%
\end{pgfscope}%
\end{pgfscope}%
\begin{pgfscope}%
\pgfsetbuttcap%
\pgfsetroundjoin%
\definecolor{currentfill}{rgb}{0.000000,0.000000,0.000000}%
\pgfsetfillcolor{currentfill}%
\pgfsetlinewidth{0.501875pt}%
\definecolor{currentstroke}{rgb}{0.000000,0.000000,0.000000}%
\pgfsetstrokecolor{currentstroke}%
\pgfsetdash{}{0pt}%
\pgfsys@defobject{currentmarker}{\pgfqpoint{0.000000in}{-0.020833in}}{\pgfqpoint{0.000000in}{0.000000in}}{%
\pgfpathmoveto{\pgfqpoint{0.000000in}{0.000000in}}%
\pgfpathlineto{\pgfqpoint{0.000000in}{-0.020833in}}%
\pgfusepath{stroke,fill}%
}%
\begin{pgfscope}%
\pgfsys@transformshift{6.079927in}{0.893003in}%
\pgfsys@useobject{currentmarker}{}%
\end{pgfscope}%
\end{pgfscope}%
\begin{pgfscope}%
\pgfsetbuttcap%
\pgfsetroundjoin%
\definecolor{currentfill}{rgb}{0.000000,0.000000,0.000000}%
\pgfsetfillcolor{currentfill}%
\pgfsetlinewidth{0.501875pt}%
\definecolor{currentstroke}{rgb}{0.000000,0.000000,0.000000}%
\pgfsetstrokecolor{currentstroke}%
\pgfsetdash{}{0pt}%
\pgfsys@defobject{currentmarker}{\pgfqpoint{0.000000in}{0.000000in}}{\pgfqpoint{0.000000in}{0.020833in}}{%
\pgfpathmoveto{\pgfqpoint{0.000000in}{0.000000in}}%
\pgfpathlineto{\pgfqpoint{0.000000in}{0.020833in}}%
\pgfusepath{stroke,fill}%
}%
\begin{pgfscope}%
\pgfsys@transformshift{6.115625in}{0.586309in}%
\pgfsys@useobject{currentmarker}{}%
\end{pgfscope}%
\end{pgfscope}%
\begin{pgfscope}%
\pgfsetbuttcap%
\pgfsetroundjoin%
\definecolor{currentfill}{rgb}{0.000000,0.000000,0.000000}%
\pgfsetfillcolor{currentfill}%
\pgfsetlinewidth{0.501875pt}%
\definecolor{currentstroke}{rgb}{0.000000,0.000000,0.000000}%
\pgfsetstrokecolor{currentstroke}%
\pgfsetdash{}{0pt}%
\pgfsys@defobject{currentmarker}{\pgfqpoint{0.000000in}{-0.020833in}}{\pgfqpoint{0.000000in}{0.000000in}}{%
\pgfpathmoveto{\pgfqpoint{0.000000in}{0.000000in}}%
\pgfpathlineto{\pgfqpoint{0.000000in}{-0.020833in}}%
\pgfusepath{stroke,fill}%
}%
\begin{pgfscope}%
\pgfsys@transformshift{6.115625in}{0.893003in}%
\pgfsys@useobject{currentmarker}{}%
\end{pgfscope}%
\end{pgfscope}%
\begin{pgfscope}%
\pgfsetbuttcap%
\pgfsetroundjoin%
\definecolor{currentfill}{rgb}{0.000000,0.000000,0.000000}%
\pgfsetfillcolor{currentfill}%
\pgfsetlinewidth{0.501875pt}%
\definecolor{currentstroke}{rgb}{0.000000,0.000000,0.000000}%
\pgfsetstrokecolor{currentstroke}%
\pgfsetdash{}{0pt}%
\pgfsys@defobject{currentmarker}{\pgfqpoint{0.000000in}{0.000000in}}{\pgfqpoint{0.000000in}{0.020833in}}{%
\pgfpathmoveto{\pgfqpoint{0.000000in}{0.000000in}}%
\pgfpathlineto{\pgfqpoint{0.000000in}{0.020833in}}%
\pgfusepath{stroke,fill}%
}%
\begin{pgfscope}%
\pgfsys@transformshift{6.151322in}{0.586309in}%
\pgfsys@useobject{currentmarker}{}%
\end{pgfscope}%
\end{pgfscope}%
\begin{pgfscope}%
\pgfsetbuttcap%
\pgfsetroundjoin%
\definecolor{currentfill}{rgb}{0.000000,0.000000,0.000000}%
\pgfsetfillcolor{currentfill}%
\pgfsetlinewidth{0.501875pt}%
\definecolor{currentstroke}{rgb}{0.000000,0.000000,0.000000}%
\pgfsetstrokecolor{currentstroke}%
\pgfsetdash{}{0pt}%
\pgfsys@defobject{currentmarker}{\pgfqpoint{0.000000in}{-0.020833in}}{\pgfqpoint{0.000000in}{0.000000in}}{%
\pgfpathmoveto{\pgfqpoint{0.000000in}{0.000000in}}%
\pgfpathlineto{\pgfqpoint{0.000000in}{-0.020833in}}%
\pgfusepath{stroke,fill}%
}%
\begin{pgfscope}%
\pgfsys@transformshift{6.151322in}{0.893003in}%
\pgfsys@useobject{currentmarker}{}%
\end{pgfscope}%
\end{pgfscope}%
\begin{pgfscope}%
\pgfsetbuttcap%
\pgfsetroundjoin%
\definecolor{currentfill}{rgb}{0.000000,0.000000,0.000000}%
\pgfsetfillcolor{currentfill}%
\pgfsetlinewidth{0.501875pt}%
\definecolor{currentstroke}{rgb}{0.000000,0.000000,0.000000}%
\pgfsetstrokecolor{currentstroke}%
\pgfsetdash{}{0pt}%
\pgfsys@defobject{currentmarker}{\pgfqpoint{0.000000in}{0.000000in}}{\pgfqpoint{0.000000in}{0.020833in}}{%
\pgfpathmoveto{\pgfqpoint{0.000000in}{0.000000in}}%
\pgfpathlineto{\pgfqpoint{0.000000in}{0.020833in}}%
\pgfusepath{stroke,fill}%
}%
\begin{pgfscope}%
\pgfsys@transformshift{6.187019in}{0.586309in}%
\pgfsys@useobject{currentmarker}{}%
\end{pgfscope}%
\end{pgfscope}%
\begin{pgfscope}%
\pgfsetbuttcap%
\pgfsetroundjoin%
\definecolor{currentfill}{rgb}{0.000000,0.000000,0.000000}%
\pgfsetfillcolor{currentfill}%
\pgfsetlinewidth{0.501875pt}%
\definecolor{currentstroke}{rgb}{0.000000,0.000000,0.000000}%
\pgfsetstrokecolor{currentstroke}%
\pgfsetdash{}{0pt}%
\pgfsys@defobject{currentmarker}{\pgfqpoint{0.000000in}{-0.020833in}}{\pgfqpoint{0.000000in}{0.000000in}}{%
\pgfpathmoveto{\pgfqpoint{0.000000in}{0.000000in}}%
\pgfpathlineto{\pgfqpoint{0.000000in}{-0.020833in}}%
\pgfusepath{stroke,fill}%
}%
\begin{pgfscope}%
\pgfsys@transformshift{6.187019in}{0.893003in}%
\pgfsys@useobject{currentmarker}{}%
\end{pgfscope}%
\end{pgfscope}%
\begin{pgfscope}%
\pgfsetbuttcap%
\pgfsetroundjoin%
\definecolor{currentfill}{rgb}{0.000000,0.000000,0.000000}%
\pgfsetfillcolor{currentfill}%
\pgfsetlinewidth{0.501875pt}%
\definecolor{currentstroke}{rgb}{0.000000,0.000000,0.000000}%
\pgfsetstrokecolor{currentstroke}%
\pgfsetdash{}{0pt}%
\pgfsys@defobject{currentmarker}{\pgfqpoint{0.000000in}{0.000000in}}{\pgfqpoint{0.000000in}{0.020833in}}{%
\pgfpathmoveto{\pgfqpoint{0.000000in}{0.000000in}}%
\pgfpathlineto{\pgfqpoint{0.000000in}{0.020833in}}%
\pgfusepath{stroke,fill}%
}%
\begin{pgfscope}%
\pgfsys@transformshift{6.222717in}{0.586309in}%
\pgfsys@useobject{currentmarker}{}%
\end{pgfscope}%
\end{pgfscope}%
\begin{pgfscope}%
\pgfsetbuttcap%
\pgfsetroundjoin%
\definecolor{currentfill}{rgb}{0.000000,0.000000,0.000000}%
\pgfsetfillcolor{currentfill}%
\pgfsetlinewidth{0.501875pt}%
\definecolor{currentstroke}{rgb}{0.000000,0.000000,0.000000}%
\pgfsetstrokecolor{currentstroke}%
\pgfsetdash{}{0pt}%
\pgfsys@defobject{currentmarker}{\pgfqpoint{0.000000in}{-0.020833in}}{\pgfqpoint{0.000000in}{0.000000in}}{%
\pgfpathmoveto{\pgfqpoint{0.000000in}{0.000000in}}%
\pgfpathlineto{\pgfqpoint{0.000000in}{-0.020833in}}%
\pgfusepath{stroke,fill}%
}%
\begin{pgfscope}%
\pgfsys@transformshift{6.222717in}{0.893003in}%
\pgfsys@useobject{currentmarker}{}%
\end{pgfscope}%
\end{pgfscope}%
\begin{pgfscope}%
\definecolor{textcolor}{rgb}{0.000000,0.000000,0.000000}%
\pgfsetstrokecolor{textcolor}%
\pgfsetfillcolor{textcolor}%
\pgftext[x=3.374517in,y=0.148667in,,top]{\color{textcolor}\rmfamily\fontsize{8.000000}{9.600000}\selectfont Zeit}%
\end{pgfscope}%
\begin{pgfscope}%
\pgfsetbuttcap%
\pgfsetroundjoin%
\definecolor{currentfill}{rgb}{0.000000,0.000000,0.000000}%
\pgfsetfillcolor{currentfill}%
\pgfsetlinewidth{0.501875pt}%
\definecolor{currentstroke}{rgb}{0.000000,0.000000,0.000000}%
\pgfsetstrokecolor{currentstroke}%
\pgfsetdash{}{0pt}%
\pgfsys@defobject{currentmarker}{\pgfqpoint{0.000000in}{0.000000in}}{\pgfqpoint{0.041667in}{0.000000in}}{%
\pgfpathmoveto{\pgfqpoint{0.000000in}{0.000000in}}%
\pgfpathlineto{\pgfqpoint{0.041667in}{0.000000in}}%
\pgfusepath{stroke,fill}%
}%
\begin{pgfscope}%
\pgfsys@transformshift{0.481681in}{0.607142in}%
\pgfsys@useobject{currentmarker}{}%
\end{pgfscope}%
\end{pgfscope}%
\begin{pgfscope}%
\pgfsetbuttcap%
\pgfsetroundjoin%
\definecolor{currentfill}{rgb}{0.000000,0.000000,0.000000}%
\pgfsetfillcolor{currentfill}%
\pgfsetlinewidth{0.501875pt}%
\definecolor{currentstroke}{rgb}{0.000000,0.000000,0.000000}%
\pgfsetstrokecolor{currentstroke}%
\pgfsetdash{}{0pt}%
\pgfsys@defobject{currentmarker}{\pgfqpoint{-0.041667in}{0.000000in}}{\pgfqpoint{-0.000000in}{0.000000in}}{%
\pgfpathmoveto{\pgfqpoint{-0.000000in}{0.000000in}}%
\pgfpathlineto{\pgfqpoint{-0.041667in}{0.000000in}}%
\pgfusepath{stroke,fill}%
}%
\begin{pgfscope}%
\pgfsys@transformshift{6.267353in}{0.607142in}%
\pgfsys@useobject{currentmarker}{}%
\end{pgfscope}%
\end{pgfscope}%
\begin{pgfscope}%
\definecolor{textcolor}{rgb}{0.000000,0.000000,0.000000}%
\pgfsetstrokecolor{textcolor}%
\pgfsetfillcolor{textcolor}%
\pgftext[x=0.290944in, y=0.573406in, left, base]{\color{textcolor}\rmfamily\fontsize{7.000000}{8.400000}\selectfont \ensuremath{-}1}%
\end{pgfscope}%
\begin{pgfscope}%
\pgfsetbuttcap%
\pgfsetroundjoin%
\definecolor{currentfill}{rgb}{0.000000,0.000000,0.000000}%
\pgfsetfillcolor{currentfill}%
\pgfsetlinewidth{0.501875pt}%
\definecolor{currentstroke}{rgb}{0.000000,0.000000,0.000000}%
\pgfsetstrokecolor{currentstroke}%
\pgfsetdash{}{0pt}%
\pgfsys@defobject{currentmarker}{\pgfqpoint{0.000000in}{0.000000in}}{\pgfqpoint{0.041667in}{0.000000in}}{%
\pgfpathmoveto{\pgfqpoint{0.000000in}{0.000000in}}%
\pgfpathlineto{\pgfqpoint{0.041667in}{0.000000in}}%
\pgfusepath{stroke,fill}%
}%
\begin{pgfscope}%
\pgfsys@transformshift{0.481681in}{0.739656in}%
\pgfsys@useobject{currentmarker}{}%
\end{pgfscope}%
\end{pgfscope}%
\begin{pgfscope}%
\pgfsetbuttcap%
\pgfsetroundjoin%
\definecolor{currentfill}{rgb}{0.000000,0.000000,0.000000}%
\pgfsetfillcolor{currentfill}%
\pgfsetlinewidth{0.501875pt}%
\definecolor{currentstroke}{rgb}{0.000000,0.000000,0.000000}%
\pgfsetstrokecolor{currentstroke}%
\pgfsetdash{}{0pt}%
\pgfsys@defobject{currentmarker}{\pgfqpoint{-0.041667in}{0.000000in}}{\pgfqpoint{-0.000000in}{0.000000in}}{%
\pgfpathmoveto{\pgfqpoint{-0.000000in}{0.000000in}}%
\pgfpathlineto{\pgfqpoint{-0.041667in}{0.000000in}}%
\pgfusepath{stroke,fill}%
}%
\begin{pgfscope}%
\pgfsys@transformshift{6.267353in}{0.739656in}%
\pgfsys@useobject{currentmarker}{}%
\end{pgfscope}%
\end{pgfscope}%
\begin{pgfscope}%
\definecolor{textcolor}{rgb}{0.000000,0.000000,0.000000}%
\pgfsetstrokecolor{textcolor}%
\pgfsetfillcolor{textcolor}%
\pgftext[x=0.377750in, y=0.705920in, left, base]{\color{textcolor}\rmfamily\fontsize{7.000000}{8.400000}\selectfont 0}%
\end{pgfscope}%
\begin{pgfscope}%
\pgfsetbuttcap%
\pgfsetroundjoin%
\definecolor{currentfill}{rgb}{0.000000,0.000000,0.000000}%
\pgfsetfillcolor{currentfill}%
\pgfsetlinewidth{0.501875pt}%
\definecolor{currentstroke}{rgb}{0.000000,0.000000,0.000000}%
\pgfsetstrokecolor{currentstroke}%
\pgfsetdash{}{0pt}%
\pgfsys@defobject{currentmarker}{\pgfqpoint{0.000000in}{0.000000in}}{\pgfqpoint{0.041667in}{0.000000in}}{%
\pgfpathmoveto{\pgfqpoint{0.000000in}{0.000000in}}%
\pgfpathlineto{\pgfqpoint{0.041667in}{0.000000in}}%
\pgfusepath{stroke,fill}%
}%
\begin{pgfscope}%
\pgfsys@transformshift{0.481681in}{0.872170in}%
\pgfsys@useobject{currentmarker}{}%
\end{pgfscope}%
\end{pgfscope}%
\begin{pgfscope}%
\pgfsetbuttcap%
\pgfsetroundjoin%
\definecolor{currentfill}{rgb}{0.000000,0.000000,0.000000}%
\pgfsetfillcolor{currentfill}%
\pgfsetlinewidth{0.501875pt}%
\definecolor{currentstroke}{rgb}{0.000000,0.000000,0.000000}%
\pgfsetstrokecolor{currentstroke}%
\pgfsetdash{}{0pt}%
\pgfsys@defobject{currentmarker}{\pgfqpoint{-0.041667in}{0.000000in}}{\pgfqpoint{-0.000000in}{0.000000in}}{%
\pgfpathmoveto{\pgfqpoint{-0.000000in}{0.000000in}}%
\pgfpathlineto{\pgfqpoint{-0.041667in}{0.000000in}}%
\pgfusepath{stroke,fill}%
}%
\begin{pgfscope}%
\pgfsys@transformshift{6.267353in}{0.872170in}%
\pgfsys@useobject{currentmarker}{}%
\end{pgfscope}%
\end{pgfscope}%
\begin{pgfscope}%
\definecolor{textcolor}{rgb}{0.000000,0.000000,0.000000}%
\pgfsetstrokecolor{textcolor}%
\pgfsetfillcolor{textcolor}%
\pgftext[x=0.377750in, y=0.838434in, left, base]{\color{textcolor}\rmfamily\fontsize{7.000000}{8.400000}\selectfont 1}%
\end{pgfscope}%
\begin{pgfscope}%
\pgfsetbuttcap%
\pgfsetroundjoin%
\definecolor{currentfill}{rgb}{0.000000,0.000000,0.000000}%
\pgfsetfillcolor{currentfill}%
\pgfsetlinewidth{0.501875pt}%
\definecolor{currentstroke}{rgb}{0.000000,0.000000,0.000000}%
\pgfsetstrokecolor{currentstroke}%
\pgfsetdash{}{0pt}%
\pgfsys@defobject{currentmarker}{\pgfqpoint{0.000000in}{0.000000in}}{\pgfqpoint{0.020833in}{0.000000in}}{%
\pgfpathmoveto{\pgfqpoint{0.000000in}{0.000000in}}%
\pgfpathlineto{\pgfqpoint{0.020833in}{0.000000in}}%
\pgfusepath{stroke,fill}%
}%
\begin{pgfscope}%
\pgfsys@transformshift{0.481681in}{0.633645in}%
\pgfsys@useobject{currentmarker}{}%
\end{pgfscope}%
\end{pgfscope}%
\begin{pgfscope}%
\pgfsetbuttcap%
\pgfsetroundjoin%
\definecolor{currentfill}{rgb}{0.000000,0.000000,0.000000}%
\pgfsetfillcolor{currentfill}%
\pgfsetlinewidth{0.501875pt}%
\definecolor{currentstroke}{rgb}{0.000000,0.000000,0.000000}%
\pgfsetstrokecolor{currentstroke}%
\pgfsetdash{}{0pt}%
\pgfsys@defobject{currentmarker}{\pgfqpoint{-0.020833in}{0.000000in}}{\pgfqpoint{-0.000000in}{0.000000in}}{%
\pgfpathmoveto{\pgfqpoint{-0.000000in}{0.000000in}}%
\pgfpathlineto{\pgfqpoint{-0.020833in}{0.000000in}}%
\pgfusepath{stroke,fill}%
}%
\begin{pgfscope}%
\pgfsys@transformshift{6.267353in}{0.633645in}%
\pgfsys@useobject{currentmarker}{}%
\end{pgfscope}%
\end{pgfscope}%
\begin{pgfscope}%
\pgfsetbuttcap%
\pgfsetroundjoin%
\definecolor{currentfill}{rgb}{0.000000,0.000000,0.000000}%
\pgfsetfillcolor{currentfill}%
\pgfsetlinewidth{0.501875pt}%
\definecolor{currentstroke}{rgb}{0.000000,0.000000,0.000000}%
\pgfsetstrokecolor{currentstroke}%
\pgfsetdash{}{0pt}%
\pgfsys@defobject{currentmarker}{\pgfqpoint{0.000000in}{0.000000in}}{\pgfqpoint{0.020833in}{0.000000in}}{%
\pgfpathmoveto{\pgfqpoint{0.000000in}{0.000000in}}%
\pgfpathlineto{\pgfqpoint{0.020833in}{0.000000in}}%
\pgfusepath{stroke,fill}%
}%
\begin{pgfscope}%
\pgfsys@transformshift{0.481681in}{0.660148in}%
\pgfsys@useobject{currentmarker}{}%
\end{pgfscope}%
\end{pgfscope}%
\begin{pgfscope}%
\pgfsetbuttcap%
\pgfsetroundjoin%
\definecolor{currentfill}{rgb}{0.000000,0.000000,0.000000}%
\pgfsetfillcolor{currentfill}%
\pgfsetlinewidth{0.501875pt}%
\definecolor{currentstroke}{rgb}{0.000000,0.000000,0.000000}%
\pgfsetstrokecolor{currentstroke}%
\pgfsetdash{}{0pt}%
\pgfsys@defobject{currentmarker}{\pgfqpoint{-0.020833in}{0.000000in}}{\pgfqpoint{-0.000000in}{0.000000in}}{%
\pgfpathmoveto{\pgfqpoint{-0.000000in}{0.000000in}}%
\pgfpathlineto{\pgfqpoint{-0.020833in}{0.000000in}}%
\pgfusepath{stroke,fill}%
}%
\begin{pgfscope}%
\pgfsys@transformshift{6.267353in}{0.660148in}%
\pgfsys@useobject{currentmarker}{}%
\end{pgfscope}%
\end{pgfscope}%
\begin{pgfscope}%
\pgfsetbuttcap%
\pgfsetroundjoin%
\definecolor{currentfill}{rgb}{0.000000,0.000000,0.000000}%
\pgfsetfillcolor{currentfill}%
\pgfsetlinewidth{0.501875pt}%
\definecolor{currentstroke}{rgb}{0.000000,0.000000,0.000000}%
\pgfsetstrokecolor{currentstroke}%
\pgfsetdash{}{0pt}%
\pgfsys@defobject{currentmarker}{\pgfqpoint{0.000000in}{0.000000in}}{\pgfqpoint{0.020833in}{0.000000in}}{%
\pgfpathmoveto{\pgfqpoint{0.000000in}{0.000000in}}%
\pgfpathlineto{\pgfqpoint{0.020833in}{0.000000in}}%
\pgfusepath{stroke,fill}%
}%
\begin{pgfscope}%
\pgfsys@transformshift{0.481681in}{0.686651in}%
\pgfsys@useobject{currentmarker}{}%
\end{pgfscope}%
\end{pgfscope}%
\begin{pgfscope}%
\pgfsetbuttcap%
\pgfsetroundjoin%
\definecolor{currentfill}{rgb}{0.000000,0.000000,0.000000}%
\pgfsetfillcolor{currentfill}%
\pgfsetlinewidth{0.501875pt}%
\definecolor{currentstroke}{rgb}{0.000000,0.000000,0.000000}%
\pgfsetstrokecolor{currentstroke}%
\pgfsetdash{}{0pt}%
\pgfsys@defobject{currentmarker}{\pgfqpoint{-0.020833in}{0.000000in}}{\pgfqpoint{-0.000000in}{0.000000in}}{%
\pgfpathmoveto{\pgfqpoint{-0.000000in}{0.000000in}}%
\pgfpathlineto{\pgfqpoint{-0.020833in}{0.000000in}}%
\pgfusepath{stroke,fill}%
}%
\begin{pgfscope}%
\pgfsys@transformshift{6.267353in}{0.686651in}%
\pgfsys@useobject{currentmarker}{}%
\end{pgfscope}%
\end{pgfscope}%
\begin{pgfscope}%
\pgfsetbuttcap%
\pgfsetroundjoin%
\definecolor{currentfill}{rgb}{0.000000,0.000000,0.000000}%
\pgfsetfillcolor{currentfill}%
\pgfsetlinewidth{0.501875pt}%
\definecolor{currentstroke}{rgb}{0.000000,0.000000,0.000000}%
\pgfsetstrokecolor{currentstroke}%
\pgfsetdash{}{0pt}%
\pgfsys@defobject{currentmarker}{\pgfqpoint{0.000000in}{0.000000in}}{\pgfqpoint{0.020833in}{0.000000in}}{%
\pgfpathmoveto{\pgfqpoint{0.000000in}{0.000000in}}%
\pgfpathlineto{\pgfqpoint{0.020833in}{0.000000in}}%
\pgfusepath{stroke,fill}%
}%
\begin{pgfscope}%
\pgfsys@transformshift{0.481681in}{0.713153in}%
\pgfsys@useobject{currentmarker}{}%
\end{pgfscope}%
\end{pgfscope}%
\begin{pgfscope}%
\pgfsetbuttcap%
\pgfsetroundjoin%
\definecolor{currentfill}{rgb}{0.000000,0.000000,0.000000}%
\pgfsetfillcolor{currentfill}%
\pgfsetlinewidth{0.501875pt}%
\definecolor{currentstroke}{rgb}{0.000000,0.000000,0.000000}%
\pgfsetstrokecolor{currentstroke}%
\pgfsetdash{}{0pt}%
\pgfsys@defobject{currentmarker}{\pgfqpoint{-0.020833in}{0.000000in}}{\pgfqpoint{-0.000000in}{0.000000in}}{%
\pgfpathmoveto{\pgfqpoint{-0.000000in}{0.000000in}}%
\pgfpathlineto{\pgfqpoint{-0.020833in}{0.000000in}}%
\pgfusepath{stroke,fill}%
}%
\begin{pgfscope}%
\pgfsys@transformshift{6.267353in}{0.713153in}%
\pgfsys@useobject{currentmarker}{}%
\end{pgfscope}%
\end{pgfscope}%
\begin{pgfscope}%
\pgfsetbuttcap%
\pgfsetroundjoin%
\definecolor{currentfill}{rgb}{0.000000,0.000000,0.000000}%
\pgfsetfillcolor{currentfill}%
\pgfsetlinewidth{0.501875pt}%
\definecolor{currentstroke}{rgb}{0.000000,0.000000,0.000000}%
\pgfsetstrokecolor{currentstroke}%
\pgfsetdash{}{0pt}%
\pgfsys@defobject{currentmarker}{\pgfqpoint{0.000000in}{0.000000in}}{\pgfqpoint{0.020833in}{0.000000in}}{%
\pgfpathmoveto{\pgfqpoint{0.000000in}{0.000000in}}%
\pgfpathlineto{\pgfqpoint{0.020833in}{0.000000in}}%
\pgfusepath{stroke,fill}%
}%
\begin{pgfscope}%
\pgfsys@transformshift{0.481681in}{0.766159in}%
\pgfsys@useobject{currentmarker}{}%
\end{pgfscope}%
\end{pgfscope}%
\begin{pgfscope}%
\pgfsetbuttcap%
\pgfsetroundjoin%
\definecolor{currentfill}{rgb}{0.000000,0.000000,0.000000}%
\pgfsetfillcolor{currentfill}%
\pgfsetlinewidth{0.501875pt}%
\definecolor{currentstroke}{rgb}{0.000000,0.000000,0.000000}%
\pgfsetstrokecolor{currentstroke}%
\pgfsetdash{}{0pt}%
\pgfsys@defobject{currentmarker}{\pgfqpoint{-0.020833in}{0.000000in}}{\pgfqpoint{-0.000000in}{0.000000in}}{%
\pgfpathmoveto{\pgfqpoint{-0.000000in}{0.000000in}}%
\pgfpathlineto{\pgfqpoint{-0.020833in}{0.000000in}}%
\pgfusepath{stroke,fill}%
}%
\begin{pgfscope}%
\pgfsys@transformshift{6.267353in}{0.766159in}%
\pgfsys@useobject{currentmarker}{}%
\end{pgfscope}%
\end{pgfscope}%
\begin{pgfscope}%
\pgfsetbuttcap%
\pgfsetroundjoin%
\definecolor{currentfill}{rgb}{0.000000,0.000000,0.000000}%
\pgfsetfillcolor{currentfill}%
\pgfsetlinewidth{0.501875pt}%
\definecolor{currentstroke}{rgb}{0.000000,0.000000,0.000000}%
\pgfsetstrokecolor{currentstroke}%
\pgfsetdash{}{0pt}%
\pgfsys@defobject{currentmarker}{\pgfqpoint{0.000000in}{0.000000in}}{\pgfqpoint{0.020833in}{0.000000in}}{%
\pgfpathmoveto{\pgfqpoint{0.000000in}{0.000000in}}%
\pgfpathlineto{\pgfqpoint{0.020833in}{0.000000in}}%
\pgfusepath{stroke,fill}%
}%
\begin{pgfscope}%
\pgfsys@transformshift{0.481681in}{0.792662in}%
\pgfsys@useobject{currentmarker}{}%
\end{pgfscope}%
\end{pgfscope}%
\begin{pgfscope}%
\pgfsetbuttcap%
\pgfsetroundjoin%
\definecolor{currentfill}{rgb}{0.000000,0.000000,0.000000}%
\pgfsetfillcolor{currentfill}%
\pgfsetlinewidth{0.501875pt}%
\definecolor{currentstroke}{rgb}{0.000000,0.000000,0.000000}%
\pgfsetstrokecolor{currentstroke}%
\pgfsetdash{}{0pt}%
\pgfsys@defobject{currentmarker}{\pgfqpoint{-0.020833in}{0.000000in}}{\pgfqpoint{-0.000000in}{0.000000in}}{%
\pgfpathmoveto{\pgfqpoint{-0.000000in}{0.000000in}}%
\pgfpathlineto{\pgfqpoint{-0.020833in}{0.000000in}}%
\pgfusepath{stroke,fill}%
}%
\begin{pgfscope}%
\pgfsys@transformshift{6.267353in}{0.792662in}%
\pgfsys@useobject{currentmarker}{}%
\end{pgfscope}%
\end{pgfscope}%
\begin{pgfscope}%
\pgfsetbuttcap%
\pgfsetroundjoin%
\definecolor{currentfill}{rgb}{0.000000,0.000000,0.000000}%
\pgfsetfillcolor{currentfill}%
\pgfsetlinewidth{0.501875pt}%
\definecolor{currentstroke}{rgb}{0.000000,0.000000,0.000000}%
\pgfsetstrokecolor{currentstroke}%
\pgfsetdash{}{0pt}%
\pgfsys@defobject{currentmarker}{\pgfqpoint{0.000000in}{0.000000in}}{\pgfqpoint{0.020833in}{0.000000in}}{%
\pgfpathmoveto{\pgfqpoint{0.000000in}{0.000000in}}%
\pgfpathlineto{\pgfqpoint{0.020833in}{0.000000in}}%
\pgfusepath{stroke,fill}%
}%
\begin{pgfscope}%
\pgfsys@transformshift{0.481681in}{0.819164in}%
\pgfsys@useobject{currentmarker}{}%
\end{pgfscope}%
\end{pgfscope}%
\begin{pgfscope}%
\pgfsetbuttcap%
\pgfsetroundjoin%
\definecolor{currentfill}{rgb}{0.000000,0.000000,0.000000}%
\pgfsetfillcolor{currentfill}%
\pgfsetlinewidth{0.501875pt}%
\definecolor{currentstroke}{rgb}{0.000000,0.000000,0.000000}%
\pgfsetstrokecolor{currentstroke}%
\pgfsetdash{}{0pt}%
\pgfsys@defobject{currentmarker}{\pgfqpoint{-0.020833in}{0.000000in}}{\pgfqpoint{-0.000000in}{0.000000in}}{%
\pgfpathmoveto{\pgfqpoint{-0.000000in}{0.000000in}}%
\pgfpathlineto{\pgfqpoint{-0.020833in}{0.000000in}}%
\pgfusepath{stroke,fill}%
}%
\begin{pgfscope}%
\pgfsys@transformshift{6.267353in}{0.819164in}%
\pgfsys@useobject{currentmarker}{}%
\end{pgfscope}%
\end{pgfscope}%
\begin{pgfscope}%
\pgfsetbuttcap%
\pgfsetroundjoin%
\definecolor{currentfill}{rgb}{0.000000,0.000000,0.000000}%
\pgfsetfillcolor{currentfill}%
\pgfsetlinewidth{0.501875pt}%
\definecolor{currentstroke}{rgb}{0.000000,0.000000,0.000000}%
\pgfsetstrokecolor{currentstroke}%
\pgfsetdash{}{0pt}%
\pgfsys@defobject{currentmarker}{\pgfqpoint{0.000000in}{0.000000in}}{\pgfqpoint{0.020833in}{0.000000in}}{%
\pgfpathmoveto{\pgfqpoint{0.000000in}{0.000000in}}%
\pgfpathlineto{\pgfqpoint{0.020833in}{0.000000in}}%
\pgfusepath{stroke,fill}%
}%
\begin{pgfscope}%
\pgfsys@transformshift{0.481681in}{0.845667in}%
\pgfsys@useobject{currentmarker}{}%
\end{pgfscope}%
\end{pgfscope}%
\begin{pgfscope}%
\pgfsetbuttcap%
\pgfsetroundjoin%
\definecolor{currentfill}{rgb}{0.000000,0.000000,0.000000}%
\pgfsetfillcolor{currentfill}%
\pgfsetlinewidth{0.501875pt}%
\definecolor{currentstroke}{rgb}{0.000000,0.000000,0.000000}%
\pgfsetstrokecolor{currentstroke}%
\pgfsetdash{}{0pt}%
\pgfsys@defobject{currentmarker}{\pgfqpoint{-0.020833in}{0.000000in}}{\pgfqpoint{-0.000000in}{0.000000in}}{%
\pgfpathmoveto{\pgfqpoint{-0.000000in}{0.000000in}}%
\pgfpathlineto{\pgfqpoint{-0.020833in}{0.000000in}}%
\pgfusepath{stroke,fill}%
}%
\begin{pgfscope}%
\pgfsys@transformshift{6.267353in}{0.845667in}%
\pgfsys@useobject{currentmarker}{}%
\end{pgfscope}%
\end{pgfscope}%
\begin{pgfscope}%
\definecolor{textcolor}{rgb}{0.000000,0.000000,0.000000}%
\pgfsetstrokecolor{textcolor}%
\pgfsetfillcolor{textcolor}%
\pgftext[x=0.235389in,y=0.739656in,,bottom,rotate=90.000000]{\color{textcolor}\rmfamily\fontsize{8.000000}{9.600000}\selectfont Unterschied}%
\end{pgfscope}%
\begin{pgfscope}%
\pgfsetrectcap%
\pgfsetmiterjoin%
\pgfsetlinewidth{0.501875pt}%
\definecolor{currentstroke}{rgb}{0.000000,0.000000,0.000000}%
\pgfsetstrokecolor{currentstroke}%
\pgfsetdash{}{0pt}%
\pgfpathmoveto{\pgfqpoint{0.481681in}{0.586309in}}%
\pgfpathlineto{\pgfqpoint{0.481681in}{0.893003in}}%
\pgfusepath{stroke}%
\end{pgfscope}%
\begin{pgfscope}%
\pgfsetrectcap%
\pgfsetmiterjoin%
\pgfsetlinewidth{0.501875pt}%
\definecolor{currentstroke}{rgb}{0.000000,0.000000,0.000000}%
\pgfsetstrokecolor{currentstroke}%
\pgfsetdash{}{0pt}%
\pgfpathmoveto{\pgfqpoint{6.267353in}{0.586309in}}%
\pgfpathlineto{\pgfqpoint{6.267353in}{0.893003in}}%
\pgfusepath{stroke}%
\end{pgfscope}%
\begin{pgfscope}%
\pgfsetrectcap%
\pgfsetmiterjoin%
\pgfsetlinewidth{0.501875pt}%
\definecolor{currentstroke}{rgb}{0.000000,0.000000,0.000000}%
\pgfsetstrokecolor{currentstroke}%
\pgfsetdash{}{0pt}%
\pgfpathmoveto{\pgfqpoint{0.481681in}{0.586309in}}%
\pgfpathlineto{\pgfqpoint{6.267353in}{0.586309in}}%
\pgfusepath{stroke}%
\end{pgfscope}%
\begin{pgfscope}%
\pgfsetrectcap%
\pgfsetmiterjoin%
\pgfsetlinewidth{0.501875pt}%
\definecolor{currentstroke}{rgb}{0.000000,0.000000,0.000000}%
\pgfsetstrokecolor{currentstroke}%
\pgfsetdash{}{0pt}%
\pgfpathmoveto{\pgfqpoint{0.481681in}{0.893003in}}%
\pgfpathlineto{\pgfqpoint{6.267353in}{0.893003in}}%
\pgfusepath{stroke}%
\end{pgfscope}%
\end{pgfpicture}%
\makeatother%
\endgroup%

  \end{center}
  \caption{Verhältnis von Aufwand und Komplexität im inGRID Projekt}
\end{figure}

In dem Graph sind die Story Points, logischen Codezeilen, die
zyklomatische Komplexität und die Einrückungskomplexität genau wie in
dem NDA Projekt (siehe \ref{nda-Auswertung}) abgezeichnet. Die Differenz zwischen dem
Durchschnitt aller Codemetriken und den Storypoint Abschätzungen findet
sich, wie auch in dem ersten Projekt in einem separaten Diagramm unten
in der Grafik.

Zunächst verlaufen die Storypoint-Abschätzungen nahe bei den
Codekomplexitätsmetriken. Bis Ende Juni 2020 sind Abweichungen von 20
bis 30 Prozent ersichtlich. Anfang Juli 2020 ist in allen
Codekomplexitätsmetriken ein plötzlicher Abfall der Werte zu beobachten.
Dieser Abfall ist in den Aufwandsabschätzungen nicht vermerkt. In
Zusammenarbeit mit dem Entwickler konnte eine Migration der
Entwicklungsumgebung als möglicher Grund hierfür identifiziert
werden\footnote{Analyse Tabelle Zeile 38, Spalte K TODO}. Bis Januar 2021
fallen in den Codemetriken keine Veränderungen auf. Gleichzeitig steigen
die Komplexitätsabschätzungen in diesem Zeitraum aber kontinuierlich
weiter. Eine Erklärung für diese Abweichung konnte nicht gefunden
werden. Für die plötzlichen Anstiege und Abfälle der
Codekomplexitätsmetriken von Januar 2021 bis März 2021 konnte ein
Handover zwischen zwei Entwicklern als Erklärung gefunden werden. Von
März bis August 2021 bleiben sowohl die Komplexitätsmessungen, als auch
die Aufwandsabschätzungen grö\ss tenteils stabil. Diese Messdaten sprechen
also für eine Korrelation. Ab August bis zum Ende der Messdaten im
November 2021 lassen sich Anstiege in den Codekomplexitätsmetriken
beobachten. Hier wurden bereits Funktionen in der Anwendung entwickelt,
die aber noch nicht als geschlossene Storys in der
Projektmanagementsoftware vermerkt sind.

Insgesamt verliefen in diesem Projekt die Aufwandsabschätzungen nur in
Teilen ähnlich wie die Codekomplexitätsmetriken. Für die Abweichungen
konnten nicht an allen Stellen Erklärungen gefunden werden. Es ist also
zu erwarten, dass die berechneten Korrelationskoeffizienten nur eine
geringe Korrelation anzeigen.

TODO

Diese Erwartung lie\ss  sich durch eine Berechnung der
Korrelationskoeffizienten bestätigen. Für die Komplexitätsma\ss en der
logischen Codezeilen und der Einrückungskomplexität konnten über die
Koeffizienten hinweg nur Werte zwischen 0,64 und 0,43 erreicht werden.
Diese Werte schlie\ss en zwar keine Korrelation aus, sprechen aber deutlich
gegen eine Existenz dieser. Für die Komplexitätsma\ss en der zyklomatischen
Komplexität und des Halstead Aufwandes konnten deutlich höhere Werte
zwischen 0,69 und 0,92 erreicht werden. Bei diesen Komplexitätsma\ss en ist
in diesem Projekt also eine Korrelation mit den
Storypoint-Aufwandsabschätzungen durchaus anzunehmen. Als Erklärung für
die Abweichung zwischen den Komplexitätsma\ss en kann ihre
Berechnungsmethodik vermutet werden. Sowohl die Ma\ss zahl der logischen
Codezeilen als auch die der Einrückungskomplexität orientieren sich, wie
in 2.3 erklärt, zumindest in Teilen an den Zeilen im Sourcecode. Bei den
anderen beiden Ma\ss en spielt die Aufteilung des Codes auf Codezeilen
keine Rolle. Hier wird lediglich der Inhalt des Codes betrachtet. Also
könnte die Hypothese aufgestellt werden, dass eine Aufteilung oder
nicht-Aufteilung des Quelltextes auf Codezeilen Einfluss auf die Ma\ss zahl
der logischen Codezeilen und der Einrückungskomplexität genommen haben
könnte.

\subsection{Fazit}\label{ingrid-fazit}

Insgesamt konnte in diesem Projekt nur eine eingeschränkte Korrelation
zwischen den Codekomplexitätsmetriken und den Aufwandsabschätzungen
nachgewiesen werden. Für die Metriken der zyklomatischen Komplexität und
die von Halstead konnten starke Korrelationen nachgewiesen werden. Hier
spricht das Projekt also für eine generelle Korrelation. Bei den anderen
beiden Metriken konnten aber nur schwache Korrelationen festgestellt
werden, was wiederum gegen eine generelle Korrelation spricht. Insgesamt
schwächt dieses Projekt also die Hypothese, dass
Codekomplexitätsmetriken und Aufwandsabschätzungen zusammenhängen.

\subsection{Kritik}\label{ingrid-kritik}

In diesem Projekt wurden grö\ss ere Abweichungen zwischen den Story Point
Abschätzungen und den Codemetriken festgestellt. Die Existenz dieser
Abweichungen deutet zunächst daraufhin, dass ein Fehler in der Messung
existieren könnte. Des Weiteren konnten nicht für alle Abweichungen
Erklärungen gefunden werden. Das könnte bedeuten, dass das Interview mit
dem Entwickler nicht ausreichend umfangreich durchgeführt wurde.
Insgesamt sollte die Validität der Ergebnisse dieses Falls stark
hinterfragt werden.

\section{Alstonii}\label{alstonii}

Der zweite Fall dieser Fallstudie befasst sich mit dem Alstonii
Projekt. Mit dem Alstonii Projekt soll ein mobiles Frontend für
Vertriebsmitarbeiter eines gro\ss en Telekommunikationsanbieters geschaffen
werden. Es soll diesen Mitarbeitern bei dem Verkauf von Breitband
Internetanbindungen an Privat- und Firmenkunden unterstützen. Als
digitales System soll das Alstonii Projekt einen analogen Prozess auf Basis
von Papierformularen ersetzen. Ziel ist die Unterstützung der gleichen
Funktionalität wie der analoge Prozess.

Das Projekt wird agil nach Scrum entwickelt und in mehreren
Programmiersprachen umgesetzt. Dabei kommen zu 82\% Vue, zu 10\%
TypeScript und zusätzlich (\textless5\%) noch \ac{HTML}, JavaScript, Python,
CSS, SCSS, Shell und Batch zum Einsatz. Realisiert
wird die Anwendung hauptsächlich von einem Nearshore Team in Bulgarien.
Unterstützt wird dieses Team von einem Offshore Team in Indien und
einigen Onsite Mitarbeitern. Insgesamt arbeiten so zwischen sechs und
neun Mitarbeiter an dem Projekt. Hauptsächlich kommen erfahrene
Mitarbeiter zum Einsatz, wobei die Erfahrung der Mitarbeiter in dem
Projekt trotz der langen Laufzeit als gering einzustufen ist. Das lässt
sich auf eine hohe Mitarbeiterfluktuation zurückführen. Das aktuelle
Entwicklungsteam ist seit sechs Monaten in dem Projekt.

\subsection{Datenerhebung}\label{Alstonii-Datenerhebung}

Auch in diesem Projekt konnten die Aufwandsabschätzungen aus der
Projektmanagementsoftware Jira exportiert werden. Nach den in Kapitel
4.4 definierten Kriterien konnten 215 Datensätze identifiziert werden.
Diese wurden als CSV-Datei exportiert und in die Analysesoftware
eingelesen.

Als Quelle für die Komplexitätsma\ss zahlen wurde das GitHub Repository der
Software identifiziert. Es konnte ein Zugang beantragt werden und die
Daten konnten über eine \ac{SSH} Verbindung in eine lokale Kopie geladen
werden. Diese Kopie wurde dann von der Analysesoftware gelesen.

\subsection{Auswertung}\label{Alstonii-Auswertung}

In der Auswertung werden alle Daten nun miteinander verbunden, um so
einen Schluss auf die Hypothese zu erlangen. Die Relation der
Aufwandsabschätzungen mit den Komplexitätsmetriken ist in Abbildung TODO
erkenntlich.

\begin{figure}\label{alstonii-graph}
  \begin{center}
      %% Creator: Matplotlib, PGF backend
%%
%% To include the figure in your LaTeX document, write
%%   \input{<filename>.pgf}
%%
%% Make sure the required packages are loaded in your preamble
%%   \usepackage{pgf}
%%
%% Also ensure that all the required font packages are loaded; for instance,
%% the lmodern package is sometimes necessary when using math font.
%%   \usepackage{lmodern}
%%
%% Figures using additional raster images can only be included by \input if
%% they are in the same directory as the main LaTeX file. For loading figures
%% from other directories you can use the `import` package
%%   \usepackage{import}
%%
%% and then include the figures with
%%   \import{<path to file>}{<filename>.pgf}
%%
%% Matplotlib used the following preamble
%%   \usepackage{fontspec}
%%
\begingroup%
\makeatletter%
\begin{pgfpicture}%
\pgfpathrectangle{\pgfpointorigin}{\pgfqpoint{6.317353in}{3.277753in}}%
\pgfusepath{use as bounding box, clip}%
\begin{pgfscope}%
\pgfsetbuttcap%
\pgfsetmiterjoin%
\definecolor{currentfill}{rgb}{1.000000,1.000000,1.000000}%
\pgfsetfillcolor{currentfill}%
\pgfsetlinewidth{0.000000pt}%
\definecolor{currentstroke}{rgb}{1.000000,1.000000,1.000000}%
\pgfsetstrokecolor{currentstroke}%
\pgfsetdash{}{0pt}%
\pgfpathmoveto{\pgfqpoint{0.000000in}{-0.000000in}}%
\pgfpathlineto{\pgfqpoint{6.317353in}{-0.000000in}}%
\pgfpathlineto{\pgfqpoint{6.317353in}{3.277753in}}%
\pgfpathlineto{\pgfqpoint{0.000000in}{3.277753in}}%
\pgfpathlineto{\pgfqpoint{0.000000in}{-0.000000in}}%
\pgfpathclose%
\pgfusepath{fill}%
\end{pgfscope}%
\begin{pgfscope}%
\pgfsetbuttcap%
\pgfsetmiterjoin%
\definecolor{currentfill}{rgb}{1.000000,1.000000,1.000000}%
\pgfsetfillcolor{currentfill}%
\pgfsetlinewidth{0.000000pt}%
\definecolor{currentstroke}{rgb}{0.000000,0.000000,0.000000}%
\pgfsetstrokecolor{currentstroke}%
\pgfsetstrokeopacity{0.000000}%
\pgfsetdash{}{0pt}%
\pgfpathmoveto{\pgfqpoint{0.481681in}{1.080890in}}%
\pgfpathlineto{\pgfqpoint{6.267353in}{1.080890in}}%
\pgfpathlineto{\pgfqpoint{6.267353in}{3.227753in}}%
\pgfpathlineto{\pgfqpoint{0.481681in}{3.227753in}}%
\pgfpathlineto{\pgfqpoint{0.481681in}{1.080890in}}%
\pgfpathclose%
\pgfusepath{fill}%
\end{pgfscope}%
\begin{pgfscope}%
\pgfpathrectangle{\pgfqpoint{0.481681in}{1.080890in}}{\pgfqpoint{5.785672in}{2.146863in}}%
\pgfusepath{clip}%
\pgfsetrectcap%
\pgfsetroundjoin%
\pgfsetlinewidth{0.100375pt}%
\definecolor{currentstroke}{rgb}{0.501961,0.501961,0.501961}%
\pgfsetstrokecolor{currentstroke}%
\pgfsetdash{}{0pt}%
\pgfpathmoveto{\pgfqpoint{0.643728in}{1.080890in}}%
\pgfpathlineto{\pgfqpoint{0.643728in}{3.227753in}}%
\pgfusepath{stroke}%
\end{pgfscope}%
\begin{pgfscope}%
\pgfsetbuttcap%
\pgfsetroundjoin%
\definecolor{currentfill}{rgb}{0.000000,0.000000,0.000000}%
\pgfsetfillcolor{currentfill}%
\pgfsetlinewidth{0.501875pt}%
\definecolor{currentstroke}{rgb}{0.000000,0.000000,0.000000}%
\pgfsetstrokecolor{currentstroke}%
\pgfsetdash{}{0pt}%
\pgfsys@defobject{currentmarker}{\pgfqpoint{0.000000in}{0.000000in}}{\pgfqpoint{0.000000in}{0.041667in}}{%
\pgfpathmoveto{\pgfqpoint{0.000000in}{0.000000in}}%
\pgfpathlineto{\pgfqpoint{0.000000in}{0.041667in}}%
\pgfusepath{stroke,fill}%
}%
\begin{pgfscope}%
\pgfsys@transformshift{0.643728in}{1.080890in}%
\pgfsys@useobject{currentmarker}{}%
\end{pgfscope}%
\end{pgfscope}%
\begin{pgfscope}%
\pgfsetbuttcap%
\pgfsetroundjoin%
\definecolor{currentfill}{rgb}{0.000000,0.000000,0.000000}%
\pgfsetfillcolor{currentfill}%
\pgfsetlinewidth{0.501875pt}%
\definecolor{currentstroke}{rgb}{0.000000,0.000000,0.000000}%
\pgfsetstrokecolor{currentstroke}%
\pgfsetdash{}{0pt}%
\pgfsys@defobject{currentmarker}{\pgfqpoint{0.000000in}{-0.041667in}}{\pgfqpoint{0.000000in}{0.000000in}}{%
\pgfpathmoveto{\pgfqpoint{0.000000in}{0.000000in}}%
\pgfpathlineto{\pgfqpoint{0.000000in}{-0.041667in}}%
\pgfusepath{stroke,fill}%
}%
\begin{pgfscope}%
\pgfsys@transformshift{0.643728in}{3.227753in}%
\pgfsys@useobject{currentmarker}{}%
\end{pgfscope}%
\end{pgfscope}%
\begin{pgfscope}%
\pgfpathrectangle{\pgfqpoint{0.481681in}{1.080890in}}{\pgfqpoint{5.785672in}{2.146863in}}%
\pgfusepath{clip}%
\pgfsetrectcap%
\pgfsetroundjoin%
\pgfsetlinewidth{0.100375pt}%
\definecolor{currentstroke}{rgb}{0.501961,0.501961,0.501961}%
\pgfsetstrokecolor{currentstroke}%
\pgfsetdash{}{0pt}%
\pgfpathmoveto{\pgfqpoint{1.069975in}{1.080890in}}%
\pgfpathlineto{\pgfqpoint{1.069975in}{3.227753in}}%
\pgfusepath{stroke}%
\end{pgfscope}%
\begin{pgfscope}%
\pgfsetbuttcap%
\pgfsetroundjoin%
\definecolor{currentfill}{rgb}{0.000000,0.000000,0.000000}%
\pgfsetfillcolor{currentfill}%
\pgfsetlinewidth{0.501875pt}%
\definecolor{currentstroke}{rgb}{0.000000,0.000000,0.000000}%
\pgfsetstrokecolor{currentstroke}%
\pgfsetdash{}{0pt}%
\pgfsys@defobject{currentmarker}{\pgfqpoint{0.000000in}{0.000000in}}{\pgfqpoint{0.000000in}{0.041667in}}{%
\pgfpathmoveto{\pgfqpoint{0.000000in}{0.000000in}}%
\pgfpathlineto{\pgfqpoint{0.000000in}{0.041667in}}%
\pgfusepath{stroke,fill}%
}%
\begin{pgfscope}%
\pgfsys@transformshift{1.069975in}{1.080890in}%
\pgfsys@useobject{currentmarker}{}%
\end{pgfscope}%
\end{pgfscope}%
\begin{pgfscope}%
\pgfsetbuttcap%
\pgfsetroundjoin%
\definecolor{currentfill}{rgb}{0.000000,0.000000,0.000000}%
\pgfsetfillcolor{currentfill}%
\pgfsetlinewidth{0.501875pt}%
\definecolor{currentstroke}{rgb}{0.000000,0.000000,0.000000}%
\pgfsetstrokecolor{currentstroke}%
\pgfsetdash{}{0pt}%
\pgfsys@defobject{currentmarker}{\pgfqpoint{0.000000in}{-0.041667in}}{\pgfqpoint{0.000000in}{0.000000in}}{%
\pgfpathmoveto{\pgfqpoint{0.000000in}{0.000000in}}%
\pgfpathlineto{\pgfqpoint{0.000000in}{-0.041667in}}%
\pgfusepath{stroke,fill}%
}%
\begin{pgfscope}%
\pgfsys@transformshift{1.069975in}{3.227753in}%
\pgfsys@useobject{currentmarker}{}%
\end{pgfscope}%
\end{pgfscope}%
\begin{pgfscope}%
\pgfpathrectangle{\pgfqpoint{0.481681in}{1.080890in}}{\pgfqpoint{5.785672in}{2.146863in}}%
\pgfusepath{clip}%
\pgfsetrectcap%
\pgfsetroundjoin%
\pgfsetlinewidth{0.100375pt}%
\definecolor{currentstroke}{rgb}{0.501961,0.501961,0.501961}%
\pgfsetstrokecolor{currentstroke}%
\pgfsetdash{}{0pt}%
\pgfpathmoveto{\pgfqpoint{1.496221in}{1.080890in}}%
\pgfpathlineto{\pgfqpoint{1.496221in}{3.227753in}}%
\pgfusepath{stroke}%
\end{pgfscope}%
\begin{pgfscope}%
\pgfsetbuttcap%
\pgfsetroundjoin%
\definecolor{currentfill}{rgb}{0.000000,0.000000,0.000000}%
\pgfsetfillcolor{currentfill}%
\pgfsetlinewidth{0.501875pt}%
\definecolor{currentstroke}{rgb}{0.000000,0.000000,0.000000}%
\pgfsetstrokecolor{currentstroke}%
\pgfsetdash{}{0pt}%
\pgfsys@defobject{currentmarker}{\pgfqpoint{0.000000in}{0.000000in}}{\pgfqpoint{0.000000in}{0.041667in}}{%
\pgfpathmoveto{\pgfqpoint{0.000000in}{0.000000in}}%
\pgfpathlineto{\pgfqpoint{0.000000in}{0.041667in}}%
\pgfusepath{stroke,fill}%
}%
\begin{pgfscope}%
\pgfsys@transformshift{1.496221in}{1.080890in}%
\pgfsys@useobject{currentmarker}{}%
\end{pgfscope}%
\end{pgfscope}%
\begin{pgfscope}%
\pgfsetbuttcap%
\pgfsetroundjoin%
\definecolor{currentfill}{rgb}{0.000000,0.000000,0.000000}%
\pgfsetfillcolor{currentfill}%
\pgfsetlinewidth{0.501875pt}%
\definecolor{currentstroke}{rgb}{0.000000,0.000000,0.000000}%
\pgfsetstrokecolor{currentstroke}%
\pgfsetdash{}{0pt}%
\pgfsys@defobject{currentmarker}{\pgfqpoint{0.000000in}{-0.041667in}}{\pgfqpoint{0.000000in}{0.000000in}}{%
\pgfpathmoveto{\pgfqpoint{0.000000in}{0.000000in}}%
\pgfpathlineto{\pgfqpoint{0.000000in}{-0.041667in}}%
\pgfusepath{stroke,fill}%
}%
\begin{pgfscope}%
\pgfsys@transformshift{1.496221in}{3.227753in}%
\pgfsys@useobject{currentmarker}{}%
\end{pgfscope}%
\end{pgfscope}%
\begin{pgfscope}%
\pgfpathrectangle{\pgfqpoint{0.481681in}{1.080890in}}{\pgfqpoint{5.785672in}{2.146863in}}%
\pgfusepath{clip}%
\pgfsetrectcap%
\pgfsetroundjoin%
\pgfsetlinewidth{0.100375pt}%
\definecolor{currentstroke}{rgb}{0.501961,0.501961,0.501961}%
\pgfsetstrokecolor{currentstroke}%
\pgfsetdash{}{0pt}%
\pgfpathmoveto{\pgfqpoint{1.922467in}{1.080890in}}%
\pgfpathlineto{\pgfqpoint{1.922467in}{3.227753in}}%
\pgfusepath{stroke}%
\end{pgfscope}%
\begin{pgfscope}%
\pgfsetbuttcap%
\pgfsetroundjoin%
\definecolor{currentfill}{rgb}{0.000000,0.000000,0.000000}%
\pgfsetfillcolor{currentfill}%
\pgfsetlinewidth{0.501875pt}%
\definecolor{currentstroke}{rgb}{0.000000,0.000000,0.000000}%
\pgfsetstrokecolor{currentstroke}%
\pgfsetdash{}{0pt}%
\pgfsys@defobject{currentmarker}{\pgfqpoint{0.000000in}{0.000000in}}{\pgfqpoint{0.000000in}{0.041667in}}{%
\pgfpathmoveto{\pgfqpoint{0.000000in}{0.000000in}}%
\pgfpathlineto{\pgfqpoint{0.000000in}{0.041667in}}%
\pgfusepath{stroke,fill}%
}%
\begin{pgfscope}%
\pgfsys@transformshift{1.922467in}{1.080890in}%
\pgfsys@useobject{currentmarker}{}%
\end{pgfscope}%
\end{pgfscope}%
\begin{pgfscope}%
\pgfsetbuttcap%
\pgfsetroundjoin%
\definecolor{currentfill}{rgb}{0.000000,0.000000,0.000000}%
\pgfsetfillcolor{currentfill}%
\pgfsetlinewidth{0.501875pt}%
\definecolor{currentstroke}{rgb}{0.000000,0.000000,0.000000}%
\pgfsetstrokecolor{currentstroke}%
\pgfsetdash{}{0pt}%
\pgfsys@defobject{currentmarker}{\pgfqpoint{0.000000in}{-0.041667in}}{\pgfqpoint{0.000000in}{0.000000in}}{%
\pgfpathmoveto{\pgfqpoint{0.000000in}{0.000000in}}%
\pgfpathlineto{\pgfqpoint{0.000000in}{-0.041667in}}%
\pgfusepath{stroke,fill}%
}%
\begin{pgfscope}%
\pgfsys@transformshift{1.922467in}{3.227753in}%
\pgfsys@useobject{currentmarker}{}%
\end{pgfscope}%
\end{pgfscope}%
\begin{pgfscope}%
\pgfpathrectangle{\pgfqpoint{0.481681in}{1.080890in}}{\pgfqpoint{5.785672in}{2.146863in}}%
\pgfusepath{clip}%
\pgfsetrectcap%
\pgfsetroundjoin%
\pgfsetlinewidth{0.100375pt}%
\definecolor{currentstroke}{rgb}{0.501961,0.501961,0.501961}%
\pgfsetstrokecolor{currentstroke}%
\pgfsetdash{}{0pt}%
\pgfpathmoveto{\pgfqpoint{2.348714in}{1.080890in}}%
\pgfpathlineto{\pgfqpoint{2.348714in}{3.227753in}}%
\pgfusepath{stroke}%
\end{pgfscope}%
\begin{pgfscope}%
\pgfsetbuttcap%
\pgfsetroundjoin%
\definecolor{currentfill}{rgb}{0.000000,0.000000,0.000000}%
\pgfsetfillcolor{currentfill}%
\pgfsetlinewidth{0.501875pt}%
\definecolor{currentstroke}{rgb}{0.000000,0.000000,0.000000}%
\pgfsetstrokecolor{currentstroke}%
\pgfsetdash{}{0pt}%
\pgfsys@defobject{currentmarker}{\pgfqpoint{0.000000in}{0.000000in}}{\pgfqpoint{0.000000in}{0.041667in}}{%
\pgfpathmoveto{\pgfqpoint{0.000000in}{0.000000in}}%
\pgfpathlineto{\pgfqpoint{0.000000in}{0.041667in}}%
\pgfusepath{stroke,fill}%
}%
\begin{pgfscope}%
\pgfsys@transformshift{2.348714in}{1.080890in}%
\pgfsys@useobject{currentmarker}{}%
\end{pgfscope}%
\end{pgfscope}%
\begin{pgfscope}%
\pgfsetbuttcap%
\pgfsetroundjoin%
\definecolor{currentfill}{rgb}{0.000000,0.000000,0.000000}%
\pgfsetfillcolor{currentfill}%
\pgfsetlinewidth{0.501875pt}%
\definecolor{currentstroke}{rgb}{0.000000,0.000000,0.000000}%
\pgfsetstrokecolor{currentstroke}%
\pgfsetdash{}{0pt}%
\pgfsys@defobject{currentmarker}{\pgfqpoint{0.000000in}{-0.041667in}}{\pgfqpoint{0.000000in}{0.000000in}}{%
\pgfpathmoveto{\pgfqpoint{0.000000in}{0.000000in}}%
\pgfpathlineto{\pgfqpoint{0.000000in}{-0.041667in}}%
\pgfusepath{stroke,fill}%
}%
\begin{pgfscope}%
\pgfsys@transformshift{2.348714in}{3.227753in}%
\pgfsys@useobject{currentmarker}{}%
\end{pgfscope}%
\end{pgfscope}%
\begin{pgfscope}%
\pgfpathrectangle{\pgfqpoint{0.481681in}{1.080890in}}{\pgfqpoint{5.785672in}{2.146863in}}%
\pgfusepath{clip}%
\pgfsetrectcap%
\pgfsetroundjoin%
\pgfsetlinewidth{0.100375pt}%
\definecolor{currentstroke}{rgb}{0.501961,0.501961,0.501961}%
\pgfsetstrokecolor{currentstroke}%
\pgfsetdash{}{0pt}%
\pgfpathmoveto{\pgfqpoint{2.774960in}{1.080890in}}%
\pgfpathlineto{\pgfqpoint{2.774960in}{3.227753in}}%
\pgfusepath{stroke}%
\end{pgfscope}%
\begin{pgfscope}%
\pgfsetbuttcap%
\pgfsetroundjoin%
\definecolor{currentfill}{rgb}{0.000000,0.000000,0.000000}%
\pgfsetfillcolor{currentfill}%
\pgfsetlinewidth{0.501875pt}%
\definecolor{currentstroke}{rgb}{0.000000,0.000000,0.000000}%
\pgfsetstrokecolor{currentstroke}%
\pgfsetdash{}{0pt}%
\pgfsys@defobject{currentmarker}{\pgfqpoint{0.000000in}{0.000000in}}{\pgfqpoint{0.000000in}{0.041667in}}{%
\pgfpathmoveto{\pgfqpoint{0.000000in}{0.000000in}}%
\pgfpathlineto{\pgfqpoint{0.000000in}{0.041667in}}%
\pgfusepath{stroke,fill}%
}%
\begin{pgfscope}%
\pgfsys@transformshift{2.774960in}{1.080890in}%
\pgfsys@useobject{currentmarker}{}%
\end{pgfscope}%
\end{pgfscope}%
\begin{pgfscope}%
\pgfsetbuttcap%
\pgfsetroundjoin%
\definecolor{currentfill}{rgb}{0.000000,0.000000,0.000000}%
\pgfsetfillcolor{currentfill}%
\pgfsetlinewidth{0.501875pt}%
\definecolor{currentstroke}{rgb}{0.000000,0.000000,0.000000}%
\pgfsetstrokecolor{currentstroke}%
\pgfsetdash{}{0pt}%
\pgfsys@defobject{currentmarker}{\pgfqpoint{0.000000in}{-0.041667in}}{\pgfqpoint{0.000000in}{0.000000in}}{%
\pgfpathmoveto{\pgfqpoint{0.000000in}{0.000000in}}%
\pgfpathlineto{\pgfqpoint{0.000000in}{-0.041667in}}%
\pgfusepath{stroke,fill}%
}%
\begin{pgfscope}%
\pgfsys@transformshift{2.774960in}{3.227753in}%
\pgfsys@useobject{currentmarker}{}%
\end{pgfscope}%
\end{pgfscope}%
\begin{pgfscope}%
\pgfpathrectangle{\pgfqpoint{0.481681in}{1.080890in}}{\pgfqpoint{5.785672in}{2.146863in}}%
\pgfusepath{clip}%
\pgfsetrectcap%
\pgfsetroundjoin%
\pgfsetlinewidth{0.100375pt}%
\definecolor{currentstroke}{rgb}{0.501961,0.501961,0.501961}%
\pgfsetstrokecolor{currentstroke}%
\pgfsetdash{}{0pt}%
\pgfpathmoveto{\pgfqpoint{3.201206in}{1.080890in}}%
\pgfpathlineto{\pgfqpoint{3.201206in}{3.227753in}}%
\pgfusepath{stroke}%
\end{pgfscope}%
\begin{pgfscope}%
\pgfsetbuttcap%
\pgfsetroundjoin%
\definecolor{currentfill}{rgb}{0.000000,0.000000,0.000000}%
\pgfsetfillcolor{currentfill}%
\pgfsetlinewidth{0.501875pt}%
\definecolor{currentstroke}{rgb}{0.000000,0.000000,0.000000}%
\pgfsetstrokecolor{currentstroke}%
\pgfsetdash{}{0pt}%
\pgfsys@defobject{currentmarker}{\pgfqpoint{0.000000in}{0.000000in}}{\pgfqpoint{0.000000in}{0.041667in}}{%
\pgfpathmoveto{\pgfqpoint{0.000000in}{0.000000in}}%
\pgfpathlineto{\pgfqpoint{0.000000in}{0.041667in}}%
\pgfusepath{stroke,fill}%
}%
\begin{pgfscope}%
\pgfsys@transformshift{3.201206in}{1.080890in}%
\pgfsys@useobject{currentmarker}{}%
\end{pgfscope}%
\end{pgfscope}%
\begin{pgfscope}%
\pgfsetbuttcap%
\pgfsetroundjoin%
\definecolor{currentfill}{rgb}{0.000000,0.000000,0.000000}%
\pgfsetfillcolor{currentfill}%
\pgfsetlinewidth{0.501875pt}%
\definecolor{currentstroke}{rgb}{0.000000,0.000000,0.000000}%
\pgfsetstrokecolor{currentstroke}%
\pgfsetdash{}{0pt}%
\pgfsys@defobject{currentmarker}{\pgfqpoint{0.000000in}{-0.041667in}}{\pgfqpoint{0.000000in}{0.000000in}}{%
\pgfpathmoveto{\pgfqpoint{0.000000in}{0.000000in}}%
\pgfpathlineto{\pgfqpoint{0.000000in}{-0.041667in}}%
\pgfusepath{stroke,fill}%
}%
\begin{pgfscope}%
\pgfsys@transformshift{3.201206in}{3.227753in}%
\pgfsys@useobject{currentmarker}{}%
\end{pgfscope}%
\end{pgfscope}%
\begin{pgfscope}%
\pgfpathrectangle{\pgfqpoint{0.481681in}{1.080890in}}{\pgfqpoint{5.785672in}{2.146863in}}%
\pgfusepath{clip}%
\pgfsetrectcap%
\pgfsetroundjoin%
\pgfsetlinewidth{0.100375pt}%
\definecolor{currentstroke}{rgb}{0.501961,0.501961,0.501961}%
\pgfsetstrokecolor{currentstroke}%
\pgfsetdash{}{0pt}%
\pgfpathmoveto{\pgfqpoint{3.627453in}{1.080890in}}%
\pgfpathlineto{\pgfqpoint{3.627453in}{3.227753in}}%
\pgfusepath{stroke}%
\end{pgfscope}%
\begin{pgfscope}%
\pgfsetbuttcap%
\pgfsetroundjoin%
\definecolor{currentfill}{rgb}{0.000000,0.000000,0.000000}%
\pgfsetfillcolor{currentfill}%
\pgfsetlinewidth{0.501875pt}%
\definecolor{currentstroke}{rgb}{0.000000,0.000000,0.000000}%
\pgfsetstrokecolor{currentstroke}%
\pgfsetdash{}{0pt}%
\pgfsys@defobject{currentmarker}{\pgfqpoint{0.000000in}{0.000000in}}{\pgfqpoint{0.000000in}{0.041667in}}{%
\pgfpathmoveto{\pgfqpoint{0.000000in}{0.000000in}}%
\pgfpathlineto{\pgfqpoint{0.000000in}{0.041667in}}%
\pgfusepath{stroke,fill}%
}%
\begin{pgfscope}%
\pgfsys@transformshift{3.627453in}{1.080890in}%
\pgfsys@useobject{currentmarker}{}%
\end{pgfscope}%
\end{pgfscope}%
\begin{pgfscope}%
\pgfsetbuttcap%
\pgfsetroundjoin%
\definecolor{currentfill}{rgb}{0.000000,0.000000,0.000000}%
\pgfsetfillcolor{currentfill}%
\pgfsetlinewidth{0.501875pt}%
\definecolor{currentstroke}{rgb}{0.000000,0.000000,0.000000}%
\pgfsetstrokecolor{currentstroke}%
\pgfsetdash{}{0pt}%
\pgfsys@defobject{currentmarker}{\pgfqpoint{0.000000in}{-0.041667in}}{\pgfqpoint{0.000000in}{0.000000in}}{%
\pgfpathmoveto{\pgfqpoint{0.000000in}{0.000000in}}%
\pgfpathlineto{\pgfqpoint{0.000000in}{-0.041667in}}%
\pgfusepath{stroke,fill}%
}%
\begin{pgfscope}%
\pgfsys@transformshift{3.627453in}{3.227753in}%
\pgfsys@useobject{currentmarker}{}%
\end{pgfscope}%
\end{pgfscope}%
\begin{pgfscope}%
\pgfpathrectangle{\pgfqpoint{0.481681in}{1.080890in}}{\pgfqpoint{5.785672in}{2.146863in}}%
\pgfusepath{clip}%
\pgfsetrectcap%
\pgfsetroundjoin%
\pgfsetlinewidth{0.100375pt}%
\definecolor{currentstroke}{rgb}{0.501961,0.501961,0.501961}%
\pgfsetstrokecolor{currentstroke}%
\pgfsetdash{}{0pt}%
\pgfpathmoveto{\pgfqpoint{4.053699in}{1.080890in}}%
\pgfpathlineto{\pgfqpoint{4.053699in}{3.227753in}}%
\pgfusepath{stroke}%
\end{pgfscope}%
\begin{pgfscope}%
\pgfsetbuttcap%
\pgfsetroundjoin%
\definecolor{currentfill}{rgb}{0.000000,0.000000,0.000000}%
\pgfsetfillcolor{currentfill}%
\pgfsetlinewidth{0.501875pt}%
\definecolor{currentstroke}{rgb}{0.000000,0.000000,0.000000}%
\pgfsetstrokecolor{currentstroke}%
\pgfsetdash{}{0pt}%
\pgfsys@defobject{currentmarker}{\pgfqpoint{0.000000in}{0.000000in}}{\pgfqpoint{0.000000in}{0.041667in}}{%
\pgfpathmoveto{\pgfqpoint{0.000000in}{0.000000in}}%
\pgfpathlineto{\pgfqpoint{0.000000in}{0.041667in}}%
\pgfusepath{stroke,fill}%
}%
\begin{pgfscope}%
\pgfsys@transformshift{4.053699in}{1.080890in}%
\pgfsys@useobject{currentmarker}{}%
\end{pgfscope}%
\end{pgfscope}%
\begin{pgfscope}%
\pgfsetbuttcap%
\pgfsetroundjoin%
\definecolor{currentfill}{rgb}{0.000000,0.000000,0.000000}%
\pgfsetfillcolor{currentfill}%
\pgfsetlinewidth{0.501875pt}%
\definecolor{currentstroke}{rgb}{0.000000,0.000000,0.000000}%
\pgfsetstrokecolor{currentstroke}%
\pgfsetdash{}{0pt}%
\pgfsys@defobject{currentmarker}{\pgfqpoint{0.000000in}{-0.041667in}}{\pgfqpoint{0.000000in}{0.000000in}}{%
\pgfpathmoveto{\pgfqpoint{0.000000in}{0.000000in}}%
\pgfpathlineto{\pgfqpoint{0.000000in}{-0.041667in}}%
\pgfusepath{stroke,fill}%
}%
\begin{pgfscope}%
\pgfsys@transformshift{4.053699in}{3.227753in}%
\pgfsys@useobject{currentmarker}{}%
\end{pgfscope}%
\end{pgfscope}%
\begin{pgfscope}%
\pgfpathrectangle{\pgfqpoint{0.481681in}{1.080890in}}{\pgfqpoint{5.785672in}{2.146863in}}%
\pgfusepath{clip}%
\pgfsetrectcap%
\pgfsetroundjoin%
\pgfsetlinewidth{0.100375pt}%
\definecolor{currentstroke}{rgb}{0.501961,0.501961,0.501961}%
\pgfsetstrokecolor{currentstroke}%
\pgfsetdash{}{0pt}%
\pgfpathmoveto{\pgfqpoint{4.479945in}{1.080890in}}%
\pgfpathlineto{\pgfqpoint{4.479945in}{3.227753in}}%
\pgfusepath{stroke}%
\end{pgfscope}%
\begin{pgfscope}%
\pgfsetbuttcap%
\pgfsetroundjoin%
\definecolor{currentfill}{rgb}{0.000000,0.000000,0.000000}%
\pgfsetfillcolor{currentfill}%
\pgfsetlinewidth{0.501875pt}%
\definecolor{currentstroke}{rgb}{0.000000,0.000000,0.000000}%
\pgfsetstrokecolor{currentstroke}%
\pgfsetdash{}{0pt}%
\pgfsys@defobject{currentmarker}{\pgfqpoint{0.000000in}{0.000000in}}{\pgfqpoint{0.000000in}{0.041667in}}{%
\pgfpathmoveto{\pgfqpoint{0.000000in}{0.000000in}}%
\pgfpathlineto{\pgfqpoint{0.000000in}{0.041667in}}%
\pgfusepath{stroke,fill}%
}%
\begin{pgfscope}%
\pgfsys@transformshift{4.479945in}{1.080890in}%
\pgfsys@useobject{currentmarker}{}%
\end{pgfscope}%
\end{pgfscope}%
\begin{pgfscope}%
\pgfsetbuttcap%
\pgfsetroundjoin%
\definecolor{currentfill}{rgb}{0.000000,0.000000,0.000000}%
\pgfsetfillcolor{currentfill}%
\pgfsetlinewidth{0.501875pt}%
\definecolor{currentstroke}{rgb}{0.000000,0.000000,0.000000}%
\pgfsetstrokecolor{currentstroke}%
\pgfsetdash{}{0pt}%
\pgfsys@defobject{currentmarker}{\pgfqpoint{0.000000in}{-0.041667in}}{\pgfqpoint{0.000000in}{0.000000in}}{%
\pgfpathmoveto{\pgfqpoint{0.000000in}{0.000000in}}%
\pgfpathlineto{\pgfqpoint{0.000000in}{-0.041667in}}%
\pgfusepath{stroke,fill}%
}%
\begin{pgfscope}%
\pgfsys@transformshift{4.479945in}{3.227753in}%
\pgfsys@useobject{currentmarker}{}%
\end{pgfscope}%
\end{pgfscope}%
\begin{pgfscope}%
\pgfpathrectangle{\pgfqpoint{0.481681in}{1.080890in}}{\pgfqpoint{5.785672in}{2.146863in}}%
\pgfusepath{clip}%
\pgfsetrectcap%
\pgfsetroundjoin%
\pgfsetlinewidth{0.100375pt}%
\definecolor{currentstroke}{rgb}{0.501961,0.501961,0.501961}%
\pgfsetstrokecolor{currentstroke}%
\pgfsetdash{}{0pt}%
\pgfpathmoveto{\pgfqpoint{4.906192in}{1.080890in}}%
\pgfpathlineto{\pgfqpoint{4.906192in}{3.227753in}}%
\pgfusepath{stroke}%
\end{pgfscope}%
\begin{pgfscope}%
\pgfsetbuttcap%
\pgfsetroundjoin%
\definecolor{currentfill}{rgb}{0.000000,0.000000,0.000000}%
\pgfsetfillcolor{currentfill}%
\pgfsetlinewidth{0.501875pt}%
\definecolor{currentstroke}{rgb}{0.000000,0.000000,0.000000}%
\pgfsetstrokecolor{currentstroke}%
\pgfsetdash{}{0pt}%
\pgfsys@defobject{currentmarker}{\pgfqpoint{0.000000in}{0.000000in}}{\pgfqpoint{0.000000in}{0.041667in}}{%
\pgfpathmoveto{\pgfqpoint{0.000000in}{0.000000in}}%
\pgfpathlineto{\pgfqpoint{0.000000in}{0.041667in}}%
\pgfusepath{stroke,fill}%
}%
\begin{pgfscope}%
\pgfsys@transformshift{4.906192in}{1.080890in}%
\pgfsys@useobject{currentmarker}{}%
\end{pgfscope}%
\end{pgfscope}%
\begin{pgfscope}%
\pgfsetbuttcap%
\pgfsetroundjoin%
\definecolor{currentfill}{rgb}{0.000000,0.000000,0.000000}%
\pgfsetfillcolor{currentfill}%
\pgfsetlinewidth{0.501875pt}%
\definecolor{currentstroke}{rgb}{0.000000,0.000000,0.000000}%
\pgfsetstrokecolor{currentstroke}%
\pgfsetdash{}{0pt}%
\pgfsys@defobject{currentmarker}{\pgfqpoint{0.000000in}{-0.041667in}}{\pgfqpoint{0.000000in}{0.000000in}}{%
\pgfpathmoveto{\pgfqpoint{0.000000in}{0.000000in}}%
\pgfpathlineto{\pgfqpoint{0.000000in}{-0.041667in}}%
\pgfusepath{stroke,fill}%
}%
\begin{pgfscope}%
\pgfsys@transformshift{4.906192in}{3.227753in}%
\pgfsys@useobject{currentmarker}{}%
\end{pgfscope}%
\end{pgfscope}%
\begin{pgfscope}%
\pgfpathrectangle{\pgfqpoint{0.481681in}{1.080890in}}{\pgfqpoint{5.785672in}{2.146863in}}%
\pgfusepath{clip}%
\pgfsetrectcap%
\pgfsetroundjoin%
\pgfsetlinewidth{0.100375pt}%
\definecolor{currentstroke}{rgb}{0.501961,0.501961,0.501961}%
\pgfsetstrokecolor{currentstroke}%
\pgfsetdash{}{0pt}%
\pgfpathmoveto{\pgfqpoint{5.332438in}{1.080890in}}%
\pgfpathlineto{\pgfqpoint{5.332438in}{3.227753in}}%
\pgfusepath{stroke}%
\end{pgfscope}%
\begin{pgfscope}%
\pgfsetbuttcap%
\pgfsetroundjoin%
\definecolor{currentfill}{rgb}{0.000000,0.000000,0.000000}%
\pgfsetfillcolor{currentfill}%
\pgfsetlinewidth{0.501875pt}%
\definecolor{currentstroke}{rgb}{0.000000,0.000000,0.000000}%
\pgfsetstrokecolor{currentstroke}%
\pgfsetdash{}{0pt}%
\pgfsys@defobject{currentmarker}{\pgfqpoint{0.000000in}{0.000000in}}{\pgfqpoint{0.000000in}{0.041667in}}{%
\pgfpathmoveto{\pgfqpoint{0.000000in}{0.000000in}}%
\pgfpathlineto{\pgfqpoint{0.000000in}{0.041667in}}%
\pgfusepath{stroke,fill}%
}%
\begin{pgfscope}%
\pgfsys@transformshift{5.332438in}{1.080890in}%
\pgfsys@useobject{currentmarker}{}%
\end{pgfscope}%
\end{pgfscope}%
\begin{pgfscope}%
\pgfsetbuttcap%
\pgfsetroundjoin%
\definecolor{currentfill}{rgb}{0.000000,0.000000,0.000000}%
\pgfsetfillcolor{currentfill}%
\pgfsetlinewidth{0.501875pt}%
\definecolor{currentstroke}{rgb}{0.000000,0.000000,0.000000}%
\pgfsetstrokecolor{currentstroke}%
\pgfsetdash{}{0pt}%
\pgfsys@defobject{currentmarker}{\pgfqpoint{0.000000in}{-0.041667in}}{\pgfqpoint{0.000000in}{0.000000in}}{%
\pgfpathmoveto{\pgfqpoint{0.000000in}{0.000000in}}%
\pgfpathlineto{\pgfqpoint{0.000000in}{-0.041667in}}%
\pgfusepath{stroke,fill}%
}%
\begin{pgfscope}%
\pgfsys@transformshift{5.332438in}{3.227753in}%
\pgfsys@useobject{currentmarker}{}%
\end{pgfscope}%
\end{pgfscope}%
\begin{pgfscope}%
\pgfpathrectangle{\pgfqpoint{0.481681in}{1.080890in}}{\pgfqpoint{5.785672in}{2.146863in}}%
\pgfusepath{clip}%
\pgfsetrectcap%
\pgfsetroundjoin%
\pgfsetlinewidth{0.100375pt}%
\definecolor{currentstroke}{rgb}{0.501961,0.501961,0.501961}%
\pgfsetstrokecolor{currentstroke}%
\pgfsetdash{}{0pt}%
\pgfpathmoveto{\pgfqpoint{5.758684in}{1.080890in}}%
\pgfpathlineto{\pgfqpoint{5.758684in}{3.227753in}}%
\pgfusepath{stroke}%
\end{pgfscope}%
\begin{pgfscope}%
\pgfsetbuttcap%
\pgfsetroundjoin%
\definecolor{currentfill}{rgb}{0.000000,0.000000,0.000000}%
\pgfsetfillcolor{currentfill}%
\pgfsetlinewidth{0.501875pt}%
\definecolor{currentstroke}{rgb}{0.000000,0.000000,0.000000}%
\pgfsetstrokecolor{currentstroke}%
\pgfsetdash{}{0pt}%
\pgfsys@defobject{currentmarker}{\pgfqpoint{0.000000in}{0.000000in}}{\pgfqpoint{0.000000in}{0.041667in}}{%
\pgfpathmoveto{\pgfqpoint{0.000000in}{0.000000in}}%
\pgfpathlineto{\pgfqpoint{0.000000in}{0.041667in}}%
\pgfusepath{stroke,fill}%
}%
\begin{pgfscope}%
\pgfsys@transformshift{5.758684in}{1.080890in}%
\pgfsys@useobject{currentmarker}{}%
\end{pgfscope}%
\end{pgfscope}%
\begin{pgfscope}%
\pgfsetbuttcap%
\pgfsetroundjoin%
\definecolor{currentfill}{rgb}{0.000000,0.000000,0.000000}%
\pgfsetfillcolor{currentfill}%
\pgfsetlinewidth{0.501875pt}%
\definecolor{currentstroke}{rgb}{0.000000,0.000000,0.000000}%
\pgfsetstrokecolor{currentstroke}%
\pgfsetdash{}{0pt}%
\pgfsys@defobject{currentmarker}{\pgfqpoint{0.000000in}{-0.041667in}}{\pgfqpoint{0.000000in}{0.000000in}}{%
\pgfpathmoveto{\pgfqpoint{0.000000in}{0.000000in}}%
\pgfpathlineto{\pgfqpoint{0.000000in}{-0.041667in}}%
\pgfusepath{stroke,fill}%
}%
\begin{pgfscope}%
\pgfsys@transformshift{5.758684in}{3.227753in}%
\pgfsys@useobject{currentmarker}{}%
\end{pgfscope}%
\end{pgfscope}%
\begin{pgfscope}%
\pgfpathrectangle{\pgfqpoint{0.481681in}{1.080890in}}{\pgfqpoint{5.785672in}{2.146863in}}%
\pgfusepath{clip}%
\pgfsetrectcap%
\pgfsetroundjoin%
\pgfsetlinewidth{0.100375pt}%
\definecolor{currentstroke}{rgb}{0.501961,0.501961,0.501961}%
\pgfsetstrokecolor{currentstroke}%
\pgfsetdash{}{0pt}%
\pgfpathmoveto{\pgfqpoint{6.184931in}{1.080890in}}%
\pgfpathlineto{\pgfqpoint{6.184931in}{3.227753in}}%
\pgfusepath{stroke}%
\end{pgfscope}%
\begin{pgfscope}%
\pgfsetbuttcap%
\pgfsetroundjoin%
\definecolor{currentfill}{rgb}{0.000000,0.000000,0.000000}%
\pgfsetfillcolor{currentfill}%
\pgfsetlinewidth{0.501875pt}%
\definecolor{currentstroke}{rgb}{0.000000,0.000000,0.000000}%
\pgfsetstrokecolor{currentstroke}%
\pgfsetdash{}{0pt}%
\pgfsys@defobject{currentmarker}{\pgfqpoint{0.000000in}{0.000000in}}{\pgfqpoint{0.000000in}{0.041667in}}{%
\pgfpathmoveto{\pgfqpoint{0.000000in}{0.000000in}}%
\pgfpathlineto{\pgfqpoint{0.000000in}{0.041667in}}%
\pgfusepath{stroke,fill}%
}%
\begin{pgfscope}%
\pgfsys@transformshift{6.184931in}{1.080890in}%
\pgfsys@useobject{currentmarker}{}%
\end{pgfscope}%
\end{pgfscope}%
\begin{pgfscope}%
\pgfsetbuttcap%
\pgfsetroundjoin%
\definecolor{currentfill}{rgb}{0.000000,0.000000,0.000000}%
\pgfsetfillcolor{currentfill}%
\pgfsetlinewidth{0.501875pt}%
\definecolor{currentstroke}{rgb}{0.000000,0.000000,0.000000}%
\pgfsetstrokecolor{currentstroke}%
\pgfsetdash{}{0pt}%
\pgfsys@defobject{currentmarker}{\pgfqpoint{0.000000in}{-0.041667in}}{\pgfqpoint{0.000000in}{0.000000in}}{%
\pgfpathmoveto{\pgfqpoint{0.000000in}{0.000000in}}%
\pgfpathlineto{\pgfqpoint{0.000000in}{-0.041667in}}%
\pgfusepath{stroke,fill}%
}%
\begin{pgfscope}%
\pgfsys@transformshift{6.184931in}{3.227753in}%
\pgfsys@useobject{currentmarker}{}%
\end{pgfscope}%
\end{pgfscope}%
\begin{pgfscope}%
\pgfpathrectangle{\pgfqpoint{0.481681in}{1.080890in}}{\pgfqpoint{5.785672in}{2.146863in}}%
\pgfusepath{clip}%
\pgfsetrectcap%
\pgfsetroundjoin%
\pgfsetlinewidth{0.100375pt}%
\definecolor{currentstroke}{rgb}{0.827451,0.827451,0.827451}%
\pgfsetstrokecolor{currentstroke}%
\pgfsetdash{}{0pt}%
\pgfpathmoveto{\pgfqpoint{0.501646in}{1.080890in}}%
\pgfpathlineto{\pgfqpoint{0.501646in}{3.227753in}}%
\pgfusepath{stroke}%
\end{pgfscope}%
\begin{pgfscope}%
\pgfsetbuttcap%
\pgfsetroundjoin%
\definecolor{currentfill}{rgb}{0.000000,0.000000,0.000000}%
\pgfsetfillcolor{currentfill}%
\pgfsetlinewidth{0.501875pt}%
\definecolor{currentstroke}{rgb}{0.000000,0.000000,0.000000}%
\pgfsetstrokecolor{currentstroke}%
\pgfsetdash{}{0pt}%
\pgfsys@defobject{currentmarker}{\pgfqpoint{0.000000in}{0.000000in}}{\pgfqpoint{0.000000in}{0.020833in}}{%
\pgfpathmoveto{\pgfqpoint{0.000000in}{0.000000in}}%
\pgfpathlineto{\pgfqpoint{0.000000in}{0.020833in}}%
\pgfusepath{stroke,fill}%
}%
\begin{pgfscope}%
\pgfsys@transformshift{0.501646in}{1.080890in}%
\pgfsys@useobject{currentmarker}{}%
\end{pgfscope}%
\end{pgfscope}%
\begin{pgfscope}%
\pgfsetbuttcap%
\pgfsetroundjoin%
\definecolor{currentfill}{rgb}{0.000000,0.000000,0.000000}%
\pgfsetfillcolor{currentfill}%
\pgfsetlinewidth{0.501875pt}%
\definecolor{currentstroke}{rgb}{0.000000,0.000000,0.000000}%
\pgfsetstrokecolor{currentstroke}%
\pgfsetdash{}{0pt}%
\pgfsys@defobject{currentmarker}{\pgfqpoint{0.000000in}{-0.020833in}}{\pgfqpoint{0.000000in}{0.000000in}}{%
\pgfpathmoveto{\pgfqpoint{0.000000in}{0.000000in}}%
\pgfpathlineto{\pgfqpoint{0.000000in}{-0.020833in}}%
\pgfusepath{stroke,fill}%
}%
\begin{pgfscope}%
\pgfsys@transformshift{0.501646in}{3.227753in}%
\pgfsys@useobject{currentmarker}{}%
\end{pgfscope}%
\end{pgfscope}%
\begin{pgfscope}%
\pgfpathrectangle{\pgfqpoint{0.481681in}{1.080890in}}{\pgfqpoint{5.785672in}{2.146863in}}%
\pgfusepath{clip}%
\pgfsetrectcap%
\pgfsetroundjoin%
\pgfsetlinewidth{0.100375pt}%
\definecolor{currentstroke}{rgb}{0.827451,0.827451,0.827451}%
\pgfsetstrokecolor{currentstroke}%
\pgfsetdash{}{0pt}%
\pgfpathmoveto{\pgfqpoint{0.537167in}{1.080890in}}%
\pgfpathlineto{\pgfqpoint{0.537167in}{3.227753in}}%
\pgfusepath{stroke}%
\end{pgfscope}%
\begin{pgfscope}%
\pgfsetbuttcap%
\pgfsetroundjoin%
\definecolor{currentfill}{rgb}{0.000000,0.000000,0.000000}%
\pgfsetfillcolor{currentfill}%
\pgfsetlinewidth{0.501875pt}%
\definecolor{currentstroke}{rgb}{0.000000,0.000000,0.000000}%
\pgfsetstrokecolor{currentstroke}%
\pgfsetdash{}{0pt}%
\pgfsys@defobject{currentmarker}{\pgfqpoint{0.000000in}{0.000000in}}{\pgfqpoint{0.000000in}{0.020833in}}{%
\pgfpathmoveto{\pgfqpoint{0.000000in}{0.000000in}}%
\pgfpathlineto{\pgfqpoint{0.000000in}{0.020833in}}%
\pgfusepath{stroke,fill}%
}%
\begin{pgfscope}%
\pgfsys@transformshift{0.537167in}{1.080890in}%
\pgfsys@useobject{currentmarker}{}%
\end{pgfscope}%
\end{pgfscope}%
\begin{pgfscope}%
\pgfsetbuttcap%
\pgfsetroundjoin%
\definecolor{currentfill}{rgb}{0.000000,0.000000,0.000000}%
\pgfsetfillcolor{currentfill}%
\pgfsetlinewidth{0.501875pt}%
\definecolor{currentstroke}{rgb}{0.000000,0.000000,0.000000}%
\pgfsetstrokecolor{currentstroke}%
\pgfsetdash{}{0pt}%
\pgfsys@defobject{currentmarker}{\pgfqpoint{0.000000in}{-0.020833in}}{\pgfqpoint{0.000000in}{0.000000in}}{%
\pgfpathmoveto{\pgfqpoint{0.000000in}{0.000000in}}%
\pgfpathlineto{\pgfqpoint{0.000000in}{-0.020833in}}%
\pgfusepath{stroke,fill}%
}%
\begin{pgfscope}%
\pgfsys@transformshift{0.537167in}{3.227753in}%
\pgfsys@useobject{currentmarker}{}%
\end{pgfscope}%
\end{pgfscope}%
\begin{pgfscope}%
\pgfpathrectangle{\pgfqpoint{0.481681in}{1.080890in}}{\pgfqpoint{5.785672in}{2.146863in}}%
\pgfusepath{clip}%
\pgfsetrectcap%
\pgfsetroundjoin%
\pgfsetlinewidth{0.100375pt}%
\definecolor{currentstroke}{rgb}{0.827451,0.827451,0.827451}%
\pgfsetstrokecolor{currentstroke}%
\pgfsetdash{}{0pt}%
\pgfpathmoveto{\pgfqpoint{0.572687in}{1.080890in}}%
\pgfpathlineto{\pgfqpoint{0.572687in}{3.227753in}}%
\pgfusepath{stroke}%
\end{pgfscope}%
\begin{pgfscope}%
\pgfsetbuttcap%
\pgfsetroundjoin%
\definecolor{currentfill}{rgb}{0.000000,0.000000,0.000000}%
\pgfsetfillcolor{currentfill}%
\pgfsetlinewidth{0.501875pt}%
\definecolor{currentstroke}{rgb}{0.000000,0.000000,0.000000}%
\pgfsetstrokecolor{currentstroke}%
\pgfsetdash{}{0pt}%
\pgfsys@defobject{currentmarker}{\pgfqpoint{0.000000in}{0.000000in}}{\pgfqpoint{0.000000in}{0.020833in}}{%
\pgfpathmoveto{\pgfqpoint{0.000000in}{0.000000in}}%
\pgfpathlineto{\pgfqpoint{0.000000in}{0.020833in}}%
\pgfusepath{stroke,fill}%
}%
\begin{pgfscope}%
\pgfsys@transformshift{0.572687in}{1.080890in}%
\pgfsys@useobject{currentmarker}{}%
\end{pgfscope}%
\end{pgfscope}%
\begin{pgfscope}%
\pgfsetbuttcap%
\pgfsetroundjoin%
\definecolor{currentfill}{rgb}{0.000000,0.000000,0.000000}%
\pgfsetfillcolor{currentfill}%
\pgfsetlinewidth{0.501875pt}%
\definecolor{currentstroke}{rgb}{0.000000,0.000000,0.000000}%
\pgfsetstrokecolor{currentstroke}%
\pgfsetdash{}{0pt}%
\pgfsys@defobject{currentmarker}{\pgfqpoint{0.000000in}{-0.020833in}}{\pgfqpoint{0.000000in}{0.000000in}}{%
\pgfpathmoveto{\pgfqpoint{0.000000in}{0.000000in}}%
\pgfpathlineto{\pgfqpoint{0.000000in}{-0.020833in}}%
\pgfusepath{stroke,fill}%
}%
\begin{pgfscope}%
\pgfsys@transformshift{0.572687in}{3.227753in}%
\pgfsys@useobject{currentmarker}{}%
\end{pgfscope}%
\end{pgfscope}%
\begin{pgfscope}%
\pgfpathrectangle{\pgfqpoint{0.481681in}{1.080890in}}{\pgfqpoint{5.785672in}{2.146863in}}%
\pgfusepath{clip}%
\pgfsetrectcap%
\pgfsetroundjoin%
\pgfsetlinewidth{0.100375pt}%
\definecolor{currentstroke}{rgb}{0.827451,0.827451,0.827451}%
\pgfsetstrokecolor{currentstroke}%
\pgfsetdash{}{0pt}%
\pgfpathmoveto{\pgfqpoint{0.608208in}{1.080890in}}%
\pgfpathlineto{\pgfqpoint{0.608208in}{3.227753in}}%
\pgfusepath{stroke}%
\end{pgfscope}%
\begin{pgfscope}%
\pgfsetbuttcap%
\pgfsetroundjoin%
\definecolor{currentfill}{rgb}{0.000000,0.000000,0.000000}%
\pgfsetfillcolor{currentfill}%
\pgfsetlinewidth{0.501875pt}%
\definecolor{currentstroke}{rgb}{0.000000,0.000000,0.000000}%
\pgfsetstrokecolor{currentstroke}%
\pgfsetdash{}{0pt}%
\pgfsys@defobject{currentmarker}{\pgfqpoint{0.000000in}{0.000000in}}{\pgfqpoint{0.000000in}{0.020833in}}{%
\pgfpathmoveto{\pgfqpoint{0.000000in}{0.000000in}}%
\pgfpathlineto{\pgfqpoint{0.000000in}{0.020833in}}%
\pgfusepath{stroke,fill}%
}%
\begin{pgfscope}%
\pgfsys@transformshift{0.608208in}{1.080890in}%
\pgfsys@useobject{currentmarker}{}%
\end{pgfscope}%
\end{pgfscope}%
\begin{pgfscope}%
\pgfsetbuttcap%
\pgfsetroundjoin%
\definecolor{currentfill}{rgb}{0.000000,0.000000,0.000000}%
\pgfsetfillcolor{currentfill}%
\pgfsetlinewidth{0.501875pt}%
\definecolor{currentstroke}{rgb}{0.000000,0.000000,0.000000}%
\pgfsetstrokecolor{currentstroke}%
\pgfsetdash{}{0pt}%
\pgfsys@defobject{currentmarker}{\pgfqpoint{0.000000in}{-0.020833in}}{\pgfqpoint{0.000000in}{0.000000in}}{%
\pgfpathmoveto{\pgfqpoint{0.000000in}{0.000000in}}%
\pgfpathlineto{\pgfqpoint{0.000000in}{-0.020833in}}%
\pgfusepath{stroke,fill}%
}%
\begin{pgfscope}%
\pgfsys@transformshift{0.608208in}{3.227753in}%
\pgfsys@useobject{currentmarker}{}%
\end{pgfscope}%
\end{pgfscope}%
\begin{pgfscope}%
\pgfpathrectangle{\pgfqpoint{0.481681in}{1.080890in}}{\pgfqpoint{5.785672in}{2.146863in}}%
\pgfusepath{clip}%
\pgfsetrectcap%
\pgfsetroundjoin%
\pgfsetlinewidth{0.100375pt}%
\definecolor{currentstroke}{rgb}{0.827451,0.827451,0.827451}%
\pgfsetstrokecolor{currentstroke}%
\pgfsetdash{}{0pt}%
\pgfpathmoveto{\pgfqpoint{0.679249in}{1.080890in}}%
\pgfpathlineto{\pgfqpoint{0.679249in}{3.227753in}}%
\pgfusepath{stroke}%
\end{pgfscope}%
\begin{pgfscope}%
\pgfsetbuttcap%
\pgfsetroundjoin%
\definecolor{currentfill}{rgb}{0.000000,0.000000,0.000000}%
\pgfsetfillcolor{currentfill}%
\pgfsetlinewidth{0.501875pt}%
\definecolor{currentstroke}{rgb}{0.000000,0.000000,0.000000}%
\pgfsetstrokecolor{currentstroke}%
\pgfsetdash{}{0pt}%
\pgfsys@defobject{currentmarker}{\pgfqpoint{0.000000in}{0.000000in}}{\pgfqpoint{0.000000in}{0.020833in}}{%
\pgfpathmoveto{\pgfqpoint{0.000000in}{0.000000in}}%
\pgfpathlineto{\pgfqpoint{0.000000in}{0.020833in}}%
\pgfusepath{stroke,fill}%
}%
\begin{pgfscope}%
\pgfsys@transformshift{0.679249in}{1.080890in}%
\pgfsys@useobject{currentmarker}{}%
\end{pgfscope}%
\end{pgfscope}%
\begin{pgfscope}%
\pgfsetbuttcap%
\pgfsetroundjoin%
\definecolor{currentfill}{rgb}{0.000000,0.000000,0.000000}%
\pgfsetfillcolor{currentfill}%
\pgfsetlinewidth{0.501875pt}%
\definecolor{currentstroke}{rgb}{0.000000,0.000000,0.000000}%
\pgfsetstrokecolor{currentstroke}%
\pgfsetdash{}{0pt}%
\pgfsys@defobject{currentmarker}{\pgfqpoint{0.000000in}{-0.020833in}}{\pgfqpoint{0.000000in}{0.000000in}}{%
\pgfpathmoveto{\pgfqpoint{0.000000in}{0.000000in}}%
\pgfpathlineto{\pgfqpoint{0.000000in}{-0.020833in}}%
\pgfusepath{stroke,fill}%
}%
\begin{pgfscope}%
\pgfsys@transformshift{0.679249in}{3.227753in}%
\pgfsys@useobject{currentmarker}{}%
\end{pgfscope}%
\end{pgfscope}%
\begin{pgfscope}%
\pgfpathrectangle{\pgfqpoint{0.481681in}{1.080890in}}{\pgfqpoint{5.785672in}{2.146863in}}%
\pgfusepath{clip}%
\pgfsetrectcap%
\pgfsetroundjoin%
\pgfsetlinewidth{0.100375pt}%
\definecolor{currentstroke}{rgb}{0.827451,0.827451,0.827451}%
\pgfsetstrokecolor{currentstroke}%
\pgfsetdash{}{0pt}%
\pgfpathmoveto{\pgfqpoint{0.714769in}{1.080890in}}%
\pgfpathlineto{\pgfqpoint{0.714769in}{3.227753in}}%
\pgfusepath{stroke}%
\end{pgfscope}%
\begin{pgfscope}%
\pgfsetbuttcap%
\pgfsetroundjoin%
\definecolor{currentfill}{rgb}{0.000000,0.000000,0.000000}%
\pgfsetfillcolor{currentfill}%
\pgfsetlinewidth{0.501875pt}%
\definecolor{currentstroke}{rgb}{0.000000,0.000000,0.000000}%
\pgfsetstrokecolor{currentstroke}%
\pgfsetdash{}{0pt}%
\pgfsys@defobject{currentmarker}{\pgfqpoint{0.000000in}{0.000000in}}{\pgfqpoint{0.000000in}{0.020833in}}{%
\pgfpathmoveto{\pgfqpoint{0.000000in}{0.000000in}}%
\pgfpathlineto{\pgfqpoint{0.000000in}{0.020833in}}%
\pgfusepath{stroke,fill}%
}%
\begin{pgfscope}%
\pgfsys@transformshift{0.714769in}{1.080890in}%
\pgfsys@useobject{currentmarker}{}%
\end{pgfscope}%
\end{pgfscope}%
\begin{pgfscope}%
\pgfsetbuttcap%
\pgfsetroundjoin%
\definecolor{currentfill}{rgb}{0.000000,0.000000,0.000000}%
\pgfsetfillcolor{currentfill}%
\pgfsetlinewidth{0.501875pt}%
\definecolor{currentstroke}{rgb}{0.000000,0.000000,0.000000}%
\pgfsetstrokecolor{currentstroke}%
\pgfsetdash{}{0pt}%
\pgfsys@defobject{currentmarker}{\pgfqpoint{0.000000in}{-0.020833in}}{\pgfqpoint{0.000000in}{0.000000in}}{%
\pgfpathmoveto{\pgfqpoint{0.000000in}{0.000000in}}%
\pgfpathlineto{\pgfqpoint{0.000000in}{-0.020833in}}%
\pgfusepath{stroke,fill}%
}%
\begin{pgfscope}%
\pgfsys@transformshift{0.714769in}{3.227753in}%
\pgfsys@useobject{currentmarker}{}%
\end{pgfscope}%
\end{pgfscope}%
\begin{pgfscope}%
\pgfpathrectangle{\pgfqpoint{0.481681in}{1.080890in}}{\pgfqpoint{5.785672in}{2.146863in}}%
\pgfusepath{clip}%
\pgfsetrectcap%
\pgfsetroundjoin%
\pgfsetlinewidth{0.100375pt}%
\definecolor{currentstroke}{rgb}{0.827451,0.827451,0.827451}%
\pgfsetstrokecolor{currentstroke}%
\pgfsetdash{}{0pt}%
\pgfpathmoveto{\pgfqpoint{0.750290in}{1.080890in}}%
\pgfpathlineto{\pgfqpoint{0.750290in}{3.227753in}}%
\pgfusepath{stroke}%
\end{pgfscope}%
\begin{pgfscope}%
\pgfsetbuttcap%
\pgfsetroundjoin%
\definecolor{currentfill}{rgb}{0.000000,0.000000,0.000000}%
\pgfsetfillcolor{currentfill}%
\pgfsetlinewidth{0.501875pt}%
\definecolor{currentstroke}{rgb}{0.000000,0.000000,0.000000}%
\pgfsetstrokecolor{currentstroke}%
\pgfsetdash{}{0pt}%
\pgfsys@defobject{currentmarker}{\pgfqpoint{0.000000in}{0.000000in}}{\pgfqpoint{0.000000in}{0.020833in}}{%
\pgfpathmoveto{\pgfqpoint{0.000000in}{0.000000in}}%
\pgfpathlineto{\pgfqpoint{0.000000in}{0.020833in}}%
\pgfusepath{stroke,fill}%
}%
\begin{pgfscope}%
\pgfsys@transformshift{0.750290in}{1.080890in}%
\pgfsys@useobject{currentmarker}{}%
\end{pgfscope}%
\end{pgfscope}%
\begin{pgfscope}%
\pgfsetbuttcap%
\pgfsetroundjoin%
\definecolor{currentfill}{rgb}{0.000000,0.000000,0.000000}%
\pgfsetfillcolor{currentfill}%
\pgfsetlinewidth{0.501875pt}%
\definecolor{currentstroke}{rgb}{0.000000,0.000000,0.000000}%
\pgfsetstrokecolor{currentstroke}%
\pgfsetdash{}{0pt}%
\pgfsys@defobject{currentmarker}{\pgfqpoint{0.000000in}{-0.020833in}}{\pgfqpoint{0.000000in}{0.000000in}}{%
\pgfpathmoveto{\pgfqpoint{0.000000in}{0.000000in}}%
\pgfpathlineto{\pgfqpoint{0.000000in}{-0.020833in}}%
\pgfusepath{stroke,fill}%
}%
\begin{pgfscope}%
\pgfsys@transformshift{0.750290in}{3.227753in}%
\pgfsys@useobject{currentmarker}{}%
\end{pgfscope}%
\end{pgfscope}%
\begin{pgfscope}%
\pgfpathrectangle{\pgfqpoint{0.481681in}{1.080890in}}{\pgfqpoint{5.785672in}{2.146863in}}%
\pgfusepath{clip}%
\pgfsetrectcap%
\pgfsetroundjoin%
\pgfsetlinewidth{0.100375pt}%
\definecolor{currentstroke}{rgb}{0.827451,0.827451,0.827451}%
\pgfsetstrokecolor{currentstroke}%
\pgfsetdash{}{0pt}%
\pgfpathmoveto{\pgfqpoint{0.785811in}{1.080890in}}%
\pgfpathlineto{\pgfqpoint{0.785811in}{3.227753in}}%
\pgfusepath{stroke}%
\end{pgfscope}%
\begin{pgfscope}%
\pgfsetbuttcap%
\pgfsetroundjoin%
\definecolor{currentfill}{rgb}{0.000000,0.000000,0.000000}%
\pgfsetfillcolor{currentfill}%
\pgfsetlinewidth{0.501875pt}%
\definecolor{currentstroke}{rgb}{0.000000,0.000000,0.000000}%
\pgfsetstrokecolor{currentstroke}%
\pgfsetdash{}{0pt}%
\pgfsys@defobject{currentmarker}{\pgfqpoint{0.000000in}{0.000000in}}{\pgfqpoint{0.000000in}{0.020833in}}{%
\pgfpathmoveto{\pgfqpoint{0.000000in}{0.000000in}}%
\pgfpathlineto{\pgfqpoint{0.000000in}{0.020833in}}%
\pgfusepath{stroke,fill}%
}%
\begin{pgfscope}%
\pgfsys@transformshift{0.785811in}{1.080890in}%
\pgfsys@useobject{currentmarker}{}%
\end{pgfscope}%
\end{pgfscope}%
\begin{pgfscope}%
\pgfsetbuttcap%
\pgfsetroundjoin%
\definecolor{currentfill}{rgb}{0.000000,0.000000,0.000000}%
\pgfsetfillcolor{currentfill}%
\pgfsetlinewidth{0.501875pt}%
\definecolor{currentstroke}{rgb}{0.000000,0.000000,0.000000}%
\pgfsetstrokecolor{currentstroke}%
\pgfsetdash{}{0pt}%
\pgfsys@defobject{currentmarker}{\pgfqpoint{0.000000in}{-0.020833in}}{\pgfqpoint{0.000000in}{0.000000in}}{%
\pgfpathmoveto{\pgfqpoint{0.000000in}{0.000000in}}%
\pgfpathlineto{\pgfqpoint{0.000000in}{-0.020833in}}%
\pgfusepath{stroke,fill}%
}%
\begin{pgfscope}%
\pgfsys@transformshift{0.785811in}{3.227753in}%
\pgfsys@useobject{currentmarker}{}%
\end{pgfscope}%
\end{pgfscope}%
\begin{pgfscope}%
\pgfpathrectangle{\pgfqpoint{0.481681in}{1.080890in}}{\pgfqpoint{5.785672in}{2.146863in}}%
\pgfusepath{clip}%
\pgfsetrectcap%
\pgfsetroundjoin%
\pgfsetlinewidth{0.100375pt}%
\definecolor{currentstroke}{rgb}{0.827451,0.827451,0.827451}%
\pgfsetstrokecolor{currentstroke}%
\pgfsetdash{}{0pt}%
\pgfpathmoveto{\pgfqpoint{0.821331in}{1.080890in}}%
\pgfpathlineto{\pgfqpoint{0.821331in}{3.227753in}}%
\pgfusepath{stroke}%
\end{pgfscope}%
\begin{pgfscope}%
\pgfsetbuttcap%
\pgfsetroundjoin%
\definecolor{currentfill}{rgb}{0.000000,0.000000,0.000000}%
\pgfsetfillcolor{currentfill}%
\pgfsetlinewidth{0.501875pt}%
\definecolor{currentstroke}{rgb}{0.000000,0.000000,0.000000}%
\pgfsetstrokecolor{currentstroke}%
\pgfsetdash{}{0pt}%
\pgfsys@defobject{currentmarker}{\pgfqpoint{0.000000in}{0.000000in}}{\pgfqpoint{0.000000in}{0.020833in}}{%
\pgfpathmoveto{\pgfqpoint{0.000000in}{0.000000in}}%
\pgfpathlineto{\pgfqpoint{0.000000in}{0.020833in}}%
\pgfusepath{stroke,fill}%
}%
\begin{pgfscope}%
\pgfsys@transformshift{0.821331in}{1.080890in}%
\pgfsys@useobject{currentmarker}{}%
\end{pgfscope}%
\end{pgfscope}%
\begin{pgfscope}%
\pgfsetbuttcap%
\pgfsetroundjoin%
\definecolor{currentfill}{rgb}{0.000000,0.000000,0.000000}%
\pgfsetfillcolor{currentfill}%
\pgfsetlinewidth{0.501875pt}%
\definecolor{currentstroke}{rgb}{0.000000,0.000000,0.000000}%
\pgfsetstrokecolor{currentstroke}%
\pgfsetdash{}{0pt}%
\pgfsys@defobject{currentmarker}{\pgfqpoint{0.000000in}{-0.020833in}}{\pgfqpoint{0.000000in}{0.000000in}}{%
\pgfpathmoveto{\pgfqpoint{0.000000in}{0.000000in}}%
\pgfpathlineto{\pgfqpoint{0.000000in}{-0.020833in}}%
\pgfusepath{stroke,fill}%
}%
\begin{pgfscope}%
\pgfsys@transformshift{0.821331in}{3.227753in}%
\pgfsys@useobject{currentmarker}{}%
\end{pgfscope}%
\end{pgfscope}%
\begin{pgfscope}%
\pgfpathrectangle{\pgfqpoint{0.481681in}{1.080890in}}{\pgfqpoint{5.785672in}{2.146863in}}%
\pgfusepath{clip}%
\pgfsetrectcap%
\pgfsetroundjoin%
\pgfsetlinewidth{0.100375pt}%
\definecolor{currentstroke}{rgb}{0.827451,0.827451,0.827451}%
\pgfsetstrokecolor{currentstroke}%
\pgfsetdash{}{0pt}%
\pgfpathmoveto{\pgfqpoint{0.856852in}{1.080890in}}%
\pgfpathlineto{\pgfqpoint{0.856852in}{3.227753in}}%
\pgfusepath{stroke}%
\end{pgfscope}%
\begin{pgfscope}%
\pgfsetbuttcap%
\pgfsetroundjoin%
\definecolor{currentfill}{rgb}{0.000000,0.000000,0.000000}%
\pgfsetfillcolor{currentfill}%
\pgfsetlinewidth{0.501875pt}%
\definecolor{currentstroke}{rgb}{0.000000,0.000000,0.000000}%
\pgfsetstrokecolor{currentstroke}%
\pgfsetdash{}{0pt}%
\pgfsys@defobject{currentmarker}{\pgfqpoint{0.000000in}{0.000000in}}{\pgfqpoint{0.000000in}{0.020833in}}{%
\pgfpathmoveto{\pgfqpoint{0.000000in}{0.000000in}}%
\pgfpathlineto{\pgfqpoint{0.000000in}{0.020833in}}%
\pgfusepath{stroke,fill}%
}%
\begin{pgfscope}%
\pgfsys@transformshift{0.856852in}{1.080890in}%
\pgfsys@useobject{currentmarker}{}%
\end{pgfscope}%
\end{pgfscope}%
\begin{pgfscope}%
\pgfsetbuttcap%
\pgfsetroundjoin%
\definecolor{currentfill}{rgb}{0.000000,0.000000,0.000000}%
\pgfsetfillcolor{currentfill}%
\pgfsetlinewidth{0.501875pt}%
\definecolor{currentstroke}{rgb}{0.000000,0.000000,0.000000}%
\pgfsetstrokecolor{currentstroke}%
\pgfsetdash{}{0pt}%
\pgfsys@defobject{currentmarker}{\pgfqpoint{0.000000in}{-0.020833in}}{\pgfqpoint{0.000000in}{0.000000in}}{%
\pgfpathmoveto{\pgfqpoint{0.000000in}{0.000000in}}%
\pgfpathlineto{\pgfqpoint{0.000000in}{-0.020833in}}%
\pgfusepath{stroke,fill}%
}%
\begin{pgfscope}%
\pgfsys@transformshift{0.856852in}{3.227753in}%
\pgfsys@useobject{currentmarker}{}%
\end{pgfscope}%
\end{pgfscope}%
\begin{pgfscope}%
\pgfpathrectangle{\pgfqpoint{0.481681in}{1.080890in}}{\pgfqpoint{5.785672in}{2.146863in}}%
\pgfusepath{clip}%
\pgfsetrectcap%
\pgfsetroundjoin%
\pgfsetlinewidth{0.100375pt}%
\definecolor{currentstroke}{rgb}{0.827451,0.827451,0.827451}%
\pgfsetstrokecolor{currentstroke}%
\pgfsetdash{}{0pt}%
\pgfpathmoveto{\pgfqpoint{0.892372in}{1.080890in}}%
\pgfpathlineto{\pgfqpoint{0.892372in}{3.227753in}}%
\pgfusepath{stroke}%
\end{pgfscope}%
\begin{pgfscope}%
\pgfsetbuttcap%
\pgfsetroundjoin%
\definecolor{currentfill}{rgb}{0.000000,0.000000,0.000000}%
\pgfsetfillcolor{currentfill}%
\pgfsetlinewidth{0.501875pt}%
\definecolor{currentstroke}{rgb}{0.000000,0.000000,0.000000}%
\pgfsetstrokecolor{currentstroke}%
\pgfsetdash{}{0pt}%
\pgfsys@defobject{currentmarker}{\pgfqpoint{0.000000in}{0.000000in}}{\pgfqpoint{0.000000in}{0.020833in}}{%
\pgfpathmoveto{\pgfqpoint{0.000000in}{0.000000in}}%
\pgfpathlineto{\pgfqpoint{0.000000in}{0.020833in}}%
\pgfusepath{stroke,fill}%
}%
\begin{pgfscope}%
\pgfsys@transformshift{0.892372in}{1.080890in}%
\pgfsys@useobject{currentmarker}{}%
\end{pgfscope}%
\end{pgfscope}%
\begin{pgfscope}%
\pgfsetbuttcap%
\pgfsetroundjoin%
\definecolor{currentfill}{rgb}{0.000000,0.000000,0.000000}%
\pgfsetfillcolor{currentfill}%
\pgfsetlinewidth{0.501875pt}%
\definecolor{currentstroke}{rgb}{0.000000,0.000000,0.000000}%
\pgfsetstrokecolor{currentstroke}%
\pgfsetdash{}{0pt}%
\pgfsys@defobject{currentmarker}{\pgfqpoint{0.000000in}{-0.020833in}}{\pgfqpoint{0.000000in}{0.000000in}}{%
\pgfpathmoveto{\pgfqpoint{0.000000in}{0.000000in}}%
\pgfpathlineto{\pgfqpoint{0.000000in}{-0.020833in}}%
\pgfusepath{stroke,fill}%
}%
\begin{pgfscope}%
\pgfsys@transformshift{0.892372in}{3.227753in}%
\pgfsys@useobject{currentmarker}{}%
\end{pgfscope}%
\end{pgfscope}%
\begin{pgfscope}%
\pgfpathrectangle{\pgfqpoint{0.481681in}{1.080890in}}{\pgfqpoint{5.785672in}{2.146863in}}%
\pgfusepath{clip}%
\pgfsetrectcap%
\pgfsetroundjoin%
\pgfsetlinewidth{0.100375pt}%
\definecolor{currentstroke}{rgb}{0.827451,0.827451,0.827451}%
\pgfsetstrokecolor{currentstroke}%
\pgfsetdash{}{0pt}%
\pgfpathmoveto{\pgfqpoint{0.927893in}{1.080890in}}%
\pgfpathlineto{\pgfqpoint{0.927893in}{3.227753in}}%
\pgfusepath{stroke}%
\end{pgfscope}%
\begin{pgfscope}%
\pgfsetbuttcap%
\pgfsetroundjoin%
\definecolor{currentfill}{rgb}{0.000000,0.000000,0.000000}%
\pgfsetfillcolor{currentfill}%
\pgfsetlinewidth{0.501875pt}%
\definecolor{currentstroke}{rgb}{0.000000,0.000000,0.000000}%
\pgfsetstrokecolor{currentstroke}%
\pgfsetdash{}{0pt}%
\pgfsys@defobject{currentmarker}{\pgfqpoint{0.000000in}{0.000000in}}{\pgfqpoint{0.000000in}{0.020833in}}{%
\pgfpathmoveto{\pgfqpoint{0.000000in}{0.000000in}}%
\pgfpathlineto{\pgfqpoint{0.000000in}{0.020833in}}%
\pgfusepath{stroke,fill}%
}%
\begin{pgfscope}%
\pgfsys@transformshift{0.927893in}{1.080890in}%
\pgfsys@useobject{currentmarker}{}%
\end{pgfscope}%
\end{pgfscope}%
\begin{pgfscope}%
\pgfsetbuttcap%
\pgfsetroundjoin%
\definecolor{currentfill}{rgb}{0.000000,0.000000,0.000000}%
\pgfsetfillcolor{currentfill}%
\pgfsetlinewidth{0.501875pt}%
\definecolor{currentstroke}{rgb}{0.000000,0.000000,0.000000}%
\pgfsetstrokecolor{currentstroke}%
\pgfsetdash{}{0pt}%
\pgfsys@defobject{currentmarker}{\pgfqpoint{0.000000in}{-0.020833in}}{\pgfqpoint{0.000000in}{0.000000in}}{%
\pgfpathmoveto{\pgfqpoint{0.000000in}{0.000000in}}%
\pgfpathlineto{\pgfqpoint{0.000000in}{-0.020833in}}%
\pgfusepath{stroke,fill}%
}%
\begin{pgfscope}%
\pgfsys@transformshift{0.927893in}{3.227753in}%
\pgfsys@useobject{currentmarker}{}%
\end{pgfscope}%
\end{pgfscope}%
\begin{pgfscope}%
\pgfpathrectangle{\pgfqpoint{0.481681in}{1.080890in}}{\pgfqpoint{5.785672in}{2.146863in}}%
\pgfusepath{clip}%
\pgfsetrectcap%
\pgfsetroundjoin%
\pgfsetlinewidth{0.100375pt}%
\definecolor{currentstroke}{rgb}{0.827451,0.827451,0.827451}%
\pgfsetstrokecolor{currentstroke}%
\pgfsetdash{}{0pt}%
\pgfpathmoveto{\pgfqpoint{0.963413in}{1.080890in}}%
\pgfpathlineto{\pgfqpoint{0.963413in}{3.227753in}}%
\pgfusepath{stroke}%
\end{pgfscope}%
\begin{pgfscope}%
\pgfsetbuttcap%
\pgfsetroundjoin%
\definecolor{currentfill}{rgb}{0.000000,0.000000,0.000000}%
\pgfsetfillcolor{currentfill}%
\pgfsetlinewidth{0.501875pt}%
\definecolor{currentstroke}{rgb}{0.000000,0.000000,0.000000}%
\pgfsetstrokecolor{currentstroke}%
\pgfsetdash{}{0pt}%
\pgfsys@defobject{currentmarker}{\pgfqpoint{0.000000in}{0.000000in}}{\pgfqpoint{0.000000in}{0.020833in}}{%
\pgfpathmoveto{\pgfqpoint{0.000000in}{0.000000in}}%
\pgfpathlineto{\pgfqpoint{0.000000in}{0.020833in}}%
\pgfusepath{stroke,fill}%
}%
\begin{pgfscope}%
\pgfsys@transformshift{0.963413in}{1.080890in}%
\pgfsys@useobject{currentmarker}{}%
\end{pgfscope}%
\end{pgfscope}%
\begin{pgfscope}%
\pgfsetbuttcap%
\pgfsetroundjoin%
\definecolor{currentfill}{rgb}{0.000000,0.000000,0.000000}%
\pgfsetfillcolor{currentfill}%
\pgfsetlinewidth{0.501875pt}%
\definecolor{currentstroke}{rgb}{0.000000,0.000000,0.000000}%
\pgfsetstrokecolor{currentstroke}%
\pgfsetdash{}{0pt}%
\pgfsys@defobject{currentmarker}{\pgfqpoint{0.000000in}{-0.020833in}}{\pgfqpoint{0.000000in}{0.000000in}}{%
\pgfpathmoveto{\pgfqpoint{0.000000in}{0.000000in}}%
\pgfpathlineto{\pgfqpoint{0.000000in}{-0.020833in}}%
\pgfusepath{stroke,fill}%
}%
\begin{pgfscope}%
\pgfsys@transformshift{0.963413in}{3.227753in}%
\pgfsys@useobject{currentmarker}{}%
\end{pgfscope}%
\end{pgfscope}%
\begin{pgfscope}%
\pgfpathrectangle{\pgfqpoint{0.481681in}{1.080890in}}{\pgfqpoint{5.785672in}{2.146863in}}%
\pgfusepath{clip}%
\pgfsetrectcap%
\pgfsetroundjoin%
\pgfsetlinewidth{0.100375pt}%
\definecolor{currentstroke}{rgb}{0.827451,0.827451,0.827451}%
\pgfsetstrokecolor{currentstroke}%
\pgfsetdash{}{0pt}%
\pgfpathmoveto{\pgfqpoint{0.998934in}{1.080890in}}%
\pgfpathlineto{\pgfqpoint{0.998934in}{3.227753in}}%
\pgfusepath{stroke}%
\end{pgfscope}%
\begin{pgfscope}%
\pgfsetbuttcap%
\pgfsetroundjoin%
\definecolor{currentfill}{rgb}{0.000000,0.000000,0.000000}%
\pgfsetfillcolor{currentfill}%
\pgfsetlinewidth{0.501875pt}%
\definecolor{currentstroke}{rgb}{0.000000,0.000000,0.000000}%
\pgfsetstrokecolor{currentstroke}%
\pgfsetdash{}{0pt}%
\pgfsys@defobject{currentmarker}{\pgfqpoint{0.000000in}{0.000000in}}{\pgfqpoint{0.000000in}{0.020833in}}{%
\pgfpathmoveto{\pgfqpoint{0.000000in}{0.000000in}}%
\pgfpathlineto{\pgfqpoint{0.000000in}{0.020833in}}%
\pgfusepath{stroke,fill}%
}%
\begin{pgfscope}%
\pgfsys@transformshift{0.998934in}{1.080890in}%
\pgfsys@useobject{currentmarker}{}%
\end{pgfscope}%
\end{pgfscope}%
\begin{pgfscope}%
\pgfsetbuttcap%
\pgfsetroundjoin%
\definecolor{currentfill}{rgb}{0.000000,0.000000,0.000000}%
\pgfsetfillcolor{currentfill}%
\pgfsetlinewidth{0.501875pt}%
\definecolor{currentstroke}{rgb}{0.000000,0.000000,0.000000}%
\pgfsetstrokecolor{currentstroke}%
\pgfsetdash{}{0pt}%
\pgfsys@defobject{currentmarker}{\pgfqpoint{0.000000in}{-0.020833in}}{\pgfqpoint{0.000000in}{0.000000in}}{%
\pgfpathmoveto{\pgfqpoint{0.000000in}{0.000000in}}%
\pgfpathlineto{\pgfqpoint{0.000000in}{-0.020833in}}%
\pgfusepath{stroke,fill}%
}%
\begin{pgfscope}%
\pgfsys@transformshift{0.998934in}{3.227753in}%
\pgfsys@useobject{currentmarker}{}%
\end{pgfscope}%
\end{pgfscope}%
\begin{pgfscope}%
\pgfpathrectangle{\pgfqpoint{0.481681in}{1.080890in}}{\pgfqpoint{5.785672in}{2.146863in}}%
\pgfusepath{clip}%
\pgfsetrectcap%
\pgfsetroundjoin%
\pgfsetlinewidth{0.100375pt}%
\definecolor{currentstroke}{rgb}{0.827451,0.827451,0.827451}%
\pgfsetstrokecolor{currentstroke}%
\pgfsetdash{}{0pt}%
\pgfpathmoveto{\pgfqpoint{1.034454in}{1.080890in}}%
\pgfpathlineto{\pgfqpoint{1.034454in}{3.227753in}}%
\pgfusepath{stroke}%
\end{pgfscope}%
\begin{pgfscope}%
\pgfsetbuttcap%
\pgfsetroundjoin%
\definecolor{currentfill}{rgb}{0.000000,0.000000,0.000000}%
\pgfsetfillcolor{currentfill}%
\pgfsetlinewidth{0.501875pt}%
\definecolor{currentstroke}{rgb}{0.000000,0.000000,0.000000}%
\pgfsetstrokecolor{currentstroke}%
\pgfsetdash{}{0pt}%
\pgfsys@defobject{currentmarker}{\pgfqpoint{0.000000in}{0.000000in}}{\pgfqpoint{0.000000in}{0.020833in}}{%
\pgfpathmoveto{\pgfqpoint{0.000000in}{0.000000in}}%
\pgfpathlineto{\pgfqpoint{0.000000in}{0.020833in}}%
\pgfusepath{stroke,fill}%
}%
\begin{pgfscope}%
\pgfsys@transformshift{1.034454in}{1.080890in}%
\pgfsys@useobject{currentmarker}{}%
\end{pgfscope}%
\end{pgfscope}%
\begin{pgfscope}%
\pgfsetbuttcap%
\pgfsetroundjoin%
\definecolor{currentfill}{rgb}{0.000000,0.000000,0.000000}%
\pgfsetfillcolor{currentfill}%
\pgfsetlinewidth{0.501875pt}%
\definecolor{currentstroke}{rgb}{0.000000,0.000000,0.000000}%
\pgfsetstrokecolor{currentstroke}%
\pgfsetdash{}{0pt}%
\pgfsys@defobject{currentmarker}{\pgfqpoint{0.000000in}{-0.020833in}}{\pgfqpoint{0.000000in}{0.000000in}}{%
\pgfpathmoveto{\pgfqpoint{0.000000in}{0.000000in}}%
\pgfpathlineto{\pgfqpoint{0.000000in}{-0.020833in}}%
\pgfusepath{stroke,fill}%
}%
\begin{pgfscope}%
\pgfsys@transformshift{1.034454in}{3.227753in}%
\pgfsys@useobject{currentmarker}{}%
\end{pgfscope}%
\end{pgfscope}%
\begin{pgfscope}%
\pgfpathrectangle{\pgfqpoint{0.481681in}{1.080890in}}{\pgfqpoint{5.785672in}{2.146863in}}%
\pgfusepath{clip}%
\pgfsetrectcap%
\pgfsetroundjoin%
\pgfsetlinewidth{0.100375pt}%
\definecolor{currentstroke}{rgb}{0.827451,0.827451,0.827451}%
\pgfsetstrokecolor{currentstroke}%
\pgfsetdash{}{0pt}%
\pgfpathmoveto{\pgfqpoint{1.105495in}{1.080890in}}%
\pgfpathlineto{\pgfqpoint{1.105495in}{3.227753in}}%
\pgfusepath{stroke}%
\end{pgfscope}%
\begin{pgfscope}%
\pgfsetbuttcap%
\pgfsetroundjoin%
\definecolor{currentfill}{rgb}{0.000000,0.000000,0.000000}%
\pgfsetfillcolor{currentfill}%
\pgfsetlinewidth{0.501875pt}%
\definecolor{currentstroke}{rgb}{0.000000,0.000000,0.000000}%
\pgfsetstrokecolor{currentstroke}%
\pgfsetdash{}{0pt}%
\pgfsys@defobject{currentmarker}{\pgfqpoint{0.000000in}{0.000000in}}{\pgfqpoint{0.000000in}{0.020833in}}{%
\pgfpathmoveto{\pgfqpoint{0.000000in}{0.000000in}}%
\pgfpathlineto{\pgfqpoint{0.000000in}{0.020833in}}%
\pgfusepath{stroke,fill}%
}%
\begin{pgfscope}%
\pgfsys@transformshift{1.105495in}{1.080890in}%
\pgfsys@useobject{currentmarker}{}%
\end{pgfscope}%
\end{pgfscope}%
\begin{pgfscope}%
\pgfsetbuttcap%
\pgfsetroundjoin%
\definecolor{currentfill}{rgb}{0.000000,0.000000,0.000000}%
\pgfsetfillcolor{currentfill}%
\pgfsetlinewidth{0.501875pt}%
\definecolor{currentstroke}{rgb}{0.000000,0.000000,0.000000}%
\pgfsetstrokecolor{currentstroke}%
\pgfsetdash{}{0pt}%
\pgfsys@defobject{currentmarker}{\pgfqpoint{0.000000in}{-0.020833in}}{\pgfqpoint{0.000000in}{0.000000in}}{%
\pgfpathmoveto{\pgfqpoint{0.000000in}{0.000000in}}%
\pgfpathlineto{\pgfqpoint{0.000000in}{-0.020833in}}%
\pgfusepath{stroke,fill}%
}%
\begin{pgfscope}%
\pgfsys@transformshift{1.105495in}{3.227753in}%
\pgfsys@useobject{currentmarker}{}%
\end{pgfscope}%
\end{pgfscope}%
\begin{pgfscope}%
\pgfpathrectangle{\pgfqpoint{0.481681in}{1.080890in}}{\pgfqpoint{5.785672in}{2.146863in}}%
\pgfusepath{clip}%
\pgfsetrectcap%
\pgfsetroundjoin%
\pgfsetlinewidth{0.100375pt}%
\definecolor{currentstroke}{rgb}{0.827451,0.827451,0.827451}%
\pgfsetstrokecolor{currentstroke}%
\pgfsetdash{}{0pt}%
\pgfpathmoveto{\pgfqpoint{1.141016in}{1.080890in}}%
\pgfpathlineto{\pgfqpoint{1.141016in}{3.227753in}}%
\pgfusepath{stroke}%
\end{pgfscope}%
\begin{pgfscope}%
\pgfsetbuttcap%
\pgfsetroundjoin%
\definecolor{currentfill}{rgb}{0.000000,0.000000,0.000000}%
\pgfsetfillcolor{currentfill}%
\pgfsetlinewidth{0.501875pt}%
\definecolor{currentstroke}{rgb}{0.000000,0.000000,0.000000}%
\pgfsetstrokecolor{currentstroke}%
\pgfsetdash{}{0pt}%
\pgfsys@defobject{currentmarker}{\pgfqpoint{0.000000in}{0.000000in}}{\pgfqpoint{0.000000in}{0.020833in}}{%
\pgfpathmoveto{\pgfqpoint{0.000000in}{0.000000in}}%
\pgfpathlineto{\pgfqpoint{0.000000in}{0.020833in}}%
\pgfusepath{stroke,fill}%
}%
\begin{pgfscope}%
\pgfsys@transformshift{1.141016in}{1.080890in}%
\pgfsys@useobject{currentmarker}{}%
\end{pgfscope}%
\end{pgfscope}%
\begin{pgfscope}%
\pgfsetbuttcap%
\pgfsetroundjoin%
\definecolor{currentfill}{rgb}{0.000000,0.000000,0.000000}%
\pgfsetfillcolor{currentfill}%
\pgfsetlinewidth{0.501875pt}%
\definecolor{currentstroke}{rgb}{0.000000,0.000000,0.000000}%
\pgfsetstrokecolor{currentstroke}%
\pgfsetdash{}{0pt}%
\pgfsys@defobject{currentmarker}{\pgfqpoint{0.000000in}{-0.020833in}}{\pgfqpoint{0.000000in}{0.000000in}}{%
\pgfpathmoveto{\pgfqpoint{0.000000in}{0.000000in}}%
\pgfpathlineto{\pgfqpoint{0.000000in}{-0.020833in}}%
\pgfusepath{stroke,fill}%
}%
\begin{pgfscope}%
\pgfsys@transformshift{1.141016in}{3.227753in}%
\pgfsys@useobject{currentmarker}{}%
\end{pgfscope}%
\end{pgfscope}%
\begin{pgfscope}%
\pgfpathrectangle{\pgfqpoint{0.481681in}{1.080890in}}{\pgfqpoint{5.785672in}{2.146863in}}%
\pgfusepath{clip}%
\pgfsetrectcap%
\pgfsetroundjoin%
\pgfsetlinewidth{0.100375pt}%
\definecolor{currentstroke}{rgb}{0.827451,0.827451,0.827451}%
\pgfsetstrokecolor{currentstroke}%
\pgfsetdash{}{0pt}%
\pgfpathmoveto{\pgfqpoint{1.176536in}{1.080890in}}%
\pgfpathlineto{\pgfqpoint{1.176536in}{3.227753in}}%
\pgfusepath{stroke}%
\end{pgfscope}%
\begin{pgfscope}%
\pgfsetbuttcap%
\pgfsetroundjoin%
\definecolor{currentfill}{rgb}{0.000000,0.000000,0.000000}%
\pgfsetfillcolor{currentfill}%
\pgfsetlinewidth{0.501875pt}%
\definecolor{currentstroke}{rgb}{0.000000,0.000000,0.000000}%
\pgfsetstrokecolor{currentstroke}%
\pgfsetdash{}{0pt}%
\pgfsys@defobject{currentmarker}{\pgfqpoint{0.000000in}{0.000000in}}{\pgfqpoint{0.000000in}{0.020833in}}{%
\pgfpathmoveto{\pgfqpoint{0.000000in}{0.000000in}}%
\pgfpathlineto{\pgfqpoint{0.000000in}{0.020833in}}%
\pgfusepath{stroke,fill}%
}%
\begin{pgfscope}%
\pgfsys@transformshift{1.176536in}{1.080890in}%
\pgfsys@useobject{currentmarker}{}%
\end{pgfscope}%
\end{pgfscope}%
\begin{pgfscope}%
\pgfsetbuttcap%
\pgfsetroundjoin%
\definecolor{currentfill}{rgb}{0.000000,0.000000,0.000000}%
\pgfsetfillcolor{currentfill}%
\pgfsetlinewidth{0.501875pt}%
\definecolor{currentstroke}{rgb}{0.000000,0.000000,0.000000}%
\pgfsetstrokecolor{currentstroke}%
\pgfsetdash{}{0pt}%
\pgfsys@defobject{currentmarker}{\pgfqpoint{0.000000in}{-0.020833in}}{\pgfqpoint{0.000000in}{0.000000in}}{%
\pgfpathmoveto{\pgfqpoint{0.000000in}{0.000000in}}%
\pgfpathlineto{\pgfqpoint{0.000000in}{-0.020833in}}%
\pgfusepath{stroke,fill}%
}%
\begin{pgfscope}%
\pgfsys@transformshift{1.176536in}{3.227753in}%
\pgfsys@useobject{currentmarker}{}%
\end{pgfscope}%
\end{pgfscope}%
\begin{pgfscope}%
\pgfpathrectangle{\pgfqpoint{0.481681in}{1.080890in}}{\pgfqpoint{5.785672in}{2.146863in}}%
\pgfusepath{clip}%
\pgfsetrectcap%
\pgfsetroundjoin%
\pgfsetlinewidth{0.100375pt}%
\definecolor{currentstroke}{rgb}{0.827451,0.827451,0.827451}%
\pgfsetstrokecolor{currentstroke}%
\pgfsetdash{}{0pt}%
\pgfpathmoveto{\pgfqpoint{1.212057in}{1.080890in}}%
\pgfpathlineto{\pgfqpoint{1.212057in}{3.227753in}}%
\pgfusepath{stroke}%
\end{pgfscope}%
\begin{pgfscope}%
\pgfsetbuttcap%
\pgfsetroundjoin%
\definecolor{currentfill}{rgb}{0.000000,0.000000,0.000000}%
\pgfsetfillcolor{currentfill}%
\pgfsetlinewidth{0.501875pt}%
\definecolor{currentstroke}{rgb}{0.000000,0.000000,0.000000}%
\pgfsetstrokecolor{currentstroke}%
\pgfsetdash{}{0pt}%
\pgfsys@defobject{currentmarker}{\pgfqpoint{0.000000in}{0.000000in}}{\pgfqpoint{0.000000in}{0.020833in}}{%
\pgfpathmoveto{\pgfqpoint{0.000000in}{0.000000in}}%
\pgfpathlineto{\pgfqpoint{0.000000in}{0.020833in}}%
\pgfusepath{stroke,fill}%
}%
\begin{pgfscope}%
\pgfsys@transformshift{1.212057in}{1.080890in}%
\pgfsys@useobject{currentmarker}{}%
\end{pgfscope}%
\end{pgfscope}%
\begin{pgfscope}%
\pgfsetbuttcap%
\pgfsetroundjoin%
\definecolor{currentfill}{rgb}{0.000000,0.000000,0.000000}%
\pgfsetfillcolor{currentfill}%
\pgfsetlinewidth{0.501875pt}%
\definecolor{currentstroke}{rgb}{0.000000,0.000000,0.000000}%
\pgfsetstrokecolor{currentstroke}%
\pgfsetdash{}{0pt}%
\pgfsys@defobject{currentmarker}{\pgfqpoint{0.000000in}{-0.020833in}}{\pgfqpoint{0.000000in}{0.000000in}}{%
\pgfpathmoveto{\pgfqpoint{0.000000in}{0.000000in}}%
\pgfpathlineto{\pgfqpoint{0.000000in}{-0.020833in}}%
\pgfusepath{stroke,fill}%
}%
\begin{pgfscope}%
\pgfsys@transformshift{1.212057in}{3.227753in}%
\pgfsys@useobject{currentmarker}{}%
\end{pgfscope}%
\end{pgfscope}%
\begin{pgfscope}%
\pgfpathrectangle{\pgfqpoint{0.481681in}{1.080890in}}{\pgfqpoint{5.785672in}{2.146863in}}%
\pgfusepath{clip}%
\pgfsetrectcap%
\pgfsetroundjoin%
\pgfsetlinewidth{0.100375pt}%
\definecolor{currentstroke}{rgb}{0.827451,0.827451,0.827451}%
\pgfsetstrokecolor{currentstroke}%
\pgfsetdash{}{0pt}%
\pgfpathmoveto{\pgfqpoint{1.247577in}{1.080890in}}%
\pgfpathlineto{\pgfqpoint{1.247577in}{3.227753in}}%
\pgfusepath{stroke}%
\end{pgfscope}%
\begin{pgfscope}%
\pgfsetbuttcap%
\pgfsetroundjoin%
\definecolor{currentfill}{rgb}{0.000000,0.000000,0.000000}%
\pgfsetfillcolor{currentfill}%
\pgfsetlinewidth{0.501875pt}%
\definecolor{currentstroke}{rgb}{0.000000,0.000000,0.000000}%
\pgfsetstrokecolor{currentstroke}%
\pgfsetdash{}{0pt}%
\pgfsys@defobject{currentmarker}{\pgfqpoint{0.000000in}{0.000000in}}{\pgfqpoint{0.000000in}{0.020833in}}{%
\pgfpathmoveto{\pgfqpoint{0.000000in}{0.000000in}}%
\pgfpathlineto{\pgfqpoint{0.000000in}{0.020833in}}%
\pgfusepath{stroke,fill}%
}%
\begin{pgfscope}%
\pgfsys@transformshift{1.247577in}{1.080890in}%
\pgfsys@useobject{currentmarker}{}%
\end{pgfscope}%
\end{pgfscope}%
\begin{pgfscope}%
\pgfsetbuttcap%
\pgfsetroundjoin%
\definecolor{currentfill}{rgb}{0.000000,0.000000,0.000000}%
\pgfsetfillcolor{currentfill}%
\pgfsetlinewidth{0.501875pt}%
\definecolor{currentstroke}{rgb}{0.000000,0.000000,0.000000}%
\pgfsetstrokecolor{currentstroke}%
\pgfsetdash{}{0pt}%
\pgfsys@defobject{currentmarker}{\pgfqpoint{0.000000in}{-0.020833in}}{\pgfqpoint{0.000000in}{0.000000in}}{%
\pgfpathmoveto{\pgfqpoint{0.000000in}{0.000000in}}%
\pgfpathlineto{\pgfqpoint{0.000000in}{-0.020833in}}%
\pgfusepath{stroke,fill}%
}%
\begin{pgfscope}%
\pgfsys@transformshift{1.247577in}{3.227753in}%
\pgfsys@useobject{currentmarker}{}%
\end{pgfscope}%
\end{pgfscope}%
\begin{pgfscope}%
\pgfpathrectangle{\pgfqpoint{0.481681in}{1.080890in}}{\pgfqpoint{5.785672in}{2.146863in}}%
\pgfusepath{clip}%
\pgfsetrectcap%
\pgfsetroundjoin%
\pgfsetlinewidth{0.100375pt}%
\definecolor{currentstroke}{rgb}{0.827451,0.827451,0.827451}%
\pgfsetstrokecolor{currentstroke}%
\pgfsetdash{}{0pt}%
\pgfpathmoveto{\pgfqpoint{1.283098in}{1.080890in}}%
\pgfpathlineto{\pgfqpoint{1.283098in}{3.227753in}}%
\pgfusepath{stroke}%
\end{pgfscope}%
\begin{pgfscope}%
\pgfsetbuttcap%
\pgfsetroundjoin%
\definecolor{currentfill}{rgb}{0.000000,0.000000,0.000000}%
\pgfsetfillcolor{currentfill}%
\pgfsetlinewidth{0.501875pt}%
\definecolor{currentstroke}{rgb}{0.000000,0.000000,0.000000}%
\pgfsetstrokecolor{currentstroke}%
\pgfsetdash{}{0pt}%
\pgfsys@defobject{currentmarker}{\pgfqpoint{0.000000in}{0.000000in}}{\pgfqpoint{0.000000in}{0.020833in}}{%
\pgfpathmoveto{\pgfqpoint{0.000000in}{0.000000in}}%
\pgfpathlineto{\pgfqpoint{0.000000in}{0.020833in}}%
\pgfusepath{stroke,fill}%
}%
\begin{pgfscope}%
\pgfsys@transformshift{1.283098in}{1.080890in}%
\pgfsys@useobject{currentmarker}{}%
\end{pgfscope}%
\end{pgfscope}%
\begin{pgfscope}%
\pgfsetbuttcap%
\pgfsetroundjoin%
\definecolor{currentfill}{rgb}{0.000000,0.000000,0.000000}%
\pgfsetfillcolor{currentfill}%
\pgfsetlinewidth{0.501875pt}%
\definecolor{currentstroke}{rgb}{0.000000,0.000000,0.000000}%
\pgfsetstrokecolor{currentstroke}%
\pgfsetdash{}{0pt}%
\pgfsys@defobject{currentmarker}{\pgfqpoint{0.000000in}{-0.020833in}}{\pgfqpoint{0.000000in}{0.000000in}}{%
\pgfpathmoveto{\pgfqpoint{0.000000in}{0.000000in}}%
\pgfpathlineto{\pgfqpoint{0.000000in}{-0.020833in}}%
\pgfusepath{stroke,fill}%
}%
\begin{pgfscope}%
\pgfsys@transformshift{1.283098in}{3.227753in}%
\pgfsys@useobject{currentmarker}{}%
\end{pgfscope}%
\end{pgfscope}%
\begin{pgfscope}%
\pgfpathrectangle{\pgfqpoint{0.481681in}{1.080890in}}{\pgfqpoint{5.785672in}{2.146863in}}%
\pgfusepath{clip}%
\pgfsetrectcap%
\pgfsetroundjoin%
\pgfsetlinewidth{0.100375pt}%
\definecolor{currentstroke}{rgb}{0.827451,0.827451,0.827451}%
\pgfsetstrokecolor{currentstroke}%
\pgfsetdash{}{0pt}%
\pgfpathmoveto{\pgfqpoint{1.318618in}{1.080890in}}%
\pgfpathlineto{\pgfqpoint{1.318618in}{3.227753in}}%
\pgfusepath{stroke}%
\end{pgfscope}%
\begin{pgfscope}%
\pgfsetbuttcap%
\pgfsetroundjoin%
\definecolor{currentfill}{rgb}{0.000000,0.000000,0.000000}%
\pgfsetfillcolor{currentfill}%
\pgfsetlinewidth{0.501875pt}%
\definecolor{currentstroke}{rgb}{0.000000,0.000000,0.000000}%
\pgfsetstrokecolor{currentstroke}%
\pgfsetdash{}{0pt}%
\pgfsys@defobject{currentmarker}{\pgfqpoint{0.000000in}{0.000000in}}{\pgfqpoint{0.000000in}{0.020833in}}{%
\pgfpathmoveto{\pgfqpoint{0.000000in}{0.000000in}}%
\pgfpathlineto{\pgfqpoint{0.000000in}{0.020833in}}%
\pgfusepath{stroke,fill}%
}%
\begin{pgfscope}%
\pgfsys@transformshift{1.318618in}{1.080890in}%
\pgfsys@useobject{currentmarker}{}%
\end{pgfscope}%
\end{pgfscope}%
\begin{pgfscope}%
\pgfsetbuttcap%
\pgfsetroundjoin%
\definecolor{currentfill}{rgb}{0.000000,0.000000,0.000000}%
\pgfsetfillcolor{currentfill}%
\pgfsetlinewidth{0.501875pt}%
\definecolor{currentstroke}{rgb}{0.000000,0.000000,0.000000}%
\pgfsetstrokecolor{currentstroke}%
\pgfsetdash{}{0pt}%
\pgfsys@defobject{currentmarker}{\pgfqpoint{0.000000in}{-0.020833in}}{\pgfqpoint{0.000000in}{0.000000in}}{%
\pgfpathmoveto{\pgfqpoint{0.000000in}{0.000000in}}%
\pgfpathlineto{\pgfqpoint{0.000000in}{-0.020833in}}%
\pgfusepath{stroke,fill}%
}%
\begin{pgfscope}%
\pgfsys@transformshift{1.318618in}{3.227753in}%
\pgfsys@useobject{currentmarker}{}%
\end{pgfscope}%
\end{pgfscope}%
\begin{pgfscope}%
\pgfpathrectangle{\pgfqpoint{0.481681in}{1.080890in}}{\pgfqpoint{5.785672in}{2.146863in}}%
\pgfusepath{clip}%
\pgfsetrectcap%
\pgfsetroundjoin%
\pgfsetlinewidth{0.100375pt}%
\definecolor{currentstroke}{rgb}{0.827451,0.827451,0.827451}%
\pgfsetstrokecolor{currentstroke}%
\pgfsetdash{}{0pt}%
\pgfpathmoveto{\pgfqpoint{1.354139in}{1.080890in}}%
\pgfpathlineto{\pgfqpoint{1.354139in}{3.227753in}}%
\pgfusepath{stroke}%
\end{pgfscope}%
\begin{pgfscope}%
\pgfsetbuttcap%
\pgfsetroundjoin%
\definecolor{currentfill}{rgb}{0.000000,0.000000,0.000000}%
\pgfsetfillcolor{currentfill}%
\pgfsetlinewidth{0.501875pt}%
\definecolor{currentstroke}{rgb}{0.000000,0.000000,0.000000}%
\pgfsetstrokecolor{currentstroke}%
\pgfsetdash{}{0pt}%
\pgfsys@defobject{currentmarker}{\pgfqpoint{0.000000in}{0.000000in}}{\pgfqpoint{0.000000in}{0.020833in}}{%
\pgfpathmoveto{\pgfqpoint{0.000000in}{0.000000in}}%
\pgfpathlineto{\pgfqpoint{0.000000in}{0.020833in}}%
\pgfusepath{stroke,fill}%
}%
\begin{pgfscope}%
\pgfsys@transformshift{1.354139in}{1.080890in}%
\pgfsys@useobject{currentmarker}{}%
\end{pgfscope}%
\end{pgfscope}%
\begin{pgfscope}%
\pgfsetbuttcap%
\pgfsetroundjoin%
\definecolor{currentfill}{rgb}{0.000000,0.000000,0.000000}%
\pgfsetfillcolor{currentfill}%
\pgfsetlinewidth{0.501875pt}%
\definecolor{currentstroke}{rgb}{0.000000,0.000000,0.000000}%
\pgfsetstrokecolor{currentstroke}%
\pgfsetdash{}{0pt}%
\pgfsys@defobject{currentmarker}{\pgfqpoint{0.000000in}{-0.020833in}}{\pgfqpoint{0.000000in}{0.000000in}}{%
\pgfpathmoveto{\pgfqpoint{0.000000in}{0.000000in}}%
\pgfpathlineto{\pgfqpoint{0.000000in}{-0.020833in}}%
\pgfusepath{stroke,fill}%
}%
\begin{pgfscope}%
\pgfsys@transformshift{1.354139in}{3.227753in}%
\pgfsys@useobject{currentmarker}{}%
\end{pgfscope}%
\end{pgfscope}%
\begin{pgfscope}%
\pgfpathrectangle{\pgfqpoint{0.481681in}{1.080890in}}{\pgfqpoint{5.785672in}{2.146863in}}%
\pgfusepath{clip}%
\pgfsetrectcap%
\pgfsetroundjoin%
\pgfsetlinewidth{0.100375pt}%
\definecolor{currentstroke}{rgb}{0.827451,0.827451,0.827451}%
\pgfsetstrokecolor{currentstroke}%
\pgfsetdash{}{0pt}%
\pgfpathmoveto{\pgfqpoint{1.389660in}{1.080890in}}%
\pgfpathlineto{\pgfqpoint{1.389660in}{3.227753in}}%
\pgfusepath{stroke}%
\end{pgfscope}%
\begin{pgfscope}%
\pgfsetbuttcap%
\pgfsetroundjoin%
\definecolor{currentfill}{rgb}{0.000000,0.000000,0.000000}%
\pgfsetfillcolor{currentfill}%
\pgfsetlinewidth{0.501875pt}%
\definecolor{currentstroke}{rgb}{0.000000,0.000000,0.000000}%
\pgfsetstrokecolor{currentstroke}%
\pgfsetdash{}{0pt}%
\pgfsys@defobject{currentmarker}{\pgfqpoint{0.000000in}{0.000000in}}{\pgfqpoint{0.000000in}{0.020833in}}{%
\pgfpathmoveto{\pgfqpoint{0.000000in}{0.000000in}}%
\pgfpathlineto{\pgfqpoint{0.000000in}{0.020833in}}%
\pgfusepath{stroke,fill}%
}%
\begin{pgfscope}%
\pgfsys@transformshift{1.389660in}{1.080890in}%
\pgfsys@useobject{currentmarker}{}%
\end{pgfscope}%
\end{pgfscope}%
\begin{pgfscope}%
\pgfsetbuttcap%
\pgfsetroundjoin%
\definecolor{currentfill}{rgb}{0.000000,0.000000,0.000000}%
\pgfsetfillcolor{currentfill}%
\pgfsetlinewidth{0.501875pt}%
\definecolor{currentstroke}{rgb}{0.000000,0.000000,0.000000}%
\pgfsetstrokecolor{currentstroke}%
\pgfsetdash{}{0pt}%
\pgfsys@defobject{currentmarker}{\pgfqpoint{0.000000in}{-0.020833in}}{\pgfqpoint{0.000000in}{0.000000in}}{%
\pgfpathmoveto{\pgfqpoint{0.000000in}{0.000000in}}%
\pgfpathlineto{\pgfqpoint{0.000000in}{-0.020833in}}%
\pgfusepath{stroke,fill}%
}%
\begin{pgfscope}%
\pgfsys@transformshift{1.389660in}{3.227753in}%
\pgfsys@useobject{currentmarker}{}%
\end{pgfscope}%
\end{pgfscope}%
\begin{pgfscope}%
\pgfpathrectangle{\pgfqpoint{0.481681in}{1.080890in}}{\pgfqpoint{5.785672in}{2.146863in}}%
\pgfusepath{clip}%
\pgfsetrectcap%
\pgfsetroundjoin%
\pgfsetlinewidth{0.100375pt}%
\definecolor{currentstroke}{rgb}{0.827451,0.827451,0.827451}%
\pgfsetstrokecolor{currentstroke}%
\pgfsetdash{}{0pt}%
\pgfpathmoveto{\pgfqpoint{1.425180in}{1.080890in}}%
\pgfpathlineto{\pgfqpoint{1.425180in}{3.227753in}}%
\pgfusepath{stroke}%
\end{pgfscope}%
\begin{pgfscope}%
\pgfsetbuttcap%
\pgfsetroundjoin%
\definecolor{currentfill}{rgb}{0.000000,0.000000,0.000000}%
\pgfsetfillcolor{currentfill}%
\pgfsetlinewidth{0.501875pt}%
\definecolor{currentstroke}{rgb}{0.000000,0.000000,0.000000}%
\pgfsetstrokecolor{currentstroke}%
\pgfsetdash{}{0pt}%
\pgfsys@defobject{currentmarker}{\pgfqpoint{0.000000in}{0.000000in}}{\pgfqpoint{0.000000in}{0.020833in}}{%
\pgfpathmoveto{\pgfqpoint{0.000000in}{0.000000in}}%
\pgfpathlineto{\pgfqpoint{0.000000in}{0.020833in}}%
\pgfusepath{stroke,fill}%
}%
\begin{pgfscope}%
\pgfsys@transformshift{1.425180in}{1.080890in}%
\pgfsys@useobject{currentmarker}{}%
\end{pgfscope}%
\end{pgfscope}%
\begin{pgfscope}%
\pgfsetbuttcap%
\pgfsetroundjoin%
\definecolor{currentfill}{rgb}{0.000000,0.000000,0.000000}%
\pgfsetfillcolor{currentfill}%
\pgfsetlinewidth{0.501875pt}%
\definecolor{currentstroke}{rgb}{0.000000,0.000000,0.000000}%
\pgfsetstrokecolor{currentstroke}%
\pgfsetdash{}{0pt}%
\pgfsys@defobject{currentmarker}{\pgfqpoint{0.000000in}{-0.020833in}}{\pgfqpoint{0.000000in}{0.000000in}}{%
\pgfpathmoveto{\pgfqpoint{0.000000in}{0.000000in}}%
\pgfpathlineto{\pgfqpoint{0.000000in}{-0.020833in}}%
\pgfusepath{stroke,fill}%
}%
\begin{pgfscope}%
\pgfsys@transformshift{1.425180in}{3.227753in}%
\pgfsys@useobject{currentmarker}{}%
\end{pgfscope}%
\end{pgfscope}%
\begin{pgfscope}%
\pgfpathrectangle{\pgfqpoint{0.481681in}{1.080890in}}{\pgfqpoint{5.785672in}{2.146863in}}%
\pgfusepath{clip}%
\pgfsetrectcap%
\pgfsetroundjoin%
\pgfsetlinewidth{0.100375pt}%
\definecolor{currentstroke}{rgb}{0.827451,0.827451,0.827451}%
\pgfsetstrokecolor{currentstroke}%
\pgfsetdash{}{0pt}%
\pgfpathmoveto{\pgfqpoint{1.460701in}{1.080890in}}%
\pgfpathlineto{\pgfqpoint{1.460701in}{3.227753in}}%
\pgfusepath{stroke}%
\end{pgfscope}%
\begin{pgfscope}%
\pgfsetbuttcap%
\pgfsetroundjoin%
\definecolor{currentfill}{rgb}{0.000000,0.000000,0.000000}%
\pgfsetfillcolor{currentfill}%
\pgfsetlinewidth{0.501875pt}%
\definecolor{currentstroke}{rgb}{0.000000,0.000000,0.000000}%
\pgfsetstrokecolor{currentstroke}%
\pgfsetdash{}{0pt}%
\pgfsys@defobject{currentmarker}{\pgfqpoint{0.000000in}{0.000000in}}{\pgfqpoint{0.000000in}{0.020833in}}{%
\pgfpathmoveto{\pgfqpoint{0.000000in}{0.000000in}}%
\pgfpathlineto{\pgfqpoint{0.000000in}{0.020833in}}%
\pgfusepath{stroke,fill}%
}%
\begin{pgfscope}%
\pgfsys@transformshift{1.460701in}{1.080890in}%
\pgfsys@useobject{currentmarker}{}%
\end{pgfscope}%
\end{pgfscope}%
\begin{pgfscope}%
\pgfsetbuttcap%
\pgfsetroundjoin%
\definecolor{currentfill}{rgb}{0.000000,0.000000,0.000000}%
\pgfsetfillcolor{currentfill}%
\pgfsetlinewidth{0.501875pt}%
\definecolor{currentstroke}{rgb}{0.000000,0.000000,0.000000}%
\pgfsetstrokecolor{currentstroke}%
\pgfsetdash{}{0pt}%
\pgfsys@defobject{currentmarker}{\pgfqpoint{0.000000in}{-0.020833in}}{\pgfqpoint{0.000000in}{0.000000in}}{%
\pgfpathmoveto{\pgfqpoint{0.000000in}{0.000000in}}%
\pgfpathlineto{\pgfqpoint{0.000000in}{-0.020833in}}%
\pgfusepath{stroke,fill}%
}%
\begin{pgfscope}%
\pgfsys@transformshift{1.460701in}{3.227753in}%
\pgfsys@useobject{currentmarker}{}%
\end{pgfscope}%
\end{pgfscope}%
\begin{pgfscope}%
\pgfpathrectangle{\pgfqpoint{0.481681in}{1.080890in}}{\pgfqpoint{5.785672in}{2.146863in}}%
\pgfusepath{clip}%
\pgfsetrectcap%
\pgfsetroundjoin%
\pgfsetlinewidth{0.100375pt}%
\definecolor{currentstroke}{rgb}{0.827451,0.827451,0.827451}%
\pgfsetstrokecolor{currentstroke}%
\pgfsetdash{}{0pt}%
\pgfpathmoveto{\pgfqpoint{1.531742in}{1.080890in}}%
\pgfpathlineto{\pgfqpoint{1.531742in}{3.227753in}}%
\pgfusepath{stroke}%
\end{pgfscope}%
\begin{pgfscope}%
\pgfsetbuttcap%
\pgfsetroundjoin%
\definecolor{currentfill}{rgb}{0.000000,0.000000,0.000000}%
\pgfsetfillcolor{currentfill}%
\pgfsetlinewidth{0.501875pt}%
\definecolor{currentstroke}{rgb}{0.000000,0.000000,0.000000}%
\pgfsetstrokecolor{currentstroke}%
\pgfsetdash{}{0pt}%
\pgfsys@defobject{currentmarker}{\pgfqpoint{0.000000in}{0.000000in}}{\pgfqpoint{0.000000in}{0.020833in}}{%
\pgfpathmoveto{\pgfqpoint{0.000000in}{0.000000in}}%
\pgfpathlineto{\pgfqpoint{0.000000in}{0.020833in}}%
\pgfusepath{stroke,fill}%
}%
\begin{pgfscope}%
\pgfsys@transformshift{1.531742in}{1.080890in}%
\pgfsys@useobject{currentmarker}{}%
\end{pgfscope}%
\end{pgfscope}%
\begin{pgfscope}%
\pgfsetbuttcap%
\pgfsetroundjoin%
\definecolor{currentfill}{rgb}{0.000000,0.000000,0.000000}%
\pgfsetfillcolor{currentfill}%
\pgfsetlinewidth{0.501875pt}%
\definecolor{currentstroke}{rgb}{0.000000,0.000000,0.000000}%
\pgfsetstrokecolor{currentstroke}%
\pgfsetdash{}{0pt}%
\pgfsys@defobject{currentmarker}{\pgfqpoint{0.000000in}{-0.020833in}}{\pgfqpoint{0.000000in}{0.000000in}}{%
\pgfpathmoveto{\pgfqpoint{0.000000in}{0.000000in}}%
\pgfpathlineto{\pgfqpoint{0.000000in}{-0.020833in}}%
\pgfusepath{stroke,fill}%
}%
\begin{pgfscope}%
\pgfsys@transformshift{1.531742in}{3.227753in}%
\pgfsys@useobject{currentmarker}{}%
\end{pgfscope}%
\end{pgfscope}%
\begin{pgfscope}%
\pgfpathrectangle{\pgfqpoint{0.481681in}{1.080890in}}{\pgfqpoint{5.785672in}{2.146863in}}%
\pgfusepath{clip}%
\pgfsetrectcap%
\pgfsetroundjoin%
\pgfsetlinewidth{0.100375pt}%
\definecolor{currentstroke}{rgb}{0.827451,0.827451,0.827451}%
\pgfsetstrokecolor{currentstroke}%
\pgfsetdash{}{0pt}%
\pgfpathmoveto{\pgfqpoint{1.567262in}{1.080890in}}%
\pgfpathlineto{\pgfqpoint{1.567262in}{3.227753in}}%
\pgfusepath{stroke}%
\end{pgfscope}%
\begin{pgfscope}%
\pgfsetbuttcap%
\pgfsetroundjoin%
\definecolor{currentfill}{rgb}{0.000000,0.000000,0.000000}%
\pgfsetfillcolor{currentfill}%
\pgfsetlinewidth{0.501875pt}%
\definecolor{currentstroke}{rgb}{0.000000,0.000000,0.000000}%
\pgfsetstrokecolor{currentstroke}%
\pgfsetdash{}{0pt}%
\pgfsys@defobject{currentmarker}{\pgfqpoint{0.000000in}{0.000000in}}{\pgfqpoint{0.000000in}{0.020833in}}{%
\pgfpathmoveto{\pgfqpoint{0.000000in}{0.000000in}}%
\pgfpathlineto{\pgfqpoint{0.000000in}{0.020833in}}%
\pgfusepath{stroke,fill}%
}%
\begin{pgfscope}%
\pgfsys@transformshift{1.567262in}{1.080890in}%
\pgfsys@useobject{currentmarker}{}%
\end{pgfscope}%
\end{pgfscope}%
\begin{pgfscope}%
\pgfsetbuttcap%
\pgfsetroundjoin%
\definecolor{currentfill}{rgb}{0.000000,0.000000,0.000000}%
\pgfsetfillcolor{currentfill}%
\pgfsetlinewidth{0.501875pt}%
\definecolor{currentstroke}{rgb}{0.000000,0.000000,0.000000}%
\pgfsetstrokecolor{currentstroke}%
\pgfsetdash{}{0pt}%
\pgfsys@defobject{currentmarker}{\pgfqpoint{0.000000in}{-0.020833in}}{\pgfqpoint{0.000000in}{0.000000in}}{%
\pgfpathmoveto{\pgfqpoint{0.000000in}{0.000000in}}%
\pgfpathlineto{\pgfqpoint{0.000000in}{-0.020833in}}%
\pgfusepath{stroke,fill}%
}%
\begin{pgfscope}%
\pgfsys@transformshift{1.567262in}{3.227753in}%
\pgfsys@useobject{currentmarker}{}%
\end{pgfscope}%
\end{pgfscope}%
\begin{pgfscope}%
\pgfpathrectangle{\pgfqpoint{0.481681in}{1.080890in}}{\pgfqpoint{5.785672in}{2.146863in}}%
\pgfusepath{clip}%
\pgfsetrectcap%
\pgfsetroundjoin%
\pgfsetlinewidth{0.100375pt}%
\definecolor{currentstroke}{rgb}{0.827451,0.827451,0.827451}%
\pgfsetstrokecolor{currentstroke}%
\pgfsetdash{}{0pt}%
\pgfpathmoveto{\pgfqpoint{1.602783in}{1.080890in}}%
\pgfpathlineto{\pgfqpoint{1.602783in}{3.227753in}}%
\pgfusepath{stroke}%
\end{pgfscope}%
\begin{pgfscope}%
\pgfsetbuttcap%
\pgfsetroundjoin%
\definecolor{currentfill}{rgb}{0.000000,0.000000,0.000000}%
\pgfsetfillcolor{currentfill}%
\pgfsetlinewidth{0.501875pt}%
\definecolor{currentstroke}{rgb}{0.000000,0.000000,0.000000}%
\pgfsetstrokecolor{currentstroke}%
\pgfsetdash{}{0pt}%
\pgfsys@defobject{currentmarker}{\pgfqpoint{0.000000in}{0.000000in}}{\pgfqpoint{0.000000in}{0.020833in}}{%
\pgfpathmoveto{\pgfqpoint{0.000000in}{0.000000in}}%
\pgfpathlineto{\pgfqpoint{0.000000in}{0.020833in}}%
\pgfusepath{stroke,fill}%
}%
\begin{pgfscope}%
\pgfsys@transformshift{1.602783in}{1.080890in}%
\pgfsys@useobject{currentmarker}{}%
\end{pgfscope}%
\end{pgfscope}%
\begin{pgfscope}%
\pgfsetbuttcap%
\pgfsetroundjoin%
\definecolor{currentfill}{rgb}{0.000000,0.000000,0.000000}%
\pgfsetfillcolor{currentfill}%
\pgfsetlinewidth{0.501875pt}%
\definecolor{currentstroke}{rgb}{0.000000,0.000000,0.000000}%
\pgfsetstrokecolor{currentstroke}%
\pgfsetdash{}{0pt}%
\pgfsys@defobject{currentmarker}{\pgfqpoint{0.000000in}{-0.020833in}}{\pgfqpoint{0.000000in}{0.000000in}}{%
\pgfpathmoveto{\pgfqpoint{0.000000in}{0.000000in}}%
\pgfpathlineto{\pgfqpoint{0.000000in}{-0.020833in}}%
\pgfusepath{stroke,fill}%
}%
\begin{pgfscope}%
\pgfsys@transformshift{1.602783in}{3.227753in}%
\pgfsys@useobject{currentmarker}{}%
\end{pgfscope}%
\end{pgfscope}%
\begin{pgfscope}%
\pgfpathrectangle{\pgfqpoint{0.481681in}{1.080890in}}{\pgfqpoint{5.785672in}{2.146863in}}%
\pgfusepath{clip}%
\pgfsetrectcap%
\pgfsetroundjoin%
\pgfsetlinewidth{0.100375pt}%
\definecolor{currentstroke}{rgb}{0.827451,0.827451,0.827451}%
\pgfsetstrokecolor{currentstroke}%
\pgfsetdash{}{0pt}%
\pgfpathmoveto{\pgfqpoint{1.638303in}{1.080890in}}%
\pgfpathlineto{\pgfqpoint{1.638303in}{3.227753in}}%
\pgfusepath{stroke}%
\end{pgfscope}%
\begin{pgfscope}%
\pgfsetbuttcap%
\pgfsetroundjoin%
\definecolor{currentfill}{rgb}{0.000000,0.000000,0.000000}%
\pgfsetfillcolor{currentfill}%
\pgfsetlinewidth{0.501875pt}%
\definecolor{currentstroke}{rgb}{0.000000,0.000000,0.000000}%
\pgfsetstrokecolor{currentstroke}%
\pgfsetdash{}{0pt}%
\pgfsys@defobject{currentmarker}{\pgfqpoint{0.000000in}{0.000000in}}{\pgfqpoint{0.000000in}{0.020833in}}{%
\pgfpathmoveto{\pgfqpoint{0.000000in}{0.000000in}}%
\pgfpathlineto{\pgfqpoint{0.000000in}{0.020833in}}%
\pgfusepath{stroke,fill}%
}%
\begin{pgfscope}%
\pgfsys@transformshift{1.638303in}{1.080890in}%
\pgfsys@useobject{currentmarker}{}%
\end{pgfscope}%
\end{pgfscope}%
\begin{pgfscope}%
\pgfsetbuttcap%
\pgfsetroundjoin%
\definecolor{currentfill}{rgb}{0.000000,0.000000,0.000000}%
\pgfsetfillcolor{currentfill}%
\pgfsetlinewidth{0.501875pt}%
\definecolor{currentstroke}{rgb}{0.000000,0.000000,0.000000}%
\pgfsetstrokecolor{currentstroke}%
\pgfsetdash{}{0pt}%
\pgfsys@defobject{currentmarker}{\pgfqpoint{0.000000in}{-0.020833in}}{\pgfqpoint{0.000000in}{0.000000in}}{%
\pgfpathmoveto{\pgfqpoint{0.000000in}{0.000000in}}%
\pgfpathlineto{\pgfqpoint{0.000000in}{-0.020833in}}%
\pgfusepath{stroke,fill}%
}%
\begin{pgfscope}%
\pgfsys@transformshift{1.638303in}{3.227753in}%
\pgfsys@useobject{currentmarker}{}%
\end{pgfscope}%
\end{pgfscope}%
\begin{pgfscope}%
\pgfpathrectangle{\pgfqpoint{0.481681in}{1.080890in}}{\pgfqpoint{5.785672in}{2.146863in}}%
\pgfusepath{clip}%
\pgfsetrectcap%
\pgfsetroundjoin%
\pgfsetlinewidth{0.100375pt}%
\definecolor{currentstroke}{rgb}{0.827451,0.827451,0.827451}%
\pgfsetstrokecolor{currentstroke}%
\pgfsetdash{}{0pt}%
\pgfpathmoveto{\pgfqpoint{1.673824in}{1.080890in}}%
\pgfpathlineto{\pgfqpoint{1.673824in}{3.227753in}}%
\pgfusepath{stroke}%
\end{pgfscope}%
\begin{pgfscope}%
\pgfsetbuttcap%
\pgfsetroundjoin%
\definecolor{currentfill}{rgb}{0.000000,0.000000,0.000000}%
\pgfsetfillcolor{currentfill}%
\pgfsetlinewidth{0.501875pt}%
\definecolor{currentstroke}{rgb}{0.000000,0.000000,0.000000}%
\pgfsetstrokecolor{currentstroke}%
\pgfsetdash{}{0pt}%
\pgfsys@defobject{currentmarker}{\pgfqpoint{0.000000in}{0.000000in}}{\pgfqpoint{0.000000in}{0.020833in}}{%
\pgfpathmoveto{\pgfqpoint{0.000000in}{0.000000in}}%
\pgfpathlineto{\pgfqpoint{0.000000in}{0.020833in}}%
\pgfusepath{stroke,fill}%
}%
\begin{pgfscope}%
\pgfsys@transformshift{1.673824in}{1.080890in}%
\pgfsys@useobject{currentmarker}{}%
\end{pgfscope}%
\end{pgfscope}%
\begin{pgfscope}%
\pgfsetbuttcap%
\pgfsetroundjoin%
\definecolor{currentfill}{rgb}{0.000000,0.000000,0.000000}%
\pgfsetfillcolor{currentfill}%
\pgfsetlinewidth{0.501875pt}%
\definecolor{currentstroke}{rgb}{0.000000,0.000000,0.000000}%
\pgfsetstrokecolor{currentstroke}%
\pgfsetdash{}{0pt}%
\pgfsys@defobject{currentmarker}{\pgfqpoint{0.000000in}{-0.020833in}}{\pgfqpoint{0.000000in}{0.000000in}}{%
\pgfpathmoveto{\pgfqpoint{0.000000in}{0.000000in}}%
\pgfpathlineto{\pgfqpoint{0.000000in}{-0.020833in}}%
\pgfusepath{stroke,fill}%
}%
\begin{pgfscope}%
\pgfsys@transformshift{1.673824in}{3.227753in}%
\pgfsys@useobject{currentmarker}{}%
\end{pgfscope}%
\end{pgfscope}%
\begin{pgfscope}%
\pgfpathrectangle{\pgfqpoint{0.481681in}{1.080890in}}{\pgfqpoint{5.785672in}{2.146863in}}%
\pgfusepath{clip}%
\pgfsetrectcap%
\pgfsetroundjoin%
\pgfsetlinewidth{0.100375pt}%
\definecolor{currentstroke}{rgb}{0.827451,0.827451,0.827451}%
\pgfsetstrokecolor{currentstroke}%
\pgfsetdash{}{0pt}%
\pgfpathmoveto{\pgfqpoint{1.709344in}{1.080890in}}%
\pgfpathlineto{\pgfqpoint{1.709344in}{3.227753in}}%
\pgfusepath{stroke}%
\end{pgfscope}%
\begin{pgfscope}%
\pgfsetbuttcap%
\pgfsetroundjoin%
\definecolor{currentfill}{rgb}{0.000000,0.000000,0.000000}%
\pgfsetfillcolor{currentfill}%
\pgfsetlinewidth{0.501875pt}%
\definecolor{currentstroke}{rgb}{0.000000,0.000000,0.000000}%
\pgfsetstrokecolor{currentstroke}%
\pgfsetdash{}{0pt}%
\pgfsys@defobject{currentmarker}{\pgfqpoint{0.000000in}{0.000000in}}{\pgfqpoint{0.000000in}{0.020833in}}{%
\pgfpathmoveto{\pgfqpoint{0.000000in}{0.000000in}}%
\pgfpathlineto{\pgfqpoint{0.000000in}{0.020833in}}%
\pgfusepath{stroke,fill}%
}%
\begin{pgfscope}%
\pgfsys@transformshift{1.709344in}{1.080890in}%
\pgfsys@useobject{currentmarker}{}%
\end{pgfscope}%
\end{pgfscope}%
\begin{pgfscope}%
\pgfsetbuttcap%
\pgfsetroundjoin%
\definecolor{currentfill}{rgb}{0.000000,0.000000,0.000000}%
\pgfsetfillcolor{currentfill}%
\pgfsetlinewidth{0.501875pt}%
\definecolor{currentstroke}{rgb}{0.000000,0.000000,0.000000}%
\pgfsetstrokecolor{currentstroke}%
\pgfsetdash{}{0pt}%
\pgfsys@defobject{currentmarker}{\pgfqpoint{0.000000in}{-0.020833in}}{\pgfqpoint{0.000000in}{0.000000in}}{%
\pgfpathmoveto{\pgfqpoint{0.000000in}{0.000000in}}%
\pgfpathlineto{\pgfqpoint{0.000000in}{-0.020833in}}%
\pgfusepath{stroke,fill}%
}%
\begin{pgfscope}%
\pgfsys@transformshift{1.709344in}{3.227753in}%
\pgfsys@useobject{currentmarker}{}%
\end{pgfscope}%
\end{pgfscope}%
\begin{pgfscope}%
\pgfpathrectangle{\pgfqpoint{0.481681in}{1.080890in}}{\pgfqpoint{5.785672in}{2.146863in}}%
\pgfusepath{clip}%
\pgfsetrectcap%
\pgfsetroundjoin%
\pgfsetlinewidth{0.100375pt}%
\definecolor{currentstroke}{rgb}{0.827451,0.827451,0.827451}%
\pgfsetstrokecolor{currentstroke}%
\pgfsetdash{}{0pt}%
\pgfpathmoveto{\pgfqpoint{1.744865in}{1.080890in}}%
\pgfpathlineto{\pgfqpoint{1.744865in}{3.227753in}}%
\pgfusepath{stroke}%
\end{pgfscope}%
\begin{pgfscope}%
\pgfsetbuttcap%
\pgfsetroundjoin%
\definecolor{currentfill}{rgb}{0.000000,0.000000,0.000000}%
\pgfsetfillcolor{currentfill}%
\pgfsetlinewidth{0.501875pt}%
\definecolor{currentstroke}{rgb}{0.000000,0.000000,0.000000}%
\pgfsetstrokecolor{currentstroke}%
\pgfsetdash{}{0pt}%
\pgfsys@defobject{currentmarker}{\pgfqpoint{0.000000in}{0.000000in}}{\pgfqpoint{0.000000in}{0.020833in}}{%
\pgfpathmoveto{\pgfqpoint{0.000000in}{0.000000in}}%
\pgfpathlineto{\pgfqpoint{0.000000in}{0.020833in}}%
\pgfusepath{stroke,fill}%
}%
\begin{pgfscope}%
\pgfsys@transformshift{1.744865in}{1.080890in}%
\pgfsys@useobject{currentmarker}{}%
\end{pgfscope}%
\end{pgfscope}%
\begin{pgfscope}%
\pgfsetbuttcap%
\pgfsetroundjoin%
\definecolor{currentfill}{rgb}{0.000000,0.000000,0.000000}%
\pgfsetfillcolor{currentfill}%
\pgfsetlinewidth{0.501875pt}%
\definecolor{currentstroke}{rgb}{0.000000,0.000000,0.000000}%
\pgfsetstrokecolor{currentstroke}%
\pgfsetdash{}{0pt}%
\pgfsys@defobject{currentmarker}{\pgfqpoint{0.000000in}{-0.020833in}}{\pgfqpoint{0.000000in}{0.000000in}}{%
\pgfpathmoveto{\pgfqpoint{0.000000in}{0.000000in}}%
\pgfpathlineto{\pgfqpoint{0.000000in}{-0.020833in}}%
\pgfusepath{stroke,fill}%
}%
\begin{pgfscope}%
\pgfsys@transformshift{1.744865in}{3.227753in}%
\pgfsys@useobject{currentmarker}{}%
\end{pgfscope}%
\end{pgfscope}%
\begin{pgfscope}%
\pgfpathrectangle{\pgfqpoint{0.481681in}{1.080890in}}{\pgfqpoint{5.785672in}{2.146863in}}%
\pgfusepath{clip}%
\pgfsetrectcap%
\pgfsetroundjoin%
\pgfsetlinewidth{0.100375pt}%
\definecolor{currentstroke}{rgb}{0.827451,0.827451,0.827451}%
\pgfsetstrokecolor{currentstroke}%
\pgfsetdash{}{0pt}%
\pgfpathmoveto{\pgfqpoint{1.780385in}{1.080890in}}%
\pgfpathlineto{\pgfqpoint{1.780385in}{3.227753in}}%
\pgfusepath{stroke}%
\end{pgfscope}%
\begin{pgfscope}%
\pgfsetbuttcap%
\pgfsetroundjoin%
\definecolor{currentfill}{rgb}{0.000000,0.000000,0.000000}%
\pgfsetfillcolor{currentfill}%
\pgfsetlinewidth{0.501875pt}%
\definecolor{currentstroke}{rgb}{0.000000,0.000000,0.000000}%
\pgfsetstrokecolor{currentstroke}%
\pgfsetdash{}{0pt}%
\pgfsys@defobject{currentmarker}{\pgfqpoint{0.000000in}{0.000000in}}{\pgfqpoint{0.000000in}{0.020833in}}{%
\pgfpathmoveto{\pgfqpoint{0.000000in}{0.000000in}}%
\pgfpathlineto{\pgfqpoint{0.000000in}{0.020833in}}%
\pgfusepath{stroke,fill}%
}%
\begin{pgfscope}%
\pgfsys@transformshift{1.780385in}{1.080890in}%
\pgfsys@useobject{currentmarker}{}%
\end{pgfscope}%
\end{pgfscope}%
\begin{pgfscope}%
\pgfsetbuttcap%
\pgfsetroundjoin%
\definecolor{currentfill}{rgb}{0.000000,0.000000,0.000000}%
\pgfsetfillcolor{currentfill}%
\pgfsetlinewidth{0.501875pt}%
\definecolor{currentstroke}{rgb}{0.000000,0.000000,0.000000}%
\pgfsetstrokecolor{currentstroke}%
\pgfsetdash{}{0pt}%
\pgfsys@defobject{currentmarker}{\pgfqpoint{0.000000in}{-0.020833in}}{\pgfqpoint{0.000000in}{0.000000in}}{%
\pgfpathmoveto{\pgfqpoint{0.000000in}{0.000000in}}%
\pgfpathlineto{\pgfqpoint{0.000000in}{-0.020833in}}%
\pgfusepath{stroke,fill}%
}%
\begin{pgfscope}%
\pgfsys@transformshift{1.780385in}{3.227753in}%
\pgfsys@useobject{currentmarker}{}%
\end{pgfscope}%
\end{pgfscope}%
\begin{pgfscope}%
\pgfpathrectangle{\pgfqpoint{0.481681in}{1.080890in}}{\pgfqpoint{5.785672in}{2.146863in}}%
\pgfusepath{clip}%
\pgfsetrectcap%
\pgfsetroundjoin%
\pgfsetlinewidth{0.100375pt}%
\definecolor{currentstroke}{rgb}{0.827451,0.827451,0.827451}%
\pgfsetstrokecolor{currentstroke}%
\pgfsetdash{}{0pt}%
\pgfpathmoveto{\pgfqpoint{1.815906in}{1.080890in}}%
\pgfpathlineto{\pgfqpoint{1.815906in}{3.227753in}}%
\pgfusepath{stroke}%
\end{pgfscope}%
\begin{pgfscope}%
\pgfsetbuttcap%
\pgfsetroundjoin%
\definecolor{currentfill}{rgb}{0.000000,0.000000,0.000000}%
\pgfsetfillcolor{currentfill}%
\pgfsetlinewidth{0.501875pt}%
\definecolor{currentstroke}{rgb}{0.000000,0.000000,0.000000}%
\pgfsetstrokecolor{currentstroke}%
\pgfsetdash{}{0pt}%
\pgfsys@defobject{currentmarker}{\pgfqpoint{0.000000in}{0.000000in}}{\pgfqpoint{0.000000in}{0.020833in}}{%
\pgfpathmoveto{\pgfqpoint{0.000000in}{0.000000in}}%
\pgfpathlineto{\pgfqpoint{0.000000in}{0.020833in}}%
\pgfusepath{stroke,fill}%
}%
\begin{pgfscope}%
\pgfsys@transformshift{1.815906in}{1.080890in}%
\pgfsys@useobject{currentmarker}{}%
\end{pgfscope}%
\end{pgfscope}%
\begin{pgfscope}%
\pgfsetbuttcap%
\pgfsetroundjoin%
\definecolor{currentfill}{rgb}{0.000000,0.000000,0.000000}%
\pgfsetfillcolor{currentfill}%
\pgfsetlinewidth{0.501875pt}%
\definecolor{currentstroke}{rgb}{0.000000,0.000000,0.000000}%
\pgfsetstrokecolor{currentstroke}%
\pgfsetdash{}{0pt}%
\pgfsys@defobject{currentmarker}{\pgfqpoint{0.000000in}{-0.020833in}}{\pgfqpoint{0.000000in}{0.000000in}}{%
\pgfpathmoveto{\pgfqpoint{0.000000in}{0.000000in}}%
\pgfpathlineto{\pgfqpoint{0.000000in}{-0.020833in}}%
\pgfusepath{stroke,fill}%
}%
\begin{pgfscope}%
\pgfsys@transformshift{1.815906in}{3.227753in}%
\pgfsys@useobject{currentmarker}{}%
\end{pgfscope}%
\end{pgfscope}%
\begin{pgfscope}%
\pgfpathrectangle{\pgfqpoint{0.481681in}{1.080890in}}{\pgfqpoint{5.785672in}{2.146863in}}%
\pgfusepath{clip}%
\pgfsetrectcap%
\pgfsetroundjoin%
\pgfsetlinewidth{0.100375pt}%
\definecolor{currentstroke}{rgb}{0.827451,0.827451,0.827451}%
\pgfsetstrokecolor{currentstroke}%
\pgfsetdash{}{0pt}%
\pgfpathmoveto{\pgfqpoint{1.851426in}{1.080890in}}%
\pgfpathlineto{\pgfqpoint{1.851426in}{3.227753in}}%
\pgfusepath{stroke}%
\end{pgfscope}%
\begin{pgfscope}%
\pgfsetbuttcap%
\pgfsetroundjoin%
\definecolor{currentfill}{rgb}{0.000000,0.000000,0.000000}%
\pgfsetfillcolor{currentfill}%
\pgfsetlinewidth{0.501875pt}%
\definecolor{currentstroke}{rgb}{0.000000,0.000000,0.000000}%
\pgfsetstrokecolor{currentstroke}%
\pgfsetdash{}{0pt}%
\pgfsys@defobject{currentmarker}{\pgfqpoint{0.000000in}{0.000000in}}{\pgfqpoint{0.000000in}{0.020833in}}{%
\pgfpathmoveto{\pgfqpoint{0.000000in}{0.000000in}}%
\pgfpathlineto{\pgfqpoint{0.000000in}{0.020833in}}%
\pgfusepath{stroke,fill}%
}%
\begin{pgfscope}%
\pgfsys@transformshift{1.851426in}{1.080890in}%
\pgfsys@useobject{currentmarker}{}%
\end{pgfscope}%
\end{pgfscope}%
\begin{pgfscope}%
\pgfsetbuttcap%
\pgfsetroundjoin%
\definecolor{currentfill}{rgb}{0.000000,0.000000,0.000000}%
\pgfsetfillcolor{currentfill}%
\pgfsetlinewidth{0.501875pt}%
\definecolor{currentstroke}{rgb}{0.000000,0.000000,0.000000}%
\pgfsetstrokecolor{currentstroke}%
\pgfsetdash{}{0pt}%
\pgfsys@defobject{currentmarker}{\pgfqpoint{0.000000in}{-0.020833in}}{\pgfqpoint{0.000000in}{0.000000in}}{%
\pgfpathmoveto{\pgfqpoint{0.000000in}{0.000000in}}%
\pgfpathlineto{\pgfqpoint{0.000000in}{-0.020833in}}%
\pgfusepath{stroke,fill}%
}%
\begin{pgfscope}%
\pgfsys@transformshift{1.851426in}{3.227753in}%
\pgfsys@useobject{currentmarker}{}%
\end{pgfscope}%
\end{pgfscope}%
\begin{pgfscope}%
\pgfpathrectangle{\pgfqpoint{0.481681in}{1.080890in}}{\pgfqpoint{5.785672in}{2.146863in}}%
\pgfusepath{clip}%
\pgfsetrectcap%
\pgfsetroundjoin%
\pgfsetlinewidth{0.100375pt}%
\definecolor{currentstroke}{rgb}{0.827451,0.827451,0.827451}%
\pgfsetstrokecolor{currentstroke}%
\pgfsetdash{}{0pt}%
\pgfpathmoveto{\pgfqpoint{1.886947in}{1.080890in}}%
\pgfpathlineto{\pgfqpoint{1.886947in}{3.227753in}}%
\pgfusepath{stroke}%
\end{pgfscope}%
\begin{pgfscope}%
\pgfsetbuttcap%
\pgfsetroundjoin%
\definecolor{currentfill}{rgb}{0.000000,0.000000,0.000000}%
\pgfsetfillcolor{currentfill}%
\pgfsetlinewidth{0.501875pt}%
\definecolor{currentstroke}{rgb}{0.000000,0.000000,0.000000}%
\pgfsetstrokecolor{currentstroke}%
\pgfsetdash{}{0pt}%
\pgfsys@defobject{currentmarker}{\pgfqpoint{0.000000in}{0.000000in}}{\pgfqpoint{0.000000in}{0.020833in}}{%
\pgfpathmoveto{\pgfqpoint{0.000000in}{0.000000in}}%
\pgfpathlineto{\pgfqpoint{0.000000in}{0.020833in}}%
\pgfusepath{stroke,fill}%
}%
\begin{pgfscope}%
\pgfsys@transformshift{1.886947in}{1.080890in}%
\pgfsys@useobject{currentmarker}{}%
\end{pgfscope}%
\end{pgfscope}%
\begin{pgfscope}%
\pgfsetbuttcap%
\pgfsetroundjoin%
\definecolor{currentfill}{rgb}{0.000000,0.000000,0.000000}%
\pgfsetfillcolor{currentfill}%
\pgfsetlinewidth{0.501875pt}%
\definecolor{currentstroke}{rgb}{0.000000,0.000000,0.000000}%
\pgfsetstrokecolor{currentstroke}%
\pgfsetdash{}{0pt}%
\pgfsys@defobject{currentmarker}{\pgfqpoint{0.000000in}{-0.020833in}}{\pgfqpoint{0.000000in}{0.000000in}}{%
\pgfpathmoveto{\pgfqpoint{0.000000in}{0.000000in}}%
\pgfpathlineto{\pgfqpoint{0.000000in}{-0.020833in}}%
\pgfusepath{stroke,fill}%
}%
\begin{pgfscope}%
\pgfsys@transformshift{1.886947in}{3.227753in}%
\pgfsys@useobject{currentmarker}{}%
\end{pgfscope}%
\end{pgfscope}%
\begin{pgfscope}%
\pgfpathrectangle{\pgfqpoint{0.481681in}{1.080890in}}{\pgfqpoint{5.785672in}{2.146863in}}%
\pgfusepath{clip}%
\pgfsetrectcap%
\pgfsetroundjoin%
\pgfsetlinewidth{0.100375pt}%
\definecolor{currentstroke}{rgb}{0.827451,0.827451,0.827451}%
\pgfsetstrokecolor{currentstroke}%
\pgfsetdash{}{0pt}%
\pgfpathmoveto{\pgfqpoint{1.957988in}{1.080890in}}%
\pgfpathlineto{\pgfqpoint{1.957988in}{3.227753in}}%
\pgfusepath{stroke}%
\end{pgfscope}%
\begin{pgfscope}%
\pgfsetbuttcap%
\pgfsetroundjoin%
\definecolor{currentfill}{rgb}{0.000000,0.000000,0.000000}%
\pgfsetfillcolor{currentfill}%
\pgfsetlinewidth{0.501875pt}%
\definecolor{currentstroke}{rgb}{0.000000,0.000000,0.000000}%
\pgfsetstrokecolor{currentstroke}%
\pgfsetdash{}{0pt}%
\pgfsys@defobject{currentmarker}{\pgfqpoint{0.000000in}{0.000000in}}{\pgfqpoint{0.000000in}{0.020833in}}{%
\pgfpathmoveto{\pgfqpoint{0.000000in}{0.000000in}}%
\pgfpathlineto{\pgfqpoint{0.000000in}{0.020833in}}%
\pgfusepath{stroke,fill}%
}%
\begin{pgfscope}%
\pgfsys@transformshift{1.957988in}{1.080890in}%
\pgfsys@useobject{currentmarker}{}%
\end{pgfscope}%
\end{pgfscope}%
\begin{pgfscope}%
\pgfsetbuttcap%
\pgfsetroundjoin%
\definecolor{currentfill}{rgb}{0.000000,0.000000,0.000000}%
\pgfsetfillcolor{currentfill}%
\pgfsetlinewidth{0.501875pt}%
\definecolor{currentstroke}{rgb}{0.000000,0.000000,0.000000}%
\pgfsetstrokecolor{currentstroke}%
\pgfsetdash{}{0pt}%
\pgfsys@defobject{currentmarker}{\pgfqpoint{0.000000in}{-0.020833in}}{\pgfqpoint{0.000000in}{0.000000in}}{%
\pgfpathmoveto{\pgfqpoint{0.000000in}{0.000000in}}%
\pgfpathlineto{\pgfqpoint{0.000000in}{-0.020833in}}%
\pgfusepath{stroke,fill}%
}%
\begin{pgfscope}%
\pgfsys@transformshift{1.957988in}{3.227753in}%
\pgfsys@useobject{currentmarker}{}%
\end{pgfscope}%
\end{pgfscope}%
\begin{pgfscope}%
\pgfpathrectangle{\pgfqpoint{0.481681in}{1.080890in}}{\pgfqpoint{5.785672in}{2.146863in}}%
\pgfusepath{clip}%
\pgfsetrectcap%
\pgfsetroundjoin%
\pgfsetlinewidth{0.100375pt}%
\definecolor{currentstroke}{rgb}{0.827451,0.827451,0.827451}%
\pgfsetstrokecolor{currentstroke}%
\pgfsetdash{}{0pt}%
\pgfpathmoveto{\pgfqpoint{1.993508in}{1.080890in}}%
\pgfpathlineto{\pgfqpoint{1.993508in}{3.227753in}}%
\pgfusepath{stroke}%
\end{pgfscope}%
\begin{pgfscope}%
\pgfsetbuttcap%
\pgfsetroundjoin%
\definecolor{currentfill}{rgb}{0.000000,0.000000,0.000000}%
\pgfsetfillcolor{currentfill}%
\pgfsetlinewidth{0.501875pt}%
\definecolor{currentstroke}{rgb}{0.000000,0.000000,0.000000}%
\pgfsetstrokecolor{currentstroke}%
\pgfsetdash{}{0pt}%
\pgfsys@defobject{currentmarker}{\pgfqpoint{0.000000in}{0.000000in}}{\pgfqpoint{0.000000in}{0.020833in}}{%
\pgfpathmoveto{\pgfqpoint{0.000000in}{0.000000in}}%
\pgfpathlineto{\pgfqpoint{0.000000in}{0.020833in}}%
\pgfusepath{stroke,fill}%
}%
\begin{pgfscope}%
\pgfsys@transformshift{1.993508in}{1.080890in}%
\pgfsys@useobject{currentmarker}{}%
\end{pgfscope}%
\end{pgfscope}%
\begin{pgfscope}%
\pgfsetbuttcap%
\pgfsetroundjoin%
\definecolor{currentfill}{rgb}{0.000000,0.000000,0.000000}%
\pgfsetfillcolor{currentfill}%
\pgfsetlinewidth{0.501875pt}%
\definecolor{currentstroke}{rgb}{0.000000,0.000000,0.000000}%
\pgfsetstrokecolor{currentstroke}%
\pgfsetdash{}{0pt}%
\pgfsys@defobject{currentmarker}{\pgfqpoint{0.000000in}{-0.020833in}}{\pgfqpoint{0.000000in}{0.000000in}}{%
\pgfpathmoveto{\pgfqpoint{0.000000in}{0.000000in}}%
\pgfpathlineto{\pgfqpoint{0.000000in}{-0.020833in}}%
\pgfusepath{stroke,fill}%
}%
\begin{pgfscope}%
\pgfsys@transformshift{1.993508in}{3.227753in}%
\pgfsys@useobject{currentmarker}{}%
\end{pgfscope}%
\end{pgfscope}%
\begin{pgfscope}%
\pgfpathrectangle{\pgfqpoint{0.481681in}{1.080890in}}{\pgfqpoint{5.785672in}{2.146863in}}%
\pgfusepath{clip}%
\pgfsetrectcap%
\pgfsetroundjoin%
\pgfsetlinewidth{0.100375pt}%
\definecolor{currentstroke}{rgb}{0.827451,0.827451,0.827451}%
\pgfsetstrokecolor{currentstroke}%
\pgfsetdash{}{0pt}%
\pgfpathmoveto{\pgfqpoint{2.029029in}{1.080890in}}%
\pgfpathlineto{\pgfqpoint{2.029029in}{3.227753in}}%
\pgfusepath{stroke}%
\end{pgfscope}%
\begin{pgfscope}%
\pgfsetbuttcap%
\pgfsetroundjoin%
\definecolor{currentfill}{rgb}{0.000000,0.000000,0.000000}%
\pgfsetfillcolor{currentfill}%
\pgfsetlinewidth{0.501875pt}%
\definecolor{currentstroke}{rgb}{0.000000,0.000000,0.000000}%
\pgfsetstrokecolor{currentstroke}%
\pgfsetdash{}{0pt}%
\pgfsys@defobject{currentmarker}{\pgfqpoint{0.000000in}{0.000000in}}{\pgfqpoint{0.000000in}{0.020833in}}{%
\pgfpathmoveto{\pgfqpoint{0.000000in}{0.000000in}}%
\pgfpathlineto{\pgfqpoint{0.000000in}{0.020833in}}%
\pgfusepath{stroke,fill}%
}%
\begin{pgfscope}%
\pgfsys@transformshift{2.029029in}{1.080890in}%
\pgfsys@useobject{currentmarker}{}%
\end{pgfscope}%
\end{pgfscope}%
\begin{pgfscope}%
\pgfsetbuttcap%
\pgfsetroundjoin%
\definecolor{currentfill}{rgb}{0.000000,0.000000,0.000000}%
\pgfsetfillcolor{currentfill}%
\pgfsetlinewidth{0.501875pt}%
\definecolor{currentstroke}{rgb}{0.000000,0.000000,0.000000}%
\pgfsetstrokecolor{currentstroke}%
\pgfsetdash{}{0pt}%
\pgfsys@defobject{currentmarker}{\pgfqpoint{0.000000in}{-0.020833in}}{\pgfqpoint{0.000000in}{0.000000in}}{%
\pgfpathmoveto{\pgfqpoint{0.000000in}{0.000000in}}%
\pgfpathlineto{\pgfqpoint{0.000000in}{-0.020833in}}%
\pgfusepath{stroke,fill}%
}%
\begin{pgfscope}%
\pgfsys@transformshift{2.029029in}{3.227753in}%
\pgfsys@useobject{currentmarker}{}%
\end{pgfscope}%
\end{pgfscope}%
\begin{pgfscope}%
\pgfpathrectangle{\pgfqpoint{0.481681in}{1.080890in}}{\pgfqpoint{5.785672in}{2.146863in}}%
\pgfusepath{clip}%
\pgfsetrectcap%
\pgfsetroundjoin%
\pgfsetlinewidth{0.100375pt}%
\definecolor{currentstroke}{rgb}{0.827451,0.827451,0.827451}%
\pgfsetstrokecolor{currentstroke}%
\pgfsetdash{}{0pt}%
\pgfpathmoveto{\pgfqpoint{2.064550in}{1.080890in}}%
\pgfpathlineto{\pgfqpoint{2.064550in}{3.227753in}}%
\pgfusepath{stroke}%
\end{pgfscope}%
\begin{pgfscope}%
\pgfsetbuttcap%
\pgfsetroundjoin%
\definecolor{currentfill}{rgb}{0.000000,0.000000,0.000000}%
\pgfsetfillcolor{currentfill}%
\pgfsetlinewidth{0.501875pt}%
\definecolor{currentstroke}{rgb}{0.000000,0.000000,0.000000}%
\pgfsetstrokecolor{currentstroke}%
\pgfsetdash{}{0pt}%
\pgfsys@defobject{currentmarker}{\pgfqpoint{0.000000in}{0.000000in}}{\pgfqpoint{0.000000in}{0.020833in}}{%
\pgfpathmoveto{\pgfqpoint{0.000000in}{0.000000in}}%
\pgfpathlineto{\pgfqpoint{0.000000in}{0.020833in}}%
\pgfusepath{stroke,fill}%
}%
\begin{pgfscope}%
\pgfsys@transformshift{2.064550in}{1.080890in}%
\pgfsys@useobject{currentmarker}{}%
\end{pgfscope}%
\end{pgfscope}%
\begin{pgfscope}%
\pgfsetbuttcap%
\pgfsetroundjoin%
\definecolor{currentfill}{rgb}{0.000000,0.000000,0.000000}%
\pgfsetfillcolor{currentfill}%
\pgfsetlinewidth{0.501875pt}%
\definecolor{currentstroke}{rgb}{0.000000,0.000000,0.000000}%
\pgfsetstrokecolor{currentstroke}%
\pgfsetdash{}{0pt}%
\pgfsys@defobject{currentmarker}{\pgfqpoint{0.000000in}{-0.020833in}}{\pgfqpoint{0.000000in}{0.000000in}}{%
\pgfpathmoveto{\pgfqpoint{0.000000in}{0.000000in}}%
\pgfpathlineto{\pgfqpoint{0.000000in}{-0.020833in}}%
\pgfusepath{stroke,fill}%
}%
\begin{pgfscope}%
\pgfsys@transformshift{2.064550in}{3.227753in}%
\pgfsys@useobject{currentmarker}{}%
\end{pgfscope}%
\end{pgfscope}%
\begin{pgfscope}%
\pgfpathrectangle{\pgfqpoint{0.481681in}{1.080890in}}{\pgfqpoint{5.785672in}{2.146863in}}%
\pgfusepath{clip}%
\pgfsetrectcap%
\pgfsetroundjoin%
\pgfsetlinewidth{0.100375pt}%
\definecolor{currentstroke}{rgb}{0.827451,0.827451,0.827451}%
\pgfsetstrokecolor{currentstroke}%
\pgfsetdash{}{0pt}%
\pgfpathmoveto{\pgfqpoint{2.100070in}{1.080890in}}%
\pgfpathlineto{\pgfqpoint{2.100070in}{3.227753in}}%
\pgfusepath{stroke}%
\end{pgfscope}%
\begin{pgfscope}%
\pgfsetbuttcap%
\pgfsetroundjoin%
\definecolor{currentfill}{rgb}{0.000000,0.000000,0.000000}%
\pgfsetfillcolor{currentfill}%
\pgfsetlinewidth{0.501875pt}%
\definecolor{currentstroke}{rgb}{0.000000,0.000000,0.000000}%
\pgfsetstrokecolor{currentstroke}%
\pgfsetdash{}{0pt}%
\pgfsys@defobject{currentmarker}{\pgfqpoint{0.000000in}{0.000000in}}{\pgfqpoint{0.000000in}{0.020833in}}{%
\pgfpathmoveto{\pgfqpoint{0.000000in}{0.000000in}}%
\pgfpathlineto{\pgfqpoint{0.000000in}{0.020833in}}%
\pgfusepath{stroke,fill}%
}%
\begin{pgfscope}%
\pgfsys@transformshift{2.100070in}{1.080890in}%
\pgfsys@useobject{currentmarker}{}%
\end{pgfscope}%
\end{pgfscope}%
\begin{pgfscope}%
\pgfsetbuttcap%
\pgfsetroundjoin%
\definecolor{currentfill}{rgb}{0.000000,0.000000,0.000000}%
\pgfsetfillcolor{currentfill}%
\pgfsetlinewidth{0.501875pt}%
\definecolor{currentstroke}{rgb}{0.000000,0.000000,0.000000}%
\pgfsetstrokecolor{currentstroke}%
\pgfsetdash{}{0pt}%
\pgfsys@defobject{currentmarker}{\pgfqpoint{0.000000in}{-0.020833in}}{\pgfqpoint{0.000000in}{0.000000in}}{%
\pgfpathmoveto{\pgfqpoint{0.000000in}{0.000000in}}%
\pgfpathlineto{\pgfqpoint{0.000000in}{-0.020833in}}%
\pgfusepath{stroke,fill}%
}%
\begin{pgfscope}%
\pgfsys@transformshift{2.100070in}{3.227753in}%
\pgfsys@useobject{currentmarker}{}%
\end{pgfscope}%
\end{pgfscope}%
\begin{pgfscope}%
\pgfpathrectangle{\pgfqpoint{0.481681in}{1.080890in}}{\pgfqpoint{5.785672in}{2.146863in}}%
\pgfusepath{clip}%
\pgfsetrectcap%
\pgfsetroundjoin%
\pgfsetlinewidth{0.100375pt}%
\definecolor{currentstroke}{rgb}{0.827451,0.827451,0.827451}%
\pgfsetstrokecolor{currentstroke}%
\pgfsetdash{}{0pt}%
\pgfpathmoveto{\pgfqpoint{2.135591in}{1.080890in}}%
\pgfpathlineto{\pgfqpoint{2.135591in}{3.227753in}}%
\pgfusepath{stroke}%
\end{pgfscope}%
\begin{pgfscope}%
\pgfsetbuttcap%
\pgfsetroundjoin%
\definecolor{currentfill}{rgb}{0.000000,0.000000,0.000000}%
\pgfsetfillcolor{currentfill}%
\pgfsetlinewidth{0.501875pt}%
\definecolor{currentstroke}{rgb}{0.000000,0.000000,0.000000}%
\pgfsetstrokecolor{currentstroke}%
\pgfsetdash{}{0pt}%
\pgfsys@defobject{currentmarker}{\pgfqpoint{0.000000in}{0.000000in}}{\pgfqpoint{0.000000in}{0.020833in}}{%
\pgfpathmoveto{\pgfqpoint{0.000000in}{0.000000in}}%
\pgfpathlineto{\pgfqpoint{0.000000in}{0.020833in}}%
\pgfusepath{stroke,fill}%
}%
\begin{pgfscope}%
\pgfsys@transformshift{2.135591in}{1.080890in}%
\pgfsys@useobject{currentmarker}{}%
\end{pgfscope}%
\end{pgfscope}%
\begin{pgfscope}%
\pgfsetbuttcap%
\pgfsetroundjoin%
\definecolor{currentfill}{rgb}{0.000000,0.000000,0.000000}%
\pgfsetfillcolor{currentfill}%
\pgfsetlinewidth{0.501875pt}%
\definecolor{currentstroke}{rgb}{0.000000,0.000000,0.000000}%
\pgfsetstrokecolor{currentstroke}%
\pgfsetdash{}{0pt}%
\pgfsys@defobject{currentmarker}{\pgfqpoint{0.000000in}{-0.020833in}}{\pgfqpoint{0.000000in}{0.000000in}}{%
\pgfpathmoveto{\pgfqpoint{0.000000in}{0.000000in}}%
\pgfpathlineto{\pgfqpoint{0.000000in}{-0.020833in}}%
\pgfusepath{stroke,fill}%
}%
\begin{pgfscope}%
\pgfsys@transformshift{2.135591in}{3.227753in}%
\pgfsys@useobject{currentmarker}{}%
\end{pgfscope}%
\end{pgfscope}%
\begin{pgfscope}%
\pgfpathrectangle{\pgfqpoint{0.481681in}{1.080890in}}{\pgfqpoint{5.785672in}{2.146863in}}%
\pgfusepath{clip}%
\pgfsetrectcap%
\pgfsetroundjoin%
\pgfsetlinewidth{0.100375pt}%
\definecolor{currentstroke}{rgb}{0.827451,0.827451,0.827451}%
\pgfsetstrokecolor{currentstroke}%
\pgfsetdash{}{0pt}%
\pgfpathmoveto{\pgfqpoint{2.171111in}{1.080890in}}%
\pgfpathlineto{\pgfqpoint{2.171111in}{3.227753in}}%
\pgfusepath{stroke}%
\end{pgfscope}%
\begin{pgfscope}%
\pgfsetbuttcap%
\pgfsetroundjoin%
\definecolor{currentfill}{rgb}{0.000000,0.000000,0.000000}%
\pgfsetfillcolor{currentfill}%
\pgfsetlinewidth{0.501875pt}%
\definecolor{currentstroke}{rgb}{0.000000,0.000000,0.000000}%
\pgfsetstrokecolor{currentstroke}%
\pgfsetdash{}{0pt}%
\pgfsys@defobject{currentmarker}{\pgfqpoint{0.000000in}{0.000000in}}{\pgfqpoint{0.000000in}{0.020833in}}{%
\pgfpathmoveto{\pgfqpoint{0.000000in}{0.000000in}}%
\pgfpathlineto{\pgfqpoint{0.000000in}{0.020833in}}%
\pgfusepath{stroke,fill}%
}%
\begin{pgfscope}%
\pgfsys@transformshift{2.171111in}{1.080890in}%
\pgfsys@useobject{currentmarker}{}%
\end{pgfscope}%
\end{pgfscope}%
\begin{pgfscope}%
\pgfsetbuttcap%
\pgfsetroundjoin%
\definecolor{currentfill}{rgb}{0.000000,0.000000,0.000000}%
\pgfsetfillcolor{currentfill}%
\pgfsetlinewidth{0.501875pt}%
\definecolor{currentstroke}{rgb}{0.000000,0.000000,0.000000}%
\pgfsetstrokecolor{currentstroke}%
\pgfsetdash{}{0pt}%
\pgfsys@defobject{currentmarker}{\pgfqpoint{0.000000in}{-0.020833in}}{\pgfqpoint{0.000000in}{0.000000in}}{%
\pgfpathmoveto{\pgfqpoint{0.000000in}{0.000000in}}%
\pgfpathlineto{\pgfqpoint{0.000000in}{-0.020833in}}%
\pgfusepath{stroke,fill}%
}%
\begin{pgfscope}%
\pgfsys@transformshift{2.171111in}{3.227753in}%
\pgfsys@useobject{currentmarker}{}%
\end{pgfscope}%
\end{pgfscope}%
\begin{pgfscope}%
\pgfpathrectangle{\pgfqpoint{0.481681in}{1.080890in}}{\pgfqpoint{5.785672in}{2.146863in}}%
\pgfusepath{clip}%
\pgfsetrectcap%
\pgfsetroundjoin%
\pgfsetlinewidth{0.100375pt}%
\definecolor{currentstroke}{rgb}{0.827451,0.827451,0.827451}%
\pgfsetstrokecolor{currentstroke}%
\pgfsetdash{}{0pt}%
\pgfpathmoveto{\pgfqpoint{2.206632in}{1.080890in}}%
\pgfpathlineto{\pgfqpoint{2.206632in}{3.227753in}}%
\pgfusepath{stroke}%
\end{pgfscope}%
\begin{pgfscope}%
\pgfsetbuttcap%
\pgfsetroundjoin%
\definecolor{currentfill}{rgb}{0.000000,0.000000,0.000000}%
\pgfsetfillcolor{currentfill}%
\pgfsetlinewidth{0.501875pt}%
\definecolor{currentstroke}{rgb}{0.000000,0.000000,0.000000}%
\pgfsetstrokecolor{currentstroke}%
\pgfsetdash{}{0pt}%
\pgfsys@defobject{currentmarker}{\pgfqpoint{0.000000in}{0.000000in}}{\pgfqpoint{0.000000in}{0.020833in}}{%
\pgfpathmoveto{\pgfqpoint{0.000000in}{0.000000in}}%
\pgfpathlineto{\pgfqpoint{0.000000in}{0.020833in}}%
\pgfusepath{stroke,fill}%
}%
\begin{pgfscope}%
\pgfsys@transformshift{2.206632in}{1.080890in}%
\pgfsys@useobject{currentmarker}{}%
\end{pgfscope}%
\end{pgfscope}%
\begin{pgfscope}%
\pgfsetbuttcap%
\pgfsetroundjoin%
\definecolor{currentfill}{rgb}{0.000000,0.000000,0.000000}%
\pgfsetfillcolor{currentfill}%
\pgfsetlinewidth{0.501875pt}%
\definecolor{currentstroke}{rgb}{0.000000,0.000000,0.000000}%
\pgfsetstrokecolor{currentstroke}%
\pgfsetdash{}{0pt}%
\pgfsys@defobject{currentmarker}{\pgfqpoint{0.000000in}{-0.020833in}}{\pgfqpoint{0.000000in}{0.000000in}}{%
\pgfpathmoveto{\pgfqpoint{0.000000in}{0.000000in}}%
\pgfpathlineto{\pgfqpoint{0.000000in}{-0.020833in}}%
\pgfusepath{stroke,fill}%
}%
\begin{pgfscope}%
\pgfsys@transformshift{2.206632in}{3.227753in}%
\pgfsys@useobject{currentmarker}{}%
\end{pgfscope}%
\end{pgfscope}%
\begin{pgfscope}%
\pgfpathrectangle{\pgfqpoint{0.481681in}{1.080890in}}{\pgfqpoint{5.785672in}{2.146863in}}%
\pgfusepath{clip}%
\pgfsetrectcap%
\pgfsetroundjoin%
\pgfsetlinewidth{0.100375pt}%
\definecolor{currentstroke}{rgb}{0.827451,0.827451,0.827451}%
\pgfsetstrokecolor{currentstroke}%
\pgfsetdash{}{0pt}%
\pgfpathmoveto{\pgfqpoint{2.242152in}{1.080890in}}%
\pgfpathlineto{\pgfqpoint{2.242152in}{3.227753in}}%
\pgfusepath{stroke}%
\end{pgfscope}%
\begin{pgfscope}%
\pgfsetbuttcap%
\pgfsetroundjoin%
\definecolor{currentfill}{rgb}{0.000000,0.000000,0.000000}%
\pgfsetfillcolor{currentfill}%
\pgfsetlinewidth{0.501875pt}%
\definecolor{currentstroke}{rgb}{0.000000,0.000000,0.000000}%
\pgfsetstrokecolor{currentstroke}%
\pgfsetdash{}{0pt}%
\pgfsys@defobject{currentmarker}{\pgfqpoint{0.000000in}{0.000000in}}{\pgfqpoint{0.000000in}{0.020833in}}{%
\pgfpathmoveto{\pgfqpoint{0.000000in}{0.000000in}}%
\pgfpathlineto{\pgfqpoint{0.000000in}{0.020833in}}%
\pgfusepath{stroke,fill}%
}%
\begin{pgfscope}%
\pgfsys@transformshift{2.242152in}{1.080890in}%
\pgfsys@useobject{currentmarker}{}%
\end{pgfscope}%
\end{pgfscope}%
\begin{pgfscope}%
\pgfsetbuttcap%
\pgfsetroundjoin%
\definecolor{currentfill}{rgb}{0.000000,0.000000,0.000000}%
\pgfsetfillcolor{currentfill}%
\pgfsetlinewidth{0.501875pt}%
\definecolor{currentstroke}{rgb}{0.000000,0.000000,0.000000}%
\pgfsetstrokecolor{currentstroke}%
\pgfsetdash{}{0pt}%
\pgfsys@defobject{currentmarker}{\pgfqpoint{0.000000in}{-0.020833in}}{\pgfqpoint{0.000000in}{0.000000in}}{%
\pgfpathmoveto{\pgfqpoint{0.000000in}{0.000000in}}%
\pgfpathlineto{\pgfqpoint{0.000000in}{-0.020833in}}%
\pgfusepath{stroke,fill}%
}%
\begin{pgfscope}%
\pgfsys@transformshift{2.242152in}{3.227753in}%
\pgfsys@useobject{currentmarker}{}%
\end{pgfscope}%
\end{pgfscope}%
\begin{pgfscope}%
\pgfpathrectangle{\pgfqpoint{0.481681in}{1.080890in}}{\pgfqpoint{5.785672in}{2.146863in}}%
\pgfusepath{clip}%
\pgfsetrectcap%
\pgfsetroundjoin%
\pgfsetlinewidth{0.100375pt}%
\definecolor{currentstroke}{rgb}{0.827451,0.827451,0.827451}%
\pgfsetstrokecolor{currentstroke}%
\pgfsetdash{}{0pt}%
\pgfpathmoveto{\pgfqpoint{2.277673in}{1.080890in}}%
\pgfpathlineto{\pgfqpoint{2.277673in}{3.227753in}}%
\pgfusepath{stroke}%
\end{pgfscope}%
\begin{pgfscope}%
\pgfsetbuttcap%
\pgfsetroundjoin%
\definecolor{currentfill}{rgb}{0.000000,0.000000,0.000000}%
\pgfsetfillcolor{currentfill}%
\pgfsetlinewidth{0.501875pt}%
\definecolor{currentstroke}{rgb}{0.000000,0.000000,0.000000}%
\pgfsetstrokecolor{currentstroke}%
\pgfsetdash{}{0pt}%
\pgfsys@defobject{currentmarker}{\pgfqpoint{0.000000in}{0.000000in}}{\pgfqpoint{0.000000in}{0.020833in}}{%
\pgfpathmoveto{\pgfqpoint{0.000000in}{0.000000in}}%
\pgfpathlineto{\pgfqpoint{0.000000in}{0.020833in}}%
\pgfusepath{stroke,fill}%
}%
\begin{pgfscope}%
\pgfsys@transformshift{2.277673in}{1.080890in}%
\pgfsys@useobject{currentmarker}{}%
\end{pgfscope}%
\end{pgfscope}%
\begin{pgfscope}%
\pgfsetbuttcap%
\pgfsetroundjoin%
\definecolor{currentfill}{rgb}{0.000000,0.000000,0.000000}%
\pgfsetfillcolor{currentfill}%
\pgfsetlinewidth{0.501875pt}%
\definecolor{currentstroke}{rgb}{0.000000,0.000000,0.000000}%
\pgfsetstrokecolor{currentstroke}%
\pgfsetdash{}{0pt}%
\pgfsys@defobject{currentmarker}{\pgfqpoint{0.000000in}{-0.020833in}}{\pgfqpoint{0.000000in}{0.000000in}}{%
\pgfpathmoveto{\pgfqpoint{0.000000in}{0.000000in}}%
\pgfpathlineto{\pgfqpoint{0.000000in}{-0.020833in}}%
\pgfusepath{stroke,fill}%
}%
\begin{pgfscope}%
\pgfsys@transformshift{2.277673in}{3.227753in}%
\pgfsys@useobject{currentmarker}{}%
\end{pgfscope}%
\end{pgfscope}%
\begin{pgfscope}%
\pgfpathrectangle{\pgfqpoint{0.481681in}{1.080890in}}{\pgfqpoint{5.785672in}{2.146863in}}%
\pgfusepath{clip}%
\pgfsetrectcap%
\pgfsetroundjoin%
\pgfsetlinewidth{0.100375pt}%
\definecolor{currentstroke}{rgb}{0.827451,0.827451,0.827451}%
\pgfsetstrokecolor{currentstroke}%
\pgfsetdash{}{0pt}%
\pgfpathmoveto{\pgfqpoint{2.313193in}{1.080890in}}%
\pgfpathlineto{\pgfqpoint{2.313193in}{3.227753in}}%
\pgfusepath{stroke}%
\end{pgfscope}%
\begin{pgfscope}%
\pgfsetbuttcap%
\pgfsetroundjoin%
\definecolor{currentfill}{rgb}{0.000000,0.000000,0.000000}%
\pgfsetfillcolor{currentfill}%
\pgfsetlinewidth{0.501875pt}%
\definecolor{currentstroke}{rgb}{0.000000,0.000000,0.000000}%
\pgfsetstrokecolor{currentstroke}%
\pgfsetdash{}{0pt}%
\pgfsys@defobject{currentmarker}{\pgfqpoint{0.000000in}{0.000000in}}{\pgfqpoint{0.000000in}{0.020833in}}{%
\pgfpathmoveto{\pgfqpoint{0.000000in}{0.000000in}}%
\pgfpathlineto{\pgfqpoint{0.000000in}{0.020833in}}%
\pgfusepath{stroke,fill}%
}%
\begin{pgfscope}%
\pgfsys@transformshift{2.313193in}{1.080890in}%
\pgfsys@useobject{currentmarker}{}%
\end{pgfscope}%
\end{pgfscope}%
\begin{pgfscope}%
\pgfsetbuttcap%
\pgfsetroundjoin%
\definecolor{currentfill}{rgb}{0.000000,0.000000,0.000000}%
\pgfsetfillcolor{currentfill}%
\pgfsetlinewidth{0.501875pt}%
\definecolor{currentstroke}{rgb}{0.000000,0.000000,0.000000}%
\pgfsetstrokecolor{currentstroke}%
\pgfsetdash{}{0pt}%
\pgfsys@defobject{currentmarker}{\pgfqpoint{0.000000in}{-0.020833in}}{\pgfqpoint{0.000000in}{0.000000in}}{%
\pgfpathmoveto{\pgfqpoint{0.000000in}{0.000000in}}%
\pgfpathlineto{\pgfqpoint{0.000000in}{-0.020833in}}%
\pgfusepath{stroke,fill}%
}%
\begin{pgfscope}%
\pgfsys@transformshift{2.313193in}{3.227753in}%
\pgfsys@useobject{currentmarker}{}%
\end{pgfscope}%
\end{pgfscope}%
\begin{pgfscope}%
\pgfpathrectangle{\pgfqpoint{0.481681in}{1.080890in}}{\pgfqpoint{5.785672in}{2.146863in}}%
\pgfusepath{clip}%
\pgfsetrectcap%
\pgfsetroundjoin%
\pgfsetlinewidth{0.100375pt}%
\definecolor{currentstroke}{rgb}{0.827451,0.827451,0.827451}%
\pgfsetstrokecolor{currentstroke}%
\pgfsetdash{}{0pt}%
\pgfpathmoveto{\pgfqpoint{2.384234in}{1.080890in}}%
\pgfpathlineto{\pgfqpoint{2.384234in}{3.227753in}}%
\pgfusepath{stroke}%
\end{pgfscope}%
\begin{pgfscope}%
\pgfsetbuttcap%
\pgfsetroundjoin%
\definecolor{currentfill}{rgb}{0.000000,0.000000,0.000000}%
\pgfsetfillcolor{currentfill}%
\pgfsetlinewidth{0.501875pt}%
\definecolor{currentstroke}{rgb}{0.000000,0.000000,0.000000}%
\pgfsetstrokecolor{currentstroke}%
\pgfsetdash{}{0pt}%
\pgfsys@defobject{currentmarker}{\pgfqpoint{0.000000in}{0.000000in}}{\pgfqpoint{0.000000in}{0.020833in}}{%
\pgfpathmoveto{\pgfqpoint{0.000000in}{0.000000in}}%
\pgfpathlineto{\pgfqpoint{0.000000in}{0.020833in}}%
\pgfusepath{stroke,fill}%
}%
\begin{pgfscope}%
\pgfsys@transformshift{2.384234in}{1.080890in}%
\pgfsys@useobject{currentmarker}{}%
\end{pgfscope}%
\end{pgfscope}%
\begin{pgfscope}%
\pgfsetbuttcap%
\pgfsetroundjoin%
\definecolor{currentfill}{rgb}{0.000000,0.000000,0.000000}%
\pgfsetfillcolor{currentfill}%
\pgfsetlinewidth{0.501875pt}%
\definecolor{currentstroke}{rgb}{0.000000,0.000000,0.000000}%
\pgfsetstrokecolor{currentstroke}%
\pgfsetdash{}{0pt}%
\pgfsys@defobject{currentmarker}{\pgfqpoint{0.000000in}{-0.020833in}}{\pgfqpoint{0.000000in}{0.000000in}}{%
\pgfpathmoveto{\pgfqpoint{0.000000in}{0.000000in}}%
\pgfpathlineto{\pgfqpoint{0.000000in}{-0.020833in}}%
\pgfusepath{stroke,fill}%
}%
\begin{pgfscope}%
\pgfsys@transformshift{2.384234in}{3.227753in}%
\pgfsys@useobject{currentmarker}{}%
\end{pgfscope}%
\end{pgfscope}%
\begin{pgfscope}%
\pgfpathrectangle{\pgfqpoint{0.481681in}{1.080890in}}{\pgfqpoint{5.785672in}{2.146863in}}%
\pgfusepath{clip}%
\pgfsetrectcap%
\pgfsetroundjoin%
\pgfsetlinewidth{0.100375pt}%
\definecolor{currentstroke}{rgb}{0.827451,0.827451,0.827451}%
\pgfsetstrokecolor{currentstroke}%
\pgfsetdash{}{0pt}%
\pgfpathmoveto{\pgfqpoint{2.419755in}{1.080890in}}%
\pgfpathlineto{\pgfqpoint{2.419755in}{3.227753in}}%
\pgfusepath{stroke}%
\end{pgfscope}%
\begin{pgfscope}%
\pgfsetbuttcap%
\pgfsetroundjoin%
\definecolor{currentfill}{rgb}{0.000000,0.000000,0.000000}%
\pgfsetfillcolor{currentfill}%
\pgfsetlinewidth{0.501875pt}%
\definecolor{currentstroke}{rgb}{0.000000,0.000000,0.000000}%
\pgfsetstrokecolor{currentstroke}%
\pgfsetdash{}{0pt}%
\pgfsys@defobject{currentmarker}{\pgfqpoint{0.000000in}{0.000000in}}{\pgfqpoint{0.000000in}{0.020833in}}{%
\pgfpathmoveto{\pgfqpoint{0.000000in}{0.000000in}}%
\pgfpathlineto{\pgfqpoint{0.000000in}{0.020833in}}%
\pgfusepath{stroke,fill}%
}%
\begin{pgfscope}%
\pgfsys@transformshift{2.419755in}{1.080890in}%
\pgfsys@useobject{currentmarker}{}%
\end{pgfscope}%
\end{pgfscope}%
\begin{pgfscope}%
\pgfsetbuttcap%
\pgfsetroundjoin%
\definecolor{currentfill}{rgb}{0.000000,0.000000,0.000000}%
\pgfsetfillcolor{currentfill}%
\pgfsetlinewidth{0.501875pt}%
\definecolor{currentstroke}{rgb}{0.000000,0.000000,0.000000}%
\pgfsetstrokecolor{currentstroke}%
\pgfsetdash{}{0pt}%
\pgfsys@defobject{currentmarker}{\pgfqpoint{0.000000in}{-0.020833in}}{\pgfqpoint{0.000000in}{0.000000in}}{%
\pgfpathmoveto{\pgfqpoint{0.000000in}{0.000000in}}%
\pgfpathlineto{\pgfqpoint{0.000000in}{-0.020833in}}%
\pgfusepath{stroke,fill}%
}%
\begin{pgfscope}%
\pgfsys@transformshift{2.419755in}{3.227753in}%
\pgfsys@useobject{currentmarker}{}%
\end{pgfscope}%
\end{pgfscope}%
\begin{pgfscope}%
\pgfpathrectangle{\pgfqpoint{0.481681in}{1.080890in}}{\pgfqpoint{5.785672in}{2.146863in}}%
\pgfusepath{clip}%
\pgfsetrectcap%
\pgfsetroundjoin%
\pgfsetlinewidth{0.100375pt}%
\definecolor{currentstroke}{rgb}{0.827451,0.827451,0.827451}%
\pgfsetstrokecolor{currentstroke}%
\pgfsetdash{}{0pt}%
\pgfpathmoveto{\pgfqpoint{2.455275in}{1.080890in}}%
\pgfpathlineto{\pgfqpoint{2.455275in}{3.227753in}}%
\pgfusepath{stroke}%
\end{pgfscope}%
\begin{pgfscope}%
\pgfsetbuttcap%
\pgfsetroundjoin%
\definecolor{currentfill}{rgb}{0.000000,0.000000,0.000000}%
\pgfsetfillcolor{currentfill}%
\pgfsetlinewidth{0.501875pt}%
\definecolor{currentstroke}{rgb}{0.000000,0.000000,0.000000}%
\pgfsetstrokecolor{currentstroke}%
\pgfsetdash{}{0pt}%
\pgfsys@defobject{currentmarker}{\pgfqpoint{0.000000in}{0.000000in}}{\pgfqpoint{0.000000in}{0.020833in}}{%
\pgfpathmoveto{\pgfqpoint{0.000000in}{0.000000in}}%
\pgfpathlineto{\pgfqpoint{0.000000in}{0.020833in}}%
\pgfusepath{stroke,fill}%
}%
\begin{pgfscope}%
\pgfsys@transformshift{2.455275in}{1.080890in}%
\pgfsys@useobject{currentmarker}{}%
\end{pgfscope}%
\end{pgfscope}%
\begin{pgfscope}%
\pgfsetbuttcap%
\pgfsetroundjoin%
\definecolor{currentfill}{rgb}{0.000000,0.000000,0.000000}%
\pgfsetfillcolor{currentfill}%
\pgfsetlinewidth{0.501875pt}%
\definecolor{currentstroke}{rgb}{0.000000,0.000000,0.000000}%
\pgfsetstrokecolor{currentstroke}%
\pgfsetdash{}{0pt}%
\pgfsys@defobject{currentmarker}{\pgfqpoint{0.000000in}{-0.020833in}}{\pgfqpoint{0.000000in}{0.000000in}}{%
\pgfpathmoveto{\pgfqpoint{0.000000in}{0.000000in}}%
\pgfpathlineto{\pgfqpoint{0.000000in}{-0.020833in}}%
\pgfusepath{stroke,fill}%
}%
\begin{pgfscope}%
\pgfsys@transformshift{2.455275in}{3.227753in}%
\pgfsys@useobject{currentmarker}{}%
\end{pgfscope}%
\end{pgfscope}%
\begin{pgfscope}%
\pgfpathrectangle{\pgfqpoint{0.481681in}{1.080890in}}{\pgfqpoint{5.785672in}{2.146863in}}%
\pgfusepath{clip}%
\pgfsetrectcap%
\pgfsetroundjoin%
\pgfsetlinewidth{0.100375pt}%
\definecolor{currentstroke}{rgb}{0.827451,0.827451,0.827451}%
\pgfsetstrokecolor{currentstroke}%
\pgfsetdash{}{0pt}%
\pgfpathmoveto{\pgfqpoint{2.490796in}{1.080890in}}%
\pgfpathlineto{\pgfqpoint{2.490796in}{3.227753in}}%
\pgfusepath{stroke}%
\end{pgfscope}%
\begin{pgfscope}%
\pgfsetbuttcap%
\pgfsetroundjoin%
\definecolor{currentfill}{rgb}{0.000000,0.000000,0.000000}%
\pgfsetfillcolor{currentfill}%
\pgfsetlinewidth{0.501875pt}%
\definecolor{currentstroke}{rgb}{0.000000,0.000000,0.000000}%
\pgfsetstrokecolor{currentstroke}%
\pgfsetdash{}{0pt}%
\pgfsys@defobject{currentmarker}{\pgfqpoint{0.000000in}{0.000000in}}{\pgfqpoint{0.000000in}{0.020833in}}{%
\pgfpathmoveto{\pgfqpoint{0.000000in}{0.000000in}}%
\pgfpathlineto{\pgfqpoint{0.000000in}{0.020833in}}%
\pgfusepath{stroke,fill}%
}%
\begin{pgfscope}%
\pgfsys@transformshift{2.490796in}{1.080890in}%
\pgfsys@useobject{currentmarker}{}%
\end{pgfscope}%
\end{pgfscope}%
\begin{pgfscope}%
\pgfsetbuttcap%
\pgfsetroundjoin%
\definecolor{currentfill}{rgb}{0.000000,0.000000,0.000000}%
\pgfsetfillcolor{currentfill}%
\pgfsetlinewidth{0.501875pt}%
\definecolor{currentstroke}{rgb}{0.000000,0.000000,0.000000}%
\pgfsetstrokecolor{currentstroke}%
\pgfsetdash{}{0pt}%
\pgfsys@defobject{currentmarker}{\pgfqpoint{0.000000in}{-0.020833in}}{\pgfqpoint{0.000000in}{0.000000in}}{%
\pgfpathmoveto{\pgfqpoint{0.000000in}{0.000000in}}%
\pgfpathlineto{\pgfqpoint{0.000000in}{-0.020833in}}%
\pgfusepath{stroke,fill}%
}%
\begin{pgfscope}%
\pgfsys@transformshift{2.490796in}{3.227753in}%
\pgfsys@useobject{currentmarker}{}%
\end{pgfscope}%
\end{pgfscope}%
\begin{pgfscope}%
\pgfpathrectangle{\pgfqpoint{0.481681in}{1.080890in}}{\pgfqpoint{5.785672in}{2.146863in}}%
\pgfusepath{clip}%
\pgfsetrectcap%
\pgfsetroundjoin%
\pgfsetlinewidth{0.100375pt}%
\definecolor{currentstroke}{rgb}{0.827451,0.827451,0.827451}%
\pgfsetstrokecolor{currentstroke}%
\pgfsetdash{}{0pt}%
\pgfpathmoveto{\pgfqpoint{2.526316in}{1.080890in}}%
\pgfpathlineto{\pgfqpoint{2.526316in}{3.227753in}}%
\pgfusepath{stroke}%
\end{pgfscope}%
\begin{pgfscope}%
\pgfsetbuttcap%
\pgfsetroundjoin%
\definecolor{currentfill}{rgb}{0.000000,0.000000,0.000000}%
\pgfsetfillcolor{currentfill}%
\pgfsetlinewidth{0.501875pt}%
\definecolor{currentstroke}{rgb}{0.000000,0.000000,0.000000}%
\pgfsetstrokecolor{currentstroke}%
\pgfsetdash{}{0pt}%
\pgfsys@defobject{currentmarker}{\pgfqpoint{0.000000in}{0.000000in}}{\pgfqpoint{0.000000in}{0.020833in}}{%
\pgfpathmoveto{\pgfqpoint{0.000000in}{0.000000in}}%
\pgfpathlineto{\pgfqpoint{0.000000in}{0.020833in}}%
\pgfusepath{stroke,fill}%
}%
\begin{pgfscope}%
\pgfsys@transformshift{2.526316in}{1.080890in}%
\pgfsys@useobject{currentmarker}{}%
\end{pgfscope}%
\end{pgfscope}%
\begin{pgfscope}%
\pgfsetbuttcap%
\pgfsetroundjoin%
\definecolor{currentfill}{rgb}{0.000000,0.000000,0.000000}%
\pgfsetfillcolor{currentfill}%
\pgfsetlinewidth{0.501875pt}%
\definecolor{currentstroke}{rgb}{0.000000,0.000000,0.000000}%
\pgfsetstrokecolor{currentstroke}%
\pgfsetdash{}{0pt}%
\pgfsys@defobject{currentmarker}{\pgfqpoint{0.000000in}{-0.020833in}}{\pgfqpoint{0.000000in}{0.000000in}}{%
\pgfpathmoveto{\pgfqpoint{0.000000in}{0.000000in}}%
\pgfpathlineto{\pgfqpoint{0.000000in}{-0.020833in}}%
\pgfusepath{stroke,fill}%
}%
\begin{pgfscope}%
\pgfsys@transformshift{2.526316in}{3.227753in}%
\pgfsys@useobject{currentmarker}{}%
\end{pgfscope}%
\end{pgfscope}%
\begin{pgfscope}%
\pgfpathrectangle{\pgfqpoint{0.481681in}{1.080890in}}{\pgfqpoint{5.785672in}{2.146863in}}%
\pgfusepath{clip}%
\pgfsetrectcap%
\pgfsetroundjoin%
\pgfsetlinewidth{0.100375pt}%
\definecolor{currentstroke}{rgb}{0.827451,0.827451,0.827451}%
\pgfsetstrokecolor{currentstroke}%
\pgfsetdash{}{0pt}%
\pgfpathmoveto{\pgfqpoint{2.561837in}{1.080890in}}%
\pgfpathlineto{\pgfqpoint{2.561837in}{3.227753in}}%
\pgfusepath{stroke}%
\end{pgfscope}%
\begin{pgfscope}%
\pgfsetbuttcap%
\pgfsetroundjoin%
\definecolor{currentfill}{rgb}{0.000000,0.000000,0.000000}%
\pgfsetfillcolor{currentfill}%
\pgfsetlinewidth{0.501875pt}%
\definecolor{currentstroke}{rgb}{0.000000,0.000000,0.000000}%
\pgfsetstrokecolor{currentstroke}%
\pgfsetdash{}{0pt}%
\pgfsys@defobject{currentmarker}{\pgfqpoint{0.000000in}{0.000000in}}{\pgfqpoint{0.000000in}{0.020833in}}{%
\pgfpathmoveto{\pgfqpoint{0.000000in}{0.000000in}}%
\pgfpathlineto{\pgfqpoint{0.000000in}{0.020833in}}%
\pgfusepath{stroke,fill}%
}%
\begin{pgfscope}%
\pgfsys@transformshift{2.561837in}{1.080890in}%
\pgfsys@useobject{currentmarker}{}%
\end{pgfscope}%
\end{pgfscope}%
\begin{pgfscope}%
\pgfsetbuttcap%
\pgfsetroundjoin%
\definecolor{currentfill}{rgb}{0.000000,0.000000,0.000000}%
\pgfsetfillcolor{currentfill}%
\pgfsetlinewidth{0.501875pt}%
\definecolor{currentstroke}{rgb}{0.000000,0.000000,0.000000}%
\pgfsetstrokecolor{currentstroke}%
\pgfsetdash{}{0pt}%
\pgfsys@defobject{currentmarker}{\pgfqpoint{0.000000in}{-0.020833in}}{\pgfqpoint{0.000000in}{0.000000in}}{%
\pgfpathmoveto{\pgfqpoint{0.000000in}{0.000000in}}%
\pgfpathlineto{\pgfqpoint{0.000000in}{-0.020833in}}%
\pgfusepath{stroke,fill}%
}%
\begin{pgfscope}%
\pgfsys@transformshift{2.561837in}{3.227753in}%
\pgfsys@useobject{currentmarker}{}%
\end{pgfscope}%
\end{pgfscope}%
\begin{pgfscope}%
\pgfpathrectangle{\pgfqpoint{0.481681in}{1.080890in}}{\pgfqpoint{5.785672in}{2.146863in}}%
\pgfusepath{clip}%
\pgfsetrectcap%
\pgfsetroundjoin%
\pgfsetlinewidth{0.100375pt}%
\definecolor{currentstroke}{rgb}{0.827451,0.827451,0.827451}%
\pgfsetstrokecolor{currentstroke}%
\pgfsetdash{}{0pt}%
\pgfpathmoveto{\pgfqpoint{2.597357in}{1.080890in}}%
\pgfpathlineto{\pgfqpoint{2.597357in}{3.227753in}}%
\pgfusepath{stroke}%
\end{pgfscope}%
\begin{pgfscope}%
\pgfsetbuttcap%
\pgfsetroundjoin%
\definecolor{currentfill}{rgb}{0.000000,0.000000,0.000000}%
\pgfsetfillcolor{currentfill}%
\pgfsetlinewidth{0.501875pt}%
\definecolor{currentstroke}{rgb}{0.000000,0.000000,0.000000}%
\pgfsetstrokecolor{currentstroke}%
\pgfsetdash{}{0pt}%
\pgfsys@defobject{currentmarker}{\pgfqpoint{0.000000in}{0.000000in}}{\pgfqpoint{0.000000in}{0.020833in}}{%
\pgfpathmoveto{\pgfqpoint{0.000000in}{0.000000in}}%
\pgfpathlineto{\pgfqpoint{0.000000in}{0.020833in}}%
\pgfusepath{stroke,fill}%
}%
\begin{pgfscope}%
\pgfsys@transformshift{2.597357in}{1.080890in}%
\pgfsys@useobject{currentmarker}{}%
\end{pgfscope}%
\end{pgfscope}%
\begin{pgfscope}%
\pgfsetbuttcap%
\pgfsetroundjoin%
\definecolor{currentfill}{rgb}{0.000000,0.000000,0.000000}%
\pgfsetfillcolor{currentfill}%
\pgfsetlinewidth{0.501875pt}%
\definecolor{currentstroke}{rgb}{0.000000,0.000000,0.000000}%
\pgfsetstrokecolor{currentstroke}%
\pgfsetdash{}{0pt}%
\pgfsys@defobject{currentmarker}{\pgfqpoint{0.000000in}{-0.020833in}}{\pgfqpoint{0.000000in}{0.000000in}}{%
\pgfpathmoveto{\pgfqpoint{0.000000in}{0.000000in}}%
\pgfpathlineto{\pgfqpoint{0.000000in}{-0.020833in}}%
\pgfusepath{stroke,fill}%
}%
\begin{pgfscope}%
\pgfsys@transformshift{2.597357in}{3.227753in}%
\pgfsys@useobject{currentmarker}{}%
\end{pgfscope}%
\end{pgfscope}%
\begin{pgfscope}%
\pgfpathrectangle{\pgfqpoint{0.481681in}{1.080890in}}{\pgfqpoint{5.785672in}{2.146863in}}%
\pgfusepath{clip}%
\pgfsetrectcap%
\pgfsetroundjoin%
\pgfsetlinewidth{0.100375pt}%
\definecolor{currentstroke}{rgb}{0.827451,0.827451,0.827451}%
\pgfsetstrokecolor{currentstroke}%
\pgfsetdash{}{0pt}%
\pgfpathmoveto{\pgfqpoint{2.632878in}{1.080890in}}%
\pgfpathlineto{\pgfqpoint{2.632878in}{3.227753in}}%
\pgfusepath{stroke}%
\end{pgfscope}%
\begin{pgfscope}%
\pgfsetbuttcap%
\pgfsetroundjoin%
\definecolor{currentfill}{rgb}{0.000000,0.000000,0.000000}%
\pgfsetfillcolor{currentfill}%
\pgfsetlinewidth{0.501875pt}%
\definecolor{currentstroke}{rgb}{0.000000,0.000000,0.000000}%
\pgfsetstrokecolor{currentstroke}%
\pgfsetdash{}{0pt}%
\pgfsys@defobject{currentmarker}{\pgfqpoint{0.000000in}{0.000000in}}{\pgfqpoint{0.000000in}{0.020833in}}{%
\pgfpathmoveto{\pgfqpoint{0.000000in}{0.000000in}}%
\pgfpathlineto{\pgfqpoint{0.000000in}{0.020833in}}%
\pgfusepath{stroke,fill}%
}%
\begin{pgfscope}%
\pgfsys@transformshift{2.632878in}{1.080890in}%
\pgfsys@useobject{currentmarker}{}%
\end{pgfscope}%
\end{pgfscope}%
\begin{pgfscope}%
\pgfsetbuttcap%
\pgfsetroundjoin%
\definecolor{currentfill}{rgb}{0.000000,0.000000,0.000000}%
\pgfsetfillcolor{currentfill}%
\pgfsetlinewidth{0.501875pt}%
\definecolor{currentstroke}{rgb}{0.000000,0.000000,0.000000}%
\pgfsetstrokecolor{currentstroke}%
\pgfsetdash{}{0pt}%
\pgfsys@defobject{currentmarker}{\pgfqpoint{0.000000in}{-0.020833in}}{\pgfqpoint{0.000000in}{0.000000in}}{%
\pgfpathmoveto{\pgfqpoint{0.000000in}{0.000000in}}%
\pgfpathlineto{\pgfqpoint{0.000000in}{-0.020833in}}%
\pgfusepath{stroke,fill}%
}%
\begin{pgfscope}%
\pgfsys@transformshift{2.632878in}{3.227753in}%
\pgfsys@useobject{currentmarker}{}%
\end{pgfscope}%
\end{pgfscope}%
\begin{pgfscope}%
\pgfpathrectangle{\pgfqpoint{0.481681in}{1.080890in}}{\pgfqpoint{5.785672in}{2.146863in}}%
\pgfusepath{clip}%
\pgfsetrectcap%
\pgfsetroundjoin%
\pgfsetlinewidth{0.100375pt}%
\definecolor{currentstroke}{rgb}{0.827451,0.827451,0.827451}%
\pgfsetstrokecolor{currentstroke}%
\pgfsetdash{}{0pt}%
\pgfpathmoveto{\pgfqpoint{2.668398in}{1.080890in}}%
\pgfpathlineto{\pgfqpoint{2.668398in}{3.227753in}}%
\pgfusepath{stroke}%
\end{pgfscope}%
\begin{pgfscope}%
\pgfsetbuttcap%
\pgfsetroundjoin%
\definecolor{currentfill}{rgb}{0.000000,0.000000,0.000000}%
\pgfsetfillcolor{currentfill}%
\pgfsetlinewidth{0.501875pt}%
\definecolor{currentstroke}{rgb}{0.000000,0.000000,0.000000}%
\pgfsetstrokecolor{currentstroke}%
\pgfsetdash{}{0pt}%
\pgfsys@defobject{currentmarker}{\pgfqpoint{0.000000in}{0.000000in}}{\pgfqpoint{0.000000in}{0.020833in}}{%
\pgfpathmoveto{\pgfqpoint{0.000000in}{0.000000in}}%
\pgfpathlineto{\pgfqpoint{0.000000in}{0.020833in}}%
\pgfusepath{stroke,fill}%
}%
\begin{pgfscope}%
\pgfsys@transformshift{2.668398in}{1.080890in}%
\pgfsys@useobject{currentmarker}{}%
\end{pgfscope}%
\end{pgfscope}%
\begin{pgfscope}%
\pgfsetbuttcap%
\pgfsetroundjoin%
\definecolor{currentfill}{rgb}{0.000000,0.000000,0.000000}%
\pgfsetfillcolor{currentfill}%
\pgfsetlinewidth{0.501875pt}%
\definecolor{currentstroke}{rgb}{0.000000,0.000000,0.000000}%
\pgfsetstrokecolor{currentstroke}%
\pgfsetdash{}{0pt}%
\pgfsys@defobject{currentmarker}{\pgfqpoint{0.000000in}{-0.020833in}}{\pgfqpoint{0.000000in}{0.000000in}}{%
\pgfpathmoveto{\pgfqpoint{0.000000in}{0.000000in}}%
\pgfpathlineto{\pgfqpoint{0.000000in}{-0.020833in}}%
\pgfusepath{stroke,fill}%
}%
\begin{pgfscope}%
\pgfsys@transformshift{2.668398in}{3.227753in}%
\pgfsys@useobject{currentmarker}{}%
\end{pgfscope}%
\end{pgfscope}%
\begin{pgfscope}%
\pgfpathrectangle{\pgfqpoint{0.481681in}{1.080890in}}{\pgfqpoint{5.785672in}{2.146863in}}%
\pgfusepath{clip}%
\pgfsetrectcap%
\pgfsetroundjoin%
\pgfsetlinewidth{0.100375pt}%
\definecolor{currentstroke}{rgb}{0.827451,0.827451,0.827451}%
\pgfsetstrokecolor{currentstroke}%
\pgfsetdash{}{0pt}%
\pgfpathmoveto{\pgfqpoint{2.703919in}{1.080890in}}%
\pgfpathlineto{\pgfqpoint{2.703919in}{3.227753in}}%
\pgfusepath{stroke}%
\end{pgfscope}%
\begin{pgfscope}%
\pgfsetbuttcap%
\pgfsetroundjoin%
\definecolor{currentfill}{rgb}{0.000000,0.000000,0.000000}%
\pgfsetfillcolor{currentfill}%
\pgfsetlinewidth{0.501875pt}%
\definecolor{currentstroke}{rgb}{0.000000,0.000000,0.000000}%
\pgfsetstrokecolor{currentstroke}%
\pgfsetdash{}{0pt}%
\pgfsys@defobject{currentmarker}{\pgfqpoint{0.000000in}{0.000000in}}{\pgfqpoint{0.000000in}{0.020833in}}{%
\pgfpathmoveto{\pgfqpoint{0.000000in}{0.000000in}}%
\pgfpathlineto{\pgfqpoint{0.000000in}{0.020833in}}%
\pgfusepath{stroke,fill}%
}%
\begin{pgfscope}%
\pgfsys@transformshift{2.703919in}{1.080890in}%
\pgfsys@useobject{currentmarker}{}%
\end{pgfscope}%
\end{pgfscope}%
\begin{pgfscope}%
\pgfsetbuttcap%
\pgfsetroundjoin%
\definecolor{currentfill}{rgb}{0.000000,0.000000,0.000000}%
\pgfsetfillcolor{currentfill}%
\pgfsetlinewidth{0.501875pt}%
\definecolor{currentstroke}{rgb}{0.000000,0.000000,0.000000}%
\pgfsetstrokecolor{currentstroke}%
\pgfsetdash{}{0pt}%
\pgfsys@defobject{currentmarker}{\pgfqpoint{0.000000in}{-0.020833in}}{\pgfqpoint{0.000000in}{0.000000in}}{%
\pgfpathmoveto{\pgfqpoint{0.000000in}{0.000000in}}%
\pgfpathlineto{\pgfqpoint{0.000000in}{-0.020833in}}%
\pgfusepath{stroke,fill}%
}%
\begin{pgfscope}%
\pgfsys@transformshift{2.703919in}{3.227753in}%
\pgfsys@useobject{currentmarker}{}%
\end{pgfscope}%
\end{pgfscope}%
\begin{pgfscope}%
\pgfpathrectangle{\pgfqpoint{0.481681in}{1.080890in}}{\pgfqpoint{5.785672in}{2.146863in}}%
\pgfusepath{clip}%
\pgfsetrectcap%
\pgfsetroundjoin%
\pgfsetlinewidth{0.100375pt}%
\definecolor{currentstroke}{rgb}{0.827451,0.827451,0.827451}%
\pgfsetstrokecolor{currentstroke}%
\pgfsetdash{}{0pt}%
\pgfpathmoveto{\pgfqpoint{2.739440in}{1.080890in}}%
\pgfpathlineto{\pgfqpoint{2.739440in}{3.227753in}}%
\pgfusepath{stroke}%
\end{pgfscope}%
\begin{pgfscope}%
\pgfsetbuttcap%
\pgfsetroundjoin%
\definecolor{currentfill}{rgb}{0.000000,0.000000,0.000000}%
\pgfsetfillcolor{currentfill}%
\pgfsetlinewidth{0.501875pt}%
\definecolor{currentstroke}{rgb}{0.000000,0.000000,0.000000}%
\pgfsetstrokecolor{currentstroke}%
\pgfsetdash{}{0pt}%
\pgfsys@defobject{currentmarker}{\pgfqpoint{0.000000in}{0.000000in}}{\pgfqpoint{0.000000in}{0.020833in}}{%
\pgfpathmoveto{\pgfqpoint{0.000000in}{0.000000in}}%
\pgfpathlineto{\pgfqpoint{0.000000in}{0.020833in}}%
\pgfusepath{stroke,fill}%
}%
\begin{pgfscope}%
\pgfsys@transformshift{2.739440in}{1.080890in}%
\pgfsys@useobject{currentmarker}{}%
\end{pgfscope}%
\end{pgfscope}%
\begin{pgfscope}%
\pgfsetbuttcap%
\pgfsetroundjoin%
\definecolor{currentfill}{rgb}{0.000000,0.000000,0.000000}%
\pgfsetfillcolor{currentfill}%
\pgfsetlinewidth{0.501875pt}%
\definecolor{currentstroke}{rgb}{0.000000,0.000000,0.000000}%
\pgfsetstrokecolor{currentstroke}%
\pgfsetdash{}{0pt}%
\pgfsys@defobject{currentmarker}{\pgfqpoint{0.000000in}{-0.020833in}}{\pgfqpoint{0.000000in}{0.000000in}}{%
\pgfpathmoveto{\pgfqpoint{0.000000in}{0.000000in}}%
\pgfpathlineto{\pgfqpoint{0.000000in}{-0.020833in}}%
\pgfusepath{stroke,fill}%
}%
\begin{pgfscope}%
\pgfsys@transformshift{2.739440in}{3.227753in}%
\pgfsys@useobject{currentmarker}{}%
\end{pgfscope}%
\end{pgfscope}%
\begin{pgfscope}%
\pgfpathrectangle{\pgfqpoint{0.481681in}{1.080890in}}{\pgfqpoint{5.785672in}{2.146863in}}%
\pgfusepath{clip}%
\pgfsetrectcap%
\pgfsetroundjoin%
\pgfsetlinewidth{0.100375pt}%
\definecolor{currentstroke}{rgb}{0.827451,0.827451,0.827451}%
\pgfsetstrokecolor{currentstroke}%
\pgfsetdash{}{0pt}%
\pgfpathmoveto{\pgfqpoint{2.810481in}{1.080890in}}%
\pgfpathlineto{\pgfqpoint{2.810481in}{3.227753in}}%
\pgfusepath{stroke}%
\end{pgfscope}%
\begin{pgfscope}%
\pgfsetbuttcap%
\pgfsetroundjoin%
\definecolor{currentfill}{rgb}{0.000000,0.000000,0.000000}%
\pgfsetfillcolor{currentfill}%
\pgfsetlinewidth{0.501875pt}%
\definecolor{currentstroke}{rgb}{0.000000,0.000000,0.000000}%
\pgfsetstrokecolor{currentstroke}%
\pgfsetdash{}{0pt}%
\pgfsys@defobject{currentmarker}{\pgfqpoint{0.000000in}{0.000000in}}{\pgfqpoint{0.000000in}{0.020833in}}{%
\pgfpathmoveto{\pgfqpoint{0.000000in}{0.000000in}}%
\pgfpathlineto{\pgfqpoint{0.000000in}{0.020833in}}%
\pgfusepath{stroke,fill}%
}%
\begin{pgfscope}%
\pgfsys@transformshift{2.810481in}{1.080890in}%
\pgfsys@useobject{currentmarker}{}%
\end{pgfscope}%
\end{pgfscope}%
\begin{pgfscope}%
\pgfsetbuttcap%
\pgfsetroundjoin%
\definecolor{currentfill}{rgb}{0.000000,0.000000,0.000000}%
\pgfsetfillcolor{currentfill}%
\pgfsetlinewidth{0.501875pt}%
\definecolor{currentstroke}{rgb}{0.000000,0.000000,0.000000}%
\pgfsetstrokecolor{currentstroke}%
\pgfsetdash{}{0pt}%
\pgfsys@defobject{currentmarker}{\pgfqpoint{0.000000in}{-0.020833in}}{\pgfqpoint{0.000000in}{0.000000in}}{%
\pgfpathmoveto{\pgfqpoint{0.000000in}{0.000000in}}%
\pgfpathlineto{\pgfqpoint{0.000000in}{-0.020833in}}%
\pgfusepath{stroke,fill}%
}%
\begin{pgfscope}%
\pgfsys@transformshift{2.810481in}{3.227753in}%
\pgfsys@useobject{currentmarker}{}%
\end{pgfscope}%
\end{pgfscope}%
\begin{pgfscope}%
\pgfpathrectangle{\pgfqpoint{0.481681in}{1.080890in}}{\pgfqpoint{5.785672in}{2.146863in}}%
\pgfusepath{clip}%
\pgfsetrectcap%
\pgfsetroundjoin%
\pgfsetlinewidth{0.100375pt}%
\definecolor{currentstroke}{rgb}{0.827451,0.827451,0.827451}%
\pgfsetstrokecolor{currentstroke}%
\pgfsetdash{}{0pt}%
\pgfpathmoveto{\pgfqpoint{2.846001in}{1.080890in}}%
\pgfpathlineto{\pgfqpoint{2.846001in}{3.227753in}}%
\pgfusepath{stroke}%
\end{pgfscope}%
\begin{pgfscope}%
\pgfsetbuttcap%
\pgfsetroundjoin%
\definecolor{currentfill}{rgb}{0.000000,0.000000,0.000000}%
\pgfsetfillcolor{currentfill}%
\pgfsetlinewidth{0.501875pt}%
\definecolor{currentstroke}{rgb}{0.000000,0.000000,0.000000}%
\pgfsetstrokecolor{currentstroke}%
\pgfsetdash{}{0pt}%
\pgfsys@defobject{currentmarker}{\pgfqpoint{0.000000in}{0.000000in}}{\pgfqpoint{0.000000in}{0.020833in}}{%
\pgfpathmoveto{\pgfqpoint{0.000000in}{0.000000in}}%
\pgfpathlineto{\pgfqpoint{0.000000in}{0.020833in}}%
\pgfusepath{stroke,fill}%
}%
\begin{pgfscope}%
\pgfsys@transformshift{2.846001in}{1.080890in}%
\pgfsys@useobject{currentmarker}{}%
\end{pgfscope}%
\end{pgfscope}%
\begin{pgfscope}%
\pgfsetbuttcap%
\pgfsetroundjoin%
\definecolor{currentfill}{rgb}{0.000000,0.000000,0.000000}%
\pgfsetfillcolor{currentfill}%
\pgfsetlinewidth{0.501875pt}%
\definecolor{currentstroke}{rgb}{0.000000,0.000000,0.000000}%
\pgfsetstrokecolor{currentstroke}%
\pgfsetdash{}{0pt}%
\pgfsys@defobject{currentmarker}{\pgfqpoint{0.000000in}{-0.020833in}}{\pgfqpoint{0.000000in}{0.000000in}}{%
\pgfpathmoveto{\pgfqpoint{0.000000in}{0.000000in}}%
\pgfpathlineto{\pgfqpoint{0.000000in}{-0.020833in}}%
\pgfusepath{stroke,fill}%
}%
\begin{pgfscope}%
\pgfsys@transformshift{2.846001in}{3.227753in}%
\pgfsys@useobject{currentmarker}{}%
\end{pgfscope}%
\end{pgfscope}%
\begin{pgfscope}%
\pgfpathrectangle{\pgfqpoint{0.481681in}{1.080890in}}{\pgfqpoint{5.785672in}{2.146863in}}%
\pgfusepath{clip}%
\pgfsetrectcap%
\pgfsetroundjoin%
\pgfsetlinewidth{0.100375pt}%
\definecolor{currentstroke}{rgb}{0.827451,0.827451,0.827451}%
\pgfsetstrokecolor{currentstroke}%
\pgfsetdash{}{0pt}%
\pgfpathmoveto{\pgfqpoint{2.881522in}{1.080890in}}%
\pgfpathlineto{\pgfqpoint{2.881522in}{3.227753in}}%
\pgfusepath{stroke}%
\end{pgfscope}%
\begin{pgfscope}%
\pgfsetbuttcap%
\pgfsetroundjoin%
\definecolor{currentfill}{rgb}{0.000000,0.000000,0.000000}%
\pgfsetfillcolor{currentfill}%
\pgfsetlinewidth{0.501875pt}%
\definecolor{currentstroke}{rgb}{0.000000,0.000000,0.000000}%
\pgfsetstrokecolor{currentstroke}%
\pgfsetdash{}{0pt}%
\pgfsys@defobject{currentmarker}{\pgfqpoint{0.000000in}{0.000000in}}{\pgfqpoint{0.000000in}{0.020833in}}{%
\pgfpathmoveto{\pgfqpoint{0.000000in}{0.000000in}}%
\pgfpathlineto{\pgfqpoint{0.000000in}{0.020833in}}%
\pgfusepath{stroke,fill}%
}%
\begin{pgfscope}%
\pgfsys@transformshift{2.881522in}{1.080890in}%
\pgfsys@useobject{currentmarker}{}%
\end{pgfscope}%
\end{pgfscope}%
\begin{pgfscope}%
\pgfsetbuttcap%
\pgfsetroundjoin%
\definecolor{currentfill}{rgb}{0.000000,0.000000,0.000000}%
\pgfsetfillcolor{currentfill}%
\pgfsetlinewidth{0.501875pt}%
\definecolor{currentstroke}{rgb}{0.000000,0.000000,0.000000}%
\pgfsetstrokecolor{currentstroke}%
\pgfsetdash{}{0pt}%
\pgfsys@defobject{currentmarker}{\pgfqpoint{0.000000in}{-0.020833in}}{\pgfqpoint{0.000000in}{0.000000in}}{%
\pgfpathmoveto{\pgfqpoint{0.000000in}{0.000000in}}%
\pgfpathlineto{\pgfqpoint{0.000000in}{-0.020833in}}%
\pgfusepath{stroke,fill}%
}%
\begin{pgfscope}%
\pgfsys@transformshift{2.881522in}{3.227753in}%
\pgfsys@useobject{currentmarker}{}%
\end{pgfscope}%
\end{pgfscope}%
\begin{pgfscope}%
\pgfpathrectangle{\pgfqpoint{0.481681in}{1.080890in}}{\pgfqpoint{5.785672in}{2.146863in}}%
\pgfusepath{clip}%
\pgfsetrectcap%
\pgfsetroundjoin%
\pgfsetlinewidth{0.100375pt}%
\definecolor{currentstroke}{rgb}{0.827451,0.827451,0.827451}%
\pgfsetstrokecolor{currentstroke}%
\pgfsetdash{}{0pt}%
\pgfpathmoveto{\pgfqpoint{2.917042in}{1.080890in}}%
\pgfpathlineto{\pgfqpoint{2.917042in}{3.227753in}}%
\pgfusepath{stroke}%
\end{pgfscope}%
\begin{pgfscope}%
\pgfsetbuttcap%
\pgfsetroundjoin%
\definecolor{currentfill}{rgb}{0.000000,0.000000,0.000000}%
\pgfsetfillcolor{currentfill}%
\pgfsetlinewidth{0.501875pt}%
\definecolor{currentstroke}{rgb}{0.000000,0.000000,0.000000}%
\pgfsetstrokecolor{currentstroke}%
\pgfsetdash{}{0pt}%
\pgfsys@defobject{currentmarker}{\pgfqpoint{0.000000in}{0.000000in}}{\pgfqpoint{0.000000in}{0.020833in}}{%
\pgfpathmoveto{\pgfqpoint{0.000000in}{0.000000in}}%
\pgfpathlineto{\pgfqpoint{0.000000in}{0.020833in}}%
\pgfusepath{stroke,fill}%
}%
\begin{pgfscope}%
\pgfsys@transformshift{2.917042in}{1.080890in}%
\pgfsys@useobject{currentmarker}{}%
\end{pgfscope}%
\end{pgfscope}%
\begin{pgfscope}%
\pgfsetbuttcap%
\pgfsetroundjoin%
\definecolor{currentfill}{rgb}{0.000000,0.000000,0.000000}%
\pgfsetfillcolor{currentfill}%
\pgfsetlinewidth{0.501875pt}%
\definecolor{currentstroke}{rgb}{0.000000,0.000000,0.000000}%
\pgfsetstrokecolor{currentstroke}%
\pgfsetdash{}{0pt}%
\pgfsys@defobject{currentmarker}{\pgfqpoint{0.000000in}{-0.020833in}}{\pgfqpoint{0.000000in}{0.000000in}}{%
\pgfpathmoveto{\pgfqpoint{0.000000in}{0.000000in}}%
\pgfpathlineto{\pgfqpoint{0.000000in}{-0.020833in}}%
\pgfusepath{stroke,fill}%
}%
\begin{pgfscope}%
\pgfsys@transformshift{2.917042in}{3.227753in}%
\pgfsys@useobject{currentmarker}{}%
\end{pgfscope}%
\end{pgfscope}%
\begin{pgfscope}%
\pgfpathrectangle{\pgfqpoint{0.481681in}{1.080890in}}{\pgfqpoint{5.785672in}{2.146863in}}%
\pgfusepath{clip}%
\pgfsetrectcap%
\pgfsetroundjoin%
\pgfsetlinewidth{0.100375pt}%
\definecolor{currentstroke}{rgb}{0.827451,0.827451,0.827451}%
\pgfsetstrokecolor{currentstroke}%
\pgfsetdash{}{0pt}%
\pgfpathmoveto{\pgfqpoint{2.952563in}{1.080890in}}%
\pgfpathlineto{\pgfqpoint{2.952563in}{3.227753in}}%
\pgfusepath{stroke}%
\end{pgfscope}%
\begin{pgfscope}%
\pgfsetbuttcap%
\pgfsetroundjoin%
\definecolor{currentfill}{rgb}{0.000000,0.000000,0.000000}%
\pgfsetfillcolor{currentfill}%
\pgfsetlinewidth{0.501875pt}%
\definecolor{currentstroke}{rgb}{0.000000,0.000000,0.000000}%
\pgfsetstrokecolor{currentstroke}%
\pgfsetdash{}{0pt}%
\pgfsys@defobject{currentmarker}{\pgfqpoint{0.000000in}{0.000000in}}{\pgfqpoint{0.000000in}{0.020833in}}{%
\pgfpathmoveto{\pgfqpoint{0.000000in}{0.000000in}}%
\pgfpathlineto{\pgfqpoint{0.000000in}{0.020833in}}%
\pgfusepath{stroke,fill}%
}%
\begin{pgfscope}%
\pgfsys@transformshift{2.952563in}{1.080890in}%
\pgfsys@useobject{currentmarker}{}%
\end{pgfscope}%
\end{pgfscope}%
\begin{pgfscope}%
\pgfsetbuttcap%
\pgfsetroundjoin%
\definecolor{currentfill}{rgb}{0.000000,0.000000,0.000000}%
\pgfsetfillcolor{currentfill}%
\pgfsetlinewidth{0.501875pt}%
\definecolor{currentstroke}{rgb}{0.000000,0.000000,0.000000}%
\pgfsetstrokecolor{currentstroke}%
\pgfsetdash{}{0pt}%
\pgfsys@defobject{currentmarker}{\pgfqpoint{0.000000in}{-0.020833in}}{\pgfqpoint{0.000000in}{0.000000in}}{%
\pgfpathmoveto{\pgfqpoint{0.000000in}{0.000000in}}%
\pgfpathlineto{\pgfqpoint{0.000000in}{-0.020833in}}%
\pgfusepath{stroke,fill}%
}%
\begin{pgfscope}%
\pgfsys@transformshift{2.952563in}{3.227753in}%
\pgfsys@useobject{currentmarker}{}%
\end{pgfscope}%
\end{pgfscope}%
\begin{pgfscope}%
\pgfpathrectangle{\pgfqpoint{0.481681in}{1.080890in}}{\pgfqpoint{5.785672in}{2.146863in}}%
\pgfusepath{clip}%
\pgfsetrectcap%
\pgfsetroundjoin%
\pgfsetlinewidth{0.100375pt}%
\definecolor{currentstroke}{rgb}{0.827451,0.827451,0.827451}%
\pgfsetstrokecolor{currentstroke}%
\pgfsetdash{}{0pt}%
\pgfpathmoveto{\pgfqpoint{2.988083in}{1.080890in}}%
\pgfpathlineto{\pgfqpoint{2.988083in}{3.227753in}}%
\pgfusepath{stroke}%
\end{pgfscope}%
\begin{pgfscope}%
\pgfsetbuttcap%
\pgfsetroundjoin%
\definecolor{currentfill}{rgb}{0.000000,0.000000,0.000000}%
\pgfsetfillcolor{currentfill}%
\pgfsetlinewidth{0.501875pt}%
\definecolor{currentstroke}{rgb}{0.000000,0.000000,0.000000}%
\pgfsetstrokecolor{currentstroke}%
\pgfsetdash{}{0pt}%
\pgfsys@defobject{currentmarker}{\pgfqpoint{0.000000in}{0.000000in}}{\pgfqpoint{0.000000in}{0.020833in}}{%
\pgfpathmoveto{\pgfqpoint{0.000000in}{0.000000in}}%
\pgfpathlineto{\pgfqpoint{0.000000in}{0.020833in}}%
\pgfusepath{stroke,fill}%
}%
\begin{pgfscope}%
\pgfsys@transformshift{2.988083in}{1.080890in}%
\pgfsys@useobject{currentmarker}{}%
\end{pgfscope}%
\end{pgfscope}%
\begin{pgfscope}%
\pgfsetbuttcap%
\pgfsetroundjoin%
\definecolor{currentfill}{rgb}{0.000000,0.000000,0.000000}%
\pgfsetfillcolor{currentfill}%
\pgfsetlinewidth{0.501875pt}%
\definecolor{currentstroke}{rgb}{0.000000,0.000000,0.000000}%
\pgfsetstrokecolor{currentstroke}%
\pgfsetdash{}{0pt}%
\pgfsys@defobject{currentmarker}{\pgfqpoint{0.000000in}{-0.020833in}}{\pgfqpoint{0.000000in}{0.000000in}}{%
\pgfpathmoveto{\pgfqpoint{0.000000in}{0.000000in}}%
\pgfpathlineto{\pgfqpoint{0.000000in}{-0.020833in}}%
\pgfusepath{stroke,fill}%
}%
\begin{pgfscope}%
\pgfsys@transformshift{2.988083in}{3.227753in}%
\pgfsys@useobject{currentmarker}{}%
\end{pgfscope}%
\end{pgfscope}%
\begin{pgfscope}%
\pgfpathrectangle{\pgfqpoint{0.481681in}{1.080890in}}{\pgfqpoint{5.785672in}{2.146863in}}%
\pgfusepath{clip}%
\pgfsetrectcap%
\pgfsetroundjoin%
\pgfsetlinewidth{0.100375pt}%
\definecolor{currentstroke}{rgb}{0.827451,0.827451,0.827451}%
\pgfsetstrokecolor{currentstroke}%
\pgfsetdash{}{0pt}%
\pgfpathmoveto{\pgfqpoint{3.023604in}{1.080890in}}%
\pgfpathlineto{\pgfqpoint{3.023604in}{3.227753in}}%
\pgfusepath{stroke}%
\end{pgfscope}%
\begin{pgfscope}%
\pgfsetbuttcap%
\pgfsetroundjoin%
\definecolor{currentfill}{rgb}{0.000000,0.000000,0.000000}%
\pgfsetfillcolor{currentfill}%
\pgfsetlinewidth{0.501875pt}%
\definecolor{currentstroke}{rgb}{0.000000,0.000000,0.000000}%
\pgfsetstrokecolor{currentstroke}%
\pgfsetdash{}{0pt}%
\pgfsys@defobject{currentmarker}{\pgfqpoint{0.000000in}{0.000000in}}{\pgfqpoint{0.000000in}{0.020833in}}{%
\pgfpathmoveto{\pgfqpoint{0.000000in}{0.000000in}}%
\pgfpathlineto{\pgfqpoint{0.000000in}{0.020833in}}%
\pgfusepath{stroke,fill}%
}%
\begin{pgfscope}%
\pgfsys@transformshift{3.023604in}{1.080890in}%
\pgfsys@useobject{currentmarker}{}%
\end{pgfscope}%
\end{pgfscope}%
\begin{pgfscope}%
\pgfsetbuttcap%
\pgfsetroundjoin%
\definecolor{currentfill}{rgb}{0.000000,0.000000,0.000000}%
\pgfsetfillcolor{currentfill}%
\pgfsetlinewidth{0.501875pt}%
\definecolor{currentstroke}{rgb}{0.000000,0.000000,0.000000}%
\pgfsetstrokecolor{currentstroke}%
\pgfsetdash{}{0pt}%
\pgfsys@defobject{currentmarker}{\pgfqpoint{0.000000in}{-0.020833in}}{\pgfqpoint{0.000000in}{0.000000in}}{%
\pgfpathmoveto{\pgfqpoint{0.000000in}{0.000000in}}%
\pgfpathlineto{\pgfqpoint{0.000000in}{-0.020833in}}%
\pgfusepath{stroke,fill}%
}%
\begin{pgfscope}%
\pgfsys@transformshift{3.023604in}{3.227753in}%
\pgfsys@useobject{currentmarker}{}%
\end{pgfscope}%
\end{pgfscope}%
\begin{pgfscope}%
\pgfpathrectangle{\pgfqpoint{0.481681in}{1.080890in}}{\pgfqpoint{5.785672in}{2.146863in}}%
\pgfusepath{clip}%
\pgfsetrectcap%
\pgfsetroundjoin%
\pgfsetlinewidth{0.100375pt}%
\definecolor{currentstroke}{rgb}{0.827451,0.827451,0.827451}%
\pgfsetstrokecolor{currentstroke}%
\pgfsetdash{}{0pt}%
\pgfpathmoveto{\pgfqpoint{3.059124in}{1.080890in}}%
\pgfpathlineto{\pgfqpoint{3.059124in}{3.227753in}}%
\pgfusepath{stroke}%
\end{pgfscope}%
\begin{pgfscope}%
\pgfsetbuttcap%
\pgfsetroundjoin%
\definecolor{currentfill}{rgb}{0.000000,0.000000,0.000000}%
\pgfsetfillcolor{currentfill}%
\pgfsetlinewidth{0.501875pt}%
\definecolor{currentstroke}{rgb}{0.000000,0.000000,0.000000}%
\pgfsetstrokecolor{currentstroke}%
\pgfsetdash{}{0pt}%
\pgfsys@defobject{currentmarker}{\pgfqpoint{0.000000in}{0.000000in}}{\pgfqpoint{0.000000in}{0.020833in}}{%
\pgfpathmoveto{\pgfqpoint{0.000000in}{0.000000in}}%
\pgfpathlineto{\pgfqpoint{0.000000in}{0.020833in}}%
\pgfusepath{stroke,fill}%
}%
\begin{pgfscope}%
\pgfsys@transformshift{3.059124in}{1.080890in}%
\pgfsys@useobject{currentmarker}{}%
\end{pgfscope}%
\end{pgfscope}%
\begin{pgfscope}%
\pgfsetbuttcap%
\pgfsetroundjoin%
\definecolor{currentfill}{rgb}{0.000000,0.000000,0.000000}%
\pgfsetfillcolor{currentfill}%
\pgfsetlinewidth{0.501875pt}%
\definecolor{currentstroke}{rgb}{0.000000,0.000000,0.000000}%
\pgfsetstrokecolor{currentstroke}%
\pgfsetdash{}{0pt}%
\pgfsys@defobject{currentmarker}{\pgfqpoint{0.000000in}{-0.020833in}}{\pgfqpoint{0.000000in}{0.000000in}}{%
\pgfpathmoveto{\pgfqpoint{0.000000in}{0.000000in}}%
\pgfpathlineto{\pgfqpoint{0.000000in}{-0.020833in}}%
\pgfusepath{stroke,fill}%
}%
\begin{pgfscope}%
\pgfsys@transformshift{3.059124in}{3.227753in}%
\pgfsys@useobject{currentmarker}{}%
\end{pgfscope}%
\end{pgfscope}%
\begin{pgfscope}%
\pgfpathrectangle{\pgfqpoint{0.481681in}{1.080890in}}{\pgfqpoint{5.785672in}{2.146863in}}%
\pgfusepath{clip}%
\pgfsetrectcap%
\pgfsetroundjoin%
\pgfsetlinewidth{0.100375pt}%
\definecolor{currentstroke}{rgb}{0.827451,0.827451,0.827451}%
\pgfsetstrokecolor{currentstroke}%
\pgfsetdash{}{0pt}%
\pgfpathmoveto{\pgfqpoint{3.094645in}{1.080890in}}%
\pgfpathlineto{\pgfqpoint{3.094645in}{3.227753in}}%
\pgfusepath{stroke}%
\end{pgfscope}%
\begin{pgfscope}%
\pgfsetbuttcap%
\pgfsetroundjoin%
\definecolor{currentfill}{rgb}{0.000000,0.000000,0.000000}%
\pgfsetfillcolor{currentfill}%
\pgfsetlinewidth{0.501875pt}%
\definecolor{currentstroke}{rgb}{0.000000,0.000000,0.000000}%
\pgfsetstrokecolor{currentstroke}%
\pgfsetdash{}{0pt}%
\pgfsys@defobject{currentmarker}{\pgfqpoint{0.000000in}{0.000000in}}{\pgfqpoint{0.000000in}{0.020833in}}{%
\pgfpathmoveto{\pgfqpoint{0.000000in}{0.000000in}}%
\pgfpathlineto{\pgfqpoint{0.000000in}{0.020833in}}%
\pgfusepath{stroke,fill}%
}%
\begin{pgfscope}%
\pgfsys@transformshift{3.094645in}{1.080890in}%
\pgfsys@useobject{currentmarker}{}%
\end{pgfscope}%
\end{pgfscope}%
\begin{pgfscope}%
\pgfsetbuttcap%
\pgfsetroundjoin%
\definecolor{currentfill}{rgb}{0.000000,0.000000,0.000000}%
\pgfsetfillcolor{currentfill}%
\pgfsetlinewidth{0.501875pt}%
\definecolor{currentstroke}{rgb}{0.000000,0.000000,0.000000}%
\pgfsetstrokecolor{currentstroke}%
\pgfsetdash{}{0pt}%
\pgfsys@defobject{currentmarker}{\pgfqpoint{0.000000in}{-0.020833in}}{\pgfqpoint{0.000000in}{0.000000in}}{%
\pgfpathmoveto{\pgfqpoint{0.000000in}{0.000000in}}%
\pgfpathlineto{\pgfqpoint{0.000000in}{-0.020833in}}%
\pgfusepath{stroke,fill}%
}%
\begin{pgfscope}%
\pgfsys@transformshift{3.094645in}{3.227753in}%
\pgfsys@useobject{currentmarker}{}%
\end{pgfscope}%
\end{pgfscope}%
\begin{pgfscope}%
\pgfpathrectangle{\pgfqpoint{0.481681in}{1.080890in}}{\pgfqpoint{5.785672in}{2.146863in}}%
\pgfusepath{clip}%
\pgfsetrectcap%
\pgfsetroundjoin%
\pgfsetlinewidth{0.100375pt}%
\definecolor{currentstroke}{rgb}{0.827451,0.827451,0.827451}%
\pgfsetstrokecolor{currentstroke}%
\pgfsetdash{}{0pt}%
\pgfpathmoveto{\pgfqpoint{3.130165in}{1.080890in}}%
\pgfpathlineto{\pgfqpoint{3.130165in}{3.227753in}}%
\pgfusepath{stroke}%
\end{pgfscope}%
\begin{pgfscope}%
\pgfsetbuttcap%
\pgfsetroundjoin%
\definecolor{currentfill}{rgb}{0.000000,0.000000,0.000000}%
\pgfsetfillcolor{currentfill}%
\pgfsetlinewidth{0.501875pt}%
\definecolor{currentstroke}{rgb}{0.000000,0.000000,0.000000}%
\pgfsetstrokecolor{currentstroke}%
\pgfsetdash{}{0pt}%
\pgfsys@defobject{currentmarker}{\pgfqpoint{0.000000in}{0.000000in}}{\pgfqpoint{0.000000in}{0.020833in}}{%
\pgfpathmoveto{\pgfqpoint{0.000000in}{0.000000in}}%
\pgfpathlineto{\pgfqpoint{0.000000in}{0.020833in}}%
\pgfusepath{stroke,fill}%
}%
\begin{pgfscope}%
\pgfsys@transformshift{3.130165in}{1.080890in}%
\pgfsys@useobject{currentmarker}{}%
\end{pgfscope}%
\end{pgfscope}%
\begin{pgfscope}%
\pgfsetbuttcap%
\pgfsetroundjoin%
\definecolor{currentfill}{rgb}{0.000000,0.000000,0.000000}%
\pgfsetfillcolor{currentfill}%
\pgfsetlinewidth{0.501875pt}%
\definecolor{currentstroke}{rgb}{0.000000,0.000000,0.000000}%
\pgfsetstrokecolor{currentstroke}%
\pgfsetdash{}{0pt}%
\pgfsys@defobject{currentmarker}{\pgfqpoint{0.000000in}{-0.020833in}}{\pgfqpoint{0.000000in}{0.000000in}}{%
\pgfpathmoveto{\pgfqpoint{0.000000in}{0.000000in}}%
\pgfpathlineto{\pgfqpoint{0.000000in}{-0.020833in}}%
\pgfusepath{stroke,fill}%
}%
\begin{pgfscope}%
\pgfsys@transformshift{3.130165in}{3.227753in}%
\pgfsys@useobject{currentmarker}{}%
\end{pgfscope}%
\end{pgfscope}%
\begin{pgfscope}%
\pgfpathrectangle{\pgfqpoint{0.481681in}{1.080890in}}{\pgfqpoint{5.785672in}{2.146863in}}%
\pgfusepath{clip}%
\pgfsetrectcap%
\pgfsetroundjoin%
\pgfsetlinewidth{0.100375pt}%
\definecolor{currentstroke}{rgb}{0.827451,0.827451,0.827451}%
\pgfsetstrokecolor{currentstroke}%
\pgfsetdash{}{0pt}%
\pgfpathmoveto{\pgfqpoint{3.165686in}{1.080890in}}%
\pgfpathlineto{\pgfqpoint{3.165686in}{3.227753in}}%
\pgfusepath{stroke}%
\end{pgfscope}%
\begin{pgfscope}%
\pgfsetbuttcap%
\pgfsetroundjoin%
\definecolor{currentfill}{rgb}{0.000000,0.000000,0.000000}%
\pgfsetfillcolor{currentfill}%
\pgfsetlinewidth{0.501875pt}%
\definecolor{currentstroke}{rgb}{0.000000,0.000000,0.000000}%
\pgfsetstrokecolor{currentstroke}%
\pgfsetdash{}{0pt}%
\pgfsys@defobject{currentmarker}{\pgfqpoint{0.000000in}{0.000000in}}{\pgfqpoint{0.000000in}{0.020833in}}{%
\pgfpathmoveto{\pgfqpoint{0.000000in}{0.000000in}}%
\pgfpathlineto{\pgfqpoint{0.000000in}{0.020833in}}%
\pgfusepath{stroke,fill}%
}%
\begin{pgfscope}%
\pgfsys@transformshift{3.165686in}{1.080890in}%
\pgfsys@useobject{currentmarker}{}%
\end{pgfscope}%
\end{pgfscope}%
\begin{pgfscope}%
\pgfsetbuttcap%
\pgfsetroundjoin%
\definecolor{currentfill}{rgb}{0.000000,0.000000,0.000000}%
\pgfsetfillcolor{currentfill}%
\pgfsetlinewidth{0.501875pt}%
\definecolor{currentstroke}{rgb}{0.000000,0.000000,0.000000}%
\pgfsetstrokecolor{currentstroke}%
\pgfsetdash{}{0pt}%
\pgfsys@defobject{currentmarker}{\pgfqpoint{0.000000in}{-0.020833in}}{\pgfqpoint{0.000000in}{0.000000in}}{%
\pgfpathmoveto{\pgfqpoint{0.000000in}{0.000000in}}%
\pgfpathlineto{\pgfqpoint{0.000000in}{-0.020833in}}%
\pgfusepath{stroke,fill}%
}%
\begin{pgfscope}%
\pgfsys@transformshift{3.165686in}{3.227753in}%
\pgfsys@useobject{currentmarker}{}%
\end{pgfscope}%
\end{pgfscope}%
\begin{pgfscope}%
\pgfpathrectangle{\pgfqpoint{0.481681in}{1.080890in}}{\pgfqpoint{5.785672in}{2.146863in}}%
\pgfusepath{clip}%
\pgfsetrectcap%
\pgfsetroundjoin%
\pgfsetlinewidth{0.100375pt}%
\definecolor{currentstroke}{rgb}{0.827451,0.827451,0.827451}%
\pgfsetstrokecolor{currentstroke}%
\pgfsetdash{}{0pt}%
\pgfpathmoveto{\pgfqpoint{3.236727in}{1.080890in}}%
\pgfpathlineto{\pgfqpoint{3.236727in}{3.227753in}}%
\pgfusepath{stroke}%
\end{pgfscope}%
\begin{pgfscope}%
\pgfsetbuttcap%
\pgfsetroundjoin%
\definecolor{currentfill}{rgb}{0.000000,0.000000,0.000000}%
\pgfsetfillcolor{currentfill}%
\pgfsetlinewidth{0.501875pt}%
\definecolor{currentstroke}{rgb}{0.000000,0.000000,0.000000}%
\pgfsetstrokecolor{currentstroke}%
\pgfsetdash{}{0pt}%
\pgfsys@defobject{currentmarker}{\pgfqpoint{0.000000in}{0.000000in}}{\pgfqpoint{0.000000in}{0.020833in}}{%
\pgfpathmoveto{\pgfqpoint{0.000000in}{0.000000in}}%
\pgfpathlineto{\pgfqpoint{0.000000in}{0.020833in}}%
\pgfusepath{stroke,fill}%
}%
\begin{pgfscope}%
\pgfsys@transformshift{3.236727in}{1.080890in}%
\pgfsys@useobject{currentmarker}{}%
\end{pgfscope}%
\end{pgfscope}%
\begin{pgfscope}%
\pgfsetbuttcap%
\pgfsetroundjoin%
\definecolor{currentfill}{rgb}{0.000000,0.000000,0.000000}%
\pgfsetfillcolor{currentfill}%
\pgfsetlinewidth{0.501875pt}%
\definecolor{currentstroke}{rgb}{0.000000,0.000000,0.000000}%
\pgfsetstrokecolor{currentstroke}%
\pgfsetdash{}{0pt}%
\pgfsys@defobject{currentmarker}{\pgfqpoint{0.000000in}{-0.020833in}}{\pgfqpoint{0.000000in}{0.000000in}}{%
\pgfpathmoveto{\pgfqpoint{0.000000in}{0.000000in}}%
\pgfpathlineto{\pgfqpoint{0.000000in}{-0.020833in}}%
\pgfusepath{stroke,fill}%
}%
\begin{pgfscope}%
\pgfsys@transformshift{3.236727in}{3.227753in}%
\pgfsys@useobject{currentmarker}{}%
\end{pgfscope}%
\end{pgfscope}%
\begin{pgfscope}%
\pgfpathrectangle{\pgfqpoint{0.481681in}{1.080890in}}{\pgfqpoint{5.785672in}{2.146863in}}%
\pgfusepath{clip}%
\pgfsetrectcap%
\pgfsetroundjoin%
\pgfsetlinewidth{0.100375pt}%
\definecolor{currentstroke}{rgb}{0.827451,0.827451,0.827451}%
\pgfsetstrokecolor{currentstroke}%
\pgfsetdash{}{0pt}%
\pgfpathmoveto{\pgfqpoint{3.272247in}{1.080890in}}%
\pgfpathlineto{\pgfqpoint{3.272247in}{3.227753in}}%
\pgfusepath{stroke}%
\end{pgfscope}%
\begin{pgfscope}%
\pgfsetbuttcap%
\pgfsetroundjoin%
\definecolor{currentfill}{rgb}{0.000000,0.000000,0.000000}%
\pgfsetfillcolor{currentfill}%
\pgfsetlinewidth{0.501875pt}%
\definecolor{currentstroke}{rgb}{0.000000,0.000000,0.000000}%
\pgfsetstrokecolor{currentstroke}%
\pgfsetdash{}{0pt}%
\pgfsys@defobject{currentmarker}{\pgfqpoint{0.000000in}{0.000000in}}{\pgfqpoint{0.000000in}{0.020833in}}{%
\pgfpathmoveto{\pgfqpoint{0.000000in}{0.000000in}}%
\pgfpathlineto{\pgfqpoint{0.000000in}{0.020833in}}%
\pgfusepath{stroke,fill}%
}%
\begin{pgfscope}%
\pgfsys@transformshift{3.272247in}{1.080890in}%
\pgfsys@useobject{currentmarker}{}%
\end{pgfscope}%
\end{pgfscope}%
\begin{pgfscope}%
\pgfsetbuttcap%
\pgfsetroundjoin%
\definecolor{currentfill}{rgb}{0.000000,0.000000,0.000000}%
\pgfsetfillcolor{currentfill}%
\pgfsetlinewidth{0.501875pt}%
\definecolor{currentstroke}{rgb}{0.000000,0.000000,0.000000}%
\pgfsetstrokecolor{currentstroke}%
\pgfsetdash{}{0pt}%
\pgfsys@defobject{currentmarker}{\pgfqpoint{0.000000in}{-0.020833in}}{\pgfqpoint{0.000000in}{0.000000in}}{%
\pgfpathmoveto{\pgfqpoint{0.000000in}{0.000000in}}%
\pgfpathlineto{\pgfqpoint{0.000000in}{-0.020833in}}%
\pgfusepath{stroke,fill}%
}%
\begin{pgfscope}%
\pgfsys@transformshift{3.272247in}{3.227753in}%
\pgfsys@useobject{currentmarker}{}%
\end{pgfscope}%
\end{pgfscope}%
\begin{pgfscope}%
\pgfpathrectangle{\pgfqpoint{0.481681in}{1.080890in}}{\pgfqpoint{5.785672in}{2.146863in}}%
\pgfusepath{clip}%
\pgfsetrectcap%
\pgfsetroundjoin%
\pgfsetlinewidth{0.100375pt}%
\definecolor{currentstroke}{rgb}{0.827451,0.827451,0.827451}%
\pgfsetstrokecolor{currentstroke}%
\pgfsetdash{}{0pt}%
\pgfpathmoveto{\pgfqpoint{3.307768in}{1.080890in}}%
\pgfpathlineto{\pgfqpoint{3.307768in}{3.227753in}}%
\pgfusepath{stroke}%
\end{pgfscope}%
\begin{pgfscope}%
\pgfsetbuttcap%
\pgfsetroundjoin%
\definecolor{currentfill}{rgb}{0.000000,0.000000,0.000000}%
\pgfsetfillcolor{currentfill}%
\pgfsetlinewidth{0.501875pt}%
\definecolor{currentstroke}{rgb}{0.000000,0.000000,0.000000}%
\pgfsetstrokecolor{currentstroke}%
\pgfsetdash{}{0pt}%
\pgfsys@defobject{currentmarker}{\pgfqpoint{0.000000in}{0.000000in}}{\pgfqpoint{0.000000in}{0.020833in}}{%
\pgfpathmoveto{\pgfqpoint{0.000000in}{0.000000in}}%
\pgfpathlineto{\pgfqpoint{0.000000in}{0.020833in}}%
\pgfusepath{stroke,fill}%
}%
\begin{pgfscope}%
\pgfsys@transformshift{3.307768in}{1.080890in}%
\pgfsys@useobject{currentmarker}{}%
\end{pgfscope}%
\end{pgfscope}%
\begin{pgfscope}%
\pgfsetbuttcap%
\pgfsetroundjoin%
\definecolor{currentfill}{rgb}{0.000000,0.000000,0.000000}%
\pgfsetfillcolor{currentfill}%
\pgfsetlinewidth{0.501875pt}%
\definecolor{currentstroke}{rgb}{0.000000,0.000000,0.000000}%
\pgfsetstrokecolor{currentstroke}%
\pgfsetdash{}{0pt}%
\pgfsys@defobject{currentmarker}{\pgfqpoint{0.000000in}{-0.020833in}}{\pgfqpoint{0.000000in}{0.000000in}}{%
\pgfpathmoveto{\pgfqpoint{0.000000in}{0.000000in}}%
\pgfpathlineto{\pgfqpoint{0.000000in}{-0.020833in}}%
\pgfusepath{stroke,fill}%
}%
\begin{pgfscope}%
\pgfsys@transformshift{3.307768in}{3.227753in}%
\pgfsys@useobject{currentmarker}{}%
\end{pgfscope}%
\end{pgfscope}%
\begin{pgfscope}%
\pgfpathrectangle{\pgfqpoint{0.481681in}{1.080890in}}{\pgfqpoint{5.785672in}{2.146863in}}%
\pgfusepath{clip}%
\pgfsetrectcap%
\pgfsetroundjoin%
\pgfsetlinewidth{0.100375pt}%
\definecolor{currentstroke}{rgb}{0.827451,0.827451,0.827451}%
\pgfsetstrokecolor{currentstroke}%
\pgfsetdash{}{0pt}%
\pgfpathmoveto{\pgfqpoint{3.343288in}{1.080890in}}%
\pgfpathlineto{\pgfqpoint{3.343288in}{3.227753in}}%
\pgfusepath{stroke}%
\end{pgfscope}%
\begin{pgfscope}%
\pgfsetbuttcap%
\pgfsetroundjoin%
\definecolor{currentfill}{rgb}{0.000000,0.000000,0.000000}%
\pgfsetfillcolor{currentfill}%
\pgfsetlinewidth{0.501875pt}%
\definecolor{currentstroke}{rgb}{0.000000,0.000000,0.000000}%
\pgfsetstrokecolor{currentstroke}%
\pgfsetdash{}{0pt}%
\pgfsys@defobject{currentmarker}{\pgfqpoint{0.000000in}{0.000000in}}{\pgfqpoint{0.000000in}{0.020833in}}{%
\pgfpathmoveto{\pgfqpoint{0.000000in}{0.000000in}}%
\pgfpathlineto{\pgfqpoint{0.000000in}{0.020833in}}%
\pgfusepath{stroke,fill}%
}%
\begin{pgfscope}%
\pgfsys@transformshift{3.343288in}{1.080890in}%
\pgfsys@useobject{currentmarker}{}%
\end{pgfscope}%
\end{pgfscope}%
\begin{pgfscope}%
\pgfsetbuttcap%
\pgfsetroundjoin%
\definecolor{currentfill}{rgb}{0.000000,0.000000,0.000000}%
\pgfsetfillcolor{currentfill}%
\pgfsetlinewidth{0.501875pt}%
\definecolor{currentstroke}{rgb}{0.000000,0.000000,0.000000}%
\pgfsetstrokecolor{currentstroke}%
\pgfsetdash{}{0pt}%
\pgfsys@defobject{currentmarker}{\pgfqpoint{0.000000in}{-0.020833in}}{\pgfqpoint{0.000000in}{0.000000in}}{%
\pgfpathmoveto{\pgfqpoint{0.000000in}{0.000000in}}%
\pgfpathlineto{\pgfqpoint{0.000000in}{-0.020833in}}%
\pgfusepath{stroke,fill}%
}%
\begin{pgfscope}%
\pgfsys@transformshift{3.343288in}{3.227753in}%
\pgfsys@useobject{currentmarker}{}%
\end{pgfscope}%
\end{pgfscope}%
\begin{pgfscope}%
\pgfpathrectangle{\pgfqpoint{0.481681in}{1.080890in}}{\pgfqpoint{5.785672in}{2.146863in}}%
\pgfusepath{clip}%
\pgfsetrectcap%
\pgfsetroundjoin%
\pgfsetlinewidth{0.100375pt}%
\definecolor{currentstroke}{rgb}{0.827451,0.827451,0.827451}%
\pgfsetstrokecolor{currentstroke}%
\pgfsetdash{}{0pt}%
\pgfpathmoveto{\pgfqpoint{3.378809in}{1.080890in}}%
\pgfpathlineto{\pgfqpoint{3.378809in}{3.227753in}}%
\pgfusepath{stroke}%
\end{pgfscope}%
\begin{pgfscope}%
\pgfsetbuttcap%
\pgfsetroundjoin%
\definecolor{currentfill}{rgb}{0.000000,0.000000,0.000000}%
\pgfsetfillcolor{currentfill}%
\pgfsetlinewidth{0.501875pt}%
\definecolor{currentstroke}{rgb}{0.000000,0.000000,0.000000}%
\pgfsetstrokecolor{currentstroke}%
\pgfsetdash{}{0pt}%
\pgfsys@defobject{currentmarker}{\pgfqpoint{0.000000in}{0.000000in}}{\pgfqpoint{0.000000in}{0.020833in}}{%
\pgfpathmoveto{\pgfqpoint{0.000000in}{0.000000in}}%
\pgfpathlineto{\pgfqpoint{0.000000in}{0.020833in}}%
\pgfusepath{stroke,fill}%
}%
\begin{pgfscope}%
\pgfsys@transformshift{3.378809in}{1.080890in}%
\pgfsys@useobject{currentmarker}{}%
\end{pgfscope}%
\end{pgfscope}%
\begin{pgfscope}%
\pgfsetbuttcap%
\pgfsetroundjoin%
\definecolor{currentfill}{rgb}{0.000000,0.000000,0.000000}%
\pgfsetfillcolor{currentfill}%
\pgfsetlinewidth{0.501875pt}%
\definecolor{currentstroke}{rgb}{0.000000,0.000000,0.000000}%
\pgfsetstrokecolor{currentstroke}%
\pgfsetdash{}{0pt}%
\pgfsys@defobject{currentmarker}{\pgfqpoint{0.000000in}{-0.020833in}}{\pgfqpoint{0.000000in}{0.000000in}}{%
\pgfpathmoveto{\pgfqpoint{0.000000in}{0.000000in}}%
\pgfpathlineto{\pgfqpoint{0.000000in}{-0.020833in}}%
\pgfusepath{stroke,fill}%
}%
\begin{pgfscope}%
\pgfsys@transformshift{3.378809in}{3.227753in}%
\pgfsys@useobject{currentmarker}{}%
\end{pgfscope}%
\end{pgfscope}%
\begin{pgfscope}%
\pgfpathrectangle{\pgfqpoint{0.481681in}{1.080890in}}{\pgfqpoint{5.785672in}{2.146863in}}%
\pgfusepath{clip}%
\pgfsetrectcap%
\pgfsetroundjoin%
\pgfsetlinewidth{0.100375pt}%
\definecolor{currentstroke}{rgb}{0.827451,0.827451,0.827451}%
\pgfsetstrokecolor{currentstroke}%
\pgfsetdash{}{0pt}%
\pgfpathmoveto{\pgfqpoint{3.414330in}{1.080890in}}%
\pgfpathlineto{\pgfqpoint{3.414330in}{3.227753in}}%
\pgfusepath{stroke}%
\end{pgfscope}%
\begin{pgfscope}%
\pgfsetbuttcap%
\pgfsetroundjoin%
\definecolor{currentfill}{rgb}{0.000000,0.000000,0.000000}%
\pgfsetfillcolor{currentfill}%
\pgfsetlinewidth{0.501875pt}%
\definecolor{currentstroke}{rgb}{0.000000,0.000000,0.000000}%
\pgfsetstrokecolor{currentstroke}%
\pgfsetdash{}{0pt}%
\pgfsys@defobject{currentmarker}{\pgfqpoint{0.000000in}{0.000000in}}{\pgfqpoint{0.000000in}{0.020833in}}{%
\pgfpathmoveto{\pgfqpoint{0.000000in}{0.000000in}}%
\pgfpathlineto{\pgfqpoint{0.000000in}{0.020833in}}%
\pgfusepath{stroke,fill}%
}%
\begin{pgfscope}%
\pgfsys@transformshift{3.414330in}{1.080890in}%
\pgfsys@useobject{currentmarker}{}%
\end{pgfscope}%
\end{pgfscope}%
\begin{pgfscope}%
\pgfsetbuttcap%
\pgfsetroundjoin%
\definecolor{currentfill}{rgb}{0.000000,0.000000,0.000000}%
\pgfsetfillcolor{currentfill}%
\pgfsetlinewidth{0.501875pt}%
\definecolor{currentstroke}{rgb}{0.000000,0.000000,0.000000}%
\pgfsetstrokecolor{currentstroke}%
\pgfsetdash{}{0pt}%
\pgfsys@defobject{currentmarker}{\pgfqpoint{0.000000in}{-0.020833in}}{\pgfqpoint{0.000000in}{0.000000in}}{%
\pgfpathmoveto{\pgfqpoint{0.000000in}{0.000000in}}%
\pgfpathlineto{\pgfqpoint{0.000000in}{-0.020833in}}%
\pgfusepath{stroke,fill}%
}%
\begin{pgfscope}%
\pgfsys@transformshift{3.414330in}{3.227753in}%
\pgfsys@useobject{currentmarker}{}%
\end{pgfscope}%
\end{pgfscope}%
\begin{pgfscope}%
\pgfpathrectangle{\pgfqpoint{0.481681in}{1.080890in}}{\pgfqpoint{5.785672in}{2.146863in}}%
\pgfusepath{clip}%
\pgfsetrectcap%
\pgfsetroundjoin%
\pgfsetlinewidth{0.100375pt}%
\definecolor{currentstroke}{rgb}{0.827451,0.827451,0.827451}%
\pgfsetstrokecolor{currentstroke}%
\pgfsetdash{}{0pt}%
\pgfpathmoveto{\pgfqpoint{3.449850in}{1.080890in}}%
\pgfpathlineto{\pgfqpoint{3.449850in}{3.227753in}}%
\pgfusepath{stroke}%
\end{pgfscope}%
\begin{pgfscope}%
\pgfsetbuttcap%
\pgfsetroundjoin%
\definecolor{currentfill}{rgb}{0.000000,0.000000,0.000000}%
\pgfsetfillcolor{currentfill}%
\pgfsetlinewidth{0.501875pt}%
\definecolor{currentstroke}{rgb}{0.000000,0.000000,0.000000}%
\pgfsetstrokecolor{currentstroke}%
\pgfsetdash{}{0pt}%
\pgfsys@defobject{currentmarker}{\pgfqpoint{0.000000in}{0.000000in}}{\pgfqpoint{0.000000in}{0.020833in}}{%
\pgfpathmoveto{\pgfqpoint{0.000000in}{0.000000in}}%
\pgfpathlineto{\pgfqpoint{0.000000in}{0.020833in}}%
\pgfusepath{stroke,fill}%
}%
\begin{pgfscope}%
\pgfsys@transformshift{3.449850in}{1.080890in}%
\pgfsys@useobject{currentmarker}{}%
\end{pgfscope}%
\end{pgfscope}%
\begin{pgfscope}%
\pgfsetbuttcap%
\pgfsetroundjoin%
\definecolor{currentfill}{rgb}{0.000000,0.000000,0.000000}%
\pgfsetfillcolor{currentfill}%
\pgfsetlinewidth{0.501875pt}%
\definecolor{currentstroke}{rgb}{0.000000,0.000000,0.000000}%
\pgfsetstrokecolor{currentstroke}%
\pgfsetdash{}{0pt}%
\pgfsys@defobject{currentmarker}{\pgfqpoint{0.000000in}{-0.020833in}}{\pgfqpoint{0.000000in}{0.000000in}}{%
\pgfpathmoveto{\pgfqpoint{0.000000in}{0.000000in}}%
\pgfpathlineto{\pgfqpoint{0.000000in}{-0.020833in}}%
\pgfusepath{stroke,fill}%
}%
\begin{pgfscope}%
\pgfsys@transformshift{3.449850in}{3.227753in}%
\pgfsys@useobject{currentmarker}{}%
\end{pgfscope}%
\end{pgfscope}%
\begin{pgfscope}%
\pgfpathrectangle{\pgfqpoint{0.481681in}{1.080890in}}{\pgfqpoint{5.785672in}{2.146863in}}%
\pgfusepath{clip}%
\pgfsetrectcap%
\pgfsetroundjoin%
\pgfsetlinewidth{0.100375pt}%
\definecolor{currentstroke}{rgb}{0.827451,0.827451,0.827451}%
\pgfsetstrokecolor{currentstroke}%
\pgfsetdash{}{0pt}%
\pgfpathmoveto{\pgfqpoint{3.485371in}{1.080890in}}%
\pgfpathlineto{\pgfqpoint{3.485371in}{3.227753in}}%
\pgfusepath{stroke}%
\end{pgfscope}%
\begin{pgfscope}%
\pgfsetbuttcap%
\pgfsetroundjoin%
\definecolor{currentfill}{rgb}{0.000000,0.000000,0.000000}%
\pgfsetfillcolor{currentfill}%
\pgfsetlinewidth{0.501875pt}%
\definecolor{currentstroke}{rgb}{0.000000,0.000000,0.000000}%
\pgfsetstrokecolor{currentstroke}%
\pgfsetdash{}{0pt}%
\pgfsys@defobject{currentmarker}{\pgfqpoint{0.000000in}{0.000000in}}{\pgfqpoint{0.000000in}{0.020833in}}{%
\pgfpathmoveto{\pgfqpoint{0.000000in}{0.000000in}}%
\pgfpathlineto{\pgfqpoint{0.000000in}{0.020833in}}%
\pgfusepath{stroke,fill}%
}%
\begin{pgfscope}%
\pgfsys@transformshift{3.485371in}{1.080890in}%
\pgfsys@useobject{currentmarker}{}%
\end{pgfscope}%
\end{pgfscope}%
\begin{pgfscope}%
\pgfsetbuttcap%
\pgfsetroundjoin%
\definecolor{currentfill}{rgb}{0.000000,0.000000,0.000000}%
\pgfsetfillcolor{currentfill}%
\pgfsetlinewidth{0.501875pt}%
\definecolor{currentstroke}{rgb}{0.000000,0.000000,0.000000}%
\pgfsetstrokecolor{currentstroke}%
\pgfsetdash{}{0pt}%
\pgfsys@defobject{currentmarker}{\pgfqpoint{0.000000in}{-0.020833in}}{\pgfqpoint{0.000000in}{0.000000in}}{%
\pgfpathmoveto{\pgfqpoint{0.000000in}{0.000000in}}%
\pgfpathlineto{\pgfqpoint{0.000000in}{-0.020833in}}%
\pgfusepath{stroke,fill}%
}%
\begin{pgfscope}%
\pgfsys@transformshift{3.485371in}{3.227753in}%
\pgfsys@useobject{currentmarker}{}%
\end{pgfscope}%
\end{pgfscope}%
\begin{pgfscope}%
\pgfpathrectangle{\pgfqpoint{0.481681in}{1.080890in}}{\pgfqpoint{5.785672in}{2.146863in}}%
\pgfusepath{clip}%
\pgfsetrectcap%
\pgfsetroundjoin%
\pgfsetlinewidth{0.100375pt}%
\definecolor{currentstroke}{rgb}{0.827451,0.827451,0.827451}%
\pgfsetstrokecolor{currentstroke}%
\pgfsetdash{}{0pt}%
\pgfpathmoveto{\pgfqpoint{3.520891in}{1.080890in}}%
\pgfpathlineto{\pgfqpoint{3.520891in}{3.227753in}}%
\pgfusepath{stroke}%
\end{pgfscope}%
\begin{pgfscope}%
\pgfsetbuttcap%
\pgfsetroundjoin%
\definecolor{currentfill}{rgb}{0.000000,0.000000,0.000000}%
\pgfsetfillcolor{currentfill}%
\pgfsetlinewidth{0.501875pt}%
\definecolor{currentstroke}{rgb}{0.000000,0.000000,0.000000}%
\pgfsetstrokecolor{currentstroke}%
\pgfsetdash{}{0pt}%
\pgfsys@defobject{currentmarker}{\pgfqpoint{0.000000in}{0.000000in}}{\pgfqpoint{0.000000in}{0.020833in}}{%
\pgfpathmoveto{\pgfqpoint{0.000000in}{0.000000in}}%
\pgfpathlineto{\pgfqpoint{0.000000in}{0.020833in}}%
\pgfusepath{stroke,fill}%
}%
\begin{pgfscope}%
\pgfsys@transformshift{3.520891in}{1.080890in}%
\pgfsys@useobject{currentmarker}{}%
\end{pgfscope}%
\end{pgfscope}%
\begin{pgfscope}%
\pgfsetbuttcap%
\pgfsetroundjoin%
\definecolor{currentfill}{rgb}{0.000000,0.000000,0.000000}%
\pgfsetfillcolor{currentfill}%
\pgfsetlinewidth{0.501875pt}%
\definecolor{currentstroke}{rgb}{0.000000,0.000000,0.000000}%
\pgfsetstrokecolor{currentstroke}%
\pgfsetdash{}{0pt}%
\pgfsys@defobject{currentmarker}{\pgfqpoint{0.000000in}{-0.020833in}}{\pgfqpoint{0.000000in}{0.000000in}}{%
\pgfpathmoveto{\pgfqpoint{0.000000in}{0.000000in}}%
\pgfpathlineto{\pgfqpoint{0.000000in}{-0.020833in}}%
\pgfusepath{stroke,fill}%
}%
\begin{pgfscope}%
\pgfsys@transformshift{3.520891in}{3.227753in}%
\pgfsys@useobject{currentmarker}{}%
\end{pgfscope}%
\end{pgfscope}%
\begin{pgfscope}%
\pgfpathrectangle{\pgfqpoint{0.481681in}{1.080890in}}{\pgfqpoint{5.785672in}{2.146863in}}%
\pgfusepath{clip}%
\pgfsetrectcap%
\pgfsetroundjoin%
\pgfsetlinewidth{0.100375pt}%
\definecolor{currentstroke}{rgb}{0.827451,0.827451,0.827451}%
\pgfsetstrokecolor{currentstroke}%
\pgfsetdash{}{0pt}%
\pgfpathmoveto{\pgfqpoint{3.556412in}{1.080890in}}%
\pgfpathlineto{\pgfqpoint{3.556412in}{3.227753in}}%
\pgfusepath{stroke}%
\end{pgfscope}%
\begin{pgfscope}%
\pgfsetbuttcap%
\pgfsetroundjoin%
\definecolor{currentfill}{rgb}{0.000000,0.000000,0.000000}%
\pgfsetfillcolor{currentfill}%
\pgfsetlinewidth{0.501875pt}%
\definecolor{currentstroke}{rgb}{0.000000,0.000000,0.000000}%
\pgfsetstrokecolor{currentstroke}%
\pgfsetdash{}{0pt}%
\pgfsys@defobject{currentmarker}{\pgfqpoint{0.000000in}{0.000000in}}{\pgfqpoint{0.000000in}{0.020833in}}{%
\pgfpathmoveto{\pgfqpoint{0.000000in}{0.000000in}}%
\pgfpathlineto{\pgfqpoint{0.000000in}{0.020833in}}%
\pgfusepath{stroke,fill}%
}%
\begin{pgfscope}%
\pgfsys@transformshift{3.556412in}{1.080890in}%
\pgfsys@useobject{currentmarker}{}%
\end{pgfscope}%
\end{pgfscope}%
\begin{pgfscope}%
\pgfsetbuttcap%
\pgfsetroundjoin%
\definecolor{currentfill}{rgb}{0.000000,0.000000,0.000000}%
\pgfsetfillcolor{currentfill}%
\pgfsetlinewidth{0.501875pt}%
\definecolor{currentstroke}{rgb}{0.000000,0.000000,0.000000}%
\pgfsetstrokecolor{currentstroke}%
\pgfsetdash{}{0pt}%
\pgfsys@defobject{currentmarker}{\pgfqpoint{0.000000in}{-0.020833in}}{\pgfqpoint{0.000000in}{0.000000in}}{%
\pgfpathmoveto{\pgfqpoint{0.000000in}{0.000000in}}%
\pgfpathlineto{\pgfqpoint{0.000000in}{-0.020833in}}%
\pgfusepath{stroke,fill}%
}%
\begin{pgfscope}%
\pgfsys@transformshift{3.556412in}{3.227753in}%
\pgfsys@useobject{currentmarker}{}%
\end{pgfscope}%
\end{pgfscope}%
\begin{pgfscope}%
\pgfpathrectangle{\pgfqpoint{0.481681in}{1.080890in}}{\pgfqpoint{5.785672in}{2.146863in}}%
\pgfusepath{clip}%
\pgfsetrectcap%
\pgfsetroundjoin%
\pgfsetlinewidth{0.100375pt}%
\definecolor{currentstroke}{rgb}{0.827451,0.827451,0.827451}%
\pgfsetstrokecolor{currentstroke}%
\pgfsetdash{}{0pt}%
\pgfpathmoveto{\pgfqpoint{3.591932in}{1.080890in}}%
\pgfpathlineto{\pgfqpoint{3.591932in}{3.227753in}}%
\pgfusepath{stroke}%
\end{pgfscope}%
\begin{pgfscope}%
\pgfsetbuttcap%
\pgfsetroundjoin%
\definecolor{currentfill}{rgb}{0.000000,0.000000,0.000000}%
\pgfsetfillcolor{currentfill}%
\pgfsetlinewidth{0.501875pt}%
\definecolor{currentstroke}{rgb}{0.000000,0.000000,0.000000}%
\pgfsetstrokecolor{currentstroke}%
\pgfsetdash{}{0pt}%
\pgfsys@defobject{currentmarker}{\pgfqpoint{0.000000in}{0.000000in}}{\pgfqpoint{0.000000in}{0.020833in}}{%
\pgfpathmoveto{\pgfqpoint{0.000000in}{0.000000in}}%
\pgfpathlineto{\pgfqpoint{0.000000in}{0.020833in}}%
\pgfusepath{stroke,fill}%
}%
\begin{pgfscope}%
\pgfsys@transformshift{3.591932in}{1.080890in}%
\pgfsys@useobject{currentmarker}{}%
\end{pgfscope}%
\end{pgfscope}%
\begin{pgfscope}%
\pgfsetbuttcap%
\pgfsetroundjoin%
\definecolor{currentfill}{rgb}{0.000000,0.000000,0.000000}%
\pgfsetfillcolor{currentfill}%
\pgfsetlinewidth{0.501875pt}%
\definecolor{currentstroke}{rgb}{0.000000,0.000000,0.000000}%
\pgfsetstrokecolor{currentstroke}%
\pgfsetdash{}{0pt}%
\pgfsys@defobject{currentmarker}{\pgfqpoint{0.000000in}{-0.020833in}}{\pgfqpoint{0.000000in}{0.000000in}}{%
\pgfpathmoveto{\pgfqpoint{0.000000in}{0.000000in}}%
\pgfpathlineto{\pgfqpoint{0.000000in}{-0.020833in}}%
\pgfusepath{stroke,fill}%
}%
\begin{pgfscope}%
\pgfsys@transformshift{3.591932in}{3.227753in}%
\pgfsys@useobject{currentmarker}{}%
\end{pgfscope}%
\end{pgfscope}%
\begin{pgfscope}%
\pgfpathrectangle{\pgfqpoint{0.481681in}{1.080890in}}{\pgfqpoint{5.785672in}{2.146863in}}%
\pgfusepath{clip}%
\pgfsetrectcap%
\pgfsetroundjoin%
\pgfsetlinewidth{0.100375pt}%
\definecolor{currentstroke}{rgb}{0.827451,0.827451,0.827451}%
\pgfsetstrokecolor{currentstroke}%
\pgfsetdash{}{0pt}%
\pgfpathmoveto{\pgfqpoint{3.662973in}{1.080890in}}%
\pgfpathlineto{\pgfqpoint{3.662973in}{3.227753in}}%
\pgfusepath{stroke}%
\end{pgfscope}%
\begin{pgfscope}%
\pgfsetbuttcap%
\pgfsetroundjoin%
\definecolor{currentfill}{rgb}{0.000000,0.000000,0.000000}%
\pgfsetfillcolor{currentfill}%
\pgfsetlinewidth{0.501875pt}%
\definecolor{currentstroke}{rgb}{0.000000,0.000000,0.000000}%
\pgfsetstrokecolor{currentstroke}%
\pgfsetdash{}{0pt}%
\pgfsys@defobject{currentmarker}{\pgfqpoint{0.000000in}{0.000000in}}{\pgfqpoint{0.000000in}{0.020833in}}{%
\pgfpathmoveto{\pgfqpoint{0.000000in}{0.000000in}}%
\pgfpathlineto{\pgfqpoint{0.000000in}{0.020833in}}%
\pgfusepath{stroke,fill}%
}%
\begin{pgfscope}%
\pgfsys@transformshift{3.662973in}{1.080890in}%
\pgfsys@useobject{currentmarker}{}%
\end{pgfscope}%
\end{pgfscope}%
\begin{pgfscope}%
\pgfsetbuttcap%
\pgfsetroundjoin%
\definecolor{currentfill}{rgb}{0.000000,0.000000,0.000000}%
\pgfsetfillcolor{currentfill}%
\pgfsetlinewidth{0.501875pt}%
\definecolor{currentstroke}{rgb}{0.000000,0.000000,0.000000}%
\pgfsetstrokecolor{currentstroke}%
\pgfsetdash{}{0pt}%
\pgfsys@defobject{currentmarker}{\pgfqpoint{0.000000in}{-0.020833in}}{\pgfqpoint{0.000000in}{0.000000in}}{%
\pgfpathmoveto{\pgfqpoint{0.000000in}{0.000000in}}%
\pgfpathlineto{\pgfqpoint{0.000000in}{-0.020833in}}%
\pgfusepath{stroke,fill}%
}%
\begin{pgfscope}%
\pgfsys@transformshift{3.662973in}{3.227753in}%
\pgfsys@useobject{currentmarker}{}%
\end{pgfscope}%
\end{pgfscope}%
\begin{pgfscope}%
\pgfpathrectangle{\pgfqpoint{0.481681in}{1.080890in}}{\pgfqpoint{5.785672in}{2.146863in}}%
\pgfusepath{clip}%
\pgfsetrectcap%
\pgfsetroundjoin%
\pgfsetlinewidth{0.100375pt}%
\definecolor{currentstroke}{rgb}{0.827451,0.827451,0.827451}%
\pgfsetstrokecolor{currentstroke}%
\pgfsetdash{}{0pt}%
\pgfpathmoveto{\pgfqpoint{3.698494in}{1.080890in}}%
\pgfpathlineto{\pgfqpoint{3.698494in}{3.227753in}}%
\pgfusepath{stroke}%
\end{pgfscope}%
\begin{pgfscope}%
\pgfsetbuttcap%
\pgfsetroundjoin%
\definecolor{currentfill}{rgb}{0.000000,0.000000,0.000000}%
\pgfsetfillcolor{currentfill}%
\pgfsetlinewidth{0.501875pt}%
\definecolor{currentstroke}{rgb}{0.000000,0.000000,0.000000}%
\pgfsetstrokecolor{currentstroke}%
\pgfsetdash{}{0pt}%
\pgfsys@defobject{currentmarker}{\pgfqpoint{0.000000in}{0.000000in}}{\pgfqpoint{0.000000in}{0.020833in}}{%
\pgfpathmoveto{\pgfqpoint{0.000000in}{0.000000in}}%
\pgfpathlineto{\pgfqpoint{0.000000in}{0.020833in}}%
\pgfusepath{stroke,fill}%
}%
\begin{pgfscope}%
\pgfsys@transformshift{3.698494in}{1.080890in}%
\pgfsys@useobject{currentmarker}{}%
\end{pgfscope}%
\end{pgfscope}%
\begin{pgfscope}%
\pgfsetbuttcap%
\pgfsetroundjoin%
\definecolor{currentfill}{rgb}{0.000000,0.000000,0.000000}%
\pgfsetfillcolor{currentfill}%
\pgfsetlinewidth{0.501875pt}%
\definecolor{currentstroke}{rgb}{0.000000,0.000000,0.000000}%
\pgfsetstrokecolor{currentstroke}%
\pgfsetdash{}{0pt}%
\pgfsys@defobject{currentmarker}{\pgfqpoint{0.000000in}{-0.020833in}}{\pgfqpoint{0.000000in}{0.000000in}}{%
\pgfpathmoveto{\pgfqpoint{0.000000in}{0.000000in}}%
\pgfpathlineto{\pgfqpoint{0.000000in}{-0.020833in}}%
\pgfusepath{stroke,fill}%
}%
\begin{pgfscope}%
\pgfsys@transformshift{3.698494in}{3.227753in}%
\pgfsys@useobject{currentmarker}{}%
\end{pgfscope}%
\end{pgfscope}%
\begin{pgfscope}%
\pgfpathrectangle{\pgfqpoint{0.481681in}{1.080890in}}{\pgfqpoint{5.785672in}{2.146863in}}%
\pgfusepath{clip}%
\pgfsetrectcap%
\pgfsetroundjoin%
\pgfsetlinewidth{0.100375pt}%
\definecolor{currentstroke}{rgb}{0.827451,0.827451,0.827451}%
\pgfsetstrokecolor{currentstroke}%
\pgfsetdash{}{0pt}%
\pgfpathmoveto{\pgfqpoint{3.734014in}{1.080890in}}%
\pgfpathlineto{\pgfqpoint{3.734014in}{3.227753in}}%
\pgfusepath{stroke}%
\end{pgfscope}%
\begin{pgfscope}%
\pgfsetbuttcap%
\pgfsetroundjoin%
\definecolor{currentfill}{rgb}{0.000000,0.000000,0.000000}%
\pgfsetfillcolor{currentfill}%
\pgfsetlinewidth{0.501875pt}%
\definecolor{currentstroke}{rgb}{0.000000,0.000000,0.000000}%
\pgfsetstrokecolor{currentstroke}%
\pgfsetdash{}{0pt}%
\pgfsys@defobject{currentmarker}{\pgfqpoint{0.000000in}{0.000000in}}{\pgfqpoint{0.000000in}{0.020833in}}{%
\pgfpathmoveto{\pgfqpoint{0.000000in}{0.000000in}}%
\pgfpathlineto{\pgfqpoint{0.000000in}{0.020833in}}%
\pgfusepath{stroke,fill}%
}%
\begin{pgfscope}%
\pgfsys@transformshift{3.734014in}{1.080890in}%
\pgfsys@useobject{currentmarker}{}%
\end{pgfscope}%
\end{pgfscope}%
\begin{pgfscope}%
\pgfsetbuttcap%
\pgfsetroundjoin%
\definecolor{currentfill}{rgb}{0.000000,0.000000,0.000000}%
\pgfsetfillcolor{currentfill}%
\pgfsetlinewidth{0.501875pt}%
\definecolor{currentstroke}{rgb}{0.000000,0.000000,0.000000}%
\pgfsetstrokecolor{currentstroke}%
\pgfsetdash{}{0pt}%
\pgfsys@defobject{currentmarker}{\pgfqpoint{0.000000in}{-0.020833in}}{\pgfqpoint{0.000000in}{0.000000in}}{%
\pgfpathmoveto{\pgfqpoint{0.000000in}{0.000000in}}%
\pgfpathlineto{\pgfqpoint{0.000000in}{-0.020833in}}%
\pgfusepath{stroke,fill}%
}%
\begin{pgfscope}%
\pgfsys@transformshift{3.734014in}{3.227753in}%
\pgfsys@useobject{currentmarker}{}%
\end{pgfscope}%
\end{pgfscope}%
\begin{pgfscope}%
\pgfpathrectangle{\pgfqpoint{0.481681in}{1.080890in}}{\pgfqpoint{5.785672in}{2.146863in}}%
\pgfusepath{clip}%
\pgfsetrectcap%
\pgfsetroundjoin%
\pgfsetlinewidth{0.100375pt}%
\definecolor{currentstroke}{rgb}{0.827451,0.827451,0.827451}%
\pgfsetstrokecolor{currentstroke}%
\pgfsetdash{}{0pt}%
\pgfpathmoveto{\pgfqpoint{3.769535in}{1.080890in}}%
\pgfpathlineto{\pgfqpoint{3.769535in}{3.227753in}}%
\pgfusepath{stroke}%
\end{pgfscope}%
\begin{pgfscope}%
\pgfsetbuttcap%
\pgfsetroundjoin%
\definecolor{currentfill}{rgb}{0.000000,0.000000,0.000000}%
\pgfsetfillcolor{currentfill}%
\pgfsetlinewidth{0.501875pt}%
\definecolor{currentstroke}{rgb}{0.000000,0.000000,0.000000}%
\pgfsetstrokecolor{currentstroke}%
\pgfsetdash{}{0pt}%
\pgfsys@defobject{currentmarker}{\pgfqpoint{0.000000in}{0.000000in}}{\pgfqpoint{0.000000in}{0.020833in}}{%
\pgfpathmoveto{\pgfqpoint{0.000000in}{0.000000in}}%
\pgfpathlineto{\pgfqpoint{0.000000in}{0.020833in}}%
\pgfusepath{stroke,fill}%
}%
\begin{pgfscope}%
\pgfsys@transformshift{3.769535in}{1.080890in}%
\pgfsys@useobject{currentmarker}{}%
\end{pgfscope}%
\end{pgfscope}%
\begin{pgfscope}%
\pgfsetbuttcap%
\pgfsetroundjoin%
\definecolor{currentfill}{rgb}{0.000000,0.000000,0.000000}%
\pgfsetfillcolor{currentfill}%
\pgfsetlinewidth{0.501875pt}%
\definecolor{currentstroke}{rgb}{0.000000,0.000000,0.000000}%
\pgfsetstrokecolor{currentstroke}%
\pgfsetdash{}{0pt}%
\pgfsys@defobject{currentmarker}{\pgfqpoint{0.000000in}{-0.020833in}}{\pgfqpoint{0.000000in}{0.000000in}}{%
\pgfpathmoveto{\pgfqpoint{0.000000in}{0.000000in}}%
\pgfpathlineto{\pgfqpoint{0.000000in}{-0.020833in}}%
\pgfusepath{stroke,fill}%
}%
\begin{pgfscope}%
\pgfsys@transformshift{3.769535in}{3.227753in}%
\pgfsys@useobject{currentmarker}{}%
\end{pgfscope}%
\end{pgfscope}%
\begin{pgfscope}%
\pgfpathrectangle{\pgfqpoint{0.481681in}{1.080890in}}{\pgfqpoint{5.785672in}{2.146863in}}%
\pgfusepath{clip}%
\pgfsetrectcap%
\pgfsetroundjoin%
\pgfsetlinewidth{0.100375pt}%
\definecolor{currentstroke}{rgb}{0.827451,0.827451,0.827451}%
\pgfsetstrokecolor{currentstroke}%
\pgfsetdash{}{0pt}%
\pgfpathmoveto{\pgfqpoint{3.805055in}{1.080890in}}%
\pgfpathlineto{\pgfqpoint{3.805055in}{3.227753in}}%
\pgfusepath{stroke}%
\end{pgfscope}%
\begin{pgfscope}%
\pgfsetbuttcap%
\pgfsetroundjoin%
\definecolor{currentfill}{rgb}{0.000000,0.000000,0.000000}%
\pgfsetfillcolor{currentfill}%
\pgfsetlinewidth{0.501875pt}%
\definecolor{currentstroke}{rgb}{0.000000,0.000000,0.000000}%
\pgfsetstrokecolor{currentstroke}%
\pgfsetdash{}{0pt}%
\pgfsys@defobject{currentmarker}{\pgfqpoint{0.000000in}{0.000000in}}{\pgfqpoint{0.000000in}{0.020833in}}{%
\pgfpathmoveto{\pgfqpoint{0.000000in}{0.000000in}}%
\pgfpathlineto{\pgfqpoint{0.000000in}{0.020833in}}%
\pgfusepath{stroke,fill}%
}%
\begin{pgfscope}%
\pgfsys@transformshift{3.805055in}{1.080890in}%
\pgfsys@useobject{currentmarker}{}%
\end{pgfscope}%
\end{pgfscope}%
\begin{pgfscope}%
\pgfsetbuttcap%
\pgfsetroundjoin%
\definecolor{currentfill}{rgb}{0.000000,0.000000,0.000000}%
\pgfsetfillcolor{currentfill}%
\pgfsetlinewidth{0.501875pt}%
\definecolor{currentstroke}{rgb}{0.000000,0.000000,0.000000}%
\pgfsetstrokecolor{currentstroke}%
\pgfsetdash{}{0pt}%
\pgfsys@defobject{currentmarker}{\pgfqpoint{0.000000in}{-0.020833in}}{\pgfqpoint{0.000000in}{0.000000in}}{%
\pgfpathmoveto{\pgfqpoint{0.000000in}{0.000000in}}%
\pgfpathlineto{\pgfqpoint{0.000000in}{-0.020833in}}%
\pgfusepath{stroke,fill}%
}%
\begin{pgfscope}%
\pgfsys@transformshift{3.805055in}{3.227753in}%
\pgfsys@useobject{currentmarker}{}%
\end{pgfscope}%
\end{pgfscope}%
\begin{pgfscope}%
\pgfpathrectangle{\pgfqpoint{0.481681in}{1.080890in}}{\pgfqpoint{5.785672in}{2.146863in}}%
\pgfusepath{clip}%
\pgfsetrectcap%
\pgfsetroundjoin%
\pgfsetlinewidth{0.100375pt}%
\definecolor{currentstroke}{rgb}{0.827451,0.827451,0.827451}%
\pgfsetstrokecolor{currentstroke}%
\pgfsetdash{}{0pt}%
\pgfpathmoveto{\pgfqpoint{3.840576in}{1.080890in}}%
\pgfpathlineto{\pgfqpoint{3.840576in}{3.227753in}}%
\pgfusepath{stroke}%
\end{pgfscope}%
\begin{pgfscope}%
\pgfsetbuttcap%
\pgfsetroundjoin%
\definecolor{currentfill}{rgb}{0.000000,0.000000,0.000000}%
\pgfsetfillcolor{currentfill}%
\pgfsetlinewidth{0.501875pt}%
\definecolor{currentstroke}{rgb}{0.000000,0.000000,0.000000}%
\pgfsetstrokecolor{currentstroke}%
\pgfsetdash{}{0pt}%
\pgfsys@defobject{currentmarker}{\pgfqpoint{0.000000in}{0.000000in}}{\pgfqpoint{0.000000in}{0.020833in}}{%
\pgfpathmoveto{\pgfqpoint{0.000000in}{0.000000in}}%
\pgfpathlineto{\pgfqpoint{0.000000in}{0.020833in}}%
\pgfusepath{stroke,fill}%
}%
\begin{pgfscope}%
\pgfsys@transformshift{3.840576in}{1.080890in}%
\pgfsys@useobject{currentmarker}{}%
\end{pgfscope}%
\end{pgfscope}%
\begin{pgfscope}%
\pgfsetbuttcap%
\pgfsetroundjoin%
\definecolor{currentfill}{rgb}{0.000000,0.000000,0.000000}%
\pgfsetfillcolor{currentfill}%
\pgfsetlinewidth{0.501875pt}%
\definecolor{currentstroke}{rgb}{0.000000,0.000000,0.000000}%
\pgfsetstrokecolor{currentstroke}%
\pgfsetdash{}{0pt}%
\pgfsys@defobject{currentmarker}{\pgfqpoint{0.000000in}{-0.020833in}}{\pgfqpoint{0.000000in}{0.000000in}}{%
\pgfpathmoveto{\pgfqpoint{0.000000in}{0.000000in}}%
\pgfpathlineto{\pgfqpoint{0.000000in}{-0.020833in}}%
\pgfusepath{stroke,fill}%
}%
\begin{pgfscope}%
\pgfsys@transformshift{3.840576in}{3.227753in}%
\pgfsys@useobject{currentmarker}{}%
\end{pgfscope}%
\end{pgfscope}%
\begin{pgfscope}%
\pgfpathrectangle{\pgfqpoint{0.481681in}{1.080890in}}{\pgfqpoint{5.785672in}{2.146863in}}%
\pgfusepath{clip}%
\pgfsetrectcap%
\pgfsetroundjoin%
\pgfsetlinewidth{0.100375pt}%
\definecolor{currentstroke}{rgb}{0.827451,0.827451,0.827451}%
\pgfsetstrokecolor{currentstroke}%
\pgfsetdash{}{0pt}%
\pgfpathmoveto{\pgfqpoint{3.876096in}{1.080890in}}%
\pgfpathlineto{\pgfqpoint{3.876096in}{3.227753in}}%
\pgfusepath{stroke}%
\end{pgfscope}%
\begin{pgfscope}%
\pgfsetbuttcap%
\pgfsetroundjoin%
\definecolor{currentfill}{rgb}{0.000000,0.000000,0.000000}%
\pgfsetfillcolor{currentfill}%
\pgfsetlinewidth{0.501875pt}%
\definecolor{currentstroke}{rgb}{0.000000,0.000000,0.000000}%
\pgfsetstrokecolor{currentstroke}%
\pgfsetdash{}{0pt}%
\pgfsys@defobject{currentmarker}{\pgfqpoint{0.000000in}{0.000000in}}{\pgfqpoint{0.000000in}{0.020833in}}{%
\pgfpathmoveto{\pgfqpoint{0.000000in}{0.000000in}}%
\pgfpathlineto{\pgfqpoint{0.000000in}{0.020833in}}%
\pgfusepath{stroke,fill}%
}%
\begin{pgfscope}%
\pgfsys@transformshift{3.876096in}{1.080890in}%
\pgfsys@useobject{currentmarker}{}%
\end{pgfscope}%
\end{pgfscope}%
\begin{pgfscope}%
\pgfsetbuttcap%
\pgfsetroundjoin%
\definecolor{currentfill}{rgb}{0.000000,0.000000,0.000000}%
\pgfsetfillcolor{currentfill}%
\pgfsetlinewidth{0.501875pt}%
\definecolor{currentstroke}{rgb}{0.000000,0.000000,0.000000}%
\pgfsetstrokecolor{currentstroke}%
\pgfsetdash{}{0pt}%
\pgfsys@defobject{currentmarker}{\pgfqpoint{0.000000in}{-0.020833in}}{\pgfqpoint{0.000000in}{0.000000in}}{%
\pgfpathmoveto{\pgfqpoint{0.000000in}{0.000000in}}%
\pgfpathlineto{\pgfqpoint{0.000000in}{-0.020833in}}%
\pgfusepath{stroke,fill}%
}%
\begin{pgfscope}%
\pgfsys@transformshift{3.876096in}{3.227753in}%
\pgfsys@useobject{currentmarker}{}%
\end{pgfscope}%
\end{pgfscope}%
\begin{pgfscope}%
\pgfpathrectangle{\pgfqpoint{0.481681in}{1.080890in}}{\pgfqpoint{5.785672in}{2.146863in}}%
\pgfusepath{clip}%
\pgfsetrectcap%
\pgfsetroundjoin%
\pgfsetlinewidth{0.100375pt}%
\definecolor{currentstroke}{rgb}{0.827451,0.827451,0.827451}%
\pgfsetstrokecolor{currentstroke}%
\pgfsetdash{}{0pt}%
\pgfpathmoveto{\pgfqpoint{3.911617in}{1.080890in}}%
\pgfpathlineto{\pgfqpoint{3.911617in}{3.227753in}}%
\pgfusepath{stroke}%
\end{pgfscope}%
\begin{pgfscope}%
\pgfsetbuttcap%
\pgfsetroundjoin%
\definecolor{currentfill}{rgb}{0.000000,0.000000,0.000000}%
\pgfsetfillcolor{currentfill}%
\pgfsetlinewidth{0.501875pt}%
\definecolor{currentstroke}{rgb}{0.000000,0.000000,0.000000}%
\pgfsetstrokecolor{currentstroke}%
\pgfsetdash{}{0pt}%
\pgfsys@defobject{currentmarker}{\pgfqpoint{0.000000in}{0.000000in}}{\pgfqpoint{0.000000in}{0.020833in}}{%
\pgfpathmoveto{\pgfqpoint{0.000000in}{0.000000in}}%
\pgfpathlineto{\pgfqpoint{0.000000in}{0.020833in}}%
\pgfusepath{stroke,fill}%
}%
\begin{pgfscope}%
\pgfsys@transformshift{3.911617in}{1.080890in}%
\pgfsys@useobject{currentmarker}{}%
\end{pgfscope}%
\end{pgfscope}%
\begin{pgfscope}%
\pgfsetbuttcap%
\pgfsetroundjoin%
\definecolor{currentfill}{rgb}{0.000000,0.000000,0.000000}%
\pgfsetfillcolor{currentfill}%
\pgfsetlinewidth{0.501875pt}%
\definecolor{currentstroke}{rgb}{0.000000,0.000000,0.000000}%
\pgfsetstrokecolor{currentstroke}%
\pgfsetdash{}{0pt}%
\pgfsys@defobject{currentmarker}{\pgfqpoint{0.000000in}{-0.020833in}}{\pgfqpoint{0.000000in}{0.000000in}}{%
\pgfpathmoveto{\pgfqpoint{0.000000in}{0.000000in}}%
\pgfpathlineto{\pgfqpoint{0.000000in}{-0.020833in}}%
\pgfusepath{stroke,fill}%
}%
\begin{pgfscope}%
\pgfsys@transformshift{3.911617in}{3.227753in}%
\pgfsys@useobject{currentmarker}{}%
\end{pgfscope}%
\end{pgfscope}%
\begin{pgfscope}%
\pgfpathrectangle{\pgfqpoint{0.481681in}{1.080890in}}{\pgfqpoint{5.785672in}{2.146863in}}%
\pgfusepath{clip}%
\pgfsetrectcap%
\pgfsetroundjoin%
\pgfsetlinewidth{0.100375pt}%
\definecolor{currentstroke}{rgb}{0.827451,0.827451,0.827451}%
\pgfsetstrokecolor{currentstroke}%
\pgfsetdash{}{0pt}%
\pgfpathmoveto{\pgfqpoint{3.947137in}{1.080890in}}%
\pgfpathlineto{\pgfqpoint{3.947137in}{3.227753in}}%
\pgfusepath{stroke}%
\end{pgfscope}%
\begin{pgfscope}%
\pgfsetbuttcap%
\pgfsetroundjoin%
\definecolor{currentfill}{rgb}{0.000000,0.000000,0.000000}%
\pgfsetfillcolor{currentfill}%
\pgfsetlinewidth{0.501875pt}%
\definecolor{currentstroke}{rgb}{0.000000,0.000000,0.000000}%
\pgfsetstrokecolor{currentstroke}%
\pgfsetdash{}{0pt}%
\pgfsys@defobject{currentmarker}{\pgfqpoint{0.000000in}{0.000000in}}{\pgfqpoint{0.000000in}{0.020833in}}{%
\pgfpathmoveto{\pgfqpoint{0.000000in}{0.000000in}}%
\pgfpathlineto{\pgfqpoint{0.000000in}{0.020833in}}%
\pgfusepath{stroke,fill}%
}%
\begin{pgfscope}%
\pgfsys@transformshift{3.947137in}{1.080890in}%
\pgfsys@useobject{currentmarker}{}%
\end{pgfscope}%
\end{pgfscope}%
\begin{pgfscope}%
\pgfsetbuttcap%
\pgfsetroundjoin%
\definecolor{currentfill}{rgb}{0.000000,0.000000,0.000000}%
\pgfsetfillcolor{currentfill}%
\pgfsetlinewidth{0.501875pt}%
\definecolor{currentstroke}{rgb}{0.000000,0.000000,0.000000}%
\pgfsetstrokecolor{currentstroke}%
\pgfsetdash{}{0pt}%
\pgfsys@defobject{currentmarker}{\pgfqpoint{0.000000in}{-0.020833in}}{\pgfqpoint{0.000000in}{0.000000in}}{%
\pgfpathmoveto{\pgfqpoint{0.000000in}{0.000000in}}%
\pgfpathlineto{\pgfqpoint{0.000000in}{-0.020833in}}%
\pgfusepath{stroke,fill}%
}%
\begin{pgfscope}%
\pgfsys@transformshift{3.947137in}{3.227753in}%
\pgfsys@useobject{currentmarker}{}%
\end{pgfscope}%
\end{pgfscope}%
\begin{pgfscope}%
\pgfpathrectangle{\pgfqpoint{0.481681in}{1.080890in}}{\pgfqpoint{5.785672in}{2.146863in}}%
\pgfusepath{clip}%
\pgfsetrectcap%
\pgfsetroundjoin%
\pgfsetlinewidth{0.100375pt}%
\definecolor{currentstroke}{rgb}{0.827451,0.827451,0.827451}%
\pgfsetstrokecolor{currentstroke}%
\pgfsetdash{}{0pt}%
\pgfpathmoveto{\pgfqpoint{3.982658in}{1.080890in}}%
\pgfpathlineto{\pgfqpoint{3.982658in}{3.227753in}}%
\pgfusepath{stroke}%
\end{pgfscope}%
\begin{pgfscope}%
\pgfsetbuttcap%
\pgfsetroundjoin%
\definecolor{currentfill}{rgb}{0.000000,0.000000,0.000000}%
\pgfsetfillcolor{currentfill}%
\pgfsetlinewidth{0.501875pt}%
\definecolor{currentstroke}{rgb}{0.000000,0.000000,0.000000}%
\pgfsetstrokecolor{currentstroke}%
\pgfsetdash{}{0pt}%
\pgfsys@defobject{currentmarker}{\pgfqpoint{0.000000in}{0.000000in}}{\pgfqpoint{0.000000in}{0.020833in}}{%
\pgfpathmoveto{\pgfqpoint{0.000000in}{0.000000in}}%
\pgfpathlineto{\pgfqpoint{0.000000in}{0.020833in}}%
\pgfusepath{stroke,fill}%
}%
\begin{pgfscope}%
\pgfsys@transformshift{3.982658in}{1.080890in}%
\pgfsys@useobject{currentmarker}{}%
\end{pgfscope}%
\end{pgfscope}%
\begin{pgfscope}%
\pgfsetbuttcap%
\pgfsetroundjoin%
\definecolor{currentfill}{rgb}{0.000000,0.000000,0.000000}%
\pgfsetfillcolor{currentfill}%
\pgfsetlinewidth{0.501875pt}%
\definecolor{currentstroke}{rgb}{0.000000,0.000000,0.000000}%
\pgfsetstrokecolor{currentstroke}%
\pgfsetdash{}{0pt}%
\pgfsys@defobject{currentmarker}{\pgfqpoint{0.000000in}{-0.020833in}}{\pgfqpoint{0.000000in}{0.000000in}}{%
\pgfpathmoveto{\pgfqpoint{0.000000in}{0.000000in}}%
\pgfpathlineto{\pgfqpoint{0.000000in}{-0.020833in}}%
\pgfusepath{stroke,fill}%
}%
\begin{pgfscope}%
\pgfsys@transformshift{3.982658in}{3.227753in}%
\pgfsys@useobject{currentmarker}{}%
\end{pgfscope}%
\end{pgfscope}%
\begin{pgfscope}%
\pgfpathrectangle{\pgfqpoint{0.481681in}{1.080890in}}{\pgfqpoint{5.785672in}{2.146863in}}%
\pgfusepath{clip}%
\pgfsetrectcap%
\pgfsetroundjoin%
\pgfsetlinewidth{0.100375pt}%
\definecolor{currentstroke}{rgb}{0.827451,0.827451,0.827451}%
\pgfsetstrokecolor{currentstroke}%
\pgfsetdash{}{0pt}%
\pgfpathmoveto{\pgfqpoint{4.018178in}{1.080890in}}%
\pgfpathlineto{\pgfqpoint{4.018178in}{3.227753in}}%
\pgfusepath{stroke}%
\end{pgfscope}%
\begin{pgfscope}%
\pgfsetbuttcap%
\pgfsetroundjoin%
\definecolor{currentfill}{rgb}{0.000000,0.000000,0.000000}%
\pgfsetfillcolor{currentfill}%
\pgfsetlinewidth{0.501875pt}%
\definecolor{currentstroke}{rgb}{0.000000,0.000000,0.000000}%
\pgfsetstrokecolor{currentstroke}%
\pgfsetdash{}{0pt}%
\pgfsys@defobject{currentmarker}{\pgfqpoint{0.000000in}{0.000000in}}{\pgfqpoint{0.000000in}{0.020833in}}{%
\pgfpathmoveto{\pgfqpoint{0.000000in}{0.000000in}}%
\pgfpathlineto{\pgfqpoint{0.000000in}{0.020833in}}%
\pgfusepath{stroke,fill}%
}%
\begin{pgfscope}%
\pgfsys@transformshift{4.018178in}{1.080890in}%
\pgfsys@useobject{currentmarker}{}%
\end{pgfscope}%
\end{pgfscope}%
\begin{pgfscope}%
\pgfsetbuttcap%
\pgfsetroundjoin%
\definecolor{currentfill}{rgb}{0.000000,0.000000,0.000000}%
\pgfsetfillcolor{currentfill}%
\pgfsetlinewidth{0.501875pt}%
\definecolor{currentstroke}{rgb}{0.000000,0.000000,0.000000}%
\pgfsetstrokecolor{currentstroke}%
\pgfsetdash{}{0pt}%
\pgfsys@defobject{currentmarker}{\pgfqpoint{0.000000in}{-0.020833in}}{\pgfqpoint{0.000000in}{0.000000in}}{%
\pgfpathmoveto{\pgfqpoint{0.000000in}{0.000000in}}%
\pgfpathlineto{\pgfqpoint{0.000000in}{-0.020833in}}%
\pgfusepath{stroke,fill}%
}%
\begin{pgfscope}%
\pgfsys@transformshift{4.018178in}{3.227753in}%
\pgfsys@useobject{currentmarker}{}%
\end{pgfscope}%
\end{pgfscope}%
\begin{pgfscope}%
\pgfpathrectangle{\pgfqpoint{0.481681in}{1.080890in}}{\pgfqpoint{5.785672in}{2.146863in}}%
\pgfusepath{clip}%
\pgfsetrectcap%
\pgfsetroundjoin%
\pgfsetlinewidth{0.100375pt}%
\definecolor{currentstroke}{rgb}{0.827451,0.827451,0.827451}%
\pgfsetstrokecolor{currentstroke}%
\pgfsetdash{}{0pt}%
\pgfpathmoveto{\pgfqpoint{4.089220in}{1.080890in}}%
\pgfpathlineto{\pgfqpoint{4.089220in}{3.227753in}}%
\pgfusepath{stroke}%
\end{pgfscope}%
\begin{pgfscope}%
\pgfsetbuttcap%
\pgfsetroundjoin%
\definecolor{currentfill}{rgb}{0.000000,0.000000,0.000000}%
\pgfsetfillcolor{currentfill}%
\pgfsetlinewidth{0.501875pt}%
\definecolor{currentstroke}{rgb}{0.000000,0.000000,0.000000}%
\pgfsetstrokecolor{currentstroke}%
\pgfsetdash{}{0pt}%
\pgfsys@defobject{currentmarker}{\pgfqpoint{0.000000in}{0.000000in}}{\pgfqpoint{0.000000in}{0.020833in}}{%
\pgfpathmoveto{\pgfqpoint{0.000000in}{0.000000in}}%
\pgfpathlineto{\pgfqpoint{0.000000in}{0.020833in}}%
\pgfusepath{stroke,fill}%
}%
\begin{pgfscope}%
\pgfsys@transformshift{4.089220in}{1.080890in}%
\pgfsys@useobject{currentmarker}{}%
\end{pgfscope}%
\end{pgfscope}%
\begin{pgfscope}%
\pgfsetbuttcap%
\pgfsetroundjoin%
\definecolor{currentfill}{rgb}{0.000000,0.000000,0.000000}%
\pgfsetfillcolor{currentfill}%
\pgfsetlinewidth{0.501875pt}%
\definecolor{currentstroke}{rgb}{0.000000,0.000000,0.000000}%
\pgfsetstrokecolor{currentstroke}%
\pgfsetdash{}{0pt}%
\pgfsys@defobject{currentmarker}{\pgfqpoint{0.000000in}{-0.020833in}}{\pgfqpoint{0.000000in}{0.000000in}}{%
\pgfpathmoveto{\pgfqpoint{0.000000in}{0.000000in}}%
\pgfpathlineto{\pgfqpoint{0.000000in}{-0.020833in}}%
\pgfusepath{stroke,fill}%
}%
\begin{pgfscope}%
\pgfsys@transformshift{4.089220in}{3.227753in}%
\pgfsys@useobject{currentmarker}{}%
\end{pgfscope}%
\end{pgfscope}%
\begin{pgfscope}%
\pgfpathrectangle{\pgfqpoint{0.481681in}{1.080890in}}{\pgfqpoint{5.785672in}{2.146863in}}%
\pgfusepath{clip}%
\pgfsetrectcap%
\pgfsetroundjoin%
\pgfsetlinewidth{0.100375pt}%
\definecolor{currentstroke}{rgb}{0.827451,0.827451,0.827451}%
\pgfsetstrokecolor{currentstroke}%
\pgfsetdash{}{0pt}%
\pgfpathmoveto{\pgfqpoint{4.124740in}{1.080890in}}%
\pgfpathlineto{\pgfqpoint{4.124740in}{3.227753in}}%
\pgfusepath{stroke}%
\end{pgfscope}%
\begin{pgfscope}%
\pgfsetbuttcap%
\pgfsetroundjoin%
\definecolor{currentfill}{rgb}{0.000000,0.000000,0.000000}%
\pgfsetfillcolor{currentfill}%
\pgfsetlinewidth{0.501875pt}%
\definecolor{currentstroke}{rgb}{0.000000,0.000000,0.000000}%
\pgfsetstrokecolor{currentstroke}%
\pgfsetdash{}{0pt}%
\pgfsys@defobject{currentmarker}{\pgfqpoint{0.000000in}{0.000000in}}{\pgfqpoint{0.000000in}{0.020833in}}{%
\pgfpathmoveto{\pgfqpoint{0.000000in}{0.000000in}}%
\pgfpathlineto{\pgfqpoint{0.000000in}{0.020833in}}%
\pgfusepath{stroke,fill}%
}%
\begin{pgfscope}%
\pgfsys@transformshift{4.124740in}{1.080890in}%
\pgfsys@useobject{currentmarker}{}%
\end{pgfscope}%
\end{pgfscope}%
\begin{pgfscope}%
\pgfsetbuttcap%
\pgfsetroundjoin%
\definecolor{currentfill}{rgb}{0.000000,0.000000,0.000000}%
\pgfsetfillcolor{currentfill}%
\pgfsetlinewidth{0.501875pt}%
\definecolor{currentstroke}{rgb}{0.000000,0.000000,0.000000}%
\pgfsetstrokecolor{currentstroke}%
\pgfsetdash{}{0pt}%
\pgfsys@defobject{currentmarker}{\pgfqpoint{0.000000in}{-0.020833in}}{\pgfqpoint{0.000000in}{0.000000in}}{%
\pgfpathmoveto{\pgfqpoint{0.000000in}{0.000000in}}%
\pgfpathlineto{\pgfqpoint{0.000000in}{-0.020833in}}%
\pgfusepath{stroke,fill}%
}%
\begin{pgfscope}%
\pgfsys@transformshift{4.124740in}{3.227753in}%
\pgfsys@useobject{currentmarker}{}%
\end{pgfscope}%
\end{pgfscope}%
\begin{pgfscope}%
\pgfpathrectangle{\pgfqpoint{0.481681in}{1.080890in}}{\pgfqpoint{5.785672in}{2.146863in}}%
\pgfusepath{clip}%
\pgfsetrectcap%
\pgfsetroundjoin%
\pgfsetlinewidth{0.100375pt}%
\definecolor{currentstroke}{rgb}{0.827451,0.827451,0.827451}%
\pgfsetstrokecolor{currentstroke}%
\pgfsetdash{}{0pt}%
\pgfpathmoveto{\pgfqpoint{4.160261in}{1.080890in}}%
\pgfpathlineto{\pgfqpoint{4.160261in}{3.227753in}}%
\pgfusepath{stroke}%
\end{pgfscope}%
\begin{pgfscope}%
\pgfsetbuttcap%
\pgfsetroundjoin%
\definecolor{currentfill}{rgb}{0.000000,0.000000,0.000000}%
\pgfsetfillcolor{currentfill}%
\pgfsetlinewidth{0.501875pt}%
\definecolor{currentstroke}{rgb}{0.000000,0.000000,0.000000}%
\pgfsetstrokecolor{currentstroke}%
\pgfsetdash{}{0pt}%
\pgfsys@defobject{currentmarker}{\pgfqpoint{0.000000in}{0.000000in}}{\pgfqpoint{0.000000in}{0.020833in}}{%
\pgfpathmoveto{\pgfqpoint{0.000000in}{0.000000in}}%
\pgfpathlineto{\pgfqpoint{0.000000in}{0.020833in}}%
\pgfusepath{stroke,fill}%
}%
\begin{pgfscope}%
\pgfsys@transformshift{4.160261in}{1.080890in}%
\pgfsys@useobject{currentmarker}{}%
\end{pgfscope}%
\end{pgfscope}%
\begin{pgfscope}%
\pgfsetbuttcap%
\pgfsetroundjoin%
\definecolor{currentfill}{rgb}{0.000000,0.000000,0.000000}%
\pgfsetfillcolor{currentfill}%
\pgfsetlinewidth{0.501875pt}%
\definecolor{currentstroke}{rgb}{0.000000,0.000000,0.000000}%
\pgfsetstrokecolor{currentstroke}%
\pgfsetdash{}{0pt}%
\pgfsys@defobject{currentmarker}{\pgfqpoint{0.000000in}{-0.020833in}}{\pgfqpoint{0.000000in}{0.000000in}}{%
\pgfpathmoveto{\pgfqpoint{0.000000in}{0.000000in}}%
\pgfpathlineto{\pgfqpoint{0.000000in}{-0.020833in}}%
\pgfusepath{stroke,fill}%
}%
\begin{pgfscope}%
\pgfsys@transformshift{4.160261in}{3.227753in}%
\pgfsys@useobject{currentmarker}{}%
\end{pgfscope}%
\end{pgfscope}%
\begin{pgfscope}%
\pgfpathrectangle{\pgfqpoint{0.481681in}{1.080890in}}{\pgfqpoint{5.785672in}{2.146863in}}%
\pgfusepath{clip}%
\pgfsetrectcap%
\pgfsetroundjoin%
\pgfsetlinewidth{0.100375pt}%
\definecolor{currentstroke}{rgb}{0.827451,0.827451,0.827451}%
\pgfsetstrokecolor{currentstroke}%
\pgfsetdash{}{0pt}%
\pgfpathmoveto{\pgfqpoint{4.195781in}{1.080890in}}%
\pgfpathlineto{\pgfqpoint{4.195781in}{3.227753in}}%
\pgfusepath{stroke}%
\end{pgfscope}%
\begin{pgfscope}%
\pgfsetbuttcap%
\pgfsetroundjoin%
\definecolor{currentfill}{rgb}{0.000000,0.000000,0.000000}%
\pgfsetfillcolor{currentfill}%
\pgfsetlinewidth{0.501875pt}%
\definecolor{currentstroke}{rgb}{0.000000,0.000000,0.000000}%
\pgfsetstrokecolor{currentstroke}%
\pgfsetdash{}{0pt}%
\pgfsys@defobject{currentmarker}{\pgfqpoint{0.000000in}{0.000000in}}{\pgfqpoint{0.000000in}{0.020833in}}{%
\pgfpathmoveto{\pgfqpoint{0.000000in}{0.000000in}}%
\pgfpathlineto{\pgfqpoint{0.000000in}{0.020833in}}%
\pgfusepath{stroke,fill}%
}%
\begin{pgfscope}%
\pgfsys@transformshift{4.195781in}{1.080890in}%
\pgfsys@useobject{currentmarker}{}%
\end{pgfscope}%
\end{pgfscope}%
\begin{pgfscope}%
\pgfsetbuttcap%
\pgfsetroundjoin%
\definecolor{currentfill}{rgb}{0.000000,0.000000,0.000000}%
\pgfsetfillcolor{currentfill}%
\pgfsetlinewidth{0.501875pt}%
\definecolor{currentstroke}{rgb}{0.000000,0.000000,0.000000}%
\pgfsetstrokecolor{currentstroke}%
\pgfsetdash{}{0pt}%
\pgfsys@defobject{currentmarker}{\pgfqpoint{0.000000in}{-0.020833in}}{\pgfqpoint{0.000000in}{0.000000in}}{%
\pgfpathmoveto{\pgfqpoint{0.000000in}{0.000000in}}%
\pgfpathlineto{\pgfqpoint{0.000000in}{-0.020833in}}%
\pgfusepath{stroke,fill}%
}%
\begin{pgfscope}%
\pgfsys@transformshift{4.195781in}{3.227753in}%
\pgfsys@useobject{currentmarker}{}%
\end{pgfscope}%
\end{pgfscope}%
\begin{pgfscope}%
\pgfpathrectangle{\pgfqpoint{0.481681in}{1.080890in}}{\pgfqpoint{5.785672in}{2.146863in}}%
\pgfusepath{clip}%
\pgfsetrectcap%
\pgfsetroundjoin%
\pgfsetlinewidth{0.100375pt}%
\definecolor{currentstroke}{rgb}{0.827451,0.827451,0.827451}%
\pgfsetstrokecolor{currentstroke}%
\pgfsetdash{}{0pt}%
\pgfpathmoveto{\pgfqpoint{4.231302in}{1.080890in}}%
\pgfpathlineto{\pgfqpoint{4.231302in}{3.227753in}}%
\pgfusepath{stroke}%
\end{pgfscope}%
\begin{pgfscope}%
\pgfsetbuttcap%
\pgfsetroundjoin%
\definecolor{currentfill}{rgb}{0.000000,0.000000,0.000000}%
\pgfsetfillcolor{currentfill}%
\pgfsetlinewidth{0.501875pt}%
\definecolor{currentstroke}{rgb}{0.000000,0.000000,0.000000}%
\pgfsetstrokecolor{currentstroke}%
\pgfsetdash{}{0pt}%
\pgfsys@defobject{currentmarker}{\pgfqpoint{0.000000in}{0.000000in}}{\pgfqpoint{0.000000in}{0.020833in}}{%
\pgfpathmoveto{\pgfqpoint{0.000000in}{0.000000in}}%
\pgfpathlineto{\pgfqpoint{0.000000in}{0.020833in}}%
\pgfusepath{stroke,fill}%
}%
\begin{pgfscope}%
\pgfsys@transformshift{4.231302in}{1.080890in}%
\pgfsys@useobject{currentmarker}{}%
\end{pgfscope}%
\end{pgfscope}%
\begin{pgfscope}%
\pgfsetbuttcap%
\pgfsetroundjoin%
\definecolor{currentfill}{rgb}{0.000000,0.000000,0.000000}%
\pgfsetfillcolor{currentfill}%
\pgfsetlinewidth{0.501875pt}%
\definecolor{currentstroke}{rgb}{0.000000,0.000000,0.000000}%
\pgfsetstrokecolor{currentstroke}%
\pgfsetdash{}{0pt}%
\pgfsys@defobject{currentmarker}{\pgfqpoint{0.000000in}{-0.020833in}}{\pgfqpoint{0.000000in}{0.000000in}}{%
\pgfpathmoveto{\pgfqpoint{0.000000in}{0.000000in}}%
\pgfpathlineto{\pgfqpoint{0.000000in}{-0.020833in}}%
\pgfusepath{stroke,fill}%
}%
\begin{pgfscope}%
\pgfsys@transformshift{4.231302in}{3.227753in}%
\pgfsys@useobject{currentmarker}{}%
\end{pgfscope}%
\end{pgfscope}%
\begin{pgfscope}%
\pgfpathrectangle{\pgfqpoint{0.481681in}{1.080890in}}{\pgfqpoint{5.785672in}{2.146863in}}%
\pgfusepath{clip}%
\pgfsetrectcap%
\pgfsetroundjoin%
\pgfsetlinewidth{0.100375pt}%
\definecolor{currentstroke}{rgb}{0.827451,0.827451,0.827451}%
\pgfsetstrokecolor{currentstroke}%
\pgfsetdash{}{0pt}%
\pgfpathmoveto{\pgfqpoint{4.266822in}{1.080890in}}%
\pgfpathlineto{\pgfqpoint{4.266822in}{3.227753in}}%
\pgfusepath{stroke}%
\end{pgfscope}%
\begin{pgfscope}%
\pgfsetbuttcap%
\pgfsetroundjoin%
\definecolor{currentfill}{rgb}{0.000000,0.000000,0.000000}%
\pgfsetfillcolor{currentfill}%
\pgfsetlinewidth{0.501875pt}%
\definecolor{currentstroke}{rgb}{0.000000,0.000000,0.000000}%
\pgfsetstrokecolor{currentstroke}%
\pgfsetdash{}{0pt}%
\pgfsys@defobject{currentmarker}{\pgfqpoint{0.000000in}{0.000000in}}{\pgfqpoint{0.000000in}{0.020833in}}{%
\pgfpathmoveto{\pgfqpoint{0.000000in}{0.000000in}}%
\pgfpathlineto{\pgfqpoint{0.000000in}{0.020833in}}%
\pgfusepath{stroke,fill}%
}%
\begin{pgfscope}%
\pgfsys@transformshift{4.266822in}{1.080890in}%
\pgfsys@useobject{currentmarker}{}%
\end{pgfscope}%
\end{pgfscope}%
\begin{pgfscope}%
\pgfsetbuttcap%
\pgfsetroundjoin%
\definecolor{currentfill}{rgb}{0.000000,0.000000,0.000000}%
\pgfsetfillcolor{currentfill}%
\pgfsetlinewidth{0.501875pt}%
\definecolor{currentstroke}{rgb}{0.000000,0.000000,0.000000}%
\pgfsetstrokecolor{currentstroke}%
\pgfsetdash{}{0pt}%
\pgfsys@defobject{currentmarker}{\pgfqpoint{0.000000in}{-0.020833in}}{\pgfqpoint{0.000000in}{0.000000in}}{%
\pgfpathmoveto{\pgfqpoint{0.000000in}{0.000000in}}%
\pgfpathlineto{\pgfqpoint{0.000000in}{-0.020833in}}%
\pgfusepath{stroke,fill}%
}%
\begin{pgfscope}%
\pgfsys@transformshift{4.266822in}{3.227753in}%
\pgfsys@useobject{currentmarker}{}%
\end{pgfscope}%
\end{pgfscope}%
\begin{pgfscope}%
\pgfpathrectangle{\pgfqpoint{0.481681in}{1.080890in}}{\pgfqpoint{5.785672in}{2.146863in}}%
\pgfusepath{clip}%
\pgfsetrectcap%
\pgfsetroundjoin%
\pgfsetlinewidth{0.100375pt}%
\definecolor{currentstroke}{rgb}{0.827451,0.827451,0.827451}%
\pgfsetstrokecolor{currentstroke}%
\pgfsetdash{}{0pt}%
\pgfpathmoveto{\pgfqpoint{4.302343in}{1.080890in}}%
\pgfpathlineto{\pgfqpoint{4.302343in}{3.227753in}}%
\pgfusepath{stroke}%
\end{pgfscope}%
\begin{pgfscope}%
\pgfsetbuttcap%
\pgfsetroundjoin%
\definecolor{currentfill}{rgb}{0.000000,0.000000,0.000000}%
\pgfsetfillcolor{currentfill}%
\pgfsetlinewidth{0.501875pt}%
\definecolor{currentstroke}{rgb}{0.000000,0.000000,0.000000}%
\pgfsetstrokecolor{currentstroke}%
\pgfsetdash{}{0pt}%
\pgfsys@defobject{currentmarker}{\pgfqpoint{0.000000in}{0.000000in}}{\pgfqpoint{0.000000in}{0.020833in}}{%
\pgfpathmoveto{\pgfqpoint{0.000000in}{0.000000in}}%
\pgfpathlineto{\pgfqpoint{0.000000in}{0.020833in}}%
\pgfusepath{stroke,fill}%
}%
\begin{pgfscope}%
\pgfsys@transformshift{4.302343in}{1.080890in}%
\pgfsys@useobject{currentmarker}{}%
\end{pgfscope}%
\end{pgfscope}%
\begin{pgfscope}%
\pgfsetbuttcap%
\pgfsetroundjoin%
\definecolor{currentfill}{rgb}{0.000000,0.000000,0.000000}%
\pgfsetfillcolor{currentfill}%
\pgfsetlinewidth{0.501875pt}%
\definecolor{currentstroke}{rgb}{0.000000,0.000000,0.000000}%
\pgfsetstrokecolor{currentstroke}%
\pgfsetdash{}{0pt}%
\pgfsys@defobject{currentmarker}{\pgfqpoint{0.000000in}{-0.020833in}}{\pgfqpoint{0.000000in}{0.000000in}}{%
\pgfpathmoveto{\pgfqpoint{0.000000in}{0.000000in}}%
\pgfpathlineto{\pgfqpoint{0.000000in}{-0.020833in}}%
\pgfusepath{stroke,fill}%
}%
\begin{pgfscope}%
\pgfsys@transformshift{4.302343in}{3.227753in}%
\pgfsys@useobject{currentmarker}{}%
\end{pgfscope}%
\end{pgfscope}%
\begin{pgfscope}%
\pgfpathrectangle{\pgfqpoint{0.481681in}{1.080890in}}{\pgfqpoint{5.785672in}{2.146863in}}%
\pgfusepath{clip}%
\pgfsetrectcap%
\pgfsetroundjoin%
\pgfsetlinewidth{0.100375pt}%
\definecolor{currentstroke}{rgb}{0.827451,0.827451,0.827451}%
\pgfsetstrokecolor{currentstroke}%
\pgfsetdash{}{0pt}%
\pgfpathmoveto{\pgfqpoint{4.337863in}{1.080890in}}%
\pgfpathlineto{\pgfqpoint{4.337863in}{3.227753in}}%
\pgfusepath{stroke}%
\end{pgfscope}%
\begin{pgfscope}%
\pgfsetbuttcap%
\pgfsetroundjoin%
\definecolor{currentfill}{rgb}{0.000000,0.000000,0.000000}%
\pgfsetfillcolor{currentfill}%
\pgfsetlinewidth{0.501875pt}%
\definecolor{currentstroke}{rgb}{0.000000,0.000000,0.000000}%
\pgfsetstrokecolor{currentstroke}%
\pgfsetdash{}{0pt}%
\pgfsys@defobject{currentmarker}{\pgfqpoint{0.000000in}{0.000000in}}{\pgfqpoint{0.000000in}{0.020833in}}{%
\pgfpathmoveto{\pgfqpoint{0.000000in}{0.000000in}}%
\pgfpathlineto{\pgfqpoint{0.000000in}{0.020833in}}%
\pgfusepath{stroke,fill}%
}%
\begin{pgfscope}%
\pgfsys@transformshift{4.337863in}{1.080890in}%
\pgfsys@useobject{currentmarker}{}%
\end{pgfscope}%
\end{pgfscope}%
\begin{pgfscope}%
\pgfsetbuttcap%
\pgfsetroundjoin%
\definecolor{currentfill}{rgb}{0.000000,0.000000,0.000000}%
\pgfsetfillcolor{currentfill}%
\pgfsetlinewidth{0.501875pt}%
\definecolor{currentstroke}{rgb}{0.000000,0.000000,0.000000}%
\pgfsetstrokecolor{currentstroke}%
\pgfsetdash{}{0pt}%
\pgfsys@defobject{currentmarker}{\pgfqpoint{0.000000in}{-0.020833in}}{\pgfqpoint{0.000000in}{0.000000in}}{%
\pgfpathmoveto{\pgfqpoint{0.000000in}{0.000000in}}%
\pgfpathlineto{\pgfqpoint{0.000000in}{-0.020833in}}%
\pgfusepath{stroke,fill}%
}%
\begin{pgfscope}%
\pgfsys@transformshift{4.337863in}{3.227753in}%
\pgfsys@useobject{currentmarker}{}%
\end{pgfscope}%
\end{pgfscope}%
\begin{pgfscope}%
\pgfpathrectangle{\pgfqpoint{0.481681in}{1.080890in}}{\pgfqpoint{5.785672in}{2.146863in}}%
\pgfusepath{clip}%
\pgfsetrectcap%
\pgfsetroundjoin%
\pgfsetlinewidth{0.100375pt}%
\definecolor{currentstroke}{rgb}{0.827451,0.827451,0.827451}%
\pgfsetstrokecolor{currentstroke}%
\pgfsetdash{}{0pt}%
\pgfpathmoveto{\pgfqpoint{4.373384in}{1.080890in}}%
\pgfpathlineto{\pgfqpoint{4.373384in}{3.227753in}}%
\pgfusepath{stroke}%
\end{pgfscope}%
\begin{pgfscope}%
\pgfsetbuttcap%
\pgfsetroundjoin%
\definecolor{currentfill}{rgb}{0.000000,0.000000,0.000000}%
\pgfsetfillcolor{currentfill}%
\pgfsetlinewidth{0.501875pt}%
\definecolor{currentstroke}{rgb}{0.000000,0.000000,0.000000}%
\pgfsetstrokecolor{currentstroke}%
\pgfsetdash{}{0pt}%
\pgfsys@defobject{currentmarker}{\pgfqpoint{0.000000in}{0.000000in}}{\pgfqpoint{0.000000in}{0.020833in}}{%
\pgfpathmoveto{\pgfqpoint{0.000000in}{0.000000in}}%
\pgfpathlineto{\pgfqpoint{0.000000in}{0.020833in}}%
\pgfusepath{stroke,fill}%
}%
\begin{pgfscope}%
\pgfsys@transformshift{4.373384in}{1.080890in}%
\pgfsys@useobject{currentmarker}{}%
\end{pgfscope}%
\end{pgfscope}%
\begin{pgfscope}%
\pgfsetbuttcap%
\pgfsetroundjoin%
\definecolor{currentfill}{rgb}{0.000000,0.000000,0.000000}%
\pgfsetfillcolor{currentfill}%
\pgfsetlinewidth{0.501875pt}%
\definecolor{currentstroke}{rgb}{0.000000,0.000000,0.000000}%
\pgfsetstrokecolor{currentstroke}%
\pgfsetdash{}{0pt}%
\pgfsys@defobject{currentmarker}{\pgfqpoint{0.000000in}{-0.020833in}}{\pgfqpoint{0.000000in}{0.000000in}}{%
\pgfpathmoveto{\pgfqpoint{0.000000in}{0.000000in}}%
\pgfpathlineto{\pgfqpoint{0.000000in}{-0.020833in}}%
\pgfusepath{stroke,fill}%
}%
\begin{pgfscope}%
\pgfsys@transformshift{4.373384in}{3.227753in}%
\pgfsys@useobject{currentmarker}{}%
\end{pgfscope}%
\end{pgfscope}%
\begin{pgfscope}%
\pgfpathrectangle{\pgfqpoint{0.481681in}{1.080890in}}{\pgfqpoint{5.785672in}{2.146863in}}%
\pgfusepath{clip}%
\pgfsetrectcap%
\pgfsetroundjoin%
\pgfsetlinewidth{0.100375pt}%
\definecolor{currentstroke}{rgb}{0.827451,0.827451,0.827451}%
\pgfsetstrokecolor{currentstroke}%
\pgfsetdash{}{0pt}%
\pgfpathmoveto{\pgfqpoint{4.408904in}{1.080890in}}%
\pgfpathlineto{\pgfqpoint{4.408904in}{3.227753in}}%
\pgfusepath{stroke}%
\end{pgfscope}%
\begin{pgfscope}%
\pgfsetbuttcap%
\pgfsetroundjoin%
\definecolor{currentfill}{rgb}{0.000000,0.000000,0.000000}%
\pgfsetfillcolor{currentfill}%
\pgfsetlinewidth{0.501875pt}%
\definecolor{currentstroke}{rgb}{0.000000,0.000000,0.000000}%
\pgfsetstrokecolor{currentstroke}%
\pgfsetdash{}{0pt}%
\pgfsys@defobject{currentmarker}{\pgfqpoint{0.000000in}{0.000000in}}{\pgfqpoint{0.000000in}{0.020833in}}{%
\pgfpathmoveto{\pgfqpoint{0.000000in}{0.000000in}}%
\pgfpathlineto{\pgfqpoint{0.000000in}{0.020833in}}%
\pgfusepath{stroke,fill}%
}%
\begin{pgfscope}%
\pgfsys@transformshift{4.408904in}{1.080890in}%
\pgfsys@useobject{currentmarker}{}%
\end{pgfscope}%
\end{pgfscope}%
\begin{pgfscope}%
\pgfsetbuttcap%
\pgfsetroundjoin%
\definecolor{currentfill}{rgb}{0.000000,0.000000,0.000000}%
\pgfsetfillcolor{currentfill}%
\pgfsetlinewidth{0.501875pt}%
\definecolor{currentstroke}{rgb}{0.000000,0.000000,0.000000}%
\pgfsetstrokecolor{currentstroke}%
\pgfsetdash{}{0pt}%
\pgfsys@defobject{currentmarker}{\pgfqpoint{0.000000in}{-0.020833in}}{\pgfqpoint{0.000000in}{0.000000in}}{%
\pgfpathmoveto{\pgfqpoint{0.000000in}{0.000000in}}%
\pgfpathlineto{\pgfqpoint{0.000000in}{-0.020833in}}%
\pgfusepath{stroke,fill}%
}%
\begin{pgfscope}%
\pgfsys@transformshift{4.408904in}{3.227753in}%
\pgfsys@useobject{currentmarker}{}%
\end{pgfscope}%
\end{pgfscope}%
\begin{pgfscope}%
\pgfpathrectangle{\pgfqpoint{0.481681in}{1.080890in}}{\pgfqpoint{5.785672in}{2.146863in}}%
\pgfusepath{clip}%
\pgfsetrectcap%
\pgfsetroundjoin%
\pgfsetlinewidth{0.100375pt}%
\definecolor{currentstroke}{rgb}{0.827451,0.827451,0.827451}%
\pgfsetstrokecolor{currentstroke}%
\pgfsetdash{}{0pt}%
\pgfpathmoveto{\pgfqpoint{4.444425in}{1.080890in}}%
\pgfpathlineto{\pgfqpoint{4.444425in}{3.227753in}}%
\pgfusepath{stroke}%
\end{pgfscope}%
\begin{pgfscope}%
\pgfsetbuttcap%
\pgfsetroundjoin%
\definecolor{currentfill}{rgb}{0.000000,0.000000,0.000000}%
\pgfsetfillcolor{currentfill}%
\pgfsetlinewidth{0.501875pt}%
\definecolor{currentstroke}{rgb}{0.000000,0.000000,0.000000}%
\pgfsetstrokecolor{currentstroke}%
\pgfsetdash{}{0pt}%
\pgfsys@defobject{currentmarker}{\pgfqpoint{0.000000in}{0.000000in}}{\pgfqpoint{0.000000in}{0.020833in}}{%
\pgfpathmoveto{\pgfqpoint{0.000000in}{0.000000in}}%
\pgfpathlineto{\pgfqpoint{0.000000in}{0.020833in}}%
\pgfusepath{stroke,fill}%
}%
\begin{pgfscope}%
\pgfsys@transformshift{4.444425in}{1.080890in}%
\pgfsys@useobject{currentmarker}{}%
\end{pgfscope}%
\end{pgfscope}%
\begin{pgfscope}%
\pgfsetbuttcap%
\pgfsetroundjoin%
\definecolor{currentfill}{rgb}{0.000000,0.000000,0.000000}%
\pgfsetfillcolor{currentfill}%
\pgfsetlinewidth{0.501875pt}%
\definecolor{currentstroke}{rgb}{0.000000,0.000000,0.000000}%
\pgfsetstrokecolor{currentstroke}%
\pgfsetdash{}{0pt}%
\pgfsys@defobject{currentmarker}{\pgfqpoint{0.000000in}{-0.020833in}}{\pgfqpoint{0.000000in}{0.000000in}}{%
\pgfpathmoveto{\pgfqpoint{0.000000in}{0.000000in}}%
\pgfpathlineto{\pgfqpoint{0.000000in}{-0.020833in}}%
\pgfusepath{stroke,fill}%
}%
\begin{pgfscope}%
\pgfsys@transformshift{4.444425in}{3.227753in}%
\pgfsys@useobject{currentmarker}{}%
\end{pgfscope}%
\end{pgfscope}%
\begin{pgfscope}%
\pgfpathrectangle{\pgfqpoint{0.481681in}{1.080890in}}{\pgfqpoint{5.785672in}{2.146863in}}%
\pgfusepath{clip}%
\pgfsetrectcap%
\pgfsetroundjoin%
\pgfsetlinewidth{0.100375pt}%
\definecolor{currentstroke}{rgb}{0.827451,0.827451,0.827451}%
\pgfsetstrokecolor{currentstroke}%
\pgfsetdash{}{0pt}%
\pgfpathmoveto{\pgfqpoint{4.515466in}{1.080890in}}%
\pgfpathlineto{\pgfqpoint{4.515466in}{3.227753in}}%
\pgfusepath{stroke}%
\end{pgfscope}%
\begin{pgfscope}%
\pgfsetbuttcap%
\pgfsetroundjoin%
\definecolor{currentfill}{rgb}{0.000000,0.000000,0.000000}%
\pgfsetfillcolor{currentfill}%
\pgfsetlinewidth{0.501875pt}%
\definecolor{currentstroke}{rgb}{0.000000,0.000000,0.000000}%
\pgfsetstrokecolor{currentstroke}%
\pgfsetdash{}{0pt}%
\pgfsys@defobject{currentmarker}{\pgfqpoint{0.000000in}{0.000000in}}{\pgfqpoint{0.000000in}{0.020833in}}{%
\pgfpathmoveto{\pgfqpoint{0.000000in}{0.000000in}}%
\pgfpathlineto{\pgfqpoint{0.000000in}{0.020833in}}%
\pgfusepath{stroke,fill}%
}%
\begin{pgfscope}%
\pgfsys@transformshift{4.515466in}{1.080890in}%
\pgfsys@useobject{currentmarker}{}%
\end{pgfscope}%
\end{pgfscope}%
\begin{pgfscope}%
\pgfsetbuttcap%
\pgfsetroundjoin%
\definecolor{currentfill}{rgb}{0.000000,0.000000,0.000000}%
\pgfsetfillcolor{currentfill}%
\pgfsetlinewidth{0.501875pt}%
\definecolor{currentstroke}{rgb}{0.000000,0.000000,0.000000}%
\pgfsetstrokecolor{currentstroke}%
\pgfsetdash{}{0pt}%
\pgfsys@defobject{currentmarker}{\pgfqpoint{0.000000in}{-0.020833in}}{\pgfqpoint{0.000000in}{0.000000in}}{%
\pgfpathmoveto{\pgfqpoint{0.000000in}{0.000000in}}%
\pgfpathlineto{\pgfqpoint{0.000000in}{-0.020833in}}%
\pgfusepath{stroke,fill}%
}%
\begin{pgfscope}%
\pgfsys@transformshift{4.515466in}{3.227753in}%
\pgfsys@useobject{currentmarker}{}%
\end{pgfscope}%
\end{pgfscope}%
\begin{pgfscope}%
\pgfpathrectangle{\pgfqpoint{0.481681in}{1.080890in}}{\pgfqpoint{5.785672in}{2.146863in}}%
\pgfusepath{clip}%
\pgfsetrectcap%
\pgfsetroundjoin%
\pgfsetlinewidth{0.100375pt}%
\definecolor{currentstroke}{rgb}{0.827451,0.827451,0.827451}%
\pgfsetstrokecolor{currentstroke}%
\pgfsetdash{}{0pt}%
\pgfpathmoveto{\pgfqpoint{4.550986in}{1.080890in}}%
\pgfpathlineto{\pgfqpoint{4.550986in}{3.227753in}}%
\pgfusepath{stroke}%
\end{pgfscope}%
\begin{pgfscope}%
\pgfsetbuttcap%
\pgfsetroundjoin%
\definecolor{currentfill}{rgb}{0.000000,0.000000,0.000000}%
\pgfsetfillcolor{currentfill}%
\pgfsetlinewidth{0.501875pt}%
\definecolor{currentstroke}{rgb}{0.000000,0.000000,0.000000}%
\pgfsetstrokecolor{currentstroke}%
\pgfsetdash{}{0pt}%
\pgfsys@defobject{currentmarker}{\pgfqpoint{0.000000in}{0.000000in}}{\pgfqpoint{0.000000in}{0.020833in}}{%
\pgfpathmoveto{\pgfqpoint{0.000000in}{0.000000in}}%
\pgfpathlineto{\pgfqpoint{0.000000in}{0.020833in}}%
\pgfusepath{stroke,fill}%
}%
\begin{pgfscope}%
\pgfsys@transformshift{4.550986in}{1.080890in}%
\pgfsys@useobject{currentmarker}{}%
\end{pgfscope}%
\end{pgfscope}%
\begin{pgfscope}%
\pgfsetbuttcap%
\pgfsetroundjoin%
\definecolor{currentfill}{rgb}{0.000000,0.000000,0.000000}%
\pgfsetfillcolor{currentfill}%
\pgfsetlinewidth{0.501875pt}%
\definecolor{currentstroke}{rgb}{0.000000,0.000000,0.000000}%
\pgfsetstrokecolor{currentstroke}%
\pgfsetdash{}{0pt}%
\pgfsys@defobject{currentmarker}{\pgfqpoint{0.000000in}{-0.020833in}}{\pgfqpoint{0.000000in}{0.000000in}}{%
\pgfpathmoveto{\pgfqpoint{0.000000in}{0.000000in}}%
\pgfpathlineto{\pgfqpoint{0.000000in}{-0.020833in}}%
\pgfusepath{stroke,fill}%
}%
\begin{pgfscope}%
\pgfsys@transformshift{4.550986in}{3.227753in}%
\pgfsys@useobject{currentmarker}{}%
\end{pgfscope}%
\end{pgfscope}%
\begin{pgfscope}%
\pgfpathrectangle{\pgfqpoint{0.481681in}{1.080890in}}{\pgfqpoint{5.785672in}{2.146863in}}%
\pgfusepath{clip}%
\pgfsetrectcap%
\pgfsetroundjoin%
\pgfsetlinewidth{0.100375pt}%
\definecolor{currentstroke}{rgb}{0.827451,0.827451,0.827451}%
\pgfsetstrokecolor{currentstroke}%
\pgfsetdash{}{0pt}%
\pgfpathmoveto{\pgfqpoint{4.586507in}{1.080890in}}%
\pgfpathlineto{\pgfqpoint{4.586507in}{3.227753in}}%
\pgfusepath{stroke}%
\end{pgfscope}%
\begin{pgfscope}%
\pgfsetbuttcap%
\pgfsetroundjoin%
\definecolor{currentfill}{rgb}{0.000000,0.000000,0.000000}%
\pgfsetfillcolor{currentfill}%
\pgfsetlinewidth{0.501875pt}%
\definecolor{currentstroke}{rgb}{0.000000,0.000000,0.000000}%
\pgfsetstrokecolor{currentstroke}%
\pgfsetdash{}{0pt}%
\pgfsys@defobject{currentmarker}{\pgfqpoint{0.000000in}{0.000000in}}{\pgfqpoint{0.000000in}{0.020833in}}{%
\pgfpathmoveto{\pgfqpoint{0.000000in}{0.000000in}}%
\pgfpathlineto{\pgfqpoint{0.000000in}{0.020833in}}%
\pgfusepath{stroke,fill}%
}%
\begin{pgfscope}%
\pgfsys@transformshift{4.586507in}{1.080890in}%
\pgfsys@useobject{currentmarker}{}%
\end{pgfscope}%
\end{pgfscope}%
\begin{pgfscope}%
\pgfsetbuttcap%
\pgfsetroundjoin%
\definecolor{currentfill}{rgb}{0.000000,0.000000,0.000000}%
\pgfsetfillcolor{currentfill}%
\pgfsetlinewidth{0.501875pt}%
\definecolor{currentstroke}{rgb}{0.000000,0.000000,0.000000}%
\pgfsetstrokecolor{currentstroke}%
\pgfsetdash{}{0pt}%
\pgfsys@defobject{currentmarker}{\pgfqpoint{0.000000in}{-0.020833in}}{\pgfqpoint{0.000000in}{0.000000in}}{%
\pgfpathmoveto{\pgfqpoint{0.000000in}{0.000000in}}%
\pgfpathlineto{\pgfqpoint{0.000000in}{-0.020833in}}%
\pgfusepath{stroke,fill}%
}%
\begin{pgfscope}%
\pgfsys@transformshift{4.586507in}{3.227753in}%
\pgfsys@useobject{currentmarker}{}%
\end{pgfscope}%
\end{pgfscope}%
\begin{pgfscope}%
\pgfpathrectangle{\pgfqpoint{0.481681in}{1.080890in}}{\pgfqpoint{5.785672in}{2.146863in}}%
\pgfusepath{clip}%
\pgfsetrectcap%
\pgfsetroundjoin%
\pgfsetlinewidth{0.100375pt}%
\definecolor{currentstroke}{rgb}{0.827451,0.827451,0.827451}%
\pgfsetstrokecolor{currentstroke}%
\pgfsetdash{}{0pt}%
\pgfpathmoveto{\pgfqpoint{4.622027in}{1.080890in}}%
\pgfpathlineto{\pgfqpoint{4.622027in}{3.227753in}}%
\pgfusepath{stroke}%
\end{pgfscope}%
\begin{pgfscope}%
\pgfsetbuttcap%
\pgfsetroundjoin%
\definecolor{currentfill}{rgb}{0.000000,0.000000,0.000000}%
\pgfsetfillcolor{currentfill}%
\pgfsetlinewidth{0.501875pt}%
\definecolor{currentstroke}{rgb}{0.000000,0.000000,0.000000}%
\pgfsetstrokecolor{currentstroke}%
\pgfsetdash{}{0pt}%
\pgfsys@defobject{currentmarker}{\pgfqpoint{0.000000in}{0.000000in}}{\pgfqpoint{0.000000in}{0.020833in}}{%
\pgfpathmoveto{\pgfqpoint{0.000000in}{0.000000in}}%
\pgfpathlineto{\pgfqpoint{0.000000in}{0.020833in}}%
\pgfusepath{stroke,fill}%
}%
\begin{pgfscope}%
\pgfsys@transformshift{4.622027in}{1.080890in}%
\pgfsys@useobject{currentmarker}{}%
\end{pgfscope}%
\end{pgfscope}%
\begin{pgfscope}%
\pgfsetbuttcap%
\pgfsetroundjoin%
\definecolor{currentfill}{rgb}{0.000000,0.000000,0.000000}%
\pgfsetfillcolor{currentfill}%
\pgfsetlinewidth{0.501875pt}%
\definecolor{currentstroke}{rgb}{0.000000,0.000000,0.000000}%
\pgfsetstrokecolor{currentstroke}%
\pgfsetdash{}{0pt}%
\pgfsys@defobject{currentmarker}{\pgfqpoint{0.000000in}{-0.020833in}}{\pgfqpoint{0.000000in}{0.000000in}}{%
\pgfpathmoveto{\pgfqpoint{0.000000in}{0.000000in}}%
\pgfpathlineto{\pgfqpoint{0.000000in}{-0.020833in}}%
\pgfusepath{stroke,fill}%
}%
\begin{pgfscope}%
\pgfsys@transformshift{4.622027in}{3.227753in}%
\pgfsys@useobject{currentmarker}{}%
\end{pgfscope}%
\end{pgfscope}%
\begin{pgfscope}%
\pgfpathrectangle{\pgfqpoint{0.481681in}{1.080890in}}{\pgfqpoint{5.785672in}{2.146863in}}%
\pgfusepath{clip}%
\pgfsetrectcap%
\pgfsetroundjoin%
\pgfsetlinewidth{0.100375pt}%
\definecolor{currentstroke}{rgb}{0.827451,0.827451,0.827451}%
\pgfsetstrokecolor{currentstroke}%
\pgfsetdash{}{0pt}%
\pgfpathmoveto{\pgfqpoint{4.657548in}{1.080890in}}%
\pgfpathlineto{\pgfqpoint{4.657548in}{3.227753in}}%
\pgfusepath{stroke}%
\end{pgfscope}%
\begin{pgfscope}%
\pgfsetbuttcap%
\pgfsetroundjoin%
\definecolor{currentfill}{rgb}{0.000000,0.000000,0.000000}%
\pgfsetfillcolor{currentfill}%
\pgfsetlinewidth{0.501875pt}%
\definecolor{currentstroke}{rgb}{0.000000,0.000000,0.000000}%
\pgfsetstrokecolor{currentstroke}%
\pgfsetdash{}{0pt}%
\pgfsys@defobject{currentmarker}{\pgfqpoint{0.000000in}{0.000000in}}{\pgfqpoint{0.000000in}{0.020833in}}{%
\pgfpathmoveto{\pgfqpoint{0.000000in}{0.000000in}}%
\pgfpathlineto{\pgfqpoint{0.000000in}{0.020833in}}%
\pgfusepath{stroke,fill}%
}%
\begin{pgfscope}%
\pgfsys@transformshift{4.657548in}{1.080890in}%
\pgfsys@useobject{currentmarker}{}%
\end{pgfscope}%
\end{pgfscope}%
\begin{pgfscope}%
\pgfsetbuttcap%
\pgfsetroundjoin%
\definecolor{currentfill}{rgb}{0.000000,0.000000,0.000000}%
\pgfsetfillcolor{currentfill}%
\pgfsetlinewidth{0.501875pt}%
\definecolor{currentstroke}{rgb}{0.000000,0.000000,0.000000}%
\pgfsetstrokecolor{currentstroke}%
\pgfsetdash{}{0pt}%
\pgfsys@defobject{currentmarker}{\pgfqpoint{0.000000in}{-0.020833in}}{\pgfqpoint{0.000000in}{0.000000in}}{%
\pgfpathmoveto{\pgfqpoint{0.000000in}{0.000000in}}%
\pgfpathlineto{\pgfqpoint{0.000000in}{-0.020833in}}%
\pgfusepath{stroke,fill}%
}%
\begin{pgfscope}%
\pgfsys@transformshift{4.657548in}{3.227753in}%
\pgfsys@useobject{currentmarker}{}%
\end{pgfscope}%
\end{pgfscope}%
\begin{pgfscope}%
\pgfpathrectangle{\pgfqpoint{0.481681in}{1.080890in}}{\pgfqpoint{5.785672in}{2.146863in}}%
\pgfusepath{clip}%
\pgfsetrectcap%
\pgfsetroundjoin%
\pgfsetlinewidth{0.100375pt}%
\definecolor{currentstroke}{rgb}{0.827451,0.827451,0.827451}%
\pgfsetstrokecolor{currentstroke}%
\pgfsetdash{}{0pt}%
\pgfpathmoveto{\pgfqpoint{4.693069in}{1.080890in}}%
\pgfpathlineto{\pgfqpoint{4.693069in}{3.227753in}}%
\pgfusepath{stroke}%
\end{pgfscope}%
\begin{pgfscope}%
\pgfsetbuttcap%
\pgfsetroundjoin%
\definecolor{currentfill}{rgb}{0.000000,0.000000,0.000000}%
\pgfsetfillcolor{currentfill}%
\pgfsetlinewidth{0.501875pt}%
\definecolor{currentstroke}{rgb}{0.000000,0.000000,0.000000}%
\pgfsetstrokecolor{currentstroke}%
\pgfsetdash{}{0pt}%
\pgfsys@defobject{currentmarker}{\pgfqpoint{0.000000in}{0.000000in}}{\pgfqpoint{0.000000in}{0.020833in}}{%
\pgfpathmoveto{\pgfqpoint{0.000000in}{0.000000in}}%
\pgfpathlineto{\pgfqpoint{0.000000in}{0.020833in}}%
\pgfusepath{stroke,fill}%
}%
\begin{pgfscope}%
\pgfsys@transformshift{4.693069in}{1.080890in}%
\pgfsys@useobject{currentmarker}{}%
\end{pgfscope}%
\end{pgfscope}%
\begin{pgfscope}%
\pgfsetbuttcap%
\pgfsetroundjoin%
\definecolor{currentfill}{rgb}{0.000000,0.000000,0.000000}%
\pgfsetfillcolor{currentfill}%
\pgfsetlinewidth{0.501875pt}%
\definecolor{currentstroke}{rgb}{0.000000,0.000000,0.000000}%
\pgfsetstrokecolor{currentstroke}%
\pgfsetdash{}{0pt}%
\pgfsys@defobject{currentmarker}{\pgfqpoint{0.000000in}{-0.020833in}}{\pgfqpoint{0.000000in}{0.000000in}}{%
\pgfpathmoveto{\pgfqpoint{0.000000in}{0.000000in}}%
\pgfpathlineto{\pgfqpoint{0.000000in}{-0.020833in}}%
\pgfusepath{stroke,fill}%
}%
\begin{pgfscope}%
\pgfsys@transformshift{4.693069in}{3.227753in}%
\pgfsys@useobject{currentmarker}{}%
\end{pgfscope}%
\end{pgfscope}%
\begin{pgfscope}%
\pgfpathrectangle{\pgfqpoint{0.481681in}{1.080890in}}{\pgfqpoint{5.785672in}{2.146863in}}%
\pgfusepath{clip}%
\pgfsetrectcap%
\pgfsetroundjoin%
\pgfsetlinewidth{0.100375pt}%
\definecolor{currentstroke}{rgb}{0.827451,0.827451,0.827451}%
\pgfsetstrokecolor{currentstroke}%
\pgfsetdash{}{0pt}%
\pgfpathmoveto{\pgfqpoint{4.728589in}{1.080890in}}%
\pgfpathlineto{\pgfqpoint{4.728589in}{3.227753in}}%
\pgfusepath{stroke}%
\end{pgfscope}%
\begin{pgfscope}%
\pgfsetbuttcap%
\pgfsetroundjoin%
\definecolor{currentfill}{rgb}{0.000000,0.000000,0.000000}%
\pgfsetfillcolor{currentfill}%
\pgfsetlinewidth{0.501875pt}%
\definecolor{currentstroke}{rgb}{0.000000,0.000000,0.000000}%
\pgfsetstrokecolor{currentstroke}%
\pgfsetdash{}{0pt}%
\pgfsys@defobject{currentmarker}{\pgfqpoint{0.000000in}{0.000000in}}{\pgfqpoint{0.000000in}{0.020833in}}{%
\pgfpathmoveto{\pgfqpoint{0.000000in}{0.000000in}}%
\pgfpathlineto{\pgfqpoint{0.000000in}{0.020833in}}%
\pgfusepath{stroke,fill}%
}%
\begin{pgfscope}%
\pgfsys@transformshift{4.728589in}{1.080890in}%
\pgfsys@useobject{currentmarker}{}%
\end{pgfscope}%
\end{pgfscope}%
\begin{pgfscope}%
\pgfsetbuttcap%
\pgfsetroundjoin%
\definecolor{currentfill}{rgb}{0.000000,0.000000,0.000000}%
\pgfsetfillcolor{currentfill}%
\pgfsetlinewidth{0.501875pt}%
\definecolor{currentstroke}{rgb}{0.000000,0.000000,0.000000}%
\pgfsetstrokecolor{currentstroke}%
\pgfsetdash{}{0pt}%
\pgfsys@defobject{currentmarker}{\pgfqpoint{0.000000in}{-0.020833in}}{\pgfqpoint{0.000000in}{0.000000in}}{%
\pgfpathmoveto{\pgfqpoint{0.000000in}{0.000000in}}%
\pgfpathlineto{\pgfqpoint{0.000000in}{-0.020833in}}%
\pgfusepath{stroke,fill}%
}%
\begin{pgfscope}%
\pgfsys@transformshift{4.728589in}{3.227753in}%
\pgfsys@useobject{currentmarker}{}%
\end{pgfscope}%
\end{pgfscope}%
\begin{pgfscope}%
\pgfpathrectangle{\pgfqpoint{0.481681in}{1.080890in}}{\pgfqpoint{5.785672in}{2.146863in}}%
\pgfusepath{clip}%
\pgfsetrectcap%
\pgfsetroundjoin%
\pgfsetlinewidth{0.100375pt}%
\definecolor{currentstroke}{rgb}{0.827451,0.827451,0.827451}%
\pgfsetstrokecolor{currentstroke}%
\pgfsetdash{}{0pt}%
\pgfpathmoveto{\pgfqpoint{4.764110in}{1.080890in}}%
\pgfpathlineto{\pgfqpoint{4.764110in}{3.227753in}}%
\pgfusepath{stroke}%
\end{pgfscope}%
\begin{pgfscope}%
\pgfsetbuttcap%
\pgfsetroundjoin%
\definecolor{currentfill}{rgb}{0.000000,0.000000,0.000000}%
\pgfsetfillcolor{currentfill}%
\pgfsetlinewidth{0.501875pt}%
\definecolor{currentstroke}{rgb}{0.000000,0.000000,0.000000}%
\pgfsetstrokecolor{currentstroke}%
\pgfsetdash{}{0pt}%
\pgfsys@defobject{currentmarker}{\pgfqpoint{0.000000in}{0.000000in}}{\pgfqpoint{0.000000in}{0.020833in}}{%
\pgfpathmoveto{\pgfqpoint{0.000000in}{0.000000in}}%
\pgfpathlineto{\pgfqpoint{0.000000in}{0.020833in}}%
\pgfusepath{stroke,fill}%
}%
\begin{pgfscope}%
\pgfsys@transformshift{4.764110in}{1.080890in}%
\pgfsys@useobject{currentmarker}{}%
\end{pgfscope}%
\end{pgfscope}%
\begin{pgfscope}%
\pgfsetbuttcap%
\pgfsetroundjoin%
\definecolor{currentfill}{rgb}{0.000000,0.000000,0.000000}%
\pgfsetfillcolor{currentfill}%
\pgfsetlinewidth{0.501875pt}%
\definecolor{currentstroke}{rgb}{0.000000,0.000000,0.000000}%
\pgfsetstrokecolor{currentstroke}%
\pgfsetdash{}{0pt}%
\pgfsys@defobject{currentmarker}{\pgfqpoint{0.000000in}{-0.020833in}}{\pgfqpoint{0.000000in}{0.000000in}}{%
\pgfpathmoveto{\pgfqpoint{0.000000in}{0.000000in}}%
\pgfpathlineto{\pgfqpoint{0.000000in}{-0.020833in}}%
\pgfusepath{stroke,fill}%
}%
\begin{pgfscope}%
\pgfsys@transformshift{4.764110in}{3.227753in}%
\pgfsys@useobject{currentmarker}{}%
\end{pgfscope}%
\end{pgfscope}%
\begin{pgfscope}%
\pgfpathrectangle{\pgfqpoint{0.481681in}{1.080890in}}{\pgfqpoint{5.785672in}{2.146863in}}%
\pgfusepath{clip}%
\pgfsetrectcap%
\pgfsetroundjoin%
\pgfsetlinewidth{0.100375pt}%
\definecolor{currentstroke}{rgb}{0.827451,0.827451,0.827451}%
\pgfsetstrokecolor{currentstroke}%
\pgfsetdash{}{0pt}%
\pgfpathmoveto{\pgfqpoint{4.799630in}{1.080890in}}%
\pgfpathlineto{\pgfqpoint{4.799630in}{3.227753in}}%
\pgfusepath{stroke}%
\end{pgfscope}%
\begin{pgfscope}%
\pgfsetbuttcap%
\pgfsetroundjoin%
\definecolor{currentfill}{rgb}{0.000000,0.000000,0.000000}%
\pgfsetfillcolor{currentfill}%
\pgfsetlinewidth{0.501875pt}%
\definecolor{currentstroke}{rgb}{0.000000,0.000000,0.000000}%
\pgfsetstrokecolor{currentstroke}%
\pgfsetdash{}{0pt}%
\pgfsys@defobject{currentmarker}{\pgfqpoint{0.000000in}{0.000000in}}{\pgfqpoint{0.000000in}{0.020833in}}{%
\pgfpathmoveto{\pgfqpoint{0.000000in}{0.000000in}}%
\pgfpathlineto{\pgfqpoint{0.000000in}{0.020833in}}%
\pgfusepath{stroke,fill}%
}%
\begin{pgfscope}%
\pgfsys@transformshift{4.799630in}{1.080890in}%
\pgfsys@useobject{currentmarker}{}%
\end{pgfscope}%
\end{pgfscope}%
\begin{pgfscope}%
\pgfsetbuttcap%
\pgfsetroundjoin%
\definecolor{currentfill}{rgb}{0.000000,0.000000,0.000000}%
\pgfsetfillcolor{currentfill}%
\pgfsetlinewidth{0.501875pt}%
\definecolor{currentstroke}{rgb}{0.000000,0.000000,0.000000}%
\pgfsetstrokecolor{currentstroke}%
\pgfsetdash{}{0pt}%
\pgfsys@defobject{currentmarker}{\pgfqpoint{0.000000in}{-0.020833in}}{\pgfqpoint{0.000000in}{0.000000in}}{%
\pgfpathmoveto{\pgfqpoint{0.000000in}{0.000000in}}%
\pgfpathlineto{\pgfqpoint{0.000000in}{-0.020833in}}%
\pgfusepath{stroke,fill}%
}%
\begin{pgfscope}%
\pgfsys@transformshift{4.799630in}{3.227753in}%
\pgfsys@useobject{currentmarker}{}%
\end{pgfscope}%
\end{pgfscope}%
\begin{pgfscope}%
\pgfpathrectangle{\pgfqpoint{0.481681in}{1.080890in}}{\pgfqpoint{5.785672in}{2.146863in}}%
\pgfusepath{clip}%
\pgfsetrectcap%
\pgfsetroundjoin%
\pgfsetlinewidth{0.100375pt}%
\definecolor{currentstroke}{rgb}{0.827451,0.827451,0.827451}%
\pgfsetstrokecolor{currentstroke}%
\pgfsetdash{}{0pt}%
\pgfpathmoveto{\pgfqpoint{4.835151in}{1.080890in}}%
\pgfpathlineto{\pgfqpoint{4.835151in}{3.227753in}}%
\pgfusepath{stroke}%
\end{pgfscope}%
\begin{pgfscope}%
\pgfsetbuttcap%
\pgfsetroundjoin%
\definecolor{currentfill}{rgb}{0.000000,0.000000,0.000000}%
\pgfsetfillcolor{currentfill}%
\pgfsetlinewidth{0.501875pt}%
\definecolor{currentstroke}{rgb}{0.000000,0.000000,0.000000}%
\pgfsetstrokecolor{currentstroke}%
\pgfsetdash{}{0pt}%
\pgfsys@defobject{currentmarker}{\pgfqpoint{0.000000in}{0.000000in}}{\pgfqpoint{0.000000in}{0.020833in}}{%
\pgfpathmoveto{\pgfqpoint{0.000000in}{0.000000in}}%
\pgfpathlineto{\pgfqpoint{0.000000in}{0.020833in}}%
\pgfusepath{stroke,fill}%
}%
\begin{pgfscope}%
\pgfsys@transformshift{4.835151in}{1.080890in}%
\pgfsys@useobject{currentmarker}{}%
\end{pgfscope}%
\end{pgfscope}%
\begin{pgfscope}%
\pgfsetbuttcap%
\pgfsetroundjoin%
\definecolor{currentfill}{rgb}{0.000000,0.000000,0.000000}%
\pgfsetfillcolor{currentfill}%
\pgfsetlinewidth{0.501875pt}%
\definecolor{currentstroke}{rgb}{0.000000,0.000000,0.000000}%
\pgfsetstrokecolor{currentstroke}%
\pgfsetdash{}{0pt}%
\pgfsys@defobject{currentmarker}{\pgfqpoint{0.000000in}{-0.020833in}}{\pgfqpoint{0.000000in}{0.000000in}}{%
\pgfpathmoveto{\pgfqpoint{0.000000in}{0.000000in}}%
\pgfpathlineto{\pgfqpoint{0.000000in}{-0.020833in}}%
\pgfusepath{stroke,fill}%
}%
\begin{pgfscope}%
\pgfsys@transformshift{4.835151in}{3.227753in}%
\pgfsys@useobject{currentmarker}{}%
\end{pgfscope}%
\end{pgfscope}%
\begin{pgfscope}%
\pgfpathrectangle{\pgfqpoint{0.481681in}{1.080890in}}{\pgfqpoint{5.785672in}{2.146863in}}%
\pgfusepath{clip}%
\pgfsetrectcap%
\pgfsetroundjoin%
\pgfsetlinewidth{0.100375pt}%
\definecolor{currentstroke}{rgb}{0.827451,0.827451,0.827451}%
\pgfsetstrokecolor{currentstroke}%
\pgfsetdash{}{0pt}%
\pgfpathmoveto{\pgfqpoint{4.870671in}{1.080890in}}%
\pgfpathlineto{\pgfqpoint{4.870671in}{3.227753in}}%
\pgfusepath{stroke}%
\end{pgfscope}%
\begin{pgfscope}%
\pgfsetbuttcap%
\pgfsetroundjoin%
\definecolor{currentfill}{rgb}{0.000000,0.000000,0.000000}%
\pgfsetfillcolor{currentfill}%
\pgfsetlinewidth{0.501875pt}%
\definecolor{currentstroke}{rgb}{0.000000,0.000000,0.000000}%
\pgfsetstrokecolor{currentstroke}%
\pgfsetdash{}{0pt}%
\pgfsys@defobject{currentmarker}{\pgfqpoint{0.000000in}{0.000000in}}{\pgfqpoint{0.000000in}{0.020833in}}{%
\pgfpathmoveto{\pgfqpoint{0.000000in}{0.000000in}}%
\pgfpathlineto{\pgfqpoint{0.000000in}{0.020833in}}%
\pgfusepath{stroke,fill}%
}%
\begin{pgfscope}%
\pgfsys@transformshift{4.870671in}{1.080890in}%
\pgfsys@useobject{currentmarker}{}%
\end{pgfscope}%
\end{pgfscope}%
\begin{pgfscope}%
\pgfsetbuttcap%
\pgfsetroundjoin%
\definecolor{currentfill}{rgb}{0.000000,0.000000,0.000000}%
\pgfsetfillcolor{currentfill}%
\pgfsetlinewidth{0.501875pt}%
\definecolor{currentstroke}{rgb}{0.000000,0.000000,0.000000}%
\pgfsetstrokecolor{currentstroke}%
\pgfsetdash{}{0pt}%
\pgfsys@defobject{currentmarker}{\pgfqpoint{0.000000in}{-0.020833in}}{\pgfqpoint{0.000000in}{0.000000in}}{%
\pgfpathmoveto{\pgfqpoint{0.000000in}{0.000000in}}%
\pgfpathlineto{\pgfqpoint{0.000000in}{-0.020833in}}%
\pgfusepath{stroke,fill}%
}%
\begin{pgfscope}%
\pgfsys@transformshift{4.870671in}{3.227753in}%
\pgfsys@useobject{currentmarker}{}%
\end{pgfscope}%
\end{pgfscope}%
\begin{pgfscope}%
\pgfpathrectangle{\pgfqpoint{0.481681in}{1.080890in}}{\pgfqpoint{5.785672in}{2.146863in}}%
\pgfusepath{clip}%
\pgfsetrectcap%
\pgfsetroundjoin%
\pgfsetlinewidth{0.100375pt}%
\definecolor{currentstroke}{rgb}{0.827451,0.827451,0.827451}%
\pgfsetstrokecolor{currentstroke}%
\pgfsetdash{}{0pt}%
\pgfpathmoveto{\pgfqpoint{4.941712in}{1.080890in}}%
\pgfpathlineto{\pgfqpoint{4.941712in}{3.227753in}}%
\pgfusepath{stroke}%
\end{pgfscope}%
\begin{pgfscope}%
\pgfsetbuttcap%
\pgfsetroundjoin%
\definecolor{currentfill}{rgb}{0.000000,0.000000,0.000000}%
\pgfsetfillcolor{currentfill}%
\pgfsetlinewidth{0.501875pt}%
\definecolor{currentstroke}{rgb}{0.000000,0.000000,0.000000}%
\pgfsetstrokecolor{currentstroke}%
\pgfsetdash{}{0pt}%
\pgfsys@defobject{currentmarker}{\pgfqpoint{0.000000in}{0.000000in}}{\pgfqpoint{0.000000in}{0.020833in}}{%
\pgfpathmoveto{\pgfqpoint{0.000000in}{0.000000in}}%
\pgfpathlineto{\pgfqpoint{0.000000in}{0.020833in}}%
\pgfusepath{stroke,fill}%
}%
\begin{pgfscope}%
\pgfsys@transformshift{4.941712in}{1.080890in}%
\pgfsys@useobject{currentmarker}{}%
\end{pgfscope}%
\end{pgfscope}%
\begin{pgfscope}%
\pgfsetbuttcap%
\pgfsetroundjoin%
\definecolor{currentfill}{rgb}{0.000000,0.000000,0.000000}%
\pgfsetfillcolor{currentfill}%
\pgfsetlinewidth{0.501875pt}%
\definecolor{currentstroke}{rgb}{0.000000,0.000000,0.000000}%
\pgfsetstrokecolor{currentstroke}%
\pgfsetdash{}{0pt}%
\pgfsys@defobject{currentmarker}{\pgfqpoint{0.000000in}{-0.020833in}}{\pgfqpoint{0.000000in}{0.000000in}}{%
\pgfpathmoveto{\pgfqpoint{0.000000in}{0.000000in}}%
\pgfpathlineto{\pgfqpoint{0.000000in}{-0.020833in}}%
\pgfusepath{stroke,fill}%
}%
\begin{pgfscope}%
\pgfsys@transformshift{4.941712in}{3.227753in}%
\pgfsys@useobject{currentmarker}{}%
\end{pgfscope}%
\end{pgfscope}%
\begin{pgfscope}%
\pgfpathrectangle{\pgfqpoint{0.481681in}{1.080890in}}{\pgfqpoint{5.785672in}{2.146863in}}%
\pgfusepath{clip}%
\pgfsetrectcap%
\pgfsetroundjoin%
\pgfsetlinewidth{0.100375pt}%
\definecolor{currentstroke}{rgb}{0.827451,0.827451,0.827451}%
\pgfsetstrokecolor{currentstroke}%
\pgfsetdash{}{0pt}%
\pgfpathmoveto{\pgfqpoint{4.977233in}{1.080890in}}%
\pgfpathlineto{\pgfqpoint{4.977233in}{3.227753in}}%
\pgfusepath{stroke}%
\end{pgfscope}%
\begin{pgfscope}%
\pgfsetbuttcap%
\pgfsetroundjoin%
\definecolor{currentfill}{rgb}{0.000000,0.000000,0.000000}%
\pgfsetfillcolor{currentfill}%
\pgfsetlinewidth{0.501875pt}%
\definecolor{currentstroke}{rgb}{0.000000,0.000000,0.000000}%
\pgfsetstrokecolor{currentstroke}%
\pgfsetdash{}{0pt}%
\pgfsys@defobject{currentmarker}{\pgfqpoint{0.000000in}{0.000000in}}{\pgfqpoint{0.000000in}{0.020833in}}{%
\pgfpathmoveto{\pgfqpoint{0.000000in}{0.000000in}}%
\pgfpathlineto{\pgfqpoint{0.000000in}{0.020833in}}%
\pgfusepath{stroke,fill}%
}%
\begin{pgfscope}%
\pgfsys@transformshift{4.977233in}{1.080890in}%
\pgfsys@useobject{currentmarker}{}%
\end{pgfscope}%
\end{pgfscope}%
\begin{pgfscope}%
\pgfsetbuttcap%
\pgfsetroundjoin%
\definecolor{currentfill}{rgb}{0.000000,0.000000,0.000000}%
\pgfsetfillcolor{currentfill}%
\pgfsetlinewidth{0.501875pt}%
\definecolor{currentstroke}{rgb}{0.000000,0.000000,0.000000}%
\pgfsetstrokecolor{currentstroke}%
\pgfsetdash{}{0pt}%
\pgfsys@defobject{currentmarker}{\pgfqpoint{0.000000in}{-0.020833in}}{\pgfqpoint{0.000000in}{0.000000in}}{%
\pgfpathmoveto{\pgfqpoint{0.000000in}{0.000000in}}%
\pgfpathlineto{\pgfqpoint{0.000000in}{-0.020833in}}%
\pgfusepath{stroke,fill}%
}%
\begin{pgfscope}%
\pgfsys@transformshift{4.977233in}{3.227753in}%
\pgfsys@useobject{currentmarker}{}%
\end{pgfscope}%
\end{pgfscope}%
\begin{pgfscope}%
\pgfpathrectangle{\pgfqpoint{0.481681in}{1.080890in}}{\pgfqpoint{5.785672in}{2.146863in}}%
\pgfusepath{clip}%
\pgfsetrectcap%
\pgfsetroundjoin%
\pgfsetlinewidth{0.100375pt}%
\definecolor{currentstroke}{rgb}{0.827451,0.827451,0.827451}%
\pgfsetstrokecolor{currentstroke}%
\pgfsetdash{}{0pt}%
\pgfpathmoveto{\pgfqpoint{5.012753in}{1.080890in}}%
\pgfpathlineto{\pgfqpoint{5.012753in}{3.227753in}}%
\pgfusepath{stroke}%
\end{pgfscope}%
\begin{pgfscope}%
\pgfsetbuttcap%
\pgfsetroundjoin%
\definecolor{currentfill}{rgb}{0.000000,0.000000,0.000000}%
\pgfsetfillcolor{currentfill}%
\pgfsetlinewidth{0.501875pt}%
\definecolor{currentstroke}{rgb}{0.000000,0.000000,0.000000}%
\pgfsetstrokecolor{currentstroke}%
\pgfsetdash{}{0pt}%
\pgfsys@defobject{currentmarker}{\pgfqpoint{0.000000in}{0.000000in}}{\pgfqpoint{0.000000in}{0.020833in}}{%
\pgfpathmoveto{\pgfqpoint{0.000000in}{0.000000in}}%
\pgfpathlineto{\pgfqpoint{0.000000in}{0.020833in}}%
\pgfusepath{stroke,fill}%
}%
\begin{pgfscope}%
\pgfsys@transformshift{5.012753in}{1.080890in}%
\pgfsys@useobject{currentmarker}{}%
\end{pgfscope}%
\end{pgfscope}%
\begin{pgfscope}%
\pgfsetbuttcap%
\pgfsetroundjoin%
\definecolor{currentfill}{rgb}{0.000000,0.000000,0.000000}%
\pgfsetfillcolor{currentfill}%
\pgfsetlinewidth{0.501875pt}%
\definecolor{currentstroke}{rgb}{0.000000,0.000000,0.000000}%
\pgfsetstrokecolor{currentstroke}%
\pgfsetdash{}{0pt}%
\pgfsys@defobject{currentmarker}{\pgfqpoint{0.000000in}{-0.020833in}}{\pgfqpoint{0.000000in}{0.000000in}}{%
\pgfpathmoveto{\pgfqpoint{0.000000in}{0.000000in}}%
\pgfpathlineto{\pgfqpoint{0.000000in}{-0.020833in}}%
\pgfusepath{stroke,fill}%
}%
\begin{pgfscope}%
\pgfsys@transformshift{5.012753in}{3.227753in}%
\pgfsys@useobject{currentmarker}{}%
\end{pgfscope}%
\end{pgfscope}%
\begin{pgfscope}%
\pgfpathrectangle{\pgfqpoint{0.481681in}{1.080890in}}{\pgfqpoint{5.785672in}{2.146863in}}%
\pgfusepath{clip}%
\pgfsetrectcap%
\pgfsetroundjoin%
\pgfsetlinewidth{0.100375pt}%
\definecolor{currentstroke}{rgb}{0.827451,0.827451,0.827451}%
\pgfsetstrokecolor{currentstroke}%
\pgfsetdash{}{0pt}%
\pgfpathmoveto{\pgfqpoint{5.048274in}{1.080890in}}%
\pgfpathlineto{\pgfqpoint{5.048274in}{3.227753in}}%
\pgfusepath{stroke}%
\end{pgfscope}%
\begin{pgfscope}%
\pgfsetbuttcap%
\pgfsetroundjoin%
\definecolor{currentfill}{rgb}{0.000000,0.000000,0.000000}%
\pgfsetfillcolor{currentfill}%
\pgfsetlinewidth{0.501875pt}%
\definecolor{currentstroke}{rgb}{0.000000,0.000000,0.000000}%
\pgfsetstrokecolor{currentstroke}%
\pgfsetdash{}{0pt}%
\pgfsys@defobject{currentmarker}{\pgfqpoint{0.000000in}{0.000000in}}{\pgfqpoint{0.000000in}{0.020833in}}{%
\pgfpathmoveto{\pgfqpoint{0.000000in}{0.000000in}}%
\pgfpathlineto{\pgfqpoint{0.000000in}{0.020833in}}%
\pgfusepath{stroke,fill}%
}%
\begin{pgfscope}%
\pgfsys@transformshift{5.048274in}{1.080890in}%
\pgfsys@useobject{currentmarker}{}%
\end{pgfscope}%
\end{pgfscope}%
\begin{pgfscope}%
\pgfsetbuttcap%
\pgfsetroundjoin%
\definecolor{currentfill}{rgb}{0.000000,0.000000,0.000000}%
\pgfsetfillcolor{currentfill}%
\pgfsetlinewidth{0.501875pt}%
\definecolor{currentstroke}{rgb}{0.000000,0.000000,0.000000}%
\pgfsetstrokecolor{currentstroke}%
\pgfsetdash{}{0pt}%
\pgfsys@defobject{currentmarker}{\pgfqpoint{0.000000in}{-0.020833in}}{\pgfqpoint{0.000000in}{0.000000in}}{%
\pgfpathmoveto{\pgfqpoint{0.000000in}{0.000000in}}%
\pgfpathlineto{\pgfqpoint{0.000000in}{-0.020833in}}%
\pgfusepath{stroke,fill}%
}%
\begin{pgfscope}%
\pgfsys@transformshift{5.048274in}{3.227753in}%
\pgfsys@useobject{currentmarker}{}%
\end{pgfscope}%
\end{pgfscope}%
\begin{pgfscope}%
\pgfpathrectangle{\pgfqpoint{0.481681in}{1.080890in}}{\pgfqpoint{5.785672in}{2.146863in}}%
\pgfusepath{clip}%
\pgfsetrectcap%
\pgfsetroundjoin%
\pgfsetlinewidth{0.100375pt}%
\definecolor{currentstroke}{rgb}{0.827451,0.827451,0.827451}%
\pgfsetstrokecolor{currentstroke}%
\pgfsetdash{}{0pt}%
\pgfpathmoveto{\pgfqpoint{5.083794in}{1.080890in}}%
\pgfpathlineto{\pgfqpoint{5.083794in}{3.227753in}}%
\pgfusepath{stroke}%
\end{pgfscope}%
\begin{pgfscope}%
\pgfsetbuttcap%
\pgfsetroundjoin%
\definecolor{currentfill}{rgb}{0.000000,0.000000,0.000000}%
\pgfsetfillcolor{currentfill}%
\pgfsetlinewidth{0.501875pt}%
\definecolor{currentstroke}{rgb}{0.000000,0.000000,0.000000}%
\pgfsetstrokecolor{currentstroke}%
\pgfsetdash{}{0pt}%
\pgfsys@defobject{currentmarker}{\pgfqpoint{0.000000in}{0.000000in}}{\pgfqpoint{0.000000in}{0.020833in}}{%
\pgfpathmoveto{\pgfqpoint{0.000000in}{0.000000in}}%
\pgfpathlineto{\pgfqpoint{0.000000in}{0.020833in}}%
\pgfusepath{stroke,fill}%
}%
\begin{pgfscope}%
\pgfsys@transformshift{5.083794in}{1.080890in}%
\pgfsys@useobject{currentmarker}{}%
\end{pgfscope}%
\end{pgfscope}%
\begin{pgfscope}%
\pgfsetbuttcap%
\pgfsetroundjoin%
\definecolor{currentfill}{rgb}{0.000000,0.000000,0.000000}%
\pgfsetfillcolor{currentfill}%
\pgfsetlinewidth{0.501875pt}%
\definecolor{currentstroke}{rgb}{0.000000,0.000000,0.000000}%
\pgfsetstrokecolor{currentstroke}%
\pgfsetdash{}{0pt}%
\pgfsys@defobject{currentmarker}{\pgfqpoint{0.000000in}{-0.020833in}}{\pgfqpoint{0.000000in}{0.000000in}}{%
\pgfpathmoveto{\pgfqpoint{0.000000in}{0.000000in}}%
\pgfpathlineto{\pgfqpoint{0.000000in}{-0.020833in}}%
\pgfusepath{stroke,fill}%
}%
\begin{pgfscope}%
\pgfsys@transformshift{5.083794in}{3.227753in}%
\pgfsys@useobject{currentmarker}{}%
\end{pgfscope}%
\end{pgfscope}%
\begin{pgfscope}%
\pgfpathrectangle{\pgfqpoint{0.481681in}{1.080890in}}{\pgfqpoint{5.785672in}{2.146863in}}%
\pgfusepath{clip}%
\pgfsetrectcap%
\pgfsetroundjoin%
\pgfsetlinewidth{0.100375pt}%
\definecolor{currentstroke}{rgb}{0.827451,0.827451,0.827451}%
\pgfsetstrokecolor{currentstroke}%
\pgfsetdash{}{0pt}%
\pgfpathmoveto{\pgfqpoint{5.119315in}{1.080890in}}%
\pgfpathlineto{\pgfqpoint{5.119315in}{3.227753in}}%
\pgfusepath{stroke}%
\end{pgfscope}%
\begin{pgfscope}%
\pgfsetbuttcap%
\pgfsetroundjoin%
\definecolor{currentfill}{rgb}{0.000000,0.000000,0.000000}%
\pgfsetfillcolor{currentfill}%
\pgfsetlinewidth{0.501875pt}%
\definecolor{currentstroke}{rgb}{0.000000,0.000000,0.000000}%
\pgfsetstrokecolor{currentstroke}%
\pgfsetdash{}{0pt}%
\pgfsys@defobject{currentmarker}{\pgfqpoint{0.000000in}{0.000000in}}{\pgfqpoint{0.000000in}{0.020833in}}{%
\pgfpathmoveto{\pgfqpoint{0.000000in}{0.000000in}}%
\pgfpathlineto{\pgfqpoint{0.000000in}{0.020833in}}%
\pgfusepath{stroke,fill}%
}%
\begin{pgfscope}%
\pgfsys@transformshift{5.119315in}{1.080890in}%
\pgfsys@useobject{currentmarker}{}%
\end{pgfscope}%
\end{pgfscope}%
\begin{pgfscope}%
\pgfsetbuttcap%
\pgfsetroundjoin%
\definecolor{currentfill}{rgb}{0.000000,0.000000,0.000000}%
\pgfsetfillcolor{currentfill}%
\pgfsetlinewidth{0.501875pt}%
\definecolor{currentstroke}{rgb}{0.000000,0.000000,0.000000}%
\pgfsetstrokecolor{currentstroke}%
\pgfsetdash{}{0pt}%
\pgfsys@defobject{currentmarker}{\pgfqpoint{0.000000in}{-0.020833in}}{\pgfqpoint{0.000000in}{0.000000in}}{%
\pgfpathmoveto{\pgfqpoint{0.000000in}{0.000000in}}%
\pgfpathlineto{\pgfqpoint{0.000000in}{-0.020833in}}%
\pgfusepath{stroke,fill}%
}%
\begin{pgfscope}%
\pgfsys@transformshift{5.119315in}{3.227753in}%
\pgfsys@useobject{currentmarker}{}%
\end{pgfscope}%
\end{pgfscope}%
\begin{pgfscope}%
\pgfpathrectangle{\pgfqpoint{0.481681in}{1.080890in}}{\pgfqpoint{5.785672in}{2.146863in}}%
\pgfusepath{clip}%
\pgfsetrectcap%
\pgfsetroundjoin%
\pgfsetlinewidth{0.100375pt}%
\definecolor{currentstroke}{rgb}{0.827451,0.827451,0.827451}%
\pgfsetstrokecolor{currentstroke}%
\pgfsetdash{}{0pt}%
\pgfpathmoveto{\pgfqpoint{5.154835in}{1.080890in}}%
\pgfpathlineto{\pgfqpoint{5.154835in}{3.227753in}}%
\pgfusepath{stroke}%
\end{pgfscope}%
\begin{pgfscope}%
\pgfsetbuttcap%
\pgfsetroundjoin%
\definecolor{currentfill}{rgb}{0.000000,0.000000,0.000000}%
\pgfsetfillcolor{currentfill}%
\pgfsetlinewidth{0.501875pt}%
\definecolor{currentstroke}{rgb}{0.000000,0.000000,0.000000}%
\pgfsetstrokecolor{currentstroke}%
\pgfsetdash{}{0pt}%
\pgfsys@defobject{currentmarker}{\pgfqpoint{0.000000in}{0.000000in}}{\pgfqpoint{0.000000in}{0.020833in}}{%
\pgfpathmoveto{\pgfqpoint{0.000000in}{0.000000in}}%
\pgfpathlineto{\pgfqpoint{0.000000in}{0.020833in}}%
\pgfusepath{stroke,fill}%
}%
\begin{pgfscope}%
\pgfsys@transformshift{5.154835in}{1.080890in}%
\pgfsys@useobject{currentmarker}{}%
\end{pgfscope}%
\end{pgfscope}%
\begin{pgfscope}%
\pgfsetbuttcap%
\pgfsetroundjoin%
\definecolor{currentfill}{rgb}{0.000000,0.000000,0.000000}%
\pgfsetfillcolor{currentfill}%
\pgfsetlinewidth{0.501875pt}%
\definecolor{currentstroke}{rgb}{0.000000,0.000000,0.000000}%
\pgfsetstrokecolor{currentstroke}%
\pgfsetdash{}{0pt}%
\pgfsys@defobject{currentmarker}{\pgfqpoint{0.000000in}{-0.020833in}}{\pgfqpoint{0.000000in}{0.000000in}}{%
\pgfpathmoveto{\pgfqpoint{0.000000in}{0.000000in}}%
\pgfpathlineto{\pgfqpoint{0.000000in}{-0.020833in}}%
\pgfusepath{stroke,fill}%
}%
\begin{pgfscope}%
\pgfsys@transformshift{5.154835in}{3.227753in}%
\pgfsys@useobject{currentmarker}{}%
\end{pgfscope}%
\end{pgfscope}%
\begin{pgfscope}%
\pgfpathrectangle{\pgfqpoint{0.481681in}{1.080890in}}{\pgfqpoint{5.785672in}{2.146863in}}%
\pgfusepath{clip}%
\pgfsetrectcap%
\pgfsetroundjoin%
\pgfsetlinewidth{0.100375pt}%
\definecolor{currentstroke}{rgb}{0.827451,0.827451,0.827451}%
\pgfsetstrokecolor{currentstroke}%
\pgfsetdash{}{0pt}%
\pgfpathmoveto{\pgfqpoint{5.190356in}{1.080890in}}%
\pgfpathlineto{\pgfqpoint{5.190356in}{3.227753in}}%
\pgfusepath{stroke}%
\end{pgfscope}%
\begin{pgfscope}%
\pgfsetbuttcap%
\pgfsetroundjoin%
\definecolor{currentfill}{rgb}{0.000000,0.000000,0.000000}%
\pgfsetfillcolor{currentfill}%
\pgfsetlinewidth{0.501875pt}%
\definecolor{currentstroke}{rgb}{0.000000,0.000000,0.000000}%
\pgfsetstrokecolor{currentstroke}%
\pgfsetdash{}{0pt}%
\pgfsys@defobject{currentmarker}{\pgfqpoint{0.000000in}{0.000000in}}{\pgfqpoint{0.000000in}{0.020833in}}{%
\pgfpathmoveto{\pgfqpoint{0.000000in}{0.000000in}}%
\pgfpathlineto{\pgfqpoint{0.000000in}{0.020833in}}%
\pgfusepath{stroke,fill}%
}%
\begin{pgfscope}%
\pgfsys@transformshift{5.190356in}{1.080890in}%
\pgfsys@useobject{currentmarker}{}%
\end{pgfscope}%
\end{pgfscope}%
\begin{pgfscope}%
\pgfsetbuttcap%
\pgfsetroundjoin%
\definecolor{currentfill}{rgb}{0.000000,0.000000,0.000000}%
\pgfsetfillcolor{currentfill}%
\pgfsetlinewidth{0.501875pt}%
\definecolor{currentstroke}{rgb}{0.000000,0.000000,0.000000}%
\pgfsetstrokecolor{currentstroke}%
\pgfsetdash{}{0pt}%
\pgfsys@defobject{currentmarker}{\pgfqpoint{0.000000in}{-0.020833in}}{\pgfqpoint{0.000000in}{0.000000in}}{%
\pgfpathmoveto{\pgfqpoint{0.000000in}{0.000000in}}%
\pgfpathlineto{\pgfqpoint{0.000000in}{-0.020833in}}%
\pgfusepath{stroke,fill}%
}%
\begin{pgfscope}%
\pgfsys@transformshift{5.190356in}{3.227753in}%
\pgfsys@useobject{currentmarker}{}%
\end{pgfscope}%
\end{pgfscope}%
\begin{pgfscope}%
\pgfpathrectangle{\pgfqpoint{0.481681in}{1.080890in}}{\pgfqpoint{5.785672in}{2.146863in}}%
\pgfusepath{clip}%
\pgfsetrectcap%
\pgfsetroundjoin%
\pgfsetlinewidth{0.100375pt}%
\definecolor{currentstroke}{rgb}{0.827451,0.827451,0.827451}%
\pgfsetstrokecolor{currentstroke}%
\pgfsetdash{}{0pt}%
\pgfpathmoveto{\pgfqpoint{5.225876in}{1.080890in}}%
\pgfpathlineto{\pgfqpoint{5.225876in}{3.227753in}}%
\pgfusepath{stroke}%
\end{pgfscope}%
\begin{pgfscope}%
\pgfsetbuttcap%
\pgfsetroundjoin%
\definecolor{currentfill}{rgb}{0.000000,0.000000,0.000000}%
\pgfsetfillcolor{currentfill}%
\pgfsetlinewidth{0.501875pt}%
\definecolor{currentstroke}{rgb}{0.000000,0.000000,0.000000}%
\pgfsetstrokecolor{currentstroke}%
\pgfsetdash{}{0pt}%
\pgfsys@defobject{currentmarker}{\pgfqpoint{0.000000in}{0.000000in}}{\pgfqpoint{0.000000in}{0.020833in}}{%
\pgfpathmoveto{\pgfqpoint{0.000000in}{0.000000in}}%
\pgfpathlineto{\pgfqpoint{0.000000in}{0.020833in}}%
\pgfusepath{stroke,fill}%
}%
\begin{pgfscope}%
\pgfsys@transformshift{5.225876in}{1.080890in}%
\pgfsys@useobject{currentmarker}{}%
\end{pgfscope}%
\end{pgfscope}%
\begin{pgfscope}%
\pgfsetbuttcap%
\pgfsetroundjoin%
\definecolor{currentfill}{rgb}{0.000000,0.000000,0.000000}%
\pgfsetfillcolor{currentfill}%
\pgfsetlinewidth{0.501875pt}%
\definecolor{currentstroke}{rgb}{0.000000,0.000000,0.000000}%
\pgfsetstrokecolor{currentstroke}%
\pgfsetdash{}{0pt}%
\pgfsys@defobject{currentmarker}{\pgfqpoint{0.000000in}{-0.020833in}}{\pgfqpoint{0.000000in}{0.000000in}}{%
\pgfpathmoveto{\pgfqpoint{0.000000in}{0.000000in}}%
\pgfpathlineto{\pgfqpoint{0.000000in}{-0.020833in}}%
\pgfusepath{stroke,fill}%
}%
\begin{pgfscope}%
\pgfsys@transformshift{5.225876in}{3.227753in}%
\pgfsys@useobject{currentmarker}{}%
\end{pgfscope}%
\end{pgfscope}%
\begin{pgfscope}%
\pgfpathrectangle{\pgfqpoint{0.481681in}{1.080890in}}{\pgfqpoint{5.785672in}{2.146863in}}%
\pgfusepath{clip}%
\pgfsetrectcap%
\pgfsetroundjoin%
\pgfsetlinewidth{0.100375pt}%
\definecolor{currentstroke}{rgb}{0.827451,0.827451,0.827451}%
\pgfsetstrokecolor{currentstroke}%
\pgfsetdash{}{0pt}%
\pgfpathmoveto{\pgfqpoint{5.261397in}{1.080890in}}%
\pgfpathlineto{\pgfqpoint{5.261397in}{3.227753in}}%
\pgfusepath{stroke}%
\end{pgfscope}%
\begin{pgfscope}%
\pgfsetbuttcap%
\pgfsetroundjoin%
\definecolor{currentfill}{rgb}{0.000000,0.000000,0.000000}%
\pgfsetfillcolor{currentfill}%
\pgfsetlinewidth{0.501875pt}%
\definecolor{currentstroke}{rgb}{0.000000,0.000000,0.000000}%
\pgfsetstrokecolor{currentstroke}%
\pgfsetdash{}{0pt}%
\pgfsys@defobject{currentmarker}{\pgfqpoint{0.000000in}{0.000000in}}{\pgfqpoint{0.000000in}{0.020833in}}{%
\pgfpathmoveto{\pgfqpoint{0.000000in}{0.000000in}}%
\pgfpathlineto{\pgfqpoint{0.000000in}{0.020833in}}%
\pgfusepath{stroke,fill}%
}%
\begin{pgfscope}%
\pgfsys@transformshift{5.261397in}{1.080890in}%
\pgfsys@useobject{currentmarker}{}%
\end{pgfscope}%
\end{pgfscope}%
\begin{pgfscope}%
\pgfsetbuttcap%
\pgfsetroundjoin%
\definecolor{currentfill}{rgb}{0.000000,0.000000,0.000000}%
\pgfsetfillcolor{currentfill}%
\pgfsetlinewidth{0.501875pt}%
\definecolor{currentstroke}{rgb}{0.000000,0.000000,0.000000}%
\pgfsetstrokecolor{currentstroke}%
\pgfsetdash{}{0pt}%
\pgfsys@defobject{currentmarker}{\pgfqpoint{0.000000in}{-0.020833in}}{\pgfqpoint{0.000000in}{0.000000in}}{%
\pgfpathmoveto{\pgfqpoint{0.000000in}{0.000000in}}%
\pgfpathlineto{\pgfqpoint{0.000000in}{-0.020833in}}%
\pgfusepath{stroke,fill}%
}%
\begin{pgfscope}%
\pgfsys@transformshift{5.261397in}{3.227753in}%
\pgfsys@useobject{currentmarker}{}%
\end{pgfscope}%
\end{pgfscope}%
\begin{pgfscope}%
\pgfpathrectangle{\pgfqpoint{0.481681in}{1.080890in}}{\pgfqpoint{5.785672in}{2.146863in}}%
\pgfusepath{clip}%
\pgfsetrectcap%
\pgfsetroundjoin%
\pgfsetlinewidth{0.100375pt}%
\definecolor{currentstroke}{rgb}{0.827451,0.827451,0.827451}%
\pgfsetstrokecolor{currentstroke}%
\pgfsetdash{}{0pt}%
\pgfpathmoveto{\pgfqpoint{5.296917in}{1.080890in}}%
\pgfpathlineto{\pgfqpoint{5.296917in}{3.227753in}}%
\pgfusepath{stroke}%
\end{pgfscope}%
\begin{pgfscope}%
\pgfsetbuttcap%
\pgfsetroundjoin%
\definecolor{currentfill}{rgb}{0.000000,0.000000,0.000000}%
\pgfsetfillcolor{currentfill}%
\pgfsetlinewidth{0.501875pt}%
\definecolor{currentstroke}{rgb}{0.000000,0.000000,0.000000}%
\pgfsetstrokecolor{currentstroke}%
\pgfsetdash{}{0pt}%
\pgfsys@defobject{currentmarker}{\pgfqpoint{0.000000in}{0.000000in}}{\pgfqpoint{0.000000in}{0.020833in}}{%
\pgfpathmoveto{\pgfqpoint{0.000000in}{0.000000in}}%
\pgfpathlineto{\pgfqpoint{0.000000in}{0.020833in}}%
\pgfusepath{stroke,fill}%
}%
\begin{pgfscope}%
\pgfsys@transformshift{5.296917in}{1.080890in}%
\pgfsys@useobject{currentmarker}{}%
\end{pgfscope}%
\end{pgfscope}%
\begin{pgfscope}%
\pgfsetbuttcap%
\pgfsetroundjoin%
\definecolor{currentfill}{rgb}{0.000000,0.000000,0.000000}%
\pgfsetfillcolor{currentfill}%
\pgfsetlinewidth{0.501875pt}%
\definecolor{currentstroke}{rgb}{0.000000,0.000000,0.000000}%
\pgfsetstrokecolor{currentstroke}%
\pgfsetdash{}{0pt}%
\pgfsys@defobject{currentmarker}{\pgfqpoint{0.000000in}{-0.020833in}}{\pgfqpoint{0.000000in}{0.000000in}}{%
\pgfpathmoveto{\pgfqpoint{0.000000in}{0.000000in}}%
\pgfpathlineto{\pgfqpoint{0.000000in}{-0.020833in}}%
\pgfusepath{stroke,fill}%
}%
\begin{pgfscope}%
\pgfsys@transformshift{5.296917in}{3.227753in}%
\pgfsys@useobject{currentmarker}{}%
\end{pgfscope}%
\end{pgfscope}%
\begin{pgfscope}%
\pgfpathrectangle{\pgfqpoint{0.481681in}{1.080890in}}{\pgfqpoint{5.785672in}{2.146863in}}%
\pgfusepath{clip}%
\pgfsetrectcap%
\pgfsetroundjoin%
\pgfsetlinewidth{0.100375pt}%
\definecolor{currentstroke}{rgb}{0.827451,0.827451,0.827451}%
\pgfsetstrokecolor{currentstroke}%
\pgfsetdash{}{0pt}%
\pgfpathmoveto{\pgfqpoint{5.367959in}{1.080890in}}%
\pgfpathlineto{\pgfqpoint{5.367959in}{3.227753in}}%
\pgfusepath{stroke}%
\end{pgfscope}%
\begin{pgfscope}%
\pgfsetbuttcap%
\pgfsetroundjoin%
\definecolor{currentfill}{rgb}{0.000000,0.000000,0.000000}%
\pgfsetfillcolor{currentfill}%
\pgfsetlinewidth{0.501875pt}%
\definecolor{currentstroke}{rgb}{0.000000,0.000000,0.000000}%
\pgfsetstrokecolor{currentstroke}%
\pgfsetdash{}{0pt}%
\pgfsys@defobject{currentmarker}{\pgfqpoint{0.000000in}{0.000000in}}{\pgfqpoint{0.000000in}{0.020833in}}{%
\pgfpathmoveto{\pgfqpoint{0.000000in}{0.000000in}}%
\pgfpathlineto{\pgfqpoint{0.000000in}{0.020833in}}%
\pgfusepath{stroke,fill}%
}%
\begin{pgfscope}%
\pgfsys@transformshift{5.367959in}{1.080890in}%
\pgfsys@useobject{currentmarker}{}%
\end{pgfscope}%
\end{pgfscope}%
\begin{pgfscope}%
\pgfsetbuttcap%
\pgfsetroundjoin%
\definecolor{currentfill}{rgb}{0.000000,0.000000,0.000000}%
\pgfsetfillcolor{currentfill}%
\pgfsetlinewidth{0.501875pt}%
\definecolor{currentstroke}{rgb}{0.000000,0.000000,0.000000}%
\pgfsetstrokecolor{currentstroke}%
\pgfsetdash{}{0pt}%
\pgfsys@defobject{currentmarker}{\pgfqpoint{0.000000in}{-0.020833in}}{\pgfqpoint{0.000000in}{0.000000in}}{%
\pgfpathmoveto{\pgfqpoint{0.000000in}{0.000000in}}%
\pgfpathlineto{\pgfqpoint{0.000000in}{-0.020833in}}%
\pgfusepath{stroke,fill}%
}%
\begin{pgfscope}%
\pgfsys@transformshift{5.367959in}{3.227753in}%
\pgfsys@useobject{currentmarker}{}%
\end{pgfscope}%
\end{pgfscope}%
\begin{pgfscope}%
\pgfpathrectangle{\pgfqpoint{0.481681in}{1.080890in}}{\pgfqpoint{5.785672in}{2.146863in}}%
\pgfusepath{clip}%
\pgfsetrectcap%
\pgfsetroundjoin%
\pgfsetlinewidth{0.100375pt}%
\definecolor{currentstroke}{rgb}{0.827451,0.827451,0.827451}%
\pgfsetstrokecolor{currentstroke}%
\pgfsetdash{}{0pt}%
\pgfpathmoveto{\pgfqpoint{5.403479in}{1.080890in}}%
\pgfpathlineto{\pgfqpoint{5.403479in}{3.227753in}}%
\pgfusepath{stroke}%
\end{pgfscope}%
\begin{pgfscope}%
\pgfsetbuttcap%
\pgfsetroundjoin%
\definecolor{currentfill}{rgb}{0.000000,0.000000,0.000000}%
\pgfsetfillcolor{currentfill}%
\pgfsetlinewidth{0.501875pt}%
\definecolor{currentstroke}{rgb}{0.000000,0.000000,0.000000}%
\pgfsetstrokecolor{currentstroke}%
\pgfsetdash{}{0pt}%
\pgfsys@defobject{currentmarker}{\pgfqpoint{0.000000in}{0.000000in}}{\pgfqpoint{0.000000in}{0.020833in}}{%
\pgfpathmoveto{\pgfqpoint{0.000000in}{0.000000in}}%
\pgfpathlineto{\pgfqpoint{0.000000in}{0.020833in}}%
\pgfusepath{stroke,fill}%
}%
\begin{pgfscope}%
\pgfsys@transformshift{5.403479in}{1.080890in}%
\pgfsys@useobject{currentmarker}{}%
\end{pgfscope}%
\end{pgfscope}%
\begin{pgfscope}%
\pgfsetbuttcap%
\pgfsetroundjoin%
\definecolor{currentfill}{rgb}{0.000000,0.000000,0.000000}%
\pgfsetfillcolor{currentfill}%
\pgfsetlinewidth{0.501875pt}%
\definecolor{currentstroke}{rgb}{0.000000,0.000000,0.000000}%
\pgfsetstrokecolor{currentstroke}%
\pgfsetdash{}{0pt}%
\pgfsys@defobject{currentmarker}{\pgfqpoint{0.000000in}{-0.020833in}}{\pgfqpoint{0.000000in}{0.000000in}}{%
\pgfpathmoveto{\pgfqpoint{0.000000in}{0.000000in}}%
\pgfpathlineto{\pgfqpoint{0.000000in}{-0.020833in}}%
\pgfusepath{stroke,fill}%
}%
\begin{pgfscope}%
\pgfsys@transformshift{5.403479in}{3.227753in}%
\pgfsys@useobject{currentmarker}{}%
\end{pgfscope}%
\end{pgfscope}%
\begin{pgfscope}%
\pgfpathrectangle{\pgfqpoint{0.481681in}{1.080890in}}{\pgfqpoint{5.785672in}{2.146863in}}%
\pgfusepath{clip}%
\pgfsetrectcap%
\pgfsetroundjoin%
\pgfsetlinewidth{0.100375pt}%
\definecolor{currentstroke}{rgb}{0.827451,0.827451,0.827451}%
\pgfsetstrokecolor{currentstroke}%
\pgfsetdash{}{0pt}%
\pgfpathmoveto{\pgfqpoint{5.439000in}{1.080890in}}%
\pgfpathlineto{\pgfqpoint{5.439000in}{3.227753in}}%
\pgfusepath{stroke}%
\end{pgfscope}%
\begin{pgfscope}%
\pgfsetbuttcap%
\pgfsetroundjoin%
\definecolor{currentfill}{rgb}{0.000000,0.000000,0.000000}%
\pgfsetfillcolor{currentfill}%
\pgfsetlinewidth{0.501875pt}%
\definecolor{currentstroke}{rgb}{0.000000,0.000000,0.000000}%
\pgfsetstrokecolor{currentstroke}%
\pgfsetdash{}{0pt}%
\pgfsys@defobject{currentmarker}{\pgfqpoint{0.000000in}{0.000000in}}{\pgfqpoint{0.000000in}{0.020833in}}{%
\pgfpathmoveto{\pgfqpoint{0.000000in}{0.000000in}}%
\pgfpathlineto{\pgfqpoint{0.000000in}{0.020833in}}%
\pgfusepath{stroke,fill}%
}%
\begin{pgfscope}%
\pgfsys@transformshift{5.439000in}{1.080890in}%
\pgfsys@useobject{currentmarker}{}%
\end{pgfscope}%
\end{pgfscope}%
\begin{pgfscope}%
\pgfsetbuttcap%
\pgfsetroundjoin%
\definecolor{currentfill}{rgb}{0.000000,0.000000,0.000000}%
\pgfsetfillcolor{currentfill}%
\pgfsetlinewidth{0.501875pt}%
\definecolor{currentstroke}{rgb}{0.000000,0.000000,0.000000}%
\pgfsetstrokecolor{currentstroke}%
\pgfsetdash{}{0pt}%
\pgfsys@defobject{currentmarker}{\pgfqpoint{0.000000in}{-0.020833in}}{\pgfqpoint{0.000000in}{0.000000in}}{%
\pgfpathmoveto{\pgfqpoint{0.000000in}{0.000000in}}%
\pgfpathlineto{\pgfqpoint{0.000000in}{-0.020833in}}%
\pgfusepath{stroke,fill}%
}%
\begin{pgfscope}%
\pgfsys@transformshift{5.439000in}{3.227753in}%
\pgfsys@useobject{currentmarker}{}%
\end{pgfscope}%
\end{pgfscope}%
\begin{pgfscope}%
\pgfpathrectangle{\pgfqpoint{0.481681in}{1.080890in}}{\pgfqpoint{5.785672in}{2.146863in}}%
\pgfusepath{clip}%
\pgfsetrectcap%
\pgfsetroundjoin%
\pgfsetlinewidth{0.100375pt}%
\definecolor{currentstroke}{rgb}{0.827451,0.827451,0.827451}%
\pgfsetstrokecolor{currentstroke}%
\pgfsetdash{}{0pt}%
\pgfpathmoveto{\pgfqpoint{5.474520in}{1.080890in}}%
\pgfpathlineto{\pgfqpoint{5.474520in}{3.227753in}}%
\pgfusepath{stroke}%
\end{pgfscope}%
\begin{pgfscope}%
\pgfsetbuttcap%
\pgfsetroundjoin%
\definecolor{currentfill}{rgb}{0.000000,0.000000,0.000000}%
\pgfsetfillcolor{currentfill}%
\pgfsetlinewidth{0.501875pt}%
\definecolor{currentstroke}{rgb}{0.000000,0.000000,0.000000}%
\pgfsetstrokecolor{currentstroke}%
\pgfsetdash{}{0pt}%
\pgfsys@defobject{currentmarker}{\pgfqpoint{0.000000in}{0.000000in}}{\pgfqpoint{0.000000in}{0.020833in}}{%
\pgfpathmoveto{\pgfqpoint{0.000000in}{0.000000in}}%
\pgfpathlineto{\pgfqpoint{0.000000in}{0.020833in}}%
\pgfusepath{stroke,fill}%
}%
\begin{pgfscope}%
\pgfsys@transformshift{5.474520in}{1.080890in}%
\pgfsys@useobject{currentmarker}{}%
\end{pgfscope}%
\end{pgfscope}%
\begin{pgfscope}%
\pgfsetbuttcap%
\pgfsetroundjoin%
\definecolor{currentfill}{rgb}{0.000000,0.000000,0.000000}%
\pgfsetfillcolor{currentfill}%
\pgfsetlinewidth{0.501875pt}%
\definecolor{currentstroke}{rgb}{0.000000,0.000000,0.000000}%
\pgfsetstrokecolor{currentstroke}%
\pgfsetdash{}{0pt}%
\pgfsys@defobject{currentmarker}{\pgfqpoint{0.000000in}{-0.020833in}}{\pgfqpoint{0.000000in}{0.000000in}}{%
\pgfpathmoveto{\pgfqpoint{0.000000in}{0.000000in}}%
\pgfpathlineto{\pgfqpoint{0.000000in}{-0.020833in}}%
\pgfusepath{stroke,fill}%
}%
\begin{pgfscope}%
\pgfsys@transformshift{5.474520in}{3.227753in}%
\pgfsys@useobject{currentmarker}{}%
\end{pgfscope}%
\end{pgfscope}%
\begin{pgfscope}%
\pgfpathrectangle{\pgfqpoint{0.481681in}{1.080890in}}{\pgfqpoint{5.785672in}{2.146863in}}%
\pgfusepath{clip}%
\pgfsetrectcap%
\pgfsetroundjoin%
\pgfsetlinewidth{0.100375pt}%
\definecolor{currentstroke}{rgb}{0.827451,0.827451,0.827451}%
\pgfsetstrokecolor{currentstroke}%
\pgfsetdash{}{0pt}%
\pgfpathmoveto{\pgfqpoint{5.510041in}{1.080890in}}%
\pgfpathlineto{\pgfqpoint{5.510041in}{3.227753in}}%
\pgfusepath{stroke}%
\end{pgfscope}%
\begin{pgfscope}%
\pgfsetbuttcap%
\pgfsetroundjoin%
\definecolor{currentfill}{rgb}{0.000000,0.000000,0.000000}%
\pgfsetfillcolor{currentfill}%
\pgfsetlinewidth{0.501875pt}%
\definecolor{currentstroke}{rgb}{0.000000,0.000000,0.000000}%
\pgfsetstrokecolor{currentstroke}%
\pgfsetdash{}{0pt}%
\pgfsys@defobject{currentmarker}{\pgfqpoint{0.000000in}{0.000000in}}{\pgfqpoint{0.000000in}{0.020833in}}{%
\pgfpathmoveto{\pgfqpoint{0.000000in}{0.000000in}}%
\pgfpathlineto{\pgfqpoint{0.000000in}{0.020833in}}%
\pgfusepath{stroke,fill}%
}%
\begin{pgfscope}%
\pgfsys@transformshift{5.510041in}{1.080890in}%
\pgfsys@useobject{currentmarker}{}%
\end{pgfscope}%
\end{pgfscope}%
\begin{pgfscope}%
\pgfsetbuttcap%
\pgfsetroundjoin%
\definecolor{currentfill}{rgb}{0.000000,0.000000,0.000000}%
\pgfsetfillcolor{currentfill}%
\pgfsetlinewidth{0.501875pt}%
\definecolor{currentstroke}{rgb}{0.000000,0.000000,0.000000}%
\pgfsetstrokecolor{currentstroke}%
\pgfsetdash{}{0pt}%
\pgfsys@defobject{currentmarker}{\pgfqpoint{0.000000in}{-0.020833in}}{\pgfqpoint{0.000000in}{0.000000in}}{%
\pgfpathmoveto{\pgfqpoint{0.000000in}{0.000000in}}%
\pgfpathlineto{\pgfqpoint{0.000000in}{-0.020833in}}%
\pgfusepath{stroke,fill}%
}%
\begin{pgfscope}%
\pgfsys@transformshift{5.510041in}{3.227753in}%
\pgfsys@useobject{currentmarker}{}%
\end{pgfscope}%
\end{pgfscope}%
\begin{pgfscope}%
\pgfpathrectangle{\pgfqpoint{0.481681in}{1.080890in}}{\pgfqpoint{5.785672in}{2.146863in}}%
\pgfusepath{clip}%
\pgfsetrectcap%
\pgfsetroundjoin%
\pgfsetlinewidth{0.100375pt}%
\definecolor{currentstroke}{rgb}{0.827451,0.827451,0.827451}%
\pgfsetstrokecolor{currentstroke}%
\pgfsetdash{}{0pt}%
\pgfpathmoveto{\pgfqpoint{5.545561in}{1.080890in}}%
\pgfpathlineto{\pgfqpoint{5.545561in}{3.227753in}}%
\pgfusepath{stroke}%
\end{pgfscope}%
\begin{pgfscope}%
\pgfsetbuttcap%
\pgfsetroundjoin%
\definecolor{currentfill}{rgb}{0.000000,0.000000,0.000000}%
\pgfsetfillcolor{currentfill}%
\pgfsetlinewidth{0.501875pt}%
\definecolor{currentstroke}{rgb}{0.000000,0.000000,0.000000}%
\pgfsetstrokecolor{currentstroke}%
\pgfsetdash{}{0pt}%
\pgfsys@defobject{currentmarker}{\pgfqpoint{0.000000in}{0.000000in}}{\pgfqpoint{0.000000in}{0.020833in}}{%
\pgfpathmoveto{\pgfqpoint{0.000000in}{0.000000in}}%
\pgfpathlineto{\pgfqpoint{0.000000in}{0.020833in}}%
\pgfusepath{stroke,fill}%
}%
\begin{pgfscope}%
\pgfsys@transformshift{5.545561in}{1.080890in}%
\pgfsys@useobject{currentmarker}{}%
\end{pgfscope}%
\end{pgfscope}%
\begin{pgfscope}%
\pgfsetbuttcap%
\pgfsetroundjoin%
\definecolor{currentfill}{rgb}{0.000000,0.000000,0.000000}%
\pgfsetfillcolor{currentfill}%
\pgfsetlinewidth{0.501875pt}%
\definecolor{currentstroke}{rgb}{0.000000,0.000000,0.000000}%
\pgfsetstrokecolor{currentstroke}%
\pgfsetdash{}{0pt}%
\pgfsys@defobject{currentmarker}{\pgfqpoint{0.000000in}{-0.020833in}}{\pgfqpoint{0.000000in}{0.000000in}}{%
\pgfpathmoveto{\pgfqpoint{0.000000in}{0.000000in}}%
\pgfpathlineto{\pgfqpoint{0.000000in}{-0.020833in}}%
\pgfusepath{stroke,fill}%
}%
\begin{pgfscope}%
\pgfsys@transformshift{5.545561in}{3.227753in}%
\pgfsys@useobject{currentmarker}{}%
\end{pgfscope}%
\end{pgfscope}%
\begin{pgfscope}%
\pgfpathrectangle{\pgfqpoint{0.481681in}{1.080890in}}{\pgfqpoint{5.785672in}{2.146863in}}%
\pgfusepath{clip}%
\pgfsetrectcap%
\pgfsetroundjoin%
\pgfsetlinewidth{0.100375pt}%
\definecolor{currentstroke}{rgb}{0.827451,0.827451,0.827451}%
\pgfsetstrokecolor{currentstroke}%
\pgfsetdash{}{0pt}%
\pgfpathmoveto{\pgfqpoint{5.581082in}{1.080890in}}%
\pgfpathlineto{\pgfqpoint{5.581082in}{3.227753in}}%
\pgfusepath{stroke}%
\end{pgfscope}%
\begin{pgfscope}%
\pgfsetbuttcap%
\pgfsetroundjoin%
\definecolor{currentfill}{rgb}{0.000000,0.000000,0.000000}%
\pgfsetfillcolor{currentfill}%
\pgfsetlinewidth{0.501875pt}%
\definecolor{currentstroke}{rgb}{0.000000,0.000000,0.000000}%
\pgfsetstrokecolor{currentstroke}%
\pgfsetdash{}{0pt}%
\pgfsys@defobject{currentmarker}{\pgfqpoint{0.000000in}{0.000000in}}{\pgfqpoint{0.000000in}{0.020833in}}{%
\pgfpathmoveto{\pgfqpoint{0.000000in}{0.000000in}}%
\pgfpathlineto{\pgfqpoint{0.000000in}{0.020833in}}%
\pgfusepath{stroke,fill}%
}%
\begin{pgfscope}%
\pgfsys@transformshift{5.581082in}{1.080890in}%
\pgfsys@useobject{currentmarker}{}%
\end{pgfscope}%
\end{pgfscope}%
\begin{pgfscope}%
\pgfsetbuttcap%
\pgfsetroundjoin%
\definecolor{currentfill}{rgb}{0.000000,0.000000,0.000000}%
\pgfsetfillcolor{currentfill}%
\pgfsetlinewidth{0.501875pt}%
\definecolor{currentstroke}{rgb}{0.000000,0.000000,0.000000}%
\pgfsetstrokecolor{currentstroke}%
\pgfsetdash{}{0pt}%
\pgfsys@defobject{currentmarker}{\pgfqpoint{0.000000in}{-0.020833in}}{\pgfqpoint{0.000000in}{0.000000in}}{%
\pgfpathmoveto{\pgfqpoint{0.000000in}{0.000000in}}%
\pgfpathlineto{\pgfqpoint{0.000000in}{-0.020833in}}%
\pgfusepath{stroke,fill}%
}%
\begin{pgfscope}%
\pgfsys@transformshift{5.581082in}{3.227753in}%
\pgfsys@useobject{currentmarker}{}%
\end{pgfscope}%
\end{pgfscope}%
\begin{pgfscope}%
\pgfpathrectangle{\pgfqpoint{0.481681in}{1.080890in}}{\pgfqpoint{5.785672in}{2.146863in}}%
\pgfusepath{clip}%
\pgfsetrectcap%
\pgfsetroundjoin%
\pgfsetlinewidth{0.100375pt}%
\definecolor{currentstroke}{rgb}{0.827451,0.827451,0.827451}%
\pgfsetstrokecolor{currentstroke}%
\pgfsetdash{}{0pt}%
\pgfpathmoveto{\pgfqpoint{5.616602in}{1.080890in}}%
\pgfpathlineto{\pgfqpoint{5.616602in}{3.227753in}}%
\pgfusepath{stroke}%
\end{pgfscope}%
\begin{pgfscope}%
\pgfsetbuttcap%
\pgfsetroundjoin%
\definecolor{currentfill}{rgb}{0.000000,0.000000,0.000000}%
\pgfsetfillcolor{currentfill}%
\pgfsetlinewidth{0.501875pt}%
\definecolor{currentstroke}{rgb}{0.000000,0.000000,0.000000}%
\pgfsetstrokecolor{currentstroke}%
\pgfsetdash{}{0pt}%
\pgfsys@defobject{currentmarker}{\pgfqpoint{0.000000in}{0.000000in}}{\pgfqpoint{0.000000in}{0.020833in}}{%
\pgfpathmoveto{\pgfqpoint{0.000000in}{0.000000in}}%
\pgfpathlineto{\pgfqpoint{0.000000in}{0.020833in}}%
\pgfusepath{stroke,fill}%
}%
\begin{pgfscope}%
\pgfsys@transformshift{5.616602in}{1.080890in}%
\pgfsys@useobject{currentmarker}{}%
\end{pgfscope}%
\end{pgfscope}%
\begin{pgfscope}%
\pgfsetbuttcap%
\pgfsetroundjoin%
\definecolor{currentfill}{rgb}{0.000000,0.000000,0.000000}%
\pgfsetfillcolor{currentfill}%
\pgfsetlinewidth{0.501875pt}%
\definecolor{currentstroke}{rgb}{0.000000,0.000000,0.000000}%
\pgfsetstrokecolor{currentstroke}%
\pgfsetdash{}{0pt}%
\pgfsys@defobject{currentmarker}{\pgfqpoint{0.000000in}{-0.020833in}}{\pgfqpoint{0.000000in}{0.000000in}}{%
\pgfpathmoveto{\pgfqpoint{0.000000in}{0.000000in}}%
\pgfpathlineto{\pgfqpoint{0.000000in}{-0.020833in}}%
\pgfusepath{stroke,fill}%
}%
\begin{pgfscope}%
\pgfsys@transformshift{5.616602in}{3.227753in}%
\pgfsys@useobject{currentmarker}{}%
\end{pgfscope}%
\end{pgfscope}%
\begin{pgfscope}%
\pgfpathrectangle{\pgfqpoint{0.481681in}{1.080890in}}{\pgfqpoint{5.785672in}{2.146863in}}%
\pgfusepath{clip}%
\pgfsetrectcap%
\pgfsetroundjoin%
\pgfsetlinewidth{0.100375pt}%
\definecolor{currentstroke}{rgb}{0.827451,0.827451,0.827451}%
\pgfsetstrokecolor{currentstroke}%
\pgfsetdash{}{0pt}%
\pgfpathmoveto{\pgfqpoint{5.652123in}{1.080890in}}%
\pgfpathlineto{\pgfqpoint{5.652123in}{3.227753in}}%
\pgfusepath{stroke}%
\end{pgfscope}%
\begin{pgfscope}%
\pgfsetbuttcap%
\pgfsetroundjoin%
\definecolor{currentfill}{rgb}{0.000000,0.000000,0.000000}%
\pgfsetfillcolor{currentfill}%
\pgfsetlinewidth{0.501875pt}%
\definecolor{currentstroke}{rgb}{0.000000,0.000000,0.000000}%
\pgfsetstrokecolor{currentstroke}%
\pgfsetdash{}{0pt}%
\pgfsys@defobject{currentmarker}{\pgfqpoint{0.000000in}{0.000000in}}{\pgfqpoint{0.000000in}{0.020833in}}{%
\pgfpathmoveto{\pgfqpoint{0.000000in}{0.000000in}}%
\pgfpathlineto{\pgfqpoint{0.000000in}{0.020833in}}%
\pgfusepath{stroke,fill}%
}%
\begin{pgfscope}%
\pgfsys@transformshift{5.652123in}{1.080890in}%
\pgfsys@useobject{currentmarker}{}%
\end{pgfscope}%
\end{pgfscope}%
\begin{pgfscope}%
\pgfsetbuttcap%
\pgfsetroundjoin%
\definecolor{currentfill}{rgb}{0.000000,0.000000,0.000000}%
\pgfsetfillcolor{currentfill}%
\pgfsetlinewidth{0.501875pt}%
\definecolor{currentstroke}{rgb}{0.000000,0.000000,0.000000}%
\pgfsetstrokecolor{currentstroke}%
\pgfsetdash{}{0pt}%
\pgfsys@defobject{currentmarker}{\pgfqpoint{0.000000in}{-0.020833in}}{\pgfqpoint{0.000000in}{0.000000in}}{%
\pgfpathmoveto{\pgfqpoint{0.000000in}{0.000000in}}%
\pgfpathlineto{\pgfqpoint{0.000000in}{-0.020833in}}%
\pgfusepath{stroke,fill}%
}%
\begin{pgfscope}%
\pgfsys@transformshift{5.652123in}{3.227753in}%
\pgfsys@useobject{currentmarker}{}%
\end{pgfscope}%
\end{pgfscope}%
\begin{pgfscope}%
\pgfpathrectangle{\pgfqpoint{0.481681in}{1.080890in}}{\pgfqpoint{5.785672in}{2.146863in}}%
\pgfusepath{clip}%
\pgfsetrectcap%
\pgfsetroundjoin%
\pgfsetlinewidth{0.100375pt}%
\definecolor{currentstroke}{rgb}{0.827451,0.827451,0.827451}%
\pgfsetstrokecolor{currentstroke}%
\pgfsetdash{}{0pt}%
\pgfpathmoveto{\pgfqpoint{5.687643in}{1.080890in}}%
\pgfpathlineto{\pgfqpoint{5.687643in}{3.227753in}}%
\pgfusepath{stroke}%
\end{pgfscope}%
\begin{pgfscope}%
\pgfsetbuttcap%
\pgfsetroundjoin%
\definecolor{currentfill}{rgb}{0.000000,0.000000,0.000000}%
\pgfsetfillcolor{currentfill}%
\pgfsetlinewidth{0.501875pt}%
\definecolor{currentstroke}{rgb}{0.000000,0.000000,0.000000}%
\pgfsetstrokecolor{currentstroke}%
\pgfsetdash{}{0pt}%
\pgfsys@defobject{currentmarker}{\pgfqpoint{0.000000in}{0.000000in}}{\pgfqpoint{0.000000in}{0.020833in}}{%
\pgfpathmoveto{\pgfqpoint{0.000000in}{0.000000in}}%
\pgfpathlineto{\pgfqpoint{0.000000in}{0.020833in}}%
\pgfusepath{stroke,fill}%
}%
\begin{pgfscope}%
\pgfsys@transformshift{5.687643in}{1.080890in}%
\pgfsys@useobject{currentmarker}{}%
\end{pgfscope}%
\end{pgfscope}%
\begin{pgfscope}%
\pgfsetbuttcap%
\pgfsetroundjoin%
\definecolor{currentfill}{rgb}{0.000000,0.000000,0.000000}%
\pgfsetfillcolor{currentfill}%
\pgfsetlinewidth{0.501875pt}%
\definecolor{currentstroke}{rgb}{0.000000,0.000000,0.000000}%
\pgfsetstrokecolor{currentstroke}%
\pgfsetdash{}{0pt}%
\pgfsys@defobject{currentmarker}{\pgfqpoint{0.000000in}{-0.020833in}}{\pgfqpoint{0.000000in}{0.000000in}}{%
\pgfpathmoveto{\pgfqpoint{0.000000in}{0.000000in}}%
\pgfpathlineto{\pgfqpoint{0.000000in}{-0.020833in}}%
\pgfusepath{stroke,fill}%
}%
\begin{pgfscope}%
\pgfsys@transformshift{5.687643in}{3.227753in}%
\pgfsys@useobject{currentmarker}{}%
\end{pgfscope}%
\end{pgfscope}%
\begin{pgfscope}%
\pgfpathrectangle{\pgfqpoint{0.481681in}{1.080890in}}{\pgfqpoint{5.785672in}{2.146863in}}%
\pgfusepath{clip}%
\pgfsetrectcap%
\pgfsetroundjoin%
\pgfsetlinewidth{0.100375pt}%
\definecolor{currentstroke}{rgb}{0.827451,0.827451,0.827451}%
\pgfsetstrokecolor{currentstroke}%
\pgfsetdash{}{0pt}%
\pgfpathmoveto{\pgfqpoint{5.723164in}{1.080890in}}%
\pgfpathlineto{\pgfqpoint{5.723164in}{3.227753in}}%
\pgfusepath{stroke}%
\end{pgfscope}%
\begin{pgfscope}%
\pgfsetbuttcap%
\pgfsetroundjoin%
\definecolor{currentfill}{rgb}{0.000000,0.000000,0.000000}%
\pgfsetfillcolor{currentfill}%
\pgfsetlinewidth{0.501875pt}%
\definecolor{currentstroke}{rgb}{0.000000,0.000000,0.000000}%
\pgfsetstrokecolor{currentstroke}%
\pgfsetdash{}{0pt}%
\pgfsys@defobject{currentmarker}{\pgfqpoint{0.000000in}{0.000000in}}{\pgfqpoint{0.000000in}{0.020833in}}{%
\pgfpathmoveto{\pgfqpoint{0.000000in}{0.000000in}}%
\pgfpathlineto{\pgfqpoint{0.000000in}{0.020833in}}%
\pgfusepath{stroke,fill}%
}%
\begin{pgfscope}%
\pgfsys@transformshift{5.723164in}{1.080890in}%
\pgfsys@useobject{currentmarker}{}%
\end{pgfscope}%
\end{pgfscope}%
\begin{pgfscope}%
\pgfsetbuttcap%
\pgfsetroundjoin%
\definecolor{currentfill}{rgb}{0.000000,0.000000,0.000000}%
\pgfsetfillcolor{currentfill}%
\pgfsetlinewidth{0.501875pt}%
\definecolor{currentstroke}{rgb}{0.000000,0.000000,0.000000}%
\pgfsetstrokecolor{currentstroke}%
\pgfsetdash{}{0pt}%
\pgfsys@defobject{currentmarker}{\pgfqpoint{0.000000in}{-0.020833in}}{\pgfqpoint{0.000000in}{0.000000in}}{%
\pgfpathmoveto{\pgfqpoint{0.000000in}{0.000000in}}%
\pgfpathlineto{\pgfqpoint{0.000000in}{-0.020833in}}%
\pgfusepath{stroke,fill}%
}%
\begin{pgfscope}%
\pgfsys@transformshift{5.723164in}{3.227753in}%
\pgfsys@useobject{currentmarker}{}%
\end{pgfscope}%
\end{pgfscope}%
\begin{pgfscope}%
\pgfpathrectangle{\pgfqpoint{0.481681in}{1.080890in}}{\pgfqpoint{5.785672in}{2.146863in}}%
\pgfusepath{clip}%
\pgfsetrectcap%
\pgfsetroundjoin%
\pgfsetlinewidth{0.100375pt}%
\definecolor{currentstroke}{rgb}{0.827451,0.827451,0.827451}%
\pgfsetstrokecolor{currentstroke}%
\pgfsetdash{}{0pt}%
\pgfpathmoveto{\pgfqpoint{5.794205in}{1.080890in}}%
\pgfpathlineto{\pgfqpoint{5.794205in}{3.227753in}}%
\pgfusepath{stroke}%
\end{pgfscope}%
\begin{pgfscope}%
\pgfsetbuttcap%
\pgfsetroundjoin%
\definecolor{currentfill}{rgb}{0.000000,0.000000,0.000000}%
\pgfsetfillcolor{currentfill}%
\pgfsetlinewidth{0.501875pt}%
\definecolor{currentstroke}{rgb}{0.000000,0.000000,0.000000}%
\pgfsetstrokecolor{currentstroke}%
\pgfsetdash{}{0pt}%
\pgfsys@defobject{currentmarker}{\pgfqpoint{0.000000in}{0.000000in}}{\pgfqpoint{0.000000in}{0.020833in}}{%
\pgfpathmoveto{\pgfqpoint{0.000000in}{0.000000in}}%
\pgfpathlineto{\pgfqpoint{0.000000in}{0.020833in}}%
\pgfusepath{stroke,fill}%
}%
\begin{pgfscope}%
\pgfsys@transformshift{5.794205in}{1.080890in}%
\pgfsys@useobject{currentmarker}{}%
\end{pgfscope}%
\end{pgfscope}%
\begin{pgfscope}%
\pgfsetbuttcap%
\pgfsetroundjoin%
\definecolor{currentfill}{rgb}{0.000000,0.000000,0.000000}%
\pgfsetfillcolor{currentfill}%
\pgfsetlinewidth{0.501875pt}%
\definecolor{currentstroke}{rgb}{0.000000,0.000000,0.000000}%
\pgfsetstrokecolor{currentstroke}%
\pgfsetdash{}{0pt}%
\pgfsys@defobject{currentmarker}{\pgfqpoint{0.000000in}{-0.020833in}}{\pgfqpoint{0.000000in}{0.000000in}}{%
\pgfpathmoveto{\pgfqpoint{0.000000in}{0.000000in}}%
\pgfpathlineto{\pgfqpoint{0.000000in}{-0.020833in}}%
\pgfusepath{stroke,fill}%
}%
\begin{pgfscope}%
\pgfsys@transformshift{5.794205in}{3.227753in}%
\pgfsys@useobject{currentmarker}{}%
\end{pgfscope}%
\end{pgfscope}%
\begin{pgfscope}%
\pgfpathrectangle{\pgfqpoint{0.481681in}{1.080890in}}{\pgfqpoint{5.785672in}{2.146863in}}%
\pgfusepath{clip}%
\pgfsetrectcap%
\pgfsetroundjoin%
\pgfsetlinewidth{0.100375pt}%
\definecolor{currentstroke}{rgb}{0.827451,0.827451,0.827451}%
\pgfsetstrokecolor{currentstroke}%
\pgfsetdash{}{0pt}%
\pgfpathmoveto{\pgfqpoint{5.829725in}{1.080890in}}%
\pgfpathlineto{\pgfqpoint{5.829725in}{3.227753in}}%
\pgfusepath{stroke}%
\end{pgfscope}%
\begin{pgfscope}%
\pgfsetbuttcap%
\pgfsetroundjoin%
\definecolor{currentfill}{rgb}{0.000000,0.000000,0.000000}%
\pgfsetfillcolor{currentfill}%
\pgfsetlinewidth{0.501875pt}%
\definecolor{currentstroke}{rgb}{0.000000,0.000000,0.000000}%
\pgfsetstrokecolor{currentstroke}%
\pgfsetdash{}{0pt}%
\pgfsys@defobject{currentmarker}{\pgfqpoint{0.000000in}{0.000000in}}{\pgfqpoint{0.000000in}{0.020833in}}{%
\pgfpathmoveto{\pgfqpoint{0.000000in}{0.000000in}}%
\pgfpathlineto{\pgfqpoint{0.000000in}{0.020833in}}%
\pgfusepath{stroke,fill}%
}%
\begin{pgfscope}%
\pgfsys@transformshift{5.829725in}{1.080890in}%
\pgfsys@useobject{currentmarker}{}%
\end{pgfscope}%
\end{pgfscope}%
\begin{pgfscope}%
\pgfsetbuttcap%
\pgfsetroundjoin%
\definecolor{currentfill}{rgb}{0.000000,0.000000,0.000000}%
\pgfsetfillcolor{currentfill}%
\pgfsetlinewidth{0.501875pt}%
\definecolor{currentstroke}{rgb}{0.000000,0.000000,0.000000}%
\pgfsetstrokecolor{currentstroke}%
\pgfsetdash{}{0pt}%
\pgfsys@defobject{currentmarker}{\pgfqpoint{0.000000in}{-0.020833in}}{\pgfqpoint{0.000000in}{0.000000in}}{%
\pgfpathmoveto{\pgfqpoint{0.000000in}{0.000000in}}%
\pgfpathlineto{\pgfqpoint{0.000000in}{-0.020833in}}%
\pgfusepath{stroke,fill}%
}%
\begin{pgfscope}%
\pgfsys@transformshift{5.829725in}{3.227753in}%
\pgfsys@useobject{currentmarker}{}%
\end{pgfscope}%
\end{pgfscope}%
\begin{pgfscope}%
\pgfpathrectangle{\pgfqpoint{0.481681in}{1.080890in}}{\pgfqpoint{5.785672in}{2.146863in}}%
\pgfusepath{clip}%
\pgfsetrectcap%
\pgfsetroundjoin%
\pgfsetlinewidth{0.100375pt}%
\definecolor{currentstroke}{rgb}{0.827451,0.827451,0.827451}%
\pgfsetstrokecolor{currentstroke}%
\pgfsetdash{}{0pt}%
\pgfpathmoveto{\pgfqpoint{5.865246in}{1.080890in}}%
\pgfpathlineto{\pgfqpoint{5.865246in}{3.227753in}}%
\pgfusepath{stroke}%
\end{pgfscope}%
\begin{pgfscope}%
\pgfsetbuttcap%
\pgfsetroundjoin%
\definecolor{currentfill}{rgb}{0.000000,0.000000,0.000000}%
\pgfsetfillcolor{currentfill}%
\pgfsetlinewidth{0.501875pt}%
\definecolor{currentstroke}{rgb}{0.000000,0.000000,0.000000}%
\pgfsetstrokecolor{currentstroke}%
\pgfsetdash{}{0pt}%
\pgfsys@defobject{currentmarker}{\pgfqpoint{0.000000in}{0.000000in}}{\pgfqpoint{0.000000in}{0.020833in}}{%
\pgfpathmoveto{\pgfqpoint{0.000000in}{0.000000in}}%
\pgfpathlineto{\pgfqpoint{0.000000in}{0.020833in}}%
\pgfusepath{stroke,fill}%
}%
\begin{pgfscope}%
\pgfsys@transformshift{5.865246in}{1.080890in}%
\pgfsys@useobject{currentmarker}{}%
\end{pgfscope}%
\end{pgfscope}%
\begin{pgfscope}%
\pgfsetbuttcap%
\pgfsetroundjoin%
\definecolor{currentfill}{rgb}{0.000000,0.000000,0.000000}%
\pgfsetfillcolor{currentfill}%
\pgfsetlinewidth{0.501875pt}%
\definecolor{currentstroke}{rgb}{0.000000,0.000000,0.000000}%
\pgfsetstrokecolor{currentstroke}%
\pgfsetdash{}{0pt}%
\pgfsys@defobject{currentmarker}{\pgfqpoint{0.000000in}{-0.020833in}}{\pgfqpoint{0.000000in}{0.000000in}}{%
\pgfpathmoveto{\pgfqpoint{0.000000in}{0.000000in}}%
\pgfpathlineto{\pgfqpoint{0.000000in}{-0.020833in}}%
\pgfusepath{stroke,fill}%
}%
\begin{pgfscope}%
\pgfsys@transformshift{5.865246in}{3.227753in}%
\pgfsys@useobject{currentmarker}{}%
\end{pgfscope}%
\end{pgfscope}%
\begin{pgfscope}%
\pgfpathrectangle{\pgfqpoint{0.481681in}{1.080890in}}{\pgfqpoint{5.785672in}{2.146863in}}%
\pgfusepath{clip}%
\pgfsetrectcap%
\pgfsetroundjoin%
\pgfsetlinewidth{0.100375pt}%
\definecolor{currentstroke}{rgb}{0.827451,0.827451,0.827451}%
\pgfsetstrokecolor{currentstroke}%
\pgfsetdash{}{0pt}%
\pgfpathmoveto{\pgfqpoint{5.900766in}{1.080890in}}%
\pgfpathlineto{\pgfqpoint{5.900766in}{3.227753in}}%
\pgfusepath{stroke}%
\end{pgfscope}%
\begin{pgfscope}%
\pgfsetbuttcap%
\pgfsetroundjoin%
\definecolor{currentfill}{rgb}{0.000000,0.000000,0.000000}%
\pgfsetfillcolor{currentfill}%
\pgfsetlinewidth{0.501875pt}%
\definecolor{currentstroke}{rgb}{0.000000,0.000000,0.000000}%
\pgfsetstrokecolor{currentstroke}%
\pgfsetdash{}{0pt}%
\pgfsys@defobject{currentmarker}{\pgfqpoint{0.000000in}{0.000000in}}{\pgfqpoint{0.000000in}{0.020833in}}{%
\pgfpathmoveto{\pgfqpoint{0.000000in}{0.000000in}}%
\pgfpathlineto{\pgfqpoint{0.000000in}{0.020833in}}%
\pgfusepath{stroke,fill}%
}%
\begin{pgfscope}%
\pgfsys@transformshift{5.900766in}{1.080890in}%
\pgfsys@useobject{currentmarker}{}%
\end{pgfscope}%
\end{pgfscope}%
\begin{pgfscope}%
\pgfsetbuttcap%
\pgfsetroundjoin%
\definecolor{currentfill}{rgb}{0.000000,0.000000,0.000000}%
\pgfsetfillcolor{currentfill}%
\pgfsetlinewidth{0.501875pt}%
\definecolor{currentstroke}{rgb}{0.000000,0.000000,0.000000}%
\pgfsetstrokecolor{currentstroke}%
\pgfsetdash{}{0pt}%
\pgfsys@defobject{currentmarker}{\pgfqpoint{0.000000in}{-0.020833in}}{\pgfqpoint{0.000000in}{0.000000in}}{%
\pgfpathmoveto{\pgfqpoint{0.000000in}{0.000000in}}%
\pgfpathlineto{\pgfqpoint{0.000000in}{-0.020833in}}%
\pgfusepath{stroke,fill}%
}%
\begin{pgfscope}%
\pgfsys@transformshift{5.900766in}{3.227753in}%
\pgfsys@useobject{currentmarker}{}%
\end{pgfscope}%
\end{pgfscope}%
\begin{pgfscope}%
\pgfpathrectangle{\pgfqpoint{0.481681in}{1.080890in}}{\pgfqpoint{5.785672in}{2.146863in}}%
\pgfusepath{clip}%
\pgfsetrectcap%
\pgfsetroundjoin%
\pgfsetlinewidth{0.100375pt}%
\definecolor{currentstroke}{rgb}{0.827451,0.827451,0.827451}%
\pgfsetstrokecolor{currentstroke}%
\pgfsetdash{}{0pt}%
\pgfpathmoveto{\pgfqpoint{5.936287in}{1.080890in}}%
\pgfpathlineto{\pgfqpoint{5.936287in}{3.227753in}}%
\pgfusepath{stroke}%
\end{pgfscope}%
\begin{pgfscope}%
\pgfsetbuttcap%
\pgfsetroundjoin%
\definecolor{currentfill}{rgb}{0.000000,0.000000,0.000000}%
\pgfsetfillcolor{currentfill}%
\pgfsetlinewidth{0.501875pt}%
\definecolor{currentstroke}{rgb}{0.000000,0.000000,0.000000}%
\pgfsetstrokecolor{currentstroke}%
\pgfsetdash{}{0pt}%
\pgfsys@defobject{currentmarker}{\pgfqpoint{0.000000in}{0.000000in}}{\pgfqpoint{0.000000in}{0.020833in}}{%
\pgfpathmoveto{\pgfqpoint{0.000000in}{0.000000in}}%
\pgfpathlineto{\pgfqpoint{0.000000in}{0.020833in}}%
\pgfusepath{stroke,fill}%
}%
\begin{pgfscope}%
\pgfsys@transformshift{5.936287in}{1.080890in}%
\pgfsys@useobject{currentmarker}{}%
\end{pgfscope}%
\end{pgfscope}%
\begin{pgfscope}%
\pgfsetbuttcap%
\pgfsetroundjoin%
\definecolor{currentfill}{rgb}{0.000000,0.000000,0.000000}%
\pgfsetfillcolor{currentfill}%
\pgfsetlinewidth{0.501875pt}%
\definecolor{currentstroke}{rgb}{0.000000,0.000000,0.000000}%
\pgfsetstrokecolor{currentstroke}%
\pgfsetdash{}{0pt}%
\pgfsys@defobject{currentmarker}{\pgfqpoint{0.000000in}{-0.020833in}}{\pgfqpoint{0.000000in}{0.000000in}}{%
\pgfpathmoveto{\pgfqpoint{0.000000in}{0.000000in}}%
\pgfpathlineto{\pgfqpoint{0.000000in}{-0.020833in}}%
\pgfusepath{stroke,fill}%
}%
\begin{pgfscope}%
\pgfsys@transformshift{5.936287in}{3.227753in}%
\pgfsys@useobject{currentmarker}{}%
\end{pgfscope}%
\end{pgfscope}%
\begin{pgfscope}%
\pgfpathrectangle{\pgfqpoint{0.481681in}{1.080890in}}{\pgfqpoint{5.785672in}{2.146863in}}%
\pgfusepath{clip}%
\pgfsetrectcap%
\pgfsetroundjoin%
\pgfsetlinewidth{0.100375pt}%
\definecolor{currentstroke}{rgb}{0.827451,0.827451,0.827451}%
\pgfsetstrokecolor{currentstroke}%
\pgfsetdash{}{0pt}%
\pgfpathmoveto{\pgfqpoint{5.971807in}{1.080890in}}%
\pgfpathlineto{\pgfqpoint{5.971807in}{3.227753in}}%
\pgfusepath{stroke}%
\end{pgfscope}%
\begin{pgfscope}%
\pgfsetbuttcap%
\pgfsetroundjoin%
\definecolor{currentfill}{rgb}{0.000000,0.000000,0.000000}%
\pgfsetfillcolor{currentfill}%
\pgfsetlinewidth{0.501875pt}%
\definecolor{currentstroke}{rgb}{0.000000,0.000000,0.000000}%
\pgfsetstrokecolor{currentstroke}%
\pgfsetdash{}{0pt}%
\pgfsys@defobject{currentmarker}{\pgfqpoint{0.000000in}{0.000000in}}{\pgfqpoint{0.000000in}{0.020833in}}{%
\pgfpathmoveto{\pgfqpoint{0.000000in}{0.000000in}}%
\pgfpathlineto{\pgfqpoint{0.000000in}{0.020833in}}%
\pgfusepath{stroke,fill}%
}%
\begin{pgfscope}%
\pgfsys@transformshift{5.971807in}{1.080890in}%
\pgfsys@useobject{currentmarker}{}%
\end{pgfscope}%
\end{pgfscope}%
\begin{pgfscope}%
\pgfsetbuttcap%
\pgfsetroundjoin%
\definecolor{currentfill}{rgb}{0.000000,0.000000,0.000000}%
\pgfsetfillcolor{currentfill}%
\pgfsetlinewidth{0.501875pt}%
\definecolor{currentstroke}{rgb}{0.000000,0.000000,0.000000}%
\pgfsetstrokecolor{currentstroke}%
\pgfsetdash{}{0pt}%
\pgfsys@defobject{currentmarker}{\pgfqpoint{0.000000in}{-0.020833in}}{\pgfqpoint{0.000000in}{0.000000in}}{%
\pgfpathmoveto{\pgfqpoint{0.000000in}{0.000000in}}%
\pgfpathlineto{\pgfqpoint{0.000000in}{-0.020833in}}%
\pgfusepath{stroke,fill}%
}%
\begin{pgfscope}%
\pgfsys@transformshift{5.971807in}{3.227753in}%
\pgfsys@useobject{currentmarker}{}%
\end{pgfscope}%
\end{pgfscope}%
\begin{pgfscope}%
\pgfpathrectangle{\pgfqpoint{0.481681in}{1.080890in}}{\pgfqpoint{5.785672in}{2.146863in}}%
\pgfusepath{clip}%
\pgfsetrectcap%
\pgfsetroundjoin%
\pgfsetlinewidth{0.100375pt}%
\definecolor{currentstroke}{rgb}{0.827451,0.827451,0.827451}%
\pgfsetstrokecolor{currentstroke}%
\pgfsetdash{}{0pt}%
\pgfpathmoveto{\pgfqpoint{6.007328in}{1.080890in}}%
\pgfpathlineto{\pgfqpoint{6.007328in}{3.227753in}}%
\pgfusepath{stroke}%
\end{pgfscope}%
\begin{pgfscope}%
\pgfsetbuttcap%
\pgfsetroundjoin%
\definecolor{currentfill}{rgb}{0.000000,0.000000,0.000000}%
\pgfsetfillcolor{currentfill}%
\pgfsetlinewidth{0.501875pt}%
\definecolor{currentstroke}{rgb}{0.000000,0.000000,0.000000}%
\pgfsetstrokecolor{currentstroke}%
\pgfsetdash{}{0pt}%
\pgfsys@defobject{currentmarker}{\pgfqpoint{0.000000in}{0.000000in}}{\pgfqpoint{0.000000in}{0.020833in}}{%
\pgfpathmoveto{\pgfqpoint{0.000000in}{0.000000in}}%
\pgfpathlineto{\pgfqpoint{0.000000in}{0.020833in}}%
\pgfusepath{stroke,fill}%
}%
\begin{pgfscope}%
\pgfsys@transformshift{6.007328in}{1.080890in}%
\pgfsys@useobject{currentmarker}{}%
\end{pgfscope}%
\end{pgfscope}%
\begin{pgfscope}%
\pgfsetbuttcap%
\pgfsetroundjoin%
\definecolor{currentfill}{rgb}{0.000000,0.000000,0.000000}%
\pgfsetfillcolor{currentfill}%
\pgfsetlinewidth{0.501875pt}%
\definecolor{currentstroke}{rgb}{0.000000,0.000000,0.000000}%
\pgfsetstrokecolor{currentstroke}%
\pgfsetdash{}{0pt}%
\pgfsys@defobject{currentmarker}{\pgfqpoint{0.000000in}{-0.020833in}}{\pgfqpoint{0.000000in}{0.000000in}}{%
\pgfpathmoveto{\pgfqpoint{0.000000in}{0.000000in}}%
\pgfpathlineto{\pgfqpoint{0.000000in}{-0.020833in}}%
\pgfusepath{stroke,fill}%
}%
\begin{pgfscope}%
\pgfsys@transformshift{6.007328in}{3.227753in}%
\pgfsys@useobject{currentmarker}{}%
\end{pgfscope}%
\end{pgfscope}%
\begin{pgfscope}%
\pgfpathrectangle{\pgfqpoint{0.481681in}{1.080890in}}{\pgfqpoint{5.785672in}{2.146863in}}%
\pgfusepath{clip}%
\pgfsetrectcap%
\pgfsetroundjoin%
\pgfsetlinewidth{0.100375pt}%
\definecolor{currentstroke}{rgb}{0.827451,0.827451,0.827451}%
\pgfsetstrokecolor{currentstroke}%
\pgfsetdash{}{0pt}%
\pgfpathmoveto{\pgfqpoint{6.042849in}{1.080890in}}%
\pgfpathlineto{\pgfqpoint{6.042849in}{3.227753in}}%
\pgfusepath{stroke}%
\end{pgfscope}%
\begin{pgfscope}%
\pgfsetbuttcap%
\pgfsetroundjoin%
\definecolor{currentfill}{rgb}{0.000000,0.000000,0.000000}%
\pgfsetfillcolor{currentfill}%
\pgfsetlinewidth{0.501875pt}%
\definecolor{currentstroke}{rgb}{0.000000,0.000000,0.000000}%
\pgfsetstrokecolor{currentstroke}%
\pgfsetdash{}{0pt}%
\pgfsys@defobject{currentmarker}{\pgfqpoint{0.000000in}{0.000000in}}{\pgfqpoint{0.000000in}{0.020833in}}{%
\pgfpathmoveto{\pgfqpoint{0.000000in}{0.000000in}}%
\pgfpathlineto{\pgfqpoint{0.000000in}{0.020833in}}%
\pgfusepath{stroke,fill}%
}%
\begin{pgfscope}%
\pgfsys@transformshift{6.042849in}{1.080890in}%
\pgfsys@useobject{currentmarker}{}%
\end{pgfscope}%
\end{pgfscope}%
\begin{pgfscope}%
\pgfsetbuttcap%
\pgfsetroundjoin%
\definecolor{currentfill}{rgb}{0.000000,0.000000,0.000000}%
\pgfsetfillcolor{currentfill}%
\pgfsetlinewidth{0.501875pt}%
\definecolor{currentstroke}{rgb}{0.000000,0.000000,0.000000}%
\pgfsetstrokecolor{currentstroke}%
\pgfsetdash{}{0pt}%
\pgfsys@defobject{currentmarker}{\pgfqpoint{0.000000in}{-0.020833in}}{\pgfqpoint{0.000000in}{0.000000in}}{%
\pgfpathmoveto{\pgfqpoint{0.000000in}{0.000000in}}%
\pgfpathlineto{\pgfqpoint{0.000000in}{-0.020833in}}%
\pgfusepath{stroke,fill}%
}%
\begin{pgfscope}%
\pgfsys@transformshift{6.042849in}{3.227753in}%
\pgfsys@useobject{currentmarker}{}%
\end{pgfscope}%
\end{pgfscope}%
\begin{pgfscope}%
\pgfpathrectangle{\pgfqpoint{0.481681in}{1.080890in}}{\pgfqpoint{5.785672in}{2.146863in}}%
\pgfusepath{clip}%
\pgfsetrectcap%
\pgfsetroundjoin%
\pgfsetlinewidth{0.100375pt}%
\definecolor{currentstroke}{rgb}{0.827451,0.827451,0.827451}%
\pgfsetstrokecolor{currentstroke}%
\pgfsetdash{}{0pt}%
\pgfpathmoveto{\pgfqpoint{6.078369in}{1.080890in}}%
\pgfpathlineto{\pgfqpoint{6.078369in}{3.227753in}}%
\pgfusepath{stroke}%
\end{pgfscope}%
\begin{pgfscope}%
\pgfsetbuttcap%
\pgfsetroundjoin%
\definecolor{currentfill}{rgb}{0.000000,0.000000,0.000000}%
\pgfsetfillcolor{currentfill}%
\pgfsetlinewidth{0.501875pt}%
\definecolor{currentstroke}{rgb}{0.000000,0.000000,0.000000}%
\pgfsetstrokecolor{currentstroke}%
\pgfsetdash{}{0pt}%
\pgfsys@defobject{currentmarker}{\pgfqpoint{0.000000in}{0.000000in}}{\pgfqpoint{0.000000in}{0.020833in}}{%
\pgfpathmoveto{\pgfqpoint{0.000000in}{0.000000in}}%
\pgfpathlineto{\pgfqpoint{0.000000in}{0.020833in}}%
\pgfusepath{stroke,fill}%
}%
\begin{pgfscope}%
\pgfsys@transformshift{6.078369in}{1.080890in}%
\pgfsys@useobject{currentmarker}{}%
\end{pgfscope}%
\end{pgfscope}%
\begin{pgfscope}%
\pgfsetbuttcap%
\pgfsetroundjoin%
\definecolor{currentfill}{rgb}{0.000000,0.000000,0.000000}%
\pgfsetfillcolor{currentfill}%
\pgfsetlinewidth{0.501875pt}%
\definecolor{currentstroke}{rgb}{0.000000,0.000000,0.000000}%
\pgfsetstrokecolor{currentstroke}%
\pgfsetdash{}{0pt}%
\pgfsys@defobject{currentmarker}{\pgfqpoint{0.000000in}{-0.020833in}}{\pgfqpoint{0.000000in}{0.000000in}}{%
\pgfpathmoveto{\pgfqpoint{0.000000in}{0.000000in}}%
\pgfpathlineto{\pgfqpoint{0.000000in}{-0.020833in}}%
\pgfusepath{stroke,fill}%
}%
\begin{pgfscope}%
\pgfsys@transformshift{6.078369in}{3.227753in}%
\pgfsys@useobject{currentmarker}{}%
\end{pgfscope}%
\end{pgfscope}%
\begin{pgfscope}%
\pgfpathrectangle{\pgfqpoint{0.481681in}{1.080890in}}{\pgfqpoint{5.785672in}{2.146863in}}%
\pgfusepath{clip}%
\pgfsetrectcap%
\pgfsetroundjoin%
\pgfsetlinewidth{0.100375pt}%
\definecolor{currentstroke}{rgb}{0.827451,0.827451,0.827451}%
\pgfsetstrokecolor{currentstroke}%
\pgfsetdash{}{0pt}%
\pgfpathmoveto{\pgfqpoint{6.113890in}{1.080890in}}%
\pgfpathlineto{\pgfqpoint{6.113890in}{3.227753in}}%
\pgfusepath{stroke}%
\end{pgfscope}%
\begin{pgfscope}%
\pgfsetbuttcap%
\pgfsetroundjoin%
\definecolor{currentfill}{rgb}{0.000000,0.000000,0.000000}%
\pgfsetfillcolor{currentfill}%
\pgfsetlinewidth{0.501875pt}%
\definecolor{currentstroke}{rgb}{0.000000,0.000000,0.000000}%
\pgfsetstrokecolor{currentstroke}%
\pgfsetdash{}{0pt}%
\pgfsys@defobject{currentmarker}{\pgfqpoint{0.000000in}{0.000000in}}{\pgfqpoint{0.000000in}{0.020833in}}{%
\pgfpathmoveto{\pgfqpoint{0.000000in}{0.000000in}}%
\pgfpathlineto{\pgfqpoint{0.000000in}{0.020833in}}%
\pgfusepath{stroke,fill}%
}%
\begin{pgfscope}%
\pgfsys@transformshift{6.113890in}{1.080890in}%
\pgfsys@useobject{currentmarker}{}%
\end{pgfscope}%
\end{pgfscope}%
\begin{pgfscope}%
\pgfsetbuttcap%
\pgfsetroundjoin%
\definecolor{currentfill}{rgb}{0.000000,0.000000,0.000000}%
\pgfsetfillcolor{currentfill}%
\pgfsetlinewidth{0.501875pt}%
\definecolor{currentstroke}{rgb}{0.000000,0.000000,0.000000}%
\pgfsetstrokecolor{currentstroke}%
\pgfsetdash{}{0pt}%
\pgfsys@defobject{currentmarker}{\pgfqpoint{0.000000in}{-0.020833in}}{\pgfqpoint{0.000000in}{0.000000in}}{%
\pgfpathmoveto{\pgfqpoint{0.000000in}{0.000000in}}%
\pgfpathlineto{\pgfqpoint{0.000000in}{-0.020833in}}%
\pgfusepath{stroke,fill}%
}%
\begin{pgfscope}%
\pgfsys@transformshift{6.113890in}{3.227753in}%
\pgfsys@useobject{currentmarker}{}%
\end{pgfscope}%
\end{pgfscope}%
\begin{pgfscope}%
\pgfpathrectangle{\pgfqpoint{0.481681in}{1.080890in}}{\pgfqpoint{5.785672in}{2.146863in}}%
\pgfusepath{clip}%
\pgfsetrectcap%
\pgfsetroundjoin%
\pgfsetlinewidth{0.100375pt}%
\definecolor{currentstroke}{rgb}{0.827451,0.827451,0.827451}%
\pgfsetstrokecolor{currentstroke}%
\pgfsetdash{}{0pt}%
\pgfpathmoveto{\pgfqpoint{6.149410in}{1.080890in}}%
\pgfpathlineto{\pgfqpoint{6.149410in}{3.227753in}}%
\pgfusepath{stroke}%
\end{pgfscope}%
\begin{pgfscope}%
\pgfsetbuttcap%
\pgfsetroundjoin%
\definecolor{currentfill}{rgb}{0.000000,0.000000,0.000000}%
\pgfsetfillcolor{currentfill}%
\pgfsetlinewidth{0.501875pt}%
\definecolor{currentstroke}{rgb}{0.000000,0.000000,0.000000}%
\pgfsetstrokecolor{currentstroke}%
\pgfsetdash{}{0pt}%
\pgfsys@defobject{currentmarker}{\pgfqpoint{0.000000in}{0.000000in}}{\pgfqpoint{0.000000in}{0.020833in}}{%
\pgfpathmoveto{\pgfqpoint{0.000000in}{0.000000in}}%
\pgfpathlineto{\pgfqpoint{0.000000in}{0.020833in}}%
\pgfusepath{stroke,fill}%
}%
\begin{pgfscope}%
\pgfsys@transformshift{6.149410in}{1.080890in}%
\pgfsys@useobject{currentmarker}{}%
\end{pgfscope}%
\end{pgfscope}%
\begin{pgfscope}%
\pgfsetbuttcap%
\pgfsetroundjoin%
\definecolor{currentfill}{rgb}{0.000000,0.000000,0.000000}%
\pgfsetfillcolor{currentfill}%
\pgfsetlinewidth{0.501875pt}%
\definecolor{currentstroke}{rgb}{0.000000,0.000000,0.000000}%
\pgfsetstrokecolor{currentstroke}%
\pgfsetdash{}{0pt}%
\pgfsys@defobject{currentmarker}{\pgfqpoint{0.000000in}{-0.020833in}}{\pgfqpoint{0.000000in}{0.000000in}}{%
\pgfpathmoveto{\pgfqpoint{0.000000in}{0.000000in}}%
\pgfpathlineto{\pgfqpoint{0.000000in}{-0.020833in}}%
\pgfusepath{stroke,fill}%
}%
\begin{pgfscope}%
\pgfsys@transformshift{6.149410in}{3.227753in}%
\pgfsys@useobject{currentmarker}{}%
\end{pgfscope}%
\end{pgfscope}%
\begin{pgfscope}%
\pgfpathrectangle{\pgfqpoint{0.481681in}{1.080890in}}{\pgfqpoint{5.785672in}{2.146863in}}%
\pgfusepath{clip}%
\pgfsetrectcap%
\pgfsetroundjoin%
\pgfsetlinewidth{0.100375pt}%
\definecolor{currentstroke}{rgb}{0.827451,0.827451,0.827451}%
\pgfsetstrokecolor{currentstroke}%
\pgfsetdash{}{0pt}%
\pgfpathmoveto{\pgfqpoint{6.220451in}{1.080890in}}%
\pgfpathlineto{\pgfqpoint{6.220451in}{3.227753in}}%
\pgfusepath{stroke}%
\end{pgfscope}%
\begin{pgfscope}%
\pgfsetbuttcap%
\pgfsetroundjoin%
\definecolor{currentfill}{rgb}{0.000000,0.000000,0.000000}%
\pgfsetfillcolor{currentfill}%
\pgfsetlinewidth{0.501875pt}%
\definecolor{currentstroke}{rgb}{0.000000,0.000000,0.000000}%
\pgfsetstrokecolor{currentstroke}%
\pgfsetdash{}{0pt}%
\pgfsys@defobject{currentmarker}{\pgfqpoint{0.000000in}{0.000000in}}{\pgfqpoint{0.000000in}{0.020833in}}{%
\pgfpathmoveto{\pgfqpoint{0.000000in}{0.000000in}}%
\pgfpathlineto{\pgfqpoint{0.000000in}{0.020833in}}%
\pgfusepath{stroke,fill}%
}%
\begin{pgfscope}%
\pgfsys@transformshift{6.220451in}{1.080890in}%
\pgfsys@useobject{currentmarker}{}%
\end{pgfscope}%
\end{pgfscope}%
\begin{pgfscope}%
\pgfsetbuttcap%
\pgfsetroundjoin%
\definecolor{currentfill}{rgb}{0.000000,0.000000,0.000000}%
\pgfsetfillcolor{currentfill}%
\pgfsetlinewidth{0.501875pt}%
\definecolor{currentstroke}{rgb}{0.000000,0.000000,0.000000}%
\pgfsetstrokecolor{currentstroke}%
\pgfsetdash{}{0pt}%
\pgfsys@defobject{currentmarker}{\pgfqpoint{0.000000in}{-0.020833in}}{\pgfqpoint{0.000000in}{0.000000in}}{%
\pgfpathmoveto{\pgfqpoint{0.000000in}{0.000000in}}%
\pgfpathlineto{\pgfqpoint{0.000000in}{-0.020833in}}%
\pgfusepath{stroke,fill}%
}%
\begin{pgfscope}%
\pgfsys@transformshift{6.220451in}{3.227753in}%
\pgfsys@useobject{currentmarker}{}%
\end{pgfscope}%
\end{pgfscope}%
\begin{pgfscope}%
\pgfpathrectangle{\pgfqpoint{0.481681in}{1.080890in}}{\pgfqpoint{5.785672in}{2.146863in}}%
\pgfusepath{clip}%
\pgfsetrectcap%
\pgfsetroundjoin%
\pgfsetlinewidth{0.100375pt}%
\definecolor{currentstroke}{rgb}{0.827451,0.827451,0.827451}%
\pgfsetstrokecolor{currentstroke}%
\pgfsetdash{}{0pt}%
\pgfpathmoveto{\pgfqpoint{6.255972in}{1.080890in}}%
\pgfpathlineto{\pgfqpoint{6.255972in}{3.227753in}}%
\pgfusepath{stroke}%
\end{pgfscope}%
\begin{pgfscope}%
\pgfsetbuttcap%
\pgfsetroundjoin%
\definecolor{currentfill}{rgb}{0.000000,0.000000,0.000000}%
\pgfsetfillcolor{currentfill}%
\pgfsetlinewidth{0.501875pt}%
\definecolor{currentstroke}{rgb}{0.000000,0.000000,0.000000}%
\pgfsetstrokecolor{currentstroke}%
\pgfsetdash{}{0pt}%
\pgfsys@defobject{currentmarker}{\pgfqpoint{0.000000in}{0.000000in}}{\pgfqpoint{0.000000in}{0.020833in}}{%
\pgfpathmoveto{\pgfqpoint{0.000000in}{0.000000in}}%
\pgfpathlineto{\pgfqpoint{0.000000in}{0.020833in}}%
\pgfusepath{stroke,fill}%
}%
\begin{pgfscope}%
\pgfsys@transformshift{6.255972in}{1.080890in}%
\pgfsys@useobject{currentmarker}{}%
\end{pgfscope}%
\end{pgfscope}%
\begin{pgfscope}%
\pgfsetbuttcap%
\pgfsetroundjoin%
\definecolor{currentfill}{rgb}{0.000000,0.000000,0.000000}%
\pgfsetfillcolor{currentfill}%
\pgfsetlinewidth{0.501875pt}%
\definecolor{currentstroke}{rgb}{0.000000,0.000000,0.000000}%
\pgfsetstrokecolor{currentstroke}%
\pgfsetdash{}{0pt}%
\pgfsys@defobject{currentmarker}{\pgfqpoint{0.000000in}{-0.020833in}}{\pgfqpoint{0.000000in}{0.000000in}}{%
\pgfpathmoveto{\pgfqpoint{0.000000in}{0.000000in}}%
\pgfpathlineto{\pgfqpoint{0.000000in}{-0.020833in}}%
\pgfusepath{stroke,fill}%
}%
\begin{pgfscope}%
\pgfsys@transformshift{6.255972in}{3.227753in}%
\pgfsys@useobject{currentmarker}{}%
\end{pgfscope}%
\end{pgfscope}%
\begin{pgfscope}%
\pgfsetbuttcap%
\pgfsetroundjoin%
\definecolor{currentfill}{rgb}{0.000000,0.000000,0.000000}%
\pgfsetfillcolor{currentfill}%
\pgfsetlinewidth{0.501875pt}%
\definecolor{currentstroke}{rgb}{0.000000,0.000000,0.000000}%
\pgfsetstrokecolor{currentstroke}%
\pgfsetdash{}{0pt}%
\pgfsys@defobject{currentmarker}{\pgfqpoint{0.000000in}{0.000000in}}{\pgfqpoint{0.041667in}{0.000000in}}{%
\pgfpathmoveto{\pgfqpoint{0.000000in}{0.000000in}}%
\pgfpathlineto{\pgfqpoint{0.041667in}{0.000000in}}%
\pgfusepath{stroke,fill}%
}%
\begin{pgfscope}%
\pgfsys@transformshift{0.481681in}{1.178475in}%
\pgfsys@useobject{currentmarker}{}%
\end{pgfscope}%
\end{pgfscope}%
\begin{pgfscope}%
\pgfsetbuttcap%
\pgfsetroundjoin%
\definecolor{currentfill}{rgb}{0.000000,0.000000,0.000000}%
\pgfsetfillcolor{currentfill}%
\pgfsetlinewidth{0.501875pt}%
\definecolor{currentstroke}{rgb}{0.000000,0.000000,0.000000}%
\pgfsetstrokecolor{currentstroke}%
\pgfsetdash{}{0pt}%
\pgfsys@defobject{currentmarker}{\pgfqpoint{-0.041667in}{0.000000in}}{\pgfqpoint{-0.000000in}{0.000000in}}{%
\pgfpathmoveto{\pgfqpoint{-0.000000in}{0.000000in}}%
\pgfpathlineto{\pgfqpoint{-0.041667in}{0.000000in}}%
\pgfusepath{stroke,fill}%
}%
\begin{pgfscope}%
\pgfsys@transformshift{6.267353in}{1.178475in}%
\pgfsys@useobject{currentmarker}{}%
\end{pgfscope}%
\end{pgfscope}%
\begin{pgfscope}%
\definecolor{textcolor}{rgb}{0.000000,0.000000,0.000000}%
\pgfsetstrokecolor{textcolor}%
\pgfsetfillcolor{textcolor}%
\pgftext[x=0.291028in, y=1.144739in, left, base]{\color{textcolor}\rmfamily\fontsize{7.000000}{8.400000}\selectfont 0.0}%
\end{pgfscope}%
\begin{pgfscope}%
\pgfsetbuttcap%
\pgfsetroundjoin%
\definecolor{currentfill}{rgb}{0.000000,0.000000,0.000000}%
\pgfsetfillcolor{currentfill}%
\pgfsetlinewidth{0.501875pt}%
\definecolor{currentstroke}{rgb}{0.000000,0.000000,0.000000}%
\pgfsetstrokecolor{currentstroke}%
\pgfsetdash{}{0pt}%
\pgfsys@defobject{currentmarker}{\pgfqpoint{0.000000in}{0.000000in}}{\pgfqpoint{0.041667in}{0.000000in}}{%
\pgfpathmoveto{\pgfqpoint{0.000000in}{0.000000in}}%
\pgfpathlineto{\pgfqpoint{0.041667in}{0.000000in}}%
\pgfusepath{stroke,fill}%
}%
\begin{pgfscope}%
\pgfsys@transformshift{0.481681in}{1.506703in}%
\pgfsys@useobject{currentmarker}{}%
\end{pgfscope}%
\end{pgfscope}%
\begin{pgfscope}%
\pgfsetbuttcap%
\pgfsetroundjoin%
\definecolor{currentfill}{rgb}{0.000000,0.000000,0.000000}%
\pgfsetfillcolor{currentfill}%
\pgfsetlinewidth{0.501875pt}%
\definecolor{currentstroke}{rgb}{0.000000,0.000000,0.000000}%
\pgfsetstrokecolor{currentstroke}%
\pgfsetdash{}{0pt}%
\pgfsys@defobject{currentmarker}{\pgfqpoint{-0.041667in}{0.000000in}}{\pgfqpoint{-0.000000in}{0.000000in}}{%
\pgfpathmoveto{\pgfqpoint{-0.000000in}{0.000000in}}%
\pgfpathlineto{\pgfqpoint{-0.041667in}{0.000000in}}%
\pgfusepath{stroke,fill}%
}%
\begin{pgfscope}%
\pgfsys@transformshift{6.267353in}{1.506703in}%
\pgfsys@useobject{currentmarker}{}%
\end{pgfscope}%
\end{pgfscope}%
\begin{pgfscope}%
\definecolor{textcolor}{rgb}{0.000000,0.000000,0.000000}%
\pgfsetstrokecolor{textcolor}%
\pgfsetfillcolor{textcolor}%
\pgftext[x=0.291028in, y=1.472967in, left, base]{\color{textcolor}\rmfamily\fontsize{7.000000}{8.400000}\selectfont 0.2}%
\end{pgfscope}%
\begin{pgfscope}%
\pgfsetbuttcap%
\pgfsetroundjoin%
\definecolor{currentfill}{rgb}{0.000000,0.000000,0.000000}%
\pgfsetfillcolor{currentfill}%
\pgfsetlinewidth{0.501875pt}%
\definecolor{currentstroke}{rgb}{0.000000,0.000000,0.000000}%
\pgfsetstrokecolor{currentstroke}%
\pgfsetdash{}{0pt}%
\pgfsys@defobject{currentmarker}{\pgfqpoint{0.000000in}{0.000000in}}{\pgfqpoint{0.041667in}{0.000000in}}{%
\pgfpathmoveto{\pgfqpoint{0.000000in}{0.000000in}}%
\pgfpathlineto{\pgfqpoint{0.041667in}{0.000000in}}%
\pgfusepath{stroke,fill}%
}%
\begin{pgfscope}%
\pgfsys@transformshift{0.481681in}{1.834931in}%
\pgfsys@useobject{currentmarker}{}%
\end{pgfscope}%
\end{pgfscope}%
\begin{pgfscope}%
\pgfsetbuttcap%
\pgfsetroundjoin%
\definecolor{currentfill}{rgb}{0.000000,0.000000,0.000000}%
\pgfsetfillcolor{currentfill}%
\pgfsetlinewidth{0.501875pt}%
\definecolor{currentstroke}{rgb}{0.000000,0.000000,0.000000}%
\pgfsetstrokecolor{currentstroke}%
\pgfsetdash{}{0pt}%
\pgfsys@defobject{currentmarker}{\pgfqpoint{-0.041667in}{0.000000in}}{\pgfqpoint{-0.000000in}{0.000000in}}{%
\pgfpathmoveto{\pgfqpoint{-0.000000in}{0.000000in}}%
\pgfpathlineto{\pgfqpoint{-0.041667in}{0.000000in}}%
\pgfusepath{stroke,fill}%
}%
\begin{pgfscope}%
\pgfsys@transformshift{6.267353in}{1.834931in}%
\pgfsys@useobject{currentmarker}{}%
\end{pgfscope}%
\end{pgfscope}%
\begin{pgfscope}%
\definecolor{textcolor}{rgb}{0.000000,0.000000,0.000000}%
\pgfsetstrokecolor{textcolor}%
\pgfsetfillcolor{textcolor}%
\pgftext[x=0.291028in, y=1.801195in, left, base]{\color{textcolor}\rmfamily\fontsize{7.000000}{8.400000}\selectfont 0.4}%
\end{pgfscope}%
\begin{pgfscope}%
\pgfsetbuttcap%
\pgfsetroundjoin%
\definecolor{currentfill}{rgb}{0.000000,0.000000,0.000000}%
\pgfsetfillcolor{currentfill}%
\pgfsetlinewidth{0.501875pt}%
\definecolor{currentstroke}{rgb}{0.000000,0.000000,0.000000}%
\pgfsetstrokecolor{currentstroke}%
\pgfsetdash{}{0pt}%
\pgfsys@defobject{currentmarker}{\pgfqpoint{0.000000in}{0.000000in}}{\pgfqpoint{0.041667in}{0.000000in}}{%
\pgfpathmoveto{\pgfqpoint{0.000000in}{0.000000in}}%
\pgfpathlineto{\pgfqpoint{0.041667in}{0.000000in}}%
\pgfusepath{stroke,fill}%
}%
\begin{pgfscope}%
\pgfsys@transformshift{0.481681in}{2.163159in}%
\pgfsys@useobject{currentmarker}{}%
\end{pgfscope}%
\end{pgfscope}%
\begin{pgfscope}%
\pgfsetbuttcap%
\pgfsetroundjoin%
\definecolor{currentfill}{rgb}{0.000000,0.000000,0.000000}%
\pgfsetfillcolor{currentfill}%
\pgfsetlinewidth{0.501875pt}%
\definecolor{currentstroke}{rgb}{0.000000,0.000000,0.000000}%
\pgfsetstrokecolor{currentstroke}%
\pgfsetdash{}{0pt}%
\pgfsys@defobject{currentmarker}{\pgfqpoint{-0.041667in}{0.000000in}}{\pgfqpoint{-0.000000in}{0.000000in}}{%
\pgfpathmoveto{\pgfqpoint{-0.000000in}{0.000000in}}%
\pgfpathlineto{\pgfqpoint{-0.041667in}{0.000000in}}%
\pgfusepath{stroke,fill}%
}%
\begin{pgfscope}%
\pgfsys@transformshift{6.267353in}{2.163159in}%
\pgfsys@useobject{currentmarker}{}%
\end{pgfscope}%
\end{pgfscope}%
\begin{pgfscope}%
\definecolor{textcolor}{rgb}{0.000000,0.000000,0.000000}%
\pgfsetstrokecolor{textcolor}%
\pgfsetfillcolor{textcolor}%
\pgftext[x=0.291028in, y=2.129423in, left, base]{\color{textcolor}\rmfamily\fontsize{7.000000}{8.400000}\selectfont 0.6}%
\end{pgfscope}%
\begin{pgfscope}%
\pgfsetbuttcap%
\pgfsetroundjoin%
\definecolor{currentfill}{rgb}{0.000000,0.000000,0.000000}%
\pgfsetfillcolor{currentfill}%
\pgfsetlinewidth{0.501875pt}%
\definecolor{currentstroke}{rgb}{0.000000,0.000000,0.000000}%
\pgfsetstrokecolor{currentstroke}%
\pgfsetdash{}{0pt}%
\pgfsys@defobject{currentmarker}{\pgfqpoint{0.000000in}{0.000000in}}{\pgfqpoint{0.041667in}{0.000000in}}{%
\pgfpathmoveto{\pgfqpoint{0.000000in}{0.000000in}}%
\pgfpathlineto{\pgfqpoint{0.041667in}{0.000000in}}%
\pgfusepath{stroke,fill}%
}%
\begin{pgfscope}%
\pgfsys@transformshift{0.481681in}{2.491386in}%
\pgfsys@useobject{currentmarker}{}%
\end{pgfscope}%
\end{pgfscope}%
\begin{pgfscope}%
\pgfsetbuttcap%
\pgfsetroundjoin%
\definecolor{currentfill}{rgb}{0.000000,0.000000,0.000000}%
\pgfsetfillcolor{currentfill}%
\pgfsetlinewidth{0.501875pt}%
\definecolor{currentstroke}{rgb}{0.000000,0.000000,0.000000}%
\pgfsetstrokecolor{currentstroke}%
\pgfsetdash{}{0pt}%
\pgfsys@defobject{currentmarker}{\pgfqpoint{-0.041667in}{0.000000in}}{\pgfqpoint{-0.000000in}{0.000000in}}{%
\pgfpathmoveto{\pgfqpoint{-0.000000in}{0.000000in}}%
\pgfpathlineto{\pgfqpoint{-0.041667in}{0.000000in}}%
\pgfusepath{stroke,fill}%
}%
\begin{pgfscope}%
\pgfsys@transformshift{6.267353in}{2.491386in}%
\pgfsys@useobject{currentmarker}{}%
\end{pgfscope}%
\end{pgfscope}%
\begin{pgfscope}%
\definecolor{textcolor}{rgb}{0.000000,0.000000,0.000000}%
\pgfsetstrokecolor{textcolor}%
\pgfsetfillcolor{textcolor}%
\pgftext[x=0.291028in, y=2.457650in, left, base]{\color{textcolor}\rmfamily\fontsize{7.000000}{8.400000}\selectfont 0.8}%
\end{pgfscope}%
\begin{pgfscope}%
\pgfsetbuttcap%
\pgfsetroundjoin%
\definecolor{currentfill}{rgb}{0.000000,0.000000,0.000000}%
\pgfsetfillcolor{currentfill}%
\pgfsetlinewidth{0.501875pt}%
\definecolor{currentstroke}{rgb}{0.000000,0.000000,0.000000}%
\pgfsetstrokecolor{currentstroke}%
\pgfsetdash{}{0pt}%
\pgfsys@defobject{currentmarker}{\pgfqpoint{0.000000in}{0.000000in}}{\pgfqpoint{0.041667in}{0.000000in}}{%
\pgfpathmoveto{\pgfqpoint{0.000000in}{0.000000in}}%
\pgfpathlineto{\pgfqpoint{0.041667in}{0.000000in}}%
\pgfusepath{stroke,fill}%
}%
\begin{pgfscope}%
\pgfsys@transformshift{0.481681in}{2.819614in}%
\pgfsys@useobject{currentmarker}{}%
\end{pgfscope}%
\end{pgfscope}%
\begin{pgfscope}%
\pgfsetbuttcap%
\pgfsetroundjoin%
\definecolor{currentfill}{rgb}{0.000000,0.000000,0.000000}%
\pgfsetfillcolor{currentfill}%
\pgfsetlinewidth{0.501875pt}%
\definecolor{currentstroke}{rgb}{0.000000,0.000000,0.000000}%
\pgfsetstrokecolor{currentstroke}%
\pgfsetdash{}{0pt}%
\pgfsys@defobject{currentmarker}{\pgfqpoint{-0.041667in}{0.000000in}}{\pgfqpoint{-0.000000in}{0.000000in}}{%
\pgfpathmoveto{\pgfqpoint{-0.000000in}{0.000000in}}%
\pgfpathlineto{\pgfqpoint{-0.041667in}{0.000000in}}%
\pgfusepath{stroke,fill}%
}%
\begin{pgfscope}%
\pgfsys@transformshift{6.267353in}{2.819614in}%
\pgfsys@useobject{currentmarker}{}%
\end{pgfscope}%
\end{pgfscope}%
\begin{pgfscope}%
\definecolor{textcolor}{rgb}{0.000000,0.000000,0.000000}%
\pgfsetstrokecolor{textcolor}%
\pgfsetfillcolor{textcolor}%
\pgftext[x=0.291028in, y=2.785878in, left, base]{\color{textcolor}\rmfamily\fontsize{7.000000}{8.400000}\selectfont 1.0}%
\end{pgfscope}%
\begin{pgfscope}%
\pgfsetbuttcap%
\pgfsetroundjoin%
\definecolor{currentfill}{rgb}{0.000000,0.000000,0.000000}%
\pgfsetfillcolor{currentfill}%
\pgfsetlinewidth{0.501875pt}%
\definecolor{currentstroke}{rgb}{0.000000,0.000000,0.000000}%
\pgfsetstrokecolor{currentstroke}%
\pgfsetdash{}{0pt}%
\pgfsys@defobject{currentmarker}{\pgfqpoint{0.000000in}{0.000000in}}{\pgfqpoint{0.041667in}{0.000000in}}{%
\pgfpathmoveto{\pgfqpoint{0.000000in}{0.000000in}}%
\pgfpathlineto{\pgfqpoint{0.041667in}{0.000000in}}%
\pgfusepath{stroke,fill}%
}%
\begin{pgfscope}%
\pgfsys@transformshift{0.481681in}{3.147842in}%
\pgfsys@useobject{currentmarker}{}%
\end{pgfscope}%
\end{pgfscope}%
\begin{pgfscope}%
\pgfsetbuttcap%
\pgfsetroundjoin%
\definecolor{currentfill}{rgb}{0.000000,0.000000,0.000000}%
\pgfsetfillcolor{currentfill}%
\pgfsetlinewidth{0.501875pt}%
\definecolor{currentstroke}{rgb}{0.000000,0.000000,0.000000}%
\pgfsetstrokecolor{currentstroke}%
\pgfsetdash{}{0pt}%
\pgfsys@defobject{currentmarker}{\pgfqpoint{-0.041667in}{0.000000in}}{\pgfqpoint{-0.000000in}{0.000000in}}{%
\pgfpathmoveto{\pgfqpoint{-0.000000in}{0.000000in}}%
\pgfpathlineto{\pgfqpoint{-0.041667in}{0.000000in}}%
\pgfusepath{stroke,fill}%
}%
\begin{pgfscope}%
\pgfsys@transformshift{6.267353in}{3.147842in}%
\pgfsys@useobject{currentmarker}{}%
\end{pgfscope}%
\end{pgfscope}%
\begin{pgfscope}%
\definecolor{textcolor}{rgb}{0.000000,0.000000,0.000000}%
\pgfsetstrokecolor{textcolor}%
\pgfsetfillcolor{textcolor}%
\pgftext[x=0.291028in, y=3.114106in, left, base]{\color{textcolor}\rmfamily\fontsize{7.000000}{8.400000}\selectfont 1.2}%
\end{pgfscope}%
\begin{pgfscope}%
\pgfsetbuttcap%
\pgfsetroundjoin%
\definecolor{currentfill}{rgb}{0.000000,0.000000,0.000000}%
\pgfsetfillcolor{currentfill}%
\pgfsetlinewidth{0.501875pt}%
\definecolor{currentstroke}{rgb}{0.000000,0.000000,0.000000}%
\pgfsetstrokecolor{currentstroke}%
\pgfsetdash{}{0pt}%
\pgfsys@defobject{currentmarker}{\pgfqpoint{0.000000in}{0.000000in}}{\pgfqpoint{0.020833in}{0.000000in}}{%
\pgfpathmoveto{\pgfqpoint{0.000000in}{0.000000in}}%
\pgfpathlineto{\pgfqpoint{0.020833in}{0.000000in}}%
\pgfusepath{stroke,fill}%
}%
\begin{pgfscope}%
\pgfsys@transformshift{0.481681in}{1.096418in}%
\pgfsys@useobject{currentmarker}{}%
\end{pgfscope}%
\end{pgfscope}%
\begin{pgfscope}%
\pgfsetbuttcap%
\pgfsetroundjoin%
\definecolor{currentfill}{rgb}{0.000000,0.000000,0.000000}%
\pgfsetfillcolor{currentfill}%
\pgfsetlinewidth{0.501875pt}%
\definecolor{currentstroke}{rgb}{0.000000,0.000000,0.000000}%
\pgfsetstrokecolor{currentstroke}%
\pgfsetdash{}{0pt}%
\pgfsys@defobject{currentmarker}{\pgfqpoint{-0.020833in}{0.000000in}}{\pgfqpoint{-0.000000in}{0.000000in}}{%
\pgfpathmoveto{\pgfqpoint{-0.000000in}{0.000000in}}%
\pgfpathlineto{\pgfqpoint{-0.020833in}{0.000000in}}%
\pgfusepath{stroke,fill}%
}%
\begin{pgfscope}%
\pgfsys@transformshift{6.267353in}{1.096418in}%
\pgfsys@useobject{currentmarker}{}%
\end{pgfscope}%
\end{pgfscope}%
\begin{pgfscope}%
\pgfsetbuttcap%
\pgfsetroundjoin%
\definecolor{currentfill}{rgb}{0.000000,0.000000,0.000000}%
\pgfsetfillcolor{currentfill}%
\pgfsetlinewidth{0.501875pt}%
\definecolor{currentstroke}{rgb}{0.000000,0.000000,0.000000}%
\pgfsetstrokecolor{currentstroke}%
\pgfsetdash{}{0pt}%
\pgfsys@defobject{currentmarker}{\pgfqpoint{0.000000in}{0.000000in}}{\pgfqpoint{0.020833in}{0.000000in}}{%
\pgfpathmoveto{\pgfqpoint{0.000000in}{0.000000in}}%
\pgfpathlineto{\pgfqpoint{0.020833in}{0.000000in}}%
\pgfusepath{stroke,fill}%
}%
\begin{pgfscope}%
\pgfsys@transformshift{0.481681in}{1.260532in}%
\pgfsys@useobject{currentmarker}{}%
\end{pgfscope}%
\end{pgfscope}%
\begin{pgfscope}%
\pgfsetbuttcap%
\pgfsetroundjoin%
\definecolor{currentfill}{rgb}{0.000000,0.000000,0.000000}%
\pgfsetfillcolor{currentfill}%
\pgfsetlinewidth{0.501875pt}%
\definecolor{currentstroke}{rgb}{0.000000,0.000000,0.000000}%
\pgfsetstrokecolor{currentstroke}%
\pgfsetdash{}{0pt}%
\pgfsys@defobject{currentmarker}{\pgfqpoint{-0.020833in}{0.000000in}}{\pgfqpoint{-0.000000in}{0.000000in}}{%
\pgfpathmoveto{\pgfqpoint{-0.000000in}{0.000000in}}%
\pgfpathlineto{\pgfqpoint{-0.020833in}{0.000000in}}%
\pgfusepath{stroke,fill}%
}%
\begin{pgfscope}%
\pgfsys@transformshift{6.267353in}{1.260532in}%
\pgfsys@useobject{currentmarker}{}%
\end{pgfscope}%
\end{pgfscope}%
\begin{pgfscope}%
\pgfsetbuttcap%
\pgfsetroundjoin%
\definecolor{currentfill}{rgb}{0.000000,0.000000,0.000000}%
\pgfsetfillcolor{currentfill}%
\pgfsetlinewidth{0.501875pt}%
\definecolor{currentstroke}{rgb}{0.000000,0.000000,0.000000}%
\pgfsetstrokecolor{currentstroke}%
\pgfsetdash{}{0pt}%
\pgfsys@defobject{currentmarker}{\pgfqpoint{0.000000in}{0.000000in}}{\pgfqpoint{0.020833in}{0.000000in}}{%
\pgfpathmoveto{\pgfqpoint{0.000000in}{0.000000in}}%
\pgfpathlineto{\pgfqpoint{0.020833in}{0.000000in}}%
\pgfusepath{stroke,fill}%
}%
\begin{pgfscope}%
\pgfsys@transformshift{0.481681in}{1.342589in}%
\pgfsys@useobject{currentmarker}{}%
\end{pgfscope}%
\end{pgfscope}%
\begin{pgfscope}%
\pgfsetbuttcap%
\pgfsetroundjoin%
\definecolor{currentfill}{rgb}{0.000000,0.000000,0.000000}%
\pgfsetfillcolor{currentfill}%
\pgfsetlinewidth{0.501875pt}%
\definecolor{currentstroke}{rgb}{0.000000,0.000000,0.000000}%
\pgfsetstrokecolor{currentstroke}%
\pgfsetdash{}{0pt}%
\pgfsys@defobject{currentmarker}{\pgfqpoint{-0.020833in}{0.000000in}}{\pgfqpoint{-0.000000in}{0.000000in}}{%
\pgfpathmoveto{\pgfqpoint{-0.000000in}{0.000000in}}%
\pgfpathlineto{\pgfqpoint{-0.020833in}{0.000000in}}%
\pgfusepath{stroke,fill}%
}%
\begin{pgfscope}%
\pgfsys@transformshift{6.267353in}{1.342589in}%
\pgfsys@useobject{currentmarker}{}%
\end{pgfscope}%
\end{pgfscope}%
\begin{pgfscope}%
\pgfsetbuttcap%
\pgfsetroundjoin%
\definecolor{currentfill}{rgb}{0.000000,0.000000,0.000000}%
\pgfsetfillcolor{currentfill}%
\pgfsetlinewidth{0.501875pt}%
\definecolor{currentstroke}{rgb}{0.000000,0.000000,0.000000}%
\pgfsetstrokecolor{currentstroke}%
\pgfsetdash{}{0pt}%
\pgfsys@defobject{currentmarker}{\pgfqpoint{0.000000in}{0.000000in}}{\pgfqpoint{0.020833in}{0.000000in}}{%
\pgfpathmoveto{\pgfqpoint{0.000000in}{0.000000in}}%
\pgfpathlineto{\pgfqpoint{0.020833in}{0.000000in}}%
\pgfusepath{stroke,fill}%
}%
\begin{pgfscope}%
\pgfsys@transformshift{0.481681in}{1.424646in}%
\pgfsys@useobject{currentmarker}{}%
\end{pgfscope}%
\end{pgfscope}%
\begin{pgfscope}%
\pgfsetbuttcap%
\pgfsetroundjoin%
\definecolor{currentfill}{rgb}{0.000000,0.000000,0.000000}%
\pgfsetfillcolor{currentfill}%
\pgfsetlinewidth{0.501875pt}%
\definecolor{currentstroke}{rgb}{0.000000,0.000000,0.000000}%
\pgfsetstrokecolor{currentstroke}%
\pgfsetdash{}{0pt}%
\pgfsys@defobject{currentmarker}{\pgfqpoint{-0.020833in}{0.000000in}}{\pgfqpoint{-0.000000in}{0.000000in}}{%
\pgfpathmoveto{\pgfqpoint{-0.000000in}{0.000000in}}%
\pgfpathlineto{\pgfqpoint{-0.020833in}{0.000000in}}%
\pgfusepath{stroke,fill}%
}%
\begin{pgfscope}%
\pgfsys@transformshift{6.267353in}{1.424646in}%
\pgfsys@useobject{currentmarker}{}%
\end{pgfscope}%
\end{pgfscope}%
\begin{pgfscope}%
\pgfsetbuttcap%
\pgfsetroundjoin%
\definecolor{currentfill}{rgb}{0.000000,0.000000,0.000000}%
\pgfsetfillcolor{currentfill}%
\pgfsetlinewidth{0.501875pt}%
\definecolor{currentstroke}{rgb}{0.000000,0.000000,0.000000}%
\pgfsetstrokecolor{currentstroke}%
\pgfsetdash{}{0pt}%
\pgfsys@defobject{currentmarker}{\pgfqpoint{0.000000in}{0.000000in}}{\pgfqpoint{0.020833in}{0.000000in}}{%
\pgfpathmoveto{\pgfqpoint{0.000000in}{0.000000in}}%
\pgfpathlineto{\pgfqpoint{0.020833in}{0.000000in}}%
\pgfusepath{stroke,fill}%
}%
\begin{pgfscope}%
\pgfsys@transformshift{0.481681in}{1.588760in}%
\pgfsys@useobject{currentmarker}{}%
\end{pgfscope}%
\end{pgfscope}%
\begin{pgfscope}%
\pgfsetbuttcap%
\pgfsetroundjoin%
\definecolor{currentfill}{rgb}{0.000000,0.000000,0.000000}%
\pgfsetfillcolor{currentfill}%
\pgfsetlinewidth{0.501875pt}%
\definecolor{currentstroke}{rgb}{0.000000,0.000000,0.000000}%
\pgfsetstrokecolor{currentstroke}%
\pgfsetdash{}{0pt}%
\pgfsys@defobject{currentmarker}{\pgfqpoint{-0.020833in}{0.000000in}}{\pgfqpoint{-0.000000in}{0.000000in}}{%
\pgfpathmoveto{\pgfqpoint{-0.000000in}{0.000000in}}%
\pgfpathlineto{\pgfqpoint{-0.020833in}{0.000000in}}%
\pgfusepath{stroke,fill}%
}%
\begin{pgfscope}%
\pgfsys@transformshift{6.267353in}{1.588760in}%
\pgfsys@useobject{currentmarker}{}%
\end{pgfscope}%
\end{pgfscope}%
\begin{pgfscope}%
\pgfsetbuttcap%
\pgfsetroundjoin%
\definecolor{currentfill}{rgb}{0.000000,0.000000,0.000000}%
\pgfsetfillcolor{currentfill}%
\pgfsetlinewidth{0.501875pt}%
\definecolor{currentstroke}{rgb}{0.000000,0.000000,0.000000}%
\pgfsetstrokecolor{currentstroke}%
\pgfsetdash{}{0pt}%
\pgfsys@defobject{currentmarker}{\pgfqpoint{0.000000in}{0.000000in}}{\pgfqpoint{0.020833in}{0.000000in}}{%
\pgfpathmoveto{\pgfqpoint{0.000000in}{0.000000in}}%
\pgfpathlineto{\pgfqpoint{0.020833in}{0.000000in}}%
\pgfusepath{stroke,fill}%
}%
\begin{pgfscope}%
\pgfsys@transformshift{0.481681in}{1.670817in}%
\pgfsys@useobject{currentmarker}{}%
\end{pgfscope}%
\end{pgfscope}%
\begin{pgfscope}%
\pgfsetbuttcap%
\pgfsetroundjoin%
\definecolor{currentfill}{rgb}{0.000000,0.000000,0.000000}%
\pgfsetfillcolor{currentfill}%
\pgfsetlinewidth{0.501875pt}%
\definecolor{currentstroke}{rgb}{0.000000,0.000000,0.000000}%
\pgfsetstrokecolor{currentstroke}%
\pgfsetdash{}{0pt}%
\pgfsys@defobject{currentmarker}{\pgfqpoint{-0.020833in}{0.000000in}}{\pgfqpoint{-0.000000in}{0.000000in}}{%
\pgfpathmoveto{\pgfqpoint{-0.000000in}{0.000000in}}%
\pgfpathlineto{\pgfqpoint{-0.020833in}{0.000000in}}%
\pgfusepath{stroke,fill}%
}%
\begin{pgfscope}%
\pgfsys@transformshift{6.267353in}{1.670817in}%
\pgfsys@useobject{currentmarker}{}%
\end{pgfscope}%
\end{pgfscope}%
\begin{pgfscope}%
\pgfsetbuttcap%
\pgfsetroundjoin%
\definecolor{currentfill}{rgb}{0.000000,0.000000,0.000000}%
\pgfsetfillcolor{currentfill}%
\pgfsetlinewidth{0.501875pt}%
\definecolor{currentstroke}{rgb}{0.000000,0.000000,0.000000}%
\pgfsetstrokecolor{currentstroke}%
\pgfsetdash{}{0pt}%
\pgfsys@defobject{currentmarker}{\pgfqpoint{0.000000in}{0.000000in}}{\pgfqpoint{0.020833in}{0.000000in}}{%
\pgfpathmoveto{\pgfqpoint{0.000000in}{0.000000in}}%
\pgfpathlineto{\pgfqpoint{0.020833in}{0.000000in}}%
\pgfusepath{stroke,fill}%
}%
\begin{pgfscope}%
\pgfsys@transformshift{0.481681in}{1.752874in}%
\pgfsys@useobject{currentmarker}{}%
\end{pgfscope}%
\end{pgfscope}%
\begin{pgfscope}%
\pgfsetbuttcap%
\pgfsetroundjoin%
\definecolor{currentfill}{rgb}{0.000000,0.000000,0.000000}%
\pgfsetfillcolor{currentfill}%
\pgfsetlinewidth{0.501875pt}%
\definecolor{currentstroke}{rgb}{0.000000,0.000000,0.000000}%
\pgfsetstrokecolor{currentstroke}%
\pgfsetdash{}{0pt}%
\pgfsys@defobject{currentmarker}{\pgfqpoint{-0.020833in}{0.000000in}}{\pgfqpoint{-0.000000in}{0.000000in}}{%
\pgfpathmoveto{\pgfqpoint{-0.000000in}{0.000000in}}%
\pgfpathlineto{\pgfqpoint{-0.020833in}{0.000000in}}%
\pgfusepath{stroke,fill}%
}%
\begin{pgfscope}%
\pgfsys@transformshift{6.267353in}{1.752874in}%
\pgfsys@useobject{currentmarker}{}%
\end{pgfscope}%
\end{pgfscope}%
\begin{pgfscope}%
\pgfsetbuttcap%
\pgfsetroundjoin%
\definecolor{currentfill}{rgb}{0.000000,0.000000,0.000000}%
\pgfsetfillcolor{currentfill}%
\pgfsetlinewidth{0.501875pt}%
\definecolor{currentstroke}{rgb}{0.000000,0.000000,0.000000}%
\pgfsetstrokecolor{currentstroke}%
\pgfsetdash{}{0pt}%
\pgfsys@defobject{currentmarker}{\pgfqpoint{0.000000in}{0.000000in}}{\pgfqpoint{0.020833in}{0.000000in}}{%
\pgfpathmoveto{\pgfqpoint{0.000000in}{0.000000in}}%
\pgfpathlineto{\pgfqpoint{0.020833in}{0.000000in}}%
\pgfusepath{stroke,fill}%
}%
\begin{pgfscope}%
\pgfsys@transformshift{0.481681in}{1.916988in}%
\pgfsys@useobject{currentmarker}{}%
\end{pgfscope}%
\end{pgfscope}%
\begin{pgfscope}%
\pgfsetbuttcap%
\pgfsetroundjoin%
\definecolor{currentfill}{rgb}{0.000000,0.000000,0.000000}%
\pgfsetfillcolor{currentfill}%
\pgfsetlinewidth{0.501875pt}%
\definecolor{currentstroke}{rgb}{0.000000,0.000000,0.000000}%
\pgfsetstrokecolor{currentstroke}%
\pgfsetdash{}{0pt}%
\pgfsys@defobject{currentmarker}{\pgfqpoint{-0.020833in}{0.000000in}}{\pgfqpoint{-0.000000in}{0.000000in}}{%
\pgfpathmoveto{\pgfqpoint{-0.000000in}{0.000000in}}%
\pgfpathlineto{\pgfqpoint{-0.020833in}{0.000000in}}%
\pgfusepath{stroke,fill}%
}%
\begin{pgfscope}%
\pgfsys@transformshift{6.267353in}{1.916988in}%
\pgfsys@useobject{currentmarker}{}%
\end{pgfscope}%
\end{pgfscope}%
\begin{pgfscope}%
\pgfsetbuttcap%
\pgfsetroundjoin%
\definecolor{currentfill}{rgb}{0.000000,0.000000,0.000000}%
\pgfsetfillcolor{currentfill}%
\pgfsetlinewidth{0.501875pt}%
\definecolor{currentstroke}{rgb}{0.000000,0.000000,0.000000}%
\pgfsetstrokecolor{currentstroke}%
\pgfsetdash{}{0pt}%
\pgfsys@defobject{currentmarker}{\pgfqpoint{0.000000in}{0.000000in}}{\pgfqpoint{0.020833in}{0.000000in}}{%
\pgfpathmoveto{\pgfqpoint{0.000000in}{0.000000in}}%
\pgfpathlineto{\pgfqpoint{0.020833in}{0.000000in}}%
\pgfusepath{stroke,fill}%
}%
\begin{pgfscope}%
\pgfsys@transformshift{0.481681in}{1.999045in}%
\pgfsys@useobject{currentmarker}{}%
\end{pgfscope}%
\end{pgfscope}%
\begin{pgfscope}%
\pgfsetbuttcap%
\pgfsetroundjoin%
\definecolor{currentfill}{rgb}{0.000000,0.000000,0.000000}%
\pgfsetfillcolor{currentfill}%
\pgfsetlinewidth{0.501875pt}%
\definecolor{currentstroke}{rgb}{0.000000,0.000000,0.000000}%
\pgfsetstrokecolor{currentstroke}%
\pgfsetdash{}{0pt}%
\pgfsys@defobject{currentmarker}{\pgfqpoint{-0.020833in}{0.000000in}}{\pgfqpoint{-0.000000in}{0.000000in}}{%
\pgfpathmoveto{\pgfqpoint{-0.000000in}{0.000000in}}%
\pgfpathlineto{\pgfqpoint{-0.020833in}{0.000000in}}%
\pgfusepath{stroke,fill}%
}%
\begin{pgfscope}%
\pgfsys@transformshift{6.267353in}{1.999045in}%
\pgfsys@useobject{currentmarker}{}%
\end{pgfscope}%
\end{pgfscope}%
\begin{pgfscope}%
\pgfsetbuttcap%
\pgfsetroundjoin%
\definecolor{currentfill}{rgb}{0.000000,0.000000,0.000000}%
\pgfsetfillcolor{currentfill}%
\pgfsetlinewidth{0.501875pt}%
\definecolor{currentstroke}{rgb}{0.000000,0.000000,0.000000}%
\pgfsetstrokecolor{currentstroke}%
\pgfsetdash{}{0pt}%
\pgfsys@defobject{currentmarker}{\pgfqpoint{0.000000in}{0.000000in}}{\pgfqpoint{0.020833in}{0.000000in}}{%
\pgfpathmoveto{\pgfqpoint{0.000000in}{0.000000in}}%
\pgfpathlineto{\pgfqpoint{0.020833in}{0.000000in}}%
\pgfusepath{stroke,fill}%
}%
\begin{pgfscope}%
\pgfsys@transformshift{0.481681in}{2.081102in}%
\pgfsys@useobject{currentmarker}{}%
\end{pgfscope}%
\end{pgfscope}%
\begin{pgfscope}%
\pgfsetbuttcap%
\pgfsetroundjoin%
\definecolor{currentfill}{rgb}{0.000000,0.000000,0.000000}%
\pgfsetfillcolor{currentfill}%
\pgfsetlinewidth{0.501875pt}%
\definecolor{currentstroke}{rgb}{0.000000,0.000000,0.000000}%
\pgfsetstrokecolor{currentstroke}%
\pgfsetdash{}{0pt}%
\pgfsys@defobject{currentmarker}{\pgfqpoint{-0.020833in}{0.000000in}}{\pgfqpoint{-0.000000in}{0.000000in}}{%
\pgfpathmoveto{\pgfqpoint{-0.000000in}{0.000000in}}%
\pgfpathlineto{\pgfqpoint{-0.020833in}{0.000000in}}%
\pgfusepath{stroke,fill}%
}%
\begin{pgfscope}%
\pgfsys@transformshift{6.267353in}{2.081102in}%
\pgfsys@useobject{currentmarker}{}%
\end{pgfscope}%
\end{pgfscope}%
\begin{pgfscope}%
\pgfsetbuttcap%
\pgfsetroundjoin%
\definecolor{currentfill}{rgb}{0.000000,0.000000,0.000000}%
\pgfsetfillcolor{currentfill}%
\pgfsetlinewidth{0.501875pt}%
\definecolor{currentstroke}{rgb}{0.000000,0.000000,0.000000}%
\pgfsetstrokecolor{currentstroke}%
\pgfsetdash{}{0pt}%
\pgfsys@defobject{currentmarker}{\pgfqpoint{0.000000in}{0.000000in}}{\pgfqpoint{0.020833in}{0.000000in}}{%
\pgfpathmoveto{\pgfqpoint{0.000000in}{0.000000in}}%
\pgfpathlineto{\pgfqpoint{0.020833in}{0.000000in}}%
\pgfusepath{stroke,fill}%
}%
\begin{pgfscope}%
\pgfsys@transformshift{0.481681in}{2.245216in}%
\pgfsys@useobject{currentmarker}{}%
\end{pgfscope}%
\end{pgfscope}%
\begin{pgfscope}%
\pgfsetbuttcap%
\pgfsetroundjoin%
\definecolor{currentfill}{rgb}{0.000000,0.000000,0.000000}%
\pgfsetfillcolor{currentfill}%
\pgfsetlinewidth{0.501875pt}%
\definecolor{currentstroke}{rgb}{0.000000,0.000000,0.000000}%
\pgfsetstrokecolor{currentstroke}%
\pgfsetdash{}{0pt}%
\pgfsys@defobject{currentmarker}{\pgfqpoint{-0.020833in}{0.000000in}}{\pgfqpoint{-0.000000in}{0.000000in}}{%
\pgfpathmoveto{\pgfqpoint{-0.000000in}{0.000000in}}%
\pgfpathlineto{\pgfqpoint{-0.020833in}{0.000000in}}%
\pgfusepath{stroke,fill}%
}%
\begin{pgfscope}%
\pgfsys@transformshift{6.267353in}{2.245216in}%
\pgfsys@useobject{currentmarker}{}%
\end{pgfscope}%
\end{pgfscope}%
\begin{pgfscope}%
\pgfsetbuttcap%
\pgfsetroundjoin%
\definecolor{currentfill}{rgb}{0.000000,0.000000,0.000000}%
\pgfsetfillcolor{currentfill}%
\pgfsetlinewidth{0.501875pt}%
\definecolor{currentstroke}{rgb}{0.000000,0.000000,0.000000}%
\pgfsetstrokecolor{currentstroke}%
\pgfsetdash{}{0pt}%
\pgfsys@defobject{currentmarker}{\pgfqpoint{0.000000in}{0.000000in}}{\pgfqpoint{0.020833in}{0.000000in}}{%
\pgfpathmoveto{\pgfqpoint{0.000000in}{0.000000in}}%
\pgfpathlineto{\pgfqpoint{0.020833in}{0.000000in}}%
\pgfusepath{stroke,fill}%
}%
\begin{pgfscope}%
\pgfsys@transformshift{0.481681in}{2.327273in}%
\pgfsys@useobject{currentmarker}{}%
\end{pgfscope}%
\end{pgfscope}%
\begin{pgfscope}%
\pgfsetbuttcap%
\pgfsetroundjoin%
\definecolor{currentfill}{rgb}{0.000000,0.000000,0.000000}%
\pgfsetfillcolor{currentfill}%
\pgfsetlinewidth{0.501875pt}%
\definecolor{currentstroke}{rgb}{0.000000,0.000000,0.000000}%
\pgfsetstrokecolor{currentstroke}%
\pgfsetdash{}{0pt}%
\pgfsys@defobject{currentmarker}{\pgfqpoint{-0.020833in}{0.000000in}}{\pgfqpoint{-0.000000in}{0.000000in}}{%
\pgfpathmoveto{\pgfqpoint{-0.000000in}{0.000000in}}%
\pgfpathlineto{\pgfqpoint{-0.020833in}{0.000000in}}%
\pgfusepath{stroke,fill}%
}%
\begin{pgfscope}%
\pgfsys@transformshift{6.267353in}{2.327273in}%
\pgfsys@useobject{currentmarker}{}%
\end{pgfscope}%
\end{pgfscope}%
\begin{pgfscope}%
\pgfsetbuttcap%
\pgfsetroundjoin%
\definecolor{currentfill}{rgb}{0.000000,0.000000,0.000000}%
\pgfsetfillcolor{currentfill}%
\pgfsetlinewidth{0.501875pt}%
\definecolor{currentstroke}{rgb}{0.000000,0.000000,0.000000}%
\pgfsetstrokecolor{currentstroke}%
\pgfsetdash{}{0pt}%
\pgfsys@defobject{currentmarker}{\pgfqpoint{0.000000in}{0.000000in}}{\pgfqpoint{0.020833in}{0.000000in}}{%
\pgfpathmoveto{\pgfqpoint{0.000000in}{0.000000in}}%
\pgfpathlineto{\pgfqpoint{0.020833in}{0.000000in}}%
\pgfusepath{stroke,fill}%
}%
\begin{pgfscope}%
\pgfsys@transformshift{0.481681in}{2.409330in}%
\pgfsys@useobject{currentmarker}{}%
\end{pgfscope}%
\end{pgfscope}%
\begin{pgfscope}%
\pgfsetbuttcap%
\pgfsetroundjoin%
\definecolor{currentfill}{rgb}{0.000000,0.000000,0.000000}%
\pgfsetfillcolor{currentfill}%
\pgfsetlinewidth{0.501875pt}%
\definecolor{currentstroke}{rgb}{0.000000,0.000000,0.000000}%
\pgfsetstrokecolor{currentstroke}%
\pgfsetdash{}{0pt}%
\pgfsys@defobject{currentmarker}{\pgfqpoint{-0.020833in}{0.000000in}}{\pgfqpoint{-0.000000in}{0.000000in}}{%
\pgfpathmoveto{\pgfqpoint{-0.000000in}{0.000000in}}%
\pgfpathlineto{\pgfqpoint{-0.020833in}{0.000000in}}%
\pgfusepath{stroke,fill}%
}%
\begin{pgfscope}%
\pgfsys@transformshift{6.267353in}{2.409330in}%
\pgfsys@useobject{currentmarker}{}%
\end{pgfscope}%
\end{pgfscope}%
\begin{pgfscope}%
\pgfsetbuttcap%
\pgfsetroundjoin%
\definecolor{currentfill}{rgb}{0.000000,0.000000,0.000000}%
\pgfsetfillcolor{currentfill}%
\pgfsetlinewidth{0.501875pt}%
\definecolor{currentstroke}{rgb}{0.000000,0.000000,0.000000}%
\pgfsetstrokecolor{currentstroke}%
\pgfsetdash{}{0pt}%
\pgfsys@defobject{currentmarker}{\pgfqpoint{0.000000in}{0.000000in}}{\pgfqpoint{0.020833in}{0.000000in}}{%
\pgfpathmoveto{\pgfqpoint{0.000000in}{0.000000in}}%
\pgfpathlineto{\pgfqpoint{0.020833in}{0.000000in}}%
\pgfusepath{stroke,fill}%
}%
\begin{pgfscope}%
\pgfsys@transformshift{0.481681in}{2.573443in}%
\pgfsys@useobject{currentmarker}{}%
\end{pgfscope}%
\end{pgfscope}%
\begin{pgfscope}%
\pgfsetbuttcap%
\pgfsetroundjoin%
\definecolor{currentfill}{rgb}{0.000000,0.000000,0.000000}%
\pgfsetfillcolor{currentfill}%
\pgfsetlinewidth{0.501875pt}%
\definecolor{currentstroke}{rgb}{0.000000,0.000000,0.000000}%
\pgfsetstrokecolor{currentstroke}%
\pgfsetdash{}{0pt}%
\pgfsys@defobject{currentmarker}{\pgfqpoint{-0.020833in}{0.000000in}}{\pgfqpoint{-0.000000in}{0.000000in}}{%
\pgfpathmoveto{\pgfqpoint{-0.000000in}{0.000000in}}%
\pgfpathlineto{\pgfqpoint{-0.020833in}{0.000000in}}%
\pgfusepath{stroke,fill}%
}%
\begin{pgfscope}%
\pgfsys@transformshift{6.267353in}{2.573443in}%
\pgfsys@useobject{currentmarker}{}%
\end{pgfscope}%
\end{pgfscope}%
\begin{pgfscope}%
\pgfsetbuttcap%
\pgfsetroundjoin%
\definecolor{currentfill}{rgb}{0.000000,0.000000,0.000000}%
\pgfsetfillcolor{currentfill}%
\pgfsetlinewidth{0.501875pt}%
\definecolor{currentstroke}{rgb}{0.000000,0.000000,0.000000}%
\pgfsetstrokecolor{currentstroke}%
\pgfsetdash{}{0pt}%
\pgfsys@defobject{currentmarker}{\pgfqpoint{0.000000in}{0.000000in}}{\pgfqpoint{0.020833in}{0.000000in}}{%
\pgfpathmoveto{\pgfqpoint{0.000000in}{0.000000in}}%
\pgfpathlineto{\pgfqpoint{0.020833in}{0.000000in}}%
\pgfusepath{stroke,fill}%
}%
\begin{pgfscope}%
\pgfsys@transformshift{0.481681in}{2.655500in}%
\pgfsys@useobject{currentmarker}{}%
\end{pgfscope}%
\end{pgfscope}%
\begin{pgfscope}%
\pgfsetbuttcap%
\pgfsetroundjoin%
\definecolor{currentfill}{rgb}{0.000000,0.000000,0.000000}%
\pgfsetfillcolor{currentfill}%
\pgfsetlinewidth{0.501875pt}%
\definecolor{currentstroke}{rgb}{0.000000,0.000000,0.000000}%
\pgfsetstrokecolor{currentstroke}%
\pgfsetdash{}{0pt}%
\pgfsys@defobject{currentmarker}{\pgfqpoint{-0.020833in}{0.000000in}}{\pgfqpoint{-0.000000in}{0.000000in}}{%
\pgfpathmoveto{\pgfqpoint{-0.000000in}{0.000000in}}%
\pgfpathlineto{\pgfqpoint{-0.020833in}{0.000000in}}%
\pgfusepath{stroke,fill}%
}%
\begin{pgfscope}%
\pgfsys@transformshift{6.267353in}{2.655500in}%
\pgfsys@useobject{currentmarker}{}%
\end{pgfscope}%
\end{pgfscope}%
\begin{pgfscope}%
\pgfsetbuttcap%
\pgfsetroundjoin%
\definecolor{currentfill}{rgb}{0.000000,0.000000,0.000000}%
\pgfsetfillcolor{currentfill}%
\pgfsetlinewidth{0.501875pt}%
\definecolor{currentstroke}{rgb}{0.000000,0.000000,0.000000}%
\pgfsetstrokecolor{currentstroke}%
\pgfsetdash{}{0pt}%
\pgfsys@defobject{currentmarker}{\pgfqpoint{0.000000in}{0.000000in}}{\pgfqpoint{0.020833in}{0.000000in}}{%
\pgfpathmoveto{\pgfqpoint{0.000000in}{0.000000in}}%
\pgfpathlineto{\pgfqpoint{0.020833in}{0.000000in}}%
\pgfusepath{stroke,fill}%
}%
\begin{pgfscope}%
\pgfsys@transformshift{0.481681in}{2.737557in}%
\pgfsys@useobject{currentmarker}{}%
\end{pgfscope}%
\end{pgfscope}%
\begin{pgfscope}%
\pgfsetbuttcap%
\pgfsetroundjoin%
\definecolor{currentfill}{rgb}{0.000000,0.000000,0.000000}%
\pgfsetfillcolor{currentfill}%
\pgfsetlinewidth{0.501875pt}%
\definecolor{currentstroke}{rgb}{0.000000,0.000000,0.000000}%
\pgfsetstrokecolor{currentstroke}%
\pgfsetdash{}{0pt}%
\pgfsys@defobject{currentmarker}{\pgfqpoint{-0.020833in}{0.000000in}}{\pgfqpoint{-0.000000in}{0.000000in}}{%
\pgfpathmoveto{\pgfqpoint{-0.000000in}{0.000000in}}%
\pgfpathlineto{\pgfqpoint{-0.020833in}{0.000000in}}%
\pgfusepath{stroke,fill}%
}%
\begin{pgfscope}%
\pgfsys@transformshift{6.267353in}{2.737557in}%
\pgfsys@useobject{currentmarker}{}%
\end{pgfscope}%
\end{pgfscope}%
\begin{pgfscope}%
\pgfsetbuttcap%
\pgfsetroundjoin%
\definecolor{currentfill}{rgb}{0.000000,0.000000,0.000000}%
\pgfsetfillcolor{currentfill}%
\pgfsetlinewidth{0.501875pt}%
\definecolor{currentstroke}{rgb}{0.000000,0.000000,0.000000}%
\pgfsetstrokecolor{currentstroke}%
\pgfsetdash{}{0pt}%
\pgfsys@defobject{currentmarker}{\pgfqpoint{0.000000in}{0.000000in}}{\pgfqpoint{0.020833in}{0.000000in}}{%
\pgfpathmoveto{\pgfqpoint{0.000000in}{0.000000in}}%
\pgfpathlineto{\pgfqpoint{0.020833in}{0.000000in}}%
\pgfusepath{stroke,fill}%
}%
\begin{pgfscope}%
\pgfsys@transformshift{0.481681in}{2.901671in}%
\pgfsys@useobject{currentmarker}{}%
\end{pgfscope}%
\end{pgfscope}%
\begin{pgfscope}%
\pgfsetbuttcap%
\pgfsetroundjoin%
\definecolor{currentfill}{rgb}{0.000000,0.000000,0.000000}%
\pgfsetfillcolor{currentfill}%
\pgfsetlinewidth{0.501875pt}%
\definecolor{currentstroke}{rgb}{0.000000,0.000000,0.000000}%
\pgfsetstrokecolor{currentstroke}%
\pgfsetdash{}{0pt}%
\pgfsys@defobject{currentmarker}{\pgfqpoint{-0.020833in}{0.000000in}}{\pgfqpoint{-0.000000in}{0.000000in}}{%
\pgfpathmoveto{\pgfqpoint{-0.000000in}{0.000000in}}%
\pgfpathlineto{\pgfqpoint{-0.020833in}{0.000000in}}%
\pgfusepath{stroke,fill}%
}%
\begin{pgfscope}%
\pgfsys@transformshift{6.267353in}{2.901671in}%
\pgfsys@useobject{currentmarker}{}%
\end{pgfscope}%
\end{pgfscope}%
\begin{pgfscope}%
\pgfsetbuttcap%
\pgfsetroundjoin%
\definecolor{currentfill}{rgb}{0.000000,0.000000,0.000000}%
\pgfsetfillcolor{currentfill}%
\pgfsetlinewidth{0.501875pt}%
\definecolor{currentstroke}{rgb}{0.000000,0.000000,0.000000}%
\pgfsetstrokecolor{currentstroke}%
\pgfsetdash{}{0pt}%
\pgfsys@defobject{currentmarker}{\pgfqpoint{0.000000in}{0.000000in}}{\pgfqpoint{0.020833in}{0.000000in}}{%
\pgfpathmoveto{\pgfqpoint{0.000000in}{0.000000in}}%
\pgfpathlineto{\pgfqpoint{0.020833in}{0.000000in}}%
\pgfusepath{stroke,fill}%
}%
\begin{pgfscope}%
\pgfsys@transformshift{0.481681in}{2.983728in}%
\pgfsys@useobject{currentmarker}{}%
\end{pgfscope}%
\end{pgfscope}%
\begin{pgfscope}%
\pgfsetbuttcap%
\pgfsetroundjoin%
\definecolor{currentfill}{rgb}{0.000000,0.000000,0.000000}%
\pgfsetfillcolor{currentfill}%
\pgfsetlinewidth{0.501875pt}%
\definecolor{currentstroke}{rgb}{0.000000,0.000000,0.000000}%
\pgfsetstrokecolor{currentstroke}%
\pgfsetdash{}{0pt}%
\pgfsys@defobject{currentmarker}{\pgfqpoint{-0.020833in}{0.000000in}}{\pgfqpoint{-0.000000in}{0.000000in}}{%
\pgfpathmoveto{\pgfqpoint{-0.000000in}{0.000000in}}%
\pgfpathlineto{\pgfqpoint{-0.020833in}{0.000000in}}%
\pgfusepath{stroke,fill}%
}%
\begin{pgfscope}%
\pgfsys@transformshift{6.267353in}{2.983728in}%
\pgfsys@useobject{currentmarker}{}%
\end{pgfscope}%
\end{pgfscope}%
\begin{pgfscope}%
\pgfsetbuttcap%
\pgfsetroundjoin%
\definecolor{currentfill}{rgb}{0.000000,0.000000,0.000000}%
\pgfsetfillcolor{currentfill}%
\pgfsetlinewidth{0.501875pt}%
\definecolor{currentstroke}{rgb}{0.000000,0.000000,0.000000}%
\pgfsetstrokecolor{currentstroke}%
\pgfsetdash{}{0pt}%
\pgfsys@defobject{currentmarker}{\pgfqpoint{0.000000in}{0.000000in}}{\pgfqpoint{0.020833in}{0.000000in}}{%
\pgfpathmoveto{\pgfqpoint{0.000000in}{0.000000in}}%
\pgfpathlineto{\pgfqpoint{0.020833in}{0.000000in}}%
\pgfusepath{stroke,fill}%
}%
\begin{pgfscope}%
\pgfsys@transformshift{0.481681in}{3.065785in}%
\pgfsys@useobject{currentmarker}{}%
\end{pgfscope}%
\end{pgfscope}%
\begin{pgfscope}%
\pgfsetbuttcap%
\pgfsetroundjoin%
\definecolor{currentfill}{rgb}{0.000000,0.000000,0.000000}%
\pgfsetfillcolor{currentfill}%
\pgfsetlinewidth{0.501875pt}%
\definecolor{currentstroke}{rgb}{0.000000,0.000000,0.000000}%
\pgfsetstrokecolor{currentstroke}%
\pgfsetdash{}{0pt}%
\pgfsys@defobject{currentmarker}{\pgfqpoint{-0.020833in}{0.000000in}}{\pgfqpoint{-0.000000in}{0.000000in}}{%
\pgfpathmoveto{\pgfqpoint{-0.000000in}{0.000000in}}%
\pgfpathlineto{\pgfqpoint{-0.020833in}{0.000000in}}%
\pgfusepath{stroke,fill}%
}%
\begin{pgfscope}%
\pgfsys@transformshift{6.267353in}{3.065785in}%
\pgfsys@useobject{currentmarker}{}%
\end{pgfscope}%
\end{pgfscope}%
\begin{pgfscope}%
\definecolor{textcolor}{rgb}{0.000000,0.000000,0.000000}%
\pgfsetstrokecolor{textcolor}%
\pgfsetfillcolor{textcolor}%
\pgftext[x=0.235472in,y=2.154322in,,bottom,rotate=90.000000]{\color{textcolor}\rmfamily\fontsize{8.000000}{9.600000}\selectfont Wert}%
\end{pgfscope}%
\begin{pgfscope}%
\pgfpathrectangle{\pgfqpoint{0.481681in}{1.080890in}}{\pgfqpoint{5.785672in}{2.146863in}}%
\pgfusepath{clip}%
\pgfsetrectcap%
\pgfsetroundjoin%
\pgfsetlinewidth{0.401500pt}%
\definecolor{currentstroke}{rgb}{0.000000,0.070588,0.098039}%
\pgfsetstrokecolor{currentstroke}%
\pgfsetdash{}{0pt}%
\pgfpathmoveto{\pgfqpoint{0.744666in}{1.178475in}}%
\pgfpathlineto{\pgfqpoint{0.775450in}{1.179687in}}%
\pgfpathlineto{\pgfqpoint{0.781074in}{1.181303in}}%
\pgfpathlineto{\pgfqpoint{0.781370in}{1.184315in}}%
\pgfpathlineto{\pgfqpoint{0.782554in}{1.184424in}}%
\pgfpathlineto{\pgfqpoint{0.924637in}{1.188521in}}%
\pgfpathlineto{\pgfqpoint{0.925525in}{1.189714in}}%
\pgfpathlineto{\pgfqpoint{0.994494in}{1.191607in}}%
\pgfpathlineto{\pgfqpoint{0.995382in}{1.197093in}}%
\pgfpathlineto{\pgfqpoint{0.996862in}{1.198471in}}%
\pgfpathlineto{\pgfqpoint{1.022318in}{1.199606in}}%
\pgfpathlineto{\pgfqpoint{1.023502in}{1.205714in}}%
\pgfpathlineto{\pgfqpoint{1.030606in}{1.207567in}}%
\pgfpathlineto{\pgfqpoint{1.031198in}{1.211322in}}%
\pgfpathlineto{\pgfqpoint{1.031790in}{1.211768in}}%
\pgfpathlineto{\pgfqpoint{1.122368in}{1.212900in}}%
\pgfpathlineto{\pgfqpoint{1.122664in}{1.213335in}}%
\pgfpathlineto{\pgfqpoint{1.123256in}{1.218996in}}%
\pgfpathlineto{\pgfqpoint{1.181272in}{1.222411in}}%
\pgfpathlineto{\pgfqpoint{1.196369in}{1.225097in}}%
\pgfpathlineto{\pgfqpoint{1.216201in}{1.225446in}}%
\pgfpathlineto{\pgfqpoint{1.217385in}{1.227464in}}%
\pgfpathlineto{\pgfqpoint{1.222121in}{1.231816in}}%
\pgfpathlineto{\pgfqpoint{1.229225in}{1.234590in}}%
\pgfpathlineto{\pgfqpoint{1.298194in}{1.236877in}}%
\pgfpathlineto{\pgfqpoint{1.321282in}{1.238743in}}%
\pgfpathlineto{\pgfqpoint{1.321874in}{1.242584in}}%
\pgfpathlineto{\pgfqpoint{1.386403in}{1.247689in}}%
\pgfpathlineto{\pgfqpoint{1.386995in}{1.249407in}}%
\pgfpathlineto{\pgfqpoint{1.415412in}{1.249856in}}%
\pgfpathlineto{\pgfqpoint{1.416004in}{1.250193in}}%
\pgfpathlineto{\pgfqpoint{1.422516in}{1.250787in}}%
\pgfpathlineto{\pgfqpoint{1.423404in}{1.250986in}}%
\pgfpathlineto{\pgfqpoint{1.445900in}{1.253399in}}%
\pgfpathlineto{\pgfqpoint{1.499181in}{1.262403in}}%
\pgfpathlineto{\pgfqpoint{1.521381in}{1.266109in}}%
\pgfpathlineto{\pgfqpoint{1.522565in}{1.268049in}}%
\pgfpathlineto{\pgfqpoint{1.569038in}{1.270280in}}%
\pgfpathlineto{\pgfqpoint{1.591534in}{1.272194in}}%
\pgfpathlineto{\pgfqpoint{1.592127in}{1.280403in}}%
\pgfpathlineto{\pgfqpoint{1.614919in}{1.295515in}}%
\pgfpathlineto{\pgfqpoint{1.621135in}{1.296524in}}%
\pgfpathlineto{\pgfqpoint{1.621727in}{1.302726in}}%
\pgfpathlineto{\pgfqpoint{1.627943in}{1.303906in}}%
\pgfpathlineto{\pgfqpoint{1.665832in}{1.307203in}}%
\pgfpathlineto{\pgfqpoint{1.679744in}{1.310423in}}%
\pgfpathlineto{\pgfqpoint{1.690400in}{1.314788in}}%
\pgfpathlineto{\pgfqpoint{1.690992in}{1.320220in}}%
\pgfpathlineto{\pgfqpoint{1.691880in}{1.321273in}}%
\pgfpathlineto{\pgfqpoint{1.692472in}{1.322803in}}%
\pgfpathlineto{\pgfqpoint{1.720000in}{1.324678in}}%
\pgfpathlineto{\pgfqpoint{1.720592in}{1.328124in}}%
\pgfpathlineto{\pgfqpoint{1.720888in}{1.329604in}}%
\pgfpathlineto{\pgfqpoint{1.723552in}{1.329886in}}%
\pgfpathlineto{\pgfqpoint{1.745161in}{1.331637in}}%
\pgfpathlineto{\pgfqpoint{1.812946in}{1.333679in}}%
\pgfpathlineto{\pgfqpoint{1.818570in}{1.335553in}}%
\pgfpathlineto{\pgfqpoint{1.826562in}{1.350547in}}%
\pgfpathlineto{\pgfqpoint{1.917731in}{1.352514in}}%
\pgfpathlineto{\pgfqpoint{1.919507in}{1.356707in}}%
\pgfpathlineto{\pgfqpoint{1.927203in}{1.376052in}}%
\pgfpathlineto{\pgfqpoint{1.939636in}{1.387261in}}%
\pgfpathlineto{\pgfqpoint{1.940820in}{1.393145in}}%
\pgfpathlineto{\pgfqpoint{1.976340in}{1.396180in}}%
\pgfpathlineto{\pgfqpoint{1.977524in}{1.401299in}}%
\pgfpathlineto{\pgfqpoint{1.992028in}{1.407324in}}%
\pgfpathlineto{\pgfqpoint{2.010677in}{1.410217in}}%
\pgfpathlineto{\pgfqpoint{2.011269in}{1.414165in}}%
\pgfpathlineto{\pgfqpoint{2.011861in}{1.414812in}}%
\pgfpathlineto{\pgfqpoint{2.093262in}{1.418621in}}%
\pgfpathlineto{\pgfqpoint{2.113982in}{1.420280in}}%
\pgfpathlineto{\pgfqpoint{2.124046in}{1.421765in}}%
\pgfpathlineto{\pgfqpoint{2.125822in}{1.424694in}}%
\pgfpathlineto{\pgfqpoint{2.137663in}{1.425627in}}%
\pgfpathlineto{\pgfqpoint{2.164007in}{1.428732in}}%
\pgfpathlineto{\pgfqpoint{2.166375in}{1.429189in}}%
\pgfpathlineto{\pgfqpoint{2.167263in}{1.430462in}}%
\pgfpathlineto{\pgfqpoint{2.184135in}{1.432362in}}%
\pgfpathlineto{\pgfqpoint{2.209592in}{1.433279in}}%
\pgfpathlineto{\pgfqpoint{2.215512in}{1.434871in}}%
\pgfpathlineto{\pgfqpoint{2.216104in}{1.436506in}}%
\pgfpathlineto{\pgfqpoint{2.217288in}{1.438997in}}%
\pgfpathlineto{\pgfqpoint{2.218768in}{1.440316in}}%
\pgfpathlineto{\pgfqpoint{2.219064in}{1.441415in}}%
\pgfpathlineto{\pgfqpoint{2.237416in}{1.441840in}}%
\pgfpathlineto{\pgfqpoint{2.238896in}{1.443303in}}%
\pgfpathlineto{\pgfqpoint{2.245408in}{1.443751in}}%
\pgfpathlineto{\pgfqpoint{2.269681in}{1.445628in}}%
\pgfpathlineto{\pgfqpoint{2.273825in}{1.446388in}}%
\pgfpathlineto{\pgfqpoint{2.275305in}{1.449752in}}%
\pgfpathlineto{\pgfqpoint{2.276193in}{1.451371in}}%
\pgfpathlineto{\pgfqpoint{2.282705in}{1.453703in}}%
\pgfpathlineto{\pgfqpoint{2.289809in}{1.458215in}}%
\pgfpathlineto{\pgfqpoint{2.311121in}{1.461154in}}%
\pgfpathlineto{\pgfqpoint{2.311713in}{1.462564in}}%
\pgfpathlineto{\pgfqpoint{2.315561in}{1.465229in}}%
\pgfpathlineto{\pgfqpoint{2.316449in}{1.467724in}}%
\pgfpathlineto{\pgfqpoint{2.317633in}{1.469137in}}%
\pgfpathlineto{\pgfqpoint{2.318225in}{1.489424in}}%
\pgfpathlineto{\pgfqpoint{2.322369in}{1.491312in}}%
\pgfpathlineto{\pgfqpoint{2.322961in}{1.506719in}}%
\pgfpathlineto{\pgfqpoint{2.325033in}{1.515622in}}%
\pgfpathlineto{\pgfqpoint{2.325625in}{1.515936in}}%
\pgfpathlineto{\pgfqpoint{2.365882in}{1.516359in}}%
\pgfpathlineto{\pgfqpoint{2.366770in}{1.518590in}}%
\pgfpathlineto{\pgfqpoint{2.367362in}{1.519658in}}%
\pgfpathlineto{\pgfqpoint{2.372690in}{1.524740in}}%
\pgfpathlineto{\pgfqpoint{2.374466in}{1.529314in}}%
\pgfpathlineto{\pgfqpoint{2.379202in}{1.531128in}}%
\pgfpathlineto{\pgfqpoint{2.380682in}{1.533291in}}%
\pgfpathlineto{\pgfqpoint{2.386306in}{1.535928in}}%
\pgfpathlineto{\pgfqpoint{2.386602in}{1.536656in}}%
\pgfpathlineto{\pgfqpoint{2.387194in}{1.538752in}}%
\pgfpathlineto{\pgfqpoint{2.388082in}{1.540416in}}%
\pgfpathlineto{\pgfqpoint{2.388970in}{1.542946in}}%
\pgfpathlineto{\pgfqpoint{2.402587in}{1.544062in}}%
\pgfpathlineto{\pgfqpoint{2.409691in}{1.545623in}}%
\pgfpathlineto{\pgfqpoint{2.410579in}{1.548650in}}%
\pgfpathlineto{\pgfqpoint{2.416203in}{1.548998in}}%
\pgfpathlineto{\pgfqpoint{2.416499in}{1.549374in}}%
\pgfpathlineto{\pgfqpoint{2.417683in}{1.554485in}}%
\pgfpathlineto{\pgfqpoint{2.418275in}{1.561600in}}%
\pgfpathlineto{\pgfqpoint{2.429819in}{1.565989in}}%
\pgfpathlineto{\pgfqpoint{2.431003in}{1.566438in}}%
\pgfpathlineto{\pgfqpoint{2.431299in}{1.567155in}}%
\pgfpathlineto{\pgfqpoint{2.431891in}{1.570103in}}%
\pgfpathlineto{\pgfqpoint{2.432483in}{1.579843in}}%
\pgfpathlineto{\pgfqpoint{2.458531in}{1.582831in}}%
\pgfpathlineto{\pgfqpoint{2.459419in}{1.587512in}}%
\pgfpathlineto{\pgfqpoint{2.460603in}{1.588647in}}%
\pgfpathlineto{\pgfqpoint{2.460899in}{1.589356in}}%
\pgfpathlineto{\pgfqpoint{2.461491in}{1.591164in}}%
\pgfpathlineto{\pgfqpoint{2.465043in}{1.593835in}}%
\pgfpathlineto{\pgfqpoint{2.466819in}{1.597268in}}%
\pgfpathlineto{\pgfqpoint{2.470076in}{1.597721in}}%
\pgfpathlineto{\pgfqpoint{2.481916in}{1.599472in}}%
\pgfpathlineto{\pgfqpoint{2.486948in}{1.602113in}}%
\pgfpathlineto{\pgfqpoint{2.487540in}{1.605401in}}%
\pgfpathlineto{\pgfqpoint{2.508852in}{1.612882in}}%
\pgfpathlineto{\pgfqpoint{2.509148in}{1.613431in}}%
\pgfpathlineto{\pgfqpoint{2.510628in}{1.666615in}}%
\pgfpathlineto{\pgfqpoint{2.519508in}{1.672946in}}%
\pgfpathlineto{\pgfqpoint{2.521876in}{1.674488in}}%
\pgfpathlineto{\pgfqpoint{2.706583in}{1.678663in}}%
\pgfpathlineto{\pgfqpoint{2.706879in}{1.680579in}}%
\pgfpathlineto{\pgfqpoint{2.733815in}{1.683205in}}%
\pgfpathlineto{\pgfqpoint{2.740624in}{1.683882in}}%
\pgfpathlineto{\pgfqpoint{2.764600in}{1.685802in}}%
\pgfpathlineto{\pgfqpoint{2.765784in}{1.685925in}}%
\pgfpathlineto{\pgfqpoint{2.785320in}{1.686904in}}%
\pgfpathlineto{\pgfqpoint{2.821137in}{1.695810in}}%
\pgfpathlineto{\pgfqpoint{2.836529in}{1.699151in}}%
\pgfpathlineto{\pgfqpoint{2.837121in}{1.706137in}}%
\pgfpathlineto{\pgfqpoint{2.855473in}{1.710876in}}%
\pgfpathlineto{\pgfqpoint{2.855769in}{1.711529in}}%
\pgfpathlineto{\pgfqpoint{2.856065in}{1.715948in}}%
\pgfpathlineto{\pgfqpoint{2.877082in}{1.718444in}}%
\pgfpathlineto{\pgfqpoint{2.882706in}{1.718694in}}%
\pgfpathlineto{\pgfqpoint{2.907274in}{1.719561in}}%
\pgfpathlineto{\pgfqpoint{2.907866in}{1.720148in}}%
\pgfpathlineto{\pgfqpoint{2.922370in}{1.721536in}}%
\pgfpathlineto{\pgfqpoint{2.935690in}{1.726138in}}%
\pgfpathlineto{\pgfqpoint{2.936282in}{1.732058in}}%
\pgfpathlineto{\pgfqpoint{2.937466in}{1.734194in}}%
\pgfpathlineto{\pgfqpoint{2.956115in}{1.736483in}}%
\pgfpathlineto{\pgfqpoint{2.970323in}{1.741487in}}%
\pgfpathlineto{\pgfqpoint{2.976243in}{1.745877in}}%
\pgfpathlineto{\pgfqpoint{2.976835in}{1.747930in}}%
\pgfpathlineto{\pgfqpoint{3.004659in}{1.752879in}}%
\pgfpathlineto{\pgfqpoint{3.005547in}{1.754126in}}%
\pgfpathlineto{\pgfqpoint{3.006139in}{1.761335in}}%
\pgfpathlineto{\pgfqpoint{3.054092in}{1.769842in}}%
\pgfpathlineto{\pgfqpoint{3.054388in}{1.770181in}}%
\pgfpathlineto{\pgfqpoint{3.054980in}{1.781552in}}%
\pgfpathlineto{\pgfqpoint{3.063860in}{1.789521in}}%
\pgfpathlineto{\pgfqpoint{3.070372in}{1.792269in}}%
\pgfpathlineto{\pgfqpoint{3.071556in}{1.797327in}}%
\pgfpathlineto{\pgfqpoint{3.107669in}{1.800727in}}%
\pgfpathlineto{\pgfqpoint{3.110925in}{1.801484in}}%
\pgfpathlineto{\pgfqpoint{3.112405in}{1.803961in}}%
\pgfpathlineto{\pgfqpoint{3.113589in}{1.806024in}}%
\pgfpathlineto{\pgfqpoint{3.114181in}{1.812934in}}%
\pgfpathlineto{\pgfqpoint{3.158878in}{1.819991in}}%
\pgfpathlineto{\pgfqpoint{3.177526in}{1.824470in}}%
\pgfpathlineto{\pgfqpoint{3.204462in}{1.827556in}}%
\pgfpathlineto{\pgfqpoint{3.205646in}{1.829652in}}%
\pgfpathlineto{\pgfqpoint{3.206238in}{1.832470in}}%
\pgfpathlineto{\pgfqpoint{3.210383in}{1.835168in}}%
\pgfpathlineto{\pgfqpoint{3.210679in}{1.836957in}}%
\pgfpathlineto{\pgfqpoint{3.211271in}{1.841900in}}%
\pgfpathlineto{\pgfqpoint{3.212455in}{1.843525in}}%
\pgfpathlineto{\pgfqpoint{3.213047in}{1.845388in}}%
\pgfpathlineto{\pgfqpoint{3.220743in}{1.848869in}}%
\pgfpathlineto{\pgfqpoint{3.234359in}{1.853233in}}%
\pgfpathlineto{\pgfqpoint{3.271359in}{1.855486in}}%
\pgfpathlineto{\pgfqpoint{3.303032in}{1.860570in}}%
\pgfpathlineto{\pgfqpoint{3.303624in}{1.864819in}}%
\pgfpathlineto{\pgfqpoint{3.304808in}{1.866143in}}%
\pgfpathlineto{\pgfqpoint{3.319016in}{1.867961in}}%
\pgfpathlineto{\pgfqpoint{3.319312in}{1.869486in}}%
\pgfpathlineto{\pgfqpoint{3.319608in}{1.875018in}}%
\pgfpathlineto{\pgfqpoint{3.324936in}{1.879038in}}%
\pgfpathlineto{\pgfqpoint{3.355129in}{1.879668in}}%
\pgfpathlineto{\pgfqpoint{3.361345in}{1.882381in}}%
\pgfpathlineto{\pgfqpoint{3.361937in}{1.884595in}}%
\pgfpathlineto{\pgfqpoint{3.370225in}{1.889670in}}%
\pgfpathlineto{\pgfqpoint{3.374665in}{1.891343in}}%
\pgfpathlineto{\pgfqpoint{3.382657in}{1.911459in}}%
\pgfpathlineto{\pgfqpoint{3.427058in}{1.918831in}}%
\pgfpathlineto{\pgfqpoint{3.455178in}{1.922525in}}%
\pgfpathlineto{\pgfqpoint{3.479451in}{1.923956in}}%
\pgfpathlineto{\pgfqpoint{3.481523in}{1.924243in}}%
\pgfpathlineto{\pgfqpoint{3.482115in}{1.927199in}}%
\pgfpathlineto{\pgfqpoint{3.515267in}{1.934124in}}%
\pgfpathlineto{\pgfqpoint{3.516155in}{1.940345in}}%
\pgfpathlineto{\pgfqpoint{3.518227in}{1.945420in}}%
\pgfpathlineto{\pgfqpoint{3.518819in}{1.947092in}}%
\pgfpathlineto{\pgfqpoint{3.522371in}{1.948234in}}%
\pgfpathlineto{\pgfqpoint{3.525035in}{1.949237in}}%
\pgfpathlineto{\pgfqpoint{3.525923in}{1.949769in}}%
\pgfpathlineto{\pgfqpoint{3.604364in}{1.953132in}}%
\pgfpathlineto{\pgfqpoint{3.609396in}{1.953912in}}%
\pgfpathlineto{\pgfqpoint{3.609988in}{1.955940in}}%
\pgfpathlineto{\pgfqpoint{3.611172in}{1.958218in}}%
\pgfpathlineto{\pgfqpoint{3.615613in}{1.961043in}}%
\pgfpathlineto{\pgfqpoint{3.616501in}{1.964216in}}%
\pgfpathlineto{\pgfqpoint{3.653205in}{1.965401in}}%
\pgfpathlineto{\pgfqpoint{3.654685in}{1.967314in}}%
\pgfpathlineto{\pgfqpoint{3.660605in}{1.968552in}}%
\pgfpathlineto{\pgfqpoint{3.661493in}{1.973403in}}%
\pgfpathlineto{\pgfqpoint{3.672445in}{1.978200in}}%
\pgfpathlineto{\pgfqpoint{3.673629in}{1.983323in}}%
\pgfpathlineto{\pgfqpoint{3.675405in}{1.985897in}}%
\pgfpathlineto{\pgfqpoint{3.708558in}{1.990866in}}%
\pgfpathlineto{\pgfqpoint{3.709446in}{1.994274in}}%
\pgfpathlineto{\pgfqpoint{3.710038in}{1.996670in}}%
\pgfpathlineto{\pgfqpoint{3.715366in}{2.001332in}}%
\pgfpathlineto{\pgfqpoint{3.715958in}{2.003732in}}%
\pgfpathlineto{\pgfqpoint{3.716550in}{2.006359in}}%
\pgfpathlineto{\pgfqpoint{3.718326in}{2.006490in}}%
\pgfpathlineto{\pgfqpoint{3.759471in}{2.008578in}}%
\pgfpathlineto{\pgfqpoint{3.765391in}{2.010484in}}%
\pgfpathlineto{\pgfqpoint{3.766279in}{2.011148in}}%
\pgfpathlineto{\pgfqpoint{3.767167in}{2.015766in}}%
\pgfpathlineto{\pgfqpoint{3.773383in}{2.018653in}}%
\pgfpathlineto{\pgfqpoint{3.781967in}{2.019819in}}%
\pgfpathlineto{\pgfqpoint{3.800911in}{2.024188in}}%
\pgfpathlineto{\pgfqpoint{3.801207in}{2.024637in}}%
\pgfpathlineto{\pgfqpoint{3.801799in}{2.027943in}}%
\pgfpathlineto{\pgfqpoint{3.802687in}{2.028121in}}%
\pgfpathlineto{\pgfqpoint{3.809495in}{2.028445in}}%
\pgfpathlineto{\pgfqpoint{3.809791in}{2.028895in}}%
\pgfpathlineto{\pgfqpoint{3.810975in}{2.036874in}}%
\pgfpathlineto{\pgfqpoint{3.816007in}{2.042012in}}%
\pgfpathlineto{\pgfqpoint{3.816303in}{2.044235in}}%
\pgfpathlineto{\pgfqpoint{3.816896in}{2.051050in}}%
\pgfpathlineto{\pgfqpoint{3.817784in}{2.051974in}}%
\pgfpathlineto{\pgfqpoint{3.823408in}{2.056222in}}%
\pgfpathlineto{\pgfqpoint{3.852712in}{2.057264in}}%
\pgfpathlineto{\pgfqpoint{3.859520in}{2.060566in}}%
\pgfpathlineto{\pgfqpoint{3.867512in}{2.063142in}}%
\pgfpathlineto{\pgfqpoint{3.906881in}{2.066822in}}%
\pgfpathlineto{\pgfqpoint{3.907177in}{2.068436in}}%
\pgfpathlineto{\pgfqpoint{3.907473in}{2.085430in}}%
\pgfpathlineto{\pgfqpoint{3.930265in}{2.094607in}}%
\pgfpathlineto{\pgfqpoint{3.930857in}{2.100707in}}%
\pgfpathlineto{\pgfqpoint{4.014922in}{2.103976in}}%
\pgfpathlineto{\pgfqpoint{4.022915in}{2.119013in}}%
\pgfpathlineto{\pgfqpoint{4.023507in}{2.123996in}}%
\pgfpathlineto{\pgfqpoint{4.029427in}{2.125733in}}%
\pgfpathlineto{\pgfqpoint{4.030019in}{2.128844in}}%
\pgfpathlineto{\pgfqpoint{4.106980in}{2.130742in}}%
\pgfpathlineto{\pgfqpoint{4.107572in}{2.134275in}}%
\pgfpathlineto{\pgfqpoint{4.108164in}{2.139558in}}%
\pgfpathlineto{\pgfqpoint{4.114380in}{2.141324in}}%
\pgfpathlineto{\pgfqpoint{4.114972in}{2.149357in}}%
\pgfpathlineto{\pgfqpoint{4.115268in}{2.150177in}}%
\pgfpathlineto{\pgfqpoint{4.115860in}{2.155311in}}%
\pgfpathlineto{\pgfqpoint{4.120892in}{2.162020in}}%
\pgfpathlineto{\pgfqpoint{4.121484in}{2.165415in}}%
\pgfpathlineto{\pgfqpoint{4.127700in}{2.167211in}}%
\pgfpathlineto{\pgfqpoint{4.128292in}{2.170307in}}%
\pgfpathlineto{\pgfqpoint{4.169437in}{2.173304in}}%
\pgfpathlineto{\pgfqpoint{4.170029in}{2.176462in}}%
\pgfpathlineto{\pgfqpoint{4.178909in}{2.179352in}}%
\pgfpathlineto{\pgfqpoint{4.179797in}{2.182471in}}%
\pgfpathlineto{\pgfqpoint{4.198149in}{2.183607in}}%
\pgfpathlineto{\pgfqpoint{4.199925in}{2.185925in}}%
\pgfpathlineto{\pgfqpoint{4.200221in}{2.186324in}}%
\pgfpathlineto{\pgfqpoint{4.200813in}{2.189614in}}%
\pgfpathlineto{\pgfqpoint{4.201405in}{2.191662in}}%
\pgfpathlineto{\pgfqpoint{4.207325in}{2.193524in}}%
\pgfpathlineto{\pgfqpoint{4.212357in}{2.193834in}}%
\pgfpathlineto{\pgfqpoint{4.213541in}{2.200172in}}%
\pgfpathlineto{\pgfqpoint{4.214133in}{2.201093in}}%
\pgfpathlineto{\pgfqpoint{4.265046in}{2.204305in}}%
\pgfpathlineto{\pgfqpoint{4.273630in}{2.208196in}}%
\pgfpathlineto{\pgfqpoint{4.279846in}{2.210307in}}%
\pgfpathlineto{\pgfqpoint{4.307079in}{2.213410in}}%
\pgfpathlineto{\pgfqpoint{4.312407in}{2.216294in}}%
\pgfpathlineto{\pgfqpoint{4.399432in}{2.218473in}}%
\pgfpathlineto{\pgfqpoint{4.407424in}{2.221763in}}%
\pgfpathlineto{\pgfqpoint{4.413640in}{2.226449in}}%
\pgfpathlineto{\pgfqpoint{4.420152in}{2.229119in}}%
\pgfpathlineto{\pgfqpoint{4.421040in}{2.232280in}}%
\pgfpathlineto{\pgfqpoint{4.496818in}{2.237172in}}%
\pgfpathlineto{\pgfqpoint{4.497114in}{2.237301in}}%
\pgfpathlineto{\pgfqpoint{4.498002in}{2.239933in}}%
\pgfpathlineto{\pgfqpoint{4.498890in}{2.241052in}}%
\pgfpathlineto{\pgfqpoint{4.504218in}{2.242374in}}%
\pgfpathlineto{\pgfqpoint{4.505698in}{2.245378in}}%
\pgfpathlineto{\pgfqpoint{4.511026in}{2.248285in}}%
\pgfpathlineto{\pgfqpoint{4.535002in}{2.250746in}}%
\pgfpathlineto{\pgfqpoint{4.589467in}{2.255045in}}%
\pgfpathlineto{\pgfqpoint{4.590059in}{2.261218in}}%
\pgfpathlineto{\pgfqpoint{4.612851in}{2.266167in}}%
\pgfpathlineto{\pgfqpoint{4.636236in}{2.268235in}}%
\pgfpathlineto{\pgfqpoint{4.662580in}{2.271289in}}%
\pgfpathlineto{\pgfqpoint{4.685668in}{2.274359in}}%
\pgfpathlineto{\pgfqpoint{4.702837in}{2.276323in}}%
\pgfpathlineto{\pgfqpoint{4.703725in}{2.281538in}}%
\pgfpathlineto{\pgfqpoint{4.710533in}{2.284477in}}%
\pgfpathlineto{\pgfqpoint{4.714085in}{2.284796in}}%
\pgfpathlineto{\pgfqpoint{4.794894in}{2.287758in}}%
\pgfpathlineto{\pgfqpoint{4.795486in}{2.288643in}}%
\pgfpathlineto{\pgfqpoint{4.796670in}{2.290828in}}%
\pgfpathlineto{\pgfqpoint{4.810582in}{2.294163in}}%
\pgfpathlineto{\pgfqpoint{4.825382in}{2.297588in}}%
\pgfpathlineto{\pgfqpoint{4.825974in}{2.300751in}}%
\pgfpathlineto{\pgfqpoint{4.856167in}{2.306112in}}%
\pgfpathlineto{\pgfqpoint{4.868007in}{2.307589in}}%
\pgfpathlineto{\pgfqpoint{4.869191in}{2.309753in}}%
\pgfpathlineto{\pgfqpoint{4.873631in}{2.316032in}}%
\pgfpathlineto{\pgfqpoint{4.874519in}{2.321600in}}%
\pgfpathlineto{\pgfqpoint{4.875407in}{2.324267in}}%
\pgfpathlineto{\pgfqpoint{4.901752in}{2.324643in}}%
\pgfpathlineto{\pgfqpoint{4.902048in}{2.325980in}}%
\pgfpathlineto{\pgfqpoint{4.902640in}{2.329902in}}%
\pgfpathlineto{\pgfqpoint{4.908856in}{2.335204in}}%
\pgfpathlineto{\pgfqpoint{4.909152in}{2.335581in}}%
\pgfpathlineto{\pgfqpoint{4.909744in}{2.341350in}}%
\pgfpathlineto{\pgfqpoint{4.917440in}{2.348488in}}%
\pgfpathlineto{\pgfqpoint{4.918920in}{2.349391in}}%
\pgfpathlineto{\pgfqpoint{4.944672in}{2.352416in}}%
\pgfpathlineto{\pgfqpoint{4.945264in}{2.358546in}}%
\pgfpathlineto{\pgfqpoint{4.952664in}{2.361632in}}%
\pgfpathlineto{\pgfqpoint{4.953848in}{2.364669in}}%
\pgfpathlineto{\pgfqpoint{4.961840in}{2.367920in}}%
\pgfpathlineto{\pgfqpoint{4.975161in}{2.370831in}}%
\pgfpathlineto{\pgfqpoint{5.004465in}{2.374026in}}%
\pgfpathlineto{\pgfqpoint{5.007425in}{2.375175in}}%
\pgfpathlineto{\pgfqpoint{5.007721in}{2.375662in}}%
\pgfpathlineto{\pgfqpoint{5.008313in}{2.381582in}}%
\pgfpathlineto{\pgfqpoint{5.010089in}{2.384249in}}%
\pgfpathlineto{\pgfqpoint{5.010385in}{2.387381in}}%
\pgfpathlineto{\pgfqpoint{5.095930in}{2.388517in}}%
\pgfpathlineto{\pgfqpoint{5.096522in}{2.393661in}}%
\pgfpathlineto{\pgfqpoint{5.100667in}{2.396422in}}%
\pgfpathlineto{\pgfqpoint{5.101259in}{2.399638in}}%
\pgfpathlineto{\pgfqpoint{5.102147in}{2.399864in}}%
\pgfpathlineto{\pgfqpoint{5.103331in}{2.409118in}}%
\pgfpathlineto{\pgfqpoint{5.104219in}{2.413004in}}%
\pgfpathlineto{\pgfqpoint{5.107179in}{2.415626in}}%
\pgfpathlineto{\pgfqpoint{5.107475in}{2.417487in}}%
\pgfpathlineto{\pgfqpoint{5.108067in}{2.422477in}}%
\pgfpathlineto{\pgfqpoint{5.109251in}{2.425048in}}%
\pgfpathlineto{\pgfqpoint{5.116355in}{2.426432in}}%
\pgfpathlineto{\pgfqpoint{5.118131in}{2.427331in}}%
\pgfpathlineto{\pgfqpoint{5.155427in}{2.429249in}}%
\pgfpathlineto{\pgfqpoint{5.205452in}{2.429965in}}%
\pgfpathlineto{\pgfqpoint{5.301654in}{2.432995in}}%
\pgfpathlineto{\pgfqpoint{5.302246in}{2.442532in}}%
\pgfpathlineto{\pgfqpoint{5.306686in}{2.442868in}}%
\pgfpathlineto{\pgfqpoint{5.306982in}{2.443227in}}%
\pgfpathlineto{\pgfqpoint{5.308166in}{2.450546in}}%
\pgfpathlineto{\pgfqpoint{5.313790in}{2.451592in}}%
\pgfpathlineto{\pgfqpoint{5.314382in}{2.464574in}}%
\pgfpathlineto{\pgfqpoint{5.315566in}{2.468125in}}%
\pgfpathlineto{\pgfqpoint{5.316158in}{2.469913in}}%
\pgfpathlineto{\pgfqpoint{5.356710in}{2.474960in}}%
\pgfpathlineto{\pgfqpoint{5.358782in}{2.478492in}}%
\pgfpathlineto{\pgfqpoint{5.359078in}{2.488483in}}%
\pgfpathlineto{\pgfqpoint{5.359374in}{2.518639in}}%
\pgfpathlineto{\pgfqpoint{5.359966in}{2.520079in}}%
\pgfpathlineto{\pgfqpoint{5.407031in}{2.522168in}}%
\pgfpathlineto{\pgfqpoint{5.412655in}{2.524911in}}%
\pgfpathlineto{\pgfqpoint{5.413543in}{2.531033in}}%
\pgfpathlineto{\pgfqpoint{5.414135in}{2.534136in}}%
\pgfpathlineto{\pgfqpoint{5.420055in}{2.538400in}}%
\pgfpathlineto{\pgfqpoint{5.420351in}{2.538999in}}%
\pgfpathlineto{\pgfqpoint{5.420647in}{2.545457in}}%
\pgfpathlineto{\pgfqpoint{5.421239in}{2.545827in}}%
\pgfpathlineto{\pgfqpoint{5.421831in}{2.547455in}}%
\pgfpathlineto{\pgfqpoint{5.445216in}{2.547849in}}%
\pgfpathlineto{\pgfqpoint{5.446696in}{2.549569in}}%
\pgfpathlineto{\pgfqpoint{5.449360in}{2.553090in}}%
\pgfpathlineto{\pgfqpoint{5.450248in}{2.557925in}}%
\pgfpathlineto{\pgfqpoint{5.450840in}{2.558620in}}%
\pgfpathlineto{\pgfqpoint{5.451432in}{2.562517in}}%
\pgfpathlineto{\pgfqpoint{5.458832in}{2.567390in}}%
\pgfpathlineto{\pgfqpoint{5.464752in}{2.573321in}}%
\pgfpathlineto{\pgfqpoint{5.497904in}{2.575186in}}%
\pgfpathlineto{\pgfqpoint{5.498792in}{2.580262in}}%
\pgfpathlineto{\pgfqpoint{5.513593in}{2.583572in}}%
\pgfpathlineto{\pgfqpoint{5.522769in}{2.588352in}}%
\pgfpathlineto{\pgfqpoint{5.542009in}{2.593440in}}%
\pgfpathlineto{\pgfqpoint{5.550593in}{2.598424in}}%
\pgfpathlineto{\pgfqpoint{5.563321in}{2.602910in}}%
\pgfpathlineto{\pgfqpoint{5.564209in}{2.611628in}}%
\pgfpathlineto{\pgfqpoint{5.572202in}{2.616287in}}%
\pgfpathlineto{\pgfqpoint{5.629034in}{2.618818in}}%
\pgfpathlineto{\pgfqpoint{5.647091in}{2.620056in}}%
\pgfpathlineto{\pgfqpoint{5.647387in}{2.620916in}}%
\pgfpathlineto{\pgfqpoint{5.647979in}{2.624929in}}%
\pgfpathlineto{\pgfqpoint{5.649163in}{2.625266in}}%
\pgfpathlineto{\pgfqpoint{5.650643in}{2.625713in}}%
\pgfpathlineto{\pgfqpoint{5.651827in}{2.631332in}}%
\pgfpathlineto{\pgfqpoint{5.657155in}{2.636998in}}%
\pgfpathlineto{\pgfqpoint{5.663963in}{2.645762in}}%
\pgfpathlineto{\pgfqpoint{5.689715in}{2.650076in}}%
\pgfpathlineto{\pgfqpoint{5.690307in}{2.653273in}}%
\pgfpathlineto{\pgfqpoint{5.716948in}{2.661975in}}%
\pgfpathlineto{\pgfqpoint{5.741812in}{2.672079in}}%
\pgfpathlineto{\pgfqpoint{5.749508in}{2.675984in}}%
\pgfpathlineto{\pgfqpoint{5.755428in}{2.676326in}}%
\pgfpathlineto{\pgfqpoint{5.756020in}{2.684417in}}%
\pgfpathlineto{\pgfqpoint{5.761940in}{2.686154in}}%
\pgfpathlineto{\pgfqpoint{5.762532in}{2.692277in}}%
\pgfpathlineto{\pgfqpoint{5.763420in}{2.694245in}}%
\pgfpathlineto{\pgfqpoint{5.793021in}{2.697389in}}%
\pgfpathlineto{\pgfqpoint{5.799829in}{2.697924in}}%
\pgfpathlineto{\pgfqpoint{5.803381in}{2.699472in}}%
\pgfpathlineto{\pgfqpoint{5.804565in}{2.702725in}}%
\pgfpathlineto{\pgfqpoint{5.823213in}{2.708805in}}%
\pgfpathlineto{\pgfqpoint{5.839790in}{2.715395in}}%
\pgfpathlineto{\pgfqpoint{5.840382in}{2.718779in}}%
\pgfpathlineto{\pgfqpoint{5.846302in}{2.723370in}}%
\pgfpathlineto{\pgfqpoint{5.846894in}{2.725453in}}%
\pgfpathlineto{\pgfqpoint{5.847486in}{2.728591in}}%
\pgfpathlineto{\pgfqpoint{5.848670in}{2.750214in}}%
\pgfpathlineto{\pgfqpoint{5.854294in}{2.759706in}}%
\pgfpathlineto{\pgfqpoint{5.854886in}{2.763082in}}%
\pgfpathlineto{\pgfqpoint{5.862582in}{2.765017in}}%
\pgfpathlineto{\pgfqpoint{5.863174in}{2.766836in}}%
\pgfpathlineto{\pgfqpoint{5.898990in}{2.768274in}}%
\pgfpathlineto{\pgfqpoint{5.915863in}{2.770785in}}%
\pgfpathlineto{\pgfqpoint{5.938655in}{2.774016in}}%
\pgfpathlineto{\pgfqpoint{5.939543in}{2.781100in}}%
\pgfpathlineto{\pgfqpoint{5.953455in}{2.785897in}}%
\pgfpathlineto{\pgfqpoint{5.988384in}{2.788293in}}%
\pgfpathlineto{\pgfqpoint{5.988680in}{2.790000in}}%
\pgfpathlineto{\pgfqpoint{5.989272in}{2.806738in}}%
\pgfpathlineto{\pgfqpoint{5.995784in}{2.809869in}}%
\pgfpathlineto{\pgfqpoint{5.996968in}{2.812937in}}%
\pgfpathlineto{\pgfqpoint{5.997560in}{2.813975in}}%
\pgfpathlineto{\pgfqpoint{6.004368in}{2.819614in}}%
\pgfpathlineto{\pgfqpoint{6.004368in}{2.819614in}}%
\pgfusepath{stroke}%
\end{pgfscope}%
\begin{pgfscope}%
\pgfpathrectangle{\pgfqpoint{0.481681in}{1.080890in}}{\pgfqpoint{5.785672in}{2.146863in}}%
\pgfusepath{clip}%
\pgfsetrectcap%
\pgfsetroundjoin%
\pgfsetlinewidth{0.200750pt}%
\definecolor{currentstroke}{rgb}{0.682353,0.125490,0.070588}%
\pgfsetstrokecolor{currentstroke}%
\pgfsetdash{}{0pt}%
\pgfpathmoveto{\pgfqpoint{0.744666in}{1.178475in}}%
\pgfpathlineto{\pgfqpoint{0.744962in}{1.180311in}}%
\pgfpathlineto{\pgfqpoint{0.745554in}{1.180311in}}%
\pgfpathlineto{\pgfqpoint{0.746146in}{1.186842in}}%
\pgfpathlineto{\pgfqpoint{0.746738in}{1.303106in}}%
\pgfpathlineto{\pgfqpoint{0.747034in}{1.303935in}}%
\pgfpathlineto{\pgfqpoint{0.816003in}{1.303935in}}%
\pgfpathlineto{\pgfqpoint{0.816595in}{1.305335in}}%
\pgfpathlineto{\pgfqpoint{0.825179in}{1.305480in}}%
\pgfpathlineto{\pgfqpoint{0.826659in}{1.313222in}}%
\pgfpathlineto{\pgfqpoint{0.830803in}{1.335642in}}%
\pgfpathlineto{\pgfqpoint{0.831099in}{1.332727in}}%
\pgfpathlineto{\pgfqpoint{0.831395in}{1.309934in}}%
\pgfpathlineto{\pgfqpoint{0.832283in}{1.333457in}}%
\pgfpathlineto{\pgfqpoint{0.832579in}{1.322719in}}%
\pgfpathlineto{\pgfqpoint{0.832875in}{1.322090in}}%
\pgfpathlineto{\pgfqpoint{0.840571in}{1.337524in}}%
\pgfpathlineto{\pgfqpoint{0.886748in}{1.337732in}}%
\pgfpathlineto{\pgfqpoint{0.887340in}{1.338103in}}%
\pgfpathlineto{\pgfqpoint{0.888228in}{1.337990in}}%
\pgfpathlineto{\pgfqpoint{0.889412in}{1.337990in}}%
\pgfpathlineto{\pgfqpoint{0.889708in}{1.338533in}}%
\pgfpathlineto{\pgfqpoint{0.890300in}{1.341646in}}%
\pgfpathlineto{\pgfqpoint{0.891484in}{1.341722in}}%
\pgfpathlineto{\pgfqpoint{0.894740in}{1.341722in}}%
\pgfpathlineto{\pgfqpoint{0.895036in}{1.340937in}}%
\pgfpathlineto{\pgfqpoint{0.895628in}{1.317164in}}%
\pgfpathlineto{\pgfqpoint{0.895924in}{1.309551in}}%
\pgfpathlineto{\pgfqpoint{0.900660in}{1.309534in}}%
\pgfpathlineto{\pgfqpoint{0.902140in}{1.309588in}}%
\pgfpathlineto{\pgfqpoint{0.904508in}{1.311987in}}%
\pgfpathlineto{\pgfqpoint{0.912204in}{1.313266in}}%
\pgfpathlineto{\pgfqpoint{0.915460in}{1.313266in}}%
\pgfpathlineto{\pgfqpoint{0.915756in}{1.313905in}}%
\pgfpathlineto{\pgfqpoint{0.916348in}{1.316065in}}%
\pgfpathlineto{\pgfqpoint{0.923453in}{1.316240in}}%
\pgfpathlineto{\pgfqpoint{0.932333in}{1.327727in}}%
\pgfpathlineto{\pgfqpoint{0.937661in}{1.327727in}}%
\pgfpathlineto{\pgfqpoint{0.937957in}{1.328836in}}%
\pgfpathlineto{\pgfqpoint{0.938549in}{1.337426in}}%
\pgfpathlineto{\pgfqpoint{0.939733in}{1.328021in}}%
\pgfpathlineto{\pgfqpoint{0.940325in}{1.327727in}}%
\pgfpathlineto{\pgfqpoint{0.967261in}{1.327727in}}%
\pgfpathlineto{\pgfqpoint{0.967557in}{1.334709in}}%
\pgfpathlineto{\pgfqpoint{0.967853in}{1.348720in}}%
\pgfpathlineto{\pgfqpoint{0.973181in}{1.348720in}}%
\pgfpathlineto{\pgfqpoint{0.973477in}{1.345403in}}%
\pgfpathlineto{\pgfqpoint{0.974069in}{1.332350in}}%
\pgfpathlineto{\pgfqpoint{0.974365in}{1.336441in}}%
\pgfpathlineto{\pgfqpoint{0.974957in}{1.365981in}}%
\pgfpathlineto{\pgfqpoint{0.987094in}{1.365981in}}%
\pgfpathlineto{\pgfqpoint{0.987390in}{1.363717in}}%
\pgfpathlineto{\pgfqpoint{0.989462in}{1.328976in}}%
\pgfpathlineto{\pgfqpoint{0.993310in}{1.363140in}}%
\pgfpathlineto{\pgfqpoint{0.993606in}{1.361479in}}%
\pgfpathlineto{\pgfqpoint{0.993902in}{1.366246in}}%
\pgfpathlineto{\pgfqpoint{0.994494in}{1.365981in}}%
\pgfpathlineto{\pgfqpoint{1.016694in}{1.365985in}}%
\pgfpathlineto{\pgfqpoint{1.016990in}{1.367259in}}%
\pgfpathlineto{\pgfqpoint{1.017286in}{1.365981in}}%
\pgfpathlineto{\pgfqpoint{1.038006in}{1.365981in}}%
\pgfpathlineto{\pgfqpoint{1.038302in}{1.368656in}}%
\pgfpathlineto{\pgfqpoint{1.038894in}{1.366332in}}%
\pgfpathlineto{\pgfqpoint{1.043630in}{1.372357in}}%
\pgfpathlineto{\pgfqpoint{1.044222in}{1.369814in}}%
\pgfpathlineto{\pgfqpoint{1.044814in}{1.366717in}}%
\pgfpathlineto{\pgfqpoint{1.045110in}{1.368297in}}%
\pgfpathlineto{\pgfqpoint{1.045406in}{1.372481in}}%
\pgfpathlineto{\pgfqpoint{1.048366in}{1.372512in}}%
\pgfpathlineto{\pgfqpoint{1.064943in}{1.372512in}}%
\pgfpathlineto{\pgfqpoint{1.065535in}{1.450418in}}%
\pgfpathlineto{\pgfqpoint{1.071751in}{1.450434in}}%
\pgfpathlineto{\pgfqpoint{1.072935in}{1.456462in}}%
\pgfpathlineto{\pgfqpoint{1.073527in}{1.456950in}}%
\pgfpathlineto{\pgfqpoint{1.078559in}{1.456770in}}%
\pgfpathlineto{\pgfqpoint{1.079743in}{1.453200in}}%
\pgfpathlineto{\pgfqpoint{1.081223in}{1.458303in}}%
\pgfpathlineto{\pgfqpoint{1.082407in}{1.458067in}}%
\pgfpathlineto{\pgfqpoint{1.086255in}{1.457029in}}%
\pgfpathlineto{\pgfqpoint{1.086551in}{1.458089in}}%
\pgfpathlineto{\pgfqpoint{1.086847in}{1.463550in}}%
\pgfpathlineto{\pgfqpoint{1.087143in}{1.434828in}}%
\pgfpathlineto{\pgfqpoint{1.087439in}{1.373538in}}%
\pgfpathlineto{\pgfqpoint{1.088327in}{1.475144in}}%
\pgfpathlineto{\pgfqpoint{1.089807in}{1.475777in}}%
\pgfpathlineto{\pgfqpoint{1.093063in}{1.477306in}}%
\pgfpathlineto{\pgfqpoint{1.093359in}{1.476475in}}%
\pgfpathlineto{\pgfqpoint{1.094247in}{1.406621in}}%
\pgfpathlineto{\pgfqpoint{1.094543in}{1.381088in}}%
\pgfpathlineto{\pgfqpoint{1.094839in}{1.379924in}}%
\pgfpathlineto{\pgfqpoint{1.095727in}{1.477476in}}%
\pgfpathlineto{\pgfqpoint{1.145752in}{1.477668in}}%
\pgfpathlineto{\pgfqpoint{1.164696in}{1.485873in}}%
\pgfpathlineto{\pgfqpoint{1.164992in}{1.485818in}}%
\pgfpathlineto{\pgfqpoint{1.165584in}{1.478973in}}%
\pgfpathlineto{\pgfqpoint{1.166176in}{1.491661in}}%
\pgfpathlineto{\pgfqpoint{1.168248in}{1.543720in}}%
\pgfpathlineto{\pgfqpoint{1.171504in}{1.543720in}}%
\pgfpathlineto{\pgfqpoint{1.171800in}{1.536298in}}%
\pgfpathlineto{\pgfqpoint{1.172392in}{1.497668in}}%
\pgfpathlineto{\pgfqpoint{1.172688in}{1.513253in}}%
\pgfpathlineto{\pgfqpoint{1.172984in}{1.543720in}}%
\pgfpathlineto{\pgfqpoint{1.178312in}{1.543720in}}%
\pgfpathlineto{\pgfqpoint{1.178608in}{1.542648in}}%
\pgfpathlineto{\pgfqpoint{1.179496in}{1.492385in}}%
\pgfpathlineto{\pgfqpoint{1.179792in}{1.505233in}}%
\pgfpathlineto{\pgfqpoint{1.180384in}{1.543720in}}%
\pgfpathlineto{\pgfqpoint{1.185120in}{1.544152in}}%
\pgfpathlineto{\pgfqpoint{1.185712in}{1.544381in}}%
\pgfpathlineto{\pgfqpoint{1.186304in}{1.544895in}}%
\pgfpathlineto{\pgfqpoint{1.186600in}{1.544707in}}%
\pgfpathlineto{\pgfqpoint{1.186896in}{1.538030in}}%
\pgfpathlineto{\pgfqpoint{1.187192in}{1.501338in}}%
\pgfpathlineto{\pgfqpoint{1.187785in}{1.545119in}}%
\pgfpathlineto{\pgfqpoint{1.188377in}{1.545187in}}%
\pgfpathlineto{\pgfqpoint{1.189265in}{1.547481in}}%
\pgfpathlineto{\pgfqpoint{1.194889in}{1.548273in}}%
\pgfpathlineto{\pgfqpoint{1.195481in}{1.549828in}}%
\pgfpathlineto{\pgfqpoint{1.214129in}{1.614580in}}%
\pgfpathlineto{\pgfqpoint{1.215905in}{1.614629in}}%
\pgfpathlineto{\pgfqpoint{1.216201in}{1.597628in}}%
\pgfpathlineto{\pgfqpoint{1.216497in}{1.600047in}}%
\pgfpathlineto{\pgfqpoint{1.216793in}{1.623026in}}%
\pgfpathlineto{\pgfqpoint{1.221529in}{1.623026in}}%
\pgfpathlineto{\pgfqpoint{1.221825in}{1.624606in}}%
\pgfpathlineto{\pgfqpoint{1.222121in}{1.623647in}}%
\pgfpathlineto{\pgfqpoint{1.222713in}{1.625825in}}%
\pgfpathlineto{\pgfqpoint{1.231001in}{1.626087in}}%
\pgfpathlineto{\pgfqpoint{1.235145in}{1.628986in}}%
\pgfpathlineto{\pgfqpoint{1.235737in}{1.626213in}}%
\pgfpathlineto{\pgfqpoint{1.236625in}{1.629168in}}%
\pgfpathlineto{\pgfqpoint{1.236921in}{1.627276in}}%
\pgfpathlineto{\pgfqpoint{1.237513in}{1.628348in}}%
\pgfpathlineto{\pgfqpoint{1.238105in}{1.625825in}}%
\pgfpathlineto{\pgfqpoint{1.244025in}{1.625825in}}%
\pgfpathlineto{\pgfqpoint{1.244321in}{1.627257in}}%
\pgfpathlineto{\pgfqpoint{1.244913in}{1.633289in}}%
\pgfpathlineto{\pgfqpoint{1.249945in}{1.633068in}}%
\pgfpathlineto{\pgfqpoint{1.250833in}{1.626492in}}%
\pgfpathlineto{\pgfqpoint{1.252313in}{1.638820in}}%
\pgfpathlineto{\pgfqpoint{1.263562in}{1.735810in}}%
\pgfpathlineto{\pgfqpoint{1.263858in}{1.745603in}}%
\pgfpathlineto{\pgfqpoint{1.264450in}{1.782792in}}%
\pgfpathlineto{\pgfqpoint{1.264746in}{1.784903in}}%
\pgfpathlineto{\pgfqpoint{1.267114in}{1.784903in}}%
\pgfpathlineto{\pgfqpoint{1.267410in}{1.782011in}}%
\pgfpathlineto{\pgfqpoint{1.267706in}{1.653288in}}%
\pgfpathlineto{\pgfqpoint{1.268002in}{1.646790in}}%
\pgfpathlineto{\pgfqpoint{1.270370in}{1.784903in}}%
\pgfpathlineto{\pgfqpoint{1.273034in}{1.785107in}}%
\pgfpathlineto{\pgfqpoint{1.273626in}{1.785370in}}%
\pgfpathlineto{\pgfqpoint{1.279250in}{1.785624in}}%
\pgfpathlineto{\pgfqpoint{1.279546in}{1.785777in}}%
\pgfpathlineto{\pgfqpoint{1.279842in}{1.785587in}}%
\pgfpathlineto{\pgfqpoint{1.280730in}{1.626318in}}%
\pgfpathlineto{\pgfqpoint{1.281026in}{1.635020in}}%
\pgfpathlineto{\pgfqpoint{1.284874in}{1.784703in}}%
\pgfpathlineto{\pgfqpoint{1.285170in}{1.786303in}}%
\pgfpathlineto{\pgfqpoint{1.286650in}{1.786303in}}%
\pgfpathlineto{\pgfqpoint{1.286946in}{1.710690in}}%
\pgfpathlineto{\pgfqpoint{1.287538in}{1.784790in}}%
\pgfpathlineto{\pgfqpoint{1.287834in}{1.782491in}}%
\pgfpathlineto{\pgfqpoint{1.308258in}{1.625825in}}%
\pgfpathlineto{\pgfqpoint{1.313586in}{1.625825in}}%
\pgfpathlineto{\pgfqpoint{1.313882in}{1.639930in}}%
\pgfpathlineto{\pgfqpoint{1.315362in}{1.771730in}}%
\pgfpathlineto{\pgfqpoint{1.315658in}{1.712305in}}%
\pgfpathlineto{\pgfqpoint{1.315954in}{1.786303in}}%
\pgfpathlineto{\pgfqpoint{1.320098in}{1.786752in}}%
\pgfpathlineto{\pgfqpoint{1.320394in}{1.743851in}}%
\pgfpathlineto{\pgfqpoint{1.320690in}{1.786769in}}%
\pgfpathlineto{\pgfqpoint{1.328387in}{1.786769in}}%
\pgfpathlineto{\pgfqpoint{1.328979in}{1.795167in}}%
\pgfpathlineto{\pgfqpoint{1.335195in}{1.795219in}}%
\pgfpathlineto{\pgfqpoint{1.335787in}{1.803739in}}%
\pgfpathlineto{\pgfqpoint{1.342299in}{1.916875in}}%
\pgfpathlineto{\pgfqpoint{1.343779in}{1.917399in}}%
\pgfpathlineto{\pgfqpoint{1.357691in}{1.921489in}}%
\pgfpathlineto{\pgfqpoint{1.359171in}{1.917184in}}%
\pgfpathlineto{\pgfqpoint{1.360355in}{1.919970in}}%
\pgfpathlineto{\pgfqpoint{1.363019in}{1.921487in}}%
\pgfpathlineto{\pgfqpoint{1.363315in}{1.914843in}}%
\pgfpathlineto{\pgfqpoint{1.365091in}{1.813126in}}%
\pgfpathlineto{\pgfqpoint{1.365979in}{1.917427in}}%
\pgfpathlineto{\pgfqpoint{1.370715in}{1.921528in}}%
\pgfpathlineto{\pgfqpoint{1.374859in}{1.918598in}}%
\pgfpathlineto{\pgfqpoint{1.376931in}{1.917120in}}%
\pgfpathlineto{\pgfqpoint{1.377227in}{1.920066in}}%
\pgfpathlineto{\pgfqpoint{1.377819in}{1.952000in}}%
\pgfpathlineto{\pgfqpoint{1.378411in}{1.865182in}}%
\pgfpathlineto{\pgfqpoint{1.379891in}{1.633046in}}%
\pgfpathlineto{\pgfqpoint{1.385515in}{1.955992in}}%
\pgfpathlineto{\pgfqpoint{1.386403in}{1.958417in}}%
\pgfpathlineto{\pgfqpoint{1.388771in}{1.959551in}}%
\pgfpathlineto{\pgfqpoint{1.391140in}{1.960639in}}%
\pgfpathlineto{\pgfqpoint{1.391436in}{1.959005in}}%
\pgfpathlineto{\pgfqpoint{1.391732in}{1.955645in}}%
\pgfpathlineto{\pgfqpoint{1.393804in}{1.955884in}}%
\pgfpathlineto{\pgfqpoint{1.413636in}{1.962096in}}%
\pgfpathlineto{\pgfqpoint{1.413932in}{1.961845in}}%
\pgfpathlineto{\pgfqpoint{1.415412in}{1.955645in}}%
\pgfpathlineto{\pgfqpoint{1.427252in}{1.955793in}}%
\pgfpathlineto{\pgfqpoint{1.427844in}{1.962172in}}%
\pgfpathlineto{\pgfqpoint{1.428732in}{1.961995in}}%
\pgfpathlineto{\pgfqpoint{1.429324in}{1.959566in}}%
\pgfpathlineto{\pgfqpoint{1.429620in}{1.959089in}}%
\pgfpathlineto{\pgfqpoint{1.434948in}{1.962160in}}%
\pgfpathlineto{\pgfqpoint{1.435836in}{1.965441in}}%
\pgfpathlineto{\pgfqpoint{1.444124in}{1.965189in}}%
\pgfpathlineto{\pgfqpoint{1.462181in}{1.963367in}}%
\pgfpathlineto{\pgfqpoint{1.462477in}{1.964234in}}%
\pgfpathlineto{\pgfqpoint{1.465437in}{2.001362in}}%
\pgfpathlineto{\pgfqpoint{1.471357in}{2.001362in}}%
\pgfpathlineto{\pgfqpoint{1.471949in}{2.006200in}}%
\pgfpathlineto{\pgfqpoint{1.476685in}{2.014656in}}%
\pgfpathlineto{\pgfqpoint{1.476981in}{1.996971in}}%
\pgfpathlineto{\pgfqpoint{1.477277in}{2.023755in}}%
\pgfpathlineto{\pgfqpoint{1.478757in}{2.024357in}}%
\pgfpathlineto{\pgfqpoint{1.484677in}{2.026738in}}%
\pgfpathlineto{\pgfqpoint{1.484973in}{2.025324in}}%
\pgfpathlineto{\pgfqpoint{1.485565in}{2.028230in}}%
\pgfpathlineto{\pgfqpoint{1.485861in}{2.033085in}}%
\pgfpathlineto{\pgfqpoint{1.520197in}{2.033279in}}%
\pgfpathlineto{\pgfqpoint{1.529078in}{2.036125in}}%
\pgfpathlineto{\pgfqpoint{1.533518in}{2.040902in}}%
\pgfpathlineto{\pgfqpoint{1.534406in}{2.046266in}}%
\pgfpathlineto{\pgfqpoint{1.534702in}{2.046005in}}%
\pgfpathlineto{\pgfqpoint{1.541214in}{2.041135in}}%
\pgfpathlineto{\pgfqpoint{1.541510in}{2.043854in}}%
\pgfpathlineto{\pgfqpoint{1.542102in}{2.056032in}}%
\pgfpathlineto{\pgfqpoint{1.542398in}{2.053632in}}%
\pgfpathlineto{\pgfqpoint{1.542694in}{2.046250in}}%
\pgfpathlineto{\pgfqpoint{1.543286in}{2.046465in}}%
\pgfpathlineto{\pgfqpoint{1.562230in}{2.057790in}}%
\pgfpathlineto{\pgfqpoint{1.562822in}{2.056970in}}%
\pgfpathlineto{\pgfqpoint{1.569038in}{2.046293in}}%
\pgfpathlineto{\pgfqpoint{1.569630in}{2.050845in}}%
\pgfpathlineto{\pgfqpoint{1.570222in}{2.056482in}}%
\pgfpathlineto{\pgfqpoint{1.570518in}{2.053041in}}%
\pgfpathlineto{\pgfqpoint{1.571406in}{2.057818in}}%
\pgfpathlineto{\pgfqpoint{1.578510in}{2.058517in}}%
\pgfpathlineto{\pgfqpoint{1.579694in}{2.059999in}}%
\pgfpathlineto{\pgfqpoint{1.590646in}{2.072046in}}%
\pgfpathlineto{\pgfqpoint{1.591238in}{2.073204in}}%
\pgfpathlineto{\pgfqpoint{1.611663in}{2.073204in}}%
\pgfpathlineto{\pgfqpoint{1.611959in}{2.074177in}}%
\pgfpathlineto{\pgfqpoint{1.612255in}{2.082772in}}%
\pgfpathlineto{\pgfqpoint{1.612551in}{2.079143in}}%
\pgfpathlineto{\pgfqpoint{1.612847in}{2.085800in}}%
\pgfpathlineto{\pgfqpoint{1.681816in}{2.085526in}}%
\pgfpathlineto{\pgfqpoint{1.683592in}{2.085517in}}%
\pgfpathlineto{\pgfqpoint{1.683888in}{2.085678in}}%
\pgfpathlineto{\pgfqpoint{1.684184in}{2.086585in}}%
\pgfpathlineto{\pgfqpoint{1.684776in}{2.092244in}}%
\pgfpathlineto{\pgfqpoint{1.685368in}{2.102128in}}%
\pgfpathlineto{\pgfqpoint{1.689808in}{2.102060in}}%
\pgfpathlineto{\pgfqpoint{1.690696in}{2.096558in}}%
\pgfpathlineto{\pgfqpoint{1.692472in}{2.085451in}}%
\pgfpathlineto{\pgfqpoint{1.711416in}{2.101894in}}%
\pgfpathlineto{\pgfqpoint{1.712008in}{2.103061in}}%
\pgfpathlineto{\pgfqpoint{1.712600in}{2.103061in}}%
\pgfpathlineto{\pgfqpoint{1.713488in}{2.107704in}}%
\pgfpathlineto{\pgfqpoint{1.714080in}{2.103085in}}%
\pgfpathlineto{\pgfqpoint{1.714968in}{2.094516in}}%
\pgfpathlineto{\pgfqpoint{1.715264in}{2.100091in}}%
\pgfpathlineto{\pgfqpoint{1.715560in}{2.112899in}}%
\pgfpathlineto{\pgfqpoint{1.718224in}{2.108353in}}%
\pgfpathlineto{\pgfqpoint{1.718520in}{2.108539in}}%
\pgfpathlineto{\pgfqpoint{1.718816in}{2.113774in}}%
\pgfpathlineto{\pgfqpoint{1.720888in}{2.113982in}}%
\pgfpathlineto{\pgfqpoint{1.725920in}{2.118698in}}%
\pgfpathlineto{\pgfqpoint{1.726216in}{2.119506in}}%
\pgfpathlineto{\pgfqpoint{1.726512in}{2.121182in}}%
\pgfpathlineto{\pgfqpoint{1.727697in}{2.121254in}}%
\pgfpathlineto{\pgfqpoint{1.747233in}{2.120978in}}%
\pgfpathlineto{\pgfqpoint{1.782161in}{2.119855in}}%
\pgfpathlineto{\pgfqpoint{1.786009in}{2.119618in}}%
\pgfpathlineto{\pgfqpoint{1.791337in}{2.118922in}}%
\pgfpathlineto{\pgfqpoint{1.812058in}{2.119120in}}%
\pgfpathlineto{\pgfqpoint{1.818274in}{2.120968in}}%
\pgfpathlineto{\pgfqpoint{1.818570in}{2.119077in}}%
\pgfpathlineto{\pgfqpoint{1.819458in}{2.119204in}}%
\pgfpathlineto{\pgfqpoint{1.835146in}{2.123587in}}%
\pgfpathlineto{\pgfqpoint{1.839586in}{2.123587in}}%
\pgfpathlineto{\pgfqpoint{1.839882in}{2.123043in}}%
\pgfpathlineto{\pgfqpoint{1.840474in}{2.123587in}}%
\pgfpathlineto{\pgfqpoint{1.845210in}{2.123327in}}%
\pgfpathlineto{\pgfqpoint{1.875107in}{2.121409in}}%
\pgfpathlineto{\pgfqpoint{1.884579in}{2.123587in}}%
\pgfpathlineto{\pgfqpoint{1.932828in}{2.123832in}}%
\pgfpathlineto{\pgfqpoint{1.934604in}{2.124321in}}%
\pgfpathlineto{\pgfqpoint{1.944964in}{2.124520in}}%
\pgfpathlineto{\pgfqpoint{1.960060in}{2.124386in}}%
\pgfpathlineto{\pgfqpoint{1.960356in}{2.126626in}}%
\pgfpathlineto{\pgfqpoint{1.960652in}{2.125692in}}%
\pgfpathlineto{\pgfqpoint{1.961244in}{2.164552in}}%
\pgfpathlineto{\pgfqpoint{1.962428in}{2.200695in}}%
\pgfpathlineto{\pgfqpoint{1.963020in}{2.197355in}}%
\pgfpathlineto{\pgfqpoint{1.968644in}{2.158805in}}%
\pgfpathlineto{\pgfqpoint{1.968940in}{2.158575in}}%
\pgfpathlineto{\pgfqpoint{1.969236in}{2.158972in}}%
\pgfpathlineto{\pgfqpoint{1.971900in}{2.181670in}}%
\pgfpathlineto{\pgfqpoint{1.973972in}{2.199518in}}%
\pgfpathlineto{\pgfqpoint{1.974564in}{2.158575in}}%
\pgfpathlineto{\pgfqpoint{1.974860in}{2.171192in}}%
\pgfpathlineto{\pgfqpoint{1.975452in}{2.202543in}}%
\pgfpathlineto{\pgfqpoint{1.981076in}{2.164556in}}%
\pgfpathlineto{\pgfqpoint{1.981668in}{2.204293in}}%
\pgfpathlineto{\pgfqpoint{1.982260in}{2.174843in}}%
\pgfpathlineto{\pgfqpoint{1.982852in}{2.207244in}}%
\pgfpathlineto{\pgfqpoint{1.988772in}{2.163240in}}%
\pgfpathlineto{\pgfqpoint{1.989068in}{2.176026in}}%
\pgfpathlineto{\pgfqpoint{1.989364in}{2.206052in}}%
\pgfpathlineto{\pgfqpoint{1.989956in}{2.189117in}}%
\pgfpathlineto{\pgfqpoint{1.990252in}{2.208958in}}%
\pgfpathlineto{\pgfqpoint{2.031989in}{2.209157in}}%
\pgfpathlineto{\pgfqpoint{2.033469in}{2.209376in}}%
\pgfpathlineto{\pgfqpoint{2.038501in}{2.209130in}}%
\pgfpathlineto{\pgfqpoint{2.039093in}{2.209424in}}%
\pgfpathlineto{\pgfqpoint{2.075798in}{2.209579in}}%
\pgfpathlineto{\pgfqpoint{2.082014in}{2.213623in}}%
\pgfpathlineto{\pgfqpoint{2.087934in}{2.213623in}}%
\pgfpathlineto{\pgfqpoint{2.088526in}{2.220830in}}%
\pgfpathlineto{\pgfqpoint{2.089118in}{2.215015in}}%
\pgfpathlineto{\pgfqpoint{2.090006in}{2.223835in}}%
\pgfpathlineto{\pgfqpoint{2.091782in}{2.223886in}}%
\pgfpathlineto{\pgfqpoint{2.109838in}{2.223886in}}%
\pgfpathlineto{\pgfqpoint{2.110134in}{2.220159in}}%
\pgfpathlineto{\pgfqpoint{2.110726in}{2.228073in}}%
\pgfpathlineto{\pgfqpoint{2.111318in}{2.217091in}}%
\pgfpathlineto{\pgfqpoint{2.111614in}{2.219426in}}%
\pgfpathlineto{\pgfqpoint{2.111910in}{2.227704in}}%
\pgfpathlineto{\pgfqpoint{2.112502in}{2.228084in}}%
\pgfpathlineto{\pgfqpoint{2.123454in}{2.228225in}}%
\pgfpathlineto{\pgfqpoint{2.132039in}{2.232727in}}%
\pgfpathlineto{\pgfqpoint{2.139143in}{2.231883in}}%
\pgfpathlineto{\pgfqpoint{2.140327in}{2.232749in}}%
\pgfpathlineto{\pgfqpoint{2.158975in}{2.232750in}}%
\pgfpathlineto{\pgfqpoint{2.159863in}{2.233682in}}%
\pgfpathlineto{\pgfqpoint{2.161047in}{2.237200in}}%
\pgfpathlineto{\pgfqpoint{2.166375in}{2.239182in}}%
\pgfpathlineto{\pgfqpoint{2.166671in}{2.235658in}}%
\pgfpathlineto{\pgfqpoint{2.166967in}{2.235457in}}%
\pgfpathlineto{\pgfqpoint{2.167559in}{2.239281in}}%
\pgfpathlineto{\pgfqpoint{2.169631in}{2.239519in}}%
\pgfpathlineto{\pgfqpoint{2.173775in}{2.240168in}}%
\pgfpathlineto{\pgfqpoint{2.174663in}{2.250313in}}%
\pgfpathlineto{\pgfqpoint{2.176143in}{2.252327in}}%
\pgfpathlineto{\pgfqpoint{2.180287in}{2.252346in}}%
\pgfpathlineto{\pgfqpoint{2.180879in}{2.254067in}}%
\pgfpathlineto{\pgfqpoint{2.181175in}{2.255157in}}%
\pgfpathlineto{\pgfqpoint{2.181767in}{2.232632in}}%
\pgfpathlineto{\pgfqpoint{2.182063in}{2.233074in}}%
\pgfpathlineto{\pgfqpoint{2.197751in}{2.256406in}}%
\pgfpathlineto{\pgfqpoint{2.198343in}{2.255347in}}%
\pgfpathlineto{\pgfqpoint{2.208704in}{2.232463in}}%
\pgfpathlineto{\pgfqpoint{2.209296in}{2.268745in}}%
\pgfpathlineto{\pgfqpoint{2.210480in}{2.284065in}}%
\pgfpathlineto{\pgfqpoint{2.216992in}{2.284078in}}%
\pgfpathlineto{\pgfqpoint{2.217584in}{2.284532in}}%
\pgfpathlineto{\pgfqpoint{2.224096in}{2.284695in}}%
\pgfpathlineto{\pgfqpoint{2.224688in}{2.285465in}}%
\pgfpathlineto{\pgfqpoint{2.228536in}{2.285143in}}%
\pgfpathlineto{\pgfqpoint{2.230016in}{2.285012in}}%
\pgfpathlineto{\pgfqpoint{2.230608in}{2.288840in}}%
\pgfpathlineto{\pgfqpoint{2.231200in}{2.286621in}}%
\pgfpathlineto{\pgfqpoint{2.232680in}{2.309775in}}%
\pgfpathlineto{\pgfqpoint{2.237416in}{2.310624in}}%
\pgfpathlineto{\pgfqpoint{2.238304in}{2.310189in}}%
\pgfpathlineto{\pgfqpoint{2.239488in}{2.310436in}}%
\pgfpathlineto{\pgfqpoint{2.239784in}{2.310548in}}%
\pgfpathlineto{\pgfqpoint{2.243336in}{2.305474in}}%
\pgfpathlineto{\pgfqpoint{2.255176in}{2.288444in}}%
\pgfpathlineto{\pgfqpoint{2.255768in}{2.291875in}}%
\pgfpathlineto{\pgfqpoint{2.258432in}{2.311698in}}%
\pgfpathlineto{\pgfqpoint{2.259024in}{2.302455in}}%
\pgfpathlineto{\pgfqpoint{2.259616in}{2.290745in}}%
\pgfpathlineto{\pgfqpoint{2.259912in}{2.295380in}}%
\pgfpathlineto{\pgfqpoint{2.260504in}{2.309708in}}%
\pgfpathlineto{\pgfqpoint{2.261096in}{2.289830in}}%
\pgfpathlineto{\pgfqpoint{2.265536in}{2.312581in}}%
\pgfpathlineto{\pgfqpoint{2.265832in}{2.312049in}}%
\pgfpathlineto{\pgfqpoint{2.266129in}{2.307858in}}%
\pgfpathlineto{\pgfqpoint{2.266721in}{2.315655in}}%
\pgfpathlineto{\pgfqpoint{2.267905in}{2.317921in}}%
\pgfpathlineto{\pgfqpoint{2.272937in}{2.315836in}}%
\pgfpathlineto{\pgfqpoint{2.273233in}{2.316390in}}%
\pgfpathlineto{\pgfqpoint{2.275009in}{2.337247in}}%
\pgfpathlineto{\pgfqpoint{2.280633in}{2.337247in}}%
\pgfpathlineto{\pgfqpoint{2.280929in}{2.335630in}}%
\pgfpathlineto{\pgfqpoint{2.281225in}{2.329193in}}%
\pgfpathlineto{\pgfqpoint{2.282113in}{2.338646in}}%
\pgfpathlineto{\pgfqpoint{2.288329in}{2.338646in}}%
\pgfpathlineto{\pgfqpoint{2.289513in}{2.337713in}}%
\pgfpathlineto{\pgfqpoint{2.305201in}{2.337912in}}%
\pgfpathlineto{\pgfqpoint{2.309345in}{2.339113in}}%
\pgfpathlineto{\pgfqpoint{2.309641in}{2.339113in}}%
\pgfpathlineto{\pgfqpoint{2.309937in}{2.340130in}}%
\pgfpathlineto{\pgfqpoint{2.310529in}{2.339741in}}%
\pgfpathlineto{\pgfqpoint{2.311121in}{2.339788in}}%
\pgfpathlineto{\pgfqpoint{2.311713in}{2.339643in}}%
\pgfpathlineto{\pgfqpoint{2.315265in}{2.340042in}}%
\pgfpathlineto{\pgfqpoint{2.315857in}{2.340979in}}%
\pgfpathlineto{\pgfqpoint{2.331841in}{2.341142in}}%
\pgfpathlineto{\pgfqpoint{2.336874in}{2.343254in}}%
\pgfpathlineto{\pgfqpoint{2.337466in}{2.341507in}}%
\pgfpathlineto{\pgfqpoint{2.337762in}{2.342933in}}%
\pgfpathlineto{\pgfqpoint{2.338058in}{2.342946in}}%
\pgfpathlineto{\pgfqpoint{2.339242in}{2.341748in}}%
\pgfpathlineto{\pgfqpoint{2.358186in}{2.341389in}}%
\pgfpathlineto{\pgfqpoint{2.361442in}{2.343778in}}%
\pgfpathlineto{\pgfqpoint{2.364994in}{2.343778in}}%
\pgfpathlineto{\pgfqpoint{2.365290in}{2.343475in}}%
\pgfpathlineto{\pgfqpoint{2.365882in}{2.341900in}}%
\pgfpathlineto{\pgfqpoint{2.366178in}{2.344010in}}%
\pgfpathlineto{\pgfqpoint{2.367066in}{2.341703in}}%
\pgfpathlineto{\pgfqpoint{2.367362in}{2.344582in}}%
\pgfpathlineto{\pgfqpoint{2.368250in}{2.344526in}}%
\pgfpathlineto{\pgfqpoint{2.372986in}{2.343874in}}%
\pgfpathlineto{\pgfqpoint{2.381570in}{2.344711in}}%
\pgfpathlineto{\pgfqpoint{2.386306in}{2.344711in}}%
\pgfpathlineto{\pgfqpoint{2.387194in}{2.345644in}}%
\pgfpathlineto{\pgfqpoint{2.392226in}{2.345371in}}%
\pgfpathlineto{\pgfqpoint{2.402291in}{2.344879in}}%
\pgfpathlineto{\pgfqpoint{2.407915in}{2.345616in}}%
\pgfpathlineto{\pgfqpoint{2.408507in}{2.350309in}}%
\pgfpathlineto{\pgfqpoint{2.408803in}{2.350253in}}%
\pgfpathlineto{\pgfqpoint{2.409395in}{2.346150in}}%
\pgfpathlineto{\pgfqpoint{2.409691in}{2.346250in}}%
\pgfpathlineto{\pgfqpoint{2.411467in}{2.350232in}}%
\pgfpathlineto{\pgfqpoint{2.412651in}{2.350309in}}%
\pgfpathlineto{\pgfqpoint{2.415019in}{2.350382in}}%
\pgfpathlineto{\pgfqpoint{2.415611in}{2.354819in}}%
\pgfpathlineto{\pgfqpoint{2.416499in}{2.354974in}}%
\pgfpathlineto{\pgfqpoint{2.431595in}{2.355217in}}%
\pgfpathlineto{\pgfqpoint{2.432187in}{2.355397in}}%
\pgfpathlineto{\pgfqpoint{2.437219in}{2.355089in}}%
\pgfpathlineto{\pgfqpoint{2.439291in}{2.355331in}}%
\pgfpathlineto{\pgfqpoint{2.439587in}{2.356092in}}%
\pgfpathlineto{\pgfqpoint{2.439883in}{2.359460in}}%
\pgfpathlineto{\pgfqpoint{2.457347in}{2.363798in}}%
\pgfpathlineto{\pgfqpoint{2.459123in}{2.361944in}}%
\pgfpathlineto{\pgfqpoint{2.465339in}{2.355441in}}%
\pgfpathlineto{\pgfqpoint{2.466819in}{2.355453in}}%
\pgfpathlineto{\pgfqpoint{2.467707in}{2.356801in}}%
\pgfpathlineto{\pgfqpoint{2.472148in}{2.363838in}}%
\pgfpathlineto{\pgfqpoint{2.730559in}{2.364074in}}%
\pgfpathlineto{\pgfqpoint{2.739736in}{2.365162in}}%
\pgfpathlineto{\pgfqpoint{2.761344in}{2.363838in}}%
\pgfpathlineto{\pgfqpoint{2.765784in}{2.364107in}}%
\pgfpathlineto{\pgfqpoint{2.772592in}{2.365367in}}%
\pgfpathlineto{\pgfqpoint{2.778808in}{2.363838in}}%
\pgfpathlineto{\pgfqpoint{2.808113in}{2.364047in}}%
\pgfpathlineto{\pgfqpoint{2.813441in}{2.365660in}}%
\pgfpathlineto{\pgfqpoint{2.813737in}{2.365408in}}%
\pgfpathlineto{\pgfqpoint{2.814329in}{2.365704in}}%
\pgfpathlineto{\pgfqpoint{2.827057in}{2.365763in}}%
\pgfpathlineto{\pgfqpoint{2.828537in}{2.369436in}}%
\pgfpathlineto{\pgfqpoint{2.912306in}{2.369581in}}%
\pgfpathlineto{\pgfqpoint{2.912898in}{2.369902in}}%
\pgfpathlineto{\pgfqpoint{2.925922in}{2.369902in}}%
\pgfpathlineto{\pgfqpoint{2.926218in}{2.378386in}}%
\pgfpathlineto{\pgfqpoint{2.926810in}{2.430852in}}%
\pgfpathlineto{\pgfqpoint{2.927106in}{2.431141in}}%
\pgfpathlineto{\pgfqpoint{2.928882in}{2.420325in}}%
\pgfpathlineto{\pgfqpoint{2.935690in}{2.378361in}}%
\pgfpathlineto{\pgfqpoint{2.937170in}{2.378299in}}%
\pgfpathlineto{\pgfqpoint{2.954635in}{2.378299in}}%
\pgfpathlineto{\pgfqpoint{2.954931in}{2.390930in}}%
\pgfpathlineto{\pgfqpoint{2.955523in}{2.436029in}}%
\pgfpathlineto{\pgfqpoint{2.961739in}{2.379925in}}%
\pgfpathlineto{\pgfqpoint{2.962331in}{2.406850in}}%
\pgfpathlineto{\pgfqpoint{2.963219in}{2.407116in}}%
\pgfpathlineto{\pgfqpoint{2.963515in}{2.407728in}}%
\pgfpathlineto{\pgfqpoint{2.963811in}{2.409053in}}%
\pgfpathlineto{\pgfqpoint{2.969435in}{2.407974in}}%
\pgfpathlineto{\pgfqpoint{2.971211in}{2.399342in}}%
\pgfpathlineto{\pgfqpoint{2.977427in}{2.400692in}}%
\pgfpathlineto{\pgfqpoint{3.004659in}{2.400594in}}%
\pgfpathlineto{\pgfqpoint{3.005843in}{2.399759in}}%
\pgfpathlineto{\pgfqpoint{3.020348in}{2.399769in}}%
\pgfpathlineto{\pgfqpoint{3.021532in}{2.401312in}}%
\pgfpathlineto{\pgfqpoint{3.026860in}{2.409267in}}%
\pgfpathlineto{\pgfqpoint{3.027748in}{2.413335in}}%
\pgfpathlineto{\pgfqpoint{3.028044in}{2.411821in}}%
\pgfpathlineto{\pgfqpoint{3.028340in}{2.419144in}}%
\pgfpathlineto{\pgfqpoint{3.033372in}{2.400316in}}%
\pgfpathlineto{\pgfqpoint{3.033964in}{2.419895in}}%
\pgfpathlineto{\pgfqpoint{3.036332in}{2.420067in}}%
\pgfpathlineto{\pgfqpoint{3.054388in}{2.408622in}}%
\pgfpathlineto{\pgfqpoint{3.055868in}{2.408622in}}%
\pgfpathlineto{\pgfqpoint{3.056164in}{2.411514in}}%
\pgfpathlineto{\pgfqpoint{3.056460in}{2.419929in}}%
\pgfpathlineto{\pgfqpoint{3.059124in}{2.413837in}}%
\pgfpathlineto{\pgfqpoint{3.061196in}{2.409039in}}%
\pgfpathlineto{\pgfqpoint{3.062084in}{2.425063in}}%
\pgfpathlineto{\pgfqpoint{3.062380in}{2.423982in}}%
\pgfpathlineto{\pgfqpoint{3.062676in}{2.427717in}}%
\pgfpathlineto{\pgfqpoint{3.063268in}{2.424415in}}%
\pgfpathlineto{\pgfqpoint{3.063564in}{2.429971in}}%
\pgfpathlineto{\pgfqpoint{3.076589in}{2.422465in}}%
\pgfpathlineto{\pgfqpoint{3.098789in}{2.417952in}}%
\pgfpathlineto{\pgfqpoint{3.104117in}{2.417952in}}%
\pgfpathlineto{\pgfqpoint{3.104709in}{2.428216in}}%
\pgfpathlineto{\pgfqpoint{3.113885in}{2.428433in}}%
\pgfpathlineto{\pgfqpoint{3.134309in}{2.432870in}}%
\pgfpathlineto{\pgfqpoint{3.148518in}{2.429692in}}%
\pgfpathlineto{\pgfqpoint{3.154734in}{2.428299in}}%
\pgfpathlineto{\pgfqpoint{3.155030in}{2.428804in}}%
\pgfpathlineto{\pgfqpoint{3.155622in}{2.432253in}}%
\pgfpathlineto{\pgfqpoint{3.160654in}{2.448635in}}%
\pgfpathlineto{\pgfqpoint{3.161542in}{2.433482in}}%
\pgfpathlineto{\pgfqpoint{3.162134in}{2.446666in}}%
\pgfpathlineto{\pgfqpoint{3.163318in}{2.434790in}}%
\pgfpathlineto{\pgfqpoint{3.169830in}{2.450920in}}%
\pgfpathlineto{\pgfqpoint{3.170718in}{2.451074in}}%
\pgfpathlineto{\pgfqpoint{3.181966in}{2.451015in}}%
\pgfpathlineto{\pgfqpoint{3.182262in}{2.443577in}}%
\pgfpathlineto{\pgfqpoint{3.182558in}{2.451074in}}%
\pgfpathlineto{\pgfqpoint{3.184334in}{2.451299in}}%
\pgfpathlineto{\pgfqpoint{3.184926in}{2.451527in}}%
\pgfpathlineto{\pgfqpoint{3.203870in}{2.451170in}}%
\pgfpathlineto{\pgfqpoint{3.204462in}{2.451541in}}%
\pgfpathlineto{\pgfqpoint{3.274319in}{2.451541in}}%
\pgfpathlineto{\pgfqpoint{3.274615in}{2.454961in}}%
\pgfpathlineto{\pgfqpoint{3.274911in}{2.466881in}}%
\pgfpathlineto{\pgfqpoint{3.303032in}{2.459418in}}%
\pgfpathlineto{\pgfqpoint{3.316648in}{2.466859in}}%
\pgfpathlineto{\pgfqpoint{3.316944in}{2.467905in}}%
\pgfpathlineto{\pgfqpoint{3.317240in}{2.467218in}}%
\pgfpathlineto{\pgfqpoint{3.317832in}{2.468899in}}%
\pgfpathlineto{\pgfqpoint{3.318128in}{2.468494in}}%
\pgfpathlineto{\pgfqpoint{3.318424in}{2.467438in}}%
\pgfpathlineto{\pgfqpoint{3.319016in}{2.468175in}}%
\pgfpathlineto{\pgfqpoint{3.327008in}{2.469268in}}%
\pgfpathlineto{\pgfqpoint{3.354241in}{2.469532in}}%
\pgfpathlineto{\pgfqpoint{3.360161in}{2.470168in}}%
\pgfpathlineto{\pgfqpoint{3.360457in}{2.469280in}}%
\pgfpathlineto{\pgfqpoint{3.370225in}{2.470169in}}%
\pgfpathlineto{\pgfqpoint{3.373777in}{2.470631in}}%
\pgfpathlineto{\pgfqpoint{3.374369in}{2.471601in}}%
\pgfpathlineto{\pgfqpoint{3.374961in}{2.471574in}}%
\pgfpathlineto{\pgfqpoint{3.375553in}{2.470116in}}%
\pgfpathlineto{\pgfqpoint{3.376737in}{2.470201in}}%
\pgfpathlineto{\pgfqpoint{3.380881in}{2.470310in}}%
\pgfpathlineto{\pgfqpoint{3.381473in}{2.471435in}}%
\pgfpathlineto{\pgfqpoint{3.381769in}{2.472532in}}%
\pgfpathlineto{\pgfqpoint{3.409001in}{2.472534in}}%
\pgfpathlineto{\pgfqpoint{3.409593in}{2.470305in}}%
\pgfpathlineto{\pgfqpoint{3.410481in}{2.472534in}}%
\pgfpathlineto{\pgfqpoint{3.424394in}{2.472534in}}%
\pgfpathlineto{\pgfqpoint{3.424690in}{2.471944in}}%
\pgfpathlineto{\pgfqpoint{3.425282in}{2.472534in}}%
\pgfpathlineto{\pgfqpoint{3.451330in}{2.472650in}}%
\pgfpathlineto{\pgfqpoint{3.451922in}{2.476358in}}%
\pgfpathlineto{\pgfqpoint{3.452218in}{2.476880in}}%
\pgfpathlineto{\pgfqpoint{3.459322in}{2.474940in}}%
\pgfpathlineto{\pgfqpoint{3.459914in}{2.478598in}}%
\pgfpathlineto{\pgfqpoint{3.460802in}{2.478598in}}%
\pgfpathlineto{\pgfqpoint{3.461098in}{2.476679in}}%
\pgfpathlineto{\pgfqpoint{3.461394in}{2.478589in}}%
\pgfpathlineto{\pgfqpoint{3.470866in}{2.478598in}}%
\pgfpathlineto{\pgfqpoint{3.473530in}{2.478418in}}%
\pgfpathlineto{\pgfqpoint{3.475602in}{2.475334in}}%
\pgfpathlineto{\pgfqpoint{3.501355in}{2.475333in}}%
\pgfpathlineto{\pgfqpoint{3.502243in}{2.479065in}}%
\pgfpathlineto{\pgfqpoint{3.517635in}{2.479339in}}%
\pgfpathlineto{\pgfqpoint{3.529475in}{2.479531in}}%
\pgfpathlineto{\pgfqpoint{3.607916in}{2.479531in}}%
\pgfpathlineto{\pgfqpoint{3.608508in}{2.475429in}}%
\pgfpathlineto{\pgfqpoint{3.608804in}{2.475542in}}%
\pgfpathlineto{\pgfqpoint{3.609692in}{2.475877in}}%
\pgfpathlineto{\pgfqpoint{3.609988in}{2.480464in}}%
\pgfpathlineto{\pgfqpoint{3.650541in}{2.480464in}}%
\pgfpathlineto{\pgfqpoint{3.651429in}{2.481888in}}%
\pgfpathlineto{\pgfqpoint{3.657941in}{2.482280in}}%
\pgfpathlineto{\pgfqpoint{3.660013in}{2.480464in}}%
\pgfpathlineto{\pgfqpoint{3.660605in}{2.481076in}}%
\pgfpathlineto{\pgfqpoint{3.661789in}{2.482341in}}%
\pgfpathlineto{\pgfqpoint{3.673333in}{2.482797in}}%
\pgfpathlineto{\pgfqpoint{3.673925in}{2.479531in}}%
\pgfpathlineto{\pgfqpoint{3.678957in}{2.479694in}}%
\pgfpathlineto{\pgfqpoint{3.679549in}{2.480556in}}%
\pgfpathlineto{\pgfqpoint{3.679845in}{2.479536in}}%
\pgfpathlineto{\pgfqpoint{3.680733in}{2.479716in}}%
\pgfpathlineto{\pgfqpoint{3.681325in}{2.483197in}}%
\pgfpathlineto{\pgfqpoint{3.682510in}{2.483547in}}%
\pgfpathlineto{\pgfqpoint{3.701750in}{2.487893in}}%
\pgfpathlineto{\pgfqpoint{3.702934in}{2.486277in}}%
\pgfpathlineto{\pgfqpoint{3.707670in}{2.479531in}}%
\pgfpathlineto{\pgfqpoint{3.709150in}{2.479564in}}%
\pgfpathlineto{\pgfqpoint{3.710038in}{2.487890in}}%
\pgfpathlineto{\pgfqpoint{3.710334in}{2.487831in}}%
\pgfpathlineto{\pgfqpoint{3.715958in}{2.486995in}}%
\pgfpathlineto{\pgfqpoint{3.721582in}{2.486995in}}%
\pgfpathlineto{\pgfqpoint{3.721878in}{2.487400in}}%
\pgfpathlineto{\pgfqpoint{3.722174in}{2.488861in}}%
\pgfpathlineto{\pgfqpoint{3.727502in}{2.488595in}}%
\pgfpathlineto{\pgfqpoint{3.752367in}{2.487069in}}%
\pgfpathlineto{\pgfqpoint{3.756807in}{2.488738in}}%
\pgfpathlineto{\pgfqpoint{3.757695in}{2.490261in}}%
\pgfpathlineto{\pgfqpoint{3.759471in}{2.490032in}}%
\pgfpathlineto{\pgfqpoint{3.765095in}{2.487071in}}%
\pgfpathlineto{\pgfqpoint{3.766575in}{2.492636in}}%
\pgfpathlineto{\pgfqpoint{3.771607in}{2.493042in}}%
\pgfpathlineto{\pgfqpoint{3.771903in}{2.492760in}}%
\pgfpathlineto{\pgfqpoint{3.772199in}{2.492866in}}%
\pgfpathlineto{\pgfqpoint{3.772495in}{2.491983in}}%
\pgfpathlineto{\pgfqpoint{3.772791in}{2.490216in}}%
\pgfpathlineto{\pgfqpoint{3.773383in}{2.493060in}}%
\pgfpathlineto{\pgfqpoint{3.799727in}{2.493060in}}%
\pgfpathlineto{\pgfqpoint{3.800319in}{2.493526in}}%
\pgfpathlineto{\pgfqpoint{3.804759in}{2.493249in}}%
\pgfpathlineto{\pgfqpoint{3.806831in}{2.493081in}}%
\pgfpathlineto{\pgfqpoint{3.807719in}{2.494829in}}%
\pgfpathlineto{\pgfqpoint{3.808015in}{2.494926in}}%
\pgfpathlineto{\pgfqpoint{3.809495in}{2.493526in}}%
\pgfpathlineto{\pgfqpoint{3.845312in}{2.493757in}}%
\pgfpathlineto{\pgfqpoint{3.856856in}{2.497520in}}%
\pgfpathlineto{\pgfqpoint{3.859224in}{2.493526in}}%
\pgfpathlineto{\pgfqpoint{3.870768in}{2.493526in}}%
\pgfpathlineto{\pgfqpoint{3.871360in}{2.496827in}}%
\pgfpathlineto{\pgfqpoint{3.872248in}{2.497936in}}%
\pgfpathlineto{\pgfqpoint{3.872840in}{2.493526in}}%
\pgfpathlineto{\pgfqpoint{3.878168in}{2.493620in}}%
\pgfpathlineto{\pgfqpoint{3.878464in}{2.493820in}}%
\pgfpathlineto{\pgfqpoint{3.879056in}{2.493526in}}%
\pgfpathlineto{\pgfqpoint{3.899185in}{2.493721in}}%
\pgfpathlineto{\pgfqpoint{3.899481in}{2.494402in}}%
\pgfpathlineto{\pgfqpoint{3.900073in}{2.499631in}}%
\pgfpathlineto{\pgfqpoint{3.901257in}{2.507727in}}%
\pgfpathlineto{\pgfqpoint{3.907473in}{2.515229in}}%
\pgfpathlineto{\pgfqpoint{3.907769in}{2.514401in}}%
\pgfpathlineto{\pgfqpoint{3.908657in}{2.506205in}}%
\pgfpathlineto{\pgfqpoint{3.909249in}{2.519284in}}%
\pgfpathlineto{\pgfqpoint{3.913097in}{2.516150in}}%
\pgfpathlineto{\pgfqpoint{3.913985in}{2.520584in}}%
\pgfpathlineto{\pgfqpoint{3.950393in}{2.520584in}}%
\pgfpathlineto{\pgfqpoint{3.950689in}{2.521587in}}%
\pgfpathlineto{\pgfqpoint{3.951874in}{2.642828in}}%
\pgfpathlineto{\pgfqpoint{3.971706in}{2.643729in}}%
\pgfpathlineto{\pgfqpoint{3.972002in}{2.638879in}}%
\pgfpathlineto{\pgfqpoint{3.973482in}{2.520584in}}%
\pgfpathlineto{\pgfqpoint{4.005450in}{2.520584in}}%
\pgfpathlineto{\pgfqpoint{4.005746in}{2.527033in}}%
\pgfpathlineto{\pgfqpoint{4.006634in}{2.640580in}}%
\pgfpathlineto{\pgfqpoint{4.012850in}{2.520594in}}%
\pgfpathlineto{\pgfqpoint{4.021435in}{2.520584in}}%
\pgfpathlineto{\pgfqpoint{4.021731in}{2.520584in}}%
\pgfpathlineto{\pgfqpoint{4.022027in}{2.520876in}}%
\pgfpathlineto{\pgfqpoint{4.026763in}{2.542408in}}%
\pgfpathlineto{\pgfqpoint{4.048963in}{2.643362in}}%
\pgfpathlineto{\pgfqpoint{4.049851in}{2.638646in}}%
\pgfpathlineto{\pgfqpoint{4.070275in}{2.520785in}}%
\pgfpathlineto{\pgfqpoint{4.070867in}{2.521173in}}%
\pgfpathlineto{\pgfqpoint{4.071459in}{2.521510in}}%
\pgfpathlineto{\pgfqpoint{4.107276in}{2.520706in}}%
\pgfpathlineto{\pgfqpoint{4.111124in}{2.529230in}}%
\pgfpathlineto{\pgfqpoint{4.111716in}{2.530549in}}%
\pgfpathlineto{\pgfqpoint{4.112012in}{2.545099in}}%
\pgfpathlineto{\pgfqpoint{4.112308in}{2.651832in}}%
\pgfpathlineto{\pgfqpoint{4.112604in}{2.654471in}}%
\pgfpathlineto{\pgfqpoint{4.119412in}{2.654600in}}%
\pgfpathlineto{\pgfqpoint{4.120300in}{2.655400in}}%
\pgfpathlineto{\pgfqpoint{4.129476in}{2.655762in}}%
\pgfpathlineto{\pgfqpoint{4.146940in}{2.667324in}}%
\pgfpathlineto{\pgfqpoint{4.148716in}{2.668326in}}%
\pgfpathlineto{\pgfqpoint{4.153156in}{2.659791in}}%
\pgfpathlineto{\pgfqpoint{4.155229in}{2.656301in}}%
\pgfpathlineto{\pgfqpoint{4.162333in}{2.655404in}}%
\pgfpathlineto{\pgfqpoint{4.162629in}{2.656027in}}%
\pgfpathlineto{\pgfqpoint{4.163221in}{2.665083in}}%
\pgfpathlineto{\pgfqpoint{4.163517in}{2.665031in}}%
\pgfpathlineto{\pgfqpoint{4.164109in}{2.655916in}}%
\pgfpathlineto{\pgfqpoint{4.164405in}{2.656136in}}%
\pgfpathlineto{\pgfqpoint{4.169733in}{2.667533in}}%
\pgfpathlineto{\pgfqpoint{4.170325in}{2.667533in}}%
\pgfpathlineto{\pgfqpoint{4.170621in}{2.667175in}}%
\pgfpathlineto{\pgfqpoint{4.170917in}{2.658077in}}%
\pgfpathlineto{\pgfqpoint{4.171509in}{2.667042in}}%
\pgfpathlineto{\pgfqpoint{4.171805in}{2.666951in}}%
\pgfpathlineto{\pgfqpoint{4.178613in}{2.655620in}}%
\pgfpathlineto{\pgfqpoint{4.178909in}{2.655481in}}%
\pgfpathlineto{\pgfqpoint{4.194005in}{2.665993in}}%
\pgfpathlineto{\pgfqpoint{4.197261in}{2.668262in}}%
\pgfpathlineto{\pgfqpoint{4.197853in}{2.658571in}}%
\pgfpathlineto{\pgfqpoint{4.198741in}{2.668354in}}%
\pgfpathlineto{\pgfqpoint{4.199037in}{2.668954in}}%
\pgfpathlineto{\pgfqpoint{4.199333in}{2.668512in}}%
\pgfpathlineto{\pgfqpoint{4.199925in}{2.656364in}}%
\pgfpathlineto{\pgfqpoint{4.201997in}{2.655914in}}%
\pgfpathlineto{\pgfqpoint{4.204365in}{2.655455in}}%
\pgfpathlineto{\pgfqpoint{4.204661in}{2.657333in}}%
\pgfpathlineto{\pgfqpoint{4.205253in}{2.668727in}}%
\pgfpathlineto{\pgfqpoint{4.207325in}{2.671292in}}%
\pgfpathlineto{\pgfqpoint{4.213837in}{2.671732in}}%
\pgfpathlineto{\pgfqpoint{4.263270in}{2.671732in}}%
\pgfpathlineto{\pgfqpoint{4.263566in}{2.672240in}}%
\pgfpathlineto{\pgfqpoint{4.265638in}{2.672024in}}%
\pgfpathlineto{\pgfqpoint{4.270078in}{2.671925in}}%
\pgfpathlineto{\pgfqpoint{4.271262in}{2.672267in}}%
\pgfpathlineto{\pgfqpoint{4.271558in}{2.673131in}}%
\pgfpathlineto{\pgfqpoint{4.302935in}{2.673398in}}%
\pgfpathlineto{\pgfqpoint{4.312111in}{2.673598in}}%
\pgfpathlineto{\pgfqpoint{4.328391in}{2.673865in}}%
\pgfpathlineto{\pgfqpoint{4.354735in}{2.679896in}}%
\pgfpathlineto{\pgfqpoint{4.355327in}{2.680129in}}%
\pgfpathlineto{\pgfqpoint{4.378120in}{2.679908in}}%
\pgfpathlineto{\pgfqpoint{4.399136in}{2.673598in}}%
\pgfpathlineto{\pgfqpoint{4.410384in}{2.673598in}}%
\pgfpathlineto{\pgfqpoint{4.411272in}{2.680129in}}%
\pgfpathlineto{\pgfqpoint{4.430513in}{2.680406in}}%
\pgfpathlineto{\pgfqpoint{4.457449in}{2.680830in}}%
\pgfpathlineto{\pgfqpoint{4.461593in}{2.680169in}}%
\pgfpathlineto{\pgfqpoint{4.462777in}{2.681452in}}%
\pgfpathlineto{\pgfqpoint{4.463961in}{2.681528in}}%
\pgfpathlineto{\pgfqpoint{4.497410in}{2.681528in}}%
\pgfpathlineto{\pgfqpoint{4.498002in}{2.680183in}}%
\pgfpathlineto{\pgfqpoint{4.504810in}{2.681453in}}%
\pgfpathlineto{\pgfqpoint{4.505698in}{2.680606in}}%
\pgfpathlineto{\pgfqpoint{4.511914in}{2.681528in}}%
\pgfpathlineto{\pgfqpoint{4.674716in}{2.681795in}}%
\pgfpathlineto{\pgfqpoint{4.675604in}{2.681942in}}%
\pgfpathlineto{\pgfqpoint{4.675900in}{2.681687in}}%
\pgfpathlineto{\pgfqpoint{4.676492in}{2.681995in}}%
\pgfpathlineto{\pgfqpoint{4.724445in}{2.681995in}}%
\pgfpathlineto{\pgfqpoint{4.725037in}{2.684328in}}%
\pgfpathlineto{\pgfqpoint{4.737765in}{2.684633in}}%
\pgfpathlineto{\pgfqpoint{4.744869in}{2.684623in}}%
\pgfpathlineto{\pgfqpoint{4.745461in}{2.682396in}}%
\pgfpathlineto{\pgfqpoint{4.746053in}{2.685261in}}%
\pgfpathlineto{\pgfqpoint{4.746349in}{2.684743in}}%
\pgfpathlineto{\pgfqpoint{4.746941in}{2.685261in}}%
\pgfpathlineto{\pgfqpoint{4.747533in}{2.685046in}}%
\pgfpathlineto{\pgfqpoint{4.751973in}{2.682145in}}%
\pgfpathlineto{\pgfqpoint{4.752565in}{2.685279in}}%
\pgfpathlineto{\pgfqpoint{4.752861in}{2.685447in}}%
\pgfpathlineto{\pgfqpoint{4.753157in}{2.685280in}}%
\pgfpathlineto{\pgfqpoint{4.753453in}{2.682908in}}%
\pgfpathlineto{\pgfqpoint{4.754045in}{2.685260in}}%
\pgfpathlineto{\pgfqpoint{4.758781in}{2.685261in}}%
\pgfpathlineto{\pgfqpoint{4.759077in}{2.683737in}}%
\pgfpathlineto{\pgfqpoint{4.759373in}{2.684118in}}%
\pgfpathlineto{\pgfqpoint{4.759669in}{2.685260in}}%
\pgfpathlineto{\pgfqpoint{4.760854in}{2.685402in}}%
\pgfpathlineto{\pgfqpoint{4.761446in}{2.686151in}}%
\pgfpathlineto{\pgfqpoint{4.764110in}{2.685712in}}%
\pgfpathlineto{\pgfqpoint{4.766478in}{2.685294in}}%
\pgfpathlineto{\pgfqpoint{4.766774in}{2.712157in}}%
\pgfpathlineto{\pgfqpoint{4.767070in}{2.789153in}}%
\pgfpathlineto{\pgfqpoint{4.767366in}{2.764836in}}%
\pgfpathlineto{\pgfqpoint{4.767662in}{2.827078in}}%
\pgfpathlineto{\pgfqpoint{4.775062in}{2.827264in}}%
\pgfpathlineto{\pgfqpoint{4.775654in}{2.827488in}}%
\pgfpathlineto{\pgfqpoint{4.775950in}{2.826407in}}%
\pgfpathlineto{\pgfqpoint{4.794302in}{2.683104in}}%
\pgfpathlineto{\pgfqpoint{4.794894in}{2.728527in}}%
\pgfpathlineto{\pgfqpoint{4.796078in}{2.827078in}}%
\pgfpathlineto{\pgfqpoint{4.801406in}{2.827078in}}%
\pgfpathlineto{\pgfqpoint{4.801998in}{2.831988in}}%
\pgfpathlineto{\pgfqpoint{4.802294in}{2.827211in}}%
\pgfpathlineto{\pgfqpoint{4.802590in}{2.828424in}}%
\pgfpathlineto{\pgfqpoint{4.803182in}{2.841540in}}%
\pgfpathlineto{\pgfqpoint{4.804958in}{2.841540in}}%
\pgfpathlineto{\pgfqpoint{4.805254in}{2.836243in}}%
\pgfpathlineto{\pgfqpoint{4.805550in}{2.691506in}}%
\pgfpathlineto{\pgfqpoint{4.808510in}{2.840277in}}%
\pgfpathlineto{\pgfqpoint{4.808806in}{2.841515in}}%
\pgfpathlineto{\pgfqpoint{4.809398in}{2.841119in}}%
\pgfpathlineto{\pgfqpoint{4.810582in}{2.841507in}}%
\pgfpathlineto{\pgfqpoint{4.811470in}{2.841307in}}%
\pgfpathlineto{\pgfqpoint{4.816502in}{2.841039in}}%
\pgfpathlineto{\pgfqpoint{4.823014in}{2.840831in}}%
\pgfpathlineto{\pgfqpoint{4.824198in}{2.841530in}}%
\pgfpathlineto{\pgfqpoint{4.832487in}{2.841540in}}%
\pgfpathlineto{\pgfqpoint{4.845807in}{2.841541in}}%
\pgfpathlineto{\pgfqpoint{4.846399in}{2.843029in}}%
\pgfpathlineto{\pgfqpoint{4.852615in}{2.844806in}}%
\pgfpathlineto{\pgfqpoint{4.853207in}{2.845029in}}%
\pgfpathlineto{\pgfqpoint{4.854095in}{2.845706in}}%
\pgfpathlineto{\pgfqpoint{4.858831in}{2.844297in}}%
\pgfpathlineto{\pgfqpoint{4.859127in}{2.841417in}}%
\pgfpathlineto{\pgfqpoint{4.860015in}{2.845708in}}%
\pgfpathlineto{\pgfqpoint{4.860903in}{2.904972in}}%
\pgfpathlineto{\pgfqpoint{4.861199in}{2.866237in}}%
\pgfpathlineto{\pgfqpoint{4.862087in}{2.906851in}}%
\pgfpathlineto{\pgfqpoint{4.865343in}{2.906929in}}%
\pgfpathlineto{\pgfqpoint{4.865639in}{2.907755in}}%
\pgfpathlineto{\pgfqpoint{4.866231in}{2.907108in}}%
\pgfpathlineto{\pgfqpoint{4.866823in}{2.908251in}}%
\pgfpathlineto{\pgfqpoint{4.868007in}{2.908251in}}%
\pgfpathlineto{\pgfqpoint{4.873927in}{2.908251in}}%
\pgfpathlineto{\pgfqpoint{4.874519in}{2.908709in}}%
\pgfpathlineto{\pgfqpoint{4.894351in}{2.908254in}}%
\pgfpathlineto{\pgfqpoint{4.894943in}{2.881499in}}%
\pgfpathlineto{\pgfqpoint{4.895240in}{2.908251in}}%
\pgfpathlineto{\pgfqpoint{4.896128in}{2.908477in}}%
\pgfpathlineto{\pgfqpoint{4.898200in}{2.908717in}}%
\pgfpathlineto{\pgfqpoint{4.901752in}{2.908520in}}%
\pgfpathlineto{\pgfqpoint{4.902048in}{2.907311in}}%
\pgfpathlineto{\pgfqpoint{4.902344in}{2.907674in}}%
\pgfpathlineto{\pgfqpoint{4.902640in}{2.908515in}}%
\pgfpathlineto{\pgfqpoint{4.902936in}{2.908406in}}%
\pgfpathlineto{\pgfqpoint{4.903528in}{2.908717in}}%
\pgfpathlineto{\pgfqpoint{4.904120in}{2.907744in}}%
\pgfpathlineto{\pgfqpoint{4.907968in}{2.906938in}}%
\pgfpathlineto{\pgfqpoint{4.908264in}{2.908126in}}%
\pgfpathlineto{\pgfqpoint{4.908560in}{2.921441in}}%
\pgfpathlineto{\pgfqpoint{4.909448in}{2.910342in}}%
\pgfpathlineto{\pgfqpoint{4.909744in}{2.910795in}}%
\pgfpathlineto{\pgfqpoint{4.910336in}{2.922246in}}%
\pgfpathlineto{\pgfqpoint{4.922768in}{2.922246in}}%
\pgfpathlineto{\pgfqpoint{4.923064in}{2.921065in}}%
\pgfpathlineto{\pgfqpoint{4.923952in}{2.910285in}}%
\pgfpathlineto{\pgfqpoint{4.924544in}{2.922246in}}%
\pgfpathlineto{\pgfqpoint{4.945856in}{2.922501in}}%
\pgfpathlineto{\pgfqpoint{4.946448in}{2.922620in}}%
\pgfpathlineto{\pgfqpoint{4.950888in}{2.921809in}}%
\pgfpathlineto{\pgfqpoint{4.951480in}{2.922712in}}%
\pgfpathlineto{\pgfqpoint{4.955920in}{2.922459in}}%
\pgfpathlineto{\pgfqpoint{4.958288in}{2.922246in}}%
\pgfpathlineto{\pgfqpoint{4.958584in}{2.936703in}}%
\pgfpathlineto{\pgfqpoint{4.959176in}{2.972054in}}%
\pgfpathlineto{\pgfqpoint{4.959472in}{2.942062in}}%
\pgfpathlineto{\pgfqpoint{4.960064in}{2.981052in}}%
\pgfpathlineto{\pgfqpoint{4.965985in}{2.981468in}}%
\pgfpathlineto{\pgfqpoint{4.966281in}{2.980018in}}%
\pgfpathlineto{\pgfqpoint{4.966577in}{2.947694in}}%
\pgfpathlineto{\pgfqpoint{4.966873in}{2.976074in}}%
\pgfpathlineto{\pgfqpoint{4.967465in}{2.937325in}}%
\pgfpathlineto{\pgfqpoint{4.968057in}{2.981492in}}%
\pgfpathlineto{\pgfqpoint{4.995289in}{2.981492in}}%
\pgfpathlineto{\pgfqpoint{4.995585in}{2.981138in}}%
\pgfpathlineto{\pgfqpoint{4.996177in}{2.981386in}}%
\pgfpathlineto{\pgfqpoint{4.996473in}{2.980397in}}%
\pgfpathlineto{\pgfqpoint{4.997065in}{2.977345in}}%
\pgfpathlineto{\pgfqpoint{5.001801in}{2.978666in}}%
\pgfpathlineto{\pgfqpoint{5.002097in}{2.979403in}}%
\pgfpathlineto{\pgfqpoint{5.002393in}{2.979382in}}%
\pgfpathlineto{\pgfqpoint{5.002985in}{2.978693in}}%
\pgfpathlineto{\pgfqpoint{5.010977in}{2.978698in}}%
\pgfpathlineto{\pgfqpoint{5.011569in}{2.980689in}}%
\pgfpathlineto{\pgfqpoint{5.015713in}{2.997353in}}%
\pgfpathlineto{\pgfqpoint{5.016897in}{2.997353in}}%
\pgfpathlineto{\pgfqpoint{5.017193in}{2.995221in}}%
\pgfpathlineto{\pgfqpoint{5.018081in}{3.006959in}}%
\pgfpathlineto{\pgfqpoint{5.025185in}{3.009949in}}%
\pgfpathlineto{\pgfqpoint{5.046498in}{3.009949in}}%
\pgfpathlineto{\pgfqpoint{5.046794in}{2.966867in}}%
\pgfpathlineto{\pgfqpoint{5.047090in}{2.838989in}}%
\pgfpathlineto{\pgfqpoint{5.050346in}{2.988034in}}%
\pgfpathlineto{\pgfqpoint{5.050642in}{2.831744in}}%
\pgfpathlineto{\pgfqpoint{5.072842in}{2.831744in}}%
\pgfpathlineto{\pgfqpoint{5.073138in}{2.832283in}}%
\pgfpathlineto{\pgfqpoint{5.074026in}{2.841987in}}%
\pgfpathlineto{\pgfqpoint{5.074618in}{2.839995in}}%
\pgfpathlineto{\pgfqpoint{5.075210in}{2.835303in}}%
\pgfpathlineto{\pgfqpoint{5.075506in}{2.835392in}}%
\pgfpathlineto{\pgfqpoint{5.092674in}{2.840521in}}%
\pgfpathlineto{\pgfqpoint{5.092970in}{2.840311in}}%
\pgfpathlineto{\pgfqpoint{5.093562in}{2.838741in}}%
\pgfpathlineto{\pgfqpoint{5.101555in}{2.838501in}}%
\pgfpathlineto{\pgfqpoint{5.102443in}{2.838321in}}%
\pgfpathlineto{\pgfqpoint{5.102739in}{2.837243in}}%
\pgfpathlineto{\pgfqpoint{5.103035in}{2.833130in}}%
\pgfpathlineto{\pgfqpoint{5.103627in}{2.812150in}}%
\pgfpathlineto{\pgfqpoint{5.121091in}{2.812150in}}%
\pgfpathlineto{\pgfqpoint{5.121387in}{2.812500in}}%
\pgfpathlineto{\pgfqpoint{5.122867in}{2.819115in}}%
\pgfpathlineto{\pgfqpoint{5.123163in}{2.818940in}}%
\pgfpathlineto{\pgfqpoint{5.123459in}{2.819283in}}%
\pgfpathlineto{\pgfqpoint{5.124643in}{2.830810in}}%
\pgfpathlineto{\pgfqpoint{5.144179in}{2.830810in}}%
\pgfpathlineto{\pgfqpoint{5.144771in}{2.838275in}}%
\pgfpathlineto{\pgfqpoint{5.152763in}{2.838275in}}%
\pgfpathlineto{\pgfqpoint{5.153355in}{2.831330in}}%
\pgfpathlineto{\pgfqpoint{5.157499in}{2.837993in}}%
\pgfpathlineto{\pgfqpoint{5.157795in}{2.837633in}}%
\pgfpathlineto{\pgfqpoint{5.158979in}{2.832010in}}%
\pgfpathlineto{\pgfqpoint{5.159275in}{2.838275in}}%
\pgfpathlineto{\pgfqpoint{5.174372in}{2.838035in}}%
\pgfpathlineto{\pgfqpoint{5.194204in}{2.829478in}}%
\pgfpathlineto{\pgfqpoint{5.194796in}{2.841459in}}%
\pgfpathlineto{\pgfqpoint{5.195092in}{2.843163in}}%
\pgfpathlineto{\pgfqpoint{5.195388in}{2.842934in}}%
\pgfpathlineto{\pgfqpoint{5.196276in}{2.845667in}}%
\pgfpathlineto{\pgfqpoint{5.199236in}{2.848224in}}%
\pgfpathlineto{\pgfqpoint{5.199532in}{2.848880in}}%
\pgfpathlineto{\pgfqpoint{5.200124in}{2.847502in}}%
\pgfpathlineto{\pgfqpoint{5.200420in}{2.850404in}}%
\pgfpathlineto{\pgfqpoint{5.307574in}{2.850407in}}%
\pgfpathlineto{\pgfqpoint{5.308758in}{2.851803in}}%
\pgfpathlineto{\pgfqpoint{5.314086in}{2.851550in}}%
\pgfpathlineto{\pgfqpoint{5.315566in}{2.850429in}}%
\pgfpathlineto{\pgfqpoint{5.321782in}{2.851761in}}%
\pgfpathlineto{\pgfqpoint{5.322078in}{2.850918in}}%
\pgfpathlineto{\pgfqpoint{5.322966in}{2.851803in}}%
\pgfpathlineto{\pgfqpoint{5.331550in}{2.851225in}}%
\pgfpathlineto{\pgfqpoint{5.343390in}{2.850415in}}%
\pgfpathlineto{\pgfqpoint{5.343982in}{2.851803in}}%
\pgfpathlineto{\pgfqpoint{5.443440in}{2.851803in}}%
\pgfpathlineto{\pgfqpoint{5.443736in}{2.852357in}}%
\pgfpathlineto{\pgfqpoint{5.444920in}{2.856257in}}%
\pgfpathlineto{\pgfqpoint{5.448768in}{2.851803in}}%
\pgfpathlineto{\pgfqpoint{5.449360in}{2.856468in}}%
\pgfpathlineto{\pgfqpoint{5.449952in}{2.856468in}}%
\pgfpathlineto{\pgfqpoint{5.450248in}{2.856104in}}%
\pgfpathlineto{\pgfqpoint{5.450544in}{2.853683in}}%
\pgfpathlineto{\pgfqpoint{5.450840in}{2.856468in}}%
\pgfpathlineto{\pgfqpoint{5.451136in}{2.856468in}}%
\pgfpathlineto{\pgfqpoint{5.451432in}{2.860450in}}%
\pgfpathlineto{\pgfqpoint{5.451728in}{2.868616in}}%
\pgfpathlineto{\pgfqpoint{5.452320in}{2.867675in}}%
\pgfpathlineto{\pgfqpoint{5.457056in}{2.857375in}}%
\pgfpathlineto{\pgfqpoint{5.457352in}{2.857782in}}%
\pgfpathlineto{\pgfqpoint{5.457648in}{2.868717in}}%
\pgfpathlineto{\pgfqpoint{5.463568in}{2.856717in}}%
\pgfpathlineto{\pgfqpoint{5.464160in}{2.861209in}}%
\pgfpathlineto{\pgfqpoint{5.465048in}{2.868957in}}%
\pgfpathlineto{\pgfqpoint{5.465936in}{2.869064in}}%
\pgfpathlineto{\pgfqpoint{5.470080in}{2.869064in}}%
\pgfpathlineto{\pgfqpoint{5.470376in}{2.869686in}}%
\pgfpathlineto{\pgfqpoint{5.472152in}{2.883059in}}%
\pgfpathlineto{\pgfqpoint{5.493168in}{2.883059in}}%
\pgfpathlineto{\pgfqpoint{5.493464in}{2.882551in}}%
\pgfpathlineto{\pgfqpoint{5.498792in}{2.858811in}}%
\pgfpathlineto{\pgfqpoint{5.499384in}{2.865832in}}%
\pgfpathlineto{\pgfqpoint{5.500568in}{2.882564in}}%
\pgfpathlineto{\pgfqpoint{5.500864in}{2.882416in}}%
\pgfpathlineto{\pgfqpoint{5.512705in}{2.858598in}}%
\pgfpathlineto{\pgfqpoint{5.513297in}{2.864085in}}%
\pgfpathlineto{\pgfqpoint{5.515073in}{2.882998in}}%
\pgfpathlineto{\pgfqpoint{5.516553in}{2.883059in}}%
\pgfpathlineto{\pgfqpoint{5.521881in}{2.883059in}}%
\pgfpathlineto{\pgfqpoint{5.522177in}{2.882662in}}%
\pgfpathlineto{\pgfqpoint{5.522473in}{2.874986in}}%
\pgfpathlineto{\pgfqpoint{5.541121in}{2.858537in}}%
\pgfpathlineto{\pgfqpoint{5.542009in}{2.869103in}}%
\pgfpathlineto{\pgfqpoint{5.543193in}{2.882972in}}%
\pgfpathlineto{\pgfqpoint{5.548521in}{2.879752in}}%
\pgfpathlineto{\pgfqpoint{5.549113in}{2.869173in}}%
\pgfpathlineto{\pgfqpoint{5.549409in}{2.862637in}}%
\pgfpathlineto{\pgfqpoint{5.549705in}{2.863005in}}%
\pgfpathlineto{\pgfqpoint{5.550297in}{2.880888in}}%
\pgfpathlineto{\pgfqpoint{5.550889in}{2.880918in}}%
\pgfpathlineto{\pgfqpoint{5.551185in}{2.881530in}}%
\pgfpathlineto{\pgfqpoint{5.555033in}{2.879915in}}%
\pgfpathlineto{\pgfqpoint{5.555921in}{2.881660in}}%
\pgfpathlineto{\pgfqpoint{5.593218in}{2.881660in}}%
\pgfpathlineto{\pgfqpoint{5.593514in}{2.881322in}}%
\pgfpathlineto{\pgfqpoint{5.600618in}{2.857584in}}%
\pgfpathlineto{\pgfqpoint{5.605058in}{2.881660in}}%
\pgfpathlineto{\pgfqpoint{5.605946in}{2.881660in}}%
\pgfpathlineto{\pgfqpoint{5.606242in}{2.880544in}}%
\pgfpathlineto{\pgfqpoint{5.607426in}{2.830264in}}%
\pgfpathlineto{\pgfqpoint{5.643243in}{2.821955in}}%
\pgfpathlineto{\pgfqpoint{5.643835in}{2.826657in}}%
\pgfpathlineto{\pgfqpoint{5.649163in}{2.879918in}}%
\pgfpathlineto{\pgfqpoint{5.649459in}{2.842356in}}%
\pgfpathlineto{\pgfqpoint{5.650051in}{2.881660in}}%
\pgfpathlineto{\pgfqpoint{5.650347in}{2.881341in}}%
\pgfpathlineto{\pgfqpoint{5.651827in}{2.862912in}}%
\pgfpathlineto{\pgfqpoint{5.655083in}{2.822043in}}%
\pgfpathlineto{\pgfqpoint{5.655379in}{2.827361in}}%
\pgfpathlineto{\pgfqpoint{5.655675in}{2.846353in}}%
\pgfpathlineto{\pgfqpoint{5.655971in}{2.843454in}}%
\pgfpathlineto{\pgfqpoint{5.656859in}{2.821947in}}%
\pgfpathlineto{\pgfqpoint{5.838605in}{2.821947in}}%
\pgfpathlineto{\pgfqpoint{5.838902in}{2.822332in}}%
\pgfpathlineto{\pgfqpoint{5.839494in}{2.824746in}}%
\pgfpathlineto{\pgfqpoint{5.850150in}{2.824549in}}%
\pgfpathlineto{\pgfqpoint{5.861102in}{2.820081in}}%
\pgfpathlineto{\pgfqpoint{5.863766in}{2.819813in}}%
\pgfpathlineto{\pgfqpoint{5.867022in}{2.819178in}}%
\pgfpathlineto{\pgfqpoint{5.867614in}{2.820217in}}%
\pgfpathlineto{\pgfqpoint{5.868502in}{2.821740in}}%
\pgfpathlineto{\pgfqpoint{5.869982in}{2.819148in}}%
\pgfpathlineto{\pgfqpoint{5.911423in}{2.819393in}}%
\pgfpathlineto{\pgfqpoint{5.913199in}{2.819614in}}%
\pgfpathlineto{\pgfqpoint{6.004368in}{2.819614in}}%
\pgfpathlineto{\pgfqpoint{6.004368in}{2.819614in}}%
\pgfusepath{stroke}%
\end{pgfscope}%
\begin{pgfscope}%
\pgfpathrectangle{\pgfqpoint{0.481681in}{1.080890in}}{\pgfqpoint{5.785672in}{2.146863in}}%
\pgfusepath{clip}%
\pgfsetrectcap%
\pgfsetroundjoin%
\pgfsetlinewidth{0.200750pt}%
\definecolor{currentstroke}{rgb}{0.000000,0.372549,0.450980}%
\pgfsetstrokecolor{currentstroke}%
\pgfsetdash{}{0pt}%
\pgfpathmoveto{\pgfqpoint{0.744666in}{1.178475in}}%
\pgfpathlineto{\pgfqpoint{0.746146in}{1.178475in}}%
\pgfpathlineto{\pgfqpoint{0.746738in}{1.359011in}}%
\pgfpathlineto{\pgfqpoint{0.747034in}{1.359864in}}%
\pgfpathlineto{\pgfqpoint{0.816003in}{1.359864in}}%
\pgfpathlineto{\pgfqpoint{0.816595in}{1.364183in}}%
\pgfpathlineto{\pgfqpoint{0.825179in}{1.364396in}}%
\pgfpathlineto{\pgfqpoint{0.826659in}{1.375823in}}%
\pgfpathlineto{\pgfqpoint{0.830803in}{1.408912in}}%
\pgfpathlineto{\pgfqpoint{0.831099in}{1.404610in}}%
\pgfpathlineto{\pgfqpoint{0.831395in}{1.370971in}}%
\pgfpathlineto{\pgfqpoint{0.832283in}{1.405687in}}%
\pgfpathlineto{\pgfqpoint{0.832579in}{1.389839in}}%
\pgfpathlineto{\pgfqpoint{0.832875in}{1.388911in}}%
\pgfpathlineto{\pgfqpoint{0.840571in}{1.411690in}}%
\pgfpathlineto{\pgfqpoint{0.889412in}{1.411690in}}%
\pgfpathlineto{\pgfqpoint{0.889708in}{1.412317in}}%
\pgfpathlineto{\pgfqpoint{0.890300in}{1.415920in}}%
\pgfpathlineto{\pgfqpoint{0.891484in}{1.416008in}}%
\pgfpathlineto{\pgfqpoint{0.894740in}{1.416008in}}%
\pgfpathlineto{\pgfqpoint{0.895036in}{1.414850in}}%
\pgfpathlineto{\pgfqpoint{0.895628in}{1.379763in}}%
\pgfpathlineto{\pgfqpoint{0.895924in}{1.368528in}}%
\pgfpathlineto{\pgfqpoint{0.899180in}{1.368502in}}%
\pgfpathlineto{\pgfqpoint{0.902140in}{1.368602in}}%
\pgfpathlineto{\pgfqpoint{0.904508in}{1.373194in}}%
\pgfpathlineto{\pgfqpoint{0.911316in}{1.377139in}}%
\pgfpathlineto{\pgfqpoint{0.915460in}{1.377139in}}%
\pgfpathlineto{\pgfqpoint{0.915756in}{1.379113in}}%
\pgfpathlineto{\pgfqpoint{0.916348in}{1.385777in}}%
\pgfpathlineto{\pgfqpoint{0.923453in}{1.386036in}}%
\pgfpathlineto{\pgfqpoint{0.932333in}{1.403052in}}%
\pgfpathlineto{\pgfqpoint{0.937957in}{1.403052in}}%
\pgfpathlineto{\pgfqpoint{0.938549in}{1.417418in}}%
\pgfpathlineto{\pgfqpoint{0.939733in}{1.403487in}}%
\pgfpathlineto{\pgfqpoint{0.940325in}{1.403052in}}%
\pgfpathlineto{\pgfqpoint{0.967261in}{1.403052in}}%
\pgfpathlineto{\pgfqpoint{0.967853in}{1.420327in}}%
\pgfpathlineto{\pgfqpoint{0.973181in}{1.420327in}}%
\pgfpathlineto{\pgfqpoint{0.973477in}{1.417597in}}%
\pgfpathlineto{\pgfqpoint{0.974069in}{1.406856in}}%
\pgfpathlineto{\pgfqpoint{0.974365in}{1.410222in}}%
\pgfpathlineto{\pgfqpoint{0.975253in}{1.480790in}}%
\pgfpathlineto{\pgfqpoint{0.987094in}{1.480790in}}%
\pgfpathlineto{\pgfqpoint{0.987390in}{1.476190in}}%
\pgfpathlineto{\pgfqpoint{0.989462in}{1.405420in}}%
\pgfpathlineto{\pgfqpoint{0.993310in}{1.470201in}}%
\pgfpathlineto{\pgfqpoint{0.993606in}{1.467256in}}%
\pgfpathlineto{\pgfqpoint{0.993902in}{1.476660in}}%
\pgfpathlineto{\pgfqpoint{0.994494in}{1.476471in}}%
\pgfpathlineto{\pgfqpoint{1.032086in}{1.476587in}}%
\pgfpathlineto{\pgfqpoint{1.033270in}{1.484821in}}%
\pgfpathlineto{\pgfqpoint{1.037710in}{1.477094in}}%
\pgfpathlineto{\pgfqpoint{1.038006in}{1.477649in}}%
\pgfpathlineto{\pgfqpoint{1.038302in}{1.484190in}}%
\pgfpathlineto{\pgfqpoint{1.038894in}{1.476936in}}%
\pgfpathlineto{\pgfqpoint{1.043630in}{1.484905in}}%
\pgfpathlineto{\pgfqpoint{1.044222in}{1.481541in}}%
\pgfpathlineto{\pgfqpoint{1.044814in}{1.477455in}}%
\pgfpathlineto{\pgfqpoint{1.045406in}{1.485109in}}%
\pgfpathlineto{\pgfqpoint{1.079447in}{1.485109in}}%
\pgfpathlineto{\pgfqpoint{1.079743in}{1.485794in}}%
\pgfpathlineto{\pgfqpoint{1.081223in}{1.493676in}}%
\pgfpathlineto{\pgfqpoint{1.081815in}{1.492995in}}%
\pgfpathlineto{\pgfqpoint{1.086255in}{1.485598in}}%
\pgfpathlineto{\pgfqpoint{1.086847in}{1.493747in}}%
\pgfpathlineto{\pgfqpoint{1.087439in}{1.485757in}}%
\pgfpathlineto{\pgfqpoint{1.088327in}{1.498065in}}%
\pgfpathlineto{\pgfqpoint{1.093359in}{1.497953in}}%
\pgfpathlineto{\pgfqpoint{1.094247in}{1.489832in}}%
\pgfpathlineto{\pgfqpoint{1.094543in}{1.486865in}}%
\pgfpathlineto{\pgfqpoint{1.094839in}{1.486729in}}%
\pgfpathlineto{\pgfqpoint{1.095727in}{1.498065in}}%
\pgfpathlineto{\pgfqpoint{1.145456in}{1.498256in}}%
\pgfpathlineto{\pgfqpoint{1.164696in}{1.523978in}}%
\pgfpathlineto{\pgfqpoint{1.164992in}{1.523809in}}%
\pgfpathlineto{\pgfqpoint{1.165880in}{1.498065in}}%
\pgfpathlineto{\pgfqpoint{1.171504in}{1.498065in}}%
\pgfpathlineto{\pgfqpoint{1.171800in}{1.501390in}}%
\pgfpathlineto{\pgfqpoint{1.172392in}{1.518694in}}%
\pgfpathlineto{\pgfqpoint{1.173280in}{1.498065in}}%
\pgfpathlineto{\pgfqpoint{1.178312in}{1.498065in}}%
\pgfpathlineto{\pgfqpoint{1.178608in}{1.498545in}}%
\pgfpathlineto{\pgfqpoint{1.179496in}{1.521061in}}%
\pgfpathlineto{\pgfqpoint{1.179792in}{1.515306in}}%
\pgfpathlineto{\pgfqpoint{1.180384in}{1.498065in}}%
\pgfpathlineto{\pgfqpoint{1.186600in}{1.498065in}}%
\pgfpathlineto{\pgfqpoint{1.186896in}{1.501124in}}%
\pgfpathlineto{\pgfqpoint{1.187192in}{1.517214in}}%
\pgfpathlineto{\pgfqpoint{1.187785in}{1.498065in}}%
\pgfpathlineto{\pgfqpoint{1.188377in}{1.498315in}}%
\pgfpathlineto{\pgfqpoint{1.189265in}{1.506703in}}%
\pgfpathlineto{\pgfqpoint{1.215905in}{1.506703in}}%
\pgfpathlineto{\pgfqpoint{1.216497in}{1.531168in}}%
\pgfpathlineto{\pgfqpoint{1.216793in}{1.532616in}}%
\pgfpathlineto{\pgfqpoint{1.328387in}{1.532616in}}%
\pgfpathlineto{\pgfqpoint{1.328979in}{1.567166in}}%
\pgfpathlineto{\pgfqpoint{1.359171in}{1.567411in}}%
\pgfpathlineto{\pgfqpoint{1.360059in}{1.571304in}}%
\pgfpathlineto{\pgfqpoint{1.363315in}{1.567168in}}%
\pgfpathlineto{\pgfqpoint{1.377819in}{1.567007in}}%
\pgfpathlineto{\pgfqpoint{1.378707in}{1.552249in}}%
\pgfpathlineto{\pgfqpoint{1.379891in}{1.533372in}}%
\pgfpathlineto{\pgfqpoint{1.386403in}{1.571401in}}%
\pgfpathlineto{\pgfqpoint{1.387587in}{1.571485in}}%
\pgfpathlineto{\pgfqpoint{1.391140in}{1.571485in}}%
\pgfpathlineto{\pgfqpoint{1.392028in}{1.567166in}}%
\pgfpathlineto{\pgfqpoint{1.394100in}{1.567385in}}%
\pgfpathlineto{\pgfqpoint{1.413932in}{1.571266in}}%
\pgfpathlineto{\pgfqpoint{1.415412in}{1.567166in}}%
\pgfpathlineto{\pgfqpoint{1.427252in}{1.567264in}}%
\pgfpathlineto{\pgfqpoint{1.427844in}{1.571482in}}%
\pgfpathlineto{\pgfqpoint{1.428732in}{1.571245in}}%
\pgfpathlineto{\pgfqpoint{1.429324in}{1.567201in}}%
\pgfpathlineto{\pgfqpoint{1.434060in}{1.570750in}}%
\pgfpathlineto{\pgfqpoint{1.435244in}{1.571485in}}%
\pgfpathlineto{\pgfqpoint{1.444124in}{1.571225in}}%
\pgfpathlineto{\pgfqpoint{1.462181in}{1.569351in}}%
\pgfpathlineto{\pgfqpoint{1.462477in}{1.571704in}}%
\pgfpathlineto{\pgfqpoint{1.465437in}{1.670817in}}%
\pgfpathlineto{\pgfqpoint{1.471357in}{1.670817in}}%
\pgfpathlineto{\pgfqpoint{1.471949in}{1.676380in}}%
\pgfpathlineto{\pgfqpoint{1.476389in}{1.691417in}}%
\pgfpathlineto{\pgfqpoint{1.476685in}{1.685702in}}%
\pgfpathlineto{\pgfqpoint{1.476981in}{1.636304in}}%
\pgfpathlineto{\pgfqpoint{1.477277in}{1.696730in}}%
\pgfpathlineto{\pgfqpoint{1.484381in}{1.696730in}}%
\pgfpathlineto{\pgfqpoint{1.484677in}{1.697344in}}%
\pgfpathlineto{\pgfqpoint{1.485565in}{1.706375in}}%
\pgfpathlineto{\pgfqpoint{1.485861in}{1.718324in}}%
\pgfpathlineto{\pgfqpoint{1.519605in}{1.718230in}}%
\pgfpathlineto{\pgfqpoint{1.526118in}{1.709178in}}%
\pgfpathlineto{\pgfqpoint{1.529078in}{1.705367in}}%
\pgfpathlineto{\pgfqpoint{1.533518in}{1.705394in}}%
\pgfpathlineto{\pgfqpoint{1.534110in}{1.712722in}}%
\pgfpathlineto{\pgfqpoint{1.534702in}{1.718622in}}%
\pgfpathlineto{\pgfqpoint{1.541214in}{1.728868in}}%
\pgfpathlineto{\pgfqpoint{1.541806in}{1.732598in}}%
\pgfpathlineto{\pgfqpoint{1.542102in}{1.734913in}}%
\pgfpathlineto{\pgfqpoint{1.542694in}{1.782686in}}%
\pgfpathlineto{\pgfqpoint{1.542990in}{1.782533in}}%
\pgfpathlineto{\pgfqpoint{1.562230in}{1.735682in}}%
\pgfpathlineto{\pgfqpoint{1.562822in}{1.739330in}}%
\pgfpathlineto{\pgfqpoint{1.569334in}{1.790187in}}%
\pgfpathlineto{\pgfqpoint{1.570222in}{1.802753in}}%
\pgfpathlineto{\pgfqpoint{1.570518in}{1.801719in}}%
\pgfpathlineto{\pgfqpoint{1.571406in}{1.809101in}}%
\pgfpathlineto{\pgfqpoint{1.578510in}{1.813361in}}%
\pgfpathlineto{\pgfqpoint{1.579990in}{1.816367in}}%
\pgfpathlineto{\pgfqpoint{1.590646in}{1.838818in}}%
\pgfpathlineto{\pgfqpoint{1.591238in}{1.843568in}}%
\pgfpathlineto{\pgfqpoint{1.611663in}{1.843568in}}%
\pgfpathlineto{\pgfqpoint{1.611959in}{1.844903in}}%
\pgfpathlineto{\pgfqpoint{1.612255in}{1.856691in}}%
\pgfpathlineto{\pgfqpoint{1.612551in}{1.851713in}}%
\pgfpathlineto{\pgfqpoint{1.612847in}{1.860843in}}%
\pgfpathlineto{\pgfqpoint{1.712600in}{1.860843in}}%
\pgfpathlineto{\pgfqpoint{1.713488in}{1.873817in}}%
\pgfpathlineto{\pgfqpoint{1.714080in}{1.877483in}}%
\pgfpathlineto{\pgfqpoint{1.714968in}{1.884282in}}%
\pgfpathlineto{\pgfqpoint{1.715560in}{1.897745in}}%
\pgfpathlineto{\pgfqpoint{1.718224in}{1.876703in}}%
\pgfpathlineto{\pgfqpoint{1.718520in}{1.877322in}}%
\pgfpathlineto{\pgfqpoint{1.718816in}{1.899641in}}%
\pgfpathlineto{\pgfqpoint{1.720000in}{1.899713in}}%
\pgfpathlineto{\pgfqpoint{1.720592in}{1.899734in}}%
\pgfpathlineto{\pgfqpoint{1.721480in}{1.902854in}}%
\pgfpathlineto{\pgfqpoint{1.725920in}{1.920363in}}%
\pgfpathlineto{\pgfqpoint{1.726216in}{1.915718in}}%
\pgfpathlineto{\pgfqpoint{1.726512in}{1.900387in}}%
\pgfpathlineto{\pgfqpoint{1.727105in}{1.899713in}}%
\pgfpathlineto{\pgfqpoint{1.740721in}{1.899969in}}%
\pgfpathlineto{\pgfqpoint{1.775649in}{1.921307in}}%
\pgfpathlineto{\pgfqpoint{1.811466in}{1.921208in}}%
\pgfpathlineto{\pgfqpoint{1.813242in}{1.915729in}}%
\pgfpathlineto{\pgfqpoint{1.817978in}{1.900750in}}%
\pgfpathlineto{\pgfqpoint{1.818274in}{1.902369in}}%
\pgfpathlineto{\pgfqpoint{1.818866in}{1.921307in}}%
\pgfpathlineto{\pgfqpoint{1.839586in}{1.921307in}}%
\pgfpathlineto{\pgfqpoint{1.839882in}{1.916268in}}%
\pgfpathlineto{\pgfqpoint{1.840178in}{1.921307in}}%
\pgfpathlineto{\pgfqpoint{1.841954in}{1.921037in}}%
\pgfpathlineto{\pgfqpoint{1.874515in}{1.899839in}}%
\pgfpathlineto{\pgfqpoint{1.876883in}{1.905296in}}%
\pgfpathlineto{\pgfqpoint{1.883987in}{1.921307in}}%
\pgfpathlineto{\pgfqpoint{1.959764in}{1.921307in}}%
\pgfpathlineto{\pgfqpoint{1.960060in}{1.919245in}}%
\pgfpathlineto{\pgfqpoint{1.960356in}{1.921208in}}%
\pgfpathlineto{\pgfqpoint{1.961244in}{1.921307in}}%
\pgfpathlineto{\pgfqpoint{1.975156in}{1.921173in}}%
\pgfpathlineto{\pgfqpoint{1.976340in}{1.913971in}}%
\pgfpathlineto{\pgfqpoint{1.981076in}{1.883684in}}%
\pgfpathlineto{\pgfqpoint{1.981668in}{1.921307in}}%
\pgfpathlineto{\pgfqpoint{1.982260in}{1.893423in}}%
\pgfpathlineto{\pgfqpoint{1.982556in}{1.904826in}}%
\pgfpathlineto{\pgfqpoint{1.982852in}{1.882437in}}%
\pgfpathlineto{\pgfqpoint{2.075502in}{1.882451in}}%
\pgfpathlineto{\pgfqpoint{2.076686in}{1.884020in}}%
\pgfpathlineto{\pgfqpoint{2.081718in}{1.891061in}}%
\pgfpathlineto{\pgfqpoint{2.087638in}{1.891075in}}%
\pgfpathlineto{\pgfqpoint{2.087934in}{1.891075in}}%
\pgfpathlineto{\pgfqpoint{2.088526in}{1.885010in}}%
\pgfpathlineto{\pgfqpoint{2.089118in}{1.889904in}}%
\pgfpathlineto{\pgfqpoint{2.090006in}{1.882480in}}%
\pgfpathlineto{\pgfqpoint{2.092078in}{1.882437in}}%
\pgfpathlineto{\pgfqpoint{2.109838in}{1.882437in}}%
\pgfpathlineto{\pgfqpoint{2.110726in}{1.891075in}}%
\pgfpathlineto{\pgfqpoint{2.132039in}{1.891177in}}%
\pgfpathlineto{\pgfqpoint{2.138847in}{1.895248in}}%
\pgfpathlineto{\pgfqpoint{2.139143in}{1.895086in}}%
\pgfpathlineto{\pgfqpoint{2.140327in}{1.891075in}}%
\pgfpathlineto{\pgfqpoint{2.166375in}{1.891075in}}%
\pgfpathlineto{\pgfqpoint{2.166967in}{1.893435in}}%
\pgfpathlineto{\pgfqpoint{2.167559in}{1.891075in}}%
\pgfpathlineto{\pgfqpoint{2.181175in}{1.891075in}}%
\pgfpathlineto{\pgfqpoint{2.181767in}{1.895332in}}%
\pgfpathlineto{\pgfqpoint{2.182063in}{1.895253in}}%
\pgfpathlineto{\pgfqpoint{2.198047in}{1.891171in}}%
\pgfpathlineto{\pgfqpoint{2.209296in}{1.895394in}}%
\pgfpathlineto{\pgfqpoint{2.304609in}{1.895429in}}%
\pgfpathlineto{\pgfqpoint{2.307273in}{1.898186in}}%
\pgfpathlineto{\pgfqpoint{2.309049in}{1.899713in}}%
\pgfpathlineto{\pgfqpoint{2.309641in}{1.899713in}}%
\pgfpathlineto{\pgfqpoint{2.309937in}{1.898484in}}%
\pgfpathlineto{\pgfqpoint{2.310233in}{1.895995in}}%
\pgfpathlineto{\pgfqpoint{2.310825in}{1.898451in}}%
\pgfpathlineto{\pgfqpoint{2.311417in}{1.895672in}}%
\pgfpathlineto{\pgfqpoint{2.311713in}{1.895983in}}%
\pgfpathlineto{\pgfqpoint{2.315561in}{1.899713in}}%
\pgfpathlineto{\pgfqpoint{2.316153in}{1.899713in}}%
\pgfpathlineto{\pgfqpoint{2.316449in}{1.899209in}}%
\pgfpathlineto{\pgfqpoint{2.317041in}{1.899713in}}%
\pgfpathlineto{\pgfqpoint{2.330953in}{1.899713in}}%
\pgfpathlineto{\pgfqpoint{2.331249in}{1.900001in}}%
\pgfpathlineto{\pgfqpoint{2.331545in}{1.899788in}}%
\pgfpathlineto{\pgfqpoint{2.336874in}{1.903925in}}%
\pgfpathlineto{\pgfqpoint{2.337466in}{1.900517in}}%
\pgfpathlineto{\pgfqpoint{2.338058in}{1.903354in}}%
\pgfpathlineto{\pgfqpoint{2.339242in}{1.901130in}}%
\pgfpathlineto{\pgfqpoint{2.357890in}{1.899731in}}%
\pgfpathlineto{\pgfqpoint{2.358778in}{1.901102in}}%
\pgfpathlineto{\pgfqpoint{2.360554in}{1.904020in}}%
\pgfpathlineto{\pgfqpoint{2.364994in}{1.904031in}}%
\pgfpathlineto{\pgfqpoint{2.365290in}{1.903471in}}%
\pgfpathlineto{\pgfqpoint{2.366178in}{1.899717in}}%
\pgfpathlineto{\pgfqpoint{2.367362in}{1.899797in}}%
\pgfpathlineto{\pgfqpoint{2.372690in}{1.904031in}}%
\pgfpathlineto{\pgfqpoint{2.382162in}{1.903792in}}%
\pgfpathlineto{\pgfqpoint{2.386898in}{1.899713in}}%
\pgfpathlineto{\pgfqpoint{2.389266in}{1.899945in}}%
\pgfpathlineto{\pgfqpoint{2.401107in}{1.903969in}}%
\pgfpathlineto{\pgfqpoint{2.408507in}{1.899713in}}%
\pgfpathlineto{\pgfqpoint{2.408803in}{1.899765in}}%
\pgfpathlineto{\pgfqpoint{2.409395in}{1.903563in}}%
\pgfpathlineto{\pgfqpoint{2.409691in}{1.903471in}}%
\pgfpathlineto{\pgfqpoint{2.411763in}{1.899713in}}%
\pgfpathlineto{\pgfqpoint{2.415019in}{1.899780in}}%
\pgfpathlineto{\pgfqpoint{2.415611in}{1.903887in}}%
\pgfpathlineto{\pgfqpoint{2.416499in}{1.904031in}}%
\pgfpathlineto{\pgfqpoint{2.439291in}{1.904031in}}%
\pgfpathlineto{\pgfqpoint{2.439587in}{1.905546in}}%
\pgfpathlineto{\pgfqpoint{2.439883in}{1.910588in}}%
\pgfpathlineto{\pgfqpoint{2.457347in}{1.916928in}}%
\pgfpathlineto{\pgfqpoint{2.458827in}{1.914560in}}%
\pgfpathlineto{\pgfqpoint{2.465339in}{1.904031in}}%
\pgfpathlineto{\pgfqpoint{2.466819in}{1.904052in}}%
\pgfpathlineto{\pgfqpoint{2.467707in}{1.906254in}}%
\pgfpathlineto{\pgfqpoint{2.471852in}{1.916988in}}%
\pgfpathlineto{\pgfqpoint{2.827057in}{1.917124in}}%
\pgfpathlineto{\pgfqpoint{2.828537in}{1.925625in}}%
\pgfpathlineto{\pgfqpoint{2.925922in}{1.925625in}}%
\pgfpathlineto{\pgfqpoint{2.926218in}{1.946326in}}%
\pgfpathlineto{\pgfqpoint{2.926514in}{2.061393in}}%
\pgfpathlineto{\pgfqpoint{2.926810in}{2.065328in}}%
\pgfpathlineto{\pgfqpoint{2.928290in}{2.040719in}}%
\pgfpathlineto{\pgfqpoint{2.935690in}{1.925780in}}%
\pgfpathlineto{\pgfqpoint{2.936578in}{1.925625in}}%
\pgfpathlineto{\pgfqpoint{2.954635in}{1.925625in}}%
\pgfpathlineto{\pgfqpoint{2.954931in}{1.957608in}}%
\pgfpathlineto{\pgfqpoint{2.955227in}{2.049922in}}%
\pgfpathlineto{\pgfqpoint{2.955523in}{2.053069in}}%
\pgfpathlineto{\pgfqpoint{2.961739in}{1.929213in}}%
\pgfpathlineto{\pgfqpoint{2.962331in}{1.990407in}}%
\pgfpathlineto{\pgfqpoint{2.963219in}{1.990407in}}%
\pgfpathlineto{\pgfqpoint{2.963515in}{1.991631in}}%
\pgfpathlineto{\pgfqpoint{2.963811in}{1.994726in}}%
\pgfpathlineto{\pgfqpoint{2.969139in}{1.994726in}}%
\pgfpathlineto{\pgfqpoint{2.969435in}{1.994085in}}%
\pgfpathlineto{\pgfqpoint{2.971211in}{1.964650in}}%
\pgfpathlineto{\pgfqpoint{2.976835in}{1.968813in}}%
\pgfpathlineto{\pgfqpoint{3.020348in}{1.968829in}}%
\pgfpathlineto{\pgfqpoint{3.021236in}{1.970475in}}%
\pgfpathlineto{\pgfqpoint{3.026860in}{1.981531in}}%
\pgfpathlineto{\pgfqpoint{3.027748in}{1.980024in}}%
\pgfpathlineto{\pgfqpoint{3.028044in}{1.978515in}}%
\pgfpathlineto{\pgfqpoint{3.028340in}{1.985507in}}%
\pgfpathlineto{\pgfqpoint{3.033372in}{1.969293in}}%
\pgfpathlineto{\pgfqpoint{3.033964in}{1.986088in}}%
\pgfpathlineto{\pgfqpoint{3.036924in}{1.985867in}}%
\pgfpathlineto{\pgfqpoint{3.055276in}{1.981770in}}%
\pgfpathlineto{\pgfqpoint{3.055868in}{1.981770in}}%
\pgfpathlineto{\pgfqpoint{3.056164in}{1.982840in}}%
\pgfpathlineto{\pgfqpoint{3.056460in}{1.985957in}}%
\pgfpathlineto{\pgfqpoint{3.060308in}{1.982685in}}%
\pgfpathlineto{\pgfqpoint{3.061196in}{1.981924in}}%
\pgfpathlineto{\pgfqpoint{3.062084in}{1.990835in}}%
\pgfpathlineto{\pgfqpoint{3.062380in}{1.989348in}}%
\pgfpathlineto{\pgfqpoint{3.062676in}{1.992641in}}%
\pgfpathlineto{\pgfqpoint{3.063268in}{1.989730in}}%
\pgfpathlineto{\pgfqpoint{3.063564in}{1.994662in}}%
\pgfpathlineto{\pgfqpoint{3.076589in}{1.990407in}}%
\pgfpathlineto{\pgfqpoint{3.104117in}{1.990407in}}%
\pgfpathlineto{\pgfqpoint{3.104709in}{1.999045in}}%
\pgfpathlineto{\pgfqpoint{3.155030in}{1.999074in}}%
\pgfpathlineto{\pgfqpoint{3.155918in}{2.002161in}}%
\pgfpathlineto{\pgfqpoint{3.160654in}{2.019916in}}%
\pgfpathlineto{\pgfqpoint{3.161542in}{1.999432in}}%
\pgfpathlineto{\pgfqpoint{3.162134in}{2.015425in}}%
\pgfpathlineto{\pgfqpoint{3.163318in}{1.999720in}}%
\pgfpathlineto{\pgfqpoint{3.169534in}{2.020548in}}%
\pgfpathlineto{\pgfqpoint{3.170718in}{2.020639in}}%
\pgfpathlineto{\pgfqpoint{3.181966in}{2.020562in}}%
\pgfpathlineto{\pgfqpoint{3.182262in}{2.010998in}}%
\pgfpathlineto{\pgfqpoint{3.182558in}{2.020639in}}%
\pgfpathlineto{\pgfqpoint{3.382065in}{2.020639in}}%
\pgfpathlineto{\pgfqpoint{3.382361in}{2.021707in}}%
\pgfpathlineto{\pgfqpoint{3.382657in}{2.024033in}}%
\pgfpathlineto{\pgfqpoint{3.382953in}{2.023069in}}%
\pgfpathlineto{\pgfqpoint{3.383249in}{2.020656in}}%
\pgfpathlineto{\pgfqpoint{3.388281in}{2.020639in}}%
\pgfpathlineto{\pgfqpoint{3.408705in}{2.020639in}}%
\pgfpathlineto{\pgfqpoint{3.409297in}{2.022995in}}%
\pgfpathlineto{\pgfqpoint{3.409593in}{2.020639in}}%
\pgfpathlineto{\pgfqpoint{3.423506in}{2.020639in}}%
\pgfpathlineto{\pgfqpoint{3.423802in}{2.020947in}}%
\pgfpathlineto{\pgfqpoint{3.424690in}{2.023598in}}%
\pgfpathlineto{\pgfqpoint{3.424986in}{2.020639in}}%
\pgfpathlineto{\pgfqpoint{3.431498in}{2.020639in}}%
\pgfpathlineto{\pgfqpoint{3.431794in}{2.021662in}}%
\pgfpathlineto{\pgfqpoint{3.432090in}{2.021450in}}%
\pgfpathlineto{\pgfqpoint{3.432386in}{2.020641in}}%
\pgfpathlineto{\pgfqpoint{3.437418in}{2.021768in}}%
\pgfpathlineto{\pgfqpoint{3.452514in}{2.024957in}}%
\pgfpathlineto{\pgfqpoint{3.765095in}{2.025073in}}%
\pgfpathlineto{\pgfqpoint{3.765687in}{2.042220in}}%
\pgfpathlineto{\pgfqpoint{3.772495in}{2.042233in}}%
\pgfpathlineto{\pgfqpoint{3.899777in}{2.042274in}}%
\pgfpathlineto{\pgfqpoint{3.900369in}{2.050897in}}%
\pgfpathlineto{\pgfqpoint{3.901257in}{2.064050in}}%
\pgfpathlineto{\pgfqpoint{3.907473in}{2.072221in}}%
\pgfpathlineto{\pgfqpoint{3.907769in}{2.070624in}}%
\pgfpathlineto{\pgfqpoint{3.908657in}{2.056269in}}%
\pgfpathlineto{\pgfqpoint{3.909545in}{2.072464in}}%
\pgfpathlineto{\pgfqpoint{4.107276in}{2.072710in}}%
\pgfpathlineto{\pgfqpoint{4.110828in}{2.088538in}}%
\pgfpathlineto{\pgfqpoint{4.112308in}{2.094058in}}%
\pgfpathlineto{\pgfqpoint{4.130364in}{2.093845in}}%
\pgfpathlineto{\pgfqpoint{4.148716in}{2.089789in}}%
\pgfpathlineto{\pgfqpoint{4.155821in}{2.094058in}}%
\pgfpathlineto{\pgfqpoint{4.162629in}{2.094280in}}%
\pgfpathlineto{\pgfqpoint{4.163517in}{2.097486in}}%
\pgfpathlineto{\pgfqpoint{4.164109in}{2.094240in}}%
\pgfpathlineto{\pgfqpoint{4.164701in}{2.094554in}}%
\pgfpathlineto{\pgfqpoint{4.169733in}{2.098377in}}%
\pgfpathlineto{\pgfqpoint{4.170621in}{2.098249in}}%
\pgfpathlineto{\pgfqpoint{4.170917in}{2.095010in}}%
\pgfpathlineto{\pgfqpoint{4.171805in}{2.098152in}}%
\pgfpathlineto{\pgfqpoint{4.178317in}{2.094058in}}%
\pgfpathlineto{\pgfqpoint{4.179797in}{2.094286in}}%
\pgfpathlineto{\pgfqpoint{4.197261in}{2.098309in}}%
\pgfpathlineto{\pgfqpoint{4.197853in}{2.094854in}}%
\pgfpathlineto{\pgfqpoint{4.198741in}{2.098304in}}%
\pgfpathlineto{\pgfqpoint{4.199037in}{2.098377in}}%
\pgfpathlineto{\pgfqpoint{4.199333in}{2.098105in}}%
\pgfpathlineto{\pgfqpoint{4.199925in}{2.094074in}}%
\pgfpathlineto{\pgfqpoint{4.204661in}{2.093536in}}%
\pgfpathlineto{\pgfqpoint{4.205253in}{2.089739in}}%
\pgfpathlineto{\pgfqpoint{4.328687in}{2.089976in}}%
\pgfpathlineto{\pgfqpoint{4.356512in}{2.094058in}}%
\pgfpathlineto{\pgfqpoint{4.378712in}{2.093789in}}%
\pgfpathlineto{\pgfqpoint{4.399432in}{2.089739in}}%
\pgfpathlineto{\pgfqpoint{4.410384in}{2.089739in}}%
\pgfpathlineto{\pgfqpoint{4.411272in}{2.094058in}}%
\pgfpathlineto{\pgfqpoint{4.724445in}{2.094058in}}%
\pgfpathlineto{\pgfqpoint{4.725037in}{2.098377in}}%
\pgfpathlineto{\pgfqpoint{4.744869in}{2.098114in}}%
\pgfpathlineto{\pgfqpoint{4.745461in}{2.094611in}}%
\pgfpathlineto{\pgfqpoint{4.746053in}{2.098377in}}%
\pgfpathlineto{\pgfqpoint{4.746349in}{2.097693in}}%
\pgfpathlineto{\pgfqpoint{4.746941in}{2.098377in}}%
\pgfpathlineto{\pgfqpoint{4.747533in}{2.098093in}}%
\pgfpathlineto{\pgfqpoint{4.751973in}{2.094256in}}%
\pgfpathlineto{\pgfqpoint{4.752565in}{2.098377in}}%
\pgfpathlineto{\pgfqpoint{4.752861in}{2.098377in}}%
\pgfpathlineto{\pgfqpoint{4.753157in}{2.097953in}}%
\pgfpathlineto{\pgfqpoint{4.753453in}{2.095265in}}%
\pgfpathlineto{\pgfqpoint{4.754045in}{2.098376in}}%
\pgfpathlineto{\pgfqpoint{4.758781in}{2.098377in}}%
\pgfpathlineto{\pgfqpoint{4.759077in}{2.096362in}}%
\pgfpathlineto{\pgfqpoint{4.759373in}{2.096865in}}%
\pgfpathlineto{\pgfqpoint{4.759669in}{2.098376in}}%
\pgfpathlineto{\pgfqpoint{4.766478in}{2.098377in}}%
\pgfpathlineto{\pgfqpoint{4.766774in}{2.161444in}}%
\pgfpathlineto{\pgfqpoint{4.767070in}{2.341992in}}%
\pgfpathlineto{\pgfqpoint{4.767366in}{2.284972in}}%
\pgfpathlineto{\pgfqpoint{4.767662in}{2.430923in}}%
\pgfpathlineto{\pgfqpoint{4.775654in}{2.430923in}}%
\pgfpathlineto{\pgfqpoint{4.775950in}{2.428289in}}%
\pgfpathlineto{\pgfqpoint{4.794302in}{2.096626in}}%
\pgfpathlineto{\pgfqpoint{4.794894in}{2.202099in}}%
\pgfpathlineto{\pgfqpoint{4.796078in}{2.430923in}}%
\pgfpathlineto{\pgfqpoint{4.801406in}{2.430923in}}%
\pgfpathlineto{\pgfqpoint{4.801998in}{2.441023in}}%
\pgfpathlineto{\pgfqpoint{4.802294in}{2.430923in}}%
\pgfpathlineto{\pgfqpoint{4.802590in}{2.432327in}}%
\pgfpathlineto{\pgfqpoint{4.803182in}{2.452517in}}%
\pgfpathlineto{\pgfqpoint{4.804958in}{2.452517in}}%
\pgfpathlineto{\pgfqpoint{4.805254in}{2.440593in}}%
\pgfpathlineto{\pgfqpoint{4.805550in}{2.111535in}}%
\pgfpathlineto{\pgfqpoint{4.808510in}{2.449648in}}%
\pgfpathlineto{\pgfqpoint{4.808806in}{2.452517in}}%
\pgfpathlineto{\pgfqpoint{4.846103in}{2.452618in}}%
\pgfpathlineto{\pgfqpoint{4.847583in}{2.458023in}}%
\pgfpathlineto{\pgfqpoint{4.851727in}{2.473650in}}%
\pgfpathlineto{\pgfqpoint{4.852023in}{2.472295in}}%
\pgfpathlineto{\pgfqpoint{4.852615in}{2.466325in}}%
\pgfpathlineto{\pgfqpoint{4.852911in}{2.468199in}}%
\pgfpathlineto{\pgfqpoint{4.853207in}{2.474111in}}%
\pgfpathlineto{\pgfqpoint{4.858535in}{2.474111in}}%
\pgfpathlineto{\pgfqpoint{4.858831in}{2.473661in}}%
\pgfpathlineto{\pgfqpoint{4.859127in}{2.454249in}}%
\pgfpathlineto{\pgfqpoint{4.860015in}{2.473166in}}%
\pgfpathlineto{\pgfqpoint{4.860903in}{2.634728in}}%
\pgfpathlineto{\pgfqpoint{4.861199in}{2.529157in}}%
\pgfpathlineto{\pgfqpoint{4.862087in}{2.638225in}}%
\pgfpathlineto{\pgfqpoint{4.894351in}{2.638225in}}%
\pgfpathlineto{\pgfqpoint{4.894943in}{2.563980in}}%
\pgfpathlineto{\pgfqpoint{4.895240in}{2.638225in}}%
\pgfpathlineto{\pgfqpoint{4.907968in}{2.638225in}}%
\pgfpathlineto{\pgfqpoint{4.908264in}{2.639305in}}%
\pgfpathlineto{\pgfqpoint{4.908560in}{2.650483in}}%
\pgfpathlineto{\pgfqpoint{4.909448in}{2.640517in}}%
\pgfpathlineto{\pgfqpoint{4.909744in}{2.640922in}}%
\pgfpathlineto{\pgfqpoint{4.910336in}{2.651182in}}%
\pgfpathlineto{\pgfqpoint{4.922768in}{2.651182in}}%
\pgfpathlineto{\pgfqpoint{4.923064in}{2.650124in}}%
\pgfpathlineto{\pgfqpoint{4.923952in}{2.640466in}}%
\pgfpathlineto{\pgfqpoint{4.924544in}{2.651182in}}%
\pgfpathlineto{\pgfqpoint{4.958288in}{2.651182in}}%
\pgfpathlineto{\pgfqpoint{4.958584in}{2.683625in}}%
\pgfpathlineto{\pgfqpoint{4.958880in}{2.750668in}}%
\pgfpathlineto{\pgfqpoint{4.959176in}{2.753992in}}%
\pgfpathlineto{\pgfqpoint{4.959472in}{2.691307in}}%
\pgfpathlineto{\pgfqpoint{4.960064in}{2.772108in}}%
\pgfpathlineto{\pgfqpoint{4.965985in}{2.772108in}}%
\pgfpathlineto{\pgfqpoint{4.966281in}{2.769085in}}%
\pgfpathlineto{\pgfqpoint{4.966577in}{2.702575in}}%
\pgfpathlineto{\pgfqpoint{4.966873in}{2.760961in}}%
\pgfpathlineto{\pgfqpoint{4.967465in}{2.681244in}}%
\pgfpathlineto{\pgfqpoint{4.968057in}{2.772108in}}%
\pgfpathlineto{\pgfqpoint{4.996177in}{2.772108in}}%
\pgfpathlineto{\pgfqpoint{4.996473in}{2.772944in}}%
\pgfpathlineto{\pgfqpoint{4.997065in}{2.776426in}}%
\pgfpathlineto{\pgfqpoint{5.001801in}{2.776426in}}%
\pgfpathlineto{\pgfqpoint{5.002097in}{2.775111in}}%
\pgfpathlineto{\pgfqpoint{5.002393in}{2.775150in}}%
\pgfpathlineto{\pgfqpoint{5.002689in}{2.776426in}}%
\pgfpathlineto{\pgfqpoint{5.010977in}{2.776440in}}%
\pgfpathlineto{\pgfqpoint{5.011569in}{2.781507in}}%
\pgfpathlineto{\pgfqpoint{5.015713in}{2.823933in}}%
\pgfpathlineto{\pgfqpoint{5.016897in}{2.823933in}}%
\pgfpathlineto{\pgfqpoint{5.017489in}{2.810686in}}%
\pgfpathlineto{\pgfqpoint{5.017785in}{2.828438in}}%
\pgfpathlineto{\pgfqpoint{5.024889in}{2.832571in}}%
\pgfpathlineto{\pgfqpoint{5.046498in}{2.832571in}}%
\pgfpathlineto{\pgfqpoint{5.047090in}{2.799154in}}%
\pgfpathlineto{\pgfqpoint{5.047386in}{2.801313in}}%
\pgfpathlineto{\pgfqpoint{5.050346in}{2.822475in}}%
\pgfpathlineto{\pgfqpoint{5.050642in}{2.798020in}}%
\pgfpathlineto{\pgfqpoint{5.060410in}{2.798223in}}%
\pgfpathlineto{\pgfqpoint{5.065738in}{2.802288in}}%
\pgfpathlineto{\pgfqpoint{5.066034in}{2.800398in}}%
\pgfpathlineto{\pgfqpoint{5.066330in}{2.802197in}}%
\pgfpathlineto{\pgfqpoint{5.071658in}{2.798196in}}%
\pgfpathlineto{\pgfqpoint{5.072250in}{2.802339in}}%
\pgfpathlineto{\pgfqpoint{5.072842in}{2.802339in}}%
\pgfpathlineto{\pgfqpoint{5.073138in}{2.803425in}}%
\pgfpathlineto{\pgfqpoint{5.074026in}{2.822886in}}%
\pgfpathlineto{\pgfqpoint{5.074618in}{2.818197in}}%
\pgfpathlineto{\pgfqpoint{5.075210in}{2.808940in}}%
\pgfpathlineto{\pgfqpoint{5.092674in}{2.819441in}}%
\pgfpathlineto{\pgfqpoint{5.092970in}{2.818933in}}%
\pgfpathlineto{\pgfqpoint{5.093562in}{2.815296in}}%
\pgfpathlineto{\pgfqpoint{5.100075in}{2.815296in}}%
\pgfpathlineto{\pgfqpoint{5.100371in}{2.815632in}}%
\pgfpathlineto{\pgfqpoint{5.102443in}{2.834735in}}%
\pgfpathlineto{\pgfqpoint{5.102739in}{2.829337in}}%
\pgfpathlineto{\pgfqpoint{5.103035in}{2.812606in}}%
\pgfpathlineto{\pgfqpoint{5.103627in}{2.841177in}}%
\pgfpathlineto{\pgfqpoint{5.106587in}{2.838438in}}%
\pgfpathlineto{\pgfqpoint{5.117243in}{2.828439in}}%
\pgfpathlineto{\pgfqpoint{5.117835in}{2.817090in}}%
\pgfpathlineto{\pgfqpoint{5.121091in}{2.839653in}}%
\pgfpathlineto{\pgfqpoint{5.121683in}{2.818655in}}%
\pgfpathlineto{\pgfqpoint{5.122867in}{2.829623in}}%
\pgfpathlineto{\pgfqpoint{5.123163in}{2.829047in}}%
\pgfpathlineto{\pgfqpoint{5.123459in}{2.826562in}}%
\pgfpathlineto{\pgfqpoint{5.124643in}{2.836889in}}%
\pgfpathlineto{\pgfqpoint{5.144179in}{2.836889in}}%
\pgfpathlineto{\pgfqpoint{5.144771in}{2.841208in}}%
\pgfpathlineto{\pgfqpoint{5.152763in}{2.841208in}}%
\pgfpathlineto{\pgfqpoint{5.153355in}{2.837190in}}%
\pgfpathlineto{\pgfqpoint{5.157499in}{2.841045in}}%
\pgfpathlineto{\pgfqpoint{5.157795in}{2.840837in}}%
\pgfpathlineto{\pgfqpoint{5.158979in}{2.837583in}}%
\pgfpathlineto{\pgfqpoint{5.159275in}{2.841208in}}%
\pgfpathlineto{\pgfqpoint{5.160459in}{2.841426in}}%
\pgfpathlineto{\pgfqpoint{5.167268in}{2.845527in}}%
\pgfpathlineto{\pgfqpoint{5.174668in}{2.845259in}}%
\pgfpathlineto{\pgfqpoint{5.194204in}{2.839098in}}%
\pgfpathlineto{\pgfqpoint{5.195980in}{2.871257in}}%
\pgfpathlineto{\pgfqpoint{5.199236in}{2.867535in}}%
\pgfpathlineto{\pgfqpoint{5.200716in}{2.884396in}}%
\pgfpathlineto{\pgfqpoint{5.264653in}{2.884400in}}%
\pgfpathlineto{\pgfqpoint{5.265245in}{2.885481in}}%
\pgfpathlineto{\pgfqpoint{5.266725in}{2.888676in}}%
\pgfpathlineto{\pgfqpoint{5.269685in}{2.888310in}}%
\pgfpathlineto{\pgfqpoint{5.300766in}{2.884396in}}%
\pgfpathlineto{\pgfqpoint{5.324150in}{2.884617in}}%
\pgfpathlineto{\pgfqpoint{5.343390in}{2.888680in}}%
\pgfpathlineto{\pgfqpoint{5.343686in}{2.887455in}}%
\pgfpathlineto{\pgfqpoint{5.343982in}{2.888199in}}%
\pgfpathlineto{\pgfqpoint{5.344870in}{2.886654in}}%
\pgfpathlineto{\pgfqpoint{5.349606in}{2.888715in}}%
\pgfpathlineto{\pgfqpoint{5.443440in}{2.888715in}}%
\pgfpathlineto{\pgfqpoint{5.443736in}{2.889228in}}%
\pgfpathlineto{\pgfqpoint{5.444920in}{2.892838in}}%
\pgfpathlineto{\pgfqpoint{5.448768in}{2.888715in}}%
\pgfpathlineto{\pgfqpoint{5.449360in}{2.893034in}}%
\pgfpathlineto{\pgfqpoint{5.449952in}{2.893034in}}%
\pgfpathlineto{\pgfqpoint{5.450248in}{2.892697in}}%
\pgfpathlineto{\pgfqpoint{5.450544in}{2.890455in}}%
\pgfpathlineto{\pgfqpoint{5.450840in}{2.893034in}}%
\pgfpathlineto{\pgfqpoint{5.451136in}{2.893034in}}%
\pgfpathlineto{\pgfqpoint{5.451432in}{2.902591in}}%
\pgfpathlineto{\pgfqpoint{5.451728in}{2.922191in}}%
\pgfpathlineto{\pgfqpoint{5.452024in}{2.921476in}}%
\pgfpathlineto{\pgfqpoint{5.457056in}{2.895210in}}%
\pgfpathlineto{\pgfqpoint{5.457352in}{2.896186in}}%
\pgfpathlineto{\pgfqpoint{5.457648in}{2.922434in}}%
\pgfpathlineto{\pgfqpoint{5.463568in}{2.893630in}}%
\pgfpathlineto{\pgfqpoint{5.464160in}{2.904413in}}%
\pgfpathlineto{\pgfqpoint{5.465048in}{2.923009in}}%
\pgfpathlineto{\pgfqpoint{5.465640in}{2.923265in}}%
\pgfpathlineto{\pgfqpoint{5.470376in}{2.923457in}}%
\pgfpathlineto{\pgfqpoint{5.472152in}{2.927584in}}%
\pgfpathlineto{\pgfqpoint{5.493168in}{2.927584in}}%
\pgfpathlineto{\pgfqpoint{5.493464in}{2.926963in}}%
\pgfpathlineto{\pgfqpoint{5.498792in}{2.897935in}}%
\pgfpathlineto{\pgfqpoint{5.499384in}{2.906520in}}%
\pgfpathlineto{\pgfqpoint{5.500568in}{2.926979in}}%
\pgfpathlineto{\pgfqpoint{5.500864in}{2.926798in}}%
\pgfpathlineto{\pgfqpoint{5.512705in}{2.897675in}}%
\pgfpathlineto{\pgfqpoint{5.513297in}{2.904384in}}%
\pgfpathlineto{\pgfqpoint{5.515073in}{2.927509in}}%
\pgfpathlineto{\pgfqpoint{5.516257in}{2.927584in}}%
\pgfpathlineto{\pgfqpoint{5.522473in}{2.927327in}}%
\pgfpathlineto{\pgfqpoint{5.541121in}{2.897642in}}%
\pgfpathlineto{\pgfqpoint{5.542009in}{2.910519in}}%
\pgfpathlineto{\pgfqpoint{5.543193in}{2.927571in}}%
\pgfpathlineto{\pgfqpoint{5.548225in}{2.927584in}}%
\pgfpathlineto{\pgfqpoint{5.548521in}{2.927512in}}%
\pgfpathlineto{\pgfqpoint{5.549113in}{2.912622in}}%
\pgfpathlineto{\pgfqpoint{5.549409in}{2.903414in}}%
\pgfpathlineto{\pgfqpoint{5.549705in}{2.903079in}}%
\pgfpathlineto{\pgfqpoint{5.550297in}{2.929866in}}%
\pgfpathlineto{\pgfqpoint{5.550889in}{2.930186in}}%
\pgfpathlineto{\pgfqpoint{5.551185in}{2.931603in}}%
\pgfpathlineto{\pgfqpoint{5.555033in}{2.927864in}}%
\pgfpathlineto{\pgfqpoint{5.555625in}{2.931903in}}%
\pgfpathlineto{\pgfqpoint{5.593218in}{2.931903in}}%
\pgfpathlineto{\pgfqpoint{5.593514in}{2.931137in}}%
\pgfpathlineto{\pgfqpoint{5.600618in}{2.877232in}}%
\pgfpathlineto{\pgfqpoint{5.605058in}{2.931903in}}%
\pgfpathlineto{\pgfqpoint{5.605946in}{2.931903in}}%
\pgfpathlineto{\pgfqpoint{5.606242in}{2.929462in}}%
\pgfpathlineto{\pgfqpoint{5.607426in}{2.819573in}}%
\pgfpathlineto{\pgfqpoint{5.643243in}{2.815312in}}%
\pgfpathlineto{\pgfqpoint{5.643835in}{2.824494in}}%
\pgfpathlineto{\pgfqpoint{5.649163in}{2.928502in}}%
\pgfpathlineto{\pgfqpoint{5.649459in}{2.855150in}}%
\pgfpathlineto{\pgfqpoint{5.650051in}{2.931903in}}%
\pgfpathlineto{\pgfqpoint{5.650347in}{2.931281in}}%
\pgfpathlineto{\pgfqpoint{5.651827in}{2.895292in}}%
\pgfpathlineto{\pgfqpoint{5.655083in}{2.815484in}}%
\pgfpathlineto{\pgfqpoint{5.655379in}{2.825867in}}%
\pgfpathlineto{\pgfqpoint{5.655675in}{2.862956in}}%
\pgfpathlineto{\pgfqpoint{5.655971in}{2.857294in}}%
\pgfpathlineto{\pgfqpoint{5.656859in}{2.815296in}}%
\pgfpathlineto{\pgfqpoint{5.838605in}{2.815296in}}%
\pgfpathlineto{\pgfqpoint{5.838902in}{2.815889in}}%
\pgfpathlineto{\pgfqpoint{5.839494in}{2.819614in}}%
\pgfpathlineto{\pgfqpoint{6.004368in}{2.819614in}}%
\pgfpathlineto{\pgfqpoint{6.004368in}{2.819614in}}%
\pgfusepath{stroke}%
\end{pgfscope}%
\begin{pgfscope}%
\pgfpathrectangle{\pgfqpoint{0.481681in}{1.080890in}}{\pgfqpoint{5.785672in}{2.146863in}}%
\pgfusepath{clip}%
\pgfsetrectcap%
\pgfsetroundjoin%
\pgfsetlinewidth{0.200750pt}%
\definecolor{currentstroke}{rgb}{0.580392,0.823529,0.741176}%
\pgfsetstrokecolor{currentstroke}%
\pgfsetdash{}{0pt}%
\pgfpathmoveto{\pgfqpoint{0.744666in}{1.178475in}}%
\pgfpathlineto{\pgfqpoint{0.746146in}{1.179046in}}%
\pgfpathlineto{\pgfqpoint{0.746738in}{1.287640in}}%
\pgfpathlineto{\pgfqpoint{0.747330in}{1.288385in}}%
\pgfpathlineto{\pgfqpoint{0.824883in}{1.288655in}}%
\pgfpathlineto{\pgfqpoint{0.825179in}{1.288830in}}%
\pgfpathlineto{\pgfqpoint{0.826659in}{1.298193in}}%
\pgfpathlineto{\pgfqpoint{0.830803in}{1.325306in}}%
\pgfpathlineto{\pgfqpoint{0.831099in}{1.321781in}}%
\pgfpathlineto{\pgfqpoint{0.831395in}{1.294218in}}%
\pgfpathlineto{\pgfqpoint{0.832283in}{1.322663in}}%
\pgfpathlineto{\pgfqpoint{0.832579in}{1.309677in}}%
\pgfpathlineto{\pgfqpoint{0.832875in}{1.308917in}}%
\pgfpathlineto{\pgfqpoint{0.840571in}{1.327581in}}%
\pgfpathlineto{\pgfqpoint{0.890004in}{1.327828in}}%
\pgfpathlineto{\pgfqpoint{0.890596in}{1.328005in}}%
\pgfpathlineto{\pgfqpoint{0.894740in}{1.328005in}}%
\pgfpathlineto{\pgfqpoint{0.895036in}{1.327055in}}%
\pgfpathlineto{\pgfqpoint{0.895628in}{1.298306in}}%
\pgfpathlineto{\pgfqpoint{0.895924in}{1.289101in}}%
\pgfpathlineto{\pgfqpoint{0.899772in}{1.289079in}}%
\pgfpathlineto{\pgfqpoint{0.915460in}{1.289341in}}%
\pgfpathlineto{\pgfqpoint{0.917532in}{1.289768in}}%
\pgfpathlineto{\pgfqpoint{0.923749in}{1.290006in}}%
\pgfpathlineto{\pgfqpoint{0.932629in}{1.294641in}}%
\pgfpathlineto{\pgfqpoint{0.937957in}{1.294777in}}%
\pgfpathlineto{\pgfqpoint{0.938549in}{1.296184in}}%
\pgfpathlineto{\pgfqpoint{0.940029in}{1.294641in}}%
\pgfpathlineto{\pgfqpoint{0.967261in}{1.294641in}}%
\pgfpathlineto{\pgfqpoint{0.968149in}{1.297536in}}%
\pgfpathlineto{\pgfqpoint{0.973181in}{1.297536in}}%
\pgfpathlineto{\pgfqpoint{0.973773in}{1.296178in}}%
\pgfpathlineto{\pgfqpoint{0.974069in}{1.295278in}}%
\pgfpathlineto{\pgfqpoint{0.974365in}{1.295842in}}%
\pgfpathlineto{\pgfqpoint{0.975253in}{1.305815in}}%
\pgfpathlineto{\pgfqpoint{0.987094in}{1.305673in}}%
\pgfpathlineto{\pgfqpoint{0.987390in}{1.305018in}}%
\pgfpathlineto{\pgfqpoint{0.989462in}{1.295014in}}%
\pgfpathlineto{\pgfqpoint{0.993310in}{1.305227in}}%
\pgfpathlineto{\pgfqpoint{0.993606in}{1.304714in}}%
\pgfpathlineto{\pgfqpoint{0.993902in}{1.306037in}}%
\pgfpathlineto{\pgfqpoint{0.994494in}{1.305818in}}%
\pgfpathlineto{\pgfqpoint{1.016694in}{1.305768in}}%
\pgfpathlineto{\pgfqpoint{1.016990in}{1.306158in}}%
\pgfpathlineto{\pgfqpoint{1.017582in}{1.305694in}}%
\pgfpathlineto{\pgfqpoint{1.021430in}{1.305694in}}%
\pgfpathlineto{\pgfqpoint{1.038006in}{1.305671in}}%
\pgfpathlineto{\pgfqpoint{1.038302in}{1.306392in}}%
\pgfpathlineto{\pgfqpoint{1.038894in}{1.305790in}}%
\pgfpathlineto{\pgfqpoint{1.043630in}{1.307827in}}%
\pgfpathlineto{\pgfqpoint{1.044518in}{1.306443in}}%
\pgfpathlineto{\pgfqpoint{1.044814in}{1.305920in}}%
\pgfpathlineto{\pgfqpoint{1.045110in}{1.306454in}}%
\pgfpathlineto{\pgfqpoint{1.045406in}{1.307869in}}%
\pgfpathlineto{\pgfqpoint{1.053399in}{1.307879in}}%
\pgfpathlineto{\pgfqpoint{1.064943in}{1.307879in}}%
\pgfpathlineto{\pgfqpoint{1.065535in}{1.334227in}}%
\pgfpathlineto{\pgfqpoint{1.071751in}{1.334233in}}%
\pgfpathlineto{\pgfqpoint{1.073231in}{1.336726in}}%
\pgfpathlineto{\pgfqpoint{1.078559in}{1.336593in}}%
\pgfpathlineto{\pgfqpoint{1.079447in}{1.334106in}}%
\pgfpathlineto{\pgfqpoint{1.079743in}{1.334323in}}%
\pgfpathlineto{\pgfqpoint{1.081223in}{1.342601in}}%
\pgfpathlineto{\pgfqpoint{1.081815in}{1.342157in}}%
\pgfpathlineto{\pgfqpoint{1.086255in}{1.337063in}}%
\pgfpathlineto{\pgfqpoint{1.086551in}{1.339832in}}%
\pgfpathlineto{\pgfqpoint{1.086847in}{1.345821in}}%
\pgfpathlineto{\pgfqpoint{1.087143in}{1.334422in}}%
\pgfpathlineto{\pgfqpoint{1.087439in}{1.308827in}}%
\pgfpathlineto{\pgfqpoint{1.088327in}{1.351773in}}%
\pgfpathlineto{\pgfqpoint{1.091583in}{1.352383in}}%
\pgfpathlineto{\pgfqpoint{1.093063in}{1.352672in}}%
\pgfpathlineto{\pgfqpoint{1.093359in}{1.352321in}}%
\pgfpathlineto{\pgfqpoint{1.094247in}{1.322829in}}%
\pgfpathlineto{\pgfqpoint{1.094543in}{1.312049in}}%
\pgfpathlineto{\pgfqpoint{1.094839in}{1.311558in}}%
\pgfpathlineto{\pgfqpoint{1.095727in}{1.352743in}}%
\pgfpathlineto{\pgfqpoint{1.147824in}{1.353009in}}%
\pgfpathlineto{\pgfqpoint{1.164992in}{1.354756in}}%
\pgfpathlineto{\pgfqpoint{1.165584in}{1.353241in}}%
\pgfpathlineto{\pgfqpoint{1.166176in}{1.364347in}}%
\pgfpathlineto{\pgfqpoint{1.168248in}{1.406931in}}%
\pgfpathlineto{\pgfqpoint{1.171504in}{1.406931in}}%
\pgfpathlineto{\pgfqpoint{1.171800in}{1.400238in}}%
\pgfpathlineto{\pgfqpoint{1.172392in}{1.365405in}}%
\pgfpathlineto{\pgfqpoint{1.173280in}{1.406931in}}%
\pgfpathlineto{\pgfqpoint{1.178312in}{1.406931in}}%
\pgfpathlineto{\pgfqpoint{1.178608in}{1.405965in}}%
\pgfpathlineto{\pgfqpoint{1.179496in}{1.360641in}}%
\pgfpathlineto{\pgfqpoint{1.179792in}{1.372226in}}%
\pgfpathlineto{\pgfqpoint{1.180384in}{1.406931in}}%
\pgfpathlineto{\pgfqpoint{1.184232in}{1.407350in}}%
\pgfpathlineto{\pgfqpoint{1.185712in}{1.407302in}}%
\pgfpathlineto{\pgfqpoint{1.186304in}{1.406800in}}%
\pgfpathlineto{\pgfqpoint{1.186600in}{1.406984in}}%
\pgfpathlineto{\pgfqpoint{1.186896in}{1.400557in}}%
\pgfpathlineto{\pgfqpoint{1.187192in}{1.368293in}}%
\pgfpathlineto{\pgfqpoint{1.187785in}{1.406581in}}%
\pgfpathlineto{\pgfqpoint{1.188673in}{1.406775in}}%
\pgfpathlineto{\pgfqpoint{1.189265in}{1.407043in}}%
\pgfpathlineto{\pgfqpoint{1.194889in}{1.407057in}}%
\pgfpathlineto{\pgfqpoint{1.195481in}{1.409371in}}%
\pgfpathlineto{\pgfqpoint{1.214129in}{1.499971in}}%
\pgfpathlineto{\pgfqpoint{1.215609in}{1.500039in}}%
\pgfpathlineto{\pgfqpoint{1.215905in}{1.500039in}}%
\pgfpathlineto{\pgfqpoint{1.216201in}{1.477004in}}%
\pgfpathlineto{\pgfqpoint{1.216497in}{1.477387in}}%
\pgfpathlineto{\pgfqpoint{1.216793in}{1.502065in}}%
\pgfpathlineto{\pgfqpoint{1.221529in}{1.502065in}}%
\pgfpathlineto{\pgfqpoint{1.221825in}{1.502769in}}%
\pgfpathlineto{\pgfqpoint{1.222121in}{1.502342in}}%
\pgfpathlineto{\pgfqpoint{1.222713in}{1.503312in}}%
\pgfpathlineto{\pgfqpoint{1.231593in}{1.503550in}}%
\pgfpathlineto{\pgfqpoint{1.235145in}{1.504423in}}%
\pgfpathlineto{\pgfqpoint{1.235737in}{1.503433in}}%
\pgfpathlineto{\pgfqpoint{1.236625in}{1.504348in}}%
\pgfpathlineto{\pgfqpoint{1.236921in}{1.503762in}}%
\pgfpathlineto{\pgfqpoint{1.237513in}{1.504094in}}%
\pgfpathlineto{\pgfqpoint{1.238105in}{1.503312in}}%
\pgfpathlineto{\pgfqpoint{1.244025in}{1.503312in}}%
\pgfpathlineto{\pgfqpoint{1.244321in}{1.503865in}}%
\pgfpathlineto{\pgfqpoint{1.244913in}{1.506196in}}%
\pgfpathlineto{\pgfqpoint{1.249945in}{1.506111in}}%
\pgfpathlineto{\pgfqpoint{1.250833in}{1.503564in}}%
\pgfpathlineto{\pgfqpoint{1.252313in}{1.508148in}}%
\pgfpathlineto{\pgfqpoint{1.263562in}{1.544242in}}%
\pgfpathlineto{\pgfqpoint{1.263858in}{1.552241in}}%
\pgfpathlineto{\pgfqpoint{1.264450in}{1.585603in}}%
\pgfpathlineto{\pgfqpoint{1.264746in}{1.587497in}}%
\pgfpathlineto{\pgfqpoint{1.266818in}{1.587283in}}%
\pgfpathlineto{\pgfqpoint{1.267114in}{1.587122in}}%
\pgfpathlineto{\pgfqpoint{1.267410in}{1.585467in}}%
\pgfpathlineto{\pgfqpoint{1.267706in}{1.517756in}}%
\pgfpathlineto{\pgfqpoint{1.268002in}{1.514338in}}%
\pgfpathlineto{\pgfqpoint{1.270370in}{1.586973in}}%
\pgfpathlineto{\pgfqpoint{1.279842in}{1.586976in}}%
\pgfpathlineto{\pgfqpoint{1.280730in}{1.503570in}}%
\pgfpathlineto{\pgfqpoint{1.281026in}{1.508115in}}%
\pgfpathlineto{\pgfqpoint{1.284874in}{1.586302in}}%
\pgfpathlineto{\pgfqpoint{1.285170in}{1.587138in}}%
\pgfpathlineto{\pgfqpoint{1.286650in}{1.587138in}}%
\pgfpathlineto{\pgfqpoint{1.286946in}{1.547642in}}%
\pgfpathlineto{\pgfqpoint{1.287538in}{1.586347in}}%
\pgfpathlineto{\pgfqpoint{1.287834in}{1.585147in}}%
\pgfpathlineto{\pgfqpoint{1.308258in}{1.503312in}}%
\pgfpathlineto{\pgfqpoint{1.313586in}{1.503312in}}%
\pgfpathlineto{\pgfqpoint{1.313882in}{1.510680in}}%
\pgfpathlineto{\pgfqpoint{1.315362in}{1.579526in}}%
\pgfpathlineto{\pgfqpoint{1.315658in}{1.548485in}}%
\pgfpathlineto{\pgfqpoint{1.315954in}{1.587138in}}%
\pgfpathlineto{\pgfqpoint{1.320098in}{1.587427in}}%
\pgfpathlineto{\pgfqpoint{1.320394in}{1.565004in}}%
\pgfpathlineto{\pgfqpoint{1.320690in}{1.587438in}}%
\pgfpathlineto{\pgfqpoint{1.328387in}{1.587438in}}%
\pgfpathlineto{\pgfqpoint{1.328979in}{1.597617in}}%
\pgfpathlineto{\pgfqpoint{1.335195in}{1.597686in}}%
\pgfpathlineto{\pgfqpoint{1.335787in}{1.609019in}}%
\pgfpathlineto{\pgfqpoint{1.342299in}{1.759461in}}%
\pgfpathlineto{\pgfqpoint{1.343187in}{1.759660in}}%
\pgfpathlineto{\pgfqpoint{1.357691in}{1.761165in}}%
\pgfpathlineto{\pgfqpoint{1.359171in}{1.759677in}}%
\pgfpathlineto{\pgfqpoint{1.360059in}{1.761069in}}%
\pgfpathlineto{\pgfqpoint{1.363019in}{1.761193in}}%
\pgfpathlineto{\pgfqpoint{1.363315in}{1.752351in}}%
\pgfpathlineto{\pgfqpoint{1.365091in}{1.618902in}}%
\pgfpathlineto{\pgfqpoint{1.365979in}{1.756685in}}%
\pgfpathlineto{\pgfqpoint{1.370715in}{1.761134in}}%
\pgfpathlineto{\pgfqpoint{1.374563in}{1.758184in}}%
\pgfpathlineto{\pgfqpoint{1.376931in}{1.756353in}}%
\pgfpathlineto{\pgfqpoint{1.377227in}{1.758079in}}%
\pgfpathlineto{\pgfqpoint{1.377819in}{1.775591in}}%
\pgfpathlineto{\pgfqpoint{1.378707in}{1.658551in}}%
\pgfpathlineto{\pgfqpoint{1.379891in}{1.509294in}}%
\pgfpathlineto{\pgfqpoint{1.385515in}{1.776682in}}%
\pgfpathlineto{\pgfqpoint{1.386699in}{1.778109in}}%
\pgfpathlineto{\pgfqpoint{1.391140in}{1.779397in}}%
\pgfpathlineto{\pgfqpoint{1.391436in}{1.778454in}}%
\pgfpathlineto{\pgfqpoint{1.391732in}{1.776503in}}%
\pgfpathlineto{\pgfqpoint{1.394692in}{1.776745in}}%
\pgfpathlineto{\pgfqpoint{1.413932in}{1.779402in}}%
\pgfpathlineto{\pgfqpoint{1.415412in}{1.776503in}}%
\pgfpathlineto{\pgfqpoint{1.427252in}{1.776572in}}%
\pgfpathlineto{\pgfqpoint{1.427844in}{1.779555in}}%
\pgfpathlineto{\pgfqpoint{1.428732in}{1.779459in}}%
\pgfpathlineto{\pgfqpoint{1.429324in}{1.778355in}}%
\pgfpathlineto{\pgfqpoint{1.429620in}{1.777890in}}%
\pgfpathlineto{\pgfqpoint{1.434948in}{1.779543in}}%
\pgfpathlineto{\pgfqpoint{1.435836in}{1.780863in}}%
\pgfpathlineto{\pgfqpoint{1.446492in}{1.780595in}}%
\pgfpathlineto{\pgfqpoint{1.462181in}{1.779732in}}%
\pgfpathlineto{\pgfqpoint{1.462477in}{1.780422in}}%
\pgfpathlineto{\pgfqpoint{1.465437in}{1.809700in}}%
\pgfpathlineto{\pgfqpoint{1.471357in}{1.809700in}}%
\pgfpathlineto{\pgfqpoint{1.471949in}{1.819277in}}%
\pgfpathlineto{\pgfqpoint{1.476389in}{1.826304in}}%
\pgfpathlineto{\pgfqpoint{1.476685in}{1.825853in}}%
\pgfpathlineto{\pgfqpoint{1.476981in}{1.808733in}}%
\pgfpathlineto{\pgfqpoint{1.477277in}{1.832450in}}%
\pgfpathlineto{\pgfqpoint{1.478757in}{1.833016in}}%
\pgfpathlineto{\pgfqpoint{1.484677in}{1.835357in}}%
\pgfpathlineto{\pgfqpoint{1.484973in}{1.834591in}}%
\pgfpathlineto{\pgfqpoint{1.485565in}{1.839685in}}%
\pgfpathlineto{\pgfqpoint{1.485861in}{1.849423in}}%
\pgfpathlineto{\pgfqpoint{1.519605in}{1.849331in}}%
\pgfpathlineto{\pgfqpoint{1.526118in}{1.840410in}}%
\pgfpathlineto{\pgfqpoint{1.528782in}{1.836778in}}%
\pgfpathlineto{\pgfqpoint{1.529374in}{1.837293in}}%
\pgfpathlineto{\pgfqpoint{1.533814in}{1.843008in}}%
\pgfpathlineto{\pgfqpoint{1.534406in}{1.844110in}}%
\pgfpathlineto{\pgfqpoint{1.534998in}{1.842829in}}%
\pgfpathlineto{\pgfqpoint{1.541214in}{1.839930in}}%
\pgfpathlineto{\pgfqpoint{1.541510in}{1.843797in}}%
\pgfpathlineto{\pgfqpoint{1.542102in}{1.860708in}}%
\pgfpathlineto{\pgfqpoint{1.542398in}{1.855969in}}%
\pgfpathlineto{\pgfqpoint{1.542694in}{1.843234in}}%
\pgfpathlineto{\pgfqpoint{1.543286in}{1.843605in}}%
\pgfpathlineto{\pgfqpoint{1.562230in}{1.863141in}}%
\pgfpathlineto{\pgfqpoint{1.562822in}{1.861727in}}%
\pgfpathlineto{\pgfqpoint{1.569038in}{1.843308in}}%
\pgfpathlineto{\pgfqpoint{1.569630in}{1.851161in}}%
\pgfpathlineto{\pgfqpoint{1.570222in}{1.860886in}}%
\pgfpathlineto{\pgfqpoint{1.570518in}{1.854950in}}%
\pgfpathlineto{\pgfqpoint{1.571406in}{1.863179in}}%
\pgfpathlineto{\pgfqpoint{1.579102in}{1.863597in}}%
\pgfpathlineto{\pgfqpoint{1.592127in}{1.868313in}}%
\pgfpathlineto{\pgfqpoint{1.611663in}{1.868313in}}%
\pgfpathlineto{\pgfqpoint{1.611959in}{1.869884in}}%
\pgfpathlineto{\pgfqpoint{1.612255in}{1.883767in}}%
\pgfpathlineto{\pgfqpoint{1.612551in}{1.877904in}}%
\pgfpathlineto{\pgfqpoint{1.612847in}{1.888657in}}%
\pgfpathlineto{\pgfqpoint{1.683888in}{1.888583in}}%
\pgfpathlineto{\pgfqpoint{1.684184in}{1.888990in}}%
\pgfpathlineto{\pgfqpoint{1.684776in}{1.891403in}}%
\pgfpathlineto{\pgfqpoint{1.685368in}{1.894995in}}%
\pgfpathlineto{\pgfqpoint{1.689808in}{1.894969in}}%
\pgfpathlineto{\pgfqpoint{1.690696in}{1.892800in}}%
\pgfpathlineto{\pgfqpoint{1.692472in}{1.888423in}}%
\pgfpathlineto{\pgfqpoint{1.711416in}{1.894903in}}%
\pgfpathlineto{\pgfqpoint{1.712008in}{1.896336in}}%
\pgfpathlineto{\pgfqpoint{1.712600in}{1.896336in}}%
\pgfpathlineto{\pgfqpoint{1.713488in}{1.897150in}}%
\pgfpathlineto{\pgfqpoint{1.714080in}{1.895794in}}%
\pgfpathlineto{\pgfqpoint{1.714968in}{1.893279in}}%
\pgfpathlineto{\pgfqpoint{1.715264in}{1.895678in}}%
\pgfpathlineto{\pgfqpoint{1.715560in}{1.900582in}}%
\pgfpathlineto{\pgfqpoint{1.718224in}{1.897571in}}%
\pgfpathlineto{\pgfqpoint{1.718520in}{1.897691in}}%
\pgfpathlineto{\pgfqpoint{1.718816in}{1.901132in}}%
\pgfpathlineto{\pgfqpoint{1.720592in}{1.901393in}}%
\pgfpathlineto{\pgfqpoint{1.721480in}{1.903010in}}%
\pgfpathlineto{\pgfqpoint{1.725920in}{1.911946in}}%
\pgfpathlineto{\pgfqpoint{1.726216in}{1.910739in}}%
\pgfpathlineto{\pgfqpoint{1.726512in}{1.906151in}}%
\pgfpathlineto{\pgfqpoint{1.727105in}{1.905948in}}%
\pgfpathlineto{\pgfqpoint{1.741905in}{1.905687in}}%
\pgfpathlineto{\pgfqpoint{1.776833in}{1.900214in}}%
\pgfpathlineto{\pgfqpoint{1.786305in}{1.900471in}}%
\pgfpathlineto{\pgfqpoint{1.791633in}{1.901041in}}%
\pgfpathlineto{\pgfqpoint{1.811762in}{1.901245in}}%
\pgfpathlineto{\pgfqpoint{1.817978in}{1.905712in}}%
\pgfpathlineto{\pgfqpoint{1.818274in}{1.905345in}}%
\pgfpathlineto{\pgfqpoint{1.818570in}{1.901333in}}%
\pgfpathlineto{\pgfqpoint{1.819162in}{1.901161in}}%
\pgfpathlineto{\pgfqpoint{1.835738in}{1.903906in}}%
\pgfpathlineto{\pgfqpoint{1.839586in}{1.903906in}}%
\pgfpathlineto{\pgfqpoint{1.839882in}{1.904383in}}%
\pgfpathlineto{\pgfqpoint{1.840474in}{1.903906in}}%
\pgfpathlineto{\pgfqpoint{1.845802in}{1.904170in}}%
\pgfpathlineto{\pgfqpoint{1.875107in}{1.905788in}}%
\pgfpathlineto{\pgfqpoint{1.884579in}{1.903516in}}%
\pgfpathlineto{\pgfqpoint{1.933420in}{1.903270in}}%
\pgfpathlineto{\pgfqpoint{1.935196in}{1.903148in}}%
\pgfpathlineto{\pgfqpoint{1.951476in}{1.903065in}}%
\pgfpathlineto{\pgfqpoint{1.959764in}{1.903065in}}%
\pgfpathlineto{\pgfqpoint{1.960060in}{1.903472in}}%
\pgfpathlineto{\pgfqpoint{1.960356in}{1.904645in}}%
\pgfpathlineto{\pgfqpoint{1.960652in}{1.903931in}}%
\pgfpathlineto{\pgfqpoint{1.962428in}{1.968447in}}%
\pgfpathlineto{\pgfqpoint{1.963316in}{1.961903in}}%
\pgfpathlineto{\pgfqpoint{1.968644in}{1.918633in}}%
\pgfpathlineto{\pgfqpoint{1.968940in}{1.918360in}}%
\pgfpathlineto{\pgfqpoint{1.969236in}{1.918831in}}%
\pgfpathlineto{\pgfqpoint{1.971900in}{1.945734in}}%
\pgfpathlineto{\pgfqpoint{1.973972in}{1.966888in}}%
\pgfpathlineto{\pgfqpoint{1.974564in}{1.918360in}}%
\pgfpathlineto{\pgfqpoint{1.974860in}{1.932165in}}%
\pgfpathlineto{\pgfqpoint{1.975452in}{1.966380in}}%
\pgfpathlineto{\pgfqpoint{1.981076in}{1.922958in}}%
\pgfpathlineto{\pgfqpoint{1.981668in}{1.968379in}}%
\pgfpathlineto{\pgfqpoint{1.982260in}{1.934716in}}%
\pgfpathlineto{\pgfqpoint{1.982852in}{1.969593in}}%
\pgfpathlineto{\pgfqpoint{1.988772in}{1.921453in}}%
\pgfpathlineto{\pgfqpoint{1.989068in}{1.935441in}}%
\pgfpathlineto{\pgfqpoint{1.989364in}{1.968288in}}%
\pgfpathlineto{\pgfqpoint{1.989956in}{1.949763in}}%
\pgfpathlineto{\pgfqpoint{1.990252in}{1.971468in}}%
\pgfpathlineto{\pgfqpoint{2.076390in}{1.971701in}}%
\pgfpathlineto{\pgfqpoint{2.082310in}{1.973405in}}%
\pgfpathlineto{\pgfqpoint{2.089710in}{1.973668in}}%
\pgfpathlineto{\pgfqpoint{2.090894in}{1.973742in}}%
\pgfpathlineto{\pgfqpoint{2.110134in}{1.973620in}}%
\pgfpathlineto{\pgfqpoint{2.110726in}{1.975709in}}%
\pgfpathlineto{\pgfqpoint{2.111318in}{1.973958in}}%
\pgfpathlineto{\pgfqpoint{2.111614in}{1.974331in}}%
\pgfpathlineto{\pgfqpoint{2.112206in}{1.975711in}}%
\pgfpathlineto{\pgfqpoint{2.123750in}{1.975923in}}%
\pgfpathlineto{\pgfqpoint{2.132039in}{1.978922in}}%
\pgfpathlineto{\pgfqpoint{2.139143in}{1.976388in}}%
\pgfpathlineto{\pgfqpoint{2.140327in}{1.978987in}}%
\pgfpathlineto{\pgfqpoint{2.159567in}{1.979262in}}%
\pgfpathlineto{\pgfqpoint{2.161935in}{1.979341in}}%
\pgfpathlineto{\pgfqpoint{2.166375in}{1.979063in}}%
\pgfpathlineto{\pgfqpoint{2.166967in}{1.977471in}}%
\pgfpathlineto{\pgfqpoint{2.167559in}{1.979046in}}%
\pgfpathlineto{\pgfqpoint{2.174071in}{1.979268in}}%
\pgfpathlineto{\pgfqpoint{2.174663in}{1.979700in}}%
\pgfpathlineto{\pgfqpoint{2.180287in}{1.979784in}}%
\pgfpathlineto{\pgfqpoint{2.181175in}{1.978942in}}%
\pgfpathlineto{\pgfqpoint{2.181767in}{1.976212in}}%
\pgfpathlineto{\pgfqpoint{2.198343in}{1.979346in}}%
\pgfpathlineto{\pgfqpoint{2.208704in}{1.976189in}}%
\pgfpathlineto{\pgfqpoint{2.209296in}{1.980445in}}%
\pgfpathlineto{\pgfqpoint{2.210776in}{1.981404in}}%
\pgfpathlineto{\pgfqpoint{2.216992in}{1.981425in}}%
\pgfpathlineto{\pgfqpoint{2.217584in}{1.982124in}}%
\pgfpathlineto{\pgfqpoint{2.230016in}{1.982157in}}%
\pgfpathlineto{\pgfqpoint{2.230608in}{1.984067in}}%
\pgfpathlineto{\pgfqpoint{2.231200in}{1.982832in}}%
\pgfpathlineto{\pgfqpoint{2.231496in}{1.983327in}}%
\pgfpathlineto{\pgfqpoint{2.232088in}{1.985564in}}%
\pgfpathlineto{\pgfqpoint{2.232680in}{1.986301in}}%
\pgfpathlineto{\pgfqpoint{2.241560in}{1.986174in}}%
\pgfpathlineto{\pgfqpoint{2.255176in}{1.982344in}}%
\pgfpathlineto{\pgfqpoint{2.255768in}{1.982989in}}%
\pgfpathlineto{\pgfqpoint{2.258432in}{1.986725in}}%
\pgfpathlineto{\pgfqpoint{2.259024in}{1.984983in}}%
\pgfpathlineto{\pgfqpoint{2.259616in}{1.982776in}}%
\pgfpathlineto{\pgfqpoint{2.259912in}{1.983592in}}%
\pgfpathlineto{\pgfqpoint{2.260504in}{1.986340in}}%
\pgfpathlineto{\pgfqpoint{2.261096in}{1.983532in}}%
\pgfpathlineto{\pgfqpoint{2.265832in}{1.986660in}}%
\pgfpathlineto{\pgfqpoint{2.266129in}{1.986340in}}%
\pgfpathlineto{\pgfqpoint{2.266721in}{1.987358in}}%
\pgfpathlineto{\pgfqpoint{2.268201in}{1.988418in}}%
\pgfpathlineto{\pgfqpoint{2.272937in}{1.987121in}}%
\pgfpathlineto{\pgfqpoint{2.273233in}{1.987343in}}%
\pgfpathlineto{\pgfqpoint{2.275009in}{1.996153in}}%
\pgfpathlineto{\pgfqpoint{2.280633in}{1.996153in}}%
\pgfpathlineto{\pgfqpoint{2.280929in}{1.995470in}}%
\pgfpathlineto{\pgfqpoint{2.281225in}{1.992751in}}%
\pgfpathlineto{\pgfqpoint{2.281521in}{1.993848in}}%
\pgfpathlineto{\pgfqpoint{2.281817in}{1.996252in}}%
\pgfpathlineto{\pgfqpoint{2.306089in}{1.996528in}}%
\pgfpathlineto{\pgfqpoint{2.310233in}{1.996325in}}%
\pgfpathlineto{\pgfqpoint{2.310825in}{1.995800in}}%
\pgfpathlineto{\pgfqpoint{2.311713in}{1.996219in}}%
\pgfpathlineto{\pgfqpoint{2.316153in}{1.995653in}}%
\pgfpathlineto{\pgfqpoint{2.317041in}{1.995653in}}%
\pgfpathlineto{\pgfqpoint{2.333025in}{1.995926in}}%
\pgfpathlineto{\pgfqpoint{2.336874in}{1.996592in}}%
\pgfpathlineto{\pgfqpoint{2.337466in}{1.995974in}}%
\pgfpathlineto{\pgfqpoint{2.337762in}{1.996555in}}%
\pgfpathlineto{\pgfqpoint{2.340426in}{1.995976in}}%
\pgfpathlineto{\pgfqpoint{2.359074in}{1.996276in}}%
\pgfpathlineto{\pgfqpoint{2.361738in}{1.996642in}}%
\pgfpathlineto{\pgfqpoint{2.365882in}{1.996418in}}%
\pgfpathlineto{\pgfqpoint{2.366178in}{1.996945in}}%
\pgfpathlineto{\pgfqpoint{2.367066in}{1.995952in}}%
\pgfpathlineto{\pgfqpoint{2.367362in}{1.997140in}}%
\pgfpathlineto{\pgfqpoint{2.369138in}{1.996987in}}%
\pgfpathlineto{\pgfqpoint{2.373874in}{1.996724in}}%
\pgfpathlineto{\pgfqpoint{2.381274in}{1.997341in}}%
\pgfpathlineto{\pgfqpoint{2.382162in}{1.999758in}}%
\pgfpathlineto{\pgfqpoint{2.386602in}{2.000092in}}%
\pgfpathlineto{\pgfqpoint{2.387194in}{2.000345in}}%
\pgfpathlineto{\pgfqpoint{2.394298in}{2.000071in}}%
\pgfpathlineto{\pgfqpoint{2.402883in}{1.999905in}}%
\pgfpathlineto{\pgfqpoint{2.408211in}{2.000599in}}%
\pgfpathlineto{\pgfqpoint{2.408803in}{2.000676in}}%
\pgfpathlineto{\pgfqpoint{2.409691in}{2.000135in}}%
\pgfpathlineto{\pgfqpoint{2.412355in}{2.000684in}}%
\pgfpathlineto{\pgfqpoint{2.415019in}{2.000710in}}%
\pgfpathlineto{\pgfqpoint{2.415611in}{2.002288in}}%
\pgfpathlineto{\pgfqpoint{2.417387in}{2.002343in}}%
\pgfpathlineto{\pgfqpoint{2.439291in}{2.002350in}}%
\pgfpathlineto{\pgfqpoint{2.439587in}{2.004224in}}%
\pgfpathlineto{\pgfqpoint{2.439883in}{2.010482in}}%
\pgfpathlineto{\pgfqpoint{2.457347in}{2.018351in}}%
\pgfpathlineto{\pgfqpoint{2.458827in}{2.015415in}}%
\pgfpathlineto{\pgfqpoint{2.465339in}{2.002352in}}%
\pgfpathlineto{\pgfqpoint{2.466819in}{2.002379in}}%
\pgfpathlineto{\pgfqpoint{2.467707in}{2.005249in}}%
\pgfpathlineto{\pgfqpoint{2.471852in}{2.019237in}}%
\pgfpathlineto{\pgfqpoint{2.827057in}{2.019398in}}%
\pgfpathlineto{\pgfqpoint{2.828537in}{2.020841in}}%
\pgfpathlineto{\pgfqpoint{2.925922in}{2.020948in}}%
\pgfpathlineto{\pgfqpoint{2.926218in}{2.031423in}}%
\pgfpathlineto{\pgfqpoint{2.926810in}{2.094508in}}%
\pgfpathlineto{\pgfqpoint{2.927106in}{2.093624in}}%
\pgfpathlineto{\pgfqpoint{2.930658in}{2.065547in}}%
\pgfpathlineto{\pgfqpoint{2.935690in}{2.026162in}}%
\pgfpathlineto{\pgfqpoint{2.936874in}{2.026084in}}%
\pgfpathlineto{\pgfqpoint{2.954635in}{2.026084in}}%
\pgfpathlineto{\pgfqpoint{2.954931in}{2.042139in}}%
\pgfpathlineto{\pgfqpoint{2.955523in}{2.094515in}}%
\pgfpathlineto{\pgfqpoint{2.961739in}{2.028010in}}%
\pgfpathlineto{\pgfqpoint{2.962035in}{2.047226in}}%
\pgfpathlineto{\pgfqpoint{2.962331in}{2.042537in}}%
\pgfpathlineto{\pgfqpoint{2.963515in}{2.044398in}}%
\pgfpathlineto{\pgfqpoint{2.963811in}{2.045405in}}%
\pgfpathlineto{\pgfqpoint{2.969435in}{2.042597in}}%
\pgfpathlineto{\pgfqpoint{2.971211in}{2.039152in}}%
\pgfpathlineto{\pgfqpoint{2.979203in}{2.039641in}}%
\pgfpathlineto{\pgfqpoint{3.004659in}{2.039764in}}%
\pgfpathlineto{\pgfqpoint{3.005843in}{2.040816in}}%
\pgfpathlineto{\pgfqpoint{3.020348in}{2.040837in}}%
\pgfpathlineto{\pgfqpoint{3.021236in}{2.043043in}}%
\pgfpathlineto{\pgfqpoint{3.027748in}{2.061126in}}%
\pgfpathlineto{\pgfqpoint{3.028044in}{2.058657in}}%
\pgfpathlineto{\pgfqpoint{3.028340in}{2.070265in}}%
\pgfpathlineto{\pgfqpoint{3.033372in}{2.041663in}}%
\pgfpathlineto{\pgfqpoint{3.033964in}{2.071203in}}%
\pgfpathlineto{\pgfqpoint{3.036332in}{2.070523in}}%
\pgfpathlineto{\pgfqpoint{3.054388in}{2.058178in}}%
\pgfpathlineto{\pgfqpoint{3.055868in}{2.058155in}}%
\pgfpathlineto{\pgfqpoint{3.056164in}{2.061226in}}%
\pgfpathlineto{\pgfqpoint{3.056460in}{2.070531in}}%
\pgfpathlineto{\pgfqpoint{3.059124in}{2.063789in}}%
\pgfpathlineto{\pgfqpoint{3.061196in}{2.058480in}}%
\pgfpathlineto{\pgfqpoint{3.062084in}{2.071990in}}%
\pgfpathlineto{\pgfqpoint{3.062380in}{2.072111in}}%
\pgfpathlineto{\pgfqpoint{3.062676in}{2.073309in}}%
\pgfpathlineto{\pgfqpoint{3.063268in}{2.072250in}}%
\pgfpathlineto{\pgfqpoint{3.063564in}{2.074048in}}%
\pgfpathlineto{\pgfqpoint{3.077181in}{2.072248in}}%
\pgfpathlineto{\pgfqpoint{3.098493in}{2.065584in}}%
\pgfpathlineto{\pgfqpoint{3.104117in}{2.065584in}}%
\pgfpathlineto{\pgfqpoint{3.104709in}{2.068366in}}%
\pgfpathlineto{\pgfqpoint{3.114773in}{2.068640in}}%
\pgfpathlineto{\pgfqpoint{3.134605in}{2.071379in}}%
\pgfpathlineto{\pgfqpoint{3.154734in}{2.068421in}}%
\pgfpathlineto{\pgfqpoint{3.155030in}{2.068752in}}%
\pgfpathlineto{\pgfqpoint{3.155622in}{2.071045in}}%
\pgfpathlineto{\pgfqpoint{3.160654in}{2.082092in}}%
\pgfpathlineto{\pgfqpoint{3.161542in}{2.071604in}}%
\pgfpathlineto{\pgfqpoint{3.162134in}{2.079723in}}%
\pgfpathlineto{\pgfqpoint{3.163318in}{2.071683in}}%
\pgfpathlineto{\pgfqpoint{3.170422in}{2.083654in}}%
\pgfpathlineto{\pgfqpoint{3.181966in}{2.083610in}}%
\pgfpathlineto{\pgfqpoint{3.182262in}{2.078155in}}%
\pgfpathlineto{\pgfqpoint{3.182558in}{2.083654in}}%
\pgfpathlineto{\pgfqpoint{3.184038in}{2.083654in}}%
\pgfpathlineto{\pgfqpoint{3.184926in}{2.085075in}}%
\pgfpathlineto{\pgfqpoint{3.203574in}{2.083682in}}%
\pgfpathlineto{\pgfqpoint{3.203870in}{2.083953in}}%
\pgfpathlineto{\pgfqpoint{3.204166in}{2.085118in}}%
\pgfpathlineto{\pgfqpoint{3.274319in}{2.085118in}}%
\pgfpathlineto{\pgfqpoint{3.274615in}{2.085509in}}%
\pgfpathlineto{\pgfqpoint{3.274911in}{2.086872in}}%
\pgfpathlineto{\pgfqpoint{3.305400in}{2.086166in}}%
\pgfpathlineto{\pgfqpoint{3.317536in}{2.087110in}}%
\pgfpathlineto{\pgfqpoint{3.317832in}{2.087394in}}%
\pgfpathlineto{\pgfqpoint{3.319016in}{2.087195in}}%
\pgfpathlineto{\pgfqpoint{3.336184in}{2.087358in}}%
\pgfpathlineto{\pgfqpoint{3.353353in}{2.087598in}}%
\pgfpathlineto{\pgfqpoint{3.360161in}{2.088814in}}%
\pgfpathlineto{\pgfqpoint{3.360457in}{2.087362in}}%
\pgfpathlineto{\pgfqpoint{3.375257in}{2.087325in}}%
\pgfpathlineto{\pgfqpoint{3.375849in}{2.088868in}}%
\pgfpathlineto{\pgfqpoint{3.380881in}{2.088746in}}%
\pgfpathlineto{\pgfqpoint{3.381473in}{2.087619in}}%
\pgfpathlineto{\pgfqpoint{3.381769in}{2.088803in}}%
\pgfpathlineto{\pgfqpoint{3.382065in}{2.088805in}}%
\pgfpathlineto{\pgfqpoint{3.382361in}{2.092229in}}%
\pgfpathlineto{\pgfqpoint{3.382657in}{2.099682in}}%
\pgfpathlineto{\pgfqpoint{3.382953in}{2.096592in}}%
\pgfpathlineto{\pgfqpoint{3.383249in}{2.088979in}}%
\pgfpathlineto{\pgfqpoint{3.385321in}{2.088937in}}%
\pgfpathlineto{\pgfqpoint{3.408705in}{2.088937in}}%
\pgfpathlineto{\pgfqpoint{3.409297in}{2.095755in}}%
\pgfpathlineto{\pgfqpoint{3.409593in}{2.087670in}}%
\pgfpathlineto{\pgfqpoint{3.410481in}{2.088937in}}%
\pgfpathlineto{\pgfqpoint{3.423506in}{2.088937in}}%
\pgfpathlineto{\pgfqpoint{3.423802in}{2.089916in}}%
\pgfpathlineto{\pgfqpoint{3.424690in}{2.097996in}}%
\pgfpathlineto{\pgfqpoint{3.424986in}{2.088874in}}%
\pgfpathlineto{\pgfqpoint{3.426466in}{2.088937in}}%
\pgfpathlineto{\pgfqpoint{3.431498in}{2.088937in}}%
\pgfpathlineto{\pgfqpoint{3.431794in}{2.092186in}}%
\pgfpathlineto{\pgfqpoint{3.432090in}{2.091514in}}%
\pgfpathlineto{\pgfqpoint{3.432386in}{2.088946in}}%
\pgfpathlineto{\pgfqpoint{3.434162in}{2.090174in}}%
\pgfpathlineto{\pgfqpoint{3.452514in}{2.103143in}}%
\pgfpathlineto{\pgfqpoint{3.459322in}{2.102733in}}%
\pgfpathlineto{\pgfqpoint{3.459914in}{2.103654in}}%
\pgfpathlineto{\pgfqpoint{3.460802in}{2.103654in}}%
\pgfpathlineto{\pgfqpoint{3.461098in}{2.103172in}}%
\pgfpathlineto{\pgfqpoint{3.461690in}{2.103654in}}%
\pgfpathlineto{\pgfqpoint{3.473826in}{2.103466in}}%
\pgfpathlineto{\pgfqpoint{3.475898in}{2.102705in}}%
\pgfpathlineto{\pgfqpoint{3.501355in}{2.102705in}}%
\pgfpathlineto{\pgfqpoint{3.502243in}{2.103643in}}%
\pgfpathlineto{\pgfqpoint{3.607916in}{2.103635in}}%
\pgfpathlineto{\pgfqpoint{3.608804in}{2.102717in}}%
\pgfpathlineto{\pgfqpoint{3.609692in}{2.102748in}}%
\pgfpathlineto{\pgfqpoint{3.609988in}{2.103660in}}%
\pgfpathlineto{\pgfqpoint{3.663269in}{2.103929in}}%
\pgfpathlineto{\pgfqpoint{3.673333in}{2.104632in}}%
\pgfpathlineto{\pgfqpoint{3.673925in}{2.098860in}}%
\pgfpathlineto{\pgfqpoint{3.678661in}{2.098860in}}%
\pgfpathlineto{\pgfqpoint{3.678957in}{2.099416in}}%
\pgfpathlineto{\pgfqpoint{3.679549in}{2.102374in}}%
\pgfpathlineto{\pgfqpoint{3.679845in}{2.098875in}}%
\pgfpathlineto{\pgfqpoint{3.680437in}{2.098860in}}%
\pgfpathlineto{\pgfqpoint{3.680733in}{2.099219in}}%
\pgfpathlineto{\pgfqpoint{3.681325in}{2.105974in}}%
\pgfpathlineto{\pgfqpoint{3.682214in}{2.106384in}}%
\pgfpathlineto{\pgfqpoint{3.701750in}{2.111851in}}%
\pgfpathlineto{\pgfqpoint{3.702638in}{2.110013in}}%
\pgfpathlineto{\pgfqpoint{3.707670in}{2.098860in}}%
\pgfpathlineto{\pgfqpoint{3.709150in}{2.098911in}}%
\pgfpathlineto{\pgfqpoint{3.710038in}{2.111699in}}%
\pgfpathlineto{\pgfqpoint{3.714774in}{2.107010in}}%
\pgfpathlineto{\pgfqpoint{3.765095in}{2.107020in}}%
\pgfpathlineto{\pgfqpoint{3.765983in}{2.107868in}}%
\pgfpathlineto{\pgfqpoint{3.766575in}{2.108044in}}%
\pgfpathlineto{\pgfqpoint{3.801503in}{2.108054in}}%
\pgfpathlineto{\pgfqpoint{3.807127in}{2.105745in}}%
\pgfpathlineto{\pgfqpoint{3.808015in}{2.103486in}}%
\pgfpathlineto{\pgfqpoint{3.808607in}{2.105343in}}%
\pgfpathlineto{\pgfqpoint{3.808903in}{2.105104in}}%
\pgfpathlineto{\pgfqpoint{3.809495in}{2.102650in}}%
\pgfpathlineto{\pgfqpoint{3.845608in}{2.102880in}}%
\pgfpathlineto{\pgfqpoint{3.856856in}{2.105415in}}%
\pgfpathlineto{\pgfqpoint{3.859224in}{2.102650in}}%
\pgfpathlineto{\pgfqpoint{3.870768in}{2.102650in}}%
\pgfpathlineto{\pgfqpoint{3.871360in}{2.104858in}}%
\pgfpathlineto{\pgfqpoint{3.872248in}{2.105454in}}%
\pgfpathlineto{\pgfqpoint{3.872840in}{2.102650in}}%
\pgfpathlineto{\pgfqpoint{3.899185in}{2.102651in}}%
\pgfpathlineto{\pgfqpoint{3.899481in}{2.102947in}}%
\pgfpathlineto{\pgfqpoint{3.900073in}{2.105697in}}%
\pgfpathlineto{\pgfqpoint{3.901257in}{2.108887in}}%
\pgfpathlineto{\pgfqpoint{3.907473in}{2.113977in}}%
\pgfpathlineto{\pgfqpoint{3.907769in}{2.113580in}}%
\pgfpathlineto{\pgfqpoint{3.908657in}{2.109313in}}%
\pgfpathlineto{\pgfqpoint{3.909249in}{2.116155in}}%
\pgfpathlineto{\pgfqpoint{3.913097in}{2.114290in}}%
\pgfpathlineto{\pgfqpoint{3.913985in}{2.116658in}}%
\pgfpathlineto{\pgfqpoint{3.950393in}{2.116658in}}%
\pgfpathlineto{\pgfqpoint{3.950689in}{2.117016in}}%
\pgfpathlineto{\pgfqpoint{3.951874in}{2.160189in}}%
\pgfpathlineto{\pgfqpoint{3.971706in}{2.160901in}}%
\pgfpathlineto{\pgfqpoint{3.972002in}{2.159163in}}%
\pgfpathlineto{\pgfqpoint{3.973482in}{2.116658in}}%
\pgfpathlineto{\pgfqpoint{4.005450in}{2.116658in}}%
\pgfpathlineto{\pgfqpoint{4.005746in}{2.118976in}}%
\pgfpathlineto{\pgfqpoint{4.006634in}{2.159775in}}%
\pgfpathlineto{\pgfqpoint{4.012850in}{2.116662in}}%
\pgfpathlineto{\pgfqpoint{4.022027in}{2.116764in}}%
\pgfpathlineto{\pgfqpoint{4.027651in}{2.125956in}}%
\pgfpathlineto{\pgfqpoint{4.048963in}{2.160774in}}%
\pgfpathlineto{\pgfqpoint{4.050147in}{2.158464in}}%
\pgfpathlineto{\pgfqpoint{4.070275in}{2.116717in}}%
\pgfpathlineto{\pgfqpoint{4.072347in}{2.116721in}}%
\pgfpathlineto{\pgfqpoint{4.107572in}{2.116872in}}%
\pgfpathlineto{\pgfqpoint{4.111716in}{2.119567in}}%
\pgfpathlineto{\pgfqpoint{4.112012in}{2.124753in}}%
\pgfpathlineto{\pgfqpoint{4.112308in}{2.163093in}}%
\pgfpathlineto{\pgfqpoint{4.112604in}{2.164042in}}%
\pgfpathlineto{\pgfqpoint{4.129772in}{2.164310in}}%
\pgfpathlineto{\pgfqpoint{4.148716in}{2.169453in}}%
\pgfpathlineto{\pgfqpoint{4.155821in}{2.164253in}}%
\pgfpathlineto{\pgfqpoint{4.162629in}{2.164412in}}%
\pgfpathlineto{\pgfqpoint{4.163517in}{2.167681in}}%
\pgfpathlineto{\pgfqpoint{4.164109in}{2.164372in}}%
\pgfpathlineto{\pgfqpoint{4.164701in}{2.164692in}}%
\pgfpathlineto{\pgfqpoint{4.169733in}{2.168589in}}%
\pgfpathlineto{\pgfqpoint{4.170621in}{2.168459in}}%
\pgfpathlineto{\pgfqpoint{4.170917in}{2.165156in}}%
\pgfpathlineto{\pgfqpoint{4.171805in}{2.168364in}}%
\pgfpathlineto{\pgfqpoint{4.178317in}{2.164230in}}%
\pgfpathlineto{\pgfqpoint{4.179501in}{2.164352in}}%
\pgfpathlineto{\pgfqpoint{4.197261in}{2.168595in}}%
\pgfpathlineto{\pgfqpoint{4.197853in}{2.165073in}}%
\pgfpathlineto{\pgfqpoint{4.199037in}{2.169534in}}%
\pgfpathlineto{\pgfqpoint{4.199333in}{2.169830in}}%
\pgfpathlineto{\pgfqpoint{4.199925in}{2.164283in}}%
\pgfpathlineto{\pgfqpoint{4.203773in}{2.164200in}}%
\pgfpathlineto{\pgfqpoint{4.204365in}{2.164190in}}%
\pgfpathlineto{\pgfqpoint{4.204661in}{2.164858in}}%
\pgfpathlineto{\pgfqpoint{4.205253in}{2.169605in}}%
\pgfpathlineto{\pgfqpoint{4.206733in}{2.170285in}}%
\pgfpathlineto{\pgfqpoint{4.207325in}{2.171134in}}%
\pgfpathlineto{\pgfqpoint{4.213541in}{2.170647in}}%
\pgfpathlineto{\pgfqpoint{4.328095in}{2.170865in}}%
\pgfpathlineto{\pgfqpoint{4.355920in}{2.176444in}}%
\pgfpathlineto{\pgfqpoint{4.378416in}{2.176167in}}%
\pgfpathlineto{\pgfqpoint{4.399136in}{2.170674in}}%
\pgfpathlineto{\pgfqpoint{4.410384in}{2.170674in}}%
\pgfpathlineto{\pgfqpoint{4.411272in}{2.176444in}}%
\pgfpathlineto{\pgfqpoint{4.462185in}{2.176685in}}%
\pgfpathlineto{\pgfqpoint{4.463369in}{2.177387in}}%
\pgfpathlineto{\pgfqpoint{4.471361in}{2.178668in}}%
\pgfpathlineto{\pgfqpoint{4.497410in}{2.178668in}}%
\pgfpathlineto{\pgfqpoint{4.498002in}{2.176531in}}%
\pgfpathlineto{\pgfqpoint{4.504810in}{2.178494in}}%
\pgfpathlineto{\pgfqpoint{4.505698in}{2.176489in}}%
\pgfpathlineto{\pgfqpoint{4.511026in}{2.178668in}}%
\pgfpathlineto{\pgfqpoint{4.724445in}{2.178761in}}%
\pgfpathlineto{\pgfqpoint{4.725037in}{2.181630in}}%
\pgfpathlineto{\pgfqpoint{4.744869in}{2.181489in}}%
\pgfpathlineto{\pgfqpoint{4.745461in}{2.179142in}}%
\pgfpathlineto{\pgfqpoint{4.746053in}{2.181761in}}%
\pgfpathlineto{\pgfqpoint{4.746349in}{2.181286in}}%
\pgfpathlineto{\pgfqpoint{4.746941in}{2.181761in}}%
\pgfpathlineto{\pgfqpoint{4.747829in}{2.181386in}}%
\pgfpathlineto{\pgfqpoint{4.751973in}{2.178899in}}%
\pgfpathlineto{\pgfqpoint{4.752565in}{2.181763in}}%
\pgfpathlineto{\pgfqpoint{4.752861in}{2.181786in}}%
\pgfpathlineto{\pgfqpoint{4.753157in}{2.181513in}}%
\pgfpathlineto{\pgfqpoint{4.753453in}{2.179600in}}%
\pgfpathlineto{\pgfqpoint{4.754045in}{2.181761in}}%
\pgfpathlineto{\pgfqpoint{4.758781in}{2.181761in}}%
\pgfpathlineto{\pgfqpoint{4.759077in}{2.180362in}}%
\pgfpathlineto{\pgfqpoint{4.759373in}{2.180711in}}%
\pgfpathlineto{\pgfqpoint{4.759669in}{2.181760in}}%
\pgfpathlineto{\pgfqpoint{4.766478in}{2.181764in}}%
\pgfpathlineto{\pgfqpoint{4.766774in}{2.232679in}}%
\pgfpathlineto{\pgfqpoint{4.767070in}{2.378449in}}%
\pgfpathlineto{\pgfqpoint{4.767366in}{2.332424in}}%
\pgfpathlineto{\pgfqpoint{4.767662in}{2.450248in}}%
\pgfpathlineto{\pgfqpoint{4.775654in}{2.450268in}}%
\pgfpathlineto{\pgfqpoint{4.775950in}{2.448147in}}%
\pgfpathlineto{\pgfqpoint{4.794302in}{2.180831in}}%
\pgfpathlineto{\pgfqpoint{4.794894in}{2.265833in}}%
\pgfpathlineto{\pgfqpoint{4.796078in}{2.450248in}}%
\pgfpathlineto{\pgfqpoint{4.801406in}{2.450248in}}%
\pgfpathlineto{\pgfqpoint{4.801998in}{2.477033in}}%
\pgfpathlineto{\pgfqpoint{4.802294in}{2.450258in}}%
\pgfpathlineto{\pgfqpoint{4.802590in}{2.454092in}}%
\pgfpathlineto{\pgfqpoint{4.803182in}{2.508909in}}%
\pgfpathlineto{\pgfqpoint{4.803774in}{2.509794in}}%
\pgfpathlineto{\pgfqpoint{4.804070in}{2.509609in}}%
\pgfpathlineto{\pgfqpoint{4.804958in}{2.509794in}}%
\pgfpathlineto{\pgfqpoint{4.805254in}{2.498777in}}%
\pgfpathlineto{\pgfqpoint{4.805550in}{2.194015in}}%
\pgfpathlineto{\pgfqpoint{4.808510in}{2.507136in}}%
\pgfpathlineto{\pgfqpoint{4.809102in}{2.511053in}}%
\pgfpathlineto{\pgfqpoint{4.809398in}{2.511850in}}%
\pgfpathlineto{\pgfqpoint{4.810878in}{2.509706in}}%
\pgfpathlineto{\pgfqpoint{4.813246in}{2.509667in}}%
\pgfpathlineto{\pgfqpoint{4.844919in}{2.509479in}}%
\pgfpathlineto{\pgfqpoint{4.845215in}{2.510081in}}%
\pgfpathlineto{\pgfqpoint{4.845807in}{2.513313in}}%
\pgfpathlineto{\pgfqpoint{4.846103in}{2.511467in}}%
\pgfpathlineto{\pgfqpoint{4.846399in}{2.514296in}}%
\pgfpathlineto{\pgfqpoint{4.852319in}{2.515001in}}%
\pgfpathlineto{\pgfqpoint{4.852615in}{2.514879in}}%
\pgfpathlineto{\pgfqpoint{4.853503in}{2.515342in}}%
\pgfpathlineto{\pgfqpoint{4.854391in}{2.515343in}}%
\pgfpathlineto{\pgfqpoint{4.858831in}{2.516480in}}%
\pgfpathlineto{\pgfqpoint{4.859127in}{2.509930in}}%
\pgfpathlineto{\pgfqpoint{4.859719in}{2.513035in}}%
\pgfpathlineto{\pgfqpoint{4.860015in}{2.515788in}}%
\pgfpathlineto{\pgfqpoint{4.860903in}{2.669289in}}%
\pgfpathlineto{\pgfqpoint{4.861199in}{2.568501in}}%
\pgfpathlineto{\pgfqpoint{4.862087in}{2.674052in}}%
\pgfpathlineto{\pgfqpoint{4.866231in}{2.673997in}}%
\pgfpathlineto{\pgfqpoint{4.866823in}{2.673665in}}%
\pgfpathlineto{\pgfqpoint{4.868007in}{2.673665in}}%
\pgfpathlineto{\pgfqpoint{4.891391in}{2.673929in}}%
\pgfpathlineto{\pgfqpoint{4.894351in}{2.673943in}}%
\pgfpathlineto{\pgfqpoint{4.894943in}{2.608186in}}%
\pgfpathlineto{\pgfqpoint{4.895240in}{2.673944in}}%
\pgfpathlineto{\pgfqpoint{4.901752in}{2.673755in}}%
\pgfpathlineto{\pgfqpoint{4.902048in}{2.672303in}}%
\pgfpathlineto{\pgfqpoint{4.903232in}{2.673972in}}%
\pgfpathlineto{\pgfqpoint{4.903528in}{2.673972in}}%
\pgfpathlineto{\pgfqpoint{4.904120in}{2.671954in}}%
\pgfpathlineto{\pgfqpoint{4.907968in}{2.671811in}}%
\pgfpathlineto{\pgfqpoint{4.908264in}{2.673353in}}%
\pgfpathlineto{\pgfqpoint{4.908560in}{2.689505in}}%
\pgfpathlineto{\pgfqpoint{4.909448in}{2.675242in}}%
\pgfpathlineto{\pgfqpoint{4.909744in}{2.675822in}}%
\pgfpathlineto{\pgfqpoint{4.910336in}{2.690509in}}%
\pgfpathlineto{\pgfqpoint{4.922768in}{2.690509in}}%
\pgfpathlineto{\pgfqpoint{4.923064in}{2.688995in}}%
\pgfpathlineto{\pgfqpoint{4.923952in}{2.675169in}}%
\pgfpathlineto{\pgfqpoint{4.924544in}{2.690509in}}%
\pgfpathlineto{\pgfqpoint{4.958288in}{2.690509in}}%
\pgfpathlineto{\pgfqpoint{4.958584in}{2.737896in}}%
\pgfpathlineto{\pgfqpoint{4.958880in}{2.835301in}}%
\pgfpathlineto{\pgfqpoint{4.959176in}{2.837822in}}%
\pgfpathlineto{\pgfqpoint{4.959472in}{2.747887in}}%
\pgfpathlineto{\pgfqpoint{4.960064in}{2.863564in}}%
\pgfpathlineto{\pgfqpoint{4.965985in}{2.864686in}}%
\pgfpathlineto{\pgfqpoint{4.966281in}{2.860344in}}%
\pgfpathlineto{\pgfqpoint{4.966577in}{2.764447in}}%
\pgfpathlineto{\pgfqpoint{4.966873in}{2.848568in}}%
\pgfpathlineto{\pgfqpoint{4.967465in}{2.733819in}}%
\pgfpathlineto{\pgfqpoint{4.968057in}{2.864614in}}%
\pgfpathlineto{\pgfqpoint{4.995289in}{2.864614in}}%
\pgfpathlineto{\pgfqpoint{4.995585in}{2.863763in}}%
\pgfpathlineto{\pgfqpoint{4.996177in}{2.864358in}}%
\pgfpathlineto{\pgfqpoint{4.996473in}{2.865133in}}%
\pgfpathlineto{\pgfqpoint{4.997065in}{2.871234in}}%
\pgfpathlineto{\pgfqpoint{5.001801in}{2.886556in}}%
\pgfpathlineto{\pgfqpoint{5.002097in}{2.879750in}}%
\pgfpathlineto{\pgfqpoint{5.002393in}{2.879962in}}%
\pgfpathlineto{\pgfqpoint{5.002689in}{2.886869in}}%
\pgfpathlineto{\pgfqpoint{5.010977in}{2.886875in}}%
\pgfpathlineto{\pgfqpoint{5.011569in}{2.888965in}}%
\pgfpathlineto{\pgfqpoint{5.015713in}{2.906463in}}%
\pgfpathlineto{\pgfqpoint{5.016897in}{2.906463in}}%
\pgfpathlineto{\pgfqpoint{5.017193in}{2.907272in}}%
\pgfpathlineto{\pgfqpoint{5.017785in}{2.929479in}}%
\pgfpathlineto{\pgfqpoint{5.025185in}{2.926806in}}%
\pgfpathlineto{\pgfqpoint{5.046498in}{2.926806in}}%
\pgfpathlineto{\pgfqpoint{5.046794in}{2.864413in}}%
\pgfpathlineto{\pgfqpoint{5.047090in}{2.698615in}}%
\pgfpathlineto{\pgfqpoint{5.050346in}{2.894232in}}%
\pgfpathlineto{\pgfqpoint{5.050642in}{2.689105in}}%
\pgfpathlineto{\pgfqpoint{5.072842in}{2.689096in}}%
\pgfpathlineto{\pgfqpoint{5.073138in}{2.689559in}}%
\pgfpathlineto{\pgfqpoint{5.074322in}{2.699077in}}%
\pgfpathlineto{\pgfqpoint{5.074618in}{2.699776in}}%
\pgfpathlineto{\pgfqpoint{5.074914in}{2.698651in}}%
\pgfpathlineto{\pgfqpoint{5.075210in}{2.694864in}}%
\pgfpathlineto{\pgfqpoint{5.092970in}{2.699572in}}%
\pgfpathlineto{\pgfqpoint{5.093266in}{2.700414in}}%
\pgfpathlineto{\pgfqpoint{5.094154in}{2.700049in}}%
\pgfpathlineto{\pgfqpoint{5.095338in}{2.699657in}}%
\pgfpathlineto{\pgfqpoint{5.100075in}{2.700485in}}%
\pgfpathlineto{\pgfqpoint{5.100667in}{2.699742in}}%
\pgfpathlineto{\pgfqpoint{5.102443in}{2.700602in}}%
\pgfpathlineto{\pgfqpoint{5.102739in}{2.698358in}}%
\pgfpathlineto{\pgfqpoint{5.103035in}{2.692415in}}%
\pgfpathlineto{\pgfqpoint{5.103331in}{2.692397in}}%
\pgfpathlineto{\pgfqpoint{5.103627in}{2.683496in}}%
\pgfpathlineto{\pgfqpoint{5.121091in}{2.683496in}}%
\pgfpathlineto{\pgfqpoint{5.121387in}{2.685132in}}%
\pgfpathlineto{\pgfqpoint{5.122867in}{2.716031in}}%
\pgfpathlineto{\pgfqpoint{5.123163in}{2.714786in}}%
\pgfpathlineto{\pgfqpoint{5.123459in}{2.710086in}}%
\pgfpathlineto{\pgfqpoint{5.124643in}{2.737493in}}%
\pgfpathlineto{\pgfqpoint{5.144179in}{2.737493in}}%
\pgfpathlineto{\pgfqpoint{5.144771in}{2.752347in}}%
\pgfpathlineto{\pgfqpoint{5.152763in}{2.752347in}}%
\pgfpathlineto{\pgfqpoint{5.153355in}{2.738527in}}%
\pgfpathlineto{\pgfqpoint{5.157499in}{2.751787in}}%
\pgfpathlineto{\pgfqpoint{5.157795in}{2.751071in}}%
\pgfpathlineto{\pgfqpoint{5.158979in}{2.739879in}}%
\pgfpathlineto{\pgfqpoint{5.159275in}{2.752347in}}%
\pgfpathlineto{\pgfqpoint{5.160459in}{2.752566in}}%
\pgfpathlineto{\pgfqpoint{5.167268in}{2.756700in}}%
\pgfpathlineto{\pgfqpoint{5.174076in}{2.756485in}}%
\pgfpathlineto{\pgfqpoint{5.194204in}{2.739841in}}%
\pgfpathlineto{\pgfqpoint{5.194796in}{2.757343in}}%
\pgfpathlineto{\pgfqpoint{5.195980in}{2.764537in}}%
\pgfpathlineto{\pgfqpoint{5.199236in}{2.762659in}}%
\pgfpathlineto{\pgfqpoint{5.199532in}{2.763920in}}%
\pgfpathlineto{\pgfqpoint{5.199828in}{2.766686in}}%
\pgfpathlineto{\pgfqpoint{5.200124in}{2.766099in}}%
\pgfpathlineto{\pgfqpoint{5.200420in}{2.768012in}}%
\pgfpathlineto{\pgfqpoint{5.263765in}{2.768012in}}%
\pgfpathlineto{\pgfqpoint{5.264061in}{2.768369in}}%
\pgfpathlineto{\pgfqpoint{5.264653in}{2.768012in}}%
\pgfpathlineto{\pgfqpoint{5.307870in}{2.768132in}}%
\pgfpathlineto{\pgfqpoint{5.308462in}{2.768360in}}%
\pgfpathlineto{\pgfqpoint{5.314974in}{2.768116in}}%
\pgfpathlineto{\pgfqpoint{5.316454in}{2.768064in}}%
\pgfpathlineto{\pgfqpoint{5.325926in}{2.768312in}}%
\pgfpathlineto{\pgfqpoint{5.343390in}{2.768015in}}%
\pgfpathlineto{\pgfqpoint{5.343982in}{2.768360in}}%
\pgfpathlineto{\pgfqpoint{5.443440in}{2.768360in}}%
\pgfpathlineto{\pgfqpoint{5.443736in}{2.770462in}}%
\pgfpathlineto{\pgfqpoint{5.444920in}{2.785265in}}%
\pgfpathlineto{\pgfqpoint{5.448768in}{2.768360in}}%
\pgfpathlineto{\pgfqpoint{5.449360in}{2.786067in}}%
\pgfpathlineto{\pgfqpoint{5.449952in}{2.786067in}}%
\pgfpathlineto{\pgfqpoint{5.450248in}{2.784686in}}%
\pgfpathlineto{\pgfqpoint{5.450544in}{2.775496in}}%
\pgfpathlineto{\pgfqpoint{5.450840in}{2.786067in}}%
\pgfpathlineto{\pgfqpoint{5.451136in}{2.786067in}}%
\pgfpathlineto{\pgfqpoint{5.451432in}{2.800945in}}%
\pgfpathlineto{\pgfqpoint{5.451728in}{2.831457in}}%
\pgfpathlineto{\pgfqpoint{5.452024in}{2.830344in}}%
\pgfpathlineto{\pgfqpoint{5.457056in}{2.789454in}}%
\pgfpathlineto{\pgfqpoint{5.457352in}{2.790975in}}%
\pgfpathlineto{\pgfqpoint{5.457648in}{2.831835in}}%
\pgfpathlineto{\pgfqpoint{5.463568in}{2.786995in}}%
\pgfpathlineto{\pgfqpoint{5.464160in}{2.803782in}}%
\pgfpathlineto{\pgfqpoint{5.465048in}{2.832730in}}%
\pgfpathlineto{\pgfqpoint{5.465640in}{2.833130in}}%
\pgfpathlineto{\pgfqpoint{5.470080in}{2.833130in}}%
\pgfpathlineto{\pgfqpoint{5.470376in}{2.835023in}}%
\pgfpathlineto{\pgfqpoint{5.472152in}{2.875742in}}%
\pgfpathlineto{\pgfqpoint{5.493168in}{2.875742in}}%
\pgfpathlineto{\pgfqpoint{5.493464in}{2.873916in}}%
\pgfpathlineto{\pgfqpoint{5.498792in}{2.788547in}}%
\pgfpathlineto{\pgfqpoint{5.499384in}{2.813795in}}%
\pgfpathlineto{\pgfqpoint{5.500568in}{2.873962in}}%
\pgfpathlineto{\pgfqpoint{5.500864in}{2.873431in}}%
\pgfpathlineto{\pgfqpoint{5.512705in}{2.787782in}}%
\pgfpathlineto{\pgfqpoint{5.513297in}{2.807512in}}%
\pgfpathlineto{\pgfqpoint{5.515073in}{2.875522in}}%
\pgfpathlineto{\pgfqpoint{5.515665in}{2.875742in}}%
\pgfpathlineto{\pgfqpoint{5.522177in}{2.875486in}}%
\pgfpathlineto{\pgfqpoint{5.522769in}{2.868614in}}%
\pgfpathlineto{\pgfqpoint{5.541121in}{2.787663in}}%
\pgfpathlineto{\pgfqpoint{5.542009in}{2.825556in}}%
\pgfpathlineto{\pgfqpoint{5.543193in}{2.875704in}}%
\pgfpathlineto{\pgfqpoint{5.545561in}{2.875765in}}%
\pgfpathlineto{\pgfqpoint{5.548521in}{2.875580in}}%
\pgfpathlineto{\pgfqpoint{5.549113in}{2.831764in}}%
\pgfpathlineto{\pgfqpoint{5.549409in}{2.804670in}}%
\pgfpathlineto{\pgfqpoint{5.549705in}{2.803674in}}%
\pgfpathlineto{\pgfqpoint{5.550297in}{2.876239in}}%
\pgfpathlineto{\pgfqpoint{5.552073in}{2.876412in}}%
\pgfpathlineto{\pgfqpoint{5.555033in}{2.875846in}}%
\pgfpathlineto{\pgfqpoint{5.555921in}{2.876641in}}%
\pgfpathlineto{\pgfqpoint{5.593218in}{2.876641in}}%
\pgfpathlineto{\pgfqpoint{5.593514in}{2.876210in}}%
\pgfpathlineto{\pgfqpoint{5.600618in}{2.845838in}}%
\pgfpathlineto{\pgfqpoint{5.605058in}{2.876641in}}%
\pgfpathlineto{\pgfqpoint{5.605946in}{2.876641in}}%
\pgfpathlineto{\pgfqpoint{5.606242in}{2.875253in}}%
\pgfpathlineto{\pgfqpoint{5.607426in}{2.812829in}}%
\pgfpathlineto{\pgfqpoint{5.643243in}{2.818796in}}%
\pgfpathlineto{\pgfqpoint{5.643835in}{2.823351in}}%
\pgfpathlineto{\pgfqpoint{5.649163in}{2.874954in}}%
\pgfpathlineto{\pgfqpoint{5.649459in}{2.838561in}}%
\pgfpathlineto{\pgfqpoint{5.650051in}{2.876641in}}%
\pgfpathlineto{\pgfqpoint{5.650347in}{2.876333in}}%
\pgfpathlineto{\pgfqpoint{5.651827in}{2.858477in}}%
\pgfpathlineto{\pgfqpoint{5.655083in}{2.818881in}}%
\pgfpathlineto{\pgfqpoint{5.655379in}{2.824033in}}%
\pgfpathlineto{\pgfqpoint{5.655675in}{2.842434in}}%
\pgfpathlineto{\pgfqpoint{5.655971in}{2.839625in}}%
\pgfpathlineto{\pgfqpoint{5.656859in}{2.818787in}}%
\pgfpathlineto{\pgfqpoint{5.838902in}{2.819003in}}%
\pgfpathlineto{\pgfqpoint{5.839494in}{2.820354in}}%
\pgfpathlineto{\pgfqpoint{5.851334in}{2.820114in}}%
\pgfpathlineto{\pgfqpoint{5.862286in}{2.818770in}}%
\pgfpathlineto{\pgfqpoint{5.867318in}{2.818825in}}%
\pgfpathlineto{\pgfqpoint{5.868502in}{2.819465in}}%
\pgfpathlineto{\pgfqpoint{5.870278in}{2.818698in}}%
\pgfpathlineto{\pgfqpoint{5.910831in}{2.818873in}}%
\pgfpathlineto{\pgfqpoint{5.912607in}{2.819614in}}%
\pgfpathlineto{\pgfqpoint{6.004368in}{2.819614in}}%
\pgfpathlineto{\pgfqpoint{6.004368in}{2.819614in}}%
\pgfusepath{stroke}%
\end{pgfscope}%
\begin{pgfscope}%
\pgfpathrectangle{\pgfqpoint{0.481681in}{1.080890in}}{\pgfqpoint{5.785672in}{2.146863in}}%
\pgfusepath{clip}%
\pgfsetrectcap%
\pgfsetroundjoin%
\pgfsetlinewidth{0.200750pt}%
\definecolor{currentstroke}{rgb}{0.933333,0.607843,0.000000}%
\pgfsetstrokecolor{currentstroke}%
\pgfsetdash{}{0pt}%
\pgfpathmoveto{\pgfqpoint{0.744666in}{1.178475in}}%
\pgfpathlineto{\pgfqpoint{0.744962in}{1.301537in}}%
\pgfpathlineto{\pgfqpoint{0.745554in}{1.301592in}}%
\pgfpathlineto{\pgfqpoint{0.746146in}{1.330880in}}%
\pgfpathlineto{\pgfqpoint{0.746442in}{1.330359in}}%
\pgfpathlineto{\pgfqpoint{0.747626in}{1.331972in}}%
\pgfpathlineto{\pgfqpoint{0.762130in}{1.332778in}}%
\pgfpathlineto{\pgfqpoint{0.767162in}{1.332778in}}%
\pgfpathlineto{\pgfqpoint{0.767458in}{1.333312in}}%
\pgfpathlineto{\pgfqpoint{0.768642in}{1.340828in}}%
\pgfpathlineto{\pgfqpoint{0.768938in}{1.339091in}}%
\pgfpathlineto{\pgfqpoint{0.769234in}{1.344991in}}%
\pgfpathlineto{\pgfqpoint{0.773082in}{1.344589in}}%
\pgfpathlineto{\pgfqpoint{0.773674in}{1.347442in}}%
\pgfpathlineto{\pgfqpoint{0.774266in}{1.353150in}}%
\pgfpathlineto{\pgfqpoint{0.774858in}{1.354104in}}%
\pgfpathlineto{\pgfqpoint{0.776634in}{1.354481in}}%
\pgfpathlineto{\pgfqpoint{0.784331in}{1.354897in}}%
\pgfpathlineto{\pgfqpoint{0.788475in}{1.356075in}}%
\pgfpathlineto{\pgfqpoint{0.789067in}{1.354825in}}%
\pgfpathlineto{\pgfqpoint{0.790251in}{1.362725in}}%
\pgfpathlineto{\pgfqpoint{0.791435in}{1.363336in}}%
\pgfpathlineto{\pgfqpoint{0.796763in}{1.365679in}}%
\pgfpathlineto{\pgfqpoint{0.797355in}{1.367609in}}%
\pgfpathlineto{\pgfqpoint{0.816003in}{1.367848in}}%
\pgfpathlineto{\pgfqpoint{0.816595in}{1.363366in}}%
\pgfpathlineto{\pgfqpoint{0.817187in}{1.363466in}}%
\pgfpathlineto{\pgfqpoint{0.818075in}{1.363257in}}%
\pgfpathlineto{\pgfqpoint{0.819851in}{1.364859in}}%
\pgfpathlineto{\pgfqpoint{0.822811in}{1.367626in}}%
\pgfpathlineto{\pgfqpoint{0.823403in}{1.363836in}}%
\pgfpathlineto{\pgfqpoint{0.823699in}{1.364962in}}%
\pgfpathlineto{\pgfqpoint{0.823995in}{1.364878in}}%
\pgfpathlineto{\pgfqpoint{0.824291in}{1.365269in}}%
\pgfpathlineto{\pgfqpoint{0.824883in}{1.364871in}}%
\pgfpathlineto{\pgfqpoint{0.825475in}{1.367288in}}%
\pgfpathlineto{\pgfqpoint{0.830803in}{1.372122in}}%
\pgfpathlineto{\pgfqpoint{0.831099in}{1.371918in}}%
\pgfpathlineto{\pgfqpoint{0.831395in}{1.370399in}}%
\pgfpathlineto{\pgfqpoint{0.832283in}{1.378517in}}%
\pgfpathlineto{\pgfqpoint{0.832875in}{1.375308in}}%
\pgfpathlineto{\pgfqpoint{0.840275in}{1.379959in}}%
\pgfpathlineto{\pgfqpoint{0.840867in}{1.380500in}}%
\pgfpathlineto{\pgfqpoint{0.875500in}{1.380350in}}%
\pgfpathlineto{\pgfqpoint{0.877572in}{1.380401in}}%
\pgfpathlineto{\pgfqpoint{0.883196in}{1.380943in}}%
\pgfpathlineto{\pgfqpoint{0.886452in}{1.384117in}}%
\pgfpathlineto{\pgfqpoint{0.886748in}{1.387961in}}%
\pgfpathlineto{\pgfqpoint{0.887340in}{1.403550in}}%
\pgfpathlineto{\pgfqpoint{0.887636in}{1.404579in}}%
\pgfpathlineto{\pgfqpoint{0.890300in}{1.404620in}}%
\pgfpathlineto{\pgfqpoint{0.890892in}{1.405210in}}%
\pgfpathlineto{\pgfqpoint{0.891484in}{1.405460in}}%
\pgfpathlineto{\pgfqpoint{0.894148in}{1.405429in}}%
\pgfpathlineto{\pgfqpoint{0.894740in}{1.409205in}}%
\pgfpathlineto{\pgfqpoint{0.895036in}{1.409292in}}%
\pgfpathlineto{\pgfqpoint{0.895924in}{1.404376in}}%
\pgfpathlineto{\pgfqpoint{0.896516in}{1.402846in}}%
\pgfpathlineto{\pgfqpoint{0.897108in}{1.403379in}}%
\pgfpathlineto{\pgfqpoint{0.897996in}{1.403693in}}%
\pgfpathlineto{\pgfqpoint{0.903324in}{1.405378in}}%
\pgfpathlineto{\pgfqpoint{0.903620in}{1.405425in}}%
\pgfpathlineto{\pgfqpoint{0.904212in}{1.404601in}}%
\pgfpathlineto{\pgfqpoint{0.916052in}{1.404876in}}%
\pgfpathlineto{\pgfqpoint{0.916348in}{1.403381in}}%
\pgfpathlineto{\pgfqpoint{0.918125in}{1.408145in}}%
\pgfpathlineto{\pgfqpoint{0.918421in}{1.407384in}}%
\pgfpathlineto{\pgfqpoint{0.920789in}{1.407443in}}%
\pgfpathlineto{\pgfqpoint{0.923453in}{1.407670in}}%
\pgfpathlineto{\pgfqpoint{0.932333in}{1.421237in}}%
\pgfpathlineto{\pgfqpoint{0.938253in}{1.421285in}}%
\pgfpathlineto{\pgfqpoint{0.938845in}{1.421489in}}%
\pgfpathlineto{\pgfqpoint{0.940917in}{1.421576in}}%
\pgfpathlineto{\pgfqpoint{0.965781in}{1.425039in}}%
\pgfpathlineto{\pgfqpoint{0.966373in}{1.424438in}}%
\pgfpathlineto{\pgfqpoint{0.966965in}{1.424344in}}%
\pgfpathlineto{\pgfqpoint{0.967261in}{1.424905in}}%
\pgfpathlineto{\pgfqpoint{0.967557in}{1.424214in}}%
\pgfpathlineto{\pgfqpoint{0.968149in}{1.428308in}}%
\pgfpathlineto{\pgfqpoint{0.968741in}{1.428156in}}%
\pgfpathlineto{\pgfqpoint{0.973181in}{1.429507in}}%
\pgfpathlineto{\pgfqpoint{0.974069in}{1.432033in}}%
\pgfpathlineto{\pgfqpoint{0.974365in}{1.431774in}}%
\pgfpathlineto{\pgfqpoint{0.974957in}{1.439129in}}%
\pgfpathlineto{\pgfqpoint{0.987390in}{1.438906in}}%
\pgfpathlineto{\pgfqpoint{0.989462in}{1.435494in}}%
\pgfpathlineto{\pgfqpoint{0.993310in}{1.440230in}}%
\pgfpathlineto{\pgfqpoint{0.993606in}{1.439948in}}%
\pgfpathlineto{\pgfqpoint{0.994198in}{1.442090in}}%
\pgfpathlineto{\pgfqpoint{0.994494in}{1.442296in}}%
\pgfpathlineto{\pgfqpoint{0.994790in}{1.443301in}}%
\pgfpathlineto{\pgfqpoint{0.995086in}{1.445926in}}%
\pgfpathlineto{\pgfqpoint{0.995382in}{1.444531in}}%
\pgfpathlineto{\pgfqpoint{0.995974in}{1.439018in}}%
\pgfpathlineto{\pgfqpoint{1.004854in}{1.438883in}}%
\pgfpathlineto{\pgfqpoint{1.014918in}{1.438938in}}%
\pgfpathlineto{\pgfqpoint{1.016102in}{1.439944in}}%
\pgfpathlineto{\pgfqpoint{1.016990in}{1.447081in}}%
\pgfpathlineto{\pgfqpoint{1.017878in}{1.441518in}}%
\pgfpathlineto{\pgfqpoint{1.029718in}{1.443180in}}%
\pgfpathlineto{\pgfqpoint{1.030310in}{1.440602in}}%
\pgfpathlineto{\pgfqpoint{1.030606in}{1.440762in}}%
\pgfpathlineto{\pgfqpoint{1.031198in}{1.441109in}}%
\pgfpathlineto{\pgfqpoint{1.032086in}{1.442625in}}%
\pgfpathlineto{\pgfqpoint{1.032678in}{1.442973in}}%
\pgfpathlineto{\pgfqpoint{1.032974in}{1.442404in}}%
\pgfpathlineto{\pgfqpoint{1.033270in}{1.436580in}}%
\pgfpathlineto{\pgfqpoint{1.037710in}{1.442460in}}%
\pgfpathlineto{\pgfqpoint{1.038006in}{1.442323in}}%
\pgfpathlineto{\pgfqpoint{1.038302in}{1.438562in}}%
\pgfpathlineto{\pgfqpoint{1.038894in}{1.443174in}}%
\pgfpathlineto{\pgfqpoint{1.043630in}{1.448196in}}%
\pgfpathlineto{\pgfqpoint{1.044222in}{1.446037in}}%
\pgfpathlineto{\pgfqpoint{1.045110in}{1.442288in}}%
\pgfpathlineto{\pgfqpoint{1.045702in}{1.451339in}}%
\pgfpathlineto{\pgfqpoint{1.046590in}{1.454576in}}%
\pgfpathlineto{\pgfqpoint{1.064943in}{1.456515in}}%
\pgfpathlineto{\pgfqpoint{1.065831in}{1.458149in}}%
\pgfpathlineto{\pgfqpoint{1.067607in}{1.459734in}}%
\pgfpathlineto{\pgfqpoint{1.071751in}{1.458757in}}%
\pgfpathlineto{\pgfqpoint{1.072935in}{1.461961in}}%
\pgfpathlineto{\pgfqpoint{1.074711in}{1.463160in}}%
\pgfpathlineto{\pgfqpoint{1.078559in}{1.465339in}}%
\pgfpathlineto{\pgfqpoint{1.080039in}{1.470165in}}%
\pgfpathlineto{\pgfqpoint{1.081815in}{1.476574in}}%
\pgfpathlineto{\pgfqpoint{1.086255in}{1.475761in}}%
\pgfpathlineto{\pgfqpoint{1.086551in}{1.477072in}}%
\pgfpathlineto{\pgfqpoint{1.086847in}{1.479694in}}%
\pgfpathlineto{\pgfqpoint{1.087143in}{1.469277in}}%
\pgfpathlineto{\pgfqpoint{1.087439in}{1.448011in}}%
\pgfpathlineto{\pgfqpoint{1.088031in}{1.480238in}}%
\pgfpathlineto{\pgfqpoint{1.088327in}{1.479938in}}%
\pgfpathlineto{\pgfqpoint{1.088919in}{1.479955in}}%
\pgfpathlineto{\pgfqpoint{1.093063in}{1.481308in}}%
\pgfpathlineto{\pgfqpoint{1.093359in}{1.481112in}}%
\pgfpathlineto{\pgfqpoint{1.094247in}{1.460202in}}%
\pgfpathlineto{\pgfqpoint{1.094543in}{1.452553in}}%
\pgfpathlineto{\pgfqpoint{1.094839in}{1.453411in}}%
\pgfpathlineto{\pgfqpoint{1.095727in}{1.491144in}}%
\pgfpathlineto{\pgfqpoint{1.121776in}{1.491469in}}%
\pgfpathlineto{\pgfqpoint{1.123552in}{1.492501in}}%
\pgfpathlineto{\pgfqpoint{1.129472in}{1.491379in}}%
\pgfpathlineto{\pgfqpoint{1.129768in}{1.491997in}}%
\pgfpathlineto{\pgfqpoint{1.130360in}{1.494548in}}%
\pgfpathlineto{\pgfqpoint{1.131248in}{1.493375in}}%
\pgfpathlineto{\pgfqpoint{1.138056in}{1.492939in}}%
\pgfpathlineto{\pgfqpoint{1.143680in}{1.495068in}}%
\pgfpathlineto{\pgfqpoint{1.144864in}{1.493487in}}%
\pgfpathlineto{\pgfqpoint{1.145160in}{1.495462in}}%
\pgfpathlineto{\pgfqpoint{1.145752in}{1.495149in}}%
\pgfpathlineto{\pgfqpoint{1.164992in}{1.496112in}}%
\pgfpathlineto{\pgfqpoint{1.165880in}{1.495429in}}%
\pgfpathlineto{\pgfqpoint{1.168248in}{1.499387in}}%
\pgfpathlineto{\pgfqpoint{1.171504in}{1.500184in}}%
\pgfpathlineto{\pgfqpoint{1.171800in}{1.499702in}}%
\pgfpathlineto{\pgfqpoint{1.172392in}{1.496957in}}%
\pgfpathlineto{\pgfqpoint{1.172688in}{1.507416in}}%
\pgfpathlineto{\pgfqpoint{1.173280in}{1.500024in}}%
\pgfpathlineto{\pgfqpoint{1.173872in}{1.502415in}}%
\pgfpathlineto{\pgfqpoint{1.178312in}{1.501978in}}%
\pgfpathlineto{\pgfqpoint{1.178608in}{1.502729in}}%
\pgfpathlineto{\pgfqpoint{1.179200in}{1.501753in}}%
\pgfpathlineto{\pgfqpoint{1.179496in}{1.501131in}}%
\pgfpathlineto{\pgfqpoint{1.180976in}{1.503388in}}%
\pgfpathlineto{\pgfqpoint{1.185416in}{1.503216in}}%
\pgfpathlineto{\pgfqpoint{1.186896in}{1.505482in}}%
\pgfpathlineto{\pgfqpoint{1.187192in}{1.503886in}}%
\pgfpathlineto{\pgfqpoint{1.187785in}{1.505676in}}%
\pgfpathlineto{\pgfqpoint{1.188377in}{1.505279in}}%
\pgfpathlineto{\pgfqpoint{1.189265in}{1.509005in}}%
\pgfpathlineto{\pgfqpoint{1.192817in}{1.513068in}}%
\pgfpathlineto{\pgfqpoint{1.193113in}{1.512612in}}%
\pgfpathlineto{\pgfqpoint{1.193705in}{1.513171in}}%
\pgfpathlineto{\pgfqpoint{1.194001in}{1.514061in}}%
\pgfpathlineto{\pgfqpoint{1.194593in}{1.517889in}}%
\pgfpathlineto{\pgfqpoint{1.194889in}{1.517664in}}%
\pgfpathlineto{\pgfqpoint{1.195185in}{1.516386in}}%
\pgfpathlineto{\pgfqpoint{1.215313in}{1.521499in}}%
\pgfpathlineto{\pgfqpoint{1.215905in}{1.521544in}}%
\pgfpathlineto{\pgfqpoint{1.216201in}{1.525663in}}%
\pgfpathlineto{\pgfqpoint{1.216793in}{1.542644in}}%
\pgfpathlineto{\pgfqpoint{1.218865in}{1.542919in}}%
\pgfpathlineto{\pgfqpoint{1.220641in}{1.543322in}}%
\pgfpathlineto{\pgfqpoint{1.221233in}{1.534973in}}%
\pgfpathlineto{\pgfqpoint{1.221825in}{1.543948in}}%
\pgfpathlineto{\pgfqpoint{1.222417in}{1.543701in}}%
\pgfpathlineto{\pgfqpoint{1.225081in}{1.544172in}}%
\pgfpathlineto{\pgfqpoint{1.229225in}{1.544057in}}%
\pgfpathlineto{\pgfqpoint{1.229817in}{1.545658in}}%
\pgfpathlineto{\pgfqpoint{1.230113in}{1.545409in}}%
\pgfpathlineto{\pgfqpoint{1.230705in}{1.544018in}}%
\pgfpathlineto{\pgfqpoint{1.235145in}{1.546732in}}%
\pgfpathlineto{\pgfqpoint{1.235441in}{1.543635in}}%
\pgfpathlineto{\pgfqpoint{1.235737in}{1.543772in}}%
\pgfpathlineto{\pgfqpoint{1.236625in}{1.552067in}}%
\pgfpathlineto{\pgfqpoint{1.236921in}{1.547128in}}%
\pgfpathlineto{\pgfqpoint{1.237513in}{1.553710in}}%
\pgfpathlineto{\pgfqpoint{1.238105in}{1.544634in}}%
\pgfpathlineto{\pgfqpoint{1.242841in}{1.543314in}}%
\pgfpathlineto{\pgfqpoint{1.243433in}{1.544677in}}%
\pgfpathlineto{\pgfqpoint{1.244025in}{1.544677in}}%
\pgfpathlineto{\pgfqpoint{1.244321in}{1.546801in}}%
\pgfpathlineto{\pgfqpoint{1.244913in}{1.555561in}}%
\pgfpathlineto{\pgfqpoint{1.248169in}{1.555843in}}%
\pgfpathlineto{\pgfqpoint{1.249649in}{1.556029in}}%
\pgfpathlineto{\pgfqpoint{1.249945in}{1.555688in}}%
\pgfpathlineto{\pgfqpoint{1.251129in}{1.543732in}}%
\pgfpathlineto{\pgfqpoint{1.263562in}{1.557329in}}%
\pgfpathlineto{\pgfqpoint{1.264154in}{1.559731in}}%
\pgfpathlineto{\pgfqpoint{1.264746in}{1.562656in}}%
\pgfpathlineto{\pgfqpoint{1.267410in}{1.562562in}}%
\pgfpathlineto{\pgfqpoint{1.267706in}{1.548254in}}%
\pgfpathlineto{\pgfqpoint{1.268002in}{1.547570in}}%
\pgfpathlineto{\pgfqpoint{1.270370in}{1.563181in}}%
\pgfpathlineto{\pgfqpoint{1.271258in}{1.562929in}}%
\pgfpathlineto{\pgfqpoint{1.272146in}{1.563421in}}%
\pgfpathlineto{\pgfqpoint{1.272738in}{1.563294in}}%
\pgfpathlineto{\pgfqpoint{1.273626in}{1.565279in}}%
\pgfpathlineto{\pgfqpoint{1.279250in}{1.565993in}}%
\pgfpathlineto{\pgfqpoint{1.279842in}{1.566289in}}%
\pgfpathlineto{\pgfqpoint{1.280730in}{1.544922in}}%
\pgfpathlineto{\pgfqpoint{1.281026in}{1.546174in}}%
\pgfpathlineto{\pgfqpoint{1.285170in}{1.566877in}}%
\pgfpathlineto{\pgfqpoint{1.286354in}{1.567547in}}%
\pgfpathlineto{\pgfqpoint{1.286650in}{1.567443in}}%
\pgfpathlineto{\pgfqpoint{1.286946in}{1.557407in}}%
\pgfpathlineto{\pgfqpoint{1.287538in}{1.569945in}}%
\pgfpathlineto{\pgfqpoint{1.287834in}{1.569585in}}%
\pgfpathlineto{\pgfqpoint{1.308258in}{1.545056in}}%
\pgfpathlineto{\pgfqpoint{1.313586in}{1.545568in}}%
\pgfpathlineto{\pgfqpoint{1.313882in}{1.547746in}}%
\pgfpathlineto{\pgfqpoint{1.315362in}{1.568013in}}%
\pgfpathlineto{\pgfqpoint{1.315658in}{1.559444in}}%
\pgfpathlineto{\pgfqpoint{1.315954in}{1.572577in}}%
\pgfpathlineto{\pgfqpoint{1.316546in}{1.573293in}}%
\pgfpathlineto{\pgfqpoint{1.320098in}{1.573876in}}%
\pgfpathlineto{\pgfqpoint{1.320394in}{1.566725in}}%
\pgfpathlineto{\pgfqpoint{1.320690in}{1.573922in}}%
\pgfpathlineto{\pgfqpoint{1.328387in}{1.574205in}}%
\pgfpathlineto{\pgfqpoint{1.328979in}{1.575702in}}%
\pgfpathlineto{\pgfqpoint{1.337563in}{1.575716in}}%
\pgfpathlineto{\pgfqpoint{1.342891in}{1.576351in}}%
\pgfpathlineto{\pgfqpoint{1.357691in}{1.581995in}}%
\pgfpathlineto{\pgfqpoint{1.359763in}{1.592755in}}%
\pgfpathlineto{\pgfqpoint{1.360059in}{1.592484in}}%
\pgfpathlineto{\pgfqpoint{1.363611in}{1.581242in}}%
\pgfpathlineto{\pgfqpoint{1.365683in}{1.578223in}}%
\pgfpathlineto{\pgfqpoint{1.370715in}{1.582026in}}%
\pgfpathlineto{\pgfqpoint{1.375155in}{1.579288in}}%
\pgfpathlineto{\pgfqpoint{1.376931in}{1.578185in}}%
\pgfpathlineto{\pgfqpoint{1.377227in}{1.578517in}}%
\pgfpathlineto{\pgfqpoint{1.377819in}{1.584545in}}%
\pgfpathlineto{\pgfqpoint{1.378115in}{1.581354in}}%
\pgfpathlineto{\pgfqpoint{1.379891in}{1.546735in}}%
\pgfpathlineto{\pgfqpoint{1.385219in}{1.584729in}}%
\pgfpathlineto{\pgfqpoint{1.385811in}{1.593478in}}%
\pgfpathlineto{\pgfqpoint{1.386403in}{1.602987in}}%
\pgfpathlineto{\pgfqpoint{1.387291in}{1.604590in}}%
\pgfpathlineto{\pgfqpoint{1.391140in}{1.610197in}}%
\pgfpathlineto{\pgfqpoint{1.391436in}{1.602135in}}%
\pgfpathlineto{\pgfqpoint{1.391732in}{1.586068in}}%
\pgfpathlineto{\pgfqpoint{1.393212in}{1.586071in}}%
\pgfpathlineto{\pgfqpoint{1.413636in}{1.610405in}}%
\pgfpathlineto{\pgfqpoint{1.413932in}{1.609470in}}%
\pgfpathlineto{\pgfqpoint{1.415116in}{1.587301in}}%
\pgfpathlineto{\pgfqpoint{1.415708in}{1.588044in}}%
\pgfpathlineto{\pgfqpoint{1.420444in}{1.588649in}}%
\pgfpathlineto{\pgfqpoint{1.421036in}{1.589337in}}%
\pgfpathlineto{\pgfqpoint{1.426956in}{1.588695in}}%
\pgfpathlineto{\pgfqpoint{1.427252in}{1.589199in}}%
\pgfpathlineto{\pgfqpoint{1.427844in}{1.613373in}}%
\pgfpathlineto{\pgfqpoint{1.428436in}{1.613650in}}%
\pgfpathlineto{\pgfqpoint{1.428732in}{1.612768in}}%
\pgfpathlineto{\pgfqpoint{1.429324in}{1.597330in}}%
\pgfpathlineto{\pgfqpoint{1.430212in}{1.599982in}}%
\pgfpathlineto{\pgfqpoint{1.434948in}{1.615565in}}%
\pgfpathlineto{\pgfqpoint{1.435836in}{1.626784in}}%
\pgfpathlineto{\pgfqpoint{1.441756in}{1.627174in}}%
\pgfpathlineto{\pgfqpoint{1.462477in}{1.612579in}}%
\pgfpathlineto{\pgfqpoint{1.465437in}{1.607248in}}%
\pgfpathlineto{\pgfqpoint{1.471357in}{1.606771in}}%
\pgfpathlineto{\pgfqpoint{1.471949in}{1.608796in}}%
\pgfpathlineto{\pgfqpoint{1.476389in}{1.609982in}}%
\pgfpathlineto{\pgfqpoint{1.477573in}{1.659470in}}%
\pgfpathlineto{\pgfqpoint{1.484381in}{1.665602in}}%
\pgfpathlineto{\pgfqpoint{1.484677in}{1.663550in}}%
\pgfpathlineto{\pgfqpoint{1.484973in}{1.651013in}}%
\pgfpathlineto{\pgfqpoint{1.485269in}{1.665363in}}%
\pgfpathlineto{\pgfqpoint{1.485565in}{1.661476in}}%
\pgfpathlineto{\pgfqpoint{1.485861in}{1.673110in}}%
\pgfpathlineto{\pgfqpoint{1.490893in}{1.672957in}}%
\pgfpathlineto{\pgfqpoint{1.491189in}{1.672426in}}%
\pgfpathlineto{\pgfqpoint{1.492373in}{1.661166in}}%
\pgfpathlineto{\pgfqpoint{1.519309in}{1.661203in}}%
\pgfpathlineto{\pgfqpoint{1.519605in}{1.661983in}}%
\pgfpathlineto{\pgfqpoint{1.529078in}{1.658160in}}%
\pgfpathlineto{\pgfqpoint{1.533814in}{1.667356in}}%
\pgfpathlineto{\pgfqpoint{1.534406in}{1.667366in}}%
\pgfpathlineto{\pgfqpoint{1.534998in}{1.666339in}}%
\pgfpathlineto{\pgfqpoint{1.541214in}{1.663250in}}%
\pgfpathlineto{\pgfqpoint{1.541510in}{1.666443in}}%
\pgfpathlineto{\pgfqpoint{1.542102in}{1.680482in}}%
\pgfpathlineto{\pgfqpoint{1.542398in}{1.677970in}}%
\pgfpathlineto{\pgfqpoint{1.542694in}{1.669891in}}%
\pgfpathlineto{\pgfqpoint{1.543286in}{1.670109in}}%
\pgfpathlineto{\pgfqpoint{1.562230in}{1.682857in}}%
\pgfpathlineto{\pgfqpoint{1.562526in}{1.683121in}}%
\pgfpathlineto{\pgfqpoint{1.569038in}{1.670903in}}%
\pgfpathlineto{\pgfqpoint{1.569630in}{1.677512in}}%
\pgfpathlineto{\pgfqpoint{1.570222in}{1.685681in}}%
\pgfpathlineto{\pgfqpoint{1.570518in}{1.680703in}}%
\pgfpathlineto{\pgfqpoint{1.571406in}{1.687672in}}%
\pgfpathlineto{\pgfqpoint{1.578214in}{1.687881in}}%
\pgfpathlineto{\pgfqpoint{1.578806in}{1.684304in}}%
\pgfpathlineto{\pgfqpoint{1.590646in}{1.691734in}}%
\pgfpathlineto{\pgfqpoint{1.591238in}{1.688533in}}%
\pgfpathlineto{\pgfqpoint{1.611663in}{1.677486in}}%
\pgfpathlineto{\pgfqpoint{1.611959in}{1.679562in}}%
\pgfpathlineto{\pgfqpoint{1.612255in}{1.686466in}}%
\pgfpathlineto{\pgfqpoint{1.612551in}{1.684107in}}%
\pgfpathlineto{\pgfqpoint{1.612847in}{1.686157in}}%
\pgfpathlineto{\pgfqpoint{1.632679in}{1.686420in}}%
\pgfpathlineto{\pgfqpoint{1.634455in}{1.685343in}}%
\pgfpathlineto{\pgfqpoint{1.644815in}{1.686587in}}%
\pgfpathlineto{\pgfqpoint{1.661688in}{1.687184in}}%
\pgfpathlineto{\pgfqpoint{1.661984in}{1.688435in}}%
\pgfpathlineto{\pgfqpoint{1.670272in}{1.688978in}}%
\pgfpathlineto{\pgfqpoint{1.675896in}{1.688741in}}%
\pgfpathlineto{\pgfqpoint{1.676192in}{1.686298in}}%
\pgfpathlineto{\pgfqpoint{1.676488in}{1.688835in}}%
\pgfpathlineto{\pgfqpoint{1.677376in}{1.687778in}}%
\pgfpathlineto{\pgfqpoint{1.679744in}{1.687915in}}%
\pgfpathlineto{\pgfqpoint{1.681816in}{1.685849in}}%
\pgfpathlineto{\pgfqpoint{1.683000in}{1.684634in}}%
\pgfpathlineto{\pgfqpoint{1.683296in}{1.684734in}}%
\pgfpathlineto{\pgfqpoint{1.684184in}{1.689143in}}%
\pgfpathlineto{\pgfqpoint{1.685664in}{1.689478in}}%
\pgfpathlineto{\pgfqpoint{1.690104in}{1.689064in}}%
\pgfpathlineto{\pgfqpoint{1.692472in}{1.687286in}}%
\pgfpathlineto{\pgfqpoint{1.711416in}{1.689216in}}%
\pgfpathlineto{\pgfqpoint{1.712008in}{1.688688in}}%
\pgfpathlineto{\pgfqpoint{1.712600in}{1.688760in}}%
\pgfpathlineto{\pgfqpoint{1.713192in}{1.691548in}}%
\pgfpathlineto{\pgfqpoint{1.713488in}{1.690884in}}%
\pgfpathlineto{\pgfqpoint{1.714376in}{1.689804in}}%
\pgfpathlineto{\pgfqpoint{1.714968in}{1.689252in}}%
\pgfpathlineto{\pgfqpoint{1.715264in}{1.689864in}}%
\pgfpathlineto{\pgfqpoint{1.715560in}{1.691132in}}%
\pgfpathlineto{\pgfqpoint{1.718520in}{1.689534in}}%
\pgfpathlineto{\pgfqpoint{1.719112in}{1.692061in}}%
\pgfpathlineto{\pgfqpoint{1.719704in}{1.692807in}}%
\pgfpathlineto{\pgfqpoint{1.720888in}{1.695106in}}%
\pgfpathlineto{\pgfqpoint{1.725920in}{1.695982in}}%
\pgfpathlineto{\pgfqpoint{1.726216in}{1.697383in}}%
\pgfpathlineto{\pgfqpoint{1.726512in}{1.701187in}}%
\pgfpathlineto{\pgfqpoint{1.727401in}{1.701497in}}%
\pgfpathlineto{\pgfqpoint{1.735689in}{1.702586in}}%
\pgfpathlineto{\pgfqpoint{1.740129in}{1.702887in}}%
\pgfpathlineto{\pgfqpoint{1.740721in}{1.703944in}}%
\pgfpathlineto{\pgfqpoint{1.776537in}{1.696109in}}%
\pgfpathlineto{\pgfqpoint{1.778313in}{1.696353in}}%
\pgfpathlineto{\pgfqpoint{1.784233in}{1.697177in}}%
\pgfpathlineto{\pgfqpoint{1.785713in}{1.697398in}}%
\pgfpathlineto{\pgfqpoint{1.795482in}{1.699213in}}%
\pgfpathlineto{\pgfqpoint{1.813834in}{1.702480in}}%
\pgfpathlineto{\pgfqpoint{1.818274in}{1.703765in}}%
\pgfpathlineto{\pgfqpoint{1.818866in}{1.701852in}}%
\pgfpathlineto{\pgfqpoint{1.833962in}{1.702694in}}%
\pgfpathlineto{\pgfqpoint{1.834258in}{1.703276in}}%
\pgfpathlineto{\pgfqpoint{1.834554in}{1.705241in}}%
\pgfpathlineto{\pgfqpoint{1.839882in}{1.705248in}}%
\pgfpathlineto{\pgfqpoint{1.840178in}{1.705765in}}%
\pgfpathlineto{\pgfqpoint{1.841658in}{1.705599in}}%
\pgfpathlineto{\pgfqpoint{1.874811in}{1.704218in}}%
\pgfpathlineto{\pgfqpoint{1.883987in}{1.709328in}}%
\pgfpathlineto{\pgfqpoint{1.889611in}{1.709381in}}%
\pgfpathlineto{\pgfqpoint{1.890499in}{1.715480in}}%
\pgfpathlineto{\pgfqpoint{1.918323in}{1.717065in}}%
\pgfpathlineto{\pgfqpoint{1.919211in}{1.717256in}}%
\pgfpathlineto{\pgfqpoint{1.925723in}{1.717744in}}%
\pgfpathlineto{\pgfqpoint{1.926907in}{1.718706in}}%
\pgfpathlineto{\pgfqpoint{1.931348in}{1.718957in}}%
\pgfpathlineto{\pgfqpoint{1.931644in}{1.719833in}}%
\pgfpathlineto{\pgfqpoint{1.932236in}{1.723239in}}%
\pgfpathlineto{\pgfqpoint{1.932828in}{1.724105in}}%
\pgfpathlineto{\pgfqpoint{1.939044in}{1.726893in}}%
\pgfpathlineto{\pgfqpoint{1.959764in}{1.727272in}}%
\pgfpathlineto{\pgfqpoint{1.960060in}{1.727940in}}%
\pgfpathlineto{\pgfqpoint{1.960356in}{1.738569in}}%
\pgfpathlineto{\pgfqpoint{1.960652in}{1.732026in}}%
\pgfpathlineto{\pgfqpoint{1.961244in}{1.740919in}}%
\pgfpathlineto{\pgfqpoint{1.962132in}{1.748794in}}%
\pgfpathlineto{\pgfqpoint{1.962428in}{1.748999in}}%
\pgfpathlineto{\pgfqpoint{1.968940in}{1.740829in}}%
\pgfpathlineto{\pgfqpoint{1.969532in}{1.741590in}}%
\pgfpathlineto{\pgfqpoint{1.973972in}{1.750170in}}%
\pgfpathlineto{\pgfqpoint{1.974564in}{1.740507in}}%
\pgfpathlineto{\pgfqpoint{1.974860in}{1.743485in}}%
\pgfpathlineto{\pgfqpoint{1.975452in}{1.751754in}}%
\pgfpathlineto{\pgfqpoint{1.981076in}{1.740769in}}%
\pgfpathlineto{\pgfqpoint{1.981668in}{1.752738in}}%
\pgfpathlineto{\pgfqpoint{1.982260in}{1.743955in}}%
\pgfpathlineto{\pgfqpoint{1.982852in}{1.752784in}}%
\pgfpathlineto{\pgfqpoint{1.988476in}{1.740757in}}%
\pgfpathlineto{\pgfqpoint{1.989364in}{1.753024in}}%
\pgfpathlineto{\pgfqpoint{1.989660in}{1.750558in}}%
\pgfpathlineto{\pgfqpoint{1.989956in}{1.753083in}}%
\pgfpathlineto{\pgfqpoint{1.990252in}{1.759389in}}%
\pgfpathlineto{\pgfqpoint{1.992324in}{1.754550in}}%
\pgfpathlineto{\pgfqpoint{1.993212in}{1.760531in}}%
\pgfpathlineto{\pgfqpoint{2.016893in}{1.760563in}}%
\pgfpathlineto{\pgfqpoint{2.017189in}{1.762899in}}%
\pgfpathlineto{\pgfqpoint{2.019261in}{1.763130in}}%
\pgfpathlineto{\pgfqpoint{2.023997in}{1.762921in}}%
\pgfpathlineto{\pgfqpoint{2.025181in}{1.763977in}}%
\pgfpathlineto{\pgfqpoint{2.026661in}{1.761987in}}%
\pgfpathlineto{\pgfqpoint{2.030805in}{1.763144in}}%
\pgfpathlineto{\pgfqpoint{2.031397in}{1.762904in}}%
\pgfpathlineto{\pgfqpoint{2.033173in}{1.765880in}}%
\pgfpathlineto{\pgfqpoint{2.038205in}{1.763394in}}%
\pgfpathlineto{\pgfqpoint{2.038501in}{1.764259in}}%
\pgfpathlineto{\pgfqpoint{2.038797in}{1.766029in}}%
\pgfpathlineto{\pgfqpoint{2.068990in}{1.766679in}}%
\pgfpathlineto{\pgfqpoint{2.069286in}{1.768034in}}%
\pgfpathlineto{\pgfqpoint{2.069582in}{1.777375in}}%
\pgfpathlineto{\pgfqpoint{2.075798in}{1.766764in}}%
\pgfpathlineto{\pgfqpoint{2.083198in}{1.764794in}}%
\pgfpathlineto{\pgfqpoint{2.087934in}{1.764769in}}%
\pgfpathlineto{\pgfqpoint{2.088230in}{1.772999in}}%
\pgfpathlineto{\pgfqpoint{2.088526in}{1.789208in}}%
\pgfpathlineto{\pgfqpoint{2.089118in}{1.769710in}}%
\pgfpathlineto{\pgfqpoint{2.090006in}{1.801092in}}%
\pgfpathlineto{\pgfqpoint{2.090894in}{1.801756in}}%
\pgfpathlineto{\pgfqpoint{2.109838in}{1.812188in}}%
\pgfpathlineto{\pgfqpoint{2.110134in}{1.795479in}}%
\pgfpathlineto{\pgfqpoint{2.110430in}{1.795626in}}%
\pgfpathlineto{\pgfqpoint{2.110726in}{1.811498in}}%
\pgfpathlineto{\pgfqpoint{2.111318in}{1.776653in}}%
\pgfpathlineto{\pgfqpoint{2.111614in}{1.784129in}}%
\pgfpathlineto{\pgfqpoint{2.111910in}{1.810626in}}%
\pgfpathlineto{\pgfqpoint{2.112206in}{1.811845in}}%
\pgfpathlineto{\pgfqpoint{2.116646in}{1.811845in}}%
\pgfpathlineto{\pgfqpoint{2.117238in}{1.813340in}}%
\pgfpathlineto{\pgfqpoint{2.122862in}{1.814592in}}%
\pgfpathlineto{\pgfqpoint{2.124638in}{1.819816in}}%
\pgfpathlineto{\pgfqpoint{2.132039in}{1.833857in}}%
\pgfpathlineto{\pgfqpoint{2.138847in}{1.821084in}}%
\pgfpathlineto{\pgfqpoint{2.139143in}{1.821589in}}%
\pgfpathlineto{\pgfqpoint{2.140327in}{1.834337in}}%
\pgfpathlineto{\pgfqpoint{2.156015in}{1.846118in}}%
\pgfpathlineto{\pgfqpoint{2.159271in}{1.849743in}}%
\pgfpathlineto{\pgfqpoint{2.159567in}{1.850372in}}%
\pgfpathlineto{\pgfqpoint{2.160455in}{1.853552in}}%
\pgfpathlineto{\pgfqpoint{2.164007in}{1.854065in}}%
\pgfpathlineto{\pgfqpoint{2.166375in}{1.854390in}}%
\pgfpathlineto{\pgfqpoint{2.166671in}{1.837314in}}%
\pgfpathlineto{\pgfqpoint{2.166967in}{1.836718in}}%
\pgfpathlineto{\pgfqpoint{2.167559in}{1.855610in}}%
\pgfpathlineto{\pgfqpoint{2.168447in}{1.856018in}}%
\pgfpathlineto{\pgfqpoint{2.173775in}{1.855830in}}%
\pgfpathlineto{\pgfqpoint{2.174071in}{1.856124in}}%
\pgfpathlineto{\pgfqpoint{2.176439in}{1.867312in}}%
\pgfpathlineto{\pgfqpoint{2.180287in}{1.867305in}}%
\pgfpathlineto{\pgfqpoint{2.180879in}{1.863200in}}%
\pgfpathlineto{\pgfqpoint{2.181175in}{1.860592in}}%
\pgfpathlineto{\pgfqpoint{2.181767in}{1.821838in}}%
\pgfpathlineto{\pgfqpoint{2.197751in}{1.859936in}}%
\pgfpathlineto{\pgfqpoint{2.198343in}{1.858271in}}%
\pgfpathlineto{\pgfqpoint{2.208704in}{1.822229in}}%
\pgfpathlineto{\pgfqpoint{2.209296in}{1.865075in}}%
\pgfpathlineto{\pgfqpoint{2.210184in}{1.867244in}}%
\pgfpathlineto{\pgfqpoint{2.211072in}{1.868902in}}%
\pgfpathlineto{\pgfqpoint{2.216104in}{1.868906in}}%
\pgfpathlineto{\pgfqpoint{2.216696in}{1.868594in}}%
\pgfpathlineto{\pgfqpoint{2.216992in}{1.868934in}}%
\pgfpathlineto{\pgfqpoint{2.217584in}{1.870531in}}%
\pgfpathlineto{\pgfqpoint{2.230016in}{1.870586in}}%
\pgfpathlineto{\pgfqpoint{2.230608in}{1.877663in}}%
\pgfpathlineto{\pgfqpoint{2.231200in}{1.873452in}}%
\pgfpathlineto{\pgfqpoint{2.231496in}{1.873667in}}%
\pgfpathlineto{\pgfqpoint{2.232088in}{1.878046in}}%
\pgfpathlineto{\pgfqpoint{2.232384in}{1.879098in}}%
\pgfpathlineto{\pgfqpoint{2.233864in}{1.878823in}}%
\pgfpathlineto{\pgfqpoint{2.237120in}{1.878380in}}%
\pgfpathlineto{\pgfqpoint{2.237712in}{1.883383in}}%
\pgfpathlineto{\pgfqpoint{2.238304in}{1.884682in}}%
\pgfpathlineto{\pgfqpoint{2.238600in}{1.884761in}}%
\pgfpathlineto{\pgfqpoint{2.239192in}{1.886215in}}%
\pgfpathlineto{\pgfqpoint{2.239784in}{1.886428in}}%
\pgfpathlineto{\pgfqpoint{2.244816in}{1.881835in}}%
\pgfpathlineto{\pgfqpoint{2.255176in}{1.872348in}}%
\pgfpathlineto{\pgfqpoint{2.255768in}{1.875000in}}%
\pgfpathlineto{\pgfqpoint{2.258432in}{1.890204in}}%
\pgfpathlineto{\pgfqpoint{2.259024in}{1.883066in}}%
\pgfpathlineto{\pgfqpoint{2.259616in}{1.874037in}}%
\pgfpathlineto{\pgfqpoint{2.259912in}{1.879141in}}%
\pgfpathlineto{\pgfqpoint{2.260504in}{1.892933in}}%
\pgfpathlineto{\pgfqpoint{2.261096in}{1.875215in}}%
\pgfpathlineto{\pgfqpoint{2.265536in}{1.894884in}}%
\pgfpathlineto{\pgfqpoint{2.265832in}{1.894424in}}%
\pgfpathlineto{\pgfqpoint{2.266129in}{1.890895in}}%
\pgfpathlineto{\pgfqpoint{2.266721in}{1.899898in}}%
\pgfpathlineto{\pgfqpoint{2.267313in}{1.904392in}}%
\pgfpathlineto{\pgfqpoint{2.267609in}{1.909123in}}%
\pgfpathlineto{\pgfqpoint{2.267905in}{1.920306in}}%
\pgfpathlineto{\pgfqpoint{2.272937in}{1.897337in}}%
\pgfpathlineto{\pgfqpoint{2.273233in}{1.897822in}}%
\pgfpathlineto{\pgfqpoint{2.275009in}{1.926611in}}%
\pgfpathlineto{\pgfqpoint{2.280633in}{1.927232in}}%
\pgfpathlineto{\pgfqpoint{2.280929in}{1.925339in}}%
\pgfpathlineto{\pgfqpoint{2.281225in}{1.917704in}}%
\pgfpathlineto{\pgfqpoint{2.282113in}{1.937670in}}%
\pgfpathlineto{\pgfqpoint{2.288329in}{1.938874in}}%
\pgfpathlineto{\pgfqpoint{2.288625in}{1.938509in}}%
\pgfpathlineto{\pgfqpoint{2.288921in}{1.937518in}}%
\pgfpathlineto{\pgfqpoint{2.306977in}{1.938012in}}%
\pgfpathlineto{\pgfqpoint{2.309641in}{1.938115in}}%
\pgfpathlineto{\pgfqpoint{2.309937in}{1.939718in}}%
\pgfpathlineto{\pgfqpoint{2.310233in}{1.939382in}}%
\pgfpathlineto{\pgfqpoint{2.310825in}{1.940262in}}%
\pgfpathlineto{\pgfqpoint{2.311713in}{1.939422in}}%
\pgfpathlineto{\pgfqpoint{2.317337in}{1.941097in}}%
\pgfpathlineto{\pgfqpoint{2.321481in}{1.941361in}}%
\pgfpathlineto{\pgfqpoint{2.325921in}{1.941745in}}%
\pgfpathlineto{\pgfqpoint{2.326809in}{1.942530in}}%
\pgfpathlineto{\pgfqpoint{2.330065in}{1.942551in}}%
\pgfpathlineto{\pgfqpoint{2.330953in}{1.947251in}}%
\pgfpathlineto{\pgfqpoint{2.331249in}{1.946541in}}%
\pgfpathlineto{\pgfqpoint{2.331545in}{1.947400in}}%
\pgfpathlineto{\pgfqpoint{2.336874in}{1.943662in}}%
\pgfpathlineto{\pgfqpoint{2.337466in}{1.947339in}}%
\pgfpathlineto{\pgfqpoint{2.338058in}{1.943172in}}%
\pgfpathlineto{\pgfqpoint{2.339242in}{1.946069in}}%
\pgfpathlineto{\pgfqpoint{2.357890in}{1.947684in}}%
\pgfpathlineto{\pgfqpoint{2.358778in}{1.945971in}}%
\pgfpathlineto{\pgfqpoint{2.360258in}{1.942767in}}%
\pgfpathlineto{\pgfqpoint{2.361442in}{1.944650in}}%
\pgfpathlineto{\pgfqpoint{2.364994in}{1.948584in}}%
\pgfpathlineto{\pgfqpoint{2.365290in}{1.955417in}}%
\pgfpathlineto{\pgfqpoint{2.366178in}{2.005119in}}%
\pgfpathlineto{\pgfqpoint{2.367066in}{1.960625in}}%
\pgfpathlineto{\pgfqpoint{2.367362in}{2.010986in}}%
\pgfpathlineto{\pgfqpoint{2.367658in}{2.009132in}}%
\pgfpathlineto{\pgfqpoint{2.372394in}{1.949026in}}%
\pgfpathlineto{\pgfqpoint{2.380090in}{1.953465in}}%
\pgfpathlineto{\pgfqpoint{2.380978in}{1.953559in}}%
\pgfpathlineto{\pgfqpoint{2.381274in}{1.948046in}}%
\pgfpathlineto{\pgfqpoint{2.381866in}{1.915380in}}%
\pgfpathlineto{\pgfqpoint{2.382162in}{1.917247in}}%
\pgfpathlineto{\pgfqpoint{2.387194in}{2.023067in}}%
\pgfpathlineto{\pgfqpoint{2.388378in}{2.023226in}}%
\pgfpathlineto{\pgfqpoint{2.388674in}{2.024922in}}%
\pgfpathlineto{\pgfqpoint{2.395186in}{1.965124in}}%
\pgfpathlineto{\pgfqpoint{2.401107in}{1.913201in}}%
\pgfpathlineto{\pgfqpoint{2.408507in}{2.025473in}}%
\pgfpathlineto{\pgfqpoint{2.408803in}{2.024098in}}%
\pgfpathlineto{\pgfqpoint{2.409395in}{1.923616in}}%
\pgfpathlineto{\pgfqpoint{2.409691in}{1.926068in}}%
\pgfpathlineto{\pgfqpoint{2.411467in}{2.023579in}}%
\pgfpathlineto{\pgfqpoint{2.411763in}{2.025443in}}%
\pgfpathlineto{\pgfqpoint{2.414723in}{2.025490in}}%
\pgfpathlineto{\pgfqpoint{2.415019in}{2.024884in}}%
\pgfpathlineto{\pgfqpoint{2.415611in}{1.989067in}}%
\pgfpathlineto{\pgfqpoint{2.415907in}{1.987812in}}%
\pgfpathlineto{\pgfqpoint{2.421827in}{1.987537in}}%
\pgfpathlineto{\pgfqpoint{2.422419in}{1.987136in}}%
\pgfpathlineto{\pgfqpoint{2.429227in}{1.986991in}}%
\pgfpathlineto{\pgfqpoint{2.429819in}{1.988071in}}%
\pgfpathlineto{\pgfqpoint{2.430707in}{1.989419in}}%
\pgfpathlineto{\pgfqpoint{2.431003in}{1.986538in}}%
\pgfpathlineto{\pgfqpoint{2.431299in}{1.986661in}}%
\pgfpathlineto{\pgfqpoint{2.431891in}{1.989523in}}%
\pgfpathlineto{\pgfqpoint{2.433075in}{1.988676in}}%
\pgfpathlineto{\pgfqpoint{2.436035in}{1.986806in}}%
\pgfpathlineto{\pgfqpoint{2.436627in}{1.988754in}}%
\pgfpathlineto{\pgfqpoint{2.436923in}{1.988105in}}%
\pgfpathlineto{\pgfqpoint{2.439291in}{1.990988in}}%
\pgfpathlineto{\pgfqpoint{2.439587in}{1.992687in}}%
\pgfpathlineto{\pgfqpoint{2.439883in}{1.997090in}}%
\pgfpathlineto{\pgfqpoint{2.457347in}{2.006126in}}%
\pgfpathlineto{\pgfqpoint{2.458531in}{2.003895in}}%
\pgfpathlineto{\pgfqpoint{2.465043in}{1.990942in}}%
\pgfpathlineto{\pgfqpoint{2.465635in}{1.992265in}}%
\pgfpathlineto{\pgfqpoint{2.466819in}{1.995275in}}%
\pgfpathlineto{\pgfqpoint{2.468003in}{1.997959in}}%
\pgfpathlineto{\pgfqpoint{2.471260in}{2.005814in}}%
\pgfpathlineto{\pgfqpoint{2.471556in}{2.008646in}}%
\pgfpathlineto{\pgfqpoint{2.471852in}{2.014262in}}%
\pgfpathlineto{\pgfqpoint{2.556805in}{2.014262in}}%
\pgfpathlineto{\pgfqpoint{2.557397in}{2.017995in}}%
\pgfpathlineto{\pgfqpoint{2.716943in}{2.018236in}}%
\pgfpathlineto{\pgfqpoint{2.729079in}{2.021402in}}%
\pgfpathlineto{\pgfqpoint{2.738848in}{2.031589in}}%
\pgfpathlineto{\pgfqpoint{2.758088in}{2.021378in}}%
\pgfpathlineto{\pgfqpoint{2.764600in}{2.021465in}}%
\pgfpathlineto{\pgfqpoint{2.768448in}{2.027897in}}%
\pgfpathlineto{\pgfqpoint{2.771112in}{2.032378in}}%
\pgfpathlineto{\pgfqpoint{2.771704in}{2.036567in}}%
\pgfpathlineto{\pgfqpoint{2.778216in}{2.021928in}}%
\pgfpathlineto{\pgfqpoint{2.801008in}{2.022306in}}%
\pgfpathlineto{\pgfqpoint{2.807521in}{2.024923in}}%
\pgfpathlineto{\pgfqpoint{2.813441in}{2.037352in}}%
\pgfpathlineto{\pgfqpoint{2.814033in}{2.048330in}}%
\pgfpathlineto{\pgfqpoint{2.827353in}{2.047809in}}%
\pgfpathlineto{\pgfqpoint{2.828241in}{2.046386in}}%
\pgfpathlineto{\pgfqpoint{2.828833in}{2.046898in}}%
\pgfpathlineto{\pgfqpoint{2.831497in}{2.046829in}}%
\pgfpathlineto{\pgfqpoint{2.837713in}{2.046343in}}%
\pgfpathlineto{\pgfqpoint{2.876490in}{2.048205in}}%
\pgfpathlineto{\pgfqpoint{2.877082in}{2.049886in}}%
\pgfpathlineto{\pgfqpoint{2.878562in}{2.050513in}}%
\pgfpathlineto{\pgfqpoint{2.885962in}{2.045805in}}%
\pgfpathlineto{\pgfqpoint{2.906090in}{2.046706in}}%
\pgfpathlineto{\pgfqpoint{2.910530in}{2.052283in}}%
\pgfpathlineto{\pgfqpoint{2.912010in}{2.054152in}}%
\pgfpathlineto{\pgfqpoint{2.912306in}{2.053852in}}%
\pgfpathlineto{\pgfqpoint{2.912602in}{2.052289in}}%
\pgfpathlineto{\pgfqpoint{2.914378in}{2.052274in}}%
\pgfpathlineto{\pgfqpoint{2.921186in}{2.052579in}}%
\pgfpathlineto{\pgfqpoint{2.922962in}{2.053095in}}%
\pgfpathlineto{\pgfqpoint{2.925922in}{2.053524in}}%
\pgfpathlineto{\pgfqpoint{2.926218in}{2.056211in}}%
\pgfpathlineto{\pgfqpoint{2.926514in}{2.069516in}}%
\pgfpathlineto{\pgfqpoint{2.926810in}{2.069244in}}%
\pgfpathlineto{\pgfqpoint{2.927402in}{2.067924in}}%
\pgfpathlineto{\pgfqpoint{2.936578in}{2.069950in}}%
\pgfpathlineto{\pgfqpoint{2.954635in}{2.069550in}}%
\pgfpathlineto{\pgfqpoint{2.954931in}{2.069166in}}%
\pgfpathlineto{\pgfqpoint{2.955523in}{2.082732in}}%
\pgfpathlineto{\pgfqpoint{2.961739in}{2.069791in}}%
\pgfpathlineto{\pgfqpoint{2.962331in}{2.079350in}}%
\pgfpathlineto{\pgfqpoint{2.964699in}{2.079690in}}%
\pgfpathlineto{\pgfqpoint{2.969139in}{2.080259in}}%
\pgfpathlineto{\pgfqpoint{2.969731in}{2.080703in}}%
\pgfpathlineto{\pgfqpoint{2.971211in}{2.079550in}}%
\pgfpathlineto{\pgfqpoint{2.977427in}{2.081198in}}%
\pgfpathlineto{\pgfqpoint{3.004363in}{2.083148in}}%
\pgfpathlineto{\pgfqpoint{3.004659in}{2.083922in}}%
\pgfpathlineto{\pgfqpoint{3.005547in}{2.090215in}}%
\pgfpathlineto{\pgfqpoint{3.006139in}{2.090417in}}%
\pgfpathlineto{\pgfqpoint{3.020348in}{2.090634in}}%
\pgfpathlineto{\pgfqpoint{3.021236in}{2.092619in}}%
\pgfpathlineto{\pgfqpoint{3.026564in}{2.105812in}}%
\pgfpathlineto{\pgfqpoint{3.026860in}{2.105441in}}%
\pgfpathlineto{\pgfqpoint{3.028044in}{2.100348in}}%
\pgfpathlineto{\pgfqpoint{3.028340in}{2.106632in}}%
\pgfpathlineto{\pgfqpoint{3.033372in}{2.094485in}}%
\pgfpathlineto{\pgfqpoint{3.033964in}{2.113801in}}%
\pgfpathlineto{\pgfqpoint{3.036036in}{2.130105in}}%
\pgfpathlineto{\pgfqpoint{3.036628in}{2.129496in}}%
\pgfpathlineto{\pgfqpoint{3.054684in}{2.105784in}}%
\pgfpathlineto{\pgfqpoint{3.055276in}{2.105858in}}%
\pgfpathlineto{\pgfqpoint{3.055868in}{2.106580in}}%
\pgfpathlineto{\pgfqpoint{3.056164in}{2.112958in}}%
\pgfpathlineto{\pgfqpoint{3.056460in}{2.131940in}}%
\pgfpathlineto{\pgfqpoint{3.058828in}{2.120128in}}%
\pgfpathlineto{\pgfqpoint{3.061196in}{2.108146in}}%
\pgfpathlineto{\pgfqpoint{3.062084in}{2.109038in}}%
\pgfpathlineto{\pgfqpoint{3.062380in}{2.123752in}}%
\pgfpathlineto{\pgfqpoint{3.062676in}{2.115002in}}%
\pgfpathlineto{\pgfqpoint{3.063268in}{2.122722in}}%
\pgfpathlineto{\pgfqpoint{3.063564in}{2.109777in}}%
\pgfpathlineto{\pgfqpoint{3.075997in}{2.132498in}}%
\pgfpathlineto{\pgfqpoint{3.098493in}{2.125071in}}%
\pgfpathlineto{\pgfqpoint{3.104117in}{2.125669in}}%
\pgfpathlineto{\pgfqpoint{3.104709in}{2.128124in}}%
\pgfpathlineto{\pgfqpoint{3.108853in}{2.128263in}}%
\pgfpathlineto{\pgfqpoint{3.111813in}{2.128197in}}%
\pgfpathlineto{\pgfqpoint{3.112109in}{2.174068in}}%
\pgfpathlineto{\pgfqpoint{3.112405in}{2.678739in}}%
\pgfpathlineto{\pgfqpoint{3.112701in}{2.678775in}}%
\pgfpathlineto{\pgfqpoint{3.112997in}{2.676553in}}%
\pgfpathlineto{\pgfqpoint{3.122469in}{2.426763in}}%
\pgfpathlineto{\pgfqpoint{3.133421in}{2.137679in}}%
\pgfpathlineto{\pgfqpoint{3.133717in}{2.138451in}}%
\pgfpathlineto{\pgfqpoint{3.134901in}{2.178650in}}%
\pgfpathlineto{\pgfqpoint{3.154734in}{2.671417in}}%
\pgfpathlineto{\pgfqpoint{3.155030in}{2.618186in}}%
\pgfpathlineto{\pgfqpoint{3.155622in}{2.400367in}}%
\pgfpathlineto{\pgfqpoint{3.160950in}{2.168992in}}%
\pgfpathlineto{\pgfqpoint{3.161542in}{2.164360in}}%
\pgfpathlineto{\pgfqpoint{3.162134in}{2.172496in}}%
\pgfpathlineto{\pgfqpoint{3.163318in}{2.168492in}}%
\pgfpathlineto{\pgfqpoint{3.169830in}{2.174348in}}%
\pgfpathlineto{\pgfqpoint{3.173086in}{2.174462in}}%
\pgfpathlineto{\pgfqpoint{3.181966in}{2.174424in}}%
\pgfpathlineto{\pgfqpoint{3.182262in}{2.173240in}}%
\pgfpathlineto{\pgfqpoint{3.182558in}{2.176996in}}%
\pgfpathlineto{\pgfqpoint{3.183150in}{2.175807in}}%
\pgfpathlineto{\pgfqpoint{3.184334in}{2.174590in}}%
\pgfpathlineto{\pgfqpoint{3.185814in}{2.174770in}}%
\pgfpathlineto{\pgfqpoint{3.203574in}{2.176177in}}%
\pgfpathlineto{\pgfqpoint{3.203870in}{2.176588in}}%
\pgfpathlineto{\pgfqpoint{3.204166in}{2.178030in}}%
\pgfpathlineto{\pgfqpoint{3.217191in}{2.177954in}}%
\pgfpathlineto{\pgfqpoint{3.217783in}{2.726733in}}%
\pgfpathlineto{\pgfqpoint{3.218079in}{2.726558in}}%
\pgfpathlineto{\pgfqpoint{3.233175in}{2.717838in}}%
\pgfpathlineto{\pgfqpoint{3.274319in}{2.717838in}}%
\pgfpathlineto{\pgfqpoint{3.274615in}{2.721357in}}%
\pgfpathlineto{\pgfqpoint{3.274911in}{2.730603in}}%
\pgfpathlineto{\pgfqpoint{3.302736in}{2.183829in}}%
\pgfpathlineto{\pgfqpoint{3.303032in}{2.183544in}}%
\pgfpathlineto{\pgfqpoint{3.316352in}{2.191429in}}%
\pgfpathlineto{\pgfqpoint{3.316648in}{2.200784in}}%
\pgfpathlineto{\pgfqpoint{3.316944in}{2.624372in}}%
\pgfpathlineto{\pgfqpoint{3.317240in}{2.298336in}}%
\pgfpathlineto{\pgfqpoint{3.317536in}{2.400907in}}%
\pgfpathlineto{\pgfqpoint{3.317832in}{2.653851in}}%
\pgfpathlineto{\pgfqpoint{3.318128in}{2.558920in}}%
\pgfpathlineto{\pgfqpoint{3.318424in}{2.312060in}}%
\pgfpathlineto{\pgfqpoint{3.319016in}{2.486531in}}%
\pgfpathlineto{\pgfqpoint{3.325528in}{2.743278in}}%
\pgfpathlineto{\pgfqpoint{3.326416in}{2.743427in}}%
\pgfpathlineto{\pgfqpoint{3.353649in}{2.743666in}}%
\pgfpathlineto{\pgfqpoint{3.359273in}{2.744633in}}%
\pgfpathlineto{\pgfqpoint{3.359569in}{2.745162in}}%
\pgfpathlineto{\pgfqpoint{3.359865in}{2.745089in}}%
\pgfpathlineto{\pgfqpoint{3.360161in}{2.745381in}}%
\pgfpathlineto{\pgfqpoint{3.360753in}{2.744813in}}%
\pgfpathlineto{\pgfqpoint{3.367265in}{2.741644in}}%
\pgfpathlineto{\pgfqpoint{3.373777in}{2.743291in}}%
\pgfpathlineto{\pgfqpoint{3.374961in}{2.744750in}}%
\pgfpathlineto{\pgfqpoint{3.375257in}{2.743038in}}%
\pgfpathlineto{\pgfqpoint{3.375849in}{2.745552in}}%
\pgfpathlineto{\pgfqpoint{3.380585in}{2.745666in}}%
\pgfpathlineto{\pgfqpoint{3.380881in}{2.746837in}}%
\pgfpathlineto{\pgfqpoint{3.381473in}{2.744898in}}%
\pgfpathlineto{\pgfqpoint{3.381769in}{2.747734in}}%
\pgfpathlineto{\pgfqpoint{3.382065in}{2.747753in}}%
\pgfpathlineto{\pgfqpoint{3.382361in}{2.615513in}}%
\pgfpathlineto{\pgfqpoint{3.382657in}{2.324880in}}%
\pgfpathlineto{\pgfqpoint{3.382953in}{2.444520in}}%
\pgfpathlineto{\pgfqpoint{3.383249in}{2.746747in}}%
\pgfpathlineto{\pgfqpoint{3.383545in}{2.749060in}}%
\pgfpathlineto{\pgfqpoint{3.408705in}{2.749060in}}%
\pgfpathlineto{\pgfqpoint{3.409297in}{2.450221in}}%
\pgfpathlineto{\pgfqpoint{3.409593in}{2.743110in}}%
\pgfpathlineto{\pgfqpoint{3.410481in}{2.749283in}}%
\pgfpathlineto{\pgfqpoint{3.410777in}{2.748626in}}%
\pgfpathlineto{\pgfqpoint{3.411073in}{2.740166in}}%
\pgfpathlineto{\pgfqpoint{3.411961in}{2.740028in}}%
\pgfpathlineto{\pgfqpoint{3.416106in}{2.740083in}}%
\pgfpathlineto{\pgfqpoint{3.419362in}{2.744172in}}%
\pgfpathlineto{\pgfqpoint{3.423506in}{2.749426in}}%
\pgfpathlineto{\pgfqpoint{3.423802in}{2.705433in}}%
\pgfpathlineto{\pgfqpoint{3.424690in}{2.375133in}}%
\pgfpathlineto{\pgfqpoint{3.424986in}{2.740913in}}%
\pgfpathlineto{\pgfqpoint{3.425578in}{2.744752in}}%
\pgfpathlineto{\pgfqpoint{3.431498in}{2.745759in}}%
\pgfpathlineto{\pgfqpoint{3.431794in}{2.618498in}}%
\pgfpathlineto{\pgfqpoint{3.432090in}{2.645011in}}%
\pgfpathlineto{\pgfqpoint{3.432386in}{2.746051in}}%
\pgfpathlineto{\pgfqpoint{3.433274in}{2.723026in}}%
\pgfpathlineto{\pgfqpoint{3.451034in}{2.220525in}}%
\pgfpathlineto{\pgfqpoint{3.451330in}{2.235101in}}%
\pgfpathlineto{\pgfqpoint{3.451626in}{2.758738in}}%
\pgfpathlineto{\pgfqpoint{3.452218in}{2.496139in}}%
\pgfpathlineto{\pgfqpoint{3.459914in}{2.762976in}}%
\pgfpathlineto{\pgfqpoint{3.460506in}{2.763287in}}%
\pgfpathlineto{\pgfqpoint{3.460802in}{2.763261in}}%
\pgfpathlineto{\pgfqpoint{3.461098in}{2.478361in}}%
\pgfpathlineto{\pgfqpoint{3.461394in}{2.761866in}}%
\pgfpathlineto{\pgfqpoint{3.461690in}{2.763052in}}%
\pgfpathlineto{\pgfqpoint{3.473234in}{2.757865in}}%
\pgfpathlineto{\pgfqpoint{3.473530in}{2.727676in}}%
\pgfpathlineto{\pgfqpoint{3.475602in}{2.210338in}}%
\pgfpathlineto{\pgfqpoint{3.476194in}{2.210132in}}%
\pgfpathlineto{\pgfqpoint{3.492179in}{2.210209in}}%
\pgfpathlineto{\pgfqpoint{3.501355in}{2.210364in}}%
\pgfpathlineto{\pgfqpoint{3.502243in}{2.767291in}}%
\pgfpathlineto{\pgfqpoint{3.504907in}{2.767325in}}%
\pgfpathlineto{\pgfqpoint{3.508163in}{2.767325in}}%
\pgfpathlineto{\pgfqpoint{3.508459in}{2.746338in}}%
\pgfpathlineto{\pgfqpoint{3.508755in}{2.210083in}}%
\pgfpathlineto{\pgfqpoint{3.509347in}{2.211165in}}%
\pgfpathlineto{\pgfqpoint{3.510531in}{2.211291in}}%
\pgfpathlineto{\pgfqpoint{3.510827in}{2.218384in}}%
\pgfpathlineto{\pgfqpoint{3.522371in}{2.754634in}}%
\pgfpathlineto{\pgfqpoint{3.522667in}{2.240120in}}%
\pgfpathlineto{\pgfqpoint{3.522963in}{2.211765in}}%
\pgfpathlineto{\pgfqpoint{3.524443in}{2.211911in}}%
\pgfpathlineto{\pgfqpoint{3.524739in}{2.211959in}}%
\pgfpathlineto{\pgfqpoint{3.525331in}{2.212431in}}%
\pgfpathlineto{\pgfqpoint{3.607916in}{2.212431in}}%
\pgfpathlineto{\pgfqpoint{3.608508in}{2.652512in}}%
\pgfpathlineto{\pgfqpoint{3.609692in}{2.214779in}}%
\pgfpathlineto{\pgfqpoint{3.610284in}{2.217985in}}%
\pgfpathlineto{\pgfqpoint{3.614725in}{2.217985in}}%
\pgfpathlineto{\pgfqpoint{3.615317in}{2.216964in}}%
\pgfpathlineto{\pgfqpoint{3.615909in}{2.216412in}}%
\pgfpathlineto{\pgfqpoint{3.650541in}{2.216412in}}%
\pgfpathlineto{\pgfqpoint{3.651725in}{2.212083in}}%
\pgfpathlineto{\pgfqpoint{3.657941in}{2.212281in}}%
\pgfpathlineto{\pgfqpoint{3.659717in}{2.217391in}}%
\pgfpathlineto{\pgfqpoint{3.660013in}{2.217398in}}%
\pgfpathlineto{\pgfqpoint{3.660605in}{2.215878in}}%
\pgfpathlineto{\pgfqpoint{3.661789in}{2.212764in}}%
\pgfpathlineto{\pgfqpoint{3.672445in}{2.212724in}}%
\pgfpathlineto{\pgfqpoint{3.673333in}{2.214568in}}%
\pgfpathlineto{\pgfqpoint{3.673629in}{2.214161in}}%
\pgfpathlineto{\pgfqpoint{3.673925in}{2.214672in}}%
\pgfpathlineto{\pgfqpoint{3.674813in}{2.213852in}}%
\pgfpathlineto{\pgfqpoint{3.678661in}{2.214770in}}%
\pgfpathlineto{\pgfqpoint{3.678957in}{2.213625in}}%
\pgfpathlineto{\pgfqpoint{3.679549in}{2.219056in}}%
\pgfpathlineto{\pgfqpoint{3.679845in}{2.212773in}}%
\pgfpathlineto{\pgfqpoint{3.680733in}{2.214238in}}%
\pgfpathlineto{\pgfqpoint{3.681621in}{2.227111in}}%
\pgfpathlineto{\pgfqpoint{3.701750in}{2.231842in}}%
\pgfpathlineto{\pgfqpoint{3.702638in}{2.229080in}}%
\pgfpathlineto{\pgfqpoint{3.707374in}{2.212905in}}%
\pgfpathlineto{\pgfqpoint{3.707966in}{2.214640in}}%
\pgfpathlineto{\pgfqpoint{3.709150in}{2.214855in}}%
\pgfpathlineto{\pgfqpoint{3.710038in}{2.232307in}}%
\pgfpathlineto{\pgfqpoint{3.710334in}{2.232141in}}%
\pgfpathlineto{\pgfqpoint{3.714774in}{2.229821in}}%
\pgfpathlineto{\pgfqpoint{3.716846in}{2.230048in}}%
\pgfpathlineto{\pgfqpoint{3.751479in}{2.230319in}}%
\pgfpathlineto{\pgfqpoint{3.752959in}{2.230519in}}%
\pgfpathlineto{\pgfqpoint{3.756807in}{2.230444in}}%
\pgfpathlineto{\pgfqpoint{3.757399in}{2.229377in}}%
\pgfpathlineto{\pgfqpoint{3.758879in}{2.230605in}}%
\pgfpathlineto{\pgfqpoint{3.759767in}{2.230367in}}%
\pgfpathlineto{\pgfqpoint{3.765391in}{2.229743in}}%
\pgfpathlineto{\pgfqpoint{3.765687in}{2.229857in}}%
\pgfpathlineto{\pgfqpoint{3.765983in}{2.230324in}}%
\pgfpathlineto{\pgfqpoint{3.766279in}{2.231414in}}%
\pgfpathlineto{\pgfqpoint{3.771607in}{2.231414in}}%
\pgfpathlineto{\pgfqpoint{3.771903in}{2.230871in}}%
\pgfpathlineto{\pgfqpoint{3.772495in}{2.235139in}}%
\pgfpathlineto{\pgfqpoint{3.772791in}{2.231642in}}%
\pgfpathlineto{\pgfqpoint{3.773087in}{2.232691in}}%
\pgfpathlineto{\pgfqpoint{3.773383in}{2.238641in}}%
\pgfpathlineto{\pgfqpoint{3.777527in}{2.238728in}}%
\pgfpathlineto{\pgfqpoint{3.786999in}{2.239155in}}%
\pgfpathlineto{\pgfqpoint{3.799727in}{2.239809in}}%
\pgfpathlineto{\pgfqpoint{3.800319in}{2.239216in}}%
\pgfpathlineto{\pgfqpoint{3.800911in}{2.239313in}}%
\pgfpathlineto{\pgfqpoint{3.801503in}{2.240163in}}%
\pgfpathlineto{\pgfqpoint{3.806831in}{2.228339in}}%
\pgfpathlineto{\pgfqpoint{3.807423in}{2.238286in}}%
\pgfpathlineto{\pgfqpoint{3.808607in}{2.239959in}}%
\pgfpathlineto{\pgfqpoint{3.808903in}{2.239089in}}%
\pgfpathlineto{\pgfqpoint{3.809495in}{2.236654in}}%
\pgfpathlineto{\pgfqpoint{3.811271in}{2.236731in}}%
\pgfpathlineto{\pgfqpoint{3.815119in}{2.237158in}}%
\pgfpathlineto{\pgfqpoint{3.815711in}{2.237446in}}%
\pgfpathlineto{\pgfqpoint{3.820744in}{2.237179in}}%
\pgfpathlineto{\pgfqpoint{3.821040in}{2.237641in}}%
\pgfpathlineto{\pgfqpoint{3.821632in}{2.239242in}}%
\pgfpathlineto{\pgfqpoint{3.844720in}{2.239269in}}%
\pgfpathlineto{\pgfqpoint{3.856856in}{2.248960in}}%
\pgfpathlineto{\pgfqpoint{3.859224in}{2.239200in}}%
\pgfpathlineto{\pgfqpoint{3.863664in}{2.239337in}}%
\pgfpathlineto{\pgfqpoint{3.863960in}{2.238815in}}%
\pgfpathlineto{\pgfqpoint{3.864552in}{2.239136in}}%
\pgfpathlineto{\pgfqpoint{3.870768in}{2.239136in}}%
\pgfpathlineto{\pgfqpoint{3.871360in}{2.247024in}}%
\pgfpathlineto{\pgfqpoint{3.872248in}{2.250068in}}%
\pgfpathlineto{\pgfqpoint{3.872840in}{2.239470in}}%
\pgfpathlineto{\pgfqpoint{3.878464in}{2.240097in}}%
\pgfpathlineto{\pgfqpoint{3.879056in}{2.238514in}}%
\pgfpathlineto{\pgfqpoint{3.898889in}{2.239424in}}%
\pgfpathlineto{\pgfqpoint{3.899185in}{2.240323in}}%
\pgfpathlineto{\pgfqpoint{3.899481in}{2.242429in}}%
\pgfpathlineto{\pgfqpoint{3.900073in}{2.257646in}}%
\pgfpathlineto{\pgfqpoint{3.901257in}{2.242182in}}%
\pgfpathlineto{\pgfqpoint{3.907473in}{2.243171in}}%
\pgfpathlineto{\pgfqpoint{3.907769in}{2.244558in}}%
\pgfpathlineto{\pgfqpoint{3.908953in}{2.260077in}}%
\pgfpathlineto{\pgfqpoint{3.909249in}{2.271606in}}%
\pgfpathlineto{\pgfqpoint{3.913097in}{2.246818in}}%
\pgfpathlineto{\pgfqpoint{3.913985in}{2.272643in}}%
\pgfpathlineto{\pgfqpoint{3.914281in}{2.273009in}}%
\pgfpathlineto{\pgfqpoint{3.914873in}{2.275684in}}%
\pgfpathlineto{\pgfqpoint{3.922865in}{2.275945in}}%
\pgfpathlineto{\pgfqpoint{3.934705in}{2.276054in}}%
\pgfpathlineto{\pgfqpoint{3.950689in}{2.275865in}}%
\pgfpathlineto{\pgfqpoint{3.951578in}{2.282396in}}%
\pgfpathlineto{\pgfqpoint{3.971706in}{2.282624in}}%
\pgfpathlineto{\pgfqpoint{3.972002in}{2.283525in}}%
\pgfpathlineto{\pgfqpoint{3.973482in}{2.305410in}}%
\pgfpathlineto{\pgfqpoint{3.982066in}{2.306242in}}%
\pgfpathlineto{\pgfqpoint{4.000418in}{2.306420in}}%
\pgfpathlineto{\pgfqpoint{4.002194in}{2.296179in}}%
\pgfpathlineto{\pgfqpoint{4.005746in}{2.275817in}}%
\pgfpathlineto{\pgfqpoint{4.007226in}{2.285749in}}%
\pgfpathlineto{\pgfqpoint{4.012554in}{2.307989in}}%
\pgfpathlineto{\pgfqpoint{4.012850in}{2.307900in}}%
\pgfpathlineto{\pgfqpoint{4.020251in}{2.275891in}}%
\pgfpathlineto{\pgfqpoint{4.020547in}{2.281281in}}%
\pgfpathlineto{\pgfqpoint{4.021139in}{2.306369in}}%
\pgfpathlineto{\pgfqpoint{4.021435in}{2.308304in}}%
\pgfpathlineto{\pgfqpoint{4.021731in}{2.308303in}}%
\pgfpathlineto{\pgfqpoint{4.022323in}{2.308850in}}%
\pgfpathlineto{\pgfqpoint{4.049259in}{2.316176in}}%
\pgfpathlineto{\pgfqpoint{4.069979in}{2.309281in}}%
\pgfpathlineto{\pgfqpoint{4.070275in}{2.309739in}}%
\pgfpathlineto{\pgfqpoint{4.071459in}{2.319863in}}%
\pgfpathlineto{\pgfqpoint{4.106684in}{2.311905in}}%
\pgfpathlineto{\pgfqpoint{4.106980in}{2.311905in}}%
\pgfpathlineto{\pgfqpoint{4.107276in}{2.312209in}}%
\pgfpathlineto{\pgfqpoint{4.110828in}{2.331820in}}%
\pgfpathlineto{\pgfqpoint{4.111716in}{2.336754in}}%
\pgfpathlineto{\pgfqpoint{4.112012in}{2.340134in}}%
\pgfpathlineto{\pgfqpoint{4.112308in}{2.348631in}}%
\pgfpathlineto{\pgfqpoint{4.112900in}{2.348831in}}%
\pgfpathlineto{\pgfqpoint{4.119708in}{2.349194in}}%
\pgfpathlineto{\pgfqpoint{4.120596in}{2.350029in}}%
\pgfpathlineto{\pgfqpoint{4.127108in}{2.359123in}}%
\pgfpathlineto{\pgfqpoint{4.127996in}{2.359312in}}%
\pgfpathlineto{\pgfqpoint{4.129772in}{2.364345in}}%
\pgfpathlineto{\pgfqpoint{4.148716in}{2.378578in}}%
\pgfpathlineto{\pgfqpoint{4.152860in}{2.373238in}}%
\pgfpathlineto{\pgfqpoint{4.155821in}{2.369055in}}%
\pgfpathlineto{\pgfqpoint{4.162333in}{2.358315in}}%
\pgfpathlineto{\pgfqpoint{4.162925in}{2.358516in}}%
\pgfpathlineto{\pgfqpoint{4.163221in}{2.358715in}}%
\pgfpathlineto{\pgfqpoint{4.163813in}{2.360799in}}%
\pgfpathlineto{\pgfqpoint{4.164405in}{2.362110in}}%
\pgfpathlineto{\pgfqpoint{4.170029in}{2.364058in}}%
\pgfpathlineto{\pgfqpoint{4.170621in}{2.363912in}}%
\pgfpathlineto{\pgfqpoint{4.170917in}{2.360402in}}%
\pgfpathlineto{\pgfqpoint{4.171509in}{2.365693in}}%
\pgfpathlineto{\pgfqpoint{4.172101in}{2.365321in}}%
\pgfpathlineto{\pgfqpoint{4.177725in}{2.359523in}}%
\pgfpathlineto{\pgfqpoint{4.178021in}{2.360123in}}%
\pgfpathlineto{\pgfqpoint{4.178909in}{2.369137in}}%
\pgfpathlineto{\pgfqpoint{4.180981in}{2.368841in}}%
\pgfpathlineto{\pgfqpoint{4.197261in}{2.366172in}}%
\pgfpathlineto{\pgfqpoint{4.197853in}{2.369361in}}%
\pgfpathlineto{\pgfqpoint{4.198741in}{2.366412in}}%
\pgfpathlineto{\pgfqpoint{4.199037in}{2.366623in}}%
\pgfpathlineto{\pgfqpoint{4.199333in}{2.365801in}}%
\pgfpathlineto{\pgfqpoint{4.199925in}{2.349945in}}%
\pgfpathlineto{\pgfqpoint{4.202293in}{2.354125in}}%
\pgfpathlineto{\pgfqpoint{4.204365in}{2.357841in}}%
\pgfpathlineto{\pgfqpoint{4.204957in}{2.378985in}}%
\pgfpathlineto{\pgfqpoint{4.205549in}{2.542649in}}%
\pgfpathlineto{\pgfqpoint{4.207029in}{2.971010in}}%
\pgfpathlineto{\pgfqpoint{4.208213in}{2.974062in}}%
\pgfpathlineto{\pgfqpoint{4.211469in}{2.982641in}}%
\pgfpathlineto{\pgfqpoint{4.226270in}{2.982609in}}%
\pgfpathlineto{\pgfqpoint{4.278070in}{2.982932in}}%
\pgfpathlineto{\pgfqpoint{4.278958in}{2.983063in}}%
\pgfpathlineto{\pgfqpoint{4.324543in}{2.982916in}}%
\pgfpathlineto{\pgfqpoint{4.327799in}{2.983415in}}%
\pgfpathlineto{\pgfqpoint{4.354439in}{2.995395in}}%
\pgfpathlineto{\pgfqpoint{4.355031in}{3.002853in}}%
\pgfpathlineto{\pgfqpoint{4.377528in}{3.002321in}}%
\pgfpathlineto{\pgfqpoint{4.398840in}{2.983757in}}%
\pgfpathlineto{\pgfqpoint{4.405944in}{2.983511in}}%
\pgfpathlineto{\pgfqpoint{4.410384in}{2.980807in}}%
\pgfpathlineto{\pgfqpoint{4.411272in}{3.000453in}}%
\pgfpathlineto{\pgfqpoint{4.421632in}{3.003800in}}%
\pgfpathlineto{\pgfqpoint{4.455969in}{3.018857in}}%
\pgfpathlineto{\pgfqpoint{4.457449in}{3.016812in}}%
\pgfpathlineto{\pgfqpoint{4.461593in}{3.010828in}}%
\pgfpathlineto{\pgfqpoint{4.462185in}{3.019299in}}%
\pgfpathlineto{\pgfqpoint{4.462481in}{3.019186in}}%
\pgfpathlineto{\pgfqpoint{4.463073in}{3.018802in}}%
\pgfpathlineto{\pgfqpoint{4.470177in}{3.007808in}}%
\pgfpathlineto{\pgfqpoint{4.497410in}{3.007892in}}%
\pgfpathlineto{\pgfqpoint{4.498002in}{3.010919in}}%
\pgfpathlineto{\pgfqpoint{4.504514in}{3.008396in}}%
\pgfpathlineto{\pgfqpoint{4.504810in}{3.008615in}}%
\pgfpathlineto{\pgfqpoint{4.505698in}{3.012800in}}%
\pgfpathlineto{\pgfqpoint{4.507178in}{3.011361in}}%
\pgfpathlineto{\pgfqpoint{4.509842in}{3.008624in}}%
\pgfpathlineto{\pgfqpoint{4.510434in}{3.008682in}}%
\pgfpathlineto{\pgfqpoint{4.511026in}{3.008460in}}%
\pgfpathlineto{\pgfqpoint{4.534410in}{3.008734in}}%
\pgfpathlineto{\pgfqpoint{4.575851in}{3.009256in}}%
\pgfpathlineto{\pgfqpoint{4.576443in}{3.008465in}}%
\pgfpathlineto{\pgfqpoint{4.596571in}{3.008915in}}%
\pgfpathlineto{\pgfqpoint{4.597163in}{3.009802in}}%
\pgfpathlineto{\pgfqpoint{4.597755in}{3.010063in}}%
\pgfpathlineto{\pgfqpoint{4.603675in}{3.008857in}}%
\pgfpathlineto{\pgfqpoint{4.607227in}{3.008976in}}%
\pgfpathlineto{\pgfqpoint{4.611371in}{3.009540in}}%
\pgfpathlineto{\pgfqpoint{4.625580in}{3.012728in}}%
\pgfpathlineto{\pgfqpoint{4.629132in}{3.012162in}}%
\pgfpathlineto{\pgfqpoint{4.645116in}{3.009440in}}%
\pgfpathlineto{\pgfqpoint{4.645708in}{2.998170in}}%
\pgfpathlineto{\pgfqpoint{4.646004in}{2.998328in}}%
\pgfpathlineto{\pgfqpoint{4.652516in}{3.009194in}}%
\pgfpathlineto{\pgfqpoint{4.652812in}{3.008794in}}%
\pgfpathlineto{\pgfqpoint{4.653108in}{2.998461in}}%
\pgfpathlineto{\pgfqpoint{4.659324in}{3.016437in}}%
\pgfpathlineto{\pgfqpoint{4.661396in}{3.016963in}}%
\pgfpathlineto{\pgfqpoint{4.663764in}{3.010957in}}%
\pgfpathlineto{\pgfqpoint{4.668796in}{2.998246in}}%
\pgfpathlineto{\pgfqpoint{4.669980in}{2.998630in}}%
\pgfpathlineto{\pgfqpoint{4.675012in}{2.999803in}}%
\pgfpathlineto{\pgfqpoint{4.675900in}{2.998532in}}%
\pgfpathlineto{\pgfqpoint{4.676196in}{3.000419in}}%
\pgfpathlineto{\pgfqpoint{4.702245in}{3.001001in}}%
\pgfpathlineto{\pgfqpoint{4.702837in}{3.001897in}}%
\pgfpathlineto{\pgfqpoint{4.703725in}{3.003573in}}%
\pgfpathlineto{\pgfqpoint{4.704317in}{3.001063in}}%
\pgfpathlineto{\pgfqpoint{4.709053in}{3.003666in}}%
\pgfpathlineto{\pgfqpoint{4.710237in}{3.002521in}}%
\pgfpathlineto{\pgfqpoint{4.711125in}{3.001589in}}%
\pgfpathlineto{\pgfqpoint{4.711421in}{3.002149in}}%
\pgfpathlineto{\pgfqpoint{4.712013in}{3.003843in}}%
\pgfpathlineto{\pgfqpoint{4.720301in}{3.007271in}}%
\pgfpathlineto{\pgfqpoint{4.724741in}{3.008825in}}%
\pgfpathlineto{\pgfqpoint{4.725333in}{3.009317in}}%
\pgfpathlineto{\pgfqpoint{4.725925in}{3.009080in}}%
\pgfpathlineto{\pgfqpoint{4.743981in}{3.009681in}}%
\pgfpathlineto{\pgfqpoint{4.744869in}{3.011435in}}%
\pgfpathlineto{\pgfqpoint{4.745461in}{3.019752in}}%
\pgfpathlineto{\pgfqpoint{4.746053in}{2.420220in}}%
\pgfpathlineto{\pgfqpoint{4.746349in}{2.604506in}}%
\pgfpathlineto{\pgfqpoint{4.746645in}{2.420829in}}%
\pgfpathlineto{\pgfqpoint{4.746941in}{2.421986in}}%
\pgfpathlineto{\pgfqpoint{4.747237in}{2.429883in}}%
\pgfpathlineto{\pgfqpoint{4.751085in}{2.888528in}}%
\pgfpathlineto{\pgfqpoint{4.751973in}{2.994956in}}%
\pgfpathlineto{\pgfqpoint{4.752565in}{2.663086in}}%
\pgfpathlineto{\pgfqpoint{4.752861in}{2.421901in}}%
\pgfpathlineto{\pgfqpoint{4.753157in}{2.478786in}}%
\pgfpathlineto{\pgfqpoint{4.753453in}{2.855017in}}%
\pgfpathlineto{\pgfqpoint{4.754045in}{2.423518in}}%
\pgfpathlineto{\pgfqpoint{4.755229in}{2.423529in}}%
\pgfpathlineto{\pgfqpoint{4.758781in}{2.423796in}}%
\pgfpathlineto{\pgfqpoint{4.759077in}{2.703723in}}%
\pgfpathlineto{\pgfqpoint{4.759373in}{2.633849in}}%
\pgfpathlineto{\pgfqpoint{4.759669in}{2.423957in}}%
\pgfpathlineto{\pgfqpoint{4.760558in}{2.423846in}}%
\pgfpathlineto{\pgfqpoint{4.760854in}{2.423054in}}%
\pgfpathlineto{\pgfqpoint{4.761150in}{2.420485in}}%
\pgfpathlineto{\pgfqpoint{4.761742in}{2.410618in}}%
\pgfpathlineto{\pgfqpoint{4.766478in}{2.423336in}}%
\pgfpathlineto{\pgfqpoint{4.766774in}{2.425197in}}%
\pgfpathlineto{\pgfqpoint{4.767070in}{2.465139in}}%
\pgfpathlineto{\pgfqpoint{4.767366in}{2.690242in}}%
\pgfpathlineto{\pgfqpoint{4.767662in}{2.437764in}}%
\pgfpathlineto{\pgfqpoint{4.768550in}{2.438538in}}%
\pgfpathlineto{\pgfqpoint{4.774174in}{2.439000in}}%
\pgfpathlineto{\pgfqpoint{4.775654in}{2.442832in}}%
\pgfpathlineto{\pgfqpoint{4.776246in}{2.456922in}}%
\pgfpathlineto{\pgfqpoint{4.794302in}{3.019181in}}%
\pgfpathlineto{\pgfqpoint{4.794894in}{2.837416in}}%
\pgfpathlineto{\pgfqpoint{4.796078in}{2.443131in}}%
\pgfpathlineto{\pgfqpoint{4.796374in}{2.443689in}}%
\pgfpathlineto{\pgfqpoint{4.796966in}{2.445144in}}%
\pgfpathlineto{\pgfqpoint{4.797262in}{2.444166in}}%
\pgfpathlineto{\pgfqpoint{4.799334in}{2.444366in}}%
\pgfpathlineto{\pgfqpoint{4.801998in}{2.444609in}}%
\pgfpathlineto{\pgfqpoint{4.802294in}{2.444797in}}%
\pgfpathlineto{\pgfqpoint{4.803182in}{2.449158in}}%
\pgfpathlineto{\pgfqpoint{4.803478in}{2.448912in}}%
\pgfpathlineto{\pgfqpoint{4.803774in}{2.449467in}}%
\pgfpathlineto{\pgfqpoint{4.804366in}{2.451703in}}%
\pgfpathlineto{\pgfqpoint{4.804958in}{2.452001in}}%
\pgfpathlineto{\pgfqpoint{4.805254in}{2.468002in}}%
\pgfpathlineto{\pgfqpoint{4.805550in}{2.437433in}}%
\pgfpathlineto{\pgfqpoint{4.805846in}{2.435981in}}%
\pgfpathlineto{\pgfqpoint{4.808510in}{2.452254in}}%
\pgfpathlineto{\pgfqpoint{4.809102in}{2.459290in}}%
\pgfpathlineto{\pgfqpoint{4.809398in}{2.454080in}}%
\pgfpathlineto{\pgfqpoint{4.810286in}{2.465941in}}%
\pgfpathlineto{\pgfqpoint{4.810878in}{2.463555in}}%
\pgfpathlineto{\pgfqpoint{4.812950in}{2.464340in}}%
\pgfpathlineto{\pgfqpoint{4.815910in}{2.465466in}}%
\pgfpathlineto{\pgfqpoint{4.816502in}{2.507369in}}%
\pgfpathlineto{\pgfqpoint{4.822718in}{3.034073in}}%
\pgfpathlineto{\pgfqpoint{4.823606in}{2.463247in}}%
\pgfpathlineto{\pgfqpoint{4.823902in}{2.463305in}}%
\pgfpathlineto{\pgfqpoint{4.824198in}{2.465178in}}%
\pgfpathlineto{\pgfqpoint{4.824494in}{2.472327in}}%
\pgfpathlineto{\pgfqpoint{4.825086in}{2.464159in}}%
\pgfpathlineto{\pgfqpoint{4.845215in}{2.454369in}}%
\pgfpathlineto{\pgfqpoint{4.845511in}{2.453847in}}%
\pgfpathlineto{\pgfqpoint{4.845807in}{2.455162in}}%
\pgfpathlineto{\pgfqpoint{4.846399in}{2.465711in}}%
\pgfpathlineto{\pgfqpoint{4.852615in}{2.465532in}}%
\pgfpathlineto{\pgfqpoint{4.853207in}{2.466830in}}%
\pgfpathlineto{\pgfqpoint{4.854391in}{2.466137in}}%
\pgfpathlineto{\pgfqpoint{4.858535in}{2.465425in}}%
\pgfpathlineto{\pgfqpoint{4.858831in}{2.477776in}}%
\pgfpathlineto{\pgfqpoint{4.859127in}{2.986308in}}%
\pgfpathlineto{\pgfqpoint{4.860311in}{2.467767in}}%
\pgfpathlineto{\pgfqpoint{4.860607in}{2.469903in}}%
\pgfpathlineto{\pgfqpoint{4.860903in}{2.469968in}}%
\pgfpathlineto{\pgfqpoint{4.861199in}{2.467536in}}%
\pgfpathlineto{\pgfqpoint{4.862087in}{2.469937in}}%
\pgfpathlineto{\pgfqpoint{4.865343in}{2.469726in}}%
\pgfpathlineto{\pgfqpoint{4.865639in}{2.470027in}}%
\pgfpathlineto{\pgfqpoint{4.866231in}{2.469883in}}%
\pgfpathlineto{\pgfqpoint{4.866823in}{2.471034in}}%
\pgfpathlineto{\pgfqpoint{4.867119in}{2.470794in}}%
\pgfpathlineto{\pgfqpoint{4.868007in}{2.471300in}}%
\pgfpathlineto{\pgfqpoint{4.868303in}{2.472466in}}%
\pgfpathlineto{\pgfqpoint{4.873927in}{2.472466in}}%
\pgfpathlineto{\pgfqpoint{4.874519in}{2.463572in}}%
\pgfpathlineto{\pgfqpoint{4.894351in}{2.471292in}}%
\pgfpathlineto{\pgfqpoint{4.894943in}{2.850108in}}%
\pgfpathlineto{\pgfqpoint{4.895240in}{2.474938in}}%
\pgfpathlineto{\pgfqpoint{4.895536in}{2.474703in}}%
\pgfpathlineto{\pgfqpoint{4.895832in}{2.673852in}}%
\pgfpathlineto{\pgfqpoint{4.896128in}{2.638167in}}%
\pgfpathlineto{\pgfqpoint{4.896424in}{2.535030in}}%
\pgfpathlineto{\pgfqpoint{4.896720in}{2.929310in}}%
\pgfpathlineto{\pgfqpoint{4.897312in}{2.483609in}}%
\pgfpathlineto{\pgfqpoint{4.900864in}{2.467354in}}%
\pgfpathlineto{\pgfqpoint{4.901160in}{2.467902in}}%
\pgfpathlineto{\pgfqpoint{4.901456in}{2.476871in}}%
\pgfpathlineto{\pgfqpoint{4.902048in}{2.465071in}}%
\pgfpathlineto{\pgfqpoint{4.902936in}{2.482934in}}%
\pgfpathlineto{\pgfqpoint{4.903232in}{2.487016in}}%
\pgfpathlineto{\pgfqpoint{4.903528in}{2.486437in}}%
\pgfpathlineto{\pgfqpoint{4.904120in}{2.474176in}}%
\pgfpathlineto{\pgfqpoint{4.908264in}{2.463983in}}%
\pgfpathlineto{\pgfqpoint{4.908560in}{2.474164in}}%
\pgfpathlineto{\pgfqpoint{4.909152in}{2.474560in}}%
\pgfpathlineto{\pgfqpoint{4.910040in}{2.474474in}}%
\pgfpathlineto{\pgfqpoint{4.911816in}{2.474394in}}%
\pgfpathlineto{\pgfqpoint{4.916848in}{2.474633in}}%
\pgfpathlineto{\pgfqpoint{4.918032in}{2.474602in}}%
\pgfpathlineto{\pgfqpoint{4.922768in}{2.473813in}}%
\pgfpathlineto{\pgfqpoint{4.923360in}{2.474608in}}%
\pgfpathlineto{\pgfqpoint{4.923952in}{2.474930in}}%
\pgfpathlineto{\pgfqpoint{4.924248in}{2.474491in}}%
\pgfpathlineto{\pgfqpoint{4.925136in}{2.475338in}}%
\pgfpathlineto{\pgfqpoint{4.925728in}{2.474846in}}%
\pgfpathlineto{\pgfqpoint{4.944968in}{2.474114in}}%
\pgfpathlineto{\pgfqpoint{4.945856in}{2.477626in}}%
\pgfpathlineto{\pgfqpoint{4.946744in}{2.570115in}}%
\pgfpathlineto{\pgfqpoint{4.950888in}{3.051553in}}%
\pgfpathlineto{\pgfqpoint{4.951480in}{2.476909in}}%
\pgfpathlineto{\pgfqpoint{4.952072in}{2.477072in}}%
\pgfpathlineto{\pgfqpoint{4.952368in}{2.488420in}}%
\pgfpathlineto{\pgfqpoint{4.952664in}{2.487839in}}%
\pgfpathlineto{\pgfqpoint{4.953552in}{2.478225in}}%
\pgfpathlineto{\pgfqpoint{4.954440in}{2.592368in}}%
\pgfpathlineto{\pgfqpoint{4.957992in}{3.073252in}}%
\pgfpathlineto{\pgfqpoint{4.958288in}{3.073624in}}%
\pgfpathlineto{\pgfqpoint{4.958584in}{2.897146in}}%
\pgfpathlineto{\pgfqpoint{4.958880in}{2.542609in}}%
\pgfpathlineto{\pgfqpoint{4.959176in}{2.565439in}}%
\pgfpathlineto{\pgfqpoint{4.959768in}{3.066575in}}%
\pgfpathlineto{\pgfqpoint{4.962136in}{2.856433in}}%
\pgfpathlineto{\pgfqpoint{4.965985in}{2.510164in}}%
\pgfpathlineto{\pgfqpoint{4.966281in}{2.659577in}}%
\pgfpathlineto{\pgfqpoint{4.966577in}{3.075925in}}%
\pgfpathlineto{\pgfqpoint{4.966873in}{3.022195in}}%
\pgfpathlineto{\pgfqpoint{4.967465in}{2.626332in}}%
\pgfpathlineto{\pgfqpoint{4.968057in}{3.045367in}}%
\pgfpathlineto{\pgfqpoint{4.974569in}{2.484032in}}%
\pgfpathlineto{\pgfqpoint{4.975161in}{2.483623in}}%
\pgfpathlineto{\pgfqpoint{4.994697in}{2.483389in}}%
\pgfpathlineto{\pgfqpoint{4.995289in}{2.483878in}}%
\pgfpathlineto{\pgfqpoint{4.995585in}{2.485220in}}%
\pgfpathlineto{\pgfqpoint{4.996177in}{2.484335in}}%
\pgfpathlineto{\pgfqpoint{4.996473in}{2.491952in}}%
\pgfpathlineto{\pgfqpoint{4.997065in}{2.525716in}}%
\pgfpathlineto{\pgfqpoint{5.001801in}{2.526376in}}%
\pgfpathlineto{\pgfqpoint{5.002097in}{2.514144in}}%
\pgfpathlineto{\pgfqpoint{5.002393in}{2.515290in}}%
\pgfpathlineto{\pgfqpoint{5.002689in}{2.527586in}}%
\pgfpathlineto{\pgfqpoint{5.003873in}{2.527062in}}%
\pgfpathlineto{\pgfqpoint{5.004465in}{2.527812in}}%
\pgfpathlineto{\pgfqpoint{5.008609in}{2.527551in}}%
\pgfpathlineto{\pgfqpoint{5.010977in}{2.526597in}}%
\pgfpathlineto{\pgfqpoint{5.011273in}{2.527764in}}%
\pgfpathlineto{\pgfqpoint{5.015417in}{2.524592in}}%
\pgfpathlineto{\pgfqpoint{5.016009in}{2.529686in}}%
\pgfpathlineto{\pgfqpoint{5.016601in}{2.533490in}}%
\pgfpathlineto{\pgfqpoint{5.016897in}{2.533490in}}%
\pgfpathlineto{\pgfqpoint{5.017193in}{2.533053in}}%
\pgfpathlineto{\pgfqpoint{5.018081in}{2.537895in}}%
\pgfpathlineto{\pgfqpoint{5.024889in}{2.546960in}}%
\pgfpathlineto{\pgfqpoint{5.043834in}{2.546612in}}%
\pgfpathlineto{\pgfqpoint{5.044722in}{2.544521in}}%
\pgfpathlineto{\pgfqpoint{5.045906in}{2.541795in}}%
\pgfpathlineto{\pgfqpoint{5.046498in}{2.544154in}}%
\pgfpathlineto{\pgfqpoint{5.047090in}{2.502207in}}%
\pgfpathlineto{\pgfqpoint{5.047386in}{2.504949in}}%
\pgfpathlineto{\pgfqpoint{5.050346in}{2.531821in}}%
\pgfpathlineto{\pgfqpoint{5.050642in}{2.500468in}}%
\pgfpathlineto{\pgfqpoint{5.053010in}{2.493536in}}%
\pgfpathlineto{\pgfqpoint{5.060114in}{2.495883in}}%
\pgfpathlineto{\pgfqpoint{5.065442in}{2.496381in}}%
\pgfpathlineto{\pgfqpoint{5.065738in}{2.494885in}}%
\pgfpathlineto{\pgfqpoint{5.066034in}{2.495258in}}%
\pgfpathlineto{\pgfqpoint{5.066330in}{2.494541in}}%
\pgfpathlineto{\pgfqpoint{5.071658in}{2.498903in}}%
\pgfpathlineto{\pgfqpoint{5.071954in}{2.497284in}}%
\pgfpathlineto{\pgfqpoint{5.072546in}{2.498735in}}%
\pgfpathlineto{\pgfqpoint{5.073138in}{2.498871in}}%
\pgfpathlineto{\pgfqpoint{5.074322in}{2.500007in}}%
\pgfpathlineto{\pgfqpoint{5.074914in}{2.500046in}}%
\pgfpathlineto{\pgfqpoint{5.075506in}{2.499733in}}%
\pgfpathlineto{\pgfqpoint{5.100075in}{2.500088in}}%
\pgfpathlineto{\pgfqpoint{5.100963in}{2.499684in}}%
\pgfpathlineto{\pgfqpoint{5.102739in}{2.499405in}}%
\pgfpathlineto{\pgfqpoint{5.103035in}{2.499195in}}%
\pgfpathlineto{\pgfqpoint{5.103331in}{2.499929in}}%
\pgfpathlineto{\pgfqpoint{5.103627in}{2.499235in}}%
\pgfpathlineto{\pgfqpoint{5.104811in}{2.499302in}}%
\pgfpathlineto{\pgfqpoint{5.118723in}{2.498904in}}%
\pgfpathlineto{\pgfqpoint{5.121979in}{2.498378in}}%
\pgfpathlineto{\pgfqpoint{5.122867in}{2.498630in}}%
\pgfpathlineto{\pgfqpoint{5.123163in}{2.498145in}}%
\pgfpathlineto{\pgfqpoint{5.123459in}{2.498732in}}%
\pgfpathlineto{\pgfqpoint{5.125235in}{2.510005in}}%
\pgfpathlineto{\pgfqpoint{5.144475in}{2.606169in}}%
\pgfpathlineto{\pgfqpoint{5.144771in}{2.619227in}}%
\pgfpathlineto{\pgfqpoint{5.145363in}{2.620488in}}%
\pgfpathlineto{\pgfqpoint{5.147435in}{2.620517in}}%
\pgfpathlineto{\pgfqpoint{5.152763in}{2.620475in}}%
\pgfpathlineto{\pgfqpoint{5.153355in}{2.519921in}}%
\pgfpathlineto{\pgfqpoint{5.157795in}{2.620226in}}%
\pgfpathlineto{\pgfqpoint{5.158979in}{2.614832in}}%
\pgfpathlineto{\pgfqpoint{5.159275in}{2.626561in}}%
\pgfpathlineto{\pgfqpoint{5.160163in}{2.622921in}}%
\pgfpathlineto{\pgfqpoint{5.161347in}{2.623368in}}%
\pgfpathlineto{\pgfqpoint{5.167564in}{2.625876in}}%
\pgfpathlineto{\pgfqpoint{5.173780in}{2.625830in}}%
\pgfpathlineto{\pgfqpoint{5.175260in}{2.621427in}}%
\pgfpathlineto{\pgfqpoint{5.194204in}{2.562778in}}%
\pgfpathlineto{\pgfqpoint{5.194500in}{2.626061in}}%
\pgfpathlineto{\pgfqpoint{5.194796in}{2.626436in}}%
\pgfpathlineto{\pgfqpoint{5.195092in}{2.629879in}}%
\pgfpathlineto{\pgfqpoint{5.195684in}{2.644553in}}%
\pgfpathlineto{\pgfqpoint{5.198940in}{2.633184in}}%
\pgfpathlineto{\pgfqpoint{5.199236in}{2.632160in}}%
\pgfpathlineto{\pgfqpoint{5.199532in}{2.640760in}}%
\pgfpathlineto{\pgfqpoint{5.200420in}{2.954256in}}%
\pgfpathlineto{\pgfqpoint{5.201308in}{2.852334in}}%
\pgfpathlineto{\pgfqpoint{5.201604in}{2.980239in}}%
\pgfpathlineto{\pgfqpoint{5.201900in}{2.984873in}}%
\pgfpathlineto{\pgfqpoint{5.207820in}{2.985448in}}%
\pgfpathlineto{\pgfqpoint{5.209892in}{2.986048in}}%
\pgfpathlineto{\pgfqpoint{5.227060in}{2.986800in}}%
\pgfpathlineto{\pgfqpoint{5.257845in}{2.988100in}}%
\pgfpathlineto{\pgfqpoint{5.258733in}{2.988406in}}%
\pgfpathlineto{\pgfqpoint{5.264949in}{2.987661in}}%
\pgfpathlineto{\pgfqpoint{5.266725in}{2.984817in}}%
\pgfpathlineto{\pgfqpoint{5.270573in}{2.985179in}}%
\pgfpathlineto{\pgfqpoint{5.300470in}{2.988409in}}%
\pgfpathlineto{\pgfqpoint{5.307870in}{2.990098in}}%
\pgfpathlineto{\pgfqpoint{5.308166in}{2.989962in}}%
\pgfpathlineto{\pgfqpoint{5.308758in}{2.991642in}}%
\pgfpathlineto{\pgfqpoint{5.313790in}{2.993613in}}%
\pgfpathlineto{\pgfqpoint{5.315566in}{2.992022in}}%
\pgfpathlineto{\pgfqpoint{5.321782in}{2.998889in}}%
\pgfpathlineto{\pgfqpoint{5.322078in}{2.994736in}}%
\pgfpathlineto{\pgfqpoint{5.322966in}{3.000299in}}%
\pgfpathlineto{\pgfqpoint{5.323262in}{3.000931in}}%
\pgfpathlineto{\pgfqpoint{5.343390in}{2.985419in}}%
\pgfpathlineto{\pgfqpoint{5.343982in}{2.999539in}}%
\pgfpathlineto{\pgfqpoint{5.344870in}{3.000180in}}%
\pgfpathlineto{\pgfqpoint{5.349902in}{3.000837in}}%
\pgfpathlineto{\pgfqpoint{5.350198in}{3.001053in}}%
\pgfpathlineto{\pgfqpoint{5.350494in}{3.001617in}}%
\pgfpathlineto{\pgfqpoint{5.357894in}{3.000772in}}%
\pgfpathlineto{\pgfqpoint{5.358190in}{3.001039in}}%
\pgfpathlineto{\pgfqpoint{5.358782in}{3.002042in}}%
\pgfpathlineto{\pgfqpoint{5.364998in}{3.002080in}}%
\pgfpathlineto{\pgfqpoint{5.365294in}{3.005291in}}%
\pgfpathlineto{\pgfqpoint{5.365886in}{3.013513in}}%
\pgfpathlineto{\pgfqpoint{5.391935in}{3.008732in}}%
\pgfpathlineto{\pgfqpoint{5.392527in}{3.019977in}}%
\pgfpathlineto{\pgfqpoint{5.394303in}{3.019480in}}%
\pgfpathlineto{\pgfqpoint{5.394895in}{3.024885in}}%
\pgfpathlineto{\pgfqpoint{5.400519in}{3.024512in}}%
\pgfpathlineto{\pgfqpoint{5.401407in}{3.024654in}}%
\pgfpathlineto{\pgfqpoint{5.406143in}{3.023251in}}%
\pgfpathlineto{\pgfqpoint{5.407327in}{3.024627in}}%
\pgfpathlineto{\pgfqpoint{5.407919in}{3.025998in}}%
\pgfpathlineto{\pgfqpoint{5.409695in}{3.026152in}}%
\pgfpathlineto{\pgfqpoint{5.419463in}{3.026933in}}%
\pgfpathlineto{\pgfqpoint{5.420351in}{3.027126in}}%
\pgfpathlineto{\pgfqpoint{5.422127in}{3.036093in}}%
\pgfpathlineto{\pgfqpoint{5.422719in}{3.037705in}}%
\pgfpathlineto{\pgfqpoint{5.442256in}{3.039054in}}%
\pgfpathlineto{\pgfqpoint{5.442552in}{3.037425in}}%
\pgfpathlineto{\pgfqpoint{5.443144in}{3.028821in}}%
\pgfpathlineto{\pgfqpoint{5.443440in}{3.031794in}}%
\pgfpathlineto{\pgfqpoint{5.443736in}{3.028526in}}%
\pgfpathlineto{\pgfqpoint{5.444920in}{3.038203in}}%
\pgfpathlineto{\pgfqpoint{5.448472in}{3.027663in}}%
\pgfpathlineto{\pgfqpoint{5.449360in}{3.039914in}}%
\pgfpathlineto{\pgfqpoint{5.449952in}{3.039963in}}%
\pgfpathlineto{\pgfqpoint{5.450248in}{3.039011in}}%
\pgfpathlineto{\pgfqpoint{5.450544in}{3.032387in}}%
\pgfpathlineto{\pgfqpoint{5.450840in}{3.040565in}}%
\pgfpathlineto{\pgfqpoint{5.451728in}{3.040845in}}%
\pgfpathlineto{\pgfqpoint{5.452320in}{3.040989in}}%
\pgfpathlineto{\pgfqpoint{5.457352in}{3.040944in}}%
\pgfpathlineto{\pgfqpoint{5.457648in}{3.041802in}}%
\pgfpathlineto{\pgfqpoint{5.463568in}{3.040593in}}%
\pgfpathlineto{\pgfqpoint{5.464456in}{3.042101in}}%
\pgfpathlineto{\pgfqpoint{5.465048in}{3.043101in}}%
\pgfpathlineto{\pgfqpoint{5.468896in}{3.043188in}}%
\pgfpathlineto{\pgfqpoint{5.470376in}{3.043326in}}%
\pgfpathlineto{\pgfqpoint{5.472152in}{3.045854in}}%
\pgfpathlineto{\pgfqpoint{5.497904in}{3.053470in}}%
\pgfpathlineto{\pgfqpoint{5.499088in}{3.053591in}}%
\pgfpathlineto{\pgfqpoint{5.500864in}{3.052175in}}%
\pgfpathlineto{\pgfqpoint{5.513001in}{3.054600in}}%
\pgfpathlineto{\pgfqpoint{5.515369in}{3.052831in}}%
\pgfpathlineto{\pgfqpoint{5.520993in}{3.052692in}}%
\pgfpathlineto{\pgfqpoint{5.521289in}{3.052804in}}%
\pgfpathlineto{\pgfqpoint{5.521585in}{3.051682in}}%
\pgfpathlineto{\pgfqpoint{5.522177in}{3.046894in}}%
\pgfpathlineto{\pgfqpoint{5.522473in}{3.060187in}}%
\pgfpathlineto{\pgfqpoint{5.541121in}{3.070310in}}%
\pgfpathlineto{\pgfqpoint{5.542009in}{3.062655in}}%
\pgfpathlineto{\pgfqpoint{5.543193in}{3.052914in}}%
\pgfpathlineto{\pgfqpoint{5.548817in}{3.067652in}}%
\pgfpathlineto{\pgfqpoint{5.549409in}{3.068727in}}%
\pgfpathlineto{\pgfqpoint{5.549705in}{3.066712in}}%
\pgfpathlineto{\pgfqpoint{5.550001in}{3.061005in}}%
\pgfpathlineto{\pgfqpoint{5.550297in}{3.087526in}}%
\pgfpathlineto{\pgfqpoint{5.550889in}{3.087960in}}%
\pgfpathlineto{\pgfqpoint{5.551185in}{3.099534in}}%
\pgfpathlineto{\pgfqpoint{5.555033in}{3.074627in}}%
\pgfpathlineto{\pgfqpoint{5.555625in}{3.101599in}}%
\pgfpathlineto{\pgfqpoint{5.557697in}{3.102608in}}%
\pgfpathlineto{\pgfqpoint{5.570130in}{3.102817in}}%
\pgfpathlineto{\pgfqpoint{5.571314in}{3.102562in}}%
\pgfpathlineto{\pgfqpoint{5.571906in}{3.102958in}}%
\pgfpathlineto{\pgfqpoint{5.590554in}{3.102992in}}%
\pgfpathlineto{\pgfqpoint{5.590850in}{3.103500in}}%
\pgfpathlineto{\pgfqpoint{5.593514in}{3.115696in}}%
\pgfpathlineto{\pgfqpoint{5.597066in}{3.112996in}}%
\pgfpathlineto{\pgfqpoint{5.600618in}{3.110161in}}%
\pgfpathlineto{\pgfqpoint{5.605354in}{3.102708in}}%
\pgfpathlineto{\pgfqpoint{5.606242in}{3.102359in}}%
\pgfpathlineto{\pgfqpoint{5.607426in}{3.108886in}}%
\pgfpathlineto{\pgfqpoint{5.643243in}{3.111095in}}%
\pgfpathlineto{\pgfqpoint{5.645907in}{3.116190in}}%
\pgfpathlineto{\pgfqpoint{5.649163in}{3.122312in}}%
\pgfpathlineto{\pgfqpoint{5.649459in}{3.114881in}}%
\pgfpathlineto{\pgfqpoint{5.650051in}{3.121983in}}%
\pgfpathlineto{\pgfqpoint{5.650643in}{3.122562in}}%
\pgfpathlineto{\pgfqpoint{5.655083in}{3.130169in}}%
\pgfpathlineto{\pgfqpoint{5.655379in}{3.129827in}}%
\pgfpathlineto{\pgfqpoint{5.656859in}{3.122210in}}%
\pgfpathlineto{\pgfqpoint{5.664259in}{3.126533in}}%
\pgfpathlineto{\pgfqpoint{5.668995in}{3.126775in}}%
\pgfpathlineto{\pgfqpoint{5.669883in}{3.126933in}}%
\pgfpathlineto{\pgfqpoint{5.690307in}{3.126642in}}%
\pgfpathlineto{\pgfqpoint{5.690603in}{3.126217in}}%
\pgfpathlineto{\pgfqpoint{5.691195in}{3.038631in}}%
\pgfpathlineto{\pgfqpoint{5.692675in}{2.783472in}}%
\pgfpathlineto{\pgfqpoint{5.692971in}{2.782086in}}%
\pgfpathlineto{\pgfqpoint{5.698003in}{2.785467in}}%
\pgfpathlineto{\pgfqpoint{5.698299in}{2.786681in}}%
\pgfpathlineto{\pgfqpoint{5.698891in}{2.952689in}}%
\pgfpathlineto{\pgfqpoint{5.699187in}{3.054385in}}%
\pgfpathlineto{\pgfqpoint{5.699779in}{2.786819in}}%
\pgfpathlineto{\pgfqpoint{5.717836in}{2.787275in}}%
\pgfpathlineto{\pgfqpoint{5.718132in}{2.788106in}}%
\pgfpathlineto{\pgfqpoint{5.718724in}{2.790888in}}%
\pgfpathlineto{\pgfqpoint{5.758092in}{2.790914in}}%
\pgfpathlineto{\pgfqpoint{5.791837in}{2.788996in}}%
\pgfpathlineto{\pgfqpoint{5.792133in}{2.789511in}}%
\pgfpathlineto{\pgfqpoint{5.799237in}{2.816401in}}%
\pgfpathlineto{\pgfqpoint{5.807821in}{2.817784in}}%
\pgfpathlineto{\pgfqpoint{5.814629in}{2.819771in}}%
\pgfpathlineto{\pgfqpoint{5.822917in}{2.820468in}}%
\pgfpathlineto{\pgfqpoint{5.838605in}{2.820468in}}%
\pgfpathlineto{\pgfqpoint{5.838902in}{2.822500in}}%
\pgfpathlineto{\pgfqpoint{5.839494in}{2.819015in}}%
\pgfpathlineto{\pgfqpoint{5.842454in}{2.820074in}}%
\pgfpathlineto{\pgfqpoint{5.848670in}{2.819855in}}%
\pgfpathlineto{\pgfqpoint{5.849558in}{2.819629in}}%
\pgfpathlineto{\pgfqpoint{5.849854in}{2.820181in}}%
\pgfpathlineto{\pgfqpoint{5.860806in}{2.818974in}}%
\pgfpathlineto{\pgfqpoint{5.862878in}{2.819616in}}%
\pgfpathlineto{\pgfqpoint{5.867318in}{2.818965in}}%
\pgfpathlineto{\pgfqpoint{5.868502in}{2.819339in}}%
\pgfpathlineto{\pgfqpoint{5.869982in}{2.818057in}}%
\pgfpathlineto{\pgfqpoint{5.890406in}{2.818081in}}%
\pgfpathlineto{\pgfqpoint{5.890702in}{2.817778in}}%
\pgfpathlineto{\pgfqpoint{5.891294in}{2.818452in}}%
\pgfpathlineto{\pgfqpoint{5.896622in}{2.817094in}}%
\pgfpathlineto{\pgfqpoint{5.896918in}{2.818272in}}%
\pgfpathlineto{\pgfqpoint{5.897510in}{2.817843in}}%
\pgfpathlineto{\pgfqpoint{5.898398in}{2.817175in}}%
\pgfpathlineto{\pgfqpoint{5.908167in}{2.818314in}}%
\pgfpathlineto{\pgfqpoint{5.910239in}{2.818266in}}%
\pgfpathlineto{\pgfqpoint{5.910831in}{2.818590in}}%
\pgfpathlineto{\pgfqpoint{5.912903in}{2.818128in}}%
\pgfpathlineto{\pgfqpoint{5.945167in}{2.818398in}}%
\pgfpathlineto{\pgfqpoint{5.961151in}{2.818739in}}%
\pgfpathlineto{\pgfqpoint{5.988680in}{2.818247in}}%
\pgfpathlineto{\pgfqpoint{5.989272in}{2.817497in}}%
\pgfpathlineto{\pgfqpoint{5.998744in}{2.816377in}}%
\pgfpathlineto{\pgfqpoint{6.003184in}{2.817072in}}%
\pgfpathlineto{\pgfqpoint{6.004368in}{2.819614in}}%
\pgfpathlineto{\pgfqpoint{6.004368in}{2.819614in}}%
\pgfusepath{stroke}%
\end{pgfscope}%
\begin{pgfscope}%
\pgfsetrectcap%
\pgfsetmiterjoin%
\pgfsetlinewidth{0.501875pt}%
\definecolor{currentstroke}{rgb}{0.000000,0.000000,0.000000}%
\pgfsetstrokecolor{currentstroke}%
\pgfsetdash{}{0pt}%
\pgfpathmoveto{\pgfqpoint{0.481681in}{1.080890in}}%
\pgfpathlineto{\pgfqpoint{0.481681in}{3.227753in}}%
\pgfusepath{stroke}%
\end{pgfscope}%
\begin{pgfscope}%
\pgfsetrectcap%
\pgfsetmiterjoin%
\pgfsetlinewidth{0.501875pt}%
\definecolor{currentstroke}{rgb}{0.000000,0.000000,0.000000}%
\pgfsetstrokecolor{currentstroke}%
\pgfsetdash{}{0pt}%
\pgfpathmoveto{\pgfqpoint{6.267353in}{1.080890in}}%
\pgfpathlineto{\pgfqpoint{6.267353in}{3.227753in}}%
\pgfusepath{stroke}%
\end{pgfscope}%
\begin{pgfscope}%
\pgfsetrectcap%
\pgfsetmiterjoin%
\pgfsetlinewidth{0.501875pt}%
\definecolor{currentstroke}{rgb}{0.000000,0.000000,0.000000}%
\pgfsetstrokecolor{currentstroke}%
\pgfsetdash{}{0pt}%
\pgfpathmoveto{\pgfqpoint{0.481681in}{1.080890in}}%
\pgfpathlineto{\pgfqpoint{6.267353in}{1.080890in}}%
\pgfusepath{stroke}%
\end{pgfscope}%
\begin{pgfscope}%
\pgfsetrectcap%
\pgfsetmiterjoin%
\pgfsetlinewidth{0.501875pt}%
\definecolor{currentstroke}{rgb}{0.000000,0.000000,0.000000}%
\pgfsetstrokecolor{currentstroke}%
\pgfsetdash{}{0pt}%
\pgfpathmoveto{\pgfqpoint{0.481681in}{3.227753in}}%
\pgfpathlineto{\pgfqpoint{6.267353in}{3.227753in}}%
\pgfusepath{stroke}%
\end{pgfscope}%
\begin{pgfscope}%
\pgfsetrectcap%
\pgfsetroundjoin%
\pgfsetlinewidth{0.401500pt}%
\definecolor{currentstroke}{rgb}{0.000000,0.070588,0.098039}%
\pgfsetstrokecolor{currentstroke}%
\pgfsetdash{}{0pt}%
\pgfpathmoveto{\pgfqpoint{0.569181in}{3.105934in}}%
\pgfpathlineto{\pgfqpoint{0.666403in}{3.105934in}}%
\pgfpathlineto{\pgfqpoint{0.763625in}{3.105934in}}%
\pgfusepath{stroke}%
\end{pgfscope}%
\begin{pgfscope}%
\definecolor{textcolor}{rgb}{0.000000,0.000000,0.000000}%
\pgfsetstrokecolor{textcolor}%
\pgfsetfillcolor{textcolor}%
\pgftext[x=0.841403in,y=3.071906in,left,base]{\color{textcolor}\rmfamily\fontsize{7.000000}{8.400000}\selectfont Story Points}%
\end{pgfscope}%
\begin{pgfscope}%
\pgfsetrectcap%
\pgfsetroundjoin%
\pgfsetlinewidth{0.200750pt}%
\definecolor{currentstroke}{rgb}{0.682353,0.125490,0.070588}%
\pgfsetstrokecolor{currentstroke}%
\pgfsetdash{}{0pt}%
\pgfpathmoveto{\pgfqpoint{0.569181in}{2.969142in}}%
\pgfpathlineto{\pgfqpoint{0.666403in}{2.969142in}}%
\pgfpathlineto{\pgfqpoint{0.763625in}{2.969142in}}%
\pgfusepath{stroke}%
\end{pgfscope}%
\begin{pgfscope}%
\definecolor{textcolor}{rgb}{0.000000,0.000000,0.000000}%
\pgfsetstrokecolor{textcolor}%
\pgfsetfillcolor{textcolor}%
\pgftext[x=0.841403in,y=2.935114in,left,base]{\color{textcolor}\rmfamily\fontsize{7.000000}{8.400000}\selectfont Logische Codezeilen}%
\end{pgfscope}%
\begin{pgfscope}%
\pgfsetrectcap%
\pgfsetroundjoin%
\pgfsetlinewidth{0.200750pt}%
\definecolor{currentstroke}{rgb}{0.000000,0.372549,0.450980}%
\pgfsetstrokecolor{currentstroke}%
\pgfsetdash{}{0pt}%
\pgfpathmoveto{\pgfqpoint{0.569181in}{2.832545in}}%
\pgfpathlineto{\pgfqpoint{0.666403in}{2.832545in}}%
\pgfpathlineto{\pgfqpoint{0.763625in}{2.832545in}}%
\pgfusepath{stroke}%
\end{pgfscope}%
\begin{pgfscope}%
\definecolor{textcolor}{rgb}{0.000000,0.000000,0.000000}%
\pgfsetstrokecolor{textcolor}%
\pgfsetfillcolor{textcolor}%
\pgftext[x=0.841403in,y=2.798517in,left,base]{\color{textcolor}\rmfamily\fontsize{7.000000}{8.400000}\selectfont Zyklomatische Komplexität}%
\end{pgfscope}%
\begin{pgfscope}%
\pgfsetrectcap%
\pgfsetroundjoin%
\pgfsetlinewidth{0.200750pt}%
\definecolor{currentstroke}{rgb}{0.580392,0.823529,0.741176}%
\pgfsetstrokecolor{currentstroke}%
\pgfsetdash{}{0pt}%
\pgfpathmoveto{\pgfqpoint{0.569181in}{2.696045in}}%
\pgfpathlineto{\pgfqpoint{0.666403in}{2.696045in}}%
\pgfpathlineto{\pgfqpoint{0.763625in}{2.696045in}}%
\pgfusepath{stroke}%
\end{pgfscope}%
\begin{pgfscope}%
\definecolor{textcolor}{rgb}{0.000000,0.000000,0.000000}%
\pgfsetstrokecolor{textcolor}%
\pgfsetfillcolor{textcolor}%
\pgftext[x=0.841403in,y=2.662017in,left,base]{\color{textcolor}\rmfamily\fontsize{7.000000}{8.400000}\selectfont Halstead Aufwand}%
\end{pgfscope}%
\begin{pgfscope}%
\pgfsetrectcap%
\pgfsetroundjoin%
\pgfsetlinewidth{0.200750pt}%
\definecolor{currentstroke}{rgb}{0.933333,0.607843,0.000000}%
\pgfsetstrokecolor{currentstroke}%
\pgfsetdash{}{0pt}%
\pgfpathmoveto{\pgfqpoint{0.569181in}{2.560517in}}%
\pgfpathlineto{\pgfqpoint{0.666403in}{2.560517in}}%
\pgfpathlineto{\pgfqpoint{0.763625in}{2.560517in}}%
\pgfusepath{stroke}%
\end{pgfscope}%
\begin{pgfscope}%
\definecolor{textcolor}{rgb}{0.000000,0.000000,0.000000}%
\pgfsetstrokecolor{textcolor}%
\pgfsetfillcolor{textcolor}%
\pgftext[x=0.841403in,y=2.526489in,left,base]{\color{textcolor}\rmfamily\fontsize{7.000000}{8.400000}\selectfont Einrückungskomplexität}%
\end{pgfscope}%
\begin{pgfscope}%
\pgfsetbuttcap%
\pgfsetmiterjoin%
\definecolor{currentfill}{rgb}{1.000000,1.000000,1.000000}%
\pgfsetfillcolor{currentfill}%
\pgfsetlinewidth{0.000000pt}%
\definecolor{currentstroke}{rgb}{0.000000,0.000000,0.000000}%
\pgfsetstrokecolor{currentstroke}%
\pgfsetstrokeopacity{0.000000}%
\pgfsetdash{}{0pt}%
\pgfpathmoveto{\pgfqpoint{0.481681in}{0.586309in}}%
\pgfpathlineto{\pgfqpoint{6.267353in}{0.586309in}}%
\pgfpathlineto{\pgfqpoint{6.267353in}{0.893003in}}%
\pgfpathlineto{\pgfqpoint{0.481681in}{0.893003in}}%
\pgfpathlineto{\pgfqpoint{0.481681in}{0.586309in}}%
\pgfpathclose%
\pgfusepath{fill}%
\end{pgfscope}%
\begin{pgfscope}%
\pgfpathrectangle{\pgfqpoint{0.481681in}{0.586309in}}{\pgfqpoint{5.785672in}{0.306695in}}%
\pgfusepath{clip}%
\pgfsetbuttcap%
\pgfsetroundjoin%
\definecolor{currentfill}{rgb}{0.800000,0.788235,0.760784}%
\pgfsetfillcolor{currentfill}%
\pgfsetlinewidth{0.000000pt}%
\definecolor{currentstroke}{rgb}{0.000000,0.000000,0.000000}%
\pgfsetstrokecolor{currentstroke}%
\pgfsetdash{}{0pt}%
\pgfpathmoveto{\pgfqpoint{0.744666in}{0.739656in}}%
\pgfpathlineto{\pgfqpoint{0.744666in}{0.739656in}}%
\pgfpathlineto{\pgfqpoint{0.744962in}{0.744840in}}%
\pgfpathlineto{\pgfqpoint{0.745258in}{0.744839in}}%
\pgfpathlineto{\pgfqpoint{0.745554in}{0.744839in}}%
\pgfpathlineto{\pgfqpoint{0.745850in}{0.746142in}}%
\pgfpathlineto{\pgfqpoint{0.746146in}{0.746339in}}%
\pgfpathlineto{\pgfqpoint{0.746442in}{0.758852in}}%
\pgfpathlineto{\pgfqpoint{0.746738in}{0.763146in}}%
\pgfpathlineto{\pgfqpoint{0.747034in}{0.763270in}}%
\pgfpathlineto{\pgfqpoint{0.747330in}{0.763289in}}%
\pgfpathlineto{\pgfqpoint{0.747626in}{0.763288in}}%
\pgfpathlineto{\pgfqpoint{0.747922in}{0.763288in}}%
\pgfpathlineto{\pgfqpoint{0.748218in}{0.763287in}}%
\pgfpathlineto{\pgfqpoint{0.748514in}{0.763286in}}%
\pgfpathlineto{\pgfqpoint{0.748810in}{0.763286in}}%
\pgfpathlineto{\pgfqpoint{0.749106in}{0.763285in}}%
\pgfpathlineto{\pgfqpoint{0.749402in}{0.763284in}}%
\pgfpathlineto{\pgfqpoint{0.749698in}{0.763283in}}%
\pgfpathlineto{\pgfqpoint{0.749994in}{0.763283in}}%
\pgfpathlineto{\pgfqpoint{0.750290in}{0.763282in}}%
\pgfpathlineto{\pgfqpoint{0.750586in}{0.763281in}}%
\pgfpathlineto{\pgfqpoint{0.750882in}{0.763281in}}%
\pgfpathlineto{\pgfqpoint{0.751178in}{0.763280in}}%
\pgfpathlineto{\pgfqpoint{0.751474in}{0.763279in}}%
\pgfpathlineto{\pgfqpoint{0.751770in}{0.763279in}}%
\pgfpathlineto{\pgfqpoint{0.752066in}{0.763278in}}%
\pgfpathlineto{\pgfqpoint{0.752362in}{0.763277in}}%
\pgfpathlineto{\pgfqpoint{0.752658in}{0.763277in}}%
\pgfpathlineto{\pgfqpoint{0.752954in}{0.763276in}}%
\pgfpathlineto{\pgfqpoint{0.753250in}{0.763275in}}%
\pgfpathlineto{\pgfqpoint{0.753546in}{0.763275in}}%
\pgfpathlineto{\pgfqpoint{0.753842in}{0.763274in}}%
\pgfpathlineto{\pgfqpoint{0.754138in}{0.763273in}}%
\pgfpathlineto{\pgfqpoint{0.754434in}{0.763272in}}%
\pgfpathlineto{\pgfqpoint{0.754730in}{0.763272in}}%
\pgfpathlineto{\pgfqpoint{0.755026in}{0.763271in}}%
\pgfpathlineto{\pgfqpoint{0.755322in}{0.763270in}}%
\pgfpathlineto{\pgfqpoint{0.755618in}{0.763275in}}%
\pgfpathlineto{\pgfqpoint{0.755914in}{0.763277in}}%
\pgfpathlineto{\pgfqpoint{0.756210in}{0.763276in}}%
\pgfpathlineto{\pgfqpoint{0.756506in}{0.763274in}}%
\pgfpathlineto{\pgfqpoint{0.756802in}{0.763272in}}%
\pgfpathlineto{\pgfqpoint{0.757098in}{0.763271in}}%
\pgfpathlineto{\pgfqpoint{0.757394in}{0.763269in}}%
\pgfpathlineto{\pgfqpoint{0.757690in}{0.763268in}}%
\pgfpathlineto{\pgfqpoint{0.757986in}{0.763266in}}%
\pgfpathlineto{\pgfqpoint{0.758282in}{0.763264in}}%
\pgfpathlineto{\pgfqpoint{0.758578in}{0.763263in}}%
\pgfpathlineto{\pgfqpoint{0.758874in}{0.763261in}}%
\pgfpathlineto{\pgfqpoint{0.759170in}{0.763260in}}%
\pgfpathlineto{\pgfqpoint{0.759466in}{0.763258in}}%
\pgfpathlineto{\pgfqpoint{0.759762in}{0.763257in}}%
\pgfpathlineto{\pgfqpoint{0.760058in}{0.763255in}}%
\pgfpathlineto{\pgfqpoint{0.760354in}{0.763253in}}%
\pgfpathlineto{\pgfqpoint{0.760650in}{0.763252in}}%
\pgfpathlineto{\pgfqpoint{0.760946in}{0.763250in}}%
\pgfpathlineto{\pgfqpoint{0.761242in}{0.763249in}}%
\pgfpathlineto{\pgfqpoint{0.761538in}{0.763247in}}%
\pgfpathlineto{\pgfqpoint{0.761834in}{0.763245in}}%
\pgfpathlineto{\pgfqpoint{0.762130in}{0.763244in}}%
\pgfpathlineto{\pgfqpoint{0.762426in}{0.763242in}}%
\pgfpathlineto{\pgfqpoint{0.762722in}{0.763241in}}%
\pgfpathlineto{\pgfqpoint{0.763018in}{0.763239in}}%
\pgfpathlineto{\pgfqpoint{0.763314in}{0.763237in}}%
\pgfpathlineto{\pgfqpoint{0.763610in}{0.763236in}}%
\pgfpathlineto{\pgfqpoint{0.763906in}{0.763234in}}%
\pgfpathlineto{\pgfqpoint{0.764202in}{0.763233in}}%
\pgfpathlineto{\pgfqpoint{0.764498in}{0.763231in}}%
\pgfpathlineto{\pgfqpoint{0.764794in}{0.763229in}}%
\pgfpathlineto{\pgfqpoint{0.765090in}{0.763228in}}%
\pgfpathlineto{\pgfqpoint{0.765386in}{0.763226in}}%
\pgfpathlineto{\pgfqpoint{0.765682in}{0.763225in}}%
\pgfpathlineto{\pgfqpoint{0.765978in}{0.763223in}}%
\pgfpathlineto{\pgfqpoint{0.766274in}{0.763222in}}%
\pgfpathlineto{\pgfqpoint{0.766570in}{0.763220in}}%
\pgfpathlineto{\pgfqpoint{0.766866in}{0.763218in}}%
\pgfpathlineto{\pgfqpoint{0.767162in}{0.763217in}}%
\pgfpathlineto{\pgfqpoint{0.767458in}{0.763237in}}%
\pgfpathlineto{\pgfqpoint{0.767754in}{0.763328in}}%
\pgfpathlineto{\pgfqpoint{0.768050in}{0.763423in}}%
\pgfpathlineto{\pgfqpoint{0.768346in}{0.763512in}}%
\pgfpathlineto{\pgfqpoint{0.768642in}{0.763543in}}%
\pgfpathlineto{\pgfqpoint{0.768938in}{0.763469in}}%
\pgfpathlineto{\pgfqpoint{0.769234in}{0.763712in}}%
\pgfpathlineto{\pgfqpoint{0.769530in}{0.763709in}}%
\pgfpathlineto{\pgfqpoint{0.769826in}{0.763706in}}%
\pgfpathlineto{\pgfqpoint{0.770122in}{0.763704in}}%
\pgfpathlineto{\pgfqpoint{0.770418in}{0.763701in}}%
\pgfpathlineto{\pgfqpoint{0.770714in}{0.763698in}}%
\pgfpathlineto{\pgfqpoint{0.771010in}{0.763695in}}%
\pgfpathlineto{\pgfqpoint{0.771306in}{0.763692in}}%
\pgfpathlineto{\pgfqpoint{0.771602in}{0.763689in}}%
\pgfpathlineto{\pgfqpoint{0.771898in}{0.763686in}}%
\pgfpathlineto{\pgfqpoint{0.772194in}{0.763683in}}%
\pgfpathlineto{\pgfqpoint{0.772490in}{0.763680in}}%
\pgfpathlineto{\pgfqpoint{0.772786in}{0.763677in}}%
\pgfpathlineto{\pgfqpoint{0.773082in}{0.763675in}}%
\pgfpathlineto{\pgfqpoint{0.773378in}{0.763718in}}%
\pgfpathlineto{\pgfqpoint{0.773674in}{0.763790in}}%
\pgfpathlineto{\pgfqpoint{0.773970in}{0.764008in}}%
\pgfpathlineto{\pgfqpoint{0.774266in}{0.764023in}}%
\pgfpathlineto{\pgfqpoint{0.774562in}{0.764058in}}%
\pgfpathlineto{\pgfqpoint{0.774858in}{0.764049in}}%
\pgfpathlineto{\pgfqpoint{0.775154in}{0.764040in}}%
\pgfpathlineto{\pgfqpoint{0.775450in}{0.764030in}}%
\pgfpathlineto{\pgfqpoint{0.775746in}{0.764020in}}%
\pgfpathlineto{\pgfqpoint{0.776042in}{0.764008in}}%
\pgfpathlineto{\pgfqpoint{0.776338in}{0.763994in}}%
\pgfpathlineto{\pgfqpoint{0.776634in}{0.763980in}}%
\pgfpathlineto{\pgfqpoint{0.776930in}{0.763966in}}%
\pgfpathlineto{\pgfqpoint{0.777226in}{0.763953in}}%
\pgfpathlineto{\pgfqpoint{0.777522in}{0.763939in}}%
\pgfpathlineto{\pgfqpoint{0.777818in}{0.763925in}}%
\pgfpathlineto{\pgfqpoint{0.778114in}{0.763911in}}%
\pgfpathlineto{\pgfqpoint{0.778410in}{0.763897in}}%
\pgfpathlineto{\pgfqpoint{0.778706in}{0.763883in}}%
\pgfpathlineto{\pgfqpoint{0.779002in}{0.763869in}}%
\pgfpathlineto{\pgfqpoint{0.779298in}{0.763856in}}%
\pgfpathlineto{\pgfqpoint{0.779594in}{0.763842in}}%
\pgfpathlineto{\pgfqpoint{0.779890in}{0.763828in}}%
\pgfpathlineto{\pgfqpoint{0.780186in}{0.763814in}}%
\pgfpathlineto{\pgfqpoint{0.780482in}{0.763800in}}%
\pgfpathlineto{\pgfqpoint{0.780778in}{0.763786in}}%
\pgfpathlineto{\pgfqpoint{0.781074in}{0.763772in}}%
\pgfpathlineto{\pgfqpoint{0.781370in}{0.763273in}}%
\pgfpathlineto{\pgfqpoint{0.781666in}{0.763259in}}%
\pgfpathlineto{\pgfqpoint{0.781962in}{0.763258in}}%
\pgfpathlineto{\pgfqpoint{0.782258in}{0.763257in}}%
\pgfpathlineto{\pgfqpoint{0.782554in}{0.763256in}}%
\pgfpathlineto{\pgfqpoint{0.782851in}{0.763255in}}%
\pgfpathlineto{\pgfqpoint{0.783147in}{0.763254in}}%
\pgfpathlineto{\pgfqpoint{0.783443in}{0.763253in}}%
\pgfpathlineto{\pgfqpoint{0.783739in}{0.763254in}}%
\pgfpathlineto{\pgfqpoint{0.784035in}{0.763257in}}%
\pgfpathlineto{\pgfqpoint{0.784331in}{0.763260in}}%
\pgfpathlineto{\pgfqpoint{0.784627in}{0.763263in}}%
\pgfpathlineto{\pgfqpoint{0.784923in}{0.763265in}}%
\pgfpathlineto{\pgfqpoint{0.785219in}{0.763268in}}%
\pgfpathlineto{\pgfqpoint{0.785515in}{0.763271in}}%
\pgfpathlineto{\pgfqpoint{0.785811in}{0.763274in}}%
\pgfpathlineto{\pgfqpoint{0.786107in}{0.763277in}}%
\pgfpathlineto{\pgfqpoint{0.786403in}{0.763279in}}%
\pgfpathlineto{\pgfqpoint{0.786699in}{0.763282in}}%
\pgfpathlineto{\pgfqpoint{0.786995in}{0.763285in}}%
\pgfpathlineto{\pgfqpoint{0.787291in}{0.763288in}}%
\pgfpathlineto{\pgfqpoint{0.787587in}{0.763291in}}%
\pgfpathlineto{\pgfqpoint{0.787883in}{0.763293in}}%
\pgfpathlineto{\pgfqpoint{0.788179in}{0.763296in}}%
\pgfpathlineto{\pgfqpoint{0.788475in}{0.763290in}}%
\pgfpathlineto{\pgfqpoint{0.788771in}{0.763243in}}%
\pgfpathlineto{\pgfqpoint{0.789067in}{0.763235in}}%
\pgfpathlineto{\pgfqpoint{0.789363in}{0.763327in}}%
\pgfpathlineto{\pgfqpoint{0.789659in}{0.763499in}}%
\pgfpathlineto{\pgfqpoint{0.789955in}{0.763539in}}%
\pgfpathlineto{\pgfqpoint{0.790251in}{0.763557in}}%
\pgfpathlineto{\pgfqpoint{0.790547in}{0.763565in}}%
\pgfpathlineto{\pgfqpoint{0.790843in}{0.763569in}}%
\pgfpathlineto{\pgfqpoint{0.791139in}{0.763573in}}%
\pgfpathlineto{\pgfqpoint{0.791435in}{0.763577in}}%
\pgfpathlineto{\pgfqpoint{0.791731in}{0.763581in}}%
\pgfpathlineto{\pgfqpoint{0.792027in}{0.763585in}}%
\pgfpathlineto{\pgfqpoint{0.792323in}{0.763589in}}%
\pgfpathlineto{\pgfqpoint{0.792619in}{0.763593in}}%
\pgfpathlineto{\pgfqpoint{0.792915in}{0.763597in}}%
\pgfpathlineto{\pgfqpoint{0.793211in}{0.763601in}}%
\pgfpathlineto{\pgfqpoint{0.793507in}{0.763605in}}%
\pgfpathlineto{\pgfqpoint{0.793803in}{0.763609in}}%
\pgfpathlineto{\pgfqpoint{0.794099in}{0.763613in}}%
\pgfpathlineto{\pgfqpoint{0.794395in}{0.763617in}}%
\pgfpathlineto{\pgfqpoint{0.794691in}{0.763621in}}%
\pgfpathlineto{\pgfqpoint{0.794987in}{0.763625in}}%
\pgfpathlineto{\pgfqpoint{0.795283in}{0.763629in}}%
\pgfpathlineto{\pgfqpoint{0.795579in}{0.763632in}}%
\pgfpathlineto{\pgfqpoint{0.795875in}{0.763636in}}%
\pgfpathlineto{\pgfqpoint{0.796171in}{0.763640in}}%
\pgfpathlineto{\pgfqpoint{0.796467in}{0.763644in}}%
\pgfpathlineto{\pgfqpoint{0.796763in}{0.763649in}}%
\pgfpathlineto{\pgfqpoint{0.797059in}{0.763723in}}%
\pgfpathlineto{\pgfqpoint{0.797355in}{0.763727in}}%
\pgfpathlineto{\pgfqpoint{0.797651in}{0.763726in}}%
\pgfpathlineto{\pgfqpoint{0.797947in}{0.763724in}}%
\pgfpathlineto{\pgfqpoint{0.798243in}{0.763723in}}%
\pgfpathlineto{\pgfqpoint{0.798539in}{0.763722in}}%
\pgfpathlineto{\pgfqpoint{0.798835in}{0.763721in}}%
\pgfpathlineto{\pgfqpoint{0.799131in}{0.763720in}}%
\pgfpathlineto{\pgfqpoint{0.799427in}{0.763718in}}%
\pgfpathlineto{\pgfqpoint{0.799723in}{0.763717in}}%
\pgfpathlineto{\pgfqpoint{0.800019in}{0.763716in}}%
\pgfpathlineto{\pgfqpoint{0.800315in}{0.763715in}}%
\pgfpathlineto{\pgfqpoint{0.800611in}{0.763713in}}%
\pgfpathlineto{\pgfqpoint{0.800907in}{0.763712in}}%
\pgfpathlineto{\pgfqpoint{0.801203in}{0.763711in}}%
\pgfpathlineto{\pgfqpoint{0.801499in}{0.763710in}}%
\pgfpathlineto{\pgfqpoint{0.801795in}{0.763709in}}%
\pgfpathlineto{\pgfqpoint{0.802091in}{0.763707in}}%
\pgfpathlineto{\pgfqpoint{0.802387in}{0.763706in}}%
\pgfpathlineto{\pgfqpoint{0.802683in}{0.763705in}}%
\pgfpathlineto{\pgfqpoint{0.802979in}{0.763704in}}%
\pgfpathlineto{\pgfqpoint{0.803275in}{0.763703in}}%
\pgfpathlineto{\pgfqpoint{0.803571in}{0.763701in}}%
\pgfpathlineto{\pgfqpoint{0.803867in}{0.763700in}}%
\pgfpathlineto{\pgfqpoint{0.804163in}{0.763699in}}%
\pgfpathlineto{\pgfqpoint{0.804459in}{0.763698in}}%
\pgfpathlineto{\pgfqpoint{0.804755in}{0.763697in}}%
\pgfpathlineto{\pgfqpoint{0.805051in}{0.763695in}}%
\pgfpathlineto{\pgfqpoint{0.805347in}{0.763694in}}%
\pgfpathlineto{\pgfqpoint{0.805643in}{0.763693in}}%
\pgfpathlineto{\pgfqpoint{0.805939in}{0.763692in}}%
\pgfpathlineto{\pgfqpoint{0.806235in}{0.763690in}}%
\pgfpathlineto{\pgfqpoint{0.806531in}{0.763689in}}%
\pgfpathlineto{\pgfqpoint{0.806827in}{0.763688in}}%
\pgfpathlineto{\pgfqpoint{0.807123in}{0.763687in}}%
\pgfpathlineto{\pgfqpoint{0.807419in}{0.763686in}}%
\pgfpathlineto{\pgfqpoint{0.807715in}{0.763684in}}%
\pgfpathlineto{\pgfqpoint{0.808011in}{0.763683in}}%
\pgfpathlineto{\pgfqpoint{0.808307in}{0.763682in}}%
\pgfpathlineto{\pgfqpoint{0.808603in}{0.763681in}}%
\pgfpathlineto{\pgfqpoint{0.808899in}{0.763680in}}%
\pgfpathlineto{\pgfqpoint{0.809195in}{0.763678in}}%
\pgfpathlineto{\pgfqpoint{0.809491in}{0.763677in}}%
\pgfpathlineto{\pgfqpoint{0.809787in}{0.763676in}}%
\pgfpathlineto{\pgfqpoint{0.810083in}{0.763675in}}%
\pgfpathlineto{\pgfqpoint{0.810379in}{0.763674in}}%
\pgfpathlineto{\pgfqpoint{0.810675in}{0.763672in}}%
\pgfpathlineto{\pgfqpoint{0.810971in}{0.763671in}}%
\pgfpathlineto{\pgfqpoint{0.811267in}{0.763670in}}%
\pgfpathlineto{\pgfqpoint{0.811563in}{0.763669in}}%
\pgfpathlineto{\pgfqpoint{0.811859in}{0.763667in}}%
\pgfpathlineto{\pgfqpoint{0.812155in}{0.763666in}}%
\pgfpathlineto{\pgfqpoint{0.812451in}{0.763665in}}%
\pgfpathlineto{\pgfqpoint{0.812747in}{0.763664in}}%
\pgfpathlineto{\pgfqpoint{0.813043in}{0.763663in}}%
\pgfpathlineto{\pgfqpoint{0.813339in}{0.763661in}}%
\pgfpathlineto{\pgfqpoint{0.813635in}{0.763660in}}%
\pgfpathlineto{\pgfqpoint{0.813931in}{0.763659in}}%
\pgfpathlineto{\pgfqpoint{0.814227in}{0.763658in}}%
\pgfpathlineto{\pgfqpoint{0.814523in}{0.763657in}}%
\pgfpathlineto{\pgfqpoint{0.814819in}{0.763655in}}%
\pgfpathlineto{\pgfqpoint{0.815115in}{0.763654in}}%
\pgfpathlineto{\pgfqpoint{0.815411in}{0.763653in}}%
\pgfpathlineto{\pgfqpoint{0.815707in}{0.763652in}}%
\pgfpathlineto{\pgfqpoint{0.816003in}{0.763651in}}%
\pgfpathlineto{\pgfqpoint{0.816299in}{0.763638in}}%
\pgfpathlineto{\pgfqpoint{0.816595in}{0.763710in}}%
\pgfpathlineto{\pgfqpoint{0.816891in}{0.763716in}}%
\pgfpathlineto{\pgfqpoint{0.817187in}{0.763712in}}%
\pgfpathlineto{\pgfqpoint{0.817483in}{0.763707in}}%
\pgfpathlineto{\pgfqpoint{0.817779in}{0.763702in}}%
\pgfpathlineto{\pgfqpoint{0.818075in}{0.763699in}}%
\pgfpathlineto{\pgfqpoint{0.818371in}{0.763707in}}%
\pgfpathlineto{\pgfqpoint{0.818667in}{0.763717in}}%
\pgfpathlineto{\pgfqpoint{0.818963in}{0.763727in}}%
\pgfpathlineto{\pgfqpoint{0.819259in}{0.763737in}}%
\pgfpathlineto{\pgfqpoint{0.819555in}{0.763747in}}%
\pgfpathlineto{\pgfqpoint{0.819851in}{0.763757in}}%
\pgfpathlineto{\pgfqpoint{0.820147in}{0.763767in}}%
\pgfpathlineto{\pgfqpoint{0.820443in}{0.763777in}}%
\pgfpathlineto{\pgfqpoint{0.820739in}{0.763787in}}%
\pgfpathlineto{\pgfqpoint{0.821035in}{0.763798in}}%
\pgfpathlineto{\pgfqpoint{0.821331in}{0.763808in}}%
\pgfpathlineto{\pgfqpoint{0.821627in}{0.763818in}}%
\pgfpathlineto{\pgfqpoint{0.821923in}{0.763828in}}%
\pgfpathlineto{\pgfqpoint{0.822219in}{0.763838in}}%
\pgfpathlineto{\pgfqpoint{0.822515in}{0.763848in}}%
\pgfpathlineto{\pgfqpoint{0.822811in}{0.763858in}}%
\pgfpathlineto{\pgfqpoint{0.823107in}{0.763747in}}%
\pgfpathlineto{\pgfqpoint{0.823403in}{0.763698in}}%
\pgfpathlineto{\pgfqpoint{0.823699in}{0.763744in}}%
\pgfpathlineto{\pgfqpoint{0.823995in}{0.763739in}}%
\pgfpathlineto{\pgfqpoint{0.824291in}{0.763754in}}%
\pgfpathlineto{\pgfqpoint{0.824587in}{0.763742in}}%
\pgfpathlineto{\pgfqpoint{0.824883in}{0.763734in}}%
\pgfpathlineto{\pgfqpoint{0.825179in}{0.763839in}}%
\pgfpathlineto{\pgfqpoint{0.825475in}{0.764058in}}%
\pgfpathlineto{\pgfqpoint{0.825771in}{0.764313in}}%
\pgfpathlineto{\pgfqpoint{0.826067in}{0.764567in}}%
\pgfpathlineto{\pgfqpoint{0.826363in}{0.764822in}}%
\pgfpathlineto{\pgfqpoint{0.826659in}{0.765076in}}%
\pgfpathlineto{\pgfqpoint{0.826955in}{0.765331in}}%
\pgfpathlineto{\pgfqpoint{0.827251in}{0.765585in}}%
\pgfpathlineto{\pgfqpoint{0.827547in}{0.765840in}}%
\pgfpathlineto{\pgfqpoint{0.827843in}{0.766094in}}%
\pgfpathlineto{\pgfqpoint{0.828139in}{0.766349in}}%
\pgfpathlineto{\pgfqpoint{0.828435in}{0.766603in}}%
\pgfpathlineto{\pgfqpoint{0.828731in}{0.766858in}}%
\pgfpathlineto{\pgfqpoint{0.829027in}{0.767112in}}%
\pgfpathlineto{\pgfqpoint{0.829323in}{0.767367in}}%
\pgfpathlineto{\pgfqpoint{0.829619in}{0.767621in}}%
\pgfpathlineto{\pgfqpoint{0.829915in}{0.767876in}}%
\pgfpathlineto{\pgfqpoint{0.830211in}{0.768130in}}%
\pgfpathlineto{\pgfqpoint{0.830507in}{0.768385in}}%
\pgfpathlineto{\pgfqpoint{0.830803in}{0.768639in}}%
\pgfpathlineto{\pgfqpoint{0.831099in}{0.768184in}}%
\pgfpathlineto{\pgfqpoint{0.831395in}{0.764636in}}%
\pgfpathlineto{\pgfqpoint{0.831691in}{0.766267in}}%
\pgfpathlineto{\pgfqpoint{0.831987in}{0.767922in}}%
\pgfpathlineto{\pgfqpoint{0.832283in}{0.768564in}}%
\pgfpathlineto{\pgfqpoint{0.832579in}{0.766797in}}%
\pgfpathlineto{\pgfqpoint{0.832875in}{0.766691in}}%
\pgfpathlineto{\pgfqpoint{0.833171in}{0.766790in}}%
\pgfpathlineto{\pgfqpoint{0.833467in}{0.766888in}}%
\pgfpathlineto{\pgfqpoint{0.833763in}{0.766987in}}%
\pgfpathlineto{\pgfqpoint{0.834059in}{0.767086in}}%
\pgfpathlineto{\pgfqpoint{0.834355in}{0.767185in}}%
\pgfpathlineto{\pgfqpoint{0.834651in}{0.767284in}}%
\pgfpathlineto{\pgfqpoint{0.834947in}{0.767382in}}%
\pgfpathlineto{\pgfqpoint{0.835243in}{0.767481in}}%
\pgfpathlineto{\pgfqpoint{0.835539in}{0.767580in}}%
\pgfpathlineto{\pgfqpoint{0.835835in}{0.767679in}}%
\pgfpathlineto{\pgfqpoint{0.836131in}{0.767778in}}%
\pgfpathlineto{\pgfqpoint{0.836427in}{0.767877in}}%
\pgfpathlineto{\pgfqpoint{0.836723in}{0.767975in}}%
\pgfpathlineto{\pgfqpoint{0.837019in}{0.768074in}}%
\pgfpathlineto{\pgfqpoint{0.837315in}{0.768173in}}%
\pgfpathlineto{\pgfqpoint{0.837611in}{0.768272in}}%
\pgfpathlineto{\pgfqpoint{0.837907in}{0.768371in}}%
\pgfpathlineto{\pgfqpoint{0.838203in}{0.768469in}}%
\pgfpathlineto{\pgfqpoint{0.838499in}{0.768568in}}%
\pgfpathlineto{\pgfqpoint{0.838795in}{0.768667in}}%
\pgfpathlineto{\pgfqpoint{0.839091in}{0.768766in}}%
\pgfpathlineto{\pgfqpoint{0.839387in}{0.768865in}}%
\pgfpathlineto{\pgfqpoint{0.839683in}{0.768963in}}%
\pgfpathlineto{\pgfqpoint{0.839979in}{0.769062in}}%
\pgfpathlineto{\pgfqpoint{0.840275in}{0.769161in}}%
\pgfpathlineto{\pgfqpoint{0.840571in}{0.769229in}}%
\pgfpathlineto{\pgfqpoint{0.840867in}{0.769228in}}%
\pgfpathlineto{\pgfqpoint{0.841163in}{0.769226in}}%
\pgfpathlineto{\pgfqpoint{0.841459in}{0.769225in}}%
\pgfpathlineto{\pgfqpoint{0.841755in}{0.769224in}}%
\pgfpathlineto{\pgfqpoint{0.842051in}{0.769223in}}%
\pgfpathlineto{\pgfqpoint{0.842347in}{0.769221in}}%
\pgfpathlineto{\pgfqpoint{0.842643in}{0.769220in}}%
\pgfpathlineto{\pgfqpoint{0.842939in}{0.769219in}}%
\pgfpathlineto{\pgfqpoint{0.843235in}{0.769217in}}%
\pgfpathlineto{\pgfqpoint{0.843531in}{0.769216in}}%
\pgfpathlineto{\pgfqpoint{0.843827in}{0.769215in}}%
\pgfpathlineto{\pgfqpoint{0.844123in}{0.769213in}}%
\pgfpathlineto{\pgfqpoint{0.844419in}{0.769212in}}%
\pgfpathlineto{\pgfqpoint{0.844715in}{0.769211in}}%
\pgfpathlineto{\pgfqpoint{0.845011in}{0.769209in}}%
\pgfpathlineto{\pgfqpoint{0.845307in}{0.769208in}}%
\pgfpathlineto{\pgfqpoint{0.845603in}{0.769207in}}%
\pgfpathlineto{\pgfqpoint{0.845899in}{0.769205in}}%
\pgfpathlineto{\pgfqpoint{0.846195in}{0.769204in}}%
\pgfpathlineto{\pgfqpoint{0.846491in}{0.769203in}}%
\pgfpathlineto{\pgfqpoint{0.846787in}{0.769201in}}%
\pgfpathlineto{\pgfqpoint{0.847083in}{0.769200in}}%
\pgfpathlineto{\pgfqpoint{0.847379in}{0.769199in}}%
\pgfpathlineto{\pgfqpoint{0.847675in}{0.769197in}}%
\pgfpathlineto{\pgfqpoint{0.847971in}{0.769196in}}%
\pgfpathlineto{\pgfqpoint{0.848267in}{0.769195in}}%
\pgfpathlineto{\pgfqpoint{0.848563in}{0.769193in}}%
\pgfpathlineto{\pgfqpoint{0.848859in}{0.769192in}}%
\pgfpathlineto{\pgfqpoint{0.849155in}{0.769191in}}%
\pgfpathlineto{\pgfqpoint{0.849451in}{0.769189in}}%
\pgfpathlineto{\pgfqpoint{0.849747in}{0.769188in}}%
\pgfpathlineto{\pgfqpoint{0.850043in}{0.769187in}}%
\pgfpathlineto{\pgfqpoint{0.850340in}{0.769185in}}%
\pgfpathlineto{\pgfqpoint{0.850636in}{0.769184in}}%
\pgfpathlineto{\pgfqpoint{0.850932in}{0.769183in}}%
\pgfpathlineto{\pgfqpoint{0.851228in}{0.769181in}}%
\pgfpathlineto{\pgfqpoint{0.851524in}{0.769180in}}%
\pgfpathlineto{\pgfqpoint{0.851820in}{0.769179in}}%
\pgfpathlineto{\pgfqpoint{0.852116in}{0.769177in}}%
\pgfpathlineto{\pgfqpoint{0.852412in}{0.769176in}}%
\pgfpathlineto{\pgfqpoint{0.852708in}{0.769175in}}%
\pgfpathlineto{\pgfqpoint{0.853004in}{0.769174in}}%
\pgfpathlineto{\pgfqpoint{0.853300in}{0.769172in}}%
\pgfpathlineto{\pgfqpoint{0.853596in}{0.769171in}}%
\pgfpathlineto{\pgfqpoint{0.853892in}{0.769170in}}%
\pgfpathlineto{\pgfqpoint{0.854188in}{0.769168in}}%
\pgfpathlineto{\pgfqpoint{0.854484in}{0.769167in}}%
\pgfpathlineto{\pgfqpoint{0.854780in}{0.769166in}}%
\pgfpathlineto{\pgfqpoint{0.855076in}{0.769164in}}%
\pgfpathlineto{\pgfqpoint{0.855372in}{0.769163in}}%
\pgfpathlineto{\pgfqpoint{0.855668in}{0.769162in}}%
\pgfpathlineto{\pgfqpoint{0.855964in}{0.769160in}}%
\pgfpathlineto{\pgfqpoint{0.856260in}{0.769159in}}%
\pgfpathlineto{\pgfqpoint{0.856556in}{0.769158in}}%
\pgfpathlineto{\pgfqpoint{0.856852in}{0.769156in}}%
\pgfpathlineto{\pgfqpoint{0.857148in}{0.769155in}}%
\pgfpathlineto{\pgfqpoint{0.857444in}{0.769154in}}%
\pgfpathlineto{\pgfqpoint{0.857740in}{0.769152in}}%
\pgfpathlineto{\pgfqpoint{0.858036in}{0.769151in}}%
\pgfpathlineto{\pgfqpoint{0.858332in}{0.769150in}}%
\pgfpathlineto{\pgfqpoint{0.858628in}{0.769148in}}%
\pgfpathlineto{\pgfqpoint{0.858924in}{0.769147in}}%
\pgfpathlineto{\pgfqpoint{0.859220in}{0.769146in}}%
\pgfpathlineto{\pgfqpoint{0.859516in}{0.769144in}}%
\pgfpathlineto{\pgfqpoint{0.859812in}{0.769143in}}%
\pgfpathlineto{\pgfqpoint{0.860108in}{0.769142in}}%
\pgfpathlineto{\pgfqpoint{0.860404in}{0.769140in}}%
\pgfpathlineto{\pgfqpoint{0.860700in}{0.769139in}}%
\pgfpathlineto{\pgfqpoint{0.860996in}{0.769138in}}%
\pgfpathlineto{\pgfqpoint{0.861292in}{0.769136in}}%
\pgfpathlineto{\pgfqpoint{0.861588in}{0.769135in}}%
\pgfpathlineto{\pgfqpoint{0.861884in}{0.769134in}}%
\pgfpathlineto{\pgfqpoint{0.862180in}{0.769132in}}%
\pgfpathlineto{\pgfqpoint{0.862476in}{0.769131in}}%
\pgfpathlineto{\pgfqpoint{0.862772in}{0.769130in}}%
\pgfpathlineto{\pgfqpoint{0.863068in}{0.769129in}}%
\pgfpathlineto{\pgfqpoint{0.863364in}{0.769127in}}%
\pgfpathlineto{\pgfqpoint{0.863660in}{0.769126in}}%
\pgfpathlineto{\pgfqpoint{0.863956in}{0.769125in}}%
\pgfpathlineto{\pgfqpoint{0.864252in}{0.769123in}}%
\pgfpathlineto{\pgfqpoint{0.864548in}{0.769122in}}%
\pgfpathlineto{\pgfqpoint{0.864844in}{0.769121in}}%
\pgfpathlineto{\pgfqpoint{0.865140in}{0.769119in}}%
\pgfpathlineto{\pgfqpoint{0.865436in}{0.769118in}}%
\pgfpathlineto{\pgfqpoint{0.865732in}{0.769117in}}%
\pgfpathlineto{\pgfqpoint{0.866028in}{0.769115in}}%
\pgfpathlineto{\pgfqpoint{0.866324in}{0.769114in}}%
\pgfpathlineto{\pgfqpoint{0.866620in}{0.769113in}}%
\pgfpathlineto{\pgfqpoint{0.866916in}{0.769111in}}%
\pgfpathlineto{\pgfqpoint{0.867212in}{0.769110in}}%
\pgfpathlineto{\pgfqpoint{0.867508in}{0.769109in}}%
\pgfpathlineto{\pgfqpoint{0.867804in}{0.769107in}}%
\pgfpathlineto{\pgfqpoint{0.868100in}{0.769106in}}%
\pgfpathlineto{\pgfqpoint{0.868396in}{0.769105in}}%
\pgfpathlineto{\pgfqpoint{0.868692in}{0.769103in}}%
\pgfpathlineto{\pgfqpoint{0.868988in}{0.769102in}}%
\pgfpathlineto{\pgfqpoint{0.869284in}{0.769101in}}%
\pgfpathlineto{\pgfqpoint{0.869580in}{0.769099in}}%
\pgfpathlineto{\pgfqpoint{0.869876in}{0.769098in}}%
\pgfpathlineto{\pgfqpoint{0.870172in}{0.769097in}}%
\pgfpathlineto{\pgfqpoint{0.870468in}{0.769095in}}%
\pgfpathlineto{\pgfqpoint{0.870764in}{0.769094in}}%
\pgfpathlineto{\pgfqpoint{0.871060in}{0.769093in}}%
\pgfpathlineto{\pgfqpoint{0.871356in}{0.769091in}}%
\pgfpathlineto{\pgfqpoint{0.871652in}{0.769090in}}%
\pgfpathlineto{\pgfqpoint{0.871948in}{0.769089in}}%
\pgfpathlineto{\pgfqpoint{0.872244in}{0.769087in}}%
\pgfpathlineto{\pgfqpoint{0.872540in}{0.769086in}}%
\pgfpathlineto{\pgfqpoint{0.872836in}{0.769085in}}%
\pgfpathlineto{\pgfqpoint{0.873132in}{0.769083in}}%
\pgfpathlineto{\pgfqpoint{0.873428in}{0.769081in}}%
\pgfpathlineto{\pgfqpoint{0.873724in}{0.769078in}}%
\pgfpathlineto{\pgfqpoint{0.874020in}{0.769076in}}%
\pgfpathlineto{\pgfqpoint{0.874316in}{0.769073in}}%
\pgfpathlineto{\pgfqpoint{0.874612in}{0.769070in}}%
\pgfpathlineto{\pgfqpoint{0.874908in}{0.769067in}}%
\pgfpathlineto{\pgfqpoint{0.875204in}{0.769064in}}%
\pgfpathlineto{\pgfqpoint{0.875500in}{0.769062in}}%
\pgfpathlineto{\pgfqpoint{0.875796in}{0.769059in}}%
\pgfpathlineto{\pgfqpoint{0.876092in}{0.769058in}}%
\pgfpathlineto{\pgfqpoint{0.876388in}{0.769057in}}%
\pgfpathlineto{\pgfqpoint{0.876684in}{0.769056in}}%
\pgfpathlineto{\pgfqpoint{0.876980in}{0.769056in}}%
\pgfpathlineto{\pgfqpoint{0.877276in}{0.769055in}}%
\pgfpathlineto{\pgfqpoint{0.877572in}{0.769054in}}%
\pgfpathlineto{\pgfqpoint{0.877868in}{0.769053in}}%
\pgfpathlineto{\pgfqpoint{0.878164in}{0.769053in}}%
\pgfpathlineto{\pgfqpoint{0.878460in}{0.769052in}}%
\pgfpathlineto{\pgfqpoint{0.878756in}{0.769051in}}%
\pgfpathlineto{\pgfqpoint{0.879052in}{0.769050in}}%
\pgfpathlineto{\pgfqpoint{0.879348in}{0.769050in}}%
\pgfpathlineto{\pgfqpoint{0.879644in}{0.769049in}}%
\pgfpathlineto{\pgfqpoint{0.879940in}{0.769048in}}%
\pgfpathlineto{\pgfqpoint{0.880236in}{0.769048in}}%
\pgfpathlineto{\pgfqpoint{0.880532in}{0.769047in}}%
\pgfpathlineto{\pgfqpoint{0.880828in}{0.769046in}}%
\pgfpathlineto{\pgfqpoint{0.881124in}{0.769045in}}%
\pgfpathlineto{\pgfqpoint{0.881420in}{0.769045in}}%
\pgfpathlineto{\pgfqpoint{0.881716in}{0.769044in}}%
\pgfpathlineto{\pgfqpoint{0.882012in}{0.769043in}}%
\pgfpathlineto{\pgfqpoint{0.882308in}{0.769038in}}%
\pgfpathlineto{\pgfqpoint{0.882604in}{0.769030in}}%
\pgfpathlineto{\pgfqpoint{0.882900in}{0.769040in}}%
\pgfpathlineto{\pgfqpoint{0.883196in}{0.769050in}}%
\pgfpathlineto{\pgfqpoint{0.883492in}{0.769061in}}%
\pgfpathlineto{\pgfqpoint{0.883788in}{0.769071in}}%
\pgfpathlineto{\pgfqpoint{0.884084in}{0.769082in}}%
\pgfpathlineto{\pgfqpoint{0.884380in}{0.769092in}}%
\pgfpathlineto{\pgfqpoint{0.884676in}{0.769103in}}%
\pgfpathlineto{\pgfqpoint{0.884972in}{0.769113in}}%
\pgfpathlineto{\pgfqpoint{0.885268in}{0.769124in}}%
\pgfpathlineto{\pgfqpoint{0.885564in}{0.769134in}}%
\pgfpathlineto{\pgfqpoint{0.885860in}{0.769145in}}%
\pgfpathlineto{\pgfqpoint{0.886156in}{0.769155in}}%
\pgfpathlineto{\pgfqpoint{0.886452in}{0.769166in}}%
\pgfpathlineto{\pgfqpoint{0.886748in}{0.769332in}}%
\pgfpathlineto{\pgfqpoint{0.887044in}{0.769922in}}%
\pgfpathlineto{\pgfqpoint{0.887340in}{0.769992in}}%
\pgfpathlineto{\pgfqpoint{0.887636in}{0.770028in}}%
\pgfpathlineto{\pgfqpoint{0.887932in}{0.770026in}}%
\pgfpathlineto{\pgfqpoint{0.888228in}{0.770025in}}%
\pgfpathlineto{\pgfqpoint{0.888524in}{0.770023in}}%
\pgfpathlineto{\pgfqpoint{0.888820in}{0.770022in}}%
\pgfpathlineto{\pgfqpoint{0.889116in}{0.770020in}}%
\pgfpathlineto{\pgfqpoint{0.889412in}{0.770018in}}%
\pgfpathlineto{\pgfqpoint{0.889708in}{0.770067in}}%
\pgfpathlineto{\pgfqpoint{0.890004in}{0.770229in}}%
\pgfpathlineto{\pgfqpoint{0.890300in}{0.770363in}}%
\pgfpathlineto{\pgfqpoint{0.890596in}{0.770391in}}%
\pgfpathlineto{\pgfqpoint{0.890892in}{0.770392in}}%
\pgfpathlineto{\pgfqpoint{0.891188in}{0.770400in}}%
\pgfpathlineto{\pgfqpoint{0.891484in}{0.770400in}}%
\pgfpathlineto{\pgfqpoint{0.891780in}{0.770398in}}%
\pgfpathlineto{\pgfqpoint{0.892076in}{0.770397in}}%
\pgfpathlineto{\pgfqpoint{0.892372in}{0.770395in}}%
\pgfpathlineto{\pgfqpoint{0.892668in}{0.770394in}}%
\pgfpathlineto{\pgfqpoint{0.892964in}{0.770393in}}%
\pgfpathlineto{\pgfqpoint{0.893260in}{0.770391in}}%
\pgfpathlineto{\pgfqpoint{0.893556in}{0.770390in}}%
\pgfpathlineto{\pgfqpoint{0.893852in}{0.770388in}}%
\pgfpathlineto{\pgfqpoint{0.894148in}{0.770386in}}%
\pgfpathlineto{\pgfqpoint{0.894444in}{0.770478in}}%
\pgfpathlineto{\pgfqpoint{0.894740in}{0.770540in}}%
\pgfpathlineto{\pgfqpoint{0.895036in}{0.770422in}}%
\pgfpathlineto{\pgfqpoint{0.895332in}{0.767725in}}%
\pgfpathlineto{\pgfqpoint{0.895628in}{0.766661in}}%
\pgfpathlineto{\pgfqpoint{0.895924in}{0.765418in}}%
\pgfpathlineto{\pgfqpoint{0.896220in}{0.765361in}}%
\pgfpathlineto{\pgfqpoint{0.896516in}{0.765349in}}%
\pgfpathlineto{\pgfqpoint{0.896812in}{0.765364in}}%
\pgfpathlineto{\pgfqpoint{0.897108in}{0.765368in}}%
\pgfpathlineto{\pgfqpoint{0.897404in}{0.765368in}}%
\pgfpathlineto{\pgfqpoint{0.897700in}{0.765373in}}%
\pgfpathlineto{\pgfqpoint{0.897996in}{0.765377in}}%
\pgfpathlineto{\pgfqpoint{0.898292in}{0.765380in}}%
\pgfpathlineto{\pgfqpoint{0.898588in}{0.765383in}}%
\pgfpathlineto{\pgfqpoint{0.898884in}{0.765387in}}%
\pgfpathlineto{\pgfqpoint{0.899180in}{0.765390in}}%
\pgfpathlineto{\pgfqpoint{0.899476in}{0.765393in}}%
\pgfpathlineto{\pgfqpoint{0.899772in}{0.765396in}}%
\pgfpathlineto{\pgfqpoint{0.900068in}{0.765399in}}%
\pgfpathlineto{\pgfqpoint{0.900364in}{0.765402in}}%
\pgfpathlineto{\pgfqpoint{0.900660in}{0.765405in}}%
\pgfpathlineto{\pgfqpoint{0.900956in}{0.765408in}}%
\pgfpathlineto{\pgfqpoint{0.901252in}{0.765411in}}%
\pgfpathlineto{\pgfqpoint{0.901548in}{0.765414in}}%
\pgfpathlineto{\pgfqpoint{0.901844in}{0.765417in}}%
\pgfpathlineto{\pgfqpoint{0.902140in}{0.765427in}}%
\pgfpathlineto{\pgfqpoint{0.902436in}{0.765469in}}%
\pgfpathlineto{\pgfqpoint{0.902732in}{0.765517in}}%
\pgfpathlineto{\pgfqpoint{0.903028in}{0.765565in}}%
\pgfpathlineto{\pgfqpoint{0.903324in}{0.765613in}}%
\pgfpathlineto{\pgfqpoint{0.903620in}{0.765661in}}%
\pgfpathlineto{\pgfqpoint{0.903916in}{0.765681in}}%
\pgfpathlineto{\pgfqpoint{0.904212in}{0.765679in}}%
\pgfpathlineto{\pgfqpoint{0.904508in}{0.765688in}}%
\pgfpathlineto{\pgfqpoint{0.904804in}{0.765697in}}%
\pgfpathlineto{\pgfqpoint{0.905100in}{0.765706in}}%
\pgfpathlineto{\pgfqpoint{0.905396in}{0.765715in}}%
\pgfpathlineto{\pgfqpoint{0.905692in}{0.765725in}}%
\pgfpathlineto{\pgfqpoint{0.905988in}{0.765734in}}%
\pgfpathlineto{\pgfqpoint{0.906284in}{0.765743in}}%
\pgfpathlineto{\pgfqpoint{0.906580in}{0.765752in}}%
\pgfpathlineto{\pgfqpoint{0.906876in}{0.765762in}}%
\pgfpathlineto{\pgfqpoint{0.907172in}{0.765771in}}%
\pgfpathlineto{\pgfqpoint{0.907468in}{0.765780in}}%
\pgfpathlineto{\pgfqpoint{0.907764in}{0.765789in}}%
\pgfpathlineto{\pgfqpoint{0.908060in}{0.765798in}}%
\pgfpathlineto{\pgfqpoint{0.908356in}{0.765808in}}%
\pgfpathlineto{\pgfqpoint{0.908652in}{0.765817in}}%
\pgfpathlineto{\pgfqpoint{0.908948in}{0.765826in}}%
\pgfpathlineto{\pgfqpoint{0.909244in}{0.765835in}}%
\pgfpathlineto{\pgfqpoint{0.909540in}{0.765845in}}%
\pgfpathlineto{\pgfqpoint{0.909836in}{0.765854in}}%
\pgfpathlineto{\pgfqpoint{0.910132in}{0.765863in}}%
\pgfpathlineto{\pgfqpoint{0.910428in}{0.765872in}}%
\pgfpathlineto{\pgfqpoint{0.910724in}{0.765881in}}%
\pgfpathlineto{\pgfqpoint{0.911020in}{0.765889in}}%
\pgfpathlineto{\pgfqpoint{0.911316in}{0.765888in}}%
\pgfpathlineto{\pgfqpoint{0.911612in}{0.765887in}}%
\pgfpathlineto{\pgfqpoint{0.911908in}{0.765885in}}%
\pgfpathlineto{\pgfqpoint{0.912204in}{0.765884in}}%
\pgfpathlineto{\pgfqpoint{0.912500in}{0.765883in}}%
\pgfpathlineto{\pgfqpoint{0.912796in}{0.765881in}}%
\pgfpathlineto{\pgfqpoint{0.913092in}{0.765880in}}%
\pgfpathlineto{\pgfqpoint{0.913388in}{0.765879in}}%
\pgfpathlineto{\pgfqpoint{0.913684in}{0.765877in}}%
\pgfpathlineto{\pgfqpoint{0.913980in}{0.765876in}}%
\pgfpathlineto{\pgfqpoint{0.914276in}{0.765874in}}%
\pgfpathlineto{\pgfqpoint{0.914572in}{0.765873in}}%
\pgfpathlineto{\pgfqpoint{0.914868in}{0.765872in}}%
\pgfpathlineto{\pgfqpoint{0.915164in}{0.765870in}}%
\pgfpathlineto{\pgfqpoint{0.915460in}{0.765869in}}%
\pgfpathlineto{\pgfqpoint{0.915756in}{0.765985in}}%
\pgfpathlineto{\pgfqpoint{0.916052in}{0.766279in}}%
\pgfpathlineto{\pgfqpoint{0.916348in}{0.766295in}}%
\pgfpathlineto{\pgfqpoint{0.916644in}{0.766332in}}%
\pgfpathlineto{\pgfqpoint{0.916940in}{0.766400in}}%
\pgfpathlineto{\pgfqpoint{0.917236in}{0.766397in}}%
\pgfpathlineto{\pgfqpoint{0.917532in}{0.766411in}}%
\pgfpathlineto{\pgfqpoint{0.917829in}{0.766485in}}%
\pgfpathlineto{\pgfqpoint{0.918125in}{0.766485in}}%
\pgfpathlineto{\pgfqpoint{0.918421in}{0.766452in}}%
\pgfpathlineto{\pgfqpoint{0.918717in}{0.766452in}}%
\pgfpathlineto{\pgfqpoint{0.919013in}{0.766451in}}%
\pgfpathlineto{\pgfqpoint{0.919309in}{0.766450in}}%
\pgfpathlineto{\pgfqpoint{0.919605in}{0.766448in}}%
\pgfpathlineto{\pgfqpoint{0.919901in}{0.766447in}}%
\pgfpathlineto{\pgfqpoint{0.920197in}{0.766446in}}%
\pgfpathlineto{\pgfqpoint{0.920493in}{0.766445in}}%
\pgfpathlineto{\pgfqpoint{0.920789in}{0.766443in}}%
\pgfpathlineto{\pgfqpoint{0.921085in}{0.766442in}}%
\pgfpathlineto{\pgfqpoint{0.921381in}{0.766441in}}%
\pgfpathlineto{\pgfqpoint{0.921677in}{0.766439in}}%
\pgfpathlineto{\pgfqpoint{0.921973in}{0.766438in}}%
\pgfpathlineto{\pgfqpoint{0.922269in}{0.766437in}}%
\pgfpathlineto{\pgfqpoint{0.922565in}{0.766436in}}%
\pgfpathlineto{\pgfqpoint{0.922861in}{0.766434in}}%
\pgfpathlineto{\pgfqpoint{0.923157in}{0.766433in}}%
\pgfpathlineto{\pgfqpoint{0.923453in}{0.766461in}}%
\pgfpathlineto{\pgfqpoint{0.923749in}{0.766527in}}%
\pgfpathlineto{\pgfqpoint{0.924045in}{0.766592in}}%
\pgfpathlineto{\pgfqpoint{0.924341in}{0.766658in}}%
\pgfpathlineto{\pgfqpoint{0.924637in}{0.766700in}}%
\pgfpathlineto{\pgfqpoint{0.924933in}{0.766690in}}%
\pgfpathlineto{\pgfqpoint{0.925229in}{0.766679in}}%
\pgfpathlineto{\pgfqpoint{0.925525in}{0.766703in}}%
\pgfpathlineto{\pgfqpoint{0.925821in}{0.766768in}}%
\pgfpathlineto{\pgfqpoint{0.926117in}{0.766834in}}%
\pgfpathlineto{\pgfqpoint{0.926413in}{0.766899in}}%
\pgfpathlineto{\pgfqpoint{0.926709in}{0.766964in}}%
\pgfpathlineto{\pgfqpoint{0.927005in}{0.767030in}}%
\pgfpathlineto{\pgfqpoint{0.927301in}{0.767095in}}%
\pgfpathlineto{\pgfqpoint{0.927597in}{0.767161in}}%
\pgfpathlineto{\pgfqpoint{0.927893in}{0.767226in}}%
\pgfpathlineto{\pgfqpoint{0.928189in}{0.767292in}}%
\pgfpathlineto{\pgfqpoint{0.928485in}{0.767357in}}%
\pgfpathlineto{\pgfqpoint{0.928781in}{0.767423in}}%
\pgfpathlineto{\pgfqpoint{0.929077in}{0.767488in}}%
\pgfpathlineto{\pgfqpoint{0.929373in}{0.767554in}}%
\pgfpathlineto{\pgfqpoint{0.929669in}{0.767619in}}%
\pgfpathlineto{\pgfqpoint{0.929965in}{0.767685in}}%
\pgfpathlineto{\pgfqpoint{0.930261in}{0.767750in}}%
\pgfpathlineto{\pgfqpoint{0.930557in}{0.767816in}}%
\pgfpathlineto{\pgfqpoint{0.930853in}{0.767881in}}%
\pgfpathlineto{\pgfqpoint{0.931149in}{0.767947in}}%
\pgfpathlineto{\pgfqpoint{0.931445in}{0.768012in}}%
\pgfpathlineto{\pgfqpoint{0.931741in}{0.768078in}}%
\pgfpathlineto{\pgfqpoint{0.932037in}{0.768137in}}%
\pgfpathlineto{\pgfqpoint{0.932333in}{0.768148in}}%
\pgfpathlineto{\pgfqpoint{0.932629in}{0.768146in}}%
\pgfpathlineto{\pgfqpoint{0.932925in}{0.768145in}}%
\pgfpathlineto{\pgfqpoint{0.933221in}{0.768143in}}%
\pgfpathlineto{\pgfqpoint{0.933517in}{0.768142in}}%
\pgfpathlineto{\pgfqpoint{0.933813in}{0.768140in}}%
\pgfpathlineto{\pgfqpoint{0.934109in}{0.768139in}}%
\pgfpathlineto{\pgfqpoint{0.934405in}{0.768137in}}%
\pgfpathlineto{\pgfqpoint{0.934701in}{0.768136in}}%
\pgfpathlineto{\pgfqpoint{0.934997in}{0.768134in}}%
\pgfpathlineto{\pgfqpoint{0.935293in}{0.768133in}}%
\pgfpathlineto{\pgfqpoint{0.935589in}{0.768131in}}%
\pgfpathlineto{\pgfqpoint{0.935885in}{0.768130in}}%
\pgfpathlineto{\pgfqpoint{0.936181in}{0.768128in}}%
\pgfpathlineto{\pgfqpoint{0.936477in}{0.768127in}}%
\pgfpathlineto{\pgfqpoint{0.936773in}{0.768125in}}%
\pgfpathlineto{\pgfqpoint{0.937069in}{0.768124in}}%
\pgfpathlineto{\pgfqpoint{0.937365in}{0.768123in}}%
\pgfpathlineto{\pgfqpoint{0.937661in}{0.768124in}}%
\pgfpathlineto{\pgfqpoint{0.937957in}{0.768174in}}%
\pgfpathlineto{\pgfqpoint{0.938253in}{0.768968in}}%
\pgfpathlineto{\pgfqpoint{0.938549in}{0.769193in}}%
\pgfpathlineto{\pgfqpoint{0.938845in}{0.768924in}}%
\pgfpathlineto{\pgfqpoint{0.939141in}{0.768656in}}%
\pgfpathlineto{\pgfqpoint{0.939437in}{0.768387in}}%
\pgfpathlineto{\pgfqpoint{0.939733in}{0.768154in}}%
\pgfpathlineto{\pgfqpoint{0.940029in}{0.768122in}}%
\pgfpathlineto{\pgfqpoint{0.940325in}{0.768122in}}%
\pgfpathlineto{\pgfqpoint{0.940621in}{0.768122in}}%
\pgfpathlineto{\pgfqpoint{0.940917in}{0.768123in}}%
\pgfpathlineto{\pgfqpoint{0.941213in}{0.768123in}}%
\pgfpathlineto{\pgfqpoint{0.941509in}{0.768123in}}%
\pgfpathlineto{\pgfqpoint{0.941805in}{0.768124in}}%
\pgfpathlineto{\pgfqpoint{0.942101in}{0.768124in}}%
\pgfpathlineto{\pgfqpoint{0.942397in}{0.768124in}}%
\pgfpathlineto{\pgfqpoint{0.942693in}{0.768125in}}%
\pgfpathlineto{\pgfqpoint{0.942989in}{0.768125in}}%
\pgfpathlineto{\pgfqpoint{0.943285in}{0.768126in}}%
\pgfpathlineto{\pgfqpoint{0.943581in}{0.768126in}}%
\pgfpathlineto{\pgfqpoint{0.943877in}{0.768126in}}%
\pgfpathlineto{\pgfqpoint{0.944173in}{0.768127in}}%
\pgfpathlineto{\pgfqpoint{0.944469in}{0.768127in}}%
\pgfpathlineto{\pgfqpoint{0.944765in}{0.768127in}}%
\pgfpathlineto{\pgfqpoint{0.945061in}{0.768128in}}%
\pgfpathlineto{\pgfqpoint{0.945357in}{0.768128in}}%
\pgfpathlineto{\pgfqpoint{0.945653in}{0.768129in}}%
\pgfpathlineto{\pgfqpoint{0.945949in}{0.768129in}}%
\pgfpathlineto{\pgfqpoint{0.946245in}{0.768129in}}%
\pgfpathlineto{\pgfqpoint{0.946541in}{0.768130in}}%
\pgfpathlineto{\pgfqpoint{0.946837in}{0.768130in}}%
\pgfpathlineto{\pgfqpoint{0.947133in}{0.768130in}}%
\pgfpathlineto{\pgfqpoint{0.947429in}{0.768131in}}%
\pgfpathlineto{\pgfqpoint{0.947725in}{0.768131in}}%
\pgfpathlineto{\pgfqpoint{0.948021in}{0.768132in}}%
\pgfpathlineto{\pgfqpoint{0.948317in}{0.768132in}}%
\pgfpathlineto{\pgfqpoint{0.948613in}{0.768132in}}%
\pgfpathlineto{\pgfqpoint{0.948909in}{0.768133in}}%
\pgfpathlineto{\pgfqpoint{0.949205in}{0.768133in}}%
\pgfpathlineto{\pgfqpoint{0.949501in}{0.768133in}}%
\pgfpathlineto{\pgfqpoint{0.949797in}{0.768134in}}%
\pgfpathlineto{\pgfqpoint{0.950093in}{0.768134in}}%
\pgfpathlineto{\pgfqpoint{0.950389in}{0.768134in}}%
\pgfpathlineto{\pgfqpoint{0.950685in}{0.768135in}}%
\pgfpathlineto{\pgfqpoint{0.950981in}{0.768135in}}%
\pgfpathlineto{\pgfqpoint{0.951277in}{0.768136in}}%
\pgfpathlineto{\pgfqpoint{0.951573in}{0.768136in}}%
\pgfpathlineto{\pgfqpoint{0.951869in}{0.768136in}}%
\pgfpathlineto{\pgfqpoint{0.952165in}{0.768137in}}%
\pgfpathlineto{\pgfqpoint{0.952461in}{0.768137in}}%
\pgfpathlineto{\pgfqpoint{0.952757in}{0.768137in}}%
\pgfpathlineto{\pgfqpoint{0.953053in}{0.768138in}}%
\pgfpathlineto{\pgfqpoint{0.953349in}{0.768138in}}%
\pgfpathlineto{\pgfqpoint{0.953645in}{0.768139in}}%
\pgfpathlineto{\pgfqpoint{0.953941in}{0.768139in}}%
\pgfpathlineto{\pgfqpoint{0.954237in}{0.768139in}}%
\pgfpathlineto{\pgfqpoint{0.954533in}{0.768140in}}%
\pgfpathlineto{\pgfqpoint{0.954829in}{0.768140in}}%
\pgfpathlineto{\pgfqpoint{0.955125in}{0.768140in}}%
\pgfpathlineto{\pgfqpoint{0.955421in}{0.768141in}}%
\pgfpathlineto{\pgfqpoint{0.955717in}{0.768141in}}%
\pgfpathlineto{\pgfqpoint{0.956013in}{0.768141in}}%
\pgfpathlineto{\pgfqpoint{0.956309in}{0.768142in}}%
\pgfpathlineto{\pgfqpoint{0.956605in}{0.768142in}}%
\pgfpathlineto{\pgfqpoint{0.956901in}{0.768143in}}%
\pgfpathlineto{\pgfqpoint{0.957197in}{0.768143in}}%
\pgfpathlineto{\pgfqpoint{0.957493in}{0.768143in}}%
\pgfpathlineto{\pgfqpoint{0.957789in}{0.768144in}}%
\pgfpathlineto{\pgfqpoint{0.958085in}{0.768144in}}%
\pgfpathlineto{\pgfqpoint{0.958381in}{0.768144in}}%
\pgfpathlineto{\pgfqpoint{0.958677in}{0.768145in}}%
\pgfpathlineto{\pgfqpoint{0.958973in}{0.768145in}}%
\pgfpathlineto{\pgfqpoint{0.959269in}{0.768146in}}%
\pgfpathlineto{\pgfqpoint{0.959565in}{0.768146in}}%
\pgfpathlineto{\pgfqpoint{0.959861in}{0.768146in}}%
\pgfpathlineto{\pgfqpoint{0.960157in}{0.768147in}}%
\pgfpathlineto{\pgfqpoint{0.960453in}{0.768147in}}%
\pgfpathlineto{\pgfqpoint{0.960749in}{0.768147in}}%
\pgfpathlineto{\pgfqpoint{0.961045in}{0.768148in}}%
\pgfpathlineto{\pgfqpoint{0.961341in}{0.768148in}}%
\pgfpathlineto{\pgfqpoint{0.961637in}{0.768148in}}%
\pgfpathlineto{\pgfqpoint{0.961933in}{0.768149in}}%
\pgfpathlineto{\pgfqpoint{0.962229in}{0.768149in}}%
\pgfpathlineto{\pgfqpoint{0.962525in}{0.768150in}}%
\pgfpathlineto{\pgfqpoint{0.962821in}{0.768150in}}%
\pgfpathlineto{\pgfqpoint{0.963117in}{0.768150in}}%
\pgfpathlineto{\pgfqpoint{0.963413in}{0.768151in}}%
\pgfpathlineto{\pgfqpoint{0.963709in}{0.768151in}}%
\pgfpathlineto{\pgfqpoint{0.964005in}{0.768151in}}%
\pgfpathlineto{\pgfqpoint{0.964301in}{0.768152in}}%
\pgfpathlineto{\pgfqpoint{0.964597in}{0.768152in}}%
\pgfpathlineto{\pgfqpoint{0.964893in}{0.768153in}}%
\pgfpathlineto{\pgfqpoint{0.965189in}{0.768153in}}%
\pgfpathlineto{\pgfqpoint{0.965485in}{0.768153in}}%
\pgfpathlineto{\pgfqpoint{0.965781in}{0.768153in}}%
\pgfpathlineto{\pgfqpoint{0.966077in}{0.768131in}}%
\pgfpathlineto{\pgfqpoint{0.966373in}{0.768125in}}%
\pgfpathlineto{\pgfqpoint{0.966669in}{0.768122in}}%
\pgfpathlineto{\pgfqpoint{0.966965in}{0.768119in}}%
\pgfpathlineto{\pgfqpoint{0.967261in}{0.768141in}}%
\pgfpathlineto{\pgfqpoint{0.967557in}{0.768797in}}%
\pgfpathlineto{\pgfqpoint{0.967853in}{0.769974in}}%
\pgfpathlineto{\pgfqpoint{0.968149in}{0.769985in}}%
\pgfpathlineto{\pgfqpoint{0.968445in}{0.769973in}}%
\pgfpathlineto{\pgfqpoint{0.968741in}{0.769976in}}%
\pgfpathlineto{\pgfqpoint{0.969037in}{0.769978in}}%
\pgfpathlineto{\pgfqpoint{0.969333in}{0.769981in}}%
\pgfpathlineto{\pgfqpoint{0.969629in}{0.769983in}}%
\pgfpathlineto{\pgfqpoint{0.969925in}{0.769985in}}%
\pgfpathlineto{\pgfqpoint{0.970221in}{0.769988in}}%
\pgfpathlineto{\pgfqpoint{0.970517in}{0.769990in}}%
\pgfpathlineto{\pgfqpoint{0.970813in}{0.769993in}}%
\pgfpathlineto{\pgfqpoint{0.971109in}{0.769995in}}%
\pgfpathlineto{\pgfqpoint{0.971405in}{0.769997in}}%
\pgfpathlineto{\pgfqpoint{0.971701in}{0.770000in}}%
\pgfpathlineto{\pgfqpoint{0.971997in}{0.770002in}}%
\pgfpathlineto{\pgfqpoint{0.972293in}{0.770004in}}%
\pgfpathlineto{\pgfqpoint{0.972589in}{0.770007in}}%
\pgfpathlineto{\pgfqpoint{0.972885in}{0.770009in}}%
\pgfpathlineto{\pgfqpoint{0.973181in}{0.770012in}}%
\pgfpathlineto{\pgfqpoint{0.973477in}{0.769768in}}%
\pgfpathlineto{\pgfqpoint{0.973773in}{0.769275in}}%
\pgfpathlineto{\pgfqpoint{0.974069in}{0.768781in}}%
\pgfpathlineto{\pgfqpoint{0.974365in}{0.769102in}}%
\pgfpathlineto{\pgfqpoint{0.974661in}{0.771594in}}%
\pgfpathlineto{\pgfqpoint{0.974957in}{0.773969in}}%
\pgfpathlineto{\pgfqpoint{0.975253in}{0.773968in}}%
\pgfpathlineto{\pgfqpoint{0.975549in}{0.773966in}}%
\pgfpathlineto{\pgfqpoint{0.975845in}{0.773965in}}%
\pgfpathlineto{\pgfqpoint{0.976141in}{0.773963in}}%
\pgfpathlineto{\pgfqpoint{0.976437in}{0.773962in}}%
\pgfpathlineto{\pgfqpoint{0.976733in}{0.773960in}}%
\pgfpathlineto{\pgfqpoint{0.977029in}{0.773959in}}%
\pgfpathlineto{\pgfqpoint{0.977325in}{0.773957in}}%
\pgfpathlineto{\pgfqpoint{0.977621in}{0.773956in}}%
\pgfpathlineto{\pgfqpoint{0.977917in}{0.773954in}}%
\pgfpathlineto{\pgfqpoint{0.978213in}{0.773953in}}%
\pgfpathlineto{\pgfqpoint{0.978509in}{0.773951in}}%
\pgfpathlineto{\pgfqpoint{0.978805in}{0.773950in}}%
\pgfpathlineto{\pgfqpoint{0.979101in}{0.773948in}}%
\pgfpathlineto{\pgfqpoint{0.979397in}{0.773947in}}%
\pgfpathlineto{\pgfqpoint{0.979693in}{0.773945in}}%
\pgfpathlineto{\pgfqpoint{0.979989in}{0.773944in}}%
\pgfpathlineto{\pgfqpoint{0.980285in}{0.773942in}}%
\pgfpathlineto{\pgfqpoint{0.980581in}{0.773941in}}%
\pgfpathlineto{\pgfqpoint{0.980877in}{0.773939in}}%
\pgfpathlineto{\pgfqpoint{0.981173in}{0.773938in}}%
\pgfpathlineto{\pgfqpoint{0.981469in}{0.773936in}}%
\pgfpathlineto{\pgfqpoint{0.981765in}{0.773935in}}%
\pgfpathlineto{\pgfqpoint{0.982061in}{0.773933in}}%
\pgfpathlineto{\pgfqpoint{0.982357in}{0.773932in}}%
\pgfpathlineto{\pgfqpoint{0.982653in}{0.773930in}}%
\pgfpathlineto{\pgfqpoint{0.982949in}{0.773929in}}%
\pgfpathlineto{\pgfqpoint{0.983245in}{0.773927in}}%
\pgfpathlineto{\pgfqpoint{0.983541in}{0.773926in}}%
\pgfpathlineto{\pgfqpoint{0.983837in}{0.773924in}}%
\pgfpathlineto{\pgfqpoint{0.984133in}{0.773923in}}%
\pgfpathlineto{\pgfqpoint{0.984429in}{0.773921in}}%
\pgfpathlineto{\pgfqpoint{0.984725in}{0.773920in}}%
\pgfpathlineto{\pgfqpoint{0.985021in}{0.773918in}}%
\pgfpathlineto{\pgfqpoint{0.985318in}{0.773917in}}%
\pgfpathlineto{\pgfqpoint{0.985614in}{0.773915in}}%
\pgfpathlineto{\pgfqpoint{0.985910in}{0.773914in}}%
\pgfpathlineto{\pgfqpoint{0.986206in}{0.773912in}}%
\pgfpathlineto{\pgfqpoint{0.986502in}{0.773911in}}%
\pgfpathlineto{\pgfqpoint{0.986798in}{0.773909in}}%
\pgfpathlineto{\pgfqpoint{0.987094in}{0.773908in}}%
\pgfpathlineto{\pgfqpoint{0.987390in}{0.773586in}}%
\pgfpathlineto{\pgfqpoint{0.987686in}{0.772781in}}%
\pgfpathlineto{\pgfqpoint{0.987982in}{0.771972in}}%
\pgfpathlineto{\pgfqpoint{0.988278in}{0.771163in}}%
\pgfpathlineto{\pgfqpoint{0.988574in}{0.770355in}}%
\pgfpathlineto{\pgfqpoint{0.988870in}{0.769546in}}%
\pgfpathlineto{\pgfqpoint{0.989166in}{0.768749in}}%
\pgfpathlineto{\pgfqpoint{0.989462in}{0.768644in}}%
\pgfpathlineto{\pgfqpoint{0.989758in}{0.769006in}}%
\pgfpathlineto{\pgfqpoint{0.990054in}{0.769368in}}%
\pgfpathlineto{\pgfqpoint{0.990350in}{0.769730in}}%
\pgfpathlineto{\pgfqpoint{0.990646in}{0.770092in}}%
\pgfpathlineto{\pgfqpoint{0.990942in}{0.770454in}}%
\pgfpathlineto{\pgfqpoint{0.991238in}{0.770816in}}%
\pgfpathlineto{\pgfqpoint{0.991534in}{0.771178in}}%
\pgfpathlineto{\pgfqpoint{0.991830in}{0.771540in}}%
\pgfpathlineto{\pgfqpoint{0.992126in}{0.771902in}}%
\pgfpathlineto{\pgfqpoint{0.992422in}{0.772264in}}%
\pgfpathlineto{\pgfqpoint{0.992718in}{0.772626in}}%
\pgfpathlineto{\pgfqpoint{0.993014in}{0.772988in}}%
\pgfpathlineto{\pgfqpoint{0.993310in}{0.773350in}}%
\pgfpathlineto{\pgfqpoint{0.993606in}{0.773125in}}%
\pgfpathlineto{\pgfqpoint{0.993902in}{0.773816in}}%
\pgfpathlineto{\pgfqpoint{0.994198in}{0.773825in}}%
\pgfpathlineto{\pgfqpoint{0.994494in}{0.773833in}}%
\pgfpathlineto{\pgfqpoint{0.994790in}{0.773504in}}%
\pgfpathlineto{\pgfqpoint{0.995086in}{0.773133in}}%
\pgfpathlineto{\pgfqpoint{0.995382in}{0.773015in}}%
\pgfpathlineto{\pgfqpoint{0.995678in}{0.772786in}}%
\pgfpathlineto{\pgfqpoint{0.995974in}{0.772665in}}%
\pgfpathlineto{\pgfqpoint{0.996270in}{0.772601in}}%
\pgfpathlineto{\pgfqpoint{0.996566in}{0.772555in}}%
\pgfpathlineto{\pgfqpoint{0.996862in}{0.772551in}}%
\pgfpathlineto{\pgfqpoint{0.997158in}{0.772548in}}%
\pgfpathlineto{\pgfqpoint{0.997454in}{0.772546in}}%
\pgfpathlineto{\pgfqpoint{0.997750in}{0.772544in}}%
\pgfpathlineto{\pgfqpoint{0.998046in}{0.772541in}}%
\pgfpathlineto{\pgfqpoint{0.998342in}{0.772539in}}%
\pgfpathlineto{\pgfqpoint{0.998638in}{0.772537in}}%
\pgfpathlineto{\pgfqpoint{0.998934in}{0.772534in}}%
\pgfpathlineto{\pgfqpoint{0.999230in}{0.772532in}}%
\pgfpathlineto{\pgfqpoint{0.999526in}{0.772530in}}%
\pgfpathlineto{\pgfqpoint{0.999822in}{0.772527in}}%
\pgfpathlineto{\pgfqpoint{1.000118in}{0.772525in}}%
\pgfpathlineto{\pgfqpoint{1.000414in}{0.772523in}}%
\pgfpathlineto{\pgfqpoint{1.000710in}{0.772521in}}%
\pgfpathlineto{\pgfqpoint{1.001006in}{0.772518in}}%
\pgfpathlineto{\pgfqpoint{1.001302in}{0.772516in}}%
\pgfpathlineto{\pgfqpoint{1.001598in}{0.772514in}}%
\pgfpathlineto{\pgfqpoint{1.001894in}{0.772511in}}%
\pgfpathlineto{\pgfqpoint{1.002190in}{0.772509in}}%
\pgfpathlineto{\pgfqpoint{1.002486in}{0.772507in}}%
\pgfpathlineto{\pgfqpoint{1.002782in}{0.772504in}}%
\pgfpathlineto{\pgfqpoint{1.003078in}{0.772502in}}%
\pgfpathlineto{\pgfqpoint{1.003374in}{0.772500in}}%
\pgfpathlineto{\pgfqpoint{1.003670in}{0.772497in}}%
\pgfpathlineto{\pgfqpoint{1.003966in}{0.772495in}}%
\pgfpathlineto{\pgfqpoint{1.004262in}{0.772493in}}%
\pgfpathlineto{\pgfqpoint{1.004558in}{0.772490in}}%
\pgfpathlineto{\pgfqpoint{1.004854in}{0.772488in}}%
\pgfpathlineto{\pgfqpoint{1.005150in}{0.772486in}}%
\pgfpathlineto{\pgfqpoint{1.005446in}{0.772483in}}%
\pgfpathlineto{\pgfqpoint{1.005742in}{0.772481in}}%
\pgfpathlineto{\pgfqpoint{1.006038in}{0.772479in}}%
\pgfpathlineto{\pgfqpoint{1.006334in}{0.772476in}}%
\pgfpathlineto{\pgfqpoint{1.006630in}{0.772474in}}%
\pgfpathlineto{\pgfqpoint{1.006926in}{0.772472in}}%
\pgfpathlineto{\pgfqpoint{1.007222in}{0.772469in}}%
\pgfpathlineto{\pgfqpoint{1.007518in}{0.772467in}}%
\pgfpathlineto{\pgfqpoint{1.007814in}{0.772465in}}%
\pgfpathlineto{\pgfqpoint{1.008110in}{0.772462in}}%
\pgfpathlineto{\pgfqpoint{1.008406in}{0.772460in}}%
\pgfpathlineto{\pgfqpoint{1.008702in}{0.772458in}}%
\pgfpathlineto{\pgfqpoint{1.008998in}{0.772455in}}%
\pgfpathlineto{\pgfqpoint{1.009294in}{0.772453in}}%
\pgfpathlineto{\pgfqpoint{1.009590in}{0.772451in}}%
\pgfpathlineto{\pgfqpoint{1.009886in}{0.772448in}}%
\pgfpathlineto{\pgfqpoint{1.010182in}{0.772446in}}%
\pgfpathlineto{\pgfqpoint{1.010478in}{0.772444in}}%
\pgfpathlineto{\pgfqpoint{1.010774in}{0.772441in}}%
\pgfpathlineto{\pgfqpoint{1.011070in}{0.772439in}}%
\pgfpathlineto{\pgfqpoint{1.011366in}{0.772437in}}%
\pgfpathlineto{\pgfqpoint{1.011662in}{0.772434in}}%
\pgfpathlineto{\pgfqpoint{1.011958in}{0.772432in}}%
\pgfpathlineto{\pgfqpoint{1.012254in}{0.772430in}}%
\pgfpathlineto{\pgfqpoint{1.012550in}{0.772427in}}%
\pgfpathlineto{\pgfqpoint{1.012846in}{0.772425in}}%
\pgfpathlineto{\pgfqpoint{1.013142in}{0.772423in}}%
\pgfpathlineto{\pgfqpoint{1.013438in}{0.772420in}}%
\pgfpathlineto{\pgfqpoint{1.013734in}{0.772418in}}%
\pgfpathlineto{\pgfqpoint{1.014030in}{0.772416in}}%
\pgfpathlineto{\pgfqpoint{1.014326in}{0.772413in}}%
\pgfpathlineto{\pgfqpoint{1.014622in}{0.772411in}}%
\pgfpathlineto{\pgfqpoint{1.014918in}{0.772417in}}%
\pgfpathlineto{\pgfqpoint{1.015214in}{0.772425in}}%
\pgfpathlineto{\pgfqpoint{1.015510in}{0.772433in}}%
\pgfpathlineto{\pgfqpoint{1.015806in}{0.772442in}}%
\pgfpathlineto{\pgfqpoint{1.016102in}{0.772450in}}%
\pgfpathlineto{\pgfqpoint{1.016398in}{0.772564in}}%
\pgfpathlineto{\pgfqpoint{1.016694in}{0.772676in}}%
\pgfpathlineto{\pgfqpoint{1.016990in}{0.772813in}}%
\pgfpathlineto{\pgfqpoint{1.017286in}{0.772635in}}%
\pgfpathlineto{\pgfqpoint{1.017582in}{0.772528in}}%
\pgfpathlineto{\pgfqpoint{1.017878in}{0.772502in}}%
\pgfpathlineto{\pgfqpoint{1.018174in}{0.772502in}}%
\pgfpathlineto{\pgfqpoint{1.018470in}{0.772502in}}%
\pgfpathlineto{\pgfqpoint{1.018766in}{0.772501in}}%
\pgfpathlineto{\pgfqpoint{1.019062in}{0.772501in}}%
\pgfpathlineto{\pgfqpoint{1.019358in}{0.772500in}}%
\pgfpathlineto{\pgfqpoint{1.019654in}{0.772500in}}%
\pgfpathlineto{\pgfqpoint{1.019950in}{0.772500in}}%
\pgfpathlineto{\pgfqpoint{1.020246in}{0.772499in}}%
\pgfpathlineto{\pgfqpoint{1.020542in}{0.772499in}}%
\pgfpathlineto{\pgfqpoint{1.020838in}{0.772499in}}%
\pgfpathlineto{\pgfqpoint{1.021134in}{0.772498in}}%
\pgfpathlineto{\pgfqpoint{1.021430in}{0.772498in}}%
\pgfpathlineto{\pgfqpoint{1.021726in}{0.772498in}}%
\pgfpathlineto{\pgfqpoint{1.022022in}{0.772497in}}%
\pgfpathlineto{\pgfqpoint{1.022318in}{0.772494in}}%
\pgfpathlineto{\pgfqpoint{1.022614in}{0.772228in}}%
\pgfpathlineto{\pgfqpoint{1.022910in}{0.771797in}}%
\pgfpathlineto{\pgfqpoint{1.023206in}{0.771571in}}%
\pgfpathlineto{\pgfqpoint{1.023502in}{0.771488in}}%
\pgfpathlineto{\pgfqpoint{1.023798in}{0.771477in}}%
\pgfpathlineto{\pgfqpoint{1.024094in}{0.771466in}}%
\pgfpathlineto{\pgfqpoint{1.024390in}{0.771456in}}%
\pgfpathlineto{\pgfqpoint{1.024686in}{0.771445in}}%
\pgfpathlineto{\pgfqpoint{1.024982in}{0.771434in}}%
\pgfpathlineto{\pgfqpoint{1.025278in}{0.771423in}}%
\pgfpathlineto{\pgfqpoint{1.025574in}{0.771412in}}%
\pgfpathlineto{\pgfqpoint{1.025870in}{0.771401in}}%
\pgfpathlineto{\pgfqpoint{1.026166in}{0.771390in}}%
\pgfpathlineto{\pgfqpoint{1.026462in}{0.771379in}}%
\pgfpathlineto{\pgfqpoint{1.026758in}{0.771369in}}%
\pgfpathlineto{\pgfqpoint{1.027054in}{0.771358in}}%
\pgfpathlineto{\pgfqpoint{1.027350in}{0.771347in}}%
\pgfpathlineto{\pgfqpoint{1.027646in}{0.771336in}}%
\pgfpathlineto{\pgfqpoint{1.027942in}{0.771325in}}%
\pgfpathlineto{\pgfqpoint{1.028238in}{0.771314in}}%
\pgfpathlineto{\pgfqpoint{1.028534in}{0.771303in}}%
\pgfpathlineto{\pgfqpoint{1.028830in}{0.771292in}}%
\pgfpathlineto{\pgfqpoint{1.029126in}{0.771282in}}%
\pgfpathlineto{\pgfqpoint{1.029422in}{0.771271in}}%
\pgfpathlineto{\pgfqpoint{1.029718in}{0.771260in}}%
\pgfpathlineto{\pgfqpoint{1.030014in}{0.771198in}}%
\pgfpathlineto{\pgfqpoint{1.030310in}{0.771125in}}%
\pgfpathlineto{\pgfqpoint{1.030606in}{0.771115in}}%
\pgfpathlineto{\pgfqpoint{1.030902in}{0.770670in}}%
\pgfpathlineto{\pgfqpoint{1.031198in}{0.770507in}}%
\pgfpathlineto{\pgfqpoint{1.031494in}{0.770466in}}%
\pgfpathlineto{\pgfqpoint{1.031790in}{0.770480in}}%
\pgfpathlineto{\pgfqpoint{1.032086in}{0.770500in}}%
\pgfpathlineto{\pgfqpoint{1.032382in}{0.770593in}}%
\pgfpathlineto{\pgfqpoint{1.032678in}{0.770711in}}%
\pgfpathlineto{\pgfqpoint{1.032974in}{0.770799in}}%
\pgfpathlineto{\pgfqpoint{1.033270in}{0.770588in}}%
\pgfpathlineto{\pgfqpoint{1.033566in}{0.770583in}}%
\pgfpathlineto{\pgfqpoint{1.033862in}{0.770577in}}%
\pgfpathlineto{\pgfqpoint{1.034158in}{0.770571in}}%
\pgfpathlineto{\pgfqpoint{1.034454in}{0.770565in}}%
\pgfpathlineto{\pgfqpoint{1.034750in}{0.770560in}}%
\pgfpathlineto{\pgfqpoint{1.035046in}{0.770554in}}%
\pgfpathlineto{\pgfqpoint{1.035342in}{0.770548in}}%
\pgfpathlineto{\pgfqpoint{1.035638in}{0.770543in}}%
\pgfpathlineto{\pgfqpoint{1.035934in}{0.770537in}}%
\pgfpathlineto{\pgfqpoint{1.036230in}{0.770531in}}%
\pgfpathlineto{\pgfqpoint{1.036526in}{0.770525in}}%
\pgfpathlineto{\pgfqpoint{1.036822in}{0.770520in}}%
\pgfpathlineto{\pgfqpoint{1.037118in}{0.770514in}}%
\pgfpathlineto{\pgfqpoint{1.037414in}{0.770508in}}%
\pgfpathlineto{\pgfqpoint{1.037710in}{0.770503in}}%
\pgfpathlineto{\pgfqpoint{1.038006in}{0.770519in}}%
\pgfpathlineto{\pgfqpoint{1.038302in}{0.770775in}}%
\pgfpathlineto{\pgfqpoint{1.038598in}{0.770559in}}%
\pgfpathlineto{\pgfqpoint{1.038894in}{0.770543in}}%
\pgfpathlineto{\pgfqpoint{1.039190in}{0.770597in}}%
\pgfpathlineto{\pgfqpoint{1.039486in}{0.770652in}}%
\pgfpathlineto{\pgfqpoint{1.039782in}{0.770706in}}%
\pgfpathlineto{\pgfqpoint{1.040078in}{0.770761in}}%
\pgfpathlineto{\pgfqpoint{1.040374in}{0.770815in}}%
\pgfpathlineto{\pgfqpoint{1.040670in}{0.770869in}}%
\pgfpathlineto{\pgfqpoint{1.040966in}{0.770924in}}%
\pgfpathlineto{\pgfqpoint{1.041262in}{0.770978in}}%
\pgfpathlineto{\pgfqpoint{1.041558in}{0.771033in}}%
\pgfpathlineto{\pgfqpoint{1.041854in}{0.771087in}}%
\pgfpathlineto{\pgfqpoint{1.042150in}{0.771142in}}%
\pgfpathlineto{\pgfqpoint{1.042446in}{0.771196in}}%
\pgfpathlineto{\pgfqpoint{1.042742in}{0.771251in}}%
\pgfpathlineto{\pgfqpoint{1.043038in}{0.771305in}}%
\pgfpathlineto{\pgfqpoint{1.043334in}{0.771360in}}%
\pgfpathlineto{\pgfqpoint{1.043630in}{0.771406in}}%
\pgfpathlineto{\pgfqpoint{1.043926in}{0.771261in}}%
\pgfpathlineto{\pgfqpoint{1.044222in}{0.771035in}}%
\pgfpathlineto{\pgfqpoint{1.044518in}{0.770809in}}%
\pgfpathlineto{\pgfqpoint{1.044814in}{0.770583in}}%
\pgfpathlineto{\pgfqpoint{1.045110in}{0.770908in}}%
\pgfpathlineto{\pgfqpoint{1.045406in}{0.771524in}}%
\pgfpathlineto{\pgfqpoint{1.045702in}{0.771549in}}%
\pgfpathlineto{\pgfqpoint{1.045998in}{0.771597in}}%
\pgfpathlineto{\pgfqpoint{1.046294in}{0.771681in}}%
\pgfpathlineto{\pgfqpoint{1.046590in}{0.771681in}}%
\pgfpathlineto{\pgfqpoint{1.046886in}{0.771682in}}%
\pgfpathlineto{\pgfqpoint{1.047182in}{0.771683in}}%
\pgfpathlineto{\pgfqpoint{1.047478in}{0.771683in}}%
\pgfpathlineto{\pgfqpoint{1.047774in}{0.771684in}}%
\pgfpathlineto{\pgfqpoint{1.048070in}{0.771685in}}%
\pgfpathlineto{\pgfqpoint{1.048366in}{0.771685in}}%
\pgfpathlineto{\pgfqpoint{1.048662in}{0.771686in}}%
\pgfpathlineto{\pgfqpoint{1.048958in}{0.771687in}}%
\pgfpathlineto{\pgfqpoint{1.049254in}{0.771687in}}%
\pgfpathlineto{\pgfqpoint{1.049550in}{0.771688in}}%
\pgfpathlineto{\pgfqpoint{1.049846in}{0.771689in}}%
\pgfpathlineto{\pgfqpoint{1.050142in}{0.771690in}}%
\pgfpathlineto{\pgfqpoint{1.050438in}{0.771690in}}%
\pgfpathlineto{\pgfqpoint{1.050734in}{0.771691in}}%
\pgfpathlineto{\pgfqpoint{1.051030in}{0.771692in}}%
\pgfpathlineto{\pgfqpoint{1.051326in}{0.771692in}}%
\pgfpathlineto{\pgfqpoint{1.051622in}{0.771693in}}%
\pgfpathlineto{\pgfqpoint{1.051918in}{0.771694in}}%
\pgfpathlineto{\pgfqpoint{1.052214in}{0.771694in}}%
\pgfpathlineto{\pgfqpoint{1.052511in}{0.771695in}}%
\pgfpathlineto{\pgfqpoint{1.052807in}{0.771696in}}%
\pgfpathlineto{\pgfqpoint{1.053103in}{0.771696in}}%
\pgfpathlineto{\pgfqpoint{1.053399in}{0.771697in}}%
\pgfpathlineto{\pgfqpoint{1.053695in}{0.771698in}}%
\pgfpathlineto{\pgfqpoint{1.053991in}{0.771698in}}%
\pgfpathlineto{\pgfqpoint{1.054287in}{0.771699in}}%
\pgfpathlineto{\pgfqpoint{1.054583in}{0.771700in}}%
\pgfpathlineto{\pgfqpoint{1.054879in}{0.771700in}}%
\pgfpathlineto{\pgfqpoint{1.055175in}{0.771701in}}%
\pgfpathlineto{\pgfqpoint{1.055471in}{0.771702in}}%
\pgfpathlineto{\pgfqpoint{1.055767in}{0.771703in}}%
\pgfpathlineto{\pgfqpoint{1.056063in}{0.771703in}}%
\pgfpathlineto{\pgfqpoint{1.056359in}{0.771704in}}%
\pgfpathlineto{\pgfqpoint{1.056655in}{0.771705in}}%
\pgfpathlineto{\pgfqpoint{1.056951in}{0.771705in}}%
\pgfpathlineto{\pgfqpoint{1.057247in}{0.771706in}}%
\pgfpathlineto{\pgfqpoint{1.057543in}{0.771707in}}%
\pgfpathlineto{\pgfqpoint{1.057839in}{0.771707in}}%
\pgfpathlineto{\pgfqpoint{1.058135in}{0.771708in}}%
\pgfpathlineto{\pgfqpoint{1.058431in}{0.771709in}}%
\pgfpathlineto{\pgfqpoint{1.058727in}{0.771709in}}%
\pgfpathlineto{\pgfqpoint{1.059023in}{0.771710in}}%
\pgfpathlineto{\pgfqpoint{1.059319in}{0.771711in}}%
\pgfpathlineto{\pgfqpoint{1.059615in}{0.771711in}}%
\pgfpathlineto{\pgfqpoint{1.059911in}{0.771712in}}%
\pgfpathlineto{\pgfqpoint{1.060207in}{0.771713in}}%
\pgfpathlineto{\pgfqpoint{1.060503in}{0.771713in}}%
\pgfpathlineto{\pgfqpoint{1.060799in}{0.771714in}}%
\pgfpathlineto{\pgfqpoint{1.061095in}{0.771715in}}%
\pgfpathlineto{\pgfqpoint{1.061391in}{0.771716in}}%
\pgfpathlineto{\pgfqpoint{1.061687in}{0.771716in}}%
\pgfpathlineto{\pgfqpoint{1.061983in}{0.771717in}}%
\pgfpathlineto{\pgfqpoint{1.062279in}{0.771718in}}%
\pgfpathlineto{\pgfqpoint{1.062575in}{0.771718in}}%
\pgfpathlineto{\pgfqpoint{1.062871in}{0.771719in}}%
\pgfpathlineto{\pgfqpoint{1.063167in}{0.771720in}}%
\pgfpathlineto{\pgfqpoint{1.063463in}{0.771720in}}%
\pgfpathlineto{\pgfqpoint{1.063759in}{0.771721in}}%
\pgfpathlineto{\pgfqpoint{1.064055in}{0.771722in}}%
\pgfpathlineto{\pgfqpoint{1.064351in}{0.771722in}}%
\pgfpathlineto{\pgfqpoint{1.064647in}{0.771723in}}%
\pgfpathlineto{\pgfqpoint{1.064943in}{0.771724in}}%
\pgfpathlineto{\pgfqpoint{1.065239in}{0.774275in}}%
\pgfpathlineto{\pgfqpoint{1.065535in}{0.776102in}}%
\pgfpathlineto{\pgfqpoint{1.065831in}{0.776113in}}%
\pgfpathlineto{\pgfqpoint{1.066127in}{0.776125in}}%
\pgfpathlineto{\pgfqpoint{1.066423in}{0.776137in}}%
\pgfpathlineto{\pgfqpoint{1.066719in}{0.776149in}}%
\pgfpathlineto{\pgfqpoint{1.067015in}{0.776161in}}%
\pgfpathlineto{\pgfqpoint{1.067311in}{0.776173in}}%
\pgfpathlineto{\pgfqpoint{1.067607in}{0.776175in}}%
\pgfpathlineto{\pgfqpoint{1.067903in}{0.776171in}}%
\pgfpathlineto{\pgfqpoint{1.068199in}{0.776168in}}%
\pgfpathlineto{\pgfqpoint{1.068495in}{0.776164in}}%
\pgfpathlineto{\pgfqpoint{1.068791in}{0.776161in}}%
\pgfpathlineto{\pgfqpoint{1.069087in}{0.776157in}}%
\pgfpathlineto{\pgfqpoint{1.069383in}{0.776154in}}%
\pgfpathlineto{\pgfqpoint{1.069679in}{0.776150in}}%
\pgfpathlineto{\pgfqpoint{1.069975in}{0.776147in}}%
\pgfpathlineto{\pgfqpoint{1.070271in}{0.776143in}}%
\pgfpathlineto{\pgfqpoint{1.070567in}{0.776140in}}%
\pgfpathlineto{\pgfqpoint{1.070863in}{0.776136in}}%
\pgfpathlineto{\pgfqpoint{1.071159in}{0.776133in}}%
\pgfpathlineto{\pgfqpoint{1.071455in}{0.776129in}}%
\pgfpathlineto{\pgfqpoint{1.071751in}{0.776127in}}%
\pgfpathlineto{\pgfqpoint{1.072047in}{0.776256in}}%
\pgfpathlineto{\pgfqpoint{1.072343in}{0.776473in}}%
\pgfpathlineto{\pgfqpoint{1.072639in}{0.776503in}}%
\pgfpathlineto{\pgfqpoint{1.072935in}{0.776603in}}%
\pgfpathlineto{\pgfqpoint{1.073231in}{0.776641in}}%
\pgfpathlineto{\pgfqpoint{1.073527in}{0.776648in}}%
\pgfpathlineto{\pgfqpoint{1.073823in}{0.776655in}}%
\pgfpathlineto{\pgfqpoint{1.074119in}{0.776663in}}%
\pgfpathlineto{\pgfqpoint{1.074415in}{0.776670in}}%
\pgfpathlineto{\pgfqpoint{1.074711in}{0.776677in}}%
\pgfpathlineto{\pgfqpoint{1.075007in}{0.776683in}}%
\pgfpathlineto{\pgfqpoint{1.075303in}{0.776689in}}%
\pgfpathlineto{\pgfqpoint{1.075599in}{0.776695in}}%
\pgfpathlineto{\pgfqpoint{1.075895in}{0.776701in}}%
\pgfpathlineto{\pgfqpoint{1.076191in}{0.776707in}}%
\pgfpathlineto{\pgfqpoint{1.076487in}{0.776713in}}%
\pgfpathlineto{\pgfqpoint{1.076783in}{0.776719in}}%
\pgfpathlineto{\pgfqpoint{1.077079in}{0.776725in}}%
\pgfpathlineto{\pgfqpoint{1.077375in}{0.776731in}}%
\pgfpathlineto{\pgfqpoint{1.077671in}{0.776737in}}%
\pgfpathlineto{\pgfqpoint{1.077967in}{0.776743in}}%
\pgfpathlineto{\pgfqpoint{1.078263in}{0.776749in}}%
\pgfpathlineto{\pgfqpoint{1.078559in}{0.776746in}}%
\pgfpathlineto{\pgfqpoint{1.078855in}{0.776705in}}%
\pgfpathlineto{\pgfqpoint{1.079151in}{0.776659in}}%
\pgfpathlineto{\pgfqpoint{1.079447in}{0.776613in}}%
\pgfpathlineto{\pgfqpoint{1.079743in}{0.776682in}}%
\pgfpathlineto{\pgfqpoint{1.080039in}{0.776914in}}%
\pgfpathlineto{\pgfqpoint{1.080335in}{0.777148in}}%
\pgfpathlineto{\pgfqpoint{1.080631in}{0.777381in}}%
\pgfpathlineto{\pgfqpoint{1.080927in}{0.777615in}}%
\pgfpathlineto{\pgfqpoint{1.081223in}{0.777821in}}%
\pgfpathlineto{\pgfqpoint{1.081519in}{0.777864in}}%
\pgfpathlineto{\pgfqpoint{1.081815in}{0.777824in}}%
\pgfpathlineto{\pgfqpoint{1.082111in}{0.777783in}}%
\pgfpathlineto{\pgfqpoint{1.082407in}{0.777742in}}%
\pgfpathlineto{\pgfqpoint{1.082703in}{0.777701in}}%
\pgfpathlineto{\pgfqpoint{1.082999in}{0.777661in}}%
\pgfpathlineto{\pgfqpoint{1.083295in}{0.777620in}}%
\pgfpathlineto{\pgfqpoint{1.083591in}{0.777579in}}%
\pgfpathlineto{\pgfqpoint{1.083887in}{0.777539in}}%
\pgfpathlineto{\pgfqpoint{1.084183in}{0.777498in}}%
\pgfpathlineto{\pgfqpoint{1.084479in}{0.777457in}}%
\pgfpathlineto{\pgfqpoint{1.084775in}{0.777417in}}%
\pgfpathlineto{\pgfqpoint{1.085071in}{0.777376in}}%
\pgfpathlineto{\pgfqpoint{1.085367in}{0.777335in}}%
\pgfpathlineto{\pgfqpoint{1.085663in}{0.777294in}}%
\pgfpathlineto{\pgfqpoint{1.085959in}{0.777254in}}%
\pgfpathlineto{\pgfqpoint{1.086255in}{0.777213in}}%
\pgfpathlineto{\pgfqpoint{1.086551in}{0.777575in}}%
\pgfpathlineto{\pgfqpoint{1.086847in}{0.778346in}}%
\pgfpathlineto{\pgfqpoint{1.087143in}{0.776137in}}%
\pgfpathlineto{\pgfqpoint{1.087439in}{0.771433in}}%
\pgfpathlineto{\pgfqpoint{1.087735in}{0.774810in}}%
\pgfpathlineto{\pgfqpoint{1.088031in}{0.779273in}}%
\pgfpathlineto{\pgfqpoint{1.088327in}{0.779260in}}%
\pgfpathlineto{\pgfqpoint{1.088623in}{0.779261in}}%
\pgfpathlineto{\pgfqpoint{1.088919in}{0.779272in}}%
\pgfpathlineto{\pgfqpoint{1.089215in}{0.779284in}}%
\pgfpathlineto{\pgfqpoint{1.089511in}{0.779295in}}%
\pgfpathlineto{\pgfqpoint{1.089807in}{0.779307in}}%
\pgfpathlineto{\pgfqpoint{1.090103in}{0.779319in}}%
\pgfpathlineto{\pgfqpoint{1.090399in}{0.779330in}}%
\pgfpathlineto{\pgfqpoint{1.090695in}{0.779342in}}%
\pgfpathlineto{\pgfqpoint{1.090991in}{0.779353in}}%
\pgfpathlineto{\pgfqpoint{1.091287in}{0.779365in}}%
\pgfpathlineto{\pgfqpoint{1.091583in}{0.779376in}}%
\pgfpathlineto{\pgfqpoint{1.091879in}{0.779388in}}%
\pgfpathlineto{\pgfqpoint{1.092175in}{0.779399in}}%
\pgfpathlineto{\pgfqpoint{1.092471in}{0.779411in}}%
\pgfpathlineto{\pgfqpoint{1.092767in}{0.779423in}}%
\pgfpathlineto{\pgfqpoint{1.093063in}{0.779434in}}%
\pgfpathlineto{\pgfqpoint{1.093359in}{0.779372in}}%
\pgfpathlineto{\pgfqpoint{1.093655in}{0.777940in}}%
\pgfpathlineto{\pgfqpoint{1.093951in}{0.775993in}}%
\pgfpathlineto{\pgfqpoint{1.094247in}{0.774046in}}%
\pgfpathlineto{\pgfqpoint{1.094543in}{0.772099in}}%
\pgfpathlineto{\pgfqpoint{1.094839in}{0.772060in}}%
\pgfpathlineto{\pgfqpoint{1.095135in}{0.775781in}}%
\pgfpathlineto{\pgfqpoint{1.095431in}{0.779248in}}%
\pgfpathlineto{\pgfqpoint{1.095727in}{0.779846in}}%
\pgfpathlineto{\pgfqpoint{1.096023in}{0.779846in}}%
\pgfpathlineto{\pgfqpoint{1.096319in}{0.779845in}}%
\pgfpathlineto{\pgfqpoint{1.096615in}{0.779845in}}%
\pgfpathlineto{\pgfqpoint{1.096911in}{0.779844in}}%
\pgfpathlineto{\pgfqpoint{1.097207in}{0.779844in}}%
\pgfpathlineto{\pgfqpoint{1.097503in}{0.779843in}}%
\pgfpathlineto{\pgfqpoint{1.097799in}{0.779842in}}%
\pgfpathlineto{\pgfqpoint{1.098095in}{0.779842in}}%
\pgfpathlineto{\pgfqpoint{1.098391in}{0.779841in}}%
\pgfpathlineto{\pgfqpoint{1.098687in}{0.779841in}}%
\pgfpathlineto{\pgfqpoint{1.098983in}{0.779840in}}%
\pgfpathlineto{\pgfqpoint{1.099279in}{0.779840in}}%
\pgfpathlineto{\pgfqpoint{1.099575in}{0.779839in}}%
\pgfpathlineto{\pgfqpoint{1.099871in}{0.779839in}}%
\pgfpathlineto{\pgfqpoint{1.100167in}{0.779838in}}%
\pgfpathlineto{\pgfqpoint{1.100463in}{0.779837in}}%
\pgfpathlineto{\pgfqpoint{1.100759in}{0.779837in}}%
\pgfpathlineto{\pgfqpoint{1.101055in}{0.779836in}}%
\pgfpathlineto{\pgfqpoint{1.101351in}{0.779836in}}%
\pgfpathlineto{\pgfqpoint{1.101647in}{0.779835in}}%
\pgfpathlineto{\pgfqpoint{1.101943in}{0.779835in}}%
\pgfpathlineto{\pgfqpoint{1.102239in}{0.779834in}}%
\pgfpathlineto{\pgfqpoint{1.102535in}{0.779833in}}%
\pgfpathlineto{\pgfqpoint{1.102831in}{0.779833in}}%
\pgfpathlineto{\pgfqpoint{1.103127in}{0.779832in}}%
\pgfpathlineto{\pgfqpoint{1.103423in}{0.779832in}}%
\pgfpathlineto{\pgfqpoint{1.103719in}{0.779831in}}%
\pgfpathlineto{\pgfqpoint{1.104015in}{0.779831in}}%
\pgfpathlineto{\pgfqpoint{1.104311in}{0.779830in}}%
\pgfpathlineto{\pgfqpoint{1.104607in}{0.779829in}}%
\pgfpathlineto{\pgfqpoint{1.104903in}{0.779829in}}%
\pgfpathlineto{\pgfqpoint{1.105199in}{0.779828in}}%
\pgfpathlineto{\pgfqpoint{1.105495in}{0.779828in}}%
\pgfpathlineto{\pgfqpoint{1.105791in}{0.779827in}}%
\pgfpathlineto{\pgfqpoint{1.106087in}{0.779827in}}%
\pgfpathlineto{\pgfqpoint{1.106383in}{0.779826in}}%
\pgfpathlineto{\pgfqpoint{1.106679in}{0.779825in}}%
\pgfpathlineto{\pgfqpoint{1.106975in}{0.779825in}}%
\pgfpathlineto{\pgfqpoint{1.107271in}{0.779824in}}%
\pgfpathlineto{\pgfqpoint{1.107567in}{0.779824in}}%
\pgfpathlineto{\pgfqpoint{1.107863in}{0.779823in}}%
\pgfpathlineto{\pgfqpoint{1.108159in}{0.779823in}}%
\pgfpathlineto{\pgfqpoint{1.108455in}{0.779822in}}%
\pgfpathlineto{\pgfqpoint{1.108751in}{0.779821in}}%
\pgfpathlineto{\pgfqpoint{1.109047in}{0.779821in}}%
\pgfpathlineto{\pgfqpoint{1.109343in}{0.779820in}}%
\pgfpathlineto{\pgfqpoint{1.109639in}{0.779820in}}%
\pgfpathlineto{\pgfqpoint{1.109935in}{0.779819in}}%
\pgfpathlineto{\pgfqpoint{1.110231in}{0.779819in}}%
\pgfpathlineto{\pgfqpoint{1.110527in}{0.779818in}}%
\pgfpathlineto{\pgfqpoint{1.110823in}{0.779817in}}%
\pgfpathlineto{\pgfqpoint{1.111119in}{0.779817in}}%
\pgfpathlineto{\pgfqpoint{1.111415in}{0.779816in}}%
\pgfpathlineto{\pgfqpoint{1.111711in}{0.779816in}}%
\pgfpathlineto{\pgfqpoint{1.112007in}{0.779815in}}%
\pgfpathlineto{\pgfqpoint{1.112303in}{0.779815in}}%
\pgfpathlineto{\pgfqpoint{1.112599in}{0.779814in}}%
\pgfpathlineto{\pgfqpoint{1.112895in}{0.779814in}}%
\pgfpathlineto{\pgfqpoint{1.113191in}{0.779813in}}%
\pgfpathlineto{\pgfqpoint{1.113487in}{0.779812in}}%
\pgfpathlineto{\pgfqpoint{1.113783in}{0.779812in}}%
\pgfpathlineto{\pgfqpoint{1.114079in}{0.779811in}}%
\pgfpathlineto{\pgfqpoint{1.114375in}{0.779811in}}%
\pgfpathlineto{\pgfqpoint{1.114671in}{0.779810in}}%
\pgfpathlineto{\pgfqpoint{1.114967in}{0.779810in}}%
\pgfpathlineto{\pgfqpoint{1.115263in}{0.779809in}}%
\pgfpathlineto{\pgfqpoint{1.115559in}{0.779808in}}%
\pgfpathlineto{\pgfqpoint{1.115855in}{0.779808in}}%
\pgfpathlineto{\pgfqpoint{1.116151in}{0.779807in}}%
\pgfpathlineto{\pgfqpoint{1.116447in}{0.779807in}}%
\pgfpathlineto{\pgfqpoint{1.116743in}{0.779806in}}%
\pgfpathlineto{\pgfqpoint{1.117039in}{0.779806in}}%
\pgfpathlineto{\pgfqpoint{1.117335in}{0.779805in}}%
\pgfpathlineto{\pgfqpoint{1.117631in}{0.779804in}}%
\pgfpathlineto{\pgfqpoint{1.117927in}{0.779804in}}%
\pgfpathlineto{\pgfqpoint{1.118223in}{0.779803in}}%
\pgfpathlineto{\pgfqpoint{1.118519in}{0.779803in}}%
\pgfpathlineto{\pgfqpoint{1.118815in}{0.779802in}}%
\pgfpathlineto{\pgfqpoint{1.119111in}{0.779802in}}%
\pgfpathlineto{\pgfqpoint{1.119407in}{0.779801in}}%
\pgfpathlineto{\pgfqpoint{1.119703in}{0.779800in}}%
\pgfpathlineto{\pgfqpoint{1.120000in}{0.779800in}}%
\pgfpathlineto{\pgfqpoint{1.120296in}{0.779799in}}%
\pgfpathlineto{\pgfqpoint{1.120592in}{0.779799in}}%
\pgfpathlineto{\pgfqpoint{1.120888in}{0.779798in}}%
\pgfpathlineto{\pgfqpoint{1.121184in}{0.779798in}}%
\pgfpathlineto{\pgfqpoint{1.121480in}{0.779798in}}%
\pgfpathlineto{\pgfqpoint{1.121776in}{0.779806in}}%
\pgfpathlineto{\pgfqpoint{1.122072in}{0.779814in}}%
\pgfpathlineto{\pgfqpoint{1.122368in}{0.779823in}}%
\pgfpathlineto{\pgfqpoint{1.122664in}{0.779760in}}%
\pgfpathlineto{\pgfqpoint{1.122960in}{0.779286in}}%
\pgfpathlineto{\pgfqpoint{1.123256in}{0.778837in}}%
\pgfpathlineto{\pgfqpoint{1.123552in}{0.778834in}}%
\pgfpathlineto{\pgfqpoint{1.123848in}{0.778829in}}%
\pgfpathlineto{\pgfqpoint{1.124144in}{0.778824in}}%
\pgfpathlineto{\pgfqpoint{1.124440in}{0.778819in}}%
\pgfpathlineto{\pgfqpoint{1.124736in}{0.778814in}}%
\pgfpathlineto{\pgfqpoint{1.125032in}{0.778809in}}%
\pgfpathlineto{\pgfqpoint{1.125328in}{0.778804in}}%
\pgfpathlineto{\pgfqpoint{1.125624in}{0.778799in}}%
\pgfpathlineto{\pgfqpoint{1.125920in}{0.778794in}}%
\pgfpathlineto{\pgfqpoint{1.126216in}{0.778789in}}%
\pgfpathlineto{\pgfqpoint{1.126512in}{0.778784in}}%
\pgfpathlineto{\pgfqpoint{1.126808in}{0.778779in}}%
\pgfpathlineto{\pgfqpoint{1.127104in}{0.778774in}}%
\pgfpathlineto{\pgfqpoint{1.127400in}{0.778769in}}%
\pgfpathlineto{\pgfqpoint{1.127696in}{0.778764in}}%
\pgfpathlineto{\pgfqpoint{1.127992in}{0.778759in}}%
\pgfpathlineto{\pgfqpoint{1.128288in}{0.778754in}}%
\pgfpathlineto{\pgfqpoint{1.128584in}{0.778749in}}%
\pgfpathlineto{\pgfqpoint{1.128880in}{0.778744in}}%
\pgfpathlineto{\pgfqpoint{1.129176in}{0.778739in}}%
\pgfpathlineto{\pgfqpoint{1.129472in}{0.778734in}}%
\pgfpathlineto{\pgfqpoint{1.129768in}{0.778757in}}%
\pgfpathlineto{\pgfqpoint{1.130064in}{0.778843in}}%
\pgfpathlineto{\pgfqpoint{1.130360in}{0.778857in}}%
\pgfpathlineto{\pgfqpoint{1.130656in}{0.778836in}}%
\pgfpathlineto{\pgfqpoint{1.130952in}{0.778812in}}%
\pgfpathlineto{\pgfqpoint{1.131248in}{0.778801in}}%
\pgfpathlineto{\pgfqpoint{1.131544in}{0.778797in}}%
\pgfpathlineto{\pgfqpoint{1.131840in}{0.778793in}}%
\pgfpathlineto{\pgfqpoint{1.132136in}{0.778789in}}%
\pgfpathlineto{\pgfqpoint{1.132432in}{0.778785in}}%
\pgfpathlineto{\pgfqpoint{1.132728in}{0.778781in}}%
\pgfpathlineto{\pgfqpoint{1.133024in}{0.778778in}}%
\pgfpathlineto{\pgfqpoint{1.133320in}{0.778774in}}%
\pgfpathlineto{\pgfqpoint{1.133616in}{0.778770in}}%
\pgfpathlineto{\pgfqpoint{1.133912in}{0.778766in}}%
\pgfpathlineto{\pgfqpoint{1.134208in}{0.778762in}}%
\pgfpathlineto{\pgfqpoint{1.134504in}{0.778758in}}%
\pgfpathlineto{\pgfqpoint{1.134800in}{0.778754in}}%
\pgfpathlineto{\pgfqpoint{1.135096in}{0.778751in}}%
\pgfpathlineto{\pgfqpoint{1.135392in}{0.778747in}}%
\pgfpathlineto{\pgfqpoint{1.135688in}{0.778743in}}%
\pgfpathlineto{\pgfqpoint{1.135984in}{0.778739in}}%
\pgfpathlineto{\pgfqpoint{1.136280in}{0.778736in}}%
\pgfpathlineto{\pgfqpoint{1.136576in}{0.778733in}}%
\pgfpathlineto{\pgfqpoint{1.136872in}{0.778730in}}%
\pgfpathlineto{\pgfqpoint{1.137168in}{0.778727in}}%
\pgfpathlineto{\pgfqpoint{1.137464in}{0.778717in}}%
\pgfpathlineto{\pgfqpoint{1.137760in}{0.778719in}}%
\pgfpathlineto{\pgfqpoint{1.138056in}{0.778722in}}%
\pgfpathlineto{\pgfqpoint{1.138352in}{0.778724in}}%
\pgfpathlineto{\pgfqpoint{1.138648in}{0.778726in}}%
\pgfpathlineto{\pgfqpoint{1.138944in}{0.778729in}}%
\pgfpathlineto{\pgfqpoint{1.139240in}{0.778731in}}%
\pgfpathlineto{\pgfqpoint{1.139536in}{0.778733in}}%
\pgfpathlineto{\pgfqpoint{1.139832in}{0.778736in}}%
\pgfpathlineto{\pgfqpoint{1.140128in}{0.778738in}}%
\pgfpathlineto{\pgfqpoint{1.140424in}{0.778740in}}%
\pgfpathlineto{\pgfqpoint{1.140720in}{0.778743in}}%
\pgfpathlineto{\pgfqpoint{1.141016in}{0.778745in}}%
\pgfpathlineto{\pgfqpoint{1.141312in}{0.778747in}}%
\pgfpathlineto{\pgfqpoint{1.141608in}{0.778750in}}%
\pgfpathlineto{\pgfqpoint{1.141904in}{0.778752in}}%
\pgfpathlineto{\pgfqpoint{1.142200in}{0.778755in}}%
\pgfpathlineto{\pgfqpoint{1.142496in}{0.778757in}}%
\pgfpathlineto{\pgfqpoint{1.142792in}{0.778759in}}%
\pgfpathlineto{\pgfqpoint{1.143088in}{0.778762in}}%
\pgfpathlineto{\pgfqpoint{1.143384in}{0.778765in}}%
\pgfpathlineto{\pgfqpoint{1.143680in}{0.778759in}}%
\pgfpathlineto{\pgfqpoint{1.143976in}{0.778743in}}%
\pgfpathlineto{\pgfqpoint{1.144272in}{0.778718in}}%
\pgfpathlineto{\pgfqpoint{1.144568in}{0.778692in}}%
\pgfpathlineto{\pgfqpoint{1.144864in}{0.778683in}}%
\pgfpathlineto{\pgfqpoint{1.145160in}{0.778762in}}%
\pgfpathlineto{\pgfqpoint{1.145456in}{0.778757in}}%
\pgfpathlineto{\pgfqpoint{1.145752in}{0.778779in}}%
\pgfpathlineto{\pgfqpoint{1.146048in}{0.778800in}}%
\pgfpathlineto{\pgfqpoint{1.146344in}{0.778821in}}%
\pgfpathlineto{\pgfqpoint{1.146640in}{0.778843in}}%
\pgfpathlineto{\pgfqpoint{1.146936in}{0.778864in}}%
\pgfpathlineto{\pgfqpoint{1.147232in}{0.778885in}}%
\pgfpathlineto{\pgfqpoint{1.147528in}{0.778907in}}%
\pgfpathlineto{\pgfqpoint{1.147824in}{0.778928in}}%
\pgfpathlineto{\pgfqpoint{1.148120in}{0.778949in}}%
\pgfpathlineto{\pgfqpoint{1.148416in}{0.778971in}}%
\pgfpathlineto{\pgfqpoint{1.148712in}{0.778992in}}%
\pgfpathlineto{\pgfqpoint{1.149008in}{0.779013in}}%
\pgfpathlineto{\pgfqpoint{1.149304in}{0.779035in}}%
\pgfpathlineto{\pgfqpoint{1.149600in}{0.779056in}}%
\pgfpathlineto{\pgfqpoint{1.149896in}{0.779077in}}%
\pgfpathlineto{\pgfqpoint{1.150192in}{0.779099in}}%
\pgfpathlineto{\pgfqpoint{1.150488in}{0.779120in}}%
\pgfpathlineto{\pgfqpoint{1.150784in}{0.779142in}}%
\pgfpathlineto{\pgfqpoint{1.151080in}{0.779163in}}%
\pgfpathlineto{\pgfqpoint{1.151376in}{0.779184in}}%
\pgfpathlineto{\pgfqpoint{1.151672in}{0.779206in}}%
\pgfpathlineto{\pgfqpoint{1.151968in}{0.779227in}}%
\pgfpathlineto{\pgfqpoint{1.152264in}{0.779248in}}%
\pgfpathlineto{\pgfqpoint{1.152560in}{0.779270in}}%
\pgfpathlineto{\pgfqpoint{1.152856in}{0.779291in}}%
\pgfpathlineto{\pgfqpoint{1.153152in}{0.779312in}}%
\pgfpathlineto{\pgfqpoint{1.153448in}{0.779334in}}%
\pgfpathlineto{\pgfqpoint{1.153744in}{0.779355in}}%
\pgfpathlineto{\pgfqpoint{1.154040in}{0.779376in}}%
\pgfpathlineto{\pgfqpoint{1.154336in}{0.779398in}}%
\pgfpathlineto{\pgfqpoint{1.154632in}{0.779419in}}%
\pgfpathlineto{\pgfqpoint{1.154928in}{0.779440in}}%
\pgfpathlineto{\pgfqpoint{1.155224in}{0.779462in}}%
\pgfpathlineto{\pgfqpoint{1.155520in}{0.779483in}}%
\pgfpathlineto{\pgfqpoint{1.155816in}{0.779504in}}%
\pgfpathlineto{\pgfqpoint{1.156112in}{0.779526in}}%
\pgfpathlineto{\pgfqpoint{1.156408in}{0.779547in}}%
\pgfpathlineto{\pgfqpoint{1.156704in}{0.779568in}}%
\pgfpathlineto{\pgfqpoint{1.157000in}{0.779590in}}%
\pgfpathlineto{\pgfqpoint{1.157296in}{0.779611in}}%
\pgfpathlineto{\pgfqpoint{1.157592in}{0.779632in}}%
\pgfpathlineto{\pgfqpoint{1.157888in}{0.779654in}}%
\pgfpathlineto{\pgfqpoint{1.158184in}{0.779675in}}%
\pgfpathlineto{\pgfqpoint{1.158480in}{0.779696in}}%
\pgfpathlineto{\pgfqpoint{1.158776in}{0.779718in}}%
\pgfpathlineto{\pgfqpoint{1.159072in}{0.779739in}}%
\pgfpathlineto{\pgfqpoint{1.159368in}{0.779760in}}%
\pgfpathlineto{\pgfqpoint{1.159664in}{0.779782in}}%
\pgfpathlineto{\pgfqpoint{1.159960in}{0.779803in}}%
\pgfpathlineto{\pgfqpoint{1.160256in}{0.779825in}}%
\pgfpathlineto{\pgfqpoint{1.160552in}{0.779846in}}%
\pgfpathlineto{\pgfqpoint{1.160848in}{0.779867in}}%
\pgfpathlineto{\pgfqpoint{1.161144in}{0.779889in}}%
\pgfpathlineto{\pgfqpoint{1.161440in}{0.779910in}}%
\pgfpathlineto{\pgfqpoint{1.161736in}{0.779931in}}%
\pgfpathlineto{\pgfqpoint{1.162032in}{0.779953in}}%
\pgfpathlineto{\pgfqpoint{1.162328in}{0.779974in}}%
\pgfpathlineto{\pgfqpoint{1.162624in}{0.779995in}}%
\pgfpathlineto{\pgfqpoint{1.162920in}{0.780017in}}%
\pgfpathlineto{\pgfqpoint{1.163216in}{0.780038in}}%
\pgfpathlineto{\pgfqpoint{1.163512in}{0.780059in}}%
\pgfpathlineto{\pgfqpoint{1.163808in}{0.780081in}}%
\pgfpathlineto{\pgfqpoint{1.164104in}{0.780102in}}%
\pgfpathlineto{\pgfqpoint{1.164400in}{0.780121in}}%
\pgfpathlineto{\pgfqpoint{1.164696in}{0.780121in}}%
\pgfpathlineto{\pgfqpoint{1.164992in}{0.780109in}}%
\pgfpathlineto{\pgfqpoint{1.165288in}{0.779583in}}%
\pgfpathlineto{\pgfqpoint{1.165584in}{0.778813in}}%
\pgfpathlineto{\pgfqpoint{1.165880in}{0.779031in}}%
\pgfpathlineto{\pgfqpoint{1.166176in}{0.779664in}}%
\pgfpathlineto{\pgfqpoint{1.166472in}{0.780297in}}%
\pgfpathlineto{\pgfqpoint{1.166768in}{0.780931in}}%
\pgfpathlineto{\pgfqpoint{1.167064in}{0.781564in}}%
\pgfpathlineto{\pgfqpoint{1.167360in}{0.782197in}}%
\pgfpathlineto{\pgfqpoint{1.167656in}{0.782831in}}%
\pgfpathlineto{\pgfqpoint{1.167952in}{0.783460in}}%
\pgfpathlineto{\pgfqpoint{1.168248in}{0.783712in}}%
\pgfpathlineto{\pgfqpoint{1.168544in}{0.783713in}}%
\pgfpathlineto{\pgfqpoint{1.168840in}{0.783713in}}%
\pgfpathlineto{\pgfqpoint{1.169136in}{0.783713in}}%
\pgfpathlineto{\pgfqpoint{1.169432in}{0.783714in}}%
\pgfpathlineto{\pgfqpoint{1.169728in}{0.783714in}}%
\pgfpathlineto{\pgfqpoint{1.170024in}{0.783714in}}%
\pgfpathlineto{\pgfqpoint{1.170320in}{0.783715in}}%
\pgfpathlineto{\pgfqpoint{1.170616in}{0.783715in}}%
\pgfpathlineto{\pgfqpoint{1.170912in}{0.783715in}}%
\pgfpathlineto{\pgfqpoint{1.171208in}{0.783716in}}%
\pgfpathlineto{\pgfqpoint{1.171504in}{0.783716in}}%
\pgfpathlineto{\pgfqpoint{1.171800in}{0.783246in}}%
\pgfpathlineto{\pgfqpoint{1.172096in}{0.782027in}}%
\pgfpathlineto{\pgfqpoint{1.172392in}{0.780798in}}%
\pgfpathlineto{\pgfqpoint{1.172688in}{0.782144in}}%
\pgfpathlineto{\pgfqpoint{1.172984in}{0.783724in}}%
\pgfpathlineto{\pgfqpoint{1.173280in}{0.783694in}}%
\pgfpathlineto{\pgfqpoint{1.173576in}{0.783758in}}%
\pgfpathlineto{\pgfqpoint{1.173872in}{0.783787in}}%
\pgfpathlineto{\pgfqpoint{1.174168in}{0.783783in}}%
\pgfpathlineto{\pgfqpoint{1.174464in}{0.783779in}}%
\pgfpathlineto{\pgfqpoint{1.174760in}{0.783774in}}%
\pgfpathlineto{\pgfqpoint{1.175056in}{0.783770in}}%
\pgfpathlineto{\pgfqpoint{1.175352in}{0.783765in}}%
\pgfpathlineto{\pgfqpoint{1.175648in}{0.783761in}}%
\pgfpathlineto{\pgfqpoint{1.175944in}{0.783756in}}%
\pgfpathlineto{\pgfqpoint{1.176240in}{0.783752in}}%
\pgfpathlineto{\pgfqpoint{1.176536in}{0.783748in}}%
\pgfpathlineto{\pgfqpoint{1.176832in}{0.783743in}}%
\pgfpathlineto{\pgfqpoint{1.177128in}{0.783739in}}%
\pgfpathlineto{\pgfqpoint{1.177424in}{0.783734in}}%
\pgfpathlineto{\pgfqpoint{1.177720in}{0.783730in}}%
\pgfpathlineto{\pgfqpoint{1.178016in}{0.783726in}}%
\pgfpathlineto{\pgfqpoint{1.178312in}{0.783730in}}%
\pgfpathlineto{\pgfqpoint{1.178608in}{0.783693in}}%
\pgfpathlineto{\pgfqpoint{1.178904in}{0.782810in}}%
\pgfpathlineto{\pgfqpoint{1.179200in}{0.781673in}}%
\pgfpathlineto{\pgfqpoint{1.179496in}{0.780585in}}%
\pgfpathlineto{\pgfqpoint{1.179792in}{0.781373in}}%
\pgfpathlineto{\pgfqpoint{1.180088in}{0.783085in}}%
\pgfpathlineto{\pgfqpoint{1.180384in}{0.783743in}}%
\pgfpathlineto{\pgfqpoint{1.180680in}{0.783736in}}%
\pgfpathlineto{\pgfqpoint{1.180976in}{0.783729in}}%
\pgfpathlineto{\pgfqpoint{1.181272in}{0.783721in}}%
\pgfpathlineto{\pgfqpoint{1.181568in}{0.783713in}}%
\pgfpathlineto{\pgfqpoint{1.181864in}{0.783705in}}%
\pgfpathlineto{\pgfqpoint{1.182160in}{0.783697in}}%
\pgfpathlineto{\pgfqpoint{1.182456in}{0.783689in}}%
\pgfpathlineto{\pgfqpoint{1.182752in}{0.783681in}}%
\pgfpathlineto{\pgfqpoint{1.183048in}{0.783673in}}%
\pgfpathlineto{\pgfqpoint{1.183344in}{0.783666in}}%
\pgfpathlineto{\pgfqpoint{1.183640in}{0.783658in}}%
\pgfpathlineto{\pgfqpoint{1.183936in}{0.783650in}}%
\pgfpathlineto{\pgfqpoint{1.184232in}{0.783642in}}%
\pgfpathlineto{\pgfqpoint{1.184528in}{0.783634in}}%
\pgfpathlineto{\pgfqpoint{1.184824in}{0.783626in}}%
\pgfpathlineto{\pgfqpoint{1.185120in}{0.783618in}}%
\pgfpathlineto{\pgfqpoint{1.185416in}{0.783616in}}%
\pgfpathlineto{\pgfqpoint{1.185712in}{0.783628in}}%
\pgfpathlineto{\pgfqpoint{1.186008in}{0.783629in}}%
\pgfpathlineto{\pgfqpoint{1.186304in}{0.783629in}}%
\pgfpathlineto{\pgfqpoint{1.186600in}{0.783645in}}%
\pgfpathlineto{\pgfqpoint{1.186896in}{0.783246in}}%
\pgfpathlineto{\pgfqpoint{1.187192in}{0.780978in}}%
\pgfpathlineto{\pgfqpoint{1.187489in}{0.782684in}}%
\pgfpathlineto{\pgfqpoint{1.187785in}{0.783643in}}%
\pgfpathlineto{\pgfqpoint{1.188081in}{0.783622in}}%
\pgfpathlineto{\pgfqpoint{1.188377in}{0.783621in}}%
\pgfpathlineto{\pgfqpoint{1.188673in}{0.783851in}}%
\pgfpathlineto{\pgfqpoint{1.188969in}{0.784135in}}%
\pgfpathlineto{\pgfqpoint{1.189265in}{0.784208in}}%
\pgfpathlineto{\pgfqpoint{1.189561in}{0.784214in}}%
\pgfpathlineto{\pgfqpoint{1.189857in}{0.784220in}}%
\pgfpathlineto{\pgfqpoint{1.190153in}{0.784226in}}%
\pgfpathlineto{\pgfqpoint{1.190449in}{0.784232in}}%
\pgfpathlineto{\pgfqpoint{1.190745in}{0.784238in}}%
\pgfpathlineto{\pgfqpoint{1.191041in}{0.784244in}}%
\pgfpathlineto{\pgfqpoint{1.191337in}{0.784250in}}%
\pgfpathlineto{\pgfqpoint{1.191633in}{0.784256in}}%
\pgfpathlineto{\pgfqpoint{1.191929in}{0.784262in}}%
\pgfpathlineto{\pgfqpoint{1.192225in}{0.784268in}}%
\pgfpathlineto{\pgfqpoint{1.192521in}{0.784273in}}%
\pgfpathlineto{\pgfqpoint{1.192817in}{0.784276in}}%
\pgfpathlineto{\pgfqpoint{1.193113in}{0.784239in}}%
\pgfpathlineto{\pgfqpoint{1.193409in}{0.784242in}}%
\pgfpathlineto{\pgfqpoint{1.193705in}{0.784250in}}%
\pgfpathlineto{\pgfqpoint{1.194001in}{0.784282in}}%
\pgfpathlineto{\pgfqpoint{1.194297in}{0.784359in}}%
\pgfpathlineto{\pgfqpoint{1.194593in}{0.784437in}}%
\pgfpathlineto{\pgfqpoint{1.194889in}{0.784419in}}%
\pgfpathlineto{\pgfqpoint{1.195185in}{0.784423in}}%
\pgfpathlineto{\pgfqpoint{1.195481in}{0.784528in}}%
\pgfpathlineto{\pgfqpoint{1.195777in}{0.784633in}}%
\pgfpathlineto{\pgfqpoint{1.196073in}{0.784738in}}%
\pgfpathlineto{\pgfqpoint{1.196369in}{0.784843in}}%
\pgfpathlineto{\pgfqpoint{1.196665in}{0.784948in}}%
\pgfpathlineto{\pgfqpoint{1.196961in}{0.785053in}}%
\pgfpathlineto{\pgfqpoint{1.197257in}{0.785158in}}%
\pgfpathlineto{\pgfqpoint{1.197553in}{0.785263in}}%
\pgfpathlineto{\pgfqpoint{1.197849in}{0.785368in}}%
\pgfpathlineto{\pgfqpoint{1.198145in}{0.785473in}}%
\pgfpathlineto{\pgfqpoint{1.198441in}{0.785578in}}%
\pgfpathlineto{\pgfqpoint{1.198737in}{0.785683in}}%
\pgfpathlineto{\pgfqpoint{1.199033in}{0.785788in}}%
\pgfpathlineto{\pgfqpoint{1.199329in}{0.785893in}}%
\pgfpathlineto{\pgfqpoint{1.199625in}{0.785998in}}%
\pgfpathlineto{\pgfqpoint{1.199921in}{0.786103in}}%
\pgfpathlineto{\pgfqpoint{1.200217in}{0.786208in}}%
\pgfpathlineto{\pgfqpoint{1.200513in}{0.786313in}}%
\pgfpathlineto{\pgfqpoint{1.200809in}{0.786418in}}%
\pgfpathlineto{\pgfqpoint{1.201105in}{0.786523in}}%
\pgfpathlineto{\pgfqpoint{1.201401in}{0.786629in}}%
\pgfpathlineto{\pgfqpoint{1.201697in}{0.786734in}}%
\pgfpathlineto{\pgfqpoint{1.201993in}{0.786839in}}%
\pgfpathlineto{\pgfqpoint{1.202289in}{0.786944in}}%
\pgfpathlineto{\pgfqpoint{1.202585in}{0.787049in}}%
\pgfpathlineto{\pgfqpoint{1.202881in}{0.787154in}}%
\pgfpathlineto{\pgfqpoint{1.203177in}{0.787259in}}%
\pgfpathlineto{\pgfqpoint{1.203473in}{0.787364in}}%
\pgfpathlineto{\pgfqpoint{1.203769in}{0.787469in}}%
\pgfpathlineto{\pgfqpoint{1.204065in}{0.787574in}}%
\pgfpathlineto{\pgfqpoint{1.204361in}{0.787679in}}%
\pgfpathlineto{\pgfqpoint{1.204657in}{0.787784in}}%
\pgfpathlineto{\pgfqpoint{1.204953in}{0.787889in}}%
\pgfpathlineto{\pgfqpoint{1.205249in}{0.787994in}}%
\pgfpathlineto{\pgfqpoint{1.205545in}{0.788099in}}%
\pgfpathlineto{\pgfqpoint{1.205841in}{0.788204in}}%
\pgfpathlineto{\pgfqpoint{1.206137in}{0.788309in}}%
\pgfpathlineto{\pgfqpoint{1.206433in}{0.788414in}}%
\pgfpathlineto{\pgfqpoint{1.206729in}{0.788519in}}%
\pgfpathlineto{\pgfqpoint{1.207025in}{0.788624in}}%
\pgfpathlineto{\pgfqpoint{1.207321in}{0.788729in}}%
\pgfpathlineto{\pgfqpoint{1.207617in}{0.788834in}}%
\pgfpathlineto{\pgfqpoint{1.207913in}{0.788940in}}%
\pgfpathlineto{\pgfqpoint{1.208209in}{0.789045in}}%
\pgfpathlineto{\pgfqpoint{1.208505in}{0.789150in}}%
\pgfpathlineto{\pgfqpoint{1.208801in}{0.789255in}}%
\pgfpathlineto{\pgfqpoint{1.209097in}{0.789360in}}%
\pgfpathlineto{\pgfqpoint{1.209393in}{0.789465in}}%
\pgfpathlineto{\pgfqpoint{1.209689in}{0.789570in}}%
\pgfpathlineto{\pgfqpoint{1.209985in}{0.789675in}}%
\pgfpathlineto{\pgfqpoint{1.210281in}{0.789780in}}%
\pgfpathlineto{\pgfqpoint{1.210577in}{0.789885in}}%
\pgfpathlineto{\pgfqpoint{1.210873in}{0.789990in}}%
\pgfpathlineto{\pgfqpoint{1.211169in}{0.790095in}}%
\pgfpathlineto{\pgfqpoint{1.211465in}{0.790200in}}%
\pgfpathlineto{\pgfqpoint{1.211761in}{0.790305in}}%
\pgfpathlineto{\pgfqpoint{1.212057in}{0.790410in}}%
\pgfpathlineto{\pgfqpoint{1.212353in}{0.790515in}}%
\pgfpathlineto{\pgfqpoint{1.212649in}{0.790620in}}%
\pgfpathlineto{\pgfqpoint{1.212945in}{0.790725in}}%
\pgfpathlineto{\pgfqpoint{1.213241in}{0.790830in}}%
\pgfpathlineto{\pgfqpoint{1.213537in}{0.790935in}}%
\pgfpathlineto{\pgfqpoint{1.213833in}{0.791040in}}%
\pgfpathlineto{\pgfqpoint{1.214129in}{0.791120in}}%
\pgfpathlineto{\pgfqpoint{1.214425in}{0.791125in}}%
\pgfpathlineto{\pgfqpoint{1.214721in}{0.791126in}}%
\pgfpathlineto{\pgfqpoint{1.215017in}{0.791126in}}%
\pgfpathlineto{\pgfqpoint{1.215313in}{0.791126in}}%
\pgfpathlineto{\pgfqpoint{1.215609in}{0.791126in}}%
\pgfpathlineto{\pgfqpoint{1.215905in}{0.791126in}}%
\pgfpathlineto{\pgfqpoint{1.216201in}{0.790310in}}%
\pgfpathlineto{\pgfqpoint{1.216497in}{0.791133in}}%
\pgfpathlineto{\pgfqpoint{1.216793in}{0.793270in}}%
\pgfpathlineto{\pgfqpoint{1.217089in}{0.793221in}}%
\pgfpathlineto{\pgfqpoint{1.217385in}{0.793174in}}%
\pgfpathlineto{\pgfqpoint{1.217681in}{0.793127in}}%
\pgfpathlineto{\pgfqpoint{1.217977in}{0.793079in}}%
\pgfpathlineto{\pgfqpoint{1.218273in}{0.793032in}}%
\pgfpathlineto{\pgfqpoint{1.218569in}{0.792985in}}%
\pgfpathlineto{\pgfqpoint{1.218865in}{0.792937in}}%
\pgfpathlineto{\pgfqpoint{1.219161in}{0.792890in}}%
\pgfpathlineto{\pgfqpoint{1.219457in}{0.792843in}}%
\pgfpathlineto{\pgfqpoint{1.219753in}{0.792795in}}%
\pgfpathlineto{\pgfqpoint{1.220049in}{0.792748in}}%
\pgfpathlineto{\pgfqpoint{1.220345in}{0.792701in}}%
\pgfpathlineto{\pgfqpoint{1.220641in}{0.792659in}}%
\pgfpathlineto{\pgfqpoint{1.220937in}{0.792453in}}%
\pgfpathlineto{\pgfqpoint{1.221233in}{0.792214in}}%
\pgfpathlineto{\pgfqpoint{1.221529in}{0.792494in}}%
\pgfpathlineto{\pgfqpoint{1.221825in}{0.792619in}}%
\pgfpathlineto{\pgfqpoint{1.222121in}{0.792529in}}%
\pgfpathlineto{\pgfqpoint{1.222417in}{0.792621in}}%
\pgfpathlineto{\pgfqpoint{1.222713in}{0.792624in}}%
\pgfpathlineto{\pgfqpoint{1.223009in}{0.792614in}}%
\pgfpathlineto{\pgfqpoint{1.223305in}{0.792594in}}%
\pgfpathlineto{\pgfqpoint{1.223601in}{0.792582in}}%
\pgfpathlineto{\pgfqpoint{1.223897in}{0.792561in}}%
\pgfpathlineto{\pgfqpoint{1.224193in}{0.792539in}}%
\pgfpathlineto{\pgfqpoint{1.224489in}{0.792518in}}%
\pgfpathlineto{\pgfqpoint{1.224785in}{0.792497in}}%
\pgfpathlineto{\pgfqpoint{1.225081in}{0.792476in}}%
\pgfpathlineto{\pgfqpoint{1.225377in}{0.792454in}}%
\pgfpathlineto{\pgfqpoint{1.225673in}{0.792433in}}%
\pgfpathlineto{\pgfqpoint{1.225969in}{0.792412in}}%
\pgfpathlineto{\pgfqpoint{1.226265in}{0.792391in}}%
\pgfpathlineto{\pgfqpoint{1.226561in}{0.792369in}}%
\pgfpathlineto{\pgfqpoint{1.226857in}{0.792348in}}%
\pgfpathlineto{\pgfqpoint{1.227153in}{0.792327in}}%
\pgfpathlineto{\pgfqpoint{1.227449in}{0.792306in}}%
\pgfpathlineto{\pgfqpoint{1.227745in}{0.792284in}}%
\pgfpathlineto{\pgfqpoint{1.228041in}{0.792263in}}%
\pgfpathlineto{\pgfqpoint{1.228337in}{0.792242in}}%
\pgfpathlineto{\pgfqpoint{1.228633in}{0.792221in}}%
\pgfpathlineto{\pgfqpoint{1.228929in}{0.792213in}}%
\pgfpathlineto{\pgfqpoint{1.229225in}{0.792216in}}%
\pgfpathlineto{\pgfqpoint{1.229521in}{0.792246in}}%
\pgfpathlineto{\pgfqpoint{1.229817in}{0.792280in}}%
\pgfpathlineto{\pgfqpoint{1.230113in}{0.792268in}}%
\pgfpathlineto{\pgfqpoint{1.230409in}{0.792222in}}%
\pgfpathlineto{\pgfqpoint{1.230705in}{0.792211in}}%
\pgfpathlineto{\pgfqpoint{1.231001in}{0.792227in}}%
\pgfpathlineto{\pgfqpoint{1.231297in}{0.792245in}}%
\pgfpathlineto{\pgfqpoint{1.231593in}{0.792263in}}%
\pgfpathlineto{\pgfqpoint{1.231889in}{0.792281in}}%
\pgfpathlineto{\pgfqpoint{1.232185in}{0.792299in}}%
\pgfpathlineto{\pgfqpoint{1.232481in}{0.792317in}}%
\pgfpathlineto{\pgfqpoint{1.232777in}{0.792334in}}%
\pgfpathlineto{\pgfqpoint{1.233073in}{0.792352in}}%
\pgfpathlineto{\pgfqpoint{1.233369in}{0.792370in}}%
\pgfpathlineto{\pgfqpoint{1.233665in}{0.792388in}}%
\pgfpathlineto{\pgfqpoint{1.233961in}{0.792406in}}%
\pgfpathlineto{\pgfqpoint{1.234257in}{0.792424in}}%
\pgfpathlineto{\pgfqpoint{1.234553in}{0.792442in}}%
\pgfpathlineto{\pgfqpoint{1.234849in}{0.792459in}}%
\pgfpathlineto{\pgfqpoint{1.235145in}{0.792476in}}%
\pgfpathlineto{\pgfqpoint{1.235441in}{0.792235in}}%
\pgfpathlineto{\pgfqpoint{1.235737in}{0.792194in}}%
\pgfpathlineto{\pgfqpoint{1.236033in}{0.792369in}}%
\pgfpathlineto{\pgfqpoint{1.236329in}{0.792548in}}%
\pgfpathlineto{\pgfqpoint{1.236625in}{0.792694in}}%
\pgfpathlineto{\pgfqpoint{1.236921in}{0.792385in}}%
\pgfpathlineto{\pgfqpoint{1.237217in}{0.792632in}}%
\pgfpathlineto{\pgfqpoint{1.237513in}{0.792713in}}%
\pgfpathlineto{\pgfqpoint{1.237809in}{0.792351in}}%
\pgfpathlineto{\pgfqpoint{1.238105in}{0.792197in}}%
\pgfpathlineto{\pgfqpoint{1.238401in}{0.792192in}}%
\pgfpathlineto{\pgfqpoint{1.238697in}{0.792186in}}%
\pgfpathlineto{\pgfqpoint{1.238993in}{0.792180in}}%
\pgfpathlineto{\pgfqpoint{1.239289in}{0.792175in}}%
\pgfpathlineto{\pgfqpoint{1.239585in}{0.792169in}}%
\pgfpathlineto{\pgfqpoint{1.239881in}{0.792164in}}%
\pgfpathlineto{\pgfqpoint{1.240177in}{0.792158in}}%
\pgfpathlineto{\pgfqpoint{1.240473in}{0.792152in}}%
\pgfpathlineto{\pgfqpoint{1.240769in}{0.792147in}}%
\pgfpathlineto{\pgfqpoint{1.241065in}{0.792141in}}%
\pgfpathlineto{\pgfqpoint{1.241361in}{0.792136in}}%
\pgfpathlineto{\pgfqpoint{1.241657in}{0.792130in}}%
\pgfpathlineto{\pgfqpoint{1.241953in}{0.792125in}}%
\pgfpathlineto{\pgfqpoint{1.242249in}{0.792119in}}%
\pgfpathlineto{\pgfqpoint{1.242545in}{0.792113in}}%
\pgfpathlineto{\pgfqpoint{1.242841in}{0.792119in}}%
\pgfpathlineto{\pgfqpoint{1.243137in}{0.792156in}}%
\pgfpathlineto{\pgfqpoint{1.243433in}{0.792173in}}%
\pgfpathlineto{\pgfqpoint{1.243729in}{0.792172in}}%
\pgfpathlineto{\pgfqpoint{1.244025in}{0.792170in}}%
\pgfpathlineto{\pgfqpoint{1.244321in}{0.792339in}}%
\pgfpathlineto{\pgfqpoint{1.244617in}{0.792863in}}%
\pgfpathlineto{\pgfqpoint{1.244913in}{0.793046in}}%
\pgfpathlineto{\pgfqpoint{1.245209in}{0.793045in}}%
\pgfpathlineto{\pgfqpoint{1.245505in}{0.793044in}}%
\pgfpathlineto{\pgfqpoint{1.245801in}{0.793043in}}%
\pgfpathlineto{\pgfqpoint{1.246097in}{0.793042in}}%
\pgfpathlineto{\pgfqpoint{1.246393in}{0.793042in}}%
\pgfpathlineto{\pgfqpoint{1.246689in}{0.793042in}}%
\pgfpathlineto{\pgfqpoint{1.246985in}{0.793042in}}%
\pgfpathlineto{\pgfqpoint{1.247281in}{0.793042in}}%
\pgfpathlineto{\pgfqpoint{1.247577in}{0.793042in}}%
\pgfpathlineto{\pgfqpoint{1.247873in}{0.793042in}}%
\pgfpathlineto{\pgfqpoint{1.248169in}{0.793042in}}%
\pgfpathlineto{\pgfqpoint{1.248465in}{0.793042in}}%
\pgfpathlineto{\pgfqpoint{1.248761in}{0.793043in}}%
\pgfpathlineto{\pgfqpoint{1.249057in}{0.793043in}}%
\pgfpathlineto{\pgfqpoint{1.249353in}{0.793043in}}%
\pgfpathlineto{\pgfqpoint{1.249649in}{0.793043in}}%
\pgfpathlineto{\pgfqpoint{1.249945in}{0.793015in}}%
\pgfpathlineto{\pgfqpoint{1.250241in}{0.792734in}}%
\pgfpathlineto{\pgfqpoint{1.250537in}{0.792393in}}%
\pgfpathlineto{\pgfqpoint{1.250833in}{0.792139in}}%
\pgfpathlineto{\pgfqpoint{1.251129in}{0.792255in}}%
\pgfpathlineto{\pgfqpoint{1.251425in}{0.792412in}}%
\pgfpathlineto{\pgfqpoint{1.251721in}{0.792570in}}%
\pgfpathlineto{\pgfqpoint{1.252017in}{0.792727in}}%
\pgfpathlineto{\pgfqpoint{1.252313in}{0.792884in}}%
\pgfpathlineto{\pgfqpoint{1.252609in}{0.793041in}}%
\pgfpathlineto{\pgfqpoint{1.252905in}{0.793199in}}%
\pgfpathlineto{\pgfqpoint{1.253201in}{0.793356in}}%
\pgfpathlineto{\pgfqpoint{1.253497in}{0.793513in}}%
\pgfpathlineto{\pgfqpoint{1.253793in}{0.793670in}}%
\pgfpathlineto{\pgfqpoint{1.254089in}{0.793827in}}%
\pgfpathlineto{\pgfqpoint{1.254385in}{0.793985in}}%
\pgfpathlineto{\pgfqpoint{1.254681in}{0.794142in}}%
\pgfpathlineto{\pgfqpoint{1.254978in}{0.794299in}}%
\pgfpathlineto{\pgfqpoint{1.255274in}{0.794456in}}%
\pgfpathlineto{\pgfqpoint{1.255570in}{0.794614in}}%
\pgfpathlineto{\pgfqpoint{1.255866in}{0.794771in}}%
\pgfpathlineto{\pgfqpoint{1.256162in}{0.794928in}}%
\pgfpathlineto{\pgfqpoint{1.256458in}{0.795085in}}%
\pgfpathlineto{\pgfqpoint{1.256754in}{0.795242in}}%
\pgfpathlineto{\pgfqpoint{1.257050in}{0.795400in}}%
\pgfpathlineto{\pgfqpoint{1.257346in}{0.795557in}}%
\pgfpathlineto{\pgfqpoint{1.257642in}{0.795714in}}%
\pgfpathlineto{\pgfqpoint{1.257938in}{0.795871in}}%
\pgfpathlineto{\pgfqpoint{1.258234in}{0.796029in}}%
\pgfpathlineto{\pgfqpoint{1.258530in}{0.796186in}}%
\pgfpathlineto{\pgfqpoint{1.258826in}{0.796343in}}%
\pgfpathlineto{\pgfqpoint{1.259122in}{0.796500in}}%
\pgfpathlineto{\pgfqpoint{1.259418in}{0.796657in}}%
\pgfpathlineto{\pgfqpoint{1.259714in}{0.796815in}}%
\pgfpathlineto{\pgfqpoint{1.260010in}{0.796972in}}%
\pgfpathlineto{\pgfqpoint{1.260306in}{0.797129in}}%
\pgfpathlineto{\pgfqpoint{1.260602in}{0.797286in}}%
\pgfpathlineto{\pgfqpoint{1.260898in}{0.797444in}}%
\pgfpathlineto{\pgfqpoint{1.261194in}{0.797601in}}%
\pgfpathlineto{\pgfqpoint{1.261490in}{0.797758in}}%
\pgfpathlineto{\pgfqpoint{1.261786in}{0.797915in}}%
\pgfpathlineto{\pgfqpoint{1.262082in}{0.798073in}}%
\pgfpathlineto{\pgfqpoint{1.262378in}{0.798230in}}%
\pgfpathlineto{\pgfqpoint{1.262674in}{0.798387in}}%
\pgfpathlineto{\pgfqpoint{1.262970in}{0.798544in}}%
\pgfpathlineto{\pgfqpoint{1.263266in}{0.798701in}}%
\pgfpathlineto{\pgfqpoint{1.263562in}{0.798859in}}%
\pgfpathlineto{\pgfqpoint{1.263858in}{0.799630in}}%
\pgfpathlineto{\pgfqpoint{1.264154in}{0.801269in}}%
\pgfpathlineto{\pgfqpoint{1.264450in}{0.802704in}}%
\pgfpathlineto{\pgfqpoint{1.264746in}{0.802903in}}%
\pgfpathlineto{\pgfqpoint{1.265042in}{0.802902in}}%
\pgfpathlineto{\pgfqpoint{1.265338in}{0.802901in}}%
\pgfpathlineto{\pgfqpoint{1.265634in}{0.802899in}}%
\pgfpathlineto{\pgfqpoint{1.265930in}{0.802898in}}%
\pgfpathlineto{\pgfqpoint{1.266226in}{0.802896in}}%
\pgfpathlineto{\pgfqpoint{1.266522in}{0.802894in}}%
\pgfpathlineto{\pgfqpoint{1.266818in}{0.802888in}}%
\pgfpathlineto{\pgfqpoint{1.267114in}{0.802883in}}%
\pgfpathlineto{\pgfqpoint{1.267410in}{0.802683in}}%
\pgfpathlineto{\pgfqpoint{1.267706in}{0.793942in}}%
\pgfpathlineto{\pgfqpoint{1.268002in}{0.793501in}}%
\pgfpathlineto{\pgfqpoint{1.268298in}{0.794805in}}%
\pgfpathlineto{\pgfqpoint{1.268594in}{0.796109in}}%
\pgfpathlineto{\pgfqpoint{1.268890in}{0.797414in}}%
\pgfpathlineto{\pgfqpoint{1.269186in}{0.798718in}}%
\pgfpathlineto{\pgfqpoint{1.269482in}{0.800022in}}%
\pgfpathlineto{\pgfqpoint{1.269778in}{0.801326in}}%
\pgfpathlineto{\pgfqpoint{1.270074in}{0.802568in}}%
\pgfpathlineto{\pgfqpoint{1.270370in}{0.802876in}}%
\pgfpathlineto{\pgfqpoint{1.270666in}{0.802871in}}%
\pgfpathlineto{\pgfqpoint{1.270962in}{0.802866in}}%
\pgfpathlineto{\pgfqpoint{1.271258in}{0.802862in}}%
\pgfpathlineto{\pgfqpoint{1.271554in}{0.802871in}}%
\pgfpathlineto{\pgfqpoint{1.271850in}{0.802878in}}%
\pgfpathlineto{\pgfqpoint{1.272146in}{0.802878in}}%
\pgfpathlineto{\pgfqpoint{1.272442in}{0.802874in}}%
\pgfpathlineto{\pgfqpoint{1.272738in}{0.802869in}}%
\pgfpathlineto{\pgfqpoint{1.273034in}{0.802910in}}%
\pgfpathlineto{\pgfqpoint{1.273330in}{0.802963in}}%
\pgfpathlineto{\pgfqpoint{1.273626in}{0.802976in}}%
\pgfpathlineto{\pgfqpoint{1.273922in}{0.802975in}}%
\pgfpathlineto{\pgfqpoint{1.274218in}{0.802975in}}%
\pgfpathlineto{\pgfqpoint{1.274514in}{0.802974in}}%
\pgfpathlineto{\pgfqpoint{1.274810in}{0.802974in}}%
\pgfpathlineto{\pgfqpoint{1.275106in}{0.802973in}}%
\pgfpathlineto{\pgfqpoint{1.275402in}{0.802973in}}%
\pgfpathlineto{\pgfqpoint{1.275698in}{0.802973in}}%
\pgfpathlineto{\pgfqpoint{1.275994in}{0.802972in}}%
\pgfpathlineto{\pgfqpoint{1.276290in}{0.802972in}}%
\pgfpathlineto{\pgfqpoint{1.276586in}{0.802971in}}%
\pgfpathlineto{\pgfqpoint{1.276882in}{0.802971in}}%
\pgfpathlineto{\pgfqpoint{1.277178in}{0.802971in}}%
\pgfpathlineto{\pgfqpoint{1.277474in}{0.802970in}}%
\pgfpathlineto{\pgfqpoint{1.277770in}{0.802970in}}%
\pgfpathlineto{\pgfqpoint{1.278066in}{0.802969in}}%
\pgfpathlineto{\pgfqpoint{1.278362in}{0.802969in}}%
\pgfpathlineto{\pgfqpoint{1.278658in}{0.802969in}}%
\pgfpathlineto{\pgfqpoint{1.278954in}{0.802976in}}%
\pgfpathlineto{\pgfqpoint{1.279250in}{0.802988in}}%
\pgfpathlineto{\pgfqpoint{1.279546in}{0.803000in}}%
\pgfpathlineto{\pgfqpoint{1.279842in}{0.802988in}}%
\pgfpathlineto{\pgfqpoint{1.280138in}{0.799672in}}%
\pgfpathlineto{\pgfqpoint{1.280434in}{0.794076in}}%
\pgfpathlineto{\pgfqpoint{1.280730in}{0.792034in}}%
\pgfpathlineto{\pgfqpoint{1.281026in}{0.792634in}}%
\pgfpathlineto{\pgfqpoint{1.281322in}{0.793431in}}%
\pgfpathlineto{\pgfqpoint{1.281618in}{0.794228in}}%
\pgfpathlineto{\pgfqpoint{1.281914in}{0.795025in}}%
\pgfpathlineto{\pgfqpoint{1.282210in}{0.795822in}}%
\pgfpathlineto{\pgfqpoint{1.282506in}{0.796619in}}%
\pgfpathlineto{\pgfqpoint{1.282802in}{0.797416in}}%
\pgfpathlineto{\pgfqpoint{1.283098in}{0.798213in}}%
\pgfpathlineto{\pgfqpoint{1.283394in}{0.799010in}}%
\pgfpathlineto{\pgfqpoint{1.283690in}{0.799807in}}%
\pgfpathlineto{\pgfqpoint{1.283986in}{0.800604in}}%
\pgfpathlineto{\pgfqpoint{1.284282in}{0.801401in}}%
\pgfpathlineto{\pgfqpoint{1.284578in}{0.802198in}}%
\pgfpathlineto{\pgfqpoint{1.284874in}{0.802906in}}%
\pgfpathlineto{\pgfqpoint{1.285170in}{0.803023in}}%
\pgfpathlineto{\pgfqpoint{1.285466in}{0.803030in}}%
\pgfpathlineto{\pgfqpoint{1.285762in}{0.803038in}}%
\pgfpathlineto{\pgfqpoint{1.286058in}{0.803045in}}%
\pgfpathlineto{\pgfqpoint{1.286354in}{0.803045in}}%
\pgfpathlineto{\pgfqpoint{1.286650in}{0.803039in}}%
\pgfpathlineto{\pgfqpoint{1.286946in}{0.797848in}}%
\pgfpathlineto{\pgfqpoint{1.287242in}{0.799942in}}%
\pgfpathlineto{\pgfqpoint{1.287538in}{0.803043in}}%
\pgfpathlineto{\pgfqpoint{1.287834in}{0.802881in}}%
\pgfpathlineto{\pgfqpoint{1.288130in}{0.802720in}}%
\pgfpathlineto{\pgfqpoint{1.288426in}{0.802559in}}%
\pgfpathlineto{\pgfqpoint{1.288722in}{0.802397in}}%
\pgfpathlineto{\pgfqpoint{1.289018in}{0.802236in}}%
\pgfpathlineto{\pgfqpoint{1.289314in}{0.802074in}}%
\pgfpathlineto{\pgfqpoint{1.289610in}{0.801913in}}%
\pgfpathlineto{\pgfqpoint{1.289906in}{0.801751in}}%
\pgfpathlineto{\pgfqpoint{1.290202in}{0.801590in}}%
\pgfpathlineto{\pgfqpoint{1.290498in}{0.801428in}}%
\pgfpathlineto{\pgfqpoint{1.290794in}{0.801267in}}%
\pgfpathlineto{\pgfqpoint{1.291090in}{0.801105in}}%
\pgfpathlineto{\pgfqpoint{1.291386in}{0.800944in}}%
\pgfpathlineto{\pgfqpoint{1.291682in}{0.800782in}}%
\pgfpathlineto{\pgfqpoint{1.291978in}{0.800621in}}%
\pgfpathlineto{\pgfqpoint{1.292274in}{0.800459in}}%
\pgfpathlineto{\pgfqpoint{1.292570in}{0.800298in}}%
\pgfpathlineto{\pgfqpoint{1.292866in}{0.800136in}}%
\pgfpathlineto{\pgfqpoint{1.293162in}{0.799975in}}%
\pgfpathlineto{\pgfqpoint{1.293458in}{0.799813in}}%
\pgfpathlineto{\pgfqpoint{1.293754in}{0.799652in}}%
\pgfpathlineto{\pgfqpoint{1.294050in}{0.799488in}}%
\pgfpathlineto{\pgfqpoint{1.294346in}{0.799324in}}%
\pgfpathlineto{\pgfqpoint{1.294642in}{0.799159in}}%
\pgfpathlineto{\pgfqpoint{1.294938in}{0.798994in}}%
\pgfpathlineto{\pgfqpoint{1.295234in}{0.798830in}}%
\pgfpathlineto{\pgfqpoint{1.295530in}{0.798665in}}%
\pgfpathlineto{\pgfqpoint{1.295826in}{0.798501in}}%
\pgfpathlineto{\pgfqpoint{1.296122in}{0.798336in}}%
\pgfpathlineto{\pgfqpoint{1.296418in}{0.798172in}}%
\pgfpathlineto{\pgfqpoint{1.296714in}{0.798007in}}%
\pgfpathlineto{\pgfqpoint{1.297010in}{0.797843in}}%
\pgfpathlineto{\pgfqpoint{1.297306in}{0.797678in}}%
\pgfpathlineto{\pgfqpoint{1.297602in}{0.797513in}}%
\pgfpathlineto{\pgfqpoint{1.297898in}{0.797349in}}%
\pgfpathlineto{\pgfqpoint{1.298194in}{0.797184in}}%
\pgfpathlineto{\pgfqpoint{1.298490in}{0.797020in}}%
\pgfpathlineto{\pgfqpoint{1.298786in}{0.796855in}}%
\pgfpathlineto{\pgfqpoint{1.299082in}{0.796691in}}%
\pgfpathlineto{\pgfqpoint{1.299378in}{0.796526in}}%
\pgfpathlineto{\pgfqpoint{1.299674in}{0.796362in}}%
\pgfpathlineto{\pgfqpoint{1.299970in}{0.796197in}}%
\pgfpathlineto{\pgfqpoint{1.300266in}{0.796033in}}%
\pgfpathlineto{\pgfqpoint{1.300562in}{0.795868in}}%
\pgfpathlineto{\pgfqpoint{1.300858in}{0.795703in}}%
\pgfpathlineto{\pgfqpoint{1.301154in}{0.795539in}}%
\pgfpathlineto{\pgfqpoint{1.301450in}{0.795374in}}%
\pgfpathlineto{\pgfqpoint{1.301746in}{0.795210in}}%
\pgfpathlineto{\pgfqpoint{1.302042in}{0.795045in}}%
\pgfpathlineto{\pgfqpoint{1.302338in}{0.794881in}}%
\pgfpathlineto{\pgfqpoint{1.302634in}{0.794716in}}%
\pgfpathlineto{\pgfqpoint{1.302930in}{0.794552in}}%
\pgfpathlineto{\pgfqpoint{1.303226in}{0.794387in}}%
\pgfpathlineto{\pgfqpoint{1.303522in}{0.794222in}}%
\pgfpathlineto{\pgfqpoint{1.303818in}{0.794058in}}%
\pgfpathlineto{\pgfqpoint{1.304114in}{0.793893in}}%
\pgfpathlineto{\pgfqpoint{1.304410in}{0.793729in}}%
\pgfpathlineto{\pgfqpoint{1.304706in}{0.793564in}}%
\pgfpathlineto{\pgfqpoint{1.305002in}{0.793400in}}%
\pgfpathlineto{\pgfqpoint{1.305298in}{0.793235in}}%
\pgfpathlineto{\pgfqpoint{1.305594in}{0.793071in}}%
\pgfpathlineto{\pgfqpoint{1.305890in}{0.792906in}}%
\pgfpathlineto{\pgfqpoint{1.306186in}{0.792741in}}%
\pgfpathlineto{\pgfqpoint{1.306482in}{0.792577in}}%
\pgfpathlineto{\pgfqpoint{1.306778in}{0.792412in}}%
\pgfpathlineto{\pgfqpoint{1.307074in}{0.792248in}}%
\pgfpathlineto{\pgfqpoint{1.307370in}{0.792083in}}%
\pgfpathlineto{\pgfqpoint{1.307666in}{0.791919in}}%
\pgfpathlineto{\pgfqpoint{1.307962in}{0.791764in}}%
\pgfpathlineto{\pgfqpoint{1.308258in}{0.791725in}}%
\pgfpathlineto{\pgfqpoint{1.308554in}{0.791722in}}%
\pgfpathlineto{\pgfqpoint{1.308850in}{0.791719in}}%
\pgfpathlineto{\pgfqpoint{1.309146in}{0.791715in}}%
\pgfpathlineto{\pgfqpoint{1.309442in}{0.791712in}}%
\pgfpathlineto{\pgfqpoint{1.309738in}{0.791709in}}%
\pgfpathlineto{\pgfqpoint{1.310034in}{0.791705in}}%
\pgfpathlineto{\pgfqpoint{1.310330in}{0.791702in}}%
\pgfpathlineto{\pgfqpoint{1.310626in}{0.791699in}}%
\pgfpathlineto{\pgfqpoint{1.310922in}{0.791695in}}%
\pgfpathlineto{\pgfqpoint{1.311218in}{0.791692in}}%
\pgfpathlineto{\pgfqpoint{1.311514in}{0.791689in}}%
\pgfpathlineto{\pgfqpoint{1.311810in}{0.791685in}}%
\pgfpathlineto{\pgfqpoint{1.312106in}{0.791682in}}%
\pgfpathlineto{\pgfqpoint{1.312402in}{0.791679in}}%
\pgfpathlineto{\pgfqpoint{1.312698in}{0.791675in}}%
\pgfpathlineto{\pgfqpoint{1.312994in}{0.791672in}}%
\pgfpathlineto{\pgfqpoint{1.313290in}{0.791669in}}%
\pgfpathlineto{\pgfqpoint{1.313586in}{0.791665in}}%
\pgfpathlineto{\pgfqpoint{1.313882in}{0.792642in}}%
\pgfpathlineto{\pgfqpoint{1.314178in}{0.794566in}}%
\pgfpathlineto{\pgfqpoint{1.314474in}{0.796489in}}%
\pgfpathlineto{\pgfqpoint{1.314770in}{0.798414in}}%
\pgfpathlineto{\pgfqpoint{1.315066in}{0.800340in}}%
\pgfpathlineto{\pgfqpoint{1.315362in}{0.801785in}}%
\pgfpathlineto{\pgfqpoint{1.315658in}{0.797676in}}%
\pgfpathlineto{\pgfqpoint{1.315954in}{0.802889in}}%
\pgfpathlineto{\pgfqpoint{1.316250in}{0.802916in}}%
\pgfpathlineto{\pgfqpoint{1.316546in}{0.802917in}}%
\pgfpathlineto{\pgfqpoint{1.316842in}{0.802919in}}%
\pgfpathlineto{\pgfqpoint{1.317138in}{0.802920in}}%
\pgfpathlineto{\pgfqpoint{1.317434in}{0.802922in}}%
\pgfpathlineto{\pgfqpoint{1.317730in}{0.802924in}}%
\pgfpathlineto{\pgfqpoint{1.318026in}{0.802925in}}%
\pgfpathlineto{\pgfqpoint{1.318322in}{0.802927in}}%
\pgfpathlineto{\pgfqpoint{1.318618in}{0.802928in}}%
\pgfpathlineto{\pgfqpoint{1.318914in}{0.802930in}}%
\pgfpathlineto{\pgfqpoint{1.319210in}{0.802932in}}%
\pgfpathlineto{\pgfqpoint{1.319506in}{0.802933in}}%
\pgfpathlineto{\pgfqpoint{1.319802in}{0.802935in}}%
\pgfpathlineto{\pgfqpoint{1.320098in}{0.802936in}}%
\pgfpathlineto{\pgfqpoint{1.320394in}{0.799928in}}%
\pgfpathlineto{\pgfqpoint{1.320690in}{0.802934in}}%
\pgfpathlineto{\pgfqpoint{1.320986in}{0.802932in}}%
\pgfpathlineto{\pgfqpoint{1.321282in}{0.802930in}}%
\pgfpathlineto{\pgfqpoint{1.321578in}{0.802579in}}%
\pgfpathlineto{\pgfqpoint{1.321874in}{0.802294in}}%
\pgfpathlineto{\pgfqpoint{1.322171in}{0.802290in}}%
\pgfpathlineto{\pgfqpoint{1.322467in}{0.802287in}}%
\pgfpathlineto{\pgfqpoint{1.322763in}{0.802284in}}%
\pgfpathlineto{\pgfqpoint{1.323059in}{0.802280in}}%
\pgfpathlineto{\pgfqpoint{1.323355in}{0.802277in}}%
\pgfpathlineto{\pgfqpoint{1.323651in}{0.802274in}}%
\pgfpathlineto{\pgfqpoint{1.323947in}{0.802270in}}%
\pgfpathlineto{\pgfqpoint{1.324243in}{0.802267in}}%
\pgfpathlineto{\pgfqpoint{1.324539in}{0.802264in}}%
\pgfpathlineto{\pgfqpoint{1.324835in}{0.802261in}}%
\pgfpathlineto{\pgfqpoint{1.325131in}{0.802257in}}%
\pgfpathlineto{\pgfqpoint{1.325427in}{0.802254in}}%
\pgfpathlineto{\pgfqpoint{1.325723in}{0.802251in}}%
\pgfpathlineto{\pgfqpoint{1.326019in}{0.802247in}}%
\pgfpathlineto{\pgfqpoint{1.326315in}{0.802244in}}%
\pgfpathlineto{\pgfqpoint{1.326611in}{0.802241in}}%
\pgfpathlineto{\pgfqpoint{1.326907in}{0.802238in}}%
\pgfpathlineto{\pgfqpoint{1.327203in}{0.802234in}}%
\pgfpathlineto{\pgfqpoint{1.327499in}{0.802231in}}%
\pgfpathlineto{\pgfqpoint{1.327795in}{0.802228in}}%
\pgfpathlineto{\pgfqpoint{1.328091in}{0.802224in}}%
\pgfpathlineto{\pgfqpoint{1.328387in}{0.802221in}}%
\pgfpathlineto{\pgfqpoint{1.328683in}{0.803350in}}%
\pgfpathlineto{\pgfqpoint{1.328979in}{0.804479in}}%
\pgfpathlineto{\pgfqpoint{1.329275in}{0.804475in}}%
\pgfpathlineto{\pgfqpoint{1.329571in}{0.804471in}}%
\pgfpathlineto{\pgfqpoint{1.329867in}{0.804467in}}%
\pgfpathlineto{\pgfqpoint{1.330163in}{0.804463in}}%
\pgfpathlineto{\pgfqpoint{1.330459in}{0.804458in}}%
\pgfpathlineto{\pgfqpoint{1.330755in}{0.804454in}}%
\pgfpathlineto{\pgfqpoint{1.331051in}{0.804450in}}%
\pgfpathlineto{\pgfqpoint{1.331347in}{0.804446in}}%
\pgfpathlineto{\pgfqpoint{1.331643in}{0.804442in}}%
\pgfpathlineto{\pgfqpoint{1.331939in}{0.804438in}}%
\pgfpathlineto{\pgfqpoint{1.332235in}{0.804434in}}%
\pgfpathlineto{\pgfqpoint{1.332531in}{0.804430in}}%
\pgfpathlineto{\pgfqpoint{1.332827in}{0.804426in}}%
\pgfpathlineto{\pgfqpoint{1.333123in}{0.804422in}}%
\pgfpathlineto{\pgfqpoint{1.333419in}{0.804418in}}%
\pgfpathlineto{\pgfqpoint{1.333715in}{0.804414in}}%
\pgfpathlineto{\pgfqpoint{1.334011in}{0.804410in}}%
\pgfpathlineto{\pgfqpoint{1.334307in}{0.804405in}}%
\pgfpathlineto{\pgfqpoint{1.334603in}{0.804401in}}%
\pgfpathlineto{\pgfqpoint{1.334899in}{0.804397in}}%
\pgfpathlineto{\pgfqpoint{1.335195in}{0.804398in}}%
\pgfpathlineto{\pgfqpoint{1.335491in}{0.804714in}}%
\pgfpathlineto{\pgfqpoint{1.335787in}{0.805216in}}%
\pgfpathlineto{\pgfqpoint{1.336083in}{0.805717in}}%
\pgfpathlineto{\pgfqpoint{1.336379in}{0.806219in}}%
\pgfpathlineto{\pgfqpoint{1.336675in}{0.806721in}}%
\pgfpathlineto{\pgfqpoint{1.336971in}{0.807223in}}%
\pgfpathlineto{\pgfqpoint{1.337267in}{0.807725in}}%
\pgfpathlineto{\pgfqpoint{1.337563in}{0.808227in}}%
\pgfpathlineto{\pgfqpoint{1.337859in}{0.808728in}}%
\pgfpathlineto{\pgfqpoint{1.338155in}{0.809230in}}%
\pgfpathlineto{\pgfqpoint{1.338451in}{0.809732in}}%
\pgfpathlineto{\pgfqpoint{1.338747in}{0.810234in}}%
\pgfpathlineto{\pgfqpoint{1.339043in}{0.810736in}}%
\pgfpathlineto{\pgfqpoint{1.339339in}{0.811237in}}%
\pgfpathlineto{\pgfqpoint{1.339635in}{0.811739in}}%
\pgfpathlineto{\pgfqpoint{1.339931in}{0.812241in}}%
\pgfpathlineto{\pgfqpoint{1.340227in}{0.812743in}}%
\pgfpathlineto{\pgfqpoint{1.340523in}{0.813245in}}%
\pgfpathlineto{\pgfqpoint{1.340819in}{0.813747in}}%
\pgfpathlineto{\pgfqpoint{1.341115in}{0.814248in}}%
\pgfpathlineto{\pgfqpoint{1.341411in}{0.814750in}}%
\pgfpathlineto{\pgfqpoint{1.341707in}{0.815252in}}%
\pgfpathlineto{\pgfqpoint{1.342003in}{0.815754in}}%
\pgfpathlineto{\pgfqpoint{1.342299in}{0.816087in}}%
\pgfpathlineto{\pgfqpoint{1.342595in}{0.816100in}}%
\pgfpathlineto{\pgfqpoint{1.342891in}{0.816106in}}%
\pgfpathlineto{\pgfqpoint{1.343187in}{0.816112in}}%
\pgfpathlineto{\pgfqpoint{1.343483in}{0.816118in}}%
\pgfpathlineto{\pgfqpoint{1.343779in}{0.816123in}}%
\pgfpathlineto{\pgfqpoint{1.344075in}{0.816129in}}%
\pgfpathlineto{\pgfqpoint{1.344371in}{0.816135in}}%
\pgfpathlineto{\pgfqpoint{1.344667in}{0.816141in}}%
\pgfpathlineto{\pgfqpoint{1.344963in}{0.816147in}}%
\pgfpathlineto{\pgfqpoint{1.345259in}{0.816152in}}%
\pgfpathlineto{\pgfqpoint{1.345555in}{0.816158in}}%
\pgfpathlineto{\pgfqpoint{1.345851in}{0.816164in}}%
\pgfpathlineto{\pgfqpoint{1.346147in}{0.816170in}}%
\pgfpathlineto{\pgfqpoint{1.346443in}{0.816176in}}%
\pgfpathlineto{\pgfqpoint{1.346739in}{0.816181in}}%
\pgfpathlineto{\pgfqpoint{1.347035in}{0.816187in}}%
\pgfpathlineto{\pgfqpoint{1.347331in}{0.816193in}}%
\pgfpathlineto{\pgfqpoint{1.347627in}{0.816199in}}%
\pgfpathlineto{\pgfqpoint{1.347923in}{0.816204in}}%
\pgfpathlineto{\pgfqpoint{1.348219in}{0.816210in}}%
\pgfpathlineto{\pgfqpoint{1.348515in}{0.816216in}}%
\pgfpathlineto{\pgfqpoint{1.348811in}{0.816222in}}%
\pgfpathlineto{\pgfqpoint{1.349107in}{0.816228in}}%
\pgfpathlineto{\pgfqpoint{1.349403in}{0.816233in}}%
\pgfpathlineto{\pgfqpoint{1.349699in}{0.816239in}}%
\pgfpathlineto{\pgfqpoint{1.349995in}{0.816245in}}%
\pgfpathlineto{\pgfqpoint{1.350291in}{0.816251in}}%
\pgfpathlineto{\pgfqpoint{1.350587in}{0.816256in}}%
\pgfpathlineto{\pgfqpoint{1.350883in}{0.816262in}}%
\pgfpathlineto{\pgfqpoint{1.351179in}{0.816268in}}%
\pgfpathlineto{\pgfqpoint{1.351475in}{0.816274in}}%
\pgfpathlineto{\pgfqpoint{1.351771in}{0.816280in}}%
\pgfpathlineto{\pgfqpoint{1.352067in}{0.816285in}}%
\pgfpathlineto{\pgfqpoint{1.352363in}{0.816291in}}%
\pgfpathlineto{\pgfqpoint{1.352659in}{0.816297in}}%
\pgfpathlineto{\pgfqpoint{1.352955in}{0.816303in}}%
\pgfpathlineto{\pgfqpoint{1.353251in}{0.816309in}}%
\pgfpathlineto{\pgfqpoint{1.353547in}{0.816314in}}%
\pgfpathlineto{\pgfqpoint{1.353843in}{0.816320in}}%
\pgfpathlineto{\pgfqpoint{1.354139in}{0.816326in}}%
\pgfpathlineto{\pgfqpoint{1.354435in}{0.816332in}}%
\pgfpathlineto{\pgfqpoint{1.354731in}{0.816337in}}%
\pgfpathlineto{\pgfqpoint{1.355027in}{0.816343in}}%
\pgfpathlineto{\pgfqpoint{1.355323in}{0.816349in}}%
\pgfpathlineto{\pgfqpoint{1.355619in}{0.816355in}}%
\pgfpathlineto{\pgfqpoint{1.355915in}{0.816361in}}%
\pgfpathlineto{\pgfqpoint{1.356211in}{0.816366in}}%
\pgfpathlineto{\pgfqpoint{1.356507in}{0.816372in}}%
\pgfpathlineto{\pgfqpoint{1.356803in}{0.816378in}}%
\pgfpathlineto{\pgfqpoint{1.357099in}{0.816384in}}%
\pgfpathlineto{\pgfqpoint{1.357395in}{0.816390in}}%
\pgfpathlineto{\pgfqpoint{1.357691in}{0.816397in}}%
\pgfpathlineto{\pgfqpoint{1.357987in}{0.816423in}}%
\pgfpathlineto{\pgfqpoint{1.358283in}{0.816452in}}%
\pgfpathlineto{\pgfqpoint{1.358579in}{0.816482in}}%
\pgfpathlineto{\pgfqpoint{1.358875in}{0.816511in}}%
\pgfpathlineto{\pgfqpoint{1.359171in}{0.816558in}}%
\pgfpathlineto{\pgfqpoint{1.359467in}{0.816695in}}%
\pgfpathlineto{\pgfqpoint{1.359763in}{0.816844in}}%
\pgfpathlineto{\pgfqpoint{1.360059in}{0.816900in}}%
\pgfpathlineto{\pgfqpoint{1.360355in}{0.816846in}}%
\pgfpathlineto{\pgfqpoint{1.360651in}{0.816793in}}%
\pgfpathlineto{\pgfqpoint{1.360947in}{0.816739in}}%
\pgfpathlineto{\pgfqpoint{1.361243in}{0.816685in}}%
\pgfpathlineto{\pgfqpoint{1.361539in}{0.816631in}}%
\pgfpathlineto{\pgfqpoint{1.361835in}{0.816577in}}%
\pgfpathlineto{\pgfqpoint{1.362131in}{0.816523in}}%
\pgfpathlineto{\pgfqpoint{1.362427in}{0.816469in}}%
\pgfpathlineto{\pgfqpoint{1.362723in}{0.816416in}}%
\pgfpathlineto{\pgfqpoint{1.363019in}{0.816362in}}%
\pgfpathlineto{\pgfqpoint{1.363315in}{0.815673in}}%
\pgfpathlineto{\pgfqpoint{1.363611in}{0.814033in}}%
\pgfpathlineto{\pgfqpoint{1.363907in}{0.812383in}}%
\pgfpathlineto{\pgfqpoint{1.364203in}{0.810733in}}%
\pgfpathlineto{\pgfqpoint{1.364499in}{0.809083in}}%
\pgfpathlineto{\pgfqpoint{1.364795in}{0.807433in}}%
\pgfpathlineto{\pgfqpoint{1.365091in}{0.805783in}}%
\pgfpathlineto{\pgfqpoint{1.365387in}{0.809269in}}%
\pgfpathlineto{\pgfqpoint{1.365683in}{0.815763in}}%
\pgfpathlineto{\pgfqpoint{1.365979in}{0.815791in}}%
\pgfpathlineto{\pgfqpoint{1.366275in}{0.815820in}}%
\pgfpathlineto{\pgfqpoint{1.366571in}{0.815849in}}%
\pgfpathlineto{\pgfqpoint{1.366867in}{0.815877in}}%
\pgfpathlineto{\pgfqpoint{1.367163in}{0.815906in}}%
\pgfpathlineto{\pgfqpoint{1.367459in}{0.815934in}}%
\pgfpathlineto{\pgfqpoint{1.367755in}{0.815963in}}%
\pgfpathlineto{\pgfqpoint{1.368051in}{0.815991in}}%
\pgfpathlineto{\pgfqpoint{1.368347in}{0.816020in}}%
\pgfpathlineto{\pgfqpoint{1.368643in}{0.816048in}}%
\pgfpathlineto{\pgfqpoint{1.368939in}{0.816077in}}%
\pgfpathlineto{\pgfqpoint{1.369235in}{0.816105in}}%
\pgfpathlineto{\pgfqpoint{1.369531in}{0.816134in}}%
\pgfpathlineto{\pgfqpoint{1.369827in}{0.816162in}}%
\pgfpathlineto{\pgfqpoint{1.370123in}{0.816191in}}%
\pgfpathlineto{\pgfqpoint{1.370419in}{0.816219in}}%
\pgfpathlineto{\pgfqpoint{1.370715in}{0.816234in}}%
\pgfpathlineto{\pgfqpoint{1.371011in}{0.816208in}}%
\pgfpathlineto{\pgfqpoint{1.371307in}{0.816178in}}%
\pgfpathlineto{\pgfqpoint{1.371603in}{0.816149in}}%
\pgfpathlineto{\pgfqpoint{1.371899in}{0.816119in}}%
\pgfpathlineto{\pgfqpoint{1.372195in}{0.816089in}}%
\pgfpathlineto{\pgfqpoint{1.372491in}{0.816060in}}%
\pgfpathlineto{\pgfqpoint{1.372787in}{0.816030in}}%
\pgfpathlineto{\pgfqpoint{1.373083in}{0.816000in}}%
\pgfpathlineto{\pgfqpoint{1.373379in}{0.815971in}}%
\pgfpathlineto{\pgfqpoint{1.373675in}{0.815941in}}%
\pgfpathlineto{\pgfqpoint{1.373971in}{0.815912in}}%
\pgfpathlineto{\pgfqpoint{1.374267in}{0.815882in}}%
\pgfpathlineto{\pgfqpoint{1.374563in}{0.815852in}}%
\pgfpathlineto{\pgfqpoint{1.374859in}{0.815823in}}%
\pgfpathlineto{\pgfqpoint{1.375155in}{0.815793in}}%
\pgfpathlineto{\pgfqpoint{1.375451in}{0.815763in}}%
\pgfpathlineto{\pgfqpoint{1.375747in}{0.815734in}}%
\pgfpathlineto{\pgfqpoint{1.376043in}{0.815704in}}%
\pgfpathlineto{\pgfqpoint{1.376339in}{0.815675in}}%
\pgfpathlineto{\pgfqpoint{1.376635in}{0.815645in}}%
\pgfpathlineto{\pgfqpoint{1.376931in}{0.815615in}}%
\pgfpathlineto{\pgfqpoint{1.377227in}{0.815819in}}%
\pgfpathlineto{\pgfqpoint{1.377523in}{0.817074in}}%
\pgfpathlineto{\pgfqpoint{1.377819in}{0.818106in}}%
\pgfpathlineto{\pgfqpoint{1.378115in}{0.815077in}}%
\pgfpathlineto{\pgfqpoint{1.378411in}{0.810647in}}%
\pgfpathlineto{\pgfqpoint{1.378707in}{0.806217in}}%
\pgfpathlineto{\pgfqpoint{1.379003in}{0.801787in}}%
\pgfpathlineto{\pgfqpoint{1.379299in}{0.797358in}}%
\pgfpathlineto{\pgfqpoint{1.379595in}{0.792928in}}%
\pgfpathlineto{\pgfqpoint{1.379891in}{0.790847in}}%
\pgfpathlineto{\pgfqpoint{1.380187in}{0.792291in}}%
\pgfpathlineto{\pgfqpoint{1.380483in}{0.793770in}}%
\pgfpathlineto{\pgfqpoint{1.380779in}{0.795250in}}%
\pgfpathlineto{\pgfqpoint{1.381075in}{0.796729in}}%
\pgfpathlineto{\pgfqpoint{1.381371in}{0.798208in}}%
\pgfpathlineto{\pgfqpoint{1.381667in}{0.799687in}}%
\pgfpathlineto{\pgfqpoint{1.381963in}{0.801166in}}%
\pgfpathlineto{\pgfqpoint{1.382259in}{0.802645in}}%
\pgfpathlineto{\pgfqpoint{1.382555in}{0.804124in}}%
\pgfpathlineto{\pgfqpoint{1.382851in}{0.805603in}}%
\pgfpathlineto{\pgfqpoint{1.383147in}{0.807082in}}%
\pgfpathlineto{\pgfqpoint{1.383443in}{0.808562in}}%
\pgfpathlineto{\pgfqpoint{1.383739in}{0.810041in}}%
\pgfpathlineto{\pgfqpoint{1.384035in}{0.811520in}}%
\pgfpathlineto{\pgfqpoint{1.384331in}{0.812999in}}%
\pgfpathlineto{\pgfqpoint{1.384627in}{0.814478in}}%
\pgfpathlineto{\pgfqpoint{1.384923in}{0.815957in}}%
\pgfpathlineto{\pgfqpoint{1.385219in}{0.817436in}}%
\pgfpathlineto{\pgfqpoint{1.385515in}{0.818403in}}%
\pgfpathlineto{\pgfqpoint{1.385811in}{0.818727in}}%
\pgfpathlineto{\pgfqpoint{1.386107in}{0.819049in}}%
\pgfpathlineto{\pgfqpoint{1.386403in}{0.819279in}}%
\pgfpathlineto{\pgfqpoint{1.386699in}{0.819109in}}%
\pgfpathlineto{\pgfqpoint{1.386995in}{0.819067in}}%
\pgfpathlineto{\pgfqpoint{1.387291in}{0.819094in}}%
\pgfpathlineto{\pgfqpoint{1.387587in}{0.819120in}}%
\pgfpathlineto{\pgfqpoint{1.387883in}{0.819147in}}%
\pgfpathlineto{\pgfqpoint{1.388179in}{0.819173in}}%
\pgfpathlineto{\pgfqpoint{1.388475in}{0.819200in}}%
\pgfpathlineto{\pgfqpoint{1.388771in}{0.819226in}}%
\pgfpathlineto{\pgfqpoint{1.389067in}{0.819253in}}%
\pgfpathlineto{\pgfqpoint{1.389363in}{0.819279in}}%
\pgfpathlineto{\pgfqpoint{1.389660in}{0.819305in}}%
\pgfpathlineto{\pgfqpoint{1.389956in}{0.819332in}}%
\pgfpathlineto{\pgfqpoint{1.390252in}{0.819358in}}%
\pgfpathlineto{\pgfqpoint{1.390548in}{0.819385in}}%
\pgfpathlineto{\pgfqpoint{1.390844in}{0.819411in}}%
\pgfpathlineto{\pgfqpoint{1.391140in}{0.819438in}}%
\pgfpathlineto{\pgfqpoint{1.391436in}{0.818935in}}%
\pgfpathlineto{\pgfqpoint{1.391732in}{0.817929in}}%
\pgfpathlineto{\pgfqpoint{1.392028in}{0.817929in}}%
\pgfpathlineto{\pgfqpoint{1.392324in}{0.817930in}}%
\pgfpathlineto{\pgfqpoint{1.392620in}{0.817934in}}%
\pgfpathlineto{\pgfqpoint{1.392916in}{0.817929in}}%
\pgfpathlineto{\pgfqpoint{1.393212in}{0.817931in}}%
\pgfpathlineto{\pgfqpoint{1.393508in}{0.817953in}}%
\pgfpathlineto{\pgfqpoint{1.393804in}{0.817975in}}%
\pgfpathlineto{\pgfqpoint{1.394100in}{0.817998in}}%
\pgfpathlineto{\pgfqpoint{1.394396in}{0.818020in}}%
\pgfpathlineto{\pgfqpoint{1.394692in}{0.818042in}}%
\pgfpathlineto{\pgfqpoint{1.394988in}{0.818064in}}%
\pgfpathlineto{\pgfqpoint{1.395284in}{0.818086in}}%
\pgfpathlineto{\pgfqpoint{1.395580in}{0.818108in}}%
\pgfpathlineto{\pgfqpoint{1.395876in}{0.818130in}}%
\pgfpathlineto{\pgfqpoint{1.396172in}{0.818153in}}%
\pgfpathlineto{\pgfqpoint{1.396468in}{0.818175in}}%
\pgfpathlineto{\pgfqpoint{1.396764in}{0.818197in}}%
\pgfpathlineto{\pgfqpoint{1.397060in}{0.818219in}}%
\pgfpathlineto{\pgfqpoint{1.397356in}{0.818241in}}%
\pgfpathlineto{\pgfqpoint{1.397652in}{0.818263in}}%
\pgfpathlineto{\pgfqpoint{1.397948in}{0.818286in}}%
\pgfpathlineto{\pgfqpoint{1.398244in}{0.818308in}}%
\pgfpathlineto{\pgfqpoint{1.398540in}{0.818330in}}%
\pgfpathlineto{\pgfqpoint{1.398836in}{0.818352in}}%
\pgfpathlineto{\pgfqpoint{1.399132in}{0.818374in}}%
\pgfpathlineto{\pgfqpoint{1.399428in}{0.818396in}}%
\pgfpathlineto{\pgfqpoint{1.399724in}{0.818419in}}%
\pgfpathlineto{\pgfqpoint{1.400020in}{0.818441in}}%
\pgfpathlineto{\pgfqpoint{1.400316in}{0.818463in}}%
\pgfpathlineto{\pgfqpoint{1.400612in}{0.818485in}}%
\pgfpathlineto{\pgfqpoint{1.400908in}{0.818507in}}%
\pgfpathlineto{\pgfqpoint{1.401204in}{0.818529in}}%
\pgfpathlineto{\pgfqpoint{1.401500in}{0.818552in}}%
\pgfpathlineto{\pgfqpoint{1.401796in}{0.818574in}}%
\pgfpathlineto{\pgfqpoint{1.402092in}{0.818596in}}%
\pgfpathlineto{\pgfqpoint{1.402388in}{0.818618in}}%
\pgfpathlineto{\pgfqpoint{1.402684in}{0.818640in}}%
\pgfpathlineto{\pgfqpoint{1.402980in}{0.818662in}}%
\pgfpathlineto{\pgfqpoint{1.403276in}{0.818685in}}%
\pgfpathlineto{\pgfqpoint{1.403572in}{0.818707in}}%
\pgfpathlineto{\pgfqpoint{1.403868in}{0.818729in}}%
\pgfpathlineto{\pgfqpoint{1.404164in}{0.818751in}}%
\pgfpathlineto{\pgfqpoint{1.404460in}{0.818773in}}%
\pgfpathlineto{\pgfqpoint{1.404756in}{0.818795in}}%
\pgfpathlineto{\pgfqpoint{1.405052in}{0.818817in}}%
\pgfpathlineto{\pgfqpoint{1.405348in}{0.818840in}}%
\pgfpathlineto{\pgfqpoint{1.405644in}{0.818862in}}%
\pgfpathlineto{\pgfqpoint{1.405940in}{0.818884in}}%
\pgfpathlineto{\pgfqpoint{1.406236in}{0.818906in}}%
\pgfpathlineto{\pgfqpoint{1.406532in}{0.818928in}}%
\pgfpathlineto{\pgfqpoint{1.406828in}{0.818950in}}%
\pgfpathlineto{\pgfqpoint{1.407124in}{0.818973in}}%
\pgfpathlineto{\pgfqpoint{1.407420in}{0.818995in}}%
\pgfpathlineto{\pgfqpoint{1.407716in}{0.819017in}}%
\pgfpathlineto{\pgfqpoint{1.408012in}{0.819039in}}%
\pgfpathlineto{\pgfqpoint{1.408308in}{0.819061in}}%
\pgfpathlineto{\pgfqpoint{1.408604in}{0.819083in}}%
\pgfpathlineto{\pgfqpoint{1.408900in}{0.819106in}}%
\pgfpathlineto{\pgfqpoint{1.409196in}{0.819128in}}%
\pgfpathlineto{\pgfqpoint{1.409492in}{0.819150in}}%
\pgfpathlineto{\pgfqpoint{1.409788in}{0.819172in}}%
\pgfpathlineto{\pgfqpoint{1.410084in}{0.819194in}}%
\pgfpathlineto{\pgfqpoint{1.410380in}{0.819216in}}%
\pgfpathlineto{\pgfqpoint{1.410676in}{0.819239in}}%
\pgfpathlineto{\pgfqpoint{1.410972in}{0.819261in}}%
\pgfpathlineto{\pgfqpoint{1.411268in}{0.819283in}}%
\pgfpathlineto{\pgfqpoint{1.411564in}{0.819305in}}%
\pgfpathlineto{\pgfqpoint{1.411860in}{0.819327in}}%
\pgfpathlineto{\pgfqpoint{1.412156in}{0.819349in}}%
\pgfpathlineto{\pgfqpoint{1.412452in}{0.819372in}}%
\pgfpathlineto{\pgfqpoint{1.412748in}{0.819394in}}%
\pgfpathlineto{\pgfqpoint{1.413044in}{0.819416in}}%
\pgfpathlineto{\pgfqpoint{1.413340in}{0.819438in}}%
\pgfpathlineto{\pgfqpoint{1.413636in}{0.819460in}}%
\pgfpathlineto{\pgfqpoint{1.413932in}{0.819399in}}%
\pgfpathlineto{\pgfqpoint{1.414228in}{0.819040in}}%
\pgfpathlineto{\pgfqpoint{1.414524in}{0.818655in}}%
\pgfpathlineto{\pgfqpoint{1.414820in}{0.818270in}}%
\pgfpathlineto{\pgfqpoint{1.415116in}{0.817946in}}%
\pgfpathlineto{\pgfqpoint{1.415412in}{0.817939in}}%
\pgfpathlineto{\pgfqpoint{1.415708in}{0.817894in}}%
\pgfpathlineto{\pgfqpoint{1.416004in}{0.817893in}}%
\pgfpathlineto{\pgfqpoint{1.416300in}{0.817892in}}%
\pgfpathlineto{\pgfqpoint{1.416596in}{0.817890in}}%
\pgfpathlineto{\pgfqpoint{1.416892in}{0.817889in}}%
\pgfpathlineto{\pgfqpoint{1.417188in}{0.817888in}}%
\pgfpathlineto{\pgfqpoint{1.417484in}{0.817887in}}%
\pgfpathlineto{\pgfqpoint{1.417780in}{0.817885in}}%
\pgfpathlineto{\pgfqpoint{1.418076in}{0.817884in}}%
\pgfpathlineto{\pgfqpoint{1.418372in}{0.817883in}}%
\pgfpathlineto{\pgfqpoint{1.418668in}{0.817882in}}%
\pgfpathlineto{\pgfqpoint{1.418964in}{0.817880in}}%
\pgfpathlineto{\pgfqpoint{1.419260in}{0.817879in}}%
\pgfpathlineto{\pgfqpoint{1.419556in}{0.817878in}}%
\pgfpathlineto{\pgfqpoint{1.419852in}{0.817876in}}%
\pgfpathlineto{\pgfqpoint{1.420148in}{0.817875in}}%
\pgfpathlineto{\pgfqpoint{1.420444in}{0.817874in}}%
\pgfpathlineto{\pgfqpoint{1.420740in}{0.817886in}}%
\pgfpathlineto{\pgfqpoint{1.421036in}{0.817897in}}%
\pgfpathlineto{\pgfqpoint{1.421332in}{0.817893in}}%
\pgfpathlineto{\pgfqpoint{1.421628in}{0.817888in}}%
\pgfpathlineto{\pgfqpoint{1.421924in}{0.817884in}}%
\pgfpathlineto{\pgfqpoint{1.422220in}{0.817868in}}%
\pgfpathlineto{\pgfqpoint{1.422516in}{0.817840in}}%
\pgfpathlineto{\pgfqpoint{1.422812in}{0.817816in}}%
\pgfpathlineto{\pgfqpoint{1.423108in}{0.817809in}}%
\pgfpathlineto{\pgfqpoint{1.423404in}{0.817803in}}%
\pgfpathlineto{\pgfqpoint{1.423700in}{0.817797in}}%
\pgfpathlineto{\pgfqpoint{1.423996in}{0.817791in}}%
\pgfpathlineto{\pgfqpoint{1.424292in}{0.817785in}}%
\pgfpathlineto{\pgfqpoint{1.424588in}{0.817779in}}%
\pgfpathlineto{\pgfqpoint{1.424884in}{0.817773in}}%
\pgfpathlineto{\pgfqpoint{1.425180in}{0.817767in}}%
\pgfpathlineto{\pgfqpoint{1.425476in}{0.817761in}}%
\pgfpathlineto{\pgfqpoint{1.425772in}{0.817755in}}%
\pgfpathlineto{\pgfqpoint{1.426068in}{0.817749in}}%
\pgfpathlineto{\pgfqpoint{1.426364in}{0.817743in}}%
\pgfpathlineto{\pgfqpoint{1.426660in}{0.817736in}}%
\pgfpathlineto{\pgfqpoint{1.426956in}{0.817730in}}%
\pgfpathlineto{\pgfqpoint{1.427252in}{0.817760in}}%
\pgfpathlineto{\pgfqpoint{1.427548in}{0.818617in}}%
\pgfpathlineto{\pgfqpoint{1.427844in}{0.819316in}}%
\pgfpathlineto{\pgfqpoint{1.428140in}{0.819322in}}%
\pgfpathlineto{\pgfqpoint{1.428436in}{0.819318in}}%
\pgfpathlineto{\pgfqpoint{1.428732in}{0.819256in}}%
\pgfpathlineto{\pgfqpoint{1.429028in}{0.818616in}}%
\pgfpathlineto{\pgfqpoint{1.429324in}{0.818292in}}%
\pgfpathlineto{\pgfqpoint{1.429620in}{0.818286in}}%
\pgfpathlineto{\pgfqpoint{1.429916in}{0.818342in}}%
\pgfpathlineto{\pgfqpoint{1.430212in}{0.818397in}}%
\pgfpathlineto{\pgfqpoint{1.430508in}{0.818452in}}%
\pgfpathlineto{\pgfqpoint{1.430804in}{0.818508in}}%
\pgfpathlineto{\pgfqpoint{1.431100in}{0.818563in}}%
\pgfpathlineto{\pgfqpoint{1.431396in}{0.818619in}}%
\pgfpathlineto{\pgfqpoint{1.431692in}{0.818674in}}%
\pgfpathlineto{\pgfqpoint{1.431988in}{0.818730in}}%
\pgfpathlineto{\pgfqpoint{1.432284in}{0.818785in}}%
\pgfpathlineto{\pgfqpoint{1.432580in}{0.818841in}}%
\pgfpathlineto{\pgfqpoint{1.432876in}{0.818896in}}%
\pgfpathlineto{\pgfqpoint{1.433172in}{0.818951in}}%
\pgfpathlineto{\pgfqpoint{1.433468in}{0.819007in}}%
\pgfpathlineto{\pgfqpoint{1.433764in}{0.819062in}}%
\pgfpathlineto{\pgfqpoint{1.434060in}{0.819118in}}%
\pgfpathlineto{\pgfqpoint{1.434356in}{0.819173in}}%
\pgfpathlineto{\pgfqpoint{1.434652in}{0.819229in}}%
\pgfpathlineto{\pgfqpoint{1.434948in}{0.819290in}}%
\pgfpathlineto{\pgfqpoint{1.435244in}{0.819538in}}%
\pgfpathlineto{\pgfqpoint{1.435540in}{0.819858in}}%
\pgfpathlineto{\pgfqpoint{1.435836in}{0.819935in}}%
\pgfpathlineto{\pgfqpoint{1.436132in}{0.819931in}}%
\pgfpathlineto{\pgfqpoint{1.436428in}{0.819928in}}%
\pgfpathlineto{\pgfqpoint{1.436724in}{0.819924in}}%
\pgfpathlineto{\pgfqpoint{1.437020in}{0.819920in}}%
\pgfpathlineto{\pgfqpoint{1.437316in}{0.819917in}}%
\pgfpathlineto{\pgfqpoint{1.437612in}{0.819913in}}%
\pgfpathlineto{\pgfqpoint{1.437908in}{0.819909in}}%
\pgfpathlineto{\pgfqpoint{1.438204in}{0.819906in}}%
\pgfpathlineto{\pgfqpoint{1.438500in}{0.819902in}}%
\pgfpathlineto{\pgfqpoint{1.438796in}{0.819898in}}%
\pgfpathlineto{\pgfqpoint{1.439092in}{0.819894in}}%
\pgfpathlineto{\pgfqpoint{1.439388in}{0.819891in}}%
\pgfpathlineto{\pgfqpoint{1.439684in}{0.819887in}}%
\pgfpathlineto{\pgfqpoint{1.439980in}{0.819883in}}%
\pgfpathlineto{\pgfqpoint{1.440276in}{0.819880in}}%
\pgfpathlineto{\pgfqpoint{1.440572in}{0.819876in}}%
\pgfpathlineto{\pgfqpoint{1.440868in}{0.819872in}}%
\pgfpathlineto{\pgfqpoint{1.441164in}{0.819869in}}%
\pgfpathlineto{\pgfqpoint{1.441460in}{0.819865in}}%
\pgfpathlineto{\pgfqpoint{1.441756in}{0.819856in}}%
\pgfpathlineto{\pgfqpoint{1.442052in}{0.819839in}}%
\pgfpathlineto{\pgfqpoint{1.442348in}{0.819823in}}%
\pgfpathlineto{\pgfqpoint{1.442644in}{0.819805in}}%
\pgfpathlineto{\pgfqpoint{1.442940in}{0.819785in}}%
\pgfpathlineto{\pgfqpoint{1.443236in}{0.819765in}}%
\pgfpathlineto{\pgfqpoint{1.443532in}{0.819745in}}%
\pgfpathlineto{\pgfqpoint{1.443828in}{0.819724in}}%
\pgfpathlineto{\pgfqpoint{1.444124in}{0.819704in}}%
\pgfpathlineto{\pgfqpoint{1.444420in}{0.819684in}}%
\pgfpathlineto{\pgfqpoint{1.444716in}{0.819663in}}%
\pgfpathlineto{\pgfqpoint{1.445012in}{0.819643in}}%
\pgfpathlineto{\pgfqpoint{1.445308in}{0.819623in}}%
\pgfpathlineto{\pgfqpoint{1.445604in}{0.819602in}}%
\pgfpathlineto{\pgfqpoint{1.445900in}{0.819582in}}%
\pgfpathlineto{\pgfqpoint{1.446196in}{0.819562in}}%
\pgfpathlineto{\pgfqpoint{1.446492in}{0.819541in}}%
\pgfpathlineto{\pgfqpoint{1.446788in}{0.819521in}}%
\pgfpathlineto{\pgfqpoint{1.447084in}{0.819501in}}%
\pgfpathlineto{\pgfqpoint{1.447380in}{0.819480in}}%
\pgfpathlineto{\pgfqpoint{1.447676in}{0.819460in}}%
\pgfpathlineto{\pgfqpoint{1.447972in}{0.819440in}}%
\pgfpathlineto{\pgfqpoint{1.448268in}{0.819419in}}%
\pgfpathlineto{\pgfqpoint{1.448564in}{0.819399in}}%
\pgfpathlineto{\pgfqpoint{1.448860in}{0.819379in}}%
\pgfpathlineto{\pgfqpoint{1.449156in}{0.819358in}}%
\pgfpathlineto{\pgfqpoint{1.449452in}{0.819338in}}%
\pgfpathlineto{\pgfqpoint{1.449748in}{0.819318in}}%
\pgfpathlineto{\pgfqpoint{1.450044in}{0.819297in}}%
\pgfpathlineto{\pgfqpoint{1.450340in}{0.819277in}}%
\pgfpathlineto{\pgfqpoint{1.450636in}{0.819257in}}%
\pgfpathlineto{\pgfqpoint{1.450932in}{0.819236in}}%
\pgfpathlineto{\pgfqpoint{1.451228in}{0.819216in}}%
\pgfpathlineto{\pgfqpoint{1.451524in}{0.819196in}}%
\pgfpathlineto{\pgfqpoint{1.451820in}{0.819175in}}%
\pgfpathlineto{\pgfqpoint{1.452116in}{0.819155in}}%
\pgfpathlineto{\pgfqpoint{1.452412in}{0.819135in}}%
\pgfpathlineto{\pgfqpoint{1.452708in}{0.819114in}}%
\pgfpathlineto{\pgfqpoint{1.453004in}{0.819094in}}%
\pgfpathlineto{\pgfqpoint{1.453300in}{0.819073in}}%
\pgfpathlineto{\pgfqpoint{1.453596in}{0.819053in}}%
\pgfpathlineto{\pgfqpoint{1.453892in}{0.819033in}}%
\pgfpathlineto{\pgfqpoint{1.454188in}{0.819012in}}%
\pgfpathlineto{\pgfqpoint{1.454484in}{0.818992in}}%
\pgfpathlineto{\pgfqpoint{1.454780in}{0.818972in}}%
\pgfpathlineto{\pgfqpoint{1.455076in}{0.818951in}}%
\pgfpathlineto{\pgfqpoint{1.455372in}{0.818931in}}%
\pgfpathlineto{\pgfqpoint{1.455668in}{0.818911in}}%
\pgfpathlineto{\pgfqpoint{1.455964in}{0.818890in}}%
\pgfpathlineto{\pgfqpoint{1.456260in}{0.818870in}}%
\pgfpathlineto{\pgfqpoint{1.456556in}{0.818850in}}%
\pgfpathlineto{\pgfqpoint{1.456852in}{0.818829in}}%
\pgfpathlineto{\pgfqpoint{1.457149in}{0.818809in}}%
\pgfpathlineto{\pgfqpoint{1.457445in}{0.818789in}}%
\pgfpathlineto{\pgfqpoint{1.457741in}{0.818768in}}%
\pgfpathlineto{\pgfqpoint{1.458037in}{0.818748in}}%
\pgfpathlineto{\pgfqpoint{1.458333in}{0.818728in}}%
\pgfpathlineto{\pgfqpoint{1.458629in}{0.818707in}}%
\pgfpathlineto{\pgfqpoint{1.458925in}{0.818687in}}%
\pgfpathlineto{\pgfqpoint{1.459221in}{0.818667in}}%
\pgfpathlineto{\pgfqpoint{1.459517in}{0.818646in}}%
\pgfpathlineto{\pgfqpoint{1.459813in}{0.818626in}}%
\pgfpathlineto{\pgfqpoint{1.460109in}{0.818606in}}%
\pgfpathlineto{\pgfqpoint{1.460405in}{0.818585in}}%
\pgfpathlineto{\pgfqpoint{1.460701in}{0.818565in}}%
\pgfpathlineto{\pgfqpoint{1.460997in}{0.818545in}}%
\pgfpathlineto{\pgfqpoint{1.461293in}{0.818524in}}%
\pgfpathlineto{\pgfqpoint{1.461589in}{0.818504in}}%
\pgfpathlineto{\pgfqpoint{1.461885in}{0.818484in}}%
\pgfpathlineto{\pgfqpoint{1.462181in}{0.818463in}}%
\pgfpathlineto{\pgfqpoint{1.462477in}{0.818605in}}%
\pgfpathlineto{\pgfqpoint{1.462773in}{0.819288in}}%
\pgfpathlineto{\pgfqpoint{1.463069in}{0.820015in}}%
\pgfpathlineto{\pgfqpoint{1.463365in}{0.820742in}}%
\pgfpathlineto{\pgfqpoint{1.463661in}{0.821469in}}%
\pgfpathlineto{\pgfqpoint{1.463957in}{0.822196in}}%
\pgfpathlineto{\pgfqpoint{1.464253in}{0.822923in}}%
\pgfpathlineto{\pgfqpoint{1.464549in}{0.823650in}}%
\pgfpathlineto{\pgfqpoint{1.464845in}{0.824377in}}%
\pgfpathlineto{\pgfqpoint{1.465141in}{0.825043in}}%
\pgfpathlineto{\pgfqpoint{1.465437in}{0.825162in}}%
\pgfpathlineto{\pgfqpoint{1.465733in}{0.825153in}}%
\pgfpathlineto{\pgfqpoint{1.466029in}{0.825143in}}%
\pgfpathlineto{\pgfqpoint{1.466325in}{0.825134in}}%
\pgfpathlineto{\pgfqpoint{1.466621in}{0.825124in}}%
\pgfpathlineto{\pgfqpoint{1.466917in}{0.825114in}}%
\pgfpathlineto{\pgfqpoint{1.467213in}{0.825105in}}%
\pgfpathlineto{\pgfqpoint{1.467509in}{0.825095in}}%
\pgfpathlineto{\pgfqpoint{1.467805in}{0.825086in}}%
\pgfpathlineto{\pgfqpoint{1.468101in}{0.825076in}}%
\pgfpathlineto{\pgfqpoint{1.468397in}{0.825067in}}%
\pgfpathlineto{\pgfqpoint{1.468693in}{0.825057in}}%
\pgfpathlineto{\pgfqpoint{1.468989in}{0.825048in}}%
\pgfpathlineto{\pgfqpoint{1.469285in}{0.825038in}}%
\pgfpathlineto{\pgfqpoint{1.469581in}{0.825029in}}%
\pgfpathlineto{\pgfqpoint{1.469877in}{0.825019in}}%
\pgfpathlineto{\pgfqpoint{1.470173in}{0.825010in}}%
\pgfpathlineto{\pgfqpoint{1.470469in}{0.825000in}}%
\pgfpathlineto{\pgfqpoint{1.470765in}{0.824990in}}%
\pgfpathlineto{\pgfqpoint{1.471061in}{0.824981in}}%
\pgfpathlineto{\pgfqpoint{1.471357in}{0.824971in}}%
\pgfpathlineto{\pgfqpoint{1.471653in}{0.825591in}}%
\pgfpathlineto{\pgfqpoint{1.471949in}{0.825867in}}%
\pgfpathlineto{\pgfqpoint{1.472245in}{0.825944in}}%
\pgfpathlineto{\pgfqpoint{1.472541in}{0.826021in}}%
\pgfpathlineto{\pgfqpoint{1.472837in}{0.826098in}}%
\pgfpathlineto{\pgfqpoint{1.473133in}{0.826175in}}%
\pgfpathlineto{\pgfqpoint{1.473429in}{0.826252in}}%
\pgfpathlineto{\pgfqpoint{1.473725in}{0.826329in}}%
\pgfpathlineto{\pgfqpoint{1.474021in}{0.826406in}}%
\pgfpathlineto{\pgfqpoint{1.474317in}{0.826483in}}%
\pgfpathlineto{\pgfqpoint{1.474613in}{0.826560in}}%
\pgfpathlineto{\pgfqpoint{1.474909in}{0.826637in}}%
\pgfpathlineto{\pgfqpoint{1.475205in}{0.826714in}}%
\pgfpathlineto{\pgfqpoint{1.475501in}{0.826791in}}%
\pgfpathlineto{\pgfqpoint{1.475797in}{0.826868in}}%
\pgfpathlineto{\pgfqpoint{1.476093in}{0.826945in}}%
\pgfpathlineto{\pgfqpoint{1.476389in}{0.827022in}}%
\pgfpathlineto{\pgfqpoint{1.476685in}{0.827341in}}%
\pgfpathlineto{\pgfqpoint{1.476981in}{0.824727in}}%
\pgfpathlineto{\pgfqpoint{1.477277in}{0.829926in}}%
\pgfpathlineto{\pgfqpoint{1.477573in}{0.829929in}}%
\pgfpathlineto{\pgfqpoint{1.477869in}{0.829943in}}%
\pgfpathlineto{\pgfqpoint{1.478165in}{0.829957in}}%
\pgfpathlineto{\pgfqpoint{1.478461in}{0.829971in}}%
\pgfpathlineto{\pgfqpoint{1.478757in}{0.829985in}}%
\pgfpathlineto{\pgfqpoint{1.479053in}{0.829999in}}%
\pgfpathlineto{\pgfqpoint{1.479349in}{0.830012in}}%
\pgfpathlineto{\pgfqpoint{1.479645in}{0.830026in}}%
\pgfpathlineto{\pgfqpoint{1.479941in}{0.830040in}}%
\pgfpathlineto{\pgfqpoint{1.480237in}{0.830054in}}%
\pgfpathlineto{\pgfqpoint{1.480533in}{0.830068in}}%
\pgfpathlineto{\pgfqpoint{1.480829in}{0.830082in}}%
\pgfpathlineto{\pgfqpoint{1.481125in}{0.830096in}}%
\pgfpathlineto{\pgfqpoint{1.481421in}{0.830109in}}%
\pgfpathlineto{\pgfqpoint{1.481717in}{0.830123in}}%
\pgfpathlineto{\pgfqpoint{1.482013in}{0.830137in}}%
\pgfpathlineto{\pgfqpoint{1.482309in}{0.830151in}}%
\pgfpathlineto{\pgfqpoint{1.482605in}{0.830165in}}%
\pgfpathlineto{\pgfqpoint{1.482901in}{0.830179in}}%
\pgfpathlineto{\pgfqpoint{1.483197in}{0.830192in}}%
\pgfpathlineto{\pgfqpoint{1.483493in}{0.830206in}}%
\pgfpathlineto{\pgfqpoint{1.483789in}{0.830220in}}%
\pgfpathlineto{\pgfqpoint{1.484085in}{0.830234in}}%
\pgfpathlineto{\pgfqpoint{1.484381in}{0.830248in}}%
\pgfpathlineto{\pgfqpoint{1.484677in}{0.830173in}}%
\pgfpathlineto{\pgfqpoint{1.484973in}{0.829695in}}%
\pgfpathlineto{\pgfqpoint{1.485269in}{0.830447in}}%
\pgfpathlineto{\pgfqpoint{1.485565in}{0.830679in}}%
\pgfpathlineto{\pgfqpoint{1.485861in}{0.832254in}}%
\pgfpathlineto{\pgfqpoint{1.486157in}{0.832246in}}%
\pgfpathlineto{\pgfqpoint{1.486453in}{0.832237in}}%
\pgfpathlineto{\pgfqpoint{1.486749in}{0.832229in}}%
\pgfpathlineto{\pgfqpoint{1.487045in}{0.832221in}}%
\pgfpathlineto{\pgfqpoint{1.487341in}{0.832212in}}%
\pgfpathlineto{\pgfqpoint{1.487637in}{0.832204in}}%
\pgfpathlineto{\pgfqpoint{1.487933in}{0.832196in}}%
\pgfpathlineto{\pgfqpoint{1.488229in}{0.832188in}}%
\pgfpathlineto{\pgfqpoint{1.488525in}{0.832179in}}%
\pgfpathlineto{\pgfqpoint{1.488821in}{0.832171in}}%
\pgfpathlineto{\pgfqpoint{1.489117in}{0.832163in}}%
\pgfpathlineto{\pgfqpoint{1.489413in}{0.832154in}}%
\pgfpathlineto{\pgfqpoint{1.489709in}{0.832146in}}%
\pgfpathlineto{\pgfqpoint{1.490005in}{0.832138in}}%
\pgfpathlineto{\pgfqpoint{1.490301in}{0.832129in}}%
\pgfpathlineto{\pgfqpoint{1.490597in}{0.832121in}}%
\pgfpathlineto{\pgfqpoint{1.490893in}{0.832113in}}%
\pgfpathlineto{\pgfqpoint{1.491189in}{0.832083in}}%
\pgfpathlineto{\pgfqpoint{1.491485in}{0.831943in}}%
\pgfpathlineto{\pgfqpoint{1.491781in}{0.831787in}}%
\pgfpathlineto{\pgfqpoint{1.492077in}{0.831635in}}%
\pgfpathlineto{\pgfqpoint{1.492373in}{0.831584in}}%
\pgfpathlineto{\pgfqpoint{1.492669in}{0.831576in}}%
\pgfpathlineto{\pgfqpoint{1.492965in}{0.831569in}}%
\pgfpathlineto{\pgfqpoint{1.493261in}{0.831561in}}%
\pgfpathlineto{\pgfqpoint{1.493557in}{0.831553in}}%
\pgfpathlineto{\pgfqpoint{1.493853in}{0.831545in}}%
\pgfpathlineto{\pgfqpoint{1.494149in}{0.831537in}}%
\pgfpathlineto{\pgfqpoint{1.494445in}{0.831529in}}%
\pgfpathlineto{\pgfqpoint{1.494741in}{0.831521in}}%
\pgfpathlineto{\pgfqpoint{1.495037in}{0.831513in}}%
\pgfpathlineto{\pgfqpoint{1.495333in}{0.831505in}}%
\pgfpathlineto{\pgfqpoint{1.495629in}{0.831497in}}%
\pgfpathlineto{\pgfqpoint{1.495925in}{0.831489in}}%
\pgfpathlineto{\pgfqpoint{1.496221in}{0.831482in}}%
\pgfpathlineto{\pgfqpoint{1.496517in}{0.831474in}}%
\pgfpathlineto{\pgfqpoint{1.496813in}{0.831466in}}%
\pgfpathlineto{\pgfqpoint{1.497109in}{0.831458in}}%
\pgfpathlineto{\pgfqpoint{1.497405in}{0.831450in}}%
\pgfpathlineto{\pgfqpoint{1.497701in}{0.831442in}}%
\pgfpathlineto{\pgfqpoint{1.497997in}{0.831434in}}%
\pgfpathlineto{\pgfqpoint{1.498293in}{0.831426in}}%
\pgfpathlineto{\pgfqpoint{1.498589in}{0.831418in}}%
\pgfpathlineto{\pgfqpoint{1.498885in}{0.831410in}}%
\pgfpathlineto{\pgfqpoint{1.499181in}{0.831402in}}%
\pgfpathlineto{\pgfqpoint{1.499477in}{0.831395in}}%
\pgfpathlineto{\pgfqpoint{1.499773in}{0.831387in}}%
\pgfpathlineto{\pgfqpoint{1.500069in}{0.831379in}}%
\pgfpathlineto{\pgfqpoint{1.500365in}{0.831371in}}%
\pgfpathlineto{\pgfqpoint{1.500661in}{0.831363in}}%
\pgfpathlineto{\pgfqpoint{1.500957in}{0.831355in}}%
\pgfpathlineto{\pgfqpoint{1.501253in}{0.831347in}}%
\pgfpathlineto{\pgfqpoint{1.501549in}{0.831339in}}%
\pgfpathlineto{\pgfqpoint{1.501845in}{0.831331in}}%
\pgfpathlineto{\pgfqpoint{1.502141in}{0.831323in}}%
\pgfpathlineto{\pgfqpoint{1.502437in}{0.831316in}}%
\pgfpathlineto{\pgfqpoint{1.502733in}{0.831308in}}%
\pgfpathlineto{\pgfqpoint{1.503029in}{0.831300in}}%
\pgfpathlineto{\pgfqpoint{1.503325in}{0.831292in}}%
\pgfpathlineto{\pgfqpoint{1.503621in}{0.831284in}}%
\pgfpathlineto{\pgfqpoint{1.503917in}{0.831276in}}%
\pgfpathlineto{\pgfqpoint{1.504213in}{0.831268in}}%
\pgfpathlineto{\pgfqpoint{1.504509in}{0.831260in}}%
\pgfpathlineto{\pgfqpoint{1.504805in}{0.831252in}}%
\pgfpathlineto{\pgfqpoint{1.505101in}{0.831244in}}%
\pgfpathlineto{\pgfqpoint{1.505397in}{0.831236in}}%
\pgfpathlineto{\pgfqpoint{1.505693in}{0.831229in}}%
\pgfpathlineto{\pgfqpoint{1.505989in}{0.831221in}}%
\pgfpathlineto{\pgfqpoint{1.506285in}{0.831213in}}%
\pgfpathlineto{\pgfqpoint{1.506581in}{0.831205in}}%
\pgfpathlineto{\pgfqpoint{1.506877in}{0.831197in}}%
\pgfpathlineto{\pgfqpoint{1.507173in}{0.831189in}}%
\pgfpathlineto{\pgfqpoint{1.507469in}{0.831181in}}%
\pgfpathlineto{\pgfqpoint{1.507765in}{0.831173in}}%
\pgfpathlineto{\pgfqpoint{1.508061in}{0.831165in}}%
\pgfpathlineto{\pgfqpoint{1.508357in}{0.831157in}}%
\pgfpathlineto{\pgfqpoint{1.508653in}{0.831150in}}%
\pgfpathlineto{\pgfqpoint{1.508949in}{0.831142in}}%
\pgfpathlineto{\pgfqpoint{1.509245in}{0.831134in}}%
\pgfpathlineto{\pgfqpoint{1.509541in}{0.831126in}}%
\pgfpathlineto{\pgfqpoint{1.509837in}{0.831118in}}%
\pgfpathlineto{\pgfqpoint{1.510133in}{0.831110in}}%
\pgfpathlineto{\pgfqpoint{1.510429in}{0.831102in}}%
\pgfpathlineto{\pgfqpoint{1.510725in}{0.831094in}}%
\pgfpathlineto{\pgfqpoint{1.511021in}{0.831086in}}%
\pgfpathlineto{\pgfqpoint{1.511317in}{0.831078in}}%
\pgfpathlineto{\pgfqpoint{1.511613in}{0.831070in}}%
\pgfpathlineto{\pgfqpoint{1.511909in}{0.831063in}}%
\pgfpathlineto{\pgfqpoint{1.512205in}{0.831055in}}%
\pgfpathlineto{\pgfqpoint{1.512501in}{0.831046in}}%
\pgfpathlineto{\pgfqpoint{1.512797in}{0.831039in}}%
\pgfpathlineto{\pgfqpoint{1.513093in}{0.831033in}}%
\pgfpathlineto{\pgfqpoint{1.513389in}{0.831027in}}%
\pgfpathlineto{\pgfqpoint{1.513685in}{0.831021in}}%
\pgfpathlineto{\pgfqpoint{1.513981in}{0.831013in}}%
\pgfpathlineto{\pgfqpoint{1.514277in}{0.831004in}}%
\pgfpathlineto{\pgfqpoint{1.514573in}{0.830996in}}%
\pgfpathlineto{\pgfqpoint{1.514869in}{0.830987in}}%
\pgfpathlineto{\pgfqpoint{1.515165in}{0.830979in}}%
\pgfpathlineto{\pgfqpoint{1.515461in}{0.830970in}}%
\pgfpathlineto{\pgfqpoint{1.515757in}{0.830962in}}%
\pgfpathlineto{\pgfqpoint{1.516053in}{0.830953in}}%
\pgfpathlineto{\pgfqpoint{1.516349in}{0.830945in}}%
\pgfpathlineto{\pgfqpoint{1.516645in}{0.830936in}}%
\pgfpathlineto{\pgfqpoint{1.516941in}{0.830928in}}%
\pgfpathlineto{\pgfqpoint{1.517237in}{0.830919in}}%
\pgfpathlineto{\pgfqpoint{1.517533in}{0.830911in}}%
\pgfpathlineto{\pgfqpoint{1.517829in}{0.830902in}}%
\pgfpathlineto{\pgfqpoint{1.518125in}{0.830894in}}%
\pgfpathlineto{\pgfqpoint{1.518421in}{0.830885in}}%
\pgfpathlineto{\pgfqpoint{1.518717in}{0.830877in}}%
\pgfpathlineto{\pgfqpoint{1.519013in}{0.830868in}}%
\pgfpathlineto{\pgfqpoint{1.519309in}{0.830865in}}%
\pgfpathlineto{\pgfqpoint{1.519605in}{0.830883in}}%
\pgfpathlineto{\pgfqpoint{1.519901in}{0.830840in}}%
\pgfpathlineto{\pgfqpoint{1.520197in}{0.830793in}}%
\pgfpathlineto{\pgfqpoint{1.520493in}{0.830745in}}%
\pgfpathlineto{\pgfqpoint{1.520789in}{0.830697in}}%
\pgfpathlineto{\pgfqpoint{1.521085in}{0.830649in}}%
\pgfpathlineto{\pgfqpoint{1.521381in}{0.830601in}}%
\pgfpathlineto{\pgfqpoint{1.521677in}{0.830481in}}%
\pgfpathlineto{\pgfqpoint{1.521973in}{0.830313in}}%
\pgfpathlineto{\pgfqpoint{1.522269in}{0.830179in}}%
\pgfpathlineto{\pgfqpoint{1.522565in}{0.830135in}}%
\pgfpathlineto{\pgfqpoint{1.522861in}{0.830097in}}%
\pgfpathlineto{\pgfqpoint{1.523157in}{0.830059in}}%
\pgfpathlineto{\pgfqpoint{1.523453in}{0.830021in}}%
\pgfpathlineto{\pgfqpoint{1.523749in}{0.829983in}}%
\pgfpathlineto{\pgfqpoint{1.524045in}{0.829945in}}%
\pgfpathlineto{\pgfqpoint{1.524341in}{0.829907in}}%
\pgfpathlineto{\pgfqpoint{1.524638in}{0.829869in}}%
\pgfpathlineto{\pgfqpoint{1.524934in}{0.829831in}}%
\pgfpathlineto{\pgfqpoint{1.525230in}{0.829793in}}%
\pgfpathlineto{\pgfqpoint{1.525526in}{0.829755in}}%
\pgfpathlineto{\pgfqpoint{1.525822in}{0.829717in}}%
\pgfpathlineto{\pgfqpoint{1.526118in}{0.829679in}}%
\pgfpathlineto{\pgfqpoint{1.526414in}{0.829641in}}%
\pgfpathlineto{\pgfqpoint{1.526710in}{0.829603in}}%
\pgfpathlineto{\pgfqpoint{1.527006in}{0.829565in}}%
\pgfpathlineto{\pgfqpoint{1.527302in}{0.829527in}}%
\pgfpathlineto{\pgfqpoint{1.527598in}{0.829489in}}%
\pgfpathlineto{\pgfqpoint{1.527894in}{0.829451in}}%
\pgfpathlineto{\pgfqpoint{1.528190in}{0.829413in}}%
\pgfpathlineto{\pgfqpoint{1.528486in}{0.829375in}}%
\pgfpathlineto{\pgfqpoint{1.528782in}{0.829340in}}%
\pgfpathlineto{\pgfqpoint{1.529078in}{0.829366in}}%
\pgfpathlineto{\pgfqpoint{1.529374in}{0.829417in}}%
\pgfpathlineto{\pgfqpoint{1.529670in}{0.829467in}}%
\pgfpathlineto{\pgfqpoint{1.529966in}{0.829518in}}%
\pgfpathlineto{\pgfqpoint{1.530262in}{0.829569in}}%
\pgfpathlineto{\pgfqpoint{1.530558in}{0.829619in}}%
\pgfpathlineto{\pgfqpoint{1.530854in}{0.829670in}}%
\pgfpathlineto{\pgfqpoint{1.531150in}{0.829721in}}%
\pgfpathlineto{\pgfqpoint{1.531446in}{0.829771in}}%
\pgfpathlineto{\pgfqpoint{1.531742in}{0.829822in}}%
\pgfpathlineto{\pgfqpoint{1.532038in}{0.829873in}}%
\pgfpathlineto{\pgfqpoint{1.532334in}{0.829924in}}%
\pgfpathlineto{\pgfqpoint{1.532630in}{0.829974in}}%
\pgfpathlineto{\pgfqpoint{1.532926in}{0.830025in}}%
\pgfpathlineto{\pgfqpoint{1.533222in}{0.830076in}}%
\pgfpathlineto{\pgfqpoint{1.533518in}{0.830128in}}%
\pgfpathlineto{\pgfqpoint{1.533814in}{0.830338in}}%
\pgfpathlineto{\pgfqpoint{1.534110in}{0.830654in}}%
\pgfpathlineto{\pgfqpoint{1.534406in}{0.830916in}}%
\pgfpathlineto{\pgfqpoint{1.534702in}{0.830882in}}%
\pgfpathlineto{\pgfqpoint{1.534998in}{0.830878in}}%
\pgfpathlineto{\pgfqpoint{1.535294in}{0.830874in}}%
\pgfpathlineto{\pgfqpoint{1.535590in}{0.830870in}}%
\pgfpathlineto{\pgfqpoint{1.535886in}{0.830867in}}%
\pgfpathlineto{\pgfqpoint{1.536182in}{0.830863in}}%
\pgfpathlineto{\pgfqpoint{1.536478in}{0.830859in}}%
\pgfpathlineto{\pgfqpoint{1.536774in}{0.830855in}}%
\pgfpathlineto{\pgfqpoint{1.537070in}{0.830852in}}%
\pgfpathlineto{\pgfqpoint{1.537366in}{0.830848in}}%
\pgfpathlineto{\pgfqpoint{1.537662in}{0.830844in}}%
\pgfpathlineto{\pgfqpoint{1.537958in}{0.830840in}}%
\pgfpathlineto{\pgfqpoint{1.538254in}{0.830836in}}%
\pgfpathlineto{\pgfqpoint{1.538550in}{0.830833in}}%
\pgfpathlineto{\pgfqpoint{1.538846in}{0.830829in}}%
\pgfpathlineto{\pgfqpoint{1.539142in}{0.830825in}}%
\pgfpathlineto{\pgfqpoint{1.539438in}{0.830821in}}%
\pgfpathlineto{\pgfqpoint{1.539734in}{0.830818in}}%
\pgfpathlineto{\pgfqpoint{1.540030in}{0.830814in}}%
\pgfpathlineto{\pgfqpoint{1.540326in}{0.830810in}}%
\pgfpathlineto{\pgfqpoint{1.540622in}{0.830806in}}%
\pgfpathlineto{\pgfqpoint{1.540918in}{0.830803in}}%
\pgfpathlineto{\pgfqpoint{1.541214in}{0.830799in}}%
\pgfpathlineto{\pgfqpoint{1.541510in}{0.831258in}}%
\pgfpathlineto{\pgfqpoint{1.541806in}{0.832262in}}%
\pgfpathlineto{\pgfqpoint{1.542102in}{0.833237in}}%
\pgfpathlineto{\pgfqpoint{1.542398in}{0.833569in}}%
\pgfpathlineto{\pgfqpoint{1.542694in}{0.833645in}}%
\pgfpathlineto{\pgfqpoint{1.542990in}{0.833642in}}%
\pgfpathlineto{\pgfqpoint{1.543286in}{0.833638in}}%
\pgfpathlineto{\pgfqpoint{1.543582in}{0.833634in}}%
\pgfpathlineto{\pgfqpoint{1.543878in}{0.833630in}}%
\pgfpathlineto{\pgfqpoint{1.544174in}{0.833626in}}%
\pgfpathlineto{\pgfqpoint{1.544470in}{0.833622in}}%
\pgfpathlineto{\pgfqpoint{1.544766in}{0.833618in}}%
\pgfpathlineto{\pgfqpoint{1.545062in}{0.833614in}}%
\pgfpathlineto{\pgfqpoint{1.545358in}{0.833610in}}%
\pgfpathlineto{\pgfqpoint{1.545654in}{0.833606in}}%
\pgfpathlineto{\pgfqpoint{1.545950in}{0.833602in}}%
\pgfpathlineto{\pgfqpoint{1.546246in}{0.833598in}}%
\pgfpathlineto{\pgfqpoint{1.546542in}{0.833594in}}%
\pgfpathlineto{\pgfqpoint{1.546838in}{0.833590in}}%
\pgfpathlineto{\pgfqpoint{1.547134in}{0.833586in}}%
\pgfpathlineto{\pgfqpoint{1.547430in}{0.833583in}}%
\pgfpathlineto{\pgfqpoint{1.547726in}{0.833579in}}%
\pgfpathlineto{\pgfqpoint{1.548022in}{0.833575in}}%
\pgfpathlineto{\pgfqpoint{1.548318in}{0.833571in}}%
\pgfpathlineto{\pgfqpoint{1.548614in}{0.833567in}}%
\pgfpathlineto{\pgfqpoint{1.548910in}{0.833563in}}%
\pgfpathlineto{\pgfqpoint{1.549206in}{0.833559in}}%
\pgfpathlineto{\pgfqpoint{1.549502in}{0.833555in}}%
\pgfpathlineto{\pgfqpoint{1.549798in}{0.833551in}}%
\pgfpathlineto{\pgfqpoint{1.550094in}{0.833547in}}%
\pgfpathlineto{\pgfqpoint{1.550390in}{0.833543in}}%
\pgfpathlineto{\pgfqpoint{1.550686in}{0.833539in}}%
\pgfpathlineto{\pgfqpoint{1.550982in}{0.833535in}}%
\pgfpathlineto{\pgfqpoint{1.551278in}{0.833531in}}%
\pgfpathlineto{\pgfqpoint{1.551574in}{0.833527in}}%
\pgfpathlineto{\pgfqpoint{1.551870in}{0.833523in}}%
\pgfpathlineto{\pgfqpoint{1.552166in}{0.833520in}}%
\pgfpathlineto{\pgfqpoint{1.552462in}{0.833516in}}%
\pgfpathlineto{\pgfqpoint{1.552758in}{0.833512in}}%
\pgfpathlineto{\pgfqpoint{1.553054in}{0.833508in}}%
\pgfpathlineto{\pgfqpoint{1.553350in}{0.833504in}}%
\pgfpathlineto{\pgfqpoint{1.553646in}{0.833500in}}%
\pgfpathlineto{\pgfqpoint{1.553942in}{0.833496in}}%
\pgfpathlineto{\pgfqpoint{1.554238in}{0.833492in}}%
\pgfpathlineto{\pgfqpoint{1.554534in}{0.833488in}}%
\pgfpathlineto{\pgfqpoint{1.554830in}{0.833484in}}%
\pgfpathlineto{\pgfqpoint{1.555126in}{0.833480in}}%
\pgfpathlineto{\pgfqpoint{1.555422in}{0.833476in}}%
\pgfpathlineto{\pgfqpoint{1.555718in}{0.833472in}}%
\pgfpathlineto{\pgfqpoint{1.556014in}{0.833468in}}%
\pgfpathlineto{\pgfqpoint{1.556310in}{0.833464in}}%
\pgfpathlineto{\pgfqpoint{1.556606in}{0.833461in}}%
\pgfpathlineto{\pgfqpoint{1.556902in}{0.833457in}}%
\pgfpathlineto{\pgfqpoint{1.557198in}{0.833453in}}%
\pgfpathlineto{\pgfqpoint{1.557494in}{0.833449in}}%
\pgfpathlineto{\pgfqpoint{1.557790in}{0.833445in}}%
\pgfpathlineto{\pgfqpoint{1.558086in}{0.833441in}}%
\pgfpathlineto{\pgfqpoint{1.558382in}{0.833437in}}%
\pgfpathlineto{\pgfqpoint{1.558678in}{0.833433in}}%
\pgfpathlineto{\pgfqpoint{1.558974in}{0.833429in}}%
\pgfpathlineto{\pgfqpoint{1.559270in}{0.833425in}}%
\pgfpathlineto{\pgfqpoint{1.559566in}{0.833421in}}%
\pgfpathlineto{\pgfqpoint{1.559862in}{0.833417in}}%
\pgfpathlineto{\pgfqpoint{1.560158in}{0.833413in}}%
\pgfpathlineto{\pgfqpoint{1.560454in}{0.833409in}}%
\pgfpathlineto{\pgfqpoint{1.560750in}{0.833405in}}%
\pgfpathlineto{\pgfqpoint{1.561046in}{0.833401in}}%
\pgfpathlineto{\pgfqpoint{1.561342in}{0.833398in}}%
\pgfpathlineto{\pgfqpoint{1.561638in}{0.833394in}}%
\pgfpathlineto{\pgfqpoint{1.561934in}{0.833390in}}%
\pgfpathlineto{\pgfqpoint{1.562230in}{0.833401in}}%
\pgfpathlineto{\pgfqpoint{1.562526in}{0.833432in}}%
\pgfpathlineto{\pgfqpoint{1.562822in}{0.833442in}}%
\pgfpathlineto{\pgfqpoint{1.563118in}{0.833452in}}%
\pgfpathlineto{\pgfqpoint{1.563414in}{0.833461in}}%
\pgfpathlineto{\pgfqpoint{1.563710in}{0.833470in}}%
\pgfpathlineto{\pgfqpoint{1.564006in}{0.833479in}}%
\pgfpathlineto{\pgfqpoint{1.564302in}{0.833488in}}%
\pgfpathlineto{\pgfqpoint{1.564598in}{0.833497in}}%
\pgfpathlineto{\pgfqpoint{1.564894in}{0.833506in}}%
\pgfpathlineto{\pgfqpoint{1.565190in}{0.833515in}}%
\pgfpathlineto{\pgfqpoint{1.565486in}{0.833524in}}%
\pgfpathlineto{\pgfqpoint{1.565782in}{0.833533in}}%
\pgfpathlineto{\pgfqpoint{1.566078in}{0.833543in}}%
\pgfpathlineto{\pgfqpoint{1.566374in}{0.833552in}}%
\pgfpathlineto{\pgfqpoint{1.566670in}{0.833561in}}%
\pgfpathlineto{\pgfqpoint{1.566966in}{0.833570in}}%
\pgfpathlineto{\pgfqpoint{1.567262in}{0.833579in}}%
\pgfpathlineto{\pgfqpoint{1.567558in}{0.833588in}}%
\pgfpathlineto{\pgfqpoint{1.567854in}{0.833597in}}%
\pgfpathlineto{\pgfqpoint{1.568150in}{0.833606in}}%
\pgfpathlineto{\pgfqpoint{1.568446in}{0.833615in}}%
\pgfpathlineto{\pgfqpoint{1.568742in}{0.833625in}}%
\pgfpathlineto{\pgfqpoint{1.569038in}{0.833641in}}%
\pgfpathlineto{\pgfqpoint{1.569334in}{0.834068in}}%
\pgfpathlineto{\pgfqpoint{1.569630in}{0.834728in}}%
\pgfpathlineto{\pgfqpoint{1.569926in}{0.835388in}}%
\pgfpathlineto{\pgfqpoint{1.570222in}{0.836042in}}%
\pgfpathlineto{\pgfqpoint{1.570518in}{0.835400in}}%
\pgfpathlineto{\pgfqpoint{1.570814in}{0.835801in}}%
\pgfpathlineto{\pgfqpoint{1.571110in}{0.836273in}}%
\pgfpathlineto{\pgfqpoint{1.571406in}{0.836522in}}%
\pgfpathlineto{\pgfqpoint{1.571702in}{0.836527in}}%
\pgfpathlineto{\pgfqpoint{1.571998in}{0.836531in}}%
\pgfpathlineto{\pgfqpoint{1.572294in}{0.836536in}}%
\pgfpathlineto{\pgfqpoint{1.572590in}{0.836541in}}%
\pgfpathlineto{\pgfqpoint{1.572886in}{0.836545in}}%
\pgfpathlineto{\pgfqpoint{1.573182in}{0.836550in}}%
\pgfpathlineto{\pgfqpoint{1.573478in}{0.836555in}}%
\pgfpathlineto{\pgfqpoint{1.573774in}{0.836559in}}%
\pgfpathlineto{\pgfqpoint{1.574070in}{0.836564in}}%
\pgfpathlineto{\pgfqpoint{1.574366in}{0.836569in}}%
\pgfpathlineto{\pgfqpoint{1.574662in}{0.836573in}}%
\pgfpathlineto{\pgfqpoint{1.574958in}{0.836578in}}%
\pgfpathlineto{\pgfqpoint{1.575254in}{0.836583in}}%
\pgfpathlineto{\pgfqpoint{1.575550in}{0.836587in}}%
\pgfpathlineto{\pgfqpoint{1.575846in}{0.836592in}}%
\pgfpathlineto{\pgfqpoint{1.576142in}{0.836597in}}%
\pgfpathlineto{\pgfqpoint{1.576438in}{0.836601in}}%
\pgfpathlineto{\pgfqpoint{1.576734in}{0.836606in}}%
\pgfpathlineto{\pgfqpoint{1.577030in}{0.836611in}}%
\pgfpathlineto{\pgfqpoint{1.577326in}{0.836615in}}%
\pgfpathlineto{\pgfqpoint{1.577622in}{0.836620in}}%
\pgfpathlineto{\pgfqpoint{1.577918in}{0.836625in}}%
\pgfpathlineto{\pgfqpoint{1.578214in}{0.836629in}}%
\pgfpathlineto{\pgfqpoint{1.578510in}{0.836570in}}%
\pgfpathlineto{\pgfqpoint{1.578806in}{0.836538in}}%
\pgfpathlineto{\pgfqpoint{1.579102in}{0.836586in}}%
\pgfpathlineto{\pgfqpoint{1.579398in}{0.836633in}}%
\pgfpathlineto{\pgfqpoint{1.579694in}{0.836681in}}%
\pgfpathlineto{\pgfqpoint{1.579990in}{0.836729in}}%
\pgfpathlineto{\pgfqpoint{1.580286in}{0.836776in}}%
\pgfpathlineto{\pgfqpoint{1.580582in}{0.836824in}}%
\pgfpathlineto{\pgfqpoint{1.580878in}{0.836871in}}%
\pgfpathlineto{\pgfqpoint{1.581174in}{0.836919in}}%
\pgfpathlineto{\pgfqpoint{1.581470in}{0.836967in}}%
\pgfpathlineto{\pgfqpoint{1.581766in}{0.837014in}}%
\pgfpathlineto{\pgfqpoint{1.582062in}{0.837062in}}%
\pgfpathlineto{\pgfqpoint{1.582358in}{0.837109in}}%
\pgfpathlineto{\pgfqpoint{1.582654in}{0.837157in}}%
\pgfpathlineto{\pgfqpoint{1.582950in}{0.837205in}}%
\pgfpathlineto{\pgfqpoint{1.583246in}{0.837252in}}%
\pgfpathlineto{\pgfqpoint{1.583542in}{0.837300in}}%
\pgfpathlineto{\pgfqpoint{1.583838in}{0.837347in}}%
\pgfpathlineto{\pgfqpoint{1.584134in}{0.837395in}}%
\pgfpathlineto{\pgfqpoint{1.584430in}{0.837443in}}%
\pgfpathlineto{\pgfqpoint{1.584726in}{0.837490in}}%
\pgfpathlineto{\pgfqpoint{1.585022in}{0.837538in}}%
\pgfpathlineto{\pgfqpoint{1.585318in}{0.837585in}}%
\pgfpathlineto{\pgfqpoint{1.585614in}{0.837633in}}%
\pgfpathlineto{\pgfqpoint{1.585910in}{0.837680in}}%
\pgfpathlineto{\pgfqpoint{1.586206in}{0.837728in}}%
\pgfpathlineto{\pgfqpoint{1.586502in}{0.837776in}}%
\pgfpathlineto{\pgfqpoint{1.586798in}{0.837823in}}%
\pgfpathlineto{\pgfqpoint{1.587094in}{0.837871in}}%
\pgfpathlineto{\pgfqpoint{1.587390in}{0.837918in}}%
\pgfpathlineto{\pgfqpoint{1.587686in}{0.837966in}}%
\pgfpathlineto{\pgfqpoint{1.587982in}{0.838014in}}%
\pgfpathlineto{\pgfqpoint{1.588278in}{0.838061in}}%
\pgfpathlineto{\pgfqpoint{1.588574in}{0.838109in}}%
\pgfpathlineto{\pgfqpoint{1.588870in}{0.838156in}}%
\pgfpathlineto{\pgfqpoint{1.589166in}{0.838204in}}%
\pgfpathlineto{\pgfqpoint{1.589462in}{0.838252in}}%
\pgfpathlineto{\pgfqpoint{1.589758in}{0.838299in}}%
\pgfpathlineto{\pgfqpoint{1.590054in}{0.838347in}}%
\pgfpathlineto{\pgfqpoint{1.590350in}{0.838394in}}%
\pgfpathlineto{\pgfqpoint{1.590646in}{0.838442in}}%
\pgfpathlineto{\pgfqpoint{1.590942in}{0.838548in}}%
\pgfpathlineto{\pgfqpoint{1.591238in}{0.838558in}}%
\pgfpathlineto{\pgfqpoint{1.591534in}{0.838547in}}%
\pgfpathlineto{\pgfqpoint{1.591831in}{0.837801in}}%
\pgfpathlineto{\pgfqpoint{1.592127in}{0.837172in}}%
\pgfpathlineto{\pgfqpoint{1.592423in}{0.837132in}}%
\pgfpathlineto{\pgfqpoint{1.592719in}{0.837092in}}%
\pgfpathlineto{\pgfqpoint{1.593015in}{0.837052in}}%
\pgfpathlineto{\pgfqpoint{1.593311in}{0.837012in}}%
\pgfpathlineto{\pgfqpoint{1.593607in}{0.836973in}}%
\pgfpathlineto{\pgfqpoint{1.593903in}{0.836933in}}%
\pgfpathlineto{\pgfqpoint{1.594199in}{0.836893in}}%
\pgfpathlineto{\pgfqpoint{1.594495in}{0.836853in}}%
\pgfpathlineto{\pgfqpoint{1.594791in}{0.836813in}}%
\pgfpathlineto{\pgfqpoint{1.595087in}{0.836773in}}%
\pgfpathlineto{\pgfqpoint{1.595383in}{0.836733in}}%
\pgfpathlineto{\pgfqpoint{1.595679in}{0.836693in}}%
\pgfpathlineto{\pgfqpoint{1.595975in}{0.836653in}}%
\pgfpathlineto{\pgfqpoint{1.596271in}{0.836613in}}%
\pgfpathlineto{\pgfqpoint{1.596567in}{0.836573in}}%
\pgfpathlineto{\pgfqpoint{1.596863in}{0.836534in}}%
\pgfpathlineto{\pgfqpoint{1.597159in}{0.836494in}}%
\pgfpathlineto{\pgfqpoint{1.597455in}{0.836454in}}%
\pgfpathlineto{\pgfqpoint{1.597751in}{0.836414in}}%
\pgfpathlineto{\pgfqpoint{1.598047in}{0.836374in}}%
\pgfpathlineto{\pgfqpoint{1.598343in}{0.836334in}}%
\pgfpathlineto{\pgfqpoint{1.598639in}{0.836294in}}%
\pgfpathlineto{\pgfqpoint{1.598935in}{0.836254in}}%
\pgfpathlineto{\pgfqpoint{1.599231in}{0.836214in}}%
\pgfpathlineto{\pgfqpoint{1.599527in}{0.836174in}}%
\pgfpathlineto{\pgfqpoint{1.599823in}{0.836134in}}%
\pgfpathlineto{\pgfqpoint{1.600119in}{0.836094in}}%
\pgfpathlineto{\pgfqpoint{1.600415in}{0.836055in}}%
\pgfpathlineto{\pgfqpoint{1.600711in}{0.836015in}}%
\pgfpathlineto{\pgfqpoint{1.601007in}{0.835975in}}%
\pgfpathlineto{\pgfqpoint{1.601303in}{0.835935in}}%
\pgfpathlineto{\pgfqpoint{1.601599in}{0.835895in}}%
\pgfpathlineto{\pgfqpoint{1.601895in}{0.835855in}}%
\pgfpathlineto{\pgfqpoint{1.602191in}{0.835815in}}%
\pgfpathlineto{\pgfqpoint{1.602487in}{0.835775in}}%
\pgfpathlineto{\pgfqpoint{1.602783in}{0.835735in}}%
\pgfpathlineto{\pgfqpoint{1.603079in}{0.835695in}}%
\pgfpathlineto{\pgfqpoint{1.603375in}{0.835655in}}%
\pgfpathlineto{\pgfqpoint{1.603671in}{0.835615in}}%
\pgfpathlineto{\pgfqpoint{1.603967in}{0.835576in}}%
\pgfpathlineto{\pgfqpoint{1.604263in}{0.835536in}}%
\pgfpathlineto{\pgfqpoint{1.604559in}{0.835496in}}%
\pgfpathlineto{\pgfqpoint{1.604855in}{0.835456in}}%
\pgfpathlineto{\pgfqpoint{1.605151in}{0.835416in}}%
\pgfpathlineto{\pgfqpoint{1.605447in}{0.835376in}}%
\pgfpathlineto{\pgfqpoint{1.605743in}{0.835336in}}%
\pgfpathlineto{\pgfqpoint{1.606039in}{0.835296in}}%
\pgfpathlineto{\pgfqpoint{1.606335in}{0.835256in}}%
\pgfpathlineto{\pgfqpoint{1.606631in}{0.835216in}}%
\pgfpathlineto{\pgfqpoint{1.606927in}{0.835176in}}%
\pgfpathlineto{\pgfqpoint{1.607223in}{0.835137in}}%
\pgfpathlineto{\pgfqpoint{1.607519in}{0.835097in}}%
\pgfpathlineto{\pgfqpoint{1.607815in}{0.835057in}}%
\pgfpathlineto{\pgfqpoint{1.608111in}{0.835017in}}%
\pgfpathlineto{\pgfqpoint{1.608407in}{0.834977in}}%
\pgfpathlineto{\pgfqpoint{1.608703in}{0.834937in}}%
\pgfpathlineto{\pgfqpoint{1.608999in}{0.834897in}}%
\pgfpathlineto{\pgfqpoint{1.609295in}{0.834857in}}%
\pgfpathlineto{\pgfqpoint{1.609591in}{0.834817in}}%
\pgfpathlineto{\pgfqpoint{1.609887in}{0.834777in}}%
\pgfpathlineto{\pgfqpoint{1.610183in}{0.834737in}}%
\pgfpathlineto{\pgfqpoint{1.610479in}{0.834697in}}%
\pgfpathlineto{\pgfqpoint{1.610775in}{0.834658in}}%
\pgfpathlineto{\pgfqpoint{1.611071in}{0.834618in}}%
\pgfpathlineto{\pgfqpoint{1.611367in}{0.834578in}}%
\pgfpathlineto{\pgfqpoint{1.611663in}{0.834538in}}%
\pgfpathlineto{\pgfqpoint{1.611959in}{0.834751in}}%
\pgfpathlineto{\pgfqpoint{1.612255in}{0.836425in}}%
\pgfpathlineto{\pgfqpoint{1.612551in}{0.835694in}}%
\pgfpathlineto{\pgfqpoint{1.612847in}{0.836847in}}%
\pgfpathlineto{\pgfqpoint{1.613143in}{0.836814in}}%
\pgfpathlineto{\pgfqpoint{1.613439in}{0.836780in}}%
\pgfpathlineto{\pgfqpoint{1.613735in}{0.836747in}}%
\pgfpathlineto{\pgfqpoint{1.614031in}{0.836714in}}%
\pgfpathlineto{\pgfqpoint{1.614327in}{0.836687in}}%
\pgfpathlineto{\pgfqpoint{1.614623in}{0.836678in}}%
\pgfpathlineto{\pgfqpoint{1.614919in}{0.836671in}}%
\pgfpathlineto{\pgfqpoint{1.615215in}{0.836663in}}%
\pgfpathlineto{\pgfqpoint{1.615511in}{0.836655in}}%
\pgfpathlineto{\pgfqpoint{1.615807in}{0.836647in}}%
\pgfpathlineto{\pgfqpoint{1.616103in}{0.836639in}}%
\pgfpathlineto{\pgfqpoint{1.616399in}{0.836632in}}%
\pgfpathlineto{\pgfqpoint{1.616695in}{0.836624in}}%
\pgfpathlineto{\pgfqpoint{1.616991in}{0.836616in}}%
\pgfpathlineto{\pgfqpoint{1.617287in}{0.836608in}}%
\pgfpathlineto{\pgfqpoint{1.617583in}{0.836601in}}%
\pgfpathlineto{\pgfqpoint{1.617879in}{0.836593in}}%
\pgfpathlineto{\pgfqpoint{1.618175in}{0.836585in}}%
\pgfpathlineto{\pgfqpoint{1.618471in}{0.836577in}}%
\pgfpathlineto{\pgfqpoint{1.618767in}{0.836569in}}%
\pgfpathlineto{\pgfqpoint{1.619063in}{0.836562in}}%
\pgfpathlineto{\pgfqpoint{1.619359in}{0.836554in}}%
\pgfpathlineto{\pgfqpoint{1.619655in}{0.836546in}}%
\pgfpathlineto{\pgfqpoint{1.619951in}{0.836538in}}%
\pgfpathlineto{\pgfqpoint{1.620247in}{0.836530in}}%
\pgfpathlineto{\pgfqpoint{1.620543in}{0.836523in}}%
\pgfpathlineto{\pgfqpoint{1.620839in}{0.836515in}}%
\pgfpathlineto{\pgfqpoint{1.621135in}{0.836507in}}%
\pgfpathlineto{\pgfqpoint{1.621431in}{0.835575in}}%
\pgfpathlineto{\pgfqpoint{1.621727in}{0.835479in}}%
\pgfpathlineto{\pgfqpoint{1.622023in}{0.835468in}}%
\pgfpathlineto{\pgfqpoint{1.622319in}{0.835457in}}%
\pgfpathlineto{\pgfqpoint{1.622615in}{0.835446in}}%
\pgfpathlineto{\pgfqpoint{1.622911in}{0.835435in}}%
\pgfpathlineto{\pgfqpoint{1.623207in}{0.835424in}}%
\pgfpathlineto{\pgfqpoint{1.623503in}{0.835413in}}%
\pgfpathlineto{\pgfqpoint{1.623799in}{0.835402in}}%
\pgfpathlineto{\pgfqpoint{1.624095in}{0.835390in}}%
\pgfpathlineto{\pgfqpoint{1.624391in}{0.835379in}}%
\pgfpathlineto{\pgfqpoint{1.624687in}{0.835368in}}%
\pgfpathlineto{\pgfqpoint{1.624983in}{0.835357in}}%
\pgfpathlineto{\pgfqpoint{1.625279in}{0.835346in}}%
\pgfpathlineto{\pgfqpoint{1.625575in}{0.835335in}}%
\pgfpathlineto{\pgfqpoint{1.625871in}{0.835324in}}%
\pgfpathlineto{\pgfqpoint{1.626167in}{0.835313in}}%
\pgfpathlineto{\pgfqpoint{1.626463in}{0.835306in}}%
\pgfpathlineto{\pgfqpoint{1.626759in}{0.835302in}}%
\pgfpathlineto{\pgfqpoint{1.627055in}{0.835298in}}%
\pgfpathlineto{\pgfqpoint{1.627351in}{0.835295in}}%
\pgfpathlineto{\pgfqpoint{1.627647in}{0.835291in}}%
\pgfpathlineto{\pgfqpoint{1.627943in}{0.835287in}}%
\pgfpathlineto{\pgfqpoint{1.628239in}{0.835283in}}%
\pgfpathlineto{\pgfqpoint{1.628535in}{0.835280in}}%
\pgfpathlineto{\pgfqpoint{1.628831in}{0.835276in}}%
\pgfpathlineto{\pgfqpoint{1.629127in}{0.835272in}}%
\pgfpathlineto{\pgfqpoint{1.629423in}{0.835268in}}%
\pgfpathlineto{\pgfqpoint{1.629719in}{0.835264in}}%
\pgfpathlineto{\pgfqpoint{1.630015in}{0.835261in}}%
\pgfpathlineto{\pgfqpoint{1.630311in}{0.835257in}}%
\pgfpathlineto{\pgfqpoint{1.630607in}{0.835253in}}%
\pgfpathlineto{\pgfqpoint{1.630903in}{0.835249in}}%
\pgfpathlineto{\pgfqpoint{1.631199in}{0.835246in}}%
\pgfpathlineto{\pgfqpoint{1.631495in}{0.835242in}}%
\pgfpathlineto{\pgfqpoint{1.631791in}{0.835238in}}%
\pgfpathlineto{\pgfqpoint{1.632087in}{0.835234in}}%
\pgfpathlineto{\pgfqpoint{1.632383in}{0.835231in}}%
\pgfpathlineto{\pgfqpoint{1.632679in}{0.835225in}}%
\pgfpathlineto{\pgfqpoint{1.632975in}{0.835211in}}%
\pgfpathlineto{\pgfqpoint{1.633271in}{0.835197in}}%
\pgfpathlineto{\pgfqpoint{1.633567in}{0.835182in}}%
\pgfpathlineto{\pgfqpoint{1.633863in}{0.835168in}}%
\pgfpathlineto{\pgfqpoint{1.634159in}{0.835159in}}%
\pgfpathlineto{\pgfqpoint{1.634455in}{0.835156in}}%
\pgfpathlineto{\pgfqpoint{1.634751in}{0.835154in}}%
\pgfpathlineto{\pgfqpoint{1.635047in}{0.835152in}}%
\pgfpathlineto{\pgfqpoint{1.635343in}{0.835150in}}%
\pgfpathlineto{\pgfqpoint{1.635639in}{0.835148in}}%
\pgfpathlineto{\pgfqpoint{1.635935in}{0.835145in}}%
\pgfpathlineto{\pgfqpoint{1.636231in}{0.835143in}}%
\pgfpathlineto{\pgfqpoint{1.636527in}{0.835141in}}%
\pgfpathlineto{\pgfqpoint{1.636823in}{0.835139in}}%
\pgfpathlineto{\pgfqpoint{1.637119in}{0.835137in}}%
\pgfpathlineto{\pgfqpoint{1.637415in}{0.835135in}}%
\pgfpathlineto{\pgfqpoint{1.637711in}{0.835132in}}%
\pgfpathlineto{\pgfqpoint{1.638007in}{0.835130in}}%
\pgfpathlineto{\pgfqpoint{1.638303in}{0.835128in}}%
\pgfpathlineto{\pgfqpoint{1.638599in}{0.835126in}}%
\pgfpathlineto{\pgfqpoint{1.638895in}{0.835124in}}%
\pgfpathlineto{\pgfqpoint{1.639191in}{0.835122in}}%
\pgfpathlineto{\pgfqpoint{1.639487in}{0.835119in}}%
\pgfpathlineto{\pgfqpoint{1.639783in}{0.835117in}}%
\pgfpathlineto{\pgfqpoint{1.640079in}{0.835115in}}%
\pgfpathlineto{\pgfqpoint{1.640375in}{0.835113in}}%
\pgfpathlineto{\pgfqpoint{1.640671in}{0.835111in}}%
\pgfpathlineto{\pgfqpoint{1.640967in}{0.835109in}}%
\pgfpathlineto{\pgfqpoint{1.641263in}{0.835106in}}%
\pgfpathlineto{\pgfqpoint{1.641559in}{0.835104in}}%
\pgfpathlineto{\pgfqpoint{1.641855in}{0.835102in}}%
\pgfpathlineto{\pgfqpoint{1.642151in}{0.835100in}}%
\pgfpathlineto{\pgfqpoint{1.642447in}{0.835098in}}%
\pgfpathlineto{\pgfqpoint{1.642743in}{0.835095in}}%
\pgfpathlineto{\pgfqpoint{1.643039in}{0.835092in}}%
\pgfpathlineto{\pgfqpoint{1.643335in}{0.835088in}}%
\pgfpathlineto{\pgfqpoint{1.643631in}{0.835084in}}%
\pgfpathlineto{\pgfqpoint{1.643927in}{0.835080in}}%
\pgfpathlineto{\pgfqpoint{1.644223in}{0.835077in}}%
\pgfpathlineto{\pgfqpoint{1.644519in}{0.835073in}}%
\pgfpathlineto{\pgfqpoint{1.644815in}{0.835069in}}%
\pgfpathlineto{\pgfqpoint{1.645111in}{0.835066in}}%
\pgfpathlineto{\pgfqpoint{1.645407in}{0.835062in}}%
\pgfpathlineto{\pgfqpoint{1.645703in}{0.835058in}}%
\pgfpathlineto{\pgfqpoint{1.645999in}{0.835054in}}%
\pgfpathlineto{\pgfqpoint{1.646295in}{0.835051in}}%
\pgfpathlineto{\pgfqpoint{1.646591in}{0.835047in}}%
\pgfpathlineto{\pgfqpoint{1.646887in}{0.835043in}}%
\pgfpathlineto{\pgfqpoint{1.647183in}{0.835040in}}%
\pgfpathlineto{\pgfqpoint{1.647479in}{0.835036in}}%
\pgfpathlineto{\pgfqpoint{1.647775in}{0.835032in}}%
\pgfpathlineto{\pgfqpoint{1.648071in}{0.835028in}}%
\pgfpathlineto{\pgfqpoint{1.648367in}{0.835025in}}%
\pgfpathlineto{\pgfqpoint{1.648663in}{0.835021in}}%
\pgfpathlineto{\pgfqpoint{1.648959in}{0.835017in}}%
\pgfpathlineto{\pgfqpoint{1.649255in}{0.835014in}}%
\pgfpathlineto{\pgfqpoint{1.649551in}{0.835010in}}%
\pgfpathlineto{\pgfqpoint{1.649847in}{0.835006in}}%
\pgfpathlineto{\pgfqpoint{1.650143in}{0.835003in}}%
\pgfpathlineto{\pgfqpoint{1.650439in}{0.834999in}}%
\pgfpathlineto{\pgfqpoint{1.650735in}{0.834995in}}%
\pgfpathlineto{\pgfqpoint{1.651031in}{0.834991in}}%
\pgfpathlineto{\pgfqpoint{1.651327in}{0.834988in}}%
\pgfpathlineto{\pgfqpoint{1.651623in}{0.834984in}}%
\pgfpathlineto{\pgfqpoint{1.651919in}{0.834980in}}%
\pgfpathlineto{\pgfqpoint{1.652215in}{0.834977in}}%
\pgfpathlineto{\pgfqpoint{1.652511in}{0.834973in}}%
\pgfpathlineto{\pgfqpoint{1.652807in}{0.834969in}}%
\pgfpathlineto{\pgfqpoint{1.653103in}{0.834965in}}%
\pgfpathlineto{\pgfqpoint{1.653399in}{0.834962in}}%
\pgfpathlineto{\pgfqpoint{1.653695in}{0.834958in}}%
\pgfpathlineto{\pgfqpoint{1.653991in}{0.834954in}}%
\pgfpathlineto{\pgfqpoint{1.654287in}{0.834951in}}%
\pgfpathlineto{\pgfqpoint{1.654583in}{0.834947in}}%
\pgfpathlineto{\pgfqpoint{1.654879in}{0.834943in}}%
\pgfpathlineto{\pgfqpoint{1.655175in}{0.834940in}}%
\pgfpathlineto{\pgfqpoint{1.655471in}{0.834936in}}%
\pgfpathlineto{\pgfqpoint{1.655767in}{0.834932in}}%
\pgfpathlineto{\pgfqpoint{1.656063in}{0.834928in}}%
\pgfpathlineto{\pgfqpoint{1.656359in}{0.834925in}}%
\pgfpathlineto{\pgfqpoint{1.656655in}{0.834921in}}%
\pgfpathlineto{\pgfqpoint{1.656951in}{0.834917in}}%
\pgfpathlineto{\pgfqpoint{1.657247in}{0.834914in}}%
\pgfpathlineto{\pgfqpoint{1.657543in}{0.834910in}}%
\pgfpathlineto{\pgfqpoint{1.657839in}{0.834906in}}%
\pgfpathlineto{\pgfqpoint{1.658135in}{0.834902in}}%
\pgfpathlineto{\pgfqpoint{1.658431in}{0.834899in}}%
\pgfpathlineto{\pgfqpoint{1.658727in}{0.834895in}}%
\pgfpathlineto{\pgfqpoint{1.659023in}{0.834891in}}%
\pgfpathlineto{\pgfqpoint{1.659320in}{0.834888in}}%
\pgfpathlineto{\pgfqpoint{1.659616in}{0.834884in}}%
\pgfpathlineto{\pgfqpoint{1.659912in}{0.834880in}}%
\pgfpathlineto{\pgfqpoint{1.660208in}{0.834877in}}%
\pgfpathlineto{\pgfqpoint{1.660504in}{0.834873in}}%
\pgfpathlineto{\pgfqpoint{1.660800in}{0.834869in}}%
\pgfpathlineto{\pgfqpoint{1.661096in}{0.834865in}}%
\pgfpathlineto{\pgfqpoint{1.661392in}{0.834862in}}%
\pgfpathlineto{\pgfqpoint{1.661688in}{0.834868in}}%
\pgfpathlineto{\pgfqpoint{1.661984in}{0.834916in}}%
\pgfpathlineto{\pgfqpoint{1.662280in}{0.834913in}}%
\pgfpathlineto{\pgfqpoint{1.662576in}{0.834911in}}%
\pgfpathlineto{\pgfqpoint{1.662872in}{0.834908in}}%
\pgfpathlineto{\pgfqpoint{1.663168in}{0.834905in}}%
\pgfpathlineto{\pgfqpoint{1.663464in}{0.834902in}}%
\pgfpathlineto{\pgfqpoint{1.663760in}{0.834900in}}%
\pgfpathlineto{\pgfqpoint{1.664056in}{0.834895in}}%
\pgfpathlineto{\pgfqpoint{1.664352in}{0.834887in}}%
\pgfpathlineto{\pgfqpoint{1.664648in}{0.834877in}}%
\pgfpathlineto{\pgfqpoint{1.664944in}{0.834868in}}%
\pgfpathlineto{\pgfqpoint{1.665240in}{0.834859in}}%
\pgfpathlineto{\pgfqpoint{1.665536in}{0.834850in}}%
\pgfpathlineto{\pgfqpoint{1.665832in}{0.834841in}}%
\pgfpathlineto{\pgfqpoint{1.666128in}{0.834831in}}%
\pgfpathlineto{\pgfqpoint{1.666424in}{0.834822in}}%
\pgfpathlineto{\pgfqpoint{1.666720in}{0.834813in}}%
\pgfpathlineto{\pgfqpoint{1.667016in}{0.834804in}}%
\pgfpathlineto{\pgfqpoint{1.667312in}{0.834794in}}%
\pgfpathlineto{\pgfqpoint{1.667608in}{0.834785in}}%
\pgfpathlineto{\pgfqpoint{1.667904in}{0.834776in}}%
\pgfpathlineto{\pgfqpoint{1.668200in}{0.834767in}}%
\pgfpathlineto{\pgfqpoint{1.668496in}{0.834758in}}%
\pgfpathlineto{\pgfqpoint{1.668792in}{0.834748in}}%
\pgfpathlineto{\pgfqpoint{1.669088in}{0.834733in}}%
\pgfpathlineto{\pgfqpoint{1.669384in}{0.834722in}}%
\pgfpathlineto{\pgfqpoint{1.669680in}{0.834712in}}%
\pgfpathlineto{\pgfqpoint{1.669976in}{0.834702in}}%
\pgfpathlineto{\pgfqpoint{1.670272in}{0.834691in}}%
\pgfpathlineto{\pgfqpoint{1.670568in}{0.834681in}}%
\pgfpathlineto{\pgfqpoint{1.670864in}{0.834670in}}%
\pgfpathlineto{\pgfqpoint{1.671160in}{0.834660in}}%
\pgfpathlineto{\pgfqpoint{1.671456in}{0.834649in}}%
\pgfpathlineto{\pgfqpoint{1.671752in}{0.834639in}}%
\pgfpathlineto{\pgfqpoint{1.672048in}{0.834629in}}%
\pgfpathlineto{\pgfqpoint{1.672344in}{0.834618in}}%
\pgfpathlineto{\pgfqpoint{1.672640in}{0.834608in}}%
\pgfpathlineto{\pgfqpoint{1.672936in}{0.834597in}}%
\pgfpathlineto{\pgfqpoint{1.673232in}{0.834587in}}%
\pgfpathlineto{\pgfqpoint{1.673528in}{0.834577in}}%
\pgfpathlineto{\pgfqpoint{1.673824in}{0.834566in}}%
\pgfpathlineto{\pgfqpoint{1.674120in}{0.834556in}}%
\pgfpathlineto{\pgfqpoint{1.674416in}{0.834545in}}%
\pgfpathlineto{\pgfqpoint{1.674712in}{0.834535in}}%
\pgfpathlineto{\pgfqpoint{1.675008in}{0.834524in}}%
\pgfpathlineto{\pgfqpoint{1.675304in}{0.834514in}}%
\pgfpathlineto{\pgfqpoint{1.675600in}{0.834504in}}%
\pgfpathlineto{\pgfqpoint{1.675896in}{0.834482in}}%
\pgfpathlineto{\pgfqpoint{1.676192in}{0.834354in}}%
\pgfpathlineto{\pgfqpoint{1.676488in}{0.834467in}}%
\pgfpathlineto{\pgfqpoint{1.676784in}{0.834434in}}%
\pgfpathlineto{\pgfqpoint{1.677080in}{0.834404in}}%
\pgfpathlineto{\pgfqpoint{1.677376in}{0.834392in}}%
\pgfpathlineto{\pgfqpoint{1.677672in}{0.834382in}}%
\pgfpathlineto{\pgfqpoint{1.677968in}{0.834372in}}%
\pgfpathlineto{\pgfqpoint{1.678264in}{0.834363in}}%
\pgfpathlineto{\pgfqpoint{1.678560in}{0.834347in}}%
\pgfpathlineto{\pgfqpoint{1.678856in}{0.834327in}}%
\pgfpathlineto{\pgfqpoint{1.679152in}{0.834308in}}%
\pgfpathlineto{\pgfqpoint{1.679448in}{0.834289in}}%
\pgfpathlineto{\pgfqpoint{1.679744in}{0.834269in}}%
\pgfpathlineto{\pgfqpoint{1.680040in}{0.834237in}}%
\pgfpathlineto{\pgfqpoint{1.680336in}{0.834202in}}%
\pgfpathlineto{\pgfqpoint{1.680632in}{0.834166in}}%
\pgfpathlineto{\pgfqpoint{1.680928in}{0.834131in}}%
\pgfpathlineto{\pgfqpoint{1.681224in}{0.834096in}}%
\pgfpathlineto{\pgfqpoint{1.681520in}{0.834061in}}%
\pgfpathlineto{\pgfqpoint{1.681816in}{0.834026in}}%
\pgfpathlineto{\pgfqpoint{1.682112in}{0.833991in}}%
\pgfpathlineto{\pgfqpoint{1.682408in}{0.833956in}}%
\pgfpathlineto{\pgfqpoint{1.682704in}{0.833920in}}%
\pgfpathlineto{\pgfqpoint{1.683000in}{0.833885in}}%
\pgfpathlineto{\pgfqpoint{1.683296in}{0.833870in}}%
\pgfpathlineto{\pgfqpoint{1.683592in}{0.833923in}}%
\pgfpathlineto{\pgfqpoint{1.683888in}{0.833983in}}%
\pgfpathlineto{\pgfqpoint{1.684184in}{0.834068in}}%
\pgfpathlineto{\pgfqpoint{1.684480in}{0.834217in}}%
\pgfpathlineto{\pgfqpoint{1.684776in}{0.834369in}}%
\pgfpathlineto{\pgfqpoint{1.685072in}{0.834713in}}%
\pgfpathlineto{\pgfqpoint{1.685368in}{0.834896in}}%
\pgfpathlineto{\pgfqpoint{1.685664in}{0.834875in}}%
\pgfpathlineto{\pgfqpoint{1.685960in}{0.834855in}}%
\pgfpathlineto{\pgfqpoint{1.686256in}{0.834834in}}%
\pgfpathlineto{\pgfqpoint{1.686552in}{0.834814in}}%
\pgfpathlineto{\pgfqpoint{1.686848in}{0.834793in}}%
\pgfpathlineto{\pgfqpoint{1.687144in}{0.834772in}}%
\pgfpathlineto{\pgfqpoint{1.687440in}{0.834752in}}%
\pgfpathlineto{\pgfqpoint{1.687736in}{0.834731in}}%
\pgfpathlineto{\pgfqpoint{1.688032in}{0.834711in}}%
\pgfpathlineto{\pgfqpoint{1.688328in}{0.834690in}}%
\pgfpathlineto{\pgfqpoint{1.688624in}{0.834669in}}%
\pgfpathlineto{\pgfqpoint{1.688920in}{0.834649in}}%
\pgfpathlineto{\pgfqpoint{1.689216in}{0.834628in}}%
\pgfpathlineto{\pgfqpoint{1.689512in}{0.834608in}}%
\pgfpathlineto{\pgfqpoint{1.689808in}{0.834583in}}%
\pgfpathlineto{\pgfqpoint{1.690104in}{0.834471in}}%
\pgfpathlineto{\pgfqpoint{1.690400in}{0.834318in}}%
\pgfpathlineto{\pgfqpoint{1.690696in}{0.833389in}}%
\pgfpathlineto{\pgfqpoint{1.690992in}{0.833163in}}%
\pgfpathlineto{\pgfqpoint{1.691288in}{0.832979in}}%
\pgfpathlineto{\pgfqpoint{1.691584in}{0.832794in}}%
\pgfpathlineto{\pgfqpoint{1.691880in}{0.832609in}}%
\pgfpathlineto{\pgfqpoint{1.692176in}{0.832349in}}%
\pgfpathlineto{\pgfqpoint{1.692472in}{0.832165in}}%
\pgfpathlineto{\pgfqpoint{1.692768in}{0.832178in}}%
\pgfpathlineto{\pgfqpoint{1.693064in}{0.832191in}}%
\pgfpathlineto{\pgfqpoint{1.693360in}{0.832203in}}%
\pgfpathlineto{\pgfqpoint{1.693656in}{0.832216in}}%
\pgfpathlineto{\pgfqpoint{1.693952in}{0.832229in}}%
\pgfpathlineto{\pgfqpoint{1.694248in}{0.832242in}}%
\pgfpathlineto{\pgfqpoint{1.694544in}{0.832254in}}%
\pgfpathlineto{\pgfqpoint{1.694840in}{0.832267in}}%
\pgfpathlineto{\pgfqpoint{1.695136in}{0.832280in}}%
\pgfpathlineto{\pgfqpoint{1.695432in}{0.832293in}}%
\pgfpathlineto{\pgfqpoint{1.695728in}{0.832306in}}%
\pgfpathlineto{\pgfqpoint{1.696024in}{0.832318in}}%
\pgfpathlineto{\pgfqpoint{1.696320in}{0.832331in}}%
\pgfpathlineto{\pgfqpoint{1.696616in}{0.832344in}}%
\pgfpathlineto{\pgfqpoint{1.696912in}{0.832357in}}%
\pgfpathlineto{\pgfqpoint{1.697208in}{0.832369in}}%
\pgfpathlineto{\pgfqpoint{1.697504in}{0.832382in}}%
\pgfpathlineto{\pgfqpoint{1.697800in}{0.832395in}}%
\pgfpathlineto{\pgfqpoint{1.698096in}{0.832408in}}%
\pgfpathlineto{\pgfqpoint{1.698392in}{0.832420in}}%
\pgfpathlineto{\pgfqpoint{1.698688in}{0.832433in}}%
\pgfpathlineto{\pgfqpoint{1.698984in}{0.832446in}}%
\pgfpathlineto{\pgfqpoint{1.699280in}{0.832459in}}%
\pgfpathlineto{\pgfqpoint{1.699576in}{0.832471in}}%
\pgfpathlineto{\pgfqpoint{1.699872in}{0.832484in}}%
\pgfpathlineto{\pgfqpoint{1.700168in}{0.832497in}}%
\pgfpathlineto{\pgfqpoint{1.700464in}{0.832510in}}%
\pgfpathlineto{\pgfqpoint{1.700760in}{0.832523in}}%
\pgfpathlineto{\pgfqpoint{1.701056in}{0.832535in}}%
\pgfpathlineto{\pgfqpoint{1.701352in}{0.832548in}}%
\pgfpathlineto{\pgfqpoint{1.701648in}{0.832561in}}%
\pgfpathlineto{\pgfqpoint{1.701944in}{0.832574in}}%
\pgfpathlineto{\pgfqpoint{1.702240in}{0.832586in}}%
\pgfpathlineto{\pgfqpoint{1.702536in}{0.832599in}}%
\pgfpathlineto{\pgfqpoint{1.702832in}{0.832612in}}%
\pgfpathlineto{\pgfqpoint{1.703128in}{0.832625in}}%
\pgfpathlineto{\pgfqpoint{1.703424in}{0.832637in}}%
\pgfpathlineto{\pgfqpoint{1.703720in}{0.832650in}}%
\pgfpathlineto{\pgfqpoint{1.704016in}{0.832663in}}%
\pgfpathlineto{\pgfqpoint{1.704312in}{0.832676in}}%
\pgfpathlineto{\pgfqpoint{1.704608in}{0.832689in}}%
\pgfpathlineto{\pgfqpoint{1.704904in}{0.832701in}}%
\pgfpathlineto{\pgfqpoint{1.705200in}{0.832714in}}%
\pgfpathlineto{\pgfqpoint{1.705496in}{0.832727in}}%
\pgfpathlineto{\pgfqpoint{1.705792in}{0.832740in}}%
\pgfpathlineto{\pgfqpoint{1.706088in}{0.832752in}}%
\pgfpathlineto{\pgfqpoint{1.706384in}{0.832765in}}%
\pgfpathlineto{\pgfqpoint{1.706680in}{0.832778in}}%
\pgfpathlineto{\pgfqpoint{1.706976in}{0.832791in}}%
\pgfpathlineto{\pgfqpoint{1.707272in}{0.832803in}}%
\pgfpathlineto{\pgfqpoint{1.707568in}{0.832816in}}%
\pgfpathlineto{\pgfqpoint{1.707864in}{0.832829in}}%
\pgfpathlineto{\pgfqpoint{1.708160in}{0.832842in}}%
\pgfpathlineto{\pgfqpoint{1.708456in}{0.832854in}}%
\pgfpathlineto{\pgfqpoint{1.708752in}{0.832867in}}%
\pgfpathlineto{\pgfqpoint{1.709048in}{0.832880in}}%
\pgfpathlineto{\pgfqpoint{1.709344in}{0.832893in}}%
\pgfpathlineto{\pgfqpoint{1.709640in}{0.832906in}}%
\pgfpathlineto{\pgfqpoint{1.709936in}{0.832918in}}%
\pgfpathlineto{\pgfqpoint{1.710232in}{0.832931in}}%
\pgfpathlineto{\pgfqpoint{1.710528in}{0.832944in}}%
\pgfpathlineto{\pgfqpoint{1.710824in}{0.832957in}}%
\pgfpathlineto{\pgfqpoint{1.711120in}{0.832969in}}%
\pgfpathlineto{\pgfqpoint{1.711416in}{0.832982in}}%
\pgfpathlineto{\pgfqpoint{1.711712in}{0.833026in}}%
\pgfpathlineto{\pgfqpoint{1.712008in}{0.833061in}}%
\pgfpathlineto{\pgfqpoint{1.712304in}{0.833060in}}%
\pgfpathlineto{\pgfqpoint{1.712600in}{0.833058in}}%
\pgfpathlineto{\pgfqpoint{1.712896in}{0.833454in}}%
\pgfpathlineto{\pgfqpoint{1.713192in}{0.833931in}}%
\pgfpathlineto{\pgfqpoint{1.713488in}{0.833900in}}%
\pgfpathlineto{\pgfqpoint{1.713784in}{0.833838in}}%
\pgfpathlineto{\pgfqpoint{1.714080in}{0.833764in}}%
\pgfpathlineto{\pgfqpoint{1.714376in}{0.833690in}}%
\pgfpathlineto{\pgfqpoint{1.714672in}{0.833616in}}%
\pgfpathlineto{\pgfqpoint{1.714968in}{0.833542in}}%
\pgfpathlineto{\pgfqpoint{1.715264in}{0.834177in}}%
\pgfpathlineto{\pgfqpoint{1.715560in}{0.835237in}}%
\pgfpathlineto{\pgfqpoint{1.715856in}{0.835094in}}%
\pgfpathlineto{\pgfqpoint{1.716152in}{0.834951in}}%
\pgfpathlineto{\pgfqpoint{1.716448in}{0.834809in}}%
\pgfpathlineto{\pgfqpoint{1.716744in}{0.834666in}}%
\pgfpathlineto{\pgfqpoint{1.717040in}{0.834523in}}%
\pgfpathlineto{\pgfqpoint{1.717336in}{0.834380in}}%
\pgfpathlineto{\pgfqpoint{1.717632in}{0.834238in}}%
\pgfpathlineto{\pgfqpoint{1.717928in}{0.834095in}}%
\pgfpathlineto{\pgfqpoint{1.718224in}{0.833952in}}%
\pgfpathlineto{\pgfqpoint{1.718520in}{0.833990in}}%
\pgfpathlineto{\pgfqpoint{1.718816in}{0.835362in}}%
\pgfpathlineto{\pgfqpoint{1.719112in}{0.835377in}}%
\pgfpathlineto{\pgfqpoint{1.719408in}{0.835387in}}%
\pgfpathlineto{\pgfqpoint{1.719704in}{0.835402in}}%
\pgfpathlineto{\pgfqpoint{1.720000in}{0.835430in}}%
\pgfpathlineto{\pgfqpoint{1.720296in}{0.835260in}}%
\pgfpathlineto{\pgfqpoint{1.720592in}{0.834922in}}%
\pgfpathlineto{\pgfqpoint{1.720888in}{0.834746in}}%
\pgfpathlineto{\pgfqpoint{1.721184in}{0.834822in}}%
\pgfpathlineto{\pgfqpoint{1.721480in}{0.834904in}}%
\pgfpathlineto{\pgfqpoint{1.721776in}{0.834986in}}%
\pgfpathlineto{\pgfqpoint{1.722072in}{0.835069in}}%
\pgfpathlineto{\pgfqpoint{1.722368in}{0.835151in}}%
\pgfpathlineto{\pgfqpoint{1.722664in}{0.835233in}}%
\pgfpathlineto{\pgfqpoint{1.722960in}{0.835315in}}%
\pgfpathlineto{\pgfqpoint{1.723256in}{0.835397in}}%
\pgfpathlineto{\pgfqpoint{1.723552in}{0.835479in}}%
\pgfpathlineto{\pgfqpoint{1.723848in}{0.835562in}}%
\pgfpathlineto{\pgfqpoint{1.724144in}{0.835644in}}%
\pgfpathlineto{\pgfqpoint{1.724440in}{0.835726in}}%
\pgfpathlineto{\pgfqpoint{1.724736in}{0.835808in}}%
\pgfpathlineto{\pgfqpoint{1.725032in}{0.835890in}}%
\pgfpathlineto{\pgfqpoint{1.725328in}{0.835972in}}%
\pgfpathlineto{\pgfqpoint{1.725624in}{0.836055in}}%
\pgfpathlineto{\pgfqpoint{1.725920in}{0.836137in}}%
\pgfpathlineto{\pgfqpoint{1.726216in}{0.835981in}}%
\pgfpathlineto{\pgfqpoint{1.726512in}{0.835378in}}%
\pgfpathlineto{\pgfqpoint{1.726809in}{0.835349in}}%
\pgfpathlineto{\pgfqpoint{1.727105in}{0.835346in}}%
\pgfpathlineto{\pgfqpoint{1.727401in}{0.835344in}}%
\pgfpathlineto{\pgfqpoint{1.727697in}{0.835341in}}%
\pgfpathlineto{\pgfqpoint{1.727993in}{0.835338in}}%
\pgfpathlineto{\pgfqpoint{1.728289in}{0.835336in}}%
\pgfpathlineto{\pgfqpoint{1.728585in}{0.835333in}}%
\pgfpathlineto{\pgfqpoint{1.728881in}{0.835331in}}%
\pgfpathlineto{\pgfqpoint{1.729177in}{0.835328in}}%
\pgfpathlineto{\pgfqpoint{1.729473in}{0.835326in}}%
\pgfpathlineto{\pgfqpoint{1.729769in}{0.835323in}}%
\pgfpathlineto{\pgfqpoint{1.730065in}{0.835320in}}%
\pgfpathlineto{\pgfqpoint{1.730361in}{0.835318in}}%
\pgfpathlineto{\pgfqpoint{1.730657in}{0.835315in}}%
\pgfpathlineto{\pgfqpoint{1.730953in}{0.835313in}}%
\pgfpathlineto{\pgfqpoint{1.731249in}{0.835310in}}%
\pgfpathlineto{\pgfqpoint{1.731545in}{0.835308in}}%
\pgfpathlineto{\pgfqpoint{1.731841in}{0.835305in}}%
\pgfpathlineto{\pgfqpoint{1.732137in}{0.835302in}}%
\pgfpathlineto{\pgfqpoint{1.732433in}{0.835300in}}%
\pgfpathlineto{\pgfqpoint{1.732729in}{0.835297in}}%
\pgfpathlineto{\pgfqpoint{1.733025in}{0.835294in}}%
\pgfpathlineto{\pgfqpoint{1.733321in}{0.835290in}}%
\pgfpathlineto{\pgfqpoint{1.733617in}{0.835287in}}%
\pgfpathlineto{\pgfqpoint{1.733913in}{0.835283in}}%
\pgfpathlineto{\pgfqpoint{1.734209in}{0.835279in}}%
\pgfpathlineto{\pgfqpoint{1.734505in}{0.835275in}}%
\pgfpathlineto{\pgfqpoint{1.734801in}{0.835272in}}%
\pgfpathlineto{\pgfqpoint{1.735097in}{0.835268in}}%
\pgfpathlineto{\pgfqpoint{1.735393in}{0.835264in}}%
\pgfpathlineto{\pgfqpoint{1.735689in}{0.835260in}}%
\pgfpathlineto{\pgfqpoint{1.735985in}{0.835257in}}%
\pgfpathlineto{\pgfqpoint{1.736281in}{0.835253in}}%
\pgfpathlineto{\pgfqpoint{1.736577in}{0.835249in}}%
\pgfpathlineto{\pgfqpoint{1.736873in}{0.835245in}}%
\pgfpathlineto{\pgfqpoint{1.737169in}{0.835242in}}%
\pgfpathlineto{\pgfqpoint{1.737465in}{0.835238in}}%
\pgfpathlineto{\pgfqpoint{1.737761in}{0.835234in}}%
\pgfpathlineto{\pgfqpoint{1.738057in}{0.835230in}}%
\pgfpathlineto{\pgfqpoint{1.738353in}{0.835227in}}%
\pgfpathlineto{\pgfqpoint{1.738649in}{0.835223in}}%
\pgfpathlineto{\pgfqpoint{1.738945in}{0.835219in}}%
\pgfpathlineto{\pgfqpoint{1.739241in}{0.835215in}}%
\pgfpathlineto{\pgfqpoint{1.739537in}{0.835212in}}%
\pgfpathlineto{\pgfqpoint{1.739833in}{0.835208in}}%
\pgfpathlineto{\pgfqpoint{1.740129in}{0.835204in}}%
\pgfpathlineto{\pgfqpoint{1.740425in}{0.835244in}}%
\pgfpathlineto{\pgfqpoint{1.740721in}{0.835246in}}%
\pgfpathlineto{\pgfqpoint{1.741017in}{0.835244in}}%
\pgfpathlineto{\pgfqpoint{1.741313in}{0.835243in}}%
\pgfpathlineto{\pgfqpoint{1.741609in}{0.835244in}}%
\pgfpathlineto{\pgfqpoint{1.741905in}{0.835245in}}%
\pgfpathlineto{\pgfqpoint{1.742201in}{0.835246in}}%
\pgfpathlineto{\pgfqpoint{1.742497in}{0.835247in}}%
\pgfpathlineto{\pgfqpoint{1.742793in}{0.835248in}}%
\pgfpathlineto{\pgfqpoint{1.743089in}{0.835249in}}%
\pgfpathlineto{\pgfqpoint{1.743385in}{0.835250in}}%
\pgfpathlineto{\pgfqpoint{1.743681in}{0.835251in}}%
\pgfpathlineto{\pgfqpoint{1.743977in}{0.835252in}}%
\pgfpathlineto{\pgfqpoint{1.744273in}{0.835253in}}%
\pgfpathlineto{\pgfqpoint{1.744569in}{0.835254in}}%
\pgfpathlineto{\pgfqpoint{1.744865in}{0.835255in}}%
\pgfpathlineto{\pgfqpoint{1.745161in}{0.835256in}}%
\pgfpathlineto{\pgfqpoint{1.745457in}{0.835257in}}%
\pgfpathlineto{\pgfqpoint{1.745753in}{0.835258in}}%
\pgfpathlineto{\pgfqpoint{1.746049in}{0.835259in}}%
\pgfpathlineto{\pgfqpoint{1.746345in}{0.835260in}}%
\pgfpathlineto{\pgfqpoint{1.746641in}{0.835261in}}%
\pgfpathlineto{\pgfqpoint{1.746937in}{0.835261in}}%
\pgfpathlineto{\pgfqpoint{1.747233in}{0.835262in}}%
\pgfpathlineto{\pgfqpoint{1.747529in}{0.835263in}}%
\pgfpathlineto{\pgfqpoint{1.747825in}{0.835264in}}%
\pgfpathlineto{\pgfqpoint{1.748121in}{0.835265in}}%
\pgfpathlineto{\pgfqpoint{1.748417in}{0.835266in}}%
\pgfpathlineto{\pgfqpoint{1.748713in}{0.835267in}}%
\pgfpathlineto{\pgfqpoint{1.749009in}{0.835268in}}%
\pgfpathlineto{\pgfqpoint{1.749305in}{0.835269in}}%
\pgfpathlineto{\pgfqpoint{1.749601in}{0.835270in}}%
\pgfpathlineto{\pgfqpoint{1.749897in}{0.835271in}}%
\pgfpathlineto{\pgfqpoint{1.750193in}{0.835272in}}%
\pgfpathlineto{\pgfqpoint{1.750489in}{0.835273in}}%
\pgfpathlineto{\pgfqpoint{1.750785in}{0.835274in}}%
\pgfpathlineto{\pgfqpoint{1.751081in}{0.835275in}}%
\pgfpathlineto{\pgfqpoint{1.751377in}{0.835276in}}%
\pgfpathlineto{\pgfqpoint{1.751673in}{0.835277in}}%
\pgfpathlineto{\pgfqpoint{1.751969in}{0.835278in}}%
\pgfpathlineto{\pgfqpoint{1.752265in}{0.835279in}}%
\pgfpathlineto{\pgfqpoint{1.752561in}{0.835280in}}%
\pgfpathlineto{\pgfqpoint{1.752857in}{0.835281in}}%
\pgfpathlineto{\pgfqpoint{1.753153in}{0.835282in}}%
\pgfpathlineto{\pgfqpoint{1.753449in}{0.835283in}}%
\pgfpathlineto{\pgfqpoint{1.753745in}{0.835284in}}%
\pgfpathlineto{\pgfqpoint{1.754041in}{0.835285in}}%
\pgfpathlineto{\pgfqpoint{1.754337in}{0.835286in}}%
\pgfpathlineto{\pgfqpoint{1.754633in}{0.835287in}}%
\pgfpathlineto{\pgfqpoint{1.754929in}{0.835288in}}%
\pgfpathlineto{\pgfqpoint{1.755225in}{0.835289in}}%
\pgfpathlineto{\pgfqpoint{1.755521in}{0.835290in}}%
\pgfpathlineto{\pgfqpoint{1.755817in}{0.835291in}}%
\pgfpathlineto{\pgfqpoint{1.756113in}{0.835292in}}%
\pgfpathlineto{\pgfqpoint{1.756409in}{0.835293in}}%
\pgfpathlineto{\pgfqpoint{1.756705in}{0.835294in}}%
\pgfpathlineto{\pgfqpoint{1.757001in}{0.835295in}}%
\pgfpathlineto{\pgfqpoint{1.757297in}{0.835296in}}%
\pgfpathlineto{\pgfqpoint{1.757593in}{0.835297in}}%
\pgfpathlineto{\pgfqpoint{1.757889in}{0.835297in}}%
\pgfpathlineto{\pgfqpoint{1.758185in}{0.835298in}}%
\pgfpathlineto{\pgfqpoint{1.758481in}{0.835299in}}%
\pgfpathlineto{\pgfqpoint{1.758777in}{0.835300in}}%
\pgfpathlineto{\pgfqpoint{1.759073in}{0.835301in}}%
\pgfpathlineto{\pgfqpoint{1.759369in}{0.835302in}}%
\pgfpathlineto{\pgfqpoint{1.759665in}{0.835303in}}%
\pgfpathlineto{\pgfqpoint{1.759961in}{0.835304in}}%
\pgfpathlineto{\pgfqpoint{1.760257in}{0.835305in}}%
\pgfpathlineto{\pgfqpoint{1.760553in}{0.835306in}}%
\pgfpathlineto{\pgfqpoint{1.760849in}{0.835307in}}%
\pgfpathlineto{\pgfqpoint{1.761145in}{0.835308in}}%
\pgfpathlineto{\pgfqpoint{1.761441in}{0.835309in}}%
\pgfpathlineto{\pgfqpoint{1.761737in}{0.835310in}}%
\pgfpathlineto{\pgfqpoint{1.762033in}{0.835311in}}%
\pgfpathlineto{\pgfqpoint{1.762329in}{0.835312in}}%
\pgfpathlineto{\pgfqpoint{1.762625in}{0.835313in}}%
\pgfpathlineto{\pgfqpoint{1.762921in}{0.835314in}}%
\pgfpathlineto{\pgfqpoint{1.763217in}{0.835315in}}%
\pgfpathlineto{\pgfqpoint{1.763513in}{0.835316in}}%
\pgfpathlineto{\pgfqpoint{1.763809in}{0.835317in}}%
\pgfpathlineto{\pgfqpoint{1.764105in}{0.835318in}}%
\pgfpathlineto{\pgfqpoint{1.764401in}{0.835319in}}%
\pgfpathlineto{\pgfqpoint{1.764697in}{0.835320in}}%
\pgfpathlineto{\pgfqpoint{1.764993in}{0.835321in}}%
\pgfpathlineto{\pgfqpoint{1.765289in}{0.835322in}}%
\pgfpathlineto{\pgfqpoint{1.765585in}{0.835323in}}%
\pgfpathlineto{\pgfqpoint{1.765881in}{0.835324in}}%
\pgfpathlineto{\pgfqpoint{1.766177in}{0.835325in}}%
\pgfpathlineto{\pgfqpoint{1.766473in}{0.835326in}}%
\pgfpathlineto{\pgfqpoint{1.766769in}{0.835327in}}%
\pgfpathlineto{\pgfqpoint{1.767065in}{0.835328in}}%
\pgfpathlineto{\pgfqpoint{1.767361in}{0.835329in}}%
\pgfpathlineto{\pgfqpoint{1.767657in}{0.835330in}}%
\pgfpathlineto{\pgfqpoint{1.767953in}{0.835331in}}%
\pgfpathlineto{\pgfqpoint{1.768249in}{0.835332in}}%
\pgfpathlineto{\pgfqpoint{1.768545in}{0.835333in}}%
\pgfpathlineto{\pgfqpoint{1.768841in}{0.835333in}}%
\pgfpathlineto{\pgfqpoint{1.769137in}{0.835334in}}%
\pgfpathlineto{\pgfqpoint{1.769433in}{0.835335in}}%
\pgfpathlineto{\pgfqpoint{1.769729in}{0.835336in}}%
\pgfpathlineto{\pgfqpoint{1.770025in}{0.835337in}}%
\pgfpathlineto{\pgfqpoint{1.770321in}{0.835338in}}%
\pgfpathlineto{\pgfqpoint{1.770617in}{0.835339in}}%
\pgfpathlineto{\pgfqpoint{1.770913in}{0.835340in}}%
\pgfpathlineto{\pgfqpoint{1.771209in}{0.835341in}}%
\pgfpathlineto{\pgfqpoint{1.771505in}{0.835342in}}%
\pgfpathlineto{\pgfqpoint{1.771801in}{0.835343in}}%
\pgfpathlineto{\pgfqpoint{1.772097in}{0.835344in}}%
\pgfpathlineto{\pgfqpoint{1.772393in}{0.835345in}}%
\pgfpathlineto{\pgfqpoint{1.772689in}{0.835346in}}%
\pgfpathlineto{\pgfqpoint{1.772985in}{0.835347in}}%
\pgfpathlineto{\pgfqpoint{1.773281in}{0.835348in}}%
\pgfpathlineto{\pgfqpoint{1.773577in}{0.835349in}}%
\pgfpathlineto{\pgfqpoint{1.773873in}{0.835350in}}%
\pgfpathlineto{\pgfqpoint{1.774169in}{0.835351in}}%
\pgfpathlineto{\pgfqpoint{1.774465in}{0.835352in}}%
\pgfpathlineto{\pgfqpoint{1.774761in}{0.835353in}}%
\pgfpathlineto{\pgfqpoint{1.775057in}{0.835354in}}%
\pgfpathlineto{\pgfqpoint{1.775353in}{0.835355in}}%
\pgfpathlineto{\pgfqpoint{1.775649in}{0.835353in}}%
\pgfpathlineto{\pgfqpoint{1.775945in}{0.835352in}}%
\pgfpathlineto{\pgfqpoint{1.776241in}{0.835351in}}%
\pgfpathlineto{\pgfqpoint{1.776537in}{0.835350in}}%
\pgfpathlineto{\pgfqpoint{1.776833in}{0.835348in}}%
\pgfpathlineto{\pgfqpoint{1.777129in}{0.835348in}}%
\pgfpathlineto{\pgfqpoint{1.777425in}{0.835349in}}%
\pgfpathlineto{\pgfqpoint{1.777721in}{0.835350in}}%
\pgfpathlineto{\pgfqpoint{1.778017in}{0.835351in}}%
\pgfpathlineto{\pgfqpoint{1.778313in}{0.835352in}}%
\pgfpathlineto{\pgfqpoint{1.778609in}{0.835353in}}%
\pgfpathlineto{\pgfqpoint{1.778905in}{0.835354in}}%
\pgfpathlineto{\pgfqpoint{1.779201in}{0.835355in}}%
\pgfpathlineto{\pgfqpoint{1.779497in}{0.835356in}}%
\pgfpathlineto{\pgfqpoint{1.779793in}{0.835356in}}%
\pgfpathlineto{\pgfqpoint{1.780089in}{0.835357in}}%
\pgfpathlineto{\pgfqpoint{1.780385in}{0.835358in}}%
\pgfpathlineto{\pgfqpoint{1.780681in}{0.835359in}}%
\pgfpathlineto{\pgfqpoint{1.780977in}{0.835360in}}%
\pgfpathlineto{\pgfqpoint{1.781273in}{0.835361in}}%
\pgfpathlineto{\pgfqpoint{1.781569in}{0.835362in}}%
\pgfpathlineto{\pgfqpoint{1.781865in}{0.835363in}}%
\pgfpathlineto{\pgfqpoint{1.782161in}{0.835364in}}%
\pgfpathlineto{\pgfqpoint{1.782457in}{0.835365in}}%
\pgfpathlineto{\pgfqpoint{1.782753in}{0.835366in}}%
\pgfpathlineto{\pgfqpoint{1.783049in}{0.835365in}}%
\pgfpathlineto{\pgfqpoint{1.783345in}{0.835364in}}%
\pgfpathlineto{\pgfqpoint{1.783641in}{0.835362in}}%
\pgfpathlineto{\pgfqpoint{1.783937in}{0.835361in}}%
\pgfpathlineto{\pgfqpoint{1.784233in}{0.835360in}}%
\pgfpathlineto{\pgfqpoint{1.784529in}{0.835358in}}%
\pgfpathlineto{\pgfqpoint{1.784825in}{0.835358in}}%
\pgfpathlineto{\pgfqpoint{1.785121in}{0.835359in}}%
\pgfpathlineto{\pgfqpoint{1.785417in}{0.835360in}}%
\pgfpathlineto{\pgfqpoint{1.785713in}{0.835361in}}%
\pgfpathlineto{\pgfqpoint{1.786009in}{0.835362in}}%
\pgfpathlineto{\pgfqpoint{1.786305in}{0.835363in}}%
\pgfpathlineto{\pgfqpoint{1.786601in}{0.835365in}}%
\pgfpathlineto{\pgfqpoint{1.786897in}{0.835366in}}%
\pgfpathlineto{\pgfqpoint{1.787193in}{0.835367in}}%
\pgfpathlineto{\pgfqpoint{1.787489in}{0.835368in}}%
\pgfpathlineto{\pgfqpoint{1.787785in}{0.835369in}}%
\pgfpathlineto{\pgfqpoint{1.788081in}{0.835370in}}%
\pgfpathlineto{\pgfqpoint{1.788377in}{0.835371in}}%
\pgfpathlineto{\pgfqpoint{1.788673in}{0.835372in}}%
\pgfpathlineto{\pgfqpoint{1.788969in}{0.835373in}}%
\pgfpathlineto{\pgfqpoint{1.789265in}{0.835374in}}%
\pgfpathlineto{\pgfqpoint{1.789561in}{0.835375in}}%
\pgfpathlineto{\pgfqpoint{1.789857in}{0.835376in}}%
\pgfpathlineto{\pgfqpoint{1.790153in}{0.835377in}}%
\pgfpathlineto{\pgfqpoint{1.790449in}{0.835378in}}%
\pgfpathlineto{\pgfqpoint{1.790745in}{0.835378in}}%
\pgfpathlineto{\pgfqpoint{1.791041in}{0.835379in}}%
\pgfpathlineto{\pgfqpoint{1.791337in}{0.835380in}}%
\pgfpathlineto{\pgfqpoint{1.791633in}{0.835381in}}%
\pgfpathlineto{\pgfqpoint{1.791929in}{0.835381in}}%
\pgfpathlineto{\pgfqpoint{1.792225in}{0.835382in}}%
\pgfpathlineto{\pgfqpoint{1.792521in}{0.835383in}}%
\pgfpathlineto{\pgfqpoint{1.792817in}{0.835383in}}%
\pgfpathlineto{\pgfqpoint{1.793113in}{0.835384in}}%
\pgfpathlineto{\pgfqpoint{1.793409in}{0.835385in}}%
\pgfpathlineto{\pgfqpoint{1.793705in}{0.835385in}}%
\pgfpathlineto{\pgfqpoint{1.794001in}{0.835386in}}%
\pgfpathlineto{\pgfqpoint{1.794298in}{0.835387in}}%
\pgfpathlineto{\pgfqpoint{1.794594in}{0.835388in}}%
\pgfpathlineto{\pgfqpoint{1.794890in}{0.835388in}}%
\pgfpathlineto{\pgfqpoint{1.795186in}{0.835389in}}%
\pgfpathlineto{\pgfqpoint{1.795482in}{0.835390in}}%
\pgfpathlineto{\pgfqpoint{1.795778in}{0.835390in}}%
\pgfpathlineto{\pgfqpoint{1.796074in}{0.835391in}}%
\pgfpathlineto{\pgfqpoint{1.796370in}{0.835392in}}%
\pgfpathlineto{\pgfqpoint{1.796666in}{0.835392in}}%
\pgfpathlineto{\pgfqpoint{1.796962in}{0.835393in}}%
\pgfpathlineto{\pgfqpoint{1.797258in}{0.835394in}}%
\pgfpathlineto{\pgfqpoint{1.797554in}{0.835395in}}%
\pgfpathlineto{\pgfqpoint{1.797850in}{0.835395in}}%
\pgfpathlineto{\pgfqpoint{1.798146in}{0.835396in}}%
\pgfpathlineto{\pgfqpoint{1.798442in}{0.835397in}}%
\pgfpathlineto{\pgfqpoint{1.798738in}{0.835397in}}%
\pgfpathlineto{\pgfqpoint{1.799034in}{0.835398in}}%
\pgfpathlineto{\pgfqpoint{1.799330in}{0.835399in}}%
\pgfpathlineto{\pgfqpoint{1.799626in}{0.835399in}}%
\pgfpathlineto{\pgfqpoint{1.799922in}{0.835400in}}%
\pgfpathlineto{\pgfqpoint{1.800218in}{0.835401in}}%
\pgfpathlineto{\pgfqpoint{1.800514in}{0.835402in}}%
\pgfpathlineto{\pgfqpoint{1.800810in}{0.835402in}}%
\pgfpathlineto{\pgfqpoint{1.801106in}{0.835403in}}%
\pgfpathlineto{\pgfqpoint{1.801402in}{0.835404in}}%
\pgfpathlineto{\pgfqpoint{1.801698in}{0.835404in}}%
\pgfpathlineto{\pgfqpoint{1.801994in}{0.835405in}}%
\pgfpathlineto{\pgfqpoint{1.802290in}{0.835406in}}%
\pgfpathlineto{\pgfqpoint{1.802586in}{0.835406in}}%
\pgfpathlineto{\pgfqpoint{1.802882in}{0.835407in}}%
\pgfpathlineto{\pgfqpoint{1.803178in}{0.835408in}}%
\pgfpathlineto{\pgfqpoint{1.803474in}{0.835409in}}%
\pgfpathlineto{\pgfqpoint{1.803770in}{0.835409in}}%
\pgfpathlineto{\pgfqpoint{1.804066in}{0.835410in}}%
\pgfpathlineto{\pgfqpoint{1.804362in}{0.835411in}}%
\pgfpathlineto{\pgfqpoint{1.804658in}{0.835411in}}%
\pgfpathlineto{\pgfqpoint{1.804954in}{0.835412in}}%
\pgfpathlineto{\pgfqpoint{1.805250in}{0.835413in}}%
\pgfpathlineto{\pgfqpoint{1.805546in}{0.835413in}}%
\pgfpathlineto{\pgfqpoint{1.805842in}{0.835414in}}%
\pgfpathlineto{\pgfqpoint{1.806138in}{0.835415in}}%
\pgfpathlineto{\pgfqpoint{1.806434in}{0.835416in}}%
\pgfpathlineto{\pgfqpoint{1.806730in}{0.835416in}}%
\pgfpathlineto{\pgfqpoint{1.807026in}{0.835417in}}%
\pgfpathlineto{\pgfqpoint{1.807322in}{0.835418in}}%
\pgfpathlineto{\pgfqpoint{1.807618in}{0.835418in}}%
\pgfpathlineto{\pgfqpoint{1.807914in}{0.835419in}}%
\pgfpathlineto{\pgfqpoint{1.808210in}{0.835420in}}%
\pgfpathlineto{\pgfqpoint{1.808506in}{0.835420in}}%
\pgfpathlineto{\pgfqpoint{1.808802in}{0.835421in}}%
\pgfpathlineto{\pgfqpoint{1.809098in}{0.835422in}}%
\pgfpathlineto{\pgfqpoint{1.809394in}{0.835423in}}%
\pgfpathlineto{\pgfqpoint{1.809690in}{0.835423in}}%
\pgfpathlineto{\pgfqpoint{1.809986in}{0.835424in}}%
\pgfpathlineto{\pgfqpoint{1.810282in}{0.835425in}}%
\pgfpathlineto{\pgfqpoint{1.810578in}{0.835425in}}%
\pgfpathlineto{\pgfqpoint{1.810874in}{0.835425in}}%
\pgfpathlineto{\pgfqpoint{1.811170in}{0.835423in}}%
\pgfpathlineto{\pgfqpoint{1.811466in}{0.835418in}}%
\pgfpathlineto{\pgfqpoint{1.811762in}{0.835398in}}%
\pgfpathlineto{\pgfqpoint{1.812058in}{0.835375in}}%
\pgfpathlineto{\pgfqpoint{1.812354in}{0.835342in}}%
\pgfpathlineto{\pgfqpoint{1.812650in}{0.835306in}}%
\pgfpathlineto{\pgfqpoint{1.812946in}{0.835269in}}%
\pgfpathlineto{\pgfqpoint{1.813242in}{0.835233in}}%
\pgfpathlineto{\pgfqpoint{1.813538in}{0.835197in}}%
\pgfpathlineto{\pgfqpoint{1.813834in}{0.835161in}}%
\pgfpathlineto{\pgfqpoint{1.814130in}{0.835124in}}%
\pgfpathlineto{\pgfqpoint{1.814426in}{0.835088in}}%
\pgfpathlineto{\pgfqpoint{1.814722in}{0.835052in}}%
\pgfpathlineto{\pgfqpoint{1.815018in}{0.835016in}}%
\pgfpathlineto{\pgfqpoint{1.815314in}{0.834979in}}%
\pgfpathlineto{\pgfqpoint{1.815610in}{0.834943in}}%
\pgfpathlineto{\pgfqpoint{1.815906in}{0.834907in}}%
\pgfpathlineto{\pgfqpoint{1.816202in}{0.834871in}}%
\pgfpathlineto{\pgfqpoint{1.816498in}{0.834834in}}%
\pgfpathlineto{\pgfqpoint{1.816794in}{0.834798in}}%
\pgfpathlineto{\pgfqpoint{1.817090in}{0.834762in}}%
\pgfpathlineto{\pgfqpoint{1.817386in}{0.834725in}}%
\pgfpathlineto{\pgfqpoint{1.817682in}{0.834689in}}%
\pgfpathlineto{\pgfqpoint{1.817978in}{0.834653in}}%
\pgfpathlineto{\pgfqpoint{1.818274in}{0.834676in}}%
\pgfpathlineto{\pgfqpoint{1.818570in}{0.835046in}}%
\pgfpathlineto{\pgfqpoint{1.818866in}{0.834985in}}%
\pgfpathlineto{\pgfqpoint{1.819162in}{0.834896in}}%
\pgfpathlineto{\pgfqpoint{1.819458in}{0.834806in}}%
\pgfpathlineto{\pgfqpoint{1.819754in}{0.834717in}}%
\pgfpathlineto{\pgfqpoint{1.820050in}{0.834627in}}%
\pgfpathlineto{\pgfqpoint{1.820346in}{0.834538in}}%
\pgfpathlineto{\pgfqpoint{1.820642in}{0.834448in}}%
\pgfpathlineto{\pgfqpoint{1.820938in}{0.834359in}}%
\pgfpathlineto{\pgfqpoint{1.821234in}{0.834270in}}%
\pgfpathlineto{\pgfqpoint{1.821530in}{0.834180in}}%
\pgfpathlineto{\pgfqpoint{1.821826in}{0.834091in}}%
\pgfpathlineto{\pgfqpoint{1.822122in}{0.834001in}}%
\pgfpathlineto{\pgfqpoint{1.822418in}{0.833912in}}%
\pgfpathlineto{\pgfqpoint{1.822714in}{0.833822in}}%
\pgfpathlineto{\pgfqpoint{1.823010in}{0.833733in}}%
\pgfpathlineto{\pgfqpoint{1.823306in}{0.833643in}}%
\pgfpathlineto{\pgfqpoint{1.823602in}{0.833554in}}%
\pgfpathlineto{\pgfqpoint{1.823898in}{0.833465in}}%
\pgfpathlineto{\pgfqpoint{1.824194in}{0.833375in}}%
\pgfpathlineto{\pgfqpoint{1.824490in}{0.833286in}}%
\pgfpathlineto{\pgfqpoint{1.824786in}{0.833196in}}%
\pgfpathlineto{\pgfqpoint{1.825082in}{0.833107in}}%
\pgfpathlineto{\pgfqpoint{1.825378in}{0.833017in}}%
\pgfpathlineto{\pgfqpoint{1.825674in}{0.832928in}}%
\pgfpathlineto{\pgfqpoint{1.825970in}{0.832839in}}%
\pgfpathlineto{\pgfqpoint{1.826266in}{0.832766in}}%
\pgfpathlineto{\pgfqpoint{1.826562in}{0.832763in}}%
\pgfpathlineto{\pgfqpoint{1.826858in}{0.832769in}}%
\pgfpathlineto{\pgfqpoint{1.827154in}{0.832774in}}%
\pgfpathlineto{\pgfqpoint{1.827450in}{0.832780in}}%
\pgfpathlineto{\pgfqpoint{1.827746in}{0.832785in}}%
\pgfpathlineto{\pgfqpoint{1.828042in}{0.832791in}}%
\pgfpathlineto{\pgfqpoint{1.828338in}{0.832797in}}%
\pgfpathlineto{\pgfqpoint{1.828634in}{0.832802in}}%
\pgfpathlineto{\pgfqpoint{1.828930in}{0.832808in}}%
\pgfpathlineto{\pgfqpoint{1.829226in}{0.832813in}}%
\pgfpathlineto{\pgfqpoint{1.829522in}{0.832819in}}%
\pgfpathlineto{\pgfqpoint{1.829818in}{0.832824in}}%
\pgfpathlineto{\pgfqpoint{1.830114in}{0.832830in}}%
\pgfpathlineto{\pgfqpoint{1.830410in}{0.832835in}}%
\pgfpathlineto{\pgfqpoint{1.830706in}{0.832841in}}%
\pgfpathlineto{\pgfqpoint{1.831002in}{0.832846in}}%
\pgfpathlineto{\pgfqpoint{1.831298in}{0.832852in}}%
\pgfpathlineto{\pgfqpoint{1.831594in}{0.832857in}}%
\pgfpathlineto{\pgfqpoint{1.831890in}{0.832863in}}%
\pgfpathlineto{\pgfqpoint{1.832186in}{0.832868in}}%
\pgfpathlineto{\pgfqpoint{1.832482in}{0.832874in}}%
\pgfpathlineto{\pgfqpoint{1.832778in}{0.832879in}}%
\pgfpathlineto{\pgfqpoint{1.833074in}{0.832885in}}%
\pgfpathlineto{\pgfqpoint{1.833370in}{0.832891in}}%
\pgfpathlineto{\pgfqpoint{1.833666in}{0.832896in}}%
\pgfpathlineto{\pgfqpoint{1.833962in}{0.832902in}}%
\pgfpathlineto{\pgfqpoint{1.834258in}{0.832930in}}%
\pgfpathlineto{\pgfqpoint{1.834554in}{0.833011in}}%
\pgfpathlineto{\pgfqpoint{1.834850in}{0.833010in}}%
\pgfpathlineto{\pgfqpoint{1.835146in}{0.833010in}}%
\pgfpathlineto{\pgfqpoint{1.835442in}{0.833009in}}%
\pgfpathlineto{\pgfqpoint{1.835738in}{0.833008in}}%
\pgfpathlineto{\pgfqpoint{1.836034in}{0.833007in}}%
\pgfpathlineto{\pgfqpoint{1.836330in}{0.833006in}}%
\pgfpathlineto{\pgfqpoint{1.836626in}{0.833006in}}%
\pgfpathlineto{\pgfqpoint{1.836922in}{0.833005in}}%
\pgfpathlineto{\pgfqpoint{1.837218in}{0.833004in}}%
\pgfpathlineto{\pgfqpoint{1.837514in}{0.833003in}}%
\pgfpathlineto{\pgfqpoint{1.837810in}{0.833003in}}%
\pgfpathlineto{\pgfqpoint{1.838106in}{0.833002in}}%
\pgfpathlineto{\pgfqpoint{1.838402in}{0.833001in}}%
\pgfpathlineto{\pgfqpoint{1.838698in}{0.833000in}}%
\pgfpathlineto{\pgfqpoint{1.838994in}{0.832999in}}%
\pgfpathlineto{\pgfqpoint{1.839290in}{0.832999in}}%
\pgfpathlineto{\pgfqpoint{1.839586in}{0.832999in}}%
\pgfpathlineto{\pgfqpoint{1.839882in}{0.832781in}}%
\pgfpathlineto{\pgfqpoint{1.840178in}{0.833014in}}%
\pgfpathlineto{\pgfqpoint{1.840474in}{0.833009in}}%
\pgfpathlineto{\pgfqpoint{1.840770in}{0.833007in}}%
\pgfpathlineto{\pgfqpoint{1.841066in}{0.833005in}}%
\pgfpathlineto{\pgfqpoint{1.841362in}{0.833003in}}%
\pgfpathlineto{\pgfqpoint{1.841658in}{0.832998in}}%
\pgfpathlineto{\pgfqpoint{1.841954in}{0.832989in}}%
\pgfpathlineto{\pgfqpoint{1.842250in}{0.832979in}}%
\pgfpathlineto{\pgfqpoint{1.842546in}{0.832969in}}%
\pgfpathlineto{\pgfqpoint{1.842842in}{0.832959in}}%
\pgfpathlineto{\pgfqpoint{1.843138in}{0.832950in}}%
\pgfpathlineto{\pgfqpoint{1.843434in}{0.832940in}}%
\pgfpathlineto{\pgfqpoint{1.843730in}{0.832930in}}%
\pgfpathlineto{\pgfqpoint{1.844026in}{0.832920in}}%
\pgfpathlineto{\pgfqpoint{1.844322in}{0.832911in}}%
\pgfpathlineto{\pgfqpoint{1.844618in}{0.832901in}}%
\pgfpathlineto{\pgfqpoint{1.844914in}{0.832891in}}%
\pgfpathlineto{\pgfqpoint{1.845210in}{0.832881in}}%
\pgfpathlineto{\pgfqpoint{1.845506in}{0.832872in}}%
\pgfpathlineto{\pgfqpoint{1.845802in}{0.832862in}}%
\pgfpathlineto{\pgfqpoint{1.846098in}{0.832852in}}%
\pgfpathlineto{\pgfqpoint{1.846394in}{0.832842in}}%
\pgfpathlineto{\pgfqpoint{1.846690in}{0.832833in}}%
\pgfpathlineto{\pgfqpoint{1.846986in}{0.832823in}}%
\pgfpathlineto{\pgfqpoint{1.847282in}{0.832813in}}%
\pgfpathlineto{\pgfqpoint{1.847578in}{0.832803in}}%
\pgfpathlineto{\pgfqpoint{1.847874in}{0.832794in}}%
\pgfpathlineto{\pgfqpoint{1.848170in}{0.832784in}}%
\pgfpathlineto{\pgfqpoint{1.848466in}{0.832774in}}%
\pgfpathlineto{\pgfqpoint{1.848762in}{0.832764in}}%
\pgfpathlineto{\pgfqpoint{1.849058in}{0.832755in}}%
\pgfpathlineto{\pgfqpoint{1.849354in}{0.832745in}}%
\pgfpathlineto{\pgfqpoint{1.849650in}{0.832735in}}%
\pgfpathlineto{\pgfqpoint{1.849946in}{0.832725in}}%
\pgfpathlineto{\pgfqpoint{1.850242in}{0.832715in}}%
\pgfpathlineto{\pgfqpoint{1.850538in}{0.832706in}}%
\pgfpathlineto{\pgfqpoint{1.850834in}{0.832696in}}%
\pgfpathlineto{\pgfqpoint{1.851130in}{0.832686in}}%
\pgfpathlineto{\pgfqpoint{1.851426in}{0.832676in}}%
\pgfpathlineto{\pgfqpoint{1.851722in}{0.832667in}}%
\pgfpathlineto{\pgfqpoint{1.852018in}{0.832657in}}%
\pgfpathlineto{\pgfqpoint{1.852314in}{0.832647in}}%
\pgfpathlineto{\pgfqpoint{1.852610in}{0.832637in}}%
\pgfpathlineto{\pgfqpoint{1.852906in}{0.832628in}}%
\pgfpathlineto{\pgfqpoint{1.853202in}{0.832618in}}%
\pgfpathlineto{\pgfqpoint{1.853498in}{0.832608in}}%
\pgfpathlineto{\pgfqpoint{1.853794in}{0.832598in}}%
\pgfpathlineto{\pgfqpoint{1.854090in}{0.832589in}}%
\pgfpathlineto{\pgfqpoint{1.854386in}{0.832579in}}%
\pgfpathlineto{\pgfqpoint{1.854682in}{0.832569in}}%
\pgfpathlineto{\pgfqpoint{1.854978in}{0.832559in}}%
\pgfpathlineto{\pgfqpoint{1.855274in}{0.832550in}}%
\pgfpathlineto{\pgfqpoint{1.855570in}{0.832540in}}%
\pgfpathlineto{\pgfqpoint{1.855866in}{0.832530in}}%
\pgfpathlineto{\pgfqpoint{1.856162in}{0.832520in}}%
\pgfpathlineto{\pgfqpoint{1.856458in}{0.832511in}}%
\pgfpathlineto{\pgfqpoint{1.856754in}{0.832501in}}%
\pgfpathlineto{\pgfqpoint{1.857050in}{0.832491in}}%
\pgfpathlineto{\pgfqpoint{1.857346in}{0.832481in}}%
\pgfpathlineto{\pgfqpoint{1.857642in}{0.832472in}}%
\pgfpathlineto{\pgfqpoint{1.857938in}{0.832462in}}%
\pgfpathlineto{\pgfqpoint{1.858234in}{0.832452in}}%
\pgfpathlineto{\pgfqpoint{1.858530in}{0.832442in}}%
\pgfpathlineto{\pgfqpoint{1.858826in}{0.832433in}}%
\pgfpathlineto{\pgfqpoint{1.859122in}{0.832423in}}%
\pgfpathlineto{\pgfqpoint{1.859418in}{0.832413in}}%
\pgfpathlineto{\pgfqpoint{1.859714in}{0.832403in}}%
\pgfpathlineto{\pgfqpoint{1.860010in}{0.832394in}}%
\pgfpathlineto{\pgfqpoint{1.860306in}{0.832384in}}%
\pgfpathlineto{\pgfqpoint{1.860602in}{0.832374in}}%
\pgfpathlineto{\pgfqpoint{1.860898in}{0.832364in}}%
\pgfpathlineto{\pgfqpoint{1.861194in}{0.832355in}}%
\pgfpathlineto{\pgfqpoint{1.861491in}{0.832345in}}%
\pgfpathlineto{\pgfqpoint{1.861787in}{0.832335in}}%
\pgfpathlineto{\pgfqpoint{1.862083in}{0.832325in}}%
\pgfpathlineto{\pgfqpoint{1.862379in}{0.832316in}}%
\pgfpathlineto{\pgfqpoint{1.862675in}{0.832306in}}%
\pgfpathlineto{\pgfqpoint{1.862971in}{0.832296in}}%
\pgfpathlineto{\pgfqpoint{1.863267in}{0.832286in}}%
\pgfpathlineto{\pgfqpoint{1.863563in}{0.832277in}}%
\pgfpathlineto{\pgfqpoint{1.863859in}{0.832267in}}%
\pgfpathlineto{\pgfqpoint{1.864155in}{0.832257in}}%
\pgfpathlineto{\pgfqpoint{1.864451in}{0.832247in}}%
\pgfpathlineto{\pgfqpoint{1.864747in}{0.832237in}}%
\pgfpathlineto{\pgfqpoint{1.865043in}{0.832228in}}%
\pgfpathlineto{\pgfqpoint{1.865339in}{0.832218in}}%
\pgfpathlineto{\pgfqpoint{1.865635in}{0.832208in}}%
\pgfpathlineto{\pgfqpoint{1.865931in}{0.832198in}}%
\pgfpathlineto{\pgfqpoint{1.866227in}{0.832189in}}%
\pgfpathlineto{\pgfqpoint{1.866523in}{0.832179in}}%
\pgfpathlineto{\pgfqpoint{1.866819in}{0.832169in}}%
\pgfpathlineto{\pgfqpoint{1.867115in}{0.832159in}}%
\pgfpathlineto{\pgfqpoint{1.867411in}{0.832150in}}%
\pgfpathlineto{\pgfqpoint{1.867707in}{0.832140in}}%
\pgfpathlineto{\pgfqpoint{1.868003in}{0.832130in}}%
\pgfpathlineto{\pgfqpoint{1.868299in}{0.832120in}}%
\pgfpathlineto{\pgfqpoint{1.868595in}{0.832111in}}%
\pgfpathlineto{\pgfqpoint{1.868891in}{0.832101in}}%
\pgfpathlineto{\pgfqpoint{1.869187in}{0.832091in}}%
\pgfpathlineto{\pgfqpoint{1.869483in}{0.832081in}}%
\pgfpathlineto{\pgfqpoint{1.869779in}{0.832072in}}%
\pgfpathlineto{\pgfqpoint{1.870075in}{0.832062in}}%
\pgfpathlineto{\pgfqpoint{1.870371in}{0.832052in}}%
\pgfpathlineto{\pgfqpoint{1.870667in}{0.832042in}}%
\pgfpathlineto{\pgfqpoint{1.870963in}{0.832033in}}%
\pgfpathlineto{\pgfqpoint{1.871259in}{0.832023in}}%
\pgfpathlineto{\pgfqpoint{1.871555in}{0.832013in}}%
\pgfpathlineto{\pgfqpoint{1.871851in}{0.832003in}}%
\pgfpathlineto{\pgfqpoint{1.872147in}{0.831994in}}%
\pgfpathlineto{\pgfqpoint{1.872443in}{0.831984in}}%
\pgfpathlineto{\pgfqpoint{1.872739in}{0.831974in}}%
\pgfpathlineto{\pgfqpoint{1.873035in}{0.831964in}}%
\pgfpathlineto{\pgfqpoint{1.873331in}{0.831955in}}%
\pgfpathlineto{\pgfqpoint{1.873627in}{0.831945in}}%
\pgfpathlineto{\pgfqpoint{1.873923in}{0.831935in}}%
\pgfpathlineto{\pgfqpoint{1.874219in}{0.831925in}}%
\pgfpathlineto{\pgfqpoint{1.874515in}{0.831923in}}%
\pgfpathlineto{\pgfqpoint{1.874811in}{0.831953in}}%
\pgfpathlineto{\pgfqpoint{1.875107in}{0.831987in}}%
\pgfpathlineto{\pgfqpoint{1.875403in}{0.832022in}}%
\pgfpathlineto{\pgfqpoint{1.875699in}{0.832057in}}%
\pgfpathlineto{\pgfqpoint{1.875995in}{0.832091in}}%
\pgfpathlineto{\pgfqpoint{1.876291in}{0.832126in}}%
\pgfpathlineto{\pgfqpoint{1.876587in}{0.832161in}}%
\pgfpathlineto{\pgfqpoint{1.876883in}{0.832195in}}%
\pgfpathlineto{\pgfqpoint{1.877179in}{0.832230in}}%
\pgfpathlineto{\pgfqpoint{1.877475in}{0.832264in}}%
\pgfpathlineto{\pgfqpoint{1.877771in}{0.832299in}}%
\pgfpathlineto{\pgfqpoint{1.878067in}{0.832334in}}%
\pgfpathlineto{\pgfqpoint{1.878363in}{0.832368in}}%
\pgfpathlineto{\pgfqpoint{1.878659in}{0.832403in}}%
\pgfpathlineto{\pgfqpoint{1.878955in}{0.832437in}}%
\pgfpathlineto{\pgfqpoint{1.879251in}{0.832472in}}%
\pgfpathlineto{\pgfqpoint{1.879547in}{0.832507in}}%
\pgfpathlineto{\pgfqpoint{1.879843in}{0.832541in}}%
\pgfpathlineto{\pgfqpoint{1.880139in}{0.832576in}}%
\pgfpathlineto{\pgfqpoint{1.880435in}{0.832611in}}%
\pgfpathlineto{\pgfqpoint{1.880731in}{0.832645in}}%
\pgfpathlineto{\pgfqpoint{1.881027in}{0.832680in}}%
\pgfpathlineto{\pgfqpoint{1.881323in}{0.832714in}}%
\pgfpathlineto{\pgfqpoint{1.881619in}{0.832749in}}%
\pgfpathlineto{\pgfqpoint{1.881915in}{0.832784in}}%
\pgfpathlineto{\pgfqpoint{1.882211in}{0.832818in}}%
\pgfpathlineto{\pgfqpoint{1.882507in}{0.832853in}}%
\pgfpathlineto{\pgfqpoint{1.882803in}{0.832887in}}%
\pgfpathlineto{\pgfqpoint{1.883099in}{0.832922in}}%
\pgfpathlineto{\pgfqpoint{1.883395in}{0.832957in}}%
\pgfpathlineto{\pgfqpoint{1.883691in}{0.832989in}}%
\pgfpathlineto{\pgfqpoint{1.883987in}{0.832994in}}%
\pgfpathlineto{\pgfqpoint{1.884283in}{0.832993in}}%
\pgfpathlineto{\pgfqpoint{1.884579in}{0.832992in}}%
\pgfpathlineto{\pgfqpoint{1.884875in}{0.832991in}}%
\pgfpathlineto{\pgfqpoint{1.885171in}{0.832990in}}%
\pgfpathlineto{\pgfqpoint{1.885467in}{0.832989in}}%
\pgfpathlineto{\pgfqpoint{1.885763in}{0.832988in}}%
\pgfpathlineto{\pgfqpoint{1.886059in}{0.832987in}}%
\pgfpathlineto{\pgfqpoint{1.886355in}{0.832986in}}%
\pgfpathlineto{\pgfqpoint{1.886651in}{0.832985in}}%
\pgfpathlineto{\pgfqpoint{1.886947in}{0.832984in}}%
\pgfpathlineto{\pgfqpoint{1.887243in}{0.832983in}}%
\pgfpathlineto{\pgfqpoint{1.887539in}{0.832982in}}%
\pgfpathlineto{\pgfqpoint{1.887835in}{0.832981in}}%
\pgfpathlineto{\pgfqpoint{1.888131in}{0.832980in}}%
\pgfpathlineto{\pgfqpoint{1.888427in}{0.832979in}}%
\pgfpathlineto{\pgfqpoint{1.888723in}{0.832978in}}%
\pgfpathlineto{\pgfqpoint{1.889019in}{0.832977in}}%
\pgfpathlineto{\pgfqpoint{1.889315in}{0.832986in}}%
\pgfpathlineto{\pgfqpoint{1.889611in}{0.832977in}}%
\pgfpathlineto{\pgfqpoint{1.889907in}{0.833071in}}%
\pgfpathlineto{\pgfqpoint{1.890203in}{0.833227in}}%
\pgfpathlineto{\pgfqpoint{1.890499in}{0.833227in}}%
\pgfpathlineto{\pgfqpoint{1.890795in}{0.833226in}}%
\pgfpathlineto{\pgfqpoint{1.891091in}{0.833226in}}%
\pgfpathlineto{\pgfqpoint{1.891387in}{0.833226in}}%
\pgfpathlineto{\pgfqpoint{1.891683in}{0.833225in}}%
\pgfpathlineto{\pgfqpoint{1.891979in}{0.833225in}}%
\pgfpathlineto{\pgfqpoint{1.892275in}{0.833224in}}%
\pgfpathlineto{\pgfqpoint{1.892571in}{0.833224in}}%
\pgfpathlineto{\pgfqpoint{1.892867in}{0.833224in}}%
\pgfpathlineto{\pgfqpoint{1.893163in}{0.833223in}}%
\pgfpathlineto{\pgfqpoint{1.893459in}{0.833223in}}%
\pgfpathlineto{\pgfqpoint{1.893755in}{0.833222in}}%
\pgfpathlineto{\pgfqpoint{1.894051in}{0.833222in}}%
\pgfpathlineto{\pgfqpoint{1.894347in}{0.833222in}}%
\pgfpathlineto{\pgfqpoint{1.894643in}{0.833221in}}%
\pgfpathlineto{\pgfqpoint{1.894939in}{0.833221in}}%
\pgfpathlineto{\pgfqpoint{1.895235in}{0.833220in}}%
\pgfpathlineto{\pgfqpoint{1.895531in}{0.833220in}}%
\pgfpathlineto{\pgfqpoint{1.895827in}{0.833219in}}%
\pgfpathlineto{\pgfqpoint{1.896123in}{0.833219in}}%
\pgfpathlineto{\pgfqpoint{1.896419in}{0.833219in}}%
\pgfpathlineto{\pgfqpoint{1.896715in}{0.833218in}}%
\pgfpathlineto{\pgfqpoint{1.897011in}{0.833218in}}%
\pgfpathlineto{\pgfqpoint{1.897307in}{0.833217in}}%
\pgfpathlineto{\pgfqpoint{1.897603in}{0.833217in}}%
\pgfpathlineto{\pgfqpoint{1.897899in}{0.833217in}}%
\pgfpathlineto{\pgfqpoint{1.898195in}{0.833216in}}%
\pgfpathlineto{\pgfqpoint{1.898491in}{0.833216in}}%
\pgfpathlineto{\pgfqpoint{1.898787in}{0.833215in}}%
\pgfpathlineto{\pgfqpoint{1.899083in}{0.833215in}}%
\pgfpathlineto{\pgfqpoint{1.899379in}{0.833215in}}%
\pgfpathlineto{\pgfqpoint{1.899675in}{0.833214in}}%
\pgfpathlineto{\pgfqpoint{1.899971in}{0.833214in}}%
\pgfpathlineto{\pgfqpoint{1.900267in}{0.833213in}}%
\pgfpathlineto{\pgfqpoint{1.900563in}{0.833213in}}%
\pgfpathlineto{\pgfqpoint{1.900859in}{0.833212in}}%
\pgfpathlineto{\pgfqpoint{1.901155in}{0.833212in}}%
\pgfpathlineto{\pgfqpoint{1.901451in}{0.833212in}}%
\pgfpathlineto{\pgfqpoint{1.901747in}{0.833211in}}%
\pgfpathlineto{\pgfqpoint{1.902043in}{0.833211in}}%
\pgfpathlineto{\pgfqpoint{1.902339in}{0.833210in}}%
\pgfpathlineto{\pgfqpoint{1.902635in}{0.833210in}}%
\pgfpathlineto{\pgfqpoint{1.902931in}{0.833210in}}%
\pgfpathlineto{\pgfqpoint{1.903227in}{0.833209in}}%
\pgfpathlineto{\pgfqpoint{1.903523in}{0.833209in}}%
\pgfpathlineto{\pgfqpoint{1.903819in}{0.833208in}}%
\pgfpathlineto{\pgfqpoint{1.904115in}{0.833208in}}%
\pgfpathlineto{\pgfqpoint{1.904411in}{0.833208in}}%
\pgfpathlineto{\pgfqpoint{1.904707in}{0.833207in}}%
\pgfpathlineto{\pgfqpoint{1.905003in}{0.833207in}}%
\pgfpathlineto{\pgfqpoint{1.905299in}{0.833206in}}%
\pgfpathlineto{\pgfqpoint{1.905595in}{0.833206in}}%
\pgfpathlineto{\pgfqpoint{1.905891in}{0.833205in}}%
\pgfpathlineto{\pgfqpoint{1.906187in}{0.833205in}}%
\pgfpathlineto{\pgfqpoint{1.906483in}{0.833205in}}%
\pgfpathlineto{\pgfqpoint{1.906779in}{0.833204in}}%
\pgfpathlineto{\pgfqpoint{1.907075in}{0.833204in}}%
\pgfpathlineto{\pgfqpoint{1.907371in}{0.833203in}}%
\pgfpathlineto{\pgfqpoint{1.907667in}{0.833203in}}%
\pgfpathlineto{\pgfqpoint{1.907963in}{0.833203in}}%
\pgfpathlineto{\pgfqpoint{1.908259in}{0.833202in}}%
\pgfpathlineto{\pgfqpoint{1.908555in}{0.833202in}}%
\pgfpathlineto{\pgfqpoint{1.908851in}{0.833201in}}%
\pgfpathlineto{\pgfqpoint{1.909147in}{0.833201in}}%
\pgfpathlineto{\pgfqpoint{1.909443in}{0.833201in}}%
\pgfpathlineto{\pgfqpoint{1.909739in}{0.833200in}}%
\pgfpathlineto{\pgfqpoint{1.910035in}{0.833200in}}%
\pgfpathlineto{\pgfqpoint{1.910331in}{0.833199in}}%
\pgfpathlineto{\pgfqpoint{1.910627in}{0.833199in}}%
\pgfpathlineto{\pgfqpoint{1.910923in}{0.833198in}}%
\pgfpathlineto{\pgfqpoint{1.911219in}{0.833198in}}%
\pgfpathlineto{\pgfqpoint{1.911515in}{0.833197in}}%
\pgfpathlineto{\pgfqpoint{1.911811in}{0.833197in}}%
\pgfpathlineto{\pgfqpoint{1.912107in}{0.833196in}}%
\pgfpathlineto{\pgfqpoint{1.912403in}{0.833195in}}%
\pgfpathlineto{\pgfqpoint{1.912699in}{0.833194in}}%
\pgfpathlineto{\pgfqpoint{1.912995in}{0.833193in}}%
\pgfpathlineto{\pgfqpoint{1.913291in}{0.833192in}}%
\pgfpathlineto{\pgfqpoint{1.913587in}{0.833191in}}%
\pgfpathlineto{\pgfqpoint{1.913883in}{0.833190in}}%
\pgfpathlineto{\pgfqpoint{1.914179in}{0.833189in}}%
\pgfpathlineto{\pgfqpoint{1.914475in}{0.833188in}}%
\pgfpathlineto{\pgfqpoint{1.914771in}{0.833187in}}%
\pgfpathlineto{\pgfqpoint{1.915067in}{0.833186in}}%
\pgfpathlineto{\pgfqpoint{1.915363in}{0.833185in}}%
\pgfpathlineto{\pgfqpoint{1.915659in}{0.833184in}}%
\pgfpathlineto{\pgfqpoint{1.915955in}{0.833183in}}%
\pgfpathlineto{\pgfqpoint{1.916251in}{0.833182in}}%
\pgfpathlineto{\pgfqpoint{1.916547in}{0.833181in}}%
\pgfpathlineto{\pgfqpoint{1.916843in}{0.833180in}}%
\pgfpathlineto{\pgfqpoint{1.917139in}{0.833180in}}%
\pgfpathlineto{\pgfqpoint{1.917435in}{0.833179in}}%
\pgfpathlineto{\pgfqpoint{1.917731in}{0.833167in}}%
\pgfpathlineto{\pgfqpoint{1.918027in}{0.833077in}}%
\pgfpathlineto{\pgfqpoint{1.918323in}{0.832972in}}%
\pgfpathlineto{\pgfqpoint{1.918619in}{0.832863in}}%
\pgfpathlineto{\pgfqpoint{1.918915in}{0.832747in}}%
\pgfpathlineto{\pgfqpoint{1.919211in}{0.832625in}}%
\pgfpathlineto{\pgfqpoint{1.919507in}{0.832499in}}%
\pgfpathlineto{\pgfqpoint{1.919803in}{0.832374in}}%
\pgfpathlineto{\pgfqpoint{1.920099in}{0.832248in}}%
\pgfpathlineto{\pgfqpoint{1.920395in}{0.832122in}}%
\pgfpathlineto{\pgfqpoint{1.920691in}{0.831996in}}%
\pgfpathlineto{\pgfqpoint{1.920987in}{0.831870in}}%
\pgfpathlineto{\pgfqpoint{1.921283in}{0.831744in}}%
\pgfpathlineto{\pgfqpoint{1.921579in}{0.831619in}}%
\pgfpathlineto{\pgfqpoint{1.921875in}{0.831493in}}%
\pgfpathlineto{\pgfqpoint{1.922171in}{0.831367in}}%
\pgfpathlineto{\pgfqpoint{1.922467in}{0.831241in}}%
\pgfpathlineto{\pgfqpoint{1.922763in}{0.831115in}}%
\pgfpathlineto{\pgfqpoint{1.923059in}{0.830989in}}%
\pgfpathlineto{\pgfqpoint{1.923355in}{0.830863in}}%
\pgfpathlineto{\pgfqpoint{1.923651in}{0.830738in}}%
\pgfpathlineto{\pgfqpoint{1.923947in}{0.830612in}}%
\pgfpathlineto{\pgfqpoint{1.924243in}{0.830486in}}%
\pgfpathlineto{\pgfqpoint{1.924539in}{0.830360in}}%
\pgfpathlineto{\pgfqpoint{1.924835in}{0.830234in}}%
\pgfpathlineto{\pgfqpoint{1.925131in}{0.830103in}}%
\pgfpathlineto{\pgfqpoint{1.925427in}{0.829972in}}%
\pgfpathlineto{\pgfqpoint{1.925723in}{0.829844in}}%
\pgfpathlineto{\pgfqpoint{1.926019in}{0.829720in}}%
\pgfpathlineto{\pgfqpoint{1.926315in}{0.829608in}}%
\pgfpathlineto{\pgfqpoint{1.926611in}{0.829477in}}%
\pgfpathlineto{\pgfqpoint{1.926907in}{0.829394in}}%
\pgfpathlineto{\pgfqpoint{1.927203in}{0.829351in}}%
\pgfpathlineto{\pgfqpoint{1.927499in}{0.829308in}}%
\pgfpathlineto{\pgfqpoint{1.927795in}{0.829265in}}%
\pgfpathlineto{\pgfqpoint{1.928091in}{0.829221in}}%
\pgfpathlineto{\pgfqpoint{1.928387in}{0.829178in}}%
\pgfpathlineto{\pgfqpoint{1.928683in}{0.829135in}}%
\pgfpathlineto{\pgfqpoint{1.928980in}{0.829092in}}%
\pgfpathlineto{\pgfqpoint{1.929276in}{0.829048in}}%
\pgfpathlineto{\pgfqpoint{1.929572in}{0.829005in}}%
\pgfpathlineto{\pgfqpoint{1.929868in}{0.828962in}}%
\pgfpathlineto{\pgfqpoint{1.930164in}{0.828918in}}%
\pgfpathlineto{\pgfqpoint{1.930460in}{0.828875in}}%
\pgfpathlineto{\pgfqpoint{1.930756in}{0.828832in}}%
\pgfpathlineto{\pgfqpoint{1.931052in}{0.828789in}}%
\pgfpathlineto{\pgfqpoint{1.931348in}{0.828745in}}%
\pgfpathlineto{\pgfqpoint{1.931644in}{0.828738in}}%
\pgfpathlineto{\pgfqpoint{1.931940in}{0.828793in}}%
\pgfpathlineto{\pgfqpoint{1.932236in}{0.828791in}}%
\pgfpathlineto{\pgfqpoint{1.932532in}{0.828780in}}%
\pgfpathlineto{\pgfqpoint{1.932828in}{0.828744in}}%
\pgfpathlineto{\pgfqpoint{1.933124in}{0.828709in}}%
\pgfpathlineto{\pgfqpoint{1.933420in}{0.828674in}}%
\pgfpathlineto{\pgfqpoint{1.933716in}{0.828639in}}%
\pgfpathlineto{\pgfqpoint{1.934012in}{0.828602in}}%
\pgfpathlineto{\pgfqpoint{1.934308in}{0.828564in}}%
\pgfpathlineto{\pgfqpoint{1.934604in}{0.828526in}}%
\pgfpathlineto{\pgfqpoint{1.934900in}{0.828488in}}%
\pgfpathlineto{\pgfqpoint{1.935196in}{0.828450in}}%
\pgfpathlineto{\pgfqpoint{1.935492in}{0.828411in}}%
\pgfpathlineto{\pgfqpoint{1.935788in}{0.828373in}}%
\pgfpathlineto{\pgfqpoint{1.936084in}{0.828335in}}%
\pgfpathlineto{\pgfqpoint{1.936380in}{0.828297in}}%
\pgfpathlineto{\pgfqpoint{1.936676in}{0.828259in}}%
\pgfpathlineto{\pgfqpoint{1.936972in}{0.828221in}}%
\pgfpathlineto{\pgfqpoint{1.937268in}{0.828182in}}%
\pgfpathlineto{\pgfqpoint{1.937564in}{0.828144in}}%
\pgfpathlineto{\pgfqpoint{1.937860in}{0.828106in}}%
\pgfpathlineto{\pgfqpoint{1.938156in}{0.828068in}}%
\pgfpathlineto{\pgfqpoint{1.938452in}{0.828030in}}%
\pgfpathlineto{\pgfqpoint{1.938748in}{0.827992in}}%
\pgfpathlineto{\pgfqpoint{1.939044in}{0.827952in}}%
\pgfpathlineto{\pgfqpoint{1.939340in}{0.827908in}}%
\pgfpathlineto{\pgfqpoint{1.939636in}{0.827851in}}%
\pgfpathlineto{\pgfqpoint{1.939932in}{0.827591in}}%
\pgfpathlineto{\pgfqpoint{1.940228in}{0.827263in}}%
\pgfpathlineto{\pgfqpoint{1.940524in}{0.826953in}}%
\pgfpathlineto{\pgfqpoint{1.940820in}{0.826875in}}%
\pgfpathlineto{\pgfqpoint{1.941116in}{0.826870in}}%
\pgfpathlineto{\pgfqpoint{1.941412in}{0.826866in}}%
\pgfpathlineto{\pgfqpoint{1.941708in}{0.826862in}}%
\pgfpathlineto{\pgfqpoint{1.942004in}{0.826858in}}%
\pgfpathlineto{\pgfqpoint{1.942300in}{0.826854in}}%
\pgfpathlineto{\pgfqpoint{1.942596in}{0.826850in}}%
\pgfpathlineto{\pgfqpoint{1.942892in}{0.826846in}}%
\pgfpathlineto{\pgfqpoint{1.943188in}{0.826841in}}%
\pgfpathlineto{\pgfqpoint{1.943484in}{0.826837in}}%
\pgfpathlineto{\pgfqpoint{1.943780in}{0.826833in}}%
\pgfpathlineto{\pgfqpoint{1.944076in}{0.826829in}}%
\pgfpathlineto{\pgfqpoint{1.944372in}{0.826825in}}%
\pgfpathlineto{\pgfqpoint{1.944668in}{0.826821in}}%
\pgfpathlineto{\pgfqpoint{1.944964in}{0.826816in}}%
\pgfpathlineto{\pgfqpoint{1.945260in}{0.826812in}}%
\pgfpathlineto{\pgfqpoint{1.945556in}{0.826808in}}%
\pgfpathlineto{\pgfqpoint{1.945852in}{0.826804in}}%
\pgfpathlineto{\pgfqpoint{1.946148in}{0.826800in}}%
\pgfpathlineto{\pgfqpoint{1.946444in}{0.826796in}}%
\pgfpathlineto{\pgfqpoint{1.946740in}{0.826791in}}%
\pgfpathlineto{\pgfqpoint{1.947036in}{0.826787in}}%
\pgfpathlineto{\pgfqpoint{1.947332in}{0.826783in}}%
\pgfpathlineto{\pgfqpoint{1.947628in}{0.826779in}}%
\pgfpathlineto{\pgfqpoint{1.947924in}{0.826775in}}%
\pgfpathlineto{\pgfqpoint{1.948220in}{0.826771in}}%
\pgfpathlineto{\pgfqpoint{1.948516in}{0.826766in}}%
\pgfpathlineto{\pgfqpoint{1.948812in}{0.826762in}}%
\pgfpathlineto{\pgfqpoint{1.949108in}{0.826758in}}%
\pgfpathlineto{\pgfqpoint{1.949404in}{0.826754in}}%
\pgfpathlineto{\pgfqpoint{1.949700in}{0.826750in}}%
\pgfpathlineto{\pgfqpoint{1.949996in}{0.826746in}}%
\pgfpathlineto{\pgfqpoint{1.950292in}{0.826741in}}%
\pgfpathlineto{\pgfqpoint{1.950588in}{0.826737in}}%
\pgfpathlineto{\pgfqpoint{1.950884in}{0.826733in}}%
\pgfpathlineto{\pgfqpoint{1.951180in}{0.826729in}}%
\pgfpathlineto{\pgfqpoint{1.951476in}{0.826725in}}%
\pgfpathlineto{\pgfqpoint{1.951772in}{0.826721in}}%
\pgfpathlineto{\pgfqpoint{1.952068in}{0.826716in}}%
\pgfpathlineto{\pgfqpoint{1.952364in}{0.826712in}}%
\pgfpathlineto{\pgfqpoint{1.952660in}{0.826708in}}%
\pgfpathlineto{\pgfqpoint{1.952956in}{0.826704in}}%
\pgfpathlineto{\pgfqpoint{1.953252in}{0.826700in}}%
\pgfpathlineto{\pgfqpoint{1.953548in}{0.826696in}}%
\pgfpathlineto{\pgfqpoint{1.953844in}{0.826691in}}%
\pgfpathlineto{\pgfqpoint{1.954140in}{0.826687in}}%
\pgfpathlineto{\pgfqpoint{1.954436in}{0.826683in}}%
\pgfpathlineto{\pgfqpoint{1.954732in}{0.826679in}}%
\pgfpathlineto{\pgfqpoint{1.955028in}{0.826675in}}%
\pgfpathlineto{\pgfqpoint{1.955324in}{0.826671in}}%
\pgfpathlineto{\pgfqpoint{1.955620in}{0.826666in}}%
\pgfpathlineto{\pgfqpoint{1.955916in}{0.826662in}}%
\pgfpathlineto{\pgfqpoint{1.956212in}{0.826658in}}%
\pgfpathlineto{\pgfqpoint{1.956508in}{0.826654in}}%
\pgfpathlineto{\pgfqpoint{1.956804in}{0.826650in}}%
\pgfpathlineto{\pgfqpoint{1.957100in}{0.826646in}}%
\pgfpathlineto{\pgfqpoint{1.957396in}{0.826641in}}%
\pgfpathlineto{\pgfqpoint{1.957692in}{0.826637in}}%
\pgfpathlineto{\pgfqpoint{1.957988in}{0.826633in}}%
\pgfpathlineto{\pgfqpoint{1.958284in}{0.826629in}}%
\pgfpathlineto{\pgfqpoint{1.958580in}{0.826625in}}%
\pgfpathlineto{\pgfqpoint{1.958876in}{0.826621in}}%
\pgfpathlineto{\pgfqpoint{1.959172in}{0.826617in}}%
\pgfpathlineto{\pgfqpoint{1.959468in}{0.826612in}}%
\pgfpathlineto{\pgfqpoint{1.959764in}{0.826624in}}%
\pgfpathlineto{\pgfqpoint{1.960060in}{0.826573in}}%
\pgfpathlineto{\pgfqpoint{1.960356in}{0.827233in}}%
\pgfpathlineto{\pgfqpoint{1.960652in}{0.826893in}}%
\pgfpathlineto{\pgfqpoint{1.960948in}{0.828707in}}%
\pgfpathlineto{\pgfqpoint{1.961244in}{0.829766in}}%
\pgfpathlineto{\pgfqpoint{1.961540in}{0.830860in}}%
\pgfpathlineto{\pgfqpoint{1.961836in}{0.831954in}}%
\pgfpathlineto{\pgfqpoint{1.962132in}{0.832974in}}%
\pgfpathlineto{\pgfqpoint{1.962428in}{0.833358in}}%
\pgfpathlineto{\pgfqpoint{1.962724in}{0.833213in}}%
\pgfpathlineto{\pgfqpoint{1.963020in}{0.833007in}}%
\pgfpathlineto{\pgfqpoint{1.963316in}{0.832801in}}%
\pgfpathlineto{\pgfqpoint{1.963612in}{0.832595in}}%
\pgfpathlineto{\pgfqpoint{1.963908in}{0.832389in}}%
\pgfpathlineto{\pgfqpoint{1.964204in}{0.832183in}}%
\pgfpathlineto{\pgfqpoint{1.964500in}{0.831977in}}%
\pgfpathlineto{\pgfqpoint{1.964796in}{0.831770in}}%
\pgfpathlineto{\pgfqpoint{1.965092in}{0.831564in}}%
\pgfpathlineto{\pgfqpoint{1.965388in}{0.831358in}}%
\pgfpathlineto{\pgfqpoint{1.965684in}{0.831152in}}%
\pgfpathlineto{\pgfqpoint{1.965980in}{0.830946in}}%
\pgfpathlineto{\pgfqpoint{1.966276in}{0.830740in}}%
\pgfpathlineto{\pgfqpoint{1.966572in}{0.830534in}}%
\pgfpathlineto{\pgfqpoint{1.966868in}{0.830328in}}%
\pgfpathlineto{\pgfqpoint{1.967164in}{0.830122in}}%
\pgfpathlineto{\pgfqpoint{1.967460in}{0.829915in}}%
\pgfpathlineto{\pgfqpoint{1.967756in}{0.829709in}}%
\pgfpathlineto{\pgfqpoint{1.968052in}{0.829503in}}%
\pgfpathlineto{\pgfqpoint{1.968348in}{0.829297in}}%
\pgfpathlineto{\pgfqpoint{1.968644in}{0.829120in}}%
\pgfpathlineto{\pgfqpoint{1.968940in}{0.829103in}}%
\pgfpathlineto{\pgfqpoint{1.969236in}{0.829145in}}%
\pgfpathlineto{\pgfqpoint{1.969532in}{0.829371in}}%
\pgfpathlineto{\pgfqpoint{1.969828in}{0.829622in}}%
\pgfpathlineto{\pgfqpoint{1.970124in}{0.829872in}}%
\pgfpathlineto{\pgfqpoint{1.970420in}{0.830123in}}%
\pgfpathlineto{\pgfqpoint{1.970716in}{0.830374in}}%
\pgfpathlineto{\pgfqpoint{1.971012in}{0.830624in}}%
\pgfpathlineto{\pgfqpoint{1.971308in}{0.830875in}}%
\pgfpathlineto{\pgfqpoint{1.971604in}{0.831125in}}%
\pgfpathlineto{\pgfqpoint{1.971900in}{0.831376in}}%
\pgfpathlineto{\pgfqpoint{1.972196in}{0.831627in}}%
\pgfpathlineto{\pgfqpoint{1.972492in}{0.831877in}}%
\pgfpathlineto{\pgfqpoint{1.972788in}{0.832128in}}%
\pgfpathlineto{\pgfqpoint{1.973084in}{0.832378in}}%
\pgfpathlineto{\pgfqpoint{1.973380in}{0.832629in}}%
\pgfpathlineto{\pgfqpoint{1.973676in}{0.832880in}}%
\pgfpathlineto{\pgfqpoint{1.973972in}{0.833130in}}%
\pgfpathlineto{\pgfqpoint{1.974268in}{0.830952in}}%
\pgfpathlineto{\pgfqpoint{1.974564in}{0.829011in}}%
\pgfpathlineto{\pgfqpoint{1.974860in}{0.830226in}}%
\pgfpathlineto{\pgfqpoint{1.975156in}{0.832877in}}%
\pgfpathlineto{\pgfqpoint{1.975452in}{0.833211in}}%
\pgfpathlineto{\pgfqpoint{1.975748in}{0.832926in}}%
\pgfpathlineto{\pgfqpoint{1.976044in}{0.832642in}}%
\pgfpathlineto{\pgfqpoint{1.976340in}{0.832354in}}%
\pgfpathlineto{\pgfqpoint{1.976636in}{0.831831in}}%
\pgfpathlineto{\pgfqpoint{1.976932in}{0.831170in}}%
\pgfpathlineto{\pgfqpoint{1.977228in}{0.830686in}}%
\pgfpathlineto{\pgfqpoint{1.977524in}{0.830384in}}%
\pgfpathlineto{\pgfqpoint{1.977820in}{0.830083in}}%
\pgfpathlineto{\pgfqpoint{1.978116in}{0.829781in}}%
\pgfpathlineto{\pgfqpoint{1.978412in}{0.829480in}}%
\pgfpathlineto{\pgfqpoint{1.978708in}{0.829179in}}%
\pgfpathlineto{\pgfqpoint{1.979004in}{0.828877in}}%
\pgfpathlineto{\pgfqpoint{1.979300in}{0.828576in}}%
\pgfpathlineto{\pgfqpoint{1.979596in}{0.828275in}}%
\pgfpathlineto{\pgfqpoint{1.979892in}{0.827973in}}%
\pgfpathlineto{\pgfqpoint{1.980188in}{0.827672in}}%
\pgfpathlineto{\pgfqpoint{1.980484in}{0.827370in}}%
\pgfpathlineto{\pgfqpoint{1.980780in}{0.827069in}}%
\pgfpathlineto{\pgfqpoint{1.981076in}{0.826768in}}%
\pgfpathlineto{\pgfqpoint{1.981372in}{0.831216in}}%
\pgfpathlineto{\pgfqpoint{1.981668in}{0.832313in}}%
\pgfpathlineto{\pgfqpoint{1.981964in}{0.829603in}}%
\pgfpathlineto{\pgfqpoint{1.982260in}{0.828133in}}%
\pgfpathlineto{\pgfqpoint{1.982556in}{0.831040in}}%
\pgfpathlineto{\pgfqpoint{1.982852in}{0.830791in}}%
\pgfpathlineto{\pgfqpoint{1.983148in}{0.830547in}}%
\pgfpathlineto{\pgfqpoint{1.983444in}{0.830304in}}%
\pgfpathlineto{\pgfqpoint{1.983740in}{0.830060in}}%
\pgfpathlineto{\pgfqpoint{1.984036in}{0.829816in}}%
\pgfpathlineto{\pgfqpoint{1.984332in}{0.829572in}}%
\pgfpathlineto{\pgfqpoint{1.984628in}{0.829328in}}%
\pgfpathlineto{\pgfqpoint{1.984924in}{0.829084in}}%
\pgfpathlineto{\pgfqpoint{1.985220in}{0.828840in}}%
\pgfpathlineto{\pgfqpoint{1.985516in}{0.828596in}}%
\pgfpathlineto{\pgfqpoint{1.985812in}{0.828352in}}%
\pgfpathlineto{\pgfqpoint{1.986108in}{0.828108in}}%
\pgfpathlineto{\pgfqpoint{1.986404in}{0.827864in}}%
\pgfpathlineto{\pgfqpoint{1.986700in}{0.827620in}}%
\pgfpathlineto{\pgfqpoint{1.986996in}{0.827376in}}%
\pgfpathlineto{\pgfqpoint{1.987292in}{0.827132in}}%
\pgfpathlineto{\pgfqpoint{1.987588in}{0.826889in}}%
\pgfpathlineto{\pgfqpoint{1.987884in}{0.826645in}}%
\pgfpathlineto{\pgfqpoint{1.988180in}{0.826401in}}%
\pgfpathlineto{\pgfqpoint{1.988476in}{0.826159in}}%
\pgfpathlineto{\pgfqpoint{1.988772in}{0.826203in}}%
\pgfpathlineto{\pgfqpoint{1.989068in}{0.827412in}}%
\pgfpathlineto{\pgfqpoint{1.989364in}{0.830228in}}%
\pgfpathlineto{\pgfqpoint{1.989660in}{0.828939in}}%
\pgfpathlineto{\pgfqpoint{1.989956in}{0.828718in}}%
\pgfpathlineto{\pgfqpoint{1.990252in}{0.830681in}}%
\pgfpathlineto{\pgfqpoint{1.990548in}{0.830631in}}%
\pgfpathlineto{\pgfqpoint{1.990844in}{0.830581in}}%
\pgfpathlineto{\pgfqpoint{1.991140in}{0.830533in}}%
\pgfpathlineto{\pgfqpoint{1.991436in}{0.830496in}}%
\pgfpathlineto{\pgfqpoint{1.991732in}{0.830460in}}%
\pgfpathlineto{\pgfqpoint{1.992028in}{0.830424in}}%
\pgfpathlineto{\pgfqpoint{1.992324in}{0.830389in}}%
\pgfpathlineto{\pgfqpoint{1.992620in}{0.830485in}}%
\pgfpathlineto{\pgfqpoint{1.992916in}{0.830622in}}%
\pgfpathlineto{\pgfqpoint{1.993212in}{0.830615in}}%
\pgfpathlineto{\pgfqpoint{1.993508in}{0.830608in}}%
\pgfpathlineto{\pgfqpoint{1.993804in}{0.830601in}}%
\pgfpathlineto{\pgfqpoint{1.994100in}{0.830594in}}%
\pgfpathlineto{\pgfqpoint{1.994396in}{0.830587in}}%
\pgfpathlineto{\pgfqpoint{1.994692in}{0.830579in}}%
\pgfpathlineto{\pgfqpoint{1.994988in}{0.830572in}}%
\pgfpathlineto{\pgfqpoint{1.995284in}{0.830565in}}%
\pgfpathlineto{\pgfqpoint{1.995580in}{0.830558in}}%
\pgfpathlineto{\pgfqpoint{1.995876in}{0.830551in}}%
\pgfpathlineto{\pgfqpoint{1.996172in}{0.830543in}}%
\pgfpathlineto{\pgfqpoint{1.996469in}{0.830536in}}%
\pgfpathlineto{\pgfqpoint{1.996765in}{0.830529in}}%
\pgfpathlineto{\pgfqpoint{1.997061in}{0.830522in}}%
\pgfpathlineto{\pgfqpoint{1.997357in}{0.830515in}}%
\pgfpathlineto{\pgfqpoint{1.997653in}{0.830507in}}%
\pgfpathlineto{\pgfqpoint{1.997949in}{0.830500in}}%
\pgfpathlineto{\pgfqpoint{1.998245in}{0.830493in}}%
\pgfpathlineto{\pgfqpoint{1.998541in}{0.830486in}}%
\pgfpathlineto{\pgfqpoint{1.998837in}{0.830479in}}%
\pgfpathlineto{\pgfqpoint{1.999133in}{0.830471in}}%
\pgfpathlineto{\pgfqpoint{1.999429in}{0.830464in}}%
\pgfpathlineto{\pgfqpoint{1.999725in}{0.830457in}}%
\pgfpathlineto{\pgfqpoint{2.000021in}{0.830450in}}%
\pgfpathlineto{\pgfqpoint{2.000317in}{0.830443in}}%
\pgfpathlineto{\pgfqpoint{2.000613in}{0.830435in}}%
\pgfpathlineto{\pgfqpoint{2.000909in}{0.830428in}}%
\pgfpathlineto{\pgfqpoint{2.001205in}{0.830421in}}%
\pgfpathlineto{\pgfqpoint{2.001501in}{0.830414in}}%
\pgfpathlineto{\pgfqpoint{2.001797in}{0.830407in}}%
\pgfpathlineto{\pgfqpoint{2.002093in}{0.830399in}}%
\pgfpathlineto{\pgfqpoint{2.002389in}{0.830392in}}%
\pgfpathlineto{\pgfqpoint{2.002685in}{0.830385in}}%
\pgfpathlineto{\pgfqpoint{2.002981in}{0.830378in}}%
\pgfpathlineto{\pgfqpoint{2.003277in}{0.830371in}}%
\pgfpathlineto{\pgfqpoint{2.003573in}{0.830364in}}%
\pgfpathlineto{\pgfqpoint{2.003869in}{0.830356in}}%
\pgfpathlineto{\pgfqpoint{2.004165in}{0.830349in}}%
\pgfpathlineto{\pgfqpoint{2.004461in}{0.830342in}}%
\pgfpathlineto{\pgfqpoint{2.004757in}{0.830335in}}%
\pgfpathlineto{\pgfqpoint{2.005053in}{0.830328in}}%
\pgfpathlineto{\pgfqpoint{2.005349in}{0.830320in}}%
\pgfpathlineto{\pgfqpoint{2.005645in}{0.830313in}}%
\pgfpathlineto{\pgfqpoint{2.005941in}{0.830306in}}%
\pgfpathlineto{\pgfqpoint{2.006237in}{0.830299in}}%
\pgfpathlineto{\pgfqpoint{2.006533in}{0.830292in}}%
\pgfpathlineto{\pgfqpoint{2.006829in}{0.830284in}}%
\pgfpathlineto{\pgfqpoint{2.007125in}{0.830277in}}%
\pgfpathlineto{\pgfqpoint{2.007421in}{0.830270in}}%
\pgfpathlineto{\pgfqpoint{2.007717in}{0.830263in}}%
\pgfpathlineto{\pgfqpoint{2.008013in}{0.830256in}}%
\pgfpathlineto{\pgfqpoint{2.008309in}{0.830248in}}%
\pgfpathlineto{\pgfqpoint{2.008605in}{0.830241in}}%
\pgfpathlineto{\pgfqpoint{2.008901in}{0.830234in}}%
\pgfpathlineto{\pgfqpoint{2.009197in}{0.830227in}}%
\pgfpathlineto{\pgfqpoint{2.009493in}{0.830220in}}%
\pgfpathlineto{\pgfqpoint{2.009789in}{0.830212in}}%
\pgfpathlineto{\pgfqpoint{2.010085in}{0.830205in}}%
\pgfpathlineto{\pgfqpoint{2.010381in}{0.830198in}}%
\pgfpathlineto{\pgfqpoint{2.010677in}{0.830159in}}%
\pgfpathlineto{\pgfqpoint{2.010973in}{0.829639in}}%
\pgfpathlineto{\pgfqpoint{2.011269in}{0.829504in}}%
\pgfpathlineto{\pgfqpoint{2.011565in}{0.829404in}}%
\pgfpathlineto{\pgfqpoint{2.011861in}{0.829397in}}%
\pgfpathlineto{\pgfqpoint{2.012157in}{0.829394in}}%
\pgfpathlineto{\pgfqpoint{2.012453in}{0.829392in}}%
\pgfpathlineto{\pgfqpoint{2.012749in}{0.829390in}}%
\pgfpathlineto{\pgfqpoint{2.013045in}{0.829388in}}%
\pgfpathlineto{\pgfqpoint{2.013341in}{0.829385in}}%
\pgfpathlineto{\pgfqpoint{2.013637in}{0.829383in}}%
\pgfpathlineto{\pgfqpoint{2.013933in}{0.829381in}}%
\pgfpathlineto{\pgfqpoint{2.014229in}{0.829379in}}%
\pgfpathlineto{\pgfqpoint{2.014525in}{0.829377in}}%
\pgfpathlineto{\pgfqpoint{2.014821in}{0.829374in}}%
\pgfpathlineto{\pgfqpoint{2.015117in}{0.829372in}}%
\pgfpathlineto{\pgfqpoint{2.015413in}{0.829370in}}%
\pgfpathlineto{\pgfqpoint{2.015709in}{0.829368in}}%
\pgfpathlineto{\pgfqpoint{2.016005in}{0.829366in}}%
\pgfpathlineto{\pgfqpoint{2.016301in}{0.829363in}}%
\pgfpathlineto{\pgfqpoint{2.016597in}{0.829361in}}%
\pgfpathlineto{\pgfqpoint{2.016893in}{0.829367in}}%
\pgfpathlineto{\pgfqpoint{2.017189in}{0.829461in}}%
\pgfpathlineto{\pgfqpoint{2.017485in}{0.829462in}}%
\pgfpathlineto{\pgfqpoint{2.017781in}{0.829463in}}%
\pgfpathlineto{\pgfqpoint{2.018077in}{0.829464in}}%
\pgfpathlineto{\pgfqpoint{2.018373in}{0.829464in}}%
\pgfpathlineto{\pgfqpoint{2.018669in}{0.829462in}}%
\pgfpathlineto{\pgfqpoint{2.018965in}{0.829459in}}%
\pgfpathlineto{\pgfqpoint{2.019261in}{0.829456in}}%
\pgfpathlineto{\pgfqpoint{2.019557in}{0.829453in}}%
\pgfpathlineto{\pgfqpoint{2.019853in}{0.829450in}}%
\pgfpathlineto{\pgfqpoint{2.020149in}{0.829448in}}%
\pgfpathlineto{\pgfqpoint{2.020445in}{0.829445in}}%
\pgfpathlineto{\pgfqpoint{2.020741in}{0.829442in}}%
\pgfpathlineto{\pgfqpoint{2.021037in}{0.829439in}}%
\pgfpathlineto{\pgfqpoint{2.021333in}{0.829436in}}%
\pgfpathlineto{\pgfqpoint{2.021629in}{0.829433in}}%
\pgfpathlineto{\pgfqpoint{2.021925in}{0.829431in}}%
\pgfpathlineto{\pgfqpoint{2.022221in}{0.829428in}}%
\pgfpathlineto{\pgfqpoint{2.022517in}{0.829425in}}%
\pgfpathlineto{\pgfqpoint{2.022813in}{0.829422in}}%
\pgfpathlineto{\pgfqpoint{2.023109in}{0.829419in}}%
\pgfpathlineto{\pgfqpoint{2.023405in}{0.829416in}}%
\pgfpathlineto{\pgfqpoint{2.023701in}{0.829413in}}%
\pgfpathlineto{\pgfqpoint{2.023997in}{0.829413in}}%
\pgfpathlineto{\pgfqpoint{2.024293in}{0.829424in}}%
\pgfpathlineto{\pgfqpoint{2.024589in}{0.829437in}}%
\pgfpathlineto{\pgfqpoint{2.024885in}{0.829449in}}%
\pgfpathlineto{\pgfqpoint{2.025181in}{0.829449in}}%
\pgfpathlineto{\pgfqpoint{2.025477in}{0.829427in}}%
\pgfpathlineto{\pgfqpoint{2.025773in}{0.829406in}}%
\pgfpathlineto{\pgfqpoint{2.026069in}{0.829385in}}%
\pgfpathlineto{\pgfqpoint{2.026365in}{0.829363in}}%
\pgfpathlineto{\pgfqpoint{2.026661in}{0.829355in}}%
\pgfpathlineto{\pgfqpoint{2.026957in}{0.829357in}}%
\pgfpathlineto{\pgfqpoint{2.027253in}{0.829359in}}%
\pgfpathlineto{\pgfqpoint{2.027549in}{0.829360in}}%
\pgfpathlineto{\pgfqpoint{2.027845in}{0.829362in}}%
\pgfpathlineto{\pgfqpoint{2.028141in}{0.829364in}}%
\pgfpathlineto{\pgfqpoint{2.028437in}{0.829365in}}%
\pgfpathlineto{\pgfqpoint{2.028733in}{0.829367in}}%
\pgfpathlineto{\pgfqpoint{2.029029in}{0.829369in}}%
\pgfpathlineto{\pgfqpoint{2.029325in}{0.829370in}}%
\pgfpathlineto{\pgfqpoint{2.029621in}{0.829372in}}%
\pgfpathlineto{\pgfqpoint{2.029917in}{0.829374in}}%
\pgfpathlineto{\pgfqpoint{2.030213in}{0.829375in}}%
\pgfpathlineto{\pgfqpoint{2.030509in}{0.829377in}}%
\pgfpathlineto{\pgfqpoint{2.030805in}{0.829374in}}%
\pgfpathlineto{\pgfqpoint{2.031101in}{0.829363in}}%
\pgfpathlineto{\pgfqpoint{2.031397in}{0.829360in}}%
\pgfpathlineto{\pgfqpoint{2.031693in}{0.829385in}}%
\pgfpathlineto{\pgfqpoint{2.031989in}{0.829414in}}%
\pgfpathlineto{\pgfqpoint{2.032285in}{0.829443in}}%
\pgfpathlineto{\pgfqpoint{2.032581in}{0.829472in}}%
\pgfpathlineto{\pgfqpoint{2.032877in}{0.829477in}}%
\pgfpathlineto{\pgfqpoint{2.033173in}{0.829487in}}%
\pgfpathlineto{\pgfqpoint{2.033469in}{0.829478in}}%
\pgfpathlineto{\pgfqpoint{2.033765in}{0.829469in}}%
\pgfpathlineto{\pgfqpoint{2.034061in}{0.829460in}}%
\pgfpathlineto{\pgfqpoint{2.034357in}{0.829450in}}%
\pgfpathlineto{\pgfqpoint{2.034653in}{0.829441in}}%
\pgfpathlineto{\pgfqpoint{2.034949in}{0.829432in}}%
\pgfpathlineto{\pgfqpoint{2.035245in}{0.829423in}}%
\pgfpathlineto{\pgfqpoint{2.035541in}{0.829414in}}%
\pgfpathlineto{\pgfqpoint{2.035837in}{0.829405in}}%
\pgfpathlineto{\pgfqpoint{2.036133in}{0.829396in}}%
\pgfpathlineto{\pgfqpoint{2.036429in}{0.829386in}}%
\pgfpathlineto{\pgfqpoint{2.036725in}{0.829377in}}%
\pgfpathlineto{\pgfqpoint{2.037021in}{0.829368in}}%
\pgfpathlineto{\pgfqpoint{2.037317in}{0.829359in}}%
\pgfpathlineto{\pgfqpoint{2.037613in}{0.829350in}}%
\pgfpathlineto{\pgfqpoint{2.037909in}{0.829341in}}%
\pgfpathlineto{\pgfqpoint{2.038205in}{0.829332in}}%
\pgfpathlineto{\pgfqpoint{2.038501in}{0.829371in}}%
\pgfpathlineto{\pgfqpoint{2.038797in}{0.829454in}}%
\pgfpathlineto{\pgfqpoint{2.039093in}{0.829452in}}%
\pgfpathlineto{\pgfqpoint{2.039389in}{0.829450in}}%
\pgfpathlineto{\pgfqpoint{2.039685in}{0.829448in}}%
\pgfpathlineto{\pgfqpoint{2.039981in}{0.829446in}}%
\pgfpathlineto{\pgfqpoint{2.040277in}{0.829445in}}%
\pgfpathlineto{\pgfqpoint{2.040573in}{0.829443in}}%
\pgfpathlineto{\pgfqpoint{2.040869in}{0.829441in}}%
\pgfpathlineto{\pgfqpoint{2.041165in}{0.829439in}}%
\pgfpathlineto{\pgfqpoint{2.041461in}{0.829437in}}%
\pgfpathlineto{\pgfqpoint{2.041757in}{0.829435in}}%
\pgfpathlineto{\pgfqpoint{2.042053in}{0.829434in}}%
\pgfpathlineto{\pgfqpoint{2.042349in}{0.829432in}}%
\pgfpathlineto{\pgfqpoint{2.042645in}{0.829430in}}%
\pgfpathlineto{\pgfqpoint{2.042941in}{0.829428in}}%
\pgfpathlineto{\pgfqpoint{2.043237in}{0.829426in}}%
\pgfpathlineto{\pgfqpoint{2.043533in}{0.829425in}}%
\pgfpathlineto{\pgfqpoint{2.043829in}{0.829423in}}%
\pgfpathlineto{\pgfqpoint{2.044125in}{0.829421in}}%
\pgfpathlineto{\pgfqpoint{2.044421in}{0.829419in}}%
\pgfpathlineto{\pgfqpoint{2.044717in}{0.829417in}}%
\pgfpathlineto{\pgfqpoint{2.045013in}{0.829415in}}%
\pgfpathlineto{\pgfqpoint{2.045309in}{0.829414in}}%
\pgfpathlineto{\pgfqpoint{2.045605in}{0.829412in}}%
\pgfpathlineto{\pgfqpoint{2.045901in}{0.829410in}}%
\pgfpathlineto{\pgfqpoint{2.046197in}{0.829408in}}%
\pgfpathlineto{\pgfqpoint{2.046493in}{0.829406in}}%
\pgfpathlineto{\pgfqpoint{2.046789in}{0.829405in}}%
\pgfpathlineto{\pgfqpoint{2.047085in}{0.829403in}}%
\pgfpathlineto{\pgfqpoint{2.047381in}{0.829401in}}%
\pgfpathlineto{\pgfqpoint{2.047677in}{0.829399in}}%
\pgfpathlineto{\pgfqpoint{2.047973in}{0.829397in}}%
\pgfpathlineto{\pgfqpoint{2.048269in}{0.829395in}}%
\pgfpathlineto{\pgfqpoint{2.048565in}{0.829394in}}%
\pgfpathlineto{\pgfqpoint{2.048861in}{0.829392in}}%
\pgfpathlineto{\pgfqpoint{2.049157in}{0.829390in}}%
\pgfpathlineto{\pgfqpoint{2.049453in}{0.829388in}}%
\pgfpathlineto{\pgfqpoint{2.049749in}{0.829386in}}%
\pgfpathlineto{\pgfqpoint{2.050045in}{0.829385in}}%
\pgfpathlineto{\pgfqpoint{2.050341in}{0.829383in}}%
\pgfpathlineto{\pgfqpoint{2.050637in}{0.829381in}}%
\pgfpathlineto{\pgfqpoint{2.050933in}{0.829379in}}%
\pgfpathlineto{\pgfqpoint{2.051229in}{0.829377in}}%
\pgfpathlineto{\pgfqpoint{2.051525in}{0.829375in}}%
\pgfpathlineto{\pgfqpoint{2.051821in}{0.829374in}}%
\pgfpathlineto{\pgfqpoint{2.052117in}{0.829372in}}%
\pgfpathlineto{\pgfqpoint{2.052413in}{0.829370in}}%
\pgfpathlineto{\pgfqpoint{2.052709in}{0.829368in}}%
\pgfpathlineto{\pgfqpoint{2.053005in}{0.829366in}}%
\pgfpathlineto{\pgfqpoint{2.053301in}{0.829365in}}%
\pgfpathlineto{\pgfqpoint{2.053597in}{0.829363in}}%
\pgfpathlineto{\pgfqpoint{2.053893in}{0.829361in}}%
\pgfpathlineto{\pgfqpoint{2.054189in}{0.829359in}}%
\pgfpathlineto{\pgfqpoint{2.054485in}{0.829357in}}%
\pgfpathlineto{\pgfqpoint{2.054781in}{0.829355in}}%
\pgfpathlineto{\pgfqpoint{2.055077in}{0.829354in}}%
\pgfpathlineto{\pgfqpoint{2.055373in}{0.829352in}}%
\pgfpathlineto{\pgfqpoint{2.055669in}{0.829350in}}%
\pgfpathlineto{\pgfqpoint{2.055965in}{0.829348in}}%
\pgfpathlineto{\pgfqpoint{2.056261in}{0.829346in}}%
\pgfpathlineto{\pgfqpoint{2.056557in}{0.829345in}}%
\pgfpathlineto{\pgfqpoint{2.056853in}{0.829343in}}%
\pgfpathlineto{\pgfqpoint{2.057149in}{0.829341in}}%
\pgfpathlineto{\pgfqpoint{2.057445in}{0.829339in}}%
\pgfpathlineto{\pgfqpoint{2.057741in}{0.829337in}}%
\pgfpathlineto{\pgfqpoint{2.058037in}{0.829335in}}%
\pgfpathlineto{\pgfqpoint{2.058333in}{0.829334in}}%
\pgfpathlineto{\pgfqpoint{2.058629in}{0.829332in}}%
\pgfpathlineto{\pgfqpoint{2.058925in}{0.829330in}}%
\pgfpathlineto{\pgfqpoint{2.059221in}{0.829328in}}%
\pgfpathlineto{\pgfqpoint{2.059517in}{0.829326in}}%
\pgfpathlineto{\pgfqpoint{2.059813in}{0.829325in}}%
\pgfpathlineto{\pgfqpoint{2.060109in}{0.829323in}}%
\pgfpathlineto{\pgfqpoint{2.060405in}{0.829321in}}%
\pgfpathlineto{\pgfqpoint{2.060701in}{0.829320in}}%
\pgfpathlineto{\pgfqpoint{2.060997in}{0.829318in}}%
\pgfpathlineto{\pgfqpoint{2.061293in}{0.829317in}}%
\pgfpathlineto{\pgfqpoint{2.061589in}{0.829315in}}%
\pgfpathlineto{\pgfqpoint{2.061885in}{0.829314in}}%
\pgfpathlineto{\pgfqpoint{2.062181in}{0.829313in}}%
\pgfpathlineto{\pgfqpoint{2.062477in}{0.829311in}}%
\pgfpathlineto{\pgfqpoint{2.062773in}{0.829310in}}%
\pgfpathlineto{\pgfqpoint{2.063069in}{0.829308in}}%
\pgfpathlineto{\pgfqpoint{2.063365in}{0.829307in}}%
\pgfpathlineto{\pgfqpoint{2.063661in}{0.829306in}}%
\pgfpathlineto{\pgfqpoint{2.063958in}{0.829304in}}%
\pgfpathlineto{\pgfqpoint{2.064254in}{0.829303in}}%
\pgfpathlineto{\pgfqpoint{2.064550in}{0.829302in}}%
\pgfpathlineto{\pgfqpoint{2.064846in}{0.829300in}}%
\pgfpathlineto{\pgfqpoint{2.065142in}{0.829299in}}%
\pgfpathlineto{\pgfqpoint{2.065438in}{0.829297in}}%
\pgfpathlineto{\pgfqpoint{2.065734in}{0.829296in}}%
\pgfpathlineto{\pgfqpoint{2.066030in}{0.829295in}}%
\pgfpathlineto{\pgfqpoint{2.066326in}{0.829293in}}%
\pgfpathlineto{\pgfqpoint{2.066622in}{0.829292in}}%
\pgfpathlineto{\pgfqpoint{2.066918in}{0.829289in}}%
\pgfpathlineto{\pgfqpoint{2.067214in}{0.829285in}}%
\pgfpathlineto{\pgfqpoint{2.067510in}{0.829281in}}%
\pgfpathlineto{\pgfqpoint{2.067806in}{0.829278in}}%
\pgfpathlineto{\pgfqpoint{2.068102in}{0.829274in}}%
\pgfpathlineto{\pgfqpoint{2.068398in}{0.829270in}}%
\pgfpathlineto{\pgfqpoint{2.068694in}{0.829267in}}%
\pgfpathlineto{\pgfqpoint{2.068990in}{0.829263in}}%
\pgfpathlineto{\pgfqpoint{2.069286in}{0.829317in}}%
\pgfpathlineto{\pgfqpoint{2.069582in}{0.829702in}}%
\pgfpathlineto{\pgfqpoint{2.069878in}{0.829679in}}%
\pgfpathlineto{\pgfqpoint{2.070174in}{0.829655in}}%
\pgfpathlineto{\pgfqpoint{2.070470in}{0.829631in}}%
\pgfpathlineto{\pgfqpoint{2.070766in}{0.829607in}}%
\pgfpathlineto{\pgfqpoint{2.071062in}{0.829584in}}%
\pgfpathlineto{\pgfqpoint{2.071358in}{0.829560in}}%
\pgfpathlineto{\pgfqpoint{2.071654in}{0.829536in}}%
\pgfpathlineto{\pgfqpoint{2.071950in}{0.829512in}}%
\pgfpathlineto{\pgfqpoint{2.072246in}{0.829489in}}%
\pgfpathlineto{\pgfqpoint{2.072542in}{0.829465in}}%
\pgfpathlineto{\pgfqpoint{2.072838in}{0.829441in}}%
\pgfpathlineto{\pgfqpoint{2.073134in}{0.829417in}}%
\pgfpathlineto{\pgfqpoint{2.073430in}{0.829394in}}%
\pgfpathlineto{\pgfqpoint{2.073726in}{0.829370in}}%
\pgfpathlineto{\pgfqpoint{2.074022in}{0.829346in}}%
\pgfpathlineto{\pgfqpoint{2.074318in}{0.829323in}}%
\pgfpathlineto{\pgfqpoint{2.074614in}{0.829299in}}%
\pgfpathlineto{\pgfqpoint{2.074910in}{0.829275in}}%
\pgfpathlineto{\pgfqpoint{2.075206in}{0.829251in}}%
\pgfpathlineto{\pgfqpoint{2.075502in}{0.829229in}}%
\pgfpathlineto{\pgfqpoint{2.075798in}{0.829240in}}%
\pgfpathlineto{\pgfqpoint{2.076094in}{0.829264in}}%
\pgfpathlineto{\pgfqpoint{2.076390in}{0.829288in}}%
\pgfpathlineto{\pgfqpoint{2.076686in}{0.829313in}}%
\pgfpathlineto{\pgfqpoint{2.076982in}{0.829337in}}%
\pgfpathlineto{\pgfqpoint{2.077278in}{0.829361in}}%
\pgfpathlineto{\pgfqpoint{2.077574in}{0.829385in}}%
\pgfpathlineto{\pgfqpoint{2.077870in}{0.829409in}}%
\pgfpathlineto{\pgfqpoint{2.078166in}{0.829433in}}%
\pgfpathlineto{\pgfqpoint{2.078462in}{0.829458in}}%
\pgfpathlineto{\pgfqpoint{2.078758in}{0.829482in}}%
\pgfpathlineto{\pgfqpoint{2.079054in}{0.829506in}}%
\pgfpathlineto{\pgfqpoint{2.079350in}{0.829530in}}%
\pgfpathlineto{\pgfqpoint{2.079646in}{0.829554in}}%
\pgfpathlineto{\pgfqpoint{2.079942in}{0.829578in}}%
\pgfpathlineto{\pgfqpoint{2.080238in}{0.829602in}}%
\pgfpathlineto{\pgfqpoint{2.080534in}{0.829627in}}%
\pgfpathlineto{\pgfqpoint{2.080830in}{0.829651in}}%
\pgfpathlineto{\pgfqpoint{2.081126in}{0.829675in}}%
\pgfpathlineto{\pgfqpoint{2.081422in}{0.829699in}}%
\pgfpathlineto{\pgfqpoint{2.081718in}{0.829714in}}%
\pgfpathlineto{\pgfqpoint{2.082014in}{0.829708in}}%
\pgfpathlineto{\pgfqpoint{2.082310in}{0.829702in}}%
\pgfpathlineto{\pgfqpoint{2.082606in}{0.829697in}}%
\pgfpathlineto{\pgfqpoint{2.082902in}{0.829693in}}%
\pgfpathlineto{\pgfqpoint{2.083198in}{0.829690in}}%
\pgfpathlineto{\pgfqpoint{2.083494in}{0.829687in}}%
\pgfpathlineto{\pgfqpoint{2.083790in}{0.829683in}}%
\pgfpathlineto{\pgfqpoint{2.084086in}{0.829680in}}%
\pgfpathlineto{\pgfqpoint{2.084382in}{0.829677in}}%
\pgfpathlineto{\pgfqpoint{2.084678in}{0.829673in}}%
\pgfpathlineto{\pgfqpoint{2.084974in}{0.829670in}}%
\pgfpathlineto{\pgfqpoint{2.085270in}{0.829667in}}%
\pgfpathlineto{\pgfqpoint{2.085566in}{0.829663in}}%
\pgfpathlineto{\pgfqpoint{2.085862in}{0.829660in}}%
\pgfpathlineto{\pgfqpoint{2.086158in}{0.829657in}}%
\pgfpathlineto{\pgfqpoint{2.086454in}{0.829653in}}%
\pgfpathlineto{\pgfqpoint{2.086750in}{0.829650in}}%
\pgfpathlineto{\pgfqpoint{2.087046in}{0.829647in}}%
\pgfpathlineto{\pgfqpoint{2.087342in}{0.829643in}}%
\pgfpathlineto{\pgfqpoint{2.087638in}{0.829640in}}%
\pgfpathlineto{\pgfqpoint{2.087934in}{0.829636in}}%
\pgfpathlineto{\pgfqpoint{2.088230in}{0.829997in}}%
\pgfpathlineto{\pgfqpoint{2.088526in}{0.830701in}}%
\pgfpathlineto{\pgfqpoint{2.088822in}{0.830025in}}%
\pgfpathlineto{\pgfqpoint{2.089118in}{0.829839in}}%
\pgfpathlineto{\pgfqpoint{2.089414in}{0.830348in}}%
\pgfpathlineto{\pgfqpoint{2.089710in}{0.830861in}}%
\pgfpathlineto{\pgfqpoint{2.090006in}{0.831201in}}%
\pgfpathlineto{\pgfqpoint{2.090302in}{0.831212in}}%
\pgfpathlineto{\pgfqpoint{2.090598in}{0.831215in}}%
\pgfpathlineto{\pgfqpoint{2.090894in}{0.831219in}}%
\pgfpathlineto{\pgfqpoint{2.091190in}{0.831222in}}%
\pgfpathlineto{\pgfqpoint{2.091486in}{0.831226in}}%
\pgfpathlineto{\pgfqpoint{2.091782in}{0.831229in}}%
\pgfpathlineto{\pgfqpoint{2.092078in}{0.831233in}}%
\pgfpathlineto{\pgfqpoint{2.092374in}{0.831236in}}%
\pgfpathlineto{\pgfqpoint{2.092670in}{0.831240in}}%
\pgfpathlineto{\pgfqpoint{2.092966in}{0.831243in}}%
\pgfpathlineto{\pgfqpoint{2.093262in}{0.831246in}}%
\pgfpathlineto{\pgfqpoint{2.093558in}{0.831250in}}%
\pgfpathlineto{\pgfqpoint{2.093854in}{0.831253in}}%
\pgfpathlineto{\pgfqpoint{2.094150in}{0.831257in}}%
\pgfpathlineto{\pgfqpoint{2.094446in}{0.831260in}}%
\pgfpathlineto{\pgfqpoint{2.094742in}{0.831264in}}%
\pgfpathlineto{\pgfqpoint{2.095038in}{0.831267in}}%
\pgfpathlineto{\pgfqpoint{2.095334in}{0.831271in}}%
\pgfpathlineto{\pgfqpoint{2.095630in}{0.831274in}}%
\pgfpathlineto{\pgfqpoint{2.095926in}{0.831277in}}%
\pgfpathlineto{\pgfqpoint{2.096222in}{0.831281in}}%
\pgfpathlineto{\pgfqpoint{2.096518in}{0.831284in}}%
\pgfpathlineto{\pgfqpoint{2.096814in}{0.831288in}}%
\pgfpathlineto{\pgfqpoint{2.097110in}{0.831291in}}%
\pgfpathlineto{\pgfqpoint{2.097406in}{0.831295in}}%
\pgfpathlineto{\pgfqpoint{2.097702in}{0.831298in}}%
\pgfpathlineto{\pgfqpoint{2.097998in}{0.831302in}}%
\pgfpathlineto{\pgfqpoint{2.098294in}{0.831305in}}%
\pgfpathlineto{\pgfqpoint{2.098590in}{0.831309in}}%
\pgfpathlineto{\pgfqpoint{2.098886in}{0.831312in}}%
\pgfpathlineto{\pgfqpoint{2.099182in}{0.831315in}}%
\pgfpathlineto{\pgfqpoint{2.099478in}{0.831319in}}%
\pgfpathlineto{\pgfqpoint{2.099774in}{0.831322in}}%
\pgfpathlineto{\pgfqpoint{2.100070in}{0.831326in}}%
\pgfpathlineto{\pgfqpoint{2.100366in}{0.831329in}}%
\pgfpathlineto{\pgfqpoint{2.100662in}{0.831333in}}%
\pgfpathlineto{\pgfqpoint{2.100958in}{0.831336in}}%
\pgfpathlineto{\pgfqpoint{2.101254in}{0.831340in}}%
\pgfpathlineto{\pgfqpoint{2.101550in}{0.831343in}}%
\pgfpathlineto{\pgfqpoint{2.101846in}{0.831346in}}%
\pgfpathlineto{\pgfqpoint{2.102142in}{0.831350in}}%
\pgfpathlineto{\pgfqpoint{2.102438in}{0.831353in}}%
\pgfpathlineto{\pgfqpoint{2.102734in}{0.831357in}}%
\pgfpathlineto{\pgfqpoint{2.103030in}{0.831360in}}%
\pgfpathlineto{\pgfqpoint{2.103326in}{0.831364in}}%
\pgfpathlineto{\pgfqpoint{2.103622in}{0.831367in}}%
\pgfpathlineto{\pgfqpoint{2.103918in}{0.831371in}}%
\pgfpathlineto{\pgfqpoint{2.104214in}{0.831374in}}%
\pgfpathlineto{\pgfqpoint{2.104510in}{0.831377in}}%
\pgfpathlineto{\pgfqpoint{2.104806in}{0.831381in}}%
\pgfpathlineto{\pgfqpoint{2.105102in}{0.831384in}}%
\pgfpathlineto{\pgfqpoint{2.105398in}{0.831388in}}%
\pgfpathlineto{\pgfqpoint{2.105694in}{0.831391in}}%
\pgfpathlineto{\pgfqpoint{2.105990in}{0.831395in}}%
\pgfpathlineto{\pgfqpoint{2.106286in}{0.831398in}}%
\pgfpathlineto{\pgfqpoint{2.106582in}{0.831402in}}%
\pgfpathlineto{\pgfqpoint{2.106878in}{0.831405in}}%
\pgfpathlineto{\pgfqpoint{2.107174in}{0.831408in}}%
\pgfpathlineto{\pgfqpoint{2.107470in}{0.831412in}}%
\pgfpathlineto{\pgfqpoint{2.107766in}{0.831415in}}%
\pgfpathlineto{\pgfqpoint{2.108062in}{0.831419in}}%
\pgfpathlineto{\pgfqpoint{2.108358in}{0.831422in}}%
\pgfpathlineto{\pgfqpoint{2.108654in}{0.831426in}}%
\pgfpathlineto{\pgfqpoint{2.108950in}{0.831429in}}%
\pgfpathlineto{\pgfqpoint{2.109246in}{0.831433in}}%
\pgfpathlineto{\pgfqpoint{2.109542in}{0.831436in}}%
\pgfpathlineto{\pgfqpoint{2.109838in}{0.831442in}}%
\pgfpathlineto{\pgfqpoint{2.110134in}{0.830714in}}%
\pgfpathlineto{\pgfqpoint{2.110430in}{0.831084in}}%
\pgfpathlineto{\pgfqpoint{2.110726in}{0.832008in}}%
\pgfpathlineto{\pgfqpoint{2.111022in}{0.831304in}}%
\pgfpathlineto{\pgfqpoint{2.111318in}{0.830022in}}%
\pgfpathlineto{\pgfqpoint{2.111614in}{0.830438in}}%
\pgfpathlineto{\pgfqpoint{2.111910in}{0.831928in}}%
\pgfpathlineto{\pgfqpoint{2.112206in}{0.831990in}}%
\pgfpathlineto{\pgfqpoint{2.112502in}{0.831983in}}%
\pgfpathlineto{\pgfqpoint{2.112798in}{0.831977in}}%
\pgfpathlineto{\pgfqpoint{2.113094in}{0.831970in}}%
\pgfpathlineto{\pgfqpoint{2.113390in}{0.831963in}}%
\pgfpathlineto{\pgfqpoint{2.113686in}{0.831957in}}%
\pgfpathlineto{\pgfqpoint{2.113982in}{0.831950in}}%
\pgfpathlineto{\pgfqpoint{2.114278in}{0.831943in}}%
\pgfpathlineto{\pgfqpoint{2.114574in}{0.831937in}}%
\pgfpathlineto{\pgfqpoint{2.114870in}{0.831930in}}%
\pgfpathlineto{\pgfqpoint{2.115166in}{0.831923in}}%
\pgfpathlineto{\pgfqpoint{2.115462in}{0.831917in}}%
\pgfpathlineto{\pgfqpoint{2.115758in}{0.831910in}}%
\pgfpathlineto{\pgfqpoint{2.116054in}{0.831903in}}%
\pgfpathlineto{\pgfqpoint{2.116350in}{0.831897in}}%
\pgfpathlineto{\pgfqpoint{2.116646in}{0.831890in}}%
\pgfpathlineto{\pgfqpoint{2.116942in}{0.831930in}}%
\pgfpathlineto{\pgfqpoint{2.117238in}{0.831939in}}%
\pgfpathlineto{\pgfqpoint{2.117534in}{0.831935in}}%
\pgfpathlineto{\pgfqpoint{2.117830in}{0.831931in}}%
\pgfpathlineto{\pgfqpoint{2.118126in}{0.831927in}}%
\pgfpathlineto{\pgfqpoint{2.118422in}{0.831923in}}%
\pgfpathlineto{\pgfqpoint{2.118718in}{0.831919in}}%
\pgfpathlineto{\pgfqpoint{2.119014in}{0.831915in}}%
\pgfpathlineto{\pgfqpoint{2.119310in}{0.831911in}}%
\pgfpathlineto{\pgfqpoint{2.119606in}{0.831907in}}%
\pgfpathlineto{\pgfqpoint{2.119902in}{0.831903in}}%
\pgfpathlineto{\pgfqpoint{2.120198in}{0.831899in}}%
\pgfpathlineto{\pgfqpoint{2.120494in}{0.831895in}}%
\pgfpathlineto{\pgfqpoint{2.120790in}{0.831891in}}%
\pgfpathlineto{\pgfqpoint{2.121086in}{0.831887in}}%
\pgfpathlineto{\pgfqpoint{2.121382in}{0.831883in}}%
\pgfpathlineto{\pgfqpoint{2.121678in}{0.831879in}}%
\pgfpathlineto{\pgfqpoint{2.121974in}{0.831876in}}%
\pgfpathlineto{\pgfqpoint{2.122270in}{0.831872in}}%
\pgfpathlineto{\pgfqpoint{2.122566in}{0.831868in}}%
\pgfpathlineto{\pgfqpoint{2.122862in}{0.831864in}}%
\pgfpathlineto{\pgfqpoint{2.123158in}{0.831901in}}%
\pgfpathlineto{\pgfqpoint{2.123454in}{0.831984in}}%
\pgfpathlineto{\pgfqpoint{2.123750in}{0.832012in}}%
\pgfpathlineto{\pgfqpoint{2.124046in}{0.832020in}}%
\pgfpathlineto{\pgfqpoint{2.124342in}{0.831967in}}%
\pgfpathlineto{\pgfqpoint{2.124638in}{0.831908in}}%
\pgfpathlineto{\pgfqpoint{2.124934in}{0.831850in}}%
\pgfpathlineto{\pgfqpoint{2.125230in}{0.831792in}}%
\pgfpathlineto{\pgfqpoint{2.125526in}{0.831736in}}%
\pgfpathlineto{\pgfqpoint{2.125822in}{0.831742in}}%
\pgfpathlineto{\pgfqpoint{2.126118in}{0.831773in}}%
\pgfpathlineto{\pgfqpoint{2.126414in}{0.831805in}}%
\pgfpathlineto{\pgfqpoint{2.126710in}{0.831837in}}%
\pgfpathlineto{\pgfqpoint{2.127006in}{0.831869in}}%
\pgfpathlineto{\pgfqpoint{2.127302in}{0.831901in}}%
\pgfpathlineto{\pgfqpoint{2.127598in}{0.831932in}}%
\pgfpathlineto{\pgfqpoint{2.127894in}{0.831964in}}%
\pgfpathlineto{\pgfqpoint{2.128190in}{0.831996in}}%
\pgfpathlineto{\pgfqpoint{2.128486in}{0.832028in}}%
\pgfpathlineto{\pgfqpoint{2.128782in}{0.832060in}}%
\pgfpathlineto{\pgfqpoint{2.129078in}{0.832091in}}%
\pgfpathlineto{\pgfqpoint{2.129374in}{0.832123in}}%
\pgfpathlineto{\pgfqpoint{2.129670in}{0.832155in}}%
\pgfpathlineto{\pgfqpoint{2.129966in}{0.832187in}}%
\pgfpathlineto{\pgfqpoint{2.130262in}{0.832218in}}%
\pgfpathlineto{\pgfqpoint{2.130558in}{0.832250in}}%
\pgfpathlineto{\pgfqpoint{2.130854in}{0.832282in}}%
\pgfpathlineto{\pgfqpoint{2.131151in}{0.832314in}}%
\pgfpathlineto{\pgfqpoint{2.131447in}{0.832346in}}%
\pgfpathlineto{\pgfqpoint{2.131743in}{0.832383in}}%
\pgfpathlineto{\pgfqpoint{2.132039in}{0.832403in}}%
\pgfpathlineto{\pgfqpoint{2.132335in}{0.832378in}}%
\pgfpathlineto{\pgfqpoint{2.132631in}{0.832351in}}%
\pgfpathlineto{\pgfqpoint{2.132927in}{0.832324in}}%
\pgfpathlineto{\pgfqpoint{2.133223in}{0.832297in}}%
\pgfpathlineto{\pgfqpoint{2.133519in}{0.832269in}}%
\pgfpathlineto{\pgfqpoint{2.133815in}{0.832242in}}%
\pgfpathlineto{\pgfqpoint{2.134111in}{0.832215in}}%
\pgfpathlineto{\pgfqpoint{2.134407in}{0.832187in}}%
\pgfpathlineto{\pgfqpoint{2.134703in}{0.832160in}}%
\pgfpathlineto{\pgfqpoint{2.134999in}{0.832133in}}%
\pgfpathlineto{\pgfqpoint{2.135295in}{0.832105in}}%
\pgfpathlineto{\pgfqpoint{2.135591in}{0.832078in}}%
\pgfpathlineto{\pgfqpoint{2.135887in}{0.832051in}}%
\pgfpathlineto{\pgfqpoint{2.136183in}{0.832024in}}%
\pgfpathlineto{\pgfqpoint{2.136479in}{0.831996in}}%
\pgfpathlineto{\pgfqpoint{2.136775in}{0.831969in}}%
\pgfpathlineto{\pgfqpoint{2.137071in}{0.831942in}}%
\pgfpathlineto{\pgfqpoint{2.137367in}{0.831914in}}%
\pgfpathlineto{\pgfqpoint{2.137663in}{0.831887in}}%
\pgfpathlineto{\pgfqpoint{2.137959in}{0.831860in}}%
\pgfpathlineto{\pgfqpoint{2.138255in}{0.831832in}}%
\pgfpathlineto{\pgfqpoint{2.138551in}{0.831805in}}%
\pgfpathlineto{\pgfqpoint{2.138847in}{0.831778in}}%
\pgfpathlineto{\pgfqpoint{2.139143in}{0.831793in}}%
\pgfpathlineto{\pgfqpoint{2.139439in}{0.831940in}}%
\pgfpathlineto{\pgfqpoint{2.139735in}{0.832097in}}%
\pgfpathlineto{\pgfqpoint{2.140031in}{0.832244in}}%
\pgfpathlineto{\pgfqpoint{2.140327in}{0.832278in}}%
\pgfpathlineto{\pgfqpoint{2.140623in}{0.832281in}}%
\pgfpathlineto{\pgfqpoint{2.140919in}{0.832285in}}%
\pgfpathlineto{\pgfqpoint{2.141215in}{0.832289in}}%
\pgfpathlineto{\pgfqpoint{2.141511in}{0.832293in}}%
\pgfpathlineto{\pgfqpoint{2.141807in}{0.832296in}}%
\pgfpathlineto{\pgfqpoint{2.142103in}{0.832300in}}%
\pgfpathlineto{\pgfqpoint{2.142399in}{0.832304in}}%
\pgfpathlineto{\pgfqpoint{2.142695in}{0.832308in}}%
\pgfpathlineto{\pgfqpoint{2.142991in}{0.832311in}}%
\pgfpathlineto{\pgfqpoint{2.143287in}{0.832315in}}%
\pgfpathlineto{\pgfqpoint{2.143583in}{0.832319in}}%
\pgfpathlineto{\pgfqpoint{2.143879in}{0.832323in}}%
\pgfpathlineto{\pgfqpoint{2.144175in}{0.832326in}}%
\pgfpathlineto{\pgfqpoint{2.144471in}{0.832330in}}%
\pgfpathlineto{\pgfqpoint{2.144767in}{0.832334in}}%
\pgfpathlineto{\pgfqpoint{2.145063in}{0.832338in}}%
\pgfpathlineto{\pgfqpoint{2.145359in}{0.832341in}}%
\pgfpathlineto{\pgfqpoint{2.145655in}{0.832345in}}%
\pgfpathlineto{\pgfqpoint{2.145951in}{0.832349in}}%
\pgfpathlineto{\pgfqpoint{2.146247in}{0.832353in}}%
\pgfpathlineto{\pgfqpoint{2.146543in}{0.832356in}}%
\pgfpathlineto{\pgfqpoint{2.146839in}{0.832360in}}%
\pgfpathlineto{\pgfqpoint{2.147135in}{0.832364in}}%
\pgfpathlineto{\pgfqpoint{2.147431in}{0.832368in}}%
\pgfpathlineto{\pgfqpoint{2.147727in}{0.832371in}}%
\pgfpathlineto{\pgfqpoint{2.148023in}{0.832375in}}%
\pgfpathlineto{\pgfqpoint{2.148319in}{0.832379in}}%
\pgfpathlineto{\pgfqpoint{2.148615in}{0.832383in}}%
\pgfpathlineto{\pgfqpoint{2.148911in}{0.832386in}}%
\pgfpathlineto{\pgfqpoint{2.149207in}{0.832390in}}%
\pgfpathlineto{\pgfqpoint{2.149503in}{0.832394in}}%
\pgfpathlineto{\pgfqpoint{2.149799in}{0.832398in}}%
\pgfpathlineto{\pgfqpoint{2.150095in}{0.832401in}}%
\pgfpathlineto{\pgfqpoint{2.150391in}{0.832405in}}%
\pgfpathlineto{\pgfqpoint{2.150687in}{0.832409in}}%
\pgfpathlineto{\pgfqpoint{2.150983in}{0.832413in}}%
\pgfpathlineto{\pgfqpoint{2.151279in}{0.832416in}}%
\pgfpathlineto{\pgfqpoint{2.151575in}{0.832420in}}%
\pgfpathlineto{\pgfqpoint{2.151871in}{0.832424in}}%
\pgfpathlineto{\pgfqpoint{2.152167in}{0.832428in}}%
\pgfpathlineto{\pgfqpoint{2.152463in}{0.832431in}}%
\pgfpathlineto{\pgfqpoint{2.152759in}{0.832435in}}%
\pgfpathlineto{\pgfqpoint{2.153055in}{0.832439in}}%
\pgfpathlineto{\pgfqpoint{2.153351in}{0.832443in}}%
\pgfpathlineto{\pgfqpoint{2.153647in}{0.832446in}}%
\pgfpathlineto{\pgfqpoint{2.153943in}{0.832450in}}%
\pgfpathlineto{\pgfqpoint{2.154239in}{0.832454in}}%
\pgfpathlineto{\pgfqpoint{2.154535in}{0.832458in}}%
\pgfpathlineto{\pgfqpoint{2.154831in}{0.832461in}}%
\pgfpathlineto{\pgfqpoint{2.155127in}{0.832465in}}%
\pgfpathlineto{\pgfqpoint{2.155423in}{0.832471in}}%
\pgfpathlineto{\pgfqpoint{2.155719in}{0.832479in}}%
\pgfpathlineto{\pgfqpoint{2.156015in}{0.832487in}}%
\pgfpathlineto{\pgfqpoint{2.156311in}{0.832495in}}%
\pgfpathlineto{\pgfqpoint{2.156607in}{0.832503in}}%
\pgfpathlineto{\pgfqpoint{2.156903in}{0.832511in}}%
\pgfpathlineto{\pgfqpoint{2.157199in}{0.832519in}}%
\pgfpathlineto{\pgfqpoint{2.157495in}{0.832527in}}%
\pgfpathlineto{\pgfqpoint{2.157791in}{0.832535in}}%
\pgfpathlineto{\pgfqpoint{2.158087in}{0.832543in}}%
\pgfpathlineto{\pgfqpoint{2.158383in}{0.832551in}}%
\pgfpathlineto{\pgfqpoint{2.158679in}{0.832559in}}%
\pgfpathlineto{\pgfqpoint{2.158975in}{0.832567in}}%
\pgfpathlineto{\pgfqpoint{2.159271in}{0.832595in}}%
\pgfpathlineto{\pgfqpoint{2.159567in}{0.832645in}}%
\pgfpathlineto{\pgfqpoint{2.159863in}{0.832706in}}%
\pgfpathlineto{\pgfqpoint{2.160159in}{0.832789in}}%
\pgfpathlineto{\pgfqpoint{2.160455in}{0.832907in}}%
\pgfpathlineto{\pgfqpoint{2.160751in}{0.832911in}}%
\pgfpathlineto{\pgfqpoint{2.161047in}{0.832907in}}%
\pgfpathlineto{\pgfqpoint{2.161343in}{0.832903in}}%
\pgfpathlineto{\pgfqpoint{2.161639in}{0.832899in}}%
\pgfpathlineto{\pgfqpoint{2.161935in}{0.832895in}}%
\pgfpathlineto{\pgfqpoint{2.162231in}{0.832891in}}%
\pgfpathlineto{\pgfqpoint{2.162527in}{0.832887in}}%
\pgfpathlineto{\pgfqpoint{2.162823in}{0.832883in}}%
\pgfpathlineto{\pgfqpoint{2.163119in}{0.832879in}}%
\pgfpathlineto{\pgfqpoint{2.163415in}{0.832875in}}%
\pgfpathlineto{\pgfqpoint{2.163711in}{0.832871in}}%
\pgfpathlineto{\pgfqpoint{2.164007in}{0.832868in}}%
\pgfpathlineto{\pgfqpoint{2.164303in}{0.832864in}}%
\pgfpathlineto{\pgfqpoint{2.164599in}{0.832860in}}%
\pgfpathlineto{\pgfqpoint{2.164895in}{0.832856in}}%
\pgfpathlineto{\pgfqpoint{2.165191in}{0.832852in}}%
\pgfpathlineto{\pgfqpoint{2.165487in}{0.832848in}}%
\pgfpathlineto{\pgfqpoint{2.165783in}{0.832844in}}%
\pgfpathlineto{\pgfqpoint{2.166079in}{0.832840in}}%
\pgfpathlineto{\pgfqpoint{2.166375in}{0.832836in}}%
\pgfpathlineto{\pgfqpoint{2.166671in}{0.831905in}}%
\pgfpathlineto{\pgfqpoint{2.166967in}{0.831775in}}%
\pgfpathlineto{\pgfqpoint{2.167263in}{0.832459in}}%
\pgfpathlineto{\pgfqpoint{2.167559in}{0.832672in}}%
\pgfpathlineto{\pgfqpoint{2.167855in}{0.832675in}}%
\pgfpathlineto{\pgfqpoint{2.168151in}{0.832677in}}%
\pgfpathlineto{\pgfqpoint{2.168447in}{0.832672in}}%
\pgfpathlineto{\pgfqpoint{2.168743in}{0.832667in}}%
\pgfpathlineto{\pgfqpoint{2.169039in}{0.832662in}}%
\pgfpathlineto{\pgfqpoint{2.169335in}{0.832657in}}%
\pgfpathlineto{\pgfqpoint{2.169631in}{0.832652in}}%
\pgfpathlineto{\pgfqpoint{2.169927in}{0.832647in}}%
\pgfpathlineto{\pgfqpoint{2.170223in}{0.832642in}}%
\pgfpathlineto{\pgfqpoint{2.170519in}{0.832637in}}%
\pgfpathlineto{\pgfqpoint{2.170815in}{0.832633in}}%
\pgfpathlineto{\pgfqpoint{2.171111in}{0.832628in}}%
\pgfpathlineto{\pgfqpoint{2.171407in}{0.832623in}}%
\pgfpathlineto{\pgfqpoint{2.171703in}{0.832618in}}%
\pgfpathlineto{\pgfqpoint{2.171999in}{0.832613in}}%
\pgfpathlineto{\pgfqpoint{2.172295in}{0.832608in}}%
\pgfpathlineto{\pgfqpoint{2.172591in}{0.832603in}}%
\pgfpathlineto{\pgfqpoint{2.172887in}{0.832598in}}%
\pgfpathlineto{\pgfqpoint{2.173183in}{0.832593in}}%
\pgfpathlineto{\pgfqpoint{2.173479in}{0.832588in}}%
\pgfpathlineto{\pgfqpoint{2.173775in}{0.832583in}}%
\pgfpathlineto{\pgfqpoint{2.174071in}{0.832761in}}%
\pgfpathlineto{\pgfqpoint{2.174367in}{0.833080in}}%
\pgfpathlineto{\pgfqpoint{2.174663in}{0.833161in}}%
\pgfpathlineto{\pgfqpoint{2.174959in}{0.833242in}}%
\pgfpathlineto{\pgfqpoint{2.175255in}{0.833323in}}%
\pgfpathlineto{\pgfqpoint{2.175551in}{0.833404in}}%
\pgfpathlineto{\pgfqpoint{2.175847in}{0.833485in}}%
\pgfpathlineto{\pgfqpoint{2.176143in}{0.833543in}}%
\pgfpathlineto{\pgfqpoint{2.176439in}{0.833540in}}%
\pgfpathlineto{\pgfqpoint{2.176735in}{0.833533in}}%
\pgfpathlineto{\pgfqpoint{2.177031in}{0.833527in}}%
\pgfpathlineto{\pgfqpoint{2.177327in}{0.833521in}}%
\pgfpathlineto{\pgfqpoint{2.177623in}{0.833514in}}%
\pgfpathlineto{\pgfqpoint{2.177919in}{0.833508in}}%
\pgfpathlineto{\pgfqpoint{2.178215in}{0.833502in}}%
\pgfpathlineto{\pgfqpoint{2.178511in}{0.833495in}}%
\pgfpathlineto{\pgfqpoint{2.178807in}{0.833489in}}%
\pgfpathlineto{\pgfqpoint{2.179103in}{0.833483in}}%
\pgfpathlineto{\pgfqpoint{2.179399in}{0.833476in}}%
\pgfpathlineto{\pgfqpoint{2.179695in}{0.833470in}}%
\pgfpathlineto{\pgfqpoint{2.179991in}{0.833464in}}%
\pgfpathlineto{\pgfqpoint{2.180287in}{0.833457in}}%
\pgfpathlineto{\pgfqpoint{2.180583in}{0.833407in}}%
\pgfpathlineto{\pgfqpoint{2.180879in}{0.833324in}}%
\pgfpathlineto{\pgfqpoint{2.181175in}{0.833242in}}%
\pgfpathlineto{\pgfqpoint{2.181471in}{0.832073in}}%
\pgfpathlineto{\pgfqpoint{2.181767in}{0.830757in}}%
\pgfpathlineto{\pgfqpoint{2.182063in}{0.830803in}}%
\pgfpathlineto{\pgfqpoint{2.182359in}{0.830848in}}%
\pgfpathlineto{\pgfqpoint{2.182655in}{0.830894in}}%
\pgfpathlineto{\pgfqpoint{2.182951in}{0.830939in}}%
\pgfpathlineto{\pgfqpoint{2.183247in}{0.830985in}}%
\pgfpathlineto{\pgfqpoint{2.183543in}{0.831030in}}%
\pgfpathlineto{\pgfqpoint{2.183839in}{0.831076in}}%
\pgfpathlineto{\pgfqpoint{2.184135in}{0.831121in}}%
\pgfpathlineto{\pgfqpoint{2.184431in}{0.831167in}}%
\pgfpathlineto{\pgfqpoint{2.184727in}{0.831212in}}%
\pgfpathlineto{\pgfqpoint{2.185023in}{0.831258in}}%
\pgfpathlineto{\pgfqpoint{2.185319in}{0.831303in}}%
\pgfpathlineto{\pgfqpoint{2.185615in}{0.831349in}}%
\pgfpathlineto{\pgfqpoint{2.185911in}{0.831394in}}%
\pgfpathlineto{\pgfqpoint{2.186207in}{0.831440in}}%
\pgfpathlineto{\pgfqpoint{2.186503in}{0.831485in}}%
\pgfpathlineto{\pgfqpoint{2.186799in}{0.831531in}}%
\pgfpathlineto{\pgfqpoint{2.187095in}{0.831577in}}%
\pgfpathlineto{\pgfqpoint{2.187391in}{0.831622in}}%
\pgfpathlineto{\pgfqpoint{2.187687in}{0.831668in}}%
\pgfpathlineto{\pgfqpoint{2.187983in}{0.831713in}}%
\pgfpathlineto{\pgfqpoint{2.188279in}{0.831759in}}%
\pgfpathlineto{\pgfqpoint{2.188575in}{0.831804in}}%
\pgfpathlineto{\pgfqpoint{2.188871in}{0.831850in}}%
\pgfpathlineto{\pgfqpoint{2.189167in}{0.831895in}}%
\pgfpathlineto{\pgfqpoint{2.189463in}{0.831941in}}%
\pgfpathlineto{\pgfqpoint{2.189759in}{0.831986in}}%
\pgfpathlineto{\pgfqpoint{2.190055in}{0.832032in}}%
\pgfpathlineto{\pgfqpoint{2.190351in}{0.832077in}}%
\pgfpathlineto{\pgfqpoint{2.190647in}{0.832123in}}%
\pgfpathlineto{\pgfqpoint{2.190943in}{0.832168in}}%
\pgfpathlineto{\pgfqpoint{2.191239in}{0.832214in}}%
\pgfpathlineto{\pgfqpoint{2.191535in}{0.832260in}}%
\pgfpathlineto{\pgfqpoint{2.191831in}{0.832305in}}%
\pgfpathlineto{\pgfqpoint{2.192127in}{0.832351in}}%
\pgfpathlineto{\pgfqpoint{2.192423in}{0.832396in}}%
\pgfpathlineto{\pgfqpoint{2.192719in}{0.832442in}}%
\pgfpathlineto{\pgfqpoint{2.193015in}{0.832487in}}%
\pgfpathlineto{\pgfqpoint{2.193311in}{0.832533in}}%
\pgfpathlineto{\pgfqpoint{2.193607in}{0.832578in}}%
\pgfpathlineto{\pgfqpoint{2.193903in}{0.832624in}}%
\pgfpathlineto{\pgfqpoint{2.194199in}{0.832669in}}%
\pgfpathlineto{\pgfqpoint{2.194495in}{0.832715in}}%
\pgfpathlineto{\pgfqpoint{2.194791in}{0.832760in}}%
\pgfpathlineto{\pgfqpoint{2.195087in}{0.832806in}}%
\pgfpathlineto{\pgfqpoint{2.195383in}{0.832851in}}%
\pgfpathlineto{\pgfqpoint{2.195679in}{0.832897in}}%
\pgfpathlineto{\pgfqpoint{2.195975in}{0.832942in}}%
\pgfpathlineto{\pgfqpoint{2.196271in}{0.832988in}}%
\pgfpathlineto{\pgfqpoint{2.196567in}{0.833034in}}%
\pgfpathlineto{\pgfqpoint{2.196863in}{0.833079in}}%
\pgfpathlineto{\pgfqpoint{2.197159in}{0.833125in}}%
\pgfpathlineto{\pgfqpoint{2.197455in}{0.833170in}}%
\pgfpathlineto{\pgfqpoint{2.197751in}{0.833210in}}%
\pgfpathlineto{\pgfqpoint{2.198047in}{0.833166in}}%
\pgfpathlineto{\pgfqpoint{2.198343in}{0.833096in}}%
\pgfpathlineto{\pgfqpoint{2.198640in}{0.833025in}}%
\pgfpathlineto{\pgfqpoint{2.198936in}{0.832955in}}%
\pgfpathlineto{\pgfqpoint{2.199232in}{0.832885in}}%
\pgfpathlineto{\pgfqpoint{2.199528in}{0.832815in}}%
\pgfpathlineto{\pgfqpoint{2.199824in}{0.832745in}}%
\pgfpathlineto{\pgfqpoint{2.200120in}{0.832675in}}%
\pgfpathlineto{\pgfqpoint{2.200416in}{0.832604in}}%
\pgfpathlineto{\pgfqpoint{2.200712in}{0.832534in}}%
\pgfpathlineto{\pgfqpoint{2.201008in}{0.832464in}}%
\pgfpathlineto{\pgfqpoint{2.201304in}{0.832394in}}%
\pgfpathlineto{\pgfqpoint{2.201600in}{0.832324in}}%
\pgfpathlineto{\pgfqpoint{2.201896in}{0.832254in}}%
\pgfpathlineto{\pgfqpoint{2.202192in}{0.832183in}}%
\pgfpathlineto{\pgfqpoint{2.202488in}{0.832113in}}%
\pgfpathlineto{\pgfqpoint{2.202784in}{0.832043in}}%
\pgfpathlineto{\pgfqpoint{2.203080in}{0.831973in}}%
\pgfpathlineto{\pgfqpoint{2.203376in}{0.831903in}}%
\pgfpathlineto{\pgfqpoint{2.203672in}{0.831833in}}%
\pgfpathlineto{\pgfqpoint{2.203968in}{0.831762in}}%
\pgfpathlineto{\pgfqpoint{2.204264in}{0.831692in}}%
\pgfpathlineto{\pgfqpoint{2.204560in}{0.831622in}}%
\pgfpathlineto{\pgfqpoint{2.204856in}{0.831552in}}%
\pgfpathlineto{\pgfqpoint{2.205152in}{0.831482in}}%
\pgfpathlineto{\pgfqpoint{2.205448in}{0.831412in}}%
\pgfpathlineto{\pgfqpoint{2.205744in}{0.831341in}}%
\pgfpathlineto{\pgfqpoint{2.206040in}{0.831271in}}%
\pgfpathlineto{\pgfqpoint{2.206336in}{0.831201in}}%
\pgfpathlineto{\pgfqpoint{2.206632in}{0.831131in}}%
\pgfpathlineto{\pgfqpoint{2.206928in}{0.831061in}}%
\pgfpathlineto{\pgfqpoint{2.207224in}{0.830991in}}%
\pgfpathlineto{\pgfqpoint{2.207520in}{0.830920in}}%
\pgfpathlineto{\pgfqpoint{2.207816in}{0.830850in}}%
\pgfpathlineto{\pgfqpoint{2.208112in}{0.830780in}}%
\pgfpathlineto{\pgfqpoint{2.208408in}{0.830710in}}%
\pgfpathlineto{\pgfqpoint{2.208704in}{0.830641in}}%
\pgfpathlineto{\pgfqpoint{2.209000in}{0.832824in}}%
\pgfpathlineto{\pgfqpoint{2.209296in}{0.834076in}}%
\pgfpathlineto{\pgfqpoint{2.209592in}{0.834270in}}%
\pgfpathlineto{\pgfqpoint{2.209888in}{0.834568in}}%
\pgfpathlineto{\pgfqpoint{2.210184in}{0.834784in}}%
\pgfpathlineto{\pgfqpoint{2.210480in}{0.834820in}}%
\pgfpathlineto{\pgfqpoint{2.210776in}{0.834835in}}%
\pgfpathlineto{\pgfqpoint{2.211072in}{0.834830in}}%
\pgfpathlineto{\pgfqpoint{2.211368in}{0.834817in}}%
\pgfpathlineto{\pgfqpoint{2.211664in}{0.834804in}}%
\pgfpathlineto{\pgfqpoint{2.211960in}{0.834791in}}%
\pgfpathlineto{\pgfqpoint{2.212256in}{0.834778in}}%
\pgfpathlineto{\pgfqpoint{2.212552in}{0.834765in}}%
\pgfpathlineto{\pgfqpoint{2.212848in}{0.834752in}}%
\pgfpathlineto{\pgfqpoint{2.213144in}{0.834739in}}%
\pgfpathlineto{\pgfqpoint{2.213440in}{0.834726in}}%
\pgfpathlineto{\pgfqpoint{2.213736in}{0.834713in}}%
\pgfpathlineto{\pgfqpoint{2.214032in}{0.834700in}}%
\pgfpathlineto{\pgfqpoint{2.214328in}{0.834687in}}%
\pgfpathlineto{\pgfqpoint{2.214624in}{0.834674in}}%
\pgfpathlineto{\pgfqpoint{2.214920in}{0.834661in}}%
\pgfpathlineto{\pgfqpoint{2.215216in}{0.834648in}}%
\pgfpathlineto{\pgfqpoint{2.215512in}{0.834635in}}%
\pgfpathlineto{\pgfqpoint{2.215808in}{0.834543in}}%
\pgfpathlineto{\pgfqpoint{2.216104in}{0.834361in}}%
\pgfpathlineto{\pgfqpoint{2.216400in}{0.834231in}}%
\pgfpathlineto{\pgfqpoint{2.216696in}{0.834014in}}%
\pgfpathlineto{\pgfqpoint{2.216992in}{0.833990in}}%
\pgfpathlineto{\pgfqpoint{2.217288in}{0.834037in}}%
\pgfpathlineto{\pgfqpoint{2.217584in}{0.834025in}}%
\pgfpathlineto{\pgfqpoint{2.217880in}{0.833986in}}%
\pgfpathlineto{\pgfqpoint{2.218176in}{0.833947in}}%
\pgfpathlineto{\pgfqpoint{2.218472in}{0.833908in}}%
\pgfpathlineto{\pgfqpoint{2.218768in}{0.833847in}}%
\pgfpathlineto{\pgfqpoint{2.219064in}{0.833665in}}%
\pgfpathlineto{\pgfqpoint{2.219360in}{0.833664in}}%
\pgfpathlineto{\pgfqpoint{2.219656in}{0.833663in}}%
\pgfpathlineto{\pgfqpoint{2.219952in}{0.833662in}}%
\pgfpathlineto{\pgfqpoint{2.220248in}{0.833662in}}%
\pgfpathlineto{\pgfqpoint{2.220544in}{0.833661in}}%
\pgfpathlineto{\pgfqpoint{2.220840in}{0.833660in}}%
\pgfpathlineto{\pgfqpoint{2.221136in}{0.833659in}}%
\pgfpathlineto{\pgfqpoint{2.221432in}{0.833659in}}%
\pgfpathlineto{\pgfqpoint{2.221728in}{0.833658in}}%
\pgfpathlineto{\pgfqpoint{2.222024in}{0.833657in}}%
\pgfpathlineto{\pgfqpoint{2.222320in}{0.833656in}}%
\pgfpathlineto{\pgfqpoint{2.222616in}{0.833656in}}%
\pgfpathlineto{\pgfqpoint{2.222912in}{0.833655in}}%
\pgfpathlineto{\pgfqpoint{2.223208in}{0.833654in}}%
\pgfpathlineto{\pgfqpoint{2.223504in}{0.833653in}}%
\pgfpathlineto{\pgfqpoint{2.223800in}{0.833652in}}%
\pgfpathlineto{\pgfqpoint{2.224096in}{0.833655in}}%
\pgfpathlineto{\pgfqpoint{2.224392in}{0.833687in}}%
\pgfpathlineto{\pgfqpoint{2.224688in}{0.833686in}}%
\pgfpathlineto{\pgfqpoint{2.224984in}{0.833685in}}%
\pgfpathlineto{\pgfqpoint{2.225280in}{0.833683in}}%
\pgfpathlineto{\pgfqpoint{2.225576in}{0.833681in}}%
\pgfpathlineto{\pgfqpoint{2.225872in}{0.833679in}}%
\pgfpathlineto{\pgfqpoint{2.226168in}{0.833676in}}%
\pgfpathlineto{\pgfqpoint{2.226464in}{0.833674in}}%
\pgfpathlineto{\pgfqpoint{2.226760in}{0.833672in}}%
\pgfpathlineto{\pgfqpoint{2.227056in}{0.833670in}}%
\pgfpathlineto{\pgfqpoint{2.227352in}{0.833668in}}%
\pgfpathlineto{\pgfqpoint{2.227648in}{0.833666in}}%
\pgfpathlineto{\pgfqpoint{2.227944in}{0.833664in}}%
\pgfpathlineto{\pgfqpoint{2.228240in}{0.833662in}}%
\pgfpathlineto{\pgfqpoint{2.228536in}{0.833660in}}%
\pgfpathlineto{\pgfqpoint{2.228832in}{0.833658in}}%
\pgfpathlineto{\pgfqpoint{2.229128in}{0.833656in}}%
\pgfpathlineto{\pgfqpoint{2.229424in}{0.833654in}}%
\pgfpathlineto{\pgfqpoint{2.229720in}{0.833652in}}%
\pgfpathlineto{\pgfqpoint{2.230016in}{0.833650in}}%
\pgfpathlineto{\pgfqpoint{2.230312in}{0.833917in}}%
\pgfpathlineto{\pgfqpoint{2.230608in}{0.834179in}}%
\pgfpathlineto{\pgfqpoint{2.230904in}{0.834019in}}%
\pgfpathlineto{\pgfqpoint{2.231200in}{0.833860in}}%
\pgfpathlineto{\pgfqpoint{2.231496in}{0.834104in}}%
\pgfpathlineto{\pgfqpoint{2.231792in}{0.834684in}}%
\pgfpathlineto{\pgfqpoint{2.232088in}{0.834939in}}%
\pgfpathlineto{\pgfqpoint{2.232384in}{0.835161in}}%
\pgfpathlineto{\pgfqpoint{2.232680in}{0.835188in}}%
\pgfpathlineto{\pgfqpoint{2.232976in}{0.835188in}}%
\pgfpathlineto{\pgfqpoint{2.233272in}{0.835188in}}%
\pgfpathlineto{\pgfqpoint{2.233568in}{0.835187in}}%
\pgfpathlineto{\pgfqpoint{2.233864in}{0.835187in}}%
\pgfpathlineto{\pgfqpoint{2.234160in}{0.835187in}}%
\pgfpathlineto{\pgfqpoint{2.234456in}{0.835187in}}%
\pgfpathlineto{\pgfqpoint{2.234752in}{0.835187in}}%
\pgfpathlineto{\pgfqpoint{2.235048in}{0.835187in}}%
\pgfpathlineto{\pgfqpoint{2.235344in}{0.835187in}}%
\pgfpathlineto{\pgfqpoint{2.235640in}{0.835186in}}%
\pgfpathlineto{\pgfqpoint{2.235936in}{0.835186in}}%
\pgfpathlineto{\pgfqpoint{2.236232in}{0.835186in}}%
\pgfpathlineto{\pgfqpoint{2.236528in}{0.835186in}}%
\pgfpathlineto{\pgfqpoint{2.236824in}{0.835186in}}%
\pgfpathlineto{\pgfqpoint{2.237120in}{0.835186in}}%
\pgfpathlineto{\pgfqpoint{2.237416in}{0.835293in}}%
\pgfpathlineto{\pgfqpoint{2.237712in}{0.835318in}}%
\pgfpathlineto{\pgfqpoint{2.238008in}{0.835291in}}%
\pgfpathlineto{\pgfqpoint{2.238304in}{0.835263in}}%
\pgfpathlineto{\pgfqpoint{2.238600in}{0.835217in}}%
\pgfpathlineto{\pgfqpoint{2.238896in}{0.835217in}}%
\pgfpathlineto{\pgfqpoint{2.239192in}{0.835236in}}%
\pgfpathlineto{\pgfqpoint{2.239488in}{0.835256in}}%
\pgfpathlineto{\pgfqpoint{2.239784in}{0.835265in}}%
\pgfpathlineto{\pgfqpoint{2.240080in}{0.835232in}}%
\pgfpathlineto{\pgfqpoint{2.240376in}{0.835197in}}%
\pgfpathlineto{\pgfqpoint{2.240672in}{0.835161in}}%
\pgfpathlineto{\pgfqpoint{2.240968in}{0.835125in}}%
\pgfpathlineto{\pgfqpoint{2.241264in}{0.835090in}}%
\pgfpathlineto{\pgfqpoint{2.241560in}{0.835054in}}%
\pgfpathlineto{\pgfqpoint{2.241856in}{0.835018in}}%
\pgfpathlineto{\pgfqpoint{2.242152in}{0.834982in}}%
\pgfpathlineto{\pgfqpoint{2.242448in}{0.834947in}}%
\pgfpathlineto{\pgfqpoint{2.242744in}{0.834911in}}%
\pgfpathlineto{\pgfqpoint{2.243040in}{0.834875in}}%
\pgfpathlineto{\pgfqpoint{2.243336in}{0.834840in}}%
\pgfpathlineto{\pgfqpoint{2.243632in}{0.834804in}}%
\pgfpathlineto{\pgfqpoint{2.243928in}{0.834768in}}%
\pgfpathlineto{\pgfqpoint{2.244224in}{0.834733in}}%
\pgfpathlineto{\pgfqpoint{2.244520in}{0.834697in}}%
\pgfpathlineto{\pgfqpoint{2.244816in}{0.834661in}}%
\pgfpathlineto{\pgfqpoint{2.245112in}{0.834626in}}%
\pgfpathlineto{\pgfqpoint{2.245408in}{0.834590in}}%
\pgfpathlineto{\pgfqpoint{2.245704in}{0.834554in}}%
\pgfpathlineto{\pgfqpoint{2.246000in}{0.834519in}}%
\pgfpathlineto{\pgfqpoint{2.246296in}{0.834483in}}%
\pgfpathlineto{\pgfqpoint{2.246592in}{0.834447in}}%
\pgfpathlineto{\pgfqpoint{2.246888in}{0.834412in}}%
\pgfpathlineto{\pgfqpoint{2.247184in}{0.834376in}}%
\pgfpathlineto{\pgfqpoint{2.247480in}{0.834340in}}%
\pgfpathlineto{\pgfqpoint{2.247776in}{0.834305in}}%
\pgfpathlineto{\pgfqpoint{2.248072in}{0.834269in}}%
\pgfpathlineto{\pgfqpoint{2.248368in}{0.834233in}}%
\pgfpathlineto{\pgfqpoint{2.248664in}{0.834198in}}%
\pgfpathlineto{\pgfqpoint{2.248960in}{0.834162in}}%
\pgfpathlineto{\pgfqpoint{2.249256in}{0.834126in}}%
\pgfpathlineto{\pgfqpoint{2.249552in}{0.834091in}}%
\pgfpathlineto{\pgfqpoint{2.249848in}{0.834055in}}%
\pgfpathlineto{\pgfqpoint{2.250144in}{0.834019in}}%
\pgfpathlineto{\pgfqpoint{2.250440in}{0.833984in}}%
\pgfpathlineto{\pgfqpoint{2.250736in}{0.833948in}}%
\pgfpathlineto{\pgfqpoint{2.251032in}{0.833912in}}%
\pgfpathlineto{\pgfqpoint{2.251328in}{0.833877in}}%
\pgfpathlineto{\pgfqpoint{2.251624in}{0.833841in}}%
\pgfpathlineto{\pgfqpoint{2.251920in}{0.833805in}}%
\pgfpathlineto{\pgfqpoint{2.252216in}{0.833770in}}%
\pgfpathlineto{\pgfqpoint{2.252512in}{0.833734in}}%
\pgfpathlineto{\pgfqpoint{2.252808in}{0.833698in}}%
\pgfpathlineto{\pgfqpoint{2.253104in}{0.833663in}}%
\pgfpathlineto{\pgfqpoint{2.253400in}{0.833627in}}%
\pgfpathlineto{\pgfqpoint{2.253696in}{0.833591in}}%
\pgfpathlineto{\pgfqpoint{2.253992in}{0.833556in}}%
\pgfpathlineto{\pgfqpoint{2.254288in}{0.833520in}}%
\pgfpathlineto{\pgfqpoint{2.254584in}{0.833484in}}%
\pgfpathlineto{\pgfqpoint{2.254880in}{0.833449in}}%
\pgfpathlineto{\pgfqpoint{2.255176in}{0.833415in}}%
\pgfpathlineto{\pgfqpoint{2.255472in}{0.833510in}}%
\pgfpathlineto{\pgfqpoint{2.255768in}{0.833687in}}%
\pgfpathlineto{\pgfqpoint{2.256064in}{0.833864in}}%
\pgfpathlineto{\pgfqpoint{2.256360in}{0.834041in}}%
\pgfpathlineto{\pgfqpoint{2.256656in}{0.834218in}}%
\pgfpathlineto{\pgfqpoint{2.256952in}{0.834395in}}%
\pgfpathlineto{\pgfqpoint{2.257248in}{0.834572in}}%
\pgfpathlineto{\pgfqpoint{2.257544in}{0.834748in}}%
\pgfpathlineto{\pgfqpoint{2.257840in}{0.834925in}}%
\pgfpathlineto{\pgfqpoint{2.258136in}{0.835102in}}%
\pgfpathlineto{\pgfqpoint{2.258432in}{0.835265in}}%
\pgfpathlineto{\pgfqpoint{2.258728in}{0.834989in}}%
\pgfpathlineto{\pgfqpoint{2.259024in}{0.834507in}}%
\pgfpathlineto{\pgfqpoint{2.259320in}{0.834025in}}%
\pgfpathlineto{\pgfqpoint{2.259616in}{0.833549in}}%
\pgfpathlineto{\pgfqpoint{2.259912in}{0.833983in}}%
\pgfpathlineto{\pgfqpoint{2.260208in}{0.835061in}}%
\pgfpathlineto{\pgfqpoint{2.260504in}{0.835257in}}%
\pgfpathlineto{\pgfqpoint{2.260800in}{0.834222in}}%
\pgfpathlineto{\pgfqpoint{2.261096in}{0.833575in}}%
\pgfpathlineto{\pgfqpoint{2.261392in}{0.833697in}}%
\pgfpathlineto{\pgfqpoint{2.261688in}{0.833820in}}%
\pgfpathlineto{\pgfqpoint{2.261984in}{0.833943in}}%
\pgfpathlineto{\pgfqpoint{2.262280in}{0.834066in}}%
\pgfpathlineto{\pgfqpoint{2.262576in}{0.834189in}}%
\pgfpathlineto{\pgfqpoint{2.262872in}{0.834312in}}%
\pgfpathlineto{\pgfqpoint{2.263168in}{0.834434in}}%
\pgfpathlineto{\pgfqpoint{2.263464in}{0.834557in}}%
\pgfpathlineto{\pgfqpoint{2.263760in}{0.834680in}}%
\pgfpathlineto{\pgfqpoint{2.264056in}{0.834803in}}%
\pgfpathlineto{\pgfqpoint{2.264352in}{0.834926in}}%
\pgfpathlineto{\pgfqpoint{2.264648in}{0.835049in}}%
\pgfpathlineto{\pgfqpoint{2.264944in}{0.835172in}}%
\pgfpathlineto{\pgfqpoint{2.265240in}{0.835294in}}%
\pgfpathlineto{\pgfqpoint{2.265536in}{0.835417in}}%
\pgfpathlineto{\pgfqpoint{2.265832in}{0.835370in}}%
\pgfpathlineto{\pgfqpoint{2.266129in}{0.835033in}}%
\pgfpathlineto{\pgfqpoint{2.266425in}{0.835643in}}%
\pgfpathlineto{\pgfqpoint{2.266721in}{0.835765in}}%
\pgfpathlineto{\pgfqpoint{2.267017in}{0.835891in}}%
\pgfpathlineto{\pgfqpoint{2.267313in}{0.836015in}}%
\pgfpathlineto{\pgfqpoint{2.267609in}{0.836244in}}%
\pgfpathlineto{\pgfqpoint{2.267905in}{0.836729in}}%
\pgfpathlineto{\pgfqpoint{2.268201in}{0.836656in}}%
\pgfpathlineto{\pgfqpoint{2.268497in}{0.836583in}}%
\pgfpathlineto{\pgfqpoint{2.268793in}{0.836510in}}%
\pgfpathlineto{\pgfqpoint{2.269089in}{0.836437in}}%
\pgfpathlineto{\pgfqpoint{2.269385in}{0.836364in}}%
\pgfpathlineto{\pgfqpoint{2.269681in}{0.836291in}}%
\pgfpathlineto{\pgfqpoint{2.269977in}{0.836218in}}%
\pgfpathlineto{\pgfqpoint{2.270273in}{0.836145in}}%
\pgfpathlineto{\pgfqpoint{2.270569in}{0.836072in}}%
\pgfpathlineto{\pgfqpoint{2.270865in}{0.835999in}}%
\pgfpathlineto{\pgfqpoint{2.271161in}{0.835926in}}%
\pgfpathlineto{\pgfqpoint{2.271457in}{0.835853in}}%
\pgfpathlineto{\pgfqpoint{2.271753in}{0.835780in}}%
\pgfpathlineto{\pgfqpoint{2.272049in}{0.835707in}}%
\pgfpathlineto{\pgfqpoint{2.272345in}{0.835634in}}%
\pgfpathlineto{\pgfqpoint{2.272641in}{0.835561in}}%
\pgfpathlineto{\pgfqpoint{2.272937in}{0.835488in}}%
\pgfpathlineto{\pgfqpoint{2.273233in}{0.835532in}}%
\pgfpathlineto{\pgfqpoint{2.273529in}{0.835930in}}%
\pgfpathlineto{\pgfqpoint{2.273825in}{0.836364in}}%
\pgfpathlineto{\pgfqpoint{2.274121in}{0.836719in}}%
\pgfpathlineto{\pgfqpoint{2.274417in}{0.837052in}}%
\pgfpathlineto{\pgfqpoint{2.274713in}{0.837385in}}%
\pgfpathlineto{\pgfqpoint{2.275009in}{0.837490in}}%
\pgfpathlineto{\pgfqpoint{2.275305in}{0.837376in}}%
\pgfpathlineto{\pgfqpoint{2.275601in}{0.837261in}}%
\pgfpathlineto{\pgfqpoint{2.275897in}{0.837150in}}%
\pgfpathlineto{\pgfqpoint{2.276193in}{0.837111in}}%
\pgfpathlineto{\pgfqpoint{2.276489in}{0.837097in}}%
\pgfpathlineto{\pgfqpoint{2.276785in}{0.837083in}}%
\pgfpathlineto{\pgfqpoint{2.277081in}{0.837068in}}%
\pgfpathlineto{\pgfqpoint{2.277377in}{0.837054in}}%
\pgfpathlineto{\pgfqpoint{2.277673in}{0.837040in}}%
\pgfpathlineto{\pgfqpoint{2.277969in}{0.837026in}}%
\pgfpathlineto{\pgfqpoint{2.278265in}{0.837012in}}%
\pgfpathlineto{\pgfqpoint{2.278561in}{0.836997in}}%
\pgfpathlineto{\pgfqpoint{2.278857in}{0.836983in}}%
\pgfpathlineto{\pgfqpoint{2.279153in}{0.836969in}}%
\pgfpathlineto{\pgfqpoint{2.279449in}{0.836955in}}%
\pgfpathlineto{\pgfqpoint{2.279745in}{0.836941in}}%
\pgfpathlineto{\pgfqpoint{2.280041in}{0.836926in}}%
\pgfpathlineto{\pgfqpoint{2.280337in}{0.836912in}}%
\pgfpathlineto{\pgfqpoint{2.280633in}{0.836898in}}%
\pgfpathlineto{\pgfqpoint{2.280929in}{0.836709in}}%
\pgfpathlineto{\pgfqpoint{2.281225in}{0.835997in}}%
\pgfpathlineto{\pgfqpoint{2.281521in}{0.836525in}}%
\pgfpathlineto{\pgfqpoint{2.281817in}{0.837328in}}%
\pgfpathlineto{\pgfqpoint{2.282113in}{0.837306in}}%
\pgfpathlineto{\pgfqpoint{2.282409in}{0.837275in}}%
\pgfpathlineto{\pgfqpoint{2.282705in}{0.837244in}}%
\pgfpathlineto{\pgfqpoint{2.283001in}{0.837213in}}%
\pgfpathlineto{\pgfqpoint{2.283297in}{0.837182in}}%
\pgfpathlineto{\pgfqpoint{2.283593in}{0.837151in}}%
\pgfpathlineto{\pgfqpoint{2.283889in}{0.837120in}}%
\pgfpathlineto{\pgfqpoint{2.284185in}{0.837089in}}%
\pgfpathlineto{\pgfqpoint{2.284481in}{0.837058in}}%
\pgfpathlineto{\pgfqpoint{2.284777in}{0.837027in}}%
\pgfpathlineto{\pgfqpoint{2.285073in}{0.836996in}}%
\pgfpathlineto{\pgfqpoint{2.285369in}{0.836965in}}%
\pgfpathlineto{\pgfqpoint{2.285665in}{0.836934in}}%
\pgfpathlineto{\pgfqpoint{2.285961in}{0.836904in}}%
\pgfpathlineto{\pgfqpoint{2.286257in}{0.836873in}}%
\pgfpathlineto{\pgfqpoint{2.286553in}{0.836842in}}%
\pgfpathlineto{\pgfqpoint{2.286849in}{0.836811in}}%
\pgfpathlineto{\pgfqpoint{2.287145in}{0.836780in}}%
\pgfpathlineto{\pgfqpoint{2.287441in}{0.836749in}}%
\pgfpathlineto{\pgfqpoint{2.287737in}{0.836719in}}%
\pgfpathlineto{\pgfqpoint{2.288033in}{0.836689in}}%
\pgfpathlineto{\pgfqpoint{2.288329in}{0.836659in}}%
\pgfpathlineto{\pgfqpoint{2.288625in}{0.836602in}}%
\pgfpathlineto{\pgfqpoint{2.288921in}{0.836508in}}%
\pgfpathlineto{\pgfqpoint{2.289217in}{0.836477in}}%
\pgfpathlineto{\pgfqpoint{2.289513in}{0.836464in}}%
\pgfpathlineto{\pgfqpoint{2.289809in}{0.836457in}}%
\pgfpathlineto{\pgfqpoint{2.290105in}{0.836451in}}%
\pgfpathlineto{\pgfqpoint{2.290401in}{0.836444in}}%
\pgfpathlineto{\pgfqpoint{2.290697in}{0.836437in}}%
\pgfpathlineto{\pgfqpoint{2.290993in}{0.836431in}}%
\pgfpathlineto{\pgfqpoint{2.291289in}{0.836424in}}%
\pgfpathlineto{\pgfqpoint{2.291585in}{0.836418in}}%
\pgfpathlineto{\pgfqpoint{2.291881in}{0.836411in}}%
\pgfpathlineto{\pgfqpoint{2.292177in}{0.836404in}}%
\pgfpathlineto{\pgfqpoint{2.292473in}{0.836398in}}%
\pgfpathlineto{\pgfqpoint{2.292769in}{0.836391in}}%
\pgfpathlineto{\pgfqpoint{2.293065in}{0.836385in}}%
\pgfpathlineto{\pgfqpoint{2.293361in}{0.836378in}}%
\pgfpathlineto{\pgfqpoint{2.293657in}{0.836371in}}%
\pgfpathlineto{\pgfqpoint{2.293953in}{0.836365in}}%
\pgfpathlineto{\pgfqpoint{2.294249in}{0.836358in}}%
\pgfpathlineto{\pgfqpoint{2.294545in}{0.836351in}}%
\pgfpathlineto{\pgfqpoint{2.294841in}{0.836345in}}%
\pgfpathlineto{\pgfqpoint{2.295137in}{0.836338in}}%
\pgfpathlineto{\pgfqpoint{2.295433in}{0.836332in}}%
\pgfpathlineto{\pgfqpoint{2.295729in}{0.836325in}}%
\pgfpathlineto{\pgfqpoint{2.296025in}{0.836318in}}%
\pgfpathlineto{\pgfqpoint{2.296321in}{0.836312in}}%
\pgfpathlineto{\pgfqpoint{2.296617in}{0.836305in}}%
\pgfpathlineto{\pgfqpoint{2.296913in}{0.836299in}}%
\pgfpathlineto{\pgfqpoint{2.297209in}{0.836292in}}%
\pgfpathlineto{\pgfqpoint{2.297505in}{0.836285in}}%
\pgfpathlineto{\pgfqpoint{2.297801in}{0.836279in}}%
\pgfpathlineto{\pgfqpoint{2.298097in}{0.836272in}}%
\pgfpathlineto{\pgfqpoint{2.298393in}{0.836266in}}%
\pgfpathlineto{\pgfqpoint{2.298689in}{0.836259in}}%
\pgfpathlineto{\pgfqpoint{2.298985in}{0.836252in}}%
\pgfpathlineto{\pgfqpoint{2.299281in}{0.836246in}}%
\pgfpathlineto{\pgfqpoint{2.299577in}{0.836239in}}%
\pgfpathlineto{\pgfqpoint{2.299873in}{0.836232in}}%
\pgfpathlineto{\pgfqpoint{2.300169in}{0.836226in}}%
\pgfpathlineto{\pgfqpoint{2.300465in}{0.836219in}}%
\pgfpathlineto{\pgfqpoint{2.300761in}{0.836213in}}%
\pgfpathlineto{\pgfqpoint{2.301057in}{0.836206in}}%
\pgfpathlineto{\pgfqpoint{2.301353in}{0.836199in}}%
\pgfpathlineto{\pgfqpoint{2.301649in}{0.836193in}}%
\pgfpathlineto{\pgfqpoint{2.301945in}{0.836186in}}%
\pgfpathlineto{\pgfqpoint{2.302241in}{0.836180in}}%
\pgfpathlineto{\pgfqpoint{2.302537in}{0.836173in}}%
\pgfpathlineto{\pgfqpoint{2.302833in}{0.836166in}}%
\pgfpathlineto{\pgfqpoint{2.303129in}{0.836160in}}%
\pgfpathlineto{\pgfqpoint{2.303425in}{0.836153in}}%
\pgfpathlineto{\pgfqpoint{2.303721in}{0.836146in}}%
\pgfpathlineto{\pgfqpoint{2.304017in}{0.836140in}}%
\pgfpathlineto{\pgfqpoint{2.304313in}{0.836133in}}%
\pgfpathlineto{\pgfqpoint{2.304609in}{0.836129in}}%
\pgfpathlineto{\pgfqpoint{2.304905in}{0.836138in}}%
\pgfpathlineto{\pgfqpoint{2.305201in}{0.836150in}}%
\pgfpathlineto{\pgfqpoint{2.305497in}{0.836162in}}%
\pgfpathlineto{\pgfqpoint{2.305793in}{0.836174in}}%
\pgfpathlineto{\pgfqpoint{2.306089in}{0.836186in}}%
\pgfpathlineto{\pgfqpoint{2.306385in}{0.836198in}}%
\pgfpathlineto{\pgfqpoint{2.306681in}{0.836210in}}%
\pgfpathlineto{\pgfqpoint{2.306977in}{0.836222in}}%
\pgfpathlineto{\pgfqpoint{2.307273in}{0.836234in}}%
\pgfpathlineto{\pgfqpoint{2.307569in}{0.836246in}}%
\pgfpathlineto{\pgfqpoint{2.307865in}{0.836258in}}%
\pgfpathlineto{\pgfqpoint{2.308161in}{0.836270in}}%
\pgfpathlineto{\pgfqpoint{2.308457in}{0.836282in}}%
\pgfpathlineto{\pgfqpoint{2.308753in}{0.836290in}}%
\pgfpathlineto{\pgfqpoint{2.309049in}{0.836285in}}%
\pgfpathlineto{\pgfqpoint{2.309345in}{0.836278in}}%
\pgfpathlineto{\pgfqpoint{2.309641in}{0.836267in}}%
\pgfpathlineto{\pgfqpoint{2.309937in}{0.836318in}}%
\pgfpathlineto{\pgfqpoint{2.310233in}{0.836166in}}%
\pgfpathlineto{\pgfqpoint{2.310529in}{0.836202in}}%
\pgfpathlineto{\pgfqpoint{2.310825in}{0.836278in}}%
\pgfpathlineto{\pgfqpoint{2.311121in}{0.836212in}}%
\pgfpathlineto{\pgfqpoint{2.311417in}{0.835972in}}%
\pgfpathlineto{\pgfqpoint{2.311713in}{0.835907in}}%
\pgfpathlineto{\pgfqpoint{2.312009in}{0.835889in}}%
\pgfpathlineto{\pgfqpoint{2.312305in}{0.835871in}}%
\pgfpathlineto{\pgfqpoint{2.312601in}{0.835854in}}%
\pgfpathlineto{\pgfqpoint{2.312897in}{0.835836in}}%
\pgfpathlineto{\pgfqpoint{2.313193in}{0.835818in}}%
\pgfpathlineto{\pgfqpoint{2.313489in}{0.835801in}}%
\pgfpathlineto{\pgfqpoint{2.313785in}{0.835783in}}%
\pgfpathlineto{\pgfqpoint{2.314081in}{0.835765in}}%
\pgfpathlineto{\pgfqpoint{2.314377in}{0.835748in}}%
\pgfpathlineto{\pgfqpoint{2.314673in}{0.835730in}}%
\pgfpathlineto{\pgfqpoint{2.314969in}{0.835712in}}%
\pgfpathlineto{\pgfqpoint{2.315265in}{0.835693in}}%
\pgfpathlineto{\pgfqpoint{2.315561in}{0.835712in}}%
\pgfpathlineto{\pgfqpoint{2.315857in}{0.835538in}}%
\pgfpathlineto{\pgfqpoint{2.316153in}{0.835345in}}%
\pgfpathlineto{\pgfqpoint{2.316449in}{0.835272in}}%
\pgfpathlineto{\pgfqpoint{2.316745in}{0.835260in}}%
\pgfpathlineto{\pgfqpoint{2.317041in}{0.835213in}}%
\pgfpathlineto{\pgfqpoint{2.317337in}{0.835167in}}%
\pgfpathlineto{\pgfqpoint{2.317633in}{0.835073in}}%
\pgfpathlineto{\pgfqpoint{2.317929in}{0.832240in}}%
\pgfpathlineto{\pgfqpoint{2.318225in}{0.831707in}}%
\pgfpathlineto{\pgfqpoint{2.318521in}{0.831687in}}%
\pgfpathlineto{\pgfqpoint{2.318817in}{0.831667in}}%
\pgfpathlineto{\pgfqpoint{2.319113in}{0.831648in}}%
\pgfpathlineto{\pgfqpoint{2.319409in}{0.831628in}}%
\pgfpathlineto{\pgfqpoint{2.319705in}{0.831609in}}%
\pgfpathlineto{\pgfqpoint{2.320001in}{0.831589in}}%
\pgfpathlineto{\pgfqpoint{2.320297in}{0.831570in}}%
\pgfpathlineto{\pgfqpoint{2.320593in}{0.831550in}}%
\pgfpathlineto{\pgfqpoint{2.320889in}{0.831531in}}%
\pgfpathlineto{\pgfqpoint{2.321185in}{0.831511in}}%
\pgfpathlineto{\pgfqpoint{2.321481in}{0.831491in}}%
\pgfpathlineto{\pgfqpoint{2.321777in}{0.831472in}}%
\pgfpathlineto{\pgfqpoint{2.322073in}{0.831452in}}%
\pgfpathlineto{\pgfqpoint{2.322369in}{0.831408in}}%
\pgfpathlineto{\pgfqpoint{2.322665in}{0.829167in}}%
\pgfpathlineto{\pgfqpoint{2.322961in}{0.828855in}}%
\pgfpathlineto{\pgfqpoint{2.323257in}{0.828617in}}%
\pgfpathlineto{\pgfqpoint{2.323553in}{0.828380in}}%
\pgfpathlineto{\pgfqpoint{2.323849in}{0.828142in}}%
\pgfpathlineto{\pgfqpoint{2.324145in}{0.827904in}}%
\pgfpathlineto{\pgfqpoint{2.324441in}{0.827667in}}%
\pgfpathlineto{\pgfqpoint{2.324737in}{0.827463in}}%
\pgfpathlineto{\pgfqpoint{2.325033in}{0.827385in}}%
\pgfpathlineto{\pgfqpoint{2.325329in}{0.827337in}}%
\pgfpathlineto{\pgfqpoint{2.325625in}{0.827335in}}%
\pgfpathlineto{\pgfqpoint{2.325921in}{0.827336in}}%
\pgfpathlineto{\pgfqpoint{2.326217in}{0.827352in}}%
\pgfpathlineto{\pgfqpoint{2.326513in}{0.827368in}}%
\pgfpathlineto{\pgfqpoint{2.326809in}{0.827367in}}%
\pgfpathlineto{\pgfqpoint{2.327105in}{0.827367in}}%
\pgfpathlineto{\pgfqpoint{2.327401in}{0.827366in}}%
\pgfpathlineto{\pgfqpoint{2.327697in}{0.827366in}}%
\pgfpathlineto{\pgfqpoint{2.327993in}{0.827365in}}%
\pgfpathlineto{\pgfqpoint{2.328289in}{0.827365in}}%
\pgfpathlineto{\pgfqpoint{2.328585in}{0.827364in}}%
\pgfpathlineto{\pgfqpoint{2.328881in}{0.827364in}}%
\pgfpathlineto{\pgfqpoint{2.329177in}{0.827364in}}%
\pgfpathlineto{\pgfqpoint{2.329473in}{0.827363in}}%
\pgfpathlineto{\pgfqpoint{2.329769in}{0.827363in}}%
\pgfpathlineto{\pgfqpoint{2.330065in}{0.827363in}}%
\pgfpathlineto{\pgfqpoint{2.330361in}{0.827419in}}%
\pgfpathlineto{\pgfqpoint{2.330657in}{0.827508in}}%
\pgfpathlineto{\pgfqpoint{2.330953in}{0.827556in}}%
\pgfpathlineto{\pgfqpoint{2.331249in}{0.827549in}}%
\pgfpathlineto{\pgfqpoint{2.331545in}{0.827567in}}%
\pgfpathlineto{\pgfqpoint{2.331841in}{0.827575in}}%
\pgfpathlineto{\pgfqpoint{2.332137in}{0.827582in}}%
\pgfpathlineto{\pgfqpoint{2.332433in}{0.827590in}}%
\pgfpathlineto{\pgfqpoint{2.332729in}{0.827598in}}%
\pgfpathlineto{\pgfqpoint{2.333025in}{0.827606in}}%
\pgfpathlineto{\pgfqpoint{2.333321in}{0.827614in}}%
\pgfpathlineto{\pgfqpoint{2.333618in}{0.827621in}}%
\pgfpathlineto{\pgfqpoint{2.333914in}{0.827629in}}%
\pgfpathlineto{\pgfqpoint{2.334210in}{0.827637in}}%
\pgfpathlineto{\pgfqpoint{2.334506in}{0.827645in}}%
\pgfpathlineto{\pgfqpoint{2.334802in}{0.827653in}}%
\pgfpathlineto{\pgfqpoint{2.335098in}{0.827660in}}%
\pgfpathlineto{\pgfqpoint{2.335394in}{0.827668in}}%
\pgfpathlineto{\pgfqpoint{2.335690in}{0.827676in}}%
\pgfpathlineto{\pgfqpoint{2.335986in}{0.827684in}}%
\pgfpathlineto{\pgfqpoint{2.336282in}{0.827691in}}%
\pgfpathlineto{\pgfqpoint{2.336578in}{0.827699in}}%
\pgfpathlineto{\pgfqpoint{2.336874in}{0.827706in}}%
\pgfpathlineto{\pgfqpoint{2.337170in}{0.827669in}}%
\pgfpathlineto{\pgfqpoint{2.337466in}{0.827619in}}%
\pgfpathlineto{\pgfqpoint{2.337762in}{0.827669in}}%
\pgfpathlineto{\pgfqpoint{2.338058in}{0.827643in}}%
\pgfpathlineto{\pgfqpoint{2.338354in}{0.827628in}}%
\pgfpathlineto{\pgfqpoint{2.338650in}{0.827614in}}%
\pgfpathlineto{\pgfqpoint{2.338946in}{0.827601in}}%
\pgfpathlineto{\pgfqpoint{2.339242in}{0.827599in}}%
\pgfpathlineto{\pgfqpoint{2.339538in}{0.827598in}}%
\pgfpathlineto{\pgfqpoint{2.339834in}{0.827597in}}%
\pgfpathlineto{\pgfqpoint{2.340130in}{0.827597in}}%
\pgfpathlineto{\pgfqpoint{2.340426in}{0.827596in}}%
\pgfpathlineto{\pgfqpoint{2.340722in}{0.827595in}}%
\pgfpathlineto{\pgfqpoint{2.341018in}{0.827595in}}%
\pgfpathlineto{\pgfqpoint{2.341314in}{0.827594in}}%
\pgfpathlineto{\pgfqpoint{2.341610in}{0.827593in}}%
\pgfpathlineto{\pgfqpoint{2.341906in}{0.827593in}}%
\pgfpathlineto{\pgfqpoint{2.342202in}{0.827592in}}%
\pgfpathlineto{\pgfqpoint{2.342498in}{0.827592in}}%
\pgfpathlineto{\pgfqpoint{2.342794in}{0.827591in}}%
\pgfpathlineto{\pgfqpoint{2.343090in}{0.827590in}}%
\pgfpathlineto{\pgfqpoint{2.343386in}{0.827590in}}%
\pgfpathlineto{\pgfqpoint{2.343682in}{0.827589in}}%
\pgfpathlineto{\pgfqpoint{2.343978in}{0.827588in}}%
\pgfpathlineto{\pgfqpoint{2.344274in}{0.827588in}}%
\pgfpathlineto{\pgfqpoint{2.344570in}{0.827587in}}%
\pgfpathlineto{\pgfqpoint{2.344866in}{0.827586in}}%
\pgfpathlineto{\pgfqpoint{2.345162in}{0.827586in}}%
\pgfpathlineto{\pgfqpoint{2.345458in}{0.827585in}}%
\pgfpathlineto{\pgfqpoint{2.345754in}{0.827584in}}%
\pgfpathlineto{\pgfqpoint{2.346050in}{0.827584in}}%
\pgfpathlineto{\pgfqpoint{2.346346in}{0.827583in}}%
\pgfpathlineto{\pgfqpoint{2.346642in}{0.827583in}}%
\pgfpathlineto{\pgfqpoint{2.346938in}{0.827582in}}%
\pgfpathlineto{\pgfqpoint{2.347234in}{0.827581in}}%
\pgfpathlineto{\pgfqpoint{2.347530in}{0.827581in}}%
\pgfpathlineto{\pgfqpoint{2.347826in}{0.827580in}}%
\pgfpathlineto{\pgfqpoint{2.348122in}{0.827579in}}%
\pgfpathlineto{\pgfqpoint{2.348418in}{0.827579in}}%
\pgfpathlineto{\pgfqpoint{2.348714in}{0.827578in}}%
\pgfpathlineto{\pgfqpoint{2.349010in}{0.827577in}}%
\pgfpathlineto{\pgfqpoint{2.349306in}{0.827577in}}%
\pgfpathlineto{\pgfqpoint{2.349602in}{0.827576in}}%
\pgfpathlineto{\pgfqpoint{2.349898in}{0.827575in}}%
\pgfpathlineto{\pgfqpoint{2.350194in}{0.827575in}}%
\pgfpathlineto{\pgfqpoint{2.350490in}{0.827574in}}%
\pgfpathlineto{\pgfqpoint{2.350786in}{0.827574in}}%
\pgfpathlineto{\pgfqpoint{2.351082in}{0.827573in}}%
\pgfpathlineto{\pgfqpoint{2.351378in}{0.827572in}}%
\pgfpathlineto{\pgfqpoint{2.351674in}{0.827572in}}%
\pgfpathlineto{\pgfqpoint{2.351970in}{0.827571in}}%
\pgfpathlineto{\pgfqpoint{2.352266in}{0.827570in}}%
\pgfpathlineto{\pgfqpoint{2.352562in}{0.827570in}}%
\pgfpathlineto{\pgfqpoint{2.352858in}{0.827569in}}%
\pgfpathlineto{\pgfqpoint{2.353154in}{0.827568in}}%
\pgfpathlineto{\pgfqpoint{2.353450in}{0.827568in}}%
\pgfpathlineto{\pgfqpoint{2.353746in}{0.827567in}}%
\pgfpathlineto{\pgfqpoint{2.354042in}{0.827566in}}%
\pgfpathlineto{\pgfqpoint{2.354338in}{0.827566in}}%
\pgfpathlineto{\pgfqpoint{2.354634in}{0.827565in}}%
\pgfpathlineto{\pgfqpoint{2.354930in}{0.827565in}}%
\pgfpathlineto{\pgfqpoint{2.355226in}{0.827564in}}%
\pgfpathlineto{\pgfqpoint{2.355522in}{0.827563in}}%
\pgfpathlineto{\pgfqpoint{2.355818in}{0.827563in}}%
\pgfpathlineto{\pgfqpoint{2.356114in}{0.827562in}}%
\pgfpathlineto{\pgfqpoint{2.356410in}{0.827561in}}%
\pgfpathlineto{\pgfqpoint{2.356706in}{0.827561in}}%
\pgfpathlineto{\pgfqpoint{2.357002in}{0.827560in}}%
\pgfpathlineto{\pgfqpoint{2.357298in}{0.827559in}}%
\pgfpathlineto{\pgfqpoint{2.357594in}{0.827559in}}%
\pgfpathlineto{\pgfqpoint{2.357890in}{0.827558in}}%
\pgfpathlineto{\pgfqpoint{2.358186in}{0.827563in}}%
\pgfpathlineto{\pgfqpoint{2.358482in}{0.827571in}}%
\pgfpathlineto{\pgfqpoint{2.358778in}{0.827578in}}%
\pgfpathlineto{\pgfqpoint{2.359074in}{0.827586in}}%
\pgfpathlineto{\pgfqpoint{2.359370in}{0.827594in}}%
\pgfpathlineto{\pgfqpoint{2.359666in}{0.827601in}}%
\pgfpathlineto{\pgfqpoint{2.359962in}{0.827609in}}%
\pgfpathlineto{\pgfqpoint{2.360258in}{0.827616in}}%
\pgfpathlineto{\pgfqpoint{2.360554in}{0.827681in}}%
\pgfpathlineto{\pgfqpoint{2.360850in}{0.827711in}}%
\pgfpathlineto{\pgfqpoint{2.361146in}{0.827724in}}%
\pgfpathlineto{\pgfqpoint{2.361442in}{0.827738in}}%
\pgfpathlineto{\pgfqpoint{2.361738in}{0.827751in}}%
\pgfpathlineto{\pgfqpoint{2.362034in}{0.827764in}}%
\pgfpathlineto{\pgfqpoint{2.362330in}{0.827777in}}%
\pgfpathlineto{\pgfqpoint{2.362626in}{0.827790in}}%
\pgfpathlineto{\pgfqpoint{2.362922in}{0.827803in}}%
\pgfpathlineto{\pgfqpoint{2.363218in}{0.827816in}}%
\pgfpathlineto{\pgfqpoint{2.363514in}{0.827829in}}%
\pgfpathlineto{\pgfqpoint{2.363810in}{0.827843in}}%
\pgfpathlineto{\pgfqpoint{2.364106in}{0.827856in}}%
\pgfpathlineto{\pgfqpoint{2.364402in}{0.827869in}}%
\pgfpathlineto{\pgfqpoint{2.364698in}{0.827882in}}%
\pgfpathlineto{\pgfqpoint{2.364994in}{0.827895in}}%
\pgfpathlineto{\pgfqpoint{2.365290in}{0.828141in}}%
\pgfpathlineto{\pgfqpoint{2.365586in}{0.828757in}}%
\pgfpathlineto{\pgfqpoint{2.365882in}{0.829369in}}%
\pgfpathlineto{\pgfqpoint{2.366178in}{0.829979in}}%
\pgfpathlineto{\pgfqpoint{2.366474in}{0.829070in}}%
\pgfpathlineto{\pgfqpoint{2.366770in}{0.828066in}}%
\pgfpathlineto{\pgfqpoint{2.367066in}{0.827595in}}%
\pgfpathlineto{\pgfqpoint{2.367362in}{0.829805in}}%
\pgfpathlineto{\pgfqpoint{2.367658in}{0.829698in}}%
\pgfpathlineto{\pgfqpoint{2.367954in}{0.829499in}}%
\pgfpathlineto{\pgfqpoint{2.368250in}{0.829300in}}%
\pgfpathlineto{\pgfqpoint{2.368546in}{0.829101in}}%
\pgfpathlineto{\pgfqpoint{2.368842in}{0.828902in}}%
\pgfpathlineto{\pgfqpoint{2.369138in}{0.828703in}}%
\pgfpathlineto{\pgfqpoint{2.369434in}{0.828504in}}%
\pgfpathlineto{\pgfqpoint{2.369730in}{0.828305in}}%
\pgfpathlineto{\pgfqpoint{2.370026in}{0.828106in}}%
\pgfpathlineto{\pgfqpoint{2.370322in}{0.827907in}}%
\pgfpathlineto{\pgfqpoint{2.370618in}{0.827709in}}%
\pgfpathlineto{\pgfqpoint{2.370914in}{0.827510in}}%
\pgfpathlineto{\pgfqpoint{2.371210in}{0.827311in}}%
\pgfpathlineto{\pgfqpoint{2.371506in}{0.827112in}}%
\pgfpathlineto{\pgfqpoint{2.371802in}{0.826913in}}%
\pgfpathlineto{\pgfqpoint{2.372098in}{0.826715in}}%
\pgfpathlineto{\pgfqpoint{2.372394in}{0.826610in}}%
\pgfpathlineto{\pgfqpoint{2.372690in}{0.826525in}}%
\pgfpathlineto{\pgfqpoint{2.372986in}{0.826395in}}%
\pgfpathlineto{\pgfqpoint{2.373282in}{0.826265in}}%
\pgfpathlineto{\pgfqpoint{2.373578in}{0.826135in}}%
\pgfpathlineto{\pgfqpoint{2.373874in}{0.826005in}}%
\pgfpathlineto{\pgfqpoint{2.374170in}{0.825876in}}%
\pgfpathlineto{\pgfqpoint{2.374466in}{0.825824in}}%
\pgfpathlineto{\pgfqpoint{2.374762in}{0.825815in}}%
\pgfpathlineto{\pgfqpoint{2.375058in}{0.825806in}}%
\pgfpathlineto{\pgfqpoint{2.375354in}{0.825797in}}%
\pgfpathlineto{\pgfqpoint{2.375650in}{0.825788in}}%
\pgfpathlineto{\pgfqpoint{2.375946in}{0.825779in}}%
\pgfpathlineto{\pgfqpoint{2.376242in}{0.825770in}}%
\pgfpathlineto{\pgfqpoint{2.376538in}{0.825761in}}%
\pgfpathlineto{\pgfqpoint{2.376834in}{0.825752in}}%
\pgfpathlineto{\pgfqpoint{2.377130in}{0.825743in}}%
\pgfpathlineto{\pgfqpoint{2.377426in}{0.825734in}}%
\pgfpathlineto{\pgfqpoint{2.377722in}{0.825725in}}%
\pgfpathlineto{\pgfqpoint{2.378018in}{0.825716in}}%
\pgfpathlineto{\pgfqpoint{2.378314in}{0.825707in}}%
\pgfpathlineto{\pgfqpoint{2.378610in}{0.825698in}}%
\pgfpathlineto{\pgfqpoint{2.378906in}{0.825689in}}%
\pgfpathlineto{\pgfqpoint{2.379202in}{0.825677in}}%
\pgfpathlineto{\pgfqpoint{2.379498in}{0.825605in}}%
\pgfpathlineto{\pgfqpoint{2.379794in}{0.825503in}}%
\pgfpathlineto{\pgfqpoint{2.380090in}{0.825401in}}%
\pgfpathlineto{\pgfqpoint{2.380386in}{0.825358in}}%
\pgfpathlineto{\pgfqpoint{2.380682in}{0.825336in}}%
\pgfpathlineto{\pgfqpoint{2.380978in}{0.825315in}}%
\pgfpathlineto{\pgfqpoint{2.381274in}{0.825077in}}%
\pgfpathlineto{\pgfqpoint{2.381570in}{0.824395in}}%
\pgfpathlineto{\pgfqpoint{2.381866in}{0.823766in}}%
\pgfpathlineto{\pgfqpoint{2.382162in}{0.823821in}}%
\pgfpathlineto{\pgfqpoint{2.382458in}{0.824058in}}%
\pgfpathlineto{\pgfqpoint{2.382754in}{0.824295in}}%
\pgfpathlineto{\pgfqpoint{2.383050in}{0.824532in}}%
\pgfpathlineto{\pgfqpoint{2.383346in}{0.824769in}}%
\pgfpathlineto{\pgfqpoint{2.383642in}{0.825006in}}%
\pgfpathlineto{\pgfqpoint{2.383938in}{0.825242in}}%
\pgfpathlineto{\pgfqpoint{2.384234in}{0.825479in}}%
\pgfpathlineto{\pgfqpoint{2.384530in}{0.825716in}}%
\pgfpathlineto{\pgfqpoint{2.384826in}{0.825953in}}%
\pgfpathlineto{\pgfqpoint{2.385122in}{0.826190in}}%
\pgfpathlineto{\pgfqpoint{2.385418in}{0.826427in}}%
\pgfpathlineto{\pgfqpoint{2.385714in}{0.826664in}}%
\pgfpathlineto{\pgfqpoint{2.386010in}{0.826901in}}%
\pgfpathlineto{\pgfqpoint{2.386306in}{0.827138in}}%
\pgfpathlineto{\pgfqpoint{2.386602in}{0.827359in}}%
\pgfpathlineto{\pgfqpoint{2.386898in}{0.827385in}}%
\pgfpathlineto{\pgfqpoint{2.387194in}{0.827312in}}%
\pgfpathlineto{\pgfqpoint{2.387490in}{0.827239in}}%
\pgfpathlineto{\pgfqpoint{2.387786in}{0.827166in}}%
\pgfpathlineto{\pgfqpoint{2.388082in}{0.827036in}}%
\pgfpathlineto{\pgfqpoint{2.388378in}{0.826837in}}%
\pgfpathlineto{\pgfqpoint{2.388674in}{0.826725in}}%
\pgfpathlineto{\pgfqpoint{2.388970in}{0.826589in}}%
\pgfpathlineto{\pgfqpoint{2.389266in}{0.826476in}}%
\pgfpathlineto{\pgfqpoint{2.389562in}{0.826362in}}%
\pgfpathlineto{\pgfqpoint{2.389858in}{0.826249in}}%
\pgfpathlineto{\pgfqpoint{2.390154in}{0.826136in}}%
\pgfpathlineto{\pgfqpoint{2.390450in}{0.826023in}}%
\pgfpathlineto{\pgfqpoint{2.390746in}{0.825909in}}%
\pgfpathlineto{\pgfqpoint{2.391042in}{0.825796in}}%
\pgfpathlineto{\pgfqpoint{2.391338in}{0.825683in}}%
\pgfpathlineto{\pgfqpoint{2.391634in}{0.825570in}}%
\pgfpathlineto{\pgfqpoint{2.391930in}{0.825456in}}%
\pgfpathlineto{\pgfqpoint{2.392226in}{0.825343in}}%
\pgfpathlineto{\pgfqpoint{2.392522in}{0.825230in}}%
\pgfpathlineto{\pgfqpoint{2.392818in}{0.825116in}}%
\pgfpathlineto{\pgfqpoint{2.393114in}{0.825003in}}%
\pgfpathlineto{\pgfqpoint{2.393410in}{0.824890in}}%
\pgfpathlineto{\pgfqpoint{2.393706in}{0.824777in}}%
\pgfpathlineto{\pgfqpoint{2.394002in}{0.824663in}}%
\pgfpathlineto{\pgfqpoint{2.394298in}{0.824550in}}%
\pgfpathlineto{\pgfqpoint{2.394594in}{0.824437in}}%
\pgfpathlineto{\pgfqpoint{2.394890in}{0.824324in}}%
\pgfpathlineto{\pgfqpoint{2.395186in}{0.824210in}}%
\pgfpathlineto{\pgfqpoint{2.395482in}{0.824097in}}%
\pgfpathlineto{\pgfqpoint{2.395778in}{0.823984in}}%
\pgfpathlineto{\pgfqpoint{2.396074in}{0.823871in}}%
\pgfpathlineto{\pgfqpoint{2.396370in}{0.823757in}}%
\pgfpathlineto{\pgfqpoint{2.396666in}{0.823644in}}%
\pgfpathlineto{\pgfqpoint{2.396962in}{0.823531in}}%
\pgfpathlineto{\pgfqpoint{2.397258in}{0.823418in}}%
\pgfpathlineto{\pgfqpoint{2.397554in}{0.823304in}}%
\pgfpathlineto{\pgfqpoint{2.397850in}{0.823191in}}%
\pgfpathlineto{\pgfqpoint{2.398146in}{0.823078in}}%
\pgfpathlineto{\pgfqpoint{2.398442in}{0.822965in}}%
\pgfpathlineto{\pgfqpoint{2.398738in}{0.822851in}}%
\pgfpathlineto{\pgfqpoint{2.399034in}{0.822738in}}%
\pgfpathlineto{\pgfqpoint{2.399330in}{0.822625in}}%
\pgfpathlineto{\pgfqpoint{2.399626in}{0.822512in}}%
\pgfpathlineto{\pgfqpoint{2.399922in}{0.822398in}}%
\pgfpathlineto{\pgfqpoint{2.400218in}{0.822285in}}%
\pgfpathlineto{\pgfqpoint{2.400514in}{0.822172in}}%
\pgfpathlineto{\pgfqpoint{2.400811in}{0.822058in}}%
\pgfpathlineto{\pgfqpoint{2.401107in}{0.822041in}}%
\pgfpathlineto{\pgfqpoint{2.401403in}{0.822219in}}%
\pgfpathlineto{\pgfqpoint{2.401699in}{0.822400in}}%
\pgfpathlineto{\pgfqpoint{2.401995in}{0.822582in}}%
\pgfpathlineto{\pgfqpoint{2.402291in}{0.822764in}}%
\pgfpathlineto{\pgfqpoint{2.402587in}{0.822946in}}%
\pgfpathlineto{\pgfqpoint{2.402883in}{0.823128in}}%
\pgfpathlineto{\pgfqpoint{2.403179in}{0.823309in}}%
\pgfpathlineto{\pgfqpoint{2.403475in}{0.823491in}}%
\pgfpathlineto{\pgfqpoint{2.403771in}{0.823673in}}%
\pgfpathlineto{\pgfqpoint{2.404067in}{0.823855in}}%
\pgfpathlineto{\pgfqpoint{2.404363in}{0.824037in}}%
\pgfpathlineto{\pgfqpoint{2.404659in}{0.824218in}}%
\pgfpathlineto{\pgfqpoint{2.404955in}{0.824400in}}%
\pgfpathlineto{\pgfqpoint{2.405251in}{0.824582in}}%
\pgfpathlineto{\pgfqpoint{2.405547in}{0.824764in}}%
\pgfpathlineto{\pgfqpoint{2.405843in}{0.824946in}}%
\pgfpathlineto{\pgfqpoint{2.406139in}{0.825127in}}%
\pgfpathlineto{\pgfqpoint{2.406435in}{0.825309in}}%
\pgfpathlineto{\pgfqpoint{2.406731in}{0.825491in}}%
\pgfpathlineto{\pgfqpoint{2.407027in}{0.825673in}}%
\pgfpathlineto{\pgfqpoint{2.407323in}{0.825855in}}%
\pgfpathlineto{\pgfqpoint{2.407619in}{0.826036in}}%
\pgfpathlineto{\pgfqpoint{2.407915in}{0.826218in}}%
\pgfpathlineto{\pgfqpoint{2.408211in}{0.826487in}}%
\pgfpathlineto{\pgfqpoint{2.408507in}{0.826530in}}%
\pgfpathlineto{\pgfqpoint{2.408803in}{0.826462in}}%
\pgfpathlineto{\pgfqpoint{2.409099in}{0.824629in}}%
\pgfpathlineto{\pgfqpoint{2.409395in}{0.822238in}}%
\pgfpathlineto{\pgfqpoint{2.409691in}{0.822322in}}%
\pgfpathlineto{\pgfqpoint{2.409987in}{0.822804in}}%
\pgfpathlineto{\pgfqpoint{2.410283in}{0.823246in}}%
\pgfpathlineto{\pgfqpoint{2.410579in}{0.823908in}}%
\pgfpathlineto{\pgfqpoint{2.410875in}{0.824600in}}%
\pgfpathlineto{\pgfqpoint{2.411171in}{0.825293in}}%
\pgfpathlineto{\pgfqpoint{2.411467in}{0.825889in}}%
\pgfpathlineto{\pgfqpoint{2.411763in}{0.825964in}}%
\pgfpathlineto{\pgfqpoint{2.412059in}{0.825961in}}%
\pgfpathlineto{\pgfqpoint{2.412355in}{0.825959in}}%
\pgfpathlineto{\pgfqpoint{2.412651in}{0.825956in}}%
\pgfpathlineto{\pgfqpoint{2.412947in}{0.825953in}}%
\pgfpathlineto{\pgfqpoint{2.413243in}{0.825950in}}%
\pgfpathlineto{\pgfqpoint{2.413539in}{0.825947in}}%
\pgfpathlineto{\pgfqpoint{2.413835in}{0.825944in}}%
\pgfpathlineto{\pgfqpoint{2.414131in}{0.825941in}}%
\pgfpathlineto{\pgfqpoint{2.414427in}{0.825939in}}%
\pgfpathlineto{\pgfqpoint{2.414723in}{0.825936in}}%
\pgfpathlineto{\pgfqpoint{2.415019in}{0.825914in}}%
\pgfpathlineto{\pgfqpoint{2.415315in}{0.825257in}}%
\pgfpathlineto{\pgfqpoint{2.415611in}{0.824843in}}%
\pgfpathlineto{\pgfqpoint{2.415907in}{0.824803in}}%
\pgfpathlineto{\pgfqpoint{2.416203in}{0.824799in}}%
\pgfpathlineto{\pgfqpoint{2.416499in}{0.824737in}}%
\pgfpathlineto{\pgfqpoint{2.416795in}{0.824549in}}%
\pgfpathlineto{\pgfqpoint{2.417091in}{0.824357in}}%
\pgfpathlineto{\pgfqpoint{2.417387in}{0.824165in}}%
\pgfpathlineto{\pgfqpoint{2.417683in}{0.823889in}}%
\pgfpathlineto{\pgfqpoint{2.417979in}{0.823161in}}%
\pgfpathlineto{\pgfqpoint{2.418275in}{0.822709in}}%
\pgfpathlineto{\pgfqpoint{2.418571in}{0.822689in}}%
\pgfpathlineto{\pgfqpoint{2.418867in}{0.822670in}}%
\pgfpathlineto{\pgfqpoint{2.419163in}{0.822652in}}%
\pgfpathlineto{\pgfqpoint{2.419459in}{0.822633in}}%
\pgfpathlineto{\pgfqpoint{2.419755in}{0.822614in}}%
\pgfpathlineto{\pgfqpoint{2.420051in}{0.822596in}}%
\pgfpathlineto{\pgfqpoint{2.420347in}{0.822577in}}%
\pgfpathlineto{\pgfqpoint{2.420643in}{0.822559in}}%
\pgfpathlineto{\pgfqpoint{2.420939in}{0.822540in}}%
\pgfpathlineto{\pgfqpoint{2.421235in}{0.822521in}}%
\pgfpathlineto{\pgfqpoint{2.421531in}{0.822503in}}%
\pgfpathlineto{\pgfqpoint{2.421827in}{0.822473in}}%
\pgfpathlineto{\pgfqpoint{2.422123in}{0.822438in}}%
\pgfpathlineto{\pgfqpoint{2.422419in}{0.822419in}}%
\pgfpathlineto{\pgfqpoint{2.422715in}{0.822400in}}%
\pgfpathlineto{\pgfqpoint{2.423011in}{0.822382in}}%
\pgfpathlineto{\pgfqpoint{2.423307in}{0.822363in}}%
\pgfpathlineto{\pgfqpoint{2.423603in}{0.822344in}}%
\pgfpathlineto{\pgfqpoint{2.423899in}{0.822325in}}%
\pgfpathlineto{\pgfqpoint{2.424195in}{0.822307in}}%
\pgfpathlineto{\pgfqpoint{2.424491in}{0.822288in}}%
\pgfpathlineto{\pgfqpoint{2.424787in}{0.822269in}}%
\pgfpathlineto{\pgfqpoint{2.425083in}{0.822251in}}%
\pgfpathlineto{\pgfqpoint{2.425379in}{0.822232in}}%
\pgfpathlineto{\pgfqpoint{2.425675in}{0.822213in}}%
\pgfpathlineto{\pgfqpoint{2.425971in}{0.822195in}}%
\pgfpathlineto{\pgfqpoint{2.426267in}{0.822176in}}%
\pgfpathlineto{\pgfqpoint{2.426563in}{0.822157in}}%
\pgfpathlineto{\pgfqpoint{2.426859in}{0.822139in}}%
\pgfpathlineto{\pgfqpoint{2.427155in}{0.822120in}}%
\pgfpathlineto{\pgfqpoint{2.427451in}{0.822101in}}%
\pgfpathlineto{\pgfqpoint{2.427747in}{0.822082in}}%
\pgfpathlineto{\pgfqpoint{2.428043in}{0.822064in}}%
\pgfpathlineto{\pgfqpoint{2.428339in}{0.822045in}}%
\pgfpathlineto{\pgfqpoint{2.428635in}{0.822026in}}%
\pgfpathlineto{\pgfqpoint{2.428931in}{0.822007in}}%
\pgfpathlineto{\pgfqpoint{2.429227in}{0.821984in}}%
\pgfpathlineto{\pgfqpoint{2.429523in}{0.821982in}}%
\pgfpathlineto{\pgfqpoint{2.429819in}{0.821992in}}%
\pgfpathlineto{\pgfqpoint{2.430115in}{0.822000in}}%
\pgfpathlineto{\pgfqpoint{2.430411in}{0.821980in}}%
\pgfpathlineto{\pgfqpoint{2.430707in}{0.822001in}}%
\pgfpathlineto{\pgfqpoint{2.431003in}{0.821855in}}%
\pgfpathlineto{\pgfqpoint{2.431299in}{0.821740in}}%
\pgfpathlineto{\pgfqpoint{2.431595in}{0.821603in}}%
\pgfpathlineto{\pgfqpoint{2.431891in}{0.821389in}}%
\pgfpathlineto{\pgfqpoint{2.432187in}{0.820380in}}%
\pgfpathlineto{\pgfqpoint{2.432483in}{0.819751in}}%
\pgfpathlineto{\pgfqpoint{2.432779in}{0.819736in}}%
\pgfpathlineto{\pgfqpoint{2.433075in}{0.819722in}}%
\pgfpathlineto{\pgfqpoint{2.433371in}{0.819707in}}%
\pgfpathlineto{\pgfqpoint{2.433667in}{0.819692in}}%
\pgfpathlineto{\pgfqpoint{2.433963in}{0.819677in}}%
\pgfpathlineto{\pgfqpoint{2.434259in}{0.819663in}}%
\pgfpathlineto{\pgfqpoint{2.434555in}{0.819648in}}%
\pgfpathlineto{\pgfqpoint{2.434851in}{0.819633in}}%
\pgfpathlineto{\pgfqpoint{2.435147in}{0.819619in}}%
\pgfpathlineto{\pgfqpoint{2.435443in}{0.819604in}}%
\pgfpathlineto{\pgfqpoint{2.435739in}{0.819589in}}%
\pgfpathlineto{\pgfqpoint{2.436035in}{0.819575in}}%
\pgfpathlineto{\pgfqpoint{2.436331in}{0.819625in}}%
\pgfpathlineto{\pgfqpoint{2.436627in}{0.819643in}}%
\pgfpathlineto{\pgfqpoint{2.436923in}{0.819613in}}%
\pgfpathlineto{\pgfqpoint{2.437219in}{0.819624in}}%
\pgfpathlineto{\pgfqpoint{2.437515in}{0.819635in}}%
\pgfpathlineto{\pgfqpoint{2.437811in}{0.819647in}}%
\pgfpathlineto{\pgfqpoint{2.438107in}{0.819658in}}%
\pgfpathlineto{\pgfqpoint{2.438403in}{0.819669in}}%
\pgfpathlineto{\pgfqpoint{2.438699in}{0.819680in}}%
\pgfpathlineto{\pgfqpoint{2.438995in}{0.819692in}}%
\pgfpathlineto{\pgfqpoint{2.439291in}{0.819700in}}%
\pgfpathlineto{\pgfqpoint{2.439587in}{0.819937in}}%
\pgfpathlineto{\pgfqpoint{2.439883in}{0.820722in}}%
\pgfpathlineto{\pgfqpoint{2.440179in}{0.820736in}}%
\pgfpathlineto{\pgfqpoint{2.440475in}{0.820750in}}%
\pgfpathlineto{\pgfqpoint{2.440771in}{0.820764in}}%
\pgfpathlineto{\pgfqpoint{2.441067in}{0.820778in}}%
\pgfpathlineto{\pgfqpoint{2.441363in}{0.820792in}}%
\pgfpathlineto{\pgfqpoint{2.441659in}{0.820806in}}%
\pgfpathlineto{\pgfqpoint{2.441955in}{0.820820in}}%
\pgfpathlineto{\pgfqpoint{2.442251in}{0.820834in}}%
\pgfpathlineto{\pgfqpoint{2.442547in}{0.820847in}}%
\pgfpathlineto{\pgfqpoint{2.442843in}{0.820861in}}%
\pgfpathlineto{\pgfqpoint{2.443139in}{0.820875in}}%
\pgfpathlineto{\pgfqpoint{2.443435in}{0.820889in}}%
\pgfpathlineto{\pgfqpoint{2.443731in}{0.820903in}}%
\pgfpathlineto{\pgfqpoint{2.444027in}{0.820917in}}%
\pgfpathlineto{\pgfqpoint{2.444323in}{0.820931in}}%
\pgfpathlineto{\pgfqpoint{2.444619in}{0.820945in}}%
\pgfpathlineto{\pgfqpoint{2.444915in}{0.820959in}}%
\pgfpathlineto{\pgfqpoint{2.445211in}{0.820973in}}%
\pgfpathlineto{\pgfqpoint{2.445507in}{0.820986in}}%
\pgfpathlineto{\pgfqpoint{2.445803in}{0.821000in}}%
\pgfpathlineto{\pgfqpoint{2.446099in}{0.821014in}}%
\pgfpathlineto{\pgfqpoint{2.446395in}{0.821028in}}%
\pgfpathlineto{\pgfqpoint{2.446691in}{0.821042in}}%
\pgfpathlineto{\pgfqpoint{2.446987in}{0.821056in}}%
\pgfpathlineto{\pgfqpoint{2.447283in}{0.821070in}}%
\pgfpathlineto{\pgfqpoint{2.447579in}{0.821084in}}%
\pgfpathlineto{\pgfqpoint{2.447875in}{0.821098in}}%
\pgfpathlineto{\pgfqpoint{2.448171in}{0.821112in}}%
\pgfpathlineto{\pgfqpoint{2.448467in}{0.821125in}}%
\pgfpathlineto{\pgfqpoint{2.448763in}{0.821139in}}%
\pgfpathlineto{\pgfqpoint{2.449059in}{0.821153in}}%
\pgfpathlineto{\pgfqpoint{2.449355in}{0.821167in}}%
\pgfpathlineto{\pgfqpoint{2.449651in}{0.821181in}}%
\pgfpathlineto{\pgfqpoint{2.449947in}{0.821195in}}%
\pgfpathlineto{\pgfqpoint{2.450243in}{0.821209in}}%
\pgfpathlineto{\pgfqpoint{2.450539in}{0.821223in}}%
\pgfpathlineto{\pgfqpoint{2.450835in}{0.821237in}}%
\pgfpathlineto{\pgfqpoint{2.451131in}{0.821251in}}%
\pgfpathlineto{\pgfqpoint{2.451427in}{0.821265in}}%
\pgfpathlineto{\pgfqpoint{2.451723in}{0.821278in}}%
\pgfpathlineto{\pgfqpoint{2.452019in}{0.821292in}}%
\pgfpathlineto{\pgfqpoint{2.452315in}{0.821306in}}%
\pgfpathlineto{\pgfqpoint{2.452611in}{0.821320in}}%
\pgfpathlineto{\pgfqpoint{2.452907in}{0.821334in}}%
\pgfpathlineto{\pgfqpoint{2.453203in}{0.821348in}}%
\pgfpathlineto{\pgfqpoint{2.453499in}{0.821362in}}%
\pgfpathlineto{\pgfqpoint{2.453795in}{0.821376in}}%
\pgfpathlineto{\pgfqpoint{2.454091in}{0.821390in}}%
\pgfpathlineto{\pgfqpoint{2.454387in}{0.821404in}}%
\pgfpathlineto{\pgfqpoint{2.454683in}{0.821417in}}%
\pgfpathlineto{\pgfqpoint{2.454979in}{0.821431in}}%
\pgfpathlineto{\pgfqpoint{2.455275in}{0.821445in}}%
\pgfpathlineto{\pgfqpoint{2.455571in}{0.821459in}}%
\pgfpathlineto{\pgfqpoint{2.455867in}{0.821473in}}%
\pgfpathlineto{\pgfqpoint{2.456163in}{0.821487in}}%
\pgfpathlineto{\pgfqpoint{2.456459in}{0.821501in}}%
\pgfpathlineto{\pgfqpoint{2.456755in}{0.821515in}}%
\pgfpathlineto{\pgfqpoint{2.457051in}{0.821529in}}%
\pgfpathlineto{\pgfqpoint{2.457347in}{0.821534in}}%
\pgfpathlineto{\pgfqpoint{2.457643in}{0.821463in}}%
\pgfpathlineto{\pgfqpoint{2.457939in}{0.821373in}}%
\pgfpathlineto{\pgfqpoint{2.458235in}{0.821284in}}%
\pgfpathlineto{\pgfqpoint{2.458531in}{0.821195in}}%
\pgfpathlineto{\pgfqpoint{2.458827in}{0.820753in}}%
\pgfpathlineto{\pgfqpoint{2.459123in}{0.820298in}}%
\pgfpathlineto{\pgfqpoint{2.459419in}{0.820167in}}%
\pgfpathlineto{\pgfqpoint{2.459715in}{0.820036in}}%
\pgfpathlineto{\pgfqpoint{2.460011in}{0.819905in}}%
\pgfpathlineto{\pgfqpoint{2.460307in}{0.819775in}}%
\pgfpathlineto{\pgfqpoint{2.460603in}{0.819644in}}%
\pgfpathlineto{\pgfqpoint{2.460899in}{0.819442in}}%
\pgfpathlineto{\pgfqpoint{2.461195in}{0.819096in}}%
\pgfpathlineto{\pgfqpoint{2.461491in}{0.818975in}}%
\pgfpathlineto{\pgfqpoint{2.461787in}{0.818855in}}%
\pgfpathlineto{\pgfqpoint{2.462083in}{0.818734in}}%
\pgfpathlineto{\pgfqpoint{2.462379in}{0.818614in}}%
\pgfpathlineto{\pgfqpoint{2.462675in}{0.818493in}}%
\pgfpathlineto{\pgfqpoint{2.462971in}{0.818373in}}%
\pgfpathlineto{\pgfqpoint{2.463267in}{0.818253in}}%
\pgfpathlineto{\pgfqpoint{2.463563in}{0.818132in}}%
\pgfpathlineto{\pgfqpoint{2.463859in}{0.818012in}}%
\pgfpathlineto{\pgfqpoint{2.464155in}{0.817891in}}%
\pgfpathlineto{\pgfqpoint{2.464451in}{0.817771in}}%
\pgfpathlineto{\pgfqpoint{2.464747in}{0.817651in}}%
\pgfpathlineto{\pgfqpoint{2.465043in}{0.817530in}}%
\pgfpathlineto{\pgfqpoint{2.465339in}{0.817430in}}%
\pgfpathlineto{\pgfqpoint{2.465635in}{0.817342in}}%
\pgfpathlineto{\pgfqpoint{2.465931in}{0.817254in}}%
\pgfpathlineto{\pgfqpoint{2.466227in}{0.817168in}}%
\pgfpathlineto{\pgfqpoint{2.466523in}{0.817140in}}%
\pgfpathlineto{\pgfqpoint{2.466819in}{0.817124in}}%
\pgfpathlineto{\pgfqpoint{2.467115in}{0.817205in}}%
\pgfpathlineto{\pgfqpoint{2.467411in}{0.817327in}}%
\pgfpathlineto{\pgfqpoint{2.467707in}{0.817448in}}%
\pgfpathlineto{\pgfqpoint{2.468003in}{0.817570in}}%
\pgfpathlineto{\pgfqpoint{2.468300in}{0.817692in}}%
\pgfpathlineto{\pgfqpoint{2.468596in}{0.817814in}}%
\pgfpathlineto{\pgfqpoint{2.468892in}{0.817936in}}%
\pgfpathlineto{\pgfqpoint{2.469188in}{0.818058in}}%
\pgfpathlineto{\pgfqpoint{2.469484in}{0.818179in}}%
\pgfpathlineto{\pgfqpoint{2.469780in}{0.818301in}}%
\pgfpathlineto{\pgfqpoint{2.470076in}{0.818423in}}%
\pgfpathlineto{\pgfqpoint{2.470372in}{0.818545in}}%
\pgfpathlineto{\pgfqpoint{2.470668in}{0.818667in}}%
\pgfpathlineto{\pgfqpoint{2.470964in}{0.818789in}}%
\pgfpathlineto{\pgfqpoint{2.471260in}{0.818910in}}%
\pgfpathlineto{\pgfqpoint{2.471556in}{0.819120in}}%
\pgfpathlineto{\pgfqpoint{2.471852in}{0.819381in}}%
\pgfpathlineto{\pgfqpoint{2.472148in}{0.819375in}}%
\pgfpathlineto{\pgfqpoint{2.472444in}{0.819368in}}%
\pgfpathlineto{\pgfqpoint{2.472740in}{0.819362in}}%
\pgfpathlineto{\pgfqpoint{2.473036in}{0.819356in}}%
\pgfpathlineto{\pgfqpoint{2.473332in}{0.819349in}}%
\pgfpathlineto{\pgfqpoint{2.473628in}{0.819343in}}%
\pgfpathlineto{\pgfqpoint{2.473924in}{0.819336in}}%
\pgfpathlineto{\pgfqpoint{2.474220in}{0.819330in}}%
\pgfpathlineto{\pgfqpoint{2.474516in}{0.819324in}}%
\pgfpathlineto{\pgfqpoint{2.474812in}{0.819317in}}%
\pgfpathlineto{\pgfqpoint{2.475108in}{0.819311in}}%
\pgfpathlineto{\pgfqpoint{2.475404in}{0.819304in}}%
\pgfpathlineto{\pgfqpoint{2.475700in}{0.819298in}}%
\pgfpathlineto{\pgfqpoint{2.475996in}{0.819292in}}%
\pgfpathlineto{\pgfqpoint{2.476292in}{0.819285in}}%
\pgfpathlineto{\pgfqpoint{2.476588in}{0.819279in}}%
\pgfpathlineto{\pgfqpoint{2.476884in}{0.819272in}}%
\pgfpathlineto{\pgfqpoint{2.477180in}{0.819266in}}%
\pgfpathlineto{\pgfqpoint{2.477476in}{0.819259in}}%
\pgfpathlineto{\pgfqpoint{2.477772in}{0.819253in}}%
\pgfpathlineto{\pgfqpoint{2.478068in}{0.819247in}}%
\pgfpathlineto{\pgfqpoint{2.478364in}{0.819240in}}%
\pgfpathlineto{\pgfqpoint{2.478660in}{0.819234in}}%
\pgfpathlineto{\pgfqpoint{2.478956in}{0.819227in}}%
\pgfpathlineto{\pgfqpoint{2.479252in}{0.819221in}}%
\pgfpathlineto{\pgfqpoint{2.479548in}{0.819215in}}%
\pgfpathlineto{\pgfqpoint{2.479844in}{0.819208in}}%
\pgfpathlineto{\pgfqpoint{2.480140in}{0.819202in}}%
\pgfpathlineto{\pgfqpoint{2.480436in}{0.819195in}}%
\pgfpathlineto{\pgfqpoint{2.480732in}{0.819189in}}%
\pgfpathlineto{\pgfqpoint{2.481028in}{0.819183in}}%
\pgfpathlineto{\pgfqpoint{2.481324in}{0.819175in}}%
\pgfpathlineto{\pgfqpoint{2.481620in}{0.819155in}}%
\pgfpathlineto{\pgfqpoint{2.481916in}{0.819129in}}%
\pgfpathlineto{\pgfqpoint{2.482212in}{0.819103in}}%
\pgfpathlineto{\pgfqpoint{2.482508in}{0.819078in}}%
\pgfpathlineto{\pgfqpoint{2.482804in}{0.819052in}}%
\pgfpathlineto{\pgfqpoint{2.483100in}{0.819026in}}%
\pgfpathlineto{\pgfqpoint{2.483396in}{0.819000in}}%
\pgfpathlineto{\pgfqpoint{2.483692in}{0.818975in}}%
\pgfpathlineto{\pgfqpoint{2.483988in}{0.818949in}}%
\pgfpathlineto{\pgfqpoint{2.484284in}{0.818923in}}%
\pgfpathlineto{\pgfqpoint{2.484580in}{0.818897in}}%
\pgfpathlineto{\pgfqpoint{2.484876in}{0.818872in}}%
\pgfpathlineto{\pgfqpoint{2.485172in}{0.818846in}}%
\pgfpathlineto{\pgfqpoint{2.485468in}{0.818820in}}%
\pgfpathlineto{\pgfqpoint{2.485764in}{0.818794in}}%
\pgfpathlineto{\pgfqpoint{2.486060in}{0.818769in}}%
\pgfpathlineto{\pgfqpoint{2.486356in}{0.818743in}}%
\pgfpathlineto{\pgfqpoint{2.486652in}{0.818717in}}%
\pgfpathlineto{\pgfqpoint{2.486948in}{0.818691in}}%
\pgfpathlineto{\pgfqpoint{2.487244in}{0.818336in}}%
\pgfpathlineto{\pgfqpoint{2.487540in}{0.818146in}}%
\pgfpathlineto{\pgfqpoint{2.487836in}{0.818129in}}%
\pgfpathlineto{\pgfqpoint{2.488132in}{0.818111in}}%
\pgfpathlineto{\pgfqpoint{2.488428in}{0.818094in}}%
\pgfpathlineto{\pgfqpoint{2.488724in}{0.818077in}}%
\pgfpathlineto{\pgfqpoint{2.489020in}{0.818060in}}%
\pgfpathlineto{\pgfqpoint{2.489316in}{0.818043in}}%
\pgfpathlineto{\pgfqpoint{2.489612in}{0.818026in}}%
\pgfpathlineto{\pgfqpoint{2.489908in}{0.818008in}}%
\pgfpathlineto{\pgfqpoint{2.490204in}{0.817991in}}%
\pgfpathlineto{\pgfqpoint{2.490500in}{0.817974in}}%
\pgfpathlineto{\pgfqpoint{2.490796in}{0.817957in}}%
\pgfpathlineto{\pgfqpoint{2.491092in}{0.817940in}}%
\pgfpathlineto{\pgfqpoint{2.491388in}{0.817922in}}%
\pgfpathlineto{\pgfqpoint{2.491684in}{0.817905in}}%
\pgfpathlineto{\pgfqpoint{2.491980in}{0.817888in}}%
\pgfpathlineto{\pgfqpoint{2.492276in}{0.817871in}}%
\pgfpathlineto{\pgfqpoint{2.492572in}{0.817854in}}%
\pgfpathlineto{\pgfqpoint{2.492868in}{0.817836in}}%
\pgfpathlineto{\pgfqpoint{2.493164in}{0.817819in}}%
\pgfpathlineto{\pgfqpoint{2.493460in}{0.817802in}}%
\pgfpathlineto{\pgfqpoint{2.493756in}{0.817785in}}%
\pgfpathlineto{\pgfqpoint{2.494052in}{0.817768in}}%
\pgfpathlineto{\pgfqpoint{2.494348in}{0.817751in}}%
\pgfpathlineto{\pgfqpoint{2.494644in}{0.817733in}}%
\pgfpathlineto{\pgfqpoint{2.494940in}{0.817716in}}%
\pgfpathlineto{\pgfqpoint{2.495236in}{0.817699in}}%
\pgfpathlineto{\pgfqpoint{2.495532in}{0.817682in}}%
\pgfpathlineto{\pgfqpoint{2.495828in}{0.817665in}}%
\pgfpathlineto{\pgfqpoint{2.496124in}{0.817647in}}%
\pgfpathlineto{\pgfqpoint{2.496420in}{0.817630in}}%
\pgfpathlineto{\pgfqpoint{2.496716in}{0.817613in}}%
\pgfpathlineto{\pgfqpoint{2.497012in}{0.817596in}}%
\pgfpathlineto{\pgfqpoint{2.497308in}{0.817579in}}%
\pgfpathlineto{\pgfqpoint{2.497604in}{0.817561in}}%
\pgfpathlineto{\pgfqpoint{2.497900in}{0.817544in}}%
\pgfpathlineto{\pgfqpoint{2.498196in}{0.817527in}}%
\pgfpathlineto{\pgfqpoint{2.498492in}{0.817510in}}%
\pgfpathlineto{\pgfqpoint{2.498788in}{0.817493in}}%
\pgfpathlineto{\pgfqpoint{2.499084in}{0.817475in}}%
\pgfpathlineto{\pgfqpoint{2.499380in}{0.817458in}}%
\pgfpathlineto{\pgfqpoint{2.499676in}{0.817441in}}%
\pgfpathlineto{\pgfqpoint{2.499972in}{0.817424in}}%
\pgfpathlineto{\pgfqpoint{2.500268in}{0.817407in}}%
\pgfpathlineto{\pgfqpoint{2.500564in}{0.817390in}}%
\pgfpathlineto{\pgfqpoint{2.500860in}{0.817372in}}%
\pgfpathlineto{\pgfqpoint{2.501156in}{0.817355in}}%
\pgfpathlineto{\pgfqpoint{2.501452in}{0.817338in}}%
\pgfpathlineto{\pgfqpoint{2.501748in}{0.817321in}}%
\pgfpathlineto{\pgfqpoint{2.502044in}{0.817304in}}%
\pgfpathlineto{\pgfqpoint{2.502340in}{0.817286in}}%
\pgfpathlineto{\pgfqpoint{2.502636in}{0.817269in}}%
\pgfpathlineto{\pgfqpoint{2.502932in}{0.817252in}}%
\pgfpathlineto{\pgfqpoint{2.503228in}{0.817235in}}%
\pgfpathlineto{\pgfqpoint{2.503524in}{0.817218in}}%
\pgfpathlineto{\pgfqpoint{2.503820in}{0.817200in}}%
\pgfpathlineto{\pgfqpoint{2.504116in}{0.817183in}}%
\pgfpathlineto{\pgfqpoint{2.504412in}{0.817166in}}%
\pgfpathlineto{\pgfqpoint{2.504708in}{0.817149in}}%
\pgfpathlineto{\pgfqpoint{2.505004in}{0.817132in}}%
\pgfpathlineto{\pgfqpoint{2.505300in}{0.817115in}}%
\pgfpathlineto{\pgfqpoint{2.505596in}{0.817097in}}%
\pgfpathlineto{\pgfqpoint{2.505892in}{0.817080in}}%
\pgfpathlineto{\pgfqpoint{2.506188in}{0.817063in}}%
\pgfpathlineto{\pgfqpoint{2.506484in}{0.817046in}}%
\pgfpathlineto{\pgfqpoint{2.506780in}{0.817029in}}%
\pgfpathlineto{\pgfqpoint{2.507076in}{0.817011in}}%
\pgfpathlineto{\pgfqpoint{2.507372in}{0.816994in}}%
\pgfpathlineto{\pgfqpoint{2.507668in}{0.816977in}}%
\pgfpathlineto{\pgfqpoint{2.507964in}{0.816960in}}%
\pgfpathlineto{\pgfqpoint{2.508260in}{0.816943in}}%
\pgfpathlineto{\pgfqpoint{2.508556in}{0.816925in}}%
\pgfpathlineto{\pgfqpoint{2.508852in}{0.816905in}}%
\pgfpathlineto{\pgfqpoint{2.509148in}{0.816814in}}%
\pgfpathlineto{\pgfqpoint{2.509444in}{0.814763in}}%
\pgfpathlineto{\pgfqpoint{2.509740in}{0.811918in}}%
\pgfpathlineto{\pgfqpoint{2.510036in}{0.811395in}}%
\pgfpathlineto{\pgfqpoint{2.510332in}{0.809814in}}%
\pgfpathlineto{\pgfqpoint{2.510628in}{0.807992in}}%
\pgfpathlineto{\pgfqpoint{2.510924in}{0.807959in}}%
\pgfpathlineto{\pgfqpoint{2.511220in}{0.807925in}}%
\pgfpathlineto{\pgfqpoint{2.511516in}{0.807892in}}%
\pgfpathlineto{\pgfqpoint{2.511812in}{0.807859in}}%
\pgfpathlineto{\pgfqpoint{2.512108in}{0.807826in}}%
\pgfpathlineto{\pgfqpoint{2.512404in}{0.807792in}}%
\pgfpathlineto{\pgfqpoint{2.512700in}{0.807759in}}%
\pgfpathlineto{\pgfqpoint{2.512996in}{0.807726in}}%
\pgfpathlineto{\pgfqpoint{2.513292in}{0.807693in}}%
\pgfpathlineto{\pgfqpoint{2.513588in}{0.807659in}}%
\pgfpathlineto{\pgfqpoint{2.513884in}{0.807626in}}%
\pgfpathlineto{\pgfqpoint{2.514180in}{0.807593in}}%
\pgfpathlineto{\pgfqpoint{2.514476in}{0.807560in}}%
\pgfpathlineto{\pgfqpoint{2.514772in}{0.807527in}}%
\pgfpathlineto{\pgfqpoint{2.515068in}{0.807492in}}%
\pgfpathlineto{\pgfqpoint{2.515364in}{0.807455in}}%
\pgfpathlineto{\pgfqpoint{2.515660in}{0.807419in}}%
\pgfpathlineto{\pgfqpoint{2.515956in}{0.807382in}}%
\pgfpathlineto{\pgfqpoint{2.516252in}{0.807345in}}%
\pgfpathlineto{\pgfqpoint{2.516548in}{0.807308in}}%
\pgfpathlineto{\pgfqpoint{2.516844in}{0.807272in}}%
\pgfpathlineto{\pgfqpoint{2.517140in}{0.807235in}}%
\pgfpathlineto{\pgfqpoint{2.517436in}{0.807198in}}%
\pgfpathlineto{\pgfqpoint{2.517732in}{0.807162in}}%
\pgfpathlineto{\pgfqpoint{2.518028in}{0.807125in}}%
\pgfpathlineto{\pgfqpoint{2.518324in}{0.807088in}}%
\pgfpathlineto{\pgfqpoint{2.518620in}{0.807052in}}%
\pgfpathlineto{\pgfqpoint{2.518916in}{0.807015in}}%
\pgfpathlineto{\pgfqpoint{2.519212in}{0.806978in}}%
\pgfpathlineto{\pgfqpoint{2.519508in}{0.806942in}}%
\pgfpathlineto{\pgfqpoint{2.519804in}{0.806905in}}%
\pgfpathlineto{\pgfqpoint{2.520100in}{0.806868in}}%
\pgfpathlineto{\pgfqpoint{2.520396in}{0.806832in}}%
\pgfpathlineto{\pgfqpoint{2.520692in}{0.806795in}}%
\pgfpathlineto{\pgfqpoint{2.520988in}{0.806758in}}%
\pgfpathlineto{\pgfqpoint{2.521284in}{0.806722in}}%
\pgfpathlineto{\pgfqpoint{2.521580in}{0.806690in}}%
\pgfpathlineto{\pgfqpoint{2.521876in}{0.806686in}}%
\pgfpathlineto{\pgfqpoint{2.522172in}{0.806685in}}%
\pgfpathlineto{\pgfqpoint{2.522468in}{0.806684in}}%
\pgfpathlineto{\pgfqpoint{2.522764in}{0.806683in}}%
\pgfpathlineto{\pgfqpoint{2.523060in}{0.806682in}}%
\pgfpathlineto{\pgfqpoint{2.523356in}{0.806681in}}%
\pgfpathlineto{\pgfqpoint{2.523652in}{0.806680in}}%
\pgfpathlineto{\pgfqpoint{2.523948in}{0.806678in}}%
\pgfpathlineto{\pgfqpoint{2.524244in}{0.806677in}}%
\pgfpathlineto{\pgfqpoint{2.524540in}{0.806676in}}%
\pgfpathlineto{\pgfqpoint{2.524836in}{0.806675in}}%
\pgfpathlineto{\pgfqpoint{2.525132in}{0.806674in}}%
\pgfpathlineto{\pgfqpoint{2.525428in}{0.806673in}}%
\pgfpathlineto{\pgfqpoint{2.525724in}{0.806672in}}%
\pgfpathlineto{\pgfqpoint{2.526020in}{0.806671in}}%
\pgfpathlineto{\pgfqpoint{2.526316in}{0.806670in}}%
\pgfpathlineto{\pgfqpoint{2.526612in}{0.806669in}}%
\pgfpathlineto{\pgfqpoint{2.526908in}{0.806668in}}%
\pgfpathlineto{\pgfqpoint{2.527204in}{0.806667in}}%
\pgfpathlineto{\pgfqpoint{2.527500in}{0.806666in}}%
\pgfpathlineto{\pgfqpoint{2.527796in}{0.806665in}}%
\pgfpathlineto{\pgfqpoint{2.528092in}{0.806664in}}%
\pgfpathlineto{\pgfqpoint{2.528388in}{0.806663in}}%
\pgfpathlineto{\pgfqpoint{2.528684in}{0.806662in}}%
\pgfpathlineto{\pgfqpoint{2.528980in}{0.806660in}}%
\pgfpathlineto{\pgfqpoint{2.529276in}{0.806659in}}%
\pgfpathlineto{\pgfqpoint{2.529572in}{0.806658in}}%
\pgfpathlineto{\pgfqpoint{2.529868in}{0.806657in}}%
\pgfpathlineto{\pgfqpoint{2.530164in}{0.806656in}}%
\pgfpathlineto{\pgfqpoint{2.530460in}{0.806655in}}%
\pgfpathlineto{\pgfqpoint{2.530756in}{0.806654in}}%
\pgfpathlineto{\pgfqpoint{2.531052in}{0.806653in}}%
\pgfpathlineto{\pgfqpoint{2.531348in}{0.806652in}}%
\pgfpathlineto{\pgfqpoint{2.531644in}{0.806651in}}%
\pgfpathlineto{\pgfqpoint{2.531940in}{0.806650in}}%
\pgfpathlineto{\pgfqpoint{2.532236in}{0.806649in}}%
\pgfpathlineto{\pgfqpoint{2.532532in}{0.806648in}}%
\pgfpathlineto{\pgfqpoint{2.532828in}{0.806647in}}%
\pgfpathlineto{\pgfqpoint{2.533124in}{0.806646in}}%
\pgfpathlineto{\pgfqpoint{2.533420in}{0.806645in}}%
\pgfpathlineto{\pgfqpoint{2.533716in}{0.806644in}}%
\pgfpathlineto{\pgfqpoint{2.534012in}{0.806642in}}%
\pgfpathlineto{\pgfqpoint{2.534308in}{0.806641in}}%
\pgfpathlineto{\pgfqpoint{2.534604in}{0.806640in}}%
\pgfpathlineto{\pgfqpoint{2.534900in}{0.806639in}}%
\pgfpathlineto{\pgfqpoint{2.535196in}{0.806638in}}%
\pgfpathlineto{\pgfqpoint{2.535492in}{0.806637in}}%
\pgfpathlineto{\pgfqpoint{2.535789in}{0.806636in}}%
\pgfpathlineto{\pgfqpoint{2.536085in}{0.806635in}}%
\pgfpathlineto{\pgfqpoint{2.536381in}{0.806634in}}%
\pgfpathlineto{\pgfqpoint{2.536677in}{0.806633in}}%
\pgfpathlineto{\pgfqpoint{2.536973in}{0.806632in}}%
\pgfpathlineto{\pgfqpoint{2.537269in}{0.806631in}}%
\pgfpathlineto{\pgfqpoint{2.537565in}{0.806630in}}%
\pgfpathlineto{\pgfqpoint{2.537861in}{0.806629in}}%
\pgfpathlineto{\pgfqpoint{2.538157in}{0.806628in}}%
\pgfpathlineto{\pgfqpoint{2.538453in}{0.806627in}}%
\pgfpathlineto{\pgfqpoint{2.538749in}{0.806626in}}%
\pgfpathlineto{\pgfqpoint{2.539045in}{0.806624in}}%
\pgfpathlineto{\pgfqpoint{2.539341in}{0.806623in}}%
\pgfpathlineto{\pgfqpoint{2.539637in}{0.806622in}}%
\pgfpathlineto{\pgfqpoint{2.539933in}{0.806621in}}%
\pgfpathlineto{\pgfqpoint{2.540229in}{0.806620in}}%
\pgfpathlineto{\pgfqpoint{2.540525in}{0.806619in}}%
\pgfpathlineto{\pgfqpoint{2.540821in}{0.806618in}}%
\pgfpathlineto{\pgfqpoint{2.541117in}{0.806617in}}%
\pgfpathlineto{\pgfqpoint{2.541413in}{0.806616in}}%
\pgfpathlineto{\pgfqpoint{2.541709in}{0.806615in}}%
\pgfpathlineto{\pgfqpoint{2.542005in}{0.806614in}}%
\pgfpathlineto{\pgfqpoint{2.542301in}{0.806613in}}%
\pgfpathlineto{\pgfqpoint{2.542597in}{0.806612in}}%
\pgfpathlineto{\pgfqpoint{2.542893in}{0.806611in}}%
\pgfpathlineto{\pgfqpoint{2.543189in}{0.806610in}}%
\pgfpathlineto{\pgfqpoint{2.543485in}{0.806609in}}%
\pgfpathlineto{\pgfqpoint{2.543781in}{0.806608in}}%
\pgfpathlineto{\pgfqpoint{2.544077in}{0.806606in}}%
\pgfpathlineto{\pgfqpoint{2.544373in}{0.806605in}}%
\pgfpathlineto{\pgfqpoint{2.544669in}{0.806604in}}%
\pgfpathlineto{\pgfqpoint{2.544965in}{0.806603in}}%
\pgfpathlineto{\pgfqpoint{2.545261in}{0.806602in}}%
\pgfpathlineto{\pgfqpoint{2.545557in}{0.806601in}}%
\pgfpathlineto{\pgfqpoint{2.545853in}{0.806600in}}%
\pgfpathlineto{\pgfqpoint{2.546149in}{0.806599in}}%
\pgfpathlineto{\pgfqpoint{2.546445in}{0.806598in}}%
\pgfpathlineto{\pgfqpoint{2.546741in}{0.806597in}}%
\pgfpathlineto{\pgfqpoint{2.547037in}{0.806596in}}%
\pgfpathlineto{\pgfqpoint{2.547333in}{0.806595in}}%
\pgfpathlineto{\pgfqpoint{2.547629in}{0.806594in}}%
\pgfpathlineto{\pgfqpoint{2.547925in}{0.806593in}}%
\pgfpathlineto{\pgfqpoint{2.548221in}{0.806592in}}%
\pgfpathlineto{\pgfqpoint{2.548517in}{0.806591in}}%
\pgfpathlineto{\pgfqpoint{2.548813in}{0.806590in}}%
\pgfpathlineto{\pgfqpoint{2.549109in}{0.806588in}}%
\pgfpathlineto{\pgfqpoint{2.549405in}{0.806587in}}%
\pgfpathlineto{\pgfqpoint{2.549701in}{0.806586in}}%
\pgfpathlineto{\pgfqpoint{2.549997in}{0.806585in}}%
\pgfpathlineto{\pgfqpoint{2.550293in}{0.806584in}}%
\pgfpathlineto{\pgfqpoint{2.550589in}{0.806583in}}%
\pgfpathlineto{\pgfqpoint{2.550885in}{0.806582in}}%
\pgfpathlineto{\pgfqpoint{2.551181in}{0.806581in}}%
\pgfpathlineto{\pgfqpoint{2.551477in}{0.806580in}}%
\pgfpathlineto{\pgfqpoint{2.551773in}{0.806579in}}%
\pgfpathlineto{\pgfqpoint{2.552069in}{0.806578in}}%
\pgfpathlineto{\pgfqpoint{2.552365in}{0.806577in}}%
\pgfpathlineto{\pgfqpoint{2.552661in}{0.806576in}}%
\pgfpathlineto{\pgfqpoint{2.552957in}{0.806575in}}%
\pgfpathlineto{\pgfqpoint{2.553253in}{0.806574in}}%
\pgfpathlineto{\pgfqpoint{2.553549in}{0.806573in}}%
\pgfpathlineto{\pgfqpoint{2.553845in}{0.806572in}}%
\pgfpathlineto{\pgfqpoint{2.554141in}{0.806570in}}%
\pgfpathlineto{\pgfqpoint{2.554437in}{0.806569in}}%
\pgfpathlineto{\pgfqpoint{2.554733in}{0.806568in}}%
\pgfpathlineto{\pgfqpoint{2.555029in}{0.806567in}}%
\pgfpathlineto{\pgfqpoint{2.555325in}{0.806566in}}%
\pgfpathlineto{\pgfqpoint{2.555621in}{0.806565in}}%
\pgfpathlineto{\pgfqpoint{2.555917in}{0.806564in}}%
\pgfpathlineto{\pgfqpoint{2.556213in}{0.806563in}}%
\pgfpathlineto{\pgfqpoint{2.556509in}{0.806562in}}%
\pgfpathlineto{\pgfqpoint{2.556805in}{0.806561in}}%
\pgfpathlineto{\pgfqpoint{2.557101in}{0.806703in}}%
\pgfpathlineto{\pgfqpoint{2.557397in}{0.806714in}}%
\pgfpathlineto{\pgfqpoint{2.557693in}{0.806713in}}%
\pgfpathlineto{\pgfqpoint{2.557989in}{0.806712in}}%
\pgfpathlineto{\pgfqpoint{2.558285in}{0.806710in}}%
\pgfpathlineto{\pgfqpoint{2.558581in}{0.806709in}}%
\pgfpathlineto{\pgfqpoint{2.558877in}{0.806708in}}%
\pgfpathlineto{\pgfqpoint{2.559173in}{0.806707in}}%
\pgfpathlineto{\pgfqpoint{2.559469in}{0.806706in}}%
\pgfpathlineto{\pgfqpoint{2.559765in}{0.806705in}}%
\pgfpathlineto{\pgfqpoint{2.560061in}{0.806704in}}%
\pgfpathlineto{\pgfqpoint{2.560357in}{0.806703in}}%
\pgfpathlineto{\pgfqpoint{2.560653in}{0.806702in}}%
\pgfpathlineto{\pgfqpoint{2.560949in}{0.806701in}}%
\pgfpathlineto{\pgfqpoint{2.561245in}{0.806700in}}%
\pgfpathlineto{\pgfqpoint{2.561541in}{0.806699in}}%
\pgfpathlineto{\pgfqpoint{2.561837in}{0.806698in}}%
\pgfpathlineto{\pgfqpoint{2.562133in}{0.806697in}}%
\pgfpathlineto{\pgfqpoint{2.562429in}{0.806696in}}%
\pgfpathlineto{\pgfqpoint{2.562725in}{0.806695in}}%
\pgfpathlineto{\pgfqpoint{2.563021in}{0.806694in}}%
\pgfpathlineto{\pgfqpoint{2.563317in}{0.806692in}}%
\pgfpathlineto{\pgfqpoint{2.563613in}{0.806691in}}%
\pgfpathlineto{\pgfqpoint{2.563909in}{0.806690in}}%
\pgfpathlineto{\pgfqpoint{2.564205in}{0.806689in}}%
\pgfpathlineto{\pgfqpoint{2.564501in}{0.806688in}}%
\pgfpathlineto{\pgfqpoint{2.564797in}{0.806687in}}%
\pgfpathlineto{\pgfqpoint{2.565093in}{0.806686in}}%
\pgfpathlineto{\pgfqpoint{2.565389in}{0.806685in}}%
\pgfpathlineto{\pgfqpoint{2.565685in}{0.806684in}}%
\pgfpathlineto{\pgfqpoint{2.565981in}{0.806683in}}%
\pgfpathlineto{\pgfqpoint{2.566277in}{0.806682in}}%
\pgfpathlineto{\pgfqpoint{2.566573in}{0.806681in}}%
\pgfpathlineto{\pgfqpoint{2.566869in}{0.806680in}}%
\pgfpathlineto{\pgfqpoint{2.567165in}{0.806679in}}%
\pgfpathlineto{\pgfqpoint{2.567461in}{0.806678in}}%
\pgfpathlineto{\pgfqpoint{2.567757in}{0.806677in}}%
\pgfpathlineto{\pgfqpoint{2.568053in}{0.806676in}}%
\pgfpathlineto{\pgfqpoint{2.568349in}{0.806674in}}%
\pgfpathlineto{\pgfqpoint{2.568645in}{0.806673in}}%
\pgfpathlineto{\pgfqpoint{2.568941in}{0.806672in}}%
\pgfpathlineto{\pgfqpoint{2.569237in}{0.806671in}}%
\pgfpathlineto{\pgfqpoint{2.569533in}{0.806670in}}%
\pgfpathlineto{\pgfqpoint{2.569829in}{0.806669in}}%
\pgfpathlineto{\pgfqpoint{2.570125in}{0.806668in}}%
\pgfpathlineto{\pgfqpoint{2.570421in}{0.806667in}}%
\pgfpathlineto{\pgfqpoint{2.570717in}{0.806666in}}%
\pgfpathlineto{\pgfqpoint{2.571013in}{0.806665in}}%
\pgfpathlineto{\pgfqpoint{2.571309in}{0.806664in}}%
\pgfpathlineto{\pgfqpoint{2.571605in}{0.806663in}}%
\pgfpathlineto{\pgfqpoint{2.571901in}{0.806662in}}%
\pgfpathlineto{\pgfqpoint{2.572197in}{0.806661in}}%
\pgfpathlineto{\pgfqpoint{2.572493in}{0.806660in}}%
\pgfpathlineto{\pgfqpoint{2.572789in}{0.806659in}}%
\pgfpathlineto{\pgfqpoint{2.573085in}{0.806658in}}%
\pgfpathlineto{\pgfqpoint{2.573381in}{0.806656in}}%
\pgfpathlineto{\pgfqpoint{2.573677in}{0.806655in}}%
\pgfpathlineto{\pgfqpoint{2.573973in}{0.806654in}}%
\pgfpathlineto{\pgfqpoint{2.574269in}{0.806653in}}%
\pgfpathlineto{\pgfqpoint{2.574565in}{0.806652in}}%
\pgfpathlineto{\pgfqpoint{2.574861in}{0.806651in}}%
\pgfpathlineto{\pgfqpoint{2.575157in}{0.806650in}}%
\pgfpathlineto{\pgfqpoint{2.575453in}{0.806649in}}%
\pgfpathlineto{\pgfqpoint{2.575749in}{0.806648in}}%
\pgfpathlineto{\pgfqpoint{2.576045in}{0.806647in}}%
\pgfpathlineto{\pgfqpoint{2.576341in}{0.806646in}}%
\pgfpathlineto{\pgfqpoint{2.576637in}{0.806645in}}%
\pgfpathlineto{\pgfqpoint{2.576933in}{0.806644in}}%
\pgfpathlineto{\pgfqpoint{2.577229in}{0.806643in}}%
\pgfpathlineto{\pgfqpoint{2.577525in}{0.806642in}}%
\pgfpathlineto{\pgfqpoint{2.577821in}{0.806641in}}%
\pgfpathlineto{\pgfqpoint{2.578117in}{0.806640in}}%
\pgfpathlineto{\pgfqpoint{2.578413in}{0.806638in}}%
\pgfpathlineto{\pgfqpoint{2.578709in}{0.806637in}}%
\pgfpathlineto{\pgfqpoint{2.579005in}{0.806636in}}%
\pgfpathlineto{\pgfqpoint{2.579301in}{0.806635in}}%
\pgfpathlineto{\pgfqpoint{2.579597in}{0.806634in}}%
\pgfpathlineto{\pgfqpoint{2.579893in}{0.806633in}}%
\pgfpathlineto{\pgfqpoint{2.580189in}{0.806632in}}%
\pgfpathlineto{\pgfqpoint{2.580485in}{0.806631in}}%
\pgfpathlineto{\pgfqpoint{2.580781in}{0.806630in}}%
\pgfpathlineto{\pgfqpoint{2.581077in}{0.806629in}}%
\pgfpathlineto{\pgfqpoint{2.581373in}{0.806628in}}%
\pgfpathlineto{\pgfqpoint{2.581669in}{0.806627in}}%
\pgfpathlineto{\pgfqpoint{2.581965in}{0.806626in}}%
\pgfpathlineto{\pgfqpoint{2.582261in}{0.806625in}}%
\pgfpathlineto{\pgfqpoint{2.582557in}{0.806624in}}%
\pgfpathlineto{\pgfqpoint{2.582853in}{0.806623in}}%
\pgfpathlineto{\pgfqpoint{2.583149in}{0.806622in}}%
\pgfpathlineto{\pgfqpoint{2.583445in}{0.806620in}}%
\pgfpathlineto{\pgfqpoint{2.583741in}{0.806619in}}%
\pgfpathlineto{\pgfqpoint{2.584037in}{0.806618in}}%
\pgfpathlineto{\pgfqpoint{2.584333in}{0.806617in}}%
\pgfpathlineto{\pgfqpoint{2.584629in}{0.806616in}}%
\pgfpathlineto{\pgfqpoint{2.584925in}{0.806615in}}%
\pgfpathlineto{\pgfqpoint{2.585221in}{0.806614in}}%
\pgfpathlineto{\pgfqpoint{2.585517in}{0.806613in}}%
\pgfpathlineto{\pgfqpoint{2.585813in}{0.806612in}}%
\pgfpathlineto{\pgfqpoint{2.586109in}{0.806611in}}%
\pgfpathlineto{\pgfqpoint{2.586405in}{0.806610in}}%
\pgfpathlineto{\pgfqpoint{2.586701in}{0.806609in}}%
\pgfpathlineto{\pgfqpoint{2.586997in}{0.806608in}}%
\pgfpathlineto{\pgfqpoint{2.587293in}{0.806607in}}%
\pgfpathlineto{\pgfqpoint{2.587589in}{0.806606in}}%
\pgfpathlineto{\pgfqpoint{2.587885in}{0.806605in}}%
\pgfpathlineto{\pgfqpoint{2.588181in}{0.806604in}}%
\pgfpathlineto{\pgfqpoint{2.588477in}{0.806602in}}%
\pgfpathlineto{\pgfqpoint{2.588773in}{0.806601in}}%
\pgfpathlineto{\pgfqpoint{2.589069in}{0.806600in}}%
\pgfpathlineto{\pgfqpoint{2.589365in}{0.806599in}}%
\pgfpathlineto{\pgfqpoint{2.589661in}{0.806598in}}%
\pgfpathlineto{\pgfqpoint{2.589957in}{0.806597in}}%
\pgfpathlineto{\pgfqpoint{2.590253in}{0.806596in}}%
\pgfpathlineto{\pgfqpoint{2.590549in}{0.806595in}}%
\pgfpathlineto{\pgfqpoint{2.590845in}{0.806594in}}%
\pgfpathlineto{\pgfqpoint{2.591141in}{0.806593in}}%
\pgfpathlineto{\pgfqpoint{2.591437in}{0.806592in}}%
\pgfpathlineto{\pgfqpoint{2.591733in}{0.806591in}}%
\pgfpathlineto{\pgfqpoint{2.592029in}{0.806590in}}%
\pgfpathlineto{\pgfqpoint{2.592325in}{0.806589in}}%
\pgfpathlineto{\pgfqpoint{2.592621in}{0.806588in}}%
\pgfpathlineto{\pgfqpoint{2.592917in}{0.806587in}}%
\pgfpathlineto{\pgfqpoint{2.593213in}{0.806586in}}%
\pgfpathlineto{\pgfqpoint{2.593509in}{0.806584in}}%
\pgfpathlineto{\pgfqpoint{2.593805in}{0.806583in}}%
\pgfpathlineto{\pgfqpoint{2.594101in}{0.806582in}}%
\pgfpathlineto{\pgfqpoint{2.594397in}{0.806581in}}%
\pgfpathlineto{\pgfqpoint{2.594693in}{0.806580in}}%
\pgfpathlineto{\pgfqpoint{2.594989in}{0.806579in}}%
\pgfpathlineto{\pgfqpoint{2.595285in}{0.806578in}}%
\pgfpathlineto{\pgfqpoint{2.595581in}{0.806577in}}%
\pgfpathlineto{\pgfqpoint{2.595877in}{0.806576in}}%
\pgfpathlineto{\pgfqpoint{2.596173in}{0.806575in}}%
\pgfpathlineto{\pgfqpoint{2.596469in}{0.806574in}}%
\pgfpathlineto{\pgfqpoint{2.596765in}{0.806573in}}%
\pgfpathlineto{\pgfqpoint{2.597061in}{0.806572in}}%
\pgfpathlineto{\pgfqpoint{2.597357in}{0.806571in}}%
\pgfpathlineto{\pgfqpoint{2.597653in}{0.806570in}}%
\pgfpathlineto{\pgfqpoint{2.597949in}{0.806569in}}%
\pgfpathlineto{\pgfqpoint{2.598245in}{0.806568in}}%
\pgfpathlineto{\pgfqpoint{2.598541in}{0.806566in}}%
\pgfpathlineto{\pgfqpoint{2.598837in}{0.806565in}}%
\pgfpathlineto{\pgfqpoint{2.599133in}{0.806564in}}%
\pgfpathlineto{\pgfqpoint{2.599429in}{0.806563in}}%
\pgfpathlineto{\pgfqpoint{2.599725in}{0.806562in}}%
\pgfpathlineto{\pgfqpoint{2.600021in}{0.806561in}}%
\pgfpathlineto{\pgfqpoint{2.600317in}{0.806560in}}%
\pgfpathlineto{\pgfqpoint{2.600613in}{0.806559in}}%
\pgfpathlineto{\pgfqpoint{2.600909in}{0.806558in}}%
\pgfpathlineto{\pgfqpoint{2.601205in}{0.806557in}}%
\pgfpathlineto{\pgfqpoint{2.601501in}{0.806556in}}%
\pgfpathlineto{\pgfqpoint{2.601797in}{0.806555in}}%
\pgfpathlineto{\pgfqpoint{2.602093in}{0.806554in}}%
\pgfpathlineto{\pgfqpoint{2.602389in}{0.806553in}}%
\pgfpathlineto{\pgfqpoint{2.602685in}{0.806552in}}%
\pgfpathlineto{\pgfqpoint{2.602981in}{0.806551in}}%
\pgfpathlineto{\pgfqpoint{2.603278in}{0.806550in}}%
\pgfpathlineto{\pgfqpoint{2.603574in}{0.806548in}}%
\pgfpathlineto{\pgfqpoint{2.603870in}{0.806547in}}%
\pgfpathlineto{\pgfqpoint{2.604166in}{0.806546in}}%
\pgfpathlineto{\pgfqpoint{2.604462in}{0.806545in}}%
\pgfpathlineto{\pgfqpoint{2.604758in}{0.806544in}}%
\pgfpathlineto{\pgfqpoint{2.605054in}{0.806543in}}%
\pgfpathlineto{\pgfqpoint{2.605350in}{0.806542in}}%
\pgfpathlineto{\pgfqpoint{2.605646in}{0.806541in}}%
\pgfpathlineto{\pgfqpoint{2.605942in}{0.806540in}}%
\pgfpathlineto{\pgfqpoint{2.606238in}{0.806539in}}%
\pgfpathlineto{\pgfqpoint{2.606534in}{0.806538in}}%
\pgfpathlineto{\pgfqpoint{2.606830in}{0.806537in}}%
\pgfpathlineto{\pgfqpoint{2.607126in}{0.806536in}}%
\pgfpathlineto{\pgfqpoint{2.607422in}{0.806535in}}%
\pgfpathlineto{\pgfqpoint{2.607718in}{0.806534in}}%
\pgfpathlineto{\pgfqpoint{2.608014in}{0.806533in}}%
\pgfpathlineto{\pgfqpoint{2.608310in}{0.806532in}}%
\pgfpathlineto{\pgfqpoint{2.608606in}{0.806530in}}%
\pgfpathlineto{\pgfqpoint{2.608902in}{0.806529in}}%
\pgfpathlineto{\pgfqpoint{2.609198in}{0.806528in}}%
\pgfpathlineto{\pgfqpoint{2.609494in}{0.806527in}}%
\pgfpathlineto{\pgfqpoint{2.609790in}{0.806526in}}%
\pgfpathlineto{\pgfqpoint{2.610086in}{0.806525in}}%
\pgfpathlineto{\pgfqpoint{2.610382in}{0.806524in}}%
\pgfpathlineto{\pgfqpoint{2.610678in}{0.806523in}}%
\pgfpathlineto{\pgfqpoint{2.610974in}{0.806522in}}%
\pgfpathlineto{\pgfqpoint{2.611270in}{0.806521in}}%
\pgfpathlineto{\pgfqpoint{2.611566in}{0.806520in}}%
\pgfpathlineto{\pgfqpoint{2.611862in}{0.806519in}}%
\pgfpathlineto{\pgfqpoint{2.612158in}{0.806518in}}%
\pgfpathlineto{\pgfqpoint{2.612454in}{0.806517in}}%
\pgfpathlineto{\pgfqpoint{2.612750in}{0.806516in}}%
\pgfpathlineto{\pgfqpoint{2.613046in}{0.806515in}}%
\pgfpathlineto{\pgfqpoint{2.613342in}{0.806514in}}%
\pgfpathlineto{\pgfqpoint{2.613638in}{0.806512in}}%
\pgfpathlineto{\pgfqpoint{2.613934in}{0.806511in}}%
\pgfpathlineto{\pgfqpoint{2.614230in}{0.806510in}}%
\pgfpathlineto{\pgfqpoint{2.614526in}{0.806509in}}%
\pgfpathlineto{\pgfqpoint{2.614822in}{0.806508in}}%
\pgfpathlineto{\pgfqpoint{2.615118in}{0.806507in}}%
\pgfpathlineto{\pgfqpoint{2.615414in}{0.806506in}}%
\pgfpathlineto{\pgfqpoint{2.615710in}{0.806505in}}%
\pgfpathlineto{\pgfqpoint{2.616006in}{0.806504in}}%
\pgfpathlineto{\pgfqpoint{2.616302in}{0.806503in}}%
\pgfpathlineto{\pgfqpoint{2.616598in}{0.806502in}}%
\pgfpathlineto{\pgfqpoint{2.616894in}{0.806501in}}%
\pgfpathlineto{\pgfqpoint{2.617190in}{0.806500in}}%
\pgfpathlineto{\pgfqpoint{2.617486in}{0.806499in}}%
\pgfpathlineto{\pgfqpoint{2.617782in}{0.806498in}}%
\pgfpathlineto{\pgfqpoint{2.618078in}{0.806497in}}%
\pgfpathlineto{\pgfqpoint{2.618374in}{0.806496in}}%
\pgfpathlineto{\pgfqpoint{2.618670in}{0.806494in}}%
\pgfpathlineto{\pgfqpoint{2.618966in}{0.806493in}}%
\pgfpathlineto{\pgfqpoint{2.619262in}{0.806492in}}%
\pgfpathlineto{\pgfqpoint{2.619558in}{0.806491in}}%
\pgfpathlineto{\pgfqpoint{2.619854in}{0.806490in}}%
\pgfpathlineto{\pgfqpoint{2.620150in}{0.806489in}}%
\pgfpathlineto{\pgfqpoint{2.620446in}{0.806488in}}%
\pgfpathlineto{\pgfqpoint{2.620742in}{0.806487in}}%
\pgfpathlineto{\pgfqpoint{2.621038in}{0.806486in}}%
\pgfpathlineto{\pgfqpoint{2.621334in}{0.806485in}}%
\pgfpathlineto{\pgfqpoint{2.621630in}{0.806484in}}%
\pgfpathlineto{\pgfqpoint{2.621926in}{0.806483in}}%
\pgfpathlineto{\pgfqpoint{2.622222in}{0.806482in}}%
\pgfpathlineto{\pgfqpoint{2.622518in}{0.806481in}}%
\pgfpathlineto{\pgfqpoint{2.622814in}{0.806480in}}%
\pgfpathlineto{\pgfqpoint{2.623110in}{0.806479in}}%
\pgfpathlineto{\pgfqpoint{2.623406in}{0.806478in}}%
\pgfpathlineto{\pgfqpoint{2.623702in}{0.806476in}}%
\pgfpathlineto{\pgfqpoint{2.623998in}{0.806475in}}%
\pgfpathlineto{\pgfqpoint{2.624294in}{0.806474in}}%
\pgfpathlineto{\pgfqpoint{2.624590in}{0.806473in}}%
\pgfpathlineto{\pgfqpoint{2.624886in}{0.806472in}}%
\pgfpathlineto{\pgfqpoint{2.625182in}{0.806471in}}%
\pgfpathlineto{\pgfqpoint{2.625478in}{0.806470in}}%
\pgfpathlineto{\pgfqpoint{2.625774in}{0.806469in}}%
\pgfpathlineto{\pgfqpoint{2.626070in}{0.806468in}}%
\pgfpathlineto{\pgfqpoint{2.626366in}{0.806467in}}%
\pgfpathlineto{\pgfqpoint{2.626662in}{0.806466in}}%
\pgfpathlineto{\pgfqpoint{2.626958in}{0.806465in}}%
\pgfpathlineto{\pgfqpoint{2.627254in}{0.806464in}}%
\pgfpathlineto{\pgfqpoint{2.627550in}{0.806463in}}%
\pgfpathlineto{\pgfqpoint{2.627846in}{0.806462in}}%
\pgfpathlineto{\pgfqpoint{2.628142in}{0.806461in}}%
\pgfpathlineto{\pgfqpoint{2.628438in}{0.806460in}}%
\pgfpathlineto{\pgfqpoint{2.628734in}{0.806458in}}%
\pgfpathlineto{\pgfqpoint{2.629030in}{0.806457in}}%
\pgfpathlineto{\pgfqpoint{2.629326in}{0.806456in}}%
\pgfpathlineto{\pgfqpoint{2.629622in}{0.806455in}}%
\pgfpathlineto{\pgfqpoint{2.629918in}{0.806454in}}%
\pgfpathlineto{\pgfqpoint{2.630214in}{0.806453in}}%
\pgfpathlineto{\pgfqpoint{2.630510in}{0.806452in}}%
\pgfpathlineto{\pgfqpoint{2.630806in}{0.806451in}}%
\pgfpathlineto{\pgfqpoint{2.631102in}{0.806450in}}%
\pgfpathlineto{\pgfqpoint{2.631398in}{0.806449in}}%
\pgfpathlineto{\pgfqpoint{2.631694in}{0.806448in}}%
\pgfpathlineto{\pgfqpoint{2.631990in}{0.806447in}}%
\pgfpathlineto{\pgfqpoint{2.632286in}{0.806446in}}%
\pgfpathlineto{\pgfqpoint{2.632582in}{0.806445in}}%
\pgfpathlineto{\pgfqpoint{2.632878in}{0.806444in}}%
\pgfpathlineto{\pgfqpoint{2.633174in}{0.806443in}}%
\pgfpathlineto{\pgfqpoint{2.633470in}{0.806442in}}%
\pgfpathlineto{\pgfqpoint{2.633766in}{0.806440in}}%
\pgfpathlineto{\pgfqpoint{2.634062in}{0.806439in}}%
\pgfpathlineto{\pgfqpoint{2.634358in}{0.806438in}}%
\pgfpathlineto{\pgfqpoint{2.634654in}{0.806437in}}%
\pgfpathlineto{\pgfqpoint{2.634950in}{0.806436in}}%
\pgfpathlineto{\pgfqpoint{2.635246in}{0.806435in}}%
\pgfpathlineto{\pgfqpoint{2.635542in}{0.806434in}}%
\pgfpathlineto{\pgfqpoint{2.635838in}{0.806433in}}%
\pgfpathlineto{\pgfqpoint{2.636134in}{0.806432in}}%
\pgfpathlineto{\pgfqpoint{2.636430in}{0.806431in}}%
\pgfpathlineto{\pgfqpoint{2.636726in}{0.806430in}}%
\pgfpathlineto{\pgfqpoint{2.637022in}{0.806429in}}%
\pgfpathlineto{\pgfqpoint{2.637318in}{0.806428in}}%
\pgfpathlineto{\pgfqpoint{2.637614in}{0.806427in}}%
\pgfpathlineto{\pgfqpoint{2.637910in}{0.806426in}}%
\pgfpathlineto{\pgfqpoint{2.638206in}{0.806425in}}%
\pgfpathlineto{\pgfqpoint{2.638502in}{0.806424in}}%
\pgfpathlineto{\pgfqpoint{2.638798in}{0.806422in}}%
\pgfpathlineto{\pgfqpoint{2.639094in}{0.806421in}}%
\pgfpathlineto{\pgfqpoint{2.639390in}{0.806420in}}%
\pgfpathlineto{\pgfqpoint{2.639686in}{0.806419in}}%
\pgfpathlineto{\pgfqpoint{2.639982in}{0.806418in}}%
\pgfpathlineto{\pgfqpoint{2.640278in}{0.806417in}}%
\pgfpathlineto{\pgfqpoint{2.640574in}{0.806416in}}%
\pgfpathlineto{\pgfqpoint{2.640870in}{0.806415in}}%
\pgfpathlineto{\pgfqpoint{2.641166in}{0.806414in}}%
\pgfpathlineto{\pgfqpoint{2.641462in}{0.806413in}}%
\pgfpathlineto{\pgfqpoint{2.641758in}{0.806412in}}%
\pgfpathlineto{\pgfqpoint{2.642054in}{0.806411in}}%
\pgfpathlineto{\pgfqpoint{2.642350in}{0.806410in}}%
\pgfpathlineto{\pgfqpoint{2.642646in}{0.806409in}}%
\pgfpathlineto{\pgfqpoint{2.642942in}{0.806408in}}%
\pgfpathlineto{\pgfqpoint{2.643238in}{0.806407in}}%
\pgfpathlineto{\pgfqpoint{2.643534in}{0.806406in}}%
\pgfpathlineto{\pgfqpoint{2.643830in}{0.806404in}}%
\pgfpathlineto{\pgfqpoint{2.644126in}{0.806403in}}%
\pgfpathlineto{\pgfqpoint{2.644422in}{0.806402in}}%
\pgfpathlineto{\pgfqpoint{2.644718in}{0.806401in}}%
\pgfpathlineto{\pgfqpoint{2.645014in}{0.806400in}}%
\pgfpathlineto{\pgfqpoint{2.645310in}{0.806399in}}%
\pgfpathlineto{\pgfqpoint{2.645606in}{0.806398in}}%
\pgfpathlineto{\pgfqpoint{2.645902in}{0.806397in}}%
\pgfpathlineto{\pgfqpoint{2.646198in}{0.806396in}}%
\pgfpathlineto{\pgfqpoint{2.646494in}{0.806395in}}%
\pgfpathlineto{\pgfqpoint{2.646790in}{0.806394in}}%
\pgfpathlineto{\pgfqpoint{2.647086in}{0.806393in}}%
\pgfpathlineto{\pgfqpoint{2.647382in}{0.806392in}}%
\pgfpathlineto{\pgfqpoint{2.647678in}{0.806391in}}%
\pgfpathlineto{\pgfqpoint{2.647974in}{0.806390in}}%
\pgfpathlineto{\pgfqpoint{2.648270in}{0.806389in}}%
\pgfpathlineto{\pgfqpoint{2.648566in}{0.806388in}}%
\pgfpathlineto{\pgfqpoint{2.648862in}{0.806386in}}%
\pgfpathlineto{\pgfqpoint{2.649158in}{0.806385in}}%
\pgfpathlineto{\pgfqpoint{2.649454in}{0.806384in}}%
\pgfpathlineto{\pgfqpoint{2.649750in}{0.806383in}}%
\pgfpathlineto{\pgfqpoint{2.650046in}{0.806382in}}%
\pgfpathlineto{\pgfqpoint{2.650342in}{0.806381in}}%
\pgfpathlineto{\pgfqpoint{2.650638in}{0.806380in}}%
\pgfpathlineto{\pgfqpoint{2.650934in}{0.806379in}}%
\pgfpathlineto{\pgfqpoint{2.651230in}{0.806378in}}%
\pgfpathlineto{\pgfqpoint{2.651526in}{0.806377in}}%
\pgfpathlineto{\pgfqpoint{2.651822in}{0.806376in}}%
\pgfpathlineto{\pgfqpoint{2.652118in}{0.806375in}}%
\pgfpathlineto{\pgfqpoint{2.652414in}{0.806374in}}%
\pgfpathlineto{\pgfqpoint{2.652710in}{0.806373in}}%
\pgfpathlineto{\pgfqpoint{2.653006in}{0.806372in}}%
\pgfpathlineto{\pgfqpoint{2.653302in}{0.806371in}}%
\pgfpathlineto{\pgfqpoint{2.653598in}{0.806370in}}%
\pgfpathlineto{\pgfqpoint{2.653894in}{0.806368in}}%
\pgfpathlineto{\pgfqpoint{2.654190in}{0.806367in}}%
\pgfpathlineto{\pgfqpoint{2.654486in}{0.806366in}}%
\pgfpathlineto{\pgfqpoint{2.654782in}{0.806365in}}%
\pgfpathlineto{\pgfqpoint{2.655078in}{0.806364in}}%
\pgfpathlineto{\pgfqpoint{2.655374in}{0.806363in}}%
\pgfpathlineto{\pgfqpoint{2.655670in}{0.806362in}}%
\pgfpathlineto{\pgfqpoint{2.655966in}{0.806361in}}%
\pgfpathlineto{\pgfqpoint{2.656262in}{0.806360in}}%
\pgfpathlineto{\pgfqpoint{2.656558in}{0.806359in}}%
\pgfpathlineto{\pgfqpoint{2.656854in}{0.806358in}}%
\pgfpathlineto{\pgfqpoint{2.657150in}{0.806357in}}%
\pgfpathlineto{\pgfqpoint{2.657446in}{0.806356in}}%
\pgfpathlineto{\pgfqpoint{2.657742in}{0.806355in}}%
\pgfpathlineto{\pgfqpoint{2.658038in}{0.806354in}}%
\pgfpathlineto{\pgfqpoint{2.658334in}{0.806353in}}%
\pgfpathlineto{\pgfqpoint{2.658630in}{0.806352in}}%
\pgfpathlineto{\pgfqpoint{2.658926in}{0.806350in}}%
\pgfpathlineto{\pgfqpoint{2.659222in}{0.806349in}}%
\pgfpathlineto{\pgfqpoint{2.659518in}{0.806348in}}%
\pgfpathlineto{\pgfqpoint{2.659814in}{0.806347in}}%
\pgfpathlineto{\pgfqpoint{2.660110in}{0.806346in}}%
\pgfpathlineto{\pgfqpoint{2.660406in}{0.806345in}}%
\pgfpathlineto{\pgfqpoint{2.660702in}{0.806344in}}%
\pgfpathlineto{\pgfqpoint{2.660998in}{0.806343in}}%
\pgfpathlineto{\pgfqpoint{2.661294in}{0.806342in}}%
\pgfpathlineto{\pgfqpoint{2.661590in}{0.806341in}}%
\pgfpathlineto{\pgfqpoint{2.661886in}{0.806340in}}%
\pgfpathlineto{\pgfqpoint{2.662182in}{0.806339in}}%
\pgfpathlineto{\pgfqpoint{2.662478in}{0.806338in}}%
\pgfpathlineto{\pgfqpoint{2.662774in}{0.806337in}}%
\pgfpathlineto{\pgfqpoint{2.663070in}{0.806336in}}%
\pgfpathlineto{\pgfqpoint{2.663366in}{0.806335in}}%
\pgfpathlineto{\pgfqpoint{2.663662in}{0.806334in}}%
\pgfpathlineto{\pgfqpoint{2.663958in}{0.806332in}}%
\pgfpathlineto{\pgfqpoint{2.664254in}{0.806331in}}%
\pgfpathlineto{\pgfqpoint{2.664550in}{0.806330in}}%
\pgfpathlineto{\pgfqpoint{2.664846in}{0.806329in}}%
\pgfpathlineto{\pgfqpoint{2.665142in}{0.806328in}}%
\pgfpathlineto{\pgfqpoint{2.665438in}{0.806327in}}%
\pgfpathlineto{\pgfqpoint{2.665734in}{0.806326in}}%
\pgfpathlineto{\pgfqpoint{2.666030in}{0.806325in}}%
\pgfpathlineto{\pgfqpoint{2.666326in}{0.806324in}}%
\pgfpathlineto{\pgfqpoint{2.666622in}{0.806323in}}%
\pgfpathlineto{\pgfqpoint{2.666918in}{0.806322in}}%
\pgfpathlineto{\pgfqpoint{2.667214in}{0.806321in}}%
\pgfpathlineto{\pgfqpoint{2.667510in}{0.806320in}}%
\pgfpathlineto{\pgfqpoint{2.667806in}{0.806319in}}%
\pgfpathlineto{\pgfqpoint{2.668102in}{0.806318in}}%
\pgfpathlineto{\pgfqpoint{2.668398in}{0.806317in}}%
\pgfpathlineto{\pgfqpoint{2.668694in}{0.806316in}}%
\pgfpathlineto{\pgfqpoint{2.668990in}{0.806314in}}%
\pgfpathlineto{\pgfqpoint{2.669286in}{0.806313in}}%
\pgfpathlineto{\pgfqpoint{2.669582in}{0.806312in}}%
\pgfpathlineto{\pgfqpoint{2.669878in}{0.806311in}}%
\pgfpathlineto{\pgfqpoint{2.670174in}{0.806310in}}%
\pgfpathlineto{\pgfqpoint{2.670471in}{0.806309in}}%
\pgfpathlineto{\pgfqpoint{2.670767in}{0.806308in}}%
\pgfpathlineto{\pgfqpoint{2.671063in}{0.806307in}}%
\pgfpathlineto{\pgfqpoint{2.671359in}{0.806306in}}%
\pgfpathlineto{\pgfqpoint{2.671655in}{0.806305in}}%
\pgfpathlineto{\pgfqpoint{2.671951in}{0.806304in}}%
\pgfpathlineto{\pgfqpoint{2.672247in}{0.806303in}}%
\pgfpathlineto{\pgfqpoint{2.672543in}{0.806302in}}%
\pgfpathlineto{\pgfqpoint{2.672839in}{0.806301in}}%
\pgfpathlineto{\pgfqpoint{2.673135in}{0.806300in}}%
\pgfpathlineto{\pgfqpoint{2.673431in}{0.806299in}}%
\pgfpathlineto{\pgfqpoint{2.673727in}{0.806298in}}%
\pgfpathlineto{\pgfqpoint{2.674023in}{0.806296in}}%
\pgfpathlineto{\pgfqpoint{2.674319in}{0.806295in}}%
\pgfpathlineto{\pgfqpoint{2.674615in}{0.806294in}}%
\pgfpathlineto{\pgfqpoint{2.674911in}{0.806293in}}%
\pgfpathlineto{\pgfqpoint{2.675207in}{0.806292in}}%
\pgfpathlineto{\pgfqpoint{2.675503in}{0.806291in}}%
\pgfpathlineto{\pgfqpoint{2.675799in}{0.806290in}}%
\pgfpathlineto{\pgfqpoint{2.676095in}{0.806289in}}%
\pgfpathlineto{\pgfqpoint{2.676391in}{0.806288in}}%
\pgfpathlineto{\pgfqpoint{2.676687in}{0.806287in}}%
\pgfpathlineto{\pgfqpoint{2.676983in}{0.806286in}}%
\pgfpathlineto{\pgfqpoint{2.677279in}{0.806285in}}%
\pgfpathlineto{\pgfqpoint{2.677575in}{0.806284in}}%
\pgfpathlineto{\pgfqpoint{2.677871in}{0.806283in}}%
\pgfpathlineto{\pgfqpoint{2.678167in}{0.806282in}}%
\pgfpathlineto{\pgfqpoint{2.678463in}{0.806281in}}%
\pgfpathlineto{\pgfqpoint{2.678759in}{0.806280in}}%
\pgfpathlineto{\pgfqpoint{2.679055in}{0.806278in}}%
\pgfpathlineto{\pgfqpoint{2.679351in}{0.806277in}}%
\pgfpathlineto{\pgfqpoint{2.679647in}{0.806276in}}%
\pgfpathlineto{\pgfqpoint{2.679943in}{0.806275in}}%
\pgfpathlineto{\pgfqpoint{2.680239in}{0.806274in}}%
\pgfpathlineto{\pgfqpoint{2.680535in}{0.806273in}}%
\pgfpathlineto{\pgfqpoint{2.680831in}{0.806272in}}%
\pgfpathlineto{\pgfqpoint{2.681127in}{0.806271in}}%
\pgfpathlineto{\pgfqpoint{2.681423in}{0.806270in}}%
\pgfpathlineto{\pgfqpoint{2.681719in}{0.806269in}}%
\pgfpathlineto{\pgfqpoint{2.682015in}{0.806268in}}%
\pgfpathlineto{\pgfqpoint{2.682311in}{0.806267in}}%
\pgfpathlineto{\pgfqpoint{2.682607in}{0.806266in}}%
\pgfpathlineto{\pgfqpoint{2.682903in}{0.806265in}}%
\pgfpathlineto{\pgfqpoint{2.683199in}{0.806264in}}%
\pgfpathlineto{\pgfqpoint{2.683495in}{0.806263in}}%
\pgfpathlineto{\pgfqpoint{2.683791in}{0.806262in}}%
\pgfpathlineto{\pgfqpoint{2.684087in}{0.806260in}}%
\pgfpathlineto{\pgfqpoint{2.684383in}{0.806259in}}%
\pgfpathlineto{\pgfqpoint{2.684679in}{0.806258in}}%
\pgfpathlineto{\pgfqpoint{2.684975in}{0.806257in}}%
\pgfpathlineto{\pgfqpoint{2.685271in}{0.806256in}}%
\pgfpathlineto{\pgfqpoint{2.685567in}{0.806255in}}%
\pgfpathlineto{\pgfqpoint{2.685863in}{0.806254in}}%
\pgfpathlineto{\pgfqpoint{2.686159in}{0.806253in}}%
\pgfpathlineto{\pgfqpoint{2.686455in}{0.806252in}}%
\pgfpathlineto{\pgfqpoint{2.686751in}{0.806251in}}%
\pgfpathlineto{\pgfqpoint{2.687047in}{0.806250in}}%
\pgfpathlineto{\pgfqpoint{2.687343in}{0.806249in}}%
\pgfpathlineto{\pgfqpoint{2.687639in}{0.806248in}}%
\pgfpathlineto{\pgfqpoint{2.687935in}{0.806247in}}%
\pgfpathlineto{\pgfqpoint{2.688231in}{0.806246in}}%
\pgfpathlineto{\pgfqpoint{2.688527in}{0.806245in}}%
\pgfpathlineto{\pgfqpoint{2.688823in}{0.806244in}}%
\pgfpathlineto{\pgfqpoint{2.689119in}{0.806242in}}%
\pgfpathlineto{\pgfqpoint{2.689415in}{0.806241in}}%
\pgfpathlineto{\pgfqpoint{2.689711in}{0.806240in}}%
\pgfpathlineto{\pgfqpoint{2.690007in}{0.806239in}}%
\pgfpathlineto{\pgfqpoint{2.690303in}{0.806238in}}%
\pgfpathlineto{\pgfqpoint{2.690599in}{0.806237in}}%
\pgfpathlineto{\pgfqpoint{2.690895in}{0.806236in}}%
\pgfpathlineto{\pgfqpoint{2.691191in}{0.806235in}}%
\pgfpathlineto{\pgfqpoint{2.691487in}{0.806234in}}%
\pgfpathlineto{\pgfqpoint{2.691783in}{0.806233in}}%
\pgfpathlineto{\pgfqpoint{2.692079in}{0.806232in}}%
\pgfpathlineto{\pgfqpoint{2.692375in}{0.806231in}}%
\pgfpathlineto{\pgfqpoint{2.692671in}{0.806230in}}%
\pgfpathlineto{\pgfqpoint{2.692967in}{0.806229in}}%
\pgfpathlineto{\pgfqpoint{2.693263in}{0.806228in}}%
\pgfpathlineto{\pgfqpoint{2.693559in}{0.806227in}}%
\pgfpathlineto{\pgfqpoint{2.693855in}{0.806226in}}%
\pgfpathlineto{\pgfqpoint{2.694151in}{0.806224in}}%
\pgfpathlineto{\pgfqpoint{2.694447in}{0.806223in}}%
\pgfpathlineto{\pgfqpoint{2.694743in}{0.806222in}}%
\pgfpathlineto{\pgfqpoint{2.695039in}{0.806221in}}%
\pgfpathlineto{\pgfqpoint{2.695335in}{0.806220in}}%
\pgfpathlineto{\pgfqpoint{2.695631in}{0.806219in}}%
\pgfpathlineto{\pgfqpoint{2.695927in}{0.806218in}}%
\pgfpathlineto{\pgfqpoint{2.696223in}{0.806217in}}%
\pgfpathlineto{\pgfqpoint{2.696519in}{0.806216in}}%
\pgfpathlineto{\pgfqpoint{2.696815in}{0.806215in}}%
\pgfpathlineto{\pgfqpoint{2.697111in}{0.806214in}}%
\pgfpathlineto{\pgfqpoint{2.697407in}{0.806213in}}%
\pgfpathlineto{\pgfqpoint{2.697703in}{0.806212in}}%
\pgfpathlineto{\pgfqpoint{2.697999in}{0.806211in}}%
\pgfpathlineto{\pgfqpoint{2.698295in}{0.806210in}}%
\pgfpathlineto{\pgfqpoint{2.698591in}{0.806209in}}%
\pgfpathlineto{\pgfqpoint{2.698887in}{0.806208in}}%
\pgfpathlineto{\pgfqpoint{2.699183in}{0.806206in}}%
\pgfpathlineto{\pgfqpoint{2.699479in}{0.806205in}}%
\pgfpathlineto{\pgfqpoint{2.699775in}{0.806204in}}%
\pgfpathlineto{\pgfqpoint{2.700071in}{0.806203in}}%
\pgfpathlineto{\pgfqpoint{2.700367in}{0.806202in}}%
\pgfpathlineto{\pgfqpoint{2.700663in}{0.806201in}}%
\pgfpathlineto{\pgfqpoint{2.700959in}{0.806200in}}%
\pgfpathlineto{\pgfqpoint{2.701255in}{0.806199in}}%
\pgfpathlineto{\pgfqpoint{2.701551in}{0.806198in}}%
\pgfpathlineto{\pgfqpoint{2.701847in}{0.806197in}}%
\pgfpathlineto{\pgfqpoint{2.702143in}{0.806196in}}%
\pgfpathlineto{\pgfqpoint{2.702439in}{0.806195in}}%
\pgfpathlineto{\pgfqpoint{2.702735in}{0.806194in}}%
\pgfpathlineto{\pgfqpoint{2.703031in}{0.806193in}}%
\pgfpathlineto{\pgfqpoint{2.703327in}{0.806192in}}%
\pgfpathlineto{\pgfqpoint{2.703623in}{0.806191in}}%
\pgfpathlineto{\pgfqpoint{2.703919in}{0.806190in}}%
\pgfpathlineto{\pgfqpoint{2.704215in}{0.806188in}}%
\pgfpathlineto{\pgfqpoint{2.704511in}{0.806187in}}%
\pgfpathlineto{\pgfqpoint{2.704807in}{0.806186in}}%
\pgfpathlineto{\pgfqpoint{2.705103in}{0.806185in}}%
\pgfpathlineto{\pgfqpoint{2.705399in}{0.806184in}}%
\pgfpathlineto{\pgfqpoint{2.705695in}{0.806183in}}%
\pgfpathlineto{\pgfqpoint{2.705991in}{0.806182in}}%
\pgfpathlineto{\pgfqpoint{2.706287in}{0.806181in}}%
\pgfpathlineto{\pgfqpoint{2.706583in}{0.806148in}}%
\pgfpathlineto{\pgfqpoint{2.706879in}{0.805830in}}%
\pgfpathlineto{\pgfqpoint{2.707175in}{0.805826in}}%
\pgfpathlineto{\pgfqpoint{2.707471in}{0.805822in}}%
\pgfpathlineto{\pgfqpoint{2.707767in}{0.805817in}}%
\pgfpathlineto{\pgfqpoint{2.708063in}{0.805813in}}%
\pgfpathlineto{\pgfqpoint{2.708359in}{0.805809in}}%
\pgfpathlineto{\pgfqpoint{2.708655in}{0.805805in}}%
\pgfpathlineto{\pgfqpoint{2.708951in}{0.805800in}}%
\pgfpathlineto{\pgfqpoint{2.709247in}{0.805796in}}%
\pgfpathlineto{\pgfqpoint{2.709543in}{0.805792in}}%
\pgfpathlineto{\pgfqpoint{2.709839in}{0.805787in}}%
\pgfpathlineto{\pgfqpoint{2.710135in}{0.805783in}}%
\pgfpathlineto{\pgfqpoint{2.710431in}{0.805779in}}%
\pgfpathlineto{\pgfqpoint{2.710727in}{0.805775in}}%
\pgfpathlineto{\pgfqpoint{2.711023in}{0.805770in}}%
\pgfpathlineto{\pgfqpoint{2.711319in}{0.805766in}}%
\pgfpathlineto{\pgfqpoint{2.711615in}{0.805762in}}%
\pgfpathlineto{\pgfqpoint{2.711911in}{0.805757in}}%
\pgfpathlineto{\pgfqpoint{2.712207in}{0.805753in}}%
\pgfpathlineto{\pgfqpoint{2.712503in}{0.805749in}}%
\pgfpathlineto{\pgfqpoint{2.712799in}{0.805745in}}%
\pgfpathlineto{\pgfqpoint{2.713095in}{0.805740in}}%
\pgfpathlineto{\pgfqpoint{2.713391in}{0.805736in}}%
\pgfpathlineto{\pgfqpoint{2.713687in}{0.805731in}}%
\pgfpathlineto{\pgfqpoint{2.713983in}{0.805726in}}%
\pgfpathlineto{\pgfqpoint{2.714279in}{0.805721in}}%
\pgfpathlineto{\pgfqpoint{2.714575in}{0.805715in}}%
\pgfpathlineto{\pgfqpoint{2.714871in}{0.805710in}}%
\pgfpathlineto{\pgfqpoint{2.715167in}{0.805704in}}%
\pgfpathlineto{\pgfqpoint{2.715463in}{0.805701in}}%
\pgfpathlineto{\pgfqpoint{2.715759in}{0.805700in}}%
\pgfpathlineto{\pgfqpoint{2.716055in}{0.805698in}}%
\pgfpathlineto{\pgfqpoint{2.716351in}{0.805697in}}%
\pgfpathlineto{\pgfqpoint{2.716647in}{0.805696in}}%
\pgfpathlineto{\pgfqpoint{2.716943in}{0.805695in}}%
\pgfpathlineto{\pgfqpoint{2.717239in}{0.805694in}}%
\pgfpathlineto{\pgfqpoint{2.717535in}{0.805692in}}%
\pgfpathlineto{\pgfqpoint{2.717831in}{0.805691in}}%
\pgfpathlineto{\pgfqpoint{2.718127in}{0.805690in}}%
\pgfpathlineto{\pgfqpoint{2.718423in}{0.805689in}}%
\pgfpathlineto{\pgfqpoint{2.718719in}{0.805687in}}%
\pgfpathlineto{\pgfqpoint{2.719015in}{0.805686in}}%
\pgfpathlineto{\pgfqpoint{2.719311in}{0.805685in}}%
\pgfpathlineto{\pgfqpoint{2.719607in}{0.805684in}}%
\pgfpathlineto{\pgfqpoint{2.719903in}{0.805683in}}%
\pgfpathlineto{\pgfqpoint{2.720199in}{0.805681in}}%
\pgfpathlineto{\pgfqpoint{2.720495in}{0.805680in}}%
\pgfpathlineto{\pgfqpoint{2.720791in}{0.805679in}}%
\pgfpathlineto{\pgfqpoint{2.721087in}{0.805678in}}%
\pgfpathlineto{\pgfqpoint{2.721383in}{0.805676in}}%
\pgfpathlineto{\pgfqpoint{2.721679in}{0.805675in}}%
\pgfpathlineto{\pgfqpoint{2.721975in}{0.805674in}}%
\pgfpathlineto{\pgfqpoint{2.722271in}{0.805673in}}%
\pgfpathlineto{\pgfqpoint{2.722567in}{0.805672in}}%
\pgfpathlineto{\pgfqpoint{2.722863in}{0.805670in}}%
\pgfpathlineto{\pgfqpoint{2.723159in}{0.805669in}}%
\pgfpathlineto{\pgfqpoint{2.723455in}{0.805668in}}%
\pgfpathlineto{\pgfqpoint{2.723751in}{0.805667in}}%
\pgfpathlineto{\pgfqpoint{2.724047in}{0.805665in}}%
\pgfpathlineto{\pgfqpoint{2.724343in}{0.805664in}}%
\pgfpathlineto{\pgfqpoint{2.724639in}{0.805663in}}%
\pgfpathlineto{\pgfqpoint{2.724935in}{0.805662in}}%
\pgfpathlineto{\pgfqpoint{2.725231in}{0.805660in}}%
\pgfpathlineto{\pgfqpoint{2.725527in}{0.805659in}}%
\pgfpathlineto{\pgfqpoint{2.725823in}{0.805658in}}%
\pgfpathlineto{\pgfqpoint{2.726119in}{0.805657in}}%
\pgfpathlineto{\pgfqpoint{2.726415in}{0.805656in}}%
\pgfpathlineto{\pgfqpoint{2.726711in}{0.805654in}}%
\pgfpathlineto{\pgfqpoint{2.727007in}{0.805653in}}%
\pgfpathlineto{\pgfqpoint{2.727303in}{0.805652in}}%
\pgfpathlineto{\pgfqpoint{2.727599in}{0.805651in}}%
\pgfpathlineto{\pgfqpoint{2.727895in}{0.805649in}}%
\pgfpathlineto{\pgfqpoint{2.728191in}{0.805648in}}%
\pgfpathlineto{\pgfqpoint{2.728487in}{0.805647in}}%
\pgfpathlineto{\pgfqpoint{2.728783in}{0.805644in}}%
\pgfpathlineto{\pgfqpoint{2.729079in}{0.805647in}}%
\pgfpathlineto{\pgfqpoint{2.729375in}{0.805655in}}%
\pgfpathlineto{\pgfqpoint{2.729671in}{0.805663in}}%
\pgfpathlineto{\pgfqpoint{2.729967in}{0.805671in}}%
\pgfpathlineto{\pgfqpoint{2.730263in}{0.805679in}}%
\pgfpathlineto{\pgfqpoint{2.730559in}{0.805687in}}%
\pgfpathlineto{\pgfqpoint{2.730855in}{0.805695in}}%
\pgfpathlineto{\pgfqpoint{2.731151in}{0.805704in}}%
\pgfpathlineto{\pgfqpoint{2.731447in}{0.805712in}}%
\pgfpathlineto{\pgfqpoint{2.731743in}{0.805720in}}%
\pgfpathlineto{\pgfqpoint{2.732039in}{0.805728in}}%
\pgfpathlineto{\pgfqpoint{2.732335in}{0.805736in}}%
\pgfpathlineto{\pgfqpoint{2.732631in}{0.805744in}}%
\pgfpathlineto{\pgfqpoint{2.732927in}{0.805752in}}%
\pgfpathlineto{\pgfqpoint{2.733223in}{0.805760in}}%
\pgfpathlineto{\pgfqpoint{2.733519in}{0.805769in}}%
\pgfpathlineto{\pgfqpoint{2.733815in}{0.805777in}}%
\pgfpathlineto{\pgfqpoint{2.734111in}{0.805785in}}%
\pgfpathlineto{\pgfqpoint{2.734407in}{0.805793in}}%
\pgfpathlineto{\pgfqpoint{2.734703in}{0.805801in}}%
\pgfpathlineto{\pgfqpoint{2.734999in}{0.805809in}}%
\pgfpathlineto{\pgfqpoint{2.735295in}{0.805817in}}%
\pgfpathlineto{\pgfqpoint{2.735591in}{0.805825in}}%
\pgfpathlineto{\pgfqpoint{2.735887in}{0.805833in}}%
\pgfpathlineto{\pgfqpoint{2.736183in}{0.805842in}}%
\pgfpathlineto{\pgfqpoint{2.736479in}{0.805850in}}%
\pgfpathlineto{\pgfqpoint{2.736775in}{0.805859in}}%
\pgfpathlineto{\pgfqpoint{2.737071in}{0.805870in}}%
\pgfpathlineto{\pgfqpoint{2.737367in}{0.805882in}}%
\pgfpathlineto{\pgfqpoint{2.737663in}{0.805893in}}%
\pgfpathlineto{\pgfqpoint{2.737960in}{0.805905in}}%
\pgfpathlineto{\pgfqpoint{2.738256in}{0.805916in}}%
\pgfpathlineto{\pgfqpoint{2.738552in}{0.805928in}}%
\pgfpathlineto{\pgfqpoint{2.738848in}{0.805929in}}%
\pgfpathlineto{\pgfqpoint{2.739144in}{0.805918in}}%
\pgfpathlineto{\pgfqpoint{2.739440in}{0.805907in}}%
\pgfpathlineto{\pgfqpoint{2.739736in}{0.805896in}}%
\pgfpathlineto{\pgfqpoint{2.740032in}{0.805885in}}%
\pgfpathlineto{\pgfqpoint{2.740328in}{0.805873in}}%
\pgfpathlineto{\pgfqpoint{2.740624in}{0.805862in}}%
\pgfpathlineto{\pgfqpoint{2.740920in}{0.805851in}}%
\pgfpathlineto{\pgfqpoint{2.741216in}{0.805840in}}%
\pgfpathlineto{\pgfqpoint{2.741512in}{0.805829in}}%
\pgfpathlineto{\pgfqpoint{2.741808in}{0.805818in}}%
\pgfpathlineto{\pgfqpoint{2.742104in}{0.805807in}}%
\pgfpathlineto{\pgfqpoint{2.742400in}{0.805796in}}%
\pgfpathlineto{\pgfqpoint{2.742696in}{0.805785in}}%
\pgfpathlineto{\pgfqpoint{2.742992in}{0.805774in}}%
\pgfpathlineto{\pgfqpoint{2.743288in}{0.805763in}}%
\pgfpathlineto{\pgfqpoint{2.743584in}{0.805752in}}%
\pgfpathlineto{\pgfqpoint{2.743880in}{0.805741in}}%
\pgfpathlineto{\pgfqpoint{2.744176in}{0.805730in}}%
\pgfpathlineto{\pgfqpoint{2.744472in}{0.805719in}}%
\pgfpathlineto{\pgfqpoint{2.744768in}{0.805708in}}%
\pgfpathlineto{\pgfqpoint{2.745064in}{0.805697in}}%
\pgfpathlineto{\pgfqpoint{2.745360in}{0.805686in}}%
\pgfpathlineto{\pgfqpoint{2.745656in}{0.805675in}}%
\pgfpathlineto{\pgfqpoint{2.745952in}{0.805663in}}%
\pgfpathlineto{\pgfqpoint{2.746248in}{0.805652in}}%
\pgfpathlineto{\pgfqpoint{2.746544in}{0.805641in}}%
\pgfpathlineto{\pgfqpoint{2.746840in}{0.805630in}}%
\pgfpathlineto{\pgfqpoint{2.747136in}{0.805619in}}%
\pgfpathlineto{\pgfqpoint{2.747432in}{0.805608in}}%
\pgfpathlineto{\pgfqpoint{2.747728in}{0.805597in}}%
\pgfpathlineto{\pgfqpoint{2.748024in}{0.805586in}}%
\pgfpathlineto{\pgfqpoint{2.748320in}{0.805575in}}%
\pgfpathlineto{\pgfqpoint{2.748616in}{0.805564in}}%
\pgfpathlineto{\pgfqpoint{2.748912in}{0.805553in}}%
\pgfpathlineto{\pgfqpoint{2.749208in}{0.805542in}}%
\pgfpathlineto{\pgfqpoint{2.749504in}{0.805531in}}%
\pgfpathlineto{\pgfqpoint{2.749800in}{0.805520in}}%
\pgfpathlineto{\pgfqpoint{2.750096in}{0.805509in}}%
\pgfpathlineto{\pgfqpoint{2.750392in}{0.805498in}}%
\pgfpathlineto{\pgfqpoint{2.750688in}{0.805487in}}%
\pgfpathlineto{\pgfqpoint{2.750984in}{0.805476in}}%
\pgfpathlineto{\pgfqpoint{2.751280in}{0.805464in}}%
\pgfpathlineto{\pgfqpoint{2.751576in}{0.805453in}}%
\pgfpathlineto{\pgfqpoint{2.751872in}{0.805442in}}%
\pgfpathlineto{\pgfqpoint{2.752168in}{0.805431in}}%
\pgfpathlineto{\pgfqpoint{2.752464in}{0.805420in}}%
\pgfpathlineto{\pgfqpoint{2.752760in}{0.805409in}}%
\pgfpathlineto{\pgfqpoint{2.753056in}{0.805398in}}%
\pgfpathlineto{\pgfqpoint{2.753352in}{0.805387in}}%
\pgfpathlineto{\pgfqpoint{2.753648in}{0.805376in}}%
\pgfpathlineto{\pgfqpoint{2.753944in}{0.805365in}}%
\pgfpathlineto{\pgfqpoint{2.754240in}{0.805354in}}%
\pgfpathlineto{\pgfqpoint{2.754536in}{0.805343in}}%
\pgfpathlineto{\pgfqpoint{2.754832in}{0.805332in}}%
\pgfpathlineto{\pgfqpoint{2.755128in}{0.805321in}}%
\pgfpathlineto{\pgfqpoint{2.755424in}{0.805310in}}%
\pgfpathlineto{\pgfqpoint{2.755720in}{0.805299in}}%
\pgfpathlineto{\pgfqpoint{2.756016in}{0.805288in}}%
\pgfpathlineto{\pgfqpoint{2.756312in}{0.805277in}}%
\pgfpathlineto{\pgfqpoint{2.756608in}{0.805266in}}%
\pgfpathlineto{\pgfqpoint{2.756904in}{0.805254in}}%
\pgfpathlineto{\pgfqpoint{2.757200in}{0.805243in}}%
\pgfpathlineto{\pgfqpoint{2.757496in}{0.805232in}}%
\pgfpathlineto{\pgfqpoint{2.757792in}{0.805223in}}%
\pgfpathlineto{\pgfqpoint{2.758088in}{0.805220in}}%
\pgfpathlineto{\pgfqpoint{2.758384in}{0.805216in}}%
\pgfpathlineto{\pgfqpoint{2.758680in}{0.805213in}}%
\pgfpathlineto{\pgfqpoint{2.758976in}{0.805210in}}%
\pgfpathlineto{\pgfqpoint{2.759272in}{0.805206in}}%
\pgfpathlineto{\pgfqpoint{2.759568in}{0.805203in}}%
\pgfpathlineto{\pgfqpoint{2.759864in}{0.805199in}}%
\pgfpathlineto{\pgfqpoint{2.760160in}{0.805196in}}%
\pgfpathlineto{\pgfqpoint{2.760456in}{0.805192in}}%
\pgfpathlineto{\pgfqpoint{2.760752in}{0.805189in}}%
\pgfpathlineto{\pgfqpoint{2.761048in}{0.805185in}}%
\pgfpathlineto{\pgfqpoint{2.761344in}{0.805182in}}%
\pgfpathlineto{\pgfqpoint{2.761640in}{0.805179in}}%
\pgfpathlineto{\pgfqpoint{2.761936in}{0.805175in}}%
\pgfpathlineto{\pgfqpoint{2.762232in}{0.805172in}}%
\pgfpathlineto{\pgfqpoint{2.762528in}{0.805168in}}%
\pgfpathlineto{\pgfqpoint{2.762824in}{0.805165in}}%
\pgfpathlineto{\pgfqpoint{2.763120in}{0.805161in}}%
\pgfpathlineto{\pgfqpoint{2.763416in}{0.805158in}}%
\pgfpathlineto{\pgfqpoint{2.763712in}{0.805154in}}%
\pgfpathlineto{\pgfqpoint{2.764008in}{0.805142in}}%
\pgfpathlineto{\pgfqpoint{2.764304in}{0.805123in}}%
\pgfpathlineto{\pgfqpoint{2.764600in}{0.805108in}}%
\pgfpathlineto{\pgfqpoint{2.764896in}{0.805116in}}%
\pgfpathlineto{\pgfqpoint{2.765192in}{0.805137in}}%
\pgfpathlineto{\pgfqpoint{2.765488in}{0.805159in}}%
\pgfpathlineto{\pgfqpoint{2.765784in}{0.805180in}}%
\pgfpathlineto{\pgfqpoint{2.766080in}{0.805202in}}%
\pgfpathlineto{\pgfqpoint{2.766376in}{0.805224in}}%
\pgfpathlineto{\pgfqpoint{2.766672in}{0.805245in}}%
\pgfpathlineto{\pgfqpoint{2.766968in}{0.805267in}}%
\pgfpathlineto{\pgfqpoint{2.767264in}{0.805289in}}%
\pgfpathlineto{\pgfqpoint{2.767560in}{0.805310in}}%
\pgfpathlineto{\pgfqpoint{2.767856in}{0.805332in}}%
\pgfpathlineto{\pgfqpoint{2.768152in}{0.805354in}}%
\pgfpathlineto{\pgfqpoint{2.768448in}{0.805375in}}%
\pgfpathlineto{\pgfqpoint{2.768744in}{0.805397in}}%
\pgfpathlineto{\pgfqpoint{2.769040in}{0.805419in}}%
\pgfpathlineto{\pgfqpoint{2.769336in}{0.805440in}}%
\pgfpathlineto{\pgfqpoint{2.769632in}{0.805462in}}%
\pgfpathlineto{\pgfqpoint{2.769928in}{0.805484in}}%
\pgfpathlineto{\pgfqpoint{2.770224in}{0.805505in}}%
\pgfpathlineto{\pgfqpoint{2.770520in}{0.805527in}}%
\pgfpathlineto{\pgfqpoint{2.770816in}{0.805549in}}%
\pgfpathlineto{\pgfqpoint{2.771112in}{0.805570in}}%
\pgfpathlineto{\pgfqpoint{2.771408in}{0.805690in}}%
\pgfpathlineto{\pgfqpoint{2.771704in}{0.805755in}}%
\pgfpathlineto{\pgfqpoint{2.772000in}{0.805721in}}%
\pgfpathlineto{\pgfqpoint{2.772296in}{0.805686in}}%
\pgfpathlineto{\pgfqpoint{2.772592in}{0.805652in}}%
\pgfpathlineto{\pgfqpoint{2.772888in}{0.805618in}}%
\pgfpathlineto{\pgfqpoint{2.773184in}{0.805584in}}%
\pgfpathlineto{\pgfqpoint{2.773480in}{0.805550in}}%
\pgfpathlineto{\pgfqpoint{2.773776in}{0.805516in}}%
\pgfpathlineto{\pgfqpoint{2.774072in}{0.805482in}}%
\pgfpathlineto{\pgfqpoint{2.774368in}{0.805448in}}%
\pgfpathlineto{\pgfqpoint{2.774664in}{0.805413in}}%
\pgfpathlineto{\pgfqpoint{2.774960in}{0.805379in}}%
\pgfpathlineto{\pgfqpoint{2.775256in}{0.805345in}}%
\pgfpathlineto{\pgfqpoint{2.775552in}{0.805311in}}%
\pgfpathlineto{\pgfqpoint{2.775848in}{0.805277in}}%
\pgfpathlineto{\pgfqpoint{2.776144in}{0.805243in}}%
\pgfpathlineto{\pgfqpoint{2.776440in}{0.805209in}}%
\pgfpathlineto{\pgfqpoint{2.776736in}{0.805175in}}%
\pgfpathlineto{\pgfqpoint{2.777032in}{0.805140in}}%
\pgfpathlineto{\pgfqpoint{2.777328in}{0.805106in}}%
\pgfpathlineto{\pgfqpoint{2.777624in}{0.805072in}}%
\pgfpathlineto{\pgfqpoint{2.777920in}{0.805039in}}%
\pgfpathlineto{\pgfqpoint{2.778216in}{0.805026in}}%
\pgfpathlineto{\pgfqpoint{2.778512in}{0.805024in}}%
\pgfpathlineto{\pgfqpoint{2.778808in}{0.805022in}}%
\pgfpathlineto{\pgfqpoint{2.779104in}{0.805021in}}%
\pgfpathlineto{\pgfqpoint{2.779400in}{0.805019in}}%
\pgfpathlineto{\pgfqpoint{2.779696in}{0.805017in}}%
\pgfpathlineto{\pgfqpoint{2.779992in}{0.805015in}}%
\pgfpathlineto{\pgfqpoint{2.780288in}{0.805013in}}%
\pgfpathlineto{\pgfqpoint{2.780584in}{0.805011in}}%
\pgfpathlineto{\pgfqpoint{2.780880in}{0.805010in}}%
\pgfpathlineto{\pgfqpoint{2.781176in}{0.805008in}}%
\pgfpathlineto{\pgfqpoint{2.781472in}{0.805006in}}%
\pgfpathlineto{\pgfqpoint{2.781768in}{0.805004in}}%
\pgfpathlineto{\pgfqpoint{2.782064in}{0.805002in}}%
\pgfpathlineto{\pgfqpoint{2.782360in}{0.805000in}}%
\pgfpathlineto{\pgfqpoint{2.782656in}{0.804998in}}%
\pgfpathlineto{\pgfqpoint{2.782952in}{0.804997in}}%
\pgfpathlineto{\pgfqpoint{2.783248in}{0.804995in}}%
\pgfpathlineto{\pgfqpoint{2.783544in}{0.804993in}}%
\pgfpathlineto{\pgfqpoint{2.783840in}{0.804991in}}%
\pgfpathlineto{\pgfqpoint{2.784136in}{0.804989in}}%
\pgfpathlineto{\pgfqpoint{2.784432in}{0.804983in}}%
\pgfpathlineto{\pgfqpoint{2.784728in}{0.804971in}}%
\pgfpathlineto{\pgfqpoint{2.785024in}{0.804958in}}%
\pgfpathlineto{\pgfqpoint{2.785320in}{0.804946in}}%
\pgfpathlineto{\pgfqpoint{2.785616in}{0.804933in}}%
\pgfpathlineto{\pgfqpoint{2.785912in}{0.804921in}}%
\pgfpathlineto{\pgfqpoint{2.786208in}{0.804908in}}%
\pgfpathlineto{\pgfqpoint{2.786504in}{0.804896in}}%
\pgfpathlineto{\pgfqpoint{2.786800in}{0.804883in}}%
\pgfpathlineto{\pgfqpoint{2.787096in}{0.804871in}}%
\pgfpathlineto{\pgfqpoint{2.787392in}{0.804858in}}%
\pgfpathlineto{\pgfqpoint{2.787688in}{0.804846in}}%
\pgfpathlineto{\pgfqpoint{2.787984in}{0.804833in}}%
\pgfpathlineto{\pgfqpoint{2.788280in}{0.804821in}}%
\pgfpathlineto{\pgfqpoint{2.788576in}{0.804808in}}%
\pgfpathlineto{\pgfqpoint{2.788872in}{0.804795in}}%
\pgfpathlineto{\pgfqpoint{2.789168in}{0.804783in}}%
\pgfpathlineto{\pgfqpoint{2.789464in}{0.804770in}}%
\pgfpathlineto{\pgfqpoint{2.789760in}{0.804758in}}%
\pgfpathlineto{\pgfqpoint{2.790056in}{0.804745in}}%
\pgfpathlineto{\pgfqpoint{2.790352in}{0.804733in}}%
\pgfpathlineto{\pgfqpoint{2.790648in}{0.804720in}}%
\pgfpathlineto{\pgfqpoint{2.790944in}{0.804708in}}%
\pgfpathlineto{\pgfqpoint{2.791240in}{0.804695in}}%
\pgfpathlineto{\pgfqpoint{2.791536in}{0.804683in}}%
\pgfpathlineto{\pgfqpoint{2.791832in}{0.804670in}}%
\pgfpathlineto{\pgfqpoint{2.792128in}{0.804658in}}%
\pgfpathlineto{\pgfqpoint{2.792424in}{0.804645in}}%
\pgfpathlineto{\pgfqpoint{2.792720in}{0.804633in}}%
\pgfpathlineto{\pgfqpoint{2.793016in}{0.804620in}}%
\pgfpathlineto{\pgfqpoint{2.793312in}{0.804607in}}%
\pgfpathlineto{\pgfqpoint{2.793608in}{0.804595in}}%
\pgfpathlineto{\pgfqpoint{2.793904in}{0.804582in}}%
\pgfpathlineto{\pgfqpoint{2.794200in}{0.804570in}}%
\pgfpathlineto{\pgfqpoint{2.794496in}{0.804557in}}%
\pgfpathlineto{\pgfqpoint{2.794792in}{0.804545in}}%
\pgfpathlineto{\pgfqpoint{2.795088in}{0.804532in}}%
\pgfpathlineto{\pgfqpoint{2.795384in}{0.804520in}}%
\pgfpathlineto{\pgfqpoint{2.795680in}{0.804507in}}%
\pgfpathlineto{\pgfqpoint{2.795976in}{0.804495in}}%
\pgfpathlineto{\pgfqpoint{2.796272in}{0.804482in}}%
\pgfpathlineto{\pgfqpoint{2.796568in}{0.804470in}}%
\pgfpathlineto{\pgfqpoint{2.796864in}{0.804457in}}%
\pgfpathlineto{\pgfqpoint{2.797160in}{0.804445in}}%
\pgfpathlineto{\pgfqpoint{2.797456in}{0.804432in}}%
\pgfpathlineto{\pgfqpoint{2.797752in}{0.804420in}}%
\pgfpathlineto{\pgfqpoint{2.798048in}{0.804407in}}%
\pgfpathlineto{\pgfqpoint{2.798344in}{0.804394in}}%
\pgfpathlineto{\pgfqpoint{2.798640in}{0.804382in}}%
\pgfpathlineto{\pgfqpoint{2.798936in}{0.804369in}}%
\pgfpathlineto{\pgfqpoint{2.799232in}{0.804357in}}%
\pgfpathlineto{\pgfqpoint{2.799528in}{0.804344in}}%
\pgfpathlineto{\pgfqpoint{2.799824in}{0.804332in}}%
\pgfpathlineto{\pgfqpoint{2.800120in}{0.804319in}}%
\pgfpathlineto{\pgfqpoint{2.800416in}{0.804308in}}%
\pgfpathlineto{\pgfqpoint{2.800712in}{0.804300in}}%
\pgfpathlineto{\pgfqpoint{2.801008in}{0.804292in}}%
\pgfpathlineto{\pgfqpoint{2.801304in}{0.804284in}}%
\pgfpathlineto{\pgfqpoint{2.801600in}{0.804276in}}%
\pgfpathlineto{\pgfqpoint{2.801896in}{0.804268in}}%
\pgfpathlineto{\pgfqpoint{2.802192in}{0.804260in}}%
\pgfpathlineto{\pgfqpoint{2.802488in}{0.804252in}}%
\pgfpathlineto{\pgfqpoint{2.802784in}{0.804244in}}%
\pgfpathlineto{\pgfqpoint{2.803080in}{0.804236in}}%
\pgfpathlineto{\pgfqpoint{2.803376in}{0.804228in}}%
\pgfpathlineto{\pgfqpoint{2.803672in}{0.804220in}}%
\pgfpathlineto{\pgfqpoint{2.803968in}{0.804212in}}%
\pgfpathlineto{\pgfqpoint{2.804264in}{0.804204in}}%
\pgfpathlineto{\pgfqpoint{2.804560in}{0.804196in}}%
\pgfpathlineto{\pgfqpoint{2.804856in}{0.804188in}}%
\pgfpathlineto{\pgfqpoint{2.805152in}{0.804180in}}%
\pgfpathlineto{\pgfqpoint{2.805449in}{0.804172in}}%
\pgfpathlineto{\pgfqpoint{2.805745in}{0.804164in}}%
\pgfpathlineto{\pgfqpoint{2.806041in}{0.804156in}}%
\pgfpathlineto{\pgfqpoint{2.806337in}{0.804148in}}%
\pgfpathlineto{\pgfqpoint{2.806633in}{0.804140in}}%
\pgfpathlineto{\pgfqpoint{2.806929in}{0.804132in}}%
\pgfpathlineto{\pgfqpoint{2.807225in}{0.804124in}}%
\pgfpathlineto{\pgfqpoint{2.807521in}{0.804125in}}%
\pgfpathlineto{\pgfqpoint{2.807817in}{0.804142in}}%
\pgfpathlineto{\pgfqpoint{2.808113in}{0.804159in}}%
\pgfpathlineto{\pgfqpoint{2.808409in}{0.804176in}}%
\pgfpathlineto{\pgfqpoint{2.808705in}{0.804193in}}%
\pgfpathlineto{\pgfqpoint{2.809001in}{0.804210in}}%
\pgfpathlineto{\pgfqpoint{2.809297in}{0.804228in}}%
\pgfpathlineto{\pgfqpoint{2.809593in}{0.804245in}}%
\pgfpathlineto{\pgfqpoint{2.809889in}{0.804262in}}%
\pgfpathlineto{\pgfqpoint{2.810185in}{0.804279in}}%
\pgfpathlineto{\pgfqpoint{2.810481in}{0.804296in}}%
\pgfpathlineto{\pgfqpoint{2.810777in}{0.804313in}}%
\pgfpathlineto{\pgfqpoint{2.811073in}{0.804331in}}%
\pgfpathlineto{\pgfqpoint{2.811369in}{0.804348in}}%
\pgfpathlineto{\pgfqpoint{2.811665in}{0.804365in}}%
\pgfpathlineto{\pgfqpoint{2.811961in}{0.804382in}}%
\pgfpathlineto{\pgfqpoint{2.812257in}{0.804399in}}%
\pgfpathlineto{\pgfqpoint{2.812553in}{0.804417in}}%
\pgfpathlineto{\pgfqpoint{2.812849in}{0.804434in}}%
\pgfpathlineto{\pgfqpoint{2.813145in}{0.804451in}}%
\pgfpathlineto{\pgfqpoint{2.813441in}{0.804468in}}%
\pgfpathlineto{\pgfqpoint{2.813737in}{0.804696in}}%
\pgfpathlineto{\pgfqpoint{2.814033in}{0.804901in}}%
\pgfpathlineto{\pgfqpoint{2.814329in}{0.804889in}}%
\pgfpathlineto{\pgfqpoint{2.814625in}{0.804878in}}%
\pgfpathlineto{\pgfqpoint{2.814921in}{0.804867in}}%
\pgfpathlineto{\pgfqpoint{2.815217in}{0.804855in}}%
\pgfpathlineto{\pgfqpoint{2.815513in}{0.804844in}}%
\pgfpathlineto{\pgfqpoint{2.815809in}{0.804833in}}%
\pgfpathlineto{\pgfqpoint{2.816105in}{0.804821in}}%
\pgfpathlineto{\pgfqpoint{2.816401in}{0.804810in}}%
\pgfpathlineto{\pgfqpoint{2.816697in}{0.804798in}}%
\pgfpathlineto{\pgfqpoint{2.816993in}{0.804787in}}%
\pgfpathlineto{\pgfqpoint{2.817289in}{0.804776in}}%
\pgfpathlineto{\pgfqpoint{2.817585in}{0.804764in}}%
\pgfpathlineto{\pgfqpoint{2.817881in}{0.804753in}}%
\pgfpathlineto{\pgfqpoint{2.818177in}{0.804742in}}%
\pgfpathlineto{\pgfqpoint{2.818473in}{0.804730in}}%
\pgfpathlineto{\pgfqpoint{2.818769in}{0.804719in}}%
\pgfpathlineto{\pgfqpoint{2.819065in}{0.804707in}}%
\pgfpathlineto{\pgfqpoint{2.819361in}{0.804696in}}%
\pgfpathlineto{\pgfqpoint{2.819657in}{0.804685in}}%
\pgfpathlineto{\pgfqpoint{2.819953in}{0.804673in}}%
\pgfpathlineto{\pgfqpoint{2.820249in}{0.804662in}}%
\pgfpathlineto{\pgfqpoint{2.820545in}{0.804651in}}%
\pgfpathlineto{\pgfqpoint{2.820841in}{0.804639in}}%
\pgfpathlineto{\pgfqpoint{2.821137in}{0.804628in}}%
\pgfpathlineto{\pgfqpoint{2.821433in}{0.804616in}}%
\pgfpathlineto{\pgfqpoint{2.821729in}{0.804605in}}%
\pgfpathlineto{\pgfqpoint{2.822025in}{0.804594in}}%
\pgfpathlineto{\pgfqpoint{2.822321in}{0.804582in}}%
\pgfpathlineto{\pgfqpoint{2.822617in}{0.804571in}}%
\pgfpathlineto{\pgfqpoint{2.822913in}{0.804560in}}%
\pgfpathlineto{\pgfqpoint{2.823209in}{0.804548in}}%
\pgfpathlineto{\pgfqpoint{2.823505in}{0.804537in}}%
\pgfpathlineto{\pgfqpoint{2.823801in}{0.804525in}}%
\pgfpathlineto{\pgfqpoint{2.824097in}{0.804514in}}%
\pgfpathlineto{\pgfqpoint{2.824393in}{0.804503in}}%
\pgfpathlineto{\pgfqpoint{2.824689in}{0.804491in}}%
\pgfpathlineto{\pgfqpoint{2.824985in}{0.804480in}}%
\pgfpathlineto{\pgfqpoint{2.825281in}{0.804469in}}%
\pgfpathlineto{\pgfqpoint{2.825577in}{0.804457in}}%
\pgfpathlineto{\pgfqpoint{2.825873in}{0.804446in}}%
\pgfpathlineto{\pgfqpoint{2.826169in}{0.804434in}}%
\pgfpathlineto{\pgfqpoint{2.826465in}{0.804423in}}%
\pgfpathlineto{\pgfqpoint{2.826761in}{0.804412in}}%
\pgfpathlineto{\pgfqpoint{2.827057in}{0.804419in}}%
\pgfpathlineto{\pgfqpoint{2.827353in}{0.804541in}}%
\pgfpathlineto{\pgfqpoint{2.827649in}{0.804674in}}%
\pgfpathlineto{\pgfqpoint{2.827945in}{0.804808in}}%
\pgfpathlineto{\pgfqpoint{2.828241in}{0.804882in}}%
\pgfpathlineto{\pgfqpoint{2.828537in}{0.804890in}}%
\pgfpathlineto{\pgfqpoint{2.828833in}{0.804882in}}%
\pgfpathlineto{\pgfqpoint{2.829129in}{0.804872in}}%
\pgfpathlineto{\pgfqpoint{2.829425in}{0.804862in}}%
\pgfpathlineto{\pgfqpoint{2.829721in}{0.804853in}}%
\pgfpathlineto{\pgfqpoint{2.830017in}{0.804843in}}%
\pgfpathlineto{\pgfqpoint{2.830313in}{0.804832in}}%
\pgfpathlineto{\pgfqpoint{2.830609in}{0.804820in}}%
\pgfpathlineto{\pgfqpoint{2.830905in}{0.804808in}}%
\pgfpathlineto{\pgfqpoint{2.831201in}{0.804796in}}%
\pgfpathlineto{\pgfqpoint{2.831497in}{0.804783in}}%
\pgfpathlineto{\pgfqpoint{2.831793in}{0.804771in}}%
\pgfpathlineto{\pgfqpoint{2.832089in}{0.804759in}}%
\pgfpathlineto{\pgfqpoint{2.832385in}{0.804747in}}%
\pgfpathlineto{\pgfqpoint{2.832681in}{0.804735in}}%
\pgfpathlineto{\pgfqpoint{2.832977in}{0.804723in}}%
\pgfpathlineto{\pgfqpoint{2.833273in}{0.804711in}}%
\pgfpathlineto{\pgfqpoint{2.833569in}{0.804698in}}%
\pgfpathlineto{\pgfqpoint{2.833865in}{0.804686in}}%
\pgfpathlineto{\pgfqpoint{2.834161in}{0.804674in}}%
\pgfpathlineto{\pgfqpoint{2.834457in}{0.804662in}}%
\pgfpathlineto{\pgfqpoint{2.834753in}{0.804650in}}%
\pgfpathlineto{\pgfqpoint{2.835049in}{0.804638in}}%
\pgfpathlineto{\pgfqpoint{2.835345in}{0.804626in}}%
\pgfpathlineto{\pgfqpoint{2.835641in}{0.804613in}}%
\pgfpathlineto{\pgfqpoint{2.835937in}{0.804601in}}%
\pgfpathlineto{\pgfqpoint{2.836233in}{0.804590in}}%
\pgfpathlineto{\pgfqpoint{2.836529in}{0.804580in}}%
\pgfpathlineto{\pgfqpoint{2.836825in}{0.803661in}}%
\pgfpathlineto{\pgfqpoint{2.837121in}{0.803422in}}%
\pgfpathlineto{\pgfqpoint{2.837417in}{0.803410in}}%
\pgfpathlineto{\pgfqpoint{2.837713in}{0.803398in}}%
\pgfpathlineto{\pgfqpoint{2.838009in}{0.803386in}}%
\pgfpathlineto{\pgfqpoint{2.838305in}{0.803374in}}%
\pgfpathlineto{\pgfqpoint{2.838601in}{0.803362in}}%
\pgfpathlineto{\pgfqpoint{2.838897in}{0.803350in}}%
\pgfpathlineto{\pgfqpoint{2.839193in}{0.803338in}}%
\pgfpathlineto{\pgfqpoint{2.839489in}{0.803325in}}%
\pgfpathlineto{\pgfqpoint{2.839785in}{0.803313in}}%
\pgfpathlineto{\pgfqpoint{2.840081in}{0.803301in}}%
\pgfpathlineto{\pgfqpoint{2.840377in}{0.803289in}}%
\pgfpathlineto{\pgfqpoint{2.840673in}{0.803277in}}%
\pgfpathlineto{\pgfqpoint{2.840969in}{0.803265in}}%
\pgfpathlineto{\pgfqpoint{2.841265in}{0.803253in}}%
\pgfpathlineto{\pgfqpoint{2.841561in}{0.803241in}}%
\pgfpathlineto{\pgfqpoint{2.841857in}{0.803229in}}%
\pgfpathlineto{\pgfqpoint{2.842153in}{0.803217in}}%
\pgfpathlineto{\pgfqpoint{2.842449in}{0.803205in}}%
\pgfpathlineto{\pgfqpoint{2.842745in}{0.803193in}}%
\pgfpathlineto{\pgfqpoint{2.843041in}{0.803181in}}%
\pgfpathlineto{\pgfqpoint{2.843337in}{0.803169in}}%
\pgfpathlineto{\pgfqpoint{2.843633in}{0.803157in}}%
\pgfpathlineto{\pgfqpoint{2.843929in}{0.803145in}}%
\pgfpathlineto{\pgfqpoint{2.844225in}{0.803132in}}%
\pgfpathlineto{\pgfqpoint{2.844521in}{0.803120in}}%
\pgfpathlineto{\pgfqpoint{2.844817in}{0.803108in}}%
\pgfpathlineto{\pgfqpoint{2.845113in}{0.803096in}}%
\pgfpathlineto{\pgfqpoint{2.845409in}{0.803084in}}%
\pgfpathlineto{\pgfqpoint{2.845705in}{0.803072in}}%
\pgfpathlineto{\pgfqpoint{2.846001in}{0.803060in}}%
\pgfpathlineto{\pgfqpoint{2.846297in}{0.803048in}}%
\pgfpathlineto{\pgfqpoint{2.846593in}{0.803036in}}%
\pgfpathlineto{\pgfqpoint{2.846889in}{0.803024in}}%
\pgfpathlineto{\pgfqpoint{2.847185in}{0.803012in}}%
\pgfpathlineto{\pgfqpoint{2.847481in}{0.803000in}}%
\pgfpathlineto{\pgfqpoint{2.847777in}{0.802988in}}%
\pgfpathlineto{\pgfqpoint{2.848073in}{0.802976in}}%
\pgfpathlineto{\pgfqpoint{2.848369in}{0.802964in}}%
\pgfpathlineto{\pgfqpoint{2.848665in}{0.802952in}}%
\pgfpathlineto{\pgfqpoint{2.848961in}{0.802939in}}%
\pgfpathlineto{\pgfqpoint{2.849257in}{0.802927in}}%
\pgfpathlineto{\pgfqpoint{2.849553in}{0.802915in}}%
\pgfpathlineto{\pgfqpoint{2.849849in}{0.802903in}}%
\pgfpathlineto{\pgfqpoint{2.850145in}{0.802891in}}%
\pgfpathlineto{\pgfqpoint{2.850441in}{0.802879in}}%
\pgfpathlineto{\pgfqpoint{2.850737in}{0.802867in}}%
\pgfpathlineto{\pgfqpoint{2.851033in}{0.802855in}}%
\pgfpathlineto{\pgfqpoint{2.851329in}{0.802843in}}%
\pgfpathlineto{\pgfqpoint{2.851625in}{0.802831in}}%
\pgfpathlineto{\pgfqpoint{2.851921in}{0.802819in}}%
\pgfpathlineto{\pgfqpoint{2.852217in}{0.802807in}}%
\pgfpathlineto{\pgfqpoint{2.852513in}{0.802795in}}%
\pgfpathlineto{\pgfqpoint{2.852809in}{0.802783in}}%
\pgfpathlineto{\pgfqpoint{2.853105in}{0.802771in}}%
\pgfpathlineto{\pgfqpoint{2.853401in}{0.802759in}}%
\pgfpathlineto{\pgfqpoint{2.853697in}{0.802746in}}%
\pgfpathlineto{\pgfqpoint{2.853993in}{0.802734in}}%
\pgfpathlineto{\pgfqpoint{2.854289in}{0.802722in}}%
\pgfpathlineto{\pgfqpoint{2.854585in}{0.802710in}}%
\pgfpathlineto{\pgfqpoint{2.854881in}{0.802698in}}%
\pgfpathlineto{\pgfqpoint{2.855177in}{0.802686in}}%
\pgfpathlineto{\pgfqpoint{2.855473in}{0.802674in}}%
\pgfpathlineto{\pgfqpoint{2.855769in}{0.802566in}}%
\pgfpathlineto{\pgfqpoint{2.856065in}{0.801834in}}%
\pgfpathlineto{\pgfqpoint{2.856361in}{0.801828in}}%
\pgfpathlineto{\pgfqpoint{2.856657in}{0.801822in}}%
\pgfpathlineto{\pgfqpoint{2.856953in}{0.801816in}}%
\pgfpathlineto{\pgfqpoint{2.857249in}{0.801811in}}%
\pgfpathlineto{\pgfqpoint{2.857545in}{0.801805in}}%
\pgfpathlineto{\pgfqpoint{2.857841in}{0.801799in}}%
\pgfpathlineto{\pgfqpoint{2.858137in}{0.801793in}}%
\pgfpathlineto{\pgfqpoint{2.858433in}{0.801787in}}%
\pgfpathlineto{\pgfqpoint{2.858729in}{0.801781in}}%
\pgfpathlineto{\pgfqpoint{2.859025in}{0.801775in}}%
\pgfpathlineto{\pgfqpoint{2.859321in}{0.801770in}}%
\pgfpathlineto{\pgfqpoint{2.859617in}{0.801764in}}%
\pgfpathlineto{\pgfqpoint{2.859913in}{0.801758in}}%
\pgfpathlineto{\pgfqpoint{2.860209in}{0.801752in}}%
\pgfpathlineto{\pgfqpoint{2.860505in}{0.801746in}}%
\pgfpathlineto{\pgfqpoint{2.860801in}{0.801740in}}%
\pgfpathlineto{\pgfqpoint{2.861097in}{0.801734in}}%
\pgfpathlineto{\pgfqpoint{2.861393in}{0.801728in}}%
\pgfpathlineto{\pgfqpoint{2.861689in}{0.801723in}}%
\pgfpathlineto{\pgfqpoint{2.861985in}{0.801717in}}%
\pgfpathlineto{\pgfqpoint{2.862281in}{0.801711in}}%
\pgfpathlineto{\pgfqpoint{2.862577in}{0.801705in}}%
\pgfpathlineto{\pgfqpoint{2.862873in}{0.801699in}}%
\pgfpathlineto{\pgfqpoint{2.863169in}{0.801693in}}%
\pgfpathlineto{\pgfqpoint{2.863465in}{0.801687in}}%
\pgfpathlineto{\pgfqpoint{2.863761in}{0.801681in}}%
\pgfpathlineto{\pgfqpoint{2.864057in}{0.801675in}}%
\pgfpathlineto{\pgfqpoint{2.864353in}{0.801669in}}%
\pgfpathlineto{\pgfqpoint{2.864649in}{0.801663in}}%
\pgfpathlineto{\pgfqpoint{2.864945in}{0.801657in}}%
\pgfpathlineto{\pgfqpoint{2.865241in}{0.801652in}}%
\pgfpathlineto{\pgfqpoint{2.865537in}{0.801646in}}%
\pgfpathlineto{\pgfqpoint{2.865833in}{0.801640in}}%
\pgfpathlineto{\pgfqpoint{2.866129in}{0.801634in}}%
\pgfpathlineto{\pgfqpoint{2.866425in}{0.801628in}}%
\pgfpathlineto{\pgfqpoint{2.866721in}{0.801622in}}%
\pgfpathlineto{\pgfqpoint{2.867017in}{0.801616in}}%
\pgfpathlineto{\pgfqpoint{2.867313in}{0.801610in}}%
\pgfpathlineto{\pgfqpoint{2.867609in}{0.801604in}}%
\pgfpathlineto{\pgfqpoint{2.867905in}{0.801598in}}%
\pgfpathlineto{\pgfqpoint{2.868201in}{0.801592in}}%
\pgfpathlineto{\pgfqpoint{2.868497in}{0.801586in}}%
\pgfpathlineto{\pgfqpoint{2.868793in}{0.801580in}}%
\pgfpathlineto{\pgfqpoint{2.869089in}{0.801574in}}%
\pgfpathlineto{\pgfqpoint{2.869385in}{0.801568in}}%
\pgfpathlineto{\pgfqpoint{2.869681in}{0.801562in}}%
\pgfpathlineto{\pgfqpoint{2.869977in}{0.801556in}}%
\pgfpathlineto{\pgfqpoint{2.870273in}{0.801551in}}%
\pgfpathlineto{\pgfqpoint{2.870569in}{0.801546in}}%
\pgfpathlineto{\pgfqpoint{2.870865in}{0.801543in}}%
\pgfpathlineto{\pgfqpoint{2.871161in}{0.801539in}}%
\pgfpathlineto{\pgfqpoint{2.871457in}{0.801535in}}%
\pgfpathlineto{\pgfqpoint{2.871753in}{0.801531in}}%
\pgfpathlineto{\pgfqpoint{2.872049in}{0.801527in}}%
\pgfpathlineto{\pgfqpoint{2.872345in}{0.801523in}}%
\pgfpathlineto{\pgfqpoint{2.872641in}{0.801519in}}%
\pgfpathlineto{\pgfqpoint{2.872938in}{0.801515in}}%
\pgfpathlineto{\pgfqpoint{2.873234in}{0.801511in}}%
\pgfpathlineto{\pgfqpoint{2.873530in}{0.801507in}}%
\pgfpathlineto{\pgfqpoint{2.873826in}{0.801503in}}%
\pgfpathlineto{\pgfqpoint{2.874122in}{0.801499in}}%
\pgfpathlineto{\pgfqpoint{2.874418in}{0.801495in}}%
\pgfpathlineto{\pgfqpoint{2.874714in}{0.801491in}}%
\pgfpathlineto{\pgfqpoint{2.875010in}{0.801487in}}%
\pgfpathlineto{\pgfqpoint{2.875306in}{0.801484in}}%
\pgfpathlineto{\pgfqpoint{2.875602in}{0.801480in}}%
\pgfpathlineto{\pgfqpoint{2.875898in}{0.801476in}}%
\pgfpathlineto{\pgfqpoint{2.876194in}{0.801472in}}%
\pgfpathlineto{\pgfqpoint{2.876490in}{0.801468in}}%
\pgfpathlineto{\pgfqpoint{2.876786in}{0.801511in}}%
\pgfpathlineto{\pgfqpoint{2.877082in}{0.801529in}}%
\pgfpathlineto{\pgfqpoint{2.877378in}{0.801530in}}%
\pgfpathlineto{\pgfqpoint{2.877674in}{0.801532in}}%
\pgfpathlineto{\pgfqpoint{2.877970in}{0.801533in}}%
\pgfpathlineto{\pgfqpoint{2.878266in}{0.801535in}}%
\pgfpathlineto{\pgfqpoint{2.878562in}{0.801532in}}%
\pgfpathlineto{\pgfqpoint{2.878858in}{0.801522in}}%
\pgfpathlineto{\pgfqpoint{2.879154in}{0.801512in}}%
\pgfpathlineto{\pgfqpoint{2.879450in}{0.801502in}}%
\pgfpathlineto{\pgfqpoint{2.879746in}{0.801492in}}%
\pgfpathlineto{\pgfqpoint{2.880042in}{0.801482in}}%
\pgfpathlineto{\pgfqpoint{2.880338in}{0.801472in}}%
\pgfpathlineto{\pgfqpoint{2.880634in}{0.801462in}}%
\pgfpathlineto{\pgfqpoint{2.880930in}{0.801453in}}%
\pgfpathlineto{\pgfqpoint{2.881226in}{0.801443in}}%
\pgfpathlineto{\pgfqpoint{2.881522in}{0.801433in}}%
\pgfpathlineto{\pgfqpoint{2.881818in}{0.801423in}}%
\pgfpathlineto{\pgfqpoint{2.882114in}{0.801413in}}%
\pgfpathlineto{\pgfqpoint{2.882410in}{0.801403in}}%
\pgfpathlineto{\pgfqpoint{2.882706in}{0.801393in}}%
\pgfpathlineto{\pgfqpoint{2.883002in}{0.801383in}}%
\pgfpathlineto{\pgfqpoint{2.883298in}{0.801373in}}%
\pgfpathlineto{\pgfqpoint{2.883594in}{0.801363in}}%
\pgfpathlineto{\pgfqpoint{2.883890in}{0.801353in}}%
\pgfpathlineto{\pgfqpoint{2.884186in}{0.801343in}}%
\pgfpathlineto{\pgfqpoint{2.884482in}{0.801333in}}%
\pgfpathlineto{\pgfqpoint{2.884778in}{0.801323in}}%
\pgfpathlineto{\pgfqpoint{2.885074in}{0.801313in}}%
\pgfpathlineto{\pgfqpoint{2.885370in}{0.801306in}}%
\pgfpathlineto{\pgfqpoint{2.885666in}{0.801305in}}%
\pgfpathlineto{\pgfqpoint{2.885962in}{0.801304in}}%
\pgfpathlineto{\pgfqpoint{2.886258in}{0.801303in}}%
\pgfpathlineto{\pgfqpoint{2.886554in}{0.801302in}}%
\pgfpathlineto{\pgfqpoint{2.886850in}{0.801301in}}%
\pgfpathlineto{\pgfqpoint{2.887146in}{0.801301in}}%
\pgfpathlineto{\pgfqpoint{2.887442in}{0.801300in}}%
\pgfpathlineto{\pgfqpoint{2.887738in}{0.801299in}}%
\pgfpathlineto{\pgfqpoint{2.888034in}{0.801298in}}%
\pgfpathlineto{\pgfqpoint{2.888330in}{0.801297in}}%
\pgfpathlineto{\pgfqpoint{2.888626in}{0.801297in}}%
\pgfpathlineto{\pgfqpoint{2.888922in}{0.801296in}}%
\pgfpathlineto{\pgfqpoint{2.889218in}{0.801295in}}%
\pgfpathlineto{\pgfqpoint{2.889514in}{0.801294in}}%
\pgfpathlineto{\pgfqpoint{2.889810in}{0.801294in}}%
\pgfpathlineto{\pgfqpoint{2.890106in}{0.801293in}}%
\pgfpathlineto{\pgfqpoint{2.890402in}{0.801292in}}%
\pgfpathlineto{\pgfqpoint{2.890698in}{0.801291in}}%
\pgfpathlineto{\pgfqpoint{2.890994in}{0.801290in}}%
\pgfpathlineto{\pgfqpoint{2.891290in}{0.801290in}}%
\pgfpathlineto{\pgfqpoint{2.891586in}{0.801289in}}%
\pgfpathlineto{\pgfqpoint{2.891882in}{0.801288in}}%
\pgfpathlineto{\pgfqpoint{2.892178in}{0.801287in}}%
\pgfpathlineto{\pgfqpoint{2.892474in}{0.801286in}}%
\pgfpathlineto{\pgfqpoint{2.892770in}{0.801286in}}%
\pgfpathlineto{\pgfqpoint{2.893066in}{0.801285in}}%
\pgfpathlineto{\pgfqpoint{2.893362in}{0.801284in}}%
\pgfpathlineto{\pgfqpoint{2.893658in}{0.801283in}}%
\pgfpathlineto{\pgfqpoint{2.893954in}{0.801282in}}%
\pgfpathlineto{\pgfqpoint{2.894250in}{0.801282in}}%
\pgfpathlineto{\pgfqpoint{2.894546in}{0.801281in}}%
\pgfpathlineto{\pgfqpoint{2.894842in}{0.801280in}}%
\pgfpathlineto{\pgfqpoint{2.895138in}{0.801279in}}%
\pgfpathlineto{\pgfqpoint{2.895434in}{0.801278in}}%
\pgfpathlineto{\pgfqpoint{2.895730in}{0.801278in}}%
\pgfpathlineto{\pgfqpoint{2.896026in}{0.801277in}}%
\pgfpathlineto{\pgfqpoint{2.896322in}{0.801276in}}%
\pgfpathlineto{\pgfqpoint{2.896618in}{0.801275in}}%
\pgfpathlineto{\pgfqpoint{2.896914in}{0.801275in}}%
\pgfpathlineto{\pgfqpoint{2.897210in}{0.801274in}}%
\pgfpathlineto{\pgfqpoint{2.897506in}{0.801273in}}%
\pgfpathlineto{\pgfqpoint{2.897802in}{0.801272in}}%
\pgfpathlineto{\pgfqpoint{2.898098in}{0.801271in}}%
\pgfpathlineto{\pgfqpoint{2.898394in}{0.801271in}}%
\pgfpathlineto{\pgfqpoint{2.898690in}{0.801270in}}%
\pgfpathlineto{\pgfqpoint{2.898986in}{0.801269in}}%
\pgfpathlineto{\pgfqpoint{2.899282in}{0.801268in}}%
\pgfpathlineto{\pgfqpoint{2.899578in}{0.801267in}}%
\pgfpathlineto{\pgfqpoint{2.899874in}{0.801267in}}%
\pgfpathlineto{\pgfqpoint{2.900170in}{0.801266in}}%
\pgfpathlineto{\pgfqpoint{2.900466in}{0.801265in}}%
\pgfpathlineto{\pgfqpoint{2.900762in}{0.801264in}}%
\pgfpathlineto{\pgfqpoint{2.901058in}{0.801263in}}%
\pgfpathlineto{\pgfqpoint{2.901354in}{0.801263in}}%
\pgfpathlineto{\pgfqpoint{2.901650in}{0.801262in}}%
\pgfpathlineto{\pgfqpoint{2.901946in}{0.801261in}}%
\pgfpathlineto{\pgfqpoint{2.902242in}{0.801260in}}%
\pgfpathlineto{\pgfqpoint{2.902538in}{0.801260in}}%
\pgfpathlineto{\pgfqpoint{2.902834in}{0.801259in}}%
\pgfpathlineto{\pgfqpoint{2.903130in}{0.801258in}}%
\pgfpathlineto{\pgfqpoint{2.903426in}{0.801257in}}%
\pgfpathlineto{\pgfqpoint{2.903722in}{0.801256in}}%
\pgfpathlineto{\pgfqpoint{2.904018in}{0.801256in}}%
\pgfpathlineto{\pgfqpoint{2.904314in}{0.801255in}}%
\pgfpathlineto{\pgfqpoint{2.904610in}{0.801254in}}%
\pgfpathlineto{\pgfqpoint{2.904906in}{0.801253in}}%
\pgfpathlineto{\pgfqpoint{2.905202in}{0.801252in}}%
\pgfpathlineto{\pgfqpoint{2.905498in}{0.801252in}}%
\pgfpathlineto{\pgfqpoint{2.905794in}{0.801251in}}%
\pgfpathlineto{\pgfqpoint{2.906090in}{0.801253in}}%
\pgfpathlineto{\pgfqpoint{2.906386in}{0.801266in}}%
\pgfpathlineto{\pgfqpoint{2.906682in}{0.801280in}}%
\pgfpathlineto{\pgfqpoint{2.906978in}{0.801294in}}%
\pgfpathlineto{\pgfqpoint{2.907274in}{0.801272in}}%
\pgfpathlineto{\pgfqpoint{2.907570in}{0.801194in}}%
\pgfpathlineto{\pgfqpoint{2.907866in}{0.801206in}}%
\pgfpathlineto{\pgfqpoint{2.908162in}{0.801217in}}%
\pgfpathlineto{\pgfqpoint{2.908458in}{0.801229in}}%
\pgfpathlineto{\pgfqpoint{2.908754in}{0.801241in}}%
\pgfpathlineto{\pgfqpoint{2.909050in}{0.801252in}}%
\pgfpathlineto{\pgfqpoint{2.909346in}{0.801264in}}%
\pgfpathlineto{\pgfqpoint{2.909642in}{0.801275in}}%
\pgfpathlineto{\pgfqpoint{2.909938in}{0.801287in}}%
\pgfpathlineto{\pgfqpoint{2.910234in}{0.801298in}}%
\pgfpathlineto{\pgfqpoint{2.910530in}{0.801310in}}%
\pgfpathlineto{\pgfqpoint{2.910826in}{0.801321in}}%
\pgfpathlineto{\pgfqpoint{2.911122in}{0.801333in}}%
\pgfpathlineto{\pgfqpoint{2.911418in}{0.801345in}}%
\pgfpathlineto{\pgfqpoint{2.911714in}{0.801356in}}%
\pgfpathlineto{\pgfqpoint{2.912010in}{0.801368in}}%
\pgfpathlineto{\pgfqpoint{2.912306in}{0.801359in}}%
\pgfpathlineto{\pgfqpoint{2.912602in}{0.801306in}}%
\pgfpathlineto{\pgfqpoint{2.912898in}{0.801301in}}%
\pgfpathlineto{\pgfqpoint{2.913194in}{0.801297in}}%
\pgfpathlineto{\pgfqpoint{2.913490in}{0.801293in}}%
\pgfpathlineto{\pgfqpoint{2.913786in}{0.801289in}}%
\pgfpathlineto{\pgfqpoint{2.914082in}{0.801286in}}%
\pgfpathlineto{\pgfqpoint{2.914378in}{0.801282in}}%
\pgfpathlineto{\pgfqpoint{2.914674in}{0.801278in}}%
\pgfpathlineto{\pgfqpoint{2.914970in}{0.801275in}}%
\pgfpathlineto{\pgfqpoint{2.915266in}{0.801271in}}%
\pgfpathlineto{\pgfqpoint{2.915562in}{0.801267in}}%
\pgfpathlineto{\pgfqpoint{2.915858in}{0.801263in}}%
\pgfpathlineto{\pgfqpoint{2.916154in}{0.801260in}}%
\pgfpathlineto{\pgfqpoint{2.916450in}{0.801256in}}%
\pgfpathlineto{\pgfqpoint{2.916746in}{0.801252in}}%
\pgfpathlineto{\pgfqpoint{2.917042in}{0.801248in}}%
\pgfpathlineto{\pgfqpoint{2.917338in}{0.801245in}}%
\pgfpathlineto{\pgfqpoint{2.917634in}{0.801241in}}%
\pgfpathlineto{\pgfqpoint{2.917930in}{0.801237in}}%
\pgfpathlineto{\pgfqpoint{2.918226in}{0.801234in}}%
\pgfpathlineto{\pgfqpoint{2.918522in}{0.801230in}}%
\pgfpathlineto{\pgfqpoint{2.918818in}{0.801226in}}%
\pgfpathlineto{\pgfqpoint{2.919114in}{0.801222in}}%
\pgfpathlineto{\pgfqpoint{2.919410in}{0.801219in}}%
\pgfpathlineto{\pgfqpoint{2.919706in}{0.801215in}}%
\pgfpathlineto{\pgfqpoint{2.920002in}{0.801211in}}%
\pgfpathlineto{\pgfqpoint{2.920298in}{0.801208in}}%
\pgfpathlineto{\pgfqpoint{2.920594in}{0.801205in}}%
\pgfpathlineto{\pgfqpoint{2.920890in}{0.801204in}}%
\pgfpathlineto{\pgfqpoint{2.921186in}{0.801204in}}%
\pgfpathlineto{\pgfqpoint{2.921482in}{0.801205in}}%
\pgfpathlineto{\pgfqpoint{2.921778in}{0.801200in}}%
\pgfpathlineto{\pgfqpoint{2.922074in}{0.801184in}}%
\pgfpathlineto{\pgfqpoint{2.922370in}{0.801169in}}%
\pgfpathlineto{\pgfqpoint{2.922666in}{0.801154in}}%
\pgfpathlineto{\pgfqpoint{2.922962in}{0.801139in}}%
\pgfpathlineto{\pgfqpoint{2.923258in}{0.801124in}}%
\pgfpathlineto{\pgfqpoint{2.923554in}{0.801108in}}%
\pgfpathlineto{\pgfqpoint{2.923850in}{0.801093in}}%
\pgfpathlineto{\pgfqpoint{2.924146in}{0.801078in}}%
\pgfpathlineto{\pgfqpoint{2.924442in}{0.801063in}}%
\pgfpathlineto{\pgfqpoint{2.924738in}{0.801048in}}%
\pgfpathlineto{\pgfqpoint{2.925034in}{0.801032in}}%
\pgfpathlineto{\pgfqpoint{2.925330in}{0.801017in}}%
\pgfpathlineto{\pgfqpoint{2.925626in}{0.801002in}}%
\pgfpathlineto{\pgfqpoint{2.925922in}{0.800987in}}%
\pgfpathlineto{\pgfqpoint{2.926218in}{0.802726in}}%
\pgfpathlineto{\pgfqpoint{2.926514in}{0.812455in}}%
\pgfpathlineto{\pgfqpoint{2.926810in}{0.812959in}}%
\pgfpathlineto{\pgfqpoint{2.927106in}{0.812623in}}%
\pgfpathlineto{\pgfqpoint{2.927402in}{0.812248in}}%
\pgfpathlineto{\pgfqpoint{2.927698in}{0.811867in}}%
\pgfpathlineto{\pgfqpoint{2.927994in}{0.811486in}}%
\pgfpathlineto{\pgfqpoint{2.928290in}{0.811105in}}%
\pgfpathlineto{\pgfqpoint{2.928586in}{0.810724in}}%
\pgfpathlineto{\pgfqpoint{2.928882in}{0.810343in}}%
\pgfpathlineto{\pgfqpoint{2.929178in}{0.809963in}}%
\pgfpathlineto{\pgfqpoint{2.929474in}{0.809582in}}%
\pgfpathlineto{\pgfqpoint{2.929770in}{0.809201in}}%
\pgfpathlineto{\pgfqpoint{2.930066in}{0.808820in}}%
\pgfpathlineto{\pgfqpoint{2.930362in}{0.808439in}}%
\pgfpathlineto{\pgfqpoint{2.930658in}{0.808058in}}%
\pgfpathlineto{\pgfqpoint{2.930954in}{0.807677in}}%
\pgfpathlineto{\pgfqpoint{2.931250in}{0.807296in}}%
\pgfpathlineto{\pgfqpoint{2.931546in}{0.806915in}}%
\pgfpathlineto{\pgfqpoint{2.931842in}{0.806534in}}%
\pgfpathlineto{\pgfqpoint{2.932138in}{0.806153in}}%
\pgfpathlineto{\pgfqpoint{2.932434in}{0.805772in}}%
\pgfpathlineto{\pgfqpoint{2.932730in}{0.805392in}}%
\pgfpathlineto{\pgfqpoint{2.933026in}{0.805011in}}%
\pgfpathlineto{\pgfqpoint{2.933322in}{0.804630in}}%
\pgfpathlineto{\pgfqpoint{2.933618in}{0.804249in}}%
\pgfpathlineto{\pgfqpoint{2.933914in}{0.803868in}}%
\pgfpathlineto{\pgfqpoint{2.934210in}{0.803487in}}%
\pgfpathlineto{\pgfqpoint{2.934506in}{0.803106in}}%
\pgfpathlineto{\pgfqpoint{2.934802in}{0.802725in}}%
\pgfpathlineto{\pgfqpoint{2.935098in}{0.802344in}}%
\pgfpathlineto{\pgfqpoint{2.935394in}{0.801963in}}%
\pgfpathlineto{\pgfqpoint{2.935690in}{0.801682in}}%
\pgfpathlineto{\pgfqpoint{2.935986in}{0.800806in}}%
\pgfpathlineto{\pgfqpoint{2.936282in}{0.800688in}}%
\pgfpathlineto{\pgfqpoint{2.936578in}{0.800584in}}%
\pgfpathlineto{\pgfqpoint{2.936874in}{0.800479in}}%
\pgfpathlineto{\pgfqpoint{2.937170in}{0.800376in}}%
\pgfpathlineto{\pgfqpoint{2.937466in}{0.800332in}}%
\pgfpathlineto{\pgfqpoint{2.937762in}{0.800327in}}%
\pgfpathlineto{\pgfqpoint{2.938058in}{0.800321in}}%
\pgfpathlineto{\pgfqpoint{2.938354in}{0.800316in}}%
\pgfpathlineto{\pgfqpoint{2.938650in}{0.800310in}}%
\pgfpathlineto{\pgfqpoint{2.938946in}{0.800304in}}%
\pgfpathlineto{\pgfqpoint{2.939242in}{0.800299in}}%
\pgfpathlineto{\pgfqpoint{2.939538in}{0.800293in}}%
\pgfpathlineto{\pgfqpoint{2.939834in}{0.800288in}}%
\pgfpathlineto{\pgfqpoint{2.940131in}{0.800282in}}%
\pgfpathlineto{\pgfqpoint{2.940427in}{0.800277in}}%
\pgfpathlineto{\pgfqpoint{2.940723in}{0.800271in}}%
\pgfpathlineto{\pgfqpoint{2.941019in}{0.800265in}}%
\pgfpathlineto{\pgfqpoint{2.941315in}{0.800260in}}%
\pgfpathlineto{\pgfqpoint{2.941611in}{0.800254in}}%
\pgfpathlineto{\pgfqpoint{2.941907in}{0.800249in}}%
\pgfpathlineto{\pgfqpoint{2.942203in}{0.800243in}}%
\pgfpathlineto{\pgfqpoint{2.942499in}{0.800238in}}%
\pgfpathlineto{\pgfqpoint{2.942795in}{0.800232in}}%
\pgfpathlineto{\pgfqpoint{2.943091in}{0.800226in}}%
\pgfpathlineto{\pgfqpoint{2.943387in}{0.800221in}}%
\pgfpathlineto{\pgfqpoint{2.943683in}{0.800215in}}%
\pgfpathlineto{\pgfqpoint{2.943979in}{0.800210in}}%
\pgfpathlineto{\pgfqpoint{2.944275in}{0.800204in}}%
\pgfpathlineto{\pgfqpoint{2.944571in}{0.800199in}}%
\pgfpathlineto{\pgfqpoint{2.944867in}{0.800193in}}%
\pgfpathlineto{\pgfqpoint{2.945163in}{0.800187in}}%
\pgfpathlineto{\pgfqpoint{2.945459in}{0.800182in}}%
\pgfpathlineto{\pgfqpoint{2.945755in}{0.800176in}}%
\pgfpathlineto{\pgfqpoint{2.946051in}{0.800171in}}%
\pgfpathlineto{\pgfqpoint{2.946347in}{0.800165in}}%
\pgfpathlineto{\pgfqpoint{2.946643in}{0.800160in}}%
\pgfpathlineto{\pgfqpoint{2.946939in}{0.800154in}}%
\pgfpathlineto{\pgfqpoint{2.947235in}{0.800148in}}%
\pgfpathlineto{\pgfqpoint{2.947531in}{0.800143in}}%
\pgfpathlineto{\pgfqpoint{2.947827in}{0.800137in}}%
\pgfpathlineto{\pgfqpoint{2.948123in}{0.800132in}}%
\pgfpathlineto{\pgfqpoint{2.948419in}{0.800126in}}%
\pgfpathlineto{\pgfqpoint{2.948715in}{0.800121in}}%
\pgfpathlineto{\pgfqpoint{2.949011in}{0.800115in}}%
\pgfpathlineto{\pgfqpoint{2.949307in}{0.800109in}}%
\pgfpathlineto{\pgfqpoint{2.949603in}{0.800104in}}%
\pgfpathlineto{\pgfqpoint{2.949899in}{0.800098in}}%
\pgfpathlineto{\pgfqpoint{2.950195in}{0.800093in}}%
\pgfpathlineto{\pgfqpoint{2.950491in}{0.800087in}}%
\pgfpathlineto{\pgfqpoint{2.950787in}{0.800082in}}%
\pgfpathlineto{\pgfqpoint{2.951083in}{0.800076in}}%
\pgfpathlineto{\pgfqpoint{2.951379in}{0.800070in}}%
\pgfpathlineto{\pgfqpoint{2.951675in}{0.800065in}}%
\pgfpathlineto{\pgfqpoint{2.951971in}{0.800059in}}%
\pgfpathlineto{\pgfqpoint{2.952267in}{0.800054in}}%
\pgfpathlineto{\pgfqpoint{2.952563in}{0.800048in}}%
\pgfpathlineto{\pgfqpoint{2.952859in}{0.800043in}}%
\pgfpathlineto{\pgfqpoint{2.953155in}{0.800037in}}%
\pgfpathlineto{\pgfqpoint{2.953451in}{0.800031in}}%
\pgfpathlineto{\pgfqpoint{2.953747in}{0.800026in}}%
\pgfpathlineto{\pgfqpoint{2.954043in}{0.800020in}}%
\pgfpathlineto{\pgfqpoint{2.954339in}{0.800015in}}%
\pgfpathlineto{\pgfqpoint{2.954635in}{0.800009in}}%
\pgfpathlineto{\pgfqpoint{2.954931in}{0.802502in}}%
\pgfpathlineto{\pgfqpoint{2.955227in}{0.810397in}}%
\pgfpathlineto{\pgfqpoint{2.955523in}{0.811033in}}%
\pgfpathlineto{\pgfqpoint{2.955819in}{0.810504in}}%
\pgfpathlineto{\pgfqpoint{2.956115in}{0.809976in}}%
\pgfpathlineto{\pgfqpoint{2.956411in}{0.809447in}}%
\pgfpathlineto{\pgfqpoint{2.956707in}{0.808918in}}%
\pgfpathlineto{\pgfqpoint{2.957003in}{0.808390in}}%
\pgfpathlineto{\pgfqpoint{2.957299in}{0.807861in}}%
\pgfpathlineto{\pgfqpoint{2.957595in}{0.807332in}}%
\pgfpathlineto{\pgfqpoint{2.957891in}{0.806804in}}%
\pgfpathlineto{\pgfqpoint{2.958187in}{0.806275in}}%
\pgfpathlineto{\pgfqpoint{2.958483in}{0.805746in}}%
\pgfpathlineto{\pgfqpoint{2.958779in}{0.805218in}}%
\pgfpathlineto{\pgfqpoint{2.959075in}{0.804689in}}%
\pgfpathlineto{\pgfqpoint{2.959371in}{0.804161in}}%
\pgfpathlineto{\pgfqpoint{2.959667in}{0.803632in}}%
\pgfpathlineto{\pgfqpoint{2.959963in}{0.803103in}}%
\pgfpathlineto{\pgfqpoint{2.960259in}{0.802575in}}%
\pgfpathlineto{\pgfqpoint{2.960555in}{0.802046in}}%
\pgfpathlineto{\pgfqpoint{2.960851in}{0.801517in}}%
\pgfpathlineto{\pgfqpoint{2.961147in}{0.800989in}}%
\pgfpathlineto{\pgfqpoint{2.961443in}{0.800460in}}%
\pgfpathlineto{\pgfqpoint{2.961739in}{0.799931in}}%
\pgfpathlineto{\pgfqpoint{2.962035in}{0.804378in}}%
\pgfpathlineto{\pgfqpoint{2.962331in}{0.804551in}}%
\pgfpathlineto{\pgfqpoint{2.962627in}{0.804555in}}%
\pgfpathlineto{\pgfqpoint{2.962923in}{0.804558in}}%
\pgfpathlineto{\pgfqpoint{2.963219in}{0.804561in}}%
\pgfpathlineto{\pgfqpoint{2.963515in}{0.804654in}}%
\pgfpathlineto{\pgfqpoint{2.963811in}{0.804868in}}%
\pgfpathlineto{\pgfqpoint{2.964107in}{0.804845in}}%
\pgfpathlineto{\pgfqpoint{2.964403in}{0.804823in}}%
\pgfpathlineto{\pgfqpoint{2.964699in}{0.804800in}}%
\pgfpathlineto{\pgfqpoint{2.964995in}{0.804777in}}%
\pgfpathlineto{\pgfqpoint{2.965291in}{0.804754in}}%
\pgfpathlineto{\pgfqpoint{2.965587in}{0.804731in}}%
\pgfpathlineto{\pgfqpoint{2.965883in}{0.804709in}}%
\pgfpathlineto{\pgfqpoint{2.966179in}{0.804686in}}%
\pgfpathlineto{\pgfqpoint{2.966475in}{0.804663in}}%
\pgfpathlineto{\pgfqpoint{2.966771in}{0.804640in}}%
\pgfpathlineto{\pgfqpoint{2.967067in}{0.804618in}}%
\pgfpathlineto{\pgfqpoint{2.967363in}{0.804595in}}%
\pgfpathlineto{\pgfqpoint{2.967659in}{0.804572in}}%
\pgfpathlineto{\pgfqpoint{2.967955in}{0.804549in}}%
\pgfpathlineto{\pgfqpoint{2.968251in}{0.804526in}}%
\pgfpathlineto{\pgfqpoint{2.968547in}{0.804504in}}%
\pgfpathlineto{\pgfqpoint{2.968843in}{0.804481in}}%
\pgfpathlineto{\pgfqpoint{2.969139in}{0.804458in}}%
\pgfpathlineto{\pgfqpoint{2.969435in}{0.804411in}}%
\pgfpathlineto{\pgfqpoint{2.969731in}{0.804107in}}%
\pgfpathlineto{\pgfqpoint{2.970027in}{0.803728in}}%
\pgfpathlineto{\pgfqpoint{2.970323in}{0.803348in}}%
\pgfpathlineto{\pgfqpoint{2.970619in}{0.802968in}}%
\pgfpathlineto{\pgfqpoint{2.970915in}{0.802592in}}%
\pgfpathlineto{\pgfqpoint{2.971211in}{0.802446in}}%
\pgfpathlineto{\pgfqpoint{2.971507in}{0.802427in}}%
\pgfpathlineto{\pgfqpoint{2.971803in}{0.802408in}}%
\pgfpathlineto{\pgfqpoint{2.972099in}{0.802390in}}%
\pgfpathlineto{\pgfqpoint{2.972395in}{0.802371in}}%
\pgfpathlineto{\pgfqpoint{2.972691in}{0.802353in}}%
\pgfpathlineto{\pgfqpoint{2.972987in}{0.802334in}}%
\pgfpathlineto{\pgfqpoint{2.973283in}{0.802316in}}%
\pgfpathlineto{\pgfqpoint{2.973579in}{0.802297in}}%
\pgfpathlineto{\pgfqpoint{2.973875in}{0.802279in}}%
\pgfpathlineto{\pgfqpoint{2.974171in}{0.802260in}}%
\pgfpathlineto{\pgfqpoint{2.974467in}{0.802242in}}%
\pgfpathlineto{\pgfqpoint{2.974763in}{0.802223in}}%
\pgfpathlineto{\pgfqpoint{2.975059in}{0.802205in}}%
\pgfpathlineto{\pgfqpoint{2.975355in}{0.802186in}}%
\pgfpathlineto{\pgfqpoint{2.975651in}{0.802168in}}%
\pgfpathlineto{\pgfqpoint{2.975947in}{0.802149in}}%
\pgfpathlineto{\pgfqpoint{2.976243in}{0.802129in}}%
\pgfpathlineto{\pgfqpoint{2.976539in}{0.801832in}}%
\pgfpathlineto{\pgfqpoint{2.976835in}{0.801801in}}%
\pgfpathlineto{\pgfqpoint{2.977131in}{0.801793in}}%
\pgfpathlineto{\pgfqpoint{2.977427in}{0.801786in}}%
\pgfpathlineto{\pgfqpoint{2.977723in}{0.801778in}}%
\pgfpathlineto{\pgfqpoint{2.978019in}{0.801770in}}%
\pgfpathlineto{\pgfqpoint{2.978315in}{0.801762in}}%
\pgfpathlineto{\pgfqpoint{2.978611in}{0.801755in}}%
\pgfpathlineto{\pgfqpoint{2.978907in}{0.801747in}}%
\pgfpathlineto{\pgfqpoint{2.979203in}{0.801739in}}%
\pgfpathlineto{\pgfqpoint{2.979499in}{0.801732in}}%
\pgfpathlineto{\pgfqpoint{2.979795in}{0.801724in}}%
\pgfpathlineto{\pgfqpoint{2.980091in}{0.801716in}}%
\pgfpathlineto{\pgfqpoint{2.980387in}{0.801708in}}%
\pgfpathlineto{\pgfqpoint{2.980683in}{0.801701in}}%
\pgfpathlineto{\pgfqpoint{2.980979in}{0.801693in}}%
\pgfpathlineto{\pgfqpoint{2.981275in}{0.801685in}}%
\pgfpathlineto{\pgfqpoint{2.981571in}{0.801677in}}%
\pgfpathlineto{\pgfqpoint{2.981867in}{0.801670in}}%
\pgfpathlineto{\pgfqpoint{2.982163in}{0.801662in}}%
\pgfpathlineto{\pgfqpoint{2.982459in}{0.801654in}}%
\pgfpathlineto{\pgfqpoint{2.982755in}{0.801647in}}%
\pgfpathlineto{\pgfqpoint{2.983051in}{0.801639in}}%
\pgfpathlineto{\pgfqpoint{2.983347in}{0.801631in}}%
\pgfpathlineto{\pgfqpoint{2.983643in}{0.801623in}}%
\pgfpathlineto{\pgfqpoint{2.983939in}{0.801616in}}%
\pgfpathlineto{\pgfqpoint{2.984235in}{0.801608in}}%
\pgfpathlineto{\pgfqpoint{2.984531in}{0.801600in}}%
\pgfpathlineto{\pgfqpoint{2.984827in}{0.801592in}}%
\pgfpathlineto{\pgfqpoint{2.985123in}{0.801585in}}%
\pgfpathlineto{\pgfqpoint{2.985419in}{0.801577in}}%
\pgfpathlineto{\pgfqpoint{2.985715in}{0.801569in}}%
\pgfpathlineto{\pgfqpoint{2.986011in}{0.801562in}}%
\pgfpathlineto{\pgfqpoint{2.986307in}{0.801554in}}%
\pgfpathlineto{\pgfqpoint{2.986603in}{0.801546in}}%
\pgfpathlineto{\pgfqpoint{2.986899in}{0.801538in}}%
\pgfpathlineto{\pgfqpoint{2.987195in}{0.801531in}}%
\pgfpathlineto{\pgfqpoint{2.987491in}{0.801523in}}%
\pgfpathlineto{\pgfqpoint{2.987787in}{0.801515in}}%
\pgfpathlineto{\pgfqpoint{2.988083in}{0.801507in}}%
\pgfpathlineto{\pgfqpoint{2.988379in}{0.801500in}}%
\pgfpathlineto{\pgfqpoint{2.988675in}{0.801492in}}%
\pgfpathlineto{\pgfqpoint{2.988971in}{0.801484in}}%
\pgfpathlineto{\pgfqpoint{2.989267in}{0.801476in}}%
\pgfpathlineto{\pgfqpoint{2.989563in}{0.801469in}}%
\pgfpathlineto{\pgfqpoint{2.989859in}{0.801461in}}%
\pgfpathlineto{\pgfqpoint{2.990155in}{0.801453in}}%
\pgfpathlineto{\pgfqpoint{2.990451in}{0.801446in}}%
\pgfpathlineto{\pgfqpoint{2.990747in}{0.801438in}}%
\pgfpathlineto{\pgfqpoint{2.991043in}{0.801430in}}%
\pgfpathlineto{\pgfqpoint{2.991339in}{0.801422in}}%
\pgfpathlineto{\pgfqpoint{2.991635in}{0.801415in}}%
\pgfpathlineto{\pgfqpoint{2.991931in}{0.801407in}}%
\pgfpathlineto{\pgfqpoint{2.992227in}{0.801399in}}%
\pgfpathlineto{\pgfqpoint{2.992523in}{0.801391in}}%
\pgfpathlineto{\pgfqpoint{2.992819in}{0.801384in}}%
\pgfpathlineto{\pgfqpoint{2.993115in}{0.801376in}}%
\pgfpathlineto{\pgfqpoint{2.993411in}{0.801368in}}%
\pgfpathlineto{\pgfqpoint{2.993707in}{0.801361in}}%
\pgfpathlineto{\pgfqpoint{2.994003in}{0.801353in}}%
\pgfpathlineto{\pgfqpoint{2.994299in}{0.801345in}}%
\pgfpathlineto{\pgfqpoint{2.994595in}{0.801337in}}%
\pgfpathlineto{\pgfqpoint{2.994891in}{0.801330in}}%
\pgfpathlineto{\pgfqpoint{2.995187in}{0.801322in}}%
\pgfpathlineto{\pgfqpoint{2.995483in}{0.801314in}}%
\pgfpathlineto{\pgfqpoint{2.995779in}{0.801306in}}%
\pgfpathlineto{\pgfqpoint{2.996075in}{0.801299in}}%
\pgfpathlineto{\pgfqpoint{2.996371in}{0.801291in}}%
\pgfpathlineto{\pgfqpoint{2.996667in}{0.801283in}}%
\pgfpathlineto{\pgfqpoint{2.996963in}{0.801276in}}%
\pgfpathlineto{\pgfqpoint{2.997259in}{0.801268in}}%
\pgfpathlineto{\pgfqpoint{2.997555in}{0.801260in}}%
\pgfpathlineto{\pgfqpoint{2.997851in}{0.801252in}}%
\pgfpathlineto{\pgfqpoint{2.998147in}{0.801245in}}%
\pgfpathlineto{\pgfqpoint{2.998443in}{0.801237in}}%
\pgfpathlineto{\pgfqpoint{2.998739in}{0.801229in}}%
\pgfpathlineto{\pgfqpoint{2.999035in}{0.801221in}}%
\pgfpathlineto{\pgfqpoint{2.999331in}{0.801214in}}%
\pgfpathlineto{\pgfqpoint{2.999627in}{0.801206in}}%
\pgfpathlineto{\pgfqpoint{2.999923in}{0.801198in}}%
\pgfpathlineto{\pgfqpoint{3.000219in}{0.801190in}}%
\pgfpathlineto{\pgfqpoint{3.000515in}{0.801183in}}%
\pgfpathlineto{\pgfqpoint{3.000811in}{0.801175in}}%
\pgfpathlineto{\pgfqpoint{3.001107in}{0.801167in}}%
\pgfpathlineto{\pgfqpoint{3.001403in}{0.801160in}}%
\pgfpathlineto{\pgfqpoint{3.001699in}{0.801152in}}%
\pgfpathlineto{\pgfqpoint{3.001995in}{0.801144in}}%
\pgfpathlineto{\pgfqpoint{3.002291in}{0.801136in}}%
\pgfpathlineto{\pgfqpoint{3.002587in}{0.801129in}}%
\pgfpathlineto{\pgfqpoint{3.002883in}{0.801121in}}%
\pgfpathlineto{\pgfqpoint{3.003179in}{0.801113in}}%
\pgfpathlineto{\pgfqpoint{3.003475in}{0.801105in}}%
\pgfpathlineto{\pgfqpoint{3.003771in}{0.801098in}}%
\pgfpathlineto{\pgfqpoint{3.004067in}{0.801090in}}%
\pgfpathlineto{\pgfqpoint{3.004363in}{0.801082in}}%
\pgfpathlineto{\pgfqpoint{3.004659in}{0.801096in}}%
\pgfpathlineto{\pgfqpoint{3.004955in}{0.801137in}}%
\pgfpathlineto{\pgfqpoint{3.005251in}{0.801177in}}%
\pgfpathlineto{\pgfqpoint{3.005547in}{0.801159in}}%
\pgfpathlineto{\pgfqpoint{3.005843in}{0.800317in}}%
\pgfpathlineto{\pgfqpoint{3.006139in}{0.799972in}}%
\pgfpathlineto{\pgfqpoint{3.006435in}{0.799963in}}%
\pgfpathlineto{\pgfqpoint{3.006731in}{0.799954in}}%
\pgfpathlineto{\pgfqpoint{3.007027in}{0.799946in}}%
\pgfpathlineto{\pgfqpoint{3.007323in}{0.799937in}}%
\pgfpathlineto{\pgfqpoint{3.007620in}{0.799928in}}%
\pgfpathlineto{\pgfqpoint{3.007916in}{0.799919in}}%
\pgfpathlineto{\pgfqpoint{3.008212in}{0.799911in}}%
\pgfpathlineto{\pgfqpoint{3.008508in}{0.799902in}}%
\pgfpathlineto{\pgfqpoint{3.008804in}{0.799893in}}%
\pgfpathlineto{\pgfqpoint{3.009100in}{0.799885in}}%
\pgfpathlineto{\pgfqpoint{3.009396in}{0.799876in}}%
\pgfpathlineto{\pgfqpoint{3.009692in}{0.799867in}}%
\pgfpathlineto{\pgfqpoint{3.009988in}{0.799858in}}%
\pgfpathlineto{\pgfqpoint{3.010284in}{0.799850in}}%
\pgfpathlineto{\pgfqpoint{3.010580in}{0.799841in}}%
\pgfpathlineto{\pgfqpoint{3.010876in}{0.799832in}}%
\pgfpathlineto{\pgfqpoint{3.011172in}{0.799824in}}%
\pgfpathlineto{\pgfqpoint{3.011468in}{0.799815in}}%
\pgfpathlineto{\pgfqpoint{3.011764in}{0.799806in}}%
\pgfpathlineto{\pgfqpoint{3.012060in}{0.799797in}}%
\pgfpathlineto{\pgfqpoint{3.012356in}{0.799789in}}%
\pgfpathlineto{\pgfqpoint{3.012652in}{0.799780in}}%
\pgfpathlineto{\pgfqpoint{3.012948in}{0.799771in}}%
\pgfpathlineto{\pgfqpoint{3.013244in}{0.799763in}}%
\pgfpathlineto{\pgfqpoint{3.013540in}{0.799754in}}%
\pgfpathlineto{\pgfqpoint{3.013836in}{0.799745in}}%
\pgfpathlineto{\pgfqpoint{3.014132in}{0.799736in}}%
\pgfpathlineto{\pgfqpoint{3.014428in}{0.799728in}}%
\pgfpathlineto{\pgfqpoint{3.014724in}{0.799719in}}%
\pgfpathlineto{\pgfqpoint{3.015020in}{0.799710in}}%
\pgfpathlineto{\pgfqpoint{3.015316in}{0.799702in}}%
\pgfpathlineto{\pgfqpoint{3.015612in}{0.799693in}}%
\pgfpathlineto{\pgfqpoint{3.015908in}{0.799684in}}%
\pgfpathlineto{\pgfqpoint{3.016204in}{0.799675in}}%
\pgfpathlineto{\pgfqpoint{3.016500in}{0.799667in}}%
\pgfpathlineto{\pgfqpoint{3.016796in}{0.799658in}}%
\pgfpathlineto{\pgfqpoint{3.017092in}{0.799649in}}%
\pgfpathlineto{\pgfqpoint{3.017388in}{0.799641in}}%
\pgfpathlineto{\pgfqpoint{3.017684in}{0.799632in}}%
\pgfpathlineto{\pgfqpoint{3.017980in}{0.799623in}}%
\pgfpathlineto{\pgfqpoint{3.018276in}{0.799615in}}%
\pgfpathlineto{\pgfqpoint{3.018572in}{0.799606in}}%
\pgfpathlineto{\pgfqpoint{3.018868in}{0.799597in}}%
\pgfpathlineto{\pgfqpoint{3.019164in}{0.799590in}}%
\pgfpathlineto{\pgfqpoint{3.019460in}{0.799583in}}%
\pgfpathlineto{\pgfqpoint{3.019756in}{0.799576in}}%
\pgfpathlineto{\pgfqpoint{3.020052in}{0.799569in}}%
\pgfpathlineto{\pgfqpoint{3.020348in}{0.799564in}}%
\pgfpathlineto{\pgfqpoint{3.020644in}{0.799631in}}%
\pgfpathlineto{\pgfqpoint{3.020940in}{0.799729in}}%
\pgfpathlineto{\pgfqpoint{3.021236in}{0.799827in}}%
\pgfpathlineto{\pgfqpoint{3.021532in}{0.799925in}}%
\pgfpathlineto{\pgfqpoint{3.021828in}{0.800023in}}%
\pgfpathlineto{\pgfqpoint{3.022124in}{0.800122in}}%
\pgfpathlineto{\pgfqpoint{3.022420in}{0.800220in}}%
\pgfpathlineto{\pgfqpoint{3.022716in}{0.800318in}}%
\pgfpathlineto{\pgfqpoint{3.023012in}{0.800416in}}%
\pgfpathlineto{\pgfqpoint{3.023308in}{0.800514in}}%
\pgfpathlineto{\pgfqpoint{3.023604in}{0.800612in}}%
\pgfpathlineto{\pgfqpoint{3.023900in}{0.800710in}}%
\pgfpathlineto{\pgfqpoint{3.024196in}{0.800809in}}%
\pgfpathlineto{\pgfqpoint{3.024492in}{0.800907in}}%
\pgfpathlineto{\pgfqpoint{3.024788in}{0.801005in}}%
\pgfpathlineto{\pgfqpoint{3.025084in}{0.801103in}}%
\pgfpathlineto{\pgfqpoint{3.025380in}{0.801201in}}%
\pgfpathlineto{\pgfqpoint{3.025676in}{0.801299in}}%
\pgfpathlineto{\pgfqpoint{3.025972in}{0.801397in}}%
\pgfpathlineto{\pgfqpoint{3.026268in}{0.801496in}}%
\pgfpathlineto{\pgfqpoint{3.026564in}{0.801594in}}%
\pgfpathlineto{\pgfqpoint{3.026860in}{0.801643in}}%
\pgfpathlineto{\pgfqpoint{3.027156in}{0.801639in}}%
\pgfpathlineto{\pgfqpoint{3.027452in}{0.801635in}}%
\pgfpathlineto{\pgfqpoint{3.027748in}{0.801631in}}%
\pgfpathlineto{\pgfqpoint{3.028044in}{0.801381in}}%
\pgfpathlineto{\pgfqpoint{3.028340in}{0.802708in}}%
\pgfpathlineto{\pgfqpoint{3.028636in}{0.802514in}}%
\pgfpathlineto{\pgfqpoint{3.028932in}{0.802320in}}%
\pgfpathlineto{\pgfqpoint{3.029228in}{0.802126in}}%
\pgfpathlineto{\pgfqpoint{3.029524in}{0.801933in}}%
\pgfpathlineto{\pgfqpoint{3.029820in}{0.801739in}}%
\pgfpathlineto{\pgfqpoint{3.030116in}{0.801545in}}%
\pgfpathlineto{\pgfqpoint{3.030412in}{0.801351in}}%
\pgfpathlineto{\pgfqpoint{3.030708in}{0.801157in}}%
\pgfpathlineto{\pgfqpoint{3.031004in}{0.800963in}}%
\pgfpathlineto{\pgfqpoint{3.031300in}{0.800769in}}%
\pgfpathlineto{\pgfqpoint{3.031596in}{0.800575in}}%
\pgfpathlineto{\pgfqpoint{3.031892in}{0.800381in}}%
\pgfpathlineto{\pgfqpoint{3.032188in}{0.800187in}}%
\pgfpathlineto{\pgfqpoint{3.032484in}{0.799994in}}%
\pgfpathlineto{\pgfqpoint{3.032780in}{0.799800in}}%
\pgfpathlineto{\pgfqpoint{3.033076in}{0.799606in}}%
\pgfpathlineto{\pgfqpoint{3.033372in}{0.799417in}}%
\pgfpathlineto{\pgfqpoint{3.033668in}{0.802606in}}%
\pgfpathlineto{\pgfqpoint{3.033964in}{0.802934in}}%
\pgfpathlineto{\pgfqpoint{3.034260in}{0.803025in}}%
\pgfpathlineto{\pgfqpoint{3.034556in}{0.803116in}}%
\pgfpathlineto{\pgfqpoint{3.034852in}{0.803207in}}%
\pgfpathlineto{\pgfqpoint{3.035148in}{0.803298in}}%
\pgfpathlineto{\pgfqpoint{3.035444in}{0.803389in}}%
\pgfpathlineto{\pgfqpoint{3.035740in}{0.803480in}}%
\pgfpathlineto{\pgfqpoint{3.036036in}{0.803543in}}%
\pgfpathlineto{\pgfqpoint{3.036332in}{0.803507in}}%
\pgfpathlineto{\pgfqpoint{3.036628in}{0.803463in}}%
\pgfpathlineto{\pgfqpoint{3.036924in}{0.803418in}}%
\pgfpathlineto{\pgfqpoint{3.037220in}{0.803374in}}%
\pgfpathlineto{\pgfqpoint{3.037516in}{0.803330in}}%
\pgfpathlineto{\pgfqpoint{3.037812in}{0.803285in}}%
\pgfpathlineto{\pgfqpoint{3.038108in}{0.803241in}}%
\pgfpathlineto{\pgfqpoint{3.038404in}{0.803196in}}%
\pgfpathlineto{\pgfqpoint{3.038700in}{0.803152in}}%
\pgfpathlineto{\pgfqpoint{3.038996in}{0.803107in}}%
\pgfpathlineto{\pgfqpoint{3.039292in}{0.803063in}}%
\pgfpathlineto{\pgfqpoint{3.039588in}{0.803018in}}%
\pgfpathlineto{\pgfqpoint{3.039884in}{0.802974in}}%
\pgfpathlineto{\pgfqpoint{3.040180in}{0.802930in}}%
\pgfpathlineto{\pgfqpoint{3.040476in}{0.802885in}}%
\pgfpathlineto{\pgfqpoint{3.040772in}{0.802841in}}%
\pgfpathlineto{\pgfqpoint{3.041068in}{0.802796in}}%
\pgfpathlineto{\pgfqpoint{3.041364in}{0.802752in}}%
\pgfpathlineto{\pgfqpoint{3.041660in}{0.802707in}}%
\pgfpathlineto{\pgfqpoint{3.041956in}{0.802663in}}%
\pgfpathlineto{\pgfqpoint{3.042252in}{0.802618in}}%
\pgfpathlineto{\pgfqpoint{3.042548in}{0.802574in}}%
\pgfpathlineto{\pgfqpoint{3.042844in}{0.802530in}}%
\pgfpathlineto{\pgfqpoint{3.043140in}{0.802485in}}%
\pgfpathlineto{\pgfqpoint{3.043436in}{0.802441in}}%
\pgfpathlineto{\pgfqpoint{3.043732in}{0.802396in}}%
\pgfpathlineto{\pgfqpoint{3.044028in}{0.802352in}}%
\pgfpathlineto{\pgfqpoint{3.044324in}{0.802307in}}%
\pgfpathlineto{\pgfqpoint{3.044620in}{0.802263in}}%
\pgfpathlineto{\pgfqpoint{3.044916in}{0.802218in}}%
\pgfpathlineto{\pgfqpoint{3.045212in}{0.802174in}}%
\pgfpathlineto{\pgfqpoint{3.045508in}{0.802130in}}%
\pgfpathlineto{\pgfqpoint{3.045804in}{0.802085in}}%
\pgfpathlineto{\pgfqpoint{3.046100in}{0.802041in}}%
\pgfpathlineto{\pgfqpoint{3.046396in}{0.801996in}}%
\pgfpathlineto{\pgfqpoint{3.046692in}{0.801952in}}%
\pgfpathlineto{\pgfqpoint{3.046988in}{0.801907in}}%
\pgfpathlineto{\pgfqpoint{3.047284in}{0.801863in}}%
\pgfpathlineto{\pgfqpoint{3.047580in}{0.801818in}}%
\pgfpathlineto{\pgfqpoint{3.047876in}{0.801774in}}%
\pgfpathlineto{\pgfqpoint{3.048172in}{0.801730in}}%
\pgfpathlineto{\pgfqpoint{3.048468in}{0.801685in}}%
\pgfpathlineto{\pgfqpoint{3.048764in}{0.801641in}}%
\pgfpathlineto{\pgfqpoint{3.049060in}{0.801596in}}%
\pgfpathlineto{\pgfqpoint{3.049356in}{0.801552in}}%
\pgfpathlineto{\pgfqpoint{3.049652in}{0.801507in}}%
\pgfpathlineto{\pgfqpoint{3.049948in}{0.801463in}}%
\pgfpathlineto{\pgfqpoint{3.050244in}{0.801418in}}%
\pgfpathlineto{\pgfqpoint{3.050540in}{0.801374in}}%
\pgfpathlineto{\pgfqpoint{3.050836in}{0.801330in}}%
\pgfpathlineto{\pgfqpoint{3.051132in}{0.801285in}}%
\pgfpathlineto{\pgfqpoint{3.051428in}{0.801241in}}%
\pgfpathlineto{\pgfqpoint{3.051724in}{0.801196in}}%
\pgfpathlineto{\pgfqpoint{3.052020in}{0.801152in}}%
\pgfpathlineto{\pgfqpoint{3.052316in}{0.801107in}}%
\pgfpathlineto{\pgfqpoint{3.052612in}{0.801063in}}%
\pgfpathlineto{\pgfqpoint{3.052908in}{0.801018in}}%
\pgfpathlineto{\pgfqpoint{3.053204in}{0.800974in}}%
\pgfpathlineto{\pgfqpoint{3.053500in}{0.800930in}}%
\pgfpathlineto{\pgfqpoint{3.053796in}{0.800885in}}%
\pgfpathlineto{\pgfqpoint{3.054092in}{0.800836in}}%
\pgfpathlineto{\pgfqpoint{3.054388in}{0.800765in}}%
\pgfpathlineto{\pgfqpoint{3.054684in}{0.799615in}}%
\pgfpathlineto{\pgfqpoint{3.054980in}{0.798882in}}%
\pgfpathlineto{\pgfqpoint{3.055276in}{0.798838in}}%
\pgfpathlineto{\pgfqpoint{3.055572in}{0.798816in}}%
\pgfpathlineto{\pgfqpoint{3.055868in}{0.798776in}}%
\pgfpathlineto{\pgfqpoint{3.056164in}{0.799287in}}%
\pgfpathlineto{\pgfqpoint{3.056460in}{0.800892in}}%
\pgfpathlineto{\pgfqpoint{3.056756in}{0.800729in}}%
\pgfpathlineto{\pgfqpoint{3.057052in}{0.800551in}}%
\pgfpathlineto{\pgfqpoint{3.057348in}{0.800373in}}%
\pgfpathlineto{\pgfqpoint{3.057644in}{0.800195in}}%
\pgfpathlineto{\pgfqpoint{3.057940in}{0.800017in}}%
\pgfpathlineto{\pgfqpoint{3.058236in}{0.799839in}}%
\pgfpathlineto{\pgfqpoint{3.058532in}{0.799661in}}%
\pgfpathlineto{\pgfqpoint{3.058828in}{0.799483in}}%
\pgfpathlineto{\pgfqpoint{3.059124in}{0.799305in}}%
\pgfpathlineto{\pgfqpoint{3.059420in}{0.799127in}}%
\pgfpathlineto{\pgfqpoint{3.059716in}{0.798949in}}%
\pgfpathlineto{\pgfqpoint{3.060012in}{0.798771in}}%
\pgfpathlineto{\pgfqpoint{3.060308in}{0.798593in}}%
\pgfpathlineto{\pgfqpoint{3.060604in}{0.798415in}}%
\pgfpathlineto{\pgfqpoint{3.060900in}{0.798237in}}%
\pgfpathlineto{\pgfqpoint{3.061196in}{0.798059in}}%
\pgfpathlineto{\pgfqpoint{3.061492in}{0.798673in}}%
\pgfpathlineto{\pgfqpoint{3.061788in}{0.799152in}}%
\pgfpathlineto{\pgfqpoint{3.062084in}{0.799554in}}%
\pgfpathlineto{\pgfqpoint{3.062380in}{0.800017in}}%
\pgfpathlineto{\pgfqpoint{3.062676in}{0.799950in}}%
\pgfpathlineto{\pgfqpoint{3.062972in}{0.799919in}}%
\pgfpathlineto{\pgfqpoint{3.063268in}{0.799878in}}%
\pgfpathlineto{\pgfqpoint{3.063564in}{0.799823in}}%
\pgfpathlineto{\pgfqpoint{3.063860in}{0.799812in}}%
\pgfpathlineto{\pgfqpoint{3.064156in}{0.799801in}}%
\pgfpathlineto{\pgfqpoint{3.064452in}{0.799790in}}%
\pgfpathlineto{\pgfqpoint{3.064748in}{0.799779in}}%
\pgfpathlineto{\pgfqpoint{3.065044in}{0.799768in}}%
\pgfpathlineto{\pgfqpoint{3.065340in}{0.799757in}}%
\pgfpathlineto{\pgfqpoint{3.065636in}{0.799746in}}%
\pgfpathlineto{\pgfqpoint{3.065932in}{0.799735in}}%
\pgfpathlineto{\pgfqpoint{3.066228in}{0.799724in}}%
\pgfpathlineto{\pgfqpoint{3.066524in}{0.799713in}}%
\pgfpathlineto{\pgfqpoint{3.066820in}{0.799702in}}%
\pgfpathlineto{\pgfqpoint{3.067116in}{0.799691in}}%
\pgfpathlineto{\pgfqpoint{3.067412in}{0.799680in}}%
\pgfpathlineto{\pgfqpoint{3.067708in}{0.799669in}}%
\pgfpathlineto{\pgfqpoint{3.068004in}{0.799658in}}%
\pgfpathlineto{\pgfqpoint{3.068300in}{0.799647in}}%
\pgfpathlineto{\pgfqpoint{3.068596in}{0.799636in}}%
\pgfpathlineto{\pgfqpoint{3.068892in}{0.799625in}}%
\pgfpathlineto{\pgfqpoint{3.069188in}{0.799614in}}%
\pgfpathlineto{\pgfqpoint{3.069484in}{0.799603in}}%
\pgfpathlineto{\pgfqpoint{3.069780in}{0.799592in}}%
\pgfpathlineto{\pgfqpoint{3.070076in}{0.799582in}}%
\pgfpathlineto{\pgfqpoint{3.070372in}{0.799571in}}%
\pgfpathlineto{\pgfqpoint{3.070668in}{0.799308in}}%
\pgfpathlineto{\pgfqpoint{3.070964in}{0.798904in}}%
\pgfpathlineto{\pgfqpoint{3.071260in}{0.798765in}}%
\pgfpathlineto{\pgfqpoint{3.071556in}{0.798771in}}%
\pgfpathlineto{\pgfqpoint{3.071852in}{0.798776in}}%
\pgfpathlineto{\pgfqpoint{3.072148in}{0.798781in}}%
\pgfpathlineto{\pgfqpoint{3.072444in}{0.798787in}}%
\pgfpathlineto{\pgfqpoint{3.072740in}{0.798792in}}%
\pgfpathlineto{\pgfqpoint{3.073036in}{0.798798in}}%
\pgfpathlineto{\pgfqpoint{3.073332in}{0.798803in}}%
\pgfpathlineto{\pgfqpoint{3.073628in}{0.798809in}}%
\pgfpathlineto{\pgfqpoint{3.073924in}{0.798814in}}%
\pgfpathlineto{\pgfqpoint{3.074220in}{0.798820in}}%
\pgfpathlineto{\pgfqpoint{3.074516in}{0.798825in}}%
\pgfpathlineto{\pgfqpoint{3.074812in}{0.798831in}}%
\pgfpathlineto{\pgfqpoint{3.075109in}{0.798836in}}%
\pgfpathlineto{\pgfqpoint{3.075405in}{0.798841in}}%
\pgfpathlineto{\pgfqpoint{3.075701in}{0.798847in}}%
\pgfpathlineto{\pgfqpoint{3.075997in}{0.798844in}}%
\pgfpathlineto{\pgfqpoint{3.076293in}{0.798829in}}%
\pgfpathlineto{\pgfqpoint{3.076589in}{0.798814in}}%
\pgfpathlineto{\pgfqpoint{3.076885in}{0.798798in}}%
\pgfpathlineto{\pgfqpoint{3.077181in}{0.798783in}}%
\pgfpathlineto{\pgfqpoint{3.077477in}{0.798768in}}%
\pgfpathlineto{\pgfqpoint{3.077773in}{0.798753in}}%
\pgfpathlineto{\pgfqpoint{3.078069in}{0.798738in}}%
\pgfpathlineto{\pgfqpoint{3.078365in}{0.798723in}}%
\pgfpathlineto{\pgfqpoint{3.078661in}{0.798708in}}%
\pgfpathlineto{\pgfqpoint{3.078957in}{0.798692in}}%
\pgfpathlineto{\pgfqpoint{3.079253in}{0.798677in}}%
\pgfpathlineto{\pgfqpoint{3.079549in}{0.798662in}}%
\pgfpathlineto{\pgfqpoint{3.079845in}{0.798647in}}%
\pgfpathlineto{\pgfqpoint{3.080141in}{0.798632in}}%
\pgfpathlineto{\pgfqpoint{3.080437in}{0.798617in}}%
\pgfpathlineto{\pgfqpoint{3.080733in}{0.798601in}}%
\pgfpathlineto{\pgfqpoint{3.081029in}{0.798586in}}%
\pgfpathlineto{\pgfqpoint{3.081325in}{0.798571in}}%
\pgfpathlineto{\pgfqpoint{3.081621in}{0.798556in}}%
\pgfpathlineto{\pgfqpoint{3.081917in}{0.798541in}}%
\pgfpathlineto{\pgfqpoint{3.082213in}{0.798526in}}%
\pgfpathlineto{\pgfqpoint{3.082509in}{0.798510in}}%
\pgfpathlineto{\pgfqpoint{3.082805in}{0.798495in}}%
\pgfpathlineto{\pgfqpoint{3.083101in}{0.798480in}}%
\pgfpathlineto{\pgfqpoint{3.083397in}{0.798465in}}%
\pgfpathlineto{\pgfqpoint{3.083693in}{0.798450in}}%
\pgfpathlineto{\pgfqpoint{3.083989in}{0.798435in}}%
\pgfpathlineto{\pgfqpoint{3.084285in}{0.798419in}}%
\pgfpathlineto{\pgfqpoint{3.084581in}{0.798404in}}%
\pgfpathlineto{\pgfqpoint{3.084877in}{0.798389in}}%
\pgfpathlineto{\pgfqpoint{3.085173in}{0.798374in}}%
\pgfpathlineto{\pgfqpoint{3.085469in}{0.798359in}}%
\pgfpathlineto{\pgfqpoint{3.085765in}{0.798344in}}%
\pgfpathlineto{\pgfqpoint{3.086061in}{0.798328in}}%
\pgfpathlineto{\pgfqpoint{3.086357in}{0.798313in}}%
\pgfpathlineto{\pgfqpoint{3.086653in}{0.798298in}}%
\pgfpathlineto{\pgfqpoint{3.086949in}{0.798283in}}%
\pgfpathlineto{\pgfqpoint{3.087245in}{0.798268in}}%
\pgfpathlineto{\pgfqpoint{3.087541in}{0.798253in}}%
\pgfpathlineto{\pgfqpoint{3.087837in}{0.798237in}}%
\pgfpathlineto{\pgfqpoint{3.088133in}{0.798222in}}%
\pgfpathlineto{\pgfqpoint{3.088429in}{0.798207in}}%
\pgfpathlineto{\pgfqpoint{3.088725in}{0.798192in}}%
\pgfpathlineto{\pgfqpoint{3.089021in}{0.798177in}}%
\pgfpathlineto{\pgfqpoint{3.089317in}{0.798162in}}%
\pgfpathlineto{\pgfqpoint{3.089613in}{0.798146in}}%
\pgfpathlineto{\pgfqpoint{3.089909in}{0.798131in}}%
\pgfpathlineto{\pgfqpoint{3.090205in}{0.798116in}}%
\pgfpathlineto{\pgfqpoint{3.090501in}{0.798101in}}%
\pgfpathlineto{\pgfqpoint{3.090797in}{0.798086in}}%
\pgfpathlineto{\pgfqpoint{3.091093in}{0.798071in}}%
\pgfpathlineto{\pgfqpoint{3.091389in}{0.798055in}}%
\pgfpathlineto{\pgfqpoint{3.091685in}{0.798040in}}%
\pgfpathlineto{\pgfqpoint{3.091981in}{0.798025in}}%
\pgfpathlineto{\pgfqpoint{3.092277in}{0.798010in}}%
\pgfpathlineto{\pgfqpoint{3.092573in}{0.797995in}}%
\pgfpathlineto{\pgfqpoint{3.092869in}{0.797980in}}%
\pgfpathlineto{\pgfqpoint{3.093165in}{0.797964in}}%
\pgfpathlineto{\pgfqpoint{3.093461in}{0.797949in}}%
\pgfpathlineto{\pgfqpoint{3.093757in}{0.797934in}}%
\pgfpathlineto{\pgfqpoint{3.094053in}{0.797919in}}%
\pgfpathlineto{\pgfqpoint{3.094349in}{0.797904in}}%
\pgfpathlineto{\pgfqpoint{3.094645in}{0.797889in}}%
\pgfpathlineto{\pgfqpoint{3.094941in}{0.797873in}}%
\pgfpathlineto{\pgfqpoint{3.095237in}{0.797858in}}%
\pgfpathlineto{\pgfqpoint{3.095533in}{0.797843in}}%
\pgfpathlineto{\pgfqpoint{3.095829in}{0.797828in}}%
\pgfpathlineto{\pgfqpoint{3.096125in}{0.797813in}}%
\pgfpathlineto{\pgfqpoint{3.096421in}{0.797798in}}%
\pgfpathlineto{\pgfqpoint{3.096717in}{0.797782in}}%
\pgfpathlineto{\pgfqpoint{3.097013in}{0.797767in}}%
\pgfpathlineto{\pgfqpoint{3.097309in}{0.797752in}}%
\pgfpathlineto{\pgfqpoint{3.097605in}{0.797737in}}%
\pgfpathlineto{\pgfqpoint{3.097901in}{0.797731in}}%
\pgfpathlineto{\pgfqpoint{3.098197in}{0.797728in}}%
\pgfpathlineto{\pgfqpoint{3.098493in}{0.797725in}}%
\pgfpathlineto{\pgfqpoint{3.098789in}{0.797722in}}%
\pgfpathlineto{\pgfqpoint{3.099085in}{0.797719in}}%
\pgfpathlineto{\pgfqpoint{3.099381in}{0.797716in}}%
\pgfpathlineto{\pgfqpoint{3.099677in}{0.797713in}}%
\pgfpathlineto{\pgfqpoint{3.099973in}{0.797710in}}%
\pgfpathlineto{\pgfqpoint{3.100269in}{0.797707in}}%
\pgfpathlineto{\pgfqpoint{3.100565in}{0.797704in}}%
\pgfpathlineto{\pgfqpoint{3.100861in}{0.797701in}}%
\pgfpathlineto{\pgfqpoint{3.101157in}{0.797698in}}%
\pgfpathlineto{\pgfqpoint{3.101453in}{0.797695in}}%
\pgfpathlineto{\pgfqpoint{3.101749in}{0.797692in}}%
\pgfpathlineto{\pgfqpoint{3.102045in}{0.797689in}}%
\pgfpathlineto{\pgfqpoint{3.102341in}{0.797686in}}%
\pgfpathlineto{\pgfqpoint{3.102637in}{0.797683in}}%
\pgfpathlineto{\pgfqpoint{3.102933in}{0.797680in}}%
\pgfpathlineto{\pgfqpoint{3.103229in}{0.797677in}}%
\pgfpathlineto{\pgfqpoint{3.103525in}{0.797674in}}%
\pgfpathlineto{\pgfqpoint{3.103821in}{0.797671in}}%
\pgfpathlineto{\pgfqpoint{3.104117in}{0.797668in}}%
\pgfpathlineto{\pgfqpoint{3.104413in}{0.798214in}}%
\pgfpathlineto{\pgfqpoint{3.104709in}{0.798660in}}%
\pgfpathlineto{\pgfqpoint{3.105005in}{0.798657in}}%
\pgfpathlineto{\pgfqpoint{3.105301in}{0.798654in}}%
\pgfpathlineto{\pgfqpoint{3.105597in}{0.798652in}}%
\pgfpathlineto{\pgfqpoint{3.105893in}{0.798649in}}%
\pgfpathlineto{\pgfqpoint{3.106189in}{0.798641in}}%
\pgfpathlineto{\pgfqpoint{3.106485in}{0.798629in}}%
\pgfpathlineto{\pgfqpoint{3.106781in}{0.798618in}}%
\pgfpathlineto{\pgfqpoint{3.107077in}{0.798607in}}%
\pgfpathlineto{\pgfqpoint{3.107373in}{0.798595in}}%
\pgfpathlineto{\pgfqpoint{3.107669in}{0.798584in}}%
\pgfpathlineto{\pgfqpoint{3.107965in}{0.798573in}}%
\pgfpathlineto{\pgfqpoint{3.108261in}{0.798562in}}%
\pgfpathlineto{\pgfqpoint{3.108557in}{0.798550in}}%
\pgfpathlineto{\pgfqpoint{3.108853in}{0.798539in}}%
\pgfpathlineto{\pgfqpoint{3.109149in}{0.798528in}}%
\pgfpathlineto{\pgfqpoint{3.109445in}{0.798516in}}%
\pgfpathlineto{\pgfqpoint{3.109741in}{0.798505in}}%
\pgfpathlineto{\pgfqpoint{3.110037in}{0.798494in}}%
\pgfpathlineto{\pgfqpoint{3.110333in}{0.798482in}}%
\pgfpathlineto{\pgfqpoint{3.110629in}{0.798471in}}%
\pgfpathlineto{\pgfqpoint{3.110925in}{0.798456in}}%
\pgfpathlineto{\pgfqpoint{3.111221in}{0.798387in}}%
\pgfpathlineto{\pgfqpoint{3.111517in}{0.798301in}}%
\pgfpathlineto{\pgfqpoint{3.111813in}{0.798215in}}%
\pgfpathlineto{\pgfqpoint{3.112109in}{0.800032in}}%
\pgfpathlineto{\pgfqpoint{3.112405in}{0.820875in}}%
\pgfpathlineto{\pgfqpoint{3.112701in}{0.820791in}}%
\pgfpathlineto{\pgfqpoint{3.112997in}{0.820614in}}%
\pgfpathlineto{\pgfqpoint{3.113293in}{0.820219in}}%
\pgfpathlineto{\pgfqpoint{3.113589in}{0.819814in}}%
\pgfpathlineto{\pgfqpoint{3.113885in}{0.818644in}}%
\pgfpathlineto{\pgfqpoint{3.114181in}{0.818029in}}%
\pgfpathlineto{\pgfqpoint{3.114477in}{0.817702in}}%
\pgfpathlineto{\pgfqpoint{3.114773in}{0.817375in}}%
\pgfpathlineto{\pgfqpoint{3.115069in}{0.817048in}}%
\pgfpathlineto{\pgfqpoint{3.115365in}{0.816721in}}%
\pgfpathlineto{\pgfqpoint{3.115661in}{0.816394in}}%
\pgfpathlineto{\pgfqpoint{3.115957in}{0.816067in}}%
\pgfpathlineto{\pgfqpoint{3.116253in}{0.815740in}}%
\pgfpathlineto{\pgfqpoint{3.116549in}{0.815413in}}%
\pgfpathlineto{\pgfqpoint{3.116845in}{0.815087in}}%
\pgfpathlineto{\pgfqpoint{3.117141in}{0.814760in}}%
\pgfpathlineto{\pgfqpoint{3.117437in}{0.814433in}}%
\pgfpathlineto{\pgfqpoint{3.117733in}{0.814106in}}%
\pgfpathlineto{\pgfqpoint{3.118029in}{0.813779in}}%
\pgfpathlineto{\pgfqpoint{3.118325in}{0.813452in}}%
\pgfpathlineto{\pgfqpoint{3.118621in}{0.813125in}}%
\pgfpathlineto{\pgfqpoint{3.118917in}{0.812798in}}%
\pgfpathlineto{\pgfqpoint{3.119213in}{0.812472in}}%
\pgfpathlineto{\pgfqpoint{3.119509in}{0.812145in}}%
\pgfpathlineto{\pgfqpoint{3.119805in}{0.811818in}}%
\pgfpathlineto{\pgfqpoint{3.120101in}{0.811491in}}%
\pgfpathlineto{\pgfqpoint{3.120397in}{0.811164in}}%
\pgfpathlineto{\pgfqpoint{3.120693in}{0.810837in}}%
\pgfpathlineto{\pgfqpoint{3.120989in}{0.810510in}}%
\pgfpathlineto{\pgfqpoint{3.121285in}{0.810183in}}%
\pgfpathlineto{\pgfqpoint{3.121581in}{0.809856in}}%
\pgfpathlineto{\pgfqpoint{3.121877in}{0.809530in}}%
\pgfpathlineto{\pgfqpoint{3.122173in}{0.809203in}}%
\pgfpathlineto{\pgfqpoint{3.122469in}{0.808876in}}%
\pgfpathlineto{\pgfqpoint{3.122765in}{0.808549in}}%
\pgfpathlineto{\pgfqpoint{3.123061in}{0.808222in}}%
\pgfpathlineto{\pgfqpoint{3.123357in}{0.807895in}}%
\pgfpathlineto{\pgfqpoint{3.123653in}{0.807568in}}%
\pgfpathlineto{\pgfqpoint{3.123949in}{0.807241in}}%
\pgfpathlineto{\pgfqpoint{3.124245in}{0.806915in}}%
\pgfpathlineto{\pgfqpoint{3.124541in}{0.806588in}}%
\pgfpathlineto{\pgfqpoint{3.124837in}{0.806261in}}%
\pgfpathlineto{\pgfqpoint{3.125133in}{0.805934in}}%
\pgfpathlineto{\pgfqpoint{3.125429in}{0.805607in}}%
\pgfpathlineto{\pgfqpoint{3.125725in}{0.805280in}}%
\pgfpathlineto{\pgfqpoint{3.126021in}{0.804953in}}%
\pgfpathlineto{\pgfqpoint{3.126317in}{0.804626in}}%
\pgfpathlineto{\pgfqpoint{3.126613in}{0.804299in}}%
\pgfpathlineto{\pgfqpoint{3.126909in}{0.803973in}}%
\pgfpathlineto{\pgfqpoint{3.127205in}{0.803646in}}%
\pgfpathlineto{\pgfqpoint{3.127501in}{0.803319in}}%
\pgfpathlineto{\pgfqpoint{3.127797in}{0.802992in}}%
\pgfpathlineto{\pgfqpoint{3.128093in}{0.802665in}}%
\pgfpathlineto{\pgfqpoint{3.128389in}{0.802338in}}%
\pgfpathlineto{\pgfqpoint{3.128685in}{0.802011in}}%
\pgfpathlineto{\pgfqpoint{3.128981in}{0.801684in}}%
\pgfpathlineto{\pgfqpoint{3.129277in}{0.801357in}}%
\pgfpathlineto{\pgfqpoint{3.129573in}{0.801031in}}%
\pgfpathlineto{\pgfqpoint{3.129869in}{0.800704in}}%
\pgfpathlineto{\pgfqpoint{3.130165in}{0.800377in}}%
\pgfpathlineto{\pgfqpoint{3.130461in}{0.800050in}}%
\pgfpathlineto{\pgfqpoint{3.130757in}{0.799723in}}%
\pgfpathlineto{\pgfqpoint{3.131053in}{0.799396in}}%
\pgfpathlineto{\pgfqpoint{3.131349in}{0.799069in}}%
\pgfpathlineto{\pgfqpoint{3.131645in}{0.798742in}}%
\pgfpathlineto{\pgfqpoint{3.131941in}{0.798416in}}%
\pgfpathlineto{\pgfqpoint{3.132237in}{0.798089in}}%
\pgfpathlineto{\pgfqpoint{3.132533in}{0.797762in}}%
\pgfpathlineto{\pgfqpoint{3.132829in}{0.797435in}}%
\pgfpathlineto{\pgfqpoint{3.133125in}{0.797108in}}%
\pgfpathlineto{\pgfqpoint{3.133421in}{0.796781in}}%
\pgfpathlineto{\pgfqpoint{3.133717in}{0.796808in}}%
\pgfpathlineto{\pgfqpoint{3.134013in}{0.797336in}}%
\pgfpathlineto{\pgfqpoint{3.134309in}{0.797829in}}%
\pgfpathlineto{\pgfqpoint{3.134605in}{0.798143in}}%
\pgfpathlineto{\pgfqpoint{3.134901in}{0.798436in}}%
\pgfpathlineto{\pgfqpoint{3.135197in}{0.798729in}}%
\pgfpathlineto{\pgfqpoint{3.135493in}{0.799022in}}%
\pgfpathlineto{\pgfqpoint{3.135789in}{0.799315in}}%
\pgfpathlineto{\pgfqpoint{3.136085in}{0.799608in}}%
\pgfpathlineto{\pgfqpoint{3.136381in}{0.799901in}}%
\pgfpathlineto{\pgfqpoint{3.136677in}{0.800194in}}%
\pgfpathlineto{\pgfqpoint{3.136973in}{0.800487in}}%
\pgfpathlineto{\pgfqpoint{3.137269in}{0.800780in}}%
\pgfpathlineto{\pgfqpoint{3.137565in}{0.801073in}}%
\pgfpathlineto{\pgfqpoint{3.137861in}{0.801366in}}%
\pgfpathlineto{\pgfqpoint{3.138157in}{0.801659in}}%
\pgfpathlineto{\pgfqpoint{3.138453in}{0.801952in}}%
\pgfpathlineto{\pgfqpoint{3.138749in}{0.802245in}}%
\pgfpathlineto{\pgfqpoint{3.139045in}{0.802538in}}%
\pgfpathlineto{\pgfqpoint{3.139341in}{0.802831in}}%
\pgfpathlineto{\pgfqpoint{3.139637in}{0.803124in}}%
\pgfpathlineto{\pgfqpoint{3.139933in}{0.803417in}}%
\pgfpathlineto{\pgfqpoint{3.140229in}{0.803710in}}%
\pgfpathlineto{\pgfqpoint{3.140525in}{0.804003in}}%
\pgfpathlineto{\pgfqpoint{3.140821in}{0.804296in}}%
\pgfpathlineto{\pgfqpoint{3.141117in}{0.804588in}}%
\pgfpathlineto{\pgfqpoint{3.141413in}{0.804881in}}%
\pgfpathlineto{\pgfqpoint{3.141709in}{0.805174in}}%
\pgfpathlineto{\pgfqpoint{3.142005in}{0.805467in}}%
\pgfpathlineto{\pgfqpoint{3.142301in}{0.805760in}}%
\pgfpathlineto{\pgfqpoint{3.142598in}{0.806053in}}%
\pgfpathlineto{\pgfqpoint{3.142894in}{0.806346in}}%
\pgfpathlineto{\pgfqpoint{3.143190in}{0.806639in}}%
\pgfpathlineto{\pgfqpoint{3.143486in}{0.806932in}}%
\pgfpathlineto{\pgfqpoint{3.143782in}{0.807225in}}%
\pgfpathlineto{\pgfqpoint{3.144078in}{0.807518in}}%
\pgfpathlineto{\pgfqpoint{3.144374in}{0.807811in}}%
\pgfpathlineto{\pgfqpoint{3.144670in}{0.808104in}}%
\pgfpathlineto{\pgfqpoint{3.144966in}{0.808397in}}%
\pgfpathlineto{\pgfqpoint{3.145262in}{0.808690in}}%
\pgfpathlineto{\pgfqpoint{3.145558in}{0.808983in}}%
\pgfpathlineto{\pgfqpoint{3.145854in}{0.809276in}}%
\pgfpathlineto{\pgfqpoint{3.146150in}{0.809569in}}%
\pgfpathlineto{\pgfqpoint{3.146446in}{0.809862in}}%
\pgfpathlineto{\pgfqpoint{3.146742in}{0.810155in}}%
\pgfpathlineto{\pgfqpoint{3.147038in}{0.810448in}}%
\pgfpathlineto{\pgfqpoint{3.147334in}{0.810741in}}%
\pgfpathlineto{\pgfqpoint{3.147630in}{0.811034in}}%
\pgfpathlineto{\pgfqpoint{3.147926in}{0.811327in}}%
\pgfpathlineto{\pgfqpoint{3.148222in}{0.811620in}}%
\pgfpathlineto{\pgfqpoint{3.148518in}{0.811913in}}%
\pgfpathlineto{\pgfqpoint{3.148814in}{0.812206in}}%
\pgfpathlineto{\pgfqpoint{3.149110in}{0.812499in}}%
\pgfpathlineto{\pgfqpoint{3.149406in}{0.812792in}}%
\pgfpathlineto{\pgfqpoint{3.149702in}{0.813085in}}%
\pgfpathlineto{\pgfqpoint{3.149998in}{0.813378in}}%
\pgfpathlineto{\pgfqpoint{3.150294in}{0.813671in}}%
\pgfpathlineto{\pgfqpoint{3.150590in}{0.813964in}}%
\pgfpathlineto{\pgfqpoint{3.150886in}{0.814257in}}%
\pgfpathlineto{\pgfqpoint{3.151182in}{0.814550in}}%
\pgfpathlineto{\pgfqpoint{3.151478in}{0.814843in}}%
\pgfpathlineto{\pgfqpoint{3.151774in}{0.815136in}}%
\pgfpathlineto{\pgfqpoint{3.152070in}{0.815429in}}%
\pgfpathlineto{\pgfqpoint{3.152366in}{0.815722in}}%
\pgfpathlineto{\pgfqpoint{3.152662in}{0.816014in}}%
\pgfpathlineto{\pgfqpoint{3.152958in}{0.816307in}}%
\pgfpathlineto{\pgfqpoint{3.153254in}{0.816600in}}%
\pgfpathlineto{\pgfqpoint{3.153550in}{0.816893in}}%
\pgfpathlineto{\pgfqpoint{3.153846in}{0.817186in}}%
\pgfpathlineto{\pgfqpoint{3.154142in}{0.817479in}}%
\pgfpathlineto{\pgfqpoint{3.154438in}{0.817772in}}%
\pgfpathlineto{\pgfqpoint{3.154734in}{0.818065in}}%
\pgfpathlineto{\pgfqpoint{3.155030in}{0.815886in}}%
\pgfpathlineto{\pgfqpoint{3.155326in}{0.807602in}}%
\pgfpathlineto{\pgfqpoint{3.155622in}{0.807157in}}%
\pgfpathlineto{\pgfqpoint{3.155918in}{0.806712in}}%
\pgfpathlineto{\pgfqpoint{3.156214in}{0.806268in}}%
\pgfpathlineto{\pgfqpoint{3.156510in}{0.805820in}}%
\pgfpathlineto{\pgfqpoint{3.156806in}{0.805370in}}%
\pgfpathlineto{\pgfqpoint{3.157102in}{0.804920in}}%
\pgfpathlineto{\pgfqpoint{3.157398in}{0.804471in}}%
\pgfpathlineto{\pgfqpoint{3.157694in}{0.804021in}}%
\pgfpathlineto{\pgfqpoint{3.157990in}{0.803571in}}%
\pgfpathlineto{\pgfqpoint{3.158286in}{0.803121in}}%
\pgfpathlineto{\pgfqpoint{3.158582in}{0.802672in}}%
\pgfpathlineto{\pgfqpoint{3.158878in}{0.802222in}}%
\pgfpathlineto{\pgfqpoint{3.159174in}{0.801772in}}%
\pgfpathlineto{\pgfqpoint{3.159470in}{0.801322in}}%
\pgfpathlineto{\pgfqpoint{3.159766in}{0.800872in}}%
\pgfpathlineto{\pgfqpoint{3.160062in}{0.800423in}}%
\pgfpathlineto{\pgfqpoint{3.160358in}{0.799973in}}%
\pgfpathlineto{\pgfqpoint{3.160654in}{0.799497in}}%
\pgfpathlineto{\pgfqpoint{3.160950in}{0.798696in}}%
\pgfpathlineto{\pgfqpoint{3.161246in}{0.797797in}}%
\pgfpathlineto{\pgfqpoint{3.161542in}{0.797141in}}%
\pgfpathlineto{\pgfqpoint{3.161838in}{0.798518in}}%
\pgfpathlineto{\pgfqpoint{3.162134in}{0.799000in}}%
\pgfpathlineto{\pgfqpoint{3.162430in}{0.798472in}}%
\pgfpathlineto{\pgfqpoint{3.162726in}{0.797944in}}%
\pgfpathlineto{\pgfqpoint{3.163022in}{0.797425in}}%
\pgfpathlineto{\pgfqpoint{3.163318in}{0.797307in}}%
\pgfpathlineto{\pgfqpoint{3.163614in}{0.797400in}}%
\pgfpathlineto{\pgfqpoint{3.163910in}{0.797493in}}%
\pgfpathlineto{\pgfqpoint{3.164206in}{0.797586in}}%
\pgfpathlineto{\pgfqpoint{3.164502in}{0.797679in}}%
\pgfpathlineto{\pgfqpoint{3.164798in}{0.797772in}}%
\pgfpathlineto{\pgfqpoint{3.165094in}{0.797865in}}%
\pgfpathlineto{\pgfqpoint{3.165390in}{0.797958in}}%
\pgfpathlineto{\pgfqpoint{3.165686in}{0.798051in}}%
\pgfpathlineto{\pgfqpoint{3.165982in}{0.798144in}}%
\pgfpathlineto{\pgfqpoint{3.166278in}{0.798237in}}%
\pgfpathlineto{\pgfqpoint{3.166574in}{0.798330in}}%
\pgfpathlineto{\pgfqpoint{3.166870in}{0.798423in}}%
\pgfpathlineto{\pgfqpoint{3.167166in}{0.798516in}}%
\pgfpathlineto{\pgfqpoint{3.167462in}{0.798609in}}%
\pgfpathlineto{\pgfqpoint{3.167758in}{0.798702in}}%
\pgfpathlineto{\pgfqpoint{3.168054in}{0.798795in}}%
\pgfpathlineto{\pgfqpoint{3.168350in}{0.798888in}}%
\pgfpathlineto{\pgfqpoint{3.168646in}{0.798981in}}%
\pgfpathlineto{\pgfqpoint{3.168942in}{0.799074in}}%
\pgfpathlineto{\pgfqpoint{3.169238in}{0.799167in}}%
\pgfpathlineto{\pgfqpoint{3.169534in}{0.799235in}}%
\pgfpathlineto{\pgfqpoint{3.169830in}{0.799296in}}%
\pgfpathlineto{\pgfqpoint{3.170126in}{0.799304in}}%
\pgfpathlineto{\pgfqpoint{3.170422in}{0.799293in}}%
\pgfpathlineto{\pgfqpoint{3.170718in}{0.799283in}}%
\pgfpathlineto{\pgfqpoint{3.171014in}{0.799270in}}%
\pgfpathlineto{\pgfqpoint{3.171310in}{0.799257in}}%
\pgfpathlineto{\pgfqpoint{3.171606in}{0.799245in}}%
\pgfpathlineto{\pgfqpoint{3.171902in}{0.799232in}}%
\pgfpathlineto{\pgfqpoint{3.172198in}{0.799220in}}%
\pgfpathlineto{\pgfqpoint{3.172494in}{0.799207in}}%
\pgfpathlineto{\pgfqpoint{3.172790in}{0.799195in}}%
\pgfpathlineto{\pgfqpoint{3.173086in}{0.799182in}}%
\pgfpathlineto{\pgfqpoint{3.173382in}{0.799170in}}%
\pgfpathlineto{\pgfqpoint{3.173678in}{0.799157in}}%
\pgfpathlineto{\pgfqpoint{3.173974in}{0.799145in}}%
\pgfpathlineto{\pgfqpoint{3.174270in}{0.799133in}}%
\pgfpathlineto{\pgfqpoint{3.174566in}{0.799120in}}%
\pgfpathlineto{\pgfqpoint{3.174862in}{0.799108in}}%
\pgfpathlineto{\pgfqpoint{3.175158in}{0.799095in}}%
\pgfpathlineto{\pgfqpoint{3.175454in}{0.799083in}}%
\pgfpathlineto{\pgfqpoint{3.175750in}{0.799071in}}%
\pgfpathlineto{\pgfqpoint{3.176046in}{0.799064in}}%
\pgfpathlineto{\pgfqpoint{3.176342in}{0.799058in}}%
\pgfpathlineto{\pgfqpoint{3.176638in}{0.799052in}}%
\pgfpathlineto{\pgfqpoint{3.176934in}{0.799047in}}%
\pgfpathlineto{\pgfqpoint{3.177230in}{0.799042in}}%
\pgfpathlineto{\pgfqpoint{3.177526in}{0.799037in}}%
\pgfpathlineto{\pgfqpoint{3.177822in}{0.799032in}}%
\pgfpathlineto{\pgfqpoint{3.178118in}{0.799027in}}%
\pgfpathlineto{\pgfqpoint{3.178414in}{0.799021in}}%
\pgfpathlineto{\pgfqpoint{3.178710in}{0.799016in}}%
\pgfpathlineto{\pgfqpoint{3.179006in}{0.799011in}}%
\pgfpathlineto{\pgfqpoint{3.179302in}{0.799006in}}%
\pgfpathlineto{\pgfqpoint{3.179598in}{0.799000in}}%
\pgfpathlineto{\pgfqpoint{3.179894in}{0.798995in}}%
\pgfpathlineto{\pgfqpoint{3.180190in}{0.798990in}}%
\pgfpathlineto{\pgfqpoint{3.180486in}{0.798985in}}%
\pgfpathlineto{\pgfqpoint{3.180782in}{0.798979in}}%
\pgfpathlineto{\pgfqpoint{3.181078in}{0.798974in}}%
\pgfpathlineto{\pgfqpoint{3.181374in}{0.798969in}}%
\pgfpathlineto{\pgfqpoint{3.181670in}{0.798964in}}%
\pgfpathlineto{\pgfqpoint{3.181966in}{0.798950in}}%
\pgfpathlineto{\pgfqpoint{3.182262in}{0.797965in}}%
\pgfpathlineto{\pgfqpoint{3.182558in}{0.799054in}}%
\pgfpathlineto{\pgfqpoint{3.182854in}{0.799014in}}%
\pgfpathlineto{\pgfqpoint{3.183150in}{0.798994in}}%
\pgfpathlineto{\pgfqpoint{3.183446in}{0.798975in}}%
\pgfpathlineto{\pgfqpoint{3.183742in}{0.798955in}}%
\pgfpathlineto{\pgfqpoint{3.184038in}{0.798936in}}%
\pgfpathlineto{\pgfqpoint{3.184334in}{0.798961in}}%
\pgfpathlineto{\pgfqpoint{3.184630in}{0.799000in}}%
\pgfpathlineto{\pgfqpoint{3.184926in}{0.798995in}}%
\pgfpathlineto{\pgfqpoint{3.185222in}{0.798989in}}%
\pgfpathlineto{\pgfqpoint{3.185518in}{0.798984in}}%
\pgfpathlineto{\pgfqpoint{3.185814in}{0.798978in}}%
\pgfpathlineto{\pgfqpoint{3.186110in}{0.798973in}}%
\pgfpathlineto{\pgfqpoint{3.186406in}{0.798967in}}%
\pgfpathlineto{\pgfqpoint{3.186702in}{0.798962in}}%
\pgfpathlineto{\pgfqpoint{3.186998in}{0.798956in}}%
\pgfpathlineto{\pgfqpoint{3.187294in}{0.798951in}}%
\pgfpathlineto{\pgfqpoint{3.187590in}{0.798945in}}%
\pgfpathlineto{\pgfqpoint{3.187886in}{0.798940in}}%
\pgfpathlineto{\pgfqpoint{3.188182in}{0.798934in}}%
\pgfpathlineto{\pgfqpoint{3.188478in}{0.798929in}}%
\pgfpathlineto{\pgfqpoint{3.188774in}{0.798923in}}%
\pgfpathlineto{\pgfqpoint{3.189070in}{0.798918in}}%
\pgfpathlineto{\pgfqpoint{3.189366in}{0.798912in}}%
\pgfpathlineto{\pgfqpoint{3.189662in}{0.798907in}}%
\pgfpathlineto{\pgfqpoint{3.189958in}{0.798901in}}%
\pgfpathlineto{\pgfqpoint{3.190254in}{0.798896in}}%
\pgfpathlineto{\pgfqpoint{3.190550in}{0.798891in}}%
\pgfpathlineto{\pgfqpoint{3.190846in}{0.798885in}}%
\pgfpathlineto{\pgfqpoint{3.191142in}{0.798880in}}%
\pgfpathlineto{\pgfqpoint{3.191438in}{0.798874in}}%
\pgfpathlineto{\pgfqpoint{3.191734in}{0.798869in}}%
\pgfpathlineto{\pgfqpoint{3.192030in}{0.798863in}}%
\pgfpathlineto{\pgfqpoint{3.192326in}{0.798858in}}%
\pgfpathlineto{\pgfqpoint{3.192622in}{0.798852in}}%
\pgfpathlineto{\pgfqpoint{3.192918in}{0.798847in}}%
\pgfpathlineto{\pgfqpoint{3.193214in}{0.798841in}}%
\pgfpathlineto{\pgfqpoint{3.193510in}{0.798836in}}%
\pgfpathlineto{\pgfqpoint{3.193806in}{0.798830in}}%
\pgfpathlineto{\pgfqpoint{3.194102in}{0.798825in}}%
\pgfpathlineto{\pgfqpoint{3.194398in}{0.798819in}}%
\pgfpathlineto{\pgfqpoint{3.194694in}{0.798814in}}%
\pgfpathlineto{\pgfqpoint{3.194990in}{0.798808in}}%
\pgfpathlineto{\pgfqpoint{3.195286in}{0.798803in}}%
\pgfpathlineto{\pgfqpoint{3.195582in}{0.798797in}}%
\pgfpathlineto{\pgfqpoint{3.195878in}{0.798792in}}%
\pgfpathlineto{\pgfqpoint{3.196174in}{0.798787in}}%
\pgfpathlineto{\pgfqpoint{3.196470in}{0.798781in}}%
\pgfpathlineto{\pgfqpoint{3.196766in}{0.798776in}}%
\pgfpathlineto{\pgfqpoint{3.197062in}{0.798770in}}%
\pgfpathlineto{\pgfqpoint{3.197358in}{0.798765in}}%
\pgfpathlineto{\pgfqpoint{3.197654in}{0.798759in}}%
\pgfpathlineto{\pgfqpoint{3.197950in}{0.798754in}}%
\pgfpathlineto{\pgfqpoint{3.198246in}{0.798748in}}%
\pgfpathlineto{\pgfqpoint{3.198542in}{0.798743in}}%
\pgfpathlineto{\pgfqpoint{3.198838in}{0.798737in}}%
\pgfpathlineto{\pgfqpoint{3.199134in}{0.798732in}}%
\pgfpathlineto{\pgfqpoint{3.199430in}{0.798726in}}%
\pgfpathlineto{\pgfqpoint{3.199726in}{0.798721in}}%
\pgfpathlineto{\pgfqpoint{3.200022in}{0.798715in}}%
\pgfpathlineto{\pgfqpoint{3.200318in}{0.798710in}}%
\pgfpathlineto{\pgfqpoint{3.200614in}{0.798704in}}%
\pgfpathlineto{\pgfqpoint{3.200910in}{0.798699in}}%
\pgfpathlineto{\pgfqpoint{3.201206in}{0.798693in}}%
\pgfpathlineto{\pgfqpoint{3.201502in}{0.798688in}}%
\pgfpathlineto{\pgfqpoint{3.201798in}{0.798683in}}%
\pgfpathlineto{\pgfqpoint{3.202094in}{0.798677in}}%
\pgfpathlineto{\pgfqpoint{3.202390in}{0.798672in}}%
\pgfpathlineto{\pgfqpoint{3.202686in}{0.798666in}}%
\pgfpathlineto{\pgfqpoint{3.202982in}{0.798661in}}%
\pgfpathlineto{\pgfqpoint{3.203278in}{0.798655in}}%
\pgfpathlineto{\pgfqpoint{3.203574in}{0.798650in}}%
\pgfpathlineto{\pgfqpoint{3.203870in}{0.798676in}}%
\pgfpathlineto{\pgfqpoint{3.204166in}{0.798795in}}%
\pgfpathlineto{\pgfqpoint{3.204462in}{0.798754in}}%
\pgfpathlineto{\pgfqpoint{3.204758in}{0.798684in}}%
\pgfpathlineto{\pgfqpoint{3.205054in}{0.798613in}}%
\pgfpathlineto{\pgfqpoint{3.205350in}{0.798543in}}%
\pgfpathlineto{\pgfqpoint{3.205646in}{0.798406in}}%
\pgfpathlineto{\pgfqpoint{3.205942in}{0.798019in}}%
\pgfpathlineto{\pgfqpoint{3.206238in}{0.797938in}}%
\pgfpathlineto{\pgfqpoint{3.206534in}{0.797906in}}%
\pgfpathlineto{\pgfqpoint{3.206830in}{0.797874in}}%
\pgfpathlineto{\pgfqpoint{3.207126in}{0.797842in}}%
\pgfpathlineto{\pgfqpoint{3.207422in}{0.797810in}}%
\pgfpathlineto{\pgfqpoint{3.207718in}{0.797778in}}%
\pgfpathlineto{\pgfqpoint{3.208014in}{0.797746in}}%
\pgfpathlineto{\pgfqpoint{3.208310in}{0.797714in}}%
\pgfpathlineto{\pgfqpoint{3.208606in}{0.797682in}}%
\pgfpathlineto{\pgfqpoint{3.208902in}{0.797650in}}%
\pgfpathlineto{\pgfqpoint{3.209198in}{0.797618in}}%
\pgfpathlineto{\pgfqpoint{3.209494in}{0.797586in}}%
\pgfpathlineto{\pgfqpoint{3.209791in}{0.797554in}}%
\pgfpathlineto{\pgfqpoint{3.210087in}{0.797522in}}%
\pgfpathlineto{\pgfqpoint{3.210383in}{0.797490in}}%
\pgfpathlineto{\pgfqpoint{3.210679in}{0.797193in}}%
\pgfpathlineto{\pgfqpoint{3.210975in}{0.796431in}}%
\pgfpathlineto{\pgfqpoint{3.211271in}{0.796373in}}%
\pgfpathlineto{\pgfqpoint{3.211567in}{0.796316in}}%
\pgfpathlineto{\pgfqpoint{3.211863in}{0.796259in}}%
\pgfpathlineto{\pgfqpoint{3.212159in}{0.796202in}}%
\pgfpathlineto{\pgfqpoint{3.212455in}{0.796103in}}%
\pgfpathlineto{\pgfqpoint{3.212751in}{0.795826in}}%
\pgfpathlineto{\pgfqpoint{3.213047in}{0.795794in}}%
\pgfpathlineto{\pgfqpoint{3.213343in}{0.795770in}}%
\pgfpathlineto{\pgfqpoint{3.213639in}{0.795746in}}%
\pgfpathlineto{\pgfqpoint{3.213935in}{0.795722in}}%
\pgfpathlineto{\pgfqpoint{3.214231in}{0.795698in}}%
\pgfpathlineto{\pgfqpoint{3.214527in}{0.795674in}}%
\pgfpathlineto{\pgfqpoint{3.214823in}{0.795650in}}%
\pgfpathlineto{\pgfqpoint{3.215119in}{0.795626in}}%
\pgfpathlineto{\pgfqpoint{3.215415in}{0.795602in}}%
\pgfpathlineto{\pgfqpoint{3.215711in}{0.795578in}}%
\pgfpathlineto{\pgfqpoint{3.216007in}{0.795554in}}%
\pgfpathlineto{\pgfqpoint{3.216303in}{0.795530in}}%
\pgfpathlineto{\pgfqpoint{3.216599in}{0.795506in}}%
\pgfpathlineto{\pgfqpoint{3.216895in}{0.795482in}}%
\pgfpathlineto{\pgfqpoint{3.217191in}{0.795458in}}%
\pgfpathlineto{\pgfqpoint{3.217487in}{0.806436in}}%
\pgfpathlineto{\pgfqpoint{3.217783in}{0.818168in}}%
\pgfpathlineto{\pgfqpoint{3.218079in}{0.818136in}}%
\pgfpathlineto{\pgfqpoint{3.218375in}{0.818105in}}%
\pgfpathlineto{\pgfqpoint{3.218671in}{0.818074in}}%
\pgfpathlineto{\pgfqpoint{3.218967in}{0.818045in}}%
\pgfpathlineto{\pgfqpoint{3.219263in}{0.818021in}}%
\pgfpathlineto{\pgfqpoint{3.219559in}{0.817997in}}%
\pgfpathlineto{\pgfqpoint{3.219855in}{0.817973in}}%
\pgfpathlineto{\pgfqpoint{3.220151in}{0.817949in}}%
\pgfpathlineto{\pgfqpoint{3.220447in}{0.817925in}}%
\pgfpathlineto{\pgfqpoint{3.220743in}{0.817901in}}%
\pgfpathlineto{\pgfqpoint{3.221039in}{0.817877in}}%
\pgfpathlineto{\pgfqpoint{3.221335in}{0.817853in}}%
\pgfpathlineto{\pgfqpoint{3.221631in}{0.817829in}}%
\pgfpathlineto{\pgfqpoint{3.221927in}{0.817805in}}%
\pgfpathlineto{\pgfqpoint{3.222223in}{0.817781in}}%
\pgfpathlineto{\pgfqpoint{3.222519in}{0.817757in}}%
\pgfpathlineto{\pgfqpoint{3.222815in}{0.817733in}}%
\pgfpathlineto{\pgfqpoint{3.223111in}{0.817709in}}%
\pgfpathlineto{\pgfqpoint{3.223407in}{0.817685in}}%
\pgfpathlineto{\pgfqpoint{3.223703in}{0.817661in}}%
\pgfpathlineto{\pgfqpoint{3.223999in}{0.817637in}}%
\pgfpathlineto{\pgfqpoint{3.224295in}{0.817613in}}%
\pgfpathlineto{\pgfqpoint{3.224591in}{0.817589in}}%
\pgfpathlineto{\pgfqpoint{3.224887in}{0.817565in}}%
\pgfpathlineto{\pgfqpoint{3.225183in}{0.817541in}}%
\pgfpathlineto{\pgfqpoint{3.225479in}{0.817517in}}%
\pgfpathlineto{\pgfqpoint{3.225775in}{0.817493in}}%
\pgfpathlineto{\pgfqpoint{3.226071in}{0.817469in}}%
\pgfpathlineto{\pgfqpoint{3.226367in}{0.817445in}}%
\pgfpathlineto{\pgfqpoint{3.226663in}{0.817421in}}%
\pgfpathlineto{\pgfqpoint{3.226959in}{0.817397in}}%
\pgfpathlineto{\pgfqpoint{3.227255in}{0.817373in}}%
\pgfpathlineto{\pgfqpoint{3.227551in}{0.817349in}}%
\pgfpathlineto{\pgfqpoint{3.227847in}{0.817325in}}%
\pgfpathlineto{\pgfqpoint{3.228143in}{0.817301in}}%
\pgfpathlineto{\pgfqpoint{3.228439in}{0.817277in}}%
\pgfpathlineto{\pgfqpoint{3.228735in}{0.817253in}}%
\pgfpathlineto{\pgfqpoint{3.229031in}{0.817229in}}%
\pgfpathlineto{\pgfqpoint{3.229327in}{0.817205in}}%
\pgfpathlineto{\pgfqpoint{3.229623in}{0.817181in}}%
\pgfpathlineto{\pgfqpoint{3.229919in}{0.817157in}}%
\pgfpathlineto{\pgfqpoint{3.230215in}{0.817133in}}%
\pgfpathlineto{\pgfqpoint{3.230511in}{0.817109in}}%
\pgfpathlineto{\pgfqpoint{3.230807in}{0.817085in}}%
\pgfpathlineto{\pgfqpoint{3.231103in}{0.817061in}}%
\pgfpathlineto{\pgfqpoint{3.231399in}{0.817037in}}%
\pgfpathlineto{\pgfqpoint{3.231695in}{0.817013in}}%
\pgfpathlineto{\pgfqpoint{3.231991in}{0.816989in}}%
\pgfpathlineto{\pgfqpoint{3.232287in}{0.816965in}}%
\pgfpathlineto{\pgfqpoint{3.232583in}{0.816941in}}%
\pgfpathlineto{\pgfqpoint{3.232879in}{0.816918in}}%
\pgfpathlineto{\pgfqpoint{3.233175in}{0.816901in}}%
\pgfpathlineto{\pgfqpoint{3.233471in}{0.816888in}}%
\pgfpathlineto{\pgfqpoint{3.233767in}{0.816886in}}%
\pgfpathlineto{\pgfqpoint{3.234063in}{0.816883in}}%
\pgfpathlineto{\pgfqpoint{3.234359in}{0.816880in}}%
\pgfpathlineto{\pgfqpoint{3.234655in}{0.816878in}}%
\pgfpathlineto{\pgfqpoint{3.234951in}{0.816875in}}%
\pgfpathlineto{\pgfqpoint{3.235247in}{0.816872in}}%
\pgfpathlineto{\pgfqpoint{3.235543in}{0.816870in}}%
\pgfpathlineto{\pgfqpoint{3.235839in}{0.816867in}}%
\pgfpathlineto{\pgfqpoint{3.236135in}{0.816864in}}%
\pgfpathlineto{\pgfqpoint{3.236431in}{0.816862in}}%
\pgfpathlineto{\pgfqpoint{3.236727in}{0.816859in}}%
\pgfpathlineto{\pgfqpoint{3.237023in}{0.816857in}}%
\pgfpathlineto{\pgfqpoint{3.237319in}{0.816854in}}%
\pgfpathlineto{\pgfqpoint{3.237615in}{0.816851in}}%
\pgfpathlineto{\pgfqpoint{3.237911in}{0.816849in}}%
\pgfpathlineto{\pgfqpoint{3.238207in}{0.816846in}}%
\pgfpathlineto{\pgfqpoint{3.238503in}{0.816843in}}%
\pgfpathlineto{\pgfqpoint{3.238799in}{0.816841in}}%
\pgfpathlineto{\pgfqpoint{3.239095in}{0.816838in}}%
\pgfpathlineto{\pgfqpoint{3.239391in}{0.816836in}}%
\pgfpathlineto{\pgfqpoint{3.239687in}{0.816833in}}%
\pgfpathlineto{\pgfqpoint{3.239983in}{0.816830in}}%
\pgfpathlineto{\pgfqpoint{3.240279in}{0.816828in}}%
\pgfpathlineto{\pgfqpoint{3.240575in}{0.816825in}}%
\pgfpathlineto{\pgfqpoint{3.240871in}{0.816822in}}%
\pgfpathlineto{\pgfqpoint{3.241167in}{0.816820in}}%
\pgfpathlineto{\pgfqpoint{3.241463in}{0.816817in}}%
\pgfpathlineto{\pgfqpoint{3.241759in}{0.816814in}}%
\pgfpathlineto{\pgfqpoint{3.242055in}{0.816812in}}%
\pgfpathlineto{\pgfqpoint{3.242351in}{0.816809in}}%
\pgfpathlineto{\pgfqpoint{3.242647in}{0.816807in}}%
\pgfpathlineto{\pgfqpoint{3.242943in}{0.816804in}}%
\pgfpathlineto{\pgfqpoint{3.243239in}{0.816801in}}%
\pgfpathlineto{\pgfqpoint{3.243535in}{0.816799in}}%
\pgfpathlineto{\pgfqpoint{3.243831in}{0.816796in}}%
\pgfpathlineto{\pgfqpoint{3.244127in}{0.816793in}}%
\pgfpathlineto{\pgfqpoint{3.244423in}{0.816791in}}%
\pgfpathlineto{\pgfqpoint{3.244719in}{0.816788in}}%
\pgfpathlineto{\pgfqpoint{3.245015in}{0.816786in}}%
\pgfpathlineto{\pgfqpoint{3.245311in}{0.816783in}}%
\pgfpathlineto{\pgfqpoint{3.245607in}{0.816780in}}%
\pgfpathlineto{\pgfqpoint{3.245903in}{0.816778in}}%
\pgfpathlineto{\pgfqpoint{3.246199in}{0.816775in}}%
\pgfpathlineto{\pgfqpoint{3.246495in}{0.816772in}}%
\pgfpathlineto{\pgfqpoint{3.246791in}{0.816770in}}%
\pgfpathlineto{\pgfqpoint{3.247087in}{0.816767in}}%
\pgfpathlineto{\pgfqpoint{3.247383in}{0.816764in}}%
\pgfpathlineto{\pgfqpoint{3.247679in}{0.816762in}}%
\pgfpathlineto{\pgfqpoint{3.247975in}{0.816759in}}%
\pgfpathlineto{\pgfqpoint{3.248271in}{0.816757in}}%
\pgfpathlineto{\pgfqpoint{3.248567in}{0.816754in}}%
\pgfpathlineto{\pgfqpoint{3.248863in}{0.816751in}}%
\pgfpathlineto{\pgfqpoint{3.249159in}{0.816749in}}%
\pgfpathlineto{\pgfqpoint{3.249455in}{0.816746in}}%
\pgfpathlineto{\pgfqpoint{3.249751in}{0.816743in}}%
\pgfpathlineto{\pgfqpoint{3.250047in}{0.816741in}}%
\pgfpathlineto{\pgfqpoint{3.250343in}{0.816738in}}%
\pgfpathlineto{\pgfqpoint{3.250639in}{0.816735in}}%
\pgfpathlineto{\pgfqpoint{3.250935in}{0.816733in}}%
\pgfpathlineto{\pgfqpoint{3.251231in}{0.816730in}}%
\pgfpathlineto{\pgfqpoint{3.251527in}{0.816728in}}%
\pgfpathlineto{\pgfqpoint{3.251823in}{0.816725in}}%
\pgfpathlineto{\pgfqpoint{3.252119in}{0.816722in}}%
\pgfpathlineto{\pgfqpoint{3.252415in}{0.816720in}}%
\pgfpathlineto{\pgfqpoint{3.252711in}{0.816717in}}%
\pgfpathlineto{\pgfqpoint{3.253007in}{0.816714in}}%
\pgfpathlineto{\pgfqpoint{3.253303in}{0.816712in}}%
\pgfpathlineto{\pgfqpoint{3.253599in}{0.816709in}}%
\pgfpathlineto{\pgfqpoint{3.253895in}{0.816707in}}%
\pgfpathlineto{\pgfqpoint{3.254191in}{0.816704in}}%
\pgfpathlineto{\pgfqpoint{3.254487in}{0.816701in}}%
\pgfpathlineto{\pgfqpoint{3.254783in}{0.816699in}}%
\pgfpathlineto{\pgfqpoint{3.255079in}{0.816696in}}%
\pgfpathlineto{\pgfqpoint{3.255375in}{0.816693in}}%
\pgfpathlineto{\pgfqpoint{3.255671in}{0.816691in}}%
\pgfpathlineto{\pgfqpoint{3.255967in}{0.816688in}}%
\pgfpathlineto{\pgfqpoint{3.256263in}{0.816685in}}%
\pgfpathlineto{\pgfqpoint{3.256559in}{0.816683in}}%
\pgfpathlineto{\pgfqpoint{3.256855in}{0.816680in}}%
\pgfpathlineto{\pgfqpoint{3.257151in}{0.816678in}}%
\pgfpathlineto{\pgfqpoint{3.257447in}{0.816675in}}%
\pgfpathlineto{\pgfqpoint{3.257743in}{0.816672in}}%
\pgfpathlineto{\pgfqpoint{3.258039in}{0.816670in}}%
\pgfpathlineto{\pgfqpoint{3.258335in}{0.816667in}}%
\pgfpathlineto{\pgfqpoint{3.258631in}{0.816664in}}%
\pgfpathlineto{\pgfqpoint{3.258927in}{0.816662in}}%
\pgfpathlineto{\pgfqpoint{3.259223in}{0.816659in}}%
\pgfpathlineto{\pgfqpoint{3.259519in}{0.816657in}}%
\pgfpathlineto{\pgfqpoint{3.259815in}{0.816654in}}%
\pgfpathlineto{\pgfqpoint{3.260111in}{0.816651in}}%
\pgfpathlineto{\pgfqpoint{3.260407in}{0.816649in}}%
\pgfpathlineto{\pgfqpoint{3.260703in}{0.816646in}}%
\pgfpathlineto{\pgfqpoint{3.260999in}{0.816643in}}%
\pgfpathlineto{\pgfqpoint{3.261295in}{0.816641in}}%
\pgfpathlineto{\pgfqpoint{3.261591in}{0.816638in}}%
\pgfpathlineto{\pgfqpoint{3.261887in}{0.816635in}}%
\pgfpathlineto{\pgfqpoint{3.262183in}{0.816633in}}%
\pgfpathlineto{\pgfqpoint{3.262479in}{0.816630in}}%
\pgfpathlineto{\pgfqpoint{3.262775in}{0.816628in}}%
\pgfpathlineto{\pgfqpoint{3.263071in}{0.816625in}}%
\pgfpathlineto{\pgfqpoint{3.263367in}{0.816622in}}%
\pgfpathlineto{\pgfqpoint{3.263663in}{0.816620in}}%
\pgfpathlineto{\pgfqpoint{3.263959in}{0.816617in}}%
\pgfpathlineto{\pgfqpoint{3.264255in}{0.816614in}}%
\pgfpathlineto{\pgfqpoint{3.264551in}{0.816612in}}%
\pgfpathlineto{\pgfqpoint{3.264847in}{0.816609in}}%
\pgfpathlineto{\pgfqpoint{3.265143in}{0.816607in}}%
\pgfpathlineto{\pgfqpoint{3.265439in}{0.816604in}}%
\pgfpathlineto{\pgfqpoint{3.265735in}{0.816601in}}%
\pgfpathlineto{\pgfqpoint{3.266031in}{0.816599in}}%
\pgfpathlineto{\pgfqpoint{3.266327in}{0.816596in}}%
\pgfpathlineto{\pgfqpoint{3.266623in}{0.816593in}}%
\pgfpathlineto{\pgfqpoint{3.266919in}{0.816591in}}%
\pgfpathlineto{\pgfqpoint{3.267215in}{0.816588in}}%
\pgfpathlineto{\pgfqpoint{3.267511in}{0.816585in}}%
\pgfpathlineto{\pgfqpoint{3.267807in}{0.816583in}}%
\pgfpathlineto{\pgfqpoint{3.268103in}{0.816580in}}%
\pgfpathlineto{\pgfqpoint{3.268399in}{0.816578in}}%
\pgfpathlineto{\pgfqpoint{3.268695in}{0.816575in}}%
\pgfpathlineto{\pgfqpoint{3.268991in}{0.816570in}}%
\pgfpathlineto{\pgfqpoint{3.269287in}{0.816562in}}%
\pgfpathlineto{\pgfqpoint{3.269583in}{0.816554in}}%
\pgfpathlineto{\pgfqpoint{3.269879in}{0.816546in}}%
\pgfpathlineto{\pgfqpoint{3.270175in}{0.816538in}}%
\pgfpathlineto{\pgfqpoint{3.270471in}{0.816530in}}%
\pgfpathlineto{\pgfqpoint{3.270767in}{0.816522in}}%
\pgfpathlineto{\pgfqpoint{3.271063in}{0.816514in}}%
\pgfpathlineto{\pgfqpoint{3.271359in}{0.816507in}}%
\pgfpathlineto{\pgfqpoint{3.271655in}{0.816499in}}%
\pgfpathlineto{\pgfqpoint{3.271951in}{0.816491in}}%
\pgfpathlineto{\pgfqpoint{3.272247in}{0.816483in}}%
\pgfpathlineto{\pgfqpoint{3.272543in}{0.816475in}}%
\pgfpathlineto{\pgfqpoint{3.272839in}{0.816467in}}%
\pgfpathlineto{\pgfqpoint{3.273135in}{0.816459in}}%
\pgfpathlineto{\pgfqpoint{3.273431in}{0.816451in}}%
\pgfpathlineto{\pgfqpoint{3.273727in}{0.816444in}}%
\pgfpathlineto{\pgfqpoint{3.274023in}{0.816436in}}%
\pgfpathlineto{\pgfqpoint{3.274319in}{0.816428in}}%
\pgfpathlineto{\pgfqpoint{3.274615in}{0.816724in}}%
\pgfpathlineto{\pgfqpoint{3.274911in}{0.817650in}}%
\pgfpathlineto{\pgfqpoint{3.275207in}{0.817397in}}%
\pgfpathlineto{\pgfqpoint{3.275503in}{0.817143in}}%
\pgfpathlineto{\pgfqpoint{3.275799in}{0.816890in}}%
\pgfpathlineto{\pgfqpoint{3.276095in}{0.816637in}}%
\pgfpathlineto{\pgfqpoint{3.276391in}{0.816383in}}%
\pgfpathlineto{\pgfqpoint{3.276687in}{0.816130in}}%
\pgfpathlineto{\pgfqpoint{3.276983in}{0.815877in}}%
\pgfpathlineto{\pgfqpoint{3.277280in}{0.815623in}}%
\pgfpathlineto{\pgfqpoint{3.277576in}{0.815370in}}%
\pgfpathlineto{\pgfqpoint{3.277872in}{0.815117in}}%
\pgfpathlineto{\pgfqpoint{3.278168in}{0.814863in}}%
\pgfpathlineto{\pgfqpoint{3.278464in}{0.814610in}}%
\pgfpathlineto{\pgfqpoint{3.278760in}{0.814357in}}%
\pgfpathlineto{\pgfqpoint{3.279056in}{0.814103in}}%
\pgfpathlineto{\pgfqpoint{3.279352in}{0.813850in}}%
\pgfpathlineto{\pgfqpoint{3.279648in}{0.813597in}}%
\pgfpathlineto{\pgfqpoint{3.279944in}{0.813343in}}%
\pgfpathlineto{\pgfqpoint{3.280240in}{0.813090in}}%
\pgfpathlineto{\pgfqpoint{3.280536in}{0.812837in}}%
\pgfpathlineto{\pgfqpoint{3.280832in}{0.812583in}}%
\pgfpathlineto{\pgfqpoint{3.281128in}{0.812330in}}%
\pgfpathlineto{\pgfqpoint{3.281424in}{0.812077in}}%
\pgfpathlineto{\pgfqpoint{3.281720in}{0.811823in}}%
\pgfpathlineto{\pgfqpoint{3.282016in}{0.811570in}}%
\pgfpathlineto{\pgfqpoint{3.282312in}{0.811316in}}%
\pgfpathlineto{\pgfqpoint{3.282608in}{0.811063in}}%
\pgfpathlineto{\pgfqpoint{3.282904in}{0.810810in}}%
\pgfpathlineto{\pgfqpoint{3.283200in}{0.810556in}}%
\pgfpathlineto{\pgfqpoint{3.283496in}{0.810303in}}%
\pgfpathlineto{\pgfqpoint{3.283792in}{0.810050in}}%
\pgfpathlineto{\pgfqpoint{3.284088in}{0.809796in}}%
\pgfpathlineto{\pgfqpoint{3.284384in}{0.809543in}}%
\pgfpathlineto{\pgfqpoint{3.284680in}{0.809290in}}%
\pgfpathlineto{\pgfqpoint{3.284976in}{0.809036in}}%
\pgfpathlineto{\pgfqpoint{3.285272in}{0.808783in}}%
\pgfpathlineto{\pgfqpoint{3.285568in}{0.808530in}}%
\pgfpathlineto{\pgfqpoint{3.285864in}{0.808276in}}%
\pgfpathlineto{\pgfqpoint{3.286160in}{0.808023in}}%
\pgfpathlineto{\pgfqpoint{3.286456in}{0.807770in}}%
\pgfpathlineto{\pgfqpoint{3.286752in}{0.807516in}}%
\pgfpathlineto{\pgfqpoint{3.287048in}{0.807263in}}%
\pgfpathlineto{\pgfqpoint{3.287344in}{0.807010in}}%
\pgfpathlineto{\pgfqpoint{3.287640in}{0.806756in}}%
\pgfpathlineto{\pgfqpoint{3.287936in}{0.806503in}}%
\pgfpathlineto{\pgfqpoint{3.288232in}{0.806249in}}%
\pgfpathlineto{\pgfqpoint{3.288528in}{0.805996in}}%
\pgfpathlineto{\pgfqpoint{3.288824in}{0.805743in}}%
\pgfpathlineto{\pgfqpoint{3.289120in}{0.805489in}}%
\pgfpathlineto{\pgfqpoint{3.289416in}{0.805236in}}%
\pgfpathlineto{\pgfqpoint{3.289712in}{0.804983in}}%
\pgfpathlineto{\pgfqpoint{3.290008in}{0.804729in}}%
\pgfpathlineto{\pgfqpoint{3.290304in}{0.804476in}}%
\pgfpathlineto{\pgfqpoint{3.290600in}{0.804223in}}%
\pgfpathlineto{\pgfqpoint{3.290896in}{0.803969in}}%
\pgfpathlineto{\pgfqpoint{3.291192in}{0.803716in}}%
\pgfpathlineto{\pgfqpoint{3.291488in}{0.803463in}}%
\pgfpathlineto{\pgfqpoint{3.291784in}{0.803209in}}%
\pgfpathlineto{\pgfqpoint{3.292080in}{0.802956in}}%
\pgfpathlineto{\pgfqpoint{3.292376in}{0.802703in}}%
\pgfpathlineto{\pgfqpoint{3.292672in}{0.802449in}}%
\pgfpathlineto{\pgfqpoint{3.292968in}{0.802196in}}%
\pgfpathlineto{\pgfqpoint{3.293264in}{0.801943in}}%
\pgfpathlineto{\pgfqpoint{3.293560in}{0.801689in}}%
\pgfpathlineto{\pgfqpoint{3.293856in}{0.801436in}}%
\pgfpathlineto{\pgfqpoint{3.294152in}{0.801183in}}%
\pgfpathlineto{\pgfqpoint{3.294448in}{0.800929in}}%
\pgfpathlineto{\pgfqpoint{3.294744in}{0.800676in}}%
\pgfpathlineto{\pgfqpoint{3.295040in}{0.800422in}}%
\pgfpathlineto{\pgfqpoint{3.295336in}{0.800169in}}%
\pgfpathlineto{\pgfqpoint{3.295632in}{0.799916in}}%
\pgfpathlineto{\pgfqpoint{3.295928in}{0.799662in}}%
\pgfpathlineto{\pgfqpoint{3.296224in}{0.799409in}}%
\pgfpathlineto{\pgfqpoint{3.296520in}{0.799156in}}%
\pgfpathlineto{\pgfqpoint{3.296816in}{0.798902in}}%
\pgfpathlineto{\pgfqpoint{3.297112in}{0.798649in}}%
\pgfpathlineto{\pgfqpoint{3.297408in}{0.798396in}}%
\pgfpathlineto{\pgfqpoint{3.297704in}{0.798142in}}%
\pgfpathlineto{\pgfqpoint{3.298000in}{0.797889in}}%
\pgfpathlineto{\pgfqpoint{3.298296in}{0.797636in}}%
\pgfpathlineto{\pgfqpoint{3.298592in}{0.797382in}}%
\pgfpathlineto{\pgfqpoint{3.298888in}{0.797129in}}%
\pgfpathlineto{\pgfqpoint{3.299184in}{0.796876in}}%
\pgfpathlineto{\pgfqpoint{3.299480in}{0.796622in}}%
\pgfpathlineto{\pgfqpoint{3.299776in}{0.796369in}}%
\pgfpathlineto{\pgfqpoint{3.300072in}{0.796116in}}%
\pgfpathlineto{\pgfqpoint{3.300368in}{0.795862in}}%
\pgfpathlineto{\pgfqpoint{3.300664in}{0.795609in}}%
\pgfpathlineto{\pgfqpoint{3.300960in}{0.795355in}}%
\pgfpathlineto{\pgfqpoint{3.301256in}{0.795102in}}%
\pgfpathlineto{\pgfqpoint{3.301552in}{0.794849in}}%
\pgfpathlineto{\pgfqpoint{3.301848in}{0.794595in}}%
\pgfpathlineto{\pgfqpoint{3.302144in}{0.794342in}}%
\pgfpathlineto{\pgfqpoint{3.302440in}{0.794089in}}%
\pgfpathlineto{\pgfqpoint{3.302736in}{0.793884in}}%
\pgfpathlineto{\pgfqpoint{3.303032in}{0.793870in}}%
\pgfpathlineto{\pgfqpoint{3.303328in}{0.793270in}}%
\pgfpathlineto{\pgfqpoint{3.303624in}{0.793195in}}%
\pgfpathlineto{\pgfqpoint{3.303920in}{0.793140in}}%
\pgfpathlineto{\pgfqpoint{3.304216in}{0.793086in}}%
\pgfpathlineto{\pgfqpoint{3.304512in}{0.793036in}}%
\pgfpathlineto{\pgfqpoint{3.304808in}{0.793034in}}%
\pgfpathlineto{\pgfqpoint{3.305104in}{0.793043in}}%
\pgfpathlineto{\pgfqpoint{3.305400in}{0.793051in}}%
\pgfpathlineto{\pgfqpoint{3.305696in}{0.793060in}}%
\pgfpathlineto{\pgfqpoint{3.305992in}{0.793068in}}%
\pgfpathlineto{\pgfqpoint{3.306288in}{0.793077in}}%
\pgfpathlineto{\pgfqpoint{3.306584in}{0.793085in}}%
\pgfpathlineto{\pgfqpoint{3.306880in}{0.793094in}}%
\pgfpathlineto{\pgfqpoint{3.307176in}{0.793102in}}%
\pgfpathlineto{\pgfqpoint{3.307472in}{0.793111in}}%
\pgfpathlineto{\pgfqpoint{3.307768in}{0.793119in}}%
\pgfpathlineto{\pgfqpoint{3.308064in}{0.793128in}}%
\pgfpathlineto{\pgfqpoint{3.308360in}{0.793136in}}%
\pgfpathlineto{\pgfqpoint{3.308656in}{0.793144in}}%
\pgfpathlineto{\pgfqpoint{3.308952in}{0.793153in}}%
\pgfpathlineto{\pgfqpoint{3.309248in}{0.793161in}}%
\pgfpathlineto{\pgfqpoint{3.309544in}{0.793170in}}%
\pgfpathlineto{\pgfqpoint{3.309840in}{0.793178in}}%
\pgfpathlineto{\pgfqpoint{3.310136in}{0.793187in}}%
\pgfpathlineto{\pgfqpoint{3.310432in}{0.793195in}}%
\pgfpathlineto{\pgfqpoint{3.310728in}{0.793204in}}%
\pgfpathlineto{\pgfqpoint{3.311024in}{0.793212in}}%
\pgfpathlineto{\pgfqpoint{3.311320in}{0.793221in}}%
\pgfpathlineto{\pgfqpoint{3.311616in}{0.793229in}}%
\pgfpathlineto{\pgfqpoint{3.311912in}{0.793237in}}%
\pgfpathlineto{\pgfqpoint{3.312208in}{0.793246in}}%
\pgfpathlineto{\pgfqpoint{3.312504in}{0.793254in}}%
\pgfpathlineto{\pgfqpoint{3.312800in}{0.793263in}}%
\pgfpathlineto{\pgfqpoint{3.313096in}{0.793271in}}%
\pgfpathlineto{\pgfqpoint{3.313392in}{0.793280in}}%
\pgfpathlineto{\pgfqpoint{3.313688in}{0.793288in}}%
\pgfpathlineto{\pgfqpoint{3.313984in}{0.793297in}}%
\pgfpathlineto{\pgfqpoint{3.314280in}{0.793305in}}%
\pgfpathlineto{\pgfqpoint{3.314576in}{0.793314in}}%
\pgfpathlineto{\pgfqpoint{3.314872in}{0.793322in}}%
\pgfpathlineto{\pgfqpoint{3.315168in}{0.793331in}}%
\pgfpathlineto{\pgfqpoint{3.315464in}{0.793339in}}%
\pgfpathlineto{\pgfqpoint{3.315760in}{0.793347in}}%
\pgfpathlineto{\pgfqpoint{3.316056in}{0.793356in}}%
\pgfpathlineto{\pgfqpoint{3.316352in}{0.793364in}}%
\pgfpathlineto{\pgfqpoint{3.316648in}{0.793754in}}%
\pgfpathlineto{\pgfqpoint{3.316944in}{0.811361in}}%
\pgfpathlineto{\pgfqpoint{3.317240in}{0.797803in}}%
\pgfpathlineto{\pgfqpoint{3.317536in}{0.802083in}}%
\pgfpathlineto{\pgfqpoint{3.317832in}{0.812623in}}%
\pgfpathlineto{\pgfqpoint{3.318128in}{0.808659in}}%
\pgfpathlineto{\pgfqpoint{3.318424in}{0.798360in}}%
\pgfpathlineto{\pgfqpoint{3.318720in}{0.805019in}}%
\pgfpathlineto{\pgfqpoint{3.319016in}{0.805621in}}%
\pgfpathlineto{\pgfqpoint{3.319312in}{0.805862in}}%
\pgfpathlineto{\pgfqpoint{3.319608in}{0.805439in}}%
\pgfpathlineto{\pgfqpoint{3.319904in}{0.805895in}}%
\pgfpathlineto{\pgfqpoint{3.320200in}{0.806350in}}%
\pgfpathlineto{\pgfqpoint{3.320496in}{0.806806in}}%
\pgfpathlineto{\pgfqpoint{3.320792in}{0.807261in}}%
\pgfpathlineto{\pgfqpoint{3.321088in}{0.807717in}}%
\pgfpathlineto{\pgfqpoint{3.321384in}{0.808173in}}%
\pgfpathlineto{\pgfqpoint{3.321680in}{0.808628in}}%
\pgfpathlineto{\pgfqpoint{3.321976in}{0.809084in}}%
\pgfpathlineto{\pgfqpoint{3.322272in}{0.809539in}}%
\pgfpathlineto{\pgfqpoint{3.322568in}{0.809995in}}%
\pgfpathlineto{\pgfqpoint{3.322864in}{0.810451in}}%
\pgfpathlineto{\pgfqpoint{3.323160in}{0.810906in}}%
\pgfpathlineto{\pgfqpoint{3.323456in}{0.811362in}}%
\pgfpathlineto{\pgfqpoint{3.323752in}{0.811817in}}%
\pgfpathlineto{\pgfqpoint{3.324048in}{0.812273in}}%
\pgfpathlineto{\pgfqpoint{3.324344in}{0.812729in}}%
\pgfpathlineto{\pgfqpoint{3.324640in}{0.813186in}}%
\pgfpathlineto{\pgfqpoint{3.324936in}{0.813669in}}%
\pgfpathlineto{\pgfqpoint{3.325232in}{0.814163in}}%
\pgfpathlineto{\pgfqpoint{3.325528in}{0.814482in}}%
\pgfpathlineto{\pgfqpoint{3.325824in}{0.814487in}}%
\pgfpathlineto{\pgfqpoint{3.326120in}{0.814486in}}%
\pgfpathlineto{\pgfqpoint{3.326416in}{0.814486in}}%
\pgfpathlineto{\pgfqpoint{3.326712in}{0.814485in}}%
\pgfpathlineto{\pgfqpoint{3.327008in}{0.814485in}}%
\pgfpathlineto{\pgfqpoint{3.327304in}{0.814484in}}%
\pgfpathlineto{\pgfqpoint{3.327600in}{0.814483in}}%
\pgfpathlineto{\pgfqpoint{3.327896in}{0.814483in}}%
\pgfpathlineto{\pgfqpoint{3.328192in}{0.814482in}}%
\pgfpathlineto{\pgfqpoint{3.328488in}{0.814481in}}%
\pgfpathlineto{\pgfqpoint{3.328784in}{0.814481in}}%
\pgfpathlineto{\pgfqpoint{3.329080in}{0.814480in}}%
\pgfpathlineto{\pgfqpoint{3.329376in}{0.814480in}}%
\pgfpathlineto{\pgfqpoint{3.329672in}{0.814479in}}%
\pgfpathlineto{\pgfqpoint{3.329968in}{0.814478in}}%
\pgfpathlineto{\pgfqpoint{3.330264in}{0.814478in}}%
\pgfpathlineto{\pgfqpoint{3.330560in}{0.814477in}}%
\pgfpathlineto{\pgfqpoint{3.330856in}{0.814476in}}%
\pgfpathlineto{\pgfqpoint{3.331152in}{0.814476in}}%
\pgfpathlineto{\pgfqpoint{3.331448in}{0.814475in}}%
\pgfpathlineto{\pgfqpoint{3.331744in}{0.814475in}}%
\pgfpathlineto{\pgfqpoint{3.332040in}{0.814474in}}%
\pgfpathlineto{\pgfqpoint{3.332336in}{0.814473in}}%
\pgfpathlineto{\pgfqpoint{3.332632in}{0.814473in}}%
\pgfpathlineto{\pgfqpoint{3.332928in}{0.814472in}}%
\pgfpathlineto{\pgfqpoint{3.333224in}{0.814471in}}%
\pgfpathlineto{\pgfqpoint{3.333520in}{0.814471in}}%
\pgfpathlineto{\pgfqpoint{3.333816in}{0.814470in}}%
\pgfpathlineto{\pgfqpoint{3.334112in}{0.814470in}}%
\pgfpathlineto{\pgfqpoint{3.334408in}{0.814469in}}%
\pgfpathlineto{\pgfqpoint{3.334704in}{0.814468in}}%
\pgfpathlineto{\pgfqpoint{3.335000in}{0.814468in}}%
\pgfpathlineto{\pgfqpoint{3.335296in}{0.814467in}}%
\pgfpathlineto{\pgfqpoint{3.335592in}{0.814466in}}%
\pgfpathlineto{\pgfqpoint{3.335888in}{0.814466in}}%
\pgfpathlineto{\pgfqpoint{3.336184in}{0.814465in}}%
\pgfpathlineto{\pgfqpoint{3.336480in}{0.814465in}}%
\pgfpathlineto{\pgfqpoint{3.336776in}{0.814464in}}%
\pgfpathlineto{\pgfqpoint{3.337072in}{0.814463in}}%
\pgfpathlineto{\pgfqpoint{3.337368in}{0.814463in}}%
\pgfpathlineto{\pgfqpoint{3.337664in}{0.814462in}}%
\pgfpathlineto{\pgfqpoint{3.337960in}{0.814461in}}%
\pgfpathlineto{\pgfqpoint{3.338256in}{0.814461in}}%
\pgfpathlineto{\pgfqpoint{3.338552in}{0.814460in}}%
\pgfpathlineto{\pgfqpoint{3.338848in}{0.814460in}}%
\pgfpathlineto{\pgfqpoint{3.339144in}{0.814459in}}%
\pgfpathlineto{\pgfqpoint{3.339440in}{0.814458in}}%
\pgfpathlineto{\pgfqpoint{3.339736in}{0.814458in}}%
\pgfpathlineto{\pgfqpoint{3.340032in}{0.814457in}}%
\pgfpathlineto{\pgfqpoint{3.340328in}{0.814456in}}%
\pgfpathlineto{\pgfqpoint{3.340624in}{0.814456in}}%
\pgfpathlineto{\pgfqpoint{3.340920in}{0.814455in}}%
\pgfpathlineto{\pgfqpoint{3.341216in}{0.814455in}}%
\pgfpathlineto{\pgfqpoint{3.341512in}{0.814454in}}%
\pgfpathlineto{\pgfqpoint{3.341808in}{0.814453in}}%
\pgfpathlineto{\pgfqpoint{3.342104in}{0.814453in}}%
\pgfpathlineto{\pgfqpoint{3.342400in}{0.814452in}}%
\pgfpathlineto{\pgfqpoint{3.342696in}{0.814451in}}%
\pgfpathlineto{\pgfqpoint{3.342992in}{0.814451in}}%
\pgfpathlineto{\pgfqpoint{3.343288in}{0.814450in}}%
\pgfpathlineto{\pgfqpoint{3.343584in}{0.814450in}}%
\pgfpathlineto{\pgfqpoint{3.343880in}{0.814449in}}%
\pgfpathlineto{\pgfqpoint{3.344176in}{0.814448in}}%
\pgfpathlineto{\pgfqpoint{3.344472in}{0.814448in}}%
\pgfpathlineto{\pgfqpoint{3.344769in}{0.814447in}}%
\pgfpathlineto{\pgfqpoint{3.345065in}{0.814446in}}%
\pgfpathlineto{\pgfqpoint{3.345361in}{0.814446in}}%
\pgfpathlineto{\pgfqpoint{3.345657in}{0.814445in}}%
\pgfpathlineto{\pgfqpoint{3.345953in}{0.814445in}}%
\pgfpathlineto{\pgfqpoint{3.346249in}{0.814444in}}%
\pgfpathlineto{\pgfqpoint{3.346545in}{0.814443in}}%
\pgfpathlineto{\pgfqpoint{3.346841in}{0.814443in}}%
\pgfpathlineto{\pgfqpoint{3.347137in}{0.814442in}}%
\pgfpathlineto{\pgfqpoint{3.347433in}{0.814441in}}%
\pgfpathlineto{\pgfqpoint{3.347729in}{0.814441in}}%
\pgfpathlineto{\pgfqpoint{3.348025in}{0.814440in}}%
\pgfpathlineto{\pgfqpoint{3.348321in}{0.814440in}}%
\pgfpathlineto{\pgfqpoint{3.348617in}{0.814439in}}%
\pgfpathlineto{\pgfqpoint{3.348913in}{0.814438in}}%
\pgfpathlineto{\pgfqpoint{3.349209in}{0.814438in}}%
\pgfpathlineto{\pgfqpoint{3.349505in}{0.814437in}}%
\pgfpathlineto{\pgfqpoint{3.349801in}{0.814436in}}%
\pgfpathlineto{\pgfqpoint{3.350097in}{0.814436in}}%
\pgfpathlineto{\pgfqpoint{3.350393in}{0.814435in}}%
\pgfpathlineto{\pgfqpoint{3.350689in}{0.814435in}}%
\pgfpathlineto{\pgfqpoint{3.350985in}{0.814434in}}%
\pgfpathlineto{\pgfqpoint{3.351281in}{0.814433in}}%
\pgfpathlineto{\pgfqpoint{3.351577in}{0.814433in}}%
\pgfpathlineto{\pgfqpoint{3.351873in}{0.814432in}}%
\pgfpathlineto{\pgfqpoint{3.352169in}{0.814432in}}%
\pgfpathlineto{\pgfqpoint{3.352465in}{0.814436in}}%
\pgfpathlineto{\pgfqpoint{3.352761in}{0.814442in}}%
\pgfpathlineto{\pgfqpoint{3.353057in}{0.814447in}}%
\pgfpathlineto{\pgfqpoint{3.353353in}{0.814453in}}%
\pgfpathlineto{\pgfqpoint{3.353649in}{0.814458in}}%
\pgfpathlineto{\pgfqpoint{3.353945in}{0.814464in}}%
\pgfpathlineto{\pgfqpoint{3.354241in}{0.814470in}}%
\pgfpathlineto{\pgfqpoint{3.354537in}{0.814473in}}%
\pgfpathlineto{\pgfqpoint{3.354833in}{0.814461in}}%
\pgfpathlineto{\pgfqpoint{3.355129in}{0.814446in}}%
\pgfpathlineto{\pgfqpoint{3.355425in}{0.814431in}}%
\pgfpathlineto{\pgfqpoint{3.355721in}{0.814416in}}%
\pgfpathlineto{\pgfqpoint{3.356017in}{0.814400in}}%
\pgfpathlineto{\pgfqpoint{3.356313in}{0.814385in}}%
\pgfpathlineto{\pgfqpoint{3.356609in}{0.814370in}}%
\pgfpathlineto{\pgfqpoint{3.356905in}{0.814355in}}%
\pgfpathlineto{\pgfqpoint{3.357201in}{0.814340in}}%
\pgfpathlineto{\pgfqpoint{3.357497in}{0.814325in}}%
\pgfpathlineto{\pgfqpoint{3.357793in}{0.814309in}}%
\pgfpathlineto{\pgfqpoint{3.358089in}{0.814294in}}%
\pgfpathlineto{\pgfqpoint{3.358385in}{0.814279in}}%
\pgfpathlineto{\pgfqpoint{3.358681in}{0.814264in}}%
\pgfpathlineto{\pgfqpoint{3.358977in}{0.814249in}}%
\pgfpathlineto{\pgfqpoint{3.359273in}{0.814234in}}%
\pgfpathlineto{\pgfqpoint{3.359569in}{0.814236in}}%
\pgfpathlineto{\pgfqpoint{3.359865in}{0.814212in}}%
\pgfpathlineto{\pgfqpoint{3.360161in}{0.814199in}}%
\pgfpathlineto{\pgfqpoint{3.360457in}{0.814064in}}%
\pgfpathlineto{\pgfqpoint{3.360753in}{0.814036in}}%
\pgfpathlineto{\pgfqpoint{3.361049in}{0.814009in}}%
\pgfpathlineto{\pgfqpoint{3.361345in}{0.813982in}}%
\pgfpathlineto{\pgfqpoint{3.361641in}{0.813698in}}%
\pgfpathlineto{\pgfqpoint{3.361937in}{0.813603in}}%
\pgfpathlineto{\pgfqpoint{3.362233in}{0.813566in}}%
\pgfpathlineto{\pgfqpoint{3.362529in}{0.813528in}}%
\pgfpathlineto{\pgfqpoint{3.362825in}{0.813490in}}%
\pgfpathlineto{\pgfqpoint{3.363121in}{0.813453in}}%
\pgfpathlineto{\pgfqpoint{3.363417in}{0.813415in}}%
\pgfpathlineto{\pgfqpoint{3.363713in}{0.813377in}}%
\pgfpathlineto{\pgfqpoint{3.364009in}{0.813340in}}%
\pgfpathlineto{\pgfqpoint{3.364305in}{0.813302in}}%
\pgfpathlineto{\pgfqpoint{3.364601in}{0.813264in}}%
\pgfpathlineto{\pgfqpoint{3.364897in}{0.813227in}}%
\pgfpathlineto{\pgfqpoint{3.365193in}{0.813189in}}%
\pgfpathlineto{\pgfqpoint{3.365489in}{0.813151in}}%
\pgfpathlineto{\pgfqpoint{3.365785in}{0.813114in}}%
\pgfpathlineto{\pgfqpoint{3.366081in}{0.813076in}}%
\pgfpathlineto{\pgfqpoint{3.366377in}{0.813038in}}%
\pgfpathlineto{\pgfqpoint{3.366673in}{0.813000in}}%
\pgfpathlineto{\pgfqpoint{3.366969in}{0.812966in}}%
\pgfpathlineto{\pgfqpoint{3.367265in}{0.812939in}}%
\pgfpathlineto{\pgfqpoint{3.367561in}{0.812912in}}%
\pgfpathlineto{\pgfqpoint{3.367857in}{0.812884in}}%
\pgfpathlineto{\pgfqpoint{3.368153in}{0.812857in}}%
\pgfpathlineto{\pgfqpoint{3.368449in}{0.812830in}}%
\pgfpathlineto{\pgfqpoint{3.368745in}{0.812803in}}%
\pgfpathlineto{\pgfqpoint{3.369041in}{0.812776in}}%
\pgfpathlineto{\pgfqpoint{3.369337in}{0.812758in}}%
\pgfpathlineto{\pgfqpoint{3.369633in}{0.812745in}}%
\pgfpathlineto{\pgfqpoint{3.369929in}{0.812733in}}%
\pgfpathlineto{\pgfqpoint{3.370225in}{0.812721in}}%
\pgfpathlineto{\pgfqpoint{3.370521in}{0.812709in}}%
\pgfpathlineto{\pgfqpoint{3.370817in}{0.812697in}}%
\pgfpathlineto{\pgfqpoint{3.371113in}{0.812684in}}%
\pgfpathlineto{\pgfqpoint{3.371409in}{0.812672in}}%
\pgfpathlineto{\pgfqpoint{3.371705in}{0.812660in}}%
\pgfpathlineto{\pgfqpoint{3.372001in}{0.812648in}}%
\pgfpathlineto{\pgfqpoint{3.372297in}{0.812636in}}%
\pgfpathlineto{\pgfqpoint{3.372593in}{0.812623in}}%
\pgfpathlineto{\pgfqpoint{3.372889in}{0.812611in}}%
\pgfpathlineto{\pgfqpoint{3.373185in}{0.812599in}}%
\pgfpathlineto{\pgfqpoint{3.373481in}{0.812587in}}%
\pgfpathlineto{\pgfqpoint{3.373777in}{0.812575in}}%
\pgfpathlineto{\pgfqpoint{3.374073in}{0.812596in}}%
\pgfpathlineto{\pgfqpoint{3.374369in}{0.812622in}}%
\pgfpathlineto{\pgfqpoint{3.374665in}{0.812594in}}%
\pgfpathlineto{\pgfqpoint{3.374961in}{0.812472in}}%
\pgfpathlineto{\pgfqpoint{3.375257in}{0.812227in}}%
\pgfpathlineto{\pgfqpoint{3.375553in}{0.812200in}}%
\pgfpathlineto{\pgfqpoint{3.375849in}{0.812118in}}%
\pgfpathlineto{\pgfqpoint{3.376145in}{0.811998in}}%
\pgfpathlineto{\pgfqpoint{3.376441in}{0.811879in}}%
\pgfpathlineto{\pgfqpoint{3.376737in}{0.811760in}}%
\pgfpathlineto{\pgfqpoint{3.377033in}{0.811641in}}%
\pgfpathlineto{\pgfqpoint{3.377329in}{0.811522in}}%
\pgfpathlineto{\pgfqpoint{3.377625in}{0.811403in}}%
\pgfpathlineto{\pgfqpoint{3.377921in}{0.811284in}}%
\pgfpathlineto{\pgfqpoint{3.378217in}{0.811165in}}%
\pgfpathlineto{\pgfqpoint{3.378513in}{0.811046in}}%
\pgfpathlineto{\pgfqpoint{3.378809in}{0.810927in}}%
\pgfpathlineto{\pgfqpoint{3.379105in}{0.810808in}}%
\pgfpathlineto{\pgfqpoint{3.379401in}{0.810689in}}%
\pgfpathlineto{\pgfqpoint{3.379697in}{0.810570in}}%
\pgfpathlineto{\pgfqpoint{3.379993in}{0.810450in}}%
\pgfpathlineto{\pgfqpoint{3.380289in}{0.810331in}}%
\pgfpathlineto{\pgfqpoint{3.380585in}{0.810219in}}%
\pgfpathlineto{\pgfqpoint{3.380881in}{0.810151in}}%
\pgfpathlineto{\pgfqpoint{3.381177in}{0.809989in}}%
\pgfpathlineto{\pgfqpoint{3.381473in}{0.809832in}}%
\pgfpathlineto{\pgfqpoint{3.381769in}{0.809926in}}%
\pgfpathlineto{\pgfqpoint{3.382065in}{0.809643in}}%
\pgfpathlineto{\pgfqpoint{3.382361in}{0.804197in}}%
\pgfpathlineto{\pgfqpoint{3.382657in}{0.792542in}}%
\pgfpathlineto{\pgfqpoint{3.382953in}{0.797326in}}%
\pgfpathlineto{\pgfqpoint{3.383249in}{0.809435in}}%
\pgfpathlineto{\pgfqpoint{3.383545in}{0.809520in}}%
\pgfpathlineto{\pgfqpoint{3.383841in}{0.809512in}}%
\pgfpathlineto{\pgfqpoint{3.384137in}{0.809503in}}%
\pgfpathlineto{\pgfqpoint{3.384433in}{0.809495in}}%
\pgfpathlineto{\pgfqpoint{3.384729in}{0.809486in}}%
\pgfpathlineto{\pgfqpoint{3.385025in}{0.809478in}}%
\pgfpathlineto{\pgfqpoint{3.385321in}{0.809469in}}%
\pgfpathlineto{\pgfqpoint{3.385617in}{0.809461in}}%
\pgfpathlineto{\pgfqpoint{3.385913in}{0.809453in}}%
\pgfpathlineto{\pgfqpoint{3.386209in}{0.809444in}}%
\pgfpathlineto{\pgfqpoint{3.386505in}{0.809436in}}%
\pgfpathlineto{\pgfqpoint{3.386801in}{0.809427in}}%
\pgfpathlineto{\pgfqpoint{3.387097in}{0.809419in}}%
\pgfpathlineto{\pgfqpoint{3.387393in}{0.809410in}}%
\pgfpathlineto{\pgfqpoint{3.387689in}{0.809402in}}%
\pgfpathlineto{\pgfqpoint{3.387985in}{0.809393in}}%
\pgfpathlineto{\pgfqpoint{3.388281in}{0.809385in}}%
\pgfpathlineto{\pgfqpoint{3.388577in}{0.809376in}}%
\pgfpathlineto{\pgfqpoint{3.388873in}{0.809368in}}%
\pgfpathlineto{\pgfqpoint{3.389169in}{0.809359in}}%
\pgfpathlineto{\pgfqpoint{3.389465in}{0.809351in}}%
\pgfpathlineto{\pgfqpoint{3.389761in}{0.809343in}}%
\pgfpathlineto{\pgfqpoint{3.390057in}{0.809334in}}%
\pgfpathlineto{\pgfqpoint{3.390353in}{0.809326in}}%
\pgfpathlineto{\pgfqpoint{3.390649in}{0.809317in}}%
\pgfpathlineto{\pgfqpoint{3.390945in}{0.809309in}}%
\pgfpathlineto{\pgfqpoint{3.391241in}{0.809300in}}%
\pgfpathlineto{\pgfqpoint{3.391537in}{0.809292in}}%
\pgfpathlineto{\pgfqpoint{3.391833in}{0.809283in}}%
\pgfpathlineto{\pgfqpoint{3.392129in}{0.809275in}}%
\pgfpathlineto{\pgfqpoint{3.392425in}{0.809266in}}%
\pgfpathlineto{\pgfqpoint{3.392721in}{0.809258in}}%
\pgfpathlineto{\pgfqpoint{3.393017in}{0.809249in}}%
\pgfpathlineto{\pgfqpoint{3.393313in}{0.809241in}}%
\pgfpathlineto{\pgfqpoint{3.393609in}{0.809233in}}%
\pgfpathlineto{\pgfqpoint{3.393905in}{0.809224in}}%
\pgfpathlineto{\pgfqpoint{3.394201in}{0.809216in}}%
\pgfpathlineto{\pgfqpoint{3.394497in}{0.809207in}}%
\pgfpathlineto{\pgfqpoint{3.394793in}{0.809199in}}%
\pgfpathlineto{\pgfqpoint{3.395089in}{0.809190in}}%
\pgfpathlineto{\pgfqpoint{3.395385in}{0.809182in}}%
\pgfpathlineto{\pgfqpoint{3.395681in}{0.809173in}}%
\pgfpathlineto{\pgfqpoint{3.395977in}{0.809165in}}%
\pgfpathlineto{\pgfqpoint{3.396273in}{0.809156in}}%
\pgfpathlineto{\pgfqpoint{3.396569in}{0.809148in}}%
\pgfpathlineto{\pgfqpoint{3.396865in}{0.809139in}}%
\pgfpathlineto{\pgfqpoint{3.397161in}{0.809131in}}%
\pgfpathlineto{\pgfqpoint{3.397457in}{0.809123in}}%
\pgfpathlineto{\pgfqpoint{3.397753in}{0.809114in}}%
\pgfpathlineto{\pgfqpoint{3.398049in}{0.809106in}}%
\pgfpathlineto{\pgfqpoint{3.398345in}{0.809097in}}%
\pgfpathlineto{\pgfqpoint{3.398641in}{0.809089in}}%
\pgfpathlineto{\pgfqpoint{3.398937in}{0.809080in}}%
\pgfpathlineto{\pgfqpoint{3.399233in}{0.809072in}}%
\pgfpathlineto{\pgfqpoint{3.399529in}{0.809063in}}%
\pgfpathlineto{\pgfqpoint{3.399825in}{0.809055in}}%
\pgfpathlineto{\pgfqpoint{3.400121in}{0.809046in}}%
\pgfpathlineto{\pgfqpoint{3.400417in}{0.809038in}}%
\pgfpathlineto{\pgfqpoint{3.400713in}{0.809029in}}%
\pgfpathlineto{\pgfqpoint{3.401009in}{0.809021in}}%
\pgfpathlineto{\pgfqpoint{3.401305in}{0.809013in}}%
\pgfpathlineto{\pgfqpoint{3.401601in}{0.809004in}}%
\pgfpathlineto{\pgfqpoint{3.401897in}{0.808996in}}%
\pgfpathlineto{\pgfqpoint{3.402193in}{0.808987in}}%
\pgfpathlineto{\pgfqpoint{3.402489in}{0.808979in}}%
\pgfpathlineto{\pgfqpoint{3.402785in}{0.808970in}}%
\pgfpathlineto{\pgfqpoint{3.403081in}{0.808962in}}%
\pgfpathlineto{\pgfqpoint{3.403377in}{0.808953in}}%
\pgfpathlineto{\pgfqpoint{3.403673in}{0.808945in}}%
\pgfpathlineto{\pgfqpoint{3.403969in}{0.808936in}}%
\pgfpathlineto{\pgfqpoint{3.404265in}{0.808928in}}%
\pgfpathlineto{\pgfqpoint{3.404561in}{0.808919in}}%
\pgfpathlineto{\pgfqpoint{3.404857in}{0.808911in}}%
\pgfpathlineto{\pgfqpoint{3.405153in}{0.808903in}}%
\pgfpathlineto{\pgfqpoint{3.405449in}{0.808894in}}%
\pgfpathlineto{\pgfqpoint{3.405745in}{0.808886in}}%
\pgfpathlineto{\pgfqpoint{3.406041in}{0.808877in}}%
\pgfpathlineto{\pgfqpoint{3.406337in}{0.808869in}}%
\pgfpathlineto{\pgfqpoint{3.406633in}{0.808860in}}%
\pgfpathlineto{\pgfqpoint{3.406929in}{0.808852in}}%
\pgfpathlineto{\pgfqpoint{3.407225in}{0.808843in}}%
\pgfpathlineto{\pgfqpoint{3.407521in}{0.808835in}}%
\pgfpathlineto{\pgfqpoint{3.407817in}{0.808826in}}%
\pgfpathlineto{\pgfqpoint{3.408113in}{0.808818in}}%
\pgfpathlineto{\pgfqpoint{3.408409in}{0.808809in}}%
\pgfpathlineto{\pgfqpoint{3.408705in}{0.808801in}}%
\pgfpathlineto{\pgfqpoint{3.409001in}{0.803081in}}%
\pgfpathlineto{\pgfqpoint{3.409297in}{0.796724in}}%
\pgfpathlineto{\pgfqpoint{3.409593in}{0.808384in}}%
\pgfpathlineto{\pgfqpoint{3.409889in}{0.808537in}}%
\pgfpathlineto{\pgfqpoint{3.410185in}{0.808689in}}%
\pgfpathlineto{\pgfqpoint{3.410481in}{0.808760in}}%
\pgfpathlineto{\pgfqpoint{3.410777in}{0.808724in}}%
\pgfpathlineto{\pgfqpoint{3.411073in}{0.808365in}}%
\pgfpathlineto{\pgfqpoint{3.411369in}{0.808350in}}%
\pgfpathlineto{\pgfqpoint{3.411665in}{0.808342in}}%
\pgfpathlineto{\pgfqpoint{3.411961in}{0.808333in}}%
\pgfpathlineto{\pgfqpoint{3.412258in}{0.808325in}}%
\pgfpathlineto{\pgfqpoint{3.412554in}{0.808316in}}%
\pgfpathlineto{\pgfqpoint{3.412850in}{0.808308in}}%
\pgfpathlineto{\pgfqpoint{3.413146in}{0.808300in}}%
\pgfpathlineto{\pgfqpoint{3.413442in}{0.808291in}}%
\pgfpathlineto{\pgfqpoint{3.413738in}{0.808283in}}%
\pgfpathlineto{\pgfqpoint{3.414034in}{0.808274in}}%
\pgfpathlineto{\pgfqpoint{3.414330in}{0.808266in}}%
\pgfpathlineto{\pgfqpoint{3.414626in}{0.808257in}}%
\pgfpathlineto{\pgfqpoint{3.414922in}{0.808249in}}%
\pgfpathlineto{\pgfqpoint{3.415218in}{0.808240in}}%
\pgfpathlineto{\pgfqpoint{3.415514in}{0.808232in}}%
\pgfpathlineto{\pgfqpoint{3.415810in}{0.808223in}}%
\pgfpathlineto{\pgfqpoint{3.416106in}{0.808217in}}%
\pgfpathlineto{\pgfqpoint{3.416402in}{0.808223in}}%
\pgfpathlineto{\pgfqpoint{3.416698in}{0.808230in}}%
\pgfpathlineto{\pgfqpoint{3.416994in}{0.808237in}}%
\pgfpathlineto{\pgfqpoint{3.417290in}{0.808244in}}%
\pgfpathlineto{\pgfqpoint{3.417586in}{0.808251in}}%
\pgfpathlineto{\pgfqpoint{3.417882in}{0.808258in}}%
\pgfpathlineto{\pgfqpoint{3.418178in}{0.808266in}}%
\pgfpathlineto{\pgfqpoint{3.418474in}{0.808274in}}%
\pgfpathlineto{\pgfqpoint{3.418770in}{0.808283in}}%
\pgfpathlineto{\pgfqpoint{3.419066in}{0.808292in}}%
\pgfpathlineto{\pgfqpoint{3.419362in}{0.808300in}}%
\pgfpathlineto{\pgfqpoint{3.419658in}{0.808309in}}%
\pgfpathlineto{\pgfqpoint{3.419954in}{0.808318in}}%
\pgfpathlineto{\pgfqpoint{3.420250in}{0.808326in}}%
\pgfpathlineto{\pgfqpoint{3.420546in}{0.808335in}}%
\pgfpathlineto{\pgfqpoint{3.420842in}{0.808343in}}%
\pgfpathlineto{\pgfqpoint{3.421138in}{0.808352in}}%
\pgfpathlineto{\pgfqpoint{3.421434in}{0.808361in}}%
\pgfpathlineto{\pgfqpoint{3.421730in}{0.808369in}}%
\pgfpathlineto{\pgfqpoint{3.422026in}{0.808378in}}%
\pgfpathlineto{\pgfqpoint{3.422322in}{0.808387in}}%
\pgfpathlineto{\pgfqpoint{3.422618in}{0.808395in}}%
\pgfpathlineto{\pgfqpoint{3.422914in}{0.808404in}}%
\pgfpathlineto{\pgfqpoint{3.423210in}{0.808413in}}%
\pgfpathlineto{\pgfqpoint{3.423506in}{0.808421in}}%
\pgfpathlineto{\pgfqpoint{3.423802in}{0.806643in}}%
\pgfpathlineto{\pgfqpoint{3.424098in}{0.800447in}}%
\pgfpathlineto{\pgfqpoint{3.424394in}{0.794052in}}%
\pgfpathlineto{\pgfqpoint{3.424690in}{0.793346in}}%
\pgfpathlineto{\pgfqpoint{3.424986in}{0.808026in}}%
\pgfpathlineto{\pgfqpoint{3.425282in}{0.808163in}}%
\pgfpathlineto{\pgfqpoint{3.425578in}{0.808179in}}%
\pgfpathlineto{\pgfqpoint{3.425874in}{0.808174in}}%
\pgfpathlineto{\pgfqpoint{3.426170in}{0.808169in}}%
\pgfpathlineto{\pgfqpoint{3.426466in}{0.808164in}}%
\pgfpathlineto{\pgfqpoint{3.426762in}{0.808159in}}%
\pgfpathlineto{\pgfqpoint{3.427058in}{0.808154in}}%
\pgfpathlineto{\pgfqpoint{3.427354in}{0.808149in}}%
\pgfpathlineto{\pgfqpoint{3.427650in}{0.808145in}}%
\pgfpathlineto{\pgfqpoint{3.427946in}{0.808140in}}%
\pgfpathlineto{\pgfqpoint{3.428242in}{0.808135in}}%
\pgfpathlineto{\pgfqpoint{3.428538in}{0.808130in}}%
\pgfpathlineto{\pgfqpoint{3.428834in}{0.808125in}}%
\pgfpathlineto{\pgfqpoint{3.429130in}{0.808120in}}%
\pgfpathlineto{\pgfqpoint{3.429426in}{0.808116in}}%
\pgfpathlineto{\pgfqpoint{3.429722in}{0.808111in}}%
\pgfpathlineto{\pgfqpoint{3.430018in}{0.808106in}}%
\pgfpathlineto{\pgfqpoint{3.430314in}{0.808101in}}%
\pgfpathlineto{\pgfqpoint{3.430610in}{0.808096in}}%
\pgfpathlineto{\pgfqpoint{3.430906in}{0.808091in}}%
\pgfpathlineto{\pgfqpoint{3.431202in}{0.808086in}}%
\pgfpathlineto{\pgfqpoint{3.431498in}{0.808082in}}%
\pgfpathlineto{\pgfqpoint{3.431794in}{0.802975in}}%
\pgfpathlineto{\pgfqpoint{3.432090in}{0.804030in}}%
\pgfpathlineto{\pgfqpoint{3.432386in}{0.808074in}}%
\pgfpathlineto{\pgfqpoint{3.432682in}{0.807815in}}%
\pgfpathlineto{\pgfqpoint{3.432978in}{0.807473in}}%
\pgfpathlineto{\pgfqpoint{3.433274in}{0.807130in}}%
\pgfpathlineto{\pgfqpoint{3.433570in}{0.806787in}}%
\pgfpathlineto{\pgfqpoint{3.433866in}{0.806445in}}%
\pgfpathlineto{\pgfqpoint{3.434162in}{0.806102in}}%
\pgfpathlineto{\pgfqpoint{3.434458in}{0.805760in}}%
\pgfpathlineto{\pgfqpoint{3.434754in}{0.805417in}}%
\pgfpathlineto{\pgfqpoint{3.435050in}{0.805074in}}%
\pgfpathlineto{\pgfqpoint{3.435346in}{0.804732in}}%
\pgfpathlineto{\pgfqpoint{3.435642in}{0.804389in}}%
\pgfpathlineto{\pgfqpoint{3.435938in}{0.804047in}}%
\pgfpathlineto{\pgfqpoint{3.436234in}{0.803704in}}%
\pgfpathlineto{\pgfqpoint{3.436530in}{0.803361in}}%
\pgfpathlineto{\pgfqpoint{3.436826in}{0.803019in}}%
\pgfpathlineto{\pgfqpoint{3.437122in}{0.802676in}}%
\pgfpathlineto{\pgfqpoint{3.437418in}{0.802334in}}%
\pgfpathlineto{\pgfqpoint{3.437714in}{0.801991in}}%
\pgfpathlineto{\pgfqpoint{3.438010in}{0.801648in}}%
\pgfpathlineto{\pgfqpoint{3.438306in}{0.801306in}}%
\pgfpathlineto{\pgfqpoint{3.438602in}{0.800963in}}%
\pgfpathlineto{\pgfqpoint{3.438898in}{0.800621in}}%
\pgfpathlineto{\pgfqpoint{3.439194in}{0.800278in}}%
\pgfpathlineto{\pgfqpoint{3.439490in}{0.799935in}}%
\pgfpathlineto{\pgfqpoint{3.439786in}{0.799593in}}%
\pgfpathlineto{\pgfqpoint{3.440082in}{0.799250in}}%
\pgfpathlineto{\pgfqpoint{3.440378in}{0.798908in}}%
\pgfpathlineto{\pgfqpoint{3.440674in}{0.798565in}}%
\pgfpathlineto{\pgfqpoint{3.440970in}{0.798222in}}%
\pgfpathlineto{\pgfqpoint{3.441266in}{0.797880in}}%
\pgfpathlineto{\pgfqpoint{3.441562in}{0.797537in}}%
\pgfpathlineto{\pgfqpoint{3.441858in}{0.797195in}}%
\pgfpathlineto{\pgfqpoint{3.442154in}{0.796852in}}%
\pgfpathlineto{\pgfqpoint{3.442450in}{0.796509in}}%
\pgfpathlineto{\pgfqpoint{3.442746in}{0.796167in}}%
\pgfpathlineto{\pgfqpoint{3.443042in}{0.795824in}}%
\pgfpathlineto{\pgfqpoint{3.443338in}{0.795482in}}%
\pgfpathlineto{\pgfqpoint{3.443634in}{0.795139in}}%
\pgfpathlineto{\pgfqpoint{3.443930in}{0.794796in}}%
\pgfpathlineto{\pgfqpoint{3.444226in}{0.794454in}}%
\pgfpathlineto{\pgfqpoint{3.444522in}{0.794111in}}%
\pgfpathlineto{\pgfqpoint{3.444818in}{0.793769in}}%
\pgfpathlineto{\pgfqpoint{3.445114in}{0.793426in}}%
\pgfpathlineto{\pgfqpoint{3.445410in}{0.793083in}}%
\pgfpathlineto{\pgfqpoint{3.445706in}{0.792741in}}%
\pgfpathlineto{\pgfqpoint{3.446002in}{0.792398in}}%
\pgfpathlineto{\pgfqpoint{3.446298in}{0.792056in}}%
\pgfpathlineto{\pgfqpoint{3.446594in}{0.791713in}}%
\pgfpathlineto{\pgfqpoint{3.446890in}{0.791370in}}%
\pgfpathlineto{\pgfqpoint{3.447186in}{0.791028in}}%
\pgfpathlineto{\pgfqpoint{3.447482in}{0.790685in}}%
\pgfpathlineto{\pgfqpoint{3.447778in}{0.790343in}}%
\pgfpathlineto{\pgfqpoint{3.448074in}{0.790000in}}%
\pgfpathlineto{\pgfqpoint{3.448370in}{0.789657in}}%
\pgfpathlineto{\pgfqpoint{3.448666in}{0.789315in}}%
\pgfpathlineto{\pgfqpoint{3.448962in}{0.788972in}}%
\pgfpathlineto{\pgfqpoint{3.449258in}{0.788630in}}%
\pgfpathlineto{\pgfqpoint{3.449554in}{0.788287in}}%
\pgfpathlineto{\pgfqpoint{3.449850in}{0.787944in}}%
\pgfpathlineto{\pgfqpoint{3.450146in}{0.787602in}}%
\pgfpathlineto{\pgfqpoint{3.450442in}{0.787259in}}%
\pgfpathlineto{\pgfqpoint{3.450738in}{0.786917in}}%
\pgfpathlineto{\pgfqpoint{3.451034in}{0.786574in}}%
\pgfpathlineto{\pgfqpoint{3.451330in}{0.787189in}}%
\pgfpathlineto{\pgfqpoint{3.451626in}{0.809020in}}%
\pgfpathlineto{\pgfqpoint{3.451922in}{0.800995in}}%
\pgfpathlineto{\pgfqpoint{3.452218in}{0.798195in}}%
\pgfpathlineto{\pgfqpoint{3.452514in}{0.798625in}}%
\pgfpathlineto{\pgfqpoint{3.452810in}{0.799056in}}%
\pgfpathlineto{\pgfqpoint{3.453106in}{0.799487in}}%
\pgfpathlineto{\pgfqpoint{3.453402in}{0.799918in}}%
\pgfpathlineto{\pgfqpoint{3.453698in}{0.800349in}}%
\pgfpathlineto{\pgfqpoint{3.453994in}{0.800780in}}%
\pgfpathlineto{\pgfqpoint{3.454290in}{0.801211in}}%
\pgfpathlineto{\pgfqpoint{3.454586in}{0.801641in}}%
\pgfpathlineto{\pgfqpoint{3.454882in}{0.802072in}}%
\pgfpathlineto{\pgfqpoint{3.455178in}{0.802503in}}%
\pgfpathlineto{\pgfqpoint{3.455474in}{0.802934in}}%
\pgfpathlineto{\pgfqpoint{3.455770in}{0.803365in}}%
\pgfpathlineto{\pgfqpoint{3.456066in}{0.803796in}}%
\pgfpathlineto{\pgfqpoint{3.456362in}{0.804226in}}%
\pgfpathlineto{\pgfqpoint{3.456658in}{0.804657in}}%
\pgfpathlineto{\pgfqpoint{3.456954in}{0.805088in}}%
\pgfpathlineto{\pgfqpoint{3.457250in}{0.805519in}}%
\pgfpathlineto{\pgfqpoint{3.457546in}{0.805950in}}%
\pgfpathlineto{\pgfqpoint{3.457842in}{0.806381in}}%
\pgfpathlineto{\pgfqpoint{3.458138in}{0.806812in}}%
\pgfpathlineto{\pgfqpoint{3.458434in}{0.807242in}}%
\pgfpathlineto{\pgfqpoint{3.458730in}{0.807673in}}%
\pgfpathlineto{\pgfqpoint{3.459026in}{0.808104in}}%
\pgfpathlineto{\pgfqpoint{3.459322in}{0.808535in}}%
\pgfpathlineto{\pgfqpoint{3.459618in}{0.809082in}}%
\pgfpathlineto{\pgfqpoint{3.459914in}{0.809290in}}%
\pgfpathlineto{\pgfqpoint{3.460210in}{0.809301in}}%
\pgfpathlineto{\pgfqpoint{3.460506in}{0.809300in}}%
\pgfpathlineto{\pgfqpoint{3.460802in}{0.809297in}}%
\pgfpathlineto{\pgfqpoint{3.461098in}{0.797382in}}%
\pgfpathlineto{\pgfqpoint{3.461394in}{0.809236in}}%
\pgfpathlineto{\pgfqpoint{3.461690in}{0.809284in}}%
\pgfpathlineto{\pgfqpoint{3.461986in}{0.809278in}}%
\pgfpathlineto{\pgfqpoint{3.462282in}{0.809271in}}%
\pgfpathlineto{\pgfqpoint{3.462578in}{0.809264in}}%
\pgfpathlineto{\pgfqpoint{3.462874in}{0.809257in}}%
\pgfpathlineto{\pgfqpoint{3.463170in}{0.809250in}}%
\pgfpathlineto{\pgfqpoint{3.463466in}{0.809243in}}%
\pgfpathlineto{\pgfqpoint{3.463762in}{0.809236in}}%
\pgfpathlineto{\pgfqpoint{3.464058in}{0.809229in}}%
\pgfpathlineto{\pgfqpoint{3.464354in}{0.809222in}}%
\pgfpathlineto{\pgfqpoint{3.464650in}{0.809216in}}%
\pgfpathlineto{\pgfqpoint{3.464946in}{0.809209in}}%
\pgfpathlineto{\pgfqpoint{3.465242in}{0.809202in}}%
\pgfpathlineto{\pgfqpoint{3.465538in}{0.809195in}}%
\pgfpathlineto{\pgfqpoint{3.465834in}{0.809188in}}%
\pgfpathlineto{\pgfqpoint{3.466130in}{0.809181in}}%
\pgfpathlineto{\pgfqpoint{3.466426in}{0.809174in}}%
\pgfpathlineto{\pgfqpoint{3.466722in}{0.809167in}}%
\pgfpathlineto{\pgfqpoint{3.467018in}{0.809160in}}%
\pgfpathlineto{\pgfqpoint{3.467314in}{0.809153in}}%
\pgfpathlineto{\pgfqpoint{3.467610in}{0.809147in}}%
\pgfpathlineto{\pgfqpoint{3.467906in}{0.809140in}}%
\pgfpathlineto{\pgfqpoint{3.468202in}{0.809133in}}%
\pgfpathlineto{\pgfqpoint{3.468498in}{0.809126in}}%
\pgfpathlineto{\pgfqpoint{3.468794in}{0.809119in}}%
\pgfpathlineto{\pgfqpoint{3.469090in}{0.809112in}}%
\pgfpathlineto{\pgfqpoint{3.469386in}{0.809105in}}%
\pgfpathlineto{\pgfqpoint{3.469682in}{0.809098in}}%
\pgfpathlineto{\pgfqpoint{3.469978in}{0.809091in}}%
\pgfpathlineto{\pgfqpoint{3.470274in}{0.809084in}}%
\pgfpathlineto{\pgfqpoint{3.470570in}{0.809078in}}%
\pgfpathlineto{\pgfqpoint{3.470866in}{0.809071in}}%
\pgfpathlineto{\pgfqpoint{3.471162in}{0.809064in}}%
\pgfpathlineto{\pgfqpoint{3.471458in}{0.809057in}}%
\pgfpathlineto{\pgfqpoint{3.471754in}{0.809050in}}%
\pgfpathlineto{\pgfqpoint{3.472050in}{0.809043in}}%
\pgfpathlineto{\pgfqpoint{3.472346in}{0.809036in}}%
\pgfpathlineto{\pgfqpoint{3.472642in}{0.809029in}}%
\pgfpathlineto{\pgfqpoint{3.472938in}{0.809022in}}%
\pgfpathlineto{\pgfqpoint{3.473234in}{0.809016in}}%
\pgfpathlineto{\pgfqpoint{3.473530in}{0.807752in}}%
\pgfpathlineto{\pgfqpoint{3.473826in}{0.804479in}}%
\pgfpathlineto{\pgfqpoint{3.474122in}{0.801179in}}%
\pgfpathlineto{\pgfqpoint{3.474418in}{0.797878in}}%
\pgfpathlineto{\pgfqpoint{3.474714in}{0.794578in}}%
\pgfpathlineto{\pgfqpoint{3.475010in}{0.791278in}}%
\pgfpathlineto{\pgfqpoint{3.475306in}{0.787978in}}%
\pgfpathlineto{\pgfqpoint{3.475602in}{0.786087in}}%
\pgfpathlineto{\pgfqpoint{3.475898in}{0.786071in}}%
\pgfpathlineto{\pgfqpoint{3.476194in}{0.786065in}}%
\pgfpathlineto{\pgfqpoint{3.476490in}{0.786058in}}%
\pgfpathlineto{\pgfqpoint{3.476786in}{0.786052in}}%
\pgfpathlineto{\pgfqpoint{3.477082in}{0.786045in}}%
\pgfpathlineto{\pgfqpoint{3.477378in}{0.786039in}}%
\pgfpathlineto{\pgfqpoint{3.477674in}{0.786032in}}%
\pgfpathlineto{\pgfqpoint{3.477970in}{0.786025in}}%
\pgfpathlineto{\pgfqpoint{3.478266in}{0.786019in}}%
\pgfpathlineto{\pgfqpoint{3.478562in}{0.786012in}}%
\pgfpathlineto{\pgfqpoint{3.478858in}{0.786006in}}%
\pgfpathlineto{\pgfqpoint{3.479154in}{0.785999in}}%
\pgfpathlineto{\pgfqpoint{3.479451in}{0.785993in}}%
\pgfpathlineto{\pgfqpoint{3.479747in}{0.785986in}}%
\pgfpathlineto{\pgfqpoint{3.480043in}{0.785980in}}%
\pgfpathlineto{\pgfqpoint{3.480339in}{0.785973in}}%
\pgfpathlineto{\pgfqpoint{3.480635in}{0.785967in}}%
\pgfpathlineto{\pgfqpoint{3.480931in}{0.785960in}}%
\pgfpathlineto{\pgfqpoint{3.481227in}{0.785954in}}%
\pgfpathlineto{\pgfqpoint{3.481523in}{0.785947in}}%
\pgfpathlineto{\pgfqpoint{3.481819in}{0.785672in}}%
\pgfpathlineto{\pgfqpoint{3.482115in}{0.785457in}}%
\pgfpathlineto{\pgfqpoint{3.482411in}{0.785446in}}%
\pgfpathlineto{\pgfqpoint{3.482707in}{0.785436in}}%
\pgfpathlineto{\pgfqpoint{3.483003in}{0.785425in}}%
\pgfpathlineto{\pgfqpoint{3.483299in}{0.785415in}}%
\pgfpathlineto{\pgfqpoint{3.483595in}{0.785405in}}%
\pgfpathlineto{\pgfqpoint{3.483891in}{0.785394in}}%
\pgfpathlineto{\pgfqpoint{3.484187in}{0.785384in}}%
\pgfpathlineto{\pgfqpoint{3.484483in}{0.785373in}}%
\pgfpathlineto{\pgfqpoint{3.484779in}{0.785363in}}%
\pgfpathlineto{\pgfqpoint{3.485075in}{0.785352in}}%
\pgfpathlineto{\pgfqpoint{3.485371in}{0.785342in}}%
\pgfpathlineto{\pgfqpoint{3.485667in}{0.785331in}}%
\pgfpathlineto{\pgfqpoint{3.485963in}{0.785321in}}%
\pgfpathlineto{\pgfqpoint{3.486259in}{0.785311in}}%
\pgfpathlineto{\pgfqpoint{3.486555in}{0.785300in}}%
\pgfpathlineto{\pgfqpoint{3.486851in}{0.785290in}}%
\pgfpathlineto{\pgfqpoint{3.487147in}{0.785279in}}%
\pgfpathlineto{\pgfqpoint{3.487443in}{0.785269in}}%
\pgfpathlineto{\pgfqpoint{3.487739in}{0.785258in}}%
\pgfpathlineto{\pgfqpoint{3.488035in}{0.785248in}}%
\pgfpathlineto{\pgfqpoint{3.488331in}{0.785237in}}%
\pgfpathlineto{\pgfqpoint{3.488627in}{0.785227in}}%
\pgfpathlineto{\pgfqpoint{3.488923in}{0.785216in}}%
\pgfpathlineto{\pgfqpoint{3.489219in}{0.785206in}}%
\pgfpathlineto{\pgfqpoint{3.489515in}{0.785196in}}%
\pgfpathlineto{\pgfqpoint{3.489811in}{0.785185in}}%
\pgfpathlineto{\pgfqpoint{3.490107in}{0.785175in}}%
\pgfpathlineto{\pgfqpoint{3.490403in}{0.785164in}}%
\pgfpathlineto{\pgfqpoint{3.490699in}{0.785154in}}%
\pgfpathlineto{\pgfqpoint{3.490995in}{0.785143in}}%
\pgfpathlineto{\pgfqpoint{3.491291in}{0.785133in}}%
\pgfpathlineto{\pgfqpoint{3.491587in}{0.785122in}}%
\pgfpathlineto{\pgfqpoint{3.491883in}{0.785112in}}%
\pgfpathlineto{\pgfqpoint{3.492179in}{0.785101in}}%
\pgfpathlineto{\pgfqpoint{3.492475in}{0.785091in}}%
\pgfpathlineto{\pgfqpoint{3.492771in}{0.785081in}}%
\pgfpathlineto{\pgfqpoint{3.493067in}{0.785070in}}%
\pgfpathlineto{\pgfqpoint{3.493363in}{0.785060in}}%
\pgfpathlineto{\pgfqpoint{3.493659in}{0.785049in}}%
\pgfpathlineto{\pgfqpoint{3.493955in}{0.785039in}}%
\pgfpathlineto{\pgfqpoint{3.494251in}{0.785028in}}%
\pgfpathlineto{\pgfqpoint{3.494547in}{0.785018in}}%
\pgfpathlineto{\pgfqpoint{3.494843in}{0.785007in}}%
\pgfpathlineto{\pgfqpoint{3.495139in}{0.784997in}}%
\pgfpathlineto{\pgfqpoint{3.495435in}{0.784986in}}%
\pgfpathlineto{\pgfqpoint{3.495731in}{0.784976in}}%
\pgfpathlineto{\pgfqpoint{3.496027in}{0.784966in}}%
\pgfpathlineto{\pgfqpoint{3.496323in}{0.784955in}}%
\pgfpathlineto{\pgfqpoint{3.496619in}{0.784945in}}%
\pgfpathlineto{\pgfqpoint{3.496915in}{0.784934in}}%
\pgfpathlineto{\pgfqpoint{3.497211in}{0.784924in}}%
\pgfpathlineto{\pgfqpoint{3.497507in}{0.784913in}}%
\pgfpathlineto{\pgfqpoint{3.497803in}{0.784903in}}%
\pgfpathlineto{\pgfqpoint{3.498099in}{0.784892in}}%
\pgfpathlineto{\pgfqpoint{3.498395in}{0.784882in}}%
\pgfpathlineto{\pgfqpoint{3.498691in}{0.784871in}}%
\pgfpathlineto{\pgfqpoint{3.498987in}{0.784861in}}%
\pgfpathlineto{\pgfqpoint{3.499283in}{0.784851in}}%
\pgfpathlineto{\pgfqpoint{3.499579in}{0.784840in}}%
\pgfpathlineto{\pgfqpoint{3.499875in}{0.784830in}}%
\pgfpathlineto{\pgfqpoint{3.500171in}{0.784819in}}%
\pgfpathlineto{\pgfqpoint{3.500467in}{0.784809in}}%
\pgfpathlineto{\pgfqpoint{3.500763in}{0.784798in}}%
\pgfpathlineto{\pgfqpoint{3.501059in}{0.784788in}}%
\pgfpathlineto{\pgfqpoint{3.501355in}{0.784785in}}%
\pgfpathlineto{\pgfqpoint{3.501651in}{0.794869in}}%
\pgfpathlineto{\pgfqpoint{3.501947in}{0.808039in}}%
\pgfpathlineto{\pgfqpoint{3.502243in}{0.808043in}}%
\pgfpathlineto{\pgfqpoint{3.502539in}{0.808034in}}%
\pgfpathlineto{\pgfqpoint{3.502835in}{0.808024in}}%
\pgfpathlineto{\pgfqpoint{3.503131in}{0.808013in}}%
\pgfpathlineto{\pgfqpoint{3.503427in}{0.808003in}}%
\pgfpathlineto{\pgfqpoint{3.503723in}{0.807993in}}%
\pgfpathlineto{\pgfqpoint{3.504019in}{0.807982in}}%
\pgfpathlineto{\pgfqpoint{3.504315in}{0.807972in}}%
\pgfpathlineto{\pgfqpoint{3.504611in}{0.807961in}}%
\pgfpathlineto{\pgfqpoint{3.504907in}{0.807951in}}%
\pgfpathlineto{\pgfqpoint{3.505203in}{0.807941in}}%
\pgfpathlineto{\pgfqpoint{3.505499in}{0.807930in}}%
\pgfpathlineto{\pgfqpoint{3.505795in}{0.807920in}}%
\pgfpathlineto{\pgfqpoint{3.506091in}{0.807909in}}%
\pgfpathlineto{\pgfqpoint{3.506387in}{0.807899in}}%
\pgfpathlineto{\pgfqpoint{3.506683in}{0.807889in}}%
\pgfpathlineto{\pgfqpoint{3.506979in}{0.807878in}}%
\pgfpathlineto{\pgfqpoint{3.507275in}{0.807868in}}%
\pgfpathlineto{\pgfqpoint{3.507571in}{0.807857in}}%
\pgfpathlineto{\pgfqpoint{3.507867in}{0.807847in}}%
\pgfpathlineto{\pgfqpoint{3.508163in}{0.807837in}}%
\pgfpathlineto{\pgfqpoint{3.508459in}{0.806951in}}%
\pgfpathlineto{\pgfqpoint{3.508755in}{0.784701in}}%
\pgfpathlineto{\pgfqpoint{3.509051in}{0.784737in}}%
\pgfpathlineto{\pgfqpoint{3.509347in}{0.784731in}}%
\pgfpathlineto{\pgfqpoint{3.509643in}{0.784722in}}%
\pgfpathlineto{\pgfqpoint{3.509939in}{0.784714in}}%
\pgfpathlineto{\pgfqpoint{3.510235in}{0.784706in}}%
\pgfpathlineto{\pgfqpoint{3.510531in}{0.784698in}}%
\pgfpathlineto{\pgfqpoint{3.510827in}{0.784982in}}%
\pgfpathlineto{\pgfqpoint{3.511123in}{0.785551in}}%
\pgfpathlineto{\pgfqpoint{3.511419in}{0.786119in}}%
\pgfpathlineto{\pgfqpoint{3.511715in}{0.786687in}}%
\pgfpathlineto{\pgfqpoint{3.512011in}{0.787255in}}%
\pgfpathlineto{\pgfqpoint{3.512307in}{0.787824in}}%
\pgfpathlineto{\pgfqpoint{3.512603in}{0.788392in}}%
\pgfpathlineto{\pgfqpoint{3.512899in}{0.788960in}}%
\pgfpathlineto{\pgfqpoint{3.513195in}{0.789528in}}%
\pgfpathlineto{\pgfqpoint{3.513491in}{0.790097in}}%
\pgfpathlineto{\pgfqpoint{3.513787in}{0.790665in}}%
\pgfpathlineto{\pgfqpoint{3.514083in}{0.791233in}}%
\pgfpathlineto{\pgfqpoint{3.514379in}{0.791801in}}%
\pgfpathlineto{\pgfqpoint{3.514675in}{0.792370in}}%
\pgfpathlineto{\pgfqpoint{3.514971in}{0.792938in}}%
\pgfpathlineto{\pgfqpoint{3.515267in}{0.793506in}}%
\pgfpathlineto{\pgfqpoint{3.515563in}{0.793670in}}%
\pgfpathlineto{\pgfqpoint{3.515859in}{0.793740in}}%
\pgfpathlineto{\pgfqpoint{3.516155in}{0.794208in}}%
\pgfpathlineto{\pgfqpoint{3.516451in}{0.794675in}}%
\pgfpathlineto{\pgfqpoint{3.516747in}{0.795143in}}%
\pgfpathlineto{\pgfqpoint{3.517043in}{0.795610in}}%
\pgfpathlineto{\pgfqpoint{3.517339in}{0.796078in}}%
\pgfpathlineto{\pgfqpoint{3.517635in}{0.796545in}}%
\pgfpathlineto{\pgfqpoint{3.517931in}{0.796992in}}%
\pgfpathlineto{\pgfqpoint{3.518227in}{0.797410in}}%
\pgfpathlineto{\pgfqpoint{3.518523in}{0.797829in}}%
\pgfpathlineto{\pgfqpoint{3.518819in}{0.798289in}}%
\pgfpathlineto{\pgfqpoint{3.519115in}{0.798847in}}%
\pgfpathlineto{\pgfqpoint{3.519411in}{0.799409in}}%
\pgfpathlineto{\pgfqpoint{3.519707in}{0.799972in}}%
\pgfpathlineto{\pgfqpoint{3.520003in}{0.800534in}}%
\pgfpathlineto{\pgfqpoint{3.520299in}{0.801096in}}%
\pgfpathlineto{\pgfqpoint{3.520595in}{0.801659in}}%
\pgfpathlineto{\pgfqpoint{3.520891in}{0.802221in}}%
\pgfpathlineto{\pgfqpoint{3.521187in}{0.802783in}}%
\pgfpathlineto{\pgfqpoint{3.521483in}{0.803346in}}%
\pgfpathlineto{\pgfqpoint{3.521779in}{0.803908in}}%
\pgfpathlineto{\pgfqpoint{3.522075in}{0.804471in}}%
\pgfpathlineto{\pgfqpoint{3.522371in}{0.804755in}}%
\pgfpathlineto{\pgfqpoint{3.522667in}{0.783404in}}%
\pgfpathlineto{\pgfqpoint{3.522963in}{0.782212in}}%
\pgfpathlineto{\pgfqpoint{3.523259in}{0.782196in}}%
\pgfpathlineto{\pgfqpoint{3.523555in}{0.782181in}}%
\pgfpathlineto{\pgfqpoint{3.523851in}{0.782168in}}%
\pgfpathlineto{\pgfqpoint{3.524147in}{0.782155in}}%
\pgfpathlineto{\pgfqpoint{3.524443in}{0.782141in}}%
\pgfpathlineto{\pgfqpoint{3.524739in}{0.782126in}}%
\pgfpathlineto{\pgfqpoint{3.525035in}{0.782104in}}%
\pgfpathlineto{\pgfqpoint{3.525331in}{0.782055in}}%
\pgfpathlineto{\pgfqpoint{3.525627in}{0.782020in}}%
\pgfpathlineto{\pgfqpoint{3.525923in}{0.782016in}}%
\pgfpathlineto{\pgfqpoint{3.526219in}{0.782014in}}%
\pgfpathlineto{\pgfqpoint{3.526515in}{0.782012in}}%
\pgfpathlineto{\pgfqpoint{3.526811in}{0.782010in}}%
\pgfpathlineto{\pgfqpoint{3.527107in}{0.782008in}}%
\pgfpathlineto{\pgfqpoint{3.527403in}{0.782006in}}%
\pgfpathlineto{\pgfqpoint{3.527699in}{0.782004in}}%
\pgfpathlineto{\pgfqpoint{3.527995in}{0.782002in}}%
\pgfpathlineto{\pgfqpoint{3.528291in}{0.782000in}}%
\pgfpathlineto{\pgfqpoint{3.528587in}{0.781998in}}%
\pgfpathlineto{\pgfqpoint{3.528883in}{0.781997in}}%
\pgfpathlineto{\pgfqpoint{3.529179in}{0.781995in}}%
\pgfpathlineto{\pgfqpoint{3.529475in}{0.781993in}}%
\pgfpathlineto{\pgfqpoint{3.529771in}{0.781991in}}%
\pgfpathlineto{\pgfqpoint{3.530067in}{0.781989in}}%
\pgfpathlineto{\pgfqpoint{3.530363in}{0.781987in}}%
\pgfpathlineto{\pgfqpoint{3.530659in}{0.781985in}}%
\pgfpathlineto{\pgfqpoint{3.530955in}{0.781983in}}%
\pgfpathlineto{\pgfqpoint{3.531251in}{0.781981in}}%
\pgfpathlineto{\pgfqpoint{3.531547in}{0.781979in}}%
\pgfpathlineto{\pgfqpoint{3.531843in}{0.781977in}}%
\pgfpathlineto{\pgfqpoint{3.532139in}{0.781975in}}%
\pgfpathlineto{\pgfqpoint{3.532435in}{0.781973in}}%
\pgfpathlineto{\pgfqpoint{3.532731in}{0.781971in}}%
\pgfpathlineto{\pgfqpoint{3.533027in}{0.781969in}}%
\pgfpathlineto{\pgfqpoint{3.533323in}{0.781967in}}%
\pgfpathlineto{\pgfqpoint{3.533619in}{0.781965in}}%
\pgfpathlineto{\pgfqpoint{3.533915in}{0.781963in}}%
\pgfpathlineto{\pgfqpoint{3.534211in}{0.781961in}}%
\pgfpathlineto{\pgfqpoint{3.534507in}{0.781959in}}%
\pgfpathlineto{\pgfqpoint{3.534803in}{0.781958in}}%
\pgfpathlineto{\pgfqpoint{3.535099in}{0.781956in}}%
\pgfpathlineto{\pgfqpoint{3.535395in}{0.781954in}}%
\pgfpathlineto{\pgfqpoint{3.535691in}{0.781952in}}%
\pgfpathlineto{\pgfqpoint{3.535987in}{0.781950in}}%
\pgfpathlineto{\pgfqpoint{3.536283in}{0.781948in}}%
\pgfpathlineto{\pgfqpoint{3.536579in}{0.781946in}}%
\pgfpathlineto{\pgfqpoint{3.536875in}{0.781944in}}%
\pgfpathlineto{\pgfqpoint{3.537171in}{0.781942in}}%
\pgfpathlineto{\pgfqpoint{3.537467in}{0.781940in}}%
\pgfpathlineto{\pgfqpoint{3.537763in}{0.781938in}}%
\pgfpathlineto{\pgfqpoint{3.538059in}{0.781936in}}%
\pgfpathlineto{\pgfqpoint{3.538355in}{0.781934in}}%
\pgfpathlineto{\pgfqpoint{3.538651in}{0.781932in}}%
\pgfpathlineto{\pgfqpoint{3.538947in}{0.781930in}}%
\pgfpathlineto{\pgfqpoint{3.539243in}{0.781928in}}%
\pgfpathlineto{\pgfqpoint{3.539539in}{0.781926in}}%
\pgfpathlineto{\pgfqpoint{3.539835in}{0.781924in}}%
\pgfpathlineto{\pgfqpoint{3.540131in}{0.781922in}}%
\pgfpathlineto{\pgfqpoint{3.540427in}{0.781921in}}%
\pgfpathlineto{\pgfqpoint{3.540723in}{0.781919in}}%
\pgfpathlineto{\pgfqpoint{3.541019in}{0.781917in}}%
\pgfpathlineto{\pgfqpoint{3.541315in}{0.781915in}}%
\pgfpathlineto{\pgfqpoint{3.541611in}{0.781913in}}%
\pgfpathlineto{\pgfqpoint{3.541907in}{0.781911in}}%
\pgfpathlineto{\pgfqpoint{3.542203in}{0.781909in}}%
\pgfpathlineto{\pgfqpoint{3.542499in}{0.781907in}}%
\pgfpathlineto{\pgfqpoint{3.542795in}{0.781905in}}%
\pgfpathlineto{\pgfqpoint{3.543091in}{0.781903in}}%
\pgfpathlineto{\pgfqpoint{3.543387in}{0.781901in}}%
\pgfpathlineto{\pgfqpoint{3.543683in}{0.781899in}}%
\pgfpathlineto{\pgfqpoint{3.543979in}{0.781897in}}%
\pgfpathlineto{\pgfqpoint{3.544275in}{0.781895in}}%
\pgfpathlineto{\pgfqpoint{3.544571in}{0.781893in}}%
\pgfpathlineto{\pgfqpoint{3.544867in}{0.781891in}}%
\pgfpathlineto{\pgfqpoint{3.545163in}{0.781889in}}%
\pgfpathlineto{\pgfqpoint{3.545459in}{0.781887in}}%
\pgfpathlineto{\pgfqpoint{3.545755in}{0.781885in}}%
\pgfpathlineto{\pgfqpoint{3.546051in}{0.781883in}}%
\pgfpathlineto{\pgfqpoint{3.546347in}{0.781882in}}%
\pgfpathlineto{\pgfqpoint{3.546643in}{0.781880in}}%
\pgfpathlineto{\pgfqpoint{3.546940in}{0.781878in}}%
\pgfpathlineto{\pgfqpoint{3.547236in}{0.781876in}}%
\pgfpathlineto{\pgfqpoint{3.547532in}{0.781874in}}%
\pgfpathlineto{\pgfqpoint{3.547828in}{0.781872in}}%
\pgfpathlineto{\pgfqpoint{3.548124in}{0.781870in}}%
\pgfpathlineto{\pgfqpoint{3.548420in}{0.781868in}}%
\pgfpathlineto{\pgfqpoint{3.548716in}{0.781866in}}%
\pgfpathlineto{\pgfqpoint{3.549012in}{0.781864in}}%
\pgfpathlineto{\pgfqpoint{3.549308in}{0.781862in}}%
\pgfpathlineto{\pgfqpoint{3.549604in}{0.781860in}}%
\pgfpathlineto{\pgfqpoint{3.549900in}{0.781858in}}%
\pgfpathlineto{\pgfqpoint{3.550196in}{0.781856in}}%
\pgfpathlineto{\pgfqpoint{3.550492in}{0.781854in}}%
\pgfpathlineto{\pgfqpoint{3.550788in}{0.781852in}}%
\pgfpathlineto{\pgfqpoint{3.551084in}{0.781850in}}%
\pgfpathlineto{\pgfqpoint{3.551380in}{0.781848in}}%
\pgfpathlineto{\pgfqpoint{3.551676in}{0.781846in}}%
\pgfpathlineto{\pgfqpoint{3.551972in}{0.781845in}}%
\pgfpathlineto{\pgfqpoint{3.552268in}{0.781843in}}%
\pgfpathlineto{\pgfqpoint{3.552564in}{0.781841in}}%
\pgfpathlineto{\pgfqpoint{3.552860in}{0.781839in}}%
\pgfpathlineto{\pgfqpoint{3.553156in}{0.781837in}}%
\pgfpathlineto{\pgfqpoint{3.553452in}{0.781835in}}%
\pgfpathlineto{\pgfqpoint{3.553748in}{0.781833in}}%
\pgfpathlineto{\pgfqpoint{3.554044in}{0.781831in}}%
\pgfpathlineto{\pgfqpoint{3.554340in}{0.781829in}}%
\pgfpathlineto{\pgfqpoint{3.554636in}{0.781827in}}%
\pgfpathlineto{\pgfqpoint{3.554932in}{0.781825in}}%
\pgfpathlineto{\pgfqpoint{3.555228in}{0.781823in}}%
\pgfpathlineto{\pgfqpoint{3.555524in}{0.781821in}}%
\pgfpathlineto{\pgfqpoint{3.555820in}{0.781819in}}%
\pgfpathlineto{\pgfqpoint{3.556116in}{0.781817in}}%
\pgfpathlineto{\pgfqpoint{3.556412in}{0.781815in}}%
\pgfpathlineto{\pgfqpoint{3.556708in}{0.781813in}}%
\pgfpathlineto{\pgfqpoint{3.557004in}{0.781811in}}%
\pgfpathlineto{\pgfqpoint{3.557300in}{0.781809in}}%
\pgfpathlineto{\pgfqpoint{3.557596in}{0.781807in}}%
\pgfpathlineto{\pgfqpoint{3.557892in}{0.781806in}}%
\pgfpathlineto{\pgfqpoint{3.558188in}{0.781804in}}%
\pgfpathlineto{\pgfqpoint{3.558484in}{0.781802in}}%
\pgfpathlineto{\pgfqpoint{3.558780in}{0.781800in}}%
\pgfpathlineto{\pgfqpoint{3.559076in}{0.781798in}}%
\pgfpathlineto{\pgfqpoint{3.559372in}{0.781796in}}%
\pgfpathlineto{\pgfqpoint{3.559668in}{0.781794in}}%
\pgfpathlineto{\pgfqpoint{3.559964in}{0.781792in}}%
\pgfpathlineto{\pgfqpoint{3.560260in}{0.781790in}}%
\pgfpathlineto{\pgfqpoint{3.560556in}{0.781788in}}%
\pgfpathlineto{\pgfqpoint{3.560852in}{0.781786in}}%
\pgfpathlineto{\pgfqpoint{3.561148in}{0.781784in}}%
\pgfpathlineto{\pgfqpoint{3.561444in}{0.781782in}}%
\pgfpathlineto{\pgfqpoint{3.561740in}{0.781780in}}%
\pgfpathlineto{\pgfqpoint{3.562036in}{0.781778in}}%
\pgfpathlineto{\pgfqpoint{3.562332in}{0.781776in}}%
\pgfpathlineto{\pgfqpoint{3.562628in}{0.781774in}}%
\pgfpathlineto{\pgfqpoint{3.562924in}{0.781772in}}%
\pgfpathlineto{\pgfqpoint{3.563220in}{0.781770in}}%
\pgfpathlineto{\pgfqpoint{3.563516in}{0.781769in}}%
\pgfpathlineto{\pgfqpoint{3.563812in}{0.781767in}}%
\pgfpathlineto{\pgfqpoint{3.564108in}{0.781765in}}%
\pgfpathlineto{\pgfqpoint{3.564404in}{0.781763in}}%
\pgfpathlineto{\pgfqpoint{3.564700in}{0.781761in}}%
\pgfpathlineto{\pgfqpoint{3.564996in}{0.781759in}}%
\pgfpathlineto{\pgfqpoint{3.565292in}{0.781757in}}%
\pgfpathlineto{\pgfqpoint{3.565588in}{0.781755in}}%
\pgfpathlineto{\pgfqpoint{3.565884in}{0.781753in}}%
\pgfpathlineto{\pgfqpoint{3.566180in}{0.781751in}}%
\pgfpathlineto{\pgfqpoint{3.566476in}{0.781749in}}%
\pgfpathlineto{\pgfqpoint{3.566772in}{0.781747in}}%
\pgfpathlineto{\pgfqpoint{3.567068in}{0.781745in}}%
\pgfpathlineto{\pgfqpoint{3.567364in}{0.781743in}}%
\pgfpathlineto{\pgfqpoint{3.567660in}{0.781741in}}%
\pgfpathlineto{\pgfqpoint{3.567956in}{0.781739in}}%
\pgfpathlineto{\pgfqpoint{3.568252in}{0.781737in}}%
\pgfpathlineto{\pgfqpoint{3.568548in}{0.781735in}}%
\pgfpathlineto{\pgfqpoint{3.568844in}{0.781733in}}%
\pgfpathlineto{\pgfqpoint{3.569140in}{0.781731in}}%
\pgfpathlineto{\pgfqpoint{3.569436in}{0.781730in}}%
\pgfpathlineto{\pgfqpoint{3.569732in}{0.781728in}}%
\pgfpathlineto{\pgfqpoint{3.570028in}{0.781726in}}%
\pgfpathlineto{\pgfqpoint{3.570324in}{0.781724in}}%
\pgfpathlineto{\pgfqpoint{3.570620in}{0.781722in}}%
\pgfpathlineto{\pgfqpoint{3.570916in}{0.781720in}}%
\pgfpathlineto{\pgfqpoint{3.571212in}{0.781718in}}%
\pgfpathlineto{\pgfqpoint{3.571508in}{0.781716in}}%
\pgfpathlineto{\pgfqpoint{3.571804in}{0.781714in}}%
\pgfpathlineto{\pgfqpoint{3.572100in}{0.781712in}}%
\pgfpathlineto{\pgfqpoint{3.572396in}{0.781710in}}%
\pgfpathlineto{\pgfqpoint{3.572692in}{0.781708in}}%
\pgfpathlineto{\pgfqpoint{3.572988in}{0.781706in}}%
\pgfpathlineto{\pgfqpoint{3.573284in}{0.781704in}}%
\pgfpathlineto{\pgfqpoint{3.573580in}{0.781702in}}%
\pgfpathlineto{\pgfqpoint{3.573876in}{0.781700in}}%
\pgfpathlineto{\pgfqpoint{3.574172in}{0.781698in}}%
\pgfpathlineto{\pgfqpoint{3.574468in}{0.781696in}}%
\pgfpathlineto{\pgfqpoint{3.574764in}{0.781694in}}%
\pgfpathlineto{\pgfqpoint{3.575060in}{0.781693in}}%
\pgfpathlineto{\pgfqpoint{3.575356in}{0.781691in}}%
\pgfpathlineto{\pgfqpoint{3.575652in}{0.781689in}}%
\pgfpathlineto{\pgfqpoint{3.575948in}{0.781687in}}%
\pgfpathlineto{\pgfqpoint{3.576244in}{0.781685in}}%
\pgfpathlineto{\pgfqpoint{3.576540in}{0.781683in}}%
\pgfpathlineto{\pgfqpoint{3.576836in}{0.781681in}}%
\pgfpathlineto{\pgfqpoint{3.577132in}{0.781679in}}%
\pgfpathlineto{\pgfqpoint{3.577428in}{0.781677in}}%
\pgfpathlineto{\pgfqpoint{3.577724in}{0.781675in}}%
\pgfpathlineto{\pgfqpoint{3.578020in}{0.781673in}}%
\pgfpathlineto{\pgfqpoint{3.578316in}{0.781671in}}%
\pgfpathlineto{\pgfqpoint{3.578612in}{0.781669in}}%
\pgfpathlineto{\pgfqpoint{3.578908in}{0.781667in}}%
\pgfpathlineto{\pgfqpoint{3.579204in}{0.781665in}}%
\pgfpathlineto{\pgfqpoint{3.579500in}{0.781663in}}%
\pgfpathlineto{\pgfqpoint{3.579796in}{0.781661in}}%
\pgfpathlineto{\pgfqpoint{3.580092in}{0.781659in}}%
\pgfpathlineto{\pgfqpoint{3.580388in}{0.781657in}}%
\pgfpathlineto{\pgfqpoint{3.580684in}{0.781655in}}%
\pgfpathlineto{\pgfqpoint{3.580980in}{0.781654in}}%
\pgfpathlineto{\pgfqpoint{3.581276in}{0.781652in}}%
\pgfpathlineto{\pgfqpoint{3.581572in}{0.781650in}}%
\pgfpathlineto{\pgfqpoint{3.581868in}{0.781648in}}%
\pgfpathlineto{\pgfqpoint{3.582164in}{0.781646in}}%
\pgfpathlineto{\pgfqpoint{3.582460in}{0.781644in}}%
\pgfpathlineto{\pgfqpoint{3.582756in}{0.781642in}}%
\pgfpathlineto{\pgfqpoint{3.583052in}{0.781640in}}%
\pgfpathlineto{\pgfqpoint{3.583348in}{0.781638in}}%
\pgfpathlineto{\pgfqpoint{3.583644in}{0.781636in}}%
\pgfpathlineto{\pgfqpoint{3.583940in}{0.781634in}}%
\pgfpathlineto{\pgfqpoint{3.584236in}{0.781632in}}%
\pgfpathlineto{\pgfqpoint{3.584532in}{0.781630in}}%
\pgfpathlineto{\pgfqpoint{3.584828in}{0.781628in}}%
\pgfpathlineto{\pgfqpoint{3.585124in}{0.781626in}}%
\pgfpathlineto{\pgfqpoint{3.585420in}{0.781624in}}%
\pgfpathlineto{\pgfqpoint{3.585716in}{0.781622in}}%
\pgfpathlineto{\pgfqpoint{3.586012in}{0.781620in}}%
\pgfpathlineto{\pgfqpoint{3.586308in}{0.781618in}}%
\pgfpathlineto{\pgfqpoint{3.586604in}{0.781617in}}%
\pgfpathlineto{\pgfqpoint{3.586900in}{0.781615in}}%
\pgfpathlineto{\pgfqpoint{3.587196in}{0.781613in}}%
\pgfpathlineto{\pgfqpoint{3.587492in}{0.781611in}}%
\pgfpathlineto{\pgfqpoint{3.587788in}{0.781609in}}%
\pgfpathlineto{\pgfqpoint{3.588084in}{0.781607in}}%
\pgfpathlineto{\pgfqpoint{3.588380in}{0.781605in}}%
\pgfpathlineto{\pgfqpoint{3.588676in}{0.781603in}}%
\pgfpathlineto{\pgfqpoint{3.588972in}{0.781601in}}%
\pgfpathlineto{\pgfqpoint{3.589268in}{0.781599in}}%
\pgfpathlineto{\pgfqpoint{3.589564in}{0.781597in}}%
\pgfpathlineto{\pgfqpoint{3.589860in}{0.781595in}}%
\pgfpathlineto{\pgfqpoint{3.590156in}{0.781593in}}%
\pgfpathlineto{\pgfqpoint{3.590452in}{0.781591in}}%
\pgfpathlineto{\pgfqpoint{3.590748in}{0.781589in}}%
\pgfpathlineto{\pgfqpoint{3.591044in}{0.781587in}}%
\pgfpathlineto{\pgfqpoint{3.591340in}{0.781585in}}%
\pgfpathlineto{\pgfqpoint{3.591636in}{0.781583in}}%
\pgfpathlineto{\pgfqpoint{3.591932in}{0.781581in}}%
\pgfpathlineto{\pgfqpoint{3.592228in}{0.781579in}}%
\pgfpathlineto{\pgfqpoint{3.592524in}{0.781578in}}%
\pgfpathlineto{\pgfqpoint{3.592820in}{0.781576in}}%
\pgfpathlineto{\pgfqpoint{3.593116in}{0.781574in}}%
\pgfpathlineto{\pgfqpoint{3.593412in}{0.781572in}}%
\pgfpathlineto{\pgfqpoint{3.593708in}{0.781570in}}%
\pgfpathlineto{\pgfqpoint{3.594004in}{0.781568in}}%
\pgfpathlineto{\pgfqpoint{3.594300in}{0.781566in}}%
\pgfpathlineto{\pgfqpoint{3.594596in}{0.781564in}}%
\pgfpathlineto{\pgfqpoint{3.594892in}{0.781562in}}%
\pgfpathlineto{\pgfqpoint{3.595188in}{0.781560in}}%
\pgfpathlineto{\pgfqpoint{3.595484in}{0.781558in}}%
\pgfpathlineto{\pgfqpoint{3.595780in}{0.781556in}}%
\pgfpathlineto{\pgfqpoint{3.596076in}{0.781554in}}%
\pgfpathlineto{\pgfqpoint{3.596372in}{0.781552in}}%
\pgfpathlineto{\pgfqpoint{3.596668in}{0.781550in}}%
\pgfpathlineto{\pgfqpoint{3.596964in}{0.781548in}}%
\pgfpathlineto{\pgfqpoint{3.597260in}{0.781546in}}%
\pgfpathlineto{\pgfqpoint{3.597556in}{0.781544in}}%
\pgfpathlineto{\pgfqpoint{3.597852in}{0.781542in}}%
\pgfpathlineto{\pgfqpoint{3.598148in}{0.781541in}}%
\pgfpathlineto{\pgfqpoint{3.598444in}{0.781539in}}%
\pgfpathlineto{\pgfqpoint{3.598740in}{0.781537in}}%
\pgfpathlineto{\pgfqpoint{3.599036in}{0.781535in}}%
\pgfpathlineto{\pgfqpoint{3.599332in}{0.781533in}}%
\pgfpathlineto{\pgfqpoint{3.599628in}{0.781531in}}%
\pgfpathlineto{\pgfqpoint{3.599924in}{0.781529in}}%
\pgfpathlineto{\pgfqpoint{3.600220in}{0.781527in}}%
\pgfpathlineto{\pgfqpoint{3.600516in}{0.781525in}}%
\pgfpathlineto{\pgfqpoint{3.600812in}{0.781523in}}%
\pgfpathlineto{\pgfqpoint{3.601108in}{0.781521in}}%
\pgfpathlineto{\pgfqpoint{3.601404in}{0.781519in}}%
\pgfpathlineto{\pgfqpoint{3.601700in}{0.781517in}}%
\pgfpathlineto{\pgfqpoint{3.601996in}{0.781515in}}%
\pgfpathlineto{\pgfqpoint{3.602292in}{0.781511in}}%
\pgfpathlineto{\pgfqpoint{3.602588in}{0.781504in}}%
\pgfpathlineto{\pgfqpoint{3.602884in}{0.781496in}}%
\pgfpathlineto{\pgfqpoint{3.603180in}{0.781489in}}%
\pgfpathlineto{\pgfqpoint{3.603476in}{0.781481in}}%
\pgfpathlineto{\pgfqpoint{3.603772in}{0.781473in}}%
\pgfpathlineto{\pgfqpoint{3.604068in}{0.781466in}}%
\pgfpathlineto{\pgfqpoint{3.604364in}{0.781458in}}%
\pgfpathlineto{\pgfqpoint{3.604660in}{0.781451in}}%
\pgfpathlineto{\pgfqpoint{3.604956in}{0.781443in}}%
\pgfpathlineto{\pgfqpoint{3.605252in}{0.781435in}}%
\pgfpathlineto{\pgfqpoint{3.605548in}{0.781428in}}%
\pgfpathlineto{\pgfqpoint{3.605844in}{0.781420in}}%
\pgfpathlineto{\pgfqpoint{3.606140in}{0.781412in}}%
\pgfpathlineto{\pgfqpoint{3.606436in}{0.781405in}}%
\pgfpathlineto{\pgfqpoint{3.606732in}{0.781397in}}%
\pgfpathlineto{\pgfqpoint{3.607028in}{0.781390in}}%
\pgfpathlineto{\pgfqpoint{3.607324in}{0.781382in}}%
\pgfpathlineto{\pgfqpoint{3.607620in}{0.781374in}}%
\pgfpathlineto{\pgfqpoint{3.607916in}{0.781367in}}%
\pgfpathlineto{\pgfqpoint{3.608212in}{0.789915in}}%
\pgfpathlineto{\pgfqpoint{3.608508in}{0.799393in}}%
\pgfpathlineto{\pgfqpoint{3.608804in}{0.793911in}}%
\pgfpathlineto{\pgfqpoint{3.609100in}{0.788428in}}%
\pgfpathlineto{\pgfqpoint{3.609396in}{0.783061in}}%
\pgfpathlineto{\pgfqpoint{3.609692in}{0.781020in}}%
\pgfpathlineto{\pgfqpoint{3.609988in}{0.781251in}}%
\pgfpathlineto{\pgfqpoint{3.610284in}{0.781158in}}%
\pgfpathlineto{\pgfqpoint{3.610580in}{0.781054in}}%
\pgfpathlineto{\pgfqpoint{3.610876in}{0.780950in}}%
\pgfpathlineto{\pgfqpoint{3.611172in}{0.780885in}}%
\pgfpathlineto{\pgfqpoint{3.611468in}{0.780853in}}%
\pgfpathlineto{\pgfqpoint{3.611764in}{0.780822in}}%
\pgfpathlineto{\pgfqpoint{3.612060in}{0.780791in}}%
\pgfpathlineto{\pgfqpoint{3.612356in}{0.780760in}}%
\pgfpathlineto{\pgfqpoint{3.612652in}{0.780728in}}%
\pgfpathlineto{\pgfqpoint{3.612948in}{0.780697in}}%
\pgfpathlineto{\pgfqpoint{3.613244in}{0.780666in}}%
\pgfpathlineto{\pgfqpoint{3.613540in}{0.780635in}}%
\pgfpathlineto{\pgfqpoint{3.613836in}{0.780603in}}%
\pgfpathlineto{\pgfqpoint{3.614132in}{0.780572in}}%
\pgfpathlineto{\pgfqpoint{3.614429in}{0.780541in}}%
\pgfpathlineto{\pgfqpoint{3.614725in}{0.780510in}}%
\pgfpathlineto{\pgfqpoint{3.615021in}{0.780445in}}%
\pgfpathlineto{\pgfqpoint{3.615317in}{0.780405in}}%
\pgfpathlineto{\pgfqpoint{3.615613in}{0.780351in}}%
\pgfpathlineto{\pgfqpoint{3.615909in}{0.780117in}}%
\pgfpathlineto{\pgfqpoint{3.616205in}{0.779826in}}%
\pgfpathlineto{\pgfqpoint{3.616501in}{0.779824in}}%
\pgfpathlineto{\pgfqpoint{3.616797in}{0.779823in}}%
\pgfpathlineto{\pgfqpoint{3.617093in}{0.779821in}}%
\pgfpathlineto{\pgfqpoint{3.617389in}{0.779820in}}%
\pgfpathlineto{\pgfqpoint{3.617685in}{0.779818in}}%
\pgfpathlineto{\pgfqpoint{3.617981in}{0.779817in}}%
\pgfpathlineto{\pgfqpoint{3.618277in}{0.779815in}}%
\pgfpathlineto{\pgfqpoint{3.618573in}{0.779814in}}%
\pgfpathlineto{\pgfqpoint{3.618869in}{0.779812in}}%
\pgfpathlineto{\pgfqpoint{3.619165in}{0.779811in}}%
\pgfpathlineto{\pgfqpoint{3.619461in}{0.779809in}}%
\pgfpathlineto{\pgfqpoint{3.619757in}{0.779808in}}%
\pgfpathlineto{\pgfqpoint{3.620053in}{0.779806in}}%
\pgfpathlineto{\pgfqpoint{3.620349in}{0.779805in}}%
\pgfpathlineto{\pgfqpoint{3.620645in}{0.779803in}}%
\pgfpathlineto{\pgfqpoint{3.620941in}{0.779802in}}%
\pgfpathlineto{\pgfqpoint{3.621237in}{0.779800in}}%
\pgfpathlineto{\pgfqpoint{3.621533in}{0.779799in}}%
\pgfpathlineto{\pgfqpoint{3.621829in}{0.779797in}}%
\pgfpathlineto{\pgfqpoint{3.622125in}{0.779796in}}%
\pgfpathlineto{\pgfqpoint{3.622421in}{0.779794in}}%
\pgfpathlineto{\pgfqpoint{3.622717in}{0.779793in}}%
\pgfpathlineto{\pgfqpoint{3.623013in}{0.779791in}}%
\pgfpathlineto{\pgfqpoint{3.623309in}{0.779790in}}%
\pgfpathlineto{\pgfqpoint{3.623605in}{0.779788in}}%
\pgfpathlineto{\pgfqpoint{3.623901in}{0.779787in}}%
\pgfpathlineto{\pgfqpoint{3.624197in}{0.779785in}}%
\pgfpathlineto{\pgfqpoint{3.624493in}{0.779784in}}%
\pgfpathlineto{\pgfqpoint{3.624789in}{0.779782in}}%
\pgfpathlineto{\pgfqpoint{3.625085in}{0.779781in}}%
\pgfpathlineto{\pgfqpoint{3.625381in}{0.779779in}}%
\pgfpathlineto{\pgfqpoint{3.625677in}{0.779778in}}%
\pgfpathlineto{\pgfqpoint{3.625973in}{0.779776in}}%
\pgfpathlineto{\pgfqpoint{3.626269in}{0.779775in}}%
\pgfpathlineto{\pgfqpoint{3.626565in}{0.779773in}}%
\pgfpathlineto{\pgfqpoint{3.626861in}{0.779772in}}%
\pgfpathlineto{\pgfqpoint{3.627157in}{0.779770in}}%
\pgfpathlineto{\pgfqpoint{3.627453in}{0.779769in}}%
\pgfpathlineto{\pgfqpoint{3.627749in}{0.779767in}}%
\pgfpathlineto{\pgfqpoint{3.628045in}{0.779766in}}%
\pgfpathlineto{\pgfqpoint{3.628341in}{0.779764in}}%
\pgfpathlineto{\pgfqpoint{3.628637in}{0.779763in}}%
\pgfpathlineto{\pgfqpoint{3.628933in}{0.779761in}}%
\pgfpathlineto{\pgfqpoint{3.629229in}{0.779760in}}%
\pgfpathlineto{\pgfqpoint{3.629525in}{0.779758in}}%
\pgfpathlineto{\pgfqpoint{3.629821in}{0.779757in}}%
\pgfpathlineto{\pgfqpoint{3.630117in}{0.779755in}}%
\pgfpathlineto{\pgfqpoint{3.630413in}{0.779754in}}%
\pgfpathlineto{\pgfqpoint{3.630709in}{0.779752in}}%
\pgfpathlineto{\pgfqpoint{3.631005in}{0.779751in}}%
\pgfpathlineto{\pgfqpoint{3.631301in}{0.779749in}}%
\pgfpathlineto{\pgfqpoint{3.631597in}{0.779748in}}%
\pgfpathlineto{\pgfqpoint{3.631893in}{0.779746in}}%
\pgfpathlineto{\pgfqpoint{3.632189in}{0.779744in}}%
\pgfpathlineto{\pgfqpoint{3.632485in}{0.779743in}}%
\pgfpathlineto{\pgfqpoint{3.632781in}{0.779741in}}%
\pgfpathlineto{\pgfqpoint{3.633077in}{0.779740in}}%
\pgfpathlineto{\pgfqpoint{3.633373in}{0.779738in}}%
\pgfpathlineto{\pgfqpoint{3.633669in}{0.779737in}}%
\pgfpathlineto{\pgfqpoint{3.633965in}{0.779735in}}%
\pgfpathlineto{\pgfqpoint{3.634261in}{0.779734in}}%
\pgfpathlineto{\pgfqpoint{3.634557in}{0.779732in}}%
\pgfpathlineto{\pgfqpoint{3.634853in}{0.779731in}}%
\pgfpathlineto{\pgfqpoint{3.635149in}{0.779729in}}%
\pgfpathlineto{\pgfqpoint{3.635445in}{0.779728in}}%
\pgfpathlineto{\pgfqpoint{3.635741in}{0.779726in}}%
\pgfpathlineto{\pgfqpoint{3.636037in}{0.779725in}}%
\pgfpathlineto{\pgfqpoint{3.636333in}{0.779723in}}%
\pgfpathlineto{\pgfqpoint{3.636629in}{0.779722in}}%
\pgfpathlineto{\pgfqpoint{3.636925in}{0.779720in}}%
\pgfpathlineto{\pgfqpoint{3.637221in}{0.779719in}}%
\pgfpathlineto{\pgfqpoint{3.637517in}{0.779717in}}%
\pgfpathlineto{\pgfqpoint{3.637813in}{0.779716in}}%
\pgfpathlineto{\pgfqpoint{3.638109in}{0.779714in}}%
\pgfpathlineto{\pgfqpoint{3.638405in}{0.779713in}}%
\pgfpathlineto{\pgfqpoint{3.638701in}{0.779711in}}%
\pgfpathlineto{\pgfqpoint{3.638997in}{0.779710in}}%
\pgfpathlineto{\pgfqpoint{3.639293in}{0.779708in}}%
\pgfpathlineto{\pgfqpoint{3.639589in}{0.779707in}}%
\pgfpathlineto{\pgfqpoint{3.639885in}{0.779705in}}%
\pgfpathlineto{\pgfqpoint{3.640181in}{0.779704in}}%
\pgfpathlineto{\pgfqpoint{3.640477in}{0.779702in}}%
\pgfpathlineto{\pgfqpoint{3.640773in}{0.779701in}}%
\pgfpathlineto{\pgfqpoint{3.641069in}{0.779699in}}%
\pgfpathlineto{\pgfqpoint{3.641365in}{0.779698in}}%
\pgfpathlineto{\pgfqpoint{3.641661in}{0.779696in}}%
\pgfpathlineto{\pgfqpoint{3.641957in}{0.779695in}}%
\pgfpathlineto{\pgfqpoint{3.642253in}{0.779693in}}%
\pgfpathlineto{\pgfqpoint{3.642549in}{0.779692in}}%
\pgfpathlineto{\pgfqpoint{3.642845in}{0.779690in}}%
\pgfpathlineto{\pgfqpoint{3.643141in}{0.779689in}}%
\pgfpathlineto{\pgfqpoint{3.643437in}{0.779687in}}%
\pgfpathlineto{\pgfqpoint{3.643733in}{0.779686in}}%
\pgfpathlineto{\pgfqpoint{3.644029in}{0.779684in}}%
\pgfpathlineto{\pgfqpoint{3.644325in}{0.779683in}}%
\pgfpathlineto{\pgfqpoint{3.644621in}{0.779681in}}%
\pgfpathlineto{\pgfqpoint{3.644917in}{0.779680in}}%
\pgfpathlineto{\pgfqpoint{3.645213in}{0.779678in}}%
\pgfpathlineto{\pgfqpoint{3.645509in}{0.779677in}}%
\pgfpathlineto{\pgfqpoint{3.645805in}{0.779675in}}%
\pgfpathlineto{\pgfqpoint{3.646101in}{0.779674in}}%
\pgfpathlineto{\pgfqpoint{3.646397in}{0.779672in}}%
\pgfpathlineto{\pgfqpoint{3.646693in}{0.779671in}}%
\pgfpathlineto{\pgfqpoint{3.646989in}{0.779669in}}%
\pgfpathlineto{\pgfqpoint{3.647285in}{0.779668in}}%
\pgfpathlineto{\pgfqpoint{3.647581in}{0.779666in}}%
\pgfpathlineto{\pgfqpoint{3.647877in}{0.779665in}}%
\pgfpathlineto{\pgfqpoint{3.648173in}{0.779663in}}%
\pgfpathlineto{\pgfqpoint{3.648469in}{0.779662in}}%
\pgfpathlineto{\pgfqpoint{3.648765in}{0.779660in}}%
\pgfpathlineto{\pgfqpoint{3.649061in}{0.779659in}}%
\pgfpathlineto{\pgfqpoint{3.649357in}{0.779657in}}%
\pgfpathlineto{\pgfqpoint{3.649653in}{0.779656in}}%
\pgfpathlineto{\pgfqpoint{3.649949in}{0.779654in}}%
\pgfpathlineto{\pgfqpoint{3.650245in}{0.779653in}}%
\pgfpathlineto{\pgfqpoint{3.650541in}{0.779651in}}%
\pgfpathlineto{\pgfqpoint{3.650837in}{0.779622in}}%
\pgfpathlineto{\pgfqpoint{3.651133in}{0.779563in}}%
\pgfpathlineto{\pgfqpoint{3.651429in}{0.779530in}}%
\pgfpathlineto{\pgfqpoint{3.651725in}{0.779530in}}%
\pgfpathlineto{\pgfqpoint{3.652021in}{0.779529in}}%
\pgfpathlineto{\pgfqpoint{3.652317in}{0.779529in}}%
\pgfpathlineto{\pgfqpoint{3.652613in}{0.779528in}}%
\pgfpathlineto{\pgfqpoint{3.652909in}{0.779527in}}%
\pgfpathlineto{\pgfqpoint{3.653205in}{0.779517in}}%
\pgfpathlineto{\pgfqpoint{3.653501in}{0.779444in}}%
\pgfpathlineto{\pgfqpoint{3.653797in}{0.779359in}}%
\pgfpathlineto{\pgfqpoint{3.654093in}{0.779274in}}%
\pgfpathlineto{\pgfqpoint{3.654389in}{0.779214in}}%
\pgfpathlineto{\pgfqpoint{3.654685in}{0.779204in}}%
\pgfpathlineto{\pgfqpoint{3.654981in}{0.779196in}}%
\pgfpathlineto{\pgfqpoint{3.655277in}{0.779188in}}%
\pgfpathlineto{\pgfqpoint{3.655573in}{0.779180in}}%
\pgfpathlineto{\pgfqpoint{3.655869in}{0.779173in}}%
\pgfpathlineto{\pgfqpoint{3.656165in}{0.779165in}}%
\pgfpathlineto{\pgfqpoint{3.656461in}{0.779157in}}%
\pgfpathlineto{\pgfqpoint{3.656757in}{0.779149in}}%
\pgfpathlineto{\pgfqpoint{3.657053in}{0.779141in}}%
\pgfpathlineto{\pgfqpoint{3.657349in}{0.779133in}}%
\pgfpathlineto{\pgfqpoint{3.657645in}{0.779125in}}%
\pgfpathlineto{\pgfqpoint{3.657941in}{0.779123in}}%
\pgfpathlineto{\pgfqpoint{3.658237in}{0.779136in}}%
\pgfpathlineto{\pgfqpoint{3.658533in}{0.779149in}}%
\pgfpathlineto{\pgfqpoint{3.658829in}{0.779162in}}%
\pgfpathlineto{\pgfqpoint{3.659125in}{0.779174in}}%
\pgfpathlineto{\pgfqpoint{3.659421in}{0.779188in}}%
\pgfpathlineto{\pgfqpoint{3.659717in}{0.779201in}}%
\pgfpathlineto{\pgfqpoint{3.660013in}{0.779192in}}%
\pgfpathlineto{\pgfqpoint{3.660309in}{0.779171in}}%
\pgfpathlineto{\pgfqpoint{3.660605in}{0.779110in}}%
\pgfpathlineto{\pgfqpoint{3.660901in}{0.778795in}}%
\pgfpathlineto{\pgfqpoint{3.661197in}{0.778421in}}%
\pgfpathlineto{\pgfqpoint{3.661493in}{0.778237in}}%
\pgfpathlineto{\pgfqpoint{3.661789in}{0.778212in}}%
\pgfpathlineto{\pgfqpoint{3.662085in}{0.778192in}}%
\pgfpathlineto{\pgfqpoint{3.662381in}{0.778172in}}%
\pgfpathlineto{\pgfqpoint{3.662677in}{0.778152in}}%
\pgfpathlineto{\pgfqpoint{3.662973in}{0.778132in}}%
\pgfpathlineto{\pgfqpoint{3.663269in}{0.778112in}}%
\pgfpathlineto{\pgfqpoint{3.663565in}{0.778091in}}%
\pgfpathlineto{\pgfqpoint{3.663861in}{0.778071in}}%
\pgfpathlineto{\pgfqpoint{3.664157in}{0.778051in}}%
\pgfpathlineto{\pgfqpoint{3.664453in}{0.778031in}}%
\pgfpathlineto{\pgfqpoint{3.664749in}{0.778011in}}%
\pgfpathlineto{\pgfqpoint{3.665045in}{0.777991in}}%
\pgfpathlineto{\pgfqpoint{3.665341in}{0.777971in}}%
\pgfpathlineto{\pgfqpoint{3.665637in}{0.777951in}}%
\pgfpathlineto{\pgfqpoint{3.665933in}{0.777931in}}%
\pgfpathlineto{\pgfqpoint{3.666229in}{0.777910in}}%
\pgfpathlineto{\pgfqpoint{3.666525in}{0.777890in}}%
\pgfpathlineto{\pgfqpoint{3.666821in}{0.777870in}}%
\pgfpathlineto{\pgfqpoint{3.667117in}{0.777850in}}%
\pgfpathlineto{\pgfqpoint{3.667413in}{0.777830in}}%
\pgfpathlineto{\pgfqpoint{3.667709in}{0.777810in}}%
\pgfpathlineto{\pgfqpoint{3.668005in}{0.777790in}}%
\pgfpathlineto{\pgfqpoint{3.668301in}{0.777770in}}%
\pgfpathlineto{\pgfqpoint{3.668597in}{0.777749in}}%
\pgfpathlineto{\pgfqpoint{3.668893in}{0.777729in}}%
\pgfpathlineto{\pgfqpoint{3.669189in}{0.777709in}}%
\pgfpathlineto{\pgfqpoint{3.669485in}{0.777689in}}%
\pgfpathlineto{\pgfqpoint{3.669781in}{0.777669in}}%
\pgfpathlineto{\pgfqpoint{3.670077in}{0.777649in}}%
\pgfpathlineto{\pgfqpoint{3.670373in}{0.777629in}}%
\pgfpathlineto{\pgfqpoint{3.670669in}{0.777609in}}%
\pgfpathlineto{\pgfqpoint{3.670965in}{0.777588in}}%
\pgfpathlineto{\pgfqpoint{3.671261in}{0.777568in}}%
\pgfpathlineto{\pgfqpoint{3.671557in}{0.777548in}}%
\pgfpathlineto{\pgfqpoint{3.671853in}{0.777528in}}%
\pgfpathlineto{\pgfqpoint{3.672149in}{0.777508in}}%
\pgfpathlineto{\pgfqpoint{3.672445in}{0.777488in}}%
\pgfpathlineto{\pgfqpoint{3.672741in}{0.777263in}}%
\pgfpathlineto{\pgfqpoint{3.673037in}{0.777017in}}%
\pgfpathlineto{\pgfqpoint{3.673333in}{0.776810in}}%
\pgfpathlineto{\pgfqpoint{3.673629in}{0.776355in}}%
\pgfpathlineto{\pgfqpoint{3.673925in}{0.776268in}}%
\pgfpathlineto{\pgfqpoint{3.674221in}{0.776179in}}%
\pgfpathlineto{\pgfqpoint{3.674517in}{0.776078in}}%
\pgfpathlineto{\pgfqpoint{3.674813in}{0.776004in}}%
\pgfpathlineto{\pgfqpoint{3.675109in}{0.775930in}}%
\pgfpathlineto{\pgfqpoint{3.675405in}{0.775890in}}%
\pgfpathlineto{\pgfqpoint{3.675701in}{0.775886in}}%
\pgfpathlineto{\pgfqpoint{3.675997in}{0.775881in}}%
\pgfpathlineto{\pgfqpoint{3.676293in}{0.775877in}}%
\pgfpathlineto{\pgfqpoint{3.676589in}{0.775873in}}%
\pgfpathlineto{\pgfqpoint{3.676885in}{0.775868in}}%
\pgfpathlineto{\pgfqpoint{3.677181in}{0.775864in}}%
\pgfpathlineto{\pgfqpoint{3.677477in}{0.775860in}}%
\pgfpathlineto{\pgfqpoint{3.677773in}{0.775855in}}%
\pgfpathlineto{\pgfqpoint{3.678069in}{0.775851in}}%
\pgfpathlineto{\pgfqpoint{3.678365in}{0.775847in}}%
\pgfpathlineto{\pgfqpoint{3.678661in}{0.775842in}}%
\pgfpathlineto{\pgfqpoint{3.678957in}{0.775817in}}%
\pgfpathlineto{\pgfqpoint{3.679253in}{0.776073in}}%
\pgfpathlineto{\pgfqpoint{3.679549in}{0.776186in}}%
\pgfpathlineto{\pgfqpoint{3.679845in}{0.775731in}}%
\pgfpathlineto{\pgfqpoint{3.680141in}{0.775750in}}%
\pgfpathlineto{\pgfqpoint{3.680437in}{0.775770in}}%
\pgfpathlineto{\pgfqpoint{3.680733in}{0.775792in}}%
\pgfpathlineto{\pgfqpoint{3.681029in}{0.776259in}}%
\pgfpathlineto{\pgfqpoint{3.681325in}{0.776719in}}%
\pgfpathlineto{\pgfqpoint{3.681621in}{0.776745in}}%
\pgfpathlineto{\pgfqpoint{3.681918in}{0.776746in}}%
\pgfpathlineto{\pgfqpoint{3.682214in}{0.776748in}}%
\pgfpathlineto{\pgfqpoint{3.682510in}{0.776750in}}%
\pgfpathlineto{\pgfqpoint{3.682806in}{0.776752in}}%
\pgfpathlineto{\pgfqpoint{3.683102in}{0.776754in}}%
\pgfpathlineto{\pgfqpoint{3.683398in}{0.776756in}}%
\pgfpathlineto{\pgfqpoint{3.683694in}{0.776758in}}%
\pgfpathlineto{\pgfqpoint{3.683990in}{0.776760in}}%
\pgfpathlineto{\pgfqpoint{3.684286in}{0.776761in}}%
\pgfpathlineto{\pgfqpoint{3.684582in}{0.776763in}}%
\pgfpathlineto{\pgfqpoint{3.684878in}{0.776765in}}%
\pgfpathlineto{\pgfqpoint{3.685174in}{0.776767in}}%
\pgfpathlineto{\pgfqpoint{3.685470in}{0.776769in}}%
\pgfpathlineto{\pgfqpoint{3.685766in}{0.776771in}}%
\pgfpathlineto{\pgfqpoint{3.686062in}{0.776773in}}%
\pgfpathlineto{\pgfqpoint{3.686358in}{0.776775in}}%
\pgfpathlineto{\pgfqpoint{3.686654in}{0.776776in}}%
\pgfpathlineto{\pgfqpoint{3.686950in}{0.776778in}}%
\pgfpathlineto{\pgfqpoint{3.687246in}{0.776780in}}%
\pgfpathlineto{\pgfqpoint{3.687542in}{0.776782in}}%
\pgfpathlineto{\pgfqpoint{3.687838in}{0.776784in}}%
\pgfpathlineto{\pgfqpoint{3.688134in}{0.776786in}}%
\pgfpathlineto{\pgfqpoint{3.688430in}{0.776788in}}%
\pgfpathlineto{\pgfqpoint{3.688726in}{0.776790in}}%
\pgfpathlineto{\pgfqpoint{3.689022in}{0.776792in}}%
\pgfpathlineto{\pgfqpoint{3.689318in}{0.776793in}}%
\pgfpathlineto{\pgfqpoint{3.689614in}{0.776795in}}%
\pgfpathlineto{\pgfqpoint{3.689910in}{0.776797in}}%
\pgfpathlineto{\pgfqpoint{3.690206in}{0.776799in}}%
\pgfpathlineto{\pgfqpoint{3.690502in}{0.776801in}}%
\pgfpathlineto{\pgfqpoint{3.690798in}{0.776803in}}%
\pgfpathlineto{\pgfqpoint{3.691094in}{0.776805in}}%
\pgfpathlineto{\pgfqpoint{3.691390in}{0.776807in}}%
\pgfpathlineto{\pgfqpoint{3.691686in}{0.776808in}}%
\pgfpathlineto{\pgfqpoint{3.691982in}{0.776810in}}%
\pgfpathlineto{\pgfqpoint{3.692278in}{0.776812in}}%
\pgfpathlineto{\pgfqpoint{3.692574in}{0.776814in}}%
\pgfpathlineto{\pgfqpoint{3.692870in}{0.776816in}}%
\pgfpathlineto{\pgfqpoint{3.693166in}{0.776818in}}%
\pgfpathlineto{\pgfqpoint{3.693462in}{0.776820in}}%
\pgfpathlineto{\pgfqpoint{3.693758in}{0.776822in}}%
\pgfpathlineto{\pgfqpoint{3.694054in}{0.776823in}}%
\pgfpathlineto{\pgfqpoint{3.694350in}{0.776825in}}%
\pgfpathlineto{\pgfqpoint{3.694646in}{0.776827in}}%
\pgfpathlineto{\pgfqpoint{3.694942in}{0.776829in}}%
\pgfpathlineto{\pgfqpoint{3.695238in}{0.776831in}}%
\pgfpathlineto{\pgfqpoint{3.695534in}{0.776833in}}%
\pgfpathlineto{\pgfqpoint{3.695830in}{0.776835in}}%
\pgfpathlineto{\pgfqpoint{3.696126in}{0.776837in}}%
\pgfpathlineto{\pgfqpoint{3.696422in}{0.776838in}}%
\pgfpathlineto{\pgfqpoint{3.696718in}{0.776840in}}%
\pgfpathlineto{\pgfqpoint{3.697014in}{0.776842in}}%
\pgfpathlineto{\pgfqpoint{3.697310in}{0.776844in}}%
\pgfpathlineto{\pgfqpoint{3.697606in}{0.776846in}}%
\pgfpathlineto{\pgfqpoint{3.697902in}{0.776848in}}%
\pgfpathlineto{\pgfqpoint{3.698198in}{0.776850in}}%
\pgfpathlineto{\pgfqpoint{3.698494in}{0.776852in}}%
\pgfpathlineto{\pgfqpoint{3.698790in}{0.776854in}}%
\pgfpathlineto{\pgfqpoint{3.699086in}{0.776855in}}%
\pgfpathlineto{\pgfqpoint{3.699382in}{0.776857in}}%
\pgfpathlineto{\pgfqpoint{3.699678in}{0.776859in}}%
\pgfpathlineto{\pgfqpoint{3.699974in}{0.776861in}}%
\pgfpathlineto{\pgfqpoint{3.700270in}{0.776863in}}%
\pgfpathlineto{\pgfqpoint{3.700566in}{0.776865in}}%
\pgfpathlineto{\pgfqpoint{3.700862in}{0.776867in}}%
\pgfpathlineto{\pgfqpoint{3.701158in}{0.776869in}}%
\pgfpathlineto{\pgfqpoint{3.701454in}{0.776870in}}%
\pgfpathlineto{\pgfqpoint{3.701750in}{0.776868in}}%
\pgfpathlineto{\pgfqpoint{3.702046in}{0.776797in}}%
\pgfpathlineto{\pgfqpoint{3.702342in}{0.776702in}}%
\pgfpathlineto{\pgfqpoint{3.702638in}{0.776607in}}%
\pgfpathlineto{\pgfqpoint{3.702934in}{0.776511in}}%
\pgfpathlineto{\pgfqpoint{3.703230in}{0.776416in}}%
\pgfpathlineto{\pgfqpoint{3.703526in}{0.776321in}}%
\pgfpathlineto{\pgfqpoint{3.703822in}{0.776225in}}%
\pgfpathlineto{\pgfqpoint{3.704118in}{0.776130in}}%
\pgfpathlineto{\pgfqpoint{3.704414in}{0.776035in}}%
\pgfpathlineto{\pgfqpoint{3.704710in}{0.775940in}}%
\pgfpathlineto{\pgfqpoint{3.705006in}{0.775844in}}%
\pgfpathlineto{\pgfqpoint{3.705302in}{0.775749in}}%
\pgfpathlineto{\pgfqpoint{3.705598in}{0.775654in}}%
\pgfpathlineto{\pgfqpoint{3.705894in}{0.775558in}}%
\pgfpathlineto{\pgfqpoint{3.706190in}{0.775463in}}%
\pgfpathlineto{\pgfqpoint{3.706486in}{0.775368in}}%
\pgfpathlineto{\pgfqpoint{3.706782in}{0.775272in}}%
\pgfpathlineto{\pgfqpoint{3.707078in}{0.775177in}}%
\pgfpathlineto{\pgfqpoint{3.707374in}{0.775082in}}%
\pgfpathlineto{\pgfqpoint{3.707670in}{0.775111in}}%
\pgfpathlineto{\pgfqpoint{3.707966in}{0.775117in}}%
\pgfpathlineto{\pgfqpoint{3.708262in}{0.775110in}}%
\pgfpathlineto{\pgfqpoint{3.708558in}{0.775094in}}%
\pgfpathlineto{\pgfqpoint{3.708854in}{0.774941in}}%
\pgfpathlineto{\pgfqpoint{3.709150in}{0.774744in}}%
\pgfpathlineto{\pgfqpoint{3.709446in}{0.775047in}}%
\pgfpathlineto{\pgfqpoint{3.709742in}{0.775643in}}%
\pgfpathlineto{\pgfqpoint{3.710038in}{0.775742in}}%
\pgfpathlineto{\pgfqpoint{3.710334in}{0.775677in}}%
\pgfpathlineto{\pgfqpoint{3.710630in}{0.775612in}}%
\pgfpathlineto{\pgfqpoint{3.710926in}{0.775547in}}%
\pgfpathlineto{\pgfqpoint{3.711222in}{0.775482in}}%
\pgfpathlineto{\pgfqpoint{3.711518in}{0.775417in}}%
\pgfpathlineto{\pgfqpoint{3.711814in}{0.775352in}}%
\pgfpathlineto{\pgfqpoint{3.712110in}{0.775287in}}%
\pgfpathlineto{\pgfqpoint{3.712406in}{0.775222in}}%
\pgfpathlineto{\pgfqpoint{3.712702in}{0.775157in}}%
\pgfpathlineto{\pgfqpoint{3.712998in}{0.775092in}}%
\pgfpathlineto{\pgfqpoint{3.713294in}{0.775027in}}%
\pgfpathlineto{\pgfqpoint{3.713590in}{0.774962in}}%
\pgfpathlineto{\pgfqpoint{3.713886in}{0.774897in}}%
\pgfpathlineto{\pgfqpoint{3.714182in}{0.774832in}}%
\pgfpathlineto{\pgfqpoint{3.714478in}{0.774768in}}%
\pgfpathlineto{\pgfqpoint{3.714774in}{0.774720in}}%
\pgfpathlineto{\pgfqpoint{3.715070in}{0.774679in}}%
\pgfpathlineto{\pgfqpoint{3.715366in}{0.774639in}}%
\pgfpathlineto{\pgfqpoint{3.715662in}{0.774496in}}%
\pgfpathlineto{\pgfqpoint{3.715958in}{0.774245in}}%
\pgfpathlineto{\pgfqpoint{3.716254in}{0.773993in}}%
\pgfpathlineto{\pgfqpoint{3.716550in}{0.773809in}}%
\pgfpathlineto{\pgfqpoint{3.716846in}{0.773799in}}%
\pgfpathlineto{\pgfqpoint{3.717142in}{0.773796in}}%
\pgfpathlineto{\pgfqpoint{3.717438in}{0.773794in}}%
\pgfpathlineto{\pgfqpoint{3.717734in}{0.773792in}}%
\pgfpathlineto{\pgfqpoint{3.718030in}{0.773790in}}%
\pgfpathlineto{\pgfqpoint{3.718326in}{0.773787in}}%
\pgfpathlineto{\pgfqpoint{3.718622in}{0.773785in}}%
\pgfpathlineto{\pgfqpoint{3.718918in}{0.773783in}}%
\pgfpathlineto{\pgfqpoint{3.719214in}{0.773781in}}%
\pgfpathlineto{\pgfqpoint{3.719510in}{0.773779in}}%
\pgfpathlineto{\pgfqpoint{3.719806in}{0.773776in}}%
\pgfpathlineto{\pgfqpoint{3.720102in}{0.773774in}}%
\pgfpathlineto{\pgfqpoint{3.720398in}{0.773772in}}%
\pgfpathlineto{\pgfqpoint{3.720694in}{0.773770in}}%
\pgfpathlineto{\pgfqpoint{3.720990in}{0.773767in}}%
\pgfpathlineto{\pgfqpoint{3.721286in}{0.773765in}}%
\pgfpathlineto{\pgfqpoint{3.721582in}{0.773763in}}%
\pgfpathlineto{\pgfqpoint{3.721878in}{0.773779in}}%
\pgfpathlineto{\pgfqpoint{3.722174in}{0.773842in}}%
\pgfpathlineto{\pgfqpoint{3.722470in}{0.773839in}}%
\pgfpathlineto{\pgfqpoint{3.722766in}{0.773837in}}%
\pgfpathlineto{\pgfqpoint{3.723062in}{0.773835in}}%
\pgfpathlineto{\pgfqpoint{3.723358in}{0.773833in}}%
\pgfpathlineto{\pgfqpoint{3.723654in}{0.773830in}}%
\pgfpathlineto{\pgfqpoint{3.723950in}{0.773827in}}%
\pgfpathlineto{\pgfqpoint{3.724246in}{0.773824in}}%
\pgfpathlineto{\pgfqpoint{3.724542in}{0.773821in}}%
\pgfpathlineto{\pgfqpoint{3.724838in}{0.773818in}}%
\pgfpathlineto{\pgfqpoint{3.725134in}{0.773815in}}%
\pgfpathlineto{\pgfqpoint{3.725430in}{0.773812in}}%
\pgfpathlineto{\pgfqpoint{3.725726in}{0.773809in}}%
\pgfpathlineto{\pgfqpoint{3.726022in}{0.773806in}}%
\pgfpathlineto{\pgfqpoint{3.726318in}{0.773803in}}%
\pgfpathlineto{\pgfqpoint{3.726614in}{0.773800in}}%
\pgfpathlineto{\pgfqpoint{3.726910in}{0.773796in}}%
\pgfpathlineto{\pgfqpoint{3.727206in}{0.773793in}}%
\pgfpathlineto{\pgfqpoint{3.727502in}{0.773790in}}%
\pgfpathlineto{\pgfqpoint{3.727798in}{0.773787in}}%
\pgfpathlineto{\pgfqpoint{3.728094in}{0.773784in}}%
\pgfpathlineto{\pgfqpoint{3.728390in}{0.773781in}}%
\pgfpathlineto{\pgfqpoint{3.728686in}{0.773778in}}%
\pgfpathlineto{\pgfqpoint{3.728982in}{0.773775in}}%
\pgfpathlineto{\pgfqpoint{3.729278in}{0.773772in}}%
\pgfpathlineto{\pgfqpoint{3.729574in}{0.773769in}}%
\pgfpathlineto{\pgfqpoint{3.729870in}{0.773766in}}%
\pgfpathlineto{\pgfqpoint{3.730166in}{0.773762in}}%
\pgfpathlineto{\pgfqpoint{3.730462in}{0.773759in}}%
\pgfpathlineto{\pgfqpoint{3.730758in}{0.773756in}}%
\pgfpathlineto{\pgfqpoint{3.731054in}{0.773753in}}%
\pgfpathlineto{\pgfqpoint{3.731350in}{0.773750in}}%
\pgfpathlineto{\pgfqpoint{3.731646in}{0.773747in}}%
\pgfpathlineto{\pgfqpoint{3.731942in}{0.773744in}}%
\pgfpathlineto{\pgfqpoint{3.732238in}{0.773741in}}%
\pgfpathlineto{\pgfqpoint{3.732534in}{0.773738in}}%
\pgfpathlineto{\pgfqpoint{3.732830in}{0.773735in}}%
\pgfpathlineto{\pgfqpoint{3.733126in}{0.773732in}}%
\pgfpathlineto{\pgfqpoint{3.733422in}{0.773728in}}%
\pgfpathlineto{\pgfqpoint{3.733718in}{0.773725in}}%
\pgfpathlineto{\pgfqpoint{3.734014in}{0.773722in}}%
\pgfpathlineto{\pgfqpoint{3.734310in}{0.773719in}}%
\pgfpathlineto{\pgfqpoint{3.734606in}{0.773716in}}%
\pgfpathlineto{\pgfqpoint{3.734902in}{0.773713in}}%
\pgfpathlineto{\pgfqpoint{3.735198in}{0.773710in}}%
\pgfpathlineto{\pgfqpoint{3.735494in}{0.773707in}}%
\pgfpathlineto{\pgfqpoint{3.735790in}{0.773704in}}%
\pgfpathlineto{\pgfqpoint{3.736086in}{0.773701in}}%
\pgfpathlineto{\pgfqpoint{3.736382in}{0.773697in}}%
\pgfpathlineto{\pgfqpoint{3.736678in}{0.773694in}}%
\pgfpathlineto{\pgfqpoint{3.736974in}{0.773691in}}%
\pgfpathlineto{\pgfqpoint{3.737270in}{0.773688in}}%
\pgfpathlineto{\pgfqpoint{3.737566in}{0.773685in}}%
\pgfpathlineto{\pgfqpoint{3.737862in}{0.773682in}}%
\pgfpathlineto{\pgfqpoint{3.738158in}{0.773679in}}%
\pgfpathlineto{\pgfqpoint{3.738454in}{0.773676in}}%
\pgfpathlineto{\pgfqpoint{3.738750in}{0.773673in}}%
\pgfpathlineto{\pgfqpoint{3.739046in}{0.773670in}}%
\pgfpathlineto{\pgfqpoint{3.739342in}{0.773667in}}%
\pgfpathlineto{\pgfqpoint{3.739638in}{0.773663in}}%
\pgfpathlineto{\pgfqpoint{3.739934in}{0.773660in}}%
\pgfpathlineto{\pgfqpoint{3.740230in}{0.773657in}}%
\pgfpathlineto{\pgfqpoint{3.740526in}{0.773654in}}%
\pgfpathlineto{\pgfqpoint{3.740822in}{0.773651in}}%
\pgfpathlineto{\pgfqpoint{3.741118in}{0.773648in}}%
\pgfpathlineto{\pgfqpoint{3.741414in}{0.773645in}}%
\pgfpathlineto{\pgfqpoint{3.741710in}{0.773642in}}%
\pgfpathlineto{\pgfqpoint{3.742006in}{0.773639in}}%
\pgfpathlineto{\pgfqpoint{3.742302in}{0.773636in}}%
\pgfpathlineto{\pgfqpoint{3.742598in}{0.773632in}}%
\pgfpathlineto{\pgfqpoint{3.742894in}{0.773629in}}%
\pgfpathlineto{\pgfqpoint{3.743190in}{0.773626in}}%
\pgfpathlineto{\pgfqpoint{3.743486in}{0.773623in}}%
\pgfpathlineto{\pgfqpoint{3.743782in}{0.773620in}}%
\pgfpathlineto{\pgfqpoint{3.744078in}{0.773617in}}%
\pgfpathlineto{\pgfqpoint{3.744374in}{0.773614in}}%
\pgfpathlineto{\pgfqpoint{3.744670in}{0.773611in}}%
\pgfpathlineto{\pgfqpoint{3.744966in}{0.773608in}}%
\pgfpathlineto{\pgfqpoint{3.745262in}{0.773605in}}%
\pgfpathlineto{\pgfqpoint{3.745558in}{0.773602in}}%
\pgfpathlineto{\pgfqpoint{3.745854in}{0.773598in}}%
\pgfpathlineto{\pgfqpoint{3.746150in}{0.773595in}}%
\pgfpathlineto{\pgfqpoint{3.746446in}{0.773592in}}%
\pgfpathlineto{\pgfqpoint{3.746742in}{0.773589in}}%
\pgfpathlineto{\pgfqpoint{3.747038in}{0.773586in}}%
\pgfpathlineto{\pgfqpoint{3.747334in}{0.773583in}}%
\pgfpathlineto{\pgfqpoint{3.747630in}{0.773580in}}%
\pgfpathlineto{\pgfqpoint{3.747926in}{0.773577in}}%
\pgfpathlineto{\pgfqpoint{3.748222in}{0.773574in}}%
\pgfpathlineto{\pgfqpoint{3.748518in}{0.773571in}}%
\pgfpathlineto{\pgfqpoint{3.748814in}{0.773568in}}%
\pgfpathlineto{\pgfqpoint{3.749111in}{0.773564in}}%
\pgfpathlineto{\pgfqpoint{3.749407in}{0.773561in}}%
\pgfpathlineto{\pgfqpoint{3.749703in}{0.773558in}}%
\pgfpathlineto{\pgfqpoint{3.749999in}{0.773555in}}%
\pgfpathlineto{\pgfqpoint{3.750295in}{0.773552in}}%
\pgfpathlineto{\pgfqpoint{3.750591in}{0.773549in}}%
\pgfpathlineto{\pgfqpoint{3.750887in}{0.773547in}}%
\pgfpathlineto{\pgfqpoint{3.751183in}{0.773549in}}%
\pgfpathlineto{\pgfqpoint{3.751479in}{0.773550in}}%
\pgfpathlineto{\pgfqpoint{3.751775in}{0.773552in}}%
\pgfpathlineto{\pgfqpoint{3.752071in}{0.773553in}}%
\pgfpathlineto{\pgfqpoint{3.752367in}{0.773556in}}%
\pgfpathlineto{\pgfqpoint{3.752663in}{0.773558in}}%
\pgfpathlineto{\pgfqpoint{3.752959in}{0.773561in}}%
\pgfpathlineto{\pgfqpoint{3.753255in}{0.773563in}}%
\pgfpathlineto{\pgfqpoint{3.753551in}{0.773566in}}%
\pgfpathlineto{\pgfqpoint{3.753847in}{0.773568in}}%
\pgfpathlineto{\pgfqpoint{3.754143in}{0.773571in}}%
\pgfpathlineto{\pgfqpoint{3.754439in}{0.773573in}}%
\pgfpathlineto{\pgfqpoint{3.754735in}{0.773576in}}%
\pgfpathlineto{\pgfqpoint{3.755031in}{0.773578in}}%
\pgfpathlineto{\pgfqpoint{3.755327in}{0.773581in}}%
\pgfpathlineto{\pgfqpoint{3.755623in}{0.773583in}}%
\pgfpathlineto{\pgfqpoint{3.755919in}{0.773586in}}%
\pgfpathlineto{\pgfqpoint{3.756215in}{0.773588in}}%
\pgfpathlineto{\pgfqpoint{3.756511in}{0.773591in}}%
\pgfpathlineto{\pgfqpoint{3.756807in}{0.773593in}}%
\pgfpathlineto{\pgfqpoint{3.757103in}{0.773600in}}%
\pgfpathlineto{\pgfqpoint{3.757399in}{0.773612in}}%
\pgfpathlineto{\pgfqpoint{3.757695in}{0.773622in}}%
\pgfpathlineto{\pgfqpoint{3.757991in}{0.773632in}}%
\pgfpathlineto{\pgfqpoint{3.758287in}{0.773641in}}%
\pgfpathlineto{\pgfqpoint{3.758583in}{0.773649in}}%
\pgfpathlineto{\pgfqpoint{3.758879in}{0.773639in}}%
\pgfpathlineto{\pgfqpoint{3.759175in}{0.773615in}}%
\pgfpathlineto{\pgfqpoint{3.759471in}{0.773592in}}%
\pgfpathlineto{\pgfqpoint{3.759767in}{0.773568in}}%
\pgfpathlineto{\pgfqpoint{3.760063in}{0.773545in}}%
\pgfpathlineto{\pgfqpoint{3.760359in}{0.773522in}}%
\pgfpathlineto{\pgfqpoint{3.760655in}{0.773499in}}%
\pgfpathlineto{\pgfqpoint{3.760951in}{0.773476in}}%
\pgfpathlineto{\pgfqpoint{3.761247in}{0.773453in}}%
\pgfpathlineto{\pgfqpoint{3.761543in}{0.773430in}}%
\pgfpathlineto{\pgfqpoint{3.761839in}{0.773407in}}%
\pgfpathlineto{\pgfqpoint{3.762135in}{0.773383in}}%
\pgfpathlineto{\pgfqpoint{3.762431in}{0.773360in}}%
\pgfpathlineto{\pgfqpoint{3.762727in}{0.773337in}}%
\pgfpathlineto{\pgfqpoint{3.763023in}{0.773314in}}%
\pgfpathlineto{\pgfqpoint{3.763319in}{0.773291in}}%
\pgfpathlineto{\pgfqpoint{3.763615in}{0.773268in}}%
\pgfpathlineto{\pgfqpoint{3.763911in}{0.773245in}}%
\pgfpathlineto{\pgfqpoint{3.764207in}{0.773222in}}%
\pgfpathlineto{\pgfqpoint{3.764503in}{0.773198in}}%
\pgfpathlineto{\pgfqpoint{3.764799in}{0.773175in}}%
\pgfpathlineto{\pgfqpoint{3.765095in}{0.773152in}}%
\pgfpathlineto{\pgfqpoint{3.765391in}{0.773565in}}%
\pgfpathlineto{\pgfqpoint{3.765687in}{0.773925in}}%
\pgfpathlineto{\pgfqpoint{3.765983in}{0.773952in}}%
\pgfpathlineto{\pgfqpoint{3.766279in}{0.774064in}}%
\pgfpathlineto{\pgfqpoint{3.766575in}{0.773734in}}%
\pgfpathlineto{\pgfqpoint{3.766871in}{0.773325in}}%
\pgfpathlineto{\pgfqpoint{3.767167in}{0.773301in}}%
\pgfpathlineto{\pgfqpoint{3.767463in}{0.773276in}}%
\pgfpathlineto{\pgfqpoint{3.767759in}{0.773252in}}%
\pgfpathlineto{\pgfqpoint{3.768055in}{0.773228in}}%
\pgfpathlineto{\pgfqpoint{3.768351in}{0.773204in}}%
\pgfpathlineto{\pgfqpoint{3.768647in}{0.773180in}}%
\pgfpathlineto{\pgfqpoint{3.768943in}{0.773156in}}%
\pgfpathlineto{\pgfqpoint{3.769239in}{0.773131in}}%
\pgfpathlineto{\pgfqpoint{3.769535in}{0.773107in}}%
\pgfpathlineto{\pgfqpoint{3.769831in}{0.773083in}}%
\pgfpathlineto{\pgfqpoint{3.770127in}{0.773059in}}%
\pgfpathlineto{\pgfqpoint{3.770423in}{0.773035in}}%
\pgfpathlineto{\pgfqpoint{3.770719in}{0.773010in}}%
\pgfpathlineto{\pgfqpoint{3.771015in}{0.772986in}}%
\pgfpathlineto{\pgfqpoint{3.771311in}{0.772962in}}%
\pgfpathlineto{\pgfqpoint{3.771607in}{0.772938in}}%
\pgfpathlineto{\pgfqpoint{3.771903in}{0.772878in}}%
\pgfpathlineto{\pgfqpoint{3.772199in}{0.772999in}}%
\pgfpathlineto{\pgfqpoint{3.772495in}{0.772969in}}%
\pgfpathlineto{\pgfqpoint{3.772791in}{0.772732in}}%
\pgfpathlineto{\pgfqpoint{3.773087in}{0.772864in}}%
\pgfpathlineto{\pgfqpoint{3.773383in}{0.773138in}}%
\pgfpathlineto{\pgfqpoint{3.773679in}{0.773134in}}%
\pgfpathlineto{\pgfqpoint{3.773975in}{0.773129in}}%
\pgfpathlineto{\pgfqpoint{3.774271in}{0.773124in}}%
\pgfpathlineto{\pgfqpoint{3.774567in}{0.773119in}}%
\pgfpathlineto{\pgfqpoint{3.774863in}{0.773114in}}%
\pgfpathlineto{\pgfqpoint{3.775159in}{0.773109in}}%
\pgfpathlineto{\pgfqpoint{3.775455in}{0.773104in}}%
\pgfpathlineto{\pgfqpoint{3.775751in}{0.773099in}}%
\pgfpathlineto{\pgfqpoint{3.776047in}{0.773095in}}%
\pgfpathlineto{\pgfqpoint{3.776343in}{0.773090in}}%
\pgfpathlineto{\pgfqpoint{3.776639in}{0.773085in}}%
\pgfpathlineto{\pgfqpoint{3.776935in}{0.773080in}}%
\pgfpathlineto{\pgfqpoint{3.777231in}{0.773075in}}%
\pgfpathlineto{\pgfqpoint{3.777527in}{0.773070in}}%
\pgfpathlineto{\pgfqpoint{3.777823in}{0.773065in}}%
\pgfpathlineto{\pgfqpoint{3.778119in}{0.773060in}}%
\pgfpathlineto{\pgfqpoint{3.778415in}{0.773055in}}%
\pgfpathlineto{\pgfqpoint{3.778711in}{0.773050in}}%
\pgfpathlineto{\pgfqpoint{3.779007in}{0.773045in}}%
\pgfpathlineto{\pgfqpoint{3.779303in}{0.773040in}}%
\pgfpathlineto{\pgfqpoint{3.779599in}{0.773036in}}%
\pgfpathlineto{\pgfqpoint{3.779895in}{0.773030in}}%
\pgfpathlineto{\pgfqpoint{3.780191in}{0.773020in}}%
\pgfpathlineto{\pgfqpoint{3.780487in}{0.773009in}}%
\pgfpathlineto{\pgfqpoint{3.780783in}{0.772998in}}%
\pgfpathlineto{\pgfqpoint{3.781079in}{0.772988in}}%
\pgfpathlineto{\pgfqpoint{3.781375in}{0.772977in}}%
\pgfpathlineto{\pgfqpoint{3.781671in}{0.772966in}}%
\pgfpathlineto{\pgfqpoint{3.781967in}{0.772955in}}%
\pgfpathlineto{\pgfqpoint{3.782263in}{0.772945in}}%
\pgfpathlineto{\pgfqpoint{3.782559in}{0.772934in}}%
\pgfpathlineto{\pgfqpoint{3.782855in}{0.772923in}}%
\pgfpathlineto{\pgfqpoint{3.783151in}{0.772913in}}%
\pgfpathlineto{\pgfqpoint{3.783447in}{0.772902in}}%
\pgfpathlineto{\pgfqpoint{3.783743in}{0.772891in}}%
\pgfpathlineto{\pgfqpoint{3.784039in}{0.772881in}}%
\pgfpathlineto{\pgfqpoint{3.784335in}{0.772870in}}%
\pgfpathlineto{\pgfqpoint{3.784631in}{0.772859in}}%
\pgfpathlineto{\pgfqpoint{3.784927in}{0.772849in}}%
\pgfpathlineto{\pgfqpoint{3.785223in}{0.772838in}}%
\pgfpathlineto{\pgfqpoint{3.785519in}{0.772827in}}%
\pgfpathlineto{\pgfqpoint{3.785815in}{0.772816in}}%
\pgfpathlineto{\pgfqpoint{3.786111in}{0.772806in}}%
\pgfpathlineto{\pgfqpoint{3.786407in}{0.772795in}}%
\pgfpathlineto{\pgfqpoint{3.786703in}{0.772784in}}%
\pgfpathlineto{\pgfqpoint{3.786999in}{0.772774in}}%
\pgfpathlineto{\pgfqpoint{3.787295in}{0.772763in}}%
\pgfpathlineto{\pgfqpoint{3.787591in}{0.772752in}}%
\pgfpathlineto{\pgfqpoint{3.787887in}{0.772742in}}%
\pgfpathlineto{\pgfqpoint{3.788183in}{0.772731in}}%
\pgfpathlineto{\pgfqpoint{3.788479in}{0.772720in}}%
\pgfpathlineto{\pgfqpoint{3.788775in}{0.772710in}}%
\pgfpathlineto{\pgfqpoint{3.789071in}{0.772699in}}%
\pgfpathlineto{\pgfqpoint{3.789367in}{0.772688in}}%
\pgfpathlineto{\pgfqpoint{3.789663in}{0.772677in}}%
\pgfpathlineto{\pgfqpoint{3.789959in}{0.772667in}}%
\pgfpathlineto{\pgfqpoint{3.790255in}{0.772656in}}%
\pgfpathlineto{\pgfqpoint{3.790551in}{0.772645in}}%
\pgfpathlineto{\pgfqpoint{3.790847in}{0.772635in}}%
\pgfpathlineto{\pgfqpoint{3.791143in}{0.772624in}}%
\pgfpathlineto{\pgfqpoint{3.791439in}{0.772613in}}%
\pgfpathlineto{\pgfqpoint{3.791735in}{0.772603in}}%
\pgfpathlineto{\pgfqpoint{3.792031in}{0.772592in}}%
\pgfpathlineto{\pgfqpoint{3.792327in}{0.772581in}}%
\pgfpathlineto{\pgfqpoint{3.792623in}{0.772571in}}%
\pgfpathlineto{\pgfqpoint{3.792919in}{0.772560in}}%
\pgfpathlineto{\pgfqpoint{3.793215in}{0.772549in}}%
\pgfpathlineto{\pgfqpoint{3.793511in}{0.772538in}}%
\pgfpathlineto{\pgfqpoint{3.793807in}{0.772528in}}%
\pgfpathlineto{\pgfqpoint{3.794103in}{0.772517in}}%
\pgfpathlineto{\pgfqpoint{3.794399in}{0.772506in}}%
\pgfpathlineto{\pgfqpoint{3.794695in}{0.772496in}}%
\pgfpathlineto{\pgfqpoint{3.794991in}{0.772485in}}%
\pgfpathlineto{\pgfqpoint{3.795287in}{0.772474in}}%
\pgfpathlineto{\pgfqpoint{3.795583in}{0.772464in}}%
\pgfpathlineto{\pgfqpoint{3.795879in}{0.772453in}}%
\pgfpathlineto{\pgfqpoint{3.796175in}{0.772442in}}%
\pgfpathlineto{\pgfqpoint{3.796471in}{0.772431in}}%
\pgfpathlineto{\pgfqpoint{3.796767in}{0.772421in}}%
\pgfpathlineto{\pgfqpoint{3.797063in}{0.772410in}}%
\pgfpathlineto{\pgfqpoint{3.797359in}{0.772399in}}%
\pgfpathlineto{\pgfqpoint{3.797655in}{0.772389in}}%
\pgfpathlineto{\pgfqpoint{3.797951in}{0.772378in}}%
\pgfpathlineto{\pgfqpoint{3.798247in}{0.772367in}}%
\pgfpathlineto{\pgfqpoint{3.798543in}{0.772357in}}%
\pgfpathlineto{\pgfqpoint{3.798839in}{0.772346in}}%
\pgfpathlineto{\pgfqpoint{3.799135in}{0.772335in}}%
\pgfpathlineto{\pgfqpoint{3.799431in}{0.772325in}}%
\pgfpathlineto{\pgfqpoint{3.799727in}{0.772314in}}%
\pgfpathlineto{\pgfqpoint{3.800023in}{0.772299in}}%
\pgfpathlineto{\pgfqpoint{3.800319in}{0.772287in}}%
\pgfpathlineto{\pgfqpoint{3.800615in}{0.772277in}}%
\pgfpathlineto{\pgfqpoint{3.800911in}{0.772269in}}%
\pgfpathlineto{\pgfqpoint{3.801207in}{0.772210in}}%
\pgfpathlineto{\pgfqpoint{3.801503in}{0.771843in}}%
\pgfpathlineto{\pgfqpoint{3.801799in}{0.771645in}}%
\pgfpathlineto{\pgfqpoint{3.802095in}{0.771587in}}%
\pgfpathlineto{\pgfqpoint{3.802391in}{0.771552in}}%
\pgfpathlineto{\pgfqpoint{3.802687in}{0.771517in}}%
\pgfpathlineto{\pgfqpoint{3.802983in}{0.771481in}}%
\pgfpathlineto{\pgfqpoint{3.803279in}{0.771446in}}%
\pgfpathlineto{\pgfqpoint{3.803575in}{0.771411in}}%
\pgfpathlineto{\pgfqpoint{3.803871in}{0.771375in}}%
\pgfpathlineto{\pgfqpoint{3.804167in}{0.771340in}}%
\pgfpathlineto{\pgfqpoint{3.804463in}{0.771305in}}%
\pgfpathlineto{\pgfqpoint{3.804759in}{0.771269in}}%
\pgfpathlineto{\pgfqpoint{3.805055in}{0.771234in}}%
\pgfpathlineto{\pgfqpoint{3.805351in}{0.771199in}}%
\pgfpathlineto{\pgfqpoint{3.805647in}{0.771163in}}%
\pgfpathlineto{\pgfqpoint{3.805943in}{0.771128in}}%
\pgfpathlineto{\pgfqpoint{3.806239in}{0.771093in}}%
\pgfpathlineto{\pgfqpoint{3.806535in}{0.771057in}}%
\pgfpathlineto{\pgfqpoint{3.806831in}{0.771022in}}%
\pgfpathlineto{\pgfqpoint{3.807127in}{0.771277in}}%
\pgfpathlineto{\pgfqpoint{3.807423in}{0.771436in}}%
\pgfpathlineto{\pgfqpoint{3.807719in}{0.771425in}}%
\pgfpathlineto{\pgfqpoint{3.808015in}{0.771423in}}%
\pgfpathlineto{\pgfqpoint{3.808311in}{0.771474in}}%
\pgfpathlineto{\pgfqpoint{3.808607in}{0.771515in}}%
\pgfpathlineto{\pgfqpoint{3.808903in}{0.771445in}}%
\pgfpathlineto{\pgfqpoint{3.809199in}{0.771277in}}%
\pgfpathlineto{\pgfqpoint{3.809495in}{0.771226in}}%
\pgfpathlineto{\pgfqpoint{3.809791in}{0.771150in}}%
\pgfpathlineto{\pgfqpoint{3.810087in}{0.770725in}}%
\pgfpathlineto{\pgfqpoint{3.810383in}{0.770214in}}%
\pgfpathlineto{\pgfqpoint{3.810679in}{0.769881in}}%
\pgfpathlineto{\pgfqpoint{3.810975in}{0.769830in}}%
\pgfpathlineto{\pgfqpoint{3.811271in}{0.769781in}}%
\pgfpathlineto{\pgfqpoint{3.811567in}{0.769732in}}%
\pgfpathlineto{\pgfqpoint{3.811863in}{0.769682in}}%
\pgfpathlineto{\pgfqpoint{3.812159in}{0.769633in}}%
\pgfpathlineto{\pgfqpoint{3.812455in}{0.769584in}}%
\pgfpathlineto{\pgfqpoint{3.812751in}{0.769534in}}%
\pgfpathlineto{\pgfqpoint{3.813047in}{0.769485in}}%
\pgfpathlineto{\pgfqpoint{3.813343in}{0.769436in}}%
\pgfpathlineto{\pgfqpoint{3.813639in}{0.769387in}}%
\pgfpathlineto{\pgfqpoint{3.813935in}{0.769337in}}%
\pgfpathlineto{\pgfqpoint{3.814231in}{0.769288in}}%
\pgfpathlineto{\pgfqpoint{3.814527in}{0.769239in}}%
\pgfpathlineto{\pgfqpoint{3.814823in}{0.769190in}}%
\pgfpathlineto{\pgfqpoint{3.815119in}{0.769147in}}%
\pgfpathlineto{\pgfqpoint{3.815415in}{0.769108in}}%
\pgfpathlineto{\pgfqpoint{3.815711in}{0.769058in}}%
\pgfpathlineto{\pgfqpoint{3.816007in}{0.769008in}}%
\pgfpathlineto{\pgfqpoint{3.816303in}{0.768638in}}%
\pgfpathlineto{\pgfqpoint{3.816600in}{0.767794in}}%
\pgfpathlineto{\pgfqpoint{3.816896in}{0.767507in}}%
\pgfpathlineto{\pgfqpoint{3.817192in}{0.767430in}}%
\pgfpathlineto{\pgfqpoint{3.817488in}{0.767391in}}%
\pgfpathlineto{\pgfqpoint{3.817784in}{0.767351in}}%
\pgfpathlineto{\pgfqpoint{3.818080in}{0.767312in}}%
\pgfpathlineto{\pgfqpoint{3.818376in}{0.767272in}}%
\pgfpathlineto{\pgfqpoint{3.818672in}{0.767232in}}%
\pgfpathlineto{\pgfqpoint{3.818968in}{0.767193in}}%
\pgfpathlineto{\pgfqpoint{3.819264in}{0.767153in}}%
\pgfpathlineto{\pgfqpoint{3.819560in}{0.767114in}}%
\pgfpathlineto{\pgfqpoint{3.819856in}{0.767074in}}%
\pgfpathlineto{\pgfqpoint{3.820152in}{0.767034in}}%
\pgfpathlineto{\pgfqpoint{3.820448in}{0.766995in}}%
\pgfpathlineto{\pgfqpoint{3.820744in}{0.766955in}}%
\pgfpathlineto{\pgfqpoint{3.821040in}{0.766935in}}%
\pgfpathlineto{\pgfqpoint{3.821336in}{0.766954in}}%
\pgfpathlineto{\pgfqpoint{3.821632in}{0.766924in}}%
\pgfpathlineto{\pgfqpoint{3.821928in}{0.766885in}}%
\pgfpathlineto{\pgfqpoint{3.822224in}{0.766846in}}%
\pgfpathlineto{\pgfqpoint{3.822520in}{0.766807in}}%
\pgfpathlineto{\pgfqpoint{3.822816in}{0.766768in}}%
\pgfpathlineto{\pgfqpoint{3.823112in}{0.766733in}}%
\pgfpathlineto{\pgfqpoint{3.823408in}{0.766725in}}%
\pgfpathlineto{\pgfqpoint{3.823704in}{0.766724in}}%
\pgfpathlineto{\pgfqpoint{3.824000in}{0.766723in}}%
\pgfpathlineto{\pgfqpoint{3.824296in}{0.766721in}}%
\pgfpathlineto{\pgfqpoint{3.824592in}{0.766720in}}%
\pgfpathlineto{\pgfqpoint{3.824888in}{0.766719in}}%
\pgfpathlineto{\pgfqpoint{3.825184in}{0.766717in}}%
\pgfpathlineto{\pgfqpoint{3.825480in}{0.766716in}}%
\pgfpathlineto{\pgfqpoint{3.825776in}{0.766715in}}%
\pgfpathlineto{\pgfqpoint{3.826072in}{0.766713in}}%
\pgfpathlineto{\pgfqpoint{3.826368in}{0.766712in}}%
\pgfpathlineto{\pgfqpoint{3.826664in}{0.766711in}}%
\pgfpathlineto{\pgfqpoint{3.826960in}{0.766709in}}%
\pgfpathlineto{\pgfqpoint{3.827256in}{0.766708in}}%
\pgfpathlineto{\pgfqpoint{3.827552in}{0.766707in}}%
\pgfpathlineto{\pgfqpoint{3.827848in}{0.766705in}}%
\pgfpathlineto{\pgfqpoint{3.828144in}{0.766704in}}%
\pgfpathlineto{\pgfqpoint{3.828440in}{0.766703in}}%
\pgfpathlineto{\pgfqpoint{3.828736in}{0.766701in}}%
\pgfpathlineto{\pgfqpoint{3.829032in}{0.766700in}}%
\pgfpathlineto{\pgfqpoint{3.829328in}{0.766698in}}%
\pgfpathlineto{\pgfqpoint{3.829624in}{0.766697in}}%
\pgfpathlineto{\pgfqpoint{3.829920in}{0.766696in}}%
\pgfpathlineto{\pgfqpoint{3.830216in}{0.766694in}}%
\pgfpathlineto{\pgfqpoint{3.830512in}{0.766693in}}%
\pgfpathlineto{\pgfqpoint{3.830808in}{0.766692in}}%
\pgfpathlineto{\pgfqpoint{3.831104in}{0.766691in}}%
\pgfpathlineto{\pgfqpoint{3.831400in}{0.766689in}}%
\pgfpathlineto{\pgfqpoint{3.831696in}{0.766688in}}%
\pgfpathlineto{\pgfqpoint{3.831992in}{0.766687in}}%
\pgfpathlineto{\pgfqpoint{3.832288in}{0.766685in}}%
\pgfpathlineto{\pgfqpoint{3.832584in}{0.766684in}}%
\pgfpathlineto{\pgfqpoint{3.832880in}{0.766683in}}%
\pgfpathlineto{\pgfqpoint{3.833176in}{0.766682in}}%
\pgfpathlineto{\pgfqpoint{3.833472in}{0.766680in}}%
\pgfpathlineto{\pgfqpoint{3.833768in}{0.766679in}}%
\pgfpathlineto{\pgfqpoint{3.834064in}{0.766678in}}%
\pgfpathlineto{\pgfqpoint{3.834360in}{0.766676in}}%
\pgfpathlineto{\pgfqpoint{3.834656in}{0.766675in}}%
\pgfpathlineto{\pgfqpoint{3.834952in}{0.766674in}}%
\pgfpathlineto{\pgfqpoint{3.835248in}{0.766673in}}%
\pgfpathlineto{\pgfqpoint{3.835544in}{0.766671in}}%
\pgfpathlineto{\pgfqpoint{3.835840in}{0.766670in}}%
\pgfpathlineto{\pgfqpoint{3.836136in}{0.766669in}}%
\pgfpathlineto{\pgfqpoint{3.836432in}{0.766667in}}%
\pgfpathlineto{\pgfqpoint{3.836728in}{0.766666in}}%
\pgfpathlineto{\pgfqpoint{3.837024in}{0.766665in}}%
\pgfpathlineto{\pgfqpoint{3.837320in}{0.766664in}}%
\pgfpathlineto{\pgfqpoint{3.837616in}{0.766662in}}%
\pgfpathlineto{\pgfqpoint{3.837912in}{0.766661in}}%
\pgfpathlineto{\pgfqpoint{3.838208in}{0.766660in}}%
\pgfpathlineto{\pgfqpoint{3.838504in}{0.766658in}}%
\pgfpathlineto{\pgfqpoint{3.838800in}{0.766657in}}%
\pgfpathlineto{\pgfqpoint{3.839096in}{0.766656in}}%
\pgfpathlineto{\pgfqpoint{3.839392in}{0.766655in}}%
\pgfpathlineto{\pgfqpoint{3.839688in}{0.766653in}}%
\pgfpathlineto{\pgfqpoint{3.839984in}{0.766652in}}%
\pgfpathlineto{\pgfqpoint{3.840280in}{0.766651in}}%
\pgfpathlineto{\pgfqpoint{3.840576in}{0.766649in}}%
\pgfpathlineto{\pgfqpoint{3.840872in}{0.766648in}}%
\pgfpathlineto{\pgfqpoint{3.841168in}{0.766647in}}%
\pgfpathlineto{\pgfqpoint{3.841464in}{0.766646in}}%
\pgfpathlineto{\pgfqpoint{3.841760in}{0.766644in}}%
\pgfpathlineto{\pgfqpoint{3.842056in}{0.766643in}}%
\pgfpathlineto{\pgfqpoint{3.842352in}{0.766642in}}%
\pgfpathlineto{\pgfqpoint{3.842648in}{0.766640in}}%
\pgfpathlineto{\pgfqpoint{3.842944in}{0.766639in}}%
\pgfpathlineto{\pgfqpoint{3.843240in}{0.766638in}}%
\pgfpathlineto{\pgfqpoint{3.843536in}{0.766637in}}%
\pgfpathlineto{\pgfqpoint{3.843832in}{0.766635in}}%
\pgfpathlineto{\pgfqpoint{3.844128in}{0.766634in}}%
\pgfpathlineto{\pgfqpoint{3.844424in}{0.766633in}}%
\pgfpathlineto{\pgfqpoint{3.844720in}{0.766636in}}%
\pgfpathlineto{\pgfqpoint{3.845016in}{0.766652in}}%
\pgfpathlineto{\pgfqpoint{3.845312in}{0.766668in}}%
\pgfpathlineto{\pgfqpoint{3.845608in}{0.766685in}}%
\pgfpathlineto{\pgfqpoint{3.845904in}{0.766701in}}%
\pgfpathlineto{\pgfqpoint{3.846200in}{0.766717in}}%
\pgfpathlineto{\pgfqpoint{3.846496in}{0.766734in}}%
\pgfpathlineto{\pgfqpoint{3.846792in}{0.766750in}}%
\pgfpathlineto{\pgfqpoint{3.847088in}{0.766766in}}%
\pgfpathlineto{\pgfqpoint{3.847384in}{0.766783in}}%
\pgfpathlineto{\pgfqpoint{3.847680in}{0.766799in}}%
\pgfpathlineto{\pgfqpoint{3.847976in}{0.766815in}}%
\pgfpathlineto{\pgfqpoint{3.848272in}{0.766832in}}%
\pgfpathlineto{\pgfqpoint{3.848568in}{0.766848in}}%
\pgfpathlineto{\pgfqpoint{3.848864in}{0.766864in}}%
\pgfpathlineto{\pgfqpoint{3.849160in}{0.766881in}}%
\pgfpathlineto{\pgfqpoint{3.849456in}{0.766897in}}%
\pgfpathlineto{\pgfqpoint{3.849752in}{0.766913in}}%
\pgfpathlineto{\pgfqpoint{3.850048in}{0.766930in}}%
\pgfpathlineto{\pgfqpoint{3.850344in}{0.766946in}}%
\pgfpathlineto{\pgfqpoint{3.850640in}{0.766962in}}%
\pgfpathlineto{\pgfqpoint{3.850936in}{0.766979in}}%
\pgfpathlineto{\pgfqpoint{3.851232in}{0.766995in}}%
\pgfpathlineto{\pgfqpoint{3.851528in}{0.767011in}}%
\pgfpathlineto{\pgfqpoint{3.851824in}{0.767028in}}%
\pgfpathlineto{\pgfqpoint{3.852120in}{0.767042in}}%
\pgfpathlineto{\pgfqpoint{3.852416in}{0.767039in}}%
\pgfpathlineto{\pgfqpoint{3.852712in}{0.767031in}}%
\pgfpathlineto{\pgfqpoint{3.853008in}{0.767023in}}%
\pgfpathlineto{\pgfqpoint{3.853304in}{0.767015in}}%
\pgfpathlineto{\pgfqpoint{3.853600in}{0.767007in}}%
\pgfpathlineto{\pgfqpoint{3.853896in}{0.766999in}}%
\pgfpathlineto{\pgfqpoint{3.854192in}{0.766990in}}%
\pgfpathlineto{\pgfqpoint{3.854488in}{0.766982in}}%
\pgfpathlineto{\pgfqpoint{3.854784in}{0.766974in}}%
\pgfpathlineto{\pgfqpoint{3.855080in}{0.766966in}}%
\pgfpathlineto{\pgfqpoint{3.855376in}{0.766958in}}%
\pgfpathlineto{\pgfqpoint{3.855672in}{0.766950in}}%
\pgfpathlineto{\pgfqpoint{3.855968in}{0.766941in}}%
\pgfpathlineto{\pgfqpoint{3.856264in}{0.766933in}}%
\pgfpathlineto{\pgfqpoint{3.856560in}{0.766925in}}%
\pgfpathlineto{\pgfqpoint{3.856856in}{0.766875in}}%
\pgfpathlineto{\pgfqpoint{3.857152in}{0.766759in}}%
\pgfpathlineto{\pgfqpoint{3.857448in}{0.766642in}}%
\pgfpathlineto{\pgfqpoint{3.857744in}{0.766525in}}%
\pgfpathlineto{\pgfqpoint{3.858040in}{0.766410in}}%
\pgfpathlineto{\pgfqpoint{3.858336in}{0.766301in}}%
\pgfpathlineto{\pgfqpoint{3.858632in}{0.766193in}}%
\pgfpathlineto{\pgfqpoint{3.858928in}{0.766084in}}%
\pgfpathlineto{\pgfqpoint{3.859224in}{0.766021in}}%
\pgfpathlineto{\pgfqpoint{3.859520in}{0.766003in}}%
\pgfpathlineto{\pgfqpoint{3.859816in}{0.765986in}}%
\pgfpathlineto{\pgfqpoint{3.860112in}{0.765969in}}%
\pgfpathlineto{\pgfqpoint{3.860408in}{0.765952in}}%
\pgfpathlineto{\pgfqpoint{3.860704in}{0.765934in}}%
\pgfpathlineto{\pgfqpoint{3.861000in}{0.765917in}}%
\pgfpathlineto{\pgfqpoint{3.861296in}{0.765900in}}%
\pgfpathlineto{\pgfqpoint{3.861592in}{0.765883in}}%
\pgfpathlineto{\pgfqpoint{3.861888in}{0.765865in}}%
\pgfpathlineto{\pgfqpoint{3.862184in}{0.765848in}}%
\pgfpathlineto{\pgfqpoint{3.862480in}{0.765831in}}%
\pgfpathlineto{\pgfqpoint{3.862776in}{0.765814in}}%
\pgfpathlineto{\pgfqpoint{3.863072in}{0.765796in}}%
\pgfpathlineto{\pgfqpoint{3.863368in}{0.765779in}}%
\pgfpathlineto{\pgfqpoint{3.863664in}{0.765768in}}%
\pgfpathlineto{\pgfqpoint{3.863960in}{0.765729in}}%
\pgfpathlineto{\pgfqpoint{3.864256in}{0.765722in}}%
\pgfpathlineto{\pgfqpoint{3.864552in}{0.765708in}}%
\pgfpathlineto{\pgfqpoint{3.864848in}{0.765690in}}%
\pgfpathlineto{\pgfqpoint{3.865144in}{0.765673in}}%
\pgfpathlineto{\pgfqpoint{3.865440in}{0.765656in}}%
\pgfpathlineto{\pgfqpoint{3.865736in}{0.765639in}}%
\pgfpathlineto{\pgfqpoint{3.866032in}{0.765621in}}%
\pgfpathlineto{\pgfqpoint{3.866328in}{0.765604in}}%
\pgfpathlineto{\pgfqpoint{3.866624in}{0.765589in}}%
\pgfpathlineto{\pgfqpoint{3.866920in}{0.765583in}}%
\pgfpathlineto{\pgfqpoint{3.867216in}{0.765578in}}%
\pgfpathlineto{\pgfqpoint{3.867512in}{0.765573in}}%
\pgfpathlineto{\pgfqpoint{3.867808in}{0.765569in}}%
\pgfpathlineto{\pgfqpoint{3.868104in}{0.765564in}}%
\pgfpathlineto{\pgfqpoint{3.868400in}{0.765560in}}%
\pgfpathlineto{\pgfqpoint{3.868696in}{0.765555in}}%
\pgfpathlineto{\pgfqpoint{3.868992in}{0.765550in}}%
\pgfpathlineto{\pgfqpoint{3.869288in}{0.765546in}}%
\pgfpathlineto{\pgfqpoint{3.869584in}{0.765541in}}%
\pgfpathlineto{\pgfqpoint{3.869880in}{0.765537in}}%
\pgfpathlineto{\pgfqpoint{3.870176in}{0.765532in}}%
\pgfpathlineto{\pgfqpoint{3.870472in}{0.765528in}}%
\pgfpathlineto{\pgfqpoint{3.870768in}{0.765523in}}%
\pgfpathlineto{\pgfqpoint{3.871064in}{0.765947in}}%
\pgfpathlineto{\pgfqpoint{3.871360in}{0.766069in}}%
\pgfpathlineto{\pgfqpoint{3.871656in}{0.766130in}}%
\pgfpathlineto{\pgfqpoint{3.871952in}{0.766191in}}%
\pgfpathlineto{\pgfqpoint{3.872248in}{0.766253in}}%
\pgfpathlineto{\pgfqpoint{3.872544in}{0.765710in}}%
\pgfpathlineto{\pgfqpoint{3.872840in}{0.765505in}}%
\pgfpathlineto{\pgfqpoint{3.873136in}{0.765501in}}%
\pgfpathlineto{\pgfqpoint{3.873432in}{0.765498in}}%
\pgfpathlineto{\pgfqpoint{3.873728in}{0.765494in}}%
\pgfpathlineto{\pgfqpoint{3.874024in}{0.765491in}}%
\pgfpathlineto{\pgfqpoint{3.874320in}{0.765487in}}%
\pgfpathlineto{\pgfqpoint{3.874616in}{0.765484in}}%
\pgfpathlineto{\pgfqpoint{3.874912in}{0.765480in}}%
\pgfpathlineto{\pgfqpoint{3.875208in}{0.765477in}}%
\pgfpathlineto{\pgfqpoint{3.875504in}{0.765474in}}%
\pgfpathlineto{\pgfqpoint{3.875800in}{0.765470in}}%
\pgfpathlineto{\pgfqpoint{3.876096in}{0.765467in}}%
\pgfpathlineto{\pgfqpoint{3.876392in}{0.765463in}}%
\pgfpathlineto{\pgfqpoint{3.876688in}{0.765460in}}%
\pgfpathlineto{\pgfqpoint{3.876984in}{0.765456in}}%
\pgfpathlineto{\pgfqpoint{3.877280in}{0.765453in}}%
\pgfpathlineto{\pgfqpoint{3.877576in}{0.765449in}}%
\pgfpathlineto{\pgfqpoint{3.877872in}{0.765446in}}%
\pgfpathlineto{\pgfqpoint{3.878168in}{0.765451in}}%
\pgfpathlineto{\pgfqpoint{3.878464in}{0.765456in}}%
\pgfpathlineto{\pgfqpoint{3.878760in}{0.765387in}}%
\pgfpathlineto{\pgfqpoint{3.879056in}{0.765369in}}%
\pgfpathlineto{\pgfqpoint{3.879352in}{0.765365in}}%
\pgfpathlineto{\pgfqpoint{3.879648in}{0.765361in}}%
\pgfpathlineto{\pgfqpoint{3.879944in}{0.765357in}}%
\pgfpathlineto{\pgfqpoint{3.880240in}{0.765353in}}%
\pgfpathlineto{\pgfqpoint{3.880536in}{0.765349in}}%
\pgfpathlineto{\pgfqpoint{3.880832in}{0.765345in}}%
\pgfpathlineto{\pgfqpoint{3.881128in}{0.765340in}}%
\pgfpathlineto{\pgfqpoint{3.881424in}{0.765336in}}%
\pgfpathlineto{\pgfqpoint{3.881720in}{0.765332in}}%
\pgfpathlineto{\pgfqpoint{3.882016in}{0.765328in}}%
\pgfpathlineto{\pgfqpoint{3.882312in}{0.765324in}}%
\pgfpathlineto{\pgfqpoint{3.882608in}{0.765320in}}%
\pgfpathlineto{\pgfqpoint{3.882904in}{0.765316in}}%
\pgfpathlineto{\pgfqpoint{3.883200in}{0.765312in}}%
\pgfpathlineto{\pgfqpoint{3.883496in}{0.765308in}}%
\pgfpathlineto{\pgfqpoint{3.883792in}{0.765304in}}%
\pgfpathlineto{\pgfqpoint{3.884089in}{0.765300in}}%
\pgfpathlineto{\pgfqpoint{3.884385in}{0.765296in}}%
\pgfpathlineto{\pgfqpoint{3.884681in}{0.765292in}}%
\pgfpathlineto{\pgfqpoint{3.884977in}{0.765288in}}%
\pgfpathlineto{\pgfqpoint{3.885273in}{0.765284in}}%
\pgfpathlineto{\pgfqpoint{3.885569in}{0.765280in}}%
\pgfpathlineto{\pgfqpoint{3.885865in}{0.765276in}}%
\pgfpathlineto{\pgfqpoint{3.886161in}{0.765272in}}%
\pgfpathlineto{\pgfqpoint{3.886457in}{0.765268in}}%
\pgfpathlineto{\pgfqpoint{3.886753in}{0.765264in}}%
\pgfpathlineto{\pgfqpoint{3.887049in}{0.765260in}}%
\pgfpathlineto{\pgfqpoint{3.887345in}{0.765256in}}%
\pgfpathlineto{\pgfqpoint{3.887641in}{0.765252in}}%
\pgfpathlineto{\pgfqpoint{3.887937in}{0.765248in}}%
\pgfpathlineto{\pgfqpoint{3.888233in}{0.765244in}}%
\pgfpathlineto{\pgfqpoint{3.888529in}{0.765240in}}%
\pgfpathlineto{\pgfqpoint{3.888825in}{0.765236in}}%
\pgfpathlineto{\pgfqpoint{3.889121in}{0.765232in}}%
\pgfpathlineto{\pgfqpoint{3.889417in}{0.765228in}}%
\pgfpathlineto{\pgfqpoint{3.889713in}{0.765224in}}%
\pgfpathlineto{\pgfqpoint{3.890009in}{0.765220in}}%
\pgfpathlineto{\pgfqpoint{3.890305in}{0.765216in}}%
\pgfpathlineto{\pgfqpoint{3.890601in}{0.765212in}}%
\pgfpathlineto{\pgfqpoint{3.890897in}{0.765208in}}%
\pgfpathlineto{\pgfqpoint{3.891193in}{0.765204in}}%
\pgfpathlineto{\pgfqpoint{3.891489in}{0.765200in}}%
\pgfpathlineto{\pgfqpoint{3.891785in}{0.765196in}}%
\pgfpathlineto{\pgfqpoint{3.892081in}{0.765191in}}%
\pgfpathlineto{\pgfqpoint{3.892377in}{0.765187in}}%
\pgfpathlineto{\pgfqpoint{3.892673in}{0.765183in}}%
\pgfpathlineto{\pgfqpoint{3.892969in}{0.765179in}}%
\pgfpathlineto{\pgfqpoint{3.893265in}{0.765175in}}%
\pgfpathlineto{\pgfqpoint{3.893561in}{0.765171in}}%
\pgfpathlineto{\pgfqpoint{3.893857in}{0.765167in}}%
\pgfpathlineto{\pgfqpoint{3.894153in}{0.765163in}}%
\pgfpathlineto{\pgfqpoint{3.894449in}{0.765159in}}%
\pgfpathlineto{\pgfqpoint{3.894745in}{0.765155in}}%
\pgfpathlineto{\pgfqpoint{3.895041in}{0.765151in}}%
\pgfpathlineto{\pgfqpoint{3.895337in}{0.765147in}}%
\pgfpathlineto{\pgfqpoint{3.895633in}{0.765143in}}%
\pgfpathlineto{\pgfqpoint{3.895929in}{0.765139in}}%
\pgfpathlineto{\pgfqpoint{3.896225in}{0.765135in}}%
\pgfpathlineto{\pgfqpoint{3.896521in}{0.765131in}}%
\pgfpathlineto{\pgfqpoint{3.896817in}{0.765127in}}%
\pgfpathlineto{\pgfqpoint{3.897113in}{0.765123in}}%
\pgfpathlineto{\pgfqpoint{3.897409in}{0.765119in}}%
\pgfpathlineto{\pgfqpoint{3.897705in}{0.765115in}}%
\pgfpathlineto{\pgfqpoint{3.898001in}{0.765111in}}%
\pgfpathlineto{\pgfqpoint{3.898297in}{0.765107in}}%
\pgfpathlineto{\pgfqpoint{3.898593in}{0.765103in}}%
\pgfpathlineto{\pgfqpoint{3.898889in}{0.765099in}}%
\pgfpathlineto{\pgfqpoint{3.899185in}{0.765140in}}%
\pgfpathlineto{\pgfqpoint{3.899481in}{0.765263in}}%
\pgfpathlineto{\pgfqpoint{3.899777in}{0.766044in}}%
\pgfpathlineto{\pgfqpoint{3.900073in}{0.766354in}}%
\pgfpathlineto{\pgfqpoint{3.900369in}{0.766515in}}%
\pgfpathlineto{\pgfqpoint{3.900665in}{0.766677in}}%
\pgfpathlineto{\pgfqpoint{3.900961in}{0.766838in}}%
\pgfpathlineto{\pgfqpoint{3.901257in}{0.766929in}}%
\pgfpathlineto{\pgfqpoint{3.901553in}{0.766967in}}%
\pgfpathlineto{\pgfqpoint{3.901849in}{0.767006in}}%
\pgfpathlineto{\pgfqpoint{3.902145in}{0.767044in}}%
\pgfpathlineto{\pgfqpoint{3.902441in}{0.767082in}}%
\pgfpathlineto{\pgfqpoint{3.902737in}{0.767121in}}%
\pgfpathlineto{\pgfqpoint{3.903033in}{0.767159in}}%
\pgfpathlineto{\pgfqpoint{3.903329in}{0.767197in}}%
\pgfpathlineto{\pgfqpoint{3.903625in}{0.767236in}}%
\pgfpathlineto{\pgfqpoint{3.903921in}{0.767274in}}%
\pgfpathlineto{\pgfqpoint{3.904217in}{0.767312in}}%
\pgfpathlineto{\pgfqpoint{3.904513in}{0.767351in}}%
\pgfpathlineto{\pgfqpoint{3.904809in}{0.767389in}}%
\pgfpathlineto{\pgfqpoint{3.905105in}{0.767428in}}%
\pgfpathlineto{\pgfqpoint{3.905401in}{0.767466in}}%
\pgfpathlineto{\pgfqpoint{3.905697in}{0.767504in}}%
\pgfpathlineto{\pgfqpoint{3.905993in}{0.767543in}}%
\pgfpathlineto{\pgfqpoint{3.906289in}{0.767581in}}%
\pgfpathlineto{\pgfqpoint{3.906585in}{0.767619in}}%
\pgfpathlineto{\pgfqpoint{3.906881in}{0.767658in}}%
\pgfpathlineto{\pgfqpoint{3.907177in}{0.767433in}}%
\pgfpathlineto{\pgfqpoint{3.907473in}{0.764657in}}%
\pgfpathlineto{\pgfqpoint{3.907769in}{0.764578in}}%
\pgfpathlineto{\pgfqpoint{3.908065in}{0.764336in}}%
\pgfpathlineto{\pgfqpoint{3.908361in}{0.764091in}}%
\pgfpathlineto{\pgfqpoint{3.908657in}{0.763847in}}%
\pgfpathlineto{\pgfqpoint{3.908953in}{0.764763in}}%
\pgfpathlineto{\pgfqpoint{3.909249in}{0.765987in}}%
\pgfpathlineto{\pgfqpoint{3.909545in}{0.765869in}}%
\pgfpathlineto{\pgfqpoint{3.909841in}{0.765749in}}%
\pgfpathlineto{\pgfqpoint{3.910137in}{0.765629in}}%
\pgfpathlineto{\pgfqpoint{3.910433in}{0.765509in}}%
\pgfpathlineto{\pgfqpoint{3.910729in}{0.765389in}}%
\pgfpathlineto{\pgfqpoint{3.911025in}{0.765269in}}%
\pgfpathlineto{\pgfqpoint{3.911321in}{0.765149in}}%
\pgfpathlineto{\pgfqpoint{3.911617in}{0.765029in}}%
\pgfpathlineto{\pgfqpoint{3.911913in}{0.764909in}}%
\pgfpathlineto{\pgfqpoint{3.912209in}{0.764789in}}%
\pgfpathlineto{\pgfqpoint{3.912505in}{0.764669in}}%
\pgfpathlineto{\pgfqpoint{3.912801in}{0.764549in}}%
\pgfpathlineto{\pgfqpoint{3.913097in}{0.764496in}}%
\pgfpathlineto{\pgfqpoint{3.913393in}{0.764989in}}%
\pgfpathlineto{\pgfqpoint{3.913689in}{0.765592in}}%
\pgfpathlineto{\pgfqpoint{3.913985in}{0.765790in}}%
\pgfpathlineto{\pgfqpoint{3.914281in}{0.765785in}}%
\pgfpathlineto{\pgfqpoint{3.914577in}{0.765840in}}%
\pgfpathlineto{\pgfqpoint{3.914873in}{0.765857in}}%
\pgfpathlineto{\pgfqpoint{3.915169in}{0.765838in}}%
\pgfpathlineto{\pgfqpoint{3.915465in}{0.765818in}}%
\pgfpathlineto{\pgfqpoint{3.915761in}{0.765799in}}%
\pgfpathlineto{\pgfqpoint{3.916057in}{0.765780in}}%
\pgfpathlineto{\pgfqpoint{3.916353in}{0.765761in}}%
\pgfpathlineto{\pgfqpoint{3.916649in}{0.765741in}}%
\pgfpathlineto{\pgfqpoint{3.916945in}{0.765722in}}%
\pgfpathlineto{\pgfqpoint{3.917241in}{0.765703in}}%
\pgfpathlineto{\pgfqpoint{3.917537in}{0.765683in}}%
\pgfpathlineto{\pgfqpoint{3.917833in}{0.765664in}}%
\pgfpathlineto{\pgfqpoint{3.918129in}{0.765645in}}%
\pgfpathlineto{\pgfqpoint{3.918425in}{0.765626in}}%
\pgfpathlineto{\pgfqpoint{3.918721in}{0.765606in}}%
\pgfpathlineto{\pgfqpoint{3.919017in}{0.765587in}}%
\pgfpathlineto{\pgfqpoint{3.919313in}{0.765568in}}%
\pgfpathlineto{\pgfqpoint{3.919609in}{0.765548in}}%
\pgfpathlineto{\pgfqpoint{3.919905in}{0.765529in}}%
\pgfpathlineto{\pgfqpoint{3.920201in}{0.765510in}}%
\pgfpathlineto{\pgfqpoint{3.920497in}{0.765490in}}%
\pgfpathlineto{\pgfqpoint{3.920793in}{0.765471in}}%
\pgfpathlineto{\pgfqpoint{3.921089in}{0.765452in}}%
\pgfpathlineto{\pgfqpoint{3.921385in}{0.765433in}}%
\pgfpathlineto{\pgfqpoint{3.921681in}{0.765413in}}%
\pgfpathlineto{\pgfqpoint{3.921977in}{0.765394in}}%
\pgfpathlineto{\pgfqpoint{3.922273in}{0.765375in}}%
\pgfpathlineto{\pgfqpoint{3.922569in}{0.765355in}}%
\pgfpathlineto{\pgfqpoint{3.922865in}{0.765336in}}%
\pgfpathlineto{\pgfqpoint{3.923161in}{0.765317in}}%
\pgfpathlineto{\pgfqpoint{3.923457in}{0.765298in}}%
\pgfpathlineto{\pgfqpoint{3.923753in}{0.765278in}}%
\pgfpathlineto{\pgfqpoint{3.924049in}{0.765259in}}%
\pgfpathlineto{\pgfqpoint{3.924345in}{0.765240in}}%
\pgfpathlineto{\pgfqpoint{3.924641in}{0.765220in}}%
\pgfpathlineto{\pgfqpoint{3.924937in}{0.765201in}}%
\pgfpathlineto{\pgfqpoint{3.925233in}{0.765182in}}%
\pgfpathlineto{\pgfqpoint{3.925529in}{0.765162in}}%
\pgfpathlineto{\pgfqpoint{3.925825in}{0.765143in}}%
\pgfpathlineto{\pgfqpoint{3.926121in}{0.765124in}}%
\pgfpathlineto{\pgfqpoint{3.926417in}{0.765105in}}%
\pgfpathlineto{\pgfqpoint{3.926713in}{0.765085in}}%
\pgfpathlineto{\pgfqpoint{3.927009in}{0.765066in}}%
\pgfpathlineto{\pgfqpoint{3.927305in}{0.765047in}}%
\pgfpathlineto{\pgfqpoint{3.927601in}{0.765027in}}%
\pgfpathlineto{\pgfqpoint{3.927897in}{0.765008in}}%
\pgfpathlineto{\pgfqpoint{3.928193in}{0.764989in}}%
\pgfpathlineto{\pgfqpoint{3.928489in}{0.764970in}}%
\pgfpathlineto{\pgfqpoint{3.928785in}{0.764950in}}%
\pgfpathlineto{\pgfqpoint{3.929081in}{0.764931in}}%
\pgfpathlineto{\pgfqpoint{3.929377in}{0.764912in}}%
\pgfpathlineto{\pgfqpoint{3.929673in}{0.764892in}}%
\pgfpathlineto{\pgfqpoint{3.929969in}{0.764873in}}%
\pgfpathlineto{\pgfqpoint{3.930265in}{0.764846in}}%
\pgfpathlineto{\pgfqpoint{3.930561in}{0.764077in}}%
\pgfpathlineto{\pgfqpoint{3.930857in}{0.763833in}}%
\pgfpathlineto{\pgfqpoint{3.931153in}{0.763831in}}%
\pgfpathlineto{\pgfqpoint{3.931449in}{0.763829in}}%
\pgfpathlineto{\pgfqpoint{3.931745in}{0.763827in}}%
\pgfpathlineto{\pgfqpoint{3.932041in}{0.763825in}}%
\pgfpathlineto{\pgfqpoint{3.932337in}{0.763823in}}%
\pgfpathlineto{\pgfqpoint{3.932633in}{0.763821in}}%
\pgfpathlineto{\pgfqpoint{3.932929in}{0.763819in}}%
\pgfpathlineto{\pgfqpoint{3.933225in}{0.763817in}}%
\pgfpathlineto{\pgfqpoint{3.933521in}{0.763815in}}%
\pgfpathlineto{\pgfqpoint{3.933817in}{0.763813in}}%
\pgfpathlineto{\pgfqpoint{3.934113in}{0.763811in}}%
\pgfpathlineto{\pgfqpoint{3.934409in}{0.763809in}}%
\pgfpathlineto{\pgfqpoint{3.934705in}{0.763807in}}%
\pgfpathlineto{\pgfqpoint{3.935001in}{0.763805in}}%
\pgfpathlineto{\pgfqpoint{3.935297in}{0.763803in}}%
\pgfpathlineto{\pgfqpoint{3.935593in}{0.763801in}}%
\pgfpathlineto{\pgfqpoint{3.935889in}{0.763799in}}%
\pgfpathlineto{\pgfqpoint{3.936185in}{0.763797in}}%
\pgfpathlineto{\pgfqpoint{3.936481in}{0.763795in}}%
\pgfpathlineto{\pgfqpoint{3.936777in}{0.763792in}}%
\pgfpathlineto{\pgfqpoint{3.937073in}{0.763790in}}%
\pgfpathlineto{\pgfqpoint{3.937369in}{0.763788in}}%
\pgfpathlineto{\pgfqpoint{3.937665in}{0.763786in}}%
\pgfpathlineto{\pgfqpoint{3.937961in}{0.763784in}}%
\pgfpathlineto{\pgfqpoint{3.938257in}{0.763782in}}%
\pgfpathlineto{\pgfqpoint{3.938553in}{0.763780in}}%
\pgfpathlineto{\pgfqpoint{3.938849in}{0.763778in}}%
\pgfpathlineto{\pgfqpoint{3.939145in}{0.763776in}}%
\pgfpathlineto{\pgfqpoint{3.939441in}{0.763774in}}%
\pgfpathlineto{\pgfqpoint{3.939737in}{0.763772in}}%
\pgfpathlineto{\pgfqpoint{3.940033in}{0.763770in}}%
\pgfpathlineto{\pgfqpoint{3.940329in}{0.763768in}}%
\pgfpathlineto{\pgfqpoint{3.940625in}{0.763766in}}%
\pgfpathlineto{\pgfqpoint{3.940921in}{0.763764in}}%
\pgfpathlineto{\pgfqpoint{3.941217in}{0.763762in}}%
\pgfpathlineto{\pgfqpoint{3.941513in}{0.763760in}}%
\pgfpathlineto{\pgfqpoint{3.941809in}{0.763758in}}%
\pgfpathlineto{\pgfqpoint{3.942105in}{0.763756in}}%
\pgfpathlineto{\pgfqpoint{3.942401in}{0.763754in}}%
\pgfpathlineto{\pgfqpoint{3.942697in}{0.763752in}}%
\pgfpathlineto{\pgfqpoint{3.942993in}{0.763750in}}%
\pgfpathlineto{\pgfqpoint{3.943289in}{0.763747in}}%
\pgfpathlineto{\pgfqpoint{3.943585in}{0.763745in}}%
\pgfpathlineto{\pgfqpoint{3.943881in}{0.763743in}}%
\pgfpathlineto{\pgfqpoint{3.944177in}{0.763741in}}%
\pgfpathlineto{\pgfqpoint{3.944473in}{0.763739in}}%
\pgfpathlineto{\pgfqpoint{3.944769in}{0.763737in}}%
\pgfpathlineto{\pgfqpoint{3.945065in}{0.763735in}}%
\pgfpathlineto{\pgfqpoint{3.945361in}{0.763733in}}%
\pgfpathlineto{\pgfqpoint{3.945657in}{0.763731in}}%
\pgfpathlineto{\pgfqpoint{3.945953in}{0.763729in}}%
\pgfpathlineto{\pgfqpoint{3.946249in}{0.763727in}}%
\pgfpathlineto{\pgfqpoint{3.946545in}{0.763725in}}%
\pgfpathlineto{\pgfqpoint{3.946841in}{0.763723in}}%
\pgfpathlineto{\pgfqpoint{3.947137in}{0.763721in}}%
\pgfpathlineto{\pgfqpoint{3.947433in}{0.763719in}}%
\pgfpathlineto{\pgfqpoint{3.947729in}{0.763717in}}%
\pgfpathlineto{\pgfqpoint{3.948025in}{0.763715in}}%
\pgfpathlineto{\pgfqpoint{3.948321in}{0.763713in}}%
\pgfpathlineto{\pgfqpoint{3.948617in}{0.763711in}}%
\pgfpathlineto{\pgfqpoint{3.948913in}{0.763709in}}%
\pgfpathlineto{\pgfqpoint{3.949209in}{0.763707in}}%
\pgfpathlineto{\pgfqpoint{3.949505in}{0.763705in}}%
\pgfpathlineto{\pgfqpoint{3.949801in}{0.763702in}}%
\pgfpathlineto{\pgfqpoint{3.950097in}{0.763700in}}%
\pgfpathlineto{\pgfqpoint{3.950393in}{0.763698in}}%
\pgfpathlineto{\pgfqpoint{3.950689in}{0.763760in}}%
\pgfpathlineto{\pgfqpoint{3.950985in}{0.765922in}}%
\pgfpathlineto{\pgfqpoint{3.951281in}{0.769114in}}%
\pgfpathlineto{\pgfqpoint{3.951578in}{0.770840in}}%
\pgfpathlineto{\pgfqpoint{3.951874in}{0.770842in}}%
\pgfpathlineto{\pgfqpoint{3.952170in}{0.770841in}}%
\pgfpathlineto{\pgfqpoint{3.952466in}{0.770840in}}%
\pgfpathlineto{\pgfqpoint{3.952762in}{0.770840in}}%
\pgfpathlineto{\pgfqpoint{3.953058in}{0.770839in}}%
\pgfpathlineto{\pgfqpoint{3.953354in}{0.770838in}}%
\pgfpathlineto{\pgfqpoint{3.953650in}{0.770838in}}%
\pgfpathlineto{\pgfqpoint{3.953946in}{0.770837in}}%
\pgfpathlineto{\pgfqpoint{3.954242in}{0.770837in}}%
\pgfpathlineto{\pgfqpoint{3.954538in}{0.770836in}}%
\pgfpathlineto{\pgfqpoint{3.954834in}{0.770835in}}%
\pgfpathlineto{\pgfqpoint{3.955130in}{0.770835in}}%
\pgfpathlineto{\pgfqpoint{3.955426in}{0.770834in}}%
\pgfpathlineto{\pgfqpoint{3.955722in}{0.770833in}}%
\pgfpathlineto{\pgfqpoint{3.956018in}{0.770833in}}%
\pgfpathlineto{\pgfqpoint{3.956314in}{0.770832in}}%
\pgfpathlineto{\pgfqpoint{3.956610in}{0.770831in}}%
\pgfpathlineto{\pgfqpoint{3.956906in}{0.770831in}}%
\pgfpathlineto{\pgfqpoint{3.957202in}{0.770830in}}%
\pgfpathlineto{\pgfqpoint{3.957498in}{0.770830in}}%
\pgfpathlineto{\pgfqpoint{3.957794in}{0.770829in}}%
\pgfpathlineto{\pgfqpoint{3.958090in}{0.770828in}}%
\pgfpathlineto{\pgfqpoint{3.958386in}{0.770828in}}%
\pgfpathlineto{\pgfqpoint{3.958682in}{0.770827in}}%
\pgfpathlineto{\pgfqpoint{3.958978in}{0.770826in}}%
\pgfpathlineto{\pgfqpoint{3.959274in}{0.770826in}}%
\pgfpathlineto{\pgfqpoint{3.959570in}{0.770825in}}%
\pgfpathlineto{\pgfqpoint{3.959866in}{0.770824in}}%
\pgfpathlineto{\pgfqpoint{3.960162in}{0.770824in}}%
\pgfpathlineto{\pgfqpoint{3.960458in}{0.770823in}}%
\pgfpathlineto{\pgfqpoint{3.960754in}{0.770823in}}%
\pgfpathlineto{\pgfqpoint{3.961050in}{0.770822in}}%
\pgfpathlineto{\pgfqpoint{3.961346in}{0.770821in}}%
\pgfpathlineto{\pgfqpoint{3.961642in}{0.770821in}}%
\pgfpathlineto{\pgfqpoint{3.961938in}{0.770820in}}%
\pgfpathlineto{\pgfqpoint{3.962234in}{0.770819in}}%
\pgfpathlineto{\pgfqpoint{3.962530in}{0.770819in}}%
\pgfpathlineto{\pgfqpoint{3.962826in}{0.770818in}}%
\pgfpathlineto{\pgfqpoint{3.963122in}{0.770817in}}%
\pgfpathlineto{\pgfqpoint{3.963418in}{0.770817in}}%
\pgfpathlineto{\pgfqpoint{3.963714in}{0.770816in}}%
\pgfpathlineto{\pgfqpoint{3.964010in}{0.770815in}}%
\pgfpathlineto{\pgfqpoint{3.964306in}{0.770815in}}%
\pgfpathlineto{\pgfqpoint{3.964602in}{0.770814in}}%
\pgfpathlineto{\pgfqpoint{3.964898in}{0.770814in}}%
\pgfpathlineto{\pgfqpoint{3.965194in}{0.770813in}}%
\pgfpathlineto{\pgfqpoint{3.965490in}{0.770812in}}%
\pgfpathlineto{\pgfqpoint{3.965786in}{0.770812in}}%
\pgfpathlineto{\pgfqpoint{3.966082in}{0.770811in}}%
\pgfpathlineto{\pgfqpoint{3.966378in}{0.770810in}}%
\pgfpathlineto{\pgfqpoint{3.966674in}{0.770810in}}%
\pgfpathlineto{\pgfqpoint{3.966970in}{0.770809in}}%
\pgfpathlineto{\pgfqpoint{3.967266in}{0.770808in}}%
\pgfpathlineto{\pgfqpoint{3.967562in}{0.770808in}}%
\pgfpathlineto{\pgfqpoint{3.967858in}{0.770807in}}%
\pgfpathlineto{\pgfqpoint{3.968154in}{0.770807in}}%
\pgfpathlineto{\pgfqpoint{3.968450in}{0.770806in}}%
\pgfpathlineto{\pgfqpoint{3.968746in}{0.770805in}}%
\pgfpathlineto{\pgfqpoint{3.969042in}{0.770805in}}%
\pgfpathlineto{\pgfqpoint{3.969338in}{0.770804in}}%
\pgfpathlineto{\pgfqpoint{3.969634in}{0.770803in}}%
\pgfpathlineto{\pgfqpoint{3.969930in}{0.770803in}}%
\pgfpathlineto{\pgfqpoint{3.970226in}{0.770802in}}%
\pgfpathlineto{\pgfqpoint{3.970522in}{0.770801in}}%
\pgfpathlineto{\pgfqpoint{3.970818in}{0.770801in}}%
\pgfpathlineto{\pgfqpoint{3.971114in}{0.770800in}}%
\pgfpathlineto{\pgfqpoint{3.971410in}{0.770800in}}%
\pgfpathlineto{\pgfqpoint{3.971706in}{0.770799in}}%
\pgfpathlineto{\pgfqpoint{3.972002in}{0.770561in}}%
\pgfpathlineto{\pgfqpoint{3.972298in}{0.769319in}}%
\pgfpathlineto{\pgfqpoint{3.972594in}{0.767964in}}%
\pgfpathlineto{\pgfqpoint{3.972890in}{0.766610in}}%
\pgfpathlineto{\pgfqpoint{3.973186in}{0.765272in}}%
\pgfpathlineto{\pgfqpoint{3.973482in}{0.764792in}}%
\pgfpathlineto{\pgfqpoint{3.973778in}{0.764791in}}%
\pgfpathlineto{\pgfqpoint{3.974074in}{0.764791in}}%
\pgfpathlineto{\pgfqpoint{3.974370in}{0.764791in}}%
\pgfpathlineto{\pgfqpoint{3.974666in}{0.764791in}}%
\pgfpathlineto{\pgfqpoint{3.974962in}{0.764791in}}%
\pgfpathlineto{\pgfqpoint{3.975258in}{0.764790in}}%
\pgfpathlineto{\pgfqpoint{3.975554in}{0.764790in}}%
\pgfpathlineto{\pgfqpoint{3.975850in}{0.764790in}}%
\pgfpathlineto{\pgfqpoint{3.976146in}{0.764790in}}%
\pgfpathlineto{\pgfqpoint{3.976442in}{0.764790in}}%
\pgfpathlineto{\pgfqpoint{3.976738in}{0.764789in}}%
\pgfpathlineto{\pgfqpoint{3.977034in}{0.764789in}}%
\pgfpathlineto{\pgfqpoint{3.977330in}{0.764789in}}%
\pgfpathlineto{\pgfqpoint{3.977626in}{0.764789in}}%
\pgfpathlineto{\pgfqpoint{3.977922in}{0.764789in}}%
\pgfpathlineto{\pgfqpoint{3.978218in}{0.764788in}}%
\pgfpathlineto{\pgfqpoint{3.978514in}{0.764788in}}%
\pgfpathlineto{\pgfqpoint{3.978810in}{0.764788in}}%
\pgfpathlineto{\pgfqpoint{3.979106in}{0.764788in}}%
\pgfpathlineto{\pgfqpoint{3.979402in}{0.764788in}}%
\pgfpathlineto{\pgfqpoint{3.979698in}{0.764787in}}%
\pgfpathlineto{\pgfqpoint{3.979994in}{0.764786in}}%
\pgfpathlineto{\pgfqpoint{3.980290in}{0.764784in}}%
\pgfpathlineto{\pgfqpoint{3.980586in}{0.764783in}}%
\pgfpathlineto{\pgfqpoint{3.980882in}{0.764781in}}%
\pgfpathlineto{\pgfqpoint{3.981178in}{0.764780in}}%
\pgfpathlineto{\pgfqpoint{3.981474in}{0.764778in}}%
\pgfpathlineto{\pgfqpoint{3.981770in}{0.764776in}}%
\pgfpathlineto{\pgfqpoint{3.982066in}{0.764775in}}%
\pgfpathlineto{\pgfqpoint{3.982362in}{0.764773in}}%
\pgfpathlineto{\pgfqpoint{3.982658in}{0.764772in}}%
\pgfpathlineto{\pgfqpoint{3.982954in}{0.764770in}}%
\pgfpathlineto{\pgfqpoint{3.983250in}{0.764768in}}%
\pgfpathlineto{\pgfqpoint{3.983546in}{0.764767in}}%
\pgfpathlineto{\pgfqpoint{3.983842in}{0.764765in}}%
\pgfpathlineto{\pgfqpoint{3.984138in}{0.764764in}}%
\pgfpathlineto{\pgfqpoint{3.984434in}{0.764762in}}%
\pgfpathlineto{\pgfqpoint{3.984730in}{0.764760in}}%
\pgfpathlineto{\pgfqpoint{3.985026in}{0.764759in}}%
\pgfpathlineto{\pgfqpoint{3.985322in}{0.764757in}}%
\pgfpathlineto{\pgfqpoint{3.985618in}{0.764755in}}%
\pgfpathlineto{\pgfqpoint{3.985914in}{0.764754in}}%
\pgfpathlineto{\pgfqpoint{3.986210in}{0.764752in}}%
\pgfpathlineto{\pgfqpoint{3.986506in}{0.764751in}}%
\pgfpathlineto{\pgfqpoint{3.986802in}{0.764749in}}%
\pgfpathlineto{\pgfqpoint{3.987098in}{0.764747in}}%
\pgfpathlineto{\pgfqpoint{3.987394in}{0.764746in}}%
\pgfpathlineto{\pgfqpoint{3.987690in}{0.764744in}}%
\pgfpathlineto{\pgfqpoint{3.987986in}{0.764743in}}%
\pgfpathlineto{\pgfqpoint{3.988282in}{0.764741in}}%
\pgfpathlineto{\pgfqpoint{3.988578in}{0.764739in}}%
\pgfpathlineto{\pgfqpoint{3.988874in}{0.764738in}}%
\pgfpathlineto{\pgfqpoint{3.989170in}{0.764736in}}%
\pgfpathlineto{\pgfqpoint{3.989466in}{0.764735in}}%
\pgfpathlineto{\pgfqpoint{3.989762in}{0.764733in}}%
\pgfpathlineto{\pgfqpoint{3.990058in}{0.764731in}}%
\pgfpathlineto{\pgfqpoint{3.990354in}{0.764730in}}%
\pgfpathlineto{\pgfqpoint{3.990650in}{0.764728in}}%
\pgfpathlineto{\pgfqpoint{3.990946in}{0.764727in}}%
\pgfpathlineto{\pgfqpoint{3.991242in}{0.764725in}}%
\pgfpathlineto{\pgfqpoint{3.991538in}{0.764723in}}%
\pgfpathlineto{\pgfqpoint{3.991834in}{0.764722in}}%
\pgfpathlineto{\pgfqpoint{3.992130in}{0.764720in}}%
\pgfpathlineto{\pgfqpoint{3.992426in}{0.764719in}}%
\pgfpathlineto{\pgfqpoint{3.992722in}{0.764717in}}%
\pgfpathlineto{\pgfqpoint{3.993018in}{0.764715in}}%
\pgfpathlineto{\pgfqpoint{3.993314in}{0.764714in}}%
\pgfpathlineto{\pgfqpoint{3.993610in}{0.764712in}}%
\pgfpathlineto{\pgfqpoint{3.993906in}{0.764711in}}%
\pgfpathlineto{\pgfqpoint{3.994202in}{0.764709in}}%
\pgfpathlineto{\pgfqpoint{3.994498in}{0.764707in}}%
\pgfpathlineto{\pgfqpoint{3.994794in}{0.764706in}}%
\pgfpathlineto{\pgfqpoint{3.995090in}{0.764704in}}%
\pgfpathlineto{\pgfqpoint{3.995386in}{0.764703in}}%
\pgfpathlineto{\pgfqpoint{3.995682in}{0.764701in}}%
\pgfpathlineto{\pgfqpoint{3.995978in}{0.764699in}}%
\pgfpathlineto{\pgfqpoint{3.996274in}{0.764698in}}%
\pgfpathlineto{\pgfqpoint{3.996570in}{0.764696in}}%
\pgfpathlineto{\pgfqpoint{3.996866in}{0.764695in}}%
\pgfpathlineto{\pgfqpoint{3.997162in}{0.764693in}}%
\pgfpathlineto{\pgfqpoint{3.997458in}{0.764691in}}%
\pgfpathlineto{\pgfqpoint{3.997754in}{0.764690in}}%
\pgfpathlineto{\pgfqpoint{3.998050in}{0.764688in}}%
\pgfpathlineto{\pgfqpoint{3.998346in}{0.764687in}}%
\pgfpathlineto{\pgfqpoint{3.998642in}{0.764685in}}%
\pgfpathlineto{\pgfqpoint{3.998938in}{0.764683in}}%
\pgfpathlineto{\pgfqpoint{3.999234in}{0.764682in}}%
\pgfpathlineto{\pgfqpoint{3.999530in}{0.764682in}}%
\pgfpathlineto{\pgfqpoint{3.999826in}{0.764681in}}%
\pgfpathlineto{\pgfqpoint{4.000122in}{0.764681in}}%
\pgfpathlineto{\pgfqpoint{4.000418in}{0.764672in}}%
\pgfpathlineto{\pgfqpoint{4.000714in}{0.764608in}}%
\pgfpathlineto{\pgfqpoint{4.001010in}{0.764534in}}%
\pgfpathlineto{\pgfqpoint{4.001306in}{0.764460in}}%
\pgfpathlineto{\pgfqpoint{4.001602in}{0.764385in}}%
\pgfpathlineto{\pgfqpoint{4.001898in}{0.764311in}}%
\pgfpathlineto{\pgfqpoint{4.002194in}{0.764237in}}%
\pgfpathlineto{\pgfqpoint{4.002490in}{0.764162in}}%
\pgfpathlineto{\pgfqpoint{4.002786in}{0.764088in}}%
\pgfpathlineto{\pgfqpoint{4.003082in}{0.764014in}}%
\pgfpathlineto{\pgfqpoint{4.003378in}{0.763940in}}%
\pgfpathlineto{\pgfqpoint{4.003674in}{0.763865in}}%
\pgfpathlineto{\pgfqpoint{4.003970in}{0.763791in}}%
\pgfpathlineto{\pgfqpoint{4.004266in}{0.763717in}}%
\pgfpathlineto{\pgfqpoint{4.004562in}{0.763643in}}%
\pgfpathlineto{\pgfqpoint{4.004858in}{0.763568in}}%
\pgfpathlineto{\pgfqpoint{4.005154in}{0.763494in}}%
\pgfpathlineto{\pgfqpoint{4.005450in}{0.763420in}}%
\pgfpathlineto{\pgfqpoint{4.005746in}{0.763735in}}%
\pgfpathlineto{\pgfqpoint{4.006042in}{0.766308in}}%
\pgfpathlineto{\pgfqpoint{4.006338in}{0.769242in}}%
\pgfpathlineto{\pgfqpoint{4.006634in}{0.770439in}}%
\pgfpathlineto{\pgfqpoint{4.006930in}{0.770160in}}%
\pgfpathlineto{\pgfqpoint{4.007226in}{0.769880in}}%
\pgfpathlineto{\pgfqpoint{4.007522in}{0.769601in}}%
\pgfpathlineto{\pgfqpoint{4.007818in}{0.769321in}}%
\pgfpathlineto{\pgfqpoint{4.008114in}{0.769041in}}%
\pgfpathlineto{\pgfqpoint{4.008410in}{0.768762in}}%
\pgfpathlineto{\pgfqpoint{4.008706in}{0.768482in}}%
\pgfpathlineto{\pgfqpoint{4.009002in}{0.768203in}}%
\pgfpathlineto{\pgfqpoint{4.009298in}{0.767923in}}%
\pgfpathlineto{\pgfqpoint{4.009594in}{0.767644in}}%
\pgfpathlineto{\pgfqpoint{4.009890in}{0.767364in}}%
\pgfpathlineto{\pgfqpoint{4.010186in}{0.767085in}}%
\pgfpathlineto{\pgfqpoint{4.010482in}{0.766805in}}%
\pgfpathlineto{\pgfqpoint{4.010778in}{0.766525in}}%
\pgfpathlineto{\pgfqpoint{4.011074in}{0.766246in}}%
\pgfpathlineto{\pgfqpoint{4.011370in}{0.765966in}}%
\pgfpathlineto{\pgfqpoint{4.011666in}{0.765687in}}%
\pgfpathlineto{\pgfqpoint{4.011962in}{0.765407in}}%
\pgfpathlineto{\pgfqpoint{4.012258in}{0.765128in}}%
\pgfpathlineto{\pgfqpoint{4.012554in}{0.764848in}}%
\pgfpathlineto{\pgfqpoint{4.012850in}{0.764659in}}%
\pgfpathlineto{\pgfqpoint{4.013146in}{0.764603in}}%
\pgfpathlineto{\pgfqpoint{4.013442in}{0.764548in}}%
\pgfpathlineto{\pgfqpoint{4.013738in}{0.764493in}}%
\pgfpathlineto{\pgfqpoint{4.014034in}{0.764438in}}%
\pgfpathlineto{\pgfqpoint{4.014330in}{0.764383in}}%
\pgfpathlineto{\pgfqpoint{4.014626in}{0.764329in}}%
\pgfpathlineto{\pgfqpoint{4.014922in}{0.764236in}}%
\pgfpathlineto{\pgfqpoint{4.015218in}{0.764092in}}%
\pgfpathlineto{\pgfqpoint{4.015514in}{0.763948in}}%
\pgfpathlineto{\pgfqpoint{4.015810in}{0.763803in}}%
\pgfpathlineto{\pgfqpoint{4.016106in}{0.763659in}}%
\pgfpathlineto{\pgfqpoint{4.016402in}{0.763515in}}%
\pgfpathlineto{\pgfqpoint{4.016698in}{0.763371in}}%
\pgfpathlineto{\pgfqpoint{4.016994in}{0.763227in}}%
\pgfpathlineto{\pgfqpoint{4.017290in}{0.763083in}}%
\pgfpathlineto{\pgfqpoint{4.017586in}{0.762938in}}%
\pgfpathlineto{\pgfqpoint{4.017882in}{0.762794in}}%
\pgfpathlineto{\pgfqpoint{4.018178in}{0.762650in}}%
\pgfpathlineto{\pgfqpoint{4.018474in}{0.762506in}}%
\pgfpathlineto{\pgfqpoint{4.018771in}{0.762362in}}%
\pgfpathlineto{\pgfqpoint{4.019067in}{0.762218in}}%
\pgfpathlineto{\pgfqpoint{4.019363in}{0.762073in}}%
\pgfpathlineto{\pgfqpoint{4.019659in}{0.761929in}}%
\pgfpathlineto{\pgfqpoint{4.019955in}{0.761785in}}%
\pgfpathlineto{\pgfqpoint{4.020251in}{0.761641in}}%
\pgfpathlineto{\pgfqpoint{4.020547in}{0.761773in}}%
\pgfpathlineto{\pgfqpoint{4.020843in}{0.762231in}}%
\pgfpathlineto{\pgfqpoint{4.021139in}{0.762631in}}%
\pgfpathlineto{\pgfqpoint{4.021435in}{0.762620in}}%
\pgfpathlineto{\pgfqpoint{4.021731in}{0.762529in}}%
\pgfpathlineto{\pgfqpoint{4.022027in}{0.762472in}}%
\pgfpathlineto{\pgfqpoint{4.022323in}{0.762458in}}%
\pgfpathlineto{\pgfqpoint{4.022619in}{0.762446in}}%
\pgfpathlineto{\pgfqpoint{4.022915in}{0.762401in}}%
\pgfpathlineto{\pgfqpoint{4.023211in}{0.761747in}}%
\pgfpathlineto{\pgfqpoint{4.023507in}{0.761733in}}%
\pgfpathlineto{\pgfqpoint{4.023803in}{0.761798in}}%
\pgfpathlineto{\pgfqpoint{4.024099in}{0.761864in}}%
\pgfpathlineto{\pgfqpoint{4.024395in}{0.761929in}}%
\pgfpathlineto{\pgfqpoint{4.024691in}{0.761994in}}%
\pgfpathlineto{\pgfqpoint{4.024987in}{0.762059in}}%
\pgfpathlineto{\pgfqpoint{4.025283in}{0.762124in}}%
\pgfpathlineto{\pgfqpoint{4.025579in}{0.762189in}}%
\pgfpathlineto{\pgfqpoint{4.025875in}{0.762254in}}%
\pgfpathlineto{\pgfqpoint{4.026171in}{0.762320in}}%
\pgfpathlineto{\pgfqpoint{4.026467in}{0.762385in}}%
\pgfpathlineto{\pgfqpoint{4.026763in}{0.762450in}}%
\pgfpathlineto{\pgfqpoint{4.027059in}{0.762515in}}%
\pgfpathlineto{\pgfqpoint{4.027355in}{0.762580in}}%
\pgfpathlineto{\pgfqpoint{4.027651in}{0.762645in}}%
\pgfpathlineto{\pgfqpoint{4.027947in}{0.762710in}}%
\pgfpathlineto{\pgfqpoint{4.028243in}{0.762776in}}%
\pgfpathlineto{\pgfqpoint{4.028539in}{0.762841in}}%
\pgfpathlineto{\pgfqpoint{4.028835in}{0.762906in}}%
\pgfpathlineto{\pgfqpoint{4.029131in}{0.762971in}}%
\pgfpathlineto{\pgfqpoint{4.029427in}{0.763036in}}%
\pgfpathlineto{\pgfqpoint{4.029723in}{0.762702in}}%
\pgfpathlineto{\pgfqpoint{4.030019in}{0.762679in}}%
\pgfpathlineto{\pgfqpoint{4.030315in}{0.762758in}}%
\pgfpathlineto{\pgfqpoint{4.030611in}{0.762836in}}%
\pgfpathlineto{\pgfqpoint{4.030907in}{0.762914in}}%
\pgfpathlineto{\pgfqpoint{4.031203in}{0.762993in}}%
\pgfpathlineto{\pgfqpoint{4.031499in}{0.763071in}}%
\pgfpathlineto{\pgfqpoint{4.031795in}{0.763149in}}%
\pgfpathlineto{\pgfqpoint{4.032091in}{0.763228in}}%
\pgfpathlineto{\pgfqpoint{4.032387in}{0.763306in}}%
\pgfpathlineto{\pgfqpoint{4.032683in}{0.763384in}}%
\pgfpathlineto{\pgfqpoint{4.032979in}{0.763463in}}%
\pgfpathlineto{\pgfqpoint{4.033275in}{0.763541in}}%
\pgfpathlineto{\pgfqpoint{4.033571in}{0.763619in}}%
\pgfpathlineto{\pgfqpoint{4.033867in}{0.763698in}}%
\pgfpathlineto{\pgfqpoint{4.034163in}{0.763776in}}%
\pgfpathlineto{\pgfqpoint{4.034459in}{0.763854in}}%
\pgfpathlineto{\pgfqpoint{4.034755in}{0.763933in}}%
\pgfpathlineto{\pgfqpoint{4.035051in}{0.764011in}}%
\pgfpathlineto{\pgfqpoint{4.035347in}{0.764090in}}%
\pgfpathlineto{\pgfqpoint{4.035643in}{0.764168in}}%
\pgfpathlineto{\pgfqpoint{4.035939in}{0.764246in}}%
\pgfpathlineto{\pgfqpoint{4.036235in}{0.764325in}}%
\pgfpathlineto{\pgfqpoint{4.036531in}{0.764403in}}%
\pgfpathlineto{\pgfqpoint{4.036827in}{0.764481in}}%
\pgfpathlineto{\pgfqpoint{4.037123in}{0.764560in}}%
\pgfpathlineto{\pgfqpoint{4.037419in}{0.764638in}}%
\pgfpathlineto{\pgfqpoint{4.037715in}{0.764716in}}%
\pgfpathlineto{\pgfqpoint{4.038011in}{0.764795in}}%
\pgfpathlineto{\pgfqpoint{4.038307in}{0.764873in}}%
\pgfpathlineto{\pgfqpoint{4.038603in}{0.764951in}}%
\pgfpathlineto{\pgfqpoint{4.038899in}{0.765030in}}%
\pgfpathlineto{\pgfqpoint{4.039195in}{0.765108in}}%
\pgfpathlineto{\pgfqpoint{4.039491in}{0.765186in}}%
\pgfpathlineto{\pgfqpoint{4.039787in}{0.765265in}}%
\pgfpathlineto{\pgfqpoint{4.040083in}{0.765343in}}%
\pgfpathlineto{\pgfqpoint{4.040379in}{0.765421in}}%
\pgfpathlineto{\pgfqpoint{4.040675in}{0.765500in}}%
\pgfpathlineto{\pgfqpoint{4.040971in}{0.765578in}}%
\pgfpathlineto{\pgfqpoint{4.041267in}{0.765657in}}%
\pgfpathlineto{\pgfqpoint{4.041563in}{0.765735in}}%
\pgfpathlineto{\pgfqpoint{4.041859in}{0.765813in}}%
\pgfpathlineto{\pgfqpoint{4.042155in}{0.765892in}}%
\pgfpathlineto{\pgfqpoint{4.042451in}{0.765970in}}%
\pgfpathlineto{\pgfqpoint{4.042747in}{0.766048in}}%
\pgfpathlineto{\pgfqpoint{4.043043in}{0.766127in}}%
\pgfpathlineto{\pgfqpoint{4.043339in}{0.766205in}}%
\pgfpathlineto{\pgfqpoint{4.043635in}{0.766283in}}%
\pgfpathlineto{\pgfqpoint{4.043931in}{0.766362in}}%
\pgfpathlineto{\pgfqpoint{4.044227in}{0.766440in}}%
\pgfpathlineto{\pgfqpoint{4.044523in}{0.766518in}}%
\pgfpathlineto{\pgfqpoint{4.044819in}{0.766597in}}%
\pgfpathlineto{\pgfqpoint{4.045115in}{0.766675in}}%
\pgfpathlineto{\pgfqpoint{4.045411in}{0.766753in}}%
\pgfpathlineto{\pgfqpoint{4.045707in}{0.766832in}}%
\pgfpathlineto{\pgfqpoint{4.046003in}{0.766910in}}%
\pgfpathlineto{\pgfqpoint{4.046299in}{0.766988in}}%
\pgfpathlineto{\pgfqpoint{4.046595in}{0.767067in}}%
\pgfpathlineto{\pgfqpoint{4.046891in}{0.767145in}}%
\pgfpathlineto{\pgfqpoint{4.047187in}{0.767223in}}%
\pgfpathlineto{\pgfqpoint{4.047483in}{0.767302in}}%
\pgfpathlineto{\pgfqpoint{4.047779in}{0.767380in}}%
\pgfpathlineto{\pgfqpoint{4.048075in}{0.767459in}}%
\pgfpathlineto{\pgfqpoint{4.048371in}{0.767537in}}%
\pgfpathlineto{\pgfqpoint{4.048667in}{0.767615in}}%
\pgfpathlineto{\pgfqpoint{4.048963in}{0.767673in}}%
\pgfpathlineto{\pgfqpoint{4.049259in}{0.767596in}}%
\pgfpathlineto{\pgfqpoint{4.049555in}{0.767494in}}%
\pgfpathlineto{\pgfqpoint{4.049851in}{0.767393in}}%
\pgfpathlineto{\pgfqpoint{4.050147in}{0.767291in}}%
\pgfpathlineto{\pgfqpoint{4.050443in}{0.767189in}}%
\pgfpathlineto{\pgfqpoint{4.050739in}{0.767087in}}%
\pgfpathlineto{\pgfqpoint{4.051035in}{0.766985in}}%
\pgfpathlineto{\pgfqpoint{4.051331in}{0.766883in}}%
\pgfpathlineto{\pgfqpoint{4.051627in}{0.766781in}}%
\pgfpathlineto{\pgfqpoint{4.051923in}{0.766680in}}%
\pgfpathlineto{\pgfqpoint{4.052219in}{0.766578in}}%
\pgfpathlineto{\pgfqpoint{4.052515in}{0.766476in}}%
\pgfpathlineto{\pgfqpoint{4.052811in}{0.766374in}}%
\pgfpathlineto{\pgfqpoint{4.053107in}{0.766272in}}%
\pgfpathlineto{\pgfqpoint{4.053403in}{0.766170in}}%
\pgfpathlineto{\pgfqpoint{4.053699in}{0.766069in}}%
\pgfpathlineto{\pgfqpoint{4.053995in}{0.765967in}}%
\pgfpathlineto{\pgfqpoint{4.054291in}{0.765865in}}%
\pgfpathlineto{\pgfqpoint{4.054587in}{0.765763in}}%
\pgfpathlineto{\pgfqpoint{4.054883in}{0.765661in}}%
\pgfpathlineto{\pgfqpoint{4.055179in}{0.765559in}}%
\pgfpathlineto{\pgfqpoint{4.055475in}{0.765458in}}%
\pgfpathlineto{\pgfqpoint{4.055771in}{0.765356in}}%
\pgfpathlineto{\pgfqpoint{4.056067in}{0.765254in}}%
\pgfpathlineto{\pgfqpoint{4.056363in}{0.765152in}}%
\pgfpathlineto{\pgfqpoint{4.056659in}{0.765050in}}%
\pgfpathlineto{\pgfqpoint{4.056955in}{0.764948in}}%
\pgfpathlineto{\pgfqpoint{4.057251in}{0.764846in}}%
\pgfpathlineto{\pgfqpoint{4.057547in}{0.764745in}}%
\pgfpathlineto{\pgfqpoint{4.057843in}{0.764643in}}%
\pgfpathlineto{\pgfqpoint{4.058139in}{0.764541in}}%
\pgfpathlineto{\pgfqpoint{4.058435in}{0.764439in}}%
\pgfpathlineto{\pgfqpoint{4.058731in}{0.764337in}}%
\pgfpathlineto{\pgfqpoint{4.059027in}{0.764235in}}%
\pgfpathlineto{\pgfqpoint{4.059323in}{0.764134in}}%
\pgfpathlineto{\pgfqpoint{4.059619in}{0.764032in}}%
\pgfpathlineto{\pgfqpoint{4.059915in}{0.763930in}}%
\pgfpathlineto{\pgfqpoint{4.060211in}{0.763828in}}%
\pgfpathlineto{\pgfqpoint{4.060507in}{0.763726in}}%
\pgfpathlineto{\pgfqpoint{4.060803in}{0.763624in}}%
\pgfpathlineto{\pgfqpoint{4.061099in}{0.763522in}}%
\pgfpathlineto{\pgfqpoint{4.061395in}{0.763421in}}%
\pgfpathlineto{\pgfqpoint{4.061691in}{0.763319in}}%
\pgfpathlineto{\pgfqpoint{4.061987in}{0.763217in}}%
\pgfpathlineto{\pgfqpoint{4.062283in}{0.763115in}}%
\pgfpathlineto{\pgfqpoint{4.062579in}{0.763013in}}%
\pgfpathlineto{\pgfqpoint{4.062875in}{0.762911in}}%
\pgfpathlineto{\pgfqpoint{4.063171in}{0.762810in}}%
\pgfpathlineto{\pgfqpoint{4.063467in}{0.762708in}}%
\pgfpathlineto{\pgfqpoint{4.063763in}{0.762606in}}%
\pgfpathlineto{\pgfqpoint{4.064059in}{0.762504in}}%
\pgfpathlineto{\pgfqpoint{4.064355in}{0.762402in}}%
\pgfpathlineto{\pgfqpoint{4.064651in}{0.762300in}}%
\pgfpathlineto{\pgfqpoint{4.064947in}{0.762198in}}%
\pgfpathlineto{\pgfqpoint{4.065243in}{0.762097in}}%
\pgfpathlineto{\pgfqpoint{4.065539in}{0.761995in}}%
\pgfpathlineto{\pgfqpoint{4.065835in}{0.761893in}}%
\pgfpathlineto{\pgfqpoint{4.066131in}{0.761791in}}%
\pgfpathlineto{\pgfqpoint{4.066427in}{0.761689in}}%
\pgfpathlineto{\pgfqpoint{4.066723in}{0.761587in}}%
\pgfpathlineto{\pgfqpoint{4.067019in}{0.761486in}}%
\pgfpathlineto{\pgfqpoint{4.067315in}{0.761384in}}%
\pgfpathlineto{\pgfqpoint{4.067611in}{0.761282in}}%
\pgfpathlineto{\pgfqpoint{4.067907in}{0.761180in}}%
\pgfpathlineto{\pgfqpoint{4.068203in}{0.761078in}}%
\pgfpathlineto{\pgfqpoint{4.068499in}{0.760976in}}%
\pgfpathlineto{\pgfqpoint{4.068795in}{0.760874in}}%
\pgfpathlineto{\pgfqpoint{4.069091in}{0.760773in}}%
\pgfpathlineto{\pgfqpoint{4.069387in}{0.760671in}}%
\pgfpathlineto{\pgfqpoint{4.069683in}{0.760569in}}%
\pgfpathlineto{\pgfqpoint{4.069979in}{0.760467in}}%
\pgfpathlineto{\pgfqpoint{4.070275in}{0.760406in}}%
\pgfpathlineto{\pgfqpoint{4.070571in}{0.760531in}}%
\pgfpathlineto{\pgfqpoint{4.070867in}{0.760678in}}%
\pgfpathlineto{\pgfqpoint{4.071163in}{0.760818in}}%
\pgfpathlineto{\pgfqpoint{4.071459in}{0.760852in}}%
\pgfpathlineto{\pgfqpoint{4.071755in}{0.760847in}}%
\pgfpathlineto{\pgfqpoint{4.072051in}{0.760843in}}%
\pgfpathlineto{\pgfqpoint{4.072347in}{0.760838in}}%
\pgfpathlineto{\pgfqpoint{4.072643in}{0.760834in}}%
\pgfpathlineto{\pgfqpoint{4.072939in}{0.760830in}}%
\pgfpathlineto{\pgfqpoint{4.073235in}{0.760825in}}%
\pgfpathlineto{\pgfqpoint{4.073531in}{0.760821in}}%
\pgfpathlineto{\pgfqpoint{4.073827in}{0.760816in}}%
\pgfpathlineto{\pgfqpoint{4.074123in}{0.760812in}}%
\pgfpathlineto{\pgfqpoint{4.074419in}{0.760808in}}%
\pgfpathlineto{\pgfqpoint{4.074715in}{0.760803in}}%
\pgfpathlineto{\pgfqpoint{4.075011in}{0.760799in}}%
\pgfpathlineto{\pgfqpoint{4.075307in}{0.760794in}}%
\pgfpathlineto{\pgfqpoint{4.075603in}{0.760790in}}%
\pgfpathlineto{\pgfqpoint{4.075899in}{0.760785in}}%
\pgfpathlineto{\pgfqpoint{4.076195in}{0.760781in}}%
\pgfpathlineto{\pgfqpoint{4.076491in}{0.760777in}}%
\pgfpathlineto{\pgfqpoint{4.076787in}{0.760772in}}%
\pgfpathlineto{\pgfqpoint{4.077083in}{0.760768in}}%
\pgfpathlineto{\pgfqpoint{4.077379in}{0.760763in}}%
\pgfpathlineto{\pgfqpoint{4.077675in}{0.760759in}}%
\pgfpathlineto{\pgfqpoint{4.077971in}{0.760755in}}%
\pgfpathlineto{\pgfqpoint{4.078267in}{0.760750in}}%
\pgfpathlineto{\pgfqpoint{4.078563in}{0.760746in}}%
\pgfpathlineto{\pgfqpoint{4.078859in}{0.760741in}}%
\pgfpathlineto{\pgfqpoint{4.079155in}{0.760737in}}%
\pgfpathlineto{\pgfqpoint{4.079451in}{0.760733in}}%
\pgfpathlineto{\pgfqpoint{4.079747in}{0.760728in}}%
\pgfpathlineto{\pgfqpoint{4.080043in}{0.760724in}}%
\pgfpathlineto{\pgfqpoint{4.080339in}{0.760719in}}%
\pgfpathlineto{\pgfqpoint{4.080635in}{0.760715in}}%
\pgfpathlineto{\pgfqpoint{4.080931in}{0.760710in}}%
\pgfpathlineto{\pgfqpoint{4.081227in}{0.760706in}}%
\pgfpathlineto{\pgfqpoint{4.081523in}{0.760702in}}%
\pgfpathlineto{\pgfqpoint{4.081819in}{0.760697in}}%
\pgfpathlineto{\pgfqpoint{4.082115in}{0.760693in}}%
\pgfpathlineto{\pgfqpoint{4.082411in}{0.760688in}}%
\pgfpathlineto{\pgfqpoint{4.082707in}{0.760684in}}%
\pgfpathlineto{\pgfqpoint{4.083003in}{0.760680in}}%
\pgfpathlineto{\pgfqpoint{4.083299in}{0.760675in}}%
\pgfpathlineto{\pgfqpoint{4.083595in}{0.760671in}}%
\pgfpathlineto{\pgfqpoint{4.083891in}{0.760666in}}%
\pgfpathlineto{\pgfqpoint{4.084187in}{0.760662in}}%
\pgfpathlineto{\pgfqpoint{4.084483in}{0.760658in}}%
\pgfpathlineto{\pgfqpoint{4.084779in}{0.760653in}}%
\pgfpathlineto{\pgfqpoint{4.085075in}{0.760649in}}%
\pgfpathlineto{\pgfqpoint{4.085371in}{0.760644in}}%
\pgfpathlineto{\pgfqpoint{4.085667in}{0.760640in}}%
\pgfpathlineto{\pgfqpoint{4.085963in}{0.760635in}}%
\pgfpathlineto{\pgfqpoint{4.086260in}{0.760631in}}%
\pgfpathlineto{\pgfqpoint{4.086556in}{0.760627in}}%
\pgfpathlineto{\pgfqpoint{4.086852in}{0.760622in}}%
\pgfpathlineto{\pgfqpoint{4.087148in}{0.760618in}}%
\pgfpathlineto{\pgfqpoint{4.087444in}{0.760613in}}%
\pgfpathlineto{\pgfqpoint{4.087740in}{0.760609in}}%
\pgfpathlineto{\pgfqpoint{4.088036in}{0.760605in}}%
\pgfpathlineto{\pgfqpoint{4.088332in}{0.760600in}}%
\pgfpathlineto{\pgfqpoint{4.088628in}{0.760596in}}%
\pgfpathlineto{\pgfqpoint{4.088924in}{0.760591in}}%
\pgfpathlineto{\pgfqpoint{4.089220in}{0.760587in}}%
\pgfpathlineto{\pgfqpoint{4.089516in}{0.760583in}}%
\pgfpathlineto{\pgfqpoint{4.089812in}{0.760578in}}%
\pgfpathlineto{\pgfqpoint{4.090108in}{0.760574in}}%
\pgfpathlineto{\pgfqpoint{4.090404in}{0.760569in}}%
\pgfpathlineto{\pgfqpoint{4.090700in}{0.760565in}}%
\pgfpathlineto{\pgfqpoint{4.090996in}{0.760560in}}%
\pgfpathlineto{\pgfqpoint{4.091292in}{0.760556in}}%
\pgfpathlineto{\pgfqpoint{4.091588in}{0.760552in}}%
\pgfpathlineto{\pgfqpoint{4.091884in}{0.760547in}}%
\pgfpathlineto{\pgfqpoint{4.092180in}{0.760543in}}%
\pgfpathlineto{\pgfqpoint{4.092476in}{0.760538in}}%
\pgfpathlineto{\pgfqpoint{4.092772in}{0.760534in}}%
\pgfpathlineto{\pgfqpoint{4.093068in}{0.760530in}}%
\pgfpathlineto{\pgfqpoint{4.093364in}{0.760525in}}%
\pgfpathlineto{\pgfqpoint{4.093660in}{0.760521in}}%
\pgfpathlineto{\pgfqpoint{4.093956in}{0.760516in}}%
\pgfpathlineto{\pgfqpoint{4.094252in}{0.760512in}}%
\pgfpathlineto{\pgfqpoint{4.094548in}{0.760507in}}%
\pgfpathlineto{\pgfqpoint{4.094844in}{0.760503in}}%
\pgfpathlineto{\pgfqpoint{4.095140in}{0.760499in}}%
\pgfpathlineto{\pgfqpoint{4.095436in}{0.760494in}}%
\pgfpathlineto{\pgfqpoint{4.095732in}{0.760490in}}%
\pgfpathlineto{\pgfqpoint{4.096028in}{0.760485in}}%
\pgfpathlineto{\pgfqpoint{4.096324in}{0.760481in}}%
\pgfpathlineto{\pgfqpoint{4.096620in}{0.760477in}}%
\pgfpathlineto{\pgfqpoint{4.096916in}{0.760472in}}%
\pgfpathlineto{\pgfqpoint{4.097212in}{0.760468in}}%
\pgfpathlineto{\pgfqpoint{4.097508in}{0.760463in}}%
\pgfpathlineto{\pgfqpoint{4.097804in}{0.760459in}}%
\pgfpathlineto{\pgfqpoint{4.098100in}{0.760455in}}%
\pgfpathlineto{\pgfqpoint{4.098396in}{0.760450in}}%
\pgfpathlineto{\pgfqpoint{4.098692in}{0.760446in}}%
\pgfpathlineto{\pgfqpoint{4.098988in}{0.760441in}}%
\pgfpathlineto{\pgfqpoint{4.099284in}{0.760437in}}%
\pgfpathlineto{\pgfqpoint{4.099580in}{0.760432in}}%
\pgfpathlineto{\pgfqpoint{4.099876in}{0.760428in}}%
\pgfpathlineto{\pgfqpoint{4.100172in}{0.760424in}}%
\pgfpathlineto{\pgfqpoint{4.100468in}{0.760419in}}%
\pgfpathlineto{\pgfqpoint{4.100764in}{0.760415in}}%
\pgfpathlineto{\pgfqpoint{4.101060in}{0.760410in}}%
\pgfpathlineto{\pgfqpoint{4.101356in}{0.760406in}}%
\pgfpathlineto{\pgfqpoint{4.101652in}{0.760402in}}%
\pgfpathlineto{\pgfqpoint{4.101948in}{0.760397in}}%
\pgfpathlineto{\pgfqpoint{4.102244in}{0.760393in}}%
\pgfpathlineto{\pgfqpoint{4.102540in}{0.760388in}}%
\pgfpathlineto{\pgfqpoint{4.102836in}{0.760384in}}%
\pgfpathlineto{\pgfqpoint{4.103132in}{0.760380in}}%
\pgfpathlineto{\pgfqpoint{4.103428in}{0.760375in}}%
\pgfpathlineto{\pgfqpoint{4.103724in}{0.760371in}}%
\pgfpathlineto{\pgfqpoint{4.104020in}{0.760366in}}%
\pgfpathlineto{\pgfqpoint{4.104316in}{0.760362in}}%
\pgfpathlineto{\pgfqpoint{4.104612in}{0.760357in}}%
\pgfpathlineto{\pgfqpoint{4.104908in}{0.760353in}}%
\pgfpathlineto{\pgfqpoint{4.105204in}{0.760349in}}%
\pgfpathlineto{\pgfqpoint{4.105500in}{0.760344in}}%
\pgfpathlineto{\pgfqpoint{4.105796in}{0.760341in}}%
\pgfpathlineto{\pgfqpoint{4.106092in}{0.760340in}}%
\pgfpathlineto{\pgfqpoint{4.106388in}{0.760338in}}%
\pgfpathlineto{\pgfqpoint{4.106684in}{0.760337in}}%
\pgfpathlineto{\pgfqpoint{4.106980in}{0.760334in}}%
\pgfpathlineto{\pgfqpoint{4.107276in}{0.760147in}}%
\pgfpathlineto{\pgfqpoint{4.107572in}{0.759924in}}%
\pgfpathlineto{\pgfqpoint{4.107868in}{0.759544in}}%
\pgfpathlineto{\pgfqpoint{4.108164in}{0.759365in}}%
\pgfpathlineto{\pgfqpoint{4.108460in}{0.759510in}}%
\pgfpathlineto{\pgfqpoint{4.108756in}{0.759654in}}%
\pgfpathlineto{\pgfqpoint{4.109052in}{0.759799in}}%
\pgfpathlineto{\pgfqpoint{4.109348in}{0.759944in}}%
\pgfpathlineto{\pgfqpoint{4.109644in}{0.760088in}}%
\pgfpathlineto{\pgfqpoint{4.109940in}{0.760233in}}%
\pgfpathlineto{\pgfqpoint{4.110236in}{0.760377in}}%
\pgfpathlineto{\pgfqpoint{4.110532in}{0.760522in}}%
\pgfpathlineto{\pgfqpoint{4.110828in}{0.760667in}}%
\pgfpathlineto{\pgfqpoint{4.111124in}{0.760811in}}%
\pgfpathlineto{\pgfqpoint{4.111420in}{0.760956in}}%
\pgfpathlineto{\pgfqpoint{4.111716in}{0.761101in}}%
\pgfpathlineto{\pgfqpoint{4.112012in}{0.762097in}}%
\pgfpathlineto{\pgfqpoint{4.112308in}{0.768464in}}%
\pgfpathlineto{\pgfqpoint{4.112604in}{0.768611in}}%
\pgfpathlineto{\pgfqpoint{4.112900in}{0.768593in}}%
\pgfpathlineto{\pgfqpoint{4.113196in}{0.768579in}}%
\pgfpathlineto{\pgfqpoint{4.113492in}{0.768565in}}%
\pgfpathlineto{\pgfqpoint{4.113788in}{0.768552in}}%
\pgfpathlineto{\pgfqpoint{4.114084in}{0.768538in}}%
\pgfpathlineto{\pgfqpoint{4.114380in}{0.768524in}}%
\pgfpathlineto{\pgfqpoint{4.114676in}{0.767491in}}%
\pgfpathlineto{\pgfqpoint{4.114972in}{0.767192in}}%
\pgfpathlineto{\pgfqpoint{4.115268in}{0.767056in}}%
\pgfpathlineto{\pgfqpoint{4.115564in}{0.766270in}}%
\pgfpathlineto{\pgfqpoint{4.115860in}{0.766205in}}%
\pgfpathlineto{\pgfqpoint{4.116156in}{0.766139in}}%
\pgfpathlineto{\pgfqpoint{4.116452in}{0.766074in}}%
\pgfpathlineto{\pgfqpoint{4.116748in}{0.766009in}}%
\pgfpathlineto{\pgfqpoint{4.117044in}{0.765943in}}%
\pgfpathlineto{\pgfqpoint{4.117340in}{0.765878in}}%
\pgfpathlineto{\pgfqpoint{4.117636in}{0.765813in}}%
\pgfpathlineto{\pgfqpoint{4.117932in}{0.765747in}}%
\pgfpathlineto{\pgfqpoint{4.118228in}{0.765682in}}%
\pgfpathlineto{\pgfqpoint{4.118524in}{0.765617in}}%
\pgfpathlineto{\pgfqpoint{4.118820in}{0.765552in}}%
\pgfpathlineto{\pgfqpoint{4.119116in}{0.765486in}}%
\pgfpathlineto{\pgfqpoint{4.119412in}{0.765431in}}%
\pgfpathlineto{\pgfqpoint{4.119708in}{0.765387in}}%
\pgfpathlineto{\pgfqpoint{4.120004in}{0.765344in}}%
\pgfpathlineto{\pgfqpoint{4.120300in}{0.765298in}}%
\pgfpathlineto{\pgfqpoint{4.120596in}{0.765250in}}%
\pgfpathlineto{\pgfqpoint{4.120892in}{0.765202in}}%
\pgfpathlineto{\pgfqpoint{4.121188in}{0.764859in}}%
\pgfpathlineto{\pgfqpoint{4.121484in}{0.764674in}}%
\pgfpathlineto{\pgfqpoint{4.121780in}{0.764677in}}%
\pgfpathlineto{\pgfqpoint{4.122076in}{0.764681in}}%
\pgfpathlineto{\pgfqpoint{4.122372in}{0.764684in}}%
\pgfpathlineto{\pgfqpoint{4.122668in}{0.764687in}}%
\pgfpathlineto{\pgfqpoint{4.122964in}{0.764691in}}%
\pgfpathlineto{\pgfqpoint{4.123260in}{0.764694in}}%
\pgfpathlineto{\pgfqpoint{4.123556in}{0.764698in}}%
\pgfpathlineto{\pgfqpoint{4.123852in}{0.764701in}}%
\pgfpathlineto{\pgfqpoint{4.124148in}{0.764705in}}%
\pgfpathlineto{\pgfqpoint{4.124444in}{0.764708in}}%
\pgfpathlineto{\pgfqpoint{4.124740in}{0.764711in}}%
\pgfpathlineto{\pgfqpoint{4.125036in}{0.764715in}}%
\pgfpathlineto{\pgfqpoint{4.125332in}{0.764718in}}%
\pgfpathlineto{\pgfqpoint{4.125628in}{0.764722in}}%
\pgfpathlineto{\pgfqpoint{4.125924in}{0.764725in}}%
\pgfpathlineto{\pgfqpoint{4.126220in}{0.764729in}}%
\pgfpathlineto{\pgfqpoint{4.126516in}{0.764732in}}%
\pgfpathlineto{\pgfqpoint{4.126812in}{0.764735in}}%
\pgfpathlineto{\pgfqpoint{4.127108in}{0.764731in}}%
\pgfpathlineto{\pgfqpoint{4.127404in}{0.764717in}}%
\pgfpathlineto{\pgfqpoint{4.127700in}{0.764700in}}%
\pgfpathlineto{\pgfqpoint{4.127996in}{0.764226in}}%
\pgfpathlineto{\pgfqpoint{4.128292in}{0.764237in}}%
\pgfpathlineto{\pgfqpoint{4.128588in}{0.764278in}}%
\pgfpathlineto{\pgfqpoint{4.128884in}{0.764319in}}%
\pgfpathlineto{\pgfqpoint{4.129180in}{0.764360in}}%
\pgfpathlineto{\pgfqpoint{4.129476in}{0.764394in}}%
\pgfpathlineto{\pgfqpoint{4.129772in}{0.764410in}}%
\pgfpathlineto{\pgfqpoint{4.130068in}{0.764425in}}%
\pgfpathlineto{\pgfqpoint{4.130364in}{0.764439in}}%
\pgfpathlineto{\pgfqpoint{4.130660in}{0.764454in}}%
\pgfpathlineto{\pgfqpoint{4.130956in}{0.764468in}}%
\pgfpathlineto{\pgfqpoint{4.131252in}{0.764483in}}%
\pgfpathlineto{\pgfqpoint{4.131548in}{0.764497in}}%
\pgfpathlineto{\pgfqpoint{4.131844in}{0.764512in}}%
\pgfpathlineto{\pgfqpoint{4.132140in}{0.764526in}}%
\pgfpathlineto{\pgfqpoint{4.132436in}{0.764541in}}%
\pgfpathlineto{\pgfqpoint{4.132732in}{0.764555in}}%
\pgfpathlineto{\pgfqpoint{4.133028in}{0.764570in}}%
\pgfpathlineto{\pgfqpoint{4.133324in}{0.764584in}}%
\pgfpathlineto{\pgfqpoint{4.133620in}{0.764599in}}%
\pgfpathlineto{\pgfqpoint{4.133916in}{0.764613in}}%
\pgfpathlineto{\pgfqpoint{4.134212in}{0.764628in}}%
\pgfpathlineto{\pgfqpoint{4.134508in}{0.764642in}}%
\pgfpathlineto{\pgfqpoint{4.134804in}{0.764657in}}%
\pgfpathlineto{\pgfqpoint{4.135100in}{0.764671in}}%
\pgfpathlineto{\pgfqpoint{4.135396in}{0.764686in}}%
\pgfpathlineto{\pgfqpoint{4.135692in}{0.764700in}}%
\pgfpathlineto{\pgfqpoint{4.135988in}{0.764715in}}%
\pgfpathlineto{\pgfqpoint{4.136284in}{0.764729in}}%
\pgfpathlineto{\pgfqpoint{4.136580in}{0.764743in}}%
\pgfpathlineto{\pgfqpoint{4.136876in}{0.764758in}}%
\pgfpathlineto{\pgfqpoint{4.137172in}{0.764772in}}%
\pgfpathlineto{\pgfqpoint{4.137468in}{0.764787in}}%
\pgfpathlineto{\pgfqpoint{4.137764in}{0.764801in}}%
\pgfpathlineto{\pgfqpoint{4.138060in}{0.764816in}}%
\pgfpathlineto{\pgfqpoint{4.138356in}{0.764830in}}%
\pgfpathlineto{\pgfqpoint{4.138652in}{0.764845in}}%
\pgfpathlineto{\pgfqpoint{4.138948in}{0.764859in}}%
\pgfpathlineto{\pgfqpoint{4.139244in}{0.764874in}}%
\pgfpathlineto{\pgfqpoint{4.139540in}{0.764888in}}%
\pgfpathlineto{\pgfqpoint{4.139836in}{0.764903in}}%
\pgfpathlineto{\pgfqpoint{4.140132in}{0.764917in}}%
\pgfpathlineto{\pgfqpoint{4.140428in}{0.764932in}}%
\pgfpathlineto{\pgfqpoint{4.140724in}{0.764946in}}%
\pgfpathlineto{\pgfqpoint{4.141020in}{0.764961in}}%
\pgfpathlineto{\pgfqpoint{4.141316in}{0.764975in}}%
\pgfpathlineto{\pgfqpoint{4.141612in}{0.764990in}}%
\pgfpathlineto{\pgfqpoint{4.141908in}{0.765004in}}%
\pgfpathlineto{\pgfqpoint{4.142204in}{0.765019in}}%
\pgfpathlineto{\pgfqpoint{4.142500in}{0.765033in}}%
\pgfpathlineto{\pgfqpoint{4.142796in}{0.765048in}}%
\pgfpathlineto{\pgfqpoint{4.143092in}{0.765062in}}%
\pgfpathlineto{\pgfqpoint{4.143388in}{0.765077in}}%
\pgfpathlineto{\pgfqpoint{4.143684in}{0.765091in}}%
\pgfpathlineto{\pgfqpoint{4.143980in}{0.765106in}}%
\pgfpathlineto{\pgfqpoint{4.144276in}{0.765120in}}%
\pgfpathlineto{\pgfqpoint{4.144572in}{0.765135in}}%
\pgfpathlineto{\pgfqpoint{4.144868in}{0.765149in}}%
\pgfpathlineto{\pgfqpoint{4.145164in}{0.765164in}}%
\pgfpathlineto{\pgfqpoint{4.145460in}{0.765178in}}%
\pgfpathlineto{\pgfqpoint{4.145756in}{0.765193in}}%
\pgfpathlineto{\pgfqpoint{4.146052in}{0.765207in}}%
\pgfpathlineto{\pgfqpoint{4.146348in}{0.765222in}}%
\pgfpathlineto{\pgfqpoint{4.146644in}{0.765236in}}%
\pgfpathlineto{\pgfqpoint{4.146940in}{0.765251in}}%
\pgfpathlineto{\pgfqpoint{4.147236in}{0.765265in}}%
\pgfpathlineto{\pgfqpoint{4.147532in}{0.765280in}}%
\pgfpathlineto{\pgfqpoint{4.147828in}{0.765294in}}%
\pgfpathlineto{\pgfqpoint{4.148124in}{0.765309in}}%
\pgfpathlineto{\pgfqpoint{4.148420in}{0.765323in}}%
\pgfpathlineto{\pgfqpoint{4.148716in}{0.765324in}}%
\pgfpathlineto{\pgfqpoint{4.149012in}{0.765282in}}%
\pgfpathlineto{\pgfqpoint{4.149308in}{0.765237in}}%
\pgfpathlineto{\pgfqpoint{4.149604in}{0.765192in}}%
\pgfpathlineto{\pgfqpoint{4.149900in}{0.765147in}}%
\pgfpathlineto{\pgfqpoint{4.150196in}{0.765102in}}%
\pgfpathlineto{\pgfqpoint{4.150492in}{0.765057in}}%
\pgfpathlineto{\pgfqpoint{4.150788in}{0.765012in}}%
\pgfpathlineto{\pgfqpoint{4.151084in}{0.764967in}}%
\pgfpathlineto{\pgfqpoint{4.151380in}{0.764922in}}%
\pgfpathlineto{\pgfqpoint{4.151676in}{0.764877in}}%
\pgfpathlineto{\pgfqpoint{4.151972in}{0.764832in}}%
\pgfpathlineto{\pgfqpoint{4.152268in}{0.764787in}}%
\pgfpathlineto{\pgfqpoint{4.152564in}{0.764742in}}%
\pgfpathlineto{\pgfqpoint{4.152860in}{0.764697in}}%
\pgfpathlineto{\pgfqpoint{4.153156in}{0.764652in}}%
\pgfpathlineto{\pgfqpoint{4.153452in}{0.764607in}}%
\pgfpathlineto{\pgfqpoint{4.153749in}{0.764562in}}%
\pgfpathlineto{\pgfqpoint{4.154045in}{0.764517in}}%
\pgfpathlineto{\pgfqpoint{4.154341in}{0.764472in}}%
\pgfpathlineto{\pgfqpoint{4.154637in}{0.764427in}}%
\pgfpathlineto{\pgfqpoint{4.154933in}{0.764384in}}%
\pgfpathlineto{\pgfqpoint{4.155229in}{0.764355in}}%
\pgfpathlineto{\pgfqpoint{4.155525in}{0.764329in}}%
\pgfpathlineto{\pgfqpoint{4.155821in}{0.764303in}}%
\pgfpathlineto{\pgfqpoint{4.156117in}{0.764277in}}%
\pgfpathlineto{\pgfqpoint{4.156413in}{0.764251in}}%
\pgfpathlineto{\pgfqpoint{4.156709in}{0.764225in}}%
\pgfpathlineto{\pgfqpoint{4.157005in}{0.764199in}}%
\pgfpathlineto{\pgfqpoint{4.157301in}{0.764173in}}%
\pgfpathlineto{\pgfqpoint{4.157597in}{0.764147in}}%
\pgfpathlineto{\pgfqpoint{4.157893in}{0.764121in}}%
\pgfpathlineto{\pgfqpoint{4.158189in}{0.764094in}}%
\pgfpathlineto{\pgfqpoint{4.158485in}{0.764068in}}%
\pgfpathlineto{\pgfqpoint{4.158781in}{0.764042in}}%
\pgfpathlineto{\pgfqpoint{4.159077in}{0.764016in}}%
\pgfpathlineto{\pgfqpoint{4.159373in}{0.763990in}}%
\pgfpathlineto{\pgfqpoint{4.159669in}{0.763964in}}%
\pgfpathlineto{\pgfqpoint{4.159965in}{0.763938in}}%
\pgfpathlineto{\pgfqpoint{4.160261in}{0.763912in}}%
\pgfpathlineto{\pgfqpoint{4.160557in}{0.763886in}}%
\pgfpathlineto{\pgfqpoint{4.160853in}{0.763860in}}%
\pgfpathlineto{\pgfqpoint{4.161149in}{0.763834in}}%
\pgfpathlineto{\pgfqpoint{4.161445in}{0.763808in}}%
\pgfpathlineto{\pgfqpoint{4.161741in}{0.763782in}}%
\pgfpathlineto{\pgfqpoint{4.162037in}{0.763756in}}%
\pgfpathlineto{\pgfqpoint{4.162333in}{0.763742in}}%
\pgfpathlineto{\pgfqpoint{4.162629in}{0.763784in}}%
\pgfpathlineto{\pgfqpoint{4.162925in}{0.764090in}}%
\pgfpathlineto{\pgfqpoint{4.163221in}{0.764438in}}%
\pgfpathlineto{\pgfqpoint{4.163517in}{0.764462in}}%
\pgfpathlineto{\pgfqpoint{4.163813in}{0.764151in}}%
\pgfpathlineto{\pgfqpoint{4.164109in}{0.763906in}}%
\pgfpathlineto{\pgfqpoint{4.164405in}{0.763927in}}%
\pgfpathlineto{\pgfqpoint{4.164701in}{0.763975in}}%
\pgfpathlineto{\pgfqpoint{4.164997in}{0.764023in}}%
\pgfpathlineto{\pgfqpoint{4.165293in}{0.764071in}}%
\pgfpathlineto{\pgfqpoint{4.165589in}{0.764119in}}%
\pgfpathlineto{\pgfqpoint{4.165885in}{0.764168in}}%
\pgfpathlineto{\pgfqpoint{4.166181in}{0.764216in}}%
\pgfpathlineto{\pgfqpoint{4.166477in}{0.764264in}}%
\pgfpathlineto{\pgfqpoint{4.166773in}{0.764312in}}%
\pgfpathlineto{\pgfqpoint{4.167069in}{0.764360in}}%
\pgfpathlineto{\pgfqpoint{4.167365in}{0.764409in}}%
\pgfpathlineto{\pgfqpoint{4.167661in}{0.764457in}}%
\pgfpathlineto{\pgfqpoint{4.167957in}{0.764505in}}%
\pgfpathlineto{\pgfqpoint{4.168253in}{0.764553in}}%
\pgfpathlineto{\pgfqpoint{4.168549in}{0.764601in}}%
\pgfpathlineto{\pgfqpoint{4.168845in}{0.764649in}}%
\pgfpathlineto{\pgfqpoint{4.169141in}{0.764698in}}%
\pgfpathlineto{\pgfqpoint{4.169437in}{0.764744in}}%
\pgfpathlineto{\pgfqpoint{4.169733in}{0.764426in}}%
\pgfpathlineto{\pgfqpoint{4.170029in}{0.764235in}}%
\pgfpathlineto{\pgfqpoint{4.170325in}{0.764219in}}%
\pgfpathlineto{\pgfqpoint{4.170621in}{0.764172in}}%
\pgfpathlineto{\pgfqpoint{4.170917in}{0.763362in}}%
\pgfpathlineto{\pgfqpoint{4.171213in}{0.763762in}}%
\pgfpathlineto{\pgfqpoint{4.171509in}{0.764188in}}%
\pgfpathlineto{\pgfqpoint{4.171805in}{0.764162in}}%
\pgfpathlineto{\pgfqpoint{4.172101in}{0.764096in}}%
\pgfpathlineto{\pgfqpoint{4.172397in}{0.764029in}}%
\pgfpathlineto{\pgfqpoint{4.172693in}{0.763963in}}%
\pgfpathlineto{\pgfqpoint{4.172989in}{0.763897in}}%
\pgfpathlineto{\pgfqpoint{4.173285in}{0.763831in}}%
\pgfpathlineto{\pgfqpoint{4.173581in}{0.763765in}}%
\pgfpathlineto{\pgfqpoint{4.173877in}{0.763699in}}%
\pgfpathlineto{\pgfqpoint{4.174173in}{0.763633in}}%
\pgfpathlineto{\pgfqpoint{4.174469in}{0.763567in}}%
\pgfpathlineto{\pgfqpoint{4.174765in}{0.763500in}}%
\pgfpathlineto{\pgfqpoint{4.175061in}{0.763434in}}%
\pgfpathlineto{\pgfqpoint{4.175357in}{0.763368in}}%
\pgfpathlineto{\pgfqpoint{4.175653in}{0.763302in}}%
\pgfpathlineto{\pgfqpoint{4.175949in}{0.763236in}}%
\pgfpathlineto{\pgfqpoint{4.176245in}{0.763170in}}%
\pgfpathlineto{\pgfqpoint{4.176541in}{0.763104in}}%
\pgfpathlineto{\pgfqpoint{4.176837in}{0.763038in}}%
\pgfpathlineto{\pgfqpoint{4.177133in}{0.762971in}}%
\pgfpathlineto{\pgfqpoint{4.177429in}{0.762905in}}%
\pgfpathlineto{\pgfqpoint{4.177725in}{0.762839in}}%
\pgfpathlineto{\pgfqpoint{4.178021in}{0.762816in}}%
\pgfpathlineto{\pgfqpoint{4.178317in}{0.762921in}}%
\pgfpathlineto{\pgfqpoint{4.178613in}{0.763035in}}%
\pgfpathlineto{\pgfqpoint{4.178909in}{0.763107in}}%
\pgfpathlineto{\pgfqpoint{4.179205in}{0.762880in}}%
\pgfpathlineto{\pgfqpoint{4.179501in}{0.762620in}}%
\pgfpathlineto{\pgfqpoint{4.179797in}{0.762628in}}%
\pgfpathlineto{\pgfqpoint{4.180093in}{0.762638in}}%
\pgfpathlineto{\pgfqpoint{4.180389in}{0.762647in}}%
\pgfpathlineto{\pgfqpoint{4.180685in}{0.762656in}}%
\pgfpathlineto{\pgfqpoint{4.180981in}{0.762665in}}%
\pgfpathlineto{\pgfqpoint{4.181277in}{0.762675in}}%
\pgfpathlineto{\pgfqpoint{4.181573in}{0.762684in}}%
\pgfpathlineto{\pgfqpoint{4.181869in}{0.762693in}}%
\pgfpathlineto{\pgfqpoint{4.182165in}{0.762703in}}%
\pgfpathlineto{\pgfqpoint{4.182461in}{0.762712in}}%
\pgfpathlineto{\pgfqpoint{4.182757in}{0.762721in}}%
\pgfpathlineto{\pgfqpoint{4.183053in}{0.762731in}}%
\pgfpathlineto{\pgfqpoint{4.183349in}{0.762740in}}%
\pgfpathlineto{\pgfqpoint{4.183645in}{0.762749in}}%
\pgfpathlineto{\pgfqpoint{4.183941in}{0.762759in}}%
\pgfpathlineto{\pgfqpoint{4.184237in}{0.762768in}}%
\pgfpathlineto{\pgfqpoint{4.184533in}{0.762777in}}%
\pgfpathlineto{\pgfqpoint{4.184829in}{0.762787in}}%
\pgfpathlineto{\pgfqpoint{4.185125in}{0.762796in}}%
\pgfpathlineto{\pgfqpoint{4.185421in}{0.762805in}}%
\pgfpathlineto{\pgfqpoint{4.185717in}{0.762815in}}%
\pgfpathlineto{\pgfqpoint{4.186013in}{0.762824in}}%
\pgfpathlineto{\pgfqpoint{4.186309in}{0.762833in}}%
\pgfpathlineto{\pgfqpoint{4.186605in}{0.762843in}}%
\pgfpathlineto{\pgfqpoint{4.186901in}{0.762852in}}%
\pgfpathlineto{\pgfqpoint{4.187197in}{0.762861in}}%
\pgfpathlineto{\pgfqpoint{4.187493in}{0.762871in}}%
\pgfpathlineto{\pgfqpoint{4.187789in}{0.762880in}}%
\pgfpathlineto{\pgfqpoint{4.188085in}{0.762889in}}%
\pgfpathlineto{\pgfqpoint{4.188381in}{0.762899in}}%
\pgfpathlineto{\pgfqpoint{4.188677in}{0.762908in}}%
\pgfpathlineto{\pgfqpoint{4.188973in}{0.762917in}}%
\pgfpathlineto{\pgfqpoint{4.189269in}{0.762927in}}%
\pgfpathlineto{\pgfqpoint{4.189565in}{0.762936in}}%
\pgfpathlineto{\pgfqpoint{4.189861in}{0.762945in}}%
\pgfpathlineto{\pgfqpoint{4.190157in}{0.762955in}}%
\pgfpathlineto{\pgfqpoint{4.190453in}{0.762964in}}%
\pgfpathlineto{\pgfqpoint{4.190749in}{0.762973in}}%
\pgfpathlineto{\pgfqpoint{4.191045in}{0.762983in}}%
\pgfpathlineto{\pgfqpoint{4.191341in}{0.762992in}}%
\pgfpathlineto{\pgfqpoint{4.191637in}{0.763001in}}%
\pgfpathlineto{\pgfqpoint{4.191933in}{0.763011in}}%
\pgfpathlineto{\pgfqpoint{4.192229in}{0.763020in}}%
\pgfpathlineto{\pgfqpoint{4.192525in}{0.763029in}}%
\pgfpathlineto{\pgfqpoint{4.192821in}{0.763038in}}%
\pgfpathlineto{\pgfqpoint{4.193117in}{0.763048in}}%
\pgfpathlineto{\pgfqpoint{4.193413in}{0.763057in}}%
\pgfpathlineto{\pgfqpoint{4.193709in}{0.763066in}}%
\pgfpathlineto{\pgfqpoint{4.194005in}{0.763076in}}%
\pgfpathlineto{\pgfqpoint{4.194301in}{0.763085in}}%
\pgfpathlineto{\pgfqpoint{4.194597in}{0.763094in}}%
\pgfpathlineto{\pgfqpoint{4.194893in}{0.763104in}}%
\pgfpathlineto{\pgfqpoint{4.195189in}{0.763113in}}%
\pgfpathlineto{\pgfqpoint{4.195485in}{0.763122in}}%
\pgfpathlineto{\pgfqpoint{4.195781in}{0.763132in}}%
\pgfpathlineto{\pgfqpoint{4.196077in}{0.763141in}}%
\pgfpathlineto{\pgfqpoint{4.196373in}{0.763150in}}%
\pgfpathlineto{\pgfqpoint{4.196669in}{0.763160in}}%
\pgfpathlineto{\pgfqpoint{4.196965in}{0.763169in}}%
\pgfpathlineto{\pgfqpoint{4.197261in}{0.763178in}}%
\pgfpathlineto{\pgfqpoint{4.197557in}{0.762960in}}%
\pgfpathlineto{\pgfqpoint{4.197853in}{0.762613in}}%
\pgfpathlineto{\pgfqpoint{4.198149in}{0.762810in}}%
\pgfpathlineto{\pgfqpoint{4.198445in}{0.762960in}}%
\pgfpathlineto{\pgfqpoint{4.198741in}{0.763066in}}%
\pgfpathlineto{\pgfqpoint{4.199037in}{0.763069in}}%
\pgfpathlineto{\pgfqpoint{4.199333in}{0.762951in}}%
\pgfpathlineto{\pgfqpoint{4.199629in}{0.761975in}}%
\pgfpathlineto{\pgfqpoint{4.199925in}{0.761260in}}%
\pgfpathlineto{\pgfqpoint{4.200221in}{0.761207in}}%
\pgfpathlineto{\pgfqpoint{4.200517in}{0.761039in}}%
\pgfpathlineto{\pgfqpoint{4.200813in}{0.760700in}}%
\pgfpathlineto{\pgfqpoint{4.201109in}{0.760450in}}%
\pgfpathlineto{\pgfqpoint{4.201405in}{0.760399in}}%
\pgfpathlineto{\pgfqpoint{4.201701in}{0.760401in}}%
\pgfpathlineto{\pgfqpoint{4.201997in}{0.760403in}}%
\pgfpathlineto{\pgfqpoint{4.202293in}{0.760405in}}%
\pgfpathlineto{\pgfqpoint{4.202589in}{0.760407in}}%
\pgfpathlineto{\pgfqpoint{4.202885in}{0.760409in}}%
\pgfpathlineto{\pgfqpoint{4.203181in}{0.760411in}}%
\pgfpathlineto{\pgfqpoint{4.203477in}{0.760412in}}%
\pgfpathlineto{\pgfqpoint{4.203773in}{0.760414in}}%
\pgfpathlineto{\pgfqpoint{4.204069in}{0.760416in}}%
\pgfpathlineto{\pgfqpoint{4.204365in}{0.760418in}}%
\pgfpathlineto{\pgfqpoint{4.204661in}{0.760809in}}%
\pgfpathlineto{\pgfqpoint{4.204957in}{0.761803in}}%
\pgfpathlineto{\pgfqpoint{4.205253in}{0.764565in}}%
\pgfpathlineto{\pgfqpoint{4.205549in}{0.768629in}}%
\pgfpathlineto{\pgfqpoint{4.205845in}{0.772693in}}%
\pgfpathlineto{\pgfqpoint{4.206141in}{0.776756in}}%
\pgfpathlineto{\pgfqpoint{4.206437in}{0.780821in}}%
\pgfpathlineto{\pgfqpoint{4.206733in}{0.784865in}}%
\pgfpathlineto{\pgfqpoint{4.207029in}{0.786466in}}%
\pgfpathlineto{\pgfqpoint{4.207325in}{0.786505in}}%
\pgfpathlineto{\pgfqpoint{4.207621in}{0.786535in}}%
\pgfpathlineto{\pgfqpoint{4.207917in}{0.786565in}}%
\pgfpathlineto{\pgfqpoint{4.208213in}{0.786595in}}%
\pgfpathlineto{\pgfqpoint{4.208509in}{0.786625in}}%
\pgfpathlineto{\pgfqpoint{4.208805in}{0.786656in}}%
\pgfpathlineto{\pgfqpoint{4.209101in}{0.786686in}}%
\pgfpathlineto{\pgfqpoint{4.209397in}{0.786716in}}%
\pgfpathlineto{\pgfqpoint{4.209693in}{0.786746in}}%
\pgfpathlineto{\pgfqpoint{4.209989in}{0.786777in}}%
\pgfpathlineto{\pgfqpoint{4.210285in}{0.786807in}}%
\pgfpathlineto{\pgfqpoint{4.210581in}{0.786837in}}%
\pgfpathlineto{\pgfqpoint{4.210877in}{0.786867in}}%
\pgfpathlineto{\pgfqpoint{4.211173in}{0.786898in}}%
\pgfpathlineto{\pgfqpoint{4.211469in}{0.786916in}}%
\pgfpathlineto{\pgfqpoint{4.211765in}{0.786913in}}%
\pgfpathlineto{\pgfqpoint{4.212061in}{0.786909in}}%
\pgfpathlineto{\pgfqpoint{4.212357in}{0.786906in}}%
\pgfpathlineto{\pgfqpoint{4.212653in}{0.786563in}}%
\pgfpathlineto{\pgfqpoint{4.212949in}{0.786152in}}%
\pgfpathlineto{\pgfqpoint{4.213245in}{0.786003in}}%
\pgfpathlineto{\pgfqpoint{4.213541in}{0.785854in}}%
\pgfpathlineto{\pgfqpoint{4.213837in}{0.785723in}}%
\pgfpathlineto{\pgfqpoint{4.214133in}{0.785702in}}%
\pgfpathlineto{\pgfqpoint{4.214429in}{0.785699in}}%
\pgfpathlineto{\pgfqpoint{4.214725in}{0.785696in}}%
\pgfpathlineto{\pgfqpoint{4.215021in}{0.785693in}}%
\pgfpathlineto{\pgfqpoint{4.215317in}{0.785690in}}%
\pgfpathlineto{\pgfqpoint{4.215613in}{0.785687in}}%
\pgfpathlineto{\pgfqpoint{4.215909in}{0.785684in}}%
\pgfpathlineto{\pgfqpoint{4.216205in}{0.785681in}}%
\pgfpathlineto{\pgfqpoint{4.216501in}{0.785678in}}%
\pgfpathlineto{\pgfqpoint{4.216797in}{0.785675in}}%
\pgfpathlineto{\pgfqpoint{4.217093in}{0.785672in}}%
\pgfpathlineto{\pgfqpoint{4.217389in}{0.785669in}}%
\pgfpathlineto{\pgfqpoint{4.217685in}{0.785666in}}%
\pgfpathlineto{\pgfqpoint{4.217981in}{0.785664in}}%
\pgfpathlineto{\pgfqpoint{4.218277in}{0.785661in}}%
\pgfpathlineto{\pgfqpoint{4.218573in}{0.785658in}}%
\pgfpathlineto{\pgfqpoint{4.218869in}{0.785655in}}%
\pgfpathlineto{\pgfqpoint{4.219165in}{0.785652in}}%
\pgfpathlineto{\pgfqpoint{4.219461in}{0.785649in}}%
\pgfpathlineto{\pgfqpoint{4.219757in}{0.785646in}}%
\pgfpathlineto{\pgfqpoint{4.220053in}{0.785643in}}%
\pgfpathlineto{\pgfqpoint{4.220349in}{0.785640in}}%
\pgfpathlineto{\pgfqpoint{4.220645in}{0.785637in}}%
\pgfpathlineto{\pgfqpoint{4.220941in}{0.785634in}}%
\pgfpathlineto{\pgfqpoint{4.221238in}{0.785631in}}%
\pgfpathlineto{\pgfqpoint{4.221534in}{0.785628in}}%
\pgfpathlineto{\pgfqpoint{4.221830in}{0.785625in}}%
\pgfpathlineto{\pgfqpoint{4.222126in}{0.785622in}}%
\pgfpathlineto{\pgfqpoint{4.222422in}{0.785619in}}%
\pgfpathlineto{\pgfqpoint{4.222718in}{0.785617in}}%
\pgfpathlineto{\pgfqpoint{4.223014in}{0.785614in}}%
\pgfpathlineto{\pgfqpoint{4.223310in}{0.785611in}}%
\pgfpathlineto{\pgfqpoint{4.223606in}{0.785608in}}%
\pgfpathlineto{\pgfqpoint{4.223902in}{0.785605in}}%
\pgfpathlineto{\pgfqpoint{4.224198in}{0.785602in}}%
\pgfpathlineto{\pgfqpoint{4.224494in}{0.785599in}}%
\pgfpathlineto{\pgfqpoint{4.224790in}{0.785596in}}%
\pgfpathlineto{\pgfqpoint{4.225086in}{0.785593in}}%
\pgfpathlineto{\pgfqpoint{4.225382in}{0.785590in}}%
\pgfpathlineto{\pgfqpoint{4.225678in}{0.785587in}}%
\pgfpathlineto{\pgfqpoint{4.225974in}{0.785584in}}%
\pgfpathlineto{\pgfqpoint{4.226270in}{0.785581in}}%
\pgfpathlineto{\pgfqpoint{4.226566in}{0.785578in}}%
\pgfpathlineto{\pgfqpoint{4.226862in}{0.785575in}}%
\pgfpathlineto{\pgfqpoint{4.227158in}{0.785573in}}%
\pgfpathlineto{\pgfqpoint{4.227454in}{0.785570in}}%
\pgfpathlineto{\pgfqpoint{4.227750in}{0.785567in}}%
\pgfpathlineto{\pgfqpoint{4.228046in}{0.785564in}}%
\pgfpathlineto{\pgfqpoint{4.228342in}{0.785561in}}%
\pgfpathlineto{\pgfqpoint{4.228638in}{0.785558in}}%
\pgfpathlineto{\pgfqpoint{4.228934in}{0.785555in}}%
\pgfpathlineto{\pgfqpoint{4.229230in}{0.785552in}}%
\pgfpathlineto{\pgfqpoint{4.229526in}{0.785550in}}%
\pgfpathlineto{\pgfqpoint{4.229822in}{0.785547in}}%
\pgfpathlineto{\pgfqpoint{4.230118in}{0.785544in}}%
\pgfpathlineto{\pgfqpoint{4.230414in}{0.785541in}}%
\pgfpathlineto{\pgfqpoint{4.230710in}{0.785538in}}%
\pgfpathlineto{\pgfqpoint{4.231006in}{0.785535in}}%
\pgfpathlineto{\pgfqpoint{4.231302in}{0.785532in}}%
\pgfpathlineto{\pgfqpoint{4.231598in}{0.785529in}}%
\pgfpathlineto{\pgfqpoint{4.231894in}{0.785526in}}%
\pgfpathlineto{\pgfqpoint{4.232190in}{0.785524in}}%
\pgfpathlineto{\pgfqpoint{4.232486in}{0.785521in}}%
\pgfpathlineto{\pgfqpoint{4.232782in}{0.785518in}}%
\pgfpathlineto{\pgfqpoint{4.233078in}{0.785515in}}%
\pgfpathlineto{\pgfqpoint{4.233374in}{0.785512in}}%
\pgfpathlineto{\pgfqpoint{4.233670in}{0.785509in}}%
\pgfpathlineto{\pgfqpoint{4.233966in}{0.785506in}}%
\pgfpathlineto{\pgfqpoint{4.234262in}{0.785503in}}%
\pgfpathlineto{\pgfqpoint{4.234558in}{0.785501in}}%
\pgfpathlineto{\pgfqpoint{4.234854in}{0.785498in}}%
\pgfpathlineto{\pgfqpoint{4.235150in}{0.785495in}}%
\pgfpathlineto{\pgfqpoint{4.235446in}{0.785492in}}%
\pgfpathlineto{\pgfqpoint{4.235742in}{0.785489in}}%
\pgfpathlineto{\pgfqpoint{4.236038in}{0.785486in}}%
\pgfpathlineto{\pgfqpoint{4.236334in}{0.785483in}}%
\pgfpathlineto{\pgfqpoint{4.236630in}{0.785480in}}%
\pgfpathlineto{\pgfqpoint{4.236926in}{0.785477in}}%
\pgfpathlineto{\pgfqpoint{4.237222in}{0.785475in}}%
\pgfpathlineto{\pgfqpoint{4.237518in}{0.785472in}}%
\pgfpathlineto{\pgfqpoint{4.237814in}{0.785469in}}%
\pgfpathlineto{\pgfqpoint{4.238110in}{0.785466in}}%
\pgfpathlineto{\pgfqpoint{4.238406in}{0.785463in}}%
\pgfpathlineto{\pgfqpoint{4.238702in}{0.785460in}}%
\pgfpathlineto{\pgfqpoint{4.238998in}{0.785457in}}%
\pgfpathlineto{\pgfqpoint{4.239294in}{0.785454in}}%
\pgfpathlineto{\pgfqpoint{4.239590in}{0.785452in}}%
\pgfpathlineto{\pgfqpoint{4.239886in}{0.785449in}}%
\pgfpathlineto{\pgfqpoint{4.240182in}{0.785446in}}%
\pgfpathlineto{\pgfqpoint{4.240478in}{0.785443in}}%
\pgfpathlineto{\pgfqpoint{4.240774in}{0.785440in}}%
\pgfpathlineto{\pgfqpoint{4.241070in}{0.785437in}}%
\pgfpathlineto{\pgfqpoint{4.241366in}{0.785434in}}%
\pgfpathlineto{\pgfqpoint{4.241662in}{0.785431in}}%
\pgfpathlineto{\pgfqpoint{4.241958in}{0.785428in}}%
\pgfpathlineto{\pgfqpoint{4.242254in}{0.785426in}}%
\pgfpathlineto{\pgfqpoint{4.242550in}{0.785423in}}%
\pgfpathlineto{\pgfqpoint{4.242846in}{0.785420in}}%
\pgfpathlineto{\pgfqpoint{4.243142in}{0.785417in}}%
\pgfpathlineto{\pgfqpoint{4.243438in}{0.785414in}}%
\pgfpathlineto{\pgfqpoint{4.243734in}{0.785411in}}%
\pgfpathlineto{\pgfqpoint{4.244030in}{0.785408in}}%
\pgfpathlineto{\pgfqpoint{4.244326in}{0.785405in}}%
\pgfpathlineto{\pgfqpoint{4.244622in}{0.785403in}}%
\pgfpathlineto{\pgfqpoint{4.244918in}{0.785400in}}%
\pgfpathlineto{\pgfqpoint{4.245214in}{0.785397in}}%
\pgfpathlineto{\pgfqpoint{4.245510in}{0.785394in}}%
\pgfpathlineto{\pgfqpoint{4.245806in}{0.785391in}}%
\pgfpathlineto{\pgfqpoint{4.246102in}{0.785388in}}%
\pgfpathlineto{\pgfqpoint{4.246398in}{0.785385in}}%
\pgfpathlineto{\pgfqpoint{4.246694in}{0.785382in}}%
\pgfpathlineto{\pgfqpoint{4.246990in}{0.785379in}}%
\pgfpathlineto{\pgfqpoint{4.247286in}{0.785377in}}%
\pgfpathlineto{\pgfqpoint{4.247582in}{0.785374in}}%
\pgfpathlineto{\pgfqpoint{4.247878in}{0.785371in}}%
\pgfpathlineto{\pgfqpoint{4.248174in}{0.785368in}}%
\pgfpathlineto{\pgfqpoint{4.248470in}{0.785365in}}%
\pgfpathlineto{\pgfqpoint{4.248766in}{0.785362in}}%
\pgfpathlineto{\pgfqpoint{4.249062in}{0.785359in}}%
\pgfpathlineto{\pgfqpoint{4.249358in}{0.785356in}}%
\pgfpathlineto{\pgfqpoint{4.249654in}{0.785354in}}%
\pgfpathlineto{\pgfqpoint{4.249950in}{0.785351in}}%
\pgfpathlineto{\pgfqpoint{4.250246in}{0.785348in}}%
\pgfpathlineto{\pgfqpoint{4.250542in}{0.785345in}}%
\pgfpathlineto{\pgfqpoint{4.250838in}{0.785342in}}%
\pgfpathlineto{\pgfqpoint{4.251134in}{0.785339in}}%
\pgfpathlineto{\pgfqpoint{4.251430in}{0.785336in}}%
\pgfpathlineto{\pgfqpoint{4.251726in}{0.785333in}}%
\pgfpathlineto{\pgfqpoint{4.252022in}{0.785331in}}%
\pgfpathlineto{\pgfqpoint{4.252318in}{0.785328in}}%
\pgfpathlineto{\pgfqpoint{4.252614in}{0.785325in}}%
\pgfpathlineto{\pgfqpoint{4.252910in}{0.785322in}}%
\pgfpathlineto{\pgfqpoint{4.253206in}{0.785319in}}%
\pgfpathlineto{\pgfqpoint{4.253502in}{0.785316in}}%
\pgfpathlineto{\pgfqpoint{4.253798in}{0.785313in}}%
\pgfpathlineto{\pgfqpoint{4.254094in}{0.785310in}}%
\pgfpathlineto{\pgfqpoint{4.254390in}{0.785307in}}%
\pgfpathlineto{\pgfqpoint{4.254686in}{0.785305in}}%
\pgfpathlineto{\pgfqpoint{4.254982in}{0.785302in}}%
\pgfpathlineto{\pgfqpoint{4.255278in}{0.785299in}}%
\pgfpathlineto{\pgfqpoint{4.255574in}{0.785296in}}%
\pgfpathlineto{\pgfqpoint{4.255870in}{0.785293in}}%
\pgfpathlineto{\pgfqpoint{4.256166in}{0.785290in}}%
\pgfpathlineto{\pgfqpoint{4.256462in}{0.785287in}}%
\pgfpathlineto{\pgfqpoint{4.256758in}{0.785284in}}%
\pgfpathlineto{\pgfqpoint{4.257054in}{0.785282in}}%
\pgfpathlineto{\pgfqpoint{4.257350in}{0.785279in}}%
\pgfpathlineto{\pgfqpoint{4.257646in}{0.785276in}}%
\pgfpathlineto{\pgfqpoint{4.257942in}{0.785273in}}%
\pgfpathlineto{\pgfqpoint{4.258238in}{0.785270in}}%
\pgfpathlineto{\pgfqpoint{4.258534in}{0.785267in}}%
\pgfpathlineto{\pgfqpoint{4.258830in}{0.785264in}}%
\pgfpathlineto{\pgfqpoint{4.259126in}{0.785261in}}%
\pgfpathlineto{\pgfqpoint{4.259422in}{0.785258in}}%
\pgfpathlineto{\pgfqpoint{4.259718in}{0.785256in}}%
\pgfpathlineto{\pgfqpoint{4.260014in}{0.785253in}}%
\pgfpathlineto{\pgfqpoint{4.260310in}{0.785250in}}%
\pgfpathlineto{\pgfqpoint{4.260606in}{0.785247in}}%
\pgfpathlineto{\pgfqpoint{4.260902in}{0.785244in}}%
\pgfpathlineto{\pgfqpoint{4.261198in}{0.785241in}}%
\pgfpathlineto{\pgfqpoint{4.261494in}{0.785238in}}%
\pgfpathlineto{\pgfqpoint{4.261790in}{0.785235in}}%
\pgfpathlineto{\pgfqpoint{4.262086in}{0.785233in}}%
\pgfpathlineto{\pgfqpoint{4.262382in}{0.785230in}}%
\pgfpathlineto{\pgfqpoint{4.262678in}{0.785227in}}%
\pgfpathlineto{\pgfqpoint{4.262974in}{0.785224in}}%
\pgfpathlineto{\pgfqpoint{4.263270in}{0.785221in}}%
\pgfpathlineto{\pgfqpoint{4.263566in}{0.785245in}}%
\pgfpathlineto{\pgfqpoint{4.263862in}{0.785233in}}%
\pgfpathlineto{\pgfqpoint{4.264158in}{0.785228in}}%
\pgfpathlineto{\pgfqpoint{4.264454in}{0.785223in}}%
\pgfpathlineto{\pgfqpoint{4.264750in}{0.785212in}}%
\pgfpathlineto{\pgfqpoint{4.265046in}{0.785186in}}%
\pgfpathlineto{\pgfqpoint{4.265342in}{0.785161in}}%
\pgfpathlineto{\pgfqpoint{4.265638in}{0.785135in}}%
\pgfpathlineto{\pgfqpoint{4.265934in}{0.785109in}}%
\pgfpathlineto{\pgfqpoint{4.266230in}{0.785084in}}%
\pgfpathlineto{\pgfqpoint{4.266526in}{0.785058in}}%
\pgfpathlineto{\pgfqpoint{4.266822in}{0.785033in}}%
\pgfpathlineto{\pgfqpoint{4.267118in}{0.785007in}}%
\pgfpathlineto{\pgfqpoint{4.267414in}{0.784981in}}%
\pgfpathlineto{\pgfqpoint{4.267710in}{0.784956in}}%
\pgfpathlineto{\pgfqpoint{4.268006in}{0.784930in}}%
\pgfpathlineto{\pgfqpoint{4.268302in}{0.784904in}}%
\pgfpathlineto{\pgfqpoint{4.268598in}{0.784879in}}%
\pgfpathlineto{\pgfqpoint{4.268894in}{0.784853in}}%
\pgfpathlineto{\pgfqpoint{4.269190in}{0.784829in}}%
\pgfpathlineto{\pgfqpoint{4.269486in}{0.784809in}}%
\pgfpathlineto{\pgfqpoint{4.269782in}{0.784789in}}%
\pgfpathlineto{\pgfqpoint{4.270078in}{0.784770in}}%
\pgfpathlineto{\pgfqpoint{4.270374in}{0.784751in}}%
\pgfpathlineto{\pgfqpoint{4.270670in}{0.784731in}}%
\pgfpathlineto{\pgfqpoint{4.270966in}{0.784713in}}%
\pgfpathlineto{\pgfqpoint{4.271262in}{0.784702in}}%
\pgfpathlineto{\pgfqpoint{4.271558in}{0.784721in}}%
\pgfpathlineto{\pgfqpoint{4.271854in}{0.784702in}}%
\pgfpathlineto{\pgfqpoint{4.272150in}{0.784684in}}%
\pgfpathlineto{\pgfqpoint{4.272446in}{0.784665in}}%
\pgfpathlineto{\pgfqpoint{4.272742in}{0.784647in}}%
\pgfpathlineto{\pgfqpoint{4.273038in}{0.784628in}}%
\pgfpathlineto{\pgfqpoint{4.273334in}{0.784610in}}%
\pgfpathlineto{\pgfqpoint{4.273630in}{0.784591in}}%
\pgfpathlineto{\pgfqpoint{4.273926in}{0.784573in}}%
\pgfpathlineto{\pgfqpoint{4.274222in}{0.784554in}}%
\pgfpathlineto{\pgfqpoint{4.274518in}{0.784536in}}%
\pgfpathlineto{\pgfqpoint{4.274814in}{0.784517in}}%
\pgfpathlineto{\pgfqpoint{4.275110in}{0.784499in}}%
\pgfpathlineto{\pgfqpoint{4.275406in}{0.784480in}}%
\pgfpathlineto{\pgfqpoint{4.275702in}{0.784462in}}%
\pgfpathlineto{\pgfqpoint{4.275998in}{0.784443in}}%
\pgfpathlineto{\pgfqpoint{4.276294in}{0.784425in}}%
\pgfpathlineto{\pgfqpoint{4.276590in}{0.784406in}}%
\pgfpathlineto{\pgfqpoint{4.276886in}{0.784388in}}%
\pgfpathlineto{\pgfqpoint{4.277182in}{0.784369in}}%
\pgfpathlineto{\pgfqpoint{4.277478in}{0.784351in}}%
\pgfpathlineto{\pgfqpoint{4.277774in}{0.784332in}}%
\pgfpathlineto{\pgfqpoint{4.278070in}{0.784314in}}%
\pgfpathlineto{\pgfqpoint{4.278366in}{0.784300in}}%
\pgfpathlineto{\pgfqpoint{4.278662in}{0.784282in}}%
\pgfpathlineto{\pgfqpoint{4.278958in}{0.784267in}}%
\pgfpathlineto{\pgfqpoint{4.279254in}{0.784261in}}%
\pgfpathlineto{\pgfqpoint{4.279550in}{0.784256in}}%
\pgfpathlineto{\pgfqpoint{4.279846in}{0.784250in}}%
\pgfpathlineto{\pgfqpoint{4.280142in}{0.784245in}}%
\pgfpathlineto{\pgfqpoint{4.280438in}{0.784240in}}%
\pgfpathlineto{\pgfqpoint{4.280734in}{0.784234in}}%
\pgfpathlineto{\pgfqpoint{4.281030in}{0.784229in}}%
\pgfpathlineto{\pgfqpoint{4.281326in}{0.784223in}}%
\pgfpathlineto{\pgfqpoint{4.281622in}{0.784218in}}%
\pgfpathlineto{\pgfqpoint{4.281918in}{0.784212in}}%
\pgfpathlineto{\pgfqpoint{4.282214in}{0.784207in}}%
\pgfpathlineto{\pgfqpoint{4.282510in}{0.784201in}}%
\pgfpathlineto{\pgfqpoint{4.282806in}{0.784196in}}%
\pgfpathlineto{\pgfqpoint{4.283102in}{0.784190in}}%
\pgfpathlineto{\pgfqpoint{4.283398in}{0.784185in}}%
\pgfpathlineto{\pgfqpoint{4.283694in}{0.784179in}}%
\pgfpathlineto{\pgfqpoint{4.283990in}{0.784174in}}%
\pgfpathlineto{\pgfqpoint{4.284286in}{0.784169in}}%
\pgfpathlineto{\pgfqpoint{4.284582in}{0.784163in}}%
\pgfpathlineto{\pgfqpoint{4.284878in}{0.784158in}}%
\pgfpathlineto{\pgfqpoint{4.285174in}{0.784152in}}%
\pgfpathlineto{\pgfqpoint{4.285470in}{0.784147in}}%
\pgfpathlineto{\pgfqpoint{4.285766in}{0.784141in}}%
\pgfpathlineto{\pgfqpoint{4.286062in}{0.784136in}}%
\pgfpathlineto{\pgfqpoint{4.286358in}{0.784130in}}%
\pgfpathlineto{\pgfqpoint{4.286654in}{0.784125in}}%
\pgfpathlineto{\pgfqpoint{4.286950in}{0.784119in}}%
\pgfpathlineto{\pgfqpoint{4.287246in}{0.784114in}}%
\pgfpathlineto{\pgfqpoint{4.287542in}{0.784108in}}%
\pgfpathlineto{\pgfqpoint{4.287838in}{0.784103in}}%
\pgfpathlineto{\pgfqpoint{4.288134in}{0.784098in}}%
\pgfpathlineto{\pgfqpoint{4.288431in}{0.784092in}}%
\pgfpathlineto{\pgfqpoint{4.288727in}{0.784087in}}%
\pgfpathlineto{\pgfqpoint{4.289023in}{0.784081in}}%
\pgfpathlineto{\pgfqpoint{4.289319in}{0.784076in}}%
\pgfpathlineto{\pgfqpoint{4.289615in}{0.784070in}}%
\pgfpathlineto{\pgfqpoint{4.289911in}{0.784065in}}%
\pgfpathlineto{\pgfqpoint{4.290207in}{0.784059in}}%
\pgfpathlineto{\pgfqpoint{4.290503in}{0.784054in}}%
\pgfpathlineto{\pgfqpoint{4.290799in}{0.784048in}}%
\pgfpathlineto{\pgfqpoint{4.291095in}{0.784043in}}%
\pgfpathlineto{\pgfqpoint{4.291391in}{0.784037in}}%
\pgfpathlineto{\pgfqpoint{4.291687in}{0.784032in}}%
\pgfpathlineto{\pgfqpoint{4.291983in}{0.784027in}}%
\pgfpathlineto{\pgfqpoint{4.292279in}{0.784021in}}%
\pgfpathlineto{\pgfqpoint{4.292575in}{0.784016in}}%
\pgfpathlineto{\pgfqpoint{4.292871in}{0.784010in}}%
\pgfpathlineto{\pgfqpoint{4.293167in}{0.784005in}}%
\pgfpathlineto{\pgfqpoint{4.293463in}{0.783999in}}%
\pgfpathlineto{\pgfqpoint{4.293759in}{0.783994in}}%
\pgfpathlineto{\pgfqpoint{4.294055in}{0.783988in}}%
\pgfpathlineto{\pgfqpoint{4.294351in}{0.783983in}}%
\pgfpathlineto{\pgfqpoint{4.294647in}{0.783977in}}%
\pgfpathlineto{\pgfqpoint{4.294943in}{0.783972in}}%
\pgfpathlineto{\pgfqpoint{4.295239in}{0.783966in}}%
\pgfpathlineto{\pgfqpoint{4.295535in}{0.783961in}}%
\pgfpathlineto{\pgfqpoint{4.295831in}{0.783956in}}%
\pgfpathlineto{\pgfqpoint{4.296127in}{0.783950in}}%
\pgfpathlineto{\pgfqpoint{4.296423in}{0.783945in}}%
\pgfpathlineto{\pgfqpoint{4.296719in}{0.783939in}}%
\pgfpathlineto{\pgfqpoint{4.297015in}{0.783934in}}%
\pgfpathlineto{\pgfqpoint{4.297311in}{0.783928in}}%
\pgfpathlineto{\pgfqpoint{4.297607in}{0.783923in}}%
\pgfpathlineto{\pgfqpoint{4.297903in}{0.783918in}}%
\pgfpathlineto{\pgfqpoint{4.298199in}{0.783913in}}%
\pgfpathlineto{\pgfqpoint{4.298495in}{0.783907in}}%
\pgfpathlineto{\pgfqpoint{4.298791in}{0.783902in}}%
\pgfpathlineto{\pgfqpoint{4.299087in}{0.783897in}}%
\pgfpathlineto{\pgfqpoint{4.299383in}{0.783892in}}%
\pgfpathlineto{\pgfqpoint{4.299679in}{0.783887in}}%
\pgfpathlineto{\pgfqpoint{4.299975in}{0.783882in}}%
\pgfpathlineto{\pgfqpoint{4.300271in}{0.783877in}}%
\pgfpathlineto{\pgfqpoint{4.300567in}{0.783872in}}%
\pgfpathlineto{\pgfqpoint{4.300863in}{0.783867in}}%
\pgfpathlineto{\pgfqpoint{4.301159in}{0.783862in}}%
\pgfpathlineto{\pgfqpoint{4.301455in}{0.783857in}}%
\pgfpathlineto{\pgfqpoint{4.301751in}{0.783851in}}%
\pgfpathlineto{\pgfqpoint{4.302047in}{0.783846in}}%
\pgfpathlineto{\pgfqpoint{4.302343in}{0.783841in}}%
\pgfpathlineto{\pgfqpoint{4.302639in}{0.783836in}}%
\pgfpathlineto{\pgfqpoint{4.302935in}{0.783831in}}%
\pgfpathlineto{\pgfqpoint{4.303231in}{0.783826in}}%
\pgfpathlineto{\pgfqpoint{4.303527in}{0.783821in}}%
\pgfpathlineto{\pgfqpoint{4.303823in}{0.783816in}}%
\pgfpathlineto{\pgfqpoint{4.304119in}{0.783811in}}%
\pgfpathlineto{\pgfqpoint{4.304415in}{0.783806in}}%
\pgfpathlineto{\pgfqpoint{4.304711in}{0.783801in}}%
\pgfpathlineto{\pgfqpoint{4.305007in}{0.783796in}}%
\pgfpathlineto{\pgfqpoint{4.305303in}{0.783790in}}%
\pgfpathlineto{\pgfqpoint{4.305599in}{0.783785in}}%
\pgfpathlineto{\pgfqpoint{4.305895in}{0.783780in}}%
\pgfpathlineto{\pgfqpoint{4.306191in}{0.783775in}}%
\pgfpathlineto{\pgfqpoint{4.306487in}{0.783770in}}%
\pgfpathlineto{\pgfqpoint{4.306783in}{0.783763in}}%
\pgfpathlineto{\pgfqpoint{4.307079in}{0.783737in}}%
\pgfpathlineto{\pgfqpoint{4.307375in}{0.783708in}}%
\pgfpathlineto{\pgfqpoint{4.307671in}{0.783679in}}%
\pgfpathlineto{\pgfqpoint{4.307967in}{0.783650in}}%
\pgfpathlineto{\pgfqpoint{4.308263in}{0.783621in}}%
\pgfpathlineto{\pgfqpoint{4.308559in}{0.783592in}}%
\pgfpathlineto{\pgfqpoint{4.308855in}{0.783563in}}%
\pgfpathlineto{\pgfqpoint{4.309151in}{0.783533in}}%
\pgfpathlineto{\pgfqpoint{4.309447in}{0.783504in}}%
\pgfpathlineto{\pgfqpoint{4.309743in}{0.783475in}}%
\pgfpathlineto{\pgfqpoint{4.310039in}{0.783446in}}%
\pgfpathlineto{\pgfqpoint{4.310335in}{0.783417in}}%
\pgfpathlineto{\pgfqpoint{4.310631in}{0.783388in}}%
\pgfpathlineto{\pgfqpoint{4.310927in}{0.783359in}}%
\pgfpathlineto{\pgfqpoint{4.311223in}{0.783330in}}%
\pgfpathlineto{\pgfqpoint{4.311519in}{0.783301in}}%
\pgfpathlineto{\pgfqpoint{4.311815in}{0.783272in}}%
\pgfpathlineto{\pgfqpoint{4.312111in}{0.783260in}}%
\pgfpathlineto{\pgfqpoint{4.312407in}{0.783259in}}%
\pgfpathlineto{\pgfqpoint{4.312703in}{0.783258in}}%
\pgfpathlineto{\pgfqpoint{4.312999in}{0.783256in}}%
\pgfpathlineto{\pgfqpoint{4.313295in}{0.783255in}}%
\pgfpathlineto{\pgfqpoint{4.313591in}{0.783254in}}%
\pgfpathlineto{\pgfqpoint{4.313887in}{0.783253in}}%
\pgfpathlineto{\pgfqpoint{4.314183in}{0.783252in}}%
\pgfpathlineto{\pgfqpoint{4.314479in}{0.783251in}}%
\pgfpathlineto{\pgfqpoint{4.314775in}{0.783250in}}%
\pgfpathlineto{\pgfqpoint{4.315071in}{0.783249in}}%
\pgfpathlineto{\pgfqpoint{4.315367in}{0.783248in}}%
\pgfpathlineto{\pgfqpoint{4.315663in}{0.783247in}}%
\pgfpathlineto{\pgfqpoint{4.315959in}{0.783246in}}%
\pgfpathlineto{\pgfqpoint{4.316255in}{0.783245in}}%
\pgfpathlineto{\pgfqpoint{4.316551in}{0.783244in}}%
\pgfpathlineto{\pgfqpoint{4.316847in}{0.783242in}}%
\pgfpathlineto{\pgfqpoint{4.317143in}{0.783241in}}%
\pgfpathlineto{\pgfqpoint{4.317439in}{0.783240in}}%
\pgfpathlineto{\pgfqpoint{4.317735in}{0.783239in}}%
\pgfpathlineto{\pgfqpoint{4.318031in}{0.783238in}}%
\pgfpathlineto{\pgfqpoint{4.318327in}{0.783237in}}%
\pgfpathlineto{\pgfqpoint{4.318623in}{0.783236in}}%
\pgfpathlineto{\pgfqpoint{4.318919in}{0.783235in}}%
\pgfpathlineto{\pgfqpoint{4.319215in}{0.783234in}}%
\pgfpathlineto{\pgfqpoint{4.319511in}{0.783233in}}%
\pgfpathlineto{\pgfqpoint{4.319807in}{0.783232in}}%
\pgfpathlineto{\pgfqpoint{4.320103in}{0.783231in}}%
\pgfpathlineto{\pgfqpoint{4.320399in}{0.783230in}}%
\pgfpathlineto{\pgfqpoint{4.320695in}{0.783228in}}%
\pgfpathlineto{\pgfqpoint{4.320991in}{0.783228in}}%
\pgfpathlineto{\pgfqpoint{4.321287in}{0.783228in}}%
\pgfpathlineto{\pgfqpoint{4.321583in}{0.783228in}}%
\pgfpathlineto{\pgfqpoint{4.321879in}{0.783228in}}%
\pgfpathlineto{\pgfqpoint{4.322175in}{0.783228in}}%
\pgfpathlineto{\pgfqpoint{4.322471in}{0.783228in}}%
\pgfpathlineto{\pgfqpoint{4.322767in}{0.783228in}}%
\pgfpathlineto{\pgfqpoint{4.323063in}{0.783228in}}%
\pgfpathlineto{\pgfqpoint{4.323359in}{0.783228in}}%
\pgfpathlineto{\pgfqpoint{4.323655in}{0.783227in}}%
\pgfpathlineto{\pgfqpoint{4.323951in}{0.783227in}}%
\pgfpathlineto{\pgfqpoint{4.324247in}{0.783227in}}%
\pgfpathlineto{\pgfqpoint{4.324543in}{0.783227in}}%
\pgfpathlineto{\pgfqpoint{4.324839in}{0.783227in}}%
\pgfpathlineto{\pgfqpoint{4.325135in}{0.783227in}}%
\pgfpathlineto{\pgfqpoint{4.325431in}{0.783227in}}%
\pgfpathlineto{\pgfqpoint{4.325727in}{0.783227in}}%
\pgfpathlineto{\pgfqpoint{4.326023in}{0.783227in}}%
\pgfpathlineto{\pgfqpoint{4.326319in}{0.783227in}}%
\pgfpathlineto{\pgfqpoint{4.326615in}{0.783227in}}%
\pgfpathlineto{\pgfqpoint{4.326911in}{0.783227in}}%
\pgfpathlineto{\pgfqpoint{4.327207in}{0.783229in}}%
\pgfpathlineto{\pgfqpoint{4.327503in}{0.783240in}}%
\pgfpathlineto{\pgfqpoint{4.327799in}{0.783251in}}%
\pgfpathlineto{\pgfqpoint{4.328095in}{0.783263in}}%
\pgfpathlineto{\pgfqpoint{4.328391in}{0.783274in}}%
\pgfpathlineto{\pgfqpoint{4.328687in}{0.783286in}}%
\pgfpathlineto{\pgfqpoint{4.328983in}{0.783298in}}%
\pgfpathlineto{\pgfqpoint{4.329279in}{0.783309in}}%
\pgfpathlineto{\pgfqpoint{4.329575in}{0.783321in}}%
\pgfpathlineto{\pgfqpoint{4.329871in}{0.783333in}}%
\pgfpathlineto{\pgfqpoint{4.330167in}{0.783344in}}%
\pgfpathlineto{\pgfqpoint{4.330463in}{0.783356in}}%
\pgfpathlineto{\pgfqpoint{4.330759in}{0.783368in}}%
\pgfpathlineto{\pgfqpoint{4.331055in}{0.783379in}}%
\pgfpathlineto{\pgfqpoint{4.331351in}{0.783391in}}%
\pgfpathlineto{\pgfqpoint{4.331647in}{0.783402in}}%
\pgfpathlineto{\pgfqpoint{4.331943in}{0.783414in}}%
\pgfpathlineto{\pgfqpoint{4.332239in}{0.783426in}}%
\pgfpathlineto{\pgfqpoint{4.332535in}{0.783437in}}%
\pgfpathlineto{\pgfqpoint{4.332831in}{0.783449in}}%
\pgfpathlineto{\pgfqpoint{4.333127in}{0.783461in}}%
\pgfpathlineto{\pgfqpoint{4.333423in}{0.783472in}}%
\pgfpathlineto{\pgfqpoint{4.333719in}{0.783484in}}%
\pgfpathlineto{\pgfqpoint{4.334015in}{0.783496in}}%
\pgfpathlineto{\pgfqpoint{4.334311in}{0.783507in}}%
\pgfpathlineto{\pgfqpoint{4.334607in}{0.783519in}}%
\pgfpathlineto{\pgfqpoint{4.334903in}{0.783530in}}%
\pgfpathlineto{\pgfqpoint{4.335199in}{0.783542in}}%
\pgfpathlineto{\pgfqpoint{4.335495in}{0.783554in}}%
\pgfpathlineto{\pgfqpoint{4.335791in}{0.783565in}}%
\pgfpathlineto{\pgfqpoint{4.336087in}{0.783577in}}%
\pgfpathlineto{\pgfqpoint{4.336383in}{0.783589in}}%
\pgfpathlineto{\pgfqpoint{4.336679in}{0.783600in}}%
\pgfpathlineto{\pgfqpoint{4.336975in}{0.783612in}}%
\pgfpathlineto{\pgfqpoint{4.337271in}{0.783624in}}%
\pgfpathlineto{\pgfqpoint{4.337567in}{0.783635in}}%
\pgfpathlineto{\pgfqpoint{4.337863in}{0.783647in}}%
\pgfpathlineto{\pgfqpoint{4.338159in}{0.783658in}}%
\pgfpathlineto{\pgfqpoint{4.338455in}{0.783670in}}%
\pgfpathlineto{\pgfqpoint{4.338751in}{0.783682in}}%
\pgfpathlineto{\pgfqpoint{4.339047in}{0.783693in}}%
\pgfpathlineto{\pgfqpoint{4.339343in}{0.783705in}}%
\pgfpathlineto{\pgfqpoint{4.339639in}{0.783717in}}%
\pgfpathlineto{\pgfqpoint{4.339935in}{0.783728in}}%
\pgfpathlineto{\pgfqpoint{4.340231in}{0.783740in}}%
\pgfpathlineto{\pgfqpoint{4.340527in}{0.783751in}}%
\pgfpathlineto{\pgfqpoint{4.340823in}{0.783763in}}%
\pgfpathlineto{\pgfqpoint{4.341119in}{0.783775in}}%
\pgfpathlineto{\pgfqpoint{4.341415in}{0.783786in}}%
\pgfpathlineto{\pgfqpoint{4.341711in}{0.783798in}}%
\pgfpathlineto{\pgfqpoint{4.342007in}{0.783810in}}%
\pgfpathlineto{\pgfqpoint{4.342303in}{0.783821in}}%
\pgfpathlineto{\pgfqpoint{4.342599in}{0.783833in}}%
\pgfpathlineto{\pgfqpoint{4.342895in}{0.783845in}}%
\pgfpathlineto{\pgfqpoint{4.343191in}{0.783856in}}%
\pgfpathlineto{\pgfqpoint{4.343487in}{0.783868in}}%
\pgfpathlineto{\pgfqpoint{4.343783in}{0.783879in}}%
\pgfpathlineto{\pgfqpoint{4.344079in}{0.783891in}}%
\pgfpathlineto{\pgfqpoint{4.344375in}{0.783903in}}%
\pgfpathlineto{\pgfqpoint{4.344671in}{0.783914in}}%
\pgfpathlineto{\pgfqpoint{4.344967in}{0.783926in}}%
\pgfpathlineto{\pgfqpoint{4.345263in}{0.783938in}}%
\pgfpathlineto{\pgfqpoint{4.345559in}{0.783949in}}%
\pgfpathlineto{\pgfqpoint{4.345855in}{0.783961in}}%
\pgfpathlineto{\pgfqpoint{4.346151in}{0.783973in}}%
\pgfpathlineto{\pgfqpoint{4.346447in}{0.783984in}}%
\pgfpathlineto{\pgfqpoint{4.346743in}{0.783996in}}%
\pgfpathlineto{\pgfqpoint{4.347039in}{0.784007in}}%
\pgfpathlineto{\pgfqpoint{4.347335in}{0.784019in}}%
\pgfpathlineto{\pgfqpoint{4.347631in}{0.784031in}}%
\pgfpathlineto{\pgfqpoint{4.347927in}{0.784042in}}%
\pgfpathlineto{\pgfqpoint{4.348223in}{0.784054in}}%
\pgfpathlineto{\pgfqpoint{4.348519in}{0.784066in}}%
\pgfpathlineto{\pgfqpoint{4.348815in}{0.784077in}}%
\pgfpathlineto{\pgfqpoint{4.349111in}{0.784089in}}%
\pgfpathlineto{\pgfqpoint{4.349407in}{0.784101in}}%
\pgfpathlineto{\pgfqpoint{4.349703in}{0.784112in}}%
\pgfpathlineto{\pgfqpoint{4.349999in}{0.784124in}}%
\pgfpathlineto{\pgfqpoint{4.350295in}{0.784135in}}%
\pgfpathlineto{\pgfqpoint{4.350591in}{0.784147in}}%
\pgfpathlineto{\pgfqpoint{4.350887in}{0.784159in}}%
\pgfpathlineto{\pgfqpoint{4.351183in}{0.784170in}}%
\pgfpathlineto{\pgfqpoint{4.351479in}{0.784182in}}%
\pgfpathlineto{\pgfqpoint{4.351775in}{0.784194in}}%
\pgfpathlineto{\pgfqpoint{4.352071in}{0.784205in}}%
\pgfpathlineto{\pgfqpoint{4.352367in}{0.784217in}}%
\pgfpathlineto{\pgfqpoint{4.352663in}{0.784229in}}%
\pgfpathlineto{\pgfqpoint{4.352959in}{0.784240in}}%
\pgfpathlineto{\pgfqpoint{4.353255in}{0.784252in}}%
\pgfpathlineto{\pgfqpoint{4.353551in}{0.784263in}}%
\pgfpathlineto{\pgfqpoint{4.353847in}{0.784275in}}%
\pgfpathlineto{\pgfqpoint{4.354143in}{0.784287in}}%
\pgfpathlineto{\pgfqpoint{4.354439in}{0.784298in}}%
\pgfpathlineto{\pgfqpoint{4.354735in}{0.784476in}}%
\pgfpathlineto{\pgfqpoint{4.355031in}{0.784632in}}%
\pgfpathlineto{\pgfqpoint{4.355327in}{0.784631in}}%
\pgfpathlineto{\pgfqpoint{4.355623in}{0.784630in}}%
\pgfpathlineto{\pgfqpoint{4.355920in}{0.784629in}}%
\pgfpathlineto{\pgfqpoint{4.356216in}{0.784627in}}%
\pgfpathlineto{\pgfqpoint{4.356512in}{0.784626in}}%
\pgfpathlineto{\pgfqpoint{4.356808in}{0.784625in}}%
\pgfpathlineto{\pgfqpoint{4.357104in}{0.784623in}}%
\pgfpathlineto{\pgfqpoint{4.357400in}{0.784622in}}%
\pgfpathlineto{\pgfqpoint{4.357696in}{0.784621in}}%
\pgfpathlineto{\pgfqpoint{4.357992in}{0.784619in}}%
\pgfpathlineto{\pgfqpoint{4.358288in}{0.784618in}}%
\pgfpathlineto{\pgfqpoint{4.358584in}{0.784617in}}%
\pgfpathlineto{\pgfqpoint{4.358880in}{0.784615in}}%
\pgfpathlineto{\pgfqpoint{4.359176in}{0.784614in}}%
\pgfpathlineto{\pgfqpoint{4.359472in}{0.784613in}}%
\pgfpathlineto{\pgfqpoint{4.359768in}{0.784612in}}%
\pgfpathlineto{\pgfqpoint{4.360064in}{0.784610in}}%
\pgfpathlineto{\pgfqpoint{4.360360in}{0.784609in}}%
\pgfpathlineto{\pgfqpoint{4.360656in}{0.784608in}}%
\pgfpathlineto{\pgfqpoint{4.360952in}{0.784606in}}%
\pgfpathlineto{\pgfqpoint{4.361248in}{0.784605in}}%
\pgfpathlineto{\pgfqpoint{4.361544in}{0.784604in}}%
\pgfpathlineto{\pgfqpoint{4.361840in}{0.784602in}}%
\pgfpathlineto{\pgfqpoint{4.362136in}{0.784601in}}%
\pgfpathlineto{\pgfqpoint{4.362432in}{0.784600in}}%
\pgfpathlineto{\pgfqpoint{4.362728in}{0.784598in}}%
\pgfpathlineto{\pgfqpoint{4.363024in}{0.784597in}}%
\pgfpathlineto{\pgfqpoint{4.363320in}{0.784596in}}%
\pgfpathlineto{\pgfqpoint{4.363616in}{0.784594in}}%
\pgfpathlineto{\pgfqpoint{4.363912in}{0.784593in}}%
\pgfpathlineto{\pgfqpoint{4.364208in}{0.784592in}}%
\pgfpathlineto{\pgfqpoint{4.364504in}{0.784591in}}%
\pgfpathlineto{\pgfqpoint{4.364800in}{0.784589in}}%
\pgfpathlineto{\pgfqpoint{4.365096in}{0.784588in}}%
\pgfpathlineto{\pgfqpoint{4.365392in}{0.784587in}}%
\pgfpathlineto{\pgfqpoint{4.365688in}{0.784585in}}%
\pgfpathlineto{\pgfqpoint{4.365984in}{0.784584in}}%
\pgfpathlineto{\pgfqpoint{4.366280in}{0.784583in}}%
\pgfpathlineto{\pgfqpoint{4.366576in}{0.784581in}}%
\pgfpathlineto{\pgfqpoint{4.366872in}{0.784580in}}%
\pgfpathlineto{\pgfqpoint{4.367168in}{0.784579in}}%
\pgfpathlineto{\pgfqpoint{4.367464in}{0.784577in}}%
\pgfpathlineto{\pgfqpoint{4.367760in}{0.784576in}}%
\pgfpathlineto{\pgfqpoint{4.368056in}{0.784575in}}%
\pgfpathlineto{\pgfqpoint{4.368352in}{0.784573in}}%
\pgfpathlineto{\pgfqpoint{4.368648in}{0.784572in}}%
\pgfpathlineto{\pgfqpoint{4.368944in}{0.784571in}}%
\pgfpathlineto{\pgfqpoint{4.369240in}{0.784570in}}%
\pgfpathlineto{\pgfqpoint{4.369536in}{0.784568in}}%
\pgfpathlineto{\pgfqpoint{4.369832in}{0.784567in}}%
\pgfpathlineto{\pgfqpoint{4.370128in}{0.784566in}}%
\pgfpathlineto{\pgfqpoint{4.370424in}{0.784564in}}%
\pgfpathlineto{\pgfqpoint{4.370720in}{0.784563in}}%
\pgfpathlineto{\pgfqpoint{4.371016in}{0.784562in}}%
\pgfpathlineto{\pgfqpoint{4.371312in}{0.784560in}}%
\pgfpathlineto{\pgfqpoint{4.371608in}{0.784559in}}%
\pgfpathlineto{\pgfqpoint{4.371904in}{0.784558in}}%
\pgfpathlineto{\pgfqpoint{4.372200in}{0.784556in}}%
\pgfpathlineto{\pgfqpoint{4.372496in}{0.784555in}}%
\pgfpathlineto{\pgfqpoint{4.372792in}{0.784554in}}%
\pgfpathlineto{\pgfqpoint{4.373088in}{0.784553in}}%
\pgfpathlineto{\pgfqpoint{4.373384in}{0.784551in}}%
\pgfpathlineto{\pgfqpoint{4.373680in}{0.784550in}}%
\pgfpathlineto{\pgfqpoint{4.373976in}{0.784549in}}%
\pgfpathlineto{\pgfqpoint{4.374272in}{0.784547in}}%
\pgfpathlineto{\pgfqpoint{4.374568in}{0.784546in}}%
\pgfpathlineto{\pgfqpoint{4.374864in}{0.784545in}}%
\pgfpathlineto{\pgfqpoint{4.375160in}{0.784543in}}%
\pgfpathlineto{\pgfqpoint{4.375456in}{0.784542in}}%
\pgfpathlineto{\pgfqpoint{4.375752in}{0.784541in}}%
\pgfpathlineto{\pgfqpoint{4.376048in}{0.784539in}}%
\pgfpathlineto{\pgfqpoint{4.376344in}{0.784538in}}%
\pgfpathlineto{\pgfqpoint{4.376640in}{0.784537in}}%
\pgfpathlineto{\pgfqpoint{4.376936in}{0.784535in}}%
\pgfpathlineto{\pgfqpoint{4.377232in}{0.784534in}}%
\pgfpathlineto{\pgfqpoint{4.377528in}{0.784525in}}%
\pgfpathlineto{\pgfqpoint{4.377824in}{0.784503in}}%
\pgfpathlineto{\pgfqpoint{4.378120in}{0.784481in}}%
\pgfpathlineto{\pgfqpoint{4.378416in}{0.784459in}}%
\pgfpathlineto{\pgfqpoint{4.378712in}{0.784438in}}%
\pgfpathlineto{\pgfqpoint{4.379008in}{0.784416in}}%
\pgfpathlineto{\pgfqpoint{4.379304in}{0.784394in}}%
\pgfpathlineto{\pgfqpoint{4.379600in}{0.784372in}}%
\pgfpathlineto{\pgfqpoint{4.379896in}{0.784350in}}%
\pgfpathlineto{\pgfqpoint{4.380192in}{0.784329in}}%
\pgfpathlineto{\pgfqpoint{4.380488in}{0.784307in}}%
\pgfpathlineto{\pgfqpoint{4.380784in}{0.784285in}}%
\pgfpathlineto{\pgfqpoint{4.381080in}{0.784263in}}%
\pgfpathlineto{\pgfqpoint{4.381376in}{0.784241in}}%
\pgfpathlineto{\pgfqpoint{4.381672in}{0.784219in}}%
\pgfpathlineto{\pgfqpoint{4.381968in}{0.784198in}}%
\pgfpathlineto{\pgfqpoint{4.382264in}{0.784176in}}%
\pgfpathlineto{\pgfqpoint{4.382560in}{0.784154in}}%
\pgfpathlineto{\pgfqpoint{4.382856in}{0.784132in}}%
\pgfpathlineto{\pgfqpoint{4.383152in}{0.784110in}}%
\pgfpathlineto{\pgfqpoint{4.383448in}{0.784089in}}%
\pgfpathlineto{\pgfqpoint{4.383744in}{0.784067in}}%
\pgfpathlineto{\pgfqpoint{4.384040in}{0.784045in}}%
\pgfpathlineto{\pgfqpoint{4.384336in}{0.784023in}}%
\pgfpathlineto{\pgfqpoint{4.384632in}{0.784001in}}%
\pgfpathlineto{\pgfqpoint{4.384928in}{0.783980in}}%
\pgfpathlineto{\pgfqpoint{4.385224in}{0.783958in}}%
\pgfpathlineto{\pgfqpoint{4.385520in}{0.783936in}}%
\pgfpathlineto{\pgfqpoint{4.385816in}{0.783914in}}%
\pgfpathlineto{\pgfqpoint{4.386112in}{0.783892in}}%
\pgfpathlineto{\pgfqpoint{4.386408in}{0.783870in}}%
\pgfpathlineto{\pgfqpoint{4.386704in}{0.783849in}}%
\pgfpathlineto{\pgfqpoint{4.387000in}{0.783827in}}%
\pgfpathlineto{\pgfqpoint{4.387296in}{0.783805in}}%
\pgfpathlineto{\pgfqpoint{4.387592in}{0.783783in}}%
\pgfpathlineto{\pgfqpoint{4.387888in}{0.783761in}}%
\pgfpathlineto{\pgfqpoint{4.388184in}{0.783740in}}%
\pgfpathlineto{\pgfqpoint{4.388480in}{0.783718in}}%
\pgfpathlineto{\pgfqpoint{4.388776in}{0.783696in}}%
\pgfpathlineto{\pgfqpoint{4.389072in}{0.783674in}}%
\pgfpathlineto{\pgfqpoint{4.389368in}{0.783652in}}%
\pgfpathlineto{\pgfqpoint{4.389664in}{0.783630in}}%
\pgfpathlineto{\pgfqpoint{4.389960in}{0.783609in}}%
\pgfpathlineto{\pgfqpoint{4.390256in}{0.783587in}}%
\pgfpathlineto{\pgfqpoint{4.390552in}{0.783565in}}%
\pgfpathlineto{\pgfqpoint{4.390848in}{0.783543in}}%
\pgfpathlineto{\pgfqpoint{4.391144in}{0.783521in}}%
\pgfpathlineto{\pgfqpoint{4.391440in}{0.783500in}}%
\pgfpathlineto{\pgfqpoint{4.391736in}{0.783478in}}%
\pgfpathlineto{\pgfqpoint{4.392032in}{0.783456in}}%
\pgfpathlineto{\pgfqpoint{4.392328in}{0.783434in}}%
\pgfpathlineto{\pgfqpoint{4.392624in}{0.783412in}}%
\pgfpathlineto{\pgfqpoint{4.392920in}{0.783391in}}%
\pgfpathlineto{\pgfqpoint{4.393216in}{0.783369in}}%
\pgfpathlineto{\pgfqpoint{4.393512in}{0.783347in}}%
\pgfpathlineto{\pgfqpoint{4.393808in}{0.783325in}}%
\pgfpathlineto{\pgfqpoint{4.394104in}{0.783303in}}%
\pgfpathlineto{\pgfqpoint{4.394400in}{0.783281in}}%
\pgfpathlineto{\pgfqpoint{4.394696in}{0.783260in}}%
\pgfpathlineto{\pgfqpoint{4.394992in}{0.783238in}}%
\pgfpathlineto{\pgfqpoint{4.395288in}{0.783216in}}%
\pgfpathlineto{\pgfqpoint{4.395584in}{0.783194in}}%
\pgfpathlineto{\pgfqpoint{4.395880in}{0.783172in}}%
\pgfpathlineto{\pgfqpoint{4.396176in}{0.783151in}}%
\pgfpathlineto{\pgfqpoint{4.396472in}{0.783129in}}%
\pgfpathlineto{\pgfqpoint{4.396768in}{0.783107in}}%
\pgfpathlineto{\pgfqpoint{4.397064in}{0.783085in}}%
\pgfpathlineto{\pgfqpoint{4.397360in}{0.783063in}}%
\pgfpathlineto{\pgfqpoint{4.397656in}{0.783042in}}%
\pgfpathlineto{\pgfqpoint{4.397952in}{0.783020in}}%
\pgfpathlineto{\pgfqpoint{4.398248in}{0.782999in}}%
\pgfpathlineto{\pgfqpoint{4.398544in}{0.782993in}}%
\pgfpathlineto{\pgfqpoint{4.398840in}{0.782982in}}%
\pgfpathlineto{\pgfqpoint{4.399136in}{0.782963in}}%
\pgfpathlineto{\pgfqpoint{4.399432in}{0.782944in}}%
\pgfpathlineto{\pgfqpoint{4.399728in}{0.782926in}}%
\pgfpathlineto{\pgfqpoint{4.400024in}{0.782907in}}%
\pgfpathlineto{\pgfqpoint{4.400320in}{0.782888in}}%
\pgfpathlineto{\pgfqpoint{4.400616in}{0.782870in}}%
\pgfpathlineto{\pgfqpoint{4.400912in}{0.782851in}}%
\pgfpathlineto{\pgfqpoint{4.401208in}{0.782832in}}%
\pgfpathlineto{\pgfqpoint{4.401504in}{0.782813in}}%
\pgfpathlineto{\pgfqpoint{4.401800in}{0.782795in}}%
\pgfpathlineto{\pgfqpoint{4.402096in}{0.782776in}}%
\pgfpathlineto{\pgfqpoint{4.402392in}{0.782757in}}%
\pgfpathlineto{\pgfqpoint{4.402688in}{0.782739in}}%
\pgfpathlineto{\pgfqpoint{4.402984in}{0.782720in}}%
\pgfpathlineto{\pgfqpoint{4.403280in}{0.782701in}}%
\pgfpathlineto{\pgfqpoint{4.403576in}{0.782683in}}%
\pgfpathlineto{\pgfqpoint{4.403872in}{0.782664in}}%
\pgfpathlineto{\pgfqpoint{4.404168in}{0.782645in}}%
\pgfpathlineto{\pgfqpoint{4.404464in}{0.782627in}}%
\pgfpathlineto{\pgfqpoint{4.404760in}{0.782608in}}%
\pgfpathlineto{\pgfqpoint{4.405056in}{0.782589in}}%
\pgfpathlineto{\pgfqpoint{4.405352in}{0.782570in}}%
\pgfpathlineto{\pgfqpoint{4.405648in}{0.782551in}}%
\pgfpathlineto{\pgfqpoint{4.405944in}{0.782528in}}%
\pgfpathlineto{\pgfqpoint{4.406240in}{0.782502in}}%
\pgfpathlineto{\pgfqpoint{4.406536in}{0.782476in}}%
\pgfpathlineto{\pgfqpoint{4.406832in}{0.782444in}}%
\pgfpathlineto{\pgfqpoint{4.407128in}{0.782398in}}%
\pgfpathlineto{\pgfqpoint{4.407424in}{0.782352in}}%
\pgfpathlineto{\pgfqpoint{4.407720in}{0.782305in}}%
\pgfpathlineto{\pgfqpoint{4.408016in}{0.782258in}}%
\pgfpathlineto{\pgfqpoint{4.408312in}{0.782212in}}%
\pgfpathlineto{\pgfqpoint{4.408608in}{0.782165in}}%
\pgfpathlineto{\pgfqpoint{4.408904in}{0.782119in}}%
\pgfpathlineto{\pgfqpoint{4.409200in}{0.782072in}}%
\pgfpathlineto{\pgfqpoint{4.409496in}{0.782026in}}%
\pgfpathlineto{\pgfqpoint{4.409792in}{0.781979in}}%
\pgfpathlineto{\pgfqpoint{4.410088in}{0.781932in}}%
\pgfpathlineto{\pgfqpoint{4.410384in}{0.781886in}}%
\pgfpathlineto{\pgfqpoint{4.410680in}{0.782443in}}%
\pgfpathlineto{\pgfqpoint{4.410976in}{0.783308in}}%
\pgfpathlineto{\pgfqpoint{4.411272in}{0.783272in}}%
\pgfpathlineto{\pgfqpoint{4.411568in}{0.783237in}}%
\pgfpathlineto{\pgfqpoint{4.411864in}{0.783202in}}%
\pgfpathlineto{\pgfqpoint{4.412160in}{0.783166in}}%
\pgfpathlineto{\pgfqpoint{4.412456in}{0.783131in}}%
\pgfpathlineto{\pgfqpoint{4.412752in}{0.783095in}}%
\pgfpathlineto{\pgfqpoint{4.413048in}{0.783066in}}%
\pgfpathlineto{\pgfqpoint{4.413344in}{0.783049in}}%
\pgfpathlineto{\pgfqpoint{4.413640in}{0.783033in}}%
\pgfpathlineto{\pgfqpoint{4.413936in}{0.783016in}}%
\pgfpathlineto{\pgfqpoint{4.414232in}{0.783000in}}%
\pgfpathlineto{\pgfqpoint{4.414528in}{0.782983in}}%
\pgfpathlineto{\pgfqpoint{4.414824in}{0.782967in}}%
\pgfpathlineto{\pgfqpoint{4.415120in}{0.782950in}}%
\pgfpathlineto{\pgfqpoint{4.415416in}{0.782934in}}%
\pgfpathlineto{\pgfqpoint{4.415712in}{0.782917in}}%
\pgfpathlineto{\pgfqpoint{4.416008in}{0.782901in}}%
\pgfpathlineto{\pgfqpoint{4.416304in}{0.782884in}}%
\pgfpathlineto{\pgfqpoint{4.416600in}{0.782868in}}%
\pgfpathlineto{\pgfqpoint{4.416896in}{0.782851in}}%
\pgfpathlineto{\pgfqpoint{4.417192in}{0.782835in}}%
\pgfpathlineto{\pgfqpoint{4.417488in}{0.782818in}}%
\pgfpathlineto{\pgfqpoint{4.417784in}{0.782802in}}%
\pgfpathlineto{\pgfqpoint{4.418080in}{0.782785in}}%
\pgfpathlineto{\pgfqpoint{4.418376in}{0.782769in}}%
\pgfpathlineto{\pgfqpoint{4.418672in}{0.782752in}}%
\pgfpathlineto{\pgfqpoint{4.418968in}{0.782736in}}%
\pgfpathlineto{\pgfqpoint{4.419264in}{0.782719in}}%
\pgfpathlineto{\pgfqpoint{4.419560in}{0.782703in}}%
\pgfpathlineto{\pgfqpoint{4.419856in}{0.782687in}}%
\pgfpathlineto{\pgfqpoint{4.420152in}{0.782673in}}%
\pgfpathlineto{\pgfqpoint{4.420448in}{0.782466in}}%
\pgfpathlineto{\pgfqpoint{4.420744in}{0.782163in}}%
\pgfpathlineto{\pgfqpoint{4.421040in}{0.782166in}}%
\pgfpathlineto{\pgfqpoint{4.421336in}{0.782168in}}%
\pgfpathlineto{\pgfqpoint{4.421632in}{0.782171in}}%
\pgfpathlineto{\pgfqpoint{4.421928in}{0.782174in}}%
\pgfpathlineto{\pgfqpoint{4.422224in}{0.782176in}}%
\pgfpathlineto{\pgfqpoint{4.422520in}{0.782179in}}%
\pgfpathlineto{\pgfqpoint{4.422816in}{0.782182in}}%
\pgfpathlineto{\pgfqpoint{4.423112in}{0.782185in}}%
\pgfpathlineto{\pgfqpoint{4.423409in}{0.782187in}}%
\pgfpathlineto{\pgfqpoint{4.423705in}{0.782190in}}%
\pgfpathlineto{\pgfqpoint{4.424001in}{0.782193in}}%
\pgfpathlineto{\pgfqpoint{4.424297in}{0.782196in}}%
\pgfpathlineto{\pgfqpoint{4.424593in}{0.782198in}}%
\pgfpathlineto{\pgfqpoint{4.424889in}{0.782201in}}%
\pgfpathlineto{\pgfqpoint{4.425185in}{0.782204in}}%
\pgfpathlineto{\pgfqpoint{4.425481in}{0.782207in}}%
\pgfpathlineto{\pgfqpoint{4.425777in}{0.782209in}}%
\pgfpathlineto{\pgfqpoint{4.426073in}{0.782212in}}%
\pgfpathlineto{\pgfqpoint{4.426369in}{0.782215in}}%
\pgfpathlineto{\pgfqpoint{4.426665in}{0.782217in}}%
\pgfpathlineto{\pgfqpoint{4.426961in}{0.782220in}}%
\pgfpathlineto{\pgfqpoint{4.427257in}{0.782223in}}%
\pgfpathlineto{\pgfqpoint{4.427553in}{0.782226in}}%
\pgfpathlineto{\pgfqpoint{4.427849in}{0.782228in}}%
\pgfpathlineto{\pgfqpoint{4.428145in}{0.782231in}}%
\pgfpathlineto{\pgfqpoint{4.428441in}{0.782234in}}%
\pgfpathlineto{\pgfqpoint{4.428737in}{0.782237in}}%
\pgfpathlineto{\pgfqpoint{4.429033in}{0.782239in}}%
\pgfpathlineto{\pgfqpoint{4.429329in}{0.782242in}}%
\pgfpathlineto{\pgfqpoint{4.429625in}{0.782245in}}%
\pgfpathlineto{\pgfqpoint{4.429921in}{0.782248in}}%
\pgfpathlineto{\pgfqpoint{4.430217in}{0.782250in}}%
\pgfpathlineto{\pgfqpoint{4.430513in}{0.782253in}}%
\pgfpathlineto{\pgfqpoint{4.430809in}{0.782256in}}%
\pgfpathlineto{\pgfqpoint{4.431105in}{0.782258in}}%
\pgfpathlineto{\pgfqpoint{4.431401in}{0.782261in}}%
\pgfpathlineto{\pgfqpoint{4.431697in}{0.782264in}}%
\pgfpathlineto{\pgfqpoint{4.431993in}{0.782267in}}%
\pgfpathlineto{\pgfqpoint{4.432289in}{0.782269in}}%
\pgfpathlineto{\pgfqpoint{4.432585in}{0.782272in}}%
\pgfpathlineto{\pgfqpoint{4.432881in}{0.782275in}}%
\pgfpathlineto{\pgfqpoint{4.433177in}{0.782278in}}%
\pgfpathlineto{\pgfqpoint{4.433473in}{0.782280in}}%
\pgfpathlineto{\pgfqpoint{4.433769in}{0.782283in}}%
\pgfpathlineto{\pgfqpoint{4.434065in}{0.782286in}}%
\pgfpathlineto{\pgfqpoint{4.434361in}{0.782288in}}%
\pgfpathlineto{\pgfqpoint{4.434657in}{0.782291in}}%
\pgfpathlineto{\pgfqpoint{4.434953in}{0.782294in}}%
\pgfpathlineto{\pgfqpoint{4.435249in}{0.782297in}}%
\pgfpathlineto{\pgfqpoint{4.435545in}{0.782299in}}%
\pgfpathlineto{\pgfqpoint{4.435841in}{0.782302in}}%
\pgfpathlineto{\pgfqpoint{4.436137in}{0.782305in}}%
\pgfpathlineto{\pgfqpoint{4.436433in}{0.782308in}}%
\pgfpathlineto{\pgfqpoint{4.436729in}{0.782310in}}%
\pgfpathlineto{\pgfqpoint{4.437025in}{0.782313in}}%
\pgfpathlineto{\pgfqpoint{4.437321in}{0.782316in}}%
\pgfpathlineto{\pgfqpoint{4.437617in}{0.782319in}}%
\pgfpathlineto{\pgfqpoint{4.437913in}{0.782321in}}%
\pgfpathlineto{\pgfqpoint{4.438209in}{0.782324in}}%
\pgfpathlineto{\pgfqpoint{4.438505in}{0.782327in}}%
\pgfpathlineto{\pgfqpoint{4.438801in}{0.782329in}}%
\pgfpathlineto{\pgfqpoint{4.439097in}{0.782332in}}%
\pgfpathlineto{\pgfqpoint{4.439393in}{0.782335in}}%
\pgfpathlineto{\pgfqpoint{4.439689in}{0.782338in}}%
\pgfpathlineto{\pgfqpoint{4.439985in}{0.782340in}}%
\pgfpathlineto{\pgfqpoint{4.440281in}{0.782343in}}%
\pgfpathlineto{\pgfqpoint{4.440577in}{0.782346in}}%
\pgfpathlineto{\pgfqpoint{4.440873in}{0.782349in}}%
\pgfpathlineto{\pgfqpoint{4.441169in}{0.782351in}}%
\pgfpathlineto{\pgfqpoint{4.441465in}{0.782354in}}%
\pgfpathlineto{\pgfqpoint{4.441761in}{0.782357in}}%
\pgfpathlineto{\pgfqpoint{4.442057in}{0.782360in}}%
\pgfpathlineto{\pgfqpoint{4.442353in}{0.782362in}}%
\pgfpathlineto{\pgfqpoint{4.442649in}{0.782365in}}%
\pgfpathlineto{\pgfqpoint{4.442945in}{0.782368in}}%
\pgfpathlineto{\pgfqpoint{4.443241in}{0.782370in}}%
\pgfpathlineto{\pgfqpoint{4.443537in}{0.782373in}}%
\pgfpathlineto{\pgfqpoint{4.443833in}{0.782376in}}%
\pgfpathlineto{\pgfqpoint{4.444129in}{0.782379in}}%
\pgfpathlineto{\pgfqpoint{4.444425in}{0.782381in}}%
\pgfpathlineto{\pgfqpoint{4.444721in}{0.782384in}}%
\pgfpathlineto{\pgfqpoint{4.445017in}{0.782387in}}%
\pgfpathlineto{\pgfqpoint{4.445313in}{0.782390in}}%
\pgfpathlineto{\pgfqpoint{4.445609in}{0.782392in}}%
\pgfpathlineto{\pgfqpoint{4.445905in}{0.782395in}}%
\pgfpathlineto{\pgfqpoint{4.446201in}{0.782398in}}%
\pgfpathlineto{\pgfqpoint{4.446497in}{0.782400in}}%
\pgfpathlineto{\pgfqpoint{4.446793in}{0.782403in}}%
\pgfpathlineto{\pgfqpoint{4.447089in}{0.782406in}}%
\pgfpathlineto{\pgfqpoint{4.447385in}{0.782409in}}%
\pgfpathlineto{\pgfqpoint{4.447681in}{0.782411in}}%
\pgfpathlineto{\pgfqpoint{4.447977in}{0.782414in}}%
\pgfpathlineto{\pgfqpoint{4.448273in}{0.782417in}}%
\pgfpathlineto{\pgfqpoint{4.448569in}{0.782420in}}%
\pgfpathlineto{\pgfqpoint{4.448865in}{0.782422in}}%
\pgfpathlineto{\pgfqpoint{4.449161in}{0.782425in}}%
\pgfpathlineto{\pgfqpoint{4.449457in}{0.782428in}}%
\pgfpathlineto{\pgfqpoint{4.449753in}{0.782431in}}%
\pgfpathlineto{\pgfqpoint{4.450049in}{0.782433in}}%
\pgfpathlineto{\pgfqpoint{4.450345in}{0.782436in}}%
\pgfpathlineto{\pgfqpoint{4.450641in}{0.782439in}}%
\pgfpathlineto{\pgfqpoint{4.450937in}{0.782441in}}%
\pgfpathlineto{\pgfqpoint{4.451233in}{0.782444in}}%
\pgfpathlineto{\pgfqpoint{4.451529in}{0.782447in}}%
\pgfpathlineto{\pgfqpoint{4.451825in}{0.782450in}}%
\pgfpathlineto{\pgfqpoint{4.452121in}{0.782452in}}%
\pgfpathlineto{\pgfqpoint{4.452417in}{0.782455in}}%
\pgfpathlineto{\pgfqpoint{4.452713in}{0.782458in}}%
\pgfpathlineto{\pgfqpoint{4.453009in}{0.782461in}}%
\pgfpathlineto{\pgfqpoint{4.453305in}{0.782463in}}%
\pgfpathlineto{\pgfqpoint{4.453601in}{0.782466in}}%
\pgfpathlineto{\pgfqpoint{4.453897in}{0.782469in}}%
\pgfpathlineto{\pgfqpoint{4.454193in}{0.782472in}}%
\pgfpathlineto{\pgfqpoint{4.454489in}{0.782474in}}%
\pgfpathlineto{\pgfqpoint{4.454785in}{0.782477in}}%
\pgfpathlineto{\pgfqpoint{4.455081in}{0.782480in}}%
\pgfpathlineto{\pgfqpoint{4.455377in}{0.782482in}}%
\pgfpathlineto{\pgfqpoint{4.455673in}{0.782485in}}%
\pgfpathlineto{\pgfqpoint{4.455969in}{0.782486in}}%
\pgfpathlineto{\pgfqpoint{4.456265in}{0.782467in}}%
\pgfpathlineto{\pgfqpoint{4.456561in}{0.782444in}}%
\pgfpathlineto{\pgfqpoint{4.456857in}{0.782422in}}%
\pgfpathlineto{\pgfqpoint{4.457153in}{0.782399in}}%
\pgfpathlineto{\pgfqpoint{4.457449in}{0.782376in}}%
\pgfpathlineto{\pgfqpoint{4.457745in}{0.782353in}}%
\pgfpathlineto{\pgfqpoint{4.458041in}{0.782330in}}%
\pgfpathlineto{\pgfqpoint{4.458337in}{0.782308in}}%
\pgfpathlineto{\pgfqpoint{4.458633in}{0.782285in}}%
\pgfpathlineto{\pgfqpoint{4.458929in}{0.782262in}}%
\pgfpathlineto{\pgfqpoint{4.459225in}{0.782239in}}%
\pgfpathlineto{\pgfqpoint{4.459521in}{0.782217in}}%
\pgfpathlineto{\pgfqpoint{4.459817in}{0.782194in}}%
\pgfpathlineto{\pgfqpoint{4.460113in}{0.782171in}}%
\pgfpathlineto{\pgfqpoint{4.460409in}{0.782148in}}%
\pgfpathlineto{\pgfqpoint{4.460705in}{0.782126in}}%
\pgfpathlineto{\pgfqpoint{4.461001in}{0.782103in}}%
\pgfpathlineto{\pgfqpoint{4.461297in}{0.782080in}}%
\pgfpathlineto{\pgfqpoint{4.461593in}{0.782057in}}%
\pgfpathlineto{\pgfqpoint{4.461889in}{0.782241in}}%
\pgfpathlineto{\pgfqpoint{4.462185in}{0.782454in}}%
\pgfpathlineto{\pgfqpoint{4.462481in}{0.782462in}}%
\pgfpathlineto{\pgfqpoint{4.462777in}{0.782470in}}%
\pgfpathlineto{\pgfqpoint{4.463073in}{0.782466in}}%
\pgfpathlineto{\pgfqpoint{4.463369in}{0.782446in}}%
\pgfpathlineto{\pgfqpoint{4.463665in}{0.782426in}}%
\pgfpathlineto{\pgfqpoint{4.463961in}{0.782406in}}%
\pgfpathlineto{\pgfqpoint{4.464257in}{0.782386in}}%
\pgfpathlineto{\pgfqpoint{4.464553in}{0.782366in}}%
\pgfpathlineto{\pgfqpoint{4.464849in}{0.782346in}}%
\pgfpathlineto{\pgfqpoint{4.465145in}{0.782325in}}%
\pgfpathlineto{\pgfqpoint{4.465441in}{0.782305in}}%
\pgfpathlineto{\pgfqpoint{4.465737in}{0.782285in}}%
\pgfpathlineto{\pgfqpoint{4.466033in}{0.782265in}}%
\pgfpathlineto{\pgfqpoint{4.466329in}{0.782245in}}%
\pgfpathlineto{\pgfqpoint{4.466625in}{0.782225in}}%
\pgfpathlineto{\pgfqpoint{4.466921in}{0.782205in}}%
\pgfpathlineto{\pgfqpoint{4.467217in}{0.782185in}}%
\pgfpathlineto{\pgfqpoint{4.467513in}{0.782165in}}%
\pgfpathlineto{\pgfqpoint{4.467809in}{0.782145in}}%
\pgfpathlineto{\pgfqpoint{4.468105in}{0.782125in}}%
\pgfpathlineto{\pgfqpoint{4.468401in}{0.782105in}}%
\pgfpathlineto{\pgfqpoint{4.468697in}{0.782085in}}%
\pgfpathlineto{\pgfqpoint{4.468993in}{0.782065in}}%
\pgfpathlineto{\pgfqpoint{4.469289in}{0.782045in}}%
\pgfpathlineto{\pgfqpoint{4.469585in}{0.782024in}}%
\pgfpathlineto{\pgfqpoint{4.469881in}{0.782004in}}%
\pgfpathlineto{\pgfqpoint{4.470177in}{0.781994in}}%
\pgfpathlineto{\pgfqpoint{4.470473in}{0.781990in}}%
\pgfpathlineto{\pgfqpoint{4.470769in}{0.781987in}}%
\pgfpathlineto{\pgfqpoint{4.471065in}{0.781984in}}%
\pgfpathlineto{\pgfqpoint{4.471361in}{0.781980in}}%
\pgfpathlineto{\pgfqpoint{4.471657in}{0.781977in}}%
\pgfpathlineto{\pgfqpoint{4.471953in}{0.781973in}}%
\pgfpathlineto{\pgfqpoint{4.472249in}{0.781970in}}%
\pgfpathlineto{\pgfqpoint{4.472545in}{0.781966in}}%
\pgfpathlineto{\pgfqpoint{4.472841in}{0.781963in}}%
\pgfpathlineto{\pgfqpoint{4.473137in}{0.781959in}}%
\pgfpathlineto{\pgfqpoint{4.473433in}{0.781956in}}%
\pgfpathlineto{\pgfqpoint{4.473729in}{0.781953in}}%
\pgfpathlineto{\pgfqpoint{4.474025in}{0.781949in}}%
\pgfpathlineto{\pgfqpoint{4.474321in}{0.781946in}}%
\pgfpathlineto{\pgfqpoint{4.474617in}{0.781942in}}%
\pgfpathlineto{\pgfqpoint{4.474913in}{0.781939in}}%
\pgfpathlineto{\pgfqpoint{4.475209in}{0.781935in}}%
\pgfpathlineto{\pgfqpoint{4.475505in}{0.781932in}}%
\pgfpathlineto{\pgfqpoint{4.475801in}{0.781928in}}%
\pgfpathlineto{\pgfqpoint{4.476097in}{0.781925in}}%
\pgfpathlineto{\pgfqpoint{4.476393in}{0.781921in}}%
\pgfpathlineto{\pgfqpoint{4.476689in}{0.781918in}}%
\pgfpathlineto{\pgfqpoint{4.476985in}{0.781915in}}%
\pgfpathlineto{\pgfqpoint{4.477281in}{0.781911in}}%
\pgfpathlineto{\pgfqpoint{4.477577in}{0.781908in}}%
\pgfpathlineto{\pgfqpoint{4.477873in}{0.781904in}}%
\pgfpathlineto{\pgfqpoint{4.478169in}{0.781901in}}%
\pgfpathlineto{\pgfqpoint{4.478465in}{0.781897in}}%
\pgfpathlineto{\pgfqpoint{4.478761in}{0.781894in}}%
\pgfpathlineto{\pgfqpoint{4.479057in}{0.781890in}}%
\pgfpathlineto{\pgfqpoint{4.479353in}{0.781887in}}%
\pgfpathlineto{\pgfqpoint{4.479649in}{0.781884in}}%
\pgfpathlineto{\pgfqpoint{4.479945in}{0.781880in}}%
\pgfpathlineto{\pgfqpoint{4.480241in}{0.781877in}}%
\pgfpathlineto{\pgfqpoint{4.480537in}{0.781873in}}%
\pgfpathlineto{\pgfqpoint{4.480833in}{0.781870in}}%
\pgfpathlineto{\pgfqpoint{4.481129in}{0.781866in}}%
\pgfpathlineto{\pgfqpoint{4.481425in}{0.781863in}}%
\pgfpathlineto{\pgfqpoint{4.481721in}{0.781859in}}%
\pgfpathlineto{\pgfqpoint{4.482017in}{0.781856in}}%
\pgfpathlineto{\pgfqpoint{4.482313in}{0.781852in}}%
\pgfpathlineto{\pgfqpoint{4.482609in}{0.781849in}}%
\pgfpathlineto{\pgfqpoint{4.482905in}{0.781846in}}%
\pgfpathlineto{\pgfqpoint{4.483201in}{0.781842in}}%
\pgfpathlineto{\pgfqpoint{4.483497in}{0.781839in}}%
\pgfpathlineto{\pgfqpoint{4.483793in}{0.781835in}}%
\pgfpathlineto{\pgfqpoint{4.484089in}{0.781832in}}%
\pgfpathlineto{\pgfqpoint{4.484385in}{0.781828in}}%
\pgfpathlineto{\pgfqpoint{4.484681in}{0.781825in}}%
\pgfpathlineto{\pgfqpoint{4.484977in}{0.781821in}}%
\pgfpathlineto{\pgfqpoint{4.485273in}{0.781818in}}%
\pgfpathlineto{\pgfqpoint{4.485569in}{0.781815in}}%
\pgfpathlineto{\pgfqpoint{4.485865in}{0.781811in}}%
\pgfpathlineto{\pgfqpoint{4.486161in}{0.781808in}}%
\pgfpathlineto{\pgfqpoint{4.486457in}{0.781804in}}%
\pgfpathlineto{\pgfqpoint{4.486753in}{0.781801in}}%
\pgfpathlineto{\pgfqpoint{4.487049in}{0.781797in}}%
\pgfpathlineto{\pgfqpoint{4.487345in}{0.781794in}}%
\pgfpathlineto{\pgfqpoint{4.487641in}{0.781790in}}%
\pgfpathlineto{\pgfqpoint{4.487937in}{0.781787in}}%
\pgfpathlineto{\pgfqpoint{4.488233in}{0.781783in}}%
\pgfpathlineto{\pgfqpoint{4.488529in}{0.781780in}}%
\pgfpathlineto{\pgfqpoint{4.488825in}{0.781777in}}%
\pgfpathlineto{\pgfqpoint{4.489121in}{0.781773in}}%
\pgfpathlineto{\pgfqpoint{4.489417in}{0.781770in}}%
\pgfpathlineto{\pgfqpoint{4.489713in}{0.781766in}}%
\pgfpathlineto{\pgfqpoint{4.490009in}{0.781763in}}%
\pgfpathlineto{\pgfqpoint{4.490305in}{0.781759in}}%
\pgfpathlineto{\pgfqpoint{4.490601in}{0.781756in}}%
\pgfpathlineto{\pgfqpoint{4.490898in}{0.781752in}}%
\pgfpathlineto{\pgfqpoint{4.491194in}{0.781749in}}%
\pgfpathlineto{\pgfqpoint{4.491490in}{0.781746in}}%
\pgfpathlineto{\pgfqpoint{4.491786in}{0.781742in}}%
\pgfpathlineto{\pgfqpoint{4.492082in}{0.781739in}}%
\pgfpathlineto{\pgfqpoint{4.492378in}{0.781735in}}%
\pgfpathlineto{\pgfqpoint{4.492674in}{0.781732in}}%
\pgfpathlineto{\pgfqpoint{4.492970in}{0.781728in}}%
\pgfpathlineto{\pgfqpoint{4.493266in}{0.781725in}}%
\pgfpathlineto{\pgfqpoint{4.493562in}{0.781721in}}%
\pgfpathlineto{\pgfqpoint{4.493858in}{0.781718in}}%
\pgfpathlineto{\pgfqpoint{4.494154in}{0.781715in}}%
\pgfpathlineto{\pgfqpoint{4.494450in}{0.781711in}}%
\pgfpathlineto{\pgfqpoint{4.494746in}{0.781708in}}%
\pgfpathlineto{\pgfqpoint{4.495042in}{0.781704in}}%
\pgfpathlineto{\pgfqpoint{4.495338in}{0.781701in}}%
\pgfpathlineto{\pgfqpoint{4.495634in}{0.781697in}}%
\pgfpathlineto{\pgfqpoint{4.495930in}{0.781694in}}%
\pgfpathlineto{\pgfqpoint{4.496226in}{0.781690in}}%
\pgfpathlineto{\pgfqpoint{4.496522in}{0.781687in}}%
\pgfpathlineto{\pgfqpoint{4.496818in}{0.781683in}}%
\pgfpathlineto{\pgfqpoint{4.497114in}{0.781662in}}%
\pgfpathlineto{\pgfqpoint{4.497410in}{0.781455in}}%
\pgfpathlineto{\pgfqpoint{4.497706in}{0.781301in}}%
\pgfpathlineto{\pgfqpoint{4.498002in}{0.781207in}}%
\pgfpathlineto{\pgfqpoint{4.498298in}{0.781124in}}%
\pgfpathlineto{\pgfqpoint{4.498594in}{0.781048in}}%
\pgfpathlineto{\pgfqpoint{4.498890in}{0.781026in}}%
\pgfpathlineto{\pgfqpoint{4.499186in}{0.781017in}}%
\pgfpathlineto{\pgfqpoint{4.499482in}{0.781008in}}%
\pgfpathlineto{\pgfqpoint{4.499778in}{0.781000in}}%
\pgfpathlineto{\pgfqpoint{4.500074in}{0.780991in}}%
\pgfpathlineto{\pgfqpoint{4.500370in}{0.780982in}}%
\pgfpathlineto{\pgfqpoint{4.500666in}{0.780974in}}%
\pgfpathlineto{\pgfqpoint{4.500962in}{0.780965in}}%
\pgfpathlineto{\pgfqpoint{4.501258in}{0.780957in}}%
\pgfpathlineto{\pgfqpoint{4.501554in}{0.780948in}}%
\pgfpathlineto{\pgfqpoint{4.501850in}{0.780939in}}%
\pgfpathlineto{\pgfqpoint{4.502146in}{0.780931in}}%
\pgfpathlineto{\pgfqpoint{4.502442in}{0.780922in}}%
\pgfpathlineto{\pgfqpoint{4.502738in}{0.780913in}}%
\pgfpathlineto{\pgfqpoint{4.503034in}{0.780905in}}%
\pgfpathlineto{\pgfqpoint{4.503330in}{0.780896in}}%
\pgfpathlineto{\pgfqpoint{4.503626in}{0.780887in}}%
\pgfpathlineto{\pgfqpoint{4.503922in}{0.780879in}}%
\pgfpathlineto{\pgfqpoint{4.504218in}{0.780834in}}%
\pgfpathlineto{\pgfqpoint{4.504514in}{0.780717in}}%
\pgfpathlineto{\pgfqpoint{4.504810in}{0.780601in}}%
\pgfpathlineto{\pgfqpoint{4.505106in}{0.780497in}}%
\pgfpathlineto{\pgfqpoint{4.505402in}{0.780413in}}%
\pgfpathlineto{\pgfqpoint{4.505698in}{0.780397in}}%
\pgfpathlineto{\pgfqpoint{4.505994in}{0.780367in}}%
\pgfpathlineto{\pgfqpoint{4.506290in}{0.780333in}}%
\pgfpathlineto{\pgfqpoint{4.506586in}{0.780300in}}%
\pgfpathlineto{\pgfqpoint{4.506882in}{0.780267in}}%
\pgfpathlineto{\pgfqpoint{4.507178in}{0.780233in}}%
\pgfpathlineto{\pgfqpoint{4.507474in}{0.780200in}}%
\pgfpathlineto{\pgfqpoint{4.507770in}{0.780167in}}%
\pgfpathlineto{\pgfqpoint{4.508066in}{0.780134in}}%
\pgfpathlineto{\pgfqpoint{4.508362in}{0.780100in}}%
\pgfpathlineto{\pgfqpoint{4.508658in}{0.780067in}}%
\pgfpathlineto{\pgfqpoint{4.508954in}{0.780034in}}%
\pgfpathlineto{\pgfqpoint{4.509250in}{0.780000in}}%
\pgfpathlineto{\pgfqpoint{4.509546in}{0.779967in}}%
\pgfpathlineto{\pgfqpoint{4.509842in}{0.779934in}}%
\pgfpathlineto{\pgfqpoint{4.510138in}{0.779908in}}%
\pgfpathlineto{\pgfqpoint{4.510434in}{0.779882in}}%
\pgfpathlineto{\pgfqpoint{4.510730in}{0.779869in}}%
\pgfpathlineto{\pgfqpoint{4.511026in}{0.779864in}}%
\pgfpathlineto{\pgfqpoint{4.511322in}{0.779858in}}%
\pgfpathlineto{\pgfqpoint{4.511618in}{0.779852in}}%
\pgfpathlineto{\pgfqpoint{4.511914in}{0.779847in}}%
\pgfpathlineto{\pgfqpoint{4.512210in}{0.779841in}}%
\pgfpathlineto{\pgfqpoint{4.512506in}{0.779836in}}%
\pgfpathlineto{\pgfqpoint{4.512802in}{0.779830in}}%
\pgfpathlineto{\pgfqpoint{4.513098in}{0.779825in}}%
\pgfpathlineto{\pgfqpoint{4.513394in}{0.779819in}}%
\pgfpathlineto{\pgfqpoint{4.513690in}{0.779814in}}%
\pgfpathlineto{\pgfqpoint{4.513986in}{0.779808in}}%
\pgfpathlineto{\pgfqpoint{4.514282in}{0.779803in}}%
\pgfpathlineto{\pgfqpoint{4.514578in}{0.779797in}}%
\pgfpathlineto{\pgfqpoint{4.514874in}{0.779792in}}%
\pgfpathlineto{\pgfqpoint{4.515170in}{0.779786in}}%
\pgfpathlineto{\pgfqpoint{4.515466in}{0.779781in}}%
\pgfpathlineto{\pgfqpoint{4.515762in}{0.779776in}}%
\pgfpathlineto{\pgfqpoint{4.516058in}{0.779770in}}%
\pgfpathlineto{\pgfqpoint{4.516354in}{0.779765in}}%
\pgfpathlineto{\pgfqpoint{4.516650in}{0.779759in}}%
\pgfpathlineto{\pgfqpoint{4.516946in}{0.779754in}}%
\pgfpathlineto{\pgfqpoint{4.517242in}{0.779748in}}%
\pgfpathlineto{\pgfqpoint{4.517538in}{0.779743in}}%
\pgfpathlineto{\pgfqpoint{4.517834in}{0.779737in}}%
\pgfpathlineto{\pgfqpoint{4.518130in}{0.779732in}}%
\pgfpathlineto{\pgfqpoint{4.518426in}{0.779726in}}%
\pgfpathlineto{\pgfqpoint{4.518722in}{0.779721in}}%
\pgfpathlineto{\pgfqpoint{4.519018in}{0.779715in}}%
\pgfpathlineto{\pgfqpoint{4.519314in}{0.779710in}}%
\pgfpathlineto{\pgfqpoint{4.519610in}{0.779705in}}%
\pgfpathlineto{\pgfqpoint{4.519906in}{0.779699in}}%
\pgfpathlineto{\pgfqpoint{4.520202in}{0.779694in}}%
\pgfpathlineto{\pgfqpoint{4.520498in}{0.779688in}}%
\pgfpathlineto{\pgfqpoint{4.520794in}{0.779683in}}%
\pgfpathlineto{\pgfqpoint{4.521090in}{0.779677in}}%
\pgfpathlineto{\pgfqpoint{4.521386in}{0.779672in}}%
\pgfpathlineto{\pgfqpoint{4.521682in}{0.779666in}}%
\pgfpathlineto{\pgfqpoint{4.521978in}{0.779661in}}%
\pgfpathlineto{\pgfqpoint{4.522274in}{0.779655in}}%
\pgfpathlineto{\pgfqpoint{4.522570in}{0.779650in}}%
\pgfpathlineto{\pgfqpoint{4.522866in}{0.779645in}}%
\pgfpathlineto{\pgfqpoint{4.523162in}{0.779639in}}%
\pgfpathlineto{\pgfqpoint{4.523458in}{0.779634in}}%
\pgfpathlineto{\pgfqpoint{4.523754in}{0.779628in}}%
\pgfpathlineto{\pgfqpoint{4.524050in}{0.779623in}}%
\pgfpathlineto{\pgfqpoint{4.524346in}{0.779617in}}%
\pgfpathlineto{\pgfqpoint{4.524642in}{0.779612in}}%
\pgfpathlineto{\pgfqpoint{4.524938in}{0.779606in}}%
\pgfpathlineto{\pgfqpoint{4.525234in}{0.779601in}}%
\pgfpathlineto{\pgfqpoint{4.525530in}{0.779595in}}%
\pgfpathlineto{\pgfqpoint{4.525826in}{0.779590in}}%
\pgfpathlineto{\pgfqpoint{4.526122in}{0.779585in}}%
\pgfpathlineto{\pgfqpoint{4.526418in}{0.779579in}}%
\pgfpathlineto{\pgfqpoint{4.526714in}{0.779574in}}%
\pgfpathlineto{\pgfqpoint{4.527010in}{0.779568in}}%
\pgfpathlineto{\pgfqpoint{4.527306in}{0.779564in}}%
\pgfpathlineto{\pgfqpoint{4.527602in}{0.779560in}}%
\pgfpathlineto{\pgfqpoint{4.527898in}{0.779556in}}%
\pgfpathlineto{\pgfqpoint{4.528194in}{0.779553in}}%
\pgfpathlineto{\pgfqpoint{4.528490in}{0.779549in}}%
\pgfpathlineto{\pgfqpoint{4.528786in}{0.779545in}}%
\pgfpathlineto{\pgfqpoint{4.529082in}{0.779542in}}%
\pgfpathlineto{\pgfqpoint{4.529378in}{0.779538in}}%
\pgfpathlineto{\pgfqpoint{4.529674in}{0.779534in}}%
\pgfpathlineto{\pgfqpoint{4.529970in}{0.779530in}}%
\pgfpathlineto{\pgfqpoint{4.530266in}{0.779527in}}%
\pgfpathlineto{\pgfqpoint{4.530562in}{0.779523in}}%
\pgfpathlineto{\pgfqpoint{4.530858in}{0.779519in}}%
\pgfpathlineto{\pgfqpoint{4.531154in}{0.779516in}}%
\pgfpathlineto{\pgfqpoint{4.531450in}{0.779512in}}%
\pgfpathlineto{\pgfqpoint{4.531746in}{0.779508in}}%
\pgfpathlineto{\pgfqpoint{4.532042in}{0.779504in}}%
\pgfpathlineto{\pgfqpoint{4.532338in}{0.779501in}}%
\pgfpathlineto{\pgfqpoint{4.532634in}{0.779497in}}%
\pgfpathlineto{\pgfqpoint{4.532930in}{0.779493in}}%
\pgfpathlineto{\pgfqpoint{4.533226in}{0.779489in}}%
\pgfpathlineto{\pgfqpoint{4.533522in}{0.779486in}}%
\pgfpathlineto{\pgfqpoint{4.533818in}{0.779482in}}%
\pgfpathlineto{\pgfqpoint{4.534114in}{0.779478in}}%
\pgfpathlineto{\pgfqpoint{4.534410in}{0.779475in}}%
\pgfpathlineto{\pgfqpoint{4.534706in}{0.779471in}}%
\pgfpathlineto{\pgfqpoint{4.535002in}{0.779467in}}%
\pgfpathlineto{\pgfqpoint{4.535298in}{0.779463in}}%
\pgfpathlineto{\pgfqpoint{4.535594in}{0.779460in}}%
\pgfpathlineto{\pgfqpoint{4.535890in}{0.779456in}}%
\pgfpathlineto{\pgfqpoint{4.536186in}{0.779452in}}%
\pgfpathlineto{\pgfqpoint{4.536482in}{0.779449in}}%
\pgfpathlineto{\pgfqpoint{4.536778in}{0.779445in}}%
\pgfpathlineto{\pgfqpoint{4.537074in}{0.779441in}}%
\pgfpathlineto{\pgfqpoint{4.537370in}{0.779437in}}%
\pgfpathlineto{\pgfqpoint{4.537666in}{0.779434in}}%
\pgfpathlineto{\pgfqpoint{4.537962in}{0.779430in}}%
\pgfpathlineto{\pgfqpoint{4.538258in}{0.779426in}}%
\pgfpathlineto{\pgfqpoint{4.538554in}{0.779422in}}%
\pgfpathlineto{\pgfqpoint{4.538850in}{0.779419in}}%
\pgfpathlineto{\pgfqpoint{4.539146in}{0.779415in}}%
\pgfpathlineto{\pgfqpoint{4.539442in}{0.779411in}}%
\pgfpathlineto{\pgfqpoint{4.539738in}{0.779408in}}%
\pgfpathlineto{\pgfqpoint{4.540034in}{0.779404in}}%
\pgfpathlineto{\pgfqpoint{4.540330in}{0.779400in}}%
\pgfpathlineto{\pgfqpoint{4.540626in}{0.779396in}}%
\pgfpathlineto{\pgfqpoint{4.540922in}{0.779393in}}%
\pgfpathlineto{\pgfqpoint{4.541218in}{0.779389in}}%
\pgfpathlineto{\pgfqpoint{4.541514in}{0.779385in}}%
\pgfpathlineto{\pgfqpoint{4.541810in}{0.779382in}}%
\pgfpathlineto{\pgfqpoint{4.542106in}{0.779378in}}%
\pgfpathlineto{\pgfqpoint{4.542402in}{0.779374in}}%
\pgfpathlineto{\pgfqpoint{4.542698in}{0.779370in}}%
\pgfpathlineto{\pgfqpoint{4.542994in}{0.779367in}}%
\pgfpathlineto{\pgfqpoint{4.543290in}{0.779363in}}%
\pgfpathlineto{\pgfqpoint{4.543586in}{0.779359in}}%
\pgfpathlineto{\pgfqpoint{4.543882in}{0.779355in}}%
\pgfpathlineto{\pgfqpoint{4.544178in}{0.779352in}}%
\pgfpathlineto{\pgfqpoint{4.544474in}{0.779348in}}%
\pgfpathlineto{\pgfqpoint{4.544770in}{0.779344in}}%
\pgfpathlineto{\pgfqpoint{4.545066in}{0.779341in}}%
\pgfpathlineto{\pgfqpoint{4.545362in}{0.779337in}}%
\pgfpathlineto{\pgfqpoint{4.545658in}{0.779333in}}%
\pgfpathlineto{\pgfqpoint{4.545954in}{0.779329in}}%
\pgfpathlineto{\pgfqpoint{4.546250in}{0.779326in}}%
\pgfpathlineto{\pgfqpoint{4.546546in}{0.779322in}}%
\pgfpathlineto{\pgfqpoint{4.546842in}{0.779318in}}%
\pgfpathlineto{\pgfqpoint{4.547138in}{0.779315in}}%
\pgfpathlineto{\pgfqpoint{4.547434in}{0.779311in}}%
\pgfpathlineto{\pgfqpoint{4.547730in}{0.779307in}}%
\pgfpathlineto{\pgfqpoint{4.548026in}{0.779303in}}%
\pgfpathlineto{\pgfqpoint{4.548322in}{0.779300in}}%
\pgfpathlineto{\pgfqpoint{4.548618in}{0.779296in}}%
\pgfpathlineto{\pgfqpoint{4.548914in}{0.779292in}}%
\pgfpathlineto{\pgfqpoint{4.549210in}{0.779289in}}%
\pgfpathlineto{\pgfqpoint{4.549506in}{0.779285in}}%
\pgfpathlineto{\pgfqpoint{4.549802in}{0.779281in}}%
\pgfpathlineto{\pgfqpoint{4.550098in}{0.779277in}}%
\pgfpathlineto{\pgfqpoint{4.550394in}{0.779274in}}%
\pgfpathlineto{\pgfqpoint{4.550690in}{0.779270in}}%
\pgfpathlineto{\pgfqpoint{4.550986in}{0.779266in}}%
\pgfpathlineto{\pgfqpoint{4.551282in}{0.779262in}}%
\pgfpathlineto{\pgfqpoint{4.551578in}{0.779259in}}%
\pgfpathlineto{\pgfqpoint{4.551874in}{0.779255in}}%
\pgfpathlineto{\pgfqpoint{4.552170in}{0.779251in}}%
\pgfpathlineto{\pgfqpoint{4.552466in}{0.779248in}}%
\pgfpathlineto{\pgfqpoint{4.552762in}{0.779244in}}%
\pgfpathlineto{\pgfqpoint{4.553058in}{0.779240in}}%
\pgfpathlineto{\pgfqpoint{4.553354in}{0.779236in}}%
\pgfpathlineto{\pgfqpoint{4.553650in}{0.779233in}}%
\pgfpathlineto{\pgfqpoint{4.553946in}{0.779229in}}%
\pgfpathlineto{\pgfqpoint{4.554242in}{0.779225in}}%
\pgfpathlineto{\pgfqpoint{4.554538in}{0.779222in}}%
\pgfpathlineto{\pgfqpoint{4.554834in}{0.779218in}}%
\pgfpathlineto{\pgfqpoint{4.555130in}{0.779214in}}%
\pgfpathlineto{\pgfqpoint{4.555426in}{0.779210in}}%
\pgfpathlineto{\pgfqpoint{4.555722in}{0.779207in}}%
\pgfpathlineto{\pgfqpoint{4.556018in}{0.779203in}}%
\pgfpathlineto{\pgfqpoint{4.556314in}{0.779199in}}%
\pgfpathlineto{\pgfqpoint{4.556610in}{0.779195in}}%
\pgfpathlineto{\pgfqpoint{4.556906in}{0.779192in}}%
\pgfpathlineto{\pgfqpoint{4.557202in}{0.779188in}}%
\pgfpathlineto{\pgfqpoint{4.557498in}{0.779184in}}%
\pgfpathlineto{\pgfqpoint{4.557794in}{0.779181in}}%
\pgfpathlineto{\pgfqpoint{4.558091in}{0.779177in}}%
\pgfpathlineto{\pgfqpoint{4.558387in}{0.779173in}}%
\pgfpathlineto{\pgfqpoint{4.558683in}{0.779169in}}%
\pgfpathlineto{\pgfqpoint{4.558979in}{0.779166in}}%
\pgfpathlineto{\pgfqpoint{4.559275in}{0.779162in}}%
\pgfpathlineto{\pgfqpoint{4.559571in}{0.779158in}}%
\pgfpathlineto{\pgfqpoint{4.559867in}{0.779155in}}%
\pgfpathlineto{\pgfqpoint{4.560163in}{0.779151in}}%
\pgfpathlineto{\pgfqpoint{4.560459in}{0.779147in}}%
\pgfpathlineto{\pgfqpoint{4.560755in}{0.779143in}}%
\pgfpathlineto{\pgfqpoint{4.561051in}{0.779140in}}%
\pgfpathlineto{\pgfqpoint{4.561347in}{0.779136in}}%
\pgfpathlineto{\pgfqpoint{4.561643in}{0.779132in}}%
\pgfpathlineto{\pgfqpoint{4.561939in}{0.779129in}}%
\pgfpathlineto{\pgfqpoint{4.562235in}{0.779125in}}%
\pgfpathlineto{\pgfqpoint{4.562531in}{0.779121in}}%
\pgfpathlineto{\pgfqpoint{4.562827in}{0.779117in}}%
\pgfpathlineto{\pgfqpoint{4.563123in}{0.779114in}}%
\pgfpathlineto{\pgfqpoint{4.563419in}{0.779110in}}%
\pgfpathlineto{\pgfqpoint{4.563715in}{0.779106in}}%
\pgfpathlineto{\pgfqpoint{4.564011in}{0.779102in}}%
\pgfpathlineto{\pgfqpoint{4.564307in}{0.779099in}}%
\pgfpathlineto{\pgfqpoint{4.564603in}{0.779095in}}%
\pgfpathlineto{\pgfqpoint{4.564899in}{0.779091in}}%
\pgfpathlineto{\pgfqpoint{4.565195in}{0.779088in}}%
\pgfpathlineto{\pgfqpoint{4.565491in}{0.779084in}}%
\pgfpathlineto{\pgfqpoint{4.565787in}{0.779080in}}%
\pgfpathlineto{\pgfqpoint{4.566083in}{0.779076in}}%
\pgfpathlineto{\pgfqpoint{4.566379in}{0.779073in}}%
\pgfpathlineto{\pgfqpoint{4.566675in}{0.779069in}}%
\pgfpathlineto{\pgfqpoint{4.566971in}{0.779065in}}%
\pgfpathlineto{\pgfqpoint{4.567267in}{0.779062in}}%
\pgfpathlineto{\pgfqpoint{4.567563in}{0.779058in}}%
\pgfpathlineto{\pgfqpoint{4.567859in}{0.779054in}}%
\pgfpathlineto{\pgfqpoint{4.568155in}{0.779050in}}%
\pgfpathlineto{\pgfqpoint{4.568451in}{0.779047in}}%
\pgfpathlineto{\pgfqpoint{4.568747in}{0.779043in}}%
\pgfpathlineto{\pgfqpoint{4.569043in}{0.779039in}}%
\pgfpathlineto{\pgfqpoint{4.569339in}{0.779035in}}%
\pgfpathlineto{\pgfqpoint{4.569635in}{0.779032in}}%
\pgfpathlineto{\pgfqpoint{4.569931in}{0.779028in}}%
\pgfpathlineto{\pgfqpoint{4.570227in}{0.779024in}}%
\pgfpathlineto{\pgfqpoint{4.570523in}{0.779021in}}%
\pgfpathlineto{\pgfqpoint{4.570819in}{0.779017in}}%
\pgfpathlineto{\pgfqpoint{4.571115in}{0.779013in}}%
\pgfpathlineto{\pgfqpoint{4.571411in}{0.779009in}}%
\pgfpathlineto{\pgfqpoint{4.571707in}{0.779006in}}%
\pgfpathlineto{\pgfqpoint{4.572003in}{0.779002in}}%
\pgfpathlineto{\pgfqpoint{4.572299in}{0.778998in}}%
\pgfpathlineto{\pgfqpoint{4.572595in}{0.778995in}}%
\pgfpathlineto{\pgfqpoint{4.572891in}{0.778991in}}%
\pgfpathlineto{\pgfqpoint{4.573187in}{0.778987in}}%
\pgfpathlineto{\pgfqpoint{4.573483in}{0.778983in}}%
\pgfpathlineto{\pgfqpoint{4.573779in}{0.778980in}}%
\pgfpathlineto{\pgfqpoint{4.574075in}{0.778976in}}%
\pgfpathlineto{\pgfqpoint{4.574371in}{0.778972in}}%
\pgfpathlineto{\pgfqpoint{4.574667in}{0.778969in}}%
\pgfpathlineto{\pgfqpoint{4.574963in}{0.778965in}}%
\pgfpathlineto{\pgfqpoint{4.575259in}{0.778961in}}%
\pgfpathlineto{\pgfqpoint{4.575555in}{0.778957in}}%
\pgfpathlineto{\pgfqpoint{4.575851in}{0.778954in}}%
\pgfpathlineto{\pgfqpoint{4.576147in}{0.778925in}}%
\pgfpathlineto{\pgfqpoint{4.576443in}{0.778913in}}%
\pgfpathlineto{\pgfqpoint{4.576739in}{0.778909in}}%
\pgfpathlineto{\pgfqpoint{4.577035in}{0.778906in}}%
\pgfpathlineto{\pgfqpoint{4.577331in}{0.778902in}}%
\pgfpathlineto{\pgfqpoint{4.577627in}{0.778899in}}%
\pgfpathlineto{\pgfqpoint{4.577923in}{0.778895in}}%
\pgfpathlineto{\pgfqpoint{4.578219in}{0.778891in}}%
\pgfpathlineto{\pgfqpoint{4.578515in}{0.778888in}}%
\pgfpathlineto{\pgfqpoint{4.578811in}{0.778884in}}%
\pgfpathlineto{\pgfqpoint{4.579107in}{0.778881in}}%
\pgfpathlineto{\pgfqpoint{4.579403in}{0.778877in}}%
\pgfpathlineto{\pgfqpoint{4.579699in}{0.778873in}}%
\pgfpathlineto{\pgfqpoint{4.579995in}{0.778870in}}%
\pgfpathlineto{\pgfqpoint{4.580291in}{0.778866in}}%
\pgfpathlineto{\pgfqpoint{4.580587in}{0.778863in}}%
\pgfpathlineto{\pgfqpoint{4.580883in}{0.778859in}}%
\pgfpathlineto{\pgfqpoint{4.581179in}{0.778855in}}%
\pgfpathlineto{\pgfqpoint{4.581475in}{0.778852in}}%
\pgfpathlineto{\pgfqpoint{4.581771in}{0.778848in}}%
\pgfpathlineto{\pgfqpoint{4.582067in}{0.778845in}}%
\pgfpathlineto{\pgfqpoint{4.582363in}{0.778841in}}%
\pgfpathlineto{\pgfqpoint{4.582659in}{0.778837in}}%
\pgfpathlineto{\pgfqpoint{4.582955in}{0.778834in}}%
\pgfpathlineto{\pgfqpoint{4.583251in}{0.778830in}}%
\pgfpathlineto{\pgfqpoint{4.583547in}{0.778827in}}%
\pgfpathlineto{\pgfqpoint{4.583843in}{0.778823in}}%
\pgfpathlineto{\pgfqpoint{4.584139in}{0.778819in}}%
\pgfpathlineto{\pgfqpoint{4.584435in}{0.778816in}}%
\pgfpathlineto{\pgfqpoint{4.584731in}{0.778812in}}%
\pgfpathlineto{\pgfqpoint{4.585027in}{0.778809in}}%
\pgfpathlineto{\pgfqpoint{4.585323in}{0.778805in}}%
\pgfpathlineto{\pgfqpoint{4.585619in}{0.778801in}}%
\pgfpathlineto{\pgfqpoint{4.585915in}{0.778798in}}%
\pgfpathlineto{\pgfqpoint{4.586211in}{0.778794in}}%
\pgfpathlineto{\pgfqpoint{4.586507in}{0.778791in}}%
\pgfpathlineto{\pgfqpoint{4.586803in}{0.778787in}}%
\pgfpathlineto{\pgfqpoint{4.587099in}{0.778783in}}%
\pgfpathlineto{\pgfqpoint{4.587395in}{0.778780in}}%
\pgfpathlineto{\pgfqpoint{4.587691in}{0.778776in}}%
\pgfpathlineto{\pgfqpoint{4.587987in}{0.778773in}}%
\pgfpathlineto{\pgfqpoint{4.588283in}{0.778769in}}%
\pgfpathlineto{\pgfqpoint{4.588579in}{0.778765in}}%
\pgfpathlineto{\pgfqpoint{4.588875in}{0.778762in}}%
\pgfpathlineto{\pgfqpoint{4.589171in}{0.778758in}}%
\pgfpathlineto{\pgfqpoint{4.589467in}{0.778755in}}%
\pgfpathlineto{\pgfqpoint{4.589763in}{0.778187in}}%
\pgfpathlineto{\pgfqpoint{4.590059in}{0.777731in}}%
\pgfpathlineto{\pgfqpoint{4.590355in}{0.777720in}}%
\pgfpathlineto{\pgfqpoint{4.590651in}{0.777709in}}%
\pgfpathlineto{\pgfqpoint{4.590947in}{0.777698in}}%
\pgfpathlineto{\pgfqpoint{4.591243in}{0.777687in}}%
\pgfpathlineto{\pgfqpoint{4.591539in}{0.777677in}}%
\pgfpathlineto{\pgfqpoint{4.591835in}{0.777666in}}%
\pgfpathlineto{\pgfqpoint{4.592131in}{0.777655in}}%
\pgfpathlineto{\pgfqpoint{4.592427in}{0.777644in}}%
\pgfpathlineto{\pgfqpoint{4.592723in}{0.777633in}}%
\pgfpathlineto{\pgfqpoint{4.593019in}{0.777622in}}%
\pgfpathlineto{\pgfqpoint{4.593315in}{0.777611in}}%
\pgfpathlineto{\pgfqpoint{4.593611in}{0.777600in}}%
\pgfpathlineto{\pgfqpoint{4.593907in}{0.777589in}}%
\pgfpathlineto{\pgfqpoint{4.594203in}{0.777578in}}%
\pgfpathlineto{\pgfqpoint{4.594499in}{0.777567in}}%
\pgfpathlineto{\pgfqpoint{4.594795in}{0.777556in}}%
\pgfpathlineto{\pgfqpoint{4.595091in}{0.777545in}}%
\pgfpathlineto{\pgfqpoint{4.595387in}{0.777534in}}%
\pgfpathlineto{\pgfqpoint{4.595683in}{0.777523in}}%
\pgfpathlineto{\pgfqpoint{4.595979in}{0.777512in}}%
\pgfpathlineto{\pgfqpoint{4.596275in}{0.777502in}}%
\pgfpathlineto{\pgfqpoint{4.596571in}{0.777491in}}%
\pgfpathlineto{\pgfqpoint{4.596867in}{0.777505in}}%
\pgfpathlineto{\pgfqpoint{4.597163in}{0.777505in}}%
\pgfpathlineto{\pgfqpoint{4.597459in}{0.777508in}}%
\pgfpathlineto{\pgfqpoint{4.597755in}{0.777493in}}%
\pgfpathlineto{\pgfqpoint{4.598051in}{0.777479in}}%
\pgfpathlineto{\pgfqpoint{4.598347in}{0.777465in}}%
\pgfpathlineto{\pgfqpoint{4.598643in}{0.777451in}}%
\pgfpathlineto{\pgfqpoint{4.598939in}{0.777437in}}%
\pgfpathlineto{\pgfqpoint{4.599235in}{0.777423in}}%
\pgfpathlineto{\pgfqpoint{4.599531in}{0.777410in}}%
\pgfpathlineto{\pgfqpoint{4.599827in}{0.777396in}}%
\pgfpathlineto{\pgfqpoint{4.600123in}{0.777382in}}%
\pgfpathlineto{\pgfqpoint{4.600419in}{0.777368in}}%
\pgfpathlineto{\pgfqpoint{4.600715in}{0.777354in}}%
\pgfpathlineto{\pgfqpoint{4.601011in}{0.777340in}}%
\pgfpathlineto{\pgfqpoint{4.601307in}{0.777326in}}%
\pgfpathlineto{\pgfqpoint{4.601603in}{0.777312in}}%
\pgfpathlineto{\pgfqpoint{4.601899in}{0.777298in}}%
\pgfpathlineto{\pgfqpoint{4.602195in}{0.777284in}}%
\pgfpathlineto{\pgfqpoint{4.602491in}{0.777272in}}%
\pgfpathlineto{\pgfqpoint{4.602787in}{0.777260in}}%
\pgfpathlineto{\pgfqpoint{4.603083in}{0.777249in}}%
\pgfpathlineto{\pgfqpoint{4.603379in}{0.777238in}}%
\pgfpathlineto{\pgfqpoint{4.603675in}{0.777226in}}%
\pgfpathlineto{\pgfqpoint{4.603971in}{0.777215in}}%
\pgfpathlineto{\pgfqpoint{4.604267in}{0.777204in}}%
\pgfpathlineto{\pgfqpoint{4.604563in}{0.777192in}}%
\pgfpathlineto{\pgfqpoint{4.604859in}{0.777181in}}%
\pgfpathlineto{\pgfqpoint{4.605155in}{0.777171in}}%
\pgfpathlineto{\pgfqpoint{4.605451in}{0.777161in}}%
\pgfpathlineto{\pgfqpoint{4.605747in}{0.777151in}}%
\pgfpathlineto{\pgfqpoint{4.606043in}{0.777141in}}%
\pgfpathlineto{\pgfqpoint{4.606339in}{0.777131in}}%
\pgfpathlineto{\pgfqpoint{4.606635in}{0.777121in}}%
\pgfpathlineto{\pgfqpoint{4.606931in}{0.777111in}}%
\pgfpathlineto{\pgfqpoint{4.607227in}{0.777101in}}%
\pgfpathlineto{\pgfqpoint{4.607523in}{0.777091in}}%
\pgfpathlineto{\pgfqpoint{4.607819in}{0.777081in}}%
\pgfpathlineto{\pgfqpoint{4.608115in}{0.777071in}}%
\pgfpathlineto{\pgfqpoint{4.608411in}{0.777062in}}%
\pgfpathlineto{\pgfqpoint{4.608707in}{0.777052in}}%
\pgfpathlineto{\pgfqpoint{4.609003in}{0.777042in}}%
\pgfpathlineto{\pgfqpoint{4.609299in}{0.777032in}}%
\pgfpathlineto{\pgfqpoint{4.609595in}{0.777022in}}%
\pgfpathlineto{\pgfqpoint{4.609891in}{0.777012in}}%
\pgfpathlineto{\pgfqpoint{4.610187in}{0.777004in}}%
\pgfpathlineto{\pgfqpoint{4.610483in}{0.776997in}}%
\pgfpathlineto{\pgfqpoint{4.610779in}{0.776989in}}%
\pgfpathlineto{\pgfqpoint{4.611075in}{0.776981in}}%
\pgfpathlineto{\pgfqpoint{4.611371in}{0.776973in}}%
\pgfpathlineto{\pgfqpoint{4.611667in}{0.776965in}}%
\pgfpathlineto{\pgfqpoint{4.611963in}{0.776960in}}%
\pgfpathlineto{\pgfqpoint{4.612259in}{0.776959in}}%
\pgfpathlineto{\pgfqpoint{4.612555in}{0.776958in}}%
\pgfpathlineto{\pgfqpoint{4.612851in}{0.776957in}}%
\pgfpathlineto{\pgfqpoint{4.613147in}{0.776956in}}%
\pgfpathlineto{\pgfqpoint{4.613443in}{0.776956in}}%
\pgfpathlineto{\pgfqpoint{4.613739in}{0.776955in}}%
\pgfpathlineto{\pgfqpoint{4.614035in}{0.776954in}}%
\pgfpathlineto{\pgfqpoint{4.614331in}{0.776953in}}%
\pgfpathlineto{\pgfqpoint{4.614627in}{0.776952in}}%
\pgfpathlineto{\pgfqpoint{4.614923in}{0.776952in}}%
\pgfpathlineto{\pgfqpoint{4.615219in}{0.776951in}}%
\pgfpathlineto{\pgfqpoint{4.615515in}{0.776950in}}%
\pgfpathlineto{\pgfqpoint{4.615811in}{0.776949in}}%
\pgfpathlineto{\pgfqpoint{4.616107in}{0.776948in}}%
\pgfpathlineto{\pgfqpoint{4.616403in}{0.776948in}}%
\pgfpathlineto{\pgfqpoint{4.616699in}{0.776947in}}%
\pgfpathlineto{\pgfqpoint{4.616995in}{0.776946in}}%
\pgfpathlineto{\pgfqpoint{4.617291in}{0.776945in}}%
\pgfpathlineto{\pgfqpoint{4.617587in}{0.776944in}}%
\pgfpathlineto{\pgfqpoint{4.617883in}{0.776944in}}%
\pgfpathlineto{\pgfqpoint{4.618179in}{0.776943in}}%
\pgfpathlineto{\pgfqpoint{4.618475in}{0.776942in}}%
\pgfpathlineto{\pgfqpoint{4.618771in}{0.776941in}}%
\pgfpathlineto{\pgfqpoint{4.619067in}{0.776941in}}%
\pgfpathlineto{\pgfqpoint{4.619363in}{0.776940in}}%
\pgfpathlineto{\pgfqpoint{4.619659in}{0.776939in}}%
\pgfpathlineto{\pgfqpoint{4.619955in}{0.776938in}}%
\pgfpathlineto{\pgfqpoint{4.620251in}{0.776937in}}%
\pgfpathlineto{\pgfqpoint{4.620547in}{0.776937in}}%
\pgfpathlineto{\pgfqpoint{4.620843in}{0.776936in}}%
\pgfpathlineto{\pgfqpoint{4.621139in}{0.776935in}}%
\pgfpathlineto{\pgfqpoint{4.621435in}{0.776934in}}%
\pgfpathlineto{\pgfqpoint{4.621731in}{0.776933in}}%
\pgfpathlineto{\pgfqpoint{4.622027in}{0.776933in}}%
\pgfpathlineto{\pgfqpoint{4.622323in}{0.776932in}}%
\pgfpathlineto{\pgfqpoint{4.622619in}{0.776931in}}%
\pgfpathlineto{\pgfqpoint{4.622915in}{0.776930in}}%
\pgfpathlineto{\pgfqpoint{4.623211in}{0.776929in}}%
\pgfpathlineto{\pgfqpoint{4.623507in}{0.776929in}}%
\pgfpathlineto{\pgfqpoint{4.623803in}{0.776927in}}%
\pgfpathlineto{\pgfqpoint{4.624099in}{0.776925in}}%
\pgfpathlineto{\pgfqpoint{4.624395in}{0.776922in}}%
\pgfpathlineto{\pgfqpoint{4.624691in}{0.776920in}}%
\pgfpathlineto{\pgfqpoint{4.624987in}{0.776917in}}%
\pgfpathlineto{\pgfqpoint{4.625283in}{0.776915in}}%
\pgfpathlineto{\pgfqpoint{4.625580in}{0.776912in}}%
\pgfpathlineto{\pgfqpoint{4.625876in}{0.776907in}}%
\pgfpathlineto{\pgfqpoint{4.626172in}{0.776902in}}%
\pgfpathlineto{\pgfqpoint{4.626468in}{0.776896in}}%
\pgfpathlineto{\pgfqpoint{4.626764in}{0.776890in}}%
\pgfpathlineto{\pgfqpoint{4.627060in}{0.776883in}}%
\pgfpathlineto{\pgfqpoint{4.627356in}{0.776875in}}%
\pgfpathlineto{\pgfqpoint{4.627652in}{0.776868in}}%
\pgfpathlineto{\pgfqpoint{4.627948in}{0.776861in}}%
\pgfpathlineto{\pgfqpoint{4.628244in}{0.776854in}}%
\pgfpathlineto{\pgfqpoint{4.628540in}{0.776847in}}%
\pgfpathlineto{\pgfqpoint{4.628836in}{0.776839in}}%
\pgfpathlineto{\pgfqpoint{4.629132in}{0.776832in}}%
\pgfpathlineto{\pgfqpoint{4.629428in}{0.776825in}}%
\pgfpathlineto{\pgfqpoint{4.629724in}{0.776818in}}%
\pgfpathlineto{\pgfqpoint{4.630020in}{0.776810in}}%
\pgfpathlineto{\pgfqpoint{4.630316in}{0.776803in}}%
\pgfpathlineto{\pgfqpoint{4.630612in}{0.776796in}}%
\pgfpathlineto{\pgfqpoint{4.630908in}{0.776789in}}%
\pgfpathlineto{\pgfqpoint{4.631204in}{0.776782in}}%
\pgfpathlineto{\pgfqpoint{4.631500in}{0.776774in}}%
\pgfpathlineto{\pgfqpoint{4.631796in}{0.776767in}}%
\pgfpathlineto{\pgfqpoint{4.632092in}{0.776760in}}%
\pgfpathlineto{\pgfqpoint{4.632388in}{0.776753in}}%
\pgfpathlineto{\pgfqpoint{4.632684in}{0.776746in}}%
\pgfpathlineto{\pgfqpoint{4.632980in}{0.776738in}}%
\pgfpathlineto{\pgfqpoint{4.633276in}{0.776731in}}%
\pgfpathlineto{\pgfqpoint{4.633572in}{0.776724in}}%
\pgfpathlineto{\pgfqpoint{4.633868in}{0.776717in}}%
\pgfpathlineto{\pgfqpoint{4.634164in}{0.776709in}}%
\pgfpathlineto{\pgfqpoint{4.634460in}{0.776702in}}%
\pgfpathlineto{\pgfqpoint{4.634756in}{0.776695in}}%
\pgfpathlineto{\pgfqpoint{4.635052in}{0.776688in}}%
\pgfpathlineto{\pgfqpoint{4.635348in}{0.776681in}}%
\pgfpathlineto{\pgfqpoint{4.635644in}{0.776673in}}%
\pgfpathlineto{\pgfqpoint{4.635940in}{0.776666in}}%
\pgfpathlineto{\pgfqpoint{4.636236in}{0.776659in}}%
\pgfpathlineto{\pgfqpoint{4.636532in}{0.776652in}}%
\pgfpathlineto{\pgfqpoint{4.636828in}{0.776645in}}%
\pgfpathlineto{\pgfqpoint{4.637124in}{0.776637in}}%
\pgfpathlineto{\pgfqpoint{4.637420in}{0.776630in}}%
\pgfpathlineto{\pgfqpoint{4.637716in}{0.776623in}}%
\pgfpathlineto{\pgfqpoint{4.638012in}{0.776616in}}%
\pgfpathlineto{\pgfqpoint{4.638308in}{0.776609in}}%
\pgfpathlineto{\pgfqpoint{4.638604in}{0.776601in}}%
\pgfpathlineto{\pgfqpoint{4.638900in}{0.776594in}}%
\pgfpathlineto{\pgfqpoint{4.639196in}{0.776587in}}%
\pgfpathlineto{\pgfqpoint{4.639492in}{0.776580in}}%
\pgfpathlineto{\pgfqpoint{4.639788in}{0.776572in}}%
\pgfpathlineto{\pgfqpoint{4.640084in}{0.776565in}}%
\pgfpathlineto{\pgfqpoint{4.640380in}{0.776558in}}%
\pgfpathlineto{\pgfqpoint{4.640676in}{0.776551in}}%
\pgfpathlineto{\pgfqpoint{4.640972in}{0.776544in}}%
\pgfpathlineto{\pgfqpoint{4.641268in}{0.776536in}}%
\pgfpathlineto{\pgfqpoint{4.641564in}{0.776529in}}%
\pgfpathlineto{\pgfqpoint{4.641860in}{0.776522in}}%
\pgfpathlineto{\pgfqpoint{4.642156in}{0.776515in}}%
\pgfpathlineto{\pgfqpoint{4.642452in}{0.776508in}}%
\pgfpathlineto{\pgfqpoint{4.642748in}{0.776500in}}%
\pgfpathlineto{\pgfqpoint{4.643044in}{0.776493in}}%
\pgfpathlineto{\pgfqpoint{4.643340in}{0.776486in}}%
\pgfpathlineto{\pgfqpoint{4.643636in}{0.776479in}}%
\pgfpathlineto{\pgfqpoint{4.643932in}{0.776471in}}%
\pgfpathlineto{\pgfqpoint{4.644228in}{0.776464in}}%
\pgfpathlineto{\pgfqpoint{4.644524in}{0.776457in}}%
\pgfpathlineto{\pgfqpoint{4.644820in}{0.776450in}}%
\pgfpathlineto{\pgfqpoint{4.645116in}{0.776440in}}%
\pgfpathlineto{\pgfqpoint{4.645412in}{0.776219in}}%
\pgfpathlineto{\pgfqpoint{4.645708in}{0.775962in}}%
\pgfpathlineto{\pgfqpoint{4.646004in}{0.775964in}}%
\pgfpathlineto{\pgfqpoint{4.646300in}{0.775979in}}%
\pgfpathlineto{\pgfqpoint{4.646596in}{0.775994in}}%
\pgfpathlineto{\pgfqpoint{4.646892in}{0.776009in}}%
\pgfpathlineto{\pgfqpoint{4.647188in}{0.776025in}}%
\pgfpathlineto{\pgfqpoint{4.647484in}{0.776040in}}%
\pgfpathlineto{\pgfqpoint{4.647780in}{0.776055in}}%
\pgfpathlineto{\pgfqpoint{4.648076in}{0.776071in}}%
\pgfpathlineto{\pgfqpoint{4.648372in}{0.776086in}}%
\pgfpathlineto{\pgfqpoint{4.648668in}{0.776101in}}%
\pgfpathlineto{\pgfqpoint{4.648964in}{0.776117in}}%
\pgfpathlineto{\pgfqpoint{4.649260in}{0.776132in}}%
\pgfpathlineto{\pgfqpoint{4.649556in}{0.776147in}}%
\pgfpathlineto{\pgfqpoint{4.649852in}{0.776163in}}%
\pgfpathlineto{\pgfqpoint{4.650148in}{0.776178in}}%
\pgfpathlineto{\pgfqpoint{4.650444in}{0.776193in}}%
\pgfpathlineto{\pgfqpoint{4.650740in}{0.776208in}}%
\pgfpathlineto{\pgfqpoint{4.651036in}{0.776224in}}%
\pgfpathlineto{\pgfqpoint{4.651332in}{0.776239in}}%
\pgfpathlineto{\pgfqpoint{4.651628in}{0.776254in}}%
\pgfpathlineto{\pgfqpoint{4.651924in}{0.776270in}}%
\pgfpathlineto{\pgfqpoint{4.652220in}{0.776285in}}%
\pgfpathlineto{\pgfqpoint{4.652516in}{0.776300in}}%
\pgfpathlineto{\pgfqpoint{4.652812in}{0.776279in}}%
\pgfpathlineto{\pgfqpoint{4.653108in}{0.775845in}}%
\pgfpathlineto{\pgfqpoint{4.653404in}{0.775876in}}%
\pgfpathlineto{\pgfqpoint{4.653700in}{0.775906in}}%
\pgfpathlineto{\pgfqpoint{4.653996in}{0.775937in}}%
\pgfpathlineto{\pgfqpoint{4.654292in}{0.775968in}}%
\pgfpathlineto{\pgfqpoint{4.654588in}{0.775999in}}%
\pgfpathlineto{\pgfqpoint{4.654884in}{0.776030in}}%
\pgfpathlineto{\pgfqpoint{4.655180in}{0.776061in}}%
\pgfpathlineto{\pgfqpoint{4.655476in}{0.776091in}}%
\pgfpathlineto{\pgfqpoint{4.655772in}{0.776121in}}%
\pgfpathlineto{\pgfqpoint{4.656068in}{0.776150in}}%
\pgfpathlineto{\pgfqpoint{4.656364in}{0.776179in}}%
\pgfpathlineto{\pgfqpoint{4.656660in}{0.776207in}}%
\pgfpathlineto{\pgfqpoint{4.656956in}{0.776236in}}%
\pgfpathlineto{\pgfqpoint{4.657252in}{0.776265in}}%
\pgfpathlineto{\pgfqpoint{4.657548in}{0.776294in}}%
\pgfpathlineto{\pgfqpoint{4.657844in}{0.776323in}}%
\pgfpathlineto{\pgfqpoint{4.658140in}{0.776352in}}%
\pgfpathlineto{\pgfqpoint{4.658436in}{0.776381in}}%
\pgfpathlineto{\pgfqpoint{4.658732in}{0.776409in}}%
\pgfpathlineto{\pgfqpoint{4.659028in}{0.776438in}}%
\pgfpathlineto{\pgfqpoint{4.659324in}{0.776457in}}%
\pgfpathlineto{\pgfqpoint{4.659620in}{0.776454in}}%
\pgfpathlineto{\pgfqpoint{4.659916in}{0.776451in}}%
\pgfpathlineto{\pgfqpoint{4.660212in}{0.776448in}}%
\pgfpathlineto{\pgfqpoint{4.660508in}{0.776444in}}%
\pgfpathlineto{\pgfqpoint{4.660804in}{0.776441in}}%
\pgfpathlineto{\pgfqpoint{4.661100in}{0.776438in}}%
\pgfpathlineto{\pgfqpoint{4.661396in}{0.776429in}}%
\pgfpathlineto{\pgfqpoint{4.661692in}{0.776394in}}%
\pgfpathlineto{\pgfqpoint{4.661988in}{0.776355in}}%
\pgfpathlineto{\pgfqpoint{4.662284in}{0.776316in}}%
\pgfpathlineto{\pgfqpoint{4.662580in}{0.776278in}}%
\pgfpathlineto{\pgfqpoint{4.662876in}{0.776239in}}%
\pgfpathlineto{\pgfqpoint{4.663172in}{0.776200in}}%
\pgfpathlineto{\pgfqpoint{4.663468in}{0.776162in}}%
\pgfpathlineto{\pgfqpoint{4.663764in}{0.776123in}}%
\pgfpathlineto{\pgfqpoint{4.664060in}{0.776084in}}%
\pgfpathlineto{\pgfqpoint{4.664356in}{0.776045in}}%
\pgfpathlineto{\pgfqpoint{4.664652in}{0.776007in}}%
\pgfpathlineto{\pgfqpoint{4.664948in}{0.775968in}}%
\pgfpathlineto{\pgfqpoint{4.665244in}{0.775929in}}%
\pgfpathlineto{\pgfqpoint{4.665540in}{0.775891in}}%
\pgfpathlineto{\pgfqpoint{4.665836in}{0.775852in}}%
\pgfpathlineto{\pgfqpoint{4.666132in}{0.775813in}}%
\pgfpathlineto{\pgfqpoint{4.666428in}{0.775774in}}%
\pgfpathlineto{\pgfqpoint{4.666724in}{0.775736in}}%
\pgfpathlineto{\pgfqpoint{4.667020in}{0.775697in}}%
\pgfpathlineto{\pgfqpoint{4.667316in}{0.775658in}}%
\pgfpathlineto{\pgfqpoint{4.667612in}{0.775619in}}%
\pgfpathlineto{\pgfqpoint{4.667908in}{0.775581in}}%
\pgfpathlineto{\pgfqpoint{4.668204in}{0.775542in}}%
\pgfpathlineto{\pgfqpoint{4.668500in}{0.775503in}}%
\pgfpathlineto{\pgfqpoint{4.668796in}{0.775475in}}%
\pgfpathlineto{\pgfqpoint{4.669092in}{0.775474in}}%
\pgfpathlineto{\pgfqpoint{4.669388in}{0.775470in}}%
\pgfpathlineto{\pgfqpoint{4.669684in}{0.775466in}}%
\pgfpathlineto{\pgfqpoint{4.669980in}{0.775462in}}%
\pgfpathlineto{\pgfqpoint{4.670276in}{0.775458in}}%
\pgfpathlineto{\pgfqpoint{4.670572in}{0.775454in}}%
\pgfpathlineto{\pgfqpoint{4.670868in}{0.775450in}}%
\pgfpathlineto{\pgfqpoint{4.671164in}{0.775446in}}%
\pgfpathlineto{\pgfqpoint{4.671460in}{0.775442in}}%
\pgfpathlineto{\pgfqpoint{4.671756in}{0.775438in}}%
\pgfpathlineto{\pgfqpoint{4.672052in}{0.775434in}}%
\pgfpathlineto{\pgfqpoint{4.672348in}{0.775430in}}%
\pgfpathlineto{\pgfqpoint{4.672644in}{0.775426in}}%
\pgfpathlineto{\pgfqpoint{4.672940in}{0.775422in}}%
\pgfpathlineto{\pgfqpoint{4.673236in}{0.775418in}}%
\pgfpathlineto{\pgfqpoint{4.673532in}{0.775414in}}%
\pgfpathlineto{\pgfqpoint{4.673828in}{0.775410in}}%
\pgfpathlineto{\pgfqpoint{4.674124in}{0.775407in}}%
\pgfpathlineto{\pgfqpoint{4.674420in}{0.775407in}}%
\pgfpathlineto{\pgfqpoint{4.674716in}{0.775408in}}%
\pgfpathlineto{\pgfqpoint{4.675012in}{0.775409in}}%
\pgfpathlineto{\pgfqpoint{4.675308in}{0.775387in}}%
\pgfpathlineto{\pgfqpoint{4.675604in}{0.775339in}}%
\pgfpathlineto{\pgfqpoint{4.675900in}{0.775323in}}%
\pgfpathlineto{\pgfqpoint{4.676196in}{0.775410in}}%
\pgfpathlineto{\pgfqpoint{4.676492in}{0.775403in}}%
\pgfpathlineto{\pgfqpoint{4.676788in}{0.775397in}}%
\pgfpathlineto{\pgfqpoint{4.677084in}{0.775392in}}%
\pgfpathlineto{\pgfqpoint{4.677380in}{0.775387in}}%
\pgfpathlineto{\pgfqpoint{4.677676in}{0.775381in}}%
\pgfpathlineto{\pgfqpoint{4.677972in}{0.775376in}}%
\pgfpathlineto{\pgfqpoint{4.678268in}{0.775371in}}%
\pgfpathlineto{\pgfqpoint{4.678564in}{0.775365in}}%
\pgfpathlineto{\pgfqpoint{4.678860in}{0.775360in}}%
\pgfpathlineto{\pgfqpoint{4.679156in}{0.775355in}}%
\pgfpathlineto{\pgfqpoint{4.679452in}{0.775349in}}%
\pgfpathlineto{\pgfqpoint{4.679748in}{0.775344in}}%
\pgfpathlineto{\pgfqpoint{4.680044in}{0.775339in}}%
\pgfpathlineto{\pgfqpoint{4.680340in}{0.775333in}}%
\pgfpathlineto{\pgfqpoint{4.680636in}{0.775328in}}%
\pgfpathlineto{\pgfqpoint{4.680932in}{0.775323in}}%
\pgfpathlineto{\pgfqpoint{4.681228in}{0.775317in}}%
\pgfpathlineto{\pgfqpoint{4.681524in}{0.775312in}}%
\pgfpathlineto{\pgfqpoint{4.681820in}{0.775307in}}%
\pgfpathlineto{\pgfqpoint{4.682116in}{0.775301in}}%
\pgfpathlineto{\pgfqpoint{4.682412in}{0.775296in}}%
\pgfpathlineto{\pgfqpoint{4.682708in}{0.775290in}}%
\pgfpathlineto{\pgfqpoint{4.683004in}{0.775285in}}%
\pgfpathlineto{\pgfqpoint{4.683300in}{0.775280in}}%
\pgfpathlineto{\pgfqpoint{4.683596in}{0.775274in}}%
\pgfpathlineto{\pgfqpoint{4.683892in}{0.775269in}}%
\pgfpathlineto{\pgfqpoint{4.684188in}{0.775264in}}%
\pgfpathlineto{\pgfqpoint{4.684484in}{0.775258in}}%
\pgfpathlineto{\pgfqpoint{4.684780in}{0.775253in}}%
\pgfpathlineto{\pgfqpoint{4.685076in}{0.775248in}}%
\pgfpathlineto{\pgfqpoint{4.685372in}{0.775242in}}%
\pgfpathlineto{\pgfqpoint{4.685668in}{0.775237in}}%
\pgfpathlineto{\pgfqpoint{4.685964in}{0.775232in}}%
\pgfpathlineto{\pgfqpoint{4.686260in}{0.775226in}}%
\pgfpathlineto{\pgfqpoint{4.686556in}{0.775221in}}%
\pgfpathlineto{\pgfqpoint{4.686852in}{0.775216in}}%
\pgfpathlineto{\pgfqpoint{4.687148in}{0.775210in}}%
\pgfpathlineto{\pgfqpoint{4.687444in}{0.775205in}}%
\pgfpathlineto{\pgfqpoint{4.687740in}{0.775200in}}%
\pgfpathlineto{\pgfqpoint{4.688036in}{0.775194in}}%
\pgfpathlineto{\pgfqpoint{4.688332in}{0.775189in}}%
\pgfpathlineto{\pgfqpoint{4.688628in}{0.775184in}}%
\pgfpathlineto{\pgfqpoint{4.688924in}{0.775178in}}%
\pgfpathlineto{\pgfqpoint{4.689220in}{0.775173in}}%
\pgfpathlineto{\pgfqpoint{4.689516in}{0.775168in}}%
\pgfpathlineto{\pgfqpoint{4.689812in}{0.775162in}}%
\pgfpathlineto{\pgfqpoint{4.690108in}{0.775157in}}%
\pgfpathlineto{\pgfqpoint{4.690404in}{0.775152in}}%
\pgfpathlineto{\pgfqpoint{4.690700in}{0.775146in}}%
\pgfpathlineto{\pgfqpoint{4.690996in}{0.775141in}}%
\pgfpathlineto{\pgfqpoint{4.691292in}{0.775136in}}%
\pgfpathlineto{\pgfqpoint{4.691588in}{0.775130in}}%
\pgfpathlineto{\pgfqpoint{4.691884in}{0.775125in}}%
\pgfpathlineto{\pgfqpoint{4.692180in}{0.775120in}}%
\pgfpathlineto{\pgfqpoint{4.692476in}{0.775114in}}%
\pgfpathlineto{\pgfqpoint{4.692772in}{0.775109in}}%
\pgfpathlineto{\pgfqpoint{4.693069in}{0.775104in}}%
\pgfpathlineto{\pgfqpoint{4.693365in}{0.775098in}}%
\pgfpathlineto{\pgfqpoint{4.693661in}{0.775093in}}%
\pgfpathlineto{\pgfqpoint{4.693957in}{0.775087in}}%
\pgfpathlineto{\pgfqpoint{4.694253in}{0.775082in}}%
\pgfpathlineto{\pgfqpoint{4.694549in}{0.775077in}}%
\pgfpathlineto{\pgfqpoint{4.694845in}{0.775071in}}%
\pgfpathlineto{\pgfqpoint{4.695141in}{0.775066in}}%
\pgfpathlineto{\pgfqpoint{4.695437in}{0.775061in}}%
\pgfpathlineto{\pgfqpoint{4.695733in}{0.775055in}}%
\pgfpathlineto{\pgfqpoint{4.696029in}{0.775050in}}%
\pgfpathlineto{\pgfqpoint{4.696325in}{0.775045in}}%
\pgfpathlineto{\pgfqpoint{4.696621in}{0.775039in}}%
\pgfpathlineto{\pgfqpoint{4.696917in}{0.775034in}}%
\pgfpathlineto{\pgfqpoint{4.697213in}{0.775029in}}%
\pgfpathlineto{\pgfqpoint{4.697509in}{0.775023in}}%
\pgfpathlineto{\pgfqpoint{4.697805in}{0.775018in}}%
\pgfpathlineto{\pgfqpoint{4.698101in}{0.775013in}}%
\pgfpathlineto{\pgfqpoint{4.698397in}{0.775007in}}%
\pgfpathlineto{\pgfqpoint{4.698693in}{0.775002in}}%
\pgfpathlineto{\pgfqpoint{4.698989in}{0.774997in}}%
\pgfpathlineto{\pgfqpoint{4.699285in}{0.774991in}}%
\pgfpathlineto{\pgfqpoint{4.699581in}{0.774986in}}%
\pgfpathlineto{\pgfqpoint{4.699877in}{0.774981in}}%
\pgfpathlineto{\pgfqpoint{4.700173in}{0.774975in}}%
\pgfpathlineto{\pgfqpoint{4.700469in}{0.774970in}}%
\pgfpathlineto{\pgfqpoint{4.700765in}{0.774965in}}%
\pgfpathlineto{\pgfqpoint{4.701061in}{0.774959in}}%
\pgfpathlineto{\pgfqpoint{4.701357in}{0.774954in}}%
\pgfpathlineto{\pgfqpoint{4.701653in}{0.774949in}}%
\pgfpathlineto{\pgfqpoint{4.701949in}{0.774943in}}%
\pgfpathlineto{\pgfqpoint{4.702245in}{0.774938in}}%
\pgfpathlineto{\pgfqpoint{4.702541in}{0.774944in}}%
\pgfpathlineto{\pgfqpoint{4.702837in}{0.774964in}}%
\pgfpathlineto{\pgfqpoint{4.703133in}{0.774697in}}%
\pgfpathlineto{\pgfqpoint{4.703429in}{0.774173in}}%
\pgfpathlineto{\pgfqpoint{4.703725in}{0.774168in}}%
\pgfpathlineto{\pgfqpoint{4.704021in}{0.774078in}}%
\pgfpathlineto{\pgfqpoint{4.704317in}{0.774018in}}%
\pgfpathlineto{\pgfqpoint{4.704613in}{0.774002in}}%
\pgfpathlineto{\pgfqpoint{4.704909in}{0.773986in}}%
\pgfpathlineto{\pgfqpoint{4.705205in}{0.773969in}}%
\pgfpathlineto{\pgfqpoint{4.705501in}{0.773953in}}%
\pgfpathlineto{\pgfqpoint{4.705797in}{0.773937in}}%
\pgfpathlineto{\pgfqpoint{4.706093in}{0.773920in}}%
\pgfpathlineto{\pgfqpoint{4.706389in}{0.773904in}}%
\pgfpathlineto{\pgfqpoint{4.706685in}{0.773888in}}%
\pgfpathlineto{\pgfqpoint{4.706981in}{0.773872in}}%
\pgfpathlineto{\pgfqpoint{4.707277in}{0.773855in}}%
\pgfpathlineto{\pgfqpoint{4.707573in}{0.773839in}}%
\pgfpathlineto{\pgfqpoint{4.707869in}{0.773823in}}%
\pgfpathlineto{\pgfqpoint{4.708165in}{0.773807in}}%
\pgfpathlineto{\pgfqpoint{4.708461in}{0.773790in}}%
\pgfpathlineto{\pgfqpoint{4.708757in}{0.773774in}}%
\pgfpathlineto{\pgfqpoint{4.709053in}{0.773757in}}%
\pgfpathlineto{\pgfqpoint{4.709349in}{0.773725in}}%
\pgfpathlineto{\pgfqpoint{4.709645in}{0.773690in}}%
\pgfpathlineto{\pgfqpoint{4.709941in}{0.773666in}}%
\pgfpathlineto{\pgfqpoint{4.710237in}{0.773645in}}%
\pgfpathlineto{\pgfqpoint{4.710533in}{0.773624in}}%
\pgfpathlineto{\pgfqpoint{4.710829in}{0.773603in}}%
\pgfpathlineto{\pgfqpoint{4.711125in}{0.773583in}}%
\pgfpathlineto{\pgfqpoint{4.711421in}{0.773598in}}%
\pgfpathlineto{\pgfqpoint{4.711717in}{0.773646in}}%
\pgfpathlineto{\pgfqpoint{4.712013in}{0.773653in}}%
\pgfpathlineto{\pgfqpoint{4.712309in}{0.773655in}}%
\pgfpathlineto{\pgfqpoint{4.712605in}{0.773659in}}%
\pgfpathlineto{\pgfqpoint{4.712901in}{0.773663in}}%
\pgfpathlineto{\pgfqpoint{4.713197in}{0.773666in}}%
\pgfpathlineto{\pgfqpoint{4.713493in}{0.773670in}}%
\pgfpathlineto{\pgfqpoint{4.713789in}{0.773674in}}%
\pgfpathlineto{\pgfqpoint{4.714085in}{0.773678in}}%
\pgfpathlineto{\pgfqpoint{4.714381in}{0.773681in}}%
\pgfpathlineto{\pgfqpoint{4.714677in}{0.773685in}}%
\pgfpathlineto{\pgfqpoint{4.714973in}{0.773689in}}%
\pgfpathlineto{\pgfqpoint{4.715269in}{0.773692in}}%
\pgfpathlineto{\pgfqpoint{4.715565in}{0.773696in}}%
\pgfpathlineto{\pgfqpoint{4.715861in}{0.773700in}}%
\pgfpathlineto{\pgfqpoint{4.716157in}{0.773704in}}%
\pgfpathlineto{\pgfqpoint{4.716453in}{0.773707in}}%
\pgfpathlineto{\pgfqpoint{4.716749in}{0.773711in}}%
\pgfpathlineto{\pgfqpoint{4.717045in}{0.773715in}}%
\pgfpathlineto{\pgfqpoint{4.717341in}{0.773719in}}%
\pgfpathlineto{\pgfqpoint{4.717637in}{0.773722in}}%
\pgfpathlineto{\pgfqpoint{4.717933in}{0.773726in}}%
\pgfpathlineto{\pgfqpoint{4.718229in}{0.773730in}}%
\pgfpathlineto{\pgfqpoint{4.718525in}{0.773732in}}%
\pgfpathlineto{\pgfqpoint{4.718821in}{0.773734in}}%
\pgfpathlineto{\pgfqpoint{4.719117in}{0.773736in}}%
\pgfpathlineto{\pgfqpoint{4.719413in}{0.773738in}}%
\pgfpathlineto{\pgfqpoint{4.719709in}{0.773740in}}%
\pgfpathlineto{\pgfqpoint{4.720005in}{0.773741in}}%
\pgfpathlineto{\pgfqpoint{4.720301in}{0.773743in}}%
\pgfpathlineto{\pgfqpoint{4.720597in}{0.773745in}}%
\pgfpathlineto{\pgfqpoint{4.720893in}{0.773747in}}%
\pgfpathlineto{\pgfqpoint{4.721189in}{0.773749in}}%
\pgfpathlineto{\pgfqpoint{4.721485in}{0.773751in}}%
\pgfpathlineto{\pgfqpoint{4.721781in}{0.773753in}}%
\pgfpathlineto{\pgfqpoint{4.722077in}{0.773755in}}%
\pgfpathlineto{\pgfqpoint{4.722373in}{0.773756in}}%
\pgfpathlineto{\pgfqpoint{4.722669in}{0.773758in}}%
\pgfpathlineto{\pgfqpoint{4.722965in}{0.773760in}}%
\pgfpathlineto{\pgfqpoint{4.723261in}{0.773762in}}%
\pgfpathlineto{\pgfqpoint{4.723557in}{0.773764in}}%
\pgfpathlineto{\pgfqpoint{4.723853in}{0.773766in}}%
\pgfpathlineto{\pgfqpoint{4.724149in}{0.773768in}}%
\pgfpathlineto{\pgfqpoint{4.724445in}{0.773770in}}%
\pgfpathlineto{\pgfqpoint{4.724741in}{0.774028in}}%
\pgfpathlineto{\pgfqpoint{4.725037in}{0.774191in}}%
\pgfpathlineto{\pgfqpoint{4.725333in}{0.774192in}}%
\pgfpathlineto{\pgfqpoint{4.725629in}{0.774181in}}%
\pgfpathlineto{\pgfqpoint{4.725925in}{0.774180in}}%
\pgfpathlineto{\pgfqpoint{4.726221in}{0.774179in}}%
\pgfpathlineto{\pgfqpoint{4.726517in}{0.774177in}}%
\pgfpathlineto{\pgfqpoint{4.726813in}{0.774176in}}%
\pgfpathlineto{\pgfqpoint{4.727109in}{0.774175in}}%
\pgfpathlineto{\pgfqpoint{4.727405in}{0.774174in}}%
\pgfpathlineto{\pgfqpoint{4.727701in}{0.774173in}}%
\pgfpathlineto{\pgfqpoint{4.727997in}{0.774172in}}%
\pgfpathlineto{\pgfqpoint{4.728293in}{0.774171in}}%
\pgfpathlineto{\pgfqpoint{4.728589in}{0.774170in}}%
\pgfpathlineto{\pgfqpoint{4.728885in}{0.774169in}}%
\pgfpathlineto{\pgfqpoint{4.729181in}{0.774168in}}%
\pgfpathlineto{\pgfqpoint{4.729477in}{0.774167in}}%
\pgfpathlineto{\pgfqpoint{4.729773in}{0.774166in}}%
\pgfpathlineto{\pgfqpoint{4.730069in}{0.774165in}}%
\pgfpathlineto{\pgfqpoint{4.730365in}{0.774164in}}%
\pgfpathlineto{\pgfqpoint{4.730661in}{0.774163in}}%
\pgfpathlineto{\pgfqpoint{4.730957in}{0.774162in}}%
\pgfpathlineto{\pgfqpoint{4.731253in}{0.774160in}}%
\pgfpathlineto{\pgfqpoint{4.731549in}{0.774159in}}%
\pgfpathlineto{\pgfqpoint{4.731845in}{0.774158in}}%
\pgfpathlineto{\pgfqpoint{4.732141in}{0.774157in}}%
\pgfpathlineto{\pgfqpoint{4.732437in}{0.774156in}}%
\pgfpathlineto{\pgfqpoint{4.732733in}{0.774155in}}%
\pgfpathlineto{\pgfqpoint{4.733029in}{0.774154in}}%
\pgfpathlineto{\pgfqpoint{4.733325in}{0.774153in}}%
\pgfpathlineto{\pgfqpoint{4.733621in}{0.774152in}}%
\pgfpathlineto{\pgfqpoint{4.733917in}{0.774151in}}%
\pgfpathlineto{\pgfqpoint{4.734213in}{0.774150in}}%
\pgfpathlineto{\pgfqpoint{4.734509in}{0.774149in}}%
\pgfpathlineto{\pgfqpoint{4.734805in}{0.774148in}}%
\pgfpathlineto{\pgfqpoint{4.735101in}{0.774147in}}%
\pgfpathlineto{\pgfqpoint{4.735397in}{0.774146in}}%
\pgfpathlineto{\pgfqpoint{4.735693in}{0.774144in}}%
\pgfpathlineto{\pgfqpoint{4.735989in}{0.774143in}}%
\pgfpathlineto{\pgfqpoint{4.736285in}{0.774142in}}%
\pgfpathlineto{\pgfqpoint{4.736581in}{0.774141in}}%
\pgfpathlineto{\pgfqpoint{4.736877in}{0.774140in}}%
\pgfpathlineto{\pgfqpoint{4.737173in}{0.774139in}}%
\pgfpathlineto{\pgfqpoint{4.737469in}{0.774138in}}%
\pgfpathlineto{\pgfqpoint{4.737765in}{0.774137in}}%
\pgfpathlineto{\pgfqpoint{4.738061in}{0.774136in}}%
\pgfpathlineto{\pgfqpoint{4.738357in}{0.774135in}}%
\pgfpathlineto{\pgfqpoint{4.738653in}{0.774134in}}%
\pgfpathlineto{\pgfqpoint{4.738949in}{0.774133in}}%
\pgfpathlineto{\pgfqpoint{4.739245in}{0.774132in}}%
\pgfpathlineto{\pgfqpoint{4.739541in}{0.774131in}}%
\pgfpathlineto{\pgfqpoint{4.739837in}{0.774130in}}%
\pgfpathlineto{\pgfqpoint{4.740133in}{0.774128in}}%
\pgfpathlineto{\pgfqpoint{4.740429in}{0.774127in}}%
\pgfpathlineto{\pgfqpoint{4.740725in}{0.774126in}}%
\pgfpathlineto{\pgfqpoint{4.741021in}{0.774125in}}%
\pgfpathlineto{\pgfqpoint{4.741317in}{0.774124in}}%
\pgfpathlineto{\pgfqpoint{4.741613in}{0.774123in}}%
\pgfpathlineto{\pgfqpoint{4.741909in}{0.774122in}}%
\pgfpathlineto{\pgfqpoint{4.742205in}{0.774121in}}%
\pgfpathlineto{\pgfqpoint{4.742501in}{0.774120in}}%
\pgfpathlineto{\pgfqpoint{4.742797in}{0.774119in}}%
\pgfpathlineto{\pgfqpoint{4.743093in}{0.774118in}}%
\pgfpathlineto{\pgfqpoint{4.743389in}{0.774117in}}%
\pgfpathlineto{\pgfqpoint{4.743685in}{0.774116in}}%
\pgfpathlineto{\pgfqpoint{4.743981in}{0.774115in}}%
\pgfpathlineto{\pgfqpoint{4.744277in}{0.774145in}}%
\pgfpathlineto{\pgfqpoint{4.744573in}{0.774171in}}%
\pgfpathlineto{\pgfqpoint{4.744869in}{0.774157in}}%
\pgfpathlineto{\pgfqpoint{4.745165in}{0.774188in}}%
\pgfpathlineto{\pgfqpoint{4.745461in}{0.774163in}}%
\pgfpathlineto{\pgfqpoint{4.745757in}{0.763649in}}%
\pgfpathlineto{\pgfqpoint{4.746053in}{0.749681in}}%
\pgfpathlineto{\pgfqpoint{4.746349in}{0.757252in}}%
\pgfpathlineto{\pgfqpoint{4.746645in}{0.749703in}}%
\pgfpathlineto{\pgfqpoint{4.746941in}{0.749749in}}%
\pgfpathlineto{\pgfqpoint{4.747237in}{0.750070in}}%
\pgfpathlineto{\pgfqpoint{4.747533in}{0.751410in}}%
\pgfpathlineto{\pgfqpoint{4.747829in}{0.752853in}}%
\pgfpathlineto{\pgfqpoint{4.748125in}{0.754297in}}%
\pgfpathlineto{\pgfqpoint{4.748421in}{0.755740in}}%
\pgfpathlineto{\pgfqpoint{4.748717in}{0.757183in}}%
\pgfpathlineto{\pgfqpoint{4.749013in}{0.758627in}}%
\pgfpathlineto{\pgfqpoint{4.749309in}{0.760070in}}%
\pgfpathlineto{\pgfqpoint{4.749605in}{0.761513in}}%
\pgfpathlineto{\pgfqpoint{4.749901in}{0.762957in}}%
\pgfpathlineto{\pgfqpoint{4.750197in}{0.764400in}}%
\pgfpathlineto{\pgfqpoint{4.750493in}{0.765843in}}%
\pgfpathlineto{\pgfqpoint{4.750789in}{0.767287in}}%
\pgfpathlineto{\pgfqpoint{4.751085in}{0.768730in}}%
\pgfpathlineto{\pgfqpoint{4.751381in}{0.770174in}}%
\pgfpathlineto{\pgfqpoint{4.751677in}{0.771617in}}%
\pgfpathlineto{\pgfqpoint{4.751973in}{0.773060in}}%
\pgfpathlineto{\pgfqpoint{4.752269in}{0.760542in}}%
\pgfpathlineto{\pgfqpoint{4.752565in}{0.759714in}}%
\pgfpathlineto{\pgfqpoint{4.752861in}{0.749718in}}%
\pgfpathlineto{\pgfqpoint{4.753157in}{0.752040in}}%
\pgfpathlineto{\pgfqpoint{4.753453in}{0.767351in}}%
\pgfpathlineto{\pgfqpoint{4.753749in}{0.755820in}}%
\pgfpathlineto{\pgfqpoint{4.754045in}{0.749769in}}%
\pgfpathlineto{\pgfqpoint{4.754341in}{0.749765in}}%
\pgfpathlineto{\pgfqpoint{4.754637in}{0.749764in}}%
\pgfpathlineto{\pgfqpoint{4.754933in}{0.749763in}}%
\pgfpathlineto{\pgfqpoint{4.755229in}{0.749763in}}%
\pgfpathlineto{\pgfqpoint{4.755525in}{0.749762in}}%
\pgfpathlineto{\pgfqpoint{4.755821in}{0.749761in}}%
\pgfpathlineto{\pgfqpoint{4.756117in}{0.749760in}}%
\pgfpathlineto{\pgfqpoint{4.756413in}{0.749759in}}%
\pgfpathlineto{\pgfqpoint{4.756709in}{0.749758in}}%
\pgfpathlineto{\pgfqpoint{4.757005in}{0.749757in}}%
\pgfpathlineto{\pgfqpoint{4.757301in}{0.749756in}}%
\pgfpathlineto{\pgfqpoint{4.757597in}{0.749756in}}%
\pgfpathlineto{\pgfqpoint{4.757893in}{0.749755in}}%
\pgfpathlineto{\pgfqpoint{4.758189in}{0.749754in}}%
\pgfpathlineto{\pgfqpoint{4.758485in}{0.749753in}}%
\pgfpathlineto{\pgfqpoint{4.758781in}{0.749752in}}%
\pgfpathlineto{\pgfqpoint{4.759077in}{0.761154in}}%
\pgfpathlineto{\pgfqpoint{4.759373in}{0.758306in}}%
\pgfpathlineto{\pgfqpoint{4.759669in}{0.749753in}}%
\pgfpathlineto{\pgfqpoint{4.759965in}{0.749747in}}%
\pgfpathlineto{\pgfqpoint{4.760261in}{0.749745in}}%
\pgfpathlineto{\pgfqpoint{4.760558in}{0.749743in}}%
\pgfpathlineto{\pgfqpoint{4.760854in}{0.749715in}}%
\pgfpathlineto{\pgfqpoint{4.761150in}{0.749628in}}%
\pgfpathlineto{\pgfqpoint{4.761446in}{0.749303in}}%
\pgfpathlineto{\pgfqpoint{4.761742in}{0.749227in}}%
\pgfpathlineto{\pgfqpoint{4.762038in}{0.749256in}}%
\pgfpathlineto{\pgfqpoint{4.762334in}{0.749285in}}%
\pgfpathlineto{\pgfqpoint{4.762630in}{0.749313in}}%
\pgfpathlineto{\pgfqpoint{4.762926in}{0.749342in}}%
\pgfpathlineto{\pgfqpoint{4.763222in}{0.749371in}}%
\pgfpathlineto{\pgfqpoint{4.763518in}{0.749400in}}%
\pgfpathlineto{\pgfqpoint{4.763814in}{0.749429in}}%
\pgfpathlineto{\pgfqpoint{4.764110in}{0.749457in}}%
\pgfpathlineto{\pgfqpoint{4.764406in}{0.749486in}}%
\pgfpathlineto{\pgfqpoint{4.764702in}{0.749515in}}%
\pgfpathlineto{\pgfqpoint{4.764998in}{0.749544in}}%
\pgfpathlineto{\pgfqpoint{4.765294in}{0.749573in}}%
\pgfpathlineto{\pgfqpoint{4.765590in}{0.749601in}}%
\pgfpathlineto{\pgfqpoint{4.765886in}{0.749630in}}%
\pgfpathlineto{\pgfqpoint{4.766182in}{0.749659in}}%
\pgfpathlineto{\pgfqpoint{4.766478in}{0.749688in}}%
\pgfpathlineto{\pgfqpoint{4.766774in}{0.755604in}}%
\pgfpathlineto{\pgfqpoint{4.767070in}{0.773984in}}%
\pgfpathlineto{\pgfqpoint{4.767366in}{0.778035in}}%
\pgfpathlineto{\pgfqpoint{4.767662in}{0.781083in}}%
\pgfpathlineto{\pgfqpoint{4.767958in}{0.781097in}}%
\pgfpathlineto{\pgfqpoint{4.768254in}{0.781109in}}%
\pgfpathlineto{\pgfqpoint{4.768550in}{0.781110in}}%
\pgfpathlineto{\pgfqpoint{4.768846in}{0.781109in}}%
\pgfpathlineto{\pgfqpoint{4.769142in}{0.781108in}}%
\pgfpathlineto{\pgfqpoint{4.769438in}{0.781107in}}%
\pgfpathlineto{\pgfqpoint{4.769734in}{0.781106in}}%
\pgfpathlineto{\pgfqpoint{4.770030in}{0.781106in}}%
\pgfpathlineto{\pgfqpoint{4.770326in}{0.781105in}}%
\pgfpathlineto{\pgfqpoint{4.770622in}{0.781104in}}%
\pgfpathlineto{\pgfqpoint{4.770918in}{0.781103in}}%
\pgfpathlineto{\pgfqpoint{4.771214in}{0.781102in}}%
\pgfpathlineto{\pgfqpoint{4.771510in}{0.781102in}}%
\pgfpathlineto{\pgfqpoint{4.771806in}{0.781101in}}%
\pgfpathlineto{\pgfqpoint{4.772102in}{0.781100in}}%
\pgfpathlineto{\pgfqpoint{4.772398in}{0.781099in}}%
\pgfpathlineto{\pgfqpoint{4.772694in}{0.781098in}}%
\pgfpathlineto{\pgfqpoint{4.772990in}{0.781098in}}%
\pgfpathlineto{\pgfqpoint{4.773286in}{0.781097in}}%
\pgfpathlineto{\pgfqpoint{4.773582in}{0.781096in}}%
\pgfpathlineto{\pgfqpoint{4.773878in}{0.781095in}}%
\pgfpathlineto{\pgfqpoint{4.774174in}{0.781094in}}%
\pgfpathlineto{\pgfqpoint{4.774470in}{0.781131in}}%
\pgfpathlineto{\pgfqpoint{4.774766in}{0.781175in}}%
\pgfpathlineto{\pgfqpoint{4.775062in}{0.781204in}}%
\pgfpathlineto{\pgfqpoint{4.775358in}{0.781233in}}%
\pgfpathlineto{\pgfqpoint{4.775654in}{0.781262in}}%
\pgfpathlineto{\pgfqpoint{4.775950in}{0.781220in}}%
\pgfpathlineto{\pgfqpoint{4.776246in}{0.781104in}}%
\pgfpathlineto{\pgfqpoint{4.776542in}{0.780987in}}%
\pgfpathlineto{\pgfqpoint{4.776838in}{0.780871in}}%
\pgfpathlineto{\pgfqpoint{4.777134in}{0.780755in}}%
\pgfpathlineto{\pgfqpoint{4.777430in}{0.780639in}}%
\pgfpathlineto{\pgfqpoint{4.777726in}{0.780522in}}%
\pgfpathlineto{\pgfqpoint{4.778022in}{0.780406in}}%
\pgfpathlineto{\pgfqpoint{4.778318in}{0.780290in}}%
\pgfpathlineto{\pgfqpoint{4.778614in}{0.780174in}}%
\pgfpathlineto{\pgfqpoint{4.778910in}{0.780058in}}%
\pgfpathlineto{\pgfqpoint{4.779206in}{0.779941in}}%
\pgfpathlineto{\pgfqpoint{4.779502in}{0.779825in}}%
\pgfpathlineto{\pgfqpoint{4.779798in}{0.779709in}}%
\pgfpathlineto{\pgfqpoint{4.780094in}{0.779593in}}%
\pgfpathlineto{\pgfqpoint{4.780390in}{0.779476in}}%
\pgfpathlineto{\pgfqpoint{4.780686in}{0.779360in}}%
\pgfpathlineto{\pgfqpoint{4.780982in}{0.779244in}}%
\pgfpathlineto{\pgfqpoint{4.781278in}{0.779128in}}%
\pgfpathlineto{\pgfqpoint{4.781574in}{0.779012in}}%
\pgfpathlineto{\pgfqpoint{4.781870in}{0.778895in}}%
\pgfpathlineto{\pgfqpoint{4.782166in}{0.778779in}}%
\pgfpathlineto{\pgfqpoint{4.782462in}{0.778663in}}%
\pgfpathlineto{\pgfqpoint{4.782758in}{0.778547in}}%
\pgfpathlineto{\pgfqpoint{4.783054in}{0.778430in}}%
\pgfpathlineto{\pgfqpoint{4.783350in}{0.778314in}}%
\pgfpathlineto{\pgfqpoint{4.783646in}{0.778198in}}%
\pgfpathlineto{\pgfqpoint{4.783942in}{0.778082in}}%
\pgfpathlineto{\pgfqpoint{4.784238in}{0.777965in}}%
\pgfpathlineto{\pgfqpoint{4.784534in}{0.777849in}}%
\pgfpathlineto{\pgfqpoint{4.784830in}{0.777733in}}%
\pgfpathlineto{\pgfqpoint{4.785126in}{0.777617in}}%
\pgfpathlineto{\pgfqpoint{4.785422in}{0.777501in}}%
\pgfpathlineto{\pgfqpoint{4.785718in}{0.777384in}}%
\pgfpathlineto{\pgfqpoint{4.786014in}{0.777268in}}%
\pgfpathlineto{\pgfqpoint{4.786310in}{0.777152in}}%
\pgfpathlineto{\pgfqpoint{4.786606in}{0.777036in}}%
\pgfpathlineto{\pgfqpoint{4.786902in}{0.776919in}}%
\pgfpathlineto{\pgfqpoint{4.787198in}{0.776803in}}%
\pgfpathlineto{\pgfqpoint{4.787494in}{0.776687in}}%
\pgfpathlineto{\pgfqpoint{4.787790in}{0.776571in}}%
\pgfpathlineto{\pgfqpoint{4.788086in}{0.776455in}}%
\pgfpathlineto{\pgfqpoint{4.788382in}{0.776338in}}%
\pgfpathlineto{\pgfqpoint{4.788678in}{0.776222in}}%
\pgfpathlineto{\pgfqpoint{4.788974in}{0.776106in}}%
\pgfpathlineto{\pgfqpoint{4.789270in}{0.775990in}}%
\pgfpathlineto{\pgfqpoint{4.789566in}{0.775873in}}%
\pgfpathlineto{\pgfqpoint{4.789862in}{0.775757in}}%
\pgfpathlineto{\pgfqpoint{4.790158in}{0.775641in}}%
\pgfpathlineto{\pgfqpoint{4.790454in}{0.775525in}}%
\pgfpathlineto{\pgfqpoint{4.790750in}{0.775409in}}%
\pgfpathlineto{\pgfqpoint{4.791046in}{0.775292in}}%
\pgfpathlineto{\pgfqpoint{4.791342in}{0.775176in}}%
\pgfpathlineto{\pgfqpoint{4.791638in}{0.775060in}}%
\pgfpathlineto{\pgfqpoint{4.791934in}{0.774944in}}%
\pgfpathlineto{\pgfqpoint{4.792230in}{0.774827in}}%
\pgfpathlineto{\pgfqpoint{4.792526in}{0.774711in}}%
\pgfpathlineto{\pgfqpoint{4.792822in}{0.774595in}}%
\pgfpathlineto{\pgfqpoint{4.793118in}{0.774479in}}%
\pgfpathlineto{\pgfqpoint{4.793414in}{0.774363in}}%
\pgfpathlineto{\pgfqpoint{4.793710in}{0.774246in}}%
\pgfpathlineto{\pgfqpoint{4.794006in}{0.774130in}}%
\pgfpathlineto{\pgfqpoint{4.794302in}{0.774026in}}%
\pgfpathlineto{\pgfqpoint{4.794598in}{0.774854in}}%
\pgfpathlineto{\pgfqpoint{4.794894in}{0.776267in}}%
\pgfpathlineto{\pgfqpoint{4.795190in}{0.777634in}}%
\pgfpathlineto{\pgfqpoint{4.795486in}{0.778950in}}%
\pgfpathlineto{\pgfqpoint{4.795782in}{0.780264in}}%
\pgfpathlineto{\pgfqpoint{4.796078in}{0.780795in}}%
\pgfpathlineto{\pgfqpoint{4.796374in}{0.780719in}}%
\pgfpathlineto{\pgfqpoint{4.796670in}{0.780706in}}%
\pgfpathlineto{\pgfqpoint{4.796966in}{0.780701in}}%
\pgfpathlineto{\pgfqpoint{4.797262in}{0.780649in}}%
\pgfpathlineto{\pgfqpoint{4.797558in}{0.780639in}}%
\pgfpathlineto{\pgfqpoint{4.797854in}{0.780631in}}%
\pgfpathlineto{\pgfqpoint{4.798150in}{0.780619in}}%
\pgfpathlineto{\pgfqpoint{4.798446in}{0.780607in}}%
\pgfpathlineto{\pgfqpoint{4.798742in}{0.780596in}}%
\pgfpathlineto{\pgfqpoint{4.799038in}{0.780584in}}%
\pgfpathlineto{\pgfqpoint{4.799334in}{0.780572in}}%
\pgfpathlineto{\pgfqpoint{4.799630in}{0.780561in}}%
\pgfpathlineto{\pgfqpoint{4.799926in}{0.780549in}}%
\pgfpathlineto{\pgfqpoint{4.800222in}{0.780537in}}%
\pgfpathlineto{\pgfqpoint{4.800518in}{0.780526in}}%
\pgfpathlineto{\pgfqpoint{4.800814in}{0.780514in}}%
\pgfpathlineto{\pgfqpoint{4.801110in}{0.780502in}}%
\pgfpathlineto{\pgfqpoint{4.801406in}{0.780491in}}%
\pgfpathlineto{\pgfqpoint{4.801702in}{0.781832in}}%
\pgfpathlineto{\pgfqpoint{4.801998in}{0.782207in}}%
\pgfpathlineto{\pgfqpoint{4.802294in}{0.780475in}}%
\pgfpathlineto{\pgfqpoint{4.802590in}{0.780790in}}%
\pgfpathlineto{\pgfqpoint{4.802886in}{0.783981in}}%
\pgfpathlineto{\pgfqpoint{4.803182in}{0.784542in}}%
\pgfpathlineto{\pgfqpoint{4.803478in}{0.784552in}}%
\pgfpathlineto{\pgfqpoint{4.803774in}{0.784567in}}%
\pgfpathlineto{\pgfqpoint{4.804070in}{0.784615in}}%
\pgfpathlineto{\pgfqpoint{4.804366in}{0.784635in}}%
\pgfpathlineto{\pgfqpoint{4.804662in}{0.784631in}}%
\pgfpathlineto{\pgfqpoint{4.804958in}{0.784626in}}%
\pgfpathlineto{\pgfqpoint{4.805254in}{0.784107in}}%
\pgfpathlineto{\pgfqpoint{4.805550in}{0.750541in}}%
\pgfpathlineto{\pgfqpoint{4.805846in}{0.753825in}}%
\pgfpathlineto{\pgfqpoint{4.806142in}{0.757269in}}%
\pgfpathlineto{\pgfqpoint{4.806438in}{0.760713in}}%
\pgfpathlineto{\pgfqpoint{4.806734in}{0.764158in}}%
\pgfpathlineto{\pgfqpoint{4.807030in}{0.767602in}}%
\pgfpathlineto{\pgfqpoint{4.807326in}{0.771046in}}%
\pgfpathlineto{\pgfqpoint{4.807622in}{0.774490in}}%
\pgfpathlineto{\pgfqpoint{4.807918in}{0.777934in}}%
\pgfpathlineto{\pgfqpoint{4.808214in}{0.781378in}}%
\pgfpathlineto{\pgfqpoint{4.808510in}{0.784218in}}%
\pgfpathlineto{\pgfqpoint{4.808806in}{0.784764in}}%
\pgfpathlineto{\pgfqpoint{4.809102in}{0.784810in}}%
\pgfpathlineto{\pgfqpoint{4.809398in}{0.784609in}}%
\pgfpathlineto{\pgfqpoint{4.809694in}{0.784735in}}%
\pgfpathlineto{\pgfqpoint{4.809990in}{0.784875in}}%
\pgfpathlineto{\pgfqpoint{4.810286in}{0.785010in}}%
\pgfpathlineto{\pgfqpoint{4.810582in}{0.784908in}}%
\pgfpathlineto{\pgfqpoint{4.810878in}{0.784866in}}%
\pgfpathlineto{\pgfqpoint{4.811174in}{0.784854in}}%
\pgfpathlineto{\pgfqpoint{4.811470in}{0.784848in}}%
\pgfpathlineto{\pgfqpoint{4.811766in}{0.784841in}}%
\pgfpathlineto{\pgfqpoint{4.812062in}{0.784835in}}%
\pgfpathlineto{\pgfqpoint{4.812358in}{0.784828in}}%
\pgfpathlineto{\pgfqpoint{4.812654in}{0.784822in}}%
\pgfpathlineto{\pgfqpoint{4.812950in}{0.784816in}}%
\pgfpathlineto{\pgfqpoint{4.813246in}{0.784809in}}%
\pgfpathlineto{\pgfqpoint{4.813542in}{0.784803in}}%
\pgfpathlineto{\pgfqpoint{4.813838in}{0.784796in}}%
\pgfpathlineto{\pgfqpoint{4.814134in}{0.784790in}}%
\pgfpathlineto{\pgfqpoint{4.814430in}{0.784783in}}%
\pgfpathlineto{\pgfqpoint{4.814726in}{0.784777in}}%
\pgfpathlineto{\pgfqpoint{4.815022in}{0.784770in}}%
\pgfpathlineto{\pgfqpoint{4.815318in}{0.784764in}}%
\pgfpathlineto{\pgfqpoint{4.815614in}{0.784757in}}%
\pgfpathlineto{\pgfqpoint{4.815910in}{0.784744in}}%
\pgfpathlineto{\pgfqpoint{4.816206in}{0.785377in}}%
\pgfpathlineto{\pgfqpoint{4.816502in}{0.786444in}}%
\pgfpathlineto{\pgfqpoint{4.816798in}{0.787510in}}%
\pgfpathlineto{\pgfqpoint{4.817094in}{0.788577in}}%
\pgfpathlineto{\pgfqpoint{4.817390in}{0.789644in}}%
\pgfpathlineto{\pgfqpoint{4.817686in}{0.790710in}}%
\pgfpathlineto{\pgfqpoint{4.817982in}{0.791777in}}%
\pgfpathlineto{\pgfqpoint{4.818278in}{0.792844in}}%
\pgfpathlineto{\pgfqpoint{4.818574in}{0.793910in}}%
\pgfpathlineto{\pgfqpoint{4.818870in}{0.794977in}}%
\pgfpathlineto{\pgfqpoint{4.819166in}{0.796044in}}%
\pgfpathlineto{\pgfqpoint{4.819462in}{0.797110in}}%
\pgfpathlineto{\pgfqpoint{4.819758in}{0.798177in}}%
\pgfpathlineto{\pgfqpoint{4.820054in}{0.799244in}}%
\pgfpathlineto{\pgfqpoint{4.820350in}{0.800310in}}%
\pgfpathlineto{\pgfqpoint{4.820646in}{0.801377in}}%
\pgfpathlineto{\pgfqpoint{4.820942in}{0.802444in}}%
\pgfpathlineto{\pgfqpoint{4.821238in}{0.803510in}}%
\pgfpathlineto{\pgfqpoint{4.821534in}{0.804577in}}%
\pgfpathlineto{\pgfqpoint{4.821830in}{0.805644in}}%
\pgfpathlineto{\pgfqpoint{4.822126in}{0.806710in}}%
\pgfpathlineto{\pgfqpoint{4.822422in}{0.807777in}}%
\pgfpathlineto{\pgfqpoint{4.822718in}{0.808029in}}%
\pgfpathlineto{\pgfqpoint{4.823014in}{0.797171in}}%
\pgfpathlineto{\pgfqpoint{4.823310in}{0.785454in}}%
\pgfpathlineto{\pgfqpoint{4.823606in}{0.784351in}}%
\pgfpathlineto{\pgfqpoint{4.823902in}{0.784349in}}%
\pgfpathlineto{\pgfqpoint{4.824198in}{0.784421in}}%
\pgfpathlineto{\pgfqpoint{4.824494in}{0.784706in}}%
\pgfpathlineto{\pgfqpoint{4.824790in}{0.784439in}}%
\pgfpathlineto{\pgfqpoint{4.825086in}{0.784345in}}%
\pgfpathlineto{\pgfqpoint{4.825382in}{0.784327in}}%
\pgfpathlineto{\pgfqpoint{4.825678in}{0.784008in}}%
\pgfpathlineto{\pgfqpoint{4.825974in}{0.783791in}}%
\pgfpathlineto{\pgfqpoint{4.826270in}{0.783776in}}%
\pgfpathlineto{\pgfqpoint{4.826566in}{0.783761in}}%
\pgfpathlineto{\pgfqpoint{4.826862in}{0.783746in}}%
\pgfpathlineto{\pgfqpoint{4.827158in}{0.783731in}}%
\pgfpathlineto{\pgfqpoint{4.827454in}{0.783716in}}%
\pgfpathlineto{\pgfqpoint{4.827751in}{0.783700in}}%
\pgfpathlineto{\pgfqpoint{4.828047in}{0.783685in}}%
\pgfpathlineto{\pgfqpoint{4.828343in}{0.783670in}}%
\pgfpathlineto{\pgfqpoint{4.828639in}{0.783655in}}%
\pgfpathlineto{\pgfqpoint{4.828935in}{0.783640in}}%
\pgfpathlineto{\pgfqpoint{4.829231in}{0.783625in}}%
\pgfpathlineto{\pgfqpoint{4.829527in}{0.783610in}}%
\pgfpathlineto{\pgfqpoint{4.829823in}{0.783595in}}%
\pgfpathlineto{\pgfqpoint{4.830119in}{0.783580in}}%
\pgfpathlineto{\pgfqpoint{4.830415in}{0.783565in}}%
\pgfpathlineto{\pgfqpoint{4.830711in}{0.783550in}}%
\pgfpathlineto{\pgfqpoint{4.831007in}{0.783535in}}%
\pgfpathlineto{\pgfqpoint{4.831303in}{0.783520in}}%
\pgfpathlineto{\pgfqpoint{4.831599in}{0.783505in}}%
\pgfpathlineto{\pgfqpoint{4.831895in}{0.783490in}}%
\pgfpathlineto{\pgfqpoint{4.832191in}{0.783474in}}%
\pgfpathlineto{\pgfqpoint{4.832487in}{0.783459in}}%
\pgfpathlineto{\pgfqpoint{4.832783in}{0.783444in}}%
\pgfpathlineto{\pgfqpoint{4.833079in}{0.783429in}}%
\pgfpathlineto{\pgfqpoint{4.833375in}{0.783414in}}%
\pgfpathlineto{\pgfqpoint{4.833671in}{0.783399in}}%
\pgfpathlineto{\pgfqpoint{4.833967in}{0.783384in}}%
\pgfpathlineto{\pgfqpoint{4.834263in}{0.783369in}}%
\pgfpathlineto{\pgfqpoint{4.834559in}{0.783354in}}%
\pgfpathlineto{\pgfqpoint{4.834855in}{0.783339in}}%
\pgfpathlineto{\pgfqpoint{4.835151in}{0.783324in}}%
\pgfpathlineto{\pgfqpoint{4.835447in}{0.783309in}}%
\pgfpathlineto{\pgfqpoint{4.835743in}{0.783294in}}%
\pgfpathlineto{\pgfqpoint{4.836039in}{0.783279in}}%
\pgfpathlineto{\pgfqpoint{4.836335in}{0.783264in}}%
\pgfpathlineto{\pgfqpoint{4.836631in}{0.783248in}}%
\pgfpathlineto{\pgfqpoint{4.836927in}{0.783233in}}%
\pgfpathlineto{\pgfqpoint{4.837223in}{0.783218in}}%
\pgfpathlineto{\pgfqpoint{4.837519in}{0.783203in}}%
\pgfpathlineto{\pgfqpoint{4.837815in}{0.783188in}}%
\pgfpathlineto{\pgfqpoint{4.838111in}{0.783173in}}%
\pgfpathlineto{\pgfqpoint{4.838407in}{0.783158in}}%
\pgfpathlineto{\pgfqpoint{4.838703in}{0.783143in}}%
\pgfpathlineto{\pgfqpoint{4.838999in}{0.783128in}}%
\pgfpathlineto{\pgfqpoint{4.839295in}{0.783113in}}%
\pgfpathlineto{\pgfqpoint{4.839591in}{0.783098in}}%
\pgfpathlineto{\pgfqpoint{4.839887in}{0.783083in}}%
\pgfpathlineto{\pgfqpoint{4.840183in}{0.783068in}}%
\pgfpathlineto{\pgfqpoint{4.840479in}{0.783053in}}%
\pgfpathlineto{\pgfqpoint{4.840775in}{0.783038in}}%
\pgfpathlineto{\pgfqpoint{4.841071in}{0.783022in}}%
\pgfpathlineto{\pgfqpoint{4.841367in}{0.783007in}}%
\pgfpathlineto{\pgfqpoint{4.841663in}{0.782992in}}%
\pgfpathlineto{\pgfqpoint{4.841959in}{0.782977in}}%
\pgfpathlineto{\pgfqpoint{4.842255in}{0.782962in}}%
\pgfpathlineto{\pgfqpoint{4.842551in}{0.782947in}}%
\pgfpathlineto{\pgfqpoint{4.842847in}{0.782932in}}%
\pgfpathlineto{\pgfqpoint{4.843143in}{0.782917in}}%
\pgfpathlineto{\pgfqpoint{4.843439in}{0.782902in}}%
\pgfpathlineto{\pgfqpoint{4.843735in}{0.782887in}}%
\pgfpathlineto{\pgfqpoint{4.844031in}{0.782872in}}%
\pgfpathlineto{\pgfqpoint{4.844327in}{0.782857in}}%
\pgfpathlineto{\pgfqpoint{4.844623in}{0.782842in}}%
\pgfpathlineto{\pgfqpoint{4.844919in}{0.782827in}}%
\pgfpathlineto{\pgfqpoint{4.845215in}{0.782832in}}%
\pgfpathlineto{\pgfqpoint{4.845511in}{0.782893in}}%
\pgfpathlineto{\pgfqpoint{4.845807in}{0.782981in}}%
\pgfpathlineto{\pgfqpoint{4.846103in}{0.783258in}}%
\pgfpathlineto{\pgfqpoint{4.846399in}{0.783545in}}%
\pgfpathlineto{\pgfqpoint{4.846695in}{0.783589in}}%
\pgfpathlineto{\pgfqpoint{4.846991in}{0.783632in}}%
\pgfpathlineto{\pgfqpoint{4.847287in}{0.783675in}}%
\pgfpathlineto{\pgfqpoint{4.847583in}{0.783718in}}%
\pgfpathlineto{\pgfqpoint{4.847879in}{0.783762in}}%
\pgfpathlineto{\pgfqpoint{4.848175in}{0.783805in}}%
\pgfpathlineto{\pgfqpoint{4.848471in}{0.783848in}}%
\pgfpathlineto{\pgfqpoint{4.848767in}{0.783891in}}%
\pgfpathlineto{\pgfqpoint{4.849063in}{0.783934in}}%
\pgfpathlineto{\pgfqpoint{4.849359in}{0.783978in}}%
\pgfpathlineto{\pgfqpoint{4.849655in}{0.784021in}}%
\pgfpathlineto{\pgfqpoint{4.849951in}{0.784064in}}%
\pgfpathlineto{\pgfqpoint{4.850247in}{0.784107in}}%
\pgfpathlineto{\pgfqpoint{4.850543in}{0.784151in}}%
\pgfpathlineto{\pgfqpoint{4.850839in}{0.784194in}}%
\pgfpathlineto{\pgfqpoint{4.851135in}{0.784237in}}%
\pgfpathlineto{\pgfqpoint{4.851431in}{0.784280in}}%
\pgfpathlineto{\pgfqpoint{4.851727in}{0.784323in}}%
\pgfpathlineto{\pgfqpoint{4.852023in}{0.784255in}}%
\pgfpathlineto{\pgfqpoint{4.852319in}{0.784116in}}%
\pgfpathlineto{\pgfqpoint{4.852615in}{0.783978in}}%
\pgfpathlineto{\pgfqpoint{4.852911in}{0.784082in}}%
\pgfpathlineto{\pgfqpoint{4.853207in}{0.784374in}}%
\pgfpathlineto{\pgfqpoint{4.853503in}{0.784367in}}%
\pgfpathlineto{\pgfqpoint{4.853799in}{0.784360in}}%
\pgfpathlineto{\pgfqpoint{4.854095in}{0.784352in}}%
\pgfpathlineto{\pgfqpoint{4.854391in}{0.784344in}}%
\pgfpathlineto{\pgfqpoint{4.854687in}{0.784336in}}%
\pgfpathlineto{\pgfqpoint{4.854983in}{0.784328in}}%
\pgfpathlineto{\pgfqpoint{4.855279in}{0.784321in}}%
\pgfpathlineto{\pgfqpoint{4.855575in}{0.784313in}}%
\pgfpathlineto{\pgfqpoint{4.855871in}{0.784305in}}%
\pgfpathlineto{\pgfqpoint{4.856167in}{0.784297in}}%
\pgfpathlineto{\pgfqpoint{4.856463in}{0.784289in}}%
\pgfpathlineto{\pgfqpoint{4.856759in}{0.784281in}}%
\pgfpathlineto{\pgfqpoint{4.857055in}{0.784273in}}%
\pgfpathlineto{\pgfqpoint{4.857351in}{0.784265in}}%
\pgfpathlineto{\pgfqpoint{4.857647in}{0.784257in}}%
\pgfpathlineto{\pgfqpoint{4.857943in}{0.784250in}}%
\pgfpathlineto{\pgfqpoint{4.858239in}{0.784242in}}%
\pgfpathlineto{\pgfqpoint{4.858535in}{0.784234in}}%
\pgfpathlineto{\pgfqpoint{4.858831in}{0.784712in}}%
\pgfpathlineto{\pgfqpoint{4.859127in}{0.804598in}}%
\pgfpathlineto{\pgfqpoint{4.859423in}{0.799970in}}%
\pgfpathlineto{\pgfqpoint{4.859719in}{0.793249in}}%
\pgfpathlineto{\pgfqpoint{4.860015in}{0.786729in}}%
\pgfpathlineto{\pgfqpoint{4.860311in}{0.790394in}}%
\pgfpathlineto{\pgfqpoint{4.860607in}{0.798728in}}%
\pgfpathlineto{\pgfqpoint{4.860903in}{0.799879in}}%
\pgfpathlineto{\pgfqpoint{4.861199in}{0.789609in}}%
\pgfpathlineto{\pgfqpoint{4.861495in}{0.794166in}}%
\pgfpathlineto{\pgfqpoint{4.861791in}{0.800282in}}%
\pgfpathlineto{\pgfqpoint{4.862087in}{0.800275in}}%
\pgfpathlineto{\pgfqpoint{4.862383in}{0.800268in}}%
\pgfpathlineto{\pgfqpoint{4.862679in}{0.800261in}}%
\pgfpathlineto{\pgfqpoint{4.862975in}{0.800254in}}%
\pgfpathlineto{\pgfqpoint{4.863271in}{0.800248in}}%
\pgfpathlineto{\pgfqpoint{4.863567in}{0.800241in}}%
\pgfpathlineto{\pgfqpoint{4.863863in}{0.800234in}}%
\pgfpathlineto{\pgfqpoint{4.864159in}{0.800227in}}%
\pgfpathlineto{\pgfqpoint{4.864455in}{0.800220in}}%
\pgfpathlineto{\pgfqpoint{4.864751in}{0.800213in}}%
\pgfpathlineto{\pgfqpoint{4.865047in}{0.800207in}}%
\pgfpathlineto{\pgfqpoint{4.865343in}{0.800205in}}%
\pgfpathlineto{\pgfqpoint{4.865639in}{0.800247in}}%
\pgfpathlineto{\pgfqpoint{4.865935in}{0.800215in}}%
\pgfpathlineto{\pgfqpoint{4.866231in}{0.800199in}}%
\pgfpathlineto{\pgfqpoint{4.866527in}{0.800260in}}%
\pgfpathlineto{\pgfqpoint{4.866823in}{0.800268in}}%
\pgfpathlineto{\pgfqpoint{4.867119in}{0.800252in}}%
\pgfpathlineto{\pgfqpoint{4.867415in}{0.800253in}}%
\pgfpathlineto{\pgfqpoint{4.867711in}{0.800248in}}%
\pgfpathlineto{\pgfqpoint{4.868007in}{0.800246in}}%
\pgfpathlineto{\pgfqpoint{4.868303in}{0.800177in}}%
\pgfpathlineto{\pgfqpoint{4.868599in}{0.800075in}}%
\pgfpathlineto{\pgfqpoint{4.868895in}{0.800006in}}%
\pgfpathlineto{\pgfqpoint{4.869191in}{0.799938in}}%
\pgfpathlineto{\pgfqpoint{4.869487in}{0.799869in}}%
\pgfpathlineto{\pgfqpoint{4.869783in}{0.799800in}}%
\pgfpathlineto{\pgfqpoint{4.870079in}{0.799731in}}%
\pgfpathlineto{\pgfqpoint{4.870375in}{0.799662in}}%
\pgfpathlineto{\pgfqpoint{4.870671in}{0.799593in}}%
\pgfpathlineto{\pgfqpoint{4.870967in}{0.799524in}}%
\pgfpathlineto{\pgfqpoint{4.871263in}{0.799456in}}%
\pgfpathlineto{\pgfqpoint{4.871559in}{0.799387in}}%
\pgfpathlineto{\pgfqpoint{4.871855in}{0.799318in}}%
\pgfpathlineto{\pgfqpoint{4.872151in}{0.799249in}}%
\pgfpathlineto{\pgfqpoint{4.872447in}{0.799180in}}%
\pgfpathlineto{\pgfqpoint{4.872743in}{0.799111in}}%
\pgfpathlineto{\pgfqpoint{4.873039in}{0.799042in}}%
\pgfpathlineto{\pgfqpoint{4.873335in}{0.798974in}}%
\pgfpathlineto{\pgfqpoint{4.873631in}{0.798905in}}%
\pgfpathlineto{\pgfqpoint{4.873927in}{0.798494in}}%
\pgfpathlineto{\pgfqpoint{4.874223in}{0.797884in}}%
\pgfpathlineto{\pgfqpoint{4.874519in}{0.797628in}}%
\pgfpathlineto{\pgfqpoint{4.874815in}{0.797441in}}%
\pgfpathlineto{\pgfqpoint{4.875111in}{0.797258in}}%
\pgfpathlineto{\pgfqpoint{4.875407in}{0.797199in}}%
\pgfpathlineto{\pgfqpoint{4.875703in}{0.797203in}}%
\pgfpathlineto{\pgfqpoint{4.875999in}{0.797207in}}%
\pgfpathlineto{\pgfqpoint{4.876295in}{0.797211in}}%
\pgfpathlineto{\pgfqpoint{4.876591in}{0.797215in}}%
\pgfpathlineto{\pgfqpoint{4.876887in}{0.797219in}}%
\pgfpathlineto{\pgfqpoint{4.877183in}{0.797223in}}%
\pgfpathlineto{\pgfqpoint{4.877479in}{0.797226in}}%
\pgfpathlineto{\pgfqpoint{4.877775in}{0.797230in}}%
\pgfpathlineto{\pgfqpoint{4.878071in}{0.797234in}}%
\pgfpathlineto{\pgfqpoint{4.878367in}{0.797238in}}%
\pgfpathlineto{\pgfqpoint{4.878663in}{0.797242in}}%
\pgfpathlineto{\pgfqpoint{4.878959in}{0.797246in}}%
\pgfpathlineto{\pgfqpoint{4.879255in}{0.797250in}}%
\pgfpathlineto{\pgfqpoint{4.879551in}{0.797253in}}%
\pgfpathlineto{\pgfqpoint{4.879847in}{0.797257in}}%
\pgfpathlineto{\pgfqpoint{4.880143in}{0.797261in}}%
\pgfpathlineto{\pgfqpoint{4.880439in}{0.797265in}}%
\pgfpathlineto{\pgfqpoint{4.880735in}{0.797269in}}%
\pgfpathlineto{\pgfqpoint{4.881031in}{0.797273in}}%
\pgfpathlineto{\pgfqpoint{4.881327in}{0.797277in}}%
\pgfpathlineto{\pgfqpoint{4.881623in}{0.797280in}}%
\pgfpathlineto{\pgfqpoint{4.881919in}{0.797284in}}%
\pgfpathlineto{\pgfqpoint{4.882215in}{0.797288in}}%
\pgfpathlineto{\pgfqpoint{4.882511in}{0.797292in}}%
\pgfpathlineto{\pgfqpoint{4.882807in}{0.797296in}}%
\pgfpathlineto{\pgfqpoint{4.883103in}{0.797300in}}%
\pgfpathlineto{\pgfqpoint{4.883399in}{0.797304in}}%
\pgfpathlineto{\pgfqpoint{4.883695in}{0.797307in}}%
\pgfpathlineto{\pgfqpoint{4.883991in}{0.797311in}}%
\pgfpathlineto{\pgfqpoint{4.884287in}{0.797315in}}%
\pgfpathlineto{\pgfqpoint{4.884583in}{0.797319in}}%
\pgfpathlineto{\pgfqpoint{4.884879in}{0.797323in}}%
\pgfpathlineto{\pgfqpoint{4.885175in}{0.797327in}}%
\pgfpathlineto{\pgfqpoint{4.885471in}{0.797331in}}%
\pgfpathlineto{\pgfqpoint{4.885767in}{0.797334in}}%
\pgfpathlineto{\pgfqpoint{4.886063in}{0.797338in}}%
\pgfpathlineto{\pgfqpoint{4.886359in}{0.797342in}}%
\pgfpathlineto{\pgfqpoint{4.886655in}{0.797346in}}%
\pgfpathlineto{\pgfqpoint{4.886951in}{0.797350in}}%
\pgfpathlineto{\pgfqpoint{4.887247in}{0.797354in}}%
\pgfpathlineto{\pgfqpoint{4.887543in}{0.797358in}}%
\pgfpathlineto{\pgfqpoint{4.887839in}{0.797361in}}%
\pgfpathlineto{\pgfqpoint{4.888135in}{0.797365in}}%
\pgfpathlineto{\pgfqpoint{4.888431in}{0.797369in}}%
\pgfpathlineto{\pgfqpoint{4.888727in}{0.797373in}}%
\pgfpathlineto{\pgfqpoint{4.889023in}{0.797377in}}%
\pgfpathlineto{\pgfqpoint{4.889319in}{0.797381in}}%
\pgfpathlineto{\pgfqpoint{4.889615in}{0.797385in}}%
\pgfpathlineto{\pgfqpoint{4.889911in}{0.797388in}}%
\pgfpathlineto{\pgfqpoint{4.890207in}{0.797392in}}%
\pgfpathlineto{\pgfqpoint{4.890503in}{0.797396in}}%
\pgfpathlineto{\pgfqpoint{4.890799in}{0.797400in}}%
\pgfpathlineto{\pgfqpoint{4.891095in}{0.797404in}}%
\pgfpathlineto{\pgfqpoint{4.891391in}{0.797408in}}%
\pgfpathlineto{\pgfqpoint{4.891687in}{0.797412in}}%
\pgfpathlineto{\pgfqpoint{4.891983in}{0.797415in}}%
\pgfpathlineto{\pgfqpoint{4.892279in}{0.797419in}}%
\pgfpathlineto{\pgfqpoint{4.892575in}{0.797423in}}%
\pgfpathlineto{\pgfqpoint{4.892871in}{0.797427in}}%
\pgfpathlineto{\pgfqpoint{4.893167in}{0.797431in}}%
\pgfpathlineto{\pgfqpoint{4.893463in}{0.797435in}}%
\pgfpathlineto{\pgfqpoint{4.893759in}{0.797439in}}%
\pgfpathlineto{\pgfqpoint{4.894055in}{0.797442in}}%
\pgfpathlineto{\pgfqpoint{4.894351in}{0.797446in}}%
\pgfpathlineto{\pgfqpoint{4.894647in}{0.799931in}}%
\pgfpathlineto{\pgfqpoint{4.894943in}{0.806239in}}%
\pgfpathlineto{\pgfqpoint{4.895240in}{0.797595in}}%
\pgfpathlineto{\pgfqpoint{4.895536in}{0.797585in}}%
\pgfpathlineto{\pgfqpoint{4.895832in}{0.805849in}}%
\pgfpathlineto{\pgfqpoint{4.896128in}{0.804371in}}%
\pgfpathlineto{\pgfqpoint{4.896424in}{0.800095in}}%
\pgfpathlineto{\pgfqpoint{4.896720in}{0.816455in}}%
\pgfpathlineto{\pgfqpoint{4.897016in}{0.800765in}}%
\pgfpathlineto{\pgfqpoint{4.897312in}{0.797970in}}%
\pgfpathlineto{\pgfqpoint{4.897608in}{0.797913in}}%
\pgfpathlineto{\pgfqpoint{4.897904in}{0.797856in}}%
\pgfpathlineto{\pgfqpoint{4.898200in}{0.797798in}}%
\pgfpathlineto{\pgfqpoint{4.898496in}{0.797741in}}%
\pgfpathlineto{\pgfqpoint{4.898792in}{0.797684in}}%
\pgfpathlineto{\pgfqpoint{4.899088in}{0.797627in}}%
\pgfpathlineto{\pgfqpoint{4.899384in}{0.797569in}}%
\pgfpathlineto{\pgfqpoint{4.899680in}{0.797512in}}%
\pgfpathlineto{\pgfqpoint{4.899976in}{0.797455in}}%
\pgfpathlineto{\pgfqpoint{4.900272in}{0.797398in}}%
\pgfpathlineto{\pgfqpoint{4.900568in}{0.797340in}}%
\pgfpathlineto{\pgfqpoint{4.900864in}{0.797283in}}%
\pgfpathlineto{\pgfqpoint{4.901160in}{0.797305in}}%
\pgfpathlineto{\pgfqpoint{4.901456in}{0.797679in}}%
\pgfpathlineto{\pgfqpoint{4.901752in}{0.797388in}}%
\pgfpathlineto{\pgfqpoint{4.902048in}{0.796842in}}%
\pgfpathlineto{\pgfqpoint{4.902344in}{0.796608in}}%
\pgfpathlineto{\pgfqpoint{4.902640in}{0.796849in}}%
\pgfpathlineto{\pgfqpoint{4.902936in}{0.796992in}}%
\pgfpathlineto{\pgfqpoint{4.903232in}{0.797149in}}%
\pgfpathlineto{\pgfqpoint{4.903528in}{0.797085in}}%
\pgfpathlineto{\pgfqpoint{4.903824in}{0.796622in}}%
\pgfpathlineto{\pgfqpoint{4.904120in}{0.796372in}}%
\pgfpathlineto{\pgfqpoint{4.904416in}{0.796298in}}%
\pgfpathlineto{\pgfqpoint{4.904712in}{0.796223in}}%
\pgfpathlineto{\pgfqpoint{4.905008in}{0.796148in}}%
\pgfpathlineto{\pgfqpoint{4.905304in}{0.796073in}}%
\pgfpathlineto{\pgfqpoint{4.905600in}{0.795999in}}%
\pgfpathlineto{\pgfqpoint{4.905896in}{0.795924in}}%
\pgfpathlineto{\pgfqpoint{4.906192in}{0.795849in}}%
\pgfpathlineto{\pgfqpoint{4.906488in}{0.795774in}}%
\pgfpathlineto{\pgfqpoint{4.906784in}{0.795699in}}%
\pgfpathlineto{\pgfqpoint{4.907080in}{0.795625in}}%
\pgfpathlineto{\pgfqpoint{4.907376in}{0.795550in}}%
\pgfpathlineto{\pgfqpoint{4.907672in}{0.795475in}}%
\pgfpathlineto{\pgfqpoint{4.907968in}{0.795400in}}%
\pgfpathlineto{\pgfqpoint{4.908264in}{0.795508in}}%
\pgfpathlineto{\pgfqpoint{4.908560in}{0.797559in}}%
\pgfpathlineto{\pgfqpoint{4.908856in}{0.797045in}}%
\pgfpathlineto{\pgfqpoint{4.909152in}{0.796485in}}%
\pgfpathlineto{\pgfqpoint{4.909448in}{0.795291in}}%
\pgfpathlineto{\pgfqpoint{4.909744in}{0.795089in}}%
\pgfpathlineto{\pgfqpoint{4.910040in}{0.796141in}}%
\pgfpathlineto{\pgfqpoint{4.910336in}{0.796496in}}%
\pgfpathlineto{\pgfqpoint{4.910632in}{0.796448in}}%
\pgfpathlineto{\pgfqpoint{4.910928in}{0.796400in}}%
\pgfpathlineto{\pgfqpoint{4.911224in}{0.796352in}}%
\pgfpathlineto{\pgfqpoint{4.911520in}{0.796304in}}%
\pgfpathlineto{\pgfqpoint{4.911816in}{0.796256in}}%
\pgfpathlineto{\pgfqpoint{4.912112in}{0.796209in}}%
\pgfpathlineto{\pgfqpoint{4.912408in}{0.796161in}}%
\pgfpathlineto{\pgfqpoint{4.912704in}{0.796113in}}%
\pgfpathlineto{\pgfqpoint{4.913000in}{0.796065in}}%
\pgfpathlineto{\pgfqpoint{4.913296in}{0.796018in}}%
\pgfpathlineto{\pgfqpoint{4.913592in}{0.795970in}}%
\pgfpathlineto{\pgfqpoint{4.913888in}{0.795922in}}%
\pgfpathlineto{\pgfqpoint{4.914184in}{0.795875in}}%
\pgfpathlineto{\pgfqpoint{4.914480in}{0.795827in}}%
\pgfpathlineto{\pgfqpoint{4.914776in}{0.795779in}}%
\pgfpathlineto{\pgfqpoint{4.915072in}{0.795731in}}%
\pgfpathlineto{\pgfqpoint{4.915368in}{0.795683in}}%
\pgfpathlineto{\pgfqpoint{4.915664in}{0.795634in}}%
\pgfpathlineto{\pgfqpoint{4.915960in}{0.795590in}}%
\pgfpathlineto{\pgfqpoint{4.916256in}{0.795554in}}%
\pgfpathlineto{\pgfqpoint{4.916552in}{0.795521in}}%
\pgfpathlineto{\pgfqpoint{4.916848in}{0.795488in}}%
\pgfpathlineto{\pgfqpoint{4.917144in}{0.795455in}}%
\pgfpathlineto{\pgfqpoint{4.917440in}{0.795421in}}%
\pgfpathlineto{\pgfqpoint{4.917736in}{0.795382in}}%
\pgfpathlineto{\pgfqpoint{4.918032in}{0.795343in}}%
\pgfpathlineto{\pgfqpoint{4.918328in}{0.795305in}}%
\pgfpathlineto{\pgfqpoint{4.918624in}{0.795270in}}%
\pgfpathlineto{\pgfqpoint{4.918920in}{0.795257in}}%
\pgfpathlineto{\pgfqpoint{4.919216in}{0.795249in}}%
\pgfpathlineto{\pgfqpoint{4.919512in}{0.795240in}}%
\pgfpathlineto{\pgfqpoint{4.919808in}{0.795232in}}%
\pgfpathlineto{\pgfqpoint{4.920104in}{0.795223in}}%
\pgfpathlineto{\pgfqpoint{4.920400in}{0.795215in}}%
\pgfpathlineto{\pgfqpoint{4.920696in}{0.795206in}}%
\pgfpathlineto{\pgfqpoint{4.920992in}{0.795198in}}%
\pgfpathlineto{\pgfqpoint{4.921288in}{0.795190in}}%
\pgfpathlineto{\pgfqpoint{4.921584in}{0.795181in}}%
\pgfpathlineto{\pgfqpoint{4.921880in}{0.795173in}}%
\pgfpathlineto{\pgfqpoint{4.922176in}{0.795164in}}%
\pgfpathlineto{\pgfqpoint{4.922472in}{0.795156in}}%
\pgfpathlineto{\pgfqpoint{4.922768in}{0.795158in}}%
\pgfpathlineto{\pgfqpoint{4.923064in}{0.795023in}}%
\pgfpathlineto{\pgfqpoint{4.923360in}{0.794556in}}%
\pgfpathlineto{\pgfqpoint{4.923656in}{0.794081in}}%
\pgfpathlineto{\pgfqpoint{4.923952in}{0.793605in}}%
\pgfpathlineto{\pgfqpoint{4.924248in}{0.794785in}}%
\pgfpathlineto{\pgfqpoint{4.924544in}{0.795166in}}%
\pgfpathlineto{\pgfqpoint{4.924840in}{0.795181in}}%
\pgfpathlineto{\pgfqpoint{4.925136in}{0.795176in}}%
\pgfpathlineto{\pgfqpoint{4.925432in}{0.795150in}}%
\pgfpathlineto{\pgfqpoint{4.925728in}{0.795144in}}%
\pgfpathlineto{\pgfqpoint{4.926024in}{0.795138in}}%
\pgfpathlineto{\pgfqpoint{4.926320in}{0.795132in}}%
\pgfpathlineto{\pgfqpoint{4.926616in}{0.795126in}}%
\pgfpathlineto{\pgfqpoint{4.926912in}{0.795119in}}%
\pgfpathlineto{\pgfqpoint{4.927208in}{0.795113in}}%
\pgfpathlineto{\pgfqpoint{4.927504in}{0.795107in}}%
\pgfpathlineto{\pgfqpoint{4.927800in}{0.795101in}}%
\pgfpathlineto{\pgfqpoint{4.928096in}{0.795095in}}%
\pgfpathlineto{\pgfqpoint{4.928392in}{0.795089in}}%
\pgfpathlineto{\pgfqpoint{4.928688in}{0.795082in}}%
\pgfpathlineto{\pgfqpoint{4.928984in}{0.795076in}}%
\pgfpathlineto{\pgfqpoint{4.929280in}{0.795070in}}%
\pgfpathlineto{\pgfqpoint{4.929576in}{0.795064in}}%
\pgfpathlineto{\pgfqpoint{4.929872in}{0.795058in}}%
\pgfpathlineto{\pgfqpoint{4.930168in}{0.795051in}}%
\pgfpathlineto{\pgfqpoint{4.930464in}{0.795045in}}%
\pgfpathlineto{\pgfqpoint{4.930760in}{0.795039in}}%
\pgfpathlineto{\pgfqpoint{4.931056in}{0.795033in}}%
\pgfpathlineto{\pgfqpoint{4.931352in}{0.795027in}}%
\pgfpathlineto{\pgfqpoint{4.931648in}{0.795021in}}%
\pgfpathlineto{\pgfqpoint{4.931944in}{0.795014in}}%
\pgfpathlineto{\pgfqpoint{4.932240in}{0.795008in}}%
\pgfpathlineto{\pgfqpoint{4.932536in}{0.795002in}}%
\pgfpathlineto{\pgfqpoint{4.932832in}{0.794996in}}%
\pgfpathlineto{\pgfqpoint{4.933128in}{0.794990in}}%
\pgfpathlineto{\pgfqpoint{4.933424in}{0.794984in}}%
\pgfpathlineto{\pgfqpoint{4.933720in}{0.794977in}}%
\pgfpathlineto{\pgfqpoint{4.934016in}{0.794971in}}%
\pgfpathlineto{\pgfqpoint{4.934312in}{0.794965in}}%
\pgfpathlineto{\pgfqpoint{4.934608in}{0.794959in}}%
\pgfpathlineto{\pgfqpoint{4.934904in}{0.794953in}}%
\pgfpathlineto{\pgfqpoint{4.935200in}{0.794946in}}%
\pgfpathlineto{\pgfqpoint{4.935496in}{0.794940in}}%
\pgfpathlineto{\pgfqpoint{4.935792in}{0.794934in}}%
\pgfpathlineto{\pgfqpoint{4.936088in}{0.794928in}}%
\pgfpathlineto{\pgfqpoint{4.936384in}{0.794922in}}%
\pgfpathlineto{\pgfqpoint{4.936680in}{0.794916in}}%
\pgfpathlineto{\pgfqpoint{4.936976in}{0.794909in}}%
\pgfpathlineto{\pgfqpoint{4.937272in}{0.794903in}}%
\pgfpathlineto{\pgfqpoint{4.937568in}{0.794897in}}%
\pgfpathlineto{\pgfqpoint{4.937864in}{0.794891in}}%
\pgfpathlineto{\pgfqpoint{4.938160in}{0.794885in}}%
\pgfpathlineto{\pgfqpoint{4.938456in}{0.794879in}}%
\pgfpathlineto{\pgfqpoint{4.938752in}{0.794872in}}%
\pgfpathlineto{\pgfqpoint{4.939048in}{0.794866in}}%
\pgfpathlineto{\pgfqpoint{4.939344in}{0.794860in}}%
\pgfpathlineto{\pgfqpoint{4.939640in}{0.794854in}}%
\pgfpathlineto{\pgfqpoint{4.939936in}{0.794848in}}%
\pgfpathlineto{\pgfqpoint{4.940232in}{0.794841in}}%
\pgfpathlineto{\pgfqpoint{4.940528in}{0.794835in}}%
\pgfpathlineto{\pgfqpoint{4.940824in}{0.794829in}}%
\pgfpathlineto{\pgfqpoint{4.941120in}{0.794823in}}%
\pgfpathlineto{\pgfqpoint{4.941416in}{0.794817in}}%
\pgfpathlineto{\pgfqpoint{4.941712in}{0.794811in}}%
\pgfpathlineto{\pgfqpoint{4.942008in}{0.794804in}}%
\pgfpathlineto{\pgfqpoint{4.942304in}{0.794798in}}%
\pgfpathlineto{\pgfqpoint{4.942600in}{0.794792in}}%
\pgfpathlineto{\pgfqpoint{4.942896in}{0.794786in}}%
\pgfpathlineto{\pgfqpoint{4.943192in}{0.794780in}}%
\pgfpathlineto{\pgfqpoint{4.943488in}{0.794774in}}%
\pgfpathlineto{\pgfqpoint{4.943784in}{0.794767in}}%
\pgfpathlineto{\pgfqpoint{4.944080in}{0.794761in}}%
\pgfpathlineto{\pgfqpoint{4.944376in}{0.794755in}}%
\pgfpathlineto{\pgfqpoint{4.944672in}{0.794744in}}%
\pgfpathlineto{\pgfqpoint{4.944968in}{0.794129in}}%
\pgfpathlineto{\pgfqpoint{4.945264in}{0.793766in}}%
\pgfpathlineto{\pgfqpoint{4.945560in}{0.793776in}}%
\pgfpathlineto{\pgfqpoint{4.945856in}{0.793844in}}%
\pgfpathlineto{\pgfqpoint{4.946152in}{0.794815in}}%
\pgfpathlineto{\pgfqpoint{4.946448in}{0.796219in}}%
\pgfpathlineto{\pgfqpoint{4.946744in}{0.797623in}}%
\pgfpathlineto{\pgfqpoint{4.947040in}{0.799027in}}%
\pgfpathlineto{\pgfqpoint{4.947336in}{0.800431in}}%
\pgfpathlineto{\pgfqpoint{4.947632in}{0.801835in}}%
\pgfpathlineto{\pgfqpoint{4.947928in}{0.803239in}}%
\pgfpathlineto{\pgfqpoint{4.948224in}{0.804642in}}%
\pgfpathlineto{\pgfqpoint{4.948520in}{0.806046in}}%
\pgfpathlineto{\pgfqpoint{4.948816in}{0.807450in}}%
\pgfpathlineto{\pgfqpoint{4.949112in}{0.808854in}}%
\pgfpathlineto{\pgfqpoint{4.949408in}{0.810258in}}%
\pgfpathlineto{\pgfqpoint{4.949704in}{0.811662in}}%
\pgfpathlineto{\pgfqpoint{4.950000in}{0.813066in}}%
\pgfpathlineto{\pgfqpoint{4.950296in}{0.814470in}}%
\pgfpathlineto{\pgfqpoint{4.950592in}{0.815873in}}%
\pgfpathlineto{\pgfqpoint{4.950888in}{0.817277in}}%
\pgfpathlineto{\pgfqpoint{4.951184in}{0.795517in}}%
\pgfpathlineto{\pgfqpoint{4.951480in}{0.793448in}}%
\pgfpathlineto{\pgfqpoint{4.951776in}{0.793427in}}%
\pgfpathlineto{\pgfqpoint{4.952072in}{0.793415in}}%
\pgfpathlineto{\pgfqpoint{4.952368in}{0.793866in}}%
\pgfpathlineto{\pgfqpoint{4.952664in}{0.793805in}}%
\pgfpathlineto{\pgfqpoint{4.952960in}{0.793481in}}%
\pgfpathlineto{\pgfqpoint{4.953256in}{0.793141in}}%
\pgfpathlineto{\pgfqpoint{4.953552in}{0.792958in}}%
\pgfpathlineto{\pgfqpoint{4.953848in}{0.794184in}}%
\pgfpathlineto{\pgfqpoint{4.954144in}{0.795886in}}%
\pgfpathlineto{\pgfqpoint{4.954440in}{0.797588in}}%
\pgfpathlineto{\pgfqpoint{4.954736in}{0.799290in}}%
\pgfpathlineto{\pgfqpoint{4.955032in}{0.800992in}}%
\pgfpathlineto{\pgfqpoint{4.955328in}{0.802694in}}%
\pgfpathlineto{\pgfqpoint{4.955624in}{0.804396in}}%
\pgfpathlineto{\pgfqpoint{4.955920in}{0.806098in}}%
\pgfpathlineto{\pgfqpoint{4.956216in}{0.807800in}}%
\pgfpathlineto{\pgfqpoint{4.956512in}{0.809502in}}%
\pgfpathlineto{\pgfqpoint{4.956808in}{0.811204in}}%
\pgfpathlineto{\pgfqpoint{4.957104in}{0.812907in}}%
\pgfpathlineto{\pgfqpoint{4.957400in}{0.814609in}}%
\pgfpathlineto{\pgfqpoint{4.957696in}{0.816311in}}%
\pgfpathlineto{\pgfqpoint{4.957992in}{0.817254in}}%
\pgfpathlineto{\pgfqpoint{4.958288in}{0.817248in}}%
\pgfpathlineto{\pgfqpoint{4.958584in}{0.813818in}}%
\pgfpathlineto{\pgfqpoint{4.958880in}{0.807163in}}%
\pgfpathlineto{\pgfqpoint{4.959176in}{0.808547in}}%
\pgfpathlineto{\pgfqpoint{4.959472in}{0.813873in}}%
\pgfpathlineto{\pgfqpoint{4.959768in}{0.828720in}}%
\pgfpathlineto{\pgfqpoint{4.960064in}{0.830473in}}%
\pgfpathlineto{\pgfqpoint{4.960360in}{0.829350in}}%
\pgfpathlineto{\pgfqpoint{4.960656in}{0.828232in}}%
\pgfpathlineto{\pgfqpoint{4.960952in}{0.827119in}}%
\pgfpathlineto{\pgfqpoint{4.961248in}{0.826006in}}%
\pgfpathlineto{\pgfqpoint{4.961544in}{0.824893in}}%
\pgfpathlineto{\pgfqpoint{4.961840in}{0.823780in}}%
\pgfpathlineto{\pgfqpoint{4.962136in}{0.822667in}}%
\pgfpathlineto{\pgfqpoint{4.962432in}{0.821554in}}%
\pgfpathlineto{\pgfqpoint{4.962729in}{0.820441in}}%
\pgfpathlineto{\pgfqpoint{4.963025in}{0.819328in}}%
\pgfpathlineto{\pgfqpoint{4.963321in}{0.818215in}}%
\pgfpathlineto{\pgfqpoint{4.963617in}{0.817101in}}%
\pgfpathlineto{\pgfqpoint{4.963913in}{0.815988in}}%
\pgfpathlineto{\pgfqpoint{4.964209in}{0.814875in}}%
\pgfpathlineto{\pgfqpoint{4.964505in}{0.813762in}}%
\pgfpathlineto{\pgfqpoint{4.964801in}{0.812649in}}%
\pgfpathlineto{\pgfqpoint{4.965097in}{0.811536in}}%
\pgfpathlineto{\pgfqpoint{4.965393in}{0.810423in}}%
\pgfpathlineto{\pgfqpoint{4.965689in}{0.809310in}}%
\pgfpathlineto{\pgfqpoint{4.965985in}{0.808197in}}%
\pgfpathlineto{\pgfqpoint{4.966281in}{0.814016in}}%
\pgfpathlineto{\pgfqpoint{4.966577in}{0.823195in}}%
\pgfpathlineto{\pgfqpoint{4.966873in}{0.828042in}}%
\pgfpathlineto{\pgfqpoint{4.967169in}{0.815207in}}%
\pgfpathlineto{\pgfqpoint{4.967465in}{0.801931in}}%
\pgfpathlineto{\pgfqpoint{4.967761in}{0.821852in}}%
\pgfpathlineto{\pgfqpoint{4.968057in}{0.830308in}}%
\pgfpathlineto{\pgfqpoint{4.968353in}{0.829222in}}%
\pgfpathlineto{\pgfqpoint{4.968649in}{0.828136in}}%
\pgfpathlineto{\pgfqpoint{4.968945in}{0.827049in}}%
\pgfpathlineto{\pgfqpoint{4.969241in}{0.825963in}}%
\pgfpathlineto{\pgfqpoint{4.969537in}{0.824876in}}%
\pgfpathlineto{\pgfqpoint{4.969833in}{0.823790in}}%
\pgfpathlineto{\pgfqpoint{4.970129in}{0.822704in}}%
\pgfpathlineto{\pgfqpoint{4.970425in}{0.821617in}}%
\pgfpathlineto{\pgfqpoint{4.970721in}{0.820531in}}%
\pgfpathlineto{\pgfqpoint{4.971017in}{0.819445in}}%
\pgfpathlineto{\pgfqpoint{4.971313in}{0.818358in}}%
\pgfpathlineto{\pgfqpoint{4.971609in}{0.817272in}}%
\pgfpathlineto{\pgfqpoint{4.971905in}{0.816186in}}%
\pgfpathlineto{\pgfqpoint{4.972201in}{0.815099in}}%
\pgfpathlineto{\pgfqpoint{4.972497in}{0.814013in}}%
\pgfpathlineto{\pgfqpoint{4.972793in}{0.812927in}}%
\pgfpathlineto{\pgfqpoint{4.973089in}{0.811840in}}%
\pgfpathlineto{\pgfqpoint{4.973385in}{0.810755in}}%
\pgfpathlineto{\pgfqpoint{4.973681in}{0.809675in}}%
\pgfpathlineto{\pgfqpoint{4.973977in}{0.808595in}}%
\pgfpathlineto{\pgfqpoint{4.974273in}{0.807515in}}%
\pgfpathlineto{\pgfqpoint{4.974569in}{0.806801in}}%
\pgfpathlineto{\pgfqpoint{4.974865in}{0.806780in}}%
\pgfpathlineto{\pgfqpoint{4.975161in}{0.806775in}}%
\pgfpathlineto{\pgfqpoint{4.975457in}{0.806769in}}%
\pgfpathlineto{\pgfqpoint{4.975753in}{0.806764in}}%
\pgfpathlineto{\pgfqpoint{4.976049in}{0.806759in}}%
\pgfpathlineto{\pgfqpoint{4.976345in}{0.806754in}}%
\pgfpathlineto{\pgfqpoint{4.976641in}{0.806749in}}%
\pgfpathlineto{\pgfqpoint{4.976937in}{0.806743in}}%
\pgfpathlineto{\pgfqpoint{4.977233in}{0.806738in}}%
\pgfpathlineto{\pgfqpoint{4.977529in}{0.806733in}}%
\pgfpathlineto{\pgfqpoint{4.977825in}{0.806728in}}%
\pgfpathlineto{\pgfqpoint{4.978121in}{0.806723in}}%
\pgfpathlineto{\pgfqpoint{4.978417in}{0.806717in}}%
\pgfpathlineto{\pgfqpoint{4.978713in}{0.806712in}}%
\pgfpathlineto{\pgfqpoint{4.979009in}{0.806707in}}%
\pgfpathlineto{\pgfqpoint{4.979305in}{0.806702in}}%
\pgfpathlineto{\pgfqpoint{4.979601in}{0.806697in}}%
\pgfpathlineto{\pgfqpoint{4.979897in}{0.806691in}}%
\pgfpathlineto{\pgfqpoint{4.980193in}{0.806686in}}%
\pgfpathlineto{\pgfqpoint{4.980489in}{0.806681in}}%
\pgfpathlineto{\pgfqpoint{4.980785in}{0.806676in}}%
\pgfpathlineto{\pgfqpoint{4.981081in}{0.806671in}}%
\pgfpathlineto{\pgfqpoint{4.981377in}{0.806665in}}%
\pgfpathlineto{\pgfqpoint{4.981673in}{0.806660in}}%
\pgfpathlineto{\pgfqpoint{4.981969in}{0.806655in}}%
\pgfpathlineto{\pgfqpoint{4.982265in}{0.806650in}}%
\pgfpathlineto{\pgfqpoint{4.982561in}{0.806645in}}%
\pgfpathlineto{\pgfqpoint{4.982857in}{0.806639in}}%
\pgfpathlineto{\pgfqpoint{4.983153in}{0.806634in}}%
\pgfpathlineto{\pgfqpoint{4.983449in}{0.806629in}}%
\pgfpathlineto{\pgfqpoint{4.983745in}{0.806624in}}%
\pgfpathlineto{\pgfqpoint{4.984041in}{0.806619in}}%
\pgfpathlineto{\pgfqpoint{4.984337in}{0.806613in}}%
\pgfpathlineto{\pgfqpoint{4.984633in}{0.806608in}}%
\pgfpathlineto{\pgfqpoint{4.984929in}{0.806603in}}%
\pgfpathlineto{\pgfqpoint{4.985225in}{0.806598in}}%
\pgfpathlineto{\pgfqpoint{4.985521in}{0.806593in}}%
\pgfpathlineto{\pgfqpoint{4.985817in}{0.806587in}}%
\pgfpathlineto{\pgfqpoint{4.986113in}{0.806582in}}%
\pgfpathlineto{\pgfqpoint{4.986409in}{0.806577in}}%
\pgfpathlineto{\pgfqpoint{4.986705in}{0.806572in}}%
\pgfpathlineto{\pgfqpoint{4.987001in}{0.806567in}}%
\pgfpathlineto{\pgfqpoint{4.987297in}{0.806561in}}%
\pgfpathlineto{\pgfqpoint{4.987593in}{0.806556in}}%
\pgfpathlineto{\pgfqpoint{4.987889in}{0.806551in}}%
\pgfpathlineto{\pgfqpoint{4.988185in}{0.806546in}}%
\pgfpathlineto{\pgfqpoint{4.988481in}{0.806541in}}%
\pgfpathlineto{\pgfqpoint{4.988777in}{0.806535in}}%
\pgfpathlineto{\pgfqpoint{4.989073in}{0.806530in}}%
\pgfpathlineto{\pgfqpoint{4.989369in}{0.806525in}}%
\pgfpathlineto{\pgfqpoint{4.989665in}{0.806520in}}%
\pgfpathlineto{\pgfqpoint{4.989961in}{0.806515in}}%
\pgfpathlineto{\pgfqpoint{4.990257in}{0.806509in}}%
\pgfpathlineto{\pgfqpoint{4.990553in}{0.806504in}}%
\pgfpathlineto{\pgfqpoint{4.990849in}{0.806499in}}%
\pgfpathlineto{\pgfqpoint{4.991145in}{0.806494in}}%
\pgfpathlineto{\pgfqpoint{4.991441in}{0.806489in}}%
\pgfpathlineto{\pgfqpoint{4.991737in}{0.806483in}}%
\pgfpathlineto{\pgfqpoint{4.992033in}{0.806478in}}%
\pgfpathlineto{\pgfqpoint{4.992329in}{0.806473in}}%
\pgfpathlineto{\pgfqpoint{4.992625in}{0.806468in}}%
\pgfpathlineto{\pgfqpoint{4.992921in}{0.806463in}}%
\pgfpathlineto{\pgfqpoint{4.993217in}{0.806457in}}%
\pgfpathlineto{\pgfqpoint{4.993513in}{0.806452in}}%
\pgfpathlineto{\pgfqpoint{4.993809in}{0.806447in}}%
\pgfpathlineto{\pgfqpoint{4.994105in}{0.806442in}}%
\pgfpathlineto{\pgfqpoint{4.994401in}{0.806435in}}%
\pgfpathlineto{\pgfqpoint{4.994697in}{0.806439in}}%
\pgfpathlineto{\pgfqpoint{4.994993in}{0.806444in}}%
\pgfpathlineto{\pgfqpoint{4.995289in}{0.806449in}}%
\pgfpathlineto{\pgfqpoint{4.995585in}{0.806450in}}%
\pgfpathlineto{\pgfqpoint{4.995881in}{0.806445in}}%
\pgfpathlineto{\pgfqpoint{4.996177in}{0.806439in}}%
\pgfpathlineto{\pgfqpoint{4.996473in}{0.806775in}}%
\pgfpathlineto{\pgfqpoint{4.996769in}{0.807863in}}%
\pgfpathlineto{\pgfqpoint{4.997065in}{0.808436in}}%
\pgfpathlineto{\pgfqpoint{4.997361in}{0.808476in}}%
\pgfpathlineto{\pgfqpoint{4.997657in}{0.808516in}}%
\pgfpathlineto{\pgfqpoint{4.997953in}{0.808556in}}%
\pgfpathlineto{\pgfqpoint{4.998249in}{0.808596in}}%
\pgfpathlineto{\pgfqpoint{4.998545in}{0.808636in}}%
\pgfpathlineto{\pgfqpoint{4.998841in}{0.808676in}}%
\pgfpathlineto{\pgfqpoint{4.999137in}{0.808716in}}%
\pgfpathlineto{\pgfqpoint{4.999433in}{0.808756in}}%
\pgfpathlineto{\pgfqpoint{4.999729in}{0.808796in}}%
\pgfpathlineto{\pgfqpoint{5.000025in}{0.808836in}}%
\pgfpathlineto{\pgfqpoint{5.000321in}{0.808876in}}%
\pgfpathlineto{\pgfqpoint{5.000617in}{0.808916in}}%
\pgfpathlineto{\pgfqpoint{5.000913in}{0.808956in}}%
\pgfpathlineto{\pgfqpoint{5.001209in}{0.808996in}}%
\pgfpathlineto{\pgfqpoint{5.001505in}{0.809036in}}%
\pgfpathlineto{\pgfqpoint{5.001801in}{0.809075in}}%
\pgfpathlineto{\pgfqpoint{5.002097in}{0.808257in}}%
\pgfpathlineto{\pgfqpoint{5.002393in}{0.808309in}}%
\pgfpathlineto{\pgfqpoint{5.002689in}{0.809124in}}%
\pgfpathlineto{\pgfqpoint{5.002985in}{0.809114in}}%
\pgfpathlineto{\pgfqpoint{5.003281in}{0.809104in}}%
\pgfpathlineto{\pgfqpoint{5.003577in}{0.809092in}}%
\pgfpathlineto{\pgfqpoint{5.003873in}{0.809070in}}%
\pgfpathlineto{\pgfqpoint{5.004169in}{0.809069in}}%
\pgfpathlineto{\pgfqpoint{5.004465in}{0.809063in}}%
\pgfpathlineto{\pgfqpoint{5.004761in}{0.809044in}}%
\pgfpathlineto{\pgfqpoint{5.005057in}{0.809025in}}%
\pgfpathlineto{\pgfqpoint{5.005353in}{0.809006in}}%
\pgfpathlineto{\pgfqpoint{5.005649in}{0.808987in}}%
\pgfpathlineto{\pgfqpoint{5.005945in}{0.808968in}}%
\pgfpathlineto{\pgfqpoint{5.006241in}{0.808949in}}%
\pgfpathlineto{\pgfqpoint{5.006537in}{0.808929in}}%
\pgfpathlineto{\pgfqpoint{5.006833in}{0.808910in}}%
\pgfpathlineto{\pgfqpoint{5.007129in}{0.808891in}}%
\pgfpathlineto{\pgfqpoint{5.007425in}{0.808872in}}%
\pgfpathlineto{\pgfqpoint{5.007721in}{0.808791in}}%
\pgfpathlineto{\pgfqpoint{5.008017in}{0.807880in}}%
\pgfpathlineto{\pgfqpoint{5.008313in}{0.807805in}}%
\pgfpathlineto{\pgfqpoint{5.008609in}{0.807728in}}%
\pgfpathlineto{\pgfqpoint{5.008905in}{0.807651in}}%
\pgfpathlineto{\pgfqpoint{5.009201in}{0.807575in}}%
\pgfpathlineto{\pgfqpoint{5.009497in}{0.807498in}}%
\pgfpathlineto{\pgfqpoint{5.009793in}{0.807421in}}%
\pgfpathlineto{\pgfqpoint{5.010089in}{0.807328in}}%
\pgfpathlineto{\pgfqpoint{5.010385in}{0.806803in}}%
\pgfpathlineto{\pgfqpoint{5.010681in}{0.806796in}}%
\pgfpathlineto{\pgfqpoint{5.010977in}{0.806797in}}%
\pgfpathlineto{\pgfqpoint{5.011273in}{0.806985in}}%
\pgfpathlineto{\pgfqpoint{5.011569in}{0.807214in}}%
\pgfpathlineto{\pgfqpoint{5.011865in}{0.807443in}}%
\pgfpathlineto{\pgfqpoint{5.012161in}{0.807671in}}%
\pgfpathlineto{\pgfqpoint{5.012457in}{0.807900in}}%
\pgfpathlineto{\pgfqpoint{5.012753in}{0.808129in}}%
\pgfpathlineto{\pgfqpoint{5.013049in}{0.808357in}}%
\pgfpathlineto{\pgfqpoint{5.013345in}{0.808586in}}%
\pgfpathlineto{\pgfqpoint{5.013641in}{0.808815in}}%
\pgfpathlineto{\pgfqpoint{5.013937in}{0.809043in}}%
\pgfpathlineto{\pgfqpoint{5.014233in}{0.809272in}}%
\pgfpathlineto{\pgfqpoint{5.014529in}{0.809501in}}%
\pgfpathlineto{\pgfqpoint{5.014825in}{0.809730in}}%
\pgfpathlineto{\pgfqpoint{5.015121in}{0.809958in}}%
\pgfpathlineto{\pgfqpoint{5.015417in}{0.810185in}}%
\pgfpathlineto{\pgfqpoint{5.015713in}{0.810343in}}%
\pgfpathlineto{\pgfqpoint{5.016009in}{0.810470in}}%
\pgfpathlineto{\pgfqpoint{5.016305in}{0.810592in}}%
\pgfpathlineto{\pgfqpoint{5.016601in}{0.810626in}}%
\pgfpathlineto{\pgfqpoint{5.016897in}{0.810625in}}%
\pgfpathlineto{\pgfqpoint{5.017193in}{0.810142in}}%
\pgfpathlineto{\pgfqpoint{5.017489in}{0.810796in}}%
\pgfpathlineto{\pgfqpoint{5.017785in}{0.812323in}}%
\pgfpathlineto{\pgfqpoint{5.018081in}{0.812348in}}%
\pgfpathlineto{\pgfqpoint{5.018377in}{0.812372in}}%
\pgfpathlineto{\pgfqpoint{5.018673in}{0.812397in}}%
\pgfpathlineto{\pgfqpoint{5.018969in}{0.812421in}}%
\pgfpathlineto{\pgfqpoint{5.019265in}{0.812446in}}%
\pgfpathlineto{\pgfqpoint{5.019561in}{0.812471in}}%
\pgfpathlineto{\pgfqpoint{5.019857in}{0.812495in}}%
\pgfpathlineto{\pgfqpoint{5.020153in}{0.812520in}}%
\pgfpathlineto{\pgfqpoint{5.020449in}{0.812544in}}%
\pgfpathlineto{\pgfqpoint{5.020745in}{0.812569in}}%
\pgfpathlineto{\pgfqpoint{5.021041in}{0.812594in}}%
\pgfpathlineto{\pgfqpoint{5.021337in}{0.812618in}}%
\pgfpathlineto{\pgfqpoint{5.021633in}{0.812643in}}%
\pgfpathlineto{\pgfqpoint{5.021929in}{0.812667in}}%
\pgfpathlineto{\pgfqpoint{5.022225in}{0.812692in}}%
\pgfpathlineto{\pgfqpoint{5.022521in}{0.812716in}}%
\pgfpathlineto{\pgfqpoint{5.022817in}{0.812741in}}%
\pgfpathlineto{\pgfqpoint{5.023113in}{0.812766in}}%
\pgfpathlineto{\pgfqpoint{5.023409in}{0.812790in}}%
\pgfpathlineto{\pgfqpoint{5.023705in}{0.812815in}}%
\pgfpathlineto{\pgfqpoint{5.024001in}{0.812839in}}%
\pgfpathlineto{\pgfqpoint{5.024297in}{0.812864in}}%
\pgfpathlineto{\pgfqpoint{5.024593in}{0.812887in}}%
\pgfpathlineto{\pgfqpoint{5.024889in}{0.812890in}}%
\pgfpathlineto{\pgfqpoint{5.025185in}{0.812890in}}%
\pgfpathlineto{\pgfqpoint{5.025481in}{0.812889in}}%
\pgfpathlineto{\pgfqpoint{5.025777in}{0.812888in}}%
\pgfpathlineto{\pgfqpoint{5.026073in}{0.812887in}}%
\pgfpathlineto{\pgfqpoint{5.026369in}{0.812886in}}%
\pgfpathlineto{\pgfqpoint{5.026665in}{0.812885in}}%
\pgfpathlineto{\pgfqpoint{5.026961in}{0.812884in}}%
\pgfpathlineto{\pgfqpoint{5.027257in}{0.812884in}}%
\pgfpathlineto{\pgfqpoint{5.027553in}{0.812883in}}%
\pgfpathlineto{\pgfqpoint{5.027849in}{0.812882in}}%
\pgfpathlineto{\pgfqpoint{5.028145in}{0.812881in}}%
\pgfpathlineto{\pgfqpoint{5.028441in}{0.812880in}}%
\pgfpathlineto{\pgfqpoint{5.028737in}{0.812879in}}%
\pgfpathlineto{\pgfqpoint{5.029033in}{0.812878in}}%
\pgfpathlineto{\pgfqpoint{5.029329in}{0.812877in}}%
\pgfpathlineto{\pgfqpoint{5.029625in}{0.812877in}}%
\pgfpathlineto{\pgfqpoint{5.029922in}{0.812876in}}%
\pgfpathlineto{\pgfqpoint{5.030218in}{0.812875in}}%
\pgfpathlineto{\pgfqpoint{5.030514in}{0.812874in}}%
\pgfpathlineto{\pgfqpoint{5.030810in}{0.812873in}}%
\pgfpathlineto{\pgfqpoint{5.031106in}{0.812872in}}%
\pgfpathlineto{\pgfqpoint{5.031402in}{0.812871in}}%
\pgfpathlineto{\pgfqpoint{5.031698in}{0.812871in}}%
\pgfpathlineto{\pgfqpoint{5.031994in}{0.812870in}}%
\pgfpathlineto{\pgfqpoint{5.032290in}{0.812869in}}%
\pgfpathlineto{\pgfqpoint{5.032586in}{0.812868in}}%
\pgfpathlineto{\pgfqpoint{5.032882in}{0.812867in}}%
\pgfpathlineto{\pgfqpoint{5.033178in}{0.812866in}}%
\pgfpathlineto{\pgfqpoint{5.033474in}{0.812865in}}%
\pgfpathlineto{\pgfqpoint{5.033770in}{0.812865in}}%
\pgfpathlineto{\pgfqpoint{5.034066in}{0.812864in}}%
\pgfpathlineto{\pgfqpoint{5.034362in}{0.812863in}}%
\pgfpathlineto{\pgfqpoint{5.034658in}{0.812862in}}%
\pgfpathlineto{\pgfqpoint{5.034954in}{0.812861in}}%
\pgfpathlineto{\pgfqpoint{5.035250in}{0.812860in}}%
\pgfpathlineto{\pgfqpoint{5.035546in}{0.812859in}}%
\pgfpathlineto{\pgfqpoint{5.035842in}{0.812859in}}%
\pgfpathlineto{\pgfqpoint{5.036138in}{0.812858in}}%
\pgfpathlineto{\pgfqpoint{5.036434in}{0.812857in}}%
\pgfpathlineto{\pgfqpoint{5.036730in}{0.812856in}}%
\pgfpathlineto{\pgfqpoint{5.037026in}{0.812855in}}%
\pgfpathlineto{\pgfqpoint{5.037322in}{0.812854in}}%
\pgfpathlineto{\pgfqpoint{5.037618in}{0.812853in}}%
\pgfpathlineto{\pgfqpoint{5.037914in}{0.812852in}}%
\pgfpathlineto{\pgfqpoint{5.038210in}{0.812852in}}%
\pgfpathlineto{\pgfqpoint{5.038506in}{0.812851in}}%
\pgfpathlineto{\pgfqpoint{5.038802in}{0.812850in}}%
\pgfpathlineto{\pgfqpoint{5.039098in}{0.812849in}}%
\pgfpathlineto{\pgfqpoint{5.039394in}{0.812848in}}%
\pgfpathlineto{\pgfqpoint{5.039690in}{0.812847in}}%
\pgfpathlineto{\pgfqpoint{5.039986in}{0.812846in}}%
\pgfpathlineto{\pgfqpoint{5.040282in}{0.812846in}}%
\pgfpathlineto{\pgfqpoint{5.040578in}{0.812845in}}%
\pgfpathlineto{\pgfqpoint{5.040874in}{0.812844in}}%
\pgfpathlineto{\pgfqpoint{5.041170in}{0.812843in}}%
\pgfpathlineto{\pgfqpoint{5.041466in}{0.812842in}}%
\pgfpathlineto{\pgfqpoint{5.041762in}{0.812841in}}%
\pgfpathlineto{\pgfqpoint{5.042058in}{0.812840in}}%
\pgfpathlineto{\pgfqpoint{5.042354in}{0.812840in}}%
\pgfpathlineto{\pgfqpoint{5.042650in}{0.812839in}}%
\pgfpathlineto{\pgfqpoint{5.042946in}{0.812838in}}%
\pgfpathlineto{\pgfqpoint{5.043242in}{0.812837in}}%
\pgfpathlineto{\pgfqpoint{5.043538in}{0.812836in}}%
\pgfpathlineto{\pgfqpoint{5.043834in}{0.812834in}}%
\pgfpathlineto{\pgfqpoint{5.044130in}{0.812811in}}%
\pgfpathlineto{\pgfqpoint{5.044426in}{0.812778in}}%
\pgfpathlineto{\pgfqpoint{5.044722in}{0.812746in}}%
\pgfpathlineto{\pgfqpoint{5.045018in}{0.812713in}}%
\pgfpathlineto{\pgfqpoint{5.045314in}{0.812680in}}%
\pgfpathlineto{\pgfqpoint{5.045610in}{0.812648in}}%
\pgfpathlineto{\pgfqpoint{5.045906in}{0.812630in}}%
\pgfpathlineto{\pgfqpoint{5.046202in}{0.812675in}}%
\pgfpathlineto{\pgfqpoint{5.046498in}{0.812726in}}%
\pgfpathlineto{\pgfqpoint{5.046794in}{0.807043in}}%
\pgfpathlineto{\pgfqpoint{5.047090in}{0.793047in}}%
\pgfpathlineto{\pgfqpoint{5.047386in}{0.794573in}}%
\pgfpathlineto{\pgfqpoint{5.047682in}{0.796099in}}%
\pgfpathlineto{\pgfqpoint{5.047978in}{0.797626in}}%
\pgfpathlineto{\pgfqpoint{5.048274in}{0.799152in}}%
\pgfpathlineto{\pgfqpoint{5.048570in}{0.800678in}}%
\pgfpathlineto{\pgfqpoint{5.048866in}{0.802204in}}%
\pgfpathlineto{\pgfqpoint{5.049162in}{0.803730in}}%
\pgfpathlineto{\pgfqpoint{5.049458in}{0.805256in}}%
\pgfpathlineto{\pgfqpoint{5.049754in}{0.806782in}}%
\pgfpathlineto{\pgfqpoint{5.050050in}{0.808308in}}%
\pgfpathlineto{\pgfqpoint{5.050346in}{0.809528in}}%
\pgfpathlineto{\pgfqpoint{5.050642in}{0.792226in}}%
\pgfpathlineto{\pgfqpoint{5.050938in}{0.792186in}}%
\pgfpathlineto{\pgfqpoint{5.051234in}{0.792146in}}%
\pgfpathlineto{\pgfqpoint{5.051530in}{0.792106in}}%
\pgfpathlineto{\pgfqpoint{5.051826in}{0.792067in}}%
\pgfpathlineto{\pgfqpoint{5.052122in}{0.792027in}}%
\pgfpathlineto{\pgfqpoint{5.052418in}{0.791987in}}%
\pgfpathlineto{\pgfqpoint{5.052714in}{0.791947in}}%
\pgfpathlineto{\pgfqpoint{5.053010in}{0.791933in}}%
\pgfpathlineto{\pgfqpoint{5.053306in}{0.791937in}}%
\pgfpathlineto{\pgfqpoint{5.053602in}{0.791940in}}%
\pgfpathlineto{\pgfqpoint{5.053898in}{0.791944in}}%
\pgfpathlineto{\pgfqpoint{5.054194in}{0.791947in}}%
\pgfpathlineto{\pgfqpoint{5.054490in}{0.791950in}}%
\pgfpathlineto{\pgfqpoint{5.054786in}{0.791954in}}%
\pgfpathlineto{\pgfqpoint{5.055082in}{0.791957in}}%
\pgfpathlineto{\pgfqpoint{5.055378in}{0.791961in}}%
\pgfpathlineto{\pgfqpoint{5.055674in}{0.791964in}}%
\pgfpathlineto{\pgfqpoint{5.055970in}{0.791967in}}%
\pgfpathlineto{\pgfqpoint{5.056266in}{0.791971in}}%
\pgfpathlineto{\pgfqpoint{5.056562in}{0.791974in}}%
\pgfpathlineto{\pgfqpoint{5.056858in}{0.791978in}}%
\pgfpathlineto{\pgfqpoint{5.057154in}{0.791981in}}%
\pgfpathlineto{\pgfqpoint{5.057450in}{0.791984in}}%
\pgfpathlineto{\pgfqpoint{5.057746in}{0.791988in}}%
\pgfpathlineto{\pgfqpoint{5.058042in}{0.791991in}}%
\pgfpathlineto{\pgfqpoint{5.058338in}{0.791995in}}%
\pgfpathlineto{\pgfqpoint{5.058634in}{0.791998in}}%
\pgfpathlineto{\pgfqpoint{5.058930in}{0.792001in}}%
\pgfpathlineto{\pgfqpoint{5.059226in}{0.792005in}}%
\pgfpathlineto{\pgfqpoint{5.059522in}{0.792008in}}%
\pgfpathlineto{\pgfqpoint{5.059818in}{0.792011in}}%
\pgfpathlineto{\pgfqpoint{5.060114in}{0.792015in}}%
\pgfpathlineto{\pgfqpoint{5.060410in}{0.792024in}}%
\pgfpathlineto{\pgfqpoint{5.060706in}{0.792034in}}%
\pgfpathlineto{\pgfqpoint{5.061002in}{0.792044in}}%
\pgfpathlineto{\pgfqpoint{5.061298in}{0.792054in}}%
\pgfpathlineto{\pgfqpoint{5.061594in}{0.792063in}}%
\pgfpathlineto{\pgfqpoint{5.061890in}{0.792073in}}%
\pgfpathlineto{\pgfqpoint{5.062186in}{0.792083in}}%
\pgfpathlineto{\pgfqpoint{5.062482in}{0.792093in}}%
\pgfpathlineto{\pgfqpoint{5.062778in}{0.792103in}}%
\pgfpathlineto{\pgfqpoint{5.063074in}{0.792113in}}%
\pgfpathlineto{\pgfqpoint{5.063370in}{0.792123in}}%
\pgfpathlineto{\pgfqpoint{5.063666in}{0.792132in}}%
\pgfpathlineto{\pgfqpoint{5.063962in}{0.792142in}}%
\pgfpathlineto{\pgfqpoint{5.064258in}{0.792152in}}%
\pgfpathlineto{\pgfqpoint{5.064554in}{0.792162in}}%
\pgfpathlineto{\pgfqpoint{5.064850in}{0.792172in}}%
\pgfpathlineto{\pgfqpoint{5.065146in}{0.792182in}}%
\pgfpathlineto{\pgfqpoint{5.065442in}{0.792192in}}%
\pgfpathlineto{\pgfqpoint{5.065738in}{0.792137in}}%
\pgfpathlineto{\pgfqpoint{5.066034in}{0.792074in}}%
\pgfpathlineto{\pgfqpoint{5.066330in}{0.792118in}}%
\pgfpathlineto{\pgfqpoint{5.066626in}{0.792118in}}%
\pgfpathlineto{\pgfqpoint{5.066922in}{0.792119in}}%
\pgfpathlineto{\pgfqpoint{5.067218in}{0.792119in}}%
\pgfpathlineto{\pgfqpoint{5.067514in}{0.792119in}}%
\pgfpathlineto{\pgfqpoint{5.067810in}{0.792119in}}%
\pgfpathlineto{\pgfqpoint{5.068106in}{0.792119in}}%
\pgfpathlineto{\pgfqpoint{5.068402in}{0.792120in}}%
\pgfpathlineto{\pgfqpoint{5.068698in}{0.792120in}}%
\pgfpathlineto{\pgfqpoint{5.068994in}{0.792120in}}%
\pgfpathlineto{\pgfqpoint{5.069290in}{0.792120in}}%
\pgfpathlineto{\pgfqpoint{5.069586in}{0.792120in}}%
\pgfpathlineto{\pgfqpoint{5.069882in}{0.792121in}}%
\pgfpathlineto{\pgfqpoint{5.070178in}{0.792121in}}%
\pgfpathlineto{\pgfqpoint{5.070474in}{0.792121in}}%
\pgfpathlineto{\pgfqpoint{5.070770in}{0.792121in}}%
\pgfpathlineto{\pgfqpoint{5.071066in}{0.792121in}}%
\pgfpathlineto{\pgfqpoint{5.071362in}{0.792122in}}%
\pgfpathlineto{\pgfqpoint{5.071658in}{0.792122in}}%
\pgfpathlineto{\pgfqpoint{5.071954in}{0.792172in}}%
\pgfpathlineto{\pgfqpoint{5.072250in}{0.792267in}}%
\pgfpathlineto{\pgfqpoint{5.072546in}{0.792284in}}%
\pgfpathlineto{\pgfqpoint{5.072842in}{0.792283in}}%
\pgfpathlineto{\pgfqpoint{5.073138in}{0.792375in}}%
\pgfpathlineto{\pgfqpoint{5.073434in}{0.792919in}}%
\pgfpathlineto{\pgfqpoint{5.073730in}{0.793523in}}%
\pgfpathlineto{\pgfqpoint{5.074026in}{0.793984in}}%
\pgfpathlineto{\pgfqpoint{5.074322in}{0.793907in}}%
\pgfpathlineto{\pgfqpoint{5.074618in}{0.793779in}}%
\pgfpathlineto{\pgfqpoint{5.074914in}{0.793456in}}%
\pgfpathlineto{\pgfqpoint{5.075210in}{0.792980in}}%
\pgfpathlineto{\pgfqpoint{5.075506in}{0.792994in}}%
\pgfpathlineto{\pgfqpoint{5.075802in}{0.793008in}}%
\pgfpathlineto{\pgfqpoint{5.076098in}{0.793021in}}%
\pgfpathlineto{\pgfqpoint{5.076394in}{0.793035in}}%
\pgfpathlineto{\pgfqpoint{5.076690in}{0.793049in}}%
\pgfpathlineto{\pgfqpoint{5.076986in}{0.793063in}}%
\pgfpathlineto{\pgfqpoint{5.077282in}{0.793077in}}%
\pgfpathlineto{\pgfqpoint{5.077578in}{0.793090in}}%
\pgfpathlineto{\pgfqpoint{5.077874in}{0.793104in}}%
\pgfpathlineto{\pgfqpoint{5.078170in}{0.793118in}}%
\pgfpathlineto{\pgfqpoint{5.078466in}{0.793132in}}%
\pgfpathlineto{\pgfqpoint{5.078762in}{0.793146in}}%
\pgfpathlineto{\pgfqpoint{5.079058in}{0.793159in}}%
\pgfpathlineto{\pgfqpoint{5.079354in}{0.793173in}}%
\pgfpathlineto{\pgfqpoint{5.079650in}{0.793187in}}%
\pgfpathlineto{\pgfqpoint{5.079946in}{0.793201in}}%
\pgfpathlineto{\pgfqpoint{5.080242in}{0.793214in}}%
\pgfpathlineto{\pgfqpoint{5.080538in}{0.793228in}}%
\pgfpathlineto{\pgfqpoint{5.080834in}{0.793242in}}%
\pgfpathlineto{\pgfqpoint{5.081130in}{0.793256in}}%
\pgfpathlineto{\pgfqpoint{5.081426in}{0.793270in}}%
\pgfpathlineto{\pgfqpoint{5.081722in}{0.793283in}}%
\pgfpathlineto{\pgfqpoint{5.082018in}{0.793297in}}%
\pgfpathlineto{\pgfqpoint{5.082314in}{0.793311in}}%
\pgfpathlineto{\pgfqpoint{5.082610in}{0.793325in}}%
\pgfpathlineto{\pgfqpoint{5.082906in}{0.793339in}}%
\pgfpathlineto{\pgfqpoint{5.083202in}{0.793352in}}%
\pgfpathlineto{\pgfqpoint{5.083498in}{0.793366in}}%
\pgfpathlineto{\pgfqpoint{5.083794in}{0.793380in}}%
\pgfpathlineto{\pgfqpoint{5.084090in}{0.793394in}}%
\pgfpathlineto{\pgfqpoint{5.084386in}{0.793407in}}%
\pgfpathlineto{\pgfqpoint{5.084682in}{0.793421in}}%
\pgfpathlineto{\pgfqpoint{5.084978in}{0.793435in}}%
\pgfpathlineto{\pgfqpoint{5.085274in}{0.793449in}}%
\pgfpathlineto{\pgfqpoint{5.085570in}{0.793463in}}%
\pgfpathlineto{\pgfqpoint{5.085866in}{0.793476in}}%
\pgfpathlineto{\pgfqpoint{5.086162in}{0.793490in}}%
\pgfpathlineto{\pgfqpoint{5.086458in}{0.793504in}}%
\pgfpathlineto{\pgfqpoint{5.086754in}{0.793518in}}%
\pgfpathlineto{\pgfqpoint{5.087050in}{0.793532in}}%
\pgfpathlineto{\pgfqpoint{5.087346in}{0.793545in}}%
\pgfpathlineto{\pgfqpoint{5.087642in}{0.793559in}}%
\pgfpathlineto{\pgfqpoint{5.087938in}{0.793573in}}%
\pgfpathlineto{\pgfqpoint{5.088234in}{0.793587in}}%
\pgfpathlineto{\pgfqpoint{5.088530in}{0.793600in}}%
\pgfpathlineto{\pgfqpoint{5.088826in}{0.793614in}}%
\pgfpathlineto{\pgfqpoint{5.089122in}{0.793628in}}%
\pgfpathlineto{\pgfqpoint{5.089418in}{0.793642in}}%
\pgfpathlineto{\pgfqpoint{5.089714in}{0.793656in}}%
\pgfpathlineto{\pgfqpoint{5.090010in}{0.793669in}}%
\pgfpathlineto{\pgfqpoint{5.090306in}{0.793683in}}%
\pgfpathlineto{\pgfqpoint{5.090602in}{0.793697in}}%
\pgfpathlineto{\pgfqpoint{5.090898in}{0.793711in}}%
\pgfpathlineto{\pgfqpoint{5.091194in}{0.793725in}}%
\pgfpathlineto{\pgfqpoint{5.091490in}{0.793738in}}%
\pgfpathlineto{\pgfqpoint{5.091786in}{0.793752in}}%
\pgfpathlineto{\pgfqpoint{5.092082in}{0.793766in}}%
\pgfpathlineto{\pgfqpoint{5.092378in}{0.793780in}}%
\pgfpathlineto{\pgfqpoint{5.092674in}{0.793793in}}%
\pgfpathlineto{\pgfqpoint{5.092970in}{0.793774in}}%
\pgfpathlineto{\pgfqpoint{5.093266in}{0.793622in}}%
\pgfpathlineto{\pgfqpoint{5.093562in}{0.793599in}}%
\pgfpathlineto{\pgfqpoint{5.093858in}{0.793588in}}%
\pgfpathlineto{\pgfqpoint{5.094154in}{0.793577in}}%
\pgfpathlineto{\pgfqpoint{5.094450in}{0.793566in}}%
\pgfpathlineto{\pgfqpoint{5.094746in}{0.793555in}}%
\pgfpathlineto{\pgfqpoint{5.095042in}{0.793547in}}%
\pgfpathlineto{\pgfqpoint{5.095338in}{0.793548in}}%
\pgfpathlineto{\pgfqpoint{5.095634in}{0.793550in}}%
\pgfpathlineto{\pgfqpoint{5.095930in}{0.793552in}}%
\pgfpathlineto{\pgfqpoint{5.096226in}{0.793064in}}%
\pgfpathlineto{\pgfqpoint{5.096522in}{0.792704in}}%
\pgfpathlineto{\pgfqpoint{5.096818in}{0.792674in}}%
\pgfpathlineto{\pgfqpoint{5.097114in}{0.792644in}}%
\pgfpathlineto{\pgfqpoint{5.097411in}{0.792614in}}%
\pgfpathlineto{\pgfqpoint{5.097707in}{0.792584in}}%
\pgfpathlineto{\pgfqpoint{5.098003in}{0.792554in}}%
\pgfpathlineto{\pgfqpoint{5.098299in}{0.792524in}}%
\pgfpathlineto{\pgfqpoint{5.098595in}{0.792494in}}%
\pgfpathlineto{\pgfqpoint{5.098891in}{0.792464in}}%
\pgfpathlineto{\pgfqpoint{5.099187in}{0.792434in}}%
\pgfpathlineto{\pgfqpoint{5.099483in}{0.792404in}}%
\pgfpathlineto{\pgfqpoint{5.099779in}{0.792374in}}%
\pgfpathlineto{\pgfqpoint{5.100075in}{0.792344in}}%
\pgfpathlineto{\pgfqpoint{5.100371in}{0.792295in}}%
\pgfpathlineto{\pgfqpoint{5.100667in}{0.792342in}}%
\pgfpathlineto{\pgfqpoint{5.100963in}{0.791999in}}%
\pgfpathlineto{\pgfqpoint{5.101259in}{0.792045in}}%
\pgfpathlineto{\pgfqpoint{5.101555in}{0.792151in}}%
\pgfpathlineto{\pgfqpoint{5.101851in}{0.792257in}}%
\pgfpathlineto{\pgfqpoint{5.102147in}{0.792362in}}%
\pgfpathlineto{\pgfqpoint{5.102443in}{0.792008in}}%
\pgfpathlineto{\pgfqpoint{5.102739in}{0.791161in}}%
\pgfpathlineto{\pgfqpoint{5.103035in}{0.789748in}}%
\pgfpathlineto{\pgfqpoint{5.103331in}{0.790000in}}%
\pgfpathlineto{\pgfqpoint{5.103627in}{0.789136in}}%
\pgfpathlineto{\pgfqpoint{5.103923in}{0.788882in}}%
\pgfpathlineto{\pgfqpoint{5.104219in}{0.788739in}}%
\pgfpathlineto{\pgfqpoint{5.104515in}{0.788684in}}%
\pgfpathlineto{\pgfqpoint{5.104811in}{0.788629in}}%
\pgfpathlineto{\pgfqpoint{5.105107in}{0.788574in}}%
\pgfpathlineto{\pgfqpoint{5.105403in}{0.788518in}}%
\pgfpathlineto{\pgfqpoint{5.105699in}{0.788463in}}%
\pgfpathlineto{\pgfqpoint{5.105995in}{0.788408in}}%
\pgfpathlineto{\pgfqpoint{5.106291in}{0.788353in}}%
\pgfpathlineto{\pgfqpoint{5.106587in}{0.788298in}}%
\pgfpathlineto{\pgfqpoint{5.106883in}{0.788243in}}%
\pgfpathlineto{\pgfqpoint{5.107179in}{0.788188in}}%
\pgfpathlineto{\pgfqpoint{5.107475in}{0.787867in}}%
\pgfpathlineto{\pgfqpoint{5.107771in}{0.787244in}}%
\pgfpathlineto{\pgfqpoint{5.108067in}{0.787016in}}%
\pgfpathlineto{\pgfqpoint{5.108363in}{0.786882in}}%
\pgfpathlineto{\pgfqpoint{5.108659in}{0.786748in}}%
\pgfpathlineto{\pgfqpoint{5.108955in}{0.786614in}}%
\pgfpathlineto{\pgfqpoint{5.109251in}{0.786543in}}%
\pgfpathlineto{\pgfqpoint{5.109547in}{0.786523in}}%
\pgfpathlineto{\pgfqpoint{5.109843in}{0.786504in}}%
\pgfpathlineto{\pgfqpoint{5.110139in}{0.786484in}}%
\pgfpathlineto{\pgfqpoint{5.110435in}{0.786464in}}%
\pgfpathlineto{\pgfqpoint{5.110731in}{0.786444in}}%
\pgfpathlineto{\pgfqpoint{5.111027in}{0.786425in}}%
\pgfpathlineto{\pgfqpoint{5.111323in}{0.786405in}}%
\pgfpathlineto{\pgfqpoint{5.111619in}{0.786385in}}%
\pgfpathlineto{\pgfqpoint{5.111915in}{0.786365in}}%
\pgfpathlineto{\pgfqpoint{5.112211in}{0.786345in}}%
\pgfpathlineto{\pgfqpoint{5.112507in}{0.786326in}}%
\pgfpathlineto{\pgfqpoint{5.112803in}{0.786306in}}%
\pgfpathlineto{\pgfqpoint{5.113099in}{0.786286in}}%
\pgfpathlineto{\pgfqpoint{5.113395in}{0.786266in}}%
\pgfpathlineto{\pgfqpoint{5.113691in}{0.786246in}}%
\pgfpathlineto{\pgfqpoint{5.113987in}{0.786227in}}%
\pgfpathlineto{\pgfqpoint{5.114283in}{0.786207in}}%
\pgfpathlineto{\pgfqpoint{5.114579in}{0.786187in}}%
\pgfpathlineto{\pgfqpoint{5.114875in}{0.786167in}}%
\pgfpathlineto{\pgfqpoint{5.115171in}{0.786147in}}%
\pgfpathlineto{\pgfqpoint{5.115467in}{0.786128in}}%
\pgfpathlineto{\pgfqpoint{5.115763in}{0.786108in}}%
\pgfpathlineto{\pgfqpoint{5.116059in}{0.786079in}}%
\pgfpathlineto{\pgfqpoint{5.116355in}{0.786034in}}%
\pgfpathlineto{\pgfqpoint{5.116651in}{0.785989in}}%
\pgfpathlineto{\pgfqpoint{5.116947in}{0.785944in}}%
\pgfpathlineto{\pgfqpoint{5.117243in}{0.785898in}}%
\pgfpathlineto{\pgfqpoint{5.117539in}{0.785541in}}%
\pgfpathlineto{\pgfqpoint{5.117835in}{0.785379in}}%
\pgfpathlineto{\pgfqpoint{5.118131in}{0.785459in}}%
\pgfpathlineto{\pgfqpoint{5.118427in}{0.785538in}}%
\pgfpathlineto{\pgfqpoint{5.118723in}{0.785618in}}%
\pgfpathlineto{\pgfqpoint{5.119019in}{0.785697in}}%
\pgfpathlineto{\pgfqpoint{5.119315in}{0.785777in}}%
\pgfpathlineto{\pgfqpoint{5.119611in}{0.785857in}}%
\pgfpathlineto{\pgfqpoint{5.119907in}{0.785936in}}%
\pgfpathlineto{\pgfqpoint{5.120203in}{0.786016in}}%
\pgfpathlineto{\pgfqpoint{5.120499in}{0.786095in}}%
\pgfpathlineto{\pgfqpoint{5.120795in}{0.786175in}}%
\pgfpathlineto{\pgfqpoint{5.121091in}{0.786255in}}%
\pgfpathlineto{\pgfqpoint{5.121387in}{0.785632in}}%
\pgfpathlineto{\pgfqpoint{5.121683in}{0.785757in}}%
\pgfpathlineto{\pgfqpoint{5.121979in}{0.786185in}}%
\pgfpathlineto{\pgfqpoint{5.122275in}{0.786613in}}%
\pgfpathlineto{\pgfqpoint{5.122571in}{0.787041in}}%
\pgfpathlineto{\pgfqpoint{5.122867in}{0.787469in}}%
\pgfpathlineto{\pgfqpoint{5.123163in}{0.787363in}}%
\pgfpathlineto{\pgfqpoint{5.123459in}{0.787101in}}%
\pgfpathlineto{\pgfqpoint{5.123755in}{0.787820in}}%
\pgfpathlineto{\pgfqpoint{5.124051in}{0.788538in}}%
\pgfpathlineto{\pgfqpoint{5.124347in}{0.789235in}}%
\pgfpathlineto{\pgfqpoint{5.124643in}{0.789476in}}%
\pgfpathlineto{\pgfqpoint{5.124939in}{0.789535in}}%
\pgfpathlineto{\pgfqpoint{5.125235in}{0.789594in}}%
\pgfpathlineto{\pgfqpoint{5.125531in}{0.789653in}}%
\pgfpathlineto{\pgfqpoint{5.125827in}{0.789712in}}%
\pgfpathlineto{\pgfqpoint{5.126123in}{0.789771in}}%
\pgfpathlineto{\pgfqpoint{5.126419in}{0.789830in}}%
\pgfpathlineto{\pgfqpoint{5.126715in}{0.789889in}}%
\pgfpathlineto{\pgfqpoint{5.127011in}{0.789948in}}%
\pgfpathlineto{\pgfqpoint{5.127307in}{0.790007in}}%
\pgfpathlineto{\pgfqpoint{5.127603in}{0.790066in}}%
\pgfpathlineto{\pgfqpoint{5.127899in}{0.790125in}}%
\pgfpathlineto{\pgfqpoint{5.128195in}{0.790184in}}%
\pgfpathlineto{\pgfqpoint{5.128491in}{0.790243in}}%
\pgfpathlineto{\pgfqpoint{5.128787in}{0.790302in}}%
\pgfpathlineto{\pgfqpoint{5.129083in}{0.790361in}}%
\pgfpathlineto{\pgfqpoint{5.129379in}{0.790420in}}%
\pgfpathlineto{\pgfqpoint{5.129675in}{0.790479in}}%
\pgfpathlineto{\pgfqpoint{5.129971in}{0.790538in}}%
\pgfpathlineto{\pgfqpoint{5.130267in}{0.790597in}}%
\pgfpathlineto{\pgfqpoint{5.130563in}{0.790656in}}%
\pgfpathlineto{\pgfqpoint{5.130859in}{0.790715in}}%
\pgfpathlineto{\pgfqpoint{5.131155in}{0.790774in}}%
\pgfpathlineto{\pgfqpoint{5.131451in}{0.790834in}}%
\pgfpathlineto{\pgfqpoint{5.131747in}{0.790893in}}%
\pgfpathlineto{\pgfqpoint{5.132043in}{0.790952in}}%
\pgfpathlineto{\pgfqpoint{5.132339in}{0.791011in}}%
\pgfpathlineto{\pgfqpoint{5.132635in}{0.791070in}}%
\pgfpathlineto{\pgfqpoint{5.132931in}{0.791129in}}%
\pgfpathlineto{\pgfqpoint{5.133227in}{0.791188in}}%
\pgfpathlineto{\pgfqpoint{5.133523in}{0.791247in}}%
\pgfpathlineto{\pgfqpoint{5.133819in}{0.791306in}}%
\pgfpathlineto{\pgfqpoint{5.134115in}{0.791365in}}%
\pgfpathlineto{\pgfqpoint{5.134411in}{0.791424in}}%
\pgfpathlineto{\pgfqpoint{5.134707in}{0.791483in}}%
\pgfpathlineto{\pgfqpoint{5.135003in}{0.791542in}}%
\pgfpathlineto{\pgfqpoint{5.135299in}{0.791601in}}%
\pgfpathlineto{\pgfqpoint{5.135595in}{0.791660in}}%
\pgfpathlineto{\pgfqpoint{5.135891in}{0.791719in}}%
\pgfpathlineto{\pgfqpoint{5.136187in}{0.791778in}}%
\pgfpathlineto{\pgfqpoint{5.136483in}{0.791837in}}%
\pgfpathlineto{\pgfqpoint{5.136779in}{0.791896in}}%
\pgfpathlineto{\pgfqpoint{5.137075in}{0.791955in}}%
\pgfpathlineto{\pgfqpoint{5.137371in}{0.792014in}}%
\pgfpathlineto{\pgfqpoint{5.137667in}{0.792073in}}%
\pgfpathlineto{\pgfqpoint{5.137963in}{0.792132in}}%
\pgfpathlineto{\pgfqpoint{5.138259in}{0.792191in}}%
\pgfpathlineto{\pgfqpoint{5.138555in}{0.792250in}}%
\pgfpathlineto{\pgfqpoint{5.138851in}{0.792309in}}%
\pgfpathlineto{\pgfqpoint{5.139147in}{0.792368in}}%
\pgfpathlineto{\pgfqpoint{5.139443in}{0.792427in}}%
\pgfpathlineto{\pgfqpoint{5.139739in}{0.792486in}}%
\pgfpathlineto{\pgfqpoint{5.140035in}{0.792546in}}%
\pgfpathlineto{\pgfqpoint{5.140331in}{0.792605in}}%
\pgfpathlineto{\pgfqpoint{5.140627in}{0.792664in}}%
\pgfpathlineto{\pgfqpoint{5.140923in}{0.792723in}}%
\pgfpathlineto{\pgfqpoint{5.141219in}{0.792782in}}%
\pgfpathlineto{\pgfqpoint{5.141515in}{0.792841in}}%
\pgfpathlineto{\pgfqpoint{5.141811in}{0.792900in}}%
\pgfpathlineto{\pgfqpoint{5.142107in}{0.792959in}}%
\pgfpathlineto{\pgfqpoint{5.142403in}{0.793018in}}%
\pgfpathlineto{\pgfqpoint{5.142699in}{0.793077in}}%
\pgfpathlineto{\pgfqpoint{5.142995in}{0.793136in}}%
\pgfpathlineto{\pgfqpoint{5.143291in}{0.793195in}}%
\pgfpathlineto{\pgfqpoint{5.143587in}{0.793254in}}%
\pgfpathlineto{\pgfqpoint{5.143883in}{0.793313in}}%
\pgfpathlineto{\pgfqpoint{5.144179in}{0.793372in}}%
\pgfpathlineto{\pgfqpoint{5.144475in}{0.794361in}}%
\pgfpathlineto{\pgfqpoint{5.144771in}{0.795038in}}%
\pgfpathlineto{\pgfqpoint{5.145067in}{0.795064in}}%
\pgfpathlineto{\pgfqpoint{5.145363in}{0.795085in}}%
\pgfpathlineto{\pgfqpoint{5.145659in}{0.795083in}}%
\pgfpathlineto{\pgfqpoint{5.145955in}{0.795080in}}%
\pgfpathlineto{\pgfqpoint{5.146251in}{0.795078in}}%
\pgfpathlineto{\pgfqpoint{5.146547in}{0.795075in}}%
\pgfpathlineto{\pgfqpoint{5.146843in}{0.795072in}}%
\pgfpathlineto{\pgfqpoint{5.147139in}{0.795069in}}%
\pgfpathlineto{\pgfqpoint{5.147435in}{0.795066in}}%
\pgfpathlineto{\pgfqpoint{5.147731in}{0.795063in}}%
\pgfpathlineto{\pgfqpoint{5.148027in}{0.795060in}}%
\pgfpathlineto{\pgfqpoint{5.148323in}{0.795057in}}%
\pgfpathlineto{\pgfqpoint{5.148619in}{0.795054in}}%
\pgfpathlineto{\pgfqpoint{5.148915in}{0.795051in}}%
\pgfpathlineto{\pgfqpoint{5.149211in}{0.795048in}}%
\pgfpathlineto{\pgfqpoint{5.149507in}{0.795045in}}%
\pgfpathlineto{\pgfqpoint{5.149803in}{0.795042in}}%
\pgfpathlineto{\pgfqpoint{5.150099in}{0.795039in}}%
\pgfpathlineto{\pgfqpoint{5.150395in}{0.795038in}}%
\pgfpathlineto{\pgfqpoint{5.150691in}{0.795038in}}%
\pgfpathlineto{\pgfqpoint{5.150987in}{0.795037in}}%
\pgfpathlineto{\pgfqpoint{5.151283in}{0.795037in}}%
\pgfpathlineto{\pgfqpoint{5.151579in}{0.795036in}}%
\pgfpathlineto{\pgfqpoint{5.151875in}{0.795036in}}%
\pgfpathlineto{\pgfqpoint{5.152171in}{0.795035in}}%
\pgfpathlineto{\pgfqpoint{5.152467in}{0.795035in}}%
\pgfpathlineto{\pgfqpoint{5.152763in}{0.795034in}}%
\pgfpathlineto{\pgfqpoint{5.153059in}{0.790626in}}%
\pgfpathlineto{\pgfqpoint{5.153355in}{0.789836in}}%
\pgfpathlineto{\pgfqpoint{5.153651in}{0.790193in}}%
\pgfpathlineto{\pgfqpoint{5.153947in}{0.790550in}}%
\pgfpathlineto{\pgfqpoint{5.154243in}{0.790907in}}%
\pgfpathlineto{\pgfqpoint{5.154539in}{0.791264in}}%
\pgfpathlineto{\pgfqpoint{5.154835in}{0.791621in}}%
\pgfpathlineto{\pgfqpoint{5.155131in}{0.791978in}}%
\pgfpathlineto{\pgfqpoint{5.155427in}{0.792335in}}%
\pgfpathlineto{\pgfqpoint{5.155723in}{0.792692in}}%
\pgfpathlineto{\pgfqpoint{5.156019in}{0.793049in}}%
\pgfpathlineto{\pgfqpoint{5.156315in}{0.793406in}}%
\pgfpathlineto{\pgfqpoint{5.156611in}{0.793763in}}%
\pgfpathlineto{\pgfqpoint{5.156907in}{0.794120in}}%
\pgfpathlineto{\pgfqpoint{5.157203in}{0.794477in}}%
\pgfpathlineto{\pgfqpoint{5.157499in}{0.794834in}}%
\pgfpathlineto{\pgfqpoint{5.157795in}{0.794922in}}%
\pgfpathlineto{\pgfqpoint{5.158091in}{0.794627in}}%
\pgfpathlineto{\pgfqpoint{5.158387in}{0.794330in}}%
\pgfpathlineto{\pgfqpoint{5.158683in}{0.794033in}}%
\pgfpathlineto{\pgfqpoint{5.158979in}{0.793864in}}%
\pgfpathlineto{\pgfqpoint{5.159275in}{0.795277in}}%
\pgfpathlineto{\pgfqpoint{5.159571in}{0.795222in}}%
\pgfpathlineto{\pgfqpoint{5.159867in}{0.795167in}}%
\pgfpathlineto{\pgfqpoint{5.160163in}{0.795128in}}%
\pgfpathlineto{\pgfqpoint{5.160459in}{0.795145in}}%
\pgfpathlineto{\pgfqpoint{5.160755in}{0.795165in}}%
\pgfpathlineto{\pgfqpoint{5.161051in}{0.795186in}}%
\pgfpathlineto{\pgfqpoint{5.161347in}{0.795207in}}%
\pgfpathlineto{\pgfqpoint{5.161643in}{0.795227in}}%
\pgfpathlineto{\pgfqpoint{5.161939in}{0.795248in}}%
\pgfpathlineto{\pgfqpoint{5.162235in}{0.795269in}}%
\pgfpathlineto{\pgfqpoint{5.162531in}{0.795289in}}%
\pgfpathlineto{\pgfqpoint{5.162827in}{0.795310in}}%
\pgfpathlineto{\pgfqpoint{5.163123in}{0.795331in}}%
\pgfpathlineto{\pgfqpoint{5.163419in}{0.795352in}}%
\pgfpathlineto{\pgfqpoint{5.163715in}{0.795372in}}%
\pgfpathlineto{\pgfqpoint{5.164011in}{0.795393in}}%
\pgfpathlineto{\pgfqpoint{5.164307in}{0.795414in}}%
\pgfpathlineto{\pgfqpoint{5.164603in}{0.795434in}}%
\pgfpathlineto{\pgfqpoint{5.164900in}{0.795455in}}%
\pgfpathlineto{\pgfqpoint{5.165196in}{0.795476in}}%
\pgfpathlineto{\pgfqpoint{5.165492in}{0.795496in}}%
\pgfpathlineto{\pgfqpoint{5.165788in}{0.795517in}}%
\pgfpathlineto{\pgfqpoint{5.166084in}{0.795538in}}%
\pgfpathlineto{\pgfqpoint{5.166380in}{0.795559in}}%
\pgfpathlineto{\pgfqpoint{5.166676in}{0.795579in}}%
\pgfpathlineto{\pgfqpoint{5.166972in}{0.795596in}}%
\pgfpathlineto{\pgfqpoint{5.167268in}{0.795597in}}%
\pgfpathlineto{\pgfqpoint{5.167564in}{0.795596in}}%
\pgfpathlineto{\pgfqpoint{5.167860in}{0.795596in}}%
\pgfpathlineto{\pgfqpoint{5.168156in}{0.795595in}}%
\pgfpathlineto{\pgfqpoint{5.168452in}{0.795595in}}%
\pgfpathlineto{\pgfqpoint{5.168748in}{0.795595in}}%
\pgfpathlineto{\pgfqpoint{5.169044in}{0.795594in}}%
\pgfpathlineto{\pgfqpoint{5.169340in}{0.795594in}}%
\pgfpathlineto{\pgfqpoint{5.169636in}{0.795593in}}%
\pgfpathlineto{\pgfqpoint{5.169932in}{0.795593in}}%
\pgfpathlineto{\pgfqpoint{5.170228in}{0.795593in}}%
\pgfpathlineto{\pgfqpoint{5.170524in}{0.795592in}}%
\pgfpathlineto{\pgfqpoint{5.170820in}{0.795592in}}%
\pgfpathlineto{\pgfqpoint{5.171116in}{0.795591in}}%
\pgfpathlineto{\pgfqpoint{5.171412in}{0.795591in}}%
\pgfpathlineto{\pgfqpoint{5.171708in}{0.795591in}}%
\pgfpathlineto{\pgfqpoint{5.172004in}{0.795590in}}%
\pgfpathlineto{\pgfqpoint{5.172300in}{0.795590in}}%
\pgfpathlineto{\pgfqpoint{5.172596in}{0.795589in}}%
\pgfpathlineto{\pgfqpoint{5.172892in}{0.795589in}}%
\pgfpathlineto{\pgfqpoint{5.173188in}{0.795589in}}%
\pgfpathlineto{\pgfqpoint{5.173484in}{0.795588in}}%
\pgfpathlineto{\pgfqpoint{5.173780in}{0.795584in}}%
\pgfpathlineto{\pgfqpoint{5.174076in}{0.795537in}}%
\pgfpathlineto{\pgfqpoint{5.174372in}{0.795480in}}%
\pgfpathlineto{\pgfqpoint{5.174668in}{0.795422in}}%
\pgfpathlineto{\pgfqpoint{5.174964in}{0.795364in}}%
\pgfpathlineto{\pgfqpoint{5.175260in}{0.795306in}}%
\pgfpathlineto{\pgfqpoint{5.175556in}{0.795249in}}%
\pgfpathlineto{\pgfqpoint{5.175852in}{0.795191in}}%
\pgfpathlineto{\pgfqpoint{5.176148in}{0.795133in}}%
\pgfpathlineto{\pgfqpoint{5.176444in}{0.795075in}}%
\pgfpathlineto{\pgfqpoint{5.176740in}{0.795018in}}%
\pgfpathlineto{\pgfqpoint{5.177036in}{0.794960in}}%
\pgfpathlineto{\pgfqpoint{5.177332in}{0.794902in}}%
\pgfpathlineto{\pgfqpoint{5.177628in}{0.794844in}}%
\pgfpathlineto{\pgfqpoint{5.177924in}{0.794787in}}%
\pgfpathlineto{\pgfqpoint{5.178220in}{0.794729in}}%
\pgfpathlineto{\pgfqpoint{5.178516in}{0.794671in}}%
\pgfpathlineto{\pgfqpoint{5.178812in}{0.794613in}}%
\pgfpathlineto{\pgfqpoint{5.179108in}{0.794556in}}%
\pgfpathlineto{\pgfqpoint{5.179404in}{0.794498in}}%
\pgfpathlineto{\pgfqpoint{5.179700in}{0.794440in}}%
\pgfpathlineto{\pgfqpoint{5.179996in}{0.794382in}}%
\pgfpathlineto{\pgfqpoint{5.180292in}{0.794324in}}%
\pgfpathlineto{\pgfqpoint{5.180588in}{0.794267in}}%
\pgfpathlineto{\pgfqpoint{5.180884in}{0.794209in}}%
\pgfpathlineto{\pgfqpoint{5.181180in}{0.794151in}}%
\pgfpathlineto{\pgfqpoint{5.181476in}{0.794093in}}%
\pgfpathlineto{\pgfqpoint{5.181772in}{0.794036in}}%
\pgfpathlineto{\pgfqpoint{5.182068in}{0.793978in}}%
\pgfpathlineto{\pgfqpoint{5.182364in}{0.793920in}}%
\pgfpathlineto{\pgfqpoint{5.182660in}{0.793862in}}%
\pgfpathlineto{\pgfqpoint{5.182956in}{0.793805in}}%
\pgfpathlineto{\pgfqpoint{5.183252in}{0.793747in}}%
\pgfpathlineto{\pgfqpoint{5.183548in}{0.793689in}}%
\pgfpathlineto{\pgfqpoint{5.183844in}{0.793631in}}%
\pgfpathlineto{\pgfqpoint{5.184140in}{0.793574in}}%
\pgfpathlineto{\pgfqpoint{5.184436in}{0.793516in}}%
\pgfpathlineto{\pgfqpoint{5.184732in}{0.793458in}}%
\pgfpathlineto{\pgfqpoint{5.185028in}{0.793400in}}%
\pgfpathlineto{\pgfqpoint{5.185324in}{0.793343in}}%
\pgfpathlineto{\pgfqpoint{5.185620in}{0.793285in}}%
\pgfpathlineto{\pgfqpoint{5.185916in}{0.793227in}}%
\pgfpathlineto{\pgfqpoint{5.186212in}{0.793169in}}%
\pgfpathlineto{\pgfqpoint{5.186508in}{0.793112in}}%
\pgfpathlineto{\pgfqpoint{5.186804in}{0.793054in}}%
\pgfpathlineto{\pgfqpoint{5.187100in}{0.792996in}}%
\pgfpathlineto{\pgfqpoint{5.187396in}{0.792938in}}%
\pgfpathlineto{\pgfqpoint{5.187692in}{0.792881in}}%
\pgfpathlineto{\pgfqpoint{5.187988in}{0.792823in}}%
\pgfpathlineto{\pgfqpoint{5.188284in}{0.792765in}}%
\pgfpathlineto{\pgfqpoint{5.188580in}{0.792707in}}%
\pgfpathlineto{\pgfqpoint{5.188876in}{0.792650in}}%
\pgfpathlineto{\pgfqpoint{5.189172in}{0.792592in}}%
\pgfpathlineto{\pgfqpoint{5.189468in}{0.792534in}}%
\pgfpathlineto{\pgfqpoint{5.189764in}{0.792476in}}%
\pgfpathlineto{\pgfqpoint{5.190060in}{0.792419in}}%
\pgfpathlineto{\pgfqpoint{5.190356in}{0.792361in}}%
\pgfpathlineto{\pgfqpoint{5.190652in}{0.792303in}}%
\pgfpathlineto{\pgfqpoint{5.190948in}{0.792245in}}%
\pgfpathlineto{\pgfqpoint{5.191244in}{0.792188in}}%
\pgfpathlineto{\pgfqpoint{5.191540in}{0.792130in}}%
\pgfpathlineto{\pgfqpoint{5.191836in}{0.792072in}}%
\pgfpathlineto{\pgfqpoint{5.192132in}{0.792014in}}%
\pgfpathlineto{\pgfqpoint{5.192428in}{0.791957in}}%
\pgfpathlineto{\pgfqpoint{5.192724in}{0.791899in}}%
\pgfpathlineto{\pgfqpoint{5.193020in}{0.791841in}}%
\pgfpathlineto{\pgfqpoint{5.193316in}{0.791783in}}%
\pgfpathlineto{\pgfqpoint{5.193612in}{0.791725in}}%
\pgfpathlineto{\pgfqpoint{5.193908in}{0.791666in}}%
\pgfpathlineto{\pgfqpoint{5.194204in}{0.791607in}}%
\pgfpathlineto{\pgfqpoint{5.194500in}{0.795508in}}%
\pgfpathlineto{\pgfqpoint{5.194796in}{0.795921in}}%
\pgfpathlineto{\pgfqpoint{5.195092in}{0.796431in}}%
\pgfpathlineto{\pgfqpoint{5.195388in}{0.797281in}}%
\pgfpathlineto{\pgfqpoint{5.195684in}{0.797848in}}%
\pgfpathlineto{\pgfqpoint{5.195980in}{0.797960in}}%
\pgfpathlineto{\pgfqpoint{5.196276in}{0.797906in}}%
\pgfpathlineto{\pgfqpoint{5.196572in}{0.797851in}}%
\pgfpathlineto{\pgfqpoint{5.196868in}{0.797797in}}%
\pgfpathlineto{\pgfqpoint{5.197164in}{0.797742in}}%
\pgfpathlineto{\pgfqpoint{5.197460in}{0.797688in}}%
\pgfpathlineto{\pgfqpoint{5.197756in}{0.797633in}}%
\pgfpathlineto{\pgfqpoint{5.198052in}{0.797579in}}%
\pgfpathlineto{\pgfqpoint{5.198348in}{0.797524in}}%
\pgfpathlineto{\pgfqpoint{5.198644in}{0.797470in}}%
\pgfpathlineto{\pgfqpoint{5.198940in}{0.797415in}}%
\pgfpathlineto{\pgfqpoint{5.199236in}{0.797361in}}%
\pgfpathlineto{\pgfqpoint{5.199532in}{0.797964in}}%
\pgfpathlineto{\pgfqpoint{5.199828in}{0.804153in}}%
\pgfpathlineto{\pgfqpoint{5.200124in}{0.810937in}}%
\pgfpathlineto{\pgfqpoint{5.200420in}{0.811723in}}%
\pgfpathlineto{\pgfqpoint{5.200716in}{0.809852in}}%
\pgfpathlineto{\pgfqpoint{5.201012in}{0.807981in}}%
\pgfpathlineto{\pgfqpoint{5.201308in}{0.807492in}}%
\pgfpathlineto{\pgfqpoint{5.201604in}{0.812794in}}%
\pgfpathlineto{\pgfqpoint{5.201900in}{0.812985in}}%
\pgfpathlineto{\pgfqpoint{5.202196in}{0.812984in}}%
\pgfpathlineto{\pgfqpoint{5.202492in}{0.812984in}}%
\pgfpathlineto{\pgfqpoint{5.202788in}{0.812983in}}%
\pgfpathlineto{\pgfqpoint{5.203084in}{0.812982in}}%
\pgfpathlineto{\pgfqpoint{5.203380in}{0.812981in}}%
\pgfpathlineto{\pgfqpoint{5.203676in}{0.812980in}}%
\pgfpathlineto{\pgfqpoint{5.203972in}{0.812980in}}%
\pgfpathlineto{\pgfqpoint{5.204268in}{0.812979in}}%
\pgfpathlineto{\pgfqpoint{5.204564in}{0.812978in}}%
\pgfpathlineto{\pgfqpoint{5.204860in}{0.812977in}}%
\pgfpathlineto{\pgfqpoint{5.205156in}{0.812976in}}%
\pgfpathlineto{\pgfqpoint{5.205452in}{0.812976in}}%
\pgfpathlineto{\pgfqpoint{5.205748in}{0.812975in}}%
\pgfpathlineto{\pgfqpoint{5.206044in}{0.812974in}}%
\pgfpathlineto{\pgfqpoint{5.206340in}{0.812973in}}%
\pgfpathlineto{\pgfqpoint{5.206636in}{0.812973in}}%
\pgfpathlineto{\pgfqpoint{5.206932in}{0.812971in}}%
\pgfpathlineto{\pgfqpoint{5.207228in}{0.812971in}}%
\pgfpathlineto{\pgfqpoint{5.207524in}{0.812974in}}%
\pgfpathlineto{\pgfqpoint{5.207820in}{0.812978in}}%
\pgfpathlineto{\pgfqpoint{5.208116in}{0.812982in}}%
\pgfpathlineto{\pgfqpoint{5.208412in}{0.812985in}}%
\pgfpathlineto{\pgfqpoint{5.208708in}{0.812989in}}%
\pgfpathlineto{\pgfqpoint{5.209004in}{0.812992in}}%
\pgfpathlineto{\pgfqpoint{5.209300in}{0.812994in}}%
\pgfpathlineto{\pgfqpoint{5.209596in}{0.812993in}}%
\pgfpathlineto{\pgfqpoint{5.209892in}{0.812992in}}%
\pgfpathlineto{\pgfqpoint{5.210188in}{0.812991in}}%
\pgfpathlineto{\pgfqpoint{5.210484in}{0.812990in}}%
\pgfpathlineto{\pgfqpoint{5.210780in}{0.812990in}}%
\pgfpathlineto{\pgfqpoint{5.211076in}{0.812989in}}%
\pgfpathlineto{\pgfqpoint{5.211372in}{0.812988in}}%
\pgfpathlineto{\pgfqpoint{5.211668in}{0.812987in}}%
\pgfpathlineto{\pgfqpoint{5.211964in}{0.812986in}}%
\pgfpathlineto{\pgfqpoint{5.212260in}{0.812986in}}%
\pgfpathlineto{\pgfqpoint{5.212556in}{0.812985in}}%
\pgfpathlineto{\pgfqpoint{5.212852in}{0.812984in}}%
\pgfpathlineto{\pgfqpoint{5.213148in}{0.812983in}}%
\pgfpathlineto{\pgfqpoint{5.213444in}{0.812982in}}%
\pgfpathlineto{\pgfqpoint{5.213740in}{0.812982in}}%
\pgfpathlineto{\pgfqpoint{5.214036in}{0.812981in}}%
\pgfpathlineto{\pgfqpoint{5.214332in}{0.812980in}}%
\pgfpathlineto{\pgfqpoint{5.214628in}{0.812979in}}%
\pgfpathlineto{\pgfqpoint{5.214924in}{0.812978in}}%
\pgfpathlineto{\pgfqpoint{5.215220in}{0.812977in}}%
\pgfpathlineto{\pgfqpoint{5.215516in}{0.812977in}}%
\pgfpathlineto{\pgfqpoint{5.215812in}{0.812976in}}%
\pgfpathlineto{\pgfqpoint{5.216108in}{0.812975in}}%
\pgfpathlineto{\pgfqpoint{5.216404in}{0.812974in}}%
\pgfpathlineto{\pgfqpoint{5.216700in}{0.812973in}}%
\pgfpathlineto{\pgfqpoint{5.216996in}{0.812971in}}%
\pgfpathlineto{\pgfqpoint{5.217292in}{0.812970in}}%
\pgfpathlineto{\pgfqpoint{5.217588in}{0.812969in}}%
\pgfpathlineto{\pgfqpoint{5.217884in}{0.812968in}}%
\pgfpathlineto{\pgfqpoint{5.218180in}{0.812967in}}%
\pgfpathlineto{\pgfqpoint{5.218476in}{0.812966in}}%
\pgfpathlineto{\pgfqpoint{5.218772in}{0.812965in}}%
\pgfpathlineto{\pgfqpoint{5.219068in}{0.812964in}}%
\pgfpathlineto{\pgfqpoint{5.219364in}{0.812963in}}%
\pgfpathlineto{\pgfqpoint{5.219660in}{0.812961in}}%
\pgfpathlineto{\pgfqpoint{5.219956in}{0.812960in}}%
\pgfpathlineto{\pgfqpoint{5.220252in}{0.812959in}}%
\pgfpathlineto{\pgfqpoint{5.220548in}{0.812958in}}%
\pgfpathlineto{\pgfqpoint{5.220844in}{0.812957in}}%
\pgfpathlineto{\pgfqpoint{5.221140in}{0.812956in}}%
\pgfpathlineto{\pgfqpoint{5.221436in}{0.812955in}}%
\pgfpathlineto{\pgfqpoint{5.221732in}{0.812954in}}%
\pgfpathlineto{\pgfqpoint{5.222028in}{0.812952in}}%
\pgfpathlineto{\pgfqpoint{5.222324in}{0.812951in}}%
\pgfpathlineto{\pgfqpoint{5.222620in}{0.812950in}}%
\pgfpathlineto{\pgfqpoint{5.222916in}{0.812949in}}%
\pgfpathlineto{\pgfqpoint{5.223212in}{0.812948in}}%
\pgfpathlineto{\pgfqpoint{5.223508in}{0.812947in}}%
\pgfpathlineto{\pgfqpoint{5.223804in}{0.812946in}}%
\pgfpathlineto{\pgfqpoint{5.224100in}{0.812945in}}%
\pgfpathlineto{\pgfqpoint{5.224396in}{0.812944in}}%
\pgfpathlineto{\pgfqpoint{5.224692in}{0.812942in}}%
\pgfpathlineto{\pgfqpoint{5.224988in}{0.812941in}}%
\pgfpathlineto{\pgfqpoint{5.225284in}{0.812940in}}%
\pgfpathlineto{\pgfqpoint{5.225580in}{0.812939in}}%
\pgfpathlineto{\pgfqpoint{5.225876in}{0.812938in}}%
\pgfpathlineto{\pgfqpoint{5.226172in}{0.812937in}}%
\pgfpathlineto{\pgfqpoint{5.226468in}{0.812936in}}%
\pgfpathlineto{\pgfqpoint{5.226764in}{0.812935in}}%
\pgfpathlineto{\pgfqpoint{5.227060in}{0.812934in}}%
\pgfpathlineto{\pgfqpoint{5.227356in}{0.812932in}}%
\pgfpathlineto{\pgfqpoint{5.227652in}{0.812931in}}%
\pgfpathlineto{\pgfqpoint{5.227948in}{0.812930in}}%
\pgfpathlineto{\pgfqpoint{5.228244in}{0.812929in}}%
\pgfpathlineto{\pgfqpoint{5.228540in}{0.812928in}}%
\pgfpathlineto{\pgfqpoint{5.228836in}{0.812927in}}%
\pgfpathlineto{\pgfqpoint{5.229132in}{0.812926in}}%
\pgfpathlineto{\pgfqpoint{5.229428in}{0.812925in}}%
\pgfpathlineto{\pgfqpoint{5.229724in}{0.812923in}}%
\pgfpathlineto{\pgfqpoint{5.230020in}{0.812922in}}%
\pgfpathlineto{\pgfqpoint{5.230316in}{0.812921in}}%
\pgfpathlineto{\pgfqpoint{5.230612in}{0.812920in}}%
\pgfpathlineto{\pgfqpoint{5.230908in}{0.812919in}}%
\pgfpathlineto{\pgfqpoint{5.231204in}{0.812918in}}%
\pgfpathlineto{\pgfqpoint{5.231500in}{0.812917in}}%
\pgfpathlineto{\pgfqpoint{5.231796in}{0.812916in}}%
\pgfpathlineto{\pgfqpoint{5.232092in}{0.812915in}}%
\pgfpathlineto{\pgfqpoint{5.232389in}{0.812913in}}%
\pgfpathlineto{\pgfqpoint{5.232685in}{0.812912in}}%
\pgfpathlineto{\pgfqpoint{5.232981in}{0.812911in}}%
\pgfpathlineto{\pgfqpoint{5.233277in}{0.812910in}}%
\pgfpathlineto{\pgfqpoint{5.233573in}{0.812909in}}%
\pgfpathlineto{\pgfqpoint{5.233869in}{0.812908in}}%
\pgfpathlineto{\pgfqpoint{5.234165in}{0.812907in}}%
\pgfpathlineto{\pgfqpoint{5.234461in}{0.812906in}}%
\pgfpathlineto{\pgfqpoint{5.234757in}{0.812904in}}%
\pgfpathlineto{\pgfqpoint{5.235053in}{0.812903in}}%
\pgfpathlineto{\pgfqpoint{5.235349in}{0.812902in}}%
\pgfpathlineto{\pgfqpoint{5.235645in}{0.812901in}}%
\pgfpathlineto{\pgfqpoint{5.235941in}{0.812900in}}%
\pgfpathlineto{\pgfqpoint{5.236237in}{0.812899in}}%
\pgfpathlineto{\pgfqpoint{5.236533in}{0.812898in}}%
\pgfpathlineto{\pgfqpoint{5.236829in}{0.812897in}}%
\pgfpathlineto{\pgfqpoint{5.237125in}{0.812896in}}%
\pgfpathlineto{\pgfqpoint{5.237421in}{0.812894in}}%
\pgfpathlineto{\pgfqpoint{5.237717in}{0.812893in}}%
\pgfpathlineto{\pgfqpoint{5.238013in}{0.812892in}}%
\pgfpathlineto{\pgfqpoint{5.238309in}{0.812891in}}%
\pgfpathlineto{\pgfqpoint{5.238605in}{0.812890in}}%
\pgfpathlineto{\pgfqpoint{5.238901in}{0.812889in}}%
\pgfpathlineto{\pgfqpoint{5.239197in}{0.812888in}}%
\pgfpathlineto{\pgfqpoint{5.239493in}{0.812887in}}%
\pgfpathlineto{\pgfqpoint{5.239789in}{0.812885in}}%
\pgfpathlineto{\pgfqpoint{5.240085in}{0.812884in}}%
\pgfpathlineto{\pgfqpoint{5.240381in}{0.812883in}}%
\pgfpathlineto{\pgfqpoint{5.240677in}{0.812882in}}%
\pgfpathlineto{\pgfqpoint{5.240973in}{0.812881in}}%
\pgfpathlineto{\pgfqpoint{5.241269in}{0.812880in}}%
\pgfpathlineto{\pgfqpoint{5.241565in}{0.812879in}}%
\pgfpathlineto{\pgfqpoint{5.241861in}{0.812878in}}%
\pgfpathlineto{\pgfqpoint{5.242157in}{0.812877in}}%
\pgfpathlineto{\pgfqpoint{5.242453in}{0.812875in}}%
\pgfpathlineto{\pgfqpoint{5.242749in}{0.812874in}}%
\pgfpathlineto{\pgfqpoint{5.243045in}{0.812873in}}%
\pgfpathlineto{\pgfqpoint{5.243341in}{0.812872in}}%
\pgfpathlineto{\pgfqpoint{5.243637in}{0.812871in}}%
\pgfpathlineto{\pgfqpoint{5.243933in}{0.812870in}}%
\pgfpathlineto{\pgfqpoint{5.244229in}{0.812869in}}%
\pgfpathlineto{\pgfqpoint{5.244525in}{0.812868in}}%
\pgfpathlineto{\pgfqpoint{5.244821in}{0.812867in}}%
\pgfpathlineto{\pgfqpoint{5.245117in}{0.812865in}}%
\pgfpathlineto{\pgfqpoint{5.245413in}{0.812864in}}%
\pgfpathlineto{\pgfqpoint{5.245709in}{0.812863in}}%
\pgfpathlineto{\pgfqpoint{5.246005in}{0.812862in}}%
\pgfpathlineto{\pgfqpoint{5.246301in}{0.812861in}}%
\pgfpathlineto{\pgfqpoint{5.246597in}{0.812860in}}%
\pgfpathlineto{\pgfqpoint{5.246893in}{0.812859in}}%
\pgfpathlineto{\pgfqpoint{5.247189in}{0.812858in}}%
\pgfpathlineto{\pgfqpoint{5.247485in}{0.812856in}}%
\pgfpathlineto{\pgfqpoint{5.247781in}{0.812855in}}%
\pgfpathlineto{\pgfqpoint{5.248077in}{0.812854in}}%
\pgfpathlineto{\pgfqpoint{5.248373in}{0.812853in}}%
\pgfpathlineto{\pgfqpoint{5.248669in}{0.812852in}}%
\pgfpathlineto{\pgfqpoint{5.248965in}{0.812851in}}%
\pgfpathlineto{\pgfqpoint{5.249261in}{0.812850in}}%
\pgfpathlineto{\pgfqpoint{5.249557in}{0.812849in}}%
\pgfpathlineto{\pgfqpoint{5.249853in}{0.812848in}}%
\pgfpathlineto{\pgfqpoint{5.250149in}{0.812846in}}%
\pgfpathlineto{\pgfqpoint{5.250445in}{0.812845in}}%
\pgfpathlineto{\pgfqpoint{5.250741in}{0.812844in}}%
\pgfpathlineto{\pgfqpoint{5.251037in}{0.812843in}}%
\pgfpathlineto{\pgfqpoint{5.251333in}{0.812842in}}%
\pgfpathlineto{\pgfqpoint{5.251629in}{0.812841in}}%
\pgfpathlineto{\pgfqpoint{5.251925in}{0.812840in}}%
\pgfpathlineto{\pgfqpoint{5.252221in}{0.812839in}}%
\pgfpathlineto{\pgfqpoint{5.252517in}{0.812837in}}%
\pgfpathlineto{\pgfqpoint{5.252813in}{0.812836in}}%
\pgfpathlineto{\pgfqpoint{5.253109in}{0.812835in}}%
\pgfpathlineto{\pgfqpoint{5.253405in}{0.812834in}}%
\pgfpathlineto{\pgfqpoint{5.253701in}{0.812833in}}%
\pgfpathlineto{\pgfqpoint{5.253997in}{0.812832in}}%
\pgfpathlineto{\pgfqpoint{5.254293in}{0.812831in}}%
\pgfpathlineto{\pgfqpoint{5.254589in}{0.812830in}}%
\pgfpathlineto{\pgfqpoint{5.254885in}{0.812829in}}%
\pgfpathlineto{\pgfqpoint{5.255181in}{0.812827in}}%
\pgfpathlineto{\pgfqpoint{5.255477in}{0.812826in}}%
\pgfpathlineto{\pgfqpoint{5.255773in}{0.812825in}}%
\pgfpathlineto{\pgfqpoint{5.256069in}{0.812824in}}%
\pgfpathlineto{\pgfqpoint{5.256365in}{0.812823in}}%
\pgfpathlineto{\pgfqpoint{5.256661in}{0.812822in}}%
\pgfpathlineto{\pgfqpoint{5.256957in}{0.812823in}}%
\pgfpathlineto{\pgfqpoint{5.257253in}{0.812824in}}%
\pgfpathlineto{\pgfqpoint{5.257549in}{0.812825in}}%
\pgfpathlineto{\pgfqpoint{5.257845in}{0.812827in}}%
\pgfpathlineto{\pgfqpoint{5.258141in}{0.812833in}}%
\pgfpathlineto{\pgfqpoint{5.258437in}{0.812837in}}%
\pgfpathlineto{\pgfqpoint{5.258733in}{0.812835in}}%
\pgfpathlineto{\pgfqpoint{5.259029in}{0.812832in}}%
\pgfpathlineto{\pgfqpoint{5.259325in}{0.812829in}}%
\pgfpathlineto{\pgfqpoint{5.259621in}{0.812827in}}%
\pgfpathlineto{\pgfqpoint{5.259917in}{0.812824in}}%
\pgfpathlineto{\pgfqpoint{5.260213in}{0.812822in}}%
\pgfpathlineto{\pgfqpoint{5.260509in}{0.812819in}}%
\pgfpathlineto{\pgfqpoint{5.260805in}{0.812816in}}%
\pgfpathlineto{\pgfqpoint{5.261101in}{0.812814in}}%
\pgfpathlineto{\pgfqpoint{5.261397in}{0.812811in}}%
\pgfpathlineto{\pgfqpoint{5.261693in}{0.812808in}}%
\pgfpathlineto{\pgfqpoint{5.261989in}{0.812806in}}%
\pgfpathlineto{\pgfqpoint{5.262285in}{0.812803in}}%
\pgfpathlineto{\pgfqpoint{5.262581in}{0.812801in}}%
\pgfpathlineto{\pgfqpoint{5.262877in}{0.812798in}}%
\pgfpathlineto{\pgfqpoint{5.263173in}{0.812795in}}%
\pgfpathlineto{\pgfqpoint{5.263469in}{0.812793in}}%
\pgfpathlineto{\pgfqpoint{5.263765in}{0.812790in}}%
\pgfpathlineto{\pgfqpoint{5.264061in}{0.812813in}}%
\pgfpathlineto{\pgfqpoint{5.264357in}{0.812789in}}%
\pgfpathlineto{\pgfqpoint{5.264653in}{0.812786in}}%
\pgfpathlineto{\pgfqpoint{5.264949in}{0.812788in}}%
\pgfpathlineto{\pgfqpoint{5.265245in}{0.812794in}}%
\pgfpathlineto{\pgfqpoint{5.265541in}{0.812800in}}%
\pgfpathlineto{\pgfqpoint{5.265837in}{0.812806in}}%
\pgfpathlineto{\pgfqpoint{5.266133in}{0.812812in}}%
\pgfpathlineto{\pgfqpoint{5.266429in}{0.812818in}}%
\pgfpathlineto{\pgfqpoint{5.266725in}{0.812822in}}%
\pgfpathlineto{\pgfqpoint{5.267021in}{0.812820in}}%
\pgfpathlineto{\pgfqpoint{5.267317in}{0.812818in}}%
\pgfpathlineto{\pgfqpoint{5.267613in}{0.812816in}}%
\pgfpathlineto{\pgfqpoint{5.267909in}{0.812814in}}%
\pgfpathlineto{\pgfqpoint{5.268205in}{0.812812in}}%
\pgfpathlineto{\pgfqpoint{5.268501in}{0.812810in}}%
\pgfpathlineto{\pgfqpoint{5.268797in}{0.812808in}}%
\pgfpathlineto{\pgfqpoint{5.269093in}{0.812806in}}%
\pgfpathlineto{\pgfqpoint{5.269389in}{0.812804in}}%
\pgfpathlineto{\pgfqpoint{5.269685in}{0.812802in}}%
\pgfpathlineto{\pgfqpoint{5.269981in}{0.812800in}}%
\pgfpathlineto{\pgfqpoint{5.270277in}{0.812798in}}%
\pgfpathlineto{\pgfqpoint{5.270573in}{0.812796in}}%
\pgfpathlineto{\pgfqpoint{5.270869in}{0.812794in}}%
\pgfpathlineto{\pgfqpoint{5.271165in}{0.812792in}}%
\pgfpathlineto{\pgfqpoint{5.271461in}{0.812790in}}%
\pgfpathlineto{\pgfqpoint{5.271757in}{0.812788in}}%
\pgfpathlineto{\pgfqpoint{5.272053in}{0.812786in}}%
\pgfpathlineto{\pgfqpoint{5.272349in}{0.812784in}}%
\pgfpathlineto{\pgfqpoint{5.272645in}{0.812782in}}%
\pgfpathlineto{\pgfqpoint{5.272941in}{0.812781in}}%
\pgfpathlineto{\pgfqpoint{5.273237in}{0.812779in}}%
\pgfpathlineto{\pgfqpoint{5.273533in}{0.812777in}}%
\pgfpathlineto{\pgfqpoint{5.273829in}{0.812775in}}%
\pgfpathlineto{\pgfqpoint{5.274125in}{0.812773in}}%
\pgfpathlineto{\pgfqpoint{5.274421in}{0.812771in}}%
\pgfpathlineto{\pgfqpoint{5.274717in}{0.812769in}}%
\pgfpathlineto{\pgfqpoint{5.275013in}{0.812767in}}%
\pgfpathlineto{\pgfqpoint{5.275309in}{0.812765in}}%
\pgfpathlineto{\pgfqpoint{5.275605in}{0.812763in}}%
\pgfpathlineto{\pgfqpoint{5.275901in}{0.812761in}}%
\pgfpathlineto{\pgfqpoint{5.276197in}{0.812759in}}%
\pgfpathlineto{\pgfqpoint{5.276493in}{0.812757in}}%
\pgfpathlineto{\pgfqpoint{5.276789in}{0.812755in}}%
\pgfpathlineto{\pgfqpoint{5.277085in}{0.812753in}}%
\pgfpathlineto{\pgfqpoint{5.277381in}{0.812751in}}%
\pgfpathlineto{\pgfqpoint{5.277677in}{0.812749in}}%
\pgfpathlineto{\pgfqpoint{5.277973in}{0.812747in}}%
\pgfpathlineto{\pgfqpoint{5.278269in}{0.812745in}}%
\pgfpathlineto{\pgfqpoint{5.278565in}{0.812743in}}%
\pgfpathlineto{\pgfqpoint{5.278861in}{0.812741in}}%
\pgfpathlineto{\pgfqpoint{5.279157in}{0.812739in}}%
\pgfpathlineto{\pgfqpoint{5.279453in}{0.812737in}}%
\pgfpathlineto{\pgfqpoint{5.279749in}{0.812735in}}%
\pgfpathlineto{\pgfqpoint{5.280045in}{0.812733in}}%
\pgfpathlineto{\pgfqpoint{5.280341in}{0.812731in}}%
\pgfpathlineto{\pgfqpoint{5.280637in}{0.812729in}}%
\pgfpathlineto{\pgfqpoint{5.280933in}{0.812727in}}%
\pgfpathlineto{\pgfqpoint{5.281229in}{0.812725in}}%
\pgfpathlineto{\pgfqpoint{5.281525in}{0.812723in}}%
\pgfpathlineto{\pgfqpoint{5.281821in}{0.812721in}}%
\pgfpathlineto{\pgfqpoint{5.282117in}{0.812720in}}%
\pgfpathlineto{\pgfqpoint{5.282413in}{0.812718in}}%
\pgfpathlineto{\pgfqpoint{5.282709in}{0.812716in}}%
\pgfpathlineto{\pgfqpoint{5.283005in}{0.812714in}}%
\pgfpathlineto{\pgfqpoint{5.283301in}{0.812712in}}%
\pgfpathlineto{\pgfqpoint{5.283597in}{0.812710in}}%
\pgfpathlineto{\pgfqpoint{5.283893in}{0.812708in}}%
\pgfpathlineto{\pgfqpoint{5.284189in}{0.812706in}}%
\pgfpathlineto{\pgfqpoint{5.284485in}{0.812704in}}%
\pgfpathlineto{\pgfqpoint{5.284781in}{0.812702in}}%
\pgfpathlineto{\pgfqpoint{5.285077in}{0.812700in}}%
\pgfpathlineto{\pgfqpoint{5.285373in}{0.812698in}}%
\pgfpathlineto{\pgfqpoint{5.285669in}{0.812696in}}%
\pgfpathlineto{\pgfqpoint{5.285965in}{0.812694in}}%
\pgfpathlineto{\pgfqpoint{5.286261in}{0.812692in}}%
\pgfpathlineto{\pgfqpoint{5.286557in}{0.812690in}}%
\pgfpathlineto{\pgfqpoint{5.286853in}{0.812688in}}%
\pgfpathlineto{\pgfqpoint{5.287149in}{0.812686in}}%
\pgfpathlineto{\pgfqpoint{5.287445in}{0.812684in}}%
\pgfpathlineto{\pgfqpoint{5.287741in}{0.812682in}}%
\pgfpathlineto{\pgfqpoint{5.288037in}{0.812680in}}%
\pgfpathlineto{\pgfqpoint{5.288333in}{0.812678in}}%
\pgfpathlineto{\pgfqpoint{5.288629in}{0.812676in}}%
\pgfpathlineto{\pgfqpoint{5.288925in}{0.812674in}}%
\pgfpathlineto{\pgfqpoint{5.289221in}{0.812672in}}%
\pgfpathlineto{\pgfqpoint{5.289517in}{0.812670in}}%
\pgfpathlineto{\pgfqpoint{5.289813in}{0.812668in}}%
\pgfpathlineto{\pgfqpoint{5.290109in}{0.812666in}}%
\pgfpathlineto{\pgfqpoint{5.290405in}{0.812664in}}%
\pgfpathlineto{\pgfqpoint{5.290701in}{0.812662in}}%
\pgfpathlineto{\pgfqpoint{5.290997in}{0.812660in}}%
\pgfpathlineto{\pgfqpoint{5.291293in}{0.812659in}}%
\pgfpathlineto{\pgfqpoint{5.291589in}{0.812657in}}%
\pgfpathlineto{\pgfqpoint{5.291885in}{0.812655in}}%
\pgfpathlineto{\pgfqpoint{5.292181in}{0.812653in}}%
\pgfpathlineto{\pgfqpoint{5.292477in}{0.812651in}}%
\pgfpathlineto{\pgfqpoint{5.292773in}{0.812649in}}%
\pgfpathlineto{\pgfqpoint{5.293069in}{0.812647in}}%
\pgfpathlineto{\pgfqpoint{5.293365in}{0.812645in}}%
\pgfpathlineto{\pgfqpoint{5.293661in}{0.812643in}}%
\pgfpathlineto{\pgfqpoint{5.293957in}{0.812641in}}%
\pgfpathlineto{\pgfqpoint{5.294253in}{0.812639in}}%
\pgfpathlineto{\pgfqpoint{5.294549in}{0.812637in}}%
\pgfpathlineto{\pgfqpoint{5.294845in}{0.812635in}}%
\pgfpathlineto{\pgfqpoint{5.295141in}{0.812633in}}%
\pgfpathlineto{\pgfqpoint{5.295437in}{0.812631in}}%
\pgfpathlineto{\pgfqpoint{5.295733in}{0.812629in}}%
\pgfpathlineto{\pgfqpoint{5.296029in}{0.812627in}}%
\pgfpathlineto{\pgfqpoint{5.296325in}{0.812625in}}%
\pgfpathlineto{\pgfqpoint{5.296621in}{0.812623in}}%
\pgfpathlineto{\pgfqpoint{5.296917in}{0.812621in}}%
\pgfpathlineto{\pgfqpoint{5.297213in}{0.812619in}}%
\pgfpathlineto{\pgfqpoint{5.297509in}{0.812617in}}%
\pgfpathlineto{\pgfqpoint{5.297805in}{0.812615in}}%
\pgfpathlineto{\pgfqpoint{5.298101in}{0.812613in}}%
\pgfpathlineto{\pgfqpoint{5.298397in}{0.812611in}}%
\pgfpathlineto{\pgfqpoint{5.298693in}{0.812609in}}%
\pgfpathlineto{\pgfqpoint{5.298989in}{0.812609in}}%
\pgfpathlineto{\pgfqpoint{5.299285in}{0.812610in}}%
\pgfpathlineto{\pgfqpoint{5.299582in}{0.812612in}}%
\pgfpathlineto{\pgfqpoint{5.299878in}{0.812613in}}%
\pgfpathlineto{\pgfqpoint{5.300174in}{0.812615in}}%
\pgfpathlineto{\pgfqpoint{5.300470in}{0.812617in}}%
\pgfpathlineto{\pgfqpoint{5.300766in}{0.812618in}}%
\pgfpathlineto{\pgfqpoint{5.301062in}{0.812616in}}%
\pgfpathlineto{\pgfqpoint{5.301358in}{0.812621in}}%
\pgfpathlineto{\pgfqpoint{5.301654in}{0.812623in}}%
\pgfpathlineto{\pgfqpoint{5.301950in}{0.811843in}}%
\pgfpathlineto{\pgfqpoint{5.302246in}{0.811047in}}%
\pgfpathlineto{\pgfqpoint{5.302542in}{0.811047in}}%
\pgfpathlineto{\pgfqpoint{5.302838in}{0.811046in}}%
\pgfpathlineto{\pgfqpoint{5.303134in}{0.811046in}}%
\pgfpathlineto{\pgfqpoint{5.303430in}{0.811046in}}%
\pgfpathlineto{\pgfqpoint{5.303726in}{0.811045in}}%
\pgfpathlineto{\pgfqpoint{5.304022in}{0.811045in}}%
\pgfpathlineto{\pgfqpoint{5.304318in}{0.811044in}}%
\pgfpathlineto{\pgfqpoint{5.304614in}{0.811044in}}%
\pgfpathlineto{\pgfqpoint{5.304910in}{0.811044in}}%
\pgfpathlineto{\pgfqpoint{5.305206in}{0.811043in}}%
\pgfpathlineto{\pgfqpoint{5.305502in}{0.811043in}}%
\pgfpathlineto{\pgfqpoint{5.305798in}{0.811042in}}%
\pgfpathlineto{\pgfqpoint{5.306094in}{0.811042in}}%
\pgfpathlineto{\pgfqpoint{5.306390in}{0.811042in}}%
\pgfpathlineto{\pgfqpoint{5.306686in}{0.811041in}}%
\pgfpathlineto{\pgfqpoint{5.306982in}{0.810985in}}%
\pgfpathlineto{\pgfqpoint{5.307278in}{0.810625in}}%
\pgfpathlineto{\pgfqpoint{5.307574in}{0.810225in}}%
\pgfpathlineto{\pgfqpoint{5.307870in}{0.809866in}}%
\pgfpathlineto{\pgfqpoint{5.308166in}{0.809828in}}%
\pgfpathlineto{\pgfqpoint{5.308462in}{0.809885in}}%
\pgfpathlineto{\pgfqpoint{5.308758in}{0.809887in}}%
\pgfpathlineto{\pgfqpoint{5.309054in}{0.809884in}}%
\pgfpathlineto{\pgfqpoint{5.309350in}{0.809880in}}%
\pgfpathlineto{\pgfqpoint{5.309646in}{0.809875in}}%
\pgfpathlineto{\pgfqpoint{5.309942in}{0.809871in}}%
\pgfpathlineto{\pgfqpoint{5.310238in}{0.809867in}}%
\pgfpathlineto{\pgfqpoint{5.310534in}{0.809863in}}%
\pgfpathlineto{\pgfqpoint{5.310830in}{0.809859in}}%
\pgfpathlineto{\pgfqpoint{5.311126in}{0.809854in}}%
\pgfpathlineto{\pgfqpoint{5.311422in}{0.809850in}}%
\pgfpathlineto{\pgfqpoint{5.311718in}{0.809846in}}%
\pgfpathlineto{\pgfqpoint{5.312014in}{0.809842in}}%
\pgfpathlineto{\pgfqpoint{5.312310in}{0.809838in}}%
\pgfpathlineto{\pgfqpoint{5.312606in}{0.809833in}}%
\pgfpathlineto{\pgfqpoint{5.312902in}{0.809829in}}%
\pgfpathlineto{\pgfqpoint{5.313198in}{0.809825in}}%
\pgfpathlineto{\pgfqpoint{5.313494in}{0.809821in}}%
\pgfpathlineto{\pgfqpoint{5.313790in}{0.809812in}}%
\pgfpathlineto{\pgfqpoint{5.314086in}{0.808493in}}%
\pgfpathlineto{\pgfqpoint{5.314382in}{0.807611in}}%
\pgfpathlineto{\pgfqpoint{5.314678in}{0.807465in}}%
\pgfpathlineto{\pgfqpoint{5.314974in}{0.807319in}}%
\pgfpathlineto{\pgfqpoint{5.315270in}{0.807163in}}%
\pgfpathlineto{\pgfqpoint{5.315566in}{0.806934in}}%
\pgfpathlineto{\pgfqpoint{5.315862in}{0.806721in}}%
\pgfpathlineto{\pgfqpoint{5.316158in}{0.806671in}}%
\pgfpathlineto{\pgfqpoint{5.316454in}{0.806682in}}%
\pgfpathlineto{\pgfqpoint{5.316750in}{0.806693in}}%
\pgfpathlineto{\pgfqpoint{5.317046in}{0.806703in}}%
\pgfpathlineto{\pgfqpoint{5.317342in}{0.806714in}}%
\pgfpathlineto{\pgfqpoint{5.317638in}{0.806725in}}%
\pgfpathlineto{\pgfqpoint{5.317934in}{0.806736in}}%
\pgfpathlineto{\pgfqpoint{5.318230in}{0.806747in}}%
\pgfpathlineto{\pgfqpoint{5.318526in}{0.806758in}}%
\pgfpathlineto{\pgfqpoint{5.318822in}{0.806769in}}%
\pgfpathlineto{\pgfqpoint{5.319118in}{0.806780in}}%
\pgfpathlineto{\pgfqpoint{5.319414in}{0.806791in}}%
\pgfpathlineto{\pgfqpoint{5.319710in}{0.806802in}}%
\pgfpathlineto{\pgfqpoint{5.320006in}{0.806813in}}%
\pgfpathlineto{\pgfqpoint{5.320302in}{0.806824in}}%
\pgfpathlineto{\pgfqpoint{5.320598in}{0.806835in}}%
\pgfpathlineto{\pgfqpoint{5.320894in}{0.806846in}}%
\pgfpathlineto{\pgfqpoint{5.321190in}{0.806856in}}%
\pgfpathlineto{\pgfqpoint{5.321486in}{0.806867in}}%
\pgfpathlineto{\pgfqpoint{5.321782in}{0.806878in}}%
\pgfpathlineto{\pgfqpoint{5.322078in}{0.806656in}}%
\pgfpathlineto{\pgfqpoint{5.322374in}{0.806783in}}%
\pgfpathlineto{\pgfqpoint{5.322670in}{0.806917in}}%
\pgfpathlineto{\pgfqpoint{5.322966in}{0.806915in}}%
\pgfpathlineto{\pgfqpoint{5.323262in}{0.806936in}}%
\pgfpathlineto{\pgfqpoint{5.323558in}{0.806922in}}%
\pgfpathlineto{\pgfqpoint{5.323854in}{0.806908in}}%
\pgfpathlineto{\pgfqpoint{5.324150in}{0.806895in}}%
\pgfpathlineto{\pgfqpoint{5.324446in}{0.806881in}}%
\pgfpathlineto{\pgfqpoint{5.324742in}{0.806867in}}%
\pgfpathlineto{\pgfqpoint{5.325038in}{0.806853in}}%
\pgfpathlineto{\pgfqpoint{5.325334in}{0.806839in}}%
\pgfpathlineto{\pgfqpoint{5.325630in}{0.806825in}}%
\pgfpathlineto{\pgfqpoint{5.325926in}{0.806811in}}%
\pgfpathlineto{\pgfqpoint{5.326222in}{0.806797in}}%
\pgfpathlineto{\pgfqpoint{5.326518in}{0.806784in}}%
\pgfpathlineto{\pgfqpoint{5.326814in}{0.806770in}}%
\pgfpathlineto{\pgfqpoint{5.327110in}{0.806756in}}%
\pgfpathlineto{\pgfqpoint{5.327406in}{0.806742in}}%
\pgfpathlineto{\pgfqpoint{5.327702in}{0.806728in}}%
\pgfpathlineto{\pgfqpoint{5.327998in}{0.806714in}}%
\pgfpathlineto{\pgfqpoint{5.328294in}{0.806700in}}%
\pgfpathlineto{\pgfqpoint{5.328590in}{0.806686in}}%
\pgfpathlineto{\pgfqpoint{5.328886in}{0.806673in}}%
\pgfpathlineto{\pgfqpoint{5.329182in}{0.806659in}}%
\pgfpathlineto{\pgfqpoint{5.329478in}{0.806645in}}%
\pgfpathlineto{\pgfqpoint{5.329774in}{0.806631in}}%
\pgfpathlineto{\pgfqpoint{5.330070in}{0.806617in}}%
\pgfpathlineto{\pgfqpoint{5.330366in}{0.806603in}}%
\pgfpathlineto{\pgfqpoint{5.330662in}{0.806589in}}%
\pgfpathlineto{\pgfqpoint{5.330958in}{0.806575in}}%
\pgfpathlineto{\pgfqpoint{5.331254in}{0.806562in}}%
\pgfpathlineto{\pgfqpoint{5.331550in}{0.806548in}}%
\pgfpathlineto{\pgfqpoint{5.331846in}{0.806534in}}%
\pgfpathlineto{\pgfqpoint{5.332142in}{0.806520in}}%
\pgfpathlineto{\pgfqpoint{5.332438in}{0.806506in}}%
\pgfpathlineto{\pgfqpoint{5.332734in}{0.806492in}}%
\pgfpathlineto{\pgfqpoint{5.333030in}{0.806478in}}%
\pgfpathlineto{\pgfqpoint{5.333326in}{0.806464in}}%
\pgfpathlineto{\pgfqpoint{5.333622in}{0.806450in}}%
\pgfpathlineto{\pgfqpoint{5.333918in}{0.806437in}}%
\pgfpathlineto{\pgfqpoint{5.334214in}{0.806423in}}%
\pgfpathlineto{\pgfqpoint{5.334510in}{0.806409in}}%
\pgfpathlineto{\pgfqpoint{5.334806in}{0.806395in}}%
\pgfpathlineto{\pgfqpoint{5.335102in}{0.806381in}}%
\pgfpathlineto{\pgfqpoint{5.335398in}{0.806367in}}%
\pgfpathlineto{\pgfqpoint{5.335694in}{0.806353in}}%
\pgfpathlineto{\pgfqpoint{5.335990in}{0.806339in}}%
\pgfpathlineto{\pgfqpoint{5.336286in}{0.806326in}}%
\pgfpathlineto{\pgfqpoint{5.336582in}{0.806312in}}%
\pgfpathlineto{\pgfqpoint{5.336878in}{0.806298in}}%
\pgfpathlineto{\pgfqpoint{5.337174in}{0.806284in}}%
\pgfpathlineto{\pgfqpoint{5.337470in}{0.806270in}}%
\pgfpathlineto{\pgfqpoint{5.337766in}{0.806256in}}%
\pgfpathlineto{\pgfqpoint{5.338062in}{0.806242in}}%
\pgfpathlineto{\pgfqpoint{5.338358in}{0.806228in}}%
\pgfpathlineto{\pgfqpoint{5.338654in}{0.806215in}}%
\pgfpathlineto{\pgfqpoint{5.338950in}{0.806201in}}%
\pgfpathlineto{\pgfqpoint{5.339246in}{0.806187in}}%
\pgfpathlineto{\pgfqpoint{5.339542in}{0.806173in}}%
\pgfpathlineto{\pgfqpoint{5.339838in}{0.806159in}}%
\pgfpathlineto{\pgfqpoint{5.340134in}{0.806145in}}%
\pgfpathlineto{\pgfqpoint{5.340430in}{0.806131in}}%
\pgfpathlineto{\pgfqpoint{5.340726in}{0.806117in}}%
\pgfpathlineto{\pgfqpoint{5.341022in}{0.806104in}}%
\pgfpathlineto{\pgfqpoint{5.341318in}{0.806090in}}%
\pgfpathlineto{\pgfqpoint{5.341614in}{0.806076in}}%
\pgfpathlineto{\pgfqpoint{5.341910in}{0.806062in}}%
\pgfpathlineto{\pgfqpoint{5.342206in}{0.806048in}}%
\pgfpathlineto{\pgfqpoint{5.342502in}{0.806034in}}%
\pgfpathlineto{\pgfqpoint{5.342798in}{0.806020in}}%
\pgfpathlineto{\pgfqpoint{5.343094in}{0.806006in}}%
\pgfpathlineto{\pgfqpoint{5.343390in}{0.805993in}}%
\pgfpathlineto{\pgfqpoint{5.343686in}{0.806542in}}%
\pgfpathlineto{\pgfqpoint{5.343982in}{0.806618in}}%
\pgfpathlineto{\pgfqpoint{5.344278in}{0.806594in}}%
\pgfpathlineto{\pgfqpoint{5.344574in}{0.806570in}}%
\pgfpathlineto{\pgfqpoint{5.344870in}{0.806563in}}%
\pgfpathlineto{\pgfqpoint{5.345166in}{0.806564in}}%
\pgfpathlineto{\pgfqpoint{5.345462in}{0.806565in}}%
\pgfpathlineto{\pgfqpoint{5.345758in}{0.806567in}}%
\pgfpathlineto{\pgfqpoint{5.346054in}{0.806568in}}%
\pgfpathlineto{\pgfqpoint{5.346350in}{0.806569in}}%
\pgfpathlineto{\pgfqpoint{5.346646in}{0.806570in}}%
\pgfpathlineto{\pgfqpoint{5.346942in}{0.806572in}}%
\pgfpathlineto{\pgfqpoint{5.347238in}{0.806573in}}%
\pgfpathlineto{\pgfqpoint{5.347534in}{0.806574in}}%
\pgfpathlineto{\pgfqpoint{5.347830in}{0.806575in}}%
\pgfpathlineto{\pgfqpoint{5.348126in}{0.806577in}}%
\pgfpathlineto{\pgfqpoint{5.348422in}{0.806578in}}%
\pgfpathlineto{\pgfqpoint{5.348718in}{0.806579in}}%
\pgfpathlineto{\pgfqpoint{5.349014in}{0.806580in}}%
\pgfpathlineto{\pgfqpoint{5.349310in}{0.806580in}}%
\pgfpathlineto{\pgfqpoint{5.349606in}{0.806577in}}%
\pgfpathlineto{\pgfqpoint{5.349902in}{0.806574in}}%
\pgfpathlineto{\pgfqpoint{5.350198in}{0.806577in}}%
\pgfpathlineto{\pgfqpoint{5.350494in}{0.806595in}}%
\pgfpathlineto{\pgfqpoint{5.350790in}{0.806587in}}%
\pgfpathlineto{\pgfqpoint{5.351086in}{0.806580in}}%
\pgfpathlineto{\pgfqpoint{5.351382in}{0.806572in}}%
\pgfpathlineto{\pgfqpoint{5.351678in}{0.806565in}}%
\pgfpathlineto{\pgfqpoint{5.351974in}{0.806558in}}%
\pgfpathlineto{\pgfqpoint{5.352270in}{0.806550in}}%
\pgfpathlineto{\pgfqpoint{5.352566in}{0.806543in}}%
\pgfpathlineto{\pgfqpoint{5.352862in}{0.806536in}}%
\pgfpathlineto{\pgfqpoint{5.353158in}{0.806528in}}%
\pgfpathlineto{\pgfqpoint{5.353454in}{0.806521in}}%
\pgfpathlineto{\pgfqpoint{5.353750in}{0.806514in}}%
\pgfpathlineto{\pgfqpoint{5.354046in}{0.806506in}}%
\pgfpathlineto{\pgfqpoint{5.354342in}{0.806499in}}%
\pgfpathlineto{\pgfqpoint{5.354638in}{0.806491in}}%
\pgfpathlineto{\pgfqpoint{5.354934in}{0.806484in}}%
\pgfpathlineto{\pgfqpoint{5.355230in}{0.806477in}}%
\pgfpathlineto{\pgfqpoint{5.355526in}{0.806469in}}%
\pgfpathlineto{\pgfqpoint{5.355822in}{0.806462in}}%
\pgfpathlineto{\pgfqpoint{5.356118in}{0.806455in}}%
\pgfpathlineto{\pgfqpoint{5.356414in}{0.806447in}}%
\pgfpathlineto{\pgfqpoint{5.356710in}{0.806419in}}%
\pgfpathlineto{\pgfqpoint{5.357006in}{0.806337in}}%
\pgfpathlineto{\pgfqpoint{5.357302in}{0.806251in}}%
\pgfpathlineto{\pgfqpoint{5.357598in}{0.806166in}}%
\pgfpathlineto{\pgfqpoint{5.357894in}{0.806080in}}%
\pgfpathlineto{\pgfqpoint{5.358190in}{0.806007in}}%
\pgfpathlineto{\pgfqpoint{5.358486in}{0.805963in}}%
\pgfpathlineto{\pgfqpoint{5.358782in}{0.805880in}}%
\pgfpathlineto{\pgfqpoint{5.359078in}{0.804223in}}%
\pgfpathlineto{\pgfqpoint{5.359374in}{0.799221in}}%
\pgfpathlineto{\pgfqpoint{5.359670in}{0.799063in}}%
\pgfpathlineto{\pgfqpoint{5.359966in}{0.798982in}}%
\pgfpathlineto{\pgfqpoint{5.360262in}{0.798980in}}%
\pgfpathlineto{\pgfqpoint{5.360558in}{0.798978in}}%
\pgfpathlineto{\pgfqpoint{5.360854in}{0.798975in}}%
\pgfpathlineto{\pgfqpoint{5.361150in}{0.798973in}}%
\pgfpathlineto{\pgfqpoint{5.361446in}{0.798971in}}%
\pgfpathlineto{\pgfqpoint{5.361742in}{0.798969in}}%
\pgfpathlineto{\pgfqpoint{5.362038in}{0.798967in}}%
\pgfpathlineto{\pgfqpoint{5.362334in}{0.798965in}}%
\pgfpathlineto{\pgfqpoint{5.362630in}{0.798963in}}%
\pgfpathlineto{\pgfqpoint{5.362926in}{0.798961in}}%
\pgfpathlineto{\pgfqpoint{5.363222in}{0.798959in}}%
\pgfpathlineto{\pgfqpoint{5.363518in}{0.798956in}}%
\pgfpathlineto{\pgfqpoint{5.363814in}{0.798954in}}%
\pgfpathlineto{\pgfqpoint{5.364110in}{0.798951in}}%
\pgfpathlineto{\pgfqpoint{5.364406in}{0.798949in}}%
\pgfpathlineto{\pgfqpoint{5.364702in}{0.798946in}}%
\pgfpathlineto{\pgfqpoint{5.364998in}{0.798950in}}%
\pgfpathlineto{\pgfqpoint{5.365294in}{0.799081in}}%
\pgfpathlineto{\pgfqpoint{5.365590in}{0.799394in}}%
\pgfpathlineto{\pgfqpoint{5.365886in}{0.799418in}}%
\pgfpathlineto{\pgfqpoint{5.366182in}{0.799414in}}%
\pgfpathlineto{\pgfqpoint{5.366478in}{0.799410in}}%
\pgfpathlineto{\pgfqpoint{5.366774in}{0.799405in}}%
\pgfpathlineto{\pgfqpoint{5.367071in}{0.799401in}}%
\pgfpathlineto{\pgfqpoint{5.367367in}{0.799397in}}%
\pgfpathlineto{\pgfqpoint{5.367663in}{0.799393in}}%
\pgfpathlineto{\pgfqpoint{5.367959in}{0.799388in}}%
\pgfpathlineto{\pgfqpoint{5.368255in}{0.799384in}}%
\pgfpathlineto{\pgfqpoint{5.368551in}{0.799380in}}%
\pgfpathlineto{\pgfqpoint{5.368847in}{0.799376in}}%
\pgfpathlineto{\pgfqpoint{5.369143in}{0.799371in}}%
\pgfpathlineto{\pgfqpoint{5.369439in}{0.799367in}}%
\pgfpathlineto{\pgfqpoint{5.369735in}{0.799363in}}%
\pgfpathlineto{\pgfqpoint{5.370031in}{0.799359in}}%
\pgfpathlineto{\pgfqpoint{5.370327in}{0.799354in}}%
\pgfpathlineto{\pgfqpoint{5.370623in}{0.799350in}}%
\pgfpathlineto{\pgfqpoint{5.370919in}{0.799346in}}%
\pgfpathlineto{\pgfqpoint{5.371215in}{0.799342in}}%
\pgfpathlineto{\pgfqpoint{5.371511in}{0.799337in}}%
\pgfpathlineto{\pgfqpoint{5.371807in}{0.799333in}}%
\pgfpathlineto{\pgfqpoint{5.372103in}{0.799329in}}%
\pgfpathlineto{\pgfqpoint{5.372399in}{0.799325in}}%
\pgfpathlineto{\pgfqpoint{5.372695in}{0.799320in}}%
\pgfpathlineto{\pgfqpoint{5.372991in}{0.799316in}}%
\pgfpathlineto{\pgfqpoint{5.373287in}{0.799312in}}%
\pgfpathlineto{\pgfqpoint{5.373583in}{0.799308in}}%
\pgfpathlineto{\pgfqpoint{5.373879in}{0.799303in}}%
\pgfpathlineto{\pgfqpoint{5.374175in}{0.799299in}}%
\pgfpathlineto{\pgfqpoint{5.374471in}{0.799295in}}%
\pgfpathlineto{\pgfqpoint{5.374767in}{0.799291in}}%
\pgfpathlineto{\pgfqpoint{5.375063in}{0.799286in}}%
\pgfpathlineto{\pgfqpoint{5.375359in}{0.799282in}}%
\pgfpathlineto{\pgfqpoint{5.375655in}{0.799278in}}%
\pgfpathlineto{\pgfqpoint{5.375951in}{0.799274in}}%
\pgfpathlineto{\pgfqpoint{5.376247in}{0.799269in}}%
\pgfpathlineto{\pgfqpoint{5.376543in}{0.799265in}}%
\pgfpathlineto{\pgfqpoint{5.376839in}{0.799261in}}%
\pgfpathlineto{\pgfqpoint{5.377135in}{0.799257in}}%
\pgfpathlineto{\pgfqpoint{5.377431in}{0.799252in}}%
\pgfpathlineto{\pgfqpoint{5.377727in}{0.799248in}}%
\pgfpathlineto{\pgfqpoint{5.378023in}{0.799244in}}%
\pgfpathlineto{\pgfqpoint{5.378319in}{0.799240in}}%
\pgfpathlineto{\pgfqpoint{5.378615in}{0.799235in}}%
\pgfpathlineto{\pgfqpoint{5.378911in}{0.799231in}}%
\pgfpathlineto{\pgfqpoint{5.379207in}{0.799227in}}%
\pgfpathlineto{\pgfqpoint{5.379503in}{0.799223in}}%
\pgfpathlineto{\pgfqpoint{5.379799in}{0.799218in}}%
\pgfpathlineto{\pgfqpoint{5.380095in}{0.799214in}}%
\pgfpathlineto{\pgfqpoint{5.380391in}{0.799210in}}%
\pgfpathlineto{\pgfqpoint{5.380687in}{0.799206in}}%
\pgfpathlineto{\pgfqpoint{5.380983in}{0.799201in}}%
\pgfpathlineto{\pgfqpoint{5.381279in}{0.799197in}}%
\pgfpathlineto{\pgfqpoint{5.381575in}{0.799193in}}%
\pgfpathlineto{\pgfqpoint{5.381871in}{0.799189in}}%
\pgfpathlineto{\pgfqpoint{5.382167in}{0.799184in}}%
\pgfpathlineto{\pgfqpoint{5.382463in}{0.799180in}}%
\pgfpathlineto{\pgfqpoint{5.382759in}{0.799176in}}%
\pgfpathlineto{\pgfqpoint{5.383055in}{0.799172in}}%
\pgfpathlineto{\pgfqpoint{5.383351in}{0.799167in}}%
\pgfpathlineto{\pgfqpoint{5.383647in}{0.799163in}}%
\pgfpathlineto{\pgfqpoint{5.383943in}{0.799159in}}%
\pgfpathlineto{\pgfqpoint{5.384239in}{0.799155in}}%
\pgfpathlineto{\pgfqpoint{5.384535in}{0.799150in}}%
\pgfpathlineto{\pgfqpoint{5.384831in}{0.799146in}}%
\pgfpathlineto{\pgfqpoint{5.385127in}{0.799142in}}%
\pgfpathlineto{\pgfqpoint{5.385423in}{0.799138in}}%
\pgfpathlineto{\pgfqpoint{5.385719in}{0.799133in}}%
\pgfpathlineto{\pgfqpoint{5.386015in}{0.799129in}}%
\pgfpathlineto{\pgfqpoint{5.386311in}{0.799125in}}%
\pgfpathlineto{\pgfqpoint{5.386607in}{0.799121in}}%
\pgfpathlineto{\pgfqpoint{5.386903in}{0.799116in}}%
\pgfpathlineto{\pgfqpoint{5.387199in}{0.799112in}}%
\pgfpathlineto{\pgfqpoint{5.387495in}{0.799108in}}%
\pgfpathlineto{\pgfqpoint{5.387791in}{0.799104in}}%
\pgfpathlineto{\pgfqpoint{5.388087in}{0.799099in}}%
\pgfpathlineto{\pgfqpoint{5.388383in}{0.799095in}}%
\pgfpathlineto{\pgfqpoint{5.388679in}{0.799091in}}%
\pgfpathlineto{\pgfqpoint{5.388975in}{0.799087in}}%
\pgfpathlineto{\pgfqpoint{5.389271in}{0.799082in}}%
\pgfpathlineto{\pgfqpoint{5.389567in}{0.799078in}}%
\pgfpathlineto{\pgfqpoint{5.389863in}{0.799074in}}%
\pgfpathlineto{\pgfqpoint{5.390159in}{0.799070in}}%
\pgfpathlineto{\pgfqpoint{5.390455in}{0.799066in}}%
\pgfpathlineto{\pgfqpoint{5.390751in}{0.799061in}}%
\pgfpathlineto{\pgfqpoint{5.391047in}{0.799057in}}%
\pgfpathlineto{\pgfqpoint{5.391343in}{0.799053in}}%
\pgfpathlineto{\pgfqpoint{5.391639in}{0.799049in}}%
\pgfpathlineto{\pgfqpoint{5.391935in}{0.799044in}}%
\pgfpathlineto{\pgfqpoint{5.392231in}{0.799315in}}%
\pgfpathlineto{\pgfqpoint{5.392527in}{0.799507in}}%
\pgfpathlineto{\pgfqpoint{5.392823in}{0.799501in}}%
\pgfpathlineto{\pgfqpoint{5.393119in}{0.799496in}}%
\pgfpathlineto{\pgfqpoint{5.393415in}{0.799490in}}%
\pgfpathlineto{\pgfqpoint{5.393711in}{0.799485in}}%
\pgfpathlineto{\pgfqpoint{5.394007in}{0.799479in}}%
\pgfpathlineto{\pgfqpoint{5.394303in}{0.799474in}}%
\pgfpathlineto{\pgfqpoint{5.394599in}{0.799654in}}%
\pgfpathlineto{\pgfqpoint{5.394895in}{0.799694in}}%
\pgfpathlineto{\pgfqpoint{5.395191in}{0.799691in}}%
\pgfpathlineto{\pgfqpoint{5.395487in}{0.799688in}}%
\pgfpathlineto{\pgfqpoint{5.395783in}{0.799684in}}%
\pgfpathlineto{\pgfqpoint{5.396079in}{0.799681in}}%
\pgfpathlineto{\pgfqpoint{5.396375in}{0.799678in}}%
\pgfpathlineto{\pgfqpoint{5.396671in}{0.799675in}}%
\pgfpathlineto{\pgfqpoint{5.396967in}{0.799671in}}%
\pgfpathlineto{\pgfqpoint{5.397263in}{0.799668in}}%
\pgfpathlineto{\pgfqpoint{5.397559in}{0.799665in}}%
\pgfpathlineto{\pgfqpoint{5.397855in}{0.799662in}}%
\pgfpathlineto{\pgfqpoint{5.398151in}{0.799659in}}%
\pgfpathlineto{\pgfqpoint{5.398447in}{0.799655in}}%
\pgfpathlineto{\pgfqpoint{5.398743in}{0.799652in}}%
\pgfpathlineto{\pgfqpoint{5.399039in}{0.799649in}}%
\pgfpathlineto{\pgfqpoint{5.399335in}{0.799646in}}%
\pgfpathlineto{\pgfqpoint{5.399631in}{0.799642in}}%
\pgfpathlineto{\pgfqpoint{5.399927in}{0.799639in}}%
\pgfpathlineto{\pgfqpoint{5.400223in}{0.799637in}}%
\pgfpathlineto{\pgfqpoint{5.400519in}{0.799641in}}%
\pgfpathlineto{\pgfqpoint{5.400815in}{0.799645in}}%
\pgfpathlineto{\pgfqpoint{5.401111in}{0.799645in}}%
\pgfpathlineto{\pgfqpoint{5.401407in}{0.799641in}}%
\pgfpathlineto{\pgfqpoint{5.401703in}{0.799635in}}%
\pgfpathlineto{\pgfqpoint{5.401999in}{0.799629in}}%
\pgfpathlineto{\pgfqpoint{5.402295in}{0.799622in}}%
\pgfpathlineto{\pgfqpoint{5.402591in}{0.799616in}}%
\pgfpathlineto{\pgfqpoint{5.402887in}{0.799610in}}%
\pgfpathlineto{\pgfqpoint{5.403183in}{0.799604in}}%
\pgfpathlineto{\pgfqpoint{5.403479in}{0.799598in}}%
\pgfpathlineto{\pgfqpoint{5.403775in}{0.799592in}}%
\pgfpathlineto{\pgfqpoint{5.404071in}{0.799586in}}%
\pgfpathlineto{\pgfqpoint{5.404367in}{0.799580in}}%
\pgfpathlineto{\pgfqpoint{5.404663in}{0.799574in}}%
\pgfpathlineto{\pgfqpoint{5.404959in}{0.799568in}}%
\pgfpathlineto{\pgfqpoint{5.405255in}{0.799562in}}%
\pgfpathlineto{\pgfqpoint{5.405551in}{0.799556in}}%
\pgfpathlineto{\pgfqpoint{5.405847in}{0.799549in}}%
\pgfpathlineto{\pgfqpoint{5.406143in}{0.799551in}}%
\pgfpathlineto{\pgfqpoint{5.406439in}{0.799563in}}%
\pgfpathlineto{\pgfqpoint{5.406735in}{0.799567in}}%
\pgfpathlineto{\pgfqpoint{5.407031in}{0.799558in}}%
\pgfpathlineto{\pgfqpoint{5.407327in}{0.799548in}}%
\pgfpathlineto{\pgfqpoint{5.407623in}{0.799563in}}%
\pgfpathlineto{\pgfqpoint{5.407919in}{0.799557in}}%
\pgfpathlineto{\pgfqpoint{5.408215in}{0.799533in}}%
\pgfpathlineto{\pgfqpoint{5.408511in}{0.799511in}}%
\pgfpathlineto{\pgfqpoint{5.408807in}{0.799488in}}%
\pgfpathlineto{\pgfqpoint{5.409103in}{0.799465in}}%
\pgfpathlineto{\pgfqpoint{5.409399in}{0.799443in}}%
\pgfpathlineto{\pgfqpoint{5.409695in}{0.799420in}}%
\pgfpathlineto{\pgfqpoint{5.409991in}{0.799397in}}%
\pgfpathlineto{\pgfqpoint{5.410287in}{0.799375in}}%
\pgfpathlineto{\pgfqpoint{5.410583in}{0.799352in}}%
\pgfpathlineto{\pgfqpoint{5.410879in}{0.799329in}}%
\pgfpathlineto{\pgfqpoint{5.411175in}{0.799307in}}%
\pgfpathlineto{\pgfqpoint{5.411471in}{0.799284in}}%
\pgfpathlineto{\pgfqpoint{5.411767in}{0.799261in}}%
\pgfpathlineto{\pgfqpoint{5.412063in}{0.799239in}}%
\pgfpathlineto{\pgfqpoint{5.412359in}{0.799216in}}%
\pgfpathlineto{\pgfqpoint{5.412655in}{0.799194in}}%
\pgfpathlineto{\pgfqpoint{5.412951in}{0.798938in}}%
\pgfpathlineto{\pgfqpoint{5.413247in}{0.798687in}}%
\pgfpathlineto{\pgfqpoint{5.413543in}{0.798182in}}%
\pgfpathlineto{\pgfqpoint{5.413839in}{0.797704in}}%
\pgfpathlineto{\pgfqpoint{5.414135in}{0.797670in}}%
\pgfpathlineto{\pgfqpoint{5.414431in}{0.797635in}}%
\pgfpathlineto{\pgfqpoint{5.414727in}{0.797601in}}%
\pgfpathlineto{\pgfqpoint{5.415023in}{0.797566in}}%
\pgfpathlineto{\pgfqpoint{5.415319in}{0.797531in}}%
\pgfpathlineto{\pgfqpoint{5.415615in}{0.797497in}}%
\pgfpathlineto{\pgfqpoint{5.415911in}{0.797462in}}%
\pgfpathlineto{\pgfqpoint{5.416207in}{0.797427in}}%
\pgfpathlineto{\pgfqpoint{5.416503in}{0.797393in}}%
\pgfpathlineto{\pgfqpoint{5.416799in}{0.797358in}}%
\pgfpathlineto{\pgfqpoint{5.417095in}{0.797323in}}%
\pgfpathlineto{\pgfqpoint{5.417391in}{0.797289in}}%
\pgfpathlineto{\pgfqpoint{5.417687in}{0.797254in}}%
\pgfpathlineto{\pgfqpoint{5.417983in}{0.797219in}}%
\pgfpathlineto{\pgfqpoint{5.418279in}{0.797185in}}%
\pgfpathlineto{\pgfqpoint{5.418575in}{0.797150in}}%
\pgfpathlineto{\pgfqpoint{5.418871in}{0.797115in}}%
\pgfpathlineto{\pgfqpoint{5.419167in}{0.797081in}}%
\pgfpathlineto{\pgfqpoint{5.419463in}{0.797046in}}%
\pgfpathlineto{\pgfqpoint{5.419759in}{0.797010in}}%
\pgfpathlineto{\pgfqpoint{5.420055in}{0.796966in}}%
\pgfpathlineto{\pgfqpoint{5.420351in}{0.796884in}}%
\pgfpathlineto{\pgfqpoint{5.420647in}{0.795867in}}%
\pgfpathlineto{\pgfqpoint{5.420943in}{0.795900in}}%
\pgfpathlineto{\pgfqpoint{5.421239in}{0.795933in}}%
\pgfpathlineto{\pgfqpoint{5.421535in}{0.795745in}}%
\pgfpathlineto{\pgfqpoint{5.421831in}{0.795790in}}%
\pgfpathlineto{\pgfqpoint{5.422127in}{0.795852in}}%
\pgfpathlineto{\pgfqpoint{5.422423in}{0.795913in}}%
\pgfpathlineto{\pgfqpoint{5.422719in}{0.795918in}}%
\pgfpathlineto{\pgfqpoint{5.423015in}{0.795918in}}%
\pgfpathlineto{\pgfqpoint{5.423311in}{0.795918in}}%
\pgfpathlineto{\pgfqpoint{5.423607in}{0.795918in}}%
\pgfpathlineto{\pgfqpoint{5.423903in}{0.795918in}}%
\pgfpathlineto{\pgfqpoint{5.424199in}{0.795918in}}%
\pgfpathlineto{\pgfqpoint{5.424495in}{0.795918in}}%
\pgfpathlineto{\pgfqpoint{5.424791in}{0.795918in}}%
\pgfpathlineto{\pgfqpoint{5.425087in}{0.795918in}}%
\pgfpathlineto{\pgfqpoint{5.425383in}{0.795918in}}%
\pgfpathlineto{\pgfqpoint{5.425679in}{0.795918in}}%
\pgfpathlineto{\pgfqpoint{5.425975in}{0.795918in}}%
\pgfpathlineto{\pgfqpoint{5.426271in}{0.795918in}}%
\pgfpathlineto{\pgfqpoint{5.426567in}{0.795919in}}%
\pgfpathlineto{\pgfqpoint{5.426863in}{0.795919in}}%
\pgfpathlineto{\pgfqpoint{5.427159in}{0.795919in}}%
\pgfpathlineto{\pgfqpoint{5.427455in}{0.795919in}}%
\pgfpathlineto{\pgfqpoint{5.427751in}{0.795919in}}%
\pgfpathlineto{\pgfqpoint{5.428047in}{0.795919in}}%
\pgfpathlineto{\pgfqpoint{5.428343in}{0.795919in}}%
\pgfpathlineto{\pgfqpoint{5.428639in}{0.795919in}}%
\pgfpathlineto{\pgfqpoint{5.428935in}{0.795919in}}%
\pgfpathlineto{\pgfqpoint{5.429231in}{0.795919in}}%
\pgfpathlineto{\pgfqpoint{5.429527in}{0.795919in}}%
\pgfpathlineto{\pgfqpoint{5.429823in}{0.795919in}}%
\pgfpathlineto{\pgfqpoint{5.430119in}{0.795919in}}%
\pgfpathlineto{\pgfqpoint{5.430415in}{0.795919in}}%
\pgfpathlineto{\pgfqpoint{5.430711in}{0.795919in}}%
\pgfpathlineto{\pgfqpoint{5.431007in}{0.795920in}}%
\pgfpathlineto{\pgfqpoint{5.431303in}{0.795920in}}%
\pgfpathlineto{\pgfqpoint{5.431599in}{0.795920in}}%
\pgfpathlineto{\pgfqpoint{5.431895in}{0.795920in}}%
\pgfpathlineto{\pgfqpoint{5.432191in}{0.795920in}}%
\pgfpathlineto{\pgfqpoint{5.432487in}{0.795920in}}%
\pgfpathlineto{\pgfqpoint{5.432783in}{0.795920in}}%
\pgfpathlineto{\pgfqpoint{5.433079in}{0.795920in}}%
\pgfpathlineto{\pgfqpoint{5.433375in}{0.795920in}}%
\pgfpathlineto{\pgfqpoint{5.433671in}{0.795920in}}%
\pgfpathlineto{\pgfqpoint{5.433967in}{0.795920in}}%
\pgfpathlineto{\pgfqpoint{5.434263in}{0.795920in}}%
\pgfpathlineto{\pgfqpoint{5.434560in}{0.795920in}}%
\pgfpathlineto{\pgfqpoint{5.434856in}{0.795920in}}%
\pgfpathlineto{\pgfqpoint{5.435152in}{0.795920in}}%
\pgfpathlineto{\pgfqpoint{5.435448in}{0.795920in}}%
\pgfpathlineto{\pgfqpoint{5.435744in}{0.795921in}}%
\pgfpathlineto{\pgfqpoint{5.436040in}{0.795921in}}%
\pgfpathlineto{\pgfqpoint{5.436336in}{0.795921in}}%
\pgfpathlineto{\pgfqpoint{5.436632in}{0.795921in}}%
\pgfpathlineto{\pgfqpoint{5.436928in}{0.795921in}}%
\pgfpathlineto{\pgfqpoint{5.437224in}{0.795921in}}%
\pgfpathlineto{\pgfqpoint{5.437520in}{0.795921in}}%
\pgfpathlineto{\pgfqpoint{5.437816in}{0.795921in}}%
\pgfpathlineto{\pgfqpoint{5.438112in}{0.795921in}}%
\pgfpathlineto{\pgfqpoint{5.438408in}{0.795921in}}%
\pgfpathlineto{\pgfqpoint{5.438704in}{0.795921in}}%
\pgfpathlineto{\pgfqpoint{5.439000in}{0.795921in}}%
\pgfpathlineto{\pgfqpoint{5.439296in}{0.795921in}}%
\pgfpathlineto{\pgfqpoint{5.439592in}{0.795921in}}%
\pgfpathlineto{\pgfqpoint{5.439888in}{0.795921in}}%
\pgfpathlineto{\pgfqpoint{5.440184in}{0.795922in}}%
\pgfpathlineto{\pgfqpoint{5.440480in}{0.795922in}}%
\pgfpathlineto{\pgfqpoint{5.440776in}{0.795922in}}%
\pgfpathlineto{\pgfqpoint{5.441072in}{0.795922in}}%
\pgfpathlineto{\pgfqpoint{5.441368in}{0.795922in}}%
\pgfpathlineto{\pgfqpoint{5.441664in}{0.795922in}}%
\pgfpathlineto{\pgfqpoint{5.441960in}{0.795922in}}%
\pgfpathlineto{\pgfqpoint{5.442256in}{0.795922in}}%
\pgfpathlineto{\pgfqpoint{5.442552in}{0.795854in}}%
\pgfpathlineto{\pgfqpoint{5.442848in}{0.795675in}}%
\pgfpathlineto{\pgfqpoint{5.443144in}{0.795495in}}%
\pgfpathlineto{\pgfqpoint{5.443440in}{0.795618in}}%
\pgfpathlineto{\pgfqpoint{5.443736in}{0.795613in}}%
\pgfpathlineto{\pgfqpoint{5.444032in}{0.796020in}}%
\pgfpathlineto{\pgfqpoint{5.444328in}{0.796431in}}%
\pgfpathlineto{\pgfqpoint{5.444624in}{0.796839in}}%
\pgfpathlineto{\pgfqpoint{5.444920in}{0.796936in}}%
\pgfpathlineto{\pgfqpoint{5.445216in}{0.796811in}}%
\pgfpathlineto{\pgfqpoint{5.445512in}{0.796643in}}%
\pgfpathlineto{\pgfqpoint{5.445808in}{0.796462in}}%
\pgfpathlineto{\pgfqpoint{5.446104in}{0.796281in}}%
\pgfpathlineto{\pgfqpoint{5.446400in}{0.796101in}}%
\pgfpathlineto{\pgfqpoint{5.446696in}{0.795920in}}%
\pgfpathlineto{\pgfqpoint{5.446992in}{0.795739in}}%
\pgfpathlineto{\pgfqpoint{5.447288in}{0.795558in}}%
\pgfpathlineto{\pgfqpoint{5.447584in}{0.795378in}}%
\pgfpathlineto{\pgfqpoint{5.447880in}{0.795197in}}%
\pgfpathlineto{\pgfqpoint{5.448176in}{0.795016in}}%
\pgfpathlineto{\pgfqpoint{5.448472in}{0.794835in}}%
\pgfpathlineto{\pgfqpoint{5.448768in}{0.794966in}}%
\pgfpathlineto{\pgfqpoint{5.449064in}{0.796160in}}%
\pgfpathlineto{\pgfqpoint{5.449360in}{0.796184in}}%
\pgfpathlineto{\pgfqpoint{5.449656in}{0.795876in}}%
\pgfpathlineto{\pgfqpoint{5.449952in}{0.795550in}}%
\pgfpathlineto{\pgfqpoint{5.450248in}{0.795258in}}%
\pgfpathlineto{\pgfqpoint{5.450544in}{0.794351in}}%
\pgfpathlineto{\pgfqpoint{5.450840in}{0.795293in}}%
\pgfpathlineto{\pgfqpoint{5.451136in}{0.794791in}}%
\pgfpathlineto{\pgfqpoint{5.451432in}{0.795824in}}%
\pgfpathlineto{\pgfqpoint{5.451728in}{0.798223in}}%
\pgfpathlineto{\pgfqpoint{5.452024in}{0.798110in}}%
\pgfpathlineto{\pgfqpoint{5.452320in}{0.797889in}}%
\pgfpathlineto{\pgfqpoint{5.452616in}{0.797667in}}%
\pgfpathlineto{\pgfqpoint{5.452912in}{0.797445in}}%
\pgfpathlineto{\pgfqpoint{5.453208in}{0.797224in}}%
\pgfpathlineto{\pgfqpoint{5.453504in}{0.797002in}}%
\pgfpathlineto{\pgfqpoint{5.453800in}{0.796780in}}%
\pgfpathlineto{\pgfqpoint{5.454096in}{0.796559in}}%
\pgfpathlineto{\pgfqpoint{5.454392in}{0.796337in}}%
\pgfpathlineto{\pgfqpoint{5.454688in}{0.796115in}}%
\pgfpathlineto{\pgfqpoint{5.454984in}{0.795894in}}%
\pgfpathlineto{\pgfqpoint{5.455280in}{0.795672in}}%
\pgfpathlineto{\pgfqpoint{5.455576in}{0.795450in}}%
\pgfpathlineto{\pgfqpoint{5.455872in}{0.795229in}}%
\pgfpathlineto{\pgfqpoint{5.456168in}{0.795007in}}%
\pgfpathlineto{\pgfqpoint{5.456464in}{0.794785in}}%
\pgfpathlineto{\pgfqpoint{5.456760in}{0.794564in}}%
\pgfpathlineto{\pgfqpoint{5.457056in}{0.794342in}}%
\pgfpathlineto{\pgfqpoint{5.457352in}{0.794435in}}%
\pgfpathlineto{\pgfqpoint{5.457648in}{0.797676in}}%
\pgfpathlineto{\pgfqpoint{5.457944in}{0.797464in}}%
\pgfpathlineto{\pgfqpoint{5.458240in}{0.797251in}}%
\pgfpathlineto{\pgfqpoint{5.458536in}{0.797024in}}%
\pgfpathlineto{\pgfqpoint{5.458832in}{0.796792in}}%
\pgfpathlineto{\pgfqpoint{5.459128in}{0.796559in}}%
\pgfpathlineto{\pgfqpoint{5.459424in}{0.796326in}}%
\pgfpathlineto{\pgfqpoint{5.459720in}{0.796093in}}%
\pgfpathlineto{\pgfqpoint{5.460016in}{0.795860in}}%
\pgfpathlineto{\pgfqpoint{5.460312in}{0.795628in}}%
\pgfpathlineto{\pgfqpoint{5.460608in}{0.795395in}}%
\pgfpathlineto{\pgfqpoint{5.460904in}{0.795162in}}%
\pgfpathlineto{\pgfqpoint{5.461200in}{0.794929in}}%
\pgfpathlineto{\pgfqpoint{5.461496in}{0.794696in}}%
\pgfpathlineto{\pgfqpoint{5.461792in}{0.794464in}}%
\pgfpathlineto{\pgfqpoint{5.462088in}{0.794231in}}%
\pgfpathlineto{\pgfqpoint{5.462384in}{0.793998in}}%
\pgfpathlineto{\pgfqpoint{5.462680in}{0.793765in}}%
\pgfpathlineto{\pgfqpoint{5.462976in}{0.793532in}}%
\pgfpathlineto{\pgfqpoint{5.463272in}{0.793300in}}%
\pgfpathlineto{\pgfqpoint{5.463568in}{0.793083in}}%
\pgfpathlineto{\pgfqpoint{5.463864in}{0.793538in}}%
\pgfpathlineto{\pgfqpoint{5.464160in}{0.794347in}}%
\pgfpathlineto{\pgfqpoint{5.464456in}{0.795166in}}%
\pgfpathlineto{\pgfqpoint{5.464752in}{0.796021in}}%
\pgfpathlineto{\pgfqpoint{5.465048in}{0.796653in}}%
\pgfpathlineto{\pgfqpoint{5.465344in}{0.796683in}}%
\pgfpathlineto{\pgfqpoint{5.465640in}{0.796681in}}%
\pgfpathlineto{\pgfqpoint{5.465936in}{0.796678in}}%
\pgfpathlineto{\pgfqpoint{5.466232in}{0.796676in}}%
\pgfpathlineto{\pgfqpoint{5.466528in}{0.796673in}}%
\pgfpathlineto{\pgfqpoint{5.466824in}{0.796671in}}%
\pgfpathlineto{\pgfqpoint{5.467120in}{0.796668in}}%
\pgfpathlineto{\pgfqpoint{5.467416in}{0.796665in}}%
\pgfpathlineto{\pgfqpoint{5.467712in}{0.796663in}}%
\pgfpathlineto{\pgfqpoint{5.468008in}{0.796660in}}%
\pgfpathlineto{\pgfqpoint{5.468304in}{0.796658in}}%
\pgfpathlineto{\pgfqpoint{5.468600in}{0.796655in}}%
\pgfpathlineto{\pgfqpoint{5.468896in}{0.796653in}}%
\pgfpathlineto{\pgfqpoint{5.469192in}{0.796650in}}%
\pgfpathlineto{\pgfqpoint{5.469488in}{0.796648in}}%
\pgfpathlineto{\pgfqpoint{5.469784in}{0.796645in}}%
\pgfpathlineto{\pgfqpoint{5.470080in}{0.796642in}}%
\pgfpathlineto{\pgfqpoint{5.470376in}{0.796757in}}%
\pgfpathlineto{\pgfqpoint{5.470672in}{0.797220in}}%
\pgfpathlineto{\pgfqpoint{5.470968in}{0.797706in}}%
\pgfpathlineto{\pgfqpoint{5.471264in}{0.798193in}}%
\pgfpathlineto{\pgfqpoint{5.471560in}{0.798679in}}%
\pgfpathlineto{\pgfqpoint{5.471856in}{0.799143in}}%
\pgfpathlineto{\pgfqpoint{5.472152in}{0.799259in}}%
\pgfpathlineto{\pgfqpoint{5.472448in}{0.799260in}}%
\pgfpathlineto{\pgfqpoint{5.472744in}{0.799261in}}%
\pgfpathlineto{\pgfqpoint{5.473040in}{0.799262in}}%
\pgfpathlineto{\pgfqpoint{5.473336in}{0.799262in}}%
\pgfpathlineto{\pgfqpoint{5.473632in}{0.799263in}}%
\pgfpathlineto{\pgfqpoint{5.473928in}{0.799264in}}%
\pgfpathlineto{\pgfqpoint{5.474224in}{0.799265in}}%
\pgfpathlineto{\pgfqpoint{5.474520in}{0.799265in}}%
\pgfpathlineto{\pgfqpoint{5.474816in}{0.799266in}}%
\pgfpathlineto{\pgfqpoint{5.475112in}{0.799267in}}%
\pgfpathlineto{\pgfqpoint{5.475408in}{0.799267in}}%
\pgfpathlineto{\pgfqpoint{5.475704in}{0.799268in}}%
\pgfpathlineto{\pgfqpoint{5.476000in}{0.799269in}}%
\pgfpathlineto{\pgfqpoint{5.476296in}{0.799270in}}%
\pgfpathlineto{\pgfqpoint{5.476592in}{0.799270in}}%
\pgfpathlineto{\pgfqpoint{5.476888in}{0.799271in}}%
\pgfpathlineto{\pgfqpoint{5.477184in}{0.799272in}}%
\pgfpathlineto{\pgfqpoint{5.477480in}{0.799273in}}%
\pgfpathlineto{\pgfqpoint{5.477776in}{0.799273in}}%
\pgfpathlineto{\pgfqpoint{5.478072in}{0.799274in}}%
\pgfpathlineto{\pgfqpoint{5.478368in}{0.799275in}}%
\pgfpathlineto{\pgfqpoint{5.478664in}{0.799276in}}%
\pgfpathlineto{\pgfqpoint{5.478960in}{0.799276in}}%
\pgfpathlineto{\pgfqpoint{5.479256in}{0.799277in}}%
\pgfpathlineto{\pgfqpoint{5.479552in}{0.799278in}}%
\pgfpathlineto{\pgfqpoint{5.479848in}{0.799278in}}%
\pgfpathlineto{\pgfqpoint{5.480144in}{0.799279in}}%
\pgfpathlineto{\pgfqpoint{5.480440in}{0.799280in}}%
\pgfpathlineto{\pgfqpoint{5.480736in}{0.799281in}}%
\pgfpathlineto{\pgfqpoint{5.481032in}{0.799281in}}%
\pgfpathlineto{\pgfqpoint{5.481328in}{0.799282in}}%
\pgfpathlineto{\pgfqpoint{5.481624in}{0.799283in}}%
\pgfpathlineto{\pgfqpoint{5.481920in}{0.799284in}}%
\pgfpathlineto{\pgfqpoint{5.482216in}{0.799284in}}%
\pgfpathlineto{\pgfqpoint{5.482512in}{0.799285in}}%
\pgfpathlineto{\pgfqpoint{5.482808in}{0.799286in}}%
\pgfpathlineto{\pgfqpoint{5.483104in}{0.799287in}}%
\pgfpathlineto{\pgfqpoint{5.483400in}{0.799287in}}%
\pgfpathlineto{\pgfqpoint{5.483696in}{0.799288in}}%
\pgfpathlineto{\pgfqpoint{5.483992in}{0.799289in}}%
\pgfpathlineto{\pgfqpoint{5.484288in}{0.799289in}}%
\pgfpathlineto{\pgfqpoint{5.484584in}{0.799290in}}%
\pgfpathlineto{\pgfqpoint{5.484880in}{0.799291in}}%
\pgfpathlineto{\pgfqpoint{5.485176in}{0.799292in}}%
\pgfpathlineto{\pgfqpoint{5.485472in}{0.799292in}}%
\pgfpathlineto{\pgfqpoint{5.485768in}{0.799293in}}%
\pgfpathlineto{\pgfqpoint{5.486064in}{0.799294in}}%
\pgfpathlineto{\pgfqpoint{5.486360in}{0.799295in}}%
\pgfpathlineto{\pgfqpoint{5.486656in}{0.799295in}}%
\pgfpathlineto{\pgfqpoint{5.486952in}{0.799296in}}%
\pgfpathlineto{\pgfqpoint{5.487248in}{0.799297in}}%
\pgfpathlineto{\pgfqpoint{5.487544in}{0.799297in}}%
\pgfpathlineto{\pgfqpoint{5.487840in}{0.799298in}}%
\pgfpathlineto{\pgfqpoint{5.488136in}{0.799299in}}%
\pgfpathlineto{\pgfqpoint{5.488432in}{0.799300in}}%
\pgfpathlineto{\pgfqpoint{5.488728in}{0.799300in}}%
\pgfpathlineto{\pgfqpoint{5.489024in}{0.799301in}}%
\pgfpathlineto{\pgfqpoint{5.489320in}{0.799302in}}%
\pgfpathlineto{\pgfqpoint{5.489616in}{0.799303in}}%
\pgfpathlineto{\pgfqpoint{5.489912in}{0.799303in}}%
\pgfpathlineto{\pgfqpoint{5.490208in}{0.799304in}}%
\pgfpathlineto{\pgfqpoint{5.490504in}{0.799305in}}%
\pgfpathlineto{\pgfqpoint{5.490800in}{0.799306in}}%
\pgfpathlineto{\pgfqpoint{5.491096in}{0.799306in}}%
\pgfpathlineto{\pgfqpoint{5.491392in}{0.799307in}}%
\pgfpathlineto{\pgfqpoint{5.491688in}{0.799308in}}%
\pgfpathlineto{\pgfqpoint{5.491984in}{0.799308in}}%
\pgfpathlineto{\pgfqpoint{5.492280in}{0.799309in}}%
\pgfpathlineto{\pgfqpoint{5.492576in}{0.799310in}}%
\pgfpathlineto{\pgfqpoint{5.492872in}{0.799311in}}%
\pgfpathlineto{\pgfqpoint{5.493168in}{0.799311in}}%
\pgfpathlineto{\pgfqpoint{5.493464in}{0.799190in}}%
\pgfpathlineto{\pgfqpoint{5.493760in}{0.798874in}}%
\pgfpathlineto{\pgfqpoint{5.494056in}{0.798555in}}%
\pgfpathlineto{\pgfqpoint{5.494352in}{0.798236in}}%
\pgfpathlineto{\pgfqpoint{5.494648in}{0.797918in}}%
\pgfpathlineto{\pgfqpoint{5.494944in}{0.797599in}}%
\pgfpathlineto{\pgfqpoint{5.495240in}{0.797280in}}%
\pgfpathlineto{\pgfqpoint{5.495536in}{0.796962in}}%
\pgfpathlineto{\pgfqpoint{5.495832in}{0.796643in}}%
\pgfpathlineto{\pgfqpoint{5.496128in}{0.796324in}}%
\pgfpathlineto{\pgfqpoint{5.496424in}{0.796006in}}%
\pgfpathlineto{\pgfqpoint{5.496720in}{0.795687in}}%
\pgfpathlineto{\pgfqpoint{5.497016in}{0.795368in}}%
\pgfpathlineto{\pgfqpoint{5.497312in}{0.795050in}}%
\pgfpathlineto{\pgfqpoint{5.497608in}{0.794731in}}%
\pgfpathlineto{\pgfqpoint{5.497904in}{0.794412in}}%
\pgfpathlineto{\pgfqpoint{5.498200in}{0.793772in}}%
\pgfpathlineto{\pgfqpoint{5.498496in}{0.792949in}}%
\pgfpathlineto{\pgfqpoint{5.498792in}{0.792654in}}%
\pgfpathlineto{\pgfqpoint{5.499088in}{0.793272in}}%
\pgfpathlineto{\pgfqpoint{5.499384in}{0.794308in}}%
\pgfpathlineto{\pgfqpoint{5.499680in}{0.795345in}}%
\pgfpathlineto{\pgfqpoint{5.499976in}{0.796381in}}%
\pgfpathlineto{\pgfqpoint{5.500272in}{0.797417in}}%
\pgfpathlineto{\pgfqpoint{5.500568in}{0.798257in}}%
\pgfpathlineto{\pgfqpoint{5.500864in}{0.798213in}}%
\pgfpathlineto{\pgfqpoint{5.501160in}{0.798061in}}%
\pgfpathlineto{\pgfqpoint{5.501456in}{0.797909in}}%
\pgfpathlineto{\pgfqpoint{5.501752in}{0.797758in}}%
\pgfpathlineto{\pgfqpoint{5.502049in}{0.797606in}}%
\pgfpathlineto{\pgfqpoint{5.502345in}{0.797454in}}%
\pgfpathlineto{\pgfqpoint{5.502641in}{0.797302in}}%
\pgfpathlineto{\pgfqpoint{5.502937in}{0.797150in}}%
\pgfpathlineto{\pgfqpoint{5.503233in}{0.796999in}}%
\pgfpathlineto{\pgfqpoint{5.503529in}{0.796847in}}%
\pgfpathlineto{\pgfqpoint{5.503825in}{0.796695in}}%
\pgfpathlineto{\pgfqpoint{5.504121in}{0.796543in}}%
\pgfpathlineto{\pgfqpoint{5.504417in}{0.796392in}}%
\pgfpathlineto{\pgfqpoint{5.504713in}{0.796240in}}%
\pgfpathlineto{\pgfqpoint{5.505009in}{0.796088in}}%
\pgfpathlineto{\pgfqpoint{5.505305in}{0.795936in}}%
\pgfpathlineto{\pgfqpoint{5.505601in}{0.795784in}}%
\pgfpathlineto{\pgfqpoint{5.505897in}{0.795633in}}%
\pgfpathlineto{\pgfqpoint{5.506193in}{0.795481in}}%
\pgfpathlineto{\pgfqpoint{5.506489in}{0.795329in}}%
\pgfpathlineto{\pgfqpoint{5.506785in}{0.795177in}}%
\pgfpathlineto{\pgfqpoint{5.507081in}{0.795026in}}%
\pgfpathlineto{\pgfqpoint{5.507377in}{0.794874in}}%
\pgfpathlineto{\pgfqpoint{5.507673in}{0.794722in}}%
\pgfpathlineto{\pgfqpoint{5.507969in}{0.794570in}}%
\pgfpathlineto{\pgfqpoint{5.508265in}{0.794418in}}%
\pgfpathlineto{\pgfqpoint{5.508561in}{0.794267in}}%
\pgfpathlineto{\pgfqpoint{5.508857in}{0.794115in}}%
\pgfpathlineto{\pgfqpoint{5.509153in}{0.793963in}}%
\pgfpathlineto{\pgfqpoint{5.509449in}{0.793811in}}%
\pgfpathlineto{\pgfqpoint{5.509745in}{0.793660in}}%
\pgfpathlineto{\pgfqpoint{5.510041in}{0.793508in}}%
\pgfpathlineto{\pgfqpoint{5.510337in}{0.793356in}}%
\pgfpathlineto{\pgfqpoint{5.510633in}{0.793204in}}%
\pgfpathlineto{\pgfqpoint{5.510929in}{0.793052in}}%
\pgfpathlineto{\pgfqpoint{5.511225in}{0.792901in}}%
\pgfpathlineto{\pgfqpoint{5.511521in}{0.792749in}}%
\pgfpathlineto{\pgfqpoint{5.511817in}{0.792597in}}%
\pgfpathlineto{\pgfqpoint{5.512113in}{0.792445in}}%
\pgfpathlineto{\pgfqpoint{5.512409in}{0.792294in}}%
\pgfpathlineto{\pgfqpoint{5.512705in}{0.792165in}}%
\pgfpathlineto{\pgfqpoint{5.513001in}{0.792662in}}%
\pgfpathlineto{\pgfqpoint{5.513297in}{0.793429in}}%
\pgfpathlineto{\pgfqpoint{5.513593in}{0.794195in}}%
\pgfpathlineto{\pgfqpoint{5.513889in}{0.794962in}}%
\pgfpathlineto{\pgfqpoint{5.514185in}{0.795729in}}%
\pgfpathlineto{\pgfqpoint{5.514481in}{0.796495in}}%
\pgfpathlineto{\pgfqpoint{5.514777in}{0.797262in}}%
\pgfpathlineto{\pgfqpoint{5.515073in}{0.797769in}}%
\pgfpathlineto{\pgfqpoint{5.515369in}{0.797756in}}%
\pgfpathlineto{\pgfqpoint{5.515665in}{0.797728in}}%
\pgfpathlineto{\pgfqpoint{5.515961in}{0.797700in}}%
\pgfpathlineto{\pgfqpoint{5.516257in}{0.797672in}}%
\pgfpathlineto{\pgfqpoint{5.516553in}{0.797644in}}%
\pgfpathlineto{\pgfqpoint{5.516849in}{0.797617in}}%
\pgfpathlineto{\pgfqpoint{5.517145in}{0.797589in}}%
\pgfpathlineto{\pgfqpoint{5.517441in}{0.797561in}}%
\pgfpathlineto{\pgfqpoint{5.517737in}{0.797533in}}%
\pgfpathlineto{\pgfqpoint{5.518033in}{0.797505in}}%
\pgfpathlineto{\pgfqpoint{5.518329in}{0.797478in}}%
\pgfpathlineto{\pgfqpoint{5.518625in}{0.797450in}}%
\pgfpathlineto{\pgfqpoint{5.518921in}{0.797422in}}%
\pgfpathlineto{\pgfqpoint{5.519217in}{0.797394in}}%
\pgfpathlineto{\pgfqpoint{5.519513in}{0.797366in}}%
\pgfpathlineto{\pgfqpoint{5.519809in}{0.797338in}}%
\pgfpathlineto{\pgfqpoint{5.520105in}{0.797311in}}%
\pgfpathlineto{\pgfqpoint{5.520401in}{0.797283in}}%
\pgfpathlineto{\pgfqpoint{5.520697in}{0.797260in}}%
\pgfpathlineto{\pgfqpoint{5.520993in}{0.797238in}}%
\pgfpathlineto{\pgfqpoint{5.521289in}{0.797216in}}%
\pgfpathlineto{\pgfqpoint{5.521585in}{0.797142in}}%
\pgfpathlineto{\pgfqpoint{5.521881in}{0.797001in}}%
\pgfpathlineto{\pgfqpoint{5.522177in}{0.796876in}}%
\pgfpathlineto{\pgfqpoint{5.522473in}{0.796856in}}%
\pgfpathlineto{\pgfqpoint{5.522769in}{0.796765in}}%
\pgfpathlineto{\pgfqpoint{5.523065in}{0.796674in}}%
\pgfpathlineto{\pgfqpoint{5.523361in}{0.796583in}}%
\pgfpathlineto{\pgfqpoint{5.523657in}{0.796492in}}%
\pgfpathlineto{\pgfqpoint{5.523953in}{0.796402in}}%
\pgfpathlineto{\pgfqpoint{5.524249in}{0.796311in}}%
\pgfpathlineto{\pgfqpoint{5.524545in}{0.796220in}}%
\pgfpathlineto{\pgfqpoint{5.524841in}{0.796129in}}%
\pgfpathlineto{\pgfqpoint{5.525137in}{0.796039in}}%
\pgfpathlineto{\pgfqpoint{5.525433in}{0.795948in}}%
\pgfpathlineto{\pgfqpoint{5.525729in}{0.795857in}}%
\pgfpathlineto{\pgfqpoint{5.526025in}{0.795766in}}%
\pgfpathlineto{\pgfqpoint{5.526321in}{0.795676in}}%
\pgfpathlineto{\pgfqpoint{5.526617in}{0.795585in}}%
\pgfpathlineto{\pgfqpoint{5.526913in}{0.795494in}}%
\pgfpathlineto{\pgfqpoint{5.527209in}{0.795403in}}%
\pgfpathlineto{\pgfqpoint{5.527505in}{0.795312in}}%
\pgfpathlineto{\pgfqpoint{5.527801in}{0.795222in}}%
\pgfpathlineto{\pgfqpoint{5.528097in}{0.795131in}}%
\pgfpathlineto{\pgfqpoint{5.528393in}{0.795040in}}%
\pgfpathlineto{\pgfqpoint{5.528689in}{0.794949in}}%
\pgfpathlineto{\pgfqpoint{5.528985in}{0.794859in}}%
\pgfpathlineto{\pgfqpoint{5.529281in}{0.794768in}}%
\pgfpathlineto{\pgfqpoint{5.529577in}{0.794677in}}%
\pgfpathlineto{\pgfqpoint{5.529873in}{0.794586in}}%
\pgfpathlineto{\pgfqpoint{5.530169in}{0.794496in}}%
\pgfpathlineto{\pgfqpoint{5.530465in}{0.794405in}}%
\pgfpathlineto{\pgfqpoint{5.530761in}{0.794314in}}%
\pgfpathlineto{\pgfqpoint{5.531057in}{0.794223in}}%
\pgfpathlineto{\pgfqpoint{5.531353in}{0.794133in}}%
\pgfpathlineto{\pgfqpoint{5.531649in}{0.794042in}}%
\pgfpathlineto{\pgfqpoint{5.531945in}{0.793951in}}%
\pgfpathlineto{\pgfqpoint{5.532241in}{0.793860in}}%
\pgfpathlineto{\pgfqpoint{5.532537in}{0.793769in}}%
\pgfpathlineto{\pgfqpoint{5.532833in}{0.793679in}}%
\pgfpathlineto{\pgfqpoint{5.533129in}{0.793588in}}%
\pgfpathlineto{\pgfqpoint{5.533425in}{0.793497in}}%
\pgfpathlineto{\pgfqpoint{5.533721in}{0.793406in}}%
\pgfpathlineto{\pgfqpoint{5.534017in}{0.793316in}}%
\pgfpathlineto{\pgfqpoint{5.534313in}{0.793225in}}%
\pgfpathlineto{\pgfqpoint{5.534609in}{0.793134in}}%
\pgfpathlineto{\pgfqpoint{5.534905in}{0.793043in}}%
\pgfpathlineto{\pgfqpoint{5.535201in}{0.792953in}}%
\pgfpathlineto{\pgfqpoint{5.535497in}{0.792862in}}%
\pgfpathlineto{\pgfqpoint{5.535793in}{0.792771in}}%
\pgfpathlineto{\pgfqpoint{5.536089in}{0.792680in}}%
\pgfpathlineto{\pgfqpoint{5.536385in}{0.792589in}}%
\pgfpathlineto{\pgfqpoint{5.536681in}{0.792499in}}%
\pgfpathlineto{\pgfqpoint{5.536977in}{0.792408in}}%
\pgfpathlineto{\pgfqpoint{5.537273in}{0.792317in}}%
\pgfpathlineto{\pgfqpoint{5.537569in}{0.792226in}}%
\pgfpathlineto{\pgfqpoint{5.537865in}{0.792136in}}%
\pgfpathlineto{\pgfqpoint{5.538161in}{0.792045in}}%
\pgfpathlineto{\pgfqpoint{5.538457in}{0.791954in}}%
\pgfpathlineto{\pgfqpoint{5.538753in}{0.791863in}}%
\pgfpathlineto{\pgfqpoint{5.539049in}{0.791773in}}%
\pgfpathlineto{\pgfqpoint{5.539345in}{0.791682in}}%
\pgfpathlineto{\pgfqpoint{5.539641in}{0.791591in}}%
\pgfpathlineto{\pgfqpoint{5.539937in}{0.791500in}}%
\pgfpathlineto{\pgfqpoint{5.540233in}{0.791409in}}%
\pgfpathlineto{\pgfqpoint{5.540529in}{0.791319in}}%
\pgfpathlineto{\pgfqpoint{5.540825in}{0.791228in}}%
\pgfpathlineto{\pgfqpoint{5.541121in}{0.791167in}}%
\pgfpathlineto{\pgfqpoint{5.541417in}{0.791729in}}%
\pgfpathlineto{\pgfqpoint{5.541713in}{0.792526in}}%
\pgfpathlineto{\pgfqpoint{5.542009in}{0.793320in}}%
\pgfpathlineto{\pgfqpoint{5.542305in}{0.794115in}}%
\pgfpathlineto{\pgfqpoint{5.542601in}{0.794909in}}%
\pgfpathlineto{\pgfqpoint{5.542897in}{0.795704in}}%
\pgfpathlineto{\pgfqpoint{5.543193in}{0.796157in}}%
\pgfpathlineto{\pgfqpoint{5.543489in}{0.796156in}}%
\pgfpathlineto{\pgfqpoint{5.543785in}{0.796152in}}%
\pgfpathlineto{\pgfqpoint{5.544081in}{0.796149in}}%
\pgfpathlineto{\pgfqpoint{5.544377in}{0.796145in}}%
\pgfpathlineto{\pgfqpoint{5.544673in}{0.796142in}}%
\pgfpathlineto{\pgfqpoint{5.544969in}{0.796138in}}%
\pgfpathlineto{\pgfqpoint{5.545265in}{0.796135in}}%
\pgfpathlineto{\pgfqpoint{5.545561in}{0.796131in}}%
\pgfpathlineto{\pgfqpoint{5.545857in}{0.796128in}}%
\pgfpathlineto{\pgfqpoint{5.546153in}{0.796124in}}%
\pgfpathlineto{\pgfqpoint{5.546449in}{0.796121in}}%
\pgfpathlineto{\pgfqpoint{5.546745in}{0.796117in}}%
\pgfpathlineto{\pgfqpoint{5.547041in}{0.796114in}}%
\pgfpathlineto{\pgfqpoint{5.547337in}{0.796110in}}%
\pgfpathlineto{\pgfqpoint{5.547633in}{0.796107in}}%
\pgfpathlineto{\pgfqpoint{5.547929in}{0.796103in}}%
\pgfpathlineto{\pgfqpoint{5.548225in}{0.796100in}}%
\pgfpathlineto{\pgfqpoint{5.548521in}{0.796068in}}%
\pgfpathlineto{\pgfqpoint{5.548817in}{0.794956in}}%
\pgfpathlineto{\pgfqpoint{5.549113in}{0.793171in}}%
\pgfpathlineto{\pgfqpoint{5.549409in}{0.791388in}}%
\pgfpathlineto{\pgfqpoint{5.549705in}{0.791245in}}%
\pgfpathlineto{\pgfqpoint{5.550001in}{0.794173in}}%
\pgfpathlineto{\pgfqpoint{5.550297in}{0.796935in}}%
\pgfpathlineto{\pgfqpoint{5.550593in}{0.796950in}}%
\pgfpathlineto{\pgfqpoint{5.550889in}{0.796936in}}%
\pgfpathlineto{\pgfqpoint{5.551185in}{0.797494in}}%
\pgfpathlineto{\pgfqpoint{5.551481in}{0.797378in}}%
\pgfpathlineto{\pgfqpoint{5.551777in}{0.797262in}}%
\pgfpathlineto{\pgfqpoint{5.552073in}{0.797145in}}%
\pgfpathlineto{\pgfqpoint{5.552369in}{0.797029in}}%
\pgfpathlineto{\pgfqpoint{5.552665in}{0.796913in}}%
\pgfpathlineto{\pgfqpoint{5.552961in}{0.796797in}}%
\pgfpathlineto{\pgfqpoint{5.553257in}{0.796681in}}%
\pgfpathlineto{\pgfqpoint{5.553553in}{0.796565in}}%
\pgfpathlineto{\pgfqpoint{5.553849in}{0.796448in}}%
\pgfpathlineto{\pgfqpoint{5.554145in}{0.796332in}}%
\pgfpathlineto{\pgfqpoint{5.554441in}{0.796216in}}%
\pgfpathlineto{\pgfqpoint{5.554737in}{0.796100in}}%
\pgfpathlineto{\pgfqpoint{5.555033in}{0.795984in}}%
\pgfpathlineto{\pgfqpoint{5.555329in}{0.796647in}}%
\pgfpathlineto{\pgfqpoint{5.555625in}{0.797340in}}%
\pgfpathlineto{\pgfqpoint{5.555921in}{0.797330in}}%
\pgfpathlineto{\pgfqpoint{5.556217in}{0.797320in}}%
\pgfpathlineto{\pgfqpoint{5.556513in}{0.797309in}}%
\pgfpathlineto{\pgfqpoint{5.556809in}{0.797299in}}%
\pgfpathlineto{\pgfqpoint{5.557105in}{0.797289in}}%
\pgfpathlineto{\pgfqpoint{5.557401in}{0.797277in}}%
\pgfpathlineto{\pgfqpoint{5.557697in}{0.797261in}}%
\pgfpathlineto{\pgfqpoint{5.557993in}{0.797244in}}%
\pgfpathlineto{\pgfqpoint{5.558289in}{0.797227in}}%
\pgfpathlineto{\pgfqpoint{5.558585in}{0.797211in}}%
\pgfpathlineto{\pgfqpoint{5.558881in}{0.797194in}}%
\pgfpathlineto{\pgfqpoint{5.559177in}{0.797177in}}%
\pgfpathlineto{\pgfqpoint{5.559473in}{0.797160in}}%
\pgfpathlineto{\pgfqpoint{5.559769in}{0.797143in}}%
\pgfpathlineto{\pgfqpoint{5.560065in}{0.797126in}}%
\pgfpathlineto{\pgfqpoint{5.560361in}{0.797109in}}%
\pgfpathlineto{\pgfqpoint{5.560657in}{0.797093in}}%
\pgfpathlineto{\pgfqpoint{5.560953in}{0.797076in}}%
\pgfpathlineto{\pgfqpoint{5.561249in}{0.797059in}}%
\pgfpathlineto{\pgfqpoint{5.561545in}{0.797042in}}%
\pgfpathlineto{\pgfqpoint{5.561841in}{0.797025in}}%
\pgfpathlineto{\pgfqpoint{5.562137in}{0.797008in}}%
\pgfpathlineto{\pgfqpoint{5.562433in}{0.796991in}}%
\pgfpathlineto{\pgfqpoint{5.562729in}{0.796975in}}%
\pgfpathlineto{\pgfqpoint{5.563025in}{0.796958in}}%
\pgfpathlineto{\pgfqpoint{5.563321in}{0.796941in}}%
\pgfpathlineto{\pgfqpoint{5.563617in}{0.796255in}}%
\pgfpathlineto{\pgfqpoint{5.563913in}{0.795526in}}%
\pgfpathlineto{\pgfqpoint{5.564209in}{0.795496in}}%
\pgfpathlineto{\pgfqpoint{5.564505in}{0.795466in}}%
\pgfpathlineto{\pgfqpoint{5.564801in}{0.795436in}}%
\pgfpathlineto{\pgfqpoint{5.565097in}{0.795406in}}%
\pgfpathlineto{\pgfqpoint{5.565393in}{0.795376in}}%
\pgfpathlineto{\pgfqpoint{5.565689in}{0.795347in}}%
\pgfpathlineto{\pgfqpoint{5.565985in}{0.795317in}}%
\pgfpathlineto{\pgfqpoint{5.566281in}{0.795287in}}%
\pgfpathlineto{\pgfqpoint{5.566577in}{0.795257in}}%
\pgfpathlineto{\pgfqpoint{5.566873in}{0.795227in}}%
\pgfpathlineto{\pgfqpoint{5.567169in}{0.795197in}}%
\pgfpathlineto{\pgfqpoint{5.567465in}{0.795167in}}%
\pgfpathlineto{\pgfqpoint{5.567761in}{0.795137in}}%
\pgfpathlineto{\pgfqpoint{5.568057in}{0.795107in}}%
\pgfpathlineto{\pgfqpoint{5.568353in}{0.795077in}}%
\pgfpathlineto{\pgfqpoint{5.568649in}{0.795047in}}%
\pgfpathlineto{\pgfqpoint{5.568945in}{0.795017in}}%
\pgfpathlineto{\pgfqpoint{5.569242in}{0.794986in}}%
\pgfpathlineto{\pgfqpoint{5.569538in}{0.794953in}}%
\pgfpathlineto{\pgfqpoint{5.569834in}{0.794920in}}%
\pgfpathlineto{\pgfqpoint{5.570130in}{0.794887in}}%
\pgfpathlineto{\pgfqpoint{5.570426in}{0.794854in}}%
\pgfpathlineto{\pgfqpoint{5.570722in}{0.794821in}}%
\pgfpathlineto{\pgfqpoint{5.571018in}{0.794788in}}%
\pgfpathlineto{\pgfqpoint{5.571314in}{0.794755in}}%
\pgfpathlineto{\pgfqpoint{5.571610in}{0.794739in}}%
\pgfpathlineto{\pgfqpoint{5.571906in}{0.794730in}}%
\pgfpathlineto{\pgfqpoint{5.572202in}{0.794728in}}%
\pgfpathlineto{\pgfqpoint{5.572498in}{0.794726in}}%
\pgfpathlineto{\pgfqpoint{5.572794in}{0.794724in}}%
\pgfpathlineto{\pgfqpoint{5.573090in}{0.794722in}}%
\pgfpathlineto{\pgfqpoint{5.573386in}{0.794720in}}%
\pgfpathlineto{\pgfqpoint{5.573682in}{0.794718in}}%
\pgfpathlineto{\pgfqpoint{5.573978in}{0.794716in}}%
\pgfpathlineto{\pgfqpoint{5.574274in}{0.794715in}}%
\pgfpathlineto{\pgfqpoint{5.574570in}{0.794713in}}%
\pgfpathlineto{\pgfqpoint{5.574866in}{0.794711in}}%
\pgfpathlineto{\pgfqpoint{5.575162in}{0.794709in}}%
\pgfpathlineto{\pgfqpoint{5.575458in}{0.794707in}}%
\pgfpathlineto{\pgfqpoint{5.575754in}{0.794705in}}%
\pgfpathlineto{\pgfqpoint{5.576050in}{0.794703in}}%
\pgfpathlineto{\pgfqpoint{5.576346in}{0.794701in}}%
\pgfpathlineto{\pgfqpoint{5.576642in}{0.794699in}}%
\pgfpathlineto{\pgfqpoint{5.576938in}{0.794697in}}%
\pgfpathlineto{\pgfqpoint{5.577234in}{0.794695in}}%
\pgfpathlineto{\pgfqpoint{5.577530in}{0.794693in}}%
\pgfpathlineto{\pgfqpoint{5.577826in}{0.794691in}}%
\pgfpathlineto{\pgfqpoint{5.578122in}{0.794690in}}%
\pgfpathlineto{\pgfqpoint{5.578418in}{0.794688in}}%
\pgfpathlineto{\pgfqpoint{5.578714in}{0.794686in}}%
\pgfpathlineto{\pgfqpoint{5.579010in}{0.794684in}}%
\pgfpathlineto{\pgfqpoint{5.579306in}{0.794682in}}%
\pgfpathlineto{\pgfqpoint{5.579602in}{0.794680in}}%
\pgfpathlineto{\pgfqpoint{5.579898in}{0.794678in}}%
\pgfpathlineto{\pgfqpoint{5.580194in}{0.794676in}}%
\pgfpathlineto{\pgfqpoint{5.580490in}{0.794674in}}%
\pgfpathlineto{\pgfqpoint{5.580786in}{0.794672in}}%
\pgfpathlineto{\pgfqpoint{5.581082in}{0.794670in}}%
\pgfpathlineto{\pgfqpoint{5.581378in}{0.794668in}}%
\pgfpathlineto{\pgfqpoint{5.581674in}{0.794666in}}%
\pgfpathlineto{\pgfqpoint{5.581970in}{0.794665in}}%
\pgfpathlineto{\pgfqpoint{5.582266in}{0.794663in}}%
\pgfpathlineto{\pgfqpoint{5.582562in}{0.794661in}}%
\pgfpathlineto{\pgfqpoint{5.582858in}{0.794659in}}%
\pgfpathlineto{\pgfqpoint{5.583154in}{0.794657in}}%
\pgfpathlineto{\pgfqpoint{5.583450in}{0.794655in}}%
\pgfpathlineto{\pgfqpoint{5.583746in}{0.794653in}}%
\pgfpathlineto{\pgfqpoint{5.584042in}{0.794651in}}%
\pgfpathlineto{\pgfqpoint{5.584338in}{0.794649in}}%
\pgfpathlineto{\pgfqpoint{5.584634in}{0.794647in}}%
\pgfpathlineto{\pgfqpoint{5.584930in}{0.794645in}}%
\pgfpathlineto{\pgfqpoint{5.585226in}{0.794643in}}%
\pgfpathlineto{\pgfqpoint{5.585522in}{0.794641in}}%
\pgfpathlineto{\pgfqpoint{5.585818in}{0.794640in}}%
\pgfpathlineto{\pgfqpoint{5.586114in}{0.794638in}}%
\pgfpathlineto{\pgfqpoint{5.586410in}{0.794636in}}%
\pgfpathlineto{\pgfqpoint{5.586706in}{0.794634in}}%
\pgfpathlineto{\pgfqpoint{5.587002in}{0.794632in}}%
\pgfpathlineto{\pgfqpoint{5.587298in}{0.794630in}}%
\pgfpathlineto{\pgfqpoint{5.587594in}{0.794628in}}%
\pgfpathlineto{\pgfqpoint{5.587890in}{0.794626in}}%
\pgfpathlineto{\pgfqpoint{5.588186in}{0.794624in}}%
\pgfpathlineto{\pgfqpoint{5.588482in}{0.794622in}}%
\pgfpathlineto{\pgfqpoint{5.588778in}{0.794620in}}%
\pgfpathlineto{\pgfqpoint{5.589074in}{0.794618in}}%
\pgfpathlineto{\pgfqpoint{5.589370in}{0.794616in}}%
\pgfpathlineto{\pgfqpoint{5.589666in}{0.794615in}}%
\pgfpathlineto{\pgfqpoint{5.589962in}{0.794613in}}%
\pgfpathlineto{\pgfqpoint{5.590258in}{0.794611in}}%
\pgfpathlineto{\pgfqpoint{5.590554in}{0.794608in}}%
\pgfpathlineto{\pgfqpoint{5.590850in}{0.794628in}}%
\pgfpathlineto{\pgfqpoint{5.591146in}{0.794684in}}%
\pgfpathlineto{\pgfqpoint{5.591442in}{0.794741in}}%
\pgfpathlineto{\pgfqpoint{5.591738in}{0.794797in}}%
\pgfpathlineto{\pgfqpoint{5.592034in}{0.794854in}}%
\pgfpathlineto{\pgfqpoint{5.592330in}{0.794911in}}%
\pgfpathlineto{\pgfqpoint{5.592626in}{0.794968in}}%
\pgfpathlineto{\pgfqpoint{5.592922in}{0.795024in}}%
\pgfpathlineto{\pgfqpoint{5.593218in}{0.795081in}}%
\pgfpathlineto{\pgfqpoint{5.593514in}{0.795052in}}%
\pgfpathlineto{\pgfqpoint{5.593810in}{0.794851in}}%
\pgfpathlineto{\pgfqpoint{5.594106in}{0.794644in}}%
\pgfpathlineto{\pgfqpoint{5.594402in}{0.794438in}}%
\pgfpathlineto{\pgfqpoint{5.594698in}{0.794232in}}%
\pgfpathlineto{\pgfqpoint{5.594994in}{0.794025in}}%
\pgfpathlineto{\pgfqpoint{5.595290in}{0.793819in}}%
\pgfpathlineto{\pgfqpoint{5.595586in}{0.793613in}}%
\pgfpathlineto{\pgfqpoint{5.595882in}{0.793406in}}%
\pgfpathlineto{\pgfqpoint{5.596178in}{0.793200in}}%
\pgfpathlineto{\pgfqpoint{5.596474in}{0.792994in}}%
\pgfpathlineto{\pgfqpoint{5.596770in}{0.792787in}}%
\pgfpathlineto{\pgfqpoint{5.597066in}{0.792581in}}%
\pgfpathlineto{\pgfqpoint{5.597362in}{0.792375in}}%
\pgfpathlineto{\pgfqpoint{5.597658in}{0.792168in}}%
\pgfpathlineto{\pgfqpoint{5.597954in}{0.791962in}}%
\pgfpathlineto{\pgfqpoint{5.598250in}{0.791756in}}%
\pgfpathlineto{\pgfqpoint{5.598546in}{0.791549in}}%
\pgfpathlineto{\pgfqpoint{5.598842in}{0.791343in}}%
\pgfpathlineto{\pgfqpoint{5.599138in}{0.791137in}}%
\pgfpathlineto{\pgfqpoint{5.599434in}{0.790930in}}%
\pgfpathlineto{\pgfqpoint{5.599730in}{0.790724in}}%
\pgfpathlineto{\pgfqpoint{5.600026in}{0.790518in}}%
\pgfpathlineto{\pgfqpoint{5.600322in}{0.790311in}}%
\pgfpathlineto{\pgfqpoint{5.600618in}{0.790296in}}%
\pgfpathlineto{\pgfqpoint{5.600914in}{0.790586in}}%
\pgfpathlineto{\pgfqpoint{5.601210in}{0.790879in}}%
\pgfpathlineto{\pgfqpoint{5.601506in}{0.791172in}}%
\pgfpathlineto{\pgfqpoint{5.601802in}{0.791466in}}%
\pgfpathlineto{\pgfqpoint{5.602098in}{0.791759in}}%
\pgfpathlineto{\pgfqpoint{5.602394in}{0.792052in}}%
\pgfpathlineto{\pgfqpoint{5.602690in}{0.792346in}}%
\pgfpathlineto{\pgfqpoint{5.602986in}{0.792639in}}%
\pgfpathlineto{\pgfqpoint{5.603282in}{0.792932in}}%
\pgfpathlineto{\pgfqpoint{5.603578in}{0.793225in}}%
\pgfpathlineto{\pgfqpoint{5.603874in}{0.793519in}}%
\pgfpathlineto{\pgfqpoint{5.604170in}{0.793812in}}%
\pgfpathlineto{\pgfqpoint{5.604466in}{0.794105in}}%
\pgfpathlineto{\pgfqpoint{5.604762in}{0.794397in}}%
\pgfpathlineto{\pgfqpoint{5.605058in}{0.794508in}}%
\pgfpathlineto{\pgfqpoint{5.605354in}{0.794499in}}%
\pgfpathlineto{\pgfqpoint{5.605650in}{0.794490in}}%
\pgfpathlineto{\pgfqpoint{5.605946in}{0.794481in}}%
\pgfpathlineto{\pgfqpoint{5.606242in}{0.794274in}}%
\pgfpathlineto{\pgfqpoint{5.606538in}{0.791597in}}%
\pgfpathlineto{\pgfqpoint{5.606834in}{0.788183in}}%
\pgfpathlineto{\pgfqpoint{5.607130in}{0.785508in}}%
\pgfpathlineto{\pgfqpoint{5.607426in}{0.785306in}}%
\pgfpathlineto{\pgfqpoint{5.607722in}{0.785302in}}%
\pgfpathlineto{\pgfqpoint{5.608018in}{0.785299in}}%
\pgfpathlineto{\pgfqpoint{5.608314in}{0.785295in}}%
\pgfpathlineto{\pgfqpoint{5.608610in}{0.785292in}}%
\pgfpathlineto{\pgfqpoint{5.608906in}{0.785288in}}%
\pgfpathlineto{\pgfqpoint{5.609202in}{0.785285in}}%
\pgfpathlineto{\pgfqpoint{5.609498in}{0.785281in}}%
\pgfpathlineto{\pgfqpoint{5.609794in}{0.785277in}}%
\pgfpathlineto{\pgfqpoint{5.610090in}{0.785274in}}%
\pgfpathlineto{\pgfqpoint{5.610386in}{0.785270in}}%
\pgfpathlineto{\pgfqpoint{5.610682in}{0.785267in}}%
\pgfpathlineto{\pgfqpoint{5.610978in}{0.785263in}}%
\pgfpathlineto{\pgfqpoint{5.611274in}{0.785260in}}%
\pgfpathlineto{\pgfqpoint{5.611570in}{0.785256in}}%
\pgfpathlineto{\pgfqpoint{5.611866in}{0.785253in}}%
\pgfpathlineto{\pgfqpoint{5.612162in}{0.785249in}}%
\pgfpathlineto{\pgfqpoint{5.612458in}{0.785245in}}%
\pgfpathlineto{\pgfqpoint{5.612754in}{0.785242in}}%
\pgfpathlineto{\pgfqpoint{5.613050in}{0.785238in}}%
\pgfpathlineto{\pgfqpoint{5.613346in}{0.785235in}}%
\pgfpathlineto{\pgfqpoint{5.613642in}{0.785231in}}%
\pgfpathlineto{\pgfqpoint{5.613938in}{0.785228in}}%
\pgfpathlineto{\pgfqpoint{5.614234in}{0.785224in}}%
\pgfpathlineto{\pgfqpoint{5.614530in}{0.785221in}}%
\pgfpathlineto{\pgfqpoint{5.614826in}{0.785217in}}%
\pgfpathlineto{\pgfqpoint{5.615122in}{0.785214in}}%
\pgfpathlineto{\pgfqpoint{5.615418in}{0.785210in}}%
\pgfpathlineto{\pgfqpoint{5.615714in}{0.785206in}}%
\pgfpathlineto{\pgfqpoint{5.616010in}{0.785203in}}%
\pgfpathlineto{\pgfqpoint{5.616306in}{0.785199in}}%
\pgfpathlineto{\pgfqpoint{5.616602in}{0.785196in}}%
\pgfpathlineto{\pgfqpoint{5.616898in}{0.785192in}}%
\pgfpathlineto{\pgfqpoint{5.617194in}{0.785189in}}%
\pgfpathlineto{\pgfqpoint{5.617490in}{0.785185in}}%
\pgfpathlineto{\pgfqpoint{5.617786in}{0.785182in}}%
\pgfpathlineto{\pgfqpoint{5.618082in}{0.785178in}}%
\pgfpathlineto{\pgfqpoint{5.618378in}{0.785175in}}%
\pgfpathlineto{\pgfqpoint{5.618674in}{0.785171in}}%
\pgfpathlineto{\pgfqpoint{5.618970in}{0.785167in}}%
\pgfpathlineto{\pgfqpoint{5.619266in}{0.785164in}}%
\pgfpathlineto{\pgfqpoint{5.619562in}{0.785160in}}%
\pgfpathlineto{\pgfqpoint{5.619858in}{0.785155in}}%
\pgfpathlineto{\pgfqpoint{5.620154in}{0.785150in}}%
\pgfpathlineto{\pgfqpoint{5.620450in}{0.785145in}}%
\pgfpathlineto{\pgfqpoint{5.620746in}{0.785140in}}%
\pgfpathlineto{\pgfqpoint{5.621042in}{0.785136in}}%
\pgfpathlineto{\pgfqpoint{5.621338in}{0.785131in}}%
\pgfpathlineto{\pgfqpoint{5.621634in}{0.785126in}}%
\pgfpathlineto{\pgfqpoint{5.621930in}{0.785121in}}%
\pgfpathlineto{\pgfqpoint{5.622226in}{0.785116in}}%
\pgfpathlineto{\pgfqpoint{5.622522in}{0.785111in}}%
\pgfpathlineto{\pgfqpoint{5.622818in}{0.785106in}}%
\pgfpathlineto{\pgfqpoint{5.623114in}{0.785101in}}%
\pgfpathlineto{\pgfqpoint{5.623410in}{0.785096in}}%
\pgfpathlineto{\pgfqpoint{5.623706in}{0.785091in}}%
\pgfpathlineto{\pgfqpoint{5.624002in}{0.785086in}}%
\pgfpathlineto{\pgfqpoint{5.624298in}{0.785081in}}%
\pgfpathlineto{\pgfqpoint{5.624594in}{0.785076in}}%
\pgfpathlineto{\pgfqpoint{5.624890in}{0.785071in}}%
\pgfpathlineto{\pgfqpoint{5.625186in}{0.785066in}}%
\pgfpathlineto{\pgfqpoint{5.625482in}{0.785061in}}%
\pgfpathlineto{\pgfqpoint{5.625778in}{0.785056in}}%
\pgfpathlineto{\pgfqpoint{5.626074in}{0.785051in}}%
\pgfpathlineto{\pgfqpoint{5.626370in}{0.785046in}}%
\pgfpathlineto{\pgfqpoint{5.626666in}{0.785041in}}%
\pgfpathlineto{\pgfqpoint{5.626962in}{0.785036in}}%
\pgfpathlineto{\pgfqpoint{5.627258in}{0.785031in}}%
\pgfpathlineto{\pgfqpoint{5.627554in}{0.785026in}}%
\pgfpathlineto{\pgfqpoint{5.627850in}{0.785021in}}%
\pgfpathlineto{\pgfqpoint{5.628146in}{0.785016in}}%
\pgfpathlineto{\pgfqpoint{5.628442in}{0.785011in}}%
\pgfpathlineto{\pgfqpoint{5.628738in}{0.785006in}}%
\pgfpathlineto{\pgfqpoint{5.629034in}{0.785002in}}%
\pgfpathlineto{\pgfqpoint{5.629330in}{0.784997in}}%
\pgfpathlineto{\pgfqpoint{5.629626in}{0.784992in}}%
\pgfpathlineto{\pgfqpoint{5.629922in}{0.784987in}}%
\pgfpathlineto{\pgfqpoint{5.630218in}{0.784982in}}%
\pgfpathlineto{\pgfqpoint{5.630514in}{0.784977in}}%
\pgfpathlineto{\pgfqpoint{5.630810in}{0.784972in}}%
\pgfpathlineto{\pgfqpoint{5.631106in}{0.784967in}}%
\pgfpathlineto{\pgfqpoint{5.631402in}{0.784962in}}%
\pgfpathlineto{\pgfqpoint{5.631698in}{0.784957in}}%
\pgfpathlineto{\pgfqpoint{5.631994in}{0.784952in}}%
\pgfpathlineto{\pgfqpoint{5.632290in}{0.784947in}}%
\pgfpathlineto{\pgfqpoint{5.632586in}{0.784942in}}%
\pgfpathlineto{\pgfqpoint{5.632882in}{0.784937in}}%
\pgfpathlineto{\pgfqpoint{5.633178in}{0.784932in}}%
\pgfpathlineto{\pgfqpoint{5.633474in}{0.784927in}}%
\pgfpathlineto{\pgfqpoint{5.633770in}{0.784922in}}%
\pgfpathlineto{\pgfqpoint{5.634066in}{0.784917in}}%
\pgfpathlineto{\pgfqpoint{5.634362in}{0.784912in}}%
\pgfpathlineto{\pgfqpoint{5.634658in}{0.784907in}}%
\pgfpathlineto{\pgfqpoint{5.634954in}{0.784902in}}%
\pgfpathlineto{\pgfqpoint{5.635250in}{0.784897in}}%
\pgfpathlineto{\pgfqpoint{5.635546in}{0.784892in}}%
\pgfpathlineto{\pgfqpoint{5.635842in}{0.784887in}}%
\pgfpathlineto{\pgfqpoint{5.636138in}{0.784882in}}%
\pgfpathlineto{\pgfqpoint{5.636434in}{0.784877in}}%
\pgfpathlineto{\pgfqpoint{5.636731in}{0.784872in}}%
\pgfpathlineto{\pgfqpoint{5.637027in}{0.784868in}}%
\pgfpathlineto{\pgfqpoint{5.637323in}{0.784863in}}%
\pgfpathlineto{\pgfqpoint{5.637619in}{0.784858in}}%
\pgfpathlineto{\pgfqpoint{5.637915in}{0.784853in}}%
\pgfpathlineto{\pgfqpoint{5.638211in}{0.784848in}}%
\pgfpathlineto{\pgfqpoint{5.638507in}{0.784843in}}%
\pgfpathlineto{\pgfqpoint{5.638803in}{0.784838in}}%
\pgfpathlineto{\pgfqpoint{5.639099in}{0.784833in}}%
\pgfpathlineto{\pgfqpoint{5.639395in}{0.784828in}}%
\pgfpathlineto{\pgfqpoint{5.639691in}{0.784823in}}%
\pgfpathlineto{\pgfqpoint{5.639987in}{0.784818in}}%
\pgfpathlineto{\pgfqpoint{5.640283in}{0.784813in}}%
\pgfpathlineto{\pgfqpoint{5.640579in}{0.784808in}}%
\pgfpathlineto{\pgfqpoint{5.640875in}{0.784803in}}%
\pgfpathlineto{\pgfqpoint{5.641171in}{0.784798in}}%
\pgfpathlineto{\pgfqpoint{5.641467in}{0.784793in}}%
\pgfpathlineto{\pgfqpoint{5.641763in}{0.784788in}}%
\pgfpathlineto{\pgfqpoint{5.642059in}{0.784783in}}%
\pgfpathlineto{\pgfqpoint{5.642355in}{0.784778in}}%
\pgfpathlineto{\pgfqpoint{5.642651in}{0.784773in}}%
\pgfpathlineto{\pgfqpoint{5.642947in}{0.784768in}}%
\pgfpathlineto{\pgfqpoint{5.643243in}{0.784774in}}%
\pgfpathlineto{\pgfqpoint{5.643539in}{0.785070in}}%
\pgfpathlineto{\pgfqpoint{5.643835in}{0.785577in}}%
\pgfpathlineto{\pgfqpoint{5.644131in}{0.786084in}}%
\pgfpathlineto{\pgfqpoint{5.644427in}{0.786590in}}%
\pgfpathlineto{\pgfqpoint{5.644723in}{0.787097in}}%
\pgfpathlineto{\pgfqpoint{5.645019in}{0.787604in}}%
\pgfpathlineto{\pgfqpoint{5.645315in}{0.788111in}}%
\pgfpathlineto{\pgfqpoint{5.645611in}{0.788618in}}%
\pgfpathlineto{\pgfqpoint{5.645907in}{0.789124in}}%
\pgfpathlineto{\pgfqpoint{5.646203in}{0.789631in}}%
\pgfpathlineto{\pgfqpoint{5.646499in}{0.790138in}}%
\pgfpathlineto{\pgfqpoint{5.646795in}{0.790645in}}%
\pgfpathlineto{\pgfqpoint{5.647091in}{0.791151in}}%
\pgfpathlineto{\pgfqpoint{5.647387in}{0.791519in}}%
\pgfpathlineto{\pgfqpoint{5.647683in}{0.791652in}}%
\pgfpathlineto{\pgfqpoint{5.647979in}{0.791874in}}%
\pgfpathlineto{\pgfqpoint{5.648275in}{0.792358in}}%
\pgfpathlineto{\pgfqpoint{5.648571in}{0.792858in}}%
\pgfpathlineto{\pgfqpoint{5.648867in}{0.793358in}}%
\pgfpathlineto{\pgfqpoint{5.649163in}{0.793758in}}%
\pgfpathlineto{\pgfqpoint{5.649459in}{0.787331in}}%
\pgfpathlineto{\pgfqpoint{5.649755in}{0.793249in}}%
\pgfpathlineto{\pgfqpoint{5.650051in}{0.793990in}}%
\pgfpathlineto{\pgfqpoint{5.650347in}{0.793925in}}%
\pgfpathlineto{\pgfqpoint{5.650643in}{0.793413in}}%
\pgfpathlineto{\pgfqpoint{5.650939in}{0.792531in}}%
\pgfpathlineto{\pgfqpoint{5.651235in}{0.791429in}}%
\pgfpathlineto{\pgfqpoint{5.651531in}{0.790731in}}%
\pgfpathlineto{\pgfqpoint{5.651827in}{0.790082in}}%
\pgfpathlineto{\pgfqpoint{5.652123in}{0.789432in}}%
\pgfpathlineto{\pgfqpoint{5.652419in}{0.788783in}}%
\pgfpathlineto{\pgfqpoint{5.652715in}{0.788134in}}%
\pgfpathlineto{\pgfqpoint{5.653011in}{0.787485in}}%
\pgfpathlineto{\pgfqpoint{5.653307in}{0.786836in}}%
\pgfpathlineto{\pgfqpoint{5.653603in}{0.786187in}}%
\pgfpathlineto{\pgfqpoint{5.653899in}{0.785538in}}%
\pgfpathlineto{\pgfqpoint{5.654195in}{0.784888in}}%
\pgfpathlineto{\pgfqpoint{5.654491in}{0.784239in}}%
\pgfpathlineto{\pgfqpoint{5.654787in}{0.783590in}}%
\pgfpathlineto{\pgfqpoint{5.655083in}{0.783123in}}%
\pgfpathlineto{\pgfqpoint{5.655379in}{0.783924in}}%
\pgfpathlineto{\pgfqpoint{5.655675in}{0.786893in}}%
\pgfpathlineto{\pgfqpoint{5.655971in}{0.786292in}}%
\pgfpathlineto{\pgfqpoint{5.656267in}{0.784509in}}%
\pgfpathlineto{\pgfqpoint{5.656563in}{0.782840in}}%
\pgfpathlineto{\pgfqpoint{5.656859in}{0.782444in}}%
\pgfpathlineto{\pgfqpoint{5.657155in}{0.782387in}}%
\pgfpathlineto{\pgfqpoint{5.657451in}{0.782330in}}%
\pgfpathlineto{\pgfqpoint{5.657747in}{0.782273in}}%
\pgfpathlineto{\pgfqpoint{5.658043in}{0.782216in}}%
\pgfpathlineto{\pgfqpoint{5.658339in}{0.782159in}}%
\pgfpathlineto{\pgfqpoint{5.658635in}{0.782102in}}%
\pgfpathlineto{\pgfqpoint{5.658931in}{0.782045in}}%
\pgfpathlineto{\pgfqpoint{5.659227in}{0.781988in}}%
\pgfpathlineto{\pgfqpoint{5.659523in}{0.781931in}}%
\pgfpathlineto{\pgfqpoint{5.659819in}{0.781874in}}%
\pgfpathlineto{\pgfqpoint{5.660115in}{0.781817in}}%
\pgfpathlineto{\pgfqpoint{5.660411in}{0.781760in}}%
\pgfpathlineto{\pgfqpoint{5.660707in}{0.781703in}}%
\pgfpathlineto{\pgfqpoint{5.661003in}{0.781646in}}%
\pgfpathlineto{\pgfqpoint{5.661299in}{0.781589in}}%
\pgfpathlineto{\pgfqpoint{5.661595in}{0.781532in}}%
\pgfpathlineto{\pgfqpoint{5.661891in}{0.781475in}}%
\pgfpathlineto{\pgfqpoint{5.662187in}{0.781418in}}%
\pgfpathlineto{\pgfqpoint{5.662483in}{0.781361in}}%
\pgfpathlineto{\pgfqpoint{5.662779in}{0.781304in}}%
\pgfpathlineto{\pgfqpoint{5.663075in}{0.781247in}}%
\pgfpathlineto{\pgfqpoint{5.663371in}{0.781190in}}%
\pgfpathlineto{\pgfqpoint{5.663667in}{0.781134in}}%
\pgfpathlineto{\pgfqpoint{5.663963in}{0.781105in}}%
\pgfpathlineto{\pgfqpoint{5.664259in}{0.781097in}}%
\pgfpathlineto{\pgfqpoint{5.664555in}{0.781089in}}%
\pgfpathlineto{\pgfqpoint{5.664851in}{0.781081in}}%
\pgfpathlineto{\pgfqpoint{5.665147in}{0.781073in}}%
\pgfpathlineto{\pgfqpoint{5.665443in}{0.781065in}}%
\pgfpathlineto{\pgfqpoint{5.665739in}{0.781057in}}%
\pgfpathlineto{\pgfqpoint{5.666035in}{0.781049in}}%
\pgfpathlineto{\pgfqpoint{5.666331in}{0.781041in}}%
\pgfpathlineto{\pgfqpoint{5.666627in}{0.781033in}}%
\pgfpathlineto{\pgfqpoint{5.666923in}{0.781025in}}%
\pgfpathlineto{\pgfqpoint{5.667219in}{0.781017in}}%
\pgfpathlineto{\pgfqpoint{5.667515in}{0.781009in}}%
\pgfpathlineto{\pgfqpoint{5.667811in}{0.781001in}}%
\pgfpathlineto{\pgfqpoint{5.668107in}{0.780993in}}%
\pgfpathlineto{\pgfqpoint{5.668403in}{0.780985in}}%
\pgfpathlineto{\pgfqpoint{5.668699in}{0.780978in}}%
\pgfpathlineto{\pgfqpoint{5.668995in}{0.780975in}}%
\pgfpathlineto{\pgfqpoint{5.669291in}{0.780973in}}%
\pgfpathlineto{\pgfqpoint{5.669587in}{0.780966in}}%
\pgfpathlineto{\pgfqpoint{5.669883in}{0.780957in}}%
\pgfpathlineto{\pgfqpoint{5.670179in}{0.780949in}}%
\pgfpathlineto{\pgfqpoint{5.670475in}{0.780941in}}%
\pgfpathlineto{\pgfqpoint{5.670771in}{0.780932in}}%
\pgfpathlineto{\pgfqpoint{5.671067in}{0.780924in}}%
\pgfpathlineto{\pgfqpoint{5.671363in}{0.780915in}}%
\pgfpathlineto{\pgfqpoint{5.671659in}{0.780907in}}%
\pgfpathlineto{\pgfqpoint{5.671955in}{0.780899in}}%
\pgfpathlineto{\pgfqpoint{5.672251in}{0.780890in}}%
\pgfpathlineto{\pgfqpoint{5.672547in}{0.780882in}}%
\pgfpathlineto{\pgfqpoint{5.672843in}{0.780873in}}%
\pgfpathlineto{\pgfqpoint{5.673139in}{0.780865in}}%
\pgfpathlineto{\pgfqpoint{5.673435in}{0.780857in}}%
\pgfpathlineto{\pgfqpoint{5.673731in}{0.780848in}}%
\pgfpathlineto{\pgfqpoint{5.674027in}{0.780840in}}%
\pgfpathlineto{\pgfqpoint{5.674323in}{0.780831in}}%
\pgfpathlineto{\pgfqpoint{5.674619in}{0.780823in}}%
\pgfpathlineto{\pgfqpoint{5.674915in}{0.780815in}}%
\pgfpathlineto{\pgfqpoint{5.675211in}{0.780806in}}%
\pgfpathlineto{\pgfqpoint{5.675507in}{0.780798in}}%
\pgfpathlineto{\pgfqpoint{5.675803in}{0.780789in}}%
\pgfpathlineto{\pgfqpoint{5.676099in}{0.780781in}}%
\pgfpathlineto{\pgfqpoint{5.676395in}{0.780773in}}%
\pgfpathlineto{\pgfqpoint{5.676691in}{0.780764in}}%
\pgfpathlineto{\pgfqpoint{5.676987in}{0.780756in}}%
\pgfpathlineto{\pgfqpoint{5.677283in}{0.780747in}}%
\pgfpathlineto{\pgfqpoint{5.677579in}{0.780739in}}%
\pgfpathlineto{\pgfqpoint{5.677875in}{0.780731in}}%
\pgfpathlineto{\pgfqpoint{5.678171in}{0.780722in}}%
\pgfpathlineto{\pgfqpoint{5.678467in}{0.780714in}}%
\pgfpathlineto{\pgfqpoint{5.678763in}{0.780705in}}%
\pgfpathlineto{\pgfqpoint{5.679059in}{0.780697in}}%
\pgfpathlineto{\pgfqpoint{5.679355in}{0.780689in}}%
\pgfpathlineto{\pgfqpoint{5.679651in}{0.780680in}}%
\pgfpathlineto{\pgfqpoint{5.679947in}{0.780672in}}%
\pgfpathlineto{\pgfqpoint{5.680243in}{0.780663in}}%
\pgfpathlineto{\pgfqpoint{5.680539in}{0.780655in}}%
\pgfpathlineto{\pgfqpoint{5.680835in}{0.780647in}}%
\pgfpathlineto{\pgfqpoint{5.681131in}{0.780638in}}%
\pgfpathlineto{\pgfqpoint{5.681427in}{0.780630in}}%
\pgfpathlineto{\pgfqpoint{5.681723in}{0.780621in}}%
\pgfpathlineto{\pgfqpoint{5.682019in}{0.780613in}}%
\pgfpathlineto{\pgfqpoint{5.682315in}{0.780605in}}%
\pgfpathlineto{\pgfqpoint{5.682611in}{0.780596in}}%
\pgfpathlineto{\pgfqpoint{5.682907in}{0.780588in}}%
\pgfpathlineto{\pgfqpoint{5.683203in}{0.780579in}}%
\pgfpathlineto{\pgfqpoint{5.683499in}{0.780571in}}%
\pgfpathlineto{\pgfqpoint{5.683795in}{0.780563in}}%
\pgfpathlineto{\pgfqpoint{5.684091in}{0.780554in}}%
\pgfpathlineto{\pgfqpoint{5.684387in}{0.780546in}}%
\pgfpathlineto{\pgfqpoint{5.684683in}{0.780537in}}%
\pgfpathlineto{\pgfqpoint{5.684979in}{0.780529in}}%
\pgfpathlineto{\pgfqpoint{5.685275in}{0.780521in}}%
\pgfpathlineto{\pgfqpoint{5.685571in}{0.780512in}}%
\pgfpathlineto{\pgfqpoint{5.685867in}{0.780504in}}%
\pgfpathlineto{\pgfqpoint{5.686163in}{0.780495in}}%
\pgfpathlineto{\pgfqpoint{5.686459in}{0.780487in}}%
\pgfpathlineto{\pgfqpoint{5.686755in}{0.780479in}}%
\pgfpathlineto{\pgfqpoint{5.687051in}{0.780470in}}%
\pgfpathlineto{\pgfqpoint{5.687347in}{0.780462in}}%
\pgfpathlineto{\pgfqpoint{5.687643in}{0.780453in}}%
\pgfpathlineto{\pgfqpoint{5.687939in}{0.780445in}}%
\pgfpathlineto{\pgfqpoint{5.688235in}{0.780437in}}%
\pgfpathlineto{\pgfqpoint{5.688531in}{0.780428in}}%
\pgfpathlineto{\pgfqpoint{5.688827in}{0.780420in}}%
\pgfpathlineto{\pgfqpoint{5.689123in}{0.780411in}}%
\pgfpathlineto{\pgfqpoint{5.689419in}{0.780403in}}%
\pgfpathlineto{\pgfqpoint{5.689715in}{0.780395in}}%
\pgfpathlineto{\pgfqpoint{5.690011in}{0.779947in}}%
\pgfpathlineto{\pgfqpoint{5.690307in}{0.779864in}}%
\pgfpathlineto{\pgfqpoint{5.690603in}{0.779831in}}%
\pgfpathlineto{\pgfqpoint{5.690899in}{0.778429in}}%
\pgfpathlineto{\pgfqpoint{5.691195in}{0.776168in}}%
\pgfpathlineto{\pgfqpoint{5.691491in}{0.773906in}}%
\pgfpathlineto{\pgfqpoint{5.691787in}{0.771645in}}%
\pgfpathlineto{\pgfqpoint{5.692083in}{0.769383in}}%
\pgfpathlineto{\pgfqpoint{5.692379in}{0.767122in}}%
\pgfpathlineto{\pgfqpoint{5.692675in}{0.765509in}}%
\pgfpathlineto{\pgfqpoint{5.692971in}{0.765436in}}%
\pgfpathlineto{\pgfqpoint{5.693267in}{0.765428in}}%
\pgfpathlineto{\pgfqpoint{5.693563in}{0.765421in}}%
\pgfpathlineto{\pgfqpoint{5.693859in}{0.765414in}}%
\pgfpathlineto{\pgfqpoint{5.694155in}{0.765407in}}%
\pgfpathlineto{\pgfqpoint{5.694451in}{0.765399in}}%
\pgfpathlineto{\pgfqpoint{5.694747in}{0.765392in}}%
\pgfpathlineto{\pgfqpoint{5.695043in}{0.765385in}}%
\pgfpathlineto{\pgfqpoint{5.695339in}{0.765378in}}%
\pgfpathlineto{\pgfqpoint{5.695635in}{0.765370in}}%
\pgfpathlineto{\pgfqpoint{5.695931in}{0.765363in}}%
\pgfpathlineto{\pgfqpoint{5.696227in}{0.765356in}}%
\pgfpathlineto{\pgfqpoint{5.696523in}{0.765349in}}%
\pgfpathlineto{\pgfqpoint{5.696819in}{0.765341in}}%
\pgfpathlineto{\pgfqpoint{5.697115in}{0.765334in}}%
\pgfpathlineto{\pgfqpoint{5.697411in}{0.765327in}}%
\pgfpathlineto{\pgfqpoint{5.697707in}{0.765319in}}%
\pgfpathlineto{\pgfqpoint{5.698003in}{0.765312in}}%
\pgfpathlineto{\pgfqpoint{5.698299in}{0.765347in}}%
\pgfpathlineto{\pgfqpoint{5.698595in}{0.767998in}}%
\pgfpathlineto{\pgfqpoint{5.698891in}{0.772200in}}%
\pgfpathlineto{\pgfqpoint{5.699187in}{0.776402in}}%
\pgfpathlineto{\pgfqpoint{5.699483in}{0.768487in}}%
\pgfpathlineto{\pgfqpoint{5.699779in}{0.765275in}}%
\pgfpathlineto{\pgfqpoint{5.700075in}{0.765260in}}%
\pgfpathlineto{\pgfqpoint{5.700371in}{0.765245in}}%
\pgfpathlineto{\pgfqpoint{5.700667in}{0.765230in}}%
\pgfpathlineto{\pgfqpoint{5.700963in}{0.765214in}}%
\pgfpathlineto{\pgfqpoint{5.701259in}{0.765199in}}%
\pgfpathlineto{\pgfqpoint{5.701555in}{0.765184in}}%
\pgfpathlineto{\pgfqpoint{5.701851in}{0.765169in}}%
\pgfpathlineto{\pgfqpoint{5.702147in}{0.765154in}}%
\pgfpathlineto{\pgfqpoint{5.702443in}{0.765138in}}%
\pgfpathlineto{\pgfqpoint{5.702739in}{0.765123in}}%
\pgfpathlineto{\pgfqpoint{5.703035in}{0.765108in}}%
\pgfpathlineto{\pgfqpoint{5.703331in}{0.765093in}}%
\pgfpathlineto{\pgfqpoint{5.703627in}{0.765078in}}%
\pgfpathlineto{\pgfqpoint{5.703923in}{0.765062in}}%
\pgfpathlineto{\pgfqpoint{5.704220in}{0.765047in}}%
\pgfpathlineto{\pgfqpoint{5.704516in}{0.765032in}}%
\pgfpathlineto{\pgfqpoint{5.704812in}{0.765017in}}%
\pgfpathlineto{\pgfqpoint{5.705108in}{0.765001in}}%
\pgfpathlineto{\pgfqpoint{5.705404in}{0.764986in}}%
\pgfpathlineto{\pgfqpoint{5.705700in}{0.764971in}}%
\pgfpathlineto{\pgfqpoint{5.705996in}{0.764956in}}%
\pgfpathlineto{\pgfqpoint{5.706292in}{0.764941in}}%
\pgfpathlineto{\pgfqpoint{5.706588in}{0.764925in}}%
\pgfpathlineto{\pgfqpoint{5.706884in}{0.764910in}}%
\pgfpathlineto{\pgfqpoint{5.707180in}{0.764895in}}%
\pgfpathlineto{\pgfqpoint{5.707476in}{0.764880in}}%
\pgfpathlineto{\pgfqpoint{5.707772in}{0.764865in}}%
\pgfpathlineto{\pgfqpoint{5.708068in}{0.764849in}}%
\pgfpathlineto{\pgfqpoint{5.708364in}{0.764834in}}%
\pgfpathlineto{\pgfqpoint{5.708660in}{0.764819in}}%
\pgfpathlineto{\pgfqpoint{5.708956in}{0.764804in}}%
\pgfpathlineto{\pgfqpoint{5.709252in}{0.764789in}}%
\pgfpathlineto{\pgfqpoint{5.709548in}{0.764773in}}%
\pgfpathlineto{\pgfqpoint{5.709844in}{0.764758in}}%
\pgfpathlineto{\pgfqpoint{5.710140in}{0.764743in}}%
\pgfpathlineto{\pgfqpoint{5.710436in}{0.764728in}}%
\pgfpathlineto{\pgfqpoint{5.710732in}{0.764713in}}%
\pgfpathlineto{\pgfqpoint{5.711028in}{0.764697in}}%
\pgfpathlineto{\pgfqpoint{5.711324in}{0.764682in}}%
\pgfpathlineto{\pgfqpoint{5.711620in}{0.764667in}}%
\pgfpathlineto{\pgfqpoint{5.711916in}{0.764652in}}%
\pgfpathlineto{\pgfqpoint{5.712212in}{0.764637in}}%
\pgfpathlineto{\pgfqpoint{5.712508in}{0.764621in}}%
\pgfpathlineto{\pgfqpoint{5.712804in}{0.764606in}}%
\pgfpathlineto{\pgfqpoint{5.713100in}{0.764591in}}%
\pgfpathlineto{\pgfqpoint{5.713396in}{0.764575in}}%
\pgfpathlineto{\pgfqpoint{5.713692in}{0.764556in}}%
\pgfpathlineto{\pgfqpoint{5.713988in}{0.764537in}}%
\pgfpathlineto{\pgfqpoint{5.714284in}{0.764518in}}%
\pgfpathlineto{\pgfqpoint{5.714580in}{0.764499in}}%
\pgfpathlineto{\pgfqpoint{5.714876in}{0.764480in}}%
\pgfpathlineto{\pgfqpoint{5.715172in}{0.764461in}}%
\pgfpathlineto{\pgfqpoint{5.715468in}{0.764442in}}%
\pgfpathlineto{\pgfqpoint{5.715764in}{0.764423in}}%
\pgfpathlineto{\pgfqpoint{5.716060in}{0.764404in}}%
\pgfpathlineto{\pgfqpoint{5.716356in}{0.764384in}}%
\pgfpathlineto{\pgfqpoint{5.716652in}{0.764365in}}%
\pgfpathlineto{\pgfqpoint{5.716948in}{0.764346in}}%
\pgfpathlineto{\pgfqpoint{5.717244in}{0.764327in}}%
\pgfpathlineto{\pgfqpoint{5.717540in}{0.764308in}}%
\pgfpathlineto{\pgfqpoint{5.717836in}{0.764289in}}%
\pgfpathlineto{\pgfqpoint{5.718132in}{0.764304in}}%
\pgfpathlineto{\pgfqpoint{5.718428in}{0.764379in}}%
\pgfpathlineto{\pgfqpoint{5.718724in}{0.764381in}}%
\pgfpathlineto{\pgfqpoint{5.719020in}{0.764361in}}%
\pgfpathlineto{\pgfqpoint{5.719316in}{0.764342in}}%
\pgfpathlineto{\pgfqpoint{5.719612in}{0.764323in}}%
\pgfpathlineto{\pgfqpoint{5.719908in}{0.764303in}}%
\pgfpathlineto{\pgfqpoint{5.720204in}{0.764284in}}%
\pgfpathlineto{\pgfqpoint{5.720500in}{0.764265in}}%
\pgfpathlineto{\pgfqpoint{5.720796in}{0.764245in}}%
\pgfpathlineto{\pgfqpoint{5.721092in}{0.764226in}}%
\pgfpathlineto{\pgfqpoint{5.721388in}{0.764207in}}%
\pgfpathlineto{\pgfqpoint{5.721684in}{0.764187in}}%
\pgfpathlineto{\pgfqpoint{5.721980in}{0.764168in}}%
\pgfpathlineto{\pgfqpoint{5.722276in}{0.764149in}}%
\pgfpathlineto{\pgfqpoint{5.722572in}{0.764129in}}%
\pgfpathlineto{\pgfqpoint{5.722868in}{0.764110in}}%
\pgfpathlineto{\pgfqpoint{5.723164in}{0.764091in}}%
\pgfpathlineto{\pgfqpoint{5.723460in}{0.764071in}}%
\pgfpathlineto{\pgfqpoint{5.723756in}{0.764052in}}%
\pgfpathlineto{\pgfqpoint{5.724052in}{0.764033in}}%
\pgfpathlineto{\pgfqpoint{5.724348in}{0.764013in}}%
\pgfpathlineto{\pgfqpoint{5.724644in}{0.763994in}}%
\pgfpathlineto{\pgfqpoint{5.724940in}{0.763975in}}%
\pgfpathlineto{\pgfqpoint{5.725236in}{0.763956in}}%
\pgfpathlineto{\pgfqpoint{5.725532in}{0.763936in}}%
\pgfpathlineto{\pgfqpoint{5.725828in}{0.763917in}}%
\pgfpathlineto{\pgfqpoint{5.726124in}{0.763898in}}%
\pgfpathlineto{\pgfqpoint{5.726420in}{0.763878in}}%
\pgfpathlineto{\pgfqpoint{5.726716in}{0.763859in}}%
\pgfpathlineto{\pgfqpoint{5.727012in}{0.763840in}}%
\pgfpathlineto{\pgfqpoint{5.727308in}{0.763820in}}%
\pgfpathlineto{\pgfqpoint{5.727604in}{0.763801in}}%
\pgfpathlineto{\pgfqpoint{5.727900in}{0.763782in}}%
\pgfpathlineto{\pgfqpoint{5.728196in}{0.763762in}}%
\pgfpathlineto{\pgfqpoint{5.728492in}{0.763743in}}%
\pgfpathlineto{\pgfqpoint{5.728788in}{0.763724in}}%
\pgfpathlineto{\pgfqpoint{5.729084in}{0.763704in}}%
\pgfpathlineto{\pgfqpoint{5.729380in}{0.763685in}}%
\pgfpathlineto{\pgfqpoint{5.729676in}{0.763666in}}%
\pgfpathlineto{\pgfqpoint{5.729972in}{0.763646in}}%
\pgfpathlineto{\pgfqpoint{5.730268in}{0.763627in}}%
\pgfpathlineto{\pgfqpoint{5.730564in}{0.763608in}}%
\pgfpathlineto{\pgfqpoint{5.730860in}{0.763588in}}%
\pgfpathlineto{\pgfqpoint{5.731156in}{0.763569in}}%
\pgfpathlineto{\pgfqpoint{5.731452in}{0.763550in}}%
\pgfpathlineto{\pgfqpoint{5.731748in}{0.763530in}}%
\pgfpathlineto{\pgfqpoint{5.732044in}{0.763511in}}%
\pgfpathlineto{\pgfqpoint{5.732340in}{0.763492in}}%
\pgfpathlineto{\pgfqpoint{5.732636in}{0.763473in}}%
\pgfpathlineto{\pgfqpoint{5.732932in}{0.763453in}}%
\pgfpathlineto{\pgfqpoint{5.733228in}{0.763434in}}%
\pgfpathlineto{\pgfqpoint{5.733524in}{0.763415in}}%
\pgfpathlineto{\pgfqpoint{5.733820in}{0.763395in}}%
\pgfpathlineto{\pgfqpoint{5.734116in}{0.763376in}}%
\pgfpathlineto{\pgfqpoint{5.734412in}{0.763357in}}%
\pgfpathlineto{\pgfqpoint{5.734708in}{0.763337in}}%
\pgfpathlineto{\pgfqpoint{5.735004in}{0.763318in}}%
\pgfpathlineto{\pgfqpoint{5.735300in}{0.763299in}}%
\pgfpathlineto{\pgfqpoint{5.735596in}{0.763279in}}%
\pgfpathlineto{\pgfqpoint{5.735892in}{0.763260in}}%
\pgfpathlineto{\pgfqpoint{5.736188in}{0.763241in}}%
\pgfpathlineto{\pgfqpoint{5.736484in}{0.763221in}}%
\pgfpathlineto{\pgfqpoint{5.736780in}{0.763202in}}%
\pgfpathlineto{\pgfqpoint{5.737076in}{0.763183in}}%
\pgfpathlineto{\pgfqpoint{5.737372in}{0.763163in}}%
\pgfpathlineto{\pgfqpoint{5.737668in}{0.763144in}}%
\pgfpathlineto{\pgfqpoint{5.737964in}{0.763125in}}%
\pgfpathlineto{\pgfqpoint{5.738260in}{0.763105in}}%
\pgfpathlineto{\pgfqpoint{5.738556in}{0.763086in}}%
\pgfpathlineto{\pgfqpoint{5.738852in}{0.763067in}}%
\pgfpathlineto{\pgfqpoint{5.739148in}{0.763047in}}%
\pgfpathlineto{\pgfqpoint{5.739444in}{0.763028in}}%
\pgfpathlineto{\pgfqpoint{5.739740in}{0.763009in}}%
\pgfpathlineto{\pgfqpoint{5.740036in}{0.762985in}}%
\pgfpathlineto{\pgfqpoint{5.740332in}{0.762959in}}%
\pgfpathlineto{\pgfqpoint{5.740628in}{0.762933in}}%
\pgfpathlineto{\pgfqpoint{5.740924in}{0.762907in}}%
\pgfpathlineto{\pgfqpoint{5.741220in}{0.762881in}}%
\pgfpathlineto{\pgfqpoint{5.741516in}{0.762855in}}%
\pgfpathlineto{\pgfqpoint{5.741812in}{0.762829in}}%
\pgfpathlineto{\pgfqpoint{5.742108in}{0.762803in}}%
\pgfpathlineto{\pgfqpoint{5.742404in}{0.762777in}}%
\pgfpathlineto{\pgfqpoint{5.742700in}{0.762751in}}%
\pgfpathlineto{\pgfqpoint{5.742996in}{0.762725in}}%
\pgfpathlineto{\pgfqpoint{5.743292in}{0.762699in}}%
\pgfpathlineto{\pgfqpoint{5.743588in}{0.762673in}}%
\pgfpathlineto{\pgfqpoint{5.743884in}{0.762647in}}%
\pgfpathlineto{\pgfqpoint{5.744180in}{0.762621in}}%
\pgfpathlineto{\pgfqpoint{5.744476in}{0.762595in}}%
\pgfpathlineto{\pgfqpoint{5.744772in}{0.762569in}}%
\pgfpathlineto{\pgfqpoint{5.745068in}{0.762543in}}%
\pgfpathlineto{\pgfqpoint{5.745364in}{0.762517in}}%
\pgfpathlineto{\pgfqpoint{5.745660in}{0.762491in}}%
\pgfpathlineto{\pgfqpoint{5.745956in}{0.762465in}}%
\pgfpathlineto{\pgfqpoint{5.746252in}{0.762439in}}%
\pgfpathlineto{\pgfqpoint{5.746548in}{0.762413in}}%
\pgfpathlineto{\pgfqpoint{5.746844in}{0.762387in}}%
\pgfpathlineto{\pgfqpoint{5.747140in}{0.762361in}}%
\pgfpathlineto{\pgfqpoint{5.747436in}{0.762335in}}%
\pgfpathlineto{\pgfqpoint{5.747732in}{0.762309in}}%
\pgfpathlineto{\pgfqpoint{5.748028in}{0.762283in}}%
\pgfpathlineto{\pgfqpoint{5.748324in}{0.762257in}}%
\pgfpathlineto{\pgfqpoint{5.748620in}{0.762231in}}%
\pgfpathlineto{\pgfqpoint{5.748916in}{0.762206in}}%
\pgfpathlineto{\pgfqpoint{5.749212in}{0.762195in}}%
\pgfpathlineto{\pgfqpoint{5.749508in}{0.762192in}}%
\pgfpathlineto{\pgfqpoint{5.749804in}{0.762189in}}%
\pgfpathlineto{\pgfqpoint{5.750100in}{0.762185in}}%
\pgfpathlineto{\pgfqpoint{5.750396in}{0.762182in}}%
\pgfpathlineto{\pgfqpoint{5.750692in}{0.762179in}}%
\pgfpathlineto{\pgfqpoint{5.750988in}{0.762175in}}%
\pgfpathlineto{\pgfqpoint{5.751284in}{0.762172in}}%
\pgfpathlineto{\pgfqpoint{5.751580in}{0.762169in}}%
\pgfpathlineto{\pgfqpoint{5.751876in}{0.762165in}}%
\pgfpathlineto{\pgfqpoint{5.752172in}{0.762162in}}%
\pgfpathlineto{\pgfqpoint{5.752468in}{0.762158in}}%
\pgfpathlineto{\pgfqpoint{5.752764in}{0.762155in}}%
\pgfpathlineto{\pgfqpoint{5.753060in}{0.762152in}}%
\pgfpathlineto{\pgfqpoint{5.753356in}{0.762148in}}%
\pgfpathlineto{\pgfqpoint{5.753652in}{0.762145in}}%
\pgfpathlineto{\pgfqpoint{5.753948in}{0.762141in}}%
\pgfpathlineto{\pgfqpoint{5.754244in}{0.762138in}}%
\pgfpathlineto{\pgfqpoint{5.754540in}{0.762134in}}%
\pgfpathlineto{\pgfqpoint{5.754836in}{0.762131in}}%
\pgfpathlineto{\pgfqpoint{5.755132in}{0.762127in}}%
\pgfpathlineto{\pgfqpoint{5.755428in}{0.762124in}}%
\pgfpathlineto{\pgfqpoint{5.755724in}{0.761324in}}%
\pgfpathlineto{\pgfqpoint{5.756020in}{0.760780in}}%
\pgfpathlineto{\pgfqpoint{5.756316in}{0.760765in}}%
\pgfpathlineto{\pgfqpoint{5.756612in}{0.760750in}}%
\pgfpathlineto{\pgfqpoint{5.756908in}{0.760735in}}%
\pgfpathlineto{\pgfqpoint{5.757204in}{0.760720in}}%
\pgfpathlineto{\pgfqpoint{5.757500in}{0.760705in}}%
\pgfpathlineto{\pgfqpoint{5.757796in}{0.760690in}}%
\pgfpathlineto{\pgfqpoint{5.758092in}{0.760675in}}%
\pgfpathlineto{\pgfqpoint{5.758388in}{0.760659in}}%
\pgfpathlineto{\pgfqpoint{5.758684in}{0.760644in}}%
\pgfpathlineto{\pgfqpoint{5.758980in}{0.760629in}}%
\pgfpathlineto{\pgfqpoint{5.759276in}{0.760614in}}%
\pgfpathlineto{\pgfqpoint{5.759572in}{0.760599in}}%
\pgfpathlineto{\pgfqpoint{5.759868in}{0.760584in}}%
\pgfpathlineto{\pgfqpoint{5.760164in}{0.760569in}}%
\pgfpathlineto{\pgfqpoint{5.760460in}{0.760554in}}%
\pgfpathlineto{\pgfqpoint{5.760756in}{0.760539in}}%
\pgfpathlineto{\pgfqpoint{5.761052in}{0.760523in}}%
\pgfpathlineto{\pgfqpoint{5.761348in}{0.760508in}}%
\pgfpathlineto{\pgfqpoint{5.761644in}{0.760493in}}%
\pgfpathlineto{\pgfqpoint{5.761940in}{0.760478in}}%
\pgfpathlineto{\pgfqpoint{5.762236in}{0.759893in}}%
\pgfpathlineto{\pgfqpoint{5.762532in}{0.759461in}}%
\pgfpathlineto{\pgfqpoint{5.762828in}{0.759299in}}%
\pgfpathlineto{\pgfqpoint{5.763124in}{0.759158in}}%
\pgfpathlineto{\pgfqpoint{5.763420in}{0.759133in}}%
\pgfpathlineto{\pgfqpoint{5.763716in}{0.759126in}}%
\pgfpathlineto{\pgfqpoint{5.764012in}{0.759120in}}%
\pgfpathlineto{\pgfqpoint{5.764308in}{0.759114in}}%
\pgfpathlineto{\pgfqpoint{5.764604in}{0.759107in}}%
\pgfpathlineto{\pgfqpoint{5.764900in}{0.759101in}}%
\pgfpathlineto{\pgfqpoint{5.765196in}{0.759095in}}%
\pgfpathlineto{\pgfqpoint{5.765492in}{0.759088in}}%
\pgfpathlineto{\pgfqpoint{5.765788in}{0.759082in}}%
\pgfpathlineto{\pgfqpoint{5.766084in}{0.759076in}}%
\pgfpathlineto{\pgfqpoint{5.766380in}{0.759069in}}%
\pgfpathlineto{\pgfqpoint{5.766676in}{0.759063in}}%
\pgfpathlineto{\pgfqpoint{5.766972in}{0.759057in}}%
\pgfpathlineto{\pgfqpoint{5.767268in}{0.759050in}}%
\pgfpathlineto{\pgfqpoint{5.767564in}{0.759044in}}%
\pgfpathlineto{\pgfqpoint{5.767860in}{0.759037in}}%
\pgfpathlineto{\pgfqpoint{5.768156in}{0.759031in}}%
\pgfpathlineto{\pgfqpoint{5.768452in}{0.759025in}}%
\pgfpathlineto{\pgfqpoint{5.768748in}{0.759018in}}%
\pgfpathlineto{\pgfqpoint{5.769044in}{0.759012in}}%
\pgfpathlineto{\pgfqpoint{5.769340in}{0.759006in}}%
\pgfpathlineto{\pgfqpoint{5.769636in}{0.758999in}}%
\pgfpathlineto{\pgfqpoint{5.769932in}{0.758993in}}%
\pgfpathlineto{\pgfqpoint{5.770228in}{0.758987in}}%
\pgfpathlineto{\pgfqpoint{5.770524in}{0.758980in}}%
\pgfpathlineto{\pgfqpoint{5.770820in}{0.758974in}}%
\pgfpathlineto{\pgfqpoint{5.771116in}{0.758968in}}%
\pgfpathlineto{\pgfqpoint{5.771412in}{0.758961in}}%
\pgfpathlineto{\pgfqpoint{5.771709in}{0.758955in}}%
\pgfpathlineto{\pgfqpoint{5.772005in}{0.758949in}}%
\pgfpathlineto{\pgfqpoint{5.772301in}{0.758942in}}%
\pgfpathlineto{\pgfqpoint{5.772597in}{0.758936in}}%
\pgfpathlineto{\pgfqpoint{5.772893in}{0.758930in}}%
\pgfpathlineto{\pgfqpoint{5.773189in}{0.758923in}}%
\pgfpathlineto{\pgfqpoint{5.773485in}{0.758917in}}%
\pgfpathlineto{\pgfqpoint{5.773781in}{0.758911in}}%
\pgfpathlineto{\pgfqpoint{5.774077in}{0.758904in}}%
\pgfpathlineto{\pgfqpoint{5.774373in}{0.758898in}}%
\pgfpathlineto{\pgfqpoint{5.774669in}{0.758892in}}%
\pgfpathlineto{\pgfqpoint{5.774965in}{0.758885in}}%
\pgfpathlineto{\pgfqpoint{5.775261in}{0.758879in}}%
\pgfpathlineto{\pgfqpoint{5.775557in}{0.758873in}}%
\pgfpathlineto{\pgfqpoint{5.775853in}{0.758866in}}%
\pgfpathlineto{\pgfqpoint{5.776149in}{0.758860in}}%
\pgfpathlineto{\pgfqpoint{5.776445in}{0.758854in}}%
\pgfpathlineto{\pgfqpoint{5.776741in}{0.758847in}}%
\pgfpathlineto{\pgfqpoint{5.777037in}{0.758841in}}%
\pgfpathlineto{\pgfqpoint{5.777333in}{0.758835in}}%
\pgfpathlineto{\pgfqpoint{5.777629in}{0.758828in}}%
\pgfpathlineto{\pgfqpoint{5.777925in}{0.758822in}}%
\pgfpathlineto{\pgfqpoint{5.778221in}{0.758816in}}%
\pgfpathlineto{\pgfqpoint{5.778517in}{0.758809in}}%
\pgfpathlineto{\pgfqpoint{5.778813in}{0.758803in}}%
\pgfpathlineto{\pgfqpoint{5.779109in}{0.758796in}}%
\pgfpathlineto{\pgfqpoint{5.779405in}{0.758790in}}%
\pgfpathlineto{\pgfqpoint{5.779701in}{0.758784in}}%
\pgfpathlineto{\pgfqpoint{5.779997in}{0.758777in}}%
\pgfpathlineto{\pgfqpoint{5.780293in}{0.758771in}}%
\pgfpathlineto{\pgfqpoint{5.780589in}{0.758765in}}%
\pgfpathlineto{\pgfqpoint{5.780885in}{0.758758in}}%
\pgfpathlineto{\pgfqpoint{5.781181in}{0.758752in}}%
\pgfpathlineto{\pgfqpoint{5.781477in}{0.758746in}}%
\pgfpathlineto{\pgfqpoint{5.781773in}{0.758739in}}%
\pgfpathlineto{\pgfqpoint{5.782069in}{0.758733in}}%
\pgfpathlineto{\pgfqpoint{5.782365in}{0.758727in}}%
\pgfpathlineto{\pgfqpoint{5.782661in}{0.758720in}}%
\pgfpathlineto{\pgfqpoint{5.782957in}{0.758714in}}%
\pgfpathlineto{\pgfqpoint{5.783253in}{0.758708in}}%
\pgfpathlineto{\pgfqpoint{5.783549in}{0.758701in}}%
\pgfpathlineto{\pgfqpoint{5.783845in}{0.758695in}}%
\pgfpathlineto{\pgfqpoint{5.784141in}{0.758689in}}%
\pgfpathlineto{\pgfqpoint{5.784437in}{0.758682in}}%
\pgfpathlineto{\pgfqpoint{5.784733in}{0.758676in}}%
\pgfpathlineto{\pgfqpoint{5.785029in}{0.758670in}}%
\pgfpathlineto{\pgfqpoint{5.785325in}{0.758663in}}%
\pgfpathlineto{\pgfqpoint{5.785621in}{0.758657in}}%
\pgfpathlineto{\pgfqpoint{5.785917in}{0.758651in}}%
\pgfpathlineto{\pgfqpoint{5.786213in}{0.758644in}}%
\pgfpathlineto{\pgfqpoint{5.786509in}{0.758638in}}%
\pgfpathlineto{\pgfqpoint{5.786805in}{0.758632in}}%
\pgfpathlineto{\pgfqpoint{5.787101in}{0.758625in}}%
\pgfpathlineto{\pgfqpoint{5.787397in}{0.758619in}}%
\pgfpathlineto{\pgfqpoint{5.787693in}{0.758613in}}%
\pgfpathlineto{\pgfqpoint{5.787989in}{0.758606in}}%
\pgfpathlineto{\pgfqpoint{5.788285in}{0.758600in}}%
\pgfpathlineto{\pgfqpoint{5.788581in}{0.758594in}}%
\pgfpathlineto{\pgfqpoint{5.788877in}{0.758587in}}%
\pgfpathlineto{\pgfqpoint{5.789173in}{0.758581in}}%
\pgfpathlineto{\pgfqpoint{5.789469in}{0.758575in}}%
\pgfpathlineto{\pgfqpoint{5.789765in}{0.758570in}}%
\pgfpathlineto{\pgfqpoint{5.790061in}{0.758568in}}%
\pgfpathlineto{\pgfqpoint{5.790357in}{0.758565in}}%
\pgfpathlineto{\pgfqpoint{5.790653in}{0.758563in}}%
\pgfpathlineto{\pgfqpoint{5.790949in}{0.758560in}}%
\pgfpathlineto{\pgfqpoint{5.791245in}{0.758557in}}%
\pgfpathlineto{\pgfqpoint{5.791541in}{0.758555in}}%
\pgfpathlineto{\pgfqpoint{5.791837in}{0.758552in}}%
\pgfpathlineto{\pgfqpoint{5.792133in}{0.758571in}}%
\pgfpathlineto{\pgfqpoint{5.792429in}{0.758617in}}%
\pgfpathlineto{\pgfqpoint{5.792725in}{0.758662in}}%
\pgfpathlineto{\pgfqpoint{5.793021in}{0.758707in}}%
\pgfpathlineto{\pgfqpoint{5.793317in}{0.758753in}}%
\pgfpathlineto{\pgfqpoint{5.793613in}{0.758798in}}%
\pgfpathlineto{\pgfqpoint{5.793909in}{0.758844in}}%
\pgfpathlineto{\pgfqpoint{5.794205in}{0.758889in}}%
\pgfpathlineto{\pgfqpoint{5.794501in}{0.758934in}}%
\pgfpathlineto{\pgfqpoint{5.794797in}{0.758980in}}%
\pgfpathlineto{\pgfqpoint{5.795093in}{0.759025in}}%
\pgfpathlineto{\pgfqpoint{5.795389in}{0.759071in}}%
\pgfpathlineto{\pgfqpoint{5.795685in}{0.759116in}}%
\pgfpathlineto{\pgfqpoint{5.795981in}{0.759161in}}%
\pgfpathlineto{\pgfqpoint{5.796277in}{0.759207in}}%
\pgfpathlineto{\pgfqpoint{5.796573in}{0.759252in}}%
\pgfpathlineto{\pgfqpoint{5.796869in}{0.759298in}}%
\pgfpathlineto{\pgfqpoint{5.797165in}{0.759343in}}%
\pgfpathlineto{\pgfqpoint{5.797461in}{0.759388in}}%
\pgfpathlineto{\pgfqpoint{5.797757in}{0.759434in}}%
\pgfpathlineto{\pgfqpoint{5.798053in}{0.759479in}}%
\pgfpathlineto{\pgfqpoint{5.798349in}{0.759525in}}%
\pgfpathlineto{\pgfqpoint{5.798645in}{0.759570in}}%
\pgfpathlineto{\pgfqpoint{5.798941in}{0.759615in}}%
\pgfpathlineto{\pgfqpoint{5.799237in}{0.759634in}}%
\pgfpathlineto{\pgfqpoint{5.799533in}{0.759615in}}%
\pgfpathlineto{\pgfqpoint{5.799829in}{0.759595in}}%
\pgfpathlineto{\pgfqpoint{5.800125in}{0.759575in}}%
\pgfpathlineto{\pgfqpoint{5.800421in}{0.759555in}}%
\pgfpathlineto{\pgfqpoint{5.800717in}{0.759535in}}%
\pgfpathlineto{\pgfqpoint{5.801013in}{0.759516in}}%
\pgfpathlineto{\pgfqpoint{5.801309in}{0.759496in}}%
\pgfpathlineto{\pgfqpoint{5.801605in}{0.759476in}}%
\pgfpathlineto{\pgfqpoint{5.801901in}{0.759456in}}%
\pgfpathlineto{\pgfqpoint{5.802197in}{0.759436in}}%
\pgfpathlineto{\pgfqpoint{5.802493in}{0.759416in}}%
\pgfpathlineto{\pgfqpoint{5.802789in}{0.759396in}}%
\pgfpathlineto{\pgfqpoint{5.803085in}{0.759377in}}%
\pgfpathlineto{\pgfqpoint{5.803381in}{0.759357in}}%
\pgfpathlineto{\pgfqpoint{5.803677in}{0.759209in}}%
\pgfpathlineto{\pgfqpoint{5.803973in}{0.758947in}}%
\pgfpathlineto{\pgfqpoint{5.804269in}{0.758837in}}%
\pgfpathlineto{\pgfqpoint{5.804565in}{0.758823in}}%
\pgfpathlineto{\pgfqpoint{5.804861in}{0.758810in}}%
\pgfpathlineto{\pgfqpoint{5.805157in}{0.758796in}}%
\pgfpathlineto{\pgfqpoint{5.805453in}{0.758782in}}%
\pgfpathlineto{\pgfqpoint{5.805749in}{0.758768in}}%
\pgfpathlineto{\pgfqpoint{5.806045in}{0.758755in}}%
\pgfpathlineto{\pgfqpoint{5.806341in}{0.758741in}}%
\pgfpathlineto{\pgfqpoint{5.806637in}{0.758730in}}%
\pgfpathlineto{\pgfqpoint{5.806933in}{0.758718in}}%
\pgfpathlineto{\pgfqpoint{5.807229in}{0.758707in}}%
\pgfpathlineto{\pgfqpoint{5.807525in}{0.758696in}}%
\pgfpathlineto{\pgfqpoint{5.807821in}{0.758685in}}%
\pgfpathlineto{\pgfqpoint{5.808117in}{0.758674in}}%
\pgfpathlineto{\pgfqpoint{5.808413in}{0.758662in}}%
\pgfpathlineto{\pgfqpoint{5.808709in}{0.758651in}}%
\pgfpathlineto{\pgfqpoint{5.809005in}{0.758640in}}%
\pgfpathlineto{\pgfqpoint{5.809301in}{0.758629in}}%
\pgfpathlineto{\pgfqpoint{5.809597in}{0.758618in}}%
\pgfpathlineto{\pgfqpoint{5.809893in}{0.758606in}}%
\pgfpathlineto{\pgfqpoint{5.810189in}{0.758595in}}%
\pgfpathlineto{\pgfqpoint{5.810485in}{0.758584in}}%
\pgfpathlineto{\pgfqpoint{5.810781in}{0.758573in}}%
\pgfpathlineto{\pgfqpoint{5.811077in}{0.758561in}}%
\pgfpathlineto{\pgfqpoint{5.811373in}{0.758550in}}%
\pgfpathlineto{\pgfqpoint{5.811669in}{0.758539in}}%
\pgfpathlineto{\pgfqpoint{5.811965in}{0.758528in}}%
\pgfpathlineto{\pgfqpoint{5.812261in}{0.758517in}}%
\pgfpathlineto{\pgfqpoint{5.812557in}{0.758505in}}%
\pgfpathlineto{\pgfqpoint{5.812853in}{0.758494in}}%
\pgfpathlineto{\pgfqpoint{5.813149in}{0.758483in}}%
\pgfpathlineto{\pgfqpoint{5.813445in}{0.758471in}}%
\pgfpathlineto{\pgfqpoint{5.813741in}{0.758457in}}%
\pgfpathlineto{\pgfqpoint{5.814037in}{0.758444in}}%
\pgfpathlineto{\pgfqpoint{5.814333in}{0.758430in}}%
\pgfpathlineto{\pgfqpoint{5.814629in}{0.758416in}}%
\pgfpathlineto{\pgfqpoint{5.814925in}{0.758402in}}%
\pgfpathlineto{\pgfqpoint{5.815221in}{0.758388in}}%
\pgfpathlineto{\pgfqpoint{5.815517in}{0.758374in}}%
\pgfpathlineto{\pgfqpoint{5.815813in}{0.758361in}}%
\pgfpathlineto{\pgfqpoint{5.816109in}{0.758347in}}%
\pgfpathlineto{\pgfqpoint{5.816405in}{0.758333in}}%
\pgfpathlineto{\pgfqpoint{5.816701in}{0.758319in}}%
\pgfpathlineto{\pgfqpoint{5.816997in}{0.758305in}}%
\pgfpathlineto{\pgfqpoint{5.817293in}{0.758291in}}%
\pgfpathlineto{\pgfqpoint{5.817589in}{0.758278in}}%
\pgfpathlineto{\pgfqpoint{5.817885in}{0.758264in}}%
\pgfpathlineto{\pgfqpoint{5.818181in}{0.758250in}}%
\pgfpathlineto{\pgfqpoint{5.818477in}{0.758236in}}%
\pgfpathlineto{\pgfqpoint{5.818773in}{0.758222in}}%
\pgfpathlineto{\pgfqpoint{5.819069in}{0.758209in}}%
\pgfpathlineto{\pgfqpoint{5.819365in}{0.758195in}}%
\pgfpathlineto{\pgfqpoint{5.819661in}{0.758181in}}%
\pgfpathlineto{\pgfqpoint{5.819957in}{0.758167in}}%
\pgfpathlineto{\pgfqpoint{5.820253in}{0.758149in}}%
\pgfpathlineto{\pgfqpoint{5.820549in}{0.758131in}}%
\pgfpathlineto{\pgfqpoint{5.820845in}{0.758112in}}%
\pgfpathlineto{\pgfqpoint{5.821141in}{0.758092in}}%
\pgfpathlineto{\pgfqpoint{5.821437in}{0.758073in}}%
\pgfpathlineto{\pgfqpoint{5.821733in}{0.758053in}}%
\pgfpathlineto{\pgfqpoint{5.822029in}{0.758034in}}%
\pgfpathlineto{\pgfqpoint{5.822325in}{0.758014in}}%
\pgfpathlineto{\pgfqpoint{5.822621in}{0.757995in}}%
\pgfpathlineto{\pgfqpoint{5.822917in}{0.757975in}}%
\pgfpathlineto{\pgfqpoint{5.823213in}{0.757955in}}%
\pgfpathlineto{\pgfqpoint{5.823509in}{0.757936in}}%
\pgfpathlineto{\pgfqpoint{5.823805in}{0.757916in}}%
\pgfpathlineto{\pgfqpoint{5.824101in}{0.757897in}}%
\pgfpathlineto{\pgfqpoint{5.824397in}{0.757877in}}%
\pgfpathlineto{\pgfqpoint{5.824693in}{0.757858in}}%
\pgfpathlineto{\pgfqpoint{5.824989in}{0.757838in}}%
\pgfpathlineto{\pgfqpoint{5.825285in}{0.757819in}}%
\pgfpathlineto{\pgfqpoint{5.825581in}{0.757799in}}%
\pgfpathlineto{\pgfqpoint{5.825877in}{0.757780in}}%
\pgfpathlineto{\pgfqpoint{5.826173in}{0.757760in}}%
\pgfpathlineto{\pgfqpoint{5.826469in}{0.757741in}}%
\pgfpathlineto{\pgfqpoint{5.826765in}{0.757721in}}%
\pgfpathlineto{\pgfqpoint{5.827061in}{0.757702in}}%
\pgfpathlineto{\pgfqpoint{5.827357in}{0.757682in}}%
\pgfpathlineto{\pgfqpoint{5.827653in}{0.757663in}}%
\pgfpathlineto{\pgfqpoint{5.827949in}{0.757643in}}%
\pgfpathlineto{\pgfqpoint{5.828245in}{0.757624in}}%
\pgfpathlineto{\pgfqpoint{5.828541in}{0.757604in}}%
\pgfpathlineto{\pgfqpoint{5.828837in}{0.757585in}}%
\pgfpathlineto{\pgfqpoint{5.829133in}{0.757565in}}%
\pgfpathlineto{\pgfqpoint{5.829429in}{0.757546in}}%
\pgfpathlineto{\pgfqpoint{5.829725in}{0.757526in}}%
\pgfpathlineto{\pgfqpoint{5.830021in}{0.757507in}}%
\pgfpathlineto{\pgfqpoint{5.830317in}{0.757487in}}%
\pgfpathlineto{\pgfqpoint{5.830613in}{0.757467in}}%
\pgfpathlineto{\pgfqpoint{5.830909in}{0.757448in}}%
\pgfpathlineto{\pgfqpoint{5.831205in}{0.757428in}}%
\pgfpathlineto{\pgfqpoint{5.831501in}{0.757409in}}%
\pgfpathlineto{\pgfqpoint{5.831797in}{0.757389in}}%
\pgfpathlineto{\pgfqpoint{5.832093in}{0.757370in}}%
\pgfpathlineto{\pgfqpoint{5.832389in}{0.757350in}}%
\pgfpathlineto{\pgfqpoint{5.832685in}{0.757331in}}%
\pgfpathlineto{\pgfqpoint{5.832981in}{0.757311in}}%
\pgfpathlineto{\pgfqpoint{5.833277in}{0.757292in}}%
\pgfpathlineto{\pgfqpoint{5.833573in}{0.757272in}}%
\pgfpathlineto{\pgfqpoint{5.833869in}{0.757253in}}%
\pgfpathlineto{\pgfqpoint{5.834165in}{0.757233in}}%
\pgfpathlineto{\pgfqpoint{5.834461in}{0.757214in}}%
\pgfpathlineto{\pgfqpoint{5.834757in}{0.757194in}}%
\pgfpathlineto{\pgfqpoint{5.835053in}{0.757175in}}%
\pgfpathlineto{\pgfqpoint{5.835349in}{0.757155in}}%
\pgfpathlineto{\pgfqpoint{5.835645in}{0.757136in}}%
\pgfpathlineto{\pgfqpoint{5.835941in}{0.757116in}}%
\pgfpathlineto{\pgfqpoint{5.836237in}{0.757097in}}%
\pgfpathlineto{\pgfqpoint{5.836533in}{0.757077in}}%
\pgfpathlineto{\pgfqpoint{5.836829in}{0.757058in}}%
\pgfpathlineto{\pgfqpoint{5.837125in}{0.757038in}}%
\pgfpathlineto{\pgfqpoint{5.837421in}{0.757019in}}%
\pgfpathlineto{\pgfqpoint{5.837717in}{0.756999in}}%
\pgfpathlineto{\pgfqpoint{5.838013in}{0.756979in}}%
\pgfpathlineto{\pgfqpoint{5.838309in}{0.756960in}}%
\pgfpathlineto{\pgfqpoint{5.838605in}{0.756940in}}%
\pgfpathlineto{\pgfqpoint{5.838902in}{0.757055in}}%
\pgfpathlineto{\pgfqpoint{5.839198in}{0.757204in}}%
\pgfpathlineto{\pgfqpoint{5.839494in}{0.757182in}}%
\pgfpathlineto{\pgfqpoint{5.839790in}{0.757168in}}%
\pgfpathlineto{\pgfqpoint{5.840086in}{0.756870in}}%
\pgfpathlineto{\pgfqpoint{5.840382in}{0.756617in}}%
\pgfpathlineto{\pgfqpoint{5.840678in}{0.756584in}}%
\pgfpathlineto{\pgfqpoint{5.840974in}{0.756551in}}%
\pgfpathlineto{\pgfqpoint{5.841270in}{0.756518in}}%
\pgfpathlineto{\pgfqpoint{5.841566in}{0.756485in}}%
\pgfpathlineto{\pgfqpoint{5.841862in}{0.756453in}}%
\pgfpathlineto{\pgfqpoint{5.842158in}{0.756416in}}%
\pgfpathlineto{\pgfqpoint{5.842454in}{0.756378in}}%
\pgfpathlineto{\pgfqpoint{5.842750in}{0.756340in}}%
\pgfpathlineto{\pgfqpoint{5.843046in}{0.756302in}}%
\pgfpathlineto{\pgfqpoint{5.843342in}{0.756264in}}%
\pgfpathlineto{\pgfqpoint{5.843638in}{0.756226in}}%
\pgfpathlineto{\pgfqpoint{5.843934in}{0.756188in}}%
\pgfpathlineto{\pgfqpoint{5.844230in}{0.756150in}}%
\pgfpathlineto{\pgfqpoint{5.844526in}{0.756112in}}%
\pgfpathlineto{\pgfqpoint{5.844822in}{0.756074in}}%
\pgfpathlineto{\pgfqpoint{5.845118in}{0.756036in}}%
\pgfpathlineto{\pgfqpoint{5.845414in}{0.755997in}}%
\pgfpathlineto{\pgfqpoint{5.845710in}{0.755959in}}%
\pgfpathlineto{\pgfqpoint{5.846006in}{0.755921in}}%
\pgfpathlineto{\pgfqpoint{5.846302in}{0.755883in}}%
\pgfpathlineto{\pgfqpoint{5.846598in}{0.755756in}}%
\pgfpathlineto{\pgfqpoint{5.846894in}{0.755538in}}%
\pgfpathlineto{\pgfqpoint{5.847190in}{0.755319in}}%
\pgfpathlineto{\pgfqpoint{5.847486in}{0.755017in}}%
\pgfpathlineto{\pgfqpoint{5.847782in}{0.754011in}}%
\pgfpathlineto{\pgfqpoint{5.848078in}{0.752278in}}%
\pgfpathlineto{\pgfqpoint{5.848374in}{0.751721in}}%
\pgfpathlineto{\pgfqpoint{5.848670in}{0.751422in}}%
\pgfpathlineto{\pgfqpoint{5.848966in}{0.751332in}}%
\pgfpathlineto{\pgfqpoint{5.849262in}{0.751249in}}%
\pgfpathlineto{\pgfqpoint{5.849558in}{0.751168in}}%
\pgfpathlineto{\pgfqpoint{5.849854in}{0.751108in}}%
\pgfpathlineto{\pgfqpoint{5.850150in}{0.751020in}}%
\pgfpathlineto{\pgfqpoint{5.850446in}{0.750932in}}%
\pgfpathlineto{\pgfqpoint{5.850742in}{0.750844in}}%
\pgfpathlineto{\pgfqpoint{5.851038in}{0.750756in}}%
\pgfpathlineto{\pgfqpoint{5.851334in}{0.750668in}}%
\pgfpathlineto{\pgfqpoint{5.851630in}{0.750580in}}%
\pgfpathlineto{\pgfqpoint{5.851926in}{0.750492in}}%
\pgfpathlineto{\pgfqpoint{5.852222in}{0.750404in}}%
\pgfpathlineto{\pgfqpoint{5.852518in}{0.750316in}}%
\pgfpathlineto{\pgfqpoint{5.852814in}{0.750229in}}%
\pgfpathlineto{\pgfqpoint{5.853110in}{0.750141in}}%
\pgfpathlineto{\pgfqpoint{5.853406in}{0.750052in}}%
\pgfpathlineto{\pgfqpoint{5.853702in}{0.749950in}}%
\pgfpathlineto{\pgfqpoint{5.853998in}{0.749839in}}%
\pgfpathlineto{\pgfqpoint{5.854294in}{0.749727in}}%
\pgfpathlineto{\pgfqpoint{5.854590in}{0.749344in}}%
\pgfpathlineto{\pgfqpoint{5.854886in}{0.749149in}}%
\pgfpathlineto{\pgfqpoint{5.855182in}{0.749129in}}%
\pgfpathlineto{\pgfqpoint{5.855478in}{0.749109in}}%
\pgfpathlineto{\pgfqpoint{5.855774in}{0.749088in}}%
\pgfpathlineto{\pgfqpoint{5.856070in}{0.749068in}}%
\pgfpathlineto{\pgfqpoint{5.856366in}{0.749048in}}%
\pgfpathlineto{\pgfqpoint{5.856662in}{0.749027in}}%
\pgfpathlineto{\pgfqpoint{5.856958in}{0.749007in}}%
\pgfpathlineto{\pgfqpoint{5.857254in}{0.748986in}}%
\pgfpathlineto{\pgfqpoint{5.857550in}{0.748966in}}%
\pgfpathlineto{\pgfqpoint{5.857846in}{0.748946in}}%
\pgfpathlineto{\pgfqpoint{5.858142in}{0.748925in}}%
\pgfpathlineto{\pgfqpoint{5.858438in}{0.748905in}}%
\pgfpathlineto{\pgfqpoint{5.858734in}{0.748885in}}%
\pgfpathlineto{\pgfqpoint{5.859030in}{0.748864in}}%
\pgfpathlineto{\pgfqpoint{5.859326in}{0.748844in}}%
\pgfpathlineto{\pgfqpoint{5.859622in}{0.748823in}}%
\pgfpathlineto{\pgfqpoint{5.859918in}{0.748803in}}%
\pgfpathlineto{\pgfqpoint{5.860214in}{0.748783in}}%
\pgfpathlineto{\pgfqpoint{5.860510in}{0.748764in}}%
\pgfpathlineto{\pgfqpoint{5.860806in}{0.748757in}}%
\pgfpathlineto{\pgfqpoint{5.861102in}{0.748750in}}%
\pgfpathlineto{\pgfqpoint{5.861398in}{0.748744in}}%
\pgfpathlineto{\pgfqpoint{5.861694in}{0.748738in}}%
\pgfpathlineto{\pgfqpoint{5.861990in}{0.748732in}}%
\pgfpathlineto{\pgfqpoint{5.862286in}{0.748725in}}%
\pgfpathlineto{\pgfqpoint{5.862582in}{0.748698in}}%
\pgfpathlineto{\pgfqpoint{5.862878in}{0.748498in}}%
\pgfpathlineto{\pgfqpoint{5.863174in}{0.748386in}}%
\pgfpathlineto{\pgfqpoint{5.863470in}{0.748380in}}%
\pgfpathlineto{\pgfqpoint{5.863766in}{0.748373in}}%
\pgfpathlineto{\pgfqpoint{5.864062in}{0.748367in}}%
\pgfpathlineto{\pgfqpoint{5.864358in}{0.748361in}}%
\pgfpathlineto{\pgfqpoint{5.864654in}{0.748354in}}%
\pgfpathlineto{\pgfqpoint{5.864950in}{0.748348in}}%
\pgfpathlineto{\pgfqpoint{5.865246in}{0.748342in}}%
\pgfpathlineto{\pgfqpoint{5.865542in}{0.748335in}}%
\pgfpathlineto{\pgfqpoint{5.865838in}{0.748329in}}%
\pgfpathlineto{\pgfqpoint{5.866134in}{0.748322in}}%
\pgfpathlineto{\pgfqpoint{5.866430in}{0.748316in}}%
\pgfpathlineto{\pgfqpoint{5.866726in}{0.748310in}}%
\pgfpathlineto{\pgfqpoint{5.867022in}{0.748304in}}%
\pgfpathlineto{\pgfqpoint{5.867318in}{0.748327in}}%
\pgfpathlineto{\pgfqpoint{5.867614in}{0.748365in}}%
\pgfpathlineto{\pgfqpoint{5.867910in}{0.748403in}}%
\pgfpathlineto{\pgfqpoint{5.868206in}{0.748441in}}%
\pgfpathlineto{\pgfqpoint{5.868502in}{0.748452in}}%
\pgfpathlineto{\pgfqpoint{5.868798in}{0.748406in}}%
\pgfpathlineto{\pgfqpoint{5.869094in}{0.748359in}}%
\pgfpathlineto{\pgfqpoint{5.869390in}{0.748312in}}%
\pgfpathlineto{\pgfqpoint{5.869686in}{0.748266in}}%
\pgfpathlineto{\pgfqpoint{5.869982in}{0.748251in}}%
\pgfpathlineto{\pgfqpoint{5.870278in}{0.748250in}}%
\pgfpathlineto{\pgfqpoint{5.870574in}{0.748248in}}%
\pgfpathlineto{\pgfqpoint{5.870870in}{0.748247in}}%
\pgfpathlineto{\pgfqpoint{5.871166in}{0.748245in}}%
\pgfpathlineto{\pgfqpoint{5.871462in}{0.748243in}}%
\pgfpathlineto{\pgfqpoint{5.871758in}{0.748242in}}%
\pgfpathlineto{\pgfqpoint{5.872054in}{0.748240in}}%
\pgfpathlineto{\pgfqpoint{5.872350in}{0.748239in}}%
\pgfpathlineto{\pgfqpoint{5.872646in}{0.748237in}}%
\pgfpathlineto{\pgfqpoint{5.872942in}{0.748236in}}%
\pgfpathlineto{\pgfqpoint{5.873238in}{0.748234in}}%
\pgfpathlineto{\pgfqpoint{5.873534in}{0.748233in}}%
\pgfpathlineto{\pgfqpoint{5.873830in}{0.748231in}}%
\pgfpathlineto{\pgfqpoint{5.874126in}{0.748230in}}%
\pgfpathlineto{\pgfqpoint{5.874422in}{0.748228in}}%
\pgfpathlineto{\pgfqpoint{5.874718in}{0.748227in}}%
\pgfpathlineto{\pgfqpoint{5.875014in}{0.748225in}}%
\pgfpathlineto{\pgfqpoint{5.875310in}{0.748224in}}%
\pgfpathlineto{\pgfqpoint{5.875606in}{0.748222in}}%
\pgfpathlineto{\pgfqpoint{5.875902in}{0.748221in}}%
\pgfpathlineto{\pgfqpoint{5.876198in}{0.748219in}}%
\pgfpathlineto{\pgfqpoint{5.876494in}{0.748218in}}%
\pgfpathlineto{\pgfqpoint{5.876790in}{0.748216in}}%
\pgfpathlineto{\pgfqpoint{5.877086in}{0.748215in}}%
\pgfpathlineto{\pgfqpoint{5.877382in}{0.748213in}}%
\pgfpathlineto{\pgfqpoint{5.877678in}{0.748212in}}%
\pgfpathlineto{\pgfqpoint{5.877974in}{0.748210in}}%
\pgfpathlineto{\pgfqpoint{5.878270in}{0.748209in}}%
\pgfpathlineto{\pgfqpoint{5.878566in}{0.748207in}}%
\pgfpathlineto{\pgfqpoint{5.878862in}{0.748206in}}%
\pgfpathlineto{\pgfqpoint{5.879158in}{0.748204in}}%
\pgfpathlineto{\pgfqpoint{5.879454in}{0.748203in}}%
\pgfpathlineto{\pgfqpoint{5.879750in}{0.748201in}}%
\pgfpathlineto{\pgfqpoint{5.880046in}{0.748199in}}%
\pgfpathlineto{\pgfqpoint{5.880342in}{0.748198in}}%
\pgfpathlineto{\pgfqpoint{5.880638in}{0.748196in}}%
\pgfpathlineto{\pgfqpoint{5.880934in}{0.748195in}}%
\pgfpathlineto{\pgfqpoint{5.881230in}{0.748193in}}%
\pgfpathlineto{\pgfqpoint{5.881526in}{0.748192in}}%
\pgfpathlineto{\pgfqpoint{5.881822in}{0.748190in}}%
\pgfpathlineto{\pgfqpoint{5.882118in}{0.748189in}}%
\pgfpathlineto{\pgfqpoint{5.882414in}{0.748187in}}%
\pgfpathlineto{\pgfqpoint{5.882710in}{0.748186in}}%
\pgfpathlineto{\pgfqpoint{5.883006in}{0.748184in}}%
\pgfpathlineto{\pgfqpoint{5.883302in}{0.748183in}}%
\pgfpathlineto{\pgfqpoint{5.883598in}{0.748181in}}%
\pgfpathlineto{\pgfqpoint{5.883894in}{0.748180in}}%
\pgfpathlineto{\pgfqpoint{5.884190in}{0.748178in}}%
\pgfpathlineto{\pgfqpoint{5.884486in}{0.748177in}}%
\pgfpathlineto{\pgfqpoint{5.884782in}{0.748175in}}%
\pgfpathlineto{\pgfqpoint{5.885078in}{0.748174in}}%
\pgfpathlineto{\pgfqpoint{5.885374in}{0.748172in}}%
\pgfpathlineto{\pgfqpoint{5.885670in}{0.748171in}}%
\pgfpathlineto{\pgfqpoint{5.885966in}{0.748169in}}%
\pgfpathlineto{\pgfqpoint{5.886262in}{0.748168in}}%
\pgfpathlineto{\pgfqpoint{5.886558in}{0.748166in}}%
\pgfpathlineto{\pgfqpoint{5.886854in}{0.748165in}}%
\pgfpathlineto{\pgfqpoint{5.887150in}{0.748163in}}%
\pgfpathlineto{\pgfqpoint{5.887446in}{0.748162in}}%
\pgfpathlineto{\pgfqpoint{5.887742in}{0.748160in}}%
\pgfpathlineto{\pgfqpoint{5.888038in}{0.748159in}}%
\pgfpathlineto{\pgfqpoint{5.888334in}{0.748157in}}%
\pgfpathlineto{\pgfqpoint{5.888630in}{0.748155in}}%
\pgfpathlineto{\pgfqpoint{5.888926in}{0.748154in}}%
\pgfpathlineto{\pgfqpoint{5.889222in}{0.748152in}}%
\pgfpathlineto{\pgfqpoint{5.889518in}{0.748151in}}%
\pgfpathlineto{\pgfqpoint{5.889814in}{0.748149in}}%
\pgfpathlineto{\pgfqpoint{5.890110in}{0.748148in}}%
\pgfpathlineto{\pgfqpoint{5.890406in}{0.748139in}}%
\pgfpathlineto{\pgfqpoint{5.890702in}{0.748125in}}%
\pgfpathlineto{\pgfqpoint{5.890998in}{0.748143in}}%
\pgfpathlineto{\pgfqpoint{5.891294in}{0.748149in}}%
\pgfpathlineto{\pgfqpoint{5.891590in}{0.748144in}}%
\pgfpathlineto{\pgfqpoint{5.891886in}{0.748139in}}%
\pgfpathlineto{\pgfqpoint{5.892182in}{0.748134in}}%
\pgfpathlineto{\pgfqpoint{5.892478in}{0.748130in}}%
\pgfpathlineto{\pgfqpoint{5.892774in}{0.748125in}}%
\pgfpathlineto{\pgfqpoint{5.893070in}{0.748120in}}%
\pgfpathlineto{\pgfqpoint{5.893366in}{0.748115in}}%
\pgfpathlineto{\pgfqpoint{5.893662in}{0.748110in}}%
\pgfpathlineto{\pgfqpoint{5.893958in}{0.748105in}}%
\pgfpathlineto{\pgfqpoint{5.894254in}{0.748100in}}%
\pgfpathlineto{\pgfqpoint{5.894550in}{0.748095in}}%
\pgfpathlineto{\pgfqpoint{5.894846in}{0.748090in}}%
\pgfpathlineto{\pgfqpoint{5.895142in}{0.748085in}}%
\pgfpathlineto{\pgfqpoint{5.895438in}{0.748080in}}%
\pgfpathlineto{\pgfqpoint{5.895734in}{0.748075in}}%
\pgfpathlineto{\pgfqpoint{5.896030in}{0.748070in}}%
\pgfpathlineto{\pgfqpoint{5.896326in}{0.748065in}}%
\pgfpathlineto{\pgfqpoint{5.896622in}{0.748064in}}%
\pgfpathlineto{\pgfqpoint{5.896918in}{0.748111in}}%
\pgfpathlineto{\pgfqpoint{5.897214in}{0.748102in}}%
\pgfpathlineto{\pgfqpoint{5.897510in}{0.748082in}}%
\pgfpathlineto{\pgfqpoint{5.897806in}{0.748062in}}%
\pgfpathlineto{\pgfqpoint{5.898102in}{0.748041in}}%
\pgfpathlineto{\pgfqpoint{5.898398in}{0.748030in}}%
\pgfpathlineto{\pgfqpoint{5.898694in}{0.748024in}}%
\pgfpathlineto{\pgfqpoint{5.898990in}{0.748017in}}%
\pgfpathlineto{\pgfqpoint{5.899286in}{0.748011in}}%
\pgfpathlineto{\pgfqpoint{5.899582in}{0.748005in}}%
\pgfpathlineto{\pgfqpoint{5.899878in}{0.747998in}}%
\pgfpathlineto{\pgfqpoint{5.900174in}{0.747992in}}%
\pgfpathlineto{\pgfqpoint{5.900470in}{0.747985in}}%
\pgfpathlineto{\pgfqpoint{5.900766in}{0.747979in}}%
\pgfpathlineto{\pgfqpoint{5.901062in}{0.747973in}}%
\pgfpathlineto{\pgfqpoint{5.901358in}{0.747966in}}%
\pgfpathlineto{\pgfqpoint{5.901654in}{0.747960in}}%
\pgfpathlineto{\pgfqpoint{5.901950in}{0.747954in}}%
\pgfpathlineto{\pgfqpoint{5.902246in}{0.747947in}}%
\pgfpathlineto{\pgfqpoint{5.902542in}{0.747941in}}%
\pgfpathlineto{\pgfqpoint{5.902838in}{0.747933in}}%
\pgfpathlineto{\pgfqpoint{5.903134in}{0.747925in}}%
\pgfpathlineto{\pgfqpoint{5.903430in}{0.747917in}}%
\pgfpathlineto{\pgfqpoint{5.903726in}{0.747909in}}%
\pgfpathlineto{\pgfqpoint{5.904022in}{0.747901in}}%
\pgfpathlineto{\pgfqpoint{5.904318in}{0.747894in}}%
\pgfpathlineto{\pgfqpoint{5.904614in}{0.747889in}}%
\pgfpathlineto{\pgfqpoint{5.904910in}{0.747890in}}%
\pgfpathlineto{\pgfqpoint{5.905206in}{0.747892in}}%
\pgfpathlineto{\pgfqpoint{5.905502in}{0.747893in}}%
\pgfpathlineto{\pgfqpoint{5.905798in}{0.747886in}}%
\pgfpathlineto{\pgfqpoint{5.906094in}{0.747879in}}%
\pgfpathlineto{\pgfqpoint{5.906391in}{0.747871in}}%
\pgfpathlineto{\pgfqpoint{5.906687in}{0.747863in}}%
\pgfpathlineto{\pgfqpoint{5.906983in}{0.747856in}}%
\pgfpathlineto{\pgfqpoint{5.907279in}{0.747848in}}%
\pgfpathlineto{\pgfqpoint{5.907575in}{0.747841in}}%
\pgfpathlineto{\pgfqpoint{5.907871in}{0.747833in}}%
\pgfpathlineto{\pgfqpoint{5.908167in}{0.747825in}}%
\pgfpathlineto{\pgfqpoint{5.908463in}{0.747818in}}%
\pgfpathlineto{\pgfqpoint{5.908759in}{0.747810in}}%
\pgfpathlineto{\pgfqpoint{5.909055in}{0.747803in}}%
\pgfpathlineto{\pgfqpoint{5.909351in}{0.747795in}}%
\pgfpathlineto{\pgfqpoint{5.909647in}{0.747787in}}%
\pgfpathlineto{\pgfqpoint{5.909943in}{0.747780in}}%
\pgfpathlineto{\pgfqpoint{5.910239in}{0.747775in}}%
\pgfpathlineto{\pgfqpoint{5.910535in}{0.747785in}}%
\pgfpathlineto{\pgfqpoint{5.910831in}{0.747785in}}%
\pgfpathlineto{\pgfqpoint{5.911127in}{0.747784in}}%
\pgfpathlineto{\pgfqpoint{5.911423in}{0.747782in}}%
\pgfpathlineto{\pgfqpoint{5.911719in}{0.747781in}}%
\pgfpathlineto{\pgfqpoint{5.912015in}{0.747780in}}%
\pgfpathlineto{\pgfqpoint{5.912311in}{0.747777in}}%
\pgfpathlineto{\pgfqpoint{5.912607in}{0.747771in}}%
\pgfpathlineto{\pgfqpoint{5.912903in}{0.747764in}}%
\pgfpathlineto{\pgfqpoint{5.913199in}{0.747757in}}%
\pgfpathlineto{\pgfqpoint{5.913495in}{0.747750in}}%
\pgfpathlineto{\pgfqpoint{5.913791in}{0.747743in}}%
\pgfpathlineto{\pgfqpoint{5.914087in}{0.747736in}}%
\pgfpathlineto{\pgfqpoint{5.914383in}{0.747729in}}%
\pgfpathlineto{\pgfqpoint{5.914679in}{0.747722in}}%
\pgfpathlineto{\pgfqpoint{5.914975in}{0.747715in}}%
\pgfpathlineto{\pgfqpoint{5.915271in}{0.747708in}}%
\pgfpathlineto{\pgfqpoint{5.915567in}{0.747701in}}%
\pgfpathlineto{\pgfqpoint{5.915863in}{0.747694in}}%
\pgfpathlineto{\pgfqpoint{5.916159in}{0.747687in}}%
\pgfpathlineto{\pgfqpoint{5.916455in}{0.747680in}}%
\pgfpathlineto{\pgfqpoint{5.916751in}{0.747673in}}%
\pgfpathlineto{\pgfqpoint{5.917047in}{0.747666in}}%
\pgfpathlineto{\pgfqpoint{5.917343in}{0.747659in}}%
\pgfpathlineto{\pgfqpoint{5.917639in}{0.747652in}}%
\pgfpathlineto{\pgfqpoint{5.917935in}{0.747645in}}%
\pgfpathlineto{\pgfqpoint{5.918231in}{0.747638in}}%
\pgfpathlineto{\pgfqpoint{5.918527in}{0.747631in}}%
\pgfpathlineto{\pgfqpoint{5.918823in}{0.747624in}}%
\pgfpathlineto{\pgfqpoint{5.919119in}{0.747618in}}%
\pgfpathlineto{\pgfqpoint{5.919415in}{0.747611in}}%
\pgfpathlineto{\pgfqpoint{5.919711in}{0.747604in}}%
\pgfpathlineto{\pgfqpoint{5.920007in}{0.747597in}}%
\pgfpathlineto{\pgfqpoint{5.920303in}{0.747590in}}%
\pgfpathlineto{\pgfqpoint{5.920599in}{0.747583in}}%
\pgfpathlineto{\pgfqpoint{5.920895in}{0.747576in}}%
\pgfpathlineto{\pgfqpoint{5.921191in}{0.747569in}}%
\pgfpathlineto{\pgfqpoint{5.921487in}{0.747562in}}%
\pgfpathlineto{\pgfqpoint{5.921783in}{0.747555in}}%
\pgfpathlineto{\pgfqpoint{5.922079in}{0.747548in}}%
\pgfpathlineto{\pgfqpoint{5.922375in}{0.747541in}}%
\pgfpathlineto{\pgfqpoint{5.922671in}{0.747534in}}%
\pgfpathlineto{\pgfqpoint{5.922967in}{0.747527in}}%
\pgfpathlineto{\pgfqpoint{5.923263in}{0.747520in}}%
\pgfpathlineto{\pgfqpoint{5.923559in}{0.747513in}}%
\pgfpathlineto{\pgfqpoint{5.923855in}{0.747506in}}%
\pgfpathlineto{\pgfqpoint{5.924151in}{0.747499in}}%
\pgfpathlineto{\pgfqpoint{5.924447in}{0.747492in}}%
\pgfpathlineto{\pgfqpoint{5.924743in}{0.747485in}}%
\pgfpathlineto{\pgfqpoint{5.925039in}{0.747478in}}%
\pgfpathlineto{\pgfqpoint{5.925335in}{0.747471in}}%
\pgfpathlineto{\pgfqpoint{5.925631in}{0.747464in}}%
\pgfpathlineto{\pgfqpoint{5.925927in}{0.747457in}}%
\pgfpathlineto{\pgfqpoint{5.926223in}{0.747450in}}%
\pgfpathlineto{\pgfqpoint{5.926519in}{0.747444in}}%
\pgfpathlineto{\pgfqpoint{5.926815in}{0.747437in}}%
\pgfpathlineto{\pgfqpoint{5.927111in}{0.747430in}}%
\pgfpathlineto{\pgfqpoint{5.927407in}{0.747423in}}%
\pgfpathlineto{\pgfqpoint{5.927703in}{0.747416in}}%
\pgfpathlineto{\pgfqpoint{5.927999in}{0.747409in}}%
\pgfpathlineto{\pgfqpoint{5.928295in}{0.747402in}}%
\pgfpathlineto{\pgfqpoint{5.928591in}{0.747395in}}%
\pgfpathlineto{\pgfqpoint{5.928887in}{0.747388in}}%
\pgfpathlineto{\pgfqpoint{5.929183in}{0.747381in}}%
\pgfpathlineto{\pgfqpoint{5.929479in}{0.747374in}}%
\pgfpathlineto{\pgfqpoint{5.929775in}{0.747367in}}%
\pgfpathlineto{\pgfqpoint{5.930071in}{0.747360in}}%
\pgfpathlineto{\pgfqpoint{5.930367in}{0.747353in}}%
\pgfpathlineto{\pgfqpoint{5.930663in}{0.747346in}}%
\pgfpathlineto{\pgfqpoint{5.930959in}{0.747339in}}%
\pgfpathlineto{\pgfqpoint{5.931255in}{0.747332in}}%
\pgfpathlineto{\pgfqpoint{5.931551in}{0.747325in}}%
\pgfpathlineto{\pgfqpoint{5.931847in}{0.747318in}}%
\pgfpathlineto{\pgfqpoint{5.932143in}{0.747311in}}%
\pgfpathlineto{\pgfqpoint{5.932439in}{0.747304in}}%
\pgfpathlineto{\pgfqpoint{5.932735in}{0.747297in}}%
\pgfpathlineto{\pgfqpoint{5.933031in}{0.747290in}}%
\pgfpathlineto{\pgfqpoint{5.933327in}{0.747283in}}%
\pgfpathlineto{\pgfqpoint{5.933623in}{0.747276in}}%
\pgfpathlineto{\pgfqpoint{5.933919in}{0.747270in}}%
\pgfpathlineto{\pgfqpoint{5.934215in}{0.747263in}}%
\pgfpathlineto{\pgfqpoint{5.934511in}{0.747256in}}%
\pgfpathlineto{\pgfqpoint{5.934807in}{0.747249in}}%
\pgfpathlineto{\pgfqpoint{5.935103in}{0.747242in}}%
\pgfpathlineto{\pgfqpoint{5.935399in}{0.747235in}}%
\pgfpathlineto{\pgfqpoint{5.935695in}{0.747228in}}%
\pgfpathlineto{\pgfqpoint{5.935991in}{0.747221in}}%
\pgfpathlineto{\pgfqpoint{5.936287in}{0.747214in}}%
\pgfpathlineto{\pgfqpoint{5.936583in}{0.747207in}}%
\pgfpathlineto{\pgfqpoint{5.936879in}{0.747200in}}%
\pgfpathlineto{\pgfqpoint{5.937175in}{0.747193in}}%
\pgfpathlineto{\pgfqpoint{5.937471in}{0.747186in}}%
\pgfpathlineto{\pgfqpoint{5.937767in}{0.747179in}}%
\pgfpathlineto{\pgfqpoint{5.938063in}{0.747172in}}%
\pgfpathlineto{\pgfqpoint{5.938359in}{0.747165in}}%
\pgfpathlineto{\pgfqpoint{5.938655in}{0.747159in}}%
\pgfpathlineto{\pgfqpoint{5.938951in}{0.746629in}}%
\pgfpathlineto{\pgfqpoint{5.939247in}{0.746003in}}%
\pgfpathlineto{\pgfqpoint{5.939543in}{0.745985in}}%
\pgfpathlineto{\pgfqpoint{5.939839in}{0.745968in}}%
\pgfpathlineto{\pgfqpoint{5.940135in}{0.745951in}}%
\pgfpathlineto{\pgfqpoint{5.940431in}{0.745933in}}%
\pgfpathlineto{\pgfqpoint{5.940727in}{0.745916in}}%
\pgfpathlineto{\pgfqpoint{5.941023in}{0.745899in}}%
\pgfpathlineto{\pgfqpoint{5.941319in}{0.745882in}}%
\pgfpathlineto{\pgfqpoint{5.941615in}{0.745864in}}%
\pgfpathlineto{\pgfqpoint{5.941911in}{0.745847in}}%
\pgfpathlineto{\pgfqpoint{5.942207in}{0.745830in}}%
\pgfpathlineto{\pgfqpoint{5.942503in}{0.745812in}}%
\pgfpathlineto{\pgfqpoint{5.942799in}{0.745795in}}%
\pgfpathlineto{\pgfqpoint{5.943095in}{0.745778in}}%
\pgfpathlineto{\pgfqpoint{5.943391in}{0.745761in}}%
\pgfpathlineto{\pgfqpoint{5.943687in}{0.745743in}}%
\pgfpathlineto{\pgfqpoint{5.943983in}{0.745726in}}%
\pgfpathlineto{\pgfqpoint{5.944279in}{0.745709in}}%
\pgfpathlineto{\pgfqpoint{5.944575in}{0.745691in}}%
\pgfpathlineto{\pgfqpoint{5.944871in}{0.745674in}}%
\pgfpathlineto{\pgfqpoint{5.945167in}{0.745657in}}%
\pgfpathlineto{\pgfqpoint{5.945463in}{0.745640in}}%
\pgfpathlineto{\pgfqpoint{5.945759in}{0.745622in}}%
\pgfpathlineto{\pgfqpoint{5.946055in}{0.745605in}}%
\pgfpathlineto{\pgfqpoint{5.946351in}{0.745588in}}%
\pgfpathlineto{\pgfqpoint{5.946647in}{0.745570in}}%
\pgfpathlineto{\pgfqpoint{5.946943in}{0.745553in}}%
\pgfpathlineto{\pgfqpoint{5.947239in}{0.745536in}}%
\pgfpathlineto{\pgfqpoint{5.947535in}{0.745518in}}%
\pgfpathlineto{\pgfqpoint{5.947831in}{0.745501in}}%
\pgfpathlineto{\pgfqpoint{5.948127in}{0.745484in}}%
\pgfpathlineto{\pgfqpoint{5.948423in}{0.745467in}}%
\pgfpathlineto{\pgfqpoint{5.948719in}{0.745449in}}%
\pgfpathlineto{\pgfqpoint{5.949015in}{0.745432in}}%
\pgfpathlineto{\pgfqpoint{5.949311in}{0.745415in}}%
\pgfpathlineto{\pgfqpoint{5.949607in}{0.745397in}}%
\pgfpathlineto{\pgfqpoint{5.949903in}{0.745380in}}%
\pgfpathlineto{\pgfqpoint{5.950199in}{0.745363in}}%
\pgfpathlineto{\pgfqpoint{5.950495in}{0.745346in}}%
\pgfpathlineto{\pgfqpoint{5.950791in}{0.745328in}}%
\pgfpathlineto{\pgfqpoint{5.951087in}{0.745311in}}%
\pgfpathlineto{\pgfqpoint{5.951383in}{0.745294in}}%
\pgfpathlineto{\pgfqpoint{5.951679in}{0.745276in}}%
\pgfpathlineto{\pgfqpoint{5.951975in}{0.745259in}}%
\pgfpathlineto{\pgfqpoint{5.952271in}{0.745242in}}%
\pgfpathlineto{\pgfqpoint{5.952567in}{0.745225in}}%
\pgfpathlineto{\pgfqpoint{5.952863in}{0.745218in}}%
\pgfpathlineto{\pgfqpoint{5.953159in}{0.745215in}}%
\pgfpathlineto{\pgfqpoint{5.953455in}{0.745212in}}%
\pgfpathlineto{\pgfqpoint{5.953751in}{0.745209in}}%
\pgfpathlineto{\pgfqpoint{5.954047in}{0.745206in}}%
\pgfpathlineto{\pgfqpoint{5.954343in}{0.745204in}}%
\pgfpathlineto{\pgfqpoint{5.954639in}{0.745202in}}%
\pgfpathlineto{\pgfqpoint{5.954935in}{0.745199in}}%
\pgfpathlineto{\pgfqpoint{5.955231in}{0.745195in}}%
\pgfpathlineto{\pgfqpoint{5.955527in}{0.745192in}}%
\pgfpathlineto{\pgfqpoint{5.955823in}{0.745188in}}%
\pgfpathlineto{\pgfqpoint{5.956119in}{0.745185in}}%
\pgfpathlineto{\pgfqpoint{5.956415in}{0.745181in}}%
\pgfpathlineto{\pgfqpoint{5.956711in}{0.745178in}}%
\pgfpathlineto{\pgfqpoint{5.957007in}{0.745174in}}%
\pgfpathlineto{\pgfqpoint{5.957303in}{0.745171in}}%
\pgfpathlineto{\pgfqpoint{5.957599in}{0.745167in}}%
\pgfpathlineto{\pgfqpoint{5.957895in}{0.745164in}}%
\pgfpathlineto{\pgfqpoint{5.958191in}{0.745160in}}%
\pgfpathlineto{\pgfqpoint{5.958487in}{0.745157in}}%
\pgfpathlineto{\pgfqpoint{5.958783in}{0.745153in}}%
\pgfpathlineto{\pgfqpoint{5.959079in}{0.745150in}}%
\pgfpathlineto{\pgfqpoint{5.959375in}{0.745146in}}%
\pgfpathlineto{\pgfqpoint{5.959671in}{0.745143in}}%
\pgfpathlineto{\pgfqpoint{5.959967in}{0.745139in}}%
\pgfpathlineto{\pgfqpoint{5.960263in}{0.745136in}}%
\pgfpathlineto{\pgfqpoint{5.960559in}{0.745132in}}%
\pgfpathlineto{\pgfqpoint{5.960855in}{0.745129in}}%
\pgfpathlineto{\pgfqpoint{5.961151in}{0.745125in}}%
\pgfpathlineto{\pgfqpoint{5.961447in}{0.745122in}}%
\pgfpathlineto{\pgfqpoint{5.961743in}{0.745118in}}%
\pgfpathlineto{\pgfqpoint{5.962039in}{0.745115in}}%
\pgfpathlineto{\pgfqpoint{5.962335in}{0.745111in}}%
\pgfpathlineto{\pgfqpoint{5.962631in}{0.745108in}}%
\pgfpathlineto{\pgfqpoint{5.962927in}{0.745104in}}%
\pgfpathlineto{\pgfqpoint{5.963223in}{0.745101in}}%
\pgfpathlineto{\pgfqpoint{5.963519in}{0.745097in}}%
\pgfpathlineto{\pgfqpoint{5.963815in}{0.745094in}}%
\pgfpathlineto{\pgfqpoint{5.964111in}{0.745090in}}%
\pgfpathlineto{\pgfqpoint{5.964407in}{0.745086in}}%
\pgfpathlineto{\pgfqpoint{5.964703in}{0.745083in}}%
\pgfpathlineto{\pgfqpoint{5.964999in}{0.745079in}}%
\pgfpathlineto{\pgfqpoint{5.965295in}{0.745076in}}%
\pgfpathlineto{\pgfqpoint{5.965591in}{0.745072in}}%
\pgfpathlineto{\pgfqpoint{5.965887in}{0.745069in}}%
\pgfpathlineto{\pgfqpoint{5.966183in}{0.745065in}}%
\pgfpathlineto{\pgfqpoint{5.966479in}{0.745062in}}%
\pgfpathlineto{\pgfqpoint{5.966775in}{0.745058in}}%
\pgfpathlineto{\pgfqpoint{5.967071in}{0.745055in}}%
\pgfpathlineto{\pgfqpoint{5.967367in}{0.745051in}}%
\pgfpathlineto{\pgfqpoint{5.967663in}{0.745048in}}%
\pgfpathlineto{\pgfqpoint{5.967959in}{0.745044in}}%
\pgfpathlineto{\pgfqpoint{5.968255in}{0.745041in}}%
\pgfpathlineto{\pgfqpoint{5.968551in}{0.745037in}}%
\pgfpathlineto{\pgfqpoint{5.968847in}{0.745034in}}%
\pgfpathlineto{\pgfqpoint{5.969143in}{0.745030in}}%
\pgfpathlineto{\pgfqpoint{5.969439in}{0.745027in}}%
\pgfpathlineto{\pgfqpoint{5.969735in}{0.745023in}}%
\pgfpathlineto{\pgfqpoint{5.970031in}{0.745020in}}%
\pgfpathlineto{\pgfqpoint{5.970327in}{0.745016in}}%
\pgfpathlineto{\pgfqpoint{5.970623in}{0.745013in}}%
\pgfpathlineto{\pgfqpoint{5.970919in}{0.745009in}}%
\pgfpathlineto{\pgfqpoint{5.971215in}{0.745006in}}%
\pgfpathlineto{\pgfqpoint{5.971511in}{0.745002in}}%
\pgfpathlineto{\pgfqpoint{5.971807in}{0.744999in}}%
\pgfpathlineto{\pgfqpoint{5.972103in}{0.744995in}}%
\pgfpathlineto{\pgfqpoint{5.972399in}{0.744992in}}%
\pgfpathlineto{\pgfqpoint{5.972695in}{0.744988in}}%
\pgfpathlineto{\pgfqpoint{5.972991in}{0.744985in}}%
\pgfpathlineto{\pgfqpoint{5.973287in}{0.744981in}}%
\pgfpathlineto{\pgfqpoint{5.973583in}{0.744978in}}%
\pgfpathlineto{\pgfqpoint{5.973880in}{0.744974in}}%
\pgfpathlineto{\pgfqpoint{5.974176in}{0.744970in}}%
\pgfpathlineto{\pgfqpoint{5.974472in}{0.744967in}}%
\pgfpathlineto{\pgfqpoint{5.974768in}{0.744963in}}%
\pgfpathlineto{\pgfqpoint{5.975064in}{0.744960in}}%
\pgfpathlineto{\pgfqpoint{5.975360in}{0.744956in}}%
\pgfpathlineto{\pgfqpoint{5.975656in}{0.744953in}}%
\pgfpathlineto{\pgfqpoint{5.975952in}{0.744949in}}%
\pgfpathlineto{\pgfqpoint{5.976248in}{0.744946in}}%
\pgfpathlineto{\pgfqpoint{5.976544in}{0.744942in}}%
\pgfpathlineto{\pgfqpoint{5.976840in}{0.744939in}}%
\pgfpathlineto{\pgfqpoint{5.977136in}{0.744935in}}%
\pgfpathlineto{\pgfqpoint{5.977432in}{0.744932in}}%
\pgfpathlineto{\pgfqpoint{5.977728in}{0.744928in}}%
\pgfpathlineto{\pgfqpoint{5.978024in}{0.744925in}}%
\pgfpathlineto{\pgfqpoint{5.978320in}{0.744921in}}%
\pgfpathlineto{\pgfqpoint{5.978616in}{0.744918in}}%
\pgfpathlineto{\pgfqpoint{5.978912in}{0.744914in}}%
\pgfpathlineto{\pgfqpoint{5.979208in}{0.744911in}}%
\pgfpathlineto{\pgfqpoint{5.979504in}{0.744907in}}%
\pgfpathlineto{\pgfqpoint{5.979800in}{0.744904in}}%
\pgfpathlineto{\pgfqpoint{5.980096in}{0.744900in}}%
\pgfpathlineto{\pgfqpoint{5.980392in}{0.744897in}}%
\pgfpathlineto{\pgfqpoint{5.980688in}{0.744893in}}%
\pgfpathlineto{\pgfqpoint{5.980984in}{0.744890in}}%
\pgfpathlineto{\pgfqpoint{5.981280in}{0.744886in}}%
\pgfpathlineto{\pgfqpoint{5.981576in}{0.744883in}}%
\pgfpathlineto{\pgfqpoint{5.981872in}{0.744879in}}%
\pgfpathlineto{\pgfqpoint{5.982168in}{0.744876in}}%
\pgfpathlineto{\pgfqpoint{5.982464in}{0.744872in}}%
\pgfpathlineto{\pgfqpoint{5.982760in}{0.744869in}}%
\pgfpathlineto{\pgfqpoint{5.983056in}{0.744865in}}%
\pgfpathlineto{\pgfqpoint{5.983352in}{0.744862in}}%
\pgfpathlineto{\pgfqpoint{5.983648in}{0.744858in}}%
\pgfpathlineto{\pgfqpoint{5.983944in}{0.744854in}}%
\pgfpathlineto{\pgfqpoint{5.984240in}{0.744851in}}%
\pgfpathlineto{\pgfqpoint{5.984536in}{0.744847in}}%
\pgfpathlineto{\pgfqpoint{5.984832in}{0.744844in}}%
\pgfpathlineto{\pgfqpoint{5.985128in}{0.744840in}}%
\pgfpathlineto{\pgfqpoint{5.985424in}{0.744837in}}%
\pgfpathlineto{\pgfqpoint{5.985720in}{0.744833in}}%
\pgfpathlineto{\pgfqpoint{5.986016in}{0.744830in}}%
\pgfpathlineto{\pgfqpoint{5.986312in}{0.744826in}}%
\pgfpathlineto{\pgfqpoint{5.986608in}{0.744823in}}%
\pgfpathlineto{\pgfqpoint{5.986904in}{0.744819in}}%
\pgfpathlineto{\pgfqpoint{5.987200in}{0.744816in}}%
\pgfpathlineto{\pgfqpoint{5.987496in}{0.744812in}}%
\pgfpathlineto{\pgfqpoint{5.987792in}{0.744809in}}%
\pgfpathlineto{\pgfqpoint{5.988088in}{0.744805in}}%
\pgfpathlineto{\pgfqpoint{5.988384in}{0.744802in}}%
\pgfpathlineto{\pgfqpoint{5.988680in}{0.744512in}}%
\pgfpathlineto{\pgfqpoint{5.988976in}{0.742012in}}%
\pgfpathlineto{\pgfqpoint{5.989272in}{0.741704in}}%
\pgfpathlineto{\pgfqpoint{5.989568in}{0.741681in}}%
\pgfpathlineto{\pgfqpoint{5.989864in}{0.741657in}}%
\pgfpathlineto{\pgfqpoint{5.990160in}{0.741634in}}%
\pgfpathlineto{\pgfqpoint{5.990456in}{0.741610in}}%
\pgfpathlineto{\pgfqpoint{5.990752in}{0.741586in}}%
\pgfpathlineto{\pgfqpoint{5.991048in}{0.741563in}}%
\pgfpathlineto{\pgfqpoint{5.991344in}{0.741539in}}%
\pgfpathlineto{\pgfqpoint{5.991640in}{0.741516in}}%
\pgfpathlineto{\pgfqpoint{5.991936in}{0.741492in}}%
\pgfpathlineto{\pgfqpoint{5.992232in}{0.741468in}}%
\pgfpathlineto{\pgfqpoint{5.992528in}{0.741445in}}%
\pgfpathlineto{\pgfqpoint{5.992824in}{0.741421in}}%
\pgfpathlineto{\pgfqpoint{5.993120in}{0.741398in}}%
\pgfpathlineto{\pgfqpoint{5.993416in}{0.741374in}}%
\pgfpathlineto{\pgfqpoint{5.993712in}{0.741351in}}%
\pgfpathlineto{\pgfqpoint{5.994008in}{0.741327in}}%
\pgfpathlineto{\pgfqpoint{5.994304in}{0.741303in}}%
\pgfpathlineto{\pgfqpoint{5.994600in}{0.741280in}}%
\pgfpathlineto{\pgfqpoint{5.994896in}{0.741256in}}%
\pgfpathlineto{\pgfqpoint{5.995192in}{0.741233in}}%
\pgfpathlineto{\pgfqpoint{5.995488in}{0.741209in}}%
\pgfpathlineto{\pgfqpoint{5.995784in}{0.741147in}}%
\pgfpathlineto{\pgfqpoint{5.996080in}{0.740986in}}%
\pgfpathlineto{\pgfqpoint{5.996376in}{0.740861in}}%
\pgfpathlineto{\pgfqpoint{5.996672in}{0.740746in}}%
\pgfpathlineto{\pgfqpoint{5.996968in}{0.740631in}}%
\pgfpathlineto{\pgfqpoint{5.997264in}{0.740518in}}%
\pgfpathlineto{\pgfqpoint{5.997560in}{0.740455in}}%
\pgfpathlineto{\pgfqpoint{5.997856in}{0.740413in}}%
\pgfpathlineto{\pgfqpoint{5.998152in}{0.740372in}}%
\pgfpathlineto{\pgfqpoint{5.998448in}{0.740333in}}%
\pgfpathlineto{\pgfqpoint{5.998744in}{0.740293in}}%
\pgfpathlineto{\pgfqpoint{5.999040in}{0.740253in}}%
\pgfpathlineto{\pgfqpoint{5.999336in}{0.740214in}}%
\pgfpathlineto{\pgfqpoint{5.999632in}{0.740174in}}%
\pgfpathlineto{\pgfqpoint{5.999928in}{0.740134in}}%
\pgfpathlineto{\pgfqpoint{6.000224in}{0.740095in}}%
\pgfpathlineto{\pgfqpoint{6.000520in}{0.740055in}}%
\pgfpathlineto{\pgfqpoint{6.000816in}{0.740015in}}%
\pgfpathlineto{\pgfqpoint{6.001112in}{0.739975in}}%
\pgfpathlineto{\pgfqpoint{6.001408in}{0.739936in}}%
\pgfpathlineto{\pgfqpoint{6.001704in}{0.739896in}}%
\pgfpathlineto{\pgfqpoint{6.002000in}{0.739856in}}%
\pgfpathlineto{\pgfqpoint{6.002296in}{0.739817in}}%
\pgfpathlineto{\pgfqpoint{6.002592in}{0.739777in}}%
\pgfpathlineto{\pgfqpoint{6.002888in}{0.739733in}}%
\pgfpathlineto{\pgfqpoint{6.003184in}{0.739706in}}%
\pgfpathlineto{\pgfqpoint{6.003480in}{0.739691in}}%
\pgfpathlineto{\pgfqpoint{6.003776in}{0.739676in}}%
\pgfpathlineto{\pgfqpoint{6.004072in}{0.739662in}}%
\pgfpathlineto{\pgfqpoint{6.004368in}{0.739656in}}%
\pgfpathlineto{\pgfqpoint{6.004368in}{0.739656in}}%
\pgfpathlineto{\pgfqpoint{6.004368in}{0.739656in}}%
\pgfpathlineto{\pgfqpoint{6.004072in}{0.739656in}}%
\pgfpathlineto{\pgfqpoint{6.003776in}{0.739656in}}%
\pgfpathlineto{\pgfqpoint{6.003480in}{0.739656in}}%
\pgfpathlineto{\pgfqpoint{6.003184in}{0.739656in}}%
\pgfpathlineto{\pgfqpoint{6.002888in}{0.739656in}}%
\pgfpathlineto{\pgfqpoint{6.002592in}{0.739656in}}%
\pgfpathlineto{\pgfqpoint{6.002296in}{0.739656in}}%
\pgfpathlineto{\pgfqpoint{6.002000in}{0.739656in}}%
\pgfpathlineto{\pgfqpoint{6.001704in}{0.739656in}}%
\pgfpathlineto{\pgfqpoint{6.001408in}{0.739656in}}%
\pgfpathlineto{\pgfqpoint{6.001112in}{0.739656in}}%
\pgfpathlineto{\pgfqpoint{6.000816in}{0.739656in}}%
\pgfpathlineto{\pgfqpoint{6.000520in}{0.739656in}}%
\pgfpathlineto{\pgfqpoint{6.000224in}{0.739656in}}%
\pgfpathlineto{\pgfqpoint{5.999928in}{0.739656in}}%
\pgfpathlineto{\pgfqpoint{5.999632in}{0.739656in}}%
\pgfpathlineto{\pgfqpoint{5.999336in}{0.739656in}}%
\pgfpathlineto{\pgfqpoint{5.999040in}{0.739656in}}%
\pgfpathlineto{\pgfqpoint{5.998744in}{0.739656in}}%
\pgfpathlineto{\pgfqpoint{5.998448in}{0.739656in}}%
\pgfpathlineto{\pgfqpoint{5.998152in}{0.739656in}}%
\pgfpathlineto{\pgfqpoint{5.997856in}{0.739656in}}%
\pgfpathlineto{\pgfqpoint{5.997560in}{0.739656in}}%
\pgfpathlineto{\pgfqpoint{5.997264in}{0.739656in}}%
\pgfpathlineto{\pgfqpoint{5.996968in}{0.739656in}}%
\pgfpathlineto{\pgfqpoint{5.996672in}{0.739656in}}%
\pgfpathlineto{\pgfqpoint{5.996376in}{0.739656in}}%
\pgfpathlineto{\pgfqpoint{5.996080in}{0.739656in}}%
\pgfpathlineto{\pgfqpoint{5.995784in}{0.739656in}}%
\pgfpathlineto{\pgfqpoint{5.995488in}{0.739656in}}%
\pgfpathlineto{\pgfqpoint{5.995192in}{0.739656in}}%
\pgfpathlineto{\pgfqpoint{5.994896in}{0.739656in}}%
\pgfpathlineto{\pgfqpoint{5.994600in}{0.739656in}}%
\pgfpathlineto{\pgfqpoint{5.994304in}{0.739656in}}%
\pgfpathlineto{\pgfqpoint{5.994008in}{0.739656in}}%
\pgfpathlineto{\pgfqpoint{5.993712in}{0.739656in}}%
\pgfpathlineto{\pgfqpoint{5.993416in}{0.739656in}}%
\pgfpathlineto{\pgfqpoint{5.993120in}{0.739656in}}%
\pgfpathlineto{\pgfqpoint{5.992824in}{0.739656in}}%
\pgfpathlineto{\pgfqpoint{5.992528in}{0.739656in}}%
\pgfpathlineto{\pgfqpoint{5.992232in}{0.739656in}}%
\pgfpathlineto{\pgfqpoint{5.991936in}{0.739656in}}%
\pgfpathlineto{\pgfqpoint{5.991640in}{0.739656in}}%
\pgfpathlineto{\pgfqpoint{5.991344in}{0.739656in}}%
\pgfpathlineto{\pgfqpoint{5.991048in}{0.739656in}}%
\pgfpathlineto{\pgfqpoint{5.990752in}{0.739656in}}%
\pgfpathlineto{\pgfqpoint{5.990456in}{0.739656in}}%
\pgfpathlineto{\pgfqpoint{5.990160in}{0.739656in}}%
\pgfpathlineto{\pgfqpoint{5.989864in}{0.739656in}}%
\pgfpathlineto{\pgfqpoint{5.989568in}{0.739656in}}%
\pgfpathlineto{\pgfqpoint{5.989272in}{0.739656in}}%
\pgfpathlineto{\pgfqpoint{5.988976in}{0.739656in}}%
\pgfpathlineto{\pgfqpoint{5.988680in}{0.739656in}}%
\pgfpathlineto{\pgfqpoint{5.988384in}{0.739656in}}%
\pgfpathlineto{\pgfqpoint{5.988088in}{0.739656in}}%
\pgfpathlineto{\pgfqpoint{5.987792in}{0.739656in}}%
\pgfpathlineto{\pgfqpoint{5.987496in}{0.739656in}}%
\pgfpathlineto{\pgfqpoint{5.987200in}{0.739656in}}%
\pgfpathlineto{\pgfqpoint{5.986904in}{0.739656in}}%
\pgfpathlineto{\pgfqpoint{5.986608in}{0.739656in}}%
\pgfpathlineto{\pgfqpoint{5.986312in}{0.739656in}}%
\pgfpathlineto{\pgfqpoint{5.986016in}{0.739656in}}%
\pgfpathlineto{\pgfqpoint{5.985720in}{0.739656in}}%
\pgfpathlineto{\pgfqpoint{5.985424in}{0.739656in}}%
\pgfpathlineto{\pgfqpoint{5.985128in}{0.739656in}}%
\pgfpathlineto{\pgfqpoint{5.984832in}{0.739656in}}%
\pgfpathlineto{\pgfqpoint{5.984536in}{0.739656in}}%
\pgfpathlineto{\pgfqpoint{5.984240in}{0.739656in}}%
\pgfpathlineto{\pgfqpoint{5.983944in}{0.739656in}}%
\pgfpathlineto{\pgfqpoint{5.983648in}{0.739656in}}%
\pgfpathlineto{\pgfqpoint{5.983352in}{0.739656in}}%
\pgfpathlineto{\pgfqpoint{5.983056in}{0.739656in}}%
\pgfpathlineto{\pgfqpoint{5.982760in}{0.739656in}}%
\pgfpathlineto{\pgfqpoint{5.982464in}{0.739656in}}%
\pgfpathlineto{\pgfqpoint{5.982168in}{0.739656in}}%
\pgfpathlineto{\pgfqpoint{5.981872in}{0.739656in}}%
\pgfpathlineto{\pgfqpoint{5.981576in}{0.739656in}}%
\pgfpathlineto{\pgfqpoint{5.981280in}{0.739656in}}%
\pgfpathlineto{\pgfqpoint{5.980984in}{0.739656in}}%
\pgfpathlineto{\pgfqpoint{5.980688in}{0.739656in}}%
\pgfpathlineto{\pgfqpoint{5.980392in}{0.739656in}}%
\pgfpathlineto{\pgfqpoint{5.980096in}{0.739656in}}%
\pgfpathlineto{\pgfqpoint{5.979800in}{0.739656in}}%
\pgfpathlineto{\pgfqpoint{5.979504in}{0.739656in}}%
\pgfpathlineto{\pgfqpoint{5.979208in}{0.739656in}}%
\pgfpathlineto{\pgfqpoint{5.978912in}{0.739656in}}%
\pgfpathlineto{\pgfqpoint{5.978616in}{0.739656in}}%
\pgfpathlineto{\pgfqpoint{5.978320in}{0.739656in}}%
\pgfpathlineto{\pgfqpoint{5.978024in}{0.739656in}}%
\pgfpathlineto{\pgfqpoint{5.977728in}{0.739656in}}%
\pgfpathlineto{\pgfqpoint{5.977432in}{0.739656in}}%
\pgfpathlineto{\pgfqpoint{5.977136in}{0.739656in}}%
\pgfpathlineto{\pgfqpoint{5.976840in}{0.739656in}}%
\pgfpathlineto{\pgfqpoint{5.976544in}{0.739656in}}%
\pgfpathlineto{\pgfqpoint{5.976248in}{0.739656in}}%
\pgfpathlineto{\pgfqpoint{5.975952in}{0.739656in}}%
\pgfpathlineto{\pgfqpoint{5.975656in}{0.739656in}}%
\pgfpathlineto{\pgfqpoint{5.975360in}{0.739656in}}%
\pgfpathlineto{\pgfqpoint{5.975064in}{0.739656in}}%
\pgfpathlineto{\pgfqpoint{5.974768in}{0.739656in}}%
\pgfpathlineto{\pgfqpoint{5.974472in}{0.739656in}}%
\pgfpathlineto{\pgfqpoint{5.974176in}{0.739656in}}%
\pgfpathlineto{\pgfqpoint{5.973880in}{0.739656in}}%
\pgfpathlineto{\pgfqpoint{5.973583in}{0.739656in}}%
\pgfpathlineto{\pgfqpoint{5.973287in}{0.739656in}}%
\pgfpathlineto{\pgfqpoint{5.972991in}{0.739656in}}%
\pgfpathlineto{\pgfqpoint{5.972695in}{0.739656in}}%
\pgfpathlineto{\pgfqpoint{5.972399in}{0.739656in}}%
\pgfpathlineto{\pgfqpoint{5.972103in}{0.739656in}}%
\pgfpathlineto{\pgfqpoint{5.971807in}{0.739656in}}%
\pgfpathlineto{\pgfqpoint{5.971511in}{0.739656in}}%
\pgfpathlineto{\pgfqpoint{5.971215in}{0.739656in}}%
\pgfpathlineto{\pgfqpoint{5.970919in}{0.739656in}}%
\pgfpathlineto{\pgfqpoint{5.970623in}{0.739656in}}%
\pgfpathlineto{\pgfqpoint{5.970327in}{0.739656in}}%
\pgfpathlineto{\pgfqpoint{5.970031in}{0.739656in}}%
\pgfpathlineto{\pgfqpoint{5.969735in}{0.739656in}}%
\pgfpathlineto{\pgfqpoint{5.969439in}{0.739656in}}%
\pgfpathlineto{\pgfqpoint{5.969143in}{0.739656in}}%
\pgfpathlineto{\pgfqpoint{5.968847in}{0.739656in}}%
\pgfpathlineto{\pgfqpoint{5.968551in}{0.739656in}}%
\pgfpathlineto{\pgfqpoint{5.968255in}{0.739656in}}%
\pgfpathlineto{\pgfqpoint{5.967959in}{0.739656in}}%
\pgfpathlineto{\pgfqpoint{5.967663in}{0.739656in}}%
\pgfpathlineto{\pgfqpoint{5.967367in}{0.739656in}}%
\pgfpathlineto{\pgfqpoint{5.967071in}{0.739656in}}%
\pgfpathlineto{\pgfqpoint{5.966775in}{0.739656in}}%
\pgfpathlineto{\pgfqpoint{5.966479in}{0.739656in}}%
\pgfpathlineto{\pgfqpoint{5.966183in}{0.739656in}}%
\pgfpathlineto{\pgfqpoint{5.965887in}{0.739656in}}%
\pgfpathlineto{\pgfqpoint{5.965591in}{0.739656in}}%
\pgfpathlineto{\pgfqpoint{5.965295in}{0.739656in}}%
\pgfpathlineto{\pgfqpoint{5.964999in}{0.739656in}}%
\pgfpathlineto{\pgfqpoint{5.964703in}{0.739656in}}%
\pgfpathlineto{\pgfqpoint{5.964407in}{0.739656in}}%
\pgfpathlineto{\pgfqpoint{5.964111in}{0.739656in}}%
\pgfpathlineto{\pgfqpoint{5.963815in}{0.739656in}}%
\pgfpathlineto{\pgfqpoint{5.963519in}{0.739656in}}%
\pgfpathlineto{\pgfqpoint{5.963223in}{0.739656in}}%
\pgfpathlineto{\pgfqpoint{5.962927in}{0.739656in}}%
\pgfpathlineto{\pgfqpoint{5.962631in}{0.739656in}}%
\pgfpathlineto{\pgfqpoint{5.962335in}{0.739656in}}%
\pgfpathlineto{\pgfqpoint{5.962039in}{0.739656in}}%
\pgfpathlineto{\pgfqpoint{5.961743in}{0.739656in}}%
\pgfpathlineto{\pgfqpoint{5.961447in}{0.739656in}}%
\pgfpathlineto{\pgfqpoint{5.961151in}{0.739656in}}%
\pgfpathlineto{\pgfqpoint{5.960855in}{0.739656in}}%
\pgfpathlineto{\pgfqpoint{5.960559in}{0.739656in}}%
\pgfpathlineto{\pgfqpoint{5.960263in}{0.739656in}}%
\pgfpathlineto{\pgfqpoint{5.959967in}{0.739656in}}%
\pgfpathlineto{\pgfqpoint{5.959671in}{0.739656in}}%
\pgfpathlineto{\pgfqpoint{5.959375in}{0.739656in}}%
\pgfpathlineto{\pgfqpoint{5.959079in}{0.739656in}}%
\pgfpathlineto{\pgfqpoint{5.958783in}{0.739656in}}%
\pgfpathlineto{\pgfqpoint{5.958487in}{0.739656in}}%
\pgfpathlineto{\pgfqpoint{5.958191in}{0.739656in}}%
\pgfpathlineto{\pgfqpoint{5.957895in}{0.739656in}}%
\pgfpathlineto{\pgfqpoint{5.957599in}{0.739656in}}%
\pgfpathlineto{\pgfqpoint{5.957303in}{0.739656in}}%
\pgfpathlineto{\pgfqpoint{5.957007in}{0.739656in}}%
\pgfpathlineto{\pgfqpoint{5.956711in}{0.739656in}}%
\pgfpathlineto{\pgfqpoint{5.956415in}{0.739656in}}%
\pgfpathlineto{\pgfqpoint{5.956119in}{0.739656in}}%
\pgfpathlineto{\pgfqpoint{5.955823in}{0.739656in}}%
\pgfpathlineto{\pgfqpoint{5.955527in}{0.739656in}}%
\pgfpathlineto{\pgfqpoint{5.955231in}{0.739656in}}%
\pgfpathlineto{\pgfqpoint{5.954935in}{0.739656in}}%
\pgfpathlineto{\pgfqpoint{5.954639in}{0.739656in}}%
\pgfpathlineto{\pgfqpoint{5.954343in}{0.739656in}}%
\pgfpathlineto{\pgfqpoint{5.954047in}{0.739656in}}%
\pgfpathlineto{\pgfqpoint{5.953751in}{0.739656in}}%
\pgfpathlineto{\pgfqpoint{5.953455in}{0.739656in}}%
\pgfpathlineto{\pgfqpoint{5.953159in}{0.739656in}}%
\pgfpathlineto{\pgfqpoint{5.952863in}{0.739656in}}%
\pgfpathlineto{\pgfqpoint{5.952567in}{0.739656in}}%
\pgfpathlineto{\pgfqpoint{5.952271in}{0.739656in}}%
\pgfpathlineto{\pgfqpoint{5.951975in}{0.739656in}}%
\pgfpathlineto{\pgfqpoint{5.951679in}{0.739656in}}%
\pgfpathlineto{\pgfqpoint{5.951383in}{0.739656in}}%
\pgfpathlineto{\pgfqpoint{5.951087in}{0.739656in}}%
\pgfpathlineto{\pgfqpoint{5.950791in}{0.739656in}}%
\pgfpathlineto{\pgfqpoint{5.950495in}{0.739656in}}%
\pgfpathlineto{\pgfqpoint{5.950199in}{0.739656in}}%
\pgfpathlineto{\pgfqpoint{5.949903in}{0.739656in}}%
\pgfpathlineto{\pgfqpoint{5.949607in}{0.739656in}}%
\pgfpathlineto{\pgfqpoint{5.949311in}{0.739656in}}%
\pgfpathlineto{\pgfqpoint{5.949015in}{0.739656in}}%
\pgfpathlineto{\pgfqpoint{5.948719in}{0.739656in}}%
\pgfpathlineto{\pgfqpoint{5.948423in}{0.739656in}}%
\pgfpathlineto{\pgfqpoint{5.948127in}{0.739656in}}%
\pgfpathlineto{\pgfqpoint{5.947831in}{0.739656in}}%
\pgfpathlineto{\pgfqpoint{5.947535in}{0.739656in}}%
\pgfpathlineto{\pgfqpoint{5.947239in}{0.739656in}}%
\pgfpathlineto{\pgfqpoint{5.946943in}{0.739656in}}%
\pgfpathlineto{\pgfqpoint{5.946647in}{0.739656in}}%
\pgfpathlineto{\pgfqpoint{5.946351in}{0.739656in}}%
\pgfpathlineto{\pgfqpoint{5.946055in}{0.739656in}}%
\pgfpathlineto{\pgfqpoint{5.945759in}{0.739656in}}%
\pgfpathlineto{\pgfqpoint{5.945463in}{0.739656in}}%
\pgfpathlineto{\pgfqpoint{5.945167in}{0.739656in}}%
\pgfpathlineto{\pgfqpoint{5.944871in}{0.739656in}}%
\pgfpathlineto{\pgfqpoint{5.944575in}{0.739656in}}%
\pgfpathlineto{\pgfqpoint{5.944279in}{0.739656in}}%
\pgfpathlineto{\pgfqpoint{5.943983in}{0.739656in}}%
\pgfpathlineto{\pgfqpoint{5.943687in}{0.739656in}}%
\pgfpathlineto{\pgfqpoint{5.943391in}{0.739656in}}%
\pgfpathlineto{\pgfqpoint{5.943095in}{0.739656in}}%
\pgfpathlineto{\pgfqpoint{5.942799in}{0.739656in}}%
\pgfpathlineto{\pgfqpoint{5.942503in}{0.739656in}}%
\pgfpathlineto{\pgfqpoint{5.942207in}{0.739656in}}%
\pgfpathlineto{\pgfqpoint{5.941911in}{0.739656in}}%
\pgfpathlineto{\pgfqpoint{5.941615in}{0.739656in}}%
\pgfpathlineto{\pgfqpoint{5.941319in}{0.739656in}}%
\pgfpathlineto{\pgfqpoint{5.941023in}{0.739656in}}%
\pgfpathlineto{\pgfqpoint{5.940727in}{0.739656in}}%
\pgfpathlineto{\pgfqpoint{5.940431in}{0.739656in}}%
\pgfpathlineto{\pgfqpoint{5.940135in}{0.739656in}}%
\pgfpathlineto{\pgfqpoint{5.939839in}{0.739656in}}%
\pgfpathlineto{\pgfqpoint{5.939543in}{0.739656in}}%
\pgfpathlineto{\pgfqpoint{5.939247in}{0.739656in}}%
\pgfpathlineto{\pgfqpoint{5.938951in}{0.739656in}}%
\pgfpathlineto{\pgfqpoint{5.938655in}{0.739656in}}%
\pgfpathlineto{\pgfqpoint{5.938359in}{0.739656in}}%
\pgfpathlineto{\pgfqpoint{5.938063in}{0.739656in}}%
\pgfpathlineto{\pgfqpoint{5.937767in}{0.739656in}}%
\pgfpathlineto{\pgfqpoint{5.937471in}{0.739656in}}%
\pgfpathlineto{\pgfqpoint{5.937175in}{0.739656in}}%
\pgfpathlineto{\pgfqpoint{5.936879in}{0.739656in}}%
\pgfpathlineto{\pgfqpoint{5.936583in}{0.739656in}}%
\pgfpathlineto{\pgfqpoint{5.936287in}{0.739656in}}%
\pgfpathlineto{\pgfqpoint{5.935991in}{0.739656in}}%
\pgfpathlineto{\pgfqpoint{5.935695in}{0.739656in}}%
\pgfpathlineto{\pgfqpoint{5.935399in}{0.739656in}}%
\pgfpathlineto{\pgfqpoint{5.935103in}{0.739656in}}%
\pgfpathlineto{\pgfqpoint{5.934807in}{0.739656in}}%
\pgfpathlineto{\pgfqpoint{5.934511in}{0.739656in}}%
\pgfpathlineto{\pgfqpoint{5.934215in}{0.739656in}}%
\pgfpathlineto{\pgfqpoint{5.933919in}{0.739656in}}%
\pgfpathlineto{\pgfqpoint{5.933623in}{0.739656in}}%
\pgfpathlineto{\pgfqpoint{5.933327in}{0.739656in}}%
\pgfpathlineto{\pgfqpoint{5.933031in}{0.739656in}}%
\pgfpathlineto{\pgfqpoint{5.932735in}{0.739656in}}%
\pgfpathlineto{\pgfqpoint{5.932439in}{0.739656in}}%
\pgfpathlineto{\pgfqpoint{5.932143in}{0.739656in}}%
\pgfpathlineto{\pgfqpoint{5.931847in}{0.739656in}}%
\pgfpathlineto{\pgfqpoint{5.931551in}{0.739656in}}%
\pgfpathlineto{\pgfqpoint{5.931255in}{0.739656in}}%
\pgfpathlineto{\pgfqpoint{5.930959in}{0.739656in}}%
\pgfpathlineto{\pgfqpoint{5.930663in}{0.739656in}}%
\pgfpathlineto{\pgfqpoint{5.930367in}{0.739656in}}%
\pgfpathlineto{\pgfqpoint{5.930071in}{0.739656in}}%
\pgfpathlineto{\pgfqpoint{5.929775in}{0.739656in}}%
\pgfpathlineto{\pgfqpoint{5.929479in}{0.739656in}}%
\pgfpathlineto{\pgfqpoint{5.929183in}{0.739656in}}%
\pgfpathlineto{\pgfqpoint{5.928887in}{0.739656in}}%
\pgfpathlineto{\pgfqpoint{5.928591in}{0.739656in}}%
\pgfpathlineto{\pgfqpoint{5.928295in}{0.739656in}}%
\pgfpathlineto{\pgfqpoint{5.927999in}{0.739656in}}%
\pgfpathlineto{\pgfqpoint{5.927703in}{0.739656in}}%
\pgfpathlineto{\pgfqpoint{5.927407in}{0.739656in}}%
\pgfpathlineto{\pgfqpoint{5.927111in}{0.739656in}}%
\pgfpathlineto{\pgfqpoint{5.926815in}{0.739656in}}%
\pgfpathlineto{\pgfqpoint{5.926519in}{0.739656in}}%
\pgfpathlineto{\pgfqpoint{5.926223in}{0.739656in}}%
\pgfpathlineto{\pgfqpoint{5.925927in}{0.739656in}}%
\pgfpathlineto{\pgfqpoint{5.925631in}{0.739656in}}%
\pgfpathlineto{\pgfqpoint{5.925335in}{0.739656in}}%
\pgfpathlineto{\pgfqpoint{5.925039in}{0.739656in}}%
\pgfpathlineto{\pgfqpoint{5.924743in}{0.739656in}}%
\pgfpathlineto{\pgfqpoint{5.924447in}{0.739656in}}%
\pgfpathlineto{\pgfqpoint{5.924151in}{0.739656in}}%
\pgfpathlineto{\pgfqpoint{5.923855in}{0.739656in}}%
\pgfpathlineto{\pgfqpoint{5.923559in}{0.739656in}}%
\pgfpathlineto{\pgfqpoint{5.923263in}{0.739656in}}%
\pgfpathlineto{\pgfqpoint{5.922967in}{0.739656in}}%
\pgfpathlineto{\pgfqpoint{5.922671in}{0.739656in}}%
\pgfpathlineto{\pgfqpoint{5.922375in}{0.739656in}}%
\pgfpathlineto{\pgfqpoint{5.922079in}{0.739656in}}%
\pgfpathlineto{\pgfqpoint{5.921783in}{0.739656in}}%
\pgfpathlineto{\pgfqpoint{5.921487in}{0.739656in}}%
\pgfpathlineto{\pgfqpoint{5.921191in}{0.739656in}}%
\pgfpathlineto{\pgfqpoint{5.920895in}{0.739656in}}%
\pgfpathlineto{\pgfqpoint{5.920599in}{0.739656in}}%
\pgfpathlineto{\pgfqpoint{5.920303in}{0.739656in}}%
\pgfpathlineto{\pgfqpoint{5.920007in}{0.739656in}}%
\pgfpathlineto{\pgfqpoint{5.919711in}{0.739656in}}%
\pgfpathlineto{\pgfqpoint{5.919415in}{0.739656in}}%
\pgfpathlineto{\pgfqpoint{5.919119in}{0.739656in}}%
\pgfpathlineto{\pgfqpoint{5.918823in}{0.739656in}}%
\pgfpathlineto{\pgfqpoint{5.918527in}{0.739656in}}%
\pgfpathlineto{\pgfqpoint{5.918231in}{0.739656in}}%
\pgfpathlineto{\pgfqpoint{5.917935in}{0.739656in}}%
\pgfpathlineto{\pgfqpoint{5.917639in}{0.739656in}}%
\pgfpathlineto{\pgfqpoint{5.917343in}{0.739656in}}%
\pgfpathlineto{\pgfqpoint{5.917047in}{0.739656in}}%
\pgfpathlineto{\pgfqpoint{5.916751in}{0.739656in}}%
\pgfpathlineto{\pgfqpoint{5.916455in}{0.739656in}}%
\pgfpathlineto{\pgfqpoint{5.916159in}{0.739656in}}%
\pgfpathlineto{\pgfqpoint{5.915863in}{0.739656in}}%
\pgfpathlineto{\pgfqpoint{5.915567in}{0.739656in}}%
\pgfpathlineto{\pgfqpoint{5.915271in}{0.739656in}}%
\pgfpathlineto{\pgfqpoint{5.914975in}{0.739656in}}%
\pgfpathlineto{\pgfqpoint{5.914679in}{0.739656in}}%
\pgfpathlineto{\pgfqpoint{5.914383in}{0.739656in}}%
\pgfpathlineto{\pgfqpoint{5.914087in}{0.739656in}}%
\pgfpathlineto{\pgfqpoint{5.913791in}{0.739656in}}%
\pgfpathlineto{\pgfqpoint{5.913495in}{0.739656in}}%
\pgfpathlineto{\pgfqpoint{5.913199in}{0.739656in}}%
\pgfpathlineto{\pgfqpoint{5.912903in}{0.739656in}}%
\pgfpathlineto{\pgfqpoint{5.912607in}{0.739656in}}%
\pgfpathlineto{\pgfqpoint{5.912311in}{0.739656in}}%
\pgfpathlineto{\pgfqpoint{5.912015in}{0.739656in}}%
\pgfpathlineto{\pgfqpoint{5.911719in}{0.739656in}}%
\pgfpathlineto{\pgfqpoint{5.911423in}{0.739656in}}%
\pgfpathlineto{\pgfqpoint{5.911127in}{0.739656in}}%
\pgfpathlineto{\pgfqpoint{5.910831in}{0.739656in}}%
\pgfpathlineto{\pgfqpoint{5.910535in}{0.739656in}}%
\pgfpathlineto{\pgfqpoint{5.910239in}{0.739656in}}%
\pgfpathlineto{\pgfqpoint{5.909943in}{0.739656in}}%
\pgfpathlineto{\pgfqpoint{5.909647in}{0.739656in}}%
\pgfpathlineto{\pgfqpoint{5.909351in}{0.739656in}}%
\pgfpathlineto{\pgfqpoint{5.909055in}{0.739656in}}%
\pgfpathlineto{\pgfqpoint{5.908759in}{0.739656in}}%
\pgfpathlineto{\pgfqpoint{5.908463in}{0.739656in}}%
\pgfpathlineto{\pgfqpoint{5.908167in}{0.739656in}}%
\pgfpathlineto{\pgfqpoint{5.907871in}{0.739656in}}%
\pgfpathlineto{\pgfqpoint{5.907575in}{0.739656in}}%
\pgfpathlineto{\pgfqpoint{5.907279in}{0.739656in}}%
\pgfpathlineto{\pgfqpoint{5.906983in}{0.739656in}}%
\pgfpathlineto{\pgfqpoint{5.906687in}{0.739656in}}%
\pgfpathlineto{\pgfqpoint{5.906391in}{0.739656in}}%
\pgfpathlineto{\pgfqpoint{5.906094in}{0.739656in}}%
\pgfpathlineto{\pgfqpoint{5.905798in}{0.739656in}}%
\pgfpathlineto{\pgfqpoint{5.905502in}{0.739656in}}%
\pgfpathlineto{\pgfqpoint{5.905206in}{0.739656in}}%
\pgfpathlineto{\pgfqpoint{5.904910in}{0.739656in}}%
\pgfpathlineto{\pgfqpoint{5.904614in}{0.739656in}}%
\pgfpathlineto{\pgfqpoint{5.904318in}{0.739656in}}%
\pgfpathlineto{\pgfqpoint{5.904022in}{0.739656in}}%
\pgfpathlineto{\pgfqpoint{5.903726in}{0.739656in}}%
\pgfpathlineto{\pgfqpoint{5.903430in}{0.739656in}}%
\pgfpathlineto{\pgfqpoint{5.903134in}{0.739656in}}%
\pgfpathlineto{\pgfqpoint{5.902838in}{0.739656in}}%
\pgfpathlineto{\pgfqpoint{5.902542in}{0.739656in}}%
\pgfpathlineto{\pgfqpoint{5.902246in}{0.739656in}}%
\pgfpathlineto{\pgfqpoint{5.901950in}{0.739656in}}%
\pgfpathlineto{\pgfqpoint{5.901654in}{0.739656in}}%
\pgfpathlineto{\pgfqpoint{5.901358in}{0.739656in}}%
\pgfpathlineto{\pgfqpoint{5.901062in}{0.739656in}}%
\pgfpathlineto{\pgfqpoint{5.900766in}{0.739656in}}%
\pgfpathlineto{\pgfqpoint{5.900470in}{0.739656in}}%
\pgfpathlineto{\pgfqpoint{5.900174in}{0.739656in}}%
\pgfpathlineto{\pgfqpoint{5.899878in}{0.739656in}}%
\pgfpathlineto{\pgfqpoint{5.899582in}{0.739656in}}%
\pgfpathlineto{\pgfqpoint{5.899286in}{0.739656in}}%
\pgfpathlineto{\pgfqpoint{5.898990in}{0.739656in}}%
\pgfpathlineto{\pgfqpoint{5.898694in}{0.739656in}}%
\pgfpathlineto{\pgfqpoint{5.898398in}{0.739656in}}%
\pgfpathlineto{\pgfqpoint{5.898102in}{0.739656in}}%
\pgfpathlineto{\pgfqpoint{5.897806in}{0.739656in}}%
\pgfpathlineto{\pgfqpoint{5.897510in}{0.739656in}}%
\pgfpathlineto{\pgfqpoint{5.897214in}{0.739656in}}%
\pgfpathlineto{\pgfqpoint{5.896918in}{0.739656in}}%
\pgfpathlineto{\pgfqpoint{5.896622in}{0.739656in}}%
\pgfpathlineto{\pgfqpoint{5.896326in}{0.739656in}}%
\pgfpathlineto{\pgfqpoint{5.896030in}{0.739656in}}%
\pgfpathlineto{\pgfqpoint{5.895734in}{0.739656in}}%
\pgfpathlineto{\pgfqpoint{5.895438in}{0.739656in}}%
\pgfpathlineto{\pgfqpoint{5.895142in}{0.739656in}}%
\pgfpathlineto{\pgfqpoint{5.894846in}{0.739656in}}%
\pgfpathlineto{\pgfqpoint{5.894550in}{0.739656in}}%
\pgfpathlineto{\pgfqpoint{5.894254in}{0.739656in}}%
\pgfpathlineto{\pgfqpoint{5.893958in}{0.739656in}}%
\pgfpathlineto{\pgfqpoint{5.893662in}{0.739656in}}%
\pgfpathlineto{\pgfqpoint{5.893366in}{0.739656in}}%
\pgfpathlineto{\pgfqpoint{5.893070in}{0.739656in}}%
\pgfpathlineto{\pgfqpoint{5.892774in}{0.739656in}}%
\pgfpathlineto{\pgfqpoint{5.892478in}{0.739656in}}%
\pgfpathlineto{\pgfqpoint{5.892182in}{0.739656in}}%
\pgfpathlineto{\pgfqpoint{5.891886in}{0.739656in}}%
\pgfpathlineto{\pgfqpoint{5.891590in}{0.739656in}}%
\pgfpathlineto{\pgfqpoint{5.891294in}{0.739656in}}%
\pgfpathlineto{\pgfqpoint{5.890998in}{0.739656in}}%
\pgfpathlineto{\pgfqpoint{5.890702in}{0.739656in}}%
\pgfpathlineto{\pgfqpoint{5.890406in}{0.739656in}}%
\pgfpathlineto{\pgfqpoint{5.890110in}{0.739656in}}%
\pgfpathlineto{\pgfqpoint{5.889814in}{0.739656in}}%
\pgfpathlineto{\pgfqpoint{5.889518in}{0.739656in}}%
\pgfpathlineto{\pgfqpoint{5.889222in}{0.739656in}}%
\pgfpathlineto{\pgfqpoint{5.888926in}{0.739656in}}%
\pgfpathlineto{\pgfqpoint{5.888630in}{0.739656in}}%
\pgfpathlineto{\pgfqpoint{5.888334in}{0.739656in}}%
\pgfpathlineto{\pgfqpoint{5.888038in}{0.739656in}}%
\pgfpathlineto{\pgfqpoint{5.887742in}{0.739656in}}%
\pgfpathlineto{\pgfqpoint{5.887446in}{0.739656in}}%
\pgfpathlineto{\pgfqpoint{5.887150in}{0.739656in}}%
\pgfpathlineto{\pgfqpoint{5.886854in}{0.739656in}}%
\pgfpathlineto{\pgfqpoint{5.886558in}{0.739656in}}%
\pgfpathlineto{\pgfqpoint{5.886262in}{0.739656in}}%
\pgfpathlineto{\pgfqpoint{5.885966in}{0.739656in}}%
\pgfpathlineto{\pgfqpoint{5.885670in}{0.739656in}}%
\pgfpathlineto{\pgfqpoint{5.885374in}{0.739656in}}%
\pgfpathlineto{\pgfqpoint{5.885078in}{0.739656in}}%
\pgfpathlineto{\pgfqpoint{5.884782in}{0.739656in}}%
\pgfpathlineto{\pgfqpoint{5.884486in}{0.739656in}}%
\pgfpathlineto{\pgfqpoint{5.884190in}{0.739656in}}%
\pgfpathlineto{\pgfqpoint{5.883894in}{0.739656in}}%
\pgfpathlineto{\pgfqpoint{5.883598in}{0.739656in}}%
\pgfpathlineto{\pgfqpoint{5.883302in}{0.739656in}}%
\pgfpathlineto{\pgfqpoint{5.883006in}{0.739656in}}%
\pgfpathlineto{\pgfqpoint{5.882710in}{0.739656in}}%
\pgfpathlineto{\pgfqpoint{5.882414in}{0.739656in}}%
\pgfpathlineto{\pgfqpoint{5.882118in}{0.739656in}}%
\pgfpathlineto{\pgfqpoint{5.881822in}{0.739656in}}%
\pgfpathlineto{\pgfqpoint{5.881526in}{0.739656in}}%
\pgfpathlineto{\pgfqpoint{5.881230in}{0.739656in}}%
\pgfpathlineto{\pgfqpoint{5.880934in}{0.739656in}}%
\pgfpathlineto{\pgfqpoint{5.880638in}{0.739656in}}%
\pgfpathlineto{\pgfqpoint{5.880342in}{0.739656in}}%
\pgfpathlineto{\pgfqpoint{5.880046in}{0.739656in}}%
\pgfpathlineto{\pgfqpoint{5.879750in}{0.739656in}}%
\pgfpathlineto{\pgfqpoint{5.879454in}{0.739656in}}%
\pgfpathlineto{\pgfqpoint{5.879158in}{0.739656in}}%
\pgfpathlineto{\pgfqpoint{5.878862in}{0.739656in}}%
\pgfpathlineto{\pgfqpoint{5.878566in}{0.739656in}}%
\pgfpathlineto{\pgfqpoint{5.878270in}{0.739656in}}%
\pgfpathlineto{\pgfqpoint{5.877974in}{0.739656in}}%
\pgfpathlineto{\pgfqpoint{5.877678in}{0.739656in}}%
\pgfpathlineto{\pgfqpoint{5.877382in}{0.739656in}}%
\pgfpathlineto{\pgfqpoint{5.877086in}{0.739656in}}%
\pgfpathlineto{\pgfqpoint{5.876790in}{0.739656in}}%
\pgfpathlineto{\pgfqpoint{5.876494in}{0.739656in}}%
\pgfpathlineto{\pgfqpoint{5.876198in}{0.739656in}}%
\pgfpathlineto{\pgfqpoint{5.875902in}{0.739656in}}%
\pgfpathlineto{\pgfqpoint{5.875606in}{0.739656in}}%
\pgfpathlineto{\pgfqpoint{5.875310in}{0.739656in}}%
\pgfpathlineto{\pgfqpoint{5.875014in}{0.739656in}}%
\pgfpathlineto{\pgfqpoint{5.874718in}{0.739656in}}%
\pgfpathlineto{\pgfqpoint{5.874422in}{0.739656in}}%
\pgfpathlineto{\pgfqpoint{5.874126in}{0.739656in}}%
\pgfpathlineto{\pgfqpoint{5.873830in}{0.739656in}}%
\pgfpathlineto{\pgfqpoint{5.873534in}{0.739656in}}%
\pgfpathlineto{\pgfqpoint{5.873238in}{0.739656in}}%
\pgfpathlineto{\pgfqpoint{5.872942in}{0.739656in}}%
\pgfpathlineto{\pgfqpoint{5.872646in}{0.739656in}}%
\pgfpathlineto{\pgfqpoint{5.872350in}{0.739656in}}%
\pgfpathlineto{\pgfqpoint{5.872054in}{0.739656in}}%
\pgfpathlineto{\pgfqpoint{5.871758in}{0.739656in}}%
\pgfpathlineto{\pgfqpoint{5.871462in}{0.739656in}}%
\pgfpathlineto{\pgfqpoint{5.871166in}{0.739656in}}%
\pgfpathlineto{\pgfqpoint{5.870870in}{0.739656in}}%
\pgfpathlineto{\pgfqpoint{5.870574in}{0.739656in}}%
\pgfpathlineto{\pgfqpoint{5.870278in}{0.739656in}}%
\pgfpathlineto{\pgfqpoint{5.869982in}{0.739656in}}%
\pgfpathlineto{\pgfqpoint{5.869686in}{0.739656in}}%
\pgfpathlineto{\pgfqpoint{5.869390in}{0.739656in}}%
\pgfpathlineto{\pgfqpoint{5.869094in}{0.739656in}}%
\pgfpathlineto{\pgfqpoint{5.868798in}{0.739656in}}%
\pgfpathlineto{\pgfqpoint{5.868502in}{0.739656in}}%
\pgfpathlineto{\pgfqpoint{5.868206in}{0.739656in}}%
\pgfpathlineto{\pgfqpoint{5.867910in}{0.739656in}}%
\pgfpathlineto{\pgfqpoint{5.867614in}{0.739656in}}%
\pgfpathlineto{\pgfqpoint{5.867318in}{0.739656in}}%
\pgfpathlineto{\pgfqpoint{5.867022in}{0.739656in}}%
\pgfpathlineto{\pgfqpoint{5.866726in}{0.739656in}}%
\pgfpathlineto{\pgfqpoint{5.866430in}{0.739656in}}%
\pgfpathlineto{\pgfqpoint{5.866134in}{0.739656in}}%
\pgfpathlineto{\pgfqpoint{5.865838in}{0.739656in}}%
\pgfpathlineto{\pgfqpoint{5.865542in}{0.739656in}}%
\pgfpathlineto{\pgfqpoint{5.865246in}{0.739656in}}%
\pgfpathlineto{\pgfqpoint{5.864950in}{0.739656in}}%
\pgfpathlineto{\pgfqpoint{5.864654in}{0.739656in}}%
\pgfpathlineto{\pgfqpoint{5.864358in}{0.739656in}}%
\pgfpathlineto{\pgfqpoint{5.864062in}{0.739656in}}%
\pgfpathlineto{\pgfqpoint{5.863766in}{0.739656in}}%
\pgfpathlineto{\pgfqpoint{5.863470in}{0.739656in}}%
\pgfpathlineto{\pgfqpoint{5.863174in}{0.739656in}}%
\pgfpathlineto{\pgfqpoint{5.862878in}{0.739656in}}%
\pgfpathlineto{\pgfqpoint{5.862582in}{0.739656in}}%
\pgfpathlineto{\pgfqpoint{5.862286in}{0.739656in}}%
\pgfpathlineto{\pgfqpoint{5.861990in}{0.739656in}}%
\pgfpathlineto{\pgfqpoint{5.861694in}{0.739656in}}%
\pgfpathlineto{\pgfqpoint{5.861398in}{0.739656in}}%
\pgfpathlineto{\pgfqpoint{5.861102in}{0.739656in}}%
\pgfpathlineto{\pgfqpoint{5.860806in}{0.739656in}}%
\pgfpathlineto{\pgfqpoint{5.860510in}{0.739656in}}%
\pgfpathlineto{\pgfqpoint{5.860214in}{0.739656in}}%
\pgfpathlineto{\pgfqpoint{5.859918in}{0.739656in}}%
\pgfpathlineto{\pgfqpoint{5.859622in}{0.739656in}}%
\pgfpathlineto{\pgfqpoint{5.859326in}{0.739656in}}%
\pgfpathlineto{\pgfqpoint{5.859030in}{0.739656in}}%
\pgfpathlineto{\pgfqpoint{5.858734in}{0.739656in}}%
\pgfpathlineto{\pgfqpoint{5.858438in}{0.739656in}}%
\pgfpathlineto{\pgfqpoint{5.858142in}{0.739656in}}%
\pgfpathlineto{\pgfqpoint{5.857846in}{0.739656in}}%
\pgfpathlineto{\pgfqpoint{5.857550in}{0.739656in}}%
\pgfpathlineto{\pgfqpoint{5.857254in}{0.739656in}}%
\pgfpathlineto{\pgfqpoint{5.856958in}{0.739656in}}%
\pgfpathlineto{\pgfqpoint{5.856662in}{0.739656in}}%
\pgfpathlineto{\pgfqpoint{5.856366in}{0.739656in}}%
\pgfpathlineto{\pgfqpoint{5.856070in}{0.739656in}}%
\pgfpathlineto{\pgfqpoint{5.855774in}{0.739656in}}%
\pgfpathlineto{\pgfqpoint{5.855478in}{0.739656in}}%
\pgfpathlineto{\pgfqpoint{5.855182in}{0.739656in}}%
\pgfpathlineto{\pgfqpoint{5.854886in}{0.739656in}}%
\pgfpathlineto{\pgfqpoint{5.854590in}{0.739656in}}%
\pgfpathlineto{\pgfqpoint{5.854294in}{0.739656in}}%
\pgfpathlineto{\pgfqpoint{5.853998in}{0.739656in}}%
\pgfpathlineto{\pgfqpoint{5.853702in}{0.739656in}}%
\pgfpathlineto{\pgfqpoint{5.853406in}{0.739656in}}%
\pgfpathlineto{\pgfqpoint{5.853110in}{0.739656in}}%
\pgfpathlineto{\pgfqpoint{5.852814in}{0.739656in}}%
\pgfpathlineto{\pgfqpoint{5.852518in}{0.739656in}}%
\pgfpathlineto{\pgfqpoint{5.852222in}{0.739656in}}%
\pgfpathlineto{\pgfqpoint{5.851926in}{0.739656in}}%
\pgfpathlineto{\pgfqpoint{5.851630in}{0.739656in}}%
\pgfpathlineto{\pgfqpoint{5.851334in}{0.739656in}}%
\pgfpathlineto{\pgfqpoint{5.851038in}{0.739656in}}%
\pgfpathlineto{\pgfqpoint{5.850742in}{0.739656in}}%
\pgfpathlineto{\pgfqpoint{5.850446in}{0.739656in}}%
\pgfpathlineto{\pgfqpoint{5.850150in}{0.739656in}}%
\pgfpathlineto{\pgfqpoint{5.849854in}{0.739656in}}%
\pgfpathlineto{\pgfqpoint{5.849558in}{0.739656in}}%
\pgfpathlineto{\pgfqpoint{5.849262in}{0.739656in}}%
\pgfpathlineto{\pgfqpoint{5.848966in}{0.739656in}}%
\pgfpathlineto{\pgfqpoint{5.848670in}{0.739656in}}%
\pgfpathlineto{\pgfqpoint{5.848374in}{0.739656in}}%
\pgfpathlineto{\pgfqpoint{5.848078in}{0.739656in}}%
\pgfpathlineto{\pgfqpoint{5.847782in}{0.739656in}}%
\pgfpathlineto{\pgfqpoint{5.847486in}{0.739656in}}%
\pgfpathlineto{\pgfqpoint{5.847190in}{0.739656in}}%
\pgfpathlineto{\pgfqpoint{5.846894in}{0.739656in}}%
\pgfpathlineto{\pgfqpoint{5.846598in}{0.739656in}}%
\pgfpathlineto{\pgfqpoint{5.846302in}{0.739656in}}%
\pgfpathlineto{\pgfqpoint{5.846006in}{0.739656in}}%
\pgfpathlineto{\pgfqpoint{5.845710in}{0.739656in}}%
\pgfpathlineto{\pgfqpoint{5.845414in}{0.739656in}}%
\pgfpathlineto{\pgfqpoint{5.845118in}{0.739656in}}%
\pgfpathlineto{\pgfqpoint{5.844822in}{0.739656in}}%
\pgfpathlineto{\pgfqpoint{5.844526in}{0.739656in}}%
\pgfpathlineto{\pgfqpoint{5.844230in}{0.739656in}}%
\pgfpathlineto{\pgfqpoint{5.843934in}{0.739656in}}%
\pgfpathlineto{\pgfqpoint{5.843638in}{0.739656in}}%
\pgfpathlineto{\pgfqpoint{5.843342in}{0.739656in}}%
\pgfpathlineto{\pgfqpoint{5.843046in}{0.739656in}}%
\pgfpathlineto{\pgfqpoint{5.842750in}{0.739656in}}%
\pgfpathlineto{\pgfqpoint{5.842454in}{0.739656in}}%
\pgfpathlineto{\pgfqpoint{5.842158in}{0.739656in}}%
\pgfpathlineto{\pgfqpoint{5.841862in}{0.739656in}}%
\pgfpathlineto{\pgfqpoint{5.841566in}{0.739656in}}%
\pgfpathlineto{\pgfqpoint{5.841270in}{0.739656in}}%
\pgfpathlineto{\pgfqpoint{5.840974in}{0.739656in}}%
\pgfpathlineto{\pgfqpoint{5.840678in}{0.739656in}}%
\pgfpathlineto{\pgfqpoint{5.840382in}{0.739656in}}%
\pgfpathlineto{\pgfqpoint{5.840086in}{0.739656in}}%
\pgfpathlineto{\pgfqpoint{5.839790in}{0.739656in}}%
\pgfpathlineto{\pgfqpoint{5.839494in}{0.739656in}}%
\pgfpathlineto{\pgfqpoint{5.839198in}{0.739656in}}%
\pgfpathlineto{\pgfqpoint{5.838902in}{0.739656in}}%
\pgfpathlineto{\pgfqpoint{5.838605in}{0.739656in}}%
\pgfpathlineto{\pgfqpoint{5.838309in}{0.739656in}}%
\pgfpathlineto{\pgfqpoint{5.838013in}{0.739656in}}%
\pgfpathlineto{\pgfqpoint{5.837717in}{0.739656in}}%
\pgfpathlineto{\pgfqpoint{5.837421in}{0.739656in}}%
\pgfpathlineto{\pgfqpoint{5.837125in}{0.739656in}}%
\pgfpathlineto{\pgfqpoint{5.836829in}{0.739656in}}%
\pgfpathlineto{\pgfqpoint{5.836533in}{0.739656in}}%
\pgfpathlineto{\pgfqpoint{5.836237in}{0.739656in}}%
\pgfpathlineto{\pgfqpoint{5.835941in}{0.739656in}}%
\pgfpathlineto{\pgfqpoint{5.835645in}{0.739656in}}%
\pgfpathlineto{\pgfqpoint{5.835349in}{0.739656in}}%
\pgfpathlineto{\pgfqpoint{5.835053in}{0.739656in}}%
\pgfpathlineto{\pgfqpoint{5.834757in}{0.739656in}}%
\pgfpathlineto{\pgfqpoint{5.834461in}{0.739656in}}%
\pgfpathlineto{\pgfqpoint{5.834165in}{0.739656in}}%
\pgfpathlineto{\pgfqpoint{5.833869in}{0.739656in}}%
\pgfpathlineto{\pgfqpoint{5.833573in}{0.739656in}}%
\pgfpathlineto{\pgfqpoint{5.833277in}{0.739656in}}%
\pgfpathlineto{\pgfqpoint{5.832981in}{0.739656in}}%
\pgfpathlineto{\pgfqpoint{5.832685in}{0.739656in}}%
\pgfpathlineto{\pgfqpoint{5.832389in}{0.739656in}}%
\pgfpathlineto{\pgfqpoint{5.832093in}{0.739656in}}%
\pgfpathlineto{\pgfqpoint{5.831797in}{0.739656in}}%
\pgfpathlineto{\pgfqpoint{5.831501in}{0.739656in}}%
\pgfpathlineto{\pgfqpoint{5.831205in}{0.739656in}}%
\pgfpathlineto{\pgfqpoint{5.830909in}{0.739656in}}%
\pgfpathlineto{\pgfqpoint{5.830613in}{0.739656in}}%
\pgfpathlineto{\pgfqpoint{5.830317in}{0.739656in}}%
\pgfpathlineto{\pgfqpoint{5.830021in}{0.739656in}}%
\pgfpathlineto{\pgfqpoint{5.829725in}{0.739656in}}%
\pgfpathlineto{\pgfqpoint{5.829429in}{0.739656in}}%
\pgfpathlineto{\pgfqpoint{5.829133in}{0.739656in}}%
\pgfpathlineto{\pgfqpoint{5.828837in}{0.739656in}}%
\pgfpathlineto{\pgfqpoint{5.828541in}{0.739656in}}%
\pgfpathlineto{\pgfqpoint{5.828245in}{0.739656in}}%
\pgfpathlineto{\pgfqpoint{5.827949in}{0.739656in}}%
\pgfpathlineto{\pgfqpoint{5.827653in}{0.739656in}}%
\pgfpathlineto{\pgfqpoint{5.827357in}{0.739656in}}%
\pgfpathlineto{\pgfqpoint{5.827061in}{0.739656in}}%
\pgfpathlineto{\pgfqpoint{5.826765in}{0.739656in}}%
\pgfpathlineto{\pgfqpoint{5.826469in}{0.739656in}}%
\pgfpathlineto{\pgfqpoint{5.826173in}{0.739656in}}%
\pgfpathlineto{\pgfqpoint{5.825877in}{0.739656in}}%
\pgfpathlineto{\pgfqpoint{5.825581in}{0.739656in}}%
\pgfpathlineto{\pgfqpoint{5.825285in}{0.739656in}}%
\pgfpathlineto{\pgfqpoint{5.824989in}{0.739656in}}%
\pgfpathlineto{\pgfqpoint{5.824693in}{0.739656in}}%
\pgfpathlineto{\pgfqpoint{5.824397in}{0.739656in}}%
\pgfpathlineto{\pgfqpoint{5.824101in}{0.739656in}}%
\pgfpathlineto{\pgfqpoint{5.823805in}{0.739656in}}%
\pgfpathlineto{\pgfqpoint{5.823509in}{0.739656in}}%
\pgfpathlineto{\pgfqpoint{5.823213in}{0.739656in}}%
\pgfpathlineto{\pgfqpoint{5.822917in}{0.739656in}}%
\pgfpathlineto{\pgfqpoint{5.822621in}{0.739656in}}%
\pgfpathlineto{\pgfqpoint{5.822325in}{0.739656in}}%
\pgfpathlineto{\pgfqpoint{5.822029in}{0.739656in}}%
\pgfpathlineto{\pgfqpoint{5.821733in}{0.739656in}}%
\pgfpathlineto{\pgfqpoint{5.821437in}{0.739656in}}%
\pgfpathlineto{\pgfqpoint{5.821141in}{0.739656in}}%
\pgfpathlineto{\pgfqpoint{5.820845in}{0.739656in}}%
\pgfpathlineto{\pgfqpoint{5.820549in}{0.739656in}}%
\pgfpathlineto{\pgfqpoint{5.820253in}{0.739656in}}%
\pgfpathlineto{\pgfqpoint{5.819957in}{0.739656in}}%
\pgfpathlineto{\pgfqpoint{5.819661in}{0.739656in}}%
\pgfpathlineto{\pgfqpoint{5.819365in}{0.739656in}}%
\pgfpathlineto{\pgfqpoint{5.819069in}{0.739656in}}%
\pgfpathlineto{\pgfqpoint{5.818773in}{0.739656in}}%
\pgfpathlineto{\pgfqpoint{5.818477in}{0.739656in}}%
\pgfpathlineto{\pgfqpoint{5.818181in}{0.739656in}}%
\pgfpathlineto{\pgfqpoint{5.817885in}{0.739656in}}%
\pgfpathlineto{\pgfqpoint{5.817589in}{0.739656in}}%
\pgfpathlineto{\pgfqpoint{5.817293in}{0.739656in}}%
\pgfpathlineto{\pgfqpoint{5.816997in}{0.739656in}}%
\pgfpathlineto{\pgfqpoint{5.816701in}{0.739656in}}%
\pgfpathlineto{\pgfqpoint{5.816405in}{0.739656in}}%
\pgfpathlineto{\pgfqpoint{5.816109in}{0.739656in}}%
\pgfpathlineto{\pgfqpoint{5.815813in}{0.739656in}}%
\pgfpathlineto{\pgfqpoint{5.815517in}{0.739656in}}%
\pgfpathlineto{\pgfqpoint{5.815221in}{0.739656in}}%
\pgfpathlineto{\pgfqpoint{5.814925in}{0.739656in}}%
\pgfpathlineto{\pgfqpoint{5.814629in}{0.739656in}}%
\pgfpathlineto{\pgfqpoint{5.814333in}{0.739656in}}%
\pgfpathlineto{\pgfqpoint{5.814037in}{0.739656in}}%
\pgfpathlineto{\pgfqpoint{5.813741in}{0.739656in}}%
\pgfpathlineto{\pgfqpoint{5.813445in}{0.739656in}}%
\pgfpathlineto{\pgfqpoint{5.813149in}{0.739656in}}%
\pgfpathlineto{\pgfqpoint{5.812853in}{0.739656in}}%
\pgfpathlineto{\pgfqpoint{5.812557in}{0.739656in}}%
\pgfpathlineto{\pgfqpoint{5.812261in}{0.739656in}}%
\pgfpathlineto{\pgfqpoint{5.811965in}{0.739656in}}%
\pgfpathlineto{\pgfqpoint{5.811669in}{0.739656in}}%
\pgfpathlineto{\pgfqpoint{5.811373in}{0.739656in}}%
\pgfpathlineto{\pgfqpoint{5.811077in}{0.739656in}}%
\pgfpathlineto{\pgfqpoint{5.810781in}{0.739656in}}%
\pgfpathlineto{\pgfqpoint{5.810485in}{0.739656in}}%
\pgfpathlineto{\pgfqpoint{5.810189in}{0.739656in}}%
\pgfpathlineto{\pgfqpoint{5.809893in}{0.739656in}}%
\pgfpathlineto{\pgfqpoint{5.809597in}{0.739656in}}%
\pgfpathlineto{\pgfqpoint{5.809301in}{0.739656in}}%
\pgfpathlineto{\pgfqpoint{5.809005in}{0.739656in}}%
\pgfpathlineto{\pgfqpoint{5.808709in}{0.739656in}}%
\pgfpathlineto{\pgfqpoint{5.808413in}{0.739656in}}%
\pgfpathlineto{\pgfqpoint{5.808117in}{0.739656in}}%
\pgfpathlineto{\pgfqpoint{5.807821in}{0.739656in}}%
\pgfpathlineto{\pgfqpoint{5.807525in}{0.739656in}}%
\pgfpathlineto{\pgfqpoint{5.807229in}{0.739656in}}%
\pgfpathlineto{\pgfqpoint{5.806933in}{0.739656in}}%
\pgfpathlineto{\pgfqpoint{5.806637in}{0.739656in}}%
\pgfpathlineto{\pgfqpoint{5.806341in}{0.739656in}}%
\pgfpathlineto{\pgfqpoint{5.806045in}{0.739656in}}%
\pgfpathlineto{\pgfqpoint{5.805749in}{0.739656in}}%
\pgfpathlineto{\pgfqpoint{5.805453in}{0.739656in}}%
\pgfpathlineto{\pgfqpoint{5.805157in}{0.739656in}}%
\pgfpathlineto{\pgfqpoint{5.804861in}{0.739656in}}%
\pgfpathlineto{\pgfqpoint{5.804565in}{0.739656in}}%
\pgfpathlineto{\pgfqpoint{5.804269in}{0.739656in}}%
\pgfpathlineto{\pgfqpoint{5.803973in}{0.739656in}}%
\pgfpathlineto{\pgfqpoint{5.803677in}{0.739656in}}%
\pgfpathlineto{\pgfqpoint{5.803381in}{0.739656in}}%
\pgfpathlineto{\pgfqpoint{5.803085in}{0.739656in}}%
\pgfpathlineto{\pgfqpoint{5.802789in}{0.739656in}}%
\pgfpathlineto{\pgfqpoint{5.802493in}{0.739656in}}%
\pgfpathlineto{\pgfqpoint{5.802197in}{0.739656in}}%
\pgfpathlineto{\pgfqpoint{5.801901in}{0.739656in}}%
\pgfpathlineto{\pgfqpoint{5.801605in}{0.739656in}}%
\pgfpathlineto{\pgfqpoint{5.801309in}{0.739656in}}%
\pgfpathlineto{\pgfqpoint{5.801013in}{0.739656in}}%
\pgfpathlineto{\pgfqpoint{5.800717in}{0.739656in}}%
\pgfpathlineto{\pgfqpoint{5.800421in}{0.739656in}}%
\pgfpathlineto{\pgfqpoint{5.800125in}{0.739656in}}%
\pgfpathlineto{\pgfqpoint{5.799829in}{0.739656in}}%
\pgfpathlineto{\pgfqpoint{5.799533in}{0.739656in}}%
\pgfpathlineto{\pgfqpoint{5.799237in}{0.739656in}}%
\pgfpathlineto{\pgfqpoint{5.798941in}{0.739656in}}%
\pgfpathlineto{\pgfqpoint{5.798645in}{0.739656in}}%
\pgfpathlineto{\pgfqpoint{5.798349in}{0.739656in}}%
\pgfpathlineto{\pgfqpoint{5.798053in}{0.739656in}}%
\pgfpathlineto{\pgfqpoint{5.797757in}{0.739656in}}%
\pgfpathlineto{\pgfqpoint{5.797461in}{0.739656in}}%
\pgfpathlineto{\pgfqpoint{5.797165in}{0.739656in}}%
\pgfpathlineto{\pgfqpoint{5.796869in}{0.739656in}}%
\pgfpathlineto{\pgfqpoint{5.796573in}{0.739656in}}%
\pgfpathlineto{\pgfqpoint{5.796277in}{0.739656in}}%
\pgfpathlineto{\pgfqpoint{5.795981in}{0.739656in}}%
\pgfpathlineto{\pgfqpoint{5.795685in}{0.739656in}}%
\pgfpathlineto{\pgfqpoint{5.795389in}{0.739656in}}%
\pgfpathlineto{\pgfqpoint{5.795093in}{0.739656in}}%
\pgfpathlineto{\pgfqpoint{5.794797in}{0.739656in}}%
\pgfpathlineto{\pgfqpoint{5.794501in}{0.739656in}}%
\pgfpathlineto{\pgfqpoint{5.794205in}{0.739656in}}%
\pgfpathlineto{\pgfqpoint{5.793909in}{0.739656in}}%
\pgfpathlineto{\pgfqpoint{5.793613in}{0.739656in}}%
\pgfpathlineto{\pgfqpoint{5.793317in}{0.739656in}}%
\pgfpathlineto{\pgfqpoint{5.793021in}{0.739656in}}%
\pgfpathlineto{\pgfqpoint{5.792725in}{0.739656in}}%
\pgfpathlineto{\pgfqpoint{5.792429in}{0.739656in}}%
\pgfpathlineto{\pgfqpoint{5.792133in}{0.739656in}}%
\pgfpathlineto{\pgfqpoint{5.791837in}{0.739656in}}%
\pgfpathlineto{\pgfqpoint{5.791541in}{0.739656in}}%
\pgfpathlineto{\pgfqpoint{5.791245in}{0.739656in}}%
\pgfpathlineto{\pgfqpoint{5.790949in}{0.739656in}}%
\pgfpathlineto{\pgfqpoint{5.790653in}{0.739656in}}%
\pgfpathlineto{\pgfqpoint{5.790357in}{0.739656in}}%
\pgfpathlineto{\pgfqpoint{5.790061in}{0.739656in}}%
\pgfpathlineto{\pgfqpoint{5.789765in}{0.739656in}}%
\pgfpathlineto{\pgfqpoint{5.789469in}{0.739656in}}%
\pgfpathlineto{\pgfqpoint{5.789173in}{0.739656in}}%
\pgfpathlineto{\pgfqpoint{5.788877in}{0.739656in}}%
\pgfpathlineto{\pgfqpoint{5.788581in}{0.739656in}}%
\pgfpathlineto{\pgfqpoint{5.788285in}{0.739656in}}%
\pgfpathlineto{\pgfqpoint{5.787989in}{0.739656in}}%
\pgfpathlineto{\pgfqpoint{5.787693in}{0.739656in}}%
\pgfpathlineto{\pgfqpoint{5.787397in}{0.739656in}}%
\pgfpathlineto{\pgfqpoint{5.787101in}{0.739656in}}%
\pgfpathlineto{\pgfqpoint{5.786805in}{0.739656in}}%
\pgfpathlineto{\pgfqpoint{5.786509in}{0.739656in}}%
\pgfpathlineto{\pgfqpoint{5.786213in}{0.739656in}}%
\pgfpathlineto{\pgfqpoint{5.785917in}{0.739656in}}%
\pgfpathlineto{\pgfqpoint{5.785621in}{0.739656in}}%
\pgfpathlineto{\pgfqpoint{5.785325in}{0.739656in}}%
\pgfpathlineto{\pgfqpoint{5.785029in}{0.739656in}}%
\pgfpathlineto{\pgfqpoint{5.784733in}{0.739656in}}%
\pgfpathlineto{\pgfqpoint{5.784437in}{0.739656in}}%
\pgfpathlineto{\pgfqpoint{5.784141in}{0.739656in}}%
\pgfpathlineto{\pgfqpoint{5.783845in}{0.739656in}}%
\pgfpathlineto{\pgfqpoint{5.783549in}{0.739656in}}%
\pgfpathlineto{\pgfqpoint{5.783253in}{0.739656in}}%
\pgfpathlineto{\pgfqpoint{5.782957in}{0.739656in}}%
\pgfpathlineto{\pgfqpoint{5.782661in}{0.739656in}}%
\pgfpathlineto{\pgfqpoint{5.782365in}{0.739656in}}%
\pgfpathlineto{\pgfqpoint{5.782069in}{0.739656in}}%
\pgfpathlineto{\pgfqpoint{5.781773in}{0.739656in}}%
\pgfpathlineto{\pgfqpoint{5.781477in}{0.739656in}}%
\pgfpathlineto{\pgfqpoint{5.781181in}{0.739656in}}%
\pgfpathlineto{\pgfqpoint{5.780885in}{0.739656in}}%
\pgfpathlineto{\pgfqpoint{5.780589in}{0.739656in}}%
\pgfpathlineto{\pgfqpoint{5.780293in}{0.739656in}}%
\pgfpathlineto{\pgfqpoint{5.779997in}{0.739656in}}%
\pgfpathlineto{\pgfqpoint{5.779701in}{0.739656in}}%
\pgfpathlineto{\pgfqpoint{5.779405in}{0.739656in}}%
\pgfpathlineto{\pgfqpoint{5.779109in}{0.739656in}}%
\pgfpathlineto{\pgfqpoint{5.778813in}{0.739656in}}%
\pgfpathlineto{\pgfqpoint{5.778517in}{0.739656in}}%
\pgfpathlineto{\pgfqpoint{5.778221in}{0.739656in}}%
\pgfpathlineto{\pgfqpoint{5.777925in}{0.739656in}}%
\pgfpathlineto{\pgfqpoint{5.777629in}{0.739656in}}%
\pgfpathlineto{\pgfqpoint{5.777333in}{0.739656in}}%
\pgfpathlineto{\pgfqpoint{5.777037in}{0.739656in}}%
\pgfpathlineto{\pgfqpoint{5.776741in}{0.739656in}}%
\pgfpathlineto{\pgfqpoint{5.776445in}{0.739656in}}%
\pgfpathlineto{\pgfqpoint{5.776149in}{0.739656in}}%
\pgfpathlineto{\pgfqpoint{5.775853in}{0.739656in}}%
\pgfpathlineto{\pgfqpoint{5.775557in}{0.739656in}}%
\pgfpathlineto{\pgfqpoint{5.775261in}{0.739656in}}%
\pgfpathlineto{\pgfqpoint{5.774965in}{0.739656in}}%
\pgfpathlineto{\pgfqpoint{5.774669in}{0.739656in}}%
\pgfpathlineto{\pgfqpoint{5.774373in}{0.739656in}}%
\pgfpathlineto{\pgfqpoint{5.774077in}{0.739656in}}%
\pgfpathlineto{\pgfqpoint{5.773781in}{0.739656in}}%
\pgfpathlineto{\pgfqpoint{5.773485in}{0.739656in}}%
\pgfpathlineto{\pgfqpoint{5.773189in}{0.739656in}}%
\pgfpathlineto{\pgfqpoint{5.772893in}{0.739656in}}%
\pgfpathlineto{\pgfqpoint{5.772597in}{0.739656in}}%
\pgfpathlineto{\pgfqpoint{5.772301in}{0.739656in}}%
\pgfpathlineto{\pgfqpoint{5.772005in}{0.739656in}}%
\pgfpathlineto{\pgfqpoint{5.771709in}{0.739656in}}%
\pgfpathlineto{\pgfqpoint{5.771412in}{0.739656in}}%
\pgfpathlineto{\pgfqpoint{5.771116in}{0.739656in}}%
\pgfpathlineto{\pgfqpoint{5.770820in}{0.739656in}}%
\pgfpathlineto{\pgfqpoint{5.770524in}{0.739656in}}%
\pgfpathlineto{\pgfqpoint{5.770228in}{0.739656in}}%
\pgfpathlineto{\pgfqpoint{5.769932in}{0.739656in}}%
\pgfpathlineto{\pgfqpoint{5.769636in}{0.739656in}}%
\pgfpathlineto{\pgfqpoint{5.769340in}{0.739656in}}%
\pgfpathlineto{\pgfqpoint{5.769044in}{0.739656in}}%
\pgfpathlineto{\pgfqpoint{5.768748in}{0.739656in}}%
\pgfpathlineto{\pgfqpoint{5.768452in}{0.739656in}}%
\pgfpathlineto{\pgfqpoint{5.768156in}{0.739656in}}%
\pgfpathlineto{\pgfqpoint{5.767860in}{0.739656in}}%
\pgfpathlineto{\pgfqpoint{5.767564in}{0.739656in}}%
\pgfpathlineto{\pgfqpoint{5.767268in}{0.739656in}}%
\pgfpathlineto{\pgfqpoint{5.766972in}{0.739656in}}%
\pgfpathlineto{\pgfqpoint{5.766676in}{0.739656in}}%
\pgfpathlineto{\pgfqpoint{5.766380in}{0.739656in}}%
\pgfpathlineto{\pgfqpoint{5.766084in}{0.739656in}}%
\pgfpathlineto{\pgfqpoint{5.765788in}{0.739656in}}%
\pgfpathlineto{\pgfqpoint{5.765492in}{0.739656in}}%
\pgfpathlineto{\pgfqpoint{5.765196in}{0.739656in}}%
\pgfpathlineto{\pgfqpoint{5.764900in}{0.739656in}}%
\pgfpathlineto{\pgfqpoint{5.764604in}{0.739656in}}%
\pgfpathlineto{\pgfqpoint{5.764308in}{0.739656in}}%
\pgfpathlineto{\pgfqpoint{5.764012in}{0.739656in}}%
\pgfpathlineto{\pgfqpoint{5.763716in}{0.739656in}}%
\pgfpathlineto{\pgfqpoint{5.763420in}{0.739656in}}%
\pgfpathlineto{\pgfqpoint{5.763124in}{0.739656in}}%
\pgfpathlineto{\pgfqpoint{5.762828in}{0.739656in}}%
\pgfpathlineto{\pgfqpoint{5.762532in}{0.739656in}}%
\pgfpathlineto{\pgfqpoint{5.762236in}{0.739656in}}%
\pgfpathlineto{\pgfqpoint{5.761940in}{0.739656in}}%
\pgfpathlineto{\pgfqpoint{5.761644in}{0.739656in}}%
\pgfpathlineto{\pgfqpoint{5.761348in}{0.739656in}}%
\pgfpathlineto{\pgfqpoint{5.761052in}{0.739656in}}%
\pgfpathlineto{\pgfqpoint{5.760756in}{0.739656in}}%
\pgfpathlineto{\pgfqpoint{5.760460in}{0.739656in}}%
\pgfpathlineto{\pgfqpoint{5.760164in}{0.739656in}}%
\pgfpathlineto{\pgfqpoint{5.759868in}{0.739656in}}%
\pgfpathlineto{\pgfqpoint{5.759572in}{0.739656in}}%
\pgfpathlineto{\pgfqpoint{5.759276in}{0.739656in}}%
\pgfpathlineto{\pgfqpoint{5.758980in}{0.739656in}}%
\pgfpathlineto{\pgfqpoint{5.758684in}{0.739656in}}%
\pgfpathlineto{\pgfqpoint{5.758388in}{0.739656in}}%
\pgfpathlineto{\pgfqpoint{5.758092in}{0.739656in}}%
\pgfpathlineto{\pgfqpoint{5.757796in}{0.739656in}}%
\pgfpathlineto{\pgfqpoint{5.757500in}{0.739656in}}%
\pgfpathlineto{\pgfqpoint{5.757204in}{0.739656in}}%
\pgfpathlineto{\pgfqpoint{5.756908in}{0.739656in}}%
\pgfpathlineto{\pgfqpoint{5.756612in}{0.739656in}}%
\pgfpathlineto{\pgfqpoint{5.756316in}{0.739656in}}%
\pgfpathlineto{\pgfqpoint{5.756020in}{0.739656in}}%
\pgfpathlineto{\pgfqpoint{5.755724in}{0.739656in}}%
\pgfpathlineto{\pgfqpoint{5.755428in}{0.739656in}}%
\pgfpathlineto{\pgfqpoint{5.755132in}{0.739656in}}%
\pgfpathlineto{\pgfqpoint{5.754836in}{0.739656in}}%
\pgfpathlineto{\pgfqpoint{5.754540in}{0.739656in}}%
\pgfpathlineto{\pgfqpoint{5.754244in}{0.739656in}}%
\pgfpathlineto{\pgfqpoint{5.753948in}{0.739656in}}%
\pgfpathlineto{\pgfqpoint{5.753652in}{0.739656in}}%
\pgfpathlineto{\pgfqpoint{5.753356in}{0.739656in}}%
\pgfpathlineto{\pgfqpoint{5.753060in}{0.739656in}}%
\pgfpathlineto{\pgfqpoint{5.752764in}{0.739656in}}%
\pgfpathlineto{\pgfqpoint{5.752468in}{0.739656in}}%
\pgfpathlineto{\pgfqpoint{5.752172in}{0.739656in}}%
\pgfpathlineto{\pgfqpoint{5.751876in}{0.739656in}}%
\pgfpathlineto{\pgfqpoint{5.751580in}{0.739656in}}%
\pgfpathlineto{\pgfqpoint{5.751284in}{0.739656in}}%
\pgfpathlineto{\pgfqpoint{5.750988in}{0.739656in}}%
\pgfpathlineto{\pgfqpoint{5.750692in}{0.739656in}}%
\pgfpathlineto{\pgfqpoint{5.750396in}{0.739656in}}%
\pgfpathlineto{\pgfqpoint{5.750100in}{0.739656in}}%
\pgfpathlineto{\pgfqpoint{5.749804in}{0.739656in}}%
\pgfpathlineto{\pgfqpoint{5.749508in}{0.739656in}}%
\pgfpathlineto{\pgfqpoint{5.749212in}{0.739656in}}%
\pgfpathlineto{\pgfqpoint{5.748916in}{0.739656in}}%
\pgfpathlineto{\pgfqpoint{5.748620in}{0.739656in}}%
\pgfpathlineto{\pgfqpoint{5.748324in}{0.739656in}}%
\pgfpathlineto{\pgfqpoint{5.748028in}{0.739656in}}%
\pgfpathlineto{\pgfqpoint{5.747732in}{0.739656in}}%
\pgfpathlineto{\pgfqpoint{5.747436in}{0.739656in}}%
\pgfpathlineto{\pgfqpoint{5.747140in}{0.739656in}}%
\pgfpathlineto{\pgfqpoint{5.746844in}{0.739656in}}%
\pgfpathlineto{\pgfqpoint{5.746548in}{0.739656in}}%
\pgfpathlineto{\pgfqpoint{5.746252in}{0.739656in}}%
\pgfpathlineto{\pgfqpoint{5.745956in}{0.739656in}}%
\pgfpathlineto{\pgfqpoint{5.745660in}{0.739656in}}%
\pgfpathlineto{\pgfqpoint{5.745364in}{0.739656in}}%
\pgfpathlineto{\pgfqpoint{5.745068in}{0.739656in}}%
\pgfpathlineto{\pgfqpoint{5.744772in}{0.739656in}}%
\pgfpathlineto{\pgfqpoint{5.744476in}{0.739656in}}%
\pgfpathlineto{\pgfqpoint{5.744180in}{0.739656in}}%
\pgfpathlineto{\pgfqpoint{5.743884in}{0.739656in}}%
\pgfpathlineto{\pgfqpoint{5.743588in}{0.739656in}}%
\pgfpathlineto{\pgfqpoint{5.743292in}{0.739656in}}%
\pgfpathlineto{\pgfqpoint{5.742996in}{0.739656in}}%
\pgfpathlineto{\pgfqpoint{5.742700in}{0.739656in}}%
\pgfpathlineto{\pgfqpoint{5.742404in}{0.739656in}}%
\pgfpathlineto{\pgfqpoint{5.742108in}{0.739656in}}%
\pgfpathlineto{\pgfqpoint{5.741812in}{0.739656in}}%
\pgfpathlineto{\pgfqpoint{5.741516in}{0.739656in}}%
\pgfpathlineto{\pgfqpoint{5.741220in}{0.739656in}}%
\pgfpathlineto{\pgfqpoint{5.740924in}{0.739656in}}%
\pgfpathlineto{\pgfqpoint{5.740628in}{0.739656in}}%
\pgfpathlineto{\pgfqpoint{5.740332in}{0.739656in}}%
\pgfpathlineto{\pgfqpoint{5.740036in}{0.739656in}}%
\pgfpathlineto{\pgfqpoint{5.739740in}{0.739656in}}%
\pgfpathlineto{\pgfqpoint{5.739444in}{0.739656in}}%
\pgfpathlineto{\pgfqpoint{5.739148in}{0.739656in}}%
\pgfpathlineto{\pgfqpoint{5.738852in}{0.739656in}}%
\pgfpathlineto{\pgfqpoint{5.738556in}{0.739656in}}%
\pgfpathlineto{\pgfqpoint{5.738260in}{0.739656in}}%
\pgfpathlineto{\pgfqpoint{5.737964in}{0.739656in}}%
\pgfpathlineto{\pgfqpoint{5.737668in}{0.739656in}}%
\pgfpathlineto{\pgfqpoint{5.737372in}{0.739656in}}%
\pgfpathlineto{\pgfqpoint{5.737076in}{0.739656in}}%
\pgfpathlineto{\pgfqpoint{5.736780in}{0.739656in}}%
\pgfpathlineto{\pgfqpoint{5.736484in}{0.739656in}}%
\pgfpathlineto{\pgfqpoint{5.736188in}{0.739656in}}%
\pgfpathlineto{\pgfqpoint{5.735892in}{0.739656in}}%
\pgfpathlineto{\pgfqpoint{5.735596in}{0.739656in}}%
\pgfpathlineto{\pgfqpoint{5.735300in}{0.739656in}}%
\pgfpathlineto{\pgfqpoint{5.735004in}{0.739656in}}%
\pgfpathlineto{\pgfqpoint{5.734708in}{0.739656in}}%
\pgfpathlineto{\pgfqpoint{5.734412in}{0.739656in}}%
\pgfpathlineto{\pgfqpoint{5.734116in}{0.739656in}}%
\pgfpathlineto{\pgfqpoint{5.733820in}{0.739656in}}%
\pgfpathlineto{\pgfqpoint{5.733524in}{0.739656in}}%
\pgfpathlineto{\pgfqpoint{5.733228in}{0.739656in}}%
\pgfpathlineto{\pgfqpoint{5.732932in}{0.739656in}}%
\pgfpathlineto{\pgfqpoint{5.732636in}{0.739656in}}%
\pgfpathlineto{\pgfqpoint{5.732340in}{0.739656in}}%
\pgfpathlineto{\pgfqpoint{5.732044in}{0.739656in}}%
\pgfpathlineto{\pgfqpoint{5.731748in}{0.739656in}}%
\pgfpathlineto{\pgfqpoint{5.731452in}{0.739656in}}%
\pgfpathlineto{\pgfqpoint{5.731156in}{0.739656in}}%
\pgfpathlineto{\pgfqpoint{5.730860in}{0.739656in}}%
\pgfpathlineto{\pgfqpoint{5.730564in}{0.739656in}}%
\pgfpathlineto{\pgfqpoint{5.730268in}{0.739656in}}%
\pgfpathlineto{\pgfqpoint{5.729972in}{0.739656in}}%
\pgfpathlineto{\pgfqpoint{5.729676in}{0.739656in}}%
\pgfpathlineto{\pgfqpoint{5.729380in}{0.739656in}}%
\pgfpathlineto{\pgfqpoint{5.729084in}{0.739656in}}%
\pgfpathlineto{\pgfqpoint{5.728788in}{0.739656in}}%
\pgfpathlineto{\pgfqpoint{5.728492in}{0.739656in}}%
\pgfpathlineto{\pgfqpoint{5.728196in}{0.739656in}}%
\pgfpathlineto{\pgfqpoint{5.727900in}{0.739656in}}%
\pgfpathlineto{\pgfqpoint{5.727604in}{0.739656in}}%
\pgfpathlineto{\pgfqpoint{5.727308in}{0.739656in}}%
\pgfpathlineto{\pgfqpoint{5.727012in}{0.739656in}}%
\pgfpathlineto{\pgfqpoint{5.726716in}{0.739656in}}%
\pgfpathlineto{\pgfqpoint{5.726420in}{0.739656in}}%
\pgfpathlineto{\pgfqpoint{5.726124in}{0.739656in}}%
\pgfpathlineto{\pgfqpoint{5.725828in}{0.739656in}}%
\pgfpathlineto{\pgfqpoint{5.725532in}{0.739656in}}%
\pgfpathlineto{\pgfqpoint{5.725236in}{0.739656in}}%
\pgfpathlineto{\pgfqpoint{5.724940in}{0.739656in}}%
\pgfpathlineto{\pgfqpoint{5.724644in}{0.739656in}}%
\pgfpathlineto{\pgfqpoint{5.724348in}{0.739656in}}%
\pgfpathlineto{\pgfqpoint{5.724052in}{0.739656in}}%
\pgfpathlineto{\pgfqpoint{5.723756in}{0.739656in}}%
\pgfpathlineto{\pgfqpoint{5.723460in}{0.739656in}}%
\pgfpathlineto{\pgfqpoint{5.723164in}{0.739656in}}%
\pgfpathlineto{\pgfqpoint{5.722868in}{0.739656in}}%
\pgfpathlineto{\pgfqpoint{5.722572in}{0.739656in}}%
\pgfpathlineto{\pgfqpoint{5.722276in}{0.739656in}}%
\pgfpathlineto{\pgfqpoint{5.721980in}{0.739656in}}%
\pgfpathlineto{\pgfqpoint{5.721684in}{0.739656in}}%
\pgfpathlineto{\pgfqpoint{5.721388in}{0.739656in}}%
\pgfpathlineto{\pgfqpoint{5.721092in}{0.739656in}}%
\pgfpathlineto{\pgfqpoint{5.720796in}{0.739656in}}%
\pgfpathlineto{\pgfqpoint{5.720500in}{0.739656in}}%
\pgfpathlineto{\pgfqpoint{5.720204in}{0.739656in}}%
\pgfpathlineto{\pgfqpoint{5.719908in}{0.739656in}}%
\pgfpathlineto{\pgfqpoint{5.719612in}{0.739656in}}%
\pgfpathlineto{\pgfqpoint{5.719316in}{0.739656in}}%
\pgfpathlineto{\pgfqpoint{5.719020in}{0.739656in}}%
\pgfpathlineto{\pgfqpoint{5.718724in}{0.739656in}}%
\pgfpathlineto{\pgfqpoint{5.718428in}{0.739656in}}%
\pgfpathlineto{\pgfqpoint{5.718132in}{0.739656in}}%
\pgfpathlineto{\pgfqpoint{5.717836in}{0.739656in}}%
\pgfpathlineto{\pgfqpoint{5.717540in}{0.739656in}}%
\pgfpathlineto{\pgfqpoint{5.717244in}{0.739656in}}%
\pgfpathlineto{\pgfqpoint{5.716948in}{0.739656in}}%
\pgfpathlineto{\pgfqpoint{5.716652in}{0.739656in}}%
\pgfpathlineto{\pgfqpoint{5.716356in}{0.739656in}}%
\pgfpathlineto{\pgfqpoint{5.716060in}{0.739656in}}%
\pgfpathlineto{\pgfqpoint{5.715764in}{0.739656in}}%
\pgfpathlineto{\pgfqpoint{5.715468in}{0.739656in}}%
\pgfpathlineto{\pgfqpoint{5.715172in}{0.739656in}}%
\pgfpathlineto{\pgfqpoint{5.714876in}{0.739656in}}%
\pgfpathlineto{\pgfqpoint{5.714580in}{0.739656in}}%
\pgfpathlineto{\pgfqpoint{5.714284in}{0.739656in}}%
\pgfpathlineto{\pgfqpoint{5.713988in}{0.739656in}}%
\pgfpathlineto{\pgfqpoint{5.713692in}{0.739656in}}%
\pgfpathlineto{\pgfqpoint{5.713396in}{0.739656in}}%
\pgfpathlineto{\pgfqpoint{5.713100in}{0.739656in}}%
\pgfpathlineto{\pgfqpoint{5.712804in}{0.739656in}}%
\pgfpathlineto{\pgfqpoint{5.712508in}{0.739656in}}%
\pgfpathlineto{\pgfqpoint{5.712212in}{0.739656in}}%
\pgfpathlineto{\pgfqpoint{5.711916in}{0.739656in}}%
\pgfpathlineto{\pgfqpoint{5.711620in}{0.739656in}}%
\pgfpathlineto{\pgfqpoint{5.711324in}{0.739656in}}%
\pgfpathlineto{\pgfqpoint{5.711028in}{0.739656in}}%
\pgfpathlineto{\pgfqpoint{5.710732in}{0.739656in}}%
\pgfpathlineto{\pgfqpoint{5.710436in}{0.739656in}}%
\pgfpathlineto{\pgfqpoint{5.710140in}{0.739656in}}%
\pgfpathlineto{\pgfqpoint{5.709844in}{0.739656in}}%
\pgfpathlineto{\pgfqpoint{5.709548in}{0.739656in}}%
\pgfpathlineto{\pgfqpoint{5.709252in}{0.739656in}}%
\pgfpathlineto{\pgfqpoint{5.708956in}{0.739656in}}%
\pgfpathlineto{\pgfqpoint{5.708660in}{0.739656in}}%
\pgfpathlineto{\pgfqpoint{5.708364in}{0.739656in}}%
\pgfpathlineto{\pgfqpoint{5.708068in}{0.739656in}}%
\pgfpathlineto{\pgfqpoint{5.707772in}{0.739656in}}%
\pgfpathlineto{\pgfqpoint{5.707476in}{0.739656in}}%
\pgfpathlineto{\pgfqpoint{5.707180in}{0.739656in}}%
\pgfpathlineto{\pgfqpoint{5.706884in}{0.739656in}}%
\pgfpathlineto{\pgfqpoint{5.706588in}{0.739656in}}%
\pgfpathlineto{\pgfqpoint{5.706292in}{0.739656in}}%
\pgfpathlineto{\pgfqpoint{5.705996in}{0.739656in}}%
\pgfpathlineto{\pgfqpoint{5.705700in}{0.739656in}}%
\pgfpathlineto{\pgfqpoint{5.705404in}{0.739656in}}%
\pgfpathlineto{\pgfqpoint{5.705108in}{0.739656in}}%
\pgfpathlineto{\pgfqpoint{5.704812in}{0.739656in}}%
\pgfpathlineto{\pgfqpoint{5.704516in}{0.739656in}}%
\pgfpathlineto{\pgfqpoint{5.704220in}{0.739656in}}%
\pgfpathlineto{\pgfqpoint{5.703923in}{0.739656in}}%
\pgfpathlineto{\pgfqpoint{5.703627in}{0.739656in}}%
\pgfpathlineto{\pgfqpoint{5.703331in}{0.739656in}}%
\pgfpathlineto{\pgfqpoint{5.703035in}{0.739656in}}%
\pgfpathlineto{\pgfqpoint{5.702739in}{0.739656in}}%
\pgfpathlineto{\pgfqpoint{5.702443in}{0.739656in}}%
\pgfpathlineto{\pgfqpoint{5.702147in}{0.739656in}}%
\pgfpathlineto{\pgfqpoint{5.701851in}{0.739656in}}%
\pgfpathlineto{\pgfqpoint{5.701555in}{0.739656in}}%
\pgfpathlineto{\pgfqpoint{5.701259in}{0.739656in}}%
\pgfpathlineto{\pgfqpoint{5.700963in}{0.739656in}}%
\pgfpathlineto{\pgfqpoint{5.700667in}{0.739656in}}%
\pgfpathlineto{\pgfqpoint{5.700371in}{0.739656in}}%
\pgfpathlineto{\pgfqpoint{5.700075in}{0.739656in}}%
\pgfpathlineto{\pgfqpoint{5.699779in}{0.739656in}}%
\pgfpathlineto{\pgfqpoint{5.699483in}{0.739656in}}%
\pgfpathlineto{\pgfqpoint{5.699187in}{0.739656in}}%
\pgfpathlineto{\pgfqpoint{5.698891in}{0.739656in}}%
\pgfpathlineto{\pgfqpoint{5.698595in}{0.739656in}}%
\pgfpathlineto{\pgfqpoint{5.698299in}{0.739656in}}%
\pgfpathlineto{\pgfqpoint{5.698003in}{0.739656in}}%
\pgfpathlineto{\pgfqpoint{5.697707in}{0.739656in}}%
\pgfpathlineto{\pgfqpoint{5.697411in}{0.739656in}}%
\pgfpathlineto{\pgfqpoint{5.697115in}{0.739656in}}%
\pgfpathlineto{\pgfqpoint{5.696819in}{0.739656in}}%
\pgfpathlineto{\pgfqpoint{5.696523in}{0.739656in}}%
\pgfpathlineto{\pgfqpoint{5.696227in}{0.739656in}}%
\pgfpathlineto{\pgfqpoint{5.695931in}{0.739656in}}%
\pgfpathlineto{\pgfqpoint{5.695635in}{0.739656in}}%
\pgfpathlineto{\pgfqpoint{5.695339in}{0.739656in}}%
\pgfpathlineto{\pgfqpoint{5.695043in}{0.739656in}}%
\pgfpathlineto{\pgfqpoint{5.694747in}{0.739656in}}%
\pgfpathlineto{\pgfqpoint{5.694451in}{0.739656in}}%
\pgfpathlineto{\pgfqpoint{5.694155in}{0.739656in}}%
\pgfpathlineto{\pgfqpoint{5.693859in}{0.739656in}}%
\pgfpathlineto{\pgfqpoint{5.693563in}{0.739656in}}%
\pgfpathlineto{\pgfqpoint{5.693267in}{0.739656in}}%
\pgfpathlineto{\pgfqpoint{5.692971in}{0.739656in}}%
\pgfpathlineto{\pgfqpoint{5.692675in}{0.739656in}}%
\pgfpathlineto{\pgfqpoint{5.692379in}{0.739656in}}%
\pgfpathlineto{\pgfqpoint{5.692083in}{0.739656in}}%
\pgfpathlineto{\pgfqpoint{5.691787in}{0.739656in}}%
\pgfpathlineto{\pgfqpoint{5.691491in}{0.739656in}}%
\pgfpathlineto{\pgfqpoint{5.691195in}{0.739656in}}%
\pgfpathlineto{\pgfqpoint{5.690899in}{0.739656in}}%
\pgfpathlineto{\pgfqpoint{5.690603in}{0.739656in}}%
\pgfpathlineto{\pgfqpoint{5.690307in}{0.739656in}}%
\pgfpathlineto{\pgfqpoint{5.690011in}{0.739656in}}%
\pgfpathlineto{\pgfqpoint{5.689715in}{0.739656in}}%
\pgfpathlineto{\pgfqpoint{5.689419in}{0.739656in}}%
\pgfpathlineto{\pgfqpoint{5.689123in}{0.739656in}}%
\pgfpathlineto{\pgfqpoint{5.688827in}{0.739656in}}%
\pgfpathlineto{\pgfqpoint{5.688531in}{0.739656in}}%
\pgfpathlineto{\pgfqpoint{5.688235in}{0.739656in}}%
\pgfpathlineto{\pgfqpoint{5.687939in}{0.739656in}}%
\pgfpathlineto{\pgfqpoint{5.687643in}{0.739656in}}%
\pgfpathlineto{\pgfqpoint{5.687347in}{0.739656in}}%
\pgfpathlineto{\pgfqpoint{5.687051in}{0.739656in}}%
\pgfpathlineto{\pgfqpoint{5.686755in}{0.739656in}}%
\pgfpathlineto{\pgfqpoint{5.686459in}{0.739656in}}%
\pgfpathlineto{\pgfqpoint{5.686163in}{0.739656in}}%
\pgfpathlineto{\pgfqpoint{5.685867in}{0.739656in}}%
\pgfpathlineto{\pgfqpoint{5.685571in}{0.739656in}}%
\pgfpathlineto{\pgfqpoint{5.685275in}{0.739656in}}%
\pgfpathlineto{\pgfqpoint{5.684979in}{0.739656in}}%
\pgfpathlineto{\pgfqpoint{5.684683in}{0.739656in}}%
\pgfpathlineto{\pgfqpoint{5.684387in}{0.739656in}}%
\pgfpathlineto{\pgfqpoint{5.684091in}{0.739656in}}%
\pgfpathlineto{\pgfqpoint{5.683795in}{0.739656in}}%
\pgfpathlineto{\pgfqpoint{5.683499in}{0.739656in}}%
\pgfpathlineto{\pgfqpoint{5.683203in}{0.739656in}}%
\pgfpathlineto{\pgfqpoint{5.682907in}{0.739656in}}%
\pgfpathlineto{\pgfqpoint{5.682611in}{0.739656in}}%
\pgfpathlineto{\pgfqpoint{5.682315in}{0.739656in}}%
\pgfpathlineto{\pgfqpoint{5.682019in}{0.739656in}}%
\pgfpathlineto{\pgfqpoint{5.681723in}{0.739656in}}%
\pgfpathlineto{\pgfqpoint{5.681427in}{0.739656in}}%
\pgfpathlineto{\pgfqpoint{5.681131in}{0.739656in}}%
\pgfpathlineto{\pgfqpoint{5.680835in}{0.739656in}}%
\pgfpathlineto{\pgfqpoint{5.680539in}{0.739656in}}%
\pgfpathlineto{\pgfqpoint{5.680243in}{0.739656in}}%
\pgfpathlineto{\pgfqpoint{5.679947in}{0.739656in}}%
\pgfpathlineto{\pgfqpoint{5.679651in}{0.739656in}}%
\pgfpathlineto{\pgfqpoint{5.679355in}{0.739656in}}%
\pgfpathlineto{\pgfqpoint{5.679059in}{0.739656in}}%
\pgfpathlineto{\pgfqpoint{5.678763in}{0.739656in}}%
\pgfpathlineto{\pgfqpoint{5.678467in}{0.739656in}}%
\pgfpathlineto{\pgfqpoint{5.678171in}{0.739656in}}%
\pgfpathlineto{\pgfqpoint{5.677875in}{0.739656in}}%
\pgfpathlineto{\pgfqpoint{5.677579in}{0.739656in}}%
\pgfpathlineto{\pgfqpoint{5.677283in}{0.739656in}}%
\pgfpathlineto{\pgfqpoint{5.676987in}{0.739656in}}%
\pgfpathlineto{\pgfqpoint{5.676691in}{0.739656in}}%
\pgfpathlineto{\pgfqpoint{5.676395in}{0.739656in}}%
\pgfpathlineto{\pgfqpoint{5.676099in}{0.739656in}}%
\pgfpathlineto{\pgfqpoint{5.675803in}{0.739656in}}%
\pgfpathlineto{\pgfqpoint{5.675507in}{0.739656in}}%
\pgfpathlineto{\pgfqpoint{5.675211in}{0.739656in}}%
\pgfpathlineto{\pgfqpoint{5.674915in}{0.739656in}}%
\pgfpathlineto{\pgfqpoint{5.674619in}{0.739656in}}%
\pgfpathlineto{\pgfqpoint{5.674323in}{0.739656in}}%
\pgfpathlineto{\pgfqpoint{5.674027in}{0.739656in}}%
\pgfpathlineto{\pgfqpoint{5.673731in}{0.739656in}}%
\pgfpathlineto{\pgfqpoint{5.673435in}{0.739656in}}%
\pgfpathlineto{\pgfqpoint{5.673139in}{0.739656in}}%
\pgfpathlineto{\pgfqpoint{5.672843in}{0.739656in}}%
\pgfpathlineto{\pgfqpoint{5.672547in}{0.739656in}}%
\pgfpathlineto{\pgfqpoint{5.672251in}{0.739656in}}%
\pgfpathlineto{\pgfqpoint{5.671955in}{0.739656in}}%
\pgfpathlineto{\pgfqpoint{5.671659in}{0.739656in}}%
\pgfpathlineto{\pgfqpoint{5.671363in}{0.739656in}}%
\pgfpathlineto{\pgfqpoint{5.671067in}{0.739656in}}%
\pgfpathlineto{\pgfqpoint{5.670771in}{0.739656in}}%
\pgfpathlineto{\pgfqpoint{5.670475in}{0.739656in}}%
\pgfpathlineto{\pgfqpoint{5.670179in}{0.739656in}}%
\pgfpathlineto{\pgfqpoint{5.669883in}{0.739656in}}%
\pgfpathlineto{\pgfqpoint{5.669587in}{0.739656in}}%
\pgfpathlineto{\pgfqpoint{5.669291in}{0.739656in}}%
\pgfpathlineto{\pgfqpoint{5.668995in}{0.739656in}}%
\pgfpathlineto{\pgfqpoint{5.668699in}{0.739656in}}%
\pgfpathlineto{\pgfqpoint{5.668403in}{0.739656in}}%
\pgfpathlineto{\pgfqpoint{5.668107in}{0.739656in}}%
\pgfpathlineto{\pgfqpoint{5.667811in}{0.739656in}}%
\pgfpathlineto{\pgfqpoint{5.667515in}{0.739656in}}%
\pgfpathlineto{\pgfqpoint{5.667219in}{0.739656in}}%
\pgfpathlineto{\pgfqpoint{5.666923in}{0.739656in}}%
\pgfpathlineto{\pgfqpoint{5.666627in}{0.739656in}}%
\pgfpathlineto{\pgfqpoint{5.666331in}{0.739656in}}%
\pgfpathlineto{\pgfqpoint{5.666035in}{0.739656in}}%
\pgfpathlineto{\pgfqpoint{5.665739in}{0.739656in}}%
\pgfpathlineto{\pgfqpoint{5.665443in}{0.739656in}}%
\pgfpathlineto{\pgfqpoint{5.665147in}{0.739656in}}%
\pgfpathlineto{\pgfqpoint{5.664851in}{0.739656in}}%
\pgfpathlineto{\pgfqpoint{5.664555in}{0.739656in}}%
\pgfpathlineto{\pgfqpoint{5.664259in}{0.739656in}}%
\pgfpathlineto{\pgfqpoint{5.663963in}{0.739656in}}%
\pgfpathlineto{\pgfqpoint{5.663667in}{0.739656in}}%
\pgfpathlineto{\pgfqpoint{5.663371in}{0.739656in}}%
\pgfpathlineto{\pgfqpoint{5.663075in}{0.739656in}}%
\pgfpathlineto{\pgfqpoint{5.662779in}{0.739656in}}%
\pgfpathlineto{\pgfqpoint{5.662483in}{0.739656in}}%
\pgfpathlineto{\pgfqpoint{5.662187in}{0.739656in}}%
\pgfpathlineto{\pgfqpoint{5.661891in}{0.739656in}}%
\pgfpathlineto{\pgfqpoint{5.661595in}{0.739656in}}%
\pgfpathlineto{\pgfqpoint{5.661299in}{0.739656in}}%
\pgfpathlineto{\pgfqpoint{5.661003in}{0.739656in}}%
\pgfpathlineto{\pgfqpoint{5.660707in}{0.739656in}}%
\pgfpathlineto{\pgfqpoint{5.660411in}{0.739656in}}%
\pgfpathlineto{\pgfqpoint{5.660115in}{0.739656in}}%
\pgfpathlineto{\pgfqpoint{5.659819in}{0.739656in}}%
\pgfpathlineto{\pgfqpoint{5.659523in}{0.739656in}}%
\pgfpathlineto{\pgfqpoint{5.659227in}{0.739656in}}%
\pgfpathlineto{\pgfqpoint{5.658931in}{0.739656in}}%
\pgfpathlineto{\pgfqpoint{5.658635in}{0.739656in}}%
\pgfpathlineto{\pgfqpoint{5.658339in}{0.739656in}}%
\pgfpathlineto{\pgfqpoint{5.658043in}{0.739656in}}%
\pgfpathlineto{\pgfqpoint{5.657747in}{0.739656in}}%
\pgfpathlineto{\pgfqpoint{5.657451in}{0.739656in}}%
\pgfpathlineto{\pgfqpoint{5.657155in}{0.739656in}}%
\pgfpathlineto{\pgfqpoint{5.656859in}{0.739656in}}%
\pgfpathlineto{\pgfqpoint{5.656563in}{0.739656in}}%
\pgfpathlineto{\pgfqpoint{5.656267in}{0.739656in}}%
\pgfpathlineto{\pgfqpoint{5.655971in}{0.739656in}}%
\pgfpathlineto{\pgfqpoint{5.655675in}{0.739656in}}%
\pgfpathlineto{\pgfqpoint{5.655379in}{0.739656in}}%
\pgfpathlineto{\pgfqpoint{5.655083in}{0.739656in}}%
\pgfpathlineto{\pgfqpoint{5.654787in}{0.739656in}}%
\pgfpathlineto{\pgfqpoint{5.654491in}{0.739656in}}%
\pgfpathlineto{\pgfqpoint{5.654195in}{0.739656in}}%
\pgfpathlineto{\pgfqpoint{5.653899in}{0.739656in}}%
\pgfpathlineto{\pgfqpoint{5.653603in}{0.739656in}}%
\pgfpathlineto{\pgfqpoint{5.653307in}{0.739656in}}%
\pgfpathlineto{\pgfqpoint{5.653011in}{0.739656in}}%
\pgfpathlineto{\pgfqpoint{5.652715in}{0.739656in}}%
\pgfpathlineto{\pgfqpoint{5.652419in}{0.739656in}}%
\pgfpathlineto{\pgfqpoint{5.652123in}{0.739656in}}%
\pgfpathlineto{\pgfqpoint{5.651827in}{0.739656in}}%
\pgfpathlineto{\pgfqpoint{5.651531in}{0.739656in}}%
\pgfpathlineto{\pgfqpoint{5.651235in}{0.739656in}}%
\pgfpathlineto{\pgfqpoint{5.650939in}{0.739656in}}%
\pgfpathlineto{\pgfqpoint{5.650643in}{0.739656in}}%
\pgfpathlineto{\pgfqpoint{5.650347in}{0.739656in}}%
\pgfpathlineto{\pgfqpoint{5.650051in}{0.739656in}}%
\pgfpathlineto{\pgfqpoint{5.649755in}{0.739656in}}%
\pgfpathlineto{\pgfqpoint{5.649459in}{0.739656in}}%
\pgfpathlineto{\pgfqpoint{5.649163in}{0.739656in}}%
\pgfpathlineto{\pgfqpoint{5.648867in}{0.739656in}}%
\pgfpathlineto{\pgfqpoint{5.648571in}{0.739656in}}%
\pgfpathlineto{\pgfqpoint{5.648275in}{0.739656in}}%
\pgfpathlineto{\pgfqpoint{5.647979in}{0.739656in}}%
\pgfpathlineto{\pgfqpoint{5.647683in}{0.739656in}}%
\pgfpathlineto{\pgfqpoint{5.647387in}{0.739656in}}%
\pgfpathlineto{\pgfqpoint{5.647091in}{0.739656in}}%
\pgfpathlineto{\pgfqpoint{5.646795in}{0.739656in}}%
\pgfpathlineto{\pgfqpoint{5.646499in}{0.739656in}}%
\pgfpathlineto{\pgfqpoint{5.646203in}{0.739656in}}%
\pgfpathlineto{\pgfqpoint{5.645907in}{0.739656in}}%
\pgfpathlineto{\pgfqpoint{5.645611in}{0.739656in}}%
\pgfpathlineto{\pgfqpoint{5.645315in}{0.739656in}}%
\pgfpathlineto{\pgfqpoint{5.645019in}{0.739656in}}%
\pgfpathlineto{\pgfqpoint{5.644723in}{0.739656in}}%
\pgfpathlineto{\pgfqpoint{5.644427in}{0.739656in}}%
\pgfpathlineto{\pgfqpoint{5.644131in}{0.739656in}}%
\pgfpathlineto{\pgfqpoint{5.643835in}{0.739656in}}%
\pgfpathlineto{\pgfqpoint{5.643539in}{0.739656in}}%
\pgfpathlineto{\pgfqpoint{5.643243in}{0.739656in}}%
\pgfpathlineto{\pgfqpoint{5.642947in}{0.739656in}}%
\pgfpathlineto{\pgfqpoint{5.642651in}{0.739656in}}%
\pgfpathlineto{\pgfqpoint{5.642355in}{0.739656in}}%
\pgfpathlineto{\pgfqpoint{5.642059in}{0.739656in}}%
\pgfpathlineto{\pgfqpoint{5.641763in}{0.739656in}}%
\pgfpathlineto{\pgfqpoint{5.641467in}{0.739656in}}%
\pgfpathlineto{\pgfqpoint{5.641171in}{0.739656in}}%
\pgfpathlineto{\pgfqpoint{5.640875in}{0.739656in}}%
\pgfpathlineto{\pgfqpoint{5.640579in}{0.739656in}}%
\pgfpathlineto{\pgfqpoint{5.640283in}{0.739656in}}%
\pgfpathlineto{\pgfqpoint{5.639987in}{0.739656in}}%
\pgfpathlineto{\pgfqpoint{5.639691in}{0.739656in}}%
\pgfpathlineto{\pgfqpoint{5.639395in}{0.739656in}}%
\pgfpathlineto{\pgfqpoint{5.639099in}{0.739656in}}%
\pgfpathlineto{\pgfqpoint{5.638803in}{0.739656in}}%
\pgfpathlineto{\pgfqpoint{5.638507in}{0.739656in}}%
\pgfpathlineto{\pgfqpoint{5.638211in}{0.739656in}}%
\pgfpathlineto{\pgfqpoint{5.637915in}{0.739656in}}%
\pgfpathlineto{\pgfqpoint{5.637619in}{0.739656in}}%
\pgfpathlineto{\pgfqpoint{5.637323in}{0.739656in}}%
\pgfpathlineto{\pgfqpoint{5.637027in}{0.739656in}}%
\pgfpathlineto{\pgfqpoint{5.636731in}{0.739656in}}%
\pgfpathlineto{\pgfqpoint{5.636434in}{0.739656in}}%
\pgfpathlineto{\pgfqpoint{5.636138in}{0.739656in}}%
\pgfpathlineto{\pgfqpoint{5.635842in}{0.739656in}}%
\pgfpathlineto{\pgfqpoint{5.635546in}{0.739656in}}%
\pgfpathlineto{\pgfqpoint{5.635250in}{0.739656in}}%
\pgfpathlineto{\pgfqpoint{5.634954in}{0.739656in}}%
\pgfpathlineto{\pgfqpoint{5.634658in}{0.739656in}}%
\pgfpathlineto{\pgfqpoint{5.634362in}{0.739656in}}%
\pgfpathlineto{\pgfqpoint{5.634066in}{0.739656in}}%
\pgfpathlineto{\pgfqpoint{5.633770in}{0.739656in}}%
\pgfpathlineto{\pgfqpoint{5.633474in}{0.739656in}}%
\pgfpathlineto{\pgfqpoint{5.633178in}{0.739656in}}%
\pgfpathlineto{\pgfqpoint{5.632882in}{0.739656in}}%
\pgfpathlineto{\pgfqpoint{5.632586in}{0.739656in}}%
\pgfpathlineto{\pgfqpoint{5.632290in}{0.739656in}}%
\pgfpathlineto{\pgfqpoint{5.631994in}{0.739656in}}%
\pgfpathlineto{\pgfqpoint{5.631698in}{0.739656in}}%
\pgfpathlineto{\pgfqpoint{5.631402in}{0.739656in}}%
\pgfpathlineto{\pgfqpoint{5.631106in}{0.739656in}}%
\pgfpathlineto{\pgfqpoint{5.630810in}{0.739656in}}%
\pgfpathlineto{\pgfqpoint{5.630514in}{0.739656in}}%
\pgfpathlineto{\pgfqpoint{5.630218in}{0.739656in}}%
\pgfpathlineto{\pgfqpoint{5.629922in}{0.739656in}}%
\pgfpathlineto{\pgfqpoint{5.629626in}{0.739656in}}%
\pgfpathlineto{\pgfqpoint{5.629330in}{0.739656in}}%
\pgfpathlineto{\pgfqpoint{5.629034in}{0.739656in}}%
\pgfpathlineto{\pgfqpoint{5.628738in}{0.739656in}}%
\pgfpathlineto{\pgfqpoint{5.628442in}{0.739656in}}%
\pgfpathlineto{\pgfqpoint{5.628146in}{0.739656in}}%
\pgfpathlineto{\pgfqpoint{5.627850in}{0.739656in}}%
\pgfpathlineto{\pgfqpoint{5.627554in}{0.739656in}}%
\pgfpathlineto{\pgfqpoint{5.627258in}{0.739656in}}%
\pgfpathlineto{\pgfqpoint{5.626962in}{0.739656in}}%
\pgfpathlineto{\pgfqpoint{5.626666in}{0.739656in}}%
\pgfpathlineto{\pgfqpoint{5.626370in}{0.739656in}}%
\pgfpathlineto{\pgfqpoint{5.626074in}{0.739656in}}%
\pgfpathlineto{\pgfqpoint{5.625778in}{0.739656in}}%
\pgfpathlineto{\pgfqpoint{5.625482in}{0.739656in}}%
\pgfpathlineto{\pgfqpoint{5.625186in}{0.739656in}}%
\pgfpathlineto{\pgfqpoint{5.624890in}{0.739656in}}%
\pgfpathlineto{\pgfqpoint{5.624594in}{0.739656in}}%
\pgfpathlineto{\pgfqpoint{5.624298in}{0.739656in}}%
\pgfpathlineto{\pgfqpoint{5.624002in}{0.739656in}}%
\pgfpathlineto{\pgfqpoint{5.623706in}{0.739656in}}%
\pgfpathlineto{\pgfqpoint{5.623410in}{0.739656in}}%
\pgfpathlineto{\pgfqpoint{5.623114in}{0.739656in}}%
\pgfpathlineto{\pgfqpoint{5.622818in}{0.739656in}}%
\pgfpathlineto{\pgfqpoint{5.622522in}{0.739656in}}%
\pgfpathlineto{\pgfqpoint{5.622226in}{0.739656in}}%
\pgfpathlineto{\pgfqpoint{5.621930in}{0.739656in}}%
\pgfpathlineto{\pgfqpoint{5.621634in}{0.739656in}}%
\pgfpathlineto{\pgfqpoint{5.621338in}{0.739656in}}%
\pgfpathlineto{\pgfqpoint{5.621042in}{0.739656in}}%
\pgfpathlineto{\pgfqpoint{5.620746in}{0.739656in}}%
\pgfpathlineto{\pgfqpoint{5.620450in}{0.739656in}}%
\pgfpathlineto{\pgfqpoint{5.620154in}{0.739656in}}%
\pgfpathlineto{\pgfqpoint{5.619858in}{0.739656in}}%
\pgfpathlineto{\pgfqpoint{5.619562in}{0.739656in}}%
\pgfpathlineto{\pgfqpoint{5.619266in}{0.739656in}}%
\pgfpathlineto{\pgfqpoint{5.618970in}{0.739656in}}%
\pgfpathlineto{\pgfqpoint{5.618674in}{0.739656in}}%
\pgfpathlineto{\pgfqpoint{5.618378in}{0.739656in}}%
\pgfpathlineto{\pgfqpoint{5.618082in}{0.739656in}}%
\pgfpathlineto{\pgfqpoint{5.617786in}{0.739656in}}%
\pgfpathlineto{\pgfqpoint{5.617490in}{0.739656in}}%
\pgfpathlineto{\pgfqpoint{5.617194in}{0.739656in}}%
\pgfpathlineto{\pgfqpoint{5.616898in}{0.739656in}}%
\pgfpathlineto{\pgfqpoint{5.616602in}{0.739656in}}%
\pgfpathlineto{\pgfqpoint{5.616306in}{0.739656in}}%
\pgfpathlineto{\pgfqpoint{5.616010in}{0.739656in}}%
\pgfpathlineto{\pgfqpoint{5.615714in}{0.739656in}}%
\pgfpathlineto{\pgfqpoint{5.615418in}{0.739656in}}%
\pgfpathlineto{\pgfqpoint{5.615122in}{0.739656in}}%
\pgfpathlineto{\pgfqpoint{5.614826in}{0.739656in}}%
\pgfpathlineto{\pgfqpoint{5.614530in}{0.739656in}}%
\pgfpathlineto{\pgfqpoint{5.614234in}{0.739656in}}%
\pgfpathlineto{\pgfqpoint{5.613938in}{0.739656in}}%
\pgfpathlineto{\pgfqpoint{5.613642in}{0.739656in}}%
\pgfpathlineto{\pgfqpoint{5.613346in}{0.739656in}}%
\pgfpathlineto{\pgfqpoint{5.613050in}{0.739656in}}%
\pgfpathlineto{\pgfqpoint{5.612754in}{0.739656in}}%
\pgfpathlineto{\pgfqpoint{5.612458in}{0.739656in}}%
\pgfpathlineto{\pgfqpoint{5.612162in}{0.739656in}}%
\pgfpathlineto{\pgfqpoint{5.611866in}{0.739656in}}%
\pgfpathlineto{\pgfqpoint{5.611570in}{0.739656in}}%
\pgfpathlineto{\pgfqpoint{5.611274in}{0.739656in}}%
\pgfpathlineto{\pgfqpoint{5.610978in}{0.739656in}}%
\pgfpathlineto{\pgfqpoint{5.610682in}{0.739656in}}%
\pgfpathlineto{\pgfqpoint{5.610386in}{0.739656in}}%
\pgfpathlineto{\pgfqpoint{5.610090in}{0.739656in}}%
\pgfpathlineto{\pgfqpoint{5.609794in}{0.739656in}}%
\pgfpathlineto{\pgfqpoint{5.609498in}{0.739656in}}%
\pgfpathlineto{\pgfqpoint{5.609202in}{0.739656in}}%
\pgfpathlineto{\pgfqpoint{5.608906in}{0.739656in}}%
\pgfpathlineto{\pgfqpoint{5.608610in}{0.739656in}}%
\pgfpathlineto{\pgfqpoint{5.608314in}{0.739656in}}%
\pgfpathlineto{\pgfqpoint{5.608018in}{0.739656in}}%
\pgfpathlineto{\pgfqpoint{5.607722in}{0.739656in}}%
\pgfpathlineto{\pgfqpoint{5.607426in}{0.739656in}}%
\pgfpathlineto{\pgfqpoint{5.607130in}{0.739656in}}%
\pgfpathlineto{\pgfqpoint{5.606834in}{0.739656in}}%
\pgfpathlineto{\pgfqpoint{5.606538in}{0.739656in}}%
\pgfpathlineto{\pgfqpoint{5.606242in}{0.739656in}}%
\pgfpathlineto{\pgfqpoint{5.605946in}{0.739656in}}%
\pgfpathlineto{\pgfqpoint{5.605650in}{0.739656in}}%
\pgfpathlineto{\pgfqpoint{5.605354in}{0.739656in}}%
\pgfpathlineto{\pgfqpoint{5.605058in}{0.739656in}}%
\pgfpathlineto{\pgfqpoint{5.604762in}{0.739656in}}%
\pgfpathlineto{\pgfqpoint{5.604466in}{0.739656in}}%
\pgfpathlineto{\pgfqpoint{5.604170in}{0.739656in}}%
\pgfpathlineto{\pgfqpoint{5.603874in}{0.739656in}}%
\pgfpathlineto{\pgfqpoint{5.603578in}{0.739656in}}%
\pgfpathlineto{\pgfqpoint{5.603282in}{0.739656in}}%
\pgfpathlineto{\pgfqpoint{5.602986in}{0.739656in}}%
\pgfpathlineto{\pgfqpoint{5.602690in}{0.739656in}}%
\pgfpathlineto{\pgfqpoint{5.602394in}{0.739656in}}%
\pgfpathlineto{\pgfqpoint{5.602098in}{0.739656in}}%
\pgfpathlineto{\pgfqpoint{5.601802in}{0.739656in}}%
\pgfpathlineto{\pgfqpoint{5.601506in}{0.739656in}}%
\pgfpathlineto{\pgfqpoint{5.601210in}{0.739656in}}%
\pgfpathlineto{\pgfqpoint{5.600914in}{0.739656in}}%
\pgfpathlineto{\pgfqpoint{5.600618in}{0.739656in}}%
\pgfpathlineto{\pgfqpoint{5.600322in}{0.739656in}}%
\pgfpathlineto{\pgfqpoint{5.600026in}{0.739656in}}%
\pgfpathlineto{\pgfqpoint{5.599730in}{0.739656in}}%
\pgfpathlineto{\pgfqpoint{5.599434in}{0.739656in}}%
\pgfpathlineto{\pgfqpoint{5.599138in}{0.739656in}}%
\pgfpathlineto{\pgfqpoint{5.598842in}{0.739656in}}%
\pgfpathlineto{\pgfqpoint{5.598546in}{0.739656in}}%
\pgfpathlineto{\pgfqpoint{5.598250in}{0.739656in}}%
\pgfpathlineto{\pgfqpoint{5.597954in}{0.739656in}}%
\pgfpathlineto{\pgfqpoint{5.597658in}{0.739656in}}%
\pgfpathlineto{\pgfqpoint{5.597362in}{0.739656in}}%
\pgfpathlineto{\pgfqpoint{5.597066in}{0.739656in}}%
\pgfpathlineto{\pgfqpoint{5.596770in}{0.739656in}}%
\pgfpathlineto{\pgfqpoint{5.596474in}{0.739656in}}%
\pgfpathlineto{\pgfqpoint{5.596178in}{0.739656in}}%
\pgfpathlineto{\pgfqpoint{5.595882in}{0.739656in}}%
\pgfpathlineto{\pgfqpoint{5.595586in}{0.739656in}}%
\pgfpathlineto{\pgfqpoint{5.595290in}{0.739656in}}%
\pgfpathlineto{\pgfqpoint{5.594994in}{0.739656in}}%
\pgfpathlineto{\pgfqpoint{5.594698in}{0.739656in}}%
\pgfpathlineto{\pgfqpoint{5.594402in}{0.739656in}}%
\pgfpathlineto{\pgfqpoint{5.594106in}{0.739656in}}%
\pgfpathlineto{\pgfqpoint{5.593810in}{0.739656in}}%
\pgfpathlineto{\pgfqpoint{5.593514in}{0.739656in}}%
\pgfpathlineto{\pgfqpoint{5.593218in}{0.739656in}}%
\pgfpathlineto{\pgfqpoint{5.592922in}{0.739656in}}%
\pgfpathlineto{\pgfqpoint{5.592626in}{0.739656in}}%
\pgfpathlineto{\pgfqpoint{5.592330in}{0.739656in}}%
\pgfpathlineto{\pgfqpoint{5.592034in}{0.739656in}}%
\pgfpathlineto{\pgfqpoint{5.591738in}{0.739656in}}%
\pgfpathlineto{\pgfqpoint{5.591442in}{0.739656in}}%
\pgfpathlineto{\pgfqpoint{5.591146in}{0.739656in}}%
\pgfpathlineto{\pgfqpoint{5.590850in}{0.739656in}}%
\pgfpathlineto{\pgfqpoint{5.590554in}{0.739656in}}%
\pgfpathlineto{\pgfqpoint{5.590258in}{0.739656in}}%
\pgfpathlineto{\pgfqpoint{5.589962in}{0.739656in}}%
\pgfpathlineto{\pgfqpoint{5.589666in}{0.739656in}}%
\pgfpathlineto{\pgfqpoint{5.589370in}{0.739656in}}%
\pgfpathlineto{\pgfqpoint{5.589074in}{0.739656in}}%
\pgfpathlineto{\pgfqpoint{5.588778in}{0.739656in}}%
\pgfpathlineto{\pgfqpoint{5.588482in}{0.739656in}}%
\pgfpathlineto{\pgfqpoint{5.588186in}{0.739656in}}%
\pgfpathlineto{\pgfqpoint{5.587890in}{0.739656in}}%
\pgfpathlineto{\pgfqpoint{5.587594in}{0.739656in}}%
\pgfpathlineto{\pgfqpoint{5.587298in}{0.739656in}}%
\pgfpathlineto{\pgfqpoint{5.587002in}{0.739656in}}%
\pgfpathlineto{\pgfqpoint{5.586706in}{0.739656in}}%
\pgfpathlineto{\pgfqpoint{5.586410in}{0.739656in}}%
\pgfpathlineto{\pgfqpoint{5.586114in}{0.739656in}}%
\pgfpathlineto{\pgfqpoint{5.585818in}{0.739656in}}%
\pgfpathlineto{\pgfqpoint{5.585522in}{0.739656in}}%
\pgfpathlineto{\pgfqpoint{5.585226in}{0.739656in}}%
\pgfpathlineto{\pgfqpoint{5.584930in}{0.739656in}}%
\pgfpathlineto{\pgfqpoint{5.584634in}{0.739656in}}%
\pgfpathlineto{\pgfqpoint{5.584338in}{0.739656in}}%
\pgfpathlineto{\pgfqpoint{5.584042in}{0.739656in}}%
\pgfpathlineto{\pgfqpoint{5.583746in}{0.739656in}}%
\pgfpathlineto{\pgfqpoint{5.583450in}{0.739656in}}%
\pgfpathlineto{\pgfqpoint{5.583154in}{0.739656in}}%
\pgfpathlineto{\pgfqpoint{5.582858in}{0.739656in}}%
\pgfpathlineto{\pgfqpoint{5.582562in}{0.739656in}}%
\pgfpathlineto{\pgfqpoint{5.582266in}{0.739656in}}%
\pgfpathlineto{\pgfqpoint{5.581970in}{0.739656in}}%
\pgfpathlineto{\pgfqpoint{5.581674in}{0.739656in}}%
\pgfpathlineto{\pgfqpoint{5.581378in}{0.739656in}}%
\pgfpathlineto{\pgfqpoint{5.581082in}{0.739656in}}%
\pgfpathlineto{\pgfqpoint{5.580786in}{0.739656in}}%
\pgfpathlineto{\pgfqpoint{5.580490in}{0.739656in}}%
\pgfpathlineto{\pgfqpoint{5.580194in}{0.739656in}}%
\pgfpathlineto{\pgfqpoint{5.579898in}{0.739656in}}%
\pgfpathlineto{\pgfqpoint{5.579602in}{0.739656in}}%
\pgfpathlineto{\pgfqpoint{5.579306in}{0.739656in}}%
\pgfpathlineto{\pgfqpoint{5.579010in}{0.739656in}}%
\pgfpathlineto{\pgfqpoint{5.578714in}{0.739656in}}%
\pgfpathlineto{\pgfqpoint{5.578418in}{0.739656in}}%
\pgfpathlineto{\pgfqpoint{5.578122in}{0.739656in}}%
\pgfpathlineto{\pgfqpoint{5.577826in}{0.739656in}}%
\pgfpathlineto{\pgfqpoint{5.577530in}{0.739656in}}%
\pgfpathlineto{\pgfqpoint{5.577234in}{0.739656in}}%
\pgfpathlineto{\pgfqpoint{5.576938in}{0.739656in}}%
\pgfpathlineto{\pgfqpoint{5.576642in}{0.739656in}}%
\pgfpathlineto{\pgfqpoint{5.576346in}{0.739656in}}%
\pgfpathlineto{\pgfqpoint{5.576050in}{0.739656in}}%
\pgfpathlineto{\pgfqpoint{5.575754in}{0.739656in}}%
\pgfpathlineto{\pgfqpoint{5.575458in}{0.739656in}}%
\pgfpathlineto{\pgfqpoint{5.575162in}{0.739656in}}%
\pgfpathlineto{\pgfqpoint{5.574866in}{0.739656in}}%
\pgfpathlineto{\pgfqpoint{5.574570in}{0.739656in}}%
\pgfpathlineto{\pgfqpoint{5.574274in}{0.739656in}}%
\pgfpathlineto{\pgfqpoint{5.573978in}{0.739656in}}%
\pgfpathlineto{\pgfqpoint{5.573682in}{0.739656in}}%
\pgfpathlineto{\pgfqpoint{5.573386in}{0.739656in}}%
\pgfpathlineto{\pgfqpoint{5.573090in}{0.739656in}}%
\pgfpathlineto{\pgfqpoint{5.572794in}{0.739656in}}%
\pgfpathlineto{\pgfqpoint{5.572498in}{0.739656in}}%
\pgfpathlineto{\pgfqpoint{5.572202in}{0.739656in}}%
\pgfpathlineto{\pgfqpoint{5.571906in}{0.739656in}}%
\pgfpathlineto{\pgfqpoint{5.571610in}{0.739656in}}%
\pgfpathlineto{\pgfqpoint{5.571314in}{0.739656in}}%
\pgfpathlineto{\pgfqpoint{5.571018in}{0.739656in}}%
\pgfpathlineto{\pgfqpoint{5.570722in}{0.739656in}}%
\pgfpathlineto{\pgfqpoint{5.570426in}{0.739656in}}%
\pgfpathlineto{\pgfqpoint{5.570130in}{0.739656in}}%
\pgfpathlineto{\pgfqpoint{5.569834in}{0.739656in}}%
\pgfpathlineto{\pgfqpoint{5.569538in}{0.739656in}}%
\pgfpathlineto{\pgfqpoint{5.569242in}{0.739656in}}%
\pgfpathlineto{\pgfqpoint{5.568945in}{0.739656in}}%
\pgfpathlineto{\pgfqpoint{5.568649in}{0.739656in}}%
\pgfpathlineto{\pgfqpoint{5.568353in}{0.739656in}}%
\pgfpathlineto{\pgfqpoint{5.568057in}{0.739656in}}%
\pgfpathlineto{\pgfqpoint{5.567761in}{0.739656in}}%
\pgfpathlineto{\pgfqpoint{5.567465in}{0.739656in}}%
\pgfpathlineto{\pgfqpoint{5.567169in}{0.739656in}}%
\pgfpathlineto{\pgfqpoint{5.566873in}{0.739656in}}%
\pgfpathlineto{\pgfqpoint{5.566577in}{0.739656in}}%
\pgfpathlineto{\pgfqpoint{5.566281in}{0.739656in}}%
\pgfpathlineto{\pgfqpoint{5.565985in}{0.739656in}}%
\pgfpathlineto{\pgfqpoint{5.565689in}{0.739656in}}%
\pgfpathlineto{\pgfqpoint{5.565393in}{0.739656in}}%
\pgfpathlineto{\pgfqpoint{5.565097in}{0.739656in}}%
\pgfpathlineto{\pgfqpoint{5.564801in}{0.739656in}}%
\pgfpathlineto{\pgfqpoint{5.564505in}{0.739656in}}%
\pgfpathlineto{\pgfqpoint{5.564209in}{0.739656in}}%
\pgfpathlineto{\pgfqpoint{5.563913in}{0.739656in}}%
\pgfpathlineto{\pgfqpoint{5.563617in}{0.739656in}}%
\pgfpathlineto{\pgfqpoint{5.563321in}{0.739656in}}%
\pgfpathlineto{\pgfqpoint{5.563025in}{0.739656in}}%
\pgfpathlineto{\pgfqpoint{5.562729in}{0.739656in}}%
\pgfpathlineto{\pgfqpoint{5.562433in}{0.739656in}}%
\pgfpathlineto{\pgfqpoint{5.562137in}{0.739656in}}%
\pgfpathlineto{\pgfqpoint{5.561841in}{0.739656in}}%
\pgfpathlineto{\pgfqpoint{5.561545in}{0.739656in}}%
\pgfpathlineto{\pgfqpoint{5.561249in}{0.739656in}}%
\pgfpathlineto{\pgfqpoint{5.560953in}{0.739656in}}%
\pgfpathlineto{\pgfqpoint{5.560657in}{0.739656in}}%
\pgfpathlineto{\pgfqpoint{5.560361in}{0.739656in}}%
\pgfpathlineto{\pgfqpoint{5.560065in}{0.739656in}}%
\pgfpathlineto{\pgfqpoint{5.559769in}{0.739656in}}%
\pgfpathlineto{\pgfqpoint{5.559473in}{0.739656in}}%
\pgfpathlineto{\pgfqpoint{5.559177in}{0.739656in}}%
\pgfpathlineto{\pgfqpoint{5.558881in}{0.739656in}}%
\pgfpathlineto{\pgfqpoint{5.558585in}{0.739656in}}%
\pgfpathlineto{\pgfqpoint{5.558289in}{0.739656in}}%
\pgfpathlineto{\pgfqpoint{5.557993in}{0.739656in}}%
\pgfpathlineto{\pgfqpoint{5.557697in}{0.739656in}}%
\pgfpathlineto{\pgfqpoint{5.557401in}{0.739656in}}%
\pgfpathlineto{\pgfqpoint{5.557105in}{0.739656in}}%
\pgfpathlineto{\pgfqpoint{5.556809in}{0.739656in}}%
\pgfpathlineto{\pgfqpoint{5.556513in}{0.739656in}}%
\pgfpathlineto{\pgfqpoint{5.556217in}{0.739656in}}%
\pgfpathlineto{\pgfqpoint{5.555921in}{0.739656in}}%
\pgfpathlineto{\pgfqpoint{5.555625in}{0.739656in}}%
\pgfpathlineto{\pgfqpoint{5.555329in}{0.739656in}}%
\pgfpathlineto{\pgfqpoint{5.555033in}{0.739656in}}%
\pgfpathlineto{\pgfqpoint{5.554737in}{0.739656in}}%
\pgfpathlineto{\pgfqpoint{5.554441in}{0.739656in}}%
\pgfpathlineto{\pgfqpoint{5.554145in}{0.739656in}}%
\pgfpathlineto{\pgfqpoint{5.553849in}{0.739656in}}%
\pgfpathlineto{\pgfqpoint{5.553553in}{0.739656in}}%
\pgfpathlineto{\pgfqpoint{5.553257in}{0.739656in}}%
\pgfpathlineto{\pgfqpoint{5.552961in}{0.739656in}}%
\pgfpathlineto{\pgfqpoint{5.552665in}{0.739656in}}%
\pgfpathlineto{\pgfqpoint{5.552369in}{0.739656in}}%
\pgfpathlineto{\pgfqpoint{5.552073in}{0.739656in}}%
\pgfpathlineto{\pgfqpoint{5.551777in}{0.739656in}}%
\pgfpathlineto{\pgfqpoint{5.551481in}{0.739656in}}%
\pgfpathlineto{\pgfqpoint{5.551185in}{0.739656in}}%
\pgfpathlineto{\pgfqpoint{5.550889in}{0.739656in}}%
\pgfpathlineto{\pgfqpoint{5.550593in}{0.739656in}}%
\pgfpathlineto{\pgfqpoint{5.550297in}{0.739656in}}%
\pgfpathlineto{\pgfqpoint{5.550001in}{0.739656in}}%
\pgfpathlineto{\pgfqpoint{5.549705in}{0.739656in}}%
\pgfpathlineto{\pgfqpoint{5.549409in}{0.739656in}}%
\pgfpathlineto{\pgfqpoint{5.549113in}{0.739656in}}%
\pgfpathlineto{\pgfqpoint{5.548817in}{0.739656in}}%
\pgfpathlineto{\pgfqpoint{5.548521in}{0.739656in}}%
\pgfpathlineto{\pgfqpoint{5.548225in}{0.739656in}}%
\pgfpathlineto{\pgfqpoint{5.547929in}{0.739656in}}%
\pgfpathlineto{\pgfqpoint{5.547633in}{0.739656in}}%
\pgfpathlineto{\pgfqpoint{5.547337in}{0.739656in}}%
\pgfpathlineto{\pgfqpoint{5.547041in}{0.739656in}}%
\pgfpathlineto{\pgfqpoint{5.546745in}{0.739656in}}%
\pgfpathlineto{\pgfqpoint{5.546449in}{0.739656in}}%
\pgfpathlineto{\pgfqpoint{5.546153in}{0.739656in}}%
\pgfpathlineto{\pgfqpoint{5.545857in}{0.739656in}}%
\pgfpathlineto{\pgfqpoint{5.545561in}{0.739656in}}%
\pgfpathlineto{\pgfqpoint{5.545265in}{0.739656in}}%
\pgfpathlineto{\pgfqpoint{5.544969in}{0.739656in}}%
\pgfpathlineto{\pgfqpoint{5.544673in}{0.739656in}}%
\pgfpathlineto{\pgfqpoint{5.544377in}{0.739656in}}%
\pgfpathlineto{\pgfqpoint{5.544081in}{0.739656in}}%
\pgfpathlineto{\pgfqpoint{5.543785in}{0.739656in}}%
\pgfpathlineto{\pgfqpoint{5.543489in}{0.739656in}}%
\pgfpathlineto{\pgfqpoint{5.543193in}{0.739656in}}%
\pgfpathlineto{\pgfqpoint{5.542897in}{0.739656in}}%
\pgfpathlineto{\pgfqpoint{5.542601in}{0.739656in}}%
\pgfpathlineto{\pgfqpoint{5.542305in}{0.739656in}}%
\pgfpathlineto{\pgfqpoint{5.542009in}{0.739656in}}%
\pgfpathlineto{\pgfqpoint{5.541713in}{0.739656in}}%
\pgfpathlineto{\pgfqpoint{5.541417in}{0.739656in}}%
\pgfpathlineto{\pgfqpoint{5.541121in}{0.739656in}}%
\pgfpathlineto{\pgfqpoint{5.540825in}{0.739656in}}%
\pgfpathlineto{\pgfqpoint{5.540529in}{0.739656in}}%
\pgfpathlineto{\pgfqpoint{5.540233in}{0.739656in}}%
\pgfpathlineto{\pgfqpoint{5.539937in}{0.739656in}}%
\pgfpathlineto{\pgfqpoint{5.539641in}{0.739656in}}%
\pgfpathlineto{\pgfqpoint{5.539345in}{0.739656in}}%
\pgfpathlineto{\pgfqpoint{5.539049in}{0.739656in}}%
\pgfpathlineto{\pgfqpoint{5.538753in}{0.739656in}}%
\pgfpathlineto{\pgfqpoint{5.538457in}{0.739656in}}%
\pgfpathlineto{\pgfqpoint{5.538161in}{0.739656in}}%
\pgfpathlineto{\pgfqpoint{5.537865in}{0.739656in}}%
\pgfpathlineto{\pgfqpoint{5.537569in}{0.739656in}}%
\pgfpathlineto{\pgfqpoint{5.537273in}{0.739656in}}%
\pgfpathlineto{\pgfqpoint{5.536977in}{0.739656in}}%
\pgfpathlineto{\pgfqpoint{5.536681in}{0.739656in}}%
\pgfpathlineto{\pgfqpoint{5.536385in}{0.739656in}}%
\pgfpathlineto{\pgfqpoint{5.536089in}{0.739656in}}%
\pgfpathlineto{\pgfqpoint{5.535793in}{0.739656in}}%
\pgfpathlineto{\pgfqpoint{5.535497in}{0.739656in}}%
\pgfpathlineto{\pgfqpoint{5.535201in}{0.739656in}}%
\pgfpathlineto{\pgfqpoint{5.534905in}{0.739656in}}%
\pgfpathlineto{\pgfqpoint{5.534609in}{0.739656in}}%
\pgfpathlineto{\pgfqpoint{5.534313in}{0.739656in}}%
\pgfpathlineto{\pgfqpoint{5.534017in}{0.739656in}}%
\pgfpathlineto{\pgfqpoint{5.533721in}{0.739656in}}%
\pgfpathlineto{\pgfqpoint{5.533425in}{0.739656in}}%
\pgfpathlineto{\pgfqpoint{5.533129in}{0.739656in}}%
\pgfpathlineto{\pgfqpoint{5.532833in}{0.739656in}}%
\pgfpathlineto{\pgfqpoint{5.532537in}{0.739656in}}%
\pgfpathlineto{\pgfqpoint{5.532241in}{0.739656in}}%
\pgfpathlineto{\pgfqpoint{5.531945in}{0.739656in}}%
\pgfpathlineto{\pgfqpoint{5.531649in}{0.739656in}}%
\pgfpathlineto{\pgfqpoint{5.531353in}{0.739656in}}%
\pgfpathlineto{\pgfqpoint{5.531057in}{0.739656in}}%
\pgfpathlineto{\pgfqpoint{5.530761in}{0.739656in}}%
\pgfpathlineto{\pgfqpoint{5.530465in}{0.739656in}}%
\pgfpathlineto{\pgfqpoint{5.530169in}{0.739656in}}%
\pgfpathlineto{\pgfqpoint{5.529873in}{0.739656in}}%
\pgfpathlineto{\pgfqpoint{5.529577in}{0.739656in}}%
\pgfpathlineto{\pgfqpoint{5.529281in}{0.739656in}}%
\pgfpathlineto{\pgfqpoint{5.528985in}{0.739656in}}%
\pgfpathlineto{\pgfqpoint{5.528689in}{0.739656in}}%
\pgfpathlineto{\pgfqpoint{5.528393in}{0.739656in}}%
\pgfpathlineto{\pgfqpoint{5.528097in}{0.739656in}}%
\pgfpathlineto{\pgfqpoint{5.527801in}{0.739656in}}%
\pgfpathlineto{\pgfqpoint{5.527505in}{0.739656in}}%
\pgfpathlineto{\pgfqpoint{5.527209in}{0.739656in}}%
\pgfpathlineto{\pgfqpoint{5.526913in}{0.739656in}}%
\pgfpathlineto{\pgfqpoint{5.526617in}{0.739656in}}%
\pgfpathlineto{\pgfqpoint{5.526321in}{0.739656in}}%
\pgfpathlineto{\pgfqpoint{5.526025in}{0.739656in}}%
\pgfpathlineto{\pgfqpoint{5.525729in}{0.739656in}}%
\pgfpathlineto{\pgfqpoint{5.525433in}{0.739656in}}%
\pgfpathlineto{\pgfqpoint{5.525137in}{0.739656in}}%
\pgfpathlineto{\pgfqpoint{5.524841in}{0.739656in}}%
\pgfpathlineto{\pgfqpoint{5.524545in}{0.739656in}}%
\pgfpathlineto{\pgfqpoint{5.524249in}{0.739656in}}%
\pgfpathlineto{\pgfqpoint{5.523953in}{0.739656in}}%
\pgfpathlineto{\pgfqpoint{5.523657in}{0.739656in}}%
\pgfpathlineto{\pgfqpoint{5.523361in}{0.739656in}}%
\pgfpathlineto{\pgfqpoint{5.523065in}{0.739656in}}%
\pgfpathlineto{\pgfqpoint{5.522769in}{0.739656in}}%
\pgfpathlineto{\pgfqpoint{5.522473in}{0.739656in}}%
\pgfpathlineto{\pgfqpoint{5.522177in}{0.739656in}}%
\pgfpathlineto{\pgfqpoint{5.521881in}{0.739656in}}%
\pgfpathlineto{\pgfqpoint{5.521585in}{0.739656in}}%
\pgfpathlineto{\pgfqpoint{5.521289in}{0.739656in}}%
\pgfpathlineto{\pgfqpoint{5.520993in}{0.739656in}}%
\pgfpathlineto{\pgfqpoint{5.520697in}{0.739656in}}%
\pgfpathlineto{\pgfqpoint{5.520401in}{0.739656in}}%
\pgfpathlineto{\pgfqpoint{5.520105in}{0.739656in}}%
\pgfpathlineto{\pgfqpoint{5.519809in}{0.739656in}}%
\pgfpathlineto{\pgfqpoint{5.519513in}{0.739656in}}%
\pgfpathlineto{\pgfqpoint{5.519217in}{0.739656in}}%
\pgfpathlineto{\pgfqpoint{5.518921in}{0.739656in}}%
\pgfpathlineto{\pgfqpoint{5.518625in}{0.739656in}}%
\pgfpathlineto{\pgfqpoint{5.518329in}{0.739656in}}%
\pgfpathlineto{\pgfqpoint{5.518033in}{0.739656in}}%
\pgfpathlineto{\pgfqpoint{5.517737in}{0.739656in}}%
\pgfpathlineto{\pgfqpoint{5.517441in}{0.739656in}}%
\pgfpathlineto{\pgfqpoint{5.517145in}{0.739656in}}%
\pgfpathlineto{\pgfqpoint{5.516849in}{0.739656in}}%
\pgfpathlineto{\pgfqpoint{5.516553in}{0.739656in}}%
\pgfpathlineto{\pgfqpoint{5.516257in}{0.739656in}}%
\pgfpathlineto{\pgfqpoint{5.515961in}{0.739656in}}%
\pgfpathlineto{\pgfqpoint{5.515665in}{0.739656in}}%
\pgfpathlineto{\pgfqpoint{5.515369in}{0.739656in}}%
\pgfpathlineto{\pgfqpoint{5.515073in}{0.739656in}}%
\pgfpathlineto{\pgfqpoint{5.514777in}{0.739656in}}%
\pgfpathlineto{\pgfqpoint{5.514481in}{0.739656in}}%
\pgfpathlineto{\pgfqpoint{5.514185in}{0.739656in}}%
\pgfpathlineto{\pgfqpoint{5.513889in}{0.739656in}}%
\pgfpathlineto{\pgfqpoint{5.513593in}{0.739656in}}%
\pgfpathlineto{\pgfqpoint{5.513297in}{0.739656in}}%
\pgfpathlineto{\pgfqpoint{5.513001in}{0.739656in}}%
\pgfpathlineto{\pgfqpoint{5.512705in}{0.739656in}}%
\pgfpathlineto{\pgfqpoint{5.512409in}{0.739656in}}%
\pgfpathlineto{\pgfqpoint{5.512113in}{0.739656in}}%
\pgfpathlineto{\pgfqpoint{5.511817in}{0.739656in}}%
\pgfpathlineto{\pgfqpoint{5.511521in}{0.739656in}}%
\pgfpathlineto{\pgfqpoint{5.511225in}{0.739656in}}%
\pgfpathlineto{\pgfqpoint{5.510929in}{0.739656in}}%
\pgfpathlineto{\pgfqpoint{5.510633in}{0.739656in}}%
\pgfpathlineto{\pgfqpoint{5.510337in}{0.739656in}}%
\pgfpathlineto{\pgfqpoint{5.510041in}{0.739656in}}%
\pgfpathlineto{\pgfqpoint{5.509745in}{0.739656in}}%
\pgfpathlineto{\pgfqpoint{5.509449in}{0.739656in}}%
\pgfpathlineto{\pgfqpoint{5.509153in}{0.739656in}}%
\pgfpathlineto{\pgfqpoint{5.508857in}{0.739656in}}%
\pgfpathlineto{\pgfqpoint{5.508561in}{0.739656in}}%
\pgfpathlineto{\pgfqpoint{5.508265in}{0.739656in}}%
\pgfpathlineto{\pgfqpoint{5.507969in}{0.739656in}}%
\pgfpathlineto{\pgfqpoint{5.507673in}{0.739656in}}%
\pgfpathlineto{\pgfqpoint{5.507377in}{0.739656in}}%
\pgfpathlineto{\pgfqpoint{5.507081in}{0.739656in}}%
\pgfpathlineto{\pgfqpoint{5.506785in}{0.739656in}}%
\pgfpathlineto{\pgfqpoint{5.506489in}{0.739656in}}%
\pgfpathlineto{\pgfqpoint{5.506193in}{0.739656in}}%
\pgfpathlineto{\pgfqpoint{5.505897in}{0.739656in}}%
\pgfpathlineto{\pgfqpoint{5.505601in}{0.739656in}}%
\pgfpathlineto{\pgfqpoint{5.505305in}{0.739656in}}%
\pgfpathlineto{\pgfqpoint{5.505009in}{0.739656in}}%
\pgfpathlineto{\pgfqpoint{5.504713in}{0.739656in}}%
\pgfpathlineto{\pgfqpoint{5.504417in}{0.739656in}}%
\pgfpathlineto{\pgfqpoint{5.504121in}{0.739656in}}%
\pgfpathlineto{\pgfqpoint{5.503825in}{0.739656in}}%
\pgfpathlineto{\pgfqpoint{5.503529in}{0.739656in}}%
\pgfpathlineto{\pgfqpoint{5.503233in}{0.739656in}}%
\pgfpathlineto{\pgfqpoint{5.502937in}{0.739656in}}%
\pgfpathlineto{\pgfqpoint{5.502641in}{0.739656in}}%
\pgfpathlineto{\pgfqpoint{5.502345in}{0.739656in}}%
\pgfpathlineto{\pgfqpoint{5.502049in}{0.739656in}}%
\pgfpathlineto{\pgfqpoint{5.501752in}{0.739656in}}%
\pgfpathlineto{\pgfqpoint{5.501456in}{0.739656in}}%
\pgfpathlineto{\pgfqpoint{5.501160in}{0.739656in}}%
\pgfpathlineto{\pgfqpoint{5.500864in}{0.739656in}}%
\pgfpathlineto{\pgfqpoint{5.500568in}{0.739656in}}%
\pgfpathlineto{\pgfqpoint{5.500272in}{0.739656in}}%
\pgfpathlineto{\pgfqpoint{5.499976in}{0.739656in}}%
\pgfpathlineto{\pgfqpoint{5.499680in}{0.739656in}}%
\pgfpathlineto{\pgfqpoint{5.499384in}{0.739656in}}%
\pgfpathlineto{\pgfqpoint{5.499088in}{0.739656in}}%
\pgfpathlineto{\pgfqpoint{5.498792in}{0.739656in}}%
\pgfpathlineto{\pgfqpoint{5.498496in}{0.739656in}}%
\pgfpathlineto{\pgfqpoint{5.498200in}{0.739656in}}%
\pgfpathlineto{\pgfqpoint{5.497904in}{0.739656in}}%
\pgfpathlineto{\pgfqpoint{5.497608in}{0.739656in}}%
\pgfpathlineto{\pgfqpoint{5.497312in}{0.739656in}}%
\pgfpathlineto{\pgfqpoint{5.497016in}{0.739656in}}%
\pgfpathlineto{\pgfqpoint{5.496720in}{0.739656in}}%
\pgfpathlineto{\pgfqpoint{5.496424in}{0.739656in}}%
\pgfpathlineto{\pgfqpoint{5.496128in}{0.739656in}}%
\pgfpathlineto{\pgfqpoint{5.495832in}{0.739656in}}%
\pgfpathlineto{\pgfqpoint{5.495536in}{0.739656in}}%
\pgfpathlineto{\pgfqpoint{5.495240in}{0.739656in}}%
\pgfpathlineto{\pgfqpoint{5.494944in}{0.739656in}}%
\pgfpathlineto{\pgfqpoint{5.494648in}{0.739656in}}%
\pgfpathlineto{\pgfqpoint{5.494352in}{0.739656in}}%
\pgfpathlineto{\pgfqpoint{5.494056in}{0.739656in}}%
\pgfpathlineto{\pgfqpoint{5.493760in}{0.739656in}}%
\pgfpathlineto{\pgfqpoint{5.493464in}{0.739656in}}%
\pgfpathlineto{\pgfqpoint{5.493168in}{0.739656in}}%
\pgfpathlineto{\pgfqpoint{5.492872in}{0.739656in}}%
\pgfpathlineto{\pgfqpoint{5.492576in}{0.739656in}}%
\pgfpathlineto{\pgfqpoint{5.492280in}{0.739656in}}%
\pgfpathlineto{\pgfqpoint{5.491984in}{0.739656in}}%
\pgfpathlineto{\pgfqpoint{5.491688in}{0.739656in}}%
\pgfpathlineto{\pgfqpoint{5.491392in}{0.739656in}}%
\pgfpathlineto{\pgfqpoint{5.491096in}{0.739656in}}%
\pgfpathlineto{\pgfqpoint{5.490800in}{0.739656in}}%
\pgfpathlineto{\pgfqpoint{5.490504in}{0.739656in}}%
\pgfpathlineto{\pgfqpoint{5.490208in}{0.739656in}}%
\pgfpathlineto{\pgfqpoint{5.489912in}{0.739656in}}%
\pgfpathlineto{\pgfqpoint{5.489616in}{0.739656in}}%
\pgfpathlineto{\pgfqpoint{5.489320in}{0.739656in}}%
\pgfpathlineto{\pgfqpoint{5.489024in}{0.739656in}}%
\pgfpathlineto{\pgfqpoint{5.488728in}{0.739656in}}%
\pgfpathlineto{\pgfqpoint{5.488432in}{0.739656in}}%
\pgfpathlineto{\pgfqpoint{5.488136in}{0.739656in}}%
\pgfpathlineto{\pgfqpoint{5.487840in}{0.739656in}}%
\pgfpathlineto{\pgfqpoint{5.487544in}{0.739656in}}%
\pgfpathlineto{\pgfqpoint{5.487248in}{0.739656in}}%
\pgfpathlineto{\pgfqpoint{5.486952in}{0.739656in}}%
\pgfpathlineto{\pgfqpoint{5.486656in}{0.739656in}}%
\pgfpathlineto{\pgfqpoint{5.486360in}{0.739656in}}%
\pgfpathlineto{\pgfqpoint{5.486064in}{0.739656in}}%
\pgfpathlineto{\pgfqpoint{5.485768in}{0.739656in}}%
\pgfpathlineto{\pgfqpoint{5.485472in}{0.739656in}}%
\pgfpathlineto{\pgfqpoint{5.485176in}{0.739656in}}%
\pgfpathlineto{\pgfqpoint{5.484880in}{0.739656in}}%
\pgfpathlineto{\pgfqpoint{5.484584in}{0.739656in}}%
\pgfpathlineto{\pgfqpoint{5.484288in}{0.739656in}}%
\pgfpathlineto{\pgfqpoint{5.483992in}{0.739656in}}%
\pgfpathlineto{\pgfqpoint{5.483696in}{0.739656in}}%
\pgfpathlineto{\pgfqpoint{5.483400in}{0.739656in}}%
\pgfpathlineto{\pgfqpoint{5.483104in}{0.739656in}}%
\pgfpathlineto{\pgfqpoint{5.482808in}{0.739656in}}%
\pgfpathlineto{\pgfqpoint{5.482512in}{0.739656in}}%
\pgfpathlineto{\pgfqpoint{5.482216in}{0.739656in}}%
\pgfpathlineto{\pgfqpoint{5.481920in}{0.739656in}}%
\pgfpathlineto{\pgfqpoint{5.481624in}{0.739656in}}%
\pgfpathlineto{\pgfqpoint{5.481328in}{0.739656in}}%
\pgfpathlineto{\pgfqpoint{5.481032in}{0.739656in}}%
\pgfpathlineto{\pgfqpoint{5.480736in}{0.739656in}}%
\pgfpathlineto{\pgfqpoint{5.480440in}{0.739656in}}%
\pgfpathlineto{\pgfqpoint{5.480144in}{0.739656in}}%
\pgfpathlineto{\pgfqpoint{5.479848in}{0.739656in}}%
\pgfpathlineto{\pgfqpoint{5.479552in}{0.739656in}}%
\pgfpathlineto{\pgfqpoint{5.479256in}{0.739656in}}%
\pgfpathlineto{\pgfqpoint{5.478960in}{0.739656in}}%
\pgfpathlineto{\pgfqpoint{5.478664in}{0.739656in}}%
\pgfpathlineto{\pgfqpoint{5.478368in}{0.739656in}}%
\pgfpathlineto{\pgfqpoint{5.478072in}{0.739656in}}%
\pgfpathlineto{\pgfqpoint{5.477776in}{0.739656in}}%
\pgfpathlineto{\pgfqpoint{5.477480in}{0.739656in}}%
\pgfpathlineto{\pgfqpoint{5.477184in}{0.739656in}}%
\pgfpathlineto{\pgfqpoint{5.476888in}{0.739656in}}%
\pgfpathlineto{\pgfqpoint{5.476592in}{0.739656in}}%
\pgfpathlineto{\pgfqpoint{5.476296in}{0.739656in}}%
\pgfpathlineto{\pgfqpoint{5.476000in}{0.739656in}}%
\pgfpathlineto{\pgfqpoint{5.475704in}{0.739656in}}%
\pgfpathlineto{\pgfqpoint{5.475408in}{0.739656in}}%
\pgfpathlineto{\pgfqpoint{5.475112in}{0.739656in}}%
\pgfpathlineto{\pgfqpoint{5.474816in}{0.739656in}}%
\pgfpathlineto{\pgfqpoint{5.474520in}{0.739656in}}%
\pgfpathlineto{\pgfqpoint{5.474224in}{0.739656in}}%
\pgfpathlineto{\pgfqpoint{5.473928in}{0.739656in}}%
\pgfpathlineto{\pgfqpoint{5.473632in}{0.739656in}}%
\pgfpathlineto{\pgfqpoint{5.473336in}{0.739656in}}%
\pgfpathlineto{\pgfqpoint{5.473040in}{0.739656in}}%
\pgfpathlineto{\pgfqpoint{5.472744in}{0.739656in}}%
\pgfpathlineto{\pgfqpoint{5.472448in}{0.739656in}}%
\pgfpathlineto{\pgfqpoint{5.472152in}{0.739656in}}%
\pgfpathlineto{\pgfqpoint{5.471856in}{0.739656in}}%
\pgfpathlineto{\pgfqpoint{5.471560in}{0.739656in}}%
\pgfpathlineto{\pgfqpoint{5.471264in}{0.739656in}}%
\pgfpathlineto{\pgfqpoint{5.470968in}{0.739656in}}%
\pgfpathlineto{\pgfqpoint{5.470672in}{0.739656in}}%
\pgfpathlineto{\pgfqpoint{5.470376in}{0.739656in}}%
\pgfpathlineto{\pgfqpoint{5.470080in}{0.739656in}}%
\pgfpathlineto{\pgfqpoint{5.469784in}{0.739656in}}%
\pgfpathlineto{\pgfqpoint{5.469488in}{0.739656in}}%
\pgfpathlineto{\pgfqpoint{5.469192in}{0.739656in}}%
\pgfpathlineto{\pgfqpoint{5.468896in}{0.739656in}}%
\pgfpathlineto{\pgfqpoint{5.468600in}{0.739656in}}%
\pgfpathlineto{\pgfqpoint{5.468304in}{0.739656in}}%
\pgfpathlineto{\pgfqpoint{5.468008in}{0.739656in}}%
\pgfpathlineto{\pgfqpoint{5.467712in}{0.739656in}}%
\pgfpathlineto{\pgfqpoint{5.467416in}{0.739656in}}%
\pgfpathlineto{\pgfqpoint{5.467120in}{0.739656in}}%
\pgfpathlineto{\pgfqpoint{5.466824in}{0.739656in}}%
\pgfpathlineto{\pgfqpoint{5.466528in}{0.739656in}}%
\pgfpathlineto{\pgfqpoint{5.466232in}{0.739656in}}%
\pgfpathlineto{\pgfqpoint{5.465936in}{0.739656in}}%
\pgfpathlineto{\pgfqpoint{5.465640in}{0.739656in}}%
\pgfpathlineto{\pgfqpoint{5.465344in}{0.739656in}}%
\pgfpathlineto{\pgfqpoint{5.465048in}{0.739656in}}%
\pgfpathlineto{\pgfqpoint{5.464752in}{0.739656in}}%
\pgfpathlineto{\pgfqpoint{5.464456in}{0.739656in}}%
\pgfpathlineto{\pgfqpoint{5.464160in}{0.739656in}}%
\pgfpathlineto{\pgfqpoint{5.463864in}{0.739656in}}%
\pgfpathlineto{\pgfqpoint{5.463568in}{0.739656in}}%
\pgfpathlineto{\pgfqpoint{5.463272in}{0.739656in}}%
\pgfpathlineto{\pgfqpoint{5.462976in}{0.739656in}}%
\pgfpathlineto{\pgfqpoint{5.462680in}{0.739656in}}%
\pgfpathlineto{\pgfqpoint{5.462384in}{0.739656in}}%
\pgfpathlineto{\pgfqpoint{5.462088in}{0.739656in}}%
\pgfpathlineto{\pgfqpoint{5.461792in}{0.739656in}}%
\pgfpathlineto{\pgfqpoint{5.461496in}{0.739656in}}%
\pgfpathlineto{\pgfqpoint{5.461200in}{0.739656in}}%
\pgfpathlineto{\pgfqpoint{5.460904in}{0.739656in}}%
\pgfpathlineto{\pgfqpoint{5.460608in}{0.739656in}}%
\pgfpathlineto{\pgfqpoint{5.460312in}{0.739656in}}%
\pgfpathlineto{\pgfqpoint{5.460016in}{0.739656in}}%
\pgfpathlineto{\pgfqpoint{5.459720in}{0.739656in}}%
\pgfpathlineto{\pgfqpoint{5.459424in}{0.739656in}}%
\pgfpathlineto{\pgfqpoint{5.459128in}{0.739656in}}%
\pgfpathlineto{\pgfqpoint{5.458832in}{0.739656in}}%
\pgfpathlineto{\pgfqpoint{5.458536in}{0.739656in}}%
\pgfpathlineto{\pgfqpoint{5.458240in}{0.739656in}}%
\pgfpathlineto{\pgfqpoint{5.457944in}{0.739656in}}%
\pgfpathlineto{\pgfqpoint{5.457648in}{0.739656in}}%
\pgfpathlineto{\pgfqpoint{5.457352in}{0.739656in}}%
\pgfpathlineto{\pgfqpoint{5.457056in}{0.739656in}}%
\pgfpathlineto{\pgfqpoint{5.456760in}{0.739656in}}%
\pgfpathlineto{\pgfqpoint{5.456464in}{0.739656in}}%
\pgfpathlineto{\pgfqpoint{5.456168in}{0.739656in}}%
\pgfpathlineto{\pgfqpoint{5.455872in}{0.739656in}}%
\pgfpathlineto{\pgfqpoint{5.455576in}{0.739656in}}%
\pgfpathlineto{\pgfqpoint{5.455280in}{0.739656in}}%
\pgfpathlineto{\pgfqpoint{5.454984in}{0.739656in}}%
\pgfpathlineto{\pgfqpoint{5.454688in}{0.739656in}}%
\pgfpathlineto{\pgfqpoint{5.454392in}{0.739656in}}%
\pgfpathlineto{\pgfqpoint{5.454096in}{0.739656in}}%
\pgfpathlineto{\pgfqpoint{5.453800in}{0.739656in}}%
\pgfpathlineto{\pgfqpoint{5.453504in}{0.739656in}}%
\pgfpathlineto{\pgfqpoint{5.453208in}{0.739656in}}%
\pgfpathlineto{\pgfqpoint{5.452912in}{0.739656in}}%
\pgfpathlineto{\pgfqpoint{5.452616in}{0.739656in}}%
\pgfpathlineto{\pgfqpoint{5.452320in}{0.739656in}}%
\pgfpathlineto{\pgfqpoint{5.452024in}{0.739656in}}%
\pgfpathlineto{\pgfqpoint{5.451728in}{0.739656in}}%
\pgfpathlineto{\pgfqpoint{5.451432in}{0.739656in}}%
\pgfpathlineto{\pgfqpoint{5.451136in}{0.739656in}}%
\pgfpathlineto{\pgfqpoint{5.450840in}{0.739656in}}%
\pgfpathlineto{\pgfqpoint{5.450544in}{0.739656in}}%
\pgfpathlineto{\pgfqpoint{5.450248in}{0.739656in}}%
\pgfpathlineto{\pgfqpoint{5.449952in}{0.739656in}}%
\pgfpathlineto{\pgfqpoint{5.449656in}{0.739656in}}%
\pgfpathlineto{\pgfqpoint{5.449360in}{0.739656in}}%
\pgfpathlineto{\pgfqpoint{5.449064in}{0.739656in}}%
\pgfpathlineto{\pgfqpoint{5.448768in}{0.739656in}}%
\pgfpathlineto{\pgfqpoint{5.448472in}{0.739656in}}%
\pgfpathlineto{\pgfqpoint{5.448176in}{0.739656in}}%
\pgfpathlineto{\pgfqpoint{5.447880in}{0.739656in}}%
\pgfpathlineto{\pgfqpoint{5.447584in}{0.739656in}}%
\pgfpathlineto{\pgfqpoint{5.447288in}{0.739656in}}%
\pgfpathlineto{\pgfqpoint{5.446992in}{0.739656in}}%
\pgfpathlineto{\pgfqpoint{5.446696in}{0.739656in}}%
\pgfpathlineto{\pgfqpoint{5.446400in}{0.739656in}}%
\pgfpathlineto{\pgfqpoint{5.446104in}{0.739656in}}%
\pgfpathlineto{\pgfqpoint{5.445808in}{0.739656in}}%
\pgfpathlineto{\pgfqpoint{5.445512in}{0.739656in}}%
\pgfpathlineto{\pgfqpoint{5.445216in}{0.739656in}}%
\pgfpathlineto{\pgfqpoint{5.444920in}{0.739656in}}%
\pgfpathlineto{\pgfqpoint{5.444624in}{0.739656in}}%
\pgfpathlineto{\pgfqpoint{5.444328in}{0.739656in}}%
\pgfpathlineto{\pgfqpoint{5.444032in}{0.739656in}}%
\pgfpathlineto{\pgfqpoint{5.443736in}{0.739656in}}%
\pgfpathlineto{\pgfqpoint{5.443440in}{0.739656in}}%
\pgfpathlineto{\pgfqpoint{5.443144in}{0.739656in}}%
\pgfpathlineto{\pgfqpoint{5.442848in}{0.739656in}}%
\pgfpathlineto{\pgfqpoint{5.442552in}{0.739656in}}%
\pgfpathlineto{\pgfqpoint{5.442256in}{0.739656in}}%
\pgfpathlineto{\pgfqpoint{5.441960in}{0.739656in}}%
\pgfpathlineto{\pgfqpoint{5.441664in}{0.739656in}}%
\pgfpathlineto{\pgfqpoint{5.441368in}{0.739656in}}%
\pgfpathlineto{\pgfqpoint{5.441072in}{0.739656in}}%
\pgfpathlineto{\pgfqpoint{5.440776in}{0.739656in}}%
\pgfpathlineto{\pgfqpoint{5.440480in}{0.739656in}}%
\pgfpathlineto{\pgfqpoint{5.440184in}{0.739656in}}%
\pgfpathlineto{\pgfqpoint{5.439888in}{0.739656in}}%
\pgfpathlineto{\pgfqpoint{5.439592in}{0.739656in}}%
\pgfpathlineto{\pgfqpoint{5.439296in}{0.739656in}}%
\pgfpathlineto{\pgfqpoint{5.439000in}{0.739656in}}%
\pgfpathlineto{\pgfqpoint{5.438704in}{0.739656in}}%
\pgfpathlineto{\pgfqpoint{5.438408in}{0.739656in}}%
\pgfpathlineto{\pgfqpoint{5.438112in}{0.739656in}}%
\pgfpathlineto{\pgfqpoint{5.437816in}{0.739656in}}%
\pgfpathlineto{\pgfqpoint{5.437520in}{0.739656in}}%
\pgfpathlineto{\pgfqpoint{5.437224in}{0.739656in}}%
\pgfpathlineto{\pgfqpoint{5.436928in}{0.739656in}}%
\pgfpathlineto{\pgfqpoint{5.436632in}{0.739656in}}%
\pgfpathlineto{\pgfqpoint{5.436336in}{0.739656in}}%
\pgfpathlineto{\pgfqpoint{5.436040in}{0.739656in}}%
\pgfpathlineto{\pgfqpoint{5.435744in}{0.739656in}}%
\pgfpathlineto{\pgfqpoint{5.435448in}{0.739656in}}%
\pgfpathlineto{\pgfqpoint{5.435152in}{0.739656in}}%
\pgfpathlineto{\pgfqpoint{5.434856in}{0.739656in}}%
\pgfpathlineto{\pgfqpoint{5.434560in}{0.739656in}}%
\pgfpathlineto{\pgfqpoint{5.434263in}{0.739656in}}%
\pgfpathlineto{\pgfqpoint{5.433967in}{0.739656in}}%
\pgfpathlineto{\pgfqpoint{5.433671in}{0.739656in}}%
\pgfpathlineto{\pgfqpoint{5.433375in}{0.739656in}}%
\pgfpathlineto{\pgfqpoint{5.433079in}{0.739656in}}%
\pgfpathlineto{\pgfqpoint{5.432783in}{0.739656in}}%
\pgfpathlineto{\pgfqpoint{5.432487in}{0.739656in}}%
\pgfpathlineto{\pgfqpoint{5.432191in}{0.739656in}}%
\pgfpathlineto{\pgfqpoint{5.431895in}{0.739656in}}%
\pgfpathlineto{\pgfqpoint{5.431599in}{0.739656in}}%
\pgfpathlineto{\pgfqpoint{5.431303in}{0.739656in}}%
\pgfpathlineto{\pgfqpoint{5.431007in}{0.739656in}}%
\pgfpathlineto{\pgfqpoint{5.430711in}{0.739656in}}%
\pgfpathlineto{\pgfqpoint{5.430415in}{0.739656in}}%
\pgfpathlineto{\pgfqpoint{5.430119in}{0.739656in}}%
\pgfpathlineto{\pgfqpoint{5.429823in}{0.739656in}}%
\pgfpathlineto{\pgfqpoint{5.429527in}{0.739656in}}%
\pgfpathlineto{\pgfqpoint{5.429231in}{0.739656in}}%
\pgfpathlineto{\pgfqpoint{5.428935in}{0.739656in}}%
\pgfpathlineto{\pgfqpoint{5.428639in}{0.739656in}}%
\pgfpathlineto{\pgfqpoint{5.428343in}{0.739656in}}%
\pgfpathlineto{\pgfqpoint{5.428047in}{0.739656in}}%
\pgfpathlineto{\pgfqpoint{5.427751in}{0.739656in}}%
\pgfpathlineto{\pgfqpoint{5.427455in}{0.739656in}}%
\pgfpathlineto{\pgfqpoint{5.427159in}{0.739656in}}%
\pgfpathlineto{\pgfqpoint{5.426863in}{0.739656in}}%
\pgfpathlineto{\pgfqpoint{5.426567in}{0.739656in}}%
\pgfpathlineto{\pgfqpoint{5.426271in}{0.739656in}}%
\pgfpathlineto{\pgfqpoint{5.425975in}{0.739656in}}%
\pgfpathlineto{\pgfqpoint{5.425679in}{0.739656in}}%
\pgfpathlineto{\pgfqpoint{5.425383in}{0.739656in}}%
\pgfpathlineto{\pgfqpoint{5.425087in}{0.739656in}}%
\pgfpathlineto{\pgfqpoint{5.424791in}{0.739656in}}%
\pgfpathlineto{\pgfqpoint{5.424495in}{0.739656in}}%
\pgfpathlineto{\pgfqpoint{5.424199in}{0.739656in}}%
\pgfpathlineto{\pgfqpoint{5.423903in}{0.739656in}}%
\pgfpathlineto{\pgfqpoint{5.423607in}{0.739656in}}%
\pgfpathlineto{\pgfqpoint{5.423311in}{0.739656in}}%
\pgfpathlineto{\pgfqpoint{5.423015in}{0.739656in}}%
\pgfpathlineto{\pgfqpoint{5.422719in}{0.739656in}}%
\pgfpathlineto{\pgfqpoint{5.422423in}{0.739656in}}%
\pgfpathlineto{\pgfqpoint{5.422127in}{0.739656in}}%
\pgfpathlineto{\pgfqpoint{5.421831in}{0.739656in}}%
\pgfpathlineto{\pgfqpoint{5.421535in}{0.739656in}}%
\pgfpathlineto{\pgfqpoint{5.421239in}{0.739656in}}%
\pgfpathlineto{\pgfqpoint{5.420943in}{0.739656in}}%
\pgfpathlineto{\pgfqpoint{5.420647in}{0.739656in}}%
\pgfpathlineto{\pgfqpoint{5.420351in}{0.739656in}}%
\pgfpathlineto{\pgfqpoint{5.420055in}{0.739656in}}%
\pgfpathlineto{\pgfqpoint{5.419759in}{0.739656in}}%
\pgfpathlineto{\pgfqpoint{5.419463in}{0.739656in}}%
\pgfpathlineto{\pgfqpoint{5.419167in}{0.739656in}}%
\pgfpathlineto{\pgfqpoint{5.418871in}{0.739656in}}%
\pgfpathlineto{\pgfqpoint{5.418575in}{0.739656in}}%
\pgfpathlineto{\pgfqpoint{5.418279in}{0.739656in}}%
\pgfpathlineto{\pgfqpoint{5.417983in}{0.739656in}}%
\pgfpathlineto{\pgfqpoint{5.417687in}{0.739656in}}%
\pgfpathlineto{\pgfqpoint{5.417391in}{0.739656in}}%
\pgfpathlineto{\pgfqpoint{5.417095in}{0.739656in}}%
\pgfpathlineto{\pgfqpoint{5.416799in}{0.739656in}}%
\pgfpathlineto{\pgfqpoint{5.416503in}{0.739656in}}%
\pgfpathlineto{\pgfqpoint{5.416207in}{0.739656in}}%
\pgfpathlineto{\pgfqpoint{5.415911in}{0.739656in}}%
\pgfpathlineto{\pgfqpoint{5.415615in}{0.739656in}}%
\pgfpathlineto{\pgfqpoint{5.415319in}{0.739656in}}%
\pgfpathlineto{\pgfqpoint{5.415023in}{0.739656in}}%
\pgfpathlineto{\pgfqpoint{5.414727in}{0.739656in}}%
\pgfpathlineto{\pgfqpoint{5.414431in}{0.739656in}}%
\pgfpathlineto{\pgfqpoint{5.414135in}{0.739656in}}%
\pgfpathlineto{\pgfqpoint{5.413839in}{0.739656in}}%
\pgfpathlineto{\pgfqpoint{5.413543in}{0.739656in}}%
\pgfpathlineto{\pgfqpoint{5.413247in}{0.739656in}}%
\pgfpathlineto{\pgfqpoint{5.412951in}{0.739656in}}%
\pgfpathlineto{\pgfqpoint{5.412655in}{0.739656in}}%
\pgfpathlineto{\pgfqpoint{5.412359in}{0.739656in}}%
\pgfpathlineto{\pgfqpoint{5.412063in}{0.739656in}}%
\pgfpathlineto{\pgfqpoint{5.411767in}{0.739656in}}%
\pgfpathlineto{\pgfqpoint{5.411471in}{0.739656in}}%
\pgfpathlineto{\pgfqpoint{5.411175in}{0.739656in}}%
\pgfpathlineto{\pgfqpoint{5.410879in}{0.739656in}}%
\pgfpathlineto{\pgfqpoint{5.410583in}{0.739656in}}%
\pgfpathlineto{\pgfqpoint{5.410287in}{0.739656in}}%
\pgfpathlineto{\pgfqpoint{5.409991in}{0.739656in}}%
\pgfpathlineto{\pgfqpoint{5.409695in}{0.739656in}}%
\pgfpathlineto{\pgfqpoint{5.409399in}{0.739656in}}%
\pgfpathlineto{\pgfqpoint{5.409103in}{0.739656in}}%
\pgfpathlineto{\pgfqpoint{5.408807in}{0.739656in}}%
\pgfpathlineto{\pgfqpoint{5.408511in}{0.739656in}}%
\pgfpathlineto{\pgfqpoint{5.408215in}{0.739656in}}%
\pgfpathlineto{\pgfqpoint{5.407919in}{0.739656in}}%
\pgfpathlineto{\pgfqpoint{5.407623in}{0.739656in}}%
\pgfpathlineto{\pgfqpoint{5.407327in}{0.739656in}}%
\pgfpathlineto{\pgfqpoint{5.407031in}{0.739656in}}%
\pgfpathlineto{\pgfqpoint{5.406735in}{0.739656in}}%
\pgfpathlineto{\pgfqpoint{5.406439in}{0.739656in}}%
\pgfpathlineto{\pgfqpoint{5.406143in}{0.739656in}}%
\pgfpathlineto{\pgfqpoint{5.405847in}{0.739656in}}%
\pgfpathlineto{\pgfqpoint{5.405551in}{0.739656in}}%
\pgfpathlineto{\pgfqpoint{5.405255in}{0.739656in}}%
\pgfpathlineto{\pgfqpoint{5.404959in}{0.739656in}}%
\pgfpathlineto{\pgfqpoint{5.404663in}{0.739656in}}%
\pgfpathlineto{\pgfqpoint{5.404367in}{0.739656in}}%
\pgfpathlineto{\pgfqpoint{5.404071in}{0.739656in}}%
\pgfpathlineto{\pgfqpoint{5.403775in}{0.739656in}}%
\pgfpathlineto{\pgfqpoint{5.403479in}{0.739656in}}%
\pgfpathlineto{\pgfqpoint{5.403183in}{0.739656in}}%
\pgfpathlineto{\pgfqpoint{5.402887in}{0.739656in}}%
\pgfpathlineto{\pgfqpoint{5.402591in}{0.739656in}}%
\pgfpathlineto{\pgfqpoint{5.402295in}{0.739656in}}%
\pgfpathlineto{\pgfqpoint{5.401999in}{0.739656in}}%
\pgfpathlineto{\pgfqpoint{5.401703in}{0.739656in}}%
\pgfpathlineto{\pgfqpoint{5.401407in}{0.739656in}}%
\pgfpathlineto{\pgfqpoint{5.401111in}{0.739656in}}%
\pgfpathlineto{\pgfqpoint{5.400815in}{0.739656in}}%
\pgfpathlineto{\pgfqpoint{5.400519in}{0.739656in}}%
\pgfpathlineto{\pgfqpoint{5.400223in}{0.739656in}}%
\pgfpathlineto{\pgfqpoint{5.399927in}{0.739656in}}%
\pgfpathlineto{\pgfqpoint{5.399631in}{0.739656in}}%
\pgfpathlineto{\pgfqpoint{5.399335in}{0.739656in}}%
\pgfpathlineto{\pgfqpoint{5.399039in}{0.739656in}}%
\pgfpathlineto{\pgfqpoint{5.398743in}{0.739656in}}%
\pgfpathlineto{\pgfqpoint{5.398447in}{0.739656in}}%
\pgfpathlineto{\pgfqpoint{5.398151in}{0.739656in}}%
\pgfpathlineto{\pgfqpoint{5.397855in}{0.739656in}}%
\pgfpathlineto{\pgfqpoint{5.397559in}{0.739656in}}%
\pgfpathlineto{\pgfqpoint{5.397263in}{0.739656in}}%
\pgfpathlineto{\pgfqpoint{5.396967in}{0.739656in}}%
\pgfpathlineto{\pgfqpoint{5.396671in}{0.739656in}}%
\pgfpathlineto{\pgfqpoint{5.396375in}{0.739656in}}%
\pgfpathlineto{\pgfqpoint{5.396079in}{0.739656in}}%
\pgfpathlineto{\pgfqpoint{5.395783in}{0.739656in}}%
\pgfpathlineto{\pgfqpoint{5.395487in}{0.739656in}}%
\pgfpathlineto{\pgfqpoint{5.395191in}{0.739656in}}%
\pgfpathlineto{\pgfqpoint{5.394895in}{0.739656in}}%
\pgfpathlineto{\pgfqpoint{5.394599in}{0.739656in}}%
\pgfpathlineto{\pgfqpoint{5.394303in}{0.739656in}}%
\pgfpathlineto{\pgfqpoint{5.394007in}{0.739656in}}%
\pgfpathlineto{\pgfqpoint{5.393711in}{0.739656in}}%
\pgfpathlineto{\pgfqpoint{5.393415in}{0.739656in}}%
\pgfpathlineto{\pgfqpoint{5.393119in}{0.739656in}}%
\pgfpathlineto{\pgfqpoint{5.392823in}{0.739656in}}%
\pgfpathlineto{\pgfqpoint{5.392527in}{0.739656in}}%
\pgfpathlineto{\pgfqpoint{5.392231in}{0.739656in}}%
\pgfpathlineto{\pgfqpoint{5.391935in}{0.739656in}}%
\pgfpathlineto{\pgfqpoint{5.391639in}{0.739656in}}%
\pgfpathlineto{\pgfqpoint{5.391343in}{0.739656in}}%
\pgfpathlineto{\pgfqpoint{5.391047in}{0.739656in}}%
\pgfpathlineto{\pgfqpoint{5.390751in}{0.739656in}}%
\pgfpathlineto{\pgfqpoint{5.390455in}{0.739656in}}%
\pgfpathlineto{\pgfqpoint{5.390159in}{0.739656in}}%
\pgfpathlineto{\pgfqpoint{5.389863in}{0.739656in}}%
\pgfpathlineto{\pgfqpoint{5.389567in}{0.739656in}}%
\pgfpathlineto{\pgfqpoint{5.389271in}{0.739656in}}%
\pgfpathlineto{\pgfqpoint{5.388975in}{0.739656in}}%
\pgfpathlineto{\pgfqpoint{5.388679in}{0.739656in}}%
\pgfpathlineto{\pgfqpoint{5.388383in}{0.739656in}}%
\pgfpathlineto{\pgfqpoint{5.388087in}{0.739656in}}%
\pgfpathlineto{\pgfqpoint{5.387791in}{0.739656in}}%
\pgfpathlineto{\pgfqpoint{5.387495in}{0.739656in}}%
\pgfpathlineto{\pgfqpoint{5.387199in}{0.739656in}}%
\pgfpathlineto{\pgfqpoint{5.386903in}{0.739656in}}%
\pgfpathlineto{\pgfqpoint{5.386607in}{0.739656in}}%
\pgfpathlineto{\pgfqpoint{5.386311in}{0.739656in}}%
\pgfpathlineto{\pgfqpoint{5.386015in}{0.739656in}}%
\pgfpathlineto{\pgfqpoint{5.385719in}{0.739656in}}%
\pgfpathlineto{\pgfqpoint{5.385423in}{0.739656in}}%
\pgfpathlineto{\pgfqpoint{5.385127in}{0.739656in}}%
\pgfpathlineto{\pgfqpoint{5.384831in}{0.739656in}}%
\pgfpathlineto{\pgfqpoint{5.384535in}{0.739656in}}%
\pgfpathlineto{\pgfqpoint{5.384239in}{0.739656in}}%
\pgfpathlineto{\pgfqpoint{5.383943in}{0.739656in}}%
\pgfpathlineto{\pgfqpoint{5.383647in}{0.739656in}}%
\pgfpathlineto{\pgfqpoint{5.383351in}{0.739656in}}%
\pgfpathlineto{\pgfqpoint{5.383055in}{0.739656in}}%
\pgfpathlineto{\pgfqpoint{5.382759in}{0.739656in}}%
\pgfpathlineto{\pgfqpoint{5.382463in}{0.739656in}}%
\pgfpathlineto{\pgfqpoint{5.382167in}{0.739656in}}%
\pgfpathlineto{\pgfqpoint{5.381871in}{0.739656in}}%
\pgfpathlineto{\pgfqpoint{5.381575in}{0.739656in}}%
\pgfpathlineto{\pgfqpoint{5.381279in}{0.739656in}}%
\pgfpathlineto{\pgfqpoint{5.380983in}{0.739656in}}%
\pgfpathlineto{\pgfqpoint{5.380687in}{0.739656in}}%
\pgfpathlineto{\pgfqpoint{5.380391in}{0.739656in}}%
\pgfpathlineto{\pgfqpoint{5.380095in}{0.739656in}}%
\pgfpathlineto{\pgfqpoint{5.379799in}{0.739656in}}%
\pgfpathlineto{\pgfqpoint{5.379503in}{0.739656in}}%
\pgfpathlineto{\pgfqpoint{5.379207in}{0.739656in}}%
\pgfpathlineto{\pgfqpoint{5.378911in}{0.739656in}}%
\pgfpathlineto{\pgfqpoint{5.378615in}{0.739656in}}%
\pgfpathlineto{\pgfqpoint{5.378319in}{0.739656in}}%
\pgfpathlineto{\pgfqpoint{5.378023in}{0.739656in}}%
\pgfpathlineto{\pgfqpoint{5.377727in}{0.739656in}}%
\pgfpathlineto{\pgfqpoint{5.377431in}{0.739656in}}%
\pgfpathlineto{\pgfqpoint{5.377135in}{0.739656in}}%
\pgfpathlineto{\pgfqpoint{5.376839in}{0.739656in}}%
\pgfpathlineto{\pgfqpoint{5.376543in}{0.739656in}}%
\pgfpathlineto{\pgfqpoint{5.376247in}{0.739656in}}%
\pgfpathlineto{\pgfqpoint{5.375951in}{0.739656in}}%
\pgfpathlineto{\pgfqpoint{5.375655in}{0.739656in}}%
\pgfpathlineto{\pgfqpoint{5.375359in}{0.739656in}}%
\pgfpathlineto{\pgfqpoint{5.375063in}{0.739656in}}%
\pgfpathlineto{\pgfqpoint{5.374767in}{0.739656in}}%
\pgfpathlineto{\pgfqpoint{5.374471in}{0.739656in}}%
\pgfpathlineto{\pgfqpoint{5.374175in}{0.739656in}}%
\pgfpathlineto{\pgfqpoint{5.373879in}{0.739656in}}%
\pgfpathlineto{\pgfqpoint{5.373583in}{0.739656in}}%
\pgfpathlineto{\pgfqpoint{5.373287in}{0.739656in}}%
\pgfpathlineto{\pgfqpoint{5.372991in}{0.739656in}}%
\pgfpathlineto{\pgfqpoint{5.372695in}{0.739656in}}%
\pgfpathlineto{\pgfqpoint{5.372399in}{0.739656in}}%
\pgfpathlineto{\pgfqpoint{5.372103in}{0.739656in}}%
\pgfpathlineto{\pgfqpoint{5.371807in}{0.739656in}}%
\pgfpathlineto{\pgfqpoint{5.371511in}{0.739656in}}%
\pgfpathlineto{\pgfqpoint{5.371215in}{0.739656in}}%
\pgfpathlineto{\pgfqpoint{5.370919in}{0.739656in}}%
\pgfpathlineto{\pgfqpoint{5.370623in}{0.739656in}}%
\pgfpathlineto{\pgfqpoint{5.370327in}{0.739656in}}%
\pgfpathlineto{\pgfqpoint{5.370031in}{0.739656in}}%
\pgfpathlineto{\pgfqpoint{5.369735in}{0.739656in}}%
\pgfpathlineto{\pgfqpoint{5.369439in}{0.739656in}}%
\pgfpathlineto{\pgfqpoint{5.369143in}{0.739656in}}%
\pgfpathlineto{\pgfqpoint{5.368847in}{0.739656in}}%
\pgfpathlineto{\pgfqpoint{5.368551in}{0.739656in}}%
\pgfpathlineto{\pgfqpoint{5.368255in}{0.739656in}}%
\pgfpathlineto{\pgfqpoint{5.367959in}{0.739656in}}%
\pgfpathlineto{\pgfqpoint{5.367663in}{0.739656in}}%
\pgfpathlineto{\pgfqpoint{5.367367in}{0.739656in}}%
\pgfpathlineto{\pgfqpoint{5.367071in}{0.739656in}}%
\pgfpathlineto{\pgfqpoint{5.366774in}{0.739656in}}%
\pgfpathlineto{\pgfqpoint{5.366478in}{0.739656in}}%
\pgfpathlineto{\pgfqpoint{5.366182in}{0.739656in}}%
\pgfpathlineto{\pgfqpoint{5.365886in}{0.739656in}}%
\pgfpathlineto{\pgfqpoint{5.365590in}{0.739656in}}%
\pgfpathlineto{\pgfqpoint{5.365294in}{0.739656in}}%
\pgfpathlineto{\pgfqpoint{5.364998in}{0.739656in}}%
\pgfpathlineto{\pgfqpoint{5.364702in}{0.739656in}}%
\pgfpathlineto{\pgfqpoint{5.364406in}{0.739656in}}%
\pgfpathlineto{\pgfqpoint{5.364110in}{0.739656in}}%
\pgfpathlineto{\pgfqpoint{5.363814in}{0.739656in}}%
\pgfpathlineto{\pgfqpoint{5.363518in}{0.739656in}}%
\pgfpathlineto{\pgfqpoint{5.363222in}{0.739656in}}%
\pgfpathlineto{\pgfqpoint{5.362926in}{0.739656in}}%
\pgfpathlineto{\pgfqpoint{5.362630in}{0.739656in}}%
\pgfpathlineto{\pgfqpoint{5.362334in}{0.739656in}}%
\pgfpathlineto{\pgfqpoint{5.362038in}{0.739656in}}%
\pgfpathlineto{\pgfqpoint{5.361742in}{0.739656in}}%
\pgfpathlineto{\pgfqpoint{5.361446in}{0.739656in}}%
\pgfpathlineto{\pgfqpoint{5.361150in}{0.739656in}}%
\pgfpathlineto{\pgfqpoint{5.360854in}{0.739656in}}%
\pgfpathlineto{\pgfqpoint{5.360558in}{0.739656in}}%
\pgfpathlineto{\pgfqpoint{5.360262in}{0.739656in}}%
\pgfpathlineto{\pgfqpoint{5.359966in}{0.739656in}}%
\pgfpathlineto{\pgfqpoint{5.359670in}{0.739656in}}%
\pgfpathlineto{\pgfqpoint{5.359374in}{0.739656in}}%
\pgfpathlineto{\pgfqpoint{5.359078in}{0.739656in}}%
\pgfpathlineto{\pgfqpoint{5.358782in}{0.739656in}}%
\pgfpathlineto{\pgfqpoint{5.358486in}{0.739656in}}%
\pgfpathlineto{\pgfqpoint{5.358190in}{0.739656in}}%
\pgfpathlineto{\pgfqpoint{5.357894in}{0.739656in}}%
\pgfpathlineto{\pgfqpoint{5.357598in}{0.739656in}}%
\pgfpathlineto{\pgfqpoint{5.357302in}{0.739656in}}%
\pgfpathlineto{\pgfqpoint{5.357006in}{0.739656in}}%
\pgfpathlineto{\pgfqpoint{5.356710in}{0.739656in}}%
\pgfpathlineto{\pgfqpoint{5.356414in}{0.739656in}}%
\pgfpathlineto{\pgfqpoint{5.356118in}{0.739656in}}%
\pgfpathlineto{\pgfqpoint{5.355822in}{0.739656in}}%
\pgfpathlineto{\pgfqpoint{5.355526in}{0.739656in}}%
\pgfpathlineto{\pgfqpoint{5.355230in}{0.739656in}}%
\pgfpathlineto{\pgfqpoint{5.354934in}{0.739656in}}%
\pgfpathlineto{\pgfqpoint{5.354638in}{0.739656in}}%
\pgfpathlineto{\pgfqpoint{5.354342in}{0.739656in}}%
\pgfpathlineto{\pgfqpoint{5.354046in}{0.739656in}}%
\pgfpathlineto{\pgfqpoint{5.353750in}{0.739656in}}%
\pgfpathlineto{\pgfqpoint{5.353454in}{0.739656in}}%
\pgfpathlineto{\pgfqpoint{5.353158in}{0.739656in}}%
\pgfpathlineto{\pgfqpoint{5.352862in}{0.739656in}}%
\pgfpathlineto{\pgfqpoint{5.352566in}{0.739656in}}%
\pgfpathlineto{\pgfqpoint{5.352270in}{0.739656in}}%
\pgfpathlineto{\pgfqpoint{5.351974in}{0.739656in}}%
\pgfpathlineto{\pgfqpoint{5.351678in}{0.739656in}}%
\pgfpathlineto{\pgfqpoint{5.351382in}{0.739656in}}%
\pgfpathlineto{\pgfqpoint{5.351086in}{0.739656in}}%
\pgfpathlineto{\pgfqpoint{5.350790in}{0.739656in}}%
\pgfpathlineto{\pgfqpoint{5.350494in}{0.739656in}}%
\pgfpathlineto{\pgfqpoint{5.350198in}{0.739656in}}%
\pgfpathlineto{\pgfqpoint{5.349902in}{0.739656in}}%
\pgfpathlineto{\pgfqpoint{5.349606in}{0.739656in}}%
\pgfpathlineto{\pgfqpoint{5.349310in}{0.739656in}}%
\pgfpathlineto{\pgfqpoint{5.349014in}{0.739656in}}%
\pgfpathlineto{\pgfqpoint{5.348718in}{0.739656in}}%
\pgfpathlineto{\pgfqpoint{5.348422in}{0.739656in}}%
\pgfpathlineto{\pgfqpoint{5.348126in}{0.739656in}}%
\pgfpathlineto{\pgfqpoint{5.347830in}{0.739656in}}%
\pgfpathlineto{\pgfqpoint{5.347534in}{0.739656in}}%
\pgfpathlineto{\pgfqpoint{5.347238in}{0.739656in}}%
\pgfpathlineto{\pgfqpoint{5.346942in}{0.739656in}}%
\pgfpathlineto{\pgfqpoint{5.346646in}{0.739656in}}%
\pgfpathlineto{\pgfqpoint{5.346350in}{0.739656in}}%
\pgfpathlineto{\pgfqpoint{5.346054in}{0.739656in}}%
\pgfpathlineto{\pgfqpoint{5.345758in}{0.739656in}}%
\pgfpathlineto{\pgfqpoint{5.345462in}{0.739656in}}%
\pgfpathlineto{\pgfqpoint{5.345166in}{0.739656in}}%
\pgfpathlineto{\pgfqpoint{5.344870in}{0.739656in}}%
\pgfpathlineto{\pgfqpoint{5.344574in}{0.739656in}}%
\pgfpathlineto{\pgfqpoint{5.344278in}{0.739656in}}%
\pgfpathlineto{\pgfqpoint{5.343982in}{0.739656in}}%
\pgfpathlineto{\pgfqpoint{5.343686in}{0.739656in}}%
\pgfpathlineto{\pgfqpoint{5.343390in}{0.739656in}}%
\pgfpathlineto{\pgfqpoint{5.343094in}{0.739656in}}%
\pgfpathlineto{\pgfqpoint{5.342798in}{0.739656in}}%
\pgfpathlineto{\pgfqpoint{5.342502in}{0.739656in}}%
\pgfpathlineto{\pgfqpoint{5.342206in}{0.739656in}}%
\pgfpathlineto{\pgfqpoint{5.341910in}{0.739656in}}%
\pgfpathlineto{\pgfqpoint{5.341614in}{0.739656in}}%
\pgfpathlineto{\pgfqpoint{5.341318in}{0.739656in}}%
\pgfpathlineto{\pgfqpoint{5.341022in}{0.739656in}}%
\pgfpathlineto{\pgfqpoint{5.340726in}{0.739656in}}%
\pgfpathlineto{\pgfqpoint{5.340430in}{0.739656in}}%
\pgfpathlineto{\pgfqpoint{5.340134in}{0.739656in}}%
\pgfpathlineto{\pgfqpoint{5.339838in}{0.739656in}}%
\pgfpathlineto{\pgfqpoint{5.339542in}{0.739656in}}%
\pgfpathlineto{\pgfqpoint{5.339246in}{0.739656in}}%
\pgfpathlineto{\pgfqpoint{5.338950in}{0.739656in}}%
\pgfpathlineto{\pgfqpoint{5.338654in}{0.739656in}}%
\pgfpathlineto{\pgfqpoint{5.338358in}{0.739656in}}%
\pgfpathlineto{\pgfqpoint{5.338062in}{0.739656in}}%
\pgfpathlineto{\pgfqpoint{5.337766in}{0.739656in}}%
\pgfpathlineto{\pgfqpoint{5.337470in}{0.739656in}}%
\pgfpathlineto{\pgfqpoint{5.337174in}{0.739656in}}%
\pgfpathlineto{\pgfqpoint{5.336878in}{0.739656in}}%
\pgfpathlineto{\pgfqpoint{5.336582in}{0.739656in}}%
\pgfpathlineto{\pgfqpoint{5.336286in}{0.739656in}}%
\pgfpathlineto{\pgfqpoint{5.335990in}{0.739656in}}%
\pgfpathlineto{\pgfqpoint{5.335694in}{0.739656in}}%
\pgfpathlineto{\pgfqpoint{5.335398in}{0.739656in}}%
\pgfpathlineto{\pgfqpoint{5.335102in}{0.739656in}}%
\pgfpathlineto{\pgfqpoint{5.334806in}{0.739656in}}%
\pgfpathlineto{\pgfqpoint{5.334510in}{0.739656in}}%
\pgfpathlineto{\pgfqpoint{5.334214in}{0.739656in}}%
\pgfpathlineto{\pgfqpoint{5.333918in}{0.739656in}}%
\pgfpathlineto{\pgfqpoint{5.333622in}{0.739656in}}%
\pgfpathlineto{\pgfqpoint{5.333326in}{0.739656in}}%
\pgfpathlineto{\pgfqpoint{5.333030in}{0.739656in}}%
\pgfpathlineto{\pgfqpoint{5.332734in}{0.739656in}}%
\pgfpathlineto{\pgfqpoint{5.332438in}{0.739656in}}%
\pgfpathlineto{\pgfqpoint{5.332142in}{0.739656in}}%
\pgfpathlineto{\pgfqpoint{5.331846in}{0.739656in}}%
\pgfpathlineto{\pgfqpoint{5.331550in}{0.739656in}}%
\pgfpathlineto{\pgfqpoint{5.331254in}{0.739656in}}%
\pgfpathlineto{\pgfqpoint{5.330958in}{0.739656in}}%
\pgfpathlineto{\pgfqpoint{5.330662in}{0.739656in}}%
\pgfpathlineto{\pgfqpoint{5.330366in}{0.739656in}}%
\pgfpathlineto{\pgfqpoint{5.330070in}{0.739656in}}%
\pgfpathlineto{\pgfqpoint{5.329774in}{0.739656in}}%
\pgfpathlineto{\pgfqpoint{5.329478in}{0.739656in}}%
\pgfpathlineto{\pgfqpoint{5.329182in}{0.739656in}}%
\pgfpathlineto{\pgfqpoint{5.328886in}{0.739656in}}%
\pgfpathlineto{\pgfqpoint{5.328590in}{0.739656in}}%
\pgfpathlineto{\pgfqpoint{5.328294in}{0.739656in}}%
\pgfpathlineto{\pgfqpoint{5.327998in}{0.739656in}}%
\pgfpathlineto{\pgfqpoint{5.327702in}{0.739656in}}%
\pgfpathlineto{\pgfqpoint{5.327406in}{0.739656in}}%
\pgfpathlineto{\pgfqpoint{5.327110in}{0.739656in}}%
\pgfpathlineto{\pgfqpoint{5.326814in}{0.739656in}}%
\pgfpathlineto{\pgfqpoint{5.326518in}{0.739656in}}%
\pgfpathlineto{\pgfqpoint{5.326222in}{0.739656in}}%
\pgfpathlineto{\pgfqpoint{5.325926in}{0.739656in}}%
\pgfpathlineto{\pgfqpoint{5.325630in}{0.739656in}}%
\pgfpathlineto{\pgfqpoint{5.325334in}{0.739656in}}%
\pgfpathlineto{\pgfqpoint{5.325038in}{0.739656in}}%
\pgfpathlineto{\pgfqpoint{5.324742in}{0.739656in}}%
\pgfpathlineto{\pgfqpoint{5.324446in}{0.739656in}}%
\pgfpathlineto{\pgfqpoint{5.324150in}{0.739656in}}%
\pgfpathlineto{\pgfqpoint{5.323854in}{0.739656in}}%
\pgfpathlineto{\pgfqpoint{5.323558in}{0.739656in}}%
\pgfpathlineto{\pgfqpoint{5.323262in}{0.739656in}}%
\pgfpathlineto{\pgfqpoint{5.322966in}{0.739656in}}%
\pgfpathlineto{\pgfqpoint{5.322670in}{0.739656in}}%
\pgfpathlineto{\pgfqpoint{5.322374in}{0.739656in}}%
\pgfpathlineto{\pgfqpoint{5.322078in}{0.739656in}}%
\pgfpathlineto{\pgfqpoint{5.321782in}{0.739656in}}%
\pgfpathlineto{\pgfqpoint{5.321486in}{0.739656in}}%
\pgfpathlineto{\pgfqpoint{5.321190in}{0.739656in}}%
\pgfpathlineto{\pgfqpoint{5.320894in}{0.739656in}}%
\pgfpathlineto{\pgfqpoint{5.320598in}{0.739656in}}%
\pgfpathlineto{\pgfqpoint{5.320302in}{0.739656in}}%
\pgfpathlineto{\pgfqpoint{5.320006in}{0.739656in}}%
\pgfpathlineto{\pgfqpoint{5.319710in}{0.739656in}}%
\pgfpathlineto{\pgfqpoint{5.319414in}{0.739656in}}%
\pgfpathlineto{\pgfqpoint{5.319118in}{0.739656in}}%
\pgfpathlineto{\pgfqpoint{5.318822in}{0.739656in}}%
\pgfpathlineto{\pgfqpoint{5.318526in}{0.739656in}}%
\pgfpathlineto{\pgfqpoint{5.318230in}{0.739656in}}%
\pgfpathlineto{\pgfqpoint{5.317934in}{0.739656in}}%
\pgfpathlineto{\pgfqpoint{5.317638in}{0.739656in}}%
\pgfpathlineto{\pgfqpoint{5.317342in}{0.739656in}}%
\pgfpathlineto{\pgfqpoint{5.317046in}{0.739656in}}%
\pgfpathlineto{\pgfqpoint{5.316750in}{0.739656in}}%
\pgfpathlineto{\pgfqpoint{5.316454in}{0.739656in}}%
\pgfpathlineto{\pgfqpoint{5.316158in}{0.739656in}}%
\pgfpathlineto{\pgfqpoint{5.315862in}{0.739656in}}%
\pgfpathlineto{\pgfqpoint{5.315566in}{0.739656in}}%
\pgfpathlineto{\pgfqpoint{5.315270in}{0.739656in}}%
\pgfpathlineto{\pgfqpoint{5.314974in}{0.739656in}}%
\pgfpathlineto{\pgfqpoint{5.314678in}{0.739656in}}%
\pgfpathlineto{\pgfqpoint{5.314382in}{0.739656in}}%
\pgfpathlineto{\pgfqpoint{5.314086in}{0.739656in}}%
\pgfpathlineto{\pgfqpoint{5.313790in}{0.739656in}}%
\pgfpathlineto{\pgfqpoint{5.313494in}{0.739656in}}%
\pgfpathlineto{\pgfqpoint{5.313198in}{0.739656in}}%
\pgfpathlineto{\pgfqpoint{5.312902in}{0.739656in}}%
\pgfpathlineto{\pgfqpoint{5.312606in}{0.739656in}}%
\pgfpathlineto{\pgfqpoint{5.312310in}{0.739656in}}%
\pgfpathlineto{\pgfqpoint{5.312014in}{0.739656in}}%
\pgfpathlineto{\pgfqpoint{5.311718in}{0.739656in}}%
\pgfpathlineto{\pgfqpoint{5.311422in}{0.739656in}}%
\pgfpathlineto{\pgfqpoint{5.311126in}{0.739656in}}%
\pgfpathlineto{\pgfqpoint{5.310830in}{0.739656in}}%
\pgfpathlineto{\pgfqpoint{5.310534in}{0.739656in}}%
\pgfpathlineto{\pgfqpoint{5.310238in}{0.739656in}}%
\pgfpathlineto{\pgfqpoint{5.309942in}{0.739656in}}%
\pgfpathlineto{\pgfqpoint{5.309646in}{0.739656in}}%
\pgfpathlineto{\pgfqpoint{5.309350in}{0.739656in}}%
\pgfpathlineto{\pgfqpoint{5.309054in}{0.739656in}}%
\pgfpathlineto{\pgfqpoint{5.308758in}{0.739656in}}%
\pgfpathlineto{\pgfqpoint{5.308462in}{0.739656in}}%
\pgfpathlineto{\pgfqpoint{5.308166in}{0.739656in}}%
\pgfpathlineto{\pgfqpoint{5.307870in}{0.739656in}}%
\pgfpathlineto{\pgfqpoint{5.307574in}{0.739656in}}%
\pgfpathlineto{\pgfqpoint{5.307278in}{0.739656in}}%
\pgfpathlineto{\pgfqpoint{5.306982in}{0.739656in}}%
\pgfpathlineto{\pgfqpoint{5.306686in}{0.739656in}}%
\pgfpathlineto{\pgfqpoint{5.306390in}{0.739656in}}%
\pgfpathlineto{\pgfqpoint{5.306094in}{0.739656in}}%
\pgfpathlineto{\pgfqpoint{5.305798in}{0.739656in}}%
\pgfpathlineto{\pgfqpoint{5.305502in}{0.739656in}}%
\pgfpathlineto{\pgfqpoint{5.305206in}{0.739656in}}%
\pgfpathlineto{\pgfqpoint{5.304910in}{0.739656in}}%
\pgfpathlineto{\pgfqpoint{5.304614in}{0.739656in}}%
\pgfpathlineto{\pgfqpoint{5.304318in}{0.739656in}}%
\pgfpathlineto{\pgfqpoint{5.304022in}{0.739656in}}%
\pgfpathlineto{\pgfqpoint{5.303726in}{0.739656in}}%
\pgfpathlineto{\pgfqpoint{5.303430in}{0.739656in}}%
\pgfpathlineto{\pgfqpoint{5.303134in}{0.739656in}}%
\pgfpathlineto{\pgfqpoint{5.302838in}{0.739656in}}%
\pgfpathlineto{\pgfqpoint{5.302542in}{0.739656in}}%
\pgfpathlineto{\pgfqpoint{5.302246in}{0.739656in}}%
\pgfpathlineto{\pgfqpoint{5.301950in}{0.739656in}}%
\pgfpathlineto{\pgfqpoint{5.301654in}{0.739656in}}%
\pgfpathlineto{\pgfqpoint{5.301358in}{0.739656in}}%
\pgfpathlineto{\pgfqpoint{5.301062in}{0.739656in}}%
\pgfpathlineto{\pgfqpoint{5.300766in}{0.739656in}}%
\pgfpathlineto{\pgfqpoint{5.300470in}{0.739656in}}%
\pgfpathlineto{\pgfqpoint{5.300174in}{0.739656in}}%
\pgfpathlineto{\pgfqpoint{5.299878in}{0.739656in}}%
\pgfpathlineto{\pgfqpoint{5.299582in}{0.739656in}}%
\pgfpathlineto{\pgfqpoint{5.299285in}{0.739656in}}%
\pgfpathlineto{\pgfqpoint{5.298989in}{0.739656in}}%
\pgfpathlineto{\pgfqpoint{5.298693in}{0.739656in}}%
\pgfpathlineto{\pgfqpoint{5.298397in}{0.739656in}}%
\pgfpathlineto{\pgfqpoint{5.298101in}{0.739656in}}%
\pgfpathlineto{\pgfqpoint{5.297805in}{0.739656in}}%
\pgfpathlineto{\pgfqpoint{5.297509in}{0.739656in}}%
\pgfpathlineto{\pgfqpoint{5.297213in}{0.739656in}}%
\pgfpathlineto{\pgfqpoint{5.296917in}{0.739656in}}%
\pgfpathlineto{\pgfqpoint{5.296621in}{0.739656in}}%
\pgfpathlineto{\pgfqpoint{5.296325in}{0.739656in}}%
\pgfpathlineto{\pgfqpoint{5.296029in}{0.739656in}}%
\pgfpathlineto{\pgfqpoint{5.295733in}{0.739656in}}%
\pgfpathlineto{\pgfqpoint{5.295437in}{0.739656in}}%
\pgfpathlineto{\pgfqpoint{5.295141in}{0.739656in}}%
\pgfpathlineto{\pgfqpoint{5.294845in}{0.739656in}}%
\pgfpathlineto{\pgfqpoint{5.294549in}{0.739656in}}%
\pgfpathlineto{\pgfqpoint{5.294253in}{0.739656in}}%
\pgfpathlineto{\pgfqpoint{5.293957in}{0.739656in}}%
\pgfpathlineto{\pgfqpoint{5.293661in}{0.739656in}}%
\pgfpathlineto{\pgfqpoint{5.293365in}{0.739656in}}%
\pgfpathlineto{\pgfqpoint{5.293069in}{0.739656in}}%
\pgfpathlineto{\pgfqpoint{5.292773in}{0.739656in}}%
\pgfpathlineto{\pgfqpoint{5.292477in}{0.739656in}}%
\pgfpathlineto{\pgfqpoint{5.292181in}{0.739656in}}%
\pgfpathlineto{\pgfqpoint{5.291885in}{0.739656in}}%
\pgfpathlineto{\pgfqpoint{5.291589in}{0.739656in}}%
\pgfpathlineto{\pgfqpoint{5.291293in}{0.739656in}}%
\pgfpathlineto{\pgfqpoint{5.290997in}{0.739656in}}%
\pgfpathlineto{\pgfqpoint{5.290701in}{0.739656in}}%
\pgfpathlineto{\pgfqpoint{5.290405in}{0.739656in}}%
\pgfpathlineto{\pgfqpoint{5.290109in}{0.739656in}}%
\pgfpathlineto{\pgfqpoint{5.289813in}{0.739656in}}%
\pgfpathlineto{\pgfqpoint{5.289517in}{0.739656in}}%
\pgfpathlineto{\pgfqpoint{5.289221in}{0.739656in}}%
\pgfpathlineto{\pgfqpoint{5.288925in}{0.739656in}}%
\pgfpathlineto{\pgfqpoint{5.288629in}{0.739656in}}%
\pgfpathlineto{\pgfqpoint{5.288333in}{0.739656in}}%
\pgfpathlineto{\pgfqpoint{5.288037in}{0.739656in}}%
\pgfpathlineto{\pgfqpoint{5.287741in}{0.739656in}}%
\pgfpathlineto{\pgfqpoint{5.287445in}{0.739656in}}%
\pgfpathlineto{\pgfqpoint{5.287149in}{0.739656in}}%
\pgfpathlineto{\pgfqpoint{5.286853in}{0.739656in}}%
\pgfpathlineto{\pgfqpoint{5.286557in}{0.739656in}}%
\pgfpathlineto{\pgfqpoint{5.286261in}{0.739656in}}%
\pgfpathlineto{\pgfqpoint{5.285965in}{0.739656in}}%
\pgfpathlineto{\pgfqpoint{5.285669in}{0.739656in}}%
\pgfpathlineto{\pgfqpoint{5.285373in}{0.739656in}}%
\pgfpathlineto{\pgfqpoint{5.285077in}{0.739656in}}%
\pgfpathlineto{\pgfqpoint{5.284781in}{0.739656in}}%
\pgfpathlineto{\pgfqpoint{5.284485in}{0.739656in}}%
\pgfpathlineto{\pgfqpoint{5.284189in}{0.739656in}}%
\pgfpathlineto{\pgfqpoint{5.283893in}{0.739656in}}%
\pgfpathlineto{\pgfqpoint{5.283597in}{0.739656in}}%
\pgfpathlineto{\pgfqpoint{5.283301in}{0.739656in}}%
\pgfpathlineto{\pgfqpoint{5.283005in}{0.739656in}}%
\pgfpathlineto{\pgfqpoint{5.282709in}{0.739656in}}%
\pgfpathlineto{\pgfqpoint{5.282413in}{0.739656in}}%
\pgfpathlineto{\pgfqpoint{5.282117in}{0.739656in}}%
\pgfpathlineto{\pgfqpoint{5.281821in}{0.739656in}}%
\pgfpathlineto{\pgfqpoint{5.281525in}{0.739656in}}%
\pgfpathlineto{\pgfqpoint{5.281229in}{0.739656in}}%
\pgfpathlineto{\pgfqpoint{5.280933in}{0.739656in}}%
\pgfpathlineto{\pgfqpoint{5.280637in}{0.739656in}}%
\pgfpathlineto{\pgfqpoint{5.280341in}{0.739656in}}%
\pgfpathlineto{\pgfqpoint{5.280045in}{0.739656in}}%
\pgfpathlineto{\pgfqpoint{5.279749in}{0.739656in}}%
\pgfpathlineto{\pgfqpoint{5.279453in}{0.739656in}}%
\pgfpathlineto{\pgfqpoint{5.279157in}{0.739656in}}%
\pgfpathlineto{\pgfqpoint{5.278861in}{0.739656in}}%
\pgfpathlineto{\pgfqpoint{5.278565in}{0.739656in}}%
\pgfpathlineto{\pgfqpoint{5.278269in}{0.739656in}}%
\pgfpathlineto{\pgfqpoint{5.277973in}{0.739656in}}%
\pgfpathlineto{\pgfqpoint{5.277677in}{0.739656in}}%
\pgfpathlineto{\pgfqpoint{5.277381in}{0.739656in}}%
\pgfpathlineto{\pgfqpoint{5.277085in}{0.739656in}}%
\pgfpathlineto{\pgfqpoint{5.276789in}{0.739656in}}%
\pgfpathlineto{\pgfqpoint{5.276493in}{0.739656in}}%
\pgfpathlineto{\pgfqpoint{5.276197in}{0.739656in}}%
\pgfpathlineto{\pgfqpoint{5.275901in}{0.739656in}}%
\pgfpathlineto{\pgfqpoint{5.275605in}{0.739656in}}%
\pgfpathlineto{\pgfqpoint{5.275309in}{0.739656in}}%
\pgfpathlineto{\pgfqpoint{5.275013in}{0.739656in}}%
\pgfpathlineto{\pgfqpoint{5.274717in}{0.739656in}}%
\pgfpathlineto{\pgfqpoint{5.274421in}{0.739656in}}%
\pgfpathlineto{\pgfqpoint{5.274125in}{0.739656in}}%
\pgfpathlineto{\pgfqpoint{5.273829in}{0.739656in}}%
\pgfpathlineto{\pgfqpoint{5.273533in}{0.739656in}}%
\pgfpathlineto{\pgfqpoint{5.273237in}{0.739656in}}%
\pgfpathlineto{\pgfqpoint{5.272941in}{0.739656in}}%
\pgfpathlineto{\pgfqpoint{5.272645in}{0.739656in}}%
\pgfpathlineto{\pgfqpoint{5.272349in}{0.739656in}}%
\pgfpathlineto{\pgfqpoint{5.272053in}{0.739656in}}%
\pgfpathlineto{\pgfqpoint{5.271757in}{0.739656in}}%
\pgfpathlineto{\pgfqpoint{5.271461in}{0.739656in}}%
\pgfpathlineto{\pgfqpoint{5.271165in}{0.739656in}}%
\pgfpathlineto{\pgfqpoint{5.270869in}{0.739656in}}%
\pgfpathlineto{\pgfqpoint{5.270573in}{0.739656in}}%
\pgfpathlineto{\pgfqpoint{5.270277in}{0.739656in}}%
\pgfpathlineto{\pgfqpoint{5.269981in}{0.739656in}}%
\pgfpathlineto{\pgfqpoint{5.269685in}{0.739656in}}%
\pgfpathlineto{\pgfqpoint{5.269389in}{0.739656in}}%
\pgfpathlineto{\pgfqpoint{5.269093in}{0.739656in}}%
\pgfpathlineto{\pgfqpoint{5.268797in}{0.739656in}}%
\pgfpathlineto{\pgfqpoint{5.268501in}{0.739656in}}%
\pgfpathlineto{\pgfqpoint{5.268205in}{0.739656in}}%
\pgfpathlineto{\pgfqpoint{5.267909in}{0.739656in}}%
\pgfpathlineto{\pgfqpoint{5.267613in}{0.739656in}}%
\pgfpathlineto{\pgfqpoint{5.267317in}{0.739656in}}%
\pgfpathlineto{\pgfqpoint{5.267021in}{0.739656in}}%
\pgfpathlineto{\pgfqpoint{5.266725in}{0.739656in}}%
\pgfpathlineto{\pgfqpoint{5.266429in}{0.739656in}}%
\pgfpathlineto{\pgfqpoint{5.266133in}{0.739656in}}%
\pgfpathlineto{\pgfqpoint{5.265837in}{0.739656in}}%
\pgfpathlineto{\pgfqpoint{5.265541in}{0.739656in}}%
\pgfpathlineto{\pgfqpoint{5.265245in}{0.739656in}}%
\pgfpathlineto{\pgfqpoint{5.264949in}{0.739656in}}%
\pgfpathlineto{\pgfqpoint{5.264653in}{0.739656in}}%
\pgfpathlineto{\pgfqpoint{5.264357in}{0.739656in}}%
\pgfpathlineto{\pgfqpoint{5.264061in}{0.739656in}}%
\pgfpathlineto{\pgfqpoint{5.263765in}{0.739656in}}%
\pgfpathlineto{\pgfqpoint{5.263469in}{0.739656in}}%
\pgfpathlineto{\pgfqpoint{5.263173in}{0.739656in}}%
\pgfpathlineto{\pgfqpoint{5.262877in}{0.739656in}}%
\pgfpathlineto{\pgfqpoint{5.262581in}{0.739656in}}%
\pgfpathlineto{\pgfqpoint{5.262285in}{0.739656in}}%
\pgfpathlineto{\pgfqpoint{5.261989in}{0.739656in}}%
\pgfpathlineto{\pgfqpoint{5.261693in}{0.739656in}}%
\pgfpathlineto{\pgfqpoint{5.261397in}{0.739656in}}%
\pgfpathlineto{\pgfqpoint{5.261101in}{0.739656in}}%
\pgfpathlineto{\pgfqpoint{5.260805in}{0.739656in}}%
\pgfpathlineto{\pgfqpoint{5.260509in}{0.739656in}}%
\pgfpathlineto{\pgfqpoint{5.260213in}{0.739656in}}%
\pgfpathlineto{\pgfqpoint{5.259917in}{0.739656in}}%
\pgfpathlineto{\pgfqpoint{5.259621in}{0.739656in}}%
\pgfpathlineto{\pgfqpoint{5.259325in}{0.739656in}}%
\pgfpathlineto{\pgfqpoint{5.259029in}{0.739656in}}%
\pgfpathlineto{\pgfqpoint{5.258733in}{0.739656in}}%
\pgfpathlineto{\pgfqpoint{5.258437in}{0.739656in}}%
\pgfpathlineto{\pgfqpoint{5.258141in}{0.739656in}}%
\pgfpathlineto{\pgfqpoint{5.257845in}{0.739656in}}%
\pgfpathlineto{\pgfqpoint{5.257549in}{0.739656in}}%
\pgfpathlineto{\pgfqpoint{5.257253in}{0.739656in}}%
\pgfpathlineto{\pgfqpoint{5.256957in}{0.739656in}}%
\pgfpathlineto{\pgfqpoint{5.256661in}{0.739656in}}%
\pgfpathlineto{\pgfqpoint{5.256365in}{0.739656in}}%
\pgfpathlineto{\pgfqpoint{5.256069in}{0.739656in}}%
\pgfpathlineto{\pgfqpoint{5.255773in}{0.739656in}}%
\pgfpathlineto{\pgfqpoint{5.255477in}{0.739656in}}%
\pgfpathlineto{\pgfqpoint{5.255181in}{0.739656in}}%
\pgfpathlineto{\pgfqpoint{5.254885in}{0.739656in}}%
\pgfpathlineto{\pgfqpoint{5.254589in}{0.739656in}}%
\pgfpathlineto{\pgfqpoint{5.254293in}{0.739656in}}%
\pgfpathlineto{\pgfqpoint{5.253997in}{0.739656in}}%
\pgfpathlineto{\pgfqpoint{5.253701in}{0.739656in}}%
\pgfpathlineto{\pgfqpoint{5.253405in}{0.739656in}}%
\pgfpathlineto{\pgfqpoint{5.253109in}{0.739656in}}%
\pgfpathlineto{\pgfqpoint{5.252813in}{0.739656in}}%
\pgfpathlineto{\pgfqpoint{5.252517in}{0.739656in}}%
\pgfpathlineto{\pgfqpoint{5.252221in}{0.739656in}}%
\pgfpathlineto{\pgfqpoint{5.251925in}{0.739656in}}%
\pgfpathlineto{\pgfqpoint{5.251629in}{0.739656in}}%
\pgfpathlineto{\pgfqpoint{5.251333in}{0.739656in}}%
\pgfpathlineto{\pgfqpoint{5.251037in}{0.739656in}}%
\pgfpathlineto{\pgfqpoint{5.250741in}{0.739656in}}%
\pgfpathlineto{\pgfqpoint{5.250445in}{0.739656in}}%
\pgfpathlineto{\pgfqpoint{5.250149in}{0.739656in}}%
\pgfpathlineto{\pgfqpoint{5.249853in}{0.739656in}}%
\pgfpathlineto{\pgfqpoint{5.249557in}{0.739656in}}%
\pgfpathlineto{\pgfqpoint{5.249261in}{0.739656in}}%
\pgfpathlineto{\pgfqpoint{5.248965in}{0.739656in}}%
\pgfpathlineto{\pgfqpoint{5.248669in}{0.739656in}}%
\pgfpathlineto{\pgfqpoint{5.248373in}{0.739656in}}%
\pgfpathlineto{\pgfqpoint{5.248077in}{0.739656in}}%
\pgfpathlineto{\pgfqpoint{5.247781in}{0.739656in}}%
\pgfpathlineto{\pgfqpoint{5.247485in}{0.739656in}}%
\pgfpathlineto{\pgfqpoint{5.247189in}{0.739656in}}%
\pgfpathlineto{\pgfqpoint{5.246893in}{0.739656in}}%
\pgfpathlineto{\pgfqpoint{5.246597in}{0.739656in}}%
\pgfpathlineto{\pgfqpoint{5.246301in}{0.739656in}}%
\pgfpathlineto{\pgfqpoint{5.246005in}{0.739656in}}%
\pgfpathlineto{\pgfqpoint{5.245709in}{0.739656in}}%
\pgfpathlineto{\pgfqpoint{5.245413in}{0.739656in}}%
\pgfpathlineto{\pgfqpoint{5.245117in}{0.739656in}}%
\pgfpathlineto{\pgfqpoint{5.244821in}{0.739656in}}%
\pgfpathlineto{\pgfqpoint{5.244525in}{0.739656in}}%
\pgfpathlineto{\pgfqpoint{5.244229in}{0.739656in}}%
\pgfpathlineto{\pgfqpoint{5.243933in}{0.739656in}}%
\pgfpathlineto{\pgfqpoint{5.243637in}{0.739656in}}%
\pgfpathlineto{\pgfqpoint{5.243341in}{0.739656in}}%
\pgfpathlineto{\pgfqpoint{5.243045in}{0.739656in}}%
\pgfpathlineto{\pgfqpoint{5.242749in}{0.739656in}}%
\pgfpathlineto{\pgfqpoint{5.242453in}{0.739656in}}%
\pgfpathlineto{\pgfqpoint{5.242157in}{0.739656in}}%
\pgfpathlineto{\pgfqpoint{5.241861in}{0.739656in}}%
\pgfpathlineto{\pgfqpoint{5.241565in}{0.739656in}}%
\pgfpathlineto{\pgfqpoint{5.241269in}{0.739656in}}%
\pgfpathlineto{\pgfqpoint{5.240973in}{0.739656in}}%
\pgfpathlineto{\pgfqpoint{5.240677in}{0.739656in}}%
\pgfpathlineto{\pgfqpoint{5.240381in}{0.739656in}}%
\pgfpathlineto{\pgfqpoint{5.240085in}{0.739656in}}%
\pgfpathlineto{\pgfqpoint{5.239789in}{0.739656in}}%
\pgfpathlineto{\pgfqpoint{5.239493in}{0.739656in}}%
\pgfpathlineto{\pgfqpoint{5.239197in}{0.739656in}}%
\pgfpathlineto{\pgfqpoint{5.238901in}{0.739656in}}%
\pgfpathlineto{\pgfqpoint{5.238605in}{0.739656in}}%
\pgfpathlineto{\pgfqpoint{5.238309in}{0.739656in}}%
\pgfpathlineto{\pgfqpoint{5.238013in}{0.739656in}}%
\pgfpathlineto{\pgfqpoint{5.237717in}{0.739656in}}%
\pgfpathlineto{\pgfqpoint{5.237421in}{0.739656in}}%
\pgfpathlineto{\pgfqpoint{5.237125in}{0.739656in}}%
\pgfpathlineto{\pgfqpoint{5.236829in}{0.739656in}}%
\pgfpathlineto{\pgfqpoint{5.236533in}{0.739656in}}%
\pgfpathlineto{\pgfqpoint{5.236237in}{0.739656in}}%
\pgfpathlineto{\pgfqpoint{5.235941in}{0.739656in}}%
\pgfpathlineto{\pgfqpoint{5.235645in}{0.739656in}}%
\pgfpathlineto{\pgfqpoint{5.235349in}{0.739656in}}%
\pgfpathlineto{\pgfqpoint{5.235053in}{0.739656in}}%
\pgfpathlineto{\pgfqpoint{5.234757in}{0.739656in}}%
\pgfpathlineto{\pgfqpoint{5.234461in}{0.739656in}}%
\pgfpathlineto{\pgfqpoint{5.234165in}{0.739656in}}%
\pgfpathlineto{\pgfqpoint{5.233869in}{0.739656in}}%
\pgfpathlineto{\pgfqpoint{5.233573in}{0.739656in}}%
\pgfpathlineto{\pgfqpoint{5.233277in}{0.739656in}}%
\pgfpathlineto{\pgfqpoint{5.232981in}{0.739656in}}%
\pgfpathlineto{\pgfqpoint{5.232685in}{0.739656in}}%
\pgfpathlineto{\pgfqpoint{5.232389in}{0.739656in}}%
\pgfpathlineto{\pgfqpoint{5.232092in}{0.739656in}}%
\pgfpathlineto{\pgfqpoint{5.231796in}{0.739656in}}%
\pgfpathlineto{\pgfqpoint{5.231500in}{0.739656in}}%
\pgfpathlineto{\pgfqpoint{5.231204in}{0.739656in}}%
\pgfpathlineto{\pgfqpoint{5.230908in}{0.739656in}}%
\pgfpathlineto{\pgfqpoint{5.230612in}{0.739656in}}%
\pgfpathlineto{\pgfqpoint{5.230316in}{0.739656in}}%
\pgfpathlineto{\pgfqpoint{5.230020in}{0.739656in}}%
\pgfpathlineto{\pgfqpoint{5.229724in}{0.739656in}}%
\pgfpathlineto{\pgfqpoint{5.229428in}{0.739656in}}%
\pgfpathlineto{\pgfqpoint{5.229132in}{0.739656in}}%
\pgfpathlineto{\pgfqpoint{5.228836in}{0.739656in}}%
\pgfpathlineto{\pgfqpoint{5.228540in}{0.739656in}}%
\pgfpathlineto{\pgfqpoint{5.228244in}{0.739656in}}%
\pgfpathlineto{\pgfqpoint{5.227948in}{0.739656in}}%
\pgfpathlineto{\pgfqpoint{5.227652in}{0.739656in}}%
\pgfpathlineto{\pgfqpoint{5.227356in}{0.739656in}}%
\pgfpathlineto{\pgfqpoint{5.227060in}{0.739656in}}%
\pgfpathlineto{\pgfqpoint{5.226764in}{0.739656in}}%
\pgfpathlineto{\pgfqpoint{5.226468in}{0.739656in}}%
\pgfpathlineto{\pgfqpoint{5.226172in}{0.739656in}}%
\pgfpathlineto{\pgfqpoint{5.225876in}{0.739656in}}%
\pgfpathlineto{\pgfqpoint{5.225580in}{0.739656in}}%
\pgfpathlineto{\pgfqpoint{5.225284in}{0.739656in}}%
\pgfpathlineto{\pgfqpoint{5.224988in}{0.739656in}}%
\pgfpathlineto{\pgfqpoint{5.224692in}{0.739656in}}%
\pgfpathlineto{\pgfqpoint{5.224396in}{0.739656in}}%
\pgfpathlineto{\pgfqpoint{5.224100in}{0.739656in}}%
\pgfpathlineto{\pgfqpoint{5.223804in}{0.739656in}}%
\pgfpathlineto{\pgfqpoint{5.223508in}{0.739656in}}%
\pgfpathlineto{\pgfqpoint{5.223212in}{0.739656in}}%
\pgfpathlineto{\pgfqpoint{5.222916in}{0.739656in}}%
\pgfpathlineto{\pgfqpoint{5.222620in}{0.739656in}}%
\pgfpathlineto{\pgfqpoint{5.222324in}{0.739656in}}%
\pgfpathlineto{\pgfqpoint{5.222028in}{0.739656in}}%
\pgfpathlineto{\pgfqpoint{5.221732in}{0.739656in}}%
\pgfpathlineto{\pgfqpoint{5.221436in}{0.739656in}}%
\pgfpathlineto{\pgfqpoint{5.221140in}{0.739656in}}%
\pgfpathlineto{\pgfqpoint{5.220844in}{0.739656in}}%
\pgfpathlineto{\pgfqpoint{5.220548in}{0.739656in}}%
\pgfpathlineto{\pgfqpoint{5.220252in}{0.739656in}}%
\pgfpathlineto{\pgfqpoint{5.219956in}{0.739656in}}%
\pgfpathlineto{\pgfqpoint{5.219660in}{0.739656in}}%
\pgfpathlineto{\pgfqpoint{5.219364in}{0.739656in}}%
\pgfpathlineto{\pgfqpoint{5.219068in}{0.739656in}}%
\pgfpathlineto{\pgfqpoint{5.218772in}{0.739656in}}%
\pgfpathlineto{\pgfqpoint{5.218476in}{0.739656in}}%
\pgfpathlineto{\pgfqpoint{5.218180in}{0.739656in}}%
\pgfpathlineto{\pgfqpoint{5.217884in}{0.739656in}}%
\pgfpathlineto{\pgfqpoint{5.217588in}{0.739656in}}%
\pgfpathlineto{\pgfqpoint{5.217292in}{0.739656in}}%
\pgfpathlineto{\pgfqpoint{5.216996in}{0.739656in}}%
\pgfpathlineto{\pgfqpoint{5.216700in}{0.739656in}}%
\pgfpathlineto{\pgfqpoint{5.216404in}{0.739656in}}%
\pgfpathlineto{\pgfqpoint{5.216108in}{0.739656in}}%
\pgfpathlineto{\pgfqpoint{5.215812in}{0.739656in}}%
\pgfpathlineto{\pgfqpoint{5.215516in}{0.739656in}}%
\pgfpathlineto{\pgfqpoint{5.215220in}{0.739656in}}%
\pgfpathlineto{\pgfqpoint{5.214924in}{0.739656in}}%
\pgfpathlineto{\pgfqpoint{5.214628in}{0.739656in}}%
\pgfpathlineto{\pgfqpoint{5.214332in}{0.739656in}}%
\pgfpathlineto{\pgfqpoint{5.214036in}{0.739656in}}%
\pgfpathlineto{\pgfqpoint{5.213740in}{0.739656in}}%
\pgfpathlineto{\pgfqpoint{5.213444in}{0.739656in}}%
\pgfpathlineto{\pgfqpoint{5.213148in}{0.739656in}}%
\pgfpathlineto{\pgfqpoint{5.212852in}{0.739656in}}%
\pgfpathlineto{\pgfqpoint{5.212556in}{0.739656in}}%
\pgfpathlineto{\pgfqpoint{5.212260in}{0.739656in}}%
\pgfpathlineto{\pgfqpoint{5.211964in}{0.739656in}}%
\pgfpathlineto{\pgfqpoint{5.211668in}{0.739656in}}%
\pgfpathlineto{\pgfqpoint{5.211372in}{0.739656in}}%
\pgfpathlineto{\pgfqpoint{5.211076in}{0.739656in}}%
\pgfpathlineto{\pgfqpoint{5.210780in}{0.739656in}}%
\pgfpathlineto{\pgfqpoint{5.210484in}{0.739656in}}%
\pgfpathlineto{\pgfqpoint{5.210188in}{0.739656in}}%
\pgfpathlineto{\pgfqpoint{5.209892in}{0.739656in}}%
\pgfpathlineto{\pgfqpoint{5.209596in}{0.739656in}}%
\pgfpathlineto{\pgfqpoint{5.209300in}{0.739656in}}%
\pgfpathlineto{\pgfqpoint{5.209004in}{0.739656in}}%
\pgfpathlineto{\pgfqpoint{5.208708in}{0.739656in}}%
\pgfpathlineto{\pgfqpoint{5.208412in}{0.739656in}}%
\pgfpathlineto{\pgfqpoint{5.208116in}{0.739656in}}%
\pgfpathlineto{\pgfqpoint{5.207820in}{0.739656in}}%
\pgfpathlineto{\pgfqpoint{5.207524in}{0.739656in}}%
\pgfpathlineto{\pgfqpoint{5.207228in}{0.739656in}}%
\pgfpathlineto{\pgfqpoint{5.206932in}{0.739656in}}%
\pgfpathlineto{\pgfqpoint{5.206636in}{0.739656in}}%
\pgfpathlineto{\pgfqpoint{5.206340in}{0.739656in}}%
\pgfpathlineto{\pgfqpoint{5.206044in}{0.739656in}}%
\pgfpathlineto{\pgfqpoint{5.205748in}{0.739656in}}%
\pgfpathlineto{\pgfqpoint{5.205452in}{0.739656in}}%
\pgfpathlineto{\pgfqpoint{5.205156in}{0.739656in}}%
\pgfpathlineto{\pgfqpoint{5.204860in}{0.739656in}}%
\pgfpathlineto{\pgfqpoint{5.204564in}{0.739656in}}%
\pgfpathlineto{\pgfqpoint{5.204268in}{0.739656in}}%
\pgfpathlineto{\pgfqpoint{5.203972in}{0.739656in}}%
\pgfpathlineto{\pgfqpoint{5.203676in}{0.739656in}}%
\pgfpathlineto{\pgfqpoint{5.203380in}{0.739656in}}%
\pgfpathlineto{\pgfqpoint{5.203084in}{0.739656in}}%
\pgfpathlineto{\pgfqpoint{5.202788in}{0.739656in}}%
\pgfpathlineto{\pgfqpoint{5.202492in}{0.739656in}}%
\pgfpathlineto{\pgfqpoint{5.202196in}{0.739656in}}%
\pgfpathlineto{\pgfqpoint{5.201900in}{0.739656in}}%
\pgfpathlineto{\pgfqpoint{5.201604in}{0.739656in}}%
\pgfpathlineto{\pgfqpoint{5.201308in}{0.739656in}}%
\pgfpathlineto{\pgfqpoint{5.201012in}{0.739656in}}%
\pgfpathlineto{\pgfqpoint{5.200716in}{0.739656in}}%
\pgfpathlineto{\pgfqpoint{5.200420in}{0.739656in}}%
\pgfpathlineto{\pgfqpoint{5.200124in}{0.739656in}}%
\pgfpathlineto{\pgfqpoint{5.199828in}{0.739656in}}%
\pgfpathlineto{\pgfqpoint{5.199532in}{0.739656in}}%
\pgfpathlineto{\pgfqpoint{5.199236in}{0.739656in}}%
\pgfpathlineto{\pgfqpoint{5.198940in}{0.739656in}}%
\pgfpathlineto{\pgfqpoint{5.198644in}{0.739656in}}%
\pgfpathlineto{\pgfqpoint{5.198348in}{0.739656in}}%
\pgfpathlineto{\pgfqpoint{5.198052in}{0.739656in}}%
\pgfpathlineto{\pgfqpoint{5.197756in}{0.739656in}}%
\pgfpathlineto{\pgfqpoint{5.197460in}{0.739656in}}%
\pgfpathlineto{\pgfqpoint{5.197164in}{0.739656in}}%
\pgfpathlineto{\pgfqpoint{5.196868in}{0.739656in}}%
\pgfpathlineto{\pgfqpoint{5.196572in}{0.739656in}}%
\pgfpathlineto{\pgfqpoint{5.196276in}{0.739656in}}%
\pgfpathlineto{\pgfqpoint{5.195980in}{0.739656in}}%
\pgfpathlineto{\pgfqpoint{5.195684in}{0.739656in}}%
\pgfpathlineto{\pgfqpoint{5.195388in}{0.739656in}}%
\pgfpathlineto{\pgfqpoint{5.195092in}{0.739656in}}%
\pgfpathlineto{\pgfqpoint{5.194796in}{0.739656in}}%
\pgfpathlineto{\pgfqpoint{5.194500in}{0.739656in}}%
\pgfpathlineto{\pgfqpoint{5.194204in}{0.739656in}}%
\pgfpathlineto{\pgfqpoint{5.193908in}{0.739656in}}%
\pgfpathlineto{\pgfqpoint{5.193612in}{0.739656in}}%
\pgfpathlineto{\pgfqpoint{5.193316in}{0.739656in}}%
\pgfpathlineto{\pgfqpoint{5.193020in}{0.739656in}}%
\pgfpathlineto{\pgfqpoint{5.192724in}{0.739656in}}%
\pgfpathlineto{\pgfqpoint{5.192428in}{0.739656in}}%
\pgfpathlineto{\pgfqpoint{5.192132in}{0.739656in}}%
\pgfpathlineto{\pgfqpoint{5.191836in}{0.739656in}}%
\pgfpathlineto{\pgfqpoint{5.191540in}{0.739656in}}%
\pgfpathlineto{\pgfqpoint{5.191244in}{0.739656in}}%
\pgfpathlineto{\pgfqpoint{5.190948in}{0.739656in}}%
\pgfpathlineto{\pgfqpoint{5.190652in}{0.739656in}}%
\pgfpathlineto{\pgfqpoint{5.190356in}{0.739656in}}%
\pgfpathlineto{\pgfqpoint{5.190060in}{0.739656in}}%
\pgfpathlineto{\pgfqpoint{5.189764in}{0.739656in}}%
\pgfpathlineto{\pgfqpoint{5.189468in}{0.739656in}}%
\pgfpathlineto{\pgfqpoint{5.189172in}{0.739656in}}%
\pgfpathlineto{\pgfqpoint{5.188876in}{0.739656in}}%
\pgfpathlineto{\pgfqpoint{5.188580in}{0.739656in}}%
\pgfpathlineto{\pgfqpoint{5.188284in}{0.739656in}}%
\pgfpathlineto{\pgfqpoint{5.187988in}{0.739656in}}%
\pgfpathlineto{\pgfqpoint{5.187692in}{0.739656in}}%
\pgfpathlineto{\pgfqpoint{5.187396in}{0.739656in}}%
\pgfpathlineto{\pgfqpoint{5.187100in}{0.739656in}}%
\pgfpathlineto{\pgfqpoint{5.186804in}{0.739656in}}%
\pgfpathlineto{\pgfqpoint{5.186508in}{0.739656in}}%
\pgfpathlineto{\pgfqpoint{5.186212in}{0.739656in}}%
\pgfpathlineto{\pgfqpoint{5.185916in}{0.739656in}}%
\pgfpathlineto{\pgfqpoint{5.185620in}{0.739656in}}%
\pgfpathlineto{\pgfqpoint{5.185324in}{0.739656in}}%
\pgfpathlineto{\pgfqpoint{5.185028in}{0.739656in}}%
\pgfpathlineto{\pgfqpoint{5.184732in}{0.739656in}}%
\pgfpathlineto{\pgfqpoint{5.184436in}{0.739656in}}%
\pgfpathlineto{\pgfqpoint{5.184140in}{0.739656in}}%
\pgfpathlineto{\pgfqpoint{5.183844in}{0.739656in}}%
\pgfpathlineto{\pgfqpoint{5.183548in}{0.739656in}}%
\pgfpathlineto{\pgfqpoint{5.183252in}{0.739656in}}%
\pgfpathlineto{\pgfqpoint{5.182956in}{0.739656in}}%
\pgfpathlineto{\pgfqpoint{5.182660in}{0.739656in}}%
\pgfpathlineto{\pgfqpoint{5.182364in}{0.739656in}}%
\pgfpathlineto{\pgfqpoint{5.182068in}{0.739656in}}%
\pgfpathlineto{\pgfqpoint{5.181772in}{0.739656in}}%
\pgfpathlineto{\pgfqpoint{5.181476in}{0.739656in}}%
\pgfpathlineto{\pgfqpoint{5.181180in}{0.739656in}}%
\pgfpathlineto{\pgfqpoint{5.180884in}{0.739656in}}%
\pgfpathlineto{\pgfqpoint{5.180588in}{0.739656in}}%
\pgfpathlineto{\pgfqpoint{5.180292in}{0.739656in}}%
\pgfpathlineto{\pgfqpoint{5.179996in}{0.739656in}}%
\pgfpathlineto{\pgfqpoint{5.179700in}{0.739656in}}%
\pgfpathlineto{\pgfqpoint{5.179404in}{0.739656in}}%
\pgfpathlineto{\pgfqpoint{5.179108in}{0.739656in}}%
\pgfpathlineto{\pgfqpoint{5.178812in}{0.739656in}}%
\pgfpathlineto{\pgfqpoint{5.178516in}{0.739656in}}%
\pgfpathlineto{\pgfqpoint{5.178220in}{0.739656in}}%
\pgfpathlineto{\pgfqpoint{5.177924in}{0.739656in}}%
\pgfpathlineto{\pgfqpoint{5.177628in}{0.739656in}}%
\pgfpathlineto{\pgfqpoint{5.177332in}{0.739656in}}%
\pgfpathlineto{\pgfqpoint{5.177036in}{0.739656in}}%
\pgfpathlineto{\pgfqpoint{5.176740in}{0.739656in}}%
\pgfpathlineto{\pgfqpoint{5.176444in}{0.739656in}}%
\pgfpathlineto{\pgfqpoint{5.176148in}{0.739656in}}%
\pgfpathlineto{\pgfqpoint{5.175852in}{0.739656in}}%
\pgfpathlineto{\pgfqpoint{5.175556in}{0.739656in}}%
\pgfpathlineto{\pgfqpoint{5.175260in}{0.739656in}}%
\pgfpathlineto{\pgfqpoint{5.174964in}{0.739656in}}%
\pgfpathlineto{\pgfqpoint{5.174668in}{0.739656in}}%
\pgfpathlineto{\pgfqpoint{5.174372in}{0.739656in}}%
\pgfpathlineto{\pgfqpoint{5.174076in}{0.739656in}}%
\pgfpathlineto{\pgfqpoint{5.173780in}{0.739656in}}%
\pgfpathlineto{\pgfqpoint{5.173484in}{0.739656in}}%
\pgfpathlineto{\pgfqpoint{5.173188in}{0.739656in}}%
\pgfpathlineto{\pgfqpoint{5.172892in}{0.739656in}}%
\pgfpathlineto{\pgfqpoint{5.172596in}{0.739656in}}%
\pgfpathlineto{\pgfqpoint{5.172300in}{0.739656in}}%
\pgfpathlineto{\pgfqpoint{5.172004in}{0.739656in}}%
\pgfpathlineto{\pgfqpoint{5.171708in}{0.739656in}}%
\pgfpathlineto{\pgfqpoint{5.171412in}{0.739656in}}%
\pgfpathlineto{\pgfqpoint{5.171116in}{0.739656in}}%
\pgfpathlineto{\pgfqpoint{5.170820in}{0.739656in}}%
\pgfpathlineto{\pgfqpoint{5.170524in}{0.739656in}}%
\pgfpathlineto{\pgfqpoint{5.170228in}{0.739656in}}%
\pgfpathlineto{\pgfqpoint{5.169932in}{0.739656in}}%
\pgfpathlineto{\pgfqpoint{5.169636in}{0.739656in}}%
\pgfpathlineto{\pgfqpoint{5.169340in}{0.739656in}}%
\pgfpathlineto{\pgfqpoint{5.169044in}{0.739656in}}%
\pgfpathlineto{\pgfqpoint{5.168748in}{0.739656in}}%
\pgfpathlineto{\pgfqpoint{5.168452in}{0.739656in}}%
\pgfpathlineto{\pgfqpoint{5.168156in}{0.739656in}}%
\pgfpathlineto{\pgfqpoint{5.167860in}{0.739656in}}%
\pgfpathlineto{\pgfqpoint{5.167564in}{0.739656in}}%
\pgfpathlineto{\pgfqpoint{5.167268in}{0.739656in}}%
\pgfpathlineto{\pgfqpoint{5.166972in}{0.739656in}}%
\pgfpathlineto{\pgfqpoint{5.166676in}{0.739656in}}%
\pgfpathlineto{\pgfqpoint{5.166380in}{0.739656in}}%
\pgfpathlineto{\pgfqpoint{5.166084in}{0.739656in}}%
\pgfpathlineto{\pgfqpoint{5.165788in}{0.739656in}}%
\pgfpathlineto{\pgfqpoint{5.165492in}{0.739656in}}%
\pgfpathlineto{\pgfqpoint{5.165196in}{0.739656in}}%
\pgfpathlineto{\pgfqpoint{5.164900in}{0.739656in}}%
\pgfpathlineto{\pgfqpoint{5.164603in}{0.739656in}}%
\pgfpathlineto{\pgfqpoint{5.164307in}{0.739656in}}%
\pgfpathlineto{\pgfqpoint{5.164011in}{0.739656in}}%
\pgfpathlineto{\pgfqpoint{5.163715in}{0.739656in}}%
\pgfpathlineto{\pgfqpoint{5.163419in}{0.739656in}}%
\pgfpathlineto{\pgfqpoint{5.163123in}{0.739656in}}%
\pgfpathlineto{\pgfqpoint{5.162827in}{0.739656in}}%
\pgfpathlineto{\pgfqpoint{5.162531in}{0.739656in}}%
\pgfpathlineto{\pgfqpoint{5.162235in}{0.739656in}}%
\pgfpathlineto{\pgfqpoint{5.161939in}{0.739656in}}%
\pgfpathlineto{\pgfqpoint{5.161643in}{0.739656in}}%
\pgfpathlineto{\pgfqpoint{5.161347in}{0.739656in}}%
\pgfpathlineto{\pgfqpoint{5.161051in}{0.739656in}}%
\pgfpathlineto{\pgfqpoint{5.160755in}{0.739656in}}%
\pgfpathlineto{\pgfqpoint{5.160459in}{0.739656in}}%
\pgfpathlineto{\pgfqpoint{5.160163in}{0.739656in}}%
\pgfpathlineto{\pgfqpoint{5.159867in}{0.739656in}}%
\pgfpathlineto{\pgfqpoint{5.159571in}{0.739656in}}%
\pgfpathlineto{\pgfqpoint{5.159275in}{0.739656in}}%
\pgfpathlineto{\pgfqpoint{5.158979in}{0.739656in}}%
\pgfpathlineto{\pgfqpoint{5.158683in}{0.739656in}}%
\pgfpathlineto{\pgfqpoint{5.158387in}{0.739656in}}%
\pgfpathlineto{\pgfqpoint{5.158091in}{0.739656in}}%
\pgfpathlineto{\pgfqpoint{5.157795in}{0.739656in}}%
\pgfpathlineto{\pgfqpoint{5.157499in}{0.739656in}}%
\pgfpathlineto{\pgfqpoint{5.157203in}{0.739656in}}%
\pgfpathlineto{\pgfqpoint{5.156907in}{0.739656in}}%
\pgfpathlineto{\pgfqpoint{5.156611in}{0.739656in}}%
\pgfpathlineto{\pgfqpoint{5.156315in}{0.739656in}}%
\pgfpathlineto{\pgfqpoint{5.156019in}{0.739656in}}%
\pgfpathlineto{\pgfqpoint{5.155723in}{0.739656in}}%
\pgfpathlineto{\pgfqpoint{5.155427in}{0.739656in}}%
\pgfpathlineto{\pgfqpoint{5.155131in}{0.739656in}}%
\pgfpathlineto{\pgfqpoint{5.154835in}{0.739656in}}%
\pgfpathlineto{\pgfqpoint{5.154539in}{0.739656in}}%
\pgfpathlineto{\pgfqpoint{5.154243in}{0.739656in}}%
\pgfpathlineto{\pgfqpoint{5.153947in}{0.739656in}}%
\pgfpathlineto{\pgfqpoint{5.153651in}{0.739656in}}%
\pgfpathlineto{\pgfqpoint{5.153355in}{0.739656in}}%
\pgfpathlineto{\pgfqpoint{5.153059in}{0.739656in}}%
\pgfpathlineto{\pgfqpoint{5.152763in}{0.739656in}}%
\pgfpathlineto{\pgfqpoint{5.152467in}{0.739656in}}%
\pgfpathlineto{\pgfqpoint{5.152171in}{0.739656in}}%
\pgfpathlineto{\pgfqpoint{5.151875in}{0.739656in}}%
\pgfpathlineto{\pgfqpoint{5.151579in}{0.739656in}}%
\pgfpathlineto{\pgfqpoint{5.151283in}{0.739656in}}%
\pgfpathlineto{\pgfqpoint{5.150987in}{0.739656in}}%
\pgfpathlineto{\pgfqpoint{5.150691in}{0.739656in}}%
\pgfpathlineto{\pgfqpoint{5.150395in}{0.739656in}}%
\pgfpathlineto{\pgfqpoint{5.150099in}{0.739656in}}%
\pgfpathlineto{\pgfqpoint{5.149803in}{0.739656in}}%
\pgfpathlineto{\pgfqpoint{5.149507in}{0.739656in}}%
\pgfpathlineto{\pgfqpoint{5.149211in}{0.739656in}}%
\pgfpathlineto{\pgfqpoint{5.148915in}{0.739656in}}%
\pgfpathlineto{\pgfqpoint{5.148619in}{0.739656in}}%
\pgfpathlineto{\pgfqpoint{5.148323in}{0.739656in}}%
\pgfpathlineto{\pgfqpoint{5.148027in}{0.739656in}}%
\pgfpathlineto{\pgfqpoint{5.147731in}{0.739656in}}%
\pgfpathlineto{\pgfqpoint{5.147435in}{0.739656in}}%
\pgfpathlineto{\pgfqpoint{5.147139in}{0.739656in}}%
\pgfpathlineto{\pgfqpoint{5.146843in}{0.739656in}}%
\pgfpathlineto{\pgfqpoint{5.146547in}{0.739656in}}%
\pgfpathlineto{\pgfqpoint{5.146251in}{0.739656in}}%
\pgfpathlineto{\pgfqpoint{5.145955in}{0.739656in}}%
\pgfpathlineto{\pgfqpoint{5.145659in}{0.739656in}}%
\pgfpathlineto{\pgfqpoint{5.145363in}{0.739656in}}%
\pgfpathlineto{\pgfqpoint{5.145067in}{0.739656in}}%
\pgfpathlineto{\pgfqpoint{5.144771in}{0.739656in}}%
\pgfpathlineto{\pgfqpoint{5.144475in}{0.739656in}}%
\pgfpathlineto{\pgfqpoint{5.144179in}{0.739656in}}%
\pgfpathlineto{\pgfqpoint{5.143883in}{0.739656in}}%
\pgfpathlineto{\pgfqpoint{5.143587in}{0.739656in}}%
\pgfpathlineto{\pgfqpoint{5.143291in}{0.739656in}}%
\pgfpathlineto{\pgfqpoint{5.142995in}{0.739656in}}%
\pgfpathlineto{\pgfqpoint{5.142699in}{0.739656in}}%
\pgfpathlineto{\pgfqpoint{5.142403in}{0.739656in}}%
\pgfpathlineto{\pgfqpoint{5.142107in}{0.739656in}}%
\pgfpathlineto{\pgfqpoint{5.141811in}{0.739656in}}%
\pgfpathlineto{\pgfqpoint{5.141515in}{0.739656in}}%
\pgfpathlineto{\pgfqpoint{5.141219in}{0.739656in}}%
\pgfpathlineto{\pgfqpoint{5.140923in}{0.739656in}}%
\pgfpathlineto{\pgfqpoint{5.140627in}{0.739656in}}%
\pgfpathlineto{\pgfqpoint{5.140331in}{0.739656in}}%
\pgfpathlineto{\pgfqpoint{5.140035in}{0.739656in}}%
\pgfpathlineto{\pgfqpoint{5.139739in}{0.739656in}}%
\pgfpathlineto{\pgfqpoint{5.139443in}{0.739656in}}%
\pgfpathlineto{\pgfqpoint{5.139147in}{0.739656in}}%
\pgfpathlineto{\pgfqpoint{5.138851in}{0.739656in}}%
\pgfpathlineto{\pgfqpoint{5.138555in}{0.739656in}}%
\pgfpathlineto{\pgfqpoint{5.138259in}{0.739656in}}%
\pgfpathlineto{\pgfqpoint{5.137963in}{0.739656in}}%
\pgfpathlineto{\pgfqpoint{5.137667in}{0.739656in}}%
\pgfpathlineto{\pgfqpoint{5.137371in}{0.739656in}}%
\pgfpathlineto{\pgfqpoint{5.137075in}{0.739656in}}%
\pgfpathlineto{\pgfqpoint{5.136779in}{0.739656in}}%
\pgfpathlineto{\pgfqpoint{5.136483in}{0.739656in}}%
\pgfpathlineto{\pgfqpoint{5.136187in}{0.739656in}}%
\pgfpathlineto{\pgfqpoint{5.135891in}{0.739656in}}%
\pgfpathlineto{\pgfqpoint{5.135595in}{0.739656in}}%
\pgfpathlineto{\pgfqpoint{5.135299in}{0.739656in}}%
\pgfpathlineto{\pgfqpoint{5.135003in}{0.739656in}}%
\pgfpathlineto{\pgfqpoint{5.134707in}{0.739656in}}%
\pgfpathlineto{\pgfqpoint{5.134411in}{0.739656in}}%
\pgfpathlineto{\pgfqpoint{5.134115in}{0.739656in}}%
\pgfpathlineto{\pgfqpoint{5.133819in}{0.739656in}}%
\pgfpathlineto{\pgfqpoint{5.133523in}{0.739656in}}%
\pgfpathlineto{\pgfqpoint{5.133227in}{0.739656in}}%
\pgfpathlineto{\pgfqpoint{5.132931in}{0.739656in}}%
\pgfpathlineto{\pgfqpoint{5.132635in}{0.739656in}}%
\pgfpathlineto{\pgfqpoint{5.132339in}{0.739656in}}%
\pgfpathlineto{\pgfqpoint{5.132043in}{0.739656in}}%
\pgfpathlineto{\pgfqpoint{5.131747in}{0.739656in}}%
\pgfpathlineto{\pgfqpoint{5.131451in}{0.739656in}}%
\pgfpathlineto{\pgfqpoint{5.131155in}{0.739656in}}%
\pgfpathlineto{\pgfqpoint{5.130859in}{0.739656in}}%
\pgfpathlineto{\pgfqpoint{5.130563in}{0.739656in}}%
\pgfpathlineto{\pgfqpoint{5.130267in}{0.739656in}}%
\pgfpathlineto{\pgfqpoint{5.129971in}{0.739656in}}%
\pgfpathlineto{\pgfqpoint{5.129675in}{0.739656in}}%
\pgfpathlineto{\pgfqpoint{5.129379in}{0.739656in}}%
\pgfpathlineto{\pgfqpoint{5.129083in}{0.739656in}}%
\pgfpathlineto{\pgfqpoint{5.128787in}{0.739656in}}%
\pgfpathlineto{\pgfqpoint{5.128491in}{0.739656in}}%
\pgfpathlineto{\pgfqpoint{5.128195in}{0.739656in}}%
\pgfpathlineto{\pgfqpoint{5.127899in}{0.739656in}}%
\pgfpathlineto{\pgfqpoint{5.127603in}{0.739656in}}%
\pgfpathlineto{\pgfqpoint{5.127307in}{0.739656in}}%
\pgfpathlineto{\pgfqpoint{5.127011in}{0.739656in}}%
\pgfpathlineto{\pgfqpoint{5.126715in}{0.739656in}}%
\pgfpathlineto{\pgfqpoint{5.126419in}{0.739656in}}%
\pgfpathlineto{\pgfqpoint{5.126123in}{0.739656in}}%
\pgfpathlineto{\pgfqpoint{5.125827in}{0.739656in}}%
\pgfpathlineto{\pgfqpoint{5.125531in}{0.739656in}}%
\pgfpathlineto{\pgfqpoint{5.125235in}{0.739656in}}%
\pgfpathlineto{\pgfqpoint{5.124939in}{0.739656in}}%
\pgfpathlineto{\pgfqpoint{5.124643in}{0.739656in}}%
\pgfpathlineto{\pgfqpoint{5.124347in}{0.739656in}}%
\pgfpathlineto{\pgfqpoint{5.124051in}{0.739656in}}%
\pgfpathlineto{\pgfqpoint{5.123755in}{0.739656in}}%
\pgfpathlineto{\pgfqpoint{5.123459in}{0.739656in}}%
\pgfpathlineto{\pgfqpoint{5.123163in}{0.739656in}}%
\pgfpathlineto{\pgfqpoint{5.122867in}{0.739656in}}%
\pgfpathlineto{\pgfqpoint{5.122571in}{0.739656in}}%
\pgfpathlineto{\pgfqpoint{5.122275in}{0.739656in}}%
\pgfpathlineto{\pgfqpoint{5.121979in}{0.739656in}}%
\pgfpathlineto{\pgfqpoint{5.121683in}{0.739656in}}%
\pgfpathlineto{\pgfqpoint{5.121387in}{0.739656in}}%
\pgfpathlineto{\pgfqpoint{5.121091in}{0.739656in}}%
\pgfpathlineto{\pgfqpoint{5.120795in}{0.739656in}}%
\pgfpathlineto{\pgfqpoint{5.120499in}{0.739656in}}%
\pgfpathlineto{\pgfqpoint{5.120203in}{0.739656in}}%
\pgfpathlineto{\pgfqpoint{5.119907in}{0.739656in}}%
\pgfpathlineto{\pgfqpoint{5.119611in}{0.739656in}}%
\pgfpathlineto{\pgfqpoint{5.119315in}{0.739656in}}%
\pgfpathlineto{\pgfqpoint{5.119019in}{0.739656in}}%
\pgfpathlineto{\pgfqpoint{5.118723in}{0.739656in}}%
\pgfpathlineto{\pgfqpoint{5.118427in}{0.739656in}}%
\pgfpathlineto{\pgfqpoint{5.118131in}{0.739656in}}%
\pgfpathlineto{\pgfqpoint{5.117835in}{0.739656in}}%
\pgfpathlineto{\pgfqpoint{5.117539in}{0.739656in}}%
\pgfpathlineto{\pgfqpoint{5.117243in}{0.739656in}}%
\pgfpathlineto{\pgfqpoint{5.116947in}{0.739656in}}%
\pgfpathlineto{\pgfqpoint{5.116651in}{0.739656in}}%
\pgfpathlineto{\pgfqpoint{5.116355in}{0.739656in}}%
\pgfpathlineto{\pgfqpoint{5.116059in}{0.739656in}}%
\pgfpathlineto{\pgfqpoint{5.115763in}{0.739656in}}%
\pgfpathlineto{\pgfqpoint{5.115467in}{0.739656in}}%
\pgfpathlineto{\pgfqpoint{5.115171in}{0.739656in}}%
\pgfpathlineto{\pgfqpoint{5.114875in}{0.739656in}}%
\pgfpathlineto{\pgfqpoint{5.114579in}{0.739656in}}%
\pgfpathlineto{\pgfqpoint{5.114283in}{0.739656in}}%
\pgfpathlineto{\pgfqpoint{5.113987in}{0.739656in}}%
\pgfpathlineto{\pgfqpoint{5.113691in}{0.739656in}}%
\pgfpathlineto{\pgfqpoint{5.113395in}{0.739656in}}%
\pgfpathlineto{\pgfqpoint{5.113099in}{0.739656in}}%
\pgfpathlineto{\pgfqpoint{5.112803in}{0.739656in}}%
\pgfpathlineto{\pgfqpoint{5.112507in}{0.739656in}}%
\pgfpathlineto{\pgfqpoint{5.112211in}{0.739656in}}%
\pgfpathlineto{\pgfqpoint{5.111915in}{0.739656in}}%
\pgfpathlineto{\pgfqpoint{5.111619in}{0.739656in}}%
\pgfpathlineto{\pgfqpoint{5.111323in}{0.739656in}}%
\pgfpathlineto{\pgfqpoint{5.111027in}{0.739656in}}%
\pgfpathlineto{\pgfqpoint{5.110731in}{0.739656in}}%
\pgfpathlineto{\pgfqpoint{5.110435in}{0.739656in}}%
\pgfpathlineto{\pgfqpoint{5.110139in}{0.739656in}}%
\pgfpathlineto{\pgfqpoint{5.109843in}{0.739656in}}%
\pgfpathlineto{\pgfqpoint{5.109547in}{0.739656in}}%
\pgfpathlineto{\pgfqpoint{5.109251in}{0.739656in}}%
\pgfpathlineto{\pgfqpoint{5.108955in}{0.739656in}}%
\pgfpathlineto{\pgfqpoint{5.108659in}{0.739656in}}%
\pgfpathlineto{\pgfqpoint{5.108363in}{0.739656in}}%
\pgfpathlineto{\pgfqpoint{5.108067in}{0.739656in}}%
\pgfpathlineto{\pgfqpoint{5.107771in}{0.739656in}}%
\pgfpathlineto{\pgfqpoint{5.107475in}{0.739656in}}%
\pgfpathlineto{\pgfqpoint{5.107179in}{0.739656in}}%
\pgfpathlineto{\pgfqpoint{5.106883in}{0.739656in}}%
\pgfpathlineto{\pgfqpoint{5.106587in}{0.739656in}}%
\pgfpathlineto{\pgfqpoint{5.106291in}{0.739656in}}%
\pgfpathlineto{\pgfqpoint{5.105995in}{0.739656in}}%
\pgfpathlineto{\pgfqpoint{5.105699in}{0.739656in}}%
\pgfpathlineto{\pgfqpoint{5.105403in}{0.739656in}}%
\pgfpathlineto{\pgfqpoint{5.105107in}{0.739656in}}%
\pgfpathlineto{\pgfqpoint{5.104811in}{0.739656in}}%
\pgfpathlineto{\pgfqpoint{5.104515in}{0.739656in}}%
\pgfpathlineto{\pgfqpoint{5.104219in}{0.739656in}}%
\pgfpathlineto{\pgfqpoint{5.103923in}{0.739656in}}%
\pgfpathlineto{\pgfqpoint{5.103627in}{0.739656in}}%
\pgfpathlineto{\pgfqpoint{5.103331in}{0.739656in}}%
\pgfpathlineto{\pgfqpoint{5.103035in}{0.739656in}}%
\pgfpathlineto{\pgfqpoint{5.102739in}{0.739656in}}%
\pgfpathlineto{\pgfqpoint{5.102443in}{0.739656in}}%
\pgfpathlineto{\pgfqpoint{5.102147in}{0.739656in}}%
\pgfpathlineto{\pgfqpoint{5.101851in}{0.739656in}}%
\pgfpathlineto{\pgfqpoint{5.101555in}{0.739656in}}%
\pgfpathlineto{\pgfqpoint{5.101259in}{0.739656in}}%
\pgfpathlineto{\pgfqpoint{5.100963in}{0.739656in}}%
\pgfpathlineto{\pgfqpoint{5.100667in}{0.739656in}}%
\pgfpathlineto{\pgfqpoint{5.100371in}{0.739656in}}%
\pgfpathlineto{\pgfqpoint{5.100075in}{0.739656in}}%
\pgfpathlineto{\pgfqpoint{5.099779in}{0.739656in}}%
\pgfpathlineto{\pgfqpoint{5.099483in}{0.739656in}}%
\pgfpathlineto{\pgfqpoint{5.099187in}{0.739656in}}%
\pgfpathlineto{\pgfqpoint{5.098891in}{0.739656in}}%
\pgfpathlineto{\pgfqpoint{5.098595in}{0.739656in}}%
\pgfpathlineto{\pgfqpoint{5.098299in}{0.739656in}}%
\pgfpathlineto{\pgfqpoint{5.098003in}{0.739656in}}%
\pgfpathlineto{\pgfqpoint{5.097707in}{0.739656in}}%
\pgfpathlineto{\pgfqpoint{5.097411in}{0.739656in}}%
\pgfpathlineto{\pgfqpoint{5.097114in}{0.739656in}}%
\pgfpathlineto{\pgfqpoint{5.096818in}{0.739656in}}%
\pgfpathlineto{\pgfqpoint{5.096522in}{0.739656in}}%
\pgfpathlineto{\pgfqpoint{5.096226in}{0.739656in}}%
\pgfpathlineto{\pgfqpoint{5.095930in}{0.739656in}}%
\pgfpathlineto{\pgfqpoint{5.095634in}{0.739656in}}%
\pgfpathlineto{\pgfqpoint{5.095338in}{0.739656in}}%
\pgfpathlineto{\pgfqpoint{5.095042in}{0.739656in}}%
\pgfpathlineto{\pgfqpoint{5.094746in}{0.739656in}}%
\pgfpathlineto{\pgfqpoint{5.094450in}{0.739656in}}%
\pgfpathlineto{\pgfqpoint{5.094154in}{0.739656in}}%
\pgfpathlineto{\pgfqpoint{5.093858in}{0.739656in}}%
\pgfpathlineto{\pgfqpoint{5.093562in}{0.739656in}}%
\pgfpathlineto{\pgfqpoint{5.093266in}{0.739656in}}%
\pgfpathlineto{\pgfqpoint{5.092970in}{0.739656in}}%
\pgfpathlineto{\pgfqpoint{5.092674in}{0.739656in}}%
\pgfpathlineto{\pgfqpoint{5.092378in}{0.739656in}}%
\pgfpathlineto{\pgfqpoint{5.092082in}{0.739656in}}%
\pgfpathlineto{\pgfqpoint{5.091786in}{0.739656in}}%
\pgfpathlineto{\pgfqpoint{5.091490in}{0.739656in}}%
\pgfpathlineto{\pgfqpoint{5.091194in}{0.739656in}}%
\pgfpathlineto{\pgfqpoint{5.090898in}{0.739656in}}%
\pgfpathlineto{\pgfqpoint{5.090602in}{0.739656in}}%
\pgfpathlineto{\pgfqpoint{5.090306in}{0.739656in}}%
\pgfpathlineto{\pgfqpoint{5.090010in}{0.739656in}}%
\pgfpathlineto{\pgfqpoint{5.089714in}{0.739656in}}%
\pgfpathlineto{\pgfqpoint{5.089418in}{0.739656in}}%
\pgfpathlineto{\pgfqpoint{5.089122in}{0.739656in}}%
\pgfpathlineto{\pgfqpoint{5.088826in}{0.739656in}}%
\pgfpathlineto{\pgfqpoint{5.088530in}{0.739656in}}%
\pgfpathlineto{\pgfqpoint{5.088234in}{0.739656in}}%
\pgfpathlineto{\pgfqpoint{5.087938in}{0.739656in}}%
\pgfpathlineto{\pgfqpoint{5.087642in}{0.739656in}}%
\pgfpathlineto{\pgfqpoint{5.087346in}{0.739656in}}%
\pgfpathlineto{\pgfqpoint{5.087050in}{0.739656in}}%
\pgfpathlineto{\pgfqpoint{5.086754in}{0.739656in}}%
\pgfpathlineto{\pgfqpoint{5.086458in}{0.739656in}}%
\pgfpathlineto{\pgfqpoint{5.086162in}{0.739656in}}%
\pgfpathlineto{\pgfqpoint{5.085866in}{0.739656in}}%
\pgfpathlineto{\pgfqpoint{5.085570in}{0.739656in}}%
\pgfpathlineto{\pgfqpoint{5.085274in}{0.739656in}}%
\pgfpathlineto{\pgfqpoint{5.084978in}{0.739656in}}%
\pgfpathlineto{\pgfqpoint{5.084682in}{0.739656in}}%
\pgfpathlineto{\pgfqpoint{5.084386in}{0.739656in}}%
\pgfpathlineto{\pgfqpoint{5.084090in}{0.739656in}}%
\pgfpathlineto{\pgfqpoint{5.083794in}{0.739656in}}%
\pgfpathlineto{\pgfqpoint{5.083498in}{0.739656in}}%
\pgfpathlineto{\pgfqpoint{5.083202in}{0.739656in}}%
\pgfpathlineto{\pgfqpoint{5.082906in}{0.739656in}}%
\pgfpathlineto{\pgfqpoint{5.082610in}{0.739656in}}%
\pgfpathlineto{\pgfqpoint{5.082314in}{0.739656in}}%
\pgfpathlineto{\pgfqpoint{5.082018in}{0.739656in}}%
\pgfpathlineto{\pgfqpoint{5.081722in}{0.739656in}}%
\pgfpathlineto{\pgfqpoint{5.081426in}{0.739656in}}%
\pgfpathlineto{\pgfqpoint{5.081130in}{0.739656in}}%
\pgfpathlineto{\pgfqpoint{5.080834in}{0.739656in}}%
\pgfpathlineto{\pgfqpoint{5.080538in}{0.739656in}}%
\pgfpathlineto{\pgfqpoint{5.080242in}{0.739656in}}%
\pgfpathlineto{\pgfqpoint{5.079946in}{0.739656in}}%
\pgfpathlineto{\pgfqpoint{5.079650in}{0.739656in}}%
\pgfpathlineto{\pgfqpoint{5.079354in}{0.739656in}}%
\pgfpathlineto{\pgfqpoint{5.079058in}{0.739656in}}%
\pgfpathlineto{\pgfqpoint{5.078762in}{0.739656in}}%
\pgfpathlineto{\pgfqpoint{5.078466in}{0.739656in}}%
\pgfpathlineto{\pgfqpoint{5.078170in}{0.739656in}}%
\pgfpathlineto{\pgfqpoint{5.077874in}{0.739656in}}%
\pgfpathlineto{\pgfqpoint{5.077578in}{0.739656in}}%
\pgfpathlineto{\pgfqpoint{5.077282in}{0.739656in}}%
\pgfpathlineto{\pgfqpoint{5.076986in}{0.739656in}}%
\pgfpathlineto{\pgfqpoint{5.076690in}{0.739656in}}%
\pgfpathlineto{\pgfqpoint{5.076394in}{0.739656in}}%
\pgfpathlineto{\pgfqpoint{5.076098in}{0.739656in}}%
\pgfpathlineto{\pgfqpoint{5.075802in}{0.739656in}}%
\pgfpathlineto{\pgfqpoint{5.075506in}{0.739656in}}%
\pgfpathlineto{\pgfqpoint{5.075210in}{0.739656in}}%
\pgfpathlineto{\pgfqpoint{5.074914in}{0.739656in}}%
\pgfpathlineto{\pgfqpoint{5.074618in}{0.739656in}}%
\pgfpathlineto{\pgfqpoint{5.074322in}{0.739656in}}%
\pgfpathlineto{\pgfqpoint{5.074026in}{0.739656in}}%
\pgfpathlineto{\pgfqpoint{5.073730in}{0.739656in}}%
\pgfpathlineto{\pgfqpoint{5.073434in}{0.739656in}}%
\pgfpathlineto{\pgfqpoint{5.073138in}{0.739656in}}%
\pgfpathlineto{\pgfqpoint{5.072842in}{0.739656in}}%
\pgfpathlineto{\pgfqpoint{5.072546in}{0.739656in}}%
\pgfpathlineto{\pgfqpoint{5.072250in}{0.739656in}}%
\pgfpathlineto{\pgfqpoint{5.071954in}{0.739656in}}%
\pgfpathlineto{\pgfqpoint{5.071658in}{0.739656in}}%
\pgfpathlineto{\pgfqpoint{5.071362in}{0.739656in}}%
\pgfpathlineto{\pgfqpoint{5.071066in}{0.739656in}}%
\pgfpathlineto{\pgfqpoint{5.070770in}{0.739656in}}%
\pgfpathlineto{\pgfqpoint{5.070474in}{0.739656in}}%
\pgfpathlineto{\pgfqpoint{5.070178in}{0.739656in}}%
\pgfpathlineto{\pgfqpoint{5.069882in}{0.739656in}}%
\pgfpathlineto{\pgfqpoint{5.069586in}{0.739656in}}%
\pgfpathlineto{\pgfqpoint{5.069290in}{0.739656in}}%
\pgfpathlineto{\pgfqpoint{5.068994in}{0.739656in}}%
\pgfpathlineto{\pgfqpoint{5.068698in}{0.739656in}}%
\pgfpathlineto{\pgfqpoint{5.068402in}{0.739656in}}%
\pgfpathlineto{\pgfqpoint{5.068106in}{0.739656in}}%
\pgfpathlineto{\pgfqpoint{5.067810in}{0.739656in}}%
\pgfpathlineto{\pgfqpoint{5.067514in}{0.739656in}}%
\pgfpathlineto{\pgfqpoint{5.067218in}{0.739656in}}%
\pgfpathlineto{\pgfqpoint{5.066922in}{0.739656in}}%
\pgfpathlineto{\pgfqpoint{5.066626in}{0.739656in}}%
\pgfpathlineto{\pgfqpoint{5.066330in}{0.739656in}}%
\pgfpathlineto{\pgfqpoint{5.066034in}{0.739656in}}%
\pgfpathlineto{\pgfqpoint{5.065738in}{0.739656in}}%
\pgfpathlineto{\pgfqpoint{5.065442in}{0.739656in}}%
\pgfpathlineto{\pgfqpoint{5.065146in}{0.739656in}}%
\pgfpathlineto{\pgfqpoint{5.064850in}{0.739656in}}%
\pgfpathlineto{\pgfqpoint{5.064554in}{0.739656in}}%
\pgfpathlineto{\pgfqpoint{5.064258in}{0.739656in}}%
\pgfpathlineto{\pgfqpoint{5.063962in}{0.739656in}}%
\pgfpathlineto{\pgfqpoint{5.063666in}{0.739656in}}%
\pgfpathlineto{\pgfqpoint{5.063370in}{0.739656in}}%
\pgfpathlineto{\pgfqpoint{5.063074in}{0.739656in}}%
\pgfpathlineto{\pgfqpoint{5.062778in}{0.739656in}}%
\pgfpathlineto{\pgfqpoint{5.062482in}{0.739656in}}%
\pgfpathlineto{\pgfqpoint{5.062186in}{0.739656in}}%
\pgfpathlineto{\pgfqpoint{5.061890in}{0.739656in}}%
\pgfpathlineto{\pgfqpoint{5.061594in}{0.739656in}}%
\pgfpathlineto{\pgfqpoint{5.061298in}{0.739656in}}%
\pgfpathlineto{\pgfqpoint{5.061002in}{0.739656in}}%
\pgfpathlineto{\pgfqpoint{5.060706in}{0.739656in}}%
\pgfpathlineto{\pgfqpoint{5.060410in}{0.739656in}}%
\pgfpathlineto{\pgfqpoint{5.060114in}{0.739656in}}%
\pgfpathlineto{\pgfqpoint{5.059818in}{0.739656in}}%
\pgfpathlineto{\pgfqpoint{5.059522in}{0.739656in}}%
\pgfpathlineto{\pgfqpoint{5.059226in}{0.739656in}}%
\pgfpathlineto{\pgfqpoint{5.058930in}{0.739656in}}%
\pgfpathlineto{\pgfqpoint{5.058634in}{0.739656in}}%
\pgfpathlineto{\pgfqpoint{5.058338in}{0.739656in}}%
\pgfpathlineto{\pgfqpoint{5.058042in}{0.739656in}}%
\pgfpathlineto{\pgfqpoint{5.057746in}{0.739656in}}%
\pgfpathlineto{\pgfqpoint{5.057450in}{0.739656in}}%
\pgfpathlineto{\pgfqpoint{5.057154in}{0.739656in}}%
\pgfpathlineto{\pgfqpoint{5.056858in}{0.739656in}}%
\pgfpathlineto{\pgfqpoint{5.056562in}{0.739656in}}%
\pgfpathlineto{\pgfqpoint{5.056266in}{0.739656in}}%
\pgfpathlineto{\pgfqpoint{5.055970in}{0.739656in}}%
\pgfpathlineto{\pgfqpoint{5.055674in}{0.739656in}}%
\pgfpathlineto{\pgfqpoint{5.055378in}{0.739656in}}%
\pgfpathlineto{\pgfqpoint{5.055082in}{0.739656in}}%
\pgfpathlineto{\pgfqpoint{5.054786in}{0.739656in}}%
\pgfpathlineto{\pgfqpoint{5.054490in}{0.739656in}}%
\pgfpathlineto{\pgfqpoint{5.054194in}{0.739656in}}%
\pgfpathlineto{\pgfqpoint{5.053898in}{0.739656in}}%
\pgfpathlineto{\pgfqpoint{5.053602in}{0.739656in}}%
\pgfpathlineto{\pgfqpoint{5.053306in}{0.739656in}}%
\pgfpathlineto{\pgfqpoint{5.053010in}{0.739656in}}%
\pgfpathlineto{\pgfqpoint{5.052714in}{0.739656in}}%
\pgfpathlineto{\pgfqpoint{5.052418in}{0.739656in}}%
\pgfpathlineto{\pgfqpoint{5.052122in}{0.739656in}}%
\pgfpathlineto{\pgfqpoint{5.051826in}{0.739656in}}%
\pgfpathlineto{\pgfqpoint{5.051530in}{0.739656in}}%
\pgfpathlineto{\pgfqpoint{5.051234in}{0.739656in}}%
\pgfpathlineto{\pgfqpoint{5.050938in}{0.739656in}}%
\pgfpathlineto{\pgfqpoint{5.050642in}{0.739656in}}%
\pgfpathlineto{\pgfqpoint{5.050346in}{0.739656in}}%
\pgfpathlineto{\pgfqpoint{5.050050in}{0.739656in}}%
\pgfpathlineto{\pgfqpoint{5.049754in}{0.739656in}}%
\pgfpathlineto{\pgfqpoint{5.049458in}{0.739656in}}%
\pgfpathlineto{\pgfqpoint{5.049162in}{0.739656in}}%
\pgfpathlineto{\pgfqpoint{5.048866in}{0.739656in}}%
\pgfpathlineto{\pgfqpoint{5.048570in}{0.739656in}}%
\pgfpathlineto{\pgfqpoint{5.048274in}{0.739656in}}%
\pgfpathlineto{\pgfqpoint{5.047978in}{0.739656in}}%
\pgfpathlineto{\pgfqpoint{5.047682in}{0.739656in}}%
\pgfpathlineto{\pgfqpoint{5.047386in}{0.739656in}}%
\pgfpathlineto{\pgfqpoint{5.047090in}{0.739656in}}%
\pgfpathlineto{\pgfqpoint{5.046794in}{0.739656in}}%
\pgfpathlineto{\pgfqpoint{5.046498in}{0.739656in}}%
\pgfpathlineto{\pgfqpoint{5.046202in}{0.739656in}}%
\pgfpathlineto{\pgfqpoint{5.045906in}{0.739656in}}%
\pgfpathlineto{\pgfqpoint{5.045610in}{0.739656in}}%
\pgfpathlineto{\pgfqpoint{5.045314in}{0.739656in}}%
\pgfpathlineto{\pgfqpoint{5.045018in}{0.739656in}}%
\pgfpathlineto{\pgfqpoint{5.044722in}{0.739656in}}%
\pgfpathlineto{\pgfqpoint{5.044426in}{0.739656in}}%
\pgfpathlineto{\pgfqpoint{5.044130in}{0.739656in}}%
\pgfpathlineto{\pgfqpoint{5.043834in}{0.739656in}}%
\pgfpathlineto{\pgfqpoint{5.043538in}{0.739656in}}%
\pgfpathlineto{\pgfqpoint{5.043242in}{0.739656in}}%
\pgfpathlineto{\pgfqpoint{5.042946in}{0.739656in}}%
\pgfpathlineto{\pgfqpoint{5.042650in}{0.739656in}}%
\pgfpathlineto{\pgfqpoint{5.042354in}{0.739656in}}%
\pgfpathlineto{\pgfqpoint{5.042058in}{0.739656in}}%
\pgfpathlineto{\pgfqpoint{5.041762in}{0.739656in}}%
\pgfpathlineto{\pgfqpoint{5.041466in}{0.739656in}}%
\pgfpathlineto{\pgfqpoint{5.041170in}{0.739656in}}%
\pgfpathlineto{\pgfqpoint{5.040874in}{0.739656in}}%
\pgfpathlineto{\pgfqpoint{5.040578in}{0.739656in}}%
\pgfpathlineto{\pgfqpoint{5.040282in}{0.739656in}}%
\pgfpathlineto{\pgfqpoint{5.039986in}{0.739656in}}%
\pgfpathlineto{\pgfqpoint{5.039690in}{0.739656in}}%
\pgfpathlineto{\pgfqpoint{5.039394in}{0.739656in}}%
\pgfpathlineto{\pgfqpoint{5.039098in}{0.739656in}}%
\pgfpathlineto{\pgfqpoint{5.038802in}{0.739656in}}%
\pgfpathlineto{\pgfqpoint{5.038506in}{0.739656in}}%
\pgfpathlineto{\pgfqpoint{5.038210in}{0.739656in}}%
\pgfpathlineto{\pgfqpoint{5.037914in}{0.739656in}}%
\pgfpathlineto{\pgfqpoint{5.037618in}{0.739656in}}%
\pgfpathlineto{\pgfqpoint{5.037322in}{0.739656in}}%
\pgfpathlineto{\pgfqpoint{5.037026in}{0.739656in}}%
\pgfpathlineto{\pgfqpoint{5.036730in}{0.739656in}}%
\pgfpathlineto{\pgfqpoint{5.036434in}{0.739656in}}%
\pgfpathlineto{\pgfqpoint{5.036138in}{0.739656in}}%
\pgfpathlineto{\pgfqpoint{5.035842in}{0.739656in}}%
\pgfpathlineto{\pgfqpoint{5.035546in}{0.739656in}}%
\pgfpathlineto{\pgfqpoint{5.035250in}{0.739656in}}%
\pgfpathlineto{\pgfqpoint{5.034954in}{0.739656in}}%
\pgfpathlineto{\pgfqpoint{5.034658in}{0.739656in}}%
\pgfpathlineto{\pgfqpoint{5.034362in}{0.739656in}}%
\pgfpathlineto{\pgfqpoint{5.034066in}{0.739656in}}%
\pgfpathlineto{\pgfqpoint{5.033770in}{0.739656in}}%
\pgfpathlineto{\pgfqpoint{5.033474in}{0.739656in}}%
\pgfpathlineto{\pgfqpoint{5.033178in}{0.739656in}}%
\pgfpathlineto{\pgfqpoint{5.032882in}{0.739656in}}%
\pgfpathlineto{\pgfqpoint{5.032586in}{0.739656in}}%
\pgfpathlineto{\pgfqpoint{5.032290in}{0.739656in}}%
\pgfpathlineto{\pgfqpoint{5.031994in}{0.739656in}}%
\pgfpathlineto{\pgfqpoint{5.031698in}{0.739656in}}%
\pgfpathlineto{\pgfqpoint{5.031402in}{0.739656in}}%
\pgfpathlineto{\pgfqpoint{5.031106in}{0.739656in}}%
\pgfpathlineto{\pgfqpoint{5.030810in}{0.739656in}}%
\pgfpathlineto{\pgfqpoint{5.030514in}{0.739656in}}%
\pgfpathlineto{\pgfqpoint{5.030218in}{0.739656in}}%
\pgfpathlineto{\pgfqpoint{5.029922in}{0.739656in}}%
\pgfpathlineto{\pgfqpoint{5.029625in}{0.739656in}}%
\pgfpathlineto{\pgfqpoint{5.029329in}{0.739656in}}%
\pgfpathlineto{\pgfqpoint{5.029033in}{0.739656in}}%
\pgfpathlineto{\pgfqpoint{5.028737in}{0.739656in}}%
\pgfpathlineto{\pgfqpoint{5.028441in}{0.739656in}}%
\pgfpathlineto{\pgfqpoint{5.028145in}{0.739656in}}%
\pgfpathlineto{\pgfqpoint{5.027849in}{0.739656in}}%
\pgfpathlineto{\pgfqpoint{5.027553in}{0.739656in}}%
\pgfpathlineto{\pgfqpoint{5.027257in}{0.739656in}}%
\pgfpathlineto{\pgfqpoint{5.026961in}{0.739656in}}%
\pgfpathlineto{\pgfqpoint{5.026665in}{0.739656in}}%
\pgfpathlineto{\pgfqpoint{5.026369in}{0.739656in}}%
\pgfpathlineto{\pgfqpoint{5.026073in}{0.739656in}}%
\pgfpathlineto{\pgfqpoint{5.025777in}{0.739656in}}%
\pgfpathlineto{\pgfqpoint{5.025481in}{0.739656in}}%
\pgfpathlineto{\pgfqpoint{5.025185in}{0.739656in}}%
\pgfpathlineto{\pgfqpoint{5.024889in}{0.739656in}}%
\pgfpathlineto{\pgfqpoint{5.024593in}{0.739656in}}%
\pgfpathlineto{\pgfqpoint{5.024297in}{0.739656in}}%
\pgfpathlineto{\pgfqpoint{5.024001in}{0.739656in}}%
\pgfpathlineto{\pgfqpoint{5.023705in}{0.739656in}}%
\pgfpathlineto{\pgfqpoint{5.023409in}{0.739656in}}%
\pgfpathlineto{\pgfqpoint{5.023113in}{0.739656in}}%
\pgfpathlineto{\pgfqpoint{5.022817in}{0.739656in}}%
\pgfpathlineto{\pgfqpoint{5.022521in}{0.739656in}}%
\pgfpathlineto{\pgfqpoint{5.022225in}{0.739656in}}%
\pgfpathlineto{\pgfqpoint{5.021929in}{0.739656in}}%
\pgfpathlineto{\pgfqpoint{5.021633in}{0.739656in}}%
\pgfpathlineto{\pgfqpoint{5.021337in}{0.739656in}}%
\pgfpathlineto{\pgfqpoint{5.021041in}{0.739656in}}%
\pgfpathlineto{\pgfqpoint{5.020745in}{0.739656in}}%
\pgfpathlineto{\pgfqpoint{5.020449in}{0.739656in}}%
\pgfpathlineto{\pgfqpoint{5.020153in}{0.739656in}}%
\pgfpathlineto{\pgfqpoint{5.019857in}{0.739656in}}%
\pgfpathlineto{\pgfqpoint{5.019561in}{0.739656in}}%
\pgfpathlineto{\pgfqpoint{5.019265in}{0.739656in}}%
\pgfpathlineto{\pgfqpoint{5.018969in}{0.739656in}}%
\pgfpathlineto{\pgfqpoint{5.018673in}{0.739656in}}%
\pgfpathlineto{\pgfqpoint{5.018377in}{0.739656in}}%
\pgfpathlineto{\pgfqpoint{5.018081in}{0.739656in}}%
\pgfpathlineto{\pgfqpoint{5.017785in}{0.739656in}}%
\pgfpathlineto{\pgfqpoint{5.017489in}{0.739656in}}%
\pgfpathlineto{\pgfqpoint{5.017193in}{0.739656in}}%
\pgfpathlineto{\pgfqpoint{5.016897in}{0.739656in}}%
\pgfpathlineto{\pgfqpoint{5.016601in}{0.739656in}}%
\pgfpathlineto{\pgfqpoint{5.016305in}{0.739656in}}%
\pgfpathlineto{\pgfqpoint{5.016009in}{0.739656in}}%
\pgfpathlineto{\pgfqpoint{5.015713in}{0.739656in}}%
\pgfpathlineto{\pgfqpoint{5.015417in}{0.739656in}}%
\pgfpathlineto{\pgfqpoint{5.015121in}{0.739656in}}%
\pgfpathlineto{\pgfqpoint{5.014825in}{0.739656in}}%
\pgfpathlineto{\pgfqpoint{5.014529in}{0.739656in}}%
\pgfpathlineto{\pgfqpoint{5.014233in}{0.739656in}}%
\pgfpathlineto{\pgfqpoint{5.013937in}{0.739656in}}%
\pgfpathlineto{\pgfqpoint{5.013641in}{0.739656in}}%
\pgfpathlineto{\pgfqpoint{5.013345in}{0.739656in}}%
\pgfpathlineto{\pgfqpoint{5.013049in}{0.739656in}}%
\pgfpathlineto{\pgfqpoint{5.012753in}{0.739656in}}%
\pgfpathlineto{\pgfqpoint{5.012457in}{0.739656in}}%
\pgfpathlineto{\pgfqpoint{5.012161in}{0.739656in}}%
\pgfpathlineto{\pgfqpoint{5.011865in}{0.739656in}}%
\pgfpathlineto{\pgfqpoint{5.011569in}{0.739656in}}%
\pgfpathlineto{\pgfqpoint{5.011273in}{0.739656in}}%
\pgfpathlineto{\pgfqpoint{5.010977in}{0.739656in}}%
\pgfpathlineto{\pgfqpoint{5.010681in}{0.739656in}}%
\pgfpathlineto{\pgfqpoint{5.010385in}{0.739656in}}%
\pgfpathlineto{\pgfqpoint{5.010089in}{0.739656in}}%
\pgfpathlineto{\pgfqpoint{5.009793in}{0.739656in}}%
\pgfpathlineto{\pgfqpoint{5.009497in}{0.739656in}}%
\pgfpathlineto{\pgfqpoint{5.009201in}{0.739656in}}%
\pgfpathlineto{\pgfqpoint{5.008905in}{0.739656in}}%
\pgfpathlineto{\pgfqpoint{5.008609in}{0.739656in}}%
\pgfpathlineto{\pgfqpoint{5.008313in}{0.739656in}}%
\pgfpathlineto{\pgfqpoint{5.008017in}{0.739656in}}%
\pgfpathlineto{\pgfqpoint{5.007721in}{0.739656in}}%
\pgfpathlineto{\pgfqpoint{5.007425in}{0.739656in}}%
\pgfpathlineto{\pgfqpoint{5.007129in}{0.739656in}}%
\pgfpathlineto{\pgfqpoint{5.006833in}{0.739656in}}%
\pgfpathlineto{\pgfqpoint{5.006537in}{0.739656in}}%
\pgfpathlineto{\pgfqpoint{5.006241in}{0.739656in}}%
\pgfpathlineto{\pgfqpoint{5.005945in}{0.739656in}}%
\pgfpathlineto{\pgfqpoint{5.005649in}{0.739656in}}%
\pgfpathlineto{\pgfqpoint{5.005353in}{0.739656in}}%
\pgfpathlineto{\pgfqpoint{5.005057in}{0.739656in}}%
\pgfpathlineto{\pgfqpoint{5.004761in}{0.739656in}}%
\pgfpathlineto{\pgfqpoint{5.004465in}{0.739656in}}%
\pgfpathlineto{\pgfqpoint{5.004169in}{0.739656in}}%
\pgfpathlineto{\pgfqpoint{5.003873in}{0.739656in}}%
\pgfpathlineto{\pgfqpoint{5.003577in}{0.739656in}}%
\pgfpathlineto{\pgfqpoint{5.003281in}{0.739656in}}%
\pgfpathlineto{\pgfqpoint{5.002985in}{0.739656in}}%
\pgfpathlineto{\pgfqpoint{5.002689in}{0.739656in}}%
\pgfpathlineto{\pgfqpoint{5.002393in}{0.739656in}}%
\pgfpathlineto{\pgfqpoint{5.002097in}{0.739656in}}%
\pgfpathlineto{\pgfqpoint{5.001801in}{0.739656in}}%
\pgfpathlineto{\pgfqpoint{5.001505in}{0.739656in}}%
\pgfpathlineto{\pgfqpoint{5.001209in}{0.739656in}}%
\pgfpathlineto{\pgfqpoint{5.000913in}{0.739656in}}%
\pgfpathlineto{\pgfqpoint{5.000617in}{0.739656in}}%
\pgfpathlineto{\pgfqpoint{5.000321in}{0.739656in}}%
\pgfpathlineto{\pgfqpoint{5.000025in}{0.739656in}}%
\pgfpathlineto{\pgfqpoint{4.999729in}{0.739656in}}%
\pgfpathlineto{\pgfqpoint{4.999433in}{0.739656in}}%
\pgfpathlineto{\pgfqpoint{4.999137in}{0.739656in}}%
\pgfpathlineto{\pgfqpoint{4.998841in}{0.739656in}}%
\pgfpathlineto{\pgfqpoint{4.998545in}{0.739656in}}%
\pgfpathlineto{\pgfqpoint{4.998249in}{0.739656in}}%
\pgfpathlineto{\pgfqpoint{4.997953in}{0.739656in}}%
\pgfpathlineto{\pgfqpoint{4.997657in}{0.739656in}}%
\pgfpathlineto{\pgfqpoint{4.997361in}{0.739656in}}%
\pgfpathlineto{\pgfqpoint{4.997065in}{0.739656in}}%
\pgfpathlineto{\pgfqpoint{4.996769in}{0.739656in}}%
\pgfpathlineto{\pgfqpoint{4.996473in}{0.739656in}}%
\pgfpathlineto{\pgfqpoint{4.996177in}{0.739656in}}%
\pgfpathlineto{\pgfqpoint{4.995881in}{0.739656in}}%
\pgfpathlineto{\pgfqpoint{4.995585in}{0.739656in}}%
\pgfpathlineto{\pgfqpoint{4.995289in}{0.739656in}}%
\pgfpathlineto{\pgfqpoint{4.994993in}{0.739656in}}%
\pgfpathlineto{\pgfqpoint{4.994697in}{0.739656in}}%
\pgfpathlineto{\pgfqpoint{4.994401in}{0.739656in}}%
\pgfpathlineto{\pgfqpoint{4.994105in}{0.739656in}}%
\pgfpathlineto{\pgfqpoint{4.993809in}{0.739656in}}%
\pgfpathlineto{\pgfqpoint{4.993513in}{0.739656in}}%
\pgfpathlineto{\pgfqpoint{4.993217in}{0.739656in}}%
\pgfpathlineto{\pgfqpoint{4.992921in}{0.739656in}}%
\pgfpathlineto{\pgfqpoint{4.992625in}{0.739656in}}%
\pgfpathlineto{\pgfqpoint{4.992329in}{0.739656in}}%
\pgfpathlineto{\pgfqpoint{4.992033in}{0.739656in}}%
\pgfpathlineto{\pgfqpoint{4.991737in}{0.739656in}}%
\pgfpathlineto{\pgfqpoint{4.991441in}{0.739656in}}%
\pgfpathlineto{\pgfqpoint{4.991145in}{0.739656in}}%
\pgfpathlineto{\pgfqpoint{4.990849in}{0.739656in}}%
\pgfpathlineto{\pgfqpoint{4.990553in}{0.739656in}}%
\pgfpathlineto{\pgfqpoint{4.990257in}{0.739656in}}%
\pgfpathlineto{\pgfqpoint{4.989961in}{0.739656in}}%
\pgfpathlineto{\pgfqpoint{4.989665in}{0.739656in}}%
\pgfpathlineto{\pgfqpoint{4.989369in}{0.739656in}}%
\pgfpathlineto{\pgfqpoint{4.989073in}{0.739656in}}%
\pgfpathlineto{\pgfqpoint{4.988777in}{0.739656in}}%
\pgfpathlineto{\pgfqpoint{4.988481in}{0.739656in}}%
\pgfpathlineto{\pgfqpoint{4.988185in}{0.739656in}}%
\pgfpathlineto{\pgfqpoint{4.987889in}{0.739656in}}%
\pgfpathlineto{\pgfqpoint{4.987593in}{0.739656in}}%
\pgfpathlineto{\pgfqpoint{4.987297in}{0.739656in}}%
\pgfpathlineto{\pgfqpoint{4.987001in}{0.739656in}}%
\pgfpathlineto{\pgfqpoint{4.986705in}{0.739656in}}%
\pgfpathlineto{\pgfqpoint{4.986409in}{0.739656in}}%
\pgfpathlineto{\pgfqpoint{4.986113in}{0.739656in}}%
\pgfpathlineto{\pgfqpoint{4.985817in}{0.739656in}}%
\pgfpathlineto{\pgfqpoint{4.985521in}{0.739656in}}%
\pgfpathlineto{\pgfqpoint{4.985225in}{0.739656in}}%
\pgfpathlineto{\pgfqpoint{4.984929in}{0.739656in}}%
\pgfpathlineto{\pgfqpoint{4.984633in}{0.739656in}}%
\pgfpathlineto{\pgfqpoint{4.984337in}{0.739656in}}%
\pgfpathlineto{\pgfqpoint{4.984041in}{0.739656in}}%
\pgfpathlineto{\pgfqpoint{4.983745in}{0.739656in}}%
\pgfpathlineto{\pgfqpoint{4.983449in}{0.739656in}}%
\pgfpathlineto{\pgfqpoint{4.983153in}{0.739656in}}%
\pgfpathlineto{\pgfqpoint{4.982857in}{0.739656in}}%
\pgfpathlineto{\pgfqpoint{4.982561in}{0.739656in}}%
\pgfpathlineto{\pgfqpoint{4.982265in}{0.739656in}}%
\pgfpathlineto{\pgfqpoint{4.981969in}{0.739656in}}%
\pgfpathlineto{\pgfqpoint{4.981673in}{0.739656in}}%
\pgfpathlineto{\pgfqpoint{4.981377in}{0.739656in}}%
\pgfpathlineto{\pgfqpoint{4.981081in}{0.739656in}}%
\pgfpathlineto{\pgfqpoint{4.980785in}{0.739656in}}%
\pgfpathlineto{\pgfqpoint{4.980489in}{0.739656in}}%
\pgfpathlineto{\pgfqpoint{4.980193in}{0.739656in}}%
\pgfpathlineto{\pgfqpoint{4.979897in}{0.739656in}}%
\pgfpathlineto{\pgfqpoint{4.979601in}{0.739656in}}%
\pgfpathlineto{\pgfqpoint{4.979305in}{0.739656in}}%
\pgfpathlineto{\pgfqpoint{4.979009in}{0.739656in}}%
\pgfpathlineto{\pgfqpoint{4.978713in}{0.739656in}}%
\pgfpathlineto{\pgfqpoint{4.978417in}{0.739656in}}%
\pgfpathlineto{\pgfqpoint{4.978121in}{0.739656in}}%
\pgfpathlineto{\pgfqpoint{4.977825in}{0.739656in}}%
\pgfpathlineto{\pgfqpoint{4.977529in}{0.739656in}}%
\pgfpathlineto{\pgfqpoint{4.977233in}{0.739656in}}%
\pgfpathlineto{\pgfqpoint{4.976937in}{0.739656in}}%
\pgfpathlineto{\pgfqpoint{4.976641in}{0.739656in}}%
\pgfpathlineto{\pgfqpoint{4.976345in}{0.739656in}}%
\pgfpathlineto{\pgfqpoint{4.976049in}{0.739656in}}%
\pgfpathlineto{\pgfqpoint{4.975753in}{0.739656in}}%
\pgfpathlineto{\pgfqpoint{4.975457in}{0.739656in}}%
\pgfpathlineto{\pgfqpoint{4.975161in}{0.739656in}}%
\pgfpathlineto{\pgfqpoint{4.974865in}{0.739656in}}%
\pgfpathlineto{\pgfqpoint{4.974569in}{0.739656in}}%
\pgfpathlineto{\pgfqpoint{4.974273in}{0.739656in}}%
\pgfpathlineto{\pgfqpoint{4.973977in}{0.739656in}}%
\pgfpathlineto{\pgfqpoint{4.973681in}{0.739656in}}%
\pgfpathlineto{\pgfqpoint{4.973385in}{0.739656in}}%
\pgfpathlineto{\pgfqpoint{4.973089in}{0.739656in}}%
\pgfpathlineto{\pgfqpoint{4.972793in}{0.739656in}}%
\pgfpathlineto{\pgfqpoint{4.972497in}{0.739656in}}%
\pgfpathlineto{\pgfqpoint{4.972201in}{0.739656in}}%
\pgfpathlineto{\pgfqpoint{4.971905in}{0.739656in}}%
\pgfpathlineto{\pgfqpoint{4.971609in}{0.739656in}}%
\pgfpathlineto{\pgfqpoint{4.971313in}{0.739656in}}%
\pgfpathlineto{\pgfqpoint{4.971017in}{0.739656in}}%
\pgfpathlineto{\pgfqpoint{4.970721in}{0.739656in}}%
\pgfpathlineto{\pgfqpoint{4.970425in}{0.739656in}}%
\pgfpathlineto{\pgfqpoint{4.970129in}{0.739656in}}%
\pgfpathlineto{\pgfqpoint{4.969833in}{0.739656in}}%
\pgfpathlineto{\pgfqpoint{4.969537in}{0.739656in}}%
\pgfpathlineto{\pgfqpoint{4.969241in}{0.739656in}}%
\pgfpathlineto{\pgfqpoint{4.968945in}{0.739656in}}%
\pgfpathlineto{\pgfqpoint{4.968649in}{0.739656in}}%
\pgfpathlineto{\pgfqpoint{4.968353in}{0.739656in}}%
\pgfpathlineto{\pgfqpoint{4.968057in}{0.739656in}}%
\pgfpathlineto{\pgfqpoint{4.967761in}{0.739656in}}%
\pgfpathlineto{\pgfqpoint{4.967465in}{0.739656in}}%
\pgfpathlineto{\pgfqpoint{4.967169in}{0.739656in}}%
\pgfpathlineto{\pgfqpoint{4.966873in}{0.739656in}}%
\pgfpathlineto{\pgfqpoint{4.966577in}{0.739656in}}%
\pgfpathlineto{\pgfqpoint{4.966281in}{0.739656in}}%
\pgfpathlineto{\pgfqpoint{4.965985in}{0.739656in}}%
\pgfpathlineto{\pgfqpoint{4.965689in}{0.739656in}}%
\pgfpathlineto{\pgfqpoint{4.965393in}{0.739656in}}%
\pgfpathlineto{\pgfqpoint{4.965097in}{0.739656in}}%
\pgfpathlineto{\pgfqpoint{4.964801in}{0.739656in}}%
\pgfpathlineto{\pgfqpoint{4.964505in}{0.739656in}}%
\pgfpathlineto{\pgfqpoint{4.964209in}{0.739656in}}%
\pgfpathlineto{\pgfqpoint{4.963913in}{0.739656in}}%
\pgfpathlineto{\pgfqpoint{4.963617in}{0.739656in}}%
\pgfpathlineto{\pgfqpoint{4.963321in}{0.739656in}}%
\pgfpathlineto{\pgfqpoint{4.963025in}{0.739656in}}%
\pgfpathlineto{\pgfqpoint{4.962729in}{0.739656in}}%
\pgfpathlineto{\pgfqpoint{4.962432in}{0.739656in}}%
\pgfpathlineto{\pgfqpoint{4.962136in}{0.739656in}}%
\pgfpathlineto{\pgfqpoint{4.961840in}{0.739656in}}%
\pgfpathlineto{\pgfqpoint{4.961544in}{0.739656in}}%
\pgfpathlineto{\pgfqpoint{4.961248in}{0.739656in}}%
\pgfpathlineto{\pgfqpoint{4.960952in}{0.739656in}}%
\pgfpathlineto{\pgfqpoint{4.960656in}{0.739656in}}%
\pgfpathlineto{\pgfqpoint{4.960360in}{0.739656in}}%
\pgfpathlineto{\pgfqpoint{4.960064in}{0.739656in}}%
\pgfpathlineto{\pgfqpoint{4.959768in}{0.739656in}}%
\pgfpathlineto{\pgfqpoint{4.959472in}{0.739656in}}%
\pgfpathlineto{\pgfqpoint{4.959176in}{0.739656in}}%
\pgfpathlineto{\pgfqpoint{4.958880in}{0.739656in}}%
\pgfpathlineto{\pgfqpoint{4.958584in}{0.739656in}}%
\pgfpathlineto{\pgfqpoint{4.958288in}{0.739656in}}%
\pgfpathlineto{\pgfqpoint{4.957992in}{0.739656in}}%
\pgfpathlineto{\pgfqpoint{4.957696in}{0.739656in}}%
\pgfpathlineto{\pgfqpoint{4.957400in}{0.739656in}}%
\pgfpathlineto{\pgfqpoint{4.957104in}{0.739656in}}%
\pgfpathlineto{\pgfqpoint{4.956808in}{0.739656in}}%
\pgfpathlineto{\pgfqpoint{4.956512in}{0.739656in}}%
\pgfpathlineto{\pgfqpoint{4.956216in}{0.739656in}}%
\pgfpathlineto{\pgfqpoint{4.955920in}{0.739656in}}%
\pgfpathlineto{\pgfqpoint{4.955624in}{0.739656in}}%
\pgfpathlineto{\pgfqpoint{4.955328in}{0.739656in}}%
\pgfpathlineto{\pgfqpoint{4.955032in}{0.739656in}}%
\pgfpathlineto{\pgfqpoint{4.954736in}{0.739656in}}%
\pgfpathlineto{\pgfqpoint{4.954440in}{0.739656in}}%
\pgfpathlineto{\pgfqpoint{4.954144in}{0.739656in}}%
\pgfpathlineto{\pgfqpoint{4.953848in}{0.739656in}}%
\pgfpathlineto{\pgfqpoint{4.953552in}{0.739656in}}%
\pgfpathlineto{\pgfqpoint{4.953256in}{0.739656in}}%
\pgfpathlineto{\pgfqpoint{4.952960in}{0.739656in}}%
\pgfpathlineto{\pgfqpoint{4.952664in}{0.739656in}}%
\pgfpathlineto{\pgfqpoint{4.952368in}{0.739656in}}%
\pgfpathlineto{\pgfqpoint{4.952072in}{0.739656in}}%
\pgfpathlineto{\pgfqpoint{4.951776in}{0.739656in}}%
\pgfpathlineto{\pgfqpoint{4.951480in}{0.739656in}}%
\pgfpathlineto{\pgfqpoint{4.951184in}{0.739656in}}%
\pgfpathlineto{\pgfqpoint{4.950888in}{0.739656in}}%
\pgfpathlineto{\pgfqpoint{4.950592in}{0.739656in}}%
\pgfpathlineto{\pgfqpoint{4.950296in}{0.739656in}}%
\pgfpathlineto{\pgfqpoint{4.950000in}{0.739656in}}%
\pgfpathlineto{\pgfqpoint{4.949704in}{0.739656in}}%
\pgfpathlineto{\pgfqpoint{4.949408in}{0.739656in}}%
\pgfpathlineto{\pgfqpoint{4.949112in}{0.739656in}}%
\pgfpathlineto{\pgfqpoint{4.948816in}{0.739656in}}%
\pgfpathlineto{\pgfqpoint{4.948520in}{0.739656in}}%
\pgfpathlineto{\pgfqpoint{4.948224in}{0.739656in}}%
\pgfpathlineto{\pgfqpoint{4.947928in}{0.739656in}}%
\pgfpathlineto{\pgfqpoint{4.947632in}{0.739656in}}%
\pgfpathlineto{\pgfqpoint{4.947336in}{0.739656in}}%
\pgfpathlineto{\pgfqpoint{4.947040in}{0.739656in}}%
\pgfpathlineto{\pgfqpoint{4.946744in}{0.739656in}}%
\pgfpathlineto{\pgfqpoint{4.946448in}{0.739656in}}%
\pgfpathlineto{\pgfqpoint{4.946152in}{0.739656in}}%
\pgfpathlineto{\pgfqpoint{4.945856in}{0.739656in}}%
\pgfpathlineto{\pgfqpoint{4.945560in}{0.739656in}}%
\pgfpathlineto{\pgfqpoint{4.945264in}{0.739656in}}%
\pgfpathlineto{\pgfqpoint{4.944968in}{0.739656in}}%
\pgfpathlineto{\pgfqpoint{4.944672in}{0.739656in}}%
\pgfpathlineto{\pgfqpoint{4.944376in}{0.739656in}}%
\pgfpathlineto{\pgfqpoint{4.944080in}{0.739656in}}%
\pgfpathlineto{\pgfqpoint{4.943784in}{0.739656in}}%
\pgfpathlineto{\pgfqpoint{4.943488in}{0.739656in}}%
\pgfpathlineto{\pgfqpoint{4.943192in}{0.739656in}}%
\pgfpathlineto{\pgfqpoint{4.942896in}{0.739656in}}%
\pgfpathlineto{\pgfqpoint{4.942600in}{0.739656in}}%
\pgfpathlineto{\pgfqpoint{4.942304in}{0.739656in}}%
\pgfpathlineto{\pgfqpoint{4.942008in}{0.739656in}}%
\pgfpathlineto{\pgfqpoint{4.941712in}{0.739656in}}%
\pgfpathlineto{\pgfqpoint{4.941416in}{0.739656in}}%
\pgfpathlineto{\pgfqpoint{4.941120in}{0.739656in}}%
\pgfpathlineto{\pgfqpoint{4.940824in}{0.739656in}}%
\pgfpathlineto{\pgfqpoint{4.940528in}{0.739656in}}%
\pgfpathlineto{\pgfqpoint{4.940232in}{0.739656in}}%
\pgfpathlineto{\pgfqpoint{4.939936in}{0.739656in}}%
\pgfpathlineto{\pgfqpoint{4.939640in}{0.739656in}}%
\pgfpathlineto{\pgfqpoint{4.939344in}{0.739656in}}%
\pgfpathlineto{\pgfqpoint{4.939048in}{0.739656in}}%
\pgfpathlineto{\pgfqpoint{4.938752in}{0.739656in}}%
\pgfpathlineto{\pgfqpoint{4.938456in}{0.739656in}}%
\pgfpathlineto{\pgfqpoint{4.938160in}{0.739656in}}%
\pgfpathlineto{\pgfqpoint{4.937864in}{0.739656in}}%
\pgfpathlineto{\pgfqpoint{4.937568in}{0.739656in}}%
\pgfpathlineto{\pgfqpoint{4.937272in}{0.739656in}}%
\pgfpathlineto{\pgfqpoint{4.936976in}{0.739656in}}%
\pgfpathlineto{\pgfqpoint{4.936680in}{0.739656in}}%
\pgfpathlineto{\pgfqpoint{4.936384in}{0.739656in}}%
\pgfpathlineto{\pgfqpoint{4.936088in}{0.739656in}}%
\pgfpathlineto{\pgfqpoint{4.935792in}{0.739656in}}%
\pgfpathlineto{\pgfqpoint{4.935496in}{0.739656in}}%
\pgfpathlineto{\pgfqpoint{4.935200in}{0.739656in}}%
\pgfpathlineto{\pgfqpoint{4.934904in}{0.739656in}}%
\pgfpathlineto{\pgfqpoint{4.934608in}{0.739656in}}%
\pgfpathlineto{\pgfqpoint{4.934312in}{0.739656in}}%
\pgfpathlineto{\pgfqpoint{4.934016in}{0.739656in}}%
\pgfpathlineto{\pgfqpoint{4.933720in}{0.739656in}}%
\pgfpathlineto{\pgfqpoint{4.933424in}{0.739656in}}%
\pgfpathlineto{\pgfqpoint{4.933128in}{0.739656in}}%
\pgfpathlineto{\pgfqpoint{4.932832in}{0.739656in}}%
\pgfpathlineto{\pgfqpoint{4.932536in}{0.739656in}}%
\pgfpathlineto{\pgfqpoint{4.932240in}{0.739656in}}%
\pgfpathlineto{\pgfqpoint{4.931944in}{0.739656in}}%
\pgfpathlineto{\pgfqpoint{4.931648in}{0.739656in}}%
\pgfpathlineto{\pgfqpoint{4.931352in}{0.739656in}}%
\pgfpathlineto{\pgfqpoint{4.931056in}{0.739656in}}%
\pgfpathlineto{\pgfqpoint{4.930760in}{0.739656in}}%
\pgfpathlineto{\pgfqpoint{4.930464in}{0.739656in}}%
\pgfpathlineto{\pgfqpoint{4.930168in}{0.739656in}}%
\pgfpathlineto{\pgfqpoint{4.929872in}{0.739656in}}%
\pgfpathlineto{\pgfqpoint{4.929576in}{0.739656in}}%
\pgfpathlineto{\pgfqpoint{4.929280in}{0.739656in}}%
\pgfpathlineto{\pgfqpoint{4.928984in}{0.739656in}}%
\pgfpathlineto{\pgfqpoint{4.928688in}{0.739656in}}%
\pgfpathlineto{\pgfqpoint{4.928392in}{0.739656in}}%
\pgfpathlineto{\pgfqpoint{4.928096in}{0.739656in}}%
\pgfpathlineto{\pgfqpoint{4.927800in}{0.739656in}}%
\pgfpathlineto{\pgfqpoint{4.927504in}{0.739656in}}%
\pgfpathlineto{\pgfqpoint{4.927208in}{0.739656in}}%
\pgfpathlineto{\pgfqpoint{4.926912in}{0.739656in}}%
\pgfpathlineto{\pgfqpoint{4.926616in}{0.739656in}}%
\pgfpathlineto{\pgfqpoint{4.926320in}{0.739656in}}%
\pgfpathlineto{\pgfqpoint{4.926024in}{0.739656in}}%
\pgfpathlineto{\pgfqpoint{4.925728in}{0.739656in}}%
\pgfpathlineto{\pgfqpoint{4.925432in}{0.739656in}}%
\pgfpathlineto{\pgfqpoint{4.925136in}{0.739656in}}%
\pgfpathlineto{\pgfqpoint{4.924840in}{0.739656in}}%
\pgfpathlineto{\pgfqpoint{4.924544in}{0.739656in}}%
\pgfpathlineto{\pgfqpoint{4.924248in}{0.739656in}}%
\pgfpathlineto{\pgfqpoint{4.923952in}{0.739656in}}%
\pgfpathlineto{\pgfqpoint{4.923656in}{0.739656in}}%
\pgfpathlineto{\pgfqpoint{4.923360in}{0.739656in}}%
\pgfpathlineto{\pgfqpoint{4.923064in}{0.739656in}}%
\pgfpathlineto{\pgfqpoint{4.922768in}{0.739656in}}%
\pgfpathlineto{\pgfqpoint{4.922472in}{0.739656in}}%
\pgfpathlineto{\pgfqpoint{4.922176in}{0.739656in}}%
\pgfpathlineto{\pgfqpoint{4.921880in}{0.739656in}}%
\pgfpathlineto{\pgfqpoint{4.921584in}{0.739656in}}%
\pgfpathlineto{\pgfqpoint{4.921288in}{0.739656in}}%
\pgfpathlineto{\pgfqpoint{4.920992in}{0.739656in}}%
\pgfpathlineto{\pgfqpoint{4.920696in}{0.739656in}}%
\pgfpathlineto{\pgfqpoint{4.920400in}{0.739656in}}%
\pgfpathlineto{\pgfqpoint{4.920104in}{0.739656in}}%
\pgfpathlineto{\pgfqpoint{4.919808in}{0.739656in}}%
\pgfpathlineto{\pgfqpoint{4.919512in}{0.739656in}}%
\pgfpathlineto{\pgfqpoint{4.919216in}{0.739656in}}%
\pgfpathlineto{\pgfqpoint{4.918920in}{0.739656in}}%
\pgfpathlineto{\pgfqpoint{4.918624in}{0.739656in}}%
\pgfpathlineto{\pgfqpoint{4.918328in}{0.739656in}}%
\pgfpathlineto{\pgfqpoint{4.918032in}{0.739656in}}%
\pgfpathlineto{\pgfqpoint{4.917736in}{0.739656in}}%
\pgfpathlineto{\pgfqpoint{4.917440in}{0.739656in}}%
\pgfpathlineto{\pgfqpoint{4.917144in}{0.739656in}}%
\pgfpathlineto{\pgfqpoint{4.916848in}{0.739656in}}%
\pgfpathlineto{\pgfqpoint{4.916552in}{0.739656in}}%
\pgfpathlineto{\pgfqpoint{4.916256in}{0.739656in}}%
\pgfpathlineto{\pgfqpoint{4.915960in}{0.739656in}}%
\pgfpathlineto{\pgfqpoint{4.915664in}{0.739656in}}%
\pgfpathlineto{\pgfqpoint{4.915368in}{0.739656in}}%
\pgfpathlineto{\pgfqpoint{4.915072in}{0.739656in}}%
\pgfpathlineto{\pgfqpoint{4.914776in}{0.739656in}}%
\pgfpathlineto{\pgfqpoint{4.914480in}{0.739656in}}%
\pgfpathlineto{\pgfqpoint{4.914184in}{0.739656in}}%
\pgfpathlineto{\pgfqpoint{4.913888in}{0.739656in}}%
\pgfpathlineto{\pgfqpoint{4.913592in}{0.739656in}}%
\pgfpathlineto{\pgfqpoint{4.913296in}{0.739656in}}%
\pgfpathlineto{\pgfqpoint{4.913000in}{0.739656in}}%
\pgfpathlineto{\pgfqpoint{4.912704in}{0.739656in}}%
\pgfpathlineto{\pgfqpoint{4.912408in}{0.739656in}}%
\pgfpathlineto{\pgfqpoint{4.912112in}{0.739656in}}%
\pgfpathlineto{\pgfqpoint{4.911816in}{0.739656in}}%
\pgfpathlineto{\pgfqpoint{4.911520in}{0.739656in}}%
\pgfpathlineto{\pgfqpoint{4.911224in}{0.739656in}}%
\pgfpathlineto{\pgfqpoint{4.910928in}{0.739656in}}%
\pgfpathlineto{\pgfqpoint{4.910632in}{0.739656in}}%
\pgfpathlineto{\pgfqpoint{4.910336in}{0.739656in}}%
\pgfpathlineto{\pgfqpoint{4.910040in}{0.739656in}}%
\pgfpathlineto{\pgfqpoint{4.909744in}{0.739656in}}%
\pgfpathlineto{\pgfqpoint{4.909448in}{0.739656in}}%
\pgfpathlineto{\pgfqpoint{4.909152in}{0.739656in}}%
\pgfpathlineto{\pgfqpoint{4.908856in}{0.739656in}}%
\pgfpathlineto{\pgfqpoint{4.908560in}{0.739656in}}%
\pgfpathlineto{\pgfqpoint{4.908264in}{0.739656in}}%
\pgfpathlineto{\pgfqpoint{4.907968in}{0.739656in}}%
\pgfpathlineto{\pgfqpoint{4.907672in}{0.739656in}}%
\pgfpathlineto{\pgfqpoint{4.907376in}{0.739656in}}%
\pgfpathlineto{\pgfqpoint{4.907080in}{0.739656in}}%
\pgfpathlineto{\pgfqpoint{4.906784in}{0.739656in}}%
\pgfpathlineto{\pgfqpoint{4.906488in}{0.739656in}}%
\pgfpathlineto{\pgfqpoint{4.906192in}{0.739656in}}%
\pgfpathlineto{\pgfqpoint{4.905896in}{0.739656in}}%
\pgfpathlineto{\pgfqpoint{4.905600in}{0.739656in}}%
\pgfpathlineto{\pgfqpoint{4.905304in}{0.739656in}}%
\pgfpathlineto{\pgfqpoint{4.905008in}{0.739656in}}%
\pgfpathlineto{\pgfqpoint{4.904712in}{0.739656in}}%
\pgfpathlineto{\pgfqpoint{4.904416in}{0.739656in}}%
\pgfpathlineto{\pgfqpoint{4.904120in}{0.739656in}}%
\pgfpathlineto{\pgfqpoint{4.903824in}{0.739656in}}%
\pgfpathlineto{\pgfqpoint{4.903528in}{0.739656in}}%
\pgfpathlineto{\pgfqpoint{4.903232in}{0.739656in}}%
\pgfpathlineto{\pgfqpoint{4.902936in}{0.739656in}}%
\pgfpathlineto{\pgfqpoint{4.902640in}{0.739656in}}%
\pgfpathlineto{\pgfqpoint{4.902344in}{0.739656in}}%
\pgfpathlineto{\pgfqpoint{4.902048in}{0.739656in}}%
\pgfpathlineto{\pgfqpoint{4.901752in}{0.739656in}}%
\pgfpathlineto{\pgfqpoint{4.901456in}{0.739656in}}%
\pgfpathlineto{\pgfqpoint{4.901160in}{0.739656in}}%
\pgfpathlineto{\pgfqpoint{4.900864in}{0.739656in}}%
\pgfpathlineto{\pgfqpoint{4.900568in}{0.739656in}}%
\pgfpathlineto{\pgfqpoint{4.900272in}{0.739656in}}%
\pgfpathlineto{\pgfqpoint{4.899976in}{0.739656in}}%
\pgfpathlineto{\pgfqpoint{4.899680in}{0.739656in}}%
\pgfpathlineto{\pgfqpoint{4.899384in}{0.739656in}}%
\pgfpathlineto{\pgfqpoint{4.899088in}{0.739656in}}%
\pgfpathlineto{\pgfqpoint{4.898792in}{0.739656in}}%
\pgfpathlineto{\pgfqpoint{4.898496in}{0.739656in}}%
\pgfpathlineto{\pgfqpoint{4.898200in}{0.739656in}}%
\pgfpathlineto{\pgfqpoint{4.897904in}{0.739656in}}%
\pgfpathlineto{\pgfqpoint{4.897608in}{0.739656in}}%
\pgfpathlineto{\pgfqpoint{4.897312in}{0.739656in}}%
\pgfpathlineto{\pgfqpoint{4.897016in}{0.739656in}}%
\pgfpathlineto{\pgfqpoint{4.896720in}{0.739656in}}%
\pgfpathlineto{\pgfqpoint{4.896424in}{0.739656in}}%
\pgfpathlineto{\pgfqpoint{4.896128in}{0.739656in}}%
\pgfpathlineto{\pgfqpoint{4.895832in}{0.739656in}}%
\pgfpathlineto{\pgfqpoint{4.895536in}{0.739656in}}%
\pgfpathlineto{\pgfqpoint{4.895240in}{0.739656in}}%
\pgfpathlineto{\pgfqpoint{4.894943in}{0.739656in}}%
\pgfpathlineto{\pgfqpoint{4.894647in}{0.739656in}}%
\pgfpathlineto{\pgfqpoint{4.894351in}{0.739656in}}%
\pgfpathlineto{\pgfqpoint{4.894055in}{0.739656in}}%
\pgfpathlineto{\pgfqpoint{4.893759in}{0.739656in}}%
\pgfpathlineto{\pgfqpoint{4.893463in}{0.739656in}}%
\pgfpathlineto{\pgfqpoint{4.893167in}{0.739656in}}%
\pgfpathlineto{\pgfqpoint{4.892871in}{0.739656in}}%
\pgfpathlineto{\pgfqpoint{4.892575in}{0.739656in}}%
\pgfpathlineto{\pgfqpoint{4.892279in}{0.739656in}}%
\pgfpathlineto{\pgfqpoint{4.891983in}{0.739656in}}%
\pgfpathlineto{\pgfqpoint{4.891687in}{0.739656in}}%
\pgfpathlineto{\pgfqpoint{4.891391in}{0.739656in}}%
\pgfpathlineto{\pgfqpoint{4.891095in}{0.739656in}}%
\pgfpathlineto{\pgfqpoint{4.890799in}{0.739656in}}%
\pgfpathlineto{\pgfqpoint{4.890503in}{0.739656in}}%
\pgfpathlineto{\pgfqpoint{4.890207in}{0.739656in}}%
\pgfpathlineto{\pgfqpoint{4.889911in}{0.739656in}}%
\pgfpathlineto{\pgfqpoint{4.889615in}{0.739656in}}%
\pgfpathlineto{\pgfqpoint{4.889319in}{0.739656in}}%
\pgfpathlineto{\pgfqpoint{4.889023in}{0.739656in}}%
\pgfpathlineto{\pgfqpoint{4.888727in}{0.739656in}}%
\pgfpathlineto{\pgfqpoint{4.888431in}{0.739656in}}%
\pgfpathlineto{\pgfqpoint{4.888135in}{0.739656in}}%
\pgfpathlineto{\pgfqpoint{4.887839in}{0.739656in}}%
\pgfpathlineto{\pgfqpoint{4.887543in}{0.739656in}}%
\pgfpathlineto{\pgfqpoint{4.887247in}{0.739656in}}%
\pgfpathlineto{\pgfqpoint{4.886951in}{0.739656in}}%
\pgfpathlineto{\pgfqpoint{4.886655in}{0.739656in}}%
\pgfpathlineto{\pgfqpoint{4.886359in}{0.739656in}}%
\pgfpathlineto{\pgfqpoint{4.886063in}{0.739656in}}%
\pgfpathlineto{\pgfqpoint{4.885767in}{0.739656in}}%
\pgfpathlineto{\pgfqpoint{4.885471in}{0.739656in}}%
\pgfpathlineto{\pgfqpoint{4.885175in}{0.739656in}}%
\pgfpathlineto{\pgfqpoint{4.884879in}{0.739656in}}%
\pgfpathlineto{\pgfqpoint{4.884583in}{0.739656in}}%
\pgfpathlineto{\pgfqpoint{4.884287in}{0.739656in}}%
\pgfpathlineto{\pgfqpoint{4.883991in}{0.739656in}}%
\pgfpathlineto{\pgfqpoint{4.883695in}{0.739656in}}%
\pgfpathlineto{\pgfqpoint{4.883399in}{0.739656in}}%
\pgfpathlineto{\pgfqpoint{4.883103in}{0.739656in}}%
\pgfpathlineto{\pgfqpoint{4.882807in}{0.739656in}}%
\pgfpathlineto{\pgfqpoint{4.882511in}{0.739656in}}%
\pgfpathlineto{\pgfqpoint{4.882215in}{0.739656in}}%
\pgfpathlineto{\pgfqpoint{4.881919in}{0.739656in}}%
\pgfpathlineto{\pgfqpoint{4.881623in}{0.739656in}}%
\pgfpathlineto{\pgfqpoint{4.881327in}{0.739656in}}%
\pgfpathlineto{\pgfqpoint{4.881031in}{0.739656in}}%
\pgfpathlineto{\pgfqpoint{4.880735in}{0.739656in}}%
\pgfpathlineto{\pgfqpoint{4.880439in}{0.739656in}}%
\pgfpathlineto{\pgfqpoint{4.880143in}{0.739656in}}%
\pgfpathlineto{\pgfqpoint{4.879847in}{0.739656in}}%
\pgfpathlineto{\pgfqpoint{4.879551in}{0.739656in}}%
\pgfpathlineto{\pgfqpoint{4.879255in}{0.739656in}}%
\pgfpathlineto{\pgfqpoint{4.878959in}{0.739656in}}%
\pgfpathlineto{\pgfqpoint{4.878663in}{0.739656in}}%
\pgfpathlineto{\pgfqpoint{4.878367in}{0.739656in}}%
\pgfpathlineto{\pgfqpoint{4.878071in}{0.739656in}}%
\pgfpathlineto{\pgfqpoint{4.877775in}{0.739656in}}%
\pgfpathlineto{\pgfqpoint{4.877479in}{0.739656in}}%
\pgfpathlineto{\pgfqpoint{4.877183in}{0.739656in}}%
\pgfpathlineto{\pgfqpoint{4.876887in}{0.739656in}}%
\pgfpathlineto{\pgfqpoint{4.876591in}{0.739656in}}%
\pgfpathlineto{\pgfqpoint{4.876295in}{0.739656in}}%
\pgfpathlineto{\pgfqpoint{4.875999in}{0.739656in}}%
\pgfpathlineto{\pgfqpoint{4.875703in}{0.739656in}}%
\pgfpathlineto{\pgfqpoint{4.875407in}{0.739656in}}%
\pgfpathlineto{\pgfqpoint{4.875111in}{0.739656in}}%
\pgfpathlineto{\pgfqpoint{4.874815in}{0.739656in}}%
\pgfpathlineto{\pgfqpoint{4.874519in}{0.739656in}}%
\pgfpathlineto{\pgfqpoint{4.874223in}{0.739656in}}%
\pgfpathlineto{\pgfqpoint{4.873927in}{0.739656in}}%
\pgfpathlineto{\pgfqpoint{4.873631in}{0.739656in}}%
\pgfpathlineto{\pgfqpoint{4.873335in}{0.739656in}}%
\pgfpathlineto{\pgfqpoint{4.873039in}{0.739656in}}%
\pgfpathlineto{\pgfqpoint{4.872743in}{0.739656in}}%
\pgfpathlineto{\pgfqpoint{4.872447in}{0.739656in}}%
\pgfpathlineto{\pgfqpoint{4.872151in}{0.739656in}}%
\pgfpathlineto{\pgfqpoint{4.871855in}{0.739656in}}%
\pgfpathlineto{\pgfqpoint{4.871559in}{0.739656in}}%
\pgfpathlineto{\pgfqpoint{4.871263in}{0.739656in}}%
\pgfpathlineto{\pgfqpoint{4.870967in}{0.739656in}}%
\pgfpathlineto{\pgfqpoint{4.870671in}{0.739656in}}%
\pgfpathlineto{\pgfqpoint{4.870375in}{0.739656in}}%
\pgfpathlineto{\pgfqpoint{4.870079in}{0.739656in}}%
\pgfpathlineto{\pgfqpoint{4.869783in}{0.739656in}}%
\pgfpathlineto{\pgfqpoint{4.869487in}{0.739656in}}%
\pgfpathlineto{\pgfqpoint{4.869191in}{0.739656in}}%
\pgfpathlineto{\pgfqpoint{4.868895in}{0.739656in}}%
\pgfpathlineto{\pgfqpoint{4.868599in}{0.739656in}}%
\pgfpathlineto{\pgfqpoint{4.868303in}{0.739656in}}%
\pgfpathlineto{\pgfqpoint{4.868007in}{0.739656in}}%
\pgfpathlineto{\pgfqpoint{4.867711in}{0.739656in}}%
\pgfpathlineto{\pgfqpoint{4.867415in}{0.739656in}}%
\pgfpathlineto{\pgfqpoint{4.867119in}{0.739656in}}%
\pgfpathlineto{\pgfqpoint{4.866823in}{0.739656in}}%
\pgfpathlineto{\pgfqpoint{4.866527in}{0.739656in}}%
\pgfpathlineto{\pgfqpoint{4.866231in}{0.739656in}}%
\pgfpathlineto{\pgfqpoint{4.865935in}{0.739656in}}%
\pgfpathlineto{\pgfqpoint{4.865639in}{0.739656in}}%
\pgfpathlineto{\pgfqpoint{4.865343in}{0.739656in}}%
\pgfpathlineto{\pgfqpoint{4.865047in}{0.739656in}}%
\pgfpathlineto{\pgfqpoint{4.864751in}{0.739656in}}%
\pgfpathlineto{\pgfqpoint{4.864455in}{0.739656in}}%
\pgfpathlineto{\pgfqpoint{4.864159in}{0.739656in}}%
\pgfpathlineto{\pgfqpoint{4.863863in}{0.739656in}}%
\pgfpathlineto{\pgfqpoint{4.863567in}{0.739656in}}%
\pgfpathlineto{\pgfqpoint{4.863271in}{0.739656in}}%
\pgfpathlineto{\pgfqpoint{4.862975in}{0.739656in}}%
\pgfpathlineto{\pgfqpoint{4.862679in}{0.739656in}}%
\pgfpathlineto{\pgfqpoint{4.862383in}{0.739656in}}%
\pgfpathlineto{\pgfqpoint{4.862087in}{0.739656in}}%
\pgfpathlineto{\pgfqpoint{4.861791in}{0.739656in}}%
\pgfpathlineto{\pgfqpoint{4.861495in}{0.739656in}}%
\pgfpathlineto{\pgfqpoint{4.861199in}{0.739656in}}%
\pgfpathlineto{\pgfqpoint{4.860903in}{0.739656in}}%
\pgfpathlineto{\pgfqpoint{4.860607in}{0.739656in}}%
\pgfpathlineto{\pgfqpoint{4.860311in}{0.739656in}}%
\pgfpathlineto{\pgfqpoint{4.860015in}{0.739656in}}%
\pgfpathlineto{\pgfqpoint{4.859719in}{0.739656in}}%
\pgfpathlineto{\pgfqpoint{4.859423in}{0.739656in}}%
\pgfpathlineto{\pgfqpoint{4.859127in}{0.739656in}}%
\pgfpathlineto{\pgfqpoint{4.858831in}{0.739656in}}%
\pgfpathlineto{\pgfqpoint{4.858535in}{0.739656in}}%
\pgfpathlineto{\pgfqpoint{4.858239in}{0.739656in}}%
\pgfpathlineto{\pgfqpoint{4.857943in}{0.739656in}}%
\pgfpathlineto{\pgfqpoint{4.857647in}{0.739656in}}%
\pgfpathlineto{\pgfqpoint{4.857351in}{0.739656in}}%
\pgfpathlineto{\pgfqpoint{4.857055in}{0.739656in}}%
\pgfpathlineto{\pgfqpoint{4.856759in}{0.739656in}}%
\pgfpathlineto{\pgfqpoint{4.856463in}{0.739656in}}%
\pgfpathlineto{\pgfqpoint{4.856167in}{0.739656in}}%
\pgfpathlineto{\pgfqpoint{4.855871in}{0.739656in}}%
\pgfpathlineto{\pgfqpoint{4.855575in}{0.739656in}}%
\pgfpathlineto{\pgfqpoint{4.855279in}{0.739656in}}%
\pgfpathlineto{\pgfqpoint{4.854983in}{0.739656in}}%
\pgfpathlineto{\pgfqpoint{4.854687in}{0.739656in}}%
\pgfpathlineto{\pgfqpoint{4.854391in}{0.739656in}}%
\pgfpathlineto{\pgfqpoint{4.854095in}{0.739656in}}%
\pgfpathlineto{\pgfqpoint{4.853799in}{0.739656in}}%
\pgfpathlineto{\pgfqpoint{4.853503in}{0.739656in}}%
\pgfpathlineto{\pgfqpoint{4.853207in}{0.739656in}}%
\pgfpathlineto{\pgfqpoint{4.852911in}{0.739656in}}%
\pgfpathlineto{\pgfqpoint{4.852615in}{0.739656in}}%
\pgfpathlineto{\pgfqpoint{4.852319in}{0.739656in}}%
\pgfpathlineto{\pgfqpoint{4.852023in}{0.739656in}}%
\pgfpathlineto{\pgfqpoint{4.851727in}{0.739656in}}%
\pgfpathlineto{\pgfqpoint{4.851431in}{0.739656in}}%
\pgfpathlineto{\pgfqpoint{4.851135in}{0.739656in}}%
\pgfpathlineto{\pgfqpoint{4.850839in}{0.739656in}}%
\pgfpathlineto{\pgfqpoint{4.850543in}{0.739656in}}%
\pgfpathlineto{\pgfqpoint{4.850247in}{0.739656in}}%
\pgfpathlineto{\pgfqpoint{4.849951in}{0.739656in}}%
\pgfpathlineto{\pgfqpoint{4.849655in}{0.739656in}}%
\pgfpathlineto{\pgfqpoint{4.849359in}{0.739656in}}%
\pgfpathlineto{\pgfqpoint{4.849063in}{0.739656in}}%
\pgfpathlineto{\pgfqpoint{4.848767in}{0.739656in}}%
\pgfpathlineto{\pgfqpoint{4.848471in}{0.739656in}}%
\pgfpathlineto{\pgfqpoint{4.848175in}{0.739656in}}%
\pgfpathlineto{\pgfqpoint{4.847879in}{0.739656in}}%
\pgfpathlineto{\pgfqpoint{4.847583in}{0.739656in}}%
\pgfpathlineto{\pgfqpoint{4.847287in}{0.739656in}}%
\pgfpathlineto{\pgfqpoint{4.846991in}{0.739656in}}%
\pgfpathlineto{\pgfqpoint{4.846695in}{0.739656in}}%
\pgfpathlineto{\pgfqpoint{4.846399in}{0.739656in}}%
\pgfpathlineto{\pgfqpoint{4.846103in}{0.739656in}}%
\pgfpathlineto{\pgfqpoint{4.845807in}{0.739656in}}%
\pgfpathlineto{\pgfqpoint{4.845511in}{0.739656in}}%
\pgfpathlineto{\pgfqpoint{4.845215in}{0.739656in}}%
\pgfpathlineto{\pgfqpoint{4.844919in}{0.739656in}}%
\pgfpathlineto{\pgfqpoint{4.844623in}{0.739656in}}%
\pgfpathlineto{\pgfqpoint{4.844327in}{0.739656in}}%
\pgfpathlineto{\pgfqpoint{4.844031in}{0.739656in}}%
\pgfpathlineto{\pgfqpoint{4.843735in}{0.739656in}}%
\pgfpathlineto{\pgfqpoint{4.843439in}{0.739656in}}%
\pgfpathlineto{\pgfqpoint{4.843143in}{0.739656in}}%
\pgfpathlineto{\pgfqpoint{4.842847in}{0.739656in}}%
\pgfpathlineto{\pgfqpoint{4.842551in}{0.739656in}}%
\pgfpathlineto{\pgfqpoint{4.842255in}{0.739656in}}%
\pgfpathlineto{\pgfqpoint{4.841959in}{0.739656in}}%
\pgfpathlineto{\pgfqpoint{4.841663in}{0.739656in}}%
\pgfpathlineto{\pgfqpoint{4.841367in}{0.739656in}}%
\pgfpathlineto{\pgfqpoint{4.841071in}{0.739656in}}%
\pgfpathlineto{\pgfqpoint{4.840775in}{0.739656in}}%
\pgfpathlineto{\pgfqpoint{4.840479in}{0.739656in}}%
\pgfpathlineto{\pgfqpoint{4.840183in}{0.739656in}}%
\pgfpathlineto{\pgfqpoint{4.839887in}{0.739656in}}%
\pgfpathlineto{\pgfqpoint{4.839591in}{0.739656in}}%
\pgfpathlineto{\pgfqpoint{4.839295in}{0.739656in}}%
\pgfpathlineto{\pgfqpoint{4.838999in}{0.739656in}}%
\pgfpathlineto{\pgfqpoint{4.838703in}{0.739656in}}%
\pgfpathlineto{\pgfqpoint{4.838407in}{0.739656in}}%
\pgfpathlineto{\pgfqpoint{4.838111in}{0.739656in}}%
\pgfpathlineto{\pgfqpoint{4.837815in}{0.739656in}}%
\pgfpathlineto{\pgfqpoint{4.837519in}{0.739656in}}%
\pgfpathlineto{\pgfqpoint{4.837223in}{0.739656in}}%
\pgfpathlineto{\pgfqpoint{4.836927in}{0.739656in}}%
\pgfpathlineto{\pgfqpoint{4.836631in}{0.739656in}}%
\pgfpathlineto{\pgfqpoint{4.836335in}{0.739656in}}%
\pgfpathlineto{\pgfqpoint{4.836039in}{0.739656in}}%
\pgfpathlineto{\pgfqpoint{4.835743in}{0.739656in}}%
\pgfpathlineto{\pgfqpoint{4.835447in}{0.739656in}}%
\pgfpathlineto{\pgfqpoint{4.835151in}{0.739656in}}%
\pgfpathlineto{\pgfqpoint{4.834855in}{0.739656in}}%
\pgfpathlineto{\pgfqpoint{4.834559in}{0.739656in}}%
\pgfpathlineto{\pgfqpoint{4.834263in}{0.739656in}}%
\pgfpathlineto{\pgfqpoint{4.833967in}{0.739656in}}%
\pgfpathlineto{\pgfqpoint{4.833671in}{0.739656in}}%
\pgfpathlineto{\pgfqpoint{4.833375in}{0.739656in}}%
\pgfpathlineto{\pgfqpoint{4.833079in}{0.739656in}}%
\pgfpathlineto{\pgfqpoint{4.832783in}{0.739656in}}%
\pgfpathlineto{\pgfqpoint{4.832487in}{0.739656in}}%
\pgfpathlineto{\pgfqpoint{4.832191in}{0.739656in}}%
\pgfpathlineto{\pgfqpoint{4.831895in}{0.739656in}}%
\pgfpathlineto{\pgfqpoint{4.831599in}{0.739656in}}%
\pgfpathlineto{\pgfqpoint{4.831303in}{0.739656in}}%
\pgfpathlineto{\pgfqpoint{4.831007in}{0.739656in}}%
\pgfpathlineto{\pgfqpoint{4.830711in}{0.739656in}}%
\pgfpathlineto{\pgfqpoint{4.830415in}{0.739656in}}%
\pgfpathlineto{\pgfqpoint{4.830119in}{0.739656in}}%
\pgfpathlineto{\pgfqpoint{4.829823in}{0.739656in}}%
\pgfpathlineto{\pgfqpoint{4.829527in}{0.739656in}}%
\pgfpathlineto{\pgfqpoint{4.829231in}{0.739656in}}%
\pgfpathlineto{\pgfqpoint{4.828935in}{0.739656in}}%
\pgfpathlineto{\pgfqpoint{4.828639in}{0.739656in}}%
\pgfpathlineto{\pgfqpoint{4.828343in}{0.739656in}}%
\pgfpathlineto{\pgfqpoint{4.828047in}{0.739656in}}%
\pgfpathlineto{\pgfqpoint{4.827751in}{0.739656in}}%
\pgfpathlineto{\pgfqpoint{4.827454in}{0.739656in}}%
\pgfpathlineto{\pgfqpoint{4.827158in}{0.739656in}}%
\pgfpathlineto{\pgfqpoint{4.826862in}{0.739656in}}%
\pgfpathlineto{\pgfqpoint{4.826566in}{0.739656in}}%
\pgfpathlineto{\pgfqpoint{4.826270in}{0.739656in}}%
\pgfpathlineto{\pgfqpoint{4.825974in}{0.739656in}}%
\pgfpathlineto{\pgfqpoint{4.825678in}{0.739656in}}%
\pgfpathlineto{\pgfqpoint{4.825382in}{0.739656in}}%
\pgfpathlineto{\pgfqpoint{4.825086in}{0.739656in}}%
\pgfpathlineto{\pgfqpoint{4.824790in}{0.739656in}}%
\pgfpathlineto{\pgfqpoint{4.824494in}{0.739656in}}%
\pgfpathlineto{\pgfqpoint{4.824198in}{0.739656in}}%
\pgfpathlineto{\pgfqpoint{4.823902in}{0.739656in}}%
\pgfpathlineto{\pgfqpoint{4.823606in}{0.739656in}}%
\pgfpathlineto{\pgfqpoint{4.823310in}{0.739656in}}%
\pgfpathlineto{\pgfqpoint{4.823014in}{0.739656in}}%
\pgfpathlineto{\pgfqpoint{4.822718in}{0.739656in}}%
\pgfpathlineto{\pgfqpoint{4.822422in}{0.739656in}}%
\pgfpathlineto{\pgfqpoint{4.822126in}{0.739656in}}%
\pgfpathlineto{\pgfqpoint{4.821830in}{0.739656in}}%
\pgfpathlineto{\pgfqpoint{4.821534in}{0.739656in}}%
\pgfpathlineto{\pgfqpoint{4.821238in}{0.739656in}}%
\pgfpathlineto{\pgfqpoint{4.820942in}{0.739656in}}%
\pgfpathlineto{\pgfqpoint{4.820646in}{0.739656in}}%
\pgfpathlineto{\pgfqpoint{4.820350in}{0.739656in}}%
\pgfpathlineto{\pgfqpoint{4.820054in}{0.739656in}}%
\pgfpathlineto{\pgfqpoint{4.819758in}{0.739656in}}%
\pgfpathlineto{\pgfqpoint{4.819462in}{0.739656in}}%
\pgfpathlineto{\pgfqpoint{4.819166in}{0.739656in}}%
\pgfpathlineto{\pgfqpoint{4.818870in}{0.739656in}}%
\pgfpathlineto{\pgfqpoint{4.818574in}{0.739656in}}%
\pgfpathlineto{\pgfqpoint{4.818278in}{0.739656in}}%
\pgfpathlineto{\pgfqpoint{4.817982in}{0.739656in}}%
\pgfpathlineto{\pgfqpoint{4.817686in}{0.739656in}}%
\pgfpathlineto{\pgfqpoint{4.817390in}{0.739656in}}%
\pgfpathlineto{\pgfqpoint{4.817094in}{0.739656in}}%
\pgfpathlineto{\pgfqpoint{4.816798in}{0.739656in}}%
\pgfpathlineto{\pgfqpoint{4.816502in}{0.739656in}}%
\pgfpathlineto{\pgfqpoint{4.816206in}{0.739656in}}%
\pgfpathlineto{\pgfqpoint{4.815910in}{0.739656in}}%
\pgfpathlineto{\pgfqpoint{4.815614in}{0.739656in}}%
\pgfpathlineto{\pgfqpoint{4.815318in}{0.739656in}}%
\pgfpathlineto{\pgfqpoint{4.815022in}{0.739656in}}%
\pgfpathlineto{\pgfqpoint{4.814726in}{0.739656in}}%
\pgfpathlineto{\pgfqpoint{4.814430in}{0.739656in}}%
\pgfpathlineto{\pgfqpoint{4.814134in}{0.739656in}}%
\pgfpathlineto{\pgfqpoint{4.813838in}{0.739656in}}%
\pgfpathlineto{\pgfqpoint{4.813542in}{0.739656in}}%
\pgfpathlineto{\pgfqpoint{4.813246in}{0.739656in}}%
\pgfpathlineto{\pgfqpoint{4.812950in}{0.739656in}}%
\pgfpathlineto{\pgfqpoint{4.812654in}{0.739656in}}%
\pgfpathlineto{\pgfqpoint{4.812358in}{0.739656in}}%
\pgfpathlineto{\pgfqpoint{4.812062in}{0.739656in}}%
\pgfpathlineto{\pgfqpoint{4.811766in}{0.739656in}}%
\pgfpathlineto{\pgfqpoint{4.811470in}{0.739656in}}%
\pgfpathlineto{\pgfqpoint{4.811174in}{0.739656in}}%
\pgfpathlineto{\pgfqpoint{4.810878in}{0.739656in}}%
\pgfpathlineto{\pgfqpoint{4.810582in}{0.739656in}}%
\pgfpathlineto{\pgfqpoint{4.810286in}{0.739656in}}%
\pgfpathlineto{\pgfqpoint{4.809990in}{0.739656in}}%
\pgfpathlineto{\pgfqpoint{4.809694in}{0.739656in}}%
\pgfpathlineto{\pgfqpoint{4.809398in}{0.739656in}}%
\pgfpathlineto{\pgfqpoint{4.809102in}{0.739656in}}%
\pgfpathlineto{\pgfqpoint{4.808806in}{0.739656in}}%
\pgfpathlineto{\pgfqpoint{4.808510in}{0.739656in}}%
\pgfpathlineto{\pgfqpoint{4.808214in}{0.739656in}}%
\pgfpathlineto{\pgfqpoint{4.807918in}{0.739656in}}%
\pgfpathlineto{\pgfqpoint{4.807622in}{0.739656in}}%
\pgfpathlineto{\pgfqpoint{4.807326in}{0.739656in}}%
\pgfpathlineto{\pgfqpoint{4.807030in}{0.739656in}}%
\pgfpathlineto{\pgfqpoint{4.806734in}{0.739656in}}%
\pgfpathlineto{\pgfqpoint{4.806438in}{0.739656in}}%
\pgfpathlineto{\pgfqpoint{4.806142in}{0.739656in}}%
\pgfpathlineto{\pgfqpoint{4.805846in}{0.739656in}}%
\pgfpathlineto{\pgfqpoint{4.805550in}{0.739656in}}%
\pgfpathlineto{\pgfqpoint{4.805254in}{0.739656in}}%
\pgfpathlineto{\pgfqpoint{4.804958in}{0.739656in}}%
\pgfpathlineto{\pgfqpoint{4.804662in}{0.739656in}}%
\pgfpathlineto{\pgfqpoint{4.804366in}{0.739656in}}%
\pgfpathlineto{\pgfqpoint{4.804070in}{0.739656in}}%
\pgfpathlineto{\pgfqpoint{4.803774in}{0.739656in}}%
\pgfpathlineto{\pgfqpoint{4.803478in}{0.739656in}}%
\pgfpathlineto{\pgfqpoint{4.803182in}{0.739656in}}%
\pgfpathlineto{\pgfqpoint{4.802886in}{0.739656in}}%
\pgfpathlineto{\pgfqpoint{4.802590in}{0.739656in}}%
\pgfpathlineto{\pgfqpoint{4.802294in}{0.739656in}}%
\pgfpathlineto{\pgfqpoint{4.801998in}{0.739656in}}%
\pgfpathlineto{\pgfqpoint{4.801702in}{0.739656in}}%
\pgfpathlineto{\pgfqpoint{4.801406in}{0.739656in}}%
\pgfpathlineto{\pgfqpoint{4.801110in}{0.739656in}}%
\pgfpathlineto{\pgfqpoint{4.800814in}{0.739656in}}%
\pgfpathlineto{\pgfqpoint{4.800518in}{0.739656in}}%
\pgfpathlineto{\pgfqpoint{4.800222in}{0.739656in}}%
\pgfpathlineto{\pgfqpoint{4.799926in}{0.739656in}}%
\pgfpathlineto{\pgfqpoint{4.799630in}{0.739656in}}%
\pgfpathlineto{\pgfqpoint{4.799334in}{0.739656in}}%
\pgfpathlineto{\pgfqpoint{4.799038in}{0.739656in}}%
\pgfpathlineto{\pgfqpoint{4.798742in}{0.739656in}}%
\pgfpathlineto{\pgfqpoint{4.798446in}{0.739656in}}%
\pgfpathlineto{\pgfqpoint{4.798150in}{0.739656in}}%
\pgfpathlineto{\pgfqpoint{4.797854in}{0.739656in}}%
\pgfpathlineto{\pgfqpoint{4.797558in}{0.739656in}}%
\pgfpathlineto{\pgfqpoint{4.797262in}{0.739656in}}%
\pgfpathlineto{\pgfqpoint{4.796966in}{0.739656in}}%
\pgfpathlineto{\pgfqpoint{4.796670in}{0.739656in}}%
\pgfpathlineto{\pgfqpoint{4.796374in}{0.739656in}}%
\pgfpathlineto{\pgfqpoint{4.796078in}{0.739656in}}%
\pgfpathlineto{\pgfqpoint{4.795782in}{0.739656in}}%
\pgfpathlineto{\pgfqpoint{4.795486in}{0.739656in}}%
\pgfpathlineto{\pgfqpoint{4.795190in}{0.739656in}}%
\pgfpathlineto{\pgfqpoint{4.794894in}{0.739656in}}%
\pgfpathlineto{\pgfqpoint{4.794598in}{0.739656in}}%
\pgfpathlineto{\pgfqpoint{4.794302in}{0.739656in}}%
\pgfpathlineto{\pgfqpoint{4.794006in}{0.739656in}}%
\pgfpathlineto{\pgfqpoint{4.793710in}{0.739656in}}%
\pgfpathlineto{\pgfqpoint{4.793414in}{0.739656in}}%
\pgfpathlineto{\pgfqpoint{4.793118in}{0.739656in}}%
\pgfpathlineto{\pgfqpoint{4.792822in}{0.739656in}}%
\pgfpathlineto{\pgfqpoint{4.792526in}{0.739656in}}%
\pgfpathlineto{\pgfqpoint{4.792230in}{0.739656in}}%
\pgfpathlineto{\pgfqpoint{4.791934in}{0.739656in}}%
\pgfpathlineto{\pgfqpoint{4.791638in}{0.739656in}}%
\pgfpathlineto{\pgfqpoint{4.791342in}{0.739656in}}%
\pgfpathlineto{\pgfqpoint{4.791046in}{0.739656in}}%
\pgfpathlineto{\pgfqpoint{4.790750in}{0.739656in}}%
\pgfpathlineto{\pgfqpoint{4.790454in}{0.739656in}}%
\pgfpathlineto{\pgfqpoint{4.790158in}{0.739656in}}%
\pgfpathlineto{\pgfqpoint{4.789862in}{0.739656in}}%
\pgfpathlineto{\pgfqpoint{4.789566in}{0.739656in}}%
\pgfpathlineto{\pgfqpoint{4.789270in}{0.739656in}}%
\pgfpathlineto{\pgfqpoint{4.788974in}{0.739656in}}%
\pgfpathlineto{\pgfqpoint{4.788678in}{0.739656in}}%
\pgfpathlineto{\pgfqpoint{4.788382in}{0.739656in}}%
\pgfpathlineto{\pgfqpoint{4.788086in}{0.739656in}}%
\pgfpathlineto{\pgfqpoint{4.787790in}{0.739656in}}%
\pgfpathlineto{\pgfqpoint{4.787494in}{0.739656in}}%
\pgfpathlineto{\pgfqpoint{4.787198in}{0.739656in}}%
\pgfpathlineto{\pgfqpoint{4.786902in}{0.739656in}}%
\pgfpathlineto{\pgfqpoint{4.786606in}{0.739656in}}%
\pgfpathlineto{\pgfqpoint{4.786310in}{0.739656in}}%
\pgfpathlineto{\pgfqpoint{4.786014in}{0.739656in}}%
\pgfpathlineto{\pgfqpoint{4.785718in}{0.739656in}}%
\pgfpathlineto{\pgfqpoint{4.785422in}{0.739656in}}%
\pgfpathlineto{\pgfqpoint{4.785126in}{0.739656in}}%
\pgfpathlineto{\pgfqpoint{4.784830in}{0.739656in}}%
\pgfpathlineto{\pgfqpoint{4.784534in}{0.739656in}}%
\pgfpathlineto{\pgfqpoint{4.784238in}{0.739656in}}%
\pgfpathlineto{\pgfqpoint{4.783942in}{0.739656in}}%
\pgfpathlineto{\pgfqpoint{4.783646in}{0.739656in}}%
\pgfpathlineto{\pgfqpoint{4.783350in}{0.739656in}}%
\pgfpathlineto{\pgfqpoint{4.783054in}{0.739656in}}%
\pgfpathlineto{\pgfqpoint{4.782758in}{0.739656in}}%
\pgfpathlineto{\pgfqpoint{4.782462in}{0.739656in}}%
\pgfpathlineto{\pgfqpoint{4.782166in}{0.739656in}}%
\pgfpathlineto{\pgfqpoint{4.781870in}{0.739656in}}%
\pgfpathlineto{\pgfqpoint{4.781574in}{0.739656in}}%
\pgfpathlineto{\pgfqpoint{4.781278in}{0.739656in}}%
\pgfpathlineto{\pgfqpoint{4.780982in}{0.739656in}}%
\pgfpathlineto{\pgfqpoint{4.780686in}{0.739656in}}%
\pgfpathlineto{\pgfqpoint{4.780390in}{0.739656in}}%
\pgfpathlineto{\pgfqpoint{4.780094in}{0.739656in}}%
\pgfpathlineto{\pgfqpoint{4.779798in}{0.739656in}}%
\pgfpathlineto{\pgfqpoint{4.779502in}{0.739656in}}%
\pgfpathlineto{\pgfqpoint{4.779206in}{0.739656in}}%
\pgfpathlineto{\pgfqpoint{4.778910in}{0.739656in}}%
\pgfpathlineto{\pgfqpoint{4.778614in}{0.739656in}}%
\pgfpathlineto{\pgfqpoint{4.778318in}{0.739656in}}%
\pgfpathlineto{\pgfqpoint{4.778022in}{0.739656in}}%
\pgfpathlineto{\pgfqpoint{4.777726in}{0.739656in}}%
\pgfpathlineto{\pgfqpoint{4.777430in}{0.739656in}}%
\pgfpathlineto{\pgfqpoint{4.777134in}{0.739656in}}%
\pgfpathlineto{\pgfqpoint{4.776838in}{0.739656in}}%
\pgfpathlineto{\pgfqpoint{4.776542in}{0.739656in}}%
\pgfpathlineto{\pgfqpoint{4.776246in}{0.739656in}}%
\pgfpathlineto{\pgfqpoint{4.775950in}{0.739656in}}%
\pgfpathlineto{\pgfqpoint{4.775654in}{0.739656in}}%
\pgfpathlineto{\pgfqpoint{4.775358in}{0.739656in}}%
\pgfpathlineto{\pgfqpoint{4.775062in}{0.739656in}}%
\pgfpathlineto{\pgfqpoint{4.774766in}{0.739656in}}%
\pgfpathlineto{\pgfqpoint{4.774470in}{0.739656in}}%
\pgfpathlineto{\pgfqpoint{4.774174in}{0.739656in}}%
\pgfpathlineto{\pgfqpoint{4.773878in}{0.739656in}}%
\pgfpathlineto{\pgfqpoint{4.773582in}{0.739656in}}%
\pgfpathlineto{\pgfqpoint{4.773286in}{0.739656in}}%
\pgfpathlineto{\pgfqpoint{4.772990in}{0.739656in}}%
\pgfpathlineto{\pgfqpoint{4.772694in}{0.739656in}}%
\pgfpathlineto{\pgfqpoint{4.772398in}{0.739656in}}%
\pgfpathlineto{\pgfqpoint{4.772102in}{0.739656in}}%
\pgfpathlineto{\pgfqpoint{4.771806in}{0.739656in}}%
\pgfpathlineto{\pgfqpoint{4.771510in}{0.739656in}}%
\pgfpathlineto{\pgfqpoint{4.771214in}{0.739656in}}%
\pgfpathlineto{\pgfqpoint{4.770918in}{0.739656in}}%
\pgfpathlineto{\pgfqpoint{4.770622in}{0.739656in}}%
\pgfpathlineto{\pgfqpoint{4.770326in}{0.739656in}}%
\pgfpathlineto{\pgfqpoint{4.770030in}{0.739656in}}%
\pgfpathlineto{\pgfqpoint{4.769734in}{0.739656in}}%
\pgfpathlineto{\pgfqpoint{4.769438in}{0.739656in}}%
\pgfpathlineto{\pgfqpoint{4.769142in}{0.739656in}}%
\pgfpathlineto{\pgfqpoint{4.768846in}{0.739656in}}%
\pgfpathlineto{\pgfqpoint{4.768550in}{0.739656in}}%
\pgfpathlineto{\pgfqpoint{4.768254in}{0.739656in}}%
\pgfpathlineto{\pgfqpoint{4.767958in}{0.739656in}}%
\pgfpathlineto{\pgfqpoint{4.767662in}{0.739656in}}%
\pgfpathlineto{\pgfqpoint{4.767366in}{0.739656in}}%
\pgfpathlineto{\pgfqpoint{4.767070in}{0.739656in}}%
\pgfpathlineto{\pgfqpoint{4.766774in}{0.739656in}}%
\pgfpathlineto{\pgfqpoint{4.766478in}{0.739656in}}%
\pgfpathlineto{\pgfqpoint{4.766182in}{0.739656in}}%
\pgfpathlineto{\pgfqpoint{4.765886in}{0.739656in}}%
\pgfpathlineto{\pgfqpoint{4.765590in}{0.739656in}}%
\pgfpathlineto{\pgfqpoint{4.765294in}{0.739656in}}%
\pgfpathlineto{\pgfqpoint{4.764998in}{0.739656in}}%
\pgfpathlineto{\pgfqpoint{4.764702in}{0.739656in}}%
\pgfpathlineto{\pgfqpoint{4.764406in}{0.739656in}}%
\pgfpathlineto{\pgfqpoint{4.764110in}{0.739656in}}%
\pgfpathlineto{\pgfqpoint{4.763814in}{0.739656in}}%
\pgfpathlineto{\pgfqpoint{4.763518in}{0.739656in}}%
\pgfpathlineto{\pgfqpoint{4.763222in}{0.739656in}}%
\pgfpathlineto{\pgfqpoint{4.762926in}{0.739656in}}%
\pgfpathlineto{\pgfqpoint{4.762630in}{0.739656in}}%
\pgfpathlineto{\pgfqpoint{4.762334in}{0.739656in}}%
\pgfpathlineto{\pgfqpoint{4.762038in}{0.739656in}}%
\pgfpathlineto{\pgfqpoint{4.761742in}{0.739656in}}%
\pgfpathlineto{\pgfqpoint{4.761446in}{0.739656in}}%
\pgfpathlineto{\pgfqpoint{4.761150in}{0.739656in}}%
\pgfpathlineto{\pgfqpoint{4.760854in}{0.739656in}}%
\pgfpathlineto{\pgfqpoint{4.760558in}{0.739656in}}%
\pgfpathlineto{\pgfqpoint{4.760261in}{0.739656in}}%
\pgfpathlineto{\pgfqpoint{4.759965in}{0.739656in}}%
\pgfpathlineto{\pgfqpoint{4.759669in}{0.739656in}}%
\pgfpathlineto{\pgfqpoint{4.759373in}{0.739656in}}%
\pgfpathlineto{\pgfqpoint{4.759077in}{0.739656in}}%
\pgfpathlineto{\pgfqpoint{4.758781in}{0.739656in}}%
\pgfpathlineto{\pgfqpoint{4.758485in}{0.739656in}}%
\pgfpathlineto{\pgfqpoint{4.758189in}{0.739656in}}%
\pgfpathlineto{\pgfqpoint{4.757893in}{0.739656in}}%
\pgfpathlineto{\pgfqpoint{4.757597in}{0.739656in}}%
\pgfpathlineto{\pgfqpoint{4.757301in}{0.739656in}}%
\pgfpathlineto{\pgfqpoint{4.757005in}{0.739656in}}%
\pgfpathlineto{\pgfqpoint{4.756709in}{0.739656in}}%
\pgfpathlineto{\pgfqpoint{4.756413in}{0.739656in}}%
\pgfpathlineto{\pgfqpoint{4.756117in}{0.739656in}}%
\pgfpathlineto{\pgfqpoint{4.755821in}{0.739656in}}%
\pgfpathlineto{\pgfqpoint{4.755525in}{0.739656in}}%
\pgfpathlineto{\pgfqpoint{4.755229in}{0.739656in}}%
\pgfpathlineto{\pgfqpoint{4.754933in}{0.739656in}}%
\pgfpathlineto{\pgfqpoint{4.754637in}{0.739656in}}%
\pgfpathlineto{\pgfqpoint{4.754341in}{0.739656in}}%
\pgfpathlineto{\pgfqpoint{4.754045in}{0.739656in}}%
\pgfpathlineto{\pgfqpoint{4.753749in}{0.739656in}}%
\pgfpathlineto{\pgfqpoint{4.753453in}{0.739656in}}%
\pgfpathlineto{\pgfqpoint{4.753157in}{0.739656in}}%
\pgfpathlineto{\pgfqpoint{4.752861in}{0.739656in}}%
\pgfpathlineto{\pgfqpoint{4.752565in}{0.739656in}}%
\pgfpathlineto{\pgfqpoint{4.752269in}{0.739656in}}%
\pgfpathlineto{\pgfqpoint{4.751973in}{0.739656in}}%
\pgfpathlineto{\pgfqpoint{4.751677in}{0.739656in}}%
\pgfpathlineto{\pgfqpoint{4.751381in}{0.739656in}}%
\pgfpathlineto{\pgfqpoint{4.751085in}{0.739656in}}%
\pgfpathlineto{\pgfqpoint{4.750789in}{0.739656in}}%
\pgfpathlineto{\pgfqpoint{4.750493in}{0.739656in}}%
\pgfpathlineto{\pgfqpoint{4.750197in}{0.739656in}}%
\pgfpathlineto{\pgfqpoint{4.749901in}{0.739656in}}%
\pgfpathlineto{\pgfqpoint{4.749605in}{0.739656in}}%
\pgfpathlineto{\pgfqpoint{4.749309in}{0.739656in}}%
\pgfpathlineto{\pgfqpoint{4.749013in}{0.739656in}}%
\pgfpathlineto{\pgfqpoint{4.748717in}{0.739656in}}%
\pgfpathlineto{\pgfqpoint{4.748421in}{0.739656in}}%
\pgfpathlineto{\pgfqpoint{4.748125in}{0.739656in}}%
\pgfpathlineto{\pgfqpoint{4.747829in}{0.739656in}}%
\pgfpathlineto{\pgfqpoint{4.747533in}{0.739656in}}%
\pgfpathlineto{\pgfqpoint{4.747237in}{0.739656in}}%
\pgfpathlineto{\pgfqpoint{4.746941in}{0.739656in}}%
\pgfpathlineto{\pgfqpoint{4.746645in}{0.739656in}}%
\pgfpathlineto{\pgfqpoint{4.746349in}{0.739656in}}%
\pgfpathlineto{\pgfqpoint{4.746053in}{0.739656in}}%
\pgfpathlineto{\pgfqpoint{4.745757in}{0.739656in}}%
\pgfpathlineto{\pgfqpoint{4.745461in}{0.739656in}}%
\pgfpathlineto{\pgfqpoint{4.745165in}{0.739656in}}%
\pgfpathlineto{\pgfqpoint{4.744869in}{0.739656in}}%
\pgfpathlineto{\pgfqpoint{4.744573in}{0.739656in}}%
\pgfpathlineto{\pgfqpoint{4.744277in}{0.739656in}}%
\pgfpathlineto{\pgfqpoint{4.743981in}{0.739656in}}%
\pgfpathlineto{\pgfqpoint{4.743685in}{0.739656in}}%
\pgfpathlineto{\pgfqpoint{4.743389in}{0.739656in}}%
\pgfpathlineto{\pgfqpoint{4.743093in}{0.739656in}}%
\pgfpathlineto{\pgfqpoint{4.742797in}{0.739656in}}%
\pgfpathlineto{\pgfqpoint{4.742501in}{0.739656in}}%
\pgfpathlineto{\pgfqpoint{4.742205in}{0.739656in}}%
\pgfpathlineto{\pgfqpoint{4.741909in}{0.739656in}}%
\pgfpathlineto{\pgfqpoint{4.741613in}{0.739656in}}%
\pgfpathlineto{\pgfqpoint{4.741317in}{0.739656in}}%
\pgfpathlineto{\pgfqpoint{4.741021in}{0.739656in}}%
\pgfpathlineto{\pgfqpoint{4.740725in}{0.739656in}}%
\pgfpathlineto{\pgfqpoint{4.740429in}{0.739656in}}%
\pgfpathlineto{\pgfqpoint{4.740133in}{0.739656in}}%
\pgfpathlineto{\pgfqpoint{4.739837in}{0.739656in}}%
\pgfpathlineto{\pgfqpoint{4.739541in}{0.739656in}}%
\pgfpathlineto{\pgfqpoint{4.739245in}{0.739656in}}%
\pgfpathlineto{\pgfqpoint{4.738949in}{0.739656in}}%
\pgfpathlineto{\pgfqpoint{4.738653in}{0.739656in}}%
\pgfpathlineto{\pgfqpoint{4.738357in}{0.739656in}}%
\pgfpathlineto{\pgfqpoint{4.738061in}{0.739656in}}%
\pgfpathlineto{\pgfqpoint{4.737765in}{0.739656in}}%
\pgfpathlineto{\pgfqpoint{4.737469in}{0.739656in}}%
\pgfpathlineto{\pgfqpoint{4.737173in}{0.739656in}}%
\pgfpathlineto{\pgfqpoint{4.736877in}{0.739656in}}%
\pgfpathlineto{\pgfqpoint{4.736581in}{0.739656in}}%
\pgfpathlineto{\pgfqpoint{4.736285in}{0.739656in}}%
\pgfpathlineto{\pgfqpoint{4.735989in}{0.739656in}}%
\pgfpathlineto{\pgfqpoint{4.735693in}{0.739656in}}%
\pgfpathlineto{\pgfqpoint{4.735397in}{0.739656in}}%
\pgfpathlineto{\pgfqpoint{4.735101in}{0.739656in}}%
\pgfpathlineto{\pgfqpoint{4.734805in}{0.739656in}}%
\pgfpathlineto{\pgfqpoint{4.734509in}{0.739656in}}%
\pgfpathlineto{\pgfqpoint{4.734213in}{0.739656in}}%
\pgfpathlineto{\pgfqpoint{4.733917in}{0.739656in}}%
\pgfpathlineto{\pgfqpoint{4.733621in}{0.739656in}}%
\pgfpathlineto{\pgfqpoint{4.733325in}{0.739656in}}%
\pgfpathlineto{\pgfqpoint{4.733029in}{0.739656in}}%
\pgfpathlineto{\pgfqpoint{4.732733in}{0.739656in}}%
\pgfpathlineto{\pgfqpoint{4.732437in}{0.739656in}}%
\pgfpathlineto{\pgfqpoint{4.732141in}{0.739656in}}%
\pgfpathlineto{\pgfqpoint{4.731845in}{0.739656in}}%
\pgfpathlineto{\pgfqpoint{4.731549in}{0.739656in}}%
\pgfpathlineto{\pgfqpoint{4.731253in}{0.739656in}}%
\pgfpathlineto{\pgfqpoint{4.730957in}{0.739656in}}%
\pgfpathlineto{\pgfqpoint{4.730661in}{0.739656in}}%
\pgfpathlineto{\pgfqpoint{4.730365in}{0.739656in}}%
\pgfpathlineto{\pgfqpoint{4.730069in}{0.739656in}}%
\pgfpathlineto{\pgfqpoint{4.729773in}{0.739656in}}%
\pgfpathlineto{\pgfqpoint{4.729477in}{0.739656in}}%
\pgfpathlineto{\pgfqpoint{4.729181in}{0.739656in}}%
\pgfpathlineto{\pgfqpoint{4.728885in}{0.739656in}}%
\pgfpathlineto{\pgfqpoint{4.728589in}{0.739656in}}%
\pgfpathlineto{\pgfqpoint{4.728293in}{0.739656in}}%
\pgfpathlineto{\pgfqpoint{4.727997in}{0.739656in}}%
\pgfpathlineto{\pgfqpoint{4.727701in}{0.739656in}}%
\pgfpathlineto{\pgfqpoint{4.727405in}{0.739656in}}%
\pgfpathlineto{\pgfqpoint{4.727109in}{0.739656in}}%
\pgfpathlineto{\pgfqpoint{4.726813in}{0.739656in}}%
\pgfpathlineto{\pgfqpoint{4.726517in}{0.739656in}}%
\pgfpathlineto{\pgfqpoint{4.726221in}{0.739656in}}%
\pgfpathlineto{\pgfqpoint{4.725925in}{0.739656in}}%
\pgfpathlineto{\pgfqpoint{4.725629in}{0.739656in}}%
\pgfpathlineto{\pgfqpoint{4.725333in}{0.739656in}}%
\pgfpathlineto{\pgfqpoint{4.725037in}{0.739656in}}%
\pgfpathlineto{\pgfqpoint{4.724741in}{0.739656in}}%
\pgfpathlineto{\pgfqpoint{4.724445in}{0.739656in}}%
\pgfpathlineto{\pgfqpoint{4.724149in}{0.739656in}}%
\pgfpathlineto{\pgfqpoint{4.723853in}{0.739656in}}%
\pgfpathlineto{\pgfqpoint{4.723557in}{0.739656in}}%
\pgfpathlineto{\pgfqpoint{4.723261in}{0.739656in}}%
\pgfpathlineto{\pgfqpoint{4.722965in}{0.739656in}}%
\pgfpathlineto{\pgfqpoint{4.722669in}{0.739656in}}%
\pgfpathlineto{\pgfqpoint{4.722373in}{0.739656in}}%
\pgfpathlineto{\pgfqpoint{4.722077in}{0.739656in}}%
\pgfpathlineto{\pgfqpoint{4.721781in}{0.739656in}}%
\pgfpathlineto{\pgfqpoint{4.721485in}{0.739656in}}%
\pgfpathlineto{\pgfqpoint{4.721189in}{0.739656in}}%
\pgfpathlineto{\pgfqpoint{4.720893in}{0.739656in}}%
\pgfpathlineto{\pgfqpoint{4.720597in}{0.739656in}}%
\pgfpathlineto{\pgfqpoint{4.720301in}{0.739656in}}%
\pgfpathlineto{\pgfqpoint{4.720005in}{0.739656in}}%
\pgfpathlineto{\pgfqpoint{4.719709in}{0.739656in}}%
\pgfpathlineto{\pgfqpoint{4.719413in}{0.739656in}}%
\pgfpathlineto{\pgfqpoint{4.719117in}{0.739656in}}%
\pgfpathlineto{\pgfqpoint{4.718821in}{0.739656in}}%
\pgfpathlineto{\pgfqpoint{4.718525in}{0.739656in}}%
\pgfpathlineto{\pgfqpoint{4.718229in}{0.739656in}}%
\pgfpathlineto{\pgfqpoint{4.717933in}{0.739656in}}%
\pgfpathlineto{\pgfqpoint{4.717637in}{0.739656in}}%
\pgfpathlineto{\pgfqpoint{4.717341in}{0.739656in}}%
\pgfpathlineto{\pgfqpoint{4.717045in}{0.739656in}}%
\pgfpathlineto{\pgfqpoint{4.716749in}{0.739656in}}%
\pgfpathlineto{\pgfqpoint{4.716453in}{0.739656in}}%
\pgfpathlineto{\pgfqpoint{4.716157in}{0.739656in}}%
\pgfpathlineto{\pgfqpoint{4.715861in}{0.739656in}}%
\pgfpathlineto{\pgfqpoint{4.715565in}{0.739656in}}%
\pgfpathlineto{\pgfqpoint{4.715269in}{0.739656in}}%
\pgfpathlineto{\pgfqpoint{4.714973in}{0.739656in}}%
\pgfpathlineto{\pgfqpoint{4.714677in}{0.739656in}}%
\pgfpathlineto{\pgfqpoint{4.714381in}{0.739656in}}%
\pgfpathlineto{\pgfqpoint{4.714085in}{0.739656in}}%
\pgfpathlineto{\pgfqpoint{4.713789in}{0.739656in}}%
\pgfpathlineto{\pgfqpoint{4.713493in}{0.739656in}}%
\pgfpathlineto{\pgfqpoint{4.713197in}{0.739656in}}%
\pgfpathlineto{\pgfqpoint{4.712901in}{0.739656in}}%
\pgfpathlineto{\pgfqpoint{4.712605in}{0.739656in}}%
\pgfpathlineto{\pgfqpoint{4.712309in}{0.739656in}}%
\pgfpathlineto{\pgfqpoint{4.712013in}{0.739656in}}%
\pgfpathlineto{\pgfqpoint{4.711717in}{0.739656in}}%
\pgfpathlineto{\pgfqpoint{4.711421in}{0.739656in}}%
\pgfpathlineto{\pgfqpoint{4.711125in}{0.739656in}}%
\pgfpathlineto{\pgfqpoint{4.710829in}{0.739656in}}%
\pgfpathlineto{\pgfqpoint{4.710533in}{0.739656in}}%
\pgfpathlineto{\pgfqpoint{4.710237in}{0.739656in}}%
\pgfpathlineto{\pgfqpoint{4.709941in}{0.739656in}}%
\pgfpathlineto{\pgfqpoint{4.709645in}{0.739656in}}%
\pgfpathlineto{\pgfqpoint{4.709349in}{0.739656in}}%
\pgfpathlineto{\pgfqpoint{4.709053in}{0.739656in}}%
\pgfpathlineto{\pgfqpoint{4.708757in}{0.739656in}}%
\pgfpathlineto{\pgfqpoint{4.708461in}{0.739656in}}%
\pgfpathlineto{\pgfqpoint{4.708165in}{0.739656in}}%
\pgfpathlineto{\pgfqpoint{4.707869in}{0.739656in}}%
\pgfpathlineto{\pgfqpoint{4.707573in}{0.739656in}}%
\pgfpathlineto{\pgfqpoint{4.707277in}{0.739656in}}%
\pgfpathlineto{\pgfqpoint{4.706981in}{0.739656in}}%
\pgfpathlineto{\pgfqpoint{4.706685in}{0.739656in}}%
\pgfpathlineto{\pgfqpoint{4.706389in}{0.739656in}}%
\pgfpathlineto{\pgfqpoint{4.706093in}{0.739656in}}%
\pgfpathlineto{\pgfqpoint{4.705797in}{0.739656in}}%
\pgfpathlineto{\pgfqpoint{4.705501in}{0.739656in}}%
\pgfpathlineto{\pgfqpoint{4.705205in}{0.739656in}}%
\pgfpathlineto{\pgfqpoint{4.704909in}{0.739656in}}%
\pgfpathlineto{\pgfqpoint{4.704613in}{0.739656in}}%
\pgfpathlineto{\pgfqpoint{4.704317in}{0.739656in}}%
\pgfpathlineto{\pgfqpoint{4.704021in}{0.739656in}}%
\pgfpathlineto{\pgfqpoint{4.703725in}{0.739656in}}%
\pgfpathlineto{\pgfqpoint{4.703429in}{0.739656in}}%
\pgfpathlineto{\pgfqpoint{4.703133in}{0.739656in}}%
\pgfpathlineto{\pgfqpoint{4.702837in}{0.739656in}}%
\pgfpathlineto{\pgfqpoint{4.702541in}{0.739656in}}%
\pgfpathlineto{\pgfqpoint{4.702245in}{0.739656in}}%
\pgfpathlineto{\pgfqpoint{4.701949in}{0.739656in}}%
\pgfpathlineto{\pgfqpoint{4.701653in}{0.739656in}}%
\pgfpathlineto{\pgfqpoint{4.701357in}{0.739656in}}%
\pgfpathlineto{\pgfqpoint{4.701061in}{0.739656in}}%
\pgfpathlineto{\pgfqpoint{4.700765in}{0.739656in}}%
\pgfpathlineto{\pgfqpoint{4.700469in}{0.739656in}}%
\pgfpathlineto{\pgfqpoint{4.700173in}{0.739656in}}%
\pgfpathlineto{\pgfqpoint{4.699877in}{0.739656in}}%
\pgfpathlineto{\pgfqpoint{4.699581in}{0.739656in}}%
\pgfpathlineto{\pgfqpoint{4.699285in}{0.739656in}}%
\pgfpathlineto{\pgfqpoint{4.698989in}{0.739656in}}%
\pgfpathlineto{\pgfqpoint{4.698693in}{0.739656in}}%
\pgfpathlineto{\pgfqpoint{4.698397in}{0.739656in}}%
\pgfpathlineto{\pgfqpoint{4.698101in}{0.739656in}}%
\pgfpathlineto{\pgfqpoint{4.697805in}{0.739656in}}%
\pgfpathlineto{\pgfqpoint{4.697509in}{0.739656in}}%
\pgfpathlineto{\pgfqpoint{4.697213in}{0.739656in}}%
\pgfpathlineto{\pgfqpoint{4.696917in}{0.739656in}}%
\pgfpathlineto{\pgfqpoint{4.696621in}{0.739656in}}%
\pgfpathlineto{\pgfqpoint{4.696325in}{0.739656in}}%
\pgfpathlineto{\pgfqpoint{4.696029in}{0.739656in}}%
\pgfpathlineto{\pgfqpoint{4.695733in}{0.739656in}}%
\pgfpathlineto{\pgfqpoint{4.695437in}{0.739656in}}%
\pgfpathlineto{\pgfqpoint{4.695141in}{0.739656in}}%
\pgfpathlineto{\pgfqpoint{4.694845in}{0.739656in}}%
\pgfpathlineto{\pgfqpoint{4.694549in}{0.739656in}}%
\pgfpathlineto{\pgfqpoint{4.694253in}{0.739656in}}%
\pgfpathlineto{\pgfqpoint{4.693957in}{0.739656in}}%
\pgfpathlineto{\pgfqpoint{4.693661in}{0.739656in}}%
\pgfpathlineto{\pgfqpoint{4.693365in}{0.739656in}}%
\pgfpathlineto{\pgfqpoint{4.693069in}{0.739656in}}%
\pgfpathlineto{\pgfqpoint{4.692772in}{0.739656in}}%
\pgfpathlineto{\pgfqpoint{4.692476in}{0.739656in}}%
\pgfpathlineto{\pgfqpoint{4.692180in}{0.739656in}}%
\pgfpathlineto{\pgfqpoint{4.691884in}{0.739656in}}%
\pgfpathlineto{\pgfqpoint{4.691588in}{0.739656in}}%
\pgfpathlineto{\pgfqpoint{4.691292in}{0.739656in}}%
\pgfpathlineto{\pgfqpoint{4.690996in}{0.739656in}}%
\pgfpathlineto{\pgfqpoint{4.690700in}{0.739656in}}%
\pgfpathlineto{\pgfqpoint{4.690404in}{0.739656in}}%
\pgfpathlineto{\pgfqpoint{4.690108in}{0.739656in}}%
\pgfpathlineto{\pgfqpoint{4.689812in}{0.739656in}}%
\pgfpathlineto{\pgfqpoint{4.689516in}{0.739656in}}%
\pgfpathlineto{\pgfqpoint{4.689220in}{0.739656in}}%
\pgfpathlineto{\pgfqpoint{4.688924in}{0.739656in}}%
\pgfpathlineto{\pgfqpoint{4.688628in}{0.739656in}}%
\pgfpathlineto{\pgfqpoint{4.688332in}{0.739656in}}%
\pgfpathlineto{\pgfqpoint{4.688036in}{0.739656in}}%
\pgfpathlineto{\pgfqpoint{4.687740in}{0.739656in}}%
\pgfpathlineto{\pgfqpoint{4.687444in}{0.739656in}}%
\pgfpathlineto{\pgfqpoint{4.687148in}{0.739656in}}%
\pgfpathlineto{\pgfqpoint{4.686852in}{0.739656in}}%
\pgfpathlineto{\pgfqpoint{4.686556in}{0.739656in}}%
\pgfpathlineto{\pgfqpoint{4.686260in}{0.739656in}}%
\pgfpathlineto{\pgfqpoint{4.685964in}{0.739656in}}%
\pgfpathlineto{\pgfqpoint{4.685668in}{0.739656in}}%
\pgfpathlineto{\pgfqpoint{4.685372in}{0.739656in}}%
\pgfpathlineto{\pgfqpoint{4.685076in}{0.739656in}}%
\pgfpathlineto{\pgfqpoint{4.684780in}{0.739656in}}%
\pgfpathlineto{\pgfqpoint{4.684484in}{0.739656in}}%
\pgfpathlineto{\pgfqpoint{4.684188in}{0.739656in}}%
\pgfpathlineto{\pgfqpoint{4.683892in}{0.739656in}}%
\pgfpathlineto{\pgfqpoint{4.683596in}{0.739656in}}%
\pgfpathlineto{\pgfqpoint{4.683300in}{0.739656in}}%
\pgfpathlineto{\pgfqpoint{4.683004in}{0.739656in}}%
\pgfpathlineto{\pgfqpoint{4.682708in}{0.739656in}}%
\pgfpathlineto{\pgfqpoint{4.682412in}{0.739656in}}%
\pgfpathlineto{\pgfqpoint{4.682116in}{0.739656in}}%
\pgfpathlineto{\pgfqpoint{4.681820in}{0.739656in}}%
\pgfpathlineto{\pgfqpoint{4.681524in}{0.739656in}}%
\pgfpathlineto{\pgfqpoint{4.681228in}{0.739656in}}%
\pgfpathlineto{\pgfqpoint{4.680932in}{0.739656in}}%
\pgfpathlineto{\pgfqpoint{4.680636in}{0.739656in}}%
\pgfpathlineto{\pgfqpoint{4.680340in}{0.739656in}}%
\pgfpathlineto{\pgfqpoint{4.680044in}{0.739656in}}%
\pgfpathlineto{\pgfqpoint{4.679748in}{0.739656in}}%
\pgfpathlineto{\pgfqpoint{4.679452in}{0.739656in}}%
\pgfpathlineto{\pgfqpoint{4.679156in}{0.739656in}}%
\pgfpathlineto{\pgfqpoint{4.678860in}{0.739656in}}%
\pgfpathlineto{\pgfqpoint{4.678564in}{0.739656in}}%
\pgfpathlineto{\pgfqpoint{4.678268in}{0.739656in}}%
\pgfpathlineto{\pgfqpoint{4.677972in}{0.739656in}}%
\pgfpathlineto{\pgfqpoint{4.677676in}{0.739656in}}%
\pgfpathlineto{\pgfqpoint{4.677380in}{0.739656in}}%
\pgfpathlineto{\pgfqpoint{4.677084in}{0.739656in}}%
\pgfpathlineto{\pgfqpoint{4.676788in}{0.739656in}}%
\pgfpathlineto{\pgfqpoint{4.676492in}{0.739656in}}%
\pgfpathlineto{\pgfqpoint{4.676196in}{0.739656in}}%
\pgfpathlineto{\pgfqpoint{4.675900in}{0.739656in}}%
\pgfpathlineto{\pgfqpoint{4.675604in}{0.739656in}}%
\pgfpathlineto{\pgfqpoint{4.675308in}{0.739656in}}%
\pgfpathlineto{\pgfqpoint{4.675012in}{0.739656in}}%
\pgfpathlineto{\pgfqpoint{4.674716in}{0.739656in}}%
\pgfpathlineto{\pgfqpoint{4.674420in}{0.739656in}}%
\pgfpathlineto{\pgfqpoint{4.674124in}{0.739656in}}%
\pgfpathlineto{\pgfqpoint{4.673828in}{0.739656in}}%
\pgfpathlineto{\pgfqpoint{4.673532in}{0.739656in}}%
\pgfpathlineto{\pgfqpoint{4.673236in}{0.739656in}}%
\pgfpathlineto{\pgfqpoint{4.672940in}{0.739656in}}%
\pgfpathlineto{\pgfqpoint{4.672644in}{0.739656in}}%
\pgfpathlineto{\pgfqpoint{4.672348in}{0.739656in}}%
\pgfpathlineto{\pgfqpoint{4.672052in}{0.739656in}}%
\pgfpathlineto{\pgfqpoint{4.671756in}{0.739656in}}%
\pgfpathlineto{\pgfqpoint{4.671460in}{0.739656in}}%
\pgfpathlineto{\pgfqpoint{4.671164in}{0.739656in}}%
\pgfpathlineto{\pgfqpoint{4.670868in}{0.739656in}}%
\pgfpathlineto{\pgfqpoint{4.670572in}{0.739656in}}%
\pgfpathlineto{\pgfqpoint{4.670276in}{0.739656in}}%
\pgfpathlineto{\pgfqpoint{4.669980in}{0.739656in}}%
\pgfpathlineto{\pgfqpoint{4.669684in}{0.739656in}}%
\pgfpathlineto{\pgfqpoint{4.669388in}{0.739656in}}%
\pgfpathlineto{\pgfqpoint{4.669092in}{0.739656in}}%
\pgfpathlineto{\pgfqpoint{4.668796in}{0.739656in}}%
\pgfpathlineto{\pgfqpoint{4.668500in}{0.739656in}}%
\pgfpathlineto{\pgfqpoint{4.668204in}{0.739656in}}%
\pgfpathlineto{\pgfqpoint{4.667908in}{0.739656in}}%
\pgfpathlineto{\pgfqpoint{4.667612in}{0.739656in}}%
\pgfpathlineto{\pgfqpoint{4.667316in}{0.739656in}}%
\pgfpathlineto{\pgfqpoint{4.667020in}{0.739656in}}%
\pgfpathlineto{\pgfqpoint{4.666724in}{0.739656in}}%
\pgfpathlineto{\pgfqpoint{4.666428in}{0.739656in}}%
\pgfpathlineto{\pgfqpoint{4.666132in}{0.739656in}}%
\pgfpathlineto{\pgfqpoint{4.665836in}{0.739656in}}%
\pgfpathlineto{\pgfqpoint{4.665540in}{0.739656in}}%
\pgfpathlineto{\pgfqpoint{4.665244in}{0.739656in}}%
\pgfpathlineto{\pgfqpoint{4.664948in}{0.739656in}}%
\pgfpathlineto{\pgfqpoint{4.664652in}{0.739656in}}%
\pgfpathlineto{\pgfqpoint{4.664356in}{0.739656in}}%
\pgfpathlineto{\pgfqpoint{4.664060in}{0.739656in}}%
\pgfpathlineto{\pgfqpoint{4.663764in}{0.739656in}}%
\pgfpathlineto{\pgfqpoint{4.663468in}{0.739656in}}%
\pgfpathlineto{\pgfqpoint{4.663172in}{0.739656in}}%
\pgfpathlineto{\pgfqpoint{4.662876in}{0.739656in}}%
\pgfpathlineto{\pgfqpoint{4.662580in}{0.739656in}}%
\pgfpathlineto{\pgfqpoint{4.662284in}{0.739656in}}%
\pgfpathlineto{\pgfqpoint{4.661988in}{0.739656in}}%
\pgfpathlineto{\pgfqpoint{4.661692in}{0.739656in}}%
\pgfpathlineto{\pgfqpoint{4.661396in}{0.739656in}}%
\pgfpathlineto{\pgfqpoint{4.661100in}{0.739656in}}%
\pgfpathlineto{\pgfqpoint{4.660804in}{0.739656in}}%
\pgfpathlineto{\pgfqpoint{4.660508in}{0.739656in}}%
\pgfpathlineto{\pgfqpoint{4.660212in}{0.739656in}}%
\pgfpathlineto{\pgfqpoint{4.659916in}{0.739656in}}%
\pgfpathlineto{\pgfqpoint{4.659620in}{0.739656in}}%
\pgfpathlineto{\pgfqpoint{4.659324in}{0.739656in}}%
\pgfpathlineto{\pgfqpoint{4.659028in}{0.739656in}}%
\pgfpathlineto{\pgfqpoint{4.658732in}{0.739656in}}%
\pgfpathlineto{\pgfqpoint{4.658436in}{0.739656in}}%
\pgfpathlineto{\pgfqpoint{4.658140in}{0.739656in}}%
\pgfpathlineto{\pgfqpoint{4.657844in}{0.739656in}}%
\pgfpathlineto{\pgfqpoint{4.657548in}{0.739656in}}%
\pgfpathlineto{\pgfqpoint{4.657252in}{0.739656in}}%
\pgfpathlineto{\pgfqpoint{4.656956in}{0.739656in}}%
\pgfpathlineto{\pgfqpoint{4.656660in}{0.739656in}}%
\pgfpathlineto{\pgfqpoint{4.656364in}{0.739656in}}%
\pgfpathlineto{\pgfqpoint{4.656068in}{0.739656in}}%
\pgfpathlineto{\pgfqpoint{4.655772in}{0.739656in}}%
\pgfpathlineto{\pgfqpoint{4.655476in}{0.739656in}}%
\pgfpathlineto{\pgfqpoint{4.655180in}{0.739656in}}%
\pgfpathlineto{\pgfqpoint{4.654884in}{0.739656in}}%
\pgfpathlineto{\pgfqpoint{4.654588in}{0.739656in}}%
\pgfpathlineto{\pgfqpoint{4.654292in}{0.739656in}}%
\pgfpathlineto{\pgfqpoint{4.653996in}{0.739656in}}%
\pgfpathlineto{\pgfqpoint{4.653700in}{0.739656in}}%
\pgfpathlineto{\pgfqpoint{4.653404in}{0.739656in}}%
\pgfpathlineto{\pgfqpoint{4.653108in}{0.739656in}}%
\pgfpathlineto{\pgfqpoint{4.652812in}{0.739656in}}%
\pgfpathlineto{\pgfqpoint{4.652516in}{0.739656in}}%
\pgfpathlineto{\pgfqpoint{4.652220in}{0.739656in}}%
\pgfpathlineto{\pgfqpoint{4.651924in}{0.739656in}}%
\pgfpathlineto{\pgfqpoint{4.651628in}{0.739656in}}%
\pgfpathlineto{\pgfqpoint{4.651332in}{0.739656in}}%
\pgfpathlineto{\pgfqpoint{4.651036in}{0.739656in}}%
\pgfpathlineto{\pgfqpoint{4.650740in}{0.739656in}}%
\pgfpathlineto{\pgfqpoint{4.650444in}{0.739656in}}%
\pgfpathlineto{\pgfqpoint{4.650148in}{0.739656in}}%
\pgfpathlineto{\pgfqpoint{4.649852in}{0.739656in}}%
\pgfpathlineto{\pgfqpoint{4.649556in}{0.739656in}}%
\pgfpathlineto{\pgfqpoint{4.649260in}{0.739656in}}%
\pgfpathlineto{\pgfqpoint{4.648964in}{0.739656in}}%
\pgfpathlineto{\pgfqpoint{4.648668in}{0.739656in}}%
\pgfpathlineto{\pgfqpoint{4.648372in}{0.739656in}}%
\pgfpathlineto{\pgfqpoint{4.648076in}{0.739656in}}%
\pgfpathlineto{\pgfqpoint{4.647780in}{0.739656in}}%
\pgfpathlineto{\pgfqpoint{4.647484in}{0.739656in}}%
\pgfpathlineto{\pgfqpoint{4.647188in}{0.739656in}}%
\pgfpathlineto{\pgfqpoint{4.646892in}{0.739656in}}%
\pgfpathlineto{\pgfqpoint{4.646596in}{0.739656in}}%
\pgfpathlineto{\pgfqpoint{4.646300in}{0.739656in}}%
\pgfpathlineto{\pgfqpoint{4.646004in}{0.739656in}}%
\pgfpathlineto{\pgfqpoint{4.645708in}{0.739656in}}%
\pgfpathlineto{\pgfqpoint{4.645412in}{0.739656in}}%
\pgfpathlineto{\pgfqpoint{4.645116in}{0.739656in}}%
\pgfpathlineto{\pgfqpoint{4.644820in}{0.739656in}}%
\pgfpathlineto{\pgfqpoint{4.644524in}{0.739656in}}%
\pgfpathlineto{\pgfqpoint{4.644228in}{0.739656in}}%
\pgfpathlineto{\pgfqpoint{4.643932in}{0.739656in}}%
\pgfpathlineto{\pgfqpoint{4.643636in}{0.739656in}}%
\pgfpathlineto{\pgfqpoint{4.643340in}{0.739656in}}%
\pgfpathlineto{\pgfqpoint{4.643044in}{0.739656in}}%
\pgfpathlineto{\pgfqpoint{4.642748in}{0.739656in}}%
\pgfpathlineto{\pgfqpoint{4.642452in}{0.739656in}}%
\pgfpathlineto{\pgfqpoint{4.642156in}{0.739656in}}%
\pgfpathlineto{\pgfqpoint{4.641860in}{0.739656in}}%
\pgfpathlineto{\pgfqpoint{4.641564in}{0.739656in}}%
\pgfpathlineto{\pgfqpoint{4.641268in}{0.739656in}}%
\pgfpathlineto{\pgfqpoint{4.640972in}{0.739656in}}%
\pgfpathlineto{\pgfqpoint{4.640676in}{0.739656in}}%
\pgfpathlineto{\pgfqpoint{4.640380in}{0.739656in}}%
\pgfpathlineto{\pgfqpoint{4.640084in}{0.739656in}}%
\pgfpathlineto{\pgfqpoint{4.639788in}{0.739656in}}%
\pgfpathlineto{\pgfqpoint{4.639492in}{0.739656in}}%
\pgfpathlineto{\pgfqpoint{4.639196in}{0.739656in}}%
\pgfpathlineto{\pgfqpoint{4.638900in}{0.739656in}}%
\pgfpathlineto{\pgfqpoint{4.638604in}{0.739656in}}%
\pgfpathlineto{\pgfqpoint{4.638308in}{0.739656in}}%
\pgfpathlineto{\pgfqpoint{4.638012in}{0.739656in}}%
\pgfpathlineto{\pgfqpoint{4.637716in}{0.739656in}}%
\pgfpathlineto{\pgfqpoint{4.637420in}{0.739656in}}%
\pgfpathlineto{\pgfqpoint{4.637124in}{0.739656in}}%
\pgfpathlineto{\pgfqpoint{4.636828in}{0.739656in}}%
\pgfpathlineto{\pgfqpoint{4.636532in}{0.739656in}}%
\pgfpathlineto{\pgfqpoint{4.636236in}{0.739656in}}%
\pgfpathlineto{\pgfqpoint{4.635940in}{0.739656in}}%
\pgfpathlineto{\pgfqpoint{4.635644in}{0.739656in}}%
\pgfpathlineto{\pgfqpoint{4.635348in}{0.739656in}}%
\pgfpathlineto{\pgfqpoint{4.635052in}{0.739656in}}%
\pgfpathlineto{\pgfqpoint{4.634756in}{0.739656in}}%
\pgfpathlineto{\pgfqpoint{4.634460in}{0.739656in}}%
\pgfpathlineto{\pgfqpoint{4.634164in}{0.739656in}}%
\pgfpathlineto{\pgfqpoint{4.633868in}{0.739656in}}%
\pgfpathlineto{\pgfqpoint{4.633572in}{0.739656in}}%
\pgfpathlineto{\pgfqpoint{4.633276in}{0.739656in}}%
\pgfpathlineto{\pgfqpoint{4.632980in}{0.739656in}}%
\pgfpathlineto{\pgfqpoint{4.632684in}{0.739656in}}%
\pgfpathlineto{\pgfqpoint{4.632388in}{0.739656in}}%
\pgfpathlineto{\pgfqpoint{4.632092in}{0.739656in}}%
\pgfpathlineto{\pgfqpoint{4.631796in}{0.739656in}}%
\pgfpathlineto{\pgfqpoint{4.631500in}{0.739656in}}%
\pgfpathlineto{\pgfqpoint{4.631204in}{0.739656in}}%
\pgfpathlineto{\pgfqpoint{4.630908in}{0.739656in}}%
\pgfpathlineto{\pgfqpoint{4.630612in}{0.739656in}}%
\pgfpathlineto{\pgfqpoint{4.630316in}{0.739656in}}%
\pgfpathlineto{\pgfqpoint{4.630020in}{0.739656in}}%
\pgfpathlineto{\pgfqpoint{4.629724in}{0.739656in}}%
\pgfpathlineto{\pgfqpoint{4.629428in}{0.739656in}}%
\pgfpathlineto{\pgfqpoint{4.629132in}{0.739656in}}%
\pgfpathlineto{\pgfqpoint{4.628836in}{0.739656in}}%
\pgfpathlineto{\pgfqpoint{4.628540in}{0.739656in}}%
\pgfpathlineto{\pgfqpoint{4.628244in}{0.739656in}}%
\pgfpathlineto{\pgfqpoint{4.627948in}{0.739656in}}%
\pgfpathlineto{\pgfqpoint{4.627652in}{0.739656in}}%
\pgfpathlineto{\pgfqpoint{4.627356in}{0.739656in}}%
\pgfpathlineto{\pgfqpoint{4.627060in}{0.739656in}}%
\pgfpathlineto{\pgfqpoint{4.626764in}{0.739656in}}%
\pgfpathlineto{\pgfqpoint{4.626468in}{0.739656in}}%
\pgfpathlineto{\pgfqpoint{4.626172in}{0.739656in}}%
\pgfpathlineto{\pgfqpoint{4.625876in}{0.739656in}}%
\pgfpathlineto{\pgfqpoint{4.625580in}{0.739656in}}%
\pgfpathlineto{\pgfqpoint{4.625283in}{0.739656in}}%
\pgfpathlineto{\pgfqpoint{4.624987in}{0.739656in}}%
\pgfpathlineto{\pgfqpoint{4.624691in}{0.739656in}}%
\pgfpathlineto{\pgfqpoint{4.624395in}{0.739656in}}%
\pgfpathlineto{\pgfqpoint{4.624099in}{0.739656in}}%
\pgfpathlineto{\pgfqpoint{4.623803in}{0.739656in}}%
\pgfpathlineto{\pgfqpoint{4.623507in}{0.739656in}}%
\pgfpathlineto{\pgfqpoint{4.623211in}{0.739656in}}%
\pgfpathlineto{\pgfqpoint{4.622915in}{0.739656in}}%
\pgfpathlineto{\pgfqpoint{4.622619in}{0.739656in}}%
\pgfpathlineto{\pgfqpoint{4.622323in}{0.739656in}}%
\pgfpathlineto{\pgfqpoint{4.622027in}{0.739656in}}%
\pgfpathlineto{\pgfqpoint{4.621731in}{0.739656in}}%
\pgfpathlineto{\pgfqpoint{4.621435in}{0.739656in}}%
\pgfpathlineto{\pgfqpoint{4.621139in}{0.739656in}}%
\pgfpathlineto{\pgfqpoint{4.620843in}{0.739656in}}%
\pgfpathlineto{\pgfqpoint{4.620547in}{0.739656in}}%
\pgfpathlineto{\pgfqpoint{4.620251in}{0.739656in}}%
\pgfpathlineto{\pgfqpoint{4.619955in}{0.739656in}}%
\pgfpathlineto{\pgfqpoint{4.619659in}{0.739656in}}%
\pgfpathlineto{\pgfqpoint{4.619363in}{0.739656in}}%
\pgfpathlineto{\pgfqpoint{4.619067in}{0.739656in}}%
\pgfpathlineto{\pgfqpoint{4.618771in}{0.739656in}}%
\pgfpathlineto{\pgfqpoint{4.618475in}{0.739656in}}%
\pgfpathlineto{\pgfqpoint{4.618179in}{0.739656in}}%
\pgfpathlineto{\pgfqpoint{4.617883in}{0.739656in}}%
\pgfpathlineto{\pgfqpoint{4.617587in}{0.739656in}}%
\pgfpathlineto{\pgfqpoint{4.617291in}{0.739656in}}%
\pgfpathlineto{\pgfqpoint{4.616995in}{0.739656in}}%
\pgfpathlineto{\pgfqpoint{4.616699in}{0.739656in}}%
\pgfpathlineto{\pgfqpoint{4.616403in}{0.739656in}}%
\pgfpathlineto{\pgfqpoint{4.616107in}{0.739656in}}%
\pgfpathlineto{\pgfqpoint{4.615811in}{0.739656in}}%
\pgfpathlineto{\pgfqpoint{4.615515in}{0.739656in}}%
\pgfpathlineto{\pgfqpoint{4.615219in}{0.739656in}}%
\pgfpathlineto{\pgfqpoint{4.614923in}{0.739656in}}%
\pgfpathlineto{\pgfqpoint{4.614627in}{0.739656in}}%
\pgfpathlineto{\pgfqpoint{4.614331in}{0.739656in}}%
\pgfpathlineto{\pgfqpoint{4.614035in}{0.739656in}}%
\pgfpathlineto{\pgfqpoint{4.613739in}{0.739656in}}%
\pgfpathlineto{\pgfqpoint{4.613443in}{0.739656in}}%
\pgfpathlineto{\pgfqpoint{4.613147in}{0.739656in}}%
\pgfpathlineto{\pgfqpoint{4.612851in}{0.739656in}}%
\pgfpathlineto{\pgfqpoint{4.612555in}{0.739656in}}%
\pgfpathlineto{\pgfqpoint{4.612259in}{0.739656in}}%
\pgfpathlineto{\pgfqpoint{4.611963in}{0.739656in}}%
\pgfpathlineto{\pgfqpoint{4.611667in}{0.739656in}}%
\pgfpathlineto{\pgfqpoint{4.611371in}{0.739656in}}%
\pgfpathlineto{\pgfqpoint{4.611075in}{0.739656in}}%
\pgfpathlineto{\pgfqpoint{4.610779in}{0.739656in}}%
\pgfpathlineto{\pgfqpoint{4.610483in}{0.739656in}}%
\pgfpathlineto{\pgfqpoint{4.610187in}{0.739656in}}%
\pgfpathlineto{\pgfqpoint{4.609891in}{0.739656in}}%
\pgfpathlineto{\pgfqpoint{4.609595in}{0.739656in}}%
\pgfpathlineto{\pgfqpoint{4.609299in}{0.739656in}}%
\pgfpathlineto{\pgfqpoint{4.609003in}{0.739656in}}%
\pgfpathlineto{\pgfqpoint{4.608707in}{0.739656in}}%
\pgfpathlineto{\pgfqpoint{4.608411in}{0.739656in}}%
\pgfpathlineto{\pgfqpoint{4.608115in}{0.739656in}}%
\pgfpathlineto{\pgfqpoint{4.607819in}{0.739656in}}%
\pgfpathlineto{\pgfqpoint{4.607523in}{0.739656in}}%
\pgfpathlineto{\pgfqpoint{4.607227in}{0.739656in}}%
\pgfpathlineto{\pgfqpoint{4.606931in}{0.739656in}}%
\pgfpathlineto{\pgfqpoint{4.606635in}{0.739656in}}%
\pgfpathlineto{\pgfqpoint{4.606339in}{0.739656in}}%
\pgfpathlineto{\pgfqpoint{4.606043in}{0.739656in}}%
\pgfpathlineto{\pgfqpoint{4.605747in}{0.739656in}}%
\pgfpathlineto{\pgfqpoint{4.605451in}{0.739656in}}%
\pgfpathlineto{\pgfqpoint{4.605155in}{0.739656in}}%
\pgfpathlineto{\pgfqpoint{4.604859in}{0.739656in}}%
\pgfpathlineto{\pgfqpoint{4.604563in}{0.739656in}}%
\pgfpathlineto{\pgfqpoint{4.604267in}{0.739656in}}%
\pgfpathlineto{\pgfqpoint{4.603971in}{0.739656in}}%
\pgfpathlineto{\pgfqpoint{4.603675in}{0.739656in}}%
\pgfpathlineto{\pgfqpoint{4.603379in}{0.739656in}}%
\pgfpathlineto{\pgfqpoint{4.603083in}{0.739656in}}%
\pgfpathlineto{\pgfqpoint{4.602787in}{0.739656in}}%
\pgfpathlineto{\pgfqpoint{4.602491in}{0.739656in}}%
\pgfpathlineto{\pgfqpoint{4.602195in}{0.739656in}}%
\pgfpathlineto{\pgfqpoint{4.601899in}{0.739656in}}%
\pgfpathlineto{\pgfqpoint{4.601603in}{0.739656in}}%
\pgfpathlineto{\pgfqpoint{4.601307in}{0.739656in}}%
\pgfpathlineto{\pgfqpoint{4.601011in}{0.739656in}}%
\pgfpathlineto{\pgfqpoint{4.600715in}{0.739656in}}%
\pgfpathlineto{\pgfqpoint{4.600419in}{0.739656in}}%
\pgfpathlineto{\pgfqpoint{4.600123in}{0.739656in}}%
\pgfpathlineto{\pgfqpoint{4.599827in}{0.739656in}}%
\pgfpathlineto{\pgfqpoint{4.599531in}{0.739656in}}%
\pgfpathlineto{\pgfqpoint{4.599235in}{0.739656in}}%
\pgfpathlineto{\pgfqpoint{4.598939in}{0.739656in}}%
\pgfpathlineto{\pgfqpoint{4.598643in}{0.739656in}}%
\pgfpathlineto{\pgfqpoint{4.598347in}{0.739656in}}%
\pgfpathlineto{\pgfqpoint{4.598051in}{0.739656in}}%
\pgfpathlineto{\pgfqpoint{4.597755in}{0.739656in}}%
\pgfpathlineto{\pgfqpoint{4.597459in}{0.739656in}}%
\pgfpathlineto{\pgfqpoint{4.597163in}{0.739656in}}%
\pgfpathlineto{\pgfqpoint{4.596867in}{0.739656in}}%
\pgfpathlineto{\pgfqpoint{4.596571in}{0.739656in}}%
\pgfpathlineto{\pgfqpoint{4.596275in}{0.739656in}}%
\pgfpathlineto{\pgfqpoint{4.595979in}{0.739656in}}%
\pgfpathlineto{\pgfqpoint{4.595683in}{0.739656in}}%
\pgfpathlineto{\pgfqpoint{4.595387in}{0.739656in}}%
\pgfpathlineto{\pgfqpoint{4.595091in}{0.739656in}}%
\pgfpathlineto{\pgfqpoint{4.594795in}{0.739656in}}%
\pgfpathlineto{\pgfqpoint{4.594499in}{0.739656in}}%
\pgfpathlineto{\pgfqpoint{4.594203in}{0.739656in}}%
\pgfpathlineto{\pgfqpoint{4.593907in}{0.739656in}}%
\pgfpathlineto{\pgfqpoint{4.593611in}{0.739656in}}%
\pgfpathlineto{\pgfqpoint{4.593315in}{0.739656in}}%
\pgfpathlineto{\pgfqpoint{4.593019in}{0.739656in}}%
\pgfpathlineto{\pgfqpoint{4.592723in}{0.739656in}}%
\pgfpathlineto{\pgfqpoint{4.592427in}{0.739656in}}%
\pgfpathlineto{\pgfqpoint{4.592131in}{0.739656in}}%
\pgfpathlineto{\pgfqpoint{4.591835in}{0.739656in}}%
\pgfpathlineto{\pgfqpoint{4.591539in}{0.739656in}}%
\pgfpathlineto{\pgfqpoint{4.591243in}{0.739656in}}%
\pgfpathlineto{\pgfqpoint{4.590947in}{0.739656in}}%
\pgfpathlineto{\pgfqpoint{4.590651in}{0.739656in}}%
\pgfpathlineto{\pgfqpoint{4.590355in}{0.739656in}}%
\pgfpathlineto{\pgfqpoint{4.590059in}{0.739656in}}%
\pgfpathlineto{\pgfqpoint{4.589763in}{0.739656in}}%
\pgfpathlineto{\pgfqpoint{4.589467in}{0.739656in}}%
\pgfpathlineto{\pgfqpoint{4.589171in}{0.739656in}}%
\pgfpathlineto{\pgfqpoint{4.588875in}{0.739656in}}%
\pgfpathlineto{\pgfqpoint{4.588579in}{0.739656in}}%
\pgfpathlineto{\pgfqpoint{4.588283in}{0.739656in}}%
\pgfpathlineto{\pgfqpoint{4.587987in}{0.739656in}}%
\pgfpathlineto{\pgfqpoint{4.587691in}{0.739656in}}%
\pgfpathlineto{\pgfqpoint{4.587395in}{0.739656in}}%
\pgfpathlineto{\pgfqpoint{4.587099in}{0.739656in}}%
\pgfpathlineto{\pgfqpoint{4.586803in}{0.739656in}}%
\pgfpathlineto{\pgfqpoint{4.586507in}{0.739656in}}%
\pgfpathlineto{\pgfqpoint{4.586211in}{0.739656in}}%
\pgfpathlineto{\pgfqpoint{4.585915in}{0.739656in}}%
\pgfpathlineto{\pgfqpoint{4.585619in}{0.739656in}}%
\pgfpathlineto{\pgfqpoint{4.585323in}{0.739656in}}%
\pgfpathlineto{\pgfqpoint{4.585027in}{0.739656in}}%
\pgfpathlineto{\pgfqpoint{4.584731in}{0.739656in}}%
\pgfpathlineto{\pgfqpoint{4.584435in}{0.739656in}}%
\pgfpathlineto{\pgfqpoint{4.584139in}{0.739656in}}%
\pgfpathlineto{\pgfqpoint{4.583843in}{0.739656in}}%
\pgfpathlineto{\pgfqpoint{4.583547in}{0.739656in}}%
\pgfpathlineto{\pgfqpoint{4.583251in}{0.739656in}}%
\pgfpathlineto{\pgfqpoint{4.582955in}{0.739656in}}%
\pgfpathlineto{\pgfqpoint{4.582659in}{0.739656in}}%
\pgfpathlineto{\pgfqpoint{4.582363in}{0.739656in}}%
\pgfpathlineto{\pgfqpoint{4.582067in}{0.739656in}}%
\pgfpathlineto{\pgfqpoint{4.581771in}{0.739656in}}%
\pgfpathlineto{\pgfqpoint{4.581475in}{0.739656in}}%
\pgfpathlineto{\pgfqpoint{4.581179in}{0.739656in}}%
\pgfpathlineto{\pgfqpoint{4.580883in}{0.739656in}}%
\pgfpathlineto{\pgfqpoint{4.580587in}{0.739656in}}%
\pgfpathlineto{\pgfqpoint{4.580291in}{0.739656in}}%
\pgfpathlineto{\pgfqpoint{4.579995in}{0.739656in}}%
\pgfpathlineto{\pgfqpoint{4.579699in}{0.739656in}}%
\pgfpathlineto{\pgfqpoint{4.579403in}{0.739656in}}%
\pgfpathlineto{\pgfqpoint{4.579107in}{0.739656in}}%
\pgfpathlineto{\pgfqpoint{4.578811in}{0.739656in}}%
\pgfpathlineto{\pgfqpoint{4.578515in}{0.739656in}}%
\pgfpathlineto{\pgfqpoint{4.578219in}{0.739656in}}%
\pgfpathlineto{\pgfqpoint{4.577923in}{0.739656in}}%
\pgfpathlineto{\pgfqpoint{4.577627in}{0.739656in}}%
\pgfpathlineto{\pgfqpoint{4.577331in}{0.739656in}}%
\pgfpathlineto{\pgfqpoint{4.577035in}{0.739656in}}%
\pgfpathlineto{\pgfqpoint{4.576739in}{0.739656in}}%
\pgfpathlineto{\pgfqpoint{4.576443in}{0.739656in}}%
\pgfpathlineto{\pgfqpoint{4.576147in}{0.739656in}}%
\pgfpathlineto{\pgfqpoint{4.575851in}{0.739656in}}%
\pgfpathlineto{\pgfqpoint{4.575555in}{0.739656in}}%
\pgfpathlineto{\pgfqpoint{4.575259in}{0.739656in}}%
\pgfpathlineto{\pgfqpoint{4.574963in}{0.739656in}}%
\pgfpathlineto{\pgfqpoint{4.574667in}{0.739656in}}%
\pgfpathlineto{\pgfqpoint{4.574371in}{0.739656in}}%
\pgfpathlineto{\pgfqpoint{4.574075in}{0.739656in}}%
\pgfpathlineto{\pgfqpoint{4.573779in}{0.739656in}}%
\pgfpathlineto{\pgfqpoint{4.573483in}{0.739656in}}%
\pgfpathlineto{\pgfqpoint{4.573187in}{0.739656in}}%
\pgfpathlineto{\pgfqpoint{4.572891in}{0.739656in}}%
\pgfpathlineto{\pgfqpoint{4.572595in}{0.739656in}}%
\pgfpathlineto{\pgfqpoint{4.572299in}{0.739656in}}%
\pgfpathlineto{\pgfqpoint{4.572003in}{0.739656in}}%
\pgfpathlineto{\pgfqpoint{4.571707in}{0.739656in}}%
\pgfpathlineto{\pgfqpoint{4.571411in}{0.739656in}}%
\pgfpathlineto{\pgfqpoint{4.571115in}{0.739656in}}%
\pgfpathlineto{\pgfqpoint{4.570819in}{0.739656in}}%
\pgfpathlineto{\pgfqpoint{4.570523in}{0.739656in}}%
\pgfpathlineto{\pgfqpoint{4.570227in}{0.739656in}}%
\pgfpathlineto{\pgfqpoint{4.569931in}{0.739656in}}%
\pgfpathlineto{\pgfqpoint{4.569635in}{0.739656in}}%
\pgfpathlineto{\pgfqpoint{4.569339in}{0.739656in}}%
\pgfpathlineto{\pgfqpoint{4.569043in}{0.739656in}}%
\pgfpathlineto{\pgfqpoint{4.568747in}{0.739656in}}%
\pgfpathlineto{\pgfqpoint{4.568451in}{0.739656in}}%
\pgfpathlineto{\pgfqpoint{4.568155in}{0.739656in}}%
\pgfpathlineto{\pgfqpoint{4.567859in}{0.739656in}}%
\pgfpathlineto{\pgfqpoint{4.567563in}{0.739656in}}%
\pgfpathlineto{\pgfqpoint{4.567267in}{0.739656in}}%
\pgfpathlineto{\pgfqpoint{4.566971in}{0.739656in}}%
\pgfpathlineto{\pgfqpoint{4.566675in}{0.739656in}}%
\pgfpathlineto{\pgfqpoint{4.566379in}{0.739656in}}%
\pgfpathlineto{\pgfqpoint{4.566083in}{0.739656in}}%
\pgfpathlineto{\pgfqpoint{4.565787in}{0.739656in}}%
\pgfpathlineto{\pgfqpoint{4.565491in}{0.739656in}}%
\pgfpathlineto{\pgfqpoint{4.565195in}{0.739656in}}%
\pgfpathlineto{\pgfqpoint{4.564899in}{0.739656in}}%
\pgfpathlineto{\pgfqpoint{4.564603in}{0.739656in}}%
\pgfpathlineto{\pgfqpoint{4.564307in}{0.739656in}}%
\pgfpathlineto{\pgfqpoint{4.564011in}{0.739656in}}%
\pgfpathlineto{\pgfqpoint{4.563715in}{0.739656in}}%
\pgfpathlineto{\pgfqpoint{4.563419in}{0.739656in}}%
\pgfpathlineto{\pgfqpoint{4.563123in}{0.739656in}}%
\pgfpathlineto{\pgfqpoint{4.562827in}{0.739656in}}%
\pgfpathlineto{\pgfqpoint{4.562531in}{0.739656in}}%
\pgfpathlineto{\pgfqpoint{4.562235in}{0.739656in}}%
\pgfpathlineto{\pgfqpoint{4.561939in}{0.739656in}}%
\pgfpathlineto{\pgfqpoint{4.561643in}{0.739656in}}%
\pgfpathlineto{\pgfqpoint{4.561347in}{0.739656in}}%
\pgfpathlineto{\pgfqpoint{4.561051in}{0.739656in}}%
\pgfpathlineto{\pgfqpoint{4.560755in}{0.739656in}}%
\pgfpathlineto{\pgfqpoint{4.560459in}{0.739656in}}%
\pgfpathlineto{\pgfqpoint{4.560163in}{0.739656in}}%
\pgfpathlineto{\pgfqpoint{4.559867in}{0.739656in}}%
\pgfpathlineto{\pgfqpoint{4.559571in}{0.739656in}}%
\pgfpathlineto{\pgfqpoint{4.559275in}{0.739656in}}%
\pgfpathlineto{\pgfqpoint{4.558979in}{0.739656in}}%
\pgfpathlineto{\pgfqpoint{4.558683in}{0.739656in}}%
\pgfpathlineto{\pgfqpoint{4.558387in}{0.739656in}}%
\pgfpathlineto{\pgfqpoint{4.558091in}{0.739656in}}%
\pgfpathlineto{\pgfqpoint{4.557794in}{0.739656in}}%
\pgfpathlineto{\pgfqpoint{4.557498in}{0.739656in}}%
\pgfpathlineto{\pgfqpoint{4.557202in}{0.739656in}}%
\pgfpathlineto{\pgfqpoint{4.556906in}{0.739656in}}%
\pgfpathlineto{\pgfqpoint{4.556610in}{0.739656in}}%
\pgfpathlineto{\pgfqpoint{4.556314in}{0.739656in}}%
\pgfpathlineto{\pgfqpoint{4.556018in}{0.739656in}}%
\pgfpathlineto{\pgfqpoint{4.555722in}{0.739656in}}%
\pgfpathlineto{\pgfqpoint{4.555426in}{0.739656in}}%
\pgfpathlineto{\pgfqpoint{4.555130in}{0.739656in}}%
\pgfpathlineto{\pgfqpoint{4.554834in}{0.739656in}}%
\pgfpathlineto{\pgfqpoint{4.554538in}{0.739656in}}%
\pgfpathlineto{\pgfqpoint{4.554242in}{0.739656in}}%
\pgfpathlineto{\pgfqpoint{4.553946in}{0.739656in}}%
\pgfpathlineto{\pgfqpoint{4.553650in}{0.739656in}}%
\pgfpathlineto{\pgfqpoint{4.553354in}{0.739656in}}%
\pgfpathlineto{\pgfqpoint{4.553058in}{0.739656in}}%
\pgfpathlineto{\pgfqpoint{4.552762in}{0.739656in}}%
\pgfpathlineto{\pgfqpoint{4.552466in}{0.739656in}}%
\pgfpathlineto{\pgfqpoint{4.552170in}{0.739656in}}%
\pgfpathlineto{\pgfqpoint{4.551874in}{0.739656in}}%
\pgfpathlineto{\pgfqpoint{4.551578in}{0.739656in}}%
\pgfpathlineto{\pgfqpoint{4.551282in}{0.739656in}}%
\pgfpathlineto{\pgfqpoint{4.550986in}{0.739656in}}%
\pgfpathlineto{\pgfqpoint{4.550690in}{0.739656in}}%
\pgfpathlineto{\pgfqpoint{4.550394in}{0.739656in}}%
\pgfpathlineto{\pgfqpoint{4.550098in}{0.739656in}}%
\pgfpathlineto{\pgfqpoint{4.549802in}{0.739656in}}%
\pgfpathlineto{\pgfqpoint{4.549506in}{0.739656in}}%
\pgfpathlineto{\pgfqpoint{4.549210in}{0.739656in}}%
\pgfpathlineto{\pgfqpoint{4.548914in}{0.739656in}}%
\pgfpathlineto{\pgfqpoint{4.548618in}{0.739656in}}%
\pgfpathlineto{\pgfqpoint{4.548322in}{0.739656in}}%
\pgfpathlineto{\pgfqpoint{4.548026in}{0.739656in}}%
\pgfpathlineto{\pgfqpoint{4.547730in}{0.739656in}}%
\pgfpathlineto{\pgfqpoint{4.547434in}{0.739656in}}%
\pgfpathlineto{\pgfqpoint{4.547138in}{0.739656in}}%
\pgfpathlineto{\pgfqpoint{4.546842in}{0.739656in}}%
\pgfpathlineto{\pgfqpoint{4.546546in}{0.739656in}}%
\pgfpathlineto{\pgfqpoint{4.546250in}{0.739656in}}%
\pgfpathlineto{\pgfqpoint{4.545954in}{0.739656in}}%
\pgfpathlineto{\pgfqpoint{4.545658in}{0.739656in}}%
\pgfpathlineto{\pgfqpoint{4.545362in}{0.739656in}}%
\pgfpathlineto{\pgfqpoint{4.545066in}{0.739656in}}%
\pgfpathlineto{\pgfqpoint{4.544770in}{0.739656in}}%
\pgfpathlineto{\pgfqpoint{4.544474in}{0.739656in}}%
\pgfpathlineto{\pgfqpoint{4.544178in}{0.739656in}}%
\pgfpathlineto{\pgfqpoint{4.543882in}{0.739656in}}%
\pgfpathlineto{\pgfqpoint{4.543586in}{0.739656in}}%
\pgfpathlineto{\pgfqpoint{4.543290in}{0.739656in}}%
\pgfpathlineto{\pgfqpoint{4.542994in}{0.739656in}}%
\pgfpathlineto{\pgfqpoint{4.542698in}{0.739656in}}%
\pgfpathlineto{\pgfqpoint{4.542402in}{0.739656in}}%
\pgfpathlineto{\pgfqpoint{4.542106in}{0.739656in}}%
\pgfpathlineto{\pgfqpoint{4.541810in}{0.739656in}}%
\pgfpathlineto{\pgfqpoint{4.541514in}{0.739656in}}%
\pgfpathlineto{\pgfqpoint{4.541218in}{0.739656in}}%
\pgfpathlineto{\pgfqpoint{4.540922in}{0.739656in}}%
\pgfpathlineto{\pgfqpoint{4.540626in}{0.739656in}}%
\pgfpathlineto{\pgfqpoint{4.540330in}{0.739656in}}%
\pgfpathlineto{\pgfqpoint{4.540034in}{0.739656in}}%
\pgfpathlineto{\pgfqpoint{4.539738in}{0.739656in}}%
\pgfpathlineto{\pgfqpoint{4.539442in}{0.739656in}}%
\pgfpathlineto{\pgfqpoint{4.539146in}{0.739656in}}%
\pgfpathlineto{\pgfqpoint{4.538850in}{0.739656in}}%
\pgfpathlineto{\pgfqpoint{4.538554in}{0.739656in}}%
\pgfpathlineto{\pgfqpoint{4.538258in}{0.739656in}}%
\pgfpathlineto{\pgfqpoint{4.537962in}{0.739656in}}%
\pgfpathlineto{\pgfqpoint{4.537666in}{0.739656in}}%
\pgfpathlineto{\pgfqpoint{4.537370in}{0.739656in}}%
\pgfpathlineto{\pgfqpoint{4.537074in}{0.739656in}}%
\pgfpathlineto{\pgfqpoint{4.536778in}{0.739656in}}%
\pgfpathlineto{\pgfqpoint{4.536482in}{0.739656in}}%
\pgfpathlineto{\pgfqpoint{4.536186in}{0.739656in}}%
\pgfpathlineto{\pgfqpoint{4.535890in}{0.739656in}}%
\pgfpathlineto{\pgfqpoint{4.535594in}{0.739656in}}%
\pgfpathlineto{\pgfqpoint{4.535298in}{0.739656in}}%
\pgfpathlineto{\pgfqpoint{4.535002in}{0.739656in}}%
\pgfpathlineto{\pgfqpoint{4.534706in}{0.739656in}}%
\pgfpathlineto{\pgfqpoint{4.534410in}{0.739656in}}%
\pgfpathlineto{\pgfqpoint{4.534114in}{0.739656in}}%
\pgfpathlineto{\pgfqpoint{4.533818in}{0.739656in}}%
\pgfpathlineto{\pgfqpoint{4.533522in}{0.739656in}}%
\pgfpathlineto{\pgfqpoint{4.533226in}{0.739656in}}%
\pgfpathlineto{\pgfqpoint{4.532930in}{0.739656in}}%
\pgfpathlineto{\pgfqpoint{4.532634in}{0.739656in}}%
\pgfpathlineto{\pgfqpoint{4.532338in}{0.739656in}}%
\pgfpathlineto{\pgfqpoint{4.532042in}{0.739656in}}%
\pgfpathlineto{\pgfqpoint{4.531746in}{0.739656in}}%
\pgfpathlineto{\pgfqpoint{4.531450in}{0.739656in}}%
\pgfpathlineto{\pgfqpoint{4.531154in}{0.739656in}}%
\pgfpathlineto{\pgfqpoint{4.530858in}{0.739656in}}%
\pgfpathlineto{\pgfqpoint{4.530562in}{0.739656in}}%
\pgfpathlineto{\pgfqpoint{4.530266in}{0.739656in}}%
\pgfpathlineto{\pgfqpoint{4.529970in}{0.739656in}}%
\pgfpathlineto{\pgfqpoint{4.529674in}{0.739656in}}%
\pgfpathlineto{\pgfqpoint{4.529378in}{0.739656in}}%
\pgfpathlineto{\pgfqpoint{4.529082in}{0.739656in}}%
\pgfpathlineto{\pgfqpoint{4.528786in}{0.739656in}}%
\pgfpathlineto{\pgfqpoint{4.528490in}{0.739656in}}%
\pgfpathlineto{\pgfqpoint{4.528194in}{0.739656in}}%
\pgfpathlineto{\pgfqpoint{4.527898in}{0.739656in}}%
\pgfpathlineto{\pgfqpoint{4.527602in}{0.739656in}}%
\pgfpathlineto{\pgfqpoint{4.527306in}{0.739656in}}%
\pgfpathlineto{\pgfqpoint{4.527010in}{0.739656in}}%
\pgfpathlineto{\pgfqpoint{4.526714in}{0.739656in}}%
\pgfpathlineto{\pgfqpoint{4.526418in}{0.739656in}}%
\pgfpathlineto{\pgfqpoint{4.526122in}{0.739656in}}%
\pgfpathlineto{\pgfqpoint{4.525826in}{0.739656in}}%
\pgfpathlineto{\pgfqpoint{4.525530in}{0.739656in}}%
\pgfpathlineto{\pgfqpoint{4.525234in}{0.739656in}}%
\pgfpathlineto{\pgfqpoint{4.524938in}{0.739656in}}%
\pgfpathlineto{\pgfqpoint{4.524642in}{0.739656in}}%
\pgfpathlineto{\pgfqpoint{4.524346in}{0.739656in}}%
\pgfpathlineto{\pgfqpoint{4.524050in}{0.739656in}}%
\pgfpathlineto{\pgfqpoint{4.523754in}{0.739656in}}%
\pgfpathlineto{\pgfqpoint{4.523458in}{0.739656in}}%
\pgfpathlineto{\pgfqpoint{4.523162in}{0.739656in}}%
\pgfpathlineto{\pgfqpoint{4.522866in}{0.739656in}}%
\pgfpathlineto{\pgfqpoint{4.522570in}{0.739656in}}%
\pgfpathlineto{\pgfqpoint{4.522274in}{0.739656in}}%
\pgfpathlineto{\pgfqpoint{4.521978in}{0.739656in}}%
\pgfpathlineto{\pgfqpoint{4.521682in}{0.739656in}}%
\pgfpathlineto{\pgfqpoint{4.521386in}{0.739656in}}%
\pgfpathlineto{\pgfqpoint{4.521090in}{0.739656in}}%
\pgfpathlineto{\pgfqpoint{4.520794in}{0.739656in}}%
\pgfpathlineto{\pgfqpoint{4.520498in}{0.739656in}}%
\pgfpathlineto{\pgfqpoint{4.520202in}{0.739656in}}%
\pgfpathlineto{\pgfqpoint{4.519906in}{0.739656in}}%
\pgfpathlineto{\pgfqpoint{4.519610in}{0.739656in}}%
\pgfpathlineto{\pgfqpoint{4.519314in}{0.739656in}}%
\pgfpathlineto{\pgfqpoint{4.519018in}{0.739656in}}%
\pgfpathlineto{\pgfqpoint{4.518722in}{0.739656in}}%
\pgfpathlineto{\pgfqpoint{4.518426in}{0.739656in}}%
\pgfpathlineto{\pgfqpoint{4.518130in}{0.739656in}}%
\pgfpathlineto{\pgfqpoint{4.517834in}{0.739656in}}%
\pgfpathlineto{\pgfqpoint{4.517538in}{0.739656in}}%
\pgfpathlineto{\pgfqpoint{4.517242in}{0.739656in}}%
\pgfpathlineto{\pgfqpoint{4.516946in}{0.739656in}}%
\pgfpathlineto{\pgfqpoint{4.516650in}{0.739656in}}%
\pgfpathlineto{\pgfqpoint{4.516354in}{0.739656in}}%
\pgfpathlineto{\pgfqpoint{4.516058in}{0.739656in}}%
\pgfpathlineto{\pgfqpoint{4.515762in}{0.739656in}}%
\pgfpathlineto{\pgfqpoint{4.515466in}{0.739656in}}%
\pgfpathlineto{\pgfqpoint{4.515170in}{0.739656in}}%
\pgfpathlineto{\pgfqpoint{4.514874in}{0.739656in}}%
\pgfpathlineto{\pgfqpoint{4.514578in}{0.739656in}}%
\pgfpathlineto{\pgfqpoint{4.514282in}{0.739656in}}%
\pgfpathlineto{\pgfqpoint{4.513986in}{0.739656in}}%
\pgfpathlineto{\pgfqpoint{4.513690in}{0.739656in}}%
\pgfpathlineto{\pgfqpoint{4.513394in}{0.739656in}}%
\pgfpathlineto{\pgfqpoint{4.513098in}{0.739656in}}%
\pgfpathlineto{\pgfqpoint{4.512802in}{0.739656in}}%
\pgfpathlineto{\pgfqpoint{4.512506in}{0.739656in}}%
\pgfpathlineto{\pgfqpoint{4.512210in}{0.739656in}}%
\pgfpathlineto{\pgfqpoint{4.511914in}{0.739656in}}%
\pgfpathlineto{\pgfqpoint{4.511618in}{0.739656in}}%
\pgfpathlineto{\pgfqpoint{4.511322in}{0.739656in}}%
\pgfpathlineto{\pgfqpoint{4.511026in}{0.739656in}}%
\pgfpathlineto{\pgfqpoint{4.510730in}{0.739656in}}%
\pgfpathlineto{\pgfqpoint{4.510434in}{0.739656in}}%
\pgfpathlineto{\pgfqpoint{4.510138in}{0.739656in}}%
\pgfpathlineto{\pgfqpoint{4.509842in}{0.739656in}}%
\pgfpathlineto{\pgfqpoint{4.509546in}{0.739656in}}%
\pgfpathlineto{\pgfqpoint{4.509250in}{0.739656in}}%
\pgfpathlineto{\pgfqpoint{4.508954in}{0.739656in}}%
\pgfpathlineto{\pgfqpoint{4.508658in}{0.739656in}}%
\pgfpathlineto{\pgfqpoint{4.508362in}{0.739656in}}%
\pgfpathlineto{\pgfqpoint{4.508066in}{0.739656in}}%
\pgfpathlineto{\pgfqpoint{4.507770in}{0.739656in}}%
\pgfpathlineto{\pgfqpoint{4.507474in}{0.739656in}}%
\pgfpathlineto{\pgfqpoint{4.507178in}{0.739656in}}%
\pgfpathlineto{\pgfqpoint{4.506882in}{0.739656in}}%
\pgfpathlineto{\pgfqpoint{4.506586in}{0.739656in}}%
\pgfpathlineto{\pgfqpoint{4.506290in}{0.739656in}}%
\pgfpathlineto{\pgfqpoint{4.505994in}{0.739656in}}%
\pgfpathlineto{\pgfqpoint{4.505698in}{0.739656in}}%
\pgfpathlineto{\pgfqpoint{4.505402in}{0.739656in}}%
\pgfpathlineto{\pgfqpoint{4.505106in}{0.739656in}}%
\pgfpathlineto{\pgfqpoint{4.504810in}{0.739656in}}%
\pgfpathlineto{\pgfqpoint{4.504514in}{0.739656in}}%
\pgfpathlineto{\pgfqpoint{4.504218in}{0.739656in}}%
\pgfpathlineto{\pgfqpoint{4.503922in}{0.739656in}}%
\pgfpathlineto{\pgfqpoint{4.503626in}{0.739656in}}%
\pgfpathlineto{\pgfqpoint{4.503330in}{0.739656in}}%
\pgfpathlineto{\pgfqpoint{4.503034in}{0.739656in}}%
\pgfpathlineto{\pgfqpoint{4.502738in}{0.739656in}}%
\pgfpathlineto{\pgfqpoint{4.502442in}{0.739656in}}%
\pgfpathlineto{\pgfqpoint{4.502146in}{0.739656in}}%
\pgfpathlineto{\pgfqpoint{4.501850in}{0.739656in}}%
\pgfpathlineto{\pgfqpoint{4.501554in}{0.739656in}}%
\pgfpathlineto{\pgfqpoint{4.501258in}{0.739656in}}%
\pgfpathlineto{\pgfqpoint{4.500962in}{0.739656in}}%
\pgfpathlineto{\pgfqpoint{4.500666in}{0.739656in}}%
\pgfpathlineto{\pgfqpoint{4.500370in}{0.739656in}}%
\pgfpathlineto{\pgfqpoint{4.500074in}{0.739656in}}%
\pgfpathlineto{\pgfqpoint{4.499778in}{0.739656in}}%
\pgfpathlineto{\pgfqpoint{4.499482in}{0.739656in}}%
\pgfpathlineto{\pgfqpoint{4.499186in}{0.739656in}}%
\pgfpathlineto{\pgfqpoint{4.498890in}{0.739656in}}%
\pgfpathlineto{\pgfqpoint{4.498594in}{0.739656in}}%
\pgfpathlineto{\pgfqpoint{4.498298in}{0.739656in}}%
\pgfpathlineto{\pgfqpoint{4.498002in}{0.739656in}}%
\pgfpathlineto{\pgfqpoint{4.497706in}{0.739656in}}%
\pgfpathlineto{\pgfqpoint{4.497410in}{0.739656in}}%
\pgfpathlineto{\pgfqpoint{4.497114in}{0.739656in}}%
\pgfpathlineto{\pgfqpoint{4.496818in}{0.739656in}}%
\pgfpathlineto{\pgfqpoint{4.496522in}{0.739656in}}%
\pgfpathlineto{\pgfqpoint{4.496226in}{0.739656in}}%
\pgfpathlineto{\pgfqpoint{4.495930in}{0.739656in}}%
\pgfpathlineto{\pgfqpoint{4.495634in}{0.739656in}}%
\pgfpathlineto{\pgfqpoint{4.495338in}{0.739656in}}%
\pgfpathlineto{\pgfqpoint{4.495042in}{0.739656in}}%
\pgfpathlineto{\pgfqpoint{4.494746in}{0.739656in}}%
\pgfpathlineto{\pgfqpoint{4.494450in}{0.739656in}}%
\pgfpathlineto{\pgfqpoint{4.494154in}{0.739656in}}%
\pgfpathlineto{\pgfqpoint{4.493858in}{0.739656in}}%
\pgfpathlineto{\pgfqpoint{4.493562in}{0.739656in}}%
\pgfpathlineto{\pgfqpoint{4.493266in}{0.739656in}}%
\pgfpathlineto{\pgfqpoint{4.492970in}{0.739656in}}%
\pgfpathlineto{\pgfqpoint{4.492674in}{0.739656in}}%
\pgfpathlineto{\pgfqpoint{4.492378in}{0.739656in}}%
\pgfpathlineto{\pgfqpoint{4.492082in}{0.739656in}}%
\pgfpathlineto{\pgfqpoint{4.491786in}{0.739656in}}%
\pgfpathlineto{\pgfqpoint{4.491490in}{0.739656in}}%
\pgfpathlineto{\pgfqpoint{4.491194in}{0.739656in}}%
\pgfpathlineto{\pgfqpoint{4.490898in}{0.739656in}}%
\pgfpathlineto{\pgfqpoint{4.490601in}{0.739656in}}%
\pgfpathlineto{\pgfqpoint{4.490305in}{0.739656in}}%
\pgfpathlineto{\pgfqpoint{4.490009in}{0.739656in}}%
\pgfpathlineto{\pgfqpoint{4.489713in}{0.739656in}}%
\pgfpathlineto{\pgfqpoint{4.489417in}{0.739656in}}%
\pgfpathlineto{\pgfqpoint{4.489121in}{0.739656in}}%
\pgfpathlineto{\pgfqpoint{4.488825in}{0.739656in}}%
\pgfpathlineto{\pgfqpoint{4.488529in}{0.739656in}}%
\pgfpathlineto{\pgfqpoint{4.488233in}{0.739656in}}%
\pgfpathlineto{\pgfqpoint{4.487937in}{0.739656in}}%
\pgfpathlineto{\pgfqpoint{4.487641in}{0.739656in}}%
\pgfpathlineto{\pgfqpoint{4.487345in}{0.739656in}}%
\pgfpathlineto{\pgfqpoint{4.487049in}{0.739656in}}%
\pgfpathlineto{\pgfqpoint{4.486753in}{0.739656in}}%
\pgfpathlineto{\pgfqpoint{4.486457in}{0.739656in}}%
\pgfpathlineto{\pgfqpoint{4.486161in}{0.739656in}}%
\pgfpathlineto{\pgfqpoint{4.485865in}{0.739656in}}%
\pgfpathlineto{\pgfqpoint{4.485569in}{0.739656in}}%
\pgfpathlineto{\pgfqpoint{4.485273in}{0.739656in}}%
\pgfpathlineto{\pgfqpoint{4.484977in}{0.739656in}}%
\pgfpathlineto{\pgfqpoint{4.484681in}{0.739656in}}%
\pgfpathlineto{\pgfqpoint{4.484385in}{0.739656in}}%
\pgfpathlineto{\pgfqpoint{4.484089in}{0.739656in}}%
\pgfpathlineto{\pgfqpoint{4.483793in}{0.739656in}}%
\pgfpathlineto{\pgfqpoint{4.483497in}{0.739656in}}%
\pgfpathlineto{\pgfqpoint{4.483201in}{0.739656in}}%
\pgfpathlineto{\pgfqpoint{4.482905in}{0.739656in}}%
\pgfpathlineto{\pgfqpoint{4.482609in}{0.739656in}}%
\pgfpathlineto{\pgfqpoint{4.482313in}{0.739656in}}%
\pgfpathlineto{\pgfqpoint{4.482017in}{0.739656in}}%
\pgfpathlineto{\pgfqpoint{4.481721in}{0.739656in}}%
\pgfpathlineto{\pgfqpoint{4.481425in}{0.739656in}}%
\pgfpathlineto{\pgfqpoint{4.481129in}{0.739656in}}%
\pgfpathlineto{\pgfqpoint{4.480833in}{0.739656in}}%
\pgfpathlineto{\pgfqpoint{4.480537in}{0.739656in}}%
\pgfpathlineto{\pgfqpoint{4.480241in}{0.739656in}}%
\pgfpathlineto{\pgfqpoint{4.479945in}{0.739656in}}%
\pgfpathlineto{\pgfqpoint{4.479649in}{0.739656in}}%
\pgfpathlineto{\pgfqpoint{4.479353in}{0.739656in}}%
\pgfpathlineto{\pgfqpoint{4.479057in}{0.739656in}}%
\pgfpathlineto{\pgfqpoint{4.478761in}{0.739656in}}%
\pgfpathlineto{\pgfqpoint{4.478465in}{0.739656in}}%
\pgfpathlineto{\pgfqpoint{4.478169in}{0.739656in}}%
\pgfpathlineto{\pgfqpoint{4.477873in}{0.739656in}}%
\pgfpathlineto{\pgfqpoint{4.477577in}{0.739656in}}%
\pgfpathlineto{\pgfqpoint{4.477281in}{0.739656in}}%
\pgfpathlineto{\pgfqpoint{4.476985in}{0.739656in}}%
\pgfpathlineto{\pgfqpoint{4.476689in}{0.739656in}}%
\pgfpathlineto{\pgfqpoint{4.476393in}{0.739656in}}%
\pgfpathlineto{\pgfqpoint{4.476097in}{0.739656in}}%
\pgfpathlineto{\pgfqpoint{4.475801in}{0.739656in}}%
\pgfpathlineto{\pgfqpoint{4.475505in}{0.739656in}}%
\pgfpathlineto{\pgfqpoint{4.475209in}{0.739656in}}%
\pgfpathlineto{\pgfqpoint{4.474913in}{0.739656in}}%
\pgfpathlineto{\pgfqpoint{4.474617in}{0.739656in}}%
\pgfpathlineto{\pgfqpoint{4.474321in}{0.739656in}}%
\pgfpathlineto{\pgfqpoint{4.474025in}{0.739656in}}%
\pgfpathlineto{\pgfqpoint{4.473729in}{0.739656in}}%
\pgfpathlineto{\pgfqpoint{4.473433in}{0.739656in}}%
\pgfpathlineto{\pgfqpoint{4.473137in}{0.739656in}}%
\pgfpathlineto{\pgfqpoint{4.472841in}{0.739656in}}%
\pgfpathlineto{\pgfqpoint{4.472545in}{0.739656in}}%
\pgfpathlineto{\pgfqpoint{4.472249in}{0.739656in}}%
\pgfpathlineto{\pgfqpoint{4.471953in}{0.739656in}}%
\pgfpathlineto{\pgfqpoint{4.471657in}{0.739656in}}%
\pgfpathlineto{\pgfqpoint{4.471361in}{0.739656in}}%
\pgfpathlineto{\pgfqpoint{4.471065in}{0.739656in}}%
\pgfpathlineto{\pgfqpoint{4.470769in}{0.739656in}}%
\pgfpathlineto{\pgfqpoint{4.470473in}{0.739656in}}%
\pgfpathlineto{\pgfqpoint{4.470177in}{0.739656in}}%
\pgfpathlineto{\pgfqpoint{4.469881in}{0.739656in}}%
\pgfpathlineto{\pgfqpoint{4.469585in}{0.739656in}}%
\pgfpathlineto{\pgfqpoint{4.469289in}{0.739656in}}%
\pgfpathlineto{\pgfqpoint{4.468993in}{0.739656in}}%
\pgfpathlineto{\pgfqpoint{4.468697in}{0.739656in}}%
\pgfpathlineto{\pgfqpoint{4.468401in}{0.739656in}}%
\pgfpathlineto{\pgfqpoint{4.468105in}{0.739656in}}%
\pgfpathlineto{\pgfqpoint{4.467809in}{0.739656in}}%
\pgfpathlineto{\pgfqpoint{4.467513in}{0.739656in}}%
\pgfpathlineto{\pgfqpoint{4.467217in}{0.739656in}}%
\pgfpathlineto{\pgfqpoint{4.466921in}{0.739656in}}%
\pgfpathlineto{\pgfqpoint{4.466625in}{0.739656in}}%
\pgfpathlineto{\pgfqpoint{4.466329in}{0.739656in}}%
\pgfpathlineto{\pgfqpoint{4.466033in}{0.739656in}}%
\pgfpathlineto{\pgfqpoint{4.465737in}{0.739656in}}%
\pgfpathlineto{\pgfqpoint{4.465441in}{0.739656in}}%
\pgfpathlineto{\pgfqpoint{4.465145in}{0.739656in}}%
\pgfpathlineto{\pgfqpoint{4.464849in}{0.739656in}}%
\pgfpathlineto{\pgfqpoint{4.464553in}{0.739656in}}%
\pgfpathlineto{\pgfqpoint{4.464257in}{0.739656in}}%
\pgfpathlineto{\pgfqpoint{4.463961in}{0.739656in}}%
\pgfpathlineto{\pgfqpoint{4.463665in}{0.739656in}}%
\pgfpathlineto{\pgfqpoint{4.463369in}{0.739656in}}%
\pgfpathlineto{\pgfqpoint{4.463073in}{0.739656in}}%
\pgfpathlineto{\pgfqpoint{4.462777in}{0.739656in}}%
\pgfpathlineto{\pgfqpoint{4.462481in}{0.739656in}}%
\pgfpathlineto{\pgfqpoint{4.462185in}{0.739656in}}%
\pgfpathlineto{\pgfqpoint{4.461889in}{0.739656in}}%
\pgfpathlineto{\pgfqpoint{4.461593in}{0.739656in}}%
\pgfpathlineto{\pgfqpoint{4.461297in}{0.739656in}}%
\pgfpathlineto{\pgfqpoint{4.461001in}{0.739656in}}%
\pgfpathlineto{\pgfqpoint{4.460705in}{0.739656in}}%
\pgfpathlineto{\pgfqpoint{4.460409in}{0.739656in}}%
\pgfpathlineto{\pgfqpoint{4.460113in}{0.739656in}}%
\pgfpathlineto{\pgfqpoint{4.459817in}{0.739656in}}%
\pgfpathlineto{\pgfqpoint{4.459521in}{0.739656in}}%
\pgfpathlineto{\pgfqpoint{4.459225in}{0.739656in}}%
\pgfpathlineto{\pgfqpoint{4.458929in}{0.739656in}}%
\pgfpathlineto{\pgfqpoint{4.458633in}{0.739656in}}%
\pgfpathlineto{\pgfqpoint{4.458337in}{0.739656in}}%
\pgfpathlineto{\pgfqpoint{4.458041in}{0.739656in}}%
\pgfpathlineto{\pgfqpoint{4.457745in}{0.739656in}}%
\pgfpathlineto{\pgfqpoint{4.457449in}{0.739656in}}%
\pgfpathlineto{\pgfqpoint{4.457153in}{0.739656in}}%
\pgfpathlineto{\pgfqpoint{4.456857in}{0.739656in}}%
\pgfpathlineto{\pgfqpoint{4.456561in}{0.739656in}}%
\pgfpathlineto{\pgfqpoint{4.456265in}{0.739656in}}%
\pgfpathlineto{\pgfqpoint{4.455969in}{0.739656in}}%
\pgfpathlineto{\pgfqpoint{4.455673in}{0.739656in}}%
\pgfpathlineto{\pgfqpoint{4.455377in}{0.739656in}}%
\pgfpathlineto{\pgfqpoint{4.455081in}{0.739656in}}%
\pgfpathlineto{\pgfqpoint{4.454785in}{0.739656in}}%
\pgfpathlineto{\pgfqpoint{4.454489in}{0.739656in}}%
\pgfpathlineto{\pgfqpoint{4.454193in}{0.739656in}}%
\pgfpathlineto{\pgfqpoint{4.453897in}{0.739656in}}%
\pgfpathlineto{\pgfqpoint{4.453601in}{0.739656in}}%
\pgfpathlineto{\pgfqpoint{4.453305in}{0.739656in}}%
\pgfpathlineto{\pgfqpoint{4.453009in}{0.739656in}}%
\pgfpathlineto{\pgfqpoint{4.452713in}{0.739656in}}%
\pgfpathlineto{\pgfqpoint{4.452417in}{0.739656in}}%
\pgfpathlineto{\pgfqpoint{4.452121in}{0.739656in}}%
\pgfpathlineto{\pgfqpoint{4.451825in}{0.739656in}}%
\pgfpathlineto{\pgfqpoint{4.451529in}{0.739656in}}%
\pgfpathlineto{\pgfqpoint{4.451233in}{0.739656in}}%
\pgfpathlineto{\pgfqpoint{4.450937in}{0.739656in}}%
\pgfpathlineto{\pgfqpoint{4.450641in}{0.739656in}}%
\pgfpathlineto{\pgfqpoint{4.450345in}{0.739656in}}%
\pgfpathlineto{\pgfqpoint{4.450049in}{0.739656in}}%
\pgfpathlineto{\pgfqpoint{4.449753in}{0.739656in}}%
\pgfpathlineto{\pgfqpoint{4.449457in}{0.739656in}}%
\pgfpathlineto{\pgfqpoint{4.449161in}{0.739656in}}%
\pgfpathlineto{\pgfqpoint{4.448865in}{0.739656in}}%
\pgfpathlineto{\pgfqpoint{4.448569in}{0.739656in}}%
\pgfpathlineto{\pgfqpoint{4.448273in}{0.739656in}}%
\pgfpathlineto{\pgfqpoint{4.447977in}{0.739656in}}%
\pgfpathlineto{\pgfqpoint{4.447681in}{0.739656in}}%
\pgfpathlineto{\pgfqpoint{4.447385in}{0.739656in}}%
\pgfpathlineto{\pgfqpoint{4.447089in}{0.739656in}}%
\pgfpathlineto{\pgfqpoint{4.446793in}{0.739656in}}%
\pgfpathlineto{\pgfqpoint{4.446497in}{0.739656in}}%
\pgfpathlineto{\pgfqpoint{4.446201in}{0.739656in}}%
\pgfpathlineto{\pgfqpoint{4.445905in}{0.739656in}}%
\pgfpathlineto{\pgfqpoint{4.445609in}{0.739656in}}%
\pgfpathlineto{\pgfqpoint{4.445313in}{0.739656in}}%
\pgfpathlineto{\pgfqpoint{4.445017in}{0.739656in}}%
\pgfpathlineto{\pgfqpoint{4.444721in}{0.739656in}}%
\pgfpathlineto{\pgfqpoint{4.444425in}{0.739656in}}%
\pgfpathlineto{\pgfqpoint{4.444129in}{0.739656in}}%
\pgfpathlineto{\pgfqpoint{4.443833in}{0.739656in}}%
\pgfpathlineto{\pgfqpoint{4.443537in}{0.739656in}}%
\pgfpathlineto{\pgfqpoint{4.443241in}{0.739656in}}%
\pgfpathlineto{\pgfqpoint{4.442945in}{0.739656in}}%
\pgfpathlineto{\pgfqpoint{4.442649in}{0.739656in}}%
\pgfpathlineto{\pgfqpoint{4.442353in}{0.739656in}}%
\pgfpathlineto{\pgfqpoint{4.442057in}{0.739656in}}%
\pgfpathlineto{\pgfqpoint{4.441761in}{0.739656in}}%
\pgfpathlineto{\pgfqpoint{4.441465in}{0.739656in}}%
\pgfpathlineto{\pgfqpoint{4.441169in}{0.739656in}}%
\pgfpathlineto{\pgfqpoint{4.440873in}{0.739656in}}%
\pgfpathlineto{\pgfqpoint{4.440577in}{0.739656in}}%
\pgfpathlineto{\pgfqpoint{4.440281in}{0.739656in}}%
\pgfpathlineto{\pgfqpoint{4.439985in}{0.739656in}}%
\pgfpathlineto{\pgfqpoint{4.439689in}{0.739656in}}%
\pgfpathlineto{\pgfqpoint{4.439393in}{0.739656in}}%
\pgfpathlineto{\pgfqpoint{4.439097in}{0.739656in}}%
\pgfpathlineto{\pgfqpoint{4.438801in}{0.739656in}}%
\pgfpathlineto{\pgfqpoint{4.438505in}{0.739656in}}%
\pgfpathlineto{\pgfqpoint{4.438209in}{0.739656in}}%
\pgfpathlineto{\pgfqpoint{4.437913in}{0.739656in}}%
\pgfpathlineto{\pgfqpoint{4.437617in}{0.739656in}}%
\pgfpathlineto{\pgfqpoint{4.437321in}{0.739656in}}%
\pgfpathlineto{\pgfqpoint{4.437025in}{0.739656in}}%
\pgfpathlineto{\pgfqpoint{4.436729in}{0.739656in}}%
\pgfpathlineto{\pgfqpoint{4.436433in}{0.739656in}}%
\pgfpathlineto{\pgfqpoint{4.436137in}{0.739656in}}%
\pgfpathlineto{\pgfqpoint{4.435841in}{0.739656in}}%
\pgfpathlineto{\pgfqpoint{4.435545in}{0.739656in}}%
\pgfpathlineto{\pgfqpoint{4.435249in}{0.739656in}}%
\pgfpathlineto{\pgfqpoint{4.434953in}{0.739656in}}%
\pgfpathlineto{\pgfqpoint{4.434657in}{0.739656in}}%
\pgfpathlineto{\pgfqpoint{4.434361in}{0.739656in}}%
\pgfpathlineto{\pgfqpoint{4.434065in}{0.739656in}}%
\pgfpathlineto{\pgfqpoint{4.433769in}{0.739656in}}%
\pgfpathlineto{\pgfqpoint{4.433473in}{0.739656in}}%
\pgfpathlineto{\pgfqpoint{4.433177in}{0.739656in}}%
\pgfpathlineto{\pgfqpoint{4.432881in}{0.739656in}}%
\pgfpathlineto{\pgfqpoint{4.432585in}{0.739656in}}%
\pgfpathlineto{\pgfqpoint{4.432289in}{0.739656in}}%
\pgfpathlineto{\pgfqpoint{4.431993in}{0.739656in}}%
\pgfpathlineto{\pgfqpoint{4.431697in}{0.739656in}}%
\pgfpathlineto{\pgfqpoint{4.431401in}{0.739656in}}%
\pgfpathlineto{\pgfqpoint{4.431105in}{0.739656in}}%
\pgfpathlineto{\pgfqpoint{4.430809in}{0.739656in}}%
\pgfpathlineto{\pgfqpoint{4.430513in}{0.739656in}}%
\pgfpathlineto{\pgfqpoint{4.430217in}{0.739656in}}%
\pgfpathlineto{\pgfqpoint{4.429921in}{0.739656in}}%
\pgfpathlineto{\pgfqpoint{4.429625in}{0.739656in}}%
\pgfpathlineto{\pgfqpoint{4.429329in}{0.739656in}}%
\pgfpathlineto{\pgfqpoint{4.429033in}{0.739656in}}%
\pgfpathlineto{\pgfqpoint{4.428737in}{0.739656in}}%
\pgfpathlineto{\pgfqpoint{4.428441in}{0.739656in}}%
\pgfpathlineto{\pgfqpoint{4.428145in}{0.739656in}}%
\pgfpathlineto{\pgfqpoint{4.427849in}{0.739656in}}%
\pgfpathlineto{\pgfqpoint{4.427553in}{0.739656in}}%
\pgfpathlineto{\pgfqpoint{4.427257in}{0.739656in}}%
\pgfpathlineto{\pgfqpoint{4.426961in}{0.739656in}}%
\pgfpathlineto{\pgfqpoint{4.426665in}{0.739656in}}%
\pgfpathlineto{\pgfqpoint{4.426369in}{0.739656in}}%
\pgfpathlineto{\pgfqpoint{4.426073in}{0.739656in}}%
\pgfpathlineto{\pgfqpoint{4.425777in}{0.739656in}}%
\pgfpathlineto{\pgfqpoint{4.425481in}{0.739656in}}%
\pgfpathlineto{\pgfqpoint{4.425185in}{0.739656in}}%
\pgfpathlineto{\pgfqpoint{4.424889in}{0.739656in}}%
\pgfpathlineto{\pgfqpoint{4.424593in}{0.739656in}}%
\pgfpathlineto{\pgfqpoint{4.424297in}{0.739656in}}%
\pgfpathlineto{\pgfqpoint{4.424001in}{0.739656in}}%
\pgfpathlineto{\pgfqpoint{4.423705in}{0.739656in}}%
\pgfpathlineto{\pgfqpoint{4.423409in}{0.739656in}}%
\pgfpathlineto{\pgfqpoint{4.423112in}{0.739656in}}%
\pgfpathlineto{\pgfqpoint{4.422816in}{0.739656in}}%
\pgfpathlineto{\pgfqpoint{4.422520in}{0.739656in}}%
\pgfpathlineto{\pgfqpoint{4.422224in}{0.739656in}}%
\pgfpathlineto{\pgfqpoint{4.421928in}{0.739656in}}%
\pgfpathlineto{\pgfqpoint{4.421632in}{0.739656in}}%
\pgfpathlineto{\pgfqpoint{4.421336in}{0.739656in}}%
\pgfpathlineto{\pgfqpoint{4.421040in}{0.739656in}}%
\pgfpathlineto{\pgfqpoint{4.420744in}{0.739656in}}%
\pgfpathlineto{\pgfqpoint{4.420448in}{0.739656in}}%
\pgfpathlineto{\pgfqpoint{4.420152in}{0.739656in}}%
\pgfpathlineto{\pgfqpoint{4.419856in}{0.739656in}}%
\pgfpathlineto{\pgfqpoint{4.419560in}{0.739656in}}%
\pgfpathlineto{\pgfqpoint{4.419264in}{0.739656in}}%
\pgfpathlineto{\pgfqpoint{4.418968in}{0.739656in}}%
\pgfpathlineto{\pgfqpoint{4.418672in}{0.739656in}}%
\pgfpathlineto{\pgfqpoint{4.418376in}{0.739656in}}%
\pgfpathlineto{\pgfqpoint{4.418080in}{0.739656in}}%
\pgfpathlineto{\pgfqpoint{4.417784in}{0.739656in}}%
\pgfpathlineto{\pgfqpoint{4.417488in}{0.739656in}}%
\pgfpathlineto{\pgfqpoint{4.417192in}{0.739656in}}%
\pgfpathlineto{\pgfqpoint{4.416896in}{0.739656in}}%
\pgfpathlineto{\pgfqpoint{4.416600in}{0.739656in}}%
\pgfpathlineto{\pgfqpoint{4.416304in}{0.739656in}}%
\pgfpathlineto{\pgfqpoint{4.416008in}{0.739656in}}%
\pgfpathlineto{\pgfqpoint{4.415712in}{0.739656in}}%
\pgfpathlineto{\pgfqpoint{4.415416in}{0.739656in}}%
\pgfpathlineto{\pgfqpoint{4.415120in}{0.739656in}}%
\pgfpathlineto{\pgfqpoint{4.414824in}{0.739656in}}%
\pgfpathlineto{\pgfqpoint{4.414528in}{0.739656in}}%
\pgfpathlineto{\pgfqpoint{4.414232in}{0.739656in}}%
\pgfpathlineto{\pgfqpoint{4.413936in}{0.739656in}}%
\pgfpathlineto{\pgfqpoint{4.413640in}{0.739656in}}%
\pgfpathlineto{\pgfqpoint{4.413344in}{0.739656in}}%
\pgfpathlineto{\pgfqpoint{4.413048in}{0.739656in}}%
\pgfpathlineto{\pgfqpoint{4.412752in}{0.739656in}}%
\pgfpathlineto{\pgfqpoint{4.412456in}{0.739656in}}%
\pgfpathlineto{\pgfqpoint{4.412160in}{0.739656in}}%
\pgfpathlineto{\pgfqpoint{4.411864in}{0.739656in}}%
\pgfpathlineto{\pgfqpoint{4.411568in}{0.739656in}}%
\pgfpathlineto{\pgfqpoint{4.411272in}{0.739656in}}%
\pgfpathlineto{\pgfqpoint{4.410976in}{0.739656in}}%
\pgfpathlineto{\pgfqpoint{4.410680in}{0.739656in}}%
\pgfpathlineto{\pgfqpoint{4.410384in}{0.739656in}}%
\pgfpathlineto{\pgfqpoint{4.410088in}{0.739656in}}%
\pgfpathlineto{\pgfqpoint{4.409792in}{0.739656in}}%
\pgfpathlineto{\pgfqpoint{4.409496in}{0.739656in}}%
\pgfpathlineto{\pgfqpoint{4.409200in}{0.739656in}}%
\pgfpathlineto{\pgfqpoint{4.408904in}{0.739656in}}%
\pgfpathlineto{\pgfqpoint{4.408608in}{0.739656in}}%
\pgfpathlineto{\pgfqpoint{4.408312in}{0.739656in}}%
\pgfpathlineto{\pgfqpoint{4.408016in}{0.739656in}}%
\pgfpathlineto{\pgfqpoint{4.407720in}{0.739656in}}%
\pgfpathlineto{\pgfqpoint{4.407424in}{0.739656in}}%
\pgfpathlineto{\pgfqpoint{4.407128in}{0.739656in}}%
\pgfpathlineto{\pgfqpoint{4.406832in}{0.739656in}}%
\pgfpathlineto{\pgfqpoint{4.406536in}{0.739656in}}%
\pgfpathlineto{\pgfqpoint{4.406240in}{0.739656in}}%
\pgfpathlineto{\pgfqpoint{4.405944in}{0.739656in}}%
\pgfpathlineto{\pgfqpoint{4.405648in}{0.739656in}}%
\pgfpathlineto{\pgfqpoint{4.405352in}{0.739656in}}%
\pgfpathlineto{\pgfqpoint{4.405056in}{0.739656in}}%
\pgfpathlineto{\pgfqpoint{4.404760in}{0.739656in}}%
\pgfpathlineto{\pgfqpoint{4.404464in}{0.739656in}}%
\pgfpathlineto{\pgfqpoint{4.404168in}{0.739656in}}%
\pgfpathlineto{\pgfqpoint{4.403872in}{0.739656in}}%
\pgfpathlineto{\pgfqpoint{4.403576in}{0.739656in}}%
\pgfpathlineto{\pgfqpoint{4.403280in}{0.739656in}}%
\pgfpathlineto{\pgfqpoint{4.402984in}{0.739656in}}%
\pgfpathlineto{\pgfqpoint{4.402688in}{0.739656in}}%
\pgfpathlineto{\pgfqpoint{4.402392in}{0.739656in}}%
\pgfpathlineto{\pgfqpoint{4.402096in}{0.739656in}}%
\pgfpathlineto{\pgfqpoint{4.401800in}{0.739656in}}%
\pgfpathlineto{\pgfqpoint{4.401504in}{0.739656in}}%
\pgfpathlineto{\pgfqpoint{4.401208in}{0.739656in}}%
\pgfpathlineto{\pgfqpoint{4.400912in}{0.739656in}}%
\pgfpathlineto{\pgfqpoint{4.400616in}{0.739656in}}%
\pgfpathlineto{\pgfqpoint{4.400320in}{0.739656in}}%
\pgfpathlineto{\pgfqpoint{4.400024in}{0.739656in}}%
\pgfpathlineto{\pgfqpoint{4.399728in}{0.739656in}}%
\pgfpathlineto{\pgfqpoint{4.399432in}{0.739656in}}%
\pgfpathlineto{\pgfqpoint{4.399136in}{0.739656in}}%
\pgfpathlineto{\pgfqpoint{4.398840in}{0.739656in}}%
\pgfpathlineto{\pgfqpoint{4.398544in}{0.739656in}}%
\pgfpathlineto{\pgfqpoint{4.398248in}{0.739656in}}%
\pgfpathlineto{\pgfqpoint{4.397952in}{0.739656in}}%
\pgfpathlineto{\pgfqpoint{4.397656in}{0.739656in}}%
\pgfpathlineto{\pgfqpoint{4.397360in}{0.739656in}}%
\pgfpathlineto{\pgfqpoint{4.397064in}{0.739656in}}%
\pgfpathlineto{\pgfqpoint{4.396768in}{0.739656in}}%
\pgfpathlineto{\pgfqpoint{4.396472in}{0.739656in}}%
\pgfpathlineto{\pgfqpoint{4.396176in}{0.739656in}}%
\pgfpathlineto{\pgfqpoint{4.395880in}{0.739656in}}%
\pgfpathlineto{\pgfqpoint{4.395584in}{0.739656in}}%
\pgfpathlineto{\pgfqpoint{4.395288in}{0.739656in}}%
\pgfpathlineto{\pgfqpoint{4.394992in}{0.739656in}}%
\pgfpathlineto{\pgfqpoint{4.394696in}{0.739656in}}%
\pgfpathlineto{\pgfqpoint{4.394400in}{0.739656in}}%
\pgfpathlineto{\pgfqpoint{4.394104in}{0.739656in}}%
\pgfpathlineto{\pgfqpoint{4.393808in}{0.739656in}}%
\pgfpathlineto{\pgfqpoint{4.393512in}{0.739656in}}%
\pgfpathlineto{\pgfqpoint{4.393216in}{0.739656in}}%
\pgfpathlineto{\pgfqpoint{4.392920in}{0.739656in}}%
\pgfpathlineto{\pgfqpoint{4.392624in}{0.739656in}}%
\pgfpathlineto{\pgfqpoint{4.392328in}{0.739656in}}%
\pgfpathlineto{\pgfqpoint{4.392032in}{0.739656in}}%
\pgfpathlineto{\pgfqpoint{4.391736in}{0.739656in}}%
\pgfpathlineto{\pgfqpoint{4.391440in}{0.739656in}}%
\pgfpathlineto{\pgfqpoint{4.391144in}{0.739656in}}%
\pgfpathlineto{\pgfqpoint{4.390848in}{0.739656in}}%
\pgfpathlineto{\pgfqpoint{4.390552in}{0.739656in}}%
\pgfpathlineto{\pgfqpoint{4.390256in}{0.739656in}}%
\pgfpathlineto{\pgfqpoint{4.389960in}{0.739656in}}%
\pgfpathlineto{\pgfqpoint{4.389664in}{0.739656in}}%
\pgfpathlineto{\pgfqpoint{4.389368in}{0.739656in}}%
\pgfpathlineto{\pgfqpoint{4.389072in}{0.739656in}}%
\pgfpathlineto{\pgfqpoint{4.388776in}{0.739656in}}%
\pgfpathlineto{\pgfqpoint{4.388480in}{0.739656in}}%
\pgfpathlineto{\pgfqpoint{4.388184in}{0.739656in}}%
\pgfpathlineto{\pgfqpoint{4.387888in}{0.739656in}}%
\pgfpathlineto{\pgfqpoint{4.387592in}{0.739656in}}%
\pgfpathlineto{\pgfqpoint{4.387296in}{0.739656in}}%
\pgfpathlineto{\pgfqpoint{4.387000in}{0.739656in}}%
\pgfpathlineto{\pgfqpoint{4.386704in}{0.739656in}}%
\pgfpathlineto{\pgfqpoint{4.386408in}{0.739656in}}%
\pgfpathlineto{\pgfqpoint{4.386112in}{0.739656in}}%
\pgfpathlineto{\pgfqpoint{4.385816in}{0.739656in}}%
\pgfpathlineto{\pgfqpoint{4.385520in}{0.739656in}}%
\pgfpathlineto{\pgfqpoint{4.385224in}{0.739656in}}%
\pgfpathlineto{\pgfqpoint{4.384928in}{0.739656in}}%
\pgfpathlineto{\pgfqpoint{4.384632in}{0.739656in}}%
\pgfpathlineto{\pgfqpoint{4.384336in}{0.739656in}}%
\pgfpathlineto{\pgfqpoint{4.384040in}{0.739656in}}%
\pgfpathlineto{\pgfqpoint{4.383744in}{0.739656in}}%
\pgfpathlineto{\pgfqpoint{4.383448in}{0.739656in}}%
\pgfpathlineto{\pgfqpoint{4.383152in}{0.739656in}}%
\pgfpathlineto{\pgfqpoint{4.382856in}{0.739656in}}%
\pgfpathlineto{\pgfqpoint{4.382560in}{0.739656in}}%
\pgfpathlineto{\pgfqpoint{4.382264in}{0.739656in}}%
\pgfpathlineto{\pgfqpoint{4.381968in}{0.739656in}}%
\pgfpathlineto{\pgfqpoint{4.381672in}{0.739656in}}%
\pgfpathlineto{\pgfqpoint{4.381376in}{0.739656in}}%
\pgfpathlineto{\pgfqpoint{4.381080in}{0.739656in}}%
\pgfpathlineto{\pgfqpoint{4.380784in}{0.739656in}}%
\pgfpathlineto{\pgfqpoint{4.380488in}{0.739656in}}%
\pgfpathlineto{\pgfqpoint{4.380192in}{0.739656in}}%
\pgfpathlineto{\pgfqpoint{4.379896in}{0.739656in}}%
\pgfpathlineto{\pgfqpoint{4.379600in}{0.739656in}}%
\pgfpathlineto{\pgfqpoint{4.379304in}{0.739656in}}%
\pgfpathlineto{\pgfqpoint{4.379008in}{0.739656in}}%
\pgfpathlineto{\pgfqpoint{4.378712in}{0.739656in}}%
\pgfpathlineto{\pgfqpoint{4.378416in}{0.739656in}}%
\pgfpathlineto{\pgfqpoint{4.378120in}{0.739656in}}%
\pgfpathlineto{\pgfqpoint{4.377824in}{0.739656in}}%
\pgfpathlineto{\pgfqpoint{4.377528in}{0.739656in}}%
\pgfpathlineto{\pgfqpoint{4.377232in}{0.739656in}}%
\pgfpathlineto{\pgfqpoint{4.376936in}{0.739656in}}%
\pgfpathlineto{\pgfqpoint{4.376640in}{0.739656in}}%
\pgfpathlineto{\pgfqpoint{4.376344in}{0.739656in}}%
\pgfpathlineto{\pgfqpoint{4.376048in}{0.739656in}}%
\pgfpathlineto{\pgfqpoint{4.375752in}{0.739656in}}%
\pgfpathlineto{\pgfqpoint{4.375456in}{0.739656in}}%
\pgfpathlineto{\pgfqpoint{4.375160in}{0.739656in}}%
\pgfpathlineto{\pgfqpoint{4.374864in}{0.739656in}}%
\pgfpathlineto{\pgfqpoint{4.374568in}{0.739656in}}%
\pgfpathlineto{\pgfqpoint{4.374272in}{0.739656in}}%
\pgfpathlineto{\pgfqpoint{4.373976in}{0.739656in}}%
\pgfpathlineto{\pgfqpoint{4.373680in}{0.739656in}}%
\pgfpathlineto{\pgfqpoint{4.373384in}{0.739656in}}%
\pgfpathlineto{\pgfqpoint{4.373088in}{0.739656in}}%
\pgfpathlineto{\pgfqpoint{4.372792in}{0.739656in}}%
\pgfpathlineto{\pgfqpoint{4.372496in}{0.739656in}}%
\pgfpathlineto{\pgfqpoint{4.372200in}{0.739656in}}%
\pgfpathlineto{\pgfqpoint{4.371904in}{0.739656in}}%
\pgfpathlineto{\pgfqpoint{4.371608in}{0.739656in}}%
\pgfpathlineto{\pgfqpoint{4.371312in}{0.739656in}}%
\pgfpathlineto{\pgfqpoint{4.371016in}{0.739656in}}%
\pgfpathlineto{\pgfqpoint{4.370720in}{0.739656in}}%
\pgfpathlineto{\pgfqpoint{4.370424in}{0.739656in}}%
\pgfpathlineto{\pgfqpoint{4.370128in}{0.739656in}}%
\pgfpathlineto{\pgfqpoint{4.369832in}{0.739656in}}%
\pgfpathlineto{\pgfqpoint{4.369536in}{0.739656in}}%
\pgfpathlineto{\pgfqpoint{4.369240in}{0.739656in}}%
\pgfpathlineto{\pgfqpoint{4.368944in}{0.739656in}}%
\pgfpathlineto{\pgfqpoint{4.368648in}{0.739656in}}%
\pgfpathlineto{\pgfqpoint{4.368352in}{0.739656in}}%
\pgfpathlineto{\pgfqpoint{4.368056in}{0.739656in}}%
\pgfpathlineto{\pgfqpoint{4.367760in}{0.739656in}}%
\pgfpathlineto{\pgfqpoint{4.367464in}{0.739656in}}%
\pgfpathlineto{\pgfqpoint{4.367168in}{0.739656in}}%
\pgfpathlineto{\pgfqpoint{4.366872in}{0.739656in}}%
\pgfpathlineto{\pgfqpoint{4.366576in}{0.739656in}}%
\pgfpathlineto{\pgfqpoint{4.366280in}{0.739656in}}%
\pgfpathlineto{\pgfqpoint{4.365984in}{0.739656in}}%
\pgfpathlineto{\pgfqpoint{4.365688in}{0.739656in}}%
\pgfpathlineto{\pgfqpoint{4.365392in}{0.739656in}}%
\pgfpathlineto{\pgfqpoint{4.365096in}{0.739656in}}%
\pgfpathlineto{\pgfqpoint{4.364800in}{0.739656in}}%
\pgfpathlineto{\pgfqpoint{4.364504in}{0.739656in}}%
\pgfpathlineto{\pgfqpoint{4.364208in}{0.739656in}}%
\pgfpathlineto{\pgfqpoint{4.363912in}{0.739656in}}%
\pgfpathlineto{\pgfqpoint{4.363616in}{0.739656in}}%
\pgfpathlineto{\pgfqpoint{4.363320in}{0.739656in}}%
\pgfpathlineto{\pgfqpoint{4.363024in}{0.739656in}}%
\pgfpathlineto{\pgfqpoint{4.362728in}{0.739656in}}%
\pgfpathlineto{\pgfqpoint{4.362432in}{0.739656in}}%
\pgfpathlineto{\pgfqpoint{4.362136in}{0.739656in}}%
\pgfpathlineto{\pgfqpoint{4.361840in}{0.739656in}}%
\pgfpathlineto{\pgfqpoint{4.361544in}{0.739656in}}%
\pgfpathlineto{\pgfqpoint{4.361248in}{0.739656in}}%
\pgfpathlineto{\pgfqpoint{4.360952in}{0.739656in}}%
\pgfpathlineto{\pgfqpoint{4.360656in}{0.739656in}}%
\pgfpathlineto{\pgfqpoint{4.360360in}{0.739656in}}%
\pgfpathlineto{\pgfqpoint{4.360064in}{0.739656in}}%
\pgfpathlineto{\pgfqpoint{4.359768in}{0.739656in}}%
\pgfpathlineto{\pgfqpoint{4.359472in}{0.739656in}}%
\pgfpathlineto{\pgfqpoint{4.359176in}{0.739656in}}%
\pgfpathlineto{\pgfqpoint{4.358880in}{0.739656in}}%
\pgfpathlineto{\pgfqpoint{4.358584in}{0.739656in}}%
\pgfpathlineto{\pgfqpoint{4.358288in}{0.739656in}}%
\pgfpathlineto{\pgfqpoint{4.357992in}{0.739656in}}%
\pgfpathlineto{\pgfqpoint{4.357696in}{0.739656in}}%
\pgfpathlineto{\pgfqpoint{4.357400in}{0.739656in}}%
\pgfpathlineto{\pgfqpoint{4.357104in}{0.739656in}}%
\pgfpathlineto{\pgfqpoint{4.356808in}{0.739656in}}%
\pgfpathlineto{\pgfqpoint{4.356512in}{0.739656in}}%
\pgfpathlineto{\pgfqpoint{4.356216in}{0.739656in}}%
\pgfpathlineto{\pgfqpoint{4.355920in}{0.739656in}}%
\pgfpathlineto{\pgfqpoint{4.355623in}{0.739656in}}%
\pgfpathlineto{\pgfqpoint{4.355327in}{0.739656in}}%
\pgfpathlineto{\pgfqpoint{4.355031in}{0.739656in}}%
\pgfpathlineto{\pgfqpoint{4.354735in}{0.739656in}}%
\pgfpathlineto{\pgfqpoint{4.354439in}{0.739656in}}%
\pgfpathlineto{\pgfqpoint{4.354143in}{0.739656in}}%
\pgfpathlineto{\pgfqpoint{4.353847in}{0.739656in}}%
\pgfpathlineto{\pgfqpoint{4.353551in}{0.739656in}}%
\pgfpathlineto{\pgfqpoint{4.353255in}{0.739656in}}%
\pgfpathlineto{\pgfqpoint{4.352959in}{0.739656in}}%
\pgfpathlineto{\pgfqpoint{4.352663in}{0.739656in}}%
\pgfpathlineto{\pgfqpoint{4.352367in}{0.739656in}}%
\pgfpathlineto{\pgfqpoint{4.352071in}{0.739656in}}%
\pgfpathlineto{\pgfqpoint{4.351775in}{0.739656in}}%
\pgfpathlineto{\pgfqpoint{4.351479in}{0.739656in}}%
\pgfpathlineto{\pgfqpoint{4.351183in}{0.739656in}}%
\pgfpathlineto{\pgfqpoint{4.350887in}{0.739656in}}%
\pgfpathlineto{\pgfqpoint{4.350591in}{0.739656in}}%
\pgfpathlineto{\pgfqpoint{4.350295in}{0.739656in}}%
\pgfpathlineto{\pgfqpoint{4.349999in}{0.739656in}}%
\pgfpathlineto{\pgfqpoint{4.349703in}{0.739656in}}%
\pgfpathlineto{\pgfqpoint{4.349407in}{0.739656in}}%
\pgfpathlineto{\pgfqpoint{4.349111in}{0.739656in}}%
\pgfpathlineto{\pgfqpoint{4.348815in}{0.739656in}}%
\pgfpathlineto{\pgfqpoint{4.348519in}{0.739656in}}%
\pgfpathlineto{\pgfqpoint{4.348223in}{0.739656in}}%
\pgfpathlineto{\pgfqpoint{4.347927in}{0.739656in}}%
\pgfpathlineto{\pgfqpoint{4.347631in}{0.739656in}}%
\pgfpathlineto{\pgfqpoint{4.347335in}{0.739656in}}%
\pgfpathlineto{\pgfqpoint{4.347039in}{0.739656in}}%
\pgfpathlineto{\pgfqpoint{4.346743in}{0.739656in}}%
\pgfpathlineto{\pgfqpoint{4.346447in}{0.739656in}}%
\pgfpathlineto{\pgfqpoint{4.346151in}{0.739656in}}%
\pgfpathlineto{\pgfqpoint{4.345855in}{0.739656in}}%
\pgfpathlineto{\pgfqpoint{4.345559in}{0.739656in}}%
\pgfpathlineto{\pgfqpoint{4.345263in}{0.739656in}}%
\pgfpathlineto{\pgfqpoint{4.344967in}{0.739656in}}%
\pgfpathlineto{\pgfqpoint{4.344671in}{0.739656in}}%
\pgfpathlineto{\pgfqpoint{4.344375in}{0.739656in}}%
\pgfpathlineto{\pgfqpoint{4.344079in}{0.739656in}}%
\pgfpathlineto{\pgfqpoint{4.343783in}{0.739656in}}%
\pgfpathlineto{\pgfqpoint{4.343487in}{0.739656in}}%
\pgfpathlineto{\pgfqpoint{4.343191in}{0.739656in}}%
\pgfpathlineto{\pgfqpoint{4.342895in}{0.739656in}}%
\pgfpathlineto{\pgfqpoint{4.342599in}{0.739656in}}%
\pgfpathlineto{\pgfqpoint{4.342303in}{0.739656in}}%
\pgfpathlineto{\pgfqpoint{4.342007in}{0.739656in}}%
\pgfpathlineto{\pgfqpoint{4.341711in}{0.739656in}}%
\pgfpathlineto{\pgfqpoint{4.341415in}{0.739656in}}%
\pgfpathlineto{\pgfqpoint{4.341119in}{0.739656in}}%
\pgfpathlineto{\pgfqpoint{4.340823in}{0.739656in}}%
\pgfpathlineto{\pgfqpoint{4.340527in}{0.739656in}}%
\pgfpathlineto{\pgfqpoint{4.340231in}{0.739656in}}%
\pgfpathlineto{\pgfqpoint{4.339935in}{0.739656in}}%
\pgfpathlineto{\pgfqpoint{4.339639in}{0.739656in}}%
\pgfpathlineto{\pgfqpoint{4.339343in}{0.739656in}}%
\pgfpathlineto{\pgfqpoint{4.339047in}{0.739656in}}%
\pgfpathlineto{\pgfqpoint{4.338751in}{0.739656in}}%
\pgfpathlineto{\pgfqpoint{4.338455in}{0.739656in}}%
\pgfpathlineto{\pgfqpoint{4.338159in}{0.739656in}}%
\pgfpathlineto{\pgfqpoint{4.337863in}{0.739656in}}%
\pgfpathlineto{\pgfqpoint{4.337567in}{0.739656in}}%
\pgfpathlineto{\pgfqpoint{4.337271in}{0.739656in}}%
\pgfpathlineto{\pgfqpoint{4.336975in}{0.739656in}}%
\pgfpathlineto{\pgfqpoint{4.336679in}{0.739656in}}%
\pgfpathlineto{\pgfqpoint{4.336383in}{0.739656in}}%
\pgfpathlineto{\pgfqpoint{4.336087in}{0.739656in}}%
\pgfpathlineto{\pgfqpoint{4.335791in}{0.739656in}}%
\pgfpathlineto{\pgfqpoint{4.335495in}{0.739656in}}%
\pgfpathlineto{\pgfqpoint{4.335199in}{0.739656in}}%
\pgfpathlineto{\pgfqpoint{4.334903in}{0.739656in}}%
\pgfpathlineto{\pgfqpoint{4.334607in}{0.739656in}}%
\pgfpathlineto{\pgfqpoint{4.334311in}{0.739656in}}%
\pgfpathlineto{\pgfqpoint{4.334015in}{0.739656in}}%
\pgfpathlineto{\pgfqpoint{4.333719in}{0.739656in}}%
\pgfpathlineto{\pgfqpoint{4.333423in}{0.739656in}}%
\pgfpathlineto{\pgfqpoint{4.333127in}{0.739656in}}%
\pgfpathlineto{\pgfqpoint{4.332831in}{0.739656in}}%
\pgfpathlineto{\pgfqpoint{4.332535in}{0.739656in}}%
\pgfpathlineto{\pgfqpoint{4.332239in}{0.739656in}}%
\pgfpathlineto{\pgfqpoint{4.331943in}{0.739656in}}%
\pgfpathlineto{\pgfqpoint{4.331647in}{0.739656in}}%
\pgfpathlineto{\pgfqpoint{4.331351in}{0.739656in}}%
\pgfpathlineto{\pgfqpoint{4.331055in}{0.739656in}}%
\pgfpathlineto{\pgfqpoint{4.330759in}{0.739656in}}%
\pgfpathlineto{\pgfqpoint{4.330463in}{0.739656in}}%
\pgfpathlineto{\pgfqpoint{4.330167in}{0.739656in}}%
\pgfpathlineto{\pgfqpoint{4.329871in}{0.739656in}}%
\pgfpathlineto{\pgfqpoint{4.329575in}{0.739656in}}%
\pgfpathlineto{\pgfqpoint{4.329279in}{0.739656in}}%
\pgfpathlineto{\pgfqpoint{4.328983in}{0.739656in}}%
\pgfpathlineto{\pgfqpoint{4.328687in}{0.739656in}}%
\pgfpathlineto{\pgfqpoint{4.328391in}{0.739656in}}%
\pgfpathlineto{\pgfqpoint{4.328095in}{0.739656in}}%
\pgfpathlineto{\pgfqpoint{4.327799in}{0.739656in}}%
\pgfpathlineto{\pgfqpoint{4.327503in}{0.739656in}}%
\pgfpathlineto{\pgfqpoint{4.327207in}{0.739656in}}%
\pgfpathlineto{\pgfqpoint{4.326911in}{0.739656in}}%
\pgfpathlineto{\pgfqpoint{4.326615in}{0.739656in}}%
\pgfpathlineto{\pgfqpoint{4.326319in}{0.739656in}}%
\pgfpathlineto{\pgfqpoint{4.326023in}{0.739656in}}%
\pgfpathlineto{\pgfqpoint{4.325727in}{0.739656in}}%
\pgfpathlineto{\pgfqpoint{4.325431in}{0.739656in}}%
\pgfpathlineto{\pgfqpoint{4.325135in}{0.739656in}}%
\pgfpathlineto{\pgfqpoint{4.324839in}{0.739656in}}%
\pgfpathlineto{\pgfqpoint{4.324543in}{0.739656in}}%
\pgfpathlineto{\pgfqpoint{4.324247in}{0.739656in}}%
\pgfpathlineto{\pgfqpoint{4.323951in}{0.739656in}}%
\pgfpathlineto{\pgfqpoint{4.323655in}{0.739656in}}%
\pgfpathlineto{\pgfqpoint{4.323359in}{0.739656in}}%
\pgfpathlineto{\pgfqpoint{4.323063in}{0.739656in}}%
\pgfpathlineto{\pgfqpoint{4.322767in}{0.739656in}}%
\pgfpathlineto{\pgfqpoint{4.322471in}{0.739656in}}%
\pgfpathlineto{\pgfqpoint{4.322175in}{0.739656in}}%
\pgfpathlineto{\pgfqpoint{4.321879in}{0.739656in}}%
\pgfpathlineto{\pgfqpoint{4.321583in}{0.739656in}}%
\pgfpathlineto{\pgfqpoint{4.321287in}{0.739656in}}%
\pgfpathlineto{\pgfqpoint{4.320991in}{0.739656in}}%
\pgfpathlineto{\pgfqpoint{4.320695in}{0.739656in}}%
\pgfpathlineto{\pgfqpoint{4.320399in}{0.739656in}}%
\pgfpathlineto{\pgfqpoint{4.320103in}{0.739656in}}%
\pgfpathlineto{\pgfqpoint{4.319807in}{0.739656in}}%
\pgfpathlineto{\pgfqpoint{4.319511in}{0.739656in}}%
\pgfpathlineto{\pgfqpoint{4.319215in}{0.739656in}}%
\pgfpathlineto{\pgfqpoint{4.318919in}{0.739656in}}%
\pgfpathlineto{\pgfqpoint{4.318623in}{0.739656in}}%
\pgfpathlineto{\pgfqpoint{4.318327in}{0.739656in}}%
\pgfpathlineto{\pgfqpoint{4.318031in}{0.739656in}}%
\pgfpathlineto{\pgfqpoint{4.317735in}{0.739656in}}%
\pgfpathlineto{\pgfqpoint{4.317439in}{0.739656in}}%
\pgfpathlineto{\pgfqpoint{4.317143in}{0.739656in}}%
\pgfpathlineto{\pgfqpoint{4.316847in}{0.739656in}}%
\pgfpathlineto{\pgfqpoint{4.316551in}{0.739656in}}%
\pgfpathlineto{\pgfqpoint{4.316255in}{0.739656in}}%
\pgfpathlineto{\pgfqpoint{4.315959in}{0.739656in}}%
\pgfpathlineto{\pgfqpoint{4.315663in}{0.739656in}}%
\pgfpathlineto{\pgfqpoint{4.315367in}{0.739656in}}%
\pgfpathlineto{\pgfqpoint{4.315071in}{0.739656in}}%
\pgfpathlineto{\pgfqpoint{4.314775in}{0.739656in}}%
\pgfpathlineto{\pgfqpoint{4.314479in}{0.739656in}}%
\pgfpathlineto{\pgfqpoint{4.314183in}{0.739656in}}%
\pgfpathlineto{\pgfqpoint{4.313887in}{0.739656in}}%
\pgfpathlineto{\pgfqpoint{4.313591in}{0.739656in}}%
\pgfpathlineto{\pgfqpoint{4.313295in}{0.739656in}}%
\pgfpathlineto{\pgfqpoint{4.312999in}{0.739656in}}%
\pgfpathlineto{\pgfqpoint{4.312703in}{0.739656in}}%
\pgfpathlineto{\pgfqpoint{4.312407in}{0.739656in}}%
\pgfpathlineto{\pgfqpoint{4.312111in}{0.739656in}}%
\pgfpathlineto{\pgfqpoint{4.311815in}{0.739656in}}%
\pgfpathlineto{\pgfqpoint{4.311519in}{0.739656in}}%
\pgfpathlineto{\pgfqpoint{4.311223in}{0.739656in}}%
\pgfpathlineto{\pgfqpoint{4.310927in}{0.739656in}}%
\pgfpathlineto{\pgfqpoint{4.310631in}{0.739656in}}%
\pgfpathlineto{\pgfqpoint{4.310335in}{0.739656in}}%
\pgfpathlineto{\pgfqpoint{4.310039in}{0.739656in}}%
\pgfpathlineto{\pgfqpoint{4.309743in}{0.739656in}}%
\pgfpathlineto{\pgfqpoint{4.309447in}{0.739656in}}%
\pgfpathlineto{\pgfqpoint{4.309151in}{0.739656in}}%
\pgfpathlineto{\pgfqpoint{4.308855in}{0.739656in}}%
\pgfpathlineto{\pgfqpoint{4.308559in}{0.739656in}}%
\pgfpathlineto{\pgfqpoint{4.308263in}{0.739656in}}%
\pgfpathlineto{\pgfqpoint{4.307967in}{0.739656in}}%
\pgfpathlineto{\pgfqpoint{4.307671in}{0.739656in}}%
\pgfpathlineto{\pgfqpoint{4.307375in}{0.739656in}}%
\pgfpathlineto{\pgfqpoint{4.307079in}{0.739656in}}%
\pgfpathlineto{\pgfqpoint{4.306783in}{0.739656in}}%
\pgfpathlineto{\pgfqpoint{4.306487in}{0.739656in}}%
\pgfpathlineto{\pgfqpoint{4.306191in}{0.739656in}}%
\pgfpathlineto{\pgfqpoint{4.305895in}{0.739656in}}%
\pgfpathlineto{\pgfqpoint{4.305599in}{0.739656in}}%
\pgfpathlineto{\pgfqpoint{4.305303in}{0.739656in}}%
\pgfpathlineto{\pgfqpoint{4.305007in}{0.739656in}}%
\pgfpathlineto{\pgfqpoint{4.304711in}{0.739656in}}%
\pgfpathlineto{\pgfqpoint{4.304415in}{0.739656in}}%
\pgfpathlineto{\pgfqpoint{4.304119in}{0.739656in}}%
\pgfpathlineto{\pgfqpoint{4.303823in}{0.739656in}}%
\pgfpathlineto{\pgfqpoint{4.303527in}{0.739656in}}%
\pgfpathlineto{\pgfqpoint{4.303231in}{0.739656in}}%
\pgfpathlineto{\pgfqpoint{4.302935in}{0.739656in}}%
\pgfpathlineto{\pgfqpoint{4.302639in}{0.739656in}}%
\pgfpathlineto{\pgfqpoint{4.302343in}{0.739656in}}%
\pgfpathlineto{\pgfqpoint{4.302047in}{0.739656in}}%
\pgfpathlineto{\pgfqpoint{4.301751in}{0.739656in}}%
\pgfpathlineto{\pgfqpoint{4.301455in}{0.739656in}}%
\pgfpathlineto{\pgfqpoint{4.301159in}{0.739656in}}%
\pgfpathlineto{\pgfqpoint{4.300863in}{0.739656in}}%
\pgfpathlineto{\pgfqpoint{4.300567in}{0.739656in}}%
\pgfpathlineto{\pgfqpoint{4.300271in}{0.739656in}}%
\pgfpathlineto{\pgfqpoint{4.299975in}{0.739656in}}%
\pgfpathlineto{\pgfqpoint{4.299679in}{0.739656in}}%
\pgfpathlineto{\pgfqpoint{4.299383in}{0.739656in}}%
\pgfpathlineto{\pgfqpoint{4.299087in}{0.739656in}}%
\pgfpathlineto{\pgfqpoint{4.298791in}{0.739656in}}%
\pgfpathlineto{\pgfqpoint{4.298495in}{0.739656in}}%
\pgfpathlineto{\pgfqpoint{4.298199in}{0.739656in}}%
\pgfpathlineto{\pgfqpoint{4.297903in}{0.739656in}}%
\pgfpathlineto{\pgfqpoint{4.297607in}{0.739656in}}%
\pgfpathlineto{\pgfqpoint{4.297311in}{0.739656in}}%
\pgfpathlineto{\pgfqpoint{4.297015in}{0.739656in}}%
\pgfpathlineto{\pgfqpoint{4.296719in}{0.739656in}}%
\pgfpathlineto{\pgfqpoint{4.296423in}{0.739656in}}%
\pgfpathlineto{\pgfqpoint{4.296127in}{0.739656in}}%
\pgfpathlineto{\pgfqpoint{4.295831in}{0.739656in}}%
\pgfpathlineto{\pgfqpoint{4.295535in}{0.739656in}}%
\pgfpathlineto{\pgfqpoint{4.295239in}{0.739656in}}%
\pgfpathlineto{\pgfqpoint{4.294943in}{0.739656in}}%
\pgfpathlineto{\pgfqpoint{4.294647in}{0.739656in}}%
\pgfpathlineto{\pgfqpoint{4.294351in}{0.739656in}}%
\pgfpathlineto{\pgfqpoint{4.294055in}{0.739656in}}%
\pgfpathlineto{\pgfqpoint{4.293759in}{0.739656in}}%
\pgfpathlineto{\pgfqpoint{4.293463in}{0.739656in}}%
\pgfpathlineto{\pgfqpoint{4.293167in}{0.739656in}}%
\pgfpathlineto{\pgfqpoint{4.292871in}{0.739656in}}%
\pgfpathlineto{\pgfqpoint{4.292575in}{0.739656in}}%
\pgfpathlineto{\pgfqpoint{4.292279in}{0.739656in}}%
\pgfpathlineto{\pgfqpoint{4.291983in}{0.739656in}}%
\pgfpathlineto{\pgfqpoint{4.291687in}{0.739656in}}%
\pgfpathlineto{\pgfqpoint{4.291391in}{0.739656in}}%
\pgfpathlineto{\pgfqpoint{4.291095in}{0.739656in}}%
\pgfpathlineto{\pgfqpoint{4.290799in}{0.739656in}}%
\pgfpathlineto{\pgfqpoint{4.290503in}{0.739656in}}%
\pgfpathlineto{\pgfqpoint{4.290207in}{0.739656in}}%
\pgfpathlineto{\pgfqpoint{4.289911in}{0.739656in}}%
\pgfpathlineto{\pgfqpoint{4.289615in}{0.739656in}}%
\pgfpathlineto{\pgfqpoint{4.289319in}{0.739656in}}%
\pgfpathlineto{\pgfqpoint{4.289023in}{0.739656in}}%
\pgfpathlineto{\pgfqpoint{4.288727in}{0.739656in}}%
\pgfpathlineto{\pgfqpoint{4.288431in}{0.739656in}}%
\pgfpathlineto{\pgfqpoint{4.288134in}{0.739656in}}%
\pgfpathlineto{\pgfqpoint{4.287838in}{0.739656in}}%
\pgfpathlineto{\pgfqpoint{4.287542in}{0.739656in}}%
\pgfpathlineto{\pgfqpoint{4.287246in}{0.739656in}}%
\pgfpathlineto{\pgfqpoint{4.286950in}{0.739656in}}%
\pgfpathlineto{\pgfqpoint{4.286654in}{0.739656in}}%
\pgfpathlineto{\pgfqpoint{4.286358in}{0.739656in}}%
\pgfpathlineto{\pgfqpoint{4.286062in}{0.739656in}}%
\pgfpathlineto{\pgfqpoint{4.285766in}{0.739656in}}%
\pgfpathlineto{\pgfqpoint{4.285470in}{0.739656in}}%
\pgfpathlineto{\pgfqpoint{4.285174in}{0.739656in}}%
\pgfpathlineto{\pgfqpoint{4.284878in}{0.739656in}}%
\pgfpathlineto{\pgfqpoint{4.284582in}{0.739656in}}%
\pgfpathlineto{\pgfqpoint{4.284286in}{0.739656in}}%
\pgfpathlineto{\pgfqpoint{4.283990in}{0.739656in}}%
\pgfpathlineto{\pgfqpoint{4.283694in}{0.739656in}}%
\pgfpathlineto{\pgfqpoint{4.283398in}{0.739656in}}%
\pgfpathlineto{\pgfqpoint{4.283102in}{0.739656in}}%
\pgfpathlineto{\pgfqpoint{4.282806in}{0.739656in}}%
\pgfpathlineto{\pgfqpoint{4.282510in}{0.739656in}}%
\pgfpathlineto{\pgfqpoint{4.282214in}{0.739656in}}%
\pgfpathlineto{\pgfqpoint{4.281918in}{0.739656in}}%
\pgfpathlineto{\pgfqpoint{4.281622in}{0.739656in}}%
\pgfpathlineto{\pgfqpoint{4.281326in}{0.739656in}}%
\pgfpathlineto{\pgfqpoint{4.281030in}{0.739656in}}%
\pgfpathlineto{\pgfqpoint{4.280734in}{0.739656in}}%
\pgfpathlineto{\pgfqpoint{4.280438in}{0.739656in}}%
\pgfpathlineto{\pgfqpoint{4.280142in}{0.739656in}}%
\pgfpathlineto{\pgfqpoint{4.279846in}{0.739656in}}%
\pgfpathlineto{\pgfqpoint{4.279550in}{0.739656in}}%
\pgfpathlineto{\pgfqpoint{4.279254in}{0.739656in}}%
\pgfpathlineto{\pgfqpoint{4.278958in}{0.739656in}}%
\pgfpathlineto{\pgfqpoint{4.278662in}{0.739656in}}%
\pgfpathlineto{\pgfqpoint{4.278366in}{0.739656in}}%
\pgfpathlineto{\pgfqpoint{4.278070in}{0.739656in}}%
\pgfpathlineto{\pgfqpoint{4.277774in}{0.739656in}}%
\pgfpathlineto{\pgfqpoint{4.277478in}{0.739656in}}%
\pgfpathlineto{\pgfqpoint{4.277182in}{0.739656in}}%
\pgfpathlineto{\pgfqpoint{4.276886in}{0.739656in}}%
\pgfpathlineto{\pgfqpoint{4.276590in}{0.739656in}}%
\pgfpathlineto{\pgfqpoint{4.276294in}{0.739656in}}%
\pgfpathlineto{\pgfqpoint{4.275998in}{0.739656in}}%
\pgfpathlineto{\pgfqpoint{4.275702in}{0.739656in}}%
\pgfpathlineto{\pgfqpoint{4.275406in}{0.739656in}}%
\pgfpathlineto{\pgfqpoint{4.275110in}{0.739656in}}%
\pgfpathlineto{\pgfqpoint{4.274814in}{0.739656in}}%
\pgfpathlineto{\pgfqpoint{4.274518in}{0.739656in}}%
\pgfpathlineto{\pgfqpoint{4.274222in}{0.739656in}}%
\pgfpathlineto{\pgfqpoint{4.273926in}{0.739656in}}%
\pgfpathlineto{\pgfqpoint{4.273630in}{0.739656in}}%
\pgfpathlineto{\pgfqpoint{4.273334in}{0.739656in}}%
\pgfpathlineto{\pgfqpoint{4.273038in}{0.739656in}}%
\pgfpathlineto{\pgfqpoint{4.272742in}{0.739656in}}%
\pgfpathlineto{\pgfqpoint{4.272446in}{0.739656in}}%
\pgfpathlineto{\pgfqpoint{4.272150in}{0.739656in}}%
\pgfpathlineto{\pgfqpoint{4.271854in}{0.739656in}}%
\pgfpathlineto{\pgfqpoint{4.271558in}{0.739656in}}%
\pgfpathlineto{\pgfqpoint{4.271262in}{0.739656in}}%
\pgfpathlineto{\pgfqpoint{4.270966in}{0.739656in}}%
\pgfpathlineto{\pgfqpoint{4.270670in}{0.739656in}}%
\pgfpathlineto{\pgfqpoint{4.270374in}{0.739656in}}%
\pgfpathlineto{\pgfqpoint{4.270078in}{0.739656in}}%
\pgfpathlineto{\pgfqpoint{4.269782in}{0.739656in}}%
\pgfpathlineto{\pgfqpoint{4.269486in}{0.739656in}}%
\pgfpathlineto{\pgfqpoint{4.269190in}{0.739656in}}%
\pgfpathlineto{\pgfqpoint{4.268894in}{0.739656in}}%
\pgfpathlineto{\pgfqpoint{4.268598in}{0.739656in}}%
\pgfpathlineto{\pgfqpoint{4.268302in}{0.739656in}}%
\pgfpathlineto{\pgfqpoint{4.268006in}{0.739656in}}%
\pgfpathlineto{\pgfqpoint{4.267710in}{0.739656in}}%
\pgfpathlineto{\pgfqpoint{4.267414in}{0.739656in}}%
\pgfpathlineto{\pgfqpoint{4.267118in}{0.739656in}}%
\pgfpathlineto{\pgfqpoint{4.266822in}{0.739656in}}%
\pgfpathlineto{\pgfqpoint{4.266526in}{0.739656in}}%
\pgfpathlineto{\pgfqpoint{4.266230in}{0.739656in}}%
\pgfpathlineto{\pgfqpoint{4.265934in}{0.739656in}}%
\pgfpathlineto{\pgfqpoint{4.265638in}{0.739656in}}%
\pgfpathlineto{\pgfqpoint{4.265342in}{0.739656in}}%
\pgfpathlineto{\pgfqpoint{4.265046in}{0.739656in}}%
\pgfpathlineto{\pgfqpoint{4.264750in}{0.739656in}}%
\pgfpathlineto{\pgfqpoint{4.264454in}{0.739656in}}%
\pgfpathlineto{\pgfqpoint{4.264158in}{0.739656in}}%
\pgfpathlineto{\pgfqpoint{4.263862in}{0.739656in}}%
\pgfpathlineto{\pgfqpoint{4.263566in}{0.739656in}}%
\pgfpathlineto{\pgfqpoint{4.263270in}{0.739656in}}%
\pgfpathlineto{\pgfqpoint{4.262974in}{0.739656in}}%
\pgfpathlineto{\pgfqpoint{4.262678in}{0.739656in}}%
\pgfpathlineto{\pgfqpoint{4.262382in}{0.739656in}}%
\pgfpathlineto{\pgfqpoint{4.262086in}{0.739656in}}%
\pgfpathlineto{\pgfqpoint{4.261790in}{0.739656in}}%
\pgfpathlineto{\pgfqpoint{4.261494in}{0.739656in}}%
\pgfpathlineto{\pgfqpoint{4.261198in}{0.739656in}}%
\pgfpathlineto{\pgfqpoint{4.260902in}{0.739656in}}%
\pgfpathlineto{\pgfqpoint{4.260606in}{0.739656in}}%
\pgfpathlineto{\pgfqpoint{4.260310in}{0.739656in}}%
\pgfpathlineto{\pgfqpoint{4.260014in}{0.739656in}}%
\pgfpathlineto{\pgfqpoint{4.259718in}{0.739656in}}%
\pgfpathlineto{\pgfqpoint{4.259422in}{0.739656in}}%
\pgfpathlineto{\pgfqpoint{4.259126in}{0.739656in}}%
\pgfpathlineto{\pgfqpoint{4.258830in}{0.739656in}}%
\pgfpathlineto{\pgfqpoint{4.258534in}{0.739656in}}%
\pgfpathlineto{\pgfqpoint{4.258238in}{0.739656in}}%
\pgfpathlineto{\pgfqpoint{4.257942in}{0.739656in}}%
\pgfpathlineto{\pgfqpoint{4.257646in}{0.739656in}}%
\pgfpathlineto{\pgfqpoint{4.257350in}{0.739656in}}%
\pgfpathlineto{\pgfqpoint{4.257054in}{0.739656in}}%
\pgfpathlineto{\pgfqpoint{4.256758in}{0.739656in}}%
\pgfpathlineto{\pgfqpoint{4.256462in}{0.739656in}}%
\pgfpathlineto{\pgfqpoint{4.256166in}{0.739656in}}%
\pgfpathlineto{\pgfqpoint{4.255870in}{0.739656in}}%
\pgfpathlineto{\pgfqpoint{4.255574in}{0.739656in}}%
\pgfpathlineto{\pgfqpoint{4.255278in}{0.739656in}}%
\pgfpathlineto{\pgfqpoint{4.254982in}{0.739656in}}%
\pgfpathlineto{\pgfqpoint{4.254686in}{0.739656in}}%
\pgfpathlineto{\pgfqpoint{4.254390in}{0.739656in}}%
\pgfpathlineto{\pgfqpoint{4.254094in}{0.739656in}}%
\pgfpathlineto{\pgfqpoint{4.253798in}{0.739656in}}%
\pgfpathlineto{\pgfqpoint{4.253502in}{0.739656in}}%
\pgfpathlineto{\pgfqpoint{4.253206in}{0.739656in}}%
\pgfpathlineto{\pgfqpoint{4.252910in}{0.739656in}}%
\pgfpathlineto{\pgfqpoint{4.252614in}{0.739656in}}%
\pgfpathlineto{\pgfqpoint{4.252318in}{0.739656in}}%
\pgfpathlineto{\pgfqpoint{4.252022in}{0.739656in}}%
\pgfpathlineto{\pgfqpoint{4.251726in}{0.739656in}}%
\pgfpathlineto{\pgfqpoint{4.251430in}{0.739656in}}%
\pgfpathlineto{\pgfqpoint{4.251134in}{0.739656in}}%
\pgfpathlineto{\pgfqpoint{4.250838in}{0.739656in}}%
\pgfpathlineto{\pgfqpoint{4.250542in}{0.739656in}}%
\pgfpathlineto{\pgfqpoint{4.250246in}{0.739656in}}%
\pgfpathlineto{\pgfqpoint{4.249950in}{0.739656in}}%
\pgfpathlineto{\pgfqpoint{4.249654in}{0.739656in}}%
\pgfpathlineto{\pgfqpoint{4.249358in}{0.739656in}}%
\pgfpathlineto{\pgfqpoint{4.249062in}{0.739656in}}%
\pgfpathlineto{\pgfqpoint{4.248766in}{0.739656in}}%
\pgfpathlineto{\pgfqpoint{4.248470in}{0.739656in}}%
\pgfpathlineto{\pgfqpoint{4.248174in}{0.739656in}}%
\pgfpathlineto{\pgfqpoint{4.247878in}{0.739656in}}%
\pgfpathlineto{\pgfqpoint{4.247582in}{0.739656in}}%
\pgfpathlineto{\pgfqpoint{4.247286in}{0.739656in}}%
\pgfpathlineto{\pgfqpoint{4.246990in}{0.739656in}}%
\pgfpathlineto{\pgfqpoint{4.246694in}{0.739656in}}%
\pgfpathlineto{\pgfqpoint{4.246398in}{0.739656in}}%
\pgfpathlineto{\pgfqpoint{4.246102in}{0.739656in}}%
\pgfpathlineto{\pgfqpoint{4.245806in}{0.739656in}}%
\pgfpathlineto{\pgfqpoint{4.245510in}{0.739656in}}%
\pgfpathlineto{\pgfqpoint{4.245214in}{0.739656in}}%
\pgfpathlineto{\pgfqpoint{4.244918in}{0.739656in}}%
\pgfpathlineto{\pgfqpoint{4.244622in}{0.739656in}}%
\pgfpathlineto{\pgfqpoint{4.244326in}{0.739656in}}%
\pgfpathlineto{\pgfqpoint{4.244030in}{0.739656in}}%
\pgfpathlineto{\pgfqpoint{4.243734in}{0.739656in}}%
\pgfpathlineto{\pgfqpoint{4.243438in}{0.739656in}}%
\pgfpathlineto{\pgfqpoint{4.243142in}{0.739656in}}%
\pgfpathlineto{\pgfqpoint{4.242846in}{0.739656in}}%
\pgfpathlineto{\pgfqpoint{4.242550in}{0.739656in}}%
\pgfpathlineto{\pgfqpoint{4.242254in}{0.739656in}}%
\pgfpathlineto{\pgfqpoint{4.241958in}{0.739656in}}%
\pgfpathlineto{\pgfqpoint{4.241662in}{0.739656in}}%
\pgfpathlineto{\pgfqpoint{4.241366in}{0.739656in}}%
\pgfpathlineto{\pgfqpoint{4.241070in}{0.739656in}}%
\pgfpathlineto{\pgfqpoint{4.240774in}{0.739656in}}%
\pgfpathlineto{\pgfqpoint{4.240478in}{0.739656in}}%
\pgfpathlineto{\pgfqpoint{4.240182in}{0.739656in}}%
\pgfpathlineto{\pgfqpoint{4.239886in}{0.739656in}}%
\pgfpathlineto{\pgfqpoint{4.239590in}{0.739656in}}%
\pgfpathlineto{\pgfqpoint{4.239294in}{0.739656in}}%
\pgfpathlineto{\pgfqpoint{4.238998in}{0.739656in}}%
\pgfpathlineto{\pgfqpoint{4.238702in}{0.739656in}}%
\pgfpathlineto{\pgfqpoint{4.238406in}{0.739656in}}%
\pgfpathlineto{\pgfqpoint{4.238110in}{0.739656in}}%
\pgfpathlineto{\pgfqpoint{4.237814in}{0.739656in}}%
\pgfpathlineto{\pgfqpoint{4.237518in}{0.739656in}}%
\pgfpathlineto{\pgfqpoint{4.237222in}{0.739656in}}%
\pgfpathlineto{\pgfqpoint{4.236926in}{0.739656in}}%
\pgfpathlineto{\pgfqpoint{4.236630in}{0.739656in}}%
\pgfpathlineto{\pgfqpoint{4.236334in}{0.739656in}}%
\pgfpathlineto{\pgfqpoint{4.236038in}{0.739656in}}%
\pgfpathlineto{\pgfqpoint{4.235742in}{0.739656in}}%
\pgfpathlineto{\pgfqpoint{4.235446in}{0.739656in}}%
\pgfpathlineto{\pgfqpoint{4.235150in}{0.739656in}}%
\pgfpathlineto{\pgfqpoint{4.234854in}{0.739656in}}%
\pgfpathlineto{\pgfqpoint{4.234558in}{0.739656in}}%
\pgfpathlineto{\pgfqpoint{4.234262in}{0.739656in}}%
\pgfpathlineto{\pgfqpoint{4.233966in}{0.739656in}}%
\pgfpathlineto{\pgfqpoint{4.233670in}{0.739656in}}%
\pgfpathlineto{\pgfqpoint{4.233374in}{0.739656in}}%
\pgfpathlineto{\pgfqpoint{4.233078in}{0.739656in}}%
\pgfpathlineto{\pgfqpoint{4.232782in}{0.739656in}}%
\pgfpathlineto{\pgfqpoint{4.232486in}{0.739656in}}%
\pgfpathlineto{\pgfqpoint{4.232190in}{0.739656in}}%
\pgfpathlineto{\pgfqpoint{4.231894in}{0.739656in}}%
\pgfpathlineto{\pgfqpoint{4.231598in}{0.739656in}}%
\pgfpathlineto{\pgfqpoint{4.231302in}{0.739656in}}%
\pgfpathlineto{\pgfqpoint{4.231006in}{0.739656in}}%
\pgfpathlineto{\pgfqpoint{4.230710in}{0.739656in}}%
\pgfpathlineto{\pgfqpoint{4.230414in}{0.739656in}}%
\pgfpathlineto{\pgfqpoint{4.230118in}{0.739656in}}%
\pgfpathlineto{\pgfqpoint{4.229822in}{0.739656in}}%
\pgfpathlineto{\pgfqpoint{4.229526in}{0.739656in}}%
\pgfpathlineto{\pgfqpoint{4.229230in}{0.739656in}}%
\pgfpathlineto{\pgfqpoint{4.228934in}{0.739656in}}%
\pgfpathlineto{\pgfqpoint{4.228638in}{0.739656in}}%
\pgfpathlineto{\pgfqpoint{4.228342in}{0.739656in}}%
\pgfpathlineto{\pgfqpoint{4.228046in}{0.739656in}}%
\pgfpathlineto{\pgfqpoint{4.227750in}{0.739656in}}%
\pgfpathlineto{\pgfqpoint{4.227454in}{0.739656in}}%
\pgfpathlineto{\pgfqpoint{4.227158in}{0.739656in}}%
\pgfpathlineto{\pgfqpoint{4.226862in}{0.739656in}}%
\pgfpathlineto{\pgfqpoint{4.226566in}{0.739656in}}%
\pgfpathlineto{\pgfqpoint{4.226270in}{0.739656in}}%
\pgfpathlineto{\pgfqpoint{4.225974in}{0.739656in}}%
\pgfpathlineto{\pgfqpoint{4.225678in}{0.739656in}}%
\pgfpathlineto{\pgfqpoint{4.225382in}{0.739656in}}%
\pgfpathlineto{\pgfqpoint{4.225086in}{0.739656in}}%
\pgfpathlineto{\pgfqpoint{4.224790in}{0.739656in}}%
\pgfpathlineto{\pgfqpoint{4.224494in}{0.739656in}}%
\pgfpathlineto{\pgfqpoint{4.224198in}{0.739656in}}%
\pgfpathlineto{\pgfqpoint{4.223902in}{0.739656in}}%
\pgfpathlineto{\pgfqpoint{4.223606in}{0.739656in}}%
\pgfpathlineto{\pgfqpoint{4.223310in}{0.739656in}}%
\pgfpathlineto{\pgfqpoint{4.223014in}{0.739656in}}%
\pgfpathlineto{\pgfqpoint{4.222718in}{0.739656in}}%
\pgfpathlineto{\pgfqpoint{4.222422in}{0.739656in}}%
\pgfpathlineto{\pgfqpoint{4.222126in}{0.739656in}}%
\pgfpathlineto{\pgfqpoint{4.221830in}{0.739656in}}%
\pgfpathlineto{\pgfqpoint{4.221534in}{0.739656in}}%
\pgfpathlineto{\pgfqpoint{4.221238in}{0.739656in}}%
\pgfpathlineto{\pgfqpoint{4.220941in}{0.739656in}}%
\pgfpathlineto{\pgfqpoint{4.220645in}{0.739656in}}%
\pgfpathlineto{\pgfqpoint{4.220349in}{0.739656in}}%
\pgfpathlineto{\pgfqpoint{4.220053in}{0.739656in}}%
\pgfpathlineto{\pgfqpoint{4.219757in}{0.739656in}}%
\pgfpathlineto{\pgfqpoint{4.219461in}{0.739656in}}%
\pgfpathlineto{\pgfqpoint{4.219165in}{0.739656in}}%
\pgfpathlineto{\pgfqpoint{4.218869in}{0.739656in}}%
\pgfpathlineto{\pgfqpoint{4.218573in}{0.739656in}}%
\pgfpathlineto{\pgfqpoint{4.218277in}{0.739656in}}%
\pgfpathlineto{\pgfqpoint{4.217981in}{0.739656in}}%
\pgfpathlineto{\pgfqpoint{4.217685in}{0.739656in}}%
\pgfpathlineto{\pgfqpoint{4.217389in}{0.739656in}}%
\pgfpathlineto{\pgfqpoint{4.217093in}{0.739656in}}%
\pgfpathlineto{\pgfqpoint{4.216797in}{0.739656in}}%
\pgfpathlineto{\pgfqpoint{4.216501in}{0.739656in}}%
\pgfpathlineto{\pgfqpoint{4.216205in}{0.739656in}}%
\pgfpathlineto{\pgfqpoint{4.215909in}{0.739656in}}%
\pgfpathlineto{\pgfqpoint{4.215613in}{0.739656in}}%
\pgfpathlineto{\pgfqpoint{4.215317in}{0.739656in}}%
\pgfpathlineto{\pgfqpoint{4.215021in}{0.739656in}}%
\pgfpathlineto{\pgfqpoint{4.214725in}{0.739656in}}%
\pgfpathlineto{\pgfqpoint{4.214429in}{0.739656in}}%
\pgfpathlineto{\pgfqpoint{4.214133in}{0.739656in}}%
\pgfpathlineto{\pgfqpoint{4.213837in}{0.739656in}}%
\pgfpathlineto{\pgfqpoint{4.213541in}{0.739656in}}%
\pgfpathlineto{\pgfqpoint{4.213245in}{0.739656in}}%
\pgfpathlineto{\pgfqpoint{4.212949in}{0.739656in}}%
\pgfpathlineto{\pgfqpoint{4.212653in}{0.739656in}}%
\pgfpathlineto{\pgfqpoint{4.212357in}{0.739656in}}%
\pgfpathlineto{\pgfqpoint{4.212061in}{0.739656in}}%
\pgfpathlineto{\pgfqpoint{4.211765in}{0.739656in}}%
\pgfpathlineto{\pgfqpoint{4.211469in}{0.739656in}}%
\pgfpathlineto{\pgfqpoint{4.211173in}{0.739656in}}%
\pgfpathlineto{\pgfqpoint{4.210877in}{0.739656in}}%
\pgfpathlineto{\pgfqpoint{4.210581in}{0.739656in}}%
\pgfpathlineto{\pgfqpoint{4.210285in}{0.739656in}}%
\pgfpathlineto{\pgfqpoint{4.209989in}{0.739656in}}%
\pgfpathlineto{\pgfqpoint{4.209693in}{0.739656in}}%
\pgfpathlineto{\pgfqpoint{4.209397in}{0.739656in}}%
\pgfpathlineto{\pgfqpoint{4.209101in}{0.739656in}}%
\pgfpathlineto{\pgfqpoint{4.208805in}{0.739656in}}%
\pgfpathlineto{\pgfqpoint{4.208509in}{0.739656in}}%
\pgfpathlineto{\pgfqpoint{4.208213in}{0.739656in}}%
\pgfpathlineto{\pgfqpoint{4.207917in}{0.739656in}}%
\pgfpathlineto{\pgfqpoint{4.207621in}{0.739656in}}%
\pgfpathlineto{\pgfqpoint{4.207325in}{0.739656in}}%
\pgfpathlineto{\pgfqpoint{4.207029in}{0.739656in}}%
\pgfpathlineto{\pgfqpoint{4.206733in}{0.739656in}}%
\pgfpathlineto{\pgfqpoint{4.206437in}{0.739656in}}%
\pgfpathlineto{\pgfqpoint{4.206141in}{0.739656in}}%
\pgfpathlineto{\pgfqpoint{4.205845in}{0.739656in}}%
\pgfpathlineto{\pgfqpoint{4.205549in}{0.739656in}}%
\pgfpathlineto{\pgfqpoint{4.205253in}{0.739656in}}%
\pgfpathlineto{\pgfqpoint{4.204957in}{0.739656in}}%
\pgfpathlineto{\pgfqpoint{4.204661in}{0.739656in}}%
\pgfpathlineto{\pgfqpoint{4.204365in}{0.739656in}}%
\pgfpathlineto{\pgfqpoint{4.204069in}{0.739656in}}%
\pgfpathlineto{\pgfqpoint{4.203773in}{0.739656in}}%
\pgfpathlineto{\pgfqpoint{4.203477in}{0.739656in}}%
\pgfpathlineto{\pgfqpoint{4.203181in}{0.739656in}}%
\pgfpathlineto{\pgfqpoint{4.202885in}{0.739656in}}%
\pgfpathlineto{\pgfqpoint{4.202589in}{0.739656in}}%
\pgfpathlineto{\pgfqpoint{4.202293in}{0.739656in}}%
\pgfpathlineto{\pgfqpoint{4.201997in}{0.739656in}}%
\pgfpathlineto{\pgfqpoint{4.201701in}{0.739656in}}%
\pgfpathlineto{\pgfqpoint{4.201405in}{0.739656in}}%
\pgfpathlineto{\pgfqpoint{4.201109in}{0.739656in}}%
\pgfpathlineto{\pgfqpoint{4.200813in}{0.739656in}}%
\pgfpathlineto{\pgfqpoint{4.200517in}{0.739656in}}%
\pgfpathlineto{\pgfqpoint{4.200221in}{0.739656in}}%
\pgfpathlineto{\pgfqpoint{4.199925in}{0.739656in}}%
\pgfpathlineto{\pgfqpoint{4.199629in}{0.739656in}}%
\pgfpathlineto{\pgfqpoint{4.199333in}{0.739656in}}%
\pgfpathlineto{\pgfqpoint{4.199037in}{0.739656in}}%
\pgfpathlineto{\pgfqpoint{4.198741in}{0.739656in}}%
\pgfpathlineto{\pgfqpoint{4.198445in}{0.739656in}}%
\pgfpathlineto{\pgfqpoint{4.198149in}{0.739656in}}%
\pgfpathlineto{\pgfqpoint{4.197853in}{0.739656in}}%
\pgfpathlineto{\pgfqpoint{4.197557in}{0.739656in}}%
\pgfpathlineto{\pgfqpoint{4.197261in}{0.739656in}}%
\pgfpathlineto{\pgfqpoint{4.196965in}{0.739656in}}%
\pgfpathlineto{\pgfqpoint{4.196669in}{0.739656in}}%
\pgfpathlineto{\pgfqpoint{4.196373in}{0.739656in}}%
\pgfpathlineto{\pgfqpoint{4.196077in}{0.739656in}}%
\pgfpathlineto{\pgfqpoint{4.195781in}{0.739656in}}%
\pgfpathlineto{\pgfqpoint{4.195485in}{0.739656in}}%
\pgfpathlineto{\pgfqpoint{4.195189in}{0.739656in}}%
\pgfpathlineto{\pgfqpoint{4.194893in}{0.739656in}}%
\pgfpathlineto{\pgfqpoint{4.194597in}{0.739656in}}%
\pgfpathlineto{\pgfqpoint{4.194301in}{0.739656in}}%
\pgfpathlineto{\pgfqpoint{4.194005in}{0.739656in}}%
\pgfpathlineto{\pgfqpoint{4.193709in}{0.739656in}}%
\pgfpathlineto{\pgfqpoint{4.193413in}{0.739656in}}%
\pgfpathlineto{\pgfqpoint{4.193117in}{0.739656in}}%
\pgfpathlineto{\pgfqpoint{4.192821in}{0.739656in}}%
\pgfpathlineto{\pgfqpoint{4.192525in}{0.739656in}}%
\pgfpathlineto{\pgfqpoint{4.192229in}{0.739656in}}%
\pgfpathlineto{\pgfqpoint{4.191933in}{0.739656in}}%
\pgfpathlineto{\pgfqpoint{4.191637in}{0.739656in}}%
\pgfpathlineto{\pgfqpoint{4.191341in}{0.739656in}}%
\pgfpathlineto{\pgfqpoint{4.191045in}{0.739656in}}%
\pgfpathlineto{\pgfqpoint{4.190749in}{0.739656in}}%
\pgfpathlineto{\pgfqpoint{4.190453in}{0.739656in}}%
\pgfpathlineto{\pgfqpoint{4.190157in}{0.739656in}}%
\pgfpathlineto{\pgfqpoint{4.189861in}{0.739656in}}%
\pgfpathlineto{\pgfqpoint{4.189565in}{0.739656in}}%
\pgfpathlineto{\pgfqpoint{4.189269in}{0.739656in}}%
\pgfpathlineto{\pgfqpoint{4.188973in}{0.739656in}}%
\pgfpathlineto{\pgfqpoint{4.188677in}{0.739656in}}%
\pgfpathlineto{\pgfqpoint{4.188381in}{0.739656in}}%
\pgfpathlineto{\pgfqpoint{4.188085in}{0.739656in}}%
\pgfpathlineto{\pgfqpoint{4.187789in}{0.739656in}}%
\pgfpathlineto{\pgfqpoint{4.187493in}{0.739656in}}%
\pgfpathlineto{\pgfqpoint{4.187197in}{0.739656in}}%
\pgfpathlineto{\pgfqpoint{4.186901in}{0.739656in}}%
\pgfpathlineto{\pgfqpoint{4.186605in}{0.739656in}}%
\pgfpathlineto{\pgfqpoint{4.186309in}{0.739656in}}%
\pgfpathlineto{\pgfqpoint{4.186013in}{0.739656in}}%
\pgfpathlineto{\pgfqpoint{4.185717in}{0.739656in}}%
\pgfpathlineto{\pgfqpoint{4.185421in}{0.739656in}}%
\pgfpathlineto{\pgfqpoint{4.185125in}{0.739656in}}%
\pgfpathlineto{\pgfqpoint{4.184829in}{0.739656in}}%
\pgfpathlineto{\pgfqpoint{4.184533in}{0.739656in}}%
\pgfpathlineto{\pgfqpoint{4.184237in}{0.739656in}}%
\pgfpathlineto{\pgfqpoint{4.183941in}{0.739656in}}%
\pgfpathlineto{\pgfqpoint{4.183645in}{0.739656in}}%
\pgfpathlineto{\pgfqpoint{4.183349in}{0.739656in}}%
\pgfpathlineto{\pgfqpoint{4.183053in}{0.739656in}}%
\pgfpathlineto{\pgfqpoint{4.182757in}{0.739656in}}%
\pgfpathlineto{\pgfqpoint{4.182461in}{0.739656in}}%
\pgfpathlineto{\pgfqpoint{4.182165in}{0.739656in}}%
\pgfpathlineto{\pgfqpoint{4.181869in}{0.739656in}}%
\pgfpathlineto{\pgfqpoint{4.181573in}{0.739656in}}%
\pgfpathlineto{\pgfqpoint{4.181277in}{0.739656in}}%
\pgfpathlineto{\pgfqpoint{4.180981in}{0.739656in}}%
\pgfpathlineto{\pgfqpoint{4.180685in}{0.739656in}}%
\pgfpathlineto{\pgfqpoint{4.180389in}{0.739656in}}%
\pgfpathlineto{\pgfqpoint{4.180093in}{0.739656in}}%
\pgfpathlineto{\pgfqpoint{4.179797in}{0.739656in}}%
\pgfpathlineto{\pgfqpoint{4.179501in}{0.739656in}}%
\pgfpathlineto{\pgfqpoint{4.179205in}{0.739656in}}%
\pgfpathlineto{\pgfqpoint{4.178909in}{0.739656in}}%
\pgfpathlineto{\pgfqpoint{4.178613in}{0.739656in}}%
\pgfpathlineto{\pgfqpoint{4.178317in}{0.739656in}}%
\pgfpathlineto{\pgfqpoint{4.178021in}{0.739656in}}%
\pgfpathlineto{\pgfqpoint{4.177725in}{0.739656in}}%
\pgfpathlineto{\pgfqpoint{4.177429in}{0.739656in}}%
\pgfpathlineto{\pgfqpoint{4.177133in}{0.739656in}}%
\pgfpathlineto{\pgfqpoint{4.176837in}{0.739656in}}%
\pgfpathlineto{\pgfqpoint{4.176541in}{0.739656in}}%
\pgfpathlineto{\pgfqpoint{4.176245in}{0.739656in}}%
\pgfpathlineto{\pgfqpoint{4.175949in}{0.739656in}}%
\pgfpathlineto{\pgfqpoint{4.175653in}{0.739656in}}%
\pgfpathlineto{\pgfqpoint{4.175357in}{0.739656in}}%
\pgfpathlineto{\pgfqpoint{4.175061in}{0.739656in}}%
\pgfpathlineto{\pgfqpoint{4.174765in}{0.739656in}}%
\pgfpathlineto{\pgfqpoint{4.174469in}{0.739656in}}%
\pgfpathlineto{\pgfqpoint{4.174173in}{0.739656in}}%
\pgfpathlineto{\pgfqpoint{4.173877in}{0.739656in}}%
\pgfpathlineto{\pgfqpoint{4.173581in}{0.739656in}}%
\pgfpathlineto{\pgfqpoint{4.173285in}{0.739656in}}%
\pgfpathlineto{\pgfqpoint{4.172989in}{0.739656in}}%
\pgfpathlineto{\pgfqpoint{4.172693in}{0.739656in}}%
\pgfpathlineto{\pgfqpoint{4.172397in}{0.739656in}}%
\pgfpathlineto{\pgfqpoint{4.172101in}{0.739656in}}%
\pgfpathlineto{\pgfqpoint{4.171805in}{0.739656in}}%
\pgfpathlineto{\pgfqpoint{4.171509in}{0.739656in}}%
\pgfpathlineto{\pgfqpoint{4.171213in}{0.739656in}}%
\pgfpathlineto{\pgfqpoint{4.170917in}{0.739656in}}%
\pgfpathlineto{\pgfqpoint{4.170621in}{0.739656in}}%
\pgfpathlineto{\pgfqpoint{4.170325in}{0.739656in}}%
\pgfpathlineto{\pgfqpoint{4.170029in}{0.739656in}}%
\pgfpathlineto{\pgfqpoint{4.169733in}{0.739656in}}%
\pgfpathlineto{\pgfqpoint{4.169437in}{0.739656in}}%
\pgfpathlineto{\pgfqpoint{4.169141in}{0.739656in}}%
\pgfpathlineto{\pgfqpoint{4.168845in}{0.739656in}}%
\pgfpathlineto{\pgfqpoint{4.168549in}{0.739656in}}%
\pgfpathlineto{\pgfqpoint{4.168253in}{0.739656in}}%
\pgfpathlineto{\pgfqpoint{4.167957in}{0.739656in}}%
\pgfpathlineto{\pgfqpoint{4.167661in}{0.739656in}}%
\pgfpathlineto{\pgfqpoint{4.167365in}{0.739656in}}%
\pgfpathlineto{\pgfqpoint{4.167069in}{0.739656in}}%
\pgfpathlineto{\pgfqpoint{4.166773in}{0.739656in}}%
\pgfpathlineto{\pgfqpoint{4.166477in}{0.739656in}}%
\pgfpathlineto{\pgfqpoint{4.166181in}{0.739656in}}%
\pgfpathlineto{\pgfqpoint{4.165885in}{0.739656in}}%
\pgfpathlineto{\pgfqpoint{4.165589in}{0.739656in}}%
\pgfpathlineto{\pgfqpoint{4.165293in}{0.739656in}}%
\pgfpathlineto{\pgfqpoint{4.164997in}{0.739656in}}%
\pgfpathlineto{\pgfqpoint{4.164701in}{0.739656in}}%
\pgfpathlineto{\pgfqpoint{4.164405in}{0.739656in}}%
\pgfpathlineto{\pgfqpoint{4.164109in}{0.739656in}}%
\pgfpathlineto{\pgfqpoint{4.163813in}{0.739656in}}%
\pgfpathlineto{\pgfqpoint{4.163517in}{0.739656in}}%
\pgfpathlineto{\pgfqpoint{4.163221in}{0.739656in}}%
\pgfpathlineto{\pgfqpoint{4.162925in}{0.739656in}}%
\pgfpathlineto{\pgfqpoint{4.162629in}{0.739656in}}%
\pgfpathlineto{\pgfqpoint{4.162333in}{0.739656in}}%
\pgfpathlineto{\pgfqpoint{4.162037in}{0.739656in}}%
\pgfpathlineto{\pgfqpoint{4.161741in}{0.739656in}}%
\pgfpathlineto{\pgfqpoint{4.161445in}{0.739656in}}%
\pgfpathlineto{\pgfqpoint{4.161149in}{0.739656in}}%
\pgfpathlineto{\pgfqpoint{4.160853in}{0.739656in}}%
\pgfpathlineto{\pgfqpoint{4.160557in}{0.739656in}}%
\pgfpathlineto{\pgfqpoint{4.160261in}{0.739656in}}%
\pgfpathlineto{\pgfqpoint{4.159965in}{0.739656in}}%
\pgfpathlineto{\pgfqpoint{4.159669in}{0.739656in}}%
\pgfpathlineto{\pgfqpoint{4.159373in}{0.739656in}}%
\pgfpathlineto{\pgfqpoint{4.159077in}{0.739656in}}%
\pgfpathlineto{\pgfqpoint{4.158781in}{0.739656in}}%
\pgfpathlineto{\pgfqpoint{4.158485in}{0.739656in}}%
\pgfpathlineto{\pgfqpoint{4.158189in}{0.739656in}}%
\pgfpathlineto{\pgfqpoint{4.157893in}{0.739656in}}%
\pgfpathlineto{\pgfqpoint{4.157597in}{0.739656in}}%
\pgfpathlineto{\pgfqpoint{4.157301in}{0.739656in}}%
\pgfpathlineto{\pgfqpoint{4.157005in}{0.739656in}}%
\pgfpathlineto{\pgfqpoint{4.156709in}{0.739656in}}%
\pgfpathlineto{\pgfqpoint{4.156413in}{0.739656in}}%
\pgfpathlineto{\pgfqpoint{4.156117in}{0.739656in}}%
\pgfpathlineto{\pgfqpoint{4.155821in}{0.739656in}}%
\pgfpathlineto{\pgfqpoint{4.155525in}{0.739656in}}%
\pgfpathlineto{\pgfqpoint{4.155229in}{0.739656in}}%
\pgfpathlineto{\pgfqpoint{4.154933in}{0.739656in}}%
\pgfpathlineto{\pgfqpoint{4.154637in}{0.739656in}}%
\pgfpathlineto{\pgfqpoint{4.154341in}{0.739656in}}%
\pgfpathlineto{\pgfqpoint{4.154045in}{0.739656in}}%
\pgfpathlineto{\pgfqpoint{4.153749in}{0.739656in}}%
\pgfpathlineto{\pgfqpoint{4.153452in}{0.739656in}}%
\pgfpathlineto{\pgfqpoint{4.153156in}{0.739656in}}%
\pgfpathlineto{\pgfqpoint{4.152860in}{0.739656in}}%
\pgfpathlineto{\pgfqpoint{4.152564in}{0.739656in}}%
\pgfpathlineto{\pgfqpoint{4.152268in}{0.739656in}}%
\pgfpathlineto{\pgfqpoint{4.151972in}{0.739656in}}%
\pgfpathlineto{\pgfqpoint{4.151676in}{0.739656in}}%
\pgfpathlineto{\pgfqpoint{4.151380in}{0.739656in}}%
\pgfpathlineto{\pgfqpoint{4.151084in}{0.739656in}}%
\pgfpathlineto{\pgfqpoint{4.150788in}{0.739656in}}%
\pgfpathlineto{\pgfqpoint{4.150492in}{0.739656in}}%
\pgfpathlineto{\pgfqpoint{4.150196in}{0.739656in}}%
\pgfpathlineto{\pgfqpoint{4.149900in}{0.739656in}}%
\pgfpathlineto{\pgfqpoint{4.149604in}{0.739656in}}%
\pgfpathlineto{\pgfqpoint{4.149308in}{0.739656in}}%
\pgfpathlineto{\pgfqpoint{4.149012in}{0.739656in}}%
\pgfpathlineto{\pgfqpoint{4.148716in}{0.739656in}}%
\pgfpathlineto{\pgfqpoint{4.148420in}{0.739656in}}%
\pgfpathlineto{\pgfqpoint{4.148124in}{0.739656in}}%
\pgfpathlineto{\pgfqpoint{4.147828in}{0.739656in}}%
\pgfpathlineto{\pgfqpoint{4.147532in}{0.739656in}}%
\pgfpathlineto{\pgfqpoint{4.147236in}{0.739656in}}%
\pgfpathlineto{\pgfqpoint{4.146940in}{0.739656in}}%
\pgfpathlineto{\pgfqpoint{4.146644in}{0.739656in}}%
\pgfpathlineto{\pgfqpoint{4.146348in}{0.739656in}}%
\pgfpathlineto{\pgfqpoint{4.146052in}{0.739656in}}%
\pgfpathlineto{\pgfqpoint{4.145756in}{0.739656in}}%
\pgfpathlineto{\pgfqpoint{4.145460in}{0.739656in}}%
\pgfpathlineto{\pgfqpoint{4.145164in}{0.739656in}}%
\pgfpathlineto{\pgfqpoint{4.144868in}{0.739656in}}%
\pgfpathlineto{\pgfqpoint{4.144572in}{0.739656in}}%
\pgfpathlineto{\pgfqpoint{4.144276in}{0.739656in}}%
\pgfpathlineto{\pgfqpoint{4.143980in}{0.739656in}}%
\pgfpathlineto{\pgfqpoint{4.143684in}{0.739656in}}%
\pgfpathlineto{\pgfqpoint{4.143388in}{0.739656in}}%
\pgfpathlineto{\pgfqpoint{4.143092in}{0.739656in}}%
\pgfpathlineto{\pgfqpoint{4.142796in}{0.739656in}}%
\pgfpathlineto{\pgfqpoint{4.142500in}{0.739656in}}%
\pgfpathlineto{\pgfqpoint{4.142204in}{0.739656in}}%
\pgfpathlineto{\pgfqpoint{4.141908in}{0.739656in}}%
\pgfpathlineto{\pgfqpoint{4.141612in}{0.739656in}}%
\pgfpathlineto{\pgfqpoint{4.141316in}{0.739656in}}%
\pgfpathlineto{\pgfqpoint{4.141020in}{0.739656in}}%
\pgfpathlineto{\pgfqpoint{4.140724in}{0.739656in}}%
\pgfpathlineto{\pgfqpoint{4.140428in}{0.739656in}}%
\pgfpathlineto{\pgfqpoint{4.140132in}{0.739656in}}%
\pgfpathlineto{\pgfqpoint{4.139836in}{0.739656in}}%
\pgfpathlineto{\pgfqpoint{4.139540in}{0.739656in}}%
\pgfpathlineto{\pgfqpoint{4.139244in}{0.739656in}}%
\pgfpathlineto{\pgfqpoint{4.138948in}{0.739656in}}%
\pgfpathlineto{\pgfqpoint{4.138652in}{0.739656in}}%
\pgfpathlineto{\pgfqpoint{4.138356in}{0.739656in}}%
\pgfpathlineto{\pgfqpoint{4.138060in}{0.739656in}}%
\pgfpathlineto{\pgfqpoint{4.137764in}{0.739656in}}%
\pgfpathlineto{\pgfqpoint{4.137468in}{0.739656in}}%
\pgfpathlineto{\pgfqpoint{4.137172in}{0.739656in}}%
\pgfpathlineto{\pgfqpoint{4.136876in}{0.739656in}}%
\pgfpathlineto{\pgfqpoint{4.136580in}{0.739656in}}%
\pgfpathlineto{\pgfqpoint{4.136284in}{0.739656in}}%
\pgfpathlineto{\pgfqpoint{4.135988in}{0.739656in}}%
\pgfpathlineto{\pgfqpoint{4.135692in}{0.739656in}}%
\pgfpathlineto{\pgfqpoint{4.135396in}{0.739656in}}%
\pgfpathlineto{\pgfqpoint{4.135100in}{0.739656in}}%
\pgfpathlineto{\pgfqpoint{4.134804in}{0.739656in}}%
\pgfpathlineto{\pgfqpoint{4.134508in}{0.739656in}}%
\pgfpathlineto{\pgfqpoint{4.134212in}{0.739656in}}%
\pgfpathlineto{\pgfqpoint{4.133916in}{0.739656in}}%
\pgfpathlineto{\pgfqpoint{4.133620in}{0.739656in}}%
\pgfpathlineto{\pgfqpoint{4.133324in}{0.739656in}}%
\pgfpathlineto{\pgfqpoint{4.133028in}{0.739656in}}%
\pgfpathlineto{\pgfqpoint{4.132732in}{0.739656in}}%
\pgfpathlineto{\pgfqpoint{4.132436in}{0.739656in}}%
\pgfpathlineto{\pgfqpoint{4.132140in}{0.739656in}}%
\pgfpathlineto{\pgfqpoint{4.131844in}{0.739656in}}%
\pgfpathlineto{\pgfqpoint{4.131548in}{0.739656in}}%
\pgfpathlineto{\pgfqpoint{4.131252in}{0.739656in}}%
\pgfpathlineto{\pgfqpoint{4.130956in}{0.739656in}}%
\pgfpathlineto{\pgfqpoint{4.130660in}{0.739656in}}%
\pgfpathlineto{\pgfqpoint{4.130364in}{0.739656in}}%
\pgfpathlineto{\pgfqpoint{4.130068in}{0.739656in}}%
\pgfpathlineto{\pgfqpoint{4.129772in}{0.739656in}}%
\pgfpathlineto{\pgfqpoint{4.129476in}{0.739656in}}%
\pgfpathlineto{\pgfqpoint{4.129180in}{0.739656in}}%
\pgfpathlineto{\pgfqpoint{4.128884in}{0.739656in}}%
\pgfpathlineto{\pgfqpoint{4.128588in}{0.739656in}}%
\pgfpathlineto{\pgfqpoint{4.128292in}{0.739656in}}%
\pgfpathlineto{\pgfqpoint{4.127996in}{0.739656in}}%
\pgfpathlineto{\pgfqpoint{4.127700in}{0.739656in}}%
\pgfpathlineto{\pgfqpoint{4.127404in}{0.739656in}}%
\pgfpathlineto{\pgfqpoint{4.127108in}{0.739656in}}%
\pgfpathlineto{\pgfqpoint{4.126812in}{0.739656in}}%
\pgfpathlineto{\pgfqpoint{4.126516in}{0.739656in}}%
\pgfpathlineto{\pgfqpoint{4.126220in}{0.739656in}}%
\pgfpathlineto{\pgfqpoint{4.125924in}{0.739656in}}%
\pgfpathlineto{\pgfqpoint{4.125628in}{0.739656in}}%
\pgfpathlineto{\pgfqpoint{4.125332in}{0.739656in}}%
\pgfpathlineto{\pgfqpoint{4.125036in}{0.739656in}}%
\pgfpathlineto{\pgfqpoint{4.124740in}{0.739656in}}%
\pgfpathlineto{\pgfqpoint{4.124444in}{0.739656in}}%
\pgfpathlineto{\pgfqpoint{4.124148in}{0.739656in}}%
\pgfpathlineto{\pgfqpoint{4.123852in}{0.739656in}}%
\pgfpathlineto{\pgfqpoint{4.123556in}{0.739656in}}%
\pgfpathlineto{\pgfqpoint{4.123260in}{0.739656in}}%
\pgfpathlineto{\pgfqpoint{4.122964in}{0.739656in}}%
\pgfpathlineto{\pgfqpoint{4.122668in}{0.739656in}}%
\pgfpathlineto{\pgfqpoint{4.122372in}{0.739656in}}%
\pgfpathlineto{\pgfqpoint{4.122076in}{0.739656in}}%
\pgfpathlineto{\pgfqpoint{4.121780in}{0.739656in}}%
\pgfpathlineto{\pgfqpoint{4.121484in}{0.739656in}}%
\pgfpathlineto{\pgfqpoint{4.121188in}{0.739656in}}%
\pgfpathlineto{\pgfqpoint{4.120892in}{0.739656in}}%
\pgfpathlineto{\pgfqpoint{4.120596in}{0.739656in}}%
\pgfpathlineto{\pgfqpoint{4.120300in}{0.739656in}}%
\pgfpathlineto{\pgfqpoint{4.120004in}{0.739656in}}%
\pgfpathlineto{\pgfqpoint{4.119708in}{0.739656in}}%
\pgfpathlineto{\pgfqpoint{4.119412in}{0.739656in}}%
\pgfpathlineto{\pgfqpoint{4.119116in}{0.739656in}}%
\pgfpathlineto{\pgfqpoint{4.118820in}{0.739656in}}%
\pgfpathlineto{\pgfqpoint{4.118524in}{0.739656in}}%
\pgfpathlineto{\pgfqpoint{4.118228in}{0.739656in}}%
\pgfpathlineto{\pgfqpoint{4.117932in}{0.739656in}}%
\pgfpathlineto{\pgfqpoint{4.117636in}{0.739656in}}%
\pgfpathlineto{\pgfqpoint{4.117340in}{0.739656in}}%
\pgfpathlineto{\pgfqpoint{4.117044in}{0.739656in}}%
\pgfpathlineto{\pgfqpoint{4.116748in}{0.739656in}}%
\pgfpathlineto{\pgfqpoint{4.116452in}{0.739656in}}%
\pgfpathlineto{\pgfqpoint{4.116156in}{0.739656in}}%
\pgfpathlineto{\pgfqpoint{4.115860in}{0.739656in}}%
\pgfpathlineto{\pgfqpoint{4.115564in}{0.739656in}}%
\pgfpathlineto{\pgfqpoint{4.115268in}{0.739656in}}%
\pgfpathlineto{\pgfqpoint{4.114972in}{0.739656in}}%
\pgfpathlineto{\pgfqpoint{4.114676in}{0.739656in}}%
\pgfpathlineto{\pgfqpoint{4.114380in}{0.739656in}}%
\pgfpathlineto{\pgfqpoint{4.114084in}{0.739656in}}%
\pgfpathlineto{\pgfqpoint{4.113788in}{0.739656in}}%
\pgfpathlineto{\pgfqpoint{4.113492in}{0.739656in}}%
\pgfpathlineto{\pgfqpoint{4.113196in}{0.739656in}}%
\pgfpathlineto{\pgfqpoint{4.112900in}{0.739656in}}%
\pgfpathlineto{\pgfqpoint{4.112604in}{0.739656in}}%
\pgfpathlineto{\pgfqpoint{4.112308in}{0.739656in}}%
\pgfpathlineto{\pgfqpoint{4.112012in}{0.739656in}}%
\pgfpathlineto{\pgfqpoint{4.111716in}{0.739656in}}%
\pgfpathlineto{\pgfqpoint{4.111420in}{0.739656in}}%
\pgfpathlineto{\pgfqpoint{4.111124in}{0.739656in}}%
\pgfpathlineto{\pgfqpoint{4.110828in}{0.739656in}}%
\pgfpathlineto{\pgfqpoint{4.110532in}{0.739656in}}%
\pgfpathlineto{\pgfqpoint{4.110236in}{0.739656in}}%
\pgfpathlineto{\pgfqpoint{4.109940in}{0.739656in}}%
\pgfpathlineto{\pgfqpoint{4.109644in}{0.739656in}}%
\pgfpathlineto{\pgfqpoint{4.109348in}{0.739656in}}%
\pgfpathlineto{\pgfqpoint{4.109052in}{0.739656in}}%
\pgfpathlineto{\pgfqpoint{4.108756in}{0.739656in}}%
\pgfpathlineto{\pgfqpoint{4.108460in}{0.739656in}}%
\pgfpathlineto{\pgfqpoint{4.108164in}{0.739656in}}%
\pgfpathlineto{\pgfqpoint{4.107868in}{0.739656in}}%
\pgfpathlineto{\pgfqpoint{4.107572in}{0.739656in}}%
\pgfpathlineto{\pgfqpoint{4.107276in}{0.739656in}}%
\pgfpathlineto{\pgfqpoint{4.106980in}{0.739656in}}%
\pgfpathlineto{\pgfqpoint{4.106684in}{0.739656in}}%
\pgfpathlineto{\pgfqpoint{4.106388in}{0.739656in}}%
\pgfpathlineto{\pgfqpoint{4.106092in}{0.739656in}}%
\pgfpathlineto{\pgfqpoint{4.105796in}{0.739656in}}%
\pgfpathlineto{\pgfqpoint{4.105500in}{0.739656in}}%
\pgfpathlineto{\pgfqpoint{4.105204in}{0.739656in}}%
\pgfpathlineto{\pgfqpoint{4.104908in}{0.739656in}}%
\pgfpathlineto{\pgfqpoint{4.104612in}{0.739656in}}%
\pgfpathlineto{\pgfqpoint{4.104316in}{0.739656in}}%
\pgfpathlineto{\pgfqpoint{4.104020in}{0.739656in}}%
\pgfpathlineto{\pgfqpoint{4.103724in}{0.739656in}}%
\pgfpathlineto{\pgfqpoint{4.103428in}{0.739656in}}%
\pgfpathlineto{\pgfqpoint{4.103132in}{0.739656in}}%
\pgfpathlineto{\pgfqpoint{4.102836in}{0.739656in}}%
\pgfpathlineto{\pgfqpoint{4.102540in}{0.739656in}}%
\pgfpathlineto{\pgfqpoint{4.102244in}{0.739656in}}%
\pgfpathlineto{\pgfqpoint{4.101948in}{0.739656in}}%
\pgfpathlineto{\pgfqpoint{4.101652in}{0.739656in}}%
\pgfpathlineto{\pgfqpoint{4.101356in}{0.739656in}}%
\pgfpathlineto{\pgfqpoint{4.101060in}{0.739656in}}%
\pgfpathlineto{\pgfqpoint{4.100764in}{0.739656in}}%
\pgfpathlineto{\pgfqpoint{4.100468in}{0.739656in}}%
\pgfpathlineto{\pgfqpoint{4.100172in}{0.739656in}}%
\pgfpathlineto{\pgfqpoint{4.099876in}{0.739656in}}%
\pgfpathlineto{\pgfqpoint{4.099580in}{0.739656in}}%
\pgfpathlineto{\pgfqpoint{4.099284in}{0.739656in}}%
\pgfpathlineto{\pgfqpoint{4.098988in}{0.739656in}}%
\pgfpathlineto{\pgfqpoint{4.098692in}{0.739656in}}%
\pgfpathlineto{\pgfqpoint{4.098396in}{0.739656in}}%
\pgfpathlineto{\pgfqpoint{4.098100in}{0.739656in}}%
\pgfpathlineto{\pgfqpoint{4.097804in}{0.739656in}}%
\pgfpathlineto{\pgfqpoint{4.097508in}{0.739656in}}%
\pgfpathlineto{\pgfqpoint{4.097212in}{0.739656in}}%
\pgfpathlineto{\pgfqpoint{4.096916in}{0.739656in}}%
\pgfpathlineto{\pgfqpoint{4.096620in}{0.739656in}}%
\pgfpathlineto{\pgfqpoint{4.096324in}{0.739656in}}%
\pgfpathlineto{\pgfqpoint{4.096028in}{0.739656in}}%
\pgfpathlineto{\pgfqpoint{4.095732in}{0.739656in}}%
\pgfpathlineto{\pgfqpoint{4.095436in}{0.739656in}}%
\pgfpathlineto{\pgfqpoint{4.095140in}{0.739656in}}%
\pgfpathlineto{\pgfqpoint{4.094844in}{0.739656in}}%
\pgfpathlineto{\pgfqpoint{4.094548in}{0.739656in}}%
\pgfpathlineto{\pgfqpoint{4.094252in}{0.739656in}}%
\pgfpathlineto{\pgfqpoint{4.093956in}{0.739656in}}%
\pgfpathlineto{\pgfqpoint{4.093660in}{0.739656in}}%
\pgfpathlineto{\pgfqpoint{4.093364in}{0.739656in}}%
\pgfpathlineto{\pgfqpoint{4.093068in}{0.739656in}}%
\pgfpathlineto{\pgfqpoint{4.092772in}{0.739656in}}%
\pgfpathlineto{\pgfqpoint{4.092476in}{0.739656in}}%
\pgfpathlineto{\pgfqpoint{4.092180in}{0.739656in}}%
\pgfpathlineto{\pgfqpoint{4.091884in}{0.739656in}}%
\pgfpathlineto{\pgfqpoint{4.091588in}{0.739656in}}%
\pgfpathlineto{\pgfqpoint{4.091292in}{0.739656in}}%
\pgfpathlineto{\pgfqpoint{4.090996in}{0.739656in}}%
\pgfpathlineto{\pgfqpoint{4.090700in}{0.739656in}}%
\pgfpathlineto{\pgfqpoint{4.090404in}{0.739656in}}%
\pgfpathlineto{\pgfqpoint{4.090108in}{0.739656in}}%
\pgfpathlineto{\pgfqpoint{4.089812in}{0.739656in}}%
\pgfpathlineto{\pgfqpoint{4.089516in}{0.739656in}}%
\pgfpathlineto{\pgfqpoint{4.089220in}{0.739656in}}%
\pgfpathlineto{\pgfqpoint{4.088924in}{0.739656in}}%
\pgfpathlineto{\pgfqpoint{4.088628in}{0.739656in}}%
\pgfpathlineto{\pgfqpoint{4.088332in}{0.739656in}}%
\pgfpathlineto{\pgfqpoint{4.088036in}{0.739656in}}%
\pgfpathlineto{\pgfqpoint{4.087740in}{0.739656in}}%
\pgfpathlineto{\pgfqpoint{4.087444in}{0.739656in}}%
\pgfpathlineto{\pgfqpoint{4.087148in}{0.739656in}}%
\pgfpathlineto{\pgfqpoint{4.086852in}{0.739656in}}%
\pgfpathlineto{\pgfqpoint{4.086556in}{0.739656in}}%
\pgfpathlineto{\pgfqpoint{4.086260in}{0.739656in}}%
\pgfpathlineto{\pgfqpoint{4.085963in}{0.739656in}}%
\pgfpathlineto{\pgfqpoint{4.085667in}{0.739656in}}%
\pgfpathlineto{\pgfqpoint{4.085371in}{0.739656in}}%
\pgfpathlineto{\pgfqpoint{4.085075in}{0.739656in}}%
\pgfpathlineto{\pgfqpoint{4.084779in}{0.739656in}}%
\pgfpathlineto{\pgfqpoint{4.084483in}{0.739656in}}%
\pgfpathlineto{\pgfqpoint{4.084187in}{0.739656in}}%
\pgfpathlineto{\pgfqpoint{4.083891in}{0.739656in}}%
\pgfpathlineto{\pgfqpoint{4.083595in}{0.739656in}}%
\pgfpathlineto{\pgfqpoint{4.083299in}{0.739656in}}%
\pgfpathlineto{\pgfqpoint{4.083003in}{0.739656in}}%
\pgfpathlineto{\pgfqpoint{4.082707in}{0.739656in}}%
\pgfpathlineto{\pgfqpoint{4.082411in}{0.739656in}}%
\pgfpathlineto{\pgfqpoint{4.082115in}{0.739656in}}%
\pgfpathlineto{\pgfqpoint{4.081819in}{0.739656in}}%
\pgfpathlineto{\pgfqpoint{4.081523in}{0.739656in}}%
\pgfpathlineto{\pgfqpoint{4.081227in}{0.739656in}}%
\pgfpathlineto{\pgfqpoint{4.080931in}{0.739656in}}%
\pgfpathlineto{\pgfqpoint{4.080635in}{0.739656in}}%
\pgfpathlineto{\pgfqpoint{4.080339in}{0.739656in}}%
\pgfpathlineto{\pgfqpoint{4.080043in}{0.739656in}}%
\pgfpathlineto{\pgfqpoint{4.079747in}{0.739656in}}%
\pgfpathlineto{\pgfqpoint{4.079451in}{0.739656in}}%
\pgfpathlineto{\pgfqpoint{4.079155in}{0.739656in}}%
\pgfpathlineto{\pgfqpoint{4.078859in}{0.739656in}}%
\pgfpathlineto{\pgfqpoint{4.078563in}{0.739656in}}%
\pgfpathlineto{\pgfqpoint{4.078267in}{0.739656in}}%
\pgfpathlineto{\pgfqpoint{4.077971in}{0.739656in}}%
\pgfpathlineto{\pgfqpoint{4.077675in}{0.739656in}}%
\pgfpathlineto{\pgfqpoint{4.077379in}{0.739656in}}%
\pgfpathlineto{\pgfqpoint{4.077083in}{0.739656in}}%
\pgfpathlineto{\pgfqpoint{4.076787in}{0.739656in}}%
\pgfpathlineto{\pgfqpoint{4.076491in}{0.739656in}}%
\pgfpathlineto{\pgfqpoint{4.076195in}{0.739656in}}%
\pgfpathlineto{\pgfqpoint{4.075899in}{0.739656in}}%
\pgfpathlineto{\pgfqpoint{4.075603in}{0.739656in}}%
\pgfpathlineto{\pgfqpoint{4.075307in}{0.739656in}}%
\pgfpathlineto{\pgfqpoint{4.075011in}{0.739656in}}%
\pgfpathlineto{\pgfqpoint{4.074715in}{0.739656in}}%
\pgfpathlineto{\pgfqpoint{4.074419in}{0.739656in}}%
\pgfpathlineto{\pgfqpoint{4.074123in}{0.739656in}}%
\pgfpathlineto{\pgfqpoint{4.073827in}{0.739656in}}%
\pgfpathlineto{\pgfqpoint{4.073531in}{0.739656in}}%
\pgfpathlineto{\pgfqpoint{4.073235in}{0.739656in}}%
\pgfpathlineto{\pgfqpoint{4.072939in}{0.739656in}}%
\pgfpathlineto{\pgfqpoint{4.072643in}{0.739656in}}%
\pgfpathlineto{\pgfqpoint{4.072347in}{0.739656in}}%
\pgfpathlineto{\pgfqpoint{4.072051in}{0.739656in}}%
\pgfpathlineto{\pgfqpoint{4.071755in}{0.739656in}}%
\pgfpathlineto{\pgfqpoint{4.071459in}{0.739656in}}%
\pgfpathlineto{\pgfqpoint{4.071163in}{0.739656in}}%
\pgfpathlineto{\pgfqpoint{4.070867in}{0.739656in}}%
\pgfpathlineto{\pgfqpoint{4.070571in}{0.739656in}}%
\pgfpathlineto{\pgfqpoint{4.070275in}{0.739656in}}%
\pgfpathlineto{\pgfqpoint{4.069979in}{0.739656in}}%
\pgfpathlineto{\pgfqpoint{4.069683in}{0.739656in}}%
\pgfpathlineto{\pgfqpoint{4.069387in}{0.739656in}}%
\pgfpathlineto{\pgfqpoint{4.069091in}{0.739656in}}%
\pgfpathlineto{\pgfqpoint{4.068795in}{0.739656in}}%
\pgfpathlineto{\pgfqpoint{4.068499in}{0.739656in}}%
\pgfpathlineto{\pgfqpoint{4.068203in}{0.739656in}}%
\pgfpathlineto{\pgfqpoint{4.067907in}{0.739656in}}%
\pgfpathlineto{\pgfqpoint{4.067611in}{0.739656in}}%
\pgfpathlineto{\pgfqpoint{4.067315in}{0.739656in}}%
\pgfpathlineto{\pgfqpoint{4.067019in}{0.739656in}}%
\pgfpathlineto{\pgfqpoint{4.066723in}{0.739656in}}%
\pgfpathlineto{\pgfqpoint{4.066427in}{0.739656in}}%
\pgfpathlineto{\pgfqpoint{4.066131in}{0.739656in}}%
\pgfpathlineto{\pgfqpoint{4.065835in}{0.739656in}}%
\pgfpathlineto{\pgfqpoint{4.065539in}{0.739656in}}%
\pgfpathlineto{\pgfqpoint{4.065243in}{0.739656in}}%
\pgfpathlineto{\pgfqpoint{4.064947in}{0.739656in}}%
\pgfpathlineto{\pgfqpoint{4.064651in}{0.739656in}}%
\pgfpathlineto{\pgfqpoint{4.064355in}{0.739656in}}%
\pgfpathlineto{\pgfqpoint{4.064059in}{0.739656in}}%
\pgfpathlineto{\pgfqpoint{4.063763in}{0.739656in}}%
\pgfpathlineto{\pgfqpoint{4.063467in}{0.739656in}}%
\pgfpathlineto{\pgfqpoint{4.063171in}{0.739656in}}%
\pgfpathlineto{\pgfqpoint{4.062875in}{0.739656in}}%
\pgfpathlineto{\pgfqpoint{4.062579in}{0.739656in}}%
\pgfpathlineto{\pgfqpoint{4.062283in}{0.739656in}}%
\pgfpathlineto{\pgfqpoint{4.061987in}{0.739656in}}%
\pgfpathlineto{\pgfqpoint{4.061691in}{0.739656in}}%
\pgfpathlineto{\pgfqpoint{4.061395in}{0.739656in}}%
\pgfpathlineto{\pgfqpoint{4.061099in}{0.739656in}}%
\pgfpathlineto{\pgfqpoint{4.060803in}{0.739656in}}%
\pgfpathlineto{\pgfqpoint{4.060507in}{0.739656in}}%
\pgfpathlineto{\pgfqpoint{4.060211in}{0.739656in}}%
\pgfpathlineto{\pgfqpoint{4.059915in}{0.739656in}}%
\pgfpathlineto{\pgfqpoint{4.059619in}{0.739656in}}%
\pgfpathlineto{\pgfqpoint{4.059323in}{0.739656in}}%
\pgfpathlineto{\pgfqpoint{4.059027in}{0.739656in}}%
\pgfpathlineto{\pgfqpoint{4.058731in}{0.739656in}}%
\pgfpathlineto{\pgfqpoint{4.058435in}{0.739656in}}%
\pgfpathlineto{\pgfqpoint{4.058139in}{0.739656in}}%
\pgfpathlineto{\pgfqpoint{4.057843in}{0.739656in}}%
\pgfpathlineto{\pgfqpoint{4.057547in}{0.739656in}}%
\pgfpathlineto{\pgfqpoint{4.057251in}{0.739656in}}%
\pgfpathlineto{\pgfqpoint{4.056955in}{0.739656in}}%
\pgfpathlineto{\pgfqpoint{4.056659in}{0.739656in}}%
\pgfpathlineto{\pgfqpoint{4.056363in}{0.739656in}}%
\pgfpathlineto{\pgfqpoint{4.056067in}{0.739656in}}%
\pgfpathlineto{\pgfqpoint{4.055771in}{0.739656in}}%
\pgfpathlineto{\pgfqpoint{4.055475in}{0.739656in}}%
\pgfpathlineto{\pgfqpoint{4.055179in}{0.739656in}}%
\pgfpathlineto{\pgfqpoint{4.054883in}{0.739656in}}%
\pgfpathlineto{\pgfqpoint{4.054587in}{0.739656in}}%
\pgfpathlineto{\pgfqpoint{4.054291in}{0.739656in}}%
\pgfpathlineto{\pgfqpoint{4.053995in}{0.739656in}}%
\pgfpathlineto{\pgfqpoint{4.053699in}{0.739656in}}%
\pgfpathlineto{\pgfqpoint{4.053403in}{0.739656in}}%
\pgfpathlineto{\pgfqpoint{4.053107in}{0.739656in}}%
\pgfpathlineto{\pgfqpoint{4.052811in}{0.739656in}}%
\pgfpathlineto{\pgfqpoint{4.052515in}{0.739656in}}%
\pgfpathlineto{\pgfqpoint{4.052219in}{0.739656in}}%
\pgfpathlineto{\pgfqpoint{4.051923in}{0.739656in}}%
\pgfpathlineto{\pgfqpoint{4.051627in}{0.739656in}}%
\pgfpathlineto{\pgfqpoint{4.051331in}{0.739656in}}%
\pgfpathlineto{\pgfqpoint{4.051035in}{0.739656in}}%
\pgfpathlineto{\pgfqpoint{4.050739in}{0.739656in}}%
\pgfpathlineto{\pgfqpoint{4.050443in}{0.739656in}}%
\pgfpathlineto{\pgfqpoint{4.050147in}{0.739656in}}%
\pgfpathlineto{\pgfqpoint{4.049851in}{0.739656in}}%
\pgfpathlineto{\pgfqpoint{4.049555in}{0.739656in}}%
\pgfpathlineto{\pgfqpoint{4.049259in}{0.739656in}}%
\pgfpathlineto{\pgfqpoint{4.048963in}{0.739656in}}%
\pgfpathlineto{\pgfqpoint{4.048667in}{0.739656in}}%
\pgfpathlineto{\pgfqpoint{4.048371in}{0.739656in}}%
\pgfpathlineto{\pgfqpoint{4.048075in}{0.739656in}}%
\pgfpathlineto{\pgfqpoint{4.047779in}{0.739656in}}%
\pgfpathlineto{\pgfqpoint{4.047483in}{0.739656in}}%
\pgfpathlineto{\pgfqpoint{4.047187in}{0.739656in}}%
\pgfpathlineto{\pgfqpoint{4.046891in}{0.739656in}}%
\pgfpathlineto{\pgfqpoint{4.046595in}{0.739656in}}%
\pgfpathlineto{\pgfqpoint{4.046299in}{0.739656in}}%
\pgfpathlineto{\pgfqpoint{4.046003in}{0.739656in}}%
\pgfpathlineto{\pgfqpoint{4.045707in}{0.739656in}}%
\pgfpathlineto{\pgfqpoint{4.045411in}{0.739656in}}%
\pgfpathlineto{\pgfqpoint{4.045115in}{0.739656in}}%
\pgfpathlineto{\pgfqpoint{4.044819in}{0.739656in}}%
\pgfpathlineto{\pgfqpoint{4.044523in}{0.739656in}}%
\pgfpathlineto{\pgfqpoint{4.044227in}{0.739656in}}%
\pgfpathlineto{\pgfqpoint{4.043931in}{0.739656in}}%
\pgfpathlineto{\pgfqpoint{4.043635in}{0.739656in}}%
\pgfpathlineto{\pgfqpoint{4.043339in}{0.739656in}}%
\pgfpathlineto{\pgfqpoint{4.043043in}{0.739656in}}%
\pgfpathlineto{\pgfqpoint{4.042747in}{0.739656in}}%
\pgfpathlineto{\pgfqpoint{4.042451in}{0.739656in}}%
\pgfpathlineto{\pgfqpoint{4.042155in}{0.739656in}}%
\pgfpathlineto{\pgfqpoint{4.041859in}{0.739656in}}%
\pgfpathlineto{\pgfqpoint{4.041563in}{0.739656in}}%
\pgfpathlineto{\pgfqpoint{4.041267in}{0.739656in}}%
\pgfpathlineto{\pgfqpoint{4.040971in}{0.739656in}}%
\pgfpathlineto{\pgfqpoint{4.040675in}{0.739656in}}%
\pgfpathlineto{\pgfqpoint{4.040379in}{0.739656in}}%
\pgfpathlineto{\pgfqpoint{4.040083in}{0.739656in}}%
\pgfpathlineto{\pgfqpoint{4.039787in}{0.739656in}}%
\pgfpathlineto{\pgfqpoint{4.039491in}{0.739656in}}%
\pgfpathlineto{\pgfqpoint{4.039195in}{0.739656in}}%
\pgfpathlineto{\pgfqpoint{4.038899in}{0.739656in}}%
\pgfpathlineto{\pgfqpoint{4.038603in}{0.739656in}}%
\pgfpathlineto{\pgfqpoint{4.038307in}{0.739656in}}%
\pgfpathlineto{\pgfqpoint{4.038011in}{0.739656in}}%
\pgfpathlineto{\pgfqpoint{4.037715in}{0.739656in}}%
\pgfpathlineto{\pgfqpoint{4.037419in}{0.739656in}}%
\pgfpathlineto{\pgfqpoint{4.037123in}{0.739656in}}%
\pgfpathlineto{\pgfqpoint{4.036827in}{0.739656in}}%
\pgfpathlineto{\pgfqpoint{4.036531in}{0.739656in}}%
\pgfpathlineto{\pgfqpoint{4.036235in}{0.739656in}}%
\pgfpathlineto{\pgfqpoint{4.035939in}{0.739656in}}%
\pgfpathlineto{\pgfqpoint{4.035643in}{0.739656in}}%
\pgfpathlineto{\pgfqpoint{4.035347in}{0.739656in}}%
\pgfpathlineto{\pgfqpoint{4.035051in}{0.739656in}}%
\pgfpathlineto{\pgfqpoint{4.034755in}{0.739656in}}%
\pgfpathlineto{\pgfqpoint{4.034459in}{0.739656in}}%
\pgfpathlineto{\pgfqpoint{4.034163in}{0.739656in}}%
\pgfpathlineto{\pgfqpoint{4.033867in}{0.739656in}}%
\pgfpathlineto{\pgfqpoint{4.033571in}{0.739656in}}%
\pgfpathlineto{\pgfqpoint{4.033275in}{0.739656in}}%
\pgfpathlineto{\pgfqpoint{4.032979in}{0.739656in}}%
\pgfpathlineto{\pgfqpoint{4.032683in}{0.739656in}}%
\pgfpathlineto{\pgfqpoint{4.032387in}{0.739656in}}%
\pgfpathlineto{\pgfqpoint{4.032091in}{0.739656in}}%
\pgfpathlineto{\pgfqpoint{4.031795in}{0.739656in}}%
\pgfpathlineto{\pgfqpoint{4.031499in}{0.739656in}}%
\pgfpathlineto{\pgfqpoint{4.031203in}{0.739656in}}%
\pgfpathlineto{\pgfqpoint{4.030907in}{0.739656in}}%
\pgfpathlineto{\pgfqpoint{4.030611in}{0.739656in}}%
\pgfpathlineto{\pgfqpoint{4.030315in}{0.739656in}}%
\pgfpathlineto{\pgfqpoint{4.030019in}{0.739656in}}%
\pgfpathlineto{\pgfqpoint{4.029723in}{0.739656in}}%
\pgfpathlineto{\pgfqpoint{4.029427in}{0.739656in}}%
\pgfpathlineto{\pgfqpoint{4.029131in}{0.739656in}}%
\pgfpathlineto{\pgfqpoint{4.028835in}{0.739656in}}%
\pgfpathlineto{\pgfqpoint{4.028539in}{0.739656in}}%
\pgfpathlineto{\pgfqpoint{4.028243in}{0.739656in}}%
\pgfpathlineto{\pgfqpoint{4.027947in}{0.739656in}}%
\pgfpathlineto{\pgfqpoint{4.027651in}{0.739656in}}%
\pgfpathlineto{\pgfqpoint{4.027355in}{0.739656in}}%
\pgfpathlineto{\pgfqpoint{4.027059in}{0.739656in}}%
\pgfpathlineto{\pgfqpoint{4.026763in}{0.739656in}}%
\pgfpathlineto{\pgfqpoint{4.026467in}{0.739656in}}%
\pgfpathlineto{\pgfqpoint{4.026171in}{0.739656in}}%
\pgfpathlineto{\pgfqpoint{4.025875in}{0.739656in}}%
\pgfpathlineto{\pgfqpoint{4.025579in}{0.739656in}}%
\pgfpathlineto{\pgfqpoint{4.025283in}{0.739656in}}%
\pgfpathlineto{\pgfqpoint{4.024987in}{0.739656in}}%
\pgfpathlineto{\pgfqpoint{4.024691in}{0.739656in}}%
\pgfpathlineto{\pgfqpoint{4.024395in}{0.739656in}}%
\pgfpathlineto{\pgfqpoint{4.024099in}{0.739656in}}%
\pgfpathlineto{\pgfqpoint{4.023803in}{0.739656in}}%
\pgfpathlineto{\pgfqpoint{4.023507in}{0.739656in}}%
\pgfpathlineto{\pgfqpoint{4.023211in}{0.739656in}}%
\pgfpathlineto{\pgfqpoint{4.022915in}{0.739656in}}%
\pgfpathlineto{\pgfqpoint{4.022619in}{0.739656in}}%
\pgfpathlineto{\pgfqpoint{4.022323in}{0.739656in}}%
\pgfpathlineto{\pgfqpoint{4.022027in}{0.739656in}}%
\pgfpathlineto{\pgfqpoint{4.021731in}{0.739656in}}%
\pgfpathlineto{\pgfqpoint{4.021435in}{0.739656in}}%
\pgfpathlineto{\pgfqpoint{4.021139in}{0.739656in}}%
\pgfpathlineto{\pgfqpoint{4.020843in}{0.739656in}}%
\pgfpathlineto{\pgfqpoint{4.020547in}{0.739656in}}%
\pgfpathlineto{\pgfqpoint{4.020251in}{0.739656in}}%
\pgfpathlineto{\pgfqpoint{4.019955in}{0.739656in}}%
\pgfpathlineto{\pgfqpoint{4.019659in}{0.739656in}}%
\pgfpathlineto{\pgfqpoint{4.019363in}{0.739656in}}%
\pgfpathlineto{\pgfqpoint{4.019067in}{0.739656in}}%
\pgfpathlineto{\pgfqpoint{4.018771in}{0.739656in}}%
\pgfpathlineto{\pgfqpoint{4.018474in}{0.739656in}}%
\pgfpathlineto{\pgfqpoint{4.018178in}{0.739656in}}%
\pgfpathlineto{\pgfqpoint{4.017882in}{0.739656in}}%
\pgfpathlineto{\pgfqpoint{4.017586in}{0.739656in}}%
\pgfpathlineto{\pgfqpoint{4.017290in}{0.739656in}}%
\pgfpathlineto{\pgfqpoint{4.016994in}{0.739656in}}%
\pgfpathlineto{\pgfqpoint{4.016698in}{0.739656in}}%
\pgfpathlineto{\pgfqpoint{4.016402in}{0.739656in}}%
\pgfpathlineto{\pgfqpoint{4.016106in}{0.739656in}}%
\pgfpathlineto{\pgfqpoint{4.015810in}{0.739656in}}%
\pgfpathlineto{\pgfqpoint{4.015514in}{0.739656in}}%
\pgfpathlineto{\pgfqpoint{4.015218in}{0.739656in}}%
\pgfpathlineto{\pgfqpoint{4.014922in}{0.739656in}}%
\pgfpathlineto{\pgfqpoint{4.014626in}{0.739656in}}%
\pgfpathlineto{\pgfqpoint{4.014330in}{0.739656in}}%
\pgfpathlineto{\pgfqpoint{4.014034in}{0.739656in}}%
\pgfpathlineto{\pgfqpoint{4.013738in}{0.739656in}}%
\pgfpathlineto{\pgfqpoint{4.013442in}{0.739656in}}%
\pgfpathlineto{\pgfqpoint{4.013146in}{0.739656in}}%
\pgfpathlineto{\pgfqpoint{4.012850in}{0.739656in}}%
\pgfpathlineto{\pgfqpoint{4.012554in}{0.739656in}}%
\pgfpathlineto{\pgfqpoint{4.012258in}{0.739656in}}%
\pgfpathlineto{\pgfqpoint{4.011962in}{0.739656in}}%
\pgfpathlineto{\pgfqpoint{4.011666in}{0.739656in}}%
\pgfpathlineto{\pgfqpoint{4.011370in}{0.739656in}}%
\pgfpathlineto{\pgfqpoint{4.011074in}{0.739656in}}%
\pgfpathlineto{\pgfqpoint{4.010778in}{0.739656in}}%
\pgfpathlineto{\pgfqpoint{4.010482in}{0.739656in}}%
\pgfpathlineto{\pgfqpoint{4.010186in}{0.739656in}}%
\pgfpathlineto{\pgfqpoint{4.009890in}{0.739656in}}%
\pgfpathlineto{\pgfqpoint{4.009594in}{0.739656in}}%
\pgfpathlineto{\pgfqpoint{4.009298in}{0.739656in}}%
\pgfpathlineto{\pgfqpoint{4.009002in}{0.739656in}}%
\pgfpathlineto{\pgfqpoint{4.008706in}{0.739656in}}%
\pgfpathlineto{\pgfqpoint{4.008410in}{0.739656in}}%
\pgfpathlineto{\pgfqpoint{4.008114in}{0.739656in}}%
\pgfpathlineto{\pgfqpoint{4.007818in}{0.739656in}}%
\pgfpathlineto{\pgfqpoint{4.007522in}{0.739656in}}%
\pgfpathlineto{\pgfqpoint{4.007226in}{0.739656in}}%
\pgfpathlineto{\pgfqpoint{4.006930in}{0.739656in}}%
\pgfpathlineto{\pgfqpoint{4.006634in}{0.739656in}}%
\pgfpathlineto{\pgfqpoint{4.006338in}{0.739656in}}%
\pgfpathlineto{\pgfqpoint{4.006042in}{0.739656in}}%
\pgfpathlineto{\pgfqpoint{4.005746in}{0.739656in}}%
\pgfpathlineto{\pgfqpoint{4.005450in}{0.739656in}}%
\pgfpathlineto{\pgfqpoint{4.005154in}{0.739656in}}%
\pgfpathlineto{\pgfqpoint{4.004858in}{0.739656in}}%
\pgfpathlineto{\pgfqpoint{4.004562in}{0.739656in}}%
\pgfpathlineto{\pgfqpoint{4.004266in}{0.739656in}}%
\pgfpathlineto{\pgfqpoint{4.003970in}{0.739656in}}%
\pgfpathlineto{\pgfqpoint{4.003674in}{0.739656in}}%
\pgfpathlineto{\pgfqpoint{4.003378in}{0.739656in}}%
\pgfpathlineto{\pgfqpoint{4.003082in}{0.739656in}}%
\pgfpathlineto{\pgfqpoint{4.002786in}{0.739656in}}%
\pgfpathlineto{\pgfqpoint{4.002490in}{0.739656in}}%
\pgfpathlineto{\pgfqpoint{4.002194in}{0.739656in}}%
\pgfpathlineto{\pgfqpoint{4.001898in}{0.739656in}}%
\pgfpathlineto{\pgfqpoint{4.001602in}{0.739656in}}%
\pgfpathlineto{\pgfqpoint{4.001306in}{0.739656in}}%
\pgfpathlineto{\pgfqpoint{4.001010in}{0.739656in}}%
\pgfpathlineto{\pgfqpoint{4.000714in}{0.739656in}}%
\pgfpathlineto{\pgfqpoint{4.000418in}{0.739656in}}%
\pgfpathlineto{\pgfqpoint{4.000122in}{0.739656in}}%
\pgfpathlineto{\pgfqpoint{3.999826in}{0.739656in}}%
\pgfpathlineto{\pgfqpoint{3.999530in}{0.739656in}}%
\pgfpathlineto{\pgfqpoint{3.999234in}{0.739656in}}%
\pgfpathlineto{\pgfqpoint{3.998938in}{0.739656in}}%
\pgfpathlineto{\pgfqpoint{3.998642in}{0.739656in}}%
\pgfpathlineto{\pgfqpoint{3.998346in}{0.739656in}}%
\pgfpathlineto{\pgfqpoint{3.998050in}{0.739656in}}%
\pgfpathlineto{\pgfqpoint{3.997754in}{0.739656in}}%
\pgfpathlineto{\pgfqpoint{3.997458in}{0.739656in}}%
\pgfpathlineto{\pgfqpoint{3.997162in}{0.739656in}}%
\pgfpathlineto{\pgfqpoint{3.996866in}{0.739656in}}%
\pgfpathlineto{\pgfqpoint{3.996570in}{0.739656in}}%
\pgfpathlineto{\pgfqpoint{3.996274in}{0.739656in}}%
\pgfpathlineto{\pgfqpoint{3.995978in}{0.739656in}}%
\pgfpathlineto{\pgfqpoint{3.995682in}{0.739656in}}%
\pgfpathlineto{\pgfqpoint{3.995386in}{0.739656in}}%
\pgfpathlineto{\pgfqpoint{3.995090in}{0.739656in}}%
\pgfpathlineto{\pgfqpoint{3.994794in}{0.739656in}}%
\pgfpathlineto{\pgfqpoint{3.994498in}{0.739656in}}%
\pgfpathlineto{\pgfqpoint{3.994202in}{0.739656in}}%
\pgfpathlineto{\pgfqpoint{3.993906in}{0.739656in}}%
\pgfpathlineto{\pgfqpoint{3.993610in}{0.739656in}}%
\pgfpathlineto{\pgfqpoint{3.993314in}{0.739656in}}%
\pgfpathlineto{\pgfqpoint{3.993018in}{0.739656in}}%
\pgfpathlineto{\pgfqpoint{3.992722in}{0.739656in}}%
\pgfpathlineto{\pgfqpoint{3.992426in}{0.739656in}}%
\pgfpathlineto{\pgfqpoint{3.992130in}{0.739656in}}%
\pgfpathlineto{\pgfqpoint{3.991834in}{0.739656in}}%
\pgfpathlineto{\pgfqpoint{3.991538in}{0.739656in}}%
\pgfpathlineto{\pgfqpoint{3.991242in}{0.739656in}}%
\pgfpathlineto{\pgfqpoint{3.990946in}{0.739656in}}%
\pgfpathlineto{\pgfqpoint{3.990650in}{0.739656in}}%
\pgfpathlineto{\pgfqpoint{3.990354in}{0.739656in}}%
\pgfpathlineto{\pgfqpoint{3.990058in}{0.739656in}}%
\pgfpathlineto{\pgfqpoint{3.989762in}{0.739656in}}%
\pgfpathlineto{\pgfqpoint{3.989466in}{0.739656in}}%
\pgfpathlineto{\pgfqpoint{3.989170in}{0.739656in}}%
\pgfpathlineto{\pgfqpoint{3.988874in}{0.739656in}}%
\pgfpathlineto{\pgfqpoint{3.988578in}{0.739656in}}%
\pgfpathlineto{\pgfqpoint{3.988282in}{0.739656in}}%
\pgfpathlineto{\pgfqpoint{3.987986in}{0.739656in}}%
\pgfpathlineto{\pgfqpoint{3.987690in}{0.739656in}}%
\pgfpathlineto{\pgfqpoint{3.987394in}{0.739656in}}%
\pgfpathlineto{\pgfqpoint{3.987098in}{0.739656in}}%
\pgfpathlineto{\pgfqpoint{3.986802in}{0.739656in}}%
\pgfpathlineto{\pgfqpoint{3.986506in}{0.739656in}}%
\pgfpathlineto{\pgfqpoint{3.986210in}{0.739656in}}%
\pgfpathlineto{\pgfqpoint{3.985914in}{0.739656in}}%
\pgfpathlineto{\pgfqpoint{3.985618in}{0.739656in}}%
\pgfpathlineto{\pgfqpoint{3.985322in}{0.739656in}}%
\pgfpathlineto{\pgfqpoint{3.985026in}{0.739656in}}%
\pgfpathlineto{\pgfqpoint{3.984730in}{0.739656in}}%
\pgfpathlineto{\pgfqpoint{3.984434in}{0.739656in}}%
\pgfpathlineto{\pgfqpoint{3.984138in}{0.739656in}}%
\pgfpathlineto{\pgfqpoint{3.983842in}{0.739656in}}%
\pgfpathlineto{\pgfqpoint{3.983546in}{0.739656in}}%
\pgfpathlineto{\pgfqpoint{3.983250in}{0.739656in}}%
\pgfpathlineto{\pgfqpoint{3.982954in}{0.739656in}}%
\pgfpathlineto{\pgfqpoint{3.982658in}{0.739656in}}%
\pgfpathlineto{\pgfqpoint{3.982362in}{0.739656in}}%
\pgfpathlineto{\pgfqpoint{3.982066in}{0.739656in}}%
\pgfpathlineto{\pgfqpoint{3.981770in}{0.739656in}}%
\pgfpathlineto{\pgfqpoint{3.981474in}{0.739656in}}%
\pgfpathlineto{\pgfqpoint{3.981178in}{0.739656in}}%
\pgfpathlineto{\pgfqpoint{3.980882in}{0.739656in}}%
\pgfpathlineto{\pgfqpoint{3.980586in}{0.739656in}}%
\pgfpathlineto{\pgfqpoint{3.980290in}{0.739656in}}%
\pgfpathlineto{\pgfqpoint{3.979994in}{0.739656in}}%
\pgfpathlineto{\pgfqpoint{3.979698in}{0.739656in}}%
\pgfpathlineto{\pgfqpoint{3.979402in}{0.739656in}}%
\pgfpathlineto{\pgfqpoint{3.979106in}{0.739656in}}%
\pgfpathlineto{\pgfqpoint{3.978810in}{0.739656in}}%
\pgfpathlineto{\pgfqpoint{3.978514in}{0.739656in}}%
\pgfpathlineto{\pgfqpoint{3.978218in}{0.739656in}}%
\pgfpathlineto{\pgfqpoint{3.977922in}{0.739656in}}%
\pgfpathlineto{\pgfqpoint{3.977626in}{0.739656in}}%
\pgfpathlineto{\pgfqpoint{3.977330in}{0.739656in}}%
\pgfpathlineto{\pgfqpoint{3.977034in}{0.739656in}}%
\pgfpathlineto{\pgfqpoint{3.976738in}{0.739656in}}%
\pgfpathlineto{\pgfqpoint{3.976442in}{0.739656in}}%
\pgfpathlineto{\pgfqpoint{3.976146in}{0.739656in}}%
\pgfpathlineto{\pgfqpoint{3.975850in}{0.739656in}}%
\pgfpathlineto{\pgfqpoint{3.975554in}{0.739656in}}%
\pgfpathlineto{\pgfqpoint{3.975258in}{0.739656in}}%
\pgfpathlineto{\pgfqpoint{3.974962in}{0.739656in}}%
\pgfpathlineto{\pgfqpoint{3.974666in}{0.739656in}}%
\pgfpathlineto{\pgfqpoint{3.974370in}{0.739656in}}%
\pgfpathlineto{\pgfqpoint{3.974074in}{0.739656in}}%
\pgfpathlineto{\pgfqpoint{3.973778in}{0.739656in}}%
\pgfpathlineto{\pgfqpoint{3.973482in}{0.739656in}}%
\pgfpathlineto{\pgfqpoint{3.973186in}{0.739656in}}%
\pgfpathlineto{\pgfqpoint{3.972890in}{0.739656in}}%
\pgfpathlineto{\pgfqpoint{3.972594in}{0.739656in}}%
\pgfpathlineto{\pgfqpoint{3.972298in}{0.739656in}}%
\pgfpathlineto{\pgfqpoint{3.972002in}{0.739656in}}%
\pgfpathlineto{\pgfqpoint{3.971706in}{0.739656in}}%
\pgfpathlineto{\pgfqpoint{3.971410in}{0.739656in}}%
\pgfpathlineto{\pgfqpoint{3.971114in}{0.739656in}}%
\pgfpathlineto{\pgfqpoint{3.970818in}{0.739656in}}%
\pgfpathlineto{\pgfqpoint{3.970522in}{0.739656in}}%
\pgfpathlineto{\pgfqpoint{3.970226in}{0.739656in}}%
\pgfpathlineto{\pgfqpoint{3.969930in}{0.739656in}}%
\pgfpathlineto{\pgfqpoint{3.969634in}{0.739656in}}%
\pgfpathlineto{\pgfqpoint{3.969338in}{0.739656in}}%
\pgfpathlineto{\pgfqpoint{3.969042in}{0.739656in}}%
\pgfpathlineto{\pgfqpoint{3.968746in}{0.739656in}}%
\pgfpathlineto{\pgfqpoint{3.968450in}{0.739656in}}%
\pgfpathlineto{\pgfqpoint{3.968154in}{0.739656in}}%
\pgfpathlineto{\pgfqpoint{3.967858in}{0.739656in}}%
\pgfpathlineto{\pgfqpoint{3.967562in}{0.739656in}}%
\pgfpathlineto{\pgfqpoint{3.967266in}{0.739656in}}%
\pgfpathlineto{\pgfqpoint{3.966970in}{0.739656in}}%
\pgfpathlineto{\pgfqpoint{3.966674in}{0.739656in}}%
\pgfpathlineto{\pgfqpoint{3.966378in}{0.739656in}}%
\pgfpathlineto{\pgfqpoint{3.966082in}{0.739656in}}%
\pgfpathlineto{\pgfqpoint{3.965786in}{0.739656in}}%
\pgfpathlineto{\pgfqpoint{3.965490in}{0.739656in}}%
\pgfpathlineto{\pgfqpoint{3.965194in}{0.739656in}}%
\pgfpathlineto{\pgfqpoint{3.964898in}{0.739656in}}%
\pgfpathlineto{\pgfqpoint{3.964602in}{0.739656in}}%
\pgfpathlineto{\pgfqpoint{3.964306in}{0.739656in}}%
\pgfpathlineto{\pgfqpoint{3.964010in}{0.739656in}}%
\pgfpathlineto{\pgfqpoint{3.963714in}{0.739656in}}%
\pgfpathlineto{\pgfqpoint{3.963418in}{0.739656in}}%
\pgfpathlineto{\pgfqpoint{3.963122in}{0.739656in}}%
\pgfpathlineto{\pgfqpoint{3.962826in}{0.739656in}}%
\pgfpathlineto{\pgfqpoint{3.962530in}{0.739656in}}%
\pgfpathlineto{\pgfqpoint{3.962234in}{0.739656in}}%
\pgfpathlineto{\pgfqpoint{3.961938in}{0.739656in}}%
\pgfpathlineto{\pgfqpoint{3.961642in}{0.739656in}}%
\pgfpathlineto{\pgfqpoint{3.961346in}{0.739656in}}%
\pgfpathlineto{\pgfqpoint{3.961050in}{0.739656in}}%
\pgfpathlineto{\pgfqpoint{3.960754in}{0.739656in}}%
\pgfpathlineto{\pgfqpoint{3.960458in}{0.739656in}}%
\pgfpathlineto{\pgfqpoint{3.960162in}{0.739656in}}%
\pgfpathlineto{\pgfqpoint{3.959866in}{0.739656in}}%
\pgfpathlineto{\pgfqpoint{3.959570in}{0.739656in}}%
\pgfpathlineto{\pgfqpoint{3.959274in}{0.739656in}}%
\pgfpathlineto{\pgfqpoint{3.958978in}{0.739656in}}%
\pgfpathlineto{\pgfqpoint{3.958682in}{0.739656in}}%
\pgfpathlineto{\pgfqpoint{3.958386in}{0.739656in}}%
\pgfpathlineto{\pgfqpoint{3.958090in}{0.739656in}}%
\pgfpathlineto{\pgfqpoint{3.957794in}{0.739656in}}%
\pgfpathlineto{\pgfqpoint{3.957498in}{0.739656in}}%
\pgfpathlineto{\pgfqpoint{3.957202in}{0.739656in}}%
\pgfpathlineto{\pgfqpoint{3.956906in}{0.739656in}}%
\pgfpathlineto{\pgfqpoint{3.956610in}{0.739656in}}%
\pgfpathlineto{\pgfqpoint{3.956314in}{0.739656in}}%
\pgfpathlineto{\pgfqpoint{3.956018in}{0.739656in}}%
\pgfpathlineto{\pgfqpoint{3.955722in}{0.739656in}}%
\pgfpathlineto{\pgfqpoint{3.955426in}{0.739656in}}%
\pgfpathlineto{\pgfqpoint{3.955130in}{0.739656in}}%
\pgfpathlineto{\pgfqpoint{3.954834in}{0.739656in}}%
\pgfpathlineto{\pgfqpoint{3.954538in}{0.739656in}}%
\pgfpathlineto{\pgfqpoint{3.954242in}{0.739656in}}%
\pgfpathlineto{\pgfqpoint{3.953946in}{0.739656in}}%
\pgfpathlineto{\pgfqpoint{3.953650in}{0.739656in}}%
\pgfpathlineto{\pgfqpoint{3.953354in}{0.739656in}}%
\pgfpathlineto{\pgfqpoint{3.953058in}{0.739656in}}%
\pgfpathlineto{\pgfqpoint{3.952762in}{0.739656in}}%
\pgfpathlineto{\pgfqpoint{3.952466in}{0.739656in}}%
\pgfpathlineto{\pgfqpoint{3.952170in}{0.739656in}}%
\pgfpathlineto{\pgfqpoint{3.951874in}{0.739656in}}%
\pgfpathlineto{\pgfqpoint{3.951578in}{0.739656in}}%
\pgfpathlineto{\pgfqpoint{3.951281in}{0.739656in}}%
\pgfpathlineto{\pgfqpoint{3.950985in}{0.739656in}}%
\pgfpathlineto{\pgfqpoint{3.950689in}{0.739656in}}%
\pgfpathlineto{\pgfqpoint{3.950393in}{0.739656in}}%
\pgfpathlineto{\pgfqpoint{3.950097in}{0.739656in}}%
\pgfpathlineto{\pgfqpoint{3.949801in}{0.739656in}}%
\pgfpathlineto{\pgfqpoint{3.949505in}{0.739656in}}%
\pgfpathlineto{\pgfqpoint{3.949209in}{0.739656in}}%
\pgfpathlineto{\pgfqpoint{3.948913in}{0.739656in}}%
\pgfpathlineto{\pgfqpoint{3.948617in}{0.739656in}}%
\pgfpathlineto{\pgfqpoint{3.948321in}{0.739656in}}%
\pgfpathlineto{\pgfqpoint{3.948025in}{0.739656in}}%
\pgfpathlineto{\pgfqpoint{3.947729in}{0.739656in}}%
\pgfpathlineto{\pgfqpoint{3.947433in}{0.739656in}}%
\pgfpathlineto{\pgfqpoint{3.947137in}{0.739656in}}%
\pgfpathlineto{\pgfqpoint{3.946841in}{0.739656in}}%
\pgfpathlineto{\pgfqpoint{3.946545in}{0.739656in}}%
\pgfpathlineto{\pgfqpoint{3.946249in}{0.739656in}}%
\pgfpathlineto{\pgfqpoint{3.945953in}{0.739656in}}%
\pgfpathlineto{\pgfqpoint{3.945657in}{0.739656in}}%
\pgfpathlineto{\pgfqpoint{3.945361in}{0.739656in}}%
\pgfpathlineto{\pgfqpoint{3.945065in}{0.739656in}}%
\pgfpathlineto{\pgfqpoint{3.944769in}{0.739656in}}%
\pgfpathlineto{\pgfqpoint{3.944473in}{0.739656in}}%
\pgfpathlineto{\pgfqpoint{3.944177in}{0.739656in}}%
\pgfpathlineto{\pgfqpoint{3.943881in}{0.739656in}}%
\pgfpathlineto{\pgfqpoint{3.943585in}{0.739656in}}%
\pgfpathlineto{\pgfqpoint{3.943289in}{0.739656in}}%
\pgfpathlineto{\pgfqpoint{3.942993in}{0.739656in}}%
\pgfpathlineto{\pgfqpoint{3.942697in}{0.739656in}}%
\pgfpathlineto{\pgfqpoint{3.942401in}{0.739656in}}%
\pgfpathlineto{\pgfqpoint{3.942105in}{0.739656in}}%
\pgfpathlineto{\pgfqpoint{3.941809in}{0.739656in}}%
\pgfpathlineto{\pgfqpoint{3.941513in}{0.739656in}}%
\pgfpathlineto{\pgfqpoint{3.941217in}{0.739656in}}%
\pgfpathlineto{\pgfqpoint{3.940921in}{0.739656in}}%
\pgfpathlineto{\pgfqpoint{3.940625in}{0.739656in}}%
\pgfpathlineto{\pgfqpoint{3.940329in}{0.739656in}}%
\pgfpathlineto{\pgfqpoint{3.940033in}{0.739656in}}%
\pgfpathlineto{\pgfqpoint{3.939737in}{0.739656in}}%
\pgfpathlineto{\pgfqpoint{3.939441in}{0.739656in}}%
\pgfpathlineto{\pgfqpoint{3.939145in}{0.739656in}}%
\pgfpathlineto{\pgfqpoint{3.938849in}{0.739656in}}%
\pgfpathlineto{\pgfqpoint{3.938553in}{0.739656in}}%
\pgfpathlineto{\pgfqpoint{3.938257in}{0.739656in}}%
\pgfpathlineto{\pgfqpoint{3.937961in}{0.739656in}}%
\pgfpathlineto{\pgfqpoint{3.937665in}{0.739656in}}%
\pgfpathlineto{\pgfqpoint{3.937369in}{0.739656in}}%
\pgfpathlineto{\pgfqpoint{3.937073in}{0.739656in}}%
\pgfpathlineto{\pgfqpoint{3.936777in}{0.739656in}}%
\pgfpathlineto{\pgfqpoint{3.936481in}{0.739656in}}%
\pgfpathlineto{\pgfqpoint{3.936185in}{0.739656in}}%
\pgfpathlineto{\pgfqpoint{3.935889in}{0.739656in}}%
\pgfpathlineto{\pgfqpoint{3.935593in}{0.739656in}}%
\pgfpathlineto{\pgfqpoint{3.935297in}{0.739656in}}%
\pgfpathlineto{\pgfqpoint{3.935001in}{0.739656in}}%
\pgfpathlineto{\pgfqpoint{3.934705in}{0.739656in}}%
\pgfpathlineto{\pgfqpoint{3.934409in}{0.739656in}}%
\pgfpathlineto{\pgfqpoint{3.934113in}{0.739656in}}%
\pgfpathlineto{\pgfqpoint{3.933817in}{0.739656in}}%
\pgfpathlineto{\pgfqpoint{3.933521in}{0.739656in}}%
\pgfpathlineto{\pgfqpoint{3.933225in}{0.739656in}}%
\pgfpathlineto{\pgfqpoint{3.932929in}{0.739656in}}%
\pgfpathlineto{\pgfqpoint{3.932633in}{0.739656in}}%
\pgfpathlineto{\pgfqpoint{3.932337in}{0.739656in}}%
\pgfpathlineto{\pgfqpoint{3.932041in}{0.739656in}}%
\pgfpathlineto{\pgfqpoint{3.931745in}{0.739656in}}%
\pgfpathlineto{\pgfqpoint{3.931449in}{0.739656in}}%
\pgfpathlineto{\pgfqpoint{3.931153in}{0.739656in}}%
\pgfpathlineto{\pgfqpoint{3.930857in}{0.739656in}}%
\pgfpathlineto{\pgfqpoint{3.930561in}{0.739656in}}%
\pgfpathlineto{\pgfqpoint{3.930265in}{0.739656in}}%
\pgfpathlineto{\pgfqpoint{3.929969in}{0.739656in}}%
\pgfpathlineto{\pgfqpoint{3.929673in}{0.739656in}}%
\pgfpathlineto{\pgfqpoint{3.929377in}{0.739656in}}%
\pgfpathlineto{\pgfqpoint{3.929081in}{0.739656in}}%
\pgfpathlineto{\pgfqpoint{3.928785in}{0.739656in}}%
\pgfpathlineto{\pgfqpoint{3.928489in}{0.739656in}}%
\pgfpathlineto{\pgfqpoint{3.928193in}{0.739656in}}%
\pgfpathlineto{\pgfqpoint{3.927897in}{0.739656in}}%
\pgfpathlineto{\pgfqpoint{3.927601in}{0.739656in}}%
\pgfpathlineto{\pgfqpoint{3.927305in}{0.739656in}}%
\pgfpathlineto{\pgfqpoint{3.927009in}{0.739656in}}%
\pgfpathlineto{\pgfqpoint{3.926713in}{0.739656in}}%
\pgfpathlineto{\pgfqpoint{3.926417in}{0.739656in}}%
\pgfpathlineto{\pgfqpoint{3.926121in}{0.739656in}}%
\pgfpathlineto{\pgfqpoint{3.925825in}{0.739656in}}%
\pgfpathlineto{\pgfqpoint{3.925529in}{0.739656in}}%
\pgfpathlineto{\pgfqpoint{3.925233in}{0.739656in}}%
\pgfpathlineto{\pgfqpoint{3.924937in}{0.739656in}}%
\pgfpathlineto{\pgfqpoint{3.924641in}{0.739656in}}%
\pgfpathlineto{\pgfqpoint{3.924345in}{0.739656in}}%
\pgfpathlineto{\pgfqpoint{3.924049in}{0.739656in}}%
\pgfpathlineto{\pgfqpoint{3.923753in}{0.739656in}}%
\pgfpathlineto{\pgfqpoint{3.923457in}{0.739656in}}%
\pgfpathlineto{\pgfqpoint{3.923161in}{0.739656in}}%
\pgfpathlineto{\pgfqpoint{3.922865in}{0.739656in}}%
\pgfpathlineto{\pgfqpoint{3.922569in}{0.739656in}}%
\pgfpathlineto{\pgfqpoint{3.922273in}{0.739656in}}%
\pgfpathlineto{\pgfqpoint{3.921977in}{0.739656in}}%
\pgfpathlineto{\pgfqpoint{3.921681in}{0.739656in}}%
\pgfpathlineto{\pgfqpoint{3.921385in}{0.739656in}}%
\pgfpathlineto{\pgfqpoint{3.921089in}{0.739656in}}%
\pgfpathlineto{\pgfqpoint{3.920793in}{0.739656in}}%
\pgfpathlineto{\pgfqpoint{3.920497in}{0.739656in}}%
\pgfpathlineto{\pgfqpoint{3.920201in}{0.739656in}}%
\pgfpathlineto{\pgfqpoint{3.919905in}{0.739656in}}%
\pgfpathlineto{\pgfqpoint{3.919609in}{0.739656in}}%
\pgfpathlineto{\pgfqpoint{3.919313in}{0.739656in}}%
\pgfpathlineto{\pgfqpoint{3.919017in}{0.739656in}}%
\pgfpathlineto{\pgfqpoint{3.918721in}{0.739656in}}%
\pgfpathlineto{\pgfqpoint{3.918425in}{0.739656in}}%
\pgfpathlineto{\pgfqpoint{3.918129in}{0.739656in}}%
\pgfpathlineto{\pgfqpoint{3.917833in}{0.739656in}}%
\pgfpathlineto{\pgfqpoint{3.917537in}{0.739656in}}%
\pgfpathlineto{\pgfqpoint{3.917241in}{0.739656in}}%
\pgfpathlineto{\pgfqpoint{3.916945in}{0.739656in}}%
\pgfpathlineto{\pgfqpoint{3.916649in}{0.739656in}}%
\pgfpathlineto{\pgfqpoint{3.916353in}{0.739656in}}%
\pgfpathlineto{\pgfqpoint{3.916057in}{0.739656in}}%
\pgfpathlineto{\pgfqpoint{3.915761in}{0.739656in}}%
\pgfpathlineto{\pgfqpoint{3.915465in}{0.739656in}}%
\pgfpathlineto{\pgfqpoint{3.915169in}{0.739656in}}%
\pgfpathlineto{\pgfqpoint{3.914873in}{0.739656in}}%
\pgfpathlineto{\pgfqpoint{3.914577in}{0.739656in}}%
\pgfpathlineto{\pgfqpoint{3.914281in}{0.739656in}}%
\pgfpathlineto{\pgfqpoint{3.913985in}{0.739656in}}%
\pgfpathlineto{\pgfqpoint{3.913689in}{0.739656in}}%
\pgfpathlineto{\pgfqpoint{3.913393in}{0.739656in}}%
\pgfpathlineto{\pgfqpoint{3.913097in}{0.739656in}}%
\pgfpathlineto{\pgfqpoint{3.912801in}{0.739656in}}%
\pgfpathlineto{\pgfqpoint{3.912505in}{0.739656in}}%
\pgfpathlineto{\pgfqpoint{3.912209in}{0.739656in}}%
\pgfpathlineto{\pgfqpoint{3.911913in}{0.739656in}}%
\pgfpathlineto{\pgfqpoint{3.911617in}{0.739656in}}%
\pgfpathlineto{\pgfqpoint{3.911321in}{0.739656in}}%
\pgfpathlineto{\pgfqpoint{3.911025in}{0.739656in}}%
\pgfpathlineto{\pgfqpoint{3.910729in}{0.739656in}}%
\pgfpathlineto{\pgfqpoint{3.910433in}{0.739656in}}%
\pgfpathlineto{\pgfqpoint{3.910137in}{0.739656in}}%
\pgfpathlineto{\pgfqpoint{3.909841in}{0.739656in}}%
\pgfpathlineto{\pgfqpoint{3.909545in}{0.739656in}}%
\pgfpathlineto{\pgfqpoint{3.909249in}{0.739656in}}%
\pgfpathlineto{\pgfqpoint{3.908953in}{0.739656in}}%
\pgfpathlineto{\pgfqpoint{3.908657in}{0.739656in}}%
\pgfpathlineto{\pgfqpoint{3.908361in}{0.739656in}}%
\pgfpathlineto{\pgfqpoint{3.908065in}{0.739656in}}%
\pgfpathlineto{\pgfqpoint{3.907769in}{0.739656in}}%
\pgfpathlineto{\pgfqpoint{3.907473in}{0.739656in}}%
\pgfpathlineto{\pgfqpoint{3.907177in}{0.739656in}}%
\pgfpathlineto{\pgfqpoint{3.906881in}{0.739656in}}%
\pgfpathlineto{\pgfqpoint{3.906585in}{0.739656in}}%
\pgfpathlineto{\pgfqpoint{3.906289in}{0.739656in}}%
\pgfpathlineto{\pgfqpoint{3.905993in}{0.739656in}}%
\pgfpathlineto{\pgfqpoint{3.905697in}{0.739656in}}%
\pgfpathlineto{\pgfqpoint{3.905401in}{0.739656in}}%
\pgfpathlineto{\pgfqpoint{3.905105in}{0.739656in}}%
\pgfpathlineto{\pgfqpoint{3.904809in}{0.739656in}}%
\pgfpathlineto{\pgfqpoint{3.904513in}{0.739656in}}%
\pgfpathlineto{\pgfqpoint{3.904217in}{0.739656in}}%
\pgfpathlineto{\pgfqpoint{3.903921in}{0.739656in}}%
\pgfpathlineto{\pgfqpoint{3.903625in}{0.739656in}}%
\pgfpathlineto{\pgfqpoint{3.903329in}{0.739656in}}%
\pgfpathlineto{\pgfqpoint{3.903033in}{0.739656in}}%
\pgfpathlineto{\pgfqpoint{3.902737in}{0.739656in}}%
\pgfpathlineto{\pgfqpoint{3.902441in}{0.739656in}}%
\pgfpathlineto{\pgfqpoint{3.902145in}{0.739656in}}%
\pgfpathlineto{\pgfqpoint{3.901849in}{0.739656in}}%
\pgfpathlineto{\pgfqpoint{3.901553in}{0.739656in}}%
\pgfpathlineto{\pgfqpoint{3.901257in}{0.739656in}}%
\pgfpathlineto{\pgfqpoint{3.900961in}{0.739656in}}%
\pgfpathlineto{\pgfqpoint{3.900665in}{0.739656in}}%
\pgfpathlineto{\pgfqpoint{3.900369in}{0.739656in}}%
\pgfpathlineto{\pgfqpoint{3.900073in}{0.739656in}}%
\pgfpathlineto{\pgfqpoint{3.899777in}{0.739656in}}%
\pgfpathlineto{\pgfqpoint{3.899481in}{0.739656in}}%
\pgfpathlineto{\pgfqpoint{3.899185in}{0.739656in}}%
\pgfpathlineto{\pgfqpoint{3.898889in}{0.739656in}}%
\pgfpathlineto{\pgfqpoint{3.898593in}{0.739656in}}%
\pgfpathlineto{\pgfqpoint{3.898297in}{0.739656in}}%
\pgfpathlineto{\pgfqpoint{3.898001in}{0.739656in}}%
\pgfpathlineto{\pgfqpoint{3.897705in}{0.739656in}}%
\pgfpathlineto{\pgfqpoint{3.897409in}{0.739656in}}%
\pgfpathlineto{\pgfqpoint{3.897113in}{0.739656in}}%
\pgfpathlineto{\pgfqpoint{3.896817in}{0.739656in}}%
\pgfpathlineto{\pgfqpoint{3.896521in}{0.739656in}}%
\pgfpathlineto{\pgfqpoint{3.896225in}{0.739656in}}%
\pgfpathlineto{\pgfqpoint{3.895929in}{0.739656in}}%
\pgfpathlineto{\pgfqpoint{3.895633in}{0.739656in}}%
\pgfpathlineto{\pgfqpoint{3.895337in}{0.739656in}}%
\pgfpathlineto{\pgfqpoint{3.895041in}{0.739656in}}%
\pgfpathlineto{\pgfqpoint{3.894745in}{0.739656in}}%
\pgfpathlineto{\pgfqpoint{3.894449in}{0.739656in}}%
\pgfpathlineto{\pgfqpoint{3.894153in}{0.739656in}}%
\pgfpathlineto{\pgfqpoint{3.893857in}{0.739656in}}%
\pgfpathlineto{\pgfqpoint{3.893561in}{0.739656in}}%
\pgfpathlineto{\pgfqpoint{3.893265in}{0.739656in}}%
\pgfpathlineto{\pgfqpoint{3.892969in}{0.739656in}}%
\pgfpathlineto{\pgfqpoint{3.892673in}{0.739656in}}%
\pgfpathlineto{\pgfqpoint{3.892377in}{0.739656in}}%
\pgfpathlineto{\pgfqpoint{3.892081in}{0.739656in}}%
\pgfpathlineto{\pgfqpoint{3.891785in}{0.739656in}}%
\pgfpathlineto{\pgfqpoint{3.891489in}{0.739656in}}%
\pgfpathlineto{\pgfqpoint{3.891193in}{0.739656in}}%
\pgfpathlineto{\pgfqpoint{3.890897in}{0.739656in}}%
\pgfpathlineto{\pgfqpoint{3.890601in}{0.739656in}}%
\pgfpathlineto{\pgfqpoint{3.890305in}{0.739656in}}%
\pgfpathlineto{\pgfqpoint{3.890009in}{0.739656in}}%
\pgfpathlineto{\pgfqpoint{3.889713in}{0.739656in}}%
\pgfpathlineto{\pgfqpoint{3.889417in}{0.739656in}}%
\pgfpathlineto{\pgfqpoint{3.889121in}{0.739656in}}%
\pgfpathlineto{\pgfqpoint{3.888825in}{0.739656in}}%
\pgfpathlineto{\pgfqpoint{3.888529in}{0.739656in}}%
\pgfpathlineto{\pgfqpoint{3.888233in}{0.739656in}}%
\pgfpathlineto{\pgfqpoint{3.887937in}{0.739656in}}%
\pgfpathlineto{\pgfqpoint{3.887641in}{0.739656in}}%
\pgfpathlineto{\pgfqpoint{3.887345in}{0.739656in}}%
\pgfpathlineto{\pgfqpoint{3.887049in}{0.739656in}}%
\pgfpathlineto{\pgfqpoint{3.886753in}{0.739656in}}%
\pgfpathlineto{\pgfqpoint{3.886457in}{0.739656in}}%
\pgfpathlineto{\pgfqpoint{3.886161in}{0.739656in}}%
\pgfpathlineto{\pgfqpoint{3.885865in}{0.739656in}}%
\pgfpathlineto{\pgfqpoint{3.885569in}{0.739656in}}%
\pgfpathlineto{\pgfqpoint{3.885273in}{0.739656in}}%
\pgfpathlineto{\pgfqpoint{3.884977in}{0.739656in}}%
\pgfpathlineto{\pgfqpoint{3.884681in}{0.739656in}}%
\pgfpathlineto{\pgfqpoint{3.884385in}{0.739656in}}%
\pgfpathlineto{\pgfqpoint{3.884089in}{0.739656in}}%
\pgfpathlineto{\pgfqpoint{3.883792in}{0.739656in}}%
\pgfpathlineto{\pgfqpoint{3.883496in}{0.739656in}}%
\pgfpathlineto{\pgfqpoint{3.883200in}{0.739656in}}%
\pgfpathlineto{\pgfqpoint{3.882904in}{0.739656in}}%
\pgfpathlineto{\pgfqpoint{3.882608in}{0.739656in}}%
\pgfpathlineto{\pgfqpoint{3.882312in}{0.739656in}}%
\pgfpathlineto{\pgfqpoint{3.882016in}{0.739656in}}%
\pgfpathlineto{\pgfqpoint{3.881720in}{0.739656in}}%
\pgfpathlineto{\pgfqpoint{3.881424in}{0.739656in}}%
\pgfpathlineto{\pgfqpoint{3.881128in}{0.739656in}}%
\pgfpathlineto{\pgfqpoint{3.880832in}{0.739656in}}%
\pgfpathlineto{\pgfqpoint{3.880536in}{0.739656in}}%
\pgfpathlineto{\pgfqpoint{3.880240in}{0.739656in}}%
\pgfpathlineto{\pgfqpoint{3.879944in}{0.739656in}}%
\pgfpathlineto{\pgfqpoint{3.879648in}{0.739656in}}%
\pgfpathlineto{\pgfqpoint{3.879352in}{0.739656in}}%
\pgfpathlineto{\pgfqpoint{3.879056in}{0.739656in}}%
\pgfpathlineto{\pgfqpoint{3.878760in}{0.739656in}}%
\pgfpathlineto{\pgfqpoint{3.878464in}{0.739656in}}%
\pgfpathlineto{\pgfqpoint{3.878168in}{0.739656in}}%
\pgfpathlineto{\pgfqpoint{3.877872in}{0.739656in}}%
\pgfpathlineto{\pgfqpoint{3.877576in}{0.739656in}}%
\pgfpathlineto{\pgfqpoint{3.877280in}{0.739656in}}%
\pgfpathlineto{\pgfqpoint{3.876984in}{0.739656in}}%
\pgfpathlineto{\pgfqpoint{3.876688in}{0.739656in}}%
\pgfpathlineto{\pgfqpoint{3.876392in}{0.739656in}}%
\pgfpathlineto{\pgfqpoint{3.876096in}{0.739656in}}%
\pgfpathlineto{\pgfqpoint{3.875800in}{0.739656in}}%
\pgfpathlineto{\pgfqpoint{3.875504in}{0.739656in}}%
\pgfpathlineto{\pgfqpoint{3.875208in}{0.739656in}}%
\pgfpathlineto{\pgfqpoint{3.874912in}{0.739656in}}%
\pgfpathlineto{\pgfqpoint{3.874616in}{0.739656in}}%
\pgfpathlineto{\pgfqpoint{3.874320in}{0.739656in}}%
\pgfpathlineto{\pgfqpoint{3.874024in}{0.739656in}}%
\pgfpathlineto{\pgfqpoint{3.873728in}{0.739656in}}%
\pgfpathlineto{\pgfqpoint{3.873432in}{0.739656in}}%
\pgfpathlineto{\pgfqpoint{3.873136in}{0.739656in}}%
\pgfpathlineto{\pgfqpoint{3.872840in}{0.739656in}}%
\pgfpathlineto{\pgfqpoint{3.872544in}{0.739656in}}%
\pgfpathlineto{\pgfqpoint{3.872248in}{0.739656in}}%
\pgfpathlineto{\pgfqpoint{3.871952in}{0.739656in}}%
\pgfpathlineto{\pgfqpoint{3.871656in}{0.739656in}}%
\pgfpathlineto{\pgfqpoint{3.871360in}{0.739656in}}%
\pgfpathlineto{\pgfqpoint{3.871064in}{0.739656in}}%
\pgfpathlineto{\pgfqpoint{3.870768in}{0.739656in}}%
\pgfpathlineto{\pgfqpoint{3.870472in}{0.739656in}}%
\pgfpathlineto{\pgfqpoint{3.870176in}{0.739656in}}%
\pgfpathlineto{\pgfqpoint{3.869880in}{0.739656in}}%
\pgfpathlineto{\pgfqpoint{3.869584in}{0.739656in}}%
\pgfpathlineto{\pgfqpoint{3.869288in}{0.739656in}}%
\pgfpathlineto{\pgfqpoint{3.868992in}{0.739656in}}%
\pgfpathlineto{\pgfqpoint{3.868696in}{0.739656in}}%
\pgfpathlineto{\pgfqpoint{3.868400in}{0.739656in}}%
\pgfpathlineto{\pgfqpoint{3.868104in}{0.739656in}}%
\pgfpathlineto{\pgfqpoint{3.867808in}{0.739656in}}%
\pgfpathlineto{\pgfqpoint{3.867512in}{0.739656in}}%
\pgfpathlineto{\pgfqpoint{3.867216in}{0.739656in}}%
\pgfpathlineto{\pgfqpoint{3.866920in}{0.739656in}}%
\pgfpathlineto{\pgfqpoint{3.866624in}{0.739656in}}%
\pgfpathlineto{\pgfqpoint{3.866328in}{0.739656in}}%
\pgfpathlineto{\pgfqpoint{3.866032in}{0.739656in}}%
\pgfpathlineto{\pgfqpoint{3.865736in}{0.739656in}}%
\pgfpathlineto{\pgfqpoint{3.865440in}{0.739656in}}%
\pgfpathlineto{\pgfqpoint{3.865144in}{0.739656in}}%
\pgfpathlineto{\pgfqpoint{3.864848in}{0.739656in}}%
\pgfpathlineto{\pgfqpoint{3.864552in}{0.739656in}}%
\pgfpathlineto{\pgfqpoint{3.864256in}{0.739656in}}%
\pgfpathlineto{\pgfqpoint{3.863960in}{0.739656in}}%
\pgfpathlineto{\pgfqpoint{3.863664in}{0.739656in}}%
\pgfpathlineto{\pgfqpoint{3.863368in}{0.739656in}}%
\pgfpathlineto{\pgfqpoint{3.863072in}{0.739656in}}%
\pgfpathlineto{\pgfqpoint{3.862776in}{0.739656in}}%
\pgfpathlineto{\pgfqpoint{3.862480in}{0.739656in}}%
\pgfpathlineto{\pgfqpoint{3.862184in}{0.739656in}}%
\pgfpathlineto{\pgfqpoint{3.861888in}{0.739656in}}%
\pgfpathlineto{\pgfqpoint{3.861592in}{0.739656in}}%
\pgfpathlineto{\pgfqpoint{3.861296in}{0.739656in}}%
\pgfpathlineto{\pgfqpoint{3.861000in}{0.739656in}}%
\pgfpathlineto{\pgfqpoint{3.860704in}{0.739656in}}%
\pgfpathlineto{\pgfqpoint{3.860408in}{0.739656in}}%
\pgfpathlineto{\pgfqpoint{3.860112in}{0.739656in}}%
\pgfpathlineto{\pgfqpoint{3.859816in}{0.739656in}}%
\pgfpathlineto{\pgfqpoint{3.859520in}{0.739656in}}%
\pgfpathlineto{\pgfqpoint{3.859224in}{0.739656in}}%
\pgfpathlineto{\pgfqpoint{3.858928in}{0.739656in}}%
\pgfpathlineto{\pgfqpoint{3.858632in}{0.739656in}}%
\pgfpathlineto{\pgfqpoint{3.858336in}{0.739656in}}%
\pgfpathlineto{\pgfqpoint{3.858040in}{0.739656in}}%
\pgfpathlineto{\pgfqpoint{3.857744in}{0.739656in}}%
\pgfpathlineto{\pgfqpoint{3.857448in}{0.739656in}}%
\pgfpathlineto{\pgfqpoint{3.857152in}{0.739656in}}%
\pgfpathlineto{\pgfqpoint{3.856856in}{0.739656in}}%
\pgfpathlineto{\pgfqpoint{3.856560in}{0.739656in}}%
\pgfpathlineto{\pgfqpoint{3.856264in}{0.739656in}}%
\pgfpathlineto{\pgfqpoint{3.855968in}{0.739656in}}%
\pgfpathlineto{\pgfqpoint{3.855672in}{0.739656in}}%
\pgfpathlineto{\pgfqpoint{3.855376in}{0.739656in}}%
\pgfpathlineto{\pgfqpoint{3.855080in}{0.739656in}}%
\pgfpathlineto{\pgfqpoint{3.854784in}{0.739656in}}%
\pgfpathlineto{\pgfqpoint{3.854488in}{0.739656in}}%
\pgfpathlineto{\pgfqpoint{3.854192in}{0.739656in}}%
\pgfpathlineto{\pgfqpoint{3.853896in}{0.739656in}}%
\pgfpathlineto{\pgfqpoint{3.853600in}{0.739656in}}%
\pgfpathlineto{\pgfqpoint{3.853304in}{0.739656in}}%
\pgfpathlineto{\pgfqpoint{3.853008in}{0.739656in}}%
\pgfpathlineto{\pgfqpoint{3.852712in}{0.739656in}}%
\pgfpathlineto{\pgfqpoint{3.852416in}{0.739656in}}%
\pgfpathlineto{\pgfqpoint{3.852120in}{0.739656in}}%
\pgfpathlineto{\pgfqpoint{3.851824in}{0.739656in}}%
\pgfpathlineto{\pgfqpoint{3.851528in}{0.739656in}}%
\pgfpathlineto{\pgfqpoint{3.851232in}{0.739656in}}%
\pgfpathlineto{\pgfqpoint{3.850936in}{0.739656in}}%
\pgfpathlineto{\pgfqpoint{3.850640in}{0.739656in}}%
\pgfpathlineto{\pgfqpoint{3.850344in}{0.739656in}}%
\pgfpathlineto{\pgfqpoint{3.850048in}{0.739656in}}%
\pgfpathlineto{\pgfqpoint{3.849752in}{0.739656in}}%
\pgfpathlineto{\pgfqpoint{3.849456in}{0.739656in}}%
\pgfpathlineto{\pgfqpoint{3.849160in}{0.739656in}}%
\pgfpathlineto{\pgfqpoint{3.848864in}{0.739656in}}%
\pgfpathlineto{\pgfqpoint{3.848568in}{0.739656in}}%
\pgfpathlineto{\pgfqpoint{3.848272in}{0.739656in}}%
\pgfpathlineto{\pgfqpoint{3.847976in}{0.739656in}}%
\pgfpathlineto{\pgfqpoint{3.847680in}{0.739656in}}%
\pgfpathlineto{\pgfqpoint{3.847384in}{0.739656in}}%
\pgfpathlineto{\pgfqpoint{3.847088in}{0.739656in}}%
\pgfpathlineto{\pgfqpoint{3.846792in}{0.739656in}}%
\pgfpathlineto{\pgfqpoint{3.846496in}{0.739656in}}%
\pgfpathlineto{\pgfqpoint{3.846200in}{0.739656in}}%
\pgfpathlineto{\pgfqpoint{3.845904in}{0.739656in}}%
\pgfpathlineto{\pgfqpoint{3.845608in}{0.739656in}}%
\pgfpathlineto{\pgfqpoint{3.845312in}{0.739656in}}%
\pgfpathlineto{\pgfqpoint{3.845016in}{0.739656in}}%
\pgfpathlineto{\pgfqpoint{3.844720in}{0.739656in}}%
\pgfpathlineto{\pgfqpoint{3.844424in}{0.739656in}}%
\pgfpathlineto{\pgfqpoint{3.844128in}{0.739656in}}%
\pgfpathlineto{\pgfqpoint{3.843832in}{0.739656in}}%
\pgfpathlineto{\pgfqpoint{3.843536in}{0.739656in}}%
\pgfpathlineto{\pgfqpoint{3.843240in}{0.739656in}}%
\pgfpathlineto{\pgfqpoint{3.842944in}{0.739656in}}%
\pgfpathlineto{\pgfqpoint{3.842648in}{0.739656in}}%
\pgfpathlineto{\pgfqpoint{3.842352in}{0.739656in}}%
\pgfpathlineto{\pgfqpoint{3.842056in}{0.739656in}}%
\pgfpathlineto{\pgfqpoint{3.841760in}{0.739656in}}%
\pgfpathlineto{\pgfqpoint{3.841464in}{0.739656in}}%
\pgfpathlineto{\pgfqpoint{3.841168in}{0.739656in}}%
\pgfpathlineto{\pgfqpoint{3.840872in}{0.739656in}}%
\pgfpathlineto{\pgfqpoint{3.840576in}{0.739656in}}%
\pgfpathlineto{\pgfqpoint{3.840280in}{0.739656in}}%
\pgfpathlineto{\pgfqpoint{3.839984in}{0.739656in}}%
\pgfpathlineto{\pgfqpoint{3.839688in}{0.739656in}}%
\pgfpathlineto{\pgfqpoint{3.839392in}{0.739656in}}%
\pgfpathlineto{\pgfqpoint{3.839096in}{0.739656in}}%
\pgfpathlineto{\pgfqpoint{3.838800in}{0.739656in}}%
\pgfpathlineto{\pgfqpoint{3.838504in}{0.739656in}}%
\pgfpathlineto{\pgfqpoint{3.838208in}{0.739656in}}%
\pgfpathlineto{\pgfqpoint{3.837912in}{0.739656in}}%
\pgfpathlineto{\pgfqpoint{3.837616in}{0.739656in}}%
\pgfpathlineto{\pgfqpoint{3.837320in}{0.739656in}}%
\pgfpathlineto{\pgfqpoint{3.837024in}{0.739656in}}%
\pgfpathlineto{\pgfqpoint{3.836728in}{0.739656in}}%
\pgfpathlineto{\pgfqpoint{3.836432in}{0.739656in}}%
\pgfpathlineto{\pgfqpoint{3.836136in}{0.739656in}}%
\pgfpathlineto{\pgfqpoint{3.835840in}{0.739656in}}%
\pgfpathlineto{\pgfqpoint{3.835544in}{0.739656in}}%
\pgfpathlineto{\pgfqpoint{3.835248in}{0.739656in}}%
\pgfpathlineto{\pgfqpoint{3.834952in}{0.739656in}}%
\pgfpathlineto{\pgfqpoint{3.834656in}{0.739656in}}%
\pgfpathlineto{\pgfqpoint{3.834360in}{0.739656in}}%
\pgfpathlineto{\pgfqpoint{3.834064in}{0.739656in}}%
\pgfpathlineto{\pgfqpoint{3.833768in}{0.739656in}}%
\pgfpathlineto{\pgfqpoint{3.833472in}{0.739656in}}%
\pgfpathlineto{\pgfqpoint{3.833176in}{0.739656in}}%
\pgfpathlineto{\pgfqpoint{3.832880in}{0.739656in}}%
\pgfpathlineto{\pgfqpoint{3.832584in}{0.739656in}}%
\pgfpathlineto{\pgfqpoint{3.832288in}{0.739656in}}%
\pgfpathlineto{\pgfqpoint{3.831992in}{0.739656in}}%
\pgfpathlineto{\pgfqpoint{3.831696in}{0.739656in}}%
\pgfpathlineto{\pgfqpoint{3.831400in}{0.739656in}}%
\pgfpathlineto{\pgfqpoint{3.831104in}{0.739656in}}%
\pgfpathlineto{\pgfqpoint{3.830808in}{0.739656in}}%
\pgfpathlineto{\pgfqpoint{3.830512in}{0.739656in}}%
\pgfpathlineto{\pgfqpoint{3.830216in}{0.739656in}}%
\pgfpathlineto{\pgfqpoint{3.829920in}{0.739656in}}%
\pgfpathlineto{\pgfqpoint{3.829624in}{0.739656in}}%
\pgfpathlineto{\pgfqpoint{3.829328in}{0.739656in}}%
\pgfpathlineto{\pgfqpoint{3.829032in}{0.739656in}}%
\pgfpathlineto{\pgfqpoint{3.828736in}{0.739656in}}%
\pgfpathlineto{\pgfqpoint{3.828440in}{0.739656in}}%
\pgfpathlineto{\pgfqpoint{3.828144in}{0.739656in}}%
\pgfpathlineto{\pgfqpoint{3.827848in}{0.739656in}}%
\pgfpathlineto{\pgfqpoint{3.827552in}{0.739656in}}%
\pgfpathlineto{\pgfqpoint{3.827256in}{0.739656in}}%
\pgfpathlineto{\pgfqpoint{3.826960in}{0.739656in}}%
\pgfpathlineto{\pgfqpoint{3.826664in}{0.739656in}}%
\pgfpathlineto{\pgfqpoint{3.826368in}{0.739656in}}%
\pgfpathlineto{\pgfqpoint{3.826072in}{0.739656in}}%
\pgfpathlineto{\pgfqpoint{3.825776in}{0.739656in}}%
\pgfpathlineto{\pgfqpoint{3.825480in}{0.739656in}}%
\pgfpathlineto{\pgfqpoint{3.825184in}{0.739656in}}%
\pgfpathlineto{\pgfqpoint{3.824888in}{0.739656in}}%
\pgfpathlineto{\pgfqpoint{3.824592in}{0.739656in}}%
\pgfpathlineto{\pgfqpoint{3.824296in}{0.739656in}}%
\pgfpathlineto{\pgfqpoint{3.824000in}{0.739656in}}%
\pgfpathlineto{\pgfqpoint{3.823704in}{0.739656in}}%
\pgfpathlineto{\pgfqpoint{3.823408in}{0.739656in}}%
\pgfpathlineto{\pgfqpoint{3.823112in}{0.739656in}}%
\pgfpathlineto{\pgfqpoint{3.822816in}{0.739656in}}%
\pgfpathlineto{\pgfqpoint{3.822520in}{0.739656in}}%
\pgfpathlineto{\pgfqpoint{3.822224in}{0.739656in}}%
\pgfpathlineto{\pgfqpoint{3.821928in}{0.739656in}}%
\pgfpathlineto{\pgfqpoint{3.821632in}{0.739656in}}%
\pgfpathlineto{\pgfqpoint{3.821336in}{0.739656in}}%
\pgfpathlineto{\pgfqpoint{3.821040in}{0.739656in}}%
\pgfpathlineto{\pgfqpoint{3.820744in}{0.739656in}}%
\pgfpathlineto{\pgfqpoint{3.820448in}{0.739656in}}%
\pgfpathlineto{\pgfqpoint{3.820152in}{0.739656in}}%
\pgfpathlineto{\pgfqpoint{3.819856in}{0.739656in}}%
\pgfpathlineto{\pgfqpoint{3.819560in}{0.739656in}}%
\pgfpathlineto{\pgfqpoint{3.819264in}{0.739656in}}%
\pgfpathlineto{\pgfqpoint{3.818968in}{0.739656in}}%
\pgfpathlineto{\pgfqpoint{3.818672in}{0.739656in}}%
\pgfpathlineto{\pgfqpoint{3.818376in}{0.739656in}}%
\pgfpathlineto{\pgfqpoint{3.818080in}{0.739656in}}%
\pgfpathlineto{\pgfqpoint{3.817784in}{0.739656in}}%
\pgfpathlineto{\pgfqpoint{3.817488in}{0.739656in}}%
\pgfpathlineto{\pgfqpoint{3.817192in}{0.739656in}}%
\pgfpathlineto{\pgfqpoint{3.816896in}{0.739656in}}%
\pgfpathlineto{\pgfqpoint{3.816600in}{0.739656in}}%
\pgfpathlineto{\pgfqpoint{3.816303in}{0.739656in}}%
\pgfpathlineto{\pgfqpoint{3.816007in}{0.739656in}}%
\pgfpathlineto{\pgfqpoint{3.815711in}{0.739656in}}%
\pgfpathlineto{\pgfqpoint{3.815415in}{0.739656in}}%
\pgfpathlineto{\pgfqpoint{3.815119in}{0.739656in}}%
\pgfpathlineto{\pgfqpoint{3.814823in}{0.739656in}}%
\pgfpathlineto{\pgfqpoint{3.814527in}{0.739656in}}%
\pgfpathlineto{\pgfqpoint{3.814231in}{0.739656in}}%
\pgfpathlineto{\pgfqpoint{3.813935in}{0.739656in}}%
\pgfpathlineto{\pgfqpoint{3.813639in}{0.739656in}}%
\pgfpathlineto{\pgfqpoint{3.813343in}{0.739656in}}%
\pgfpathlineto{\pgfqpoint{3.813047in}{0.739656in}}%
\pgfpathlineto{\pgfqpoint{3.812751in}{0.739656in}}%
\pgfpathlineto{\pgfqpoint{3.812455in}{0.739656in}}%
\pgfpathlineto{\pgfqpoint{3.812159in}{0.739656in}}%
\pgfpathlineto{\pgfqpoint{3.811863in}{0.739656in}}%
\pgfpathlineto{\pgfqpoint{3.811567in}{0.739656in}}%
\pgfpathlineto{\pgfqpoint{3.811271in}{0.739656in}}%
\pgfpathlineto{\pgfqpoint{3.810975in}{0.739656in}}%
\pgfpathlineto{\pgfqpoint{3.810679in}{0.739656in}}%
\pgfpathlineto{\pgfqpoint{3.810383in}{0.739656in}}%
\pgfpathlineto{\pgfqpoint{3.810087in}{0.739656in}}%
\pgfpathlineto{\pgfqpoint{3.809791in}{0.739656in}}%
\pgfpathlineto{\pgfqpoint{3.809495in}{0.739656in}}%
\pgfpathlineto{\pgfqpoint{3.809199in}{0.739656in}}%
\pgfpathlineto{\pgfqpoint{3.808903in}{0.739656in}}%
\pgfpathlineto{\pgfqpoint{3.808607in}{0.739656in}}%
\pgfpathlineto{\pgfqpoint{3.808311in}{0.739656in}}%
\pgfpathlineto{\pgfqpoint{3.808015in}{0.739656in}}%
\pgfpathlineto{\pgfqpoint{3.807719in}{0.739656in}}%
\pgfpathlineto{\pgfqpoint{3.807423in}{0.739656in}}%
\pgfpathlineto{\pgfqpoint{3.807127in}{0.739656in}}%
\pgfpathlineto{\pgfqpoint{3.806831in}{0.739656in}}%
\pgfpathlineto{\pgfqpoint{3.806535in}{0.739656in}}%
\pgfpathlineto{\pgfqpoint{3.806239in}{0.739656in}}%
\pgfpathlineto{\pgfqpoint{3.805943in}{0.739656in}}%
\pgfpathlineto{\pgfqpoint{3.805647in}{0.739656in}}%
\pgfpathlineto{\pgfqpoint{3.805351in}{0.739656in}}%
\pgfpathlineto{\pgfqpoint{3.805055in}{0.739656in}}%
\pgfpathlineto{\pgfqpoint{3.804759in}{0.739656in}}%
\pgfpathlineto{\pgfqpoint{3.804463in}{0.739656in}}%
\pgfpathlineto{\pgfqpoint{3.804167in}{0.739656in}}%
\pgfpathlineto{\pgfqpoint{3.803871in}{0.739656in}}%
\pgfpathlineto{\pgfqpoint{3.803575in}{0.739656in}}%
\pgfpathlineto{\pgfqpoint{3.803279in}{0.739656in}}%
\pgfpathlineto{\pgfqpoint{3.802983in}{0.739656in}}%
\pgfpathlineto{\pgfqpoint{3.802687in}{0.739656in}}%
\pgfpathlineto{\pgfqpoint{3.802391in}{0.739656in}}%
\pgfpathlineto{\pgfqpoint{3.802095in}{0.739656in}}%
\pgfpathlineto{\pgfqpoint{3.801799in}{0.739656in}}%
\pgfpathlineto{\pgfqpoint{3.801503in}{0.739656in}}%
\pgfpathlineto{\pgfqpoint{3.801207in}{0.739656in}}%
\pgfpathlineto{\pgfqpoint{3.800911in}{0.739656in}}%
\pgfpathlineto{\pgfqpoint{3.800615in}{0.739656in}}%
\pgfpathlineto{\pgfqpoint{3.800319in}{0.739656in}}%
\pgfpathlineto{\pgfqpoint{3.800023in}{0.739656in}}%
\pgfpathlineto{\pgfqpoint{3.799727in}{0.739656in}}%
\pgfpathlineto{\pgfqpoint{3.799431in}{0.739656in}}%
\pgfpathlineto{\pgfqpoint{3.799135in}{0.739656in}}%
\pgfpathlineto{\pgfqpoint{3.798839in}{0.739656in}}%
\pgfpathlineto{\pgfqpoint{3.798543in}{0.739656in}}%
\pgfpathlineto{\pgfqpoint{3.798247in}{0.739656in}}%
\pgfpathlineto{\pgfqpoint{3.797951in}{0.739656in}}%
\pgfpathlineto{\pgfqpoint{3.797655in}{0.739656in}}%
\pgfpathlineto{\pgfqpoint{3.797359in}{0.739656in}}%
\pgfpathlineto{\pgfqpoint{3.797063in}{0.739656in}}%
\pgfpathlineto{\pgfqpoint{3.796767in}{0.739656in}}%
\pgfpathlineto{\pgfqpoint{3.796471in}{0.739656in}}%
\pgfpathlineto{\pgfqpoint{3.796175in}{0.739656in}}%
\pgfpathlineto{\pgfqpoint{3.795879in}{0.739656in}}%
\pgfpathlineto{\pgfqpoint{3.795583in}{0.739656in}}%
\pgfpathlineto{\pgfqpoint{3.795287in}{0.739656in}}%
\pgfpathlineto{\pgfqpoint{3.794991in}{0.739656in}}%
\pgfpathlineto{\pgfqpoint{3.794695in}{0.739656in}}%
\pgfpathlineto{\pgfqpoint{3.794399in}{0.739656in}}%
\pgfpathlineto{\pgfqpoint{3.794103in}{0.739656in}}%
\pgfpathlineto{\pgfqpoint{3.793807in}{0.739656in}}%
\pgfpathlineto{\pgfqpoint{3.793511in}{0.739656in}}%
\pgfpathlineto{\pgfqpoint{3.793215in}{0.739656in}}%
\pgfpathlineto{\pgfqpoint{3.792919in}{0.739656in}}%
\pgfpathlineto{\pgfqpoint{3.792623in}{0.739656in}}%
\pgfpathlineto{\pgfqpoint{3.792327in}{0.739656in}}%
\pgfpathlineto{\pgfqpoint{3.792031in}{0.739656in}}%
\pgfpathlineto{\pgfqpoint{3.791735in}{0.739656in}}%
\pgfpathlineto{\pgfqpoint{3.791439in}{0.739656in}}%
\pgfpathlineto{\pgfqpoint{3.791143in}{0.739656in}}%
\pgfpathlineto{\pgfqpoint{3.790847in}{0.739656in}}%
\pgfpathlineto{\pgfqpoint{3.790551in}{0.739656in}}%
\pgfpathlineto{\pgfqpoint{3.790255in}{0.739656in}}%
\pgfpathlineto{\pgfqpoint{3.789959in}{0.739656in}}%
\pgfpathlineto{\pgfqpoint{3.789663in}{0.739656in}}%
\pgfpathlineto{\pgfqpoint{3.789367in}{0.739656in}}%
\pgfpathlineto{\pgfqpoint{3.789071in}{0.739656in}}%
\pgfpathlineto{\pgfqpoint{3.788775in}{0.739656in}}%
\pgfpathlineto{\pgfqpoint{3.788479in}{0.739656in}}%
\pgfpathlineto{\pgfqpoint{3.788183in}{0.739656in}}%
\pgfpathlineto{\pgfqpoint{3.787887in}{0.739656in}}%
\pgfpathlineto{\pgfqpoint{3.787591in}{0.739656in}}%
\pgfpathlineto{\pgfqpoint{3.787295in}{0.739656in}}%
\pgfpathlineto{\pgfqpoint{3.786999in}{0.739656in}}%
\pgfpathlineto{\pgfqpoint{3.786703in}{0.739656in}}%
\pgfpathlineto{\pgfqpoint{3.786407in}{0.739656in}}%
\pgfpathlineto{\pgfqpoint{3.786111in}{0.739656in}}%
\pgfpathlineto{\pgfqpoint{3.785815in}{0.739656in}}%
\pgfpathlineto{\pgfqpoint{3.785519in}{0.739656in}}%
\pgfpathlineto{\pgfqpoint{3.785223in}{0.739656in}}%
\pgfpathlineto{\pgfqpoint{3.784927in}{0.739656in}}%
\pgfpathlineto{\pgfqpoint{3.784631in}{0.739656in}}%
\pgfpathlineto{\pgfqpoint{3.784335in}{0.739656in}}%
\pgfpathlineto{\pgfqpoint{3.784039in}{0.739656in}}%
\pgfpathlineto{\pgfqpoint{3.783743in}{0.739656in}}%
\pgfpathlineto{\pgfqpoint{3.783447in}{0.739656in}}%
\pgfpathlineto{\pgfqpoint{3.783151in}{0.739656in}}%
\pgfpathlineto{\pgfqpoint{3.782855in}{0.739656in}}%
\pgfpathlineto{\pgfqpoint{3.782559in}{0.739656in}}%
\pgfpathlineto{\pgfqpoint{3.782263in}{0.739656in}}%
\pgfpathlineto{\pgfqpoint{3.781967in}{0.739656in}}%
\pgfpathlineto{\pgfqpoint{3.781671in}{0.739656in}}%
\pgfpathlineto{\pgfqpoint{3.781375in}{0.739656in}}%
\pgfpathlineto{\pgfqpoint{3.781079in}{0.739656in}}%
\pgfpathlineto{\pgfqpoint{3.780783in}{0.739656in}}%
\pgfpathlineto{\pgfqpoint{3.780487in}{0.739656in}}%
\pgfpathlineto{\pgfqpoint{3.780191in}{0.739656in}}%
\pgfpathlineto{\pgfqpoint{3.779895in}{0.739656in}}%
\pgfpathlineto{\pgfqpoint{3.779599in}{0.739656in}}%
\pgfpathlineto{\pgfqpoint{3.779303in}{0.739656in}}%
\pgfpathlineto{\pgfqpoint{3.779007in}{0.739656in}}%
\pgfpathlineto{\pgfqpoint{3.778711in}{0.739656in}}%
\pgfpathlineto{\pgfqpoint{3.778415in}{0.739656in}}%
\pgfpathlineto{\pgfqpoint{3.778119in}{0.739656in}}%
\pgfpathlineto{\pgfqpoint{3.777823in}{0.739656in}}%
\pgfpathlineto{\pgfqpoint{3.777527in}{0.739656in}}%
\pgfpathlineto{\pgfqpoint{3.777231in}{0.739656in}}%
\pgfpathlineto{\pgfqpoint{3.776935in}{0.739656in}}%
\pgfpathlineto{\pgfqpoint{3.776639in}{0.739656in}}%
\pgfpathlineto{\pgfqpoint{3.776343in}{0.739656in}}%
\pgfpathlineto{\pgfqpoint{3.776047in}{0.739656in}}%
\pgfpathlineto{\pgfqpoint{3.775751in}{0.739656in}}%
\pgfpathlineto{\pgfqpoint{3.775455in}{0.739656in}}%
\pgfpathlineto{\pgfqpoint{3.775159in}{0.739656in}}%
\pgfpathlineto{\pgfqpoint{3.774863in}{0.739656in}}%
\pgfpathlineto{\pgfqpoint{3.774567in}{0.739656in}}%
\pgfpathlineto{\pgfqpoint{3.774271in}{0.739656in}}%
\pgfpathlineto{\pgfqpoint{3.773975in}{0.739656in}}%
\pgfpathlineto{\pgfqpoint{3.773679in}{0.739656in}}%
\pgfpathlineto{\pgfqpoint{3.773383in}{0.739656in}}%
\pgfpathlineto{\pgfqpoint{3.773087in}{0.739656in}}%
\pgfpathlineto{\pgfqpoint{3.772791in}{0.739656in}}%
\pgfpathlineto{\pgfqpoint{3.772495in}{0.739656in}}%
\pgfpathlineto{\pgfqpoint{3.772199in}{0.739656in}}%
\pgfpathlineto{\pgfqpoint{3.771903in}{0.739656in}}%
\pgfpathlineto{\pgfqpoint{3.771607in}{0.739656in}}%
\pgfpathlineto{\pgfqpoint{3.771311in}{0.739656in}}%
\pgfpathlineto{\pgfqpoint{3.771015in}{0.739656in}}%
\pgfpathlineto{\pgfqpoint{3.770719in}{0.739656in}}%
\pgfpathlineto{\pgfqpoint{3.770423in}{0.739656in}}%
\pgfpathlineto{\pgfqpoint{3.770127in}{0.739656in}}%
\pgfpathlineto{\pgfqpoint{3.769831in}{0.739656in}}%
\pgfpathlineto{\pgfqpoint{3.769535in}{0.739656in}}%
\pgfpathlineto{\pgfqpoint{3.769239in}{0.739656in}}%
\pgfpathlineto{\pgfqpoint{3.768943in}{0.739656in}}%
\pgfpathlineto{\pgfqpoint{3.768647in}{0.739656in}}%
\pgfpathlineto{\pgfqpoint{3.768351in}{0.739656in}}%
\pgfpathlineto{\pgfqpoint{3.768055in}{0.739656in}}%
\pgfpathlineto{\pgfqpoint{3.767759in}{0.739656in}}%
\pgfpathlineto{\pgfqpoint{3.767463in}{0.739656in}}%
\pgfpathlineto{\pgfqpoint{3.767167in}{0.739656in}}%
\pgfpathlineto{\pgfqpoint{3.766871in}{0.739656in}}%
\pgfpathlineto{\pgfqpoint{3.766575in}{0.739656in}}%
\pgfpathlineto{\pgfqpoint{3.766279in}{0.739656in}}%
\pgfpathlineto{\pgfqpoint{3.765983in}{0.739656in}}%
\pgfpathlineto{\pgfqpoint{3.765687in}{0.739656in}}%
\pgfpathlineto{\pgfqpoint{3.765391in}{0.739656in}}%
\pgfpathlineto{\pgfqpoint{3.765095in}{0.739656in}}%
\pgfpathlineto{\pgfqpoint{3.764799in}{0.739656in}}%
\pgfpathlineto{\pgfqpoint{3.764503in}{0.739656in}}%
\pgfpathlineto{\pgfqpoint{3.764207in}{0.739656in}}%
\pgfpathlineto{\pgfqpoint{3.763911in}{0.739656in}}%
\pgfpathlineto{\pgfqpoint{3.763615in}{0.739656in}}%
\pgfpathlineto{\pgfqpoint{3.763319in}{0.739656in}}%
\pgfpathlineto{\pgfqpoint{3.763023in}{0.739656in}}%
\pgfpathlineto{\pgfqpoint{3.762727in}{0.739656in}}%
\pgfpathlineto{\pgfqpoint{3.762431in}{0.739656in}}%
\pgfpathlineto{\pgfqpoint{3.762135in}{0.739656in}}%
\pgfpathlineto{\pgfqpoint{3.761839in}{0.739656in}}%
\pgfpathlineto{\pgfqpoint{3.761543in}{0.739656in}}%
\pgfpathlineto{\pgfqpoint{3.761247in}{0.739656in}}%
\pgfpathlineto{\pgfqpoint{3.760951in}{0.739656in}}%
\pgfpathlineto{\pgfqpoint{3.760655in}{0.739656in}}%
\pgfpathlineto{\pgfqpoint{3.760359in}{0.739656in}}%
\pgfpathlineto{\pgfqpoint{3.760063in}{0.739656in}}%
\pgfpathlineto{\pgfqpoint{3.759767in}{0.739656in}}%
\pgfpathlineto{\pgfqpoint{3.759471in}{0.739656in}}%
\pgfpathlineto{\pgfqpoint{3.759175in}{0.739656in}}%
\pgfpathlineto{\pgfqpoint{3.758879in}{0.739656in}}%
\pgfpathlineto{\pgfqpoint{3.758583in}{0.739656in}}%
\pgfpathlineto{\pgfqpoint{3.758287in}{0.739656in}}%
\pgfpathlineto{\pgfqpoint{3.757991in}{0.739656in}}%
\pgfpathlineto{\pgfqpoint{3.757695in}{0.739656in}}%
\pgfpathlineto{\pgfqpoint{3.757399in}{0.739656in}}%
\pgfpathlineto{\pgfqpoint{3.757103in}{0.739656in}}%
\pgfpathlineto{\pgfqpoint{3.756807in}{0.739656in}}%
\pgfpathlineto{\pgfqpoint{3.756511in}{0.739656in}}%
\pgfpathlineto{\pgfqpoint{3.756215in}{0.739656in}}%
\pgfpathlineto{\pgfqpoint{3.755919in}{0.739656in}}%
\pgfpathlineto{\pgfqpoint{3.755623in}{0.739656in}}%
\pgfpathlineto{\pgfqpoint{3.755327in}{0.739656in}}%
\pgfpathlineto{\pgfqpoint{3.755031in}{0.739656in}}%
\pgfpathlineto{\pgfqpoint{3.754735in}{0.739656in}}%
\pgfpathlineto{\pgfqpoint{3.754439in}{0.739656in}}%
\pgfpathlineto{\pgfqpoint{3.754143in}{0.739656in}}%
\pgfpathlineto{\pgfqpoint{3.753847in}{0.739656in}}%
\pgfpathlineto{\pgfqpoint{3.753551in}{0.739656in}}%
\pgfpathlineto{\pgfqpoint{3.753255in}{0.739656in}}%
\pgfpathlineto{\pgfqpoint{3.752959in}{0.739656in}}%
\pgfpathlineto{\pgfqpoint{3.752663in}{0.739656in}}%
\pgfpathlineto{\pgfqpoint{3.752367in}{0.739656in}}%
\pgfpathlineto{\pgfqpoint{3.752071in}{0.739656in}}%
\pgfpathlineto{\pgfqpoint{3.751775in}{0.739656in}}%
\pgfpathlineto{\pgfqpoint{3.751479in}{0.739656in}}%
\pgfpathlineto{\pgfqpoint{3.751183in}{0.739656in}}%
\pgfpathlineto{\pgfqpoint{3.750887in}{0.739656in}}%
\pgfpathlineto{\pgfqpoint{3.750591in}{0.739656in}}%
\pgfpathlineto{\pgfqpoint{3.750295in}{0.739656in}}%
\pgfpathlineto{\pgfqpoint{3.749999in}{0.739656in}}%
\pgfpathlineto{\pgfqpoint{3.749703in}{0.739656in}}%
\pgfpathlineto{\pgfqpoint{3.749407in}{0.739656in}}%
\pgfpathlineto{\pgfqpoint{3.749111in}{0.739656in}}%
\pgfpathlineto{\pgfqpoint{3.748814in}{0.739656in}}%
\pgfpathlineto{\pgfqpoint{3.748518in}{0.739656in}}%
\pgfpathlineto{\pgfqpoint{3.748222in}{0.739656in}}%
\pgfpathlineto{\pgfqpoint{3.747926in}{0.739656in}}%
\pgfpathlineto{\pgfqpoint{3.747630in}{0.739656in}}%
\pgfpathlineto{\pgfqpoint{3.747334in}{0.739656in}}%
\pgfpathlineto{\pgfqpoint{3.747038in}{0.739656in}}%
\pgfpathlineto{\pgfqpoint{3.746742in}{0.739656in}}%
\pgfpathlineto{\pgfqpoint{3.746446in}{0.739656in}}%
\pgfpathlineto{\pgfqpoint{3.746150in}{0.739656in}}%
\pgfpathlineto{\pgfqpoint{3.745854in}{0.739656in}}%
\pgfpathlineto{\pgfqpoint{3.745558in}{0.739656in}}%
\pgfpathlineto{\pgfqpoint{3.745262in}{0.739656in}}%
\pgfpathlineto{\pgfqpoint{3.744966in}{0.739656in}}%
\pgfpathlineto{\pgfqpoint{3.744670in}{0.739656in}}%
\pgfpathlineto{\pgfqpoint{3.744374in}{0.739656in}}%
\pgfpathlineto{\pgfqpoint{3.744078in}{0.739656in}}%
\pgfpathlineto{\pgfqpoint{3.743782in}{0.739656in}}%
\pgfpathlineto{\pgfqpoint{3.743486in}{0.739656in}}%
\pgfpathlineto{\pgfqpoint{3.743190in}{0.739656in}}%
\pgfpathlineto{\pgfqpoint{3.742894in}{0.739656in}}%
\pgfpathlineto{\pgfqpoint{3.742598in}{0.739656in}}%
\pgfpathlineto{\pgfqpoint{3.742302in}{0.739656in}}%
\pgfpathlineto{\pgfqpoint{3.742006in}{0.739656in}}%
\pgfpathlineto{\pgfqpoint{3.741710in}{0.739656in}}%
\pgfpathlineto{\pgfqpoint{3.741414in}{0.739656in}}%
\pgfpathlineto{\pgfqpoint{3.741118in}{0.739656in}}%
\pgfpathlineto{\pgfqpoint{3.740822in}{0.739656in}}%
\pgfpathlineto{\pgfqpoint{3.740526in}{0.739656in}}%
\pgfpathlineto{\pgfqpoint{3.740230in}{0.739656in}}%
\pgfpathlineto{\pgfqpoint{3.739934in}{0.739656in}}%
\pgfpathlineto{\pgfqpoint{3.739638in}{0.739656in}}%
\pgfpathlineto{\pgfqpoint{3.739342in}{0.739656in}}%
\pgfpathlineto{\pgfqpoint{3.739046in}{0.739656in}}%
\pgfpathlineto{\pgfqpoint{3.738750in}{0.739656in}}%
\pgfpathlineto{\pgfqpoint{3.738454in}{0.739656in}}%
\pgfpathlineto{\pgfqpoint{3.738158in}{0.739656in}}%
\pgfpathlineto{\pgfqpoint{3.737862in}{0.739656in}}%
\pgfpathlineto{\pgfqpoint{3.737566in}{0.739656in}}%
\pgfpathlineto{\pgfqpoint{3.737270in}{0.739656in}}%
\pgfpathlineto{\pgfqpoint{3.736974in}{0.739656in}}%
\pgfpathlineto{\pgfqpoint{3.736678in}{0.739656in}}%
\pgfpathlineto{\pgfqpoint{3.736382in}{0.739656in}}%
\pgfpathlineto{\pgfqpoint{3.736086in}{0.739656in}}%
\pgfpathlineto{\pgfqpoint{3.735790in}{0.739656in}}%
\pgfpathlineto{\pgfqpoint{3.735494in}{0.739656in}}%
\pgfpathlineto{\pgfqpoint{3.735198in}{0.739656in}}%
\pgfpathlineto{\pgfqpoint{3.734902in}{0.739656in}}%
\pgfpathlineto{\pgfqpoint{3.734606in}{0.739656in}}%
\pgfpathlineto{\pgfqpoint{3.734310in}{0.739656in}}%
\pgfpathlineto{\pgfqpoint{3.734014in}{0.739656in}}%
\pgfpathlineto{\pgfqpoint{3.733718in}{0.739656in}}%
\pgfpathlineto{\pgfqpoint{3.733422in}{0.739656in}}%
\pgfpathlineto{\pgfqpoint{3.733126in}{0.739656in}}%
\pgfpathlineto{\pgfqpoint{3.732830in}{0.739656in}}%
\pgfpathlineto{\pgfqpoint{3.732534in}{0.739656in}}%
\pgfpathlineto{\pgfqpoint{3.732238in}{0.739656in}}%
\pgfpathlineto{\pgfqpoint{3.731942in}{0.739656in}}%
\pgfpathlineto{\pgfqpoint{3.731646in}{0.739656in}}%
\pgfpathlineto{\pgfqpoint{3.731350in}{0.739656in}}%
\pgfpathlineto{\pgfqpoint{3.731054in}{0.739656in}}%
\pgfpathlineto{\pgfqpoint{3.730758in}{0.739656in}}%
\pgfpathlineto{\pgfqpoint{3.730462in}{0.739656in}}%
\pgfpathlineto{\pgfqpoint{3.730166in}{0.739656in}}%
\pgfpathlineto{\pgfqpoint{3.729870in}{0.739656in}}%
\pgfpathlineto{\pgfqpoint{3.729574in}{0.739656in}}%
\pgfpathlineto{\pgfqpoint{3.729278in}{0.739656in}}%
\pgfpathlineto{\pgfqpoint{3.728982in}{0.739656in}}%
\pgfpathlineto{\pgfqpoint{3.728686in}{0.739656in}}%
\pgfpathlineto{\pgfqpoint{3.728390in}{0.739656in}}%
\pgfpathlineto{\pgfqpoint{3.728094in}{0.739656in}}%
\pgfpathlineto{\pgfqpoint{3.727798in}{0.739656in}}%
\pgfpathlineto{\pgfqpoint{3.727502in}{0.739656in}}%
\pgfpathlineto{\pgfqpoint{3.727206in}{0.739656in}}%
\pgfpathlineto{\pgfqpoint{3.726910in}{0.739656in}}%
\pgfpathlineto{\pgfqpoint{3.726614in}{0.739656in}}%
\pgfpathlineto{\pgfqpoint{3.726318in}{0.739656in}}%
\pgfpathlineto{\pgfqpoint{3.726022in}{0.739656in}}%
\pgfpathlineto{\pgfqpoint{3.725726in}{0.739656in}}%
\pgfpathlineto{\pgfqpoint{3.725430in}{0.739656in}}%
\pgfpathlineto{\pgfqpoint{3.725134in}{0.739656in}}%
\pgfpathlineto{\pgfqpoint{3.724838in}{0.739656in}}%
\pgfpathlineto{\pgfqpoint{3.724542in}{0.739656in}}%
\pgfpathlineto{\pgfqpoint{3.724246in}{0.739656in}}%
\pgfpathlineto{\pgfqpoint{3.723950in}{0.739656in}}%
\pgfpathlineto{\pgfqpoint{3.723654in}{0.739656in}}%
\pgfpathlineto{\pgfqpoint{3.723358in}{0.739656in}}%
\pgfpathlineto{\pgfqpoint{3.723062in}{0.739656in}}%
\pgfpathlineto{\pgfqpoint{3.722766in}{0.739656in}}%
\pgfpathlineto{\pgfqpoint{3.722470in}{0.739656in}}%
\pgfpathlineto{\pgfqpoint{3.722174in}{0.739656in}}%
\pgfpathlineto{\pgfqpoint{3.721878in}{0.739656in}}%
\pgfpathlineto{\pgfqpoint{3.721582in}{0.739656in}}%
\pgfpathlineto{\pgfqpoint{3.721286in}{0.739656in}}%
\pgfpathlineto{\pgfqpoint{3.720990in}{0.739656in}}%
\pgfpathlineto{\pgfqpoint{3.720694in}{0.739656in}}%
\pgfpathlineto{\pgfqpoint{3.720398in}{0.739656in}}%
\pgfpathlineto{\pgfqpoint{3.720102in}{0.739656in}}%
\pgfpathlineto{\pgfqpoint{3.719806in}{0.739656in}}%
\pgfpathlineto{\pgfqpoint{3.719510in}{0.739656in}}%
\pgfpathlineto{\pgfqpoint{3.719214in}{0.739656in}}%
\pgfpathlineto{\pgfqpoint{3.718918in}{0.739656in}}%
\pgfpathlineto{\pgfqpoint{3.718622in}{0.739656in}}%
\pgfpathlineto{\pgfqpoint{3.718326in}{0.739656in}}%
\pgfpathlineto{\pgfqpoint{3.718030in}{0.739656in}}%
\pgfpathlineto{\pgfqpoint{3.717734in}{0.739656in}}%
\pgfpathlineto{\pgfqpoint{3.717438in}{0.739656in}}%
\pgfpathlineto{\pgfqpoint{3.717142in}{0.739656in}}%
\pgfpathlineto{\pgfqpoint{3.716846in}{0.739656in}}%
\pgfpathlineto{\pgfqpoint{3.716550in}{0.739656in}}%
\pgfpathlineto{\pgfqpoint{3.716254in}{0.739656in}}%
\pgfpathlineto{\pgfqpoint{3.715958in}{0.739656in}}%
\pgfpathlineto{\pgfqpoint{3.715662in}{0.739656in}}%
\pgfpathlineto{\pgfqpoint{3.715366in}{0.739656in}}%
\pgfpathlineto{\pgfqpoint{3.715070in}{0.739656in}}%
\pgfpathlineto{\pgfqpoint{3.714774in}{0.739656in}}%
\pgfpathlineto{\pgfqpoint{3.714478in}{0.739656in}}%
\pgfpathlineto{\pgfqpoint{3.714182in}{0.739656in}}%
\pgfpathlineto{\pgfqpoint{3.713886in}{0.739656in}}%
\pgfpathlineto{\pgfqpoint{3.713590in}{0.739656in}}%
\pgfpathlineto{\pgfqpoint{3.713294in}{0.739656in}}%
\pgfpathlineto{\pgfqpoint{3.712998in}{0.739656in}}%
\pgfpathlineto{\pgfqpoint{3.712702in}{0.739656in}}%
\pgfpathlineto{\pgfqpoint{3.712406in}{0.739656in}}%
\pgfpathlineto{\pgfqpoint{3.712110in}{0.739656in}}%
\pgfpathlineto{\pgfqpoint{3.711814in}{0.739656in}}%
\pgfpathlineto{\pgfqpoint{3.711518in}{0.739656in}}%
\pgfpathlineto{\pgfqpoint{3.711222in}{0.739656in}}%
\pgfpathlineto{\pgfqpoint{3.710926in}{0.739656in}}%
\pgfpathlineto{\pgfqpoint{3.710630in}{0.739656in}}%
\pgfpathlineto{\pgfqpoint{3.710334in}{0.739656in}}%
\pgfpathlineto{\pgfqpoint{3.710038in}{0.739656in}}%
\pgfpathlineto{\pgfqpoint{3.709742in}{0.739656in}}%
\pgfpathlineto{\pgfqpoint{3.709446in}{0.739656in}}%
\pgfpathlineto{\pgfqpoint{3.709150in}{0.739656in}}%
\pgfpathlineto{\pgfqpoint{3.708854in}{0.739656in}}%
\pgfpathlineto{\pgfqpoint{3.708558in}{0.739656in}}%
\pgfpathlineto{\pgfqpoint{3.708262in}{0.739656in}}%
\pgfpathlineto{\pgfqpoint{3.707966in}{0.739656in}}%
\pgfpathlineto{\pgfqpoint{3.707670in}{0.739656in}}%
\pgfpathlineto{\pgfqpoint{3.707374in}{0.739656in}}%
\pgfpathlineto{\pgfqpoint{3.707078in}{0.739656in}}%
\pgfpathlineto{\pgfqpoint{3.706782in}{0.739656in}}%
\pgfpathlineto{\pgfqpoint{3.706486in}{0.739656in}}%
\pgfpathlineto{\pgfqpoint{3.706190in}{0.739656in}}%
\pgfpathlineto{\pgfqpoint{3.705894in}{0.739656in}}%
\pgfpathlineto{\pgfqpoint{3.705598in}{0.739656in}}%
\pgfpathlineto{\pgfqpoint{3.705302in}{0.739656in}}%
\pgfpathlineto{\pgfqpoint{3.705006in}{0.739656in}}%
\pgfpathlineto{\pgfqpoint{3.704710in}{0.739656in}}%
\pgfpathlineto{\pgfqpoint{3.704414in}{0.739656in}}%
\pgfpathlineto{\pgfqpoint{3.704118in}{0.739656in}}%
\pgfpathlineto{\pgfqpoint{3.703822in}{0.739656in}}%
\pgfpathlineto{\pgfqpoint{3.703526in}{0.739656in}}%
\pgfpathlineto{\pgfqpoint{3.703230in}{0.739656in}}%
\pgfpathlineto{\pgfqpoint{3.702934in}{0.739656in}}%
\pgfpathlineto{\pgfqpoint{3.702638in}{0.739656in}}%
\pgfpathlineto{\pgfqpoint{3.702342in}{0.739656in}}%
\pgfpathlineto{\pgfqpoint{3.702046in}{0.739656in}}%
\pgfpathlineto{\pgfqpoint{3.701750in}{0.739656in}}%
\pgfpathlineto{\pgfqpoint{3.701454in}{0.739656in}}%
\pgfpathlineto{\pgfqpoint{3.701158in}{0.739656in}}%
\pgfpathlineto{\pgfqpoint{3.700862in}{0.739656in}}%
\pgfpathlineto{\pgfqpoint{3.700566in}{0.739656in}}%
\pgfpathlineto{\pgfqpoint{3.700270in}{0.739656in}}%
\pgfpathlineto{\pgfqpoint{3.699974in}{0.739656in}}%
\pgfpathlineto{\pgfqpoint{3.699678in}{0.739656in}}%
\pgfpathlineto{\pgfqpoint{3.699382in}{0.739656in}}%
\pgfpathlineto{\pgfqpoint{3.699086in}{0.739656in}}%
\pgfpathlineto{\pgfqpoint{3.698790in}{0.739656in}}%
\pgfpathlineto{\pgfqpoint{3.698494in}{0.739656in}}%
\pgfpathlineto{\pgfqpoint{3.698198in}{0.739656in}}%
\pgfpathlineto{\pgfqpoint{3.697902in}{0.739656in}}%
\pgfpathlineto{\pgfqpoint{3.697606in}{0.739656in}}%
\pgfpathlineto{\pgfqpoint{3.697310in}{0.739656in}}%
\pgfpathlineto{\pgfqpoint{3.697014in}{0.739656in}}%
\pgfpathlineto{\pgfqpoint{3.696718in}{0.739656in}}%
\pgfpathlineto{\pgfqpoint{3.696422in}{0.739656in}}%
\pgfpathlineto{\pgfqpoint{3.696126in}{0.739656in}}%
\pgfpathlineto{\pgfqpoint{3.695830in}{0.739656in}}%
\pgfpathlineto{\pgfqpoint{3.695534in}{0.739656in}}%
\pgfpathlineto{\pgfqpoint{3.695238in}{0.739656in}}%
\pgfpathlineto{\pgfqpoint{3.694942in}{0.739656in}}%
\pgfpathlineto{\pgfqpoint{3.694646in}{0.739656in}}%
\pgfpathlineto{\pgfqpoint{3.694350in}{0.739656in}}%
\pgfpathlineto{\pgfqpoint{3.694054in}{0.739656in}}%
\pgfpathlineto{\pgfqpoint{3.693758in}{0.739656in}}%
\pgfpathlineto{\pgfqpoint{3.693462in}{0.739656in}}%
\pgfpathlineto{\pgfqpoint{3.693166in}{0.739656in}}%
\pgfpathlineto{\pgfqpoint{3.692870in}{0.739656in}}%
\pgfpathlineto{\pgfqpoint{3.692574in}{0.739656in}}%
\pgfpathlineto{\pgfqpoint{3.692278in}{0.739656in}}%
\pgfpathlineto{\pgfqpoint{3.691982in}{0.739656in}}%
\pgfpathlineto{\pgfqpoint{3.691686in}{0.739656in}}%
\pgfpathlineto{\pgfqpoint{3.691390in}{0.739656in}}%
\pgfpathlineto{\pgfqpoint{3.691094in}{0.739656in}}%
\pgfpathlineto{\pgfqpoint{3.690798in}{0.739656in}}%
\pgfpathlineto{\pgfqpoint{3.690502in}{0.739656in}}%
\pgfpathlineto{\pgfqpoint{3.690206in}{0.739656in}}%
\pgfpathlineto{\pgfqpoint{3.689910in}{0.739656in}}%
\pgfpathlineto{\pgfqpoint{3.689614in}{0.739656in}}%
\pgfpathlineto{\pgfqpoint{3.689318in}{0.739656in}}%
\pgfpathlineto{\pgfqpoint{3.689022in}{0.739656in}}%
\pgfpathlineto{\pgfqpoint{3.688726in}{0.739656in}}%
\pgfpathlineto{\pgfqpoint{3.688430in}{0.739656in}}%
\pgfpathlineto{\pgfqpoint{3.688134in}{0.739656in}}%
\pgfpathlineto{\pgfqpoint{3.687838in}{0.739656in}}%
\pgfpathlineto{\pgfqpoint{3.687542in}{0.739656in}}%
\pgfpathlineto{\pgfqpoint{3.687246in}{0.739656in}}%
\pgfpathlineto{\pgfqpoint{3.686950in}{0.739656in}}%
\pgfpathlineto{\pgfqpoint{3.686654in}{0.739656in}}%
\pgfpathlineto{\pgfqpoint{3.686358in}{0.739656in}}%
\pgfpathlineto{\pgfqpoint{3.686062in}{0.739656in}}%
\pgfpathlineto{\pgfqpoint{3.685766in}{0.739656in}}%
\pgfpathlineto{\pgfqpoint{3.685470in}{0.739656in}}%
\pgfpathlineto{\pgfqpoint{3.685174in}{0.739656in}}%
\pgfpathlineto{\pgfqpoint{3.684878in}{0.739656in}}%
\pgfpathlineto{\pgfqpoint{3.684582in}{0.739656in}}%
\pgfpathlineto{\pgfqpoint{3.684286in}{0.739656in}}%
\pgfpathlineto{\pgfqpoint{3.683990in}{0.739656in}}%
\pgfpathlineto{\pgfqpoint{3.683694in}{0.739656in}}%
\pgfpathlineto{\pgfqpoint{3.683398in}{0.739656in}}%
\pgfpathlineto{\pgfqpoint{3.683102in}{0.739656in}}%
\pgfpathlineto{\pgfqpoint{3.682806in}{0.739656in}}%
\pgfpathlineto{\pgfqpoint{3.682510in}{0.739656in}}%
\pgfpathlineto{\pgfqpoint{3.682214in}{0.739656in}}%
\pgfpathlineto{\pgfqpoint{3.681918in}{0.739656in}}%
\pgfpathlineto{\pgfqpoint{3.681621in}{0.739656in}}%
\pgfpathlineto{\pgfqpoint{3.681325in}{0.739656in}}%
\pgfpathlineto{\pgfqpoint{3.681029in}{0.739656in}}%
\pgfpathlineto{\pgfqpoint{3.680733in}{0.739656in}}%
\pgfpathlineto{\pgfqpoint{3.680437in}{0.739656in}}%
\pgfpathlineto{\pgfqpoint{3.680141in}{0.739656in}}%
\pgfpathlineto{\pgfqpoint{3.679845in}{0.739656in}}%
\pgfpathlineto{\pgfqpoint{3.679549in}{0.739656in}}%
\pgfpathlineto{\pgfqpoint{3.679253in}{0.739656in}}%
\pgfpathlineto{\pgfqpoint{3.678957in}{0.739656in}}%
\pgfpathlineto{\pgfqpoint{3.678661in}{0.739656in}}%
\pgfpathlineto{\pgfqpoint{3.678365in}{0.739656in}}%
\pgfpathlineto{\pgfqpoint{3.678069in}{0.739656in}}%
\pgfpathlineto{\pgfqpoint{3.677773in}{0.739656in}}%
\pgfpathlineto{\pgfqpoint{3.677477in}{0.739656in}}%
\pgfpathlineto{\pgfqpoint{3.677181in}{0.739656in}}%
\pgfpathlineto{\pgfqpoint{3.676885in}{0.739656in}}%
\pgfpathlineto{\pgfqpoint{3.676589in}{0.739656in}}%
\pgfpathlineto{\pgfqpoint{3.676293in}{0.739656in}}%
\pgfpathlineto{\pgfqpoint{3.675997in}{0.739656in}}%
\pgfpathlineto{\pgfqpoint{3.675701in}{0.739656in}}%
\pgfpathlineto{\pgfqpoint{3.675405in}{0.739656in}}%
\pgfpathlineto{\pgfqpoint{3.675109in}{0.739656in}}%
\pgfpathlineto{\pgfqpoint{3.674813in}{0.739656in}}%
\pgfpathlineto{\pgfqpoint{3.674517in}{0.739656in}}%
\pgfpathlineto{\pgfqpoint{3.674221in}{0.739656in}}%
\pgfpathlineto{\pgfqpoint{3.673925in}{0.739656in}}%
\pgfpathlineto{\pgfqpoint{3.673629in}{0.739656in}}%
\pgfpathlineto{\pgfqpoint{3.673333in}{0.739656in}}%
\pgfpathlineto{\pgfqpoint{3.673037in}{0.739656in}}%
\pgfpathlineto{\pgfqpoint{3.672741in}{0.739656in}}%
\pgfpathlineto{\pgfqpoint{3.672445in}{0.739656in}}%
\pgfpathlineto{\pgfqpoint{3.672149in}{0.739656in}}%
\pgfpathlineto{\pgfqpoint{3.671853in}{0.739656in}}%
\pgfpathlineto{\pgfqpoint{3.671557in}{0.739656in}}%
\pgfpathlineto{\pgfqpoint{3.671261in}{0.739656in}}%
\pgfpathlineto{\pgfqpoint{3.670965in}{0.739656in}}%
\pgfpathlineto{\pgfqpoint{3.670669in}{0.739656in}}%
\pgfpathlineto{\pgfqpoint{3.670373in}{0.739656in}}%
\pgfpathlineto{\pgfqpoint{3.670077in}{0.739656in}}%
\pgfpathlineto{\pgfqpoint{3.669781in}{0.739656in}}%
\pgfpathlineto{\pgfqpoint{3.669485in}{0.739656in}}%
\pgfpathlineto{\pgfqpoint{3.669189in}{0.739656in}}%
\pgfpathlineto{\pgfqpoint{3.668893in}{0.739656in}}%
\pgfpathlineto{\pgfqpoint{3.668597in}{0.739656in}}%
\pgfpathlineto{\pgfqpoint{3.668301in}{0.739656in}}%
\pgfpathlineto{\pgfqpoint{3.668005in}{0.739656in}}%
\pgfpathlineto{\pgfqpoint{3.667709in}{0.739656in}}%
\pgfpathlineto{\pgfqpoint{3.667413in}{0.739656in}}%
\pgfpathlineto{\pgfqpoint{3.667117in}{0.739656in}}%
\pgfpathlineto{\pgfqpoint{3.666821in}{0.739656in}}%
\pgfpathlineto{\pgfqpoint{3.666525in}{0.739656in}}%
\pgfpathlineto{\pgfqpoint{3.666229in}{0.739656in}}%
\pgfpathlineto{\pgfqpoint{3.665933in}{0.739656in}}%
\pgfpathlineto{\pgfqpoint{3.665637in}{0.739656in}}%
\pgfpathlineto{\pgfqpoint{3.665341in}{0.739656in}}%
\pgfpathlineto{\pgfqpoint{3.665045in}{0.739656in}}%
\pgfpathlineto{\pgfqpoint{3.664749in}{0.739656in}}%
\pgfpathlineto{\pgfqpoint{3.664453in}{0.739656in}}%
\pgfpathlineto{\pgfqpoint{3.664157in}{0.739656in}}%
\pgfpathlineto{\pgfqpoint{3.663861in}{0.739656in}}%
\pgfpathlineto{\pgfqpoint{3.663565in}{0.739656in}}%
\pgfpathlineto{\pgfqpoint{3.663269in}{0.739656in}}%
\pgfpathlineto{\pgfqpoint{3.662973in}{0.739656in}}%
\pgfpathlineto{\pgfqpoint{3.662677in}{0.739656in}}%
\pgfpathlineto{\pgfqpoint{3.662381in}{0.739656in}}%
\pgfpathlineto{\pgfqpoint{3.662085in}{0.739656in}}%
\pgfpathlineto{\pgfqpoint{3.661789in}{0.739656in}}%
\pgfpathlineto{\pgfqpoint{3.661493in}{0.739656in}}%
\pgfpathlineto{\pgfqpoint{3.661197in}{0.739656in}}%
\pgfpathlineto{\pgfqpoint{3.660901in}{0.739656in}}%
\pgfpathlineto{\pgfqpoint{3.660605in}{0.739656in}}%
\pgfpathlineto{\pgfqpoint{3.660309in}{0.739656in}}%
\pgfpathlineto{\pgfqpoint{3.660013in}{0.739656in}}%
\pgfpathlineto{\pgfqpoint{3.659717in}{0.739656in}}%
\pgfpathlineto{\pgfqpoint{3.659421in}{0.739656in}}%
\pgfpathlineto{\pgfqpoint{3.659125in}{0.739656in}}%
\pgfpathlineto{\pgfqpoint{3.658829in}{0.739656in}}%
\pgfpathlineto{\pgfqpoint{3.658533in}{0.739656in}}%
\pgfpathlineto{\pgfqpoint{3.658237in}{0.739656in}}%
\pgfpathlineto{\pgfqpoint{3.657941in}{0.739656in}}%
\pgfpathlineto{\pgfqpoint{3.657645in}{0.739656in}}%
\pgfpathlineto{\pgfqpoint{3.657349in}{0.739656in}}%
\pgfpathlineto{\pgfqpoint{3.657053in}{0.739656in}}%
\pgfpathlineto{\pgfqpoint{3.656757in}{0.739656in}}%
\pgfpathlineto{\pgfqpoint{3.656461in}{0.739656in}}%
\pgfpathlineto{\pgfqpoint{3.656165in}{0.739656in}}%
\pgfpathlineto{\pgfqpoint{3.655869in}{0.739656in}}%
\pgfpathlineto{\pgfqpoint{3.655573in}{0.739656in}}%
\pgfpathlineto{\pgfqpoint{3.655277in}{0.739656in}}%
\pgfpathlineto{\pgfqpoint{3.654981in}{0.739656in}}%
\pgfpathlineto{\pgfqpoint{3.654685in}{0.739656in}}%
\pgfpathlineto{\pgfqpoint{3.654389in}{0.739656in}}%
\pgfpathlineto{\pgfqpoint{3.654093in}{0.739656in}}%
\pgfpathlineto{\pgfqpoint{3.653797in}{0.739656in}}%
\pgfpathlineto{\pgfqpoint{3.653501in}{0.739656in}}%
\pgfpathlineto{\pgfqpoint{3.653205in}{0.739656in}}%
\pgfpathlineto{\pgfqpoint{3.652909in}{0.739656in}}%
\pgfpathlineto{\pgfqpoint{3.652613in}{0.739656in}}%
\pgfpathlineto{\pgfqpoint{3.652317in}{0.739656in}}%
\pgfpathlineto{\pgfqpoint{3.652021in}{0.739656in}}%
\pgfpathlineto{\pgfqpoint{3.651725in}{0.739656in}}%
\pgfpathlineto{\pgfqpoint{3.651429in}{0.739656in}}%
\pgfpathlineto{\pgfqpoint{3.651133in}{0.739656in}}%
\pgfpathlineto{\pgfqpoint{3.650837in}{0.739656in}}%
\pgfpathlineto{\pgfqpoint{3.650541in}{0.739656in}}%
\pgfpathlineto{\pgfqpoint{3.650245in}{0.739656in}}%
\pgfpathlineto{\pgfqpoint{3.649949in}{0.739656in}}%
\pgfpathlineto{\pgfqpoint{3.649653in}{0.739656in}}%
\pgfpathlineto{\pgfqpoint{3.649357in}{0.739656in}}%
\pgfpathlineto{\pgfqpoint{3.649061in}{0.739656in}}%
\pgfpathlineto{\pgfqpoint{3.648765in}{0.739656in}}%
\pgfpathlineto{\pgfqpoint{3.648469in}{0.739656in}}%
\pgfpathlineto{\pgfqpoint{3.648173in}{0.739656in}}%
\pgfpathlineto{\pgfqpoint{3.647877in}{0.739656in}}%
\pgfpathlineto{\pgfqpoint{3.647581in}{0.739656in}}%
\pgfpathlineto{\pgfqpoint{3.647285in}{0.739656in}}%
\pgfpathlineto{\pgfqpoint{3.646989in}{0.739656in}}%
\pgfpathlineto{\pgfqpoint{3.646693in}{0.739656in}}%
\pgfpathlineto{\pgfqpoint{3.646397in}{0.739656in}}%
\pgfpathlineto{\pgfqpoint{3.646101in}{0.739656in}}%
\pgfpathlineto{\pgfqpoint{3.645805in}{0.739656in}}%
\pgfpathlineto{\pgfqpoint{3.645509in}{0.739656in}}%
\pgfpathlineto{\pgfqpoint{3.645213in}{0.739656in}}%
\pgfpathlineto{\pgfqpoint{3.644917in}{0.739656in}}%
\pgfpathlineto{\pgfqpoint{3.644621in}{0.739656in}}%
\pgfpathlineto{\pgfqpoint{3.644325in}{0.739656in}}%
\pgfpathlineto{\pgfqpoint{3.644029in}{0.739656in}}%
\pgfpathlineto{\pgfqpoint{3.643733in}{0.739656in}}%
\pgfpathlineto{\pgfqpoint{3.643437in}{0.739656in}}%
\pgfpathlineto{\pgfqpoint{3.643141in}{0.739656in}}%
\pgfpathlineto{\pgfqpoint{3.642845in}{0.739656in}}%
\pgfpathlineto{\pgfqpoint{3.642549in}{0.739656in}}%
\pgfpathlineto{\pgfqpoint{3.642253in}{0.739656in}}%
\pgfpathlineto{\pgfqpoint{3.641957in}{0.739656in}}%
\pgfpathlineto{\pgfqpoint{3.641661in}{0.739656in}}%
\pgfpathlineto{\pgfqpoint{3.641365in}{0.739656in}}%
\pgfpathlineto{\pgfqpoint{3.641069in}{0.739656in}}%
\pgfpathlineto{\pgfqpoint{3.640773in}{0.739656in}}%
\pgfpathlineto{\pgfqpoint{3.640477in}{0.739656in}}%
\pgfpathlineto{\pgfqpoint{3.640181in}{0.739656in}}%
\pgfpathlineto{\pgfqpoint{3.639885in}{0.739656in}}%
\pgfpathlineto{\pgfqpoint{3.639589in}{0.739656in}}%
\pgfpathlineto{\pgfqpoint{3.639293in}{0.739656in}}%
\pgfpathlineto{\pgfqpoint{3.638997in}{0.739656in}}%
\pgfpathlineto{\pgfqpoint{3.638701in}{0.739656in}}%
\pgfpathlineto{\pgfqpoint{3.638405in}{0.739656in}}%
\pgfpathlineto{\pgfqpoint{3.638109in}{0.739656in}}%
\pgfpathlineto{\pgfqpoint{3.637813in}{0.739656in}}%
\pgfpathlineto{\pgfqpoint{3.637517in}{0.739656in}}%
\pgfpathlineto{\pgfqpoint{3.637221in}{0.739656in}}%
\pgfpathlineto{\pgfqpoint{3.636925in}{0.739656in}}%
\pgfpathlineto{\pgfqpoint{3.636629in}{0.739656in}}%
\pgfpathlineto{\pgfqpoint{3.636333in}{0.739656in}}%
\pgfpathlineto{\pgfqpoint{3.636037in}{0.739656in}}%
\pgfpathlineto{\pgfqpoint{3.635741in}{0.739656in}}%
\pgfpathlineto{\pgfqpoint{3.635445in}{0.739656in}}%
\pgfpathlineto{\pgfqpoint{3.635149in}{0.739656in}}%
\pgfpathlineto{\pgfqpoint{3.634853in}{0.739656in}}%
\pgfpathlineto{\pgfqpoint{3.634557in}{0.739656in}}%
\pgfpathlineto{\pgfqpoint{3.634261in}{0.739656in}}%
\pgfpathlineto{\pgfqpoint{3.633965in}{0.739656in}}%
\pgfpathlineto{\pgfqpoint{3.633669in}{0.739656in}}%
\pgfpathlineto{\pgfqpoint{3.633373in}{0.739656in}}%
\pgfpathlineto{\pgfqpoint{3.633077in}{0.739656in}}%
\pgfpathlineto{\pgfqpoint{3.632781in}{0.739656in}}%
\pgfpathlineto{\pgfqpoint{3.632485in}{0.739656in}}%
\pgfpathlineto{\pgfqpoint{3.632189in}{0.739656in}}%
\pgfpathlineto{\pgfqpoint{3.631893in}{0.739656in}}%
\pgfpathlineto{\pgfqpoint{3.631597in}{0.739656in}}%
\pgfpathlineto{\pgfqpoint{3.631301in}{0.739656in}}%
\pgfpathlineto{\pgfqpoint{3.631005in}{0.739656in}}%
\pgfpathlineto{\pgfqpoint{3.630709in}{0.739656in}}%
\pgfpathlineto{\pgfqpoint{3.630413in}{0.739656in}}%
\pgfpathlineto{\pgfqpoint{3.630117in}{0.739656in}}%
\pgfpathlineto{\pgfqpoint{3.629821in}{0.739656in}}%
\pgfpathlineto{\pgfqpoint{3.629525in}{0.739656in}}%
\pgfpathlineto{\pgfqpoint{3.629229in}{0.739656in}}%
\pgfpathlineto{\pgfqpoint{3.628933in}{0.739656in}}%
\pgfpathlineto{\pgfqpoint{3.628637in}{0.739656in}}%
\pgfpathlineto{\pgfqpoint{3.628341in}{0.739656in}}%
\pgfpathlineto{\pgfqpoint{3.628045in}{0.739656in}}%
\pgfpathlineto{\pgfqpoint{3.627749in}{0.739656in}}%
\pgfpathlineto{\pgfqpoint{3.627453in}{0.739656in}}%
\pgfpathlineto{\pgfqpoint{3.627157in}{0.739656in}}%
\pgfpathlineto{\pgfqpoint{3.626861in}{0.739656in}}%
\pgfpathlineto{\pgfqpoint{3.626565in}{0.739656in}}%
\pgfpathlineto{\pgfqpoint{3.626269in}{0.739656in}}%
\pgfpathlineto{\pgfqpoint{3.625973in}{0.739656in}}%
\pgfpathlineto{\pgfqpoint{3.625677in}{0.739656in}}%
\pgfpathlineto{\pgfqpoint{3.625381in}{0.739656in}}%
\pgfpathlineto{\pgfqpoint{3.625085in}{0.739656in}}%
\pgfpathlineto{\pgfqpoint{3.624789in}{0.739656in}}%
\pgfpathlineto{\pgfqpoint{3.624493in}{0.739656in}}%
\pgfpathlineto{\pgfqpoint{3.624197in}{0.739656in}}%
\pgfpathlineto{\pgfqpoint{3.623901in}{0.739656in}}%
\pgfpathlineto{\pgfqpoint{3.623605in}{0.739656in}}%
\pgfpathlineto{\pgfqpoint{3.623309in}{0.739656in}}%
\pgfpathlineto{\pgfqpoint{3.623013in}{0.739656in}}%
\pgfpathlineto{\pgfqpoint{3.622717in}{0.739656in}}%
\pgfpathlineto{\pgfqpoint{3.622421in}{0.739656in}}%
\pgfpathlineto{\pgfqpoint{3.622125in}{0.739656in}}%
\pgfpathlineto{\pgfqpoint{3.621829in}{0.739656in}}%
\pgfpathlineto{\pgfqpoint{3.621533in}{0.739656in}}%
\pgfpathlineto{\pgfqpoint{3.621237in}{0.739656in}}%
\pgfpathlineto{\pgfqpoint{3.620941in}{0.739656in}}%
\pgfpathlineto{\pgfqpoint{3.620645in}{0.739656in}}%
\pgfpathlineto{\pgfqpoint{3.620349in}{0.739656in}}%
\pgfpathlineto{\pgfqpoint{3.620053in}{0.739656in}}%
\pgfpathlineto{\pgfqpoint{3.619757in}{0.739656in}}%
\pgfpathlineto{\pgfqpoint{3.619461in}{0.739656in}}%
\pgfpathlineto{\pgfqpoint{3.619165in}{0.739656in}}%
\pgfpathlineto{\pgfqpoint{3.618869in}{0.739656in}}%
\pgfpathlineto{\pgfqpoint{3.618573in}{0.739656in}}%
\pgfpathlineto{\pgfqpoint{3.618277in}{0.739656in}}%
\pgfpathlineto{\pgfqpoint{3.617981in}{0.739656in}}%
\pgfpathlineto{\pgfqpoint{3.617685in}{0.739656in}}%
\pgfpathlineto{\pgfqpoint{3.617389in}{0.739656in}}%
\pgfpathlineto{\pgfqpoint{3.617093in}{0.739656in}}%
\pgfpathlineto{\pgfqpoint{3.616797in}{0.739656in}}%
\pgfpathlineto{\pgfqpoint{3.616501in}{0.739656in}}%
\pgfpathlineto{\pgfqpoint{3.616205in}{0.739656in}}%
\pgfpathlineto{\pgfqpoint{3.615909in}{0.739656in}}%
\pgfpathlineto{\pgfqpoint{3.615613in}{0.739656in}}%
\pgfpathlineto{\pgfqpoint{3.615317in}{0.739656in}}%
\pgfpathlineto{\pgfqpoint{3.615021in}{0.739656in}}%
\pgfpathlineto{\pgfqpoint{3.614725in}{0.739656in}}%
\pgfpathlineto{\pgfqpoint{3.614429in}{0.739656in}}%
\pgfpathlineto{\pgfqpoint{3.614132in}{0.739656in}}%
\pgfpathlineto{\pgfqpoint{3.613836in}{0.739656in}}%
\pgfpathlineto{\pgfqpoint{3.613540in}{0.739656in}}%
\pgfpathlineto{\pgfqpoint{3.613244in}{0.739656in}}%
\pgfpathlineto{\pgfqpoint{3.612948in}{0.739656in}}%
\pgfpathlineto{\pgfqpoint{3.612652in}{0.739656in}}%
\pgfpathlineto{\pgfqpoint{3.612356in}{0.739656in}}%
\pgfpathlineto{\pgfqpoint{3.612060in}{0.739656in}}%
\pgfpathlineto{\pgfqpoint{3.611764in}{0.739656in}}%
\pgfpathlineto{\pgfqpoint{3.611468in}{0.739656in}}%
\pgfpathlineto{\pgfqpoint{3.611172in}{0.739656in}}%
\pgfpathlineto{\pgfqpoint{3.610876in}{0.739656in}}%
\pgfpathlineto{\pgfqpoint{3.610580in}{0.739656in}}%
\pgfpathlineto{\pgfqpoint{3.610284in}{0.739656in}}%
\pgfpathlineto{\pgfqpoint{3.609988in}{0.739656in}}%
\pgfpathlineto{\pgfqpoint{3.609692in}{0.739656in}}%
\pgfpathlineto{\pgfqpoint{3.609396in}{0.739656in}}%
\pgfpathlineto{\pgfqpoint{3.609100in}{0.739656in}}%
\pgfpathlineto{\pgfqpoint{3.608804in}{0.739656in}}%
\pgfpathlineto{\pgfqpoint{3.608508in}{0.739656in}}%
\pgfpathlineto{\pgfqpoint{3.608212in}{0.739656in}}%
\pgfpathlineto{\pgfqpoint{3.607916in}{0.739656in}}%
\pgfpathlineto{\pgfqpoint{3.607620in}{0.739656in}}%
\pgfpathlineto{\pgfqpoint{3.607324in}{0.739656in}}%
\pgfpathlineto{\pgfqpoint{3.607028in}{0.739656in}}%
\pgfpathlineto{\pgfqpoint{3.606732in}{0.739656in}}%
\pgfpathlineto{\pgfqpoint{3.606436in}{0.739656in}}%
\pgfpathlineto{\pgfqpoint{3.606140in}{0.739656in}}%
\pgfpathlineto{\pgfqpoint{3.605844in}{0.739656in}}%
\pgfpathlineto{\pgfqpoint{3.605548in}{0.739656in}}%
\pgfpathlineto{\pgfqpoint{3.605252in}{0.739656in}}%
\pgfpathlineto{\pgfqpoint{3.604956in}{0.739656in}}%
\pgfpathlineto{\pgfqpoint{3.604660in}{0.739656in}}%
\pgfpathlineto{\pgfqpoint{3.604364in}{0.739656in}}%
\pgfpathlineto{\pgfqpoint{3.604068in}{0.739656in}}%
\pgfpathlineto{\pgfqpoint{3.603772in}{0.739656in}}%
\pgfpathlineto{\pgfqpoint{3.603476in}{0.739656in}}%
\pgfpathlineto{\pgfqpoint{3.603180in}{0.739656in}}%
\pgfpathlineto{\pgfqpoint{3.602884in}{0.739656in}}%
\pgfpathlineto{\pgfqpoint{3.602588in}{0.739656in}}%
\pgfpathlineto{\pgfqpoint{3.602292in}{0.739656in}}%
\pgfpathlineto{\pgfqpoint{3.601996in}{0.739656in}}%
\pgfpathlineto{\pgfqpoint{3.601700in}{0.739656in}}%
\pgfpathlineto{\pgfqpoint{3.601404in}{0.739656in}}%
\pgfpathlineto{\pgfqpoint{3.601108in}{0.739656in}}%
\pgfpathlineto{\pgfqpoint{3.600812in}{0.739656in}}%
\pgfpathlineto{\pgfqpoint{3.600516in}{0.739656in}}%
\pgfpathlineto{\pgfqpoint{3.600220in}{0.739656in}}%
\pgfpathlineto{\pgfqpoint{3.599924in}{0.739656in}}%
\pgfpathlineto{\pgfqpoint{3.599628in}{0.739656in}}%
\pgfpathlineto{\pgfqpoint{3.599332in}{0.739656in}}%
\pgfpathlineto{\pgfqpoint{3.599036in}{0.739656in}}%
\pgfpathlineto{\pgfqpoint{3.598740in}{0.739656in}}%
\pgfpathlineto{\pgfqpoint{3.598444in}{0.739656in}}%
\pgfpathlineto{\pgfqpoint{3.598148in}{0.739656in}}%
\pgfpathlineto{\pgfqpoint{3.597852in}{0.739656in}}%
\pgfpathlineto{\pgfqpoint{3.597556in}{0.739656in}}%
\pgfpathlineto{\pgfqpoint{3.597260in}{0.739656in}}%
\pgfpathlineto{\pgfqpoint{3.596964in}{0.739656in}}%
\pgfpathlineto{\pgfqpoint{3.596668in}{0.739656in}}%
\pgfpathlineto{\pgfqpoint{3.596372in}{0.739656in}}%
\pgfpathlineto{\pgfqpoint{3.596076in}{0.739656in}}%
\pgfpathlineto{\pgfqpoint{3.595780in}{0.739656in}}%
\pgfpathlineto{\pgfqpoint{3.595484in}{0.739656in}}%
\pgfpathlineto{\pgfqpoint{3.595188in}{0.739656in}}%
\pgfpathlineto{\pgfqpoint{3.594892in}{0.739656in}}%
\pgfpathlineto{\pgfqpoint{3.594596in}{0.739656in}}%
\pgfpathlineto{\pgfqpoint{3.594300in}{0.739656in}}%
\pgfpathlineto{\pgfqpoint{3.594004in}{0.739656in}}%
\pgfpathlineto{\pgfqpoint{3.593708in}{0.739656in}}%
\pgfpathlineto{\pgfqpoint{3.593412in}{0.739656in}}%
\pgfpathlineto{\pgfqpoint{3.593116in}{0.739656in}}%
\pgfpathlineto{\pgfqpoint{3.592820in}{0.739656in}}%
\pgfpathlineto{\pgfqpoint{3.592524in}{0.739656in}}%
\pgfpathlineto{\pgfqpoint{3.592228in}{0.739656in}}%
\pgfpathlineto{\pgfqpoint{3.591932in}{0.739656in}}%
\pgfpathlineto{\pgfqpoint{3.591636in}{0.739656in}}%
\pgfpathlineto{\pgfqpoint{3.591340in}{0.739656in}}%
\pgfpathlineto{\pgfqpoint{3.591044in}{0.739656in}}%
\pgfpathlineto{\pgfqpoint{3.590748in}{0.739656in}}%
\pgfpathlineto{\pgfqpoint{3.590452in}{0.739656in}}%
\pgfpathlineto{\pgfqpoint{3.590156in}{0.739656in}}%
\pgfpathlineto{\pgfqpoint{3.589860in}{0.739656in}}%
\pgfpathlineto{\pgfqpoint{3.589564in}{0.739656in}}%
\pgfpathlineto{\pgfqpoint{3.589268in}{0.739656in}}%
\pgfpathlineto{\pgfqpoint{3.588972in}{0.739656in}}%
\pgfpathlineto{\pgfqpoint{3.588676in}{0.739656in}}%
\pgfpathlineto{\pgfqpoint{3.588380in}{0.739656in}}%
\pgfpathlineto{\pgfqpoint{3.588084in}{0.739656in}}%
\pgfpathlineto{\pgfqpoint{3.587788in}{0.739656in}}%
\pgfpathlineto{\pgfqpoint{3.587492in}{0.739656in}}%
\pgfpathlineto{\pgfqpoint{3.587196in}{0.739656in}}%
\pgfpathlineto{\pgfqpoint{3.586900in}{0.739656in}}%
\pgfpathlineto{\pgfqpoint{3.586604in}{0.739656in}}%
\pgfpathlineto{\pgfqpoint{3.586308in}{0.739656in}}%
\pgfpathlineto{\pgfqpoint{3.586012in}{0.739656in}}%
\pgfpathlineto{\pgfqpoint{3.585716in}{0.739656in}}%
\pgfpathlineto{\pgfqpoint{3.585420in}{0.739656in}}%
\pgfpathlineto{\pgfqpoint{3.585124in}{0.739656in}}%
\pgfpathlineto{\pgfqpoint{3.584828in}{0.739656in}}%
\pgfpathlineto{\pgfqpoint{3.584532in}{0.739656in}}%
\pgfpathlineto{\pgfqpoint{3.584236in}{0.739656in}}%
\pgfpathlineto{\pgfqpoint{3.583940in}{0.739656in}}%
\pgfpathlineto{\pgfqpoint{3.583644in}{0.739656in}}%
\pgfpathlineto{\pgfqpoint{3.583348in}{0.739656in}}%
\pgfpathlineto{\pgfqpoint{3.583052in}{0.739656in}}%
\pgfpathlineto{\pgfqpoint{3.582756in}{0.739656in}}%
\pgfpathlineto{\pgfqpoint{3.582460in}{0.739656in}}%
\pgfpathlineto{\pgfqpoint{3.582164in}{0.739656in}}%
\pgfpathlineto{\pgfqpoint{3.581868in}{0.739656in}}%
\pgfpathlineto{\pgfqpoint{3.581572in}{0.739656in}}%
\pgfpathlineto{\pgfqpoint{3.581276in}{0.739656in}}%
\pgfpathlineto{\pgfqpoint{3.580980in}{0.739656in}}%
\pgfpathlineto{\pgfqpoint{3.580684in}{0.739656in}}%
\pgfpathlineto{\pgfqpoint{3.580388in}{0.739656in}}%
\pgfpathlineto{\pgfqpoint{3.580092in}{0.739656in}}%
\pgfpathlineto{\pgfqpoint{3.579796in}{0.739656in}}%
\pgfpathlineto{\pgfqpoint{3.579500in}{0.739656in}}%
\pgfpathlineto{\pgfqpoint{3.579204in}{0.739656in}}%
\pgfpathlineto{\pgfqpoint{3.578908in}{0.739656in}}%
\pgfpathlineto{\pgfqpoint{3.578612in}{0.739656in}}%
\pgfpathlineto{\pgfqpoint{3.578316in}{0.739656in}}%
\pgfpathlineto{\pgfqpoint{3.578020in}{0.739656in}}%
\pgfpathlineto{\pgfqpoint{3.577724in}{0.739656in}}%
\pgfpathlineto{\pgfqpoint{3.577428in}{0.739656in}}%
\pgfpathlineto{\pgfqpoint{3.577132in}{0.739656in}}%
\pgfpathlineto{\pgfqpoint{3.576836in}{0.739656in}}%
\pgfpathlineto{\pgfqpoint{3.576540in}{0.739656in}}%
\pgfpathlineto{\pgfqpoint{3.576244in}{0.739656in}}%
\pgfpathlineto{\pgfqpoint{3.575948in}{0.739656in}}%
\pgfpathlineto{\pgfqpoint{3.575652in}{0.739656in}}%
\pgfpathlineto{\pgfqpoint{3.575356in}{0.739656in}}%
\pgfpathlineto{\pgfqpoint{3.575060in}{0.739656in}}%
\pgfpathlineto{\pgfqpoint{3.574764in}{0.739656in}}%
\pgfpathlineto{\pgfqpoint{3.574468in}{0.739656in}}%
\pgfpathlineto{\pgfqpoint{3.574172in}{0.739656in}}%
\pgfpathlineto{\pgfqpoint{3.573876in}{0.739656in}}%
\pgfpathlineto{\pgfqpoint{3.573580in}{0.739656in}}%
\pgfpathlineto{\pgfqpoint{3.573284in}{0.739656in}}%
\pgfpathlineto{\pgfqpoint{3.572988in}{0.739656in}}%
\pgfpathlineto{\pgfqpoint{3.572692in}{0.739656in}}%
\pgfpathlineto{\pgfqpoint{3.572396in}{0.739656in}}%
\pgfpathlineto{\pgfqpoint{3.572100in}{0.739656in}}%
\pgfpathlineto{\pgfqpoint{3.571804in}{0.739656in}}%
\pgfpathlineto{\pgfqpoint{3.571508in}{0.739656in}}%
\pgfpathlineto{\pgfqpoint{3.571212in}{0.739656in}}%
\pgfpathlineto{\pgfqpoint{3.570916in}{0.739656in}}%
\pgfpathlineto{\pgfqpoint{3.570620in}{0.739656in}}%
\pgfpathlineto{\pgfqpoint{3.570324in}{0.739656in}}%
\pgfpathlineto{\pgfqpoint{3.570028in}{0.739656in}}%
\pgfpathlineto{\pgfqpoint{3.569732in}{0.739656in}}%
\pgfpathlineto{\pgfqpoint{3.569436in}{0.739656in}}%
\pgfpathlineto{\pgfqpoint{3.569140in}{0.739656in}}%
\pgfpathlineto{\pgfqpoint{3.568844in}{0.739656in}}%
\pgfpathlineto{\pgfqpoint{3.568548in}{0.739656in}}%
\pgfpathlineto{\pgfqpoint{3.568252in}{0.739656in}}%
\pgfpathlineto{\pgfqpoint{3.567956in}{0.739656in}}%
\pgfpathlineto{\pgfqpoint{3.567660in}{0.739656in}}%
\pgfpathlineto{\pgfqpoint{3.567364in}{0.739656in}}%
\pgfpathlineto{\pgfqpoint{3.567068in}{0.739656in}}%
\pgfpathlineto{\pgfqpoint{3.566772in}{0.739656in}}%
\pgfpathlineto{\pgfqpoint{3.566476in}{0.739656in}}%
\pgfpathlineto{\pgfqpoint{3.566180in}{0.739656in}}%
\pgfpathlineto{\pgfqpoint{3.565884in}{0.739656in}}%
\pgfpathlineto{\pgfqpoint{3.565588in}{0.739656in}}%
\pgfpathlineto{\pgfqpoint{3.565292in}{0.739656in}}%
\pgfpathlineto{\pgfqpoint{3.564996in}{0.739656in}}%
\pgfpathlineto{\pgfqpoint{3.564700in}{0.739656in}}%
\pgfpathlineto{\pgfqpoint{3.564404in}{0.739656in}}%
\pgfpathlineto{\pgfqpoint{3.564108in}{0.739656in}}%
\pgfpathlineto{\pgfqpoint{3.563812in}{0.739656in}}%
\pgfpathlineto{\pgfqpoint{3.563516in}{0.739656in}}%
\pgfpathlineto{\pgfqpoint{3.563220in}{0.739656in}}%
\pgfpathlineto{\pgfqpoint{3.562924in}{0.739656in}}%
\pgfpathlineto{\pgfqpoint{3.562628in}{0.739656in}}%
\pgfpathlineto{\pgfqpoint{3.562332in}{0.739656in}}%
\pgfpathlineto{\pgfqpoint{3.562036in}{0.739656in}}%
\pgfpathlineto{\pgfqpoint{3.561740in}{0.739656in}}%
\pgfpathlineto{\pgfqpoint{3.561444in}{0.739656in}}%
\pgfpathlineto{\pgfqpoint{3.561148in}{0.739656in}}%
\pgfpathlineto{\pgfqpoint{3.560852in}{0.739656in}}%
\pgfpathlineto{\pgfqpoint{3.560556in}{0.739656in}}%
\pgfpathlineto{\pgfqpoint{3.560260in}{0.739656in}}%
\pgfpathlineto{\pgfqpoint{3.559964in}{0.739656in}}%
\pgfpathlineto{\pgfqpoint{3.559668in}{0.739656in}}%
\pgfpathlineto{\pgfqpoint{3.559372in}{0.739656in}}%
\pgfpathlineto{\pgfqpoint{3.559076in}{0.739656in}}%
\pgfpathlineto{\pgfqpoint{3.558780in}{0.739656in}}%
\pgfpathlineto{\pgfqpoint{3.558484in}{0.739656in}}%
\pgfpathlineto{\pgfqpoint{3.558188in}{0.739656in}}%
\pgfpathlineto{\pgfqpoint{3.557892in}{0.739656in}}%
\pgfpathlineto{\pgfqpoint{3.557596in}{0.739656in}}%
\pgfpathlineto{\pgfqpoint{3.557300in}{0.739656in}}%
\pgfpathlineto{\pgfqpoint{3.557004in}{0.739656in}}%
\pgfpathlineto{\pgfqpoint{3.556708in}{0.739656in}}%
\pgfpathlineto{\pgfqpoint{3.556412in}{0.739656in}}%
\pgfpathlineto{\pgfqpoint{3.556116in}{0.739656in}}%
\pgfpathlineto{\pgfqpoint{3.555820in}{0.739656in}}%
\pgfpathlineto{\pgfqpoint{3.555524in}{0.739656in}}%
\pgfpathlineto{\pgfqpoint{3.555228in}{0.739656in}}%
\pgfpathlineto{\pgfqpoint{3.554932in}{0.739656in}}%
\pgfpathlineto{\pgfqpoint{3.554636in}{0.739656in}}%
\pgfpathlineto{\pgfqpoint{3.554340in}{0.739656in}}%
\pgfpathlineto{\pgfqpoint{3.554044in}{0.739656in}}%
\pgfpathlineto{\pgfqpoint{3.553748in}{0.739656in}}%
\pgfpathlineto{\pgfqpoint{3.553452in}{0.739656in}}%
\pgfpathlineto{\pgfqpoint{3.553156in}{0.739656in}}%
\pgfpathlineto{\pgfqpoint{3.552860in}{0.739656in}}%
\pgfpathlineto{\pgfqpoint{3.552564in}{0.739656in}}%
\pgfpathlineto{\pgfqpoint{3.552268in}{0.739656in}}%
\pgfpathlineto{\pgfqpoint{3.551972in}{0.739656in}}%
\pgfpathlineto{\pgfqpoint{3.551676in}{0.739656in}}%
\pgfpathlineto{\pgfqpoint{3.551380in}{0.739656in}}%
\pgfpathlineto{\pgfqpoint{3.551084in}{0.739656in}}%
\pgfpathlineto{\pgfqpoint{3.550788in}{0.739656in}}%
\pgfpathlineto{\pgfqpoint{3.550492in}{0.739656in}}%
\pgfpathlineto{\pgfqpoint{3.550196in}{0.739656in}}%
\pgfpathlineto{\pgfqpoint{3.549900in}{0.739656in}}%
\pgfpathlineto{\pgfqpoint{3.549604in}{0.739656in}}%
\pgfpathlineto{\pgfqpoint{3.549308in}{0.739656in}}%
\pgfpathlineto{\pgfqpoint{3.549012in}{0.739656in}}%
\pgfpathlineto{\pgfqpoint{3.548716in}{0.739656in}}%
\pgfpathlineto{\pgfqpoint{3.548420in}{0.739656in}}%
\pgfpathlineto{\pgfqpoint{3.548124in}{0.739656in}}%
\pgfpathlineto{\pgfqpoint{3.547828in}{0.739656in}}%
\pgfpathlineto{\pgfqpoint{3.547532in}{0.739656in}}%
\pgfpathlineto{\pgfqpoint{3.547236in}{0.739656in}}%
\pgfpathlineto{\pgfqpoint{3.546940in}{0.739656in}}%
\pgfpathlineto{\pgfqpoint{3.546643in}{0.739656in}}%
\pgfpathlineto{\pgfqpoint{3.546347in}{0.739656in}}%
\pgfpathlineto{\pgfqpoint{3.546051in}{0.739656in}}%
\pgfpathlineto{\pgfqpoint{3.545755in}{0.739656in}}%
\pgfpathlineto{\pgfqpoint{3.545459in}{0.739656in}}%
\pgfpathlineto{\pgfqpoint{3.545163in}{0.739656in}}%
\pgfpathlineto{\pgfqpoint{3.544867in}{0.739656in}}%
\pgfpathlineto{\pgfqpoint{3.544571in}{0.739656in}}%
\pgfpathlineto{\pgfqpoint{3.544275in}{0.739656in}}%
\pgfpathlineto{\pgfqpoint{3.543979in}{0.739656in}}%
\pgfpathlineto{\pgfqpoint{3.543683in}{0.739656in}}%
\pgfpathlineto{\pgfqpoint{3.543387in}{0.739656in}}%
\pgfpathlineto{\pgfqpoint{3.543091in}{0.739656in}}%
\pgfpathlineto{\pgfqpoint{3.542795in}{0.739656in}}%
\pgfpathlineto{\pgfqpoint{3.542499in}{0.739656in}}%
\pgfpathlineto{\pgfqpoint{3.542203in}{0.739656in}}%
\pgfpathlineto{\pgfqpoint{3.541907in}{0.739656in}}%
\pgfpathlineto{\pgfqpoint{3.541611in}{0.739656in}}%
\pgfpathlineto{\pgfqpoint{3.541315in}{0.739656in}}%
\pgfpathlineto{\pgfqpoint{3.541019in}{0.739656in}}%
\pgfpathlineto{\pgfqpoint{3.540723in}{0.739656in}}%
\pgfpathlineto{\pgfqpoint{3.540427in}{0.739656in}}%
\pgfpathlineto{\pgfqpoint{3.540131in}{0.739656in}}%
\pgfpathlineto{\pgfqpoint{3.539835in}{0.739656in}}%
\pgfpathlineto{\pgfqpoint{3.539539in}{0.739656in}}%
\pgfpathlineto{\pgfqpoint{3.539243in}{0.739656in}}%
\pgfpathlineto{\pgfqpoint{3.538947in}{0.739656in}}%
\pgfpathlineto{\pgfqpoint{3.538651in}{0.739656in}}%
\pgfpathlineto{\pgfqpoint{3.538355in}{0.739656in}}%
\pgfpathlineto{\pgfqpoint{3.538059in}{0.739656in}}%
\pgfpathlineto{\pgfqpoint{3.537763in}{0.739656in}}%
\pgfpathlineto{\pgfqpoint{3.537467in}{0.739656in}}%
\pgfpathlineto{\pgfqpoint{3.537171in}{0.739656in}}%
\pgfpathlineto{\pgfqpoint{3.536875in}{0.739656in}}%
\pgfpathlineto{\pgfqpoint{3.536579in}{0.739656in}}%
\pgfpathlineto{\pgfqpoint{3.536283in}{0.739656in}}%
\pgfpathlineto{\pgfqpoint{3.535987in}{0.739656in}}%
\pgfpathlineto{\pgfqpoint{3.535691in}{0.739656in}}%
\pgfpathlineto{\pgfqpoint{3.535395in}{0.739656in}}%
\pgfpathlineto{\pgfqpoint{3.535099in}{0.739656in}}%
\pgfpathlineto{\pgfqpoint{3.534803in}{0.739656in}}%
\pgfpathlineto{\pgfqpoint{3.534507in}{0.739656in}}%
\pgfpathlineto{\pgfqpoint{3.534211in}{0.739656in}}%
\pgfpathlineto{\pgfqpoint{3.533915in}{0.739656in}}%
\pgfpathlineto{\pgfqpoint{3.533619in}{0.739656in}}%
\pgfpathlineto{\pgfqpoint{3.533323in}{0.739656in}}%
\pgfpathlineto{\pgfqpoint{3.533027in}{0.739656in}}%
\pgfpathlineto{\pgfqpoint{3.532731in}{0.739656in}}%
\pgfpathlineto{\pgfqpoint{3.532435in}{0.739656in}}%
\pgfpathlineto{\pgfqpoint{3.532139in}{0.739656in}}%
\pgfpathlineto{\pgfqpoint{3.531843in}{0.739656in}}%
\pgfpathlineto{\pgfqpoint{3.531547in}{0.739656in}}%
\pgfpathlineto{\pgfqpoint{3.531251in}{0.739656in}}%
\pgfpathlineto{\pgfqpoint{3.530955in}{0.739656in}}%
\pgfpathlineto{\pgfqpoint{3.530659in}{0.739656in}}%
\pgfpathlineto{\pgfqpoint{3.530363in}{0.739656in}}%
\pgfpathlineto{\pgfqpoint{3.530067in}{0.739656in}}%
\pgfpathlineto{\pgfqpoint{3.529771in}{0.739656in}}%
\pgfpathlineto{\pgfqpoint{3.529475in}{0.739656in}}%
\pgfpathlineto{\pgfqpoint{3.529179in}{0.739656in}}%
\pgfpathlineto{\pgfqpoint{3.528883in}{0.739656in}}%
\pgfpathlineto{\pgfqpoint{3.528587in}{0.739656in}}%
\pgfpathlineto{\pgfqpoint{3.528291in}{0.739656in}}%
\pgfpathlineto{\pgfqpoint{3.527995in}{0.739656in}}%
\pgfpathlineto{\pgfqpoint{3.527699in}{0.739656in}}%
\pgfpathlineto{\pgfqpoint{3.527403in}{0.739656in}}%
\pgfpathlineto{\pgfqpoint{3.527107in}{0.739656in}}%
\pgfpathlineto{\pgfqpoint{3.526811in}{0.739656in}}%
\pgfpathlineto{\pgfqpoint{3.526515in}{0.739656in}}%
\pgfpathlineto{\pgfqpoint{3.526219in}{0.739656in}}%
\pgfpathlineto{\pgfqpoint{3.525923in}{0.739656in}}%
\pgfpathlineto{\pgfqpoint{3.525627in}{0.739656in}}%
\pgfpathlineto{\pgfqpoint{3.525331in}{0.739656in}}%
\pgfpathlineto{\pgfqpoint{3.525035in}{0.739656in}}%
\pgfpathlineto{\pgfqpoint{3.524739in}{0.739656in}}%
\pgfpathlineto{\pgfqpoint{3.524443in}{0.739656in}}%
\pgfpathlineto{\pgfqpoint{3.524147in}{0.739656in}}%
\pgfpathlineto{\pgfqpoint{3.523851in}{0.739656in}}%
\pgfpathlineto{\pgfqpoint{3.523555in}{0.739656in}}%
\pgfpathlineto{\pgfqpoint{3.523259in}{0.739656in}}%
\pgfpathlineto{\pgfqpoint{3.522963in}{0.739656in}}%
\pgfpathlineto{\pgfqpoint{3.522667in}{0.739656in}}%
\pgfpathlineto{\pgfqpoint{3.522371in}{0.739656in}}%
\pgfpathlineto{\pgfqpoint{3.522075in}{0.739656in}}%
\pgfpathlineto{\pgfqpoint{3.521779in}{0.739656in}}%
\pgfpathlineto{\pgfqpoint{3.521483in}{0.739656in}}%
\pgfpathlineto{\pgfqpoint{3.521187in}{0.739656in}}%
\pgfpathlineto{\pgfqpoint{3.520891in}{0.739656in}}%
\pgfpathlineto{\pgfqpoint{3.520595in}{0.739656in}}%
\pgfpathlineto{\pgfqpoint{3.520299in}{0.739656in}}%
\pgfpathlineto{\pgfqpoint{3.520003in}{0.739656in}}%
\pgfpathlineto{\pgfqpoint{3.519707in}{0.739656in}}%
\pgfpathlineto{\pgfqpoint{3.519411in}{0.739656in}}%
\pgfpathlineto{\pgfqpoint{3.519115in}{0.739656in}}%
\pgfpathlineto{\pgfqpoint{3.518819in}{0.739656in}}%
\pgfpathlineto{\pgfqpoint{3.518523in}{0.739656in}}%
\pgfpathlineto{\pgfqpoint{3.518227in}{0.739656in}}%
\pgfpathlineto{\pgfqpoint{3.517931in}{0.739656in}}%
\pgfpathlineto{\pgfqpoint{3.517635in}{0.739656in}}%
\pgfpathlineto{\pgfqpoint{3.517339in}{0.739656in}}%
\pgfpathlineto{\pgfqpoint{3.517043in}{0.739656in}}%
\pgfpathlineto{\pgfqpoint{3.516747in}{0.739656in}}%
\pgfpathlineto{\pgfqpoint{3.516451in}{0.739656in}}%
\pgfpathlineto{\pgfqpoint{3.516155in}{0.739656in}}%
\pgfpathlineto{\pgfqpoint{3.515859in}{0.739656in}}%
\pgfpathlineto{\pgfqpoint{3.515563in}{0.739656in}}%
\pgfpathlineto{\pgfqpoint{3.515267in}{0.739656in}}%
\pgfpathlineto{\pgfqpoint{3.514971in}{0.739656in}}%
\pgfpathlineto{\pgfqpoint{3.514675in}{0.739656in}}%
\pgfpathlineto{\pgfqpoint{3.514379in}{0.739656in}}%
\pgfpathlineto{\pgfqpoint{3.514083in}{0.739656in}}%
\pgfpathlineto{\pgfqpoint{3.513787in}{0.739656in}}%
\pgfpathlineto{\pgfqpoint{3.513491in}{0.739656in}}%
\pgfpathlineto{\pgfqpoint{3.513195in}{0.739656in}}%
\pgfpathlineto{\pgfqpoint{3.512899in}{0.739656in}}%
\pgfpathlineto{\pgfqpoint{3.512603in}{0.739656in}}%
\pgfpathlineto{\pgfqpoint{3.512307in}{0.739656in}}%
\pgfpathlineto{\pgfqpoint{3.512011in}{0.739656in}}%
\pgfpathlineto{\pgfqpoint{3.511715in}{0.739656in}}%
\pgfpathlineto{\pgfqpoint{3.511419in}{0.739656in}}%
\pgfpathlineto{\pgfqpoint{3.511123in}{0.739656in}}%
\pgfpathlineto{\pgfqpoint{3.510827in}{0.739656in}}%
\pgfpathlineto{\pgfqpoint{3.510531in}{0.739656in}}%
\pgfpathlineto{\pgfqpoint{3.510235in}{0.739656in}}%
\pgfpathlineto{\pgfqpoint{3.509939in}{0.739656in}}%
\pgfpathlineto{\pgfqpoint{3.509643in}{0.739656in}}%
\pgfpathlineto{\pgfqpoint{3.509347in}{0.739656in}}%
\pgfpathlineto{\pgfqpoint{3.509051in}{0.739656in}}%
\pgfpathlineto{\pgfqpoint{3.508755in}{0.739656in}}%
\pgfpathlineto{\pgfqpoint{3.508459in}{0.739656in}}%
\pgfpathlineto{\pgfqpoint{3.508163in}{0.739656in}}%
\pgfpathlineto{\pgfqpoint{3.507867in}{0.739656in}}%
\pgfpathlineto{\pgfqpoint{3.507571in}{0.739656in}}%
\pgfpathlineto{\pgfqpoint{3.507275in}{0.739656in}}%
\pgfpathlineto{\pgfqpoint{3.506979in}{0.739656in}}%
\pgfpathlineto{\pgfqpoint{3.506683in}{0.739656in}}%
\pgfpathlineto{\pgfqpoint{3.506387in}{0.739656in}}%
\pgfpathlineto{\pgfqpoint{3.506091in}{0.739656in}}%
\pgfpathlineto{\pgfqpoint{3.505795in}{0.739656in}}%
\pgfpathlineto{\pgfqpoint{3.505499in}{0.739656in}}%
\pgfpathlineto{\pgfqpoint{3.505203in}{0.739656in}}%
\pgfpathlineto{\pgfqpoint{3.504907in}{0.739656in}}%
\pgfpathlineto{\pgfqpoint{3.504611in}{0.739656in}}%
\pgfpathlineto{\pgfqpoint{3.504315in}{0.739656in}}%
\pgfpathlineto{\pgfqpoint{3.504019in}{0.739656in}}%
\pgfpathlineto{\pgfqpoint{3.503723in}{0.739656in}}%
\pgfpathlineto{\pgfqpoint{3.503427in}{0.739656in}}%
\pgfpathlineto{\pgfqpoint{3.503131in}{0.739656in}}%
\pgfpathlineto{\pgfqpoint{3.502835in}{0.739656in}}%
\pgfpathlineto{\pgfqpoint{3.502539in}{0.739656in}}%
\pgfpathlineto{\pgfqpoint{3.502243in}{0.739656in}}%
\pgfpathlineto{\pgfqpoint{3.501947in}{0.739656in}}%
\pgfpathlineto{\pgfqpoint{3.501651in}{0.739656in}}%
\pgfpathlineto{\pgfqpoint{3.501355in}{0.739656in}}%
\pgfpathlineto{\pgfqpoint{3.501059in}{0.739656in}}%
\pgfpathlineto{\pgfqpoint{3.500763in}{0.739656in}}%
\pgfpathlineto{\pgfqpoint{3.500467in}{0.739656in}}%
\pgfpathlineto{\pgfqpoint{3.500171in}{0.739656in}}%
\pgfpathlineto{\pgfqpoint{3.499875in}{0.739656in}}%
\pgfpathlineto{\pgfqpoint{3.499579in}{0.739656in}}%
\pgfpathlineto{\pgfqpoint{3.499283in}{0.739656in}}%
\pgfpathlineto{\pgfqpoint{3.498987in}{0.739656in}}%
\pgfpathlineto{\pgfqpoint{3.498691in}{0.739656in}}%
\pgfpathlineto{\pgfqpoint{3.498395in}{0.739656in}}%
\pgfpathlineto{\pgfqpoint{3.498099in}{0.739656in}}%
\pgfpathlineto{\pgfqpoint{3.497803in}{0.739656in}}%
\pgfpathlineto{\pgfqpoint{3.497507in}{0.739656in}}%
\pgfpathlineto{\pgfqpoint{3.497211in}{0.739656in}}%
\pgfpathlineto{\pgfqpoint{3.496915in}{0.739656in}}%
\pgfpathlineto{\pgfqpoint{3.496619in}{0.739656in}}%
\pgfpathlineto{\pgfqpoint{3.496323in}{0.739656in}}%
\pgfpathlineto{\pgfqpoint{3.496027in}{0.739656in}}%
\pgfpathlineto{\pgfqpoint{3.495731in}{0.739656in}}%
\pgfpathlineto{\pgfqpoint{3.495435in}{0.739656in}}%
\pgfpathlineto{\pgfqpoint{3.495139in}{0.739656in}}%
\pgfpathlineto{\pgfqpoint{3.494843in}{0.739656in}}%
\pgfpathlineto{\pgfqpoint{3.494547in}{0.739656in}}%
\pgfpathlineto{\pgfqpoint{3.494251in}{0.739656in}}%
\pgfpathlineto{\pgfqpoint{3.493955in}{0.739656in}}%
\pgfpathlineto{\pgfqpoint{3.493659in}{0.739656in}}%
\pgfpathlineto{\pgfqpoint{3.493363in}{0.739656in}}%
\pgfpathlineto{\pgfqpoint{3.493067in}{0.739656in}}%
\pgfpathlineto{\pgfqpoint{3.492771in}{0.739656in}}%
\pgfpathlineto{\pgfqpoint{3.492475in}{0.739656in}}%
\pgfpathlineto{\pgfqpoint{3.492179in}{0.739656in}}%
\pgfpathlineto{\pgfqpoint{3.491883in}{0.739656in}}%
\pgfpathlineto{\pgfqpoint{3.491587in}{0.739656in}}%
\pgfpathlineto{\pgfqpoint{3.491291in}{0.739656in}}%
\pgfpathlineto{\pgfqpoint{3.490995in}{0.739656in}}%
\pgfpathlineto{\pgfqpoint{3.490699in}{0.739656in}}%
\pgfpathlineto{\pgfqpoint{3.490403in}{0.739656in}}%
\pgfpathlineto{\pgfqpoint{3.490107in}{0.739656in}}%
\pgfpathlineto{\pgfqpoint{3.489811in}{0.739656in}}%
\pgfpathlineto{\pgfqpoint{3.489515in}{0.739656in}}%
\pgfpathlineto{\pgfqpoint{3.489219in}{0.739656in}}%
\pgfpathlineto{\pgfqpoint{3.488923in}{0.739656in}}%
\pgfpathlineto{\pgfqpoint{3.488627in}{0.739656in}}%
\pgfpathlineto{\pgfqpoint{3.488331in}{0.739656in}}%
\pgfpathlineto{\pgfqpoint{3.488035in}{0.739656in}}%
\pgfpathlineto{\pgfqpoint{3.487739in}{0.739656in}}%
\pgfpathlineto{\pgfqpoint{3.487443in}{0.739656in}}%
\pgfpathlineto{\pgfqpoint{3.487147in}{0.739656in}}%
\pgfpathlineto{\pgfqpoint{3.486851in}{0.739656in}}%
\pgfpathlineto{\pgfqpoint{3.486555in}{0.739656in}}%
\pgfpathlineto{\pgfqpoint{3.486259in}{0.739656in}}%
\pgfpathlineto{\pgfqpoint{3.485963in}{0.739656in}}%
\pgfpathlineto{\pgfqpoint{3.485667in}{0.739656in}}%
\pgfpathlineto{\pgfqpoint{3.485371in}{0.739656in}}%
\pgfpathlineto{\pgfqpoint{3.485075in}{0.739656in}}%
\pgfpathlineto{\pgfqpoint{3.484779in}{0.739656in}}%
\pgfpathlineto{\pgfqpoint{3.484483in}{0.739656in}}%
\pgfpathlineto{\pgfqpoint{3.484187in}{0.739656in}}%
\pgfpathlineto{\pgfqpoint{3.483891in}{0.739656in}}%
\pgfpathlineto{\pgfqpoint{3.483595in}{0.739656in}}%
\pgfpathlineto{\pgfqpoint{3.483299in}{0.739656in}}%
\pgfpathlineto{\pgfqpoint{3.483003in}{0.739656in}}%
\pgfpathlineto{\pgfqpoint{3.482707in}{0.739656in}}%
\pgfpathlineto{\pgfqpoint{3.482411in}{0.739656in}}%
\pgfpathlineto{\pgfqpoint{3.482115in}{0.739656in}}%
\pgfpathlineto{\pgfqpoint{3.481819in}{0.739656in}}%
\pgfpathlineto{\pgfqpoint{3.481523in}{0.739656in}}%
\pgfpathlineto{\pgfqpoint{3.481227in}{0.739656in}}%
\pgfpathlineto{\pgfqpoint{3.480931in}{0.739656in}}%
\pgfpathlineto{\pgfqpoint{3.480635in}{0.739656in}}%
\pgfpathlineto{\pgfqpoint{3.480339in}{0.739656in}}%
\pgfpathlineto{\pgfqpoint{3.480043in}{0.739656in}}%
\pgfpathlineto{\pgfqpoint{3.479747in}{0.739656in}}%
\pgfpathlineto{\pgfqpoint{3.479451in}{0.739656in}}%
\pgfpathlineto{\pgfqpoint{3.479154in}{0.739656in}}%
\pgfpathlineto{\pgfqpoint{3.478858in}{0.739656in}}%
\pgfpathlineto{\pgfqpoint{3.478562in}{0.739656in}}%
\pgfpathlineto{\pgfqpoint{3.478266in}{0.739656in}}%
\pgfpathlineto{\pgfqpoint{3.477970in}{0.739656in}}%
\pgfpathlineto{\pgfqpoint{3.477674in}{0.739656in}}%
\pgfpathlineto{\pgfqpoint{3.477378in}{0.739656in}}%
\pgfpathlineto{\pgfqpoint{3.477082in}{0.739656in}}%
\pgfpathlineto{\pgfqpoint{3.476786in}{0.739656in}}%
\pgfpathlineto{\pgfqpoint{3.476490in}{0.739656in}}%
\pgfpathlineto{\pgfqpoint{3.476194in}{0.739656in}}%
\pgfpathlineto{\pgfqpoint{3.475898in}{0.739656in}}%
\pgfpathlineto{\pgfqpoint{3.475602in}{0.739656in}}%
\pgfpathlineto{\pgfqpoint{3.475306in}{0.739656in}}%
\pgfpathlineto{\pgfqpoint{3.475010in}{0.739656in}}%
\pgfpathlineto{\pgfqpoint{3.474714in}{0.739656in}}%
\pgfpathlineto{\pgfqpoint{3.474418in}{0.739656in}}%
\pgfpathlineto{\pgfqpoint{3.474122in}{0.739656in}}%
\pgfpathlineto{\pgfqpoint{3.473826in}{0.739656in}}%
\pgfpathlineto{\pgfqpoint{3.473530in}{0.739656in}}%
\pgfpathlineto{\pgfqpoint{3.473234in}{0.739656in}}%
\pgfpathlineto{\pgfqpoint{3.472938in}{0.739656in}}%
\pgfpathlineto{\pgfqpoint{3.472642in}{0.739656in}}%
\pgfpathlineto{\pgfqpoint{3.472346in}{0.739656in}}%
\pgfpathlineto{\pgfqpoint{3.472050in}{0.739656in}}%
\pgfpathlineto{\pgfqpoint{3.471754in}{0.739656in}}%
\pgfpathlineto{\pgfqpoint{3.471458in}{0.739656in}}%
\pgfpathlineto{\pgfqpoint{3.471162in}{0.739656in}}%
\pgfpathlineto{\pgfqpoint{3.470866in}{0.739656in}}%
\pgfpathlineto{\pgfqpoint{3.470570in}{0.739656in}}%
\pgfpathlineto{\pgfqpoint{3.470274in}{0.739656in}}%
\pgfpathlineto{\pgfqpoint{3.469978in}{0.739656in}}%
\pgfpathlineto{\pgfqpoint{3.469682in}{0.739656in}}%
\pgfpathlineto{\pgfqpoint{3.469386in}{0.739656in}}%
\pgfpathlineto{\pgfqpoint{3.469090in}{0.739656in}}%
\pgfpathlineto{\pgfqpoint{3.468794in}{0.739656in}}%
\pgfpathlineto{\pgfqpoint{3.468498in}{0.739656in}}%
\pgfpathlineto{\pgfqpoint{3.468202in}{0.739656in}}%
\pgfpathlineto{\pgfqpoint{3.467906in}{0.739656in}}%
\pgfpathlineto{\pgfqpoint{3.467610in}{0.739656in}}%
\pgfpathlineto{\pgfqpoint{3.467314in}{0.739656in}}%
\pgfpathlineto{\pgfqpoint{3.467018in}{0.739656in}}%
\pgfpathlineto{\pgfqpoint{3.466722in}{0.739656in}}%
\pgfpathlineto{\pgfqpoint{3.466426in}{0.739656in}}%
\pgfpathlineto{\pgfqpoint{3.466130in}{0.739656in}}%
\pgfpathlineto{\pgfqpoint{3.465834in}{0.739656in}}%
\pgfpathlineto{\pgfqpoint{3.465538in}{0.739656in}}%
\pgfpathlineto{\pgfqpoint{3.465242in}{0.739656in}}%
\pgfpathlineto{\pgfqpoint{3.464946in}{0.739656in}}%
\pgfpathlineto{\pgfqpoint{3.464650in}{0.739656in}}%
\pgfpathlineto{\pgfqpoint{3.464354in}{0.739656in}}%
\pgfpathlineto{\pgfqpoint{3.464058in}{0.739656in}}%
\pgfpathlineto{\pgfqpoint{3.463762in}{0.739656in}}%
\pgfpathlineto{\pgfqpoint{3.463466in}{0.739656in}}%
\pgfpathlineto{\pgfqpoint{3.463170in}{0.739656in}}%
\pgfpathlineto{\pgfqpoint{3.462874in}{0.739656in}}%
\pgfpathlineto{\pgfqpoint{3.462578in}{0.739656in}}%
\pgfpathlineto{\pgfqpoint{3.462282in}{0.739656in}}%
\pgfpathlineto{\pgfqpoint{3.461986in}{0.739656in}}%
\pgfpathlineto{\pgfqpoint{3.461690in}{0.739656in}}%
\pgfpathlineto{\pgfqpoint{3.461394in}{0.739656in}}%
\pgfpathlineto{\pgfqpoint{3.461098in}{0.739656in}}%
\pgfpathlineto{\pgfqpoint{3.460802in}{0.739656in}}%
\pgfpathlineto{\pgfqpoint{3.460506in}{0.739656in}}%
\pgfpathlineto{\pgfqpoint{3.460210in}{0.739656in}}%
\pgfpathlineto{\pgfqpoint{3.459914in}{0.739656in}}%
\pgfpathlineto{\pgfqpoint{3.459618in}{0.739656in}}%
\pgfpathlineto{\pgfqpoint{3.459322in}{0.739656in}}%
\pgfpathlineto{\pgfqpoint{3.459026in}{0.739656in}}%
\pgfpathlineto{\pgfqpoint{3.458730in}{0.739656in}}%
\pgfpathlineto{\pgfqpoint{3.458434in}{0.739656in}}%
\pgfpathlineto{\pgfqpoint{3.458138in}{0.739656in}}%
\pgfpathlineto{\pgfqpoint{3.457842in}{0.739656in}}%
\pgfpathlineto{\pgfqpoint{3.457546in}{0.739656in}}%
\pgfpathlineto{\pgfqpoint{3.457250in}{0.739656in}}%
\pgfpathlineto{\pgfqpoint{3.456954in}{0.739656in}}%
\pgfpathlineto{\pgfqpoint{3.456658in}{0.739656in}}%
\pgfpathlineto{\pgfqpoint{3.456362in}{0.739656in}}%
\pgfpathlineto{\pgfqpoint{3.456066in}{0.739656in}}%
\pgfpathlineto{\pgfqpoint{3.455770in}{0.739656in}}%
\pgfpathlineto{\pgfqpoint{3.455474in}{0.739656in}}%
\pgfpathlineto{\pgfqpoint{3.455178in}{0.739656in}}%
\pgfpathlineto{\pgfqpoint{3.454882in}{0.739656in}}%
\pgfpathlineto{\pgfqpoint{3.454586in}{0.739656in}}%
\pgfpathlineto{\pgfqpoint{3.454290in}{0.739656in}}%
\pgfpathlineto{\pgfqpoint{3.453994in}{0.739656in}}%
\pgfpathlineto{\pgfqpoint{3.453698in}{0.739656in}}%
\pgfpathlineto{\pgfqpoint{3.453402in}{0.739656in}}%
\pgfpathlineto{\pgfqpoint{3.453106in}{0.739656in}}%
\pgfpathlineto{\pgfqpoint{3.452810in}{0.739656in}}%
\pgfpathlineto{\pgfqpoint{3.452514in}{0.739656in}}%
\pgfpathlineto{\pgfqpoint{3.452218in}{0.739656in}}%
\pgfpathlineto{\pgfqpoint{3.451922in}{0.739656in}}%
\pgfpathlineto{\pgfqpoint{3.451626in}{0.739656in}}%
\pgfpathlineto{\pgfqpoint{3.451330in}{0.739656in}}%
\pgfpathlineto{\pgfqpoint{3.451034in}{0.739656in}}%
\pgfpathlineto{\pgfqpoint{3.450738in}{0.739656in}}%
\pgfpathlineto{\pgfqpoint{3.450442in}{0.739656in}}%
\pgfpathlineto{\pgfqpoint{3.450146in}{0.739656in}}%
\pgfpathlineto{\pgfqpoint{3.449850in}{0.739656in}}%
\pgfpathlineto{\pgfqpoint{3.449554in}{0.739656in}}%
\pgfpathlineto{\pgfqpoint{3.449258in}{0.739656in}}%
\pgfpathlineto{\pgfqpoint{3.448962in}{0.739656in}}%
\pgfpathlineto{\pgfqpoint{3.448666in}{0.739656in}}%
\pgfpathlineto{\pgfqpoint{3.448370in}{0.739656in}}%
\pgfpathlineto{\pgfqpoint{3.448074in}{0.739656in}}%
\pgfpathlineto{\pgfqpoint{3.447778in}{0.739656in}}%
\pgfpathlineto{\pgfqpoint{3.447482in}{0.739656in}}%
\pgfpathlineto{\pgfqpoint{3.447186in}{0.739656in}}%
\pgfpathlineto{\pgfqpoint{3.446890in}{0.739656in}}%
\pgfpathlineto{\pgfqpoint{3.446594in}{0.739656in}}%
\pgfpathlineto{\pgfqpoint{3.446298in}{0.739656in}}%
\pgfpathlineto{\pgfqpoint{3.446002in}{0.739656in}}%
\pgfpathlineto{\pgfqpoint{3.445706in}{0.739656in}}%
\pgfpathlineto{\pgfqpoint{3.445410in}{0.739656in}}%
\pgfpathlineto{\pgfqpoint{3.445114in}{0.739656in}}%
\pgfpathlineto{\pgfqpoint{3.444818in}{0.739656in}}%
\pgfpathlineto{\pgfqpoint{3.444522in}{0.739656in}}%
\pgfpathlineto{\pgfqpoint{3.444226in}{0.739656in}}%
\pgfpathlineto{\pgfqpoint{3.443930in}{0.739656in}}%
\pgfpathlineto{\pgfqpoint{3.443634in}{0.739656in}}%
\pgfpathlineto{\pgfqpoint{3.443338in}{0.739656in}}%
\pgfpathlineto{\pgfqpoint{3.443042in}{0.739656in}}%
\pgfpathlineto{\pgfqpoint{3.442746in}{0.739656in}}%
\pgfpathlineto{\pgfqpoint{3.442450in}{0.739656in}}%
\pgfpathlineto{\pgfqpoint{3.442154in}{0.739656in}}%
\pgfpathlineto{\pgfqpoint{3.441858in}{0.739656in}}%
\pgfpathlineto{\pgfqpoint{3.441562in}{0.739656in}}%
\pgfpathlineto{\pgfqpoint{3.441266in}{0.739656in}}%
\pgfpathlineto{\pgfqpoint{3.440970in}{0.739656in}}%
\pgfpathlineto{\pgfqpoint{3.440674in}{0.739656in}}%
\pgfpathlineto{\pgfqpoint{3.440378in}{0.739656in}}%
\pgfpathlineto{\pgfqpoint{3.440082in}{0.739656in}}%
\pgfpathlineto{\pgfqpoint{3.439786in}{0.739656in}}%
\pgfpathlineto{\pgfqpoint{3.439490in}{0.739656in}}%
\pgfpathlineto{\pgfqpoint{3.439194in}{0.739656in}}%
\pgfpathlineto{\pgfqpoint{3.438898in}{0.739656in}}%
\pgfpathlineto{\pgfqpoint{3.438602in}{0.739656in}}%
\pgfpathlineto{\pgfqpoint{3.438306in}{0.739656in}}%
\pgfpathlineto{\pgfqpoint{3.438010in}{0.739656in}}%
\pgfpathlineto{\pgfqpoint{3.437714in}{0.739656in}}%
\pgfpathlineto{\pgfqpoint{3.437418in}{0.739656in}}%
\pgfpathlineto{\pgfqpoint{3.437122in}{0.739656in}}%
\pgfpathlineto{\pgfqpoint{3.436826in}{0.739656in}}%
\pgfpathlineto{\pgfqpoint{3.436530in}{0.739656in}}%
\pgfpathlineto{\pgfqpoint{3.436234in}{0.739656in}}%
\pgfpathlineto{\pgfqpoint{3.435938in}{0.739656in}}%
\pgfpathlineto{\pgfqpoint{3.435642in}{0.739656in}}%
\pgfpathlineto{\pgfqpoint{3.435346in}{0.739656in}}%
\pgfpathlineto{\pgfqpoint{3.435050in}{0.739656in}}%
\pgfpathlineto{\pgfqpoint{3.434754in}{0.739656in}}%
\pgfpathlineto{\pgfqpoint{3.434458in}{0.739656in}}%
\pgfpathlineto{\pgfqpoint{3.434162in}{0.739656in}}%
\pgfpathlineto{\pgfqpoint{3.433866in}{0.739656in}}%
\pgfpathlineto{\pgfqpoint{3.433570in}{0.739656in}}%
\pgfpathlineto{\pgfqpoint{3.433274in}{0.739656in}}%
\pgfpathlineto{\pgfqpoint{3.432978in}{0.739656in}}%
\pgfpathlineto{\pgfqpoint{3.432682in}{0.739656in}}%
\pgfpathlineto{\pgfqpoint{3.432386in}{0.739656in}}%
\pgfpathlineto{\pgfqpoint{3.432090in}{0.739656in}}%
\pgfpathlineto{\pgfqpoint{3.431794in}{0.739656in}}%
\pgfpathlineto{\pgfqpoint{3.431498in}{0.739656in}}%
\pgfpathlineto{\pgfqpoint{3.431202in}{0.739656in}}%
\pgfpathlineto{\pgfqpoint{3.430906in}{0.739656in}}%
\pgfpathlineto{\pgfqpoint{3.430610in}{0.739656in}}%
\pgfpathlineto{\pgfqpoint{3.430314in}{0.739656in}}%
\pgfpathlineto{\pgfqpoint{3.430018in}{0.739656in}}%
\pgfpathlineto{\pgfqpoint{3.429722in}{0.739656in}}%
\pgfpathlineto{\pgfqpoint{3.429426in}{0.739656in}}%
\pgfpathlineto{\pgfqpoint{3.429130in}{0.739656in}}%
\pgfpathlineto{\pgfqpoint{3.428834in}{0.739656in}}%
\pgfpathlineto{\pgfqpoint{3.428538in}{0.739656in}}%
\pgfpathlineto{\pgfqpoint{3.428242in}{0.739656in}}%
\pgfpathlineto{\pgfqpoint{3.427946in}{0.739656in}}%
\pgfpathlineto{\pgfqpoint{3.427650in}{0.739656in}}%
\pgfpathlineto{\pgfqpoint{3.427354in}{0.739656in}}%
\pgfpathlineto{\pgfqpoint{3.427058in}{0.739656in}}%
\pgfpathlineto{\pgfqpoint{3.426762in}{0.739656in}}%
\pgfpathlineto{\pgfqpoint{3.426466in}{0.739656in}}%
\pgfpathlineto{\pgfqpoint{3.426170in}{0.739656in}}%
\pgfpathlineto{\pgfqpoint{3.425874in}{0.739656in}}%
\pgfpathlineto{\pgfqpoint{3.425578in}{0.739656in}}%
\pgfpathlineto{\pgfqpoint{3.425282in}{0.739656in}}%
\pgfpathlineto{\pgfqpoint{3.424986in}{0.739656in}}%
\pgfpathlineto{\pgfqpoint{3.424690in}{0.739656in}}%
\pgfpathlineto{\pgfqpoint{3.424394in}{0.739656in}}%
\pgfpathlineto{\pgfqpoint{3.424098in}{0.739656in}}%
\pgfpathlineto{\pgfqpoint{3.423802in}{0.739656in}}%
\pgfpathlineto{\pgfqpoint{3.423506in}{0.739656in}}%
\pgfpathlineto{\pgfqpoint{3.423210in}{0.739656in}}%
\pgfpathlineto{\pgfqpoint{3.422914in}{0.739656in}}%
\pgfpathlineto{\pgfqpoint{3.422618in}{0.739656in}}%
\pgfpathlineto{\pgfqpoint{3.422322in}{0.739656in}}%
\pgfpathlineto{\pgfqpoint{3.422026in}{0.739656in}}%
\pgfpathlineto{\pgfqpoint{3.421730in}{0.739656in}}%
\pgfpathlineto{\pgfqpoint{3.421434in}{0.739656in}}%
\pgfpathlineto{\pgfqpoint{3.421138in}{0.739656in}}%
\pgfpathlineto{\pgfqpoint{3.420842in}{0.739656in}}%
\pgfpathlineto{\pgfqpoint{3.420546in}{0.739656in}}%
\pgfpathlineto{\pgfqpoint{3.420250in}{0.739656in}}%
\pgfpathlineto{\pgfqpoint{3.419954in}{0.739656in}}%
\pgfpathlineto{\pgfqpoint{3.419658in}{0.739656in}}%
\pgfpathlineto{\pgfqpoint{3.419362in}{0.739656in}}%
\pgfpathlineto{\pgfqpoint{3.419066in}{0.739656in}}%
\pgfpathlineto{\pgfqpoint{3.418770in}{0.739656in}}%
\pgfpathlineto{\pgfqpoint{3.418474in}{0.739656in}}%
\pgfpathlineto{\pgfqpoint{3.418178in}{0.739656in}}%
\pgfpathlineto{\pgfqpoint{3.417882in}{0.739656in}}%
\pgfpathlineto{\pgfqpoint{3.417586in}{0.739656in}}%
\pgfpathlineto{\pgfqpoint{3.417290in}{0.739656in}}%
\pgfpathlineto{\pgfqpoint{3.416994in}{0.739656in}}%
\pgfpathlineto{\pgfqpoint{3.416698in}{0.739656in}}%
\pgfpathlineto{\pgfqpoint{3.416402in}{0.739656in}}%
\pgfpathlineto{\pgfqpoint{3.416106in}{0.739656in}}%
\pgfpathlineto{\pgfqpoint{3.415810in}{0.739656in}}%
\pgfpathlineto{\pgfqpoint{3.415514in}{0.739656in}}%
\pgfpathlineto{\pgfqpoint{3.415218in}{0.739656in}}%
\pgfpathlineto{\pgfqpoint{3.414922in}{0.739656in}}%
\pgfpathlineto{\pgfqpoint{3.414626in}{0.739656in}}%
\pgfpathlineto{\pgfqpoint{3.414330in}{0.739656in}}%
\pgfpathlineto{\pgfqpoint{3.414034in}{0.739656in}}%
\pgfpathlineto{\pgfqpoint{3.413738in}{0.739656in}}%
\pgfpathlineto{\pgfqpoint{3.413442in}{0.739656in}}%
\pgfpathlineto{\pgfqpoint{3.413146in}{0.739656in}}%
\pgfpathlineto{\pgfqpoint{3.412850in}{0.739656in}}%
\pgfpathlineto{\pgfqpoint{3.412554in}{0.739656in}}%
\pgfpathlineto{\pgfqpoint{3.412258in}{0.739656in}}%
\pgfpathlineto{\pgfqpoint{3.411961in}{0.739656in}}%
\pgfpathlineto{\pgfqpoint{3.411665in}{0.739656in}}%
\pgfpathlineto{\pgfqpoint{3.411369in}{0.739656in}}%
\pgfpathlineto{\pgfqpoint{3.411073in}{0.739656in}}%
\pgfpathlineto{\pgfqpoint{3.410777in}{0.739656in}}%
\pgfpathlineto{\pgfqpoint{3.410481in}{0.739656in}}%
\pgfpathlineto{\pgfqpoint{3.410185in}{0.739656in}}%
\pgfpathlineto{\pgfqpoint{3.409889in}{0.739656in}}%
\pgfpathlineto{\pgfqpoint{3.409593in}{0.739656in}}%
\pgfpathlineto{\pgfqpoint{3.409297in}{0.739656in}}%
\pgfpathlineto{\pgfqpoint{3.409001in}{0.739656in}}%
\pgfpathlineto{\pgfqpoint{3.408705in}{0.739656in}}%
\pgfpathlineto{\pgfqpoint{3.408409in}{0.739656in}}%
\pgfpathlineto{\pgfqpoint{3.408113in}{0.739656in}}%
\pgfpathlineto{\pgfqpoint{3.407817in}{0.739656in}}%
\pgfpathlineto{\pgfqpoint{3.407521in}{0.739656in}}%
\pgfpathlineto{\pgfqpoint{3.407225in}{0.739656in}}%
\pgfpathlineto{\pgfqpoint{3.406929in}{0.739656in}}%
\pgfpathlineto{\pgfqpoint{3.406633in}{0.739656in}}%
\pgfpathlineto{\pgfqpoint{3.406337in}{0.739656in}}%
\pgfpathlineto{\pgfqpoint{3.406041in}{0.739656in}}%
\pgfpathlineto{\pgfqpoint{3.405745in}{0.739656in}}%
\pgfpathlineto{\pgfqpoint{3.405449in}{0.739656in}}%
\pgfpathlineto{\pgfqpoint{3.405153in}{0.739656in}}%
\pgfpathlineto{\pgfqpoint{3.404857in}{0.739656in}}%
\pgfpathlineto{\pgfqpoint{3.404561in}{0.739656in}}%
\pgfpathlineto{\pgfqpoint{3.404265in}{0.739656in}}%
\pgfpathlineto{\pgfqpoint{3.403969in}{0.739656in}}%
\pgfpathlineto{\pgfqpoint{3.403673in}{0.739656in}}%
\pgfpathlineto{\pgfqpoint{3.403377in}{0.739656in}}%
\pgfpathlineto{\pgfqpoint{3.403081in}{0.739656in}}%
\pgfpathlineto{\pgfqpoint{3.402785in}{0.739656in}}%
\pgfpathlineto{\pgfqpoint{3.402489in}{0.739656in}}%
\pgfpathlineto{\pgfqpoint{3.402193in}{0.739656in}}%
\pgfpathlineto{\pgfqpoint{3.401897in}{0.739656in}}%
\pgfpathlineto{\pgfqpoint{3.401601in}{0.739656in}}%
\pgfpathlineto{\pgfqpoint{3.401305in}{0.739656in}}%
\pgfpathlineto{\pgfqpoint{3.401009in}{0.739656in}}%
\pgfpathlineto{\pgfqpoint{3.400713in}{0.739656in}}%
\pgfpathlineto{\pgfqpoint{3.400417in}{0.739656in}}%
\pgfpathlineto{\pgfqpoint{3.400121in}{0.739656in}}%
\pgfpathlineto{\pgfqpoint{3.399825in}{0.739656in}}%
\pgfpathlineto{\pgfqpoint{3.399529in}{0.739656in}}%
\pgfpathlineto{\pgfqpoint{3.399233in}{0.739656in}}%
\pgfpathlineto{\pgfqpoint{3.398937in}{0.739656in}}%
\pgfpathlineto{\pgfqpoint{3.398641in}{0.739656in}}%
\pgfpathlineto{\pgfqpoint{3.398345in}{0.739656in}}%
\pgfpathlineto{\pgfqpoint{3.398049in}{0.739656in}}%
\pgfpathlineto{\pgfqpoint{3.397753in}{0.739656in}}%
\pgfpathlineto{\pgfqpoint{3.397457in}{0.739656in}}%
\pgfpathlineto{\pgfqpoint{3.397161in}{0.739656in}}%
\pgfpathlineto{\pgfqpoint{3.396865in}{0.739656in}}%
\pgfpathlineto{\pgfqpoint{3.396569in}{0.739656in}}%
\pgfpathlineto{\pgfqpoint{3.396273in}{0.739656in}}%
\pgfpathlineto{\pgfqpoint{3.395977in}{0.739656in}}%
\pgfpathlineto{\pgfqpoint{3.395681in}{0.739656in}}%
\pgfpathlineto{\pgfqpoint{3.395385in}{0.739656in}}%
\pgfpathlineto{\pgfqpoint{3.395089in}{0.739656in}}%
\pgfpathlineto{\pgfqpoint{3.394793in}{0.739656in}}%
\pgfpathlineto{\pgfqpoint{3.394497in}{0.739656in}}%
\pgfpathlineto{\pgfqpoint{3.394201in}{0.739656in}}%
\pgfpathlineto{\pgfqpoint{3.393905in}{0.739656in}}%
\pgfpathlineto{\pgfqpoint{3.393609in}{0.739656in}}%
\pgfpathlineto{\pgfqpoint{3.393313in}{0.739656in}}%
\pgfpathlineto{\pgfqpoint{3.393017in}{0.739656in}}%
\pgfpathlineto{\pgfqpoint{3.392721in}{0.739656in}}%
\pgfpathlineto{\pgfqpoint{3.392425in}{0.739656in}}%
\pgfpathlineto{\pgfqpoint{3.392129in}{0.739656in}}%
\pgfpathlineto{\pgfqpoint{3.391833in}{0.739656in}}%
\pgfpathlineto{\pgfqpoint{3.391537in}{0.739656in}}%
\pgfpathlineto{\pgfqpoint{3.391241in}{0.739656in}}%
\pgfpathlineto{\pgfqpoint{3.390945in}{0.739656in}}%
\pgfpathlineto{\pgfqpoint{3.390649in}{0.739656in}}%
\pgfpathlineto{\pgfqpoint{3.390353in}{0.739656in}}%
\pgfpathlineto{\pgfqpoint{3.390057in}{0.739656in}}%
\pgfpathlineto{\pgfqpoint{3.389761in}{0.739656in}}%
\pgfpathlineto{\pgfqpoint{3.389465in}{0.739656in}}%
\pgfpathlineto{\pgfqpoint{3.389169in}{0.739656in}}%
\pgfpathlineto{\pgfqpoint{3.388873in}{0.739656in}}%
\pgfpathlineto{\pgfqpoint{3.388577in}{0.739656in}}%
\pgfpathlineto{\pgfqpoint{3.388281in}{0.739656in}}%
\pgfpathlineto{\pgfqpoint{3.387985in}{0.739656in}}%
\pgfpathlineto{\pgfqpoint{3.387689in}{0.739656in}}%
\pgfpathlineto{\pgfqpoint{3.387393in}{0.739656in}}%
\pgfpathlineto{\pgfqpoint{3.387097in}{0.739656in}}%
\pgfpathlineto{\pgfqpoint{3.386801in}{0.739656in}}%
\pgfpathlineto{\pgfqpoint{3.386505in}{0.739656in}}%
\pgfpathlineto{\pgfqpoint{3.386209in}{0.739656in}}%
\pgfpathlineto{\pgfqpoint{3.385913in}{0.739656in}}%
\pgfpathlineto{\pgfqpoint{3.385617in}{0.739656in}}%
\pgfpathlineto{\pgfqpoint{3.385321in}{0.739656in}}%
\pgfpathlineto{\pgfqpoint{3.385025in}{0.739656in}}%
\pgfpathlineto{\pgfqpoint{3.384729in}{0.739656in}}%
\pgfpathlineto{\pgfqpoint{3.384433in}{0.739656in}}%
\pgfpathlineto{\pgfqpoint{3.384137in}{0.739656in}}%
\pgfpathlineto{\pgfqpoint{3.383841in}{0.739656in}}%
\pgfpathlineto{\pgfqpoint{3.383545in}{0.739656in}}%
\pgfpathlineto{\pgfqpoint{3.383249in}{0.739656in}}%
\pgfpathlineto{\pgfqpoint{3.382953in}{0.739656in}}%
\pgfpathlineto{\pgfqpoint{3.382657in}{0.739656in}}%
\pgfpathlineto{\pgfqpoint{3.382361in}{0.739656in}}%
\pgfpathlineto{\pgfqpoint{3.382065in}{0.739656in}}%
\pgfpathlineto{\pgfqpoint{3.381769in}{0.739656in}}%
\pgfpathlineto{\pgfqpoint{3.381473in}{0.739656in}}%
\pgfpathlineto{\pgfqpoint{3.381177in}{0.739656in}}%
\pgfpathlineto{\pgfqpoint{3.380881in}{0.739656in}}%
\pgfpathlineto{\pgfqpoint{3.380585in}{0.739656in}}%
\pgfpathlineto{\pgfqpoint{3.380289in}{0.739656in}}%
\pgfpathlineto{\pgfqpoint{3.379993in}{0.739656in}}%
\pgfpathlineto{\pgfqpoint{3.379697in}{0.739656in}}%
\pgfpathlineto{\pgfqpoint{3.379401in}{0.739656in}}%
\pgfpathlineto{\pgfqpoint{3.379105in}{0.739656in}}%
\pgfpathlineto{\pgfqpoint{3.378809in}{0.739656in}}%
\pgfpathlineto{\pgfqpoint{3.378513in}{0.739656in}}%
\pgfpathlineto{\pgfqpoint{3.378217in}{0.739656in}}%
\pgfpathlineto{\pgfqpoint{3.377921in}{0.739656in}}%
\pgfpathlineto{\pgfqpoint{3.377625in}{0.739656in}}%
\pgfpathlineto{\pgfqpoint{3.377329in}{0.739656in}}%
\pgfpathlineto{\pgfqpoint{3.377033in}{0.739656in}}%
\pgfpathlineto{\pgfqpoint{3.376737in}{0.739656in}}%
\pgfpathlineto{\pgfqpoint{3.376441in}{0.739656in}}%
\pgfpathlineto{\pgfqpoint{3.376145in}{0.739656in}}%
\pgfpathlineto{\pgfqpoint{3.375849in}{0.739656in}}%
\pgfpathlineto{\pgfqpoint{3.375553in}{0.739656in}}%
\pgfpathlineto{\pgfqpoint{3.375257in}{0.739656in}}%
\pgfpathlineto{\pgfqpoint{3.374961in}{0.739656in}}%
\pgfpathlineto{\pgfqpoint{3.374665in}{0.739656in}}%
\pgfpathlineto{\pgfqpoint{3.374369in}{0.739656in}}%
\pgfpathlineto{\pgfqpoint{3.374073in}{0.739656in}}%
\pgfpathlineto{\pgfqpoint{3.373777in}{0.739656in}}%
\pgfpathlineto{\pgfqpoint{3.373481in}{0.739656in}}%
\pgfpathlineto{\pgfqpoint{3.373185in}{0.739656in}}%
\pgfpathlineto{\pgfqpoint{3.372889in}{0.739656in}}%
\pgfpathlineto{\pgfqpoint{3.372593in}{0.739656in}}%
\pgfpathlineto{\pgfqpoint{3.372297in}{0.739656in}}%
\pgfpathlineto{\pgfqpoint{3.372001in}{0.739656in}}%
\pgfpathlineto{\pgfqpoint{3.371705in}{0.739656in}}%
\pgfpathlineto{\pgfqpoint{3.371409in}{0.739656in}}%
\pgfpathlineto{\pgfqpoint{3.371113in}{0.739656in}}%
\pgfpathlineto{\pgfqpoint{3.370817in}{0.739656in}}%
\pgfpathlineto{\pgfqpoint{3.370521in}{0.739656in}}%
\pgfpathlineto{\pgfqpoint{3.370225in}{0.739656in}}%
\pgfpathlineto{\pgfqpoint{3.369929in}{0.739656in}}%
\pgfpathlineto{\pgfqpoint{3.369633in}{0.739656in}}%
\pgfpathlineto{\pgfqpoint{3.369337in}{0.739656in}}%
\pgfpathlineto{\pgfqpoint{3.369041in}{0.739656in}}%
\pgfpathlineto{\pgfqpoint{3.368745in}{0.739656in}}%
\pgfpathlineto{\pgfqpoint{3.368449in}{0.739656in}}%
\pgfpathlineto{\pgfqpoint{3.368153in}{0.739656in}}%
\pgfpathlineto{\pgfqpoint{3.367857in}{0.739656in}}%
\pgfpathlineto{\pgfqpoint{3.367561in}{0.739656in}}%
\pgfpathlineto{\pgfqpoint{3.367265in}{0.739656in}}%
\pgfpathlineto{\pgfqpoint{3.366969in}{0.739656in}}%
\pgfpathlineto{\pgfqpoint{3.366673in}{0.739656in}}%
\pgfpathlineto{\pgfqpoint{3.366377in}{0.739656in}}%
\pgfpathlineto{\pgfqpoint{3.366081in}{0.739656in}}%
\pgfpathlineto{\pgfqpoint{3.365785in}{0.739656in}}%
\pgfpathlineto{\pgfqpoint{3.365489in}{0.739656in}}%
\pgfpathlineto{\pgfqpoint{3.365193in}{0.739656in}}%
\pgfpathlineto{\pgfqpoint{3.364897in}{0.739656in}}%
\pgfpathlineto{\pgfqpoint{3.364601in}{0.739656in}}%
\pgfpathlineto{\pgfqpoint{3.364305in}{0.739656in}}%
\pgfpathlineto{\pgfqpoint{3.364009in}{0.739656in}}%
\pgfpathlineto{\pgfqpoint{3.363713in}{0.739656in}}%
\pgfpathlineto{\pgfqpoint{3.363417in}{0.739656in}}%
\pgfpathlineto{\pgfqpoint{3.363121in}{0.739656in}}%
\pgfpathlineto{\pgfqpoint{3.362825in}{0.739656in}}%
\pgfpathlineto{\pgfqpoint{3.362529in}{0.739656in}}%
\pgfpathlineto{\pgfqpoint{3.362233in}{0.739656in}}%
\pgfpathlineto{\pgfqpoint{3.361937in}{0.739656in}}%
\pgfpathlineto{\pgfqpoint{3.361641in}{0.739656in}}%
\pgfpathlineto{\pgfqpoint{3.361345in}{0.739656in}}%
\pgfpathlineto{\pgfqpoint{3.361049in}{0.739656in}}%
\pgfpathlineto{\pgfqpoint{3.360753in}{0.739656in}}%
\pgfpathlineto{\pgfqpoint{3.360457in}{0.739656in}}%
\pgfpathlineto{\pgfqpoint{3.360161in}{0.739656in}}%
\pgfpathlineto{\pgfqpoint{3.359865in}{0.739656in}}%
\pgfpathlineto{\pgfqpoint{3.359569in}{0.739656in}}%
\pgfpathlineto{\pgfqpoint{3.359273in}{0.739656in}}%
\pgfpathlineto{\pgfqpoint{3.358977in}{0.739656in}}%
\pgfpathlineto{\pgfqpoint{3.358681in}{0.739656in}}%
\pgfpathlineto{\pgfqpoint{3.358385in}{0.739656in}}%
\pgfpathlineto{\pgfqpoint{3.358089in}{0.739656in}}%
\pgfpathlineto{\pgfqpoint{3.357793in}{0.739656in}}%
\pgfpathlineto{\pgfqpoint{3.357497in}{0.739656in}}%
\pgfpathlineto{\pgfqpoint{3.357201in}{0.739656in}}%
\pgfpathlineto{\pgfqpoint{3.356905in}{0.739656in}}%
\pgfpathlineto{\pgfqpoint{3.356609in}{0.739656in}}%
\pgfpathlineto{\pgfqpoint{3.356313in}{0.739656in}}%
\pgfpathlineto{\pgfqpoint{3.356017in}{0.739656in}}%
\pgfpathlineto{\pgfqpoint{3.355721in}{0.739656in}}%
\pgfpathlineto{\pgfqpoint{3.355425in}{0.739656in}}%
\pgfpathlineto{\pgfqpoint{3.355129in}{0.739656in}}%
\pgfpathlineto{\pgfqpoint{3.354833in}{0.739656in}}%
\pgfpathlineto{\pgfqpoint{3.354537in}{0.739656in}}%
\pgfpathlineto{\pgfqpoint{3.354241in}{0.739656in}}%
\pgfpathlineto{\pgfqpoint{3.353945in}{0.739656in}}%
\pgfpathlineto{\pgfqpoint{3.353649in}{0.739656in}}%
\pgfpathlineto{\pgfqpoint{3.353353in}{0.739656in}}%
\pgfpathlineto{\pgfqpoint{3.353057in}{0.739656in}}%
\pgfpathlineto{\pgfqpoint{3.352761in}{0.739656in}}%
\pgfpathlineto{\pgfqpoint{3.352465in}{0.739656in}}%
\pgfpathlineto{\pgfqpoint{3.352169in}{0.739656in}}%
\pgfpathlineto{\pgfqpoint{3.351873in}{0.739656in}}%
\pgfpathlineto{\pgfqpoint{3.351577in}{0.739656in}}%
\pgfpathlineto{\pgfqpoint{3.351281in}{0.739656in}}%
\pgfpathlineto{\pgfqpoint{3.350985in}{0.739656in}}%
\pgfpathlineto{\pgfqpoint{3.350689in}{0.739656in}}%
\pgfpathlineto{\pgfqpoint{3.350393in}{0.739656in}}%
\pgfpathlineto{\pgfqpoint{3.350097in}{0.739656in}}%
\pgfpathlineto{\pgfqpoint{3.349801in}{0.739656in}}%
\pgfpathlineto{\pgfqpoint{3.349505in}{0.739656in}}%
\pgfpathlineto{\pgfqpoint{3.349209in}{0.739656in}}%
\pgfpathlineto{\pgfqpoint{3.348913in}{0.739656in}}%
\pgfpathlineto{\pgfqpoint{3.348617in}{0.739656in}}%
\pgfpathlineto{\pgfqpoint{3.348321in}{0.739656in}}%
\pgfpathlineto{\pgfqpoint{3.348025in}{0.739656in}}%
\pgfpathlineto{\pgfqpoint{3.347729in}{0.739656in}}%
\pgfpathlineto{\pgfqpoint{3.347433in}{0.739656in}}%
\pgfpathlineto{\pgfqpoint{3.347137in}{0.739656in}}%
\pgfpathlineto{\pgfqpoint{3.346841in}{0.739656in}}%
\pgfpathlineto{\pgfqpoint{3.346545in}{0.739656in}}%
\pgfpathlineto{\pgfqpoint{3.346249in}{0.739656in}}%
\pgfpathlineto{\pgfqpoint{3.345953in}{0.739656in}}%
\pgfpathlineto{\pgfqpoint{3.345657in}{0.739656in}}%
\pgfpathlineto{\pgfqpoint{3.345361in}{0.739656in}}%
\pgfpathlineto{\pgfqpoint{3.345065in}{0.739656in}}%
\pgfpathlineto{\pgfqpoint{3.344769in}{0.739656in}}%
\pgfpathlineto{\pgfqpoint{3.344472in}{0.739656in}}%
\pgfpathlineto{\pgfqpoint{3.344176in}{0.739656in}}%
\pgfpathlineto{\pgfqpoint{3.343880in}{0.739656in}}%
\pgfpathlineto{\pgfqpoint{3.343584in}{0.739656in}}%
\pgfpathlineto{\pgfqpoint{3.343288in}{0.739656in}}%
\pgfpathlineto{\pgfqpoint{3.342992in}{0.739656in}}%
\pgfpathlineto{\pgfqpoint{3.342696in}{0.739656in}}%
\pgfpathlineto{\pgfqpoint{3.342400in}{0.739656in}}%
\pgfpathlineto{\pgfqpoint{3.342104in}{0.739656in}}%
\pgfpathlineto{\pgfqpoint{3.341808in}{0.739656in}}%
\pgfpathlineto{\pgfqpoint{3.341512in}{0.739656in}}%
\pgfpathlineto{\pgfqpoint{3.341216in}{0.739656in}}%
\pgfpathlineto{\pgfqpoint{3.340920in}{0.739656in}}%
\pgfpathlineto{\pgfqpoint{3.340624in}{0.739656in}}%
\pgfpathlineto{\pgfqpoint{3.340328in}{0.739656in}}%
\pgfpathlineto{\pgfqpoint{3.340032in}{0.739656in}}%
\pgfpathlineto{\pgfqpoint{3.339736in}{0.739656in}}%
\pgfpathlineto{\pgfqpoint{3.339440in}{0.739656in}}%
\pgfpathlineto{\pgfqpoint{3.339144in}{0.739656in}}%
\pgfpathlineto{\pgfqpoint{3.338848in}{0.739656in}}%
\pgfpathlineto{\pgfqpoint{3.338552in}{0.739656in}}%
\pgfpathlineto{\pgfqpoint{3.338256in}{0.739656in}}%
\pgfpathlineto{\pgfqpoint{3.337960in}{0.739656in}}%
\pgfpathlineto{\pgfqpoint{3.337664in}{0.739656in}}%
\pgfpathlineto{\pgfqpoint{3.337368in}{0.739656in}}%
\pgfpathlineto{\pgfqpoint{3.337072in}{0.739656in}}%
\pgfpathlineto{\pgfqpoint{3.336776in}{0.739656in}}%
\pgfpathlineto{\pgfqpoint{3.336480in}{0.739656in}}%
\pgfpathlineto{\pgfqpoint{3.336184in}{0.739656in}}%
\pgfpathlineto{\pgfqpoint{3.335888in}{0.739656in}}%
\pgfpathlineto{\pgfqpoint{3.335592in}{0.739656in}}%
\pgfpathlineto{\pgfqpoint{3.335296in}{0.739656in}}%
\pgfpathlineto{\pgfqpoint{3.335000in}{0.739656in}}%
\pgfpathlineto{\pgfqpoint{3.334704in}{0.739656in}}%
\pgfpathlineto{\pgfqpoint{3.334408in}{0.739656in}}%
\pgfpathlineto{\pgfqpoint{3.334112in}{0.739656in}}%
\pgfpathlineto{\pgfqpoint{3.333816in}{0.739656in}}%
\pgfpathlineto{\pgfqpoint{3.333520in}{0.739656in}}%
\pgfpathlineto{\pgfqpoint{3.333224in}{0.739656in}}%
\pgfpathlineto{\pgfqpoint{3.332928in}{0.739656in}}%
\pgfpathlineto{\pgfqpoint{3.332632in}{0.739656in}}%
\pgfpathlineto{\pgfqpoint{3.332336in}{0.739656in}}%
\pgfpathlineto{\pgfqpoint{3.332040in}{0.739656in}}%
\pgfpathlineto{\pgfqpoint{3.331744in}{0.739656in}}%
\pgfpathlineto{\pgfqpoint{3.331448in}{0.739656in}}%
\pgfpathlineto{\pgfqpoint{3.331152in}{0.739656in}}%
\pgfpathlineto{\pgfqpoint{3.330856in}{0.739656in}}%
\pgfpathlineto{\pgfqpoint{3.330560in}{0.739656in}}%
\pgfpathlineto{\pgfqpoint{3.330264in}{0.739656in}}%
\pgfpathlineto{\pgfqpoint{3.329968in}{0.739656in}}%
\pgfpathlineto{\pgfqpoint{3.329672in}{0.739656in}}%
\pgfpathlineto{\pgfqpoint{3.329376in}{0.739656in}}%
\pgfpathlineto{\pgfqpoint{3.329080in}{0.739656in}}%
\pgfpathlineto{\pgfqpoint{3.328784in}{0.739656in}}%
\pgfpathlineto{\pgfqpoint{3.328488in}{0.739656in}}%
\pgfpathlineto{\pgfqpoint{3.328192in}{0.739656in}}%
\pgfpathlineto{\pgfqpoint{3.327896in}{0.739656in}}%
\pgfpathlineto{\pgfqpoint{3.327600in}{0.739656in}}%
\pgfpathlineto{\pgfqpoint{3.327304in}{0.739656in}}%
\pgfpathlineto{\pgfqpoint{3.327008in}{0.739656in}}%
\pgfpathlineto{\pgfqpoint{3.326712in}{0.739656in}}%
\pgfpathlineto{\pgfqpoint{3.326416in}{0.739656in}}%
\pgfpathlineto{\pgfqpoint{3.326120in}{0.739656in}}%
\pgfpathlineto{\pgfqpoint{3.325824in}{0.739656in}}%
\pgfpathlineto{\pgfqpoint{3.325528in}{0.739656in}}%
\pgfpathlineto{\pgfqpoint{3.325232in}{0.739656in}}%
\pgfpathlineto{\pgfqpoint{3.324936in}{0.739656in}}%
\pgfpathlineto{\pgfqpoint{3.324640in}{0.739656in}}%
\pgfpathlineto{\pgfqpoint{3.324344in}{0.739656in}}%
\pgfpathlineto{\pgfqpoint{3.324048in}{0.739656in}}%
\pgfpathlineto{\pgfqpoint{3.323752in}{0.739656in}}%
\pgfpathlineto{\pgfqpoint{3.323456in}{0.739656in}}%
\pgfpathlineto{\pgfqpoint{3.323160in}{0.739656in}}%
\pgfpathlineto{\pgfqpoint{3.322864in}{0.739656in}}%
\pgfpathlineto{\pgfqpoint{3.322568in}{0.739656in}}%
\pgfpathlineto{\pgfqpoint{3.322272in}{0.739656in}}%
\pgfpathlineto{\pgfqpoint{3.321976in}{0.739656in}}%
\pgfpathlineto{\pgfqpoint{3.321680in}{0.739656in}}%
\pgfpathlineto{\pgfqpoint{3.321384in}{0.739656in}}%
\pgfpathlineto{\pgfqpoint{3.321088in}{0.739656in}}%
\pgfpathlineto{\pgfqpoint{3.320792in}{0.739656in}}%
\pgfpathlineto{\pgfqpoint{3.320496in}{0.739656in}}%
\pgfpathlineto{\pgfqpoint{3.320200in}{0.739656in}}%
\pgfpathlineto{\pgfqpoint{3.319904in}{0.739656in}}%
\pgfpathlineto{\pgfqpoint{3.319608in}{0.739656in}}%
\pgfpathlineto{\pgfqpoint{3.319312in}{0.739656in}}%
\pgfpathlineto{\pgfqpoint{3.319016in}{0.739656in}}%
\pgfpathlineto{\pgfqpoint{3.318720in}{0.739656in}}%
\pgfpathlineto{\pgfqpoint{3.318424in}{0.739656in}}%
\pgfpathlineto{\pgfqpoint{3.318128in}{0.739656in}}%
\pgfpathlineto{\pgfqpoint{3.317832in}{0.739656in}}%
\pgfpathlineto{\pgfqpoint{3.317536in}{0.739656in}}%
\pgfpathlineto{\pgfqpoint{3.317240in}{0.739656in}}%
\pgfpathlineto{\pgfqpoint{3.316944in}{0.739656in}}%
\pgfpathlineto{\pgfqpoint{3.316648in}{0.739656in}}%
\pgfpathlineto{\pgfqpoint{3.316352in}{0.739656in}}%
\pgfpathlineto{\pgfqpoint{3.316056in}{0.739656in}}%
\pgfpathlineto{\pgfqpoint{3.315760in}{0.739656in}}%
\pgfpathlineto{\pgfqpoint{3.315464in}{0.739656in}}%
\pgfpathlineto{\pgfqpoint{3.315168in}{0.739656in}}%
\pgfpathlineto{\pgfqpoint{3.314872in}{0.739656in}}%
\pgfpathlineto{\pgfqpoint{3.314576in}{0.739656in}}%
\pgfpathlineto{\pgfqpoint{3.314280in}{0.739656in}}%
\pgfpathlineto{\pgfqpoint{3.313984in}{0.739656in}}%
\pgfpathlineto{\pgfqpoint{3.313688in}{0.739656in}}%
\pgfpathlineto{\pgfqpoint{3.313392in}{0.739656in}}%
\pgfpathlineto{\pgfqpoint{3.313096in}{0.739656in}}%
\pgfpathlineto{\pgfqpoint{3.312800in}{0.739656in}}%
\pgfpathlineto{\pgfqpoint{3.312504in}{0.739656in}}%
\pgfpathlineto{\pgfqpoint{3.312208in}{0.739656in}}%
\pgfpathlineto{\pgfqpoint{3.311912in}{0.739656in}}%
\pgfpathlineto{\pgfqpoint{3.311616in}{0.739656in}}%
\pgfpathlineto{\pgfqpoint{3.311320in}{0.739656in}}%
\pgfpathlineto{\pgfqpoint{3.311024in}{0.739656in}}%
\pgfpathlineto{\pgfqpoint{3.310728in}{0.739656in}}%
\pgfpathlineto{\pgfqpoint{3.310432in}{0.739656in}}%
\pgfpathlineto{\pgfqpoint{3.310136in}{0.739656in}}%
\pgfpathlineto{\pgfqpoint{3.309840in}{0.739656in}}%
\pgfpathlineto{\pgfqpoint{3.309544in}{0.739656in}}%
\pgfpathlineto{\pgfqpoint{3.309248in}{0.739656in}}%
\pgfpathlineto{\pgfqpoint{3.308952in}{0.739656in}}%
\pgfpathlineto{\pgfqpoint{3.308656in}{0.739656in}}%
\pgfpathlineto{\pgfqpoint{3.308360in}{0.739656in}}%
\pgfpathlineto{\pgfqpoint{3.308064in}{0.739656in}}%
\pgfpathlineto{\pgfqpoint{3.307768in}{0.739656in}}%
\pgfpathlineto{\pgfqpoint{3.307472in}{0.739656in}}%
\pgfpathlineto{\pgfqpoint{3.307176in}{0.739656in}}%
\pgfpathlineto{\pgfqpoint{3.306880in}{0.739656in}}%
\pgfpathlineto{\pgfqpoint{3.306584in}{0.739656in}}%
\pgfpathlineto{\pgfqpoint{3.306288in}{0.739656in}}%
\pgfpathlineto{\pgfqpoint{3.305992in}{0.739656in}}%
\pgfpathlineto{\pgfqpoint{3.305696in}{0.739656in}}%
\pgfpathlineto{\pgfqpoint{3.305400in}{0.739656in}}%
\pgfpathlineto{\pgfqpoint{3.305104in}{0.739656in}}%
\pgfpathlineto{\pgfqpoint{3.304808in}{0.739656in}}%
\pgfpathlineto{\pgfqpoint{3.304512in}{0.739656in}}%
\pgfpathlineto{\pgfqpoint{3.304216in}{0.739656in}}%
\pgfpathlineto{\pgfqpoint{3.303920in}{0.739656in}}%
\pgfpathlineto{\pgfqpoint{3.303624in}{0.739656in}}%
\pgfpathlineto{\pgfqpoint{3.303328in}{0.739656in}}%
\pgfpathlineto{\pgfqpoint{3.303032in}{0.739656in}}%
\pgfpathlineto{\pgfqpoint{3.302736in}{0.739656in}}%
\pgfpathlineto{\pgfqpoint{3.302440in}{0.739656in}}%
\pgfpathlineto{\pgfqpoint{3.302144in}{0.739656in}}%
\pgfpathlineto{\pgfqpoint{3.301848in}{0.739656in}}%
\pgfpathlineto{\pgfqpoint{3.301552in}{0.739656in}}%
\pgfpathlineto{\pgfqpoint{3.301256in}{0.739656in}}%
\pgfpathlineto{\pgfqpoint{3.300960in}{0.739656in}}%
\pgfpathlineto{\pgfqpoint{3.300664in}{0.739656in}}%
\pgfpathlineto{\pgfqpoint{3.300368in}{0.739656in}}%
\pgfpathlineto{\pgfqpoint{3.300072in}{0.739656in}}%
\pgfpathlineto{\pgfqpoint{3.299776in}{0.739656in}}%
\pgfpathlineto{\pgfqpoint{3.299480in}{0.739656in}}%
\pgfpathlineto{\pgfqpoint{3.299184in}{0.739656in}}%
\pgfpathlineto{\pgfqpoint{3.298888in}{0.739656in}}%
\pgfpathlineto{\pgfqpoint{3.298592in}{0.739656in}}%
\pgfpathlineto{\pgfqpoint{3.298296in}{0.739656in}}%
\pgfpathlineto{\pgfqpoint{3.298000in}{0.739656in}}%
\pgfpathlineto{\pgfqpoint{3.297704in}{0.739656in}}%
\pgfpathlineto{\pgfqpoint{3.297408in}{0.739656in}}%
\pgfpathlineto{\pgfqpoint{3.297112in}{0.739656in}}%
\pgfpathlineto{\pgfqpoint{3.296816in}{0.739656in}}%
\pgfpathlineto{\pgfqpoint{3.296520in}{0.739656in}}%
\pgfpathlineto{\pgfqpoint{3.296224in}{0.739656in}}%
\pgfpathlineto{\pgfqpoint{3.295928in}{0.739656in}}%
\pgfpathlineto{\pgfqpoint{3.295632in}{0.739656in}}%
\pgfpathlineto{\pgfqpoint{3.295336in}{0.739656in}}%
\pgfpathlineto{\pgfqpoint{3.295040in}{0.739656in}}%
\pgfpathlineto{\pgfqpoint{3.294744in}{0.739656in}}%
\pgfpathlineto{\pgfqpoint{3.294448in}{0.739656in}}%
\pgfpathlineto{\pgfqpoint{3.294152in}{0.739656in}}%
\pgfpathlineto{\pgfqpoint{3.293856in}{0.739656in}}%
\pgfpathlineto{\pgfqpoint{3.293560in}{0.739656in}}%
\pgfpathlineto{\pgfqpoint{3.293264in}{0.739656in}}%
\pgfpathlineto{\pgfqpoint{3.292968in}{0.739656in}}%
\pgfpathlineto{\pgfqpoint{3.292672in}{0.739656in}}%
\pgfpathlineto{\pgfqpoint{3.292376in}{0.739656in}}%
\pgfpathlineto{\pgfqpoint{3.292080in}{0.739656in}}%
\pgfpathlineto{\pgfqpoint{3.291784in}{0.739656in}}%
\pgfpathlineto{\pgfqpoint{3.291488in}{0.739656in}}%
\pgfpathlineto{\pgfqpoint{3.291192in}{0.739656in}}%
\pgfpathlineto{\pgfqpoint{3.290896in}{0.739656in}}%
\pgfpathlineto{\pgfqpoint{3.290600in}{0.739656in}}%
\pgfpathlineto{\pgfqpoint{3.290304in}{0.739656in}}%
\pgfpathlineto{\pgfqpoint{3.290008in}{0.739656in}}%
\pgfpathlineto{\pgfqpoint{3.289712in}{0.739656in}}%
\pgfpathlineto{\pgfqpoint{3.289416in}{0.739656in}}%
\pgfpathlineto{\pgfqpoint{3.289120in}{0.739656in}}%
\pgfpathlineto{\pgfqpoint{3.288824in}{0.739656in}}%
\pgfpathlineto{\pgfqpoint{3.288528in}{0.739656in}}%
\pgfpathlineto{\pgfqpoint{3.288232in}{0.739656in}}%
\pgfpathlineto{\pgfqpoint{3.287936in}{0.739656in}}%
\pgfpathlineto{\pgfqpoint{3.287640in}{0.739656in}}%
\pgfpathlineto{\pgfqpoint{3.287344in}{0.739656in}}%
\pgfpathlineto{\pgfqpoint{3.287048in}{0.739656in}}%
\pgfpathlineto{\pgfqpoint{3.286752in}{0.739656in}}%
\pgfpathlineto{\pgfqpoint{3.286456in}{0.739656in}}%
\pgfpathlineto{\pgfqpoint{3.286160in}{0.739656in}}%
\pgfpathlineto{\pgfqpoint{3.285864in}{0.739656in}}%
\pgfpathlineto{\pgfqpoint{3.285568in}{0.739656in}}%
\pgfpathlineto{\pgfqpoint{3.285272in}{0.739656in}}%
\pgfpathlineto{\pgfqpoint{3.284976in}{0.739656in}}%
\pgfpathlineto{\pgfqpoint{3.284680in}{0.739656in}}%
\pgfpathlineto{\pgfqpoint{3.284384in}{0.739656in}}%
\pgfpathlineto{\pgfqpoint{3.284088in}{0.739656in}}%
\pgfpathlineto{\pgfqpoint{3.283792in}{0.739656in}}%
\pgfpathlineto{\pgfqpoint{3.283496in}{0.739656in}}%
\pgfpathlineto{\pgfqpoint{3.283200in}{0.739656in}}%
\pgfpathlineto{\pgfqpoint{3.282904in}{0.739656in}}%
\pgfpathlineto{\pgfqpoint{3.282608in}{0.739656in}}%
\pgfpathlineto{\pgfqpoint{3.282312in}{0.739656in}}%
\pgfpathlineto{\pgfqpoint{3.282016in}{0.739656in}}%
\pgfpathlineto{\pgfqpoint{3.281720in}{0.739656in}}%
\pgfpathlineto{\pgfqpoint{3.281424in}{0.739656in}}%
\pgfpathlineto{\pgfqpoint{3.281128in}{0.739656in}}%
\pgfpathlineto{\pgfqpoint{3.280832in}{0.739656in}}%
\pgfpathlineto{\pgfqpoint{3.280536in}{0.739656in}}%
\pgfpathlineto{\pgfqpoint{3.280240in}{0.739656in}}%
\pgfpathlineto{\pgfqpoint{3.279944in}{0.739656in}}%
\pgfpathlineto{\pgfqpoint{3.279648in}{0.739656in}}%
\pgfpathlineto{\pgfqpoint{3.279352in}{0.739656in}}%
\pgfpathlineto{\pgfqpoint{3.279056in}{0.739656in}}%
\pgfpathlineto{\pgfqpoint{3.278760in}{0.739656in}}%
\pgfpathlineto{\pgfqpoint{3.278464in}{0.739656in}}%
\pgfpathlineto{\pgfqpoint{3.278168in}{0.739656in}}%
\pgfpathlineto{\pgfqpoint{3.277872in}{0.739656in}}%
\pgfpathlineto{\pgfqpoint{3.277576in}{0.739656in}}%
\pgfpathlineto{\pgfqpoint{3.277280in}{0.739656in}}%
\pgfpathlineto{\pgfqpoint{3.276983in}{0.739656in}}%
\pgfpathlineto{\pgfqpoint{3.276687in}{0.739656in}}%
\pgfpathlineto{\pgfqpoint{3.276391in}{0.739656in}}%
\pgfpathlineto{\pgfqpoint{3.276095in}{0.739656in}}%
\pgfpathlineto{\pgfqpoint{3.275799in}{0.739656in}}%
\pgfpathlineto{\pgfqpoint{3.275503in}{0.739656in}}%
\pgfpathlineto{\pgfqpoint{3.275207in}{0.739656in}}%
\pgfpathlineto{\pgfqpoint{3.274911in}{0.739656in}}%
\pgfpathlineto{\pgfqpoint{3.274615in}{0.739656in}}%
\pgfpathlineto{\pgfqpoint{3.274319in}{0.739656in}}%
\pgfpathlineto{\pgfqpoint{3.274023in}{0.739656in}}%
\pgfpathlineto{\pgfqpoint{3.273727in}{0.739656in}}%
\pgfpathlineto{\pgfqpoint{3.273431in}{0.739656in}}%
\pgfpathlineto{\pgfqpoint{3.273135in}{0.739656in}}%
\pgfpathlineto{\pgfqpoint{3.272839in}{0.739656in}}%
\pgfpathlineto{\pgfqpoint{3.272543in}{0.739656in}}%
\pgfpathlineto{\pgfqpoint{3.272247in}{0.739656in}}%
\pgfpathlineto{\pgfqpoint{3.271951in}{0.739656in}}%
\pgfpathlineto{\pgfqpoint{3.271655in}{0.739656in}}%
\pgfpathlineto{\pgfqpoint{3.271359in}{0.739656in}}%
\pgfpathlineto{\pgfqpoint{3.271063in}{0.739656in}}%
\pgfpathlineto{\pgfqpoint{3.270767in}{0.739656in}}%
\pgfpathlineto{\pgfqpoint{3.270471in}{0.739656in}}%
\pgfpathlineto{\pgfqpoint{3.270175in}{0.739656in}}%
\pgfpathlineto{\pgfqpoint{3.269879in}{0.739656in}}%
\pgfpathlineto{\pgfqpoint{3.269583in}{0.739656in}}%
\pgfpathlineto{\pgfqpoint{3.269287in}{0.739656in}}%
\pgfpathlineto{\pgfqpoint{3.268991in}{0.739656in}}%
\pgfpathlineto{\pgfqpoint{3.268695in}{0.739656in}}%
\pgfpathlineto{\pgfqpoint{3.268399in}{0.739656in}}%
\pgfpathlineto{\pgfqpoint{3.268103in}{0.739656in}}%
\pgfpathlineto{\pgfqpoint{3.267807in}{0.739656in}}%
\pgfpathlineto{\pgfqpoint{3.267511in}{0.739656in}}%
\pgfpathlineto{\pgfqpoint{3.267215in}{0.739656in}}%
\pgfpathlineto{\pgfqpoint{3.266919in}{0.739656in}}%
\pgfpathlineto{\pgfqpoint{3.266623in}{0.739656in}}%
\pgfpathlineto{\pgfqpoint{3.266327in}{0.739656in}}%
\pgfpathlineto{\pgfqpoint{3.266031in}{0.739656in}}%
\pgfpathlineto{\pgfqpoint{3.265735in}{0.739656in}}%
\pgfpathlineto{\pgfqpoint{3.265439in}{0.739656in}}%
\pgfpathlineto{\pgfqpoint{3.265143in}{0.739656in}}%
\pgfpathlineto{\pgfqpoint{3.264847in}{0.739656in}}%
\pgfpathlineto{\pgfqpoint{3.264551in}{0.739656in}}%
\pgfpathlineto{\pgfqpoint{3.264255in}{0.739656in}}%
\pgfpathlineto{\pgfqpoint{3.263959in}{0.739656in}}%
\pgfpathlineto{\pgfqpoint{3.263663in}{0.739656in}}%
\pgfpathlineto{\pgfqpoint{3.263367in}{0.739656in}}%
\pgfpathlineto{\pgfqpoint{3.263071in}{0.739656in}}%
\pgfpathlineto{\pgfqpoint{3.262775in}{0.739656in}}%
\pgfpathlineto{\pgfqpoint{3.262479in}{0.739656in}}%
\pgfpathlineto{\pgfqpoint{3.262183in}{0.739656in}}%
\pgfpathlineto{\pgfqpoint{3.261887in}{0.739656in}}%
\pgfpathlineto{\pgfqpoint{3.261591in}{0.739656in}}%
\pgfpathlineto{\pgfqpoint{3.261295in}{0.739656in}}%
\pgfpathlineto{\pgfqpoint{3.260999in}{0.739656in}}%
\pgfpathlineto{\pgfqpoint{3.260703in}{0.739656in}}%
\pgfpathlineto{\pgfqpoint{3.260407in}{0.739656in}}%
\pgfpathlineto{\pgfqpoint{3.260111in}{0.739656in}}%
\pgfpathlineto{\pgfqpoint{3.259815in}{0.739656in}}%
\pgfpathlineto{\pgfqpoint{3.259519in}{0.739656in}}%
\pgfpathlineto{\pgfqpoint{3.259223in}{0.739656in}}%
\pgfpathlineto{\pgfqpoint{3.258927in}{0.739656in}}%
\pgfpathlineto{\pgfqpoint{3.258631in}{0.739656in}}%
\pgfpathlineto{\pgfqpoint{3.258335in}{0.739656in}}%
\pgfpathlineto{\pgfqpoint{3.258039in}{0.739656in}}%
\pgfpathlineto{\pgfqpoint{3.257743in}{0.739656in}}%
\pgfpathlineto{\pgfqpoint{3.257447in}{0.739656in}}%
\pgfpathlineto{\pgfqpoint{3.257151in}{0.739656in}}%
\pgfpathlineto{\pgfqpoint{3.256855in}{0.739656in}}%
\pgfpathlineto{\pgfqpoint{3.256559in}{0.739656in}}%
\pgfpathlineto{\pgfqpoint{3.256263in}{0.739656in}}%
\pgfpathlineto{\pgfqpoint{3.255967in}{0.739656in}}%
\pgfpathlineto{\pgfqpoint{3.255671in}{0.739656in}}%
\pgfpathlineto{\pgfqpoint{3.255375in}{0.739656in}}%
\pgfpathlineto{\pgfqpoint{3.255079in}{0.739656in}}%
\pgfpathlineto{\pgfqpoint{3.254783in}{0.739656in}}%
\pgfpathlineto{\pgfqpoint{3.254487in}{0.739656in}}%
\pgfpathlineto{\pgfqpoint{3.254191in}{0.739656in}}%
\pgfpathlineto{\pgfqpoint{3.253895in}{0.739656in}}%
\pgfpathlineto{\pgfqpoint{3.253599in}{0.739656in}}%
\pgfpathlineto{\pgfqpoint{3.253303in}{0.739656in}}%
\pgfpathlineto{\pgfqpoint{3.253007in}{0.739656in}}%
\pgfpathlineto{\pgfqpoint{3.252711in}{0.739656in}}%
\pgfpathlineto{\pgfqpoint{3.252415in}{0.739656in}}%
\pgfpathlineto{\pgfqpoint{3.252119in}{0.739656in}}%
\pgfpathlineto{\pgfqpoint{3.251823in}{0.739656in}}%
\pgfpathlineto{\pgfqpoint{3.251527in}{0.739656in}}%
\pgfpathlineto{\pgfqpoint{3.251231in}{0.739656in}}%
\pgfpathlineto{\pgfqpoint{3.250935in}{0.739656in}}%
\pgfpathlineto{\pgfqpoint{3.250639in}{0.739656in}}%
\pgfpathlineto{\pgfqpoint{3.250343in}{0.739656in}}%
\pgfpathlineto{\pgfqpoint{3.250047in}{0.739656in}}%
\pgfpathlineto{\pgfqpoint{3.249751in}{0.739656in}}%
\pgfpathlineto{\pgfqpoint{3.249455in}{0.739656in}}%
\pgfpathlineto{\pgfqpoint{3.249159in}{0.739656in}}%
\pgfpathlineto{\pgfqpoint{3.248863in}{0.739656in}}%
\pgfpathlineto{\pgfqpoint{3.248567in}{0.739656in}}%
\pgfpathlineto{\pgfqpoint{3.248271in}{0.739656in}}%
\pgfpathlineto{\pgfqpoint{3.247975in}{0.739656in}}%
\pgfpathlineto{\pgfqpoint{3.247679in}{0.739656in}}%
\pgfpathlineto{\pgfqpoint{3.247383in}{0.739656in}}%
\pgfpathlineto{\pgfqpoint{3.247087in}{0.739656in}}%
\pgfpathlineto{\pgfqpoint{3.246791in}{0.739656in}}%
\pgfpathlineto{\pgfqpoint{3.246495in}{0.739656in}}%
\pgfpathlineto{\pgfqpoint{3.246199in}{0.739656in}}%
\pgfpathlineto{\pgfqpoint{3.245903in}{0.739656in}}%
\pgfpathlineto{\pgfqpoint{3.245607in}{0.739656in}}%
\pgfpathlineto{\pgfqpoint{3.245311in}{0.739656in}}%
\pgfpathlineto{\pgfqpoint{3.245015in}{0.739656in}}%
\pgfpathlineto{\pgfqpoint{3.244719in}{0.739656in}}%
\pgfpathlineto{\pgfqpoint{3.244423in}{0.739656in}}%
\pgfpathlineto{\pgfqpoint{3.244127in}{0.739656in}}%
\pgfpathlineto{\pgfqpoint{3.243831in}{0.739656in}}%
\pgfpathlineto{\pgfqpoint{3.243535in}{0.739656in}}%
\pgfpathlineto{\pgfqpoint{3.243239in}{0.739656in}}%
\pgfpathlineto{\pgfqpoint{3.242943in}{0.739656in}}%
\pgfpathlineto{\pgfqpoint{3.242647in}{0.739656in}}%
\pgfpathlineto{\pgfqpoint{3.242351in}{0.739656in}}%
\pgfpathlineto{\pgfqpoint{3.242055in}{0.739656in}}%
\pgfpathlineto{\pgfqpoint{3.241759in}{0.739656in}}%
\pgfpathlineto{\pgfqpoint{3.241463in}{0.739656in}}%
\pgfpathlineto{\pgfqpoint{3.241167in}{0.739656in}}%
\pgfpathlineto{\pgfqpoint{3.240871in}{0.739656in}}%
\pgfpathlineto{\pgfqpoint{3.240575in}{0.739656in}}%
\pgfpathlineto{\pgfqpoint{3.240279in}{0.739656in}}%
\pgfpathlineto{\pgfqpoint{3.239983in}{0.739656in}}%
\pgfpathlineto{\pgfqpoint{3.239687in}{0.739656in}}%
\pgfpathlineto{\pgfqpoint{3.239391in}{0.739656in}}%
\pgfpathlineto{\pgfqpoint{3.239095in}{0.739656in}}%
\pgfpathlineto{\pgfqpoint{3.238799in}{0.739656in}}%
\pgfpathlineto{\pgfqpoint{3.238503in}{0.739656in}}%
\pgfpathlineto{\pgfqpoint{3.238207in}{0.739656in}}%
\pgfpathlineto{\pgfqpoint{3.237911in}{0.739656in}}%
\pgfpathlineto{\pgfqpoint{3.237615in}{0.739656in}}%
\pgfpathlineto{\pgfqpoint{3.237319in}{0.739656in}}%
\pgfpathlineto{\pgfqpoint{3.237023in}{0.739656in}}%
\pgfpathlineto{\pgfqpoint{3.236727in}{0.739656in}}%
\pgfpathlineto{\pgfqpoint{3.236431in}{0.739656in}}%
\pgfpathlineto{\pgfqpoint{3.236135in}{0.739656in}}%
\pgfpathlineto{\pgfqpoint{3.235839in}{0.739656in}}%
\pgfpathlineto{\pgfqpoint{3.235543in}{0.739656in}}%
\pgfpathlineto{\pgfqpoint{3.235247in}{0.739656in}}%
\pgfpathlineto{\pgfqpoint{3.234951in}{0.739656in}}%
\pgfpathlineto{\pgfqpoint{3.234655in}{0.739656in}}%
\pgfpathlineto{\pgfqpoint{3.234359in}{0.739656in}}%
\pgfpathlineto{\pgfqpoint{3.234063in}{0.739656in}}%
\pgfpathlineto{\pgfqpoint{3.233767in}{0.739656in}}%
\pgfpathlineto{\pgfqpoint{3.233471in}{0.739656in}}%
\pgfpathlineto{\pgfqpoint{3.233175in}{0.739656in}}%
\pgfpathlineto{\pgfqpoint{3.232879in}{0.739656in}}%
\pgfpathlineto{\pgfqpoint{3.232583in}{0.739656in}}%
\pgfpathlineto{\pgfqpoint{3.232287in}{0.739656in}}%
\pgfpathlineto{\pgfqpoint{3.231991in}{0.739656in}}%
\pgfpathlineto{\pgfqpoint{3.231695in}{0.739656in}}%
\pgfpathlineto{\pgfqpoint{3.231399in}{0.739656in}}%
\pgfpathlineto{\pgfqpoint{3.231103in}{0.739656in}}%
\pgfpathlineto{\pgfqpoint{3.230807in}{0.739656in}}%
\pgfpathlineto{\pgfqpoint{3.230511in}{0.739656in}}%
\pgfpathlineto{\pgfqpoint{3.230215in}{0.739656in}}%
\pgfpathlineto{\pgfqpoint{3.229919in}{0.739656in}}%
\pgfpathlineto{\pgfqpoint{3.229623in}{0.739656in}}%
\pgfpathlineto{\pgfqpoint{3.229327in}{0.739656in}}%
\pgfpathlineto{\pgfqpoint{3.229031in}{0.739656in}}%
\pgfpathlineto{\pgfqpoint{3.228735in}{0.739656in}}%
\pgfpathlineto{\pgfqpoint{3.228439in}{0.739656in}}%
\pgfpathlineto{\pgfqpoint{3.228143in}{0.739656in}}%
\pgfpathlineto{\pgfqpoint{3.227847in}{0.739656in}}%
\pgfpathlineto{\pgfqpoint{3.227551in}{0.739656in}}%
\pgfpathlineto{\pgfqpoint{3.227255in}{0.739656in}}%
\pgfpathlineto{\pgfqpoint{3.226959in}{0.739656in}}%
\pgfpathlineto{\pgfqpoint{3.226663in}{0.739656in}}%
\pgfpathlineto{\pgfqpoint{3.226367in}{0.739656in}}%
\pgfpathlineto{\pgfqpoint{3.226071in}{0.739656in}}%
\pgfpathlineto{\pgfqpoint{3.225775in}{0.739656in}}%
\pgfpathlineto{\pgfqpoint{3.225479in}{0.739656in}}%
\pgfpathlineto{\pgfqpoint{3.225183in}{0.739656in}}%
\pgfpathlineto{\pgfqpoint{3.224887in}{0.739656in}}%
\pgfpathlineto{\pgfqpoint{3.224591in}{0.739656in}}%
\pgfpathlineto{\pgfqpoint{3.224295in}{0.739656in}}%
\pgfpathlineto{\pgfqpoint{3.223999in}{0.739656in}}%
\pgfpathlineto{\pgfqpoint{3.223703in}{0.739656in}}%
\pgfpathlineto{\pgfqpoint{3.223407in}{0.739656in}}%
\pgfpathlineto{\pgfqpoint{3.223111in}{0.739656in}}%
\pgfpathlineto{\pgfqpoint{3.222815in}{0.739656in}}%
\pgfpathlineto{\pgfqpoint{3.222519in}{0.739656in}}%
\pgfpathlineto{\pgfqpoint{3.222223in}{0.739656in}}%
\pgfpathlineto{\pgfqpoint{3.221927in}{0.739656in}}%
\pgfpathlineto{\pgfqpoint{3.221631in}{0.739656in}}%
\pgfpathlineto{\pgfqpoint{3.221335in}{0.739656in}}%
\pgfpathlineto{\pgfqpoint{3.221039in}{0.739656in}}%
\pgfpathlineto{\pgfqpoint{3.220743in}{0.739656in}}%
\pgfpathlineto{\pgfqpoint{3.220447in}{0.739656in}}%
\pgfpathlineto{\pgfqpoint{3.220151in}{0.739656in}}%
\pgfpathlineto{\pgfqpoint{3.219855in}{0.739656in}}%
\pgfpathlineto{\pgfqpoint{3.219559in}{0.739656in}}%
\pgfpathlineto{\pgfqpoint{3.219263in}{0.739656in}}%
\pgfpathlineto{\pgfqpoint{3.218967in}{0.739656in}}%
\pgfpathlineto{\pgfqpoint{3.218671in}{0.739656in}}%
\pgfpathlineto{\pgfqpoint{3.218375in}{0.739656in}}%
\pgfpathlineto{\pgfqpoint{3.218079in}{0.739656in}}%
\pgfpathlineto{\pgfqpoint{3.217783in}{0.739656in}}%
\pgfpathlineto{\pgfqpoint{3.217487in}{0.739656in}}%
\pgfpathlineto{\pgfqpoint{3.217191in}{0.739656in}}%
\pgfpathlineto{\pgfqpoint{3.216895in}{0.739656in}}%
\pgfpathlineto{\pgfqpoint{3.216599in}{0.739656in}}%
\pgfpathlineto{\pgfqpoint{3.216303in}{0.739656in}}%
\pgfpathlineto{\pgfqpoint{3.216007in}{0.739656in}}%
\pgfpathlineto{\pgfqpoint{3.215711in}{0.739656in}}%
\pgfpathlineto{\pgfqpoint{3.215415in}{0.739656in}}%
\pgfpathlineto{\pgfqpoint{3.215119in}{0.739656in}}%
\pgfpathlineto{\pgfqpoint{3.214823in}{0.739656in}}%
\pgfpathlineto{\pgfqpoint{3.214527in}{0.739656in}}%
\pgfpathlineto{\pgfqpoint{3.214231in}{0.739656in}}%
\pgfpathlineto{\pgfqpoint{3.213935in}{0.739656in}}%
\pgfpathlineto{\pgfqpoint{3.213639in}{0.739656in}}%
\pgfpathlineto{\pgfqpoint{3.213343in}{0.739656in}}%
\pgfpathlineto{\pgfqpoint{3.213047in}{0.739656in}}%
\pgfpathlineto{\pgfqpoint{3.212751in}{0.739656in}}%
\pgfpathlineto{\pgfqpoint{3.212455in}{0.739656in}}%
\pgfpathlineto{\pgfqpoint{3.212159in}{0.739656in}}%
\pgfpathlineto{\pgfqpoint{3.211863in}{0.739656in}}%
\pgfpathlineto{\pgfqpoint{3.211567in}{0.739656in}}%
\pgfpathlineto{\pgfqpoint{3.211271in}{0.739656in}}%
\pgfpathlineto{\pgfqpoint{3.210975in}{0.739656in}}%
\pgfpathlineto{\pgfqpoint{3.210679in}{0.739656in}}%
\pgfpathlineto{\pgfqpoint{3.210383in}{0.739656in}}%
\pgfpathlineto{\pgfqpoint{3.210087in}{0.739656in}}%
\pgfpathlineto{\pgfqpoint{3.209791in}{0.739656in}}%
\pgfpathlineto{\pgfqpoint{3.209494in}{0.739656in}}%
\pgfpathlineto{\pgfqpoint{3.209198in}{0.739656in}}%
\pgfpathlineto{\pgfqpoint{3.208902in}{0.739656in}}%
\pgfpathlineto{\pgfqpoint{3.208606in}{0.739656in}}%
\pgfpathlineto{\pgfqpoint{3.208310in}{0.739656in}}%
\pgfpathlineto{\pgfqpoint{3.208014in}{0.739656in}}%
\pgfpathlineto{\pgfqpoint{3.207718in}{0.739656in}}%
\pgfpathlineto{\pgfqpoint{3.207422in}{0.739656in}}%
\pgfpathlineto{\pgfqpoint{3.207126in}{0.739656in}}%
\pgfpathlineto{\pgfqpoint{3.206830in}{0.739656in}}%
\pgfpathlineto{\pgfqpoint{3.206534in}{0.739656in}}%
\pgfpathlineto{\pgfqpoint{3.206238in}{0.739656in}}%
\pgfpathlineto{\pgfqpoint{3.205942in}{0.739656in}}%
\pgfpathlineto{\pgfqpoint{3.205646in}{0.739656in}}%
\pgfpathlineto{\pgfqpoint{3.205350in}{0.739656in}}%
\pgfpathlineto{\pgfqpoint{3.205054in}{0.739656in}}%
\pgfpathlineto{\pgfqpoint{3.204758in}{0.739656in}}%
\pgfpathlineto{\pgfqpoint{3.204462in}{0.739656in}}%
\pgfpathlineto{\pgfqpoint{3.204166in}{0.739656in}}%
\pgfpathlineto{\pgfqpoint{3.203870in}{0.739656in}}%
\pgfpathlineto{\pgfqpoint{3.203574in}{0.739656in}}%
\pgfpathlineto{\pgfqpoint{3.203278in}{0.739656in}}%
\pgfpathlineto{\pgfqpoint{3.202982in}{0.739656in}}%
\pgfpathlineto{\pgfqpoint{3.202686in}{0.739656in}}%
\pgfpathlineto{\pgfqpoint{3.202390in}{0.739656in}}%
\pgfpathlineto{\pgfqpoint{3.202094in}{0.739656in}}%
\pgfpathlineto{\pgfqpoint{3.201798in}{0.739656in}}%
\pgfpathlineto{\pgfqpoint{3.201502in}{0.739656in}}%
\pgfpathlineto{\pgfqpoint{3.201206in}{0.739656in}}%
\pgfpathlineto{\pgfqpoint{3.200910in}{0.739656in}}%
\pgfpathlineto{\pgfqpoint{3.200614in}{0.739656in}}%
\pgfpathlineto{\pgfqpoint{3.200318in}{0.739656in}}%
\pgfpathlineto{\pgfqpoint{3.200022in}{0.739656in}}%
\pgfpathlineto{\pgfqpoint{3.199726in}{0.739656in}}%
\pgfpathlineto{\pgfqpoint{3.199430in}{0.739656in}}%
\pgfpathlineto{\pgfqpoint{3.199134in}{0.739656in}}%
\pgfpathlineto{\pgfqpoint{3.198838in}{0.739656in}}%
\pgfpathlineto{\pgfqpoint{3.198542in}{0.739656in}}%
\pgfpathlineto{\pgfqpoint{3.198246in}{0.739656in}}%
\pgfpathlineto{\pgfqpoint{3.197950in}{0.739656in}}%
\pgfpathlineto{\pgfqpoint{3.197654in}{0.739656in}}%
\pgfpathlineto{\pgfqpoint{3.197358in}{0.739656in}}%
\pgfpathlineto{\pgfqpoint{3.197062in}{0.739656in}}%
\pgfpathlineto{\pgfqpoint{3.196766in}{0.739656in}}%
\pgfpathlineto{\pgfqpoint{3.196470in}{0.739656in}}%
\pgfpathlineto{\pgfqpoint{3.196174in}{0.739656in}}%
\pgfpathlineto{\pgfqpoint{3.195878in}{0.739656in}}%
\pgfpathlineto{\pgfqpoint{3.195582in}{0.739656in}}%
\pgfpathlineto{\pgfqpoint{3.195286in}{0.739656in}}%
\pgfpathlineto{\pgfqpoint{3.194990in}{0.739656in}}%
\pgfpathlineto{\pgfqpoint{3.194694in}{0.739656in}}%
\pgfpathlineto{\pgfqpoint{3.194398in}{0.739656in}}%
\pgfpathlineto{\pgfqpoint{3.194102in}{0.739656in}}%
\pgfpathlineto{\pgfqpoint{3.193806in}{0.739656in}}%
\pgfpathlineto{\pgfqpoint{3.193510in}{0.739656in}}%
\pgfpathlineto{\pgfqpoint{3.193214in}{0.739656in}}%
\pgfpathlineto{\pgfqpoint{3.192918in}{0.739656in}}%
\pgfpathlineto{\pgfqpoint{3.192622in}{0.739656in}}%
\pgfpathlineto{\pgfqpoint{3.192326in}{0.739656in}}%
\pgfpathlineto{\pgfqpoint{3.192030in}{0.739656in}}%
\pgfpathlineto{\pgfqpoint{3.191734in}{0.739656in}}%
\pgfpathlineto{\pgfqpoint{3.191438in}{0.739656in}}%
\pgfpathlineto{\pgfqpoint{3.191142in}{0.739656in}}%
\pgfpathlineto{\pgfqpoint{3.190846in}{0.739656in}}%
\pgfpathlineto{\pgfqpoint{3.190550in}{0.739656in}}%
\pgfpathlineto{\pgfqpoint{3.190254in}{0.739656in}}%
\pgfpathlineto{\pgfqpoint{3.189958in}{0.739656in}}%
\pgfpathlineto{\pgfqpoint{3.189662in}{0.739656in}}%
\pgfpathlineto{\pgfqpoint{3.189366in}{0.739656in}}%
\pgfpathlineto{\pgfqpoint{3.189070in}{0.739656in}}%
\pgfpathlineto{\pgfqpoint{3.188774in}{0.739656in}}%
\pgfpathlineto{\pgfqpoint{3.188478in}{0.739656in}}%
\pgfpathlineto{\pgfqpoint{3.188182in}{0.739656in}}%
\pgfpathlineto{\pgfqpoint{3.187886in}{0.739656in}}%
\pgfpathlineto{\pgfqpoint{3.187590in}{0.739656in}}%
\pgfpathlineto{\pgfqpoint{3.187294in}{0.739656in}}%
\pgfpathlineto{\pgfqpoint{3.186998in}{0.739656in}}%
\pgfpathlineto{\pgfqpoint{3.186702in}{0.739656in}}%
\pgfpathlineto{\pgfqpoint{3.186406in}{0.739656in}}%
\pgfpathlineto{\pgfqpoint{3.186110in}{0.739656in}}%
\pgfpathlineto{\pgfqpoint{3.185814in}{0.739656in}}%
\pgfpathlineto{\pgfqpoint{3.185518in}{0.739656in}}%
\pgfpathlineto{\pgfqpoint{3.185222in}{0.739656in}}%
\pgfpathlineto{\pgfqpoint{3.184926in}{0.739656in}}%
\pgfpathlineto{\pgfqpoint{3.184630in}{0.739656in}}%
\pgfpathlineto{\pgfqpoint{3.184334in}{0.739656in}}%
\pgfpathlineto{\pgfqpoint{3.184038in}{0.739656in}}%
\pgfpathlineto{\pgfqpoint{3.183742in}{0.739656in}}%
\pgfpathlineto{\pgfqpoint{3.183446in}{0.739656in}}%
\pgfpathlineto{\pgfqpoint{3.183150in}{0.739656in}}%
\pgfpathlineto{\pgfqpoint{3.182854in}{0.739656in}}%
\pgfpathlineto{\pgfqpoint{3.182558in}{0.739656in}}%
\pgfpathlineto{\pgfqpoint{3.182262in}{0.739656in}}%
\pgfpathlineto{\pgfqpoint{3.181966in}{0.739656in}}%
\pgfpathlineto{\pgfqpoint{3.181670in}{0.739656in}}%
\pgfpathlineto{\pgfqpoint{3.181374in}{0.739656in}}%
\pgfpathlineto{\pgfqpoint{3.181078in}{0.739656in}}%
\pgfpathlineto{\pgfqpoint{3.180782in}{0.739656in}}%
\pgfpathlineto{\pgfqpoint{3.180486in}{0.739656in}}%
\pgfpathlineto{\pgfqpoint{3.180190in}{0.739656in}}%
\pgfpathlineto{\pgfqpoint{3.179894in}{0.739656in}}%
\pgfpathlineto{\pgfqpoint{3.179598in}{0.739656in}}%
\pgfpathlineto{\pgfqpoint{3.179302in}{0.739656in}}%
\pgfpathlineto{\pgfqpoint{3.179006in}{0.739656in}}%
\pgfpathlineto{\pgfqpoint{3.178710in}{0.739656in}}%
\pgfpathlineto{\pgfqpoint{3.178414in}{0.739656in}}%
\pgfpathlineto{\pgfqpoint{3.178118in}{0.739656in}}%
\pgfpathlineto{\pgfqpoint{3.177822in}{0.739656in}}%
\pgfpathlineto{\pgfqpoint{3.177526in}{0.739656in}}%
\pgfpathlineto{\pgfqpoint{3.177230in}{0.739656in}}%
\pgfpathlineto{\pgfqpoint{3.176934in}{0.739656in}}%
\pgfpathlineto{\pgfqpoint{3.176638in}{0.739656in}}%
\pgfpathlineto{\pgfqpoint{3.176342in}{0.739656in}}%
\pgfpathlineto{\pgfqpoint{3.176046in}{0.739656in}}%
\pgfpathlineto{\pgfqpoint{3.175750in}{0.739656in}}%
\pgfpathlineto{\pgfqpoint{3.175454in}{0.739656in}}%
\pgfpathlineto{\pgfqpoint{3.175158in}{0.739656in}}%
\pgfpathlineto{\pgfqpoint{3.174862in}{0.739656in}}%
\pgfpathlineto{\pgfqpoint{3.174566in}{0.739656in}}%
\pgfpathlineto{\pgfqpoint{3.174270in}{0.739656in}}%
\pgfpathlineto{\pgfqpoint{3.173974in}{0.739656in}}%
\pgfpathlineto{\pgfqpoint{3.173678in}{0.739656in}}%
\pgfpathlineto{\pgfqpoint{3.173382in}{0.739656in}}%
\pgfpathlineto{\pgfqpoint{3.173086in}{0.739656in}}%
\pgfpathlineto{\pgfqpoint{3.172790in}{0.739656in}}%
\pgfpathlineto{\pgfqpoint{3.172494in}{0.739656in}}%
\pgfpathlineto{\pgfqpoint{3.172198in}{0.739656in}}%
\pgfpathlineto{\pgfqpoint{3.171902in}{0.739656in}}%
\pgfpathlineto{\pgfqpoint{3.171606in}{0.739656in}}%
\pgfpathlineto{\pgfqpoint{3.171310in}{0.739656in}}%
\pgfpathlineto{\pgfqpoint{3.171014in}{0.739656in}}%
\pgfpathlineto{\pgfqpoint{3.170718in}{0.739656in}}%
\pgfpathlineto{\pgfqpoint{3.170422in}{0.739656in}}%
\pgfpathlineto{\pgfqpoint{3.170126in}{0.739656in}}%
\pgfpathlineto{\pgfqpoint{3.169830in}{0.739656in}}%
\pgfpathlineto{\pgfqpoint{3.169534in}{0.739656in}}%
\pgfpathlineto{\pgfqpoint{3.169238in}{0.739656in}}%
\pgfpathlineto{\pgfqpoint{3.168942in}{0.739656in}}%
\pgfpathlineto{\pgfqpoint{3.168646in}{0.739656in}}%
\pgfpathlineto{\pgfqpoint{3.168350in}{0.739656in}}%
\pgfpathlineto{\pgfqpoint{3.168054in}{0.739656in}}%
\pgfpathlineto{\pgfqpoint{3.167758in}{0.739656in}}%
\pgfpathlineto{\pgfqpoint{3.167462in}{0.739656in}}%
\pgfpathlineto{\pgfqpoint{3.167166in}{0.739656in}}%
\pgfpathlineto{\pgfqpoint{3.166870in}{0.739656in}}%
\pgfpathlineto{\pgfqpoint{3.166574in}{0.739656in}}%
\pgfpathlineto{\pgfqpoint{3.166278in}{0.739656in}}%
\pgfpathlineto{\pgfqpoint{3.165982in}{0.739656in}}%
\pgfpathlineto{\pgfqpoint{3.165686in}{0.739656in}}%
\pgfpathlineto{\pgfqpoint{3.165390in}{0.739656in}}%
\pgfpathlineto{\pgfqpoint{3.165094in}{0.739656in}}%
\pgfpathlineto{\pgfqpoint{3.164798in}{0.739656in}}%
\pgfpathlineto{\pgfqpoint{3.164502in}{0.739656in}}%
\pgfpathlineto{\pgfqpoint{3.164206in}{0.739656in}}%
\pgfpathlineto{\pgfqpoint{3.163910in}{0.739656in}}%
\pgfpathlineto{\pgfqpoint{3.163614in}{0.739656in}}%
\pgfpathlineto{\pgfqpoint{3.163318in}{0.739656in}}%
\pgfpathlineto{\pgfqpoint{3.163022in}{0.739656in}}%
\pgfpathlineto{\pgfqpoint{3.162726in}{0.739656in}}%
\pgfpathlineto{\pgfqpoint{3.162430in}{0.739656in}}%
\pgfpathlineto{\pgfqpoint{3.162134in}{0.739656in}}%
\pgfpathlineto{\pgfqpoint{3.161838in}{0.739656in}}%
\pgfpathlineto{\pgfqpoint{3.161542in}{0.739656in}}%
\pgfpathlineto{\pgfqpoint{3.161246in}{0.739656in}}%
\pgfpathlineto{\pgfqpoint{3.160950in}{0.739656in}}%
\pgfpathlineto{\pgfqpoint{3.160654in}{0.739656in}}%
\pgfpathlineto{\pgfqpoint{3.160358in}{0.739656in}}%
\pgfpathlineto{\pgfqpoint{3.160062in}{0.739656in}}%
\pgfpathlineto{\pgfqpoint{3.159766in}{0.739656in}}%
\pgfpathlineto{\pgfqpoint{3.159470in}{0.739656in}}%
\pgfpathlineto{\pgfqpoint{3.159174in}{0.739656in}}%
\pgfpathlineto{\pgfqpoint{3.158878in}{0.739656in}}%
\pgfpathlineto{\pgfqpoint{3.158582in}{0.739656in}}%
\pgfpathlineto{\pgfqpoint{3.158286in}{0.739656in}}%
\pgfpathlineto{\pgfqpoint{3.157990in}{0.739656in}}%
\pgfpathlineto{\pgfqpoint{3.157694in}{0.739656in}}%
\pgfpathlineto{\pgfqpoint{3.157398in}{0.739656in}}%
\pgfpathlineto{\pgfqpoint{3.157102in}{0.739656in}}%
\pgfpathlineto{\pgfqpoint{3.156806in}{0.739656in}}%
\pgfpathlineto{\pgfqpoint{3.156510in}{0.739656in}}%
\pgfpathlineto{\pgfqpoint{3.156214in}{0.739656in}}%
\pgfpathlineto{\pgfqpoint{3.155918in}{0.739656in}}%
\pgfpathlineto{\pgfqpoint{3.155622in}{0.739656in}}%
\pgfpathlineto{\pgfqpoint{3.155326in}{0.739656in}}%
\pgfpathlineto{\pgfqpoint{3.155030in}{0.739656in}}%
\pgfpathlineto{\pgfqpoint{3.154734in}{0.739656in}}%
\pgfpathlineto{\pgfqpoint{3.154438in}{0.739656in}}%
\pgfpathlineto{\pgfqpoint{3.154142in}{0.739656in}}%
\pgfpathlineto{\pgfqpoint{3.153846in}{0.739656in}}%
\pgfpathlineto{\pgfqpoint{3.153550in}{0.739656in}}%
\pgfpathlineto{\pgfqpoint{3.153254in}{0.739656in}}%
\pgfpathlineto{\pgfqpoint{3.152958in}{0.739656in}}%
\pgfpathlineto{\pgfqpoint{3.152662in}{0.739656in}}%
\pgfpathlineto{\pgfqpoint{3.152366in}{0.739656in}}%
\pgfpathlineto{\pgfqpoint{3.152070in}{0.739656in}}%
\pgfpathlineto{\pgfqpoint{3.151774in}{0.739656in}}%
\pgfpathlineto{\pgfqpoint{3.151478in}{0.739656in}}%
\pgfpathlineto{\pgfqpoint{3.151182in}{0.739656in}}%
\pgfpathlineto{\pgfqpoint{3.150886in}{0.739656in}}%
\pgfpathlineto{\pgfqpoint{3.150590in}{0.739656in}}%
\pgfpathlineto{\pgfqpoint{3.150294in}{0.739656in}}%
\pgfpathlineto{\pgfqpoint{3.149998in}{0.739656in}}%
\pgfpathlineto{\pgfqpoint{3.149702in}{0.739656in}}%
\pgfpathlineto{\pgfqpoint{3.149406in}{0.739656in}}%
\pgfpathlineto{\pgfqpoint{3.149110in}{0.739656in}}%
\pgfpathlineto{\pgfqpoint{3.148814in}{0.739656in}}%
\pgfpathlineto{\pgfqpoint{3.148518in}{0.739656in}}%
\pgfpathlineto{\pgfqpoint{3.148222in}{0.739656in}}%
\pgfpathlineto{\pgfqpoint{3.147926in}{0.739656in}}%
\pgfpathlineto{\pgfqpoint{3.147630in}{0.739656in}}%
\pgfpathlineto{\pgfqpoint{3.147334in}{0.739656in}}%
\pgfpathlineto{\pgfqpoint{3.147038in}{0.739656in}}%
\pgfpathlineto{\pgfqpoint{3.146742in}{0.739656in}}%
\pgfpathlineto{\pgfqpoint{3.146446in}{0.739656in}}%
\pgfpathlineto{\pgfqpoint{3.146150in}{0.739656in}}%
\pgfpathlineto{\pgfqpoint{3.145854in}{0.739656in}}%
\pgfpathlineto{\pgfqpoint{3.145558in}{0.739656in}}%
\pgfpathlineto{\pgfqpoint{3.145262in}{0.739656in}}%
\pgfpathlineto{\pgfqpoint{3.144966in}{0.739656in}}%
\pgfpathlineto{\pgfqpoint{3.144670in}{0.739656in}}%
\pgfpathlineto{\pgfqpoint{3.144374in}{0.739656in}}%
\pgfpathlineto{\pgfqpoint{3.144078in}{0.739656in}}%
\pgfpathlineto{\pgfqpoint{3.143782in}{0.739656in}}%
\pgfpathlineto{\pgfqpoint{3.143486in}{0.739656in}}%
\pgfpathlineto{\pgfqpoint{3.143190in}{0.739656in}}%
\pgfpathlineto{\pgfqpoint{3.142894in}{0.739656in}}%
\pgfpathlineto{\pgfqpoint{3.142598in}{0.739656in}}%
\pgfpathlineto{\pgfqpoint{3.142301in}{0.739656in}}%
\pgfpathlineto{\pgfqpoint{3.142005in}{0.739656in}}%
\pgfpathlineto{\pgfqpoint{3.141709in}{0.739656in}}%
\pgfpathlineto{\pgfqpoint{3.141413in}{0.739656in}}%
\pgfpathlineto{\pgfqpoint{3.141117in}{0.739656in}}%
\pgfpathlineto{\pgfqpoint{3.140821in}{0.739656in}}%
\pgfpathlineto{\pgfqpoint{3.140525in}{0.739656in}}%
\pgfpathlineto{\pgfqpoint{3.140229in}{0.739656in}}%
\pgfpathlineto{\pgfqpoint{3.139933in}{0.739656in}}%
\pgfpathlineto{\pgfqpoint{3.139637in}{0.739656in}}%
\pgfpathlineto{\pgfqpoint{3.139341in}{0.739656in}}%
\pgfpathlineto{\pgfqpoint{3.139045in}{0.739656in}}%
\pgfpathlineto{\pgfqpoint{3.138749in}{0.739656in}}%
\pgfpathlineto{\pgfqpoint{3.138453in}{0.739656in}}%
\pgfpathlineto{\pgfqpoint{3.138157in}{0.739656in}}%
\pgfpathlineto{\pgfqpoint{3.137861in}{0.739656in}}%
\pgfpathlineto{\pgfqpoint{3.137565in}{0.739656in}}%
\pgfpathlineto{\pgfqpoint{3.137269in}{0.739656in}}%
\pgfpathlineto{\pgfqpoint{3.136973in}{0.739656in}}%
\pgfpathlineto{\pgfqpoint{3.136677in}{0.739656in}}%
\pgfpathlineto{\pgfqpoint{3.136381in}{0.739656in}}%
\pgfpathlineto{\pgfqpoint{3.136085in}{0.739656in}}%
\pgfpathlineto{\pgfqpoint{3.135789in}{0.739656in}}%
\pgfpathlineto{\pgfqpoint{3.135493in}{0.739656in}}%
\pgfpathlineto{\pgfqpoint{3.135197in}{0.739656in}}%
\pgfpathlineto{\pgfqpoint{3.134901in}{0.739656in}}%
\pgfpathlineto{\pgfqpoint{3.134605in}{0.739656in}}%
\pgfpathlineto{\pgfqpoint{3.134309in}{0.739656in}}%
\pgfpathlineto{\pgfqpoint{3.134013in}{0.739656in}}%
\pgfpathlineto{\pgfqpoint{3.133717in}{0.739656in}}%
\pgfpathlineto{\pgfqpoint{3.133421in}{0.739656in}}%
\pgfpathlineto{\pgfqpoint{3.133125in}{0.739656in}}%
\pgfpathlineto{\pgfqpoint{3.132829in}{0.739656in}}%
\pgfpathlineto{\pgfqpoint{3.132533in}{0.739656in}}%
\pgfpathlineto{\pgfqpoint{3.132237in}{0.739656in}}%
\pgfpathlineto{\pgfqpoint{3.131941in}{0.739656in}}%
\pgfpathlineto{\pgfqpoint{3.131645in}{0.739656in}}%
\pgfpathlineto{\pgfqpoint{3.131349in}{0.739656in}}%
\pgfpathlineto{\pgfqpoint{3.131053in}{0.739656in}}%
\pgfpathlineto{\pgfqpoint{3.130757in}{0.739656in}}%
\pgfpathlineto{\pgfqpoint{3.130461in}{0.739656in}}%
\pgfpathlineto{\pgfqpoint{3.130165in}{0.739656in}}%
\pgfpathlineto{\pgfqpoint{3.129869in}{0.739656in}}%
\pgfpathlineto{\pgfqpoint{3.129573in}{0.739656in}}%
\pgfpathlineto{\pgfqpoint{3.129277in}{0.739656in}}%
\pgfpathlineto{\pgfqpoint{3.128981in}{0.739656in}}%
\pgfpathlineto{\pgfqpoint{3.128685in}{0.739656in}}%
\pgfpathlineto{\pgfqpoint{3.128389in}{0.739656in}}%
\pgfpathlineto{\pgfqpoint{3.128093in}{0.739656in}}%
\pgfpathlineto{\pgfqpoint{3.127797in}{0.739656in}}%
\pgfpathlineto{\pgfqpoint{3.127501in}{0.739656in}}%
\pgfpathlineto{\pgfqpoint{3.127205in}{0.739656in}}%
\pgfpathlineto{\pgfqpoint{3.126909in}{0.739656in}}%
\pgfpathlineto{\pgfqpoint{3.126613in}{0.739656in}}%
\pgfpathlineto{\pgfqpoint{3.126317in}{0.739656in}}%
\pgfpathlineto{\pgfqpoint{3.126021in}{0.739656in}}%
\pgfpathlineto{\pgfqpoint{3.125725in}{0.739656in}}%
\pgfpathlineto{\pgfqpoint{3.125429in}{0.739656in}}%
\pgfpathlineto{\pgfqpoint{3.125133in}{0.739656in}}%
\pgfpathlineto{\pgfqpoint{3.124837in}{0.739656in}}%
\pgfpathlineto{\pgfqpoint{3.124541in}{0.739656in}}%
\pgfpathlineto{\pgfqpoint{3.124245in}{0.739656in}}%
\pgfpathlineto{\pgfqpoint{3.123949in}{0.739656in}}%
\pgfpathlineto{\pgfqpoint{3.123653in}{0.739656in}}%
\pgfpathlineto{\pgfqpoint{3.123357in}{0.739656in}}%
\pgfpathlineto{\pgfqpoint{3.123061in}{0.739656in}}%
\pgfpathlineto{\pgfqpoint{3.122765in}{0.739656in}}%
\pgfpathlineto{\pgfqpoint{3.122469in}{0.739656in}}%
\pgfpathlineto{\pgfqpoint{3.122173in}{0.739656in}}%
\pgfpathlineto{\pgfqpoint{3.121877in}{0.739656in}}%
\pgfpathlineto{\pgfqpoint{3.121581in}{0.739656in}}%
\pgfpathlineto{\pgfqpoint{3.121285in}{0.739656in}}%
\pgfpathlineto{\pgfqpoint{3.120989in}{0.739656in}}%
\pgfpathlineto{\pgfqpoint{3.120693in}{0.739656in}}%
\pgfpathlineto{\pgfqpoint{3.120397in}{0.739656in}}%
\pgfpathlineto{\pgfqpoint{3.120101in}{0.739656in}}%
\pgfpathlineto{\pgfqpoint{3.119805in}{0.739656in}}%
\pgfpathlineto{\pgfqpoint{3.119509in}{0.739656in}}%
\pgfpathlineto{\pgfqpoint{3.119213in}{0.739656in}}%
\pgfpathlineto{\pgfqpoint{3.118917in}{0.739656in}}%
\pgfpathlineto{\pgfqpoint{3.118621in}{0.739656in}}%
\pgfpathlineto{\pgfqpoint{3.118325in}{0.739656in}}%
\pgfpathlineto{\pgfqpoint{3.118029in}{0.739656in}}%
\pgfpathlineto{\pgfqpoint{3.117733in}{0.739656in}}%
\pgfpathlineto{\pgfqpoint{3.117437in}{0.739656in}}%
\pgfpathlineto{\pgfqpoint{3.117141in}{0.739656in}}%
\pgfpathlineto{\pgfqpoint{3.116845in}{0.739656in}}%
\pgfpathlineto{\pgfqpoint{3.116549in}{0.739656in}}%
\pgfpathlineto{\pgfqpoint{3.116253in}{0.739656in}}%
\pgfpathlineto{\pgfqpoint{3.115957in}{0.739656in}}%
\pgfpathlineto{\pgfqpoint{3.115661in}{0.739656in}}%
\pgfpathlineto{\pgfqpoint{3.115365in}{0.739656in}}%
\pgfpathlineto{\pgfqpoint{3.115069in}{0.739656in}}%
\pgfpathlineto{\pgfqpoint{3.114773in}{0.739656in}}%
\pgfpathlineto{\pgfqpoint{3.114477in}{0.739656in}}%
\pgfpathlineto{\pgfqpoint{3.114181in}{0.739656in}}%
\pgfpathlineto{\pgfqpoint{3.113885in}{0.739656in}}%
\pgfpathlineto{\pgfqpoint{3.113589in}{0.739656in}}%
\pgfpathlineto{\pgfqpoint{3.113293in}{0.739656in}}%
\pgfpathlineto{\pgfqpoint{3.112997in}{0.739656in}}%
\pgfpathlineto{\pgfqpoint{3.112701in}{0.739656in}}%
\pgfpathlineto{\pgfqpoint{3.112405in}{0.739656in}}%
\pgfpathlineto{\pgfqpoint{3.112109in}{0.739656in}}%
\pgfpathlineto{\pgfqpoint{3.111813in}{0.739656in}}%
\pgfpathlineto{\pgfqpoint{3.111517in}{0.739656in}}%
\pgfpathlineto{\pgfqpoint{3.111221in}{0.739656in}}%
\pgfpathlineto{\pgfqpoint{3.110925in}{0.739656in}}%
\pgfpathlineto{\pgfqpoint{3.110629in}{0.739656in}}%
\pgfpathlineto{\pgfqpoint{3.110333in}{0.739656in}}%
\pgfpathlineto{\pgfqpoint{3.110037in}{0.739656in}}%
\pgfpathlineto{\pgfqpoint{3.109741in}{0.739656in}}%
\pgfpathlineto{\pgfqpoint{3.109445in}{0.739656in}}%
\pgfpathlineto{\pgfqpoint{3.109149in}{0.739656in}}%
\pgfpathlineto{\pgfqpoint{3.108853in}{0.739656in}}%
\pgfpathlineto{\pgfqpoint{3.108557in}{0.739656in}}%
\pgfpathlineto{\pgfqpoint{3.108261in}{0.739656in}}%
\pgfpathlineto{\pgfqpoint{3.107965in}{0.739656in}}%
\pgfpathlineto{\pgfqpoint{3.107669in}{0.739656in}}%
\pgfpathlineto{\pgfqpoint{3.107373in}{0.739656in}}%
\pgfpathlineto{\pgfqpoint{3.107077in}{0.739656in}}%
\pgfpathlineto{\pgfqpoint{3.106781in}{0.739656in}}%
\pgfpathlineto{\pgfqpoint{3.106485in}{0.739656in}}%
\pgfpathlineto{\pgfqpoint{3.106189in}{0.739656in}}%
\pgfpathlineto{\pgfqpoint{3.105893in}{0.739656in}}%
\pgfpathlineto{\pgfqpoint{3.105597in}{0.739656in}}%
\pgfpathlineto{\pgfqpoint{3.105301in}{0.739656in}}%
\pgfpathlineto{\pgfqpoint{3.105005in}{0.739656in}}%
\pgfpathlineto{\pgfqpoint{3.104709in}{0.739656in}}%
\pgfpathlineto{\pgfqpoint{3.104413in}{0.739656in}}%
\pgfpathlineto{\pgfqpoint{3.104117in}{0.739656in}}%
\pgfpathlineto{\pgfqpoint{3.103821in}{0.739656in}}%
\pgfpathlineto{\pgfqpoint{3.103525in}{0.739656in}}%
\pgfpathlineto{\pgfqpoint{3.103229in}{0.739656in}}%
\pgfpathlineto{\pgfqpoint{3.102933in}{0.739656in}}%
\pgfpathlineto{\pgfqpoint{3.102637in}{0.739656in}}%
\pgfpathlineto{\pgfqpoint{3.102341in}{0.739656in}}%
\pgfpathlineto{\pgfqpoint{3.102045in}{0.739656in}}%
\pgfpathlineto{\pgfqpoint{3.101749in}{0.739656in}}%
\pgfpathlineto{\pgfqpoint{3.101453in}{0.739656in}}%
\pgfpathlineto{\pgfqpoint{3.101157in}{0.739656in}}%
\pgfpathlineto{\pgfqpoint{3.100861in}{0.739656in}}%
\pgfpathlineto{\pgfqpoint{3.100565in}{0.739656in}}%
\pgfpathlineto{\pgfqpoint{3.100269in}{0.739656in}}%
\pgfpathlineto{\pgfqpoint{3.099973in}{0.739656in}}%
\pgfpathlineto{\pgfqpoint{3.099677in}{0.739656in}}%
\pgfpathlineto{\pgfqpoint{3.099381in}{0.739656in}}%
\pgfpathlineto{\pgfqpoint{3.099085in}{0.739656in}}%
\pgfpathlineto{\pgfqpoint{3.098789in}{0.739656in}}%
\pgfpathlineto{\pgfqpoint{3.098493in}{0.739656in}}%
\pgfpathlineto{\pgfqpoint{3.098197in}{0.739656in}}%
\pgfpathlineto{\pgfqpoint{3.097901in}{0.739656in}}%
\pgfpathlineto{\pgfqpoint{3.097605in}{0.739656in}}%
\pgfpathlineto{\pgfqpoint{3.097309in}{0.739656in}}%
\pgfpathlineto{\pgfqpoint{3.097013in}{0.739656in}}%
\pgfpathlineto{\pgfqpoint{3.096717in}{0.739656in}}%
\pgfpathlineto{\pgfqpoint{3.096421in}{0.739656in}}%
\pgfpathlineto{\pgfqpoint{3.096125in}{0.739656in}}%
\pgfpathlineto{\pgfqpoint{3.095829in}{0.739656in}}%
\pgfpathlineto{\pgfqpoint{3.095533in}{0.739656in}}%
\pgfpathlineto{\pgfqpoint{3.095237in}{0.739656in}}%
\pgfpathlineto{\pgfqpoint{3.094941in}{0.739656in}}%
\pgfpathlineto{\pgfqpoint{3.094645in}{0.739656in}}%
\pgfpathlineto{\pgfqpoint{3.094349in}{0.739656in}}%
\pgfpathlineto{\pgfqpoint{3.094053in}{0.739656in}}%
\pgfpathlineto{\pgfqpoint{3.093757in}{0.739656in}}%
\pgfpathlineto{\pgfqpoint{3.093461in}{0.739656in}}%
\pgfpathlineto{\pgfqpoint{3.093165in}{0.739656in}}%
\pgfpathlineto{\pgfqpoint{3.092869in}{0.739656in}}%
\pgfpathlineto{\pgfqpoint{3.092573in}{0.739656in}}%
\pgfpathlineto{\pgfqpoint{3.092277in}{0.739656in}}%
\pgfpathlineto{\pgfqpoint{3.091981in}{0.739656in}}%
\pgfpathlineto{\pgfqpoint{3.091685in}{0.739656in}}%
\pgfpathlineto{\pgfqpoint{3.091389in}{0.739656in}}%
\pgfpathlineto{\pgfqpoint{3.091093in}{0.739656in}}%
\pgfpathlineto{\pgfqpoint{3.090797in}{0.739656in}}%
\pgfpathlineto{\pgfqpoint{3.090501in}{0.739656in}}%
\pgfpathlineto{\pgfqpoint{3.090205in}{0.739656in}}%
\pgfpathlineto{\pgfqpoint{3.089909in}{0.739656in}}%
\pgfpathlineto{\pgfqpoint{3.089613in}{0.739656in}}%
\pgfpathlineto{\pgfqpoint{3.089317in}{0.739656in}}%
\pgfpathlineto{\pgfqpoint{3.089021in}{0.739656in}}%
\pgfpathlineto{\pgfqpoint{3.088725in}{0.739656in}}%
\pgfpathlineto{\pgfqpoint{3.088429in}{0.739656in}}%
\pgfpathlineto{\pgfqpoint{3.088133in}{0.739656in}}%
\pgfpathlineto{\pgfqpoint{3.087837in}{0.739656in}}%
\pgfpathlineto{\pgfqpoint{3.087541in}{0.739656in}}%
\pgfpathlineto{\pgfqpoint{3.087245in}{0.739656in}}%
\pgfpathlineto{\pgfqpoint{3.086949in}{0.739656in}}%
\pgfpathlineto{\pgfqpoint{3.086653in}{0.739656in}}%
\pgfpathlineto{\pgfqpoint{3.086357in}{0.739656in}}%
\pgfpathlineto{\pgfqpoint{3.086061in}{0.739656in}}%
\pgfpathlineto{\pgfqpoint{3.085765in}{0.739656in}}%
\pgfpathlineto{\pgfqpoint{3.085469in}{0.739656in}}%
\pgfpathlineto{\pgfqpoint{3.085173in}{0.739656in}}%
\pgfpathlineto{\pgfqpoint{3.084877in}{0.739656in}}%
\pgfpathlineto{\pgfqpoint{3.084581in}{0.739656in}}%
\pgfpathlineto{\pgfqpoint{3.084285in}{0.739656in}}%
\pgfpathlineto{\pgfqpoint{3.083989in}{0.739656in}}%
\pgfpathlineto{\pgfqpoint{3.083693in}{0.739656in}}%
\pgfpathlineto{\pgfqpoint{3.083397in}{0.739656in}}%
\pgfpathlineto{\pgfqpoint{3.083101in}{0.739656in}}%
\pgfpathlineto{\pgfqpoint{3.082805in}{0.739656in}}%
\pgfpathlineto{\pgfqpoint{3.082509in}{0.739656in}}%
\pgfpathlineto{\pgfqpoint{3.082213in}{0.739656in}}%
\pgfpathlineto{\pgfqpoint{3.081917in}{0.739656in}}%
\pgfpathlineto{\pgfqpoint{3.081621in}{0.739656in}}%
\pgfpathlineto{\pgfqpoint{3.081325in}{0.739656in}}%
\pgfpathlineto{\pgfqpoint{3.081029in}{0.739656in}}%
\pgfpathlineto{\pgfqpoint{3.080733in}{0.739656in}}%
\pgfpathlineto{\pgfqpoint{3.080437in}{0.739656in}}%
\pgfpathlineto{\pgfqpoint{3.080141in}{0.739656in}}%
\pgfpathlineto{\pgfqpoint{3.079845in}{0.739656in}}%
\pgfpathlineto{\pgfqpoint{3.079549in}{0.739656in}}%
\pgfpathlineto{\pgfqpoint{3.079253in}{0.739656in}}%
\pgfpathlineto{\pgfqpoint{3.078957in}{0.739656in}}%
\pgfpathlineto{\pgfqpoint{3.078661in}{0.739656in}}%
\pgfpathlineto{\pgfqpoint{3.078365in}{0.739656in}}%
\pgfpathlineto{\pgfqpoint{3.078069in}{0.739656in}}%
\pgfpathlineto{\pgfqpoint{3.077773in}{0.739656in}}%
\pgfpathlineto{\pgfqpoint{3.077477in}{0.739656in}}%
\pgfpathlineto{\pgfqpoint{3.077181in}{0.739656in}}%
\pgfpathlineto{\pgfqpoint{3.076885in}{0.739656in}}%
\pgfpathlineto{\pgfqpoint{3.076589in}{0.739656in}}%
\pgfpathlineto{\pgfqpoint{3.076293in}{0.739656in}}%
\pgfpathlineto{\pgfqpoint{3.075997in}{0.739656in}}%
\pgfpathlineto{\pgfqpoint{3.075701in}{0.739656in}}%
\pgfpathlineto{\pgfqpoint{3.075405in}{0.739656in}}%
\pgfpathlineto{\pgfqpoint{3.075109in}{0.739656in}}%
\pgfpathlineto{\pgfqpoint{3.074812in}{0.739656in}}%
\pgfpathlineto{\pgfqpoint{3.074516in}{0.739656in}}%
\pgfpathlineto{\pgfqpoint{3.074220in}{0.739656in}}%
\pgfpathlineto{\pgfqpoint{3.073924in}{0.739656in}}%
\pgfpathlineto{\pgfqpoint{3.073628in}{0.739656in}}%
\pgfpathlineto{\pgfqpoint{3.073332in}{0.739656in}}%
\pgfpathlineto{\pgfqpoint{3.073036in}{0.739656in}}%
\pgfpathlineto{\pgfqpoint{3.072740in}{0.739656in}}%
\pgfpathlineto{\pgfqpoint{3.072444in}{0.739656in}}%
\pgfpathlineto{\pgfqpoint{3.072148in}{0.739656in}}%
\pgfpathlineto{\pgfqpoint{3.071852in}{0.739656in}}%
\pgfpathlineto{\pgfqpoint{3.071556in}{0.739656in}}%
\pgfpathlineto{\pgfqpoint{3.071260in}{0.739656in}}%
\pgfpathlineto{\pgfqpoint{3.070964in}{0.739656in}}%
\pgfpathlineto{\pgfqpoint{3.070668in}{0.739656in}}%
\pgfpathlineto{\pgfqpoint{3.070372in}{0.739656in}}%
\pgfpathlineto{\pgfqpoint{3.070076in}{0.739656in}}%
\pgfpathlineto{\pgfqpoint{3.069780in}{0.739656in}}%
\pgfpathlineto{\pgfqpoint{3.069484in}{0.739656in}}%
\pgfpathlineto{\pgfqpoint{3.069188in}{0.739656in}}%
\pgfpathlineto{\pgfqpoint{3.068892in}{0.739656in}}%
\pgfpathlineto{\pgfqpoint{3.068596in}{0.739656in}}%
\pgfpathlineto{\pgfqpoint{3.068300in}{0.739656in}}%
\pgfpathlineto{\pgfqpoint{3.068004in}{0.739656in}}%
\pgfpathlineto{\pgfqpoint{3.067708in}{0.739656in}}%
\pgfpathlineto{\pgfqpoint{3.067412in}{0.739656in}}%
\pgfpathlineto{\pgfqpoint{3.067116in}{0.739656in}}%
\pgfpathlineto{\pgfqpoint{3.066820in}{0.739656in}}%
\pgfpathlineto{\pgfqpoint{3.066524in}{0.739656in}}%
\pgfpathlineto{\pgfqpoint{3.066228in}{0.739656in}}%
\pgfpathlineto{\pgfqpoint{3.065932in}{0.739656in}}%
\pgfpathlineto{\pgfqpoint{3.065636in}{0.739656in}}%
\pgfpathlineto{\pgfqpoint{3.065340in}{0.739656in}}%
\pgfpathlineto{\pgfqpoint{3.065044in}{0.739656in}}%
\pgfpathlineto{\pgfqpoint{3.064748in}{0.739656in}}%
\pgfpathlineto{\pgfqpoint{3.064452in}{0.739656in}}%
\pgfpathlineto{\pgfqpoint{3.064156in}{0.739656in}}%
\pgfpathlineto{\pgfqpoint{3.063860in}{0.739656in}}%
\pgfpathlineto{\pgfqpoint{3.063564in}{0.739656in}}%
\pgfpathlineto{\pgfqpoint{3.063268in}{0.739656in}}%
\pgfpathlineto{\pgfqpoint{3.062972in}{0.739656in}}%
\pgfpathlineto{\pgfqpoint{3.062676in}{0.739656in}}%
\pgfpathlineto{\pgfqpoint{3.062380in}{0.739656in}}%
\pgfpathlineto{\pgfqpoint{3.062084in}{0.739656in}}%
\pgfpathlineto{\pgfqpoint{3.061788in}{0.739656in}}%
\pgfpathlineto{\pgfqpoint{3.061492in}{0.739656in}}%
\pgfpathlineto{\pgfqpoint{3.061196in}{0.739656in}}%
\pgfpathlineto{\pgfqpoint{3.060900in}{0.739656in}}%
\pgfpathlineto{\pgfqpoint{3.060604in}{0.739656in}}%
\pgfpathlineto{\pgfqpoint{3.060308in}{0.739656in}}%
\pgfpathlineto{\pgfqpoint{3.060012in}{0.739656in}}%
\pgfpathlineto{\pgfqpoint{3.059716in}{0.739656in}}%
\pgfpathlineto{\pgfqpoint{3.059420in}{0.739656in}}%
\pgfpathlineto{\pgfqpoint{3.059124in}{0.739656in}}%
\pgfpathlineto{\pgfqpoint{3.058828in}{0.739656in}}%
\pgfpathlineto{\pgfqpoint{3.058532in}{0.739656in}}%
\pgfpathlineto{\pgfqpoint{3.058236in}{0.739656in}}%
\pgfpathlineto{\pgfqpoint{3.057940in}{0.739656in}}%
\pgfpathlineto{\pgfqpoint{3.057644in}{0.739656in}}%
\pgfpathlineto{\pgfqpoint{3.057348in}{0.739656in}}%
\pgfpathlineto{\pgfqpoint{3.057052in}{0.739656in}}%
\pgfpathlineto{\pgfqpoint{3.056756in}{0.739656in}}%
\pgfpathlineto{\pgfqpoint{3.056460in}{0.739656in}}%
\pgfpathlineto{\pgfqpoint{3.056164in}{0.739656in}}%
\pgfpathlineto{\pgfqpoint{3.055868in}{0.739656in}}%
\pgfpathlineto{\pgfqpoint{3.055572in}{0.739656in}}%
\pgfpathlineto{\pgfqpoint{3.055276in}{0.739656in}}%
\pgfpathlineto{\pgfqpoint{3.054980in}{0.739656in}}%
\pgfpathlineto{\pgfqpoint{3.054684in}{0.739656in}}%
\pgfpathlineto{\pgfqpoint{3.054388in}{0.739656in}}%
\pgfpathlineto{\pgfqpoint{3.054092in}{0.739656in}}%
\pgfpathlineto{\pgfqpoint{3.053796in}{0.739656in}}%
\pgfpathlineto{\pgfqpoint{3.053500in}{0.739656in}}%
\pgfpathlineto{\pgfqpoint{3.053204in}{0.739656in}}%
\pgfpathlineto{\pgfqpoint{3.052908in}{0.739656in}}%
\pgfpathlineto{\pgfqpoint{3.052612in}{0.739656in}}%
\pgfpathlineto{\pgfqpoint{3.052316in}{0.739656in}}%
\pgfpathlineto{\pgfqpoint{3.052020in}{0.739656in}}%
\pgfpathlineto{\pgfqpoint{3.051724in}{0.739656in}}%
\pgfpathlineto{\pgfqpoint{3.051428in}{0.739656in}}%
\pgfpathlineto{\pgfqpoint{3.051132in}{0.739656in}}%
\pgfpathlineto{\pgfqpoint{3.050836in}{0.739656in}}%
\pgfpathlineto{\pgfqpoint{3.050540in}{0.739656in}}%
\pgfpathlineto{\pgfqpoint{3.050244in}{0.739656in}}%
\pgfpathlineto{\pgfqpoint{3.049948in}{0.739656in}}%
\pgfpathlineto{\pgfqpoint{3.049652in}{0.739656in}}%
\pgfpathlineto{\pgfqpoint{3.049356in}{0.739656in}}%
\pgfpathlineto{\pgfqpoint{3.049060in}{0.739656in}}%
\pgfpathlineto{\pgfqpoint{3.048764in}{0.739656in}}%
\pgfpathlineto{\pgfqpoint{3.048468in}{0.739656in}}%
\pgfpathlineto{\pgfqpoint{3.048172in}{0.739656in}}%
\pgfpathlineto{\pgfqpoint{3.047876in}{0.739656in}}%
\pgfpathlineto{\pgfqpoint{3.047580in}{0.739656in}}%
\pgfpathlineto{\pgfqpoint{3.047284in}{0.739656in}}%
\pgfpathlineto{\pgfqpoint{3.046988in}{0.739656in}}%
\pgfpathlineto{\pgfqpoint{3.046692in}{0.739656in}}%
\pgfpathlineto{\pgfqpoint{3.046396in}{0.739656in}}%
\pgfpathlineto{\pgfqpoint{3.046100in}{0.739656in}}%
\pgfpathlineto{\pgfqpoint{3.045804in}{0.739656in}}%
\pgfpathlineto{\pgfqpoint{3.045508in}{0.739656in}}%
\pgfpathlineto{\pgfqpoint{3.045212in}{0.739656in}}%
\pgfpathlineto{\pgfqpoint{3.044916in}{0.739656in}}%
\pgfpathlineto{\pgfqpoint{3.044620in}{0.739656in}}%
\pgfpathlineto{\pgfqpoint{3.044324in}{0.739656in}}%
\pgfpathlineto{\pgfqpoint{3.044028in}{0.739656in}}%
\pgfpathlineto{\pgfqpoint{3.043732in}{0.739656in}}%
\pgfpathlineto{\pgfqpoint{3.043436in}{0.739656in}}%
\pgfpathlineto{\pgfqpoint{3.043140in}{0.739656in}}%
\pgfpathlineto{\pgfqpoint{3.042844in}{0.739656in}}%
\pgfpathlineto{\pgfqpoint{3.042548in}{0.739656in}}%
\pgfpathlineto{\pgfqpoint{3.042252in}{0.739656in}}%
\pgfpathlineto{\pgfqpoint{3.041956in}{0.739656in}}%
\pgfpathlineto{\pgfqpoint{3.041660in}{0.739656in}}%
\pgfpathlineto{\pgfqpoint{3.041364in}{0.739656in}}%
\pgfpathlineto{\pgfqpoint{3.041068in}{0.739656in}}%
\pgfpathlineto{\pgfqpoint{3.040772in}{0.739656in}}%
\pgfpathlineto{\pgfqpoint{3.040476in}{0.739656in}}%
\pgfpathlineto{\pgfqpoint{3.040180in}{0.739656in}}%
\pgfpathlineto{\pgfqpoint{3.039884in}{0.739656in}}%
\pgfpathlineto{\pgfqpoint{3.039588in}{0.739656in}}%
\pgfpathlineto{\pgfqpoint{3.039292in}{0.739656in}}%
\pgfpathlineto{\pgfqpoint{3.038996in}{0.739656in}}%
\pgfpathlineto{\pgfqpoint{3.038700in}{0.739656in}}%
\pgfpathlineto{\pgfqpoint{3.038404in}{0.739656in}}%
\pgfpathlineto{\pgfqpoint{3.038108in}{0.739656in}}%
\pgfpathlineto{\pgfqpoint{3.037812in}{0.739656in}}%
\pgfpathlineto{\pgfqpoint{3.037516in}{0.739656in}}%
\pgfpathlineto{\pgfqpoint{3.037220in}{0.739656in}}%
\pgfpathlineto{\pgfqpoint{3.036924in}{0.739656in}}%
\pgfpathlineto{\pgfqpoint{3.036628in}{0.739656in}}%
\pgfpathlineto{\pgfqpoint{3.036332in}{0.739656in}}%
\pgfpathlineto{\pgfqpoint{3.036036in}{0.739656in}}%
\pgfpathlineto{\pgfqpoint{3.035740in}{0.739656in}}%
\pgfpathlineto{\pgfqpoint{3.035444in}{0.739656in}}%
\pgfpathlineto{\pgfqpoint{3.035148in}{0.739656in}}%
\pgfpathlineto{\pgfqpoint{3.034852in}{0.739656in}}%
\pgfpathlineto{\pgfqpoint{3.034556in}{0.739656in}}%
\pgfpathlineto{\pgfqpoint{3.034260in}{0.739656in}}%
\pgfpathlineto{\pgfqpoint{3.033964in}{0.739656in}}%
\pgfpathlineto{\pgfqpoint{3.033668in}{0.739656in}}%
\pgfpathlineto{\pgfqpoint{3.033372in}{0.739656in}}%
\pgfpathlineto{\pgfqpoint{3.033076in}{0.739656in}}%
\pgfpathlineto{\pgfqpoint{3.032780in}{0.739656in}}%
\pgfpathlineto{\pgfqpoint{3.032484in}{0.739656in}}%
\pgfpathlineto{\pgfqpoint{3.032188in}{0.739656in}}%
\pgfpathlineto{\pgfqpoint{3.031892in}{0.739656in}}%
\pgfpathlineto{\pgfqpoint{3.031596in}{0.739656in}}%
\pgfpathlineto{\pgfqpoint{3.031300in}{0.739656in}}%
\pgfpathlineto{\pgfqpoint{3.031004in}{0.739656in}}%
\pgfpathlineto{\pgfqpoint{3.030708in}{0.739656in}}%
\pgfpathlineto{\pgfqpoint{3.030412in}{0.739656in}}%
\pgfpathlineto{\pgfqpoint{3.030116in}{0.739656in}}%
\pgfpathlineto{\pgfqpoint{3.029820in}{0.739656in}}%
\pgfpathlineto{\pgfqpoint{3.029524in}{0.739656in}}%
\pgfpathlineto{\pgfqpoint{3.029228in}{0.739656in}}%
\pgfpathlineto{\pgfqpoint{3.028932in}{0.739656in}}%
\pgfpathlineto{\pgfqpoint{3.028636in}{0.739656in}}%
\pgfpathlineto{\pgfqpoint{3.028340in}{0.739656in}}%
\pgfpathlineto{\pgfqpoint{3.028044in}{0.739656in}}%
\pgfpathlineto{\pgfqpoint{3.027748in}{0.739656in}}%
\pgfpathlineto{\pgfqpoint{3.027452in}{0.739656in}}%
\pgfpathlineto{\pgfqpoint{3.027156in}{0.739656in}}%
\pgfpathlineto{\pgfqpoint{3.026860in}{0.739656in}}%
\pgfpathlineto{\pgfqpoint{3.026564in}{0.739656in}}%
\pgfpathlineto{\pgfqpoint{3.026268in}{0.739656in}}%
\pgfpathlineto{\pgfqpoint{3.025972in}{0.739656in}}%
\pgfpathlineto{\pgfqpoint{3.025676in}{0.739656in}}%
\pgfpathlineto{\pgfqpoint{3.025380in}{0.739656in}}%
\pgfpathlineto{\pgfqpoint{3.025084in}{0.739656in}}%
\pgfpathlineto{\pgfqpoint{3.024788in}{0.739656in}}%
\pgfpathlineto{\pgfqpoint{3.024492in}{0.739656in}}%
\pgfpathlineto{\pgfqpoint{3.024196in}{0.739656in}}%
\pgfpathlineto{\pgfqpoint{3.023900in}{0.739656in}}%
\pgfpathlineto{\pgfqpoint{3.023604in}{0.739656in}}%
\pgfpathlineto{\pgfqpoint{3.023308in}{0.739656in}}%
\pgfpathlineto{\pgfqpoint{3.023012in}{0.739656in}}%
\pgfpathlineto{\pgfqpoint{3.022716in}{0.739656in}}%
\pgfpathlineto{\pgfqpoint{3.022420in}{0.739656in}}%
\pgfpathlineto{\pgfqpoint{3.022124in}{0.739656in}}%
\pgfpathlineto{\pgfqpoint{3.021828in}{0.739656in}}%
\pgfpathlineto{\pgfqpoint{3.021532in}{0.739656in}}%
\pgfpathlineto{\pgfqpoint{3.021236in}{0.739656in}}%
\pgfpathlineto{\pgfqpoint{3.020940in}{0.739656in}}%
\pgfpathlineto{\pgfqpoint{3.020644in}{0.739656in}}%
\pgfpathlineto{\pgfqpoint{3.020348in}{0.739656in}}%
\pgfpathlineto{\pgfqpoint{3.020052in}{0.739656in}}%
\pgfpathlineto{\pgfqpoint{3.019756in}{0.739656in}}%
\pgfpathlineto{\pgfqpoint{3.019460in}{0.739656in}}%
\pgfpathlineto{\pgfqpoint{3.019164in}{0.739656in}}%
\pgfpathlineto{\pgfqpoint{3.018868in}{0.739656in}}%
\pgfpathlineto{\pgfqpoint{3.018572in}{0.739656in}}%
\pgfpathlineto{\pgfqpoint{3.018276in}{0.739656in}}%
\pgfpathlineto{\pgfqpoint{3.017980in}{0.739656in}}%
\pgfpathlineto{\pgfqpoint{3.017684in}{0.739656in}}%
\pgfpathlineto{\pgfqpoint{3.017388in}{0.739656in}}%
\pgfpathlineto{\pgfqpoint{3.017092in}{0.739656in}}%
\pgfpathlineto{\pgfqpoint{3.016796in}{0.739656in}}%
\pgfpathlineto{\pgfqpoint{3.016500in}{0.739656in}}%
\pgfpathlineto{\pgfqpoint{3.016204in}{0.739656in}}%
\pgfpathlineto{\pgfqpoint{3.015908in}{0.739656in}}%
\pgfpathlineto{\pgfqpoint{3.015612in}{0.739656in}}%
\pgfpathlineto{\pgfqpoint{3.015316in}{0.739656in}}%
\pgfpathlineto{\pgfqpoint{3.015020in}{0.739656in}}%
\pgfpathlineto{\pgfqpoint{3.014724in}{0.739656in}}%
\pgfpathlineto{\pgfqpoint{3.014428in}{0.739656in}}%
\pgfpathlineto{\pgfqpoint{3.014132in}{0.739656in}}%
\pgfpathlineto{\pgfqpoint{3.013836in}{0.739656in}}%
\pgfpathlineto{\pgfqpoint{3.013540in}{0.739656in}}%
\pgfpathlineto{\pgfqpoint{3.013244in}{0.739656in}}%
\pgfpathlineto{\pgfqpoint{3.012948in}{0.739656in}}%
\pgfpathlineto{\pgfqpoint{3.012652in}{0.739656in}}%
\pgfpathlineto{\pgfqpoint{3.012356in}{0.739656in}}%
\pgfpathlineto{\pgfqpoint{3.012060in}{0.739656in}}%
\pgfpathlineto{\pgfqpoint{3.011764in}{0.739656in}}%
\pgfpathlineto{\pgfqpoint{3.011468in}{0.739656in}}%
\pgfpathlineto{\pgfqpoint{3.011172in}{0.739656in}}%
\pgfpathlineto{\pgfqpoint{3.010876in}{0.739656in}}%
\pgfpathlineto{\pgfqpoint{3.010580in}{0.739656in}}%
\pgfpathlineto{\pgfqpoint{3.010284in}{0.739656in}}%
\pgfpathlineto{\pgfqpoint{3.009988in}{0.739656in}}%
\pgfpathlineto{\pgfqpoint{3.009692in}{0.739656in}}%
\pgfpathlineto{\pgfqpoint{3.009396in}{0.739656in}}%
\pgfpathlineto{\pgfqpoint{3.009100in}{0.739656in}}%
\pgfpathlineto{\pgfqpoint{3.008804in}{0.739656in}}%
\pgfpathlineto{\pgfqpoint{3.008508in}{0.739656in}}%
\pgfpathlineto{\pgfqpoint{3.008212in}{0.739656in}}%
\pgfpathlineto{\pgfqpoint{3.007916in}{0.739656in}}%
\pgfpathlineto{\pgfqpoint{3.007620in}{0.739656in}}%
\pgfpathlineto{\pgfqpoint{3.007323in}{0.739656in}}%
\pgfpathlineto{\pgfqpoint{3.007027in}{0.739656in}}%
\pgfpathlineto{\pgfqpoint{3.006731in}{0.739656in}}%
\pgfpathlineto{\pgfqpoint{3.006435in}{0.739656in}}%
\pgfpathlineto{\pgfqpoint{3.006139in}{0.739656in}}%
\pgfpathlineto{\pgfqpoint{3.005843in}{0.739656in}}%
\pgfpathlineto{\pgfqpoint{3.005547in}{0.739656in}}%
\pgfpathlineto{\pgfqpoint{3.005251in}{0.739656in}}%
\pgfpathlineto{\pgfqpoint{3.004955in}{0.739656in}}%
\pgfpathlineto{\pgfqpoint{3.004659in}{0.739656in}}%
\pgfpathlineto{\pgfqpoint{3.004363in}{0.739656in}}%
\pgfpathlineto{\pgfqpoint{3.004067in}{0.739656in}}%
\pgfpathlineto{\pgfqpoint{3.003771in}{0.739656in}}%
\pgfpathlineto{\pgfqpoint{3.003475in}{0.739656in}}%
\pgfpathlineto{\pgfqpoint{3.003179in}{0.739656in}}%
\pgfpathlineto{\pgfqpoint{3.002883in}{0.739656in}}%
\pgfpathlineto{\pgfqpoint{3.002587in}{0.739656in}}%
\pgfpathlineto{\pgfqpoint{3.002291in}{0.739656in}}%
\pgfpathlineto{\pgfqpoint{3.001995in}{0.739656in}}%
\pgfpathlineto{\pgfqpoint{3.001699in}{0.739656in}}%
\pgfpathlineto{\pgfqpoint{3.001403in}{0.739656in}}%
\pgfpathlineto{\pgfqpoint{3.001107in}{0.739656in}}%
\pgfpathlineto{\pgfqpoint{3.000811in}{0.739656in}}%
\pgfpathlineto{\pgfqpoint{3.000515in}{0.739656in}}%
\pgfpathlineto{\pgfqpoint{3.000219in}{0.739656in}}%
\pgfpathlineto{\pgfqpoint{2.999923in}{0.739656in}}%
\pgfpathlineto{\pgfqpoint{2.999627in}{0.739656in}}%
\pgfpathlineto{\pgfqpoint{2.999331in}{0.739656in}}%
\pgfpathlineto{\pgfqpoint{2.999035in}{0.739656in}}%
\pgfpathlineto{\pgfqpoint{2.998739in}{0.739656in}}%
\pgfpathlineto{\pgfqpoint{2.998443in}{0.739656in}}%
\pgfpathlineto{\pgfqpoint{2.998147in}{0.739656in}}%
\pgfpathlineto{\pgfqpoint{2.997851in}{0.739656in}}%
\pgfpathlineto{\pgfqpoint{2.997555in}{0.739656in}}%
\pgfpathlineto{\pgfqpoint{2.997259in}{0.739656in}}%
\pgfpathlineto{\pgfqpoint{2.996963in}{0.739656in}}%
\pgfpathlineto{\pgfqpoint{2.996667in}{0.739656in}}%
\pgfpathlineto{\pgfqpoint{2.996371in}{0.739656in}}%
\pgfpathlineto{\pgfqpoint{2.996075in}{0.739656in}}%
\pgfpathlineto{\pgfqpoint{2.995779in}{0.739656in}}%
\pgfpathlineto{\pgfqpoint{2.995483in}{0.739656in}}%
\pgfpathlineto{\pgfqpoint{2.995187in}{0.739656in}}%
\pgfpathlineto{\pgfqpoint{2.994891in}{0.739656in}}%
\pgfpathlineto{\pgfqpoint{2.994595in}{0.739656in}}%
\pgfpathlineto{\pgfqpoint{2.994299in}{0.739656in}}%
\pgfpathlineto{\pgfqpoint{2.994003in}{0.739656in}}%
\pgfpathlineto{\pgfqpoint{2.993707in}{0.739656in}}%
\pgfpathlineto{\pgfqpoint{2.993411in}{0.739656in}}%
\pgfpathlineto{\pgfqpoint{2.993115in}{0.739656in}}%
\pgfpathlineto{\pgfqpoint{2.992819in}{0.739656in}}%
\pgfpathlineto{\pgfqpoint{2.992523in}{0.739656in}}%
\pgfpathlineto{\pgfqpoint{2.992227in}{0.739656in}}%
\pgfpathlineto{\pgfqpoint{2.991931in}{0.739656in}}%
\pgfpathlineto{\pgfqpoint{2.991635in}{0.739656in}}%
\pgfpathlineto{\pgfqpoint{2.991339in}{0.739656in}}%
\pgfpathlineto{\pgfqpoint{2.991043in}{0.739656in}}%
\pgfpathlineto{\pgfqpoint{2.990747in}{0.739656in}}%
\pgfpathlineto{\pgfqpoint{2.990451in}{0.739656in}}%
\pgfpathlineto{\pgfqpoint{2.990155in}{0.739656in}}%
\pgfpathlineto{\pgfqpoint{2.989859in}{0.739656in}}%
\pgfpathlineto{\pgfqpoint{2.989563in}{0.739656in}}%
\pgfpathlineto{\pgfqpoint{2.989267in}{0.739656in}}%
\pgfpathlineto{\pgfqpoint{2.988971in}{0.739656in}}%
\pgfpathlineto{\pgfqpoint{2.988675in}{0.739656in}}%
\pgfpathlineto{\pgfqpoint{2.988379in}{0.739656in}}%
\pgfpathlineto{\pgfqpoint{2.988083in}{0.739656in}}%
\pgfpathlineto{\pgfqpoint{2.987787in}{0.739656in}}%
\pgfpathlineto{\pgfqpoint{2.987491in}{0.739656in}}%
\pgfpathlineto{\pgfqpoint{2.987195in}{0.739656in}}%
\pgfpathlineto{\pgfqpoint{2.986899in}{0.739656in}}%
\pgfpathlineto{\pgfqpoint{2.986603in}{0.739656in}}%
\pgfpathlineto{\pgfqpoint{2.986307in}{0.739656in}}%
\pgfpathlineto{\pgfqpoint{2.986011in}{0.739656in}}%
\pgfpathlineto{\pgfqpoint{2.985715in}{0.739656in}}%
\pgfpathlineto{\pgfqpoint{2.985419in}{0.739656in}}%
\pgfpathlineto{\pgfqpoint{2.985123in}{0.739656in}}%
\pgfpathlineto{\pgfqpoint{2.984827in}{0.739656in}}%
\pgfpathlineto{\pgfqpoint{2.984531in}{0.739656in}}%
\pgfpathlineto{\pgfqpoint{2.984235in}{0.739656in}}%
\pgfpathlineto{\pgfqpoint{2.983939in}{0.739656in}}%
\pgfpathlineto{\pgfqpoint{2.983643in}{0.739656in}}%
\pgfpathlineto{\pgfqpoint{2.983347in}{0.739656in}}%
\pgfpathlineto{\pgfqpoint{2.983051in}{0.739656in}}%
\pgfpathlineto{\pgfqpoint{2.982755in}{0.739656in}}%
\pgfpathlineto{\pgfqpoint{2.982459in}{0.739656in}}%
\pgfpathlineto{\pgfqpoint{2.982163in}{0.739656in}}%
\pgfpathlineto{\pgfqpoint{2.981867in}{0.739656in}}%
\pgfpathlineto{\pgfqpoint{2.981571in}{0.739656in}}%
\pgfpathlineto{\pgfqpoint{2.981275in}{0.739656in}}%
\pgfpathlineto{\pgfqpoint{2.980979in}{0.739656in}}%
\pgfpathlineto{\pgfqpoint{2.980683in}{0.739656in}}%
\pgfpathlineto{\pgfqpoint{2.980387in}{0.739656in}}%
\pgfpathlineto{\pgfqpoint{2.980091in}{0.739656in}}%
\pgfpathlineto{\pgfqpoint{2.979795in}{0.739656in}}%
\pgfpathlineto{\pgfqpoint{2.979499in}{0.739656in}}%
\pgfpathlineto{\pgfqpoint{2.979203in}{0.739656in}}%
\pgfpathlineto{\pgfqpoint{2.978907in}{0.739656in}}%
\pgfpathlineto{\pgfqpoint{2.978611in}{0.739656in}}%
\pgfpathlineto{\pgfqpoint{2.978315in}{0.739656in}}%
\pgfpathlineto{\pgfqpoint{2.978019in}{0.739656in}}%
\pgfpathlineto{\pgfqpoint{2.977723in}{0.739656in}}%
\pgfpathlineto{\pgfqpoint{2.977427in}{0.739656in}}%
\pgfpathlineto{\pgfqpoint{2.977131in}{0.739656in}}%
\pgfpathlineto{\pgfqpoint{2.976835in}{0.739656in}}%
\pgfpathlineto{\pgfqpoint{2.976539in}{0.739656in}}%
\pgfpathlineto{\pgfqpoint{2.976243in}{0.739656in}}%
\pgfpathlineto{\pgfqpoint{2.975947in}{0.739656in}}%
\pgfpathlineto{\pgfqpoint{2.975651in}{0.739656in}}%
\pgfpathlineto{\pgfqpoint{2.975355in}{0.739656in}}%
\pgfpathlineto{\pgfqpoint{2.975059in}{0.739656in}}%
\pgfpathlineto{\pgfqpoint{2.974763in}{0.739656in}}%
\pgfpathlineto{\pgfqpoint{2.974467in}{0.739656in}}%
\pgfpathlineto{\pgfqpoint{2.974171in}{0.739656in}}%
\pgfpathlineto{\pgfqpoint{2.973875in}{0.739656in}}%
\pgfpathlineto{\pgfqpoint{2.973579in}{0.739656in}}%
\pgfpathlineto{\pgfqpoint{2.973283in}{0.739656in}}%
\pgfpathlineto{\pgfqpoint{2.972987in}{0.739656in}}%
\pgfpathlineto{\pgfqpoint{2.972691in}{0.739656in}}%
\pgfpathlineto{\pgfqpoint{2.972395in}{0.739656in}}%
\pgfpathlineto{\pgfqpoint{2.972099in}{0.739656in}}%
\pgfpathlineto{\pgfqpoint{2.971803in}{0.739656in}}%
\pgfpathlineto{\pgfqpoint{2.971507in}{0.739656in}}%
\pgfpathlineto{\pgfqpoint{2.971211in}{0.739656in}}%
\pgfpathlineto{\pgfqpoint{2.970915in}{0.739656in}}%
\pgfpathlineto{\pgfqpoint{2.970619in}{0.739656in}}%
\pgfpathlineto{\pgfqpoint{2.970323in}{0.739656in}}%
\pgfpathlineto{\pgfqpoint{2.970027in}{0.739656in}}%
\pgfpathlineto{\pgfqpoint{2.969731in}{0.739656in}}%
\pgfpathlineto{\pgfqpoint{2.969435in}{0.739656in}}%
\pgfpathlineto{\pgfqpoint{2.969139in}{0.739656in}}%
\pgfpathlineto{\pgfqpoint{2.968843in}{0.739656in}}%
\pgfpathlineto{\pgfqpoint{2.968547in}{0.739656in}}%
\pgfpathlineto{\pgfqpoint{2.968251in}{0.739656in}}%
\pgfpathlineto{\pgfqpoint{2.967955in}{0.739656in}}%
\pgfpathlineto{\pgfqpoint{2.967659in}{0.739656in}}%
\pgfpathlineto{\pgfqpoint{2.967363in}{0.739656in}}%
\pgfpathlineto{\pgfqpoint{2.967067in}{0.739656in}}%
\pgfpathlineto{\pgfqpoint{2.966771in}{0.739656in}}%
\pgfpathlineto{\pgfqpoint{2.966475in}{0.739656in}}%
\pgfpathlineto{\pgfqpoint{2.966179in}{0.739656in}}%
\pgfpathlineto{\pgfqpoint{2.965883in}{0.739656in}}%
\pgfpathlineto{\pgfqpoint{2.965587in}{0.739656in}}%
\pgfpathlineto{\pgfqpoint{2.965291in}{0.739656in}}%
\pgfpathlineto{\pgfqpoint{2.964995in}{0.739656in}}%
\pgfpathlineto{\pgfqpoint{2.964699in}{0.739656in}}%
\pgfpathlineto{\pgfqpoint{2.964403in}{0.739656in}}%
\pgfpathlineto{\pgfqpoint{2.964107in}{0.739656in}}%
\pgfpathlineto{\pgfqpoint{2.963811in}{0.739656in}}%
\pgfpathlineto{\pgfqpoint{2.963515in}{0.739656in}}%
\pgfpathlineto{\pgfqpoint{2.963219in}{0.739656in}}%
\pgfpathlineto{\pgfqpoint{2.962923in}{0.739656in}}%
\pgfpathlineto{\pgfqpoint{2.962627in}{0.739656in}}%
\pgfpathlineto{\pgfqpoint{2.962331in}{0.739656in}}%
\pgfpathlineto{\pgfqpoint{2.962035in}{0.739656in}}%
\pgfpathlineto{\pgfqpoint{2.961739in}{0.739656in}}%
\pgfpathlineto{\pgfqpoint{2.961443in}{0.739656in}}%
\pgfpathlineto{\pgfqpoint{2.961147in}{0.739656in}}%
\pgfpathlineto{\pgfqpoint{2.960851in}{0.739656in}}%
\pgfpathlineto{\pgfqpoint{2.960555in}{0.739656in}}%
\pgfpathlineto{\pgfqpoint{2.960259in}{0.739656in}}%
\pgfpathlineto{\pgfqpoint{2.959963in}{0.739656in}}%
\pgfpathlineto{\pgfqpoint{2.959667in}{0.739656in}}%
\pgfpathlineto{\pgfqpoint{2.959371in}{0.739656in}}%
\pgfpathlineto{\pgfqpoint{2.959075in}{0.739656in}}%
\pgfpathlineto{\pgfqpoint{2.958779in}{0.739656in}}%
\pgfpathlineto{\pgfqpoint{2.958483in}{0.739656in}}%
\pgfpathlineto{\pgfqpoint{2.958187in}{0.739656in}}%
\pgfpathlineto{\pgfqpoint{2.957891in}{0.739656in}}%
\pgfpathlineto{\pgfqpoint{2.957595in}{0.739656in}}%
\pgfpathlineto{\pgfqpoint{2.957299in}{0.739656in}}%
\pgfpathlineto{\pgfqpoint{2.957003in}{0.739656in}}%
\pgfpathlineto{\pgfqpoint{2.956707in}{0.739656in}}%
\pgfpathlineto{\pgfqpoint{2.956411in}{0.739656in}}%
\pgfpathlineto{\pgfqpoint{2.956115in}{0.739656in}}%
\pgfpathlineto{\pgfqpoint{2.955819in}{0.739656in}}%
\pgfpathlineto{\pgfqpoint{2.955523in}{0.739656in}}%
\pgfpathlineto{\pgfqpoint{2.955227in}{0.739656in}}%
\pgfpathlineto{\pgfqpoint{2.954931in}{0.739656in}}%
\pgfpathlineto{\pgfqpoint{2.954635in}{0.739656in}}%
\pgfpathlineto{\pgfqpoint{2.954339in}{0.739656in}}%
\pgfpathlineto{\pgfqpoint{2.954043in}{0.739656in}}%
\pgfpathlineto{\pgfqpoint{2.953747in}{0.739656in}}%
\pgfpathlineto{\pgfqpoint{2.953451in}{0.739656in}}%
\pgfpathlineto{\pgfqpoint{2.953155in}{0.739656in}}%
\pgfpathlineto{\pgfqpoint{2.952859in}{0.739656in}}%
\pgfpathlineto{\pgfqpoint{2.952563in}{0.739656in}}%
\pgfpathlineto{\pgfqpoint{2.952267in}{0.739656in}}%
\pgfpathlineto{\pgfqpoint{2.951971in}{0.739656in}}%
\pgfpathlineto{\pgfqpoint{2.951675in}{0.739656in}}%
\pgfpathlineto{\pgfqpoint{2.951379in}{0.739656in}}%
\pgfpathlineto{\pgfqpoint{2.951083in}{0.739656in}}%
\pgfpathlineto{\pgfqpoint{2.950787in}{0.739656in}}%
\pgfpathlineto{\pgfqpoint{2.950491in}{0.739656in}}%
\pgfpathlineto{\pgfqpoint{2.950195in}{0.739656in}}%
\pgfpathlineto{\pgfqpoint{2.949899in}{0.739656in}}%
\pgfpathlineto{\pgfqpoint{2.949603in}{0.739656in}}%
\pgfpathlineto{\pgfqpoint{2.949307in}{0.739656in}}%
\pgfpathlineto{\pgfqpoint{2.949011in}{0.739656in}}%
\pgfpathlineto{\pgfqpoint{2.948715in}{0.739656in}}%
\pgfpathlineto{\pgfqpoint{2.948419in}{0.739656in}}%
\pgfpathlineto{\pgfqpoint{2.948123in}{0.739656in}}%
\pgfpathlineto{\pgfqpoint{2.947827in}{0.739656in}}%
\pgfpathlineto{\pgfqpoint{2.947531in}{0.739656in}}%
\pgfpathlineto{\pgfqpoint{2.947235in}{0.739656in}}%
\pgfpathlineto{\pgfqpoint{2.946939in}{0.739656in}}%
\pgfpathlineto{\pgfqpoint{2.946643in}{0.739656in}}%
\pgfpathlineto{\pgfqpoint{2.946347in}{0.739656in}}%
\pgfpathlineto{\pgfqpoint{2.946051in}{0.739656in}}%
\pgfpathlineto{\pgfqpoint{2.945755in}{0.739656in}}%
\pgfpathlineto{\pgfqpoint{2.945459in}{0.739656in}}%
\pgfpathlineto{\pgfqpoint{2.945163in}{0.739656in}}%
\pgfpathlineto{\pgfqpoint{2.944867in}{0.739656in}}%
\pgfpathlineto{\pgfqpoint{2.944571in}{0.739656in}}%
\pgfpathlineto{\pgfqpoint{2.944275in}{0.739656in}}%
\pgfpathlineto{\pgfqpoint{2.943979in}{0.739656in}}%
\pgfpathlineto{\pgfqpoint{2.943683in}{0.739656in}}%
\pgfpathlineto{\pgfqpoint{2.943387in}{0.739656in}}%
\pgfpathlineto{\pgfqpoint{2.943091in}{0.739656in}}%
\pgfpathlineto{\pgfqpoint{2.942795in}{0.739656in}}%
\pgfpathlineto{\pgfqpoint{2.942499in}{0.739656in}}%
\pgfpathlineto{\pgfqpoint{2.942203in}{0.739656in}}%
\pgfpathlineto{\pgfqpoint{2.941907in}{0.739656in}}%
\pgfpathlineto{\pgfqpoint{2.941611in}{0.739656in}}%
\pgfpathlineto{\pgfqpoint{2.941315in}{0.739656in}}%
\pgfpathlineto{\pgfqpoint{2.941019in}{0.739656in}}%
\pgfpathlineto{\pgfqpoint{2.940723in}{0.739656in}}%
\pgfpathlineto{\pgfqpoint{2.940427in}{0.739656in}}%
\pgfpathlineto{\pgfqpoint{2.940131in}{0.739656in}}%
\pgfpathlineto{\pgfqpoint{2.939834in}{0.739656in}}%
\pgfpathlineto{\pgfqpoint{2.939538in}{0.739656in}}%
\pgfpathlineto{\pgfqpoint{2.939242in}{0.739656in}}%
\pgfpathlineto{\pgfqpoint{2.938946in}{0.739656in}}%
\pgfpathlineto{\pgfqpoint{2.938650in}{0.739656in}}%
\pgfpathlineto{\pgfqpoint{2.938354in}{0.739656in}}%
\pgfpathlineto{\pgfqpoint{2.938058in}{0.739656in}}%
\pgfpathlineto{\pgfqpoint{2.937762in}{0.739656in}}%
\pgfpathlineto{\pgfqpoint{2.937466in}{0.739656in}}%
\pgfpathlineto{\pgfqpoint{2.937170in}{0.739656in}}%
\pgfpathlineto{\pgfqpoint{2.936874in}{0.739656in}}%
\pgfpathlineto{\pgfqpoint{2.936578in}{0.739656in}}%
\pgfpathlineto{\pgfqpoint{2.936282in}{0.739656in}}%
\pgfpathlineto{\pgfqpoint{2.935986in}{0.739656in}}%
\pgfpathlineto{\pgfqpoint{2.935690in}{0.739656in}}%
\pgfpathlineto{\pgfqpoint{2.935394in}{0.739656in}}%
\pgfpathlineto{\pgfqpoint{2.935098in}{0.739656in}}%
\pgfpathlineto{\pgfqpoint{2.934802in}{0.739656in}}%
\pgfpathlineto{\pgfqpoint{2.934506in}{0.739656in}}%
\pgfpathlineto{\pgfqpoint{2.934210in}{0.739656in}}%
\pgfpathlineto{\pgfqpoint{2.933914in}{0.739656in}}%
\pgfpathlineto{\pgfqpoint{2.933618in}{0.739656in}}%
\pgfpathlineto{\pgfqpoint{2.933322in}{0.739656in}}%
\pgfpathlineto{\pgfqpoint{2.933026in}{0.739656in}}%
\pgfpathlineto{\pgfqpoint{2.932730in}{0.739656in}}%
\pgfpathlineto{\pgfqpoint{2.932434in}{0.739656in}}%
\pgfpathlineto{\pgfqpoint{2.932138in}{0.739656in}}%
\pgfpathlineto{\pgfqpoint{2.931842in}{0.739656in}}%
\pgfpathlineto{\pgfqpoint{2.931546in}{0.739656in}}%
\pgfpathlineto{\pgfqpoint{2.931250in}{0.739656in}}%
\pgfpathlineto{\pgfqpoint{2.930954in}{0.739656in}}%
\pgfpathlineto{\pgfqpoint{2.930658in}{0.739656in}}%
\pgfpathlineto{\pgfqpoint{2.930362in}{0.739656in}}%
\pgfpathlineto{\pgfqpoint{2.930066in}{0.739656in}}%
\pgfpathlineto{\pgfqpoint{2.929770in}{0.739656in}}%
\pgfpathlineto{\pgfqpoint{2.929474in}{0.739656in}}%
\pgfpathlineto{\pgfqpoint{2.929178in}{0.739656in}}%
\pgfpathlineto{\pgfqpoint{2.928882in}{0.739656in}}%
\pgfpathlineto{\pgfqpoint{2.928586in}{0.739656in}}%
\pgfpathlineto{\pgfqpoint{2.928290in}{0.739656in}}%
\pgfpathlineto{\pgfqpoint{2.927994in}{0.739656in}}%
\pgfpathlineto{\pgfqpoint{2.927698in}{0.739656in}}%
\pgfpathlineto{\pgfqpoint{2.927402in}{0.739656in}}%
\pgfpathlineto{\pgfqpoint{2.927106in}{0.739656in}}%
\pgfpathlineto{\pgfqpoint{2.926810in}{0.739656in}}%
\pgfpathlineto{\pgfqpoint{2.926514in}{0.739656in}}%
\pgfpathlineto{\pgfqpoint{2.926218in}{0.739656in}}%
\pgfpathlineto{\pgfqpoint{2.925922in}{0.739656in}}%
\pgfpathlineto{\pgfqpoint{2.925626in}{0.739656in}}%
\pgfpathlineto{\pgfqpoint{2.925330in}{0.739656in}}%
\pgfpathlineto{\pgfqpoint{2.925034in}{0.739656in}}%
\pgfpathlineto{\pgfqpoint{2.924738in}{0.739656in}}%
\pgfpathlineto{\pgfqpoint{2.924442in}{0.739656in}}%
\pgfpathlineto{\pgfqpoint{2.924146in}{0.739656in}}%
\pgfpathlineto{\pgfqpoint{2.923850in}{0.739656in}}%
\pgfpathlineto{\pgfqpoint{2.923554in}{0.739656in}}%
\pgfpathlineto{\pgfqpoint{2.923258in}{0.739656in}}%
\pgfpathlineto{\pgfqpoint{2.922962in}{0.739656in}}%
\pgfpathlineto{\pgfqpoint{2.922666in}{0.739656in}}%
\pgfpathlineto{\pgfqpoint{2.922370in}{0.739656in}}%
\pgfpathlineto{\pgfqpoint{2.922074in}{0.739656in}}%
\pgfpathlineto{\pgfqpoint{2.921778in}{0.739656in}}%
\pgfpathlineto{\pgfqpoint{2.921482in}{0.739656in}}%
\pgfpathlineto{\pgfqpoint{2.921186in}{0.739656in}}%
\pgfpathlineto{\pgfqpoint{2.920890in}{0.739656in}}%
\pgfpathlineto{\pgfqpoint{2.920594in}{0.739656in}}%
\pgfpathlineto{\pgfqpoint{2.920298in}{0.739656in}}%
\pgfpathlineto{\pgfqpoint{2.920002in}{0.739656in}}%
\pgfpathlineto{\pgfqpoint{2.919706in}{0.739656in}}%
\pgfpathlineto{\pgfqpoint{2.919410in}{0.739656in}}%
\pgfpathlineto{\pgfqpoint{2.919114in}{0.739656in}}%
\pgfpathlineto{\pgfqpoint{2.918818in}{0.739656in}}%
\pgfpathlineto{\pgfqpoint{2.918522in}{0.739656in}}%
\pgfpathlineto{\pgfqpoint{2.918226in}{0.739656in}}%
\pgfpathlineto{\pgfqpoint{2.917930in}{0.739656in}}%
\pgfpathlineto{\pgfqpoint{2.917634in}{0.739656in}}%
\pgfpathlineto{\pgfqpoint{2.917338in}{0.739656in}}%
\pgfpathlineto{\pgfqpoint{2.917042in}{0.739656in}}%
\pgfpathlineto{\pgfqpoint{2.916746in}{0.739656in}}%
\pgfpathlineto{\pgfqpoint{2.916450in}{0.739656in}}%
\pgfpathlineto{\pgfqpoint{2.916154in}{0.739656in}}%
\pgfpathlineto{\pgfqpoint{2.915858in}{0.739656in}}%
\pgfpathlineto{\pgfqpoint{2.915562in}{0.739656in}}%
\pgfpathlineto{\pgfqpoint{2.915266in}{0.739656in}}%
\pgfpathlineto{\pgfqpoint{2.914970in}{0.739656in}}%
\pgfpathlineto{\pgfqpoint{2.914674in}{0.739656in}}%
\pgfpathlineto{\pgfqpoint{2.914378in}{0.739656in}}%
\pgfpathlineto{\pgfqpoint{2.914082in}{0.739656in}}%
\pgfpathlineto{\pgfqpoint{2.913786in}{0.739656in}}%
\pgfpathlineto{\pgfqpoint{2.913490in}{0.739656in}}%
\pgfpathlineto{\pgfqpoint{2.913194in}{0.739656in}}%
\pgfpathlineto{\pgfqpoint{2.912898in}{0.739656in}}%
\pgfpathlineto{\pgfqpoint{2.912602in}{0.739656in}}%
\pgfpathlineto{\pgfqpoint{2.912306in}{0.739656in}}%
\pgfpathlineto{\pgfqpoint{2.912010in}{0.739656in}}%
\pgfpathlineto{\pgfqpoint{2.911714in}{0.739656in}}%
\pgfpathlineto{\pgfqpoint{2.911418in}{0.739656in}}%
\pgfpathlineto{\pgfqpoint{2.911122in}{0.739656in}}%
\pgfpathlineto{\pgfqpoint{2.910826in}{0.739656in}}%
\pgfpathlineto{\pgfqpoint{2.910530in}{0.739656in}}%
\pgfpathlineto{\pgfqpoint{2.910234in}{0.739656in}}%
\pgfpathlineto{\pgfqpoint{2.909938in}{0.739656in}}%
\pgfpathlineto{\pgfqpoint{2.909642in}{0.739656in}}%
\pgfpathlineto{\pgfqpoint{2.909346in}{0.739656in}}%
\pgfpathlineto{\pgfqpoint{2.909050in}{0.739656in}}%
\pgfpathlineto{\pgfqpoint{2.908754in}{0.739656in}}%
\pgfpathlineto{\pgfqpoint{2.908458in}{0.739656in}}%
\pgfpathlineto{\pgfqpoint{2.908162in}{0.739656in}}%
\pgfpathlineto{\pgfqpoint{2.907866in}{0.739656in}}%
\pgfpathlineto{\pgfqpoint{2.907570in}{0.739656in}}%
\pgfpathlineto{\pgfqpoint{2.907274in}{0.739656in}}%
\pgfpathlineto{\pgfqpoint{2.906978in}{0.739656in}}%
\pgfpathlineto{\pgfqpoint{2.906682in}{0.739656in}}%
\pgfpathlineto{\pgfqpoint{2.906386in}{0.739656in}}%
\pgfpathlineto{\pgfqpoint{2.906090in}{0.739656in}}%
\pgfpathlineto{\pgfqpoint{2.905794in}{0.739656in}}%
\pgfpathlineto{\pgfqpoint{2.905498in}{0.739656in}}%
\pgfpathlineto{\pgfqpoint{2.905202in}{0.739656in}}%
\pgfpathlineto{\pgfqpoint{2.904906in}{0.739656in}}%
\pgfpathlineto{\pgfqpoint{2.904610in}{0.739656in}}%
\pgfpathlineto{\pgfqpoint{2.904314in}{0.739656in}}%
\pgfpathlineto{\pgfqpoint{2.904018in}{0.739656in}}%
\pgfpathlineto{\pgfqpoint{2.903722in}{0.739656in}}%
\pgfpathlineto{\pgfqpoint{2.903426in}{0.739656in}}%
\pgfpathlineto{\pgfqpoint{2.903130in}{0.739656in}}%
\pgfpathlineto{\pgfqpoint{2.902834in}{0.739656in}}%
\pgfpathlineto{\pgfqpoint{2.902538in}{0.739656in}}%
\pgfpathlineto{\pgfqpoint{2.902242in}{0.739656in}}%
\pgfpathlineto{\pgfqpoint{2.901946in}{0.739656in}}%
\pgfpathlineto{\pgfqpoint{2.901650in}{0.739656in}}%
\pgfpathlineto{\pgfqpoint{2.901354in}{0.739656in}}%
\pgfpathlineto{\pgfqpoint{2.901058in}{0.739656in}}%
\pgfpathlineto{\pgfqpoint{2.900762in}{0.739656in}}%
\pgfpathlineto{\pgfqpoint{2.900466in}{0.739656in}}%
\pgfpathlineto{\pgfqpoint{2.900170in}{0.739656in}}%
\pgfpathlineto{\pgfqpoint{2.899874in}{0.739656in}}%
\pgfpathlineto{\pgfqpoint{2.899578in}{0.739656in}}%
\pgfpathlineto{\pgfqpoint{2.899282in}{0.739656in}}%
\pgfpathlineto{\pgfqpoint{2.898986in}{0.739656in}}%
\pgfpathlineto{\pgfqpoint{2.898690in}{0.739656in}}%
\pgfpathlineto{\pgfqpoint{2.898394in}{0.739656in}}%
\pgfpathlineto{\pgfqpoint{2.898098in}{0.739656in}}%
\pgfpathlineto{\pgfqpoint{2.897802in}{0.739656in}}%
\pgfpathlineto{\pgfqpoint{2.897506in}{0.739656in}}%
\pgfpathlineto{\pgfqpoint{2.897210in}{0.739656in}}%
\pgfpathlineto{\pgfqpoint{2.896914in}{0.739656in}}%
\pgfpathlineto{\pgfqpoint{2.896618in}{0.739656in}}%
\pgfpathlineto{\pgfqpoint{2.896322in}{0.739656in}}%
\pgfpathlineto{\pgfqpoint{2.896026in}{0.739656in}}%
\pgfpathlineto{\pgfqpoint{2.895730in}{0.739656in}}%
\pgfpathlineto{\pgfqpoint{2.895434in}{0.739656in}}%
\pgfpathlineto{\pgfqpoint{2.895138in}{0.739656in}}%
\pgfpathlineto{\pgfqpoint{2.894842in}{0.739656in}}%
\pgfpathlineto{\pgfqpoint{2.894546in}{0.739656in}}%
\pgfpathlineto{\pgfqpoint{2.894250in}{0.739656in}}%
\pgfpathlineto{\pgfqpoint{2.893954in}{0.739656in}}%
\pgfpathlineto{\pgfqpoint{2.893658in}{0.739656in}}%
\pgfpathlineto{\pgfqpoint{2.893362in}{0.739656in}}%
\pgfpathlineto{\pgfqpoint{2.893066in}{0.739656in}}%
\pgfpathlineto{\pgfqpoint{2.892770in}{0.739656in}}%
\pgfpathlineto{\pgfqpoint{2.892474in}{0.739656in}}%
\pgfpathlineto{\pgfqpoint{2.892178in}{0.739656in}}%
\pgfpathlineto{\pgfqpoint{2.891882in}{0.739656in}}%
\pgfpathlineto{\pgfqpoint{2.891586in}{0.739656in}}%
\pgfpathlineto{\pgfqpoint{2.891290in}{0.739656in}}%
\pgfpathlineto{\pgfqpoint{2.890994in}{0.739656in}}%
\pgfpathlineto{\pgfqpoint{2.890698in}{0.739656in}}%
\pgfpathlineto{\pgfqpoint{2.890402in}{0.739656in}}%
\pgfpathlineto{\pgfqpoint{2.890106in}{0.739656in}}%
\pgfpathlineto{\pgfqpoint{2.889810in}{0.739656in}}%
\pgfpathlineto{\pgfqpoint{2.889514in}{0.739656in}}%
\pgfpathlineto{\pgfqpoint{2.889218in}{0.739656in}}%
\pgfpathlineto{\pgfqpoint{2.888922in}{0.739656in}}%
\pgfpathlineto{\pgfqpoint{2.888626in}{0.739656in}}%
\pgfpathlineto{\pgfqpoint{2.888330in}{0.739656in}}%
\pgfpathlineto{\pgfqpoint{2.888034in}{0.739656in}}%
\pgfpathlineto{\pgfqpoint{2.887738in}{0.739656in}}%
\pgfpathlineto{\pgfqpoint{2.887442in}{0.739656in}}%
\pgfpathlineto{\pgfqpoint{2.887146in}{0.739656in}}%
\pgfpathlineto{\pgfqpoint{2.886850in}{0.739656in}}%
\pgfpathlineto{\pgfqpoint{2.886554in}{0.739656in}}%
\pgfpathlineto{\pgfqpoint{2.886258in}{0.739656in}}%
\pgfpathlineto{\pgfqpoint{2.885962in}{0.739656in}}%
\pgfpathlineto{\pgfqpoint{2.885666in}{0.739656in}}%
\pgfpathlineto{\pgfqpoint{2.885370in}{0.739656in}}%
\pgfpathlineto{\pgfqpoint{2.885074in}{0.739656in}}%
\pgfpathlineto{\pgfqpoint{2.884778in}{0.739656in}}%
\pgfpathlineto{\pgfqpoint{2.884482in}{0.739656in}}%
\pgfpathlineto{\pgfqpoint{2.884186in}{0.739656in}}%
\pgfpathlineto{\pgfqpoint{2.883890in}{0.739656in}}%
\pgfpathlineto{\pgfqpoint{2.883594in}{0.739656in}}%
\pgfpathlineto{\pgfqpoint{2.883298in}{0.739656in}}%
\pgfpathlineto{\pgfqpoint{2.883002in}{0.739656in}}%
\pgfpathlineto{\pgfqpoint{2.882706in}{0.739656in}}%
\pgfpathlineto{\pgfqpoint{2.882410in}{0.739656in}}%
\pgfpathlineto{\pgfqpoint{2.882114in}{0.739656in}}%
\pgfpathlineto{\pgfqpoint{2.881818in}{0.739656in}}%
\pgfpathlineto{\pgfqpoint{2.881522in}{0.739656in}}%
\pgfpathlineto{\pgfqpoint{2.881226in}{0.739656in}}%
\pgfpathlineto{\pgfqpoint{2.880930in}{0.739656in}}%
\pgfpathlineto{\pgfqpoint{2.880634in}{0.739656in}}%
\pgfpathlineto{\pgfqpoint{2.880338in}{0.739656in}}%
\pgfpathlineto{\pgfqpoint{2.880042in}{0.739656in}}%
\pgfpathlineto{\pgfqpoint{2.879746in}{0.739656in}}%
\pgfpathlineto{\pgfqpoint{2.879450in}{0.739656in}}%
\pgfpathlineto{\pgfqpoint{2.879154in}{0.739656in}}%
\pgfpathlineto{\pgfqpoint{2.878858in}{0.739656in}}%
\pgfpathlineto{\pgfqpoint{2.878562in}{0.739656in}}%
\pgfpathlineto{\pgfqpoint{2.878266in}{0.739656in}}%
\pgfpathlineto{\pgfqpoint{2.877970in}{0.739656in}}%
\pgfpathlineto{\pgfqpoint{2.877674in}{0.739656in}}%
\pgfpathlineto{\pgfqpoint{2.877378in}{0.739656in}}%
\pgfpathlineto{\pgfqpoint{2.877082in}{0.739656in}}%
\pgfpathlineto{\pgfqpoint{2.876786in}{0.739656in}}%
\pgfpathlineto{\pgfqpoint{2.876490in}{0.739656in}}%
\pgfpathlineto{\pgfqpoint{2.876194in}{0.739656in}}%
\pgfpathlineto{\pgfqpoint{2.875898in}{0.739656in}}%
\pgfpathlineto{\pgfqpoint{2.875602in}{0.739656in}}%
\pgfpathlineto{\pgfqpoint{2.875306in}{0.739656in}}%
\pgfpathlineto{\pgfqpoint{2.875010in}{0.739656in}}%
\pgfpathlineto{\pgfqpoint{2.874714in}{0.739656in}}%
\pgfpathlineto{\pgfqpoint{2.874418in}{0.739656in}}%
\pgfpathlineto{\pgfqpoint{2.874122in}{0.739656in}}%
\pgfpathlineto{\pgfqpoint{2.873826in}{0.739656in}}%
\pgfpathlineto{\pgfqpoint{2.873530in}{0.739656in}}%
\pgfpathlineto{\pgfqpoint{2.873234in}{0.739656in}}%
\pgfpathlineto{\pgfqpoint{2.872938in}{0.739656in}}%
\pgfpathlineto{\pgfqpoint{2.872641in}{0.739656in}}%
\pgfpathlineto{\pgfqpoint{2.872345in}{0.739656in}}%
\pgfpathlineto{\pgfqpoint{2.872049in}{0.739656in}}%
\pgfpathlineto{\pgfqpoint{2.871753in}{0.739656in}}%
\pgfpathlineto{\pgfqpoint{2.871457in}{0.739656in}}%
\pgfpathlineto{\pgfqpoint{2.871161in}{0.739656in}}%
\pgfpathlineto{\pgfqpoint{2.870865in}{0.739656in}}%
\pgfpathlineto{\pgfqpoint{2.870569in}{0.739656in}}%
\pgfpathlineto{\pgfqpoint{2.870273in}{0.739656in}}%
\pgfpathlineto{\pgfqpoint{2.869977in}{0.739656in}}%
\pgfpathlineto{\pgfqpoint{2.869681in}{0.739656in}}%
\pgfpathlineto{\pgfqpoint{2.869385in}{0.739656in}}%
\pgfpathlineto{\pgfqpoint{2.869089in}{0.739656in}}%
\pgfpathlineto{\pgfqpoint{2.868793in}{0.739656in}}%
\pgfpathlineto{\pgfqpoint{2.868497in}{0.739656in}}%
\pgfpathlineto{\pgfqpoint{2.868201in}{0.739656in}}%
\pgfpathlineto{\pgfqpoint{2.867905in}{0.739656in}}%
\pgfpathlineto{\pgfqpoint{2.867609in}{0.739656in}}%
\pgfpathlineto{\pgfqpoint{2.867313in}{0.739656in}}%
\pgfpathlineto{\pgfqpoint{2.867017in}{0.739656in}}%
\pgfpathlineto{\pgfqpoint{2.866721in}{0.739656in}}%
\pgfpathlineto{\pgfqpoint{2.866425in}{0.739656in}}%
\pgfpathlineto{\pgfqpoint{2.866129in}{0.739656in}}%
\pgfpathlineto{\pgfqpoint{2.865833in}{0.739656in}}%
\pgfpathlineto{\pgfqpoint{2.865537in}{0.739656in}}%
\pgfpathlineto{\pgfqpoint{2.865241in}{0.739656in}}%
\pgfpathlineto{\pgfqpoint{2.864945in}{0.739656in}}%
\pgfpathlineto{\pgfqpoint{2.864649in}{0.739656in}}%
\pgfpathlineto{\pgfqpoint{2.864353in}{0.739656in}}%
\pgfpathlineto{\pgfqpoint{2.864057in}{0.739656in}}%
\pgfpathlineto{\pgfqpoint{2.863761in}{0.739656in}}%
\pgfpathlineto{\pgfqpoint{2.863465in}{0.739656in}}%
\pgfpathlineto{\pgfqpoint{2.863169in}{0.739656in}}%
\pgfpathlineto{\pgfqpoint{2.862873in}{0.739656in}}%
\pgfpathlineto{\pgfqpoint{2.862577in}{0.739656in}}%
\pgfpathlineto{\pgfqpoint{2.862281in}{0.739656in}}%
\pgfpathlineto{\pgfqpoint{2.861985in}{0.739656in}}%
\pgfpathlineto{\pgfqpoint{2.861689in}{0.739656in}}%
\pgfpathlineto{\pgfqpoint{2.861393in}{0.739656in}}%
\pgfpathlineto{\pgfqpoint{2.861097in}{0.739656in}}%
\pgfpathlineto{\pgfqpoint{2.860801in}{0.739656in}}%
\pgfpathlineto{\pgfqpoint{2.860505in}{0.739656in}}%
\pgfpathlineto{\pgfqpoint{2.860209in}{0.739656in}}%
\pgfpathlineto{\pgfqpoint{2.859913in}{0.739656in}}%
\pgfpathlineto{\pgfqpoint{2.859617in}{0.739656in}}%
\pgfpathlineto{\pgfqpoint{2.859321in}{0.739656in}}%
\pgfpathlineto{\pgfqpoint{2.859025in}{0.739656in}}%
\pgfpathlineto{\pgfqpoint{2.858729in}{0.739656in}}%
\pgfpathlineto{\pgfqpoint{2.858433in}{0.739656in}}%
\pgfpathlineto{\pgfqpoint{2.858137in}{0.739656in}}%
\pgfpathlineto{\pgfqpoint{2.857841in}{0.739656in}}%
\pgfpathlineto{\pgfqpoint{2.857545in}{0.739656in}}%
\pgfpathlineto{\pgfqpoint{2.857249in}{0.739656in}}%
\pgfpathlineto{\pgfqpoint{2.856953in}{0.739656in}}%
\pgfpathlineto{\pgfqpoint{2.856657in}{0.739656in}}%
\pgfpathlineto{\pgfqpoint{2.856361in}{0.739656in}}%
\pgfpathlineto{\pgfqpoint{2.856065in}{0.739656in}}%
\pgfpathlineto{\pgfqpoint{2.855769in}{0.739656in}}%
\pgfpathlineto{\pgfqpoint{2.855473in}{0.739656in}}%
\pgfpathlineto{\pgfqpoint{2.855177in}{0.739656in}}%
\pgfpathlineto{\pgfqpoint{2.854881in}{0.739656in}}%
\pgfpathlineto{\pgfqpoint{2.854585in}{0.739656in}}%
\pgfpathlineto{\pgfqpoint{2.854289in}{0.739656in}}%
\pgfpathlineto{\pgfqpoint{2.853993in}{0.739656in}}%
\pgfpathlineto{\pgfqpoint{2.853697in}{0.739656in}}%
\pgfpathlineto{\pgfqpoint{2.853401in}{0.739656in}}%
\pgfpathlineto{\pgfqpoint{2.853105in}{0.739656in}}%
\pgfpathlineto{\pgfqpoint{2.852809in}{0.739656in}}%
\pgfpathlineto{\pgfqpoint{2.852513in}{0.739656in}}%
\pgfpathlineto{\pgfqpoint{2.852217in}{0.739656in}}%
\pgfpathlineto{\pgfqpoint{2.851921in}{0.739656in}}%
\pgfpathlineto{\pgfqpoint{2.851625in}{0.739656in}}%
\pgfpathlineto{\pgfqpoint{2.851329in}{0.739656in}}%
\pgfpathlineto{\pgfqpoint{2.851033in}{0.739656in}}%
\pgfpathlineto{\pgfqpoint{2.850737in}{0.739656in}}%
\pgfpathlineto{\pgfqpoint{2.850441in}{0.739656in}}%
\pgfpathlineto{\pgfqpoint{2.850145in}{0.739656in}}%
\pgfpathlineto{\pgfqpoint{2.849849in}{0.739656in}}%
\pgfpathlineto{\pgfqpoint{2.849553in}{0.739656in}}%
\pgfpathlineto{\pgfqpoint{2.849257in}{0.739656in}}%
\pgfpathlineto{\pgfqpoint{2.848961in}{0.739656in}}%
\pgfpathlineto{\pgfqpoint{2.848665in}{0.739656in}}%
\pgfpathlineto{\pgfqpoint{2.848369in}{0.739656in}}%
\pgfpathlineto{\pgfqpoint{2.848073in}{0.739656in}}%
\pgfpathlineto{\pgfqpoint{2.847777in}{0.739656in}}%
\pgfpathlineto{\pgfqpoint{2.847481in}{0.739656in}}%
\pgfpathlineto{\pgfqpoint{2.847185in}{0.739656in}}%
\pgfpathlineto{\pgfqpoint{2.846889in}{0.739656in}}%
\pgfpathlineto{\pgfqpoint{2.846593in}{0.739656in}}%
\pgfpathlineto{\pgfqpoint{2.846297in}{0.739656in}}%
\pgfpathlineto{\pgfqpoint{2.846001in}{0.739656in}}%
\pgfpathlineto{\pgfqpoint{2.845705in}{0.739656in}}%
\pgfpathlineto{\pgfqpoint{2.845409in}{0.739656in}}%
\pgfpathlineto{\pgfqpoint{2.845113in}{0.739656in}}%
\pgfpathlineto{\pgfqpoint{2.844817in}{0.739656in}}%
\pgfpathlineto{\pgfqpoint{2.844521in}{0.739656in}}%
\pgfpathlineto{\pgfqpoint{2.844225in}{0.739656in}}%
\pgfpathlineto{\pgfqpoint{2.843929in}{0.739656in}}%
\pgfpathlineto{\pgfqpoint{2.843633in}{0.739656in}}%
\pgfpathlineto{\pgfqpoint{2.843337in}{0.739656in}}%
\pgfpathlineto{\pgfqpoint{2.843041in}{0.739656in}}%
\pgfpathlineto{\pgfqpoint{2.842745in}{0.739656in}}%
\pgfpathlineto{\pgfqpoint{2.842449in}{0.739656in}}%
\pgfpathlineto{\pgfqpoint{2.842153in}{0.739656in}}%
\pgfpathlineto{\pgfqpoint{2.841857in}{0.739656in}}%
\pgfpathlineto{\pgfqpoint{2.841561in}{0.739656in}}%
\pgfpathlineto{\pgfqpoint{2.841265in}{0.739656in}}%
\pgfpathlineto{\pgfqpoint{2.840969in}{0.739656in}}%
\pgfpathlineto{\pgfqpoint{2.840673in}{0.739656in}}%
\pgfpathlineto{\pgfqpoint{2.840377in}{0.739656in}}%
\pgfpathlineto{\pgfqpoint{2.840081in}{0.739656in}}%
\pgfpathlineto{\pgfqpoint{2.839785in}{0.739656in}}%
\pgfpathlineto{\pgfqpoint{2.839489in}{0.739656in}}%
\pgfpathlineto{\pgfqpoint{2.839193in}{0.739656in}}%
\pgfpathlineto{\pgfqpoint{2.838897in}{0.739656in}}%
\pgfpathlineto{\pgfqpoint{2.838601in}{0.739656in}}%
\pgfpathlineto{\pgfqpoint{2.838305in}{0.739656in}}%
\pgfpathlineto{\pgfqpoint{2.838009in}{0.739656in}}%
\pgfpathlineto{\pgfqpoint{2.837713in}{0.739656in}}%
\pgfpathlineto{\pgfqpoint{2.837417in}{0.739656in}}%
\pgfpathlineto{\pgfqpoint{2.837121in}{0.739656in}}%
\pgfpathlineto{\pgfqpoint{2.836825in}{0.739656in}}%
\pgfpathlineto{\pgfqpoint{2.836529in}{0.739656in}}%
\pgfpathlineto{\pgfqpoint{2.836233in}{0.739656in}}%
\pgfpathlineto{\pgfqpoint{2.835937in}{0.739656in}}%
\pgfpathlineto{\pgfqpoint{2.835641in}{0.739656in}}%
\pgfpathlineto{\pgfqpoint{2.835345in}{0.739656in}}%
\pgfpathlineto{\pgfqpoint{2.835049in}{0.739656in}}%
\pgfpathlineto{\pgfqpoint{2.834753in}{0.739656in}}%
\pgfpathlineto{\pgfqpoint{2.834457in}{0.739656in}}%
\pgfpathlineto{\pgfqpoint{2.834161in}{0.739656in}}%
\pgfpathlineto{\pgfqpoint{2.833865in}{0.739656in}}%
\pgfpathlineto{\pgfqpoint{2.833569in}{0.739656in}}%
\pgfpathlineto{\pgfqpoint{2.833273in}{0.739656in}}%
\pgfpathlineto{\pgfqpoint{2.832977in}{0.739656in}}%
\pgfpathlineto{\pgfqpoint{2.832681in}{0.739656in}}%
\pgfpathlineto{\pgfqpoint{2.832385in}{0.739656in}}%
\pgfpathlineto{\pgfqpoint{2.832089in}{0.739656in}}%
\pgfpathlineto{\pgfqpoint{2.831793in}{0.739656in}}%
\pgfpathlineto{\pgfqpoint{2.831497in}{0.739656in}}%
\pgfpathlineto{\pgfqpoint{2.831201in}{0.739656in}}%
\pgfpathlineto{\pgfqpoint{2.830905in}{0.739656in}}%
\pgfpathlineto{\pgfqpoint{2.830609in}{0.739656in}}%
\pgfpathlineto{\pgfqpoint{2.830313in}{0.739656in}}%
\pgfpathlineto{\pgfqpoint{2.830017in}{0.739656in}}%
\pgfpathlineto{\pgfqpoint{2.829721in}{0.739656in}}%
\pgfpathlineto{\pgfqpoint{2.829425in}{0.739656in}}%
\pgfpathlineto{\pgfqpoint{2.829129in}{0.739656in}}%
\pgfpathlineto{\pgfqpoint{2.828833in}{0.739656in}}%
\pgfpathlineto{\pgfqpoint{2.828537in}{0.739656in}}%
\pgfpathlineto{\pgfqpoint{2.828241in}{0.739656in}}%
\pgfpathlineto{\pgfqpoint{2.827945in}{0.739656in}}%
\pgfpathlineto{\pgfqpoint{2.827649in}{0.739656in}}%
\pgfpathlineto{\pgfqpoint{2.827353in}{0.739656in}}%
\pgfpathlineto{\pgfqpoint{2.827057in}{0.739656in}}%
\pgfpathlineto{\pgfqpoint{2.826761in}{0.739656in}}%
\pgfpathlineto{\pgfqpoint{2.826465in}{0.739656in}}%
\pgfpathlineto{\pgfqpoint{2.826169in}{0.739656in}}%
\pgfpathlineto{\pgfqpoint{2.825873in}{0.739656in}}%
\pgfpathlineto{\pgfqpoint{2.825577in}{0.739656in}}%
\pgfpathlineto{\pgfqpoint{2.825281in}{0.739656in}}%
\pgfpathlineto{\pgfqpoint{2.824985in}{0.739656in}}%
\pgfpathlineto{\pgfqpoint{2.824689in}{0.739656in}}%
\pgfpathlineto{\pgfqpoint{2.824393in}{0.739656in}}%
\pgfpathlineto{\pgfqpoint{2.824097in}{0.739656in}}%
\pgfpathlineto{\pgfqpoint{2.823801in}{0.739656in}}%
\pgfpathlineto{\pgfqpoint{2.823505in}{0.739656in}}%
\pgfpathlineto{\pgfqpoint{2.823209in}{0.739656in}}%
\pgfpathlineto{\pgfqpoint{2.822913in}{0.739656in}}%
\pgfpathlineto{\pgfqpoint{2.822617in}{0.739656in}}%
\pgfpathlineto{\pgfqpoint{2.822321in}{0.739656in}}%
\pgfpathlineto{\pgfqpoint{2.822025in}{0.739656in}}%
\pgfpathlineto{\pgfqpoint{2.821729in}{0.739656in}}%
\pgfpathlineto{\pgfqpoint{2.821433in}{0.739656in}}%
\pgfpathlineto{\pgfqpoint{2.821137in}{0.739656in}}%
\pgfpathlineto{\pgfqpoint{2.820841in}{0.739656in}}%
\pgfpathlineto{\pgfqpoint{2.820545in}{0.739656in}}%
\pgfpathlineto{\pgfqpoint{2.820249in}{0.739656in}}%
\pgfpathlineto{\pgfqpoint{2.819953in}{0.739656in}}%
\pgfpathlineto{\pgfqpoint{2.819657in}{0.739656in}}%
\pgfpathlineto{\pgfqpoint{2.819361in}{0.739656in}}%
\pgfpathlineto{\pgfqpoint{2.819065in}{0.739656in}}%
\pgfpathlineto{\pgfqpoint{2.818769in}{0.739656in}}%
\pgfpathlineto{\pgfqpoint{2.818473in}{0.739656in}}%
\pgfpathlineto{\pgfqpoint{2.818177in}{0.739656in}}%
\pgfpathlineto{\pgfqpoint{2.817881in}{0.739656in}}%
\pgfpathlineto{\pgfqpoint{2.817585in}{0.739656in}}%
\pgfpathlineto{\pgfqpoint{2.817289in}{0.739656in}}%
\pgfpathlineto{\pgfqpoint{2.816993in}{0.739656in}}%
\pgfpathlineto{\pgfqpoint{2.816697in}{0.739656in}}%
\pgfpathlineto{\pgfqpoint{2.816401in}{0.739656in}}%
\pgfpathlineto{\pgfqpoint{2.816105in}{0.739656in}}%
\pgfpathlineto{\pgfqpoint{2.815809in}{0.739656in}}%
\pgfpathlineto{\pgfqpoint{2.815513in}{0.739656in}}%
\pgfpathlineto{\pgfqpoint{2.815217in}{0.739656in}}%
\pgfpathlineto{\pgfqpoint{2.814921in}{0.739656in}}%
\pgfpathlineto{\pgfqpoint{2.814625in}{0.739656in}}%
\pgfpathlineto{\pgfqpoint{2.814329in}{0.739656in}}%
\pgfpathlineto{\pgfqpoint{2.814033in}{0.739656in}}%
\pgfpathlineto{\pgfqpoint{2.813737in}{0.739656in}}%
\pgfpathlineto{\pgfqpoint{2.813441in}{0.739656in}}%
\pgfpathlineto{\pgfqpoint{2.813145in}{0.739656in}}%
\pgfpathlineto{\pgfqpoint{2.812849in}{0.739656in}}%
\pgfpathlineto{\pgfqpoint{2.812553in}{0.739656in}}%
\pgfpathlineto{\pgfqpoint{2.812257in}{0.739656in}}%
\pgfpathlineto{\pgfqpoint{2.811961in}{0.739656in}}%
\pgfpathlineto{\pgfqpoint{2.811665in}{0.739656in}}%
\pgfpathlineto{\pgfqpoint{2.811369in}{0.739656in}}%
\pgfpathlineto{\pgfqpoint{2.811073in}{0.739656in}}%
\pgfpathlineto{\pgfqpoint{2.810777in}{0.739656in}}%
\pgfpathlineto{\pgfqpoint{2.810481in}{0.739656in}}%
\pgfpathlineto{\pgfqpoint{2.810185in}{0.739656in}}%
\pgfpathlineto{\pgfqpoint{2.809889in}{0.739656in}}%
\pgfpathlineto{\pgfqpoint{2.809593in}{0.739656in}}%
\pgfpathlineto{\pgfqpoint{2.809297in}{0.739656in}}%
\pgfpathlineto{\pgfqpoint{2.809001in}{0.739656in}}%
\pgfpathlineto{\pgfqpoint{2.808705in}{0.739656in}}%
\pgfpathlineto{\pgfqpoint{2.808409in}{0.739656in}}%
\pgfpathlineto{\pgfqpoint{2.808113in}{0.739656in}}%
\pgfpathlineto{\pgfqpoint{2.807817in}{0.739656in}}%
\pgfpathlineto{\pgfqpoint{2.807521in}{0.739656in}}%
\pgfpathlineto{\pgfqpoint{2.807225in}{0.739656in}}%
\pgfpathlineto{\pgfqpoint{2.806929in}{0.739656in}}%
\pgfpathlineto{\pgfqpoint{2.806633in}{0.739656in}}%
\pgfpathlineto{\pgfqpoint{2.806337in}{0.739656in}}%
\pgfpathlineto{\pgfqpoint{2.806041in}{0.739656in}}%
\pgfpathlineto{\pgfqpoint{2.805745in}{0.739656in}}%
\pgfpathlineto{\pgfqpoint{2.805449in}{0.739656in}}%
\pgfpathlineto{\pgfqpoint{2.805152in}{0.739656in}}%
\pgfpathlineto{\pgfqpoint{2.804856in}{0.739656in}}%
\pgfpathlineto{\pgfqpoint{2.804560in}{0.739656in}}%
\pgfpathlineto{\pgfqpoint{2.804264in}{0.739656in}}%
\pgfpathlineto{\pgfqpoint{2.803968in}{0.739656in}}%
\pgfpathlineto{\pgfqpoint{2.803672in}{0.739656in}}%
\pgfpathlineto{\pgfqpoint{2.803376in}{0.739656in}}%
\pgfpathlineto{\pgfqpoint{2.803080in}{0.739656in}}%
\pgfpathlineto{\pgfqpoint{2.802784in}{0.739656in}}%
\pgfpathlineto{\pgfqpoint{2.802488in}{0.739656in}}%
\pgfpathlineto{\pgfqpoint{2.802192in}{0.739656in}}%
\pgfpathlineto{\pgfqpoint{2.801896in}{0.739656in}}%
\pgfpathlineto{\pgfqpoint{2.801600in}{0.739656in}}%
\pgfpathlineto{\pgfqpoint{2.801304in}{0.739656in}}%
\pgfpathlineto{\pgfqpoint{2.801008in}{0.739656in}}%
\pgfpathlineto{\pgfqpoint{2.800712in}{0.739656in}}%
\pgfpathlineto{\pgfqpoint{2.800416in}{0.739656in}}%
\pgfpathlineto{\pgfqpoint{2.800120in}{0.739656in}}%
\pgfpathlineto{\pgfqpoint{2.799824in}{0.739656in}}%
\pgfpathlineto{\pgfqpoint{2.799528in}{0.739656in}}%
\pgfpathlineto{\pgfqpoint{2.799232in}{0.739656in}}%
\pgfpathlineto{\pgfqpoint{2.798936in}{0.739656in}}%
\pgfpathlineto{\pgfqpoint{2.798640in}{0.739656in}}%
\pgfpathlineto{\pgfqpoint{2.798344in}{0.739656in}}%
\pgfpathlineto{\pgfqpoint{2.798048in}{0.739656in}}%
\pgfpathlineto{\pgfqpoint{2.797752in}{0.739656in}}%
\pgfpathlineto{\pgfqpoint{2.797456in}{0.739656in}}%
\pgfpathlineto{\pgfqpoint{2.797160in}{0.739656in}}%
\pgfpathlineto{\pgfqpoint{2.796864in}{0.739656in}}%
\pgfpathlineto{\pgfqpoint{2.796568in}{0.739656in}}%
\pgfpathlineto{\pgfqpoint{2.796272in}{0.739656in}}%
\pgfpathlineto{\pgfqpoint{2.795976in}{0.739656in}}%
\pgfpathlineto{\pgfqpoint{2.795680in}{0.739656in}}%
\pgfpathlineto{\pgfqpoint{2.795384in}{0.739656in}}%
\pgfpathlineto{\pgfqpoint{2.795088in}{0.739656in}}%
\pgfpathlineto{\pgfqpoint{2.794792in}{0.739656in}}%
\pgfpathlineto{\pgfqpoint{2.794496in}{0.739656in}}%
\pgfpathlineto{\pgfqpoint{2.794200in}{0.739656in}}%
\pgfpathlineto{\pgfqpoint{2.793904in}{0.739656in}}%
\pgfpathlineto{\pgfqpoint{2.793608in}{0.739656in}}%
\pgfpathlineto{\pgfqpoint{2.793312in}{0.739656in}}%
\pgfpathlineto{\pgfqpoint{2.793016in}{0.739656in}}%
\pgfpathlineto{\pgfqpoint{2.792720in}{0.739656in}}%
\pgfpathlineto{\pgfqpoint{2.792424in}{0.739656in}}%
\pgfpathlineto{\pgfqpoint{2.792128in}{0.739656in}}%
\pgfpathlineto{\pgfqpoint{2.791832in}{0.739656in}}%
\pgfpathlineto{\pgfqpoint{2.791536in}{0.739656in}}%
\pgfpathlineto{\pgfqpoint{2.791240in}{0.739656in}}%
\pgfpathlineto{\pgfqpoint{2.790944in}{0.739656in}}%
\pgfpathlineto{\pgfqpoint{2.790648in}{0.739656in}}%
\pgfpathlineto{\pgfqpoint{2.790352in}{0.739656in}}%
\pgfpathlineto{\pgfqpoint{2.790056in}{0.739656in}}%
\pgfpathlineto{\pgfqpoint{2.789760in}{0.739656in}}%
\pgfpathlineto{\pgfqpoint{2.789464in}{0.739656in}}%
\pgfpathlineto{\pgfqpoint{2.789168in}{0.739656in}}%
\pgfpathlineto{\pgfqpoint{2.788872in}{0.739656in}}%
\pgfpathlineto{\pgfqpoint{2.788576in}{0.739656in}}%
\pgfpathlineto{\pgfqpoint{2.788280in}{0.739656in}}%
\pgfpathlineto{\pgfqpoint{2.787984in}{0.739656in}}%
\pgfpathlineto{\pgfqpoint{2.787688in}{0.739656in}}%
\pgfpathlineto{\pgfqpoint{2.787392in}{0.739656in}}%
\pgfpathlineto{\pgfqpoint{2.787096in}{0.739656in}}%
\pgfpathlineto{\pgfqpoint{2.786800in}{0.739656in}}%
\pgfpathlineto{\pgfqpoint{2.786504in}{0.739656in}}%
\pgfpathlineto{\pgfqpoint{2.786208in}{0.739656in}}%
\pgfpathlineto{\pgfqpoint{2.785912in}{0.739656in}}%
\pgfpathlineto{\pgfqpoint{2.785616in}{0.739656in}}%
\pgfpathlineto{\pgfqpoint{2.785320in}{0.739656in}}%
\pgfpathlineto{\pgfqpoint{2.785024in}{0.739656in}}%
\pgfpathlineto{\pgfqpoint{2.784728in}{0.739656in}}%
\pgfpathlineto{\pgfqpoint{2.784432in}{0.739656in}}%
\pgfpathlineto{\pgfqpoint{2.784136in}{0.739656in}}%
\pgfpathlineto{\pgfqpoint{2.783840in}{0.739656in}}%
\pgfpathlineto{\pgfqpoint{2.783544in}{0.739656in}}%
\pgfpathlineto{\pgfqpoint{2.783248in}{0.739656in}}%
\pgfpathlineto{\pgfqpoint{2.782952in}{0.739656in}}%
\pgfpathlineto{\pgfqpoint{2.782656in}{0.739656in}}%
\pgfpathlineto{\pgfqpoint{2.782360in}{0.739656in}}%
\pgfpathlineto{\pgfqpoint{2.782064in}{0.739656in}}%
\pgfpathlineto{\pgfqpoint{2.781768in}{0.739656in}}%
\pgfpathlineto{\pgfqpoint{2.781472in}{0.739656in}}%
\pgfpathlineto{\pgfqpoint{2.781176in}{0.739656in}}%
\pgfpathlineto{\pgfqpoint{2.780880in}{0.739656in}}%
\pgfpathlineto{\pgfqpoint{2.780584in}{0.739656in}}%
\pgfpathlineto{\pgfqpoint{2.780288in}{0.739656in}}%
\pgfpathlineto{\pgfqpoint{2.779992in}{0.739656in}}%
\pgfpathlineto{\pgfqpoint{2.779696in}{0.739656in}}%
\pgfpathlineto{\pgfqpoint{2.779400in}{0.739656in}}%
\pgfpathlineto{\pgfqpoint{2.779104in}{0.739656in}}%
\pgfpathlineto{\pgfqpoint{2.778808in}{0.739656in}}%
\pgfpathlineto{\pgfqpoint{2.778512in}{0.739656in}}%
\pgfpathlineto{\pgfqpoint{2.778216in}{0.739656in}}%
\pgfpathlineto{\pgfqpoint{2.777920in}{0.739656in}}%
\pgfpathlineto{\pgfqpoint{2.777624in}{0.739656in}}%
\pgfpathlineto{\pgfqpoint{2.777328in}{0.739656in}}%
\pgfpathlineto{\pgfqpoint{2.777032in}{0.739656in}}%
\pgfpathlineto{\pgfqpoint{2.776736in}{0.739656in}}%
\pgfpathlineto{\pgfqpoint{2.776440in}{0.739656in}}%
\pgfpathlineto{\pgfqpoint{2.776144in}{0.739656in}}%
\pgfpathlineto{\pgfqpoint{2.775848in}{0.739656in}}%
\pgfpathlineto{\pgfqpoint{2.775552in}{0.739656in}}%
\pgfpathlineto{\pgfqpoint{2.775256in}{0.739656in}}%
\pgfpathlineto{\pgfqpoint{2.774960in}{0.739656in}}%
\pgfpathlineto{\pgfqpoint{2.774664in}{0.739656in}}%
\pgfpathlineto{\pgfqpoint{2.774368in}{0.739656in}}%
\pgfpathlineto{\pgfqpoint{2.774072in}{0.739656in}}%
\pgfpathlineto{\pgfqpoint{2.773776in}{0.739656in}}%
\pgfpathlineto{\pgfqpoint{2.773480in}{0.739656in}}%
\pgfpathlineto{\pgfqpoint{2.773184in}{0.739656in}}%
\pgfpathlineto{\pgfqpoint{2.772888in}{0.739656in}}%
\pgfpathlineto{\pgfqpoint{2.772592in}{0.739656in}}%
\pgfpathlineto{\pgfqpoint{2.772296in}{0.739656in}}%
\pgfpathlineto{\pgfqpoint{2.772000in}{0.739656in}}%
\pgfpathlineto{\pgfqpoint{2.771704in}{0.739656in}}%
\pgfpathlineto{\pgfqpoint{2.771408in}{0.739656in}}%
\pgfpathlineto{\pgfqpoint{2.771112in}{0.739656in}}%
\pgfpathlineto{\pgfqpoint{2.770816in}{0.739656in}}%
\pgfpathlineto{\pgfqpoint{2.770520in}{0.739656in}}%
\pgfpathlineto{\pgfqpoint{2.770224in}{0.739656in}}%
\pgfpathlineto{\pgfqpoint{2.769928in}{0.739656in}}%
\pgfpathlineto{\pgfqpoint{2.769632in}{0.739656in}}%
\pgfpathlineto{\pgfqpoint{2.769336in}{0.739656in}}%
\pgfpathlineto{\pgfqpoint{2.769040in}{0.739656in}}%
\pgfpathlineto{\pgfqpoint{2.768744in}{0.739656in}}%
\pgfpathlineto{\pgfqpoint{2.768448in}{0.739656in}}%
\pgfpathlineto{\pgfqpoint{2.768152in}{0.739656in}}%
\pgfpathlineto{\pgfqpoint{2.767856in}{0.739656in}}%
\pgfpathlineto{\pgfqpoint{2.767560in}{0.739656in}}%
\pgfpathlineto{\pgfqpoint{2.767264in}{0.739656in}}%
\pgfpathlineto{\pgfqpoint{2.766968in}{0.739656in}}%
\pgfpathlineto{\pgfqpoint{2.766672in}{0.739656in}}%
\pgfpathlineto{\pgfqpoint{2.766376in}{0.739656in}}%
\pgfpathlineto{\pgfqpoint{2.766080in}{0.739656in}}%
\pgfpathlineto{\pgfqpoint{2.765784in}{0.739656in}}%
\pgfpathlineto{\pgfqpoint{2.765488in}{0.739656in}}%
\pgfpathlineto{\pgfqpoint{2.765192in}{0.739656in}}%
\pgfpathlineto{\pgfqpoint{2.764896in}{0.739656in}}%
\pgfpathlineto{\pgfqpoint{2.764600in}{0.739656in}}%
\pgfpathlineto{\pgfqpoint{2.764304in}{0.739656in}}%
\pgfpathlineto{\pgfqpoint{2.764008in}{0.739656in}}%
\pgfpathlineto{\pgfqpoint{2.763712in}{0.739656in}}%
\pgfpathlineto{\pgfqpoint{2.763416in}{0.739656in}}%
\pgfpathlineto{\pgfqpoint{2.763120in}{0.739656in}}%
\pgfpathlineto{\pgfqpoint{2.762824in}{0.739656in}}%
\pgfpathlineto{\pgfqpoint{2.762528in}{0.739656in}}%
\pgfpathlineto{\pgfqpoint{2.762232in}{0.739656in}}%
\pgfpathlineto{\pgfqpoint{2.761936in}{0.739656in}}%
\pgfpathlineto{\pgfqpoint{2.761640in}{0.739656in}}%
\pgfpathlineto{\pgfqpoint{2.761344in}{0.739656in}}%
\pgfpathlineto{\pgfqpoint{2.761048in}{0.739656in}}%
\pgfpathlineto{\pgfqpoint{2.760752in}{0.739656in}}%
\pgfpathlineto{\pgfqpoint{2.760456in}{0.739656in}}%
\pgfpathlineto{\pgfqpoint{2.760160in}{0.739656in}}%
\pgfpathlineto{\pgfqpoint{2.759864in}{0.739656in}}%
\pgfpathlineto{\pgfqpoint{2.759568in}{0.739656in}}%
\pgfpathlineto{\pgfqpoint{2.759272in}{0.739656in}}%
\pgfpathlineto{\pgfqpoint{2.758976in}{0.739656in}}%
\pgfpathlineto{\pgfqpoint{2.758680in}{0.739656in}}%
\pgfpathlineto{\pgfqpoint{2.758384in}{0.739656in}}%
\pgfpathlineto{\pgfqpoint{2.758088in}{0.739656in}}%
\pgfpathlineto{\pgfqpoint{2.757792in}{0.739656in}}%
\pgfpathlineto{\pgfqpoint{2.757496in}{0.739656in}}%
\pgfpathlineto{\pgfqpoint{2.757200in}{0.739656in}}%
\pgfpathlineto{\pgfqpoint{2.756904in}{0.739656in}}%
\pgfpathlineto{\pgfqpoint{2.756608in}{0.739656in}}%
\pgfpathlineto{\pgfqpoint{2.756312in}{0.739656in}}%
\pgfpathlineto{\pgfqpoint{2.756016in}{0.739656in}}%
\pgfpathlineto{\pgfqpoint{2.755720in}{0.739656in}}%
\pgfpathlineto{\pgfqpoint{2.755424in}{0.739656in}}%
\pgfpathlineto{\pgfqpoint{2.755128in}{0.739656in}}%
\pgfpathlineto{\pgfqpoint{2.754832in}{0.739656in}}%
\pgfpathlineto{\pgfqpoint{2.754536in}{0.739656in}}%
\pgfpathlineto{\pgfqpoint{2.754240in}{0.739656in}}%
\pgfpathlineto{\pgfqpoint{2.753944in}{0.739656in}}%
\pgfpathlineto{\pgfqpoint{2.753648in}{0.739656in}}%
\pgfpathlineto{\pgfqpoint{2.753352in}{0.739656in}}%
\pgfpathlineto{\pgfqpoint{2.753056in}{0.739656in}}%
\pgfpathlineto{\pgfqpoint{2.752760in}{0.739656in}}%
\pgfpathlineto{\pgfqpoint{2.752464in}{0.739656in}}%
\pgfpathlineto{\pgfqpoint{2.752168in}{0.739656in}}%
\pgfpathlineto{\pgfqpoint{2.751872in}{0.739656in}}%
\pgfpathlineto{\pgfqpoint{2.751576in}{0.739656in}}%
\pgfpathlineto{\pgfqpoint{2.751280in}{0.739656in}}%
\pgfpathlineto{\pgfqpoint{2.750984in}{0.739656in}}%
\pgfpathlineto{\pgfqpoint{2.750688in}{0.739656in}}%
\pgfpathlineto{\pgfqpoint{2.750392in}{0.739656in}}%
\pgfpathlineto{\pgfqpoint{2.750096in}{0.739656in}}%
\pgfpathlineto{\pgfqpoint{2.749800in}{0.739656in}}%
\pgfpathlineto{\pgfqpoint{2.749504in}{0.739656in}}%
\pgfpathlineto{\pgfqpoint{2.749208in}{0.739656in}}%
\pgfpathlineto{\pgfqpoint{2.748912in}{0.739656in}}%
\pgfpathlineto{\pgfqpoint{2.748616in}{0.739656in}}%
\pgfpathlineto{\pgfqpoint{2.748320in}{0.739656in}}%
\pgfpathlineto{\pgfqpoint{2.748024in}{0.739656in}}%
\pgfpathlineto{\pgfqpoint{2.747728in}{0.739656in}}%
\pgfpathlineto{\pgfqpoint{2.747432in}{0.739656in}}%
\pgfpathlineto{\pgfqpoint{2.747136in}{0.739656in}}%
\pgfpathlineto{\pgfqpoint{2.746840in}{0.739656in}}%
\pgfpathlineto{\pgfqpoint{2.746544in}{0.739656in}}%
\pgfpathlineto{\pgfqpoint{2.746248in}{0.739656in}}%
\pgfpathlineto{\pgfqpoint{2.745952in}{0.739656in}}%
\pgfpathlineto{\pgfqpoint{2.745656in}{0.739656in}}%
\pgfpathlineto{\pgfqpoint{2.745360in}{0.739656in}}%
\pgfpathlineto{\pgfqpoint{2.745064in}{0.739656in}}%
\pgfpathlineto{\pgfqpoint{2.744768in}{0.739656in}}%
\pgfpathlineto{\pgfqpoint{2.744472in}{0.739656in}}%
\pgfpathlineto{\pgfqpoint{2.744176in}{0.739656in}}%
\pgfpathlineto{\pgfqpoint{2.743880in}{0.739656in}}%
\pgfpathlineto{\pgfqpoint{2.743584in}{0.739656in}}%
\pgfpathlineto{\pgfqpoint{2.743288in}{0.739656in}}%
\pgfpathlineto{\pgfqpoint{2.742992in}{0.739656in}}%
\pgfpathlineto{\pgfqpoint{2.742696in}{0.739656in}}%
\pgfpathlineto{\pgfqpoint{2.742400in}{0.739656in}}%
\pgfpathlineto{\pgfqpoint{2.742104in}{0.739656in}}%
\pgfpathlineto{\pgfqpoint{2.741808in}{0.739656in}}%
\pgfpathlineto{\pgfqpoint{2.741512in}{0.739656in}}%
\pgfpathlineto{\pgfqpoint{2.741216in}{0.739656in}}%
\pgfpathlineto{\pgfqpoint{2.740920in}{0.739656in}}%
\pgfpathlineto{\pgfqpoint{2.740624in}{0.739656in}}%
\pgfpathlineto{\pgfqpoint{2.740328in}{0.739656in}}%
\pgfpathlineto{\pgfqpoint{2.740032in}{0.739656in}}%
\pgfpathlineto{\pgfqpoint{2.739736in}{0.739656in}}%
\pgfpathlineto{\pgfqpoint{2.739440in}{0.739656in}}%
\pgfpathlineto{\pgfqpoint{2.739144in}{0.739656in}}%
\pgfpathlineto{\pgfqpoint{2.738848in}{0.739656in}}%
\pgfpathlineto{\pgfqpoint{2.738552in}{0.739656in}}%
\pgfpathlineto{\pgfqpoint{2.738256in}{0.739656in}}%
\pgfpathlineto{\pgfqpoint{2.737960in}{0.739656in}}%
\pgfpathlineto{\pgfqpoint{2.737663in}{0.739656in}}%
\pgfpathlineto{\pgfqpoint{2.737367in}{0.739656in}}%
\pgfpathlineto{\pgfqpoint{2.737071in}{0.739656in}}%
\pgfpathlineto{\pgfqpoint{2.736775in}{0.739656in}}%
\pgfpathlineto{\pgfqpoint{2.736479in}{0.739656in}}%
\pgfpathlineto{\pgfqpoint{2.736183in}{0.739656in}}%
\pgfpathlineto{\pgfqpoint{2.735887in}{0.739656in}}%
\pgfpathlineto{\pgfqpoint{2.735591in}{0.739656in}}%
\pgfpathlineto{\pgfqpoint{2.735295in}{0.739656in}}%
\pgfpathlineto{\pgfqpoint{2.734999in}{0.739656in}}%
\pgfpathlineto{\pgfqpoint{2.734703in}{0.739656in}}%
\pgfpathlineto{\pgfqpoint{2.734407in}{0.739656in}}%
\pgfpathlineto{\pgfqpoint{2.734111in}{0.739656in}}%
\pgfpathlineto{\pgfqpoint{2.733815in}{0.739656in}}%
\pgfpathlineto{\pgfqpoint{2.733519in}{0.739656in}}%
\pgfpathlineto{\pgfqpoint{2.733223in}{0.739656in}}%
\pgfpathlineto{\pgfqpoint{2.732927in}{0.739656in}}%
\pgfpathlineto{\pgfqpoint{2.732631in}{0.739656in}}%
\pgfpathlineto{\pgfqpoint{2.732335in}{0.739656in}}%
\pgfpathlineto{\pgfqpoint{2.732039in}{0.739656in}}%
\pgfpathlineto{\pgfqpoint{2.731743in}{0.739656in}}%
\pgfpathlineto{\pgfqpoint{2.731447in}{0.739656in}}%
\pgfpathlineto{\pgfqpoint{2.731151in}{0.739656in}}%
\pgfpathlineto{\pgfqpoint{2.730855in}{0.739656in}}%
\pgfpathlineto{\pgfqpoint{2.730559in}{0.739656in}}%
\pgfpathlineto{\pgfqpoint{2.730263in}{0.739656in}}%
\pgfpathlineto{\pgfqpoint{2.729967in}{0.739656in}}%
\pgfpathlineto{\pgfqpoint{2.729671in}{0.739656in}}%
\pgfpathlineto{\pgfqpoint{2.729375in}{0.739656in}}%
\pgfpathlineto{\pgfqpoint{2.729079in}{0.739656in}}%
\pgfpathlineto{\pgfqpoint{2.728783in}{0.739656in}}%
\pgfpathlineto{\pgfqpoint{2.728487in}{0.739656in}}%
\pgfpathlineto{\pgfqpoint{2.728191in}{0.739656in}}%
\pgfpathlineto{\pgfqpoint{2.727895in}{0.739656in}}%
\pgfpathlineto{\pgfqpoint{2.727599in}{0.739656in}}%
\pgfpathlineto{\pgfqpoint{2.727303in}{0.739656in}}%
\pgfpathlineto{\pgfqpoint{2.727007in}{0.739656in}}%
\pgfpathlineto{\pgfqpoint{2.726711in}{0.739656in}}%
\pgfpathlineto{\pgfqpoint{2.726415in}{0.739656in}}%
\pgfpathlineto{\pgfqpoint{2.726119in}{0.739656in}}%
\pgfpathlineto{\pgfqpoint{2.725823in}{0.739656in}}%
\pgfpathlineto{\pgfqpoint{2.725527in}{0.739656in}}%
\pgfpathlineto{\pgfqpoint{2.725231in}{0.739656in}}%
\pgfpathlineto{\pgfqpoint{2.724935in}{0.739656in}}%
\pgfpathlineto{\pgfqpoint{2.724639in}{0.739656in}}%
\pgfpathlineto{\pgfqpoint{2.724343in}{0.739656in}}%
\pgfpathlineto{\pgfqpoint{2.724047in}{0.739656in}}%
\pgfpathlineto{\pgfqpoint{2.723751in}{0.739656in}}%
\pgfpathlineto{\pgfqpoint{2.723455in}{0.739656in}}%
\pgfpathlineto{\pgfqpoint{2.723159in}{0.739656in}}%
\pgfpathlineto{\pgfqpoint{2.722863in}{0.739656in}}%
\pgfpathlineto{\pgfqpoint{2.722567in}{0.739656in}}%
\pgfpathlineto{\pgfqpoint{2.722271in}{0.739656in}}%
\pgfpathlineto{\pgfqpoint{2.721975in}{0.739656in}}%
\pgfpathlineto{\pgfqpoint{2.721679in}{0.739656in}}%
\pgfpathlineto{\pgfqpoint{2.721383in}{0.739656in}}%
\pgfpathlineto{\pgfqpoint{2.721087in}{0.739656in}}%
\pgfpathlineto{\pgfqpoint{2.720791in}{0.739656in}}%
\pgfpathlineto{\pgfqpoint{2.720495in}{0.739656in}}%
\pgfpathlineto{\pgfqpoint{2.720199in}{0.739656in}}%
\pgfpathlineto{\pgfqpoint{2.719903in}{0.739656in}}%
\pgfpathlineto{\pgfqpoint{2.719607in}{0.739656in}}%
\pgfpathlineto{\pgfqpoint{2.719311in}{0.739656in}}%
\pgfpathlineto{\pgfqpoint{2.719015in}{0.739656in}}%
\pgfpathlineto{\pgfqpoint{2.718719in}{0.739656in}}%
\pgfpathlineto{\pgfqpoint{2.718423in}{0.739656in}}%
\pgfpathlineto{\pgfqpoint{2.718127in}{0.739656in}}%
\pgfpathlineto{\pgfqpoint{2.717831in}{0.739656in}}%
\pgfpathlineto{\pgfqpoint{2.717535in}{0.739656in}}%
\pgfpathlineto{\pgfqpoint{2.717239in}{0.739656in}}%
\pgfpathlineto{\pgfqpoint{2.716943in}{0.739656in}}%
\pgfpathlineto{\pgfqpoint{2.716647in}{0.739656in}}%
\pgfpathlineto{\pgfqpoint{2.716351in}{0.739656in}}%
\pgfpathlineto{\pgfqpoint{2.716055in}{0.739656in}}%
\pgfpathlineto{\pgfqpoint{2.715759in}{0.739656in}}%
\pgfpathlineto{\pgfqpoint{2.715463in}{0.739656in}}%
\pgfpathlineto{\pgfqpoint{2.715167in}{0.739656in}}%
\pgfpathlineto{\pgfqpoint{2.714871in}{0.739656in}}%
\pgfpathlineto{\pgfqpoint{2.714575in}{0.739656in}}%
\pgfpathlineto{\pgfqpoint{2.714279in}{0.739656in}}%
\pgfpathlineto{\pgfqpoint{2.713983in}{0.739656in}}%
\pgfpathlineto{\pgfqpoint{2.713687in}{0.739656in}}%
\pgfpathlineto{\pgfqpoint{2.713391in}{0.739656in}}%
\pgfpathlineto{\pgfqpoint{2.713095in}{0.739656in}}%
\pgfpathlineto{\pgfqpoint{2.712799in}{0.739656in}}%
\pgfpathlineto{\pgfqpoint{2.712503in}{0.739656in}}%
\pgfpathlineto{\pgfqpoint{2.712207in}{0.739656in}}%
\pgfpathlineto{\pgfqpoint{2.711911in}{0.739656in}}%
\pgfpathlineto{\pgfqpoint{2.711615in}{0.739656in}}%
\pgfpathlineto{\pgfqpoint{2.711319in}{0.739656in}}%
\pgfpathlineto{\pgfqpoint{2.711023in}{0.739656in}}%
\pgfpathlineto{\pgfqpoint{2.710727in}{0.739656in}}%
\pgfpathlineto{\pgfqpoint{2.710431in}{0.739656in}}%
\pgfpathlineto{\pgfqpoint{2.710135in}{0.739656in}}%
\pgfpathlineto{\pgfqpoint{2.709839in}{0.739656in}}%
\pgfpathlineto{\pgfqpoint{2.709543in}{0.739656in}}%
\pgfpathlineto{\pgfqpoint{2.709247in}{0.739656in}}%
\pgfpathlineto{\pgfqpoint{2.708951in}{0.739656in}}%
\pgfpathlineto{\pgfqpoint{2.708655in}{0.739656in}}%
\pgfpathlineto{\pgfqpoint{2.708359in}{0.739656in}}%
\pgfpathlineto{\pgfqpoint{2.708063in}{0.739656in}}%
\pgfpathlineto{\pgfqpoint{2.707767in}{0.739656in}}%
\pgfpathlineto{\pgfqpoint{2.707471in}{0.739656in}}%
\pgfpathlineto{\pgfqpoint{2.707175in}{0.739656in}}%
\pgfpathlineto{\pgfqpoint{2.706879in}{0.739656in}}%
\pgfpathlineto{\pgfqpoint{2.706583in}{0.739656in}}%
\pgfpathlineto{\pgfqpoint{2.706287in}{0.739656in}}%
\pgfpathlineto{\pgfqpoint{2.705991in}{0.739656in}}%
\pgfpathlineto{\pgfqpoint{2.705695in}{0.739656in}}%
\pgfpathlineto{\pgfqpoint{2.705399in}{0.739656in}}%
\pgfpathlineto{\pgfqpoint{2.705103in}{0.739656in}}%
\pgfpathlineto{\pgfqpoint{2.704807in}{0.739656in}}%
\pgfpathlineto{\pgfqpoint{2.704511in}{0.739656in}}%
\pgfpathlineto{\pgfqpoint{2.704215in}{0.739656in}}%
\pgfpathlineto{\pgfqpoint{2.703919in}{0.739656in}}%
\pgfpathlineto{\pgfqpoint{2.703623in}{0.739656in}}%
\pgfpathlineto{\pgfqpoint{2.703327in}{0.739656in}}%
\pgfpathlineto{\pgfqpoint{2.703031in}{0.739656in}}%
\pgfpathlineto{\pgfqpoint{2.702735in}{0.739656in}}%
\pgfpathlineto{\pgfqpoint{2.702439in}{0.739656in}}%
\pgfpathlineto{\pgfqpoint{2.702143in}{0.739656in}}%
\pgfpathlineto{\pgfqpoint{2.701847in}{0.739656in}}%
\pgfpathlineto{\pgfqpoint{2.701551in}{0.739656in}}%
\pgfpathlineto{\pgfqpoint{2.701255in}{0.739656in}}%
\pgfpathlineto{\pgfqpoint{2.700959in}{0.739656in}}%
\pgfpathlineto{\pgfqpoint{2.700663in}{0.739656in}}%
\pgfpathlineto{\pgfqpoint{2.700367in}{0.739656in}}%
\pgfpathlineto{\pgfqpoint{2.700071in}{0.739656in}}%
\pgfpathlineto{\pgfqpoint{2.699775in}{0.739656in}}%
\pgfpathlineto{\pgfqpoint{2.699479in}{0.739656in}}%
\pgfpathlineto{\pgfqpoint{2.699183in}{0.739656in}}%
\pgfpathlineto{\pgfqpoint{2.698887in}{0.739656in}}%
\pgfpathlineto{\pgfqpoint{2.698591in}{0.739656in}}%
\pgfpathlineto{\pgfqpoint{2.698295in}{0.739656in}}%
\pgfpathlineto{\pgfqpoint{2.697999in}{0.739656in}}%
\pgfpathlineto{\pgfqpoint{2.697703in}{0.739656in}}%
\pgfpathlineto{\pgfqpoint{2.697407in}{0.739656in}}%
\pgfpathlineto{\pgfqpoint{2.697111in}{0.739656in}}%
\pgfpathlineto{\pgfqpoint{2.696815in}{0.739656in}}%
\pgfpathlineto{\pgfqpoint{2.696519in}{0.739656in}}%
\pgfpathlineto{\pgfqpoint{2.696223in}{0.739656in}}%
\pgfpathlineto{\pgfqpoint{2.695927in}{0.739656in}}%
\pgfpathlineto{\pgfqpoint{2.695631in}{0.739656in}}%
\pgfpathlineto{\pgfqpoint{2.695335in}{0.739656in}}%
\pgfpathlineto{\pgfqpoint{2.695039in}{0.739656in}}%
\pgfpathlineto{\pgfqpoint{2.694743in}{0.739656in}}%
\pgfpathlineto{\pgfqpoint{2.694447in}{0.739656in}}%
\pgfpathlineto{\pgfqpoint{2.694151in}{0.739656in}}%
\pgfpathlineto{\pgfqpoint{2.693855in}{0.739656in}}%
\pgfpathlineto{\pgfqpoint{2.693559in}{0.739656in}}%
\pgfpathlineto{\pgfqpoint{2.693263in}{0.739656in}}%
\pgfpathlineto{\pgfqpoint{2.692967in}{0.739656in}}%
\pgfpathlineto{\pgfqpoint{2.692671in}{0.739656in}}%
\pgfpathlineto{\pgfqpoint{2.692375in}{0.739656in}}%
\pgfpathlineto{\pgfqpoint{2.692079in}{0.739656in}}%
\pgfpathlineto{\pgfqpoint{2.691783in}{0.739656in}}%
\pgfpathlineto{\pgfqpoint{2.691487in}{0.739656in}}%
\pgfpathlineto{\pgfqpoint{2.691191in}{0.739656in}}%
\pgfpathlineto{\pgfqpoint{2.690895in}{0.739656in}}%
\pgfpathlineto{\pgfqpoint{2.690599in}{0.739656in}}%
\pgfpathlineto{\pgfqpoint{2.690303in}{0.739656in}}%
\pgfpathlineto{\pgfqpoint{2.690007in}{0.739656in}}%
\pgfpathlineto{\pgfqpoint{2.689711in}{0.739656in}}%
\pgfpathlineto{\pgfqpoint{2.689415in}{0.739656in}}%
\pgfpathlineto{\pgfqpoint{2.689119in}{0.739656in}}%
\pgfpathlineto{\pgfqpoint{2.688823in}{0.739656in}}%
\pgfpathlineto{\pgfqpoint{2.688527in}{0.739656in}}%
\pgfpathlineto{\pgfqpoint{2.688231in}{0.739656in}}%
\pgfpathlineto{\pgfqpoint{2.687935in}{0.739656in}}%
\pgfpathlineto{\pgfqpoint{2.687639in}{0.739656in}}%
\pgfpathlineto{\pgfqpoint{2.687343in}{0.739656in}}%
\pgfpathlineto{\pgfqpoint{2.687047in}{0.739656in}}%
\pgfpathlineto{\pgfqpoint{2.686751in}{0.739656in}}%
\pgfpathlineto{\pgfqpoint{2.686455in}{0.739656in}}%
\pgfpathlineto{\pgfqpoint{2.686159in}{0.739656in}}%
\pgfpathlineto{\pgfqpoint{2.685863in}{0.739656in}}%
\pgfpathlineto{\pgfqpoint{2.685567in}{0.739656in}}%
\pgfpathlineto{\pgfqpoint{2.685271in}{0.739656in}}%
\pgfpathlineto{\pgfqpoint{2.684975in}{0.739656in}}%
\pgfpathlineto{\pgfqpoint{2.684679in}{0.739656in}}%
\pgfpathlineto{\pgfqpoint{2.684383in}{0.739656in}}%
\pgfpathlineto{\pgfqpoint{2.684087in}{0.739656in}}%
\pgfpathlineto{\pgfqpoint{2.683791in}{0.739656in}}%
\pgfpathlineto{\pgfqpoint{2.683495in}{0.739656in}}%
\pgfpathlineto{\pgfqpoint{2.683199in}{0.739656in}}%
\pgfpathlineto{\pgfqpoint{2.682903in}{0.739656in}}%
\pgfpathlineto{\pgfqpoint{2.682607in}{0.739656in}}%
\pgfpathlineto{\pgfqpoint{2.682311in}{0.739656in}}%
\pgfpathlineto{\pgfqpoint{2.682015in}{0.739656in}}%
\pgfpathlineto{\pgfqpoint{2.681719in}{0.739656in}}%
\pgfpathlineto{\pgfqpoint{2.681423in}{0.739656in}}%
\pgfpathlineto{\pgfqpoint{2.681127in}{0.739656in}}%
\pgfpathlineto{\pgfqpoint{2.680831in}{0.739656in}}%
\pgfpathlineto{\pgfqpoint{2.680535in}{0.739656in}}%
\pgfpathlineto{\pgfqpoint{2.680239in}{0.739656in}}%
\pgfpathlineto{\pgfqpoint{2.679943in}{0.739656in}}%
\pgfpathlineto{\pgfqpoint{2.679647in}{0.739656in}}%
\pgfpathlineto{\pgfqpoint{2.679351in}{0.739656in}}%
\pgfpathlineto{\pgfqpoint{2.679055in}{0.739656in}}%
\pgfpathlineto{\pgfqpoint{2.678759in}{0.739656in}}%
\pgfpathlineto{\pgfqpoint{2.678463in}{0.739656in}}%
\pgfpathlineto{\pgfqpoint{2.678167in}{0.739656in}}%
\pgfpathlineto{\pgfqpoint{2.677871in}{0.739656in}}%
\pgfpathlineto{\pgfqpoint{2.677575in}{0.739656in}}%
\pgfpathlineto{\pgfqpoint{2.677279in}{0.739656in}}%
\pgfpathlineto{\pgfqpoint{2.676983in}{0.739656in}}%
\pgfpathlineto{\pgfqpoint{2.676687in}{0.739656in}}%
\pgfpathlineto{\pgfqpoint{2.676391in}{0.739656in}}%
\pgfpathlineto{\pgfqpoint{2.676095in}{0.739656in}}%
\pgfpathlineto{\pgfqpoint{2.675799in}{0.739656in}}%
\pgfpathlineto{\pgfqpoint{2.675503in}{0.739656in}}%
\pgfpathlineto{\pgfqpoint{2.675207in}{0.739656in}}%
\pgfpathlineto{\pgfqpoint{2.674911in}{0.739656in}}%
\pgfpathlineto{\pgfqpoint{2.674615in}{0.739656in}}%
\pgfpathlineto{\pgfqpoint{2.674319in}{0.739656in}}%
\pgfpathlineto{\pgfqpoint{2.674023in}{0.739656in}}%
\pgfpathlineto{\pgfqpoint{2.673727in}{0.739656in}}%
\pgfpathlineto{\pgfqpoint{2.673431in}{0.739656in}}%
\pgfpathlineto{\pgfqpoint{2.673135in}{0.739656in}}%
\pgfpathlineto{\pgfqpoint{2.672839in}{0.739656in}}%
\pgfpathlineto{\pgfqpoint{2.672543in}{0.739656in}}%
\pgfpathlineto{\pgfqpoint{2.672247in}{0.739656in}}%
\pgfpathlineto{\pgfqpoint{2.671951in}{0.739656in}}%
\pgfpathlineto{\pgfqpoint{2.671655in}{0.739656in}}%
\pgfpathlineto{\pgfqpoint{2.671359in}{0.739656in}}%
\pgfpathlineto{\pgfqpoint{2.671063in}{0.739656in}}%
\pgfpathlineto{\pgfqpoint{2.670767in}{0.739656in}}%
\pgfpathlineto{\pgfqpoint{2.670471in}{0.739656in}}%
\pgfpathlineto{\pgfqpoint{2.670174in}{0.739656in}}%
\pgfpathlineto{\pgfqpoint{2.669878in}{0.739656in}}%
\pgfpathlineto{\pgfqpoint{2.669582in}{0.739656in}}%
\pgfpathlineto{\pgfqpoint{2.669286in}{0.739656in}}%
\pgfpathlineto{\pgfqpoint{2.668990in}{0.739656in}}%
\pgfpathlineto{\pgfqpoint{2.668694in}{0.739656in}}%
\pgfpathlineto{\pgfqpoint{2.668398in}{0.739656in}}%
\pgfpathlineto{\pgfqpoint{2.668102in}{0.739656in}}%
\pgfpathlineto{\pgfqpoint{2.667806in}{0.739656in}}%
\pgfpathlineto{\pgfqpoint{2.667510in}{0.739656in}}%
\pgfpathlineto{\pgfqpoint{2.667214in}{0.739656in}}%
\pgfpathlineto{\pgfqpoint{2.666918in}{0.739656in}}%
\pgfpathlineto{\pgfqpoint{2.666622in}{0.739656in}}%
\pgfpathlineto{\pgfqpoint{2.666326in}{0.739656in}}%
\pgfpathlineto{\pgfqpoint{2.666030in}{0.739656in}}%
\pgfpathlineto{\pgfqpoint{2.665734in}{0.739656in}}%
\pgfpathlineto{\pgfqpoint{2.665438in}{0.739656in}}%
\pgfpathlineto{\pgfqpoint{2.665142in}{0.739656in}}%
\pgfpathlineto{\pgfqpoint{2.664846in}{0.739656in}}%
\pgfpathlineto{\pgfqpoint{2.664550in}{0.739656in}}%
\pgfpathlineto{\pgfqpoint{2.664254in}{0.739656in}}%
\pgfpathlineto{\pgfqpoint{2.663958in}{0.739656in}}%
\pgfpathlineto{\pgfqpoint{2.663662in}{0.739656in}}%
\pgfpathlineto{\pgfqpoint{2.663366in}{0.739656in}}%
\pgfpathlineto{\pgfqpoint{2.663070in}{0.739656in}}%
\pgfpathlineto{\pgfqpoint{2.662774in}{0.739656in}}%
\pgfpathlineto{\pgfqpoint{2.662478in}{0.739656in}}%
\pgfpathlineto{\pgfqpoint{2.662182in}{0.739656in}}%
\pgfpathlineto{\pgfqpoint{2.661886in}{0.739656in}}%
\pgfpathlineto{\pgfqpoint{2.661590in}{0.739656in}}%
\pgfpathlineto{\pgfqpoint{2.661294in}{0.739656in}}%
\pgfpathlineto{\pgfqpoint{2.660998in}{0.739656in}}%
\pgfpathlineto{\pgfqpoint{2.660702in}{0.739656in}}%
\pgfpathlineto{\pgfqpoint{2.660406in}{0.739656in}}%
\pgfpathlineto{\pgfqpoint{2.660110in}{0.739656in}}%
\pgfpathlineto{\pgfqpoint{2.659814in}{0.739656in}}%
\pgfpathlineto{\pgfqpoint{2.659518in}{0.739656in}}%
\pgfpathlineto{\pgfqpoint{2.659222in}{0.739656in}}%
\pgfpathlineto{\pgfqpoint{2.658926in}{0.739656in}}%
\pgfpathlineto{\pgfqpoint{2.658630in}{0.739656in}}%
\pgfpathlineto{\pgfqpoint{2.658334in}{0.739656in}}%
\pgfpathlineto{\pgfqpoint{2.658038in}{0.739656in}}%
\pgfpathlineto{\pgfqpoint{2.657742in}{0.739656in}}%
\pgfpathlineto{\pgfqpoint{2.657446in}{0.739656in}}%
\pgfpathlineto{\pgfqpoint{2.657150in}{0.739656in}}%
\pgfpathlineto{\pgfqpoint{2.656854in}{0.739656in}}%
\pgfpathlineto{\pgfqpoint{2.656558in}{0.739656in}}%
\pgfpathlineto{\pgfqpoint{2.656262in}{0.739656in}}%
\pgfpathlineto{\pgfqpoint{2.655966in}{0.739656in}}%
\pgfpathlineto{\pgfqpoint{2.655670in}{0.739656in}}%
\pgfpathlineto{\pgfqpoint{2.655374in}{0.739656in}}%
\pgfpathlineto{\pgfqpoint{2.655078in}{0.739656in}}%
\pgfpathlineto{\pgfqpoint{2.654782in}{0.739656in}}%
\pgfpathlineto{\pgfqpoint{2.654486in}{0.739656in}}%
\pgfpathlineto{\pgfqpoint{2.654190in}{0.739656in}}%
\pgfpathlineto{\pgfqpoint{2.653894in}{0.739656in}}%
\pgfpathlineto{\pgfqpoint{2.653598in}{0.739656in}}%
\pgfpathlineto{\pgfqpoint{2.653302in}{0.739656in}}%
\pgfpathlineto{\pgfqpoint{2.653006in}{0.739656in}}%
\pgfpathlineto{\pgfqpoint{2.652710in}{0.739656in}}%
\pgfpathlineto{\pgfqpoint{2.652414in}{0.739656in}}%
\pgfpathlineto{\pgfqpoint{2.652118in}{0.739656in}}%
\pgfpathlineto{\pgfqpoint{2.651822in}{0.739656in}}%
\pgfpathlineto{\pgfqpoint{2.651526in}{0.739656in}}%
\pgfpathlineto{\pgfqpoint{2.651230in}{0.739656in}}%
\pgfpathlineto{\pgfqpoint{2.650934in}{0.739656in}}%
\pgfpathlineto{\pgfqpoint{2.650638in}{0.739656in}}%
\pgfpathlineto{\pgfqpoint{2.650342in}{0.739656in}}%
\pgfpathlineto{\pgfqpoint{2.650046in}{0.739656in}}%
\pgfpathlineto{\pgfqpoint{2.649750in}{0.739656in}}%
\pgfpathlineto{\pgfqpoint{2.649454in}{0.739656in}}%
\pgfpathlineto{\pgfqpoint{2.649158in}{0.739656in}}%
\pgfpathlineto{\pgfqpoint{2.648862in}{0.739656in}}%
\pgfpathlineto{\pgfqpoint{2.648566in}{0.739656in}}%
\pgfpathlineto{\pgfqpoint{2.648270in}{0.739656in}}%
\pgfpathlineto{\pgfqpoint{2.647974in}{0.739656in}}%
\pgfpathlineto{\pgfqpoint{2.647678in}{0.739656in}}%
\pgfpathlineto{\pgfqpoint{2.647382in}{0.739656in}}%
\pgfpathlineto{\pgfqpoint{2.647086in}{0.739656in}}%
\pgfpathlineto{\pgfqpoint{2.646790in}{0.739656in}}%
\pgfpathlineto{\pgfqpoint{2.646494in}{0.739656in}}%
\pgfpathlineto{\pgfqpoint{2.646198in}{0.739656in}}%
\pgfpathlineto{\pgfqpoint{2.645902in}{0.739656in}}%
\pgfpathlineto{\pgfqpoint{2.645606in}{0.739656in}}%
\pgfpathlineto{\pgfqpoint{2.645310in}{0.739656in}}%
\pgfpathlineto{\pgfqpoint{2.645014in}{0.739656in}}%
\pgfpathlineto{\pgfqpoint{2.644718in}{0.739656in}}%
\pgfpathlineto{\pgfqpoint{2.644422in}{0.739656in}}%
\pgfpathlineto{\pgfqpoint{2.644126in}{0.739656in}}%
\pgfpathlineto{\pgfqpoint{2.643830in}{0.739656in}}%
\pgfpathlineto{\pgfqpoint{2.643534in}{0.739656in}}%
\pgfpathlineto{\pgfqpoint{2.643238in}{0.739656in}}%
\pgfpathlineto{\pgfqpoint{2.642942in}{0.739656in}}%
\pgfpathlineto{\pgfqpoint{2.642646in}{0.739656in}}%
\pgfpathlineto{\pgfqpoint{2.642350in}{0.739656in}}%
\pgfpathlineto{\pgfqpoint{2.642054in}{0.739656in}}%
\pgfpathlineto{\pgfqpoint{2.641758in}{0.739656in}}%
\pgfpathlineto{\pgfqpoint{2.641462in}{0.739656in}}%
\pgfpathlineto{\pgfqpoint{2.641166in}{0.739656in}}%
\pgfpathlineto{\pgfqpoint{2.640870in}{0.739656in}}%
\pgfpathlineto{\pgfqpoint{2.640574in}{0.739656in}}%
\pgfpathlineto{\pgfqpoint{2.640278in}{0.739656in}}%
\pgfpathlineto{\pgfqpoint{2.639982in}{0.739656in}}%
\pgfpathlineto{\pgfqpoint{2.639686in}{0.739656in}}%
\pgfpathlineto{\pgfqpoint{2.639390in}{0.739656in}}%
\pgfpathlineto{\pgfqpoint{2.639094in}{0.739656in}}%
\pgfpathlineto{\pgfqpoint{2.638798in}{0.739656in}}%
\pgfpathlineto{\pgfqpoint{2.638502in}{0.739656in}}%
\pgfpathlineto{\pgfqpoint{2.638206in}{0.739656in}}%
\pgfpathlineto{\pgfqpoint{2.637910in}{0.739656in}}%
\pgfpathlineto{\pgfqpoint{2.637614in}{0.739656in}}%
\pgfpathlineto{\pgfqpoint{2.637318in}{0.739656in}}%
\pgfpathlineto{\pgfqpoint{2.637022in}{0.739656in}}%
\pgfpathlineto{\pgfqpoint{2.636726in}{0.739656in}}%
\pgfpathlineto{\pgfqpoint{2.636430in}{0.739656in}}%
\pgfpathlineto{\pgfqpoint{2.636134in}{0.739656in}}%
\pgfpathlineto{\pgfqpoint{2.635838in}{0.739656in}}%
\pgfpathlineto{\pgfqpoint{2.635542in}{0.739656in}}%
\pgfpathlineto{\pgfqpoint{2.635246in}{0.739656in}}%
\pgfpathlineto{\pgfqpoint{2.634950in}{0.739656in}}%
\pgfpathlineto{\pgfqpoint{2.634654in}{0.739656in}}%
\pgfpathlineto{\pgfqpoint{2.634358in}{0.739656in}}%
\pgfpathlineto{\pgfqpoint{2.634062in}{0.739656in}}%
\pgfpathlineto{\pgfqpoint{2.633766in}{0.739656in}}%
\pgfpathlineto{\pgfqpoint{2.633470in}{0.739656in}}%
\pgfpathlineto{\pgfqpoint{2.633174in}{0.739656in}}%
\pgfpathlineto{\pgfqpoint{2.632878in}{0.739656in}}%
\pgfpathlineto{\pgfqpoint{2.632582in}{0.739656in}}%
\pgfpathlineto{\pgfqpoint{2.632286in}{0.739656in}}%
\pgfpathlineto{\pgfqpoint{2.631990in}{0.739656in}}%
\pgfpathlineto{\pgfqpoint{2.631694in}{0.739656in}}%
\pgfpathlineto{\pgfqpoint{2.631398in}{0.739656in}}%
\pgfpathlineto{\pgfqpoint{2.631102in}{0.739656in}}%
\pgfpathlineto{\pgfqpoint{2.630806in}{0.739656in}}%
\pgfpathlineto{\pgfqpoint{2.630510in}{0.739656in}}%
\pgfpathlineto{\pgfqpoint{2.630214in}{0.739656in}}%
\pgfpathlineto{\pgfqpoint{2.629918in}{0.739656in}}%
\pgfpathlineto{\pgfqpoint{2.629622in}{0.739656in}}%
\pgfpathlineto{\pgfqpoint{2.629326in}{0.739656in}}%
\pgfpathlineto{\pgfqpoint{2.629030in}{0.739656in}}%
\pgfpathlineto{\pgfqpoint{2.628734in}{0.739656in}}%
\pgfpathlineto{\pgfqpoint{2.628438in}{0.739656in}}%
\pgfpathlineto{\pgfqpoint{2.628142in}{0.739656in}}%
\pgfpathlineto{\pgfqpoint{2.627846in}{0.739656in}}%
\pgfpathlineto{\pgfqpoint{2.627550in}{0.739656in}}%
\pgfpathlineto{\pgfqpoint{2.627254in}{0.739656in}}%
\pgfpathlineto{\pgfqpoint{2.626958in}{0.739656in}}%
\pgfpathlineto{\pgfqpoint{2.626662in}{0.739656in}}%
\pgfpathlineto{\pgfqpoint{2.626366in}{0.739656in}}%
\pgfpathlineto{\pgfqpoint{2.626070in}{0.739656in}}%
\pgfpathlineto{\pgfqpoint{2.625774in}{0.739656in}}%
\pgfpathlineto{\pgfqpoint{2.625478in}{0.739656in}}%
\pgfpathlineto{\pgfqpoint{2.625182in}{0.739656in}}%
\pgfpathlineto{\pgfqpoint{2.624886in}{0.739656in}}%
\pgfpathlineto{\pgfqpoint{2.624590in}{0.739656in}}%
\pgfpathlineto{\pgfqpoint{2.624294in}{0.739656in}}%
\pgfpathlineto{\pgfqpoint{2.623998in}{0.739656in}}%
\pgfpathlineto{\pgfqpoint{2.623702in}{0.739656in}}%
\pgfpathlineto{\pgfqpoint{2.623406in}{0.739656in}}%
\pgfpathlineto{\pgfqpoint{2.623110in}{0.739656in}}%
\pgfpathlineto{\pgfqpoint{2.622814in}{0.739656in}}%
\pgfpathlineto{\pgfqpoint{2.622518in}{0.739656in}}%
\pgfpathlineto{\pgfqpoint{2.622222in}{0.739656in}}%
\pgfpathlineto{\pgfqpoint{2.621926in}{0.739656in}}%
\pgfpathlineto{\pgfqpoint{2.621630in}{0.739656in}}%
\pgfpathlineto{\pgfqpoint{2.621334in}{0.739656in}}%
\pgfpathlineto{\pgfqpoint{2.621038in}{0.739656in}}%
\pgfpathlineto{\pgfqpoint{2.620742in}{0.739656in}}%
\pgfpathlineto{\pgfqpoint{2.620446in}{0.739656in}}%
\pgfpathlineto{\pgfqpoint{2.620150in}{0.739656in}}%
\pgfpathlineto{\pgfqpoint{2.619854in}{0.739656in}}%
\pgfpathlineto{\pgfqpoint{2.619558in}{0.739656in}}%
\pgfpathlineto{\pgfqpoint{2.619262in}{0.739656in}}%
\pgfpathlineto{\pgfqpoint{2.618966in}{0.739656in}}%
\pgfpathlineto{\pgfqpoint{2.618670in}{0.739656in}}%
\pgfpathlineto{\pgfqpoint{2.618374in}{0.739656in}}%
\pgfpathlineto{\pgfqpoint{2.618078in}{0.739656in}}%
\pgfpathlineto{\pgfqpoint{2.617782in}{0.739656in}}%
\pgfpathlineto{\pgfqpoint{2.617486in}{0.739656in}}%
\pgfpathlineto{\pgfqpoint{2.617190in}{0.739656in}}%
\pgfpathlineto{\pgfqpoint{2.616894in}{0.739656in}}%
\pgfpathlineto{\pgfqpoint{2.616598in}{0.739656in}}%
\pgfpathlineto{\pgfqpoint{2.616302in}{0.739656in}}%
\pgfpathlineto{\pgfqpoint{2.616006in}{0.739656in}}%
\pgfpathlineto{\pgfqpoint{2.615710in}{0.739656in}}%
\pgfpathlineto{\pgfqpoint{2.615414in}{0.739656in}}%
\pgfpathlineto{\pgfqpoint{2.615118in}{0.739656in}}%
\pgfpathlineto{\pgfqpoint{2.614822in}{0.739656in}}%
\pgfpathlineto{\pgfqpoint{2.614526in}{0.739656in}}%
\pgfpathlineto{\pgfqpoint{2.614230in}{0.739656in}}%
\pgfpathlineto{\pgfqpoint{2.613934in}{0.739656in}}%
\pgfpathlineto{\pgfqpoint{2.613638in}{0.739656in}}%
\pgfpathlineto{\pgfqpoint{2.613342in}{0.739656in}}%
\pgfpathlineto{\pgfqpoint{2.613046in}{0.739656in}}%
\pgfpathlineto{\pgfqpoint{2.612750in}{0.739656in}}%
\pgfpathlineto{\pgfqpoint{2.612454in}{0.739656in}}%
\pgfpathlineto{\pgfqpoint{2.612158in}{0.739656in}}%
\pgfpathlineto{\pgfqpoint{2.611862in}{0.739656in}}%
\pgfpathlineto{\pgfqpoint{2.611566in}{0.739656in}}%
\pgfpathlineto{\pgfqpoint{2.611270in}{0.739656in}}%
\pgfpathlineto{\pgfqpoint{2.610974in}{0.739656in}}%
\pgfpathlineto{\pgfqpoint{2.610678in}{0.739656in}}%
\pgfpathlineto{\pgfqpoint{2.610382in}{0.739656in}}%
\pgfpathlineto{\pgfqpoint{2.610086in}{0.739656in}}%
\pgfpathlineto{\pgfqpoint{2.609790in}{0.739656in}}%
\pgfpathlineto{\pgfqpoint{2.609494in}{0.739656in}}%
\pgfpathlineto{\pgfqpoint{2.609198in}{0.739656in}}%
\pgfpathlineto{\pgfqpoint{2.608902in}{0.739656in}}%
\pgfpathlineto{\pgfqpoint{2.608606in}{0.739656in}}%
\pgfpathlineto{\pgfqpoint{2.608310in}{0.739656in}}%
\pgfpathlineto{\pgfqpoint{2.608014in}{0.739656in}}%
\pgfpathlineto{\pgfqpoint{2.607718in}{0.739656in}}%
\pgfpathlineto{\pgfqpoint{2.607422in}{0.739656in}}%
\pgfpathlineto{\pgfqpoint{2.607126in}{0.739656in}}%
\pgfpathlineto{\pgfqpoint{2.606830in}{0.739656in}}%
\pgfpathlineto{\pgfqpoint{2.606534in}{0.739656in}}%
\pgfpathlineto{\pgfqpoint{2.606238in}{0.739656in}}%
\pgfpathlineto{\pgfqpoint{2.605942in}{0.739656in}}%
\pgfpathlineto{\pgfqpoint{2.605646in}{0.739656in}}%
\pgfpathlineto{\pgfqpoint{2.605350in}{0.739656in}}%
\pgfpathlineto{\pgfqpoint{2.605054in}{0.739656in}}%
\pgfpathlineto{\pgfqpoint{2.604758in}{0.739656in}}%
\pgfpathlineto{\pgfqpoint{2.604462in}{0.739656in}}%
\pgfpathlineto{\pgfqpoint{2.604166in}{0.739656in}}%
\pgfpathlineto{\pgfqpoint{2.603870in}{0.739656in}}%
\pgfpathlineto{\pgfqpoint{2.603574in}{0.739656in}}%
\pgfpathlineto{\pgfqpoint{2.603278in}{0.739656in}}%
\pgfpathlineto{\pgfqpoint{2.602981in}{0.739656in}}%
\pgfpathlineto{\pgfqpoint{2.602685in}{0.739656in}}%
\pgfpathlineto{\pgfqpoint{2.602389in}{0.739656in}}%
\pgfpathlineto{\pgfqpoint{2.602093in}{0.739656in}}%
\pgfpathlineto{\pgfqpoint{2.601797in}{0.739656in}}%
\pgfpathlineto{\pgfqpoint{2.601501in}{0.739656in}}%
\pgfpathlineto{\pgfqpoint{2.601205in}{0.739656in}}%
\pgfpathlineto{\pgfqpoint{2.600909in}{0.739656in}}%
\pgfpathlineto{\pgfqpoint{2.600613in}{0.739656in}}%
\pgfpathlineto{\pgfqpoint{2.600317in}{0.739656in}}%
\pgfpathlineto{\pgfqpoint{2.600021in}{0.739656in}}%
\pgfpathlineto{\pgfqpoint{2.599725in}{0.739656in}}%
\pgfpathlineto{\pgfqpoint{2.599429in}{0.739656in}}%
\pgfpathlineto{\pgfqpoint{2.599133in}{0.739656in}}%
\pgfpathlineto{\pgfqpoint{2.598837in}{0.739656in}}%
\pgfpathlineto{\pgfqpoint{2.598541in}{0.739656in}}%
\pgfpathlineto{\pgfqpoint{2.598245in}{0.739656in}}%
\pgfpathlineto{\pgfqpoint{2.597949in}{0.739656in}}%
\pgfpathlineto{\pgfqpoint{2.597653in}{0.739656in}}%
\pgfpathlineto{\pgfqpoint{2.597357in}{0.739656in}}%
\pgfpathlineto{\pgfqpoint{2.597061in}{0.739656in}}%
\pgfpathlineto{\pgfqpoint{2.596765in}{0.739656in}}%
\pgfpathlineto{\pgfqpoint{2.596469in}{0.739656in}}%
\pgfpathlineto{\pgfqpoint{2.596173in}{0.739656in}}%
\pgfpathlineto{\pgfqpoint{2.595877in}{0.739656in}}%
\pgfpathlineto{\pgfqpoint{2.595581in}{0.739656in}}%
\pgfpathlineto{\pgfqpoint{2.595285in}{0.739656in}}%
\pgfpathlineto{\pgfqpoint{2.594989in}{0.739656in}}%
\pgfpathlineto{\pgfqpoint{2.594693in}{0.739656in}}%
\pgfpathlineto{\pgfqpoint{2.594397in}{0.739656in}}%
\pgfpathlineto{\pgfqpoint{2.594101in}{0.739656in}}%
\pgfpathlineto{\pgfqpoint{2.593805in}{0.739656in}}%
\pgfpathlineto{\pgfqpoint{2.593509in}{0.739656in}}%
\pgfpathlineto{\pgfqpoint{2.593213in}{0.739656in}}%
\pgfpathlineto{\pgfqpoint{2.592917in}{0.739656in}}%
\pgfpathlineto{\pgfqpoint{2.592621in}{0.739656in}}%
\pgfpathlineto{\pgfqpoint{2.592325in}{0.739656in}}%
\pgfpathlineto{\pgfqpoint{2.592029in}{0.739656in}}%
\pgfpathlineto{\pgfqpoint{2.591733in}{0.739656in}}%
\pgfpathlineto{\pgfqpoint{2.591437in}{0.739656in}}%
\pgfpathlineto{\pgfqpoint{2.591141in}{0.739656in}}%
\pgfpathlineto{\pgfqpoint{2.590845in}{0.739656in}}%
\pgfpathlineto{\pgfqpoint{2.590549in}{0.739656in}}%
\pgfpathlineto{\pgfqpoint{2.590253in}{0.739656in}}%
\pgfpathlineto{\pgfqpoint{2.589957in}{0.739656in}}%
\pgfpathlineto{\pgfqpoint{2.589661in}{0.739656in}}%
\pgfpathlineto{\pgfqpoint{2.589365in}{0.739656in}}%
\pgfpathlineto{\pgfqpoint{2.589069in}{0.739656in}}%
\pgfpathlineto{\pgfqpoint{2.588773in}{0.739656in}}%
\pgfpathlineto{\pgfqpoint{2.588477in}{0.739656in}}%
\pgfpathlineto{\pgfqpoint{2.588181in}{0.739656in}}%
\pgfpathlineto{\pgfqpoint{2.587885in}{0.739656in}}%
\pgfpathlineto{\pgfqpoint{2.587589in}{0.739656in}}%
\pgfpathlineto{\pgfqpoint{2.587293in}{0.739656in}}%
\pgfpathlineto{\pgfqpoint{2.586997in}{0.739656in}}%
\pgfpathlineto{\pgfqpoint{2.586701in}{0.739656in}}%
\pgfpathlineto{\pgfqpoint{2.586405in}{0.739656in}}%
\pgfpathlineto{\pgfqpoint{2.586109in}{0.739656in}}%
\pgfpathlineto{\pgfqpoint{2.585813in}{0.739656in}}%
\pgfpathlineto{\pgfqpoint{2.585517in}{0.739656in}}%
\pgfpathlineto{\pgfqpoint{2.585221in}{0.739656in}}%
\pgfpathlineto{\pgfqpoint{2.584925in}{0.739656in}}%
\pgfpathlineto{\pgfqpoint{2.584629in}{0.739656in}}%
\pgfpathlineto{\pgfqpoint{2.584333in}{0.739656in}}%
\pgfpathlineto{\pgfqpoint{2.584037in}{0.739656in}}%
\pgfpathlineto{\pgfqpoint{2.583741in}{0.739656in}}%
\pgfpathlineto{\pgfqpoint{2.583445in}{0.739656in}}%
\pgfpathlineto{\pgfqpoint{2.583149in}{0.739656in}}%
\pgfpathlineto{\pgfqpoint{2.582853in}{0.739656in}}%
\pgfpathlineto{\pgfqpoint{2.582557in}{0.739656in}}%
\pgfpathlineto{\pgfqpoint{2.582261in}{0.739656in}}%
\pgfpathlineto{\pgfqpoint{2.581965in}{0.739656in}}%
\pgfpathlineto{\pgfqpoint{2.581669in}{0.739656in}}%
\pgfpathlineto{\pgfqpoint{2.581373in}{0.739656in}}%
\pgfpathlineto{\pgfqpoint{2.581077in}{0.739656in}}%
\pgfpathlineto{\pgfqpoint{2.580781in}{0.739656in}}%
\pgfpathlineto{\pgfqpoint{2.580485in}{0.739656in}}%
\pgfpathlineto{\pgfqpoint{2.580189in}{0.739656in}}%
\pgfpathlineto{\pgfqpoint{2.579893in}{0.739656in}}%
\pgfpathlineto{\pgfqpoint{2.579597in}{0.739656in}}%
\pgfpathlineto{\pgfqpoint{2.579301in}{0.739656in}}%
\pgfpathlineto{\pgfqpoint{2.579005in}{0.739656in}}%
\pgfpathlineto{\pgfqpoint{2.578709in}{0.739656in}}%
\pgfpathlineto{\pgfqpoint{2.578413in}{0.739656in}}%
\pgfpathlineto{\pgfqpoint{2.578117in}{0.739656in}}%
\pgfpathlineto{\pgfqpoint{2.577821in}{0.739656in}}%
\pgfpathlineto{\pgfqpoint{2.577525in}{0.739656in}}%
\pgfpathlineto{\pgfqpoint{2.577229in}{0.739656in}}%
\pgfpathlineto{\pgfqpoint{2.576933in}{0.739656in}}%
\pgfpathlineto{\pgfqpoint{2.576637in}{0.739656in}}%
\pgfpathlineto{\pgfqpoint{2.576341in}{0.739656in}}%
\pgfpathlineto{\pgfqpoint{2.576045in}{0.739656in}}%
\pgfpathlineto{\pgfqpoint{2.575749in}{0.739656in}}%
\pgfpathlineto{\pgfqpoint{2.575453in}{0.739656in}}%
\pgfpathlineto{\pgfqpoint{2.575157in}{0.739656in}}%
\pgfpathlineto{\pgfqpoint{2.574861in}{0.739656in}}%
\pgfpathlineto{\pgfqpoint{2.574565in}{0.739656in}}%
\pgfpathlineto{\pgfqpoint{2.574269in}{0.739656in}}%
\pgfpathlineto{\pgfqpoint{2.573973in}{0.739656in}}%
\pgfpathlineto{\pgfqpoint{2.573677in}{0.739656in}}%
\pgfpathlineto{\pgfqpoint{2.573381in}{0.739656in}}%
\pgfpathlineto{\pgfqpoint{2.573085in}{0.739656in}}%
\pgfpathlineto{\pgfqpoint{2.572789in}{0.739656in}}%
\pgfpathlineto{\pgfqpoint{2.572493in}{0.739656in}}%
\pgfpathlineto{\pgfqpoint{2.572197in}{0.739656in}}%
\pgfpathlineto{\pgfqpoint{2.571901in}{0.739656in}}%
\pgfpathlineto{\pgfqpoint{2.571605in}{0.739656in}}%
\pgfpathlineto{\pgfqpoint{2.571309in}{0.739656in}}%
\pgfpathlineto{\pgfqpoint{2.571013in}{0.739656in}}%
\pgfpathlineto{\pgfqpoint{2.570717in}{0.739656in}}%
\pgfpathlineto{\pgfqpoint{2.570421in}{0.739656in}}%
\pgfpathlineto{\pgfqpoint{2.570125in}{0.739656in}}%
\pgfpathlineto{\pgfqpoint{2.569829in}{0.739656in}}%
\pgfpathlineto{\pgfqpoint{2.569533in}{0.739656in}}%
\pgfpathlineto{\pgfqpoint{2.569237in}{0.739656in}}%
\pgfpathlineto{\pgfqpoint{2.568941in}{0.739656in}}%
\pgfpathlineto{\pgfqpoint{2.568645in}{0.739656in}}%
\pgfpathlineto{\pgfqpoint{2.568349in}{0.739656in}}%
\pgfpathlineto{\pgfqpoint{2.568053in}{0.739656in}}%
\pgfpathlineto{\pgfqpoint{2.567757in}{0.739656in}}%
\pgfpathlineto{\pgfqpoint{2.567461in}{0.739656in}}%
\pgfpathlineto{\pgfqpoint{2.567165in}{0.739656in}}%
\pgfpathlineto{\pgfqpoint{2.566869in}{0.739656in}}%
\pgfpathlineto{\pgfqpoint{2.566573in}{0.739656in}}%
\pgfpathlineto{\pgfqpoint{2.566277in}{0.739656in}}%
\pgfpathlineto{\pgfqpoint{2.565981in}{0.739656in}}%
\pgfpathlineto{\pgfqpoint{2.565685in}{0.739656in}}%
\pgfpathlineto{\pgfqpoint{2.565389in}{0.739656in}}%
\pgfpathlineto{\pgfqpoint{2.565093in}{0.739656in}}%
\pgfpathlineto{\pgfqpoint{2.564797in}{0.739656in}}%
\pgfpathlineto{\pgfqpoint{2.564501in}{0.739656in}}%
\pgfpathlineto{\pgfqpoint{2.564205in}{0.739656in}}%
\pgfpathlineto{\pgfqpoint{2.563909in}{0.739656in}}%
\pgfpathlineto{\pgfqpoint{2.563613in}{0.739656in}}%
\pgfpathlineto{\pgfqpoint{2.563317in}{0.739656in}}%
\pgfpathlineto{\pgfqpoint{2.563021in}{0.739656in}}%
\pgfpathlineto{\pgfqpoint{2.562725in}{0.739656in}}%
\pgfpathlineto{\pgfqpoint{2.562429in}{0.739656in}}%
\pgfpathlineto{\pgfqpoint{2.562133in}{0.739656in}}%
\pgfpathlineto{\pgfqpoint{2.561837in}{0.739656in}}%
\pgfpathlineto{\pgfqpoint{2.561541in}{0.739656in}}%
\pgfpathlineto{\pgfqpoint{2.561245in}{0.739656in}}%
\pgfpathlineto{\pgfqpoint{2.560949in}{0.739656in}}%
\pgfpathlineto{\pgfqpoint{2.560653in}{0.739656in}}%
\pgfpathlineto{\pgfqpoint{2.560357in}{0.739656in}}%
\pgfpathlineto{\pgfqpoint{2.560061in}{0.739656in}}%
\pgfpathlineto{\pgfqpoint{2.559765in}{0.739656in}}%
\pgfpathlineto{\pgfqpoint{2.559469in}{0.739656in}}%
\pgfpathlineto{\pgfqpoint{2.559173in}{0.739656in}}%
\pgfpathlineto{\pgfqpoint{2.558877in}{0.739656in}}%
\pgfpathlineto{\pgfqpoint{2.558581in}{0.739656in}}%
\pgfpathlineto{\pgfqpoint{2.558285in}{0.739656in}}%
\pgfpathlineto{\pgfqpoint{2.557989in}{0.739656in}}%
\pgfpathlineto{\pgfqpoint{2.557693in}{0.739656in}}%
\pgfpathlineto{\pgfqpoint{2.557397in}{0.739656in}}%
\pgfpathlineto{\pgfqpoint{2.557101in}{0.739656in}}%
\pgfpathlineto{\pgfqpoint{2.556805in}{0.739656in}}%
\pgfpathlineto{\pgfqpoint{2.556509in}{0.739656in}}%
\pgfpathlineto{\pgfqpoint{2.556213in}{0.739656in}}%
\pgfpathlineto{\pgfqpoint{2.555917in}{0.739656in}}%
\pgfpathlineto{\pgfqpoint{2.555621in}{0.739656in}}%
\pgfpathlineto{\pgfqpoint{2.555325in}{0.739656in}}%
\pgfpathlineto{\pgfqpoint{2.555029in}{0.739656in}}%
\pgfpathlineto{\pgfqpoint{2.554733in}{0.739656in}}%
\pgfpathlineto{\pgfqpoint{2.554437in}{0.739656in}}%
\pgfpathlineto{\pgfqpoint{2.554141in}{0.739656in}}%
\pgfpathlineto{\pgfqpoint{2.553845in}{0.739656in}}%
\pgfpathlineto{\pgfqpoint{2.553549in}{0.739656in}}%
\pgfpathlineto{\pgfqpoint{2.553253in}{0.739656in}}%
\pgfpathlineto{\pgfqpoint{2.552957in}{0.739656in}}%
\pgfpathlineto{\pgfqpoint{2.552661in}{0.739656in}}%
\pgfpathlineto{\pgfqpoint{2.552365in}{0.739656in}}%
\pgfpathlineto{\pgfqpoint{2.552069in}{0.739656in}}%
\pgfpathlineto{\pgfqpoint{2.551773in}{0.739656in}}%
\pgfpathlineto{\pgfqpoint{2.551477in}{0.739656in}}%
\pgfpathlineto{\pgfqpoint{2.551181in}{0.739656in}}%
\pgfpathlineto{\pgfqpoint{2.550885in}{0.739656in}}%
\pgfpathlineto{\pgfqpoint{2.550589in}{0.739656in}}%
\pgfpathlineto{\pgfqpoint{2.550293in}{0.739656in}}%
\pgfpathlineto{\pgfqpoint{2.549997in}{0.739656in}}%
\pgfpathlineto{\pgfqpoint{2.549701in}{0.739656in}}%
\pgfpathlineto{\pgfqpoint{2.549405in}{0.739656in}}%
\pgfpathlineto{\pgfqpoint{2.549109in}{0.739656in}}%
\pgfpathlineto{\pgfqpoint{2.548813in}{0.739656in}}%
\pgfpathlineto{\pgfqpoint{2.548517in}{0.739656in}}%
\pgfpathlineto{\pgfqpoint{2.548221in}{0.739656in}}%
\pgfpathlineto{\pgfqpoint{2.547925in}{0.739656in}}%
\pgfpathlineto{\pgfqpoint{2.547629in}{0.739656in}}%
\pgfpathlineto{\pgfqpoint{2.547333in}{0.739656in}}%
\pgfpathlineto{\pgfqpoint{2.547037in}{0.739656in}}%
\pgfpathlineto{\pgfqpoint{2.546741in}{0.739656in}}%
\pgfpathlineto{\pgfqpoint{2.546445in}{0.739656in}}%
\pgfpathlineto{\pgfqpoint{2.546149in}{0.739656in}}%
\pgfpathlineto{\pgfqpoint{2.545853in}{0.739656in}}%
\pgfpathlineto{\pgfqpoint{2.545557in}{0.739656in}}%
\pgfpathlineto{\pgfqpoint{2.545261in}{0.739656in}}%
\pgfpathlineto{\pgfqpoint{2.544965in}{0.739656in}}%
\pgfpathlineto{\pgfqpoint{2.544669in}{0.739656in}}%
\pgfpathlineto{\pgfqpoint{2.544373in}{0.739656in}}%
\pgfpathlineto{\pgfqpoint{2.544077in}{0.739656in}}%
\pgfpathlineto{\pgfqpoint{2.543781in}{0.739656in}}%
\pgfpathlineto{\pgfqpoint{2.543485in}{0.739656in}}%
\pgfpathlineto{\pgfqpoint{2.543189in}{0.739656in}}%
\pgfpathlineto{\pgfqpoint{2.542893in}{0.739656in}}%
\pgfpathlineto{\pgfqpoint{2.542597in}{0.739656in}}%
\pgfpathlineto{\pgfqpoint{2.542301in}{0.739656in}}%
\pgfpathlineto{\pgfqpoint{2.542005in}{0.739656in}}%
\pgfpathlineto{\pgfqpoint{2.541709in}{0.739656in}}%
\pgfpathlineto{\pgfqpoint{2.541413in}{0.739656in}}%
\pgfpathlineto{\pgfqpoint{2.541117in}{0.739656in}}%
\pgfpathlineto{\pgfqpoint{2.540821in}{0.739656in}}%
\pgfpathlineto{\pgfqpoint{2.540525in}{0.739656in}}%
\pgfpathlineto{\pgfqpoint{2.540229in}{0.739656in}}%
\pgfpathlineto{\pgfqpoint{2.539933in}{0.739656in}}%
\pgfpathlineto{\pgfqpoint{2.539637in}{0.739656in}}%
\pgfpathlineto{\pgfqpoint{2.539341in}{0.739656in}}%
\pgfpathlineto{\pgfqpoint{2.539045in}{0.739656in}}%
\pgfpathlineto{\pgfqpoint{2.538749in}{0.739656in}}%
\pgfpathlineto{\pgfqpoint{2.538453in}{0.739656in}}%
\pgfpathlineto{\pgfqpoint{2.538157in}{0.739656in}}%
\pgfpathlineto{\pgfqpoint{2.537861in}{0.739656in}}%
\pgfpathlineto{\pgfqpoint{2.537565in}{0.739656in}}%
\pgfpathlineto{\pgfqpoint{2.537269in}{0.739656in}}%
\pgfpathlineto{\pgfqpoint{2.536973in}{0.739656in}}%
\pgfpathlineto{\pgfqpoint{2.536677in}{0.739656in}}%
\pgfpathlineto{\pgfqpoint{2.536381in}{0.739656in}}%
\pgfpathlineto{\pgfqpoint{2.536085in}{0.739656in}}%
\pgfpathlineto{\pgfqpoint{2.535789in}{0.739656in}}%
\pgfpathlineto{\pgfqpoint{2.535492in}{0.739656in}}%
\pgfpathlineto{\pgfqpoint{2.535196in}{0.739656in}}%
\pgfpathlineto{\pgfqpoint{2.534900in}{0.739656in}}%
\pgfpathlineto{\pgfqpoint{2.534604in}{0.739656in}}%
\pgfpathlineto{\pgfqpoint{2.534308in}{0.739656in}}%
\pgfpathlineto{\pgfqpoint{2.534012in}{0.739656in}}%
\pgfpathlineto{\pgfqpoint{2.533716in}{0.739656in}}%
\pgfpathlineto{\pgfqpoint{2.533420in}{0.739656in}}%
\pgfpathlineto{\pgfqpoint{2.533124in}{0.739656in}}%
\pgfpathlineto{\pgfqpoint{2.532828in}{0.739656in}}%
\pgfpathlineto{\pgfqpoint{2.532532in}{0.739656in}}%
\pgfpathlineto{\pgfqpoint{2.532236in}{0.739656in}}%
\pgfpathlineto{\pgfqpoint{2.531940in}{0.739656in}}%
\pgfpathlineto{\pgfqpoint{2.531644in}{0.739656in}}%
\pgfpathlineto{\pgfqpoint{2.531348in}{0.739656in}}%
\pgfpathlineto{\pgfqpoint{2.531052in}{0.739656in}}%
\pgfpathlineto{\pgfqpoint{2.530756in}{0.739656in}}%
\pgfpathlineto{\pgfqpoint{2.530460in}{0.739656in}}%
\pgfpathlineto{\pgfqpoint{2.530164in}{0.739656in}}%
\pgfpathlineto{\pgfqpoint{2.529868in}{0.739656in}}%
\pgfpathlineto{\pgfqpoint{2.529572in}{0.739656in}}%
\pgfpathlineto{\pgfqpoint{2.529276in}{0.739656in}}%
\pgfpathlineto{\pgfqpoint{2.528980in}{0.739656in}}%
\pgfpathlineto{\pgfqpoint{2.528684in}{0.739656in}}%
\pgfpathlineto{\pgfqpoint{2.528388in}{0.739656in}}%
\pgfpathlineto{\pgfqpoint{2.528092in}{0.739656in}}%
\pgfpathlineto{\pgfqpoint{2.527796in}{0.739656in}}%
\pgfpathlineto{\pgfqpoint{2.527500in}{0.739656in}}%
\pgfpathlineto{\pgfqpoint{2.527204in}{0.739656in}}%
\pgfpathlineto{\pgfqpoint{2.526908in}{0.739656in}}%
\pgfpathlineto{\pgfqpoint{2.526612in}{0.739656in}}%
\pgfpathlineto{\pgfqpoint{2.526316in}{0.739656in}}%
\pgfpathlineto{\pgfqpoint{2.526020in}{0.739656in}}%
\pgfpathlineto{\pgfqpoint{2.525724in}{0.739656in}}%
\pgfpathlineto{\pgfqpoint{2.525428in}{0.739656in}}%
\pgfpathlineto{\pgfqpoint{2.525132in}{0.739656in}}%
\pgfpathlineto{\pgfqpoint{2.524836in}{0.739656in}}%
\pgfpathlineto{\pgfqpoint{2.524540in}{0.739656in}}%
\pgfpathlineto{\pgfqpoint{2.524244in}{0.739656in}}%
\pgfpathlineto{\pgfqpoint{2.523948in}{0.739656in}}%
\pgfpathlineto{\pgfqpoint{2.523652in}{0.739656in}}%
\pgfpathlineto{\pgfqpoint{2.523356in}{0.739656in}}%
\pgfpathlineto{\pgfqpoint{2.523060in}{0.739656in}}%
\pgfpathlineto{\pgfqpoint{2.522764in}{0.739656in}}%
\pgfpathlineto{\pgfqpoint{2.522468in}{0.739656in}}%
\pgfpathlineto{\pgfqpoint{2.522172in}{0.739656in}}%
\pgfpathlineto{\pgfqpoint{2.521876in}{0.739656in}}%
\pgfpathlineto{\pgfqpoint{2.521580in}{0.739656in}}%
\pgfpathlineto{\pgfqpoint{2.521284in}{0.739656in}}%
\pgfpathlineto{\pgfqpoint{2.520988in}{0.739656in}}%
\pgfpathlineto{\pgfqpoint{2.520692in}{0.739656in}}%
\pgfpathlineto{\pgfqpoint{2.520396in}{0.739656in}}%
\pgfpathlineto{\pgfqpoint{2.520100in}{0.739656in}}%
\pgfpathlineto{\pgfqpoint{2.519804in}{0.739656in}}%
\pgfpathlineto{\pgfqpoint{2.519508in}{0.739656in}}%
\pgfpathlineto{\pgfqpoint{2.519212in}{0.739656in}}%
\pgfpathlineto{\pgfqpoint{2.518916in}{0.739656in}}%
\pgfpathlineto{\pgfqpoint{2.518620in}{0.739656in}}%
\pgfpathlineto{\pgfqpoint{2.518324in}{0.739656in}}%
\pgfpathlineto{\pgfqpoint{2.518028in}{0.739656in}}%
\pgfpathlineto{\pgfqpoint{2.517732in}{0.739656in}}%
\pgfpathlineto{\pgfqpoint{2.517436in}{0.739656in}}%
\pgfpathlineto{\pgfqpoint{2.517140in}{0.739656in}}%
\pgfpathlineto{\pgfqpoint{2.516844in}{0.739656in}}%
\pgfpathlineto{\pgfqpoint{2.516548in}{0.739656in}}%
\pgfpathlineto{\pgfqpoint{2.516252in}{0.739656in}}%
\pgfpathlineto{\pgfqpoint{2.515956in}{0.739656in}}%
\pgfpathlineto{\pgfqpoint{2.515660in}{0.739656in}}%
\pgfpathlineto{\pgfqpoint{2.515364in}{0.739656in}}%
\pgfpathlineto{\pgfqpoint{2.515068in}{0.739656in}}%
\pgfpathlineto{\pgfqpoint{2.514772in}{0.739656in}}%
\pgfpathlineto{\pgfqpoint{2.514476in}{0.739656in}}%
\pgfpathlineto{\pgfqpoint{2.514180in}{0.739656in}}%
\pgfpathlineto{\pgfqpoint{2.513884in}{0.739656in}}%
\pgfpathlineto{\pgfqpoint{2.513588in}{0.739656in}}%
\pgfpathlineto{\pgfqpoint{2.513292in}{0.739656in}}%
\pgfpathlineto{\pgfqpoint{2.512996in}{0.739656in}}%
\pgfpathlineto{\pgfqpoint{2.512700in}{0.739656in}}%
\pgfpathlineto{\pgfqpoint{2.512404in}{0.739656in}}%
\pgfpathlineto{\pgfqpoint{2.512108in}{0.739656in}}%
\pgfpathlineto{\pgfqpoint{2.511812in}{0.739656in}}%
\pgfpathlineto{\pgfqpoint{2.511516in}{0.739656in}}%
\pgfpathlineto{\pgfqpoint{2.511220in}{0.739656in}}%
\pgfpathlineto{\pgfqpoint{2.510924in}{0.739656in}}%
\pgfpathlineto{\pgfqpoint{2.510628in}{0.739656in}}%
\pgfpathlineto{\pgfqpoint{2.510332in}{0.739656in}}%
\pgfpathlineto{\pgfqpoint{2.510036in}{0.739656in}}%
\pgfpathlineto{\pgfqpoint{2.509740in}{0.739656in}}%
\pgfpathlineto{\pgfqpoint{2.509444in}{0.739656in}}%
\pgfpathlineto{\pgfqpoint{2.509148in}{0.739656in}}%
\pgfpathlineto{\pgfqpoint{2.508852in}{0.739656in}}%
\pgfpathlineto{\pgfqpoint{2.508556in}{0.739656in}}%
\pgfpathlineto{\pgfqpoint{2.508260in}{0.739656in}}%
\pgfpathlineto{\pgfqpoint{2.507964in}{0.739656in}}%
\pgfpathlineto{\pgfqpoint{2.507668in}{0.739656in}}%
\pgfpathlineto{\pgfqpoint{2.507372in}{0.739656in}}%
\pgfpathlineto{\pgfqpoint{2.507076in}{0.739656in}}%
\pgfpathlineto{\pgfqpoint{2.506780in}{0.739656in}}%
\pgfpathlineto{\pgfqpoint{2.506484in}{0.739656in}}%
\pgfpathlineto{\pgfqpoint{2.506188in}{0.739656in}}%
\pgfpathlineto{\pgfqpoint{2.505892in}{0.739656in}}%
\pgfpathlineto{\pgfqpoint{2.505596in}{0.739656in}}%
\pgfpathlineto{\pgfqpoint{2.505300in}{0.739656in}}%
\pgfpathlineto{\pgfqpoint{2.505004in}{0.739656in}}%
\pgfpathlineto{\pgfqpoint{2.504708in}{0.739656in}}%
\pgfpathlineto{\pgfqpoint{2.504412in}{0.739656in}}%
\pgfpathlineto{\pgfqpoint{2.504116in}{0.739656in}}%
\pgfpathlineto{\pgfqpoint{2.503820in}{0.739656in}}%
\pgfpathlineto{\pgfqpoint{2.503524in}{0.739656in}}%
\pgfpathlineto{\pgfqpoint{2.503228in}{0.739656in}}%
\pgfpathlineto{\pgfqpoint{2.502932in}{0.739656in}}%
\pgfpathlineto{\pgfqpoint{2.502636in}{0.739656in}}%
\pgfpathlineto{\pgfqpoint{2.502340in}{0.739656in}}%
\pgfpathlineto{\pgfqpoint{2.502044in}{0.739656in}}%
\pgfpathlineto{\pgfqpoint{2.501748in}{0.739656in}}%
\pgfpathlineto{\pgfqpoint{2.501452in}{0.739656in}}%
\pgfpathlineto{\pgfqpoint{2.501156in}{0.739656in}}%
\pgfpathlineto{\pgfqpoint{2.500860in}{0.739656in}}%
\pgfpathlineto{\pgfqpoint{2.500564in}{0.739656in}}%
\pgfpathlineto{\pgfqpoint{2.500268in}{0.739656in}}%
\pgfpathlineto{\pgfqpoint{2.499972in}{0.739656in}}%
\pgfpathlineto{\pgfqpoint{2.499676in}{0.739656in}}%
\pgfpathlineto{\pgfqpoint{2.499380in}{0.739656in}}%
\pgfpathlineto{\pgfqpoint{2.499084in}{0.739656in}}%
\pgfpathlineto{\pgfqpoint{2.498788in}{0.739656in}}%
\pgfpathlineto{\pgfqpoint{2.498492in}{0.739656in}}%
\pgfpathlineto{\pgfqpoint{2.498196in}{0.739656in}}%
\pgfpathlineto{\pgfqpoint{2.497900in}{0.739656in}}%
\pgfpathlineto{\pgfqpoint{2.497604in}{0.739656in}}%
\pgfpathlineto{\pgfqpoint{2.497308in}{0.739656in}}%
\pgfpathlineto{\pgfqpoint{2.497012in}{0.739656in}}%
\pgfpathlineto{\pgfqpoint{2.496716in}{0.739656in}}%
\pgfpathlineto{\pgfqpoint{2.496420in}{0.739656in}}%
\pgfpathlineto{\pgfqpoint{2.496124in}{0.739656in}}%
\pgfpathlineto{\pgfqpoint{2.495828in}{0.739656in}}%
\pgfpathlineto{\pgfqpoint{2.495532in}{0.739656in}}%
\pgfpathlineto{\pgfqpoint{2.495236in}{0.739656in}}%
\pgfpathlineto{\pgfqpoint{2.494940in}{0.739656in}}%
\pgfpathlineto{\pgfqpoint{2.494644in}{0.739656in}}%
\pgfpathlineto{\pgfqpoint{2.494348in}{0.739656in}}%
\pgfpathlineto{\pgfqpoint{2.494052in}{0.739656in}}%
\pgfpathlineto{\pgfqpoint{2.493756in}{0.739656in}}%
\pgfpathlineto{\pgfqpoint{2.493460in}{0.739656in}}%
\pgfpathlineto{\pgfqpoint{2.493164in}{0.739656in}}%
\pgfpathlineto{\pgfqpoint{2.492868in}{0.739656in}}%
\pgfpathlineto{\pgfqpoint{2.492572in}{0.739656in}}%
\pgfpathlineto{\pgfqpoint{2.492276in}{0.739656in}}%
\pgfpathlineto{\pgfqpoint{2.491980in}{0.739656in}}%
\pgfpathlineto{\pgfqpoint{2.491684in}{0.739656in}}%
\pgfpathlineto{\pgfqpoint{2.491388in}{0.739656in}}%
\pgfpathlineto{\pgfqpoint{2.491092in}{0.739656in}}%
\pgfpathlineto{\pgfqpoint{2.490796in}{0.739656in}}%
\pgfpathlineto{\pgfqpoint{2.490500in}{0.739656in}}%
\pgfpathlineto{\pgfqpoint{2.490204in}{0.739656in}}%
\pgfpathlineto{\pgfqpoint{2.489908in}{0.739656in}}%
\pgfpathlineto{\pgfqpoint{2.489612in}{0.739656in}}%
\pgfpathlineto{\pgfqpoint{2.489316in}{0.739656in}}%
\pgfpathlineto{\pgfqpoint{2.489020in}{0.739656in}}%
\pgfpathlineto{\pgfqpoint{2.488724in}{0.739656in}}%
\pgfpathlineto{\pgfqpoint{2.488428in}{0.739656in}}%
\pgfpathlineto{\pgfqpoint{2.488132in}{0.739656in}}%
\pgfpathlineto{\pgfqpoint{2.487836in}{0.739656in}}%
\pgfpathlineto{\pgfqpoint{2.487540in}{0.739656in}}%
\pgfpathlineto{\pgfqpoint{2.487244in}{0.739656in}}%
\pgfpathlineto{\pgfqpoint{2.486948in}{0.739656in}}%
\pgfpathlineto{\pgfqpoint{2.486652in}{0.739656in}}%
\pgfpathlineto{\pgfqpoint{2.486356in}{0.739656in}}%
\pgfpathlineto{\pgfqpoint{2.486060in}{0.739656in}}%
\pgfpathlineto{\pgfqpoint{2.485764in}{0.739656in}}%
\pgfpathlineto{\pgfqpoint{2.485468in}{0.739656in}}%
\pgfpathlineto{\pgfqpoint{2.485172in}{0.739656in}}%
\pgfpathlineto{\pgfqpoint{2.484876in}{0.739656in}}%
\pgfpathlineto{\pgfqpoint{2.484580in}{0.739656in}}%
\pgfpathlineto{\pgfqpoint{2.484284in}{0.739656in}}%
\pgfpathlineto{\pgfqpoint{2.483988in}{0.739656in}}%
\pgfpathlineto{\pgfqpoint{2.483692in}{0.739656in}}%
\pgfpathlineto{\pgfqpoint{2.483396in}{0.739656in}}%
\pgfpathlineto{\pgfqpoint{2.483100in}{0.739656in}}%
\pgfpathlineto{\pgfqpoint{2.482804in}{0.739656in}}%
\pgfpathlineto{\pgfqpoint{2.482508in}{0.739656in}}%
\pgfpathlineto{\pgfqpoint{2.482212in}{0.739656in}}%
\pgfpathlineto{\pgfqpoint{2.481916in}{0.739656in}}%
\pgfpathlineto{\pgfqpoint{2.481620in}{0.739656in}}%
\pgfpathlineto{\pgfqpoint{2.481324in}{0.739656in}}%
\pgfpathlineto{\pgfqpoint{2.481028in}{0.739656in}}%
\pgfpathlineto{\pgfqpoint{2.480732in}{0.739656in}}%
\pgfpathlineto{\pgfqpoint{2.480436in}{0.739656in}}%
\pgfpathlineto{\pgfqpoint{2.480140in}{0.739656in}}%
\pgfpathlineto{\pgfqpoint{2.479844in}{0.739656in}}%
\pgfpathlineto{\pgfqpoint{2.479548in}{0.739656in}}%
\pgfpathlineto{\pgfqpoint{2.479252in}{0.739656in}}%
\pgfpathlineto{\pgfqpoint{2.478956in}{0.739656in}}%
\pgfpathlineto{\pgfqpoint{2.478660in}{0.739656in}}%
\pgfpathlineto{\pgfqpoint{2.478364in}{0.739656in}}%
\pgfpathlineto{\pgfqpoint{2.478068in}{0.739656in}}%
\pgfpathlineto{\pgfqpoint{2.477772in}{0.739656in}}%
\pgfpathlineto{\pgfqpoint{2.477476in}{0.739656in}}%
\pgfpathlineto{\pgfqpoint{2.477180in}{0.739656in}}%
\pgfpathlineto{\pgfqpoint{2.476884in}{0.739656in}}%
\pgfpathlineto{\pgfqpoint{2.476588in}{0.739656in}}%
\pgfpathlineto{\pgfqpoint{2.476292in}{0.739656in}}%
\pgfpathlineto{\pgfqpoint{2.475996in}{0.739656in}}%
\pgfpathlineto{\pgfqpoint{2.475700in}{0.739656in}}%
\pgfpathlineto{\pgfqpoint{2.475404in}{0.739656in}}%
\pgfpathlineto{\pgfqpoint{2.475108in}{0.739656in}}%
\pgfpathlineto{\pgfqpoint{2.474812in}{0.739656in}}%
\pgfpathlineto{\pgfqpoint{2.474516in}{0.739656in}}%
\pgfpathlineto{\pgfqpoint{2.474220in}{0.739656in}}%
\pgfpathlineto{\pgfqpoint{2.473924in}{0.739656in}}%
\pgfpathlineto{\pgfqpoint{2.473628in}{0.739656in}}%
\pgfpathlineto{\pgfqpoint{2.473332in}{0.739656in}}%
\pgfpathlineto{\pgfqpoint{2.473036in}{0.739656in}}%
\pgfpathlineto{\pgfqpoint{2.472740in}{0.739656in}}%
\pgfpathlineto{\pgfqpoint{2.472444in}{0.739656in}}%
\pgfpathlineto{\pgfqpoint{2.472148in}{0.739656in}}%
\pgfpathlineto{\pgfqpoint{2.471852in}{0.739656in}}%
\pgfpathlineto{\pgfqpoint{2.471556in}{0.739656in}}%
\pgfpathlineto{\pgfqpoint{2.471260in}{0.739656in}}%
\pgfpathlineto{\pgfqpoint{2.470964in}{0.739656in}}%
\pgfpathlineto{\pgfqpoint{2.470668in}{0.739656in}}%
\pgfpathlineto{\pgfqpoint{2.470372in}{0.739656in}}%
\pgfpathlineto{\pgfqpoint{2.470076in}{0.739656in}}%
\pgfpathlineto{\pgfqpoint{2.469780in}{0.739656in}}%
\pgfpathlineto{\pgfqpoint{2.469484in}{0.739656in}}%
\pgfpathlineto{\pgfqpoint{2.469188in}{0.739656in}}%
\pgfpathlineto{\pgfqpoint{2.468892in}{0.739656in}}%
\pgfpathlineto{\pgfqpoint{2.468596in}{0.739656in}}%
\pgfpathlineto{\pgfqpoint{2.468300in}{0.739656in}}%
\pgfpathlineto{\pgfqpoint{2.468003in}{0.739656in}}%
\pgfpathlineto{\pgfqpoint{2.467707in}{0.739656in}}%
\pgfpathlineto{\pgfqpoint{2.467411in}{0.739656in}}%
\pgfpathlineto{\pgfqpoint{2.467115in}{0.739656in}}%
\pgfpathlineto{\pgfqpoint{2.466819in}{0.739656in}}%
\pgfpathlineto{\pgfqpoint{2.466523in}{0.739656in}}%
\pgfpathlineto{\pgfqpoint{2.466227in}{0.739656in}}%
\pgfpathlineto{\pgfqpoint{2.465931in}{0.739656in}}%
\pgfpathlineto{\pgfqpoint{2.465635in}{0.739656in}}%
\pgfpathlineto{\pgfqpoint{2.465339in}{0.739656in}}%
\pgfpathlineto{\pgfqpoint{2.465043in}{0.739656in}}%
\pgfpathlineto{\pgfqpoint{2.464747in}{0.739656in}}%
\pgfpathlineto{\pgfqpoint{2.464451in}{0.739656in}}%
\pgfpathlineto{\pgfqpoint{2.464155in}{0.739656in}}%
\pgfpathlineto{\pgfqpoint{2.463859in}{0.739656in}}%
\pgfpathlineto{\pgfqpoint{2.463563in}{0.739656in}}%
\pgfpathlineto{\pgfqpoint{2.463267in}{0.739656in}}%
\pgfpathlineto{\pgfqpoint{2.462971in}{0.739656in}}%
\pgfpathlineto{\pgfqpoint{2.462675in}{0.739656in}}%
\pgfpathlineto{\pgfqpoint{2.462379in}{0.739656in}}%
\pgfpathlineto{\pgfqpoint{2.462083in}{0.739656in}}%
\pgfpathlineto{\pgfqpoint{2.461787in}{0.739656in}}%
\pgfpathlineto{\pgfqpoint{2.461491in}{0.739656in}}%
\pgfpathlineto{\pgfqpoint{2.461195in}{0.739656in}}%
\pgfpathlineto{\pgfqpoint{2.460899in}{0.739656in}}%
\pgfpathlineto{\pgfqpoint{2.460603in}{0.739656in}}%
\pgfpathlineto{\pgfqpoint{2.460307in}{0.739656in}}%
\pgfpathlineto{\pgfqpoint{2.460011in}{0.739656in}}%
\pgfpathlineto{\pgfqpoint{2.459715in}{0.739656in}}%
\pgfpathlineto{\pgfqpoint{2.459419in}{0.739656in}}%
\pgfpathlineto{\pgfqpoint{2.459123in}{0.739656in}}%
\pgfpathlineto{\pgfqpoint{2.458827in}{0.739656in}}%
\pgfpathlineto{\pgfqpoint{2.458531in}{0.739656in}}%
\pgfpathlineto{\pgfqpoint{2.458235in}{0.739656in}}%
\pgfpathlineto{\pgfqpoint{2.457939in}{0.739656in}}%
\pgfpathlineto{\pgfqpoint{2.457643in}{0.739656in}}%
\pgfpathlineto{\pgfqpoint{2.457347in}{0.739656in}}%
\pgfpathlineto{\pgfqpoint{2.457051in}{0.739656in}}%
\pgfpathlineto{\pgfqpoint{2.456755in}{0.739656in}}%
\pgfpathlineto{\pgfqpoint{2.456459in}{0.739656in}}%
\pgfpathlineto{\pgfqpoint{2.456163in}{0.739656in}}%
\pgfpathlineto{\pgfqpoint{2.455867in}{0.739656in}}%
\pgfpathlineto{\pgfqpoint{2.455571in}{0.739656in}}%
\pgfpathlineto{\pgfqpoint{2.455275in}{0.739656in}}%
\pgfpathlineto{\pgfqpoint{2.454979in}{0.739656in}}%
\pgfpathlineto{\pgfqpoint{2.454683in}{0.739656in}}%
\pgfpathlineto{\pgfqpoint{2.454387in}{0.739656in}}%
\pgfpathlineto{\pgfqpoint{2.454091in}{0.739656in}}%
\pgfpathlineto{\pgfqpoint{2.453795in}{0.739656in}}%
\pgfpathlineto{\pgfqpoint{2.453499in}{0.739656in}}%
\pgfpathlineto{\pgfqpoint{2.453203in}{0.739656in}}%
\pgfpathlineto{\pgfqpoint{2.452907in}{0.739656in}}%
\pgfpathlineto{\pgfqpoint{2.452611in}{0.739656in}}%
\pgfpathlineto{\pgfqpoint{2.452315in}{0.739656in}}%
\pgfpathlineto{\pgfqpoint{2.452019in}{0.739656in}}%
\pgfpathlineto{\pgfqpoint{2.451723in}{0.739656in}}%
\pgfpathlineto{\pgfqpoint{2.451427in}{0.739656in}}%
\pgfpathlineto{\pgfqpoint{2.451131in}{0.739656in}}%
\pgfpathlineto{\pgfqpoint{2.450835in}{0.739656in}}%
\pgfpathlineto{\pgfqpoint{2.450539in}{0.739656in}}%
\pgfpathlineto{\pgfqpoint{2.450243in}{0.739656in}}%
\pgfpathlineto{\pgfqpoint{2.449947in}{0.739656in}}%
\pgfpathlineto{\pgfqpoint{2.449651in}{0.739656in}}%
\pgfpathlineto{\pgfqpoint{2.449355in}{0.739656in}}%
\pgfpathlineto{\pgfqpoint{2.449059in}{0.739656in}}%
\pgfpathlineto{\pgfqpoint{2.448763in}{0.739656in}}%
\pgfpathlineto{\pgfqpoint{2.448467in}{0.739656in}}%
\pgfpathlineto{\pgfqpoint{2.448171in}{0.739656in}}%
\pgfpathlineto{\pgfqpoint{2.447875in}{0.739656in}}%
\pgfpathlineto{\pgfqpoint{2.447579in}{0.739656in}}%
\pgfpathlineto{\pgfqpoint{2.447283in}{0.739656in}}%
\pgfpathlineto{\pgfqpoint{2.446987in}{0.739656in}}%
\pgfpathlineto{\pgfqpoint{2.446691in}{0.739656in}}%
\pgfpathlineto{\pgfqpoint{2.446395in}{0.739656in}}%
\pgfpathlineto{\pgfqpoint{2.446099in}{0.739656in}}%
\pgfpathlineto{\pgfqpoint{2.445803in}{0.739656in}}%
\pgfpathlineto{\pgfqpoint{2.445507in}{0.739656in}}%
\pgfpathlineto{\pgfqpoint{2.445211in}{0.739656in}}%
\pgfpathlineto{\pgfqpoint{2.444915in}{0.739656in}}%
\pgfpathlineto{\pgfqpoint{2.444619in}{0.739656in}}%
\pgfpathlineto{\pgfqpoint{2.444323in}{0.739656in}}%
\pgfpathlineto{\pgfqpoint{2.444027in}{0.739656in}}%
\pgfpathlineto{\pgfqpoint{2.443731in}{0.739656in}}%
\pgfpathlineto{\pgfqpoint{2.443435in}{0.739656in}}%
\pgfpathlineto{\pgfqpoint{2.443139in}{0.739656in}}%
\pgfpathlineto{\pgfqpoint{2.442843in}{0.739656in}}%
\pgfpathlineto{\pgfqpoint{2.442547in}{0.739656in}}%
\pgfpathlineto{\pgfqpoint{2.442251in}{0.739656in}}%
\pgfpathlineto{\pgfqpoint{2.441955in}{0.739656in}}%
\pgfpathlineto{\pgfqpoint{2.441659in}{0.739656in}}%
\pgfpathlineto{\pgfqpoint{2.441363in}{0.739656in}}%
\pgfpathlineto{\pgfqpoint{2.441067in}{0.739656in}}%
\pgfpathlineto{\pgfqpoint{2.440771in}{0.739656in}}%
\pgfpathlineto{\pgfqpoint{2.440475in}{0.739656in}}%
\pgfpathlineto{\pgfqpoint{2.440179in}{0.739656in}}%
\pgfpathlineto{\pgfqpoint{2.439883in}{0.739656in}}%
\pgfpathlineto{\pgfqpoint{2.439587in}{0.739656in}}%
\pgfpathlineto{\pgfqpoint{2.439291in}{0.739656in}}%
\pgfpathlineto{\pgfqpoint{2.438995in}{0.739656in}}%
\pgfpathlineto{\pgfqpoint{2.438699in}{0.739656in}}%
\pgfpathlineto{\pgfqpoint{2.438403in}{0.739656in}}%
\pgfpathlineto{\pgfqpoint{2.438107in}{0.739656in}}%
\pgfpathlineto{\pgfqpoint{2.437811in}{0.739656in}}%
\pgfpathlineto{\pgfqpoint{2.437515in}{0.739656in}}%
\pgfpathlineto{\pgfqpoint{2.437219in}{0.739656in}}%
\pgfpathlineto{\pgfqpoint{2.436923in}{0.739656in}}%
\pgfpathlineto{\pgfqpoint{2.436627in}{0.739656in}}%
\pgfpathlineto{\pgfqpoint{2.436331in}{0.739656in}}%
\pgfpathlineto{\pgfqpoint{2.436035in}{0.739656in}}%
\pgfpathlineto{\pgfqpoint{2.435739in}{0.739656in}}%
\pgfpathlineto{\pgfqpoint{2.435443in}{0.739656in}}%
\pgfpathlineto{\pgfqpoint{2.435147in}{0.739656in}}%
\pgfpathlineto{\pgfqpoint{2.434851in}{0.739656in}}%
\pgfpathlineto{\pgfqpoint{2.434555in}{0.739656in}}%
\pgfpathlineto{\pgfqpoint{2.434259in}{0.739656in}}%
\pgfpathlineto{\pgfqpoint{2.433963in}{0.739656in}}%
\pgfpathlineto{\pgfqpoint{2.433667in}{0.739656in}}%
\pgfpathlineto{\pgfqpoint{2.433371in}{0.739656in}}%
\pgfpathlineto{\pgfqpoint{2.433075in}{0.739656in}}%
\pgfpathlineto{\pgfqpoint{2.432779in}{0.739656in}}%
\pgfpathlineto{\pgfqpoint{2.432483in}{0.739656in}}%
\pgfpathlineto{\pgfqpoint{2.432187in}{0.739656in}}%
\pgfpathlineto{\pgfqpoint{2.431891in}{0.739656in}}%
\pgfpathlineto{\pgfqpoint{2.431595in}{0.739656in}}%
\pgfpathlineto{\pgfqpoint{2.431299in}{0.739656in}}%
\pgfpathlineto{\pgfqpoint{2.431003in}{0.739656in}}%
\pgfpathlineto{\pgfqpoint{2.430707in}{0.739656in}}%
\pgfpathlineto{\pgfqpoint{2.430411in}{0.739656in}}%
\pgfpathlineto{\pgfqpoint{2.430115in}{0.739656in}}%
\pgfpathlineto{\pgfqpoint{2.429819in}{0.739656in}}%
\pgfpathlineto{\pgfqpoint{2.429523in}{0.739656in}}%
\pgfpathlineto{\pgfqpoint{2.429227in}{0.739656in}}%
\pgfpathlineto{\pgfqpoint{2.428931in}{0.739656in}}%
\pgfpathlineto{\pgfqpoint{2.428635in}{0.739656in}}%
\pgfpathlineto{\pgfqpoint{2.428339in}{0.739656in}}%
\pgfpathlineto{\pgfqpoint{2.428043in}{0.739656in}}%
\pgfpathlineto{\pgfqpoint{2.427747in}{0.739656in}}%
\pgfpathlineto{\pgfqpoint{2.427451in}{0.739656in}}%
\pgfpathlineto{\pgfqpoint{2.427155in}{0.739656in}}%
\pgfpathlineto{\pgfqpoint{2.426859in}{0.739656in}}%
\pgfpathlineto{\pgfqpoint{2.426563in}{0.739656in}}%
\pgfpathlineto{\pgfqpoint{2.426267in}{0.739656in}}%
\pgfpathlineto{\pgfqpoint{2.425971in}{0.739656in}}%
\pgfpathlineto{\pgfqpoint{2.425675in}{0.739656in}}%
\pgfpathlineto{\pgfqpoint{2.425379in}{0.739656in}}%
\pgfpathlineto{\pgfqpoint{2.425083in}{0.739656in}}%
\pgfpathlineto{\pgfqpoint{2.424787in}{0.739656in}}%
\pgfpathlineto{\pgfqpoint{2.424491in}{0.739656in}}%
\pgfpathlineto{\pgfqpoint{2.424195in}{0.739656in}}%
\pgfpathlineto{\pgfqpoint{2.423899in}{0.739656in}}%
\pgfpathlineto{\pgfqpoint{2.423603in}{0.739656in}}%
\pgfpathlineto{\pgfqpoint{2.423307in}{0.739656in}}%
\pgfpathlineto{\pgfqpoint{2.423011in}{0.739656in}}%
\pgfpathlineto{\pgfqpoint{2.422715in}{0.739656in}}%
\pgfpathlineto{\pgfqpoint{2.422419in}{0.739656in}}%
\pgfpathlineto{\pgfqpoint{2.422123in}{0.739656in}}%
\pgfpathlineto{\pgfqpoint{2.421827in}{0.739656in}}%
\pgfpathlineto{\pgfqpoint{2.421531in}{0.739656in}}%
\pgfpathlineto{\pgfqpoint{2.421235in}{0.739656in}}%
\pgfpathlineto{\pgfqpoint{2.420939in}{0.739656in}}%
\pgfpathlineto{\pgfqpoint{2.420643in}{0.739656in}}%
\pgfpathlineto{\pgfqpoint{2.420347in}{0.739656in}}%
\pgfpathlineto{\pgfqpoint{2.420051in}{0.739656in}}%
\pgfpathlineto{\pgfqpoint{2.419755in}{0.739656in}}%
\pgfpathlineto{\pgfqpoint{2.419459in}{0.739656in}}%
\pgfpathlineto{\pgfqpoint{2.419163in}{0.739656in}}%
\pgfpathlineto{\pgfqpoint{2.418867in}{0.739656in}}%
\pgfpathlineto{\pgfqpoint{2.418571in}{0.739656in}}%
\pgfpathlineto{\pgfqpoint{2.418275in}{0.739656in}}%
\pgfpathlineto{\pgfqpoint{2.417979in}{0.739656in}}%
\pgfpathlineto{\pgfqpoint{2.417683in}{0.739656in}}%
\pgfpathlineto{\pgfqpoint{2.417387in}{0.739656in}}%
\pgfpathlineto{\pgfqpoint{2.417091in}{0.739656in}}%
\pgfpathlineto{\pgfqpoint{2.416795in}{0.739656in}}%
\pgfpathlineto{\pgfqpoint{2.416499in}{0.739656in}}%
\pgfpathlineto{\pgfqpoint{2.416203in}{0.739656in}}%
\pgfpathlineto{\pgfqpoint{2.415907in}{0.739656in}}%
\pgfpathlineto{\pgfqpoint{2.415611in}{0.739656in}}%
\pgfpathlineto{\pgfqpoint{2.415315in}{0.739656in}}%
\pgfpathlineto{\pgfqpoint{2.415019in}{0.739656in}}%
\pgfpathlineto{\pgfqpoint{2.414723in}{0.739656in}}%
\pgfpathlineto{\pgfqpoint{2.414427in}{0.739656in}}%
\pgfpathlineto{\pgfqpoint{2.414131in}{0.739656in}}%
\pgfpathlineto{\pgfqpoint{2.413835in}{0.739656in}}%
\pgfpathlineto{\pgfqpoint{2.413539in}{0.739656in}}%
\pgfpathlineto{\pgfqpoint{2.413243in}{0.739656in}}%
\pgfpathlineto{\pgfqpoint{2.412947in}{0.739656in}}%
\pgfpathlineto{\pgfqpoint{2.412651in}{0.739656in}}%
\pgfpathlineto{\pgfqpoint{2.412355in}{0.739656in}}%
\pgfpathlineto{\pgfqpoint{2.412059in}{0.739656in}}%
\pgfpathlineto{\pgfqpoint{2.411763in}{0.739656in}}%
\pgfpathlineto{\pgfqpoint{2.411467in}{0.739656in}}%
\pgfpathlineto{\pgfqpoint{2.411171in}{0.739656in}}%
\pgfpathlineto{\pgfqpoint{2.410875in}{0.739656in}}%
\pgfpathlineto{\pgfqpoint{2.410579in}{0.739656in}}%
\pgfpathlineto{\pgfqpoint{2.410283in}{0.739656in}}%
\pgfpathlineto{\pgfqpoint{2.409987in}{0.739656in}}%
\pgfpathlineto{\pgfqpoint{2.409691in}{0.739656in}}%
\pgfpathlineto{\pgfqpoint{2.409395in}{0.739656in}}%
\pgfpathlineto{\pgfqpoint{2.409099in}{0.739656in}}%
\pgfpathlineto{\pgfqpoint{2.408803in}{0.739656in}}%
\pgfpathlineto{\pgfqpoint{2.408507in}{0.739656in}}%
\pgfpathlineto{\pgfqpoint{2.408211in}{0.739656in}}%
\pgfpathlineto{\pgfqpoint{2.407915in}{0.739656in}}%
\pgfpathlineto{\pgfqpoint{2.407619in}{0.739656in}}%
\pgfpathlineto{\pgfqpoint{2.407323in}{0.739656in}}%
\pgfpathlineto{\pgfqpoint{2.407027in}{0.739656in}}%
\pgfpathlineto{\pgfqpoint{2.406731in}{0.739656in}}%
\pgfpathlineto{\pgfqpoint{2.406435in}{0.739656in}}%
\pgfpathlineto{\pgfqpoint{2.406139in}{0.739656in}}%
\pgfpathlineto{\pgfqpoint{2.405843in}{0.739656in}}%
\pgfpathlineto{\pgfqpoint{2.405547in}{0.739656in}}%
\pgfpathlineto{\pgfqpoint{2.405251in}{0.739656in}}%
\pgfpathlineto{\pgfqpoint{2.404955in}{0.739656in}}%
\pgfpathlineto{\pgfqpoint{2.404659in}{0.739656in}}%
\pgfpathlineto{\pgfqpoint{2.404363in}{0.739656in}}%
\pgfpathlineto{\pgfqpoint{2.404067in}{0.739656in}}%
\pgfpathlineto{\pgfqpoint{2.403771in}{0.739656in}}%
\pgfpathlineto{\pgfqpoint{2.403475in}{0.739656in}}%
\pgfpathlineto{\pgfqpoint{2.403179in}{0.739656in}}%
\pgfpathlineto{\pgfqpoint{2.402883in}{0.739656in}}%
\pgfpathlineto{\pgfqpoint{2.402587in}{0.739656in}}%
\pgfpathlineto{\pgfqpoint{2.402291in}{0.739656in}}%
\pgfpathlineto{\pgfqpoint{2.401995in}{0.739656in}}%
\pgfpathlineto{\pgfqpoint{2.401699in}{0.739656in}}%
\pgfpathlineto{\pgfqpoint{2.401403in}{0.739656in}}%
\pgfpathlineto{\pgfqpoint{2.401107in}{0.739656in}}%
\pgfpathlineto{\pgfqpoint{2.400811in}{0.739656in}}%
\pgfpathlineto{\pgfqpoint{2.400514in}{0.739656in}}%
\pgfpathlineto{\pgfqpoint{2.400218in}{0.739656in}}%
\pgfpathlineto{\pgfqpoint{2.399922in}{0.739656in}}%
\pgfpathlineto{\pgfqpoint{2.399626in}{0.739656in}}%
\pgfpathlineto{\pgfqpoint{2.399330in}{0.739656in}}%
\pgfpathlineto{\pgfqpoint{2.399034in}{0.739656in}}%
\pgfpathlineto{\pgfqpoint{2.398738in}{0.739656in}}%
\pgfpathlineto{\pgfqpoint{2.398442in}{0.739656in}}%
\pgfpathlineto{\pgfqpoint{2.398146in}{0.739656in}}%
\pgfpathlineto{\pgfqpoint{2.397850in}{0.739656in}}%
\pgfpathlineto{\pgfqpoint{2.397554in}{0.739656in}}%
\pgfpathlineto{\pgfqpoint{2.397258in}{0.739656in}}%
\pgfpathlineto{\pgfqpoint{2.396962in}{0.739656in}}%
\pgfpathlineto{\pgfqpoint{2.396666in}{0.739656in}}%
\pgfpathlineto{\pgfqpoint{2.396370in}{0.739656in}}%
\pgfpathlineto{\pgfqpoint{2.396074in}{0.739656in}}%
\pgfpathlineto{\pgfqpoint{2.395778in}{0.739656in}}%
\pgfpathlineto{\pgfqpoint{2.395482in}{0.739656in}}%
\pgfpathlineto{\pgfqpoint{2.395186in}{0.739656in}}%
\pgfpathlineto{\pgfqpoint{2.394890in}{0.739656in}}%
\pgfpathlineto{\pgfqpoint{2.394594in}{0.739656in}}%
\pgfpathlineto{\pgfqpoint{2.394298in}{0.739656in}}%
\pgfpathlineto{\pgfqpoint{2.394002in}{0.739656in}}%
\pgfpathlineto{\pgfqpoint{2.393706in}{0.739656in}}%
\pgfpathlineto{\pgfqpoint{2.393410in}{0.739656in}}%
\pgfpathlineto{\pgfqpoint{2.393114in}{0.739656in}}%
\pgfpathlineto{\pgfqpoint{2.392818in}{0.739656in}}%
\pgfpathlineto{\pgfqpoint{2.392522in}{0.739656in}}%
\pgfpathlineto{\pgfqpoint{2.392226in}{0.739656in}}%
\pgfpathlineto{\pgfqpoint{2.391930in}{0.739656in}}%
\pgfpathlineto{\pgfqpoint{2.391634in}{0.739656in}}%
\pgfpathlineto{\pgfqpoint{2.391338in}{0.739656in}}%
\pgfpathlineto{\pgfqpoint{2.391042in}{0.739656in}}%
\pgfpathlineto{\pgfqpoint{2.390746in}{0.739656in}}%
\pgfpathlineto{\pgfqpoint{2.390450in}{0.739656in}}%
\pgfpathlineto{\pgfqpoint{2.390154in}{0.739656in}}%
\pgfpathlineto{\pgfqpoint{2.389858in}{0.739656in}}%
\pgfpathlineto{\pgfqpoint{2.389562in}{0.739656in}}%
\pgfpathlineto{\pgfqpoint{2.389266in}{0.739656in}}%
\pgfpathlineto{\pgfqpoint{2.388970in}{0.739656in}}%
\pgfpathlineto{\pgfqpoint{2.388674in}{0.739656in}}%
\pgfpathlineto{\pgfqpoint{2.388378in}{0.739656in}}%
\pgfpathlineto{\pgfqpoint{2.388082in}{0.739656in}}%
\pgfpathlineto{\pgfqpoint{2.387786in}{0.739656in}}%
\pgfpathlineto{\pgfqpoint{2.387490in}{0.739656in}}%
\pgfpathlineto{\pgfqpoint{2.387194in}{0.739656in}}%
\pgfpathlineto{\pgfqpoint{2.386898in}{0.739656in}}%
\pgfpathlineto{\pgfqpoint{2.386602in}{0.739656in}}%
\pgfpathlineto{\pgfqpoint{2.386306in}{0.739656in}}%
\pgfpathlineto{\pgfqpoint{2.386010in}{0.739656in}}%
\pgfpathlineto{\pgfqpoint{2.385714in}{0.739656in}}%
\pgfpathlineto{\pgfqpoint{2.385418in}{0.739656in}}%
\pgfpathlineto{\pgfqpoint{2.385122in}{0.739656in}}%
\pgfpathlineto{\pgfqpoint{2.384826in}{0.739656in}}%
\pgfpathlineto{\pgfqpoint{2.384530in}{0.739656in}}%
\pgfpathlineto{\pgfqpoint{2.384234in}{0.739656in}}%
\pgfpathlineto{\pgfqpoint{2.383938in}{0.739656in}}%
\pgfpathlineto{\pgfqpoint{2.383642in}{0.739656in}}%
\pgfpathlineto{\pgfqpoint{2.383346in}{0.739656in}}%
\pgfpathlineto{\pgfqpoint{2.383050in}{0.739656in}}%
\pgfpathlineto{\pgfqpoint{2.382754in}{0.739656in}}%
\pgfpathlineto{\pgfqpoint{2.382458in}{0.739656in}}%
\pgfpathlineto{\pgfqpoint{2.382162in}{0.739656in}}%
\pgfpathlineto{\pgfqpoint{2.381866in}{0.739656in}}%
\pgfpathlineto{\pgfqpoint{2.381570in}{0.739656in}}%
\pgfpathlineto{\pgfqpoint{2.381274in}{0.739656in}}%
\pgfpathlineto{\pgfqpoint{2.380978in}{0.739656in}}%
\pgfpathlineto{\pgfqpoint{2.380682in}{0.739656in}}%
\pgfpathlineto{\pgfqpoint{2.380386in}{0.739656in}}%
\pgfpathlineto{\pgfqpoint{2.380090in}{0.739656in}}%
\pgfpathlineto{\pgfqpoint{2.379794in}{0.739656in}}%
\pgfpathlineto{\pgfqpoint{2.379498in}{0.739656in}}%
\pgfpathlineto{\pgfqpoint{2.379202in}{0.739656in}}%
\pgfpathlineto{\pgfqpoint{2.378906in}{0.739656in}}%
\pgfpathlineto{\pgfqpoint{2.378610in}{0.739656in}}%
\pgfpathlineto{\pgfqpoint{2.378314in}{0.739656in}}%
\pgfpathlineto{\pgfqpoint{2.378018in}{0.739656in}}%
\pgfpathlineto{\pgfqpoint{2.377722in}{0.739656in}}%
\pgfpathlineto{\pgfqpoint{2.377426in}{0.739656in}}%
\pgfpathlineto{\pgfqpoint{2.377130in}{0.739656in}}%
\pgfpathlineto{\pgfqpoint{2.376834in}{0.739656in}}%
\pgfpathlineto{\pgfqpoint{2.376538in}{0.739656in}}%
\pgfpathlineto{\pgfqpoint{2.376242in}{0.739656in}}%
\pgfpathlineto{\pgfqpoint{2.375946in}{0.739656in}}%
\pgfpathlineto{\pgfqpoint{2.375650in}{0.739656in}}%
\pgfpathlineto{\pgfqpoint{2.375354in}{0.739656in}}%
\pgfpathlineto{\pgfqpoint{2.375058in}{0.739656in}}%
\pgfpathlineto{\pgfqpoint{2.374762in}{0.739656in}}%
\pgfpathlineto{\pgfqpoint{2.374466in}{0.739656in}}%
\pgfpathlineto{\pgfqpoint{2.374170in}{0.739656in}}%
\pgfpathlineto{\pgfqpoint{2.373874in}{0.739656in}}%
\pgfpathlineto{\pgfqpoint{2.373578in}{0.739656in}}%
\pgfpathlineto{\pgfqpoint{2.373282in}{0.739656in}}%
\pgfpathlineto{\pgfqpoint{2.372986in}{0.739656in}}%
\pgfpathlineto{\pgfqpoint{2.372690in}{0.739656in}}%
\pgfpathlineto{\pgfqpoint{2.372394in}{0.739656in}}%
\pgfpathlineto{\pgfqpoint{2.372098in}{0.739656in}}%
\pgfpathlineto{\pgfqpoint{2.371802in}{0.739656in}}%
\pgfpathlineto{\pgfqpoint{2.371506in}{0.739656in}}%
\pgfpathlineto{\pgfqpoint{2.371210in}{0.739656in}}%
\pgfpathlineto{\pgfqpoint{2.370914in}{0.739656in}}%
\pgfpathlineto{\pgfqpoint{2.370618in}{0.739656in}}%
\pgfpathlineto{\pgfqpoint{2.370322in}{0.739656in}}%
\pgfpathlineto{\pgfqpoint{2.370026in}{0.739656in}}%
\pgfpathlineto{\pgfqpoint{2.369730in}{0.739656in}}%
\pgfpathlineto{\pgfqpoint{2.369434in}{0.739656in}}%
\pgfpathlineto{\pgfqpoint{2.369138in}{0.739656in}}%
\pgfpathlineto{\pgfqpoint{2.368842in}{0.739656in}}%
\pgfpathlineto{\pgfqpoint{2.368546in}{0.739656in}}%
\pgfpathlineto{\pgfqpoint{2.368250in}{0.739656in}}%
\pgfpathlineto{\pgfqpoint{2.367954in}{0.739656in}}%
\pgfpathlineto{\pgfqpoint{2.367658in}{0.739656in}}%
\pgfpathlineto{\pgfqpoint{2.367362in}{0.739656in}}%
\pgfpathlineto{\pgfqpoint{2.367066in}{0.739656in}}%
\pgfpathlineto{\pgfqpoint{2.366770in}{0.739656in}}%
\pgfpathlineto{\pgfqpoint{2.366474in}{0.739656in}}%
\pgfpathlineto{\pgfqpoint{2.366178in}{0.739656in}}%
\pgfpathlineto{\pgfqpoint{2.365882in}{0.739656in}}%
\pgfpathlineto{\pgfqpoint{2.365586in}{0.739656in}}%
\pgfpathlineto{\pgfqpoint{2.365290in}{0.739656in}}%
\pgfpathlineto{\pgfqpoint{2.364994in}{0.739656in}}%
\pgfpathlineto{\pgfqpoint{2.364698in}{0.739656in}}%
\pgfpathlineto{\pgfqpoint{2.364402in}{0.739656in}}%
\pgfpathlineto{\pgfqpoint{2.364106in}{0.739656in}}%
\pgfpathlineto{\pgfqpoint{2.363810in}{0.739656in}}%
\pgfpathlineto{\pgfqpoint{2.363514in}{0.739656in}}%
\pgfpathlineto{\pgfqpoint{2.363218in}{0.739656in}}%
\pgfpathlineto{\pgfqpoint{2.362922in}{0.739656in}}%
\pgfpathlineto{\pgfqpoint{2.362626in}{0.739656in}}%
\pgfpathlineto{\pgfqpoint{2.362330in}{0.739656in}}%
\pgfpathlineto{\pgfqpoint{2.362034in}{0.739656in}}%
\pgfpathlineto{\pgfqpoint{2.361738in}{0.739656in}}%
\pgfpathlineto{\pgfqpoint{2.361442in}{0.739656in}}%
\pgfpathlineto{\pgfqpoint{2.361146in}{0.739656in}}%
\pgfpathlineto{\pgfqpoint{2.360850in}{0.739656in}}%
\pgfpathlineto{\pgfqpoint{2.360554in}{0.739656in}}%
\pgfpathlineto{\pgfqpoint{2.360258in}{0.739656in}}%
\pgfpathlineto{\pgfqpoint{2.359962in}{0.739656in}}%
\pgfpathlineto{\pgfqpoint{2.359666in}{0.739656in}}%
\pgfpathlineto{\pgfqpoint{2.359370in}{0.739656in}}%
\pgfpathlineto{\pgfqpoint{2.359074in}{0.739656in}}%
\pgfpathlineto{\pgfqpoint{2.358778in}{0.739656in}}%
\pgfpathlineto{\pgfqpoint{2.358482in}{0.739656in}}%
\pgfpathlineto{\pgfqpoint{2.358186in}{0.739656in}}%
\pgfpathlineto{\pgfqpoint{2.357890in}{0.739656in}}%
\pgfpathlineto{\pgfqpoint{2.357594in}{0.739656in}}%
\pgfpathlineto{\pgfqpoint{2.357298in}{0.739656in}}%
\pgfpathlineto{\pgfqpoint{2.357002in}{0.739656in}}%
\pgfpathlineto{\pgfqpoint{2.356706in}{0.739656in}}%
\pgfpathlineto{\pgfqpoint{2.356410in}{0.739656in}}%
\pgfpathlineto{\pgfqpoint{2.356114in}{0.739656in}}%
\pgfpathlineto{\pgfqpoint{2.355818in}{0.739656in}}%
\pgfpathlineto{\pgfqpoint{2.355522in}{0.739656in}}%
\pgfpathlineto{\pgfqpoint{2.355226in}{0.739656in}}%
\pgfpathlineto{\pgfqpoint{2.354930in}{0.739656in}}%
\pgfpathlineto{\pgfqpoint{2.354634in}{0.739656in}}%
\pgfpathlineto{\pgfqpoint{2.354338in}{0.739656in}}%
\pgfpathlineto{\pgfqpoint{2.354042in}{0.739656in}}%
\pgfpathlineto{\pgfqpoint{2.353746in}{0.739656in}}%
\pgfpathlineto{\pgfqpoint{2.353450in}{0.739656in}}%
\pgfpathlineto{\pgfqpoint{2.353154in}{0.739656in}}%
\pgfpathlineto{\pgfqpoint{2.352858in}{0.739656in}}%
\pgfpathlineto{\pgfqpoint{2.352562in}{0.739656in}}%
\pgfpathlineto{\pgfqpoint{2.352266in}{0.739656in}}%
\pgfpathlineto{\pgfqpoint{2.351970in}{0.739656in}}%
\pgfpathlineto{\pgfqpoint{2.351674in}{0.739656in}}%
\pgfpathlineto{\pgfqpoint{2.351378in}{0.739656in}}%
\pgfpathlineto{\pgfqpoint{2.351082in}{0.739656in}}%
\pgfpathlineto{\pgfqpoint{2.350786in}{0.739656in}}%
\pgfpathlineto{\pgfqpoint{2.350490in}{0.739656in}}%
\pgfpathlineto{\pgfqpoint{2.350194in}{0.739656in}}%
\pgfpathlineto{\pgfqpoint{2.349898in}{0.739656in}}%
\pgfpathlineto{\pgfqpoint{2.349602in}{0.739656in}}%
\pgfpathlineto{\pgfqpoint{2.349306in}{0.739656in}}%
\pgfpathlineto{\pgfqpoint{2.349010in}{0.739656in}}%
\pgfpathlineto{\pgfqpoint{2.348714in}{0.739656in}}%
\pgfpathlineto{\pgfqpoint{2.348418in}{0.739656in}}%
\pgfpathlineto{\pgfqpoint{2.348122in}{0.739656in}}%
\pgfpathlineto{\pgfqpoint{2.347826in}{0.739656in}}%
\pgfpathlineto{\pgfqpoint{2.347530in}{0.739656in}}%
\pgfpathlineto{\pgfqpoint{2.347234in}{0.739656in}}%
\pgfpathlineto{\pgfqpoint{2.346938in}{0.739656in}}%
\pgfpathlineto{\pgfqpoint{2.346642in}{0.739656in}}%
\pgfpathlineto{\pgfqpoint{2.346346in}{0.739656in}}%
\pgfpathlineto{\pgfqpoint{2.346050in}{0.739656in}}%
\pgfpathlineto{\pgfqpoint{2.345754in}{0.739656in}}%
\pgfpathlineto{\pgfqpoint{2.345458in}{0.739656in}}%
\pgfpathlineto{\pgfqpoint{2.345162in}{0.739656in}}%
\pgfpathlineto{\pgfqpoint{2.344866in}{0.739656in}}%
\pgfpathlineto{\pgfqpoint{2.344570in}{0.739656in}}%
\pgfpathlineto{\pgfqpoint{2.344274in}{0.739656in}}%
\pgfpathlineto{\pgfqpoint{2.343978in}{0.739656in}}%
\pgfpathlineto{\pgfqpoint{2.343682in}{0.739656in}}%
\pgfpathlineto{\pgfqpoint{2.343386in}{0.739656in}}%
\pgfpathlineto{\pgfqpoint{2.343090in}{0.739656in}}%
\pgfpathlineto{\pgfqpoint{2.342794in}{0.739656in}}%
\pgfpathlineto{\pgfqpoint{2.342498in}{0.739656in}}%
\pgfpathlineto{\pgfqpoint{2.342202in}{0.739656in}}%
\pgfpathlineto{\pgfqpoint{2.341906in}{0.739656in}}%
\pgfpathlineto{\pgfqpoint{2.341610in}{0.739656in}}%
\pgfpathlineto{\pgfqpoint{2.341314in}{0.739656in}}%
\pgfpathlineto{\pgfqpoint{2.341018in}{0.739656in}}%
\pgfpathlineto{\pgfqpoint{2.340722in}{0.739656in}}%
\pgfpathlineto{\pgfqpoint{2.340426in}{0.739656in}}%
\pgfpathlineto{\pgfqpoint{2.340130in}{0.739656in}}%
\pgfpathlineto{\pgfqpoint{2.339834in}{0.739656in}}%
\pgfpathlineto{\pgfqpoint{2.339538in}{0.739656in}}%
\pgfpathlineto{\pgfqpoint{2.339242in}{0.739656in}}%
\pgfpathlineto{\pgfqpoint{2.338946in}{0.739656in}}%
\pgfpathlineto{\pgfqpoint{2.338650in}{0.739656in}}%
\pgfpathlineto{\pgfqpoint{2.338354in}{0.739656in}}%
\pgfpathlineto{\pgfqpoint{2.338058in}{0.739656in}}%
\pgfpathlineto{\pgfqpoint{2.337762in}{0.739656in}}%
\pgfpathlineto{\pgfqpoint{2.337466in}{0.739656in}}%
\pgfpathlineto{\pgfqpoint{2.337170in}{0.739656in}}%
\pgfpathlineto{\pgfqpoint{2.336874in}{0.739656in}}%
\pgfpathlineto{\pgfqpoint{2.336578in}{0.739656in}}%
\pgfpathlineto{\pgfqpoint{2.336282in}{0.739656in}}%
\pgfpathlineto{\pgfqpoint{2.335986in}{0.739656in}}%
\pgfpathlineto{\pgfqpoint{2.335690in}{0.739656in}}%
\pgfpathlineto{\pgfqpoint{2.335394in}{0.739656in}}%
\pgfpathlineto{\pgfqpoint{2.335098in}{0.739656in}}%
\pgfpathlineto{\pgfqpoint{2.334802in}{0.739656in}}%
\pgfpathlineto{\pgfqpoint{2.334506in}{0.739656in}}%
\pgfpathlineto{\pgfqpoint{2.334210in}{0.739656in}}%
\pgfpathlineto{\pgfqpoint{2.333914in}{0.739656in}}%
\pgfpathlineto{\pgfqpoint{2.333618in}{0.739656in}}%
\pgfpathlineto{\pgfqpoint{2.333321in}{0.739656in}}%
\pgfpathlineto{\pgfqpoint{2.333025in}{0.739656in}}%
\pgfpathlineto{\pgfqpoint{2.332729in}{0.739656in}}%
\pgfpathlineto{\pgfqpoint{2.332433in}{0.739656in}}%
\pgfpathlineto{\pgfqpoint{2.332137in}{0.739656in}}%
\pgfpathlineto{\pgfqpoint{2.331841in}{0.739656in}}%
\pgfpathlineto{\pgfqpoint{2.331545in}{0.739656in}}%
\pgfpathlineto{\pgfqpoint{2.331249in}{0.739656in}}%
\pgfpathlineto{\pgfqpoint{2.330953in}{0.739656in}}%
\pgfpathlineto{\pgfqpoint{2.330657in}{0.739656in}}%
\pgfpathlineto{\pgfqpoint{2.330361in}{0.739656in}}%
\pgfpathlineto{\pgfqpoint{2.330065in}{0.739656in}}%
\pgfpathlineto{\pgfqpoint{2.329769in}{0.739656in}}%
\pgfpathlineto{\pgfqpoint{2.329473in}{0.739656in}}%
\pgfpathlineto{\pgfqpoint{2.329177in}{0.739656in}}%
\pgfpathlineto{\pgfqpoint{2.328881in}{0.739656in}}%
\pgfpathlineto{\pgfqpoint{2.328585in}{0.739656in}}%
\pgfpathlineto{\pgfqpoint{2.328289in}{0.739656in}}%
\pgfpathlineto{\pgfqpoint{2.327993in}{0.739656in}}%
\pgfpathlineto{\pgfqpoint{2.327697in}{0.739656in}}%
\pgfpathlineto{\pgfqpoint{2.327401in}{0.739656in}}%
\pgfpathlineto{\pgfqpoint{2.327105in}{0.739656in}}%
\pgfpathlineto{\pgfqpoint{2.326809in}{0.739656in}}%
\pgfpathlineto{\pgfqpoint{2.326513in}{0.739656in}}%
\pgfpathlineto{\pgfqpoint{2.326217in}{0.739656in}}%
\pgfpathlineto{\pgfqpoint{2.325921in}{0.739656in}}%
\pgfpathlineto{\pgfqpoint{2.325625in}{0.739656in}}%
\pgfpathlineto{\pgfqpoint{2.325329in}{0.739656in}}%
\pgfpathlineto{\pgfqpoint{2.325033in}{0.739656in}}%
\pgfpathlineto{\pgfqpoint{2.324737in}{0.739656in}}%
\pgfpathlineto{\pgfqpoint{2.324441in}{0.739656in}}%
\pgfpathlineto{\pgfqpoint{2.324145in}{0.739656in}}%
\pgfpathlineto{\pgfqpoint{2.323849in}{0.739656in}}%
\pgfpathlineto{\pgfqpoint{2.323553in}{0.739656in}}%
\pgfpathlineto{\pgfqpoint{2.323257in}{0.739656in}}%
\pgfpathlineto{\pgfqpoint{2.322961in}{0.739656in}}%
\pgfpathlineto{\pgfqpoint{2.322665in}{0.739656in}}%
\pgfpathlineto{\pgfqpoint{2.322369in}{0.739656in}}%
\pgfpathlineto{\pgfqpoint{2.322073in}{0.739656in}}%
\pgfpathlineto{\pgfqpoint{2.321777in}{0.739656in}}%
\pgfpathlineto{\pgfqpoint{2.321481in}{0.739656in}}%
\pgfpathlineto{\pgfqpoint{2.321185in}{0.739656in}}%
\pgfpathlineto{\pgfqpoint{2.320889in}{0.739656in}}%
\pgfpathlineto{\pgfqpoint{2.320593in}{0.739656in}}%
\pgfpathlineto{\pgfqpoint{2.320297in}{0.739656in}}%
\pgfpathlineto{\pgfqpoint{2.320001in}{0.739656in}}%
\pgfpathlineto{\pgfqpoint{2.319705in}{0.739656in}}%
\pgfpathlineto{\pgfqpoint{2.319409in}{0.739656in}}%
\pgfpathlineto{\pgfqpoint{2.319113in}{0.739656in}}%
\pgfpathlineto{\pgfqpoint{2.318817in}{0.739656in}}%
\pgfpathlineto{\pgfqpoint{2.318521in}{0.739656in}}%
\pgfpathlineto{\pgfqpoint{2.318225in}{0.739656in}}%
\pgfpathlineto{\pgfqpoint{2.317929in}{0.739656in}}%
\pgfpathlineto{\pgfqpoint{2.317633in}{0.739656in}}%
\pgfpathlineto{\pgfqpoint{2.317337in}{0.739656in}}%
\pgfpathlineto{\pgfqpoint{2.317041in}{0.739656in}}%
\pgfpathlineto{\pgfqpoint{2.316745in}{0.739656in}}%
\pgfpathlineto{\pgfqpoint{2.316449in}{0.739656in}}%
\pgfpathlineto{\pgfqpoint{2.316153in}{0.739656in}}%
\pgfpathlineto{\pgfqpoint{2.315857in}{0.739656in}}%
\pgfpathlineto{\pgfqpoint{2.315561in}{0.739656in}}%
\pgfpathlineto{\pgfqpoint{2.315265in}{0.739656in}}%
\pgfpathlineto{\pgfqpoint{2.314969in}{0.739656in}}%
\pgfpathlineto{\pgfqpoint{2.314673in}{0.739656in}}%
\pgfpathlineto{\pgfqpoint{2.314377in}{0.739656in}}%
\pgfpathlineto{\pgfqpoint{2.314081in}{0.739656in}}%
\pgfpathlineto{\pgfqpoint{2.313785in}{0.739656in}}%
\pgfpathlineto{\pgfqpoint{2.313489in}{0.739656in}}%
\pgfpathlineto{\pgfqpoint{2.313193in}{0.739656in}}%
\pgfpathlineto{\pgfqpoint{2.312897in}{0.739656in}}%
\pgfpathlineto{\pgfqpoint{2.312601in}{0.739656in}}%
\pgfpathlineto{\pgfqpoint{2.312305in}{0.739656in}}%
\pgfpathlineto{\pgfqpoint{2.312009in}{0.739656in}}%
\pgfpathlineto{\pgfqpoint{2.311713in}{0.739656in}}%
\pgfpathlineto{\pgfqpoint{2.311417in}{0.739656in}}%
\pgfpathlineto{\pgfqpoint{2.311121in}{0.739656in}}%
\pgfpathlineto{\pgfqpoint{2.310825in}{0.739656in}}%
\pgfpathlineto{\pgfqpoint{2.310529in}{0.739656in}}%
\pgfpathlineto{\pgfqpoint{2.310233in}{0.739656in}}%
\pgfpathlineto{\pgfqpoint{2.309937in}{0.739656in}}%
\pgfpathlineto{\pgfqpoint{2.309641in}{0.739656in}}%
\pgfpathlineto{\pgfqpoint{2.309345in}{0.739656in}}%
\pgfpathlineto{\pgfqpoint{2.309049in}{0.739656in}}%
\pgfpathlineto{\pgfqpoint{2.308753in}{0.739656in}}%
\pgfpathlineto{\pgfqpoint{2.308457in}{0.739656in}}%
\pgfpathlineto{\pgfqpoint{2.308161in}{0.739656in}}%
\pgfpathlineto{\pgfqpoint{2.307865in}{0.739656in}}%
\pgfpathlineto{\pgfqpoint{2.307569in}{0.739656in}}%
\pgfpathlineto{\pgfqpoint{2.307273in}{0.739656in}}%
\pgfpathlineto{\pgfqpoint{2.306977in}{0.739656in}}%
\pgfpathlineto{\pgfqpoint{2.306681in}{0.739656in}}%
\pgfpathlineto{\pgfqpoint{2.306385in}{0.739656in}}%
\pgfpathlineto{\pgfqpoint{2.306089in}{0.739656in}}%
\pgfpathlineto{\pgfqpoint{2.305793in}{0.739656in}}%
\pgfpathlineto{\pgfqpoint{2.305497in}{0.739656in}}%
\pgfpathlineto{\pgfqpoint{2.305201in}{0.739656in}}%
\pgfpathlineto{\pgfqpoint{2.304905in}{0.739656in}}%
\pgfpathlineto{\pgfqpoint{2.304609in}{0.739656in}}%
\pgfpathlineto{\pgfqpoint{2.304313in}{0.739656in}}%
\pgfpathlineto{\pgfqpoint{2.304017in}{0.739656in}}%
\pgfpathlineto{\pgfqpoint{2.303721in}{0.739656in}}%
\pgfpathlineto{\pgfqpoint{2.303425in}{0.739656in}}%
\pgfpathlineto{\pgfqpoint{2.303129in}{0.739656in}}%
\pgfpathlineto{\pgfqpoint{2.302833in}{0.739656in}}%
\pgfpathlineto{\pgfqpoint{2.302537in}{0.739656in}}%
\pgfpathlineto{\pgfqpoint{2.302241in}{0.739656in}}%
\pgfpathlineto{\pgfqpoint{2.301945in}{0.739656in}}%
\pgfpathlineto{\pgfqpoint{2.301649in}{0.739656in}}%
\pgfpathlineto{\pgfqpoint{2.301353in}{0.739656in}}%
\pgfpathlineto{\pgfqpoint{2.301057in}{0.739656in}}%
\pgfpathlineto{\pgfqpoint{2.300761in}{0.739656in}}%
\pgfpathlineto{\pgfqpoint{2.300465in}{0.739656in}}%
\pgfpathlineto{\pgfqpoint{2.300169in}{0.739656in}}%
\pgfpathlineto{\pgfqpoint{2.299873in}{0.739656in}}%
\pgfpathlineto{\pgfqpoint{2.299577in}{0.739656in}}%
\pgfpathlineto{\pgfqpoint{2.299281in}{0.739656in}}%
\pgfpathlineto{\pgfqpoint{2.298985in}{0.739656in}}%
\pgfpathlineto{\pgfqpoint{2.298689in}{0.739656in}}%
\pgfpathlineto{\pgfqpoint{2.298393in}{0.739656in}}%
\pgfpathlineto{\pgfqpoint{2.298097in}{0.739656in}}%
\pgfpathlineto{\pgfqpoint{2.297801in}{0.739656in}}%
\pgfpathlineto{\pgfqpoint{2.297505in}{0.739656in}}%
\pgfpathlineto{\pgfqpoint{2.297209in}{0.739656in}}%
\pgfpathlineto{\pgfqpoint{2.296913in}{0.739656in}}%
\pgfpathlineto{\pgfqpoint{2.296617in}{0.739656in}}%
\pgfpathlineto{\pgfqpoint{2.296321in}{0.739656in}}%
\pgfpathlineto{\pgfqpoint{2.296025in}{0.739656in}}%
\pgfpathlineto{\pgfqpoint{2.295729in}{0.739656in}}%
\pgfpathlineto{\pgfqpoint{2.295433in}{0.739656in}}%
\pgfpathlineto{\pgfqpoint{2.295137in}{0.739656in}}%
\pgfpathlineto{\pgfqpoint{2.294841in}{0.739656in}}%
\pgfpathlineto{\pgfqpoint{2.294545in}{0.739656in}}%
\pgfpathlineto{\pgfqpoint{2.294249in}{0.739656in}}%
\pgfpathlineto{\pgfqpoint{2.293953in}{0.739656in}}%
\pgfpathlineto{\pgfqpoint{2.293657in}{0.739656in}}%
\pgfpathlineto{\pgfqpoint{2.293361in}{0.739656in}}%
\pgfpathlineto{\pgfqpoint{2.293065in}{0.739656in}}%
\pgfpathlineto{\pgfqpoint{2.292769in}{0.739656in}}%
\pgfpathlineto{\pgfqpoint{2.292473in}{0.739656in}}%
\pgfpathlineto{\pgfqpoint{2.292177in}{0.739656in}}%
\pgfpathlineto{\pgfqpoint{2.291881in}{0.739656in}}%
\pgfpathlineto{\pgfqpoint{2.291585in}{0.739656in}}%
\pgfpathlineto{\pgfqpoint{2.291289in}{0.739656in}}%
\pgfpathlineto{\pgfqpoint{2.290993in}{0.739656in}}%
\pgfpathlineto{\pgfqpoint{2.290697in}{0.739656in}}%
\pgfpathlineto{\pgfqpoint{2.290401in}{0.739656in}}%
\pgfpathlineto{\pgfqpoint{2.290105in}{0.739656in}}%
\pgfpathlineto{\pgfqpoint{2.289809in}{0.739656in}}%
\pgfpathlineto{\pgfqpoint{2.289513in}{0.739656in}}%
\pgfpathlineto{\pgfqpoint{2.289217in}{0.739656in}}%
\pgfpathlineto{\pgfqpoint{2.288921in}{0.739656in}}%
\pgfpathlineto{\pgfqpoint{2.288625in}{0.739656in}}%
\pgfpathlineto{\pgfqpoint{2.288329in}{0.739656in}}%
\pgfpathlineto{\pgfqpoint{2.288033in}{0.739656in}}%
\pgfpathlineto{\pgfqpoint{2.287737in}{0.739656in}}%
\pgfpathlineto{\pgfqpoint{2.287441in}{0.739656in}}%
\pgfpathlineto{\pgfqpoint{2.287145in}{0.739656in}}%
\pgfpathlineto{\pgfqpoint{2.286849in}{0.739656in}}%
\pgfpathlineto{\pgfqpoint{2.286553in}{0.739656in}}%
\pgfpathlineto{\pgfqpoint{2.286257in}{0.739656in}}%
\pgfpathlineto{\pgfqpoint{2.285961in}{0.739656in}}%
\pgfpathlineto{\pgfqpoint{2.285665in}{0.739656in}}%
\pgfpathlineto{\pgfqpoint{2.285369in}{0.739656in}}%
\pgfpathlineto{\pgfqpoint{2.285073in}{0.739656in}}%
\pgfpathlineto{\pgfqpoint{2.284777in}{0.739656in}}%
\pgfpathlineto{\pgfqpoint{2.284481in}{0.739656in}}%
\pgfpathlineto{\pgfqpoint{2.284185in}{0.739656in}}%
\pgfpathlineto{\pgfqpoint{2.283889in}{0.739656in}}%
\pgfpathlineto{\pgfqpoint{2.283593in}{0.739656in}}%
\pgfpathlineto{\pgfqpoint{2.283297in}{0.739656in}}%
\pgfpathlineto{\pgfqpoint{2.283001in}{0.739656in}}%
\pgfpathlineto{\pgfqpoint{2.282705in}{0.739656in}}%
\pgfpathlineto{\pgfqpoint{2.282409in}{0.739656in}}%
\pgfpathlineto{\pgfqpoint{2.282113in}{0.739656in}}%
\pgfpathlineto{\pgfqpoint{2.281817in}{0.739656in}}%
\pgfpathlineto{\pgfqpoint{2.281521in}{0.739656in}}%
\pgfpathlineto{\pgfqpoint{2.281225in}{0.739656in}}%
\pgfpathlineto{\pgfqpoint{2.280929in}{0.739656in}}%
\pgfpathlineto{\pgfqpoint{2.280633in}{0.739656in}}%
\pgfpathlineto{\pgfqpoint{2.280337in}{0.739656in}}%
\pgfpathlineto{\pgfqpoint{2.280041in}{0.739656in}}%
\pgfpathlineto{\pgfqpoint{2.279745in}{0.739656in}}%
\pgfpathlineto{\pgfqpoint{2.279449in}{0.739656in}}%
\pgfpathlineto{\pgfqpoint{2.279153in}{0.739656in}}%
\pgfpathlineto{\pgfqpoint{2.278857in}{0.739656in}}%
\pgfpathlineto{\pgfqpoint{2.278561in}{0.739656in}}%
\pgfpathlineto{\pgfqpoint{2.278265in}{0.739656in}}%
\pgfpathlineto{\pgfqpoint{2.277969in}{0.739656in}}%
\pgfpathlineto{\pgfqpoint{2.277673in}{0.739656in}}%
\pgfpathlineto{\pgfqpoint{2.277377in}{0.739656in}}%
\pgfpathlineto{\pgfqpoint{2.277081in}{0.739656in}}%
\pgfpathlineto{\pgfqpoint{2.276785in}{0.739656in}}%
\pgfpathlineto{\pgfqpoint{2.276489in}{0.739656in}}%
\pgfpathlineto{\pgfqpoint{2.276193in}{0.739656in}}%
\pgfpathlineto{\pgfqpoint{2.275897in}{0.739656in}}%
\pgfpathlineto{\pgfqpoint{2.275601in}{0.739656in}}%
\pgfpathlineto{\pgfqpoint{2.275305in}{0.739656in}}%
\pgfpathlineto{\pgfqpoint{2.275009in}{0.739656in}}%
\pgfpathlineto{\pgfqpoint{2.274713in}{0.739656in}}%
\pgfpathlineto{\pgfqpoint{2.274417in}{0.739656in}}%
\pgfpathlineto{\pgfqpoint{2.274121in}{0.739656in}}%
\pgfpathlineto{\pgfqpoint{2.273825in}{0.739656in}}%
\pgfpathlineto{\pgfqpoint{2.273529in}{0.739656in}}%
\pgfpathlineto{\pgfqpoint{2.273233in}{0.739656in}}%
\pgfpathlineto{\pgfqpoint{2.272937in}{0.739656in}}%
\pgfpathlineto{\pgfqpoint{2.272641in}{0.739656in}}%
\pgfpathlineto{\pgfqpoint{2.272345in}{0.739656in}}%
\pgfpathlineto{\pgfqpoint{2.272049in}{0.739656in}}%
\pgfpathlineto{\pgfqpoint{2.271753in}{0.739656in}}%
\pgfpathlineto{\pgfqpoint{2.271457in}{0.739656in}}%
\pgfpathlineto{\pgfqpoint{2.271161in}{0.739656in}}%
\pgfpathlineto{\pgfqpoint{2.270865in}{0.739656in}}%
\pgfpathlineto{\pgfqpoint{2.270569in}{0.739656in}}%
\pgfpathlineto{\pgfqpoint{2.270273in}{0.739656in}}%
\pgfpathlineto{\pgfqpoint{2.269977in}{0.739656in}}%
\pgfpathlineto{\pgfqpoint{2.269681in}{0.739656in}}%
\pgfpathlineto{\pgfqpoint{2.269385in}{0.739656in}}%
\pgfpathlineto{\pgfqpoint{2.269089in}{0.739656in}}%
\pgfpathlineto{\pgfqpoint{2.268793in}{0.739656in}}%
\pgfpathlineto{\pgfqpoint{2.268497in}{0.739656in}}%
\pgfpathlineto{\pgfqpoint{2.268201in}{0.739656in}}%
\pgfpathlineto{\pgfqpoint{2.267905in}{0.739656in}}%
\pgfpathlineto{\pgfqpoint{2.267609in}{0.739656in}}%
\pgfpathlineto{\pgfqpoint{2.267313in}{0.739656in}}%
\pgfpathlineto{\pgfqpoint{2.267017in}{0.739656in}}%
\pgfpathlineto{\pgfqpoint{2.266721in}{0.739656in}}%
\pgfpathlineto{\pgfqpoint{2.266425in}{0.739656in}}%
\pgfpathlineto{\pgfqpoint{2.266129in}{0.739656in}}%
\pgfpathlineto{\pgfqpoint{2.265832in}{0.739656in}}%
\pgfpathlineto{\pgfqpoint{2.265536in}{0.739656in}}%
\pgfpathlineto{\pgfqpoint{2.265240in}{0.739656in}}%
\pgfpathlineto{\pgfqpoint{2.264944in}{0.739656in}}%
\pgfpathlineto{\pgfqpoint{2.264648in}{0.739656in}}%
\pgfpathlineto{\pgfqpoint{2.264352in}{0.739656in}}%
\pgfpathlineto{\pgfqpoint{2.264056in}{0.739656in}}%
\pgfpathlineto{\pgfqpoint{2.263760in}{0.739656in}}%
\pgfpathlineto{\pgfqpoint{2.263464in}{0.739656in}}%
\pgfpathlineto{\pgfqpoint{2.263168in}{0.739656in}}%
\pgfpathlineto{\pgfqpoint{2.262872in}{0.739656in}}%
\pgfpathlineto{\pgfqpoint{2.262576in}{0.739656in}}%
\pgfpathlineto{\pgfqpoint{2.262280in}{0.739656in}}%
\pgfpathlineto{\pgfqpoint{2.261984in}{0.739656in}}%
\pgfpathlineto{\pgfqpoint{2.261688in}{0.739656in}}%
\pgfpathlineto{\pgfqpoint{2.261392in}{0.739656in}}%
\pgfpathlineto{\pgfqpoint{2.261096in}{0.739656in}}%
\pgfpathlineto{\pgfqpoint{2.260800in}{0.739656in}}%
\pgfpathlineto{\pgfqpoint{2.260504in}{0.739656in}}%
\pgfpathlineto{\pgfqpoint{2.260208in}{0.739656in}}%
\pgfpathlineto{\pgfqpoint{2.259912in}{0.739656in}}%
\pgfpathlineto{\pgfqpoint{2.259616in}{0.739656in}}%
\pgfpathlineto{\pgfqpoint{2.259320in}{0.739656in}}%
\pgfpathlineto{\pgfqpoint{2.259024in}{0.739656in}}%
\pgfpathlineto{\pgfqpoint{2.258728in}{0.739656in}}%
\pgfpathlineto{\pgfqpoint{2.258432in}{0.739656in}}%
\pgfpathlineto{\pgfqpoint{2.258136in}{0.739656in}}%
\pgfpathlineto{\pgfqpoint{2.257840in}{0.739656in}}%
\pgfpathlineto{\pgfqpoint{2.257544in}{0.739656in}}%
\pgfpathlineto{\pgfqpoint{2.257248in}{0.739656in}}%
\pgfpathlineto{\pgfqpoint{2.256952in}{0.739656in}}%
\pgfpathlineto{\pgfqpoint{2.256656in}{0.739656in}}%
\pgfpathlineto{\pgfqpoint{2.256360in}{0.739656in}}%
\pgfpathlineto{\pgfqpoint{2.256064in}{0.739656in}}%
\pgfpathlineto{\pgfqpoint{2.255768in}{0.739656in}}%
\pgfpathlineto{\pgfqpoint{2.255472in}{0.739656in}}%
\pgfpathlineto{\pgfqpoint{2.255176in}{0.739656in}}%
\pgfpathlineto{\pgfqpoint{2.254880in}{0.739656in}}%
\pgfpathlineto{\pgfqpoint{2.254584in}{0.739656in}}%
\pgfpathlineto{\pgfqpoint{2.254288in}{0.739656in}}%
\pgfpathlineto{\pgfqpoint{2.253992in}{0.739656in}}%
\pgfpathlineto{\pgfqpoint{2.253696in}{0.739656in}}%
\pgfpathlineto{\pgfqpoint{2.253400in}{0.739656in}}%
\pgfpathlineto{\pgfqpoint{2.253104in}{0.739656in}}%
\pgfpathlineto{\pgfqpoint{2.252808in}{0.739656in}}%
\pgfpathlineto{\pgfqpoint{2.252512in}{0.739656in}}%
\pgfpathlineto{\pgfqpoint{2.252216in}{0.739656in}}%
\pgfpathlineto{\pgfqpoint{2.251920in}{0.739656in}}%
\pgfpathlineto{\pgfqpoint{2.251624in}{0.739656in}}%
\pgfpathlineto{\pgfqpoint{2.251328in}{0.739656in}}%
\pgfpathlineto{\pgfqpoint{2.251032in}{0.739656in}}%
\pgfpathlineto{\pgfqpoint{2.250736in}{0.739656in}}%
\pgfpathlineto{\pgfqpoint{2.250440in}{0.739656in}}%
\pgfpathlineto{\pgfqpoint{2.250144in}{0.739656in}}%
\pgfpathlineto{\pgfqpoint{2.249848in}{0.739656in}}%
\pgfpathlineto{\pgfqpoint{2.249552in}{0.739656in}}%
\pgfpathlineto{\pgfqpoint{2.249256in}{0.739656in}}%
\pgfpathlineto{\pgfqpoint{2.248960in}{0.739656in}}%
\pgfpathlineto{\pgfqpoint{2.248664in}{0.739656in}}%
\pgfpathlineto{\pgfqpoint{2.248368in}{0.739656in}}%
\pgfpathlineto{\pgfqpoint{2.248072in}{0.739656in}}%
\pgfpathlineto{\pgfqpoint{2.247776in}{0.739656in}}%
\pgfpathlineto{\pgfqpoint{2.247480in}{0.739656in}}%
\pgfpathlineto{\pgfqpoint{2.247184in}{0.739656in}}%
\pgfpathlineto{\pgfqpoint{2.246888in}{0.739656in}}%
\pgfpathlineto{\pgfqpoint{2.246592in}{0.739656in}}%
\pgfpathlineto{\pgfqpoint{2.246296in}{0.739656in}}%
\pgfpathlineto{\pgfqpoint{2.246000in}{0.739656in}}%
\pgfpathlineto{\pgfqpoint{2.245704in}{0.739656in}}%
\pgfpathlineto{\pgfqpoint{2.245408in}{0.739656in}}%
\pgfpathlineto{\pgfqpoint{2.245112in}{0.739656in}}%
\pgfpathlineto{\pgfqpoint{2.244816in}{0.739656in}}%
\pgfpathlineto{\pgfqpoint{2.244520in}{0.739656in}}%
\pgfpathlineto{\pgfqpoint{2.244224in}{0.739656in}}%
\pgfpathlineto{\pgfqpoint{2.243928in}{0.739656in}}%
\pgfpathlineto{\pgfqpoint{2.243632in}{0.739656in}}%
\pgfpathlineto{\pgfqpoint{2.243336in}{0.739656in}}%
\pgfpathlineto{\pgfqpoint{2.243040in}{0.739656in}}%
\pgfpathlineto{\pgfqpoint{2.242744in}{0.739656in}}%
\pgfpathlineto{\pgfqpoint{2.242448in}{0.739656in}}%
\pgfpathlineto{\pgfqpoint{2.242152in}{0.739656in}}%
\pgfpathlineto{\pgfqpoint{2.241856in}{0.739656in}}%
\pgfpathlineto{\pgfqpoint{2.241560in}{0.739656in}}%
\pgfpathlineto{\pgfqpoint{2.241264in}{0.739656in}}%
\pgfpathlineto{\pgfqpoint{2.240968in}{0.739656in}}%
\pgfpathlineto{\pgfqpoint{2.240672in}{0.739656in}}%
\pgfpathlineto{\pgfqpoint{2.240376in}{0.739656in}}%
\pgfpathlineto{\pgfqpoint{2.240080in}{0.739656in}}%
\pgfpathlineto{\pgfqpoint{2.239784in}{0.739656in}}%
\pgfpathlineto{\pgfqpoint{2.239488in}{0.739656in}}%
\pgfpathlineto{\pgfqpoint{2.239192in}{0.739656in}}%
\pgfpathlineto{\pgfqpoint{2.238896in}{0.739656in}}%
\pgfpathlineto{\pgfqpoint{2.238600in}{0.739656in}}%
\pgfpathlineto{\pgfqpoint{2.238304in}{0.739656in}}%
\pgfpathlineto{\pgfqpoint{2.238008in}{0.739656in}}%
\pgfpathlineto{\pgfqpoint{2.237712in}{0.739656in}}%
\pgfpathlineto{\pgfqpoint{2.237416in}{0.739656in}}%
\pgfpathlineto{\pgfqpoint{2.237120in}{0.739656in}}%
\pgfpathlineto{\pgfqpoint{2.236824in}{0.739656in}}%
\pgfpathlineto{\pgfqpoint{2.236528in}{0.739656in}}%
\pgfpathlineto{\pgfqpoint{2.236232in}{0.739656in}}%
\pgfpathlineto{\pgfqpoint{2.235936in}{0.739656in}}%
\pgfpathlineto{\pgfqpoint{2.235640in}{0.739656in}}%
\pgfpathlineto{\pgfqpoint{2.235344in}{0.739656in}}%
\pgfpathlineto{\pgfqpoint{2.235048in}{0.739656in}}%
\pgfpathlineto{\pgfqpoint{2.234752in}{0.739656in}}%
\pgfpathlineto{\pgfqpoint{2.234456in}{0.739656in}}%
\pgfpathlineto{\pgfqpoint{2.234160in}{0.739656in}}%
\pgfpathlineto{\pgfqpoint{2.233864in}{0.739656in}}%
\pgfpathlineto{\pgfqpoint{2.233568in}{0.739656in}}%
\pgfpathlineto{\pgfqpoint{2.233272in}{0.739656in}}%
\pgfpathlineto{\pgfqpoint{2.232976in}{0.739656in}}%
\pgfpathlineto{\pgfqpoint{2.232680in}{0.739656in}}%
\pgfpathlineto{\pgfqpoint{2.232384in}{0.739656in}}%
\pgfpathlineto{\pgfqpoint{2.232088in}{0.739656in}}%
\pgfpathlineto{\pgfqpoint{2.231792in}{0.739656in}}%
\pgfpathlineto{\pgfqpoint{2.231496in}{0.739656in}}%
\pgfpathlineto{\pgfqpoint{2.231200in}{0.739656in}}%
\pgfpathlineto{\pgfqpoint{2.230904in}{0.739656in}}%
\pgfpathlineto{\pgfqpoint{2.230608in}{0.739656in}}%
\pgfpathlineto{\pgfqpoint{2.230312in}{0.739656in}}%
\pgfpathlineto{\pgfqpoint{2.230016in}{0.739656in}}%
\pgfpathlineto{\pgfqpoint{2.229720in}{0.739656in}}%
\pgfpathlineto{\pgfqpoint{2.229424in}{0.739656in}}%
\pgfpathlineto{\pgfqpoint{2.229128in}{0.739656in}}%
\pgfpathlineto{\pgfqpoint{2.228832in}{0.739656in}}%
\pgfpathlineto{\pgfqpoint{2.228536in}{0.739656in}}%
\pgfpathlineto{\pgfqpoint{2.228240in}{0.739656in}}%
\pgfpathlineto{\pgfqpoint{2.227944in}{0.739656in}}%
\pgfpathlineto{\pgfqpoint{2.227648in}{0.739656in}}%
\pgfpathlineto{\pgfqpoint{2.227352in}{0.739656in}}%
\pgfpathlineto{\pgfqpoint{2.227056in}{0.739656in}}%
\pgfpathlineto{\pgfqpoint{2.226760in}{0.739656in}}%
\pgfpathlineto{\pgfqpoint{2.226464in}{0.739656in}}%
\pgfpathlineto{\pgfqpoint{2.226168in}{0.739656in}}%
\pgfpathlineto{\pgfqpoint{2.225872in}{0.739656in}}%
\pgfpathlineto{\pgfqpoint{2.225576in}{0.739656in}}%
\pgfpathlineto{\pgfqpoint{2.225280in}{0.739656in}}%
\pgfpathlineto{\pgfqpoint{2.224984in}{0.739656in}}%
\pgfpathlineto{\pgfqpoint{2.224688in}{0.739656in}}%
\pgfpathlineto{\pgfqpoint{2.224392in}{0.739656in}}%
\pgfpathlineto{\pgfqpoint{2.224096in}{0.739656in}}%
\pgfpathlineto{\pgfqpoint{2.223800in}{0.739656in}}%
\pgfpathlineto{\pgfqpoint{2.223504in}{0.739656in}}%
\pgfpathlineto{\pgfqpoint{2.223208in}{0.739656in}}%
\pgfpathlineto{\pgfqpoint{2.222912in}{0.739656in}}%
\pgfpathlineto{\pgfqpoint{2.222616in}{0.739656in}}%
\pgfpathlineto{\pgfqpoint{2.222320in}{0.739656in}}%
\pgfpathlineto{\pgfqpoint{2.222024in}{0.739656in}}%
\pgfpathlineto{\pgfqpoint{2.221728in}{0.739656in}}%
\pgfpathlineto{\pgfqpoint{2.221432in}{0.739656in}}%
\pgfpathlineto{\pgfqpoint{2.221136in}{0.739656in}}%
\pgfpathlineto{\pgfqpoint{2.220840in}{0.739656in}}%
\pgfpathlineto{\pgfqpoint{2.220544in}{0.739656in}}%
\pgfpathlineto{\pgfqpoint{2.220248in}{0.739656in}}%
\pgfpathlineto{\pgfqpoint{2.219952in}{0.739656in}}%
\pgfpathlineto{\pgfqpoint{2.219656in}{0.739656in}}%
\pgfpathlineto{\pgfqpoint{2.219360in}{0.739656in}}%
\pgfpathlineto{\pgfqpoint{2.219064in}{0.739656in}}%
\pgfpathlineto{\pgfqpoint{2.218768in}{0.739656in}}%
\pgfpathlineto{\pgfqpoint{2.218472in}{0.739656in}}%
\pgfpathlineto{\pgfqpoint{2.218176in}{0.739656in}}%
\pgfpathlineto{\pgfqpoint{2.217880in}{0.739656in}}%
\pgfpathlineto{\pgfqpoint{2.217584in}{0.739656in}}%
\pgfpathlineto{\pgfqpoint{2.217288in}{0.739656in}}%
\pgfpathlineto{\pgfqpoint{2.216992in}{0.739656in}}%
\pgfpathlineto{\pgfqpoint{2.216696in}{0.739656in}}%
\pgfpathlineto{\pgfqpoint{2.216400in}{0.739656in}}%
\pgfpathlineto{\pgfqpoint{2.216104in}{0.739656in}}%
\pgfpathlineto{\pgfqpoint{2.215808in}{0.739656in}}%
\pgfpathlineto{\pgfqpoint{2.215512in}{0.739656in}}%
\pgfpathlineto{\pgfqpoint{2.215216in}{0.739656in}}%
\pgfpathlineto{\pgfqpoint{2.214920in}{0.739656in}}%
\pgfpathlineto{\pgfqpoint{2.214624in}{0.739656in}}%
\pgfpathlineto{\pgfqpoint{2.214328in}{0.739656in}}%
\pgfpathlineto{\pgfqpoint{2.214032in}{0.739656in}}%
\pgfpathlineto{\pgfqpoint{2.213736in}{0.739656in}}%
\pgfpathlineto{\pgfqpoint{2.213440in}{0.739656in}}%
\pgfpathlineto{\pgfqpoint{2.213144in}{0.739656in}}%
\pgfpathlineto{\pgfqpoint{2.212848in}{0.739656in}}%
\pgfpathlineto{\pgfqpoint{2.212552in}{0.739656in}}%
\pgfpathlineto{\pgfqpoint{2.212256in}{0.739656in}}%
\pgfpathlineto{\pgfqpoint{2.211960in}{0.739656in}}%
\pgfpathlineto{\pgfqpoint{2.211664in}{0.739656in}}%
\pgfpathlineto{\pgfqpoint{2.211368in}{0.739656in}}%
\pgfpathlineto{\pgfqpoint{2.211072in}{0.739656in}}%
\pgfpathlineto{\pgfqpoint{2.210776in}{0.739656in}}%
\pgfpathlineto{\pgfqpoint{2.210480in}{0.739656in}}%
\pgfpathlineto{\pgfqpoint{2.210184in}{0.739656in}}%
\pgfpathlineto{\pgfqpoint{2.209888in}{0.739656in}}%
\pgfpathlineto{\pgfqpoint{2.209592in}{0.739656in}}%
\pgfpathlineto{\pgfqpoint{2.209296in}{0.739656in}}%
\pgfpathlineto{\pgfqpoint{2.209000in}{0.739656in}}%
\pgfpathlineto{\pgfqpoint{2.208704in}{0.739656in}}%
\pgfpathlineto{\pgfqpoint{2.208408in}{0.739656in}}%
\pgfpathlineto{\pgfqpoint{2.208112in}{0.739656in}}%
\pgfpathlineto{\pgfqpoint{2.207816in}{0.739656in}}%
\pgfpathlineto{\pgfqpoint{2.207520in}{0.739656in}}%
\pgfpathlineto{\pgfqpoint{2.207224in}{0.739656in}}%
\pgfpathlineto{\pgfqpoint{2.206928in}{0.739656in}}%
\pgfpathlineto{\pgfqpoint{2.206632in}{0.739656in}}%
\pgfpathlineto{\pgfqpoint{2.206336in}{0.739656in}}%
\pgfpathlineto{\pgfqpoint{2.206040in}{0.739656in}}%
\pgfpathlineto{\pgfqpoint{2.205744in}{0.739656in}}%
\pgfpathlineto{\pgfqpoint{2.205448in}{0.739656in}}%
\pgfpathlineto{\pgfqpoint{2.205152in}{0.739656in}}%
\pgfpathlineto{\pgfqpoint{2.204856in}{0.739656in}}%
\pgfpathlineto{\pgfqpoint{2.204560in}{0.739656in}}%
\pgfpathlineto{\pgfqpoint{2.204264in}{0.739656in}}%
\pgfpathlineto{\pgfqpoint{2.203968in}{0.739656in}}%
\pgfpathlineto{\pgfqpoint{2.203672in}{0.739656in}}%
\pgfpathlineto{\pgfqpoint{2.203376in}{0.739656in}}%
\pgfpathlineto{\pgfqpoint{2.203080in}{0.739656in}}%
\pgfpathlineto{\pgfqpoint{2.202784in}{0.739656in}}%
\pgfpathlineto{\pgfqpoint{2.202488in}{0.739656in}}%
\pgfpathlineto{\pgfqpoint{2.202192in}{0.739656in}}%
\pgfpathlineto{\pgfqpoint{2.201896in}{0.739656in}}%
\pgfpathlineto{\pgfqpoint{2.201600in}{0.739656in}}%
\pgfpathlineto{\pgfqpoint{2.201304in}{0.739656in}}%
\pgfpathlineto{\pgfqpoint{2.201008in}{0.739656in}}%
\pgfpathlineto{\pgfqpoint{2.200712in}{0.739656in}}%
\pgfpathlineto{\pgfqpoint{2.200416in}{0.739656in}}%
\pgfpathlineto{\pgfqpoint{2.200120in}{0.739656in}}%
\pgfpathlineto{\pgfqpoint{2.199824in}{0.739656in}}%
\pgfpathlineto{\pgfqpoint{2.199528in}{0.739656in}}%
\pgfpathlineto{\pgfqpoint{2.199232in}{0.739656in}}%
\pgfpathlineto{\pgfqpoint{2.198936in}{0.739656in}}%
\pgfpathlineto{\pgfqpoint{2.198640in}{0.739656in}}%
\pgfpathlineto{\pgfqpoint{2.198343in}{0.739656in}}%
\pgfpathlineto{\pgfqpoint{2.198047in}{0.739656in}}%
\pgfpathlineto{\pgfqpoint{2.197751in}{0.739656in}}%
\pgfpathlineto{\pgfqpoint{2.197455in}{0.739656in}}%
\pgfpathlineto{\pgfqpoint{2.197159in}{0.739656in}}%
\pgfpathlineto{\pgfqpoint{2.196863in}{0.739656in}}%
\pgfpathlineto{\pgfqpoint{2.196567in}{0.739656in}}%
\pgfpathlineto{\pgfqpoint{2.196271in}{0.739656in}}%
\pgfpathlineto{\pgfqpoint{2.195975in}{0.739656in}}%
\pgfpathlineto{\pgfqpoint{2.195679in}{0.739656in}}%
\pgfpathlineto{\pgfqpoint{2.195383in}{0.739656in}}%
\pgfpathlineto{\pgfqpoint{2.195087in}{0.739656in}}%
\pgfpathlineto{\pgfqpoint{2.194791in}{0.739656in}}%
\pgfpathlineto{\pgfqpoint{2.194495in}{0.739656in}}%
\pgfpathlineto{\pgfqpoint{2.194199in}{0.739656in}}%
\pgfpathlineto{\pgfqpoint{2.193903in}{0.739656in}}%
\pgfpathlineto{\pgfqpoint{2.193607in}{0.739656in}}%
\pgfpathlineto{\pgfqpoint{2.193311in}{0.739656in}}%
\pgfpathlineto{\pgfqpoint{2.193015in}{0.739656in}}%
\pgfpathlineto{\pgfqpoint{2.192719in}{0.739656in}}%
\pgfpathlineto{\pgfqpoint{2.192423in}{0.739656in}}%
\pgfpathlineto{\pgfqpoint{2.192127in}{0.739656in}}%
\pgfpathlineto{\pgfqpoint{2.191831in}{0.739656in}}%
\pgfpathlineto{\pgfqpoint{2.191535in}{0.739656in}}%
\pgfpathlineto{\pgfqpoint{2.191239in}{0.739656in}}%
\pgfpathlineto{\pgfqpoint{2.190943in}{0.739656in}}%
\pgfpathlineto{\pgfqpoint{2.190647in}{0.739656in}}%
\pgfpathlineto{\pgfqpoint{2.190351in}{0.739656in}}%
\pgfpathlineto{\pgfqpoint{2.190055in}{0.739656in}}%
\pgfpathlineto{\pgfqpoint{2.189759in}{0.739656in}}%
\pgfpathlineto{\pgfqpoint{2.189463in}{0.739656in}}%
\pgfpathlineto{\pgfqpoint{2.189167in}{0.739656in}}%
\pgfpathlineto{\pgfqpoint{2.188871in}{0.739656in}}%
\pgfpathlineto{\pgfqpoint{2.188575in}{0.739656in}}%
\pgfpathlineto{\pgfqpoint{2.188279in}{0.739656in}}%
\pgfpathlineto{\pgfqpoint{2.187983in}{0.739656in}}%
\pgfpathlineto{\pgfqpoint{2.187687in}{0.739656in}}%
\pgfpathlineto{\pgfqpoint{2.187391in}{0.739656in}}%
\pgfpathlineto{\pgfqpoint{2.187095in}{0.739656in}}%
\pgfpathlineto{\pgfqpoint{2.186799in}{0.739656in}}%
\pgfpathlineto{\pgfqpoint{2.186503in}{0.739656in}}%
\pgfpathlineto{\pgfqpoint{2.186207in}{0.739656in}}%
\pgfpathlineto{\pgfqpoint{2.185911in}{0.739656in}}%
\pgfpathlineto{\pgfqpoint{2.185615in}{0.739656in}}%
\pgfpathlineto{\pgfqpoint{2.185319in}{0.739656in}}%
\pgfpathlineto{\pgfqpoint{2.185023in}{0.739656in}}%
\pgfpathlineto{\pgfqpoint{2.184727in}{0.739656in}}%
\pgfpathlineto{\pgfqpoint{2.184431in}{0.739656in}}%
\pgfpathlineto{\pgfqpoint{2.184135in}{0.739656in}}%
\pgfpathlineto{\pgfqpoint{2.183839in}{0.739656in}}%
\pgfpathlineto{\pgfqpoint{2.183543in}{0.739656in}}%
\pgfpathlineto{\pgfqpoint{2.183247in}{0.739656in}}%
\pgfpathlineto{\pgfqpoint{2.182951in}{0.739656in}}%
\pgfpathlineto{\pgfqpoint{2.182655in}{0.739656in}}%
\pgfpathlineto{\pgfqpoint{2.182359in}{0.739656in}}%
\pgfpathlineto{\pgfqpoint{2.182063in}{0.739656in}}%
\pgfpathlineto{\pgfqpoint{2.181767in}{0.739656in}}%
\pgfpathlineto{\pgfqpoint{2.181471in}{0.739656in}}%
\pgfpathlineto{\pgfqpoint{2.181175in}{0.739656in}}%
\pgfpathlineto{\pgfqpoint{2.180879in}{0.739656in}}%
\pgfpathlineto{\pgfqpoint{2.180583in}{0.739656in}}%
\pgfpathlineto{\pgfqpoint{2.180287in}{0.739656in}}%
\pgfpathlineto{\pgfqpoint{2.179991in}{0.739656in}}%
\pgfpathlineto{\pgfqpoint{2.179695in}{0.739656in}}%
\pgfpathlineto{\pgfqpoint{2.179399in}{0.739656in}}%
\pgfpathlineto{\pgfqpoint{2.179103in}{0.739656in}}%
\pgfpathlineto{\pgfqpoint{2.178807in}{0.739656in}}%
\pgfpathlineto{\pgfqpoint{2.178511in}{0.739656in}}%
\pgfpathlineto{\pgfqpoint{2.178215in}{0.739656in}}%
\pgfpathlineto{\pgfqpoint{2.177919in}{0.739656in}}%
\pgfpathlineto{\pgfqpoint{2.177623in}{0.739656in}}%
\pgfpathlineto{\pgfqpoint{2.177327in}{0.739656in}}%
\pgfpathlineto{\pgfqpoint{2.177031in}{0.739656in}}%
\pgfpathlineto{\pgfqpoint{2.176735in}{0.739656in}}%
\pgfpathlineto{\pgfqpoint{2.176439in}{0.739656in}}%
\pgfpathlineto{\pgfqpoint{2.176143in}{0.739656in}}%
\pgfpathlineto{\pgfqpoint{2.175847in}{0.739656in}}%
\pgfpathlineto{\pgfqpoint{2.175551in}{0.739656in}}%
\pgfpathlineto{\pgfqpoint{2.175255in}{0.739656in}}%
\pgfpathlineto{\pgfqpoint{2.174959in}{0.739656in}}%
\pgfpathlineto{\pgfqpoint{2.174663in}{0.739656in}}%
\pgfpathlineto{\pgfqpoint{2.174367in}{0.739656in}}%
\pgfpathlineto{\pgfqpoint{2.174071in}{0.739656in}}%
\pgfpathlineto{\pgfqpoint{2.173775in}{0.739656in}}%
\pgfpathlineto{\pgfqpoint{2.173479in}{0.739656in}}%
\pgfpathlineto{\pgfqpoint{2.173183in}{0.739656in}}%
\pgfpathlineto{\pgfqpoint{2.172887in}{0.739656in}}%
\pgfpathlineto{\pgfqpoint{2.172591in}{0.739656in}}%
\pgfpathlineto{\pgfqpoint{2.172295in}{0.739656in}}%
\pgfpathlineto{\pgfqpoint{2.171999in}{0.739656in}}%
\pgfpathlineto{\pgfqpoint{2.171703in}{0.739656in}}%
\pgfpathlineto{\pgfqpoint{2.171407in}{0.739656in}}%
\pgfpathlineto{\pgfqpoint{2.171111in}{0.739656in}}%
\pgfpathlineto{\pgfqpoint{2.170815in}{0.739656in}}%
\pgfpathlineto{\pgfqpoint{2.170519in}{0.739656in}}%
\pgfpathlineto{\pgfqpoint{2.170223in}{0.739656in}}%
\pgfpathlineto{\pgfqpoint{2.169927in}{0.739656in}}%
\pgfpathlineto{\pgfqpoint{2.169631in}{0.739656in}}%
\pgfpathlineto{\pgfqpoint{2.169335in}{0.739656in}}%
\pgfpathlineto{\pgfqpoint{2.169039in}{0.739656in}}%
\pgfpathlineto{\pgfqpoint{2.168743in}{0.739656in}}%
\pgfpathlineto{\pgfqpoint{2.168447in}{0.739656in}}%
\pgfpathlineto{\pgfqpoint{2.168151in}{0.739656in}}%
\pgfpathlineto{\pgfqpoint{2.167855in}{0.739656in}}%
\pgfpathlineto{\pgfqpoint{2.167559in}{0.739656in}}%
\pgfpathlineto{\pgfqpoint{2.167263in}{0.739656in}}%
\pgfpathlineto{\pgfqpoint{2.166967in}{0.739656in}}%
\pgfpathlineto{\pgfqpoint{2.166671in}{0.739656in}}%
\pgfpathlineto{\pgfqpoint{2.166375in}{0.739656in}}%
\pgfpathlineto{\pgfqpoint{2.166079in}{0.739656in}}%
\pgfpathlineto{\pgfqpoint{2.165783in}{0.739656in}}%
\pgfpathlineto{\pgfqpoint{2.165487in}{0.739656in}}%
\pgfpathlineto{\pgfqpoint{2.165191in}{0.739656in}}%
\pgfpathlineto{\pgfqpoint{2.164895in}{0.739656in}}%
\pgfpathlineto{\pgfqpoint{2.164599in}{0.739656in}}%
\pgfpathlineto{\pgfqpoint{2.164303in}{0.739656in}}%
\pgfpathlineto{\pgfqpoint{2.164007in}{0.739656in}}%
\pgfpathlineto{\pgfqpoint{2.163711in}{0.739656in}}%
\pgfpathlineto{\pgfqpoint{2.163415in}{0.739656in}}%
\pgfpathlineto{\pgfqpoint{2.163119in}{0.739656in}}%
\pgfpathlineto{\pgfqpoint{2.162823in}{0.739656in}}%
\pgfpathlineto{\pgfqpoint{2.162527in}{0.739656in}}%
\pgfpathlineto{\pgfqpoint{2.162231in}{0.739656in}}%
\pgfpathlineto{\pgfqpoint{2.161935in}{0.739656in}}%
\pgfpathlineto{\pgfqpoint{2.161639in}{0.739656in}}%
\pgfpathlineto{\pgfqpoint{2.161343in}{0.739656in}}%
\pgfpathlineto{\pgfqpoint{2.161047in}{0.739656in}}%
\pgfpathlineto{\pgfqpoint{2.160751in}{0.739656in}}%
\pgfpathlineto{\pgfqpoint{2.160455in}{0.739656in}}%
\pgfpathlineto{\pgfqpoint{2.160159in}{0.739656in}}%
\pgfpathlineto{\pgfqpoint{2.159863in}{0.739656in}}%
\pgfpathlineto{\pgfqpoint{2.159567in}{0.739656in}}%
\pgfpathlineto{\pgfqpoint{2.159271in}{0.739656in}}%
\pgfpathlineto{\pgfqpoint{2.158975in}{0.739656in}}%
\pgfpathlineto{\pgfqpoint{2.158679in}{0.739656in}}%
\pgfpathlineto{\pgfqpoint{2.158383in}{0.739656in}}%
\pgfpathlineto{\pgfqpoint{2.158087in}{0.739656in}}%
\pgfpathlineto{\pgfqpoint{2.157791in}{0.739656in}}%
\pgfpathlineto{\pgfqpoint{2.157495in}{0.739656in}}%
\pgfpathlineto{\pgfqpoint{2.157199in}{0.739656in}}%
\pgfpathlineto{\pgfqpoint{2.156903in}{0.739656in}}%
\pgfpathlineto{\pgfqpoint{2.156607in}{0.739656in}}%
\pgfpathlineto{\pgfqpoint{2.156311in}{0.739656in}}%
\pgfpathlineto{\pgfqpoint{2.156015in}{0.739656in}}%
\pgfpathlineto{\pgfqpoint{2.155719in}{0.739656in}}%
\pgfpathlineto{\pgfqpoint{2.155423in}{0.739656in}}%
\pgfpathlineto{\pgfqpoint{2.155127in}{0.739656in}}%
\pgfpathlineto{\pgfqpoint{2.154831in}{0.739656in}}%
\pgfpathlineto{\pgfqpoint{2.154535in}{0.739656in}}%
\pgfpathlineto{\pgfqpoint{2.154239in}{0.739656in}}%
\pgfpathlineto{\pgfqpoint{2.153943in}{0.739656in}}%
\pgfpathlineto{\pgfqpoint{2.153647in}{0.739656in}}%
\pgfpathlineto{\pgfqpoint{2.153351in}{0.739656in}}%
\pgfpathlineto{\pgfqpoint{2.153055in}{0.739656in}}%
\pgfpathlineto{\pgfqpoint{2.152759in}{0.739656in}}%
\pgfpathlineto{\pgfqpoint{2.152463in}{0.739656in}}%
\pgfpathlineto{\pgfqpoint{2.152167in}{0.739656in}}%
\pgfpathlineto{\pgfqpoint{2.151871in}{0.739656in}}%
\pgfpathlineto{\pgfqpoint{2.151575in}{0.739656in}}%
\pgfpathlineto{\pgfqpoint{2.151279in}{0.739656in}}%
\pgfpathlineto{\pgfqpoint{2.150983in}{0.739656in}}%
\pgfpathlineto{\pgfqpoint{2.150687in}{0.739656in}}%
\pgfpathlineto{\pgfqpoint{2.150391in}{0.739656in}}%
\pgfpathlineto{\pgfqpoint{2.150095in}{0.739656in}}%
\pgfpathlineto{\pgfqpoint{2.149799in}{0.739656in}}%
\pgfpathlineto{\pgfqpoint{2.149503in}{0.739656in}}%
\pgfpathlineto{\pgfqpoint{2.149207in}{0.739656in}}%
\pgfpathlineto{\pgfqpoint{2.148911in}{0.739656in}}%
\pgfpathlineto{\pgfqpoint{2.148615in}{0.739656in}}%
\pgfpathlineto{\pgfqpoint{2.148319in}{0.739656in}}%
\pgfpathlineto{\pgfqpoint{2.148023in}{0.739656in}}%
\pgfpathlineto{\pgfqpoint{2.147727in}{0.739656in}}%
\pgfpathlineto{\pgfqpoint{2.147431in}{0.739656in}}%
\pgfpathlineto{\pgfqpoint{2.147135in}{0.739656in}}%
\pgfpathlineto{\pgfqpoint{2.146839in}{0.739656in}}%
\pgfpathlineto{\pgfqpoint{2.146543in}{0.739656in}}%
\pgfpathlineto{\pgfqpoint{2.146247in}{0.739656in}}%
\pgfpathlineto{\pgfqpoint{2.145951in}{0.739656in}}%
\pgfpathlineto{\pgfqpoint{2.145655in}{0.739656in}}%
\pgfpathlineto{\pgfqpoint{2.145359in}{0.739656in}}%
\pgfpathlineto{\pgfqpoint{2.145063in}{0.739656in}}%
\pgfpathlineto{\pgfqpoint{2.144767in}{0.739656in}}%
\pgfpathlineto{\pgfqpoint{2.144471in}{0.739656in}}%
\pgfpathlineto{\pgfqpoint{2.144175in}{0.739656in}}%
\pgfpathlineto{\pgfqpoint{2.143879in}{0.739656in}}%
\pgfpathlineto{\pgfqpoint{2.143583in}{0.739656in}}%
\pgfpathlineto{\pgfqpoint{2.143287in}{0.739656in}}%
\pgfpathlineto{\pgfqpoint{2.142991in}{0.739656in}}%
\pgfpathlineto{\pgfqpoint{2.142695in}{0.739656in}}%
\pgfpathlineto{\pgfqpoint{2.142399in}{0.739656in}}%
\pgfpathlineto{\pgfqpoint{2.142103in}{0.739656in}}%
\pgfpathlineto{\pgfqpoint{2.141807in}{0.739656in}}%
\pgfpathlineto{\pgfqpoint{2.141511in}{0.739656in}}%
\pgfpathlineto{\pgfqpoint{2.141215in}{0.739656in}}%
\pgfpathlineto{\pgfqpoint{2.140919in}{0.739656in}}%
\pgfpathlineto{\pgfqpoint{2.140623in}{0.739656in}}%
\pgfpathlineto{\pgfqpoint{2.140327in}{0.739656in}}%
\pgfpathlineto{\pgfqpoint{2.140031in}{0.739656in}}%
\pgfpathlineto{\pgfqpoint{2.139735in}{0.739656in}}%
\pgfpathlineto{\pgfqpoint{2.139439in}{0.739656in}}%
\pgfpathlineto{\pgfqpoint{2.139143in}{0.739656in}}%
\pgfpathlineto{\pgfqpoint{2.138847in}{0.739656in}}%
\pgfpathlineto{\pgfqpoint{2.138551in}{0.739656in}}%
\pgfpathlineto{\pgfqpoint{2.138255in}{0.739656in}}%
\pgfpathlineto{\pgfqpoint{2.137959in}{0.739656in}}%
\pgfpathlineto{\pgfqpoint{2.137663in}{0.739656in}}%
\pgfpathlineto{\pgfqpoint{2.137367in}{0.739656in}}%
\pgfpathlineto{\pgfqpoint{2.137071in}{0.739656in}}%
\pgfpathlineto{\pgfqpoint{2.136775in}{0.739656in}}%
\pgfpathlineto{\pgfqpoint{2.136479in}{0.739656in}}%
\pgfpathlineto{\pgfqpoint{2.136183in}{0.739656in}}%
\pgfpathlineto{\pgfqpoint{2.135887in}{0.739656in}}%
\pgfpathlineto{\pgfqpoint{2.135591in}{0.739656in}}%
\pgfpathlineto{\pgfqpoint{2.135295in}{0.739656in}}%
\pgfpathlineto{\pgfqpoint{2.134999in}{0.739656in}}%
\pgfpathlineto{\pgfqpoint{2.134703in}{0.739656in}}%
\pgfpathlineto{\pgfqpoint{2.134407in}{0.739656in}}%
\pgfpathlineto{\pgfqpoint{2.134111in}{0.739656in}}%
\pgfpathlineto{\pgfqpoint{2.133815in}{0.739656in}}%
\pgfpathlineto{\pgfqpoint{2.133519in}{0.739656in}}%
\pgfpathlineto{\pgfqpoint{2.133223in}{0.739656in}}%
\pgfpathlineto{\pgfqpoint{2.132927in}{0.739656in}}%
\pgfpathlineto{\pgfqpoint{2.132631in}{0.739656in}}%
\pgfpathlineto{\pgfqpoint{2.132335in}{0.739656in}}%
\pgfpathlineto{\pgfqpoint{2.132039in}{0.739656in}}%
\pgfpathlineto{\pgfqpoint{2.131743in}{0.739656in}}%
\pgfpathlineto{\pgfqpoint{2.131447in}{0.739656in}}%
\pgfpathlineto{\pgfqpoint{2.131151in}{0.739656in}}%
\pgfpathlineto{\pgfqpoint{2.130854in}{0.739656in}}%
\pgfpathlineto{\pgfqpoint{2.130558in}{0.739656in}}%
\pgfpathlineto{\pgfqpoint{2.130262in}{0.739656in}}%
\pgfpathlineto{\pgfqpoint{2.129966in}{0.739656in}}%
\pgfpathlineto{\pgfqpoint{2.129670in}{0.739656in}}%
\pgfpathlineto{\pgfqpoint{2.129374in}{0.739656in}}%
\pgfpathlineto{\pgfqpoint{2.129078in}{0.739656in}}%
\pgfpathlineto{\pgfqpoint{2.128782in}{0.739656in}}%
\pgfpathlineto{\pgfqpoint{2.128486in}{0.739656in}}%
\pgfpathlineto{\pgfqpoint{2.128190in}{0.739656in}}%
\pgfpathlineto{\pgfqpoint{2.127894in}{0.739656in}}%
\pgfpathlineto{\pgfqpoint{2.127598in}{0.739656in}}%
\pgfpathlineto{\pgfqpoint{2.127302in}{0.739656in}}%
\pgfpathlineto{\pgfqpoint{2.127006in}{0.739656in}}%
\pgfpathlineto{\pgfqpoint{2.126710in}{0.739656in}}%
\pgfpathlineto{\pgfqpoint{2.126414in}{0.739656in}}%
\pgfpathlineto{\pgfqpoint{2.126118in}{0.739656in}}%
\pgfpathlineto{\pgfqpoint{2.125822in}{0.739656in}}%
\pgfpathlineto{\pgfqpoint{2.125526in}{0.739656in}}%
\pgfpathlineto{\pgfqpoint{2.125230in}{0.739656in}}%
\pgfpathlineto{\pgfqpoint{2.124934in}{0.739656in}}%
\pgfpathlineto{\pgfqpoint{2.124638in}{0.739656in}}%
\pgfpathlineto{\pgfqpoint{2.124342in}{0.739656in}}%
\pgfpathlineto{\pgfqpoint{2.124046in}{0.739656in}}%
\pgfpathlineto{\pgfqpoint{2.123750in}{0.739656in}}%
\pgfpathlineto{\pgfqpoint{2.123454in}{0.739656in}}%
\pgfpathlineto{\pgfqpoint{2.123158in}{0.739656in}}%
\pgfpathlineto{\pgfqpoint{2.122862in}{0.739656in}}%
\pgfpathlineto{\pgfqpoint{2.122566in}{0.739656in}}%
\pgfpathlineto{\pgfqpoint{2.122270in}{0.739656in}}%
\pgfpathlineto{\pgfqpoint{2.121974in}{0.739656in}}%
\pgfpathlineto{\pgfqpoint{2.121678in}{0.739656in}}%
\pgfpathlineto{\pgfqpoint{2.121382in}{0.739656in}}%
\pgfpathlineto{\pgfqpoint{2.121086in}{0.739656in}}%
\pgfpathlineto{\pgfqpoint{2.120790in}{0.739656in}}%
\pgfpathlineto{\pgfqpoint{2.120494in}{0.739656in}}%
\pgfpathlineto{\pgfqpoint{2.120198in}{0.739656in}}%
\pgfpathlineto{\pgfqpoint{2.119902in}{0.739656in}}%
\pgfpathlineto{\pgfqpoint{2.119606in}{0.739656in}}%
\pgfpathlineto{\pgfqpoint{2.119310in}{0.739656in}}%
\pgfpathlineto{\pgfqpoint{2.119014in}{0.739656in}}%
\pgfpathlineto{\pgfqpoint{2.118718in}{0.739656in}}%
\pgfpathlineto{\pgfqpoint{2.118422in}{0.739656in}}%
\pgfpathlineto{\pgfqpoint{2.118126in}{0.739656in}}%
\pgfpathlineto{\pgfqpoint{2.117830in}{0.739656in}}%
\pgfpathlineto{\pgfqpoint{2.117534in}{0.739656in}}%
\pgfpathlineto{\pgfqpoint{2.117238in}{0.739656in}}%
\pgfpathlineto{\pgfqpoint{2.116942in}{0.739656in}}%
\pgfpathlineto{\pgfqpoint{2.116646in}{0.739656in}}%
\pgfpathlineto{\pgfqpoint{2.116350in}{0.739656in}}%
\pgfpathlineto{\pgfqpoint{2.116054in}{0.739656in}}%
\pgfpathlineto{\pgfqpoint{2.115758in}{0.739656in}}%
\pgfpathlineto{\pgfqpoint{2.115462in}{0.739656in}}%
\pgfpathlineto{\pgfqpoint{2.115166in}{0.739656in}}%
\pgfpathlineto{\pgfqpoint{2.114870in}{0.739656in}}%
\pgfpathlineto{\pgfqpoint{2.114574in}{0.739656in}}%
\pgfpathlineto{\pgfqpoint{2.114278in}{0.739656in}}%
\pgfpathlineto{\pgfqpoint{2.113982in}{0.739656in}}%
\pgfpathlineto{\pgfqpoint{2.113686in}{0.739656in}}%
\pgfpathlineto{\pgfqpoint{2.113390in}{0.739656in}}%
\pgfpathlineto{\pgfqpoint{2.113094in}{0.739656in}}%
\pgfpathlineto{\pgfqpoint{2.112798in}{0.739656in}}%
\pgfpathlineto{\pgfqpoint{2.112502in}{0.739656in}}%
\pgfpathlineto{\pgfqpoint{2.112206in}{0.739656in}}%
\pgfpathlineto{\pgfqpoint{2.111910in}{0.739656in}}%
\pgfpathlineto{\pgfqpoint{2.111614in}{0.739656in}}%
\pgfpathlineto{\pgfqpoint{2.111318in}{0.739656in}}%
\pgfpathlineto{\pgfqpoint{2.111022in}{0.739656in}}%
\pgfpathlineto{\pgfqpoint{2.110726in}{0.739656in}}%
\pgfpathlineto{\pgfqpoint{2.110430in}{0.739656in}}%
\pgfpathlineto{\pgfqpoint{2.110134in}{0.739656in}}%
\pgfpathlineto{\pgfqpoint{2.109838in}{0.739656in}}%
\pgfpathlineto{\pgfqpoint{2.109542in}{0.739656in}}%
\pgfpathlineto{\pgfqpoint{2.109246in}{0.739656in}}%
\pgfpathlineto{\pgfqpoint{2.108950in}{0.739656in}}%
\pgfpathlineto{\pgfqpoint{2.108654in}{0.739656in}}%
\pgfpathlineto{\pgfqpoint{2.108358in}{0.739656in}}%
\pgfpathlineto{\pgfqpoint{2.108062in}{0.739656in}}%
\pgfpathlineto{\pgfqpoint{2.107766in}{0.739656in}}%
\pgfpathlineto{\pgfqpoint{2.107470in}{0.739656in}}%
\pgfpathlineto{\pgfqpoint{2.107174in}{0.739656in}}%
\pgfpathlineto{\pgfqpoint{2.106878in}{0.739656in}}%
\pgfpathlineto{\pgfqpoint{2.106582in}{0.739656in}}%
\pgfpathlineto{\pgfqpoint{2.106286in}{0.739656in}}%
\pgfpathlineto{\pgfqpoint{2.105990in}{0.739656in}}%
\pgfpathlineto{\pgfqpoint{2.105694in}{0.739656in}}%
\pgfpathlineto{\pgfqpoint{2.105398in}{0.739656in}}%
\pgfpathlineto{\pgfqpoint{2.105102in}{0.739656in}}%
\pgfpathlineto{\pgfqpoint{2.104806in}{0.739656in}}%
\pgfpathlineto{\pgfqpoint{2.104510in}{0.739656in}}%
\pgfpathlineto{\pgfqpoint{2.104214in}{0.739656in}}%
\pgfpathlineto{\pgfqpoint{2.103918in}{0.739656in}}%
\pgfpathlineto{\pgfqpoint{2.103622in}{0.739656in}}%
\pgfpathlineto{\pgfqpoint{2.103326in}{0.739656in}}%
\pgfpathlineto{\pgfqpoint{2.103030in}{0.739656in}}%
\pgfpathlineto{\pgfqpoint{2.102734in}{0.739656in}}%
\pgfpathlineto{\pgfqpoint{2.102438in}{0.739656in}}%
\pgfpathlineto{\pgfqpoint{2.102142in}{0.739656in}}%
\pgfpathlineto{\pgfqpoint{2.101846in}{0.739656in}}%
\pgfpathlineto{\pgfqpoint{2.101550in}{0.739656in}}%
\pgfpathlineto{\pgfqpoint{2.101254in}{0.739656in}}%
\pgfpathlineto{\pgfqpoint{2.100958in}{0.739656in}}%
\pgfpathlineto{\pgfqpoint{2.100662in}{0.739656in}}%
\pgfpathlineto{\pgfqpoint{2.100366in}{0.739656in}}%
\pgfpathlineto{\pgfqpoint{2.100070in}{0.739656in}}%
\pgfpathlineto{\pgfqpoint{2.099774in}{0.739656in}}%
\pgfpathlineto{\pgfqpoint{2.099478in}{0.739656in}}%
\pgfpathlineto{\pgfqpoint{2.099182in}{0.739656in}}%
\pgfpathlineto{\pgfqpoint{2.098886in}{0.739656in}}%
\pgfpathlineto{\pgfqpoint{2.098590in}{0.739656in}}%
\pgfpathlineto{\pgfqpoint{2.098294in}{0.739656in}}%
\pgfpathlineto{\pgfqpoint{2.097998in}{0.739656in}}%
\pgfpathlineto{\pgfqpoint{2.097702in}{0.739656in}}%
\pgfpathlineto{\pgfqpoint{2.097406in}{0.739656in}}%
\pgfpathlineto{\pgfqpoint{2.097110in}{0.739656in}}%
\pgfpathlineto{\pgfqpoint{2.096814in}{0.739656in}}%
\pgfpathlineto{\pgfqpoint{2.096518in}{0.739656in}}%
\pgfpathlineto{\pgfqpoint{2.096222in}{0.739656in}}%
\pgfpathlineto{\pgfqpoint{2.095926in}{0.739656in}}%
\pgfpathlineto{\pgfqpoint{2.095630in}{0.739656in}}%
\pgfpathlineto{\pgfqpoint{2.095334in}{0.739656in}}%
\pgfpathlineto{\pgfqpoint{2.095038in}{0.739656in}}%
\pgfpathlineto{\pgfqpoint{2.094742in}{0.739656in}}%
\pgfpathlineto{\pgfqpoint{2.094446in}{0.739656in}}%
\pgfpathlineto{\pgfqpoint{2.094150in}{0.739656in}}%
\pgfpathlineto{\pgfqpoint{2.093854in}{0.739656in}}%
\pgfpathlineto{\pgfqpoint{2.093558in}{0.739656in}}%
\pgfpathlineto{\pgfqpoint{2.093262in}{0.739656in}}%
\pgfpathlineto{\pgfqpoint{2.092966in}{0.739656in}}%
\pgfpathlineto{\pgfqpoint{2.092670in}{0.739656in}}%
\pgfpathlineto{\pgfqpoint{2.092374in}{0.739656in}}%
\pgfpathlineto{\pgfqpoint{2.092078in}{0.739656in}}%
\pgfpathlineto{\pgfqpoint{2.091782in}{0.739656in}}%
\pgfpathlineto{\pgfqpoint{2.091486in}{0.739656in}}%
\pgfpathlineto{\pgfqpoint{2.091190in}{0.739656in}}%
\pgfpathlineto{\pgfqpoint{2.090894in}{0.739656in}}%
\pgfpathlineto{\pgfqpoint{2.090598in}{0.739656in}}%
\pgfpathlineto{\pgfqpoint{2.090302in}{0.739656in}}%
\pgfpathlineto{\pgfqpoint{2.090006in}{0.739656in}}%
\pgfpathlineto{\pgfqpoint{2.089710in}{0.739656in}}%
\pgfpathlineto{\pgfqpoint{2.089414in}{0.739656in}}%
\pgfpathlineto{\pgfqpoint{2.089118in}{0.739656in}}%
\pgfpathlineto{\pgfqpoint{2.088822in}{0.739656in}}%
\pgfpathlineto{\pgfqpoint{2.088526in}{0.739656in}}%
\pgfpathlineto{\pgfqpoint{2.088230in}{0.739656in}}%
\pgfpathlineto{\pgfqpoint{2.087934in}{0.739656in}}%
\pgfpathlineto{\pgfqpoint{2.087638in}{0.739656in}}%
\pgfpathlineto{\pgfqpoint{2.087342in}{0.739656in}}%
\pgfpathlineto{\pgfqpoint{2.087046in}{0.739656in}}%
\pgfpathlineto{\pgfqpoint{2.086750in}{0.739656in}}%
\pgfpathlineto{\pgfqpoint{2.086454in}{0.739656in}}%
\pgfpathlineto{\pgfqpoint{2.086158in}{0.739656in}}%
\pgfpathlineto{\pgfqpoint{2.085862in}{0.739656in}}%
\pgfpathlineto{\pgfqpoint{2.085566in}{0.739656in}}%
\pgfpathlineto{\pgfqpoint{2.085270in}{0.739656in}}%
\pgfpathlineto{\pgfqpoint{2.084974in}{0.739656in}}%
\pgfpathlineto{\pgfqpoint{2.084678in}{0.739656in}}%
\pgfpathlineto{\pgfqpoint{2.084382in}{0.739656in}}%
\pgfpathlineto{\pgfqpoint{2.084086in}{0.739656in}}%
\pgfpathlineto{\pgfqpoint{2.083790in}{0.739656in}}%
\pgfpathlineto{\pgfqpoint{2.083494in}{0.739656in}}%
\pgfpathlineto{\pgfqpoint{2.083198in}{0.739656in}}%
\pgfpathlineto{\pgfqpoint{2.082902in}{0.739656in}}%
\pgfpathlineto{\pgfqpoint{2.082606in}{0.739656in}}%
\pgfpathlineto{\pgfqpoint{2.082310in}{0.739656in}}%
\pgfpathlineto{\pgfqpoint{2.082014in}{0.739656in}}%
\pgfpathlineto{\pgfqpoint{2.081718in}{0.739656in}}%
\pgfpathlineto{\pgfqpoint{2.081422in}{0.739656in}}%
\pgfpathlineto{\pgfqpoint{2.081126in}{0.739656in}}%
\pgfpathlineto{\pgfqpoint{2.080830in}{0.739656in}}%
\pgfpathlineto{\pgfqpoint{2.080534in}{0.739656in}}%
\pgfpathlineto{\pgfqpoint{2.080238in}{0.739656in}}%
\pgfpathlineto{\pgfqpoint{2.079942in}{0.739656in}}%
\pgfpathlineto{\pgfqpoint{2.079646in}{0.739656in}}%
\pgfpathlineto{\pgfqpoint{2.079350in}{0.739656in}}%
\pgfpathlineto{\pgfqpoint{2.079054in}{0.739656in}}%
\pgfpathlineto{\pgfqpoint{2.078758in}{0.739656in}}%
\pgfpathlineto{\pgfqpoint{2.078462in}{0.739656in}}%
\pgfpathlineto{\pgfqpoint{2.078166in}{0.739656in}}%
\pgfpathlineto{\pgfqpoint{2.077870in}{0.739656in}}%
\pgfpathlineto{\pgfqpoint{2.077574in}{0.739656in}}%
\pgfpathlineto{\pgfqpoint{2.077278in}{0.739656in}}%
\pgfpathlineto{\pgfqpoint{2.076982in}{0.739656in}}%
\pgfpathlineto{\pgfqpoint{2.076686in}{0.739656in}}%
\pgfpathlineto{\pgfqpoint{2.076390in}{0.739656in}}%
\pgfpathlineto{\pgfqpoint{2.076094in}{0.739656in}}%
\pgfpathlineto{\pgfqpoint{2.075798in}{0.739656in}}%
\pgfpathlineto{\pgfqpoint{2.075502in}{0.739656in}}%
\pgfpathlineto{\pgfqpoint{2.075206in}{0.739656in}}%
\pgfpathlineto{\pgfqpoint{2.074910in}{0.739656in}}%
\pgfpathlineto{\pgfqpoint{2.074614in}{0.739656in}}%
\pgfpathlineto{\pgfqpoint{2.074318in}{0.739656in}}%
\pgfpathlineto{\pgfqpoint{2.074022in}{0.739656in}}%
\pgfpathlineto{\pgfqpoint{2.073726in}{0.739656in}}%
\pgfpathlineto{\pgfqpoint{2.073430in}{0.739656in}}%
\pgfpathlineto{\pgfqpoint{2.073134in}{0.739656in}}%
\pgfpathlineto{\pgfqpoint{2.072838in}{0.739656in}}%
\pgfpathlineto{\pgfqpoint{2.072542in}{0.739656in}}%
\pgfpathlineto{\pgfqpoint{2.072246in}{0.739656in}}%
\pgfpathlineto{\pgfqpoint{2.071950in}{0.739656in}}%
\pgfpathlineto{\pgfqpoint{2.071654in}{0.739656in}}%
\pgfpathlineto{\pgfqpoint{2.071358in}{0.739656in}}%
\pgfpathlineto{\pgfqpoint{2.071062in}{0.739656in}}%
\pgfpathlineto{\pgfqpoint{2.070766in}{0.739656in}}%
\pgfpathlineto{\pgfqpoint{2.070470in}{0.739656in}}%
\pgfpathlineto{\pgfqpoint{2.070174in}{0.739656in}}%
\pgfpathlineto{\pgfqpoint{2.069878in}{0.739656in}}%
\pgfpathlineto{\pgfqpoint{2.069582in}{0.739656in}}%
\pgfpathlineto{\pgfqpoint{2.069286in}{0.739656in}}%
\pgfpathlineto{\pgfqpoint{2.068990in}{0.739656in}}%
\pgfpathlineto{\pgfqpoint{2.068694in}{0.739656in}}%
\pgfpathlineto{\pgfqpoint{2.068398in}{0.739656in}}%
\pgfpathlineto{\pgfqpoint{2.068102in}{0.739656in}}%
\pgfpathlineto{\pgfqpoint{2.067806in}{0.739656in}}%
\pgfpathlineto{\pgfqpoint{2.067510in}{0.739656in}}%
\pgfpathlineto{\pgfqpoint{2.067214in}{0.739656in}}%
\pgfpathlineto{\pgfqpoint{2.066918in}{0.739656in}}%
\pgfpathlineto{\pgfqpoint{2.066622in}{0.739656in}}%
\pgfpathlineto{\pgfqpoint{2.066326in}{0.739656in}}%
\pgfpathlineto{\pgfqpoint{2.066030in}{0.739656in}}%
\pgfpathlineto{\pgfqpoint{2.065734in}{0.739656in}}%
\pgfpathlineto{\pgfqpoint{2.065438in}{0.739656in}}%
\pgfpathlineto{\pgfqpoint{2.065142in}{0.739656in}}%
\pgfpathlineto{\pgfqpoint{2.064846in}{0.739656in}}%
\pgfpathlineto{\pgfqpoint{2.064550in}{0.739656in}}%
\pgfpathlineto{\pgfqpoint{2.064254in}{0.739656in}}%
\pgfpathlineto{\pgfqpoint{2.063958in}{0.739656in}}%
\pgfpathlineto{\pgfqpoint{2.063661in}{0.739656in}}%
\pgfpathlineto{\pgfqpoint{2.063365in}{0.739656in}}%
\pgfpathlineto{\pgfqpoint{2.063069in}{0.739656in}}%
\pgfpathlineto{\pgfqpoint{2.062773in}{0.739656in}}%
\pgfpathlineto{\pgfqpoint{2.062477in}{0.739656in}}%
\pgfpathlineto{\pgfqpoint{2.062181in}{0.739656in}}%
\pgfpathlineto{\pgfqpoint{2.061885in}{0.739656in}}%
\pgfpathlineto{\pgfqpoint{2.061589in}{0.739656in}}%
\pgfpathlineto{\pgfqpoint{2.061293in}{0.739656in}}%
\pgfpathlineto{\pgfqpoint{2.060997in}{0.739656in}}%
\pgfpathlineto{\pgfqpoint{2.060701in}{0.739656in}}%
\pgfpathlineto{\pgfqpoint{2.060405in}{0.739656in}}%
\pgfpathlineto{\pgfqpoint{2.060109in}{0.739656in}}%
\pgfpathlineto{\pgfqpoint{2.059813in}{0.739656in}}%
\pgfpathlineto{\pgfqpoint{2.059517in}{0.739656in}}%
\pgfpathlineto{\pgfqpoint{2.059221in}{0.739656in}}%
\pgfpathlineto{\pgfqpoint{2.058925in}{0.739656in}}%
\pgfpathlineto{\pgfqpoint{2.058629in}{0.739656in}}%
\pgfpathlineto{\pgfqpoint{2.058333in}{0.739656in}}%
\pgfpathlineto{\pgfqpoint{2.058037in}{0.739656in}}%
\pgfpathlineto{\pgfqpoint{2.057741in}{0.739656in}}%
\pgfpathlineto{\pgfqpoint{2.057445in}{0.739656in}}%
\pgfpathlineto{\pgfqpoint{2.057149in}{0.739656in}}%
\pgfpathlineto{\pgfqpoint{2.056853in}{0.739656in}}%
\pgfpathlineto{\pgfqpoint{2.056557in}{0.739656in}}%
\pgfpathlineto{\pgfqpoint{2.056261in}{0.739656in}}%
\pgfpathlineto{\pgfqpoint{2.055965in}{0.739656in}}%
\pgfpathlineto{\pgfqpoint{2.055669in}{0.739656in}}%
\pgfpathlineto{\pgfqpoint{2.055373in}{0.739656in}}%
\pgfpathlineto{\pgfqpoint{2.055077in}{0.739656in}}%
\pgfpathlineto{\pgfqpoint{2.054781in}{0.739656in}}%
\pgfpathlineto{\pgfqpoint{2.054485in}{0.739656in}}%
\pgfpathlineto{\pgfqpoint{2.054189in}{0.739656in}}%
\pgfpathlineto{\pgfqpoint{2.053893in}{0.739656in}}%
\pgfpathlineto{\pgfqpoint{2.053597in}{0.739656in}}%
\pgfpathlineto{\pgfqpoint{2.053301in}{0.739656in}}%
\pgfpathlineto{\pgfqpoint{2.053005in}{0.739656in}}%
\pgfpathlineto{\pgfqpoint{2.052709in}{0.739656in}}%
\pgfpathlineto{\pgfqpoint{2.052413in}{0.739656in}}%
\pgfpathlineto{\pgfqpoint{2.052117in}{0.739656in}}%
\pgfpathlineto{\pgfqpoint{2.051821in}{0.739656in}}%
\pgfpathlineto{\pgfqpoint{2.051525in}{0.739656in}}%
\pgfpathlineto{\pgfqpoint{2.051229in}{0.739656in}}%
\pgfpathlineto{\pgfqpoint{2.050933in}{0.739656in}}%
\pgfpathlineto{\pgfqpoint{2.050637in}{0.739656in}}%
\pgfpathlineto{\pgfqpoint{2.050341in}{0.739656in}}%
\pgfpathlineto{\pgfqpoint{2.050045in}{0.739656in}}%
\pgfpathlineto{\pgfqpoint{2.049749in}{0.739656in}}%
\pgfpathlineto{\pgfqpoint{2.049453in}{0.739656in}}%
\pgfpathlineto{\pgfqpoint{2.049157in}{0.739656in}}%
\pgfpathlineto{\pgfqpoint{2.048861in}{0.739656in}}%
\pgfpathlineto{\pgfqpoint{2.048565in}{0.739656in}}%
\pgfpathlineto{\pgfqpoint{2.048269in}{0.739656in}}%
\pgfpathlineto{\pgfqpoint{2.047973in}{0.739656in}}%
\pgfpathlineto{\pgfqpoint{2.047677in}{0.739656in}}%
\pgfpathlineto{\pgfqpoint{2.047381in}{0.739656in}}%
\pgfpathlineto{\pgfqpoint{2.047085in}{0.739656in}}%
\pgfpathlineto{\pgfqpoint{2.046789in}{0.739656in}}%
\pgfpathlineto{\pgfqpoint{2.046493in}{0.739656in}}%
\pgfpathlineto{\pgfqpoint{2.046197in}{0.739656in}}%
\pgfpathlineto{\pgfqpoint{2.045901in}{0.739656in}}%
\pgfpathlineto{\pgfqpoint{2.045605in}{0.739656in}}%
\pgfpathlineto{\pgfqpoint{2.045309in}{0.739656in}}%
\pgfpathlineto{\pgfqpoint{2.045013in}{0.739656in}}%
\pgfpathlineto{\pgfqpoint{2.044717in}{0.739656in}}%
\pgfpathlineto{\pgfqpoint{2.044421in}{0.739656in}}%
\pgfpathlineto{\pgfqpoint{2.044125in}{0.739656in}}%
\pgfpathlineto{\pgfqpoint{2.043829in}{0.739656in}}%
\pgfpathlineto{\pgfqpoint{2.043533in}{0.739656in}}%
\pgfpathlineto{\pgfqpoint{2.043237in}{0.739656in}}%
\pgfpathlineto{\pgfqpoint{2.042941in}{0.739656in}}%
\pgfpathlineto{\pgfqpoint{2.042645in}{0.739656in}}%
\pgfpathlineto{\pgfqpoint{2.042349in}{0.739656in}}%
\pgfpathlineto{\pgfqpoint{2.042053in}{0.739656in}}%
\pgfpathlineto{\pgfqpoint{2.041757in}{0.739656in}}%
\pgfpathlineto{\pgfqpoint{2.041461in}{0.739656in}}%
\pgfpathlineto{\pgfqpoint{2.041165in}{0.739656in}}%
\pgfpathlineto{\pgfqpoint{2.040869in}{0.739656in}}%
\pgfpathlineto{\pgfqpoint{2.040573in}{0.739656in}}%
\pgfpathlineto{\pgfqpoint{2.040277in}{0.739656in}}%
\pgfpathlineto{\pgfqpoint{2.039981in}{0.739656in}}%
\pgfpathlineto{\pgfqpoint{2.039685in}{0.739656in}}%
\pgfpathlineto{\pgfqpoint{2.039389in}{0.739656in}}%
\pgfpathlineto{\pgfqpoint{2.039093in}{0.739656in}}%
\pgfpathlineto{\pgfqpoint{2.038797in}{0.739656in}}%
\pgfpathlineto{\pgfqpoint{2.038501in}{0.739656in}}%
\pgfpathlineto{\pgfqpoint{2.038205in}{0.739656in}}%
\pgfpathlineto{\pgfqpoint{2.037909in}{0.739656in}}%
\pgfpathlineto{\pgfqpoint{2.037613in}{0.739656in}}%
\pgfpathlineto{\pgfqpoint{2.037317in}{0.739656in}}%
\pgfpathlineto{\pgfqpoint{2.037021in}{0.739656in}}%
\pgfpathlineto{\pgfqpoint{2.036725in}{0.739656in}}%
\pgfpathlineto{\pgfqpoint{2.036429in}{0.739656in}}%
\pgfpathlineto{\pgfqpoint{2.036133in}{0.739656in}}%
\pgfpathlineto{\pgfqpoint{2.035837in}{0.739656in}}%
\pgfpathlineto{\pgfqpoint{2.035541in}{0.739656in}}%
\pgfpathlineto{\pgfqpoint{2.035245in}{0.739656in}}%
\pgfpathlineto{\pgfqpoint{2.034949in}{0.739656in}}%
\pgfpathlineto{\pgfqpoint{2.034653in}{0.739656in}}%
\pgfpathlineto{\pgfqpoint{2.034357in}{0.739656in}}%
\pgfpathlineto{\pgfqpoint{2.034061in}{0.739656in}}%
\pgfpathlineto{\pgfqpoint{2.033765in}{0.739656in}}%
\pgfpathlineto{\pgfqpoint{2.033469in}{0.739656in}}%
\pgfpathlineto{\pgfqpoint{2.033173in}{0.739656in}}%
\pgfpathlineto{\pgfqpoint{2.032877in}{0.739656in}}%
\pgfpathlineto{\pgfqpoint{2.032581in}{0.739656in}}%
\pgfpathlineto{\pgfqpoint{2.032285in}{0.739656in}}%
\pgfpathlineto{\pgfqpoint{2.031989in}{0.739656in}}%
\pgfpathlineto{\pgfqpoint{2.031693in}{0.739656in}}%
\pgfpathlineto{\pgfqpoint{2.031397in}{0.739656in}}%
\pgfpathlineto{\pgfqpoint{2.031101in}{0.739656in}}%
\pgfpathlineto{\pgfqpoint{2.030805in}{0.739656in}}%
\pgfpathlineto{\pgfqpoint{2.030509in}{0.739656in}}%
\pgfpathlineto{\pgfqpoint{2.030213in}{0.739656in}}%
\pgfpathlineto{\pgfqpoint{2.029917in}{0.739656in}}%
\pgfpathlineto{\pgfqpoint{2.029621in}{0.739656in}}%
\pgfpathlineto{\pgfqpoint{2.029325in}{0.739656in}}%
\pgfpathlineto{\pgfqpoint{2.029029in}{0.739656in}}%
\pgfpathlineto{\pgfqpoint{2.028733in}{0.739656in}}%
\pgfpathlineto{\pgfqpoint{2.028437in}{0.739656in}}%
\pgfpathlineto{\pgfqpoint{2.028141in}{0.739656in}}%
\pgfpathlineto{\pgfqpoint{2.027845in}{0.739656in}}%
\pgfpathlineto{\pgfqpoint{2.027549in}{0.739656in}}%
\pgfpathlineto{\pgfqpoint{2.027253in}{0.739656in}}%
\pgfpathlineto{\pgfqpoint{2.026957in}{0.739656in}}%
\pgfpathlineto{\pgfqpoint{2.026661in}{0.739656in}}%
\pgfpathlineto{\pgfqpoint{2.026365in}{0.739656in}}%
\pgfpathlineto{\pgfqpoint{2.026069in}{0.739656in}}%
\pgfpathlineto{\pgfqpoint{2.025773in}{0.739656in}}%
\pgfpathlineto{\pgfqpoint{2.025477in}{0.739656in}}%
\pgfpathlineto{\pgfqpoint{2.025181in}{0.739656in}}%
\pgfpathlineto{\pgfqpoint{2.024885in}{0.739656in}}%
\pgfpathlineto{\pgfqpoint{2.024589in}{0.739656in}}%
\pgfpathlineto{\pgfqpoint{2.024293in}{0.739656in}}%
\pgfpathlineto{\pgfqpoint{2.023997in}{0.739656in}}%
\pgfpathlineto{\pgfqpoint{2.023701in}{0.739656in}}%
\pgfpathlineto{\pgfqpoint{2.023405in}{0.739656in}}%
\pgfpathlineto{\pgfqpoint{2.023109in}{0.739656in}}%
\pgfpathlineto{\pgfqpoint{2.022813in}{0.739656in}}%
\pgfpathlineto{\pgfqpoint{2.022517in}{0.739656in}}%
\pgfpathlineto{\pgfqpoint{2.022221in}{0.739656in}}%
\pgfpathlineto{\pgfqpoint{2.021925in}{0.739656in}}%
\pgfpathlineto{\pgfqpoint{2.021629in}{0.739656in}}%
\pgfpathlineto{\pgfqpoint{2.021333in}{0.739656in}}%
\pgfpathlineto{\pgfqpoint{2.021037in}{0.739656in}}%
\pgfpathlineto{\pgfqpoint{2.020741in}{0.739656in}}%
\pgfpathlineto{\pgfqpoint{2.020445in}{0.739656in}}%
\pgfpathlineto{\pgfqpoint{2.020149in}{0.739656in}}%
\pgfpathlineto{\pgfqpoint{2.019853in}{0.739656in}}%
\pgfpathlineto{\pgfqpoint{2.019557in}{0.739656in}}%
\pgfpathlineto{\pgfqpoint{2.019261in}{0.739656in}}%
\pgfpathlineto{\pgfqpoint{2.018965in}{0.739656in}}%
\pgfpathlineto{\pgfqpoint{2.018669in}{0.739656in}}%
\pgfpathlineto{\pgfqpoint{2.018373in}{0.739656in}}%
\pgfpathlineto{\pgfqpoint{2.018077in}{0.739656in}}%
\pgfpathlineto{\pgfqpoint{2.017781in}{0.739656in}}%
\pgfpathlineto{\pgfqpoint{2.017485in}{0.739656in}}%
\pgfpathlineto{\pgfqpoint{2.017189in}{0.739656in}}%
\pgfpathlineto{\pgfqpoint{2.016893in}{0.739656in}}%
\pgfpathlineto{\pgfqpoint{2.016597in}{0.739656in}}%
\pgfpathlineto{\pgfqpoint{2.016301in}{0.739656in}}%
\pgfpathlineto{\pgfqpoint{2.016005in}{0.739656in}}%
\pgfpathlineto{\pgfqpoint{2.015709in}{0.739656in}}%
\pgfpathlineto{\pgfqpoint{2.015413in}{0.739656in}}%
\pgfpathlineto{\pgfqpoint{2.015117in}{0.739656in}}%
\pgfpathlineto{\pgfqpoint{2.014821in}{0.739656in}}%
\pgfpathlineto{\pgfqpoint{2.014525in}{0.739656in}}%
\pgfpathlineto{\pgfqpoint{2.014229in}{0.739656in}}%
\pgfpathlineto{\pgfqpoint{2.013933in}{0.739656in}}%
\pgfpathlineto{\pgfqpoint{2.013637in}{0.739656in}}%
\pgfpathlineto{\pgfqpoint{2.013341in}{0.739656in}}%
\pgfpathlineto{\pgfqpoint{2.013045in}{0.739656in}}%
\pgfpathlineto{\pgfqpoint{2.012749in}{0.739656in}}%
\pgfpathlineto{\pgfqpoint{2.012453in}{0.739656in}}%
\pgfpathlineto{\pgfqpoint{2.012157in}{0.739656in}}%
\pgfpathlineto{\pgfqpoint{2.011861in}{0.739656in}}%
\pgfpathlineto{\pgfqpoint{2.011565in}{0.739656in}}%
\pgfpathlineto{\pgfqpoint{2.011269in}{0.739656in}}%
\pgfpathlineto{\pgfqpoint{2.010973in}{0.739656in}}%
\pgfpathlineto{\pgfqpoint{2.010677in}{0.739656in}}%
\pgfpathlineto{\pgfqpoint{2.010381in}{0.739656in}}%
\pgfpathlineto{\pgfqpoint{2.010085in}{0.739656in}}%
\pgfpathlineto{\pgfqpoint{2.009789in}{0.739656in}}%
\pgfpathlineto{\pgfqpoint{2.009493in}{0.739656in}}%
\pgfpathlineto{\pgfqpoint{2.009197in}{0.739656in}}%
\pgfpathlineto{\pgfqpoint{2.008901in}{0.739656in}}%
\pgfpathlineto{\pgfqpoint{2.008605in}{0.739656in}}%
\pgfpathlineto{\pgfqpoint{2.008309in}{0.739656in}}%
\pgfpathlineto{\pgfqpoint{2.008013in}{0.739656in}}%
\pgfpathlineto{\pgfqpoint{2.007717in}{0.739656in}}%
\pgfpathlineto{\pgfqpoint{2.007421in}{0.739656in}}%
\pgfpathlineto{\pgfqpoint{2.007125in}{0.739656in}}%
\pgfpathlineto{\pgfqpoint{2.006829in}{0.739656in}}%
\pgfpathlineto{\pgfqpoint{2.006533in}{0.739656in}}%
\pgfpathlineto{\pgfqpoint{2.006237in}{0.739656in}}%
\pgfpathlineto{\pgfqpoint{2.005941in}{0.739656in}}%
\pgfpathlineto{\pgfqpoint{2.005645in}{0.739656in}}%
\pgfpathlineto{\pgfqpoint{2.005349in}{0.739656in}}%
\pgfpathlineto{\pgfqpoint{2.005053in}{0.739656in}}%
\pgfpathlineto{\pgfqpoint{2.004757in}{0.739656in}}%
\pgfpathlineto{\pgfqpoint{2.004461in}{0.739656in}}%
\pgfpathlineto{\pgfqpoint{2.004165in}{0.739656in}}%
\pgfpathlineto{\pgfqpoint{2.003869in}{0.739656in}}%
\pgfpathlineto{\pgfqpoint{2.003573in}{0.739656in}}%
\pgfpathlineto{\pgfqpoint{2.003277in}{0.739656in}}%
\pgfpathlineto{\pgfqpoint{2.002981in}{0.739656in}}%
\pgfpathlineto{\pgfqpoint{2.002685in}{0.739656in}}%
\pgfpathlineto{\pgfqpoint{2.002389in}{0.739656in}}%
\pgfpathlineto{\pgfqpoint{2.002093in}{0.739656in}}%
\pgfpathlineto{\pgfqpoint{2.001797in}{0.739656in}}%
\pgfpathlineto{\pgfqpoint{2.001501in}{0.739656in}}%
\pgfpathlineto{\pgfqpoint{2.001205in}{0.739656in}}%
\pgfpathlineto{\pgfqpoint{2.000909in}{0.739656in}}%
\pgfpathlineto{\pgfqpoint{2.000613in}{0.739656in}}%
\pgfpathlineto{\pgfqpoint{2.000317in}{0.739656in}}%
\pgfpathlineto{\pgfqpoint{2.000021in}{0.739656in}}%
\pgfpathlineto{\pgfqpoint{1.999725in}{0.739656in}}%
\pgfpathlineto{\pgfqpoint{1.999429in}{0.739656in}}%
\pgfpathlineto{\pgfqpoint{1.999133in}{0.739656in}}%
\pgfpathlineto{\pgfqpoint{1.998837in}{0.739656in}}%
\pgfpathlineto{\pgfqpoint{1.998541in}{0.739656in}}%
\pgfpathlineto{\pgfqpoint{1.998245in}{0.739656in}}%
\pgfpathlineto{\pgfqpoint{1.997949in}{0.739656in}}%
\pgfpathlineto{\pgfqpoint{1.997653in}{0.739656in}}%
\pgfpathlineto{\pgfqpoint{1.997357in}{0.739656in}}%
\pgfpathlineto{\pgfqpoint{1.997061in}{0.739656in}}%
\pgfpathlineto{\pgfqpoint{1.996765in}{0.739656in}}%
\pgfpathlineto{\pgfqpoint{1.996469in}{0.739656in}}%
\pgfpathlineto{\pgfqpoint{1.996172in}{0.739656in}}%
\pgfpathlineto{\pgfqpoint{1.995876in}{0.739656in}}%
\pgfpathlineto{\pgfqpoint{1.995580in}{0.739656in}}%
\pgfpathlineto{\pgfqpoint{1.995284in}{0.739656in}}%
\pgfpathlineto{\pgfqpoint{1.994988in}{0.739656in}}%
\pgfpathlineto{\pgfqpoint{1.994692in}{0.739656in}}%
\pgfpathlineto{\pgfqpoint{1.994396in}{0.739656in}}%
\pgfpathlineto{\pgfqpoint{1.994100in}{0.739656in}}%
\pgfpathlineto{\pgfqpoint{1.993804in}{0.739656in}}%
\pgfpathlineto{\pgfqpoint{1.993508in}{0.739656in}}%
\pgfpathlineto{\pgfqpoint{1.993212in}{0.739656in}}%
\pgfpathlineto{\pgfqpoint{1.992916in}{0.739656in}}%
\pgfpathlineto{\pgfqpoint{1.992620in}{0.739656in}}%
\pgfpathlineto{\pgfqpoint{1.992324in}{0.739656in}}%
\pgfpathlineto{\pgfqpoint{1.992028in}{0.739656in}}%
\pgfpathlineto{\pgfqpoint{1.991732in}{0.739656in}}%
\pgfpathlineto{\pgfqpoint{1.991436in}{0.739656in}}%
\pgfpathlineto{\pgfqpoint{1.991140in}{0.739656in}}%
\pgfpathlineto{\pgfqpoint{1.990844in}{0.739656in}}%
\pgfpathlineto{\pgfqpoint{1.990548in}{0.739656in}}%
\pgfpathlineto{\pgfqpoint{1.990252in}{0.739656in}}%
\pgfpathlineto{\pgfqpoint{1.989956in}{0.739656in}}%
\pgfpathlineto{\pgfqpoint{1.989660in}{0.739656in}}%
\pgfpathlineto{\pgfqpoint{1.989364in}{0.739656in}}%
\pgfpathlineto{\pgfqpoint{1.989068in}{0.739656in}}%
\pgfpathlineto{\pgfqpoint{1.988772in}{0.739656in}}%
\pgfpathlineto{\pgfqpoint{1.988476in}{0.739656in}}%
\pgfpathlineto{\pgfqpoint{1.988180in}{0.739656in}}%
\pgfpathlineto{\pgfqpoint{1.987884in}{0.739656in}}%
\pgfpathlineto{\pgfqpoint{1.987588in}{0.739656in}}%
\pgfpathlineto{\pgfqpoint{1.987292in}{0.739656in}}%
\pgfpathlineto{\pgfqpoint{1.986996in}{0.739656in}}%
\pgfpathlineto{\pgfqpoint{1.986700in}{0.739656in}}%
\pgfpathlineto{\pgfqpoint{1.986404in}{0.739656in}}%
\pgfpathlineto{\pgfqpoint{1.986108in}{0.739656in}}%
\pgfpathlineto{\pgfqpoint{1.985812in}{0.739656in}}%
\pgfpathlineto{\pgfqpoint{1.985516in}{0.739656in}}%
\pgfpathlineto{\pgfqpoint{1.985220in}{0.739656in}}%
\pgfpathlineto{\pgfqpoint{1.984924in}{0.739656in}}%
\pgfpathlineto{\pgfqpoint{1.984628in}{0.739656in}}%
\pgfpathlineto{\pgfqpoint{1.984332in}{0.739656in}}%
\pgfpathlineto{\pgfqpoint{1.984036in}{0.739656in}}%
\pgfpathlineto{\pgfqpoint{1.983740in}{0.739656in}}%
\pgfpathlineto{\pgfqpoint{1.983444in}{0.739656in}}%
\pgfpathlineto{\pgfqpoint{1.983148in}{0.739656in}}%
\pgfpathlineto{\pgfqpoint{1.982852in}{0.739656in}}%
\pgfpathlineto{\pgfqpoint{1.982556in}{0.739656in}}%
\pgfpathlineto{\pgfqpoint{1.982260in}{0.739656in}}%
\pgfpathlineto{\pgfqpoint{1.981964in}{0.739656in}}%
\pgfpathlineto{\pgfqpoint{1.981668in}{0.739656in}}%
\pgfpathlineto{\pgfqpoint{1.981372in}{0.739656in}}%
\pgfpathlineto{\pgfqpoint{1.981076in}{0.739656in}}%
\pgfpathlineto{\pgfqpoint{1.980780in}{0.739656in}}%
\pgfpathlineto{\pgfqpoint{1.980484in}{0.739656in}}%
\pgfpathlineto{\pgfqpoint{1.980188in}{0.739656in}}%
\pgfpathlineto{\pgfqpoint{1.979892in}{0.739656in}}%
\pgfpathlineto{\pgfqpoint{1.979596in}{0.739656in}}%
\pgfpathlineto{\pgfqpoint{1.979300in}{0.739656in}}%
\pgfpathlineto{\pgfqpoint{1.979004in}{0.739656in}}%
\pgfpathlineto{\pgfqpoint{1.978708in}{0.739656in}}%
\pgfpathlineto{\pgfqpoint{1.978412in}{0.739656in}}%
\pgfpathlineto{\pgfqpoint{1.978116in}{0.739656in}}%
\pgfpathlineto{\pgfqpoint{1.977820in}{0.739656in}}%
\pgfpathlineto{\pgfqpoint{1.977524in}{0.739656in}}%
\pgfpathlineto{\pgfqpoint{1.977228in}{0.739656in}}%
\pgfpathlineto{\pgfqpoint{1.976932in}{0.739656in}}%
\pgfpathlineto{\pgfqpoint{1.976636in}{0.739656in}}%
\pgfpathlineto{\pgfqpoint{1.976340in}{0.739656in}}%
\pgfpathlineto{\pgfqpoint{1.976044in}{0.739656in}}%
\pgfpathlineto{\pgfqpoint{1.975748in}{0.739656in}}%
\pgfpathlineto{\pgfqpoint{1.975452in}{0.739656in}}%
\pgfpathlineto{\pgfqpoint{1.975156in}{0.739656in}}%
\pgfpathlineto{\pgfqpoint{1.974860in}{0.739656in}}%
\pgfpathlineto{\pgfqpoint{1.974564in}{0.739656in}}%
\pgfpathlineto{\pgfqpoint{1.974268in}{0.739656in}}%
\pgfpathlineto{\pgfqpoint{1.973972in}{0.739656in}}%
\pgfpathlineto{\pgfqpoint{1.973676in}{0.739656in}}%
\pgfpathlineto{\pgfqpoint{1.973380in}{0.739656in}}%
\pgfpathlineto{\pgfqpoint{1.973084in}{0.739656in}}%
\pgfpathlineto{\pgfqpoint{1.972788in}{0.739656in}}%
\pgfpathlineto{\pgfqpoint{1.972492in}{0.739656in}}%
\pgfpathlineto{\pgfqpoint{1.972196in}{0.739656in}}%
\pgfpathlineto{\pgfqpoint{1.971900in}{0.739656in}}%
\pgfpathlineto{\pgfqpoint{1.971604in}{0.739656in}}%
\pgfpathlineto{\pgfqpoint{1.971308in}{0.739656in}}%
\pgfpathlineto{\pgfqpoint{1.971012in}{0.739656in}}%
\pgfpathlineto{\pgfqpoint{1.970716in}{0.739656in}}%
\pgfpathlineto{\pgfqpoint{1.970420in}{0.739656in}}%
\pgfpathlineto{\pgfqpoint{1.970124in}{0.739656in}}%
\pgfpathlineto{\pgfqpoint{1.969828in}{0.739656in}}%
\pgfpathlineto{\pgfqpoint{1.969532in}{0.739656in}}%
\pgfpathlineto{\pgfqpoint{1.969236in}{0.739656in}}%
\pgfpathlineto{\pgfqpoint{1.968940in}{0.739656in}}%
\pgfpathlineto{\pgfqpoint{1.968644in}{0.739656in}}%
\pgfpathlineto{\pgfqpoint{1.968348in}{0.739656in}}%
\pgfpathlineto{\pgfqpoint{1.968052in}{0.739656in}}%
\pgfpathlineto{\pgfqpoint{1.967756in}{0.739656in}}%
\pgfpathlineto{\pgfqpoint{1.967460in}{0.739656in}}%
\pgfpathlineto{\pgfqpoint{1.967164in}{0.739656in}}%
\pgfpathlineto{\pgfqpoint{1.966868in}{0.739656in}}%
\pgfpathlineto{\pgfqpoint{1.966572in}{0.739656in}}%
\pgfpathlineto{\pgfqpoint{1.966276in}{0.739656in}}%
\pgfpathlineto{\pgfqpoint{1.965980in}{0.739656in}}%
\pgfpathlineto{\pgfqpoint{1.965684in}{0.739656in}}%
\pgfpathlineto{\pgfqpoint{1.965388in}{0.739656in}}%
\pgfpathlineto{\pgfqpoint{1.965092in}{0.739656in}}%
\pgfpathlineto{\pgfqpoint{1.964796in}{0.739656in}}%
\pgfpathlineto{\pgfqpoint{1.964500in}{0.739656in}}%
\pgfpathlineto{\pgfqpoint{1.964204in}{0.739656in}}%
\pgfpathlineto{\pgfqpoint{1.963908in}{0.739656in}}%
\pgfpathlineto{\pgfqpoint{1.963612in}{0.739656in}}%
\pgfpathlineto{\pgfqpoint{1.963316in}{0.739656in}}%
\pgfpathlineto{\pgfqpoint{1.963020in}{0.739656in}}%
\pgfpathlineto{\pgfqpoint{1.962724in}{0.739656in}}%
\pgfpathlineto{\pgfqpoint{1.962428in}{0.739656in}}%
\pgfpathlineto{\pgfqpoint{1.962132in}{0.739656in}}%
\pgfpathlineto{\pgfqpoint{1.961836in}{0.739656in}}%
\pgfpathlineto{\pgfqpoint{1.961540in}{0.739656in}}%
\pgfpathlineto{\pgfqpoint{1.961244in}{0.739656in}}%
\pgfpathlineto{\pgfqpoint{1.960948in}{0.739656in}}%
\pgfpathlineto{\pgfqpoint{1.960652in}{0.739656in}}%
\pgfpathlineto{\pgfqpoint{1.960356in}{0.739656in}}%
\pgfpathlineto{\pgfqpoint{1.960060in}{0.739656in}}%
\pgfpathlineto{\pgfqpoint{1.959764in}{0.739656in}}%
\pgfpathlineto{\pgfqpoint{1.959468in}{0.739656in}}%
\pgfpathlineto{\pgfqpoint{1.959172in}{0.739656in}}%
\pgfpathlineto{\pgfqpoint{1.958876in}{0.739656in}}%
\pgfpathlineto{\pgfqpoint{1.958580in}{0.739656in}}%
\pgfpathlineto{\pgfqpoint{1.958284in}{0.739656in}}%
\pgfpathlineto{\pgfqpoint{1.957988in}{0.739656in}}%
\pgfpathlineto{\pgfqpoint{1.957692in}{0.739656in}}%
\pgfpathlineto{\pgfqpoint{1.957396in}{0.739656in}}%
\pgfpathlineto{\pgfqpoint{1.957100in}{0.739656in}}%
\pgfpathlineto{\pgfqpoint{1.956804in}{0.739656in}}%
\pgfpathlineto{\pgfqpoint{1.956508in}{0.739656in}}%
\pgfpathlineto{\pgfqpoint{1.956212in}{0.739656in}}%
\pgfpathlineto{\pgfqpoint{1.955916in}{0.739656in}}%
\pgfpathlineto{\pgfqpoint{1.955620in}{0.739656in}}%
\pgfpathlineto{\pgfqpoint{1.955324in}{0.739656in}}%
\pgfpathlineto{\pgfqpoint{1.955028in}{0.739656in}}%
\pgfpathlineto{\pgfqpoint{1.954732in}{0.739656in}}%
\pgfpathlineto{\pgfqpoint{1.954436in}{0.739656in}}%
\pgfpathlineto{\pgfqpoint{1.954140in}{0.739656in}}%
\pgfpathlineto{\pgfqpoint{1.953844in}{0.739656in}}%
\pgfpathlineto{\pgfqpoint{1.953548in}{0.739656in}}%
\pgfpathlineto{\pgfqpoint{1.953252in}{0.739656in}}%
\pgfpathlineto{\pgfqpoint{1.952956in}{0.739656in}}%
\pgfpathlineto{\pgfqpoint{1.952660in}{0.739656in}}%
\pgfpathlineto{\pgfqpoint{1.952364in}{0.739656in}}%
\pgfpathlineto{\pgfqpoint{1.952068in}{0.739656in}}%
\pgfpathlineto{\pgfqpoint{1.951772in}{0.739656in}}%
\pgfpathlineto{\pgfqpoint{1.951476in}{0.739656in}}%
\pgfpathlineto{\pgfqpoint{1.951180in}{0.739656in}}%
\pgfpathlineto{\pgfqpoint{1.950884in}{0.739656in}}%
\pgfpathlineto{\pgfqpoint{1.950588in}{0.739656in}}%
\pgfpathlineto{\pgfqpoint{1.950292in}{0.739656in}}%
\pgfpathlineto{\pgfqpoint{1.949996in}{0.739656in}}%
\pgfpathlineto{\pgfqpoint{1.949700in}{0.739656in}}%
\pgfpathlineto{\pgfqpoint{1.949404in}{0.739656in}}%
\pgfpathlineto{\pgfqpoint{1.949108in}{0.739656in}}%
\pgfpathlineto{\pgfqpoint{1.948812in}{0.739656in}}%
\pgfpathlineto{\pgfqpoint{1.948516in}{0.739656in}}%
\pgfpathlineto{\pgfqpoint{1.948220in}{0.739656in}}%
\pgfpathlineto{\pgfqpoint{1.947924in}{0.739656in}}%
\pgfpathlineto{\pgfqpoint{1.947628in}{0.739656in}}%
\pgfpathlineto{\pgfqpoint{1.947332in}{0.739656in}}%
\pgfpathlineto{\pgfqpoint{1.947036in}{0.739656in}}%
\pgfpathlineto{\pgfqpoint{1.946740in}{0.739656in}}%
\pgfpathlineto{\pgfqpoint{1.946444in}{0.739656in}}%
\pgfpathlineto{\pgfqpoint{1.946148in}{0.739656in}}%
\pgfpathlineto{\pgfqpoint{1.945852in}{0.739656in}}%
\pgfpathlineto{\pgfqpoint{1.945556in}{0.739656in}}%
\pgfpathlineto{\pgfqpoint{1.945260in}{0.739656in}}%
\pgfpathlineto{\pgfqpoint{1.944964in}{0.739656in}}%
\pgfpathlineto{\pgfqpoint{1.944668in}{0.739656in}}%
\pgfpathlineto{\pgfqpoint{1.944372in}{0.739656in}}%
\pgfpathlineto{\pgfqpoint{1.944076in}{0.739656in}}%
\pgfpathlineto{\pgfqpoint{1.943780in}{0.739656in}}%
\pgfpathlineto{\pgfqpoint{1.943484in}{0.739656in}}%
\pgfpathlineto{\pgfqpoint{1.943188in}{0.739656in}}%
\pgfpathlineto{\pgfqpoint{1.942892in}{0.739656in}}%
\pgfpathlineto{\pgfqpoint{1.942596in}{0.739656in}}%
\pgfpathlineto{\pgfqpoint{1.942300in}{0.739656in}}%
\pgfpathlineto{\pgfqpoint{1.942004in}{0.739656in}}%
\pgfpathlineto{\pgfqpoint{1.941708in}{0.739656in}}%
\pgfpathlineto{\pgfqpoint{1.941412in}{0.739656in}}%
\pgfpathlineto{\pgfqpoint{1.941116in}{0.739656in}}%
\pgfpathlineto{\pgfqpoint{1.940820in}{0.739656in}}%
\pgfpathlineto{\pgfqpoint{1.940524in}{0.739656in}}%
\pgfpathlineto{\pgfqpoint{1.940228in}{0.739656in}}%
\pgfpathlineto{\pgfqpoint{1.939932in}{0.739656in}}%
\pgfpathlineto{\pgfqpoint{1.939636in}{0.739656in}}%
\pgfpathlineto{\pgfqpoint{1.939340in}{0.739656in}}%
\pgfpathlineto{\pgfqpoint{1.939044in}{0.739656in}}%
\pgfpathlineto{\pgfqpoint{1.938748in}{0.739656in}}%
\pgfpathlineto{\pgfqpoint{1.938452in}{0.739656in}}%
\pgfpathlineto{\pgfqpoint{1.938156in}{0.739656in}}%
\pgfpathlineto{\pgfqpoint{1.937860in}{0.739656in}}%
\pgfpathlineto{\pgfqpoint{1.937564in}{0.739656in}}%
\pgfpathlineto{\pgfqpoint{1.937268in}{0.739656in}}%
\pgfpathlineto{\pgfqpoint{1.936972in}{0.739656in}}%
\pgfpathlineto{\pgfqpoint{1.936676in}{0.739656in}}%
\pgfpathlineto{\pgfqpoint{1.936380in}{0.739656in}}%
\pgfpathlineto{\pgfqpoint{1.936084in}{0.739656in}}%
\pgfpathlineto{\pgfqpoint{1.935788in}{0.739656in}}%
\pgfpathlineto{\pgfqpoint{1.935492in}{0.739656in}}%
\pgfpathlineto{\pgfqpoint{1.935196in}{0.739656in}}%
\pgfpathlineto{\pgfqpoint{1.934900in}{0.739656in}}%
\pgfpathlineto{\pgfqpoint{1.934604in}{0.739656in}}%
\pgfpathlineto{\pgfqpoint{1.934308in}{0.739656in}}%
\pgfpathlineto{\pgfqpoint{1.934012in}{0.739656in}}%
\pgfpathlineto{\pgfqpoint{1.933716in}{0.739656in}}%
\pgfpathlineto{\pgfqpoint{1.933420in}{0.739656in}}%
\pgfpathlineto{\pgfqpoint{1.933124in}{0.739656in}}%
\pgfpathlineto{\pgfqpoint{1.932828in}{0.739656in}}%
\pgfpathlineto{\pgfqpoint{1.932532in}{0.739656in}}%
\pgfpathlineto{\pgfqpoint{1.932236in}{0.739656in}}%
\pgfpathlineto{\pgfqpoint{1.931940in}{0.739656in}}%
\pgfpathlineto{\pgfqpoint{1.931644in}{0.739656in}}%
\pgfpathlineto{\pgfqpoint{1.931348in}{0.739656in}}%
\pgfpathlineto{\pgfqpoint{1.931052in}{0.739656in}}%
\pgfpathlineto{\pgfqpoint{1.930756in}{0.739656in}}%
\pgfpathlineto{\pgfqpoint{1.930460in}{0.739656in}}%
\pgfpathlineto{\pgfqpoint{1.930164in}{0.739656in}}%
\pgfpathlineto{\pgfqpoint{1.929868in}{0.739656in}}%
\pgfpathlineto{\pgfqpoint{1.929572in}{0.739656in}}%
\pgfpathlineto{\pgfqpoint{1.929276in}{0.739656in}}%
\pgfpathlineto{\pgfqpoint{1.928980in}{0.739656in}}%
\pgfpathlineto{\pgfqpoint{1.928683in}{0.739656in}}%
\pgfpathlineto{\pgfqpoint{1.928387in}{0.739656in}}%
\pgfpathlineto{\pgfqpoint{1.928091in}{0.739656in}}%
\pgfpathlineto{\pgfqpoint{1.927795in}{0.739656in}}%
\pgfpathlineto{\pgfqpoint{1.927499in}{0.739656in}}%
\pgfpathlineto{\pgfqpoint{1.927203in}{0.739656in}}%
\pgfpathlineto{\pgfqpoint{1.926907in}{0.739656in}}%
\pgfpathlineto{\pgfqpoint{1.926611in}{0.739656in}}%
\pgfpathlineto{\pgfqpoint{1.926315in}{0.739656in}}%
\pgfpathlineto{\pgfqpoint{1.926019in}{0.739656in}}%
\pgfpathlineto{\pgfqpoint{1.925723in}{0.739656in}}%
\pgfpathlineto{\pgfqpoint{1.925427in}{0.739656in}}%
\pgfpathlineto{\pgfqpoint{1.925131in}{0.739656in}}%
\pgfpathlineto{\pgfqpoint{1.924835in}{0.739656in}}%
\pgfpathlineto{\pgfqpoint{1.924539in}{0.739656in}}%
\pgfpathlineto{\pgfqpoint{1.924243in}{0.739656in}}%
\pgfpathlineto{\pgfqpoint{1.923947in}{0.739656in}}%
\pgfpathlineto{\pgfqpoint{1.923651in}{0.739656in}}%
\pgfpathlineto{\pgfqpoint{1.923355in}{0.739656in}}%
\pgfpathlineto{\pgfqpoint{1.923059in}{0.739656in}}%
\pgfpathlineto{\pgfqpoint{1.922763in}{0.739656in}}%
\pgfpathlineto{\pgfqpoint{1.922467in}{0.739656in}}%
\pgfpathlineto{\pgfqpoint{1.922171in}{0.739656in}}%
\pgfpathlineto{\pgfqpoint{1.921875in}{0.739656in}}%
\pgfpathlineto{\pgfqpoint{1.921579in}{0.739656in}}%
\pgfpathlineto{\pgfqpoint{1.921283in}{0.739656in}}%
\pgfpathlineto{\pgfqpoint{1.920987in}{0.739656in}}%
\pgfpathlineto{\pgfqpoint{1.920691in}{0.739656in}}%
\pgfpathlineto{\pgfqpoint{1.920395in}{0.739656in}}%
\pgfpathlineto{\pgfqpoint{1.920099in}{0.739656in}}%
\pgfpathlineto{\pgfqpoint{1.919803in}{0.739656in}}%
\pgfpathlineto{\pgfqpoint{1.919507in}{0.739656in}}%
\pgfpathlineto{\pgfqpoint{1.919211in}{0.739656in}}%
\pgfpathlineto{\pgfqpoint{1.918915in}{0.739656in}}%
\pgfpathlineto{\pgfqpoint{1.918619in}{0.739656in}}%
\pgfpathlineto{\pgfqpoint{1.918323in}{0.739656in}}%
\pgfpathlineto{\pgfqpoint{1.918027in}{0.739656in}}%
\pgfpathlineto{\pgfqpoint{1.917731in}{0.739656in}}%
\pgfpathlineto{\pgfqpoint{1.917435in}{0.739656in}}%
\pgfpathlineto{\pgfqpoint{1.917139in}{0.739656in}}%
\pgfpathlineto{\pgfqpoint{1.916843in}{0.739656in}}%
\pgfpathlineto{\pgfqpoint{1.916547in}{0.739656in}}%
\pgfpathlineto{\pgfqpoint{1.916251in}{0.739656in}}%
\pgfpathlineto{\pgfqpoint{1.915955in}{0.739656in}}%
\pgfpathlineto{\pgfqpoint{1.915659in}{0.739656in}}%
\pgfpathlineto{\pgfqpoint{1.915363in}{0.739656in}}%
\pgfpathlineto{\pgfqpoint{1.915067in}{0.739656in}}%
\pgfpathlineto{\pgfqpoint{1.914771in}{0.739656in}}%
\pgfpathlineto{\pgfqpoint{1.914475in}{0.739656in}}%
\pgfpathlineto{\pgfqpoint{1.914179in}{0.739656in}}%
\pgfpathlineto{\pgfqpoint{1.913883in}{0.739656in}}%
\pgfpathlineto{\pgfqpoint{1.913587in}{0.739656in}}%
\pgfpathlineto{\pgfqpoint{1.913291in}{0.739656in}}%
\pgfpathlineto{\pgfqpoint{1.912995in}{0.739656in}}%
\pgfpathlineto{\pgfqpoint{1.912699in}{0.739656in}}%
\pgfpathlineto{\pgfqpoint{1.912403in}{0.739656in}}%
\pgfpathlineto{\pgfqpoint{1.912107in}{0.739656in}}%
\pgfpathlineto{\pgfqpoint{1.911811in}{0.739656in}}%
\pgfpathlineto{\pgfqpoint{1.911515in}{0.739656in}}%
\pgfpathlineto{\pgfqpoint{1.911219in}{0.739656in}}%
\pgfpathlineto{\pgfqpoint{1.910923in}{0.739656in}}%
\pgfpathlineto{\pgfqpoint{1.910627in}{0.739656in}}%
\pgfpathlineto{\pgfqpoint{1.910331in}{0.739656in}}%
\pgfpathlineto{\pgfqpoint{1.910035in}{0.739656in}}%
\pgfpathlineto{\pgfqpoint{1.909739in}{0.739656in}}%
\pgfpathlineto{\pgfqpoint{1.909443in}{0.739656in}}%
\pgfpathlineto{\pgfqpoint{1.909147in}{0.739656in}}%
\pgfpathlineto{\pgfqpoint{1.908851in}{0.739656in}}%
\pgfpathlineto{\pgfqpoint{1.908555in}{0.739656in}}%
\pgfpathlineto{\pgfqpoint{1.908259in}{0.739656in}}%
\pgfpathlineto{\pgfqpoint{1.907963in}{0.739656in}}%
\pgfpathlineto{\pgfqpoint{1.907667in}{0.739656in}}%
\pgfpathlineto{\pgfqpoint{1.907371in}{0.739656in}}%
\pgfpathlineto{\pgfqpoint{1.907075in}{0.739656in}}%
\pgfpathlineto{\pgfqpoint{1.906779in}{0.739656in}}%
\pgfpathlineto{\pgfqpoint{1.906483in}{0.739656in}}%
\pgfpathlineto{\pgfqpoint{1.906187in}{0.739656in}}%
\pgfpathlineto{\pgfqpoint{1.905891in}{0.739656in}}%
\pgfpathlineto{\pgfqpoint{1.905595in}{0.739656in}}%
\pgfpathlineto{\pgfqpoint{1.905299in}{0.739656in}}%
\pgfpathlineto{\pgfqpoint{1.905003in}{0.739656in}}%
\pgfpathlineto{\pgfqpoint{1.904707in}{0.739656in}}%
\pgfpathlineto{\pgfqpoint{1.904411in}{0.739656in}}%
\pgfpathlineto{\pgfqpoint{1.904115in}{0.739656in}}%
\pgfpathlineto{\pgfqpoint{1.903819in}{0.739656in}}%
\pgfpathlineto{\pgfqpoint{1.903523in}{0.739656in}}%
\pgfpathlineto{\pgfqpoint{1.903227in}{0.739656in}}%
\pgfpathlineto{\pgfqpoint{1.902931in}{0.739656in}}%
\pgfpathlineto{\pgfqpoint{1.902635in}{0.739656in}}%
\pgfpathlineto{\pgfqpoint{1.902339in}{0.739656in}}%
\pgfpathlineto{\pgfqpoint{1.902043in}{0.739656in}}%
\pgfpathlineto{\pgfqpoint{1.901747in}{0.739656in}}%
\pgfpathlineto{\pgfqpoint{1.901451in}{0.739656in}}%
\pgfpathlineto{\pgfqpoint{1.901155in}{0.739656in}}%
\pgfpathlineto{\pgfqpoint{1.900859in}{0.739656in}}%
\pgfpathlineto{\pgfqpoint{1.900563in}{0.739656in}}%
\pgfpathlineto{\pgfqpoint{1.900267in}{0.739656in}}%
\pgfpathlineto{\pgfqpoint{1.899971in}{0.739656in}}%
\pgfpathlineto{\pgfqpoint{1.899675in}{0.739656in}}%
\pgfpathlineto{\pgfqpoint{1.899379in}{0.739656in}}%
\pgfpathlineto{\pgfqpoint{1.899083in}{0.739656in}}%
\pgfpathlineto{\pgfqpoint{1.898787in}{0.739656in}}%
\pgfpathlineto{\pgfqpoint{1.898491in}{0.739656in}}%
\pgfpathlineto{\pgfqpoint{1.898195in}{0.739656in}}%
\pgfpathlineto{\pgfqpoint{1.897899in}{0.739656in}}%
\pgfpathlineto{\pgfqpoint{1.897603in}{0.739656in}}%
\pgfpathlineto{\pgfqpoint{1.897307in}{0.739656in}}%
\pgfpathlineto{\pgfqpoint{1.897011in}{0.739656in}}%
\pgfpathlineto{\pgfqpoint{1.896715in}{0.739656in}}%
\pgfpathlineto{\pgfqpoint{1.896419in}{0.739656in}}%
\pgfpathlineto{\pgfqpoint{1.896123in}{0.739656in}}%
\pgfpathlineto{\pgfqpoint{1.895827in}{0.739656in}}%
\pgfpathlineto{\pgfqpoint{1.895531in}{0.739656in}}%
\pgfpathlineto{\pgfqpoint{1.895235in}{0.739656in}}%
\pgfpathlineto{\pgfqpoint{1.894939in}{0.739656in}}%
\pgfpathlineto{\pgfqpoint{1.894643in}{0.739656in}}%
\pgfpathlineto{\pgfqpoint{1.894347in}{0.739656in}}%
\pgfpathlineto{\pgfqpoint{1.894051in}{0.739656in}}%
\pgfpathlineto{\pgfqpoint{1.893755in}{0.739656in}}%
\pgfpathlineto{\pgfqpoint{1.893459in}{0.739656in}}%
\pgfpathlineto{\pgfqpoint{1.893163in}{0.739656in}}%
\pgfpathlineto{\pgfqpoint{1.892867in}{0.739656in}}%
\pgfpathlineto{\pgfqpoint{1.892571in}{0.739656in}}%
\pgfpathlineto{\pgfqpoint{1.892275in}{0.739656in}}%
\pgfpathlineto{\pgfqpoint{1.891979in}{0.739656in}}%
\pgfpathlineto{\pgfqpoint{1.891683in}{0.739656in}}%
\pgfpathlineto{\pgfqpoint{1.891387in}{0.739656in}}%
\pgfpathlineto{\pgfqpoint{1.891091in}{0.739656in}}%
\pgfpathlineto{\pgfqpoint{1.890795in}{0.739656in}}%
\pgfpathlineto{\pgfqpoint{1.890499in}{0.739656in}}%
\pgfpathlineto{\pgfqpoint{1.890203in}{0.739656in}}%
\pgfpathlineto{\pgfqpoint{1.889907in}{0.739656in}}%
\pgfpathlineto{\pgfqpoint{1.889611in}{0.739656in}}%
\pgfpathlineto{\pgfqpoint{1.889315in}{0.739656in}}%
\pgfpathlineto{\pgfqpoint{1.889019in}{0.739656in}}%
\pgfpathlineto{\pgfqpoint{1.888723in}{0.739656in}}%
\pgfpathlineto{\pgfqpoint{1.888427in}{0.739656in}}%
\pgfpathlineto{\pgfqpoint{1.888131in}{0.739656in}}%
\pgfpathlineto{\pgfqpoint{1.887835in}{0.739656in}}%
\pgfpathlineto{\pgfqpoint{1.887539in}{0.739656in}}%
\pgfpathlineto{\pgfqpoint{1.887243in}{0.739656in}}%
\pgfpathlineto{\pgfqpoint{1.886947in}{0.739656in}}%
\pgfpathlineto{\pgfqpoint{1.886651in}{0.739656in}}%
\pgfpathlineto{\pgfqpoint{1.886355in}{0.739656in}}%
\pgfpathlineto{\pgfqpoint{1.886059in}{0.739656in}}%
\pgfpathlineto{\pgfqpoint{1.885763in}{0.739656in}}%
\pgfpathlineto{\pgfqpoint{1.885467in}{0.739656in}}%
\pgfpathlineto{\pgfqpoint{1.885171in}{0.739656in}}%
\pgfpathlineto{\pgfqpoint{1.884875in}{0.739656in}}%
\pgfpathlineto{\pgfqpoint{1.884579in}{0.739656in}}%
\pgfpathlineto{\pgfqpoint{1.884283in}{0.739656in}}%
\pgfpathlineto{\pgfqpoint{1.883987in}{0.739656in}}%
\pgfpathlineto{\pgfqpoint{1.883691in}{0.739656in}}%
\pgfpathlineto{\pgfqpoint{1.883395in}{0.739656in}}%
\pgfpathlineto{\pgfqpoint{1.883099in}{0.739656in}}%
\pgfpathlineto{\pgfqpoint{1.882803in}{0.739656in}}%
\pgfpathlineto{\pgfqpoint{1.882507in}{0.739656in}}%
\pgfpathlineto{\pgfqpoint{1.882211in}{0.739656in}}%
\pgfpathlineto{\pgfqpoint{1.881915in}{0.739656in}}%
\pgfpathlineto{\pgfqpoint{1.881619in}{0.739656in}}%
\pgfpathlineto{\pgfqpoint{1.881323in}{0.739656in}}%
\pgfpathlineto{\pgfqpoint{1.881027in}{0.739656in}}%
\pgfpathlineto{\pgfqpoint{1.880731in}{0.739656in}}%
\pgfpathlineto{\pgfqpoint{1.880435in}{0.739656in}}%
\pgfpathlineto{\pgfqpoint{1.880139in}{0.739656in}}%
\pgfpathlineto{\pgfqpoint{1.879843in}{0.739656in}}%
\pgfpathlineto{\pgfqpoint{1.879547in}{0.739656in}}%
\pgfpathlineto{\pgfqpoint{1.879251in}{0.739656in}}%
\pgfpathlineto{\pgfqpoint{1.878955in}{0.739656in}}%
\pgfpathlineto{\pgfqpoint{1.878659in}{0.739656in}}%
\pgfpathlineto{\pgfqpoint{1.878363in}{0.739656in}}%
\pgfpathlineto{\pgfqpoint{1.878067in}{0.739656in}}%
\pgfpathlineto{\pgfqpoint{1.877771in}{0.739656in}}%
\pgfpathlineto{\pgfqpoint{1.877475in}{0.739656in}}%
\pgfpathlineto{\pgfqpoint{1.877179in}{0.739656in}}%
\pgfpathlineto{\pgfqpoint{1.876883in}{0.739656in}}%
\pgfpathlineto{\pgfqpoint{1.876587in}{0.739656in}}%
\pgfpathlineto{\pgfqpoint{1.876291in}{0.739656in}}%
\pgfpathlineto{\pgfqpoint{1.875995in}{0.739656in}}%
\pgfpathlineto{\pgfqpoint{1.875699in}{0.739656in}}%
\pgfpathlineto{\pgfqpoint{1.875403in}{0.739656in}}%
\pgfpathlineto{\pgfqpoint{1.875107in}{0.739656in}}%
\pgfpathlineto{\pgfqpoint{1.874811in}{0.739656in}}%
\pgfpathlineto{\pgfqpoint{1.874515in}{0.739656in}}%
\pgfpathlineto{\pgfqpoint{1.874219in}{0.739656in}}%
\pgfpathlineto{\pgfqpoint{1.873923in}{0.739656in}}%
\pgfpathlineto{\pgfqpoint{1.873627in}{0.739656in}}%
\pgfpathlineto{\pgfqpoint{1.873331in}{0.739656in}}%
\pgfpathlineto{\pgfqpoint{1.873035in}{0.739656in}}%
\pgfpathlineto{\pgfqpoint{1.872739in}{0.739656in}}%
\pgfpathlineto{\pgfqpoint{1.872443in}{0.739656in}}%
\pgfpathlineto{\pgfqpoint{1.872147in}{0.739656in}}%
\pgfpathlineto{\pgfqpoint{1.871851in}{0.739656in}}%
\pgfpathlineto{\pgfqpoint{1.871555in}{0.739656in}}%
\pgfpathlineto{\pgfqpoint{1.871259in}{0.739656in}}%
\pgfpathlineto{\pgfqpoint{1.870963in}{0.739656in}}%
\pgfpathlineto{\pgfqpoint{1.870667in}{0.739656in}}%
\pgfpathlineto{\pgfqpoint{1.870371in}{0.739656in}}%
\pgfpathlineto{\pgfqpoint{1.870075in}{0.739656in}}%
\pgfpathlineto{\pgfqpoint{1.869779in}{0.739656in}}%
\pgfpathlineto{\pgfqpoint{1.869483in}{0.739656in}}%
\pgfpathlineto{\pgfqpoint{1.869187in}{0.739656in}}%
\pgfpathlineto{\pgfqpoint{1.868891in}{0.739656in}}%
\pgfpathlineto{\pgfqpoint{1.868595in}{0.739656in}}%
\pgfpathlineto{\pgfqpoint{1.868299in}{0.739656in}}%
\pgfpathlineto{\pgfqpoint{1.868003in}{0.739656in}}%
\pgfpathlineto{\pgfqpoint{1.867707in}{0.739656in}}%
\pgfpathlineto{\pgfqpoint{1.867411in}{0.739656in}}%
\pgfpathlineto{\pgfqpoint{1.867115in}{0.739656in}}%
\pgfpathlineto{\pgfqpoint{1.866819in}{0.739656in}}%
\pgfpathlineto{\pgfqpoint{1.866523in}{0.739656in}}%
\pgfpathlineto{\pgfqpoint{1.866227in}{0.739656in}}%
\pgfpathlineto{\pgfqpoint{1.865931in}{0.739656in}}%
\pgfpathlineto{\pgfqpoint{1.865635in}{0.739656in}}%
\pgfpathlineto{\pgfqpoint{1.865339in}{0.739656in}}%
\pgfpathlineto{\pgfqpoint{1.865043in}{0.739656in}}%
\pgfpathlineto{\pgfqpoint{1.864747in}{0.739656in}}%
\pgfpathlineto{\pgfqpoint{1.864451in}{0.739656in}}%
\pgfpathlineto{\pgfqpoint{1.864155in}{0.739656in}}%
\pgfpathlineto{\pgfqpoint{1.863859in}{0.739656in}}%
\pgfpathlineto{\pgfqpoint{1.863563in}{0.739656in}}%
\pgfpathlineto{\pgfqpoint{1.863267in}{0.739656in}}%
\pgfpathlineto{\pgfqpoint{1.862971in}{0.739656in}}%
\pgfpathlineto{\pgfqpoint{1.862675in}{0.739656in}}%
\pgfpathlineto{\pgfqpoint{1.862379in}{0.739656in}}%
\pgfpathlineto{\pgfqpoint{1.862083in}{0.739656in}}%
\pgfpathlineto{\pgfqpoint{1.861787in}{0.739656in}}%
\pgfpathlineto{\pgfqpoint{1.861491in}{0.739656in}}%
\pgfpathlineto{\pgfqpoint{1.861194in}{0.739656in}}%
\pgfpathlineto{\pgfqpoint{1.860898in}{0.739656in}}%
\pgfpathlineto{\pgfqpoint{1.860602in}{0.739656in}}%
\pgfpathlineto{\pgfqpoint{1.860306in}{0.739656in}}%
\pgfpathlineto{\pgfqpoint{1.860010in}{0.739656in}}%
\pgfpathlineto{\pgfqpoint{1.859714in}{0.739656in}}%
\pgfpathlineto{\pgfqpoint{1.859418in}{0.739656in}}%
\pgfpathlineto{\pgfqpoint{1.859122in}{0.739656in}}%
\pgfpathlineto{\pgfqpoint{1.858826in}{0.739656in}}%
\pgfpathlineto{\pgfqpoint{1.858530in}{0.739656in}}%
\pgfpathlineto{\pgfqpoint{1.858234in}{0.739656in}}%
\pgfpathlineto{\pgfqpoint{1.857938in}{0.739656in}}%
\pgfpathlineto{\pgfqpoint{1.857642in}{0.739656in}}%
\pgfpathlineto{\pgfqpoint{1.857346in}{0.739656in}}%
\pgfpathlineto{\pgfqpoint{1.857050in}{0.739656in}}%
\pgfpathlineto{\pgfqpoint{1.856754in}{0.739656in}}%
\pgfpathlineto{\pgfqpoint{1.856458in}{0.739656in}}%
\pgfpathlineto{\pgfqpoint{1.856162in}{0.739656in}}%
\pgfpathlineto{\pgfqpoint{1.855866in}{0.739656in}}%
\pgfpathlineto{\pgfqpoint{1.855570in}{0.739656in}}%
\pgfpathlineto{\pgfqpoint{1.855274in}{0.739656in}}%
\pgfpathlineto{\pgfqpoint{1.854978in}{0.739656in}}%
\pgfpathlineto{\pgfqpoint{1.854682in}{0.739656in}}%
\pgfpathlineto{\pgfqpoint{1.854386in}{0.739656in}}%
\pgfpathlineto{\pgfqpoint{1.854090in}{0.739656in}}%
\pgfpathlineto{\pgfqpoint{1.853794in}{0.739656in}}%
\pgfpathlineto{\pgfqpoint{1.853498in}{0.739656in}}%
\pgfpathlineto{\pgfqpoint{1.853202in}{0.739656in}}%
\pgfpathlineto{\pgfqpoint{1.852906in}{0.739656in}}%
\pgfpathlineto{\pgfqpoint{1.852610in}{0.739656in}}%
\pgfpathlineto{\pgfqpoint{1.852314in}{0.739656in}}%
\pgfpathlineto{\pgfqpoint{1.852018in}{0.739656in}}%
\pgfpathlineto{\pgfqpoint{1.851722in}{0.739656in}}%
\pgfpathlineto{\pgfqpoint{1.851426in}{0.739656in}}%
\pgfpathlineto{\pgfqpoint{1.851130in}{0.739656in}}%
\pgfpathlineto{\pgfqpoint{1.850834in}{0.739656in}}%
\pgfpathlineto{\pgfqpoint{1.850538in}{0.739656in}}%
\pgfpathlineto{\pgfqpoint{1.850242in}{0.739656in}}%
\pgfpathlineto{\pgfqpoint{1.849946in}{0.739656in}}%
\pgfpathlineto{\pgfqpoint{1.849650in}{0.739656in}}%
\pgfpathlineto{\pgfqpoint{1.849354in}{0.739656in}}%
\pgfpathlineto{\pgfqpoint{1.849058in}{0.739656in}}%
\pgfpathlineto{\pgfqpoint{1.848762in}{0.739656in}}%
\pgfpathlineto{\pgfqpoint{1.848466in}{0.739656in}}%
\pgfpathlineto{\pgfqpoint{1.848170in}{0.739656in}}%
\pgfpathlineto{\pgfqpoint{1.847874in}{0.739656in}}%
\pgfpathlineto{\pgfqpoint{1.847578in}{0.739656in}}%
\pgfpathlineto{\pgfqpoint{1.847282in}{0.739656in}}%
\pgfpathlineto{\pgfqpoint{1.846986in}{0.739656in}}%
\pgfpathlineto{\pgfqpoint{1.846690in}{0.739656in}}%
\pgfpathlineto{\pgfqpoint{1.846394in}{0.739656in}}%
\pgfpathlineto{\pgfqpoint{1.846098in}{0.739656in}}%
\pgfpathlineto{\pgfqpoint{1.845802in}{0.739656in}}%
\pgfpathlineto{\pgfqpoint{1.845506in}{0.739656in}}%
\pgfpathlineto{\pgfqpoint{1.845210in}{0.739656in}}%
\pgfpathlineto{\pgfqpoint{1.844914in}{0.739656in}}%
\pgfpathlineto{\pgfqpoint{1.844618in}{0.739656in}}%
\pgfpathlineto{\pgfqpoint{1.844322in}{0.739656in}}%
\pgfpathlineto{\pgfqpoint{1.844026in}{0.739656in}}%
\pgfpathlineto{\pgfqpoint{1.843730in}{0.739656in}}%
\pgfpathlineto{\pgfqpoint{1.843434in}{0.739656in}}%
\pgfpathlineto{\pgfqpoint{1.843138in}{0.739656in}}%
\pgfpathlineto{\pgfqpoint{1.842842in}{0.739656in}}%
\pgfpathlineto{\pgfqpoint{1.842546in}{0.739656in}}%
\pgfpathlineto{\pgfqpoint{1.842250in}{0.739656in}}%
\pgfpathlineto{\pgfqpoint{1.841954in}{0.739656in}}%
\pgfpathlineto{\pgfqpoint{1.841658in}{0.739656in}}%
\pgfpathlineto{\pgfqpoint{1.841362in}{0.739656in}}%
\pgfpathlineto{\pgfqpoint{1.841066in}{0.739656in}}%
\pgfpathlineto{\pgfqpoint{1.840770in}{0.739656in}}%
\pgfpathlineto{\pgfqpoint{1.840474in}{0.739656in}}%
\pgfpathlineto{\pgfqpoint{1.840178in}{0.739656in}}%
\pgfpathlineto{\pgfqpoint{1.839882in}{0.739656in}}%
\pgfpathlineto{\pgfqpoint{1.839586in}{0.739656in}}%
\pgfpathlineto{\pgfqpoint{1.839290in}{0.739656in}}%
\pgfpathlineto{\pgfqpoint{1.838994in}{0.739656in}}%
\pgfpathlineto{\pgfqpoint{1.838698in}{0.739656in}}%
\pgfpathlineto{\pgfqpoint{1.838402in}{0.739656in}}%
\pgfpathlineto{\pgfqpoint{1.838106in}{0.739656in}}%
\pgfpathlineto{\pgfqpoint{1.837810in}{0.739656in}}%
\pgfpathlineto{\pgfqpoint{1.837514in}{0.739656in}}%
\pgfpathlineto{\pgfqpoint{1.837218in}{0.739656in}}%
\pgfpathlineto{\pgfqpoint{1.836922in}{0.739656in}}%
\pgfpathlineto{\pgfqpoint{1.836626in}{0.739656in}}%
\pgfpathlineto{\pgfqpoint{1.836330in}{0.739656in}}%
\pgfpathlineto{\pgfqpoint{1.836034in}{0.739656in}}%
\pgfpathlineto{\pgfqpoint{1.835738in}{0.739656in}}%
\pgfpathlineto{\pgfqpoint{1.835442in}{0.739656in}}%
\pgfpathlineto{\pgfqpoint{1.835146in}{0.739656in}}%
\pgfpathlineto{\pgfqpoint{1.834850in}{0.739656in}}%
\pgfpathlineto{\pgfqpoint{1.834554in}{0.739656in}}%
\pgfpathlineto{\pgfqpoint{1.834258in}{0.739656in}}%
\pgfpathlineto{\pgfqpoint{1.833962in}{0.739656in}}%
\pgfpathlineto{\pgfqpoint{1.833666in}{0.739656in}}%
\pgfpathlineto{\pgfqpoint{1.833370in}{0.739656in}}%
\pgfpathlineto{\pgfqpoint{1.833074in}{0.739656in}}%
\pgfpathlineto{\pgfqpoint{1.832778in}{0.739656in}}%
\pgfpathlineto{\pgfqpoint{1.832482in}{0.739656in}}%
\pgfpathlineto{\pgfqpoint{1.832186in}{0.739656in}}%
\pgfpathlineto{\pgfqpoint{1.831890in}{0.739656in}}%
\pgfpathlineto{\pgfqpoint{1.831594in}{0.739656in}}%
\pgfpathlineto{\pgfqpoint{1.831298in}{0.739656in}}%
\pgfpathlineto{\pgfqpoint{1.831002in}{0.739656in}}%
\pgfpathlineto{\pgfqpoint{1.830706in}{0.739656in}}%
\pgfpathlineto{\pgfqpoint{1.830410in}{0.739656in}}%
\pgfpathlineto{\pgfqpoint{1.830114in}{0.739656in}}%
\pgfpathlineto{\pgfqpoint{1.829818in}{0.739656in}}%
\pgfpathlineto{\pgfqpoint{1.829522in}{0.739656in}}%
\pgfpathlineto{\pgfqpoint{1.829226in}{0.739656in}}%
\pgfpathlineto{\pgfqpoint{1.828930in}{0.739656in}}%
\pgfpathlineto{\pgfqpoint{1.828634in}{0.739656in}}%
\pgfpathlineto{\pgfqpoint{1.828338in}{0.739656in}}%
\pgfpathlineto{\pgfqpoint{1.828042in}{0.739656in}}%
\pgfpathlineto{\pgfqpoint{1.827746in}{0.739656in}}%
\pgfpathlineto{\pgfqpoint{1.827450in}{0.739656in}}%
\pgfpathlineto{\pgfqpoint{1.827154in}{0.739656in}}%
\pgfpathlineto{\pgfqpoint{1.826858in}{0.739656in}}%
\pgfpathlineto{\pgfqpoint{1.826562in}{0.739656in}}%
\pgfpathlineto{\pgfqpoint{1.826266in}{0.739656in}}%
\pgfpathlineto{\pgfqpoint{1.825970in}{0.739656in}}%
\pgfpathlineto{\pgfqpoint{1.825674in}{0.739656in}}%
\pgfpathlineto{\pgfqpoint{1.825378in}{0.739656in}}%
\pgfpathlineto{\pgfqpoint{1.825082in}{0.739656in}}%
\pgfpathlineto{\pgfqpoint{1.824786in}{0.739656in}}%
\pgfpathlineto{\pgfqpoint{1.824490in}{0.739656in}}%
\pgfpathlineto{\pgfqpoint{1.824194in}{0.739656in}}%
\pgfpathlineto{\pgfqpoint{1.823898in}{0.739656in}}%
\pgfpathlineto{\pgfqpoint{1.823602in}{0.739656in}}%
\pgfpathlineto{\pgfqpoint{1.823306in}{0.739656in}}%
\pgfpathlineto{\pgfqpoint{1.823010in}{0.739656in}}%
\pgfpathlineto{\pgfqpoint{1.822714in}{0.739656in}}%
\pgfpathlineto{\pgfqpoint{1.822418in}{0.739656in}}%
\pgfpathlineto{\pgfqpoint{1.822122in}{0.739656in}}%
\pgfpathlineto{\pgfqpoint{1.821826in}{0.739656in}}%
\pgfpathlineto{\pgfqpoint{1.821530in}{0.739656in}}%
\pgfpathlineto{\pgfqpoint{1.821234in}{0.739656in}}%
\pgfpathlineto{\pgfqpoint{1.820938in}{0.739656in}}%
\pgfpathlineto{\pgfqpoint{1.820642in}{0.739656in}}%
\pgfpathlineto{\pgfqpoint{1.820346in}{0.739656in}}%
\pgfpathlineto{\pgfqpoint{1.820050in}{0.739656in}}%
\pgfpathlineto{\pgfqpoint{1.819754in}{0.739656in}}%
\pgfpathlineto{\pgfqpoint{1.819458in}{0.739656in}}%
\pgfpathlineto{\pgfqpoint{1.819162in}{0.739656in}}%
\pgfpathlineto{\pgfqpoint{1.818866in}{0.739656in}}%
\pgfpathlineto{\pgfqpoint{1.818570in}{0.739656in}}%
\pgfpathlineto{\pgfqpoint{1.818274in}{0.739656in}}%
\pgfpathlineto{\pgfqpoint{1.817978in}{0.739656in}}%
\pgfpathlineto{\pgfqpoint{1.817682in}{0.739656in}}%
\pgfpathlineto{\pgfqpoint{1.817386in}{0.739656in}}%
\pgfpathlineto{\pgfqpoint{1.817090in}{0.739656in}}%
\pgfpathlineto{\pgfqpoint{1.816794in}{0.739656in}}%
\pgfpathlineto{\pgfqpoint{1.816498in}{0.739656in}}%
\pgfpathlineto{\pgfqpoint{1.816202in}{0.739656in}}%
\pgfpathlineto{\pgfqpoint{1.815906in}{0.739656in}}%
\pgfpathlineto{\pgfqpoint{1.815610in}{0.739656in}}%
\pgfpathlineto{\pgfqpoint{1.815314in}{0.739656in}}%
\pgfpathlineto{\pgfqpoint{1.815018in}{0.739656in}}%
\pgfpathlineto{\pgfqpoint{1.814722in}{0.739656in}}%
\pgfpathlineto{\pgfqpoint{1.814426in}{0.739656in}}%
\pgfpathlineto{\pgfqpoint{1.814130in}{0.739656in}}%
\pgfpathlineto{\pgfqpoint{1.813834in}{0.739656in}}%
\pgfpathlineto{\pgfqpoint{1.813538in}{0.739656in}}%
\pgfpathlineto{\pgfqpoint{1.813242in}{0.739656in}}%
\pgfpathlineto{\pgfqpoint{1.812946in}{0.739656in}}%
\pgfpathlineto{\pgfqpoint{1.812650in}{0.739656in}}%
\pgfpathlineto{\pgfqpoint{1.812354in}{0.739656in}}%
\pgfpathlineto{\pgfqpoint{1.812058in}{0.739656in}}%
\pgfpathlineto{\pgfqpoint{1.811762in}{0.739656in}}%
\pgfpathlineto{\pgfqpoint{1.811466in}{0.739656in}}%
\pgfpathlineto{\pgfqpoint{1.811170in}{0.739656in}}%
\pgfpathlineto{\pgfqpoint{1.810874in}{0.739656in}}%
\pgfpathlineto{\pgfqpoint{1.810578in}{0.739656in}}%
\pgfpathlineto{\pgfqpoint{1.810282in}{0.739656in}}%
\pgfpathlineto{\pgfqpoint{1.809986in}{0.739656in}}%
\pgfpathlineto{\pgfqpoint{1.809690in}{0.739656in}}%
\pgfpathlineto{\pgfqpoint{1.809394in}{0.739656in}}%
\pgfpathlineto{\pgfqpoint{1.809098in}{0.739656in}}%
\pgfpathlineto{\pgfqpoint{1.808802in}{0.739656in}}%
\pgfpathlineto{\pgfqpoint{1.808506in}{0.739656in}}%
\pgfpathlineto{\pgfqpoint{1.808210in}{0.739656in}}%
\pgfpathlineto{\pgfqpoint{1.807914in}{0.739656in}}%
\pgfpathlineto{\pgfqpoint{1.807618in}{0.739656in}}%
\pgfpathlineto{\pgfqpoint{1.807322in}{0.739656in}}%
\pgfpathlineto{\pgfqpoint{1.807026in}{0.739656in}}%
\pgfpathlineto{\pgfqpoint{1.806730in}{0.739656in}}%
\pgfpathlineto{\pgfqpoint{1.806434in}{0.739656in}}%
\pgfpathlineto{\pgfqpoint{1.806138in}{0.739656in}}%
\pgfpathlineto{\pgfqpoint{1.805842in}{0.739656in}}%
\pgfpathlineto{\pgfqpoint{1.805546in}{0.739656in}}%
\pgfpathlineto{\pgfqpoint{1.805250in}{0.739656in}}%
\pgfpathlineto{\pgfqpoint{1.804954in}{0.739656in}}%
\pgfpathlineto{\pgfqpoint{1.804658in}{0.739656in}}%
\pgfpathlineto{\pgfqpoint{1.804362in}{0.739656in}}%
\pgfpathlineto{\pgfqpoint{1.804066in}{0.739656in}}%
\pgfpathlineto{\pgfqpoint{1.803770in}{0.739656in}}%
\pgfpathlineto{\pgfqpoint{1.803474in}{0.739656in}}%
\pgfpathlineto{\pgfqpoint{1.803178in}{0.739656in}}%
\pgfpathlineto{\pgfqpoint{1.802882in}{0.739656in}}%
\pgfpathlineto{\pgfqpoint{1.802586in}{0.739656in}}%
\pgfpathlineto{\pgfqpoint{1.802290in}{0.739656in}}%
\pgfpathlineto{\pgfqpoint{1.801994in}{0.739656in}}%
\pgfpathlineto{\pgfqpoint{1.801698in}{0.739656in}}%
\pgfpathlineto{\pgfqpoint{1.801402in}{0.739656in}}%
\pgfpathlineto{\pgfqpoint{1.801106in}{0.739656in}}%
\pgfpathlineto{\pgfqpoint{1.800810in}{0.739656in}}%
\pgfpathlineto{\pgfqpoint{1.800514in}{0.739656in}}%
\pgfpathlineto{\pgfqpoint{1.800218in}{0.739656in}}%
\pgfpathlineto{\pgfqpoint{1.799922in}{0.739656in}}%
\pgfpathlineto{\pgfqpoint{1.799626in}{0.739656in}}%
\pgfpathlineto{\pgfqpoint{1.799330in}{0.739656in}}%
\pgfpathlineto{\pgfqpoint{1.799034in}{0.739656in}}%
\pgfpathlineto{\pgfqpoint{1.798738in}{0.739656in}}%
\pgfpathlineto{\pgfqpoint{1.798442in}{0.739656in}}%
\pgfpathlineto{\pgfqpoint{1.798146in}{0.739656in}}%
\pgfpathlineto{\pgfqpoint{1.797850in}{0.739656in}}%
\pgfpathlineto{\pgfqpoint{1.797554in}{0.739656in}}%
\pgfpathlineto{\pgfqpoint{1.797258in}{0.739656in}}%
\pgfpathlineto{\pgfqpoint{1.796962in}{0.739656in}}%
\pgfpathlineto{\pgfqpoint{1.796666in}{0.739656in}}%
\pgfpathlineto{\pgfqpoint{1.796370in}{0.739656in}}%
\pgfpathlineto{\pgfqpoint{1.796074in}{0.739656in}}%
\pgfpathlineto{\pgfqpoint{1.795778in}{0.739656in}}%
\pgfpathlineto{\pgfqpoint{1.795482in}{0.739656in}}%
\pgfpathlineto{\pgfqpoint{1.795186in}{0.739656in}}%
\pgfpathlineto{\pgfqpoint{1.794890in}{0.739656in}}%
\pgfpathlineto{\pgfqpoint{1.794594in}{0.739656in}}%
\pgfpathlineto{\pgfqpoint{1.794298in}{0.739656in}}%
\pgfpathlineto{\pgfqpoint{1.794001in}{0.739656in}}%
\pgfpathlineto{\pgfqpoint{1.793705in}{0.739656in}}%
\pgfpathlineto{\pgfqpoint{1.793409in}{0.739656in}}%
\pgfpathlineto{\pgfqpoint{1.793113in}{0.739656in}}%
\pgfpathlineto{\pgfqpoint{1.792817in}{0.739656in}}%
\pgfpathlineto{\pgfqpoint{1.792521in}{0.739656in}}%
\pgfpathlineto{\pgfqpoint{1.792225in}{0.739656in}}%
\pgfpathlineto{\pgfqpoint{1.791929in}{0.739656in}}%
\pgfpathlineto{\pgfqpoint{1.791633in}{0.739656in}}%
\pgfpathlineto{\pgfqpoint{1.791337in}{0.739656in}}%
\pgfpathlineto{\pgfqpoint{1.791041in}{0.739656in}}%
\pgfpathlineto{\pgfqpoint{1.790745in}{0.739656in}}%
\pgfpathlineto{\pgfqpoint{1.790449in}{0.739656in}}%
\pgfpathlineto{\pgfqpoint{1.790153in}{0.739656in}}%
\pgfpathlineto{\pgfqpoint{1.789857in}{0.739656in}}%
\pgfpathlineto{\pgfqpoint{1.789561in}{0.739656in}}%
\pgfpathlineto{\pgfqpoint{1.789265in}{0.739656in}}%
\pgfpathlineto{\pgfqpoint{1.788969in}{0.739656in}}%
\pgfpathlineto{\pgfqpoint{1.788673in}{0.739656in}}%
\pgfpathlineto{\pgfqpoint{1.788377in}{0.739656in}}%
\pgfpathlineto{\pgfqpoint{1.788081in}{0.739656in}}%
\pgfpathlineto{\pgfqpoint{1.787785in}{0.739656in}}%
\pgfpathlineto{\pgfqpoint{1.787489in}{0.739656in}}%
\pgfpathlineto{\pgfqpoint{1.787193in}{0.739656in}}%
\pgfpathlineto{\pgfqpoint{1.786897in}{0.739656in}}%
\pgfpathlineto{\pgfqpoint{1.786601in}{0.739656in}}%
\pgfpathlineto{\pgfqpoint{1.786305in}{0.739656in}}%
\pgfpathlineto{\pgfqpoint{1.786009in}{0.739656in}}%
\pgfpathlineto{\pgfqpoint{1.785713in}{0.739656in}}%
\pgfpathlineto{\pgfqpoint{1.785417in}{0.739656in}}%
\pgfpathlineto{\pgfqpoint{1.785121in}{0.739656in}}%
\pgfpathlineto{\pgfqpoint{1.784825in}{0.739656in}}%
\pgfpathlineto{\pgfqpoint{1.784529in}{0.739656in}}%
\pgfpathlineto{\pgfqpoint{1.784233in}{0.739656in}}%
\pgfpathlineto{\pgfqpoint{1.783937in}{0.739656in}}%
\pgfpathlineto{\pgfqpoint{1.783641in}{0.739656in}}%
\pgfpathlineto{\pgfqpoint{1.783345in}{0.739656in}}%
\pgfpathlineto{\pgfqpoint{1.783049in}{0.739656in}}%
\pgfpathlineto{\pgfqpoint{1.782753in}{0.739656in}}%
\pgfpathlineto{\pgfqpoint{1.782457in}{0.739656in}}%
\pgfpathlineto{\pgfqpoint{1.782161in}{0.739656in}}%
\pgfpathlineto{\pgfqpoint{1.781865in}{0.739656in}}%
\pgfpathlineto{\pgfqpoint{1.781569in}{0.739656in}}%
\pgfpathlineto{\pgfqpoint{1.781273in}{0.739656in}}%
\pgfpathlineto{\pgfqpoint{1.780977in}{0.739656in}}%
\pgfpathlineto{\pgfqpoint{1.780681in}{0.739656in}}%
\pgfpathlineto{\pgfqpoint{1.780385in}{0.739656in}}%
\pgfpathlineto{\pgfqpoint{1.780089in}{0.739656in}}%
\pgfpathlineto{\pgfqpoint{1.779793in}{0.739656in}}%
\pgfpathlineto{\pgfqpoint{1.779497in}{0.739656in}}%
\pgfpathlineto{\pgfqpoint{1.779201in}{0.739656in}}%
\pgfpathlineto{\pgfqpoint{1.778905in}{0.739656in}}%
\pgfpathlineto{\pgfqpoint{1.778609in}{0.739656in}}%
\pgfpathlineto{\pgfqpoint{1.778313in}{0.739656in}}%
\pgfpathlineto{\pgfqpoint{1.778017in}{0.739656in}}%
\pgfpathlineto{\pgfqpoint{1.777721in}{0.739656in}}%
\pgfpathlineto{\pgfqpoint{1.777425in}{0.739656in}}%
\pgfpathlineto{\pgfqpoint{1.777129in}{0.739656in}}%
\pgfpathlineto{\pgfqpoint{1.776833in}{0.739656in}}%
\pgfpathlineto{\pgfqpoint{1.776537in}{0.739656in}}%
\pgfpathlineto{\pgfqpoint{1.776241in}{0.739656in}}%
\pgfpathlineto{\pgfqpoint{1.775945in}{0.739656in}}%
\pgfpathlineto{\pgfqpoint{1.775649in}{0.739656in}}%
\pgfpathlineto{\pgfqpoint{1.775353in}{0.739656in}}%
\pgfpathlineto{\pgfqpoint{1.775057in}{0.739656in}}%
\pgfpathlineto{\pgfqpoint{1.774761in}{0.739656in}}%
\pgfpathlineto{\pgfqpoint{1.774465in}{0.739656in}}%
\pgfpathlineto{\pgfqpoint{1.774169in}{0.739656in}}%
\pgfpathlineto{\pgfqpoint{1.773873in}{0.739656in}}%
\pgfpathlineto{\pgfqpoint{1.773577in}{0.739656in}}%
\pgfpathlineto{\pgfqpoint{1.773281in}{0.739656in}}%
\pgfpathlineto{\pgfqpoint{1.772985in}{0.739656in}}%
\pgfpathlineto{\pgfqpoint{1.772689in}{0.739656in}}%
\pgfpathlineto{\pgfqpoint{1.772393in}{0.739656in}}%
\pgfpathlineto{\pgfqpoint{1.772097in}{0.739656in}}%
\pgfpathlineto{\pgfqpoint{1.771801in}{0.739656in}}%
\pgfpathlineto{\pgfqpoint{1.771505in}{0.739656in}}%
\pgfpathlineto{\pgfqpoint{1.771209in}{0.739656in}}%
\pgfpathlineto{\pgfqpoint{1.770913in}{0.739656in}}%
\pgfpathlineto{\pgfqpoint{1.770617in}{0.739656in}}%
\pgfpathlineto{\pgfqpoint{1.770321in}{0.739656in}}%
\pgfpathlineto{\pgfqpoint{1.770025in}{0.739656in}}%
\pgfpathlineto{\pgfqpoint{1.769729in}{0.739656in}}%
\pgfpathlineto{\pgfqpoint{1.769433in}{0.739656in}}%
\pgfpathlineto{\pgfqpoint{1.769137in}{0.739656in}}%
\pgfpathlineto{\pgfqpoint{1.768841in}{0.739656in}}%
\pgfpathlineto{\pgfqpoint{1.768545in}{0.739656in}}%
\pgfpathlineto{\pgfqpoint{1.768249in}{0.739656in}}%
\pgfpathlineto{\pgfqpoint{1.767953in}{0.739656in}}%
\pgfpathlineto{\pgfqpoint{1.767657in}{0.739656in}}%
\pgfpathlineto{\pgfqpoint{1.767361in}{0.739656in}}%
\pgfpathlineto{\pgfqpoint{1.767065in}{0.739656in}}%
\pgfpathlineto{\pgfqpoint{1.766769in}{0.739656in}}%
\pgfpathlineto{\pgfqpoint{1.766473in}{0.739656in}}%
\pgfpathlineto{\pgfqpoint{1.766177in}{0.739656in}}%
\pgfpathlineto{\pgfqpoint{1.765881in}{0.739656in}}%
\pgfpathlineto{\pgfqpoint{1.765585in}{0.739656in}}%
\pgfpathlineto{\pgfqpoint{1.765289in}{0.739656in}}%
\pgfpathlineto{\pgfqpoint{1.764993in}{0.739656in}}%
\pgfpathlineto{\pgfqpoint{1.764697in}{0.739656in}}%
\pgfpathlineto{\pgfqpoint{1.764401in}{0.739656in}}%
\pgfpathlineto{\pgfqpoint{1.764105in}{0.739656in}}%
\pgfpathlineto{\pgfqpoint{1.763809in}{0.739656in}}%
\pgfpathlineto{\pgfqpoint{1.763513in}{0.739656in}}%
\pgfpathlineto{\pgfqpoint{1.763217in}{0.739656in}}%
\pgfpathlineto{\pgfqpoint{1.762921in}{0.739656in}}%
\pgfpathlineto{\pgfqpoint{1.762625in}{0.739656in}}%
\pgfpathlineto{\pgfqpoint{1.762329in}{0.739656in}}%
\pgfpathlineto{\pgfqpoint{1.762033in}{0.739656in}}%
\pgfpathlineto{\pgfqpoint{1.761737in}{0.739656in}}%
\pgfpathlineto{\pgfqpoint{1.761441in}{0.739656in}}%
\pgfpathlineto{\pgfqpoint{1.761145in}{0.739656in}}%
\pgfpathlineto{\pgfqpoint{1.760849in}{0.739656in}}%
\pgfpathlineto{\pgfqpoint{1.760553in}{0.739656in}}%
\pgfpathlineto{\pgfqpoint{1.760257in}{0.739656in}}%
\pgfpathlineto{\pgfqpoint{1.759961in}{0.739656in}}%
\pgfpathlineto{\pgfqpoint{1.759665in}{0.739656in}}%
\pgfpathlineto{\pgfqpoint{1.759369in}{0.739656in}}%
\pgfpathlineto{\pgfqpoint{1.759073in}{0.739656in}}%
\pgfpathlineto{\pgfqpoint{1.758777in}{0.739656in}}%
\pgfpathlineto{\pgfqpoint{1.758481in}{0.739656in}}%
\pgfpathlineto{\pgfqpoint{1.758185in}{0.739656in}}%
\pgfpathlineto{\pgfqpoint{1.757889in}{0.739656in}}%
\pgfpathlineto{\pgfqpoint{1.757593in}{0.739656in}}%
\pgfpathlineto{\pgfqpoint{1.757297in}{0.739656in}}%
\pgfpathlineto{\pgfqpoint{1.757001in}{0.739656in}}%
\pgfpathlineto{\pgfqpoint{1.756705in}{0.739656in}}%
\pgfpathlineto{\pgfqpoint{1.756409in}{0.739656in}}%
\pgfpathlineto{\pgfqpoint{1.756113in}{0.739656in}}%
\pgfpathlineto{\pgfqpoint{1.755817in}{0.739656in}}%
\pgfpathlineto{\pgfqpoint{1.755521in}{0.739656in}}%
\pgfpathlineto{\pgfqpoint{1.755225in}{0.739656in}}%
\pgfpathlineto{\pgfqpoint{1.754929in}{0.739656in}}%
\pgfpathlineto{\pgfqpoint{1.754633in}{0.739656in}}%
\pgfpathlineto{\pgfqpoint{1.754337in}{0.739656in}}%
\pgfpathlineto{\pgfqpoint{1.754041in}{0.739656in}}%
\pgfpathlineto{\pgfqpoint{1.753745in}{0.739656in}}%
\pgfpathlineto{\pgfqpoint{1.753449in}{0.739656in}}%
\pgfpathlineto{\pgfqpoint{1.753153in}{0.739656in}}%
\pgfpathlineto{\pgfqpoint{1.752857in}{0.739656in}}%
\pgfpathlineto{\pgfqpoint{1.752561in}{0.739656in}}%
\pgfpathlineto{\pgfqpoint{1.752265in}{0.739656in}}%
\pgfpathlineto{\pgfqpoint{1.751969in}{0.739656in}}%
\pgfpathlineto{\pgfqpoint{1.751673in}{0.739656in}}%
\pgfpathlineto{\pgfqpoint{1.751377in}{0.739656in}}%
\pgfpathlineto{\pgfqpoint{1.751081in}{0.739656in}}%
\pgfpathlineto{\pgfqpoint{1.750785in}{0.739656in}}%
\pgfpathlineto{\pgfqpoint{1.750489in}{0.739656in}}%
\pgfpathlineto{\pgfqpoint{1.750193in}{0.739656in}}%
\pgfpathlineto{\pgfqpoint{1.749897in}{0.739656in}}%
\pgfpathlineto{\pgfqpoint{1.749601in}{0.739656in}}%
\pgfpathlineto{\pgfqpoint{1.749305in}{0.739656in}}%
\pgfpathlineto{\pgfqpoint{1.749009in}{0.739656in}}%
\pgfpathlineto{\pgfqpoint{1.748713in}{0.739656in}}%
\pgfpathlineto{\pgfqpoint{1.748417in}{0.739656in}}%
\pgfpathlineto{\pgfqpoint{1.748121in}{0.739656in}}%
\pgfpathlineto{\pgfqpoint{1.747825in}{0.739656in}}%
\pgfpathlineto{\pgfqpoint{1.747529in}{0.739656in}}%
\pgfpathlineto{\pgfqpoint{1.747233in}{0.739656in}}%
\pgfpathlineto{\pgfqpoint{1.746937in}{0.739656in}}%
\pgfpathlineto{\pgfqpoint{1.746641in}{0.739656in}}%
\pgfpathlineto{\pgfqpoint{1.746345in}{0.739656in}}%
\pgfpathlineto{\pgfqpoint{1.746049in}{0.739656in}}%
\pgfpathlineto{\pgfqpoint{1.745753in}{0.739656in}}%
\pgfpathlineto{\pgfqpoint{1.745457in}{0.739656in}}%
\pgfpathlineto{\pgfqpoint{1.745161in}{0.739656in}}%
\pgfpathlineto{\pgfqpoint{1.744865in}{0.739656in}}%
\pgfpathlineto{\pgfqpoint{1.744569in}{0.739656in}}%
\pgfpathlineto{\pgfqpoint{1.744273in}{0.739656in}}%
\pgfpathlineto{\pgfqpoint{1.743977in}{0.739656in}}%
\pgfpathlineto{\pgfqpoint{1.743681in}{0.739656in}}%
\pgfpathlineto{\pgfqpoint{1.743385in}{0.739656in}}%
\pgfpathlineto{\pgfqpoint{1.743089in}{0.739656in}}%
\pgfpathlineto{\pgfqpoint{1.742793in}{0.739656in}}%
\pgfpathlineto{\pgfqpoint{1.742497in}{0.739656in}}%
\pgfpathlineto{\pgfqpoint{1.742201in}{0.739656in}}%
\pgfpathlineto{\pgfqpoint{1.741905in}{0.739656in}}%
\pgfpathlineto{\pgfqpoint{1.741609in}{0.739656in}}%
\pgfpathlineto{\pgfqpoint{1.741313in}{0.739656in}}%
\pgfpathlineto{\pgfqpoint{1.741017in}{0.739656in}}%
\pgfpathlineto{\pgfqpoint{1.740721in}{0.739656in}}%
\pgfpathlineto{\pgfqpoint{1.740425in}{0.739656in}}%
\pgfpathlineto{\pgfqpoint{1.740129in}{0.739656in}}%
\pgfpathlineto{\pgfqpoint{1.739833in}{0.739656in}}%
\pgfpathlineto{\pgfqpoint{1.739537in}{0.739656in}}%
\pgfpathlineto{\pgfqpoint{1.739241in}{0.739656in}}%
\pgfpathlineto{\pgfqpoint{1.738945in}{0.739656in}}%
\pgfpathlineto{\pgfqpoint{1.738649in}{0.739656in}}%
\pgfpathlineto{\pgfqpoint{1.738353in}{0.739656in}}%
\pgfpathlineto{\pgfqpoint{1.738057in}{0.739656in}}%
\pgfpathlineto{\pgfqpoint{1.737761in}{0.739656in}}%
\pgfpathlineto{\pgfqpoint{1.737465in}{0.739656in}}%
\pgfpathlineto{\pgfqpoint{1.737169in}{0.739656in}}%
\pgfpathlineto{\pgfqpoint{1.736873in}{0.739656in}}%
\pgfpathlineto{\pgfqpoint{1.736577in}{0.739656in}}%
\pgfpathlineto{\pgfqpoint{1.736281in}{0.739656in}}%
\pgfpathlineto{\pgfqpoint{1.735985in}{0.739656in}}%
\pgfpathlineto{\pgfqpoint{1.735689in}{0.739656in}}%
\pgfpathlineto{\pgfqpoint{1.735393in}{0.739656in}}%
\pgfpathlineto{\pgfqpoint{1.735097in}{0.739656in}}%
\pgfpathlineto{\pgfqpoint{1.734801in}{0.739656in}}%
\pgfpathlineto{\pgfqpoint{1.734505in}{0.739656in}}%
\pgfpathlineto{\pgfqpoint{1.734209in}{0.739656in}}%
\pgfpathlineto{\pgfqpoint{1.733913in}{0.739656in}}%
\pgfpathlineto{\pgfqpoint{1.733617in}{0.739656in}}%
\pgfpathlineto{\pgfqpoint{1.733321in}{0.739656in}}%
\pgfpathlineto{\pgfqpoint{1.733025in}{0.739656in}}%
\pgfpathlineto{\pgfqpoint{1.732729in}{0.739656in}}%
\pgfpathlineto{\pgfqpoint{1.732433in}{0.739656in}}%
\pgfpathlineto{\pgfqpoint{1.732137in}{0.739656in}}%
\pgfpathlineto{\pgfqpoint{1.731841in}{0.739656in}}%
\pgfpathlineto{\pgfqpoint{1.731545in}{0.739656in}}%
\pgfpathlineto{\pgfqpoint{1.731249in}{0.739656in}}%
\pgfpathlineto{\pgfqpoint{1.730953in}{0.739656in}}%
\pgfpathlineto{\pgfqpoint{1.730657in}{0.739656in}}%
\pgfpathlineto{\pgfqpoint{1.730361in}{0.739656in}}%
\pgfpathlineto{\pgfqpoint{1.730065in}{0.739656in}}%
\pgfpathlineto{\pgfqpoint{1.729769in}{0.739656in}}%
\pgfpathlineto{\pgfqpoint{1.729473in}{0.739656in}}%
\pgfpathlineto{\pgfqpoint{1.729177in}{0.739656in}}%
\pgfpathlineto{\pgfqpoint{1.728881in}{0.739656in}}%
\pgfpathlineto{\pgfqpoint{1.728585in}{0.739656in}}%
\pgfpathlineto{\pgfqpoint{1.728289in}{0.739656in}}%
\pgfpathlineto{\pgfqpoint{1.727993in}{0.739656in}}%
\pgfpathlineto{\pgfqpoint{1.727697in}{0.739656in}}%
\pgfpathlineto{\pgfqpoint{1.727401in}{0.739656in}}%
\pgfpathlineto{\pgfqpoint{1.727105in}{0.739656in}}%
\pgfpathlineto{\pgfqpoint{1.726809in}{0.739656in}}%
\pgfpathlineto{\pgfqpoint{1.726512in}{0.739656in}}%
\pgfpathlineto{\pgfqpoint{1.726216in}{0.739656in}}%
\pgfpathlineto{\pgfqpoint{1.725920in}{0.739656in}}%
\pgfpathlineto{\pgfqpoint{1.725624in}{0.739656in}}%
\pgfpathlineto{\pgfqpoint{1.725328in}{0.739656in}}%
\pgfpathlineto{\pgfqpoint{1.725032in}{0.739656in}}%
\pgfpathlineto{\pgfqpoint{1.724736in}{0.739656in}}%
\pgfpathlineto{\pgfqpoint{1.724440in}{0.739656in}}%
\pgfpathlineto{\pgfqpoint{1.724144in}{0.739656in}}%
\pgfpathlineto{\pgfqpoint{1.723848in}{0.739656in}}%
\pgfpathlineto{\pgfqpoint{1.723552in}{0.739656in}}%
\pgfpathlineto{\pgfqpoint{1.723256in}{0.739656in}}%
\pgfpathlineto{\pgfqpoint{1.722960in}{0.739656in}}%
\pgfpathlineto{\pgfqpoint{1.722664in}{0.739656in}}%
\pgfpathlineto{\pgfqpoint{1.722368in}{0.739656in}}%
\pgfpathlineto{\pgfqpoint{1.722072in}{0.739656in}}%
\pgfpathlineto{\pgfqpoint{1.721776in}{0.739656in}}%
\pgfpathlineto{\pgfqpoint{1.721480in}{0.739656in}}%
\pgfpathlineto{\pgfqpoint{1.721184in}{0.739656in}}%
\pgfpathlineto{\pgfqpoint{1.720888in}{0.739656in}}%
\pgfpathlineto{\pgfqpoint{1.720592in}{0.739656in}}%
\pgfpathlineto{\pgfqpoint{1.720296in}{0.739656in}}%
\pgfpathlineto{\pgfqpoint{1.720000in}{0.739656in}}%
\pgfpathlineto{\pgfqpoint{1.719704in}{0.739656in}}%
\pgfpathlineto{\pgfqpoint{1.719408in}{0.739656in}}%
\pgfpathlineto{\pgfqpoint{1.719112in}{0.739656in}}%
\pgfpathlineto{\pgfqpoint{1.718816in}{0.739656in}}%
\pgfpathlineto{\pgfqpoint{1.718520in}{0.739656in}}%
\pgfpathlineto{\pgfqpoint{1.718224in}{0.739656in}}%
\pgfpathlineto{\pgfqpoint{1.717928in}{0.739656in}}%
\pgfpathlineto{\pgfqpoint{1.717632in}{0.739656in}}%
\pgfpathlineto{\pgfqpoint{1.717336in}{0.739656in}}%
\pgfpathlineto{\pgfqpoint{1.717040in}{0.739656in}}%
\pgfpathlineto{\pgfqpoint{1.716744in}{0.739656in}}%
\pgfpathlineto{\pgfqpoint{1.716448in}{0.739656in}}%
\pgfpathlineto{\pgfqpoint{1.716152in}{0.739656in}}%
\pgfpathlineto{\pgfqpoint{1.715856in}{0.739656in}}%
\pgfpathlineto{\pgfqpoint{1.715560in}{0.739656in}}%
\pgfpathlineto{\pgfqpoint{1.715264in}{0.739656in}}%
\pgfpathlineto{\pgfqpoint{1.714968in}{0.739656in}}%
\pgfpathlineto{\pgfqpoint{1.714672in}{0.739656in}}%
\pgfpathlineto{\pgfqpoint{1.714376in}{0.739656in}}%
\pgfpathlineto{\pgfqpoint{1.714080in}{0.739656in}}%
\pgfpathlineto{\pgfqpoint{1.713784in}{0.739656in}}%
\pgfpathlineto{\pgfqpoint{1.713488in}{0.739656in}}%
\pgfpathlineto{\pgfqpoint{1.713192in}{0.739656in}}%
\pgfpathlineto{\pgfqpoint{1.712896in}{0.739656in}}%
\pgfpathlineto{\pgfqpoint{1.712600in}{0.739656in}}%
\pgfpathlineto{\pgfqpoint{1.712304in}{0.739656in}}%
\pgfpathlineto{\pgfqpoint{1.712008in}{0.739656in}}%
\pgfpathlineto{\pgfqpoint{1.711712in}{0.739656in}}%
\pgfpathlineto{\pgfqpoint{1.711416in}{0.739656in}}%
\pgfpathlineto{\pgfqpoint{1.711120in}{0.739656in}}%
\pgfpathlineto{\pgfqpoint{1.710824in}{0.739656in}}%
\pgfpathlineto{\pgfqpoint{1.710528in}{0.739656in}}%
\pgfpathlineto{\pgfqpoint{1.710232in}{0.739656in}}%
\pgfpathlineto{\pgfqpoint{1.709936in}{0.739656in}}%
\pgfpathlineto{\pgfqpoint{1.709640in}{0.739656in}}%
\pgfpathlineto{\pgfqpoint{1.709344in}{0.739656in}}%
\pgfpathlineto{\pgfqpoint{1.709048in}{0.739656in}}%
\pgfpathlineto{\pgfqpoint{1.708752in}{0.739656in}}%
\pgfpathlineto{\pgfqpoint{1.708456in}{0.739656in}}%
\pgfpathlineto{\pgfqpoint{1.708160in}{0.739656in}}%
\pgfpathlineto{\pgfqpoint{1.707864in}{0.739656in}}%
\pgfpathlineto{\pgfqpoint{1.707568in}{0.739656in}}%
\pgfpathlineto{\pgfqpoint{1.707272in}{0.739656in}}%
\pgfpathlineto{\pgfqpoint{1.706976in}{0.739656in}}%
\pgfpathlineto{\pgfqpoint{1.706680in}{0.739656in}}%
\pgfpathlineto{\pgfqpoint{1.706384in}{0.739656in}}%
\pgfpathlineto{\pgfqpoint{1.706088in}{0.739656in}}%
\pgfpathlineto{\pgfqpoint{1.705792in}{0.739656in}}%
\pgfpathlineto{\pgfqpoint{1.705496in}{0.739656in}}%
\pgfpathlineto{\pgfqpoint{1.705200in}{0.739656in}}%
\pgfpathlineto{\pgfqpoint{1.704904in}{0.739656in}}%
\pgfpathlineto{\pgfqpoint{1.704608in}{0.739656in}}%
\pgfpathlineto{\pgfqpoint{1.704312in}{0.739656in}}%
\pgfpathlineto{\pgfqpoint{1.704016in}{0.739656in}}%
\pgfpathlineto{\pgfqpoint{1.703720in}{0.739656in}}%
\pgfpathlineto{\pgfqpoint{1.703424in}{0.739656in}}%
\pgfpathlineto{\pgfqpoint{1.703128in}{0.739656in}}%
\pgfpathlineto{\pgfqpoint{1.702832in}{0.739656in}}%
\pgfpathlineto{\pgfqpoint{1.702536in}{0.739656in}}%
\pgfpathlineto{\pgfqpoint{1.702240in}{0.739656in}}%
\pgfpathlineto{\pgfqpoint{1.701944in}{0.739656in}}%
\pgfpathlineto{\pgfqpoint{1.701648in}{0.739656in}}%
\pgfpathlineto{\pgfqpoint{1.701352in}{0.739656in}}%
\pgfpathlineto{\pgfqpoint{1.701056in}{0.739656in}}%
\pgfpathlineto{\pgfqpoint{1.700760in}{0.739656in}}%
\pgfpathlineto{\pgfqpoint{1.700464in}{0.739656in}}%
\pgfpathlineto{\pgfqpoint{1.700168in}{0.739656in}}%
\pgfpathlineto{\pgfqpoint{1.699872in}{0.739656in}}%
\pgfpathlineto{\pgfqpoint{1.699576in}{0.739656in}}%
\pgfpathlineto{\pgfqpoint{1.699280in}{0.739656in}}%
\pgfpathlineto{\pgfqpoint{1.698984in}{0.739656in}}%
\pgfpathlineto{\pgfqpoint{1.698688in}{0.739656in}}%
\pgfpathlineto{\pgfqpoint{1.698392in}{0.739656in}}%
\pgfpathlineto{\pgfqpoint{1.698096in}{0.739656in}}%
\pgfpathlineto{\pgfqpoint{1.697800in}{0.739656in}}%
\pgfpathlineto{\pgfqpoint{1.697504in}{0.739656in}}%
\pgfpathlineto{\pgfqpoint{1.697208in}{0.739656in}}%
\pgfpathlineto{\pgfqpoint{1.696912in}{0.739656in}}%
\pgfpathlineto{\pgfqpoint{1.696616in}{0.739656in}}%
\pgfpathlineto{\pgfqpoint{1.696320in}{0.739656in}}%
\pgfpathlineto{\pgfqpoint{1.696024in}{0.739656in}}%
\pgfpathlineto{\pgfqpoint{1.695728in}{0.739656in}}%
\pgfpathlineto{\pgfqpoint{1.695432in}{0.739656in}}%
\pgfpathlineto{\pgfqpoint{1.695136in}{0.739656in}}%
\pgfpathlineto{\pgfqpoint{1.694840in}{0.739656in}}%
\pgfpathlineto{\pgfqpoint{1.694544in}{0.739656in}}%
\pgfpathlineto{\pgfqpoint{1.694248in}{0.739656in}}%
\pgfpathlineto{\pgfqpoint{1.693952in}{0.739656in}}%
\pgfpathlineto{\pgfqpoint{1.693656in}{0.739656in}}%
\pgfpathlineto{\pgfqpoint{1.693360in}{0.739656in}}%
\pgfpathlineto{\pgfqpoint{1.693064in}{0.739656in}}%
\pgfpathlineto{\pgfqpoint{1.692768in}{0.739656in}}%
\pgfpathlineto{\pgfqpoint{1.692472in}{0.739656in}}%
\pgfpathlineto{\pgfqpoint{1.692176in}{0.739656in}}%
\pgfpathlineto{\pgfqpoint{1.691880in}{0.739656in}}%
\pgfpathlineto{\pgfqpoint{1.691584in}{0.739656in}}%
\pgfpathlineto{\pgfqpoint{1.691288in}{0.739656in}}%
\pgfpathlineto{\pgfqpoint{1.690992in}{0.739656in}}%
\pgfpathlineto{\pgfqpoint{1.690696in}{0.739656in}}%
\pgfpathlineto{\pgfqpoint{1.690400in}{0.739656in}}%
\pgfpathlineto{\pgfqpoint{1.690104in}{0.739656in}}%
\pgfpathlineto{\pgfqpoint{1.689808in}{0.739656in}}%
\pgfpathlineto{\pgfqpoint{1.689512in}{0.739656in}}%
\pgfpathlineto{\pgfqpoint{1.689216in}{0.739656in}}%
\pgfpathlineto{\pgfqpoint{1.688920in}{0.739656in}}%
\pgfpathlineto{\pgfqpoint{1.688624in}{0.739656in}}%
\pgfpathlineto{\pgfqpoint{1.688328in}{0.739656in}}%
\pgfpathlineto{\pgfqpoint{1.688032in}{0.739656in}}%
\pgfpathlineto{\pgfqpoint{1.687736in}{0.739656in}}%
\pgfpathlineto{\pgfqpoint{1.687440in}{0.739656in}}%
\pgfpathlineto{\pgfqpoint{1.687144in}{0.739656in}}%
\pgfpathlineto{\pgfqpoint{1.686848in}{0.739656in}}%
\pgfpathlineto{\pgfqpoint{1.686552in}{0.739656in}}%
\pgfpathlineto{\pgfqpoint{1.686256in}{0.739656in}}%
\pgfpathlineto{\pgfqpoint{1.685960in}{0.739656in}}%
\pgfpathlineto{\pgfqpoint{1.685664in}{0.739656in}}%
\pgfpathlineto{\pgfqpoint{1.685368in}{0.739656in}}%
\pgfpathlineto{\pgfqpoint{1.685072in}{0.739656in}}%
\pgfpathlineto{\pgfqpoint{1.684776in}{0.739656in}}%
\pgfpathlineto{\pgfqpoint{1.684480in}{0.739656in}}%
\pgfpathlineto{\pgfqpoint{1.684184in}{0.739656in}}%
\pgfpathlineto{\pgfqpoint{1.683888in}{0.739656in}}%
\pgfpathlineto{\pgfqpoint{1.683592in}{0.739656in}}%
\pgfpathlineto{\pgfqpoint{1.683296in}{0.739656in}}%
\pgfpathlineto{\pgfqpoint{1.683000in}{0.739656in}}%
\pgfpathlineto{\pgfqpoint{1.682704in}{0.739656in}}%
\pgfpathlineto{\pgfqpoint{1.682408in}{0.739656in}}%
\pgfpathlineto{\pgfqpoint{1.682112in}{0.739656in}}%
\pgfpathlineto{\pgfqpoint{1.681816in}{0.739656in}}%
\pgfpathlineto{\pgfqpoint{1.681520in}{0.739656in}}%
\pgfpathlineto{\pgfqpoint{1.681224in}{0.739656in}}%
\pgfpathlineto{\pgfqpoint{1.680928in}{0.739656in}}%
\pgfpathlineto{\pgfqpoint{1.680632in}{0.739656in}}%
\pgfpathlineto{\pgfqpoint{1.680336in}{0.739656in}}%
\pgfpathlineto{\pgfqpoint{1.680040in}{0.739656in}}%
\pgfpathlineto{\pgfqpoint{1.679744in}{0.739656in}}%
\pgfpathlineto{\pgfqpoint{1.679448in}{0.739656in}}%
\pgfpathlineto{\pgfqpoint{1.679152in}{0.739656in}}%
\pgfpathlineto{\pgfqpoint{1.678856in}{0.739656in}}%
\pgfpathlineto{\pgfqpoint{1.678560in}{0.739656in}}%
\pgfpathlineto{\pgfqpoint{1.678264in}{0.739656in}}%
\pgfpathlineto{\pgfqpoint{1.677968in}{0.739656in}}%
\pgfpathlineto{\pgfqpoint{1.677672in}{0.739656in}}%
\pgfpathlineto{\pgfqpoint{1.677376in}{0.739656in}}%
\pgfpathlineto{\pgfqpoint{1.677080in}{0.739656in}}%
\pgfpathlineto{\pgfqpoint{1.676784in}{0.739656in}}%
\pgfpathlineto{\pgfqpoint{1.676488in}{0.739656in}}%
\pgfpathlineto{\pgfqpoint{1.676192in}{0.739656in}}%
\pgfpathlineto{\pgfqpoint{1.675896in}{0.739656in}}%
\pgfpathlineto{\pgfqpoint{1.675600in}{0.739656in}}%
\pgfpathlineto{\pgfqpoint{1.675304in}{0.739656in}}%
\pgfpathlineto{\pgfqpoint{1.675008in}{0.739656in}}%
\pgfpathlineto{\pgfqpoint{1.674712in}{0.739656in}}%
\pgfpathlineto{\pgfqpoint{1.674416in}{0.739656in}}%
\pgfpathlineto{\pgfqpoint{1.674120in}{0.739656in}}%
\pgfpathlineto{\pgfqpoint{1.673824in}{0.739656in}}%
\pgfpathlineto{\pgfqpoint{1.673528in}{0.739656in}}%
\pgfpathlineto{\pgfqpoint{1.673232in}{0.739656in}}%
\pgfpathlineto{\pgfqpoint{1.672936in}{0.739656in}}%
\pgfpathlineto{\pgfqpoint{1.672640in}{0.739656in}}%
\pgfpathlineto{\pgfqpoint{1.672344in}{0.739656in}}%
\pgfpathlineto{\pgfqpoint{1.672048in}{0.739656in}}%
\pgfpathlineto{\pgfqpoint{1.671752in}{0.739656in}}%
\pgfpathlineto{\pgfqpoint{1.671456in}{0.739656in}}%
\pgfpathlineto{\pgfqpoint{1.671160in}{0.739656in}}%
\pgfpathlineto{\pgfqpoint{1.670864in}{0.739656in}}%
\pgfpathlineto{\pgfqpoint{1.670568in}{0.739656in}}%
\pgfpathlineto{\pgfqpoint{1.670272in}{0.739656in}}%
\pgfpathlineto{\pgfqpoint{1.669976in}{0.739656in}}%
\pgfpathlineto{\pgfqpoint{1.669680in}{0.739656in}}%
\pgfpathlineto{\pgfqpoint{1.669384in}{0.739656in}}%
\pgfpathlineto{\pgfqpoint{1.669088in}{0.739656in}}%
\pgfpathlineto{\pgfqpoint{1.668792in}{0.739656in}}%
\pgfpathlineto{\pgfqpoint{1.668496in}{0.739656in}}%
\pgfpathlineto{\pgfqpoint{1.668200in}{0.739656in}}%
\pgfpathlineto{\pgfqpoint{1.667904in}{0.739656in}}%
\pgfpathlineto{\pgfqpoint{1.667608in}{0.739656in}}%
\pgfpathlineto{\pgfqpoint{1.667312in}{0.739656in}}%
\pgfpathlineto{\pgfqpoint{1.667016in}{0.739656in}}%
\pgfpathlineto{\pgfqpoint{1.666720in}{0.739656in}}%
\pgfpathlineto{\pgfqpoint{1.666424in}{0.739656in}}%
\pgfpathlineto{\pgfqpoint{1.666128in}{0.739656in}}%
\pgfpathlineto{\pgfqpoint{1.665832in}{0.739656in}}%
\pgfpathlineto{\pgfqpoint{1.665536in}{0.739656in}}%
\pgfpathlineto{\pgfqpoint{1.665240in}{0.739656in}}%
\pgfpathlineto{\pgfqpoint{1.664944in}{0.739656in}}%
\pgfpathlineto{\pgfqpoint{1.664648in}{0.739656in}}%
\pgfpathlineto{\pgfqpoint{1.664352in}{0.739656in}}%
\pgfpathlineto{\pgfqpoint{1.664056in}{0.739656in}}%
\pgfpathlineto{\pgfqpoint{1.663760in}{0.739656in}}%
\pgfpathlineto{\pgfqpoint{1.663464in}{0.739656in}}%
\pgfpathlineto{\pgfqpoint{1.663168in}{0.739656in}}%
\pgfpathlineto{\pgfqpoint{1.662872in}{0.739656in}}%
\pgfpathlineto{\pgfqpoint{1.662576in}{0.739656in}}%
\pgfpathlineto{\pgfqpoint{1.662280in}{0.739656in}}%
\pgfpathlineto{\pgfqpoint{1.661984in}{0.739656in}}%
\pgfpathlineto{\pgfqpoint{1.661688in}{0.739656in}}%
\pgfpathlineto{\pgfqpoint{1.661392in}{0.739656in}}%
\pgfpathlineto{\pgfqpoint{1.661096in}{0.739656in}}%
\pgfpathlineto{\pgfqpoint{1.660800in}{0.739656in}}%
\pgfpathlineto{\pgfqpoint{1.660504in}{0.739656in}}%
\pgfpathlineto{\pgfqpoint{1.660208in}{0.739656in}}%
\pgfpathlineto{\pgfqpoint{1.659912in}{0.739656in}}%
\pgfpathlineto{\pgfqpoint{1.659616in}{0.739656in}}%
\pgfpathlineto{\pgfqpoint{1.659320in}{0.739656in}}%
\pgfpathlineto{\pgfqpoint{1.659023in}{0.739656in}}%
\pgfpathlineto{\pgfqpoint{1.658727in}{0.739656in}}%
\pgfpathlineto{\pgfqpoint{1.658431in}{0.739656in}}%
\pgfpathlineto{\pgfqpoint{1.658135in}{0.739656in}}%
\pgfpathlineto{\pgfqpoint{1.657839in}{0.739656in}}%
\pgfpathlineto{\pgfqpoint{1.657543in}{0.739656in}}%
\pgfpathlineto{\pgfqpoint{1.657247in}{0.739656in}}%
\pgfpathlineto{\pgfqpoint{1.656951in}{0.739656in}}%
\pgfpathlineto{\pgfqpoint{1.656655in}{0.739656in}}%
\pgfpathlineto{\pgfqpoint{1.656359in}{0.739656in}}%
\pgfpathlineto{\pgfqpoint{1.656063in}{0.739656in}}%
\pgfpathlineto{\pgfqpoint{1.655767in}{0.739656in}}%
\pgfpathlineto{\pgfqpoint{1.655471in}{0.739656in}}%
\pgfpathlineto{\pgfqpoint{1.655175in}{0.739656in}}%
\pgfpathlineto{\pgfqpoint{1.654879in}{0.739656in}}%
\pgfpathlineto{\pgfqpoint{1.654583in}{0.739656in}}%
\pgfpathlineto{\pgfqpoint{1.654287in}{0.739656in}}%
\pgfpathlineto{\pgfqpoint{1.653991in}{0.739656in}}%
\pgfpathlineto{\pgfqpoint{1.653695in}{0.739656in}}%
\pgfpathlineto{\pgfqpoint{1.653399in}{0.739656in}}%
\pgfpathlineto{\pgfqpoint{1.653103in}{0.739656in}}%
\pgfpathlineto{\pgfqpoint{1.652807in}{0.739656in}}%
\pgfpathlineto{\pgfqpoint{1.652511in}{0.739656in}}%
\pgfpathlineto{\pgfqpoint{1.652215in}{0.739656in}}%
\pgfpathlineto{\pgfqpoint{1.651919in}{0.739656in}}%
\pgfpathlineto{\pgfqpoint{1.651623in}{0.739656in}}%
\pgfpathlineto{\pgfqpoint{1.651327in}{0.739656in}}%
\pgfpathlineto{\pgfqpoint{1.651031in}{0.739656in}}%
\pgfpathlineto{\pgfqpoint{1.650735in}{0.739656in}}%
\pgfpathlineto{\pgfqpoint{1.650439in}{0.739656in}}%
\pgfpathlineto{\pgfqpoint{1.650143in}{0.739656in}}%
\pgfpathlineto{\pgfqpoint{1.649847in}{0.739656in}}%
\pgfpathlineto{\pgfqpoint{1.649551in}{0.739656in}}%
\pgfpathlineto{\pgfqpoint{1.649255in}{0.739656in}}%
\pgfpathlineto{\pgfqpoint{1.648959in}{0.739656in}}%
\pgfpathlineto{\pgfqpoint{1.648663in}{0.739656in}}%
\pgfpathlineto{\pgfqpoint{1.648367in}{0.739656in}}%
\pgfpathlineto{\pgfqpoint{1.648071in}{0.739656in}}%
\pgfpathlineto{\pgfqpoint{1.647775in}{0.739656in}}%
\pgfpathlineto{\pgfqpoint{1.647479in}{0.739656in}}%
\pgfpathlineto{\pgfqpoint{1.647183in}{0.739656in}}%
\pgfpathlineto{\pgfqpoint{1.646887in}{0.739656in}}%
\pgfpathlineto{\pgfqpoint{1.646591in}{0.739656in}}%
\pgfpathlineto{\pgfqpoint{1.646295in}{0.739656in}}%
\pgfpathlineto{\pgfqpoint{1.645999in}{0.739656in}}%
\pgfpathlineto{\pgfqpoint{1.645703in}{0.739656in}}%
\pgfpathlineto{\pgfqpoint{1.645407in}{0.739656in}}%
\pgfpathlineto{\pgfqpoint{1.645111in}{0.739656in}}%
\pgfpathlineto{\pgfqpoint{1.644815in}{0.739656in}}%
\pgfpathlineto{\pgfqpoint{1.644519in}{0.739656in}}%
\pgfpathlineto{\pgfqpoint{1.644223in}{0.739656in}}%
\pgfpathlineto{\pgfqpoint{1.643927in}{0.739656in}}%
\pgfpathlineto{\pgfqpoint{1.643631in}{0.739656in}}%
\pgfpathlineto{\pgfqpoint{1.643335in}{0.739656in}}%
\pgfpathlineto{\pgfqpoint{1.643039in}{0.739656in}}%
\pgfpathlineto{\pgfqpoint{1.642743in}{0.739656in}}%
\pgfpathlineto{\pgfqpoint{1.642447in}{0.739656in}}%
\pgfpathlineto{\pgfqpoint{1.642151in}{0.739656in}}%
\pgfpathlineto{\pgfqpoint{1.641855in}{0.739656in}}%
\pgfpathlineto{\pgfqpoint{1.641559in}{0.739656in}}%
\pgfpathlineto{\pgfqpoint{1.641263in}{0.739656in}}%
\pgfpathlineto{\pgfqpoint{1.640967in}{0.739656in}}%
\pgfpathlineto{\pgfqpoint{1.640671in}{0.739656in}}%
\pgfpathlineto{\pgfqpoint{1.640375in}{0.739656in}}%
\pgfpathlineto{\pgfqpoint{1.640079in}{0.739656in}}%
\pgfpathlineto{\pgfqpoint{1.639783in}{0.739656in}}%
\pgfpathlineto{\pgfqpoint{1.639487in}{0.739656in}}%
\pgfpathlineto{\pgfqpoint{1.639191in}{0.739656in}}%
\pgfpathlineto{\pgfqpoint{1.638895in}{0.739656in}}%
\pgfpathlineto{\pgfqpoint{1.638599in}{0.739656in}}%
\pgfpathlineto{\pgfqpoint{1.638303in}{0.739656in}}%
\pgfpathlineto{\pgfqpoint{1.638007in}{0.739656in}}%
\pgfpathlineto{\pgfqpoint{1.637711in}{0.739656in}}%
\pgfpathlineto{\pgfqpoint{1.637415in}{0.739656in}}%
\pgfpathlineto{\pgfqpoint{1.637119in}{0.739656in}}%
\pgfpathlineto{\pgfqpoint{1.636823in}{0.739656in}}%
\pgfpathlineto{\pgfqpoint{1.636527in}{0.739656in}}%
\pgfpathlineto{\pgfqpoint{1.636231in}{0.739656in}}%
\pgfpathlineto{\pgfqpoint{1.635935in}{0.739656in}}%
\pgfpathlineto{\pgfqpoint{1.635639in}{0.739656in}}%
\pgfpathlineto{\pgfqpoint{1.635343in}{0.739656in}}%
\pgfpathlineto{\pgfqpoint{1.635047in}{0.739656in}}%
\pgfpathlineto{\pgfqpoint{1.634751in}{0.739656in}}%
\pgfpathlineto{\pgfqpoint{1.634455in}{0.739656in}}%
\pgfpathlineto{\pgfqpoint{1.634159in}{0.739656in}}%
\pgfpathlineto{\pgfqpoint{1.633863in}{0.739656in}}%
\pgfpathlineto{\pgfqpoint{1.633567in}{0.739656in}}%
\pgfpathlineto{\pgfqpoint{1.633271in}{0.739656in}}%
\pgfpathlineto{\pgfqpoint{1.632975in}{0.739656in}}%
\pgfpathlineto{\pgfqpoint{1.632679in}{0.739656in}}%
\pgfpathlineto{\pgfqpoint{1.632383in}{0.739656in}}%
\pgfpathlineto{\pgfqpoint{1.632087in}{0.739656in}}%
\pgfpathlineto{\pgfqpoint{1.631791in}{0.739656in}}%
\pgfpathlineto{\pgfqpoint{1.631495in}{0.739656in}}%
\pgfpathlineto{\pgfqpoint{1.631199in}{0.739656in}}%
\pgfpathlineto{\pgfqpoint{1.630903in}{0.739656in}}%
\pgfpathlineto{\pgfqpoint{1.630607in}{0.739656in}}%
\pgfpathlineto{\pgfqpoint{1.630311in}{0.739656in}}%
\pgfpathlineto{\pgfqpoint{1.630015in}{0.739656in}}%
\pgfpathlineto{\pgfqpoint{1.629719in}{0.739656in}}%
\pgfpathlineto{\pgfqpoint{1.629423in}{0.739656in}}%
\pgfpathlineto{\pgfqpoint{1.629127in}{0.739656in}}%
\pgfpathlineto{\pgfqpoint{1.628831in}{0.739656in}}%
\pgfpathlineto{\pgfqpoint{1.628535in}{0.739656in}}%
\pgfpathlineto{\pgfqpoint{1.628239in}{0.739656in}}%
\pgfpathlineto{\pgfqpoint{1.627943in}{0.739656in}}%
\pgfpathlineto{\pgfqpoint{1.627647in}{0.739656in}}%
\pgfpathlineto{\pgfqpoint{1.627351in}{0.739656in}}%
\pgfpathlineto{\pgfqpoint{1.627055in}{0.739656in}}%
\pgfpathlineto{\pgfqpoint{1.626759in}{0.739656in}}%
\pgfpathlineto{\pgfqpoint{1.626463in}{0.739656in}}%
\pgfpathlineto{\pgfqpoint{1.626167in}{0.739656in}}%
\pgfpathlineto{\pgfqpoint{1.625871in}{0.739656in}}%
\pgfpathlineto{\pgfqpoint{1.625575in}{0.739656in}}%
\pgfpathlineto{\pgfqpoint{1.625279in}{0.739656in}}%
\pgfpathlineto{\pgfqpoint{1.624983in}{0.739656in}}%
\pgfpathlineto{\pgfqpoint{1.624687in}{0.739656in}}%
\pgfpathlineto{\pgfqpoint{1.624391in}{0.739656in}}%
\pgfpathlineto{\pgfqpoint{1.624095in}{0.739656in}}%
\pgfpathlineto{\pgfqpoint{1.623799in}{0.739656in}}%
\pgfpathlineto{\pgfqpoint{1.623503in}{0.739656in}}%
\pgfpathlineto{\pgfqpoint{1.623207in}{0.739656in}}%
\pgfpathlineto{\pgfqpoint{1.622911in}{0.739656in}}%
\pgfpathlineto{\pgfqpoint{1.622615in}{0.739656in}}%
\pgfpathlineto{\pgfqpoint{1.622319in}{0.739656in}}%
\pgfpathlineto{\pgfqpoint{1.622023in}{0.739656in}}%
\pgfpathlineto{\pgfqpoint{1.621727in}{0.739656in}}%
\pgfpathlineto{\pgfqpoint{1.621431in}{0.739656in}}%
\pgfpathlineto{\pgfqpoint{1.621135in}{0.739656in}}%
\pgfpathlineto{\pgfqpoint{1.620839in}{0.739656in}}%
\pgfpathlineto{\pgfqpoint{1.620543in}{0.739656in}}%
\pgfpathlineto{\pgfqpoint{1.620247in}{0.739656in}}%
\pgfpathlineto{\pgfqpoint{1.619951in}{0.739656in}}%
\pgfpathlineto{\pgfqpoint{1.619655in}{0.739656in}}%
\pgfpathlineto{\pgfqpoint{1.619359in}{0.739656in}}%
\pgfpathlineto{\pgfqpoint{1.619063in}{0.739656in}}%
\pgfpathlineto{\pgfqpoint{1.618767in}{0.739656in}}%
\pgfpathlineto{\pgfqpoint{1.618471in}{0.739656in}}%
\pgfpathlineto{\pgfqpoint{1.618175in}{0.739656in}}%
\pgfpathlineto{\pgfqpoint{1.617879in}{0.739656in}}%
\pgfpathlineto{\pgfqpoint{1.617583in}{0.739656in}}%
\pgfpathlineto{\pgfqpoint{1.617287in}{0.739656in}}%
\pgfpathlineto{\pgfqpoint{1.616991in}{0.739656in}}%
\pgfpathlineto{\pgfqpoint{1.616695in}{0.739656in}}%
\pgfpathlineto{\pgfqpoint{1.616399in}{0.739656in}}%
\pgfpathlineto{\pgfqpoint{1.616103in}{0.739656in}}%
\pgfpathlineto{\pgfqpoint{1.615807in}{0.739656in}}%
\pgfpathlineto{\pgfqpoint{1.615511in}{0.739656in}}%
\pgfpathlineto{\pgfqpoint{1.615215in}{0.739656in}}%
\pgfpathlineto{\pgfqpoint{1.614919in}{0.739656in}}%
\pgfpathlineto{\pgfqpoint{1.614623in}{0.739656in}}%
\pgfpathlineto{\pgfqpoint{1.614327in}{0.739656in}}%
\pgfpathlineto{\pgfqpoint{1.614031in}{0.739656in}}%
\pgfpathlineto{\pgfqpoint{1.613735in}{0.739656in}}%
\pgfpathlineto{\pgfqpoint{1.613439in}{0.739656in}}%
\pgfpathlineto{\pgfqpoint{1.613143in}{0.739656in}}%
\pgfpathlineto{\pgfqpoint{1.612847in}{0.739656in}}%
\pgfpathlineto{\pgfqpoint{1.612551in}{0.739656in}}%
\pgfpathlineto{\pgfqpoint{1.612255in}{0.739656in}}%
\pgfpathlineto{\pgfqpoint{1.611959in}{0.739656in}}%
\pgfpathlineto{\pgfqpoint{1.611663in}{0.739656in}}%
\pgfpathlineto{\pgfqpoint{1.611367in}{0.739656in}}%
\pgfpathlineto{\pgfqpoint{1.611071in}{0.739656in}}%
\pgfpathlineto{\pgfqpoint{1.610775in}{0.739656in}}%
\pgfpathlineto{\pgfqpoint{1.610479in}{0.739656in}}%
\pgfpathlineto{\pgfqpoint{1.610183in}{0.739656in}}%
\pgfpathlineto{\pgfqpoint{1.609887in}{0.739656in}}%
\pgfpathlineto{\pgfqpoint{1.609591in}{0.739656in}}%
\pgfpathlineto{\pgfqpoint{1.609295in}{0.739656in}}%
\pgfpathlineto{\pgfqpoint{1.608999in}{0.739656in}}%
\pgfpathlineto{\pgfqpoint{1.608703in}{0.739656in}}%
\pgfpathlineto{\pgfqpoint{1.608407in}{0.739656in}}%
\pgfpathlineto{\pgfqpoint{1.608111in}{0.739656in}}%
\pgfpathlineto{\pgfqpoint{1.607815in}{0.739656in}}%
\pgfpathlineto{\pgfqpoint{1.607519in}{0.739656in}}%
\pgfpathlineto{\pgfqpoint{1.607223in}{0.739656in}}%
\pgfpathlineto{\pgfqpoint{1.606927in}{0.739656in}}%
\pgfpathlineto{\pgfqpoint{1.606631in}{0.739656in}}%
\pgfpathlineto{\pgfqpoint{1.606335in}{0.739656in}}%
\pgfpathlineto{\pgfqpoint{1.606039in}{0.739656in}}%
\pgfpathlineto{\pgfqpoint{1.605743in}{0.739656in}}%
\pgfpathlineto{\pgfqpoint{1.605447in}{0.739656in}}%
\pgfpathlineto{\pgfqpoint{1.605151in}{0.739656in}}%
\pgfpathlineto{\pgfqpoint{1.604855in}{0.739656in}}%
\pgfpathlineto{\pgfqpoint{1.604559in}{0.739656in}}%
\pgfpathlineto{\pgfqpoint{1.604263in}{0.739656in}}%
\pgfpathlineto{\pgfqpoint{1.603967in}{0.739656in}}%
\pgfpathlineto{\pgfqpoint{1.603671in}{0.739656in}}%
\pgfpathlineto{\pgfqpoint{1.603375in}{0.739656in}}%
\pgfpathlineto{\pgfqpoint{1.603079in}{0.739656in}}%
\pgfpathlineto{\pgfqpoint{1.602783in}{0.739656in}}%
\pgfpathlineto{\pgfqpoint{1.602487in}{0.739656in}}%
\pgfpathlineto{\pgfqpoint{1.602191in}{0.739656in}}%
\pgfpathlineto{\pgfqpoint{1.601895in}{0.739656in}}%
\pgfpathlineto{\pgfqpoint{1.601599in}{0.739656in}}%
\pgfpathlineto{\pgfqpoint{1.601303in}{0.739656in}}%
\pgfpathlineto{\pgfqpoint{1.601007in}{0.739656in}}%
\pgfpathlineto{\pgfqpoint{1.600711in}{0.739656in}}%
\pgfpathlineto{\pgfqpoint{1.600415in}{0.739656in}}%
\pgfpathlineto{\pgfqpoint{1.600119in}{0.739656in}}%
\pgfpathlineto{\pgfqpoint{1.599823in}{0.739656in}}%
\pgfpathlineto{\pgfqpoint{1.599527in}{0.739656in}}%
\pgfpathlineto{\pgfqpoint{1.599231in}{0.739656in}}%
\pgfpathlineto{\pgfqpoint{1.598935in}{0.739656in}}%
\pgfpathlineto{\pgfqpoint{1.598639in}{0.739656in}}%
\pgfpathlineto{\pgfqpoint{1.598343in}{0.739656in}}%
\pgfpathlineto{\pgfqpoint{1.598047in}{0.739656in}}%
\pgfpathlineto{\pgfqpoint{1.597751in}{0.739656in}}%
\pgfpathlineto{\pgfqpoint{1.597455in}{0.739656in}}%
\pgfpathlineto{\pgfqpoint{1.597159in}{0.739656in}}%
\pgfpathlineto{\pgfqpoint{1.596863in}{0.739656in}}%
\pgfpathlineto{\pgfqpoint{1.596567in}{0.739656in}}%
\pgfpathlineto{\pgfqpoint{1.596271in}{0.739656in}}%
\pgfpathlineto{\pgfqpoint{1.595975in}{0.739656in}}%
\pgfpathlineto{\pgfqpoint{1.595679in}{0.739656in}}%
\pgfpathlineto{\pgfqpoint{1.595383in}{0.739656in}}%
\pgfpathlineto{\pgfqpoint{1.595087in}{0.739656in}}%
\pgfpathlineto{\pgfqpoint{1.594791in}{0.739656in}}%
\pgfpathlineto{\pgfqpoint{1.594495in}{0.739656in}}%
\pgfpathlineto{\pgfqpoint{1.594199in}{0.739656in}}%
\pgfpathlineto{\pgfqpoint{1.593903in}{0.739656in}}%
\pgfpathlineto{\pgfqpoint{1.593607in}{0.739656in}}%
\pgfpathlineto{\pgfqpoint{1.593311in}{0.739656in}}%
\pgfpathlineto{\pgfqpoint{1.593015in}{0.739656in}}%
\pgfpathlineto{\pgfqpoint{1.592719in}{0.739656in}}%
\pgfpathlineto{\pgfqpoint{1.592423in}{0.739656in}}%
\pgfpathlineto{\pgfqpoint{1.592127in}{0.739656in}}%
\pgfpathlineto{\pgfqpoint{1.591831in}{0.739656in}}%
\pgfpathlineto{\pgfqpoint{1.591534in}{0.739656in}}%
\pgfpathlineto{\pgfqpoint{1.591238in}{0.739656in}}%
\pgfpathlineto{\pgfqpoint{1.590942in}{0.739656in}}%
\pgfpathlineto{\pgfqpoint{1.590646in}{0.739656in}}%
\pgfpathlineto{\pgfqpoint{1.590350in}{0.739656in}}%
\pgfpathlineto{\pgfqpoint{1.590054in}{0.739656in}}%
\pgfpathlineto{\pgfqpoint{1.589758in}{0.739656in}}%
\pgfpathlineto{\pgfqpoint{1.589462in}{0.739656in}}%
\pgfpathlineto{\pgfqpoint{1.589166in}{0.739656in}}%
\pgfpathlineto{\pgfqpoint{1.588870in}{0.739656in}}%
\pgfpathlineto{\pgfqpoint{1.588574in}{0.739656in}}%
\pgfpathlineto{\pgfqpoint{1.588278in}{0.739656in}}%
\pgfpathlineto{\pgfqpoint{1.587982in}{0.739656in}}%
\pgfpathlineto{\pgfqpoint{1.587686in}{0.739656in}}%
\pgfpathlineto{\pgfqpoint{1.587390in}{0.739656in}}%
\pgfpathlineto{\pgfqpoint{1.587094in}{0.739656in}}%
\pgfpathlineto{\pgfqpoint{1.586798in}{0.739656in}}%
\pgfpathlineto{\pgfqpoint{1.586502in}{0.739656in}}%
\pgfpathlineto{\pgfqpoint{1.586206in}{0.739656in}}%
\pgfpathlineto{\pgfqpoint{1.585910in}{0.739656in}}%
\pgfpathlineto{\pgfqpoint{1.585614in}{0.739656in}}%
\pgfpathlineto{\pgfqpoint{1.585318in}{0.739656in}}%
\pgfpathlineto{\pgfqpoint{1.585022in}{0.739656in}}%
\pgfpathlineto{\pgfqpoint{1.584726in}{0.739656in}}%
\pgfpathlineto{\pgfqpoint{1.584430in}{0.739656in}}%
\pgfpathlineto{\pgfqpoint{1.584134in}{0.739656in}}%
\pgfpathlineto{\pgfqpoint{1.583838in}{0.739656in}}%
\pgfpathlineto{\pgfqpoint{1.583542in}{0.739656in}}%
\pgfpathlineto{\pgfqpoint{1.583246in}{0.739656in}}%
\pgfpathlineto{\pgfqpoint{1.582950in}{0.739656in}}%
\pgfpathlineto{\pgfqpoint{1.582654in}{0.739656in}}%
\pgfpathlineto{\pgfqpoint{1.582358in}{0.739656in}}%
\pgfpathlineto{\pgfqpoint{1.582062in}{0.739656in}}%
\pgfpathlineto{\pgfqpoint{1.581766in}{0.739656in}}%
\pgfpathlineto{\pgfqpoint{1.581470in}{0.739656in}}%
\pgfpathlineto{\pgfqpoint{1.581174in}{0.739656in}}%
\pgfpathlineto{\pgfqpoint{1.580878in}{0.739656in}}%
\pgfpathlineto{\pgfqpoint{1.580582in}{0.739656in}}%
\pgfpathlineto{\pgfqpoint{1.580286in}{0.739656in}}%
\pgfpathlineto{\pgfqpoint{1.579990in}{0.739656in}}%
\pgfpathlineto{\pgfqpoint{1.579694in}{0.739656in}}%
\pgfpathlineto{\pgfqpoint{1.579398in}{0.739656in}}%
\pgfpathlineto{\pgfqpoint{1.579102in}{0.739656in}}%
\pgfpathlineto{\pgfqpoint{1.578806in}{0.739656in}}%
\pgfpathlineto{\pgfqpoint{1.578510in}{0.739656in}}%
\pgfpathlineto{\pgfqpoint{1.578214in}{0.739656in}}%
\pgfpathlineto{\pgfqpoint{1.577918in}{0.739656in}}%
\pgfpathlineto{\pgfqpoint{1.577622in}{0.739656in}}%
\pgfpathlineto{\pgfqpoint{1.577326in}{0.739656in}}%
\pgfpathlineto{\pgfqpoint{1.577030in}{0.739656in}}%
\pgfpathlineto{\pgfqpoint{1.576734in}{0.739656in}}%
\pgfpathlineto{\pgfqpoint{1.576438in}{0.739656in}}%
\pgfpathlineto{\pgfqpoint{1.576142in}{0.739656in}}%
\pgfpathlineto{\pgfqpoint{1.575846in}{0.739656in}}%
\pgfpathlineto{\pgfqpoint{1.575550in}{0.739656in}}%
\pgfpathlineto{\pgfqpoint{1.575254in}{0.739656in}}%
\pgfpathlineto{\pgfqpoint{1.574958in}{0.739656in}}%
\pgfpathlineto{\pgfqpoint{1.574662in}{0.739656in}}%
\pgfpathlineto{\pgfqpoint{1.574366in}{0.739656in}}%
\pgfpathlineto{\pgfqpoint{1.574070in}{0.739656in}}%
\pgfpathlineto{\pgfqpoint{1.573774in}{0.739656in}}%
\pgfpathlineto{\pgfqpoint{1.573478in}{0.739656in}}%
\pgfpathlineto{\pgfqpoint{1.573182in}{0.739656in}}%
\pgfpathlineto{\pgfqpoint{1.572886in}{0.739656in}}%
\pgfpathlineto{\pgfqpoint{1.572590in}{0.739656in}}%
\pgfpathlineto{\pgfqpoint{1.572294in}{0.739656in}}%
\pgfpathlineto{\pgfqpoint{1.571998in}{0.739656in}}%
\pgfpathlineto{\pgfqpoint{1.571702in}{0.739656in}}%
\pgfpathlineto{\pgfqpoint{1.571406in}{0.739656in}}%
\pgfpathlineto{\pgfqpoint{1.571110in}{0.739656in}}%
\pgfpathlineto{\pgfqpoint{1.570814in}{0.739656in}}%
\pgfpathlineto{\pgfqpoint{1.570518in}{0.739656in}}%
\pgfpathlineto{\pgfqpoint{1.570222in}{0.739656in}}%
\pgfpathlineto{\pgfqpoint{1.569926in}{0.739656in}}%
\pgfpathlineto{\pgfqpoint{1.569630in}{0.739656in}}%
\pgfpathlineto{\pgfqpoint{1.569334in}{0.739656in}}%
\pgfpathlineto{\pgfqpoint{1.569038in}{0.739656in}}%
\pgfpathlineto{\pgfqpoint{1.568742in}{0.739656in}}%
\pgfpathlineto{\pgfqpoint{1.568446in}{0.739656in}}%
\pgfpathlineto{\pgfqpoint{1.568150in}{0.739656in}}%
\pgfpathlineto{\pgfqpoint{1.567854in}{0.739656in}}%
\pgfpathlineto{\pgfqpoint{1.567558in}{0.739656in}}%
\pgfpathlineto{\pgfqpoint{1.567262in}{0.739656in}}%
\pgfpathlineto{\pgfqpoint{1.566966in}{0.739656in}}%
\pgfpathlineto{\pgfqpoint{1.566670in}{0.739656in}}%
\pgfpathlineto{\pgfqpoint{1.566374in}{0.739656in}}%
\pgfpathlineto{\pgfqpoint{1.566078in}{0.739656in}}%
\pgfpathlineto{\pgfqpoint{1.565782in}{0.739656in}}%
\pgfpathlineto{\pgfqpoint{1.565486in}{0.739656in}}%
\pgfpathlineto{\pgfqpoint{1.565190in}{0.739656in}}%
\pgfpathlineto{\pgfqpoint{1.564894in}{0.739656in}}%
\pgfpathlineto{\pgfqpoint{1.564598in}{0.739656in}}%
\pgfpathlineto{\pgfqpoint{1.564302in}{0.739656in}}%
\pgfpathlineto{\pgfqpoint{1.564006in}{0.739656in}}%
\pgfpathlineto{\pgfqpoint{1.563710in}{0.739656in}}%
\pgfpathlineto{\pgfqpoint{1.563414in}{0.739656in}}%
\pgfpathlineto{\pgfqpoint{1.563118in}{0.739656in}}%
\pgfpathlineto{\pgfqpoint{1.562822in}{0.739656in}}%
\pgfpathlineto{\pgfqpoint{1.562526in}{0.739656in}}%
\pgfpathlineto{\pgfqpoint{1.562230in}{0.739656in}}%
\pgfpathlineto{\pgfqpoint{1.561934in}{0.739656in}}%
\pgfpathlineto{\pgfqpoint{1.561638in}{0.739656in}}%
\pgfpathlineto{\pgfqpoint{1.561342in}{0.739656in}}%
\pgfpathlineto{\pgfqpoint{1.561046in}{0.739656in}}%
\pgfpathlineto{\pgfqpoint{1.560750in}{0.739656in}}%
\pgfpathlineto{\pgfqpoint{1.560454in}{0.739656in}}%
\pgfpathlineto{\pgfqpoint{1.560158in}{0.739656in}}%
\pgfpathlineto{\pgfqpoint{1.559862in}{0.739656in}}%
\pgfpathlineto{\pgfqpoint{1.559566in}{0.739656in}}%
\pgfpathlineto{\pgfqpoint{1.559270in}{0.739656in}}%
\pgfpathlineto{\pgfqpoint{1.558974in}{0.739656in}}%
\pgfpathlineto{\pgfqpoint{1.558678in}{0.739656in}}%
\pgfpathlineto{\pgfqpoint{1.558382in}{0.739656in}}%
\pgfpathlineto{\pgfqpoint{1.558086in}{0.739656in}}%
\pgfpathlineto{\pgfqpoint{1.557790in}{0.739656in}}%
\pgfpathlineto{\pgfqpoint{1.557494in}{0.739656in}}%
\pgfpathlineto{\pgfqpoint{1.557198in}{0.739656in}}%
\pgfpathlineto{\pgfqpoint{1.556902in}{0.739656in}}%
\pgfpathlineto{\pgfqpoint{1.556606in}{0.739656in}}%
\pgfpathlineto{\pgfqpoint{1.556310in}{0.739656in}}%
\pgfpathlineto{\pgfqpoint{1.556014in}{0.739656in}}%
\pgfpathlineto{\pgfqpoint{1.555718in}{0.739656in}}%
\pgfpathlineto{\pgfqpoint{1.555422in}{0.739656in}}%
\pgfpathlineto{\pgfqpoint{1.555126in}{0.739656in}}%
\pgfpathlineto{\pgfqpoint{1.554830in}{0.739656in}}%
\pgfpathlineto{\pgfqpoint{1.554534in}{0.739656in}}%
\pgfpathlineto{\pgfqpoint{1.554238in}{0.739656in}}%
\pgfpathlineto{\pgfqpoint{1.553942in}{0.739656in}}%
\pgfpathlineto{\pgfqpoint{1.553646in}{0.739656in}}%
\pgfpathlineto{\pgfqpoint{1.553350in}{0.739656in}}%
\pgfpathlineto{\pgfqpoint{1.553054in}{0.739656in}}%
\pgfpathlineto{\pgfqpoint{1.552758in}{0.739656in}}%
\pgfpathlineto{\pgfqpoint{1.552462in}{0.739656in}}%
\pgfpathlineto{\pgfqpoint{1.552166in}{0.739656in}}%
\pgfpathlineto{\pgfqpoint{1.551870in}{0.739656in}}%
\pgfpathlineto{\pgfqpoint{1.551574in}{0.739656in}}%
\pgfpathlineto{\pgfqpoint{1.551278in}{0.739656in}}%
\pgfpathlineto{\pgfqpoint{1.550982in}{0.739656in}}%
\pgfpathlineto{\pgfqpoint{1.550686in}{0.739656in}}%
\pgfpathlineto{\pgfqpoint{1.550390in}{0.739656in}}%
\pgfpathlineto{\pgfqpoint{1.550094in}{0.739656in}}%
\pgfpathlineto{\pgfqpoint{1.549798in}{0.739656in}}%
\pgfpathlineto{\pgfqpoint{1.549502in}{0.739656in}}%
\pgfpathlineto{\pgfqpoint{1.549206in}{0.739656in}}%
\pgfpathlineto{\pgfqpoint{1.548910in}{0.739656in}}%
\pgfpathlineto{\pgfqpoint{1.548614in}{0.739656in}}%
\pgfpathlineto{\pgfqpoint{1.548318in}{0.739656in}}%
\pgfpathlineto{\pgfqpoint{1.548022in}{0.739656in}}%
\pgfpathlineto{\pgfqpoint{1.547726in}{0.739656in}}%
\pgfpathlineto{\pgfqpoint{1.547430in}{0.739656in}}%
\pgfpathlineto{\pgfqpoint{1.547134in}{0.739656in}}%
\pgfpathlineto{\pgfqpoint{1.546838in}{0.739656in}}%
\pgfpathlineto{\pgfqpoint{1.546542in}{0.739656in}}%
\pgfpathlineto{\pgfqpoint{1.546246in}{0.739656in}}%
\pgfpathlineto{\pgfqpoint{1.545950in}{0.739656in}}%
\pgfpathlineto{\pgfqpoint{1.545654in}{0.739656in}}%
\pgfpathlineto{\pgfqpoint{1.545358in}{0.739656in}}%
\pgfpathlineto{\pgfqpoint{1.545062in}{0.739656in}}%
\pgfpathlineto{\pgfqpoint{1.544766in}{0.739656in}}%
\pgfpathlineto{\pgfqpoint{1.544470in}{0.739656in}}%
\pgfpathlineto{\pgfqpoint{1.544174in}{0.739656in}}%
\pgfpathlineto{\pgfqpoint{1.543878in}{0.739656in}}%
\pgfpathlineto{\pgfqpoint{1.543582in}{0.739656in}}%
\pgfpathlineto{\pgfqpoint{1.543286in}{0.739656in}}%
\pgfpathlineto{\pgfqpoint{1.542990in}{0.739656in}}%
\pgfpathlineto{\pgfqpoint{1.542694in}{0.739656in}}%
\pgfpathlineto{\pgfqpoint{1.542398in}{0.739656in}}%
\pgfpathlineto{\pgfqpoint{1.542102in}{0.739656in}}%
\pgfpathlineto{\pgfqpoint{1.541806in}{0.739656in}}%
\pgfpathlineto{\pgfqpoint{1.541510in}{0.739656in}}%
\pgfpathlineto{\pgfqpoint{1.541214in}{0.739656in}}%
\pgfpathlineto{\pgfqpoint{1.540918in}{0.739656in}}%
\pgfpathlineto{\pgfqpoint{1.540622in}{0.739656in}}%
\pgfpathlineto{\pgfqpoint{1.540326in}{0.739656in}}%
\pgfpathlineto{\pgfqpoint{1.540030in}{0.739656in}}%
\pgfpathlineto{\pgfqpoint{1.539734in}{0.739656in}}%
\pgfpathlineto{\pgfqpoint{1.539438in}{0.739656in}}%
\pgfpathlineto{\pgfqpoint{1.539142in}{0.739656in}}%
\pgfpathlineto{\pgfqpoint{1.538846in}{0.739656in}}%
\pgfpathlineto{\pgfqpoint{1.538550in}{0.739656in}}%
\pgfpathlineto{\pgfqpoint{1.538254in}{0.739656in}}%
\pgfpathlineto{\pgfqpoint{1.537958in}{0.739656in}}%
\pgfpathlineto{\pgfqpoint{1.537662in}{0.739656in}}%
\pgfpathlineto{\pgfqpoint{1.537366in}{0.739656in}}%
\pgfpathlineto{\pgfqpoint{1.537070in}{0.739656in}}%
\pgfpathlineto{\pgfqpoint{1.536774in}{0.739656in}}%
\pgfpathlineto{\pgfqpoint{1.536478in}{0.739656in}}%
\pgfpathlineto{\pgfqpoint{1.536182in}{0.739656in}}%
\pgfpathlineto{\pgfqpoint{1.535886in}{0.739656in}}%
\pgfpathlineto{\pgfqpoint{1.535590in}{0.739656in}}%
\pgfpathlineto{\pgfqpoint{1.535294in}{0.739656in}}%
\pgfpathlineto{\pgfqpoint{1.534998in}{0.739656in}}%
\pgfpathlineto{\pgfqpoint{1.534702in}{0.739656in}}%
\pgfpathlineto{\pgfqpoint{1.534406in}{0.739656in}}%
\pgfpathlineto{\pgfqpoint{1.534110in}{0.739656in}}%
\pgfpathlineto{\pgfqpoint{1.533814in}{0.739656in}}%
\pgfpathlineto{\pgfqpoint{1.533518in}{0.739656in}}%
\pgfpathlineto{\pgfqpoint{1.533222in}{0.739656in}}%
\pgfpathlineto{\pgfqpoint{1.532926in}{0.739656in}}%
\pgfpathlineto{\pgfqpoint{1.532630in}{0.739656in}}%
\pgfpathlineto{\pgfqpoint{1.532334in}{0.739656in}}%
\pgfpathlineto{\pgfqpoint{1.532038in}{0.739656in}}%
\pgfpathlineto{\pgfqpoint{1.531742in}{0.739656in}}%
\pgfpathlineto{\pgfqpoint{1.531446in}{0.739656in}}%
\pgfpathlineto{\pgfqpoint{1.531150in}{0.739656in}}%
\pgfpathlineto{\pgfqpoint{1.530854in}{0.739656in}}%
\pgfpathlineto{\pgfqpoint{1.530558in}{0.739656in}}%
\pgfpathlineto{\pgfqpoint{1.530262in}{0.739656in}}%
\pgfpathlineto{\pgfqpoint{1.529966in}{0.739656in}}%
\pgfpathlineto{\pgfqpoint{1.529670in}{0.739656in}}%
\pgfpathlineto{\pgfqpoint{1.529374in}{0.739656in}}%
\pgfpathlineto{\pgfqpoint{1.529078in}{0.739656in}}%
\pgfpathlineto{\pgfqpoint{1.528782in}{0.739656in}}%
\pgfpathlineto{\pgfqpoint{1.528486in}{0.739656in}}%
\pgfpathlineto{\pgfqpoint{1.528190in}{0.739656in}}%
\pgfpathlineto{\pgfqpoint{1.527894in}{0.739656in}}%
\pgfpathlineto{\pgfqpoint{1.527598in}{0.739656in}}%
\pgfpathlineto{\pgfqpoint{1.527302in}{0.739656in}}%
\pgfpathlineto{\pgfqpoint{1.527006in}{0.739656in}}%
\pgfpathlineto{\pgfqpoint{1.526710in}{0.739656in}}%
\pgfpathlineto{\pgfqpoint{1.526414in}{0.739656in}}%
\pgfpathlineto{\pgfqpoint{1.526118in}{0.739656in}}%
\pgfpathlineto{\pgfqpoint{1.525822in}{0.739656in}}%
\pgfpathlineto{\pgfqpoint{1.525526in}{0.739656in}}%
\pgfpathlineto{\pgfqpoint{1.525230in}{0.739656in}}%
\pgfpathlineto{\pgfqpoint{1.524934in}{0.739656in}}%
\pgfpathlineto{\pgfqpoint{1.524638in}{0.739656in}}%
\pgfpathlineto{\pgfqpoint{1.524341in}{0.739656in}}%
\pgfpathlineto{\pgfqpoint{1.524045in}{0.739656in}}%
\pgfpathlineto{\pgfqpoint{1.523749in}{0.739656in}}%
\pgfpathlineto{\pgfqpoint{1.523453in}{0.739656in}}%
\pgfpathlineto{\pgfqpoint{1.523157in}{0.739656in}}%
\pgfpathlineto{\pgfqpoint{1.522861in}{0.739656in}}%
\pgfpathlineto{\pgfqpoint{1.522565in}{0.739656in}}%
\pgfpathlineto{\pgfqpoint{1.522269in}{0.739656in}}%
\pgfpathlineto{\pgfqpoint{1.521973in}{0.739656in}}%
\pgfpathlineto{\pgfqpoint{1.521677in}{0.739656in}}%
\pgfpathlineto{\pgfqpoint{1.521381in}{0.739656in}}%
\pgfpathlineto{\pgfqpoint{1.521085in}{0.739656in}}%
\pgfpathlineto{\pgfqpoint{1.520789in}{0.739656in}}%
\pgfpathlineto{\pgfqpoint{1.520493in}{0.739656in}}%
\pgfpathlineto{\pgfqpoint{1.520197in}{0.739656in}}%
\pgfpathlineto{\pgfqpoint{1.519901in}{0.739656in}}%
\pgfpathlineto{\pgfqpoint{1.519605in}{0.739656in}}%
\pgfpathlineto{\pgfqpoint{1.519309in}{0.739656in}}%
\pgfpathlineto{\pgfqpoint{1.519013in}{0.739656in}}%
\pgfpathlineto{\pgfqpoint{1.518717in}{0.739656in}}%
\pgfpathlineto{\pgfqpoint{1.518421in}{0.739656in}}%
\pgfpathlineto{\pgfqpoint{1.518125in}{0.739656in}}%
\pgfpathlineto{\pgfqpoint{1.517829in}{0.739656in}}%
\pgfpathlineto{\pgfqpoint{1.517533in}{0.739656in}}%
\pgfpathlineto{\pgfqpoint{1.517237in}{0.739656in}}%
\pgfpathlineto{\pgfqpoint{1.516941in}{0.739656in}}%
\pgfpathlineto{\pgfqpoint{1.516645in}{0.739656in}}%
\pgfpathlineto{\pgfqpoint{1.516349in}{0.739656in}}%
\pgfpathlineto{\pgfqpoint{1.516053in}{0.739656in}}%
\pgfpathlineto{\pgfqpoint{1.515757in}{0.739656in}}%
\pgfpathlineto{\pgfqpoint{1.515461in}{0.739656in}}%
\pgfpathlineto{\pgfqpoint{1.515165in}{0.739656in}}%
\pgfpathlineto{\pgfqpoint{1.514869in}{0.739656in}}%
\pgfpathlineto{\pgfqpoint{1.514573in}{0.739656in}}%
\pgfpathlineto{\pgfqpoint{1.514277in}{0.739656in}}%
\pgfpathlineto{\pgfqpoint{1.513981in}{0.739656in}}%
\pgfpathlineto{\pgfqpoint{1.513685in}{0.739656in}}%
\pgfpathlineto{\pgfqpoint{1.513389in}{0.739656in}}%
\pgfpathlineto{\pgfqpoint{1.513093in}{0.739656in}}%
\pgfpathlineto{\pgfqpoint{1.512797in}{0.739656in}}%
\pgfpathlineto{\pgfqpoint{1.512501in}{0.739656in}}%
\pgfpathlineto{\pgfqpoint{1.512205in}{0.739656in}}%
\pgfpathlineto{\pgfqpoint{1.511909in}{0.739656in}}%
\pgfpathlineto{\pgfqpoint{1.511613in}{0.739656in}}%
\pgfpathlineto{\pgfqpoint{1.511317in}{0.739656in}}%
\pgfpathlineto{\pgfqpoint{1.511021in}{0.739656in}}%
\pgfpathlineto{\pgfqpoint{1.510725in}{0.739656in}}%
\pgfpathlineto{\pgfqpoint{1.510429in}{0.739656in}}%
\pgfpathlineto{\pgfqpoint{1.510133in}{0.739656in}}%
\pgfpathlineto{\pgfqpoint{1.509837in}{0.739656in}}%
\pgfpathlineto{\pgfqpoint{1.509541in}{0.739656in}}%
\pgfpathlineto{\pgfqpoint{1.509245in}{0.739656in}}%
\pgfpathlineto{\pgfqpoint{1.508949in}{0.739656in}}%
\pgfpathlineto{\pgfqpoint{1.508653in}{0.739656in}}%
\pgfpathlineto{\pgfqpoint{1.508357in}{0.739656in}}%
\pgfpathlineto{\pgfqpoint{1.508061in}{0.739656in}}%
\pgfpathlineto{\pgfqpoint{1.507765in}{0.739656in}}%
\pgfpathlineto{\pgfqpoint{1.507469in}{0.739656in}}%
\pgfpathlineto{\pgfqpoint{1.507173in}{0.739656in}}%
\pgfpathlineto{\pgfqpoint{1.506877in}{0.739656in}}%
\pgfpathlineto{\pgfqpoint{1.506581in}{0.739656in}}%
\pgfpathlineto{\pgfqpoint{1.506285in}{0.739656in}}%
\pgfpathlineto{\pgfqpoint{1.505989in}{0.739656in}}%
\pgfpathlineto{\pgfqpoint{1.505693in}{0.739656in}}%
\pgfpathlineto{\pgfqpoint{1.505397in}{0.739656in}}%
\pgfpathlineto{\pgfqpoint{1.505101in}{0.739656in}}%
\pgfpathlineto{\pgfqpoint{1.504805in}{0.739656in}}%
\pgfpathlineto{\pgfqpoint{1.504509in}{0.739656in}}%
\pgfpathlineto{\pgfqpoint{1.504213in}{0.739656in}}%
\pgfpathlineto{\pgfqpoint{1.503917in}{0.739656in}}%
\pgfpathlineto{\pgfqpoint{1.503621in}{0.739656in}}%
\pgfpathlineto{\pgfqpoint{1.503325in}{0.739656in}}%
\pgfpathlineto{\pgfqpoint{1.503029in}{0.739656in}}%
\pgfpathlineto{\pgfqpoint{1.502733in}{0.739656in}}%
\pgfpathlineto{\pgfqpoint{1.502437in}{0.739656in}}%
\pgfpathlineto{\pgfqpoint{1.502141in}{0.739656in}}%
\pgfpathlineto{\pgfqpoint{1.501845in}{0.739656in}}%
\pgfpathlineto{\pgfqpoint{1.501549in}{0.739656in}}%
\pgfpathlineto{\pgfqpoint{1.501253in}{0.739656in}}%
\pgfpathlineto{\pgfqpoint{1.500957in}{0.739656in}}%
\pgfpathlineto{\pgfqpoint{1.500661in}{0.739656in}}%
\pgfpathlineto{\pgfqpoint{1.500365in}{0.739656in}}%
\pgfpathlineto{\pgfqpoint{1.500069in}{0.739656in}}%
\pgfpathlineto{\pgfqpoint{1.499773in}{0.739656in}}%
\pgfpathlineto{\pgfqpoint{1.499477in}{0.739656in}}%
\pgfpathlineto{\pgfqpoint{1.499181in}{0.739656in}}%
\pgfpathlineto{\pgfqpoint{1.498885in}{0.739656in}}%
\pgfpathlineto{\pgfqpoint{1.498589in}{0.739656in}}%
\pgfpathlineto{\pgfqpoint{1.498293in}{0.739656in}}%
\pgfpathlineto{\pgfqpoint{1.497997in}{0.739656in}}%
\pgfpathlineto{\pgfqpoint{1.497701in}{0.739656in}}%
\pgfpathlineto{\pgfqpoint{1.497405in}{0.739656in}}%
\pgfpathlineto{\pgfqpoint{1.497109in}{0.739656in}}%
\pgfpathlineto{\pgfqpoint{1.496813in}{0.739656in}}%
\pgfpathlineto{\pgfqpoint{1.496517in}{0.739656in}}%
\pgfpathlineto{\pgfqpoint{1.496221in}{0.739656in}}%
\pgfpathlineto{\pgfqpoint{1.495925in}{0.739656in}}%
\pgfpathlineto{\pgfqpoint{1.495629in}{0.739656in}}%
\pgfpathlineto{\pgfqpoint{1.495333in}{0.739656in}}%
\pgfpathlineto{\pgfqpoint{1.495037in}{0.739656in}}%
\pgfpathlineto{\pgfqpoint{1.494741in}{0.739656in}}%
\pgfpathlineto{\pgfqpoint{1.494445in}{0.739656in}}%
\pgfpathlineto{\pgfqpoint{1.494149in}{0.739656in}}%
\pgfpathlineto{\pgfqpoint{1.493853in}{0.739656in}}%
\pgfpathlineto{\pgfqpoint{1.493557in}{0.739656in}}%
\pgfpathlineto{\pgfqpoint{1.493261in}{0.739656in}}%
\pgfpathlineto{\pgfqpoint{1.492965in}{0.739656in}}%
\pgfpathlineto{\pgfqpoint{1.492669in}{0.739656in}}%
\pgfpathlineto{\pgfqpoint{1.492373in}{0.739656in}}%
\pgfpathlineto{\pgfqpoint{1.492077in}{0.739656in}}%
\pgfpathlineto{\pgfqpoint{1.491781in}{0.739656in}}%
\pgfpathlineto{\pgfqpoint{1.491485in}{0.739656in}}%
\pgfpathlineto{\pgfqpoint{1.491189in}{0.739656in}}%
\pgfpathlineto{\pgfqpoint{1.490893in}{0.739656in}}%
\pgfpathlineto{\pgfqpoint{1.490597in}{0.739656in}}%
\pgfpathlineto{\pgfqpoint{1.490301in}{0.739656in}}%
\pgfpathlineto{\pgfqpoint{1.490005in}{0.739656in}}%
\pgfpathlineto{\pgfqpoint{1.489709in}{0.739656in}}%
\pgfpathlineto{\pgfqpoint{1.489413in}{0.739656in}}%
\pgfpathlineto{\pgfqpoint{1.489117in}{0.739656in}}%
\pgfpathlineto{\pgfqpoint{1.488821in}{0.739656in}}%
\pgfpathlineto{\pgfqpoint{1.488525in}{0.739656in}}%
\pgfpathlineto{\pgfqpoint{1.488229in}{0.739656in}}%
\pgfpathlineto{\pgfqpoint{1.487933in}{0.739656in}}%
\pgfpathlineto{\pgfqpoint{1.487637in}{0.739656in}}%
\pgfpathlineto{\pgfqpoint{1.487341in}{0.739656in}}%
\pgfpathlineto{\pgfqpoint{1.487045in}{0.739656in}}%
\pgfpathlineto{\pgfqpoint{1.486749in}{0.739656in}}%
\pgfpathlineto{\pgfqpoint{1.486453in}{0.739656in}}%
\pgfpathlineto{\pgfqpoint{1.486157in}{0.739656in}}%
\pgfpathlineto{\pgfqpoint{1.485861in}{0.739656in}}%
\pgfpathlineto{\pgfqpoint{1.485565in}{0.739656in}}%
\pgfpathlineto{\pgfqpoint{1.485269in}{0.739656in}}%
\pgfpathlineto{\pgfqpoint{1.484973in}{0.739656in}}%
\pgfpathlineto{\pgfqpoint{1.484677in}{0.739656in}}%
\pgfpathlineto{\pgfqpoint{1.484381in}{0.739656in}}%
\pgfpathlineto{\pgfqpoint{1.484085in}{0.739656in}}%
\pgfpathlineto{\pgfqpoint{1.483789in}{0.739656in}}%
\pgfpathlineto{\pgfqpoint{1.483493in}{0.739656in}}%
\pgfpathlineto{\pgfqpoint{1.483197in}{0.739656in}}%
\pgfpathlineto{\pgfqpoint{1.482901in}{0.739656in}}%
\pgfpathlineto{\pgfqpoint{1.482605in}{0.739656in}}%
\pgfpathlineto{\pgfqpoint{1.482309in}{0.739656in}}%
\pgfpathlineto{\pgfqpoint{1.482013in}{0.739656in}}%
\pgfpathlineto{\pgfqpoint{1.481717in}{0.739656in}}%
\pgfpathlineto{\pgfqpoint{1.481421in}{0.739656in}}%
\pgfpathlineto{\pgfqpoint{1.481125in}{0.739656in}}%
\pgfpathlineto{\pgfqpoint{1.480829in}{0.739656in}}%
\pgfpathlineto{\pgfqpoint{1.480533in}{0.739656in}}%
\pgfpathlineto{\pgfqpoint{1.480237in}{0.739656in}}%
\pgfpathlineto{\pgfqpoint{1.479941in}{0.739656in}}%
\pgfpathlineto{\pgfqpoint{1.479645in}{0.739656in}}%
\pgfpathlineto{\pgfqpoint{1.479349in}{0.739656in}}%
\pgfpathlineto{\pgfqpoint{1.479053in}{0.739656in}}%
\pgfpathlineto{\pgfqpoint{1.478757in}{0.739656in}}%
\pgfpathlineto{\pgfqpoint{1.478461in}{0.739656in}}%
\pgfpathlineto{\pgfqpoint{1.478165in}{0.739656in}}%
\pgfpathlineto{\pgfqpoint{1.477869in}{0.739656in}}%
\pgfpathlineto{\pgfqpoint{1.477573in}{0.739656in}}%
\pgfpathlineto{\pgfqpoint{1.477277in}{0.739656in}}%
\pgfpathlineto{\pgfqpoint{1.476981in}{0.739656in}}%
\pgfpathlineto{\pgfqpoint{1.476685in}{0.739656in}}%
\pgfpathlineto{\pgfqpoint{1.476389in}{0.739656in}}%
\pgfpathlineto{\pgfqpoint{1.476093in}{0.739656in}}%
\pgfpathlineto{\pgfqpoint{1.475797in}{0.739656in}}%
\pgfpathlineto{\pgfqpoint{1.475501in}{0.739656in}}%
\pgfpathlineto{\pgfqpoint{1.475205in}{0.739656in}}%
\pgfpathlineto{\pgfqpoint{1.474909in}{0.739656in}}%
\pgfpathlineto{\pgfqpoint{1.474613in}{0.739656in}}%
\pgfpathlineto{\pgfqpoint{1.474317in}{0.739656in}}%
\pgfpathlineto{\pgfqpoint{1.474021in}{0.739656in}}%
\pgfpathlineto{\pgfqpoint{1.473725in}{0.739656in}}%
\pgfpathlineto{\pgfqpoint{1.473429in}{0.739656in}}%
\pgfpathlineto{\pgfqpoint{1.473133in}{0.739656in}}%
\pgfpathlineto{\pgfqpoint{1.472837in}{0.739656in}}%
\pgfpathlineto{\pgfqpoint{1.472541in}{0.739656in}}%
\pgfpathlineto{\pgfqpoint{1.472245in}{0.739656in}}%
\pgfpathlineto{\pgfqpoint{1.471949in}{0.739656in}}%
\pgfpathlineto{\pgfqpoint{1.471653in}{0.739656in}}%
\pgfpathlineto{\pgfqpoint{1.471357in}{0.739656in}}%
\pgfpathlineto{\pgfqpoint{1.471061in}{0.739656in}}%
\pgfpathlineto{\pgfqpoint{1.470765in}{0.739656in}}%
\pgfpathlineto{\pgfqpoint{1.470469in}{0.739656in}}%
\pgfpathlineto{\pgfqpoint{1.470173in}{0.739656in}}%
\pgfpathlineto{\pgfqpoint{1.469877in}{0.739656in}}%
\pgfpathlineto{\pgfqpoint{1.469581in}{0.739656in}}%
\pgfpathlineto{\pgfqpoint{1.469285in}{0.739656in}}%
\pgfpathlineto{\pgfqpoint{1.468989in}{0.739656in}}%
\pgfpathlineto{\pgfqpoint{1.468693in}{0.739656in}}%
\pgfpathlineto{\pgfqpoint{1.468397in}{0.739656in}}%
\pgfpathlineto{\pgfqpoint{1.468101in}{0.739656in}}%
\pgfpathlineto{\pgfqpoint{1.467805in}{0.739656in}}%
\pgfpathlineto{\pgfqpoint{1.467509in}{0.739656in}}%
\pgfpathlineto{\pgfqpoint{1.467213in}{0.739656in}}%
\pgfpathlineto{\pgfqpoint{1.466917in}{0.739656in}}%
\pgfpathlineto{\pgfqpoint{1.466621in}{0.739656in}}%
\pgfpathlineto{\pgfqpoint{1.466325in}{0.739656in}}%
\pgfpathlineto{\pgfqpoint{1.466029in}{0.739656in}}%
\pgfpathlineto{\pgfqpoint{1.465733in}{0.739656in}}%
\pgfpathlineto{\pgfqpoint{1.465437in}{0.739656in}}%
\pgfpathlineto{\pgfqpoint{1.465141in}{0.739656in}}%
\pgfpathlineto{\pgfqpoint{1.464845in}{0.739656in}}%
\pgfpathlineto{\pgfqpoint{1.464549in}{0.739656in}}%
\pgfpathlineto{\pgfqpoint{1.464253in}{0.739656in}}%
\pgfpathlineto{\pgfqpoint{1.463957in}{0.739656in}}%
\pgfpathlineto{\pgfqpoint{1.463661in}{0.739656in}}%
\pgfpathlineto{\pgfqpoint{1.463365in}{0.739656in}}%
\pgfpathlineto{\pgfqpoint{1.463069in}{0.739656in}}%
\pgfpathlineto{\pgfqpoint{1.462773in}{0.739656in}}%
\pgfpathlineto{\pgfqpoint{1.462477in}{0.739656in}}%
\pgfpathlineto{\pgfqpoint{1.462181in}{0.739656in}}%
\pgfpathlineto{\pgfqpoint{1.461885in}{0.739656in}}%
\pgfpathlineto{\pgfqpoint{1.461589in}{0.739656in}}%
\pgfpathlineto{\pgfqpoint{1.461293in}{0.739656in}}%
\pgfpathlineto{\pgfqpoint{1.460997in}{0.739656in}}%
\pgfpathlineto{\pgfqpoint{1.460701in}{0.739656in}}%
\pgfpathlineto{\pgfqpoint{1.460405in}{0.739656in}}%
\pgfpathlineto{\pgfqpoint{1.460109in}{0.739656in}}%
\pgfpathlineto{\pgfqpoint{1.459813in}{0.739656in}}%
\pgfpathlineto{\pgfqpoint{1.459517in}{0.739656in}}%
\pgfpathlineto{\pgfqpoint{1.459221in}{0.739656in}}%
\pgfpathlineto{\pgfqpoint{1.458925in}{0.739656in}}%
\pgfpathlineto{\pgfqpoint{1.458629in}{0.739656in}}%
\pgfpathlineto{\pgfqpoint{1.458333in}{0.739656in}}%
\pgfpathlineto{\pgfqpoint{1.458037in}{0.739656in}}%
\pgfpathlineto{\pgfqpoint{1.457741in}{0.739656in}}%
\pgfpathlineto{\pgfqpoint{1.457445in}{0.739656in}}%
\pgfpathlineto{\pgfqpoint{1.457149in}{0.739656in}}%
\pgfpathlineto{\pgfqpoint{1.456852in}{0.739656in}}%
\pgfpathlineto{\pgfqpoint{1.456556in}{0.739656in}}%
\pgfpathlineto{\pgfqpoint{1.456260in}{0.739656in}}%
\pgfpathlineto{\pgfqpoint{1.455964in}{0.739656in}}%
\pgfpathlineto{\pgfqpoint{1.455668in}{0.739656in}}%
\pgfpathlineto{\pgfqpoint{1.455372in}{0.739656in}}%
\pgfpathlineto{\pgfqpoint{1.455076in}{0.739656in}}%
\pgfpathlineto{\pgfqpoint{1.454780in}{0.739656in}}%
\pgfpathlineto{\pgfqpoint{1.454484in}{0.739656in}}%
\pgfpathlineto{\pgfqpoint{1.454188in}{0.739656in}}%
\pgfpathlineto{\pgfqpoint{1.453892in}{0.739656in}}%
\pgfpathlineto{\pgfqpoint{1.453596in}{0.739656in}}%
\pgfpathlineto{\pgfqpoint{1.453300in}{0.739656in}}%
\pgfpathlineto{\pgfqpoint{1.453004in}{0.739656in}}%
\pgfpathlineto{\pgfqpoint{1.452708in}{0.739656in}}%
\pgfpathlineto{\pgfqpoint{1.452412in}{0.739656in}}%
\pgfpathlineto{\pgfqpoint{1.452116in}{0.739656in}}%
\pgfpathlineto{\pgfqpoint{1.451820in}{0.739656in}}%
\pgfpathlineto{\pgfqpoint{1.451524in}{0.739656in}}%
\pgfpathlineto{\pgfqpoint{1.451228in}{0.739656in}}%
\pgfpathlineto{\pgfqpoint{1.450932in}{0.739656in}}%
\pgfpathlineto{\pgfqpoint{1.450636in}{0.739656in}}%
\pgfpathlineto{\pgfqpoint{1.450340in}{0.739656in}}%
\pgfpathlineto{\pgfqpoint{1.450044in}{0.739656in}}%
\pgfpathlineto{\pgfqpoint{1.449748in}{0.739656in}}%
\pgfpathlineto{\pgfqpoint{1.449452in}{0.739656in}}%
\pgfpathlineto{\pgfqpoint{1.449156in}{0.739656in}}%
\pgfpathlineto{\pgfqpoint{1.448860in}{0.739656in}}%
\pgfpathlineto{\pgfqpoint{1.448564in}{0.739656in}}%
\pgfpathlineto{\pgfqpoint{1.448268in}{0.739656in}}%
\pgfpathlineto{\pgfqpoint{1.447972in}{0.739656in}}%
\pgfpathlineto{\pgfqpoint{1.447676in}{0.739656in}}%
\pgfpathlineto{\pgfqpoint{1.447380in}{0.739656in}}%
\pgfpathlineto{\pgfqpoint{1.447084in}{0.739656in}}%
\pgfpathlineto{\pgfqpoint{1.446788in}{0.739656in}}%
\pgfpathlineto{\pgfqpoint{1.446492in}{0.739656in}}%
\pgfpathlineto{\pgfqpoint{1.446196in}{0.739656in}}%
\pgfpathlineto{\pgfqpoint{1.445900in}{0.739656in}}%
\pgfpathlineto{\pgfqpoint{1.445604in}{0.739656in}}%
\pgfpathlineto{\pgfqpoint{1.445308in}{0.739656in}}%
\pgfpathlineto{\pgfqpoint{1.445012in}{0.739656in}}%
\pgfpathlineto{\pgfqpoint{1.444716in}{0.739656in}}%
\pgfpathlineto{\pgfqpoint{1.444420in}{0.739656in}}%
\pgfpathlineto{\pgfqpoint{1.444124in}{0.739656in}}%
\pgfpathlineto{\pgfqpoint{1.443828in}{0.739656in}}%
\pgfpathlineto{\pgfqpoint{1.443532in}{0.739656in}}%
\pgfpathlineto{\pgfqpoint{1.443236in}{0.739656in}}%
\pgfpathlineto{\pgfqpoint{1.442940in}{0.739656in}}%
\pgfpathlineto{\pgfqpoint{1.442644in}{0.739656in}}%
\pgfpathlineto{\pgfqpoint{1.442348in}{0.739656in}}%
\pgfpathlineto{\pgfqpoint{1.442052in}{0.739656in}}%
\pgfpathlineto{\pgfqpoint{1.441756in}{0.739656in}}%
\pgfpathlineto{\pgfqpoint{1.441460in}{0.739656in}}%
\pgfpathlineto{\pgfqpoint{1.441164in}{0.739656in}}%
\pgfpathlineto{\pgfqpoint{1.440868in}{0.739656in}}%
\pgfpathlineto{\pgfqpoint{1.440572in}{0.739656in}}%
\pgfpathlineto{\pgfqpoint{1.440276in}{0.739656in}}%
\pgfpathlineto{\pgfqpoint{1.439980in}{0.739656in}}%
\pgfpathlineto{\pgfqpoint{1.439684in}{0.739656in}}%
\pgfpathlineto{\pgfqpoint{1.439388in}{0.739656in}}%
\pgfpathlineto{\pgfqpoint{1.439092in}{0.739656in}}%
\pgfpathlineto{\pgfqpoint{1.438796in}{0.739656in}}%
\pgfpathlineto{\pgfqpoint{1.438500in}{0.739656in}}%
\pgfpathlineto{\pgfqpoint{1.438204in}{0.739656in}}%
\pgfpathlineto{\pgfqpoint{1.437908in}{0.739656in}}%
\pgfpathlineto{\pgfqpoint{1.437612in}{0.739656in}}%
\pgfpathlineto{\pgfqpoint{1.437316in}{0.739656in}}%
\pgfpathlineto{\pgfqpoint{1.437020in}{0.739656in}}%
\pgfpathlineto{\pgfqpoint{1.436724in}{0.739656in}}%
\pgfpathlineto{\pgfqpoint{1.436428in}{0.739656in}}%
\pgfpathlineto{\pgfqpoint{1.436132in}{0.739656in}}%
\pgfpathlineto{\pgfqpoint{1.435836in}{0.739656in}}%
\pgfpathlineto{\pgfqpoint{1.435540in}{0.739656in}}%
\pgfpathlineto{\pgfqpoint{1.435244in}{0.739656in}}%
\pgfpathlineto{\pgfqpoint{1.434948in}{0.739656in}}%
\pgfpathlineto{\pgfqpoint{1.434652in}{0.739656in}}%
\pgfpathlineto{\pgfqpoint{1.434356in}{0.739656in}}%
\pgfpathlineto{\pgfqpoint{1.434060in}{0.739656in}}%
\pgfpathlineto{\pgfqpoint{1.433764in}{0.739656in}}%
\pgfpathlineto{\pgfqpoint{1.433468in}{0.739656in}}%
\pgfpathlineto{\pgfqpoint{1.433172in}{0.739656in}}%
\pgfpathlineto{\pgfqpoint{1.432876in}{0.739656in}}%
\pgfpathlineto{\pgfqpoint{1.432580in}{0.739656in}}%
\pgfpathlineto{\pgfqpoint{1.432284in}{0.739656in}}%
\pgfpathlineto{\pgfqpoint{1.431988in}{0.739656in}}%
\pgfpathlineto{\pgfqpoint{1.431692in}{0.739656in}}%
\pgfpathlineto{\pgfqpoint{1.431396in}{0.739656in}}%
\pgfpathlineto{\pgfqpoint{1.431100in}{0.739656in}}%
\pgfpathlineto{\pgfqpoint{1.430804in}{0.739656in}}%
\pgfpathlineto{\pgfqpoint{1.430508in}{0.739656in}}%
\pgfpathlineto{\pgfqpoint{1.430212in}{0.739656in}}%
\pgfpathlineto{\pgfqpoint{1.429916in}{0.739656in}}%
\pgfpathlineto{\pgfqpoint{1.429620in}{0.739656in}}%
\pgfpathlineto{\pgfqpoint{1.429324in}{0.739656in}}%
\pgfpathlineto{\pgfqpoint{1.429028in}{0.739656in}}%
\pgfpathlineto{\pgfqpoint{1.428732in}{0.739656in}}%
\pgfpathlineto{\pgfqpoint{1.428436in}{0.739656in}}%
\pgfpathlineto{\pgfqpoint{1.428140in}{0.739656in}}%
\pgfpathlineto{\pgfqpoint{1.427844in}{0.739656in}}%
\pgfpathlineto{\pgfqpoint{1.427548in}{0.739656in}}%
\pgfpathlineto{\pgfqpoint{1.427252in}{0.739656in}}%
\pgfpathlineto{\pgfqpoint{1.426956in}{0.739656in}}%
\pgfpathlineto{\pgfqpoint{1.426660in}{0.739656in}}%
\pgfpathlineto{\pgfqpoint{1.426364in}{0.739656in}}%
\pgfpathlineto{\pgfqpoint{1.426068in}{0.739656in}}%
\pgfpathlineto{\pgfqpoint{1.425772in}{0.739656in}}%
\pgfpathlineto{\pgfqpoint{1.425476in}{0.739656in}}%
\pgfpathlineto{\pgfqpoint{1.425180in}{0.739656in}}%
\pgfpathlineto{\pgfqpoint{1.424884in}{0.739656in}}%
\pgfpathlineto{\pgfqpoint{1.424588in}{0.739656in}}%
\pgfpathlineto{\pgfqpoint{1.424292in}{0.739656in}}%
\pgfpathlineto{\pgfqpoint{1.423996in}{0.739656in}}%
\pgfpathlineto{\pgfqpoint{1.423700in}{0.739656in}}%
\pgfpathlineto{\pgfqpoint{1.423404in}{0.739656in}}%
\pgfpathlineto{\pgfqpoint{1.423108in}{0.739656in}}%
\pgfpathlineto{\pgfqpoint{1.422812in}{0.739656in}}%
\pgfpathlineto{\pgfqpoint{1.422516in}{0.739656in}}%
\pgfpathlineto{\pgfqpoint{1.422220in}{0.739656in}}%
\pgfpathlineto{\pgfqpoint{1.421924in}{0.739656in}}%
\pgfpathlineto{\pgfqpoint{1.421628in}{0.739656in}}%
\pgfpathlineto{\pgfqpoint{1.421332in}{0.739656in}}%
\pgfpathlineto{\pgfqpoint{1.421036in}{0.739656in}}%
\pgfpathlineto{\pgfqpoint{1.420740in}{0.739656in}}%
\pgfpathlineto{\pgfqpoint{1.420444in}{0.739656in}}%
\pgfpathlineto{\pgfqpoint{1.420148in}{0.739656in}}%
\pgfpathlineto{\pgfqpoint{1.419852in}{0.739656in}}%
\pgfpathlineto{\pgfqpoint{1.419556in}{0.739656in}}%
\pgfpathlineto{\pgfqpoint{1.419260in}{0.739656in}}%
\pgfpathlineto{\pgfqpoint{1.418964in}{0.739656in}}%
\pgfpathlineto{\pgfqpoint{1.418668in}{0.739656in}}%
\pgfpathlineto{\pgfqpoint{1.418372in}{0.739656in}}%
\pgfpathlineto{\pgfqpoint{1.418076in}{0.739656in}}%
\pgfpathlineto{\pgfqpoint{1.417780in}{0.739656in}}%
\pgfpathlineto{\pgfqpoint{1.417484in}{0.739656in}}%
\pgfpathlineto{\pgfqpoint{1.417188in}{0.739656in}}%
\pgfpathlineto{\pgfqpoint{1.416892in}{0.739656in}}%
\pgfpathlineto{\pgfqpoint{1.416596in}{0.739656in}}%
\pgfpathlineto{\pgfqpoint{1.416300in}{0.739656in}}%
\pgfpathlineto{\pgfqpoint{1.416004in}{0.739656in}}%
\pgfpathlineto{\pgfqpoint{1.415708in}{0.739656in}}%
\pgfpathlineto{\pgfqpoint{1.415412in}{0.739656in}}%
\pgfpathlineto{\pgfqpoint{1.415116in}{0.739656in}}%
\pgfpathlineto{\pgfqpoint{1.414820in}{0.739656in}}%
\pgfpathlineto{\pgfqpoint{1.414524in}{0.739656in}}%
\pgfpathlineto{\pgfqpoint{1.414228in}{0.739656in}}%
\pgfpathlineto{\pgfqpoint{1.413932in}{0.739656in}}%
\pgfpathlineto{\pgfqpoint{1.413636in}{0.739656in}}%
\pgfpathlineto{\pgfqpoint{1.413340in}{0.739656in}}%
\pgfpathlineto{\pgfqpoint{1.413044in}{0.739656in}}%
\pgfpathlineto{\pgfqpoint{1.412748in}{0.739656in}}%
\pgfpathlineto{\pgfqpoint{1.412452in}{0.739656in}}%
\pgfpathlineto{\pgfqpoint{1.412156in}{0.739656in}}%
\pgfpathlineto{\pgfqpoint{1.411860in}{0.739656in}}%
\pgfpathlineto{\pgfqpoint{1.411564in}{0.739656in}}%
\pgfpathlineto{\pgfqpoint{1.411268in}{0.739656in}}%
\pgfpathlineto{\pgfqpoint{1.410972in}{0.739656in}}%
\pgfpathlineto{\pgfqpoint{1.410676in}{0.739656in}}%
\pgfpathlineto{\pgfqpoint{1.410380in}{0.739656in}}%
\pgfpathlineto{\pgfqpoint{1.410084in}{0.739656in}}%
\pgfpathlineto{\pgfqpoint{1.409788in}{0.739656in}}%
\pgfpathlineto{\pgfqpoint{1.409492in}{0.739656in}}%
\pgfpathlineto{\pgfqpoint{1.409196in}{0.739656in}}%
\pgfpathlineto{\pgfqpoint{1.408900in}{0.739656in}}%
\pgfpathlineto{\pgfqpoint{1.408604in}{0.739656in}}%
\pgfpathlineto{\pgfqpoint{1.408308in}{0.739656in}}%
\pgfpathlineto{\pgfqpoint{1.408012in}{0.739656in}}%
\pgfpathlineto{\pgfqpoint{1.407716in}{0.739656in}}%
\pgfpathlineto{\pgfqpoint{1.407420in}{0.739656in}}%
\pgfpathlineto{\pgfqpoint{1.407124in}{0.739656in}}%
\pgfpathlineto{\pgfqpoint{1.406828in}{0.739656in}}%
\pgfpathlineto{\pgfqpoint{1.406532in}{0.739656in}}%
\pgfpathlineto{\pgfqpoint{1.406236in}{0.739656in}}%
\pgfpathlineto{\pgfqpoint{1.405940in}{0.739656in}}%
\pgfpathlineto{\pgfqpoint{1.405644in}{0.739656in}}%
\pgfpathlineto{\pgfqpoint{1.405348in}{0.739656in}}%
\pgfpathlineto{\pgfqpoint{1.405052in}{0.739656in}}%
\pgfpathlineto{\pgfqpoint{1.404756in}{0.739656in}}%
\pgfpathlineto{\pgfqpoint{1.404460in}{0.739656in}}%
\pgfpathlineto{\pgfqpoint{1.404164in}{0.739656in}}%
\pgfpathlineto{\pgfqpoint{1.403868in}{0.739656in}}%
\pgfpathlineto{\pgfqpoint{1.403572in}{0.739656in}}%
\pgfpathlineto{\pgfqpoint{1.403276in}{0.739656in}}%
\pgfpathlineto{\pgfqpoint{1.402980in}{0.739656in}}%
\pgfpathlineto{\pgfqpoint{1.402684in}{0.739656in}}%
\pgfpathlineto{\pgfqpoint{1.402388in}{0.739656in}}%
\pgfpathlineto{\pgfqpoint{1.402092in}{0.739656in}}%
\pgfpathlineto{\pgfqpoint{1.401796in}{0.739656in}}%
\pgfpathlineto{\pgfqpoint{1.401500in}{0.739656in}}%
\pgfpathlineto{\pgfqpoint{1.401204in}{0.739656in}}%
\pgfpathlineto{\pgfqpoint{1.400908in}{0.739656in}}%
\pgfpathlineto{\pgfqpoint{1.400612in}{0.739656in}}%
\pgfpathlineto{\pgfqpoint{1.400316in}{0.739656in}}%
\pgfpathlineto{\pgfqpoint{1.400020in}{0.739656in}}%
\pgfpathlineto{\pgfqpoint{1.399724in}{0.739656in}}%
\pgfpathlineto{\pgfqpoint{1.399428in}{0.739656in}}%
\pgfpathlineto{\pgfqpoint{1.399132in}{0.739656in}}%
\pgfpathlineto{\pgfqpoint{1.398836in}{0.739656in}}%
\pgfpathlineto{\pgfqpoint{1.398540in}{0.739656in}}%
\pgfpathlineto{\pgfqpoint{1.398244in}{0.739656in}}%
\pgfpathlineto{\pgfqpoint{1.397948in}{0.739656in}}%
\pgfpathlineto{\pgfqpoint{1.397652in}{0.739656in}}%
\pgfpathlineto{\pgfqpoint{1.397356in}{0.739656in}}%
\pgfpathlineto{\pgfqpoint{1.397060in}{0.739656in}}%
\pgfpathlineto{\pgfqpoint{1.396764in}{0.739656in}}%
\pgfpathlineto{\pgfqpoint{1.396468in}{0.739656in}}%
\pgfpathlineto{\pgfqpoint{1.396172in}{0.739656in}}%
\pgfpathlineto{\pgfqpoint{1.395876in}{0.739656in}}%
\pgfpathlineto{\pgfqpoint{1.395580in}{0.739656in}}%
\pgfpathlineto{\pgfqpoint{1.395284in}{0.739656in}}%
\pgfpathlineto{\pgfqpoint{1.394988in}{0.739656in}}%
\pgfpathlineto{\pgfqpoint{1.394692in}{0.739656in}}%
\pgfpathlineto{\pgfqpoint{1.394396in}{0.739656in}}%
\pgfpathlineto{\pgfqpoint{1.394100in}{0.739656in}}%
\pgfpathlineto{\pgfqpoint{1.393804in}{0.739656in}}%
\pgfpathlineto{\pgfqpoint{1.393508in}{0.739656in}}%
\pgfpathlineto{\pgfqpoint{1.393212in}{0.739656in}}%
\pgfpathlineto{\pgfqpoint{1.392916in}{0.739656in}}%
\pgfpathlineto{\pgfqpoint{1.392620in}{0.739656in}}%
\pgfpathlineto{\pgfqpoint{1.392324in}{0.739656in}}%
\pgfpathlineto{\pgfqpoint{1.392028in}{0.739656in}}%
\pgfpathlineto{\pgfqpoint{1.391732in}{0.739656in}}%
\pgfpathlineto{\pgfqpoint{1.391436in}{0.739656in}}%
\pgfpathlineto{\pgfqpoint{1.391140in}{0.739656in}}%
\pgfpathlineto{\pgfqpoint{1.390844in}{0.739656in}}%
\pgfpathlineto{\pgfqpoint{1.390548in}{0.739656in}}%
\pgfpathlineto{\pgfqpoint{1.390252in}{0.739656in}}%
\pgfpathlineto{\pgfqpoint{1.389956in}{0.739656in}}%
\pgfpathlineto{\pgfqpoint{1.389660in}{0.739656in}}%
\pgfpathlineto{\pgfqpoint{1.389363in}{0.739656in}}%
\pgfpathlineto{\pgfqpoint{1.389067in}{0.739656in}}%
\pgfpathlineto{\pgfqpoint{1.388771in}{0.739656in}}%
\pgfpathlineto{\pgfqpoint{1.388475in}{0.739656in}}%
\pgfpathlineto{\pgfqpoint{1.388179in}{0.739656in}}%
\pgfpathlineto{\pgfqpoint{1.387883in}{0.739656in}}%
\pgfpathlineto{\pgfqpoint{1.387587in}{0.739656in}}%
\pgfpathlineto{\pgfqpoint{1.387291in}{0.739656in}}%
\pgfpathlineto{\pgfqpoint{1.386995in}{0.739656in}}%
\pgfpathlineto{\pgfqpoint{1.386699in}{0.739656in}}%
\pgfpathlineto{\pgfqpoint{1.386403in}{0.739656in}}%
\pgfpathlineto{\pgfqpoint{1.386107in}{0.739656in}}%
\pgfpathlineto{\pgfqpoint{1.385811in}{0.739656in}}%
\pgfpathlineto{\pgfqpoint{1.385515in}{0.739656in}}%
\pgfpathlineto{\pgfqpoint{1.385219in}{0.739656in}}%
\pgfpathlineto{\pgfqpoint{1.384923in}{0.739656in}}%
\pgfpathlineto{\pgfqpoint{1.384627in}{0.739656in}}%
\pgfpathlineto{\pgfqpoint{1.384331in}{0.739656in}}%
\pgfpathlineto{\pgfqpoint{1.384035in}{0.739656in}}%
\pgfpathlineto{\pgfqpoint{1.383739in}{0.739656in}}%
\pgfpathlineto{\pgfqpoint{1.383443in}{0.739656in}}%
\pgfpathlineto{\pgfqpoint{1.383147in}{0.739656in}}%
\pgfpathlineto{\pgfqpoint{1.382851in}{0.739656in}}%
\pgfpathlineto{\pgfqpoint{1.382555in}{0.739656in}}%
\pgfpathlineto{\pgfqpoint{1.382259in}{0.739656in}}%
\pgfpathlineto{\pgfqpoint{1.381963in}{0.739656in}}%
\pgfpathlineto{\pgfqpoint{1.381667in}{0.739656in}}%
\pgfpathlineto{\pgfqpoint{1.381371in}{0.739656in}}%
\pgfpathlineto{\pgfqpoint{1.381075in}{0.739656in}}%
\pgfpathlineto{\pgfqpoint{1.380779in}{0.739656in}}%
\pgfpathlineto{\pgfqpoint{1.380483in}{0.739656in}}%
\pgfpathlineto{\pgfqpoint{1.380187in}{0.739656in}}%
\pgfpathlineto{\pgfqpoint{1.379891in}{0.739656in}}%
\pgfpathlineto{\pgfqpoint{1.379595in}{0.739656in}}%
\pgfpathlineto{\pgfqpoint{1.379299in}{0.739656in}}%
\pgfpathlineto{\pgfqpoint{1.379003in}{0.739656in}}%
\pgfpathlineto{\pgfqpoint{1.378707in}{0.739656in}}%
\pgfpathlineto{\pgfqpoint{1.378411in}{0.739656in}}%
\pgfpathlineto{\pgfqpoint{1.378115in}{0.739656in}}%
\pgfpathlineto{\pgfqpoint{1.377819in}{0.739656in}}%
\pgfpathlineto{\pgfqpoint{1.377523in}{0.739656in}}%
\pgfpathlineto{\pgfqpoint{1.377227in}{0.739656in}}%
\pgfpathlineto{\pgfqpoint{1.376931in}{0.739656in}}%
\pgfpathlineto{\pgfqpoint{1.376635in}{0.739656in}}%
\pgfpathlineto{\pgfqpoint{1.376339in}{0.739656in}}%
\pgfpathlineto{\pgfqpoint{1.376043in}{0.739656in}}%
\pgfpathlineto{\pgfqpoint{1.375747in}{0.739656in}}%
\pgfpathlineto{\pgfqpoint{1.375451in}{0.739656in}}%
\pgfpathlineto{\pgfqpoint{1.375155in}{0.739656in}}%
\pgfpathlineto{\pgfqpoint{1.374859in}{0.739656in}}%
\pgfpathlineto{\pgfqpoint{1.374563in}{0.739656in}}%
\pgfpathlineto{\pgfqpoint{1.374267in}{0.739656in}}%
\pgfpathlineto{\pgfqpoint{1.373971in}{0.739656in}}%
\pgfpathlineto{\pgfqpoint{1.373675in}{0.739656in}}%
\pgfpathlineto{\pgfqpoint{1.373379in}{0.739656in}}%
\pgfpathlineto{\pgfqpoint{1.373083in}{0.739656in}}%
\pgfpathlineto{\pgfqpoint{1.372787in}{0.739656in}}%
\pgfpathlineto{\pgfqpoint{1.372491in}{0.739656in}}%
\pgfpathlineto{\pgfqpoint{1.372195in}{0.739656in}}%
\pgfpathlineto{\pgfqpoint{1.371899in}{0.739656in}}%
\pgfpathlineto{\pgfqpoint{1.371603in}{0.739656in}}%
\pgfpathlineto{\pgfqpoint{1.371307in}{0.739656in}}%
\pgfpathlineto{\pgfqpoint{1.371011in}{0.739656in}}%
\pgfpathlineto{\pgfqpoint{1.370715in}{0.739656in}}%
\pgfpathlineto{\pgfqpoint{1.370419in}{0.739656in}}%
\pgfpathlineto{\pgfqpoint{1.370123in}{0.739656in}}%
\pgfpathlineto{\pgfqpoint{1.369827in}{0.739656in}}%
\pgfpathlineto{\pgfqpoint{1.369531in}{0.739656in}}%
\pgfpathlineto{\pgfqpoint{1.369235in}{0.739656in}}%
\pgfpathlineto{\pgfqpoint{1.368939in}{0.739656in}}%
\pgfpathlineto{\pgfqpoint{1.368643in}{0.739656in}}%
\pgfpathlineto{\pgfqpoint{1.368347in}{0.739656in}}%
\pgfpathlineto{\pgfqpoint{1.368051in}{0.739656in}}%
\pgfpathlineto{\pgfqpoint{1.367755in}{0.739656in}}%
\pgfpathlineto{\pgfqpoint{1.367459in}{0.739656in}}%
\pgfpathlineto{\pgfqpoint{1.367163in}{0.739656in}}%
\pgfpathlineto{\pgfqpoint{1.366867in}{0.739656in}}%
\pgfpathlineto{\pgfqpoint{1.366571in}{0.739656in}}%
\pgfpathlineto{\pgfqpoint{1.366275in}{0.739656in}}%
\pgfpathlineto{\pgfqpoint{1.365979in}{0.739656in}}%
\pgfpathlineto{\pgfqpoint{1.365683in}{0.739656in}}%
\pgfpathlineto{\pgfqpoint{1.365387in}{0.739656in}}%
\pgfpathlineto{\pgfqpoint{1.365091in}{0.739656in}}%
\pgfpathlineto{\pgfqpoint{1.364795in}{0.739656in}}%
\pgfpathlineto{\pgfqpoint{1.364499in}{0.739656in}}%
\pgfpathlineto{\pgfqpoint{1.364203in}{0.739656in}}%
\pgfpathlineto{\pgfqpoint{1.363907in}{0.739656in}}%
\pgfpathlineto{\pgfqpoint{1.363611in}{0.739656in}}%
\pgfpathlineto{\pgfqpoint{1.363315in}{0.739656in}}%
\pgfpathlineto{\pgfqpoint{1.363019in}{0.739656in}}%
\pgfpathlineto{\pgfqpoint{1.362723in}{0.739656in}}%
\pgfpathlineto{\pgfqpoint{1.362427in}{0.739656in}}%
\pgfpathlineto{\pgfqpoint{1.362131in}{0.739656in}}%
\pgfpathlineto{\pgfqpoint{1.361835in}{0.739656in}}%
\pgfpathlineto{\pgfqpoint{1.361539in}{0.739656in}}%
\pgfpathlineto{\pgfqpoint{1.361243in}{0.739656in}}%
\pgfpathlineto{\pgfqpoint{1.360947in}{0.739656in}}%
\pgfpathlineto{\pgfqpoint{1.360651in}{0.739656in}}%
\pgfpathlineto{\pgfqpoint{1.360355in}{0.739656in}}%
\pgfpathlineto{\pgfqpoint{1.360059in}{0.739656in}}%
\pgfpathlineto{\pgfqpoint{1.359763in}{0.739656in}}%
\pgfpathlineto{\pgfqpoint{1.359467in}{0.739656in}}%
\pgfpathlineto{\pgfqpoint{1.359171in}{0.739656in}}%
\pgfpathlineto{\pgfqpoint{1.358875in}{0.739656in}}%
\pgfpathlineto{\pgfqpoint{1.358579in}{0.739656in}}%
\pgfpathlineto{\pgfqpoint{1.358283in}{0.739656in}}%
\pgfpathlineto{\pgfqpoint{1.357987in}{0.739656in}}%
\pgfpathlineto{\pgfqpoint{1.357691in}{0.739656in}}%
\pgfpathlineto{\pgfqpoint{1.357395in}{0.739656in}}%
\pgfpathlineto{\pgfqpoint{1.357099in}{0.739656in}}%
\pgfpathlineto{\pgfqpoint{1.356803in}{0.739656in}}%
\pgfpathlineto{\pgfqpoint{1.356507in}{0.739656in}}%
\pgfpathlineto{\pgfqpoint{1.356211in}{0.739656in}}%
\pgfpathlineto{\pgfqpoint{1.355915in}{0.739656in}}%
\pgfpathlineto{\pgfqpoint{1.355619in}{0.739656in}}%
\pgfpathlineto{\pgfqpoint{1.355323in}{0.739656in}}%
\pgfpathlineto{\pgfqpoint{1.355027in}{0.739656in}}%
\pgfpathlineto{\pgfqpoint{1.354731in}{0.739656in}}%
\pgfpathlineto{\pgfqpoint{1.354435in}{0.739656in}}%
\pgfpathlineto{\pgfqpoint{1.354139in}{0.739656in}}%
\pgfpathlineto{\pgfqpoint{1.353843in}{0.739656in}}%
\pgfpathlineto{\pgfqpoint{1.353547in}{0.739656in}}%
\pgfpathlineto{\pgfqpoint{1.353251in}{0.739656in}}%
\pgfpathlineto{\pgfqpoint{1.352955in}{0.739656in}}%
\pgfpathlineto{\pgfqpoint{1.352659in}{0.739656in}}%
\pgfpathlineto{\pgfqpoint{1.352363in}{0.739656in}}%
\pgfpathlineto{\pgfqpoint{1.352067in}{0.739656in}}%
\pgfpathlineto{\pgfqpoint{1.351771in}{0.739656in}}%
\pgfpathlineto{\pgfqpoint{1.351475in}{0.739656in}}%
\pgfpathlineto{\pgfqpoint{1.351179in}{0.739656in}}%
\pgfpathlineto{\pgfqpoint{1.350883in}{0.739656in}}%
\pgfpathlineto{\pgfqpoint{1.350587in}{0.739656in}}%
\pgfpathlineto{\pgfqpoint{1.350291in}{0.739656in}}%
\pgfpathlineto{\pgfqpoint{1.349995in}{0.739656in}}%
\pgfpathlineto{\pgfqpoint{1.349699in}{0.739656in}}%
\pgfpathlineto{\pgfqpoint{1.349403in}{0.739656in}}%
\pgfpathlineto{\pgfqpoint{1.349107in}{0.739656in}}%
\pgfpathlineto{\pgfqpoint{1.348811in}{0.739656in}}%
\pgfpathlineto{\pgfqpoint{1.348515in}{0.739656in}}%
\pgfpathlineto{\pgfqpoint{1.348219in}{0.739656in}}%
\pgfpathlineto{\pgfqpoint{1.347923in}{0.739656in}}%
\pgfpathlineto{\pgfqpoint{1.347627in}{0.739656in}}%
\pgfpathlineto{\pgfqpoint{1.347331in}{0.739656in}}%
\pgfpathlineto{\pgfqpoint{1.347035in}{0.739656in}}%
\pgfpathlineto{\pgfqpoint{1.346739in}{0.739656in}}%
\pgfpathlineto{\pgfqpoint{1.346443in}{0.739656in}}%
\pgfpathlineto{\pgfqpoint{1.346147in}{0.739656in}}%
\pgfpathlineto{\pgfqpoint{1.345851in}{0.739656in}}%
\pgfpathlineto{\pgfqpoint{1.345555in}{0.739656in}}%
\pgfpathlineto{\pgfqpoint{1.345259in}{0.739656in}}%
\pgfpathlineto{\pgfqpoint{1.344963in}{0.739656in}}%
\pgfpathlineto{\pgfqpoint{1.344667in}{0.739656in}}%
\pgfpathlineto{\pgfqpoint{1.344371in}{0.739656in}}%
\pgfpathlineto{\pgfqpoint{1.344075in}{0.739656in}}%
\pgfpathlineto{\pgfqpoint{1.343779in}{0.739656in}}%
\pgfpathlineto{\pgfqpoint{1.343483in}{0.739656in}}%
\pgfpathlineto{\pgfqpoint{1.343187in}{0.739656in}}%
\pgfpathlineto{\pgfqpoint{1.342891in}{0.739656in}}%
\pgfpathlineto{\pgfqpoint{1.342595in}{0.739656in}}%
\pgfpathlineto{\pgfqpoint{1.342299in}{0.739656in}}%
\pgfpathlineto{\pgfqpoint{1.342003in}{0.739656in}}%
\pgfpathlineto{\pgfqpoint{1.341707in}{0.739656in}}%
\pgfpathlineto{\pgfqpoint{1.341411in}{0.739656in}}%
\pgfpathlineto{\pgfqpoint{1.341115in}{0.739656in}}%
\pgfpathlineto{\pgfqpoint{1.340819in}{0.739656in}}%
\pgfpathlineto{\pgfqpoint{1.340523in}{0.739656in}}%
\pgfpathlineto{\pgfqpoint{1.340227in}{0.739656in}}%
\pgfpathlineto{\pgfqpoint{1.339931in}{0.739656in}}%
\pgfpathlineto{\pgfqpoint{1.339635in}{0.739656in}}%
\pgfpathlineto{\pgfqpoint{1.339339in}{0.739656in}}%
\pgfpathlineto{\pgfqpoint{1.339043in}{0.739656in}}%
\pgfpathlineto{\pgfqpoint{1.338747in}{0.739656in}}%
\pgfpathlineto{\pgfqpoint{1.338451in}{0.739656in}}%
\pgfpathlineto{\pgfqpoint{1.338155in}{0.739656in}}%
\pgfpathlineto{\pgfqpoint{1.337859in}{0.739656in}}%
\pgfpathlineto{\pgfqpoint{1.337563in}{0.739656in}}%
\pgfpathlineto{\pgfqpoint{1.337267in}{0.739656in}}%
\pgfpathlineto{\pgfqpoint{1.336971in}{0.739656in}}%
\pgfpathlineto{\pgfqpoint{1.336675in}{0.739656in}}%
\pgfpathlineto{\pgfqpoint{1.336379in}{0.739656in}}%
\pgfpathlineto{\pgfqpoint{1.336083in}{0.739656in}}%
\pgfpathlineto{\pgfqpoint{1.335787in}{0.739656in}}%
\pgfpathlineto{\pgfqpoint{1.335491in}{0.739656in}}%
\pgfpathlineto{\pgfqpoint{1.335195in}{0.739656in}}%
\pgfpathlineto{\pgfqpoint{1.334899in}{0.739656in}}%
\pgfpathlineto{\pgfqpoint{1.334603in}{0.739656in}}%
\pgfpathlineto{\pgfqpoint{1.334307in}{0.739656in}}%
\pgfpathlineto{\pgfqpoint{1.334011in}{0.739656in}}%
\pgfpathlineto{\pgfqpoint{1.333715in}{0.739656in}}%
\pgfpathlineto{\pgfqpoint{1.333419in}{0.739656in}}%
\pgfpathlineto{\pgfqpoint{1.333123in}{0.739656in}}%
\pgfpathlineto{\pgfqpoint{1.332827in}{0.739656in}}%
\pgfpathlineto{\pgfqpoint{1.332531in}{0.739656in}}%
\pgfpathlineto{\pgfqpoint{1.332235in}{0.739656in}}%
\pgfpathlineto{\pgfqpoint{1.331939in}{0.739656in}}%
\pgfpathlineto{\pgfqpoint{1.331643in}{0.739656in}}%
\pgfpathlineto{\pgfqpoint{1.331347in}{0.739656in}}%
\pgfpathlineto{\pgfqpoint{1.331051in}{0.739656in}}%
\pgfpathlineto{\pgfqpoint{1.330755in}{0.739656in}}%
\pgfpathlineto{\pgfqpoint{1.330459in}{0.739656in}}%
\pgfpathlineto{\pgfqpoint{1.330163in}{0.739656in}}%
\pgfpathlineto{\pgfqpoint{1.329867in}{0.739656in}}%
\pgfpathlineto{\pgfqpoint{1.329571in}{0.739656in}}%
\pgfpathlineto{\pgfqpoint{1.329275in}{0.739656in}}%
\pgfpathlineto{\pgfqpoint{1.328979in}{0.739656in}}%
\pgfpathlineto{\pgfqpoint{1.328683in}{0.739656in}}%
\pgfpathlineto{\pgfqpoint{1.328387in}{0.739656in}}%
\pgfpathlineto{\pgfqpoint{1.328091in}{0.739656in}}%
\pgfpathlineto{\pgfqpoint{1.327795in}{0.739656in}}%
\pgfpathlineto{\pgfqpoint{1.327499in}{0.739656in}}%
\pgfpathlineto{\pgfqpoint{1.327203in}{0.739656in}}%
\pgfpathlineto{\pgfqpoint{1.326907in}{0.739656in}}%
\pgfpathlineto{\pgfqpoint{1.326611in}{0.739656in}}%
\pgfpathlineto{\pgfqpoint{1.326315in}{0.739656in}}%
\pgfpathlineto{\pgfqpoint{1.326019in}{0.739656in}}%
\pgfpathlineto{\pgfqpoint{1.325723in}{0.739656in}}%
\pgfpathlineto{\pgfqpoint{1.325427in}{0.739656in}}%
\pgfpathlineto{\pgfqpoint{1.325131in}{0.739656in}}%
\pgfpathlineto{\pgfqpoint{1.324835in}{0.739656in}}%
\pgfpathlineto{\pgfqpoint{1.324539in}{0.739656in}}%
\pgfpathlineto{\pgfqpoint{1.324243in}{0.739656in}}%
\pgfpathlineto{\pgfqpoint{1.323947in}{0.739656in}}%
\pgfpathlineto{\pgfqpoint{1.323651in}{0.739656in}}%
\pgfpathlineto{\pgfqpoint{1.323355in}{0.739656in}}%
\pgfpathlineto{\pgfqpoint{1.323059in}{0.739656in}}%
\pgfpathlineto{\pgfqpoint{1.322763in}{0.739656in}}%
\pgfpathlineto{\pgfqpoint{1.322467in}{0.739656in}}%
\pgfpathlineto{\pgfqpoint{1.322171in}{0.739656in}}%
\pgfpathlineto{\pgfqpoint{1.321874in}{0.739656in}}%
\pgfpathlineto{\pgfqpoint{1.321578in}{0.739656in}}%
\pgfpathlineto{\pgfqpoint{1.321282in}{0.739656in}}%
\pgfpathlineto{\pgfqpoint{1.320986in}{0.739656in}}%
\pgfpathlineto{\pgfqpoint{1.320690in}{0.739656in}}%
\pgfpathlineto{\pgfqpoint{1.320394in}{0.739656in}}%
\pgfpathlineto{\pgfqpoint{1.320098in}{0.739656in}}%
\pgfpathlineto{\pgfqpoint{1.319802in}{0.739656in}}%
\pgfpathlineto{\pgfqpoint{1.319506in}{0.739656in}}%
\pgfpathlineto{\pgfqpoint{1.319210in}{0.739656in}}%
\pgfpathlineto{\pgfqpoint{1.318914in}{0.739656in}}%
\pgfpathlineto{\pgfqpoint{1.318618in}{0.739656in}}%
\pgfpathlineto{\pgfqpoint{1.318322in}{0.739656in}}%
\pgfpathlineto{\pgfqpoint{1.318026in}{0.739656in}}%
\pgfpathlineto{\pgfqpoint{1.317730in}{0.739656in}}%
\pgfpathlineto{\pgfqpoint{1.317434in}{0.739656in}}%
\pgfpathlineto{\pgfqpoint{1.317138in}{0.739656in}}%
\pgfpathlineto{\pgfqpoint{1.316842in}{0.739656in}}%
\pgfpathlineto{\pgfqpoint{1.316546in}{0.739656in}}%
\pgfpathlineto{\pgfqpoint{1.316250in}{0.739656in}}%
\pgfpathlineto{\pgfqpoint{1.315954in}{0.739656in}}%
\pgfpathlineto{\pgfqpoint{1.315658in}{0.739656in}}%
\pgfpathlineto{\pgfqpoint{1.315362in}{0.739656in}}%
\pgfpathlineto{\pgfqpoint{1.315066in}{0.739656in}}%
\pgfpathlineto{\pgfqpoint{1.314770in}{0.739656in}}%
\pgfpathlineto{\pgfqpoint{1.314474in}{0.739656in}}%
\pgfpathlineto{\pgfqpoint{1.314178in}{0.739656in}}%
\pgfpathlineto{\pgfqpoint{1.313882in}{0.739656in}}%
\pgfpathlineto{\pgfqpoint{1.313586in}{0.739656in}}%
\pgfpathlineto{\pgfqpoint{1.313290in}{0.739656in}}%
\pgfpathlineto{\pgfqpoint{1.312994in}{0.739656in}}%
\pgfpathlineto{\pgfqpoint{1.312698in}{0.739656in}}%
\pgfpathlineto{\pgfqpoint{1.312402in}{0.739656in}}%
\pgfpathlineto{\pgfqpoint{1.312106in}{0.739656in}}%
\pgfpathlineto{\pgfqpoint{1.311810in}{0.739656in}}%
\pgfpathlineto{\pgfqpoint{1.311514in}{0.739656in}}%
\pgfpathlineto{\pgfqpoint{1.311218in}{0.739656in}}%
\pgfpathlineto{\pgfqpoint{1.310922in}{0.739656in}}%
\pgfpathlineto{\pgfqpoint{1.310626in}{0.739656in}}%
\pgfpathlineto{\pgfqpoint{1.310330in}{0.739656in}}%
\pgfpathlineto{\pgfqpoint{1.310034in}{0.739656in}}%
\pgfpathlineto{\pgfqpoint{1.309738in}{0.739656in}}%
\pgfpathlineto{\pgfqpoint{1.309442in}{0.739656in}}%
\pgfpathlineto{\pgfqpoint{1.309146in}{0.739656in}}%
\pgfpathlineto{\pgfqpoint{1.308850in}{0.739656in}}%
\pgfpathlineto{\pgfqpoint{1.308554in}{0.739656in}}%
\pgfpathlineto{\pgfqpoint{1.308258in}{0.739656in}}%
\pgfpathlineto{\pgfqpoint{1.307962in}{0.739656in}}%
\pgfpathlineto{\pgfqpoint{1.307666in}{0.739656in}}%
\pgfpathlineto{\pgfqpoint{1.307370in}{0.739656in}}%
\pgfpathlineto{\pgfqpoint{1.307074in}{0.739656in}}%
\pgfpathlineto{\pgfqpoint{1.306778in}{0.739656in}}%
\pgfpathlineto{\pgfqpoint{1.306482in}{0.739656in}}%
\pgfpathlineto{\pgfqpoint{1.306186in}{0.739656in}}%
\pgfpathlineto{\pgfqpoint{1.305890in}{0.739656in}}%
\pgfpathlineto{\pgfqpoint{1.305594in}{0.739656in}}%
\pgfpathlineto{\pgfqpoint{1.305298in}{0.739656in}}%
\pgfpathlineto{\pgfqpoint{1.305002in}{0.739656in}}%
\pgfpathlineto{\pgfqpoint{1.304706in}{0.739656in}}%
\pgfpathlineto{\pgfqpoint{1.304410in}{0.739656in}}%
\pgfpathlineto{\pgfqpoint{1.304114in}{0.739656in}}%
\pgfpathlineto{\pgfqpoint{1.303818in}{0.739656in}}%
\pgfpathlineto{\pgfqpoint{1.303522in}{0.739656in}}%
\pgfpathlineto{\pgfqpoint{1.303226in}{0.739656in}}%
\pgfpathlineto{\pgfqpoint{1.302930in}{0.739656in}}%
\pgfpathlineto{\pgfqpoint{1.302634in}{0.739656in}}%
\pgfpathlineto{\pgfqpoint{1.302338in}{0.739656in}}%
\pgfpathlineto{\pgfqpoint{1.302042in}{0.739656in}}%
\pgfpathlineto{\pgfqpoint{1.301746in}{0.739656in}}%
\pgfpathlineto{\pgfqpoint{1.301450in}{0.739656in}}%
\pgfpathlineto{\pgfqpoint{1.301154in}{0.739656in}}%
\pgfpathlineto{\pgfqpoint{1.300858in}{0.739656in}}%
\pgfpathlineto{\pgfqpoint{1.300562in}{0.739656in}}%
\pgfpathlineto{\pgfqpoint{1.300266in}{0.739656in}}%
\pgfpathlineto{\pgfqpoint{1.299970in}{0.739656in}}%
\pgfpathlineto{\pgfqpoint{1.299674in}{0.739656in}}%
\pgfpathlineto{\pgfqpoint{1.299378in}{0.739656in}}%
\pgfpathlineto{\pgfqpoint{1.299082in}{0.739656in}}%
\pgfpathlineto{\pgfqpoint{1.298786in}{0.739656in}}%
\pgfpathlineto{\pgfqpoint{1.298490in}{0.739656in}}%
\pgfpathlineto{\pgfqpoint{1.298194in}{0.739656in}}%
\pgfpathlineto{\pgfqpoint{1.297898in}{0.739656in}}%
\pgfpathlineto{\pgfqpoint{1.297602in}{0.739656in}}%
\pgfpathlineto{\pgfqpoint{1.297306in}{0.739656in}}%
\pgfpathlineto{\pgfqpoint{1.297010in}{0.739656in}}%
\pgfpathlineto{\pgfqpoint{1.296714in}{0.739656in}}%
\pgfpathlineto{\pgfqpoint{1.296418in}{0.739656in}}%
\pgfpathlineto{\pgfqpoint{1.296122in}{0.739656in}}%
\pgfpathlineto{\pgfqpoint{1.295826in}{0.739656in}}%
\pgfpathlineto{\pgfqpoint{1.295530in}{0.739656in}}%
\pgfpathlineto{\pgfqpoint{1.295234in}{0.739656in}}%
\pgfpathlineto{\pgfqpoint{1.294938in}{0.739656in}}%
\pgfpathlineto{\pgfqpoint{1.294642in}{0.739656in}}%
\pgfpathlineto{\pgfqpoint{1.294346in}{0.739656in}}%
\pgfpathlineto{\pgfqpoint{1.294050in}{0.739656in}}%
\pgfpathlineto{\pgfqpoint{1.293754in}{0.739656in}}%
\pgfpathlineto{\pgfqpoint{1.293458in}{0.739656in}}%
\pgfpathlineto{\pgfqpoint{1.293162in}{0.739656in}}%
\pgfpathlineto{\pgfqpoint{1.292866in}{0.739656in}}%
\pgfpathlineto{\pgfqpoint{1.292570in}{0.739656in}}%
\pgfpathlineto{\pgfqpoint{1.292274in}{0.739656in}}%
\pgfpathlineto{\pgfqpoint{1.291978in}{0.739656in}}%
\pgfpathlineto{\pgfqpoint{1.291682in}{0.739656in}}%
\pgfpathlineto{\pgfqpoint{1.291386in}{0.739656in}}%
\pgfpathlineto{\pgfqpoint{1.291090in}{0.739656in}}%
\pgfpathlineto{\pgfqpoint{1.290794in}{0.739656in}}%
\pgfpathlineto{\pgfqpoint{1.290498in}{0.739656in}}%
\pgfpathlineto{\pgfqpoint{1.290202in}{0.739656in}}%
\pgfpathlineto{\pgfqpoint{1.289906in}{0.739656in}}%
\pgfpathlineto{\pgfqpoint{1.289610in}{0.739656in}}%
\pgfpathlineto{\pgfqpoint{1.289314in}{0.739656in}}%
\pgfpathlineto{\pgfqpoint{1.289018in}{0.739656in}}%
\pgfpathlineto{\pgfqpoint{1.288722in}{0.739656in}}%
\pgfpathlineto{\pgfqpoint{1.288426in}{0.739656in}}%
\pgfpathlineto{\pgfqpoint{1.288130in}{0.739656in}}%
\pgfpathlineto{\pgfqpoint{1.287834in}{0.739656in}}%
\pgfpathlineto{\pgfqpoint{1.287538in}{0.739656in}}%
\pgfpathlineto{\pgfqpoint{1.287242in}{0.739656in}}%
\pgfpathlineto{\pgfqpoint{1.286946in}{0.739656in}}%
\pgfpathlineto{\pgfqpoint{1.286650in}{0.739656in}}%
\pgfpathlineto{\pgfqpoint{1.286354in}{0.739656in}}%
\pgfpathlineto{\pgfqpoint{1.286058in}{0.739656in}}%
\pgfpathlineto{\pgfqpoint{1.285762in}{0.739656in}}%
\pgfpathlineto{\pgfqpoint{1.285466in}{0.739656in}}%
\pgfpathlineto{\pgfqpoint{1.285170in}{0.739656in}}%
\pgfpathlineto{\pgfqpoint{1.284874in}{0.739656in}}%
\pgfpathlineto{\pgfqpoint{1.284578in}{0.739656in}}%
\pgfpathlineto{\pgfqpoint{1.284282in}{0.739656in}}%
\pgfpathlineto{\pgfqpoint{1.283986in}{0.739656in}}%
\pgfpathlineto{\pgfqpoint{1.283690in}{0.739656in}}%
\pgfpathlineto{\pgfqpoint{1.283394in}{0.739656in}}%
\pgfpathlineto{\pgfqpoint{1.283098in}{0.739656in}}%
\pgfpathlineto{\pgfqpoint{1.282802in}{0.739656in}}%
\pgfpathlineto{\pgfqpoint{1.282506in}{0.739656in}}%
\pgfpathlineto{\pgfqpoint{1.282210in}{0.739656in}}%
\pgfpathlineto{\pgfqpoint{1.281914in}{0.739656in}}%
\pgfpathlineto{\pgfqpoint{1.281618in}{0.739656in}}%
\pgfpathlineto{\pgfqpoint{1.281322in}{0.739656in}}%
\pgfpathlineto{\pgfqpoint{1.281026in}{0.739656in}}%
\pgfpathlineto{\pgfqpoint{1.280730in}{0.739656in}}%
\pgfpathlineto{\pgfqpoint{1.280434in}{0.739656in}}%
\pgfpathlineto{\pgfqpoint{1.280138in}{0.739656in}}%
\pgfpathlineto{\pgfqpoint{1.279842in}{0.739656in}}%
\pgfpathlineto{\pgfqpoint{1.279546in}{0.739656in}}%
\pgfpathlineto{\pgfqpoint{1.279250in}{0.739656in}}%
\pgfpathlineto{\pgfqpoint{1.278954in}{0.739656in}}%
\pgfpathlineto{\pgfqpoint{1.278658in}{0.739656in}}%
\pgfpathlineto{\pgfqpoint{1.278362in}{0.739656in}}%
\pgfpathlineto{\pgfqpoint{1.278066in}{0.739656in}}%
\pgfpathlineto{\pgfqpoint{1.277770in}{0.739656in}}%
\pgfpathlineto{\pgfqpoint{1.277474in}{0.739656in}}%
\pgfpathlineto{\pgfqpoint{1.277178in}{0.739656in}}%
\pgfpathlineto{\pgfqpoint{1.276882in}{0.739656in}}%
\pgfpathlineto{\pgfqpoint{1.276586in}{0.739656in}}%
\pgfpathlineto{\pgfqpoint{1.276290in}{0.739656in}}%
\pgfpathlineto{\pgfqpoint{1.275994in}{0.739656in}}%
\pgfpathlineto{\pgfqpoint{1.275698in}{0.739656in}}%
\pgfpathlineto{\pgfqpoint{1.275402in}{0.739656in}}%
\pgfpathlineto{\pgfqpoint{1.275106in}{0.739656in}}%
\pgfpathlineto{\pgfqpoint{1.274810in}{0.739656in}}%
\pgfpathlineto{\pgfqpoint{1.274514in}{0.739656in}}%
\pgfpathlineto{\pgfqpoint{1.274218in}{0.739656in}}%
\pgfpathlineto{\pgfqpoint{1.273922in}{0.739656in}}%
\pgfpathlineto{\pgfqpoint{1.273626in}{0.739656in}}%
\pgfpathlineto{\pgfqpoint{1.273330in}{0.739656in}}%
\pgfpathlineto{\pgfqpoint{1.273034in}{0.739656in}}%
\pgfpathlineto{\pgfqpoint{1.272738in}{0.739656in}}%
\pgfpathlineto{\pgfqpoint{1.272442in}{0.739656in}}%
\pgfpathlineto{\pgfqpoint{1.272146in}{0.739656in}}%
\pgfpathlineto{\pgfqpoint{1.271850in}{0.739656in}}%
\pgfpathlineto{\pgfqpoint{1.271554in}{0.739656in}}%
\pgfpathlineto{\pgfqpoint{1.271258in}{0.739656in}}%
\pgfpathlineto{\pgfqpoint{1.270962in}{0.739656in}}%
\pgfpathlineto{\pgfqpoint{1.270666in}{0.739656in}}%
\pgfpathlineto{\pgfqpoint{1.270370in}{0.739656in}}%
\pgfpathlineto{\pgfqpoint{1.270074in}{0.739656in}}%
\pgfpathlineto{\pgfqpoint{1.269778in}{0.739656in}}%
\pgfpathlineto{\pgfqpoint{1.269482in}{0.739656in}}%
\pgfpathlineto{\pgfqpoint{1.269186in}{0.739656in}}%
\pgfpathlineto{\pgfqpoint{1.268890in}{0.739656in}}%
\pgfpathlineto{\pgfqpoint{1.268594in}{0.739656in}}%
\pgfpathlineto{\pgfqpoint{1.268298in}{0.739656in}}%
\pgfpathlineto{\pgfqpoint{1.268002in}{0.739656in}}%
\pgfpathlineto{\pgfqpoint{1.267706in}{0.739656in}}%
\pgfpathlineto{\pgfqpoint{1.267410in}{0.739656in}}%
\pgfpathlineto{\pgfqpoint{1.267114in}{0.739656in}}%
\pgfpathlineto{\pgfqpoint{1.266818in}{0.739656in}}%
\pgfpathlineto{\pgfqpoint{1.266522in}{0.739656in}}%
\pgfpathlineto{\pgfqpoint{1.266226in}{0.739656in}}%
\pgfpathlineto{\pgfqpoint{1.265930in}{0.739656in}}%
\pgfpathlineto{\pgfqpoint{1.265634in}{0.739656in}}%
\pgfpathlineto{\pgfqpoint{1.265338in}{0.739656in}}%
\pgfpathlineto{\pgfqpoint{1.265042in}{0.739656in}}%
\pgfpathlineto{\pgfqpoint{1.264746in}{0.739656in}}%
\pgfpathlineto{\pgfqpoint{1.264450in}{0.739656in}}%
\pgfpathlineto{\pgfqpoint{1.264154in}{0.739656in}}%
\pgfpathlineto{\pgfqpoint{1.263858in}{0.739656in}}%
\pgfpathlineto{\pgfqpoint{1.263562in}{0.739656in}}%
\pgfpathlineto{\pgfqpoint{1.263266in}{0.739656in}}%
\pgfpathlineto{\pgfqpoint{1.262970in}{0.739656in}}%
\pgfpathlineto{\pgfqpoint{1.262674in}{0.739656in}}%
\pgfpathlineto{\pgfqpoint{1.262378in}{0.739656in}}%
\pgfpathlineto{\pgfqpoint{1.262082in}{0.739656in}}%
\pgfpathlineto{\pgfqpoint{1.261786in}{0.739656in}}%
\pgfpathlineto{\pgfqpoint{1.261490in}{0.739656in}}%
\pgfpathlineto{\pgfqpoint{1.261194in}{0.739656in}}%
\pgfpathlineto{\pgfqpoint{1.260898in}{0.739656in}}%
\pgfpathlineto{\pgfqpoint{1.260602in}{0.739656in}}%
\pgfpathlineto{\pgfqpoint{1.260306in}{0.739656in}}%
\pgfpathlineto{\pgfqpoint{1.260010in}{0.739656in}}%
\pgfpathlineto{\pgfqpoint{1.259714in}{0.739656in}}%
\pgfpathlineto{\pgfqpoint{1.259418in}{0.739656in}}%
\pgfpathlineto{\pgfqpoint{1.259122in}{0.739656in}}%
\pgfpathlineto{\pgfqpoint{1.258826in}{0.739656in}}%
\pgfpathlineto{\pgfqpoint{1.258530in}{0.739656in}}%
\pgfpathlineto{\pgfqpoint{1.258234in}{0.739656in}}%
\pgfpathlineto{\pgfqpoint{1.257938in}{0.739656in}}%
\pgfpathlineto{\pgfqpoint{1.257642in}{0.739656in}}%
\pgfpathlineto{\pgfqpoint{1.257346in}{0.739656in}}%
\pgfpathlineto{\pgfqpoint{1.257050in}{0.739656in}}%
\pgfpathlineto{\pgfqpoint{1.256754in}{0.739656in}}%
\pgfpathlineto{\pgfqpoint{1.256458in}{0.739656in}}%
\pgfpathlineto{\pgfqpoint{1.256162in}{0.739656in}}%
\pgfpathlineto{\pgfqpoint{1.255866in}{0.739656in}}%
\pgfpathlineto{\pgfqpoint{1.255570in}{0.739656in}}%
\pgfpathlineto{\pgfqpoint{1.255274in}{0.739656in}}%
\pgfpathlineto{\pgfqpoint{1.254978in}{0.739656in}}%
\pgfpathlineto{\pgfqpoint{1.254681in}{0.739656in}}%
\pgfpathlineto{\pgfqpoint{1.254385in}{0.739656in}}%
\pgfpathlineto{\pgfqpoint{1.254089in}{0.739656in}}%
\pgfpathlineto{\pgfqpoint{1.253793in}{0.739656in}}%
\pgfpathlineto{\pgfqpoint{1.253497in}{0.739656in}}%
\pgfpathlineto{\pgfqpoint{1.253201in}{0.739656in}}%
\pgfpathlineto{\pgfqpoint{1.252905in}{0.739656in}}%
\pgfpathlineto{\pgfqpoint{1.252609in}{0.739656in}}%
\pgfpathlineto{\pgfqpoint{1.252313in}{0.739656in}}%
\pgfpathlineto{\pgfqpoint{1.252017in}{0.739656in}}%
\pgfpathlineto{\pgfqpoint{1.251721in}{0.739656in}}%
\pgfpathlineto{\pgfqpoint{1.251425in}{0.739656in}}%
\pgfpathlineto{\pgfqpoint{1.251129in}{0.739656in}}%
\pgfpathlineto{\pgfqpoint{1.250833in}{0.739656in}}%
\pgfpathlineto{\pgfqpoint{1.250537in}{0.739656in}}%
\pgfpathlineto{\pgfqpoint{1.250241in}{0.739656in}}%
\pgfpathlineto{\pgfqpoint{1.249945in}{0.739656in}}%
\pgfpathlineto{\pgfqpoint{1.249649in}{0.739656in}}%
\pgfpathlineto{\pgfqpoint{1.249353in}{0.739656in}}%
\pgfpathlineto{\pgfqpoint{1.249057in}{0.739656in}}%
\pgfpathlineto{\pgfqpoint{1.248761in}{0.739656in}}%
\pgfpathlineto{\pgfqpoint{1.248465in}{0.739656in}}%
\pgfpathlineto{\pgfqpoint{1.248169in}{0.739656in}}%
\pgfpathlineto{\pgfqpoint{1.247873in}{0.739656in}}%
\pgfpathlineto{\pgfqpoint{1.247577in}{0.739656in}}%
\pgfpathlineto{\pgfqpoint{1.247281in}{0.739656in}}%
\pgfpathlineto{\pgfqpoint{1.246985in}{0.739656in}}%
\pgfpathlineto{\pgfqpoint{1.246689in}{0.739656in}}%
\pgfpathlineto{\pgfqpoint{1.246393in}{0.739656in}}%
\pgfpathlineto{\pgfqpoint{1.246097in}{0.739656in}}%
\pgfpathlineto{\pgfqpoint{1.245801in}{0.739656in}}%
\pgfpathlineto{\pgfqpoint{1.245505in}{0.739656in}}%
\pgfpathlineto{\pgfqpoint{1.245209in}{0.739656in}}%
\pgfpathlineto{\pgfqpoint{1.244913in}{0.739656in}}%
\pgfpathlineto{\pgfqpoint{1.244617in}{0.739656in}}%
\pgfpathlineto{\pgfqpoint{1.244321in}{0.739656in}}%
\pgfpathlineto{\pgfqpoint{1.244025in}{0.739656in}}%
\pgfpathlineto{\pgfqpoint{1.243729in}{0.739656in}}%
\pgfpathlineto{\pgfqpoint{1.243433in}{0.739656in}}%
\pgfpathlineto{\pgfqpoint{1.243137in}{0.739656in}}%
\pgfpathlineto{\pgfqpoint{1.242841in}{0.739656in}}%
\pgfpathlineto{\pgfqpoint{1.242545in}{0.739656in}}%
\pgfpathlineto{\pgfqpoint{1.242249in}{0.739656in}}%
\pgfpathlineto{\pgfqpoint{1.241953in}{0.739656in}}%
\pgfpathlineto{\pgfqpoint{1.241657in}{0.739656in}}%
\pgfpathlineto{\pgfqpoint{1.241361in}{0.739656in}}%
\pgfpathlineto{\pgfqpoint{1.241065in}{0.739656in}}%
\pgfpathlineto{\pgfqpoint{1.240769in}{0.739656in}}%
\pgfpathlineto{\pgfqpoint{1.240473in}{0.739656in}}%
\pgfpathlineto{\pgfqpoint{1.240177in}{0.739656in}}%
\pgfpathlineto{\pgfqpoint{1.239881in}{0.739656in}}%
\pgfpathlineto{\pgfqpoint{1.239585in}{0.739656in}}%
\pgfpathlineto{\pgfqpoint{1.239289in}{0.739656in}}%
\pgfpathlineto{\pgfqpoint{1.238993in}{0.739656in}}%
\pgfpathlineto{\pgfqpoint{1.238697in}{0.739656in}}%
\pgfpathlineto{\pgfqpoint{1.238401in}{0.739656in}}%
\pgfpathlineto{\pgfqpoint{1.238105in}{0.739656in}}%
\pgfpathlineto{\pgfqpoint{1.237809in}{0.739656in}}%
\pgfpathlineto{\pgfqpoint{1.237513in}{0.739656in}}%
\pgfpathlineto{\pgfqpoint{1.237217in}{0.739656in}}%
\pgfpathlineto{\pgfqpoint{1.236921in}{0.739656in}}%
\pgfpathlineto{\pgfqpoint{1.236625in}{0.739656in}}%
\pgfpathlineto{\pgfqpoint{1.236329in}{0.739656in}}%
\pgfpathlineto{\pgfqpoint{1.236033in}{0.739656in}}%
\pgfpathlineto{\pgfqpoint{1.235737in}{0.739656in}}%
\pgfpathlineto{\pgfqpoint{1.235441in}{0.739656in}}%
\pgfpathlineto{\pgfqpoint{1.235145in}{0.739656in}}%
\pgfpathlineto{\pgfqpoint{1.234849in}{0.739656in}}%
\pgfpathlineto{\pgfqpoint{1.234553in}{0.739656in}}%
\pgfpathlineto{\pgfqpoint{1.234257in}{0.739656in}}%
\pgfpathlineto{\pgfqpoint{1.233961in}{0.739656in}}%
\pgfpathlineto{\pgfqpoint{1.233665in}{0.739656in}}%
\pgfpathlineto{\pgfqpoint{1.233369in}{0.739656in}}%
\pgfpathlineto{\pgfqpoint{1.233073in}{0.739656in}}%
\pgfpathlineto{\pgfqpoint{1.232777in}{0.739656in}}%
\pgfpathlineto{\pgfqpoint{1.232481in}{0.739656in}}%
\pgfpathlineto{\pgfqpoint{1.232185in}{0.739656in}}%
\pgfpathlineto{\pgfqpoint{1.231889in}{0.739656in}}%
\pgfpathlineto{\pgfqpoint{1.231593in}{0.739656in}}%
\pgfpathlineto{\pgfqpoint{1.231297in}{0.739656in}}%
\pgfpathlineto{\pgfqpoint{1.231001in}{0.739656in}}%
\pgfpathlineto{\pgfqpoint{1.230705in}{0.739656in}}%
\pgfpathlineto{\pgfqpoint{1.230409in}{0.739656in}}%
\pgfpathlineto{\pgfqpoint{1.230113in}{0.739656in}}%
\pgfpathlineto{\pgfqpoint{1.229817in}{0.739656in}}%
\pgfpathlineto{\pgfqpoint{1.229521in}{0.739656in}}%
\pgfpathlineto{\pgfqpoint{1.229225in}{0.739656in}}%
\pgfpathlineto{\pgfqpoint{1.228929in}{0.739656in}}%
\pgfpathlineto{\pgfqpoint{1.228633in}{0.739656in}}%
\pgfpathlineto{\pgfqpoint{1.228337in}{0.739656in}}%
\pgfpathlineto{\pgfqpoint{1.228041in}{0.739656in}}%
\pgfpathlineto{\pgfqpoint{1.227745in}{0.739656in}}%
\pgfpathlineto{\pgfqpoint{1.227449in}{0.739656in}}%
\pgfpathlineto{\pgfqpoint{1.227153in}{0.739656in}}%
\pgfpathlineto{\pgfqpoint{1.226857in}{0.739656in}}%
\pgfpathlineto{\pgfqpoint{1.226561in}{0.739656in}}%
\pgfpathlineto{\pgfqpoint{1.226265in}{0.739656in}}%
\pgfpathlineto{\pgfqpoint{1.225969in}{0.739656in}}%
\pgfpathlineto{\pgfqpoint{1.225673in}{0.739656in}}%
\pgfpathlineto{\pgfqpoint{1.225377in}{0.739656in}}%
\pgfpathlineto{\pgfqpoint{1.225081in}{0.739656in}}%
\pgfpathlineto{\pgfqpoint{1.224785in}{0.739656in}}%
\pgfpathlineto{\pgfqpoint{1.224489in}{0.739656in}}%
\pgfpathlineto{\pgfqpoint{1.224193in}{0.739656in}}%
\pgfpathlineto{\pgfqpoint{1.223897in}{0.739656in}}%
\pgfpathlineto{\pgfqpoint{1.223601in}{0.739656in}}%
\pgfpathlineto{\pgfqpoint{1.223305in}{0.739656in}}%
\pgfpathlineto{\pgfqpoint{1.223009in}{0.739656in}}%
\pgfpathlineto{\pgfqpoint{1.222713in}{0.739656in}}%
\pgfpathlineto{\pgfqpoint{1.222417in}{0.739656in}}%
\pgfpathlineto{\pgfqpoint{1.222121in}{0.739656in}}%
\pgfpathlineto{\pgfqpoint{1.221825in}{0.739656in}}%
\pgfpathlineto{\pgfqpoint{1.221529in}{0.739656in}}%
\pgfpathlineto{\pgfqpoint{1.221233in}{0.739656in}}%
\pgfpathlineto{\pgfqpoint{1.220937in}{0.739656in}}%
\pgfpathlineto{\pgfqpoint{1.220641in}{0.739656in}}%
\pgfpathlineto{\pgfqpoint{1.220345in}{0.739656in}}%
\pgfpathlineto{\pgfqpoint{1.220049in}{0.739656in}}%
\pgfpathlineto{\pgfqpoint{1.219753in}{0.739656in}}%
\pgfpathlineto{\pgfqpoint{1.219457in}{0.739656in}}%
\pgfpathlineto{\pgfqpoint{1.219161in}{0.739656in}}%
\pgfpathlineto{\pgfqpoint{1.218865in}{0.739656in}}%
\pgfpathlineto{\pgfqpoint{1.218569in}{0.739656in}}%
\pgfpathlineto{\pgfqpoint{1.218273in}{0.739656in}}%
\pgfpathlineto{\pgfqpoint{1.217977in}{0.739656in}}%
\pgfpathlineto{\pgfqpoint{1.217681in}{0.739656in}}%
\pgfpathlineto{\pgfqpoint{1.217385in}{0.739656in}}%
\pgfpathlineto{\pgfqpoint{1.217089in}{0.739656in}}%
\pgfpathlineto{\pgfqpoint{1.216793in}{0.739656in}}%
\pgfpathlineto{\pgfqpoint{1.216497in}{0.739656in}}%
\pgfpathlineto{\pgfqpoint{1.216201in}{0.739656in}}%
\pgfpathlineto{\pgfqpoint{1.215905in}{0.739656in}}%
\pgfpathlineto{\pgfqpoint{1.215609in}{0.739656in}}%
\pgfpathlineto{\pgfqpoint{1.215313in}{0.739656in}}%
\pgfpathlineto{\pgfqpoint{1.215017in}{0.739656in}}%
\pgfpathlineto{\pgfqpoint{1.214721in}{0.739656in}}%
\pgfpathlineto{\pgfqpoint{1.214425in}{0.739656in}}%
\pgfpathlineto{\pgfqpoint{1.214129in}{0.739656in}}%
\pgfpathlineto{\pgfqpoint{1.213833in}{0.739656in}}%
\pgfpathlineto{\pgfqpoint{1.213537in}{0.739656in}}%
\pgfpathlineto{\pgfqpoint{1.213241in}{0.739656in}}%
\pgfpathlineto{\pgfqpoint{1.212945in}{0.739656in}}%
\pgfpathlineto{\pgfqpoint{1.212649in}{0.739656in}}%
\pgfpathlineto{\pgfqpoint{1.212353in}{0.739656in}}%
\pgfpathlineto{\pgfqpoint{1.212057in}{0.739656in}}%
\pgfpathlineto{\pgfqpoint{1.211761in}{0.739656in}}%
\pgfpathlineto{\pgfqpoint{1.211465in}{0.739656in}}%
\pgfpathlineto{\pgfqpoint{1.211169in}{0.739656in}}%
\pgfpathlineto{\pgfqpoint{1.210873in}{0.739656in}}%
\pgfpathlineto{\pgfqpoint{1.210577in}{0.739656in}}%
\pgfpathlineto{\pgfqpoint{1.210281in}{0.739656in}}%
\pgfpathlineto{\pgfqpoint{1.209985in}{0.739656in}}%
\pgfpathlineto{\pgfqpoint{1.209689in}{0.739656in}}%
\pgfpathlineto{\pgfqpoint{1.209393in}{0.739656in}}%
\pgfpathlineto{\pgfqpoint{1.209097in}{0.739656in}}%
\pgfpathlineto{\pgfqpoint{1.208801in}{0.739656in}}%
\pgfpathlineto{\pgfqpoint{1.208505in}{0.739656in}}%
\pgfpathlineto{\pgfqpoint{1.208209in}{0.739656in}}%
\pgfpathlineto{\pgfqpoint{1.207913in}{0.739656in}}%
\pgfpathlineto{\pgfqpoint{1.207617in}{0.739656in}}%
\pgfpathlineto{\pgfqpoint{1.207321in}{0.739656in}}%
\pgfpathlineto{\pgfqpoint{1.207025in}{0.739656in}}%
\pgfpathlineto{\pgfqpoint{1.206729in}{0.739656in}}%
\pgfpathlineto{\pgfqpoint{1.206433in}{0.739656in}}%
\pgfpathlineto{\pgfqpoint{1.206137in}{0.739656in}}%
\pgfpathlineto{\pgfqpoint{1.205841in}{0.739656in}}%
\pgfpathlineto{\pgfqpoint{1.205545in}{0.739656in}}%
\pgfpathlineto{\pgfqpoint{1.205249in}{0.739656in}}%
\pgfpathlineto{\pgfqpoint{1.204953in}{0.739656in}}%
\pgfpathlineto{\pgfqpoint{1.204657in}{0.739656in}}%
\pgfpathlineto{\pgfqpoint{1.204361in}{0.739656in}}%
\pgfpathlineto{\pgfqpoint{1.204065in}{0.739656in}}%
\pgfpathlineto{\pgfqpoint{1.203769in}{0.739656in}}%
\pgfpathlineto{\pgfqpoint{1.203473in}{0.739656in}}%
\pgfpathlineto{\pgfqpoint{1.203177in}{0.739656in}}%
\pgfpathlineto{\pgfqpoint{1.202881in}{0.739656in}}%
\pgfpathlineto{\pgfqpoint{1.202585in}{0.739656in}}%
\pgfpathlineto{\pgfqpoint{1.202289in}{0.739656in}}%
\pgfpathlineto{\pgfqpoint{1.201993in}{0.739656in}}%
\pgfpathlineto{\pgfqpoint{1.201697in}{0.739656in}}%
\pgfpathlineto{\pgfqpoint{1.201401in}{0.739656in}}%
\pgfpathlineto{\pgfqpoint{1.201105in}{0.739656in}}%
\pgfpathlineto{\pgfqpoint{1.200809in}{0.739656in}}%
\pgfpathlineto{\pgfqpoint{1.200513in}{0.739656in}}%
\pgfpathlineto{\pgfqpoint{1.200217in}{0.739656in}}%
\pgfpathlineto{\pgfqpoint{1.199921in}{0.739656in}}%
\pgfpathlineto{\pgfqpoint{1.199625in}{0.739656in}}%
\pgfpathlineto{\pgfqpoint{1.199329in}{0.739656in}}%
\pgfpathlineto{\pgfqpoint{1.199033in}{0.739656in}}%
\pgfpathlineto{\pgfqpoint{1.198737in}{0.739656in}}%
\pgfpathlineto{\pgfqpoint{1.198441in}{0.739656in}}%
\pgfpathlineto{\pgfqpoint{1.198145in}{0.739656in}}%
\pgfpathlineto{\pgfqpoint{1.197849in}{0.739656in}}%
\pgfpathlineto{\pgfqpoint{1.197553in}{0.739656in}}%
\pgfpathlineto{\pgfqpoint{1.197257in}{0.739656in}}%
\pgfpathlineto{\pgfqpoint{1.196961in}{0.739656in}}%
\pgfpathlineto{\pgfqpoint{1.196665in}{0.739656in}}%
\pgfpathlineto{\pgfqpoint{1.196369in}{0.739656in}}%
\pgfpathlineto{\pgfqpoint{1.196073in}{0.739656in}}%
\pgfpathlineto{\pgfqpoint{1.195777in}{0.739656in}}%
\pgfpathlineto{\pgfqpoint{1.195481in}{0.739656in}}%
\pgfpathlineto{\pgfqpoint{1.195185in}{0.739656in}}%
\pgfpathlineto{\pgfqpoint{1.194889in}{0.739656in}}%
\pgfpathlineto{\pgfqpoint{1.194593in}{0.739656in}}%
\pgfpathlineto{\pgfqpoint{1.194297in}{0.739656in}}%
\pgfpathlineto{\pgfqpoint{1.194001in}{0.739656in}}%
\pgfpathlineto{\pgfqpoint{1.193705in}{0.739656in}}%
\pgfpathlineto{\pgfqpoint{1.193409in}{0.739656in}}%
\pgfpathlineto{\pgfqpoint{1.193113in}{0.739656in}}%
\pgfpathlineto{\pgfqpoint{1.192817in}{0.739656in}}%
\pgfpathlineto{\pgfqpoint{1.192521in}{0.739656in}}%
\pgfpathlineto{\pgfqpoint{1.192225in}{0.739656in}}%
\pgfpathlineto{\pgfqpoint{1.191929in}{0.739656in}}%
\pgfpathlineto{\pgfqpoint{1.191633in}{0.739656in}}%
\pgfpathlineto{\pgfqpoint{1.191337in}{0.739656in}}%
\pgfpathlineto{\pgfqpoint{1.191041in}{0.739656in}}%
\pgfpathlineto{\pgfqpoint{1.190745in}{0.739656in}}%
\pgfpathlineto{\pgfqpoint{1.190449in}{0.739656in}}%
\pgfpathlineto{\pgfqpoint{1.190153in}{0.739656in}}%
\pgfpathlineto{\pgfqpoint{1.189857in}{0.739656in}}%
\pgfpathlineto{\pgfqpoint{1.189561in}{0.739656in}}%
\pgfpathlineto{\pgfqpoint{1.189265in}{0.739656in}}%
\pgfpathlineto{\pgfqpoint{1.188969in}{0.739656in}}%
\pgfpathlineto{\pgfqpoint{1.188673in}{0.739656in}}%
\pgfpathlineto{\pgfqpoint{1.188377in}{0.739656in}}%
\pgfpathlineto{\pgfqpoint{1.188081in}{0.739656in}}%
\pgfpathlineto{\pgfqpoint{1.187785in}{0.739656in}}%
\pgfpathlineto{\pgfqpoint{1.187489in}{0.739656in}}%
\pgfpathlineto{\pgfqpoint{1.187192in}{0.739656in}}%
\pgfpathlineto{\pgfqpoint{1.186896in}{0.739656in}}%
\pgfpathlineto{\pgfqpoint{1.186600in}{0.739656in}}%
\pgfpathlineto{\pgfqpoint{1.186304in}{0.739656in}}%
\pgfpathlineto{\pgfqpoint{1.186008in}{0.739656in}}%
\pgfpathlineto{\pgfqpoint{1.185712in}{0.739656in}}%
\pgfpathlineto{\pgfqpoint{1.185416in}{0.739656in}}%
\pgfpathlineto{\pgfqpoint{1.185120in}{0.739656in}}%
\pgfpathlineto{\pgfqpoint{1.184824in}{0.739656in}}%
\pgfpathlineto{\pgfqpoint{1.184528in}{0.739656in}}%
\pgfpathlineto{\pgfqpoint{1.184232in}{0.739656in}}%
\pgfpathlineto{\pgfqpoint{1.183936in}{0.739656in}}%
\pgfpathlineto{\pgfqpoint{1.183640in}{0.739656in}}%
\pgfpathlineto{\pgfqpoint{1.183344in}{0.739656in}}%
\pgfpathlineto{\pgfqpoint{1.183048in}{0.739656in}}%
\pgfpathlineto{\pgfqpoint{1.182752in}{0.739656in}}%
\pgfpathlineto{\pgfqpoint{1.182456in}{0.739656in}}%
\pgfpathlineto{\pgfqpoint{1.182160in}{0.739656in}}%
\pgfpathlineto{\pgfqpoint{1.181864in}{0.739656in}}%
\pgfpathlineto{\pgfqpoint{1.181568in}{0.739656in}}%
\pgfpathlineto{\pgfqpoint{1.181272in}{0.739656in}}%
\pgfpathlineto{\pgfqpoint{1.180976in}{0.739656in}}%
\pgfpathlineto{\pgfqpoint{1.180680in}{0.739656in}}%
\pgfpathlineto{\pgfqpoint{1.180384in}{0.739656in}}%
\pgfpathlineto{\pgfqpoint{1.180088in}{0.739656in}}%
\pgfpathlineto{\pgfqpoint{1.179792in}{0.739656in}}%
\pgfpathlineto{\pgfqpoint{1.179496in}{0.739656in}}%
\pgfpathlineto{\pgfqpoint{1.179200in}{0.739656in}}%
\pgfpathlineto{\pgfqpoint{1.178904in}{0.739656in}}%
\pgfpathlineto{\pgfqpoint{1.178608in}{0.739656in}}%
\pgfpathlineto{\pgfqpoint{1.178312in}{0.739656in}}%
\pgfpathlineto{\pgfqpoint{1.178016in}{0.739656in}}%
\pgfpathlineto{\pgfqpoint{1.177720in}{0.739656in}}%
\pgfpathlineto{\pgfqpoint{1.177424in}{0.739656in}}%
\pgfpathlineto{\pgfqpoint{1.177128in}{0.739656in}}%
\pgfpathlineto{\pgfqpoint{1.176832in}{0.739656in}}%
\pgfpathlineto{\pgfqpoint{1.176536in}{0.739656in}}%
\pgfpathlineto{\pgfqpoint{1.176240in}{0.739656in}}%
\pgfpathlineto{\pgfqpoint{1.175944in}{0.739656in}}%
\pgfpathlineto{\pgfqpoint{1.175648in}{0.739656in}}%
\pgfpathlineto{\pgfqpoint{1.175352in}{0.739656in}}%
\pgfpathlineto{\pgfqpoint{1.175056in}{0.739656in}}%
\pgfpathlineto{\pgfqpoint{1.174760in}{0.739656in}}%
\pgfpathlineto{\pgfqpoint{1.174464in}{0.739656in}}%
\pgfpathlineto{\pgfqpoint{1.174168in}{0.739656in}}%
\pgfpathlineto{\pgfqpoint{1.173872in}{0.739656in}}%
\pgfpathlineto{\pgfqpoint{1.173576in}{0.739656in}}%
\pgfpathlineto{\pgfqpoint{1.173280in}{0.739656in}}%
\pgfpathlineto{\pgfqpoint{1.172984in}{0.739656in}}%
\pgfpathlineto{\pgfqpoint{1.172688in}{0.739656in}}%
\pgfpathlineto{\pgfqpoint{1.172392in}{0.739656in}}%
\pgfpathlineto{\pgfqpoint{1.172096in}{0.739656in}}%
\pgfpathlineto{\pgfqpoint{1.171800in}{0.739656in}}%
\pgfpathlineto{\pgfqpoint{1.171504in}{0.739656in}}%
\pgfpathlineto{\pgfqpoint{1.171208in}{0.739656in}}%
\pgfpathlineto{\pgfqpoint{1.170912in}{0.739656in}}%
\pgfpathlineto{\pgfqpoint{1.170616in}{0.739656in}}%
\pgfpathlineto{\pgfqpoint{1.170320in}{0.739656in}}%
\pgfpathlineto{\pgfqpoint{1.170024in}{0.739656in}}%
\pgfpathlineto{\pgfqpoint{1.169728in}{0.739656in}}%
\pgfpathlineto{\pgfqpoint{1.169432in}{0.739656in}}%
\pgfpathlineto{\pgfqpoint{1.169136in}{0.739656in}}%
\pgfpathlineto{\pgfqpoint{1.168840in}{0.739656in}}%
\pgfpathlineto{\pgfqpoint{1.168544in}{0.739656in}}%
\pgfpathlineto{\pgfqpoint{1.168248in}{0.739656in}}%
\pgfpathlineto{\pgfqpoint{1.167952in}{0.739656in}}%
\pgfpathlineto{\pgfqpoint{1.167656in}{0.739656in}}%
\pgfpathlineto{\pgfqpoint{1.167360in}{0.739656in}}%
\pgfpathlineto{\pgfqpoint{1.167064in}{0.739656in}}%
\pgfpathlineto{\pgfqpoint{1.166768in}{0.739656in}}%
\pgfpathlineto{\pgfqpoint{1.166472in}{0.739656in}}%
\pgfpathlineto{\pgfqpoint{1.166176in}{0.739656in}}%
\pgfpathlineto{\pgfqpoint{1.165880in}{0.739656in}}%
\pgfpathlineto{\pgfqpoint{1.165584in}{0.739656in}}%
\pgfpathlineto{\pgfqpoint{1.165288in}{0.739656in}}%
\pgfpathlineto{\pgfqpoint{1.164992in}{0.739656in}}%
\pgfpathlineto{\pgfqpoint{1.164696in}{0.739656in}}%
\pgfpathlineto{\pgfqpoint{1.164400in}{0.739656in}}%
\pgfpathlineto{\pgfqpoint{1.164104in}{0.739656in}}%
\pgfpathlineto{\pgfqpoint{1.163808in}{0.739656in}}%
\pgfpathlineto{\pgfqpoint{1.163512in}{0.739656in}}%
\pgfpathlineto{\pgfqpoint{1.163216in}{0.739656in}}%
\pgfpathlineto{\pgfqpoint{1.162920in}{0.739656in}}%
\pgfpathlineto{\pgfqpoint{1.162624in}{0.739656in}}%
\pgfpathlineto{\pgfqpoint{1.162328in}{0.739656in}}%
\pgfpathlineto{\pgfqpoint{1.162032in}{0.739656in}}%
\pgfpathlineto{\pgfqpoint{1.161736in}{0.739656in}}%
\pgfpathlineto{\pgfqpoint{1.161440in}{0.739656in}}%
\pgfpathlineto{\pgfqpoint{1.161144in}{0.739656in}}%
\pgfpathlineto{\pgfqpoint{1.160848in}{0.739656in}}%
\pgfpathlineto{\pgfqpoint{1.160552in}{0.739656in}}%
\pgfpathlineto{\pgfqpoint{1.160256in}{0.739656in}}%
\pgfpathlineto{\pgfqpoint{1.159960in}{0.739656in}}%
\pgfpathlineto{\pgfqpoint{1.159664in}{0.739656in}}%
\pgfpathlineto{\pgfqpoint{1.159368in}{0.739656in}}%
\pgfpathlineto{\pgfqpoint{1.159072in}{0.739656in}}%
\pgfpathlineto{\pgfqpoint{1.158776in}{0.739656in}}%
\pgfpathlineto{\pgfqpoint{1.158480in}{0.739656in}}%
\pgfpathlineto{\pgfqpoint{1.158184in}{0.739656in}}%
\pgfpathlineto{\pgfqpoint{1.157888in}{0.739656in}}%
\pgfpathlineto{\pgfqpoint{1.157592in}{0.739656in}}%
\pgfpathlineto{\pgfqpoint{1.157296in}{0.739656in}}%
\pgfpathlineto{\pgfqpoint{1.157000in}{0.739656in}}%
\pgfpathlineto{\pgfqpoint{1.156704in}{0.739656in}}%
\pgfpathlineto{\pgfqpoint{1.156408in}{0.739656in}}%
\pgfpathlineto{\pgfqpoint{1.156112in}{0.739656in}}%
\pgfpathlineto{\pgfqpoint{1.155816in}{0.739656in}}%
\pgfpathlineto{\pgfqpoint{1.155520in}{0.739656in}}%
\pgfpathlineto{\pgfqpoint{1.155224in}{0.739656in}}%
\pgfpathlineto{\pgfqpoint{1.154928in}{0.739656in}}%
\pgfpathlineto{\pgfqpoint{1.154632in}{0.739656in}}%
\pgfpathlineto{\pgfqpoint{1.154336in}{0.739656in}}%
\pgfpathlineto{\pgfqpoint{1.154040in}{0.739656in}}%
\pgfpathlineto{\pgfqpoint{1.153744in}{0.739656in}}%
\pgfpathlineto{\pgfqpoint{1.153448in}{0.739656in}}%
\pgfpathlineto{\pgfqpoint{1.153152in}{0.739656in}}%
\pgfpathlineto{\pgfqpoint{1.152856in}{0.739656in}}%
\pgfpathlineto{\pgfqpoint{1.152560in}{0.739656in}}%
\pgfpathlineto{\pgfqpoint{1.152264in}{0.739656in}}%
\pgfpathlineto{\pgfqpoint{1.151968in}{0.739656in}}%
\pgfpathlineto{\pgfqpoint{1.151672in}{0.739656in}}%
\pgfpathlineto{\pgfqpoint{1.151376in}{0.739656in}}%
\pgfpathlineto{\pgfqpoint{1.151080in}{0.739656in}}%
\pgfpathlineto{\pgfqpoint{1.150784in}{0.739656in}}%
\pgfpathlineto{\pgfqpoint{1.150488in}{0.739656in}}%
\pgfpathlineto{\pgfqpoint{1.150192in}{0.739656in}}%
\pgfpathlineto{\pgfqpoint{1.149896in}{0.739656in}}%
\pgfpathlineto{\pgfqpoint{1.149600in}{0.739656in}}%
\pgfpathlineto{\pgfqpoint{1.149304in}{0.739656in}}%
\pgfpathlineto{\pgfqpoint{1.149008in}{0.739656in}}%
\pgfpathlineto{\pgfqpoint{1.148712in}{0.739656in}}%
\pgfpathlineto{\pgfqpoint{1.148416in}{0.739656in}}%
\pgfpathlineto{\pgfqpoint{1.148120in}{0.739656in}}%
\pgfpathlineto{\pgfqpoint{1.147824in}{0.739656in}}%
\pgfpathlineto{\pgfqpoint{1.147528in}{0.739656in}}%
\pgfpathlineto{\pgfqpoint{1.147232in}{0.739656in}}%
\pgfpathlineto{\pgfqpoint{1.146936in}{0.739656in}}%
\pgfpathlineto{\pgfqpoint{1.146640in}{0.739656in}}%
\pgfpathlineto{\pgfqpoint{1.146344in}{0.739656in}}%
\pgfpathlineto{\pgfqpoint{1.146048in}{0.739656in}}%
\pgfpathlineto{\pgfqpoint{1.145752in}{0.739656in}}%
\pgfpathlineto{\pgfqpoint{1.145456in}{0.739656in}}%
\pgfpathlineto{\pgfqpoint{1.145160in}{0.739656in}}%
\pgfpathlineto{\pgfqpoint{1.144864in}{0.739656in}}%
\pgfpathlineto{\pgfqpoint{1.144568in}{0.739656in}}%
\pgfpathlineto{\pgfqpoint{1.144272in}{0.739656in}}%
\pgfpathlineto{\pgfqpoint{1.143976in}{0.739656in}}%
\pgfpathlineto{\pgfqpoint{1.143680in}{0.739656in}}%
\pgfpathlineto{\pgfqpoint{1.143384in}{0.739656in}}%
\pgfpathlineto{\pgfqpoint{1.143088in}{0.739656in}}%
\pgfpathlineto{\pgfqpoint{1.142792in}{0.739656in}}%
\pgfpathlineto{\pgfqpoint{1.142496in}{0.739656in}}%
\pgfpathlineto{\pgfqpoint{1.142200in}{0.739656in}}%
\pgfpathlineto{\pgfqpoint{1.141904in}{0.739656in}}%
\pgfpathlineto{\pgfqpoint{1.141608in}{0.739656in}}%
\pgfpathlineto{\pgfqpoint{1.141312in}{0.739656in}}%
\pgfpathlineto{\pgfqpoint{1.141016in}{0.739656in}}%
\pgfpathlineto{\pgfqpoint{1.140720in}{0.739656in}}%
\pgfpathlineto{\pgfqpoint{1.140424in}{0.739656in}}%
\pgfpathlineto{\pgfqpoint{1.140128in}{0.739656in}}%
\pgfpathlineto{\pgfqpoint{1.139832in}{0.739656in}}%
\pgfpathlineto{\pgfqpoint{1.139536in}{0.739656in}}%
\pgfpathlineto{\pgfqpoint{1.139240in}{0.739656in}}%
\pgfpathlineto{\pgfqpoint{1.138944in}{0.739656in}}%
\pgfpathlineto{\pgfqpoint{1.138648in}{0.739656in}}%
\pgfpathlineto{\pgfqpoint{1.138352in}{0.739656in}}%
\pgfpathlineto{\pgfqpoint{1.138056in}{0.739656in}}%
\pgfpathlineto{\pgfqpoint{1.137760in}{0.739656in}}%
\pgfpathlineto{\pgfqpoint{1.137464in}{0.739656in}}%
\pgfpathlineto{\pgfqpoint{1.137168in}{0.739656in}}%
\pgfpathlineto{\pgfqpoint{1.136872in}{0.739656in}}%
\pgfpathlineto{\pgfqpoint{1.136576in}{0.739656in}}%
\pgfpathlineto{\pgfqpoint{1.136280in}{0.739656in}}%
\pgfpathlineto{\pgfqpoint{1.135984in}{0.739656in}}%
\pgfpathlineto{\pgfqpoint{1.135688in}{0.739656in}}%
\pgfpathlineto{\pgfqpoint{1.135392in}{0.739656in}}%
\pgfpathlineto{\pgfqpoint{1.135096in}{0.739656in}}%
\pgfpathlineto{\pgfqpoint{1.134800in}{0.739656in}}%
\pgfpathlineto{\pgfqpoint{1.134504in}{0.739656in}}%
\pgfpathlineto{\pgfqpoint{1.134208in}{0.739656in}}%
\pgfpathlineto{\pgfqpoint{1.133912in}{0.739656in}}%
\pgfpathlineto{\pgfqpoint{1.133616in}{0.739656in}}%
\pgfpathlineto{\pgfqpoint{1.133320in}{0.739656in}}%
\pgfpathlineto{\pgfqpoint{1.133024in}{0.739656in}}%
\pgfpathlineto{\pgfqpoint{1.132728in}{0.739656in}}%
\pgfpathlineto{\pgfqpoint{1.132432in}{0.739656in}}%
\pgfpathlineto{\pgfqpoint{1.132136in}{0.739656in}}%
\pgfpathlineto{\pgfqpoint{1.131840in}{0.739656in}}%
\pgfpathlineto{\pgfqpoint{1.131544in}{0.739656in}}%
\pgfpathlineto{\pgfqpoint{1.131248in}{0.739656in}}%
\pgfpathlineto{\pgfqpoint{1.130952in}{0.739656in}}%
\pgfpathlineto{\pgfqpoint{1.130656in}{0.739656in}}%
\pgfpathlineto{\pgfqpoint{1.130360in}{0.739656in}}%
\pgfpathlineto{\pgfqpoint{1.130064in}{0.739656in}}%
\pgfpathlineto{\pgfqpoint{1.129768in}{0.739656in}}%
\pgfpathlineto{\pgfqpoint{1.129472in}{0.739656in}}%
\pgfpathlineto{\pgfqpoint{1.129176in}{0.739656in}}%
\pgfpathlineto{\pgfqpoint{1.128880in}{0.739656in}}%
\pgfpathlineto{\pgfqpoint{1.128584in}{0.739656in}}%
\pgfpathlineto{\pgfqpoint{1.128288in}{0.739656in}}%
\pgfpathlineto{\pgfqpoint{1.127992in}{0.739656in}}%
\pgfpathlineto{\pgfqpoint{1.127696in}{0.739656in}}%
\pgfpathlineto{\pgfqpoint{1.127400in}{0.739656in}}%
\pgfpathlineto{\pgfqpoint{1.127104in}{0.739656in}}%
\pgfpathlineto{\pgfqpoint{1.126808in}{0.739656in}}%
\pgfpathlineto{\pgfqpoint{1.126512in}{0.739656in}}%
\pgfpathlineto{\pgfqpoint{1.126216in}{0.739656in}}%
\pgfpathlineto{\pgfqpoint{1.125920in}{0.739656in}}%
\pgfpathlineto{\pgfqpoint{1.125624in}{0.739656in}}%
\pgfpathlineto{\pgfqpoint{1.125328in}{0.739656in}}%
\pgfpathlineto{\pgfqpoint{1.125032in}{0.739656in}}%
\pgfpathlineto{\pgfqpoint{1.124736in}{0.739656in}}%
\pgfpathlineto{\pgfqpoint{1.124440in}{0.739656in}}%
\pgfpathlineto{\pgfqpoint{1.124144in}{0.739656in}}%
\pgfpathlineto{\pgfqpoint{1.123848in}{0.739656in}}%
\pgfpathlineto{\pgfqpoint{1.123552in}{0.739656in}}%
\pgfpathlineto{\pgfqpoint{1.123256in}{0.739656in}}%
\pgfpathlineto{\pgfqpoint{1.122960in}{0.739656in}}%
\pgfpathlineto{\pgfqpoint{1.122664in}{0.739656in}}%
\pgfpathlineto{\pgfqpoint{1.122368in}{0.739656in}}%
\pgfpathlineto{\pgfqpoint{1.122072in}{0.739656in}}%
\pgfpathlineto{\pgfqpoint{1.121776in}{0.739656in}}%
\pgfpathlineto{\pgfqpoint{1.121480in}{0.739656in}}%
\pgfpathlineto{\pgfqpoint{1.121184in}{0.739656in}}%
\pgfpathlineto{\pgfqpoint{1.120888in}{0.739656in}}%
\pgfpathlineto{\pgfqpoint{1.120592in}{0.739656in}}%
\pgfpathlineto{\pgfqpoint{1.120296in}{0.739656in}}%
\pgfpathlineto{\pgfqpoint{1.120000in}{0.739656in}}%
\pgfpathlineto{\pgfqpoint{1.119703in}{0.739656in}}%
\pgfpathlineto{\pgfqpoint{1.119407in}{0.739656in}}%
\pgfpathlineto{\pgfqpoint{1.119111in}{0.739656in}}%
\pgfpathlineto{\pgfqpoint{1.118815in}{0.739656in}}%
\pgfpathlineto{\pgfqpoint{1.118519in}{0.739656in}}%
\pgfpathlineto{\pgfqpoint{1.118223in}{0.739656in}}%
\pgfpathlineto{\pgfqpoint{1.117927in}{0.739656in}}%
\pgfpathlineto{\pgfqpoint{1.117631in}{0.739656in}}%
\pgfpathlineto{\pgfqpoint{1.117335in}{0.739656in}}%
\pgfpathlineto{\pgfqpoint{1.117039in}{0.739656in}}%
\pgfpathlineto{\pgfqpoint{1.116743in}{0.739656in}}%
\pgfpathlineto{\pgfqpoint{1.116447in}{0.739656in}}%
\pgfpathlineto{\pgfqpoint{1.116151in}{0.739656in}}%
\pgfpathlineto{\pgfqpoint{1.115855in}{0.739656in}}%
\pgfpathlineto{\pgfqpoint{1.115559in}{0.739656in}}%
\pgfpathlineto{\pgfqpoint{1.115263in}{0.739656in}}%
\pgfpathlineto{\pgfqpoint{1.114967in}{0.739656in}}%
\pgfpathlineto{\pgfqpoint{1.114671in}{0.739656in}}%
\pgfpathlineto{\pgfqpoint{1.114375in}{0.739656in}}%
\pgfpathlineto{\pgfqpoint{1.114079in}{0.739656in}}%
\pgfpathlineto{\pgfqpoint{1.113783in}{0.739656in}}%
\pgfpathlineto{\pgfqpoint{1.113487in}{0.739656in}}%
\pgfpathlineto{\pgfqpoint{1.113191in}{0.739656in}}%
\pgfpathlineto{\pgfqpoint{1.112895in}{0.739656in}}%
\pgfpathlineto{\pgfqpoint{1.112599in}{0.739656in}}%
\pgfpathlineto{\pgfqpoint{1.112303in}{0.739656in}}%
\pgfpathlineto{\pgfqpoint{1.112007in}{0.739656in}}%
\pgfpathlineto{\pgfqpoint{1.111711in}{0.739656in}}%
\pgfpathlineto{\pgfqpoint{1.111415in}{0.739656in}}%
\pgfpathlineto{\pgfqpoint{1.111119in}{0.739656in}}%
\pgfpathlineto{\pgfqpoint{1.110823in}{0.739656in}}%
\pgfpathlineto{\pgfqpoint{1.110527in}{0.739656in}}%
\pgfpathlineto{\pgfqpoint{1.110231in}{0.739656in}}%
\pgfpathlineto{\pgfqpoint{1.109935in}{0.739656in}}%
\pgfpathlineto{\pgfqpoint{1.109639in}{0.739656in}}%
\pgfpathlineto{\pgfqpoint{1.109343in}{0.739656in}}%
\pgfpathlineto{\pgfqpoint{1.109047in}{0.739656in}}%
\pgfpathlineto{\pgfqpoint{1.108751in}{0.739656in}}%
\pgfpathlineto{\pgfqpoint{1.108455in}{0.739656in}}%
\pgfpathlineto{\pgfqpoint{1.108159in}{0.739656in}}%
\pgfpathlineto{\pgfqpoint{1.107863in}{0.739656in}}%
\pgfpathlineto{\pgfqpoint{1.107567in}{0.739656in}}%
\pgfpathlineto{\pgfqpoint{1.107271in}{0.739656in}}%
\pgfpathlineto{\pgfqpoint{1.106975in}{0.739656in}}%
\pgfpathlineto{\pgfqpoint{1.106679in}{0.739656in}}%
\pgfpathlineto{\pgfqpoint{1.106383in}{0.739656in}}%
\pgfpathlineto{\pgfqpoint{1.106087in}{0.739656in}}%
\pgfpathlineto{\pgfqpoint{1.105791in}{0.739656in}}%
\pgfpathlineto{\pgfqpoint{1.105495in}{0.739656in}}%
\pgfpathlineto{\pgfqpoint{1.105199in}{0.739656in}}%
\pgfpathlineto{\pgfqpoint{1.104903in}{0.739656in}}%
\pgfpathlineto{\pgfqpoint{1.104607in}{0.739656in}}%
\pgfpathlineto{\pgfqpoint{1.104311in}{0.739656in}}%
\pgfpathlineto{\pgfqpoint{1.104015in}{0.739656in}}%
\pgfpathlineto{\pgfqpoint{1.103719in}{0.739656in}}%
\pgfpathlineto{\pgfqpoint{1.103423in}{0.739656in}}%
\pgfpathlineto{\pgfqpoint{1.103127in}{0.739656in}}%
\pgfpathlineto{\pgfqpoint{1.102831in}{0.739656in}}%
\pgfpathlineto{\pgfqpoint{1.102535in}{0.739656in}}%
\pgfpathlineto{\pgfqpoint{1.102239in}{0.739656in}}%
\pgfpathlineto{\pgfqpoint{1.101943in}{0.739656in}}%
\pgfpathlineto{\pgfqpoint{1.101647in}{0.739656in}}%
\pgfpathlineto{\pgfqpoint{1.101351in}{0.739656in}}%
\pgfpathlineto{\pgfqpoint{1.101055in}{0.739656in}}%
\pgfpathlineto{\pgfqpoint{1.100759in}{0.739656in}}%
\pgfpathlineto{\pgfqpoint{1.100463in}{0.739656in}}%
\pgfpathlineto{\pgfqpoint{1.100167in}{0.739656in}}%
\pgfpathlineto{\pgfqpoint{1.099871in}{0.739656in}}%
\pgfpathlineto{\pgfqpoint{1.099575in}{0.739656in}}%
\pgfpathlineto{\pgfqpoint{1.099279in}{0.739656in}}%
\pgfpathlineto{\pgfqpoint{1.098983in}{0.739656in}}%
\pgfpathlineto{\pgfqpoint{1.098687in}{0.739656in}}%
\pgfpathlineto{\pgfqpoint{1.098391in}{0.739656in}}%
\pgfpathlineto{\pgfqpoint{1.098095in}{0.739656in}}%
\pgfpathlineto{\pgfqpoint{1.097799in}{0.739656in}}%
\pgfpathlineto{\pgfqpoint{1.097503in}{0.739656in}}%
\pgfpathlineto{\pgfqpoint{1.097207in}{0.739656in}}%
\pgfpathlineto{\pgfqpoint{1.096911in}{0.739656in}}%
\pgfpathlineto{\pgfqpoint{1.096615in}{0.739656in}}%
\pgfpathlineto{\pgfqpoint{1.096319in}{0.739656in}}%
\pgfpathlineto{\pgfqpoint{1.096023in}{0.739656in}}%
\pgfpathlineto{\pgfqpoint{1.095727in}{0.739656in}}%
\pgfpathlineto{\pgfqpoint{1.095431in}{0.739656in}}%
\pgfpathlineto{\pgfqpoint{1.095135in}{0.739656in}}%
\pgfpathlineto{\pgfqpoint{1.094839in}{0.739656in}}%
\pgfpathlineto{\pgfqpoint{1.094543in}{0.739656in}}%
\pgfpathlineto{\pgfqpoint{1.094247in}{0.739656in}}%
\pgfpathlineto{\pgfqpoint{1.093951in}{0.739656in}}%
\pgfpathlineto{\pgfqpoint{1.093655in}{0.739656in}}%
\pgfpathlineto{\pgfqpoint{1.093359in}{0.739656in}}%
\pgfpathlineto{\pgfqpoint{1.093063in}{0.739656in}}%
\pgfpathlineto{\pgfqpoint{1.092767in}{0.739656in}}%
\pgfpathlineto{\pgfqpoint{1.092471in}{0.739656in}}%
\pgfpathlineto{\pgfqpoint{1.092175in}{0.739656in}}%
\pgfpathlineto{\pgfqpoint{1.091879in}{0.739656in}}%
\pgfpathlineto{\pgfqpoint{1.091583in}{0.739656in}}%
\pgfpathlineto{\pgfqpoint{1.091287in}{0.739656in}}%
\pgfpathlineto{\pgfqpoint{1.090991in}{0.739656in}}%
\pgfpathlineto{\pgfqpoint{1.090695in}{0.739656in}}%
\pgfpathlineto{\pgfqpoint{1.090399in}{0.739656in}}%
\pgfpathlineto{\pgfqpoint{1.090103in}{0.739656in}}%
\pgfpathlineto{\pgfqpoint{1.089807in}{0.739656in}}%
\pgfpathlineto{\pgfqpoint{1.089511in}{0.739656in}}%
\pgfpathlineto{\pgfqpoint{1.089215in}{0.739656in}}%
\pgfpathlineto{\pgfqpoint{1.088919in}{0.739656in}}%
\pgfpathlineto{\pgfqpoint{1.088623in}{0.739656in}}%
\pgfpathlineto{\pgfqpoint{1.088327in}{0.739656in}}%
\pgfpathlineto{\pgfqpoint{1.088031in}{0.739656in}}%
\pgfpathlineto{\pgfqpoint{1.087735in}{0.739656in}}%
\pgfpathlineto{\pgfqpoint{1.087439in}{0.739656in}}%
\pgfpathlineto{\pgfqpoint{1.087143in}{0.739656in}}%
\pgfpathlineto{\pgfqpoint{1.086847in}{0.739656in}}%
\pgfpathlineto{\pgfqpoint{1.086551in}{0.739656in}}%
\pgfpathlineto{\pgfqpoint{1.086255in}{0.739656in}}%
\pgfpathlineto{\pgfqpoint{1.085959in}{0.739656in}}%
\pgfpathlineto{\pgfqpoint{1.085663in}{0.739656in}}%
\pgfpathlineto{\pgfqpoint{1.085367in}{0.739656in}}%
\pgfpathlineto{\pgfqpoint{1.085071in}{0.739656in}}%
\pgfpathlineto{\pgfqpoint{1.084775in}{0.739656in}}%
\pgfpathlineto{\pgfqpoint{1.084479in}{0.739656in}}%
\pgfpathlineto{\pgfqpoint{1.084183in}{0.739656in}}%
\pgfpathlineto{\pgfqpoint{1.083887in}{0.739656in}}%
\pgfpathlineto{\pgfqpoint{1.083591in}{0.739656in}}%
\pgfpathlineto{\pgfqpoint{1.083295in}{0.739656in}}%
\pgfpathlineto{\pgfqpoint{1.082999in}{0.739656in}}%
\pgfpathlineto{\pgfqpoint{1.082703in}{0.739656in}}%
\pgfpathlineto{\pgfqpoint{1.082407in}{0.739656in}}%
\pgfpathlineto{\pgfqpoint{1.082111in}{0.739656in}}%
\pgfpathlineto{\pgfqpoint{1.081815in}{0.739656in}}%
\pgfpathlineto{\pgfqpoint{1.081519in}{0.739656in}}%
\pgfpathlineto{\pgfqpoint{1.081223in}{0.739656in}}%
\pgfpathlineto{\pgfqpoint{1.080927in}{0.739656in}}%
\pgfpathlineto{\pgfqpoint{1.080631in}{0.739656in}}%
\pgfpathlineto{\pgfqpoint{1.080335in}{0.739656in}}%
\pgfpathlineto{\pgfqpoint{1.080039in}{0.739656in}}%
\pgfpathlineto{\pgfqpoint{1.079743in}{0.739656in}}%
\pgfpathlineto{\pgfqpoint{1.079447in}{0.739656in}}%
\pgfpathlineto{\pgfqpoint{1.079151in}{0.739656in}}%
\pgfpathlineto{\pgfqpoint{1.078855in}{0.739656in}}%
\pgfpathlineto{\pgfqpoint{1.078559in}{0.739656in}}%
\pgfpathlineto{\pgfqpoint{1.078263in}{0.739656in}}%
\pgfpathlineto{\pgfqpoint{1.077967in}{0.739656in}}%
\pgfpathlineto{\pgfqpoint{1.077671in}{0.739656in}}%
\pgfpathlineto{\pgfqpoint{1.077375in}{0.739656in}}%
\pgfpathlineto{\pgfqpoint{1.077079in}{0.739656in}}%
\pgfpathlineto{\pgfqpoint{1.076783in}{0.739656in}}%
\pgfpathlineto{\pgfqpoint{1.076487in}{0.739656in}}%
\pgfpathlineto{\pgfqpoint{1.076191in}{0.739656in}}%
\pgfpathlineto{\pgfqpoint{1.075895in}{0.739656in}}%
\pgfpathlineto{\pgfqpoint{1.075599in}{0.739656in}}%
\pgfpathlineto{\pgfqpoint{1.075303in}{0.739656in}}%
\pgfpathlineto{\pgfqpoint{1.075007in}{0.739656in}}%
\pgfpathlineto{\pgfqpoint{1.074711in}{0.739656in}}%
\pgfpathlineto{\pgfqpoint{1.074415in}{0.739656in}}%
\pgfpathlineto{\pgfqpoint{1.074119in}{0.739656in}}%
\pgfpathlineto{\pgfqpoint{1.073823in}{0.739656in}}%
\pgfpathlineto{\pgfqpoint{1.073527in}{0.739656in}}%
\pgfpathlineto{\pgfqpoint{1.073231in}{0.739656in}}%
\pgfpathlineto{\pgfqpoint{1.072935in}{0.739656in}}%
\pgfpathlineto{\pgfqpoint{1.072639in}{0.739656in}}%
\pgfpathlineto{\pgfqpoint{1.072343in}{0.739656in}}%
\pgfpathlineto{\pgfqpoint{1.072047in}{0.739656in}}%
\pgfpathlineto{\pgfqpoint{1.071751in}{0.739656in}}%
\pgfpathlineto{\pgfqpoint{1.071455in}{0.739656in}}%
\pgfpathlineto{\pgfqpoint{1.071159in}{0.739656in}}%
\pgfpathlineto{\pgfqpoint{1.070863in}{0.739656in}}%
\pgfpathlineto{\pgfqpoint{1.070567in}{0.739656in}}%
\pgfpathlineto{\pgfqpoint{1.070271in}{0.739656in}}%
\pgfpathlineto{\pgfqpoint{1.069975in}{0.739656in}}%
\pgfpathlineto{\pgfqpoint{1.069679in}{0.739656in}}%
\pgfpathlineto{\pgfqpoint{1.069383in}{0.739656in}}%
\pgfpathlineto{\pgfqpoint{1.069087in}{0.739656in}}%
\pgfpathlineto{\pgfqpoint{1.068791in}{0.739656in}}%
\pgfpathlineto{\pgfqpoint{1.068495in}{0.739656in}}%
\pgfpathlineto{\pgfqpoint{1.068199in}{0.739656in}}%
\pgfpathlineto{\pgfqpoint{1.067903in}{0.739656in}}%
\pgfpathlineto{\pgfqpoint{1.067607in}{0.739656in}}%
\pgfpathlineto{\pgfqpoint{1.067311in}{0.739656in}}%
\pgfpathlineto{\pgfqpoint{1.067015in}{0.739656in}}%
\pgfpathlineto{\pgfqpoint{1.066719in}{0.739656in}}%
\pgfpathlineto{\pgfqpoint{1.066423in}{0.739656in}}%
\pgfpathlineto{\pgfqpoint{1.066127in}{0.739656in}}%
\pgfpathlineto{\pgfqpoint{1.065831in}{0.739656in}}%
\pgfpathlineto{\pgfqpoint{1.065535in}{0.739656in}}%
\pgfpathlineto{\pgfqpoint{1.065239in}{0.739656in}}%
\pgfpathlineto{\pgfqpoint{1.064943in}{0.739656in}}%
\pgfpathlineto{\pgfqpoint{1.064647in}{0.739656in}}%
\pgfpathlineto{\pgfqpoint{1.064351in}{0.739656in}}%
\pgfpathlineto{\pgfqpoint{1.064055in}{0.739656in}}%
\pgfpathlineto{\pgfqpoint{1.063759in}{0.739656in}}%
\pgfpathlineto{\pgfqpoint{1.063463in}{0.739656in}}%
\pgfpathlineto{\pgfqpoint{1.063167in}{0.739656in}}%
\pgfpathlineto{\pgfqpoint{1.062871in}{0.739656in}}%
\pgfpathlineto{\pgfqpoint{1.062575in}{0.739656in}}%
\pgfpathlineto{\pgfqpoint{1.062279in}{0.739656in}}%
\pgfpathlineto{\pgfqpoint{1.061983in}{0.739656in}}%
\pgfpathlineto{\pgfqpoint{1.061687in}{0.739656in}}%
\pgfpathlineto{\pgfqpoint{1.061391in}{0.739656in}}%
\pgfpathlineto{\pgfqpoint{1.061095in}{0.739656in}}%
\pgfpathlineto{\pgfqpoint{1.060799in}{0.739656in}}%
\pgfpathlineto{\pgfqpoint{1.060503in}{0.739656in}}%
\pgfpathlineto{\pgfqpoint{1.060207in}{0.739656in}}%
\pgfpathlineto{\pgfqpoint{1.059911in}{0.739656in}}%
\pgfpathlineto{\pgfqpoint{1.059615in}{0.739656in}}%
\pgfpathlineto{\pgfqpoint{1.059319in}{0.739656in}}%
\pgfpathlineto{\pgfqpoint{1.059023in}{0.739656in}}%
\pgfpathlineto{\pgfqpoint{1.058727in}{0.739656in}}%
\pgfpathlineto{\pgfqpoint{1.058431in}{0.739656in}}%
\pgfpathlineto{\pgfqpoint{1.058135in}{0.739656in}}%
\pgfpathlineto{\pgfqpoint{1.057839in}{0.739656in}}%
\pgfpathlineto{\pgfqpoint{1.057543in}{0.739656in}}%
\pgfpathlineto{\pgfqpoint{1.057247in}{0.739656in}}%
\pgfpathlineto{\pgfqpoint{1.056951in}{0.739656in}}%
\pgfpathlineto{\pgfqpoint{1.056655in}{0.739656in}}%
\pgfpathlineto{\pgfqpoint{1.056359in}{0.739656in}}%
\pgfpathlineto{\pgfqpoint{1.056063in}{0.739656in}}%
\pgfpathlineto{\pgfqpoint{1.055767in}{0.739656in}}%
\pgfpathlineto{\pgfqpoint{1.055471in}{0.739656in}}%
\pgfpathlineto{\pgfqpoint{1.055175in}{0.739656in}}%
\pgfpathlineto{\pgfqpoint{1.054879in}{0.739656in}}%
\pgfpathlineto{\pgfqpoint{1.054583in}{0.739656in}}%
\pgfpathlineto{\pgfqpoint{1.054287in}{0.739656in}}%
\pgfpathlineto{\pgfqpoint{1.053991in}{0.739656in}}%
\pgfpathlineto{\pgfqpoint{1.053695in}{0.739656in}}%
\pgfpathlineto{\pgfqpoint{1.053399in}{0.739656in}}%
\pgfpathlineto{\pgfqpoint{1.053103in}{0.739656in}}%
\pgfpathlineto{\pgfqpoint{1.052807in}{0.739656in}}%
\pgfpathlineto{\pgfqpoint{1.052511in}{0.739656in}}%
\pgfpathlineto{\pgfqpoint{1.052214in}{0.739656in}}%
\pgfpathlineto{\pgfqpoint{1.051918in}{0.739656in}}%
\pgfpathlineto{\pgfqpoint{1.051622in}{0.739656in}}%
\pgfpathlineto{\pgfqpoint{1.051326in}{0.739656in}}%
\pgfpathlineto{\pgfqpoint{1.051030in}{0.739656in}}%
\pgfpathlineto{\pgfqpoint{1.050734in}{0.739656in}}%
\pgfpathlineto{\pgfqpoint{1.050438in}{0.739656in}}%
\pgfpathlineto{\pgfqpoint{1.050142in}{0.739656in}}%
\pgfpathlineto{\pgfqpoint{1.049846in}{0.739656in}}%
\pgfpathlineto{\pgfqpoint{1.049550in}{0.739656in}}%
\pgfpathlineto{\pgfqpoint{1.049254in}{0.739656in}}%
\pgfpathlineto{\pgfqpoint{1.048958in}{0.739656in}}%
\pgfpathlineto{\pgfqpoint{1.048662in}{0.739656in}}%
\pgfpathlineto{\pgfqpoint{1.048366in}{0.739656in}}%
\pgfpathlineto{\pgfqpoint{1.048070in}{0.739656in}}%
\pgfpathlineto{\pgfqpoint{1.047774in}{0.739656in}}%
\pgfpathlineto{\pgfqpoint{1.047478in}{0.739656in}}%
\pgfpathlineto{\pgfqpoint{1.047182in}{0.739656in}}%
\pgfpathlineto{\pgfqpoint{1.046886in}{0.739656in}}%
\pgfpathlineto{\pgfqpoint{1.046590in}{0.739656in}}%
\pgfpathlineto{\pgfqpoint{1.046294in}{0.739656in}}%
\pgfpathlineto{\pgfqpoint{1.045998in}{0.739656in}}%
\pgfpathlineto{\pgfqpoint{1.045702in}{0.739656in}}%
\pgfpathlineto{\pgfqpoint{1.045406in}{0.739656in}}%
\pgfpathlineto{\pgfqpoint{1.045110in}{0.739656in}}%
\pgfpathlineto{\pgfqpoint{1.044814in}{0.739656in}}%
\pgfpathlineto{\pgfqpoint{1.044518in}{0.739656in}}%
\pgfpathlineto{\pgfqpoint{1.044222in}{0.739656in}}%
\pgfpathlineto{\pgfqpoint{1.043926in}{0.739656in}}%
\pgfpathlineto{\pgfqpoint{1.043630in}{0.739656in}}%
\pgfpathlineto{\pgfqpoint{1.043334in}{0.739656in}}%
\pgfpathlineto{\pgfqpoint{1.043038in}{0.739656in}}%
\pgfpathlineto{\pgfqpoint{1.042742in}{0.739656in}}%
\pgfpathlineto{\pgfqpoint{1.042446in}{0.739656in}}%
\pgfpathlineto{\pgfqpoint{1.042150in}{0.739656in}}%
\pgfpathlineto{\pgfqpoint{1.041854in}{0.739656in}}%
\pgfpathlineto{\pgfqpoint{1.041558in}{0.739656in}}%
\pgfpathlineto{\pgfqpoint{1.041262in}{0.739656in}}%
\pgfpathlineto{\pgfqpoint{1.040966in}{0.739656in}}%
\pgfpathlineto{\pgfqpoint{1.040670in}{0.739656in}}%
\pgfpathlineto{\pgfqpoint{1.040374in}{0.739656in}}%
\pgfpathlineto{\pgfqpoint{1.040078in}{0.739656in}}%
\pgfpathlineto{\pgfqpoint{1.039782in}{0.739656in}}%
\pgfpathlineto{\pgfqpoint{1.039486in}{0.739656in}}%
\pgfpathlineto{\pgfqpoint{1.039190in}{0.739656in}}%
\pgfpathlineto{\pgfqpoint{1.038894in}{0.739656in}}%
\pgfpathlineto{\pgfqpoint{1.038598in}{0.739656in}}%
\pgfpathlineto{\pgfqpoint{1.038302in}{0.739656in}}%
\pgfpathlineto{\pgfqpoint{1.038006in}{0.739656in}}%
\pgfpathlineto{\pgfqpoint{1.037710in}{0.739656in}}%
\pgfpathlineto{\pgfqpoint{1.037414in}{0.739656in}}%
\pgfpathlineto{\pgfqpoint{1.037118in}{0.739656in}}%
\pgfpathlineto{\pgfqpoint{1.036822in}{0.739656in}}%
\pgfpathlineto{\pgfqpoint{1.036526in}{0.739656in}}%
\pgfpathlineto{\pgfqpoint{1.036230in}{0.739656in}}%
\pgfpathlineto{\pgfqpoint{1.035934in}{0.739656in}}%
\pgfpathlineto{\pgfqpoint{1.035638in}{0.739656in}}%
\pgfpathlineto{\pgfqpoint{1.035342in}{0.739656in}}%
\pgfpathlineto{\pgfqpoint{1.035046in}{0.739656in}}%
\pgfpathlineto{\pgfqpoint{1.034750in}{0.739656in}}%
\pgfpathlineto{\pgfqpoint{1.034454in}{0.739656in}}%
\pgfpathlineto{\pgfqpoint{1.034158in}{0.739656in}}%
\pgfpathlineto{\pgfqpoint{1.033862in}{0.739656in}}%
\pgfpathlineto{\pgfqpoint{1.033566in}{0.739656in}}%
\pgfpathlineto{\pgfqpoint{1.033270in}{0.739656in}}%
\pgfpathlineto{\pgfqpoint{1.032974in}{0.739656in}}%
\pgfpathlineto{\pgfqpoint{1.032678in}{0.739656in}}%
\pgfpathlineto{\pgfqpoint{1.032382in}{0.739656in}}%
\pgfpathlineto{\pgfqpoint{1.032086in}{0.739656in}}%
\pgfpathlineto{\pgfqpoint{1.031790in}{0.739656in}}%
\pgfpathlineto{\pgfqpoint{1.031494in}{0.739656in}}%
\pgfpathlineto{\pgfqpoint{1.031198in}{0.739656in}}%
\pgfpathlineto{\pgfqpoint{1.030902in}{0.739656in}}%
\pgfpathlineto{\pgfqpoint{1.030606in}{0.739656in}}%
\pgfpathlineto{\pgfqpoint{1.030310in}{0.739656in}}%
\pgfpathlineto{\pgfqpoint{1.030014in}{0.739656in}}%
\pgfpathlineto{\pgfqpoint{1.029718in}{0.739656in}}%
\pgfpathlineto{\pgfqpoint{1.029422in}{0.739656in}}%
\pgfpathlineto{\pgfqpoint{1.029126in}{0.739656in}}%
\pgfpathlineto{\pgfqpoint{1.028830in}{0.739656in}}%
\pgfpathlineto{\pgfqpoint{1.028534in}{0.739656in}}%
\pgfpathlineto{\pgfqpoint{1.028238in}{0.739656in}}%
\pgfpathlineto{\pgfqpoint{1.027942in}{0.739656in}}%
\pgfpathlineto{\pgfqpoint{1.027646in}{0.739656in}}%
\pgfpathlineto{\pgfqpoint{1.027350in}{0.739656in}}%
\pgfpathlineto{\pgfqpoint{1.027054in}{0.739656in}}%
\pgfpathlineto{\pgfqpoint{1.026758in}{0.739656in}}%
\pgfpathlineto{\pgfqpoint{1.026462in}{0.739656in}}%
\pgfpathlineto{\pgfqpoint{1.026166in}{0.739656in}}%
\pgfpathlineto{\pgfqpoint{1.025870in}{0.739656in}}%
\pgfpathlineto{\pgfqpoint{1.025574in}{0.739656in}}%
\pgfpathlineto{\pgfqpoint{1.025278in}{0.739656in}}%
\pgfpathlineto{\pgfqpoint{1.024982in}{0.739656in}}%
\pgfpathlineto{\pgfqpoint{1.024686in}{0.739656in}}%
\pgfpathlineto{\pgfqpoint{1.024390in}{0.739656in}}%
\pgfpathlineto{\pgfqpoint{1.024094in}{0.739656in}}%
\pgfpathlineto{\pgfqpoint{1.023798in}{0.739656in}}%
\pgfpathlineto{\pgfqpoint{1.023502in}{0.739656in}}%
\pgfpathlineto{\pgfqpoint{1.023206in}{0.739656in}}%
\pgfpathlineto{\pgfqpoint{1.022910in}{0.739656in}}%
\pgfpathlineto{\pgfqpoint{1.022614in}{0.739656in}}%
\pgfpathlineto{\pgfqpoint{1.022318in}{0.739656in}}%
\pgfpathlineto{\pgfqpoint{1.022022in}{0.739656in}}%
\pgfpathlineto{\pgfqpoint{1.021726in}{0.739656in}}%
\pgfpathlineto{\pgfqpoint{1.021430in}{0.739656in}}%
\pgfpathlineto{\pgfqpoint{1.021134in}{0.739656in}}%
\pgfpathlineto{\pgfqpoint{1.020838in}{0.739656in}}%
\pgfpathlineto{\pgfqpoint{1.020542in}{0.739656in}}%
\pgfpathlineto{\pgfqpoint{1.020246in}{0.739656in}}%
\pgfpathlineto{\pgfqpoint{1.019950in}{0.739656in}}%
\pgfpathlineto{\pgfqpoint{1.019654in}{0.739656in}}%
\pgfpathlineto{\pgfqpoint{1.019358in}{0.739656in}}%
\pgfpathlineto{\pgfqpoint{1.019062in}{0.739656in}}%
\pgfpathlineto{\pgfqpoint{1.018766in}{0.739656in}}%
\pgfpathlineto{\pgfqpoint{1.018470in}{0.739656in}}%
\pgfpathlineto{\pgfqpoint{1.018174in}{0.739656in}}%
\pgfpathlineto{\pgfqpoint{1.017878in}{0.739656in}}%
\pgfpathlineto{\pgfqpoint{1.017582in}{0.739656in}}%
\pgfpathlineto{\pgfqpoint{1.017286in}{0.739656in}}%
\pgfpathlineto{\pgfqpoint{1.016990in}{0.739656in}}%
\pgfpathlineto{\pgfqpoint{1.016694in}{0.739656in}}%
\pgfpathlineto{\pgfqpoint{1.016398in}{0.739656in}}%
\pgfpathlineto{\pgfqpoint{1.016102in}{0.739656in}}%
\pgfpathlineto{\pgfqpoint{1.015806in}{0.739656in}}%
\pgfpathlineto{\pgfqpoint{1.015510in}{0.739656in}}%
\pgfpathlineto{\pgfqpoint{1.015214in}{0.739656in}}%
\pgfpathlineto{\pgfqpoint{1.014918in}{0.739656in}}%
\pgfpathlineto{\pgfqpoint{1.014622in}{0.739656in}}%
\pgfpathlineto{\pgfqpoint{1.014326in}{0.739656in}}%
\pgfpathlineto{\pgfqpoint{1.014030in}{0.739656in}}%
\pgfpathlineto{\pgfqpoint{1.013734in}{0.739656in}}%
\pgfpathlineto{\pgfqpoint{1.013438in}{0.739656in}}%
\pgfpathlineto{\pgfqpoint{1.013142in}{0.739656in}}%
\pgfpathlineto{\pgfqpoint{1.012846in}{0.739656in}}%
\pgfpathlineto{\pgfqpoint{1.012550in}{0.739656in}}%
\pgfpathlineto{\pgfqpoint{1.012254in}{0.739656in}}%
\pgfpathlineto{\pgfqpoint{1.011958in}{0.739656in}}%
\pgfpathlineto{\pgfqpoint{1.011662in}{0.739656in}}%
\pgfpathlineto{\pgfqpoint{1.011366in}{0.739656in}}%
\pgfpathlineto{\pgfqpoint{1.011070in}{0.739656in}}%
\pgfpathlineto{\pgfqpoint{1.010774in}{0.739656in}}%
\pgfpathlineto{\pgfqpoint{1.010478in}{0.739656in}}%
\pgfpathlineto{\pgfqpoint{1.010182in}{0.739656in}}%
\pgfpathlineto{\pgfqpoint{1.009886in}{0.739656in}}%
\pgfpathlineto{\pgfqpoint{1.009590in}{0.739656in}}%
\pgfpathlineto{\pgfqpoint{1.009294in}{0.739656in}}%
\pgfpathlineto{\pgfqpoint{1.008998in}{0.739656in}}%
\pgfpathlineto{\pgfqpoint{1.008702in}{0.739656in}}%
\pgfpathlineto{\pgfqpoint{1.008406in}{0.739656in}}%
\pgfpathlineto{\pgfqpoint{1.008110in}{0.739656in}}%
\pgfpathlineto{\pgfqpoint{1.007814in}{0.739656in}}%
\pgfpathlineto{\pgfqpoint{1.007518in}{0.739656in}}%
\pgfpathlineto{\pgfqpoint{1.007222in}{0.739656in}}%
\pgfpathlineto{\pgfqpoint{1.006926in}{0.739656in}}%
\pgfpathlineto{\pgfqpoint{1.006630in}{0.739656in}}%
\pgfpathlineto{\pgfqpoint{1.006334in}{0.739656in}}%
\pgfpathlineto{\pgfqpoint{1.006038in}{0.739656in}}%
\pgfpathlineto{\pgfqpoint{1.005742in}{0.739656in}}%
\pgfpathlineto{\pgfqpoint{1.005446in}{0.739656in}}%
\pgfpathlineto{\pgfqpoint{1.005150in}{0.739656in}}%
\pgfpathlineto{\pgfqpoint{1.004854in}{0.739656in}}%
\pgfpathlineto{\pgfqpoint{1.004558in}{0.739656in}}%
\pgfpathlineto{\pgfqpoint{1.004262in}{0.739656in}}%
\pgfpathlineto{\pgfqpoint{1.003966in}{0.739656in}}%
\pgfpathlineto{\pgfqpoint{1.003670in}{0.739656in}}%
\pgfpathlineto{\pgfqpoint{1.003374in}{0.739656in}}%
\pgfpathlineto{\pgfqpoint{1.003078in}{0.739656in}}%
\pgfpathlineto{\pgfqpoint{1.002782in}{0.739656in}}%
\pgfpathlineto{\pgfqpoint{1.002486in}{0.739656in}}%
\pgfpathlineto{\pgfqpoint{1.002190in}{0.739656in}}%
\pgfpathlineto{\pgfqpoint{1.001894in}{0.739656in}}%
\pgfpathlineto{\pgfqpoint{1.001598in}{0.739656in}}%
\pgfpathlineto{\pgfqpoint{1.001302in}{0.739656in}}%
\pgfpathlineto{\pgfqpoint{1.001006in}{0.739656in}}%
\pgfpathlineto{\pgfqpoint{1.000710in}{0.739656in}}%
\pgfpathlineto{\pgfqpoint{1.000414in}{0.739656in}}%
\pgfpathlineto{\pgfqpoint{1.000118in}{0.739656in}}%
\pgfpathlineto{\pgfqpoint{0.999822in}{0.739656in}}%
\pgfpathlineto{\pgfqpoint{0.999526in}{0.739656in}}%
\pgfpathlineto{\pgfqpoint{0.999230in}{0.739656in}}%
\pgfpathlineto{\pgfqpoint{0.998934in}{0.739656in}}%
\pgfpathlineto{\pgfqpoint{0.998638in}{0.739656in}}%
\pgfpathlineto{\pgfqpoint{0.998342in}{0.739656in}}%
\pgfpathlineto{\pgfqpoint{0.998046in}{0.739656in}}%
\pgfpathlineto{\pgfqpoint{0.997750in}{0.739656in}}%
\pgfpathlineto{\pgfqpoint{0.997454in}{0.739656in}}%
\pgfpathlineto{\pgfqpoint{0.997158in}{0.739656in}}%
\pgfpathlineto{\pgfqpoint{0.996862in}{0.739656in}}%
\pgfpathlineto{\pgfqpoint{0.996566in}{0.739656in}}%
\pgfpathlineto{\pgfqpoint{0.996270in}{0.739656in}}%
\pgfpathlineto{\pgfqpoint{0.995974in}{0.739656in}}%
\pgfpathlineto{\pgfqpoint{0.995678in}{0.739656in}}%
\pgfpathlineto{\pgfqpoint{0.995382in}{0.739656in}}%
\pgfpathlineto{\pgfqpoint{0.995086in}{0.739656in}}%
\pgfpathlineto{\pgfqpoint{0.994790in}{0.739656in}}%
\pgfpathlineto{\pgfqpoint{0.994494in}{0.739656in}}%
\pgfpathlineto{\pgfqpoint{0.994198in}{0.739656in}}%
\pgfpathlineto{\pgfqpoint{0.993902in}{0.739656in}}%
\pgfpathlineto{\pgfqpoint{0.993606in}{0.739656in}}%
\pgfpathlineto{\pgfqpoint{0.993310in}{0.739656in}}%
\pgfpathlineto{\pgfqpoint{0.993014in}{0.739656in}}%
\pgfpathlineto{\pgfqpoint{0.992718in}{0.739656in}}%
\pgfpathlineto{\pgfqpoint{0.992422in}{0.739656in}}%
\pgfpathlineto{\pgfqpoint{0.992126in}{0.739656in}}%
\pgfpathlineto{\pgfqpoint{0.991830in}{0.739656in}}%
\pgfpathlineto{\pgfqpoint{0.991534in}{0.739656in}}%
\pgfpathlineto{\pgfqpoint{0.991238in}{0.739656in}}%
\pgfpathlineto{\pgfqpoint{0.990942in}{0.739656in}}%
\pgfpathlineto{\pgfqpoint{0.990646in}{0.739656in}}%
\pgfpathlineto{\pgfqpoint{0.990350in}{0.739656in}}%
\pgfpathlineto{\pgfqpoint{0.990054in}{0.739656in}}%
\pgfpathlineto{\pgfqpoint{0.989758in}{0.739656in}}%
\pgfpathlineto{\pgfqpoint{0.989462in}{0.739656in}}%
\pgfpathlineto{\pgfqpoint{0.989166in}{0.739656in}}%
\pgfpathlineto{\pgfqpoint{0.988870in}{0.739656in}}%
\pgfpathlineto{\pgfqpoint{0.988574in}{0.739656in}}%
\pgfpathlineto{\pgfqpoint{0.988278in}{0.739656in}}%
\pgfpathlineto{\pgfqpoint{0.987982in}{0.739656in}}%
\pgfpathlineto{\pgfqpoint{0.987686in}{0.739656in}}%
\pgfpathlineto{\pgfqpoint{0.987390in}{0.739656in}}%
\pgfpathlineto{\pgfqpoint{0.987094in}{0.739656in}}%
\pgfpathlineto{\pgfqpoint{0.986798in}{0.739656in}}%
\pgfpathlineto{\pgfqpoint{0.986502in}{0.739656in}}%
\pgfpathlineto{\pgfqpoint{0.986206in}{0.739656in}}%
\pgfpathlineto{\pgfqpoint{0.985910in}{0.739656in}}%
\pgfpathlineto{\pgfqpoint{0.985614in}{0.739656in}}%
\pgfpathlineto{\pgfqpoint{0.985318in}{0.739656in}}%
\pgfpathlineto{\pgfqpoint{0.985021in}{0.739656in}}%
\pgfpathlineto{\pgfqpoint{0.984725in}{0.739656in}}%
\pgfpathlineto{\pgfqpoint{0.984429in}{0.739656in}}%
\pgfpathlineto{\pgfqpoint{0.984133in}{0.739656in}}%
\pgfpathlineto{\pgfqpoint{0.983837in}{0.739656in}}%
\pgfpathlineto{\pgfqpoint{0.983541in}{0.739656in}}%
\pgfpathlineto{\pgfqpoint{0.983245in}{0.739656in}}%
\pgfpathlineto{\pgfqpoint{0.982949in}{0.739656in}}%
\pgfpathlineto{\pgfqpoint{0.982653in}{0.739656in}}%
\pgfpathlineto{\pgfqpoint{0.982357in}{0.739656in}}%
\pgfpathlineto{\pgfqpoint{0.982061in}{0.739656in}}%
\pgfpathlineto{\pgfqpoint{0.981765in}{0.739656in}}%
\pgfpathlineto{\pgfqpoint{0.981469in}{0.739656in}}%
\pgfpathlineto{\pgfqpoint{0.981173in}{0.739656in}}%
\pgfpathlineto{\pgfqpoint{0.980877in}{0.739656in}}%
\pgfpathlineto{\pgfqpoint{0.980581in}{0.739656in}}%
\pgfpathlineto{\pgfqpoint{0.980285in}{0.739656in}}%
\pgfpathlineto{\pgfqpoint{0.979989in}{0.739656in}}%
\pgfpathlineto{\pgfqpoint{0.979693in}{0.739656in}}%
\pgfpathlineto{\pgfqpoint{0.979397in}{0.739656in}}%
\pgfpathlineto{\pgfqpoint{0.979101in}{0.739656in}}%
\pgfpathlineto{\pgfqpoint{0.978805in}{0.739656in}}%
\pgfpathlineto{\pgfqpoint{0.978509in}{0.739656in}}%
\pgfpathlineto{\pgfqpoint{0.978213in}{0.739656in}}%
\pgfpathlineto{\pgfqpoint{0.977917in}{0.739656in}}%
\pgfpathlineto{\pgfqpoint{0.977621in}{0.739656in}}%
\pgfpathlineto{\pgfqpoint{0.977325in}{0.739656in}}%
\pgfpathlineto{\pgfqpoint{0.977029in}{0.739656in}}%
\pgfpathlineto{\pgfqpoint{0.976733in}{0.739656in}}%
\pgfpathlineto{\pgfqpoint{0.976437in}{0.739656in}}%
\pgfpathlineto{\pgfqpoint{0.976141in}{0.739656in}}%
\pgfpathlineto{\pgfqpoint{0.975845in}{0.739656in}}%
\pgfpathlineto{\pgfqpoint{0.975549in}{0.739656in}}%
\pgfpathlineto{\pgfqpoint{0.975253in}{0.739656in}}%
\pgfpathlineto{\pgfqpoint{0.974957in}{0.739656in}}%
\pgfpathlineto{\pgfqpoint{0.974661in}{0.739656in}}%
\pgfpathlineto{\pgfqpoint{0.974365in}{0.739656in}}%
\pgfpathlineto{\pgfqpoint{0.974069in}{0.739656in}}%
\pgfpathlineto{\pgfqpoint{0.973773in}{0.739656in}}%
\pgfpathlineto{\pgfqpoint{0.973477in}{0.739656in}}%
\pgfpathlineto{\pgfqpoint{0.973181in}{0.739656in}}%
\pgfpathlineto{\pgfqpoint{0.972885in}{0.739656in}}%
\pgfpathlineto{\pgfqpoint{0.972589in}{0.739656in}}%
\pgfpathlineto{\pgfqpoint{0.972293in}{0.739656in}}%
\pgfpathlineto{\pgfqpoint{0.971997in}{0.739656in}}%
\pgfpathlineto{\pgfqpoint{0.971701in}{0.739656in}}%
\pgfpathlineto{\pgfqpoint{0.971405in}{0.739656in}}%
\pgfpathlineto{\pgfqpoint{0.971109in}{0.739656in}}%
\pgfpathlineto{\pgfqpoint{0.970813in}{0.739656in}}%
\pgfpathlineto{\pgfqpoint{0.970517in}{0.739656in}}%
\pgfpathlineto{\pgfqpoint{0.970221in}{0.739656in}}%
\pgfpathlineto{\pgfqpoint{0.969925in}{0.739656in}}%
\pgfpathlineto{\pgfqpoint{0.969629in}{0.739656in}}%
\pgfpathlineto{\pgfqpoint{0.969333in}{0.739656in}}%
\pgfpathlineto{\pgfqpoint{0.969037in}{0.739656in}}%
\pgfpathlineto{\pgfqpoint{0.968741in}{0.739656in}}%
\pgfpathlineto{\pgfqpoint{0.968445in}{0.739656in}}%
\pgfpathlineto{\pgfqpoint{0.968149in}{0.739656in}}%
\pgfpathlineto{\pgfqpoint{0.967853in}{0.739656in}}%
\pgfpathlineto{\pgfqpoint{0.967557in}{0.739656in}}%
\pgfpathlineto{\pgfqpoint{0.967261in}{0.739656in}}%
\pgfpathlineto{\pgfqpoint{0.966965in}{0.739656in}}%
\pgfpathlineto{\pgfqpoint{0.966669in}{0.739656in}}%
\pgfpathlineto{\pgfqpoint{0.966373in}{0.739656in}}%
\pgfpathlineto{\pgfqpoint{0.966077in}{0.739656in}}%
\pgfpathlineto{\pgfqpoint{0.965781in}{0.739656in}}%
\pgfpathlineto{\pgfqpoint{0.965485in}{0.739656in}}%
\pgfpathlineto{\pgfqpoint{0.965189in}{0.739656in}}%
\pgfpathlineto{\pgfqpoint{0.964893in}{0.739656in}}%
\pgfpathlineto{\pgfqpoint{0.964597in}{0.739656in}}%
\pgfpathlineto{\pgfqpoint{0.964301in}{0.739656in}}%
\pgfpathlineto{\pgfqpoint{0.964005in}{0.739656in}}%
\pgfpathlineto{\pgfqpoint{0.963709in}{0.739656in}}%
\pgfpathlineto{\pgfqpoint{0.963413in}{0.739656in}}%
\pgfpathlineto{\pgfqpoint{0.963117in}{0.739656in}}%
\pgfpathlineto{\pgfqpoint{0.962821in}{0.739656in}}%
\pgfpathlineto{\pgfqpoint{0.962525in}{0.739656in}}%
\pgfpathlineto{\pgfqpoint{0.962229in}{0.739656in}}%
\pgfpathlineto{\pgfqpoint{0.961933in}{0.739656in}}%
\pgfpathlineto{\pgfqpoint{0.961637in}{0.739656in}}%
\pgfpathlineto{\pgfqpoint{0.961341in}{0.739656in}}%
\pgfpathlineto{\pgfqpoint{0.961045in}{0.739656in}}%
\pgfpathlineto{\pgfqpoint{0.960749in}{0.739656in}}%
\pgfpathlineto{\pgfqpoint{0.960453in}{0.739656in}}%
\pgfpathlineto{\pgfqpoint{0.960157in}{0.739656in}}%
\pgfpathlineto{\pgfqpoint{0.959861in}{0.739656in}}%
\pgfpathlineto{\pgfqpoint{0.959565in}{0.739656in}}%
\pgfpathlineto{\pgfqpoint{0.959269in}{0.739656in}}%
\pgfpathlineto{\pgfqpoint{0.958973in}{0.739656in}}%
\pgfpathlineto{\pgfqpoint{0.958677in}{0.739656in}}%
\pgfpathlineto{\pgfqpoint{0.958381in}{0.739656in}}%
\pgfpathlineto{\pgfqpoint{0.958085in}{0.739656in}}%
\pgfpathlineto{\pgfqpoint{0.957789in}{0.739656in}}%
\pgfpathlineto{\pgfqpoint{0.957493in}{0.739656in}}%
\pgfpathlineto{\pgfqpoint{0.957197in}{0.739656in}}%
\pgfpathlineto{\pgfqpoint{0.956901in}{0.739656in}}%
\pgfpathlineto{\pgfqpoint{0.956605in}{0.739656in}}%
\pgfpathlineto{\pgfqpoint{0.956309in}{0.739656in}}%
\pgfpathlineto{\pgfqpoint{0.956013in}{0.739656in}}%
\pgfpathlineto{\pgfqpoint{0.955717in}{0.739656in}}%
\pgfpathlineto{\pgfqpoint{0.955421in}{0.739656in}}%
\pgfpathlineto{\pgfqpoint{0.955125in}{0.739656in}}%
\pgfpathlineto{\pgfqpoint{0.954829in}{0.739656in}}%
\pgfpathlineto{\pgfqpoint{0.954533in}{0.739656in}}%
\pgfpathlineto{\pgfqpoint{0.954237in}{0.739656in}}%
\pgfpathlineto{\pgfqpoint{0.953941in}{0.739656in}}%
\pgfpathlineto{\pgfqpoint{0.953645in}{0.739656in}}%
\pgfpathlineto{\pgfqpoint{0.953349in}{0.739656in}}%
\pgfpathlineto{\pgfqpoint{0.953053in}{0.739656in}}%
\pgfpathlineto{\pgfqpoint{0.952757in}{0.739656in}}%
\pgfpathlineto{\pgfqpoint{0.952461in}{0.739656in}}%
\pgfpathlineto{\pgfqpoint{0.952165in}{0.739656in}}%
\pgfpathlineto{\pgfqpoint{0.951869in}{0.739656in}}%
\pgfpathlineto{\pgfqpoint{0.951573in}{0.739656in}}%
\pgfpathlineto{\pgfqpoint{0.951277in}{0.739656in}}%
\pgfpathlineto{\pgfqpoint{0.950981in}{0.739656in}}%
\pgfpathlineto{\pgfqpoint{0.950685in}{0.739656in}}%
\pgfpathlineto{\pgfqpoint{0.950389in}{0.739656in}}%
\pgfpathlineto{\pgfqpoint{0.950093in}{0.739656in}}%
\pgfpathlineto{\pgfqpoint{0.949797in}{0.739656in}}%
\pgfpathlineto{\pgfqpoint{0.949501in}{0.739656in}}%
\pgfpathlineto{\pgfqpoint{0.949205in}{0.739656in}}%
\pgfpathlineto{\pgfqpoint{0.948909in}{0.739656in}}%
\pgfpathlineto{\pgfqpoint{0.948613in}{0.739656in}}%
\pgfpathlineto{\pgfqpoint{0.948317in}{0.739656in}}%
\pgfpathlineto{\pgfqpoint{0.948021in}{0.739656in}}%
\pgfpathlineto{\pgfqpoint{0.947725in}{0.739656in}}%
\pgfpathlineto{\pgfqpoint{0.947429in}{0.739656in}}%
\pgfpathlineto{\pgfqpoint{0.947133in}{0.739656in}}%
\pgfpathlineto{\pgfqpoint{0.946837in}{0.739656in}}%
\pgfpathlineto{\pgfqpoint{0.946541in}{0.739656in}}%
\pgfpathlineto{\pgfqpoint{0.946245in}{0.739656in}}%
\pgfpathlineto{\pgfqpoint{0.945949in}{0.739656in}}%
\pgfpathlineto{\pgfqpoint{0.945653in}{0.739656in}}%
\pgfpathlineto{\pgfqpoint{0.945357in}{0.739656in}}%
\pgfpathlineto{\pgfqpoint{0.945061in}{0.739656in}}%
\pgfpathlineto{\pgfqpoint{0.944765in}{0.739656in}}%
\pgfpathlineto{\pgfqpoint{0.944469in}{0.739656in}}%
\pgfpathlineto{\pgfqpoint{0.944173in}{0.739656in}}%
\pgfpathlineto{\pgfqpoint{0.943877in}{0.739656in}}%
\pgfpathlineto{\pgfqpoint{0.943581in}{0.739656in}}%
\pgfpathlineto{\pgfqpoint{0.943285in}{0.739656in}}%
\pgfpathlineto{\pgfqpoint{0.942989in}{0.739656in}}%
\pgfpathlineto{\pgfqpoint{0.942693in}{0.739656in}}%
\pgfpathlineto{\pgfqpoint{0.942397in}{0.739656in}}%
\pgfpathlineto{\pgfqpoint{0.942101in}{0.739656in}}%
\pgfpathlineto{\pgfqpoint{0.941805in}{0.739656in}}%
\pgfpathlineto{\pgfqpoint{0.941509in}{0.739656in}}%
\pgfpathlineto{\pgfqpoint{0.941213in}{0.739656in}}%
\pgfpathlineto{\pgfqpoint{0.940917in}{0.739656in}}%
\pgfpathlineto{\pgfqpoint{0.940621in}{0.739656in}}%
\pgfpathlineto{\pgfqpoint{0.940325in}{0.739656in}}%
\pgfpathlineto{\pgfqpoint{0.940029in}{0.739656in}}%
\pgfpathlineto{\pgfqpoint{0.939733in}{0.739656in}}%
\pgfpathlineto{\pgfqpoint{0.939437in}{0.739656in}}%
\pgfpathlineto{\pgfqpoint{0.939141in}{0.739656in}}%
\pgfpathlineto{\pgfqpoint{0.938845in}{0.739656in}}%
\pgfpathlineto{\pgfqpoint{0.938549in}{0.739656in}}%
\pgfpathlineto{\pgfqpoint{0.938253in}{0.739656in}}%
\pgfpathlineto{\pgfqpoint{0.937957in}{0.739656in}}%
\pgfpathlineto{\pgfqpoint{0.937661in}{0.739656in}}%
\pgfpathlineto{\pgfqpoint{0.937365in}{0.739656in}}%
\pgfpathlineto{\pgfqpoint{0.937069in}{0.739656in}}%
\pgfpathlineto{\pgfqpoint{0.936773in}{0.739656in}}%
\pgfpathlineto{\pgfqpoint{0.936477in}{0.739656in}}%
\pgfpathlineto{\pgfqpoint{0.936181in}{0.739656in}}%
\pgfpathlineto{\pgfqpoint{0.935885in}{0.739656in}}%
\pgfpathlineto{\pgfqpoint{0.935589in}{0.739656in}}%
\pgfpathlineto{\pgfqpoint{0.935293in}{0.739656in}}%
\pgfpathlineto{\pgfqpoint{0.934997in}{0.739656in}}%
\pgfpathlineto{\pgfqpoint{0.934701in}{0.739656in}}%
\pgfpathlineto{\pgfqpoint{0.934405in}{0.739656in}}%
\pgfpathlineto{\pgfqpoint{0.934109in}{0.739656in}}%
\pgfpathlineto{\pgfqpoint{0.933813in}{0.739656in}}%
\pgfpathlineto{\pgfqpoint{0.933517in}{0.739656in}}%
\pgfpathlineto{\pgfqpoint{0.933221in}{0.739656in}}%
\pgfpathlineto{\pgfqpoint{0.932925in}{0.739656in}}%
\pgfpathlineto{\pgfqpoint{0.932629in}{0.739656in}}%
\pgfpathlineto{\pgfqpoint{0.932333in}{0.739656in}}%
\pgfpathlineto{\pgfqpoint{0.932037in}{0.739656in}}%
\pgfpathlineto{\pgfqpoint{0.931741in}{0.739656in}}%
\pgfpathlineto{\pgfqpoint{0.931445in}{0.739656in}}%
\pgfpathlineto{\pgfqpoint{0.931149in}{0.739656in}}%
\pgfpathlineto{\pgfqpoint{0.930853in}{0.739656in}}%
\pgfpathlineto{\pgfqpoint{0.930557in}{0.739656in}}%
\pgfpathlineto{\pgfqpoint{0.930261in}{0.739656in}}%
\pgfpathlineto{\pgfqpoint{0.929965in}{0.739656in}}%
\pgfpathlineto{\pgfqpoint{0.929669in}{0.739656in}}%
\pgfpathlineto{\pgfqpoint{0.929373in}{0.739656in}}%
\pgfpathlineto{\pgfqpoint{0.929077in}{0.739656in}}%
\pgfpathlineto{\pgfqpoint{0.928781in}{0.739656in}}%
\pgfpathlineto{\pgfqpoint{0.928485in}{0.739656in}}%
\pgfpathlineto{\pgfqpoint{0.928189in}{0.739656in}}%
\pgfpathlineto{\pgfqpoint{0.927893in}{0.739656in}}%
\pgfpathlineto{\pgfqpoint{0.927597in}{0.739656in}}%
\pgfpathlineto{\pgfqpoint{0.927301in}{0.739656in}}%
\pgfpathlineto{\pgfqpoint{0.927005in}{0.739656in}}%
\pgfpathlineto{\pgfqpoint{0.926709in}{0.739656in}}%
\pgfpathlineto{\pgfqpoint{0.926413in}{0.739656in}}%
\pgfpathlineto{\pgfqpoint{0.926117in}{0.739656in}}%
\pgfpathlineto{\pgfqpoint{0.925821in}{0.739656in}}%
\pgfpathlineto{\pgfqpoint{0.925525in}{0.739656in}}%
\pgfpathlineto{\pgfqpoint{0.925229in}{0.739656in}}%
\pgfpathlineto{\pgfqpoint{0.924933in}{0.739656in}}%
\pgfpathlineto{\pgfqpoint{0.924637in}{0.739656in}}%
\pgfpathlineto{\pgfqpoint{0.924341in}{0.739656in}}%
\pgfpathlineto{\pgfqpoint{0.924045in}{0.739656in}}%
\pgfpathlineto{\pgfqpoint{0.923749in}{0.739656in}}%
\pgfpathlineto{\pgfqpoint{0.923453in}{0.739656in}}%
\pgfpathlineto{\pgfqpoint{0.923157in}{0.739656in}}%
\pgfpathlineto{\pgfqpoint{0.922861in}{0.739656in}}%
\pgfpathlineto{\pgfqpoint{0.922565in}{0.739656in}}%
\pgfpathlineto{\pgfqpoint{0.922269in}{0.739656in}}%
\pgfpathlineto{\pgfqpoint{0.921973in}{0.739656in}}%
\pgfpathlineto{\pgfqpoint{0.921677in}{0.739656in}}%
\pgfpathlineto{\pgfqpoint{0.921381in}{0.739656in}}%
\pgfpathlineto{\pgfqpoint{0.921085in}{0.739656in}}%
\pgfpathlineto{\pgfqpoint{0.920789in}{0.739656in}}%
\pgfpathlineto{\pgfqpoint{0.920493in}{0.739656in}}%
\pgfpathlineto{\pgfqpoint{0.920197in}{0.739656in}}%
\pgfpathlineto{\pgfqpoint{0.919901in}{0.739656in}}%
\pgfpathlineto{\pgfqpoint{0.919605in}{0.739656in}}%
\pgfpathlineto{\pgfqpoint{0.919309in}{0.739656in}}%
\pgfpathlineto{\pgfqpoint{0.919013in}{0.739656in}}%
\pgfpathlineto{\pgfqpoint{0.918717in}{0.739656in}}%
\pgfpathlineto{\pgfqpoint{0.918421in}{0.739656in}}%
\pgfpathlineto{\pgfqpoint{0.918125in}{0.739656in}}%
\pgfpathlineto{\pgfqpoint{0.917829in}{0.739656in}}%
\pgfpathlineto{\pgfqpoint{0.917532in}{0.739656in}}%
\pgfpathlineto{\pgfqpoint{0.917236in}{0.739656in}}%
\pgfpathlineto{\pgfqpoint{0.916940in}{0.739656in}}%
\pgfpathlineto{\pgfqpoint{0.916644in}{0.739656in}}%
\pgfpathlineto{\pgfqpoint{0.916348in}{0.739656in}}%
\pgfpathlineto{\pgfqpoint{0.916052in}{0.739656in}}%
\pgfpathlineto{\pgfqpoint{0.915756in}{0.739656in}}%
\pgfpathlineto{\pgfqpoint{0.915460in}{0.739656in}}%
\pgfpathlineto{\pgfqpoint{0.915164in}{0.739656in}}%
\pgfpathlineto{\pgfqpoint{0.914868in}{0.739656in}}%
\pgfpathlineto{\pgfqpoint{0.914572in}{0.739656in}}%
\pgfpathlineto{\pgfqpoint{0.914276in}{0.739656in}}%
\pgfpathlineto{\pgfqpoint{0.913980in}{0.739656in}}%
\pgfpathlineto{\pgfqpoint{0.913684in}{0.739656in}}%
\pgfpathlineto{\pgfqpoint{0.913388in}{0.739656in}}%
\pgfpathlineto{\pgfqpoint{0.913092in}{0.739656in}}%
\pgfpathlineto{\pgfqpoint{0.912796in}{0.739656in}}%
\pgfpathlineto{\pgfqpoint{0.912500in}{0.739656in}}%
\pgfpathlineto{\pgfqpoint{0.912204in}{0.739656in}}%
\pgfpathlineto{\pgfqpoint{0.911908in}{0.739656in}}%
\pgfpathlineto{\pgfqpoint{0.911612in}{0.739656in}}%
\pgfpathlineto{\pgfqpoint{0.911316in}{0.739656in}}%
\pgfpathlineto{\pgfqpoint{0.911020in}{0.739656in}}%
\pgfpathlineto{\pgfqpoint{0.910724in}{0.739656in}}%
\pgfpathlineto{\pgfqpoint{0.910428in}{0.739656in}}%
\pgfpathlineto{\pgfqpoint{0.910132in}{0.739656in}}%
\pgfpathlineto{\pgfqpoint{0.909836in}{0.739656in}}%
\pgfpathlineto{\pgfqpoint{0.909540in}{0.739656in}}%
\pgfpathlineto{\pgfqpoint{0.909244in}{0.739656in}}%
\pgfpathlineto{\pgfqpoint{0.908948in}{0.739656in}}%
\pgfpathlineto{\pgfqpoint{0.908652in}{0.739656in}}%
\pgfpathlineto{\pgfqpoint{0.908356in}{0.739656in}}%
\pgfpathlineto{\pgfqpoint{0.908060in}{0.739656in}}%
\pgfpathlineto{\pgfqpoint{0.907764in}{0.739656in}}%
\pgfpathlineto{\pgfqpoint{0.907468in}{0.739656in}}%
\pgfpathlineto{\pgfqpoint{0.907172in}{0.739656in}}%
\pgfpathlineto{\pgfqpoint{0.906876in}{0.739656in}}%
\pgfpathlineto{\pgfqpoint{0.906580in}{0.739656in}}%
\pgfpathlineto{\pgfqpoint{0.906284in}{0.739656in}}%
\pgfpathlineto{\pgfqpoint{0.905988in}{0.739656in}}%
\pgfpathlineto{\pgfqpoint{0.905692in}{0.739656in}}%
\pgfpathlineto{\pgfqpoint{0.905396in}{0.739656in}}%
\pgfpathlineto{\pgfqpoint{0.905100in}{0.739656in}}%
\pgfpathlineto{\pgfqpoint{0.904804in}{0.739656in}}%
\pgfpathlineto{\pgfqpoint{0.904508in}{0.739656in}}%
\pgfpathlineto{\pgfqpoint{0.904212in}{0.739656in}}%
\pgfpathlineto{\pgfqpoint{0.903916in}{0.739656in}}%
\pgfpathlineto{\pgfqpoint{0.903620in}{0.739656in}}%
\pgfpathlineto{\pgfqpoint{0.903324in}{0.739656in}}%
\pgfpathlineto{\pgfqpoint{0.903028in}{0.739656in}}%
\pgfpathlineto{\pgfqpoint{0.902732in}{0.739656in}}%
\pgfpathlineto{\pgfqpoint{0.902436in}{0.739656in}}%
\pgfpathlineto{\pgfqpoint{0.902140in}{0.739656in}}%
\pgfpathlineto{\pgfqpoint{0.901844in}{0.739656in}}%
\pgfpathlineto{\pgfqpoint{0.901548in}{0.739656in}}%
\pgfpathlineto{\pgfqpoint{0.901252in}{0.739656in}}%
\pgfpathlineto{\pgfqpoint{0.900956in}{0.739656in}}%
\pgfpathlineto{\pgfqpoint{0.900660in}{0.739656in}}%
\pgfpathlineto{\pgfqpoint{0.900364in}{0.739656in}}%
\pgfpathlineto{\pgfqpoint{0.900068in}{0.739656in}}%
\pgfpathlineto{\pgfqpoint{0.899772in}{0.739656in}}%
\pgfpathlineto{\pgfqpoint{0.899476in}{0.739656in}}%
\pgfpathlineto{\pgfqpoint{0.899180in}{0.739656in}}%
\pgfpathlineto{\pgfqpoint{0.898884in}{0.739656in}}%
\pgfpathlineto{\pgfqpoint{0.898588in}{0.739656in}}%
\pgfpathlineto{\pgfqpoint{0.898292in}{0.739656in}}%
\pgfpathlineto{\pgfqpoint{0.897996in}{0.739656in}}%
\pgfpathlineto{\pgfqpoint{0.897700in}{0.739656in}}%
\pgfpathlineto{\pgfqpoint{0.897404in}{0.739656in}}%
\pgfpathlineto{\pgfqpoint{0.897108in}{0.739656in}}%
\pgfpathlineto{\pgfqpoint{0.896812in}{0.739656in}}%
\pgfpathlineto{\pgfqpoint{0.896516in}{0.739656in}}%
\pgfpathlineto{\pgfqpoint{0.896220in}{0.739656in}}%
\pgfpathlineto{\pgfqpoint{0.895924in}{0.739656in}}%
\pgfpathlineto{\pgfqpoint{0.895628in}{0.739656in}}%
\pgfpathlineto{\pgfqpoint{0.895332in}{0.739656in}}%
\pgfpathlineto{\pgfqpoint{0.895036in}{0.739656in}}%
\pgfpathlineto{\pgfqpoint{0.894740in}{0.739656in}}%
\pgfpathlineto{\pgfqpoint{0.894444in}{0.739656in}}%
\pgfpathlineto{\pgfqpoint{0.894148in}{0.739656in}}%
\pgfpathlineto{\pgfqpoint{0.893852in}{0.739656in}}%
\pgfpathlineto{\pgfqpoint{0.893556in}{0.739656in}}%
\pgfpathlineto{\pgfqpoint{0.893260in}{0.739656in}}%
\pgfpathlineto{\pgfqpoint{0.892964in}{0.739656in}}%
\pgfpathlineto{\pgfqpoint{0.892668in}{0.739656in}}%
\pgfpathlineto{\pgfqpoint{0.892372in}{0.739656in}}%
\pgfpathlineto{\pgfqpoint{0.892076in}{0.739656in}}%
\pgfpathlineto{\pgfqpoint{0.891780in}{0.739656in}}%
\pgfpathlineto{\pgfqpoint{0.891484in}{0.739656in}}%
\pgfpathlineto{\pgfqpoint{0.891188in}{0.739656in}}%
\pgfpathlineto{\pgfqpoint{0.890892in}{0.739656in}}%
\pgfpathlineto{\pgfqpoint{0.890596in}{0.739656in}}%
\pgfpathlineto{\pgfqpoint{0.890300in}{0.739656in}}%
\pgfpathlineto{\pgfqpoint{0.890004in}{0.739656in}}%
\pgfpathlineto{\pgfqpoint{0.889708in}{0.739656in}}%
\pgfpathlineto{\pgfqpoint{0.889412in}{0.739656in}}%
\pgfpathlineto{\pgfqpoint{0.889116in}{0.739656in}}%
\pgfpathlineto{\pgfqpoint{0.888820in}{0.739656in}}%
\pgfpathlineto{\pgfqpoint{0.888524in}{0.739656in}}%
\pgfpathlineto{\pgfqpoint{0.888228in}{0.739656in}}%
\pgfpathlineto{\pgfqpoint{0.887932in}{0.739656in}}%
\pgfpathlineto{\pgfqpoint{0.887636in}{0.739656in}}%
\pgfpathlineto{\pgfqpoint{0.887340in}{0.739656in}}%
\pgfpathlineto{\pgfqpoint{0.887044in}{0.739656in}}%
\pgfpathlineto{\pgfqpoint{0.886748in}{0.739656in}}%
\pgfpathlineto{\pgfqpoint{0.886452in}{0.739656in}}%
\pgfpathlineto{\pgfqpoint{0.886156in}{0.739656in}}%
\pgfpathlineto{\pgfqpoint{0.885860in}{0.739656in}}%
\pgfpathlineto{\pgfqpoint{0.885564in}{0.739656in}}%
\pgfpathlineto{\pgfqpoint{0.885268in}{0.739656in}}%
\pgfpathlineto{\pgfqpoint{0.884972in}{0.739656in}}%
\pgfpathlineto{\pgfqpoint{0.884676in}{0.739656in}}%
\pgfpathlineto{\pgfqpoint{0.884380in}{0.739656in}}%
\pgfpathlineto{\pgfqpoint{0.884084in}{0.739656in}}%
\pgfpathlineto{\pgfqpoint{0.883788in}{0.739656in}}%
\pgfpathlineto{\pgfqpoint{0.883492in}{0.739656in}}%
\pgfpathlineto{\pgfqpoint{0.883196in}{0.739656in}}%
\pgfpathlineto{\pgfqpoint{0.882900in}{0.739656in}}%
\pgfpathlineto{\pgfqpoint{0.882604in}{0.739656in}}%
\pgfpathlineto{\pgfqpoint{0.882308in}{0.739656in}}%
\pgfpathlineto{\pgfqpoint{0.882012in}{0.739656in}}%
\pgfpathlineto{\pgfqpoint{0.881716in}{0.739656in}}%
\pgfpathlineto{\pgfqpoint{0.881420in}{0.739656in}}%
\pgfpathlineto{\pgfqpoint{0.881124in}{0.739656in}}%
\pgfpathlineto{\pgfqpoint{0.880828in}{0.739656in}}%
\pgfpathlineto{\pgfqpoint{0.880532in}{0.739656in}}%
\pgfpathlineto{\pgfqpoint{0.880236in}{0.739656in}}%
\pgfpathlineto{\pgfqpoint{0.879940in}{0.739656in}}%
\pgfpathlineto{\pgfqpoint{0.879644in}{0.739656in}}%
\pgfpathlineto{\pgfqpoint{0.879348in}{0.739656in}}%
\pgfpathlineto{\pgfqpoint{0.879052in}{0.739656in}}%
\pgfpathlineto{\pgfqpoint{0.878756in}{0.739656in}}%
\pgfpathlineto{\pgfqpoint{0.878460in}{0.739656in}}%
\pgfpathlineto{\pgfqpoint{0.878164in}{0.739656in}}%
\pgfpathlineto{\pgfqpoint{0.877868in}{0.739656in}}%
\pgfpathlineto{\pgfqpoint{0.877572in}{0.739656in}}%
\pgfpathlineto{\pgfqpoint{0.877276in}{0.739656in}}%
\pgfpathlineto{\pgfqpoint{0.876980in}{0.739656in}}%
\pgfpathlineto{\pgfqpoint{0.876684in}{0.739656in}}%
\pgfpathlineto{\pgfqpoint{0.876388in}{0.739656in}}%
\pgfpathlineto{\pgfqpoint{0.876092in}{0.739656in}}%
\pgfpathlineto{\pgfqpoint{0.875796in}{0.739656in}}%
\pgfpathlineto{\pgfqpoint{0.875500in}{0.739656in}}%
\pgfpathlineto{\pgfqpoint{0.875204in}{0.739656in}}%
\pgfpathlineto{\pgfqpoint{0.874908in}{0.739656in}}%
\pgfpathlineto{\pgfqpoint{0.874612in}{0.739656in}}%
\pgfpathlineto{\pgfqpoint{0.874316in}{0.739656in}}%
\pgfpathlineto{\pgfqpoint{0.874020in}{0.739656in}}%
\pgfpathlineto{\pgfqpoint{0.873724in}{0.739656in}}%
\pgfpathlineto{\pgfqpoint{0.873428in}{0.739656in}}%
\pgfpathlineto{\pgfqpoint{0.873132in}{0.739656in}}%
\pgfpathlineto{\pgfqpoint{0.872836in}{0.739656in}}%
\pgfpathlineto{\pgfqpoint{0.872540in}{0.739656in}}%
\pgfpathlineto{\pgfqpoint{0.872244in}{0.739656in}}%
\pgfpathlineto{\pgfqpoint{0.871948in}{0.739656in}}%
\pgfpathlineto{\pgfqpoint{0.871652in}{0.739656in}}%
\pgfpathlineto{\pgfqpoint{0.871356in}{0.739656in}}%
\pgfpathlineto{\pgfqpoint{0.871060in}{0.739656in}}%
\pgfpathlineto{\pgfqpoint{0.870764in}{0.739656in}}%
\pgfpathlineto{\pgfqpoint{0.870468in}{0.739656in}}%
\pgfpathlineto{\pgfqpoint{0.870172in}{0.739656in}}%
\pgfpathlineto{\pgfqpoint{0.869876in}{0.739656in}}%
\pgfpathlineto{\pgfqpoint{0.869580in}{0.739656in}}%
\pgfpathlineto{\pgfqpoint{0.869284in}{0.739656in}}%
\pgfpathlineto{\pgfqpoint{0.868988in}{0.739656in}}%
\pgfpathlineto{\pgfqpoint{0.868692in}{0.739656in}}%
\pgfpathlineto{\pgfqpoint{0.868396in}{0.739656in}}%
\pgfpathlineto{\pgfqpoint{0.868100in}{0.739656in}}%
\pgfpathlineto{\pgfqpoint{0.867804in}{0.739656in}}%
\pgfpathlineto{\pgfqpoint{0.867508in}{0.739656in}}%
\pgfpathlineto{\pgfqpoint{0.867212in}{0.739656in}}%
\pgfpathlineto{\pgfqpoint{0.866916in}{0.739656in}}%
\pgfpathlineto{\pgfqpoint{0.866620in}{0.739656in}}%
\pgfpathlineto{\pgfqpoint{0.866324in}{0.739656in}}%
\pgfpathlineto{\pgfqpoint{0.866028in}{0.739656in}}%
\pgfpathlineto{\pgfqpoint{0.865732in}{0.739656in}}%
\pgfpathlineto{\pgfqpoint{0.865436in}{0.739656in}}%
\pgfpathlineto{\pgfqpoint{0.865140in}{0.739656in}}%
\pgfpathlineto{\pgfqpoint{0.864844in}{0.739656in}}%
\pgfpathlineto{\pgfqpoint{0.864548in}{0.739656in}}%
\pgfpathlineto{\pgfqpoint{0.864252in}{0.739656in}}%
\pgfpathlineto{\pgfqpoint{0.863956in}{0.739656in}}%
\pgfpathlineto{\pgfqpoint{0.863660in}{0.739656in}}%
\pgfpathlineto{\pgfqpoint{0.863364in}{0.739656in}}%
\pgfpathlineto{\pgfqpoint{0.863068in}{0.739656in}}%
\pgfpathlineto{\pgfqpoint{0.862772in}{0.739656in}}%
\pgfpathlineto{\pgfqpoint{0.862476in}{0.739656in}}%
\pgfpathlineto{\pgfqpoint{0.862180in}{0.739656in}}%
\pgfpathlineto{\pgfqpoint{0.861884in}{0.739656in}}%
\pgfpathlineto{\pgfqpoint{0.861588in}{0.739656in}}%
\pgfpathlineto{\pgfqpoint{0.861292in}{0.739656in}}%
\pgfpathlineto{\pgfqpoint{0.860996in}{0.739656in}}%
\pgfpathlineto{\pgfqpoint{0.860700in}{0.739656in}}%
\pgfpathlineto{\pgfqpoint{0.860404in}{0.739656in}}%
\pgfpathlineto{\pgfqpoint{0.860108in}{0.739656in}}%
\pgfpathlineto{\pgfqpoint{0.859812in}{0.739656in}}%
\pgfpathlineto{\pgfqpoint{0.859516in}{0.739656in}}%
\pgfpathlineto{\pgfqpoint{0.859220in}{0.739656in}}%
\pgfpathlineto{\pgfqpoint{0.858924in}{0.739656in}}%
\pgfpathlineto{\pgfqpoint{0.858628in}{0.739656in}}%
\pgfpathlineto{\pgfqpoint{0.858332in}{0.739656in}}%
\pgfpathlineto{\pgfqpoint{0.858036in}{0.739656in}}%
\pgfpathlineto{\pgfqpoint{0.857740in}{0.739656in}}%
\pgfpathlineto{\pgfqpoint{0.857444in}{0.739656in}}%
\pgfpathlineto{\pgfqpoint{0.857148in}{0.739656in}}%
\pgfpathlineto{\pgfqpoint{0.856852in}{0.739656in}}%
\pgfpathlineto{\pgfqpoint{0.856556in}{0.739656in}}%
\pgfpathlineto{\pgfqpoint{0.856260in}{0.739656in}}%
\pgfpathlineto{\pgfqpoint{0.855964in}{0.739656in}}%
\pgfpathlineto{\pgfqpoint{0.855668in}{0.739656in}}%
\pgfpathlineto{\pgfqpoint{0.855372in}{0.739656in}}%
\pgfpathlineto{\pgfqpoint{0.855076in}{0.739656in}}%
\pgfpathlineto{\pgfqpoint{0.854780in}{0.739656in}}%
\pgfpathlineto{\pgfqpoint{0.854484in}{0.739656in}}%
\pgfpathlineto{\pgfqpoint{0.854188in}{0.739656in}}%
\pgfpathlineto{\pgfqpoint{0.853892in}{0.739656in}}%
\pgfpathlineto{\pgfqpoint{0.853596in}{0.739656in}}%
\pgfpathlineto{\pgfqpoint{0.853300in}{0.739656in}}%
\pgfpathlineto{\pgfqpoint{0.853004in}{0.739656in}}%
\pgfpathlineto{\pgfqpoint{0.852708in}{0.739656in}}%
\pgfpathlineto{\pgfqpoint{0.852412in}{0.739656in}}%
\pgfpathlineto{\pgfqpoint{0.852116in}{0.739656in}}%
\pgfpathlineto{\pgfqpoint{0.851820in}{0.739656in}}%
\pgfpathlineto{\pgfqpoint{0.851524in}{0.739656in}}%
\pgfpathlineto{\pgfqpoint{0.851228in}{0.739656in}}%
\pgfpathlineto{\pgfqpoint{0.850932in}{0.739656in}}%
\pgfpathlineto{\pgfqpoint{0.850636in}{0.739656in}}%
\pgfpathlineto{\pgfqpoint{0.850340in}{0.739656in}}%
\pgfpathlineto{\pgfqpoint{0.850043in}{0.739656in}}%
\pgfpathlineto{\pgfqpoint{0.849747in}{0.739656in}}%
\pgfpathlineto{\pgfqpoint{0.849451in}{0.739656in}}%
\pgfpathlineto{\pgfqpoint{0.849155in}{0.739656in}}%
\pgfpathlineto{\pgfqpoint{0.848859in}{0.739656in}}%
\pgfpathlineto{\pgfqpoint{0.848563in}{0.739656in}}%
\pgfpathlineto{\pgfqpoint{0.848267in}{0.739656in}}%
\pgfpathlineto{\pgfqpoint{0.847971in}{0.739656in}}%
\pgfpathlineto{\pgfqpoint{0.847675in}{0.739656in}}%
\pgfpathlineto{\pgfqpoint{0.847379in}{0.739656in}}%
\pgfpathlineto{\pgfqpoint{0.847083in}{0.739656in}}%
\pgfpathlineto{\pgfqpoint{0.846787in}{0.739656in}}%
\pgfpathlineto{\pgfqpoint{0.846491in}{0.739656in}}%
\pgfpathlineto{\pgfqpoint{0.846195in}{0.739656in}}%
\pgfpathlineto{\pgfqpoint{0.845899in}{0.739656in}}%
\pgfpathlineto{\pgfqpoint{0.845603in}{0.739656in}}%
\pgfpathlineto{\pgfqpoint{0.845307in}{0.739656in}}%
\pgfpathlineto{\pgfqpoint{0.845011in}{0.739656in}}%
\pgfpathlineto{\pgfqpoint{0.844715in}{0.739656in}}%
\pgfpathlineto{\pgfqpoint{0.844419in}{0.739656in}}%
\pgfpathlineto{\pgfqpoint{0.844123in}{0.739656in}}%
\pgfpathlineto{\pgfqpoint{0.843827in}{0.739656in}}%
\pgfpathlineto{\pgfqpoint{0.843531in}{0.739656in}}%
\pgfpathlineto{\pgfqpoint{0.843235in}{0.739656in}}%
\pgfpathlineto{\pgfqpoint{0.842939in}{0.739656in}}%
\pgfpathlineto{\pgfqpoint{0.842643in}{0.739656in}}%
\pgfpathlineto{\pgfqpoint{0.842347in}{0.739656in}}%
\pgfpathlineto{\pgfqpoint{0.842051in}{0.739656in}}%
\pgfpathlineto{\pgfqpoint{0.841755in}{0.739656in}}%
\pgfpathlineto{\pgfqpoint{0.841459in}{0.739656in}}%
\pgfpathlineto{\pgfqpoint{0.841163in}{0.739656in}}%
\pgfpathlineto{\pgfqpoint{0.840867in}{0.739656in}}%
\pgfpathlineto{\pgfqpoint{0.840571in}{0.739656in}}%
\pgfpathlineto{\pgfqpoint{0.840275in}{0.739656in}}%
\pgfpathlineto{\pgfqpoint{0.839979in}{0.739656in}}%
\pgfpathlineto{\pgfqpoint{0.839683in}{0.739656in}}%
\pgfpathlineto{\pgfqpoint{0.839387in}{0.739656in}}%
\pgfpathlineto{\pgfqpoint{0.839091in}{0.739656in}}%
\pgfpathlineto{\pgfqpoint{0.838795in}{0.739656in}}%
\pgfpathlineto{\pgfqpoint{0.838499in}{0.739656in}}%
\pgfpathlineto{\pgfqpoint{0.838203in}{0.739656in}}%
\pgfpathlineto{\pgfqpoint{0.837907in}{0.739656in}}%
\pgfpathlineto{\pgfqpoint{0.837611in}{0.739656in}}%
\pgfpathlineto{\pgfqpoint{0.837315in}{0.739656in}}%
\pgfpathlineto{\pgfqpoint{0.837019in}{0.739656in}}%
\pgfpathlineto{\pgfqpoint{0.836723in}{0.739656in}}%
\pgfpathlineto{\pgfqpoint{0.836427in}{0.739656in}}%
\pgfpathlineto{\pgfqpoint{0.836131in}{0.739656in}}%
\pgfpathlineto{\pgfqpoint{0.835835in}{0.739656in}}%
\pgfpathlineto{\pgfqpoint{0.835539in}{0.739656in}}%
\pgfpathlineto{\pgfqpoint{0.835243in}{0.739656in}}%
\pgfpathlineto{\pgfqpoint{0.834947in}{0.739656in}}%
\pgfpathlineto{\pgfqpoint{0.834651in}{0.739656in}}%
\pgfpathlineto{\pgfqpoint{0.834355in}{0.739656in}}%
\pgfpathlineto{\pgfqpoint{0.834059in}{0.739656in}}%
\pgfpathlineto{\pgfqpoint{0.833763in}{0.739656in}}%
\pgfpathlineto{\pgfqpoint{0.833467in}{0.739656in}}%
\pgfpathlineto{\pgfqpoint{0.833171in}{0.739656in}}%
\pgfpathlineto{\pgfqpoint{0.832875in}{0.739656in}}%
\pgfpathlineto{\pgfqpoint{0.832579in}{0.739656in}}%
\pgfpathlineto{\pgfqpoint{0.832283in}{0.739656in}}%
\pgfpathlineto{\pgfqpoint{0.831987in}{0.739656in}}%
\pgfpathlineto{\pgfqpoint{0.831691in}{0.739656in}}%
\pgfpathlineto{\pgfqpoint{0.831395in}{0.739656in}}%
\pgfpathlineto{\pgfqpoint{0.831099in}{0.739656in}}%
\pgfpathlineto{\pgfqpoint{0.830803in}{0.739656in}}%
\pgfpathlineto{\pgfqpoint{0.830507in}{0.739656in}}%
\pgfpathlineto{\pgfqpoint{0.830211in}{0.739656in}}%
\pgfpathlineto{\pgfqpoint{0.829915in}{0.739656in}}%
\pgfpathlineto{\pgfqpoint{0.829619in}{0.739656in}}%
\pgfpathlineto{\pgfqpoint{0.829323in}{0.739656in}}%
\pgfpathlineto{\pgfqpoint{0.829027in}{0.739656in}}%
\pgfpathlineto{\pgfqpoint{0.828731in}{0.739656in}}%
\pgfpathlineto{\pgfqpoint{0.828435in}{0.739656in}}%
\pgfpathlineto{\pgfqpoint{0.828139in}{0.739656in}}%
\pgfpathlineto{\pgfqpoint{0.827843in}{0.739656in}}%
\pgfpathlineto{\pgfqpoint{0.827547in}{0.739656in}}%
\pgfpathlineto{\pgfqpoint{0.827251in}{0.739656in}}%
\pgfpathlineto{\pgfqpoint{0.826955in}{0.739656in}}%
\pgfpathlineto{\pgfqpoint{0.826659in}{0.739656in}}%
\pgfpathlineto{\pgfqpoint{0.826363in}{0.739656in}}%
\pgfpathlineto{\pgfqpoint{0.826067in}{0.739656in}}%
\pgfpathlineto{\pgfqpoint{0.825771in}{0.739656in}}%
\pgfpathlineto{\pgfqpoint{0.825475in}{0.739656in}}%
\pgfpathlineto{\pgfqpoint{0.825179in}{0.739656in}}%
\pgfpathlineto{\pgfqpoint{0.824883in}{0.739656in}}%
\pgfpathlineto{\pgfqpoint{0.824587in}{0.739656in}}%
\pgfpathlineto{\pgfqpoint{0.824291in}{0.739656in}}%
\pgfpathlineto{\pgfqpoint{0.823995in}{0.739656in}}%
\pgfpathlineto{\pgfqpoint{0.823699in}{0.739656in}}%
\pgfpathlineto{\pgfqpoint{0.823403in}{0.739656in}}%
\pgfpathlineto{\pgfqpoint{0.823107in}{0.739656in}}%
\pgfpathlineto{\pgfqpoint{0.822811in}{0.739656in}}%
\pgfpathlineto{\pgfqpoint{0.822515in}{0.739656in}}%
\pgfpathlineto{\pgfqpoint{0.822219in}{0.739656in}}%
\pgfpathlineto{\pgfqpoint{0.821923in}{0.739656in}}%
\pgfpathlineto{\pgfqpoint{0.821627in}{0.739656in}}%
\pgfpathlineto{\pgfqpoint{0.821331in}{0.739656in}}%
\pgfpathlineto{\pgfqpoint{0.821035in}{0.739656in}}%
\pgfpathlineto{\pgfqpoint{0.820739in}{0.739656in}}%
\pgfpathlineto{\pgfqpoint{0.820443in}{0.739656in}}%
\pgfpathlineto{\pgfqpoint{0.820147in}{0.739656in}}%
\pgfpathlineto{\pgfqpoint{0.819851in}{0.739656in}}%
\pgfpathlineto{\pgfqpoint{0.819555in}{0.739656in}}%
\pgfpathlineto{\pgfqpoint{0.819259in}{0.739656in}}%
\pgfpathlineto{\pgfqpoint{0.818963in}{0.739656in}}%
\pgfpathlineto{\pgfqpoint{0.818667in}{0.739656in}}%
\pgfpathlineto{\pgfqpoint{0.818371in}{0.739656in}}%
\pgfpathlineto{\pgfqpoint{0.818075in}{0.739656in}}%
\pgfpathlineto{\pgfqpoint{0.817779in}{0.739656in}}%
\pgfpathlineto{\pgfqpoint{0.817483in}{0.739656in}}%
\pgfpathlineto{\pgfqpoint{0.817187in}{0.739656in}}%
\pgfpathlineto{\pgfqpoint{0.816891in}{0.739656in}}%
\pgfpathlineto{\pgfqpoint{0.816595in}{0.739656in}}%
\pgfpathlineto{\pgfqpoint{0.816299in}{0.739656in}}%
\pgfpathlineto{\pgfqpoint{0.816003in}{0.739656in}}%
\pgfpathlineto{\pgfqpoint{0.815707in}{0.739656in}}%
\pgfpathlineto{\pgfqpoint{0.815411in}{0.739656in}}%
\pgfpathlineto{\pgfqpoint{0.815115in}{0.739656in}}%
\pgfpathlineto{\pgfqpoint{0.814819in}{0.739656in}}%
\pgfpathlineto{\pgfqpoint{0.814523in}{0.739656in}}%
\pgfpathlineto{\pgfqpoint{0.814227in}{0.739656in}}%
\pgfpathlineto{\pgfqpoint{0.813931in}{0.739656in}}%
\pgfpathlineto{\pgfqpoint{0.813635in}{0.739656in}}%
\pgfpathlineto{\pgfqpoint{0.813339in}{0.739656in}}%
\pgfpathlineto{\pgfqpoint{0.813043in}{0.739656in}}%
\pgfpathlineto{\pgfqpoint{0.812747in}{0.739656in}}%
\pgfpathlineto{\pgfqpoint{0.812451in}{0.739656in}}%
\pgfpathlineto{\pgfqpoint{0.812155in}{0.739656in}}%
\pgfpathlineto{\pgfqpoint{0.811859in}{0.739656in}}%
\pgfpathlineto{\pgfqpoint{0.811563in}{0.739656in}}%
\pgfpathlineto{\pgfqpoint{0.811267in}{0.739656in}}%
\pgfpathlineto{\pgfqpoint{0.810971in}{0.739656in}}%
\pgfpathlineto{\pgfqpoint{0.810675in}{0.739656in}}%
\pgfpathlineto{\pgfqpoint{0.810379in}{0.739656in}}%
\pgfpathlineto{\pgfqpoint{0.810083in}{0.739656in}}%
\pgfpathlineto{\pgfqpoint{0.809787in}{0.739656in}}%
\pgfpathlineto{\pgfqpoint{0.809491in}{0.739656in}}%
\pgfpathlineto{\pgfqpoint{0.809195in}{0.739656in}}%
\pgfpathlineto{\pgfqpoint{0.808899in}{0.739656in}}%
\pgfpathlineto{\pgfqpoint{0.808603in}{0.739656in}}%
\pgfpathlineto{\pgfqpoint{0.808307in}{0.739656in}}%
\pgfpathlineto{\pgfqpoint{0.808011in}{0.739656in}}%
\pgfpathlineto{\pgfqpoint{0.807715in}{0.739656in}}%
\pgfpathlineto{\pgfqpoint{0.807419in}{0.739656in}}%
\pgfpathlineto{\pgfqpoint{0.807123in}{0.739656in}}%
\pgfpathlineto{\pgfqpoint{0.806827in}{0.739656in}}%
\pgfpathlineto{\pgfqpoint{0.806531in}{0.739656in}}%
\pgfpathlineto{\pgfqpoint{0.806235in}{0.739656in}}%
\pgfpathlineto{\pgfqpoint{0.805939in}{0.739656in}}%
\pgfpathlineto{\pgfqpoint{0.805643in}{0.739656in}}%
\pgfpathlineto{\pgfqpoint{0.805347in}{0.739656in}}%
\pgfpathlineto{\pgfqpoint{0.805051in}{0.739656in}}%
\pgfpathlineto{\pgfqpoint{0.804755in}{0.739656in}}%
\pgfpathlineto{\pgfqpoint{0.804459in}{0.739656in}}%
\pgfpathlineto{\pgfqpoint{0.804163in}{0.739656in}}%
\pgfpathlineto{\pgfqpoint{0.803867in}{0.739656in}}%
\pgfpathlineto{\pgfqpoint{0.803571in}{0.739656in}}%
\pgfpathlineto{\pgfqpoint{0.803275in}{0.739656in}}%
\pgfpathlineto{\pgfqpoint{0.802979in}{0.739656in}}%
\pgfpathlineto{\pgfqpoint{0.802683in}{0.739656in}}%
\pgfpathlineto{\pgfqpoint{0.802387in}{0.739656in}}%
\pgfpathlineto{\pgfqpoint{0.802091in}{0.739656in}}%
\pgfpathlineto{\pgfqpoint{0.801795in}{0.739656in}}%
\pgfpathlineto{\pgfqpoint{0.801499in}{0.739656in}}%
\pgfpathlineto{\pgfqpoint{0.801203in}{0.739656in}}%
\pgfpathlineto{\pgfqpoint{0.800907in}{0.739656in}}%
\pgfpathlineto{\pgfqpoint{0.800611in}{0.739656in}}%
\pgfpathlineto{\pgfqpoint{0.800315in}{0.739656in}}%
\pgfpathlineto{\pgfqpoint{0.800019in}{0.739656in}}%
\pgfpathlineto{\pgfqpoint{0.799723in}{0.739656in}}%
\pgfpathlineto{\pgfqpoint{0.799427in}{0.739656in}}%
\pgfpathlineto{\pgfqpoint{0.799131in}{0.739656in}}%
\pgfpathlineto{\pgfqpoint{0.798835in}{0.739656in}}%
\pgfpathlineto{\pgfqpoint{0.798539in}{0.739656in}}%
\pgfpathlineto{\pgfqpoint{0.798243in}{0.739656in}}%
\pgfpathlineto{\pgfqpoint{0.797947in}{0.739656in}}%
\pgfpathlineto{\pgfqpoint{0.797651in}{0.739656in}}%
\pgfpathlineto{\pgfqpoint{0.797355in}{0.739656in}}%
\pgfpathlineto{\pgfqpoint{0.797059in}{0.739656in}}%
\pgfpathlineto{\pgfqpoint{0.796763in}{0.739656in}}%
\pgfpathlineto{\pgfqpoint{0.796467in}{0.739656in}}%
\pgfpathlineto{\pgfqpoint{0.796171in}{0.739656in}}%
\pgfpathlineto{\pgfqpoint{0.795875in}{0.739656in}}%
\pgfpathlineto{\pgfqpoint{0.795579in}{0.739656in}}%
\pgfpathlineto{\pgfqpoint{0.795283in}{0.739656in}}%
\pgfpathlineto{\pgfqpoint{0.794987in}{0.739656in}}%
\pgfpathlineto{\pgfqpoint{0.794691in}{0.739656in}}%
\pgfpathlineto{\pgfqpoint{0.794395in}{0.739656in}}%
\pgfpathlineto{\pgfqpoint{0.794099in}{0.739656in}}%
\pgfpathlineto{\pgfqpoint{0.793803in}{0.739656in}}%
\pgfpathlineto{\pgfqpoint{0.793507in}{0.739656in}}%
\pgfpathlineto{\pgfqpoint{0.793211in}{0.739656in}}%
\pgfpathlineto{\pgfqpoint{0.792915in}{0.739656in}}%
\pgfpathlineto{\pgfqpoint{0.792619in}{0.739656in}}%
\pgfpathlineto{\pgfqpoint{0.792323in}{0.739656in}}%
\pgfpathlineto{\pgfqpoint{0.792027in}{0.739656in}}%
\pgfpathlineto{\pgfqpoint{0.791731in}{0.739656in}}%
\pgfpathlineto{\pgfqpoint{0.791435in}{0.739656in}}%
\pgfpathlineto{\pgfqpoint{0.791139in}{0.739656in}}%
\pgfpathlineto{\pgfqpoint{0.790843in}{0.739656in}}%
\pgfpathlineto{\pgfqpoint{0.790547in}{0.739656in}}%
\pgfpathlineto{\pgfqpoint{0.790251in}{0.739656in}}%
\pgfpathlineto{\pgfqpoint{0.789955in}{0.739656in}}%
\pgfpathlineto{\pgfqpoint{0.789659in}{0.739656in}}%
\pgfpathlineto{\pgfqpoint{0.789363in}{0.739656in}}%
\pgfpathlineto{\pgfqpoint{0.789067in}{0.739656in}}%
\pgfpathlineto{\pgfqpoint{0.788771in}{0.739656in}}%
\pgfpathlineto{\pgfqpoint{0.788475in}{0.739656in}}%
\pgfpathlineto{\pgfqpoint{0.788179in}{0.739656in}}%
\pgfpathlineto{\pgfqpoint{0.787883in}{0.739656in}}%
\pgfpathlineto{\pgfqpoint{0.787587in}{0.739656in}}%
\pgfpathlineto{\pgfqpoint{0.787291in}{0.739656in}}%
\pgfpathlineto{\pgfqpoint{0.786995in}{0.739656in}}%
\pgfpathlineto{\pgfqpoint{0.786699in}{0.739656in}}%
\pgfpathlineto{\pgfqpoint{0.786403in}{0.739656in}}%
\pgfpathlineto{\pgfqpoint{0.786107in}{0.739656in}}%
\pgfpathlineto{\pgfqpoint{0.785811in}{0.739656in}}%
\pgfpathlineto{\pgfqpoint{0.785515in}{0.739656in}}%
\pgfpathlineto{\pgfqpoint{0.785219in}{0.739656in}}%
\pgfpathlineto{\pgfqpoint{0.784923in}{0.739656in}}%
\pgfpathlineto{\pgfqpoint{0.784627in}{0.739656in}}%
\pgfpathlineto{\pgfqpoint{0.784331in}{0.739656in}}%
\pgfpathlineto{\pgfqpoint{0.784035in}{0.739656in}}%
\pgfpathlineto{\pgfqpoint{0.783739in}{0.739656in}}%
\pgfpathlineto{\pgfqpoint{0.783443in}{0.739656in}}%
\pgfpathlineto{\pgfqpoint{0.783147in}{0.739656in}}%
\pgfpathlineto{\pgfqpoint{0.782851in}{0.739656in}}%
\pgfpathlineto{\pgfqpoint{0.782554in}{0.739656in}}%
\pgfpathlineto{\pgfqpoint{0.782258in}{0.739656in}}%
\pgfpathlineto{\pgfqpoint{0.781962in}{0.739656in}}%
\pgfpathlineto{\pgfqpoint{0.781666in}{0.739656in}}%
\pgfpathlineto{\pgfqpoint{0.781370in}{0.739656in}}%
\pgfpathlineto{\pgfqpoint{0.781074in}{0.739656in}}%
\pgfpathlineto{\pgfqpoint{0.780778in}{0.739656in}}%
\pgfpathlineto{\pgfqpoint{0.780482in}{0.739656in}}%
\pgfpathlineto{\pgfqpoint{0.780186in}{0.739656in}}%
\pgfpathlineto{\pgfqpoint{0.779890in}{0.739656in}}%
\pgfpathlineto{\pgfqpoint{0.779594in}{0.739656in}}%
\pgfpathlineto{\pgfqpoint{0.779298in}{0.739656in}}%
\pgfpathlineto{\pgfqpoint{0.779002in}{0.739656in}}%
\pgfpathlineto{\pgfqpoint{0.778706in}{0.739656in}}%
\pgfpathlineto{\pgfqpoint{0.778410in}{0.739656in}}%
\pgfpathlineto{\pgfqpoint{0.778114in}{0.739656in}}%
\pgfpathlineto{\pgfqpoint{0.777818in}{0.739656in}}%
\pgfpathlineto{\pgfqpoint{0.777522in}{0.739656in}}%
\pgfpathlineto{\pgfqpoint{0.777226in}{0.739656in}}%
\pgfpathlineto{\pgfqpoint{0.776930in}{0.739656in}}%
\pgfpathlineto{\pgfqpoint{0.776634in}{0.739656in}}%
\pgfpathlineto{\pgfqpoint{0.776338in}{0.739656in}}%
\pgfpathlineto{\pgfqpoint{0.776042in}{0.739656in}}%
\pgfpathlineto{\pgfqpoint{0.775746in}{0.739656in}}%
\pgfpathlineto{\pgfqpoint{0.775450in}{0.739656in}}%
\pgfpathlineto{\pgfqpoint{0.775154in}{0.739656in}}%
\pgfpathlineto{\pgfqpoint{0.774858in}{0.739656in}}%
\pgfpathlineto{\pgfqpoint{0.774562in}{0.739656in}}%
\pgfpathlineto{\pgfqpoint{0.774266in}{0.739656in}}%
\pgfpathlineto{\pgfqpoint{0.773970in}{0.739656in}}%
\pgfpathlineto{\pgfqpoint{0.773674in}{0.739656in}}%
\pgfpathlineto{\pgfqpoint{0.773378in}{0.739656in}}%
\pgfpathlineto{\pgfqpoint{0.773082in}{0.739656in}}%
\pgfpathlineto{\pgfqpoint{0.772786in}{0.739656in}}%
\pgfpathlineto{\pgfqpoint{0.772490in}{0.739656in}}%
\pgfpathlineto{\pgfqpoint{0.772194in}{0.739656in}}%
\pgfpathlineto{\pgfqpoint{0.771898in}{0.739656in}}%
\pgfpathlineto{\pgfqpoint{0.771602in}{0.739656in}}%
\pgfpathlineto{\pgfqpoint{0.771306in}{0.739656in}}%
\pgfpathlineto{\pgfqpoint{0.771010in}{0.739656in}}%
\pgfpathlineto{\pgfqpoint{0.770714in}{0.739656in}}%
\pgfpathlineto{\pgfqpoint{0.770418in}{0.739656in}}%
\pgfpathlineto{\pgfqpoint{0.770122in}{0.739656in}}%
\pgfpathlineto{\pgfqpoint{0.769826in}{0.739656in}}%
\pgfpathlineto{\pgfqpoint{0.769530in}{0.739656in}}%
\pgfpathlineto{\pgfqpoint{0.769234in}{0.739656in}}%
\pgfpathlineto{\pgfqpoint{0.768938in}{0.739656in}}%
\pgfpathlineto{\pgfqpoint{0.768642in}{0.739656in}}%
\pgfpathlineto{\pgfqpoint{0.768346in}{0.739656in}}%
\pgfpathlineto{\pgfqpoint{0.768050in}{0.739656in}}%
\pgfpathlineto{\pgfqpoint{0.767754in}{0.739656in}}%
\pgfpathlineto{\pgfqpoint{0.767458in}{0.739656in}}%
\pgfpathlineto{\pgfqpoint{0.767162in}{0.739656in}}%
\pgfpathlineto{\pgfqpoint{0.766866in}{0.739656in}}%
\pgfpathlineto{\pgfqpoint{0.766570in}{0.739656in}}%
\pgfpathlineto{\pgfqpoint{0.766274in}{0.739656in}}%
\pgfpathlineto{\pgfqpoint{0.765978in}{0.739656in}}%
\pgfpathlineto{\pgfqpoint{0.765682in}{0.739656in}}%
\pgfpathlineto{\pgfqpoint{0.765386in}{0.739656in}}%
\pgfpathlineto{\pgfqpoint{0.765090in}{0.739656in}}%
\pgfpathlineto{\pgfqpoint{0.764794in}{0.739656in}}%
\pgfpathlineto{\pgfqpoint{0.764498in}{0.739656in}}%
\pgfpathlineto{\pgfqpoint{0.764202in}{0.739656in}}%
\pgfpathlineto{\pgfqpoint{0.763906in}{0.739656in}}%
\pgfpathlineto{\pgfqpoint{0.763610in}{0.739656in}}%
\pgfpathlineto{\pgfqpoint{0.763314in}{0.739656in}}%
\pgfpathlineto{\pgfqpoint{0.763018in}{0.739656in}}%
\pgfpathlineto{\pgfqpoint{0.762722in}{0.739656in}}%
\pgfpathlineto{\pgfqpoint{0.762426in}{0.739656in}}%
\pgfpathlineto{\pgfqpoint{0.762130in}{0.739656in}}%
\pgfpathlineto{\pgfqpoint{0.761834in}{0.739656in}}%
\pgfpathlineto{\pgfqpoint{0.761538in}{0.739656in}}%
\pgfpathlineto{\pgfqpoint{0.761242in}{0.739656in}}%
\pgfpathlineto{\pgfqpoint{0.760946in}{0.739656in}}%
\pgfpathlineto{\pgfqpoint{0.760650in}{0.739656in}}%
\pgfpathlineto{\pgfqpoint{0.760354in}{0.739656in}}%
\pgfpathlineto{\pgfqpoint{0.760058in}{0.739656in}}%
\pgfpathlineto{\pgfqpoint{0.759762in}{0.739656in}}%
\pgfpathlineto{\pgfqpoint{0.759466in}{0.739656in}}%
\pgfpathlineto{\pgfqpoint{0.759170in}{0.739656in}}%
\pgfpathlineto{\pgfqpoint{0.758874in}{0.739656in}}%
\pgfpathlineto{\pgfqpoint{0.758578in}{0.739656in}}%
\pgfpathlineto{\pgfqpoint{0.758282in}{0.739656in}}%
\pgfpathlineto{\pgfqpoint{0.757986in}{0.739656in}}%
\pgfpathlineto{\pgfqpoint{0.757690in}{0.739656in}}%
\pgfpathlineto{\pgfqpoint{0.757394in}{0.739656in}}%
\pgfpathlineto{\pgfqpoint{0.757098in}{0.739656in}}%
\pgfpathlineto{\pgfqpoint{0.756802in}{0.739656in}}%
\pgfpathlineto{\pgfqpoint{0.756506in}{0.739656in}}%
\pgfpathlineto{\pgfqpoint{0.756210in}{0.739656in}}%
\pgfpathlineto{\pgfqpoint{0.755914in}{0.739656in}}%
\pgfpathlineto{\pgfqpoint{0.755618in}{0.739656in}}%
\pgfpathlineto{\pgfqpoint{0.755322in}{0.739656in}}%
\pgfpathlineto{\pgfqpoint{0.755026in}{0.739656in}}%
\pgfpathlineto{\pgfqpoint{0.754730in}{0.739656in}}%
\pgfpathlineto{\pgfqpoint{0.754434in}{0.739656in}}%
\pgfpathlineto{\pgfqpoint{0.754138in}{0.739656in}}%
\pgfpathlineto{\pgfqpoint{0.753842in}{0.739656in}}%
\pgfpathlineto{\pgfqpoint{0.753546in}{0.739656in}}%
\pgfpathlineto{\pgfqpoint{0.753250in}{0.739656in}}%
\pgfpathlineto{\pgfqpoint{0.752954in}{0.739656in}}%
\pgfpathlineto{\pgfqpoint{0.752658in}{0.739656in}}%
\pgfpathlineto{\pgfqpoint{0.752362in}{0.739656in}}%
\pgfpathlineto{\pgfqpoint{0.752066in}{0.739656in}}%
\pgfpathlineto{\pgfqpoint{0.751770in}{0.739656in}}%
\pgfpathlineto{\pgfqpoint{0.751474in}{0.739656in}}%
\pgfpathlineto{\pgfqpoint{0.751178in}{0.739656in}}%
\pgfpathlineto{\pgfqpoint{0.750882in}{0.739656in}}%
\pgfpathlineto{\pgfqpoint{0.750586in}{0.739656in}}%
\pgfpathlineto{\pgfqpoint{0.750290in}{0.739656in}}%
\pgfpathlineto{\pgfqpoint{0.749994in}{0.739656in}}%
\pgfpathlineto{\pgfqpoint{0.749698in}{0.739656in}}%
\pgfpathlineto{\pgfqpoint{0.749402in}{0.739656in}}%
\pgfpathlineto{\pgfqpoint{0.749106in}{0.739656in}}%
\pgfpathlineto{\pgfqpoint{0.748810in}{0.739656in}}%
\pgfpathlineto{\pgfqpoint{0.748514in}{0.739656in}}%
\pgfpathlineto{\pgfqpoint{0.748218in}{0.739656in}}%
\pgfpathlineto{\pgfqpoint{0.747922in}{0.739656in}}%
\pgfpathlineto{\pgfqpoint{0.747626in}{0.739656in}}%
\pgfpathlineto{\pgfqpoint{0.747330in}{0.739656in}}%
\pgfpathlineto{\pgfqpoint{0.747034in}{0.739656in}}%
\pgfpathlineto{\pgfqpoint{0.746738in}{0.739656in}}%
\pgfpathlineto{\pgfqpoint{0.746442in}{0.739656in}}%
\pgfpathlineto{\pgfqpoint{0.746146in}{0.739656in}}%
\pgfpathlineto{\pgfqpoint{0.745850in}{0.739656in}}%
\pgfpathlineto{\pgfqpoint{0.745554in}{0.739656in}}%
\pgfpathlineto{\pgfqpoint{0.745258in}{0.739656in}}%
\pgfpathlineto{\pgfqpoint{0.744962in}{0.739656in}}%
\pgfpathlineto{\pgfqpoint{0.744666in}{0.739656in}}%
\pgfpathlineto{\pgfqpoint{0.744666in}{0.739656in}}%
\pgfpathclose%
\pgfusepath{fill}%
\end{pgfscope}%
\begin{pgfscope}%
\pgfsetbuttcap%
\pgfsetroundjoin%
\definecolor{currentfill}{rgb}{0.000000,0.000000,0.000000}%
\pgfsetfillcolor{currentfill}%
\pgfsetlinewidth{0.501875pt}%
\definecolor{currentstroke}{rgb}{0.000000,0.000000,0.000000}%
\pgfsetstrokecolor{currentstroke}%
\pgfsetdash{}{0pt}%
\pgfsys@defobject{currentmarker}{\pgfqpoint{0.000000in}{0.000000in}}{\pgfqpoint{0.000000in}{0.041667in}}{%
\pgfpathmoveto{\pgfqpoint{0.000000in}{0.000000in}}%
\pgfpathlineto{\pgfqpoint{0.000000in}{0.041667in}}%
\pgfusepath{stroke,fill}%
}%
\begin{pgfscope}%
\pgfsys@transformshift{0.643728in}{0.586309in}%
\pgfsys@useobject{currentmarker}{}%
\end{pgfscope}%
\end{pgfscope}%
\begin{pgfscope}%
\pgfsetbuttcap%
\pgfsetroundjoin%
\definecolor{currentfill}{rgb}{0.000000,0.000000,0.000000}%
\pgfsetfillcolor{currentfill}%
\pgfsetlinewidth{0.501875pt}%
\definecolor{currentstroke}{rgb}{0.000000,0.000000,0.000000}%
\pgfsetstrokecolor{currentstroke}%
\pgfsetdash{}{0pt}%
\pgfsys@defobject{currentmarker}{\pgfqpoint{0.000000in}{-0.041667in}}{\pgfqpoint{0.000000in}{0.000000in}}{%
\pgfpathmoveto{\pgfqpoint{0.000000in}{0.000000in}}%
\pgfpathlineto{\pgfqpoint{0.000000in}{-0.041667in}}%
\pgfusepath{stroke,fill}%
}%
\begin{pgfscope}%
\pgfsys@transformshift{0.643728in}{0.893003in}%
\pgfsys@useobject{currentmarker}{}%
\end{pgfscope}%
\end{pgfscope}%
\begin{pgfscope}%
\definecolor{textcolor}{rgb}{0.000000,0.000000,0.000000}%
\pgfsetstrokecolor{textcolor}%
\pgfsetfillcolor{textcolor}%
\pgftext[x=0.186202in, y=0.220556in, left, base,rotate=30.000000]{\color{textcolor}\rmfamily\fontsize{7.000000}{8.400000}\selectfont 2020-03-20}%
\end{pgfscope}%
\begin{pgfscope}%
\pgfsetbuttcap%
\pgfsetroundjoin%
\definecolor{currentfill}{rgb}{0.000000,0.000000,0.000000}%
\pgfsetfillcolor{currentfill}%
\pgfsetlinewidth{0.501875pt}%
\definecolor{currentstroke}{rgb}{0.000000,0.000000,0.000000}%
\pgfsetstrokecolor{currentstroke}%
\pgfsetdash{}{0pt}%
\pgfsys@defobject{currentmarker}{\pgfqpoint{0.000000in}{0.000000in}}{\pgfqpoint{0.000000in}{0.041667in}}{%
\pgfpathmoveto{\pgfqpoint{0.000000in}{0.000000in}}%
\pgfpathlineto{\pgfqpoint{0.000000in}{0.041667in}}%
\pgfusepath{stroke,fill}%
}%
\begin{pgfscope}%
\pgfsys@transformshift{1.069975in}{0.586309in}%
\pgfsys@useobject{currentmarker}{}%
\end{pgfscope}%
\end{pgfscope}%
\begin{pgfscope}%
\pgfsetbuttcap%
\pgfsetroundjoin%
\definecolor{currentfill}{rgb}{0.000000,0.000000,0.000000}%
\pgfsetfillcolor{currentfill}%
\pgfsetlinewidth{0.501875pt}%
\definecolor{currentstroke}{rgb}{0.000000,0.000000,0.000000}%
\pgfsetstrokecolor{currentstroke}%
\pgfsetdash{}{0pt}%
\pgfsys@defobject{currentmarker}{\pgfqpoint{0.000000in}{-0.041667in}}{\pgfqpoint{0.000000in}{0.000000in}}{%
\pgfpathmoveto{\pgfqpoint{0.000000in}{0.000000in}}%
\pgfpathlineto{\pgfqpoint{0.000000in}{-0.041667in}}%
\pgfusepath{stroke,fill}%
}%
\begin{pgfscope}%
\pgfsys@transformshift{1.069975in}{0.893003in}%
\pgfsys@useobject{currentmarker}{}%
\end{pgfscope}%
\end{pgfscope}%
\begin{pgfscope}%
\definecolor{textcolor}{rgb}{0.000000,0.000000,0.000000}%
\pgfsetstrokecolor{textcolor}%
\pgfsetfillcolor{textcolor}%
\pgftext[x=0.612448in, y=0.220556in, left, base,rotate=30.000000]{\color{textcolor}\rmfamily\fontsize{7.000000}{8.400000}\selectfont 2020-05-19}%
\end{pgfscope}%
\begin{pgfscope}%
\pgfsetbuttcap%
\pgfsetroundjoin%
\definecolor{currentfill}{rgb}{0.000000,0.000000,0.000000}%
\pgfsetfillcolor{currentfill}%
\pgfsetlinewidth{0.501875pt}%
\definecolor{currentstroke}{rgb}{0.000000,0.000000,0.000000}%
\pgfsetstrokecolor{currentstroke}%
\pgfsetdash{}{0pt}%
\pgfsys@defobject{currentmarker}{\pgfqpoint{0.000000in}{0.000000in}}{\pgfqpoint{0.000000in}{0.041667in}}{%
\pgfpathmoveto{\pgfqpoint{0.000000in}{0.000000in}}%
\pgfpathlineto{\pgfqpoint{0.000000in}{0.041667in}}%
\pgfusepath{stroke,fill}%
}%
\begin{pgfscope}%
\pgfsys@transformshift{1.496221in}{0.586309in}%
\pgfsys@useobject{currentmarker}{}%
\end{pgfscope}%
\end{pgfscope}%
\begin{pgfscope}%
\pgfsetbuttcap%
\pgfsetroundjoin%
\definecolor{currentfill}{rgb}{0.000000,0.000000,0.000000}%
\pgfsetfillcolor{currentfill}%
\pgfsetlinewidth{0.501875pt}%
\definecolor{currentstroke}{rgb}{0.000000,0.000000,0.000000}%
\pgfsetstrokecolor{currentstroke}%
\pgfsetdash{}{0pt}%
\pgfsys@defobject{currentmarker}{\pgfqpoint{0.000000in}{-0.041667in}}{\pgfqpoint{0.000000in}{0.000000in}}{%
\pgfpathmoveto{\pgfqpoint{0.000000in}{0.000000in}}%
\pgfpathlineto{\pgfqpoint{0.000000in}{-0.041667in}}%
\pgfusepath{stroke,fill}%
}%
\begin{pgfscope}%
\pgfsys@transformshift{1.496221in}{0.893003in}%
\pgfsys@useobject{currentmarker}{}%
\end{pgfscope}%
\end{pgfscope}%
\begin{pgfscope}%
\definecolor{textcolor}{rgb}{0.000000,0.000000,0.000000}%
\pgfsetstrokecolor{textcolor}%
\pgfsetfillcolor{textcolor}%
\pgftext[x=1.038694in, y=0.220556in, left, base,rotate=30.000000]{\color{textcolor}\rmfamily\fontsize{7.000000}{8.400000}\selectfont 2020-07-18}%
\end{pgfscope}%
\begin{pgfscope}%
\pgfsetbuttcap%
\pgfsetroundjoin%
\definecolor{currentfill}{rgb}{0.000000,0.000000,0.000000}%
\pgfsetfillcolor{currentfill}%
\pgfsetlinewidth{0.501875pt}%
\definecolor{currentstroke}{rgb}{0.000000,0.000000,0.000000}%
\pgfsetstrokecolor{currentstroke}%
\pgfsetdash{}{0pt}%
\pgfsys@defobject{currentmarker}{\pgfqpoint{0.000000in}{0.000000in}}{\pgfqpoint{0.000000in}{0.041667in}}{%
\pgfpathmoveto{\pgfqpoint{0.000000in}{0.000000in}}%
\pgfpathlineto{\pgfqpoint{0.000000in}{0.041667in}}%
\pgfusepath{stroke,fill}%
}%
\begin{pgfscope}%
\pgfsys@transformshift{1.922467in}{0.586309in}%
\pgfsys@useobject{currentmarker}{}%
\end{pgfscope}%
\end{pgfscope}%
\begin{pgfscope}%
\pgfsetbuttcap%
\pgfsetroundjoin%
\definecolor{currentfill}{rgb}{0.000000,0.000000,0.000000}%
\pgfsetfillcolor{currentfill}%
\pgfsetlinewidth{0.501875pt}%
\definecolor{currentstroke}{rgb}{0.000000,0.000000,0.000000}%
\pgfsetstrokecolor{currentstroke}%
\pgfsetdash{}{0pt}%
\pgfsys@defobject{currentmarker}{\pgfqpoint{0.000000in}{-0.041667in}}{\pgfqpoint{0.000000in}{0.000000in}}{%
\pgfpathmoveto{\pgfqpoint{0.000000in}{0.000000in}}%
\pgfpathlineto{\pgfqpoint{0.000000in}{-0.041667in}}%
\pgfusepath{stroke,fill}%
}%
\begin{pgfscope}%
\pgfsys@transformshift{1.922467in}{0.893003in}%
\pgfsys@useobject{currentmarker}{}%
\end{pgfscope}%
\end{pgfscope}%
\begin{pgfscope}%
\definecolor{textcolor}{rgb}{0.000000,0.000000,0.000000}%
\pgfsetstrokecolor{textcolor}%
\pgfsetfillcolor{textcolor}%
\pgftext[x=1.464941in, y=0.220556in, left, base,rotate=30.000000]{\color{textcolor}\rmfamily\fontsize{7.000000}{8.400000}\selectfont 2020-09-16}%
\end{pgfscope}%
\begin{pgfscope}%
\pgfsetbuttcap%
\pgfsetroundjoin%
\definecolor{currentfill}{rgb}{0.000000,0.000000,0.000000}%
\pgfsetfillcolor{currentfill}%
\pgfsetlinewidth{0.501875pt}%
\definecolor{currentstroke}{rgb}{0.000000,0.000000,0.000000}%
\pgfsetstrokecolor{currentstroke}%
\pgfsetdash{}{0pt}%
\pgfsys@defobject{currentmarker}{\pgfqpoint{0.000000in}{0.000000in}}{\pgfqpoint{0.000000in}{0.041667in}}{%
\pgfpathmoveto{\pgfqpoint{0.000000in}{0.000000in}}%
\pgfpathlineto{\pgfqpoint{0.000000in}{0.041667in}}%
\pgfusepath{stroke,fill}%
}%
\begin{pgfscope}%
\pgfsys@transformshift{2.348714in}{0.586309in}%
\pgfsys@useobject{currentmarker}{}%
\end{pgfscope}%
\end{pgfscope}%
\begin{pgfscope}%
\pgfsetbuttcap%
\pgfsetroundjoin%
\definecolor{currentfill}{rgb}{0.000000,0.000000,0.000000}%
\pgfsetfillcolor{currentfill}%
\pgfsetlinewidth{0.501875pt}%
\definecolor{currentstroke}{rgb}{0.000000,0.000000,0.000000}%
\pgfsetstrokecolor{currentstroke}%
\pgfsetdash{}{0pt}%
\pgfsys@defobject{currentmarker}{\pgfqpoint{0.000000in}{-0.041667in}}{\pgfqpoint{0.000000in}{0.000000in}}{%
\pgfpathmoveto{\pgfqpoint{0.000000in}{0.000000in}}%
\pgfpathlineto{\pgfqpoint{0.000000in}{-0.041667in}}%
\pgfusepath{stroke,fill}%
}%
\begin{pgfscope}%
\pgfsys@transformshift{2.348714in}{0.893003in}%
\pgfsys@useobject{currentmarker}{}%
\end{pgfscope}%
\end{pgfscope}%
\begin{pgfscope}%
\definecolor{textcolor}{rgb}{0.000000,0.000000,0.000000}%
\pgfsetstrokecolor{textcolor}%
\pgfsetfillcolor{textcolor}%
\pgftext[x=1.891187in, y=0.220556in, left, base,rotate=30.000000]{\color{textcolor}\rmfamily\fontsize{7.000000}{8.400000}\selectfont 2020-11-15}%
\end{pgfscope}%
\begin{pgfscope}%
\pgfsetbuttcap%
\pgfsetroundjoin%
\definecolor{currentfill}{rgb}{0.000000,0.000000,0.000000}%
\pgfsetfillcolor{currentfill}%
\pgfsetlinewidth{0.501875pt}%
\definecolor{currentstroke}{rgb}{0.000000,0.000000,0.000000}%
\pgfsetstrokecolor{currentstroke}%
\pgfsetdash{}{0pt}%
\pgfsys@defobject{currentmarker}{\pgfqpoint{0.000000in}{0.000000in}}{\pgfqpoint{0.000000in}{0.041667in}}{%
\pgfpathmoveto{\pgfqpoint{0.000000in}{0.000000in}}%
\pgfpathlineto{\pgfqpoint{0.000000in}{0.041667in}}%
\pgfusepath{stroke,fill}%
}%
\begin{pgfscope}%
\pgfsys@transformshift{2.774960in}{0.586309in}%
\pgfsys@useobject{currentmarker}{}%
\end{pgfscope}%
\end{pgfscope}%
\begin{pgfscope}%
\pgfsetbuttcap%
\pgfsetroundjoin%
\definecolor{currentfill}{rgb}{0.000000,0.000000,0.000000}%
\pgfsetfillcolor{currentfill}%
\pgfsetlinewidth{0.501875pt}%
\definecolor{currentstroke}{rgb}{0.000000,0.000000,0.000000}%
\pgfsetstrokecolor{currentstroke}%
\pgfsetdash{}{0pt}%
\pgfsys@defobject{currentmarker}{\pgfqpoint{0.000000in}{-0.041667in}}{\pgfqpoint{0.000000in}{0.000000in}}{%
\pgfpathmoveto{\pgfqpoint{0.000000in}{0.000000in}}%
\pgfpathlineto{\pgfqpoint{0.000000in}{-0.041667in}}%
\pgfusepath{stroke,fill}%
}%
\begin{pgfscope}%
\pgfsys@transformshift{2.774960in}{0.893003in}%
\pgfsys@useobject{currentmarker}{}%
\end{pgfscope}%
\end{pgfscope}%
\begin{pgfscope}%
\definecolor{textcolor}{rgb}{0.000000,0.000000,0.000000}%
\pgfsetstrokecolor{textcolor}%
\pgfsetfillcolor{textcolor}%
\pgftext[x=2.317433in, y=0.220556in, left, base,rotate=30.000000]{\color{textcolor}\rmfamily\fontsize{7.000000}{8.400000}\selectfont 2021-01-14}%
\end{pgfscope}%
\begin{pgfscope}%
\pgfsetbuttcap%
\pgfsetroundjoin%
\definecolor{currentfill}{rgb}{0.000000,0.000000,0.000000}%
\pgfsetfillcolor{currentfill}%
\pgfsetlinewidth{0.501875pt}%
\definecolor{currentstroke}{rgb}{0.000000,0.000000,0.000000}%
\pgfsetstrokecolor{currentstroke}%
\pgfsetdash{}{0pt}%
\pgfsys@defobject{currentmarker}{\pgfqpoint{0.000000in}{0.000000in}}{\pgfqpoint{0.000000in}{0.041667in}}{%
\pgfpathmoveto{\pgfqpoint{0.000000in}{0.000000in}}%
\pgfpathlineto{\pgfqpoint{0.000000in}{0.041667in}}%
\pgfusepath{stroke,fill}%
}%
\begin{pgfscope}%
\pgfsys@transformshift{3.201206in}{0.586309in}%
\pgfsys@useobject{currentmarker}{}%
\end{pgfscope}%
\end{pgfscope}%
\begin{pgfscope}%
\pgfsetbuttcap%
\pgfsetroundjoin%
\definecolor{currentfill}{rgb}{0.000000,0.000000,0.000000}%
\pgfsetfillcolor{currentfill}%
\pgfsetlinewidth{0.501875pt}%
\definecolor{currentstroke}{rgb}{0.000000,0.000000,0.000000}%
\pgfsetstrokecolor{currentstroke}%
\pgfsetdash{}{0pt}%
\pgfsys@defobject{currentmarker}{\pgfqpoint{0.000000in}{-0.041667in}}{\pgfqpoint{0.000000in}{0.000000in}}{%
\pgfpathmoveto{\pgfqpoint{0.000000in}{0.000000in}}%
\pgfpathlineto{\pgfqpoint{0.000000in}{-0.041667in}}%
\pgfusepath{stroke,fill}%
}%
\begin{pgfscope}%
\pgfsys@transformshift{3.201206in}{0.893003in}%
\pgfsys@useobject{currentmarker}{}%
\end{pgfscope}%
\end{pgfscope}%
\begin{pgfscope}%
\definecolor{textcolor}{rgb}{0.000000,0.000000,0.000000}%
\pgfsetstrokecolor{textcolor}%
\pgfsetfillcolor{textcolor}%
\pgftext[x=2.743679in, y=0.220556in, left, base,rotate=30.000000]{\color{textcolor}\rmfamily\fontsize{7.000000}{8.400000}\selectfont 2021-03-15}%
\end{pgfscope}%
\begin{pgfscope}%
\pgfsetbuttcap%
\pgfsetroundjoin%
\definecolor{currentfill}{rgb}{0.000000,0.000000,0.000000}%
\pgfsetfillcolor{currentfill}%
\pgfsetlinewidth{0.501875pt}%
\definecolor{currentstroke}{rgb}{0.000000,0.000000,0.000000}%
\pgfsetstrokecolor{currentstroke}%
\pgfsetdash{}{0pt}%
\pgfsys@defobject{currentmarker}{\pgfqpoint{0.000000in}{0.000000in}}{\pgfqpoint{0.000000in}{0.041667in}}{%
\pgfpathmoveto{\pgfqpoint{0.000000in}{0.000000in}}%
\pgfpathlineto{\pgfqpoint{0.000000in}{0.041667in}}%
\pgfusepath{stroke,fill}%
}%
\begin{pgfscope}%
\pgfsys@transformshift{3.627453in}{0.586309in}%
\pgfsys@useobject{currentmarker}{}%
\end{pgfscope}%
\end{pgfscope}%
\begin{pgfscope}%
\pgfsetbuttcap%
\pgfsetroundjoin%
\definecolor{currentfill}{rgb}{0.000000,0.000000,0.000000}%
\pgfsetfillcolor{currentfill}%
\pgfsetlinewidth{0.501875pt}%
\definecolor{currentstroke}{rgb}{0.000000,0.000000,0.000000}%
\pgfsetstrokecolor{currentstroke}%
\pgfsetdash{}{0pt}%
\pgfsys@defobject{currentmarker}{\pgfqpoint{0.000000in}{-0.041667in}}{\pgfqpoint{0.000000in}{0.000000in}}{%
\pgfpathmoveto{\pgfqpoint{0.000000in}{0.000000in}}%
\pgfpathlineto{\pgfqpoint{0.000000in}{-0.041667in}}%
\pgfusepath{stroke,fill}%
}%
\begin{pgfscope}%
\pgfsys@transformshift{3.627453in}{0.893003in}%
\pgfsys@useobject{currentmarker}{}%
\end{pgfscope}%
\end{pgfscope}%
\begin{pgfscope}%
\definecolor{textcolor}{rgb}{0.000000,0.000000,0.000000}%
\pgfsetstrokecolor{textcolor}%
\pgfsetfillcolor{textcolor}%
\pgftext[x=3.169926in, y=0.220556in, left, base,rotate=30.000000]{\color{textcolor}\rmfamily\fontsize{7.000000}{8.400000}\selectfont 2021-05-14}%
\end{pgfscope}%
\begin{pgfscope}%
\pgfsetbuttcap%
\pgfsetroundjoin%
\definecolor{currentfill}{rgb}{0.000000,0.000000,0.000000}%
\pgfsetfillcolor{currentfill}%
\pgfsetlinewidth{0.501875pt}%
\definecolor{currentstroke}{rgb}{0.000000,0.000000,0.000000}%
\pgfsetstrokecolor{currentstroke}%
\pgfsetdash{}{0pt}%
\pgfsys@defobject{currentmarker}{\pgfqpoint{0.000000in}{0.000000in}}{\pgfqpoint{0.000000in}{0.041667in}}{%
\pgfpathmoveto{\pgfqpoint{0.000000in}{0.000000in}}%
\pgfpathlineto{\pgfqpoint{0.000000in}{0.041667in}}%
\pgfusepath{stroke,fill}%
}%
\begin{pgfscope}%
\pgfsys@transformshift{4.053699in}{0.586309in}%
\pgfsys@useobject{currentmarker}{}%
\end{pgfscope}%
\end{pgfscope}%
\begin{pgfscope}%
\pgfsetbuttcap%
\pgfsetroundjoin%
\definecolor{currentfill}{rgb}{0.000000,0.000000,0.000000}%
\pgfsetfillcolor{currentfill}%
\pgfsetlinewidth{0.501875pt}%
\definecolor{currentstroke}{rgb}{0.000000,0.000000,0.000000}%
\pgfsetstrokecolor{currentstroke}%
\pgfsetdash{}{0pt}%
\pgfsys@defobject{currentmarker}{\pgfqpoint{0.000000in}{-0.041667in}}{\pgfqpoint{0.000000in}{0.000000in}}{%
\pgfpathmoveto{\pgfqpoint{0.000000in}{0.000000in}}%
\pgfpathlineto{\pgfqpoint{0.000000in}{-0.041667in}}%
\pgfusepath{stroke,fill}%
}%
\begin{pgfscope}%
\pgfsys@transformshift{4.053699in}{0.893003in}%
\pgfsys@useobject{currentmarker}{}%
\end{pgfscope}%
\end{pgfscope}%
\begin{pgfscope}%
\definecolor{textcolor}{rgb}{0.000000,0.000000,0.000000}%
\pgfsetstrokecolor{textcolor}%
\pgfsetfillcolor{textcolor}%
\pgftext[x=3.596172in, y=0.220556in, left, base,rotate=30.000000]{\color{textcolor}\rmfamily\fontsize{7.000000}{8.400000}\selectfont 2021-07-13}%
\end{pgfscope}%
\begin{pgfscope}%
\pgfsetbuttcap%
\pgfsetroundjoin%
\definecolor{currentfill}{rgb}{0.000000,0.000000,0.000000}%
\pgfsetfillcolor{currentfill}%
\pgfsetlinewidth{0.501875pt}%
\definecolor{currentstroke}{rgb}{0.000000,0.000000,0.000000}%
\pgfsetstrokecolor{currentstroke}%
\pgfsetdash{}{0pt}%
\pgfsys@defobject{currentmarker}{\pgfqpoint{0.000000in}{0.000000in}}{\pgfqpoint{0.000000in}{0.041667in}}{%
\pgfpathmoveto{\pgfqpoint{0.000000in}{0.000000in}}%
\pgfpathlineto{\pgfqpoint{0.000000in}{0.041667in}}%
\pgfusepath{stroke,fill}%
}%
\begin{pgfscope}%
\pgfsys@transformshift{4.479945in}{0.586309in}%
\pgfsys@useobject{currentmarker}{}%
\end{pgfscope}%
\end{pgfscope}%
\begin{pgfscope}%
\pgfsetbuttcap%
\pgfsetroundjoin%
\definecolor{currentfill}{rgb}{0.000000,0.000000,0.000000}%
\pgfsetfillcolor{currentfill}%
\pgfsetlinewidth{0.501875pt}%
\definecolor{currentstroke}{rgb}{0.000000,0.000000,0.000000}%
\pgfsetstrokecolor{currentstroke}%
\pgfsetdash{}{0pt}%
\pgfsys@defobject{currentmarker}{\pgfqpoint{0.000000in}{-0.041667in}}{\pgfqpoint{0.000000in}{0.000000in}}{%
\pgfpathmoveto{\pgfqpoint{0.000000in}{0.000000in}}%
\pgfpathlineto{\pgfqpoint{0.000000in}{-0.041667in}}%
\pgfusepath{stroke,fill}%
}%
\begin{pgfscope}%
\pgfsys@transformshift{4.479945in}{0.893003in}%
\pgfsys@useobject{currentmarker}{}%
\end{pgfscope}%
\end{pgfscope}%
\begin{pgfscope}%
\definecolor{textcolor}{rgb}{0.000000,0.000000,0.000000}%
\pgfsetstrokecolor{textcolor}%
\pgfsetfillcolor{textcolor}%
\pgftext[x=4.022418in, y=0.220556in, left, base,rotate=30.000000]{\color{textcolor}\rmfamily\fontsize{7.000000}{8.400000}\selectfont 2021-09-11}%
\end{pgfscope}%
\begin{pgfscope}%
\pgfsetbuttcap%
\pgfsetroundjoin%
\definecolor{currentfill}{rgb}{0.000000,0.000000,0.000000}%
\pgfsetfillcolor{currentfill}%
\pgfsetlinewidth{0.501875pt}%
\definecolor{currentstroke}{rgb}{0.000000,0.000000,0.000000}%
\pgfsetstrokecolor{currentstroke}%
\pgfsetdash{}{0pt}%
\pgfsys@defobject{currentmarker}{\pgfqpoint{0.000000in}{0.000000in}}{\pgfqpoint{0.000000in}{0.041667in}}{%
\pgfpathmoveto{\pgfqpoint{0.000000in}{0.000000in}}%
\pgfpathlineto{\pgfqpoint{0.000000in}{0.041667in}}%
\pgfusepath{stroke,fill}%
}%
\begin{pgfscope}%
\pgfsys@transformshift{4.906192in}{0.586309in}%
\pgfsys@useobject{currentmarker}{}%
\end{pgfscope}%
\end{pgfscope}%
\begin{pgfscope}%
\pgfsetbuttcap%
\pgfsetroundjoin%
\definecolor{currentfill}{rgb}{0.000000,0.000000,0.000000}%
\pgfsetfillcolor{currentfill}%
\pgfsetlinewidth{0.501875pt}%
\definecolor{currentstroke}{rgb}{0.000000,0.000000,0.000000}%
\pgfsetstrokecolor{currentstroke}%
\pgfsetdash{}{0pt}%
\pgfsys@defobject{currentmarker}{\pgfqpoint{0.000000in}{-0.041667in}}{\pgfqpoint{0.000000in}{0.000000in}}{%
\pgfpathmoveto{\pgfqpoint{0.000000in}{0.000000in}}%
\pgfpathlineto{\pgfqpoint{0.000000in}{-0.041667in}}%
\pgfusepath{stroke,fill}%
}%
\begin{pgfscope}%
\pgfsys@transformshift{4.906192in}{0.893003in}%
\pgfsys@useobject{currentmarker}{}%
\end{pgfscope}%
\end{pgfscope}%
\begin{pgfscope}%
\definecolor{textcolor}{rgb}{0.000000,0.000000,0.000000}%
\pgfsetstrokecolor{textcolor}%
\pgfsetfillcolor{textcolor}%
\pgftext[x=4.448665in, y=0.220556in, left, base,rotate=30.000000]{\color{textcolor}\rmfamily\fontsize{7.000000}{8.400000}\selectfont 2021-11-10}%
\end{pgfscope}%
\begin{pgfscope}%
\pgfsetbuttcap%
\pgfsetroundjoin%
\definecolor{currentfill}{rgb}{0.000000,0.000000,0.000000}%
\pgfsetfillcolor{currentfill}%
\pgfsetlinewidth{0.501875pt}%
\definecolor{currentstroke}{rgb}{0.000000,0.000000,0.000000}%
\pgfsetstrokecolor{currentstroke}%
\pgfsetdash{}{0pt}%
\pgfsys@defobject{currentmarker}{\pgfqpoint{0.000000in}{0.000000in}}{\pgfqpoint{0.000000in}{0.041667in}}{%
\pgfpathmoveto{\pgfqpoint{0.000000in}{0.000000in}}%
\pgfpathlineto{\pgfqpoint{0.000000in}{0.041667in}}%
\pgfusepath{stroke,fill}%
}%
\begin{pgfscope}%
\pgfsys@transformshift{5.332438in}{0.586309in}%
\pgfsys@useobject{currentmarker}{}%
\end{pgfscope}%
\end{pgfscope}%
\begin{pgfscope}%
\pgfsetbuttcap%
\pgfsetroundjoin%
\definecolor{currentfill}{rgb}{0.000000,0.000000,0.000000}%
\pgfsetfillcolor{currentfill}%
\pgfsetlinewidth{0.501875pt}%
\definecolor{currentstroke}{rgb}{0.000000,0.000000,0.000000}%
\pgfsetstrokecolor{currentstroke}%
\pgfsetdash{}{0pt}%
\pgfsys@defobject{currentmarker}{\pgfqpoint{0.000000in}{-0.041667in}}{\pgfqpoint{0.000000in}{0.000000in}}{%
\pgfpathmoveto{\pgfqpoint{0.000000in}{0.000000in}}%
\pgfpathlineto{\pgfqpoint{0.000000in}{-0.041667in}}%
\pgfusepath{stroke,fill}%
}%
\begin{pgfscope}%
\pgfsys@transformshift{5.332438in}{0.893003in}%
\pgfsys@useobject{currentmarker}{}%
\end{pgfscope}%
\end{pgfscope}%
\begin{pgfscope}%
\definecolor{textcolor}{rgb}{0.000000,0.000000,0.000000}%
\pgfsetstrokecolor{textcolor}%
\pgfsetfillcolor{textcolor}%
\pgftext[x=4.874911in, y=0.220556in, left, base,rotate=30.000000]{\color{textcolor}\rmfamily\fontsize{7.000000}{8.400000}\selectfont 2022-01-09}%
\end{pgfscope}%
\begin{pgfscope}%
\pgfsetbuttcap%
\pgfsetroundjoin%
\definecolor{currentfill}{rgb}{0.000000,0.000000,0.000000}%
\pgfsetfillcolor{currentfill}%
\pgfsetlinewidth{0.501875pt}%
\definecolor{currentstroke}{rgb}{0.000000,0.000000,0.000000}%
\pgfsetstrokecolor{currentstroke}%
\pgfsetdash{}{0pt}%
\pgfsys@defobject{currentmarker}{\pgfqpoint{0.000000in}{0.000000in}}{\pgfqpoint{0.000000in}{0.041667in}}{%
\pgfpathmoveto{\pgfqpoint{0.000000in}{0.000000in}}%
\pgfpathlineto{\pgfqpoint{0.000000in}{0.041667in}}%
\pgfusepath{stroke,fill}%
}%
\begin{pgfscope}%
\pgfsys@transformshift{5.758684in}{0.586309in}%
\pgfsys@useobject{currentmarker}{}%
\end{pgfscope}%
\end{pgfscope}%
\begin{pgfscope}%
\pgfsetbuttcap%
\pgfsetroundjoin%
\definecolor{currentfill}{rgb}{0.000000,0.000000,0.000000}%
\pgfsetfillcolor{currentfill}%
\pgfsetlinewidth{0.501875pt}%
\definecolor{currentstroke}{rgb}{0.000000,0.000000,0.000000}%
\pgfsetstrokecolor{currentstroke}%
\pgfsetdash{}{0pt}%
\pgfsys@defobject{currentmarker}{\pgfqpoint{0.000000in}{-0.041667in}}{\pgfqpoint{0.000000in}{0.000000in}}{%
\pgfpathmoveto{\pgfqpoint{0.000000in}{0.000000in}}%
\pgfpathlineto{\pgfqpoint{0.000000in}{-0.041667in}}%
\pgfusepath{stroke,fill}%
}%
\begin{pgfscope}%
\pgfsys@transformshift{5.758684in}{0.893003in}%
\pgfsys@useobject{currentmarker}{}%
\end{pgfscope}%
\end{pgfscope}%
\begin{pgfscope}%
\definecolor{textcolor}{rgb}{0.000000,0.000000,0.000000}%
\pgfsetstrokecolor{textcolor}%
\pgfsetfillcolor{textcolor}%
\pgftext[x=5.301157in, y=0.220556in, left, base,rotate=30.000000]{\color{textcolor}\rmfamily\fontsize{7.000000}{8.400000}\selectfont 2022-03-10}%
\end{pgfscope}%
\begin{pgfscope}%
\pgfsetbuttcap%
\pgfsetroundjoin%
\definecolor{currentfill}{rgb}{0.000000,0.000000,0.000000}%
\pgfsetfillcolor{currentfill}%
\pgfsetlinewidth{0.501875pt}%
\definecolor{currentstroke}{rgb}{0.000000,0.000000,0.000000}%
\pgfsetstrokecolor{currentstroke}%
\pgfsetdash{}{0pt}%
\pgfsys@defobject{currentmarker}{\pgfqpoint{0.000000in}{0.000000in}}{\pgfqpoint{0.000000in}{0.041667in}}{%
\pgfpathmoveto{\pgfqpoint{0.000000in}{0.000000in}}%
\pgfpathlineto{\pgfqpoint{0.000000in}{0.041667in}}%
\pgfusepath{stroke,fill}%
}%
\begin{pgfscope}%
\pgfsys@transformshift{6.184931in}{0.586309in}%
\pgfsys@useobject{currentmarker}{}%
\end{pgfscope}%
\end{pgfscope}%
\begin{pgfscope}%
\pgfsetbuttcap%
\pgfsetroundjoin%
\definecolor{currentfill}{rgb}{0.000000,0.000000,0.000000}%
\pgfsetfillcolor{currentfill}%
\pgfsetlinewidth{0.501875pt}%
\definecolor{currentstroke}{rgb}{0.000000,0.000000,0.000000}%
\pgfsetstrokecolor{currentstroke}%
\pgfsetdash{}{0pt}%
\pgfsys@defobject{currentmarker}{\pgfqpoint{0.000000in}{-0.041667in}}{\pgfqpoint{0.000000in}{0.000000in}}{%
\pgfpathmoveto{\pgfqpoint{0.000000in}{0.000000in}}%
\pgfpathlineto{\pgfqpoint{0.000000in}{-0.041667in}}%
\pgfusepath{stroke,fill}%
}%
\begin{pgfscope}%
\pgfsys@transformshift{6.184931in}{0.893003in}%
\pgfsys@useobject{currentmarker}{}%
\end{pgfscope}%
\end{pgfscope}%
\begin{pgfscope}%
\definecolor{textcolor}{rgb}{0.000000,0.000000,0.000000}%
\pgfsetstrokecolor{textcolor}%
\pgfsetfillcolor{textcolor}%
\pgftext[x=5.727404in, y=0.220556in, left, base,rotate=30.000000]{\color{textcolor}\rmfamily\fontsize{7.000000}{8.400000}\selectfont 2022-05-09}%
\end{pgfscope}%
\begin{pgfscope}%
\pgfsetbuttcap%
\pgfsetroundjoin%
\definecolor{currentfill}{rgb}{0.000000,0.000000,0.000000}%
\pgfsetfillcolor{currentfill}%
\pgfsetlinewidth{0.501875pt}%
\definecolor{currentstroke}{rgb}{0.000000,0.000000,0.000000}%
\pgfsetstrokecolor{currentstroke}%
\pgfsetdash{}{0pt}%
\pgfsys@defobject{currentmarker}{\pgfqpoint{0.000000in}{0.000000in}}{\pgfqpoint{0.000000in}{0.020833in}}{%
\pgfpathmoveto{\pgfqpoint{0.000000in}{0.000000in}}%
\pgfpathlineto{\pgfqpoint{0.000000in}{0.020833in}}%
\pgfusepath{stroke,fill}%
}%
\begin{pgfscope}%
\pgfsys@transformshift{0.501646in}{0.586309in}%
\pgfsys@useobject{currentmarker}{}%
\end{pgfscope}%
\end{pgfscope}%
\begin{pgfscope}%
\pgfsetbuttcap%
\pgfsetroundjoin%
\definecolor{currentfill}{rgb}{0.000000,0.000000,0.000000}%
\pgfsetfillcolor{currentfill}%
\pgfsetlinewidth{0.501875pt}%
\definecolor{currentstroke}{rgb}{0.000000,0.000000,0.000000}%
\pgfsetstrokecolor{currentstroke}%
\pgfsetdash{}{0pt}%
\pgfsys@defobject{currentmarker}{\pgfqpoint{0.000000in}{-0.020833in}}{\pgfqpoint{0.000000in}{0.000000in}}{%
\pgfpathmoveto{\pgfqpoint{0.000000in}{0.000000in}}%
\pgfpathlineto{\pgfqpoint{0.000000in}{-0.020833in}}%
\pgfusepath{stroke,fill}%
}%
\begin{pgfscope}%
\pgfsys@transformshift{0.501646in}{0.893003in}%
\pgfsys@useobject{currentmarker}{}%
\end{pgfscope}%
\end{pgfscope}%
\begin{pgfscope}%
\pgfsetbuttcap%
\pgfsetroundjoin%
\definecolor{currentfill}{rgb}{0.000000,0.000000,0.000000}%
\pgfsetfillcolor{currentfill}%
\pgfsetlinewidth{0.501875pt}%
\definecolor{currentstroke}{rgb}{0.000000,0.000000,0.000000}%
\pgfsetstrokecolor{currentstroke}%
\pgfsetdash{}{0pt}%
\pgfsys@defobject{currentmarker}{\pgfqpoint{0.000000in}{0.000000in}}{\pgfqpoint{0.000000in}{0.020833in}}{%
\pgfpathmoveto{\pgfqpoint{0.000000in}{0.000000in}}%
\pgfpathlineto{\pgfqpoint{0.000000in}{0.020833in}}%
\pgfusepath{stroke,fill}%
}%
\begin{pgfscope}%
\pgfsys@transformshift{0.537167in}{0.586309in}%
\pgfsys@useobject{currentmarker}{}%
\end{pgfscope}%
\end{pgfscope}%
\begin{pgfscope}%
\pgfsetbuttcap%
\pgfsetroundjoin%
\definecolor{currentfill}{rgb}{0.000000,0.000000,0.000000}%
\pgfsetfillcolor{currentfill}%
\pgfsetlinewidth{0.501875pt}%
\definecolor{currentstroke}{rgb}{0.000000,0.000000,0.000000}%
\pgfsetstrokecolor{currentstroke}%
\pgfsetdash{}{0pt}%
\pgfsys@defobject{currentmarker}{\pgfqpoint{0.000000in}{-0.020833in}}{\pgfqpoint{0.000000in}{0.000000in}}{%
\pgfpathmoveto{\pgfqpoint{0.000000in}{0.000000in}}%
\pgfpathlineto{\pgfqpoint{0.000000in}{-0.020833in}}%
\pgfusepath{stroke,fill}%
}%
\begin{pgfscope}%
\pgfsys@transformshift{0.537167in}{0.893003in}%
\pgfsys@useobject{currentmarker}{}%
\end{pgfscope}%
\end{pgfscope}%
\begin{pgfscope}%
\pgfsetbuttcap%
\pgfsetroundjoin%
\definecolor{currentfill}{rgb}{0.000000,0.000000,0.000000}%
\pgfsetfillcolor{currentfill}%
\pgfsetlinewidth{0.501875pt}%
\definecolor{currentstroke}{rgb}{0.000000,0.000000,0.000000}%
\pgfsetstrokecolor{currentstroke}%
\pgfsetdash{}{0pt}%
\pgfsys@defobject{currentmarker}{\pgfqpoint{0.000000in}{0.000000in}}{\pgfqpoint{0.000000in}{0.020833in}}{%
\pgfpathmoveto{\pgfqpoint{0.000000in}{0.000000in}}%
\pgfpathlineto{\pgfqpoint{0.000000in}{0.020833in}}%
\pgfusepath{stroke,fill}%
}%
\begin{pgfscope}%
\pgfsys@transformshift{0.572687in}{0.586309in}%
\pgfsys@useobject{currentmarker}{}%
\end{pgfscope}%
\end{pgfscope}%
\begin{pgfscope}%
\pgfsetbuttcap%
\pgfsetroundjoin%
\definecolor{currentfill}{rgb}{0.000000,0.000000,0.000000}%
\pgfsetfillcolor{currentfill}%
\pgfsetlinewidth{0.501875pt}%
\definecolor{currentstroke}{rgb}{0.000000,0.000000,0.000000}%
\pgfsetstrokecolor{currentstroke}%
\pgfsetdash{}{0pt}%
\pgfsys@defobject{currentmarker}{\pgfqpoint{0.000000in}{-0.020833in}}{\pgfqpoint{0.000000in}{0.000000in}}{%
\pgfpathmoveto{\pgfqpoint{0.000000in}{0.000000in}}%
\pgfpathlineto{\pgfqpoint{0.000000in}{-0.020833in}}%
\pgfusepath{stroke,fill}%
}%
\begin{pgfscope}%
\pgfsys@transformshift{0.572687in}{0.893003in}%
\pgfsys@useobject{currentmarker}{}%
\end{pgfscope}%
\end{pgfscope}%
\begin{pgfscope}%
\pgfsetbuttcap%
\pgfsetroundjoin%
\definecolor{currentfill}{rgb}{0.000000,0.000000,0.000000}%
\pgfsetfillcolor{currentfill}%
\pgfsetlinewidth{0.501875pt}%
\definecolor{currentstroke}{rgb}{0.000000,0.000000,0.000000}%
\pgfsetstrokecolor{currentstroke}%
\pgfsetdash{}{0pt}%
\pgfsys@defobject{currentmarker}{\pgfqpoint{0.000000in}{0.000000in}}{\pgfqpoint{0.000000in}{0.020833in}}{%
\pgfpathmoveto{\pgfqpoint{0.000000in}{0.000000in}}%
\pgfpathlineto{\pgfqpoint{0.000000in}{0.020833in}}%
\pgfusepath{stroke,fill}%
}%
\begin{pgfscope}%
\pgfsys@transformshift{0.608208in}{0.586309in}%
\pgfsys@useobject{currentmarker}{}%
\end{pgfscope}%
\end{pgfscope}%
\begin{pgfscope}%
\pgfsetbuttcap%
\pgfsetroundjoin%
\definecolor{currentfill}{rgb}{0.000000,0.000000,0.000000}%
\pgfsetfillcolor{currentfill}%
\pgfsetlinewidth{0.501875pt}%
\definecolor{currentstroke}{rgb}{0.000000,0.000000,0.000000}%
\pgfsetstrokecolor{currentstroke}%
\pgfsetdash{}{0pt}%
\pgfsys@defobject{currentmarker}{\pgfqpoint{0.000000in}{-0.020833in}}{\pgfqpoint{0.000000in}{0.000000in}}{%
\pgfpathmoveto{\pgfqpoint{0.000000in}{0.000000in}}%
\pgfpathlineto{\pgfqpoint{0.000000in}{-0.020833in}}%
\pgfusepath{stroke,fill}%
}%
\begin{pgfscope}%
\pgfsys@transformshift{0.608208in}{0.893003in}%
\pgfsys@useobject{currentmarker}{}%
\end{pgfscope}%
\end{pgfscope}%
\begin{pgfscope}%
\pgfsetbuttcap%
\pgfsetroundjoin%
\definecolor{currentfill}{rgb}{0.000000,0.000000,0.000000}%
\pgfsetfillcolor{currentfill}%
\pgfsetlinewidth{0.501875pt}%
\definecolor{currentstroke}{rgb}{0.000000,0.000000,0.000000}%
\pgfsetstrokecolor{currentstroke}%
\pgfsetdash{}{0pt}%
\pgfsys@defobject{currentmarker}{\pgfqpoint{0.000000in}{0.000000in}}{\pgfqpoint{0.000000in}{0.020833in}}{%
\pgfpathmoveto{\pgfqpoint{0.000000in}{0.000000in}}%
\pgfpathlineto{\pgfqpoint{0.000000in}{0.020833in}}%
\pgfusepath{stroke,fill}%
}%
\begin{pgfscope}%
\pgfsys@transformshift{0.679249in}{0.586309in}%
\pgfsys@useobject{currentmarker}{}%
\end{pgfscope}%
\end{pgfscope}%
\begin{pgfscope}%
\pgfsetbuttcap%
\pgfsetroundjoin%
\definecolor{currentfill}{rgb}{0.000000,0.000000,0.000000}%
\pgfsetfillcolor{currentfill}%
\pgfsetlinewidth{0.501875pt}%
\definecolor{currentstroke}{rgb}{0.000000,0.000000,0.000000}%
\pgfsetstrokecolor{currentstroke}%
\pgfsetdash{}{0pt}%
\pgfsys@defobject{currentmarker}{\pgfqpoint{0.000000in}{-0.020833in}}{\pgfqpoint{0.000000in}{0.000000in}}{%
\pgfpathmoveto{\pgfqpoint{0.000000in}{0.000000in}}%
\pgfpathlineto{\pgfqpoint{0.000000in}{-0.020833in}}%
\pgfusepath{stroke,fill}%
}%
\begin{pgfscope}%
\pgfsys@transformshift{0.679249in}{0.893003in}%
\pgfsys@useobject{currentmarker}{}%
\end{pgfscope}%
\end{pgfscope}%
\begin{pgfscope}%
\pgfsetbuttcap%
\pgfsetroundjoin%
\definecolor{currentfill}{rgb}{0.000000,0.000000,0.000000}%
\pgfsetfillcolor{currentfill}%
\pgfsetlinewidth{0.501875pt}%
\definecolor{currentstroke}{rgb}{0.000000,0.000000,0.000000}%
\pgfsetstrokecolor{currentstroke}%
\pgfsetdash{}{0pt}%
\pgfsys@defobject{currentmarker}{\pgfqpoint{0.000000in}{0.000000in}}{\pgfqpoint{0.000000in}{0.020833in}}{%
\pgfpathmoveto{\pgfqpoint{0.000000in}{0.000000in}}%
\pgfpathlineto{\pgfqpoint{0.000000in}{0.020833in}}%
\pgfusepath{stroke,fill}%
}%
\begin{pgfscope}%
\pgfsys@transformshift{0.714769in}{0.586309in}%
\pgfsys@useobject{currentmarker}{}%
\end{pgfscope}%
\end{pgfscope}%
\begin{pgfscope}%
\pgfsetbuttcap%
\pgfsetroundjoin%
\definecolor{currentfill}{rgb}{0.000000,0.000000,0.000000}%
\pgfsetfillcolor{currentfill}%
\pgfsetlinewidth{0.501875pt}%
\definecolor{currentstroke}{rgb}{0.000000,0.000000,0.000000}%
\pgfsetstrokecolor{currentstroke}%
\pgfsetdash{}{0pt}%
\pgfsys@defobject{currentmarker}{\pgfqpoint{0.000000in}{-0.020833in}}{\pgfqpoint{0.000000in}{0.000000in}}{%
\pgfpathmoveto{\pgfqpoint{0.000000in}{0.000000in}}%
\pgfpathlineto{\pgfqpoint{0.000000in}{-0.020833in}}%
\pgfusepath{stroke,fill}%
}%
\begin{pgfscope}%
\pgfsys@transformshift{0.714769in}{0.893003in}%
\pgfsys@useobject{currentmarker}{}%
\end{pgfscope}%
\end{pgfscope}%
\begin{pgfscope}%
\pgfsetbuttcap%
\pgfsetroundjoin%
\definecolor{currentfill}{rgb}{0.000000,0.000000,0.000000}%
\pgfsetfillcolor{currentfill}%
\pgfsetlinewidth{0.501875pt}%
\definecolor{currentstroke}{rgb}{0.000000,0.000000,0.000000}%
\pgfsetstrokecolor{currentstroke}%
\pgfsetdash{}{0pt}%
\pgfsys@defobject{currentmarker}{\pgfqpoint{0.000000in}{0.000000in}}{\pgfqpoint{0.000000in}{0.020833in}}{%
\pgfpathmoveto{\pgfqpoint{0.000000in}{0.000000in}}%
\pgfpathlineto{\pgfqpoint{0.000000in}{0.020833in}}%
\pgfusepath{stroke,fill}%
}%
\begin{pgfscope}%
\pgfsys@transformshift{0.750290in}{0.586309in}%
\pgfsys@useobject{currentmarker}{}%
\end{pgfscope}%
\end{pgfscope}%
\begin{pgfscope}%
\pgfsetbuttcap%
\pgfsetroundjoin%
\definecolor{currentfill}{rgb}{0.000000,0.000000,0.000000}%
\pgfsetfillcolor{currentfill}%
\pgfsetlinewidth{0.501875pt}%
\definecolor{currentstroke}{rgb}{0.000000,0.000000,0.000000}%
\pgfsetstrokecolor{currentstroke}%
\pgfsetdash{}{0pt}%
\pgfsys@defobject{currentmarker}{\pgfqpoint{0.000000in}{-0.020833in}}{\pgfqpoint{0.000000in}{0.000000in}}{%
\pgfpathmoveto{\pgfqpoint{0.000000in}{0.000000in}}%
\pgfpathlineto{\pgfqpoint{0.000000in}{-0.020833in}}%
\pgfusepath{stroke,fill}%
}%
\begin{pgfscope}%
\pgfsys@transformshift{0.750290in}{0.893003in}%
\pgfsys@useobject{currentmarker}{}%
\end{pgfscope}%
\end{pgfscope}%
\begin{pgfscope}%
\pgfsetbuttcap%
\pgfsetroundjoin%
\definecolor{currentfill}{rgb}{0.000000,0.000000,0.000000}%
\pgfsetfillcolor{currentfill}%
\pgfsetlinewidth{0.501875pt}%
\definecolor{currentstroke}{rgb}{0.000000,0.000000,0.000000}%
\pgfsetstrokecolor{currentstroke}%
\pgfsetdash{}{0pt}%
\pgfsys@defobject{currentmarker}{\pgfqpoint{0.000000in}{0.000000in}}{\pgfqpoint{0.000000in}{0.020833in}}{%
\pgfpathmoveto{\pgfqpoint{0.000000in}{0.000000in}}%
\pgfpathlineto{\pgfqpoint{0.000000in}{0.020833in}}%
\pgfusepath{stroke,fill}%
}%
\begin{pgfscope}%
\pgfsys@transformshift{0.785811in}{0.586309in}%
\pgfsys@useobject{currentmarker}{}%
\end{pgfscope}%
\end{pgfscope}%
\begin{pgfscope}%
\pgfsetbuttcap%
\pgfsetroundjoin%
\definecolor{currentfill}{rgb}{0.000000,0.000000,0.000000}%
\pgfsetfillcolor{currentfill}%
\pgfsetlinewidth{0.501875pt}%
\definecolor{currentstroke}{rgb}{0.000000,0.000000,0.000000}%
\pgfsetstrokecolor{currentstroke}%
\pgfsetdash{}{0pt}%
\pgfsys@defobject{currentmarker}{\pgfqpoint{0.000000in}{-0.020833in}}{\pgfqpoint{0.000000in}{0.000000in}}{%
\pgfpathmoveto{\pgfqpoint{0.000000in}{0.000000in}}%
\pgfpathlineto{\pgfqpoint{0.000000in}{-0.020833in}}%
\pgfusepath{stroke,fill}%
}%
\begin{pgfscope}%
\pgfsys@transformshift{0.785811in}{0.893003in}%
\pgfsys@useobject{currentmarker}{}%
\end{pgfscope}%
\end{pgfscope}%
\begin{pgfscope}%
\pgfsetbuttcap%
\pgfsetroundjoin%
\definecolor{currentfill}{rgb}{0.000000,0.000000,0.000000}%
\pgfsetfillcolor{currentfill}%
\pgfsetlinewidth{0.501875pt}%
\definecolor{currentstroke}{rgb}{0.000000,0.000000,0.000000}%
\pgfsetstrokecolor{currentstroke}%
\pgfsetdash{}{0pt}%
\pgfsys@defobject{currentmarker}{\pgfqpoint{0.000000in}{0.000000in}}{\pgfqpoint{0.000000in}{0.020833in}}{%
\pgfpathmoveto{\pgfqpoint{0.000000in}{0.000000in}}%
\pgfpathlineto{\pgfqpoint{0.000000in}{0.020833in}}%
\pgfusepath{stroke,fill}%
}%
\begin{pgfscope}%
\pgfsys@transformshift{0.821331in}{0.586309in}%
\pgfsys@useobject{currentmarker}{}%
\end{pgfscope}%
\end{pgfscope}%
\begin{pgfscope}%
\pgfsetbuttcap%
\pgfsetroundjoin%
\definecolor{currentfill}{rgb}{0.000000,0.000000,0.000000}%
\pgfsetfillcolor{currentfill}%
\pgfsetlinewidth{0.501875pt}%
\definecolor{currentstroke}{rgb}{0.000000,0.000000,0.000000}%
\pgfsetstrokecolor{currentstroke}%
\pgfsetdash{}{0pt}%
\pgfsys@defobject{currentmarker}{\pgfqpoint{0.000000in}{-0.020833in}}{\pgfqpoint{0.000000in}{0.000000in}}{%
\pgfpathmoveto{\pgfqpoint{0.000000in}{0.000000in}}%
\pgfpathlineto{\pgfqpoint{0.000000in}{-0.020833in}}%
\pgfusepath{stroke,fill}%
}%
\begin{pgfscope}%
\pgfsys@transformshift{0.821331in}{0.893003in}%
\pgfsys@useobject{currentmarker}{}%
\end{pgfscope}%
\end{pgfscope}%
\begin{pgfscope}%
\pgfsetbuttcap%
\pgfsetroundjoin%
\definecolor{currentfill}{rgb}{0.000000,0.000000,0.000000}%
\pgfsetfillcolor{currentfill}%
\pgfsetlinewidth{0.501875pt}%
\definecolor{currentstroke}{rgb}{0.000000,0.000000,0.000000}%
\pgfsetstrokecolor{currentstroke}%
\pgfsetdash{}{0pt}%
\pgfsys@defobject{currentmarker}{\pgfqpoint{0.000000in}{0.000000in}}{\pgfqpoint{0.000000in}{0.020833in}}{%
\pgfpathmoveto{\pgfqpoint{0.000000in}{0.000000in}}%
\pgfpathlineto{\pgfqpoint{0.000000in}{0.020833in}}%
\pgfusepath{stroke,fill}%
}%
\begin{pgfscope}%
\pgfsys@transformshift{0.856852in}{0.586309in}%
\pgfsys@useobject{currentmarker}{}%
\end{pgfscope}%
\end{pgfscope}%
\begin{pgfscope}%
\pgfsetbuttcap%
\pgfsetroundjoin%
\definecolor{currentfill}{rgb}{0.000000,0.000000,0.000000}%
\pgfsetfillcolor{currentfill}%
\pgfsetlinewidth{0.501875pt}%
\definecolor{currentstroke}{rgb}{0.000000,0.000000,0.000000}%
\pgfsetstrokecolor{currentstroke}%
\pgfsetdash{}{0pt}%
\pgfsys@defobject{currentmarker}{\pgfqpoint{0.000000in}{-0.020833in}}{\pgfqpoint{0.000000in}{0.000000in}}{%
\pgfpathmoveto{\pgfqpoint{0.000000in}{0.000000in}}%
\pgfpathlineto{\pgfqpoint{0.000000in}{-0.020833in}}%
\pgfusepath{stroke,fill}%
}%
\begin{pgfscope}%
\pgfsys@transformshift{0.856852in}{0.893003in}%
\pgfsys@useobject{currentmarker}{}%
\end{pgfscope}%
\end{pgfscope}%
\begin{pgfscope}%
\pgfsetbuttcap%
\pgfsetroundjoin%
\definecolor{currentfill}{rgb}{0.000000,0.000000,0.000000}%
\pgfsetfillcolor{currentfill}%
\pgfsetlinewidth{0.501875pt}%
\definecolor{currentstroke}{rgb}{0.000000,0.000000,0.000000}%
\pgfsetstrokecolor{currentstroke}%
\pgfsetdash{}{0pt}%
\pgfsys@defobject{currentmarker}{\pgfqpoint{0.000000in}{0.000000in}}{\pgfqpoint{0.000000in}{0.020833in}}{%
\pgfpathmoveto{\pgfqpoint{0.000000in}{0.000000in}}%
\pgfpathlineto{\pgfqpoint{0.000000in}{0.020833in}}%
\pgfusepath{stroke,fill}%
}%
\begin{pgfscope}%
\pgfsys@transformshift{0.892372in}{0.586309in}%
\pgfsys@useobject{currentmarker}{}%
\end{pgfscope}%
\end{pgfscope}%
\begin{pgfscope}%
\pgfsetbuttcap%
\pgfsetroundjoin%
\definecolor{currentfill}{rgb}{0.000000,0.000000,0.000000}%
\pgfsetfillcolor{currentfill}%
\pgfsetlinewidth{0.501875pt}%
\definecolor{currentstroke}{rgb}{0.000000,0.000000,0.000000}%
\pgfsetstrokecolor{currentstroke}%
\pgfsetdash{}{0pt}%
\pgfsys@defobject{currentmarker}{\pgfqpoint{0.000000in}{-0.020833in}}{\pgfqpoint{0.000000in}{0.000000in}}{%
\pgfpathmoveto{\pgfqpoint{0.000000in}{0.000000in}}%
\pgfpathlineto{\pgfqpoint{0.000000in}{-0.020833in}}%
\pgfusepath{stroke,fill}%
}%
\begin{pgfscope}%
\pgfsys@transformshift{0.892372in}{0.893003in}%
\pgfsys@useobject{currentmarker}{}%
\end{pgfscope}%
\end{pgfscope}%
\begin{pgfscope}%
\pgfsetbuttcap%
\pgfsetroundjoin%
\definecolor{currentfill}{rgb}{0.000000,0.000000,0.000000}%
\pgfsetfillcolor{currentfill}%
\pgfsetlinewidth{0.501875pt}%
\definecolor{currentstroke}{rgb}{0.000000,0.000000,0.000000}%
\pgfsetstrokecolor{currentstroke}%
\pgfsetdash{}{0pt}%
\pgfsys@defobject{currentmarker}{\pgfqpoint{0.000000in}{0.000000in}}{\pgfqpoint{0.000000in}{0.020833in}}{%
\pgfpathmoveto{\pgfqpoint{0.000000in}{0.000000in}}%
\pgfpathlineto{\pgfqpoint{0.000000in}{0.020833in}}%
\pgfusepath{stroke,fill}%
}%
\begin{pgfscope}%
\pgfsys@transformshift{0.927893in}{0.586309in}%
\pgfsys@useobject{currentmarker}{}%
\end{pgfscope}%
\end{pgfscope}%
\begin{pgfscope}%
\pgfsetbuttcap%
\pgfsetroundjoin%
\definecolor{currentfill}{rgb}{0.000000,0.000000,0.000000}%
\pgfsetfillcolor{currentfill}%
\pgfsetlinewidth{0.501875pt}%
\definecolor{currentstroke}{rgb}{0.000000,0.000000,0.000000}%
\pgfsetstrokecolor{currentstroke}%
\pgfsetdash{}{0pt}%
\pgfsys@defobject{currentmarker}{\pgfqpoint{0.000000in}{-0.020833in}}{\pgfqpoint{0.000000in}{0.000000in}}{%
\pgfpathmoveto{\pgfqpoint{0.000000in}{0.000000in}}%
\pgfpathlineto{\pgfqpoint{0.000000in}{-0.020833in}}%
\pgfusepath{stroke,fill}%
}%
\begin{pgfscope}%
\pgfsys@transformshift{0.927893in}{0.893003in}%
\pgfsys@useobject{currentmarker}{}%
\end{pgfscope}%
\end{pgfscope}%
\begin{pgfscope}%
\pgfsetbuttcap%
\pgfsetroundjoin%
\definecolor{currentfill}{rgb}{0.000000,0.000000,0.000000}%
\pgfsetfillcolor{currentfill}%
\pgfsetlinewidth{0.501875pt}%
\definecolor{currentstroke}{rgb}{0.000000,0.000000,0.000000}%
\pgfsetstrokecolor{currentstroke}%
\pgfsetdash{}{0pt}%
\pgfsys@defobject{currentmarker}{\pgfqpoint{0.000000in}{0.000000in}}{\pgfqpoint{0.000000in}{0.020833in}}{%
\pgfpathmoveto{\pgfqpoint{0.000000in}{0.000000in}}%
\pgfpathlineto{\pgfqpoint{0.000000in}{0.020833in}}%
\pgfusepath{stroke,fill}%
}%
\begin{pgfscope}%
\pgfsys@transformshift{0.963413in}{0.586309in}%
\pgfsys@useobject{currentmarker}{}%
\end{pgfscope}%
\end{pgfscope}%
\begin{pgfscope}%
\pgfsetbuttcap%
\pgfsetroundjoin%
\definecolor{currentfill}{rgb}{0.000000,0.000000,0.000000}%
\pgfsetfillcolor{currentfill}%
\pgfsetlinewidth{0.501875pt}%
\definecolor{currentstroke}{rgb}{0.000000,0.000000,0.000000}%
\pgfsetstrokecolor{currentstroke}%
\pgfsetdash{}{0pt}%
\pgfsys@defobject{currentmarker}{\pgfqpoint{0.000000in}{-0.020833in}}{\pgfqpoint{0.000000in}{0.000000in}}{%
\pgfpathmoveto{\pgfqpoint{0.000000in}{0.000000in}}%
\pgfpathlineto{\pgfqpoint{0.000000in}{-0.020833in}}%
\pgfusepath{stroke,fill}%
}%
\begin{pgfscope}%
\pgfsys@transformshift{0.963413in}{0.893003in}%
\pgfsys@useobject{currentmarker}{}%
\end{pgfscope}%
\end{pgfscope}%
\begin{pgfscope}%
\pgfsetbuttcap%
\pgfsetroundjoin%
\definecolor{currentfill}{rgb}{0.000000,0.000000,0.000000}%
\pgfsetfillcolor{currentfill}%
\pgfsetlinewidth{0.501875pt}%
\definecolor{currentstroke}{rgb}{0.000000,0.000000,0.000000}%
\pgfsetstrokecolor{currentstroke}%
\pgfsetdash{}{0pt}%
\pgfsys@defobject{currentmarker}{\pgfqpoint{0.000000in}{0.000000in}}{\pgfqpoint{0.000000in}{0.020833in}}{%
\pgfpathmoveto{\pgfqpoint{0.000000in}{0.000000in}}%
\pgfpathlineto{\pgfqpoint{0.000000in}{0.020833in}}%
\pgfusepath{stroke,fill}%
}%
\begin{pgfscope}%
\pgfsys@transformshift{0.998934in}{0.586309in}%
\pgfsys@useobject{currentmarker}{}%
\end{pgfscope}%
\end{pgfscope}%
\begin{pgfscope}%
\pgfsetbuttcap%
\pgfsetroundjoin%
\definecolor{currentfill}{rgb}{0.000000,0.000000,0.000000}%
\pgfsetfillcolor{currentfill}%
\pgfsetlinewidth{0.501875pt}%
\definecolor{currentstroke}{rgb}{0.000000,0.000000,0.000000}%
\pgfsetstrokecolor{currentstroke}%
\pgfsetdash{}{0pt}%
\pgfsys@defobject{currentmarker}{\pgfqpoint{0.000000in}{-0.020833in}}{\pgfqpoint{0.000000in}{0.000000in}}{%
\pgfpathmoveto{\pgfqpoint{0.000000in}{0.000000in}}%
\pgfpathlineto{\pgfqpoint{0.000000in}{-0.020833in}}%
\pgfusepath{stroke,fill}%
}%
\begin{pgfscope}%
\pgfsys@transformshift{0.998934in}{0.893003in}%
\pgfsys@useobject{currentmarker}{}%
\end{pgfscope}%
\end{pgfscope}%
\begin{pgfscope}%
\pgfsetbuttcap%
\pgfsetroundjoin%
\definecolor{currentfill}{rgb}{0.000000,0.000000,0.000000}%
\pgfsetfillcolor{currentfill}%
\pgfsetlinewidth{0.501875pt}%
\definecolor{currentstroke}{rgb}{0.000000,0.000000,0.000000}%
\pgfsetstrokecolor{currentstroke}%
\pgfsetdash{}{0pt}%
\pgfsys@defobject{currentmarker}{\pgfqpoint{0.000000in}{0.000000in}}{\pgfqpoint{0.000000in}{0.020833in}}{%
\pgfpathmoveto{\pgfqpoint{0.000000in}{0.000000in}}%
\pgfpathlineto{\pgfqpoint{0.000000in}{0.020833in}}%
\pgfusepath{stroke,fill}%
}%
\begin{pgfscope}%
\pgfsys@transformshift{1.034454in}{0.586309in}%
\pgfsys@useobject{currentmarker}{}%
\end{pgfscope}%
\end{pgfscope}%
\begin{pgfscope}%
\pgfsetbuttcap%
\pgfsetroundjoin%
\definecolor{currentfill}{rgb}{0.000000,0.000000,0.000000}%
\pgfsetfillcolor{currentfill}%
\pgfsetlinewidth{0.501875pt}%
\definecolor{currentstroke}{rgb}{0.000000,0.000000,0.000000}%
\pgfsetstrokecolor{currentstroke}%
\pgfsetdash{}{0pt}%
\pgfsys@defobject{currentmarker}{\pgfqpoint{0.000000in}{-0.020833in}}{\pgfqpoint{0.000000in}{0.000000in}}{%
\pgfpathmoveto{\pgfqpoint{0.000000in}{0.000000in}}%
\pgfpathlineto{\pgfqpoint{0.000000in}{-0.020833in}}%
\pgfusepath{stroke,fill}%
}%
\begin{pgfscope}%
\pgfsys@transformshift{1.034454in}{0.893003in}%
\pgfsys@useobject{currentmarker}{}%
\end{pgfscope}%
\end{pgfscope}%
\begin{pgfscope}%
\pgfsetbuttcap%
\pgfsetroundjoin%
\definecolor{currentfill}{rgb}{0.000000,0.000000,0.000000}%
\pgfsetfillcolor{currentfill}%
\pgfsetlinewidth{0.501875pt}%
\definecolor{currentstroke}{rgb}{0.000000,0.000000,0.000000}%
\pgfsetstrokecolor{currentstroke}%
\pgfsetdash{}{0pt}%
\pgfsys@defobject{currentmarker}{\pgfqpoint{0.000000in}{0.000000in}}{\pgfqpoint{0.000000in}{0.020833in}}{%
\pgfpathmoveto{\pgfqpoint{0.000000in}{0.000000in}}%
\pgfpathlineto{\pgfqpoint{0.000000in}{0.020833in}}%
\pgfusepath{stroke,fill}%
}%
\begin{pgfscope}%
\pgfsys@transformshift{1.105495in}{0.586309in}%
\pgfsys@useobject{currentmarker}{}%
\end{pgfscope}%
\end{pgfscope}%
\begin{pgfscope}%
\pgfsetbuttcap%
\pgfsetroundjoin%
\definecolor{currentfill}{rgb}{0.000000,0.000000,0.000000}%
\pgfsetfillcolor{currentfill}%
\pgfsetlinewidth{0.501875pt}%
\definecolor{currentstroke}{rgb}{0.000000,0.000000,0.000000}%
\pgfsetstrokecolor{currentstroke}%
\pgfsetdash{}{0pt}%
\pgfsys@defobject{currentmarker}{\pgfqpoint{0.000000in}{-0.020833in}}{\pgfqpoint{0.000000in}{0.000000in}}{%
\pgfpathmoveto{\pgfqpoint{0.000000in}{0.000000in}}%
\pgfpathlineto{\pgfqpoint{0.000000in}{-0.020833in}}%
\pgfusepath{stroke,fill}%
}%
\begin{pgfscope}%
\pgfsys@transformshift{1.105495in}{0.893003in}%
\pgfsys@useobject{currentmarker}{}%
\end{pgfscope}%
\end{pgfscope}%
\begin{pgfscope}%
\pgfsetbuttcap%
\pgfsetroundjoin%
\definecolor{currentfill}{rgb}{0.000000,0.000000,0.000000}%
\pgfsetfillcolor{currentfill}%
\pgfsetlinewidth{0.501875pt}%
\definecolor{currentstroke}{rgb}{0.000000,0.000000,0.000000}%
\pgfsetstrokecolor{currentstroke}%
\pgfsetdash{}{0pt}%
\pgfsys@defobject{currentmarker}{\pgfqpoint{0.000000in}{0.000000in}}{\pgfqpoint{0.000000in}{0.020833in}}{%
\pgfpathmoveto{\pgfqpoint{0.000000in}{0.000000in}}%
\pgfpathlineto{\pgfqpoint{0.000000in}{0.020833in}}%
\pgfusepath{stroke,fill}%
}%
\begin{pgfscope}%
\pgfsys@transformshift{1.141016in}{0.586309in}%
\pgfsys@useobject{currentmarker}{}%
\end{pgfscope}%
\end{pgfscope}%
\begin{pgfscope}%
\pgfsetbuttcap%
\pgfsetroundjoin%
\definecolor{currentfill}{rgb}{0.000000,0.000000,0.000000}%
\pgfsetfillcolor{currentfill}%
\pgfsetlinewidth{0.501875pt}%
\definecolor{currentstroke}{rgb}{0.000000,0.000000,0.000000}%
\pgfsetstrokecolor{currentstroke}%
\pgfsetdash{}{0pt}%
\pgfsys@defobject{currentmarker}{\pgfqpoint{0.000000in}{-0.020833in}}{\pgfqpoint{0.000000in}{0.000000in}}{%
\pgfpathmoveto{\pgfqpoint{0.000000in}{0.000000in}}%
\pgfpathlineto{\pgfqpoint{0.000000in}{-0.020833in}}%
\pgfusepath{stroke,fill}%
}%
\begin{pgfscope}%
\pgfsys@transformshift{1.141016in}{0.893003in}%
\pgfsys@useobject{currentmarker}{}%
\end{pgfscope}%
\end{pgfscope}%
\begin{pgfscope}%
\pgfsetbuttcap%
\pgfsetroundjoin%
\definecolor{currentfill}{rgb}{0.000000,0.000000,0.000000}%
\pgfsetfillcolor{currentfill}%
\pgfsetlinewidth{0.501875pt}%
\definecolor{currentstroke}{rgb}{0.000000,0.000000,0.000000}%
\pgfsetstrokecolor{currentstroke}%
\pgfsetdash{}{0pt}%
\pgfsys@defobject{currentmarker}{\pgfqpoint{0.000000in}{0.000000in}}{\pgfqpoint{0.000000in}{0.020833in}}{%
\pgfpathmoveto{\pgfqpoint{0.000000in}{0.000000in}}%
\pgfpathlineto{\pgfqpoint{0.000000in}{0.020833in}}%
\pgfusepath{stroke,fill}%
}%
\begin{pgfscope}%
\pgfsys@transformshift{1.176536in}{0.586309in}%
\pgfsys@useobject{currentmarker}{}%
\end{pgfscope}%
\end{pgfscope}%
\begin{pgfscope}%
\pgfsetbuttcap%
\pgfsetroundjoin%
\definecolor{currentfill}{rgb}{0.000000,0.000000,0.000000}%
\pgfsetfillcolor{currentfill}%
\pgfsetlinewidth{0.501875pt}%
\definecolor{currentstroke}{rgb}{0.000000,0.000000,0.000000}%
\pgfsetstrokecolor{currentstroke}%
\pgfsetdash{}{0pt}%
\pgfsys@defobject{currentmarker}{\pgfqpoint{0.000000in}{-0.020833in}}{\pgfqpoint{0.000000in}{0.000000in}}{%
\pgfpathmoveto{\pgfqpoint{0.000000in}{0.000000in}}%
\pgfpathlineto{\pgfqpoint{0.000000in}{-0.020833in}}%
\pgfusepath{stroke,fill}%
}%
\begin{pgfscope}%
\pgfsys@transformshift{1.176536in}{0.893003in}%
\pgfsys@useobject{currentmarker}{}%
\end{pgfscope}%
\end{pgfscope}%
\begin{pgfscope}%
\pgfsetbuttcap%
\pgfsetroundjoin%
\definecolor{currentfill}{rgb}{0.000000,0.000000,0.000000}%
\pgfsetfillcolor{currentfill}%
\pgfsetlinewidth{0.501875pt}%
\definecolor{currentstroke}{rgb}{0.000000,0.000000,0.000000}%
\pgfsetstrokecolor{currentstroke}%
\pgfsetdash{}{0pt}%
\pgfsys@defobject{currentmarker}{\pgfqpoint{0.000000in}{0.000000in}}{\pgfqpoint{0.000000in}{0.020833in}}{%
\pgfpathmoveto{\pgfqpoint{0.000000in}{0.000000in}}%
\pgfpathlineto{\pgfqpoint{0.000000in}{0.020833in}}%
\pgfusepath{stroke,fill}%
}%
\begin{pgfscope}%
\pgfsys@transformshift{1.212057in}{0.586309in}%
\pgfsys@useobject{currentmarker}{}%
\end{pgfscope}%
\end{pgfscope}%
\begin{pgfscope}%
\pgfsetbuttcap%
\pgfsetroundjoin%
\definecolor{currentfill}{rgb}{0.000000,0.000000,0.000000}%
\pgfsetfillcolor{currentfill}%
\pgfsetlinewidth{0.501875pt}%
\definecolor{currentstroke}{rgb}{0.000000,0.000000,0.000000}%
\pgfsetstrokecolor{currentstroke}%
\pgfsetdash{}{0pt}%
\pgfsys@defobject{currentmarker}{\pgfqpoint{0.000000in}{-0.020833in}}{\pgfqpoint{0.000000in}{0.000000in}}{%
\pgfpathmoveto{\pgfqpoint{0.000000in}{0.000000in}}%
\pgfpathlineto{\pgfqpoint{0.000000in}{-0.020833in}}%
\pgfusepath{stroke,fill}%
}%
\begin{pgfscope}%
\pgfsys@transformshift{1.212057in}{0.893003in}%
\pgfsys@useobject{currentmarker}{}%
\end{pgfscope}%
\end{pgfscope}%
\begin{pgfscope}%
\pgfsetbuttcap%
\pgfsetroundjoin%
\definecolor{currentfill}{rgb}{0.000000,0.000000,0.000000}%
\pgfsetfillcolor{currentfill}%
\pgfsetlinewidth{0.501875pt}%
\definecolor{currentstroke}{rgb}{0.000000,0.000000,0.000000}%
\pgfsetstrokecolor{currentstroke}%
\pgfsetdash{}{0pt}%
\pgfsys@defobject{currentmarker}{\pgfqpoint{0.000000in}{0.000000in}}{\pgfqpoint{0.000000in}{0.020833in}}{%
\pgfpathmoveto{\pgfqpoint{0.000000in}{0.000000in}}%
\pgfpathlineto{\pgfqpoint{0.000000in}{0.020833in}}%
\pgfusepath{stroke,fill}%
}%
\begin{pgfscope}%
\pgfsys@transformshift{1.247577in}{0.586309in}%
\pgfsys@useobject{currentmarker}{}%
\end{pgfscope}%
\end{pgfscope}%
\begin{pgfscope}%
\pgfsetbuttcap%
\pgfsetroundjoin%
\definecolor{currentfill}{rgb}{0.000000,0.000000,0.000000}%
\pgfsetfillcolor{currentfill}%
\pgfsetlinewidth{0.501875pt}%
\definecolor{currentstroke}{rgb}{0.000000,0.000000,0.000000}%
\pgfsetstrokecolor{currentstroke}%
\pgfsetdash{}{0pt}%
\pgfsys@defobject{currentmarker}{\pgfqpoint{0.000000in}{-0.020833in}}{\pgfqpoint{0.000000in}{0.000000in}}{%
\pgfpathmoveto{\pgfqpoint{0.000000in}{0.000000in}}%
\pgfpathlineto{\pgfqpoint{0.000000in}{-0.020833in}}%
\pgfusepath{stroke,fill}%
}%
\begin{pgfscope}%
\pgfsys@transformshift{1.247577in}{0.893003in}%
\pgfsys@useobject{currentmarker}{}%
\end{pgfscope}%
\end{pgfscope}%
\begin{pgfscope}%
\pgfsetbuttcap%
\pgfsetroundjoin%
\definecolor{currentfill}{rgb}{0.000000,0.000000,0.000000}%
\pgfsetfillcolor{currentfill}%
\pgfsetlinewidth{0.501875pt}%
\definecolor{currentstroke}{rgb}{0.000000,0.000000,0.000000}%
\pgfsetstrokecolor{currentstroke}%
\pgfsetdash{}{0pt}%
\pgfsys@defobject{currentmarker}{\pgfqpoint{0.000000in}{0.000000in}}{\pgfqpoint{0.000000in}{0.020833in}}{%
\pgfpathmoveto{\pgfqpoint{0.000000in}{0.000000in}}%
\pgfpathlineto{\pgfqpoint{0.000000in}{0.020833in}}%
\pgfusepath{stroke,fill}%
}%
\begin{pgfscope}%
\pgfsys@transformshift{1.283098in}{0.586309in}%
\pgfsys@useobject{currentmarker}{}%
\end{pgfscope}%
\end{pgfscope}%
\begin{pgfscope}%
\pgfsetbuttcap%
\pgfsetroundjoin%
\definecolor{currentfill}{rgb}{0.000000,0.000000,0.000000}%
\pgfsetfillcolor{currentfill}%
\pgfsetlinewidth{0.501875pt}%
\definecolor{currentstroke}{rgb}{0.000000,0.000000,0.000000}%
\pgfsetstrokecolor{currentstroke}%
\pgfsetdash{}{0pt}%
\pgfsys@defobject{currentmarker}{\pgfqpoint{0.000000in}{-0.020833in}}{\pgfqpoint{0.000000in}{0.000000in}}{%
\pgfpathmoveto{\pgfqpoint{0.000000in}{0.000000in}}%
\pgfpathlineto{\pgfqpoint{0.000000in}{-0.020833in}}%
\pgfusepath{stroke,fill}%
}%
\begin{pgfscope}%
\pgfsys@transformshift{1.283098in}{0.893003in}%
\pgfsys@useobject{currentmarker}{}%
\end{pgfscope}%
\end{pgfscope}%
\begin{pgfscope}%
\pgfsetbuttcap%
\pgfsetroundjoin%
\definecolor{currentfill}{rgb}{0.000000,0.000000,0.000000}%
\pgfsetfillcolor{currentfill}%
\pgfsetlinewidth{0.501875pt}%
\definecolor{currentstroke}{rgb}{0.000000,0.000000,0.000000}%
\pgfsetstrokecolor{currentstroke}%
\pgfsetdash{}{0pt}%
\pgfsys@defobject{currentmarker}{\pgfqpoint{0.000000in}{0.000000in}}{\pgfqpoint{0.000000in}{0.020833in}}{%
\pgfpathmoveto{\pgfqpoint{0.000000in}{0.000000in}}%
\pgfpathlineto{\pgfqpoint{0.000000in}{0.020833in}}%
\pgfusepath{stroke,fill}%
}%
\begin{pgfscope}%
\pgfsys@transformshift{1.318618in}{0.586309in}%
\pgfsys@useobject{currentmarker}{}%
\end{pgfscope}%
\end{pgfscope}%
\begin{pgfscope}%
\pgfsetbuttcap%
\pgfsetroundjoin%
\definecolor{currentfill}{rgb}{0.000000,0.000000,0.000000}%
\pgfsetfillcolor{currentfill}%
\pgfsetlinewidth{0.501875pt}%
\definecolor{currentstroke}{rgb}{0.000000,0.000000,0.000000}%
\pgfsetstrokecolor{currentstroke}%
\pgfsetdash{}{0pt}%
\pgfsys@defobject{currentmarker}{\pgfqpoint{0.000000in}{-0.020833in}}{\pgfqpoint{0.000000in}{0.000000in}}{%
\pgfpathmoveto{\pgfqpoint{0.000000in}{0.000000in}}%
\pgfpathlineto{\pgfqpoint{0.000000in}{-0.020833in}}%
\pgfusepath{stroke,fill}%
}%
\begin{pgfscope}%
\pgfsys@transformshift{1.318618in}{0.893003in}%
\pgfsys@useobject{currentmarker}{}%
\end{pgfscope}%
\end{pgfscope}%
\begin{pgfscope}%
\pgfsetbuttcap%
\pgfsetroundjoin%
\definecolor{currentfill}{rgb}{0.000000,0.000000,0.000000}%
\pgfsetfillcolor{currentfill}%
\pgfsetlinewidth{0.501875pt}%
\definecolor{currentstroke}{rgb}{0.000000,0.000000,0.000000}%
\pgfsetstrokecolor{currentstroke}%
\pgfsetdash{}{0pt}%
\pgfsys@defobject{currentmarker}{\pgfqpoint{0.000000in}{0.000000in}}{\pgfqpoint{0.000000in}{0.020833in}}{%
\pgfpathmoveto{\pgfqpoint{0.000000in}{0.000000in}}%
\pgfpathlineto{\pgfqpoint{0.000000in}{0.020833in}}%
\pgfusepath{stroke,fill}%
}%
\begin{pgfscope}%
\pgfsys@transformshift{1.354139in}{0.586309in}%
\pgfsys@useobject{currentmarker}{}%
\end{pgfscope}%
\end{pgfscope}%
\begin{pgfscope}%
\pgfsetbuttcap%
\pgfsetroundjoin%
\definecolor{currentfill}{rgb}{0.000000,0.000000,0.000000}%
\pgfsetfillcolor{currentfill}%
\pgfsetlinewidth{0.501875pt}%
\definecolor{currentstroke}{rgb}{0.000000,0.000000,0.000000}%
\pgfsetstrokecolor{currentstroke}%
\pgfsetdash{}{0pt}%
\pgfsys@defobject{currentmarker}{\pgfqpoint{0.000000in}{-0.020833in}}{\pgfqpoint{0.000000in}{0.000000in}}{%
\pgfpathmoveto{\pgfqpoint{0.000000in}{0.000000in}}%
\pgfpathlineto{\pgfqpoint{0.000000in}{-0.020833in}}%
\pgfusepath{stroke,fill}%
}%
\begin{pgfscope}%
\pgfsys@transformshift{1.354139in}{0.893003in}%
\pgfsys@useobject{currentmarker}{}%
\end{pgfscope}%
\end{pgfscope}%
\begin{pgfscope}%
\pgfsetbuttcap%
\pgfsetroundjoin%
\definecolor{currentfill}{rgb}{0.000000,0.000000,0.000000}%
\pgfsetfillcolor{currentfill}%
\pgfsetlinewidth{0.501875pt}%
\definecolor{currentstroke}{rgb}{0.000000,0.000000,0.000000}%
\pgfsetstrokecolor{currentstroke}%
\pgfsetdash{}{0pt}%
\pgfsys@defobject{currentmarker}{\pgfqpoint{0.000000in}{0.000000in}}{\pgfqpoint{0.000000in}{0.020833in}}{%
\pgfpathmoveto{\pgfqpoint{0.000000in}{0.000000in}}%
\pgfpathlineto{\pgfqpoint{0.000000in}{0.020833in}}%
\pgfusepath{stroke,fill}%
}%
\begin{pgfscope}%
\pgfsys@transformshift{1.389660in}{0.586309in}%
\pgfsys@useobject{currentmarker}{}%
\end{pgfscope}%
\end{pgfscope}%
\begin{pgfscope}%
\pgfsetbuttcap%
\pgfsetroundjoin%
\definecolor{currentfill}{rgb}{0.000000,0.000000,0.000000}%
\pgfsetfillcolor{currentfill}%
\pgfsetlinewidth{0.501875pt}%
\definecolor{currentstroke}{rgb}{0.000000,0.000000,0.000000}%
\pgfsetstrokecolor{currentstroke}%
\pgfsetdash{}{0pt}%
\pgfsys@defobject{currentmarker}{\pgfqpoint{0.000000in}{-0.020833in}}{\pgfqpoint{0.000000in}{0.000000in}}{%
\pgfpathmoveto{\pgfqpoint{0.000000in}{0.000000in}}%
\pgfpathlineto{\pgfqpoint{0.000000in}{-0.020833in}}%
\pgfusepath{stroke,fill}%
}%
\begin{pgfscope}%
\pgfsys@transformshift{1.389660in}{0.893003in}%
\pgfsys@useobject{currentmarker}{}%
\end{pgfscope}%
\end{pgfscope}%
\begin{pgfscope}%
\pgfsetbuttcap%
\pgfsetroundjoin%
\definecolor{currentfill}{rgb}{0.000000,0.000000,0.000000}%
\pgfsetfillcolor{currentfill}%
\pgfsetlinewidth{0.501875pt}%
\definecolor{currentstroke}{rgb}{0.000000,0.000000,0.000000}%
\pgfsetstrokecolor{currentstroke}%
\pgfsetdash{}{0pt}%
\pgfsys@defobject{currentmarker}{\pgfqpoint{0.000000in}{0.000000in}}{\pgfqpoint{0.000000in}{0.020833in}}{%
\pgfpathmoveto{\pgfqpoint{0.000000in}{0.000000in}}%
\pgfpathlineto{\pgfqpoint{0.000000in}{0.020833in}}%
\pgfusepath{stroke,fill}%
}%
\begin{pgfscope}%
\pgfsys@transformshift{1.425180in}{0.586309in}%
\pgfsys@useobject{currentmarker}{}%
\end{pgfscope}%
\end{pgfscope}%
\begin{pgfscope}%
\pgfsetbuttcap%
\pgfsetroundjoin%
\definecolor{currentfill}{rgb}{0.000000,0.000000,0.000000}%
\pgfsetfillcolor{currentfill}%
\pgfsetlinewidth{0.501875pt}%
\definecolor{currentstroke}{rgb}{0.000000,0.000000,0.000000}%
\pgfsetstrokecolor{currentstroke}%
\pgfsetdash{}{0pt}%
\pgfsys@defobject{currentmarker}{\pgfqpoint{0.000000in}{-0.020833in}}{\pgfqpoint{0.000000in}{0.000000in}}{%
\pgfpathmoveto{\pgfqpoint{0.000000in}{0.000000in}}%
\pgfpathlineto{\pgfqpoint{0.000000in}{-0.020833in}}%
\pgfusepath{stroke,fill}%
}%
\begin{pgfscope}%
\pgfsys@transformshift{1.425180in}{0.893003in}%
\pgfsys@useobject{currentmarker}{}%
\end{pgfscope}%
\end{pgfscope}%
\begin{pgfscope}%
\pgfsetbuttcap%
\pgfsetroundjoin%
\definecolor{currentfill}{rgb}{0.000000,0.000000,0.000000}%
\pgfsetfillcolor{currentfill}%
\pgfsetlinewidth{0.501875pt}%
\definecolor{currentstroke}{rgb}{0.000000,0.000000,0.000000}%
\pgfsetstrokecolor{currentstroke}%
\pgfsetdash{}{0pt}%
\pgfsys@defobject{currentmarker}{\pgfqpoint{0.000000in}{0.000000in}}{\pgfqpoint{0.000000in}{0.020833in}}{%
\pgfpathmoveto{\pgfqpoint{0.000000in}{0.000000in}}%
\pgfpathlineto{\pgfqpoint{0.000000in}{0.020833in}}%
\pgfusepath{stroke,fill}%
}%
\begin{pgfscope}%
\pgfsys@transformshift{1.460701in}{0.586309in}%
\pgfsys@useobject{currentmarker}{}%
\end{pgfscope}%
\end{pgfscope}%
\begin{pgfscope}%
\pgfsetbuttcap%
\pgfsetroundjoin%
\definecolor{currentfill}{rgb}{0.000000,0.000000,0.000000}%
\pgfsetfillcolor{currentfill}%
\pgfsetlinewidth{0.501875pt}%
\definecolor{currentstroke}{rgb}{0.000000,0.000000,0.000000}%
\pgfsetstrokecolor{currentstroke}%
\pgfsetdash{}{0pt}%
\pgfsys@defobject{currentmarker}{\pgfqpoint{0.000000in}{-0.020833in}}{\pgfqpoint{0.000000in}{0.000000in}}{%
\pgfpathmoveto{\pgfqpoint{0.000000in}{0.000000in}}%
\pgfpathlineto{\pgfqpoint{0.000000in}{-0.020833in}}%
\pgfusepath{stroke,fill}%
}%
\begin{pgfscope}%
\pgfsys@transformshift{1.460701in}{0.893003in}%
\pgfsys@useobject{currentmarker}{}%
\end{pgfscope}%
\end{pgfscope}%
\begin{pgfscope}%
\pgfsetbuttcap%
\pgfsetroundjoin%
\definecolor{currentfill}{rgb}{0.000000,0.000000,0.000000}%
\pgfsetfillcolor{currentfill}%
\pgfsetlinewidth{0.501875pt}%
\definecolor{currentstroke}{rgb}{0.000000,0.000000,0.000000}%
\pgfsetstrokecolor{currentstroke}%
\pgfsetdash{}{0pt}%
\pgfsys@defobject{currentmarker}{\pgfqpoint{0.000000in}{0.000000in}}{\pgfqpoint{0.000000in}{0.020833in}}{%
\pgfpathmoveto{\pgfqpoint{0.000000in}{0.000000in}}%
\pgfpathlineto{\pgfqpoint{0.000000in}{0.020833in}}%
\pgfusepath{stroke,fill}%
}%
\begin{pgfscope}%
\pgfsys@transformshift{1.531742in}{0.586309in}%
\pgfsys@useobject{currentmarker}{}%
\end{pgfscope}%
\end{pgfscope}%
\begin{pgfscope}%
\pgfsetbuttcap%
\pgfsetroundjoin%
\definecolor{currentfill}{rgb}{0.000000,0.000000,0.000000}%
\pgfsetfillcolor{currentfill}%
\pgfsetlinewidth{0.501875pt}%
\definecolor{currentstroke}{rgb}{0.000000,0.000000,0.000000}%
\pgfsetstrokecolor{currentstroke}%
\pgfsetdash{}{0pt}%
\pgfsys@defobject{currentmarker}{\pgfqpoint{0.000000in}{-0.020833in}}{\pgfqpoint{0.000000in}{0.000000in}}{%
\pgfpathmoveto{\pgfqpoint{0.000000in}{0.000000in}}%
\pgfpathlineto{\pgfqpoint{0.000000in}{-0.020833in}}%
\pgfusepath{stroke,fill}%
}%
\begin{pgfscope}%
\pgfsys@transformshift{1.531742in}{0.893003in}%
\pgfsys@useobject{currentmarker}{}%
\end{pgfscope}%
\end{pgfscope}%
\begin{pgfscope}%
\pgfsetbuttcap%
\pgfsetroundjoin%
\definecolor{currentfill}{rgb}{0.000000,0.000000,0.000000}%
\pgfsetfillcolor{currentfill}%
\pgfsetlinewidth{0.501875pt}%
\definecolor{currentstroke}{rgb}{0.000000,0.000000,0.000000}%
\pgfsetstrokecolor{currentstroke}%
\pgfsetdash{}{0pt}%
\pgfsys@defobject{currentmarker}{\pgfqpoint{0.000000in}{0.000000in}}{\pgfqpoint{0.000000in}{0.020833in}}{%
\pgfpathmoveto{\pgfqpoint{0.000000in}{0.000000in}}%
\pgfpathlineto{\pgfqpoint{0.000000in}{0.020833in}}%
\pgfusepath{stroke,fill}%
}%
\begin{pgfscope}%
\pgfsys@transformshift{1.567262in}{0.586309in}%
\pgfsys@useobject{currentmarker}{}%
\end{pgfscope}%
\end{pgfscope}%
\begin{pgfscope}%
\pgfsetbuttcap%
\pgfsetroundjoin%
\definecolor{currentfill}{rgb}{0.000000,0.000000,0.000000}%
\pgfsetfillcolor{currentfill}%
\pgfsetlinewidth{0.501875pt}%
\definecolor{currentstroke}{rgb}{0.000000,0.000000,0.000000}%
\pgfsetstrokecolor{currentstroke}%
\pgfsetdash{}{0pt}%
\pgfsys@defobject{currentmarker}{\pgfqpoint{0.000000in}{-0.020833in}}{\pgfqpoint{0.000000in}{0.000000in}}{%
\pgfpathmoveto{\pgfqpoint{0.000000in}{0.000000in}}%
\pgfpathlineto{\pgfqpoint{0.000000in}{-0.020833in}}%
\pgfusepath{stroke,fill}%
}%
\begin{pgfscope}%
\pgfsys@transformshift{1.567262in}{0.893003in}%
\pgfsys@useobject{currentmarker}{}%
\end{pgfscope}%
\end{pgfscope}%
\begin{pgfscope}%
\pgfsetbuttcap%
\pgfsetroundjoin%
\definecolor{currentfill}{rgb}{0.000000,0.000000,0.000000}%
\pgfsetfillcolor{currentfill}%
\pgfsetlinewidth{0.501875pt}%
\definecolor{currentstroke}{rgb}{0.000000,0.000000,0.000000}%
\pgfsetstrokecolor{currentstroke}%
\pgfsetdash{}{0pt}%
\pgfsys@defobject{currentmarker}{\pgfqpoint{0.000000in}{0.000000in}}{\pgfqpoint{0.000000in}{0.020833in}}{%
\pgfpathmoveto{\pgfqpoint{0.000000in}{0.000000in}}%
\pgfpathlineto{\pgfqpoint{0.000000in}{0.020833in}}%
\pgfusepath{stroke,fill}%
}%
\begin{pgfscope}%
\pgfsys@transformshift{1.602783in}{0.586309in}%
\pgfsys@useobject{currentmarker}{}%
\end{pgfscope}%
\end{pgfscope}%
\begin{pgfscope}%
\pgfsetbuttcap%
\pgfsetroundjoin%
\definecolor{currentfill}{rgb}{0.000000,0.000000,0.000000}%
\pgfsetfillcolor{currentfill}%
\pgfsetlinewidth{0.501875pt}%
\definecolor{currentstroke}{rgb}{0.000000,0.000000,0.000000}%
\pgfsetstrokecolor{currentstroke}%
\pgfsetdash{}{0pt}%
\pgfsys@defobject{currentmarker}{\pgfqpoint{0.000000in}{-0.020833in}}{\pgfqpoint{0.000000in}{0.000000in}}{%
\pgfpathmoveto{\pgfqpoint{0.000000in}{0.000000in}}%
\pgfpathlineto{\pgfqpoint{0.000000in}{-0.020833in}}%
\pgfusepath{stroke,fill}%
}%
\begin{pgfscope}%
\pgfsys@transformshift{1.602783in}{0.893003in}%
\pgfsys@useobject{currentmarker}{}%
\end{pgfscope}%
\end{pgfscope}%
\begin{pgfscope}%
\pgfsetbuttcap%
\pgfsetroundjoin%
\definecolor{currentfill}{rgb}{0.000000,0.000000,0.000000}%
\pgfsetfillcolor{currentfill}%
\pgfsetlinewidth{0.501875pt}%
\definecolor{currentstroke}{rgb}{0.000000,0.000000,0.000000}%
\pgfsetstrokecolor{currentstroke}%
\pgfsetdash{}{0pt}%
\pgfsys@defobject{currentmarker}{\pgfqpoint{0.000000in}{0.000000in}}{\pgfqpoint{0.000000in}{0.020833in}}{%
\pgfpathmoveto{\pgfqpoint{0.000000in}{0.000000in}}%
\pgfpathlineto{\pgfqpoint{0.000000in}{0.020833in}}%
\pgfusepath{stroke,fill}%
}%
\begin{pgfscope}%
\pgfsys@transformshift{1.638303in}{0.586309in}%
\pgfsys@useobject{currentmarker}{}%
\end{pgfscope}%
\end{pgfscope}%
\begin{pgfscope}%
\pgfsetbuttcap%
\pgfsetroundjoin%
\definecolor{currentfill}{rgb}{0.000000,0.000000,0.000000}%
\pgfsetfillcolor{currentfill}%
\pgfsetlinewidth{0.501875pt}%
\definecolor{currentstroke}{rgb}{0.000000,0.000000,0.000000}%
\pgfsetstrokecolor{currentstroke}%
\pgfsetdash{}{0pt}%
\pgfsys@defobject{currentmarker}{\pgfqpoint{0.000000in}{-0.020833in}}{\pgfqpoint{0.000000in}{0.000000in}}{%
\pgfpathmoveto{\pgfqpoint{0.000000in}{0.000000in}}%
\pgfpathlineto{\pgfqpoint{0.000000in}{-0.020833in}}%
\pgfusepath{stroke,fill}%
}%
\begin{pgfscope}%
\pgfsys@transformshift{1.638303in}{0.893003in}%
\pgfsys@useobject{currentmarker}{}%
\end{pgfscope}%
\end{pgfscope}%
\begin{pgfscope}%
\pgfsetbuttcap%
\pgfsetroundjoin%
\definecolor{currentfill}{rgb}{0.000000,0.000000,0.000000}%
\pgfsetfillcolor{currentfill}%
\pgfsetlinewidth{0.501875pt}%
\definecolor{currentstroke}{rgb}{0.000000,0.000000,0.000000}%
\pgfsetstrokecolor{currentstroke}%
\pgfsetdash{}{0pt}%
\pgfsys@defobject{currentmarker}{\pgfqpoint{0.000000in}{0.000000in}}{\pgfqpoint{0.000000in}{0.020833in}}{%
\pgfpathmoveto{\pgfqpoint{0.000000in}{0.000000in}}%
\pgfpathlineto{\pgfqpoint{0.000000in}{0.020833in}}%
\pgfusepath{stroke,fill}%
}%
\begin{pgfscope}%
\pgfsys@transformshift{1.673824in}{0.586309in}%
\pgfsys@useobject{currentmarker}{}%
\end{pgfscope}%
\end{pgfscope}%
\begin{pgfscope}%
\pgfsetbuttcap%
\pgfsetroundjoin%
\definecolor{currentfill}{rgb}{0.000000,0.000000,0.000000}%
\pgfsetfillcolor{currentfill}%
\pgfsetlinewidth{0.501875pt}%
\definecolor{currentstroke}{rgb}{0.000000,0.000000,0.000000}%
\pgfsetstrokecolor{currentstroke}%
\pgfsetdash{}{0pt}%
\pgfsys@defobject{currentmarker}{\pgfqpoint{0.000000in}{-0.020833in}}{\pgfqpoint{0.000000in}{0.000000in}}{%
\pgfpathmoveto{\pgfqpoint{0.000000in}{0.000000in}}%
\pgfpathlineto{\pgfqpoint{0.000000in}{-0.020833in}}%
\pgfusepath{stroke,fill}%
}%
\begin{pgfscope}%
\pgfsys@transformshift{1.673824in}{0.893003in}%
\pgfsys@useobject{currentmarker}{}%
\end{pgfscope}%
\end{pgfscope}%
\begin{pgfscope}%
\pgfsetbuttcap%
\pgfsetroundjoin%
\definecolor{currentfill}{rgb}{0.000000,0.000000,0.000000}%
\pgfsetfillcolor{currentfill}%
\pgfsetlinewidth{0.501875pt}%
\definecolor{currentstroke}{rgb}{0.000000,0.000000,0.000000}%
\pgfsetstrokecolor{currentstroke}%
\pgfsetdash{}{0pt}%
\pgfsys@defobject{currentmarker}{\pgfqpoint{0.000000in}{0.000000in}}{\pgfqpoint{0.000000in}{0.020833in}}{%
\pgfpathmoveto{\pgfqpoint{0.000000in}{0.000000in}}%
\pgfpathlineto{\pgfqpoint{0.000000in}{0.020833in}}%
\pgfusepath{stroke,fill}%
}%
\begin{pgfscope}%
\pgfsys@transformshift{1.709344in}{0.586309in}%
\pgfsys@useobject{currentmarker}{}%
\end{pgfscope}%
\end{pgfscope}%
\begin{pgfscope}%
\pgfsetbuttcap%
\pgfsetroundjoin%
\definecolor{currentfill}{rgb}{0.000000,0.000000,0.000000}%
\pgfsetfillcolor{currentfill}%
\pgfsetlinewidth{0.501875pt}%
\definecolor{currentstroke}{rgb}{0.000000,0.000000,0.000000}%
\pgfsetstrokecolor{currentstroke}%
\pgfsetdash{}{0pt}%
\pgfsys@defobject{currentmarker}{\pgfqpoint{0.000000in}{-0.020833in}}{\pgfqpoint{0.000000in}{0.000000in}}{%
\pgfpathmoveto{\pgfqpoint{0.000000in}{0.000000in}}%
\pgfpathlineto{\pgfqpoint{0.000000in}{-0.020833in}}%
\pgfusepath{stroke,fill}%
}%
\begin{pgfscope}%
\pgfsys@transformshift{1.709344in}{0.893003in}%
\pgfsys@useobject{currentmarker}{}%
\end{pgfscope}%
\end{pgfscope}%
\begin{pgfscope}%
\pgfsetbuttcap%
\pgfsetroundjoin%
\definecolor{currentfill}{rgb}{0.000000,0.000000,0.000000}%
\pgfsetfillcolor{currentfill}%
\pgfsetlinewidth{0.501875pt}%
\definecolor{currentstroke}{rgb}{0.000000,0.000000,0.000000}%
\pgfsetstrokecolor{currentstroke}%
\pgfsetdash{}{0pt}%
\pgfsys@defobject{currentmarker}{\pgfqpoint{0.000000in}{0.000000in}}{\pgfqpoint{0.000000in}{0.020833in}}{%
\pgfpathmoveto{\pgfqpoint{0.000000in}{0.000000in}}%
\pgfpathlineto{\pgfqpoint{0.000000in}{0.020833in}}%
\pgfusepath{stroke,fill}%
}%
\begin{pgfscope}%
\pgfsys@transformshift{1.744865in}{0.586309in}%
\pgfsys@useobject{currentmarker}{}%
\end{pgfscope}%
\end{pgfscope}%
\begin{pgfscope}%
\pgfsetbuttcap%
\pgfsetroundjoin%
\definecolor{currentfill}{rgb}{0.000000,0.000000,0.000000}%
\pgfsetfillcolor{currentfill}%
\pgfsetlinewidth{0.501875pt}%
\definecolor{currentstroke}{rgb}{0.000000,0.000000,0.000000}%
\pgfsetstrokecolor{currentstroke}%
\pgfsetdash{}{0pt}%
\pgfsys@defobject{currentmarker}{\pgfqpoint{0.000000in}{-0.020833in}}{\pgfqpoint{0.000000in}{0.000000in}}{%
\pgfpathmoveto{\pgfqpoint{0.000000in}{0.000000in}}%
\pgfpathlineto{\pgfqpoint{0.000000in}{-0.020833in}}%
\pgfusepath{stroke,fill}%
}%
\begin{pgfscope}%
\pgfsys@transformshift{1.744865in}{0.893003in}%
\pgfsys@useobject{currentmarker}{}%
\end{pgfscope}%
\end{pgfscope}%
\begin{pgfscope}%
\pgfsetbuttcap%
\pgfsetroundjoin%
\definecolor{currentfill}{rgb}{0.000000,0.000000,0.000000}%
\pgfsetfillcolor{currentfill}%
\pgfsetlinewidth{0.501875pt}%
\definecolor{currentstroke}{rgb}{0.000000,0.000000,0.000000}%
\pgfsetstrokecolor{currentstroke}%
\pgfsetdash{}{0pt}%
\pgfsys@defobject{currentmarker}{\pgfqpoint{0.000000in}{0.000000in}}{\pgfqpoint{0.000000in}{0.020833in}}{%
\pgfpathmoveto{\pgfqpoint{0.000000in}{0.000000in}}%
\pgfpathlineto{\pgfqpoint{0.000000in}{0.020833in}}%
\pgfusepath{stroke,fill}%
}%
\begin{pgfscope}%
\pgfsys@transformshift{1.780385in}{0.586309in}%
\pgfsys@useobject{currentmarker}{}%
\end{pgfscope}%
\end{pgfscope}%
\begin{pgfscope}%
\pgfsetbuttcap%
\pgfsetroundjoin%
\definecolor{currentfill}{rgb}{0.000000,0.000000,0.000000}%
\pgfsetfillcolor{currentfill}%
\pgfsetlinewidth{0.501875pt}%
\definecolor{currentstroke}{rgb}{0.000000,0.000000,0.000000}%
\pgfsetstrokecolor{currentstroke}%
\pgfsetdash{}{0pt}%
\pgfsys@defobject{currentmarker}{\pgfqpoint{0.000000in}{-0.020833in}}{\pgfqpoint{0.000000in}{0.000000in}}{%
\pgfpathmoveto{\pgfqpoint{0.000000in}{0.000000in}}%
\pgfpathlineto{\pgfqpoint{0.000000in}{-0.020833in}}%
\pgfusepath{stroke,fill}%
}%
\begin{pgfscope}%
\pgfsys@transformshift{1.780385in}{0.893003in}%
\pgfsys@useobject{currentmarker}{}%
\end{pgfscope}%
\end{pgfscope}%
\begin{pgfscope}%
\pgfsetbuttcap%
\pgfsetroundjoin%
\definecolor{currentfill}{rgb}{0.000000,0.000000,0.000000}%
\pgfsetfillcolor{currentfill}%
\pgfsetlinewidth{0.501875pt}%
\definecolor{currentstroke}{rgb}{0.000000,0.000000,0.000000}%
\pgfsetstrokecolor{currentstroke}%
\pgfsetdash{}{0pt}%
\pgfsys@defobject{currentmarker}{\pgfqpoint{0.000000in}{0.000000in}}{\pgfqpoint{0.000000in}{0.020833in}}{%
\pgfpathmoveto{\pgfqpoint{0.000000in}{0.000000in}}%
\pgfpathlineto{\pgfqpoint{0.000000in}{0.020833in}}%
\pgfusepath{stroke,fill}%
}%
\begin{pgfscope}%
\pgfsys@transformshift{1.815906in}{0.586309in}%
\pgfsys@useobject{currentmarker}{}%
\end{pgfscope}%
\end{pgfscope}%
\begin{pgfscope}%
\pgfsetbuttcap%
\pgfsetroundjoin%
\definecolor{currentfill}{rgb}{0.000000,0.000000,0.000000}%
\pgfsetfillcolor{currentfill}%
\pgfsetlinewidth{0.501875pt}%
\definecolor{currentstroke}{rgb}{0.000000,0.000000,0.000000}%
\pgfsetstrokecolor{currentstroke}%
\pgfsetdash{}{0pt}%
\pgfsys@defobject{currentmarker}{\pgfqpoint{0.000000in}{-0.020833in}}{\pgfqpoint{0.000000in}{0.000000in}}{%
\pgfpathmoveto{\pgfqpoint{0.000000in}{0.000000in}}%
\pgfpathlineto{\pgfqpoint{0.000000in}{-0.020833in}}%
\pgfusepath{stroke,fill}%
}%
\begin{pgfscope}%
\pgfsys@transformshift{1.815906in}{0.893003in}%
\pgfsys@useobject{currentmarker}{}%
\end{pgfscope}%
\end{pgfscope}%
\begin{pgfscope}%
\pgfsetbuttcap%
\pgfsetroundjoin%
\definecolor{currentfill}{rgb}{0.000000,0.000000,0.000000}%
\pgfsetfillcolor{currentfill}%
\pgfsetlinewidth{0.501875pt}%
\definecolor{currentstroke}{rgb}{0.000000,0.000000,0.000000}%
\pgfsetstrokecolor{currentstroke}%
\pgfsetdash{}{0pt}%
\pgfsys@defobject{currentmarker}{\pgfqpoint{0.000000in}{0.000000in}}{\pgfqpoint{0.000000in}{0.020833in}}{%
\pgfpathmoveto{\pgfqpoint{0.000000in}{0.000000in}}%
\pgfpathlineto{\pgfqpoint{0.000000in}{0.020833in}}%
\pgfusepath{stroke,fill}%
}%
\begin{pgfscope}%
\pgfsys@transformshift{1.851426in}{0.586309in}%
\pgfsys@useobject{currentmarker}{}%
\end{pgfscope}%
\end{pgfscope}%
\begin{pgfscope}%
\pgfsetbuttcap%
\pgfsetroundjoin%
\definecolor{currentfill}{rgb}{0.000000,0.000000,0.000000}%
\pgfsetfillcolor{currentfill}%
\pgfsetlinewidth{0.501875pt}%
\definecolor{currentstroke}{rgb}{0.000000,0.000000,0.000000}%
\pgfsetstrokecolor{currentstroke}%
\pgfsetdash{}{0pt}%
\pgfsys@defobject{currentmarker}{\pgfqpoint{0.000000in}{-0.020833in}}{\pgfqpoint{0.000000in}{0.000000in}}{%
\pgfpathmoveto{\pgfqpoint{0.000000in}{0.000000in}}%
\pgfpathlineto{\pgfqpoint{0.000000in}{-0.020833in}}%
\pgfusepath{stroke,fill}%
}%
\begin{pgfscope}%
\pgfsys@transformshift{1.851426in}{0.893003in}%
\pgfsys@useobject{currentmarker}{}%
\end{pgfscope}%
\end{pgfscope}%
\begin{pgfscope}%
\pgfsetbuttcap%
\pgfsetroundjoin%
\definecolor{currentfill}{rgb}{0.000000,0.000000,0.000000}%
\pgfsetfillcolor{currentfill}%
\pgfsetlinewidth{0.501875pt}%
\definecolor{currentstroke}{rgb}{0.000000,0.000000,0.000000}%
\pgfsetstrokecolor{currentstroke}%
\pgfsetdash{}{0pt}%
\pgfsys@defobject{currentmarker}{\pgfqpoint{0.000000in}{0.000000in}}{\pgfqpoint{0.000000in}{0.020833in}}{%
\pgfpathmoveto{\pgfqpoint{0.000000in}{0.000000in}}%
\pgfpathlineto{\pgfqpoint{0.000000in}{0.020833in}}%
\pgfusepath{stroke,fill}%
}%
\begin{pgfscope}%
\pgfsys@transformshift{1.886947in}{0.586309in}%
\pgfsys@useobject{currentmarker}{}%
\end{pgfscope}%
\end{pgfscope}%
\begin{pgfscope}%
\pgfsetbuttcap%
\pgfsetroundjoin%
\definecolor{currentfill}{rgb}{0.000000,0.000000,0.000000}%
\pgfsetfillcolor{currentfill}%
\pgfsetlinewidth{0.501875pt}%
\definecolor{currentstroke}{rgb}{0.000000,0.000000,0.000000}%
\pgfsetstrokecolor{currentstroke}%
\pgfsetdash{}{0pt}%
\pgfsys@defobject{currentmarker}{\pgfqpoint{0.000000in}{-0.020833in}}{\pgfqpoint{0.000000in}{0.000000in}}{%
\pgfpathmoveto{\pgfqpoint{0.000000in}{0.000000in}}%
\pgfpathlineto{\pgfqpoint{0.000000in}{-0.020833in}}%
\pgfusepath{stroke,fill}%
}%
\begin{pgfscope}%
\pgfsys@transformshift{1.886947in}{0.893003in}%
\pgfsys@useobject{currentmarker}{}%
\end{pgfscope}%
\end{pgfscope}%
\begin{pgfscope}%
\pgfsetbuttcap%
\pgfsetroundjoin%
\definecolor{currentfill}{rgb}{0.000000,0.000000,0.000000}%
\pgfsetfillcolor{currentfill}%
\pgfsetlinewidth{0.501875pt}%
\definecolor{currentstroke}{rgb}{0.000000,0.000000,0.000000}%
\pgfsetstrokecolor{currentstroke}%
\pgfsetdash{}{0pt}%
\pgfsys@defobject{currentmarker}{\pgfqpoint{0.000000in}{0.000000in}}{\pgfqpoint{0.000000in}{0.020833in}}{%
\pgfpathmoveto{\pgfqpoint{0.000000in}{0.000000in}}%
\pgfpathlineto{\pgfqpoint{0.000000in}{0.020833in}}%
\pgfusepath{stroke,fill}%
}%
\begin{pgfscope}%
\pgfsys@transformshift{1.957988in}{0.586309in}%
\pgfsys@useobject{currentmarker}{}%
\end{pgfscope}%
\end{pgfscope}%
\begin{pgfscope}%
\pgfsetbuttcap%
\pgfsetroundjoin%
\definecolor{currentfill}{rgb}{0.000000,0.000000,0.000000}%
\pgfsetfillcolor{currentfill}%
\pgfsetlinewidth{0.501875pt}%
\definecolor{currentstroke}{rgb}{0.000000,0.000000,0.000000}%
\pgfsetstrokecolor{currentstroke}%
\pgfsetdash{}{0pt}%
\pgfsys@defobject{currentmarker}{\pgfqpoint{0.000000in}{-0.020833in}}{\pgfqpoint{0.000000in}{0.000000in}}{%
\pgfpathmoveto{\pgfqpoint{0.000000in}{0.000000in}}%
\pgfpathlineto{\pgfqpoint{0.000000in}{-0.020833in}}%
\pgfusepath{stroke,fill}%
}%
\begin{pgfscope}%
\pgfsys@transformshift{1.957988in}{0.893003in}%
\pgfsys@useobject{currentmarker}{}%
\end{pgfscope}%
\end{pgfscope}%
\begin{pgfscope}%
\pgfsetbuttcap%
\pgfsetroundjoin%
\definecolor{currentfill}{rgb}{0.000000,0.000000,0.000000}%
\pgfsetfillcolor{currentfill}%
\pgfsetlinewidth{0.501875pt}%
\definecolor{currentstroke}{rgb}{0.000000,0.000000,0.000000}%
\pgfsetstrokecolor{currentstroke}%
\pgfsetdash{}{0pt}%
\pgfsys@defobject{currentmarker}{\pgfqpoint{0.000000in}{0.000000in}}{\pgfqpoint{0.000000in}{0.020833in}}{%
\pgfpathmoveto{\pgfqpoint{0.000000in}{0.000000in}}%
\pgfpathlineto{\pgfqpoint{0.000000in}{0.020833in}}%
\pgfusepath{stroke,fill}%
}%
\begin{pgfscope}%
\pgfsys@transformshift{1.993508in}{0.586309in}%
\pgfsys@useobject{currentmarker}{}%
\end{pgfscope}%
\end{pgfscope}%
\begin{pgfscope}%
\pgfsetbuttcap%
\pgfsetroundjoin%
\definecolor{currentfill}{rgb}{0.000000,0.000000,0.000000}%
\pgfsetfillcolor{currentfill}%
\pgfsetlinewidth{0.501875pt}%
\definecolor{currentstroke}{rgb}{0.000000,0.000000,0.000000}%
\pgfsetstrokecolor{currentstroke}%
\pgfsetdash{}{0pt}%
\pgfsys@defobject{currentmarker}{\pgfqpoint{0.000000in}{-0.020833in}}{\pgfqpoint{0.000000in}{0.000000in}}{%
\pgfpathmoveto{\pgfqpoint{0.000000in}{0.000000in}}%
\pgfpathlineto{\pgfqpoint{0.000000in}{-0.020833in}}%
\pgfusepath{stroke,fill}%
}%
\begin{pgfscope}%
\pgfsys@transformshift{1.993508in}{0.893003in}%
\pgfsys@useobject{currentmarker}{}%
\end{pgfscope}%
\end{pgfscope}%
\begin{pgfscope}%
\pgfsetbuttcap%
\pgfsetroundjoin%
\definecolor{currentfill}{rgb}{0.000000,0.000000,0.000000}%
\pgfsetfillcolor{currentfill}%
\pgfsetlinewidth{0.501875pt}%
\definecolor{currentstroke}{rgb}{0.000000,0.000000,0.000000}%
\pgfsetstrokecolor{currentstroke}%
\pgfsetdash{}{0pt}%
\pgfsys@defobject{currentmarker}{\pgfqpoint{0.000000in}{0.000000in}}{\pgfqpoint{0.000000in}{0.020833in}}{%
\pgfpathmoveto{\pgfqpoint{0.000000in}{0.000000in}}%
\pgfpathlineto{\pgfqpoint{0.000000in}{0.020833in}}%
\pgfusepath{stroke,fill}%
}%
\begin{pgfscope}%
\pgfsys@transformshift{2.029029in}{0.586309in}%
\pgfsys@useobject{currentmarker}{}%
\end{pgfscope}%
\end{pgfscope}%
\begin{pgfscope}%
\pgfsetbuttcap%
\pgfsetroundjoin%
\definecolor{currentfill}{rgb}{0.000000,0.000000,0.000000}%
\pgfsetfillcolor{currentfill}%
\pgfsetlinewidth{0.501875pt}%
\definecolor{currentstroke}{rgb}{0.000000,0.000000,0.000000}%
\pgfsetstrokecolor{currentstroke}%
\pgfsetdash{}{0pt}%
\pgfsys@defobject{currentmarker}{\pgfqpoint{0.000000in}{-0.020833in}}{\pgfqpoint{0.000000in}{0.000000in}}{%
\pgfpathmoveto{\pgfqpoint{0.000000in}{0.000000in}}%
\pgfpathlineto{\pgfqpoint{0.000000in}{-0.020833in}}%
\pgfusepath{stroke,fill}%
}%
\begin{pgfscope}%
\pgfsys@transformshift{2.029029in}{0.893003in}%
\pgfsys@useobject{currentmarker}{}%
\end{pgfscope}%
\end{pgfscope}%
\begin{pgfscope}%
\pgfsetbuttcap%
\pgfsetroundjoin%
\definecolor{currentfill}{rgb}{0.000000,0.000000,0.000000}%
\pgfsetfillcolor{currentfill}%
\pgfsetlinewidth{0.501875pt}%
\definecolor{currentstroke}{rgb}{0.000000,0.000000,0.000000}%
\pgfsetstrokecolor{currentstroke}%
\pgfsetdash{}{0pt}%
\pgfsys@defobject{currentmarker}{\pgfqpoint{0.000000in}{0.000000in}}{\pgfqpoint{0.000000in}{0.020833in}}{%
\pgfpathmoveto{\pgfqpoint{0.000000in}{0.000000in}}%
\pgfpathlineto{\pgfqpoint{0.000000in}{0.020833in}}%
\pgfusepath{stroke,fill}%
}%
\begin{pgfscope}%
\pgfsys@transformshift{2.064550in}{0.586309in}%
\pgfsys@useobject{currentmarker}{}%
\end{pgfscope}%
\end{pgfscope}%
\begin{pgfscope}%
\pgfsetbuttcap%
\pgfsetroundjoin%
\definecolor{currentfill}{rgb}{0.000000,0.000000,0.000000}%
\pgfsetfillcolor{currentfill}%
\pgfsetlinewidth{0.501875pt}%
\definecolor{currentstroke}{rgb}{0.000000,0.000000,0.000000}%
\pgfsetstrokecolor{currentstroke}%
\pgfsetdash{}{0pt}%
\pgfsys@defobject{currentmarker}{\pgfqpoint{0.000000in}{-0.020833in}}{\pgfqpoint{0.000000in}{0.000000in}}{%
\pgfpathmoveto{\pgfqpoint{0.000000in}{0.000000in}}%
\pgfpathlineto{\pgfqpoint{0.000000in}{-0.020833in}}%
\pgfusepath{stroke,fill}%
}%
\begin{pgfscope}%
\pgfsys@transformshift{2.064550in}{0.893003in}%
\pgfsys@useobject{currentmarker}{}%
\end{pgfscope}%
\end{pgfscope}%
\begin{pgfscope}%
\pgfsetbuttcap%
\pgfsetroundjoin%
\definecolor{currentfill}{rgb}{0.000000,0.000000,0.000000}%
\pgfsetfillcolor{currentfill}%
\pgfsetlinewidth{0.501875pt}%
\definecolor{currentstroke}{rgb}{0.000000,0.000000,0.000000}%
\pgfsetstrokecolor{currentstroke}%
\pgfsetdash{}{0pt}%
\pgfsys@defobject{currentmarker}{\pgfqpoint{0.000000in}{0.000000in}}{\pgfqpoint{0.000000in}{0.020833in}}{%
\pgfpathmoveto{\pgfqpoint{0.000000in}{0.000000in}}%
\pgfpathlineto{\pgfqpoint{0.000000in}{0.020833in}}%
\pgfusepath{stroke,fill}%
}%
\begin{pgfscope}%
\pgfsys@transformshift{2.100070in}{0.586309in}%
\pgfsys@useobject{currentmarker}{}%
\end{pgfscope}%
\end{pgfscope}%
\begin{pgfscope}%
\pgfsetbuttcap%
\pgfsetroundjoin%
\definecolor{currentfill}{rgb}{0.000000,0.000000,0.000000}%
\pgfsetfillcolor{currentfill}%
\pgfsetlinewidth{0.501875pt}%
\definecolor{currentstroke}{rgb}{0.000000,0.000000,0.000000}%
\pgfsetstrokecolor{currentstroke}%
\pgfsetdash{}{0pt}%
\pgfsys@defobject{currentmarker}{\pgfqpoint{0.000000in}{-0.020833in}}{\pgfqpoint{0.000000in}{0.000000in}}{%
\pgfpathmoveto{\pgfqpoint{0.000000in}{0.000000in}}%
\pgfpathlineto{\pgfqpoint{0.000000in}{-0.020833in}}%
\pgfusepath{stroke,fill}%
}%
\begin{pgfscope}%
\pgfsys@transformshift{2.100070in}{0.893003in}%
\pgfsys@useobject{currentmarker}{}%
\end{pgfscope}%
\end{pgfscope}%
\begin{pgfscope}%
\pgfsetbuttcap%
\pgfsetroundjoin%
\definecolor{currentfill}{rgb}{0.000000,0.000000,0.000000}%
\pgfsetfillcolor{currentfill}%
\pgfsetlinewidth{0.501875pt}%
\definecolor{currentstroke}{rgb}{0.000000,0.000000,0.000000}%
\pgfsetstrokecolor{currentstroke}%
\pgfsetdash{}{0pt}%
\pgfsys@defobject{currentmarker}{\pgfqpoint{0.000000in}{0.000000in}}{\pgfqpoint{0.000000in}{0.020833in}}{%
\pgfpathmoveto{\pgfqpoint{0.000000in}{0.000000in}}%
\pgfpathlineto{\pgfqpoint{0.000000in}{0.020833in}}%
\pgfusepath{stroke,fill}%
}%
\begin{pgfscope}%
\pgfsys@transformshift{2.135591in}{0.586309in}%
\pgfsys@useobject{currentmarker}{}%
\end{pgfscope}%
\end{pgfscope}%
\begin{pgfscope}%
\pgfsetbuttcap%
\pgfsetroundjoin%
\definecolor{currentfill}{rgb}{0.000000,0.000000,0.000000}%
\pgfsetfillcolor{currentfill}%
\pgfsetlinewidth{0.501875pt}%
\definecolor{currentstroke}{rgb}{0.000000,0.000000,0.000000}%
\pgfsetstrokecolor{currentstroke}%
\pgfsetdash{}{0pt}%
\pgfsys@defobject{currentmarker}{\pgfqpoint{0.000000in}{-0.020833in}}{\pgfqpoint{0.000000in}{0.000000in}}{%
\pgfpathmoveto{\pgfqpoint{0.000000in}{0.000000in}}%
\pgfpathlineto{\pgfqpoint{0.000000in}{-0.020833in}}%
\pgfusepath{stroke,fill}%
}%
\begin{pgfscope}%
\pgfsys@transformshift{2.135591in}{0.893003in}%
\pgfsys@useobject{currentmarker}{}%
\end{pgfscope}%
\end{pgfscope}%
\begin{pgfscope}%
\pgfsetbuttcap%
\pgfsetroundjoin%
\definecolor{currentfill}{rgb}{0.000000,0.000000,0.000000}%
\pgfsetfillcolor{currentfill}%
\pgfsetlinewidth{0.501875pt}%
\definecolor{currentstroke}{rgb}{0.000000,0.000000,0.000000}%
\pgfsetstrokecolor{currentstroke}%
\pgfsetdash{}{0pt}%
\pgfsys@defobject{currentmarker}{\pgfqpoint{0.000000in}{0.000000in}}{\pgfqpoint{0.000000in}{0.020833in}}{%
\pgfpathmoveto{\pgfqpoint{0.000000in}{0.000000in}}%
\pgfpathlineto{\pgfqpoint{0.000000in}{0.020833in}}%
\pgfusepath{stroke,fill}%
}%
\begin{pgfscope}%
\pgfsys@transformshift{2.171111in}{0.586309in}%
\pgfsys@useobject{currentmarker}{}%
\end{pgfscope}%
\end{pgfscope}%
\begin{pgfscope}%
\pgfsetbuttcap%
\pgfsetroundjoin%
\definecolor{currentfill}{rgb}{0.000000,0.000000,0.000000}%
\pgfsetfillcolor{currentfill}%
\pgfsetlinewidth{0.501875pt}%
\definecolor{currentstroke}{rgb}{0.000000,0.000000,0.000000}%
\pgfsetstrokecolor{currentstroke}%
\pgfsetdash{}{0pt}%
\pgfsys@defobject{currentmarker}{\pgfqpoint{0.000000in}{-0.020833in}}{\pgfqpoint{0.000000in}{0.000000in}}{%
\pgfpathmoveto{\pgfqpoint{0.000000in}{0.000000in}}%
\pgfpathlineto{\pgfqpoint{0.000000in}{-0.020833in}}%
\pgfusepath{stroke,fill}%
}%
\begin{pgfscope}%
\pgfsys@transformshift{2.171111in}{0.893003in}%
\pgfsys@useobject{currentmarker}{}%
\end{pgfscope}%
\end{pgfscope}%
\begin{pgfscope}%
\pgfsetbuttcap%
\pgfsetroundjoin%
\definecolor{currentfill}{rgb}{0.000000,0.000000,0.000000}%
\pgfsetfillcolor{currentfill}%
\pgfsetlinewidth{0.501875pt}%
\definecolor{currentstroke}{rgb}{0.000000,0.000000,0.000000}%
\pgfsetstrokecolor{currentstroke}%
\pgfsetdash{}{0pt}%
\pgfsys@defobject{currentmarker}{\pgfqpoint{0.000000in}{0.000000in}}{\pgfqpoint{0.000000in}{0.020833in}}{%
\pgfpathmoveto{\pgfqpoint{0.000000in}{0.000000in}}%
\pgfpathlineto{\pgfqpoint{0.000000in}{0.020833in}}%
\pgfusepath{stroke,fill}%
}%
\begin{pgfscope}%
\pgfsys@transformshift{2.206632in}{0.586309in}%
\pgfsys@useobject{currentmarker}{}%
\end{pgfscope}%
\end{pgfscope}%
\begin{pgfscope}%
\pgfsetbuttcap%
\pgfsetroundjoin%
\definecolor{currentfill}{rgb}{0.000000,0.000000,0.000000}%
\pgfsetfillcolor{currentfill}%
\pgfsetlinewidth{0.501875pt}%
\definecolor{currentstroke}{rgb}{0.000000,0.000000,0.000000}%
\pgfsetstrokecolor{currentstroke}%
\pgfsetdash{}{0pt}%
\pgfsys@defobject{currentmarker}{\pgfqpoint{0.000000in}{-0.020833in}}{\pgfqpoint{0.000000in}{0.000000in}}{%
\pgfpathmoveto{\pgfqpoint{0.000000in}{0.000000in}}%
\pgfpathlineto{\pgfqpoint{0.000000in}{-0.020833in}}%
\pgfusepath{stroke,fill}%
}%
\begin{pgfscope}%
\pgfsys@transformshift{2.206632in}{0.893003in}%
\pgfsys@useobject{currentmarker}{}%
\end{pgfscope}%
\end{pgfscope}%
\begin{pgfscope}%
\pgfsetbuttcap%
\pgfsetroundjoin%
\definecolor{currentfill}{rgb}{0.000000,0.000000,0.000000}%
\pgfsetfillcolor{currentfill}%
\pgfsetlinewidth{0.501875pt}%
\definecolor{currentstroke}{rgb}{0.000000,0.000000,0.000000}%
\pgfsetstrokecolor{currentstroke}%
\pgfsetdash{}{0pt}%
\pgfsys@defobject{currentmarker}{\pgfqpoint{0.000000in}{0.000000in}}{\pgfqpoint{0.000000in}{0.020833in}}{%
\pgfpathmoveto{\pgfqpoint{0.000000in}{0.000000in}}%
\pgfpathlineto{\pgfqpoint{0.000000in}{0.020833in}}%
\pgfusepath{stroke,fill}%
}%
\begin{pgfscope}%
\pgfsys@transformshift{2.242152in}{0.586309in}%
\pgfsys@useobject{currentmarker}{}%
\end{pgfscope}%
\end{pgfscope}%
\begin{pgfscope}%
\pgfsetbuttcap%
\pgfsetroundjoin%
\definecolor{currentfill}{rgb}{0.000000,0.000000,0.000000}%
\pgfsetfillcolor{currentfill}%
\pgfsetlinewidth{0.501875pt}%
\definecolor{currentstroke}{rgb}{0.000000,0.000000,0.000000}%
\pgfsetstrokecolor{currentstroke}%
\pgfsetdash{}{0pt}%
\pgfsys@defobject{currentmarker}{\pgfqpoint{0.000000in}{-0.020833in}}{\pgfqpoint{0.000000in}{0.000000in}}{%
\pgfpathmoveto{\pgfqpoint{0.000000in}{0.000000in}}%
\pgfpathlineto{\pgfqpoint{0.000000in}{-0.020833in}}%
\pgfusepath{stroke,fill}%
}%
\begin{pgfscope}%
\pgfsys@transformshift{2.242152in}{0.893003in}%
\pgfsys@useobject{currentmarker}{}%
\end{pgfscope}%
\end{pgfscope}%
\begin{pgfscope}%
\pgfsetbuttcap%
\pgfsetroundjoin%
\definecolor{currentfill}{rgb}{0.000000,0.000000,0.000000}%
\pgfsetfillcolor{currentfill}%
\pgfsetlinewidth{0.501875pt}%
\definecolor{currentstroke}{rgb}{0.000000,0.000000,0.000000}%
\pgfsetstrokecolor{currentstroke}%
\pgfsetdash{}{0pt}%
\pgfsys@defobject{currentmarker}{\pgfqpoint{0.000000in}{0.000000in}}{\pgfqpoint{0.000000in}{0.020833in}}{%
\pgfpathmoveto{\pgfqpoint{0.000000in}{0.000000in}}%
\pgfpathlineto{\pgfqpoint{0.000000in}{0.020833in}}%
\pgfusepath{stroke,fill}%
}%
\begin{pgfscope}%
\pgfsys@transformshift{2.277673in}{0.586309in}%
\pgfsys@useobject{currentmarker}{}%
\end{pgfscope}%
\end{pgfscope}%
\begin{pgfscope}%
\pgfsetbuttcap%
\pgfsetroundjoin%
\definecolor{currentfill}{rgb}{0.000000,0.000000,0.000000}%
\pgfsetfillcolor{currentfill}%
\pgfsetlinewidth{0.501875pt}%
\definecolor{currentstroke}{rgb}{0.000000,0.000000,0.000000}%
\pgfsetstrokecolor{currentstroke}%
\pgfsetdash{}{0pt}%
\pgfsys@defobject{currentmarker}{\pgfqpoint{0.000000in}{-0.020833in}}{\pgfqpoint{0.000000in}{0.000000in}}{%
\pgfpathmoveto{\pgfqpoint{0.000000in}{0.000000in}}%
\pgfpathlineto{\pgfqpoint{0.000000in}{-0.020833in}}%
\pgfusepath{stroke,fill}%
}%
\begin{pgfscope}%
\pgfsys@transformshift{2.277673in}{0.893003in}%
\pgfsys@useobject{currentmarker}{}%
\end{pgfscope}%
\end{pgfscope}%
\begin{pgfscope}%
\pgfsetbuttcap%
\pgfsetroundjoin%
\definecolor{currentfill}{rgb}{0.000000,0.000000,0.000000}%
\pgfsetfillcolor{currentfill}%
\pgfsetlinewidth{0.501875pt}%
\definecolor{currentstroke}{rgb}{0.000000,0.000000,0.000000}%
\pgfsetstrokecolor{currentstroke}%
\pgfsetdash{}{0pt}%
\pgfsys@defobject{currentmarker}{\pgfqpoint{0.000000in}{0.000000in}}{\pgfqpoint{0.000000in}{0.020833in}}{%
\pgfpathmoveto{\pgfqpoint{0.000000in}{0.000000in}}%
\pgfpathlineto{\pgfqpoint{0.000000in}{0.020833in}}%
\pgfusepath{stroke,fill}%
}%
\begin{pgfscope}%
\pgfsys@transformshift{2.313193in}{0.586309in}%
\pgfsys@useobject{currentmarker}{}%
\end{pgfscope}%
\end{pgfscope}%
\begin{pgfscope}%
\pgfsetbuttcap%
\pgfsetroundjoin%
\definecolor{currentfill}{rgb}{0.000000,0.000000,0.000000}%
\pgfsetfillcolor{currentfill}%
\pgfsetlinewidth{0.501875pt}%
\definecolor{currentstroke}{rgb}{0.000000,0.000000,0.000000}%
\pgfsetstrokecolor{currentstroke}%
\pgfsetdash{}{0pt}%
\pgfsys@defobject{currentmarker}{\pgfqpoint{0.000000in}{-0.020833in}}{\pgfqpoint{0.000000in}{0.000000in}}{%
\pgfpathmoveto{\pgfqpoint{0.000000in}{0.000000in}}%
\pgfpathlineto{\pgfqpoint{0.000000in}{-0.020833in}}%
\pgfusepath{stroke,fill}%
}%
\begin{pgfscope}%
\pgfsys@transformshift{2.313193in}{0.893003in}%
\pgfsys@useobject{currentmarker}{}%
\end{pgfscope}%
\end{pgfscope}%
\begin{pgfscope}%
\pgfsetbuttcap%
\pgfsetroundjoin%
\definecolor{currentfill}{rgb}{0.000000,0.000000,0.000000}%
\pgfsetfillcolor{currentfill}%
\pgfsetlinewidth{0.501875pt}%
\definecolor{currentstroke}{rgb}{0.000000,0.000000,0.000000}%
\pgfsetstrokecolor{currentstroke}%
\pgfsetdash{}{0pt}%
\pgfsys@defobject{currentmarker}{\pgfqpoint{0.000000in}{0.000000in}}{\pgfqpoint{0.000000in}{0.020833in}}{%
\pgfpathmoveto{\pgfqpoint{0.000000in}{0.000000in}}%
\pgfpathlineto{\pgfqpoint{0.000000in}{0.020833in}}%
\pgfusepath{stroke,fill}%
}%
\begin{pgfscope}%
\pgfsys@transformshift{2.384234in}{0.586309in}%
\pgfsys@useobject{currentmarker}{}%
\end{pgfscope}%
\end{pgfscope}%
\begin{pgfscope}%
\pgfsetbuttcap%
\pgfsetroundjoin%
\definecolor{currentfill}{rgb}{0.000000,0.000000,0.000000}%
\pgfsetfillcolor{currentfill}%
\pgfsetlinewidth{0.501875pt}%
\definecolor{currentstroke}{rgb}{0.000000,0.000000,0.000000}%
\pgfsetstrokecolor{currentstroke}%
\pgfsetdash{}{0pt}%
\pgfsys@defobject{currentmarker}{\pgfqpoint{0.000000in}{-0.020833in}}{\pgfqpoint{0.000000in}{0.000000in}}{%
\pgfpathmoveto{\pgfqpoint{0.000000in}{0.000000in}}%
\pgfpathlineto{\pgfqpoint{0.000000in}{-0.020833in}}%
\pgfusepath{stroke,fill}%
}%
\begin{pgfscope}%
\pgfsys@transformshift{2.384234in}{0.893003in}%
\pgfsys@useobject{currentmarker}{}%
\end{pgfscope}%
\end{pgfscope}%
\begin{pgfscope}%
\pgfsetbuttcap%
\pgfsetroundjoin%
\definecolor{currentfill}{rgb}{0.000000,0.000000,0.000000}%
\pgfsetfillcolor{currentfill}%
\pgfsetlinewidth{0.501875pt}%
\definecolor{currentstroke}{rgb}{0.000000,0.000000,0.000000}%
\pgfsetstrokecolor{currentstroke}%
\pgfsetdash{}{0pt}%
\pgfsys@defobject{currentmarker}{\pgfqpoint{0.000000in}{0.000000in}}{\pgfqpoint{0.000000in}{0.020833in}}{%
\pgfpathmoveto{\pgfqpoint{0.000000in}{0.000000in}}%
\pgfpathlineto{\pgfqpoint{0.000000in}{0.020833in}}%
\pgfusepath{stroke,fill}%
}%
\begin{pgfscope}%
\pgfsys@transformshift{2.419755in}{0.586309in}%
\pgfsys@useobject{currentmarker}{}%
\end{pgfscope}%
\end{pgfscope}%
\begin{pgfscope}%
\pgfsetbuttcap%
\pgfsetroundjoin%
\definecolor{currentfill}{rgb}{0.000000,0.000000,0.000000}%
\pgfsetfillcolor{currentfill}%
\pgfsetlinewidth{0.501875pt}%
\definecolor{currentstroke}{rgb}{0.000000,0.000000,0.000000}%
\pgfsetstrokecolor{currentstroke}%
\pgfsetdash{}{0pt}%
\pgfsys@defobject{currentmarker}{\pgfqpoint{0.000000in}{-0.020833in}}{\pgfqpoint{0.000000in}{0.000000in}}{%
\pgfpathmoveto{\pgfqpoint{0.000000in}{0.000000in}}%
\pgfpathlineto{\pgfqpoint{0.000000in}{-0.020833in}}%
\pgfusepath{stroke,fill}%
}%
\begin{pgfscope}%
\pgfsys@transformshift{2.419755in}{0.893003in}%
\pgfsys@useobject{currentmarker}{}%
\end{pgfscope}%
\end{pgfscope}%
\begin{pgfscope}%
\pgfsetbuttcap%
\pgfsetroundjoin%
\definecolor{currentfill}{rgb}{0.000000,0.000000,0.000000}%
\pgfsetfillcolor{currentfill}%
\pgfsetlinewidth{0.501875pt}%
\definecolor{currentstroke}{rgb}{0.000000,0.000000,0.000000}%
\pgfsetstrokecolor{currentstroke}%
\pgfsetdash{}{0pt}%
\pgfsys@defobject{currentmarker}{\pgfqpoint{0.000000in}{0.000000in}}{\pgfqpoint{0.000000in}{0.020833in}}{%
\pgfpathmoveto{\pgfqpoint{0.000000in}{0.000000in}}%
\pgfpathlineto{\pgfqpoint{0.000000in}{0.020833in}}%
\pgfusepath{stroke,fill}%
}%
\begin{pgfscope}%
\pgfsys@transformshift{2.455275in}{0.586309in}%
\pgfsys@useobject{currentmarker}{}%
\end{pgfscope}%
\end{pgfscope}%
\begin{pgfscope}%
\pgfsetbuttcap%
\pgfsetroundjoin%
\definecolor{currentfill}{rgb}{0.000000,0.000000,0.000000}%
\pgfsetfillcolor{currentfill}%
\pgfsetlinewidth{0.501875pt}%
\definecolor{currentstroke}{rgb}{0.000000,0.000000,0.000000}%
\pgfsetstrokecolor{currentstroke}%
\pgfsetdash{}{0pt}%
\pgfsys@defobject{currentmarker}{\pgfqpoint{0.000000in}{-0.020833in}}{\pgfqpoint{0.000000in}{0.000000in}}{%
\pgfpathmoveto{\pgfqpoint{0.000000in}{0.000000in}}%
\pgfpathlineto{\pgfqpoint{0.000000in}{-0.020833in}}%
\pgfusepath{stroke,fill}%
}%
\begin{pgfscope}%
\pgfsys@transformshift{2.455275in}{0.893003in}%
\pgfsys@useobject{currentmarker}{}%
\end{pgfscope}%
\end{pgfscope}%
\begin{pgfscope}%
\pgfsetbuttcap%
\pgfsetroundjoin%
\definecolor{currentfill}{rgb}{0.000000,0.000000,0.000000}%
\pgfsetfillcolor{currentfill}%
\pgfsetlinewidth{0.501875pt}%
\definecolor{currentstroke}{rgb}{0.000000,0.000000,0.000000}%
\pgfsetstrokecolor{currentstroke}%
\pgfsetdash{}{0pt}%
\pgfsys@defobject{currentmarker}{\pgfqpoint{0.000000in}{0.000000in}}{\pgfqpoint{0.000000in}{0.020833in}}{%
\pgfpathmoveto{\pgfqpoint{0.000000in}{0.000000in}}%
\pgfpathlineto{\pgfqpoint{0.000000in}{0.020833in}}%
\pgfusepath{stroke,fill}%
}%
\begin{pgfscope}%
\pgfsys@transformshift{2.490796in}{0.586309in}%
\pgfsys@useobject{currentmarker}{}%
\end{pgfscope}%
\end{pgfscope}%
\begin{pgfscope}%
\pgfsetbuttcap%
\pgfsetroundjoin%
\definecolor{currentfill}{rgb}{0.000000,0.000000,0.000000}%
\pgfsetfillcolor{currentfill}%
\pgfsetlinewidth{0.501875pt}%
\definecolor{currentstroke}{rgb}{0.000000,0.000000,0.000000}%
\pgfsetstrokecolor{currentstroke}%
\pgfsetdash{}{0pt}%
\pgfsys@defobject{currentmarker}{\pgfqpoint{0.000000in}{-0.020833in}}{\pgfqpoint{0.000000in}{0.000000in}}{%
\pgfpathmoveto{\pgfqpoint{0.000000in}{0.000000in}}%
\pgfpathlineto{\pgfqpoint{0.000000in}{-0.020833in}}%
\pgfusepath{stroke,fill}%
}%
\begin{pgfscope}%
\pgfsys@transformshift{2.490796in}{0.893003in}%
\pgfsys@useobject{currentmarker}{}%
\end{pgfscope}%
\end{pgfscope}%
\begin{pgfscope}%
\pgfsetbuttcap%
\pgfsetroundjoin%
\definecolor{currentfill}{rgb}{0.000000,0.000000,0.000000}%
\pgfsetfillcolor{currentfill}%
\pgfsetlinewidth{0.501875pt}%
\definecolor{currentstroke}{rgb}{0.000000,0.000000,0.000000}%
\pgfsetstrokecolor{currentstroke}%
\pgfsetdash{}{0pt}%
\pgfsys@defobject{currentmarker}{\pgfqpoint{0.000000in}{0.000000in}}{\pgfqpoint{0.000000in}{0.020833in}}{%
\pgfpathmoveto{\pgfqpoint{0.000000in}{0.000000in}}%
\pgfpathlineto{\pgfqpoint{0.000000in}{0.020833in}}%
\pgfusepath{stroke,fill}%
}%
\begin{pgfscope}%
\pgfsys@transformshift{2.526316in}{0.586309in}%
\pgfsys@useobject{currentmarker}{}%
\end{pgfscope}%
\end{pgfscope}%
\begin{pgfscope}%
\pgfsetbuttcap%
\pgfsetroundjoin%
\definecolor{currentfill}{rgb}{0.000000,0.000000,0.000000}%
\pgfsetfillcolor{currentfill}%
\pgfsetlinewidth{0.501875pt}%
\definecolor{currentstroke}{rgb}{0.000000,0.000000,0.000000}%
\pgfsetstrokecolor{currentstroke}%
\pgfsetdash{}{0pt}%
\pgfsys@defobject{currentmarker}{\pgfqpoint{0.000000in}{-0.020833in}}{\pgfqpoint{0.000000in}{0.000000in}}{%
\pgfpathmoveto{\pgfqpoint{0.000000in}{0.000000in}}%
\pgfpathlineto{\pgfqpoint{0.000000in}{-0.020833in}}%
\pgfusepath{stroke,fill}%
}%
\begin{pgfscope}%
\pgfsys@transformshift{2.526316in}{0.893003in}%
\pgfsys@useobject{currentmarker}{}%
\end{pgfscope}%
\end{pgfscope}%
\begin{pgfscope}%
\pgfsetbuttcap%
\pgfsetroundjoin%
\definecolor{currentfill}{rgb}{0.000000,0.000000,0.000000}%
\pgfsetfillcolor{currentfill}%
\pgfsetlinewidth{0.501875pt}%
\definecolor{currentstroke}{rgb}{0.000000,0.000000,0.000000}%
\pgfsetstrokecolor{currentstroke}%
\pgfsetdash{}{0pt}%
\pgfsys@defobject{currentmarker}{\pgfqpoint{0.000000in}{0.000000in}}{\pgfqpoint{0.000000in}{0.020833in}}{%
\pgfpathmoveto{\pgfqpoint{0.000000in}{0.000000in}}%
\pgfpathlineto{\pgfqpoint{0.000000in}{0.020833in}}%
\pgfusepath{stroke,fill}%
}%
\begin{pgfscope}%
\pgfsys@transformshift{2.561837in}{0.586309in}%
\pgfsys@useobject{currentmarker}{}%
\end{pgfscope}%
\end{pgfscope}%
\begin{pgfscope}%
\pgfsetbuttcap%
\pgfsetroundjoin%
\definecolor{currentfill}{rgb}{0.000000,0.000000,0.000000}%
\pgfsetfillcolor{currentfill}%
\pgfsetlinewidth{0.501875pt}%
\definecolor{currentstroke}{rgb}{0.000000,0.000000,0.000000}%
\pgfsetstrokecolor{currentstroke}%
\pgfsetdash{}{0pt}%
\pgfsys@defobject{currentmarker}{\pgfqpoint{0.000000in}{-0.020833in}}{\pgfqpoint{0.000000in}{0.000000in}}{%
\pgfpathmoveto{\pgfqpoint{0.000000in}{0.000000in}}%
\pgfpathlineto{\pgfqpoint{0.000000in}{-0.020833in}}%
\pgfusepath{stroke,fill}%
}%
\begin{pgfscope}%
\pgfsys@transformshift{2.561837in}{0.893003in}%
\pgfsys@useobject{currentmarker}{}%
\end{pgfscope}%
\end{pgfscope}%
\begin{pgfscope}%
\pgfsetbuttcap%
\pgfsetroundjoin%
\definecolor{currentfill}{rgb}{0.000000,0.000000,0.000000}%
\pgfsetfillcolor{currentfill}%
\pgfsetlinewidth{0.501875pt}%
\definecolor{currentstroke}{rgb}{0.000000,0.000000,0.000000}%
\pgfsetstrokecolor{currentstroke}%
\pgfsetdash{}{0pt}%
\pgfsys@defobject{currentmarker}{\pgfqpoint{0.000000in}{0.000000in}}{\pgfqpoint{0.000000in}{0.020833in}}{%
\pgfpathmoveto{\pgfqpoint{0.000000in}{0.000000in}}%
\pgfpathlineto{\pgfqpoint{0.000000in}{0.020833in}}%
\pgfusepath{stroke,fill}%
}%
\begin{pgfscope}%
\pgfsys@transformshift{2.597357in}{0.586309in}%
\pgfsys@useobject{currentmarker}{}%
\end{pgfscope}%
\end{pgfscope}%
\begin{pgfscope}%
\pgfsetbuttcap%
\pgfsetroundjoin%
\definecolor{currentfill}{rgb}{0.000000,0.000000,0.000000}%
\pgfsetfillcolor{currentfill}%
\pgfsetlinewidth{0.501875pt}%
\definecolor{currentstroke}{rgb}{0.000000,0.000000,0.000000}%
\pgfsetstrokecolor{currentstroke}%
\pgfsetdash{}{0pt}%
\pgfsys@defobject{currentmarker}{\pgfqpoint{0.000000in}{-0.020833in}}{\pgfqpoint{0.000000in}{0.000000in}}{%
\pgfpathmoveto{\pgfqpoint{0.000000in}{0.000000in}}%
\pgfpathlineto{\pgfqpoint{0.000000in}{-0.020833in}}%
\pgfusepath{stroke,fill}%
}%
\begin{pgfscope}%
\pgfsys@transformshift{2.597357in}{0.893003in}%
\pgfsys@useobject{currentmarker}{}%
\end{pgfscope}%
\end{pgfscope}%
\begin{pgfscope}%
\pgfsetbuttcap%
\pgfsetroundjoin%
\definecolor{currentfill}{rgb}{0.000000,0.000000,0.000000}%
\pgfsetfillcolor{currentfill}%
\pgfsetlinewidth{0.501875pt}%
\definecolor{currentstroke}{rgb}{0.000000,0.000000,0.000000}%
\pgfsetstrokecolor{currentstroke}%
\pgfsetdash{}{0pt}%
\pgfsys@defobject{currentmarker}{\pgfqpoint{0.000000in}{0.000000in}}{\pgfqpoint{0.000000in}{0.020833in}}{%
\pgfpathmoveto{\pgfqpoint{0.000000in}{0.000000in}}%
\pgfpathlineto{\pgfqpoint{0.000000in}{0.020833in}}%
\pgfusepath{stroke,fill}%
}%
\begin{pgfscope}%
\pgfsys@transformshift{2.632878in}{0.586309in}%
\pgfsys@useobject{currentmarker}{}%
\end{pgfscope}%
\end{pgfscope}%
\begin{pgfscope}%
\pgfsetbuttcap%
\pgfsetroundjoin%
\definecolor{currentfill}{rgb}{0.000000,0.000000,0.000000}%
\pgfsetfillcolor{currentfill}%
\pgfsetlinewidth{0.501875pt}%
\definecolor{currentstroke}{rgb}{0.000000,0.000000,0.000000}%
\pgfsetstrokecolor{currentstroke}%
\pgfsetdash{}{0pt}%
\pgfsys@defobject{currentmarker}{\pgfqpoint{0.000000in}{-0.020833in}}{\pgfqpoint{0.000000in}{0.000000in}}{%
\pgfpathmoveto{\pgfqpoint{0.000000in}{0.000000in}}%
\pgfpathlineto{\pgfqpoint{0.000000in}{-0.020833in}}%
\pgfusepath{stroke,fill}%
}%
\begin{pgfscope}%
\pgfsys@transformshift{2.632878in}{0.893003in}%
\pgfsys@useobject{currentmarker}{}%
\end{pgfscope}%
\end{pgfscope}%
\begin{pgfscope}%
\pgfsetbuttcap%
\pgfsetroundjoin%
\definecolor{currentfill}{rgb}{0.000000,0.000000,0.000000}%
\pgfsetfillcolor{currentfill}%
\pgfsetlinewidth{0.501875pt}%
\definecolor{currentstroke}{rgb}{0.000000,0.000000,0.000000}%
\pgfsetstrokecolor{currentstroke}%
\pgfsetdash{}{0pt}%
\pgfsys@defobject{currentmarker}{\pgfqpoint{0.000000in}{0.000000in}}{\pgfqpoint{0.000000in}{0.020833in}}{%
\pgfpathmoveto{\pgfqpoint{0.000000in}{0.000000in}}%
\pgfpathlineto{\pgfqpoint{0.000000in}{0.020833in}}%
\pgfusepath{stroke,fill}%
}%
\begin{pgfscope}%
\pgfsys@transformshift{2.668398in}{0.586309in}%
\pgfsys@useobject{currentmarker}{}%
\end{pgfscope}%
\end{pgfscope}%
\begin{pgfscope}%
\pgfsetbuttcap%
\pgfsetroundjoin%
\definecolor{currentfill}{rgb}{0.000000,0.000000,0.000000}%
\pgfsetfillcolor{currentfill}%
\pgfsetlinewidth{0.501875pt}%
\definecolor{currentstroke}{rgb}{0.000000,0.000000,0.000000}%
\pgfsetstrokecolor{currentstroke}%
\pgfsetdash{}{0pt}%
\pgfsys@defobject{currentmarker}{\pgfqpoint{0.000000in}{-0.020833in}}{\pgfqpoint{0.000000in}{0.000000in}}{%
\pgfpathmoveto{\pgfqpoint{0.000000in}{0.000000in}}%
\pgfpathlineto{\pgfqpoint{0.000000in}{-0.020833in}}%
\pgfusepath{stroke,fill}%
}%
\begin{pgfscope}%
\pgfsys@transformshift{2.668398in}{0.893003in}%
\pgfsys@useobject{currentmarker}{}%
\end{pgfscope}%
\end{pgfscope}%
\begin{pgfscope}%
\pgfsetbuttcap%
\pgfsetroundjoin%
\definecolor{currentfill}{rgb}{0.000000,0.000000,0.000000}%
\pgfsetfillcolor{currentfill}%
\pgfsetlinewidth{0.501875pt}%
\definecolor{currentstroke}{rgb}{0.000000,0.000000,0.000000}%
\pgfsetstrokecolor{currentstroke}%
\pgfsetdash{}{0pt}%
\pgfsys@defobject{currentmarker}{\pgfqpoint{0.000000in}{0.000000in}}{\pgfqpoint{0.000000in}{0.020833in}}{%
\pgfpathmoveto{\pgfqpoint{0.000000in}{0.000000in}}%
\pgfpathlineto{\pgfqpoint{0.000000in}{0.020833in}}%
\pgfusepath{stroke,fill}%
}%
\begin{pgfscope}%
\pgfsys@transformshift{2.703919in}{0.586309in}%
\pgfsys@useobject{currentmarker}{}%
\end{pgfscope}%
\end{pgfscope}%
\begin{pgfscope}%
\pgfsetbuttcap%
\pgfsetroundjoin%
\definecolor{currentfill}{rgb}{0.000000,0.000000,0.000000}%
\pgfsetfillcolor{currentfill}%
\pgfsetlinewidth{0.501875pt}%
\definecolor{currentstroke}{rgb}{0.000000,0.000000,0.000000}%
\pgfsetstrokecolor{currentstroke}%
\pgfsetdash{}{0pt}%
\pgfsys@defobject{currentmarker}{\pgfqpoint{0.000000in}{-0.020833in}}{\pgfqpoint{0.000000in}{0.000000in}}{%
\pgfpathmoveto{\pgfqpoint{0.000000in}{0.000000in}}%
\pgfpathlineto{\pgfqpoint{0.000000in}{-0.020833in}}%
\pgfusepath{stroke,fill}%
}%
\begin{pgfscope}%
\pgfsys@transformshift{2.703919in}{0.893003in}%
\pgfsys@useobject{currentmarker}{}%
\end{pgfscope}%
\end{pgfscope}%
\begin{pgfscope}%
\pgfsetbuttcap%
\pgfsetroundjoin%
\definecolor{currentfill}{rgb}{0.000000,0.000000,0.000000}%
\pgfsetfillcolor{currentfill}%
\pgfsetlinewidth{0.501875pt}%
\definecolor{currentstroke}{rgb}{0.000000,0.000000,0.000000}%
\pgfsetstrokecolor{currentstroke}%
\pgfsetdash{}{0pt}%
\pgfsys@defobject{currentmarker}{\pgfqpoint{0.000000in}{0.000000in}}{\pgfqpoint{0.000000in}{0.020833in}}{%
\pgfpathmoveto{\pgfqpoint{0.000000in}{0.000000in}}%
\pgfpathlineto{\pgfqpoint{0.000000in}{0.020833in}}%
\pgfusepath{stroke,fill}%
}%
\begin{pgfscope}%
\pgfsys@transformshift{2.739440in}{0.586309in}%
\pgfsys@useobject{currentmarker}{}%
\end{pgfscope}%
\end{pgfscope}%
\begin{pgfscope}%
\pgfsetbuttcap%
\pgfsetroundjoin%
\definecolor{currentfill}{rgb}{0.000000,0.000000,0.000000}%
\pgfsetfillcolor{currentfill}%
\pgfsetlinewidth{0.501875pt}%
\definecolor{currentstroke}{rgb}{0.000000,0.000000,0.000000}%
\pgfsetstrokecolor{currentstroke}%
\pgfsetdash{}{0pt}%
\pgfsys@defobject{currentmarker}{\pgfqpoint{0.000000in}{-0.020833in}}{\pgfqpoint{0.000000in}{0.000000in}}{%
\pgfpathmoveto{\pgfqpoint{0.000000in}{0.000000in}}%
\pgfpathlineto{\pgfqpoint{0.000000in}{-0.020833in}}%
\pgfusepath{stroke,fill}%
}%
\begin{pgfscope}%
\pgfsys@transformshift{2.739440in}{0.893003in}%
\pgfsys@useobject{currentmarker}{}%
\end{pgfscope}%
\end{pgfscope}%
\begin{pgfscope}%
\pgfsetbuttcap%
\pgfsetroundjoin%
\definecolor{currentfill}{rgb}{0.000000,0.000000,0.000000}%
\pgfsetfillcolor{currentfill}%
\pgfsetlinewidth{0.501875pt}%
\definecolor{currentstroke}{rgb}{0.000000,0.000000,0.000000}%
\pgfsetstrokecolor{currentstroke}%
\pgfsetdash{}{0pt}%
\pgfsys@defobject{currentmarker}{\pgfqpoint{0.000000in}{0.000000in}}{\pgfqpoint{0.000000in}{0.020833in}}{%
\pgfpathmoveto{\pgfqpoint{0.000000in}{0.000000in}}%
\pgfpathlineto{\pgfqpoint{0.000000in}{0.020833in}}%
\pgfusepath{stroke,fill}%
}%
\begin{pgfscope}%
\pgfsys@transformshift{2.810481in}{0.586309in}%
\pgfsys@useobject{currentmarker}{}%
\end{pgfscope}%
\end{pgfscope}%
\begin{pgfscope}%
\pgfsetbuttcap%
\pgfsetroundjoin%
\definecolor{currentfill}{rgb}{0.000000,0.000000,0.000000}%
\pgfsetfillcolor{currentfill}%
\pgfsetlinewidth{0.501875pt}%
\definecolor{currentstroke}{rgb}{0.000000,0.000000,0.000000}%
\pgfsetstrokecolor{currentstroke}%
\pgfsetdash{}{0pt}%
\pgfsys@defobject{currentmarker}{\pgfqpoint{0.000000in}{-0.020833in}}{\pgfqpoint{0.000000in}{0.000000in}}{%
\pgfpathmoveto{\pgfqpoint{0.000000in}{0.000000in}}%
\pgfpathlineto{\pgfqpoint{0.000000in}{-0.020833in}}%
\pgfusepath{stroke,fill}%
}%
\begin{pgfscope}%
\pgfsys@transformshift{2.810481in}{0.893003in}%
\pgfsys@useobject{currentmarker}{}%
\end{pgfscope}%
\end{pgfscope}%
\begin{pgfscope}%
\pgfsetbuttcap%
\pgfsetroundjoin%
\definecolor{currentfill}{rgb}{0.000000,0.000000,0.000000}%
\pgfsetfillcolor{currentfill}%
\pgfsetlinewidth{0.501875pt}%
\definecolor{currentstroke}{rgb}{0.000000,0.000000,0.000000}%
\pgfsetstrokecolor{currentstroke}%
\pgfsetdash{}{0pt}%
\pgfsys@defobject{currentmarker}{\pgfqpoint{0.000000in}{0.000000in}}{\pgfqpoint{0.000000in}{0.020833in}}{%
\pgfpathmoveto{\pgfqpoint{0.000000in}{0.000000in}}%
\pgfpathlineto{\pgfqpoint{0.000000in}{0.020833in}}%
\pgfusepath{stroke,fill}%
}%
\begin{pgfscope}%
\pgfsys@transformshift{2.846001in}{0.586309in}%
\pgfsys@useobject{currentmarker}{}%
\end{pgfscope}%
\end{pgfscope}%
\begin{pgfscope}%
\pgfsetbuttcap%
\pgfsetroundjoin%
\definecolor{currentfill}{rgb}{0.000000,0.000000,0.000000}%
\pgfsetfillcolor{currentfill}%
\pgfsetlinewidth{0.501875pt}%
\definecolor{currentstroke}{rgb}{0.000000,0.000000,0.000000}%
\pgfsetstrokecolor{currentstroke}%
\pgfsetdash{}{0pt}%
\pgfsys@defobject{currentmarker}{\pgfqpoint{0.000000in}{-0.020833in}}{\pgfqpoint{0.000000in}{0.000000in}}{%
\pgfpathmoveto{\pgfqpoint{0.000000in}{0.000000in}}%
\pgfpathlineto{\pgfqpoint{0.000000in}{-0.020833in}}%
\pgfusepath{stroke,fill}%
}%
\begin{pgfscope}%
\pgfsys@transformshift{2.846001in}{0.893003in}%
\pgfsys@useobject{currentmarker}{}%
\end{pgfscope}%
\end{pgfscope}%
\begin{pgfscope}%
\pgfsetbuttcap%
\pgfsetroundjoin%
\definecolor{currentfill}{rgb}{0.000000,0.000000,0.000000}%
\pgfsetfillcolor{currentfill}%
\pgfsetlinewidth{0.501875pt}%
\definecolor{currentstroke}{rgb}{0.000000,0.000000,0.000000}%
\pgfsetstrokecolor{currentstroke}%
\pgfsetdash{}{0pt}%
\pgfsys@defobject{currentmarker}{\pgfqpoint{0.000000in}{0.000000in}}{\pgfqpoint{0.000000in}{0.020833in}}{%
\pgfpathmoveto{\pgfqpoint{0.000000in}{0.000000in}}%
\pgfpathlineto{\pgfqpoint{0.000000in}{0.020833in}}%
\pgfusepath{stroke,fill}%
}%
\begin{pgfscope}%
\pgfsys@transformshift{2.881522in}{0.586309in}%
\pgfsys@useobject{currentmarker}{}%
\end{pgfscope}%
\end{pgfscope}%
\begin{pgfscope}%
\pgfsetbuttcap%
\pgfsetroundjoin%
\definecolor{currentfill}{rgb}{0.000000,0.000000,0.000000}%
\pgfsetfillcolor{currentfill}%
\pgfsetlinewidth{0.501875pt}%
\definecolor{currentstroke}{rgb}{0.000000,0.000000,0.000000}%
\pgfsetstrokecolor{currentstroke}%
\pgfsetdash{}{0pt}%
\pgfsys@defobject{currentmarker}{\pgfqpoint{0.000000in}{-0.020833in}}{\pgfqpoint{0.000000in}{0.000000in}}{%
\pgfpathmoveto{\pgfqpoint{0.000000in}{0.000000in}}%
\pgfpathlineto{\pgfqpoint{0.000000in}{-0.020833in}}%
\pgfusepath{stroke,fill}%
}%
\begin{pgfscope}%
\pgfsys@transformshift{2.881522in}{0.893003in}%
\pgfsys@useobject{currentmarker}{}%
\end{pgfscope}%
\end{pgfscope}%
\begin{pgfscope}%
\pgfsetbuttcap%
\pgfsetroundjoin%
\definecolor{currentfill}{rgb}{0.000000,0.000000,0.000000}%
\pgfsetfillcolor{currentfill}%
\pgfsetlinewidth{0.501875pt}%
\definecolor{currentstroke}{rgb}{0.000000,0.000000,0.000000}%
\pgfsetstrokecolor{currentstroke}%
\pgfsetdash{}{0pt}%
\pgfsys@defobject{currentmarker}{\pgfqpoint{0.000000in}{0.000000in}}{\pgfqpoint{0.000000in}{0.020833in}}{%
\pgfpathmoveto{\pgfqpoint{0.000000in}{0.000000in}}%
\pgfpathlineto{\pgfqpoint{0.000000in}{0.020833in}}%
\pgfusepath{stroke,fill}%
}%
\begin{pgfscope}%
\pgfsys@transformshift{2.917042in}{0.586309in}%
\pgfsys@useobject{currentmarker}{}%
\end{pgfscope}%
\end{pgfscope}%
\begin{pgfscope}%
\pgfsetbuttcap%
\pgfsetroundjoin%
\definecolor{currentfill}{rgb}{0.000000,0.000000,0.000000}%
\pgfsetfillcolor{currentfill}%
\pgfsetlinewidth{0.501875pt}%
\definecolor{currentstroke}{rgb}{0.000000,0.000000,0.000000}%
\pgfsetstrokecolor{currentstroke}%
\pgfsetdash{}{0pt}%
\pgfsys@defobject{currentmarker}{\pgfqpoint{0.000000in}{-0.020833in}}{\pgfqpoint{0.000000in}{0.000000in}}{%
\pgfpathmoveto{\pgfqpoint{0.000000in}{0.000000in}}%
\pgfpathlineto{\pgfqpoint{0.000000in}{-0.020833in}}%
\pgfusepath{stroke,fill}%
}%
\begin{pgfscope}%
\pgfsys@transformshift{2.917042in}{0.893003in}%
\pgfsys@useobject{currentmarker}{}%
\end{pgfscope}%
\end{pgfscope}%
\begin{pgfscope}%
\pgfsetbuttcap%
\pgfsetroundjoin%
\definecolor{currentfill}{rgb}{0.000000,0.000000,0.000000}%
\pgfsetfillcolor{currentfill}%
\pgfsetlinewidth{0.501875pt}%
\definecolor{currentstroke}{rgb}{0.000000,0.000000,0.000000}%
\pgfsetstrokecolor{currentstroke}%
\pgfsetdash{}{0pt}%
\pgfsys@defobject{currentmarker}{\pgfqpoint{0.000000in}{0.000000in}}{\pgfqpoint{0.000000in}{0.020833in}}{%
\pgfpathmoveto{\pgfqpoint{0.000000in}{0.000000in}}%
\pgfpathlineto{\pgfqpoint{0.000000in}{0.020833in}}%
\pgfusepath{stroke,fill}%
}%
\begin{pgfscope}%
\pgfsys@transformshift{2.952563in}{0.586309in}%
\pgfsys@useobject{currentmarker}{}%
\end{pgfscope}%
\end{pgfscope}%
\begin{pgfscope}%
\pgfsetbuttcap%
\pgfsetroundjoin%
\definecolor{currentfill}{rgb}{0.000000,0.000000,0.000000}%
\pgfsetfillcolor{currentfill}%
\pgfsetlinewidth{0.501875pt}%
\definecolor{currentstroke}{rgb}{0.000000,0.000000,0.000000}%
\pgfsetstrokecolor{currentstroke}%
\pgfsetdash{}{0pt}%
\pgfsys@defobject{currentmarker}{\pgfqpoint{0.000000in}{-0.020833in}}{\pgfqpoint{0.000000in}{0.000000in}}{%
\pgfpathmoveto{\pgfqpoint{0.000000in}{0.000000in}}%
\pgfpathlineto{\pgfqpoint{0.000000in}{-0.020833in}}%
\pgfusepath{stroke,fill}%
}%
\begin{pgfscope}%
\pgfsys@transformshift{2.952563in}{0.893003in}%
\pgfsys@useobject{currentmarker}{}%
\end{pgfscope}%
\end{pgfscope}%
\begin{pgfscope}%
\pgfsetbuttcap%
\pgfsetroundjoin%
\definecolor{currentfill}{rgb}{0.000000,0.000000,0.000000}%
\pgfsetfillcolor{currentfill}%
\pgfsetlinewidth{0.501875pt}%
\definecolor{currentstroke}{rgb}{0.000000,0.000000,0.000000}%
\pgfsetstrokecolor{currentstroke}%
\pgfsetdash{}{0pt}%
\pgfsys@defobject{currentmarker}{\pgfqpoint{0.000000in}{0.000000in}}{\pgfqpoint{0.000000in}{0.020833in}}{%
\pgfpathmoveto{\pgfqpoint{0.000000in}{0.000000in}}%
\pgfpathlineto{\pgfqpoint{0.000000in}{0.020833in}}%
\pgfusepath{stroke,fill}%
}%
\begin{pgfscope}%
\pgfsys@transformshift{2.988083in}{0.586309in}%
\pgfsys@useobject{currentmarker}{}%
\end{pgfscope}%
\end{pgfscope}%
\begin{pgfscope}%
\pgfsetbuttcap%
\pgfsetroundjoin%
\definecolor{currentfill}{rgb}{0.000000,0.000000,0.000000}%
\pgfsetfillcolor{currentfill}%
\pgfsetlinewidth{0.501875pt}%
\definecolor{currentstroke}{rgb}{0.000000,0.000000,0.000000}%
\pgfsetstrokecolor{currentstroke}%
\pgfsetdash{}{0pt}%
\pgfsys@defobject{currentmarker}{\pgfqpoint{0.000000in}{-0.020833in}}{\pgfqpoint{0.000000in}{0.000000in}}{%
\pgfpathmoveto{\pgfqpoint{0.000000in}{0.000000in}}%
\pgfpathlineto{\pgfqpoint{0.000000in}{-0.020833in}}%
\pgfusepath{stroke,fill}%
}%
\begin{pgfscope}%
\pgfsys@transformshift{2.988083in}{0.893003in}%
\pgfsys@useobject{currentmarker}{}%
\end{pgfscope}%
\end{pgfscope}%
\begin{pgfscope}%
\pgfsetbuttcap%
\pgfsetroundjoin%
\definecolor{currentfill}{rgb}{0.000000,0.000000,0.000000}%
\pgfsetfillcolor{currentfill}%
\pgfsetlinewidth{0.501875pt}%
\definecolor{currentstroke}{rgb}{0.000000,0.000000,0.000000}%
\pgfsetstrokecolor{currentstroke}%
\pgfsetdash{}{0pt}%
\pgfsys@defobject{currentmarker}{\pgfqpoint{0.000000in}{0.000000in}}{\pgfqpoint{0.000000in}{0.020833in}}{%
\pgfpathmoveto{\pgfqpoint{0.000000in}{0.000000in}}%
\pgfpathlineto{\pgfqpoint{0.000000in}{0.020833in}}%
\pgfusepath{stroke,fill}%
}%
\begin{pgfscope}%
\pgfsys@transformshift{3.023604in}{0.586309in}%
\pgfsys@useobject{currentmarker}{}%
\end{pgfscope}%
\end{pgfscope}%
\begin{pgfscope}%
\pgfsetbuttcap%
\pgfsetroundjoin%
\definecolor{currentfill}{rgb}{0.000000,0.000000,0.000000}%
\pgfsetfillcolor{currentfill}%
\pgfsetlinewidth{0.501875pt}%
\definecolor{currentstroke}{rgb}{0.000000,0.000000,0.000000}%
\pgfsetstrokecolor{currentstroke}%
\pgfsetdash{}{0pt}%
\pgfsys@defobject{currentmarker}{\pgfqpoint{0.000000in}{-0.020833in}}{\pgfqpoint{0.000000in}{0.000000in}}{%
\pgfpathmoveto{\pgfqpoint{0.000000in}{0.000000in}}%
\pgfpathlineto{\pgfqpoint{0.000000in}{-0.020833in}}%
\pgfusepath{stroke,fill}%
}%
\begin{pgfscope}%
\pgfsys@transformshift{3.023604in}{0.893003in}%
\pgfsys@useobject{currentmarker}{}%
\end{pgfscope}%
\end{pgfscope}%
\begin{pgfscope}%
\pgfsetbuttcap%
\pgfsetroundjoin%
\definecolor{currentfill}{rgb}{0.000000,0.000000,0.000000}%
\pgfsetfillcolor{currentfill}%
\pgfsetlinewidth{0.501875pt}%
\definecolor{currentstroke}{rgb}{0.000000,0.000000,0.000000}%
\pgfsetstrokecolor{currentstroke}%
\pgfsetdash{}{0pt}%
\pgfsys@defobject{currentmarker}{\pgfqpoint{0.000000in}{0.000000in}}{\pgfqpoint{0.000000in}{0.020833in}}{%
\pgfpathmoveto{\pgfqpoint{0.000000in}{0.000000in}}%
\pgfpathlineto{\pgfqpoint{0.000000in}{0.020833in}}%
\pgfusepath{stroke,fill}%
}%
\begin{pgfscope}%
\pgfsys@transformshift{3.059124in}{0.586309in}%
\pgfsys@useobject{currentmarker}{}%
\end{pgfscope}%
\end{pgfscope}%
\begin{pgfscope}%
\pgfsetbuttcap%
\pgfsetroundjoin%
\definecolor{currentfill}{rgb}{0.000000,0.000000,0.000000}%
\pgfsetfillcolor{currentfill}%
\pgfsetlinewidth{0.501875pt}%
\definecolor{currentstroke}{rgb}{0.000000,0.000000,0.000000}%
\pgfsetstrokecolor{currentstroke}%
\pgfsetdash{}{0pt}%
\pgfsys@defobject{currentmarker}{\pgfqpoint{0.000000in}{-0.020833in}}{\pgfqpoint{0.000000in}{0.000000in}}{%
\pgfpathmoveto{\pgfqpoint{0.000000in}{0.000000in}}%
\pgfpathlineto{\pgfqpoint{0.000000in}{-0.020833in}}%
\pgfusepath{stroke,fill}%
}%
\begin{pgfscope}%
\pgfsys@transformshift{3.059124in}{0.893003in}%
\pgfsys@useobject{currentmarker}{}%
\end{pgfscope}%
\end{pgfscope}%
\begin{pgfscope}%
\pgfsetbuttcap%
\pgfsetroundjoin%
\definecolor{currentfill}{rgb}{0.000000,0.000000,0.000000}%
\pgfsetfillcolor{currentfill}%
\pgfsetlinewidth{0.501875pt}%
\definecolor{currentstroke}{rgb}{0.000000,0.000000,0.000000}%
\pgfsetstrokecolor{currentstroke}%
\pgfsetdash{}{0pt}%
\pgfsys@defobject{currentmarker}{\pgfqpoint{0.000000in}{0.000000in}}{\pgfqpoint{0.000000in}{0.020833in}}{%
\pgfpathmoveto{\pgfqpoint{0.000000in}{0.000000in}}%
\pgfpathlineto{\pgfqpoint{0.000000in}{0.020833in}}%
\pgfusepath{stroke,fill}%
}%
\begin{pgfscope}%
\pgfsys@transformshift{3.094645in}{0.586309in}%
\pgfsys@useobject{currentmarker}{}%
\end{pgfscope}%
\end{pgfscope}%
\begin{pgfscope}%
\pgfsetbuttcap%
\pgfsetroundjoin%
\definecolor{currentfill}{rgb}{0.000000,0.000000,0.000000}%
\pgfsetfillcolor{currentfill}%
\pgfsetlinewidth{0.501875pt}%
\definecolor{currentstroke}{rgb}{0.000000,0.000000,0.000000}%
\pgfsetstrokecolor{currentstroke}%
\pgfsetdash{}{0pt}%
\pgfsys@defobject{currentmarker}{\pgfqpoint{0.000000in}{-0.020833in}}{\pgfqpoint{0.000000in}{0.000000in}}{%
\pgfpathmoveto{\pgfqpoint{0.000000in}{0.000000in}}%
\pgfpathlineto{\pgfqpoint{0.000000in}{-0.020833in}}%
\pgfusepath{stroke,fill}%
}%
\begin{pgfscope}%
\pgfsys@transformshift{3.094645in}{0.893003in}%
\pgfsys@useobject{currentmarker}{}%
\end{pgfscope}%
\end{pgfscope}%
\begin{pgfscope}%
\pgfsetbuttcap%
\pgfsetroundjoin%
\definecolor{currentfill}{rgb}{0.000000,0.000000,0.000000}%
\pgfsetfillcolor{currentfill}%
\pgfsetlinewidth{0.501875pt}%
\definecolor{currentstroke}{rgb}{0.000000,0.000000,0.000000}%
\pgfsetstrokecolor{currentstroke}%
\pgfsetdash{}{0pt}%
\pgfsys@defobject{currentmarker}{\pgfqpoint{0.000000in}{0.000000in}}{\pgfqpoint{0.000000in}{0.020833in}}{%
\pgfpathmoveto{\pgfqpoint{0.000000in}{0.000000in}}%
\pgfpathlineto{\pgfqpoint{0.000000in}{0.020833in}}%
\pgfusepath{stroke,fill}%
}%
\begin{pgfscope}%
\pgfsys@transformshift{3.130165in}{0.586309in}%
\pgfsys@useobject{currentmarker}{}%
\end{pgfscope}%
\end{pgfscope}%
\begin{pgfscope}%
\pgfsetbuttcap%
\pgfsetroundjoin%
\definecolor{currentfill}{rgb}{0.000000,0.000000,0.000000}%
\pgfsetfillcolor{currentfill}%
\pgfsetlinewidth{0.501875pt}%
\definecolor{currentstroke}{rgb}{0.000000,0.000000,0.000000}%
\pgfsetstrokecolor{currentstroke}%
\pgfsetdash{}{0pt}%
\pgfsys@defobject{currentmarker}{\pgfqpoint{0.000000in}{-0.020833in}}{\pgfqpoint{0.000000in}{0.000000in}}{%
\pgfpathmoveto{\pgfqpoint{0.000000in}{0.000000in}}%
\pgfpathlineto{\pgfqpoint{0.000000in}{-0.020833in}}%
\pgfusepath{stroke,fill}%
}%
\begin{pgfscope}%
\pgfsys@transformshift{3.130165in}{0.893003in}%
\pgfsys@useobject{currentmarker}{}%
\end{pgfscope}%
\end{pgfscope}%
\begin{pgfscope}%
\pgfsetbuttcap%
\pgfsetroundjoin%
\definecolor{currentfill}{rgb}{0.000000,0.000000,0.000000}%
\pgfsetfillcolor{currentfill}%
\pgfsetlinewidth{0.501875pt}%
\definecolor{currentstroke}{rgb}{0.000000,0.000000,0.000000}%
\pgfsetstrokecolor{currentstroke}%
\pgfsetdash{}{0pt}%
\pgfsys@defobject{currentmarker}{\pgfqpoint{0.000000in}{0.000000in}}{\pgfqpoint{0.000000in}{0.020833in}}{%
\pgfpathmoveto{\pgfqpoint{0.000000in}{0.000000in}}%
\pgfpathlineto{\pgfqpoint{0.000000in}{0.020833in}}%
\pgfusepath{stroke,fill}%
}%
\begin{pgfscope}%
\pgfsys@transformshift{3.165686in}{0.586309in}%
\pgfsys@useobject{currentmarker}{}%
\end{pgfscope}%
\end{pgfscope}%
\begin{pgfscope}%
\pgfsetbuttcap%
\pgfsetroundjoin%
\definecolor{currentfill}{rgb}{0.000000,0.000000,0.000000}%
\pgfsetfillcolor{currentfill}%
\pgfsetlinewidth{0.501875pt}%
\definecolor{currentstroke}{rgb}{0.000000,0.000000,0.000000}%
\pgfsetstrokecolor{currentstroke}%
\pgfsetdash{}{0pt}%
\pgfsys@defobject{currentmarker}{\pgfqpoint{0.000000in}{-0.020833in}}{\pgfqpoint{0.000000in}{0.000000in}}{%
\pgfpathmoveto{\pgfqpoint{0.000000in}{0.000000in}}%
\pgfpathlineto{\pgfqpoint{0.000000in}{-0.020833in}}%
\pgfusepath{stroke,fill}%
}%
\begin{pgfscope}%
\pgfsys@transformshift{3.165686in}{0.893003in}%
\pgfsys@useobject{currentmarker}{}%
\end{pgfscope}%
\end{pgfscope}%
\begin{pgfscope}%
\pgfsetbuttcap%
\pgfsetroundjoin%
\definecolor{currentfill}{rgb}{0.000000,0.000000,0.000000}%
\pgfsetfillcolor{currentfill}%
\pgfsetlinewidth{0.501875pt}%
\definecolor{currentstroke}{rgb}{0.000000,0.000000,0.000000}%
\pgfsetstrokecolor{currentstroke}%
\pgfsetdash{}{0pt}%
\pgfsys@defobject{currentmarker}{\pgfqpoint{0.000000in}{0.000000in}}{\pgfqpoint{0.000000in}{0.020833in}}{%
\pgfpathmoveto{\pgfqpoint{0.000000in}{0.000000in}}%
\pgfpathlineto{\pgfqpoint{0.000000in}{0.020833in}}%
\pgfusepath{stroke,fill}%
}%
\begin{pgfscope}%
\pgfsys@transformshift{3.236727in}{0.586309in}%
\pgfsys@useobject{currentmarker}{}%
\end{pgfscope}%
\end{pgfscope}%
\begin{pgfscope}%
\pgfsetbuttcap%
\pgfsetroundjoin%
\definecolor{currentfill}{rgb}{0.000000,0.000000,0.000000}%
\pgfsetfillcolor{currentfill}%
\pgfsetlinewidth{0.501875pt}%
\definecolor{currentstroke}{rgb}{0.000000,0.000000,0.000000}%
\pgfsetstrokecolor{currentstroke}%
\pgfsetdash{}{0pt}%
\pgfsys@defobject{currentmarker}{\pgfqpoint{0.000000in}{-0.020833in}}{\pgfqpoint{0.000000in}{0.000000in}}{%
\pgfpathmoveto{\pgfqpoint{0.000000in}{0.000000in}}%
\pgfpathlineto{\pgfqpoint{0.000000in}{-0.020833in}}%
\pgfusepath{stroke,fill}%
}%
\begin{pgfscope}%
\pgfsys@transformshift{3.236727in}{0.893003in}%
\pgfsys@useobject{currentmarker}{}%
\end{pgfscope}%
\end{pgfscope}%
\begin{pgfscope}%
\pgfsetbuttcap%
\pgfsetroundjoin%
\definecolor{currentfill}{rgb}{0.000000,0.000000,0.000000}%
\pgfsetfillcolor{currentfill}%
\pgfsetlinewidth{0.501875pt}%
\definecolor{currentstroke}{rgb}{0.000000,0.000000,0.000000}%
\pgfsetstrokecolor{currentstroke}%
\pgfsetdash{}{0pt}%
\pgfsys@defobject{currentmarker}{\pgfqpoint{0.000000in}{0.000000in}}{\pgfqpoint{0.000000in}{0.020833in}}{%
\pgfpathmoveto{\pgfqpoint{0.000000in}{0.000000in}}%
\pgfpathlineto{\pgfqpoint{0.000000in}{0.020833in}}%
\pgfusepath{stroke,fill}%
}%
\begin{pgfscope}%
\pgfsys@transformshift{3.272247in}{0.586309in}%
\pgfsys@useobject{currentmarker}{}%
\end{pgfscope}%
\end{pgfscope}%
\begin{pgfscope}%
\pgfsetbuttcap%
\pgfsetroundjoin%
\definecolor{currentfill}{rgb}{0.000000,0.000000,0.000000}%
\pgfsetfillcolor{currentfill}%
\pgfsetlinewidth{0.501875pt}%
\definecolor{currentstroke}{rgb}{0.000000,0.000000,0.000000}%
\pgfsetstrokecolor{currentstroke}%
\pgfsetdash{}{0pt}%
\pgfsys@defobject{currentmarker}{\pgfqpoint{0.000000in}{-0.020833in}}{\pgfqpoint{0.000000in}{0.000000in}}{%
\pgfpathmoveto{\pgfqpoint{0.000000in}{0.000000in}}%
\pgfpathlineto{\pgfqpoint{0.000000in}{-0.020833in}}%
\pgfusepath{stroke,fill}%
}%
\begin{pgfscope}%
\pgfsys@transformshift{3.272247in}{0.893003in}%
\pgfsys@useobject{currentmarker}{}%
\end{pgfscope}%
\end{pgfscope}%
\begin{pgfscope}%
\pgfsetbuttcap%
\pgfsetroundjoin%
\definecolor{currentfill}{rgb}{0.000000,0.000000,0.000000}%
\pgfsetfillcolor{currentfill}%
\pgfsetlinewidth{0.501875pt}%
\definecolor{currentstroke}{rgb}{0.000000,0.000000,0.000000}%
\pgfsetstrokecolor{currentstroke}%
\pgfsetdash{}{0pt}%
\pgfsys@defobject{currentmarker}{\pgfqpoint{0.000000in}{0.000000in}}{\pgfqpoint{0.000000in}{0.020833in}}{%
\pgfpathmoveto{\pgfqpoint{0.000000in}{0.000000in}}%
\pgfpathlineto{\pgfqpoint{0.000000in}{0.020833in}}%
\pgfusepath{stroke,fill}%
}%
\begin{pgfscope}%
\pgfsys@transformshift{3.307768in}{0.586309in}%
\pgfsys@useobject{currentmarker}{}%
\end{pgfscope}%
\end{pgfscope}%
\begin{pgfscope}%
\pgfsetbuttcap%
\pgfsetroundjoin%
\definecolor{currentfill}{rgb}{0.000000,0.000000,0.000000}%
\pgfsetfillcolor{currentfill}%
\pgfsetlinewidth{0.501875pt}%
\definecolor{currentstroke}{rgb}{0.000000,0.000000,0.000000}%
\pgfsetstrokecolor{currentstroke}%
\pgfsetdash{}{0pt}%
\pgfsys@defobject{currentmarker}{\pgfqpoint{0.000000in}{-0.020833in}}{\pgfqpoint{0.000000in}{0.000000in}}{%
\pgfpathmoveto{\pgfqpoint{0.000000in}{0.000000in}}%
\pgfpathlineto{\pgfqpoint{0.000000in}{-0.020833in}}%
\pgfusepath{stroke,fill}%
}%
\begin{pgfscope}%
\pgfsys@transformshift{3.307768in}{0.893003in}%
\pgfsys@useobject{currentmarker}{}%
\end{pgfscope}%
\end{pgfscope}%
\begin{pgfscope}%
\pgfsetbuttcap%
\pgfsetroundjoin%
\definecolor{currentfill}{rgb}{0.000000,0.000000,0.000000}%
\pgfsetfillcolor{currentfill}%
\pgfsetlinewidth{0.501875pt}%
\definecolor{currentstroke}{rgb}{0.000000,0.000000,0.000000}%
\pgfsetstrokecolor{currentstroke}%
\pgfsetdash{}{0pt}%
\pgfsys@defobject{currentmarker}{\pgfqpoint{0.000000in}{0.000000in}}{\pgfqpoint{0.000000in}{0.020833in}}{%
\pgfpathmoveto{\pgfqpoint{0.000000in}{0.000000in}}%
\pgfpathlineto{\pgfqpoint{0.000000in}{0.020833in}}%
\pgfusepath{stroke,fill}%
}%
\begin{pgfscope}%
\pgfsys@transformshift{3.343288in}{0.586309in}%
\pgfsys@useobject{currentmarker}{}%
\end{pgfscope}%
\end{pgfscope}%
\begin{pgfscope}%
\pgfsetbuttcap%
\pgfsetroundjoin%
\definecolor{currentfill}{rgb}{0.000000,0.000000,0.000000}%
\pgfsetfillcolor{currentfill}%
\pgfsetlinewidth{0.501875pt}%
\definecolor{currentstroke}{rgb}{0.000000,0.000000,0.000000}%
\pgfsetstrokecolor{currentstroke}%
\pgfsetdash{}{0pt}%
\pgfsys@defobject{currentmarker}{\pgfqpoint{0.000000in}{-0.020833in}}{\pgfqpoint{0.000000in}{0.000000in}}{%
\pgfpathmoveto{\pgfqpoint{0.000000in}{0.000000in}}%
\pgfpathlineto{\pgfqpoint{0.000000in}{-0.020833in}}%
\pgfusepath{stroke,fill}%
}%
\begin{pgfscope}%
\pgfsys@transformshift{3.343288in}{0.893003in}%
\pgfsys@useobject{currentmarker}{}%
\end{pgfscope}%
\end{pgfscope}%
\begin{pgfscope}%
\pgfsetbuttcap%
\pgfsetroundjoin%
\definecolor{currentfill}{rgb}{0.000000,0.000000,0.000000}%
\pgfsetfillcolor{currentfill}%
\pgfsetlinewidth{0.501875pt}%
\definecolor{currentstroke}{rgb}{0.000000,0.000000,0.000000}%
\pgfsetstrokecolor{currentstroke}%
\pgfsetdash{}{0pt}%
\pgfsys@defobject{currentmarker}{\pgfqpoint{0.000000in}{0.000000in}}{\pgfqpoint{0.000000in}{0.020833in}}{%
\pgfpathmoveto{\pgfqpoint{0.000000in}{0.000000in}}%
\pgfpathlineto{\pgfqpoint{0.000000in}{0.020833in}}%
\pgfusepath{stroke,fill}%
}%
\begin{pgfscope}%
\pgfsys@transformshift{3.378809in}{0.586309in}%
\pgfsys@useobject{currentmarker}{}%
\end{pgfscope}%
\end{pgfscope}%
\begin{pgfscope}%
\pgfsetbuttcap%
\pgfsetroundjoin%
\definecolor{currentfill}{rgb}{0.000000,0.000000,0.000000}%
\pgfsetfillcolor{currentfill}%
\pgfsetlinewidth{0.501875pt}%
\definecolor{currentstroke}{rgb}{0.000000,0.000000,0.000000}%
\pgfsetstrokecolor{currentstroke}%
\pgfsetdash{}{0pt}%
\pgfsys@defobject{currentmarker}{\pgfqpoint{0.000000in}{-0.020833in}}{\pgfqpoint{0.000000in}{0.000000in}}{%
\pgfpathmoveto{\pgfqpoint{0.000000in}{0.000000in}}%
\pgfpathlineto{\pgfqpoint{0.000000in}{-0.020833in}}%
\pgfusepath{stroke,fill}%
}%
\begin{pgfscope}%
\pgfsys@transformshift{3.378809in}{0.893003in}%
\pgfsys@useobject{currentmarker}{}%
\end{pgfscope}%
\end{pgfscope}%
\begin{pgfscope}%
\pgfsetbuttcap%
\pgfsetroundjoin%
\definecolor{currentfill}{rgb}{0.000000,0.000000,0.000000}%
\pgfsetfillcolor{currentfill}%
\pgfsetlinewidth{0.501875pt}%
\definecolor{currentstroke}{rgb}{0.000000,0.000000,0.000000}%
\pgfsetstrokecolor{currentstroke}%
\pgfsetdash{}{0pt}%
\pgfsys@defobject{currentmarker}{\pgfqpoint{0.000000in}{0.000000in}}{\pgfqpoint{0.000000in}{0.020833in}}{%
\pgfpathmoveto{\pgfqpoint{0.000000in}{0.000000in}}%
\pgfpathlineto{\pgfqpoint{0.000000in}{0.020833in}}%
\pgfusepath{stroke,fill}%
}%
\begin{pgfscope}%
\pgfsys@transformshift{3.414330in}{0.586309in}%
\pgfsys@useobject{currentmarker}{}%
\end{pgfscope}%
\end{pgfscope}%
\begin{pgfscope}%
\pgfsetbuttcap%
\pgfsetroundjoin%
\definecolor{currentfill}{rgb}{0.000000,0.000000,0.000000}%
\pgfsetfillcolor{currentfill}%
\pgfsetlinewidth{0.501875pt}%
\definecolor{currentstroke}{rgb}{0.000000,0.000000,0.000000}%
\pgfsetstrokecolor{currentstroke}%
\pgfsetdash{}{0pt}%
\pgfsys@defobject{currentmarker}{\pgfqpoint{0.000000in}{-0.020833in}}{\pgfqpoint{0.000000in}{0.000000in}}{%
\pgfpathmoveto{\pgfqpoint{0.000000in}{0.000000in}}%
\pgfpathlineto{\pgfqpoint{0.000000in}{-0.020833in}}%
\pgfusepath{stroke,fill}%
}%
\begin{pgfscope}%
\pgfsys@transformshift{3.414330in}{0.893003in}%
\pgfsys@useobject{currentmarker}{}%
\end{pgfscope}%
\end{pgfscope}%
\begin{pgfscope}%
\pgfsetbuttcap%
\pgfsetroundjoin%
\definecolor{currentfill}{rgb}{0.000000,0.000000,0.000000}%
\pgfsetfillcolor{currentfill}%
\pgfsetlinewidth{0.501875pt}%
\definecolor{currentstroke}{rgb}{0.000000,0.000000,0.000000}%
\pgfsetstrokecolor{currentstroke}%
\pgfsetdash{}{0pt}%
\pgfsys@defobject{currentmarker}{\pgfqpoint{0.000000in}{0.000000in}}{\pgfqpoint{0.000000in}{0.020833in}}{%
\pgfpathmoveto{\pgfqpoint{0.000000in}{0.000000in}}%
\pgfpathlineto{\pgfqpoint{0.000000in}{0.020833in}}%
\pgfusepath{stroke,fill}%
}%
\begin{pgfscope}%
\pgfsys@transformshift{3.449850in}{0.586309in}%
\pgfsys@useobject{currentmarker}{}%
\end{pgfscope}%
\end{pgfscope}%
\begin{pgfscope}%
\pgfsetbuttcap%
\pgfsetroundjoin%
\definecolor{currentfill}{rgb}{0.000000,0.000000,0.000000}%
\pgfsetfillcolor{currentfill}%
\pgfsetlinewidth{0.501875pt}%
\definecolor{currentstroke}{rgb}{0.000000,0.000000,0.000000}%
\pgfsetstrokecolor{currentstroke}%
\pgfsetdash{}{0pt}%
\pgfsys@defobject{currentmarker}{\pgfqpoint{0.000000in}{-0.020833in}}{\pgfqpoint{0.000000in}{0.000000in}}{%
\pgfpathmoveto{\pgfqpoint{0.000000in}{0.000000in}}%
\pgfpathlineto{\pgfqpoint{0.000000in}{-0.020833in}}%
\pgfusepath{stroke,fill}%
}%
\begin{pgfscope}%
\pgfsys@transformshift{3.449850in}{0.893003in}%
\pgfsys@useobject{currentmarker}{}%
\end{pgfscope}%
\end{pgfscope}%
\begin{pgfscope}%
\pgfsetbuttcap%
\pgfsetroundjoin%
\definecolor{currentfill}{rgb}{0.000000,0.000000,0.000000}%
\pgfsetfillcolor{currentfill}%
\pgfsetlinewidth{0.501875pt}%
\definecolor{currentstroke}{rgb}{0.000000,0.000000,0.000000}%
\pgfsetstrokecolor{currentstroke}%
\pgfsetdash{}{0pt}%
\pgfsys@defobject{currentmarker}{\pgfqpoint{0.000000in}{0.000000in}}{\pgfqpoint{0.000000in}{0.020833in}}{%
\pgfpathmoveto{\pgfqpoint{0.000000in}{0.000000in}}%
\pgfpathlineto{\pgfqpoint{0.000000in}{0.020833in}}%
\pgfusepath{stroke,fill}%
}%
\begin{pgfscope}%
\pgfsys@transformshift{3.485371in}{0.586309in}%
\pgfsys@useobject{currentmarker}{}%
\end{pgfscope}%
\end{pgfscope}%
\begin{pgfscope}%
\pgfsetbuttcap%
\pgfsetroundjoin%
\definecolor{currentfill}{rgb}{0.000000,0.000000,0.000000}%
\pgfsetfillcolor{currentfill}%
\pgfsetlinewidth{0.501875pt}%
\definecolor{currentstroke}{rgb}{0.000000,0.000000,0.000000}%
\pgfsetstrokecolor{currentstroke}%
\pgfsetdash{}{0pt}%
\pgfsys@defobject{currentmarker}{\pgfqpoint{0.000000in}{-0.020833in}}{\pgfqpoint{0.000000in}{0.000000in}}{%
\pgfpathmoveto{\pgfqpoint{0.000000in}{0.000000in}}%
\pgfpathlineto{\pgfqpoint{0.000000in}{-0.020833in}}%
\pgfusepath{stroke,fill}%
}%
\begin{pgfscope}%
\pgfsys@transformshift{3.485371in}{0.893003in}%
\pgfsys@useobject{currentmarker}{}%
\end{pgfscope}%
\end{pgfscope}%
\begin{pgfscope}%
\pgfsetbuttcap%
\pgfsetroundjoin%
\definecolor{currentfill}{rgb}{0.000000,0.000000,0.000000}%
\pgfsetfillcolor{currentfill}%
\pgfsetlinewidth{0.501875pt}%
\definecolor{currentstroke}{rgb}{0.000000,0.000000,0.000000}%
\pgfsetstrokecolor{currentstroke}%
\pgfsetdash{}{0pt}%
\pgfsys@defobject{currentmarker}{\pgfqpoint{0.000000in}{0.000000in}}{\pgfqpoint{0.000000in}{0.020833in}}{%
\pgfpathmoveto{\pgfqpoint{0.000000in}{0.000000in}}%
\pgfpathlineto{\pgfqpoint{0.000000in}{0.020833in}}%
\pgfusepath{stroke,fill}%
}%
\begin{pgfscope}%
\pgfsys@transformshift{3.520891in}{0.586309in}%
\pgfsys@useobject{currentmarker}{}%
\end{pgfscope}%
\end{pgfscope}%
\begin{pgfscope}%
\pgfsetbuttcap%
\pgfsetroundjoin%
\definecolor{currentfill}{rgb}{0.000000,0.000000,0.000000}%
\pgfsetfillcolor{currentfill}%
\pgfsetlinewidth{0.501875pt}%
\definecolor{currentstroke}{rgb}{0.000000,0.000000,0.000000}%
\pgfsetstrokecolor{currentstroke}%
\pgfsetdash{}{0pt}%
\pgfsys@defobject{currentmarker}{\pgfqpoint{0.000000in}{-0.020833in}}{\pgfqpoint{0.000000in}{0.000000in}}{%
\pgfpathmoveto{\pgfqpoint{0.000000in}{0.000000in}}%
\pgfpathlineto{\pgfqpoint{0.000000in}{-0.020833in}}%
\pgfusepath{stroke,fill}%
}%
\begin{pgfscope}%
\pgfsys@transformshift{3.520891in}{0.893003in}%
\pgfsys@useobject{currentmarker}{}%
\end{pgfscope}%
\end{pgfscope}%
\begin{pgfscope}%
\pgfsetbuttcap%
\pgfsetroundjoin%
\definecolor{currentfill}{rgb}{0.000000,0.000000,0.000000}%
\pgfsetfillcolor{currentfill}%
\pgfsetlinewidth{0.501875pt}%
\definecolor{currentstroke}{rgb}{0.000000,0.000000,0.000000}%
\pgfsetstrokecolor{currentstroke}%
\pgfsetdash{}{0pt}%
\pgfsys@defobject{currentmarker}{\pgfqpoint{0.000000in}{0.000000in}}{\pgfqpoint{0.000000in}{0.020833in}}{%
\pgfpathmoveto{\pgfqpoint{0.000000in}{0.000000in}}%
\pgfpathlineto{\pgfqpoint{0.000000in}{0.020833in}}%
\pgfusepath{stroke,fill}%
}%
\begin{pgfscope}%
\pgfsys@transformshift{3.556412in}{0.586309in}%
\pgfsys@useobject{currentmarker}{}%
\end{pgfscope}%
\end{pgfscope}%
\begin{pgfscope}%
\pgfsetbuttcap%
\pgfsetroundjoin%
\definecolor{currentfill}{rgb}{0.000000,0.000000,0.000000}%
\pgfsetfillcolor{currentfill}%
\pgfsetlinewidth{0.501875pt}%
\definecolor{currentstroke}{rgb}{0.000000,0.000000,0.000000}%
\pgfsetstrokecolor{currentstroke}%
\pgfsetdash{}{0pt}%
\pgfsys@defobject{currentmarker}{\pgfqpoint{0.000000in}{-0.020833in}}{\pgfqpoint{0.000000in}{0.000000in}}{%
\pgfpathmoveto{\pgfqpoint{0.000000in}{0.000000in}}%
\pgfpathlineto{\pgfqpoint{0.000000in}{-0.020833in}}%
\pgfusepath{stroke,fill}%
}%
\begin{pgfscope}%
\pgfsys@transformshift{3.556412in}{0.893003in}%
\pgfsys@useobject{currentmarker}{}%
\end{pgfscope}%
\end{pgfscope}%
\begin{pgfscope}%
\pgfsetbuttcap%
\pgfsetroundjoin%
\definecolor{currentfill}{rgb}{0.000000,0.000000,0.000000}%
\pgfsetfillcolor{currentfill}%
\pgfsetlinewidth{0.501875pt}%
\definecolor{currentstroke}{rgb}{0.000000,0.000000,0.000000}%
\pgfsetstrokecolor{currentstroke}%
\pgfsetdash{}{0pt}%
\pgfsys@defobject{currentmarker}{\pgfqpoint{0.000000in}{0.000000in}}{\pgfqpoint{0.000000in}{0.020833in}}{%
\pgfpathmoveto{\pgfqpoint{0.000000in}{0.000000in}}%
\pgfpathlineto{\pgfqpoint{0.000000in}{0.020833in}}%
\pgfusepath{stroke,fill}%
}%
\begin{pgfscope}%
\pgfsys@transformshift{3.591932in}{0.586309in}%
\pgfsys@useobject{currentmarker}{}%
\end{pgfscope}%
\end{pgfscope}%
\begin{pgfscope}%
\pgfsetbuttcap%
\pgfsetroundjoin%
\definecolor{currentfill}{rgb}{0.000000,0.000000,0.000000}%
\pgfsetfillcolor{currentfill}%
\pgfsetlinewidth{0.501875pt}%
\definecolor{currentstroke}{rgb}{0.000000,0.000000,0.000000}%
\pgfsetstrokecolor{currentstroke}%
\pgfsetdash{}{0pt}%
\pgfsys@defobject{currentmarker}{\pgfqpoint{0.000000in}{-0.020833in}}{\pgfqpoint{0.000000in}{0.000000in}}{%
\pgfpathmoveto{\pgfqpoint{0.000000in}{0.000000in}}%
\pgfpathlineto{\pgfqpoint{0.000000in}{-0.020833in}}%
\pgfusepath{stroke,fill}%
}%
\begin{pgfscope}%
\pgfsys@transformshift{3.591932in}{0.893003in}%
\pgfsys@useobject{currentmarker}{}%
\end{pgfscope}%
\end{pgfscope}%
\begin{pgfscope}%
\pgfsetbuttcap%
\pgfsetroundjoin%
\definecolor{currentfill}{rgb}{0.000000,0.000000,0.000000}%
\pgfsetfillcolor{currentfill}%
\pgfsetlinewidth{0.501875pt}%
\definecolor{currentstroke}{rgb}{0.000000,0.000000,0.000000}%
\pgfsetstrokecolor{currentstroke}%
\pgfsetdash{}{0pt}%
\pgfsys@defobject{currentmarker}{\pgfqpoint{0.000000in}{0.000000in}}{\pgfqpoint{0.000000in}{0.020833in}}{%
\pgfpathmoveto{\pgfqpoint{0.000000in}{0.000000in}}%
\pgfpathlineto{\pgfqpoint{0.000000in}{0.020833in}}%
\pgfusepath{stroke,fill}%
}%
\begin{pgfscope}%
\pgfsys@transformshift{3.662973in}{0.586309in}%
\pgfsys@useobject{currentmarker}{}%
\end{pgfscope}%
\end{pgfscope}%
\begin{pgfscope}%
\pgfsetbuttcap%
\pgfsetroundjoin%
\definecolor{currentfill}{rgb}{0.000000,0.000000,0.000000}%
\pgfsetfillcolor{currentfill}%
\pgfsetlinewidth{0.501875pt}%
\definecolor{currentstroke}{rgb}{0.000000,0.000000,0.000000}%
\pgfsetstrokecolor{currentstroke}%
\pgfsetdash{}{0pt}%
\pgfsys@defobject{currentmarker}{\pgfqpoint{0.000000in}{-0.020833in}}{\pgfqpoint{0.000000in}{0.000000in}}{%
\pgfpathmoveto{\pgfqpoint{0.000000in}{0.000000in}}%
\pgfpathlineto{\pgfqpoint{0.000000in}{-0.020833in}}%
\pgfusepath{stroke,fill}%
}%
\begin{pgfscope}%
\pgfsys@transformshift{3.662973in}{0.893003in}%
\pgfsys@useobject{currentmarker}{}%
\end{pgfscope}%
\end{pgfscope}%
\begin{pgfscope}%
\pgfsetbuttcap%
\pgfsetroundjoin%
\definecolor{currentfill}{rgb}{0.000000,0.000000,0.000000}%
\pgfsetfillcolor{currentfill}%
\pgfsetlinewidth{0.501875pt}%
\definecolor{currentstroke}{rgb}{0.000000,0.000000,0.000000}%
\pgfsetstrokecolor{currentstroke}%
\pgfsetdash{}{0pt}%
\pgfsys@defobject{currentmarker}{\pgfqpoint{0.000000in}{0.000000in}}{\pgfqpoint{0.000000in}{0.020833in}}{%
\pgfpathmoveto{\pgfqpoint{0.000000in}{0.000000in}}%
\pgfpathlineto{\pgfqpoint{0.000000in}{0.020833in}}%
\pgfusepath{stroke,fill}%
}%
\begin{pgfscope}%
\pgfsys@transformshift{3.698494in}{0.586309in}%
\pgfsys@useobject{currentmarker}{}%
\end{pgfscope}%
\end{pgfscope}%
\begin{pgfscope}%
\pgfsetbuttcap%
\pgfsetroundjoin%
\definecolor{currentfill}{rgb}{0.000000,0.000000,0.000000}%
\pgfsetfillcolor{currentfill}%
\pgfsetlinewidth{0.501875pt}%
\definecolor{currentstroke}{rgb}{0.000000,0.000000,0.000000}%
\pgfsetstrokecolor{currentstroke}%
\pgfsetdash{}{0pt}%
\pgfsys@defobject{currentmarker}{\pgfqpoint{0.000000in}{-0.020833in}}{\pgfqpoint{0.000000in}{0.000000in}}{%
\pgfpathmoveto{\pgfqpoint{0.000000in}{0.000000in}}%
\pgfpathlineto{\pgfqpoint{0.000000in}{-0.020833in}}%
\pgfusepath{stroke,fill}%
}%
\begin{pgfscope}%
\pgfsys@transformshift{3.698494in}{0.893003in}%
\pgfsys@useobject{currentmarker}{}%
\end{pgfscope}%
\end{pgfscope}%
\begin{pgfscope}%
\pgfsetbuttcap%
\pgfsetroundjoin%
\definecolor{currentfill}{rgb}{0.000000,0.000000,0.000000}%
\pgfsetfillcolor{currentfill}%
\pgfsetlinewidth{0.501875pt}%
\definecolor{currentstroke}{rgb}{0.000000,0.000000,0.000000}%
\pgfsetstrokecolor{currentstroke}%
\pgfsetdash{}{0pt}%
\pgfsys@defobject{currentmarker}{\pgfqpoint{0.000000in}{0.000000in}}{\pgfqpoint{0.000000in}{0.020833in}}{%
\pgfpathmoveto{\pgfqpoint{0.000000in}{0.000000in}}%
\pgfpathlineto{\pgfqpoint{0.000000in}{0.020833in}}%
\pgfusepath{stroke,fill}%
}%
\begin{pgfscope}%
\pgfsys@transformshift{3.734014in}{0.586309in}%
\pgfsys@useobject{currentmarker}{}%
\end{pgfscope}%
\end{pgfscope}%
\begin{pgfscope}%
\pgfsetbuttcap%
\pgfsetroundjoin%
\definecolor{currentfill}{rgb}{0.000000,0.000000,0.000000}%
\pgfsetfillcolor{currentfill}%
\pgfsetlinewidth{0.501875pt}%
\definecolor{currentstroke}{rgb}{0.000000,0.000000,0.000000}%
\pgfsetstrokecolor{currentstroke}%
\pgfsetdash{}{0pt}%
\pgfsys@defobject{currentmarker}{\pgfqpoint{0.000000in}{-0.020833in}}{\pgfqpoint{0.000000in}{0.000000in}}{%
\pgfpathmoveto{\pgfqpoint{0.000000in}{0.000000in}}%
\pgfpathlineto{\pgfqpoint{0.000000in}{-0.020833in}}%
\pgfusepath{stroke,fill}%
}%
\begin{pgfscope}%
\pgfsys@transformshift{3.734014in}{0.893003in}%
\pgfsys@useobject{currentmarker}{}%
\end{pgfscope}%
\end{pgfscope}%
\begin{pgfscope}%
\pgfsetbuttcap%
\pgfsetroundjoin%
\definecolor{currentfill}{rgb}{0.000000,0.000000,0.000000}%
\pgfsetfillcolor{currentfill}%
\pgfsetlinewidth{0.501875pt}%
\definecolor{currentstroke}{rgb}{0.000000,0.000000,0.000000}%
\pgfsetstrokecolor{currentstroke}%
\pgfsetdash{}{0pt}%
\pgfsys@defobject{currentmarker}{\pgfqpoint{0.000000in}{0.000000in}}{\pgfqpoint{0.000000in}{0.020833in}}{%
\pgfpathmoveto{\pgfqpoint{0.000000in}{0.000000in}}%
\pgfpathlineto{\pgfqpoint{0.000000in}{0.020833in}}%
\pgfusepath{stroke,fill}%
}%
\begin{pgfscope}%
\pgfsys@transformshift{3.769535in}{0.586309in}%
\pgfsys@useobject{currentmarker}{}%
\end{pgfscope}%
\end{pgfscope}%
\begin{pgfscope}%
\pgfsetbuttcap%
\pgfsetroundjoin%
\definecolor{currentfill}{rgb}{0.000000,0.000000,0.000000}%
\pgfsetfillcolor{currentfill}%
\pgfsetlinewidth{0.501875pt}%
\definecolor{currentstroke}{rgb}{0.000000,0.000000,0.000000}%
\pgfsetstrokecolor{currentstroke}%
\pgfsetdash{}{0pt}%
\pgfsys@defobject{currentmarker}{\pgfqpoint{0.000000in}{-0.020833in}}{\pgfqpoint{0.000000in}{0.000000in}}{%
\pgfpathmoveto{\pgfqpoint{0.000000in}{0.000000in}}%
\pgfpathlineto{\pgfqpoint{0.000000in}{-0.020833in}}%
\pgfusepath{stroke,fill}%
}%
\begin{pgfscope}%
\pgfsys@transformshift{3.769535in}{0.893003in}%
\pgfsys@useobject{currentmarker}{}%
\end{pgfscope}%
\end{pgfscope}%
\begin{pgfscope}%
\pgfsetbuttcap%
\pgfsetroundjoin%
\definecolor{currentfill}{rgb}{0.000000,0.000000,0.000000}%
\pgfsetfillcolor{currentfill}%
\pgfsetlinewidth{0.501875pt}%
\definecolor{currentstroke}{rgb}{0.000000,0.000000,0.000000}%
\pgfsetstrokecolor{currentstroke}%
\pgfsetdash{}{0pt}%
\pgfsys@defobject{currentmarker}{\pgfqpoint{0.000000in}{0.000000in}}{\pgfqpoint{0.000000in}{0.020833in}}{%
\pgfpathmoveto{\pgfqpoint{0.000000in}{0.000000in}}%
\pgfpathlineto{\pgfqpoint{0.000000in}{0.020833in}}%
\pgfusepath{stroke,fill}%
}%
\begin{pgfscope}%
\pgfsys@transformshift{3.805055in}{0.586309in}%
\pgfsys@useobject{currentmarker}{}%
\end{pgfscope}%
\end{pgfscope}%
\begin{pgfscope}%
\pgfsetbuttcap%
\pgfsetroundjoin%
\definecolor{currentfill}{rgb}{0.000000,0.000000,0.000000}%
\pgfsetfillcolor{currentfill}%
\pgfsetlinewidth{0.501875pt}%
\definecolor{currentstroke}{rgb}{0.000000,0.000000,0.000000}%
\pgfsetstrokecolor{currentstroke}%
\pgfsetdash{}{0pt}%
\pgfsys@defobject{currentmarker}{\pgfqpoint{0.000000in}{-0.020833in}}{\pgfqpoint{0.000000in}{0.000000in}}{%
\pgfpathmoveto{\pgfqpoint{0.000000in}{0.000000in}}%
\pgfpathlineto{\pgfqpoint{0.000000in}{-0.020833in}}%
\pgfusepath{stroke,fill}%
}%
\begin{pgfscope}%
\pgfsys@transformshift{3.805055in}{0.893003in}%
\pgfsys@useobject{currentmarker}{}%
\end{pgfscope}%
\end{pgfscope}%
\begin{pgfscope}%
\pgfsetbuttcap%
\pgfsetroundjoin%
\definecolor{currentfill}{rgb}{0.000000,0.000000,0.000000}%
\pgfsetfillcolor{currentfill}%
\pgfsetlinewidth{0.501875pt}%
\definecolor{currentstroke}{rgb}{0.000000,0.000000,0.000000}%
\pgfsetstrokecolor{currentstroke}%
\pgfsetdash{}{0pt}%
\pgfsys@defobject{currentmarker}{\pgfqpoint{0.000000in}{0.000000in}}{\pgfqpoint{0.000000in}{0.020833in}}{%
\pgfpathmoveto{\pgfqpoint{0.000000in}{0.000000in}}%
\pgfpathlineto{\pgfqpoint{0.000000in}{0.020833in}}%
\pgfusepath{stroke,fill}%
}%
\begin{pgfscope}%
\pgfsys@transformshift{3.840576in}{0.586309in}%
\pgfsys@useobject{currentmarker}{}%
\end{pgfscope}%
\end{pgfscope}%
\begin{pgfscope}%
\pgfsetbuttcap%
\pgfsetroundjoin%
\definecolor{currentfill}{rgb}{0.000000,0.000000,0.000000}%
\pgfsetfillcolor{currentfill}%
\pgfsetlinewidth{0.501875pt}%
\definecolor{currentstroke}{rgb}{0.000000,0.000000,0.000000}%
\pgfsetstrokecolor{currentstroke}%
\pgfsetdash{}{0pt}%
\pgfsys@defobject{currentmarker}{\pgfqpoint{0.000000in}{-0.020833in}}{\pgfqpoint{0.000000in}{0.000000in}}{%
\pgfpathmoveto{\pgfqpoint{0.000000in}{0.000000in}}%
\pgfpathlineto{\pgfqpoint{0.000000in}{-0.020833in}}%
\pgfusepath{stroke,fill}%
}%
\begin{pgfscope}%
\pgfsys@transformshift{3.840576in}{0.893003in}%
\pgfsys@useobject{currentmarker}{}%
\end{pgfscope}%
\end{pgfscope}%
\begin{pgfscope}%
\pgfsetbuttcap%
\pgfsetroundjoin%
\definecolor{currentfill}{rgb}{0.000000,0.000000,0.000000}%
\pgfsetfillcolor{currentfill}%
\pgfsetlinewidth{0.501875pt}%
\definecolor{currentstroke}{rgb}{0.000000,0.000000,0.000000}%
\pgfsetstrokecolor{currentstroke}%
\pgfsetdash{}{0pt}%
\pgfsys@defobject{currentmarker}{\pgfqpoint{0.000000in}{0.000000in}}{\pgfqpoint{0.000000in}{0.020833in}}{%
\pgfpathmoveto{\pgfqpoint{0.000000in}{0.000000in}}%
\pgfpathlineto{\pgfqpoint{0.000000in}{0.020833in}}%
\pgfusepath{stroke,fill}%
}%
\begin{pgfscope}%
\pgfsys@transformshift{3.876096in}{0.586309in}%
\pgfsys@useobject{currentmarker}{}%
\end{pgfscope}%
\end{pgfscope}%
\begin{pgfscope}%
\pgfsetbuttcap%
\pgfsetroundjoin%
\definecolor{currentfill}{rgb}{0.000000,0.000000,0.000000}%
\pgfsetfillcolor{currentfill}%
\pgfsetlinewidth{0.501875pt}%
\definecolor{currentstroke}{rgb}{0.000000,0.000000,0.000000}%
\pgfsetstrokecolor{currentstroke}%
\pgfsetdash{}{0pt}%
\pgfsys@defobject{currentmarker}{\pgfqpoint{0.000000in}{-0.020833in}}{\pgfqpoint{0.000000in}{0.000000in}}{%
\pgfpathmoveto{\pgfqpoint{0.000000in}{0.000000in}}%
\pgfpathlineto{\pgfqpoint{0.000000in}{-0.020833in}}%
\pgfusepath{stroke,fill}%
}%
\begin{pgfscope}%
\pgfsys@transformshift{3.876096in}{0.893003in}%
\pgfsys@useobject{currentmarker}{}%
\end{pgfscope}%
\end{pgfscope}%
\begin{pgfscope}%
\pgfsetbuttcap%
\pgfsetroundjoin%
\definecolor{currentfill}{rgb}{0.000000,0.000000,0.000000}%
\pgfsetfillcolor{currentfill}%
\pgfsetlinewidth{0.501875pt}%
\definecolor{currentstroke}{rgb}{0.000000,0.000000,0.000000}%
\pgfsetstrokecolor{currentstroke}%
\pgfsetdash{}{0pt}%
\pgfsys@defobject{currentmarker}{\pgfqpoint{0.000000in}{0.000000in}}{\pgfqpoint{0.000000in}{0.020833in}}{%
\pgfpathmoveto{\pgfqpoint{0.000000in}{0.000000in}}%
\pgfpathlineto{\pgfqpoint{0.000000in}{0.020833in}}%
\pgfusepath{stroke,fill}%
}%
\begin{pgfscope}%
\pgfsys@transformshift{3.911617in}{0.586309in}%
\pgfsys@useobject{currentmarker}{}%
\end{pgfscope}%
\end{pgfscope}%
\begin{pgfscope}%
\pgfsetbuttcap%
\pgfsetroundjoin%
\definecolor{currentfill}{rgb}{0.000000,0.000000,0.000000}%
\pgfsetfillcolor{currentfill}%
\pgfsetlinewidth{0.501875pt}%
\definecolor{currentstroke}{rgb}{0.000000,0.000000,0.000000}%
\pgfsetstrokecolor{currentstroke}%
\pgfsetdash{}{0pt}%
\pgfsys@defobject{currentmarker}{\pgfqpoint{0.000000in}{-0.020833in}}{\pgfqpoint{0.000000in}{0.000000in}}{%
\pgfpathmoveto{\pgfqpoint{0.000000in}{0.000000in}}%
\pgfpathlineto{\pgfqpoint{0.000000in}{-0.020833in}}%
\pgfusepath{stroke,fill}%
}%
\begin{pgfscope}%
\pgfsys@transformshift{3.911617in}{0.893003in}%
\pgfsys@useobject{currentmarker}{}%
\end{pgfscope}%
\end{pgfscope}%
\begin{pgfscope}%
\pgfsetbuttcap%
\pgfsetroundjoin%
\definecolor{currentfill}{rgb}{0.000000,0.000000,0.000000}%
\pgfsetfillcolor{currentfill}%
\pgfsetlinewidth{0.501875pt}%
\definecolor{currentstroke}{rgb}{0.000000,0.000000,0.000000}%
\pgfsetstrokecolor{currentstroke}%
\pgfsetdash{}{0pt}%
\pgfsys@defobject{currentmarker}{\pgfqpoint{0.000000in}{0.000000in}}{\pgfqpoint{0.000000in}{0.020833in}}{%
\pgfpathmoveto{\pgfqpoint{0.000000in}{0.000000in}}%
\pgfpathlineto{\pgfqpoint{0.000000in}{0.020833in}}%
\pgfusepath{stroke,fill}%
}%
\begin{pgfscope}%
\pgfsys@transformshift{3.947137in}{0.586309in}%
\pgfsys@useobject{currentmarker}{}%
\end{pgfscope}%
\end{pgfscope}%
\begin{pgfscope}%
\pgfsetbuttcap%
\pgfsetroundjoin%
\definecolor{currentfill}{rgb}{0.000000,0.000000,0.000000}%
\pgfsetfillcolor{currentfill}%
\pgfsetlinewidth{0.501875pt}%
\definecolor{currentstroke}{rgb}{0.000000,0.000000,0.000000}%
\pgfsetstrokecolor{currentstroke}%
\pgfsetdash{}{0pt}%
\pgfsys@defobject{currentmarker}{\pgfqpoint{0.000000in}{-0.020833in}}{\pgfqpoint{0.000000in}{0.000000in}}{%
\pgfpathmoveto{\pgfqpoint{0.000000in}{0.000000in}}%
\pgfpathlineto{\pgfqpoint{0.000000in}{-0.020833in}}%
\pgfusepath{stroke,fill}%
}%
\begin{pgfscope}%
\pgfsys@transformshift{3.947137in}{0.893003in}%
\pgfsys@useobject{currentmarker}{}%
\end{pgfscope}%
\end{pgfscope}%
\begin{pgfscope}%
\pgfsetbuttcap%
\pgfsetroundjoin%
\definecolor{currentfill}{rgb}{0.000000,0.000000,0.000000}%
\pgfsetfillcolor{currentfill}%
\pgfsetlinewidth{0.501875pt}%
\definecolor{currentstroke}{rgb}{0.000000,0.000000,0.000000}%
\pgfsetstrokecolor{currentstroke}%
\pgfsetdash{}{0pt}%
\pgfsys@defobject{currentmarker}{\pgfqpoint{0.000000in}{0.000000in}}{\pgfqpoint{0.000000in}{0.020833in}}{%
\pgfpathmoveto{\pgfqpoint{0.000000in}{0.000000in}}%
\pgfpathlineto{\pgfqpoint{0.000000in}{0.020833in}}%
\pgfusepath{stroke,fill}%
}%
\begin{pgfscope}%
\pgfsys@transformshift{3.982658in}{0.586309in}%
\pgfsys@useobject{currentmarker}{}%
\end{pgfscope}%
\end{pgfscope}%
\begin{pgfscope}%
\pgfsetbuttcap%
\pgfsetroundjoin%
\definecolor{currentfill}{rgb}{0.000000,0.000000,0.000000}%
\pgfsetfillcolor{currentfill}%
\pgfsetlinewidth{0.501875pt}%
\definecolor{currentstroke}{rgb}{0.000000,0.000000,0.000000}%
\pgfsetstrokecolor{currentstroke}%
\pgfsetdash{}{0pt}%
\pgfsys@defobject{currentmarker}{\pgfqpoint{0.000000in}{-0.020833in}}{\pgfqpoint{0.000000in}{0.000000in}}{%
\pgfpathmoveto{\pgfqpoint{0.000000in}{0.000000in}}%
\pgfpathlineto{\pgfqpoint{0.000000in}{-0.020833in}}%
\pgfusepath{stroke,fill}%
}%
\begin{pgfscope}%
\pgfsys@transformshift{3.982658in}{0.893003in}%
\pgfsys@useobject{currentmarker}{}%
\end{pgfscope}%
\end{pgfscope}%
\begin{pgfscope}%
\pgfsetbuttcap%
\pgfsetroundjoin%
\definecolor{currentfill}{rgb}{0.000000,0.000000,0.000000}%
\pgfsetfillcolor{currentfill}%
\pgfsetlinewidth{0.501875pt}%
\definecolor{currentstroke}{rgb}{0.000000,0.000000,0.000000}%
\pgfsetstrokecolor{currentstroke}%
\pgfsetdash{}{0pt}%
\pgfsys@defobject{currentmarker}{\pgfqpoint{0.000000in}{0.000000in}}{\pgfqpoint{0.000000in}{0.020833in}}{%
\pgfpathmoveto{\pgfqpoint{0.000000in}{0.000000in}}%
\pgfpathlineto{\pgfqpoint{0.000000in}{0.020833in}}%
\pgfusepath{stroke,fill}%
}%
\begin{pgfscope}%
\pgfsys@transformshift{4.018178in}{0.586309in}%
\pgfsys@useobject{currentmarker}{}%
\end{pgfscope}%
\end{pgfscope}%
\begin{pgfscope}%
\pgfsetbuttcap%
\pgfsetroundjoin%
\definecolor{currentfill}{rgb}{0.000000,0.000000,0.000000}%
\pgfsetfillcolor{currentfill}%
\pgfsetlinewidth{0.501875pt}%
\definecolor{currentstroke}{rgb}{0.000000,0.000000,0.000000}%
\pgfsetstrokecolor{currentstroke}%
\pgfsetdash{}{0pt}%
\pgfsys@defobject{currentmarker}{\pgfqpoint{0.000000in}{-0.020833in}}{\pgfqpoint{0.000000in}{0.000000in}}{%
\pgfpathmoveto{\pgfqpoint{0.000000in}{0.000000in}}%
\pgfpathlineto{\pgfqpoint{0.000000in}{-0.020833in}}%
\pgfusepath{stroke,fill}%
}%
\begin{pgfscope}%
\pgfsys@transformshift{4.018178in}{0.893003in}%
\pgfsys@useobject{currentmarker}{}%
\end{pgfscope}%
\end{pgfscope}%
\begin{pgfscope}%
\pgfsetbuttcap%
\pgfsetroundjoin%
\definecolor{currentfill}{rgb}{0.000000,0.000000,0.000000}%
\pgfsetfillcolor{currentfill}%
\pgfsetlinewidth{0.501875pt}%
\definecolor{currentstroke}{rgb}{0.000000,0.000000,0.000000}%
\pgfsetstrokecolor{currentstroke}%
\pgfsetdash{}{0pt}%
\pgfsys@defobject{currentmarker}{\pgfqpoint{0.000000in}{0.000000in}}{\pgfqpoint{0.000000in}{0.020833in}}{%
\pgfpathmoveto{\pgfqpoint{0.000000in}{0.000000in}}%
\pgfpathlineto{\pgfqpoint{0.000000in}{0.020833in}}%
\pgfusepath{stroke,fill}%
}%
\begin{pgfscope}%
\pgfsys@transformshift{4.089220in}{0.586309in}%
\pgfsys@useobject{currentmarker}{}%
\end{pgfscope}%
\end{pgfscope}%
\begin{pgfscope}%
\pgfsetbuttcap%
\pgfsetroundjoin%
\definecolor{currentfill}{rgb}{0.000000,0.000000,0.000000}%
\pgfsetfillcolor{currentfill}%
\pgfsetlinewidth{0.501875pt}%
\definecolor{currentstroke}{rgb}{0.000000,0.000000,0.000000}%
\pgfsetstrokecolor{currentstroke}%
\pgfsetdash{}{0pt}%
\pgfsys@defobject{currentmarker}{\pgfqpoint{0.000000in}{-0.020833in}}{\pgfqpoint{0.000000in}{0.000000in}}{%
\pgfpathmoveto{\pgfqpoint{0.000000in}{0.000000in}}%
\pgfpathlineto{\pgfqpoint{0.000000in}{-0.020833in}}%
\pgfusepath{stroke,fill}%
}%
\begin{pgfscope}%
\pgfsys@transformshift{4.089220in}{0.893003in}%
\pgfsys@useobject{currentmarker}{}%
\end{pgfscope}%
\end{pgfscope}%
\begin{pgfscope}%
\pgfsetbuttcap%
\pgfsetroundjoin%
\definecolor{currentfill}{rgb}{0.000000,0.000000,0.000000}%
\pgfsetfillcolor{currentfill}%
\pgfsetlinewidth{0.501875pt}%
\definecolor{currentstroke}{rgb}{0.000000,0.000000,0.000000}%
\pgfsetstrokecolor{currentstroke}%
\pgfsetdash{}{0pt}%
\pgfsys@defobject{currentmarker}{\pgfqpoint{0.000000in}{0.000000in}}{\pgfqpoint{0.000000in}{0.020833in}}{%
\pgfpathmoveto{\pgfqpoint{0.000000in}{0.000000in}}%
\pgfpathlineto{\pgfqpoint{0.000000in}{0.020833in}}%
\pgfusepath{stroke,fill}%
}%
\begin{pgfscope}%
\pgfsys@transformshift{4.124740in}{0.586309in}%
\pgfsys@useobject{currentmarker}{}%
\end{pgfscope}%
\end{pgfscope}%
\begin{pgfscope}%
\pgfsetbuttcap%
\pgfsetroundjoin%
\definecolor{currentfill}{rgb}{0.000000,0.000000,0.000000}%
\pgfsetfillcolor{currentfill}%
\pgfsetlinewidth{0.501875pt}%
\definecolor{currentstroke}{rgb}{0.000000,0.000000,0.000000}%
\pgfsetstrokecolor{currentstroke}%
\pgfsetdash{}{0pt}%
\pgfsys@defobject{currentmarker}{\pgfqpoint{0.000000in}{-0.020833in}}{\pgfqpoint{0.000000in}{0.000000in}}{%
\pgfpathmoveto{\pgfqpoint{0.000000in}{0.000000in}}%
\pgfpathlineto{\pgfqpoint{0.000000in}{-0.020833in}}%
\pgfusepath{stroke,fill}%
}%
\begin{pgfscope}%
\pgfsys@transformshift{4.124740in}{0.893003in}%
\pgfsys@useobject{currentmarker}{}%
\end{pgfscope}%
\end{pgfscope}%
\begin{pgfscope}%
\pgfsetbuttcap%
\pgfsetroundjoin%
\definecolor{currentfill}{rgb}{0.000000,0.000000,0.000000}%
\pgfsetfillcolor{currentfill}%
\pgfsetlinewidth{0.501875pt}%
\definecolor{currentstroke}{rgb}{0.000000,0.000000,0.000000}%
\pgfsetstrokecolor{currentstroke}%
\pgfsetdash{}{0pt}%
\pgfsys@defobject{currentmarker}{\pgfqpoint{0.000000in}{0.000000in}}{\pgfqpoint{0.000000in}{0.020833in}}{%
\pgfpathmoveto{\pgfqpoint{0.000000in}{0.000000in}}%
\pgfpathlineto{\pgfqpoint{0.000000in}{0.020833in}}%
\pgfusepath{stroke,fill}%
}%
\begin{pgfscope}%
\pgfsys@transformshift{4.160261in}{0.586309in}%
\pgfsys@useobject{currentmarker}{}%
\end{pgfscope}%
\end{pgfscope}%
\begin{pgfscope}%
\pgfsetbuttcap%
\pgfsetroundjoin%
\definecolor{currentfill}{rgb}{0.000000,0.000000,0.000000}%
\pgfsetfillcolor{currentfill}%
\pgfsetlinewidth{0.501875pt}%
\definecolor{currentstroke}{rgb}{0.000000,0.000000,0.000000}%
\pgfsetstrokecolor{currentstroke}%
\pgfsetdash{}{0pt}%
\pgfsys@defobject{currentmarker}{\pgfqpoint{0.000000in}{-0.020833in}}{\pgfqpoint{0.000000in}{0.000000in}}{%
\pgfpathmoveto{\pgfqpoint{0.000000in}{0.000000in}}%
\pgfpathlineto{\pgfqpoint{0.000000in}{-0.020833in}}%
\pgfusepath{stroke,fill}%
}%
\begin{pgfscope}%
\pgfsys@transformshift{4.160261in}{0.893003in}%
\pgfsys@useobject{currentmarker}{}%
\end{pgfscope}%
\end{pgfscope}%
\begin{pgfscope}%
\pgfsetbuttcap%
\pgfsetroundjoin%
\definecolor{currentfill}{rgb}{0.000000,0.000000,0.000000}%
\pgfsetfillcolor{currentfill}%
\pgfsetlinewidth{0.501875pt}%
\definecolor{currentstroke}{rgb}{0.000000,0.000000,0.000000}%
\pgfsetstrokecolor{currentstroke}%
\pgfsetdash{}{0pt}%
\pgfsys@defobject{currentmarker}{\pgfqpoint{0.000000in}{0.000000in}}{\pgfqpoint{0.000000in}{0.020833in}}{%
\pgfpathmoveto{\pgfqpoint{0.000000in}{0.000000in}}%
\pgfpathlineto{\pgfqpoint{0.000000in}{0.020833in}}%
\pgfusepath{stroke,fill}%
}%
\begin{pgfscope}%
\pgfsys@transformshift{4.195781in}{0.586309in}%
\pgfsys@useobject{currentmarker}{}%
\end{pgfscope}%
\end{pgfscope}%
\begin{pgfscope}%
\pgfsetbuttcap%
\pgfsetroundjoin%
\definecolor{currentfill}{rgb}{0.000000,0.000000,0.000000}%
\pgfsetfillcolor{currentfill}%
\pgfsetlinewidth{0.501875pt}%
\definecolor{currentstroke}{rgb}{0.000000,0.000000,0.000000}%
\pgfsetstrokecolor{currentstroke}%
\pgfsetdash{}{0pt}%
\pgfsys@defobject{currentmarker}{\pgfqpoint{0.000000in}{-0.020833in}}{\pgfqpoint{0.000000in}{0.000000in}}{%
\pgfpathmoveto{\pgfqpoint{0.000000in}{0.000000in}}%
\pgfpathlineto{\pgfqpoint{0.000000in}{-0.020833in}}%
\pgfusepath{stroke,fill}%
}%
\begin{pgfscope}%
\pgfsys@transformshift{4.195781in}{0.893003in}%
\pgfsys@useobject{currentmarker}{}%
\end{pgfscope}%
\end{pgfscope}%
\begin{pgfscope}%
\pgfsetbuttcap%
\pgfsetroundjoin%
\definecolor{currentfill}{rgb}{0.000000,0.000000,0.000000}%
\pgfsetfillcolor{currentfill}%
\pgfsetlinewidth{0.501875pt}%
\definecolor{currentstroke}{rgb}{0.000000,0.000000,0.000000}%
\pgfsetstrokecolor{currentstroke}%
\pgfsetdash{}{0pt}%
\pgfsys@defobject{currentmarker}{\pgfqpoint{0.000000in}{0.000000in}}{\pgfqpoint{0.000000in}{0.020833in}}{%
\pgfpathmoveto{\pgfqpoint{0.000000in}{0.000000in}}%
\pgfpathlineto{\pgfqpoint{0.000000in}{0.020833in}}%
\pgfusepath{stroke,fill}%
}%
\begin{pgfscope}%
\pgfsys@transformshift{4.231302in}{0.586309in}%
\pgfsys@useobject{currentmarker}{}%
\end{pgfscope}%
\end{pgfscope}%
\begin{pgfscope}%
\pgfsetbuttcap%
\pgfsetroundjoin%
\definecolor{currentfill}{rgb}{0.000000,0.000000,0.000000}%
\pgfsetfillcolor{currentfill}%
\pgfsetlinewidth{0.501875pt}%
\definecolor{currentstroke}{rgb}{0.000000,0.000000,0.000000}%
\pgfsetstrokecolor{currentstroke}%
\pgfsetdash{}{0pt}%
\pgfsys@defobject{currentmarker}{\pgfqpoint{0.000000in}{-0.020833in}}{\pgfqpoint{0.000000in}{0.000000in}}{%
\pgfpathmoveto{\pgfqpoint{0.000000in}{0.000000in}}%
\pgfpathlineto{\pgfqpoint{0.000000in}{-0.020833in}}%
\pgfusepath{stroke,fill}%
}%
\begin{pgfscope}%
\pgfsys@transformshift{4.231302in}{0.893003in}%
\pgfsys@useobject{currentmarker}{}%
\end{pgfscope}%
\end{pgfscope}%
\begin{pgfscope}%
\pgfsetbuttcap%
\pgfsetroundjoin%
\definecolor{currentfill}{rgb}{0.000000,0.000000,0.000000}%
\pgfsetfillcolor{currentfill}%
\pgfsetlinewidth{0.501875pt}%
\definecolor{currentstroke}{rgb}{0.000000,0.000000,0.000000}%
\pgfsetstrokecolor{currentstroke}%
\pgfsetdash{}{0pt}%
\pgfsys@defobject{currentmarker}{\pgfqpoint{0.000000in}{0.000000in}}{\pgfqpoint{0.000000in}{0.020833in}}{%
\pgfpathmoveto{\pgfqpoint{0.000000in}{0.000000in}}%
\pgfpathlineto{\pgfqpoint{0.000000in}{0.020833in}}%
\pgfusepath{stroke,fill}%
}%
\begin{pgfscope}%
\pgfsys@transformshift{4.266822in}{0.586309in}%
\pgfsys@useobject{currentmarker}{}%
\end{pgfscope}%
\end{pgfscope}%
\begin{pgfscope}%
\pgfsetbuttcap%
\pgfsetroundjoin%
\definecolor{currentfill}{rgb}{0.000000,0.000000,0.000000}%
\pgfsetfillcolor{currentfill}%
\pgfsetlinewidth{0.501875pt}%
\definecolor{currentstroke}{rgb}{0.000000,0.000000,0.000000}%
\pgfsetstrokecolor{currentstroke}%
\pgfsetdash{}{0pt}%
\pgfsys@defobject{currentmarker}{\pgfqpoint{0.000000in}{-0.020833in}}{\pgfqpoint{0.000000in}{0.000000in}}{%
\pgfpathmoveto{\pgfqpoint{0.000000in}{0.000000in}}%
\pgfpathlineto{\pgfqpoint{0.000000in}{-0.020833in}}%
\pgfusepath{stroke,fill}%
}%
\begin{pgfscope}%
\pgfsys@transformshift{4.266822in}{0.893003in}%
\pgfsys@useobject{currentmarker}{}%
\end{pgfscope}%
\end{pgfscope}%
\begin{pgfscope}%
\pgfsetbuttcap%
\pgfsetroundjoin%
\definecolor{currentfill}{rgb}{0.000000,0.000000,0.000000}%
\pgfsetfillcolor{currentfill}%
\pgfsetlinewidth{0.501875pt}%
\definecolor{currentstroke}{rgb}{0.000000,0.000000,0.000000}%
\pgfsetstrokecolor{currentstroke}%
\pgfsetdash{}{0pt}%
\pgfsys@defobject{currentmarker}{\pgfqpoint{0.000000in}{0.000000in}}{\pgfqpoint{0.000000in}{0.020833in}}{%
\pgfpathmoveto{\pgfqpoint{0.000000in}{0.000000in}}%
\pgfpathlineto{\pgfqpoint{0.000000in}{0.020833in}}%
\pgfusepath{stroke,fill}%
}%
\begin{pgfscope}%
\pgfsys@transformshift{4.302343in}{0.586309in}%
\pgfsys@useobject{currentmarker}{}%
\end{pgfscope}%
\end{pgfscope}%
\begin{pgfscope}%
\pgfsetbuttcap%
\pgfsetroundjoin%
\definecolor{currentfill}{rgb}{0.000000,0.000000,0.000000}%
\pgfsetfillcolor{currentfill}%
\pgfsetlinewidth{0.501875pt}%
\definecolor{currentstroke}{rgb}{0.000000,0.000000,0.000000}%
\pgfsetstrokecolor{currentstroke}%
\pgfsetdash{}{0pt}%
\pgfsys@defobject{currentmarker}{\pgfqpoint{0.000000in}{-0.020833in}}{\pgfqpoint{0.000000in}{0.000000in}}{%
\pgfpathmoveto{\pgfqpoint{0.000000in}{0.000000in}}%
\pgfpathlineto{\pgfqpoint{0.000000in}{-0.020833in}}%
\pgfusepath{stroke,fill}%
}%
\begin{pgfscope}%
\pgfsys@transformshift{4.302343in}{0.893003in}%
\pgfsys@useobject{currentmarker}{}%
\end{pgfscope}%
\end{pgfscope}%
\begin{pgfscope}%
\pgfsetbuttcap%
\pgfsetroundjoin%
\definecolor{currentfill}{rgb}{0.000000,0.000000,0.000000}%
\pgfsetfillcolor{currentfill}%
\pgfsetlinewidth{0.501875pt}%
\definecolor{currentstroke}{rgb}{0.000000,0.000000,0.000000}%
\pgfsetstrokecolor{currentstroke}%
\pgfsetdash{}{0pt}%
\pgfsys@defobject{currentmarker}{\pgfqpoint{0.000000in}{0.000000in}}{\pgfqpoint{0.000000in}{0.020833in}}{%
\pgfpathmoveto{\pgfqpoint{0.000000in}{0.000000in}}%
\pgfpathlineto{\pgfqpoint{0.000000in}{0.020833in}}%
\pgfusepath{stroke,fill}%
}%
\begin{pgfscope}%
\pgfsys@transformshift{4.337863in}{0.586309in}%
\pgfsys@useobject{currentmarker}{}%
\end{pgfscope}%
\end{pgfscope}%
\begin{pgfscope}%
\pgfsetbuttcap%
\pgfsetroundjoin%
\definecolor{currentfill}{rgb}{0.000000,0.000000,0.000000}%
\pgfsetfillcolor{currentfill}%
\pgfsetlinewidth{0.501875pt}%
\definecolor{currentstroke}{rgb}{0.000000,0.000000,0.000000}%
\pgfsetstrokecolor{currentstroke}%
\pgfsetdash{}{0pt}%
\pgfsys@defobject{currentmarker}{\pgfqpoint{0.000000in}{-0.020833in}}{\pgfqpoint{0.000000in}{0.000000in}}{%
\pgfpathmoveto{\pgfqpoint{0.000000in}{0.000000in}}%
\pgfpathlineto{\pgfqpoint{0.000000in}{-0.020833in}}%
\pgfusepath{stroke,fill}%
}%
\begin{pgfscope}%
\pgfsys@transformshift{4.337863in}{0.893003in}%
\pgfsys@useobject{currentmarker}{}%
\end{pgfscope}%
\end{pgfscope}%
\begin{pgfscope}%
\pgfsetbuttcap%
\pgfsetroundjoin%
\definecolor{currentfill}{rgb}{0.000000,0.000000,0.000000}%
\pgfsetfillcolor{currentfill}%
\pgfsetlinewidth{0.501875pt}%
\definecolor{currentstroke}{rgb}{0.000000,0.000000,0.000000}%
\pgfsetstrokecolor{currentstroke}%
\pgfsetdash{}{0pt}%
\pgfsys@defobject{currentmarker}{\pgfqpoint{0.000000in}{0.000000in}}{\pgfqpoint{0.000000in}{0.020833in}}{%
\pgfpathmoveto{\pgfqpoint{0.000000in}{0.000000in}}%
\pgfpathlineto{\pgfqpoint{0.000000in}{0.020833in}}%
\pgfusepath{stroke,fill}%
}%
\begin{pgfscope}%
\pgfsys@transformshift{4.373384in}{0.586309in}%
\pgfsys@useobject{currentmarker}{}%
\end{pgfscope}%
\end{pgfscope}%
\begin{pgfscope}%
\pgfsetbuttcap%
\pgfsetroundjoin%
\definecolor{currentfill}{rgb}{0.000000,0.000000,0.000000}%
\pgfsetfillcolor{currentfill}%
\pgfsetlinewidth{0.501875pt}%
\definecolor{currentstroke}{rgb}{0.000000,0.000000,0.000000}%
\pgfsetstrokecolor{currentstroke}%
\pgfsetdash{}{0pt}%
\pgfsys@defobject{currentmarker}{\pgfqpoint{0.000000in}{-0.020833in}}{\pgfqpoint{0.000000in}{0.000000in}}{%
\pgfpathmoveto{\pgfqpoint{0.000000in}{0.000000in}}%
\pgfpathlineto{\pgfqpoint{0.000000in}{-0.020833in}}%
\pgfusepath{stroke,fill}%
}%
\begin{pgfscope}%
\pgfsys@transformshift{4.373384in}{0.893003in}%
\pgfsys@useobject{currentmarker}{}%
\end{pgfscope}%
\end{pgfscope}%
\begin{pgfscope}%
\pgfsetbuttcap%
\pgfsetroundjoin%
\definecolor{currentfill}{rgb}{0.000000,0.000000,0.000000}%
\pgfsetfillcolor{currentfill}%
\pgfsetlinewidth{0.501875pt}%
\definecolor{currentstroke}{rgb}{0.000000,0.000000,0.000000}%
\pgfsetstrokecolor{currentstroke}%
\pgfsetdash{}{0pt}%
\pgfsys@defobject{currentmarker}{\pgfqpoint{0.000000in}{0.000000in}}{\pgfqpoint{0.000000in}{0.020833in}}{%
\pgfpathmoveto{\pgfqpoint{0.000000in}{0.000000in}}%
\pgfpathlineto{\pgfqpoint{0.000000in}{0.020833in}}%
\pgfusepath{stroke,fill}%
}%
\begin{pgfscope}%
\pgfsys@transformshift{4.408904in}{0.586309in}%
\pgfsys@useobject{currentmarker}{}%
\end{pgfscope}%
\end{pgfscope}%
\begin{pgfscope}%
\pgfsetbuttcap%
\pgfsetroundjoin%
\definecolor{currentfill}{rgb}{0.000000,0.000000,0.000000}%
\pgfsetfillcolor{currentfill}%
\pgfsetlinewidth{0.501875pt}%
\definecolor{currentstroke}{rgb}{0.000000,0.000000,0.000000}%
\pgfsetstrokecolor{currentstroke}%
\pgfsetdash{}{0pt}%
\pgfsys@defobject{currentmarker}{\pgfqpoint{0.000000in}{-0.020833in}}{\pgfqpoint{0.000000in}{0.000000in}}{%
\pgfpathmoveto{\pgfqpoint{0.000000in}{0.000000in}}%
\pgfpathlineto{\pgfqpoint{0.000000in}{-0.020833in}}%
\pgfusepath{stroke,fill}%
}%
\begin{pgfscope}%
\pgfsys@transformshift{4.408904in}{0.893003in}%
\pgfsys@useobject{currentmarker}{}%
\end{pgfscope}%
\end{pgfscope}%
\begin{pgfscope}%
\pgfsetbuttcap%
\pgfsetroundjoin%
\definecolor{currentfill}{rgb}{0.000000,0.000000,0.000000}%
\pgfsetfillcolor{currentfill}%
\pgfsetlinewidth{0.501875pt}%
\definecolor{currentstroke}{rgb}{0.000000,0.000000,0.000000}%
\pgfsetstrokecolor{currentstroke}%
\pgfsetdash{}{0pt}%
\pgfsys@defobject{currentmarker}{\pgfqpoint{0.000000in}{0.000000in}}{\pgfqpoint{0.000000in}{0.020833in}}{%
\pgfpathmoveto{\pgfqpoint{0.000000in}{0.000000in}}%
\pgfpathlineto{\pgfqpoint{0.000000in}{0.020833in}}%
\pgfusepath{stroke,fill}%
}%
\begin{pgfscope}%
\pgfsys@transformshift{4.444425in}{0.586309in}%
\pgfsys@useobject{currentmarker}{}%
\end{pgfscope}%
\end{pgfscope}%
\begin{pgfscope}%
\pgfsetbuttcap%
\pgfsetroundjoin%
\definecolor{currentfill}{rgb}{0.000000,0.000000,0.000000}%
\pgfsetfillcolor{currentfill}%
\pgfsetlinewidth{0.501875pt}%
\definecolor{currentstroke}{rgb}{0.000000,0.000000,0.000000}%
\pgfsetstrokecolor{currentstroke}%
\pgfsetdash{}{0pt}%
\pgfsys@defobject{currentmarker}{\pgfqpoint{0.000000in}{-0.020833in}}{\pgfqpoint{0.000000in}{0.000000in}}{%
\pgfpathmoveto{\pgfqpoint{0.000000in}{0.000000in}}%
\pgfpathlineto{\pgfqpoint{0.000000in}{-0.020833in}}%
\pgfusepath{stroke,fill}%
}%
\begin{pgfscope}%
\pgfsys@transformshift{4.444425in}{0.893003in}%
\pgfsys@useobject{currentmarker}{}%
\end{pgfscope}%
\end{pgfscope}%
\begin{pgfscope}%
\pgfsetbuttcap%
\pgfsetroundjoin%
\definecolor{currentfill}{rgb}{0.000000,0.000000,0.000000}%
\pgfsetfillcolor{currentfill}%
\pgfsetlinewidth{0.501875pt}%
\definecolor{currentstroke}{rgb}{0.000000,0.000000,0.000000}%
\pgfsetstrokecolor{currentstroke}%
\pgfsetdash{}{0pt}%
\pgfsys@defobject{currentmarker}{\pgfqpoint{0.000000in}{0.000000in}}{\pgfqpoint{0.000000in}{0.020833in}}{%
\pgfpathmoveto{\pgfqpoint{0.000000in}{0.000000in}}%
\pgfpathlineto{\pgfqpoint{0.000000in}{0.020833in}}%
\pgfusepath{stroke,fill}%
}%
\begin{pgfscope}%
\pgfsys@transformshift{4.515466in}{0.586309in}%
\pgfsys@useobject{currentmarker}{}%
\end{pgfscope}%
\end{pgfscope}%
\begin{pgfscope}%
\pgfsetbuttcap%
\pgfsetroundjoin%
\definecolor{currentfill}{rgb}{0.000000,0.000000,0.000000}%
\pgfsetfillcolor{currentfill}%
\pgfsetlinewidth{0.501875pt}%
\definecolor{currentstroke}{rgb}{0.000000,0.000000,0.000000}%
\pgfsetstrokecolor{currentstroke}%
\pgfsetdash{}{0pt}%
\pgfsys@defobject{currentmarker}{\pgfqpoint{0.000000in}{-0.020833in}}{\pgfqpoint{0.000000in}{0.000000in}}{%
\pgfpathmoveto{\pgfqpoint{0.000000in}{0.000000in}}%
\pgfpathlineto{\pgfqpoint{0.000000in}{-0.020833in}}%
\pgfusepath{stroke,fill}%
}%
\begin{pgfscope}%
\pgfsys@transformshift{4.515466in}{0.893003in}%
\pgfsys@useobject{currentmarker}{}%
\end{pgfscope}%
\end{pgfscope}%
\begin{pgfscope}%
\pgfsetbuttcap%
\pgfsetroundjoin%
\definecolor{currentfill}{rgb}{0.000000,0.000000,0.000000}%
\pgfsetfillcolor{currentfill}%
\pgfsetlinewidth{0.501875pt}%
\definecolor{currentstroke}{rgb}{0.000000,0.000000,0.000000}%
\pgfsetstrokecolor{currentstroke}%
\pgfsetdash{}{0pt}%
\pgfsys@defobject{currentmarker}{\pgfqpoint{0.000000in}{0.000000in}}{\pgfqpoint{0.000000in}{0.020833in}}{%
\pgfpathmoveto{\pgfqpoint{0.000000in}{0.000000in}}%
\pgfpathlineto{\pgfqpoint{0.000000in}{0.020833in}}%
\pgfusepath{stroke,fill}%
}%
\begin{pgfscope}%
\pgfsys@transformshift{4.550986in}{0.586309in}%
\pgfsys@useobject{currentmarker}{}%
\end{pgfscope}%
\end{pgfscope}%
\begin{pgfscope}%
\pgfsetbuttcap%
\pgfsetroundjoin%
\definecolor{currentfill}{rgb}{0.000000,0.000000,0.000000}%
\pgfsetfillcolor{currentfill}%
\pgfsetlinewidth{0.501875pt}%
\definecolor{currentstroke}{rgb}{0.000000,0.000000,0.000000}%
\pgfsetstrokecolor{currentstroke}%
\pgfsetdash{}{0pt}%
\pgfsys@defobject{currentmarker}{\pgfqpoint{0.000000in}{-0.020833in}}{\pgfqpoint{0.000000in}{0.000000in}}{%
\pgfpathmoveto{\pgfqpoint{0.000000in}{0.000000in}}%
\pgfpathlineto{\pgfqpoint{0.000000in}{-0.020833in}}%
\pgfusepath{stroke,fill}%
}%
\begin{pgfscope}%
\pgfsys@transformshift{4.550986in}{0.893003in}%
\pgfsys@useobject{currentmarker}{}%
\end{pgfscope}%
\end{pgfscope}%
\begin{pgfscope}%
\pgfsetbuttcap%
\pgfsetroundjoin%
\definecolor{currentfill}{rgb}{0.000000,0.000000,0.000000}%
\pgfsetfillcolor{currentfill}%
\pgfsetlinewidth{0.501875pt}%
\definecolor{currentstroke}{rgb}{0.000000,0.000000,0.000000}%
\pgfsetstrokecolor{currentstroke}%
\pgfsetdash{}{0pt}%
\pgfsys@defobject{currentmarker}{\pgfqpoint{0.000000in}{0.000000in}}{\pgfqpoint{0.000000in}{0.020833in}}{%
\pgfpathmoveto{\pgfqpoint{0.000000in}{0.000000in}}%
\pgfpathlineto{\pgfqpoint{0.000000in}{0.020833in}}%
\pgfusepath{stroke,fill}%
}%
\begin{pgfscope}%
\pgfsys@transformshift{4.586507in}{0.586309in}%
\pgfsys@useobject{currentmarker}{}%
\end{pgfscope}%
\end{pgfscope}%
\begin{pgfscope}%
\pgfsetbuttcap%
\pgfsetroundjoin%
\definecolor{currentfill}{rgb}{0.000000,0.000000,0.000000}%
\pgfsetfillcolor{currentfill}%
\pgfsetlinewidth{0.501875pt}%
\definecolor{currentstroke}{rgb}{0.000000,0.000000,0.000000}%
\pgfsetstrokecolor{currentstroke}%
\pgfsetdash{}{0pt}%
\pgfsys@defobject{currentmarker}{\pgfqpoint{0.000000in}{-0.020833in}}{\pgfqpoint{0.000000in}{0.000000in}}{%
\pgfpathmoveto{\pgfqpoint{0.000000in}{0.000000in}}%
\pgfpathlineto{\pgfqpoint{0.000000in}{-0.020833in}}%
\pgfusepath{stroke,fill}%
}%
\begin{pgfscope}%
\pgfsys@transformshift{4.586507in}{0.893003in}%
\pgfsys@useobject{currentmarker}{}%
\end{pgfscope}%
\end{pgfscope}%
\begin{pgfscope}%
\pgfsetbuttcap%
\pgfsetroundjoin%
\definecolor{currentfill}{rgb}{0.000000,0.000000,0.000000}%
\pgfsetfillcolor{currentfill}%
\pgfsetlinewidth{0.501875pt}%
\definecolor{currentstroke}{rgb}{0.000000,0.000000,0.000000}%
\pgfsetstrokecolor{currentstroke}%
\pgfsetdash{}{0pt}%
\pgfsys@defobject{currentmarker}{\pgfqpoint{0.000000in}{0.000000in}}{\pgfqpoint{0.000000in}{0.020833in}}{%
\pgfpathmoveto{\pgfqpoint{0.000000in}{0.000000in}}%
\pgfpathlineto{\pgfqpoint{0.000000in}{0.020833in}}%
\pgfusepath{stroke,fill}%
}%
\begin{pgfscope}%
\pgfsys@transformshift{4.622027in}{0.586309in}%
\pgfsys@useobject{currentmarker}{}%
\end{pgfscope}%
\end{pgfscope}%
\begin{pgfscope}%
\pgfsetbuttcap%
\pgfsetroundjoin%
\definecolor{currentfill}{rgb}{0.000000,0.000000,0.000000}%
\pgfsetfillcolor{currentfill}%
\pgfsetlinewidth{0.501875pt}%
\definecolor{currentstroke}{rgb}{0.000000,0.000000,0.000000}%
\pgfsetstrokecolor{currentstroke}%
\pgfsetdash{}{0pt}%
\pgfsys@defobject{currentmarker}{\pgfqpoint{0.000000in}{-0.020833in}}{\pgfqpoint{0.000000in}{0.000000in}}{%
\pgfpathmoveto{\pgfqpoint{0.000000in}{0.000000in}}%
\pgfpathlineto{\pgfqpoint{0.000000in}{-0.020833in}}%
\pgfusepath{stroke,fill}%
}%
\begin{pgfscope}%
\pgfsys@transformshift{4.622027in}{0.893003in}%
\pgfsys@useobject{currentmarker}{}%
\end{pgfscope}%
\end{pgfscope}%
\begin{pgfscope}%
\pgfsetbuttcap%
\pgfsetroundjoin%
\definecolor{currentfill}{rgb}{0.000000,0.000000,0.000000}%
\pgfsetfillcolor{currentfill}%
\pgfsetlinewidth{0.501875pt}%
\definecolor{currentstroke}{rgb}{0.000000,0.000000,0.000000}%
\pgfsetstrokecolor{currentstroke}%
\pgfsetdash{}{0pt}%
\pgfsys@defobject{currentmarker}{\pgfqpoint{0.000000in}{0.000000in}}{\pgfqpoint{0.000000in}{0.020833in}}{%
\pgfpathmoveto{\pgfqpoint{0.000000in}{0.000000in}}%
\pgfpathlineto{\pgfqpoint{0.000000in}{0.020833in}}%
\pgfusepath{stroke,fill}%
}%
\begin{pgfscope}%
\pgfsys@transformshift{4.657548in}{0.586309in}%
\pgfsys@useobject{currentmarker}{}%
\end{pgfscope}%
\end{pgfscope}%
\begin{pgfscope}%
\pgfsetbuttcap%
\pgfsetroundjoin%
\definecolor{currentfill}{rgb}{0.000000,0.000000,0.000000}%
\pgfsetfillcolor{currentfill}%
\pgfsetlinewidth{0.501875pt}%
\definecolor{currentstroke}{rgb}{0.000000,0.000000,0.000000}%
\pgfsetstrokecolor{currentstroke}%
\pgfsetdash{}{0pt}%
\pgfsys@defobject{currentmarker}{\pgfqpoint{0.000000in}{-0.020833in}}{\pgfqpoint{0.000000in}{0.000000in}}{%
\pgfpathmoveto{\pgfqpoint{0.000000in}{0.000000in}}%
\pgfpathlineto{\pgfqpoint{0.000000in}{-0.020833in}}%
\pgfusepath{stroke,fill}%
}%
\begin{pgfscope}%
\pgfsys@transformshift{4.657548in}{0.893003in}%
\pgfsys@useobject{currentmarker}{}%
\end{pgfscope}%
\end{pgfscope}%
\begin{pgfscope}%
\pgfsetbuttcap%
\pgfsetroundjoin%
\definecolor{currentfill}{rgb}{0.000000,0.000000,0.000000}%
\pgfsetfillcolor{currentfill}%
\pgfsetlinewidth{0.501875pt}%
\definecolor{currentstroke}{rgb}{0.000000,0.000000,0.000000}%
\pgfsetstrokecolor{currentstroke}%
\pgfsetdash{}{0pt}%
\pgfsys@defobject{currentmarker}{\pgfqpoint{0.000000in}{0.000000in}}{\pgfqpoint{0.000000in}{0.020833in}}{%
\pgfpathmoveto{\pgfqpoint{0.000000in}{0.000000in}}%
\pgfpathlineto{\pgfqpoint{0.000000in}{0.020833in}}%
\pgfusepath{stroke,fill}%
}%
\begin{pgfscope}%
\pgfsys@transformshift{4.693069in}{0.586309in}%
\pgfsys@useobject{currentmarker}{}%
\end{pgfscope}%
\end{pgfscope}%
\begin{pgfscope}%
\pgfsetbuttcap%
\pgfsetroundjoin%
\definecolor{currentfill}{rgb}{0.000000,0.000000,0.000000}%
\pgfsetfillcolor{currentfill}%
\pgfsetlinewidth{0.501875pt}%
\definecolor{currentstroke}{rgb}{0.000000,0.000000,0.000000}%
\pgfsetstrokecolor{currentstroke}%
\pgfsetdash{}{0pt}%
\pgfsys@defobject{currentmarker}{\pgfqpoint{0.000000in}{-0.020833in}}{\pgfqpoint{0.000000in}{0.000000in}}{%
\pgfpathmoveto{\pgfqpoint{0.000000in}{0.000000in}}%
\pgfpathlineto{\pgfqpoint{0.000000in}{-0.020833in}}%
\pgfusepath{stroke,fill}%
}%
\begin{pgfscope}%
\pgfsys@transformshift{4.693069in}{0.893003in}%
\pgfsys@useobject{currentmarker}{}%
\end{pgfscope}%
\end{pgfscope}%
\begin{pgfscope}%
\pgfsetbuttcap%
\pgfsetroundjoin%
\definecolor{currentfill}{rgb}{0.000000,0.000000,0.000000}%
\pgfsetfillcolor{currentfill}%
\pgfsetlinewidth{0.501875pt}%
\definecolor{currentstroke}{rgb}{0.000000,0.000000,0.000000}%
\pgfsetstrokecolor{currentstroke}%
\pgfsetdash{}{0pt}%
\pgfsys@defobject{currentmarker}{\pgfqpoint{0.000000in}{0.000000in}}{\pgfqpoint{0.000000in}{0.020833in}}{%
\pgfpathmoveto{\pgfqpoint{0.000000in}{0.000000in}}%
\pgfpathlineto{\pgfqpoint{0.000000in}{0.020833in}}%
\pgfusepath{stroke,fill}%
}%
\begin{pgfscope}%
\pgfsys@transformshift{4.728589in}{0.586309in}%
\pgfsys@useobject{currentmarker}{}%
\end{pgfscope}%
\end{pgfscope}%
\begin{pgfscope}%
\pgfsetbuttcap%
\pgfsetroundjoin%
\definecolor{currentfill}{rgb}{0.000000,0.000000,0.000000}%
\pgfsetfillcolor{currentfill}%
\pgfsetlinewidth{0.501875pt}%
\definecolor{currentstroke}{rgb}{0.000000,0.000000,0.000000}%
\pgfsetstrokecolor{currentstroke}%
\pgfsetdash{}{0pt}%
\pgfsys@defobject{currentmarker}{\pgfqpoint{0.000000in}{-0.020833in}}{\pgfqpoint{0.000000in}{0.000000in}}{%
\pgfpathmoveto{\pgfqpoint{0.000000in}{0.000000in}}%
\pgfpathlineto{\pgfqpoint{0.000000in}{-0.020833in}}%
\pgfusepath{stroke,fill}%
}%
\begin{pgfscope}%
\pgfsys@transformshift{4.728589in}{0.893003in}%
\pgfsys@useobject{currentmarker}{}%
\end{pgfscope}%
\end{pgfscope}%
\begin{pgfscope}%
\pgfsetbuttcap%
\pgfsetroundjoin%
\definecolor{currentfill}{rgb}{0.000000,0.000000,0.000000}%
\pgfsetfillcolor{currentfill}%
\pgfsetlinewidth{0.501875pt}%
\definecolor{currentstroke}{rgb}{0.000000,0.000000,0.000000}%
\pgfsetstrokecolor{currentstroke}%
\pgfsetdash{}{0pt}%
\pgfsys@defobject{currentmarker}{\pgfqpoint{0.000000in}{0.000000in}}{\pgfqpoint{0.000000in}{0.020833in}}{%
\pgfpathmoveto{\pgfqpoint{0.000000in}{0.000000in}}%
\pgfpathlineto{\pgfqpoint{0.000000in}{0.020833in}}%
\pgfusepath{stroke,fill}%
}%
\begin{pgfscope}%
\pgfsys@transformshift{4.764110in}{0.586309in}%
\pgfsys@useobject{currentmarker}{}%
\end{pgfscope}%
\end{pgfscope}%
\begin{pgfscope}%
\pgfsetbuttcap%
\pgfsetroundjoin%
\definecolor{currentfill}{rgb}{0.000000,0.000000,0.000000}%
\pgfsetfillcolor{currentfill}%
\pgfsetlinewidth{0.501875pt}%
\definecolor{currentstroke}{rgb}{0.000000,0.000000,0.000000}%
\pgfsetstrokecolor{currentstroke}%
\pgfsetdash{}{0pt}%
\pgfsys@defobject{currentmarker}{\pgfqpoint{0.000000in}{-0.020833in}}{\pgfqpoint{0.000000in}{0.000000in}}{%
\pgfpathmoveto{\pgfqpoint{0.000000in}{0.000000in}}%
\pgfpathlineto{\pgfqpoint{0.000000in}{-0.020833in}}%
\pgfusepath{stroke,fill}%
}%
\begin{pgfscope}%
\pgfsys@transformshift{4.764110in}{0.893003in}%
\pgfsys@useobject{currentmarker}{}%
\end{pgfscope}%
\end{pgfscope}%
\begin{pgfscope}%
\pgfsetbuttcap%
\pgfsetroundjoin%
\definecolor{currentfill}{rgb}{0.000000,0.000000,0.000000}%
\pgfsetfillcolor{currentfill}%
\pgfsetlinewidth{0.501875pt}%
\definecolor{currentstroke}{rgb}{0.000000,0.000000,0.000000}%
\pgfsetstrokecolor{currentstroke}%
\pgfsetdash{}{0pt}%
\pgfsys@defobject{currentmarker}{\pgfqpoint{0.000000in}{0.000000in}}{\pgfqpoint{0.000000in}{0.020833in}}{%
\pgfpathmoveto{\pgfqpoint{0.000000in}{0.000000in}}%
\pgfpathlineto{\pgfqpoint{0.000000in}{0.020833in}}%
\pgfusepath{stroke,fill}%
}%
\begin{pgfscope}%
\pgfsys@transformshift{4.799630in}{0.586309in}%
\pgfsys@useobject{currentmarker}{}%
\end{pgfscope}%
\end{pgfscope}%
\begin{pgfscope}%
\pgfsetbuttcap%
\pgfsetroundjoin%
\definecolor{currentfill}{rgb}{0.000000,0.000000,0.000000}%
\pgfsetfillcolor{currentfill}%
\pgfsetlinewidth{0.501875pt}%
\definecolor{currentstroke}{rgb}{0.000000,0.000000,0.000000}%
\pgfsetstrokecolor{currentstroke}%
\pgfsetdash{}{0pt}%
\pgfsys@defobject{currentmarker}{\pgfqpoint{0.000000in}{-0.020833in}}{\pgfqpoint{0.000000in}{0.000000in}}{%
\pgfpathmoveto{\pgfqpoint{0.000000in}{0.000000in}}%
\pgfpathlineto{\pgfqpoint{0.000000in}{-0.020833in}}%
\pgfusepath{stroke,fill}%
}%
\begin{pgfscope}%
\pgfsys@transformshift{4.799630in}{0.893003in}%
\pgfsys@useobject{currentmarker}{}%
\end{pgfscope}%
\end{pgfscope}%
\begin{pgfscope}%
\pgfsetbuttcap%
\pgfsetroundjoin%
\definecolor{currentfill}{rgb}{0.000000,0.000000,0.000000}%
\pgfsetfillcolor{currentfill}%
\pgfsetlinewidth{0.501875pt}%
\definecolor{currentstroke}{rgb}{0.000000,0.000000,0.000000}%
\pgfsetstrokecolor{currentstroke}%
\pgfsetdash{}{0pt}%
\pgfsys@defobject{currentmarker}{\pgfqpoint{0.000000in}{0.000000in}}{\pgfqpoint{0.000000in}{0.020833in}}{%
\pgfpathmoveto{\pgfqpoint{0.000000in}{0.000000in}}%
\pgfpathlineto{\pgfqpoint{0.000000in}{0.020833in}}%
\pgfusepath{stroke,fill}%
}%
\begin{pgfscope}%
\pgfsys@transformshift{4.835151in}{0.586309in}%
\pgfsys@useobject{currentmarker}{}%
\end{pgfscope}%
\end{pgfscope}%
\begin{pgfscope}%
\pgfsetbuttcap%
\pgfsetroundjoin%
\definecolor{currentfill}{rgb}{0.000000,0.000000,0.000000}%
\pgfsetfillcolor{currentfill}%
\pgfsetlinewidth{0.501875pt}%
\definecolor{currentstroke}{rgb}{0.000000,0.000000,0.000000}%
\pgfsetstrokecolor{currentstroke}%
\pgfsetdash{}{0pt}%
\pgfsys@defobject{currentmarker}{\pgfqpoint{0.000000in}{-0.020833in}}{\pgfqpoint{0.000000in}{0.000000in}}{%
\pgfpathmoveto{\pgfqpoint{0.000000in}{0.000000in}}%
\pgfpathlineto{\pgfqpoint{0.000000in}{-0.020833in}}%
\pgfusepath{stroke,fill}%
}%
\begin{pgfscope}%
\pgfsys@transformshift{4.835151in}{0.893003in}%
\pgfsys@useobject{currentmarker}{}%
\end{pgfscope}%
\end{pgfscope}%
\begin{pgfscope}%
\pgfsetbuttcap%
\pgfsetroundjoin%
\definecolor{currentfill}{rgb}{0.000000,0.000000,0.000000}%
\pgfsetfillcolor{currentfill}%
\pgfsetlinewidth{0.501875pt}%
\definecolor{currentstroke}{rgb}{0.000000,0.000000,0.000000}%
\pgfsetstrokecolor{currentstroke}%
\pgfsetdash{}{0pt}%
\pgfsys@defobject{currentmarker}{\pgfqpoint{0.000000in}{0.000000in}}{\pgfqpoint{0.000000in}{0.020833in}}{%
\pgfpathmoveto{\pgfqpoint{0.000000in}{0.000000in}}%
\pgfpathlineto{\pgfqpoint{0.000000in}{0.020833in}}%
\pgfusepath{stroke,fill}%
}%
\begin{pgfscope}%
\pgfsys@transformshift{4.870671in}{0.586309in}%
\pgfsys@useobject{currentmarker}{}%
\end{pgfscope}%
\end{pgfscope}%
\begin{pgfscope}%
\pgfsetbuttcap%
\pgfsetroundjoin%
\definecolor{currentfill}{rgb}{0.000000,0.000000,0.000000}%
\pgfsetfillcolor{currentfill}%
\pgfsetlinewidth{0.501875pt}%
\definecolor{currentstroke}{rgb}{0.000000,0.000000,0.000000}%
\pgfsetstrokecolor{currentstroke}%
\pgfsetdash{}{0pt}%
\pgfsys@defobject{currentmarker}{\pgfqpoint{0.000000in}{-0.020833in}}{\pgfqpoint{0.000000in}{0.000000in}}{%
\pgfpathmoveto{\pgfqpoint{0.000000in}{0.000000in}}%
\pgfpathlineto{\pgfqpoint{0.000000in}{-0.020833in}}%
\pgfusepath{stroke,fill}%
}%
\begin{pgfscope}%
\pgfsys@transformshift{4.870671in}{0.893003in}%
\pgfsys@useobject{currentmarker}{}%
\end{pgfscope}%
\end{pgfscope}%
\begin{pgfscope}%
\pgfsetbuttcap%
\pgfsetroundjoin%
\definecolor{currentfill}{rgb}{0.000000,0.000000,0.000000}%
\pgfsetfillcolor{currentfill}%
\pgfsetlinewidth{0.501875pt}%
\definecolor{currentstroke}{rgb}{0.000000,0.000000,0.000000}%
\pgfsetstrokecolor{currentstroke}%
\pgfsetdash{}{0pt}%
\pgfsys@defobject{currentmarker}{\pgfqpoint{0.000000in}{0.000000in}}{\pgfqpoint{0.000000in}{0.020833in}}{%
\pgfpathmoveto{\pgfqpoint{0.000000in}{0.000000in}}%
\pgfpathlineto{\pgfqpoint{0.000000in}{0.020833in}}%
\pgfusepath{stroke,fill}%
}%
\begin{pgfscope}%
\pgfsys@transformshift{4.941712in}{0.586309in}%
\pgfsys@useobject{currentmarker}{}%
\end{pgfscope}%
\end{pgfscope}%
\begin{pgfscope}%
\pgfsetbuttcap%
\pgfsetroundjoin%
\definecolor{currentfill}{rgb}{0.000000,0.000000,0.000000}%
\pgfsetfillcolor{currentfill}%
\pgfsetlinewidth{0.501875pt}%
\definecolor{currentstroke}{rgb}{0.000000,0.000000,0.000000}%
\pgfsetstrokecolor{currentstroke}%
\pgfsetdash{}{0pt}%
\pgfsys@defobject{currentmarker}{\pgfqpoint{0.000000in}{-0.020833in}}{\pgfqpoint{0.000000in}{0.000000in}}{%
\pgfpathmoveto{\pgfqpoint{0.000000in}{0.000000in}}%
\pgfpathlineto{\pgfqpoint{0.000000in}{-0.020833in}}%
\pgfusepath{stroke,fill}%
}%
\begin{pgfscope}%
\pgfsys@transformshift{4.941712in}{0.893003in}%
\pgfsys@useobject{currentmarker}{}%
\end{pgfscope}%
\end{pgfscope}%
\begin{pgfscope}%
\pgfsetbuttcap%
\pgfsetroundjoin%
\definecolor{currentfill}{rgb}{0.000000,0.000000,0.000000}%
\pgfsetfillcolor{currentfill}%
\pgfsetlinewidth{0.501875pt}%
\definecolor{currentstroke}{rgb}{0.000000,0.000000,0.000000}%
\pgfsetstrokecolor{currentstroke}%
\pgfsetdash{}{0pt}%
\pgfsys@defobject{currentmarker}{\pgfqpoint{0.000000in}{0.000000in}}{\pgfqpoint{0.000000in}{0.020833in}}{%
\pgfpathmoveto{\pgfqpoint{0.000000in}{0.000000in}}%
\pgfpathlineto{\pgfqpoint{0.000000in}{0.020833in}}%
\pgfusepath{stroke,fill}%
}%
\begin{pgfscope}%
\pgfsys@transformshift{4.977233in}{0.586309in}%
\pgfsys@useobject{currentmarker}{}%
\end{pgfscope}%
\end{pgfscope}%
\begin{pgfscope}%
\pgfsetbuttcap%
\pgfsetroundjoin%
\definecolor{currentfill}{rgb}{0.000000,0.000000,0.000000}%
\pgfsetfillcolor{currentfill}%
\pgfsetlinewidth{0.501875pt}%
\definecolor{currentstroke}{rgb}{0.000000,0.000000,0.000000}%
\pgfsetstrokecolor{currentstroke}%
\pgfsetdash{}{0pt}%
\pgfsys@defobject{currentmarker}{\pgfqpoint{0.000000in}{-0.020833in}}{\pgfqpoint{0.000000in}{0.000000in}}{%
\pgfpathmoveto{\pgfqpoint{0.000000in}{0.000000in}}%
\pgfpathlineto{\pgfqpoint{0.000000in}{-0.020833in}}%
\pgfusepath{stroke,fill}%
}%
\begin{pgfscope}%
\pgfsys@transformshift{4.977233in}{0.893003in}%
\pgfsys@useobject{currentmarker}{}%
\end{pgfscope}%
\end{pgfscope}%
\begin{pgfscope}%
\pgfsetbuttcap%
\pgfsetroundjoin%
\definecolor{currentfill}{rgb}{0.000000,0.000000,0.000000}%
\pgfsetfillcolor{currentfill}%
\pgfsetlinewidth{0.501875pt}%
\definecolor{currentstroke}{rgb}{0.000000,0.000000,0.000000}%
\pgfsetstrokecolor{currentstroke}%
\pgfsetdash{}{0pt}%
\pgfsys@defobject{currentmarker}{\pgfqpoint{0.000000in}{0.000000in}}{\pgfqpoint{0.000000in}{0.020833in}}{%
\pgfpathmoveto{\pgfqpoint{0.000000in}{0.000000in}}%
\pgfpathlineto{\pgfqpoint{0.000000in}{0.020833in}}%
\pgfusepath{stroke,fill}%
}%
\begin{pgfscope}%
\pgfsys@transformshift{5.012753in}{0.586309in}%
\pgfsys@useobject{currentmarker}{}%
\end{pgfscope}%
\end{pgfscope}%
\begin{pgfscope}%
\pgfsetbuttcap%
\pgfsetroundjoin%
\definecolor{currentfill}{rgb}{0.000000,0.000000,0.000000}%
\pgfsetfillcolor{currentfill}%
\pgfsetlinewidth{0.501875pt}%
\definecolor{currentstroke}{rgb}{0.000000,0.000000,0.000000}%
\pgfsetstrokecolor{currentstroke}%
\pgfsetdash{}{0pt}%
\pgfsys@defobject{currentmarker}{\pgfqpoint{0.000000in}{-0.020833in}}{\pgfqpoint{0.000000in}{0.000000in}}{%
\pgfpathmoveto{\pgfqpoint{0.000000in}{0.000000in}}%
\pgfpathlineto{\pgfqpoint{0.000000in}{-0.020833in}}%
\pgfusepath{stroke,fill}%
}%
\begin{pgfscope}%
\pgfsys@transformshift{5.012753in}{0.893003in}%
\pgfsys@useobject{currentmarker}{}%
\end{pgfscope}%
\end{pgfscope}%
\begin{pgfscope}%
\pgfsetbuttcap%
\pgfsetroundjoin%
\definecolor{currentfill}{rgb}{0.000000,0.000000,0.000000}%
\pgfsetfillcolor{currentfill}%
\pgfsetlinewidth{0.501875pt}%
\definecolor{currentstroke}{rgb}{0.000000,0.000000,0.000000}%
\pgfsetstrokecolor{currentstroke}%
\pgfsetdash{}{0pt}%
\pgfsys@defobject{currentmarker}{\pgfqpoint{0.000000in}{0.000000in}}{\pgfqpoint{0.000000in}{0.020833in}}{%
\pgfpathmoveto{\pgfqpoint{0.000000in}{0.000000in}}%
\pgfpathlineto{\pgfqpoint{0.000000in}{0.020833in}}%
\pgfusepath{stroke,fill}%
}%
\begin{pgfscope}%
\pgfsys@transformshift{5.048274in}{0.586309in}%
\pgfsys@useobject{currentmarker}{}%
\end{pgfscope}%
\end{pgfscope}%
\begin{pgfscope}%
\pgfsetbuttcap%
\pgfsetroundjoin%
\definecolor{currentfill}{rgb}{0.000000,0.000000,0.000000}%
\pgfsetfillcolor{currentfill}%
\pgfsetlinewidth{0.501875pt}%
\definecolor{currentstroke}{rgb}{0.000000,0.000000,0.000000}%
\pgfsetstrokecolor{currentstroke}%
\pgfsetdash{}{0pt}%
\pgfsys@defobject{currentmarker}{\pgfqpoint{0.000000in}{-0.020833in}}{\pgfqpoint{0.000000in}{0.000000in}}{%
\pgfpathmoveto{\pgfqpoint{0.000000in}{0.000000in}}%
\pgfpathlineto{\pgfqpoint{0.000000in}{-0.020833in}}%
\pgfusepath{stroke,fill}%
}%
\begin{pgfscope}%
\pgfsys@transformshift{5.048274in}{0.893003in}%
\pgfsys@useobject{currentmarker}{}%
\end{pgfscope}%
\end{pgfscope}%
\begin{pgfscope}%
\pgfsetbuttcap%
\pgfsetroundjoin%
\definecolor{currentfill}{rgb}{0.000000,0.000000,0.000000}%
\pgfsetfillcolor{currentfill}%
\pgfsetlinewidth{0.501875pt}%
\definecolor{currentstroke}{rgb}{0.000000,0.000000,0.000000}%
\pgfsetstrokecolor{currentstroke}%
\pgfsetdash{}{0pt}%
\pgfsys@defobject{currentmarker}{\pgfqpoint{0.000000in}{0.000000in}}{\pgfqpoint{0.000000in}{0.020833in}}{%
\pgfpathmoveto{\pgfqpoint{0.000000in}{0.000000in}}%
\pgfpathlineto{\pgfqpoint{0.000000in}{0.020833in}}%
\pgfusepath{stroke,fill}%
}%
\begin{pgfscope}%
\pgfsys@transformshift{5.083794in}{0.586309in}%
\pgfsys@useobject{currentmarker}{}%
\end{pgfscope}%
\end{pgfscope}%
\begin{pgfscope}%
\pgfsetbuttcap%
\pgfsetroundjoin%
\definecolor{currentfill}{rgb}{0.000000,0.000000,0.000000}%
\pgfsetfillcolor{currentfill}%
\pgfsetlinewidth{0.501875pt}%
\definecolor{currentstroke}{rgb}{0.000000,0.000000,0.000000}%
\pgfsetstrokecolor{currentstroke}%
\pgfsetdash{}{0pt}%
\pgfsys@defobject{currentmarker}{\pgfqpoint{0.000000in}{-0.020833in}}{\pgfqpoint{0.000000in}{0.000000in}}{%
\pgfpathmoveto{\pgfqpoint{0.000000in}{0.000000in}}%
\pgfpathlineto{\pgfqpoint{0.000000in}{-0.020833in}}%
\pgfusepath{stroke,fill}%
}%
\begin{pgfscope}%
\pgfsys@transformshift{5.083794in}{0.893003in}%
\pgfsys@useobject{currentmarker}{}%
\end{pgfscope}%
\end{pgfscope}%
\begin{pgfscope}%
\pgfsetbuttcap%
\pgfsetroundjoin%
\definecolor{currentfill}{rgb}{0.000000,0.000000,0.000000}%
\pgfsetfillcolor{currentfill}%
\pgfsetlinewidth{0.501875pt}%
\definecolor{currentstroke}{rgb}{0.000000,0.000000,0.000000}%
\pgfsetstrokecolor{currentstroke}%
\pgfsetdash{}{0pt}%
\pgfsys@defobject{currentmarker}{\pgfqpoint{0.000000in}{0.000000in}}{\pgfqpoint{0.000000in}{0.020833in}}{%
\pgfpathmoveto{\pgfqpoint{0.000000in}{0.000000in}}%
\pgfpathlineto{\pgfqpoint{0.000000in}{0.020833in}}%
\pgfusepath{stroke,fill}%
}%
\begin{pgfscope}%
\pgfsys@transformshift{5.119315in}{0.586309in}%
\pgfsys@useobject{currentmarker}{}%
\end{pgfscope}%
\end{pgfscope}%
\begin{pgfscope}%
\pgfsetbuttcap%
\pgfsetroundjoin%
\definecolor{currentfill}{rgb}{0.000000,0.000000,0.000000}%
\pgfsetfillcolor{currentfill}%
\pgfsetlinewidth{0.501875pt}%
\definecolor{currentstroke}{rgb}{0.000000,0.000000,0.000000}%
\pgfsetstrokecolor{currentstroke}%
\pgfsetdash{}{0pt}%
\pgfsys@defobject{currentmarker}{\pgfqpoint{0.000000in}{-0.020833in}}{\pgfqpoint{0.000000in}{0.000000in}}{%
\pgfpathmoveto{\pgfqpoint{0.000000in}{0.000000in}}%
\pgfpathlineto{\pgfqpoint{0.000000in}{-0.020833in}}%
\pgfusepath{stroke,fill}%
}%
\begin{pgfscope}%
\pgfsys@transformshift{5.119315in}{0.893003in}%
\pgfsys@useobject{currentmarker}{}%
\end{pgfscope}%
\end{pgfscope}%
\begin{pgfscope}%
\pgfsetbuttcap%
\pgfsetroundjoin%
\definecolor{currentfill}{rgb}{0.000000,0.000000,0.000000}%
\pgfsetfillcolor{currentfill}%
\pgfsetlinewidth{0.501875pt}%
\definecolor{currentstroke}{rgb}{0.000000,0.000000,0.000000}%
\pgfsetstrokecolor{currentstroke}%
\pgfsetdash{}{0pt}%
\pgfsys@defobject{currentmarker}{\pgfqpoint{0.000000in}{0.000000in}}{\pgfqpoint{0.000000in}{0.020833in}}{%
\pgfpathmoveto{\pgfqpoint{0.000000in}{0.000000in}}%
\pgfpathlineto{\pgfqpoint{0.000000in}{0.020833in}}%
\pgfusepath{stroke,fill}%
}%
\begin{pgfscope}%
\pgfsys@transformshift{5.154835in}{0.586309in}%
\pgfsys@useobject{currentmarker}{}%
\end{pgfscope}%
\end{pgfscope}%
\begin{pgfscope}%
\pgfsetbuttcap%
\pgfsetroundjoin%
\definecolor{currentfill}{rgb}{0.000000,0.000000,0.000000}%
\pgfsetfillcolor{currentfill}%
\pgfsetlinewidth{0.501875pt}%
\definecolor{currentstroke}{rgb}{0.000000,0.000000,0.000000}%
\pgfsetstrokecolor{currentstroke}%
\pgfsetdash{}{0pt}%
\pgfsys@defobject{currentmarker}{\pgfqpoint{0.000000in}{-0.020833in}}{\pgfqpoint{0.000000in}{0.000000in}}{%
\pgfpathmoveto{\pgfqpoint{0.000000in}{0.000000in}}%
\pgfpathlineto{\pgfqpoint{0.000000in}{-0.020833in}}%
\pgfusepath{stroke,fill}%
}%
\begin{pgfscope}%
\pgfsys@transformshift{5.154835in}{0.893003in}%
\pgfsys@useobject{currentmarker}{}%
\end{pgfscope}%
\end{pgfscope}%
\begin{pgfscope}%
\pgfsetbuttcap%
\pgfsetroundjoin%
\definecolor{currentfill}{rgb}{0.000000,0.000000,0.000000}%
\pgfsetfillcolor{currentfill}%
\pgfsetlinewidth{0.501875pt}%
\definecolor{currentstroke}{rgb}{0.000000,0.000000,0.000000}%
\pgfsetstrokecolor{currentstroke}%
\pgfsetdash{}{0pt}%
\pgfsys@defobject{currentmarker}{\pgfqpoint{0.000000in}{0.000000in}}{\pgfqpoint{0.000000in}{0.020833in}}{%
\pgfpathmoveto{\pgfqpoint{0.000000in}{0.000000in}}%
\pgfpathlineto{\pgfqpoint{0.000000in}{0.020833in}}%
\pgfusepath{stroke,fill}%
}%
\begin{pgfscope}%
\pgfsys@transformshift{5.190356in}{0.586309in}%
\pgfsys@useobject{currentmarker}{}%
\end{pgfscope}%
\end{pgfscope}%
\begin{pgfscope}%
\pgfsetbuttcap%
\pgfsetroundjoin%
\definecolor{currentfill}{rgb}{0.000000,0.000000,0.000000}%
\pgfsetfillcolor{currentfill}%
\pgfsetlinewidth{0.501875pt}%
\definecolor{currentstroke}{rgb}{0.000000,0.000000,0.000000}%
\pgfsetstrokecolor{currentstroke}%
\pgfsetdash{}{0pt}%
\pgfsys@defobject{currentmarker}{\pgfqpoint{0.000000in}{-0.020833in}}{\pgfqpoint{0.000000in}{0.000000in}}{%
\pgfpathmoveto{\pgfqpoint{0.000000in}{0.000000in}}%
\pgfpathlineto{\pgfqpoint{0.000000in}{-0.020833in}}%
\pgfusepath{stroke,fill}%
}%
\begin{pgfscope}%
\pgfsys@transformshift{5.190356in}{0.893003in}%
\pgfsys@useobject{currentmarker}{}%
\end{pgfscope}%
\end{pgfscope}%
\begin{pgfscope}%
\pgfsetbuttcap%
\pgfsetroundjoin%
\definecolor{currentfill}{rgb}{0.000000,0.000000,0.000000}%
\pgfsetfillcolor{currentfill}%
\pgfsetlinewidth{0.501875pt}%
\definecolor{currentstroke}{rgb}{0.000000,0.000000,0.000000}%
\pgfsetstrokecolor{currentstroke}%
\pgfsetdash{}{0pt}%
\pgfsys@defobject{currentmarker}{\pgfqpoint{0.000000in}{0.000000in}}{\pgfqpoint{0.000000in}{0.020833in}}{%
\pgfpathmoveto{\pgfqpoint{0.000000in}{0.000000in}}%
\pgfpathlineto{\pgfqpoint{0.000000in}{0.020833in}}%
\pgfusepath{stroke,fill}%
}%
\begin{pgfscope}%
\pgfsys@transformshift{5.225876in}{0.586309in}%
\pgfsys@useobject{currentmarker}{}%
\end{pgfscope}%
\end{pgfscope}%
\begin{pgfscope}%
\pgfsetbuttcap%
\pgfsetroundjoin%
\definecolor{currentfill}{rgb}{0.000000,0.000000,0.000000}%
\pgfsetfillcolor{currentfill}%
\pgfsetlinewidth{0.501875pt}%
\definecolor{currentstroke}{rgb}{0.000000,0.000000,0.000000}%
\pgfsetstrokecolor{currentstroke}%
\pgfsetdash{}{0pt}%
\pgfsys@defobject{currentmarker}{\pgfqpoint{0.000000in}{-0.020833in}}{\pgfqpoint{0.000000in}{0.000000in}}{%
\pgfpathmoveto{\pgfqpoint{0.000000in}{0.000000in}}%
\pgfpathlineto{\pgfqpoint{0.000000in}{-0.020833in}}%
\pgfusepath{stroke,fill}%
}%
\begin{pgfscope}%
\pgfsys@transformshift{5.225876in}{0.893003in}%
\pgfsys@useobject{currentmarker}{}%
\end{pgfscope}%
\end{pgfscope}%
\begin{pgfscope}%
\pgfsetbuttcap%
\pgfsetroundjoin%
\definecolor{currentfill}{rgb}{0.000000,0.000000,0.000000}%
\pgfsetfillcolor{currentfill}%
\pgfsetlinewidth{0.501875pt}%
\definecolor{currentstroke}{rgb}{0.000000,0.000000,0.000000}%
\pgfsetstrokecolor{currentstroke}%
\pgfsetdash{}{0pt}%
\pgfsys@defobject{currentmarker}{\pgfqpoint{0.000000in}{0.000000in}}{\pgfqpoint{0.000000in}{0.020833in}}{%
\pgfpathmoveto{\pgfqpoint{0.000000in}{0.000000in}}%
\pgfpathlineto{\pgfqpoint{0.000000in}{0.020833in}}%
\pgfusepath{stroke,fill}%
}%
\begin{pgfscope}%
\pgfsys@transformshift{5.261397in}{0.586309in}%
\pgfsys@useobject{currentmarker}{}%
\end{pgfscope}%
\end{pgfscope}%
\begin{pgfscope}%
\pgfsetbuttcap%
\pgfsetroundjoin%
\definecolor{currentfill}{rgb}{0.000000,0.000000,0.000000}%
\pgfsetfillcolor{currentfill}%
\pgfsetlinewidth{0.501875pt}%
\definecolor{currentstroke}{rgb}{0.000000,0.000000,0.000000}%
\pgfsetstrokecolor{currentstroke}%
\pgfsetdash{}{0pt}%
\pgfsys@defobject{currentmarker}{\pgfqpoint{0.000000in}{-0.020833in}}{\pgfqpoint{0.000000in}{0.000000in}}{%
\pgfpathmoveto{\pgfqpoint{0.000000in}{0.000000in}}%
\pgfpathlineto{\pgfqpoint{0.000000in}{-0.020833in}}%
\pgfusepath{stroke,fill}%
}%
\begin{pgfscope}%
\pgfsys@transformshift{5.261397in}{0.893003in}%
\pgfsys@useobject{currentmarker}{}%
\end{pgfscope}%
\end{pgfscope}%
\begin{pgfscope}%
\pgfsetbuttcap%
\pgfsetroundjoin%
\definecolor{currentfill}{rgb}{0.000000,0.000000,0.000000}%
\pgfsetfillcolor{currentfill}%
\pgfsetlinewidth{0.501875pt}%
\definecolor{currentstroke}{rgb}{0.000000,0.000000,0.000000}%
\pgfsetstrokecolor{currentstroke}%
\pgfsetdash{}{0pt}%
\pgfsys@defobject{currentmarker}{\pgfqpoint{0.000000in}{0.000000in}}{\pgfqpoint{0.000000in}{0.020833in}}{%
\pgfpathmoveto{\pgfqpoint{0.000000in}{0.000000in}}%
\pgfpathlineto{\pgfqpoint{0.000000in}{0.020833in}}%
\pgfusepath{stroke,fill}%
}%
\begin{pgfscope}%
\pgfsys@transformshift{5.296917in}{0.586309in}%
\pgfsys@useobject{currentmarker}{}%
\end{pgfscope}%
\end{pgfscope}%
\begin{pgfscope}%
\pgfsetbuttcap%
\pgfsetroundjoin%
\definecolor{currentfill}{rgb}{0.000000,0.000000,0.000000}%
\pgfsetfillcolor{currentfill}%
\pgfsetlinewidth{0.501875pt}%
\definecolor{currentstroke}{rgb}{0.000000,0.000000,0.000000}%
\pgfsetstrokecolor{currentstroke}%
\pgfsetdash{}{0pt}%
\pgfsys@defobject{currentmarker}{\pgfqpoint{0.000000in}{-0.020833in}}{\pgfqpoint{0.000000in}{0.000000in}}{%
\pgfpathmoveto{\pgfqpoint{0.000000in}{0.000000in}}%
\pgfpathlineto{\pgfqpoint{0.000000in}{-0.020833in}}%
\pgfusepath{stroke,fill}%
}%
\begin{pgfscope}%
\pgfsys@transformshift{5.296917in}{0.893003in}%
\pgfsys@useobject{currentmarker}{}%
\end{pgfscope}%
\end{pgfscope}%
\begin{pgfscope}%
\pgfsetbuttcap%
\pgfsetroundjoin%
\definecolor{currentfill}{rgb}{0.000000,0.000000,0.000000}%
\pgfsetfillcolor{currentfill}%
\pgfsetlinewidth{0.501875pt}%
\definecolor{currentstroke}{rgb}{0.000000,0.000000,0.000000}%
\pgfsetstrokecolor{currentstroke}%
\pgfsetdash{}{0pt}%
\pgfsys@defobject{currentmarker}{\pgfqpoint{0.000000in}{0.000000in}}{\pgfqpoint{0.000000in}{0.020833in}}{%
\pgfpathmoveto{\pgfqpoint{0.000000in}{0.000000in}}%
\pgfpathlineto{\pgfqpoint{0.000000in}{0.020833in}}%
\pgfusepath{stroke,fill}%
}%
\begin{pgfscope}%
\pgfsys@transformshift{5.367959in}{0.586309in}%
\pgfsys@useobject{currentmarker}{}%
\end{pgfscope}%
\end{pgfscope}%
\begin{pgfscope}%
\pgfsetbuttcap%
\pgfsetroundjoin%
\definecolor{currentfill}{rgb}{0.000000,0.000000,0.000000}%
\pgfsetfillcolor{currentfill}%
\pgfsetlinewidth{0.501875pt}%
\definecolor{currentstroke}{rgb}{0.000000,0.000000,0.000000}%
\pgfsetstrokecolor{currentstroke}%
\pgfsetdash{}{0pt}%
\pgfsys@defobject{currentmarker}{\pgfqpoint{0.000000in}{-0.020833in}}{\pgfqpoint{0.000000in}{0.000000in}}{%
\pgfpathmoveto{\pgfqpoint{0.000000in}{0.000000in}}%
\pgfpathlineto{\pgfqpoint{0.000000in}{-0.020833in}}%
\pgfusepath{stroke,fill}%
}%
\begin{pgfscope}%
\pgfsys@transformshift{5.367959in}{0.893003in}%
\pgfsys@useobject{currentmarker}{}%
\end{pgfscope}%
\end{pgfscope}%
\begin{pgfscope}%
\pgfsetbuttcap%
\pgfsetroundjoin%
\definecolor{currentfill}{rgb}{0.000000,0.000000,0.000000}%
\pgfsetfillcolor{currentfill}%
\pgfsetlinewidth{0.501875pt}%
\definecolor{currentstroke}{rgb}{0.000000,0.000000,0.000000}%
\pgfsetstrokecolor{currentstroke}%
\pgfsetdash{}{0pt}%
\pgfsys@defobject{currentmarker}{\pgfqpoint{0.000000in}{0.000000in}}{\pgfqpoint{0.000000in}{0.020833in}}{%
\pgfpathmoveto{\pgfqpoint{0.000000in}{0.000000in}}%
\pgfpathlineto{\pgfqpoint{0.000000in}{0.020833in}}%
\pgfusepath{stroke,fill}%
}%
\begin{pgfscope}%
\pgfsys@transformshift{5.403479in}{0.586309in}%
\pgfsys@useobject{currentmarker}{}%
\end{pgfscope}%
\end{pgfscope}%
\begin{pgfscope}%
\pgfsetbuttcap%
\pgfsetroundjoin%
\definecolor{currentfill}{rgb}{0.000000,0.000000,0.000000}%
\pgfsetfillcolor{currentfill}%
\pgfsetlinewidth{0.501875pt}%
\definecolor{currentstroke}{rgb}{0.000000,0.000000,0.000000}%
\pgfsetstrokecolor{currentstroke}%
\pgfsetdash{}{0pt}%
\pgfsys@defobject{currentmarker}{\pgfqpoint{0.000000in}{-0.020833in}}{\pgfqpoint{0.000000in}{0.000000in}}{%
\pgfpathmoveto{\pgfqpoint{0.000000in}{0.000000in}}%
\pgfpathlineto{\pgfqpoint{0.000000in}{-0.020833in}}%
\pgfusepath{stroke,fill}%
}%
\begin{pgfscope}%
\pgfsys@transformshift{5.403479in}{0.893003in}%
\pgfsys@useobject{currentmarker}{}%
\end{pgfscope}%
\end{pgfscope}%
\begin{pgfscope}%
\pgfsetbuttcap%
\pgfsetroundjoin%
\definecolor{currentfill}{rgb}{0.000000,0.000000,0.000000}%
\pgfsetfillcolor{currentfill}%
\pgfsetlinewidth{0.501875pt}%
\definecolor{currentstroke}{rgb}{0.000000,0.000000,0.000000}%
\pgfsetstrokecolor{currentstroke}%
\pgfsetdash{}{0pt}%
\pgfsys@defobject{currentmarker}{\pgfqpoint{0.000000in}{0.000000in}}{\pgfqpoint{0.000000in}{0.020833in}}{%
\pgfpathmoveto{\pgfqpoint{0.000000in}{0.000000in}}%
\pgfpathlineto{\pgfqpoint{0.000000in}{0.020833in}}%
\pgfusepath{stroke,fill}%
}%
\begin{pgfscope}%
\pgfsys@transformshift{5.439000in}{0.586309in}%
\pgfsys@useobject{currentmarker}{}%
\end{pgfscope}%
\end{pgfscope}%
\begin{pgfscope}%
\pgfsetbuttcap%
\pgfsetroundjoin%
\definecolor{currentfill}{rgb}{0.000000,0.000000,0.000000}%
\pgfsetfillcolor{currentfill}%
\pgfsetlinewidth{0.501875pt}%
\definecolor{currentstroke}{rgb}{0.000000,0.000000,0.000000}%
\pgfsetstrokecolor{currentstroke}%
\pgfsetdash{}{0pt}%
\pgfsys@defobject{currentmarker}{\pgfqpoint{0.000000in}{-0.020833in}}{\pgfqpoint{0.000000in}{0.000000in}}{%
\pgfpathmoveto{\pgfqpoint{0.000000in}{0.000000in}}%
\pgfpathlineto{\pgfqpoint{0.000000in}{-0.020833in}}%
\pgfusepath{stroke,fill}%
}%
\begin{pgfscope}%
\pgfsys@transformshift{5.439000in}{0.893003in}%
\pgfsys@useobject{currentmarker}{}%
\end{pgfscope}%
\end{pgfscope}%
\begin{pgfscope}%
\pgfsetbuttcap%
\pgfsetroundjoin%
\definecolor{currentfill}{rgb}{0.000000,0.000000,0.000000}%
\pgfsetfillcolor{currentfill}%
\pgfsetlinewidth{0.501875pt}%
\definecolor{currentstroke}{rgb}{0.000000,0.000000,0.000000}%
\pgfsetstrokecolor{currentstroke}%
\pgfsetdash{}{0pt}%
\pgfsys@defobject{currentmarker}{\pgfqpoint{0.000000in}{0.000000in}}{\pgfqpoint{0.000000in}{0.020833in}}{%
\pgfpathmoveto{\pgfqpoint{0.000000in}{0.000000in}}%
\pgfpathlineto{\pgfqpoint{0.000000in}{0.020833in}}%
\pgfusepath{stroke,fill}%
}%
\begin{pgfscope}%
\pgfsys@transformshift{5.474520in}{0.586309in}%
\pgfsys@useobject{currentmarker}{}%
\end{pgfscope}%
\end{pgfscope}%
\begin{pgfscope}%
\pgfsetbuttcap%
\pgfsetroundjoin%
\definecolor{currentfill}{rgb}{0.000000,0.000000,0.000000}%
\pgfsetfillcolor{currentfill}%
\pgfsetlinewidth{0.501875pt}%
\definecolor{currentstroke}{rgb}{0.000000,0.000000,0.000000}%
\pgfsetstrokecolor{currentstroke}%
\pgfsetdash{}{0pt}%
\pgfsys@defobject{currentmarker}{\pgfqpoint{0.000000in}{-0.020833in}}{\pgfqpoint{0.000000in}{0.000000in}}{%
\pgfpathmoveto{\pgfqpoint{0.000000in}{0.000000in}}%
\pgfpathlineto{\pgfqpoint{0.000000in}{-0.020833in}}%
\pgfusepath{stroke,fill}%
}%
\begin{pgfscope}%
\pgfsys@transformshift{5.474520in}{0.893003in}%
\pgfsys@useobject{currentmarker}{}%
\end{pgfscope}%
\end{pgfscope}%
\begin{pgfscope}%
\pgfsetbuttcap%
\pgfsetroundjoin%
\definecolor{currentfill}{rgb}{0.000000,0.000000,0.000000}%
\pgfsetfillcolor{currentfill}%
\pgfsetlinewidth{0.501875pt}%
\definecolor{currentstroke}{rgb}{0.000000,0.000000,0.000000}%
\pgfsetstrokecolor{currentstroke}%
\pgfsetdash{}{0pt}%
\pgfsys@defobject{currentmarker}{\pgfqpoint{0.000000in}{0.000000in}}{\pgfqpoint{0.000000in}{0.020833in}}{%
\pgfpathmoveto{\pgfqpoint{0.000000in}{0.000000in}}%
\pgfpathlineto{\pgfqpoint{0.000000in}{0.020833in}}%
\pgfusepath{stroke,fill}%
}%
\begin{pgfscope}%
\pgfsys@transformshift{5.510041in}{0.586309in}%
\pgfsys@useobject{currentmarker}{}%
\end{pgfscope}%
\end{pgfscope}%
\begin{pgfscope}%
\pgfsetbuttcap%
\pgfsetroundjoin%
\definecolor{currentfill}{rgb}{0.000000,0.000000,0.000000}%
\pgfsetfillcolor{currentfill}%
\pgfsetlinewidth{0.501875pt}%
\definecolor{currentstroke}{rgb}{0.000000,0.000000,0.000000}%
\pgfsetstrokecolor{currentstroke}%
\pgfsetdash{}{0pt}%
\pgfsys@defobject{currentmarker}{\pgfqpoint{0.000000in}{-0.020833in}}{\pgfqpoint{0.000000in}{0.000000in}}{%
\pgfpathmoveto{\pgfqpoint{0.000000in}{0.000000in}}%
\pgfpathlineto{\pgfqpoint{0.000000in}{-0.020833in}}%
\pgfusepath{stroke,fill}%
}%
\begin{pgfscope}%
\pgfsys@transformshift{5.510041in}{0.893003in}%
\pgfsys@useobject{currentmarker}{}%
\end{pgfscope}%
\end{pgfscope}%
\begin{pgfscope}%
\pgfsetbuttcap%
\pgfsetroundjoin%
\definecolor{currentfill}{rgb}{0.000000,0.000000,0.000000}%
\pgfsetfillcolor{currentfill}%
\pgfsetlinewidth{0.501875pt}%
\definecolor{currentstroke}{rgb}{0.000000,0.000000,0.000000}%
\pgfsetstrokecolor{currentstroke}%
\pgfsetdash{}{0pt}%
\pgfsys@defobject{currentmarker}{\pgfqpoint{0.000000in}{0.000000in}}{\pgfqpoint{0.000000in}{0.020833in}}{%
\pgfpathmoveto{\pgfqpoint{0.000000in}{0.000000in}}%
\pgfpathlineto{\pgfqpoint{0.000000in}{0.020833in}}%
\pgfusepath{stroke,fill}%
}%
\begin{pgfscope}%
\pgfsys@transformshift{5.545561in}{0.586309in}%
\pgfsys@useobject{currentmarker}{}%
\end{pgfscope}%
\end{pgfscope}%
\begin{pgfscope}%
\pgfsetbuttcap%
\pgfsetroundjoin%
\definecolor{currentfill}{rgb}{0.000000,0.000000,0.000000}%
\pgfsetfillcolor{currentfill}%
\pgfsetlinewidth{0.501875pt}%
\definecolor{currentstroke}{rgb}{0.000000,0.000000,0.000000}%
\pgfsetstrokecolor{currentstroke}%
\pgfsetdash{}{0pt}%
\pgfsys@defobject{currentmarker}{\pgfqpoint{0.000000in}{-0.020833in}}{\pgfqpoint{0.000000in}{0.000000in}}{%
\pgfpathmoveto{\pgfqpoint{0.000000in}{0.000000in}}%
\pgfpathlineto{\pgfqpoint{0.000000in}{-0.020833in}}%
\pgfusepath{stroke,fill}%
}%
\begin{pgfscope}%
\pgfsys@transformshift{5.545561in}{0.893003in}%
\pgfsys@useobject{currentmarker}{}%
\end{pgfscope}%
\end{pgfscope}%
\begin{pgfscope}%
\pgfsetbuttcap%
\pgfsetroundjoin%
\definecolor{currentfill}{rgb}{0.000000,0.000000,0.000000}%
\pgfsetfillcolor{currentfill}%
\pgfsetlinewidth{0.501875pt}%
\definecolor{currentstroke}{rgb}{0.000000,0.000000,0.000000}%
\pgfsetstrokecolor{currentstroke}%
\pgfsetdash{}{0pt}%
\pgfsys@defobject{currentmarker}{\pgfqpoint{0.000000in}{0.000000in}}{\pgfqpoint{0.000000in}{0.020833in}}{%
\pgfpathmoveto{\pgfqpoint{0.000000in}{0.000000in}}%
\pgfpathlineto{\pgfqpoint{0.000000in}{0.020833in}}%
\pgfusepath{stroke,fill}%
}%
\begin{pgfscope}%
\pgfsys@transformshift{5.581082in}{0.586309in}%
\pgfsys@useobject{currentmarker}{}%
\end{pgfscope}%
\end{pgfscope}%
\begin{pgfscope}%
\pgfsetbuttcap%
\pgfsetroundjoin%
\definecolor{currentfill}{rgb}{0.000000,0.000000,0.000000}%
\pgfsetfillcolor{currentfill}%
\pgfsetlinewidth{0.501875pt}%
\definecolor{currentstroke}{rgb}{0.000000,0.000000,0.000000}%
\pgfsetstrokecolor{currentstroke}%
\pgfsetdash{}{0pt}%
\pgfsys@defobject{currentmarker}{\pgfqpoint{0.000000in}{-0.020833in}}{\pgfqpoint{0.000000in}{0.000000in}}{%
\pgfpathmoveto{\pgfqpoint{0.000000in}{0.000000in}}%
\pgfpathlineto{\pgfqpoint{0.000000in}{-0.020833in}}%
\pgfusepath{stroke,fill}%
}%
\begin{pgfscope}%
\pgfsys@transformshift{5.581082in}{0.893003in}%
\pgfsys@useobject{currentmarker}{}%
\end{pgfscope}%
\end{pgfscope}%
\begin{pgfscope}%
\pgfsetbuttcap%
\pgfsetroundjoin%
\definecolor{currentfill}{rgb}{0.000000,0.000000,0.000000}%
\pgfsetfillcolor{currentfill}%
\pgfsetlinewidth{0.501875pt}%
\definecolor{currentstroke}{rgb}{0.000000,0.000000,0.000000}%
\pgfsetstrokecolor{currentstroke}%
\pgfsetdash{}{0pt}%
\pgfsys@defobject{currentmarker}{\pgfqpoint{0.000000in}{0.000000in}}{\pgfqpoint{0.000000in}{0.020833in}}{%
\pgfpathmoveto{\pgfqpoint{0.000000in}{0.000000in}}%
\pgfpathlineto{\pgfqpoint{0.000000in}{0.020833in}}%
\pgfusepath{stroke,fill}%
}%
\begin{pgfscope}%
\pgfsys@transformshift{5.616602in}{0.586309in}%
\pgfsys@useobject{currentmarker}{}%
\end{pgfscope}%
\end{pgfscope}%
\begin{pgfscope}%
\pgfsetbuttcap%
\pgfsetroundjoin%
\definecolor{currentfill}{rgb}{0.000000,0.000000,0.000000}%
\pgfsetfillcolor{currentfill}%
\pgfsetlinewidth{0.501875pt}%
\definecolor{currentstroke}{rgb}{0.000000,0.000000,0.000000}%
\pgfsetstrokecolor{currentstroke}%
\pgfsetdash{}{0pt}%
\pgfsys@defobject{currentmarker}{\pgfqpoint{0.000000in}{-0.020833in}}{\pgfqpoint{0.000000in}{0.000000in}}{%
\pgfpathmoveto{\pgfqpoint{0.000000in}{0.000000in}}%
\pgfpathlineto{\pgfqpoint{0.000000in}{-0.020833in}}%
\pgfusepath{stroke,fill}%
}%
\begin{pgfscope}%
\pgfsys@transformshift{5.616602in}{0.893003in}%
\pgfsys@useobject{currentmarker}{}%
\end{pgfscope}%
\end{pgfscope}%
\begin{pgfscope}%
\pgfsetbuttcap%
\pgfsetroundjoin%
\definecolor{currentfill}{rgb}{0.000000,0.000000,0.000000}%
\pgfsetfillcolor{currentfill}%
\pgfsetlinewidth{0.501875pt}%
\definecolor{currentstroke}{rgb}{0.000000,0.000000,0.000000}%
\pgfsetstrokecolor{currentstroke}%
\pgfsetdash{}{0pt}%
\pgfsys@defobject{currentmarker}{\pgfqpoint{0.000000in}{0.000000in}}{\pgfqpoint{0.000000in}{0.020833in}}{%
\pgfpathmoveto{\pgfqpoint{0.000000in}{0.000000in}}%
\pgfpathlineto{\pgfqpoint{0.000000in}{0.020833in}}%
\pgfusepath{stroke,fill}%
}%
\begin{pgfscope}%
\pgfsys@transformshift{5.652123in}{0.586309in}%
\pgfsys@useobject{currentmarker}{}%
\end{pgfscope}%
\end{pgfscope}%
\begin{pgfscope}%
\pgfsetbuttcap%
\pgfsetroundjoin%
\definecolor{currentfill}{rgb}{0.000000,0.000000,0.000000}%
\pgfsetfillcolor{currentfill}%
\pgfsetlinewidth{0.501875pt}%
\definecolor{currentstroke}{rgb}{0.000000,0.000000,0.000000}%
\pgfsetstrokecolor{currentstroke}%
\pgfsetdash{}{0pt}%
\pgfsys@defobject{currentmarker}{\pgfqpoint{0.000000in}{-0.020833in}}{\pgfqpoint{0.000000in}{0.000000in}}{%
\pgfpathmoveto{\pgfqpoint{0.000000in}{0.000000in}}%
\pgfpathlineto{\pgfqpoint{0.000000in}{-0.020833in}}%
\pgfusepath{stroke,fill}%
}%
\begin{pgfscope}%
\pgfsys@transformshift{5.652123in}{0.893003in}%
\pgfsys@useobject{currentmarker}{}%
\end{pgfscope}%
\end{pgfscope}%
\begin{pgfscope}%
\pgfsetbuttcap%
\pgfsetroundjoin%
\definecolor{currentfill}{rgb}{0.000000,0.000000,0.000000}%
\pgfsetfillcolor{currentfill}%
\pgfsetlinewidth{0.501875pt}%
\definecolor{currentstroke}{rgb}{0.000000,0.000000,0.000000}%
\pgfsetstrokecolor{currentstroke}%
\pgfsetdash{}{0pt}%
\pgfsys@defobject{currentmarker}{\pgfqpoint{0.000000in}{0.000000in}}{\pgfqpoint{0.000000in}{0.020833in}}{%
\pgfpathmoveto{\pgfqpoint{0.000000in}{0.000000in}}%
\pgfpathlineto{\pgfqpoint{0.000000in}{0.020833in}}%
\pgfusepath{stroke,fill}%
}%
\begin{pgfscope}%
\pgfsys@transformshift{5.687643in}{0.586309in}%
\pgfsys@useobject{currentmarker}{}%
\end{pgfscope}%
\end{pgfscope}%
\begin{pgfscope}%
\pgfsetbuttcap%
\pgfsetroundjoin%
\definecolor{currentfill}{rgb}{0.000000,0.000000,0.000000}%
\pgfsetfillcolor{currentfill}%
\pgfsetlinewidth{0.501875pt}%
\definecolor{currentstroke}{rgb}{0.000000,0.000000,0.000000}%
\pgfsetstrokecolor{currentstroke}%
\pgfsetdash{}{0pt}%
\pgfsys@defobject{currentmarker}{\pgfqpoint{0.000000in}{-0.020833in}}{\pgfqpoint{0.000000in}{0.000000in}}{%
\pgfpathmoveto{\pgfqpoint{0.000000in}{0.000000in}}%
\pgfpathlineto{\pgfqpoint{0.000000in}{-0.020833in}}%
\pgfusepath{stroke,fill}%
}%
\begin{pgfscope}%
\pgfsys@transformshift{5.687643in}{0.893003in}%
\pgfsys@useobject{currentmarker}{}%
\end{pgfscope}%
\end{pgfscope}%
\begin{pgfscope}%
\pgfsetbuttcap%
\pgfsetroundjoin%
\definecolor{currentfill}{rgb}{0.000000,0.000000,0.000000}%
\pgfsetfillcolor{currentfill}%
\pgfsetlinewidth{0.501875pt}%
\definecolor{currentstroke}{rgb}{0.000000,0.000000,0.000000}%
\pgfsetstrokecolor{currentstroke}%
\pgfsetdash{}{0pt}%
\pgfsys@defobject{currentmarker}{\pgfqpoint{0.000000in}{0.000000in}}{\pgfqpoint{0.000000in}{0.020833in}}{%
\pgfpathmoveto{\pgfqpoint{0.000000in}{0.000000in}}%
\pgfpathlineto{\pgfqpoint{0.000000in}{0.020833in}}%
\pgfusepath{stroke,fill}%
}%
\begin{pgfscope}%
\pgfsys@transformshift{5.723164in}{0.586309in}%
\pgfsys@useobject{currentmarker}{}%
\end{pgfscope}%
\end{pgfscope}%
\begin{pgfscope}%
\pgfsetbuttcap%
\pgfsetroundjoin%
\definecolor{currentfill}{rgb}{0.000000,0.000000,0.000000}%
\pgfsetfillcolor{currentfill}%
\pgfsetlinewidth{0.501875pt}%
\definecolor{currentstroke}{rgb}{0.000000,0.000000,0.000000}%
\pgfsetstrokecolor{currentstroke}%
\pgfsetdash{}{0pt}%
\pgfsys@defobject{currentmarker}{\pgfqpoint{0.000000in}{-0.020833in}}{\pgfqpoint{0.000000in}{0.000000in}}{%
\pgfpathmoveto{\pgfqpoint{0.000000in}{0.000000in}}%
\pgfpathlineto{\pgfqpoint{0.000000in}{-0.020833in}}%
\pgfusepath{stroke,fill}%
}%
\begin{pgfscope}%
\pgfsys@transformshift{5.723164in}{0.893003in}%
\pgfsys@useobject{currentmarker}{}%
\end{pgfscope}%
\end{pgfscope}%
\begin{pgfscope}%
\pgfsetbuttcap%
\pgfsetroundjoin%
\definecolor{currentfill}{rgb}{0.000000,0.000000,0.000000}%
\pgfsetfillcolor{currentfill}%
\pgfsetlinewidth{0.501875pt}%
\definecolor{currentstroke}{rgb}{0.000000,0.000000,0.000000}%
\pgfsetstrokecolor{currentstroke}%
\pgfsetdash{}{0pt}%
\pgfsys@defobject{currentmarker}{\pgfqpoint{0.000000in}{0.000000in}}{\pgfqpoint{0.000000in}{0.020833in}}{%
\pgfpathmoveto{\pgfqpoint{0.000000in}{0.000000in}}%
\pgfpathlineto{\pgfqpoint{0.000000in}{0.020833in}}%
\pgfusepath{stroke,fill}%
}%
\begin{pgfscope}%
\pgfsys@transformshift{5.794205in}{0.586309in}%
\pgfsys@useobject{currentmarker}{}%
\end{pgfscope}%
\end{pgfscope}%
\begin{pgfscope}%
\pgfsetbuttcap%
\pgfsetroundjoin%
\definecolor{currentfill}{rgb}{0.000000,0.000000,0.000000}%
\pgfsetfillcolor{currentfill}%
\pgfsetlinewidth{0.501875pt}%
\definecolor{currentstroke}{rgb}{0.000000,0.000000,0.000000}%
\pgfsetstrokecolor{currentstroke}%
\pgfsetdash{}{0pt}%
\pgfsys@defobject{currentmarker}{\pgfqpoint{0.000000in}{-0.020833in}}{\pgfqpoint{0.000000in}{0.000000in}}{%
\pgfpathmoveto{\pgfqpoint{0.000000in}{0.000000in}}%
\pgfpathlineto{\pgfqpoint{0.000000in}{-0.020833in}}%
\pgfusepath{stroke,fill}%
}%
\begin{pgfscope}%
\pgfsys@transformshift{5.794205in}{0.893003in}%
\pgfsys@useobject{currentmarker}{}%
\end{pgfscope}%
\end{pgfscope}%
\begin{pgfscope}%
\pgfsetbuttcap%
\pgfsetroundjoin%
\definecolor{currentfill}{rgb}{0.000000,0.000000,0.000000}%
\pgfsetfillcolor{currentfill}%
\pgfsetlinewidth{0.501875pt}%
\definecolor{currentstroke}{rgb}{0.000000,0.000000,0.000000}%
\pgfsetstrokecolor{currentstroke}%
\pgfsetdash{}{0pt}%
\pgfsys@defobject{currentmarker}{\pgfqpoint{0.000000in}{0.000000in}}{\pgfqpoint{0.000000in}{0.020833in}}{%
\pgfpathmoveto{\pgfqpoint{0.000000in}{0.000000in}}%
\pgfpathlineto{\pgfqpoint{0.000000in}{0.020833in}}%
\pgfusepath{stroke,fill}%
}%
\begin{pgfscope}%
\pgfsys@transformshift{5.829725in}{0.586309in}%
\pgfsys@useobject{currentmarker}{}%
\end{pgfscope}%
\end{pgfscope}%
\begin{pgfscope}%
\pgfsetbuttcap%
\pgfsetroundjoin%
\definecolor{currentfill}{rgb}{0.000000,0.000000,0.000000}%
\pgfsetfillcolor{currentfill}%
\pgfsetlinewidth{0.501875pt}%
\definecolor{currentstroke}{rgb}{0.000000,0.000000,0.000000}%
\pgfsetstrokecolor{currentstroke}%
\pgfsetdash{}{0pt}%
\pgfsys@defobject{currentmarker}{\pgfqpoint{0.000000in}{-0.020833in}}{\pgfqpoint{0.000000in}{0.000000in}}{%
\pgfpathmoveto{\pgfqpoint{0.000000in}{0.000000in}}%
\pgfpathlineto{\pgfqpoint{0.000000in}{-0.020833in}}%
\pgfusepath{stroke,fill}%
}%
\begin{pgfscope}%
\pgfsys@transformshift{5.829725in}{0.893003in}%
\pgfsys@useobject{currentmarker}{}%
\end{pgfscope}%
\end{pgfscope}%
\begin{pgfscope}%
\pgfsetbuttcap%
\pgfsetroundjoin%
\definecolor{currentfill}{rgb}{0.000000,0.000000,0.000000}%
\pgfsetfillcolor{currentfill}%
\pgfsetlinewidth{0.501875pt}%
\definecolor{currentstroke}{rgb}{0.000000,0.000000,0.000000}%
\pgfsetstrokecolor{currentstroke}%
\pgfsetdash{}{0pt}%
\pgfsys@defobject{currentmarker}{\pgfqpoint{0.000000in}{0.000000in}}{\pgfqpoint{0.000000in}{0.020833in}}{%
\pgfpathmoveto{\pgfqpoint{0.000000in}{0.000000in}}%
\pgfpathlineto{\pgfqpoint{0.000000in}{0.020833in}}%
\pgfusepath{stroke,fill}%
}%
\begin{pgfscope}%
\pgfsys@transformshift{5.865246in}{0.586309in}%
\pgfsys@useobject{currentmarker}{}%
\end{pgfscope}%
\end{pgfscope}%
\begin{pgfscope}%
\pgfsetbuttcap%
\pgfsetroundjoin%
\definecolor{currentfill}{rgb}{0.000000,0.000000,0.000000}%
\pgfsetfillcolor{currentfill}%
\pgfsetlinewidth{0.501875pt}%
\definecolor{currentstroke}{rgb}{0.000000,0.000000,0.000000}%
\pgfsetstrokecolor{currentstroke}%
\pgfsetdash{}{0pt}%
\pgfsys@defobject{currentmarker}{\pgfqpoint{0.000000in}{-0.020833in}}{\pgfqpoint{0.000000in}{0.000000in}}{%
\pgfpathmoveto{\pgfqpoint{0.000000in}{0.000000in}}%
\pgfpathlineto{\pgfqpoint{0.000000in}{-0.020833in}}%
\pgfusepath{stroke,fill}%
}%
\begin{pgfscope}%
\pgfsys@transformshift{5.865246in}{0.893003in}%
\pgfsys@useobject{currentmarker}{}%
\end{pgfscope}%
\end{pgfscope}%
\begin{pgfscope}%
\pgfsetbuttcap%
\pgfsetroundjoin%
\definecolor{currentfill}{rgb}{0.000000,0.000000,0.000000}%
\pgfsetfillcolor{currentfill}%
\pgfsetlinewidth{0.501875pt}%
\definecolor{currentstroke}{rgb}{0.000000,0.000000,0.000000}%
\pgfsetstrokecolor{currentstroke}%
\pgfsetdash{}{0pt}%
\pgfsys@defobject{currentmarker}{\pgfqpoint{0.000000in}{0.000000in}}{\pgfqpoint{0.000000in}{0.020833in}}{%
\pgfpathmoveto{\pgfqpoint{0.000000in}{0.000000in}}%
\pgfpathlineto{\pgfqpoint{0.000000in}{0.020833in}}%
\pgfusepath{stroke,fill}%
}%
\begin{pgfscope}%
\pgfsys@transformshift{5.900766in}{0.586309in}%
\pgfsys@useobject{currentmarker}{}%
\end{pgfscope}%
\end{pgfscope}%
\begin{pgfscope}%
\pgfsetbuttcap%
\pgfsetroundjoin%
\definecolor{currentfill}{rgb}{0.000000,0.000000,0.000000}%
\pgfsetfillcolor{currentfill}%
\pgfsetlinewidth{0.501875pt}%
\definecolor{currentstroke}{rgb}{0.000000,0.000000,0.000000}%
\pgfsetstrokecolor{currentstroke}%
\pgfsetdash{}{0pt}%
\pgfsys@defobject{currentmarker}{\pgfqpoint{0.000000in}{-0.020833in}}{\pgfqpoint{0.000000in}{0.000000in}}{%
\pgfpathmoveto{\pgfqpoint{0.000000in}{0.000000in}}%
\pgfpathlineto{\pgfqpoint{0.000000in}{-0.020833in}}%
\pgfusepath{stroke,fill}%
}%
\begin{pgfscope}%
\pgfsys@transformshift{5.900766in}{0.893003in}%
\pgfsys@useobject{currentmarker}{}%
\end{pgfscope}%
\end{pgfscope}%
\begin{pgfscope}%
\pgfsetbuttcap%
\pgfsetroundjoin%
\definecolor{currentfill}{rgb}{0.000000,0.000000,0.000000}%
\pgfsetfillcolor{currentfill}%
\pgfsetlinewidth{0.501875pt}%
\definecolor{currentstroke}{rgb}{0.000000,0.000000,0.000000}%
\pgfsetstrokecolor{currentstroke}%
\pgfsetdash{}{0pt}%
\pgfsys@defobject{currentmarker}{\pgfqpoint{0.000000in}{0.000000in}}{\pgfqpoint{0.000000in}{0.020833in}}{%
\pgfpathmoveto{\pgfqpoint{0.000000in}{0.000000in}}%
\pgfpathlineto{\pgfqpoint{0.000000in}{0.020833in}}%
\pgfusepath{stroke,fill}%
}%
\begin{pgfscope}%
\pgfsys@transformshift{5.936287in}{0.586309in}%
\pgfsys@useobject{currentmarker}{}%
\end{pgfscope}%
\end{pgfscope}%
\begin{pgfscope}%
\pgfsetbuttcap%
\pgfsetroundjoin%
\definecolor{currentfill}{rgb}{0.000000,0.000000,0.000000}%
\pgfsetfillcolor{currentfill}%
\pgfsetlinewidth{0.501875pt}%
\definecolor{currentstroke}{rgb}{0.000000,0.000000,0.000000}%
\pgfsetstrokecolor{currentstroke}%
\pgfsetdash{}{0pt}%
\pgfsys@defobject{currentmarker}{\pgfqpoint{0.000000in}{-0.020833in}}{\pgfqpoint{0.000000in}{0.000000in}}{%
\pgfpathmoveto{\pgfqpoint{0.000000in}{0.000000in}}%
\pgfpathlineto{\pgfqpoint{0.000000in}{-0.020833in}}%
\pgfusepath{stroke,fill}%
}%
\begin{pgfscope}%
\pgfsys@transformshift{5.936287in}{0.893003in}%
\pgfsys@useobject{currentmarker}{}%
\end{pgfscope}%
\end{pgfscope}%
\begin{pgfscope}%
\pgfsetbuttcap%
\pgfsetroundjoin%
\definecolor{currentfill}{rgb}{0.000000,0.000000,0.000000}%
\pgfsetfillcolor{currentfill}%
\pgfsetlinewidth{0.501875pt}%
\definecolor{currentstroke}{rgb}{0.000000,0.000000,0.000000}%
\pgfsetstrokecolor{currentstroke}%
\pgfsetdash{}{0pt}%
\pgfsys@defobject{currentmarker}{\pgfqpoint{0.000000in}{0.000000in}}{\pgfqpoint{0.000000in}{0.020833in}}{%
\pgfpathmoveto{\pgfqpoint{0.000000in}{0.000000in}}%
\pgfpathlineto{\pgfqpoint{0.000000in}{0.020833in}}%
\pgfusepath{stroke,fill}%
}%
\begin{pgfscope}%
\pgfsys@transformshift{5.971807in}{0.586309in}%
\pgfsys@useobject{currentmarker}{}%
\end{pgfscope}%
\end{pgfscope}%
\begin{pgfscope}%
\pgfsetbuttcap%
\pgfsetroundjoin%
\definecolor{currentfill}{rgb}{0.000000,0.000000,0.000000}%
\pgfsetfillcolor{currentfill}%
\pgfsetlinewidth{0.501875pt}%
\definecolor{currentstroke}{rgb}{0.000000,0.000000,0.000000}%
\pgfsetstrokecolor{currentstroke}%
\pgfsetdash{}{0pt}%
\pgfsys@defobject{currentmarker}{\pgfqpoint{0.000000in}{-0.020833in}}{\pgfqpoint{0.000000in}{0.000000in}}{%
\pgfpathmoveto{\pgfqpoint{0.000000in}{0.000000in}}%
\pgfpathlineto{\pgfqpoint{0.000000in}{-0.020833in}}%
\pgfusepath{stroke,fill}%
}%
\begin{pgfscope}%
\pgfsys@transformshift{5.971807in}{0.893003in}%
\pgfsys@useobject{currentmarker}{}%
\end{pgfscope}%
\end{pgfscope}%
\begin{pgfscope}%
\pgfsetbuttcap%
\pgfsetroundjoin%
\definecolor{currentfill}{rgb}{0.000000,0.000000,0.000000}%
\pgfsetfillcolor{currentfill}%
\pgfsetlinewidth{0.501875pt}%
\definecolor{currentstroke}{rgb}{0.000000,0.000000,0.000000}%
\pgfsetstrokecolor{currentstroke}%
\pgfsetdash{}{0pt}%
\pgfsys@defobject{currentmarker}{\pgfqpoint{0.000000in}{0.000000in}}{\pgfqpoint{0.000000in}{0.020833in}}{%
\pgfpathmoveto{\pgfqpoint{0.000000in}{0.000000in}}%
\pgfpathlineto{\pgfqpoint{0.000000in}{0.020833in}}%
\pgfusepath{stroke,fill}%
}%
\begin{pgfscope}%
\pgfsys@transformshift{6.007328in}{0.586309in}%
\pgfsys@useobject{currentmarker}{}%
\end{pgfscope}%
\end{pgfscope}%
\begin{pgfscope}%
\pgfsetbuttcap%
\pgfsetroundjoin%
\definecolor{currentfill}{rgb}{0.000000,0.000000,0.000000}%
\pgfsetfillcolor{currentfill}%
\pgfsetlinewidth{0.501875pt}%
\definecolor{currentstroke}{rgb}{0.000000,0.000000,0.000000}%
\pgfsetstrokecolor{currentstroke}%
\pgfsetdash{}{0pt}%
\pgfsys@defobject{currentmarker}{\pgfqpoint{0.000000in}{-0.020833in}}{\pgfqpoint{0.000000in}{0.000000in}}{%
\pgfpathmoveto{\pgfqpoint{0.000000in}{0.000000in}}%
\pgfpathlineto{\pgfqpoint{0.000000in}{-0.020833in}}%
\pgfusepath{stroke,fill}%
}%
\begin{pgfscope}%
\pgfsys@transformshift{6.007328in}{0.893003in}%
\pgfsys@useobject{currentmarker}{}%
\end{pgfscope}%
\end{pgfscope}%
\begin{pgfscope}%
\pgfsetbuttcap%
\pgfsetroundjoin%
\definecolor{currentfill}{rgb}{0.000000,0.000000,0.000000}%
\pgfsetfillcolor{currentfill}%
\pgfsetlinewidth{0.501875pt}%
\definecolor{currentstroke}{rgb}{0.000000,0.000000,0.000000}%
\pgfsetstrokecolor{currentstroke}%
\pgfsetdash{}{0pt}%
\pgfsys@defobject{currentmarker}{\pgfqpoint{0.000000in}{0.000000in}}{\pgfqpoint{0.000000in}{0.020833in}}{%
\pgfpathmoveto{\pgfqpoint{0.000000in}{0.000000in}}%
\pgfpathlineto{\pgfqpoint{0.000000in}{0.020833in}}%
\pgfusepath{stroke,fill}%
}%
\begin{pgfscope}%
\pgfsys@transformshift{6.042849in}{0.586309in}%
\pgfsys@useobject{currentmarker}{}%
\end{pgfscope}%
\end{pgfscope}%
\begin{pgfscope}%
\pgfsetbuttcap%
\pgfsetroundjoin%
\definecolor{currentfill}{rgb}{0.000000,0.000000,0.000000}%
\pgfsetfillcolor{currentfill}%
\pgfsetlinewidth{0.501875pt}%
\definecolor{currentstroke}{rgb}{0.000000,0.000000,0.000000}%
\pgfsetstrokecolor{currentstroke}%
\pgfsetdash{}{0pt}%
\pgfsys@defobject{currentmarker}{\pgfqpoint{0.000000in}{-0.020833in}}{\pgfqpoint{0.000000in}{0.000000in}}{%
\pgfpathmoveto{\pgfqpoint{0.000000in}{0.000000in}}%
\pgfpathlineto{\pgfqpoint{0.000000in}{-0.020833in}}%
\pgfusepath{stroke,fill}%
}%
\begin{pgfscope}%
\pgfsys@transformshift{6.042849in}{0.893003in}%
\pgfsys@useobject{currentmarker}{}%
\end{pgfscope}%
\end{pgfscope}%
\begin{pgfscope}%
\pgfsetbuttcap%
\pgfsetroundjoin%
\definecolor{currentfill}{rgb}{0.000000,0.000000,0.000000}%
\pgfsetfillcolor{currentfill}%
\pgfsetlinewidth{0.501875pt}%
\definecolor{currentstroke}{rgb}{0.000000,0.000000,0.000000}%
\pgfsetstrokecolor{currentstroke}%
\pgfsetdash{}{0pt}%
\pgfsys@defobject{currentmarker}{\pgfqpoint{0.000000in}{0.000000in}}{\pgfqpoint{0.000000in}{0.020833in}}{%
\pgfpathmoveto{\pgfqpoint{0.000000in}{0.000000in}}%
\pgfpathlineto{\pgfqpoint{0.000000in}{0.020833in}}%
\pgfusepath{stroke,fill}%
}%
\begin{pgfscope}%
\pgfsys@transformshift{6.078369in}{0.586309in}%
\pgfsys@useobject{currentmarker}{}%
\end{pgfscope}%
\end{pgfscope}%
\begin{pgfscope}%
\pgfsetbuttcap%
\pgfsetroundjoin%
\definecolor{currentfill}{rgb}{0.000000,0.000000,0.000000}%
\pgfsetfillcolor{currentfill}%
\pgfsetlinewidth{0.501875pt}%
\definecolor{currentstroke}{rgb}{0.000000,0.000000,0.000000}%
\pgfsetstrokecolor{currentstroke}%
\pgfsetdash{}{0pt}%
\pgfsys@defobject{currentmarker}{\pgfqpoint{0.000000in}{-0.020833in}}{\pgfqpoint{0.000000in}{0.000000in}}{%
\pgfpathmoveto{\pgfqpoint{0.000000in}{0.000000in}}%
\pgfpathlineto{\pgfqpoint{0.000000in}{-0.020833in}}%
\pgfusepath{stroke,fill}%
}%
\begin{pgfscope}%
\pgfsys@transformshift{6.078369in}{0.893003in}%
\pgfsys@useobject{currentmarker}{}%
\end{pgfscope}%
\end{pgfscope}%
\begin{pgfscope}%
\pgfsetbuttcap%
\pgfsetroundjoin%
\definecolor{currentfill}{rgb}{0.000000,0.000000,0.000000}%
\pgfsetfillcolor{currentfill}%
\pgfsetlinewidth{0.501875pt}%
\definecolor{currentstroke}{rgb}{0.000000,0.000000,0.000000}%
\pgfsetstrokecolor{currentstroke}%
\pgfsetdash{}{0pt}%
\pgfsys@defobject{currentmarker}{\pgfqpoint{0.000000in}{0.000000in}}{\pgfqpoint{0.000000in}{0.020833in}}{%
\pgfpathmoveto{\pgfqpoint{0.000000in}{0.000000in}}%
\pgfpathlineto{\pgfqpoint{0.000000in}{0.020833in}}%
\pgfusepath{stroke,fill}%
}%
\begin{pgfscope}%
\pgfsys@transformshift{6.113890in}{0.586309in}%
\pgfsys@useobject{currentmarker}{}%
\end{pgfscope}%
\end{pgfscope}%
\begin{pgfscope}%
\pgfsetbuttcap%
\pgfsetroundjoin%
\definecolor{currentfill}{rgb}{0.000000,0.000000,0.000000}%
\pgfsetfillcolor{currentfill}%
\pgfsetlinewidth{0.501875pt}%
\definecolor{currentstroke}{rgb}{0.000000,0.000000,0.000000}%
\pgfsetstrokecolor{currentstroke}%
\pgfsetdash{}{0pt}%
\pgfsys@defobject{currentmarker}{\pgfqpoint{0.000000in}{-0.020833in}}{\pgfqpoint{0.000000in}{0.000000in}}{%
\pgfpathmoveto{\pgfqpoint{0.000000in}{0.000000in}}%
\pgfpathlineto{\pgfqpoint{0.000000in}{-0.020833in}}%
\pgfusepath{stroke,fill}%
}%
\begin{pgfscope}%
\pgfsys@transformshift{6.113890in}{0.893003in}%
\pgfsys@useobject{currentmarker}{}%
\end{pgfscope}%
\end{pgfscope}%
\begin{pgfscope}%
\pgfsetbuttcap%
\pgfsetroundjoin%
\definecolor{currentfill}{rgb}{0.000000,0.000000,0.000000}%
\pgfsetfillcolor{currentfill}%
\pgfsetlinewidth{0.501875pt}%
\definecolor{currentstroke}{rgb}{0.000000,0.000000,0.000000}%
\pgfsetstrokecolor{currentstroke}%
\pgfsetdash{}{0pt}%
\pgfsys@defobject{currentmarker}{\pgfqpoint{0.000000in}{0.000000in}}{\pgfqpoint{0.000000in}{0.020833in}}{%
\pgfpathmoveto{\pgfqpoint{0.000000in}{0.000000in}}%
\pgfpathlineto{\pgfqpoint{0.000000in}{0.020833in}}%
\pgfusepath{stroke,fill}%
}%
\begin{pgfscope}%
\pgfsys@transformshift{6.149410in}{0.586309in}%
\pgfsys@useobject{currentmarker}{}%
\end{pgfscope}%
\end{pgfscope}%
\begin{pgfscope}%
\pgfsetbuttcap%
\pgfsetroundjoin%
\definecolor{currentfill}{rgb}{0.000000,0.000000,0.000000}%
\pgfsetfillcolor{currentfill}%
\pgfsetlinewidth{0.501875pt}%
\definecolor{currentstroke}{rgb}{0.000000,0.000000,0.000000}%
\pgfsetstrokecolor{currentstroke}%
\pgfsetdash{}{0pt}%
\pgfsys@defobject{currentmarker}{\pgfqpoint{0.000000in}{-0.020833in}}{\pgfqpoint{0.000000in}{0.000000in}}{%
\pgfpathmoveto{\pgfqpoint{0.000000in}{0.000000in}}%
\pgfpathlineto{\pgfqpoint{0.000000in}{-0.020833in}}%
\pgfusepath{stroke,fill}%
}%
\begin{pgfscope}%
\pgfsys@transformshift{6.149410in}{0.893003in}%
\pgfsys@useobject{currentmarker}{}%
\end{pgfscope}%
\end{pgfscope}%
\begin{pgfscope}%
\pgfsetbuttcap%
\pgfsetroundjoin%
\definecolor{currentfill}{rgb}{0.000000,0.000000,0.000000}%
\pgfsetfillcolor{currentfill}%
\pgfsetlinewidth{0.501875pt}%
\definecolor{currentstroke}{rgb}{0.000000,0.000000,0.000000}%
\pgfsetstrokecolor{currentstroke}%
\pgfsetdash{}{0pt}%
\pgfsys@defobject{currentmarker}{\pgfqpoint{0.000000in}{0.000000in}}{\pgfqpoint{0.000000in}{0.020833in}}{%
\pgfpathmoveto{\pgfqpoint{0.000000in}{0.000000in}}%
\pgfpathlineto{\pgfqpoint{0.000000in}{0.020833in}}%
\pgfusepath{stroke,fill}%
}%
\begin{pgfscope}%
\pgfsys@transformshift{6.220451in}{0.586309in}%
\pgfsys@useobject{currentmarker}{}%
\end{pgfscope}%
\end{pgfscope}%
\begin{pgfscope}%
\pgfsetbuttcap%
\pgfsetroundjoin%
\definecolor{currentfill}{rgb}{0.000000,0.000000,0.000000}%
\pgfsetfillcolor{currentfill}%
\pgfsetlinewidth{0.501875pt}%
\definecolor{currentstroke}{rgb}{0.000000,0.000000,0.000000}%
\pgfsetstrokecolor{currentstroke}%
\pgfsetdash{}{0pt}%
\pgfsys@defobject{currentmarker}{\pgfqpoint{0.000000in}{-0.020833in}}{\pgfqpoint{0.000000in}{0.000000in}}{%
\pgfpathmoveto{\pgfqpoint{0.000000in}{0.000000in}}%
\pgfpathlineto{\pgfqpoint{0.000000in}{-0.020833in}}%
\pgfusepath{stroke,fill}%
}%
\begin{pgfscope}%
\pgfsys@transformshift{6.220451in}{0.893003in}%
\pgfsys@useobject{currentmarker}{}%
\end{pgfscope}%
\end{pgfscope}%
\begin{pgfscope}%
\pgfsetbuttcap%
\pgfsetroundjoin%
\definecolor{currentfill}{rgb}{0.000000,0.000000,0.000000}%
\pgfsetfillcolor{currentfill}%
\pgfsetlinewidth{0.501875pt}%
\definecolor{currentstroke}{rgb}{0.000000,0.000000,0.000000}%
\pgfsetstrokecolor{currentstroke}%
\pgfsetdash{}{0pt}%
\pgfsys@defobject{currentmarker}{\pgfqpoint{0.000000in}{0.000000in}}{\pgfqpoint{0.000000in}{0.020833in}}{%
\pgfpathmoveto{\pgfqpoint{0.000000in}{0.000000in}}%
\pgfpathlineto{\pgfqpoint{0.000000in}{0.020833in}}%
\pgfusepath{stroke,fill}%
}%
\begin{pgfscope}%
\pgfsys@transformshift{6.255972in}{0.586309in}%
\pgfsys@useobject{currentmarker}{}%
\end{pgfscope}%
\end{pgfscope}%
\begin{pgfscope}%
\pgfsetbuttcap%
\pgfsetroundjoin%
\definecolor{currentfill}{rgb}{0.000000,0.000000,0.000000}%
\pgfsetfillcolor{currentfill}%
\pgfsetlinewidth{0.501875pt}%
\definecolor{currentstroke}{rgb}{0.000000,0.000000,0.000000}%
\pgfsetstrokecolor{currentstroke}%
\pgfsetdash{}{0pt}%
\pgfsys@defobject{currentmarker}{\pgfqpoint{0.000000in}{-0.020833in}}{\pgfqpoint{0.000000in}{0.000000in}}{%
\pgfpathmoveto{\pgfqpoint{0.000000in}{0.000000in}}%
\pgfpathlineto{\pgfqpoint{0.000000in}{-0.020833in}}%
\pgfusepath{stroke,fill}%
}%
\begin{pgfscope}%
\pgfsys@transformshift{6.255972in}{0.893003in}%
\pgfsys@useobject{currentmarker}{}%
\end{pgfscope}%
\end{pgfscope}%
\begin{pgfscope}%
\definecolor{textcolor}{rgb}{0.000000,0.000000,0.000000}%
\pgfsetstrokecolor{textcolor}%
\pgfsetfillcolor{textcolor}%
\pgftext[x=3.374517in,y=0.148667in,,top]{\color{textcolor}\rmfamily\fontsize{8.000000}{9.600000}\selectfont Zeit}%
\end{pgfscope}%
\begin{pgfscope}%
\pgfsetbuttcap%
\pgfsetroundjoin%
\definecolor{currentfill}{rgb}{0.000000,0.000000,0.000000}%
\pgfsetfillcolor{currentfill}%
\pgfsetlinewidth{0.501875pt}%
\definecolor{currentstroke}{rgb}{0.000000,0.000000,0.000000}%
\pgfsetstrokecolor{currentstroke}%
\pgfsetdash{}{0pt}%
\pgfsys@defobject{currentmarker}{\pgfqpoint{0.000000in}{0.000000in}}{\pgfqpoint{0.041667in}{0.000000in}}{%
\pgfpathmoveto{\pgfqpoint{0.000000in}{0.000000in}}%
\pgfpathlineto{\pgfqpoint{0.041667in}{0.000000in}}%
\pgfusepath{stroke,fill}%
}%
\begin{pgfscope}%
\pgfsys@transformshift{0.481681in}{0.603542in}%
\pgfsys@useobject{currentmarker}{}%
\end{pgfscope}%
\end{pgfscope}%
\begin{pgfscope}%
\pgfsetbuttcap%
\pgfsetroundjoin%
\definecolor{currentfill}{rgb}{0.000000,0.000000,0.000000}%
\pgfsetfillcolor{currentfill}%
\pgfsetlinewidth{0.501875pt}%
\definecolor{currentstroke}{rgb}{0.000000,0.000000,0.000000}%
\pgfsetstrokecolor{currentstroke}%
\pgfsetdash{}{0pt}%
\pgfsys@defobject{currentmarker}{\pgfqpoint{-0.041667in}{0.000000in}}{\pgfqpoint{-0.000000in}{0.000000in}}{%
\pgfpathmoveto{\pgfqpoint{-0.000000in}{0.000000in}}%
\pgfpathlineto{\pgfqpoint{-0.041667in}{0.000000in}}%
\pgfusepath{stroke,fill}%
}%
\begin{pgfscope}%
\pgfsys@transformshift{6.267353in}{0.603542in}%
\pgfsys@useobject{currentmarker}{}%
\end{pgfscope}%
\end{pgfscope}%
\begin{pgfscope}%
\definecolor{textcolor}{rgb}{0.000000,0.000000,0.000000}%
\pgfsetstrokecolor{textcolor}%
\pgfsetfillcolor{textcolor}%
\pgftext[x=0.204222in, y=0.569806in, left, base]{\color{textcolor}\rmfamily\fontsize{7.000000}{8.400000}\selectfont \ensuremath{-}0.5}%
\end{pgfscope}%
\begin{pgfscope}%
\pgfsetbuttcap%
\pgfsetroundjoin%
\definecolor{currentfill}{rgb}{0.000000,0.000000,0.000000}%
\pgfsetfillcolor{currentfill}%
\pgfsetlinewidth{0.501875pt}%
\definecolor{currentstroke}{rgb}{0.000000,0.000000,0.000000}%
\pgfsetstrokecolor{currentstroke}%
\pgfsetdash{}{0pt}%
\pgfsys@defobject{currentmarker}{\pgfqpoint{0.000000in}{0.000000in}}{\pgfqpoint{0.041667in}{0.000000in}}{%
\pgfpathmoveto{\pgfqpoint{0.000000in}{0.000000in}}%
\pgfpathlineto{\pgfqpoint{0.041667in}{0.000000in}}%
\pgfusepath{stroke,fill}%
}%
\begin{pgfscope}%
\pgfsys@transformshift{0.481681in}{0.739656in}%
\pgfsys@useobject{currentmarker}{}%
\end{pgfscope}%
\end{pgfscope}%
\begin{pgfscope}%
\pgfsetbuttcap%
\pgfsetroundjoin%
\definecolor{currentfill}{rgb}{0.000000,0.000000,0.000000}%
\pgfsetfillcolor{currentfill}%
\pgfsetlinewidth{0.501875pt}%
\definecolor{currentstroke}{rgb}{0.000000,0.000000,0.000000}%
\pgfsetstrokecolor{currentstroke}%
\pgfsetdash{}{0pt}%
\pgfsys@defobject{currentmarker}{\pgfqpoint{-0.041667in}{0.000000in}}{\pgfqpoint{-0.000000in}{0.000000in}}{%
\pgfpathmoveto{\pgfqpoint{-0.000000in}{0.000000in}}%
\pgfpathlineto{\pgfqpoint{-0.041667in}{0.000000in}}%
\pgfusepath{stroke,fill}%
}%
\begin{pgfscope}%
\pgfsys@transformshift{6.267353in}{0.739656in}%
\pgfsys@useobject{currentmarker}{}%
\end{pgfscope}%
\end{pgfscope}%
\begin{pgfscope}%
\definecolor{textcolor}{rgb}{0.000000,0.000000,0.000000}%
\pgfsetstrokecolor{textcolor}%
\pgfsetfillcolor{textcolor}%
\pgftext[x=0.291028in, y=0.705920in, left, base]{\color{textcolor}\rmfamily\fontsize{7.000000}{8.400000}\selectfont 0.0}%
\end{pgfscope}%
\begin{pgfscope}%
\pgfsetbuttcap%
\pgfsetroundjoin%
\definecolor{currentfill}{rgb}{0.000000,0.000000,0.000000}%
\pgfsetfillcolor{currentfill}%
\pgfsetlinewidth{0.501875pt}%
\definecolor{currentstroke}{rgb}{0.000000,0.000000,0.000000}%
\pgfsetstrokecolor{currentstroke}%
\pgfsetdash{}{0pt}%
\pgfsys@defobject{currentmarker}{\pgfqpoint{0.000000in}{0.000000in}}{\pgfqpoint{0.041667in}{0.000000in}}{%
\pgfpathmoveto{\pgfqpoint{0.000000in}{0.000000in}}%
\pgfpathlineto{\pgfqpoint{0.041667in}{0.000000in}}%
\pgfusepath{stroke,fill}%
}%
\begin{pgfscope}%
\pgfsys@transformshift{0.481681in}{0.875770in}%
\pgfsys@useobject{currentmarker}{}%
\end{pgfscope}%
\end{pgfscope}%
\begin{pgfscope}%
\pgfsetbuttcap%
\pgfsetroundjoin%
\definecolor{currentfill}{rgb}{0.000000,0.000000,0.000000}%
\pgfsetfillcolor{currentfill}%
\pgfsetlinewidth{0.501875pt}%
\definecolor{currentstroke}{rgb}{0.000000,0.000000,0.000000}%
\pgfsetstrokecolor{currentstroke}%
\pgfsetdash{}{0pt}%
\pgfsys@defobject{currentmarker}{\pgfqpoint{-0.041667in}{0.000000in}}{\pgfqpoint{-0.000000in}{0.000000in}}{%
\pgfpathmoveto{\pgfqpoint{-0.000000in}{0.000000in}}%
\pgfpathlineto{\pgfqpoint{-0.041667in}{0.000000in}}%
\pgfusepath{stroke,fill}%
}%
\begin{pgfscope}%
\pgfsys@transformshift{6.267353in}{0.875770in}%
\pgfsys@useobject{currentmarker}{}%
\end{pgfscope}%
\end{pgfscope}%
\begin{pgfscope}%
\definecolor{textcolor}{rgb}{0.000000,0.000000,0.000000}%
\pgfsetstrokecolor{textcolor}%
\pgfsetfillcolor{textcolor}%
\pgftext[x=0.291028in, y=0.842034in, left, base]{\color{textcolor}\rmfamily\fontsize{7.000000}{8.400000}\selectfont 0.5}%
\end{pgfscope}%
\begin{pgfscope}%
\pgfsetbuttcap%
\pgfsetroundjoin%
\definecolor{currentfill}{rgb}{0.000000,0.000000,0.000000}%
\pgfsetfillcolor{currentfill}%
\pgfsetlinewidth{0.501875pt}%
\definecolor{currentstroke}{rgb}{0.000000,0.000000,0.000000}%
\pgfsetstrokecolor{currentstroke}%
\pgfsetdash{}{0pt}%
\pgfsys@defobject{currentmarker}{\pgfqpoint{0.000000in}{0.000000in}}{\pgfqpoint{0.020833in}{0.000000in}}{%
\pgfpathmoveto{\pgfqpoint{0.000000in}{0.000000in}}%
\pgfpathlineto{\pgfqpoint{0.020833in}{0.000000in}}%
\pgfusepath{stroke,fill}%
}%
\begin{pgfscope}%
\pgfsys@transformshift{0.481681in}{0.630765in}%
\pgfsys@useobject{currentmarker}{}%
\end{pgfscope}%
\end{pgfscope}%
\begin{pgfscope}%
\pgfsetbuttcap%
\pgfsetroundjoin%
\definecolor{currentfill}{rgb}{0.000000,0.000000,0.000000}%
\pgfsetfillcolor{currentfill}%
\pgfsetlinewidth{0.501875pt}%
\definecolor{currentstroke}{rgb}{0.000000,0.000000,0.000000}%
\pgfsetstrokecolor{currentstroke}%
\pgfsetdash{}{0pt}%
\pgfsys@defobject{currentmarker}{\pgfqpoint{-0.020833in}{0.000000in}}{\pgfqpoint{-0.000000in}{0.000000in}}{%
\pgfpathmoveto{\pgfqpoint{-0.000000in}{0.000000in}}%
\pgfpathlineto{\pgfqpoint{-0.020833in}{0.000000in}}%
\pgfusepath{stroke,fill}%
}%
\begin{pgfscope}%
\pgfsys@transformshift{6.267353in}{0.630765in}%
\pgfsys@useobject{currentmarker}{}%
\end{pgfscope}%
\end{pgfscope}%
\begin{pgfscope}%
\pgfsetbuttcap%
\pgfsetroundjoin%
\definecolor{currentfill}{rgb}{0.000000,0.000000,0.000000}%
\pgfsetfillcolor{currentfill}%
\pgfsetlinewidth{0.501875pt}%
\definecolor{currentstroke}{rgb}{0.000000,0.000000,0.000000}%
\pgfsetstrokecolor{currentstroke}%
\pgfsetdash{}{0pt}%
\pgfsys@defobject{currentmarker}{\pgfqpoint{0.000000in}{0.000000in}}{\pgfqpoint{0.020833in}{0.000000in}}{%
\pgfpathmoveto{\pgfqpoint{0.000000in}{0.000000in}}%
\pgfpathlineto{\pgfqpoint{0.020833in}{0.000000in}}%
\pgfusepath{stroke,fill}%
}%
\begin{pgfscope}%
\pgfsys@transformshift{0.481681in}{0.657988in}%
\pgfsys@useobject{currentmarker}{}%
\end{pgfscope}%
\end{pgfscope}%
\begin{pgfscope}%
\pgfsetbuttcap%
\pgfsetroundjoin%
\definecolor{currentfill}{rgb}{0.000000,0.000000,0.000000}%
\pgfsetfillcolor{currentfill}%
\pgfsetlinewidth{0.501875pt}%
\definecolor{currentstroke}{rgb}{0.000000,0.000000,0.000000}%
\pgfsetstrokecolor{currentstroke}%
\pgfsetdash{}{0pt}%
\pgfsys@defobject{currentmarker}{\pgfqpoint{-0.020833in}{0.000000in}}{\pgfqpoint{-0.000000in}{0.000000in}}{%
\pgfpathmoveto{\pgfqpoint{-0.000000in}{0.000000in}}%
\pgfpathlineto{\pgfqpoint{-0.020833in}{0.000000in}}%
\pgfusepath{stroke,fill}%
}%
\begin{pgfscope}%
\pgfsys@transformshift{6.267353in}{0.657988in}%
\pgfsys@useobject{currentmarker}{}%
\end{pgfscope}%
\end{pgfscope}%
\begin{pgfscope}%
\pgfsetbuttcap%
\pgfsetroundjoin%
\definecolor{currentfill}{rgb}{0.000000,0.000000,0.000000}%
\pgfsetfillcolor{currentfill}%
\pgfsetlinewidth{0.501875pt}%
\definecolor{currentstroke}{rgb}{0.000000,0.000000,0.000000}%
\pgfsetstrokecolor{currentstroke}%
\pgfsetdash{}{0pt}%
\pgfsys@defobject{currentmarker}{\pgfqpoint{0.000000in}{0.000000in}}{\pgfqpoint{0.020833in}{0.000000in}}{%
\pgfpathmoveto{\pgfqpoint{0.000000in}{0.000000in}}%
\pgfpathlineto{\pgfqpoint{0.020833in}{0.000000in}}%
\pgfusepath{stroke,fill}%
}%
\begin{pgfscope}%
\pgfsys@transformshift{0.481681in}{0.685211in}%
\pgfsys@useobject{currentmarker}{}%
\end{pgfscope}%
\end{pgfscope}%
\begin{pgfscope}%
\pgfsetbuttcap%
\pgfsetroundjoin%
\definecolor{currentfill}{rgb}{0.000000,0.000000,0.000000}%
\pgfsetfillcolor{currentfill}%
\pgfsetlinewidth{0.501875pt}%
\definecolor{currentstroke}{rgb}{0.000000,0.000000,0.000000}%
\pgfsetstrokecolor{currentstroke}%
\pgfsetdash{}{0pt}%
\pgfsys@defobject{currentmarker}{\pgfqpoint{-0.020833in}{0.000000in}}{\pgfqpoint{-0.000000in}{0.000000in}}{%
\pgfpathmoveto{\pgfqpoint{-0.000000in}{0.000000in}}%
\pgfpathlineto{\pgfqpoint{-0.020833in}{0.000000in}}%
\pgfusepath{stroke,fill}%
}%
\begin{pgfscope}%
\pgfsys@transformshift{6.267353in}{0.685211in}%
\pgfsys@useobject{currentmarker}{}%
\end{pgfscope}%
\end{pgfscope}%
\begin{pgfscope}%
\pgfsetbuttcap%
\pgfsetroundjoin%
\definecolor{currentfill}{rgb}{0.000000,0.000000,0.000000}%
\pgfsetfillcolor{currentfill}%
\pgfsetlinewidth{0.501875pt}%
\definecolor{currentstroke}{rgb}{0.000000,0.000000,0.000000}%
\pgfsetstrokecolor{currentstroke}%
\pgfsetdash{}{0pt}%
\pgfsys@defobject{currentmarker}{\pgfqpoint{0.000000in}{0.000000in}}{\pgfqpoint{0.020833in}{0.000000in}}{%
\pgfpathmoveto{\pgfqpoint{0.000000in}{0.000000in}}%
\pgfpathlineto{\pgfqpoint{0.020833in}{0.000000in}}%
\pgfusepath{stroke,fill}%
}%
\begin{pgfscope}%
\pgfsys@transformshift{0.481681in}{0.712433in}%
\pgfsys@useobject{currentmarker}{}%
\end{pgfscope}%
\end{pgfscope}%
\begin{pgfscope}%
\pgfsetbuttcap%
\pgfsetroundjoin%
\definecolor{currentfill}{rgb}{0.000000,0.000000,0.000000}%
\pgfsetfillcolor{currentfill}%
\pgfsetlinewidth{0.501875pt}%
\definecolor{currentstroke}{rgb}{0.000000,0.000000,0.000000}%
\pgfsetstrokecolor{currentstroke}%
\pgfsetdash{}{0pt}%
\pgfsys@defobject{currentmarker}{\pgfqpoint{-0.020833in}{0.000000in}}{\pgfqpoint{-0.000000in}{0.000000in}}{%
\pgfpathmoveto{\pgfqpoint{-0.000000in}{0.000000in}}%
\pgfpathlineto{\pgfqpoint{-0.020833in}{0.000000in}}%
\pgfusepath{stroke,fill}%
}%
\begin{pgfscope}%
\pgfsys@transformshift{6.267353in}{0.712433in}%
\pgfsys@useobject{currentmarker}{}%
\end{pgfscope}%
\end{pgfscope}%
\begin{pgfscope}%
\pgfsetbuttcap%
\pgfsetroundjoin%
\definecolor{currentfill}{rgb}{0.000000,0.000000,0.000000}%
\pgfsetfillcolor{currentfill}%
\pgfsetlinewidth{0.501875pt}%
\definecolor{currentstroke}{rgb}{0.000000,0.000000,0.000000}%
\pgfsetstrokecolor{currentstroke}%
\pgfsetdash{}{0pt}%
\pgfsys@defobject{currentmarker}{\pgfqpoint{0.000000in}{0.000000in}}{\pgfqpoint{0.020833in}{0.000000in}}{%
\pgfpathmoveto{\pgfqpoint{0.000000in}{0.000000in}}%
\pgfpathlineto{\pgfqpoint{0.020833in}{0.000000in}}%
\pgfusepath{stroke,fill}%
}%
\begin{pgfscope}%
\pgfsys@transformshift{0.481681in}{0.766879in}%
\pgfsys@useobject{currentmarker}{}%
\end{pgfscope}%
\end{pgfscope}%
\begin{pgfscope}%
\pgfsetbuttcap%
\pgfsetroundjoin%
\definecolor{currentfill}{rgb}{0.000000,0.000000,0.000000}%
\pgfsetfillcolor{currentfill}%
\pgfsetlinewidth{0.501875pt}%
\definecolor{currentstroke}{rgb}{0.000000,0.000000,0.000000}%
\pgfsetstrokecolor{currentstroke}%
\pgfsetdash{}{0pt}%
\pgfsys@defobject{currentmarker}{\pgfqpoint{-0.020833in}{0.000000in}}{\pgfqpoint{-0.000000in}{0.000000in}}{%
\pgfpathmoveto{\pgfqpoint{-0.000000in}{0.000000in}}%
\pgfpathlineto{\pgfqpoint{-0.020833in}{0.000000in}}%
\pgfusepath{stroke,fill}%
}%
\begin{pgfscope}%
\pgfsys@transformshift{6.267353in}{0.766879in}%
\pgfsys@useobject{currentmarker}{}%
\end{pgfscope}%
\end{pgfscope}%
\begin{pgfscope}%
\pgfsetbuttcap%
\pgfsetroundjoin%
\definecolor{currentfill}{rgb}{0.000000,0.000000,0.000000}%
\pgfsetfillcolor{currentfill}%
\pgfsetlinewidth{0.501875pt}%
\definecolor{currentstroke}{rgb}{0.000000,0.000000,0.000000}%
\pgfsetstrokecolor{currentstroke}%
\pgfsetdash{}{0pt}%
\pgfsys@defobject{currentmarker}{\pgfqpoint{0.000000in}{0.000000in}}{\pgfqpoint{0.020833in}{0.000000in}}{%
\pgfpathmoveto{\pgfqpoint{0.000000in}{0.000000in}}%
\pgfpathlineto{\pgfqpoint{0.020833in}{0.000000in}}%
\pgfusepath{stroke,fill}%
}%
\begin{pgfscope}%
\pgfsys@transformshift{0.481681in}{0.794102in}%
\pgfsys@useobject{currentmarker}{}%
\end{pgfscope}%
\end{pgfscope}%
\begin{pgfscope}%
\pgfsetbuttcap%
\pgfsetroundjoin%
\definecolor{currentfill}{rgb}{0.000000,0.000000,0.000000}%
\pgfsetfillcolor{currentfill}%
\pgfsetlinewidth{0.501875pt}%
\definecolor{currentstroke}{rgb}{0.000000,0.000000,0.000000}%
\pgfsetstrokecolor{currentstroke}%
\pgfsetdash{}{0pt}%
\pgfsys@defobject{currentmarker}{\pgfqpoint{-0.020833in}{0.000000in}}{\pgfqpoint{-0.000000in}{0.000000in}}{%
\pgfpathmoveto{\pgfqpoint{-0.000000in}{0.000000in}}%
\pgfpathlineto{\pgfqpoint{-0.020833in}{0.000000in}}%
\pgfusepath{stroke,fill}%
}%
\begin{pgfscope}%
\pgfsys@transformshift{6.267353in}{0.794102in}%
\pgfsys@useobject{currentmarker}{}%
\end{pgfscope}%
\end{pgfscope}%
\begin{pgfscope}%
\pgfsetbuttcap%
\pgfsetroundjoin%
\definecolor{currentfill}{rgb}{0.000000,0.000000,0.000000}%
\pgfsetfillcolor{currentfill}%
\pgfsetlinewidth{0.501875pt}%
\definecolor{currentstroke}{rgb}{0.000000,0.000000,0.000000}%
\pgfsetstrokecolor{currentstroke}%
\pgfsetdash{}{0pt}%
\pgfsys@defobject{currentmarker}{\pgfqpoint{0.000000in}{0.000000in}}{\pgfqpoint{0.020833in}{0.000000in}}{%
\pgfpathmoveto{\pgfqpoint{0.000000in}{0.000000in}}%
\pgfpathlineto{\pgfqpoint{0.020833in}{0.000000in}}%
\pgfusepath{stroke,fill}%
}%
\begin{pgfscope}%
\pgfsys@transformshift{0.481681in}{0.821324in}%
\pgfsys@useobject{currentmarker}{}%
\end{pgfscope}%
\end{pgfscope}%
\begin{pgfscope}%
\pgfsetbuttcap%
\pgfsetroundjoin%
\definecolor{currentfill}{rgb}{0.000000,0.000000,0.000000}%
\pgfsetfillcolor{currentfill}%
\pgfsetlinewidth{0.501875pt}%
\definecolor{currentstroke}{rgb}{0.000000,0.000000,0.000000}%
\pgfsetstrokecolor{currentstroke}%
\pgfsetdash{}{0pt}%
\pgfsys@defobject{currentmarker}{\pgfqpoint{-0.020833in}{0.000000in}}{\pgfqpoint{-0.000000in}{0.000000in}}{%
\pgfpathmoveto{\pgfqpoint{-0.000000in}{0.000000in}}%
\pgfpathlineto{\pgfqpoint{-0.020833in}{0.000000in}}%
\pgfusepath{stroke,fill}%
}%
\begin{pgfscope}%
\pgfsys@transformshift{6.267353in}{0.821324in}%
\pgfsys@useobject{currentmarker}{}%
\end{pgfscope}%
\end{pgfscope}%
\begin{pgfscope}%
\pgfsetbuttcap%
\pgfsetroundjoin%
\definecolor{currentfill}{rgb}{0.000000,0.000000,0.000000}%
\pgfsetfillcolor{currentfill}%
\pgfsetlinewidth{0.501875pt}%
\definecolor{currentstroke}{rgb}{0.000000,0.000000,0.000000}%
\pgfsetstrokecolor{currentstroke}%
\pgfsetdash{}{0pt}%
\pgfsys@defobject{currentmarker}{\pgfqpoint{0.000000in}{0.000000in}}{\pgfqpoint{0.020833in}{0.000000in}}{%
\pgfpathmoveto{\pgfqpoint{0.000000in}{0.000000in}}%
\pgfpathlineto{\pgfqpoint{0.020833in}{0.000000in}}%
\pgfusepath{stroke,fill}%
}%
\begin{pgfscope}%
\pgfsys@transformshift{0.481681in}{0.848547in}%
\pgfsys@useobject{currentmarker}{}%
\end{pgfscope}%
\end{pgfscope}%
\begin{pgfscope}%
\pgfsetbuttcap%
\pgfsetroundjoin%
\definecolor{currentfill}{rgb}{0.000000,0.000000,0.000000}%
\pgfsetfillcolor{currentfill}%
\pgfsetlinewidth{0.501875pt}%
\definecolor{currentstroke}{rgb}{0.000000,0.000000,0.000000}%
\pgfsetstrokecolor{currentstroke}%
\pgfsetdash{}{0pt}%
\pgfsys@defobject{currentmarker}{\pgfqpoint{-0.020833in}{0.000000in}}{\pgfqpoint{-0.000000in}{0.000000in}}{%
\pgfpathmoveto{\pgfqpoint{-0.000000in}{0.000000in}}%
\pgfpathlineto{\pgfqpoint{-0.020833in}{0.000000in}}%
\pgfusepath{stroke,fill}%
}%
\begin{pgfscope}%
\pgfsys@transformshift{6.267353in}{0.848547in}%
\pgfsys@useobject{currentmarker}{}%
\end{pgfscope}%
\end{pgfscope}%
\begin{pgfscope}%
\definecolor{textcolor}{rgb}{0.000000,0.000000,0.000000}%
\pgfsetstrokecolor{textcolor}%
\pgfsetfillcolor{textcolor}%
\pgftext[x=0.148667in,y=0.739656in,,bottom,rotate=90.000000]{\color{textcolor}\rmfamily\fontsize{8.000000}{9.600000}\selectfont Unterschied}%
\end{pgfscope}%
\begin{pgfscope}%
\pgfsetrectcap%
\pgfsetmiterjoin%
\pgfsetlinewidth{0.501875pt}%
\definecolor{currentstroke}{rgb}{0.000000,0.000000,0.000000}%
\pgfsetstrokecolor{currentstroke}%
\pgfsetdash{}{0pt}%
\pgfpathmoveto{\pgfqpoint{0.481681in}{0.586309in}}%
\pgfpathlineto{\pgfqpoint{0.481681in}{0.893003in}}%
\pgfusepath{stroke}%
\end{pgfscope}%
\begin{pgfscope}%
\pgfsetrectcap%
\pgfsetmiterjoin%
\pgfsetlinewidth{0.501875pt}%
\definecolor{currentstroke}{rgb}{0.000000,0.000000,0.000000}%
\pgfsetstrokecolor{currentstroke}%
\pgfsetdash{}{0pt}%
\pgfpathmoveto{\pgfqpoint{6.267353in}{0.586309in}}%
\pgfpathlineto{\pgfqpoint{6.267353in}{0.893003in}}%
\pgfusepath{stroke}%
\end{pgfscope}%
\begin{pgfscope}%
\pgfsetrectcap%
\pgfsetmiterjoin%
\pgfsetlinewidth{0.501875pt}%
\definecolor{currentstroke}{rgb}{0.000000,0.000000,0.000000}%
\pgfsetstrokecolor{currentstroke}%
\pgfsetdash{}{0pt}%
\pgfpathmoveto{\pgfqpoint{0.481681in}{0.586309in}}%
\pgfpathlineto{\pgfqpoint{6.267353in}{0.586309in}}%
\pgfusepath{stroke}%
\end{pgfscope}%
\begin{pgfscope}%
\pgfsetrectcap%
\pgfsetmiterjoin%
\pgfsetlinewidth{0.501875pt}%
\definecolor{currentstroke}{rgb}{0.000000,0.000000,0.000000}%
\pgfsetstrokecolor{currentstroke}%
\pgfsetdash{}{0pt}%
\pgfpathmoveto{\pgfqpoint{0.481681in}{0.893003in}}%
\pgfpathlineto{\pgfqpoint{6.267353in}{0.893003in}}%
\pgfusepath{stroke}%
\end{pgfscope}%
\end{pgfpicture}%
\makeatother%
\endgroup%

  \end{center}
  \caption{Verhältnis von Aufwand und Komplexität im Alstonii Projekt.}
\end{figure}


Das Projekt wird über einen Zeitraum von zwei Jahren von März 2020 bis
Mai 2022 betrachtet. Eine dicke schwarze Linie zeigt die
Aufwandsabschätzungen. Die restlichen, dünneren Linien zeigen die
Komplexitätsberechnungen für die zyklomatische Komplexität, die
Einrückungskomplexität, den Halstead Aufwand und die logischen
Codezeilen. Rein visuell lässt sich erkennen, dass die Linien der
Komplexitätsmetriken ähnlich zu den Aufwandsabschätzungen verlaufen. Es
werden jedoch auch grö\ss ere Abweichungen deutlich. Einige dieser
Abweichungen konnten in einem Abschlussgespräch zu dem Projekt erklärt
werden.

Zunächst konnte für die grö\ss eren Sprünge in der Einrückungskomplexität
(Nummer 1) die Verwendung eines Linters als Erklärung herangezogen
werden. Die Einrückungskomplexität wird über Einrückungen im Code
berechnet. Diese werden durch einen Linter teils automatisiert in einem
gro\ss en Umfang geändert. Der Linter passt hier die Einrückungen auf ein
festes Niveau an, wodurch die Summe der Einrückungen (siehe Kapitel \ref{Einruckungskomplexitat}) in
einem gro\ss en Umfang geändert wird. Bei den Sprüngen in der
Einrückungskomplexität fällt auch auf, dass das Komplexitätsma\ss  nach dem
Sprung wieder auf das gleiche Niveau wie vor dem Sprung zurückkehrt. Das
kann dadurch erklärt werden, dass die Änderungen in der Einrückung durch
den Linter nur temporär geschehen. Wird wieder zu dem ursprünglichen
Einrückungsformat zurückgewechselt, ist die Summe der Einrückungen
wieder auf dem alten Niveau.

Weiter ist von März 2021 bis Oktober 2021 ein Plateau in der
zyklomatischen Komplexität zu beobachten (Nummer 3). Dieses Plateau
lässt sich durch eine Fluktuation in der Besetzung des Projektteams
erklären. So wurde hier der Stil der Entwicklung grundsätzlich geändert.
Nach der Einarbeitungsphase des neuen Teams kam es zu einem
Sprungartigen Anstieg in den Komplexitätsmetriken.

Als letzter auffälliger Punkt wurde ein plötzliches Abfallen mit einem
anschlie\ss enden Abflachen der Komplexitätsmetriken zu Ende des Projektes
ab März 2022 identifiziert. Als potenzielle Erklärung hierfür wurde eine
Verschiebung des Entwicklungszieles vorgeschlagen. So wurde zum Ende des
Projektes ein verstärkter Fokus auf die Konsolidierung der Codebasis und
auf das Verbessern der Leistung der Applikation gelegt. Bei einer
Überarbeitung der Applikation ist zu erwarten, dass der Umfang der
Anwendung gleichbleibt. Diese Erwartung konnte mit den vorliegenden
Messdaten bestätigt werden.

Zusammenfassend entsprechen die hier gesammelten Daten zwar in grobem
Ma\ss e der Hypothese, es sind jedoch einige signifikante Abweichungen
zwischen den Ma\ss en und den Aufwandsabschätzungen erkenntlich. Zu einem
Teil konnten sie durch technische Störfaktoren erklärt werden. Zu einem
anderen Teil sind sie aber durch Eigenheiten in dem Verlauf des
Projektes zu erklären. Nach dieser Durchsicht der Messdaten ist also
eine weniger starke Korrelation als z.B. in dem \ac{NDA} Projekt zu erwarten.
Die Korrelation der Komplexitätsmetriken mit den Aufwandsabschätzungen
wird nun berechnet.

TODO

Für alle Untersuchungsergebnisse wurde ein P-Wert von \textless{} .00001
ermittelt.
Bei einem, für Sozial- und Wirtschaftswissenschaften üblichen
Signifikanzeniveau von .05 lässt sich also sagen, dass die Ergebnisse
signifikant sind.

Wie zu erwarten war, wurden in diesem Fall geringfügig niedrigere
Korrelationskoeffizienten ermittelt als z.B. in dem \ac{NDA} Projekt
ermittelt. Im Durchschnitt sind die Werte aber nur um etwa 5\% geringer
als in dem \ac{NDA} Projekt.

\subsection{Fazit}\label{Alstonii-fazit}

Auch in dem Alstonii Projekt konnten erfolgreich Daten entsprechend
den zuvor gestellten Anforderungen gesammelt werden. Diese Daten konnten
ebenfalls erfolgreich in einen Zusammenhang mit der These gestellt
werden. Trotz stärkerer Störfaktoren wurde hier über
Korrelationkoeffizienten und Komplexitätsmetriken hinweg eine starke
Korrelation festgestellt. Also spricht dieser Fall für eine Bestätigung
der hypothetisierten Korrelation zwischen Softwarekomplexitätsmetriken
und Aufwandsabschätzungen.

\subsection{Kritik}\label{Alstonii-kritik}

In diesem Projekt ist die Bedeutung der Störfaktoren deutlich grö\ss er als
in dem NDA Projekt. Die Abweichungen zwischen den Komplexitätsma\ss en und
den Aufwandsabschätzungen lie\ss en sich nicht ausnahmslos erklären.

\section{Lacustris}\label{lacustris}

Ein weiteres Projekt ist das Lacustris Projekt\footnote{Lacustris
  Zeile 3}. Hierbei handelt es sich um einen Microservice, welcher ein
Teil eines grö\ss eren Projektes darstellt. Der Microservice ist dafür
verantwortlich, Daten aus einer \ac{AWS} Instanz abzurufen und diese für
andere Microservices zur Verfügung zu stellen\footnote{Lacustris Zeile 4}.
Entwickelt wird dieses Projekt ebenfalls in einer agilen Arbeitsweise,
sowohl von onshore, als auch von offshore Mitarbeitern. Primär sind die
Entwickler in Ägypten lokalisiert\footnote{Lacustris Zeile 7 und 8}. Die
Teams bestehen dabei aus einem Mix aus junior und senior Mitarbeitern.
Alle Mitarbeiter sind erst seit wenigen Monaten an dem Projekt
beteiligt. Die Programmiersprache des Projektes ist ausschlie\ss lich Java.

\subsection{Datenerhebung}\label{lm-Datenerhebung}

Zum Berechnen der Korrelationen sollen auch in diesem Projekt die
Aufwandsabschätzungen und die Codekomplexitätsmetriken erhoben werden.

Die Aufwandsabschätzungen lassen sich in diesem Projekt ähnlich wie in
den anderen Projekten der Projektmanagementsoftware Jira entnehmen.
Dabei liegen die Story Point Abschätzungen sowie ein Zeitstempel in
einem Feld names „Resolved`` vor. In diesem Projekt konnten insgesamt 58
relevante User Storys gefunden werden

Auch zu dem Git Repository des Projektes konnte ein Zugang erhalten
werden. Hier konnte die Entwicklungsgeschichte des Projektes an 965
Zeitpunkten rekonstruiert werden.

\subsection{Auswertung}\label{lm-Auswertung}

Für die Auswertung werden nun auch in diesem Projekt die Zeitreihe der
Aufwandsabschätzungen mit der der Codekomplexitätsmetriken in Verbindung
gebracht. Der daraus resultierende Korrelationsgraph ist in Abbildung TODO
erkenntlich.

\begin{figure}\label{lacustris-graph}
  \begin{center}
      %% Creator: Matplotlib, PGF backend
%%
%% To include the figure in your LaTeX document, write
%%   \input{<filename>.pgf}
%%
%% Make sure the required packages are loaded in your preamble
%%   \usepackage{pgf}
%%
%% Also ensure that all the required font packages are loaded; for instance,
%% the lmodern package is sometimes necessary when using math font.
%%   \usepackage{lmodern}
%%
%% Figures using additional raster images can only be included by \input if
%% they are in the same directory as the main LaTeX file. For loading figures
%% from other directories you can use the `import` package
%%   \usepackage{import}
%%
%% and then include the figures with
%%   \import{<path to file>}{<filename>.pgf}
%%
%% Matplotlib used the following preamble
%%   \usepackage{fontspec}
%%
\begingroup%
\makeatletter%
\begin{pgfpicture}%
\pgfpathrectangle{\pgfpointorigin}{\pgfqpoint{6.317353in}{3.277753in}}%
\pgfusepath{use as bounding box, clip}%
\begin{pgfscope}%
\pgfsetbuttcap%
\pgfsetmiterjoin%
\definecolor{currentfill}{rgb}{1.000000,1.000000,1.000000}%
\pgfsetfillcolor{currentfill}%
\pgfsetlinewidth{0.000000pt}%
\definecolor{currentstroke}{rgb}{1.000000,1.000000,1.000000}%
\pgfsetstrokecolor{currentstroke}%
\pgfsetdash{}{0pt}%
\pgfpathmoveto{\pgfqpoint{0.000000in}{-0.000000in}}%
\pgfpathlineto{\pgfqpoint{6.317353in}{-0.000000in}}%
\pgfpathlineto{\pgfqpoint{6.317353in}{3.277753in}}%
\pgfpathlineto{\pgfqpoint{0.000000in}{3.277753in}}%
\pgfpathlineto{\pgfqpoint{0.000000in}{-0.000000in}}%
\pgfpathclose%
\pgfusepath{fill}%
\end{pgfscope}%
\begin{pgfscope}%
\pgfsetbuttcap%
\pgfsetmiterjoin%
\definecolor{currentfill}{rgb}{1.000000,1.000000,1.000000}%
\pgfsetfillcolor{currentfill}%
\pgfsetlinewidth{0.000000pt}%
\definecolor{currentstroke}{rgb}{0.000000,0.000000,0.000000}%
\pgfsetstrokecolor{currentstroke}%
\pgfsetstrokeopacity{0.000000}%
\pgfsetdash{}{0pt}%
\pgfpathmoveto{\pgfqpoint{0.481681in}{1.080890in}}%
\pgfpathlineto{\pgfqpoint{6.267353in}{1.080890in}}%
\pgfpathlineto{\pgfqpoint{6.267353in}{3.227753in}}%
\pgfpathlineto{\pgfqpoint{0.481681in}{3.227753in}}%
\pgfpathlineto{\pgfqpoint{0.481681in}{1.080890in}}%
\pgfpathclose%
\pgfusepath{fill}%
\end{pgfscope}%
\begin{pgfscope}%
\pgfpathrectangle{\pgfqpoint{0.481681in}{1.080890in}}{\pgfqpoint{5.785672in}{2.146863in}}%
\pgfusepath{clip}%
\pgfsetrectcap%
\pgfsetroundjoin%
\pgfsetlinewidth{0.100375pt}%
\definecolor{currentstroke}{rgb}{0.501961,0.501961,0.501961}%
\pgfsetstrokecolor{currentstroke}%
\pgfsetdash{}{0pt}%
\pgfpathmoveto{\pgfqpoint{0.854687in}{1.080890in}}%
\pgfpathlineto{\pgfqpoint{0.854687in}{3.227753in}}%
\pgfusepath{stroke}%
\end{pgfscope}%
\begin{pgfscope}%
\pgfsetbuttcap%
\pgfsetroundjoin%
\definecolor{currentfill}{rgb}{0.000000,0.000000,0.000000}%
\pgfsetfillcolor{currentfill}%
\pgfsetlinewidth{0.501875pt}%
\definecolor{currentstroke}{rgb}{0.000000,0.000000,0.000000}%
\pgfsetstrokecolor{currentstroke}%
\pgfsetdash{}{0pt}%
\pgfsys@defobject{currentmarker}{\pgfqpoint{0.000000in}{0.000000in}}{\pgfqpoint{0.000000in}{0.041667in}}{%
\pgfpathmoveto{\pgfqpoint{0.000000in}{0.000000in}}%
\pgfpathlineto{\pgfqpoint{0.000000in}{0.041667in}}%
\pgfusepath{stroke,fill}%
}%
\begin{pgfscope}%
\pgfsys@transformshift{0.854687in}{1.080890in}%
\pgfsys@useobject{currentmarker}{}%
\end{pgfscope}%
\end{pgfscope}%
\begin{pgfscope}%
\pgfsetbuttcap%
\pgfsetroundjoin%
\definecolor{currentfill}{rgb}{0.000000,0.000000,0.000000}%
\pgfsetfillcolor{currentfill}%
\pgfsetlinewidth{0.501875pt}%
\definecolor{currentstroke}{rgb}{0.000000,0.000000,0.000000}%
\pgfsetstrokecolor{currentstroke}%
\pgfsetdash{}{0pt}%
\pgfsys@defobject{currentmarker}{\pgfqpoint{0.000000in}{-0.041667in}}{\pgfqpoint{0.000000in}{0.000000in}}{%
\pgfpathmoveto{\pgfqpoint{0.000000in}{0.000000in}}%
\pgfpathlineto{\pgfqpoint{0.000000in}{-0.041667in}}%
\pgfusepath{stroke,fill}%
}%
\begin{pgfscope}%
\pgfsys@transformshift{0.854687in}{3.227753in}%
\pgfsys@useobject{currentmarker}{}%
\end{pgfscope}%
\end{pgfscope}%
\begin{pgfscope}%
\pgfpathrectangle{\pgfqpoint{0.481681in}{1.080890in}}{\pgfqpoint{5.785672in}{2.146863in}}%
\pgfusepath{clip}%
\pgfsetrectcap%
\pgfsetroundjoin%
\pgfsetlinewidth{0.100375pt}%
\definecolor{currentstroke}{rgb}{0.501961,0.501961,0.501961}%
\pgfsetstrokecolor{currentstroke}%
\pgfsetdash{}{0pt}%
\pgfpathmoveto{\pgfqpoint{1.468760in}{1.080890in}}%
\pgfpathlineto{\pgfqpoint{1.468760in}{3.227753in}}%
\pgfusepath{stroke}%
\end{pgfscope}%
\begin{pgfscope}%
\pgfsetbuttcap%
\pgfsetroundjoin%
\definecolor{currentfill}{rgb}{0.000000,0.000000,0.000000}%
\pgfsetfillcolor{currentfill}%
\pgfsetlinewidth{0.501875pt}%
\definecolor{currentstroke}{rgb}{0.000000,0.000000,0.000000}%
\pgfsetstrokecolor{currentstroke}%
\pgfsetdash{}{0pt}%
\pgfsys@defobject{currentmarker}{\pgfqpoint{0.000000in}{0.000000in}}{\pgfqpoint{0.000000in}{0.041667in}}{%
\pgfpathmoveto{\pgfqpoint{0.000000in}{0.000000in}}%
\pgfpathlineto{\pgfqpoint{0.000000in}{0.041667in}}%
\pgfusepath{stroke,fill}%
}%
\begin{pgfscope}%
\pgfsys@transformshift{1.468760in}{1.080890in}%
\pgfsys@useobject{currentmarker}{}%
\end{pgfscope}%
\end{pgfscope}%
\begin{pgfscope}%
\pgfsetbuttcap%
\pgfsetroundjoin%
\definecolor{currentfill}{rgb}{0.000000,0.000000,0.000000}%
\pgfsetfillcolor{currentfill}%
\pgfsetlinewidth{0.501875pt}%
\definecolor{currentstroke}{rgb}{0.000000,0.000000,0.000000}%
\pgfsetstrokecolor{currentstroke}%
\pgfsetdash{}{0pt}%
\pgfsys@defobject{currentmarker}{\pgfqpoint{0.000000in}{-0.041667in}}{\pgfqpoint{0.000000in}{0.000000in}}{%
\pgfpathmoveto{\pgfqpoint{0.000000in}{0.000000in}}%
\pgfpathlineto{\pgfqpoint{0.000000in}{-0.041667in}}%
\pgfusepath{stroke,fill}%
}%
\begin{pgfscope}%
\pgfsys@transformshift{1.468760in}{3.227753in}%
\pgfsys@useobject{currentmarker}{}%
\end{pgfscope}%
\end{pgfscope}%
\begin{pgfscope}%
\pgfpathrectangle{\pgfqpoint{0.481681in}{1.080890in}}{\pgfqpoint{5.785672in}{2.146863in}}%
\pgfusepath{clip}%
\pgfsetrectcap%
\pgfsetroundjoin%
\pgfsetlinewidth{0.100375pt}%
\definecolor{currentstroke}{rgb}{0.501961,0.501961,0.501961}%
\pgfsetstrokecolor{currentstroke}%
\pgfsetdash{}{0pt}%
\pgfpathmoveto{\pgfqpoint{2.082832in}{1.080890in}}%
\pgfpathlineto{\pgfqpoint{2.082832in}{3.227753in}}%
\pgfusepath{stroke}%
\end{pgfscope}%
\begin{pgfscope}%
\pgfsetbuttcap%
\pgfsetroundjoin%
\definecolor{currentfill}{rgb}{0.000000,0.000000,0.000000}%
\pgfsetfillcolor{currentfill}%
\pgfsetlinewidth{0.501875pt}%
\definecolor{currentstroke}{rgb}{0.000000,0.000000,0.000000}%
\pgfsetstrokecolor{currentstroke}%
\pgfsetdash{}{0pt}%
\pgfsys@defobject{currentmarker}{\pgfqpoint{0.000000in}{0.000000in}}{\pgfqpoint{0.000000in}{0.041667in}}{%
\pgfpathmoveto{\pgfqpoint{0.000000in}{0.000000in}}%
\pgfpathlineto{\pgfqpoint{0.000000in}{0.041667in}}%
\pgfusepath{stroke,fill}%
}%
\begin{pgfscope}%
\pgfsys@transformshift{2.082832in}{1.080890in}%
\pgfsys@useobject{currentmarker}{}%
\end{pgfscope}%
\end{pgfscope}%
\begin{pgfscope}%
\pgfsetbuttcap%
\pgfsetroundjoin%
\definecolor{currentfill}{rgb}{0.000000,0.000000,0.000000}%
\pgfsetfillcolor{currentfill}%
\pgfsetlinewidth{0.501875pt}%
\definecolor{currentstroke}{rgb}{0.000000,0.000000,0.000000}%
\pgfsetstrokecolor{currentstroke}%
\pgfsetdash{}{0pt}%
\pgfsys@defobject{currentmarker}{\pgfqpoint{0.000000in}{-0.041667in}}{\pgfqpoint{0.000000in}{0.000000in}}{%
\pgfpathmoveto{\pgfqpoint{0.000000in}{0.000000in}}%
\pgfpathlineto{\pgfqpoint{0.000000in}{-0.041667in}}%
\pgfusepath{stroke,fill}%
}%
\begin{pgfscope}%
\pgfsys@transformshift{2.082832in}{3.227753in}%
\pgfsys@useobject{currentmarker}{}%
\end{pgfscope}%
\end{pgfscope}%
\begin{pgfscope}%
\pgfpathrectangle{\pgfqpoint{0.481681in}{1.080890in}}{\pgfqpoint{5.785672in}{2.146863in}}%
\pgfusepath{clip}%
\pgfsetrectcap%
\pgfsetroundjoin%
\pgfsetlinewidth{0.100375pt}%
\definecolor{currentstroke}{rgb}{0.501961,0.501961,0.501961}%
\pgfsetstrokecolor{currentstroke}%
\pgfsetdash{}{0pt}%
\pgfpathmoveto{\pgfqpoint{2.696905in}{1.080890in}}%
\pgfpathlineto{\pgfqpoint{2.696905in}{3.227753in}}%
\pgfusepath{stroke}%
\end{pgfscope}%
\begin{pgfscope}%
\pgfsetbuttcap%
\pgfsetroundjoin%
\definecolor{currentfill}{rgb}{0.000000,0.000000,0.000000}%
\pgfsetfillcolor{currentfill}%
\pgfsetlinewidth{0.501875pt}%
\definecolor{currentstroke}{rgb}{0.000000,0.000000,0.000000}%
\pgfsetstrokecolor{currentstroke}%
\pgfsetdash{}{0pt}%
\pgfsys@defobject{currentmarker}{\pgfqpoint{0.000000in}{0.000000in}}{\pgfqpoint{0.000000in}{0.041667in}}{%
\pgfpathmoveto{\pgfqpoint{0.000000in}{0.000000in}}%
\pgfpathlineto{\pgfqpoint{0.000000in}{0.041667in}}%
\pgfusepath{stroke,fill}%
}%
\begin{pgfscope}%
\pgfsys@transformshift{2.696905in}{1.080890in}%
\pgfsys@useobject{currentmarker}{}%
\end{pgfscope}%
\end{pgfscope}%
\begin{pgfscope}%
\pgfsetbuttcap%
\pgfsetroundjoin%
\definecolor{currentfill}{rgb}{0.000000,0.000000,0.000000}%
\pgfsetfillcolor{currentfill}%
\pgfsetlinewidth{0.501875pt}%
\definecolor{currentstroke}{rgb}{0.000000,0.000000,0.000000}%
\pgfsetstrokecolor{currentstroke}%
\pgfsetdash{}{0pt}%
\pgfsys@defobject{currentmarker}{\pgfqpoint{0.000000in}{-0.041667in}}{\pgfqpoint{0.000000in}{0.000000in}}{%
\pgfpathmoveto{\pgfqpoint{0.000000in}{0.000000in}}%
\pgfpathlineto{\pgfqpoint{0.000000in}{-0.041667in}}%
\pgfusepath{stroke,fill}%
}%
\begin{pgfscope}%
\pgfsys@transformshift{2.696905in}{3.227753in}%
\pgfsys@useobject{currentmarker}{}%
\end{pgfscope}%
\end{pgfscope}%
\begin{pgfscope}%
\pgfpathrectangle{\pgfqpoint{0.481681in}{1.080890in}}{\pgfqpoint{5.785672in}{2.146863in}}%
\pgfusepath{clip}%
\pgfsetrectcap%
\pgfsetroundjoin%
\pgfsetlinewidth{0.100375pt}%
\definecolor{currentstroke}{rgb}{0.501961,0.501961,0.501961}%
\pgfsetstrokecolor{currentstroke}%
\pgfsetdash{}{0pt}%
\pgfpathmoveto{\pgfqpoint{3.310977in}{1.080890in}}%
\pgfpathlineto{\pgfqpoint{3.310977in}{3.227753in}}%
\pgfusepath{stroke}%
\end{pgfscope}%
\begin{pgfscope}%
\pgfsetbuttcap%
\pgfsetroundjoin%
\definecolor{currentfill}{rgb}{0.000000,0.000000,0.000000}%
\pgfsetfillcolor{currentfill}%
\pgfsetlinewidth{0.501875pt}%
\definecolor{currentstroke}{rgb}{0.000000,0.000000,0.000000}%
\pgfsetstrokecolor{currentstroke}%
\pgfsetdash{}{0pt}%
\pgfsys@defobject{currentmarker}{\pgfqpoint{0.000000in}{0.000000in}}{\pgfqpoint{0.000000in}{0.041667in}}{%
\pgfpathmoveto{\pgfqpoint{0.000000in}{0.000000in}}%
\pgfpathlineto{\pgfqpoint{0.000000in}{0.041667in}}%
\pgfusepath{stroke,fill}%
}%
\begin{pgfscope}%
\pgfsys@transformshift{3.310977in}{1.080890in}%
\pgfsys@useobject{currentmarker}{}%
\end{pgfscope}%
\end{pgfscope}%
\begin{pgfscope}%
\pgfsetbuttcap%
\pgfsetroundjoin%
\definecolor{currentfill}{rgb}{0.000000,0.000000,0.000000}%
\pgfsetfillcolor{currentfill}%
\pgfsetlinewidth{0.501875pt}%
\definecolor{currentstroke}{rgb}{0.000000,0.000000,0.000000}%
\pgfsetstrokecolor{currentstroke}%
\pgfsetdash{}{0pt}%
\pgfsys@defobject{currentmarker}{\pgfqpoint{0.000000in}{-0.041667in}}{\pgfqpoint{0.000000in}{0.000000in}}{%
\pgfpathmoveto{\pgfqpoint{0.000000in}{0.000000in}}%
\pgfpathlineto{\pgfqpoint{0.000000in}{-0.041667in}}%
\pgfusepath{stroke,fill}%
}%
\begin{pgfscope}%
\pgfsys@transformshift{3.310977in}{3.227753in}%
\pgfsys@useobject{currentmarker}{}%
\end{pgfscope}%
\end{pgfscope}%
\begin{pgfscope}%
\pgfpathrectangle{\pgfqpoint{0.481681in}{1.080890in}}{\pgfqpoint{5.785672in}{2.146863in}}%
\pgfusepath{clip}%
\pgfsetrectcap%
\pgfsetroundjoin%
\pgfsetlinewidth{0.100375pt}%
\definecolor{currentstroke}{rgb}{0.501961,0.501961,0.501961}%
\pgfsetstrokecolor{currentstroke}%
\pgfsetdash{}{0pt}%
\pgfpathmoveto{\pgfqpoint{3.925050in}{1.080890in}}%
\pgfpathlineto{\pgfqpoint{3.925050in}{3.227753in}}%
\pgfusepath{stroke}%
\end{pgfscope}%
\begin{pgfscope}%
\pgfsetbuttcap%
\pgfsetroundjoin%
\definecolor{currentfill}{rgb}{0.000000,0.000000,0.000000}%
\pgfsetfillcolor{currentfill}%
\pgfsetlinewidth{0.501875pt}%
\definecolor{currentstroke}{rgb}{0.000000,0.000000,0.000000}%
\pgfsetstrokecolor{currentstroke}%
\pgfsetdash{}{0pt}%
\pgfsys@defobject{currentmarker}{\pgfqpoint{0.000000in}{0.000000in}}{\pgfqpoint{0.000000in}{0.041667in}}{%
\pgfpathmoveto{\pgfqpoint{0.000000in}{0.000000in}}%
\pgfpathlineto{\pgfqpoint{0.000000in}{0.041667in}}%
\pgfusepath{stroke,fill}%
}%
\begin{pgfscope}%
\pgfsys@transformshift{3.925050in}{1.080890in}%
\pgfsys@useobject{currentmarker}{}%
\end{pgfscope}%
\end{pgfscope}%
\begin{pgfscope}%
\pgfsetbuttcap%
\pgfsetroundjoin%
\definecolor{currentfill}{rgb}{0.000000,0.000000,0.000000}%
\pgfsetfillcolor{currentfill}%
\pgfsetlinewidth{0.501875pt}%
\definecolor{currentstroke}{rgb}{0.000000,0.000000,0.000000}%
\pgfsetstrokecolor{currentstroke}%
\pgfsetdash{}{0pt}%
\pgfsys@defobject{currentmarker}{\pgfqpoint{0.000000in}{-0.041667in}}{\pgfqpoint{0.000000in}{0.000000in}}{%
\pgfpathmoveto{\pgfqpoint{0.000000in}{0.000000in}}%
\pgfpathlineto{\pgfqpoint{0.000000in}{-0.041667in}}%
\pgfusepath{stroke,fill}%
}%
\begin{pgfscope}%
\pgfsys@transformshift{3.925050in}{3.227753in}%
\pgfsys@useobject{currentmarker}{}%
\end{pgfscope}%
\end{pgfscope}%
\begin{pgfscope}%
\pgfpathrectangle{\pgfqpoint{0.481681in}{1.080890in}}{\pgfqpoint{5.785672in}{2.146863in}}%
\pgfusepath{clip}%
\pgfsetrectcap%
\pgfsetroundjoin%
\pgfsetlinewidth{0.100375pt}%
\definecolor{currentstroke}{rgb}{0.501961,0.501961,0.501961}%
\pgfsetstrokecolor{currentstroke}%
\pgfsetdash{}{0pt}%
\pgfpathmoveto{\pgfqpoint{4.539123in}{1.080890in}}%
\pgfpathlineto{\pgfqpoint{4.539123in}{3.227753in}}%
\pgfusepath{stroke}%
\end{pgfscope}%
\begin{pgfscope}%
\pgfsetbuttcap%
\pgfsetroundjoin%
\definecolor{currentfill}{rgb}{0.000000,0.000000,0.000000}%
\pgfsetfillcolor{currentfill}%
\pgfsetlinewidth{0.501875pt}%
\definecolor{currentstroke}{rgb}{0.000000,0.000000,0.000000}%
\pgfsetstrokecolor{currentstroke}%
\pgfsetdash{}{0pt}%
\pgfsys@defobject{currentmarker}{\pgfqpoint{0.000000in}{0.000000in}}{\pgfqpoint{0.000000in}{0.041667in}}{%
\pgfpathmoveto{\pgfqpoint{0.000000in}{0.000000in}}%
\pgfpathlineto{\pgfqpoint{0.000000in}{0.041667in}}%
\pgfusepath{stroke,fill}%
}%
\begin{pgfscope}%
\pgfsys@transformshift{4.539123in}{1.080890in}%
\pgfsys@useobject{currentmarker}{}%
\end{pgfscope}%
\end{pgfscope}%
\begin{pgfscope}%
\pgfsetbuttcap%
\pgfsetroundjoin%
\definecolor{currentfill}{rgb}{0.000000,0.000000,0.000000}%
\pgfsetfillcolor{currentfill}%
\pgfsetlinewidth{0.501875pt}%
\definecolor{currentstroke}{rgb}{0.000000,0.000000,0.000000}%
\pgfsetstrokecolor{currentstroke}%
\pgfsetdash{}{0pt}%
\pgfsys@defobject{currentmarker}{\pgfqpoint{0.000000in}{-0.041667in}}{\pgfqpoint{0.000000in}{0.000000in}}{%
\pgfpathmoveto{\pgfqpoint{0.000000in}{0.000000in}}%
\pgfpathlineto{\pgfqpoint{0.000000in}{-0.041667in}}%
\pgfusepath{stroke,fill}%
}%
\begin{pgfscope}%
\pgfsys@transformshift{4.539123in}{3.227753in}%
\pgfsys@useobject{currentmarker}{}%
\end{pgfscope}%
\end{pgfscope}%
\begin{pgfscope}%
\pgfpathrectangle{\pgfqpoint{0.481681in}{1.080890in}}{\pgfqpoint{5.785672in}{2.146863in}}%
\pgfusepath{clip}%
\pgfsetrectcap%
\pgfsetroundjoin%
\pgfsetlinewidth{0.100375pt}%
\definecolor{currentstroke}{rgb}{0.501961,0.501961,0.501961}%
\pgfsetstrokecolor{currentstroke}%
\pgfsetdash{}{0pt}%
\pgfpathmoveto{\pgfqpoint{5.153195in}{1.080890in}}%
\pgfpathlineto{\pgfqpoint{5.153195in}{3.227753in}}%
\pgfusepath{stroke}%
\end{pgfscope}%
\begin{pgfscope}%
\pgfsetbuttcap%
\pgfsetroundjoin%
\definecolor{currentfill}{rgb}{0.000000,0.000000,0.000000}%
\pgfsetfillcolor{currentfill}%
\pgfsetlinewidth{0.501875pt}%
\definecolor{currentstroke}{rgb}{0.000000,0.000000,0.000000}%
\pgfsetstrokecolor{currentstroke}%
\pgfsetdash{}{0pt}%
\pgfsys@defobject{currentmarker}{\pgfqpoint{0.000000in}{0.000000in}}{\pgfqpoint{0.000000in}{0.041667in}}{%
\pgfpathmoveto{\pgfqpoint{0.000000in}{0.000000in}}%
\pgfpathlineto{\pgfqpoint{0.000000in}{0.041667in}}%
\pgfusepath{stroke,fill}%
}%
\begin{pgfscope}%
\pgfsys@transformshift{5.153195in}{1.080890in}%
\pgfsys@useobject{currentmarker}{}%
\end{pgfscope}%
\end{pgfscope}%
\begin{pgfscope}%
\pgfsetbuttcap%
\pgfsetroundjoin%
\definecolor{currentfill}{rgb}{0.000000,0.000000,0.000000}%
\pgfsetfillcolor{currentfill}%
\pgfsetlinewidth{0.501875pt}%
\definecolor{currentstroke}{rgb}{0.000000,0.000000,0.000000}%
\pgfsetstrokecolor{currentstroke}%
\pgfsetdash{}{0pt}%
\pgfsys@defobject{currentmarker}{\pgfqpoint{0.000000in}{-0.041667in}}{\pgfqpoint{0.000000in}{0.000000in}}{%
\pgfpathmoveto{\pgfqpoint{0.000000in}{0.000000in}}%
\pgfpathlineto{\pgfqpoint{0.000000in}{-0.041667in}}%
\pgfusepath{stroke,fill}%
}%
\begin{pgfscope}%
\pgfsys@transformshift{5.153195in}{3.227753in}%
\pgfsys@useobject{currentmarker}{}%
\end{pgfscope}%
\end{pgfscope}%
\begin{pgfscope}%
\pgfpathrectangle{\pgfqpoint{0.481681in}{1.080890in}}{\pgfqpoint{5.785672in}{2.146863in}}%
\pgfusepath{clip}%
\pgfsetrectcap%
\pgfsetroundjoin%
\pgfsetlinewidth{0.100375pt}%
\definecolor{currentstroke}{rgb}{0.501961,0.501961,0.501961}%
\pgfsetstrokecolor{currentstroke}%
\pgfsetdash{}{0pt}%
\pgfpathmoveto{\pgfqpoint{5.767268in}{1.080890in}}%
\pgfpathlineto{\pgfqpoint{5.767268in}{3.227753in}}%
\pgfusepath{stroke}%
\end{pgfscope}%
\begin{pgfscope}%
\pgfsetbuttcap%
\pgfsetroundjoin%
\definecolor{currentfill}{rgb}{0.000000,0.000000,0.000000}%
\pgfsetfillcolor{currentfill}%
\pgfsetlinewidth{0.501875pt}%
\definecolor{currentstroke}{rgb}{0.000000,0.000000,0.000000}%
\pgfsetstrokecolor{currentstroke}%
\pgfsetdash{}{0pt}%
\pgfsys@defobject{currentmarker}{\pgfqpoint{0.000000in}{0.000000in}}{\pgfqpoint{0.000000in}{0.041667in}}{%
\pgfpathmoveto{\pgfqpoint{0.000000in}{0.000000in}}%
\pgfpathlineto{\pgfqpoint{0.000000in}{0.041667in}}%
\pgfusepath{stroke,fill}%
}%
\begin{pgfscope}%
\pgfsys@transformshift{5.767268in}{1.080890in}%
\pgfsys@useobject{currentmarker}{}%
\end{pgfscope}%
\end{pgfscope}%
\begin{pgfscope}%
\pgfsetbuttcap%
\pgfsetroundjoin%
\definecolor{currentfill}{rgb}{0.000000,0.000000,0.000000}%
\pgfsetfillcolor{currentfill}%
\pgfsetlinewidth{0.501875pt}%
\definecolor{currentstroke}{rgb}{0.000000,0.000000,0.000000}%
\pgfsetstrokecolor{currentstroke}%
\pgfsetdash{}{0pt}%
\pgfsys@defobject{currentmarker}{\pgfqpoint{0.000000in}{-0.041667in}}{\pgfqpoint{0.000000in}{0.000000in}}{%
\pgfpathmoveto{\pgfqpoint{0.000000in}{0.000000in}}%
\pgfpathlineto{\pgfqpoint{0.000000in}{-0.041667in}}%
\pgfusepath{stroke,fill}%
}%
\begin{pgfscope}%
\pgfsys@transformshift{5.767268in}{3.227753in}%
\pgfsys@useobject{currentmarker}{}%
\end{pgfscope}%
\end{pgfscope}%
\begin{pgfscope}%
\pgfpathrectangle{\pgfqpoint{0.481681in}{1.080890in}}{\pgfqpoint{5.785672in}{2.146863in}}%
\pgfusepath{clip}%
\pgfsetrectcap%
\pgfsetroundjoin%
\pgfsetlinewidth{0.100375pt}%
\definecolor{currentstroke}{rgb}{0.827451,0.827451,0.827451}%
\pgfsetstrokecolor{currentstroke}%
\pgfsetdash{}{0pt}%
\pgfpathmoveto{\pgfqpoint{0.547651in}{1.080890in}}%
\pgfpathlineto{\pgfqpoint{0.547651in}{3.227753in}}%
\pgfusepath{stroke}%
\end{pgfscope}%
\begin{pgfscope}%
\pgfsetbuttcap%
\pgfsetroundjoin%
\definecolor{currentfill}{rgb}{0.000000,0.000000,0.000000}%
\pgfsetfillcolor{currentfill}%
\pgfsetlinewidth{0.501875pt}%
\definecolor{currentstroke}{rgb}{0.000000,0.000000,0.000000}%
\pgfsetstrokecolor{currentstroke}%
\pgfsetdash{}{0pt}%
\pgfsys@defobject{currentmarker}{\pgfqpoint{0.000000in}{0.000000in}}{\pgfqpoint{0.000000in}{0.020833in}}{%
\pgfpathmoveto{\pgfqpoint{0.000000in}{0.000000in}}%
\pgfpathlineto{\pgfqpoint{0.000000in}{0.020833in}}%
\pgfusepath{stroke,fill}%
}%
\begin{pgfscope}%
\pgfsys@transformshift{0.547651in}{1.080890in}%
\pgfsys@useobject{currentmarker}{}%
\end{pgfscope}%
\end{pgfscope}%
\begin{pgfscope}%
\pgfsetbuttcap%
\pgfsetroundjoin%
\definecolor{currentfill}{rgb}{0.000000,0.000000,0.000000}%
\pgfsetfillcolor{currentfill}%
\pgfsetlinewidth{0.501875pt}%
\definecolor{currentstroke}{rgb}{0.000000,0.000000,0.000000}%
\pgfsetstrokecolor{currentstroke}%
\pgfsetdash{}{0pt}%
\pgfsys@defobject{currentmarker}{\pgfqpoint{0.000000in}{-0.020833in}}{\pgfqpoint{0.000000in}{0.000000in}}{%
\pgfpathmoveto{\pgfqpoint{0.000000in}{0.000000in}}%
\pgfpathlineto{\pgfqpoint{0.000000in}{-0.020833in}}%
\pgfusepath{stroke,fill}%
}%
\begin{pgfscope}%
\pgfsys@transformshift{0.547651in}{3.227753in}%
\pgfsys@useobject{currentmarker}{}%
\end{pgfscope}%
\end{pgfscope}%
\begin{pgfscope}%
\pgfpathrectangle{\pgfqpoint{0.481681in}{1.080890in}}{\pgfqpoint{5.785672in}{2.146863in}}%
\pgfusepath{clip}%
\pgfsetrectcap%
\pgfsetroundjoin%
\pgfsetlinewidth{0.100375pt}%
\definecolor{currentstroke}{rgb}{0.827451,0.827451,0.827451}%
\pgfsetstrokecolor{currentstroke}%
\pgfsetdash{}{0pt}%
\pgfpathmoveto{\pgfqpoint{0.649996in}{1.080890in}}%
\pgfpathlineto{\pgfqpoint{0.649996in}{3.227753in}}%
\pgfusepath{stroke}%
\end{pgfscope}%
\begin{pgfscope}%
\pgfsetbuttcap%
\pgfsetroundjoin%
\definecolor{currentfill}{rgb}{0.000000,0.000000,0.000000}%
\pgfsetfillcolor{currentfill}%
\pgfsetlinewidth{0.501875pt}%
\definecolor{currentstroke}{rgb}{0.000000,0.000000,0.000000}%
\pgfsetstrokecolor{currentstroke}%
\pgfsetdash{}{0pt}%
\pgfsys@defobject{currentmarker}{\pgfqpoint{0.000000in}{0.000000in}}{\pgfqpoint{0.000000in}{0.020833in}}{%
\pgfpathmoveto{\pgfqpoint{0.000000in}{0.000000in}}%
\pgfpathlineto{\pgfqpoint{0.000000in}{0.020833in}}%
\pgfusepath{stroke,fill}%
}%
\begin{pgfscope}%
\pgfsys@transformshift{0.649996in}{1.080890in}%
\pgfsys@useobject{currentmarker}{}%
\end{pgfscope}%
\end{pgfscope}%
\begin{pgfscope}%
\pgfsetbuttcap%
\pgfsetroundjoin%
\definecolor{currentfill}{rgb}{0.000000,0.000000,0.000000}%
\pgfsetfillcolor{currentfill}%
\pgfsetlinewidth{0.501875pt}%
\definecolor{currentstroke}{rgb}{0.000000,0.000000,0.000000}%
\pgfsetstrokecolor{currentstroke}%
\pgfsetdash{}{0pt}%
\pgfsys@defobject{currentmarker}{\pgfqpoint{0.000000in}{-0.020833in}}{\pgfqpoint{0.000000in}{0.000000in}}{%
\pgfpathmoveto{\pgfqpoint{0.000000in}{0.000000in}}%
\pgfpathlineto{\pgfqpoint{0.000000in}{-0.020833in}}%
\pgfusepath{stroke,fill}%
}%
\begin{pgfscope}%
\pgfsys@transformshift{0.649996in}{3.227753in}%
\pgfsys@useobject{currentmarker}{}%
\end{pgfscope}%
\end{pgfscope}%
\begin{pgfscope}%
\pgfpathrectangle{\pgfqpoint{0.481681in}{1.080890in}}{\pgfqpoint{5.785672in}{2.146863in}}%
\pgfusepath{clip}%
\pgfsetrectcap%
\pgfsetroundjoin%
\pgfsetlinewidth{0.100375pt}%
\definecolor{currentstroke}{rgb}{0.827451,0.827451,0.827451}%
\pgfsetstrokecolor{currentstroke}%
\pgfsetdash{}{0pt}%
\pgfpathmoveto{\pgfqpoint{0.752342in}{1.080890in}}%
\pgfpathlineto{\pgfqpoint{0.752342in}{3.227753in}}%
\pgfusepath{stroke}%
\end{pgfscope}%
\begin{pgfscope}%
\pgfsetbuttcap%
\pgfsetroundjoin%
\definecolor{currentfill}{rgb}{0.000000,0.000000,0.000000}%
\pgfsetfillcolor{currentfill}%
\pgfsetlinewidth{0.501875pt}%
\definecolor{currentstroke}{rgb}{0.000000,0.000000,0.000000}%
\pgfsetstrokecolor{currentstroke}%
\pgfsetdash{}{0pt}%
\pgfsys@defobject{currentmarker}{\pgfqpoint{0.000000in}{0.000000in}}{\pgfqpoint{0.000000in}{0.020833in}}{%
\pgfpathmoveto{\pgfqpoint{0.000000in}{0.000000in}}%
\pgfpathlineto{\pgfqpoint{0.000000in}{0.020833in}}%
\pgfusepath{stroke,fill}%
}%
\begin{pgfscope}%
\pgfsys@transformshift{0.752342in}{1.080890in}%
\pgfsys@useobject{currentmarker}{}%
\end{pgfscope}%
\end{pgfscope}%
\begin{pgfscope}%
\pgfsetbuttcap%
\pgfsetroundjoin%
\definecolor{currentfill}{rgb}{0.000000,0.000000,0.000000}%
\pgfsetfillcolor{currentfill}%
\pgfsetlinewidth{0.501875pt}%
\definecolor{currentstroke}{rgb}{0.000000,0.000000,0.000000}%
\pgfsetstrokecolor{currentstroke}%
\pgfsetdash{}{0pt}%
\pgfsys@defobject{currentmarker}{\pgfqpoint{0.000000in}{-0.020833in}}{\pgfqpoint{0.000000in}{0.000000in}}{%
\pgfpathmoveto{\pgfqpoint{0.000000in}{0.000000in}}%
\pgfpathlineto{\pgfqpoint{0.000000in}{-0.020833in}}%
\pgfusepath{stroke,fill}%
}%
\begin{pgfscope}%
\pgfsys@transformshift{0.752342in}{3.227753in}%
\pgfsys@useobject{currentmarker}{}%
\end{pgfscope}%
\end{pgfscope}%
\begin{pgfscope}%
\pgfpathrectangle{\pgfqpoint{0.481681in}{1.080890in}}{\pgfqpoint{5.785672in}{2.146863in}}%
\pgfusepath{clip}%
\pgfsetrectcap%
\pgfsetroundjoin%
\pgfsetlinewidth{0.100375pt}%
\definecolor{currentstroke}{rgb}{0.827451,0.827451,0.827451}%
\pgfsetstrokecolor{currentstroke}%
\pgfsetdash{}{0pt}%
\pgfpathmoveto{\pgfqpoint{0.957033in}{1.080890in}}%
\pgfpathlineto{\pgfqpoint{0.957033in}{3.227753in}}%
\pgfusepath{stroke}%
\end{pgfscope}%
\begin{pgfscope}%
\pgfsetbuttcap%
\pgfsetroundjoin%
\definecolor{currentfill}{rgb}{0.000000,0.000000,0.000000}%
\pgfsetfillcolor{currentfill}%
\pgfsetlinewidth{0.501875pt}%
\definecolor{currentstroke}{rgb}{0.000000,0.000000,0.000000}%
\pgfsetstrokecolor{currentstroke}%
\pgfsetdash{}{0pt}%
\pgfsys@defobject{currentmarker}{\pgfqpoint{0.000000in}{0.000000in}}{\pgfqpoint{0.000000in}{0.020833in}}{%
\pgfpathmoveto{\pgfqpoint{0.000000in}{0.000000in}}%
\pgfpathlineto{\pgfqpoint{0.000000in}{0.020833in}}%
\pgfusepath{stroke,fill}%
}%
\begin{pgfscope}%
\pgfsys@transformshift{0.957033in}{1.080890in}%
\pgfsys@useobject{currentmarker}{}%
\end{pgfscope}%
\end{pgfscope}%
\begin{pgfscope}%
\pgfsetbuttcap%
\pgfsetroundjoin%
\definecolor{currentfill}{rgb}{0.000000,0.000000,0.000000}%
\pgfsetfillcolor{currentfill}%
\pgfsetlinewidth{0.501875pt}%
\definecolor{currentstroke}{rgb}{0.000000,0.000000,0.000000}%
\pgfsetstrokecolor{currentstroke}%
\pgfsetdash{}{0pt}%
\pgfsys@defobject{currentmarker}{\pgfqpoint{0.000000in}{-0.020833in}}{\pgfqpoint{0.000000in}{0.000000in}}{%
\pgfpathmoveto{\pgfqpoint{0.000000in}{0.000000in}}%
\pgfpathlineto{\pgfqpoint{0.000000in}{-0.020833in}}%
\pgfusepath{stroke,fill}%
}%
\begin{pgfscope}%
\pgfsys@transformshift{0.957033in}{3.227753in}%
\pgfsys@useobject{currentmarker}{}%
\end{pgfscope}%
\end{pgfscope}%
\begin{pgfscope}%
\pgfpathrectangle{\pgfqpoint{0.481681in}{1.080890in}}{\pgfqpoint{5.785672in}{2.146863in}}%
\pgfusepath{clip}%
\pgfsetrectcap%
\pgfsetroundjoin%
\pgfsetlinewidth{0.100375pt}%
\definecolor{currentstroke}{rgb}{0.827451,0.827451,0.827451}%
\pgfsetstrokecolor{currentstroke}%
\pgfsetdash{}{0pt}%
\pgfpathmoveto{\pgfqpoint{1.059378in}{1.080890in}}%
\pgfpathlineto{\pgfqpoint{1.059378in}{3.227753in}}%
\pgfusepath{stroke}%
\end{pgfscope}%
\begin{pgfscope}%
\pgfsetbuttcap%
\pgfsetroundjoin%
\definecolor{currentfill}{rgb}{0.000000,0.000000,0.000000}%
\pgfsetfillcolor{currentfill}%
\pgfsetlinewidth{0.501875pt}%
\definecolor{currentstroke}{rgb}{0.000000,0.000000,0.000000}%
\pgfsetstrokecolor{currentstroke}%
\pgfsetdash{}{0pt}%
\pgfsys@defobject{currentmarker}{\pgfqpoint{0.000000in}{0.000000in}}{\pgfqpoint{0.000000in}{0.020833in}}{%
\pgfpathmoveto{\pgfqpoint{0.000000in}{0.000000in}}%
\pgfpathlineto{\pgfqpoint{0.000000in}{0.020833in}}%
\pgfusepath{stroke,fill}%
}%
\begin{pgfscope}%
\pgfsys@transformshift{1.059378in}{1.080890in}%
\pgfsys@useobject{currentmarker}{}%
\end{pgfscope}%
\end{pgfscope}%
\begin{pgfscope}%
\pgfsetbuttcap%
\pgfsetroundjoin%
\definecolor{currentfill}{rgb}{0.000000,0.000000,0.000000}%
\pgfsetfillcolor{currentfill}%
\pgfsetlinewidth{0.501875pt}%
\definecolor{currentstroke}{rgb}{0.000000,0.000000,0.000000}%
\pgfsetstrokecolor{currentstroke}%
\pgfsetdash{}{0pt}%
\pgfsys@defobject{currentmarker}{\pgfqpoint{0.000000in}{-0.020833in}}{\pgfqpoint{0.000000in}{0.000000in}}{%
\pgfpathmoveto{\pgfqpoint{0.000000in}{0.000000in}}%
\pgfpathlineto{\pgfqpoint{0.000000in}{-0.020833in}}%
\pgfusepath{stroke,fill}%
}%
\begin{pgfscope}%
\pgfsys@transformshift{1.059378in}{3.227753in}%
\pgfsys@useobject{currentmarker}{}%
\end{pgfscope}%
\end{pgfscope}%
\begin{pgfscope}%
\pgfpathrectangle{\pgfqpoint{0.481681in}{1.080890in}}{\pgfqpoint{5.785672in}{2.146863in}}%
\pgfusepath{clip}%
\pgfsetrectcap%
\pgfsetroundjoin%
\pgfsetlinewidth{0.100375pt}%
\definecolor{currentstroke}{rgb}{0.827451,0.827451,0.827451}%
\pgfsetstrokecolor{currentstroke}%
\pgfsetdash{}{0pt}%
\pgfpathmoveto{\pgfqpoint{1.161724in}{1.080890in}}%
\pgfpathlineto{\pgfqpoint{1.161724in}{3.227753in}}%
\pgfusepath{stroke}%
\end{pgfscope}%
\begin{pgfscope}%
\pgfsetbuttcap%
\pgfsetroundjoin%
\definecolor{currentfill}{rgb}{0.000000,0.000000,0.000000}%
\pgfsetfillcolor{currentfill}%
\pgfsetlinewidth{0.501875pt}%
\definecolor{currentstroke}{rgb}{0.000000,0.000000,0.000000}%
\pgfsetstrokecolor{currentstroke}%
\pgfsetdash{}{0pt}%
\pgfsys@defobject{currentmarker}{\pgfqpoint{0.000000in}{0.000000in}}{\pgfqpoint{0.000000in}{0.020833in}}{%
\pgfpathmoveto{\pgfqpoint{0.000000in}{0.000000in}}%
\pgfpathlineto{\pgfqpoint{0.000000in}{0.020833in}}%
\pgfusepath{stroke,fill}%
}%
\begin{pgfscope}%
\pgfsys@transformshift{1.161724in}{1.080890in}%
\pgfsys@useobject{currentmarker}{}%
\end{pgfscope}%
\end{pgfscope}%
\begin{pgfscope}%
\pgfsetbuttcap%
\pgfsetroundjoin%
\definecolor{currentfill}{rgb}{0.000000,0.000000,0.000000}%
\pgfsetfillcolor{currentfill}%
\pgfsetlinewidth{0.501875pt}%
\definecolor{currentstroke}{rgb}{0.000000,0.000000,0.000000}%
\pgfsetstrokecolor{currentstroke}%
\pgfsetdash{}{0pt}%
\pgfsys@defobject{currentmarker}{\pgfqpoint{0.000000in}{-0.020833in}}{\pgfqpoint{0.000000in}{0.000000in}}{%
\pgfpathmoveto{\pgfqpoint{0.000000in}{0.000000in}}%
\pgfpathlineto{\pgfqpoint{0.000000in}{-0.020833in}}%
\pgfusepath{stroke,fill}%
}%
\begin{pgfscope}%
\pgfsys@transformshift{1.161724in}{3.227753in}%
\pgfsys@useobject{currentmarker}{}%
\end{pgfscope}%
\end{pgfscope}%
\begin{pgfscope}%
\pgfpathrectangle{\pgfqpoint{0.481681in}{1.080890in}}{\pgfqpoint{5.785672in}{2.146863in}}%
\pgfusepath{clip}%
\pgfsetrectcap%
\pgfsetroundjoin%
\pgfsetlinewidth{0.100375pt}%
\definecolor{currentstroke}{rgb}{0.827451,0.827451,0.827451}%
\pgfsetstrokecolor{currentstroke}%
\pgfsetdash{}{0pt}%
\pgfpathmoveto{\pgfqpoint{1.264069in}{1.080890in}}%
\pgfpathlineto{\pgfqpoint{1.264069in}{3.227753in}}%
\pgfusepath{stroke}%
\end{pgfscope}%
\begin{pgfscope}%
\pgfsetbuttcap%
\pgfsetroundjoin%
\definecolor{currentfill}{rgb}{0.000000,0.000000,0.000000}%
\pgfsetfillcolor{currentfill}%
\pgfsetlinewidth{0.501875pt}%
\definecolor{currentstroke}{rgb}{0.000000,0.000000,0.000000}%
\pgfsetstrokecolor{currentstroke}%
\pgfsetdash{}{0pt}%
\pgfsys@defobject{currentmarker}{\pgfqpoint{0.000000in}{0.000000in}}{\pgfqpoint{0.000000in}{0.020833in}}{%
\pgfpathmoveto{\pgfqpoint{0.000000in}{0.000000in}}%
\pgfpathlineto{\pgfqpoint{0.000000in}{0.020833in}}%
\pgfusepath{stroke,fill}%
}%
\begin{pgfscope}%
\pgfsys@transformshift{1.264069in}{1.080890in}%
\pgfsys@useobject{currentmarker}{}%
\end{pgfscope}%
\end{pgfscope}%
\begin{pgfscope}%
\pgfsetbuttcap%
\pgfsetroundjoin%
\definecolor{currentfill}{rgb}{0.000000,0.000000,0.000000}%
\pgfsetfillcolor{currentfill}%
\pgfsetlinewidth{0.501875pt}%
\definecolor{currentstroke}{rgb}{0.000000,0.000000,0.000000}%
\pgfsetstrokecolor{currentstroke}%
\pgfsetdash{}{0pt}%
\pgfsys@defobject{currentmarker}{\pgfqpoint{0.000000in}{-0.020833in}}{\pgfqpoint{0.000000in}{0.000000in}}{%
\pgfpathmoveto{\pgfqpoint{0.000000in}{0.000000in}}%
\pgfpathlineto{\pgfqpoint{0.000000in}{-0.020833in}}%
\pgfusepath{stroke,fill}%
}%
\begin{pgfscope}%
\pgfsys@transformshift{1.264069in}{3.227753in}%
\pgfsys@useobject{currentmarker}{}%
\end{pgfscope}%
\end{pgfscope}%
\begin{pgfscope}%
\pgfpathrectangle{\pgfqpoint{0.481681in}{1.080890in}}{\pgfqpoint{5.785672in}{2.146863in}}%
\pgfusepath{clip}%
\pgfsetrectcap%
\pgfsetroundjoin%
\pgfsetlinewidth{0.100375pt}%
\definecolor{currentstroke}{rgb}{0.827451,0.827451,0.827451}%
\pgfsetstrokecolor{currentstroke}%
\pgfsetdash{}{0pt}%
\pgfpathmoveto{\pgfqpoint{1.366414in}{1.080890in}}%
\pgfpathlineto{\pgfqpoint{1.366414in}{3.227753in}}%
\pgfusepath{stroke}%
\end{pgfscope}%
\begin{pgfscope}%
\pgfsetbuttcap%
\pgfsetroundjoin%
\definecolor{currentfill}{rgb}{0.000000,0.000000,0.000000}%
\pgfsetfillcolor{currentfill}%
\pgfsetlinewidth{0.501875pt}%
\definecolor{currentstroke}{rgb}{0.000000,0.000000,0.000000}%
\pgfsetstrokecolor{currentstroke}%
\pgfsetdash{}{0pt}%
\pgfsys@defobject{currentmarker}{\pgfqpoint{0.000000in}{0.000000in}}{\pgfqpoint{0.000000in}{0.020833in}}{%
\pgfpathmoveto{\pgfqpoint{0.000000in}{0.000000in}}%
\pgfpathlineto{\pgfqpoint{0.000000in}{0.020833in}}%
\pgfusepath{stroke,fill}%
}%
\begin{pgfscope}%
\pgfsys@transformshift{1.366414in}{1.080890in}%
\pgfsys@useobject{currentmarker}{}%
\end{pgfscope}%
\end{pgfscope}%
\begin{pgfscope}%
\pgfsetbuttcap%
\pgfsetroundjoin%
\definecolor{currentfill}{rgb}{0.000000,0.000000,0.000000}%
\pgfsetfillcolor{currentfill}%
\pgfsetlinewidth{0.501875pt}%
\definecolor{currentstroke}{rgb}{0.000000,0.000000,0.000000}%
\pgfsetstrokecolor{currentstroke}%
\pgfsetdash{}{0pt}%
\pgfsys@defobject{currentmarker}{\pgfqpoint{0.000000in}{-0.020833in}}{\pgfqpoint{0.000000in}{0.000000in}}{%
\pgfpathmoveto{\pgfqpoint{0.000000in}{0.000000in}}%
\pgfpathlineto{\pgfqpoint{0.000000in}{-0.020833in}}%
\pgfusepath{stroke,fill}%
}%
\begin{pgfscope}%
\pgfsys@transformshift{1.366414in}{3.227753in}%
\pgfsys@useobject{currentmarker}{}%
\end{pgfscope}%
\end{pgfscope}%
\begin{pgfscope}%
\pgfpathrectangle{\pgfqpoint{0.481681in}{1.080890in}}{\pgfqpoint{5.785672in}{2.146863in}}%
\pgfusepath{clip}%
\pgfsetrectcap%
\pgfsetroundjoin%
\pgfsetlinewidth{0.100375pt}%
\definecolor{currentstroke}{rgb}{0.827451,0.827451,0.827451}%
\pgfsetstrokecolor{currentstroke}%
\pgfsetdash{}{0pt}%
\pgfpathmoveto{\pgfqpoint{1.571105in}{1.080890in}}%
\pgfpathlineto{\pgfqpoint{1.571105in}{3.227753in}}%
\pgfusepath{stroke}%
\end{pgfscope}%
\begin{pgfscope}%
\pgfsetbuttcap%
\pgfsetroundjoin%
\definecolor{currentfill}{rgb}{0.000000,0.000000,0.000000}%
\pgfsetfillcolor{currentfill}%
\pgfsetlinewidth{0.501875pt}%
\definecolor{currentstroke}{rgb}{0.000000,0.000000,0.000000}%
\pgfsetstrokecolor{currentstroke}%
\pgfsetdash{}{0pt}%
\pgfsys@defobject{currentmarker}{\pgfqpoint{0.000000in}{0.000000in}}{\pgfqpoint{0.000000in}{0.020833in}}{%
\pgfpathmoveto{\pgfqpoint{0.000000in}{0.000000in}}%
\pgfpathlineto{\pgfqpoint{0.000000in}{0.020833in}}%
\pgfusepath{stroke,fill}%
}%
\begin{pgfscope}%
\pgfsys@transformshift{1.571105in}{1.080890in}%
\pgfsys@useobject{currentmarker}{}%
\end{pgfscope}%
\end{pgfscope}%
\begin{pgfscope}%
\pgfsetbuttcap%
\pgfsetroundjoin%
\definecolor{currentfill}{rgb}{0.000000,0.000000,0.000000}%
\pgfsetfillcolor{currentfill}%
\pgfsetlinewidth{0.501875pt}%
\definecolor{currentstroke}{rgb}{0.000000,0.000000,0.000000}%
\pgfsetstrokecolor{currentstroke}%
\pgfsetdash{}{0pt}%
\pgfsys@defobject{currentmarker}{\pgfqpoint{0.000000in}{-0.020833in}}{\pgfqpoint{0.000000in}{0.000000in}}{%
\pgfpathmoveto{\pgfqpoint{0.000000in}{0.000000in}}%
\pgfpathlineto{\pgfqpoint{0.000000in}{-0.020833in}}%
\pgfusepath{stroke,fill}%
}%
\begin{pgfscope}%
\pgfsys@transformshift{1.571105in}{3.227753in}%
\pgfsys@useobject{currentmarker}{}%
\end{pgfscope}%
\end{pgfscope}%
\begin{pgfscope}%
\pgfpathrectangle{\pgfqpoint{0.481681in}{1.080890in}}{\pgfqpoint{5.785672in}{2.146863in}}%
\pgfusepath{clip}%
\pgfsetrectcap%
\pgfsetroundjoin%
\pgfsetlinewidth{0.100375pt}%
\definecolor{currentstroke}{rgb}{0.827451,0.827451,0.827451}%
\pgfsetstrokecolor{currentstroke}%
\pgfsetdash{}{0pt}%
\pgfpathmoveto{\pgfqpoint{1.673451in}{1.080890in}}%
\pgfpathlineto{\pgfqpoint{1.673451in}{3.227753in}}%
\pgfusepath{stroke}%
\end{pgfscope}%
\begin{pgfscope}%
\pgfsetbuttcap%
\pgfsetroundjoin%
\definecolor{currentfill}{rgb}{0.000000,0.000000,0.000000}%
\pgfsetfillcolor{currentfill}%
\pgfsetlinewidth{0.501875pt}%
\definecolor{currentstroke}{rgb}{0.000000,0.000000,0.000000}%
\pgfsetstrokecolor{currentstroke}%
\pgfsetdash{}{0pt}%
\pgfsys@defobject{currentmarker}{\pgfqpoint{0.000000in}{0.000000in}}{\pgfqpoint{0.000000in}{0.020833in}}{%
\pgfpathmoveto{\pgfqpoint{0.000000in}{0.000000in}}%
\pgfpathlineto{\pgfqpoint{0.000000in}{0.020833in}}%
\pgfusepath{stroke,fill}%
}%
\begin{pgfscope}%
\pgfsys@transformshift{1.673451in}{1.080890in}%
\pgfsys@useobject{currentmarker}{}%
\end{pgfscope}%
\end{pgfscope}%
\begin{pgfscope}%
\pgfsetbuttcap%
\pgfsetroundjoin%
\definecolor{currentfill}{rgb}{0.000000,0.000000,0.000000}%
\pgfsetfillcolor{currentfill}%
\pgfsetlinewidth{0.501875pt}%
\definecolor{currentstroke}{rgb}{0.000000,0.000000,0.000000}%
\pgfsetstrokecolor{currentstroke}%
\pgfsetdash{}{0pt}%
\pgfsys@defobject{currentmarker}{\pgfqpoint{0.000000in}{-0.020833in}}{\pgfqpoint{0.000000in}{0.000000in}}{%
\pgfpathmoveto{\pgfqpoint{0.000000in}{0.000000in}}%
\pgfpathlineto{\pgfqpoint{0.000000in}{-0.020833in}}%
\pgfusepath{stroke,fill}%
}%
\begin{pgfscope}%
\pgfsys@transformshift{1.673451in}{3.227753in}%
\pgfsys@useobject{currentmarker}{}%
\end{pgfscope}%
\end{pgfscope}%
\begin{pgfscope}%
\pgfpathrectangle{\pgfqpoint{0.481681in}{1.080890in}}{\pgfqpoint{5.785672in}{2.146863in}}%
\pgfusepath{clip}%
\pgfsetrectcap%
\pgfsetroundjoin%
\pgfsetlinewidth{0.100375pt}%
\definecolor{currentstroke}{rgb}{0.827451,0.827451,0.827451}%
\pgfsetstrokecolor{currentstroke}%
\pgfsetdash{}{0pt}%
\pgfpathmoveto{\pgfqpoint{1.775796in}{1.080890in}}%
\pgfpathlineto{\pgfqpoint{1.775796in}{3.227753in}}%
\pgfusepath{stroke}%
\end{pgfscope}%
\begin{pgfscope}%
\pgfsetbuttcap%
\pgfsetroundjoin%
\definecolor{currentfill}{rgb}{0.000000,0.000000,0.000000}%
\pgfsetfillcolor{currentfill}%
\pgfsetlinewidth{0.501875pt}%
\definecolor{currentstroke}{rgb}{0.000000,0.000000,0.000000}%
\pgfsetstrokecolor{currentstroke}%
\pgfsetdash{}{0pt}%
\pgfsys@defobject{currentmarker}{\pgfqpoint{0.000000in}{0.000000in}}{\pgfqpoint{0.000000in}{0.020833in}}{%
\pgfpathmoveto{\pgfqpoint{0.000000in}{0.000000in}}%
\pgfpathlineto{\pgfqpoint{0.000000in}{0.020833in}}%
\pgfusepath{stroke,fill}%
}%
\begin{pgfscope}%
\pgfsys@transformshift{1.775796in}{1.080890in}%
\pgfsys@useobject{currentmarker}{}%
\end{pgfscope}%
\end{pgfscope}%
\begin{pgfscope}%
\pgfsetbuttcap%
\pgfsetroundjoin%
\definecolor{currentfill}{rgb}{0.000000,0.000000,0.000000}%
\pgfsetfillcolor{currentfill}%
\pgfsetlinewidth{0.501875pt}%
\definecolor{currentstroke}{rgb}{0.000000,0.000000,0.000000}%
\pgfsetstrokecolor{currentstroke}%
\pgfsetdash{}{0pt}%
\pgfsys@defobject{currentmarker}{\pgfqpoint{0.000000in}{-0.020833in}}{\pgfqpoint{0.000000in}{0.000000in}}{%
\pgfpathmoveto{\pgfqpoint{0.000000in}{0.000000in}}%
\pgfpathlineto{\pgfqpoint{0.000000in}{-0.020833in}}%
\pgfusepath{stroke,fill}%
}%
\begin{pgfscope}%
\pgfsys@transformshift{1.775796in}{3.227753in}%
\pgfsys@useobject{currentmarker}{}%
\end{pgfscope}%
\end{pgfscope}%
\begin{pgfscope}%
\pgfpathrectangle{\pgfqpoint{0.481681in}{1.080890in}}{\pgfqpoint{5.785672in}{2.146863in}}%
\pgfusepath{clip}%
\pgfsetrectcap%
\pgfsetroundjoin%
\pgfsetlinewidth{0.100375pt}%
\definecolor{currentstroke}{rgb}{0.827451,0.827451,0.827451}%
\pgfsetstrokecolor{currentstroke}%
\pgfsetdash{}{0pt}%
\pgfpathmoveto{\pgfqpoint{1.878142in}{1.080890in}}%
\pgfpathlineto{\pgfqpoint{1.878142in}{3.227753in}}%
\pgfusepath{stroke}%
\end{pgfscope}%
\begin{pgfscope}%
\pgfsetbuttcap%
\pgfsetroundjoin%
\definecolor{currentfill}{rgb}{0.000000,0.000000,0.000000}%
\pgfsetfillcolor{currentfill}%
\pgfsetlinewidth{0.501875pt}%
\definecolor{currentstroke}{rgb}{0.000000,0.000000,0.000000}%
\pgfsetstrokecolor{currentstroke}%
\pgfsetdash{}{0pt}%
\pgfsys@defobject{currentmarker}{\pgfqpoint{0.000000in}{0.000000in}}{\pgfqpoint{0.000000in}{0.020833in}}{%
\pgfpathmoveto{\pgfqpoint{0.000000in}{0.000000in}}%
\pgfpathlineto{\pgfqpoint{0.000000in}{0.020833in}}%
\pgfusepath{stroke,fill}%
}%
\begin{pgfscope}%
\pgfsys@transformshift{1.878142in}{1.080890in}%
\pgfsys@useobject{currentmarker}{}%
\end{pgfscope}%
\end{pgfscope}%
\begin{pgfscope}%
\pgfsetbuttcap%
\pgfsetroundjoin%
\definecolor{currentfill}{rgb}{0.000000,0.000000,0.000000}%
\pgfsetfillcolor{currentfill}%
\pgfsetlinewidth{0.501875pt}%
\definecolor{currentstroke}{rgb}{0.000000,0.000000,0.000000}%
\pgfsetstrokecolor{currentstroke}%
\pgfsetdash{}{0pt}%
\pgfsys@defobject{currentmarker}{\pgfqpoint{0.000000in}{-0.020833in}}{\pgfqpoint{0.000000in}{0.000000in}}{%
\pgfpathmoveto{\pgfqpoint{0.000000in}{0.000000in}}%
\pgfpathlineto{\pgfqpoint{0.000000in}{-0.020833in}}%
\pgfusepath{stroke,fill}%
}%
\begin{pgfscope}%
\pgfsys@transformshift{1.878142in}{3.227753in}%
\pgfsys@useobject{currentmarker}{}%
\end{pgfscope}%
\end{pgfscope}%
\begin{pgfscope}%
\pgfpathrectangle{\pgfqpoint{0.481681in}{1.080890in}}{\pgfqpoint{5.785672in}{2.146863in}}%
\pgfusepath{clip}%
\pgfsetrectcap%
\pgfsetroundjoin%
\pgfsetlinewidth{0.100375pt}%
\definecolor{currentstroke}{rgb}{0.827451,0.827451,0.827451}%
\pgfsetstrokecolor{currentstroke}%
\pgfsetdash{}{0pt}%
\pgfpathmoveto{\pgfqpoint{1.980487in}{1.080890in}}%
\pgfpathlineto{\pgfqpoint{1.980487in}{3.227753in}}%
\pgfusepath{stroke}%
\end{pgfscope}%
\begin{pgfscope}%
\pgfsetbuttcap%
\pgfsetroundjoin%
\definecolor{currentfill}{rgb}{0.000000,0.000000,0.000000}%
\pgfsetfillcolor{currentfill}%
\pgfsetlinewidth{0.501875pt}%
\definecolor{currentstroke}{rgb}{0.000000,0.000000,0.000000}%
\pgfsetstrokecolor{currentstroke}%
\pgfsetdash{}{0pt}%
\pgfsys@defobject{currentmarker}{\pgfqpoint{0.000000in}{0.000000in}}{\pgfqpoint{0.000000in}{0.020833in}}{%
\pgfpathmoveto{\pgfqpoint{0.000000in}{0.000000in}}%
\pgfpathlineto{\pgfqpoint{0.000000in}{0.020833in}}%
\pgfusepath{stroke,fill}%
}%
\begin{pgfscope}%
\pgfsys@transformshift{1.980487in}{1.080890in}%
\pgfsys@useobject{currentmarker}{}%
\end{pgfscope}%
\end{pgfscope}%
\begin{pgfscope}%
\pgfsetbuttcap%
\pgfsetroundjoin%
\definecolor{currentfill}{rgb}{0.000000,0.000000,0.000000}%
\pgfsetfillcolor{currentfill}%
\pgfsetlinewidth{0.501875pt}%
\definecolor{currentstroke}{rgb}{0.000000,0.000000,0.000000}%
\pgfsetstrokecolor{currentstroke}%
\pgfsetdash{}{0pt}%
\pgfsys@defobject{currentmarker}{\pgfqpoint{0.000000in}{-0.020833in}}{\pgfqpoint{0.000000in}{0.000000in}}{%
\pgfpathmoveto{\pgfqpoint{0.000000in}{0.000000in}}%
\pgfpathlineto{\pgfqpoint{0.000000in}{-0.020833in}}%
\pgfusepath{stroke,fill}%
}%
\begin{pgfscope}%
\pgfsys@transformshift{1.980487in}{3.227753in}%
\pgfsys@useobject{currentmarker}{}%
\end{pgfscope}%
\end{pgfscope}%
\begin{pgfscope}%
\pgfpathrectangle{\pgfqpoint{0.481681in}{1.080890in}}{\pgfqpoint{5.785672in}{2.146863in}}%
\pgfusepath{clip}%
\pgfsetrectcap%
\pgfsetroundjoin%
\pgfsetlinewidth{0.100375pt}%
\definecolor{currentstroke}{rgb}{0.827451,0.827451,0.827451}%
\pgfsetstrokecolor{currentstroke}%
\pgfsetdash{}{0pt}%
\pgfpathmoveto{\pgfqpoint{2.185178in}{1.080890in}}%
\pgfpathlineto{\pgfqpoint{2.185178in}{3.227753in}}%
\pgfusepath{stroke}%
\end{pgfscope}%
\begin{pgfscope}%
\pgfsetbuttcap%
\pgfsetroundjoin%
\definecolor{currentfill}{rgb}{0.000000,0.000000,0.000000}%
\pgfsetfillcolor{currentfill}%
\pgfsetlinewidth{0.501875pt}%
\definecolor{currentstroke}{rgb}{0.000000,0.000000,0.000000}%
\pgfsetstrokecolor{currentstroke}%
\pgfsetdash{}{0pt}%
\pgfsys@defobject{currentmarker}{\pgfqpoint{0.000000in}{0.000000in}}{\pgfqpoint{0.000000in}{0.020833in}}{%
\pgfpathmoveto{\pgfqpoint{0.000000in}{0.000000in}}%
\pgfpathlineto{\pgfqpoint{0.000000in}{0.020833in}}%
\pgfusepath{stroke,fill}%
}%
\begin{pgfscope}%
\pgfsys@transformshift{2.185178in}{1.080890in}%
\pgfsys@useobject{currentmarker}{}%
\end{pgfscope}%
\end{pgfscope}%
\begin{pgfscope}%
\pgfsetbuttcap%
\pgfsetroundjoin%
\definecolor{currentfill}{rgb}{0.000000,0.000000,0.000000}%
\pgfsetfillcolor{currentfill}%
\pgfsetlinewidth{0.501875pt}%
\definecolor{currentstroke}{rgb}{0.000000,0.000000,0.000000}%
\pgfsetstrokecolor{currentstroke}%
\pgfsetdash{}{0pt}%
\pgfsys@defobject{currentmarker}{\pgfqpoint{0.000000in}{-0.020833in}}{\pgfqpoint{0.000000in}{0.000000in}}{%
\pgfpathmoveto{\pgfqpoint{0.000000in}{0.000000in}}%
\pgfpathlineto{\pgfqpoint{0.000000in}{-0.020833in}}%
\pgfusepath{stroke,fill}%
}%
\begin{pgfscope}%
\pgfsys@transformshift{2.185178in}{3.227753in}%
\pgfsys@useobject{currentmarker}{}%
\end{pgfscope}%
\end{pgfscope}%
\begin{pgfscope}%
\pgfpathrectangle{\pgfqpoint{0.481681in}{1.080890in}}{\pgfqpoint{5.785672in}{2.146863in}}%
\pgfusepath{clip}%
\pgfsetrectcap%
\pgfsetroundjoin%
\pgfsetlinewidth{0.100375pt}%
\definecolor{currentstroke}{rgb}{0.827451,0.827451,0.827451}%
\pgfsetstrokecolor{currentstroke}%
\pgfsetdash{}{0pt}%
\pgfpathmoveto{\pgfqpoint{2.287523in}{1.080890in}}%
\pgfpathlineto{\pgfqpoint{2.287523in}{3.227753in}}%
\pgfusepath{stroke}%
\end{pgfscope}%
\begin{pgfscope}%
\pgfsetbuttcap%
\pgfsetroundjoin%
\definecolor{currentfill}{rgb}{0.000000,0.000000,0.000000}%
\pgfsetfillcolor{currentfill}%
\pgfsetlinewidth{0.501875pt}%
\definecolor{currentstroke}{rgb}{0.000000,0.000000,0.000000}%
\pgfsetstrokecolor{currentstroke}%
\pgfsetdash{}{0pt}%
\pgfsys@defobject{currentmarker}{\pgfqpoint{0.000000in}{0.000000in}}{\pgfqpoint{0.000000in}{0.020833in}}{%
\pgfpathmoveto{\pgfqpoint{0.000000in}{0.000000in}}%
\pgfpathlineto{\pgfqpoint{0.000000in}{0.020833in}}%
\pgfusepath{stroke,fill}%
}%
\begin{pgfscope}%
\pgfsys@transformshift{2.287523in}{1.080890in}%
\pgfsys@useobject{currentmarker}{}%
\end{pgfscope}%
\end{pgfscope}%
\begin{pgfscope}%
\pgfsetbuttcap%
\pgfsetroundjoin%
\definecolor{currentfill}{rgb}{0.000000,0.000000,0.000000}%
\pgfsetfillcolor{currentfill}%
\pgfsetlinewidth{0.501875pt}%
\definecolor{currentstroke}{rgb}{0.000000,0.000000,0.000000}%
\pgfsetstrokecolor{currentstroke}%
\pgfsetdash{}{0pt}%
\pgfsys@defobject{currentmarker}{\pgfqpoint{0.000000in}{-0.020833in}}{\pgfqpoint{0.000000in}{0.000000in}}{%
\pgfpathmoveto{\pgfqpoint{0.000000in}{0.000000in}}%
\pgfpathlineto{\pgfqpoint{0.000000in}{-0.020833in}}%
\pgfusepath{stroke,fill}%
}%
\begin{pgfscope}%
\pgfsys@transformshift{2.287523in}{3.227753in}%
\pgfsys@useobject{currentmarker}{}%
\end{pgfscope}%
\end{pgfscope}%
\begin{pgfscope}%
\pgfpathrectangle{\pgfqpoint{0.481681in}{1.080890in}}{\pgfqpoint{5.785672in}{2.146863in}}%
\pgfusepath{clip}%
\pgfsetrectcap%
\pgfsetroundjoin%
\pgfsetlinewidth{0.100375pt}%
\definecolor{currentstroke}{rgb}{0.827451,0.827451,0.827451}%
\pgfsetstrokecolor{currentstroke}%
\pgfsetdash{}{0pt}%
\pgfpathmoveto{\pgfqpoint{2.389869in}{1.080890in}}%
\pgfpathlineto{\pgfqpoint{2.389869in}{3.227753in}}%
\pgfusepath{stroke}%
\end{pgfscope}%
\begin{pgfscope}%
\pgfsetbuttcap%
\pgfsetroundjoin%
\definecolor{currentfill}{rgb}{0.000000,0.000000,0.000000}%
\pgfsetfillcolor{currentfill}%
\pgfsetlinewidth{0.501875pt}%
\definecolor{currentstroke}{rgb}{0.000000,0.000000,0.000000}%
\pgfsetstrokecolor{currentstroke}%
\pgfsetdash{}{0pt}%
\pgfsys@defobject{currentmarker}{\pgfqpoint{0.000000in}{0.000000in}}{\pgfqpoint{0.000000in}{0.020833in}}{%
\pgfpathmoveto{\pgfqpoint{0.000000in}{0.000000in}}%
\pgfpathlineto{\pgfqpoint{0.000000in}{0.020833in}}%
\pgfusepath{stroke,fill}%
}%
\begin{pgfscope}%
\pgfsys@transformshift{2.389869in}{1.080890in}%
\pgfsys@useobject{currentmarker}{}%
\end{pgfscope}%
\end{pgfscope}%
\begin{pgfscope}%
\pgfsetbuttcap%
\pgfsetroundjoin%
\definecolor{currentfill}{rgb}{0.000000,0.000000,0.000000}%
\pgfsetfillcolor{currentfill}%
\pgfsetlinewidth{0.501875pt}%
\definecolor{currentstroke}{rgb}{0.000000,0.000000,0.000000}%
\pgfsetstrokecolor{currentstroke}%
\pgfsetdash{}{0pt}%
\pgfsys@defobject{currentmarker}{\pgfqpoint{0.000000in}{-0.020833in}}{\pgfqpoint{0.000000in}{0.000000in}}{%
\pgfpathmoveto{\pgfqpoint{0.000000in}{0.000000in}}%
\pgfpathlineto{\pgfqpoint{0.000000in}{-0.020833in}}%
\pgfusepath{stroke,fill}%
}%
\begin{pgfscope}%
\pgfsys@transformshift{2.389869in}{3.227753in}%
\pgfsys@useobject{currentmarker}{}%
\end{pgfscope}%
\end{pgfscope}%
\begin{pgfscope}%
\pgfpathrectangle{\pgfqpoint{0.481681in}{1.080890in}}{\pgfqpoint{5.785672in}{2.146863in}}%
\pgfusepath{clip}%
\pgfsetrectcap%
\pgfsetroundjoin%
\pgfsetlinewidth{0.100375pt}%
\definecolor{currentstroke}{rgb}{0.827451,0.827451,0.827451}%
\pgfsetstrokecolor{currentstroke}%
\pgfsetdash{}{0pt}%
\pgfpathmoveto{\pgfqpoint{2.492214in}{1.080890in}}%
\pgfpathlineto{\pgfqpoint{2.492214in}{3.227753in}}%
\pgfusepath{stroke}%
\end{pgfscope}%
\begin{pgfscope}%
\pgfsetbuttcap%
\pgfsetroundjoin%
\definecolor{currentfill}{rgb}{0.000000,0.000000,0.000000}%
\pgfsetfillcolor{currentfill}%
\pgfsetlinewidth{0.501875pt}%
\definecolor{currentstroke}{rgb}{0.000000,0.000000,0.000000}%
\pgfsetstrokecolor{currentstroke}%
\pgfsetdash{}{0pt}%
\pgfsys@defobject{currentmarker}{\pgfqpoint{0.000000in}{0.000000in}}{\pgfqpoint{0.000000in}{0.020833in}}{%
\pgfpathmoveto{\pgfqpoint{0.000000in}{0.000000in}}%
\pgfpathlineto{\pgfqpoint{0.000000in}{0.020833in}}%
\pgfusepath{stroke,fill}%
}%
\begin{pgfscope}%
\pgfsys@transformshift{2.492214in}{1.080890in}%
\pgfsys@useobject{currentmarker}{}%
\end{pgfscope}%
\end{pgfscope}%
\begin{pgfscope}%
\pgfsetbuttcap%
\pgfsetroundjoin%
\definecolor{currentfill}{rgb}{0.000000,0.000000,0.000000}%
\pgfsetfillcolor{currentfill}%
\pgfsetlinewidth{0.501875pt}%
\definecolor{currentstroke}{rgb}{0.000000,0.000000,0.000000}%
\pgfsetstrokecolor{currentstroke}%
\pgfsetdash{}{0pt}%
\pgfsys@defobject{currentmarker}{\pgfqpoint{0.000000in}{-0.020833in}}{\pgfqpoint{0.000000in}{0.000000in}}{%
\pgfpathmoveto{\pgfqpoint{0.000000in}{0.000000in}}%
\pgfpathlineto{\pgfqpoint{0.000000in}{-0.020833in}}%
\pgfusepath{stroke,fill}%
}%
\begin{pgfscope}%
\pgfsys@transformshift{2.492214in}{3.227753in}%
\pgfsys@useobject{currentmarker}{}%
\end{pgfscope}%
\end{pgfscope}%
\begin{pgfscope}%
\pgfpathrectangle{\pgfqpoint{0.481681in}{1.080890in}}{\pgfqpoint{5.785672in}{2.146863in}}%
\pgfusepath{clip}%
\pgfsetrectcap%
\pgfsetroundjoin%
\pgfsetlinewidth{0.100375pt}%
\definecolor{currentstroke}{rgb}{0.827451,0.827451,0.827451}%
\pgfsetstrokecolor{currentstroke}%
\pgfsetdash{}{0pt}%
\pgfpathmoveto{\pgfqpoint{2.594560in}{1.080890in}}%
\pgfpathlineto{\pgfqpoint{2.594560in}{3.227753in}}%
\pgfusepath{stroke}%
\end{pgfscope}%
\begin{pgfscope}%
\pgfsetbuttcap%
\pgfsetroundjoin%
\definecolor{currentfill}{rgb}{0.000000,0.000000,0.000000}%
\pgfsetfillcolor{currentfill}%
\pgfsetlinewidth{0.501875pt}%
\definecolor{currentstroke}{rgb}{0.000000,0.000000,0.000000}%
\pgfsetstrokecolor{currentstroke}%
\pgfsetdash{}{0pt}%
\pgfsys@defobject{currentmarker}{\pgfqpoint{0.000000in}{0.000000in}}{\pgfqpoint{0.000000in}{0.020833in}}{%
\pgfpathmoveto{\pgfqpoint{0.000000in}{0.000000in}}%
\pgfpathlineto{\pgfqpoint{0.000000in}{0.020833in}}%
\pgfusepath{stroke,fill}%
}%
\begin{pgfscope}%
\pgfsys@transformshift{2.594560in}{1.080890in}%
\pgfsys@useobject{currentmarker}{}%
\end{pgfscope}%
\end{pgfscope}%
\begin{pgfscope}%
\pgfsetbuttcap%
\pgfsetroundjoin%
\definecolor{currentfill}{rgb}{0.000000,0.000000,0.000000}%
\pgfsetfillcolor{currentfill}%
\pgfsetlinewidth{0.501875pt}%
\definecolor{currentstroke}{rgb}{0.000000,0.000000,0.000000}%
\pgfsetstrokecolor{currentstroke}%
\pgfsetdash{}{0pt}%
\pgfsys@defobject{currentmarker}{\pgfqpoint{0.000000in}{-0.020833in}}{\pgfqpoint{0.000000in}{0.000000in}}{%
\pgfpathmoveto{\pgfqpoint{0.000000in}{0.000000in}}%
\pgfpathlineto{\pgfqpoint{0.000000in}{-0.020833in}}%
\pgfusepath{stroke,fill}%
}%
\begin{pgfscope}%
\pgfsys@transformshift{2.594560in}{3.227753in}%
\pgfsys@useobject{currentmarker}{}%
\end{pgfscope}%
\end{pgfscope}%
\begin{pgfscope}%
\pgfpathrectangle{\pgfqpoint{0.481681in}{1.080890in}}{\pgfqpoint{5.785672in}{2.146863in}}%
\pgfusepath{clip}%
\pgfsetrectcap%
\pgfsetroundjoin%
\pgfsetlinewidth{0.100375pt}%
\definecolor{currentstroke}{rgb}{0.827451,0.827451,0.827451}%
\pgfsetstrokecolor{currentstroke}%
\pgfsetdash{}{0pt}%
\pgfpathmoveto{\pgfqpoint{2.799250in}{1.080890in}}%
\pgfpathlineto{\pgfqpoint{2.799250in}{3.227753in}}%
\pgfusepath{stroke}%
\end{pgfscope}%
\begin{pgfscope}%
\pgfsetbuttcap%
\pgfsetroundjoin%
\definecolor{currentfill}{rgb}{0.000000,0.000000,0.000000}%
\pgfsetfillcolor{currentfill}%
\pgfsetlinewidth{0.501875pt}%
\definecolor{currentstroke}{rgb}{0.000000,0.000000,0.000000}%
\pgfsetstrokecolor{currentstroke}%
\pgfsetdash{}{0pt}%
\pgfsys@defobject{currentmarker}{\pgfqpoint{0.000000in}{0.000000in}}{\pgfqpoint{0.000000in}{0.020833in}}{%
\pgfpathmoveto{\pgfqpoint{0.000000in}{0.000000in}}%
\pgfpathlineto{\pgfqpoint{0.000000in}{0.020833in}}%
\pgfusepath{stroke,fill}%
}%
\begin{pgfscope}%
\pgfsys@transformshift{2.799250in}{1.080890in}%
\pgfsys@useobject{currentmarker}{}%
\end{pgfscope}%
\end{pgfscope}%
\begin{pgfscope}%
\pgfsetbuttcap%
\pgfsetroundjoin%
\definecolor{currentfill}{rgb}{0.000000,0.000000,0.000000}%
\pgfsetfillcolor{currentfill}%
\pgfsetlinewidth{0.501875pt}%
\definecolor{currentstroke}{rgb}{0.000000,0.000000,0.000000}%
\pgfsetstrokecolor{currentstroke}%
\pgfsetdash{}{0pt}%
\pgfsys@defobject{currentmarker}{\pgfqpoint{0.000000in}{-0.020833in}}{\pgfqpoint{0.000000in}{0.000000in}}{%
\pgfpathmoveto{\pgfqpoint{0.000000in}{0.000000in}}%
\pgfpathlineto{\pgfqpoint{0.000000in}{-0.020833in}}%
\pgfusepath{stroke,fill}%
}%
\begin{pgfscope}%
\pgfsys@transformshift{2.799250in}{3.227753in}%
\pgfsys@useobject{currentmarker}{}%
\end{pgfscope}%
\end{pgfscope}%
\begin{pgfscope}%
\pgfpathrectangle{\pgfqpoint{0.481681in}{1.080890in}}{\pgfqpoint{5.785672in}{2.146863in}}%
\pgfusepath{clip}%
\pgfsetrectcap%
\pgfsetroundjoin%
\pgfsetlinewidth{0.100375pt}%
\definecolor{currentstroke}{rgb}{0.827451,0.827451,0.827451}%
\pgfsetstrokecolor{currentstroke}%
\pgfsetdash{}{0pt}%
\pgfpathmoveto{\pgfqpoint{2.901596in}{1.080890in}}%
\pgfpathlineto{\pgfqpoint{2.901596in}{3.227753in}}%
\pgfusepath{stroke}%
\end{pgfscope}%
\begin{pgfscope}%
\pgfsetbuttcap%
\pgfsetroundjoin%
\definecolor{currentfill}{rgb}{0.000000,0.000000,0.000000}%
\pgfsetfillcolor{currentfill}%
\pgfsetlinewidth{0.501875pt}%
\definecolor{currentstroke}{rgb}{0.000000,0.000000,0.000000}%
\pgfsetstrokecolor{currentstroke}%
\pgfsetdash{}{0pt}%
\pgfsys@defobject{currentmarker}{\pgfqpoint{0.000000in}{0.000000in}}{\pgfqpoint{0.000000in}{0.020833in}}{%
\pgfpathmoveto{\pgfqpoint{0.000000in}{0.000000in}}%
\pgfpathlineto{\pgfqpoint{0.000000in}{0.020833in}}%
\pgfusepath{stroke,fill}%
}%
\begin{pgfscope}%
\pgfsys@transformshift{2.901596in}{1.080890in}%
\pgfsys@useobject{currentmarker}{}%
\end{pgfscope}%
\end{pgfscope}%
\begin{pgfscope}%
\pgfsetbuttcap%
\pgfsetroundjoin%
\definecolor{currentfill}{rgb}{0.000000,0.000000,0.000000}%
\pgfsetfillcolor{currentfill}%
\pgfsetlinewidth{0.501875pt}%
\definecolor{currentstroke}{rgb}{0.000000,0.000000,0.000000}%
\pgfsetstrokecolor{currentstroke}%
\pgfsetdash{}{0pt}%
\pgfsys@defobject{currentmarker}{\pgfqpoint{0.000000in}{-0.020833in}}{\pgfqpoint{0.000000in}{0.000000in}}{%
\pgfpathmoveto{\pgfqpoint{0.000000in}{0.000000in}}%
\pgfpathlineto{\pgfqpoint{0.000000in}{-0.020833in}}%
\pgfusepath{stroke,fill}%
}%
\begin{pgfscope}%
\pgfsys@transformshift{2.901596in}{3.227753in}%
\pgfsys@useobject{currentmarker}{}%
\end{pgfscope}%
\end{pgfscope}%
\begin{pgfscope}%
\pgfpathrectangle{\pgfqpoint{0.481681in}{1.080890in}}{\pgfqpoint{5.785672in}{2.146863in}}%
\pgfusepath{clip}%
\pgfsetrectcap%
\pgfsetroundjoin%
\pgfsetlinewidth{0.100375pt}%
\definecolor{currentstroke}{rgb}{0.827451,0.827451,0.827451}%
\pgfsetstrokecolor{currentstroke}%
\pgfsetdash{}{0pt}%
\pgfpathmoveto{\pgfqpoint{3.003941in}{1.080890in}}%
\pgfpathlineto{\pgfqpoint{3.003941in}{3.227753in}}%
\pgfusepath{stroke}%
\end{pgfscope}%
\begin{pgfscope}%
\pgfsetbuttcap%
\pgfsetroundjoin%
\definecolor{currentfill}{rgb}{0.000000,0.000000,0.000000}%
\pgfsetfillcolor{currentfill}%
\pgfsetlinewidth{0.501875pt}%
\definecolor{currentstroke}{rgb}{0.000000,0.000000,0.000000}%
\pgfsetstrokecolor{currentstroke}%
\pgfsetdash{}{0pt}%
\pgfsys@defobject{currentmarker}{\pgfqpoint{0.000000in}{0.000000in}}{\pgfqpoint{0.000000in}{0.020833in}}{%
\pgfpathmoveto{\pgfqpoint{0.000000in}{0.000000in}}%
\pgfpathlineto{\pgfqpoint{0.000000in}{0.020833in}}%
\pgfusepath{stroke,fill}%
}%
\begin{pgfscope}%
\pgfsys@transformshift{3.003941in}{1.080890in}%
\pgfsys@useobject{currentmarker}{}%
\end{pgfscope}%
\end{pgfscope}%
\begin{pgfscope}%
\pgfsetbuttcap%
\pgfsetroundjoin%
\definecolor{currentfill}{rgb}{0.000000,0.000000,0.000000}%
\pgfsetfillcolor{currentfill}%
\pgfsetlinewidth{0.501875pt}%
\definecolor{currentstroke}{rgb}{0.000000,0.000000,0.000000}%
\pgfsetstrokecolor{currentstroke}%
\pgfsetdash{}{0pt}%
\pgfsys@defobject{currentmarker}{\pgfqpoint{0.000000in}{-0.020833in}}{\pgfqpoint{0.000000in}{0.000000in}}{%
\pgfpathmoveto{\pgfqpoint{0.000000in}{0.000000in}}%
\pgfpathlineto{\pgfqpoint{0.000000in}{-0.020833in}}%
\pgfusepath{stroke,fill}%
}%
\begin{pgfscope}%
\pgfsys@transformshift{3.003941in}{3.227753in}%
\pgfsys@useobject{currentmarker}{}%
\end{pgfscope}%
\end{pgfscope}%
\begin{pgfscope}%
\pgfpathrectangle{\pgfqpoint{0.481681in}{1.080890in}}{\pgfqpoint{5.785672in}{2.146863in}}%
\pgfusepath{clip}%
\pgfsetrectcap%
\pgfsetroundjoin%
\pgfsetlinewidth{0.100375pt}%
\definecolor{currentstroke}{rgb}{0.827451,0.827451,0.827451}%
\pgfsetstrokecolor{currentstroke}%
\pgfsetdash{}{0pt}%
\pgfpathmoveto{\pgfqpoint{3.106287in}{1.080890in}}%
\pgfpathlineto{\pgfqpoint{3.106287in}{3.227753in}}%
\pgfusepath{stroke}%
\end{pgfscope}%
\begin{pgfscope}%
\pgfsetbuttcap%
\pgfsetroundjoin%
\definecolor{currentfill}{rgb}{0.000000,0.000000,0.000000}%
\pgfsetfillcolor{currentfill}%
\pgfsetlinewidth{0.501875pt}%
\definecolor{currentstroke}{rgb}{0.000000,0.000000,0.000000}%
\pgfsetstrokecolor{currentstroke}%
\pgfsetdash{}{0pt}%
\pgfsys@defobject{currentmarker}{\pgfqpoint{0.000000in}{0.000000in}}{\pgfqpoint{0.000000in}{0.020833in}}{%
\pgfpathmoveto{\pgfqpoint{0.000000in}{0.000000in}}%
\pgfpathlineto{\pgfqpoint{0.000000in}{0.020833in}}%
\pgfusepath{stroke,fill}%
}%
\begin{pgfscope}%
\pgfsys@transformshift{3.106287in}{1.080890in}%
\pgfsys@useobject{currentmarker}{}%
\end{pgfscope}%
\end{pgfscope}%
\begin{pgfscope}%
\pgfsetbuttcap%
\pgfsetroundjoin%
\definecolor{currentfill}{rgb}{0.000000,0.000000,0.000000}%
\pgfsetfillcolor{currentfill}%
\pgfsetlinewidth{0.501875pt}%
\definecolor{currentstroke}{rgb}{0.000000,0.000000,0.000000}%
\pgfsetstrokecolor{currentstroke}%
\pgfsetdash{}{0pt}%
\pgfsys@defobject{currentmarker}{\pgfqpoint{0.000000in}{-0.020833in}}{\pgfqpoint{0.000000in}{0.000000in}}{%
\pgfpathmoveto{\pgfqpoint{0.000000in}{0.000000in}}%
\pgfpathlineto{\pgfqpoint{0.000000in}{-0.020833in}}%
\pgfusepath{stroke,fill}%
}%
\begin{pgfscope}%
\pgfsys@transformshift{3.106287in}{3.227753in}%
\pgfsys@useobject{currentmarker}{}%
\end{pgfscope}%
\end{pgfscope}%
\begin{pgfscope}%
\pgfpathrectangle{\pgfqpoint{0.481681in}{1.080890in}}{\pgfqpoint{5.785672in}{2.146863in}}%
\pgfusepath{clip}%
\pgfsetrectcap%
\pgfsetroundjoin%
\pgfsetlinewidth{0.100375pt}%
\definecolor{currentstroke}{rgb}{0.827451,0.827451,0.827451}%
\pgfsetstrokecolor{currentstroke}%
\pgfsetdash{}{0pt}%
\pgfpathmoveto{\pgfqpoint{3.208632in}{1.080890in}}%
\pgfpathlineto{\pgfqpoint{3.208632in}{3.227753in}}%
\pgfusepath{stroke}%
\end{pgfscope}%
\begin{pgfscope}%
\pgfsetbuttcap%
\pgfsetroundjoin%
\definecolor{currentfill}{rgb}{0.000000,0.000000,0.000000}%
\pgfsetfillcolor{currentfill}%
\pgfsetlinewidth{0.501875pt}%
\definecolor{currentstroke}{rgb}{0.000000,0.000000,0.000000}%
\pgfsetstrokecolor{currentstroke}%
\pgfsetdash{}{0pt}%
\pgfsys@defobject{currentmarker}{\pgfqpoint{0.000000in}{0.000000in}}{\pgfqpoint{0.000000in}{0.020833in}}{%
\pgfpathmoveto{\pgfqpoint{0.000000in}{0.000000in}}%
\pgfpathlineto{\pgfqpoint{0.000000in}{0.020833in}}%
\pgfusepath{stroke,fill}%
}%
\begin{pgfscope}%
\pgfsys@transformshift{3.208632in}{1.080890in}%
\pgfsys@useobject{currentmarker}{}%
\end{pgfscope}%
\end{pgfscope}%
\begin{pgfscope}%
\pgfsetbuttcap%
\pgfsetroundjoin%
\definecolor{currentfill}{rgb}{0.000000,0.000000,0.000000}%
\pgfsetfillcolor{currentfill}%
\pgfsetlinewidth{0.501875pt}%
\definecolor{currentstroke}{rgb}{0.000000,0.000000,0.000000}%
\pgfsetstrokecolor{currentstroke}%
\pgfsetdash{}{0pt}%
\pgfsys@defobject{currentmarker}{\pgfqpoint{0.000000in}{-0.020833in}}{\pgfqpoint{0.000000in}{0.000000in}}{%
\pgfpathmoveto{\pgfqpoint{0.000000in}{0.000000in}}%
\pgfpathlineto{\pgfqpoint{0.000000in}{-0.020833in}}%
\pgfusepath{stroke,fill}%
}%
\begin{pgfscope}%
\pgfsys@transformshift{3.208632in}{3.227753in}%
\pgfsys@useobject{currentmarker}{}%
\end{pgfscope}%
\end{pgfscope}%
\begin{pgfscope}%
\pgfpathrectangle{\pgfqpoint{0.481681in}{1.080890in}}{\pgfqpoint{5.785672in}{2.146863in}}%
\pgfusepath{clip}%
\pgfsetrectcap%
\pgfsetroundjoin%
\pgfsetlinewidth{0.100375pt}%
\definecolor{currentstroke}{rgb}{0.827451,0.827451,0.827451}%
\pgfsetstrokecolor{currentstroke}%
\pgfsetdash{}{0pt}%
\pgfpathmoveto{\pgfqpoint{3.413323in}{1.080890in}}%
\pgfpathlineto{\pgfqpoint{3.413323in}{3.227753in}}%
\pgfusepath{stroke}%
\end{pgfscope}%
\begin{pgfscope}%
\pgfsetbuttcap%
\pgfsetroundjoin%
\definecolor{currentfill}{rgb}{0.000000,0.000000,0.000000}%
\pgfsetfillcolor{currentfill}%
\pgfsetlinewidth{0.501875pt}%
\definecolor{currentstroke}{rgb}{0.000000,0.000000,0.000000}%
\pgfsetstrokecolor{currentstroke}%
\pgfsetdash{}{0pt}%
\pgfsys@defobject{currentmarker}{\pgfqpoint{0.000000in}{0.000000in}}{\pgfqpoint{0.000000in}{0.020833in}}{%
\pgfpathmoveto{\pgfqpoint{0.000000in}{0.000000in}}%
\pgfpathlineto{\pgfqpoint{0.000000in}{0.020833in}}%
\pgfusepath{stroke,fill}%
}%
\begin{pgfscope}%
\pgfsys@transformshift{3.413323in}{1.080890in}%
\pgfsys@useobject{currentmarker}{}%
\end{pgfscope}%
\end{pgfscope}%
\begin{pgfscope}%
\pgfsetbuttcap%
\pgfsetroundjoin%
\definecolor{currentfill}{rgb}{0.000000,0.000000,0.000000}%
\pgfsetfillcolor{currentfill}%
\pgfsetlinewidth{0.501875pt}%
\definecolor{currentstroke}{rgb}{0.000000,0.000000,0.000000}%
\pgfsetstrokecolor{currentstroke}%
\pgfsetdash{}{0pt}%
\pgfsys@defobject{currentmarker}{\pgfqpoint{0.000000in}{-0.020833in}}{\pgfqpoint{0.000000in}{0.000000in}}{%
\pgfpathmoveto{\pgfqpoint{0.000000in}{0.000000in}}%
\pgfpathlineto{\pgfqpoint{0.000000in}{-0.020833in}}%
\pgfusepath{stroke,fill}%
}%
\begin{pgfscope}%
\pgfsys@transformshift{3.413323in}{3.227753in}%
\pgfsys@useobject{currentmarker}{}%
\end{pgfscope}%
\end{pgfscope}%
\begin{pgfscope}%
\pgfpathrectangle{\pgfqpoint{0.481681in}{1.080890in}}{\pgfqpoint{5.785672in}{2.146863in}}%
\pgfusepath{clip}%
\pgfsetrectcap%
\pgfsetroundjoin%
\pgfsetlinewidth{0.100375pt}%
\definecolor{currentstroke}{rgb}{0.827451,0.827451,0.827451}%
\pgfsetstrokecolor{currentstroke}%
\pgfsetdash{}{0pt}%
\pgfpathmoveto{\pgfqpoint{3.515668in}{1.080890in}}%
\pgfpathlineto{\pgfqpoint{3.515668in}{3.227753in}}%
\pgfusepath{stroke}%
\end{pgfscope}%
\begin{pgfscope}%
\pgfsetbuttcap%
\pgfsetroundjoin%
\definecolor{currentfill}{rgb}{0.000000,0.000000,0.000000}%
\pgfsetfillcolor{currentfill}%
\pgfsetlinewidth{0.501875pt}%
\definecolor{currentstroke}{rgb}{0.000000,0.000000,0.000000}%
\pgfsetstrokecolor{currentstroke}%
\pgfsetdash{}{0pt}%
\pgfsys@defobject{currentmarker}{\pgfqpoint{0.000000in}{0.000000in}}{\pgfqpoint{0.000000in}{0.020833in}}{%
\pgfpathmoveto{\pgfqpoint{0.000000in}{0.000000in}}%
\pgfpathlineto{\pgfqpoint{0.000000in}{0.020833in}}%
\pgfusepath{stroke,fill}%
}%
\begin{pgfscope}%
\pgfsys@transformshift{3.515668in}{1.080890in}%
\pgfsys@useobject{currentmarker}{}%
\end{pgfscope}%
\end{pgfscope}%
\begin{pgfscope}%
\pgfsetbuttcap%
\pgfsetroundjoin%
\definecolor{currentfill}{rgb}{0.000000,0.000000,0.000000}%
\pgfsetfillcolor{currentfill}%
\pgfsetlinewidth{0.501875pt}%
\definecolor{currentstroke}{rgb}{0.000000,0.000000,0.000000}%
\pgfsetstrokecolor{currentstroke}%
\pgfsetdash{}{0pt}%
\pgfsys@defobject{currentmarker}{\pgfqpoint{0.000000in}{-0.020833in}}{\pgfqpoint{0.000000in}{0.000000in}}{%
\pgfpathmoveto{\pgfqpoint{0.000000in}{0.000000in}}%
\pgfpathlineto{\pgfqpoint{0.000000in}{-0.020833in}}%
\pgfusepath{stroke,fill}%
}%
\begin{pgfscope}%
\pgfsys@transformshift{3.515668in}{3.227753in}%
\pgfsys@useobject{currentmarker}{}%
\end{pgfscope}%
\end{pgfscope}%
\begin{pgfscope}%
\pgfpathrectangle{\pgfqpoint{0.481681in}{1.080890in}}{\pgfqpoint{5.785672in}{2.146863in}}%
\pgfusepath{clip}%
\pgfsetrectcap%
\pgfsetroundjoin%
\pgfsetlinewidth{0.100375pt}%
\definecolor{currentstroke}{rgb}{0.827451,0.827451,0.827451}%
\pgfsetstrokecolor{currentstroke}%
\pgfsetdash{}{0pt}%
\pgfpathmoveto{\pgfqpoint{3.618014in}{1.080890in}}%
\pgfpathlineto{\pgfqpoint{3.618014in}{3.227753in}}%
\pgfusepath{stroke}%
\end{pgfscope}%
\begin{pgfscope}%
\pgfsetbuttcap%
\pgfsetroundjoin%
\definecolor{currentfill}{rgb}{0.000000,0.000000,0.000000}%
\pgfsetfillcolor{currentfill}%
\pgfsetlinewidth{0.501875pt}%
\definecolor{currentstroke}{rgb}{0.000000,0.000000,0.000000}%
\pgfsetstrokecolor{currentstroke}%
\pgfsetdash{}{0pt}%
\pgfsys@defobject{currentmarker}{\pgfqpoint{0.000000in}{0.000000in}}{\pgfqpoint{0.000000in}{0.020833in}}{%
\pgfpathmoveto{\pgfqpoint{0.000000in}{0.000000in}}%
\pgfpathlineto{\pgfqpoint{0.000000in}{0.020833in}}%
\pgfusepath{stroke,fill}%
}%
\begin{pgfscope}%
\pgfsys@transformshift{3.618014in}{1.080890in}%
\pgfsys@useobject{currentmarker}{}%
\end{pgfscope}%
\end{pgfscope}%
\begin{pgfscope}%
\pgfsetbuttcap%
\pgfsetroundjoin%
\definecolor{currentfill}{rgb}{0.000000,0.000000,0.000000}%
\pgfsetfillcolor{currentfill}%
\pgfsetlinewidth{0.501875pt}%
\definecolor{currentstroke}{rgb}{0.000000,0.000000,0.000000}%
\pgfsetstrokecolor{currentstroke}%
\pgfsetdash{}{0pt}%
\pgfsys@defobject{currentmarker}{\pgfqpoint{0.000000in}{-0.020833in}}{\pgfqpoint{0.000000in}{0.000000in}}{%
\pgfpathmoveto{\pgfqpoint{0.000000in}{0.000000in}}%
\pgfpathlineto{\pgfqpoint{0.000000in}{-0.020833in}}%
\pgfusepath{stroke,fill}%
}%
\begin{pgfscope}%
\pgfsys@transformshift{3.618014in}{3.227753in}%
\pgfsys@useobject{currentmarker}{}%
\end{pgfscope}%
\end{pgfscope}%
\begin{pgfscope}%
\pgfpathrectangle{\pgfqpoint{0.481681in}{1.080890in}}{\pgfqpoint{5.785672in}{2.146863in}}%
\pgfusepath{clip}%
\pgfsetrectcap%
\pgfsetroundjoin%
\pgfsetlinewidth{0.100375pt}%
\definecolor{currentstroke}{rgb}{0.827451,0.827451,0.827451}%
\pgfsetstrokecolor{currentstroke}%
\pgfsetdash{}{0pt}%
\pgfpathmoveto{\pgfqpoint{3.720359in}{1.080890in}}%
\pgfpathlineto{\pgfqpoint{3.720359in}{3.227753in}}%
\pgfusepath{stroke}%
\end{pgfscope}%
\begin{pgfscope}%
\pgfsetbuttcap%
\pgfsetroundjoin%
\definecolor{currentfill}{rgb}{0.000000,0.000000,0.000000}%
\pgfsetfillcolor{currentfill}%
\pgfsetlinewidth{0.501875pt}%
\definecolor{currentstroke}{rgb}{0.000000,0.000000,0.000000}%
\pgfsetstrokecolor{currentstroke}%
\pgfsetdash{}{0pt}%
\pgfsys@defobject{currentmarker}{\pgfqpoint{0.000000in}{0.000000in}}{\pgfqpoint{0.000000in}{0.020833in}}{%
\pgfpathmoveto{\pgfqpoint{0.000000in}{0.000000in}}%
\pgfpathlineto{\pgfqpoint{0.000000in}{0.020833in}}%
\pgfusepath{stroke,fill}%
}%
\begin{pgfscope}%
\pgfsys@transformshift{3.720359in}{1.080890in}%
\pgfsys@useobject{currentmarker}{}%
\end{pgfscope}%
\end{pgfscope}%
\begin{pgfscope}%
\pgfsetbuttcap%
\pgfsetroundjoin%
\definecolor{currentfill}{rgb}{0.000000,0.000000,0.000000}%
\pgfsetfillcolor{currentfill}%
\pgfsetlinewidth{0.501875pt}%
\definecolor{currentstroke}{rgb}{0.000000,0.000000,0.000000}%
\pgfsetstrokecolor{currentstroke}%
\pgfsetdash{}{0pt}%
\pgfsys@defobject{currentmarker}{\pgfqpoint{0.000000in}{-0.020833in}}{\pgfqpoint{0.000000in}{0.000000in}}{%
\pgfpathmoveto{\pgfqpoint{0.000000in}{0.000000in}}%
\pgfpathlineto{\pgfqpoint{0.000000in}{-0.020833in}}%
\pgfusepath{stroke,fill}%
}%
\begin{pgfscope}%
\pgfsys@transformshift{3.720359in}{3.227753in}%
\pgfsys@useobject{currentmarker}{}%
\end{pgfscope}%
\end{pgfscope}%
\begin{pgfscope}%
\pgfpathrectangle{\pgfqpoint{0.481681in}{1.080890in}}{\pgfqpoint{5.785672in}{2.146863in}}%
\pgfusepath{clip}%
\pgfsetrectcap%
\pgfsetroundjoin%
\pgfsetlinewidth{0.100375pt}%
\definecolor{currentstroke}{rgb}{0.827451,0.827451,0.827451}%
\pgfsetstrokecolor{currentstroke}%
\pgfsetdash{}{0pt}%
\pgfpathmoveto{\pgfqpoint{3.822705in}{1.080890in}}%
\pgfpathlineto{\pgfqpoint{3.822705in}{3.227753in}}%
\pgfusepath{stroke}%
\end{pgfscope}%
\begin{pgfscope}%
\pgfsetbuttcap%
\pgfsetroundjoin%
\definecolor{currentfill}{rgb}{0.000000,0.000000,0.000000}%
\pgfsetfillcolor{currentfill}%
\pgfsetlinewidth{0.501875pt}%
\definecolor{currentstroke}{rgb}{0.000000,0.000000,0.000000}%
\pgfsetstrokecolor{currentstroke}%
\pgfsetdash{}{0pt}%
\pgfsys@defobject{currentmarker}{\pgfqpoint{0.000000in}{0.000000in}}{\pgfqpoint{0.000000in}{0.020833in}}{%
\pgfpathmoveto{\pgfqpoint{0.000000in}{0.000000in}}%
\pgfpathlineto{\pgfqpoint{0.000000in}{0.020833in}}%
\pgfusepath{stroke,fill}%
}%
\begin{pgfscope}%
\pgfsys@transformshift{3.822705in}{1.080890in}%
\pgfsys@useobject{currentmarker}{}%
\end{pgfscope}%
\end{pgfscope}%
\begin{pgfscope}%
\pgfsetbuttcap%
\pgfsetroundjoin%
\definecolor{currentfill}{rgb}{0.000000,0.000000,0.000000}%
\pgfsetfillcolor{currentfill}%
\pgfsetlinewidth{0.501875pt}%
\definecolor{currentstroke}{rgb}{0.000000,0.000000,0.000000}%
\pgfsetstrokecolor{currentstroke}%
\pgfsetdash{}{0pt}%
\pgfsys@defobject{currentmarker}{\pgfqpoint{0.000000in}{-0.020833in}}{\pgfqpoint{0.000000in}{0.000000in}}{%
\pgfpathmoveto{\pgfqpoint{0.000000in}{0.000000in}}%
\pgfpathlineto{\pgfqpoint{0.000000in}{-0.020833in}}%
\pgfusepath{stroke,fill}%
}%
\begin{pgfscope}%
\pgfsys@transformshift{3.822705in}{3.227753in}%
\pgfsys@useobject{currentmarker}{}%
\end{pgfscope}%
\end{pgfscope}%
\begin{pgfscope}%
\pgfpathrectangle{\pgfqpoint{0.481681in}{1.080890in}}{\pgfqpoint{5.785672in}{2.146863in}}%
\pgfusepath{clip}%
\pgfsetrectcap%
\pgfsetroundjoin%
\pgfsetlinewidth{0.100375pt}%
\definecolor{currentstroke}{rgb}{0.827451,0.827451,0.827451}%
\pgfsetstrokecolor{currentstroke}%
\pgfsetdash{}{0pt}%
\pgfpathmoveto{\pgfqpoint{4.027395in}{1.080890in}}%
\pgfpathlineto{\pgfqpoint{4.027395in}{3.227753in}}%
\pgfusepath{stroke}%
\end{pgfscope}%
\begin{pgfscope}%
\pgfsetbuttcap%
\pgfsetroundjoin%
\definecolor{currentfill}{rgb}{0.000000,0.000000,0.000000}%
\pgfsetfillcolor{currentfill}%
\pgfsetlinewidth{0.501875pt}%
\definecolor{currentstroke}{rgb}{0.000000,0.000000,0.000000}%
\pgfsetstrokecolor{currentstroke}%
\pgfsetdash{}{0pt}%
\pgfsys@defobject{currentmarker}{\pgfqpoint{0.000000in}{0.000000in}}{\pgfqpoint{0.000000in}{0.020833in}}{%
\pgfpathmoveto{\pgfqpoint{0.000000in}{0.000000in}}%
\pgfpathlineto{\pgfqpoint{0.000000in}{0.020833in}}%
\pgfusepath{stroke,fill}%
}%
\begin{pgfscope}%
\pgfsys@transformshift{4.027395in}{1.080890in}%
\pgfsys@useobject{currentmarker}{}%
\end{pgfscope}%
\end{pgfscope}%
\begin{pgfscope}%
\pgfsetbuttcap%
\pgfsetroundjoin%
\definecolor{currentfill}{rgb}{0.000000,0.000000,0.000000}%
\pgfsetfillcolor{currentfill}%
\pgfsetlinewidth{0.501875pt}%
\definecolor{currentstroke}{rgb}{0.000000,0.000000,0.000000}%
\pgfsetstrokecolor{currentstroke}%
\pgfsetdash{}{0pt}%
\pgfsys@defobject{currentmarker}{\pgfqpoint{0.000000in}{-0.020833in}}{\pgfqpoint{0.000000in}{0.000000in}}{%
\pgfpathmoveto{\pgfqpoint{0.000000in}{0.000000in}}%
\pgfpathlineto{\pgfqpoint{0.000000in}{-0.020833in}}%
\pgfusepath{stroke,fill}%
}%
\begin{pgfscope}%
\pgfsys@transformshift{4.027395in}{3.227753in}%
\pgfsys@useobject{currentmarker}{}%
\end{pgfscope}%
\end{pgfscope}%
\begin{pgfscope}%
\pgfpathrectangle{\pgfqpoint{0.481681in}{1.080890in}}{\pgfqpoint{5.785672in}{2.146863in}}%
\pgfusepath{clip}%
\pgfsetrectcap%
\pgfsetroundjoin%
\pgfsetlinewidth{0.100375pt}%
\definecolor{currentstroke}{rgb}{0.827451,0.827451,0.827451}%
\pgfsetstrokecolor{currentstroke}%
\pgfsetdash{}{0pt}%
\pgfpathmoveto{\pgfqpoint{4.129741in}{1.080890in}}%
\pgfpathlineto{\pgfqpoint{4.129741in}{3.227753in}}%
\pgfusepath{stroke}%
\end{pgfscope}%
\begin{pgfscope}%
\pgfsetbuttcap%
\pgfsetroundjoin%
\definecolor{currentfill}{rgb}{0.000000,0.000000,0.000000}%
\pgfsetfillcolor{currentfill}%
\pgfsetlinewidth{0.501875pt}%
\definecolor{currentstroke}{rgb}{0.000000,0.000000,0.000000}%
\pgfsetstrokecolor{currentstroke}%
\pgfsetdash{}{0pt}%
\pgfsys@defobject{currentmarker}{\pgfqpoint{0.000000in}{0.000000in}}{\pgfqpoint{0.000000in}{0.020833in}}{%
\pgfpathmoveto{\pgfqpoint{0.000000in}{0.000000in}}%
\pgfpathlineto{\pgfqpoint{0.000000in}{0.020833in}}%
\pgfusepath{stroke,fill}%
}%
\begin{pgfscope}%
\pgfsys@transformshift{4.129741in}{1.080890in}%
\pgfsys@useobject{currentmarker}{}%
\end{pgfscope}%
\end{pgfscope}%
\begin{pgfscope}%
\pgfsetbuttcap%
\pgfsetroundjoin%
\definecolor{currentfill}{rgb}{0.000000,0.000000,0.000000}%
\pgfsetfillcolor{currentfill}%
\pgfsetlinewidth{0.501875pt}%
\definecolor{currentstroke}{rgb}{0.000000,0.000000,0.000000}%
\pgfsetstrokecolor{currentstroke}%
\pgfsetdash{}{0pt}%
\pgfsys@defobject{currentmarker}{\pgfqpoint{0.000000in}{-0.020833in}}{\pgfqpoint{0.000000in}{0.000000in}}{%
\pgfpathmoveto{\pgfqpoint{0.000000in}{0.000000in}}%
\pgfpathlineto{\pgfqpoint{0.000000in}{-0.020833in}}%
\pgfusepath{stroke,fill}%
}%
\begin{pgfscope}%
\pgfsys@transformshift{4.129741in}{3.227753in}%
\pgfsys@useobject{currentmarker}{}%
\end{pgfscope}%
\end{pgfscope}%
\begin{pgfscope}%
\pgfpathrectangle{\pgfqpoint{0.481681in}{1.080890in}}{\pgfqpoint{5.785672in}{2.146863in}}%
\pgfusepath{clip}%
\pgfsetrectcap%
\pgfsetroundjoin%
\pgfsetlinewidth{0.100375pt}%
\definecolor{currentstroke}{rgb}{0.827451,0.827451,0.827451}%
\pgfsetstrokecolor{currentstroke}%
\pgfsetdash{}{0pt}%
\pgfpathmoveto{\pgfqpoint{4.232086in}{1.080890in}}%
\pgfpathlineto{\pgfqpoint{4.232086in}{3.227753in}}%
\pgfusepath{stroke}%
\end{pgfscope}%
\begin{pgfscope}%
\pgfsetbuttcap%
\pgfsetroundjoin%
\definecolor{currentfill}{rgb}{0.000000,0.000000,0.000000}%
\pgfsetfillcolor{currentfill}%
\pgfsetlinewidth{0.501875pt}%
\definecolor{currentstroke}{rgb}{0.000000,0.000000,0.000000}%
\pgfsetstrokecolor{currentstroke}%
\pgfsetdash{}{0pt}%
\pgfsys@defobject{currentmarker}{\pgfqpoint{0.000000in}{0.000000in}}{\pgfqpoint{0.000000in}{0.020833in}}{%
\pgfpathmoveto{\pgfqpoint{0.000000in}{0.000000in}}%
\pgfpathlineto{\pgfqpoint{0.000000in}{0.020833in}}%
\pgfusepath{stroke,fill}%
}%
\begin{pgfscope}%
\pgfsys@transformshift{4.232086in}{1.080890in}%
\pgfsys@useobject{currentmarker}{}%
\end{pgfscope}%
\end{pgfscope}%
\begin{pgfscope}%
\pgfsetbuttcap%
\pgfsetroundjoin%
\definecolor{currentfill}{rgb}{0.000000,0.000000,0.000000}%
\pgfsetfillcolor{currentfill}%
\pgfsetlinewidth{0.501875pt}%
\definecolor{currentstroke}{rgb}{0.000000,0.000000,0.000000}%
\pgfsetstrokecolor{currentstroke}%
\pgfsetdash{}{0pt}%
\pgfsys@defobject{currentmarker}{\pgfqpoint{0.000000in}{-0.020833in}}{\pgfqpoint{0.000000in}{0.000000in}}{%
\pgfpathmoveto{\pgfqpoint{0.000000in}{0.000000in}}%
\pgfpathlineto{\pgfqpoint{0.000000in}{-0.020833in}}%
\pgfusepath{stroke,fill}%
}%
\begin{pgfscope}%
\pgfsys@transformshift{4.232086in}{3.227753in}%
\pgfsys@useobject{currentmarker}{}%
\end{pgfscope}%
\end{pgfscope}%
\begin{pgfscope}%
\pgfpathrectangle{\pgfqpoint{0.481681in}{1.080890in}}{\pgfqpoint{5.785672in}{2.146863in}}%
\pgfusepath{clip}%
\pgfsetrectcap%
\pgfsetroundjoin%
\pgfsetlinewidth{0.100375pt}%
\definecolor{currentstroke}{rgb}{0.827451,0.827451,0.827451}%
\pgfsetstrokecolor{currentstroke}%
\pgfsetdash{}{0pt}%
\pgfpathmoveto{\pgfqpoint{4.334432in}{1.080890in}}%
\pgfpathlineto{\pgfqpoint{4.334432in}{3.227753in}}%
\pgfusepath{stroke}%
\end{pgfscope}%
\begin{pgfscope}%
\pgfsetbuttcap%
\pgfsetroundjoin%
\definecolor{currentfill}{rgb}{0.000000,0.000000,0.000000}%
\pgfsetfillcolor{currentfill}%
\pgfsetlinewidth{0.501875pt}%
\definecolor{currentstroke}{rgb}{0.000000,0.000000,0.000000}%
\pgfsetstrokecolor{currentstroke}%
\pgfsetdash{}{0pt}%
\pgfsys@defobject{currentmarker}{\pgfqpoint{0.000000in}{0.000000in}}{\pgfqpoint{0.000000in}{0.020833in}}{%
\pgfpathmoveto{\pgfqpoint{0.000000in}{0.000000in}}%
\pgfpathlineto{\pgfqpoint{0.000000in}{0.020833in}}%
\pgfusepath{stroke,fill}%
}%
\begin{pgfscope}%
\pgfsys@transformshift{4.334432in}{1.080890in}%
\pgfsys@useobject{currentmarker}{}%
\end{pgfscope}%
\end{pgfscope}%
\begin{pgfscope}%
\pgfsetbuttcap%
\pgfsetroundjoin%
\definecolor{currentfill}{rgb}{0.000000,0.000000,0.000000}%
\pgfsetfillcolor{currentfill}%
\pgfsetlinewidth{0.501875pt}%
\definecolor{currentstroke}{rgb}{0.000000,0.000000,0.000000}%
\pgfsetstrokecolor{currentstroke}%
\pgfsetdash{}{0pt}%
\pgfsys@defobject{currentmarker}{\pgfqpoint{0.000000in}{-0.020833in}}{\pgfqpoint{0.000000in}{0.000000in}}{%
\pgfpathmoveto{\pgfqpoint{0.000000in}{0.000000in}}%
\pgfpathlineto{\pgfqpoint{0.000000in}{-0.020833in}}%
\pgfusepath{stroke,fill}%
}%
\begin{pgfscope}%
\pgfsys@transformshift{4.334432in}{3.227753in}%
\pgfsys@useobject{currentmarker}{}%
\end{pgfscope}%
\end{pgfscope}%
\begin{pgfscope}%
\pgfpathrectangle{\pgfqpoint{0.481681in}{1.080890in}}{\pgfqpoint{5.785672in}{2.146863in}}%
\pgfusepath{clip}%
\pgfsetrectcap%
\pgfsetroundjoin%
\pgfsetlinewidth{0.100375pt}%
\definecolor{currentstroke}{rgb}{0.827451,0.827451,0.827451}%
\pgfsetstrokecolor{currentstroke}%
\pgfsetdash{}{0pt}%
\pgfpathmoveto{\pgfqpoint{4.436777in}{1.080890in}}%
\pgfpathlineto{\pgfqpoint{4.436777in}{3.227753in}}%
\pgfusepath{stroke}%
\end{pgfscope}%
\begin{pgfscope}%
\pgfsetbuttcap%
\pgfsetroundjoin%
\definecolor{currentfill}{rgb}{0.000000,0.000000,0.000000}%
\pgfsetfillcolor{currentfill}%
\pgfsetlinewidth{0.501875pt}%
\definecolor{currentstroke}{rgb}{0.000000,0.000000,0.000000}%
\pgfsetstrokecolor{currentstroke}%
\pgfsetdash{}{0pt}%
\pgfsys@defobject{currentmarker}{\pgfqpoint{0.000000in}{0.000000in}}{\pgfqpoint{0.000000in}{0.020833in}}{%
\pgfpathmoveto{\pgfqpoint{0.000000in}{0.000000in}}%
\pgfpathlineto{\pgfqpoint{0.000000in}{0.020833in}}%
\pgfusepath{stroke,fill}%
}%
\begin{pgfscope}%
\pgfsys@transformshift{4.436777in}{1.080890in}%
\pgfsys@useobject{currentmarker}{}%
\end{pgfscope}%
\end{pgfscope}%
\begin{pgfscope}%
\pgfsetbuttcap%
\pgfsetroundjoin%
\definecolor{currentfill}{rgb}{0.000000,0.000000,0.000000}%
\pgfsetfillcolor{currentfill}%
\pgfsetlinewidth{0.501875pt}%
\definecolor{currentstroke}{rgb}{0.000000,0.000000,0.000000}%
\pgfsetstrokecolor{currentstroke}%
\pgfsetdash{}{0pt}%
\pgfsys@defobject{currentmarker}{\pgfqpoint{0.000000in}{-0.020833in}}{\pgfqpoint{0.000000in}{0.000000in}}{%
\pgfpathmoveto{\pgfqpoint{0.000000in}{0.000000in}}%
\pgfpathlineto{\pgfqpoint{0.000000in}{-0.020833in}}%
\pgfusepath{stroke,fill}%
}%
\begin{pgfscope}%
\pgfsys@transformshift{4.436777in}{3.227753in}%
\pgfsys@useobject{currentmarker}{}%
\end{pgfscope}%
\end{pgfscope}%
\begin{pgfscope}%
\pgfpathrectangle{\pgfqpoint{0.481681in}{1.080890in}}{\pgfqpoint{5.785672in}{2.146863in}}%
\pgfusepath{clip}%
\pgfsetrectcap%
\pgfsetroundjoin%
\pgfsetlinewidth{0.100375pt}%
\definecolor{currentstroke}{rgb}{0.827451,0.827451,0.827451}%
\pgfsetstrokecolor{currentstroke}%
\pgfsetdash{}{0pt}%
\pgfpathmoveto{\pgfqpoint{4.641468in}{1.080890in}}%
\pgfpathlineto{\pgfqpoint{4.641468in}{3.227753in}}%
\pgfusepath{stroke}%
\end{pgfscope}%
\begin{pgfscope}%
\pgfsetbuttcap%
\pgfsetroundjoin%
\definecolor{currentfill}{rgb}{0.000000,0.000000,0.000000}%
\pgfsetfillcolor{currentfill}%
\pgfsetlinewidth{0.501875pt}%
\definecolor{currentstroke}{rgb}{0.000000,0.000000,0.000000}%
\pgfsetstrokecolor{currentstroke}%
\pgfsetdash{}{0pt}%
\pgfsys@defobject{currentmarker}{\pgfqpoint{0.000000in}{0.000000in}}{\pgfqpoint{0.000000in}{0.020833in}}{%
\pgfpathmoveto{\pgfqpoint{0.000000in}{0.000000in}}%
\pgfpathlineto{\pgfqpoint{0.000000in}{0.020833in}}%
\pgfusepath{stroke,fill}%
}%
\begin{pgfscope}%
\pgfsys@transformshift{4.641468in}{1.080890in}%
\pgfsys@useobject{currentmarker}{}%
\end{pgfscope}%
\end{pgfscope}%
\begin{pgfscope}%
\pgfsetbuttcap%
\pgfsetroundjoin%
\definecolor{currentfill}{rgb}{0.000000,0.000000,0.000000}%
\pgfsetfillcolor{currentfill}%
\pgfsetlinewidth{0.501875pt}%
\definecolor{currentstroke}{rgb}{0.000000,0.000000,0.000000}%
\pgfsetstrokecolor{currentstroke}%
\pgfsetdash{}{0pt}%
\pgfsys@defobject{currentmarker}{\pgfqpoint{0.000000in}{-0.020833in}}{\pgfqpoint{0.000000in}{0.000000in}}{%
\pgfpathmoveto{\pgfqpoint{0.000000in}{0.000000in}}%
\pgfpathlineto{\pgfqpoint{0.000000in}{-0.020833in}}%
\pgfusepath{stroke,fill}%
}%
\begin{pgfscope}%
\pgfsys@transformshift{4.641468in}{3.227753in}%
\pgfsys@useobject{currentmarker}{}%
\end{pgfscope}%
\end{pgfscope}%
\begin{pgfscope}%
\pgfpathrectangle{\pgfqpoint{0.481681in}{1.080890in}}{\pgfqpoint{5.785672in}{2.146863in}}%
\pgfusepath{clip}%
\pgfsetrectcap%
\pgfsetroundjoin%
\pgfsetlinewidth{0.100375pt}%
\definecolor{currentstroke}{rgb}{0.827451,0.827451,0.827451}%
\pgfsetstrokecolor{currentstroke}%
\pgfsetdash{}{0pt}%
\pgfpathmoveto{\pgfqpoint{4.743813in}{1.080890in}}%
\pgfpathlineto{\pgfqpoint{4.743813in}{3.227753in}}%
\pgfusepath{stroke}%
\end{pgfscope}%
\begin{pgfscope}%
\pgfsetbuttcap%
\pgfsetroundjoin%
\definecolor{currentfill}{rgb}{0.000000,0.000000,0.000000}%
\pgfsetfillcolor{currentfill}%
\pgfsetlinewidth{0.501875pt}%
\definecolor{currentstroke}{rgb}{0.000000,0.000000,0.000000}%
\pgfsetstrokecolor{currentstroke}%
\pgfsetdash{}{0pt}%
\pgfsys@defobject{currentmarker}{\pgfqpoint{0.000000in}{0.000000in}}{\pgfqpoint{0.000000in}{0.020833in}}{%
\pgfpathmoveto{\pgfqpoint{0.000000in}{0.000000in}}%
\pgfpathlineto{\pgfqpoint{0.000000in}{0.020833in}}%
\pgfusepath{stroke,fill}%
}%
\begin{pgfscope}%
\pgfsys@transformshift{4.743813in}{1.080890in}%
\pgfsys@useobject{currentmarker}{}%
\end{pgfscope}%
\end{pgfscope}%
\begin{pgfscope}%
\pgfsetbuttcap%
\pgfsetroundjoin%
\definecolor{currentfill}{rgb}{0.000000,0.000000,0.000000}%
\pgfsetfillcolor{currentfill}%
\pgfsetlinewidth{0.501875pt}%
\definecolor{currentstroke}{rgb}{0.000000,0.000000,0.000000}%
\pgfsetstrokecolor{currentstroke}%
\pgfsetdash{}{0pt}%
\pgfsys@defobject{currentmarker}{\pgfqpoint{0.000000in}{-0.020833in}}{\pgfqpoint{0.000000in}{0.000000in}}{%
\pgfpathmoveto{\pgfqpoint{0.000000in}{0.000000in}}%
\pgfpathlineto{\pgfqpoint{0.000000in}{-0.020833in}}%
\pgfusepath{stroke,fill}%
}%
\begin{pgfscope}%
\pgfsys@transformshift{4.743813in}{3.227753in}%
\pgfsys@useobject{currentmarker}{}%
\end{pgfscope}%
\end{pgfscope}%
\begin{pgfscope}%
\pgfpathrectangle{\pgfqpoint{0.481681in}{1.080890in}}{\pgfqpoint{5.785672in}{2.146863in}}%
\pgfusepath{clip}%
\pgfsetrectcap%
\pgfsetroundjoin%
\pgfsetlinewidth{0.100375pt}%
\definecolor{currentstroke}{rgb}{0.827451,0.827451,0.827451}%
\pgfsetstrokecolor{currentstroke}%
\pgfsetdash{}{0pt}%
\pgfpathmoveto{\pgfqpoint{4.846159in}{1.080890in}}%
\pgfpathlineto{\pgfqpoint{4.846159in}{3.227753in}}%
\pgfusepath{stroke}%
\end{pgfscope}%
\begin{pgfscope}%
\pgfsetbuttcap%
\pgfsetroundjoin%
\definecolor{currentfill}{rgb}{0.000000,0.000000,0.000000}%
\pgfsetfillcolor{currentfill}%
\pgfsetlinewidth{0.501875pt}%
\definecolor{currentstroke}{rgb}{0.000000,0.000000,0.000000}%
\pgfsetstrokecolor{currentstroke}%
\pgfsetdash{}{0pt}%
\pgfsys@defobject{currentmarker}{\pgfqpoint{0.000000in}{0.000000in}}{\pgfqpoint{0.000000in}{0.020833in}}{%
\pgfpathmoveto{\pgfqpoint{0.000000in}{0.000000in}}%
\pgfpathlineto{\pgfqpoint{0.000000in}{0.020833in}}%
\pgfusepath{stroke,fill}%
}%
\begin{pgfscope}%
\pgfsys@transformshift{4.846159in}{1.080890in}%
\pgfsys@useobject{currentmarker}{}%
\end{pgfscope}%
\end{pgfscope}%
\begin{pgfscope}%
\pgfsetbuttcap%
\pgfsetroundjoin%
\definecolor{currentfill}{rgb}{0.000000,0.000000,0.000000}%
\pgfsetfillcolor{currentfill}%
\pgfsetlinewidth{0.501875pt}%
\definecolor{currentstroke}{rgb}{0.000000,0.000000,0.000000}%
\pgfsetstrokecolor{currentstroke}%
\pgfsetdash{}{0pt}%
\pgfsys@defobject{currentmarker}{\pgfqpoint{0.000000in}{-0.020833in}}{\pgfqpoint{0.000000in}{0.000000in}}{%
\pgfpathmoveto{\pgfqpoint{0.000000in}{0.000000in}}%
\pgfpathlineto{\pgfqpoint{0.000000in}{-0.020833in}}%
\pgfusepath{stroke,fill}%
}%
\begin{pgfscope}%
\pgfsys@transformshift{4.846159in}{3.227753in}%
\pgfsys@useobject{currentmarker}{}%
\end{pgfscope}%
\end{pgfscope}%
\begin{pgfscope}%
\pgfpathrectangle{\pgfqpoint{0.481681in}{1.080890in}}{\pgfqpoint{5.785672in}{2.146863in}}%
\pgfusepath{clip}%
\pgfsetrectcap%
\pgfsetroundjoin%
\pgfsetlinewidth{0.100375pt}%
\definecolor{currentstroke}{rgb}{0.827451,0.827451,0.827451}%
\pgfsetstrokecolor{currentstroke}%
\pgfsetdash{}{0pt}%
\pgfpathmoveto{\pgfqpoint{4.948504in}{1.080890in}}%
\pgfpathlineto{\pgfqpoint{4.948504in}{3.227753in}}%
\pgfusepath{stroke}%
\end{pgfscope}%
\begin{pgfscope}%
\pgfsetbuttcap%
\pgfsetroundjoin%
\definecolor{currentfill}{rgb}{0.000000,0.000000,0.000000}%
\pgfsetfillcolor{currentfill}%
\pgfsetlinewidth{0.501875pt}%
\definecolor{currentstroke}{rgb}{0.000000,0.000000,0.000000}%
\pgfsetstrokecolor{currentstroke}%
\pgfsetdash{}{0pt}%
\pgfsys@defobject{currentmarker}{\pgfqpoint{0.000000in}{0.000000in}}{\pgfqpoint{0.000000in}{0.020833in}}{%
\pgfpathmoveto{\pgfqpoint{0.000000in}{0.000000in}}%
\pgfpathlineto{\pgfqpoint{0.000000in}{0.020833in}}%
\pgfusepath{stroke,fill}%
}%
\begin{pgfscope}%
\pgfsys@transformshift{4.948504in}{1.080890in}%
\pgfsys@useobject{currentmarker}{}%
\end{pgfscope}%
\end{pgfscope}%
\begin{pgfscope}%
\pgfsetbuttcap%
\pgfsetroundjoin%
\definecolor{currentfill}{rgb}{0.000000,0.000000,0.000000}%
\pgfsetfillcolor{currentfill}%
\pgfsetlinewidth{0.501875pt}%
\definecolor{currentstroke}{rgb}{0.000000,0.000000,0.000000}%
\pgfsetstrokecolor{currentstroke}%
\pgfsetdash{}{0pt}%
\pgfsys@defobject{currentmarker}{\pgfqpoint{0.000000in}{-0.020833in}}{\pgfqpoint{0.000000in}{0.000000in}}{%
\pgfpathmoveto{\pgfqpoint{0.000000in}{0.000000in}}%
\pgfpathlineto{\pgfqpoint{0.000000in}{-0.020833in}}%
\pgfusepath{stroke,fill}%
}%
\begin{pgfscope}%
\pgfsys@transformshift{4.948504in}{3.227753in}%
\pgfsys@useobject{currentmarker}{}%
\end{pgfscope}%
\end{pgfscope}%
\begin{pgfscope}%
\pgfpathrectangle{\pgfqpoint{0.481681in}{1.080890in}}{\pgfqpoint{5.785672in}{2.146863in}}%
\pgfusepath{clip}%
\pgfsetrectcap%
\pgfsetroundjoin%
\pgfsetlinewidth{0.100375pt}%
\definecolor{currentstroke}{rgb}{0.827451,0.827451,0.827451}%
\pgfsetstrokecolor{currentstroke}%
\pgfsetdash{}{0pt}%
\pgfpathmoveto{\pgfqpoint{5.050850in}{1.080890in}}%
\pgfpathlineto{\pgfqpoint{5.050850in}{3.227753in}}%
\pgfusepath{stroke}%
\end{pgfscope}%
\begin{pgfscope}%
\pgfsetbuttcap%
\pgfsetroundjoin%
\definecolor{currentfill}{rgb}{0.000000,0.000000,0.000000}%
\pgfsetfillcolor{currentfill}%
\pgfsetlinewidth{0.501875pt}%
\definecolor{currentstroke}{rgb}{0.000000,0.000000,0.000000}%
\pgfsetstrokecolor{currentstroke}%
\pgfsetdash{}{0pt}%
\pgfsys@defobject{currentmarker}{\pgfqpoint{0.000000in}{0.000000in}}{\pgfqpoint{0.000000in}{0.020833in}}{%
\pgfpathmoveto{\pgfqpoint{0.000000in}{0.000000in}}%
\pgfpathlineto{\pgfqpoint{0.000000in}{0.020833in}}%
\pgfusepath{stroke,fill}%
}%
\begin{pgfscope}%
\pgfsys@transformshift{5.050850in}{1.080890in}%
\pgfsys@useobject{currentmarker}{}%
\end{pgfscope}%
\end{pgfscope}%
\begin{pgfscope}%
\pgfsetbuttcap%
\pgfsetroundjoin%
\definecolor{currentfill}{rgb}{0.000000,0.000000,0.000000}%
\pgfsetfillcolor{currentfill}%
\pgfsetlinewidth{0.501875pt}%
\definecolor{currentstroke}{rgb}{0.000000,0.000000,0.000000}%
\pgfsetstrokecolor{currentstroke}%
\pgfsetdash{}{0pt}%
\pgfsys@defobject{currentmarker}{\pgfqpoint{0.000000in}{-0.020833in}}{\pgfqpoint{0.000000in}{0.000000in}}{%
\pgfpathmoveto{\pgfqpoint{0.000000in}{0.000000in}}%
\pgfpathlineto{\pgfqpoint{0.000000in}{-0.020833in}}%
\pgfusepath{stroke,fill}%
}%
\begin{pgfscope}%
\pgfsys@transformshift{5.050850in}{3.227753in}%
\pgfsys@useobject{currentmarker}{}%
\end{pgfscope}%
\end{pgfscope}%
\begin{pgfscope}%
\pgfpathrectangle{\pgfqpoint{0.481681in}{1.080890in}}{\pgfqpoint{5.785672in}{2.146863in}}%
\pgfusepath{clip}%
\pgfsetrectcap%
\pgfsetroundjoin%
\pgfsetlinewidth{0.100375pt}%
\definecolor{currentstroke}{rgb}{0.827451,0.827451,0.827451}%
\pgfsetstrokecolor{currentstroke}%
\pgfsetdash{}{0pt}%
\pgfpathmoveto{\pgfqpoint{5.255541in}{1.080890in}}%
\pgfpathlineto{\pgfqpoint{5.255541in}{3.227753in}}%
\pgfusepath{stroke}%
\end{pgfscope}%
\begin{pgfscope}%
\pgfsetbuttcap%
\pgfsetroundjoin%
\definecolor{currentfill}{rgb}{0.000000,0.000000,0.000000}%
\pgfsetfillcolor{currentfill}%
\pgfsetlinewidth{0.501875pt}%
\definecolor{currentstroke}{rgb}{0.000000,0.000000,0.000000}%
\pgfsetstrokecolor{currentstroke}%
\pgfsetdash{}{0pt}%
\pgfsys@defobject{currentmarker}{\pgfqpoint{0.000000in}{0.000000in}}{\pgfqpoint{0.000000in}{0.020833in}}{%
\pgfpathmoveto{\pgfqpoint{0.000000in}{0.000000in}}%
\pgfpathlineto{\pgfqpoint{0.000000in}{0.020833in}}%
\pgfusepath{stroke,fill}%
}%
\begin{pgfscope}%
\pgfsys@transformshift{5.255541in}{1.080890in}%
\pgfsys@useobject{currentmarker}{}%
\end{pgfscope}%
\end{pgfscope}%
\begin{pgfscope}%
\pgfsetbuttcap%
\pgfsetroundjoin%
\definecolor{currentfill}{rgb}{0.000000,0.000000,0.000000}%
\pgfsetfillcolor{currentfill}%
\pgfsetlinewidth{0.501875pt}%
\definecolor{currentstroke}{rgb}{0.000000,0.000000,0.000000}%
\pgfsetstrokecolor{currentstroke}%
\pgfsetdash{}{0pt}%
\pgfsys@defobject{currentmarker}{\pgfqpoint{0.000000in}{-0.020833in}}{\pgfqpoint{0.000000in}{0.000000in}}{%
\pgfpathmoveto{\pgfqpoint{0.000000in}{0.000000in}}%
\pgfpathlineto{\pgfqpoint{0.000000in}{-0.020833in}}%
\pgfusepath{stroke,fill}%
}%
\begin{pgfscope}%
\pgfsys@transformshift{5.255541in}{3.227753in}%
\pgfsys@useobject{currentmarker}{}%
\end{pgfscope}%
\end{pgfscope}%
\begin{pgfscope}%
\pgfpathrectangle{\pgfqpoint{0.481681in}{1.080890in}}{\pgfqpoint{5.785672in}{2.146863in}}%
\pgfusepath{clip}%
\pgfsetrectcap%
\pgfsetroundjoin%
\pgfsetlinewidth{0.100375pt}%
\definecolor{currentstroke}{rgb}{0.827451,0.827451,0.827451}%
\pgfsetstrokecolor{currentstroke}%
\pgfsetdash{}{0pt}%
\pgfpathmoveto{\pgfqpoint{5.357886in}{1.080890in}}%
\pgfpathlineto{\pgfqpoint{5.357886in}{3.227753in}}%
\pgfusepath{stroke}%
\end{pgfscope}%
\begin{pgfscope}%
\pgfsetbuttcap%
\pgfsetroundjoin%
\definecolor{currentfill}{rgb}{0.000000,0.000000,0.000000}%
\pgfsetfillcolor{currentfill}%
\pgfsetlinewidth{0.501875pt}%
\definecolor{currentstroke}{rgb}{0.000000,0.000000,0.000000}%
\pgfsetstrokecolor{currentstroke}%
\pgfsetdash{}{0pt}%
\pgfsys@defobject{currentmarker}{\pgfqpoint{0.000000in}{0.000000in}}{\pgfqpoint{0.000000in}{0.020833in}}{%
\pgfpathmoveto{\pgfqpoint{0.000000in}{0.000000in}}%
\pgfpathlineto{\pgfqpoint{0.000000in}{0.020833in}}%
\pgfusepath{stroke,fill}%
}%
\begin{pgfscope}%
\pgfsys@transformshift{5.357886in}{1.080890in}%
\pgfsys@useobject{currentmarker}{}%
\end{pgfscope}%
\end{pgfscope}%
\begin{pgfscope}%
\pgfsetbuttcap%
\pgfsetroundjoin%
\definecolor{currentfill}{rgb}{0.000000,0.000000,0.000000}%
\pgfsetfillcolor{currentfill}%
\pgfsetlinewidth{0.501875pt}%
\definecolor{currentstroke}{rgb}{0.000000,0.000000,0.000000}%
\pgfsetstrokecolor{currentstroke}%
\pgfsetdash{}{0pt}%
\pgfsys@defobject{currentmarker}{\pgfqpoint{0.000000in}{-0.020833in}}{\pgfqpoint{0.000000in}{0.000000in}}{%
\pgfpathmoveto{\pgfqpoint{0.000000in}{0.000000in}}%
\pgfpathlineto{\pgfqpoint{0.000000in}{-0.020833in}}%
\pgfusepath{stroke,fill}%
}%
\begin{pgfscope}%
\pgfsys@transformshift{5.357886in}{3.227753in}%
\pgfsys@useobject{currentmarker}{}%
\end{pgfscope}%
\end{pgfscope}%
\begin{pgfscope}%
\pgfpathrectangle{\pgfqpoint{0.481681in}{1.080890in}}{\pgfqpoint{5.785672in}{2.146863in}}%
\pgfusepath{clip}%
\pgfsetrectcap%
\pgfsetroundjoin%
\pgfsetlinewidth{0.100375pt}%
\definecolor{currentstroke}{rgb}{0.827451,0.827451,0.827451}%
\pgfsetstrokecolor{currentstroke}%
\pgfsetdash{}{0pt}%
\pgfpathmoveto{\pgfqpoint{5.460231in}{1.080890in}}%
\pgfpathlineto{\pgfqpoint{5.460231in}{3.227753in}}%
\pgfusepath{stroke}%
\end{pgfscope}%
\begin{pgfscope}%
\pgfsetbuttcap%
\pgfsetroundjoin%
\definecolor{currentfill}{rgb}{0.000000,0.000000,0.000000}%
\pgfsetfillcolor{currentfill}%
\pgfsetlinewidth{0.501875pt}%
\definecolor{currentstroke}{rgb}{0.000000,0.000000,0.000000}%
\pgfsetstrokecolor{currentstroke}%
\pgfsetdash{}{0pt}%
\pgfsys@defobject{currentmarker}{\pgfqpoint{0.000000in}{0.000000in}}{\pgfqpoint{0.000000in}{0.020833in}}{%
\pgfpathmoveto{\pgfqpoint{0.000000in}{0.000000in}}%
\pgfpathlineto{\pgfqpoint{0.000000in}{0.020833in}}%
\pgfusepath{stroke,fill}%
}%
\begin{pgfscope}%
\pgfsys@transformshift{5.460231in}{1.080890in}%
\pgfsys@useobject{currentmarker}{}%
\end{pgfscope}%
\end{pgfscope}%
\begin{pgfscope}%
\pgfsetbuttcap%
\pgfsetroundjoin%
\definecolor{currentfill}{rgb}{0.000000,0.000000,0.000000}%
\pgfsetfillcolor{currentfill}%
\pgfsetlinewidth{0.501875pt}%
\definecolor{currentstroke}{rgb}{0.000000,0.000000,0.000000}%
\pgfsetstrokecolor{currentstroke}%
\pgfsetdash{}{0pt}%
\pgfsys@defobject{currentmarker}{\pgfqpoint{0.000000in}{-0.020833in}}{\pgfqpoint{0.000000in}{0.000000in}}{%
\pgfpathmoveto{\pgfqpoint{0.000000in}{0.000000in}}%
\pgfpathlineto{\pgfqpoint{0.000000in}{-0.020833in}}%
\pgfusepath{stroke,fill}%
}%
\begin{pgfscope}%
\pgfsys@transformshift{5.460231in}{3.227753in}%
\pgfsys@useobject{currentmarker}{}%
\end{pgfscope}%
\end{pgfscope}%
\begin{pgfscope}%
\pgfpathrectangle{\pgfqpoint{0.481681in}{1.080890in}}{\pgfqpoint{5.785672in}{2.146863in}}%
\pgfusepath{clip}%
\pgfsetrectcap%
\pgfsetroundjoin%
\pgfsetlinewidth{0.100375pt}%
\definecolor{currentstroke}{rgb}{0.827451,0.827451,0.827451}%
\pgfsetstrokecolor{currentstroke}%
\pgfsetdash{}{0pt}%
\pgfpathmoveto{\pgfqpoint{5.562577in}{1.080890in}}%
\pgfpathlineto{\pgfqpoint{5.562577in}{3.227753in}}%
\pgfusepath{stroke}%
\end{pgfscope}%
\begin{pgfscope}%
\pgfsetbuttcap%
\pgfsetroundjoin%
\definecolor{currentfill}{rgb}{0.000000,0.000000,0.000000}%
\pgfsetfillcolor{currentfill}%
\pgfsetlinewidth{0.501875pt}%
\definecolor{currentstroke}{rgb}{0.000000,0.000000,0.000000}%
\pgfsetstrokecolor{currentstroke}%
\pgfsetdash{}{0pt}%
\pgfsys@defobject{currentmarker}{\pgfqpoint{0.000000in}{0.000000in}}{\pgfqpoint{0.000000in}{0.020833in}}{%
\pgfpathmoveto{\pgfqpoint{0.000000in}{0.000000in}}%
\pgfpathlineto{\pgfqpoint{0.000000in}{0.020833in}}%
\pgfusepath{stroke,fill}%
}%
\begin{pgfscope}%
\pgfsys@transformshift{5.562577in}{1.080890in}%
\pgfsys@useobject{currentmarker}{}%
\end{pgfscope}%
\end{pgfscope}%
\begin{pgfscope}%
\pgfsetbuttcap%
\pgfsetroundjoin%
\definecolor{currentfill}{rgb}{0.000000,0.000000,0.000000}%
\pgfsetfillcolor{currentfill}%
\pgfsetlinewidth{0.501875pt}%
\definecolor{currentstroke}{rgb}{0.000000,0.000000,0.000000}%
\pgfsetstrokecolor{currentstroke}%
\pgfsetdash{}{0pt}%
\pgfsys@defobject{currentmarker}{\pgfqpoint{0.000000in}{-0.020833in}}{\pgfqpoint{0.000000in}{0.000000in}}{%
\pgfpathmoveto{\pgfqpoint{0.000000in}{0.000000in}}%
\pgfpathlineto{\pgfqpoint{0.000000in}{-0.020833in}}%
\pgfusepath{stroke,fill}%
}%
\begin{pgfscope}%
\pgfsys@transformshift{5.562577in}{3.227753in}%
\pgfsys@useobject{currentmarker}{}%
\end{pgfscope}%
\end{pgfscope}%
\begin{pgfscope}%
\pgfpathrectangle{\pgfqpoint{0.481681in}{1.080890in}}{\pgfqpoint{5.785672in}{2.146863in}}%
\pgfusepath{clip}%
\pgfsetrectcap%
\pgfsetroundjoin%
\pgfsetlinewidth{0.100375pt}%
\definecolor{currentstroke}{rgb}{0.827451,0.827451,0.827451}%
\pgfsetstrokecolor{currentstroke}%
\pgfsetdash{}{0pt}%
\pgfpathmoveto{\pgfqpoint{5.664922in}{1.080890in}}%
\pgfpathlineto{\pgfqpoint{5.664922in}{3.227753in}}%
\pgfusepath{stroke}%
\end{pgfscope}%
\begin{pgfscope}%
\pgfsetbuttcap%
\pgfsetroundjoin%
\definecolor{currentfill}{rgb}{0.000000,0.000000,0.000000}%
\pgfsetfillcolor{currentfill}%
\pgfsetlinewidth{0.501875pt}%
\definecolor{currentstroke}{rgb}{0.000000,0.000000,0.000000}%
\pgfsetstrokecolor{currentstroke}%
\pgfsetdash{}{0pt}%
\pgfsys@defobject{currentmarker}{\pgfqpoint{0.000000in}{0.000000in}}{\pgfqpoint{0.000000in}{0.020833in}}{%
\pgfpathmoveto{\pgfqpoint{0.000000in}{0.000000in}}%
\pgfpathlineto{\pgfqpoint{0.000000in}{0.020833in}}%
\pgfusepath{stroke,fill}%
}%
\begin{pgfscope}%
\pgfsys@transformshift{5.664922in}{1.080890in}%
\pgfsys@useobject{currentmarker}{}%
\end{pgfscope}%
\end{pgfscope}%
\begin{pgfscope}%
\pgfsetbuttcap%
\pgfsetroundjoin%
\definecolor{currentfill}{rgb}{0.000000,0.000000,0.000000}%
\pgfsetfillcolor{currentfill}%
\pgfsetlinewidth{0.501875pt}%
\definecolor{currentstroke}{rgb}{0.000000,0.000000,0.000000}%
\pgfsetstrokecolor{currentstroke}%
\pgfsetdash{}{0pt}%
\pgfsys@defobject{currentmarker}{\pgfqpoint{0.000000in}{-0.020833in}}{\pgfqpoint{0.000000in}{0.000000in}}{%
\pgfpathmoveto{\pgfqpoint{0.000000in}{0.000000in}}%
\pgfpathlineto{\pgfqpoint{0.000000in}{-0.020833in}}%
\pgfusepath{stroke,fill}%
}%
\begin{pgfscope}%
\pgfsys@transformshift{5.664922in}{3.227753in}%
\pgfsys@useobject{currentmarker}{}%
\end{pgfscope}%
\end{pgfscope}%
\begin{pgfscope}%
\pgfpathrectangle{\pgfqpoint{0.481681in}{1.080890in}}{\pgfqpoint{5.785672in}{2.146863in}}%
\pgfusepath{clip}%
\pgfsetrectcap%
\pgfsetroundjoin%
\pgfsetlinewidth{0.100375pt}%
\definecolor{currentstroke}{rgb}{0.827451,0.827451,0.827451}%
\pgfsetstrokecolor{currentstroke}%
\pgfsetdash{}{0pt}%
\pgfpathmoveto{\pgfqpoint{5.869613in}{1.080890in}}%
\pgfpathlineto{\pgfqpoint{5.869613in}{3.227753in}}%
\pgfusepath{stroke}%
\end{pgfscope}%
\begin{pgfscope}%
\pgfsetbuttcap%
\pgfsetroundjoin%
\definecolor{currentfill}{rgb}{0.000000,0.000000,0.000000}%
\pgfsetfillcolor{currentfill}%
\pgfsetlinewidth{0.501875pt}%
\definecolor{currentstroke}{rgb}{0.000000,0.000000,0.000000}%
\pgfsetstrokecolor{currentstroke}%
\pgfsetdash{}{0pt}%
\pgfsys@defobject{currentmarker}{\pgfqpoint{0.000000in}{0.000000in}}{\pgfqpoint{0.000000in}{0.020833in}}{%
\pgfpathmoveto{\pgfqpoint{0.000000in}{0.000000in}}%
\pgfpathlineto{\pgfqpoint{0.000000in}{0.020833in}}%
\pgfusepath{stroke,fill}%
}%
\begin{pgfscope}%
\pgfsys@transformshift{5.869613in}{1.080890in}%
\pgfsys@useobject{currentmarker}{}%
\end{pgfscope}%
\end{pgfscope}%
\begin{pgfscope}%
\pgfsetbuttcap%
\pgfsetroundjoin%
\definecolor{currentfill}{rgb}{0.000000,0.000000,0.000000}%
\pgfsetfillcolor{currentfill}%
\pgfsetlinewidth{0.501875pt}%
\definecolor{currentstroke}{rgb}{0.000000,0.000000,0.000000}%
\pgfsetstrokecolor{currentstroke}%
\pgfsetdash{}{0pt}%
\pgfsys@defobject{currentmarker}{\pgfqpoint{0.000000in}{-0.020833in}}{\pgfqpoint{0.000000in}{0.000000in}}{%
\pgfpathmoveto{\pgfqpoint{0.000000in}{0.000000in}}%
\pgfpathlineto{\pgfqpoint{0.000000in}{-0.020833in}}%
\pgfusepath{stroke,fill}%
}%
\begin{pgfscope}%
\pgfsys@transformshift{5.869613in}{3.227753in}%
\pgfsys@useobject{currentmarker}{}%
\end{pgfscope}%
\end{pgfscope}%
\begin{pgfscope}%
\pgfpathrectangle{\pgfqpoint{0.481681in}{1.080890in}}{\pgfqpoint{5.785672in}{2.146863in}}%
\pgfusepath{clip}%
\pgfsetrectcap%
\pgfsetroundjoin%
\pgfsetlinewidth{0.100375pt}%
\definecolor{currentstroke}{rgb}{0.827451,0.827451,0.827451}%
\pgfsetstrokecolor{currentstroke}%
\pgfsetdash{}{0pt}%
\pgfpathmoveto{\pgfqpoint{5.971959in}{1.080890in}}%
\pgfpathlineto{\pgfqpoint{5.971959in}{3.227753in}}%
\pgfusepath{stroke}%
\end{pgfscope}%
\begin{pgfscope}%
\pgfsetbuttcap%
\pgfsetroundjoin%
\definecolor{currentfill}{rgb}{0.000000,0.000000,0.000000}%
\pgfsetfillcolor{currentfill}%
\pgfsetlinewidth{0.501875pt}%
\definecolor{currentstroke}{rgb}{0.000000,0.000000,0.000000}%
\pgfsetstrokecolor{currentstroke}%
\pgfsetdash{}{0pt}%
\pgfsys@defobject{currentmarker}{\pgfqpoint{0.000000in}{0.000000in}}{\pgfqpoint{0.000000in}{0.020833in}}{%
\pgfpathmoveto{\pgfqpoint{0.000000in}{0.000000in}}%
\pgfpathlineto{\pgfqpoint{0.000000in}{0.020833in}}%
\pgfusepath{stroke,fill}%
}%
\begin{pgfscope}%
\pgfsys@transformshift{5.971959in}{1.080890in}%
\pgfsys@useobject{currentmarker}{}%
\end{pgfscope}%
\end{pgfscope}%
\begin{pgfscope}%
\pgfsetbuttcap%
\pgfsetroundjoin%
\definecolor{currentfill}{rgb}{0.000000,0.000000,0.000000}%
\pgfsetfillcolor{currentfill}%
\pgfsetlinewidth{0.501875pt}%
\definecolor{currentstroke}{rgb}{0.000000,0.000000,0.000000}%
\pgfsetstrokecolor{currentstroke}%
\pgfsetdash{}{0pt}%
\pgfsys@defobject{currentmarker}{\pgfqpoint{0.000000in}{-0.020833in}}{\pgfqpoint{0.000000in}{0.000000in}}{%
\pgfpathmoveto{\pgfqpoint{0.000000in}{0.000000in}}%
\pgfpathlineto{\pgfqpoint{0.000000in}{-0.020833in}}%
\pgfusepath{stroke,fill}%
}%
\begin{pgfscope}%
\pgfsys@transformshift{5.971959in}{3.227753in}%
\pgfsys@useobject{currentmarker}{}%
\end{pgfscope}%
\end{pgfscope}%
\begin{pgfscope}%
\pgfpathrectangle{\pgfqpoint{0.481681in}{1.080890in}}{\pgfqpoint{5.785672in}{2.146863in}}%
\pgfusepath{clip}%
\pgfsetrectcap%
\pgfsetroundjoin%
\pgfsetlinewidth{0.100375pt}%
\definecolor{currentstroke}{rgb}{0.827451,0.827451,0.827451}%
\pgfsetstrokecolor{currentstroke}%
\pgfsetdash{}{0pt}%
\pgfpathmoveto{\pgfqpoint{6.074304in}{1.080890in}}%
\pgfpathlineto{\pgfqpoint{6.074304in}{3.227753in}}%
\pgfusepath{stroke}%
\end{pgfscope}%
\begin{pgfscope}%
\pgfsetbuttcap%
\pgfsetroundjoin%
\definecolor{currentfill}{rgb}{0.000000,0.000000,0.000000}%
\pgfsetfillcolor{currentfill}%
\pgfsetlinewidth{0.501875pt}%
\definecolor{currentstroke}{rgb}{0.000000,0.000000,0.000000}%
\pgfsetstrokecolor{currentstroke}%
\pgfsetdash{}{0pt}%
\pgfsys@defobject{currentmarker}{\pgfqpoint{0.000000in}{0.000000in}}{\pgfqpoint{0.000000in}{0.020833in}}{%
\pgfpathmoveto{\pgfqpoint{0.000000in}{0.000000in}}%
\pgfpathlineto{\pgfqpoint{0.000000in}{0.020833in}}%
\pgfusepath{stroke,fill}%
}%
\begin{pgfscope}%
\pgfsys@transformshift{6.074304in}{1.080890in}%
\pgfsys@useobject{currentmarker}{}%
\end{pgfscope}%
\end{pgfscope}%
\begin{pgfscope}%
\pgfsetbuttcap%
\pgfsetroundjoin%
\definecolor{currentfill}{rgb}{0.000000,0.000000,0.000000}%
\pgfsetfillcolor{currentfill}%
\pgfsetlinewidth{0.501875pt}%
\definecolor{currentstroke}{rgb}{0.000000,0.000000,0.000000}%
\pgfsetstrokecolor{currentstroke}%
\pgfsetdash{}{0pt}%
\pgfsys@defobject{currentmarker}{\pgfqpoint{0.000000in}{-0.020833in}}{\pgfqpoint{0.000000in}{0.000000in}}{%
\pgfpathmoveto{\pgfqpoint{0.000000in}{0.000000in}}%
\pgfpathlineto{\pgfqpoint{0.000000in}{-0.020833in}}%
\pgfusepath{stroke,fill}%
}%
\begin{pgfscope}%
\pgfsys@transformshift{6.074304in}{3.227753in}%
\pgfsys@useobject{currentmarker}{}%
\end{pgfscope}%
\end{pgfscope}%
\begin{pgfscope}%
\pgfpathrectangle{\pgfqpoint{0.481681in}{1.080890in}}{\pgfqpoint{5.785672in}{2.146863in}}%
\pgfusepath{clip}%
\pgfsetrectcap%
\pgfsetroundjoin%
\pgfsetlinewidth{0.100375pt}%
\definecolor{currentstroke}{rgb}{0.827451,0.827451,0.827451}%
\pgfsetstrokecolor{currentstroke}%
\pgfsetdash{}{0pt}%
\pgfpathmoveto{\pgfqpoint{6.176649in}{1.080890in}}%
\pgfpathlineto{\pgfqpoint{6.176649in}{3.227753in}}%
\pgfusepath{stroke}%
\end{pgfscope}%
\begin{pgfscope}%
\pgfsetbuttcap%
\pgfsetroundjoin%
\definecolor{currentfill}{rgb}{0.000000,0.000000,0.000000}%
\pgfsetfillcolor{currentfill}%
\pgfsetlinewidth{0.501875pt}%
\definecolor{currentstroke}{rgb}{0.000000,0.000000,0.000000}%
\pgfsetstrokecolor{currentstroke}%
\pgfsetdash{}{0pt}%
\pgfsys@defobject{currentmarker}{\pgfqpoint{0.000000in}{0.000000in}}{\pgfqpoint{0.000000in}{0.020833in}}{%
\pgfpathmoveto{\pgfqpoint{0.000000in}{0.000000in}}%
\pgfpathlineto{\pgfqpoint{0.000000in}{0.020833in}}%
\pgfusepath{stroke,fill}%
}%
\begin{pgfscope}%
\pgfsys@transformshift{6.176649in}{1.080890in}%
\pgfsys@useobject{currentmarker}{}%
\end{pgfscope}%
\end{pgfscope}%
\begin{pgfscope}%
\pgfsetbuttcap%
\pgfsetroundjoin%
\definecolor{currentfill}{rgb}{0.000000,0.000000,0.000000}%
\pgfsetfillcolor{currentfill}%
\pgfsetlinewidth{0.501875pt}%
\definecolor{currentstroke}{rgb}{0.000000,0.000000,0.000000}%
\pgfsetstrokecolor{currentstroke}%
\pgfsetdash{}{0pt}%
\pgfsys@defobject{currentmarker}{\pgfqpoint{0.000000in}{-0.020833in}}{\pgfqpoint{0.000000in}{0.000000in}}{%
\pgfpathmoveto{\pgfqpoint{0.000000in}{0.000000in}}%
\pgfpathlineto{\pgfqpoint{0.000000in}{-0.020833in}}%
\pgfusepath{stroke,fill}%
}%
\begin{pgfscope}%
\pgfsys@transformshift{6.176649in}{3.227753in}%
\pgfsys@useobject{currentmarker}{}%
\end{pgfscope}%
\end{pgfscope}%
\begin{pgfscope}%
\pgfsetbuttcap%
\pgfsetroundjoin%
\definecolor{currentfill}{rgb}{0.000000,0.000000,0.000000}%
\pgfsetfillcolor{currentfill}%
\pgfsetlinewidth{0.501875pt}%
\definecolor{currentstroke}{rgb}{0.000000,0.000000,0.000000}%
\pgfsetstrokecolor{currentstroke}%
\pgfsetdash{}{0pt}%
\pgfsys@defobject{currentmarker}{\pgfqpoint{0.000000in}{0.000000in}}{\pgfqpoint{0.041667in}{0.000000in}}{%
\pgfpathmoveto{\pgfqpoint{0.000000in}{0.000000in}}%
\pgfpathlineto{\pgfqpoint{0.041667in}{0.000000in}}%
\pgfusepath{stroke,fill}%
}%
\begin{pgfscope}%
\pgfsys@transformshift{0.481681in}{1.178475in}%
\pgfsys@useobject{currentmarker}{}%
\end{pgfscope}%
\end{pgfscope}%
\begin{pgfscope}%
\pgfsetbuttcap%
\pgfsetroundjoin%
\definecolor{currentfill}{rgb}{0.000000,0.000000,0.000000}%
\pgfsetfillcolor{currentfill}%
\pgfsetlinewidth{0.501875pt}%
\definecolor{currentstroke}{rgb}{0.000000,0.000000,0.000000}%
\pgfsetstrokecolor{currentstroke}%
\pgfsetdash{}{0pt}%
\pgfsys@defobject{currentmarker}{\pgfqpoint{-0.041667in}{0.000000in}}{\pgfqpoint{-0.000000in}{0.000000in}}{%
\pgfpathmoveto{\pgfqpoint{-0.000000in}{0.000000in}}%
\pgfpathlineto{\pgfqpoint{-0.041667in}{0.000000in}}%
\pgfusepath{stroke,fill}%
}%
\begin{pgfscope}%
\pgfsys@transformshift{6.267353in}{1.178475in}%
\pgfsys@useobject{currentmarker}{}%
\end{pgfscope}%
\end{pgfscope}%
\begin{pgfscope}%
\definecolor{textcolor}{rgb}{0.000000,0.000000,0.000000}%
\pgfsetstrokecolor{textcolor}%
\pgfsetfillcolor{textcolor}%
\pgftext[x=0.291028in, y=1.144739in, left, base]{\color{textcolor}\rmfamily\fontsize{7.000000}{8.400000}\selectfont 0.0}%
\end{pgfscope}%
\begin{pgfscope}%
\pgfsetbuttcap%
\pgfsetroundjoin%
\definecolor{currentfill}{rgb}{0.000000,0.000000,0.000000}%
\pgfsetfillcolor{currentfill}%
\pgfsetlinewidth{0.501875pt}%
\definecolor{currentstroke}{rgb}{0.000000,0.000000,0.000000}%
\pgfsetstrokecolor{currentstroke}%
\pgfsetdash{}{0pt}%
\pgfsys@defobject{currentmarker}{\pgfqpoint{0.000000in}{0.000000in}}{\pgfqpoint{0.041667in}{0.000000in}}{%
\pgfpathmoveto{\pgfqpoint{0.000000in}{0.000000in}}%
\pgfpathlineto{\pgfqpoint{0.041667in}{0.000000in}}%
\pgfusepath{stroke,fill}%
}%
\begin{pgfscope}%
\pgfsys@transformshift{0.481681in}{1.567124in}%
\pgfsys@useobject{currentmarker}{}%
\end{pgfscope}%
\end{pgfscope}%
\begin{pgfscope}%
\pgfsetbuttcap%
\pgfsetroundjoin%
\definecolor{currentfill}{rgb}{0.000000,0.000000,0.000000}%
\pgfsetfillcolor{currentfill}%
\pgfsetlinewidth{0.501875pt}%
\definecolor{currentstroke}{rgb}{0.000000,0.000000,0.000000}%
\pgfsetstrokecolor{currentstroke}%
\pgfsetdash{}{0pt}%
\pgfsys@defobject{currentmarker}{\pgfqpoint{-0.041667in}{0.000000in}}{\pgfqpoint{-0.000000in}{0.000000in}}{%
\pgfpathmoveto{\pgfqpoint{-0.000000in}{0.000000in}}%
\pgfpathlineto{\pgfqpoint{-0.041667in}{0.000000in}}%
\pgfusepath{stroke,fill}%
}%
\begin{pgfscope}%
\pgfsys@transformshift{6.267353in}{1.567124in}%
\pgfsys@useobject{currentmarker}{}%
\end{pgfscope}%
\end{pgfscope}%
\begin{pgfscope}%
\definecolor{textcolor}{rgb}{0.000000,0.000000,0.000000}%
\pgfsetstrokecolor{textcolor}%
\pgfsetfillcolor{textcolor}%
\pgftext[x=0.291028in, y=1.533388in, left, base]{\color{textcolor}\rmfamily\fontsize{7.000000}{8.400000}\selectfont 0.2}%
\end{pgfscope}%
\begin{pgfscope}%
\pgfsetbuttcap%
\pgfsetroundjoin%
\definecolor{currentfill}{rgb}{0.000000,0.000000,0.000000}%
\pgfsetfillcolor{currentfill}%
\pgfsetlinewidth{0.501875pt}%
\definecolor{currentstroke}{rgb}{0.000000,0.000000,0.000000}%
\pgfsetstrokecolor{currentstroke}%
\pgfsetdash{}{0pt}%
\pgfsys@defobject{currentmarker}{\pgfqpoint{0.000000in}{0.000000in}}{\pgfqpoint{0.041667in}{0.000000in}}{%
\pgfpathmoveto{\pgfqpoint{0.000000in}{0.000000in}}%
\pgfpathlineto{\pgfqpoint{0.041667in}{0.000000in}}%
\pgfusepath{stroke,fill}%
}%
\begin{pgfscope}%
\pgfsys@transformshift{0.481681in}{1.955772in}%
\pgfsys@useobject{currentmarker}{}%
\end{pgfscope}%
\end{pgfscope}%
\begin{pgfscope}%
\pgfsetbuttcap%
\pgfsetroundjoin%
\definecolor{currentfill}{rgb}{0.000000,0.000000,0.000000}%
\pgfsetfillcolor{currentfill}%
\pgfsetlinewidth{0.501875pt}%
\definecolor{currentstroke}{rgb}{0.000000,0.000000,0.000000}%
\pgfsetstrokecolor{currentstroke}%
\pgfsetdash{}{0pt}%
\pgfsys@defobject{currentmarker}{\pgfqpoint{-0.041667in}{0.000000in}}{\pgfqpoint{-0.000000in}{0.000000in}}{%
\pgfpathmoveto{\pgfqpoint{-0.000000in}{0.000000in}}%
\pgfpathlineto{\pgfqpoint{-0.041667in}{0.000000in}}%
\pgfusepath{stroke,fill}%
}%
\begin{pgfscope}%
\pgfsys@transformshift{6.267353in}{1.955772in}%
\pgfsys@useobject{currentmarker}{}%
\end{pgfscope}%
\end{pgfscope}%
\begin{pgfscope}%
\definecolor{textcolor}{rgb}{0.000000,0.000000,0.000000}%
\pgfsetstrokecolor{textcolor}%
\pgfsetfillcolor{textcolor}%
\pgftext[x=0.291028in, y=1.922036in, left, base]{\color{textcolor}\rmfamily\fontsize{7.000000}{8.400000}\selectfont 0.4}%
\end{pgfscope}%
\begin{pgfscope}%
\pgfsetbuttcap%
\pgfsetroundjoin%
\definecolor{currentfill}{rgb}{0.000000,0.000000,0.000000}%
\pgfsetfillcolor{currentfill}%
\pgfsetlinewidth{0.501875pt}%
\definecolor{currentstroke}{rgb}{0.000000,0.000000,0.000000}%
\pgfsetstrokecolor{currentstroke}%
\pgfsetdash{}{0pt}%
\pgfsys@defobject{currentmarker}{\pgfqpoint{0.000000in}{0.000000in}}{\pgfqpoint{0.041667in}{0.000000in}}{%
\pgfpathmoveto{\pgfqpoint{0.000000in}{0.000000in}}%
\pgfpathlineto{\pgfqpoint{0.041667in}{0.000000in}}%
\pgfusepath{stroke,fill}%
}%
\begin{pgfscope}%
\pgfsys@transformshift{0.481681in}{2.344421in}%
\pgfsys@useobject{currentmarker}{}%
\end{pgfscope}%
\end{pgfscope}%
\begin{pgfscope}%
\pgfsetbuttcap%
\pgfsetroundjoin%
\definecolor{currentfill}{rgb}{0.000000,0.000000,0.000000}%
\pgfsetfillcolor{currentfill}%
\pgfsetlinewidth{0.501875pt}%
\definecolor{currentstroke}{rgb}{0.000000,0.000000,0.000000}%
\pgfsetstrokecolor{currentstroke}%
\pgfsetdash{}{0pt}%
\pgfsys@defobject{currentmarker}{\pgfqpoint{-0.041667in}{0.000000in}}{\pgfqpoint{-0.000000in}{0.000000in}}{%
\pgfpathmoveto{\pgfqpoint{-0.000000in}{0.000000in}}%
\pgfpathlineto{\pgfqpoint{-0.041667in}{0.000000in}}%
\pgfusepath{stroke,fill}%
}%
\begin{pgfscope}%
\pgfsys@transformshift{6.267353in}{2.344421in}%
\pgfsys@useobject{currentmarker}{}%
\end{pgfscope}%
\end{pgfscope}%
\begin{pgfscope}%
\definecolor{textcolor}{rgb}{0.000000,0.000000,0.000000}%
\pgfsetstrokecolor{textcolor}%
\pgfsetfillcolor{textcolor}%
\pgftext[x=0.291028in, y=2.310685in, left, base]{\color{textcolor}\rmfamily\fontsize{7.000000}{8.400000}\selectfont 0.6}%
\end{pgfscope}%
\begin{pgfscope}%
\pgfsetbuttcap%
\pgfsetroundjoin%
\definecolor{currentfill}{rgb}{0.000000,0.000000,0.000000}%
\pgfsetfillcolor{currentfill}%
\pgfsetlinewidth{0.501875pt}%
\definecolor{currentstroke}{rgb}{0.000000,0.000000,0.000000}%
\pgfsetstrokecolor{currentstroke}%
\pgfsetdash{}{0pt}%
\pgfsys@defobject{currentmarker}{\pgfqpoint{0.000000in}{0.000000in}}{\pgfqpoint{0.041667in}{0.000000in}}{%
\pgfpathmoveto{\pgfqpoint{0.000000in}{0.000000in}}%
\pgfpathlineto{\pgfqpoint{0.041667in}{0.000000in}}%
\pgfusepath{stroke,fill}%
}%
\begin{pgfscope}%
\pgfsys@transformshift{0.481681in}{2.733069in}%
\pgfsys@useobject{currentmarker}{}%
\end{pgfscope}%
\end{pgfscope}%
\begin{pgfscope}%
\pgfsetbuttcap%
\pgfsetroundjoin%
\definecolor{currentfill}{rgb}{0.000000,0.000000,0.000000}%
\pgfsetfillcolor{currentfill}%
\pgfsetlinewidth{0.501875pt}%
\definecolor{currentstroke}{rgb}{0.000000,0.000000,0.000000}%
\pgfsetstrokecolor{currentstroke}%
\pgfsetdash{}{0pt}%
\pgfsys@defobject{currentmarker}{\pgfqpoint{-0.041667in}{0.000000in}}{\pgfqpoint{-0.000000in}{0.000000in}}{%
\pgfpathmoveto{\pgfqpoint{-0.000000in}{0.000000in}}%
\pgfpathlineto{\pgfqpoint{-0.041667in}{0.000000in}}%
\pgfusepath{stroke,fill}%
}%
\begin{pgfscope}%
\pgfsys@transformshift{6.267353in}{2.733069in}%
\pgfsys@useobject{currentmarker}{}%
\end{pgfscope}%
\end{pgfscope}%
\begin{pgfscope}%
\definecolor{textcolor}{rgb}{0.000000,0.000000,0.000000}%
\pgfsetstrokecolor{textcolor}%
\pgfsetfillcolor{textcolor}%
\pgftext[x=0.291028in, y=2.699333in, left, base]{\color{textcolor}\rmfamily\fontsize{7.000000}{8.400000}\selectfont 0.8}%
\end{pgfscope}%
\begin{pgfscope}%
\pgfsetbuttcap%
\pgfsetroundjoin%
\definecolor{currentfill}{rgb}{0.000000,0.000000,0.000000}%
\pgfsetfillcolor{currentfill}%
\pgfsetlinewidth{0.501875pt}%
\definecolor{currentstroke}{rgb}{0.000000,0.000000,0.000000}%
\pgfsetstrokecolor{currentstroke}%
\pgfsetdash{}{0pt}%
\pgfsys@defobject{currentmarker}{\pgfqpoint{0.000000in}{0.000000in}}{\pgfqpoint{0.041667in}{0.000000in}}{%
\pgfpathmoveto{\pgfqpoint{0.000000in}{0.000000in}}%
\pgfpathlineto{\pgfqpoint{0.041667in}{0.000000in}}%
\pgfusepath{stroke,fill}%
}%
\begin{pgfscope}%
\pgfsys@transformshift{0.481681in}{3.121718in}%
\pgfsys@useobject{currentmarker}{}%
\end{pgfscope}%
\end{pgfscope}%
\begin{pgfscope}%
\pgfsetbuttcap%
\pgfsetroundjoin%
\definecolor{currentfill}{rgb}{0.000000,0.000000,0.000000}%
\pgfsetfillcolor{currentfill}%
\pgfsetlinewidth{0.501875pt}%
\definecolor{currentstroke}{rgb}{0.000000,0.000000,0.000000}%
\pgfsetstrokecolor{currentstroke}%
\pgfsetdash{}{0pt}%
\pgfsys@defobject{currentmarker}{\pgfqpoint{-0.041667in}{0.000000in}}{\pgfqpoint{-0.000000in}{0.000000in}}{%
\pgfpathmoveto{\pgfqpoint{-0.000000in}{0.000000in}}%
\pgfpathlineto{\pgfqpoint{-0.041667in}{0.000000in}}%
\pgfusepath{stroke,fill}%
}%
\begin{pgfscope}%
\pgfsys@transformshift{6.267353in}{3.121718in}%
\pgfsys@useobject{currentmarker}{}%
\end{pgfscope}%
\end{pgfscope}%
\begin{pgfscope}%
\definecolor{textcolor}{rgb}{0.000000,0.000000,0.000000}%
\pgfsetstrokecolor{textcolor}%
\pgfsetfillcolor{textcolor}%
\pgftext[x=0.291028in, y=3.087982in, left, base]{\color{textcolor}\rmfamily\fontsize{7.000000}{8.400000}\selectfont 1.0}%
\end{pgfscope}%
\begin{pgfscope}%
\pgfsetbuttcap%
\pgfsetroundjoin%
\definecolor{currentfill}{rgb}{0.000000,0.000000,0.000000}%
\pgfsetfillcolor{currentfill}%
\pgfsetlinewidth{0.501875pt}%
\definecolor{currentstroke}{rgb}{0.000000,0.000000,0.000000}%
\pgfsetstrokecolor{currentstroke}%
\pgfsetdash{}{0pt}%
\pgfsys@defobject{currentmarker}{\pgfqpoint{0.000000in}{0.000000in}}{\pgfqpoint{0.020833in}{0.000000in}}{%
\pgfpathmoveto{\pgfqpoint{0.000000in}{0.000000in}}%
\pgfpathlineto{\pgfqpoint{0.020833in}{0.000000in}}%
\pgfusepath{stroke,fill}%
}%
\begin{pgfscope}%
\pgfsys@transformshift{0.481681in}{1.081313in}%
\pgfsys@useobject{currentmarker}{}%
\end{pgfscope}%
\end{pgfscope}%
\begin{pgfscope}%
\pgfsetbuttcap%
\pgfsetroundjoin%
\definecolor{currentfill}{rgb}{0.000000,0.000000,0.000000}%
\pgfsetfillcolor{currentfill}%
\pgfsetlinewidth{0.501875pt}%
\definecolor{currentstroke}{rgb}{0.000000,0.000000,0.000000}%
\pgfsetstrokecolor{currentstroke}%
\pgfsetdash{}{0pt}%
\pgfsys@defobject{currentmarker}{\pgfqpoint{-0.020833in}{0.000000in}}{\pgfqpoint{-0.000000in}{0.000000in}}{%
\pgfpathmoveto{\pgfqpoint{-0.000000in}{0.000000in}}%
\pgfpathlineto{\pgfqpoint{-0.020833in}{0.000000in}}%
\pgfusepath{stroke,fill}%
}%
\begin{pgfscope}%
\pgfsys@transformshift{6.267353in}{1.081313in}%
\pgfsys@useobject{currentmarker}{}%
\end{pgfscope}%
\end{pgfscope}%
\begin{pgfscope}%
\pgfsetbuttcap%
\pgfsetroundjoin%
\definecolor{currentfill}{rgb}{0.000000,0.000000,0.000000}%
\pgfsetfillcolor{currentfill}%
\pgfsetlinewidth{0.501875pt}%
\definecolor{currentstroke}{rgb}{0.000000,0.000000,0.000000}%
\pgfsetstrokecolor{currentstroke}%
\pgfsetdash{}{0pt}%
\pgfsys@defobject{currentmarker}{\pgfqpoint{0.000000in}{0.000000in}}{\pgfqpoint{0.020833in}{0.000000in}}{%
\pgfpathmoveto{\pgfqpoint{0.000000in}{0.000000in}}%
\pgfpathlineto{\pgfqpoint{0.020833in}{0.000000in}}%
\pgfusepath{stroke,fill}%
}%
\begin{pgfscope}%
\pgfsys@transformshift{0.481681in}{1.275637in}%
\pgfsys@useobject{currentmarker}{}%
\end{pgfscope}%
\end{pgfscope}%
\begin{pgfscope}%
\pgfsetbuttcap%
\pgfsetroundjoin%
\definecolor{currentfill}{rgb}{0.000000,0.000000,0.000000}%
\pgfsetfillcolor{currentfill}%
\pgfsetlinewidth{0.501875pt}%
\definecolor{currentstroke}{rgb}{0.000000,0.000000,0.000000}%
\pgfsetstrokecolor{currentstroke}%
\pgfsetdash{}{0pt}%
\pgfsys@defobject{currentmarker}{\pgfqpoint{-0.020833in}{0.000000in}}{\pgfqpoint{-0.000000in}{0.000000in}}{%
\pgfpathmoveto{\pgfqpoint{-0.000000in}{0.000000in}}%
\pgfpathlineto{\pgfqpoint{-0.020833in}{0.000000in}}%
\pgfusepath{stroke,fill}%
}%
\begin{pgfscope}%
\pgfsys@transformshift{6.267353in}{1.275637in}%
\pgfsys@useobject{currentmarker}{}%
\end{pgfscope}%
\end{pgfscope}%
\begin{pgfscope}%
\pgfsetbuttcap%
\pgfsetroundjoin%
\definecolor{currentfill}{rgb}{0.000000,0.000000,0.000000}%
\pgfsetfillcolor{currentfill}%
\pgfsetlinewidth{0.501875pt}%
\definecolor{currentstroke}{rgb}{0.000000,0.000000,0.000000}%
\pgfsetstrokecolor{currentstroke}%
\pgfsetdash{}{0pt}%
\pgfsys@defobject{currentmarker}{\pgfqpoint{0.000000in}{0.000000in}}{\pgfqpoint{0.020833in}{0.000000in}}{%
\pgfpathmoveto{\pgfqpoint{0.000000in}{0.000000in}}%
\pgfpathlineto{\pgfqpoint{0.020833in}{0.000000in}}%
\pgfusepath{stroke,fill}%
}%
\begin{pgfscope}%
\pgfsys@transformshift{0.481681in}{1.372799in}%
\pgfsys@useobject{currentmarker}{}%
\end{pgfscope}%
\end{pgfscope}%
\begin{pgfscope}%
\pgfsetbuttcap%
\pgfsetroundjoin%
\definecolor{currentfill}{rgb}{0.000000,0.000000,0.000000}%
\pgfsetfillcolor{currentfill}%
\pgfsetlinewidth{0.501875pt}%
\definecolor{currentstroke}{rgb}{0.000000,0.000000,0.000000}%
\pgfsetstrokecolor{currentstroke}%
\pgfsetdash{}{0pt}%
\pgfsys@defobject{currentmarker}{\pgfqpoint{-0.020833in}{0.000000in}}{\pgfqpoint{-0.000000in}{0.000000in}}{%
\pgfpathmoveto{\pgfqpoint{-0.000000in}{0.000000in}}%
\pgfpathlineto{\pgfqpoint{-0.020833in}{0.000000in}}%
\pgfusepath{stroke,fill}%
}%
\begin{pgfscope}%
\pgfsys@transformshift{6.267353in}{1.372799in}%
\pgfsys@useobject{currentmarker}{}%
\end{pgfscope}%
\end{pgfscope}%
\begin{pgfscope}%
\pgfsetbuttcap%
\pgfsetroundjoin%
\definecolor{currentfill}{rgb}{0.000000,0.000000,0.000000}%
\pgfsetfillcolor{currentfill}%
\pgfsetlinewidth{0.501875pt}%
\definecolor{currentstroke}{rgb}{0.000000,0.000000,0.000000}%
\pgfsetstrokecolor{currentstroke}%
\pgfsetdash{}{0pt}%
\pgfsys@defobject{currentmarker}{\pgfqpoint{0.000000in}{0.000000in}}{\pgfqpoint{0.020833in}{0.000000in}}{%
\pgfpathmoveto{\pgfqpoint{0.000000in}{0.000000in}}%
\pgfpathlineto{\pgfqpoint{0.020833in}{0.000000in}}%
\pgfusepath{stroke,fill}%
}%
\begin{pgfscope}%
\pgfsys@transformshift{0.481681in}{1.469961in}%
\pgfsys@useobject{currentmarker}{}%
\end{pgfscope}%
\end{pgfscope}%
\begin{pgfscope}%
\pgfsetbuttcap%
\pgfsetroundjoin%
\definecolor{currentfill}{rgb}{0.000000,0.000000,0.000000}%
\pgfsetfillcolor{currentfill}%
\pgfsetlinewidth{0.501875pt}%
\definecolor{currentstroke}{rgb}{0.000000,0.000000,0.000000}%
\pgfsetstrokecolor{currentstroke}%
\pgfsetdash{}{0pt}%
\pgfsys@defobject{currentmarker}{\pgfqpoint{-0.020833in}{0.000000in}}{\pgfqpoint{-0.000000in}{0.000000in}}{%
\pgfpathmoveto{\pgfqpoint{-0.000000in}{0.000000in}}%
\pgfpathlineto{\pgfqpoint{-0.020833in}{0.000000in}}%
\pgfusepath{stroke,fill}%
}%
\begin{pgfscope}%
\pgfsys@transformshift{6.267353in}{1.469961in}%
\pgfsys@useobject{currentmarker}{}%
\end{pgfscope}%
\end{pgfscope}%
\begin{pgfscope}%
\pgfsetbuttcap%
\pgfsetroundjoin%
\definecolor{currentfill}{rgb}{0.000000,0.000000,0.000000}%
\pgfsetfillcolor{currentfill}%
\pgfsetlinewidth{0.501875pt}%
\definecolor{currentstroke}{rgb}{0.000000,0.000000,0.000000}%
\pgfsetstrokecolor{currentstroke}%
\pgfsetdash{}{0pt}%
\pgfsys@defobject{currentmarker}{\pgfqpoint{0.000000in}{0.000000in}}{\pgfqpoint{0.020833in}{0.000000in}}{%
\pgfpathmoveto{\pgfqpoint{0.000000in}{0.000000in}}%
\pgfpathlineto{\pgfqpoint{0.020833in}{0.000000in}}%
\pgfusepath{stroke,fill}%
}%
\begin{pgfscope}%
\pgfsys@transformshift{0.481681in}{1.664286in}%
\pgfsys@useobject{currentmarker}{}%
\end{pgfscope}%
\end{pgfscope}%
\begin{pgfscope}%
\pgfsetbuttcap%
\pgfsetroundjoin%
\definecolor{currentfill}{rgb}{0.000000,0.000000,0.000000}%
\pgfsetfillcolor{currentfill}%
\pgfsetlinewidth{0.501875pt}%
\definecolor{currentstroke}{rgb}{0.000000,0.000000,0.000000}%
\pgfsetstrokecolor{currentstroke}%
\pgfsetdash{}{0pt}%
\pgfsys@defobject{currentmarker}{\pgfqpoint{-0.020833in}{0.000000in}}{\pgfqpoint{-0.000000in}{0.000000in}}{%
\pgfpathmoveto{\pgfqpoint{-0.000000in}{0.000000in}}%
\pgfpathlineto{\pgfqpoint{-0.020833in}{0.000000in}}%
\pgfusepath{stroke,fill}%
}%
\begin{pgfscope}%
\pgfsys@transformshift{6.267353in}{1.664286in}%
\pgfsys@useobject{currentmarker}{}%
\end{pgfscope}%
\end{pgfscope}%
\begin{pgfscope}%
\pgfsetbuttcap%
\pgfsetroundjoin%
\definecolor{currentfill}{rgb}{0.000000,0.000000,0.000000}%
\pgfsetfillcolor{currentfill}%
\pgfsetlinewidth{0.501875pt}%
\definecolor{currentstroke}{rgb}{0.000000,0.000000,0.000000}%
\pgfsetstrokecolor{currentstroke}%
\pgfsetdash{}{0pt}%
\pgfsys@defobject{currentmarker}{\pgfqpoint{0.000000in}{0.000000in}}{\pgfqpoint{0.020833in}{0.000000in}}{%
\pgfpathmoveto{\pgfqpoint{0.000000in}{0.000000in}}%
\pgfpathlineto{\pgfqpoint{0.020833in}{0.000000in}}%
\pgfusepath{stroke,fill}%
}%
\begin{pgfscope}%
\pgfsys@transformshift{0.481681in}{1.761448in}%
\pgfsys@useobject{currentmarker}{}%
\end{pgfscope}%
\end{pgfscope}%
\begin{pgfscope}%
\pgfsetbuttcap%
\pgfsetroundjoin%
\definecolor{currentfill}{rgb}{0.000000,0.000000,0.000000}%
\pgfsetfillcolor{currentfill}%
\pgfsetlinewidth{0.501875pt}%
\definecolor{currentstroke}{rgb}{0.000000,0.000000,0.000000}%
\pgfsetstrokecolor{currentstroke}%
\pgfsetdash{}{0pt}%
\pgfsys@defobject{currentmarker}{\pgfqpoint{-0.020833in}{0.000000in}}{\pgfqpoint{-0.000000in}{0.000000in}}{%
\pgfpathmoveto{\pgfqpoint{-0.000000in}{0.000000in}}%
\pgfpathlineto{\pgfqpoint{-0.020833in}{0.000000in}}%
\pgfusepath{stroke,fill}%
}%
\begin{pgfscope}%
\pgfsys@transformshift{6.267353in}{1.761448in}%
\pgfsys@useobject{currentmarker}{}%
\end{pgfscope}%
\end{pgfscope}%
\begin{pgfscope}%
\pgfsetbuttcap%
\pgfsetroundjoin%
\definecolor{currentfill}{rgb}{0.000000,0.000000,0.000000}%
\pgfsetfillcolor{currentfill}%
\pgfsetlinewidth{0.501875pt}%
\definecolor{currentstroke}{rgb}{0.000000,0.000000,0.000000}%
\pgfsetstrokecolor{currentstroke}%
\pgfsetdash{}{0pt}%
\pgfsys@defobject{currentmarker}{\pgfqpoint{0.000000in}{0.000000in}}{\pgfqpoint{0.020833in}{0.000000in}}{%
\pgfpathmoveto{\pgfqpoint{0.000000in}{0.000000in}}%
\pgfpathlineto{\pgfqpoint{0.020833in}{0.000000in}}%
\pgfusepath{stroke,fill}%
}%
\begin{pgfscope}%
\pgfsys@transformshift{0.481681in}{1.858610in}%
\pgfsys@useobject{currentmarker}{}%
\end{pgfscope}%
\end{pgfscope}%
\begin{pgfscope}%
\pgfsetbuttcap%
\pgfsetroundjoin%
\definecolor{currentfill}{rgb}{0.000000,0.000000,0.000000}%
\pgfsetfillcolor{currentfill}%
\pgfsetlinewidth{0.501875pt}%
\definecolor{currentstroke}{rgb}{0.000000,0.000000,0.000000}%
\pgfsetstrokecolor{currentstroke}%
\pgfsetdash{}{0pt}%
\pgfsys@defobject{currentmarker}{\pgfqpoint{-0.020833in}{0.000000in}}{\pgfqpoint{-0.000000in}{0.000000in}}{%
\pgfpathmoveto{\pgfqpoint{-0.000000in}{0.000000in}}%
\pgfpathlineto{\pgfqpoint{-0.020833in}{0.000000in}}%
\pgfusepath{stroke,fill}%
}%
\begin{pgfscope}%
\pgfsys@transformshift{6.267353in}{1.858610in}%
\pgfsys@useobject{currentmarker}{}%
\end{pgfscope}%
\end{pgfscope}%
\begin{pgfscope}%
\pgfsetbuttcap%
\pgfsetroundjoin%
\definecolor{currentfill}{rgb}{0.000000,0.000000,0.000000}%
\pgfsetfillcolor{currentfill}%
\pgfsetlinewidth{0.501875pt}%
\definecolor{currentstroke}{rgb}{0.000000,0.000000,0.000000}%
\pgfsetstrokecolor{currentstroke}%
\pgfsetdash{}{0pt}%
\pgfsys@defobject{currentmarker}{\pgfqpoint{0.000000in}{0.000000in}}{\pgfqpoint{0.020833in}{0.000000in}}{%
\pgfpathmoveto{\pgfqpoint{0.000000in}{0.000000in}}%
\pgfpathlineto{\pgfqpoint{0.020833in}{0.000000in}}%
\pgfusepath{stroke,fill}%
}%
\begin{pgfscope}%
\pgfsys@transformshift{0.481681in}{2.052934in}%
\pgfsys@useobject{currentmarker}{}%
\end{pgfscope}%
\end{pgfscope}%
\begin{pgfscope}%
\pgfsetbuttcap%
\pgfsetroundjoin%
\definecolor{currentfill}{rgb}{0.000000,0.000000,0.000000}%
\pgfsetfillcolor{currentfill}%
\pgfsetlinewidth{0.501875pt}%
\definecolor{currentstroke}{rgb}{0.000000,0.000000,0.000000}%
\pgfsetstrokecolor{currentstroke}%
\pgfsetdash{}{0pt}%
\pgfsys@defobject{currentmarker}{\pgfqpoint{-0.020833in}{0.000000in}}{\pgfqpoint{-0.000000in}{0.000000in}}{%
\pgfpathmoveto{\pgfqpoint{-0.000000in}{0.000000in}}%
\pgfpathlineto{\pgfqpoint{-0.020833in}{0.000000in}}%
\pgfusepath{stroke,fill}%
}%
\begin{pgfscope}%
\pgfsys@transformshift{6.267353in}{2.052934in}%
\pgfsys@useobject{currentmarker}{}%
\end{pgfscope}%
\end{pgfscope}%
\begin{pgfscope}%
\pgfsetbuttcap%
\pgfsetroundjoin%
\definecolor{currentfill}{rgb}{0.000000,0.000000,0.000000}%
\pgfsetfillcolor{currentfill}%
\pgfsetlinewidth{0.501875pt}%
\definecolor{currentstroke}{rgb}{0.000000,0.000000,0.000000}%
\pgfsetstrokecolor{currentstroke}%
\pgfsetdash{}{0pt}%
\pgfsys@defobject{currentmarker}{\pgfqpoint{0.000000in}{0.000000in}}{\pgfqpoint{0.020833in}{0.000000in}}{%
\pgfpathmoveto{\pgfqpoint{0.000000in}{0.000000in}}%
\pgfpathlineto{\pgfqpoint{0.020833in}{0.000000in}}%
\pgfusepath{stroke,fill}%
}%
\begin{pgfscope}%
\pgfsys@transformshift{0.481681in}{2.150096in}%
\pgfsys@useobject{currentmarker}{}%
\end{pgfscope}%
\end{pgfscope}%
\begin{pgfscope}%
\pgfsetbuttcap%
\pgfsetroundjoin%
\definecolor{currentfill}{rgb}{0.000000,0.000000,0.000000}%
\pgfsetfillcolor{currentfill}%
\pgfsetlinewidth{0.501875pt}%
\definecolor{currentstroke}{rgb}{0.000000,0.000000,0.000000}%
\pgfsetstrokecolor{currentstroke}%
\pgfsetdash{}{0pt}%
\pgfsys@defobject{currentmarker}{\pgfqpoint{-0.020833in}{0.000000in}}{\pgfqpoint{-0.000000in}{0.000000in}}{%
\pgfpathmoveto{\pgfqpoint{-0.000000in}{0.000000in}}%
\pgfpathlineto{\pgfqpoint{-0.020833in}{0.000000in}}%
\pgfusepath{stroke,fill}%
}%
\begin{pgfscope}%
\pgfsys@transformshift{6.267353in}{2.150096in}%
\pgfsys@useobject{currentmarker}{}%
\end{pgfscope}%
\end{pgfscope}%
\begin{pgfscope}%
\pgfsetbuttcap%
\pgfsetroundjoin%
\definecolor{currentfill}{rgb}{0.000000,0.000000,0.000000}%
\pgfsetfillcolor{currentfill}%
\pgfsetlinewidth{0.501875pt}%
\definecolor{currentstroke}{rgb}{0.000000,0.000000,0.000000}%
\pgfsetstrokecolor{currentstroke}%
\pgfsetdash{}{0pt}%
\pgfsys@defobject{currentmarker}{\pgfqpoint{0.000000in}{0.000000in}}{\pgfqpoint{0.020833in}{0.000000in}}{%
\pgfpathmoveto{\pgfqpoint{0.000000in}{0.000000in}}%
\pgfpathlineto{\pgfqpoint{0.020833in}{0.000000in}}%
\pgfusepath{stroke,fill}%
}%
\begin{pgfscope}%
\pgfsys@transformshift{0.481681in}{2.247259in}%
\pgfsys@useobject{currentmarker}{}%
\end{pgfscope}%
\end{pgfscope}%
\begin{pgfscope}%
\pgfsetbuttcap%
\pgfsetroundjoin%
\definecolor{currentfill}{rgb}{0.000000,0.000000,0.000000}%
\pgfsetfillcolor{currentfill}%
\pgfsetlinewidth{0.501875pt}%
\definecolor{currentstroke}{rgb}{0.000000,0.000000,0.000000}%
\pgfsetstrokecolor{currentstroke}%
\pgfsetdash{}{0pt}%
\pgfsys@defobject{currentmarker}{\pgfqpoint{-0.020833in}{0.000000in}}{\pgfqpoint{-0.000000in}{0.000000in}}{%
\pgfpathmoveto{\pgfqpoint{-0.000000in}{0.000000in}}%
\pgfpathlineto{\pgfqpoint{-0.020833in}{0.000000in}}%
\pgfusepath{stroke,fill}%
}%
\begin{pgfscope}%
\pgfsys@transformshift{6.267353in}{2.247259in}%
\pgfsys@useobject{currentmarker}{}%
\end{pgfscope}%
\end{pgfscope}%
\begin{pgfscope}%
\pgfsetbuttcap%
\pgfsetroundjoin%
\definecolor{currentfill}{rgb}{0.000000,0.000000,0.000000}%
\pgfsetfillcolor{currentfill}%
\pgfsetlinewidth{0.501875pt}%
\definecolor{currentstroke}{rgb}{0.000000,0.000000,0.000000}%
\pgfsetstrokecolor{currentstroke}%
\pgfsetdash{}{0pt}%
\pgfsys@defobject{currentmarker}{\pgfqpoint{0.000000in}{0.000000in}}{\pgfqpoint{0.020833in}{0.000000in}}{%
\pgfpathmoveto{\pgfqpoint{0.000000in}{0.000000in}}%
\pgfpathlineto{\pgfqpoint{0.020833in}{0.000000in}}%
\pgfusepath{stroke,fill}%
}%
\begin{pgfscope}%
\pgfsys@transformshift{0.481681in}{2.441583in}%
\pgfsys@useobject{currentmarker}{}%
\end{pgfscope}%
\end{pgfscope}%
\begin{pgfscope}%
\pgfsetbuttcap%
\pgfsetroundjoin%
\definecolor{currentfill}{rgb}{0.000000,0.000000,0.000000}%
\pgfsetfillcolor{currentfill}%
\pgfsetlinewidth{0.501875pt}%
\definecolor{currentstroke}{rgb}{0.000000,0.000000,0.000000}%
\pgfsetstrokecolor{currentstroke}%
\pgfsetdash{}{0pt}%
\pgfsys@defobject{currentmarker}{\pgfqpoint{-0.020833in}{0.000000in}}{\pgfqpoint{-0.000000in}{0.000000in}}{%
\pgfpathmoveto{\pgfqpoint{-0.000000in}{0.000000in}}%
\pgfpathlineto{\pgfqpoint{-0.020833in}{0.000000in}}%
\pgfusepath{stroke,fill}%
}%
\begin{pgfscope}%
\pgfsys@transformshift{6.267353in}{2.441583in}%
\pgfsys@useobject{currentmarker}{}%
\end{pgfscope}%
\end{pgfscope}%
\begin{pgfscope}%
\pgfsetbuttcap%
\pgfsetroundjoin%
\definecolor{currentfill}{rgb}{0.000000,0.000000,0.000000}%
\pgfsetfillcolor{currentfill}%
\pgfsetlinewidth{0.501875pt}%
\definecolor{currentstroke}{rgb}{0.000000,0.000000,0.000000}%
\pgfsetstrokecolor{currentstroke}%
\pgfsetdash{}{0pt}%
\pgfsys@defobject{currentmarker}{\pgfqpoint{0.000000in}{0.000000in}}{\pgfqpoint{0.020833in}{0.000000in}}{%
\pgfpathmoveto{\pgfqpoint{0.000000in}{0.000000in}}%
\pgfpathlineto{\pgfqpoint{0.020833in}{0.000000in}}%
\pgfusepath{stroke,fill}%
}%
\begin{pgfscope}%
\pgfsys@transformshift{0.481681in}{2.538745in}%
\pgfsys@useobject{currentmarker}{}%
\end{pgfscope}%
\end{pgfscope}%
\begin{pgfscope}%
\pgfsetbuttcap%
\pgfsetroundjoin%
\definecolor{currentfill}{rgb}{0.000000,0.000000,0.000000}%
\pgfsetfillcolor{currentfill}%
\pgfsetlinewidth{0.501875pt}%
\definecolor{currentstroke}{rgb}{0.000000,0.000000,0.000000}%
\pgfsetstrokecolor{currentstroke}%
\pgfsetdash{}{0pt}%
\pgfsys@defobject{currentmarker}{\pgfqpoint{-0.020833in}{0.000000in}}{\pgfqpoint{-0.000000in}{0.000000in}}{%
\pgfpathmoveto{\pgfqpoint{-0.000000in}{0.000000in}}%
\pgfpathlineto{\pgfqpoint{-0.020833in}{0.000000in}}%
\pgfusepath{stroke,fill}%
}%
\begin{pgfscope}%
\pgfsys@transformshift{6.267353in}{2.538745in}%
\pgfsys@useobject{currentmarker}{}%
\end{pgfscope}%
\end{pgfscope}%
\begin{pgfscope}%
\pgfsetbuttcap%
\pgfsetroundjoin%
\definecolor{currentfill}{rgb}{0.000000,0.000000,0.000000}%
\pgfsetfillcolor{currentfill}%
\pgfsetlinewidth{0.501875pt}%
\definecolor{currentstroke}{rgb}{0.000000,0.000000,0.000000}%
\pgfsetstrokecolor{currentstroke}%
\pgfsetdash{}{0pt}%
\pgfsys@defobject{currentmarker}{\pgfqpoint{0.000000in}{0.000000in}}{\pgfqpoint{0.020833in}{0.000000in}}{%
\pgfpathmoveto{\pgfqpoint{0.000000in}{0.000000in}}%
\pgfpathlineto{\pgfqpoint{0.020833in}{0.000000in}}%
\pgfusepath{stroke,fill}%
}%
\begin{pgfscope}%
\pgfsys@transformshift{0.481681in}{2.635907in}%
\pgfsys@useobject{currentmarker}{}%
\end{pgfscope}%
\end{pgfscope}%
\begin{pgfscope}%
\pgfsetbuttcap%
\pgfsetroundjoin%
\definecolor{currentfill}{rgb}{0.000000,0.000000,0.000000}%
\pgfsetfillcolor{currentfill}%
\pgfsetlinewidth{0.501875pt}%
\definecolor{currentstroke}{rgb}{0.000000,0.000000,0.000000}%
\pgfsetstrokecolor{currentstroke}%
\pgfsetdash{}{0pt}%
\pgfsys@defobject{currentmarker}{\pgfqpoint{-0.020833in}{0.000000in}}{\pgfqpoint{-0.000000in}{0.000000in}}{%
\pgfpathmoveto{\pgfqpoint{-0.000000in}{0.000000in}}%
\pgfpathlineto{\pgfqpoint{-0.020833in}{0.000000in}}%
\pgfusepath{stroke,fill}%
}%
\begin{pgfscope}%
\pgfsys@transformshift{6.267353in}{2.635907in}%
\pgfsys@useobject{currentmarker}{}%
\end{pgfscope}%
\end{pgfscope}%
\begin{pgfscope}%
\pgfsetbuttcap%
\pgfsetroundjoin%
\definecolor{currentfill}{rgb}{0.000000,0.000000,0.000000}%
\pgfsetfillcolor{currentfill}%
\pgfsetlinewidth{0.501875pt}%
\definecolor{currentstroke}{rgb}{0.000000,0.000000,0.000000}%
\pgfsetstrokecolor{currentstroke}%
\pgfsetdash{}{0pt}%
\pgfsys@defobject{currentmarker}{\pgfqpoint{0.000000in}{0.000000in}}{\pgfqpoint{0.020833in}{0.000000in}}{%
\pgfpathmoveto{\pgfqpoint{0.000000in}{0.000000in}}%
\pgfpathlineto{\pgfqpoint{0.020833in}{0.000000in}}%
\pgfusepath{stroke,fill}%
}%
\begin{pgfscope}%
\pgfsys@transformshift{0.481681in}{2.830231in}%
\pgfsys@useobject{currentmarker}{}%
\end{pgfscope}%
\end{pgfscope}%
\begin{pgfscope}%
\pgfsetbuttcap%
\pgfsetroundjoin%
\definecolor{currentfill}{rgb}{0.000000,0.000000,0.000000}%
\pgfsetfillcolor{currentfill}%
\pgfsetlinewidth{0.501875pt}%
\definecolor{currentstroke}{rgb}{0.000000,0.000000,0.000000}%
\pgfsetstrokecolor{currentstroke}%
\pgfsetdash{}{0pt}%
\pgfsys@defobject{currentmarker}{\pgfqpoint{-0.020833in}{0.000000in}}{\pgfqpoint{-0.000000in}{0.000000in}}{%
\pgfpathmoveto{\pgfqpoint{-0.000000in}{0.000000in}}%
\pgfpathlineto{\pgfqpoint{-0.020833in}{0.000000in}}%
\pgfusepath{stroke,fill}%
}%
\begin{pgfscope}%
\pgfsys@transformshift{6.267353in}{2.830231in}%
\pgfsys@useobject{currentmarker}{}%
\end{pgfscope}%
\end{pgfscope}%
\begin{pgfscope}%
\pgfsetbuttcap%
\pgfsetroundjoin%
\definecolor{currentfill}{rgb}{0.000000,0.000000,0.000000}%
\pgfsetfillcolor{currentfill}%
\pgfsetlinewidth{0.501875pt}%
\definecolor{currentstroke}{rgb}{0.000000,0.000000,0.000000}%
\pgfsetstrokecolor{currentstroke}%
\pgfsetdash{}{0pt}%
\pgfsys@defobject{currentmarker}{\pgfqpoint{0.000000in}{0.000000in}}{\pgfqpoint{0.020833in}{0.000000in}}{%
\pgfpathmoveto{\pgfqpoint{0.000000in}{0.000000in}}%
\pgfpathlineto{\pgfqpoint{0.020833in}{0.000000in}}%
\pgfusepath{stroke,fill}%
}%
\begin{pgfscope}%
\pgfsys@transformshift{0.481681in}{2.927394in}%
\pgfsys@useobject{currentmarker}{}%
\end{pgfscope}%
\end{pgfscope}%
\begin{pgfscope}%
\pgfsetbuttcap%
\pgfsetroundjoin%
\definecolor{currentfill}{rgb}{0.000000,0.000000,0.000000}%
\pgfsetfillcolor{currentfill}%
\pgfsetlinewidth{0.501875pt}%
\definecolor{currentstroke}{rgb}{0.000000,0.000000,0.000000}%
\pgfsetstrokecolor{currentstroke}%
\pgfsetdash{}{0pt}%
\pgfsys@defobject{currentmarker}{\pgfqpoint{-0.020833in}{0.000000in}}{\pgfqpoint{-0.000000in}{0.000000in}}{%
\pgfpathmoveto{\pgfqpoint{-0.000000in}{0.000000in}}%
\pgfpathlineto{\pgfqpoint{-0.020833in}{0.000000in}}%
\pgfusepath{stroke,fill}%
}%
\begin{pgfscope}%
\pgfsys@transformshift{6.267353in}{2.927394in}%
\pgfsys@useobject{currentmarker}{}%
\end{pgfscope}%
\end{pgfscope}%
\begin{pgfscope}%
\pgfsetbuttcap%
\pgfsetroundjoin%
\definecolor{currentfill}{rgb}{0.000000,0.000000,0.000000}%
\pgfsetfillcolor{currentfill}%
\pgfsetlinewidth{0.501875pt}%
\definecolor{currentstroke}{rgb}{0.000000,0.000000,0.000000}%
\pgfsetstrokecolor{currentstroke}%
\pgfsetdash{}{0pt}%
\pgfsys@defobject{currentmarker}{\pgfqpoint{0.000000in}{0.000000in}}{\pgfqpoint{0.020833in}{0.000000in}}{%
\pgfpathmoveto{\pgfqpoint{0.000000in}{0.000000in}}%
\pgfpathlineto{\pgfqpoint{0.020833in}{0.000000in}}%
\pgfusepath{stroke,fill}%
}%
\begin{pgfscope}%
\pgfsys@transformshift{0.481681in}{3.024556in}%
\pgfsys@useobject{currentmarker}{}%
\end{pgfscope}%
\end{pgfscope}%
\begin{pgfscope}%
\pgfsetbuttcap%
\pgfsetroundjoin%
\definecolor{currentfill}{rgb}{0.000000,0.000000,0.000000}%
\pgfsetfillcolor{currentfill}%
\pgfsetlinewidth{0.501875pt}%
\definecolor{currentstroke}{rgb}{0.000000,0.000000,0.000000}%
\pgfsetstrokecolor{currentstroke}%
\pgfsetdash{}{0pt}%
\pgfsys@defobject{currentmarker}{\pgfqpoint{-0.020833in}{0.000000in}}{\pgfqpoint{-0.000000in}{0.000000in}}{%
\pgfpathmoveto{\pgfqpoint{-0.000000in}{0.000000in}}%
\pgfpathlineto{\pgfqpoint{-0.020833in}{0.000000in}}%
\pgfusepath{stroke,fill}%
}%
\begin{pgfscope}%
\pgfsys@transformshift{6.267353in}{3.024556in}%
\pgfsys@useobject{currentmarker}{}%
\end{pgfscope}%
\end{pgfscope}%
\begin{pgfscope}%
\pgfsetbuttcap%
\pgfsetroundjoin%
\definecolor{currentfill}{rgb}{0.000000,0.000000,0.000000}%
\pgfsetfillcolor{currentfill}%
\pgfsetlinewidth{0.501875pt}%
\definecolor{currentstroke}{rgb}{0.000000,0.000000,0.000000}%
\pgfsetstrokecolor{currentstroke}%
\pgfsetdash{}{0pt}%
\pgfsys@defobject{currentmarker}{\pgfqpoint{0.000000in}{0.000000in}}{\pgfqpoint{0.020833in}{0.000000in}}{%
\pgfpathmoveto{\pgfqpoint{0.000000in}{0.000000in}}%
\pgfpathlineto{\pgfqpoint{0.020833in}{0.000000in}}%
\pgfusepath{stroke,fill}%
}%
\begin{pgfscope}%
\pgfsys@transformshift{0.481681in}{3.218880in}%
\pgfsys@useobject{currentmarker}{}%
\end{pgfscope}%
\end{pgfscope}%
\begin{pgfscope}%
\pgfsetbuttcap%
\pgfsetroundjoin%
\definecolor{currentfill}{rgb}{0.000000,0.000000,0.000000}%
\pgfsetfillcolor{currentfill}%
\pgfsetlinewidth{0.501875pt}%
\definecolor{currentstroke}{rgb}{0.000000,0.000000,0.000000}%
\pgfsetstrokecolor{currentstroke}%
\pgfsetdash{}{0pt}%
\pgfsys@defobject{currentmarker}{\pgfqpoint{-0.020833in}{0.000000in}}{\pgfqpoint{-0.000000in}{0.000000in}}{%
\pgfpathmoveto{\pgfqpoint{-0.000000in}{0.000000in}}%
\pgfpathlineto{\pgfqpoint{-0.020833in}{0.000000in}}%
\pgfusepath{stroke,fill}%
}%
\begin{pgfscope}%
\pgfsys@transformshift{6.267353in}{3.218880in}%
\pgfsys@useobject{currentmarker}{}%
\end{pgfscope}%
\end{pgfscope}%
\begin{pgfscope}%
\definecolor{textcolor}{rgb}{0.000000,0.000000,0.000000}%
\pgfsetstrokecolor{textcolor}%
\pgfsetfillcolor{textcolor}%
\pgftext[x=0.235472in,y=2.154322in,,bottom,rotate=90.000000]{\color{textcolor}\rmfamily\fontsize{8.000000}{9.600000}\selectfont Wert}%
\end{pgfscope}%
\begin{pgfscope}%
\pgfpathrectangle{\pgfqpoint{0.481681in}{1.080890in}}{\pgfqpoint{5.785672in}{2.146863in}}%
\pgfusepath{clip}%
\pgfsetrectcap%
\pgfsetroundjoin%
\pgfsetlinewidth{0.401500pt}%
\definecolor{currentstroke}{rgb}{0.000000,0.070588,0.098039}%
\pgfsetstrokecolor{currentstroke}%
\pgfsetdash{}{0pt}%
\pgfpathmoveto{\pgfqpoint{0.744666in}{1.178475in}}%
\pgfpathlineto{\pgfqpoint{0.751489in}{1.179798in}}%
\pgfpathlineto{\pgfqpoint{1.131020in}{1.255133in}}%
\pgfpathlineto{\pgfqpoint{1.335711in}{1.272925in}}%
\pgfpathlineto{\pgfqpoint{1.373237in}{1.289799in}}%
\pgfpathlineto{\pgfqpoint{1.374090in}{1.291847in}}%
\pgfpathlineto{\pgfqpoint{1.374943in}{1.302163in}}%
\pgfpathlineto{\pgfqpoint{1.461937in}{1.313963in}}%
\pgfpathlineto{\pgfqpoint{1.464495in}{1.317010in}}%
\pgfpathlineto{\pgfqpoint{1.516521in}{1.381778in}}%
\pgfpathlineto{\pgfqpoint{1.517374in}{1.410130in}}%
\pgfpathlineto{\pgfqpoint{1.518227in}{1.483318in}}%
\pgfpathlineto{\pgfqpoint{1.519080in}{1.488999in}}%
\pgfpathlineto{\pgfqpoint{1.519933in}{1.490847in}}%
\pgfpathlineto{\pgfqpoint{1.581340in}{1.520021in}}%
\pgfpathlineto{\pgfqpoint{1.756180in}{1.538686in}}%
\pgfpathlineto{\pgfqpoint{1.954901in}{1.567340in}}%
\pgfpathlineto{\pgfqpoint{1.996692in}{1.602819in}}%
\pgfpathlineto{\pgfqpoint{2.010338in}{1.614194in}}%
\pgfpathlineto{\pgfqpoint{2.026542in}{1.615108in}}%
\pgfpathlineto{\pgfqpoint{2.861510in}{1.661638in}}%
\pgfpathlineto{\pgfqpoint{2.981766in}{1.679778in}}%
\pgfpathlineto{\pgfqpoint{3.259805in}{1.740924in}}%
\pgfpathlineto{\pgfqpoint{3.264922in}{1.746183in}}%
\pgfpathlineto{\pgfqpoint{3.319506in}{1.802735in}}%
\pgfpathlineto{\pgfqpoint{3.424410in}{1.832186in}}%
\pgfpathlineto{\pgfqpoint{3.426969in}{1.836403in}}%
\pgfpathlineto{\pgfqpoint{3.451702in}{1.878948in}}%
\pgfpathlineto{\pgfqpoint{3.453408in}{1.879944in}}%
\pgfpathlineto{\pgfqpoint{3.560871in}{1.930524in}}%
\pgfpathlineto{\pgfqpoint{3.588163in}{1.944042in}}%
\pgfpathlineto{\pgfqpoint{3.736564in}{1.961770in}}%
\pgfpathlineto{\pgfqpoint{4.383899in}{1.991250in}}%
\pgfpathlineto{\pgfqpoint{4.393280in}{1.999291in}}%
\pgfpathlineto{\pgfqpoint{4.404368in}{2.008977in}}%
\pgfpathlineto{\pgfqpoint{4.413749in}{2.025042in}}%
\pgfpathlineto{\pgfqpoint{4.448717in}{2.085054in}}%
\pgfpathlineto{\pgfqpoint{4.451276in}{2.087085in}}%
\pgfpathlineto{\pgfqpoint{4.489656in}{2.114851in}}%
\pgfpathlineto{\pgfqpoint{4.696905in}{2.163089in}}%
\pgfpathlineto{\pgfqpoint{4.847012in}{2.192354in}}%
\pgfpathlineto{\pgfqpoint{5.000530in}{2.232324in}}%
\pgfpathlineto{\pgfqpoint{5.002236in}{2.287459in}}%
\pgfpathlineto{\pgfqpoint{5.017587in}{2.329893in}}%
\pgfpathlineto{\pgfqpoint{5.018440in}{2.333746in}}%
\pgfpathlineto{\pgfqpoint{5.019293in}{2.356179in}}%
\pgfpathlineto{\pgfqpoint{5.020146in}{2.390861in}}%
\pgfpathlineto{\pgfqpoint{5.020999in}{2.444188in}}%
\pgfpathlineto{\pgfqpoint{5.021852in}{2.456569in}}%
\pgfpathlineto{\pgfqpoint{5.143813in}{2.503542in}}%
\pgfpathlineto{\pgfqpoint{5.148078in}{2.509520in}}%
\pgfpathlineto{\pgfqpoint{5.164283in}{2.532698in}}%
\pgfpathlineto{\pgfqpoint{5.165988in}{2.539626in}}%
\pgfpathlineto{\pgfqpoint{5.180487in}{2.604000in}}%
\pgfpathlineto{\pgfqpoint{5.181340in}{2.612895in}}%
\pgfpathlineto{\pgfqpoint{5.182193in}{2.691089in}}%
\pgfpathlineto{\pgfqpoint{5.189016in}{2.762277in}}%
\pgfpathlineto{\pgfqpoint{5.190722in}{2.763157in}}%
\pgfpathlineto{\pgfqpoint{5.328888in}{2.810181in}}%
\pgfpathlineto{\pgfqpoint{5.446585in}{2.839211in}}%
\pgfpathlineto{\pgfqpoint{5.452556in}{2.845998in}}%
\pgfpathlineto{\pgfqpoint{5.473025in}{2.869092in}}%
\pgfpathlineto{\pgfqpoint{5.542108in}{2.916271in}}%
\pgfpathlineto{\pgfqpoint{5.599251in}{2.944656in}}%
\pgfpathlineto{\pgfqpoint{5.600104in}{2.945440in}}%
\pgfpathlineto{\pgfqpoint{5.603515in}{2.956160in}}%
\pgfpathlineto{\pgfqpoint{5.614602in}{2.991579in}}%
\pgfpathlineto{\pgfqpoint{5.615455in}{2.992261in}}%
\pgfpathlineto{\pgfqpoint{5.878142in}{3.027366in}}%
\pgfpathlineto{\pgfqpoint{5.878995in}{3.027973in}}%
\pgfpathlineto{\pgfqpoint{5.880701in}{3.043292in}}%
\pgfpathlineto{\pgfqpoint{5.881553in}{3.051867in}}%
\pgfpathlineto{\pgfqpoint{5.882406in}{3.069443in}}%
\pgfpathlineto{\pgfqpoint{5.883259in}{3.097837in}}%
\pgfpathlineto{\pgfqpoint{5.884112in}{3.104607in}}%
\pgfpathlineto{\pgfqpoint{5.898611in}{3.115601in}}%
\pgfpathlineto{\pgfqpoint{5.900317in}{3.115908in}}%
\pgfpathlineto{\pgfqpoint{6.004368in}{3.121718in}}%
\pgfpathlineto{\pgfqpoint{6.004368in}{3.121718in}}%
\pgfusepath{stroke}%
\end{pgfscope}%
\begin{pgfscope}%
\pgfpathrectangle{\pgfqpoint{0.481681in}{1.080890in}}{\pgfqpoint{5.785672in}{2.146863in}}%
\pgfusepath{clip}%
\pgfsetrectcap%
\pgfsetroundjoin%
\pgfsetlinewidth{0.200750pt}%
\definecolor{currentstroke}{rgb}{0.682353,0.125490,0.070588}%
\pgfsetstrokecolor{currentstroke}%
\pgfsetdash{}{0pt}%
\pgfpathmoveto{\pgfqpoint{0.744666in}{1.178475in}}%
\pgfpathlineto{\pgfqpoint{0.760871in}{1.179388in}}%
\pgfpathlineto{\pgfqpoint{0.761724in}{1.179727in}}%
\pgfpathlineto{\pgfqpoint{0.764282in}{1.183268in}}%
\pgfpathlineto{\pgfqpoint{0.801809in}{1.196272in}}%
\pgfpathlineto{\pgfqpoint{0.802662in}{1.209508in}}%
\pgfpathlineto{\pgfqpoint{0.803515in}{1.237653in}}%
\pgfpathlineto{\pgfqpoint{0.804367in}{1.243561in}}%
\pgfpathlineto{\pgfqpoint{0.820572in}{1.249551in}}%
\pgfpathlineto{\pgfqpoint{0.821425in}{1.248137in}}%
\pgfpathlineto{\pgfqpoint{0.823984in}{1.203353in}}%
\pgfpathlineto{\pgfqpoint{0.825689in}{1.249428in}}%
\pgfpathlineto{\pgfqpoint{0.826542in}{1.249890in}}%
\pgfpathlineto{\pgfqpoint{0.828248in}{1.249660in}}%
\pgfpathlineto{\pgfqpoint{0.835071in}{1.241625in}}%
\pgfpathlineto{\pgfqpoint{0.860657in}{1.211294in}}%
\pgfpathlineto{\pgfqpoint{0.861510in}{1.222826in}}%
\pgfpathlineto{\pgfqpoint{0.862363in}{1.292872in}}%
\pgfpathlineto{\pgfqpoint{0.864069in}{1.291810in}}%
\pgfpathlineto{\pgfqpoint{0.891361in}{1.271817in}}%
\pgfpathlineto{\pgfqpoint{0.893067in}{1.274352in}}%
\pgfpathlineto{\pgfqpoint{0.942534in}{1.263218in}}%
\pgfpathlineto{\pgfqpoint{0.943387in}{1.267669in}}%
\pgfpathlineto{\pgfqpoint{0.944240in}{1.277444in}}%
\pgfpathlineto{\pgfqpoint{0.945092in}{1.277929in}}%
\pgfpathlineto{\pgfqpoint{0.947651in}{1.270343in}}%
\pgfpathlineto{\pgfqpoint{0.949357in}{1.295201in}}%
\pgfpathlineto{\pgfqpoint{0.950210in}{1.296776in}}%
\pgfpathlineto{\pgfqpoint{0.963856in}{1.296734in}}%
\pgfpathlineto{\pgfqpoint{0.964709in}{1.297657in}}%
\pgfpathlineto{\pgfqpoint{0.965561in}{1.298111in}}%
\pgfpathlineto{\pgfqpoint{0.985178in}{1.298163in}}%
\pgfpathlineto{\pgfqpoint{0.986031in}{1.297212in}}%
\pgfpathlineto{\pgfqpoint{0.987736in}{1.290050in}}%
\pgfpathlineto{\pgfqpoint{0.991148in}{1.269193in}}%
\pgfpathlineto{\pgfqpoint{1.003941in}{1.268774in}}%
\pgfpathlineto{\pgfqpoint{1.004794in}{1.287661in}}%
\pgfpathlineto{\pgfqpoint{1.005647in}{1.353898in}}%
\pgfpathlineto{\pgfqpoint{1.006500in}{1.354232in}}%
\pgfpathlineto{\pgfqpoint{1.007353in}{1.358435in}}%
\pgfpathlineto{\pgfqpoint{1.008205in}{1.384474in}}%
\pgfpathlineto{\pgfqpoint{1.009058in}{1.385958in}}%
\pgfpathlineto{\pgfqpoint{1.013323in}{1.386504in}}%
\pgfpathlineto{\pgfqpoint{1.023557in}{1.386458in}}%
\pgfpathlineto{\pgfqpoint{1.024410in}{1.377648in}}%
\pgfpathlineto{\pgfqpoint{1.025263in}{1.339360in}}%
\pgfpathlineto{\pgfqpoint{1.026116in}{1.334006in}}%
\pgfpathlineto{\pgfqpoint{1.026969in}{1.389321in}}%
\pgfpathlineto{\pgfqpoint{1.030380in}{1.386671in}}%
\pgfpathlineto{\pgfqpoint{1.032939in}{1.386718in}}%
\pgfpathlineto{\pgfqpoint{1.085817in}{1.390281in}}%
\pgfpathlineto{\pgfqpoint{1.086670in}{1.392294in}}%
\pgfpathlineto{\pgfqpoint{1.087523in}{1.392282in}}%
\pgfpathlineto{\pgfqpoint{1.090082in}{1.389376in}}%
\pgfpathlineto{\pgfqpoint{1.090935in}{1.391118in}}%
\pgfpathlineto{\pgfqpoint{1.091788in}{1.397359in}}%
\pgfpathlineto{\pgfqpoint{1.092640in}{1.397068in}}%
\pgfpathlineto{\pgfqpoint{1.093493in}{1.357745in}}%
\pgfpathlineto{\pgfqpoint{1.094346in}{1.244409in}}%
\pgfpathlineto{\pgfqpoint{1.095199in}{1.373774in}}%
\pgfpathlineto{\pgfqpoint{1.096052in}{1.394650in}}%
\pgfpathlineto{\pgfqpoint{1.105434in}{1.397677in}}%
\pgfpathlineto{\pgfqpoint{1.106286in}{1.383716in}}%
\pgfpathlineto{\pgfqpoint{1.107992in}{1.238664in}}%
\pgfpathlineto{\pgfqpoint{1.108845in}{1.384580in}}%
\pgfpathlineto{\pgfqpoint{1.109698in}{1.399340in}}%
\pgfpathlineto{\pgfqpoint{1.113109in}{1.394405in}}%
\pgfpathlineto{\pgfqpoint{1.114815in}{1.394876in}}%
\pgfpathlineto{\pgfqpoint{1.127608in}{1.400168in}}%
\pgfpathlineto{\pgfqpoint{1.128461in}{1.398928in}}%
\pgfpathlineto{\pgfqpoint{1.129314in}{1.399670in}}%
\pgfpathlineto{\pgfqpoint{1.131020in}{1.412100in}}%
\pgfpathlineto{\pgfqpoint{1.131873in}{1.413007in}}%
\pgfpathlineto{\pgfqpoint{1.148930in}{1.412947in}}%
\pgfpathlineto{\pgfqpoint{1.252129in}{1.395946in}}%
\pgfpathlineto{\pgfqpoint{1.252982in}{1.396393in}}%
\pgfpathlineto{\pgfqpoint{1.270892in}{1.412541in}}%
\pgfpathlineto{\pgfqpoint{1.271745in}{1.412157in}}%
\pgfpathlineto{\pgfqpoint{1.272598in}{1.399104in}}%
\pgfpathlineto{\pgfqpoint{1.273451in}{1.396477in}}%
\pgfpathlineto{\pgfqpoint{1.275156in}{1.397530in}}%
\pgfpathlineto{\pgfqpoint{1.285391in}{1.396361in}}%
\pgfpathlineto{\pgfqpoint{1.287950in}{1.396063in}}%
\pgfpathlineto{\pgfqpoint{1.288802in}{1.396687in}}%
\pgfpathlineto{\pgfqpoint{1.289655in}{1.406964in}}%
\pgfpathlineto{\pgfqpoint{1.291361in}{1.416009in}}%
\pgfpathlineto{\pgfqpoint{1.296478in}{1.411845in}}%
\pgfpathlineto{\pgfqpoint{1.298184in}{1.411670in}}%
\pgfpathlineto{\pgfqpoint{1.310977in}{1.411670in}}%
\pgfpathlineto{\pgfqpoint{1.311830in}{1.412096in}}%
\pgfpathlineto{\pgfqpoint{1.312683in}{1.413377in}}%
\pgfpathlineto{\pgfqpoint{1.313536in}{1.412252in}}%
\pgfpathlineto{\pgfqpoint{1.314389in}{1.414372in}}%
\pgfpathlineto{\pgfqpoint{1.315242in}{1.414173in}}%
\pgfpathlineto{\pgfqpoint{1.316095in}{1.415188in}}%
\pgfpathlineto{\pgfqpoint{1.316947in}{1.415360in}}%
\pgfpathlineto{\pgfqpoint{1.319506in}{1.415155in}}%
\pgfpathlineto{\pgfqpoint{1.408205in}{1.415407in}}%
\pgfpathlineto{\pgfqpoint{1.435498in}{1.415740in}}%
\pgfpathlineto{\pgfqpoint{1.521638in}{1.435546in}}%
\pgfpathlineto{\pgfqpoint{1.522491in}{1.437110in}}%
\pgfpathlineto{\pgfqpoint{1.528461in}{1.478253in}}%
\pgfpathlineto{\pgfqpoint{1.537843in}{1.542100in}}%
\pgfpathlineto{\pgfqpoint{1.556606in}{1.601374in}}%
\pgfpathlineto{\pgfqpoint{1.557459in}{1.603096in}}%
\pgfpathlineto{\pgfqpoint{1.582193in}{1.612914in}}%
\pgfpathlineto{\pgfqpoint{1.584751in}{1.613589in}}%
\pgfpathlineto{\pgfqpoint{1.600103in}{1.616936in}}%
\pgfpathlineto{\pgfqpoint{1.601809in}{1.616263in}}%
\pgfpathlineto{\pgfqpoint{1.605220in}{1.619224in}}%
\pgfpathlineto{\pgfqpoint{1.644453in}{1.621476in}}%
\pgfpathlineto{\pgfqpoint{1.658952in}{1.622306in}}%
\pgfpathlineto{\pgfqpoint{1.659805in}{1.632475in}}%
\pgfpathlineto{\pgfqpoint{1.681979in}{2.376362in}}%
\pgfpathlineto{\pgfqpoint{1.682832in}{2.205689in}}%
\pgfpathlineto{\pgfqpoint{1.683685in}{1.621503in}}%
\pgfpathlineto{\pgfqpoint{1.685391in}{1.623338in}}%
\pgfpathlineto{\pgfqpoint{1.686244in}{2.164039in}}%
\pgfpathlineto{\pgfqpoint{1.687097in}{2.414587in}}%
\pgfpathlineto{\pgfqpoint{1.695626in}{2.417401in}}%
\pgfpathlineto{\pgfqpoint{1.705007in}{2.420040in}}%
\pgfpathlineto{\pgfqpoint{1.710977in}{2.419588in}}%
\pgfpathlineto{\pgfqpoint{1.711830in}{2.417661in}}%
\pgfpathlineto{\pgfqpoint{1.712683in}{2.366073in}}%
\pgfpathlineto{\pgfqpoint{1.721212in}{1.652390in}}%
\pgfpathlineto{\pgfqpoint{1.722918in}{1.865565in}}%
\pgfpathlineto{\pgfqpoint{1.726329in}{2.341703in}}%
\pgfpathlineto{\pgfqpoint{1.727182in}{2.418599in}}%
\pgfpathlineto{\pgfqpoint{1.728035in}{2.420118in}}%
\pgfpathlineto{\pgfqpoint{1.745945in}{2.420154in}}%
\pgfpathlineto{\pgfqpoint{1.746798in}{2.418849in}}%
\pgfpathlineto{\pgfqpoint{1.748504in}{2.409257in}}%
\pgfpathlineto{\pgfqpoint{1.749357in}{2.416051in}}%
\pgfpathlineto{\pgfqpoint{1.750210in}{2.420606in}}%
\pgfpathlineto{\pgfqpoint{1.751916in}{2.420474in}}%
\pgfpathlineto{\pgfqpoint{1.753621in}{2.420980in}}%
\pgfpathlineto{\pgfqpoint{1.763856in}{2.422407in}}%
\pgfpathlineto{\pgfqpoint{1.765562in}{2.423501in}}%
\pgfpathlineto{\pgfqpoint{1.766414in}{2.423502in}}%
\pgfpathlineto{\pgfqpoint{1.767267in}{2.423822in}}%
\pgfpathlineto{\pgfqpoint{1.768120in}{2.256011in}}%
\pgfpathlineto{\pgfqpoint{1.768973in}{1.756393in}}%
\pgfpathlineto{\pgfqpoint{1.769826in}{1.628342in}}%
\pgfpathlineto{\pgfqpoint{1.786884in}{1.629523in}}%
\pgfpathlineto{\pgfqpoint{1.787736in}{1.630641in}}%
\pgfpathlineto{\pgfqpoint{1.788589in}{1.651205in}}%
\pgfpathlineto{\pgfqpoint{1.789442in}{2.352809in}}%
\pgfpathlineto{\pgfqpoint{1.790295in}{2.424576in}}%
\pgfpathlineto{\pgfqpoint{1.797971in}{2.424424in}}%
\pgfpathlineto{\pgfqpoint{1.814176in}{2.424181in}}%
\pgfpathlineto{\pgfqpoint{1.815029in}{2.416076in}}%
\pgfpathlineto{\pgfqpoint{1.816734in}{2.208238in}}%
\pgfpathlineto{\pgfqpoint{1.820146in}{1.747590in}}%
\pgfpathlineto{\pgfqpoint{1.820999in}{1.680293in}}%
\pgfpathlineto{\pgfqpoint{1.822704in}{2.034330in}}%
\pgfpathlineto{\pgfqpoint{1.823557in}{2.226322in}}%
\pgfpathlineto{\pgfqpoint{1.824410in}{2.322555in}}%
\pgfpathlineto{\pgfqpoint{1.825263in}{2.466090in}}%
\pgfpathlineto{\pgfqpoint{1.826116in}{2.465277in}}%
\pgfpathlineto{\pgfqpoint{1.826969in}{2.457713in}}%
\pgfpathlineto{\pgfqpoint{1.827822in}{2.458131in}}%
\pgfpathlineto{\pgfqpoint{1.828675in}{2.458204in}}%
\pgfpathlineto{\pgfqpoint{1.829527in}{2.466964in}}%
\pgfpathlineto{\pgfqpoint{1.830380in}{2.467987in}}%
\pgfpathlineto{\pgfqpoint{1.832086in}{2.454957in}}%
\pgfpathlineto{\pgfqpoint{1.832939in}{2.458095in}}%
\pgfpathlineto{\pgfqpoint{1.833792in}{2.341776in}}%
\pgfpathlineto{\pgfqpoint{1.834645in}{2.467368in}}%
\pgfpathlineto{\pgfqpoint{1.836350in}{2.464421in}}%
\pgfpathlineto{\pgfqpoint{1.843174in}{2.451727in}}%
\pgfpathlineto{\pgfqpoint{1.844879in}{2.472803in}}%
\pgfpathlineto{\pgfqpoint{1.845732in}{2.472041in}}%
\pgfpathlineto{\pgfqpoint{1.846585in}{2.455745in}}%
\pgfpathlineto{\pgfqpoint{1.847438in}{2.464432in}}%
\pgfpathlineto{\pgfqpoint{1.848291in}{2.478827in}}%
\pgfpathlineto{\pgfqpoint{1.849144in}{2.474242in}}%
\pgfpathlineto{\pgfqpoint{1.849997in}{2.478814in}}%
\pgfpathlineto{\pgfqpoint{1.850849in}{2.479053in}}%
\pgfpathlineto{\pgfqpoint{1.854261in}{2.478881in}}%
\pgfpathlineto{\pgfqpoint{1.855114in}{2.479401in}}%
\pgfpathlineto{\pgfqpoint{1.855967in}{2.465771in}}%
\pgfpathlineto{\pgfqpoint{1.856820in}{2.478507in}}%
\pgfpathlineto{\pgfqpoint{1.857672in}{2.474622in}}%
\pgfpathlineto{\pgfqpoint{1.858525in}{2.472136in}}%
\pgfpathlineto{\pgfqpoint{1.861937in}{2.456484in}}%
\pgfpathlineto{\pgfqpoint{1.862790in}{2.455937in}}%
\pgfpathlineto{\pgfqpoint{1.863643in}{2.468464in}}%
\pgfpathlineto{\pgfqpoint{1.864495in}{2.474968in}}%
\pgfpathlineto{\pgfqpoint{1.865348in}{2.476508in}}%
\pgfpathlineto{\pgfqpoint{1.866201in}{2.469787in}}%
\pgfpathlineto{\pgfqpoint{1.867054in}{2.479337in}}%
\pgfpathlineto{\pgfqpoint{1.867907in}{2.504735in}}%
\pgfpathlineto{\pgfqpoint{1.868760in}{2.501992in}}%
\pgfpathlineto{\pgfqpoint{1.869613in}{2.485773in}}%
\pgfpathlineto{\pgfqpoint{1.870466in}{2.501063in}}%
\pgfpathlineto{\pgfqpoint{1.871318in}{2.507032in}}%
\pgfpathlineto{\pgfqpoint{1.875583in}{2.509459in}}%
\pgfpathlineto{\pgfqpoint{1.878142in}{2.507934in}}%
\pgfpathlineto{\pgfqpoint{1.882406in}{2.505150in}}%
\pgfpathlineto{\pgfqpoint{1.883259in}{2.504929in}}%
\pgfpathlineto{\pgfqpoint{1.884965in}{2.506951in}}%
\pgfpathlineto{\pgfqpoint{1.885817in}{2.507284in}}%
\pgfpathlineto{\pgfqpoint{1.886670in}{2.508324in}}%
\pgfpathlineto{\pgfqpoint{1.888376in}{2.505298in}}%
\pgfpathlineto{\pgfqpoint{1.889229in}{2.506167in}}%
\pgfpathlineto{\pgfqpoint{1.890935in}{2.509084in}}%
\pgfpathlineto{\pgfqpoint{1.891788in}{2.509572in}}%
\pgfpathlineto{\pgfqpoint{1.907139in}{2.510481in}}%
\pgfpathlineto{\pgfqpoint{1.907992in}{2.511118in}}%
\pgfpathlineto{\pgfqpoint{1.908845in}{2.511291in}}%
\pgfpathlineto{\pgfqpoint{1.947225in}{2.504771in}}%
\pgfpathlineto{\pgfqpoint{1.948078in}{2.506356in}}%
\pgfpathlineto{\pgfqpoint{1.948930in}{2.509673in}}%
\pgfpathlineto{\pgfqpoint{1.956606in}{2.509792in}}%
\pgfpathlineto{\pgfqpoint{1.965988in}{2.509747in}}%
\pgfpathlineto{\pgfqpoint{1.968547in}{2.511909in}}%
\pgfpathlineto{\pgfqpoint{1.970252in}{2.512154in}}%
\pgfpathlineto{\pgfqpoint{1.975370in}{2.512279in}}%
\pgfpathlineto{\pgfqpoint{2.021425in}{2.512285in}}%
\pgfpathlineto{\pgfqpoint{2.031660in}{2.512209in}}%
\pgfpathlineto{\pgfqpoint{2.032513in}{2.520694in}}%
\pgfpathlineto{\pgfqpoint{2.033365in}{2.522013in}}%
\pgfpathlineto{\pgfqpoint{2.053835in}{2.522096in}}%
\pgfpathlineto{\pgfqpoint{2.093920in}{2.521859in}}%
\pgfpathlineto{\pgfqpoint{2.224410in}{2.513066in}}%
\pgfpathlineto{\pgfqpoint{2.244879in}{2.517227in}}%
\pgfpathlineto{\pgfqpoint{2.245732in}{2.514851in}}%
\pgfpathlineto{\pgfqpoint{2.255967in}{2.468486in}}%
\pgfpathlineto{\pgfqpoint{2.256820in}{2.469243in}}%
\pgfpathlineto{\pgfqpoint{2.257672in}{2.538410in}}%
\pgfpathlineto{\pgfqpoint{2.258525in}{2.569230in}}%
\pgfpathlineto{\pgfqpoint{2.265348in}{2.524807in}}%
\pgfpathlineto{\pgfqpoint{2.266201in}{2.524473in}}%
\pgfpathlineto{\pgfqpoint{2.279847in}{2.526238in}}%
\pgfpathlineto{\pgfqpoint{2.280700in}{2.528169in}}%
\pgfpathlineto{\pgfqpoint{2.299464in}{2.617806in}}%
\pgfpathlineto{\pgfqpoint{2.300316in}{2.618818in}}%
\pgfpathlineto{\pgfqpoint{2.301169in}{2.578753in}}%
\pgfpathlineto{\pgfqpoint{2.306287in}{2.588707in}}%
\pgfpathlineto{\pgfqpoint{2.307139in}{2.586382in}}%
\pgfpathlineto{\pgfqpoint{2.320785in}{2.532115in}}%
\pgfpathlineto{\pgfqpoint{2.322491in}{2.532188in}}%
\pgfpathlineto{\pgfqpoint{2.346372in}{2.536142in}}%
\pgfpathlineto{\pgfqpoint{2.347225in}{2.536739in}}%
\pgfpathlineto{\pgfqpoint{2.366841in}{2.572089in}}%
\pgfpathlineto{\pgfqpoint{2.377075in}{2.590451in}}%
\pgfpathlineto{\pgfqpoint{2.377928in}{2.590808in}}%
\pgfpathlineto{\pgfqpoint{2.379634in}{2.590866in}}%
\pgfpathlineto{\pgfqpoint{2.380487in}{2.592753in}}%
\pgfpathlineto{\pgfqpoint{2.381340in}{2.595849in}}%
\pgfpathlineto{\pgfqpoint{2.394986in}{2.599716in}}%
\pgfpathlineto{\pgfqpoint{2.395839in}{2.607257in}}%
\pgfpathlineto{\pgfqpoint{2.398397in}{2.651063in}}%
\pgfpathlineto{\pgfqpoint{2.399250in}{2.657195in}}%
\pgfpathlineto{\pgfqpoint{2.400103in}{2.613262in}}%
\pgfpathlineto{\pgfqpoint{2.400956in}{2.616383in}}%
\pgfpathlineto{\pgfqpoint{2.401809in}{2.590866in}}%
\pgfpathlineto{\pgfqpoint{2.402662in}{2.598866in}}%
\pgfpathlineto{\pgfqpoint{2.403515in}{2.633635in}}%
\pgfpathlineto{\pgfqpoint{2.404368in}{2.634282in}}%
\pgfpathlineto{\pgfqpoint{2.405220in}{2.580890in}}%
\pgfpathlineto{\pgfqpoint{2.406073in}{2.654048in}}%
\pgfpathlineto{\pgfqpoint{2.408632in}{2.555099in}}%
\pgfpathlineto{\pgfqpoint{2.409485in}{2.547953in}}%
\pgfpathlineto{\pgfqpoint{2.417161in}{2.663117in}}%
\pgfpathlineto{\pgfqpoint{2.418014in}{2.643633in}}%
\pgfpathlineto{\pgfqpoint{2.418867in}{2.553101in}}%
\pgfpathlineto{\pgfqpoint{2.420572in}{2.615145in}}%
\pgfpathlineto{\pgfqpoint{2.421425in}{2.622278in}}%
\pgfpathlineto{\pgfqpoint{2.423131in}{2.622485in}}%
\pgfpathlineto{\pgfqpoint{2.437630in}{2.622690in}}%
\pgfpathlineto{\pgfqpoint{2.440188in}{2.628259in}}%
\pgfpathlineto{\pgfqpoint{2.458952in}{2.671129in}}%
\pgfpathlineto{\pgfqpoint{2.459805in}{2.671896in}}%
\pgfpathlineto{\pgfqpoint{2.460658in}{2.697945in}}%
\pgfpathlineto{\pgfqpoint{2.461510in}{2.695757in}}%
\pgfpathlineto{\pgfqpoint{2.470039in}{2.625506in}}%
\pgfpathlineto{\pgfqpoint{2.470892in}{2.624348in}}%
\pgfpathlineto{\pgfqpoint{2.525476in}{2.711759in}}%
\pgfpathlineto{\pgfqpoint{2.530594in}{2.715406in}}%
\pgfpathlineto{\pgfqpoint{2.531446in}{2.711722in}}%
\pgfpathlineto{\pgfqpoint{2.538270in}{2.636454in}}%
\pgfpathlineto{\pgfqpoint{2.539122in}{2.632544in}}%
\pgfpathlineto{\pgfqpoint{2.539975in}{2.716976in}}%
\pgfpathlineto{\pgfqpoint{2.540828in}{2.713159in}}%
\pgfpathlineto{\pgfqpoint{2.542534in}{2.719781in}}%
\pgfpathlineto{\pgfqpoint{2.561297in}{2.720019in}}%
\pgfpathlineto{\pgfqpoint{2.563856in}{2.720050in}}%
\pgfpathlineto{\pgfqpoint{2.564709in}{2.720634in}}%
\pgfpathlineto{\pgfqpoint{2.573238in}{2.721256in}}%
\pgfpathlineto{\pgfqpoint{2.580913in}{2.721776in}}%
\pgfpathlineto{\pgfqpoint{2.581766in}{2.720740in}}%
\pgfpathlineto{\pgfqpoint{2.584325in}{2.721007in}}%
\pgfpathlineto{\pgfqpoint{2.585178in}{2.722895in}}%
\pgfpathlineto{\pgfqpoint{2.586884in}{2.734527in}}%
\pgfpathlineto{\pgfqpoint{2.587736in}{2.738207in}}%
\pgfpathlineto{\pgfqpoint{2.592854in}{2.739146in}}%
\pgfpathlineto{\pgfqpoint{2.593707in}{2.739130in}}%
\pgfpathlineto{\pgfqpoint{2.600530in}{2.734280in}}%
\pgfpathlineto{\pgfqpoint{2.601383in}{2.733668in}}%
\pgfpathlineto{\pgfqpoint{2.602235in}{2.733476in}}%
\pgfpathlineto{\pgfqpoint{2.608206in}{2.741055in}}%
\pgfpathlineto{\pgfqpoint{2.614176in}{2.741992in}}%
\pgfpathlineto{\pgfqpoint{2.628675in}{2.741191in}}%
\pgfpathlineto{\pgfqpoint{2.664496in}{2.741168in}}%
\pgfpathlineto{\pgfqpoint{2.665348in}{2.741168in}}%
\pgfpathlineto{\pgfqpoint{2.666201in}{2.734666in}}%
\pgfpathlineto{\pgfqpoint{2.667054in}{2.721994in}}%
\pgfpathlineto{\pgfqpoint{2.667907in}{2.721961in}}%
\pgfpathlineto{\pgfqpoint{2.683259in}{2.741505in}}%
\pgfpathlineto{\pgfqpoint{2.684112in}{2.742111in}}%
\pgfpathlineto{\pgfqpoint{2.697758in}{2.744156in}}%
\pgfpathlineto{\pgfqpoint{2.702022in}{2.744396in}}%
\pgfpathlineto{\pgfqpoint{2.702875in}{2.743458in}}%
\pgfpathlineto{\pgfqpoint{2.704581in}{2.725621in}}%
\pgfpathlineto{\pgfqpoint{2.706287in}{2.726454in}}%
\pgfpathlineto{\pgfqpoint{2.707139in}{2.729560in}}%
\pgfpathlineto{\pgfqpoint{2.707992in}{2.726773in}}%
\pgfpathlineto{\pgfqpoint{2.713963in}{2.726896in}}%
\pgfpathlineto{\pgfqpoint{2.726756in}{2.727280in}}%
\pgfpathlineto{\pgfqpoint{2.771105in}{2.731475in}}%
\pgfpathlineto{\pgfqpoint{2.785604in}{2.732507in}}%
\pgfpathlineto{\pgfqpoint{2.835924in}{2.732267in}}%
\pgfpathlineto{\pgfqpoint{2.846159in}{2.729841in}}%
\pgfpathlineto{\pgfqpoint{2.847012in}{2.730023in}}%
\pgfpathlineto{\pgfqpoint{2.847864in}{2.732289in}}%
\pgfpathlineto{\pgfqpoint{2.848717in}{2.732759in}}%
\pgfpathlineto{\pgfqpoint{2.849570in}{2.732774in}}%
\pgfpathlineto{\pgfqpoint{2.850423in}{2.733319in}}%
\pgfpathlineto{\pgfqpoint{2.852982in}{2.733451in}}%
\pgfpathlineto{\pgfqpoint{2.856393in}{2.733615in}}%
\pgfpathlineto{\pgfqpoint{2.858952in}{2.737040in}}%
\pgfpathlineto{\pgfqpoint{2.870039in}{2.752594in}}%
\pgfpathlineto{\pgfqpoint{2.870892in}{2.738873in}}%
\pgfpathlineto{\pgfqpoint{2.871745in}{2.734365in}}%
\pgfpathlineto{\pgfqpoint{2.900743in}{2.763695in}}%
\pgfpathlineto{\pgfqpoint{2.901596in}{2.745533in}}%
\pgfpathlineto{\pgfqpoint{2.902449in}{2.735148in}}%
\pgfpathlineto{\pgfqpoint{2.915242in}{2.735662in}}%
\pgfpathlineto{\pgfqpoint{2.918653in}{2.737405in}}%
\pgfpathlineto{\pgfqpoint{2.968973in}{2.764264in}}%
\pgfpathlineto{\pgfqpoint{2.969826in}{2.763472in}}%
\pgfpathlineto{\pgfqpoint{2.970679in}{2.748099in}}%
\pgfpathlineto{\pgfqpoint{2.971532in}{2.745112in}}%
\pgfpathlineto{\pgfqpoint{2.972385in}{2.765443in}}%
\pgfpathlineto{\pgfqpoint{2.973238in}{2.768944in}}%
\pgfpathlineto{\pgfqpoint{2.974943in}{2.769683in}}%
\pgfpathlineto{\pgfqpoint{2.990295in}{2.771688in}}%
\pgfpathlineto{\pgfqpoint{2.992001in}{2.743061in}}%
\pgfpathlineto{\pgfqpoint{2.992854in}{2.759893in}}%
\pgfpathlineto{\pgfqpoint{2.993707in}{2.771135in}}%
\pgfpathlineto{\pgfqpoint{2.994560in}{2.769765in}}%
\pgfpathlineto{\pgfqpoint{2.995412in}{2.777897in}}%
\pgfpathlineto{\pgfqpoint{2.996265in}{2.779399in}}%
\pgfpathlineto{\pgfqpoint{3.003941in}{2.778680in}}%
\pgfpathlineto{\pgfqpoint{3.018440in}{2.783423in}}%
\pgfpathlineto{\pgfqpoint{3.046585in}{2.781789in}}%
\pgfpathlineto{\pgfqpoint{3.061937in}{2.781118in}}%
\pgfpathlineto{\pgfqpoint{3.134432in}{2.781980in}}%
\pgfpathlineto{\pgfqpoint{3.136137in}{2.783437in}}%
\pgfpathlineto{\pgfqpoint{3.138696in}{2.783527in}}%
\pgfpathlineto{\pgfqpoint{3.207779in}{2.782870in}}%
\pgfpathlineto{\pgfqpoint{3.213749in}{2.779694in}}%
\pgfpathlineto{\pgfqpoint{3.237630in}{2.766786in}}%
\pgfpathlineto{\pgfqpoint{3.238483in}{2.762940in}}%
\pgfpathlineto{\pgfqpoint{3.239336in}{2.753582in}}%
\pgfpathlineto{\pgfqpoint{3.240189in}{2.750380in}}%
\pgfpathlineto{\pgfqpoint{3.241041in}{2.749174in}}%
\pgfpathlineto{\pgfqpoint{3.243600in}{2.759751in}}%
\pgfpathlineto{\pgfqpoint{3.247864in}{2.777999in}}%
\pgfpathlineto{\pgfqpoint{3.248717in}{2.779421in}}%
\pgfpathlineto{\pgfqpoint{3.256393in}{2.751018in}}%
\pgfpathlineto{\pgfqpoint{3.258099in}{2.745503in}}%
\pgfpathlineto{\pgfqpoint{3.276009in}{2.745375in}}%
\pgfpathlineto{\pgfqpoint{3.276862in}{2.747271in}}%
\pgfpathlineto{\pgfqpoint{3.277715in}{2.767466in}}%
\pgfpathlineto{\pgfqpoint{3.278568in}{2.747216in}}%
\pgfpathlineto{\pgfqpoint{3.279421in}{2.746417in}}%
\pgfpathlineto{\pgfqpoint{3.296479in}{2.777836in}}%
\pgfpathlineto{\pgfqpoint{3.297331in}{2.778269in}}%
\pgfpathlineto{\pgfqpoint{3.318653in}{2.778243in}}%
\pgfpathlineto{\pgfqpoint{3.320359in}{2.743760in}}%
\pgfpathlineto{\pgfqpoint{3.323771in}{2.743631in}}%
\pgfpathlineto{\pgfqpoint{3.379208in}{2.742922in}}%
\pgfpathlineto{\pgfqpoint{3.380061in}{2.742558in}}%
\pgfpathlineto{\pgfqpoint{3.380914in}{2.741423in}}%
\pgfpathlineto{\pgfqpoint{3.382619in}{2.741270in}}%
\pgfpathlineto{\pgfqpoint{3.404794in}{2.741385in}}%
\pgfpathlineto{\pgfqpoint{3.413323in}{2.743308in}}%
\pgfpathlineto{\pgfqpoint{3.419293in}{2.741376in}}%
\pgfpathlineto{\pgfqpoint{3.420999in}{2.742635in}}%
\pgfpathlineto{\pgfqpoint{3.425263in}{2.746240in}}%
\pgfpathlineto{\pgfqpoint{3.433792in}{2.749703in}}%
\pgfpathlineto{\pgfqpoint{3.445732in}{2.754505in}}%
\pgfpathlineto{\pgfqpoint{3.446585in}{2.755307in}}%
\pgfpathlineto{\pgfqpoint{3.447438in}{2.756564in}}%
\pgfpathlineto{\pgfqpoint{3.449997in}{2.757869in}}%
\pgfpathlineto{\pgfqpoint{3.462790in}{2.763676in}}%
\pgfpathlineto{\pgfqpoint{3.463643in}{2.761556in}}%
\pgfpathlineto{\pgfqpoint{3.467054in}{2.749246in}}%
\pgfpathlineto{\pgfqpoint{3.467907in}{2.754597in}}%
\pgfpathlineto{\pgfqpoint{3.468760in}{2.754056in}}%
\pgfpathlineto{\pgfqpoint{3.469613in}{2.750783in}}%
\pgfpathlineto{\pgfqpoint{3.473877in}{2.757816in}}%
\pgfpathlineto{\pgfqpoint{3.474730in}{2.758016in}}%
\pgfpathlineto{\pgfqpoint{3.487523in}{2.755106in}}%
\pgfpathlineto{\pgfqpoint{3.488376in}{2.755135in}}%
\pgfpathlineto{\pgfqpoint{3.490082in}{2.756070in}}%
\pgfpathlineto{\pgfqpoint{3.525903in}{2.770143in}}%
\pgfpathlineto{\pgfqpoint{3.526756in}{2.769814in}}%
\pgfpathlineto{\pgfqpoint{3.527609in}{2.762301in}}%
\pgfpathlineto{\pgfqpoint{3.528462in}{2.760492in}}%
\pgfpathlineto{\pgfqpoint{3.534432in}{2.759911in}}%
\pgfpathlineto{\pgfqpoint{3.536990in}{2.762762in}}%
\pgfpathlineto{\pgfqpoint{3.544666in}{2.771706in}}%
\pgfpathlineto{\pgfqpoint{3.546372in}{2.763631in}}%
\pgfpathlineto{\pgfqpoint{3.548078in}{2.785050in}}%
\pgfpathlineto{\pgfqpoint{3.548931in}{2.785204in}}%
\pgfpathlineto{\pgfqpoint{3.550636in}{2.783028in}}%
\pgfpathlineto{\pgfqpoint{3.560871in}{2.772830in}}%
\pgfpathlineto{\pgfqpoint{3.561724in}{2.772695in}}%
\pgfpathlineto{\pgfqpoint{3.562577in}{2.774464in}}%
\pgfpathlineto{\pgfqpoint{3.563430in}{2.768683in}}%
\pgfpathlineto{\pgfqpoint{3.565135in}{2.795533in}}%
\pgfpathlineto{\pgfqpoint{3.565988in}{2.796925in}}%
\pgfpathlineto{\pgfqpoint{3.566841in}{2.789501in}}%
\pgfpathlineto{\pgfqpoint{3.567694in}{2.763735in}}%
\pgfpathlineto{\pgfqpoint{3.568547in}{2.764769in}}%
\pgfpathlineto{\pgfqpoint{3.570253in}{2.768279in}}%
\pgfpathlineto{\pgfqpoint{3.577076in}{2.788597in}}%
\pgfpathlineto{\pgfqpoint{3.582193in}{2.800615in}}%
\pgfpathlineto{\pgfqpoint{3.583046in}{2.800537in}}%
\pgfpathlineto{\pgfqpoint{3.583899in}{2.773671in}}%
\pgfpathlineto{\pgfqpoint{3.584752in}{2.787143in}}%
\pgfpathlineto{\pgfqpoint{3.585604in}{2.776664in}}%
\pgfpathlineto{\pgfqpoint{3.586457in}{2.771281in}}%
\pgfpathlineto{\pgfqpoint{3.587310in}{2.799806in}}%
\pgfpathlineto{\pgfqpoint{3.588163in}{2.799351in}}%
\pgfpathlineto{\pgfqpoint{3.589016in}{2.794562in}}%
\pgfpathlineto{\pgfqpoint{3.589869in}{2.803843in}}%
\pgfpathlineto{\pgfqpoint{3.590722in}{2.793281in}}%
\pgfpathlineto{\pgfqpoint{3.591575in}{2.790990in}}%
\pgfpathlineto{\pgfqpoint{3.604368in}{2.810853in}}%
\pgfpathlineto{\pgfqpoint{3.606073in}{2.812225in}}%
\pgfpathlineto{\pgfqpoint{3.606926in}{2.811204in}}%
\pgfpathlineto{\pgfqpoint{3.607779in}{2.791057in}}%
\pgfpathlineto{\pgfqpoint{3.608632in}{2.797956in}}%
\pgfpathlineto{\pgfqpoint{3.609485in}{2.800799in}}%
\pgfpathlineto{\pgfqpoint{3.610338in}{2.812293in}}%
\pgfpathlineto{\pgfqpoint{3.611191in}{2.794986in}}%
\pgfpathlineto{\pgfqpoint{3.612044in}{2.789236in}}%
\pgfpathlineto{\pgfqpoint{3.630807in}{2.791765in}}%
\pgfpathlineto{\pgfqpoint{3.635071in}{2.794298in}}%
\pgfpathlineto{\pgfqpoint{3.670039in}{2.815630in}}%
\pgfpathlineto{\pgfqpoint{3.675157in}{2.816340in}}%
\pgfpathlineto{\pgfqpoint{3.685391in}{2.816310in}}%
\pgfpathlineto{\pgfqpoint{3.686244in}{2.813287in}}%
\pgfpathlineto{\pgfqpoint{3.688803in}{2.797000in}}%
\pgfpathlineto{\pgfqpoint{3.689656in}{2.801461in}}%
\pgfpathlineto{\pgfqpoint{3.691361in}{2.819217in}}%
\pgfpathlineto{\pgfqpoint{3.692214in}{2.838307in}}%
\pgfpathlineto{\pgfqpoint{3.693067in}{2.847219in}}%
\pgfpathlineto{\pgfqpoint{3.693920in}{2.851597in}}%
\pgfpathlineto{\pgfqpoint{3.705007in}{2.851506in}}%
\pgfpathlineto{\pgfqpoint{3.705860in}{2.850654in}}%
\pgfpathlineto{\pgfqpoint{3.706713in}{2.836936in}}%
\pgfpathlineto{\pgfqpoint{3.707566in}{2.833085in}}%
\pgfpathlineto{\pgfqpoint{3.708419in}{2.833530in}}%
\pgfpathlineto{\pgfqpoint{3.710978in}{2.832976in}}%
\pgfpathlineto{\pgfqpoint{3.711830in}{2.833537in}}%
\pgfpathlineto{\pgfqpoint{3.712683in}{2.837966in}}%
\pgfpathlineto{\pgfqpoint{3.713536in}{2.845946in}}%
\pgfpathlineto{\pgfqpoint{3.714389in}{2.850716in}}%
\pgfpathlineto{\pgfqpoint{3.716948in}{2.853692in}}%
\pgfpathlineto{\pgfqpoint{3.717801in}{2.853119in}}%
\pgfpathlineto{\pgfqpoint{3.726329in}{2.833639in}}%
\pgfpathlineto{\pgfqpoint{3.728888in}{2.833511in}}%
\pgfpathlineto{\pgfqpoint{3.729741in}{2.833511in}}%
\pgfpathlineto{\pgfqpoint{3.731447in}{2.848044in}}%
\pgfpathlineto{\pgfqpoint{3.732299in}{2.835980in}}%
\pgfpathlineto{\pgfqpoint{3.733152in}{2.834274in}}%
\pgfpathlineto{\pgfqpoint{3.738270in}{2.850825in}}%
\pgfpathlineto{\pgfqpoint{3.739975in}{2.857586in}}%
\pgfpathlineto{\pgfqpoint{3.742534in}{2.858229in}}%
\pgfpathlineto{\pgfqpoint{3.748504in}{2.859472in}}%
\pgfpathlineto{\pgfqpoint{3.750210in}{2.852330in}}%
\pgfpathlineto{\pgfqpoint{3.754474in}{2.851610in}}%
\pgfpathlineto{\pgfqpoint{3.760444in}{2.844603in}}%
\pgfpathlineto{\pgfqpoint{3.762150in}{2.845296in}}%
\pgfpathlineto{\pgfqpoint{3.778355in}{2.853306in}}%
\pgfpathlineto{\pgfqpoint{3.811617in}{2.855337in}}%
\pgfpathlineto{\pgfqpoint{3.812470in}{2.859455in}}%
\pgfpathlineto{\pgfqpoint{3.814176in}{2.859942in}}%
\pgfpathlineto{\pgfqpoint{3.832086in}{2.863104in}}%
\pgfpathlineto{\pgfqpoint{3.832939in}{2.857890in}}%
\pgfpathlineto{\pgfqpoint{3.833792in}{2.858243in}}%
\pgfpathlineto{\pgfqpoint{3.834645in}{2.864350in}}%
\pgfpathlineto{\pgfqpoint{3.835498in}{2.860859in}}%
\pgfpathlineto{\pgfqpoint{3.836351in}{2.856026in}}%
\pgfpathlineto{\pgfqpoint{3.837204in}{2.854295in}}%
\pgfpathlineto{\pgfqpoint{3.851702in}{2.869592in}}%
\pgfpathlineto{\pgfqpoint{3.852555in}{2.870134in}}%
\pgfpathlineto{\pgfqpoint{3.853408in}{2.872593in}}%
\pgfpathlineto{\pgfqpoint{3.854261in}{2.869284in}}%
\pgfpathlineto{\pgfqpoint{3.855967in}{2.855386in}}%
\pgfpathlineto{\pgfqpoint{3.856820in}{2.855409in}}%
\pgfpathlineto{\pgfqpoint{3.868760in}{2.873612in}}%
\pgfpathlineto{\pgfqpoint{3.869613in}{2.871930in}}%
\pgfpathlineto{\pgfqpoint{3.870466in}{2.862073in}}%
\pgfpathlineto{\pgfqpoint{3.871319in}{2.876489in}}%
\pgfpathlineto{\pgfqpoint{3.872172in}{2.875173in}}%
\pgfpathlineto{\pgfqpoint{3.873024in}{2.862425in}}%
\pgfpathlineto{\pgfqpoint{3.874730in}{2.876660in}}%
\pgfpathlineto{\pgfqpoint{4.098184in}{2.877231in}}%
\pgfpathlineto{\pgfqpoint{4.115242in}{2.879172in}}%
\pgfpathlineto{\pgfqpoint{4.116095in}{2.894835in}}%
\pgfpathlineto{\pgfqpoint{4.116948in}{2.899991in}}%
\pgfpathlineto{\pgfqpoint{4.247438in}{2.868349in}}%
\pgfpathlineto{\pgfqpoint{4.263643in}{2.853162in}}%
\pgfpathlineto{\pgfqpoint{4.271319in}{2.846192in}}%
\pgfpathlineto{\pgfqpoint{4.272172in}{2.896242in}}%
\pgfpathlineto{\pgfqpoint{4.273024in}{2.906320in}}%
\pgfpathlineto{\pgfqpoint{4.278995in}{2.908489in}}%
\pgfpathlineto{\pgfqpoint{4.279848in}{2.909286in}}%
\pgfpathlineto{\pgfqpoint{4.281553in}{2.911763in}}%
\pgfpathlineto{\pgfqpoint{4.282406in}{2.910317in}}%
\pgfpathlineto{\pgfqpoint{4.283259in}{2.886731in}}%
\pgfpathlineto{\pgfqpoint{4.284112in}{2.870845in}}%
\pgfpathlineto{\pgfqpoint{4.284965in}{2.902990in}}%
\pgfpathlineto{\pgfqpoint{4.285818in}{2.920354in}}%
\pgfpathlineto{\pgfqpoint{4.286671in}{2.913187in}}%
\pgfpathlineto{\pgfqpoint{4.287523in}{2.910928in}}%
\pgfpathlineto{\pgfqpoint{4.291788in}{2.908851in}}%
\pgfpathlineto{\pgfqpoint{4.292641in}{2.909163in}}%
\pgfpathlineto{\pgfqpoint{4.302022in}{2.926201in}}%
\pgfpathlineto{\pgfqpoint{4.302875in}{2.926600in}}%
\pgfpathlineto{\pgfqpoint{4.303728in}{2.926635in}}%
\pgfpathlineto{\pgfqpoint{4.304581in}{2.921430in}}%
\pgfpathlineto{\pgfqpoint{4.305434in}{2.931788in}}%
\pgfpathlineto{\pgfqpoint{4.306287in}{2.948567in}}%
\pgfpathlineto{\pgfqpoint{4.307140in}{2.951575in}}%
\pgfpathlineto{\pgfqpoint{4.318227in}{2.952229in}}%
\pgfpathlineto{\pgfqpoint{4.319080in}{2.950597in}}%
\pgfpathlineto{\pgfqpoint{4.319933in}{2.936770in}}%
\pgfpathlineto{\pgfqpoint{4.320786in}{2.936770in}}%
\pgfpathlineto{\pgfqpoint{4.321639in}{2.940516in}}%
\pgfpathlineto{\pgfqpoint{4.322491in}{2.954196in}}%
\pgfpathlineto{\pgfqpoint{4.324197in}{2.955529in}}%
\pgfpathlineto{\pgfqpoint{4.328462in}{2.957862in}}%
\pgfpathlineto{\pgfqpoint{4.331020in}{2.957914in}}%
\pgfpathlineto{\pgfqpoint{4.334432in}{2.958118in}}%
\pgfpathlineto{\pgfqpoint{4.336138in}{2.955363in}}%
\pgfpathlineto{\pgfqpoint{4.345519in}{2.937679in}}%
\pgfpathlineto{\pgfqpoint{4.346372in}{2.937064in}}%
\pgfpathlineto{\pgfqpoint{4.348078in}{2.946670in}}%
\pgfpathlineto{\pgfqpoint{4.350636in}{2.961774in}}%
\pgfpathlineto{\pgfqpoint{4.351489in}{2.962496in}}%
\pgfpathlineto{\pgfqpoint{4.388163in}{2.962456in}}%
\pgfpathlineto{\pgfqpoint{4.389016in}{2.961980in}}%
\pgfpathlineto{\pgfqpoint{4.405221in}{2.936838in}}%
\pgfpathlineto{\pgfqpoint{4.406074in}{2.936565in}}%
\pgfpathlineto{\pgfqpoint{4.407779in}{2.963983in}}%
\pgfpathlineto{\pgfqpoint{4.411191in}{2.963981in}}%
\pgfpathlineto{\pgfqpoint{4.445306in}{2.964246in}}%
\pgfpathlineto{\pgfqpoint{4.464922in}{2.964336in}}%
\pgfpathlineto{\pgfqpoint{4.468334in}{2.964477in}}%
\pgfpathlineto{\pgfqpoint{4.534858in}{2.978408in}}%
\pgfpathlineto{\pgfqpoint{4.545093in}{2.980227in}}%
\pgfpathlineto{\pgfqpoint{4.546799in}{2.982361in}}%
\pgfpathlineto{\pgfqpoint{4.548504in}{2.978974in}}%
\pgfpathlineto{\pgfqpoint{4.552769in}{2.969635in}}%
\pgfpathlineto{\pgfqpoint{4.553622in}{2.968783in}}%
\pgfpathlineto{\pgfqpoint{4.559592in}{2.971442in}}%
\pgfpathlineto{\pgfqpoint{4.572385in}{2.973019in}}%
\pgfpathlineto{\pgfqpoint{4.579208in}{2.974599in}}%
\pgfpathlineto{\pgfqpoint{4.583472in}{2.969903in}}%
\pgfpathlineto{\pgfqpoint{4.587737in}{2.965117in}}%
\pgfpathlineto{\pgfqpoint{4.588590in}{2.979105in}}%
\pgfpathlineto{\pgfqpoint{4.589442in}{2.985294in}}%
\pgfpathlineto{\pgfqpoint{4.590295in}{2.995189in}}%
\pgfpathlineto{\pgfqpoint{4.592001in}{2.995185in}}%
\pgfpathlineto{\pgfqpoint{4.592854in}{2.994055in}}%
\pgfpathlineto{\pgfqpoint{4.600530in}{2.975937in}}%
\pgfpathlineto{\pgfqpoint{4.601383in}{2.975558in}}%
\pgfpathlineto{\pgfqpoint{4.611617in}{2.975524in}}%
\pgfpathlineto{\pgfqpoint{4.612470in}{2.976154in}}%
\pgfpathlineto{\pgfqpoint{4.632939in}{2.999515in}}%
\pgfpathlineto{\pgfqpoint{4.633792in}{2.998849in}}%
\pgfpathlineto{\pgfqpoint{4.649144in}{2.976130in}}%
\pgfpathlineto{\pgfqpoint{4.652555in}{2.979404in}}%
\pgfpathlineto{\pgfqpoint{4.674730in}{3.001078in}}%
\pgfpathlineto{\pgfqpoint{4.676436in}{3.000907in}}%
\pgfpathlineto{\pgfqpoint{4.679848in}{3.001236in}}%
\pgfpathlineto{\pgfqpoint{4.688376in}{3.001017in}}%
\pgfpathlineto{\pgfqpoint{4.690082in}{3.000663in}}%
\pgfpathlineto{\pgfqpoint{4.693494in}{3.001191in}}%
\pgfpathlineto{\pgfqpoint{4.694347in}{3.000985in}}%
\pgfpathlineto{\pgfqpoint{4.695199in}{2.993414in}}%
\pgfpathlineto{\pgfqpoint{4.696052in}{3.014807in}}%
\pgfpathlineto{\pgfqpoint{4.710551in}{3.015011in}}%
\pgfpathlineto{\pgfqpoint{4.711404in}{3.014374in}}%
\pgfpathlineto{\pgfqpoint{4.712257in}{3.015789in}}%
\pgfpathlineto{\pgfqpoint{4.713963in}{3.015789in}}%
\pgfpathlineto{\pgfqpoint{4.714816in}{3.012661in}}%
\pgfpathlineto{\pgfqpoint{4.715668in}{3.003229in}}%
\pgfpathlineto{\pgfqpoint{4.716521in}{3.006647in}}%
\pgfpathlineto{\pgfqpoint{4.718227in}{3.015500in}}%
\pgfpathlineto{\pgfqpoint{4.719933in}{3.015789in}}%
\pgfpathlineto{\pgfqpoint{4.737843in}{3.016022in}}%
\pgfpathlineto{\pgfqpoint{4.800956in}{3.020760in}}%
\pgfpathlineto{\pgfqpoint{4.818014in}{3.021000in}}%
\pgfpathlineto{\pgfqpoint{4.831660in}{3.022151in}}%
\pgfpathlineto{\pgfqpoint{4.832513in}{3.020725in}}%
\pgfpathlineto{\pgfqpoint{4.833366in}{3.017159in}}%
\pgfpathlineto{\pgfqpoint{4.834219in}{3.016096in}}%
\pgfpathlineto{\pgfqpoint{4.845306in}{3.016173in}}%
\pgfpathlineto{\pgfqpoint{4.862364in}{3.019490in}}%
\pgfpathlineto{\pgfqpoint{4.864922in}{3.019760in}}%
\pgfpathlineto{\pgfqpoint{4.868334in}{3.021737in}}%
\pgfpathlineto{\pgfqpoint{4.871745in}{3.023791in}}%
\pgfpathlineto{\pgfqpoint{4.872598in}{3.018305in}}%
\pgfpathlineto{\pgfqpoint{4.873451in}{3.018576in}}%
\pgfpathlineto{\pgfqpoint{4.874304in}{3.022856in}}%
\pgfpathlineto{\pgfqpoint{4.876010in}{3.028765in}}%
\pgfpathlineto{\pgfqpoint{4.876863in}{3.029372in}}%
\pgfpathlineto{\pgfqpoint{4.879421in}{3.029973in}}%
\pgfpathlineto{\pgfqpoint{4.880274in}{3.030673in}}%
\pgfpathlineto{\pgfqpoint{4.886244in}{3.030752in}}%
\pgfpathlineto{\pgfqpoint{4.888803in}{3.028963in}}%
\pgfpathlineto{\pgfqpoint{4.894773in}{3.024454in}}%
\pgfpathlineto{\pgfqpoint{4.895626in}{3.024270in}}%
\pgfpathlineto{\pgfqpoint{4.897332in}{3.026559in}}%
\pgfpathlineto{\pgfqpoint{4.901596in}{3.031904in}}%
\pgfpathlineto{\pgfqpoint{4.918654in}{3.031262in}}%
\pgfpathlineto{\pgfqpoint{4.919506in}{3.032003in}}%
\pgfpathlineto{\pgfqpoint{4.955327in}{3.024102in}}%
\pgfpathlineto{\pgfqpoint{4.957033in}{3.024679in}}%
\pgfpathlineto{\pgfqpoint{4.958739in}{3.024052in}}%
\pgfpathlineto{\pgfqpoint{4.959592in}{3.025066in}}%
\pgfpathlineto{\pgfqpoint{4.975796in}{3.030412in}}%
\pgfpathlineto{\pgfqpoint{4.982619in}{3.032432in}}%
\pgfpathlineto{\pgfqpoint{5.000530in}{3.032325in}}%
\pgfpathlineto{\pgfqpoint{5.063643in}{3.048535in}}%
\pgfpathlineto{\pgfqpoint{5.067054in}{3.048442in}}%
\pgfpathlineto{\pgfqpoint{5.073025in}{3.044578in}}%
\pgfpathlineto{\pgfqpoint{5.102875in}{3.025254in}}%
\pgfpathlineto{\pgfqpoint{5.104581in}{3.027024in}}%
\pgfpathlineto{\pgfqpoint{5.121639in}{3.048202in}}%
\pgfpathlineto{\pgfqpoint{5.122492in}{3.048768in}}%
\pgfpathlineto{\pgfqpoint{5.123344in}{3.048220in}}%
\pgfpathlineto{\pgfqpoint{5.143813in}{3.026426in}}%
\pgfpathlineto{\pgfqpoint{5.144666in}{3.032723in}}%
\pgfpathlineto{\pgfqpoint{5.145519in}{3.049150in}}%
\pgfpathlineto{\pgfqpoint{5.160871in}{3.049388in}}%
\pgfpathlineto{\pgfqpoint{5.161724in}{3.050225in}}%
\pgfpathlineto{\pgfqpoint{5.163430in}{3.049778in}}%
\pgfpathlineto{\pgfqpoint{5.164283in}{3.050270in}}%
\pgfpathlineto{\pgfqpoint{5.246159in}{3.049209in}}%
\pgfpathlineto{\pgfqpoint{5.247012in}{3.050004in}}%
\pgfpathlineto{\pgfqpoint{5.248718in}{3.049942in}}%
\pgfpathlineto{\pgfqpoint{5.262364in}{3.047341in}}%
\pgfpathlineto{\pgfqpoint{5.263216in}{3.048893in}}%
\pgfpathlineto{\pgfqpoint{5.264069in}{3.049866in}}%
\pgfpathlineto{\pgfqpoint{5.266628in}{3.051402in}}%
\pgfpathlineto{\pgfqpoint{5.268334in}{3.051610in}}%
\pgfpathlineto{\pgfqpoint{5.307566in}{3.052027in}}%
\pgfpathlineto{\pgfqpoint{5.330594in}{3.054600in}}%
\pgfpathlineto{\pgfqpoint{5.386031in}{3.055035in}}%
\pgfpathlineto{\pgfqpoint{5.386884in}{3.054608in}}%
\pgfpathlineto{\pgfqpoint{5.387737in}{3.050243in}}%
\pgfpathlineto{\pgfqpoint{5.388590in}{3.048910in}}%
\pgfpathlineto{\pgfqpoint{5.390295in}{3.048509in}}%
\pgfpathlineto{\pgfqpoint{5.426969in}{3.055619in}}%
\pgfpathlineto{\pgfqpoint{5.427822in}{3.055352in}}%
\pgfpathlineto{\pgfqpoint{5.431234in}{3.051895in}}%
\pgfpathlineto{\pgfqpoint{5.432086in}{3.050023in}}%
\pgfpathlineto{\pgfqpoint{5.432939in}{3.051961in}}%
\pgfpathlineto{\pgfqpoint{5.434645in}{3.059312in}}%
\pgfpathlineto{\pgfqpoint{5.435498in}{3.059724in}}%
\pgfpathlineto{\pgfqpoint{5.445733in}{3.049473in}}%
\pgfpathlineto{\pgfqpoint{5.446585in}{3.049580in}}%
\pgfpathlineto{\pgfqpoint{5.447438in}{3.052182in}}%
\pgfpathlineto{\pgfqpoint{5.448291in}{3.051870in}}%
\pgfpathlineto{\pgfqpoint{5.449144in}{3.052836in}}%
\pgfpathlineto{\pgfqpoint{5.449997in}{3.053153in}}%
\pgfpathlineto{\pgfqpoint{5.450850in}{3.052428in}}%
\pgfpathlineto{\pgfqpoint{5.451703in}{3.049265in}}%
\pgfpathlineto{\pgfqpoint{5.453408in}{3.052291in}}%
\pgfpathlineto{\pgfqpoint{5.455114in}{3.052131in}}%
\pgfpathlineto{\pgfqpoint{5.467907in}{3.049670in}}%
\pgfpathlineto{\pgfqpoint{5.469613in}{3.052658in}}%
\pgfpathlineto{\pgfqpoint{5.473877in}{3.049519in}}%
\pgfpathlineto{\pgfqpoint{5.474730in}{3.049391in}}%
\pgfpathlineto{\pgfqpoint{5.475583in}{3.058911in}}%
\pgfpathlineto{\pgfqpoint{5.476436in}{3.063131in}}%
\pgfpathlineto{\pgfqpoint{5.478142in}{3.063409in}}%
\pgfpathlineto{\pgfqpoint{5.490082in}{3.063592in}}%
\pgfpathlineto{\pgfqpoint{5.490935in}{3.054907in}}%
\pgfpathlineto{\pgfqpoint{5.491788in}{3.051609in}}%
\pgfpathlineto{\pgfqpoint{5.552342in}{3.051804in}}%
\pgfpathlineto{\pgfqpoint{5.554048in}{3.052276in}}%
\pgfpathlineto{\pgfqpoint{5.636777in}{3.052522in}}%
\pgfpathlineto{\pgfqpoint{5.637630in}{3.054156in}}%
\pgfpathlineto{\pgfqpoint{5.690509in}{3.057728in}}%
\pgfpathlineto{\pgfqpoint{5.691362in}{3.058299in}}%
\pgfpathlineto{\pgfqpoint{5.696479in}{3.065798in}}%
\pgfpathlineto{\pgfqpoint{5.697332in}{3.067210in}}%
\pgfpathlineto{\pgfqpoint{5.698185in}{3.065003in}}%
\pgfpathlineto{\pgfqpoint{5.699037in}{3.061790in}}%
\pgfpathlineto{\pgfqpoint{5.699890in}{3.065920in}}%
\pgfpathlineto{\pgfqpoint{5.716948in}{3.066054in}}%
\pgfpathlineto{\pgfqpoint{5.718654in}{3.065703in}}%
\pgfpathlineto{\pgfqpoint{5.729741in}{3.059539in}}%
\pgfpathlineto{\pgfqpoint{5.731447in}{3.063768in}}%
\pgfpathlineto{\pgfqpoint{5.733153in}{3.068256in}}%
\pgfpathlineto{\pgfqpoint{5.734005in}{3.067502in}}%
\pgfpathlineto{\pgfqpoint{5.737417in}{3.060829in}}%
\pgfpathlineto{\pgfqpoint{5.738270in}{3.063592in}}%
\pgfpathlineto{\pgfqpoint{5.739976in}{3.087728in}}%
\pgfpathlineto{\pgfqpoint{5.740828in}{3.092042in}}%
\pgfpathlineto{\pgfqpoint{5.753622in}{3.091998in}}%
\pgfpathlineto{\pgfqpoint{5.756180in}{3.090175in}}%
\pgfpathlineto{\pgfqpoint{5.757886in}{3.090434in}}%
\pgfpathlineto{\pgfqpoint{5.824411in}{3.110068in}}%
\pgfpathlineto{\pgfqpoint{5.825263in}{3.109790in}}%
\pgfpathlineto{\pgfqpoint{5.836351in}{3.093639in}}%
\pgfpathlineto{\pgfqpoint{5.837204in}{3.093980in}}%
\pgfpathlineto{\pgfqpoint{5.843174in}{3.105217in}}%
\pgfpathlineto{\pgfqpoint{5.855114in}{3.106302in}}%
\pgfpathlineto{\pgfqpoint{5.858526in}{3.106606in}}%
\pgfpathlineto{\pgfqpoint{5.859379in}{3.109663in}}%
\pgfpathlineto{\pgfqpoint{5.860231in}{3.110737in}}%
\pgfpathlineto{\pgfqpoint{5.861937in}{3.118125in}}%
\pgfpathlineto{\pgfqpoint{5.862790in}{3.118420in}}%
\pgfpathlineto{\pgfqpoint{5.880701in}{3.092547in}}%
\pgfpathlineto{\pgfqpoint{5.881553in}{3.096238in}}%
\pgfpathlineto{\pgfqpoint{5.884112in}{3.119188in}}%
\pgfpathlineto{\pgfqpoint{5.884965in}{3.119799in}}%
\pgfpathlineto{\pgfqpoint{5.965136in}{3.118627in}}%
\pgfpathlineto{\pgfqpoint{5.978782in}{3.120943in}}%
\pgfpathlineto{\pgfqpoint{5.982193in}{3.124105in}}%
\pgfpathlineto{\pgfqpoint{5.983899in}{3.125742in}}%
\pgfpathlineto{\pgfqpoint{5.984752in}{3.125277in}}%
\pgfpathlineto{\pgfqpoint{5.985605in}{3.119156in}}%
\pgfpathlineto{\pgfqpoint{5.986458in}{3.119557in}}%
\pgfpathlineto{\pgfqpoint{5.990722in}{3.123774in}}%
\pgfpathlineto{\pgfqpoint{5.997545in}{3.123773in}}%
\pgfpathlineto{\pgfqpoint{5.999251in}{3.123564in}}%
\pgfpathlineto{\pgfqpoint{6.000956in}{3.120528in}}%
\pgfpathlineto{\pgfqpoint{6.001809in}{3.121031in}}%
\pgfpathlineto{\pgfqpoint{6.003515in}{3.121107in}}%
\pgfpathlineto{\pgfqpoint{6.004368in}{3.121718in}}%
\pgfpathlineto{\pgfqpoint{6.004368in}{3.121718in}}%
\pgfusepath{stroke}%
\end{pgfscope}%
\begin{pgfscope}%
\pgfpathrectangle{\pgfqpoint{0.481681in}{1.080890in}}{\pgfqpoint{5.785672in}{2.146863in}}%
\pgfusepath{clip}%
\pgfsetrectcap%
\pgfsetroundjoin%
\pgfsetlinewidth{0.200750pt}%
\definecolor{currentstroke}{rgb}{0.000000,0.372549,0.450980}%
\pgfsetstrokecolor{currentstroke}%
\pgfsetdash{}{0pt}%
\pgfpathmoveto{\pgfqpoint{0.744666in}{1.178475in}}%
\pgfpathlineto{\pgfqpoint{0.761724in}{1.179334in}}%
\pgfpathlineto{\pgfqpoint{0.765135in}{1.181580in}}%
\pgfpathlineto{\pgfqpoint{0.801809in}{1.199400in}}%
\pgfpathlineto{\pgfqpoint{0.802662in}{1.214389in}}%
\pgfpathlineto{\pgfqpoint{0.803515in}{1.246157in}}%
\pgfpathlineto{\pgfqpoint{0.804367in}{1.252756in}}%
\pgfpathlineto{\pgfqpoint{0.820572in}{1.257867in}}%
\pgfpathlineto{\pgfqpoint{0.821425in}{1.256248in}}%
\pgfpathlineto{\pgfqpoint{0.823984in}{1.207156in}}%
\pgfpathlineto{\pgfqpoint{0.825689in}{1.257651in}}%
\pgfpathlineto{\pgfqpoint{0.826542in}{1.258157in}}%
\pgfpathlineto{\pgfqpoint{0.827395in}{1.258157in}}%
\pgfpathlineto{\pgfqpoint{0.828248in}{1.257845in}}%
\pgfpathlineto{\pgfqpoint{0.834218in}{1.248343in}}%
\pgfpathlineto{\pgfqpoint{0.860657in}{1.205942in}}%
\pgfpathlineto{\pgfqpoint{0.861510in}{1.217910in}}%
\pgfpathlineto{\pgfqpoint{0.862363in}{1.282197in}}%
\pgfpathlineto{\pgfqpoint{0.864922in}{1.281276in}}%
\pgfpathlineto{\pgfqpoint{0.891361in}{1.270482in}}%
\pgfpathlineto{\pgfqpoint{0.893067in}{1.270554in}}%
\pgfpathlineto{\pgfqpoint{0.942534in}{1.253117in}}%
\pgfpathlineto{\pgfqpoint{0.943387in}{1.257297in}}%
\pgfpathlineto{\pgfqpoint{0.944240in}{1.274149in}}%
\pgfpathlineto{\pgfqpoint{0.945092in}{1.274663in}}%
\pgfpathlineto{\pgfqpoint{0.947651in}{1.257361in}}%
\pgfpathlineto{\pgfqpoint{0.948504in}{1.264567in}}%
\pgfpathlineto{\pgfqpoint{0.949357in}{1.274621in}}%
\pgfpathlineto{\pgfqpoint{0.950210in}{1.275783in}}%
\pgfpathlineto{\pgfqpoint{0.964709in}{1.276019in}}%
\pgfpathlineto{\pgfqpoint{0.967267in}{1.276135in}}%
\pgfpathlineto{\pgfqpoint{0.985178in}{1.276135in}}%
\pgfpathlineto{\pgfqpoint{0.986031in}{1.275849in}}%
\pgfpathlineto{\pgfqpoint{0.987736in}{1.273435in}}%
\pgfpathlineto{\pgfqpoint{0.989442in}{1.263199in}}%
\pgfpathlineto{\pgfqpoint{0.990295in}{1.257639in}}%
\pgfpathlineto{\pgfqpoint{0.991148in}{1.255336in}}%
\pgfpathlineto{\pgfqpoint{1.003941in}{1.255336in}}%
\pgfpathlineto{\pgfqpoint{1.004794in}{1.276789in}}%
\pgfpathlineto{\pgfqpoint{1.005647in}{1.352688in}}%
\pgfpathlineto{\pgfqpoint{1.006500in}{1.353048in}}%
\pgfpathlineto{\pgfqpoint{1.007353in}{1.355375in}}%
\pgfpathlineto{\pgfqpoint{1.008205in}{1.373875in}}%
\pgfpathlineto{\pgfqpoint{1.009058in}{1.374840in}}%
\pgfpathlineto{\pgfqpoint{1.023557in}{1.374840in}}%
\pgfpathlineto{\pgfqpoint{1.024410in}{1.364648in}}%
\pgfpathlineto{\pgfqpoint{1.025263in}{1.320352in}}%
\pgfpathlineto{\pgfqpoint{1.026116in}{1.313394in}}%
\pgfpathlineto{\pgfqpoint{1.026969in}{1.376076in}}%
\pgfpathlineto{\pgfqpoint{1.030380in}{1.374889in}}%
\pgfpathlineto{\pgfqpoint{1.035498in}{1.374905in}}%
\pgfpathlineto{\pgfqpoint{1.085817in}{1.375689in}}%
\pgfpathlineto{\pgfqpoint{1.086670in}{1.376886in}}%
\pgfpathlineto{\pgfqpoint{1.087523in}{1.377033in}}%
\pgfpathlineto{\pgfqpoint{1.090082in}{1.375947in}}%
\pgfpathlineto{\pgfqpoint{1.090935in}{1.376563in}}%
\pgfpathlineto{\pgfqpoint{1.091788in}{1.378474in}}%
\pgfpathlineto{\pgfqpoint{1.092640in}{1.378417in}}%
\pgfpathlineto{\pgfqpoint{1.093493in}{1.344322in}}%
\pgfpathlineto{\pgfqpoint{1.094346in}{1.241934in}}%
\pgfpathlineto{\pgfqpoint{1.095199in}{1.358642in}}%
\pgfpathlineto{\pgfqpoint{1.096052in}{1.377367in}}%
\pgfpathlineto{\pgfqpoint{1.105434in}{1.377962in}}%
\pgfpathlineto{\pgfqpoint{1.106286in}{1.365331in}}%
\pgfpathlineto{\pgfqpoint{1.107992in}{1.236031in}}%
\pgfpathlineto{\pgfqpoint{1.108845in}{1.366567in}}%
\pgfpathlineto{\pgfqpoint{1.109698in}{1.379039in}}%
\pgfpathlineto{\pgfqpoint{1.113109in}{1.377328in}}%
\pgfpathlineto{\pgfqpoint{1.116521in}{1.377735in}}%
\pgfpathlineto{\pgfqpoint{1.127608in}{1.379326in}}%
\pgfpathlineto{\pgfqpoint{1.128461in}{1.378896in}}%
\pgfpathlineto{\pgfqpoint{1.129314in}{1.379354in}}%
\pgfpathlineto{\pgfqpoint{1.131020in}{1.385639in}}%
\pgfpathlineto{\pgfqpoint{1.131873in}{1.386121in}}%
\pgfpathlineto{\pgfqpoint{1.150636in}{1.385905in}}%
\pgfpathlineto{\pgfqpoint{1.252129in}{1.377690in}}%
\pgfpathlineto{\pgfqpoint{1.253834in}{1.378272in}}%
\pgfpathlineto{\pgfqpoint{1.270892in}{1.385643in}}%
\pgfpathlineto{\pgfqpoint{1.271745in}{1.385363in}}%
\pgfpathlineto{\pgfqpoint{1.272598in}{1.379160in}}%
\pgfpathlineto{\pgfqpoint{1.273451in}{1.377784in}}%
\pgfpathlineto{\pgfqpoint{1.276009in}{1.377987in}}%
\pgfpathlineto{\pgfqpoint{1.288802in}{1.377912in}}%
\pgfpathlineto{\pgfqpoint{1.289655in}{1.382691in}}%
\pgfpathlineto{\pgfqpoint{1.291361in}{1.386718in}}%
\pgfpathlineto{\pgfqpoint{1.299037in}{1.386121in}}%
\pgfpathlineto{\pgfqpoint{1.310977in}{1.386121in}}%
\pgfpathlineto{\pgfqpoint{1.311830in}{1.386465in}}%
\pgfpathlineto{\pgfqpoint{1.312683in}{1.387578in}}%
\pgfpathlineto{\pgfqpoint{1.314389in}{1.388399in}}%
\pgfpathlineto{\pgfqpoint{1.315242in}{1.387110in}}%
\pgfpathlineto{\pgfqpoint{1.316095in}{1.388374in}}%
\pgfpathlineto{\pgfqpoint{1.316947in}{1.388589in}}%
\pgfpathlineto{\pgfqpoint{1.318653in}{1.387470in}}%
\pgfpathlineto{\pgfqpoint{1.322918in}{1.387513in}}%
\pgfpathlineto{\pgfqpoint{1.435498in}{1.387026in}}%
\pgfpathlineto{\pgfqpoint{1.521638in}{1.399068in}}%
\pgfpathlineto{\pgfqpoint{1.522491in}{1.400730in}}%
\pgfpathlineto{\pgfqpoint{1.528461in}{1.446406in}}%
\pgfpathlineto{\pgfqpoint{1.537843in}{1.517295in}}%
\pgfpathlineto{\pgfqpoint{1.556606in}{1.583171in}}%
\pgfpathlineto{\pgfqpoint{1.557459in}{1.584983in}}%
\pgfpathlineto{\pgfqpoint{1.577928in}{1.588081in}}%
\pgfpathlineto{\pgfqpoint{1.588163in}{1.589898in}}%
\pgfpathlineto{\pgfqpoint{1.600103in}{1.592292in}}%
\pgfpathlineto{\pgfqpoint{1.601809in}{1.592047in}}%
\pgfpathlineto{\pgfqpoint{1.606073in}{1.592713in}}%
\pgfpathlineto{\pgfqpoint{1.658952in}{1.593395in}}%
\pgfpathlineto{\pgfqpoint{1.659805in}{1.605539in}}%
\pgfpathlineto{\pgfqpoint{1.681979in}{2.496112in}}%
\pgfpathlineto{\pgfqpoint{1.682832in}{2.292357in}}%
\pgfpathlineto{\pgfqpoint{1.683685in}{1.594164in}}%
\pgfpathlineto{\pgfqpoint{1.685391in}{1.594375in}}%
\pgfpathlineto{\pgfqpoint{1.686244in}{2.236698in}}%
\pgfpathlineto{\pgfqpoint{1.687097in}{2.534459in}}%
\pgfpathlineto{\pgfqpoint{1.705860in}{2.537653in}}%
\pgfpathlineto{\pgfqpoint{1.710977in}{2.536903in}}%
\pgfpathlineto{\pgfqpoint{1.711830in}{2.534566in}}%
\pgfpathlineto{\pgfqpoint{1.712683in}{2.473191in}}%
\pgfpathlineto{\pgfqpoint{1.721212in}{1.624282in}}%
\pgfpathlineto{\pgfqpoint{1.722918in}{1.877555in}}%
\pgfpathlineto{\pgfqpoint{1.726329in}{2.443282in}}%
\pgfpathlineto{\pgfqpoint{1.727182in}{2.534624in}}%
\pgfpathlineto{\pgfqpoint{1.728035in}{2.536392in}}%
\pgfpathlineto{\pgfqpoint{1.745945in}{2.536392in}}%
\pgfpathlineto{\pgfqpoint{1.746798in}{2.535865in}}%
\pgfpathlineto{\pgfqpoint{1.748504in}{2.532152in}}%
\pgfpathlineto{\pgfqpoint{1.750210in}{2.536448in}}%
\pgfpathlineto{\pgfqpoint{1.751916in}{2.536625in}}%
\pgfpathlineto{\pgfqpoint{1.753621in}{2.536473in}}%
\pgfpathlineto{\pgfqpoint{1.763856in}{2.537068in}}%
\pgfpathlineto{\pgfqpoint{1.765562in}{2.537802in}}%
\pgfpathlineto{\pgfqpoint{1.767267in}{2.537960in}}%
\pgfpathlineto{\pgfqpoint{1.768120in}{2.339104in}}%
\pgfpathlineto{\pgfqpoint{1.768973in}{1.747800in}}%
\pgfpathlineto{\pgfqpoint{1.769826in}{1.595876in}}%
\pgfpathlineto{\pgfqpoint{1.786884in}{1.596231in}}%
\pgfpathlineto{\pgfqpoint{1.787736in}{1.596740in}}%
\pgfpathlineto{\pgfqpoint{1.788589in}{1.620026in}}%
\pgfpathlineto{\pgfqpoint{1.789442in}{2.452880in}}%
\pgfpathlineto{\pgfqpoint{1.790295in}{2.538154in}}%
\pgfpathlineto{\pgfqpoint{1.812470in}{2.537841in}}%
\pgfpathlineto{\pgfqpoint{1.814176in}{2.537816in}}%
\pgfpathlineto{\pgfqpoint{1.815029in}{2.528133in}}%
\pgfpathlineto{\pgfqpoint{1.816734in}{2.279779in}}%
\pgfpathlineto{\pgfqpoint{1.820146in}{1.729332in}}%
\pgfpathlineto{\pgfqpoint{1.820999in}{1.649278in}}%
\pgfpathlineto{\pgfqpoint{1.822704in}{2.076752in}}%
\pgfpathlineto{\pgfqpoint{1.823557in}{2.308495in}}%
\pgfpathlineto{\pgfqpoint{1.824410in}{2.421911in}}%
\pgfpathlineto{\pgfqpoint{1.825263in}{2.583160in}}%
\pgfpathlineto{\pgfqpoint{1.826116in}{2.582617in}}%
\pgfpathlineto{\pgfqpoint{1.826969in}{2.580519in}}%
\pgfpathlineto{\pgfqpoint{1.827822in}{2.580702in}}%
\pgfpathlineto{\pgfqpoint{1.828675in}{2.580471in}}%
\pgfpathlineto{\pgfqpoint{1.829527in}{2.583357in}}%
\pgfpathlineto{\pgfqpoint{1.830380in}{2.583743in}}%
\pgfpathlineto{\pgfqpoint{1.832086in}{2.579555in}}%
\pgfpathlineto{\pgfqpoint{1.832939in}{2.580903in}}%
\pgfpathlineto{\pgfqpoint{1.833792in}{2.436393in}}%
\pgfpathlineto{\pgfqpoint{1.834645in}{2.584858in}}%
\pgfpathlineto{\pgfqpoint{1.836350in}{2.583694in}}%
\pgfpathlineto{\pgfqpoint{1.842321in}{2.578966in}}%
\pgfpathlineto{\pgfqpoint{1.843174in}{2.578658in}}%
\pgfpathlineto{\pgfqpoint{1.844026in}{2.584163in}}%
\pgfpathlineto{\pgfqpoint{1.844879in}{2.587584in}}%
\pgfpathlineto{\pgfqpoint{1.845732in}{2.587981in}}%
\pgfpathlineto{\pgfqpoint{1.846585in}{2.581542in}}%
\pgfpathlineto{\pgfqpoint{1.847438in}{2.585046in}}%
\pgfpathlineto{\pgfqpoint{1.848291in}{2.592127in}}%
\pgfpathlineto{\pgfqpoint{1.849144in}{2.589403in}}%
\pgfpathlineto{\pgfqpoint{1.849997in}{2.592134in}}%
\pgfpathlineto{\pgfqpoint{1.850849in}{2.592442in}}%
\pgfpathlineto{\pgfqpoint{1.855114in}{2.592623in}}%
\pgfpathlineto{\pgfqpoint{1.855967in}{2.585806in}}%
\pgfpathlineto{\pgfqpoint{1.856820in}{2.591999in}}%
\pgfpathlineto{\pgfqpoint{1.857672in}{2.589714in}}%
\pgfpathlineto{\pgfqpoint{1.858525in}{2.588829in}}%
\pgfpathlineto{\pgfqpoint{1.861937in}{2.581247in}}%
\pgfpathlineto{\pgfqpoint{1.862790in}{2.580982in}}%
\pgfpathlineto{\pgfqpoint{1.863643in}{2.587031in}}%
\pgfpathlineto{\pgfqpoint{1.864495in}{2.589761in}}%
\pgfpathlineto{\pgfqpoint{1.865348in}{2.590046in}}%
\pgfpathlineto{\pgfqpoint{1.866201in}{2.587385in}}%
\pgfpathlineto{\pgfqpoint{1.867054in}{2.591585in}}%
\pgfpathlineto{\pgfqpoint{1.867907in}{2.603723in}}%
\pgfpathlineto{\pgfqpoint{1.868760in}{2.602401in}}%
\pgfpathlineto{\pgfqpoint{1.869613in}{2.594701in}}%
\pgfpathlineto{\pgfqpoint{1.870466in}{2.602528in}}%
\pgfpathlineto{\pgfqpoint{1.871318in}{2.605730in}}%
\pgfpathlineto{\pgfqpoint{1.875583in}{2.607164in}}%
\pgfpathlineto{\pgfqpoint{1.878994in}{2.605741in}}%
\pgfpathlineto{\pgfqpoint{1.883259in}{2.604075in}}%
\pgfpathlineto{\pgfqpoint{1.884965in}{2.605470in}}%
\pgfpathlineto{\pgfqpoint{1.885817in}{2.605800in}}%
\pgfpathlineto{\pgfqpoint{1.886670in}{2.606545in}}%
\pgfpathlineto{\pgfqpoint{1.888376in}{2.604372in}}%
\pgfpathlineto{\pgfqpoint{1.889229in}{2.604943in}}%
\pgfpathlineto{\pgfqpoint{1.891788in}{2.607634in}}%
\pgfpathlineto{\pgfqpoint{1.908845in}{2.608558in}}%
\pgfpathlineto{\pgfqpoint{1.947225in}{2.604116in}}%
\pgfpathlineto{\pgfqpoint{1.948078in}{2.605479in}}%
\pgfpathlineto{\pgfqpoint{1.948930in}{2.607953in}}%
\pgfpathlineto{\pgfqpoint{1.959165in}{2.607712in}}%
\pgfpathlineto{\pgfqpoint{1.965988in}{2.607265in}}%
\pgfpathlineto{\pgfqpoint{1.967694in}{2.608792in}}%
\pgfpathlineto{\pgfqpoint{1.970252in}{2.609363in}}%
\pgfpathlineto{\pgfqpoint{2.018866in}{2.609110in}}%
\pgfpathlineto{\pgfqpoint{2.031660in}{2.608676in}}%
\pgfpathlineto{\pgfqpoint{2.032513in}{2.618740in}}%
\pgfpathlineto{\pgfqpoint{2.033365in}{2.620291in}}%
\pgfpathlineto{\pgfqpoint{2.093067in}{2.620078in}}%
\pgfpathlineto{\pgfqpoint{2.225263in}{2.609493in}}%
\pgfpathlineto{\pgfqpoint{2.244879in}{2.611406in}}%
\pgfpathlineto{\pgfqpoint{2.245732in}{2.608846in}}%
\pgfpathlineto{\pgfqpoint{2.255967in}{2.559612in}}%
\pgfpathlineto{\pgfqpoint{2.256820in}{2.560727in}}%
\pgfpathlineto{\pgfqpoint{2.257672in}{2.639077in}}%
\pgfpathlineto{\pgfqpoint{2.258525in}{2.673333in}}%
\pgfpathlineto{\pgfqpoint{2.265348in}{2.615228in}}%
\pgfpathlineto{\pgfqpoint{2.266201in}{2.614695in}}%
\pgfpathlineto{\pgfqpoint{2.279847in}{2.615334in}}%
\pgfpathlineto{\pgfqpoint{2.280700in}{2.617626in}}%
\pgfpathlineto{\pgfqpoint{2.299464in}{2.726404in}}%
\pgfpathlineto{\pgfqpoint{2.300316in}{2.727953in}}%
\pgfpathlineto{\pgfqpoint{2.301169in}{2.683654in}}%
\pgfpathlineto{\pgfqpoint{2.306287in}{2.693692in}}%
\pgfpathlineto{\pgfqpoint{2.307139in}{2.690674in}}%
\pgfpathlineto{\pgfqpoint{2.320785in}{2.622341in}}%
\pgfpathlineto{\pgfqpoint{2.322491in}{2.622399in}}%
\pgfpathlineto{\pgfqpoint{2.346372in}{2.626878in}}%
\pgfpathlineto{\pgfqpoint{2.347225in}{2.627616in}}%
\pgfpathlineto{\pgfqpoint{2.365135in}{2.668096in}}%
\pgfpathlineto{\pgfqpoint{2.377075in}{2.694992in}}%
\pgfpathlineto{\pgfqpoint{2.377928in}{2.695529in}}%
\pgfpathlineto{\pgfqpoint{2.379634in}{2.695730in}}%
\pgfpathlineto{\pgfqpoint{2.380487in}{2.696738in}}%
\pgfpathlineto{\pgfqpoint{2.381340in}{2.698247in}}%
\pgfpathlineto{\pgfqpoint{2.394986in}{2.699909in}}%
\pgfpathlineto{\pgfqpoint{2.395839in}{2.708133in}}%
\pgfpathlineto{\pgfqpoint{2.398397in}{2.756368in}}%
\pgfpathlineto{\pgfqpoint{2.399250in}{2.763624in}}%
\pgfpathlineto{\pgfqpoint{2.400103in}{2.718703in}}%
\pgfpathlineto{\pgfqpoint{2.400956in}{2.721904in}}%
\pgfpathlineto{\pgfqpoint{2.401809in}{2.695730in}}%
\pgfpathlineto{\pgfqpoint{2.402662in}{2.703779in}}%
\pgfpathlineto{\pgfqpoint{2.403515in}{2.738760in}}%
\pgfpathlineto{\pgfqpoint{2.404368in}{2.735854in}}%
\pgfpathlineto{\pgfqpoint{2.405220in}{2.675209in}}%
\pgfpathlineto{\pgfqpoint{2.406073in}{2.758452in}}%
\pgfpathlineto{\pgfqpoint{2.408632in}{2.645639in}}%
\pgfpathlineto{\pgfqpoint{2.409485in}{2.637453in}}%
\pgfpathlineto{\pgfqpoint{2.417161in}{2.768129in}}%
\pgfpathlineto{\pgfqpoint{2.418014in}{2.748372in}}%
\pgfpathlineto{\pgfqpoint{2.418867in}{2.645576in}}%
\pgfpathlineto{\pgfqpoint{2.420572in}{2.710787in}}%
\pgfpathlineto{\pgfqpoint{2.421425in}{2.715268in}}%
\pgfpathlineto{\pgfqpoint{2.423131in}{2.715471in}}%
\pgfpathlineto{\pgfqpoint{2.437630in}{2.715729in}}%
\pgfpathlineto{\pgfqpoint{2.440188in}{2.722728in}}%
\pgfpathlineto{\pgfqpoint{2.458952in}{2.776588in}}%
\pgfpathlineto{\pgfqpoint{2.459805in}{2.777381in}}%
\pgfpathlineto{\pgfqpoint{2.460658in}{2.792665in}}%
\pgfpathlineto{\pgfqpoint{2.461510in}{2.788613in}}%
\pgfpathlineto{\pgfqpoint{2.470039in}{2.717302in}}%
\pgfpathlineto{\pgfqpoint{2.470892in}{2.716131in}}%
\pgfpathlineto{\pgfqpoint{2.525476in}{2.805325in}}%
\pgfpathlineto{\pgfqpoint{2.528888in}{2.806881in}}%
\pgfpathlineto{\pgfqpoint{2.530594in}{2.807619in}}%
\pgfpathlineto{\pgfqpoint{2.531446in}{2.803790in}}%
\pgfpathlineto{\pgfqpoint{2.538270in}{2.728262in}}%
\pgfpathlineto{\pgfqpoint{2.539122in}{2.724281in}}%
\pgfpathlineto{\pgfqpoint{2.539975in}{2.809186in}}%
\pgfpathlineto{\pgfqpoint{2.540828in}{2.806571in}}%
\pgfpathlineto{\pgfqpoint{2.541681in}{2.809658in}}%
\pgfpathlineto{\pgfqpoint{2.542534in}{2.811661in}}%
\pgfpathlineto{\pgfqpoint{2.545093in}{2.811387in}}%
\pgfpathlineto{\pgfqpoint{2.563856in}{2.808603in}}%
\pgfpathlineto{\pgfqpoint{2.564709in}{2.811959in}}%
\pgfpathlineto{\pgfqpoint{2.565562in}{2.812414in}}%
\pgfpathlineto{\pgfqpoint{2.568973in}{2.812697in}}%
\pgfpathlineto{\pgfqpoint{2.580913in}{2.814024in}}%
\pgfpathlineto{\pgfqpoint{2.581766in}{2.813222in}}%
\pgfpathlineto{\pgfqpoint{2.584325in}{2.813809in}}%
\pgfpathlineto{\pgfqpoint{2.585178in}{2.815983in}}%
\pgfpathlineto{\pgfqpoint{2.586884in}{2.829082in}}%
\pgfpathlineto{\pgfqpoint{2.587736in}{2.833217in}}%
\pgfpathlineto{\pgfqpoint{2.592854in}{2.834147in}}%
\pgfpathlineto{\pgfqpoint{2.593707in}{2.834124in}}%
\pgfpathlineto{\pgfqpoint{2.600530in}{2.829076in}}%
\pgfpathlineto{\pgfqpoint{2.601383in}{2.828440in}}%
\pgfpathlineto{\pgfqpoint{2.602235in}{2.828275in}}%
\pgfpathlineto{\pgfqpoint{2.608206in}{2.837187in}}%
\pgfpathlineto{\pgfqpoint{2.613323in}{2.839110in}}%
\pgfpathlineto{\pgfqpoint{2.615881in}{2.838857in}}%
\pgfpathlineto{\pgfqpoint{2.626969in}{2.837443in}}%
\pgfpathlineto{\pgfqpoint{2.658525in}{2.837716in}}%
\pgfpathlineto{\pgfqpoint{2.665348in}{2.837790in}}%
\pgfpathlineto{\pgfqpoint{2.666201in}{2.830733in}}%
\pgfpathlineto{\pgfqpoint{2.667054in}{2.816971in}}%
\pgfpathlineto{\pgfqpoint{2.667907in}{2.816940in}}%
\pgfpathlineto{\pgfqpoint{2.683259in}{2.838270in}}%
\pgfpathlineto{\pgfqpoint{2.684112in}{2.838904in}}%
\pgfpathlineto{\pgfqpoint{2.702022in}{2.841188in}}%
\pgfpathlineto{\pgfqpoint{2.702875in}{2.840122in}}%
\pgfpathlineto{\pgfqpoint{2.704581in}{2.820675in}}%
\pgfpathlineto{\pgfqpoint{2.706287in}{2.821115in}}%
\pgfpathlineto{\pgfqpoint{2.707139in}{2.824091in}}%
\pgfpathlineto{\pgfqpoint{2.707992in}{2.821069in}}%
\pgfpathlineto{\pgfqpoint{2.709698in}{2.821260in}}%
\pgfpathlineto{\pgfqpoint{2.729314in}{2.821441in}}%
\pgfpathlineto{\pgfqpoint{2.752342in}{2.821016in}}%
\pgfpathlineto{\pgfqpoint{2.789016in}{2.823111in}}%
\pgfpathlineto{\pgfqpoint{2.829954in}{2.823654in}}%
\pgfpathlineto{\pgfqpoint{2.834218in}{2.823611in}}%
\pgfpathlineto{\pgfqpoint{2.846159in}{2.821058in}}%
\pgfpathlineto{\pgfqpoint{2.847012in}{2.821191in}}%
\pgfpathlineto{\pgfqpoint{2.847864in}{2.823060in}}%
\pgfpathlineto{\pgfqpoint{2.848717in}{2.823543in}}%
\pgfpathlineto{\pgfqpoint{2.849570in}{2.823491in}}%
\pgfpathlineto{\pgfqpoint{2.851276in}{2.824138in}}%
\pgfpathlineto{\pgfqpoint{2.856393in}{2.824863in}}%
\pgfpathlineto{\pgfqpoint{2.859805in}{2.829160in}}%
\pgfpathlineto{\pgfqpoint{2.870039in}{2.842464in}}%
\pgfpathlineto{\pgfqpoint{2.870892in}{2.829753in}}%
\pgfpathlineto{\pgfqpoint{2.871745in}{2.825605in}}%
\pgfpathlineto{\pgfqpoint{2.900743in}{2.853899in}}%
\pgfpathlineto{\pgfqpoint{2.901596in}{2.836724in}}%
\pgfpathlineto{\pgfqpoint{2.902449in}{2.826942in}}%
\pgfpathlineto{\pgfqpoint{2.915242in}{2.828297in}}%
\pgfpathlineto{\pgfqpoint{2.919506in}{2.830555in}}%
\pgfpathlineto{\pgfqpoint{2.968973in}{2.857559in}}%
\pgfpathlineto{\pgfqpoint{2.969826in}{2.856612in}}%
\pgfpathlineto{\pgfqpoint{2.970679in}{2.840945in}}%
\pgfpathlineto{\pgfqpoint{2.971532in}{2.838158in}}%
\pgfpathlineto{\pgfqpoint{2.972385in}{2.859246in}}%
\pgfpathlineto{\pgfqpoint{2.973238in}{2.862573in}}%
\pgfpathlineto{\pgfqpoint{2.976649in}{2.863243in}}%
\pgfpathlineto{\pgfqpoint{2.990295in}{2.866227in}}%
\pgfpathlineto{\pgfqpoint{2.992001in}{2.835618in}}%
\pgfpathlineto{\pgfqpoint{2.993707in}{2.863815in}}%
\pgfpathlineto{\pgfqpoint{2.994560in}{2.863134in}}%
\pgfpathlineto{\pgfqpoint{2.995412in}{2.874386in}}%
\pgfpathlineto{\pgfqpoint{2.996265in}{2.876610in}}%
\pgfpathlineto{\pgfqpoint{2.997971in}{2.875086in}}%
\pgfpathlineto{\pgfqpoint{3.003088in}{2.873848in}}%
\pgfpathlineto{\pgfqpoint{3.003941in}{2.874106in}}%
\pgfpathlineto{\pgfqpoint{3.018440in}{2.884959in}}%
\pgfpathlineto{\pgfqpoint{3.030380in}{2.882271in}}%
\pgfpathlineto{\pgfqpoint{3.059378in}{2.875884in}}%
\pgfpathlineto{\pgfqpoint{3.104581in}{2.876573in}}%
\pgfpathlineto{\pgfqpoint{3.116521in}{2.876825in}}%
\pgfpathlineto{\pgfqpoint{3.134432in}{2.878739in}}%
\pgfpathlineto{\pgfqpoint{3.136137in}{2.880004in}}%
\pgfpathlineto{\pgfqpoint{3.138696in}{2.880098in}}%
\pgfpathlineto{\pgfqpoint{3.207779in}{2.880022in}}%
\pgfpathlineto{\pgfqpoint{3.212896in}{2.876765in}}%
\pgfpathlineto{\pgfqpoint{3.237630in}{2.860721in}}%
\pgfpathlineto{\pgfqpoint{3.238483in}{2.855819in}}%
\pgfpathlineto{\pgfqpoint{3.239336in}{2.843921in}}%
\pgfpathlineto{\pgfqpoint{3.240189in}{2.840822in}}%
\pgfpathlineto{\pgfqpoint{3.241041in}{2.840813in}}%
\pgfpathlineto{\pgfqpoint{3.244453in}{2.857393in}}%
\pgfpathlineto{\pgfqpoint{3.247864in}{2.874346in}}%
\pgfpathlineto{\pgfqpoint{3.248717in}{2.875985in}}%
\pgfpathlineto{\pgfqpoint{3.256393in}{2.842647in}}%
\pgfpathlineto{\pgfqpoint{3.258099in}{2.833123in}}%
\pgfpathlineto{\pgfqpoint{3.260658in}{2.833213in}}%
\pgfpathlineto{\pgfqpoint{3.276009in}{2.833213in}}%
\pgfpathlineto{\pgfqpoint{3.276862in}{2.835607in}}%
\pgfpathlineto{\pgfqpoint{3.277715in}{2.861053in}}%
\pgfpathlineto{\pgfqpoint{3.278568in}{2.835977in}}%
\pgfpathlineto{\pgfqpoint{3.279421in}{2.835173in}}%
\pgfpathlineto{\pgfqpoint{3.296479in}{2.876010in}}%
\pgfpathlineto{\pgfqpoint{3.297331in}{2.876573in}}%
\pgfpathlineto{\pgfqpoint{3.318653in}{2.876538in}}%
\pgfpathlineto{\pgfqpoint{3.320359in}{2.830162in}}%
\pgfpathlineto{\pgfqpoint{3.322918in}{2.830091in}}%
\pgfpathlineto{\pgfqpoint{3.379208in}{2.831081in}}%
\pgfpathlineto{\pgfqpoint{3.380061in}{2.830211in}}%
\pgfpathlineto{\pgfqpoint{3.380914in}{2.828588in}}%
\pgfpathlineto{\pgfqpoint{3.381766in}{2.828983in}}%
\pgfpathlineto{\pgfqpoint{3.405647in}{2.829250in}}%
\pgfpathlineto{\pgfqpoint{3.413323in}{2.830651in}}%
\pgfpathlineto{\pgfqpoint{3.419293in}{2.829061in}}%
\pgfpathlineto{\pgfqpoint{3.421852in}{2.830323in}}%
\pgfpathlineto{\pgfqpoint{3.425263in}{2.832134in}}%
\pgfpathlineto{\pgfqpoint{3.433792in}{2.833165in}}%
\pgfpathlineto{\pgfqpoint{3.445732in}{2.834567in}}%
\pgfpathlineto{\pgfqpoint{3.446585in}{2.837533in}}%
\pgfpathlineto{\pgfqpoint{3.447438in}{2.843344in}}%
\pgfpathlineto{\pgfqpoint{3.448291in}{2.844562in}}%
\pgfpathlineto{\pgfqpoint{3.462790in}{2.851680in}}%
\pgfpathlineto{\pgfqpoint{3.463643in}{2.849158in}}%
\pgfpathlineto{\pgfqpoint{3.467054in}{2.834205in}}%
\pgfpathlineto{\pgfqpoint{3.467907in}{2.836174in}}%
\pgfpathlineto{\pgfqpoint{3.469613in}{2.835320in}}%
\pgfpathlineto{\pgfqpoint{3.473877in}{2.844716in}}%
\pgfpathlineto{\pgfqpoint{3.474730in}{2.845066in}}%
\pgfpathlineto{\pgfqpoint{3.487523in}{2.842731in}}%
\pgfpathlineto{\pgfqpoint{3.488376in}{2.842764in}}%
\pgfpathlineto{\pgfqpoint{3.490082in}{2.843702in}}%
\pgfpathlineto{\pgfqpoint{3.525903in}{2.854421in}}%
\pgfpathlineto{\pgfqpoint{3.526756in}{2.853633in}}%
\pgfpathlineto{\pgfqpoint{3.527609in}{2.841682in}}%
\pgfpathlineto{\pgfqpoint{3.528462in}{2.839648in}}%
\pgfpathlineto{\pgfqpoint{3.540402in}{2.851100in}}%
\pgfpathlineto{\pgfqpoint{3.544666in}{2.855047in}}%
\pgfpathlineto{\pgfqpoint{3.546372in}{2.840797in}}%
\pgfpathlineto{\pgfqpoint{3.548078in}{2.858452in}}%
\pgfpathlineto{\pgfqpoint{3.548931in}{2.859158in}}%
\pgfpathlineto{\pgfqpoint{3.550636in}{2.858409in}}%
\pgfpathlineto{\pgfqpoint{3.560871in}{2.855603in}}%
\pgfpathlineto{\pgfqpoint{3.561724in}{2.855865in}}%
\pgfpathlineto{\pgfqpoint{3.562577in}{2.857449in}}%
\pgfpathlineto{\pgfqpoint{3.563430in}{2.846203in}}%
\pgfpathlineto{\pgfqpoint{3.564282in}{2.847266in}}%
\pgfpathlineto{\pgfqpoint{3.565988in}{2.862521in}}%
\pgfpathlineto{\pgfqpoint{3.566841in}{2.862369in}}%
\pgfpathlineto{\pgfqpoint{3.567694in}{2.840809in}}%
\pgfpathlineto{\pgfqpoint{3.568547in}{2.841292in}}%
\pgfpathlineto{\pgfqpoint{3.570253in}{2.843610in}}%
\pgfpathlineto{\pgfqpoint{3.577076in}{2.862946in}}%
\pgfpathlineto{\pgfqpoint{3.582193in}{2.873751in}}%
\pgfpathlineto{\pgfqpoint{3.583046in}{2.873675in}}%
\pgfpathlineto{\pgfqpoint{3.583899in}{2.849440in}}%
\pgfpathlineto{\pgfqpoint{3.584752in}{2.860528in}}%
\pgfpathlineto{\pgfqpoint{3.585604in}{2.850421in}}%
\pgfpathlineto{\pgfqpoint{3.586457in}{2.845284in}}%
\pgfpathlineto{\pgfqpoint{3.587310in}{2.872900in}}%
\pgfpathlineto{\pgfqpoint{3.588163in}{2.872463in}}%
\pgfpathlineto{\pgfqpoint{3.589016in}{2.867721in}}%
\pgfpathlineto{\pgfqpoint{3.589869in}{2.876702in}}%
\pgfpathlineto{\pgfqpoint{3.590722in}{2.866380in}}%
\pgfpathlineto{\pgfqpoint{3.591575in}{2.864155in}}%
\pgfpathlineto{\pgfqpoint{3.605221in}{2.884985in}}%
\pgfpathlineto{\pgfqpoint{3.606073in}{2.886005in}}%
\pgfpathlineto{\pgfqpoint{3.606926in}{2.884861in}}%
\pgfpathlineto{\pgfqpoint{3.607779in}{2.864538in}}%
\pgfpathlineto{\pgfqpoint{3.608632in}{2.871580in}}%
\pgfpathlineto{\pgfqpoint{3.609485in}{2.874386in}}%
\pgfpathlineto{\pgfqpoint{3.610338in}{2.886472in}}%
\pgfpathlineto{\pgfqpoint{3.611191in}{2.868268in}}%
\pgfpathlineto{\pgfqpoint{3.612044in}{2.862161in}}%
\pgfpathlineto{\pgfqpoint{3.630807in}{2.863216in}}%
\pgfpathlineto{\pgfqpoint{3.634218in}{2.865569in}}%
\pgfpathlineto{\pgfqpoint{3.670892in}{2.891750in}}%
\pgfpathlineto{\pgfqpoint{3.675157in}{2.893476in}}%
\pgfpathlineto{\pgfqpoint{3.685391in}{2.893142in}}%
\pgfpathlineto{\pgfqpoint{3.686244in}{2.889437in}}%
\pgfpathlineto{\pgfqpoint{3.688803in}{2.869535in}}%
\pgfpathlineto{\pgfqpoint{3.689656in}{2.875671in}}%
\pgfpathlineto{\pgfqpoint{3.691361in}{2.892675in}}%
\pgfpathlineto{\pgfqpoint{3.692214in}{2.912225in}}%
\pgfpathlineto{\pgfqpoint{3.693067in}{2.919992in}}%
\pgfpathlineto{\pgfqpoint{3.693920in}{2.923810in}}%
\pgfpathlineto{\pgfqpoint{3.705007in}{2.923810in}}%
\pgfpathlineto{\pgfqpoint{3.705860in}{2.923069in}}%
\pgfpathlineto{\pgfqpoint{3.706713in}{2.911083in}}%
\pgfpathlineto{\pgfqpoint{3.707566in}{2.908141in}}%
\pgfpathlineto{\pgfqpoint{3.708419in}{2.908946in}}%
\pgfpathlineto{\pgfqpoint{3.710978in}{2.907947in}}%
\pgfpathlineto{\pgfqpoint{3.711830in}{2.909020in}}%
\pgfpathlineto{\pgfqpoint{3.712683in}{2.914066in}}%
\pgfpathlineto{\pgfqpoint{3.713536in}{2.923130in}}%
\pgfpathlineto{\pgfqpoint{3.714389in}{2.928568in}}%
\pgfpathlineto{\pgfqpoint{3.716948in}{2.932069in}}%
\pgfpathlineto{\pgfqpoint{3.717801in}{2.931429in}}%
\pgfpathlineto{\pgfqpoint{3.726329in}{2.909150in}}%
\pgfpathlineto{\pgfqpoint{3.728035in}{2.909004in}}%
\pgfpathlineto{\pgfqpoint{3.729741in}{2.909004in}}%
\pgfpathlineto{\pgfqpoint{3.731447in}{2.921738in}}%
\pgfpathlineto{\pgfqpoint{3.732299in}{2.911168in}}%
\pgfpathlineto{\pgfqpoint{3.733152in}{2.909673in}}%
\pgfpathlineto{\pgfqpoint{3.738270in}{2.924175in}}%
\pgfpathlineto{\pgfqpoint{3.739975in}{2.936338in}}%
\pgfpathlineto{\pgfqpoint{3.741681in}{2.937086in}}%
\pgfpathlineto{\pgfqpoint{3.748504in}{2.939143in}}%
\pgfpathlineto{\pgfqpoint{3.750210in}{2.925521in}}%
\pgfpathlineto{\pgfqpoint{3.754474in}{2.925089in}}%
\pgfpathlineto{\pgfqpoint{3.760444in}{2.920401in}}%
\pgfpathlineto{\pgfqpoint{3.763003in}{2.921111in}}%
\pgfpathlineto{\pgfqpoint{3.779208in}{2.926023in}}%
\pgfpathlineto{\pgfqpoint{3.811617in}{2.928975in}}%
\pgfpathlineto{\pgfqpoint{3.812470in}{2.933781in}}%
\pgfpathlineto{\pgfqpoint{3.814176in}{2.934109in}}%
\pgfpathlineto{\pgfqpoint{3.832086in}{2.935074in}}%
\pgfpathlineto{\pgfqpoint{3.832939in}{2.930006in}}%
\pgfpathlineto{\pgfqpoint{3.833792in}{2.931081in}}%
\pgfpathlineto{\pgfqpoint{3.834645in}{2.937946in}}%
\pgfpathlineto{\pgfqpoint{3.835498in}{2.934092in}}%
\pgfpathlineto{\pgfqpoint{3.836351in}{2.928739in}}%
\pgfpathlineto{\pgfqpoint{3.837204in}{2.926832in}}%
\pgfpathlineto{\pgfqpoint{3.851702in}{2.943850in}}%
\pgfpathlineto{\pgfqpoint{3.852555in}{2.942597in}}%
\pgfpathlineto{\pgfqpoint{3.853408in}{2.945037in}}%
\pgfpathlineto{\pgfqpoint{3.854261in}{2.940592in}}%
\pgfpathlineto{\pgfqpoint{3.855967in}{2.927114in}}%
\pgfpathlineto{\pgfqpoint{3.856820in}{2.927159in}}%
\pgfpathlineto{\pgfqpoint{3.868760in}{2.945181in}}%
\pgfpathlineto{\pgfqpoint{3.869613in}{2.943515in}}%
\pgfpathlineto{\pgfqpoint{3.870466in}{2.933706in}}%
\pgfpathlineto{\pgfqpoint{3.871319in}{2.947782in}}%
\pgfpathlineto{\pgfqpoint{3.872172in}{2.946531in}}%
\pgfpathlineto{\pgfqpoint{3.873024in}{2.934154in}}%
\pgfpathlineto{\pgfqpoint{3.874730in}{2.948137in}}%
\pgfpathlineto{\pgfqpoint{4.096479in}{2.949346in}}%
\pgfpathlineto{\pgfqpoint{4.115242in}{2.954019in}}%
\pgfpathlineto{\pgfqpoint{4.116095in}{2.966926in}}%
\pgfpathlineto{\pgfqpoint{4.116948in}{2.970919in}}%
\pgfpathlineto{\pgfqpoint{4.247438in}{2.941589in}}%
\pgfpathlineto{\pgfqpoint{4.264496in}{2.926614in}}%
\pgfpathlineto{\pgfqpoint{4.271319in}{2.920834in}}%
\pgfpathlineto{\pgfqpoint{4.272172in}{2.967557in}}%
\pgfpathlineto{\pgfqpoint{4.273024in}{2.976780in}}%
\pgfpathlineto{\pgfqpoint{4.278995in}{2.977354in}}%
\pgfpathlineto{\pgfqpoint{4.280700in}{2.978439in}}%
\pgfpathlineto{\pgfqpoint{4.281553in}{2.979124in}}%
\pgfpathlineto{\pgfqpoint{4.282406in}{2.977473in}}%
\pgfpathlineto{\pgfqpoint{4.283259in}{2.956559in}}%
\pgfpathlineto{\pgfqpoint{4.284112in}{2.942670in}}%
\pgfpathlineto{\pgfqpoint{4.284965in}{2.971901in}}%
\pgfpathlineto{\pgfqpoint{4.285818in}{2.987223in}}%
\pgfpathlineto{\pgfqpoint{4.286671in}{2.980326in}}%
\pgfpathlineto{\pgfqpoint{4.287523in}{2.978323in}}%
\pgfpathlineto{\pgfqpoint{4.291788in}{2.977499in}}%
\pgfpathlineto{\pgfqpoint{4.292641in}{2.977856in}}%
\pgfpathlineto{\pgfqpoint{4.302022in}{2.991547in}}%
\pgfpathlineto{\pgfqpoint{4.302875in}{2.991846in}}%
\pgfpathlineto{\pgfqpoint{4.303728in}{2.991846in}}%
\pgfpathlineto{\pgfqpoint{4.304581in}{2.988038in}}%
\pgfpathlineto{\pgfqpoint{4.307140in}{3.010904in}}%
\pgfpathlineto{\pgfqpoint{4.318227in}{3.011204in}}%
\pgfpathlineto{\pgfqpoint{4.319080in}{3.010043in}}%
\pgfpathlineto{\pgfqpoint{4.319933in}{3.000307in}}%
\pgfpathlineto{\pgfqpoint{4.320786in}{3.000307in}}%
\pgfpathlineto{\pgfqpoint{4.321639in}{3.002854in}}%
\pgfpathlineto{\pgfqpoint{4.322491in}{3.012158in}}%
\pgfpathlineto{\pgfqpoint{4.324197in}{3.013082in}}%
\pgfpathlineto{\pgfqpoint{4.330167in}{3.015172in}}%
\pgfpathlineto{\pgfqpoint{4.334432in}{3.015412in}}%
\pgfpathlineto{\pgfqpoint{4.336138in}{3.013443in}}%
\pgfpathlineto{\pgfqpoint{4.345519in}{3.000788in}}%
\pgfpathlineto{\pgfqpoint{4.346372in}{3.000316in}}%
\pgfpathlineto{\pgfqpoint{4.348078in}{3.006569in}}%
\pgfpathlineto{\pgfqpoint{4.350636in}{3.016405in}}%
\pgfpathlineto{\pgfqpoint{4.351489in}{3.016867in}}%
\pgfpathlineto{\pgfqpoint{4.387310in}{3.016755in}}%
\pgfpathlineto{\pgfqpoint{4.388163in}{3.016830in}}%
\pgfpathlineto{\pgfqpoint{4.389869in}{3.015505in}}%
\pgfpathlineto{\pgfqpoint{4.405221in}{3.002222in}}%
\pgfpathlineto{\pgfqpoint{4.406074in}{3.002070in}}%
\pgfpathlineto{\pgfqpoint{4.407779in}{3.020210in}}%
\pgfpathlineto{\pgfqpoint{4.408632in}{3.020146in}}%
\pgfpathlineto{\pgfqpoint{4.409485in}{3.020740in}}%
\pgfpathlineto{\pgfqpoint{4.428248in}{3.020753in}}%
\pgfpathlineto{\pgfqpoint{4.436777in}{3.021005in}}%
\pgfpathlineto{\pgfqpoint{4.453835in}{3.021458in}}%
\pgfpathlineto{\pgfqpoint{4.469187in}{3.021717in}}%
\pgfpathlineto{\pgfqpoint{4.530594in}{3.032104in}}%
\pgfpathlineto{\pgfqpoint{4.546799in}{3.032914in}}%
\pgfpathlineto{\pgfqpoint{4.550210in}{3.028779in}}%
\pgfpathlineto{\pgfqpoint{4.552769in}{3.025579in}}%
\pgfpathlineto{\pgfqpoint{4.553622in}{3.025199in}}%
\pgfpathlineto{\pgfqpoint{4.559592in}{3.028282in}}%
\pgfpathlineto{\pgfqpoint{4.574091in}{3.030520in}}%
\pgfpathlineto{\pgfqpoint{4.579208in}{3.030808in}}%
\pgfpathlineto{\pgfqpoint{4.584325in}{3.025626in}}%
\pgfpathlineto{\pgfqpoint{4.587737in}{3.022122in}}%
\pgfpathlineto{\pgfqpoint{4.588590in}{3.030519in}}%
\pgfpathlineto{\pgfqpoint{4.590295in}{3.039437in}}%
\pgfpathlineto{\pgfqpoint{4.592001in}{3.039435in}}%
\pgfpathlineto{\pgfqpoint{4.592854in}{3.039050in}}%
\pgfpathlineto{\pgfqpoint{4.600530in}{3.032867in}}%
\pgfpathlineto{\pgfqpoint{4.603088in}{3.032739in}}%
\pgfpathlineto{\pgfqpoint{4.612470in}{3.032925in}}%
\pgfpathlineto{\pgfqpoint{4.632939in}{3.039700in}}%
\pgfpathlineto{\pgfqpoint{4.634645in}{3.039189in}}%
\pgfpathlineto{\pgfqpoint{4.649144in}{3.033521in}}%
\pgfpathlineto{\pgfqpoint{4.657673in}{3.036099in}}%
\pgfpathlineto{\pgfqpoint{4.674730in}{3.041199in}}%
\pgfpathlineto{\pgfqpoint{4.676436in}{3.041390in}}%
\pgfpathlineto{\pgfqpoint{4.678142in}{3.043419in}}%
\pgfpathlineto{\pgfqpoint{4.679848in}{3.043667in}}%
\pgfpathlineto{\pgfqpoint{4.688376in}{3.043417in}}%
\pgfpathlineto{\pgfqpoint{4.690082in}{3.043004in}}%
\pgfpathlineto{\pgfqpoint{4.694347in}{3.043531in}}%
\pgfpathlineto{\pgfqpoint{4.695199in}{3.043081in}}%
\pgfpathlineto{\pgfqpoint{4.696052in}{3.057412in}}%
\pgfpathlineto{\pgfqpoint{4.709698in}{3.057730in}}%
\pgfpathlineto{\pgfqpoint{4.710551in}{3.057751in}}%
\pgfpathlineto{\pgfqpoint{4.711404in}{3.056950in}}%
\pgfpathlineto{\pgfqpoint{4.712257in}{3.058120in}}%
\pgfpathlineto{\pgfqpoint{4.713963in}{3.058120in}}%
\pgfpathlineto{\pgfqpoint{4.714816in}{3.054908in}}%
\pgfpathlineto{\pgfqpoint{4.715668in}{3.045224in}}%
\pgfpathlineto{\pgfqpoint{4.716521in}{3.048734in}}%
\pgfpathlineto{\pgfqpoint{4.718227in}{3.057823in}}%
\pgfpathlineto{\pgfqpoint{4.719080in}{3.058120in}}%
\pgfpathlineto{\pgfqpoint{4.737843in}{3.058351in}}%
\pgfpathlineto{\pgfqpoint{4.800956in}{3.063055in}}%
\pgfpathlineto{\pgfqpoint{4.828248in}{3.063326in}}%
\pgfpathlineto{\pgfqpoint{4.831660in}{3.063397in}}%
\pgfpathlineto{\pgfqpoint{4.832513in}{3.062219in}}%
\pgfpathlineto{\pgfqpoint{4.833366in}{3.059355in}}%
\pgfpathlineto{\pgfqpoint{4.834219in}{3.058808in}}%
\pgfpathlineto{\pgfqpoint{4.847012in}{3.059238in}}%
\pgfpathlineto{\pgfqpoint{4.866628in}{3.062908in}}%
\pgfpathlineto{\pgfqpoint{4.871745in}{3.064674in}}%
\pgfpathlineto{\pgfqpoint{4.872598in}{3.059746in}}%
\pgfpathlineto{\pgfqpoint{4.873451in}{3.059862in}}%
\pgfpathlineto{\pgfqpoint{4.875157in}{3.065653in}}%
\pgfpathlineto{\pgfqpoint{4.876010in}{3.068973in}}%
\pgfpathlineto{\pgfqpoint{4.876863in}{3.069627in}}%
\pgfpathlineto{\pgfqpoint{4.879421in}{3.070279in}}%
\pgfpathlineto{\pgfqpoint{4.880274in}{3.070917in}}%
\pgfpathlineto{\pgfqpoint{4.881980in}{3.071163in}}%
\pgfpathlineto{\pgfqpoint{4.886244in}{3.071126in}}%
\pgfpathlineto{\pgfqpoint{4.888803in}{3.069499in}}%
\pgfpathlineto{\pgfqpoint{4.895626in}{3.065049in}}%
\pgfpathlineto{\pgfqpoint{4.898184in}{3.067877in}}%
\pgfpathlineto{\pgfqpoint{4.900743in}{3.070994in}}%
\pgfpathlineto{\pgfqpoint{4.901596in}{3.073219in}}%
\pgfpathlineto{\pgfqpoint{4.918654in}{3.071648in}}%
\pgfpathlineto{\pgfqpoint{4.919506in}{3.073147in}}%
\pgfpathlineto{\pgfqpoint{4.956180in}{3.065357in}}%
\pgfpathlineto{\pgfqpoint{4.957886in}{3.065107in}}%
\pgfpathlineto{\pgfqpoint{4.958739in}{3.064912in}}%
\pgfpathlineto{\pgfqpoint{4.964709in}{3.067464in}}%
\pgfpathlineto{\pgfqpoint{4.982619in}{3.074689in}}%
\pgfpathlineto{\pgfqpoint{5.000530in}{3.074834in}}%
\pgfpathlineto{\pgfqpoint{5.063643in}{3.091962in}}%
\pgfpathlineto{\pgfqpoint{5.067054in}{3.091852in}}%
\pgfpathlineto{\pgfqpoint{5.072172in}{3.087913in}}%
\pgfpathlineto{\pgfqpoint{5.102022in}{3.064585in}}%
\pgfpathlineto{\pgfqpoint{5.102875in}{3.064293in}}%
\pgfpathlineto{\pgfqpoint{5.104581in}{3.066470in}}%
\pgfpathlineto{\pgfqpoint{5.121639in}{3.091390in}}%
\pgfpathlineto{\pgfqpoint{5.122492in}{3.091930in}}%
\pgfpathlineto{\pgfqpoint{5.124197in}{3.090077in}}%
\pgfpathlineto{\pgfqpoint{5.143813in}{3.065186in}}%
\pgfpathlineto{\pgfqpoint{5.144666in}{3.072281in}}%
\pgfpathlineto{\pgfqpoint{5.145519in}{3.090904in}}%
\pgfpathlineto{\pgfqpoint{5.160018in}{3.090904in}}%
\pgfpathlineto{\pgfqpoint{5.160871in}{3.090620in}}%
\pgfpathlineto{\pgfqpoint{5.161724in}{3.089556in}}%
\pgfpathlineto{\pgfqpoint{5.162577in}{3.089617in}}%
\pgfpathlineto{\pgfqpoint{5.163430in}{3.090147in}}%
\pgfpathlineto{\pgfqpoint{5.164283in}{3.089504in}}%
\pgfpathlineto{\pgfqpoint{5.246159in}{3.090894in}}%
\pgfpathlineto{\pgfqpoint{5.247012in}{3.089830in}}%
\pgfpathlineto{\pgfqpoint{5.247865in}{3.089513in}}%
\pgfpathlineto{\pgfqpoint{5.262364in}{3.089709in}}%
\pgfpathlineto{\pgfqpoint{5.263216in}{3.088745in}}%
\pgfpathlineto{\pgfqpoint{5.264922in}{3.089395in}}%
\pgfpathlineto{\pgfqpoint{5.267481in}{3.089847in}}%
\pgfpathlineto{\pgfqpoint{5.271745in}{3.089533in}}%
\pgfpathlineto{\pgfqpoint{5.291361in}{3.088084in}}%
\pgfpathlineto{\pgfqpoint{5.306713in}{3.088302in}}%
\pgfpathlineto{\pgfqpoint{5.329741in}{3.091971in}}%
\pgfpathlineto{\pgfqpoint{5.388590in}{3.092314in}}%
\pgfpathlineto{\pgfqpoint{5.389443in}{3.091799in}}%
\pgfpathlineto{\pgfqpoint{5.391148in}{3.091666in}}%
\pgfpathlineto{\pgfqpoint{5.426969in}{3.093002in}}%
\pgfpathlineto{\pgfqpoint{5.427822in}{3.092405in}}%
\pgfpathlineto{\pgfqpoint{5.430381in}{3.088394in}}%
\pgfpathlineto{\pgfqpoint{5.431234in}{3.086412in}}%
\pgfpathlineto{\pgfqpoint{5.432086in}{3.083011in}}%
\pgfpathlineto{\pgfqpoint{5.432939in}{3.085633in}}%
\pgfpathlineto{\pgfqpoint{5.434645in}{3.096428in}}%
\pgfpathlineto{\pgfqpoint{5.435498in}{3.097609in}}%
\pgfpathlineto{\pgfqpoint{5.445733in}{3.092138in}}%
\pgfpathlineto{\pgfqpoint{5.446585in}{3.091131in}}%
\pgfpathlineto{\pgfqpoint{5.447438in}{3.088124in}}%
\pgfpathlineto{\pgfqpoint{5.449144in}{3.090550in}}%
\pgfpathlineto{\pgfqpoint{5.449997in}{3.090810in}}%
\pgfpathlineto{\pgfqpoint{5.450850in}{3.090663in}}%
\pgfpathlineto{\pgfqpoint{5.451703in}{3.091402in}}%
\pgfpathlineto{\pgfqpoint{5.453408in}{3.090582in}}%
\pgfpathlineto{\pgfqpoint{5.460231in}{3.089880in}}%
\pgfpathlineto{\pgfqpoint{5.467907in}{3.089173in}}%
\pgfpathlineto{\pgfqpoint{5.469613in}{3.090754in}}%
\pgfpathlineto{\pgfqpoint{5.473025in}{3.089143in}}%
\pgfpathlineto{\pgfqpoint{5.473877in}{3.086709in}}%
\pgfpathlineto{\pgfqpoint{5.474730in}{3.083354in}}%
\pgfpathlineto{\pgfqpoint{5.475583in}{3.093813in}}%
\pgfpathlineto{\pgfqpoint{5.476436in}{3.098307in}}%
\pgfpathlineto{\pgfqpoint{5.489229in}{3.098623in}}%
\pgfpathlineto{\pgfqpoint{5.490082in}{3.098646in}}%
\pgfpathlineto{\pgfqpoint{5.490935in}{3.092526in}}%
\pgfpathlineto{\pgfqpoint{5.491788in}{3.090199in}}%
\pgfpathlineto{\pgfqpoint{5.641895in}{3.090451in}}%
\pgfpathlineto{\pgfqpoint{5.690509in}{3.092995in}}%
\pgfpathlineto{\pgfqpoint{5.691362in}{3.093349in}}%
\pgfpathlineto{\pgfqpoint{5.697332in}{3.098360in}}%
\pgfpathlineto{\pgfqpoint{5.698185in}{3.096836in}}%
\pgfpathlineto{\pgfqpoint{5.699037in}{3.094712in}}%
\pgfpathlineto{\pgfqpoint{5.699890in}{3.097224in}}%
\pgfpathlineto{\pgfqpoint{5.715242in}{3.096602in}}%
\pgfpathlineto{\pgfqpoint{5.718654in}{3.097120in}}%
\pgfpathlineto{\pgfqpoint{5.729741in}{3.093172in}}%
\pgfpathlineto{\pgfqpoint{5.731447in}{3.095171in}}%
\pgfpathlineto{\pgfqpoint{5.733153in}{3.097310in}}%
\pgfpathlineto{\pgfqpoint{5.734005in}{3.097007in}}%
\pgfpathlineto{\pgfqpoint{5.737417in}{3.094123in}}%
\pgfpathlineto{\pgfqpoint{5.738270in}{3.095393in}}%
\pgfpathlineto{\pgfqpoint{5.739976in}{3.108507in}}%
\pgfpathlineto{\pgfqpoint{5.740828in}{3.111351in}}%
\pgfpathlineto{\pgfqpoint{5.753622in}{3.111301in}}%
\pgfpathlineto{\pgfqpoint{5.756180in}{3.110389in}}%
\pgfpathlineto{\pgfqpoint{5.757886in}{3.110717in}}%
\pgfpathlineto{\pgfqpoint{5.824411in}{3.129523in}}%
\pgfpathlineto{\pgfqpoint{5.825263in}{3.129018in}}%
\pgfpathlineto{\pgfqpoint{5.836351in}{3.104805in}}%
\pgfpathlineto{\pgfqpoint{5.837204in}{3.104505in}}%
\pgfpathlineto{\pgfqpoint{5.843174in}{3.111358in}}%
\pgfpathlineto{\pgfqpoint{5.858526in}{3.112379in}}%
\pgfpathlineto{\pgfqpoint{5.860231in}{3.113727in}}%
\pgfpathlineto{\pgfqpoint{5.861937in}{3.123213in}}%
\pgfpathlineto{\pgfqpoint{5.862790in}{3.124287in}}%
\pgfpathlineto{\pgfqpoint{5.880701in}{3.111574in}}%
\pgfpathlineto{\pgfqpoint{5.881553in}{3.113353in}}%
\pgfpathlineto{\pgfqpoint{5.884112in}{3.124434in}}%
\pgfpathlineto{\pgfqpoint{5.884965in}{3.124688in}}%
\pgfpathlineto{\pgfqpoint{5.965136in}{3.119983in}}%
\pgfpathlineto{\pgfqpoint{5.978782in}{3.121626in}}%
\pgfpathlineto{\pgfqpoint{5.982193in}{3.124560in}}%
\pgfpathlineto{\pgfqpoint{5.983899in}{3.126082in}}%
\pgfpathlineto{\pgfqpoint{5.984752in}{3.125875in}}%
\pgfpathlineto{\pgfqpoint{5.985605in}{3.121407in}}%
\pgfpathlineto{\pgfqpoint{5.986458in}{3.121862in}}%
\pgfpathlineto{\pgfqpoint{5.990722in}{3.125752in}}%
\pgfpathlineto{\pgfqpoint{5.997545in}{3.125241in}}%
\pgfpathlineto{\pgfqpoint{5.999251in}{3.124869in}}%
\pgfpathlineto{\pgfqpoint{6.000956in}{3.120998in}}%
\pgfpathlineto{\pgfqpoint{6.002662in}{3.121221in}}%
\pgfpathlineto{\pgfqpoint{6.003515in}{3.121229in}}%
\pgfpathlineto{\pgfqpoint{6.004368in}{3.121718in}}%
\pgfpathlineto{\pgfqpoint{6.004368in}{3.121718in}}%
\pgfusepath{stroke}%
\end{pgfscope}%
\begin{pgfscope}%
\pgfpathrectangle{\pgfqpoint{0.481681in}{1.080890in}}{\pgfqpoint{5.785672in}{2.146863in}}%
\pgfusepath{clip}%
\pgfsetrectcap%
\pgfsetroundjoin%
\pgfsetlinewidth{0.200750pt}%
\definecolor{currentstroke}{rgb}{0.580392,0.823529,0.741176}%
\pgfsetstrokecolor{currentstroke}%
\pgfsetdash{}{0pt}%
\pgfpathmoveto{\pgfqpoint{0.744666in}{1.178475in}}%
\pgfpathlineto{\pgfqpoint{0.800103in}{1.178784in}}%
\pgfpathlineto{\pgfqpoint{0.801809in}{1.178796in}}%
\pgfpathlineto{\pgfqpoint{0.802662in}{1.179340in}}%
\pgfpathlineto{\pgfqpoint{0.803515in}{1.180506in}}%
\pgfpathlineto{\pgfqpoint{0.805220in}{1.180775in}}%
\pgfpathlineto{\pgfqpoint{0.821425in}{1.181064in}}%
\pgfpathlineto{\pgfqpoint{0.823984in}{1.179098in}}%
\pgfpathlineto{\pgfqpoint{0.825689in}{1.181122in}}%
\pgfpathlineto{\pgfqpoint{0.830807in}{1.180972in}}%
\pgfpathlineto{\pgfqpoint{0.860657in}{1.179095in}}%
\pgfpathlineto{\pgfqpoint{0.861510in}{1.179671in}}%
\pgfpathlineto{\pgfqpoint{0.862363in}{1.183574in}}%
\pgfpathlineto{\pgfqpoint{0.881979in}{1.182531in}}%
\pgfpathlineto{\pgfqpoint{0.894773in}{1.182092in}}%
\pgfpathlineto{\pgfqpoint{0.943387in}{1.181710in}}%
\pgfpathlineto{\pgfqpoint{0.944240in}{1.182085in}}%
\pgfpathlineto{\pgfqpoint{0.946798in}{1.181886in}}%
\pgfpathlineto{\pgfqpoint{0.947651in}{1.181814in}}%
\pgfpathlineto{\pgfqpoint{0.950210in}{1.182945in}}%
\pgfpathlineto{\pgfqpoint{0.986031in}{1.182923in}}%
\pgfpathlineto{\pgfqpoint{0.989442in}{1.181982in}}%
\pgfpathlineto{\pgfqpoint{0.992001in}{1.181753in}}%
\pgfpathlineto{\pgfqpoint{1.003941in}{1.181731in}}%
\pgfpathlineto{\pgfqpoint{1.004794in}{1.183530in}}%
\pgfpathlineto{\pgfqpoint{1.005647in}{1.189790in}}%
\pgfpathlineto{\pgfqpoint{1.006500in}{1.189819in}}%
\pgfpathlineto{\pgfqpoint{1.007353in}{1.190216in}}%
\pgfpathlineto{\pgfqpoint{1.008205in}{1.192190in}}%
\pgfpathlineto{\pgfqpoint{1.011617in}{1.192358in}}%
\pgfpathlineto{\pgfqpoint{1.023557in}{1.192390in}}%
\pgfpathlineto{\pgfqpoint{1.024410in}{1.191440in}}%
\pgfpathlineto{\pgfqpoint{1.025263in}{1.187311in}}%
\pgfpathlineto{\pgfqpoint{1.026116in}{1.186758in}}%
\pgfpathlineto{\pgfqpoint{1.026969in}{1.192764in}}%
\pgfpathlineto{\pgfqpoint{1.032086in}{1.192394in}}%
\pgfpathlineto{\pgfqpoint{1.085817in}{1.192859in}}%
\pgfpathlineto{\pgfqpoint{1.087523in}{1.193163in}}%
\pgfpathlineto{\pgfqpoint{1.090082in}{1.192780in}}%
\pgfpathlineto{\pgfqpoint{1.090935in}{1.193012in}}%
\pgfpathlineto{\pgfqpoint{1.091788in}{1.193827in}}%
\pgfpathlineto{\pgfqpoint{1.092640in}{1.193796in}}%
\pgfpathlineto{\pgfqpoint{1.093493in}{1.190758in}}%
\pgfpathlineto{\pgfqpoint{1.094346in}{1.182234in}}%
\pgfpathlineto{\pgfqpoint{1.095199in}{1.191921in}}%
\pgfpathlineto{\pgfqpoint{1.096052in}{1.193504in}}%
\pgfpathlineto{\pgfqpoint{1.105434in}{1.193885in}}%
\pgfpathlineto{\pgfqpoint{1.106286in}{1.192833in}}%
\pgfpathlineto{\pgfqpoint{1.107992in}{1.181766in}}%
\pgfpathlineto{\pgfqpoint{1.108845in}{1.192898in}}%
\pgfpathlineto{\pgfqpoint{1.109698in}{1.194089in}}%
\pgfpathlineto{\pgfqpoint{1.113962in}{1.193502in}}%
\pgfpathlineto{\pgfqpoint{1.129314in}{1.194136in}}%
\pgfpathlineto{\pgfqpoint{1.131020in}{1.195741in}}%
\pgfpathlineto{\pgfqpoint{1.133579in}{1.195858in}}%
\pgfpathlineto{\pgfqpoint{1.160018in}{1.195592in}}%
\pgfpathlineto{\pgfqpoint{1.253834in}{1.193844in}}%
\pgfpathlineto{\pgfqpoint{1.271745in}{1.195647in}}%
\pgfpathlineto{\pgfqpoint{1.272598in}{1.194062in}}%
\pgfpathlineto{\pgfqpoint{1.273451in}{1.193773in}}%
\pgfpathlineto{\pgfqpoint{1.276862in}{1.193913in}}%
\pgfpathlineto{\pgfqpoint{1.288802in}{1.193794in}}%
\pgfpathlineto{\pgfqpoint{1.289655in}{1.195009in}}%
\pgfpathlineto{\pgfqpoint{1.291361in}{1.196143in}}%
\pgfpathlineto{\pgfqpoint{1.298184in}{1.195437in}}%
\pgfpathlineto{\pgfqpoint{1.313536in}{1.195512in}}%
\pgfpathlineto{\pgfqpoint{1.315242in}{1.195784in}}%
\pgfpathlineto{\pgfqpoint{1.319506in}{1.195856in}}%
\pgfpathlineto{\pgfqpoint{1.443173in}{1.196099in}}%
\pgfpathlineto{\pgfqpoint{1.522491in}{1.198938in}}%
\pgfpathlineto{\pgfqpoint{1.530167in}{1.208740in}}%
\pgfpathlineto{\pgfqpoint{1.537843in}{1.218367in}}%
\pgfpathlineto{\pgfqpoint{1.557459in}{1.229795in}}%
\pgfpathlineto{\pgfqpoint{1.596692in}{1.232722in}}%
\pgfpathlineto{\pgfqpoint{1.600103in}{1.232978in}}%
\pgfpathlineto{\pgfqpoint{1.601809in}{1.234182in}}%
\pgfpathlineto{\pgfqpoint{1.605220in}{1.234750in}}%
\pgfpathlineto{\pgfqpoint{1.658952in}{1.235712in}}%
\pgfpathlineto{\pgfqpoint{1.659805in}{1.244158in}}%
\pgfpathlineto{\pgfqpoint{1.681979in}{1.863164in}}%
\pgfpathlineto{\pgfqpoint{1.682832in}{1.721233in}}%
\pgfpathlineto{\pgfqpoint{1.683685in}{1.235034in}}%
\pgfpathlineto{\pgfqpoint{1.685391in}{1.235244in}}%
\pgfpathlineto{\pgfqpoint{1.686244in}{1.684660in}}%
\pgfpathlineto{\pgfqpoint{1.687097in}{1.893160in}}%
\pgfpathlineto{\pgfqpoint{1.693067in}{1.895155in}}%
\pgfpathlineto{\pgfqpoint{1.708419in}{1.897603in}}%
\pgfpathlineto{\pgfqpoint{1.710977in}{1.897864in}}%
\pgfpathlineto{\pgfqpoint{1.711830in}{1.896393in}}%
\pgfpathlineto{\pgfqpoint{1.712683in}{1.853346in}}%
\pgfpathlineto{\pgfqpoint{1.721212in}{1.257226in}}%
\pgfpathlineto{\pgfqpoint{1.722918in}{1.435414in}}%
\pgfpathlineto{\pgfqpoint{1.726329in}{1.833396in}}%
\pgfpathlineto{\pgfqpoint{1.727182in}{1.897933in}}%
\pgfpathlineto{\pgfqpoint{1.728035in}{1.899594in}}%
\pgfpathlineto{\pgfqpoint{1.743387in}{1.900033in}}%
\pgfpathlineto{\pgfqpoint{1.745945in}{1.900587in}}%
\pgfpathlineto{\pgfqpoint{1.746798in}{1.899667in}}%
\pgfpathlineto{\pgfqpoint{1.748504in}{1.892228in}}%
\pgfpathlineto{\pgfqpoint{1.749357in}{1.897499in}}%
\pgfpathlineto{\pgfqpoint{1.750210in}{1.900889in}}%
\pgfpathlineto{\pgfqpoint{1.751916in}{1.900307in}}%
\pgfpathlineto{\pgfqpoint{1.752768in}{1.900874in}}%
\pgfpathlineto{\pgfqpoint{1.763856in}{1.900205in}}%
\pgfpathlineto{\pgfqpoint{1.765562in}{1.902137in}}%
\pgfpathlineto{\pgfqpoint{1.767267in}{1.902456in}}%
\pgfpathlineto{\pgfqpoint{1.768120in}{1.761998in}}%
\pgfpathlineto{\pgfqpoint{1.768973in}{1.344391in}}%
\pgfpathlineto{\pgfqpoint{1.769826in}{1.237421in}}%
\pgfpathlineto{\pgfqpoint{1.787736in}{1.238195in}}%
\pgfpathlineto{\pgfqpoint{1.788589in}{1.254709in}}%
\pgfpathlineto{\pgfqpoint{1.789442in}{1.842239in}}%
\pgfpathlineto{\pgfqpoint{1.790295in}{1.902577in}}%
\pgfpathlineto{\pgfqpoint{1.793707in}{1.902675in}}%
\pgfpathlineto{\pgfqpoint{1.814176in}{1.902591in}}%
\pgfpathlineto{\pgfqpoint{1.815029in}{1.895775in}}%
\pgfpathlineto{\pgfqpoint{1.816734in}{1.720839in}}%
\pgfpathlineto{\pgfqpoint{1.820146in}{1.333111in}}%
\pgfpathlineto{\pgfqpoint{1.820999in}{1.276974in}}%
\pgfpathlineto{\pgfqpoint{1.822704in}{1.581155in}}%
\pgfpathlineto{\pgfqpoint{1.823557in}{1.746007in}}%
\pgfpathlineto{\pgfqpoint{1.824410in}{1.828390in}}%
\pgfpathlineto{\pgfqpoint{1.825263in}{1.954203in}}%
\pgfpathlineto{\pgfqpoint{1.826116in}{1.950085in}}%
\pgfpathlineto{\pgfqpoint{1.826969in}{1.941893in}}%
\pgfpathlineto{\pgfqpoint{1.828675in}{1.941084in}}%
\pgfpathlineto{\pgfqpoint{1.829527in}{1.947595in}}%
\pgfpathlineto{\pgfqpoint{1.830380in}{1.948042in}}%
\pgfpathlineto{\pgfqpoint{1.832086in}{1.938248in}}%
\pgfpathlineto{\pgfqpoint{1.832939in}{1.940316in}}%
\pgfpathlineto{\pgfqpoint{1.833792in}{1.845819in}}%
\pgfpathlineto{\pgfqpoint{1.834645in}{1.956411in}}%
\pgfpathlineto{\pgfqpoint{1.836350in}{1.952827in}}%
\pgfpathlineto{\pgfqpoint{1.843174in}{1.937597in}}%
\pgfpathlineto{\pgfqpoint{1.844026in}{1.947494in}}%
\pgfpathlineto{\pgfqpoint{1.844879in}{1.953976in}}%
\pgfpathlineto{\pgfqpoint{1.845732in}{1.953951in}}%
\pgfpathlineto{\pgfqpoint{1.846585in}{1.940796in}}%
\pgfpathlineto{\pgfqpoint{1.847438in}{1.948044in}}%
\pgfpathlineto{\pgfqpoint{1.848291in}{1.960599in}}%
\pgfpathlineto{\pgfqpoint{1.849144in}{1.955867in}}%
\pgfpathlineto{\pgfqpoint{1.849997in}{1.960677in}}%
\pgfpathlineto{\pgfqpoint{1.850849in}{1.961229in}}%
\pgfpathlineto{\pgfqpoint{1.855114in}{1.961511in}}%
\pgfpathlineto{\pgfqpoint{1.855967in}{1.949301in}}%
\pgfpathlineto{\pgfqpoint{1.856820in}{1.961089in}}%
\pgfpathlineto{\pgfqpoint{1.858525in}{1.958904in}}%
\pgfpathlineto{\pgfqpoint{1.861937in}{1.941409in}}%
\pgfpathlineto{\pgfqpoint{1.862790in}{1.940796in}}%
\pgfpathlineto{\pgfqpoint{1.863643in}{1.954828in}}%
\pgfpathlineto{\pgfqpoint{1.864495in}{1.962198in}}%
\pgfpathlineto{\pgfqpoint{1.865348in}{1.963973in}}%
\pgfpathlineto{\pgfqpoint{1.866201in}{1.955777in}}%
\pgfpathlineto{\pgfqpoint{1.867054in}{1.964974in}}%
\pgfpathlineto{\pgfqpoint{1.867907in}{1.988741in}}%
\pgfpathlineto{\pgfqpoint{1.868760in}{1.986314in}}%
\pgfpathlineto{\pgfqpoint{1.869613in}{1.972108in}}%
\pgfpathlineto{\pgfqpoint{1.870466in}{1.987083in}}%
\pgfpathlineto{\pgfqpoint{1.871318in}{1.993911in}}%
\pgfpathlineto{\pgfqpoint{1.875583in}{1.997169in}}%
\pgfpathlineto{\pgfqpoint{1.877289in}{1.995528in}}%
\pgfpathlineto{\pgfqpoint{1.882406in}{1.989939in}}%
\pgfpathlineto{\pgfqpoint{1.883259in}{1.989568in}}%
\pgfpathlineto{\pgfqpoint{1.884112in}{1.991665in}}%
\pgfpathlineto{\pgfqpoint{1.884965in}{1.992997in}}%
\pgfpathlineto{\pgfqpoint{1.885817in}{1.993766in}}%
\pgfpathlineto{\pgfqpoint{1.886670in}{1.995592in}}%
\pgfpathlineto{\pgfqpoint{1.888376in}{1.990334in}}%
\pgfpathlineto{\pgfqpoint{1.889229in}{1.991725in}}%
\pgfpathlineto{\pgfqpoint{1.891788in}{1.998192in}}%
\pgfpathlineto{\pgfqpoint{1.907139in}{2.000243in}}%
\pgfpathlineto{\pgfqpoint{1.907992in}{2.001012in}}%
\pgfpathlineto{\pgfqpoint{1.908845in}{2.001141in}}%
\pgfpathlineto{\pgfqpoint{1.947225in}{1.989732in}}%
\pgfpathlineto{\pgfqpoint{1.948078in}{1.992512in}}%
\pgfpathlineto{\pgfqpoint{1.948930in}{1.998281in}}%
\pgfpathlineto{\pgfqpoint{1.955753in}{1.997748in}}%
\pgfpathlineto{\pgfqpoint{1.965988in}{1.997745in}}%
\pgfpathlineto{\pgfqpoint{1.967694in}{2.002003in}}%
\pgfpathlineto{\pgfqpoint{1.970252in}{2.003175in}}%
\pgfpathlineto{\pgfqpoint{1.973664in}{2.003213in}}%
\pgfpathlineto{\pgfqpoint{2.013749in}{2.003236in}}%
\pgfpathlineto{\pgfqpoint{2.031660in}{2.002279in}}%
\pgfpathlineto{\pgfqpoint{2.032513in}{2.012700in}}%
\pgfpathlineto{\pgfqpoint{2.033365in}{2.014308in}}%
\pgfpathlineto{\pgfqpoint{2.093067in}{2.014123in}}%
\pgfpathlineto{\pgfqpoint{2.224410in}{2.004606in}}%
\pgfpathlineto{\pgfqpoint{2.244879in}{2.007734in}}%
\pgfpathlineto{\pgfqpoint{2.245732in}{2.005121in}}%
\pgfpathlineto{\pgfqpoint{2.255967in}{1.954556in}}%
\pgfpathlineto{\pgfqpoint{2.256820in}{1.956403in}}%
\pgfpathlineto{\pgfqpoint{2.257672in}{2.047938in}}%
\pgfpathlineto{\pgfqpoint{2.258525in}{2.087731in}}%
\pgfpathlineto{\pgfqpoint{2.265348in}{2.017896in}}%
\pgfpathlineto{\pgfqpoint{2.266201in}{2.017468in}}%
\pgfpathlineto{\pgfqpoint{2.279847in}{2.021931in}}%
\pgfpathlineto{\pgfqpoint{2.280700in}{2.024524in}}%
\pgfpathlineto{\pgfqpoint{2.299464in}{2.141549in}}%
\pgfpathlineto{\pgfqpoint{2.300316in}{2.143514in}}%
\pgfpathlineto{\pgfqpoint{2.301169in}{2.099760in}}%
\pgfpathlineto{\pgfqpoint{2.306287in}{2.107596in}}%
\pgfpathlineto{\pgfqpoint{2.307139in}{2.104063in}}%
\pgfpathlineto{\pgfqpoint{2.320785in}{2.027015in}}%
\pgfpathlineto{\pgfqpoint{2.321638in}{2.027123in}}%
\pgfpathlineto{\pgfqpoint{2.346372in}{2.039215in}}%
\pgfpathlineto{\pgfqpoint{2.347225in}{2.040155in}}%
\pgfpathlineto{\pgfqpoint{2.368547in}{2.090569in}}%
\pgfpathlineto{\pgfqpoint{2.377075in}{2.110633in}}%
\pgfpathlineto{\pgfqpoint{2.377928in}{2.111247in}}%
\pgfpathlineto{\pgfqpoint{2.379634in}{2.111532in}}%
\pgfpathlineto{\pgfqpoint{2.381340in}{2.116298in}}%
\pgfpathlineto{\pgfqpoint{2.394986in}{2.123603in}}%
\pgfpathlineto{\pgfqpoint{2.395839in}{2.131815in}}%
\pgfpathlineto{\pgfqpoint{2.398397in}{2.178954in}}%
\pgfpathlineto{\pgfqpoint{2.399250in}{2.185272in}}%
\pgfpathlineto{\pgfqpoint{2.400103in}{2.136085in}}%
\pgfpathlineto{\pgfqpoint{2.400956in}{2.139573in}}%
\pgfpathlineto{\pgfqpoint{2.401809in}{2.111050in}}%
\pgfpathlineto{\pgfqpoint{2.402662in}{2.119704in}}%
\pgfpathlineto{\pgfqpoint{2.403515in}{2.157314in}}%
\pgfpathlineto{\pgfqpoint{2.404368in}{2.154825in}}%
\pgfpathlineto{\pgfqpoint{2.405220in}{2.091199in}}%
\pgfpathlineto{\pgfqpoint{2.406073in}{2.179910in}}%
\pgfpathlineto{\pgfqpoint{2.408632in}{2.061082in}}%
\pgfpathlineto{\pgfqpoint{2.409485in}{2.052449in}}%
\pgfpathlineto{\pgfqpoint{2.417161in}{2.189996in}}%
\pgfpathlineto{\pgfqpoint{2.418014in}{2.169261in}}%
\pgfpathlineto{\pgfqpoint{2.418867in}{2.060347in}}%
\pgfpathlineto{\pgfqpoint{2.420572in}{2.144100in}}%
\pgfpathlineto{\pgfqpoint{2.421425in}{2.152117in}}%
\pgfpathlineto{\pgfqpoint{2.423131in}{2.152221in}}%
\pgfpathlineto{\pgfqpoint{2.426542in}{2.152072in}}%
\pgfpathlineto{\pgfqpoint{2.437630in}{2.152524in}}%
\pgfpathlineto{\pgfqpoint{2.440188in}{2.157955in}}%
\pgfpathlineto{\pgfqpoint{2.458952in}{2.199738in}}%
\pgfpathlineto{\pgfqpoint{2.459805in}{2.200363in}}%
\pgfpathlineto{\pgfqpoint{2.460658in}{2.236096in}}%
\pgfpathlineto{\pgfqpoint{2.461510in}{2.234865in}}%
\pgfpathlineto{\pgfqpoint{2.470039in}{2.157480in}}%
\pgfpathlineto{\pgfqpoint{2.470892in}{2.156189in}}%
\pgfpathlineto{\pgfqpoint{2.530594in}{2.259807in}}%
\pgfpathlineto{\pgfqpoint{2.531446in}{2.256017in}}%
\pgfpathlineto{\pgfqpoint{2.538270in}{2.170069in}}%
\pgfpathlineto{\pgfqpoint{2.539122in}{2.165733in}}%
\pgfpathlineto{\pgfqpoint{2.539975in}{2.263720in}}%
\pgfpathlineto{\pgfqpoint{2.540828in}{2.254503in}}%
\pgfpathlineto{\pgfqpoint{2.542534in}{2.272131in}}%
\pgfpathlineto{\pgfqpoint{2.546798in}{2.272468in}}%
\pgfpathlineto{\pgfqpoint{2.569826in}{2.274197in}}%
\pgfpathlineto{\pgfqpoint{2.580913in}{2.275596in}}%
\pgfpathlineto{\pgfqpoint{2.581766in}{2.274204in}}%
\pgfpathlineto{\pgfqpoint{2.584325in}{2.275109in}}%
\pgfpathlineto{\pgfqpoint{2.585178in}{2.277822in}}%
\pgfpathlineto{\pgfqpoint{2.586884in}{2.294408in}}%
\pgfpathlineto{\pgfqpoint{2.587736in}{2.299651in}}%
\pgfpathlineto{\pgfqpoint{2.592001in}{2.300711in}}%
\pgfpathlineto{\pgfqpoint{2.593707in}{2.300844in}}%
\pgfpathlineto{\pgfqpoint{2.598824in}{2.294752in}}%
\pgfpathlineto{\pgfqpoint{2.601383in}{2.291669in}}%
\pgfpathlineto{\pgfqpoint{2.602235in}{2.291363in}}%
\pgfpathlineto{\pgfqpoint{2.606500in}{2.301152in}}%
\pgfpathlineto{\pgfqpoint{2.608206in}{2.304600in}}%
\pgfpathlineto{\pgfqpoint{2.613323in}{2.307636in}}%
\pgfpathlineto{\pgfqpoint{2.615029in}{2.307364in}}%
\pgfpathlineto{\pgfqpoint{2.626116in}{2.304423in}}%
\pgfpathlineto{\pgfqpoint{2.632939in}{2.304817in}}%
\pgfpathlineto{\pgfqpoint{2.634645in}{2.305031in}}%
\pgfpathlineto{\pgfqpoint{2.665348in}{2.305128in}}%
\pgfpathlineto{\pgfqpoint{2.666201in}{2.302481in}}%
\pgfpathlineto{\pgfqpoint{2.667054in}{2.297332in}}%
\pgfpathlineto{\pgfqpoint{2.667907in}{2.297379in}}%
\pgfpathlineto{\pgfqpoint{2.684112in}{2.306842in}}%
\pgfpathlineto{\pgfqpoint{2.698611in}{2.308873in}}%
\pgfpathlineto{\pgfqpoint{2.702022in}{2.309092in}}%
\pgfpathlineto{\pgfqpoint{2.702875in}{2.308823in}}%
\pgfpathlineto{\pgfqpoint{2.704581in}{2.302032in}}%
\pgfpathlineto{\pgfqpoint{2.707139in}{2.304762in}}%
\pgfpathlineto{\pgfqpoint{2.707992in}{2.303987in}}%
\pgfpathlineto{\pgfqpoint{2.711404in}{2.304134in}}%
\pgfpathlineto{\pgfqpoint{2.725050in}{2.304582in}}%
\pgfpathlineto{\pgfqpoint{2.754048in}{2.312816in}}%
\pgfpathlineto{\pgfqpoint{2.783899in}{2.319417in}}%
\pgfpathlineto{\pgfqpoint{2.832513in}{2.320509in}}%
\pgfpathlineto{\pgfqpoint{2.834218in}{2.320287in}}%
\pgfpathlineto{\pgfqpoint{2.846159in}{2.312490in}}%
\pgfpathlineto{\pgfqpoint{2.847012in}{2.312938in}}%
\pgfpathlineto{\pgfqpoint{2.847864in}{2.318872in}}%
\pgfpathlineto{\pgfqpoint{2.848717in}{2.320358in}}%
\pgfpathlineto{\pgfqpoint{2.849570in}{2.319432in}}%
\pgfpathlineto{\pgfqpoint{2.851276in}{2.319657in}}%
\pgfpathlineto{\pgfqpoint{2.856393in}{2.319590in}}%
\pgfpathlineto{\pgfqpoint{2.858952in}{2.324481in}}%
\pgfpathlineto{\pgfqpoint{2.870039in}{2.346722in}}%
\pgfpathlineto{\pgfqpoint{2.870892in}{2.327101in}}%
\pgfpathlineto{\pgfqpoint{2.871745in}{2.320650in}}%
\pgfpathlineto{\pgfqpoint{2.900743in}{2.362397in}}%
\pgfpathlineto{\pgfqpoint{2.901596in}{2.336309in}}%
\pgfpathlineto{\pgfqpoint{2.902449in}{2.321394in}}%
\pgfpathlineto{\pgfqpoint{2.915242in}{2.322114in}}%
\pgfpathlineto{\pgfqpoint{2.917800in}{2.323995in}}%
\pgfpathlineto{\pgfqpoint{2.968973in}{2.363908in}}%
\pgfpathlineto{\pgfqpoint{2.969826in}{2.363089in}}%
\pgfpathlineto{\pgfqpoint{2.970679in}{2.340784in}}%
\pgfpathlineto{\pgfqpoint{2.971532in}{2.336462in}}%
\pgfpathlineto{\pgfqpoint{2.972385in}{2.367284in}}%
\pgfpathlineto{\pgfqpoint{2.973238in}{2.373511in}}%
\pgfpathlineto{\pgfqpoint{2.977502in}{2.374559in}}%
\pgfpathlineto{\pgfqpoint{2.990295in}{2.377022in}}%
\pgfpathlineto{\pgfqpoint{2.992001in}{2.332996in}}%
\pgfpathlineto{\pgfqpoint{2.993707in}{2.375716in}}%
\pgfpathlineto{\pgfqpoint{2.994560in}{2.373699in}}%
\pgfpathlineto{\pgfqpoint{2.995412in}{2.385234in}}%
\pgfpathlineto{\pgfqpoint{2.996265in}{2.388760in}}%
\pgfpathlineto{\pgfqpoint{2.997118in}{2.389957in}}%
\pgfpathlineto{\pgfqpoint{2.997971in}{2.390643in}}%
\pgfpathlineto{\pgfqpoint{3.003088in}{2.390147in}}%
\pgfpathlineto{\pgfqpoint{3.004794in}{2.390944in}}%
\pgfpathlineto{\pgfqpoint{3.018440in}{2.399422in}}%
\pgfpathlineto{\pgfqpoint{3.033792in}{2.397526in}}%
\pgfpathlineto{\pgfqpoint{3.060231in}{2.394467in}}%
\pgfpathlineto{\pgfqpoint{3.104581in}{2.395093in}}%
\pgfpathlineto{\pgfqpoint{3.119080in}{2.395336in}}%
\pgfpathlineto{\pgfqpoint{3.134432in}{2.396281in}}%
\pgfpathlineto{\pgfqpoint{3.136137in}{2.399895in}}%
\pgfpathlineto{\pgfqpoint{3.137843in}{2.400172in}}%
\pgfpathlineto{\pgfqpoint{3.207779in}{2.400314in}}%
\pgfpathlineto{\pgfqpoint{3.212896in}{2.397332in}}%
\pgfpathlineto{\pgfqpoint{3.237630in}{2.382641in}}%
\pgfpathlineto{\pgfqpoint{3.238483in}{2.378742in}}%
\pgfpathlineto{\pgfqpoint{3.239336in}{2.369116in}}%
\pgfpathlineto{\pgfqpoint{3.240189in}{2.364319in}}%
\pgfpathlineto{\pgfqpoint{3.241041in}{2.361529in}}%
\pgfpathlineto{\pgfqpoint{3.243600in}{2.373309in}}%
\pgfpathlineto{\pgfqpoint{3.247864in}{2.393817in}}%
\pgfpathlineto{\pgfqpoint{3.248717in}{2.395407in}}%
\pgfpathlineto{\pgfqpoint{3.256393in}{2.363271in}}%
\pgfpathlineto{\pgfqpoint{3.258952in}{2.352022in}}%
\pgfpathlineto{\pgfqpoint{3.264922in}{2.351914in}}%
\pgfpathlineto{\pgfqpoint{3.276009in}{2.351793in}}%
\pgfpathlineto{\pgfqpoint{3.276862in}{2.354216in}}%
\pgfpathlineto{\pgfqpoint{3.277715in}{2.380057in}}%
\pgfpathlineto{\pgfqpoint{3.278568in}{2.354543in}}%
\pgfpathlineto{\pgfqpoint{3.279421in}{2.353763in}}%
\pgfpathlineto{\pgfqpoint{3.296479in}{2.393629in}}%
\pgfpathlineto{\pgfqpoint{3.297331in}{2.394178in}}%
\pgfpathlineto{\pgfqpoint{3.318653in}{2.394146in}}%
\pgfpathlineto{\pgfqpoint{3.320359in}{2.351115in}}%
\pgfpathlineto{\pgfqpoint{3.322918in}{2.350931in}}%
\pgfpathlineto{\pgfqpoint{3.380061in}{2.348982in}}%
\pgfpathlineto{\pgfqpoint{3.380914in}{2.347411in}}%
\pgfpathlineto{\pgfqpoint{3.381766in}{2.346822in}}%
\pgfpathlineto{\pgfqpoint{3.404794in}{2.346917in}}%
\pgfpathlineto{\pgfqpoint{3.412470in}{2.350114in}}%
\pgfpathlineto{\pgfqpoint{3.413323in}{2.350138in}}%
\pgfpathlineto{\pgfqpoint{3.419293in}{2.346932in}}%
\pgfpathlineto{\pgfqpoint{3.420999in}{2.349877in}}%
\pgfpathlineto{\pgfqpoint{3.425263in}{2.358198in}}%
\pgfpathlineto{\pgfqpoint{3.430380in}{2.362892in}}%
\pgfpathlineto{\pgfqpoint{3.445732in}{2.376744in}}%
\pgfpathlineto{\pgfqpoint{3.446585in}{2.376468in}}%
\pgfpathlineto{\pgfqpoint{3.447438in}{2.375092in}}%
\pgfpathlineto{\pgfqpoint{3.449144in}{2.375165in}}%
\pgfpathlineto{\pgfqpoint{3.462790in}{2.377468in}}%
\pgfpathlineto{\pgfqpoint{3.463643in}{2.375271in}}%
\pgfpathlineto{\pgfqpoint{3.466201in}{2.365472in}}%
\pgfpathlineto{\pgfqpoint{3.467054in}{2.363387in}}%
\pgfpathlineto{\pgfqpoint{3.467907in}{2.373639in}}%
\pgfpathlineto{\pgfqpoint{3.468760in}{2.373108in}}%
\pgfpathlineto{\pgfqpoint{3.469613in}{2.367219in}}%
\pgfpathlineto{\pgfqpoint{3.473877in}{2.378154in}}%
\pgfpathlineto{\pgfqpoint{3.474730in}{2.378172in}}%
\pgfpathlineto{\pgfqpoint{3.486670in}{2.368272in}}%
\pgfpathlineto{\pgfqpoint{3.488376in}{2.368217in}}%
\pgfpathlineto{\pgfqpoint{3.490082in}{2.368922in}}%
\pgfpathlineto{\pgfqpoint{3.525903in}{2.389113in}}%
\pgfpathlineto{\pgfqpoint{3.526756in}{2.389215in}}%
\pgfpathlineto{\pgfqpoint{3.527609in}{2.385174in}}%
\pgfpathlineto{\pgfqpoint{3.528462in}{2.384055in}}%
\pgfpathlineto{\pgfqpoint{3.534432in}{2.382698in}}%
\pgfpathlineto{\pgfqpoint{3.536990in}{2.385440in}}%
\pgfpathlineto{\pgfqpoint{3.544666in}{2.394085in}}%
\pgfpathlineto{\pgfqpoint{3.545519in}{2.390253in}}%
\pgfpathlineto{\pgfqpoint{3.546372in}{2.391034in}}%
\pgfpathlineto{\pgfqpoint{3.548078in}{2.433513in}}%
\pgfpathlineto{\pgfqpoint{3.548931in}{2.430691in}}%
\pgfpathlineto{\pgfqpoint{3.549783in}{2.424979in}}%
\pgfpathlineto{\pgfqpoint{3.551489in}{2.419760in}}%
\pgfpathlineto{\pgfqpoint{3.560871in}{2.396193in}}%
\pgfpathlineto{\pgfqpoint{3.561724in}{2.395403in}}%
\pgfpathlineto{\pgfqpoint{3.562577in}{2.398212in}}%
\pgfpathlineto{\pgfqpoint{3.563430in}{2.394973in}}%
\pgfpathlineto{\pgfqpoint{3.565135in}{2.441152in}}%
\pgfpathlineto{\pgfqpoint{3.565988in}{2.442862in}}%
\pgfpathlineto{\pgfqpoint{3.566841in}{2.430540in}}%
\pgfpathlineto{\pgfqpoint{3.567694in}{2.390009in}}%
\pgfpathlineto{\pgfqpoint{3.568547in}{2.392627in}}%
\pgfpathlineto{\pgfqpoint{3.571105in}{2.404569in}}%
\pgfpathlineto{\pgfqpoint{3.582193in}{2.449117in}}%
\pgfpathlineto{\pgfqpoint{3.583046in}{2.449204in}}%
\pgfpathlineto{\pgfqpoint{3.583899in}{2.406653in}}%
\pgfpathlineto{\pgfqpoint{3.584752in}{2.428940in}}%
\pgfpathlineto{\pgfqpoint{3.585604in}{2.414762in}}%
\pgfpathlineto{\pgfqpoint{3.586457in}{2.406762in}}%
\pgfpathlineto{\pgfqpoint{3.587310in}{2.450757in}}%
\pgfpathlineto{\pgfqpoint{3.588163in}{2.448583in}}%
\pgfpathlineto{\pgfqpoint{3.589016in}{2.440210in}}%
\pgfpathlineto{\pgfqpoint{3.589869in}{2.455903in}}%
\pgfpathlineto{\pgfqpoint{3.590722in}{2.438019in}}%
\pgfpathlineto{\pgfqpoint{3.591575in}{2.434104in}}%
\pgfpathlineto{\pgfqpoint{3.604368in}{2.466953in}}%
\pgfpathlineto{\pgfqpoint{3.606073in}{2.468455in}}%
\pgfpathlineto{\pgfqpoint{3.606926in}{2.468076in}}%
\pgfpathlineto{\pgfqpoint{3.607779in}{2.438057in}}%
\pgfpathlineto{\pgfqpoint{3.608632in}{2.445674in}}%
\pgfpathlineto{\pgfqpoint{3.609485in}{2.449740in}}%
\pgfpathlineto{\pgfqpoint{3.610338in}{2.469661in}}%
\pgfpathlineto{\pgfqpoint{3.611191in}{2.440034in}}%
\pgfpathlineto{\pgfqpoint{3.612044in}{2.430283in}}%
\pgfpathlineto{\pgfqpoint{3.630807in}{2.437677in}}%
\pgfpathlineto{\pgfqpoint{3.634218in}{2.441014in}}%
\pgfpathlineto{\pgfqpoint{3.670039in}{2.476983in}}%
\pgfpathlineto{\pgfqpoint{3.673451in}{2.477695in}}%
\pgfpathlineto{\pgfqpoint{3.677715in}{2.478242in}}%
\pgfpathlineto{\pgfqpoint{3.685391in}{2.478984in}}%
\pgfpathlineto{\pgfqpoint{3.686244in}{2.473741in}}%
\pgfpathlineto{\pgfqpoint{3.688803in}{2.445282in}}%
\pgfpathlineto{\pgfqpoint{3.689656in}{2.453050in}}%
\pgfpathlineto{\pgfqpoint{3.691361in}{2.489344in}}%
\pgfpathlineto{\pgfqpoint{3.692214in}{2.527372in}}%
\pgfpathlineto{\pgfqpoint{3.693067in}{2.546003in}}%
\pgfpathlineto{\pgfqpoint{3.693920in}{2.555163in}}%
\pgfpathlineto{\pgfqpoint{3.705007in}{2.555163in}}%
\pgfpathlineto{\pgfqpoint{3.705860in}{2.553312in}}%
\pgfpathlineto{\pgfqpoint{3.706713in}{2.523235in}}%
\pgfpathlineto{\pgfqpoint{3.707566in}{2.513924in}}%
\pgfpathlineto{\pgfqpoint{3.709272in}{2.514378in}}%
\pgfpathlineto{\pgfqpoint{3.710125in}{2.514871in}}%
\pgfpathlineto{\pgfqpoint{3.710978in}{2.514871in}}%
\pgfpathlineto{\pgfqpoint{3.711830in}{2.514066in}}%
\pgfpathlineto{\pgfqpoint{3.712683in}{2.521085in}}%
\pgfpathlineto{\pgfqpoint{3.713536in}{2.533700in}}%
\pgfpathlineto{\pgfqpoint{3.714389in}{2.541252in}}%
\pgfpathlineto{\pgfqpoint{3.716948in}{2.546027in}}%
\pgfpathlineto{\pgfqpoint{3.717801in}{2.545128in}}%
\pgfpathlineto{\pgfqpoint{3.726329in}{2.514242in}}%
\pgfpathlineto{\pgfqpoint{3.728035in}{2.513902in}}%
\pgfpathlineto{\pgfqpoint{3.729741in}{2.513744in}}%
\pgfpathlineto{\pgfqpoint{3.731447in}{2.544998in}}%
\pgfpathlineto{\pgfqpoint{3.732299in}{2.519356in}}%
\pgfpathlineto{\pgfqpoint{3.733152in}{2.515726in}}%
\pgfpathlineto{\pgfqpoint{3.738270in}{2.550933in}}%
\pgfpathlineto{\pgfqpoint{3.739123in}{2.552285in}}%
\pgfpathlineto{\pgfqpoint{3.739975in}{2.549043in}}%
\pgfpathlineto{\pgfqpoint{3.742534in}{2.549325in}}%
\pgfpathlineto{\pgfqpoint{3.748504in}{2.550271in}}%
\pgfpathlineto{\pgfqpoint{3.750210in}{2.553765in}}%
\pgfpathlineto{\pgfqpoint{3.753621in}{2.552757in}}%
\pgfpathlineto{\pgfqpoint{3.754474in}{2.551908in}}%
\pgfpathlineto{\pgfqpoint{3.758739in}{2.535566in}}%
\pgfpathlineto{\pgfqpoint{3.760444in}{2.529597in}}%
\pgfpathlineto{\pgfqpoint{3.761297in}{2.530496in}}%
\pgfpathlineto{\pgfqpoint{3.777502in}{2.557762in}}%
\pgfpathlineto{\pgfqpoint{3.778355in}{2.558508in}}%
\pgfpathlineto{\pgfqpoint{3.797118in}{2.550511in}}%
\pgfpathlineto{\pgfqpoint{3.810764in}{2.544684in}}%
\pgfpathlineto{\pgfqpoint{3.811617in}{2.544650in}}%
\pgfpathlineto{\pgfqpoint{3.812470in}{2.550425in}}%
\pgfpathlineto{\pgfqpoint{3.817587in}{2.552385in}}%
\pgfpathlineto{\pgfqpoint{3.832086in}{2.557956in}}%
\pgfpathlineto{\pgfqpoint{3.833792in}{2.559988in}}%
\pgfpathlineto{\pgfqpoint{3.834645in}{2.559640in}}%
\pgfpathlineto{\pgfqpoint{3.850850in}{2.565838in}}%
\pgfpathlineto{\pgfqpoint{3.851702in}{2.566072in}}%
\pgfpathlineto{\pgfqpoint{3.852555in}{2.566678in}}%
\pgfpathlineto{\pgfqpoint{3.853408in}{2.570734in}}%
\pgfpathlineto{\pgfqpoint{3.854261in}{2.570004in}}%
\pgfpathlineto{\pgfqpoint{3.855967in}{2.561418in}}%
\pgfpathlineto{\pgfqpoint{3.856820in}{2.561501in}}%
\pgfpathlineto{\pgfqpoint{3.868760in}{2.573866in}}%
\pgfpathlineto{\pgfqpoint{3.869613in}{2.572724in}}%
\pgfpathlineto{\pgfqpoint{3.870466in}{2.566562in}}%
\pgfpathlineto{\pgfqpoint{3.871319in}{2.578606in}}%
\pgfpathlineto{\pgfqpoint{3.872172in}{2.577560in}}%
\pgfpathlineto{\pgfqpoint{3.873024in}{2.567443in}}%
\pgfpathlineto{\pgfqpoint{3.874730in}{2.579815in}}%
\pgfpathlineto{\pgfqpoint{4.096479in}{2.582528in}}%
\pgfpathlineto{\pgfqpoint{4.115242in}{2.588239in}}%
\pgfpathlineto{\pgfqpoint{4.116095in}{2.624231in}}%
\pgfpathlineto{\pgfqpoint{4.116948in}{2.636048in}}%
\pgfpathlineto{\pgfqpoint{4.247438in}{2.570155in}}%
\pgfpathlineto{\pgfqpoint{4.261937in}{2.546013in}}%
\pgfpathlineto{\pgfqpoint{4.271319in}{2.530818in}}%
\pgfpathlineto{\pgfqpoint{4.272172in}{2.626789in}}%
\pgfpathlineto{\pgfqpoint{4.273024in}{2.646397in}}%
\pgfpathlineto{\pgfqpoint{4.278995in}{2.652853in}}%
\pgfpathlineto{\pgfqpoint{4.279848in}{2.656210in}}%
\pgfpathlineto{\pgfqpoint{4.281553in}{2.667333in}}%
\pgfpathlineto{\pgfqpoint{4.282406in}{2.666287in}}%
\pgfpathlineto{\pgfqpoint{4.284112in}{2.573145in}}%
\pgfpathlineto{\pgfqpoint{4.284965in}{2.636623in}}%
\pgfpathlineto{\pgfqpoint{4.285818in}{2.677534in}}%
\pgfpathlineto{\pgfqpoint{4.286671in}{2.671856in}}%
\pgfpathlineto{\pgfqpoint{4.287523in}{2.669668in}}%
\pgfpathlineto{\pgfqpoint{4.291788in}{2.665286in}}%
\pgfpathlineto{\pgfqpoint{4.292641in}{2.665242in}}%
\pgfpathlineto{\pgfqpoint{4.302022in}{2.680374in}}%
\pgfpathlineto{\pgfqpoint{4.302875in}{2.680747in}}%
\pgfpathlineto{\pgfqpoint{4.303728in}{2.680803in}}%
\pgfpathlineto{\pgfqpoint{4.304581in}{2.676984in}}%
\pgfpathlineto{\pgfqpoint{4.306287in}{2.729030in}}%
\pgfpathlineto{\pgfqpoint{4.307140in}{2.733704in}}%
\pgfpathlineto{\pgfqpoint{4.318227in}{2.734869in}}%
\pgfpathlineto{\pgfqpoint{4.319080in}{2.732670in}}%
\pgfpathlineto{\pgfqpoint{4.319933in}{2.713876in}}%
\pgfpathlineto{\pgfqpoint{4.320786in}{2.713876in}}%
\pgfpathlineto{\pgfqpoint{4.321639in}{2.719737in}}%
\pgfpathlineto{\pgfqpoint{4.322491in}{2.741079in}}%
\pgfpathlineto{\pgfqpoint{4.323344in}{2.742229in}}%
\pgfpathlineto{\pgfqpoint{4.329314in}{2.746282in}}%
\pgfpathlineto{\pgfqpoint{4.334432in}{2.747519in}}%
\pgfpathlineto{\pgfqpoint{4.335285in}{2.745743in}}%
\pgfpathlineto{\pgfqpoint{4.345519in}{2.715366in}}%
\pgfpathlineto{\pgfqpoint{4.346372in}{2.714503in}}%
\pgfpathlineto{\pgfqpoint{4.348078in}{2.731635in}}%
\pgfpathlineto{\pgfqpoint{4.350636in}{2.758556in}}%
\pgfpathlineto{\pgfqpoint{4.351489in}{2.759847in}}%
\pgfpathlineto{\pgfqpoint{4.388163in}{2.760337in}}%
\pgfpathlineto{\pgfqpoint{4.389016in}{2.759684in}}%
\pgfpathlineto{\pgfqpoint{4.403515in}{2.719203in}}%
\pgfpathlineto{\pgfqpoint{4.405221in}{2.714588in}}%
\pgfpathlineto{\pgfqpoint{4.406074in}{2.714097in}}%
\pgfpathlineto{\pgfqpoint{4.407779in}{2.765522in}}%
\pgfpathlineto{\pgfqpoint{4.408632in}{2.765566in}}%
\pgfpathlineto{\pgfqpoint{4.409485in}{2.765254in}}%
\pgfpathlineto{\pgfqpoint{4.432513in}{2.765318in}}%
\pgfpathlineto{\pgfqpoint{4.452129in}{2.765852in}}%
\pgfpathlineto{\pgfqpoint{4.467481in}{2.765662in}}%
\pgfpathlineto{\pgfqpoint{4.470039in}{2.766897in}}%
\pgfpathlineto{\pgfqpoint{4.529741in}{2.798319in}}%
\pgfpathlineto{\pgfqpoint{4.545093in}{2.800537in}}%
\pgfpathlineto{\pgfqpoint{4.545946in}{2.801303in}}%
\pgfpathlineto{\pgfqpoint{4.546799in}{2.801616in}}%
\pgfpathlineto{\pgfqpoint{4.549357in}{2.791993in}}%
\pgfpathlineto{\pgfqpoint{4.552769in}{2.778560in}}%
\pgfpathlineto{\pgfqpoint{4.553622in}{2.777111in}}%
\pgfpathlineto{\pgfqpoint{4.558739in}{2.782729in}}%
\pgfpathlineto{\pgfqpoint{4.560445in}{2.783066in}}%
\pgfpathlineto{\pgfqpoint{4.571532in}{2.784376in}}%
\pgfpathlineto{\pgfqpoint{4.579208in}{2.791815in}}%
\pgfpathlineto{\pgfqpoint{4.581767in}{2.784798in}}%
\pgfpathlineto{\pgfqpoint{4.587737in}{2.767830in}}%
\pgfpathlineto{\pgfqpoint{4.588590in}{2.793774in}}%
\pgfpathlineto{\pgfqpoint{4.590295in}{2.821324in}}%
\pgfpathlineto{\pgfqpoint{4.592001in}{2.821319in}}%
\pgfpathlineto{\pgfqpoint{4.592854in}{2.819855in}}%
\pgfpathlineto{\pgfqpoint{4.600530in}{2.796377in}}%
\pgfpathlineto{\pgfqpoint{4.601383in}{2.795876in}}%
\pgfpathlineto{\pgfqpoint{4.611617in}{2.795673in}}%
\pgfpathlineto{\pgfqpoint{4.612470in}{2.796541in}}%
\pgfpathlineto{\pgfqpoint{4.632939in}{2.828912in}}%
\pgfpathlineto{\pgfqpoint{4.633792in}{2.827965in}}%
\pgfpathlineto{\pgfqpoint{4.648291in}{2.797035in}}%
\pgfpathlineto{\pgfqpoint{4.649144in}{2.795907in}}%
\pgfpathlineto{\pgfqpoint{4.651703in}{2.799387in}}%
\pgfpathlineto{\pgfqpoint{4.674730in}{2.832271in}}%
\pgfpathlineto{\pgfqpoint{4.676436in}{2.832303in}}%
\pgfpathlineto{\pgfqpoint{4.678142in}{2.833529in}}%
\pgfpathlineto{\pgfqpoint{4.679848in}{2.833680in}}%
\pgfpathlineto{\pgfqpoint{4.687523in}{2.833648in}}%
\pgfpathlineto{\pgfqpoint{4.690082in}{2.832985in}}%
\pgfpathlineto{\pgfqpoint{4.693494in}{2.834526in}}%
\pgfpathlineto{\pgfqpoint{4.694347in}{2.834400in}}%
\pgfpathlineto{\pgfqpoint{4.695199in}{2.826990in}}%
\pgfpathlineto{\pgfqpoint{4.696052in}{2.871711in}}%
\pgfpathlineto{\pgfqpoint{4.710551in}{2.872202in}}%
\pgfpathlineto{\pgfqpoint{4.711404in}{2.869851in}}%
\pgfpathlineto{\pgfqpoint{4.712257in}{2.872482in}}%
\pgfpathlineto{\pgfqpoint{4.713963in}{2.872482in}}%
\pgfpathlineto{\pgfqpoint{4.714816in}{2.864068in}}%
\pgfpathlineto{\pgfqpoint{4.715668in}{2.838946in}}%
\pgfpathlineto{\pgfqpoint{4.716521in}{2.848345in}}%
\pgfpathlineto{\pgfqpoint{4.718227in}{2.872156in}}%
\pgfpathlineto{\pgfqpoint{4.719080in}{2.872928in}}%
\pgfpathlineto{\pgfqpoint{4.734432in}{2.872851in}}%
\pgfpathlineto{\pgfqpoint{4.736138in}{2.873471in}}%
\pgfpathlineto{\pgfqpoint{4.798398in}{2.884914in}}%
\pgfpathlineto{\pgfqpoint{4.815455in}{2.884150in}}%
\pgfpathlineto{\pgfqpoint{4.831660in}{2.889860in}}%
\pgfpathlineto{\pgfqpoint{4.832513in}{2.886263in}}%
\pgfpathlineto{\pgfqpoint{4.833366in}{2.877078in}}%
\pgfpathlineto{\pgfqpoint{4.834219in}{2.874130in}}%
\pgfpathlineto{\pgfqpoint{4.844453in}{2.874333in}}%
\pgfpathlineto{\pgfqpoint{4.847012in}{2.875574in}}%
\pgfpathlineto{\pgfqpoint{4.860658in}{2.882843in}}%
\pgfpathlineto{\pgfqpoint{4.864922in}{2.883629in}}%
\pgfpathlineto{\pgfqpoint{4.867481in}{2.886747in}}%
\pgfpathlineto{\pgfqpoint{4.871745in}{2.892224in}}%
\pgfpathlineto{\pgfqpoint{4.872598in}{2.872078in}}%
\pgfpathlineto{\pgfqpoint{4.873451in}{2.870988in}}%
\pgfpathlineto{\pgfqpoint{4.875157in}{2.889520in}}%
\pgfpathlineto{\pgfqpoint{4.876010in}{2.896167in}}%
\pgfpathlineto{\pgfqpoint{4.876863in}{2.898523in}}%
\pgfpathlineto{\pgfqpoint{4.879421in}{2.902644in}}%
\pgfpathlineto{\pgfqpoint{4.880274in}{2.906492in}}%
\pgfpathlineto{\pgfqpoint{4.881127in}{2.907141in}}%
\pgfpathlineto{\pgfqpoint{4.882833in}{2.906699in}}%
\pgfpathlineto{\pgfqpoint{4.886244in}{2.907203in}}%
\pgfpathlineto{\pgfqpoint{4.887950in}{2.905198in}}%
\pgfpathlineto{\pgfqpoint{4.894773in}{2.896108in}}%
\pgfpathlineto{\pgfqpoint{4.896479in}{2.891073in}}%
\pgfpathlineto{\pgfqpoint{4.900743in}{2.906937in}}%
\pgfpathlineto{\pgfqpoint{4.901596in}{2.911927in}}%
\pgfpathlineto{\pgfqpoint{4.910125in}{2.909923in}}%
\pgfpathlineto{\pgfqpoint{4.917801in}{2.908093in}}%
\pgfpathlineto{\pgfqpoint{4.918654in}{2.908878in}}%
\pgfpathlineto{\pgfqpoint{4.919506in}{2.912134in}}%
\pgfpathlineto{\pgfqpoint{4.955327in}{2.895779in}}%
\pgfpathlineto{\pgfqpoint{4.956180in}{2.892230in}}%
\pgfpathlineto{\pgfqpoint{4.957033in}{2.891113in}}%
\pgfpathlineto{\pgfqpoint{4.958739in}{2.894459in}}%
\pgfpathlineto{\pgfqpoint{4.959592in}{2.889726in}}%
\pgfpathlineto{\pgfqpoint{4.963856in}{2.894189in}}%
\pgfpathlineto{\pgfqpoint{4.981767in}{2.913308in}}%
\pgfpathlineto{\pgfqpoint{4.983472in}{2.913834in}}%
\pgfpathlineto{\pgfqpoint{4.996266in}{2.916094in}}%
\pgfpathlineto{\pgfqpoint{5.000530in}{2.916210in}}%
\pgfpathlineto{\pgfqpoint{5.064496in}{2.929858in}}%
\pgfpathlineto{\pgfqpoint{5.067054in}{2.929720in}}%
\pgfpathlineto{\pgfqpoint{5.071319in}{2.925604in}}%
\pgfpathlineto{\pgfqpoint{5.102022in}{2.895426in}}%
\pgfpathlineto{\pgfqpoint{5.102875in}{2.895235in}}%
\pgfpathlineto{\pgfqpoint{5.105434in}{2.899721in}}%
\pgfpathlineto{\pgfqpoint{5.121639in}{2.929250in}}%
\pgfpathlineto{\pgfqpoint{5.122492in}{2.930056in}}%
\pgfpathlineto{\pgfqpoint{5.123344in}{2.929235in}}%
\pgfpathlineto{\pgfqpoint{5.143813in}{2.897412in}}%
\pgfpathlineto{\pgfqpoint{5.144666in}{2.908570in}}%
\pgfpathlineto{\pgfqpoint{5.145519in}{2.937139in}}%
\pgfpathlineto{\pgfqpoint{5.160018in}{2.937139in}}%
\pgfpathlineto{\pgfqpoint{5.160871in}{2.935935in}}%
\pgfpathlineto{\pgfqpoint{5.161724in}{2.931769in}}%
\pgfpathlineto{\pgfqpoint{5.162577in}{2.932293in}}%
\pgfpathlineto{\pgfqpoint{5.163430in}{2.934290in}}%
\pgfpathlineto{\pgfqpoint{5.164283in}{2.931867in}}%
\pgfpathlineto{\pgfqpoint{5.246159in}{2.937099in}}%
\pgfpathlineto{\pgfqpoint{5.247012in}{2.933091in}}%
\pgfpathlineto{\pgfqpoint{5.247865in}{2.931887in}}%
\pgfpathlineto{\pgfqpoint{5.261511in}{2.932800in}}%
\pgfpathlineto{\pgfqpoint{5.262364in}{2.932351in}}%
\pgfpathlineto{\pgfqpoint{5.263216in}{2.930446in}}%
\pgfpathlineto{\pgfqpoint{5.264069in}{2.933952in}}%
\pgfpathlineto{\pgfqpoint{5.266628in}{2.931868in}}%
\pgfpathlineto{\pgfqpoint{5.267481in}{2.931433in}}%
\pgfpathlineto{\pgfqpoint{5.270892in}{2.931823in}}%
\pgfpathlineto{\pgfqpoint{5.289656in}{2.934308in}}%
\pgfpathlineto{\pgfqpoint{5.305860in}{2.934459in}}%
\pgfpathlineto{\pgfqpoint{5.328888in}{2.942266in}}%
\pgfpathlineto{\pgfqpoint{5.386884in}{2.944421in}}%
\pgfpathlineto{\pgfqpoint{5.387737in}{2.943563in}}%
\pgfpathlineto{\pgfqpoint{5.388590in}{2.942032in}}%
\pgfpathlineto{\pgfqpoint{5.389443in}{2.939292in}}%
\pgfpathlineto{\pgfqpoint{5.390295in}{2.937821in}}%
\pgfpathlineto{\pgfqpoint{5.426969in}{2.945797in}}%
\pgfpathlineto{\pgfqpoint{5.427822in}{2.944558in}}%
\pgfpathlineto{\pgfqpoint{5.431234in}{2.932232in}}%
\pgfpathlineto{\pgfqpoint{5.432086in}{2.927454in}}%
\pgfpathlineto{\pgfqpoint{5.432939in}{2.936953in}}%
\pgfpathlineto{\pgfqpoint{5.434645in}{2.968853in}}%
\pgfpathlineto{\pgfqpoint{5.435498in}{2.971478in}}%
\pgfpathlineto{\pgfqpoint{5.445733in}{2.940894in}}%
\pgfpathlineto{\pgfqpoint{5.447438in}{2.932498in}}%
\pgfpathlineto{\pgfqpoint{5.448291in}{2.931840in}}%
\pgfpathlineto{\pgfqpoint{5.449144in}{2.935659in}}%
\pgfpathlineto{\pgfqpoint{5.449997in}{2.936899in}}%
\pgfpathlineto{\pgfqpoint{5.450850in}{2.934843in}}%
\pgfpathlineto{\pgfqpoint{5.451703in}{2.937293in}}%
\pgfpathlineto{\pgfqpoint{5.453408in}{2.936677in}}%
\pgfpathlineto{\pgfqpoint{5.458526in}{2.937678in}}%
\pgfpathlineto{\pgfqpoint{5.469613in}{2.939832in}}%
\pgfpathlineto{\pgfqpoint{5.473025in}{2.939389in}}%
\pgfpathlineto{\pgfqpoint{5.473877in}{2.936668in}}%
\pgfpathlineto{\pgfqpoint{5.474730in}{2.932807in}}%
\pgfpathlineto{\pgfqpoint{5.475583in}{2.972534in}}%
\pgfpathlineto{\pgfqpoint{5.476436in}{2.996859in}}%
\pgfpathlineto{\pgfqpoint{5.478142in}{2.996468in}}%
\pgfpathlineto{\pgfqpoint{5.490082in}{2.996568in}}%
\pgfpathlineto{\pgfqpoint{5.490935in}{2.969002in}}%
\pgfpathlineto{\pgfqpoint{5.491788in}{2.958544in}}%
\pgfpathlineto{\pgfqpoint{5.553195in}{2.958816in}}%
\pgfpathlineto{\pgfqpoint{5.557460in}{2.958874in}}%
\pgfpathlineto{\pgfqpoint{5.635924in}{2.958989in}}%
\pgfpathlineto{\pgfqpoint{5.636777in}{2.959488in}}%
\pgfpathlineto{\pgfqpoint{5.637630in}{2.962795in}}%
\pgfpathlineto{\pgfqpoint{5.690509in}{2.970988in}}%
\pgfpathlineto{\pgfqpoint{5.691362in}{2.971745in}}%
\pgfpathlineto{\pgfqpoint{5.694773in}{2.977718in}}%
\pgfpathlineto{\pgfqpoint{5.695626in}{2.981359in}}%
\pgfpathlineto{\pgfqpoint{5.697332in}{2.992156in}}%
\pgfpathlineto{\pgfqpoint{5.699037in}{2.981212in}}%
\pgfpathlineto{\pgfqpoint{5.699890in}{3.003260in}}%
\pgfpathlineto{\pgfqpoint{5.714389in}{2.975266in}}%
\pgfpathlineto{\pgfqpoint{5.715242in}{2.976575in}}%
\pgfpathlineto{\pgfqpoint{5.717801in}{2.999736in}}%
\pgfpathlineto{\pgfqpoint{5.718654in}{3.003362in}}%
\pgfpathlineto{\pgfqpoint{5.729741in}{2.973788in}}%
\pgfpathlineto{\pgfqpoint{5.730594in}{2.974515in}}%
\pgfpathlineto{\pgfqpoint{5.733153in}{2.978383in}}%
\pgfpathlineto{\pgfqpoint{5.734005in}{2.978051in}}%
\pgfpathlineto{\pgfqpoint{5.737417in}{2.974770in}}%
\pgfpathlineto{\pgfqpoint{5.738270in}{2.977086in}}%
\pgfpathlineto{\pgfqpoint{5.739123in}{3.001864in}}%
\pgfpathlineto{\pgfqpoint{5.739976in}{3.045410in}}%
\pgfpathlineto{\pgfqpoint{5.740828in}{3.062662in}}%
\pgfpathlineto{\pgfqpoint{5.742534in}{3.062662in}}%
\pgfpathlineto{\pgfqpoint{5.744240in}{3.063265in}}%
\pgfpathlineto{\pgfqpoint{5.752769in}{3.063265in}}%
\pgfpathlineto{\pgfqpoint{5.753622in}{3.062911in}}%
\pgfpathlineto{\pgfqpoint{5.756180in}{3.056597in}}%
\pgfpathlineto{\pgfqpoint{5.757033in}{3.056369in}}%
\pgfpathlineto{\pgfqpoint{5.824411in}{3.097611in}}%
\pgfpathlineto{\pgfqpoint{5.825263in}{3.097046in}}%
\pgfpathlineto{\pgfqpoint{5.836351in}{3.063985in}}%
\pgfpathlineto{\pgfqpoint{5.837204in}{3.064923in}}%
\pgfpathlineto{\pgfqpoint{5.843174in}{3.090917in}}%
\pgfpathlineto{\pgfqpoint{5.849144in}{3.092917in}}%
\pgfpathlineto{\pgfqpoint{5.858526in}{3.095990in}}%
\pgfpathlineto{\pgfqpoint{5.859379in}{3.102014in}}%
\pgfpathlineto{\pgfqpoint{5.860231in}{3.104076in}}%
\pgfpathlineto{\pgfqpoint{5.861937in}{3.111651in}}%
\pgfpathlineto{\pgfqpoint{5.862790in}{3.111245in}}%
\pgfpathlineto{\pgfqpoint{5.880701in}{3.064082in}}%
\pgfpathlineto{\pgfqpoint{5.881553in}{3.070909in}}%
\pgfpathlineto{\pgfqpoint{5.884112in}{3.113282in}}%
\pgfpathlineto{\pgfqpoint{5.884965in}{3.114366in}}%
\pgfpathlineto{\pgfqpoint{5.963430in}{3.107399in}}%
\pgfpathlineto{\pgfqpoint{5.965136in}{3.108191in}}%
\pgfpathlineto{\pgfqpoint{5.978782in}{3.115906in}}%
\pgfpathlineto{\pgfqpoint{5.981340in}{3.120927in}}%
\pgfpathlineto{\pgfqpoint{5.983899in}{3.126158in}}%
\pgfpathlineto{\pgfqpoint{5.984752in}{3.126262in}}%
\pgfpathlineto{\pgfqpoint{5.985605in}{3.118874in}}%
\pgfpathlineto{\pgfqpoint{5.986458in}{3.119091in}}%
\pgfpathlineto{\pgfqpoint{5.990722in}{3.122830in}}%
\pgfpathlineto{\pgfqpoint{5.996692in}{3.121808in}}%
\pgfpathlineto{\pgfqpoint{5.999251in}{3.121137in}}%
\pgfpathlineto{\pgfqpoint{6.000104in}{3.118999in}}%
\pgfpathlineto{\pgfqpoint{6.000956in}{3.118426in}}%
\pgfpathlineto{\pgfqpoint{6.001809in}{3.120859in}}%
\pgfpathlineto{\pgfqpoint{6.002662in}{3.120823in}}%
\pgfpathlineto{\pgfqpoint{6.003515in}{3.120458in}}%
\pgfpathlineto{\pgfqpoint{6.004368in}{3.121718in}}%
\pgfpathlineto{\pgfqpoint{6.004368in}{3.121718in}}%
\pgfusepath{stroke}%
\end{pgfscope}%
\begin{pgfscope}%
\pgfpathrectangle{\pgfqpoint{0.481681in}{1.080890in}}{\pgfqpoint{5.785672in}{2.146863in}}%
\pgfusepath{clip}%
\pgfsetrectcap%
\pgfsetroundjoin%
\pgfsetlinewidth{0.200750pt}%
\definecolor{currentstroke}{rgb}{0.933333,0.607843,0.000000}%
\pgfsetstrokecolor{currentstroke}%
\pgfsetdash{}{0pt}%
\pgfpathmoveto{\pgfqpoint{0.744666in}{1.178475in}}%
\pgfpathlineto{\pgfqpoint{0.761724in}{1.179468in}}%
\pgfpathlineto{\pgfqpoint{0.764282in}{1.181573in}}%
\pgfpathlineto{\pgfqpoint{0.801809in}{1.194980in}}%
\pgfpathlineto{\pgfqpoint{0.802662in}{1.210946in}}%
\pgfpathlineto{\pgfqpoint{0.803515in}{1.244952in}}%
\pgfpathlineto{\pgfqpoint{0.804367in}{1.252030in}}%
\pgfpathlineto{\pgfqpoint{0.820572in}{1.257818in}}%
\pgfpathlineto{\pgfqpoint{0.821425in}{1.256091in}}%
\pgfpathlineto{\pgfqpoint{0.823984in}{1.203332in}}%
\pgfpathlineto{\pgfqpoint{0.825689in}{1.257510in}}%
\pgfpathlineto{\pgfqpoint{0.826542in}{1.258053in}}%
\pgfpathlineto{\pgfqpoint{0.827395in}{1.258053in}}%
\pgfpathlineto{\pgfqpoint{0.828248in}{1.257726in}}%
\pgfpathlineto{\pgfqpoint{0.834218in}{1.247753in}}%
\pgfpathlineto{\pgfqpoint{0.860657in}{1.203247in}}%
\pgfpathlineto{\pgfqpoint{0.861510in}{1.216374in}}%
\pgfpathlineto{\pgfqpoint{0.862363in}{1.292243in}}%
\pgfpathlineto{\pgfqpoint{0.864069in}{1.291373in}}%
\pgfpathlineto{\pgfqpoint{0.891361in}{1.274616in}}%
\pgfpathlineto{\pgfqpoint{0.893920in}{1.274530in}}%
\pgfpathlineto{\pgfqpoint{0.942534in}{1.256932in}}%
\pgfpathlineto{\pgfqpoint{0.943387in}{1.261883in}}%
\pgfpathlineto{\pgfqpoint{0.944240in}{1.279011in}}%
\pgfpathlineto{\pgfqpoint{0.945092in}{1.279519in}}%
\pgfpathlineto{\pgfqpoint{0.947651in}{1.262473in}}%
\pgfpathlineto{\pgfqpoint{0.948504in}{1.271994in}}%
\pgfpathlineto{\pgfqpoint{0.949357in}{1.284701in}}%
\pgfpathlineto{\pgfqpoint{0.950210in}{1.286261in}}%
\pgfpathlineto{\pgfqpoint{0.963856in}{1.287927in}}%
\pgfpathlineto{\pgfqpoint{0.964709in}{1.288579in}}%
\pgfpathlineto{\pgfqpoint{0.965561in}{1.289691in}}%
\pgfpathlineto{\pgfqpoint{0.968973in}{1.287616in}}%
\pgfpathlineto{\pgfqpoint{0.983472in}{1.287614in}}%
\pgfpathlineto{\pgfqpoint{0.984325in}{1.288421in}}%
\pgfpathlineto{\pgfqpoint{0.985178in}{1.288582in}}%
\pgfpathlineto{\pgfqpoint{0.986031in}{1.288037in}}%
\pgfpathlineto{\pgfqpoint{0.987736in}{1.283679in}}%
\pgfpathlineto{\pgfqpoint{0.989442in}{1.270469in}}%
\pgfpathlineto{\pgfqpoint{0.990295in}{1.263348in}}%
\pgfpathlineto{\pgfqpoint{0.991148in}{1.260345in}}%
\pgfpathlineto{\pgfqpoint{1.003941in}{1.259004in}}%
\pgfpathlineto{\pgfqpoint{1.004794in}{1.281247in}}%
\pgfpathlineto{\pgfqpoint{1.005647in}{1.360326in}}%
\pgfpathlineto{\pgfqpoint{1.006500in}{1.360707in}}%
\pgfpathlineto{\pgfqpoint{1.007353in}{1.364777in}}%
\pgfpathlineto{\pgfqpoint{1.008205in}{1.387175in}}%
\pgfpathlineto{\pgfqpoint{1.009058in}{1.390127in}}%
\pgfpathlineto{\pgfqpoint{1.013323in}{1.390576in}}%
\pgfpathlineto{\pgfqpoint{1.023557in}{1.391128in}}%
\pgfpathlineto{\pgfqpoint{1.024410in}{1.380509in}}%
\pgfpathlineto{\pgfqpoint{1.025263in}{1.334357in}}%
\pgfpathlineto{\pgfqpoint{1.026116in}{1.327388in}}%
\pgfpathlineto{\pgfqpoint{1.026969in}{1.393168in}}%
\pgfpathlineto{\pgfqpoint{1.030380in}{1.391163in}}%
\pgfpathlineto{\pgfqpoint{1.033792in}{1.391112in}}%
\pgfpathlineto{\pgfqpoint{1.085817in}{1.391813in}}%
\pgfpathlineto{\pgfqpoint{1.086670in}{1.393853in}}%
\pgfpathlineto{\pgfqpoint{1.087523in}{1.394238in}}%
\pgfpathlineto{\pgfqpoint{1.090082in}{1.392946in}}%
\pgfpathlineto{\pgfqpoint{1.090935in}{1.393813in}}%
\pgfpathlineto{\pgfqpoint{1.091788in}{1.396069in}}%
\pgfpathlineto{\pgfqpoint{1.092640in}{1.396033in}}%
\pgfpathlineto{\pgfqpoint{1.093493in}{1.357791in}}%
\pgfpathlineto{\pgfqpoint{1.094346in}{1.242564in}}%
\pgfpathlineto{\pgfqpoint{1.095199in}{1.374125in}}%
\pgfpathlineto{\pgfqpoint{1.096052in}{1.395162in}}%
\pgfpathlineto{\pgfqpoint{1.105434in}{1.395532in}}%
\pgfpathlineto{\pgfqpoint{1.106286in}{1.381323in}}%
\pgfpathlineto{\pgfqpoint{1.107992in}{1.236155in}}%
\pgfpathlineto{\pgfqpoint{1.108845in}{1.382565in}}%
\pgfpathlineto{\pgfqpoint{1.109698in}{1.396368in}}%
\pgfpathlineto{\pgfqpoint{1.113109in}{1.395139in}}%
\pgfpathlineto{\pgfqpoint{1.117374in}{1.395520in}}%
\pgfpathlineto{\pgfqpoint{1.127608in}{1.396574in}}%
\pgfpathlineto{\pgfqpoint{1.128461in}{1.396265in}}%
\pgfpathlineto{\pgfqpoint{1.129314in}{1.397034in}}%
\pgfpathlineto{\pgfqpoint{1.131020in}{1.405894in}}%
\pgfpathlineto{\pgfqpoint{1.131873in}{1.406666in}}%
\pgfpathlineto{\pgfqpoint{1.148930in}{1.407776in}}%
\pgfpathlineto{\pgfqpoint{1.252129in}{1.395641in}}%
\pgfpathlineto{\pgfqpoint{1.253834in}{1.396483in}}%
\pgfpathlineto{\pgfqpoint{1.270892in}{1.407142in}}%
\pgfpathlineto{\pgfqpoint{1.271745in}{1.406757in}}%
\pgfpathlineto{\pgfqpoint{1.272598in}{1.397772in}}%
\pgfpathlineto{\pgfqpoint{1.273451in}{1.395771in}}%
\pgfpathlineto{\pgfqpoint{1.276009in}{1.396055in}}%
\pgfpathlineto{\pgfqpoint{1.287950in}{1.395622in}}%
\pgfpathlineto{\pgfqpoint{1.288802in}{1.395945in}}%
\pgfpathlineto{\pgfqpoint{1.289655in}{1.402879in}}%
\pgfpathlineto{\pgfqpoint{1.291361in}{1.408091in}}%
\pgfpathlineto{\pgfqpoint{1.296478in}{1.405730in}}%
\pgfpathlineto{\pgfqpoint{1.299037in}{1.405635in}}%
\pgfpathlineto{\pgfqpoint{1.310977in}{1.405655in}}%
\pgfpathlineto{\pgfqpoint{1.311830in}{1.406007in}}%
\pgfpathlineto{\pgfqpoint{1.312683in}{1.407212in}}%
\pgfpathlineto{\pgfqpoint{1.313536in}{1.407267in}}%
\pgfpathlineto{\pgfqpoint{1.314389in}{1.407638in}}%
\pgfpathlineto{\pgfqpoint{1.316095in}{1.407688in}}%
\pgfpathlineto{\pgfqpoint{1.316947in}{1.407716in}}%
\pgfpathlineto{\pgfqpoint{1.318653in}{1.406700in}}%
\pgfpathlineto{\pgfqpoint{1.322065in}{1.406766in}}%
\pgfpathlineto{\pgfqpoint{1.415881in}{1.405656in}}%
\pgfpathlineto{\pgfqpoint{1.417587in}{1.405498in}}%
\pgfpathlineto{\pgfqpoint{1.436350in}{1.405732in}}%
\pgfpathlineto{\pgfqpoint{1.521638in}{1.414455in}}%
\pgfpathlineto{\pgfqpoint{1.522491in}{1.416105in}}%
\pgfpathlineto{\pgfqpoint{1.528461in}{1.462141in}}%
\pgfpathlineto{\pgfqpoint{1.537843in}{1.533414in}}%
\pgfpathlineto{\pgfqpoint{1.556606in}{1.589712in}}%
\pgfpathlineto{\pgfqpoint{1.557459in}{1.591310in}}%
\pgfpathlineto{\pgfqpoint{1.582193in}{1.598057in}}%
\pgfpathlineto{\pgfqpoint{1.589016in}{1.599529in}}%
\pgfpathlineto{\pgfqpoint{1.600103in}{1.601828in}}%
\pgfpathlineto{\pgfqpoint{1.601809in}{1.601205in}}%
\pgfpathlineto{\pgfqpoint{1.605220in}{1.602498in}}%
\pgfpathlineto{\pgfqpoint{1.658952in}{1.604165in}}%
\pgfpathlineto{\pgfqpoint{1.659805in}{1.616209in}}%
\pgfpathlineto{\pgfqpoint{1.681979in}{2.498693in}}%
\pgfpathlineto{\pgfqpoint{1.682832in}{2.296607in}}%
\pgfpathlineto{\pgfqpoint{1.683685in}{1.604238in}}%
\pgfpathlineto{\pgfqpoint{1.685391in}{1.604555in}}%
\pgfpathlineto{\pgfqpoint{1.686244in}{2.242258in}}%
\pgfpathlineto{\pgfqpoint{1.687097in}{2.537922in}}%
\pgfpathlineto{\pgfqpoint{1.694773in}{2.539700in}}%
\pgfpathlineto{\pgfqpoint{1.705860in}{2.541353in}}%
\pgfpathlineto{\pgfqpoint{1.710977in}{2.540905in}}%
\pgfpathlineto{\pgfqpoint{1.711830in}{2.538635in}}%
\pgfpathlineto{\pgfqpoint{1.712683in}{2.477774in}}%
\pgfpathlineto{\pgfqpoint{1.721212in}{1.635798in}}%
\pgfpathlineto{\pgfqpoint{1.722918in}{1.887371in}}%
\pgfpathlineto{\pgfqpoint{1.726329in}{2.449269in}}%
\pgfpathlineto{\pgfqpoint{1.727182in}{2.539982in}}%
\pgfpathlineto{\pgfqpoint{1.728035in}{2.541685in}}%
\pgfpathlineto{\pgfqpoint{1.741681in}{2.541174in}}%
\pgfpathlineto{\pgfqpoint{1.744240in}{2.541167in}}%
\pgfpathlineto{\pgfqpoint{1.745945in}{2.540947in}}%
\pgfpathlineto{\pgfqpoint{1.746798in}{2.540141in}}%
\pgfpathlineto{\pgfqpoint{1.748504in}{2.535136in}}%
\pgfpathlineto{\pgfqpoint{1.749357in}{2.539090in}}%
\pgfpathlineto{\pgfqpoint{1.750210in}{2.541642in}}%
\pgfpathlineto{\pgfqpoint{1.751916in}{2.541326in}}%
\pgfpathlineto{\pgfqpoint{1.752768in}{2.541722in}}%
\pgfpathlineto{\pgfqpoint{1.763856in}{2.541303in}}%
\pgfpathlineto{\pgfqpoint{1.765562in}{2.542054in}}%
\pgfpathlineto{\pgfqpoint{1.767267in}{2.542234in}}%
\pgfpathlineto{\pgfqpoint{1.768120in}{2.344794in}}%
\pgfpathlineto{\pgfqpoint{1.768973in}{1.757595in}}%
\pgfpathlineto{\pgfqpoint{1.769826in}{1.606550in}}%
\pgfpathlineto{\pgfqpoint{1.787736in}{1.607413in}}%
\pgfpathlineto{\pgfqpoint{1.788589in}{1.629770in}}%
\pgfpathlineto{\pgfqpoint{1.789442in}{2.457477in}}%
\pgfpathlineto{\pgfqpoint{1.790295in}{2.542419in}}%
\pgfpathlineto{\pgfqpoint{1.792001in}{2.542617in}}%
\pgfpathlineto{\pgfqpoint{1.814176in}{2.542536in}}%
\pgfpathlineto{\pgfqpoint{1.815029in}{2.532914in}}%
\pgfpathlineto{\pgfqpoint{1.816734in}{2.285874in}}%
\pgfpathlineto{\pgfqpoint{1.820146in}{1.738333in}}%
\pgfpathlineto{\pgfqpoint{1.820999in}{1.658684in}}%
\pgfpathlineto{\pgfqpoint{1.822704in}{2.083678in}}%
\pgfpathlineto{\pgfqpoint{1.823557in}{2.314080in}}%
\pgfpathlineto{\pgfqpoint{1.824410in}{2.427252in}}%
\pgfpathlineto{\pgfqpoint{1.825263in}{2.587444in}}%
\pgfpathlineto{\pgfqpoint{1.826116in}{2.588449in}}%
\pgfpathlineto{\pgfqpoint{1.826969in}{2.585452in}}%
\pgfpathlineto{\pgfqpoint{1.827822in}{2.585767in}}%
\pgfpathlineto{\pgfqpoint{1.828675in}{2.585574in}}%
\pgfpathlineto{\pgfqpoint{1.829527in}{2.589549in}}%
\pgfpathlineto{\pgfqpoint{1.830380in}{2.589687in}}%
\pgfpathlineto{\pgfqpoint{1.832086in}{2.583723in}}%
\pgfpathlineto{\pgfqpoint{1.832939in}{2.585822in}}%
\pgfpathlineto{\pgfqpoint{1.833792in}{2.440582in}}%
\pgfpathlineto{\pgfqpoint{1.834645in}{2.587738in}}%
\pgfpathlineto{\pgfqpoint{1.836350in}{2.586835in}}%
\pgfpathlineto{\pgfqpoint{1.843174in}{2.582434in}}%
\pgfpathlineto{\pgfqpoint{1.844879in}{2.590729in}}%
\pgfpathlineto{\pgfqpoint{1.845732in}{2.589811in}}%
\pgfpathlineto{\pgfqpoint{1.846585in}{2.585403in}}%
\pgfpathlineto{\pgfqpoint{1.847438in}{2.588964in}}%
\pgfpathlineto{\pgfqpoint{1.848291in}{2.596535in}}%
\pgfpathlineto{\pgfqpoint{1.849144in}{2.591603in}}%
\pgfpathlineto{\pgfqpoint{1.849997in}{2.596306in}}%
\pgfpathlineto{\pgfqpoint{1.850849in}{2.596700in}}%
\pgfpathlineto{\pgfqpoint{1.854261in}{2.596731in}}%
\pgfpathlineto{\pgfqpoint{1.855114in}{2.597100in}}%
\pgfpathlineto{\pgfqpoint{1.855967in}{2.589913in}}%
\pgfpathlineto{\pgfqpoint{1.856820in}{2.596447in}}%
\pgfpathlineto{\pgfqpoint{1.857672in}{2.592982in}}%
\pgfpathlineto{\pgfqpoint{1.859378in}{2.589950in}}%
\pgfpathlineto{\pgfqpoint{1.861937in}{2.584882in}}%
\pgfpathlineto{\pgfqpoint{1.862790in}{2.584645in}}%
\pgfpathlineto{\pgfqpoint{1.863643in}{2.590063in}}%
\pgfpathlineto{\pgfqpoint{1.864495in}{2.592529in}}%
\pgfpathlineto{\pgfqpoint{1.865348in}{2.592790in}}%
\pgfpathlineto{\pgfqpoint{1.866201in}{2.590435in}}%
\pgfpathlineto{\pgfqpoint{1.867054in}{2.594446in}}%
\pgfpathlineto{\pgfqpoint{1.867907in}{2.605081in}}%
\pgfpathlineto{\pgfqpoint{1.868760in}{2.602628in}}%
\pgfpathlineto{\pgfqpoint{1.869613in}{2.596303in}}%
\pgfpathlineto{\pgfqpoint{1.870466in}{2.604313in}}%
\pgfpathlineto{\pgfqpoint{1.871318in}{2.609263in}}%
\pgfpathlineto{\pgfqpoint{1.875583in}{2.610397in}}%
\pgfpathlineto{\pgfqpoint{1.878142in}{2.608102in}}%
\pgfpathlineto{\pgfqpoint{1.882406in}{2.604037in}}%
\pgfpathlineto{\pgfqpoint{1.883259in}{2.603713in}}%
\pgfpathlineto{\pgfqpoint{1.884112in}{2.605543in}}%
\pgfpathlineto{\pgfqpoint{1.884965in}{2.606707in}}%
\pgfpathlineto{\pgfqpoint{1.885817in}{2.607045in}}%
\pgfpathlineto{\pgfqpoint{1.886670in}{2.608484in}}%
\pgfpathlineto{\pgfqpoint{1.888376in}{2.604301in}}%
\pgfpathlineto{\pgfqpoint{1.889229in}{2.605575in}}%
\pgfpathlineto{\pgfqpoint{1.890935in}{2.609875in}}%
\pgfpathlineto{\pgfqpoint{1.891788in}{2.611104in}}%
\pgfpathlineto{\pgfqpoint{1.907139in}{2.612614in}}%
\pgfpathlineto{\pgfqpoint{1.911404in}{2.611463in}}%
\pgfpathlineto{\pgfqpoint{1.947225in}{2.603745in}}%
\pgfpathlineto{\pgfqpoint{1.948078in}{2.606027in}}%
\pgfpathlineto{\pgfqpoint{1.948930in}{2.610677in}}%
\pgfpathlineto{\pgfqpoint{1.965988in}{2.610802in}}%
\pgfpathlineto{\pgfqpoint{1.966841in}{2.611439in}}%
\pgfpathlineto{\pgfqpoint{1.967694in}{2.612632in}}%
\pgfpathlineto{\pgfqpoint{1.969400in}{2.613422in}}%
\pgfpathlineto{\pgfqpoint{1.972811in}{2.613499in}}%
\pgfpathlineto{\pgfqpoint{2.014602in}{2.613389in}}%
\pgfpathlineto{\pgfqpoint{2.031660in}{2.612493in}}%
\pgfpathlineto{\pgfqpoint{2.032513in}{2.619856in}}%
\pgfpathlineto{\pgfqpoint{2.033365in}{2.620992in}}%
\pgfpathlineto{\pgfqpoint{2.093920in}{2.620708in}}%
\pgfpathlineto{\pgfqpoint{2.226116in}{2.610252in}}%
\pgfpathlineto{\pgfqpoint{2.244879in}{2.611006in}}%
\pgfpathlineto{\pgfqpoint{2.245732in}{2.608572in}}%
\pgfpathlineto{\pgfqpoint{2.255967in}{2.562071in}}%
\pgfpathlineto{\pgfqpoint{2.256820in}{2.563463in}}%
\pgfpathlineto{\pgfqpoint{2.257672in}{2.642804in}}%
\pgfpathlineto{\pgfqpoint{2.258525in}{2.677102in}}%
\pgfpathlineto{\pgfqpoint{2.265348in}{2.613801in}}%
\pgfpathlineto{\pgfqpoint{2.266201in}{2.613259in}}%
\pgfpathlineto{\pgfqpoint{2.279847in}{2.614641in}}%
\pgfpathlineto{\pgfqpoint{2.280700in}{2.617021in}}%
\pgfpathlineto{\pgfqpoint{2.299464in}{2.728784in}}%
\pgfpathlineto{\pgfqpoint{2.300316in}{2.730502in}}%
\pgfpathlineto{\pgfqpoint{2.301169in}{2.686743in}}%
\pgfpathlineto{\pgfqpoint{2.306287in}{2.696903in}}%
\pgfpathlineto{\pgfqpoint{2.307139in}{2.693602in}}%
\pgfpathlineto{\pgfqpoint{2.320785in}{2.619462in}}%
\pgfpathlineto{\pgfqpoint{2.321638in}{2.619317in}}%
\pgfpathlineto{\pgfqpoint{2.346372in}{2.623279in}}%
\pgfpathlineto{\pgfqpoint{2.347225in}{2.624071in}}%
\pgfpathlineto{\pgfqpoint{2.365135in}{2.669012in}}%
\pgfpathlineto{\pgfqpoint{2.377075in}{2.698895in}}%
\pgfpathlineto{\pgfqpoint{2.378781in}{2.700280in}}%
\pgfpathlineto{\pgfqpoint{2.379634in}{2.700392in}}%
\pgfpathlineto{\pgfqpoint{2.380487in}{2.701776in}}%
\pgfpathlineto{\pgfqpoint{2.381340in}{2.703854in}}%
\pgfpathlineto{\pgfqpoint{2.394986in}{2.705609in}}%
\pgfpathlineto{\pgfqpoint{2.395839in}{2.713574in}}%
\pgfpathlineto{\pgfqpoint{2.398397in}{2.760264in}}%
\pgfpathlineto{\pgfqpoint{2.399250in}{2.767106in}}%
\pgfpathlineto{\pgfqpoint{2.400103in}{2.722373in}}%
\pgfpathlineto{\pgfqpoint{2.400956in}{2.725556in}}%
\pgfpathlineto{\pgfqpoint{2.401809in}{2.699525in}}%
\pgfpathlineto{\pgfqpoint{2.402662in}{2.707629in}}%
\pgfpathlineto{\pgfqpoint{2.403515in}{2.742848in}}%
\pgfpathlineto{\pgfqpoint{2.404368in}{2.738539in}}%
\pgfpathlineto{\pgfqpoint{2.405220in}{2.674430in}}%
\pgfpathlineto{\pgfqpoint{2.406073in}{2.761131in}}%
\pgfpathlineto{\pgfqpoint{2.408632in}{2.643322in}}%
\pgfpathlineto{\pgfqpoint{2.409485in}{2.634799in}}%
\pgfpathlineto{\pgfqpoint{2.417161in}{2.771654in}}%
\pgfpathlineto{\pgfqpoint{2.418014in}{2.751513in}}%
\pgfpathlineto{\pgfqpoint{2.418867in}{2.644179in}}%
\pgfpathlineto{\pgfqpoint{2.419719in}{2.683708in}}%
\pgfpathlineto{\pgfqpoint{2.420572in}{2.709283in}}%
\pgfpathlineto{\pgfqpoint{2.421425in}{2.712727in}}%
\pgfpathlineto{\pgfqpoint{2.425690in}{2.712789in}}%
\pgfpathlineto{\pgfqpoint{2.436777in}{2.712789in}}%
\pgfpathlineto{\pgfqpoint{2.437630in}{2.713068in}}%
\pgfpathlineto{\pgfqpoint{2.440188in}{2.720649in}}%
\pgfpathlineto{\pgfqpoint{2.458952in}{2.778965in}}%
\pgfpathlineto{\pgfqpoint{2.459805in}{2.779502in}}%
\pgfpathlineto{\pgfqpoint{2.460658in}{2.788712in}}%
\pgfpathlineto{\pgfqpoint{2.461510in}{2.783860in}}%
\pgfpathlineto{\pgfqpoint{2.470039in}{2.714390in}}%
\pgfpathlineto{\pgfqpoint{2.470892in}{2.713219in}}%
\pgfpathlineto{\pgfqpoint{2.525476in}{2.797376in}}%
\pgfpathlineto{\pgfqpoint{2.528888in}{2.798632in}}%
\pgfpathlineto{\pgfqpoint{2.530594in}{2.799221in}}%
\pgfpathlineto{\pgfqpoint{2.531446in}{2.795594in}}%
\pgfpathlineto{\pgfqpoint{2.538270in}{2.724650in}}%
\pgfpathlineto{\pgfqpoint{2.539122in}{2.720916in}}%
\pgfpathlineto{\pgfqpoint{2.539975in}{2.800210in}}%
\pgfpathlineto{\pgfqpoint{2.540828in}{2.798236in}}%
\pgfpathlineto{\pgfqpoint{2.541681in}{2.800266in}}%
\pgfpathlineto{\pgfqpoint{2.542534in}{2.801466in}}%
\pgfpathlineto{\pgfqpoint{2.560444in}{2.801740in}}%
\pgfpathlineto{\pgfqpoint{2.564709in}{2.801588in}}%
\pgfpathlineto{\pgfqpoint{2.566415in}{2.801397in}}%
\pgfpathlineto{\pgfqpoint{2.571532in}{2.802281in}}%
\pgfpathlineto{\pgfqpoint{2.580061in}{2.803836in}}%
\pgfpathlineto{\pgfqpoint{2.580913in}{2.803656in}}%
\pgfpathlineto{\pgfqpoint{2.581766in}{2.801542in}}%
\pgfpathlineto{\pgfqpoint{2.584325in}{2.802158in}}%
\pgfpathlineto{\pgfqpoint{2.585178in}{2.804435in}}%
\pgfpathlineto{\pgfqpoint{2.586884in}{2.818031in}}%
\pgfpathlineto{\pgfqpoint{2.587736in}{2.822314in}}%
\pgfpathlineto{\pgfqpoint{2.592854in}{2.823138in}}%
\pgfpathlineto{\pgfqpoint{2.593707in}{2.823123in}}%
\pgfpathlineto{\pgfqpoint{2.602235in}{2.818150in}}%
\pgfpathlineto{\pgfqpoint{2.609058in}{2.825190in}}%
\pgfpathlineto{\pgfqpoint{2.613323in}{2.827799in}}%
\pgfpathlineto{\pgfqpoint{2.615029in}{2.827878in}}%
\pgfpathlineto{\pgfqpoint{2.632086in}{2.827309in}}%
\pgfpathlineto{\pgfqpoint{2.665348in}{2.827309in}}%
\pgfpathlineto{\pgfqpoint{2.666201in}{2.821613in}}%
\pgfpathlineto{\pgfqpoint{2.667054in}{2.810509in}}%
\pgfpathlineto{\pgfqpoint{2.667907in}{2.810469in}}%
\pgfpathlineto{\pgfqpoint{2.684112in}{2.827900in}}%
\pgfpathlineto{\pgfqpoint{2.702022in}{2.830483in}}%
\pgfpathlineto{\pgfqpoint{2.702875in}{2.829692in}}%
\pgfpathlineto{\pgfqpoint{2.704581in}{2.814024in}}%
\pgfpathlineto{\pgfqpoint{2.706287in}{2.814964in}}%
\pgfpathlineto{\pgfqpoint{2.707139in}{2.817804in}}%
\pgfpathlineto{\pgfqpoint{2.707992in}{2.815503in}}%
\pgfpathlineto{\pgfqpoint{2.710551in}{2.815664in}}%
\pgfpathlineto{\pgfqpoint{2.733579in}{2.815681in}}%
\pgfpathlineto{\pgfqpoint{2.755754in}{2.815583in}}%
\pgfpathlineto{\pgfqpoint{2.790722in}{2.816825in}}%
\pgfpathlineto{\pgfqpoint{2.806073in}{2.817528in}}%
\pgfpathlineto{\pgfqpoint{2.835071in}{2.817735in}}%
\pgfpathlineto{\pgfqpoint{2.847012in}{2.815677in}}%
\pgfpathlineto{\pgfqpoint{2.847864in}{2.817171in}}%
\pgfpathlineto{\pgfqpoint{2.848717in}{2.817684in}}%
\pgfpathlineto{\pgfqpoint{2.849570in}{2.817309in}}%
\pgfpathlineto{\pgfqpoint{2.850423in}{2.817832in}}%
\pgfpathlineto{\pgfqpoint{2.852982in}{2.818147in}}%
\pgfpathlineto{\pgfqpoint{2.856393in}{2.818548in}}%
\pgfpathlineto{\pgfqpoint{2.858952in}{2.821700in}}%
\pgfpathlineto{\pgfqpoint{2.870039in}{2.835969in}}%
\pgfpathlineto{\pgfqpoint{2.870892in}{2.823377in}}%
\pgfpathlineto{\pgfqpoint{2.871745in}{2.819219in}}%
\pgfpathlineto{\pgfqpoint{2.900743in}{2.845243in}}%
\pgfpathlineto{\pgfqpoint{2.901596in}{2.829545in}}%
\pgfpathlineto{\pgfqpoint{2.902449in}{2.820604in}}%
\pgfpathlineto{\pgfqpoint{2.916095in}{2.822207in}}%
\pgfpathlineto{\pgfqpoint{2.968973in}{2.846312in}}%
\pgfpathlineto{\pgfqpoint{2.969826in}{2.845553in}}%
\pgfpathlineto{\pgfqpoint{2.970679in}{2.832280in}}%
\pgfpathlineto{\pgfqpoint{2.971532in}{2.829739in}}%
\pgfpathlineto{\pgfqpoint{2.972385in}{2.847327in}}%
\pgfpathlineto{\pgfqpoint{2.973238in}{2.850413in}}%
\pgfpathlineto{\pgfqpoint{2.976649in}{2.851290in}}%
\pgfpathlineto{\pgfqpoint{2.990295in}{2.853888in}}%
\pgfpathlineto{\pgfqpoint{2.992001in}{2.827549in}}%
\pgfpathlineto{\pgfqpoint{2.993707in}{2.851209in}}%
\pgfpathlineto{\pgfqpoint{2.994560in}{2.850572in}}%
\pgfpathlineto{\pgfqpoint{2.995412in}{2.856759in}}%
\pgfpathlineto{\pgfqpoint{2.996265in}{2.857229in}}%
\pgfpathlineto{\pgfqpoint{2.997971in}{2.855633in}}%
\pgfpathlineto{\pgfqpoint{3.003941in}{2.856028in}}%
\pgfpathlineto{\pgfqpoint{3.018440in}{2.859923in}}%
\pgfpathlineto{\pgfqpoint{3.049144in}{2.857049in}}%
\pgfpathlineto{\pgfqpoint{3.060231in}{2.856198in}}%
\pgfpathlineto{\pgfqpoint{3.104581in}{2.856946in}}%
\pgfpathlineto{\pgfqpoint{3.116521in}{2.857241in}}%
\pgfpathlineto{\pgfqpoint{3.134432in}{2.859380in}}%
\pgfpathlineto{\pgfqpoint{3.136137in}{2.861199in}}%
\pgfpathlineto{\pgfqpoint{3.138696in}{2.861315in}}%
\pgfpathlineto{\pgfqpoint{3.207779in}{2.860710in}}%
\pgfpathlineto{\pgfqpoint{3.215455in}{2.857823in}}%
\pgfpathlineto{\pgfqpoint{3.237630in}{2.849381in}}%
\pgfpathlineto{\pgfqpoint{3.238483in}{2.845695in}}%
\pgfpathlineto{\pgfqpoint{3.239336in}{2.836409in}}%
\pgfpathlineto{\pgfqpoint{3.240189in}{2.832733in}}%
\pgfpathlineto{\pgfqpoint{3.241041in}{2.832508in}}%
\pgfpathlineto{\pgfqpoint{3.244453in}{2.845034in}}%
\pgfpathlineto{\pgfqpoint{3.247864in}{2.857854in}}%
\pgfpathlineto{\pgfqpoint{3.248717in}{2.859066in}}%
\pgfpathlineto{\pgfqpoint{3.258952in}{2.825040in}}%
\pgfpathlineto{\pgfqpoint{3.276009in}{2.824436in}}%
\pgfpathlineto{\pgfqpoint{3.276862in}{2.826385in}}%
\pgfpathlineto{\pgfqpoint{3.277715in}{2.847386in}}%
\pgfpathlineto{\pgfqpoint{3.278568in}{2.824954in}}%
\pgfpathlineto{\pgfqpoint{3.279421in}{2.824144in}}%
\pgfpathlineto{\pgfqpoint{3.296479in}{2.860377in}}%
\pgfpathlineto{\pgfqpoint{3.297331in}{2.860876in}}%
\pgfpathlineto{\pgfqpoint{3.318653in}{2.860849in}}%
\pgfpathlineto{\pgfqpoint{3.320359in}{2.825275in}}%
\pgfpathlineto{\pgfqpoint{3.323771in}{2.824920in}}%
\pgfpathlineto{\pgfqpoint{3.379208in}{2.820588in}}%
\pgfpathlineto{\pgfqpoint{3.380061in}{2.821306in}}%
\pgfpathlineto{\pgfqpoint{3.380914in}{2.822626in}}%
\pgfpathlineto{\pgfqpoint{3.381766in}{2.822212in}}%
\pgfpathlineto{\pgfqpoint{3.405647in}{2.822473in}}%
\pgfpathlineto{\pgfqpoint{3.413323in}{2.823838in}}%
\pgfpathlineto{\pgfqpoint{3.419293in}{2.822285in}}%
\pgfpathlineto{\pgfqpoint{3.421852in}{2.823394in}}%
\pgfpathlineto{\pgfqpoint{3.426116in}{2.825199in}}%
\pgfpathlineto{\pgfqpoint{3.445732in}{2.829175in}}%
\pgfpathlineto{\pgfqpoint{3.446585in}{2.831498in}}%
\pgfpathlineto{\pgfqpoint{3.447438in}{2.835986in}}%
\pgfpathlineto{\pgfqpoint{3.448291in}{2.837207in}}%
\pgfpathlineto{\pgfqpoint{3.462790in}{2.848257in}}%
\pgfpathlineto{\pgfqpoint{3.463643in}{2.845190in}}%
\pgfpathlineto{\pgfqpoint{3.467054in}{2.826832in}}%
\pgfpathlineto{\pgfqpoint{3.467907in}{2.830149in}}%
\pgfpathlineto{\pgfqpoint{3.468760in}{2.829337in}}%
\pgfpathlineto{\pgfqpoint{3.469613in}{2.827378in}}%
\pgfpathlineto{\pgfqpoint{3.473877in}{2.836844in}}%
\pgfpathlineto{\pgfqpoint{3.474730in}{2.837275in}}%
\pgfpathlineto{\pgfqpoint{3.488376in}{2.836397in}}%
\pgfpathlineto{\pgfqpoint{3.490082in}{2.837411in}}%
\pgfpathlineto{\pgfqpoint{3.525903in}{2.851520in}}%
\pgfpathlineto{\pgfqpoint{3.526756in}{2.850706in}}%
\pgfpathlineto{\pgfqpoint{3.527609in}{2.837525in}}%
\pgfpathlineto{\pgfqpoint{3.528462in}{2.834840in}}%
\pgfpathlineto{\pgfqpoint{3.534432in}{2.837719in}}%
\pgfpathlineto{\pgfqpoint{3.538696in}{2.843461in}}%
\pgfpathlineto{\pgfqpoint{3.544666in}{2.851617in}}%
\pgfpathlineto{\pgfqpoint{3.546372in}{2.836249in}}%
\pgfpathlineto{\pgfqpoint{3.547225in}{2.843510in}}%
\pgfpathlineto{\pgfqpoint{3.548078in}{2.853278in}}%
\pgfpathlineto{\pgfqpoint{3.548931in}{2.854840in}}%
\pgfpathlineto{\pgfqpoint{3.549783in}{2.855299in}}%
\pgfpathlineto{\pgfqpoint{3.551489in}{2.855000in}}%
\pgfpathlineto{\pgfqpoint{3.560871in}{2.852351in}}%
\pgfpathlineto{\pgfqpoint{3.561724in}{2.852610in}}%
\pgfpathlineto{\pgfqpoint{3.562577in}{2.854201in}}%
\pgfpathlineto{\pgfqpoint{3.563430in}{2.844326in}}%
\pgfpathlineto{\pgfqpoint{3.565135in}{2.868428in}}%
\pgfpathlineto{\pgfqpoint{3.565988in}{2.870206in}}%
\pgfpathlineto{\pgfqpoint{3.566841in}{2.863262in}}%
\pgfpathlineto{\pgfqpoint{3.567694in}{2.836351in}}%
\pgfpathlineto{\pgfqpoint{3.569400in}{2.837346in}}%
\pgfpathlineto{\pgfqpoint{3.570253in}{2.838510in}}%
\pgfpathlineto{\pgfqpoint{3.577076in}{2.860956in}}%
\pgfpathlineto{\pgfqpoint{3.582193in}{2.872872in}}%
\pgfpathlineto{\pgfqpoint{3.583046in}{2.872776in}}%
\pgfpathlineto{\pgfqpoint{3.583899in}{2.845891in}}%
\pgfpathlineto{\pgfqpoint{3.584752in}{2.858119in}}%
\pgfpathlineto{\pgfqpoint{3.585604in}{2.846047in}}%
\pgfpathlineto{\pgfqpoint{3.586457in}{2.840048in}}%
\pgfpathlineto{\pgfqpoint{3.587310in}{2.870806in}}%
\pgfpathlineto{\pgfqpoint{3.588163in}{2.870484in}}%
\pgfpathlineto{\pgfqpoint{3.589016in}{2.865434in}}%
\pgfpathlineto{\pgfqpoint{3.589869in}{2.875722in}}%
\pgfpathlineto{\pgfqpoint{3.590722in}{2.863955in}}%
\pgfpathlineto{\pgfqpoint{3.591575in}{2.861406in}}%
\pgfpathlineto{\pgfqpoint{3.604368in}{2.883534in}}%
\pgfpathlineto{\pgfqpoint{3.606073in}{2.884320in}}%
\pgfpathlineto{\pgfqpoint{3.606926in}{2.883523in}}%
\pgfpathlineto{\pgfqpoint{3.607779in}{2.861569in}}%
\pgfpathlineto{\pgfqpoint{3.608632in}{2.869042in}}%
\pgfpathlineto{\pgfqpoint{3.609485in}{2.871891in}}%
\pgfpathlineto{\pgfqpoint{3.610338in}{2.883979in}}%
\pgfpathlineto{\pgfqpoint{3.611191in}{2.865771in}}%
\pgfpathlineto{\pgfqpoint{3.612044in}{2.859658in}}%
\pgfpathlineto{\pgfqpoint{3.630807in}{2.860587in}}%
\pgfpathlineto{\pgfqpoint{3.633366in}{2.862432in}}%
\pgfpathlineto{\pgfqpoint{3.670039in}{2.890421in}}%
\pgfpathlineto{\pgfqpoint{3.674304in}{2.890986in}}%
\pgfpathlineto{\pgfqpoint{3.676862in}{2.890941in}}%
\pgfpathlineto{\pgfqpoint{3.685391in}{2.890417in}}%
\pgfpathlineto{\pgfqpoint{3.686244in}{2.886709in}}%
\pgfpathlineto{\pgfqpoint{3.688803in}{2.866848in}}%
\pgfpathlineto{\pgfqpoint{3.689656in}{2.871720in}}%
\pgfpathlineto{\pgfqpoint{3.691361in}{2.887249in}}%
\pgfpathlineto{\pgfqpoint{3.692214in}{2.904839in}}%
\pgfpathlineto{\pgfqpoint{3.693067in}{2.912246in}}%
\pgfpathlineto{\pgfqpoint{3.693920in}{2.915888in}}%
\pgfpathlineto{\pgfqpoint{3.705007in}{2.915888in}}%
\pgfpathlineto{\pgfqpoint{3.705860in}{2.915193in}}%
\pgfpathlineto{\pgfqpoint{3.706713in}{2.903954in}}%
\pgfpathlineto{\pgfqpoint{3.707566in}{2.901113in}}%
\pgfpathlineto{\pgfqpoint{3.708419in}{2.901784in}}%
\pgfpathlineto{\pgfqpoint{3.710978in}{2.900786in}}%
\pgfpathlineto{\pgfqpoint{3.711830in}{2.901861in}}%
\pgfpathlineto{\pgfqpoint{3.712683in}{2.907015in}}%
\pgfpathlineto{\pgfqpoint{3.713536in}{2.916265in}}%
\pgfpathlineto{\pgfqpoint{3.714389in}{2.921746in}}%
\pgfpathlineto{\pgfqpoint{3.716948in}{2.924870in}}%
\pgfpathlineto{\pgfqpoint{3.717801in}{2.924175in}}%
\pgfpathlineto{\pgfqpoint{3.726329in}{2.901995in}}%
\pgfpathlineto{\pgfqpoint{3.728035in}{2.901850in}}%
\pgfpathlineto{\pgfqpoint{3.729741in}{2.901850in}}%
\pgfpathlineto{\pgfqpoint{3.731447in}{2.913999in}}%
\pgfpathlineto{\pgfqpoint{3.732299in}{2.903914in}}%
\pgfpathlineto{\pgfqpoint{3.733152in}{2.902487in}}%
\pgfpathlineto{\pgfqpoint{3.738270in}{2.916324in}}%
\pgfpathlineto{\pgfqpoint{3.739975in}{2.930352in}}%
\pgfpathlineto{\pgfqpoint{3.741681in}{2.931237in}}%
\pgfpathlineto{\pgfqpoint{3.748504in}{2.933676in}}%
\pgfpathlineto{\pgfqpoint{3.750210in}{2.917589in}}%
\pgfpathlineto{\pgfqpoint{3.754474in}{2.917053in}}%
\pgfpathlineto{\pgfqpoint{3.760444in}{2.912000in}}%
\pgfpathlineto{\pgfqpoint{3.763003in}{2.912734in}}%
\pgfpathlineto{\pgfqpoint{3.779208in}{2.917953in}}%
\pgfpathlineto{\pgfqpoint{3.811617in}{2.924305in}}%
\pgfpathlineto{\pgfqpoint{3.812470in}{2.929720in}}%
\pgfpathlineto{\pgfqpoint{3.814176in}{2.930110in}}%
\pgfpathlineto{\pgfqpoint{3.832086in}{2.931348in}}%
\pgfpathlineto{\pgfqpoint{3.832939in}{2.924344in}}%
\pgfpathlineto{\pgfqpoint{3.833792in}{2.924012in}}%
\pgfpathlineto{\pgfqpoint{3.834645in}{2.931060in}}%
\pgfpathlineto{\pgfqpoint{3.835498in}{2.926947in}}%
\pgfpathlineto{\pgfqpoint{3.836351in}{2.921275in}}%
\pgfpathlineto{\pgfqpoint{3.837204in}{2.919249in}}%
\pgfpathlineto{\pgfqpoint{3.850850in}{2.936458in}}%
\pgfpathlineto{\pgfqpoint{3.851702in}{2.937064in}}%
\pgfpathlineto{\pgfqpoint{3.852555in}{2.935049in}}%
\pgfpathlineto{\pgfqpoint{3.853408in}{2.938663in}}%
\pgfpathlineto{\pgfqpoint{3.854261in}{2.934607in}}%
\pgfpathlineto{\pgfqpoint{3.855967in}{2.919769in}}%
\pgfpathlineto{\pgfqpoint{3.856820in}{2.919807in}}%
\pgfpathlineto{\pgfqpoint{3.868760in}{2.939451in}}%
\pgfpathlineto{\pgfqpoint{3.869613in}{2.937635in}}%
\pgfpathlineto{\pgfqpoint{3.870466in}{2.926708in}}%
\pgfpathlineto{\pgfqpoint{3.871319in}{2.941063in}}%
\pgfpathlineto{\pgfqpoint{3.872172in}{2.939760in}}%
\pgfpathlineto{\pgfqpoint{3.873024in}{2.926724in}}%
\pgfpathlineto{\pgfqpoint{3.874730in}{2.940991in}}%
\pgfpathlineto{\pgfqpoint{4.097332in}{2.942001in}}%
\pgfpathlineto{\pgfqpoint{4.115242in}{2.944222in}}%
\pgfpathlineto{\pgfqpoint{4.116095in}{2.956484in}}%
\pgfpathlineto{\pgfqpoint{4.116948in}{2.960083in}}%
\pgfpathlineto{\pgfqpoint{4.247438in}{2.935548in}}%
\pgfpathlineto{\pgfqpoint{4.262790in}{2.921497in}}%
\pgfpathlineto{\pgfqpoint{4.271319in}{2.913886in}}%
\pgfpathlineto{\pgfqpoint{4.272172in}{2.957980in}}%
\pgfpathlineto{\pgfqpoint{4.273024in}{2.966772in}}%
\pgfpathlineto{\pgfqpoint{4.280700in}{2.968553in}}%
\pgfpathlineto{\pgfqpoint{4.281553in}{2.968895in}}%
\pgfpathlineto{\pgfqpoint{4.282406in}{2.967140in}}%
\pgfpathlineto{\pgfqpoint{4.283259in}{2.948081in}}%
\pgfpathlineto{\pgfqpoint{4.284112in}{2.935344in}}%
\pgfpathlineto{\pgfqpoint{4.284965in}{2.961189in}}%
\pgfpathlineto{\pgfqpoint{4.285818in}{2.974813in}}%
\pgfpathlineto{\pgfqpoint{4.286671in}{2.968661in}}%
\pgfpathlineto{\pgfqpoint{4.287523in}{2.966750in}}%
\pgfpathlineto{\pgfqpoint{4.291788in}{2.965170in}}%
\pgfpathlineto{\pgfqpoint{4.292641in}{2.965496in}}%
\pgfpathlineto{\pgfqpoint{4.302022in}{2.981124in}}%
\pgfpathlineto{\pgfqpoint{4.302875in}{2.981441in}}%
\pgfpathlineto{\pgfqpoint{4.303728in}{2.981408in}}%
\pgfpathlineto{\pgfqpoint{4.304581in}{2.976491in}}%
\pgfpathlineto{\pgfqpoint{4.305434in}{2.983457in}}%
\pgfpathlineto{\pgfqpoint{4.306287in}{2.997641in}}%
\pgfpathlineto{\pgfqpoint{4.307140in}{3.002641in}}%
\pgfpathlineto{\pgfqpoint{4.318227in}{3.003049in}}%
\pgfpathlineto{\pgfqpoint{4.319080in}{3.001367in}}%
\pgfpathlineto{\pgfqpoint{4.319933in}{2.987283in}}%
\pgfpathlineto{\pgfqpoint{4.320786in}{2.987283in}}%
\pgfpathlineto{\pgfqpoint{4.321639in}{2.991156in}}%
\pgfpathlineto{\pgfqpoint{4.322491in}{3.005157in}}%
\pgfpathlineto{\pgfqpoint{4.323344in}{3.005624in}}%
\pgfpathlineto{\pgfqpoint{4.333579in}{3.006406in}}%
\pgfpathlineto{\pgfqpoint{4.334432in}{3.006401in}}%
\pgfpathlineto{\pgfqpoint{4.336138in}{3.003965in}}%
\pgfpathlineto{\pgfqpoint{4.345519in}{2.988353in}}%
\pgfpathlineto{\pgfqpoint{4.346372in}{2.987753in}}%
\pgfpathlineto{\pgfqpoint{4.348078in}{2.995150in}}%
\pgfpathlineto{\pgfqpoint{4.350636in}{3.006786in}}%
\pgfpathlineto{\pgfqpoint{4.351489in}{3.007332in}}%
\pgfpathlineto{\pgfqpoint{4.386457in}{3.007188in}}%
\pgfpathlineto{\pgfqpoint{4.388163in}{3.007420in}}%
\pgfpathlineto{\pgfqpoint{4.389016in}{3.006749in}}%
\pgfpathlineto{\pgfqpoint{4.399251in}{2.994529in}}%
\pgfpathlineto{\pgfqpoint{4.405221in}{2.987487in}}%
\pgfpathlineto{\pgfqpoint{4.406074in}{2.987278in}}%
\pgfpathlineto{\pgfqpoint{4.407779in}{3.007912in}}%
\pgfpathlineto{\pgfqpoint{4.410338in}{3.007765in}}%
\pgfpathlineto{\pgfqpoint{4.434219in}{3.008009in}}%
\pgfpathlineto{\pgfqpoint{4.450423in}{3.008897in}}%
\pgfpathlineto{\pgfqpoint{4.468334in}{3.008730in}}%
\pgfpathlineto{\pgfqpoint{4.530594in}{3.019392in}}%
\pgfpathlineto{\pgfqpoint{4.545093in}{3.019786in}}%
\pgfpathlineto{\pgfqpoint{4.546799in}{3.020629in}}%
\pgfpathlineto{\pgfqpoint{4.549357in}{3.017481in}}%
\pgfpathlineto{\pgfqpoint{4.552769in}{3.013044in}}%
\pgfpathlineto{\pgfqpoint{4.553622in}{3.012689in}}%
\pgfpathlineto{\pgfqpoint{4.558739in}{3.016209in}}%
\pgfpathlineto{\pgfqpoint{4.561297in}{3.016352in}}%
\pgfpathlineto{\pgfqpoint{4.574091in}{3.016746in}}%
\pgfpathlineto{\pgfqpoint{4.579208in}{3.017106in}}%
\pgfpathlineto{\pgfqpoint{4.584325in}{3.012517in}}%
\pgfpathlineto{\pgfqpoint{4.587737in}{3.009414in}}%
\pgfpathlineto{\pgfqpoint{4.588590in}{3.018230in}}%
\pgfpathlineto{\pgfqpoint{4.590295in}{3.027355in}}%
\pgfpathlineto{\pgfqpoint{4.592001in}{3.027353in}}%
\pgfpathlineto{\pgfqpoint{4.592854in}{3.026846in}}%
\pgfpathlineto{\pgfqpoint{4.600530in}{3.018715in}}%
\pgfpathlineto{\pgfqpoint{4.602236in}{3.018542in}}%
\pgfpathlineto{\pgfqpoint{4.612470in}{3.018764in}}%
\pgfpathlineto{\pgfqpoint{4.632939in}{3.027799in}}%
\pgfpathlineto{\pgfqpoint{4.634645in}{3.027126in}}%
\pgfpathlineto{\pgfqpoint{4.649144in}{3.019642in}}%
\pgfpathlineto{\pgfqpoint{4.655967in}{3.022172in}}%
\pgfpathlineto{\pgfqpoint{4.674730in}{3.028993in}}%
\pgfpathlineto{\pgfqpoint{4.676436in}{3.028674in}}%
\pgfpathlineto{\pgfqpoint{4.678995in}{3.029695in}}%
\pgfpathlineto{\pgfqpoint{4.688376in}{3.029441in}}%
\pgfpathlineto{\pgfqpoint{4.690082in}{3.029009in}}%
\pgfpathlineto{\pgfqpoint{4.694347in}{3.029263in}}%
\pgfpathlineto{\pgfqpoint{4.695199in}{3.028327in}}%
\pgfpathlineto{\pgfqpoint{4.696052in}{3.041570in}}%
\pgfpathlineto{\pgfqpoint{4.710551in}{3.041455in}}%
\pgfpathlineto{\pgfqpoint{4.711404in}{3.040690in}}%
\pgfpathlineto{\pgfqpoint{4.712257in}{3.041567in}}%
\pgfpathlineto{\pgfqpoint{4.713963in}{3.041567in}}%
\pgfpathlineto{\pgfqpoint{4.714816in}{3.038832in}}%
\pgfpathlineto{\pgfqpoint{4.715668in}{3.030587in}}%
\pgfpathlineto{\pgfqpoint{4.716521in}{3.033576in}}%
\pgfpathlineto{\pgfqpoint{4.718227in}{3.041313in}}%
\pgfpathlineto{\pgfqpoint{4.719933in}{3.041567in}}%
\pgfpathlineto{\pgfqpoint{4.739549in}{3.041811in}}%
\pgfpathlineto{\pgfqpoint{4.802662in}{3.044865in}}%
\pgfpathlineto{\pgfqpoint{4.820573in}{3.045109in}}%
\pgfpathlineto{\pgfqpoint{4.831660in}{3.045611in}}%
\pgfpathlineto{\pgfqpoint{4.832513in}{3.044738in}}%
\pgfpathlineto{\pgfqpoint{4.833366in}{3.042562in}}%
\pgfpathlineto{\pgfqpoint{4.834219in}{3.041913in}}%
\pgfpathlineto{\pgfqpoint{4.846159in}{3.042061in}}%
\pgfpathlineto{\pgfqpoint{4.871745in}{3.046873in}}%
\pgfpathlineto{\pgfqpoint{4.872598in}{3.042981in}}%
\pgfpathlineto{\pgfqpoint{4.873451in}{3.043100in}}%
\pgfpathlineto{\pgfqpoint{4.874304in}{3.045903in}}%
\pgfpathlineto{\pgfqpoint{4.875157in}{3.047737in}}%
\pgfpathlineto{\pgfqpoint{4.876010in}{3.050694in}}%
\pgfpathlineto{\pgfqpoint{4.876863in}{3.051248in}}%
\pgfpathlineto{\pgfqpoint{4.879421in}{3.051735in}}%
\pgfpathlineto{\pgfqpoint{4.880274in}{3.052753in}}%
\pgfpathlineto{\pgfqpoint{4.881127in}{3.053067in}}%
\pgfpathlineto{\pgfqpoint{4.886244in}{3.052925in}}%
\pgfpathlineto{\pgfqpoint{4.888803in}{3.051254in}}%
\pgfpathlineto{\pgfqpoint{4.895626in}{3.046709in}}%
\pgfpathlineto{\pgfqpoint{4.899037in}{3.050629in}}%
\pgfpathlineto{\pgfqpoint{4.900743in}{3.052929in}}%
\pgfpathlineto{\pgfqpoint{4.901596in}{3.056930in}}%
\pgfpathlineto{\pgfqpoint{4.910978in}{3.054960in}}%
\pgfpathlineto{\pgfqpoint{4.917801in}{3.053507in}}%
\pgfpathlineto{\pgfqpoint{4.918654in}{3.054208in}}%
\pgfpathlineto{\pgfqpoint{4.919506in}{3.057179in}}%
\pgfpathlineto{\pgfqpoint{4.956180in}{3.046875in}}%
\pgfpathlineto{\pgfqpoint{4.957886in}{3.046746in}}%
\pgfpathlineto{\pgfqpoint{4.958739in}{3.046606in}}%
\pgfpathlineto{\pgfqpoint{4.962150in}{3.048327in}}%
\pgfpathlineto{\pgfqpoint{4.981767in}{3.057598in}}%
\pgfpathlineto{\pgfqpoint{4.984325in}{3.057645in}}%
\pgfpathlineto{\pgfqpoint{5.000530in}{3.057458in}}%
\pgfpathlineto{\pgfqpoint{5.063643in}{3.072090in}}%
\pgfpathlineto{\pgfqpoint{5.067054in}{3.071998in}}%
\pgfpathlineto{\pgfqpoint{5.073025in}{3.068124in}}%
\pgfpathlineto{\pgfqpoint{5.102022in}{3.049064in}}%
\pgfpathlineto{\pgfqpoint{5.102875in}{3.048848in}}%
\pgfpathlineto{\pgfqpoint{5.104581in}{3.050699in}}%
\pgfpathlineto{\pgfqpoint{5.121639in}{3.071461in}}%
\pgfpathlineto{\pgfqpoint{5.122492in}{3.071272in}}%
\pgfpathlineto{\pgfqpoint{5.124197in}{3.069199in}}%
\pgfpathlineto{\pgfqpoint{5.143813in}{3.049389in}}%
\pgfpathlineto{\pgfqpoint{5.144666in}{3.055094in}}%
\pgfpathlineto{\pgfqpoint{5.145519in}{3.070051in}}%
\pgfpathlineto{\pgfqpoint{5.160871in}{3.069957in}}%
\pgfpathlineto{\pgfqpoint{5.161724in}{3.069517in}}%
\pgfpathlineto{\pgfqpoint{5.162577in}{3.069470in}}%
\pgfpathlineto{\pgfqpoint{5.163430in}{3.069709in}}%
\pgfpathlineto{\pgfqpoint{5.164283in}{3.069418in}}%
\pgfpathlineto{\pgfqpoint{5.246159in}{3.070047in}}%
\pgfpathlineto{\pgfqpoint{5.247012in}{3.069571in}}%
\pgfpathlineto{\pgfqpoint{5.248718in}{3.069522in}}%
\pgfpathlineto{\pgfqpoint{5.262364in}{3.070377in}}%
\pgfpathlineto{\pgfqpoint{5.263216in}{3.069540in}}%
\pgfpathlineto{\pgfqpoint{5.264069in}{3.069150in}}%
\pgfpathlineto{\pgfqpoint{5.267481in}{3.070485in}}%
\pgfpathlineto{\pgfqpoint{5.273451in}{3.070860in}}%
\pgfpathlineto{\pgfqpoint{5.292214in}{3.071922in}}%
\pgfpathlineto{\pgfqpoint{5.307566in}{3.072186in}}%
\pgfpathlineto{\pgfqpoint{5.330594in}{3.074414in}}%
\pgfpathlineto{\pgfqpoint{5.386884in}{3.074871in}}%
\pgfpathlineto{\pgfqpoint{5.387737in}{3.076975in}}%
\pgfpathlineto{\pgfqpoint{5.388590in}{3.077346in}}%
\pgfpathlineto{\pgfqpoint{5.390295in}{3.079191in}}%
\pgfpathlineto{\pgfqpoint{5.426969in}{3.076946in}}%
\pgfpathlineto{\pgfqpoint{5.427822in}{3.076308in}}%
\pgfpathlineto{\pgfqpoint{5.430381in}{3.072323in}}%
\pgfpathlineto{\pgfqpoint{5.431234in}{3.070066in}}%
\pgfpathlineto{\pgfqpoint{5.432086in}{3.065712in}}%
\pgfpathlineto{\pgfqpoint{5.432939in}{3.068548in}}%
\pgfpathlineto{\pgfqpoint{5.434645in}{3.080882in}}%
\pgfpathlineto{\pgfqpoint{5.435498in}{3.082427in}}%
\pgfpathlineto{\pgfqpoint{5.445733in}{3.079420in}}%
\pgfpathlineto{\pgfqpoint{5.446585in}{3.076725in}}%
\pgfpathlineto{\pgfqpoint{5.447438in}{3.070937in}}%
\pgfpathlineto{\pgfqpoint{5.448291in}{3.070542in}}%
\pgfpathlineto{\pgfqpoint{5.449997in}{3.070982in}}%
\pgfpathlineto{\pgfqpoint{5.450850in}{3.070869in}}%
\pgfpathlineto{\pgfqpoint{5.451703in}{3.077259in}}%
\pgfpathlineto{\pgfqpoint{5.453408in}{3.070540in}}%
\pgfpathlineto{\pgfqpoint{5.454261in}{3.070660in}}%
\pgfpathlineto{\pgfqpoint{5.467907in}{3.079001in}}%
\pgfpathlineto{\pgfqpoint{5.469613in}{3.070996in}}%
\pgfpathlineto{\pgfqpoint{5.473025in}{3.078556in}}%
\pgfpathlineto{\pgfqpoint{5.473877in}{3.074260in}}%
\pgfpathlineto{\pgfqpoint{5.474730in}{3.067493in}}%
\pgfpathlineto{\pgfqpoint{5.476436in}{3.091313in}}%
\pgfpathlineto{\pgfqpoint{5.477289in}{3.091848in}}%
\pgfpathlineto{\pgfqpoint{5.480701in}{3.091970in}}%
\pgfpathlineto{\pgfqpoint{5.490082in}{3.092081in}}%
\pgfpathlineto{\pgfqpoint{5.490935in}{3.088148in}}%
\pgfpathlineto{\pgfqpoint{5.491788in}{3.086654in}}%
\pgfpathlineto{\pgfqpoint{5.552342in}{3.086476in}}%
\pgfpathlineto{\pgfqpoint{5.554048in}{3.086047in}}%
\pgfpathlineto{\pgfqpoint{5.636777in}{3.086151in}}%
\pgfpathlineto{\pgfqpoint{5.637630in}{3.086854in}}%
\pgfpathlineto{\pgfqpoint{5.692214in}{3.090318in}}%
\pgfpathlineto{\pgfqpoint{5.694773in}{3.090948in}}%
\pgfpathlineto{\pgfqpoint{5.695626in}{3.091665in}}%
\pgfpathlineto{\pgfqpoint{5.697332in}{3.093929in}}%
\pgfpathlineto{\pgfqpoint{5.698185in}{3.092561in}}%
\pgfpathlineto{\pgfqpoint{5.699037in}{3.090551in}}%
\pgfpathlineto{\pgfqpoint{5.699890in}{3.093459in}}%
\pgfpathlineto{\pgfqpoint{5.714389in}{3.087136in}}%
\pgfpathlineto{\pgfqpoint{5.715242in}{3.087432in}}%
\pgfpathlineto{\pgfqpoint{5.717801in}{3.092663in}}%
\pgfpathlineto{\pgfqpoint{5.718654in}{3.093599in}}%
\pgfpathlineto{\pgfqpoint{5.731447in}{3.090478in}}%
\pgfpathlineto{\pgfqpoint{5.737417in}{3.090241in}}%
\pgfpathlineto{\pgfqpoint{5.738270in}{3.090897in}}%
\pgfpathlineto{\pgfqpoint{5.740828in}{3.108925in}}%
\pgfpathlineto{\pgfqpoint{5.757033in}{3.109066in}}%
\pgfpathlineto{\pgfqpoint{5.824411in}{3.130169in}}%
\pgfpathlineto{\pgfqpoint{5.825263in}{3.129677in}}%
\pgfpathlineto{\pgfqpoint{5.836351in}{3.105331in}}%
\pgfpathlineto{\pgfqpoint{5.837204in}{3.105085in}}%
\pgfpathlineto{\pgfqpoint{5.843174in}{3.112654in}}%
\pgfpathlineto{\pgfqpoint{5.859379in}{3.113664in}}%
\pgfpathlineto{\pgfqpoint{5.860231in}{3.113893in}}%
\pgfpathlineto{\pgfqpoint{5.861937in}{3.123741in}}%
\pgfpathlineto{\pgfqpoint{5.862790in}{3.124782in}}%
\pgfpathlineto{\pgfqpoint{5.880701in}{3.109384in}}%
\pgfpathlineto{\pgfqpoint{5.881553in}{3.111579in}}%
\pgfpathlineto{\pgfqpoint{5.884112in}{3.125226in}}%
\pgfpathlineto{\pgfqpoint{5.884965in}{3.125568in}}%
\pgfpathlineto{\pgfqpoint{5.965988in}{3.122701in}}%
\pgfpathlineto{\pgfqpoint{5.978782in}{3.123502in}}%
\pgfpathlineto{\pgfqpoint{5.982193in}{3.126182in}}%
\pgfpathlineto{\pgfqpoint{5.983899in}{3.127576in}}%
\pgfpathlineto{\pgfqpoint{5.984752in}{3.127012in}}%
\pgfpathlineto{\pgfqpoint{5.985605in}{3.120961in}}%
\pgfpathlineto{\pgfqpoint{5.986458in}{3.121506in}}%
\pgfpathlineto{\pgfqpoint{5.990722in}{3.126418in}}%
\pgfpathlineto{\pgfqpoint{5.996692in}{3.126499in}}%
\pgfpathlineto{\pgfqpoint{5.998398in}{3.126510in}}%
\pgfpathlineto{\pgfqpoint{5.999251in}{3.126180in}}%
\pgfpathlineto{\pgfqpoint{6.000956in}{3.120832in}}%
\pgfpathlineto{\pgfqpoint{6.002662in}{3.121179in}}%
\pgfpathlineto{\pgfqpoint{6.003515in}{3.121187in}}%
\pgfpathlineto{\pgfqpoint{6.004368in}{3.121718in}}%
\pgfpathlineto{\pgfqpoint{6.004368in}{3.121718in}}%
\pgfusepath{stroke}%
\end{pgfscope}%
\begin{pgfscope}%
\pgfsetrectcap%
\pgfsetmiterjoin%
\pgfsetlinewidth{0.501875pt}%
\definecolor{currentstroke}{rgb}{0.000000,0.000000,0.000000}%
\pgfsetstrokecolor{currentstroke}%
\pgfsetdash{}{0pt}%
\pgfpathmoveto{\pgfqpoint{0.481681in}{1.080890in}}%
\pgfpathlineto{\pgfqpoint{0.481681in}{3.227753in}}%
\pgfusepath{stroke}%
\end{pgfscope}%
\begin{pgfscope}%
\pgfsetrectcap%
\pgfsetmiterjoin%
\pgfsetlinewidth{0.501875pt}%
\definecolor{currentstroke}{rgb}{0.000000,0.000000,0.000000}%
\pgfsetstrokecolor{currentstroke}%
\pgfsetdash{}{0pt}%
\pgfpathmoveto{\pgfqpoint{6.267353in}{1.080890in}}%
\pgfpathlineto{\pgfqpoint{6.267353in}{3.227753in}}%
\pgfusepath{stroke}%
\end{pgfscope}%
\begin{pgfscope}%
\pgfsetrectcap%
\pgfsetmiterjoin%
\pgfsetlinewidth{0.501875pt}%
\definecolor{currentstroke}{rgb}{0.000000,0.000000,0.000000}%
\pgfsetstrokecolor{currentstroke}%
\pgfsetdash{}{0pt}%
\pgfpathmoveto{\pgfqpoint{0.481681in}{1.080890in}}%
\pgfpathlineto{\pgfqpoint{6.267353in}{1.080890in}}%
\pgfusepath{stroke}%
\end{pgfscope}%
\begin{pgfscope}%
\pgfsetrectcap%
\pgfsetmiterjoin%
\pgfsetlinewidth{0.501875pt}%
\definecolor{currentstroke}{rgb}{0.000000,0.000000,0.000000}%
\pgfsetstrokecolor{currentstroke}%
\pgfsetdash{}{0pt}%
\pgfpathmoveto{\pgfqpoint{0.481681in}{3.227753in}}%
\pgfpathlineto{\pgfqpoint{6.267353in}{3.227753in}}%
\pgfusepath{stroke}%
\end{pgfscope}%
\begin{pgfscope}%
\pgfsetrectcap%
\pgfsetroundjoin%
\pgfsetlinewidth{0.401500pt}%
\definecolor{currentstroke}{rgb}{0.000000,0.070588,0.098039}%
\pgfsetstrokecolor{currentstroke}%
\pgfsetdash{}{0pt}%
\pgfpathmoveto{\pgfqpoint{0.569181in}{3.105934in}}%
\pgfpathlineto{\pgfqpoint{0.666403in}{3.105934in}}%
\pgfpathlineto{\pgfqpoint{0.763625in}{3.105934in}}%
\pgfusepath{stroke}%
\end{pgfscope}%
\begin{pgfscope}%
\definecolor{textcolor}{rgb}{0.000000,0.000000,0.000000}%
\pgfsetstrokecolor{textcolor}%
\pgfsetfillcolor{textcolor}%
\pgftext[x=0.841403in,y=3.071906in,left,base]{\color{textcolor}\rmfamily\fontsize{7.000000}{8.400000}\selectfont Story Points}%
\end{pgfscope}%
\begin{pgfscope}%
\pgfsetrectcap%
\pgfsetroundjoin%
\pgfsetlinewidth{0.200750pt}%
\definecolor{currentstroke}{rgb}{0.682353,0.125490,0.070588}%
\pgfsetstrokecolor{currentstroke}%
\pgfsetdash{}{0pt}%
\pgfpathmoveto{\pgfqpoint{0.569181in}{2.969142in}}%
\pgfpathlineto{\pgfqpoint{0.666403in}{2.969142in}}%
\pgfpathlineto{\pgfqpoint{0.763625in}{2.969142in}}%
\pgfusepath{stroke}%
\end{pgfscope}%
\begin{pgfscope}%
\definecolor{textcolor}{rgb}{0.000000,0.000000,0.000000}%
\pgfsetstrokecolor{textcolor}%
\pgfsetfillcolor{textcolor}%
\pgftext[x=0.841403in,y=2.935114in,left,base]{\color{textcolor}\rmfamily\fontsize{7.000000}{8.400000}\selectfont Logische Codezeilen}%
\end{pgfscope}%
\begin{pgfscope}%
\pgfsetrectcap%
\pgfsetroundjoin%
\pgfsetlinewidth{0.200750pt}%
\definecolor{currentstroke}{rgb}{0.000000,0.372549,0.450980}%
\pgfsetstrokecolor{currentstroke}%
\pgfsetdash{}{0pt}%
\pgfpathmoveto{\pgfqpoint{0.569181in}{2.832545in}}%
\pgfpathlineto{\pgfqpoint{0.666403in}{2.832545in}}%
\pgfpathlineto{\pgfqpoint{0.763625in}{2.832545in}}%
\pgfusepath{stroke}%
\end{pgfscope}%
\begin{pgfscope}%
\definecolor{textcolor}{rgb}{0.000000,0.000000,0.000000}%
\pgfsetstrokecolor{textcolor}%
\pgfsetfillcolor{textcolor}%
\pgftext[x=0.841403in,y=2.798517in,left,base]{\color{textcolor}\rmfamily\fontsize{7.000000}{8.400000}\selectfont Zyklomatische Komplexität}%
\end{pgfscope}%
\begin{pgfscope}%
\pgfsetrectcap%
\pgfsetroundjoin%
\pgfsetlinewidth{0.200750pt}%
\definecolor{currentstroke}{rgb}{0.580392,0.823529,0.741176}%
\pgfsetstrokecolor{currentstroke}%
\pgfsetdash{}{0pt}%
\pgfpathmoveto{\pgfqpoint{0.569181in}{2.696045in}}%
\pgfpathlineto{\pgfqpoint{0.666403in}{2.696045in}}%
\pgfpathlineto{\pgfqpoint{0.763625in}{2.696045in}}%
\pgfusepath{stroke}%
\end{pgfscope}%
\begin{pgfscope}%
\definecolor{textcolor}{rgb}{0.000000,0.000000,0.000000}%
\pgfsetstrokecolor{textcolor}%
\pgfsetfillcolor{textcolor}%
\pgftext[x=0.841403in,y=2.662017in,left,base]{\color{textcolor}\rmfamily\fontsize{7.000000}{8.400000}\selectfont Halstead Aufwand}%
\end{pgfscope}%
\begin{pgfscope}%
\pgfsetrectcap%
\pgfsetroundjoin%
\pgfsetlinewidth{0.200750pt}%
\definecolor{currentstroke}{rgb}{0.933333,0.607843,0.000000}%
\pgfsetstrokecolor{currentstroke}%
\pgfsetdash{}{0pt}%
\pgfpathmoveto{\pgfqpoint{0.569181in}{2.560517in}}%
\pgfpathlineto{\pgfqpoint{0.666403in}{2.560517in}}%
\pgfpathlineto{\pgfqpoint{0.763625in}{2.560517in}}%
\pgfusepath{stroke}%
\end{pgfscope}%
\begin{pgfscope}%
\definecolor{textcolor}{rgb}{0.000000,0.000000,0.000000}%
\pgfsetstrokecolor{textcolor}%
\pgfsetfillcolor{textcolor}%
\pgftext[x=0.841403in,y=2.526489in,left,base]{\color{textcolor}\rmfamily\fontsize{7.000000}{8.400000}\selectfont Einrückungskomplexität}%
\end{pgfscope}%
\begin{pgfscope}%
\pgfsetbuttcap%
\pgfsetmiterjoin%
\definecolor{currentfill}{rgb}{1.000000,1.000000,1.000000}%
\pgfsetfillcolor{currentfill}%
\pgfsetlinewidth{0.000000pt}%
\definecolor{currentstroke}{rgb}{0.000000,0.000000,0.000000}%
\pgfsetstrokecolor{currentstroke}%
\pgfsetstrokeopacity{0.000000}%
\pgfsetdash{}{0pt}%
\pgfpathmoveto{\pgfqpoint{0.481681in}{0.586309in}}%
\pgfpathlineto{\pgfqpoint{6.267353in}{0.586309in}}%
\pgfpathlineto{\pgfqpoint{6.267353in}{0.893003in}}%
\pgfpathlineto{\pgfqpoint{0.481681in}{0.893003in}}%
\pgfpathlineto{\pgfqpoint{0.481681in}{0.586309in}}%
\pgfpathclose%
\pgfusepath{fill}%
\end{pgfscope}%
\begin{pgfscope}%
\pgfpathrectangle{\pgfqpoint{0.481681in}{0.586309in}}{\pgfqpoint{5.785672in}{0.306695in}}%
\pgfusepath{clip}%
\pgfsetbuttcap%
\pgfsetroundjoin%
\definecolor{currentfill}{rgb}{0.800000,0.788235,0.760784}%
\pgfsetfillcolor{currentfill}%
\pgfsetlinewidth{0.000000pt}%
\definecolor{currentstroke}{rgb}{0.000000,0.000000,0.000000}%
\pgfsetstrokecolor{currentstroke}%
\pgfsetdash{}{0pt}%
\pgfpathmoveto{\pgfqpoint{0.744666in}{0.739656in}}%
\pgfpathlineto{\pgfqpoint{0.744666in}{0.739656in}}%
\pgfpathlineto{\pgfqpoint{0.745519in}{0.739645in}}%
\pgfpathlineto{\pgfqpoint{0.746372in}{0.739630in}}%
\pgfpathlineto{\pgfqpoint{0.747225in}{0.739615in}}%
\pgfpathlineto{\pgfqpoint{0.748077in}{0.739601in}}%
\pgfpathlineto{\pgfqpoint{0.748930in}{0.739586in}}%
\pgfpathlineto{\pgfqpoint{0.749783in}{0.739571in}}%
\pgfpathlineto{\pgfqpoint{0.750636in}{0.739556in}}%
\pgfpathlineto{\pgfqpoint{0.751489in}{0.739542in}}%
\pgfpathlineto{\pgfqpoint{0.752342in}{0.739527in}}%
\pgfpathlineto{\pgfqpoint{0.753195in}{0.739512in}}%
\pgfpathlineto{\pgfqpoint{0.754048in}{0.739497in}}%
\pgfpathlineto{\pgfqpoint{0.754900in}{0.739482in}}%
\pgfpathlineto{\pgfqpoint{0.755753in}{0.739468in}}%
\pgfpathlineto{\pgfqpoint{0.756606in}{0.739453in}}%
\pgfpathlineto{\pgfqpoint{0.757459in}{0.739438in}}%
\pgfpathlineto{\pgfqpoint{0.758312in}{0.739423in}}%
\pgfpathlineto{\pgfqpoint{0.759165in}{0.739408in}}%
\pgfpathlineto{\pgfqpoint{0.760018in}{0.739394in}}%
\pgfpathlineto{\pgfqpoint{0.760871in}{0.739379in}}%
\pgfpathlineto{\pgfqpoint{0.761724in}{0.739380in}}%
\pgfpathlineto{\pgfqpoint{0.762576in}{0.739434in}}%
\pgfpathlineto{\pgfqpoint{0.763429in}{0.739493in}}%
\pgfpathlineto{\pgfqpoint{0.764282in}{0.739527in}}%
\pgfpathlineto{\pgfqpoint{0.765135in}{0.739536in}}%
\pgfpathlineto{\pgfqpoint{0.765988in}{0.739545in}}%
\pgfpathlineto{\pgfqpoint{0.766841in}{0.739554in}}%
\pgfpathlineto{\pgfqpoint{0.767694in}{0.739563in}}%
\pgfpathlineto{\pgfqpoint{0.768547in}{0.739572in}}%
\pgfpathlineto{\pgfqpoint{0.769399in}{0.739582in}}%
\pgfpathlineto{\pgfqpoint{0.770252in}{0.739591in}}%
\pgfpathlineto{\pgfqpoint{0.771105in}{0.739600in}}%
\pgfpathlineto{\pgfqpoint{0.771958in}{0.739609in}}%
\pgfpathlineto{\pgfqpoint{0.772811in}{0.739618in}}%
\pgfpathlineto{\pgfqpoint{0.773664in}{0.739627in}}%
\pgfpathlineto{\pgfqpoint{0.774517in}{0.739636in}}%
\pgfpathlineto{\pgfqpoint{0.775370in}{0.739646in}}%
\pgfpathlineto{\pgfqpoint{0.776222in}{0.739655in}}%
\pgfpathlineto{\pgfqpoint{0.777075in}{0.739664in}}%
\pgfpathlineto{\pgfqpoint{0.777928in}{0.739673in}}%
\pgfpathlineto{\pgfqpoint{0.778781in}{0.739682in}}%
\pgfpathlineto{\pgfqpoint{0.779634in}{0.739691in}}%
\pgfpathlineto{\pgfqpoint{0.780487in}{0.739700in}}%
\pgfpathlineto{\pgfqpoint{0.781340in}{0.739710in}}%
\pgfpathlineto{\pgfqpoint{0.782193in}{0.739719in}}%
\pgfpathlineto{\pgfqpoint{0.783045in}{0.739728in}}%
\pgfpathlineto{\pgfqpoint{0.783898in}{0.739737in}}%
\pgfpathlineto{\pgfqpoint{0.784751in}{0.739746in}}%
\pgfpathlineto{\pgfqpoint{0.785604in}{0.739755in}}%
\pgfpathlineto{\pgfqpoint{0.786457in}{0.739764in}}%
\pgfpathlineto{\pgfqpoint{0.787310in}{0.739774in}}%
\pgfpathlineto{\pgfqpoint{0.788163in}{0.739783in}}%
\pgfpathlineto{\pgfqpoint{0.789016in}{0.739792in}}%
\pgfpathlineto{\pgfqpoint{0.789868in}{0.739801in}}%
\pgfpathlineto{\pgfqpoint{0.790721in}{0.739810in}}%
\pgfpathlineto{\pgfqpoint{0.791574in}{0.739819in}}%
\pgfpathlineto{\pgfqpoint{0.792427in}{0.739828in}}%
\pgfpathlineto{\pgfqpoint{0.793280in}{0.739838in}}%
\pgfpathlineto{\pgfqpoint{0.794133in}{0.739847in}}%
\pgfpathlineto{\pgfqpoint{0.794986in}{0.739856in}}%
\pgfpathlineto{\pgfqpoint{0.795839in}{0.739865in}}%
\pgfpathlineto{\pgfqpoint{0.796692in}{0.739874in}}%
\pgfpathlineto{\pgfqpoint{0.797544in}{0.739883in}}%
\pgfpathlineto{\pgfqpoint{0.798397in}{0.739892in}}%
\pgfpathlineto{\pgfqpoint{0.799250in}{0.739902in}}%
\pgfpathlineto{\pgfqpoint{0.800103in}{0.739911in}}%
\pgfpathlineto{\pgfqpoint{0.800956in}{0.739920in}}%
\pgfpathlineto{\pgfqpoint{0.801809in}{0.739929in}}%
\pgfpathlineto{\pgfqpoint{0.802662in}{0.741108in}}%
\pgfpathlineto{\pgfqpoint{0.803515in}{0.743634in}}%
\pgfpathlineto{\pgfqpoint{0.804367in}{0.744146in}}%
\pgfpathlineto{\pgfqpoint{0.805220in}{0.744153in}}%
\pgfpathlineto{\pgfqpoint{0.806073in}{0.744159in}}%
\pgfpathlineto{\pgfqpoint{0.806926in}{0.744165in}}%
\pgfpathlineto{\pgfqpoint{0.807779in}{0.744171in}}%
\pgfpathlineto{\pgfqpoint{0.808632in}{0.744177in}}%
\pgfpathlineto{\pgfqpoint{0.809485in}{0.744183in}}%
\pgfpathlineto{\pgfqpoint{0.810338in}{0.744189in}}%
\pgfpathlineto{\pgfqpoint{0.811190in}{0.744195in}}%
\pgfpathlineto{\pgfqpoint{0.812043in}{0.744202in}}%
\pgfpathlineto{\pgfqpoint{0.812896in}{0.744208in}}%
\pgfpathlineto{\pgfqpoint{0.813749in}{0.744214in}}%
\pgfpathlineto{\pgfqpoint{0.814602in}{0.744220in}}%
\pgfpathlineto{\pgfqpoint{0.815455in}{0.744226in}}%
\pgfpathlineto{\pgfqpoint{0.816308in}{0.744232in}}%
\pgfpathlineto{\pgfqpoint{0.817161in}{0.744238in}}%
\pgfpathlineto{\pgfqpoint{0.818013in}{0.744245in}}%
\pgfpathlineto{\pgfqpoint{0.818866in}{0.744251in}}%
\pgfpathlineto{\pgfqpoint{0.819719in}{0.744257in}}%
\pgfpathlineto{\pgfqpoint{0.820572in}{0.744263in}}%
\pgfpathlineto{\pgfqpoint{0.821425in}{0.744116in}}%
\pgfpathlineto{\pgfqpoint{0.822278in}{0.742705in}}%
\pgfpathlineto{\pgfqpoint{0.823131in}{0.741014in}}%
\pgfpathlineto{\pgfqpoint{0.823984in}{0.740085in}}%
\pgfpathlineto{\pgfqpoint{0.824837in}{0.742386in}}%
\pgfpathlineto{\pgfqpoint{0.825689in}{0.744137in}}%
\pgfpathlineto{\pgfqpoint{0.826542in}{0.744159in}}%
\pgfpathlineto{\pgfqpoint{0.827395in}{0.744141in}}%
\pgfpathlineto{\pgfqpoint{0.828248in}{0.744100in}}%
\pgfpathlineto{\pgfqpoint{0.829101in}{0.743983in}}%
\pgfpathlineto{\pgfqpoint{0.829954in}{0.743862in}}%
\pgfpathlineto{\pgfqpoint{0.830807in}{0.743740in}}%
\pgfpathlineto{\pgfqpoint{0.831660in}{0.743618in}}%
\pgfpathlineto{\pgfqpoint{0.832512in}{0.743497in}}%
\pgfpathlineto{\pgfqpoint{0.833365in}{0.743375in}}%
\pgfpathlineto{\pgfqpoint{0.834218in}{0.743253in}}%
\pgfpathlineto{\pgfqpoint{0.835071in}{0.743132in}}%
\pgfpathlineto{\pgfqpoint{0.835924in}{0.743010in}}%
\pgfpathlineto{\pgfqpoint{0.836777in}{0.742888in}}%
\pgfpathlineto{\pgfqpoint{0.837630in}{0.742767in}}%
\pgfpathlineto{\pgfqpoint{0.838483in}{0.742645in}}%
\pgfpathlineto{\pgfqpoint{0.839335in}{0.742523in}}%
\pgfpathlineto{\pgfqpoint{0.840188in}{0.742402in}}%
\pgfpathlineto{\pgfqpoint{0.841041in}{0.742280in}}%
\pgfpathlineto{\pgfqpoint{0.841894in}{0.742158in}}%
\pgfpathlineto{\pgfqpoint{0.842747in}{0.742037in}}%
\pgfpathlineto{\pgfqpoint{0.843600in}{0.741915in}}%
\pgfpathlineto{\pgfqpoint{0.844453in}{0.741793in}}%
\pgfpathlineto{\pgfqpoint{0.845306in}{0.741672in}}%
\pgfpathlineto{\pgfqpoint{0.846158in}{0.741550in}}%
\pgfpathlineto{\pgfqpoint{0.847011in}{0.741428in}}%
\pgfpathlineto{\pgfqpoint{0.847864in}{0.741307in}}%
\pgfpathlineto{\pgfqpoint{0.848717in}{0.741185in}}%
\pgfpathlineto{\pgfqpoint{0.849570in}{0.741063in}}%
\pgfpathlineto{\pgfqpoint{0.850423in}{0.740942in}}%
\pgfpathlineto{\pgfqpoint{0.851276in}{0.740820in}}%
\pgfpathlineto{\pgfqpoint{0.852129in}{0.740698in}}%
\pgfpathlineto{\pgfqpoint{0.852982in}{0.740577in}}%
\pgfpathlineto{\pgfqpoint{0.853834in}{0.740455in}}%
\pgfpathlineto{\pgfqpoint{0.854687in}{0.740333in}}%
\pgfpathlineto{\pgfqpoint{0.855540in}{0.740212in}}%
\pgfpathlineto{\pgfqpoint{0.856393in}{0.740090in}}%
\pgfpathlineto{\pgfqpoint{0.857246in}{0.739968in}}%
\pgfpathlineto{\pgfqpoint{0.858099in}{0.739847in}}%
\pgfpathlineto{\pgfqpoint{0.858952in}{0.739725in}}%
\pgfpathlineto{\pgfqpoint{0.859805in}{0.739603in}}%
\pgfpathlineto{\pgfqpoint{0.860657in}{0.739482in}}%
\pgfpathlineto{\pgfqpoint{0.861510in}{0.740459in}}%
\pgfpathlineto{\pgfqpoint{0.862363in}{0.746170in}}%
\pgfpathlineto{\pgfqpoint{0.863216in}{0.746124in}}%
\pgfpathlineto{\pgfqpoint{0.864069in}{0.746064in}}%
\pgfpathlineto{\pgfqpoint{0.864922in}{0.746005in}}%
\pgfpathlineto{\pgfqpoint{0.865775in}{0.745945in}}%
\pgfpathlineto{\pgfqpoint{0.866628in}{0.745886in}}%
\pgfpathlineto{\pgfqpoint{0.867480in}{0.745826in}}%
\pgfpathlineto{\pgfqpoint{0.868333in}{0.745767in}}%
\pgfpathlineto{\pgfqpoint{0.869186in}{0.745707in}}%
\pgfpathlineto{\pgfqpoint{0.870039in}{0.745648in}}%
\pgfpathlineto{\pgfqpoint{0.870892in}{0.745589in}}%
\pgfpathlineto{\pgfqpoint{0.871745in}{0.745529in}}%
\pgfpathlineto{\pgfqpoint{0.872598in}{0.745470in}}%
\pgfpathlineto{\pgfqpoint{0.873451in}{0.745410in}}%
\pgfpathlineto{\pgfqpoint{0.874303in}{0.745351in}}%
\pgfpathlineto{\pgfqpoint{0.875156in}{0.745291in}}%
\pgfpathlineto{\pgfqpoint{0.876009in}{0.745232in}}%
\pgfpathlineto{\pgfqpoint{0.876862in}{0.745173in}}%
\pgfpathlineto{\pgfqpoint{0.877715in}{0.745113in}}%
\pgfpathlineto{\pgfqpoint{0.878568in}{0.745054in}}%
\pgfpathlineto{\pgfqpoint{0.879421in}{0.744994in}}%
\pgfpathlineto{\pgfqpoint{0.880274in}{0.744935in}}%
\pgfpathlineto{\pgfqpoint{0.881127in}{0.744875in}}%
\pgfpathlineto{\pgfqpoint{0.881979in}{0.744816in}}%
\pgfpathlineto{\pgfqpoint{0.882832in}{0.744756in}}%
\pgfpathlineto{\pgfqpoint{0.883685in}{0.744697in}}%
\pgfpathlineto{\pgfqpoint{0.884538in}{0.744638in}}%
\pgfpathlineto{\pgfqpoint{0.885391in}{0.744578in}}%
\pgfpathlineto{\pgfqpoint{0.886244in}{0.744519in}}%
\pgfpathlineto{\pgfqpoint{0.887097in}{0.744459in}}%
\pgfpathlineto{\pgfqpoint{0.887950in}{0.744400in}}%
\pgfpathlineto{\pgfqpoint{0.888802in}{0.744340in}}%
\pgfpathlineto{\pgfqpoint{0.889655in}{0.744281in}}%
\pgfpathlineto{\pgfqpoint{0.890508in}{0.744221in}}%
\pgfpathlineto{\pgfqpoint{0.891361in}{0.744162in}}%
\pgfpathlineto{\pgfqpoint{0.892214in}{0.744176in}}%
\pgfpathlineto{\pgfqpoint{0.893067in}{0.744204in}}%
\pgfpathlineto{\pgfqpoint{0.893920in}{0.744164in}}%
\pgfpathlineto{\pgfqpoint{0.894773in}{0.744124in}}%
\pgfpathlineto{\pgfqpoint{0.895625in}{0.744084in}}%
\pgfpathlineto{\pgfqpoint{0.896478in}{0.744044in}}%
\pgfpathlineto{\pgfqpoint{0.897331in}{0.744004in}}%
\pgfpathlineto{\pgfqpoint{0.898184in}{0.743964in}}%
\pgfpathlineto{\pgfqpoint{0.899037in}{0.743924in}}%
\pgfpathlineto{\pgfqpoint{0.899890in}{0.743884in}}%
\pgfpathlineto{\pgfqpoint{0.900743in}{0.743845in}}%
\pgfpathlineto{\pgfqpoint{0.901596in}{0.743805in}}%
\pgfpathlineto{\pgfqpoint{0.902448in}{0.743765in}}%
\pgfpathlineto{\pgfqpoint{0.903301in}{0.743725in}}%
\pgfpathlineto{\pgfqpoint{0.904154in}{0.743685in}}%
\pgfpathlineto{\pgfqpoint{0.905007in}{0.743645in}}%
\pgfpathlineto{\pgfqpoint{0.905860in}{0.743605in}}%
\pgfpathlineto{\pgfqpoint{0.906713in}{0.743565in}}%
\pgfpathlineto{\pgfqpoint{0.907566in}{0.743525in}}%
\pgfpathlineto{\pgfqpoint{0.908419in}{0.743485in}}%
\pgfpathlineto{\pgfqpoint{0.909271in}{0.743445in}}%
\pgfpathlineto{\pgfqpoint{0.910124in}{0.743405in}}%
\pgfpathlineto{\pgfqpoint{0.910977in}{0.743366in}}%
\pgfpathlineto{\pgfqpoint{0.911830in}{0.743326in}}%
\pgfpathlineto{\pgfqpoint{0.912683in}{0.743286in}}%
\pgfpathlineto{\pgfqpoint{0.913536in}{0.743246in}}%
\pgfpathlineto{\pgfqpoint{0.914389in}{0.743206in}}%
\pgfpathlineto{\pgfqpoint{0.915242in}{0.743166in}}%
\pgfpathlineto{\pgfqpoint{0.916095in}{0.743126in}}%
\pgfpathlineto{\pgfqpoint{0.916947in}{0.743086in}}%
\pgfpathlineto{\pgfqpoint{0.917800in}{0.743046in}}%
\pgfpathlineto{\pgfqpoint{0.918653in}{0.743006in}}%
\pgfpathlineto{\pgfqpoint{0.919506in}{0.742966in}}%
\pgfpathlineto{\pgfqpoint{0.920359in}{0.742926in}}%
\pgfpathlineto{\pgfqpoint{0.921212in}{0.742887in}}%
\pgfpathlineto{\pgfqpoint{0.922065in}{0.742847in}}%
\pgfpathlineto{\pgfqpoint{0.922918in}{0.742807in}}%
\pgfpathlineto{\pgfqpoint{0.923770in}{0.742767in}}%
\pgfpathlineto{\pgfqpoint{0.924623in}{0.742727in}}%
\pgfpathlineto{\pgfqpoint{0.925476in}{0.742687in}}%
\pgfpathlineto{\pgfqpoint{0.926329in}{0.742647in}}%
\pgfpathlineto{\pgfqpoint{0.927182in}{0.742607in}}%
\pgfpathlineto{\pgfqpoint{0.928035in}{0.742567in}}%
\pgfpathlineto{\pgfqpoint{0.928888in}{0.742527in}}%
\pgfpathlineto{\pgfqpoint{0.929741in}{0.742487in}}%
\pgfpathlineto{\pgfqpoint{0.930593in}{0.742447in}}%
\pgfpathlineto{\pgfqpoint{0.931446in}{0.742408in}}%
\pgfpathlineto{\pgfqpoint{0.932299in}{0.742368in}}%
\pgfpathlineto{\pgfqpoint{0.933152in}{0.742328in}}%
\pgfpathlineto{\pgfqpoint{0.934005in}{0.742288in}}%
\pgfpathlineto{\pgfqpoint{0.934858in}{0.742248in}}%
\pgfpathlineto{\pgfqpoint{0.935711in}{0.742208in}}%
\pgfpathlineto{\pgfqpoint{0.936564in}{0.742168in}}%
\pgfpathlineto{\pgfqpoint{0.937416in}{0.742128in}}%
\pgfpathlineto{\pgfqpoint{0.938269in}{0.742088in}}%
\pgfpathlineto{\pgfqpoint{0.939122in}{0.742048in}}%
\pgfpathlineto{\pgfqpoint{0.939975in}{0.742008in}}%
\pgfpathlineto{\pgfqpoint{0.940828in}{0.741968in}}%
\pgfpathlineto{\pgfqpoint{0.941681in}{0.741929in}}%
\pgfpathlineto{\pgfqpoint{0.942534in}{0.741889in}}%
\pgfpathlineto{\pgfqpoint{0.943387in}{0.742241in}}%
\pgfpathlineto{\pgfqpoint{0.944240in}{0.743403in}}%
\pgfpathlineto{\pgfqpoint{0.945092in}{0.743426in}}%
\pgfpathlineto{\pgfqpoint{0.945945in}{0.743011in}}%
\pgfpathlineto{\pgfqpoint{0.946798in}{0.742597in}}%
\pgfpathlineto{\pgfqpoint{0.947651in}{0.742242in}}%
\pgfpathlineto{\pgfqpoint{0.948504in}{0.742984in}}%
\pgfpathlineto{\pgfqpoint{0.949357in}{0.743956in}}%
\pgfpathlineto{\pgfqpoint{0.950210in}{0.744054in}}%
\pgfpathlineto{\pgfqpoint{0.951063in}{0.744039in}}%
\pgfpathlineto{\pgfqpoint{0.951915in}{0.744023in}}%
\pgfpathlineto{\pgfqpoint{0.952768in}{0.744008in}}%
\pgfpathlineto{\pgfqpoint{0.953621in}{0.743992in}}%
\pgfpathlineto{\pgfqpoint{0.954474in}{0.743977in}}%
\pgfpathlineto{\pgfqpoint{0.955327in}{0.743962in}}%
\pgfpathlineto{\pgfqpoint{0.956180in}{0.743946in}}%
\pgfpathlineto{\pgfqpoint{0.957033in}{0.743931in}}%
\pgfpathlineto{\pgfqpoint{0.957886in}{0.743915in}}%
\pgfpathlineto{\pgfqpoint{0.958738in}{0.743900in}}%
\pgfpathlineto{\pgfqpoint{0.959591in}{0.743884in}}%
\pgfpathlineto{\pgfqpoint{0.960444in}{0.743869in}}%
\pgfpathlineto{\pgfqpoint{0.961297in}{0.743853in}}%
\pgfpathlineto{\pgfqpoint{0.962150in}{0.743838in}}%
\pgfpathlineto{\pgfqpoint{0.963003in}{0.743822in}}%
\pgfpathlineto{\pgfqpoint{0.963856in}{0.743807in}}%
\pgfpathlineto{\pgfqpoint{0.964709in}{0.743839in}}%
\pgfpathlineto{\pgfqpoint{0.965561in}{0.743866in}}%
\pgfpathlineto{\pgfqpoint{0.966414in}{0.743835in}}%
\pgfpathlineto{\pgfqpoint{0.967267in}{0.743800in}}%
\pgfpathlineto{\pgfqpoint{0.968120in}{0.743763in}}%
\pgfpathlineto{\pgfqpoint{0.968973in}{0.743738in}}%
\pgfpathlineto{\pgfqpoint{0.969826in}{0.743720in}}%
\pgfpathlineto{\pgfqpoint{0.970679in}{0.743701in}}%
\pgfpathlineto{\pgfqpoint{0.971532in}{0.743683in}}%
\pgfpathlineto{\pgfqpoint{0.972385in}{0.743665in}}%
\pgfpathlineto{\pgfqpoint{0.973237in}{0.743647in}}%
\pgfpathlineto{\pgfqpoint{0.974090in}{0.743629in}}%
\pgfpathlineto{\pgfqpoint{0.974943in}{0.743611in}}%
\pgfpathlineto{\pgfqpoint{0.975796in}{0.743592in}}%
\pgfpathlineto{\pgfqpoint{0.976649in}{0.743574in}}%
\pgfpathlineto{\pgfqpoint{0.977502in}{0.743556in}}%
\pgfpathlineto{\pgfqpoint{0.978355in}{0.743538in}}%
\pgfpathlineto{\pgfqpoint{0.979208in}{0.743520in}}%
\pgfpathlineto{\pgfqpoint{0.980060in}{0.743502in}}%
\pgfpathlineto{\pgfqpoint{0.980913in}{0.743483in}}%
\pgfpathlineto{\pgfqpoint{0.981766in}{0.743465in}}%
\pgfpathlineto{\pgfqpoint{0.982619in}{0.743447in}}%
\pgfpathlineto{\pgfqpoint{0.983472in}{0.743429in}}%
\pgfpathlineto{\pgfqpoint{0.984325in}{0.743433in}}%
\pgfpathlineto{\pgfqpoint{0.985178in}{0.743420in}}%
\pgfpathlineto{\pgfqpoint{0.986031in}{0.743352in}}%
\pgfpathlineto{\pgfqpoint{0.986883in}{0.743151in}}%
\pgfpathlineto{\pgfqpoint{0.987736in}{0.742926in}}%
\pgfpathlineto{\pgfqpoint{0.988589in}{0.742467in}}%
\pgfpathlineto{\pgfqpoint{0.989442in}{0.741934in}}%
\pgfpathlineto{\pgfqpoint{0.990295in}{0.741404in}}%
\pgfpathlineto{\pgfqpoint{0.991148in}{0.741172in}}%
\pgfpathlineto{\pgfqpoint{0.992001in}{0.741150in}}%
\pgfpathlineto{\pgfqpoint{0.992854in}{0.741129in}}%
\pgfpathlineto{\pgfqpoint{0.993706in}{0.741108in}}%
\pgfpathlineto{\pgfqpoint{0.994559in}{0.741086in}}%
\pgfpathlineto{\pgfqpoint{0.995412in}{0.741065in}}%
\pgfpathlineto{\pgfqpoint{0.996265in}{0.741044in}}%
\pgfpathlineto{\pgfqpoint{0.997118in}{0.741022in}}%
\pgfpathlineto{\pgfqpoint{0.997971in}{0.741001in}}%
\pgfpathlineto{\pgfqpoint{0.998824in}{0.740980in}}%
\pgfpathlineto{\pgfqpoint{0.999677in}{0.740958in}}%
\pgfpathlineto{\pgfqpoint{1.000529in}{0.740937in}}%
\pgfpathlineto{\pgfqpoint{1.001382in}{0.740916in}}%
\pgfpathlineto{\pgfqpoint{1.002235in}{0.740894in}}%
\pgfpathlineto{\pgfqpoint{1.003088in}{0.740873in}}%
\pgfpathlineto{\pgfqpoint{1.003941in}{0.740852in}}%
\pgfpathlineto{\pgfqpoint{1.004794in}{0.742556in}}%
\pgfpathlineto{\pgfqpoint{1.005647in}{0.748625in}}%
\pgfpathlineto{\pgfqpoint{1.006500in}{0.748636in}}%
\pgfpathlineto{\pgfqpoint{1.007353in}{0.748912in}}%
\pgfpathlineto{\pgfqpoint{1.008205in}{0.750738in}}%
\pgfpathlineto{\pgfqpoint{1.009058in}{0.750867in}}%
\pgfpathlineto{\pgfqpoint{1.009911in}{0.750857in}}%
\pgfpathlineto{\pgfqpoint{1.010764in}{0.750846in}}%
\pgfpathlineto{\pgfqpoint{1.011617in}{0.750835in}}%
\pgfpathlineto{\pgfqpoint{1.012470in}{0.750822in}}%
\pgfpathlineto{\pgfqpoint{1.013323in}{0.750805in}}%
\pgfpathlineto{\pgfqpoint{1.014176in}{0.750788in}}%
\pgfpathlineto{\pgfqpoint{1.015028in}{0.750771in}}%
\pgfpathlineto{\pgfqpoint{1.015881in}{0.750754in}}%
\pgfpathlineto{\pgfqpoint{1.016734in}{0.750737in}}%
\pgfpathlineto{\pgfqpoint{1.017587in}{0.750720in}}%
\pgfpathlineto{\pgfqpoint{1.018440in}{0.750703in}}%
\pgfpathlineto{\pgfqpoint{1.019293in}{0.750686in}}%
\pgfpathlineto{\pgfqpoint{1.020146in}{0.750669in}}%
\pgfpathlineto{\pgfqpoint{1.020999in}{0.750652in}}%
\pgfpathlineto{\pgfqpoint{1.021851in}{0.750635in}}%
\pgfpathlineto{\pgfqpoint{1.022704in}{0.750618in}}%
\pgfpathlineto{\pgfqpoint{1.023557in}{0.750601in}}%
\pgfpathlineto{\pgfqpoint{1.024410in}{0.749765in}}%
\pgfpathlineto{\pgfqpoint{1.025263in}{0.746191in}}%
\pgfpathlineto{\pgfqpoint{1.026116in}{0.745642in}}%
\pgfpathlineto{\pgfqpoint{1.026969in}{0.750702in}}%
\pgfpathlineto{\pgfqpoint{1.027822in}{0.750642in}}%
\pgfpathlineto{\pgfqpoint{1.028674in}{0.750581in}}%
\pgfpathlineto{\pgfqpoint{1.029527in}{0.750520in}}%
\pgfpathlineto{\pgfqpoint{1.030380in}{0.750464in}}%
\pgfpathlineto{\pgfqpoint{1.031233in}{0.750441in}}%
\pgfpathlineto{\pgfqpoint{1.032086in}{0.750424in}}%
\pgfpathlineto{\pgfqpoint{1.032939in}{0.750408in}}%
\pgfpathlineto{\pgfqpoint{1.033792in}{0.750392in}}%
\pgfpathlineto{\pgfqpoint{1.034645in}{0.750376in}}%
\pgfpathlineto{\pgfqpoint{1.035498in}{0.750360in}}%
\pgfpathlineto{\pgfqpoint{1.036350in}{0.750344in}}%
\pgfpathlineto{\pgfqpoint{1.037203in}{0.750328in}}%
\pgfpathlineto{\pgfqpoint{1.038056in}{0.750312in}}%
\pgfpathlineto{\pgfqpoint{1.038909in}{0.750296in}}%
\pgfpathlineto{\pgfqpoint{1.039762in}{0.750280in}}%
\pgfpathlineto{\pgfqpoint{1.040615in}{0.750264in}}%
\pgfpathlineto{\pgfqpoint{1.041468in}{0.750248in}}%
\pgfpathlineto{\pgfqpoint{1.042321in}{0.750232in}}%
\pgfpathlineto{\pgfqpoint{1.043173in}{0.750216in}}%
\pgfpathlineto{\pgfqpoint{1.044026in}{0.750200in}}%
\pgfpathlineto{\pgfqpoint{1.044879in}{0.750184in}}%
\pgfpathlineto{\pgfqpoint{1.045732in}{0.750168in}}%
\pgfpathlineto{\pgfqpoint{1.046585in}{0.750151in}}%
\pgfpathlineto{\pgfqpoint{1.047438in}{0.750135in}}%
\pgfpathlineto{\pgfqpoint{1.048291in}{0.750119in}}%
\pgfpathlineto{\pgfqpoint{1.049144in}{0.750103in}}%
\pgfpathlineto{\pgfqpoint{1.049996in}{0.750087in}}%
\pgfpathlineto{\pgfqpoint{1.050849in}{0.750071in}}%
\pgfpathlineto{\pgfqpoint{1.051702in}{0.750055in}}%
\pgfpathlineto{\pgfqpoint{1.052555in}{0.750039in}}%
\pgfpathlineto{\pgfqpoint{1.053408in}{0.750023in}}%
\pgfpathlineto{\pgfqpoint{1.054261in}{0.750007in}}%
\pgfpathlineto{\pgfqpoint{1.055114in}{0.749991in}}%
\pgfpathlineto{\pgfqpoint{1.055967in}{0.749975in}}%
\pgfpathlineto{\pgfqpoint{1.056819in}{0.749959in}}%
\pgfpathlineto{\pgfqpoint{1.057672in}{0.749943in}}%
\pgfpathlineto{\pgfqpoint{1.058525in}{0.749927in}}%
\pgfpathlineto{\pgfqpoint{1.059378in}{0.749911in}}%
\pgfpathlineto{\pgfqpoint{1.060231in}{0.749894in}}%
\pgfpathlineto{\pgfqpoint{1.061084in}{0.749878in}}%
\pgfpathlineto{\pgfqpoint{1.061937in}{0.749862in}}%
\pgfpathlineto{\pgfqpoint{1.062790in}{0.749846in}}%
\pgfpathlineto{\pgfqpoint{1.063643in}{0.749830in}}%
\pgfpathlineto{\pgfqpoint{1.064495in}{0.749814in}}%
\pgfpathlineto{\pgfqpoint{1.065348in}{0.749798in}}%
\pgfpathlineto{\pgfqpoint{1.066201in}{0.749782in}}%
\pgfpathlineto{\pgfqpoint{1.067054in}{0.749766in}}%
\pgfpathlineto{\pgfqpoint{1.067907in}{0.749750in}}%
\pgfpathlineto{\pgfqpoint{1.068760in}{0.749734in}}%
\pgfpathlineto{\pgfqpoint{1.069613in}{0.749718in}}%
\pgfpathlineto{\pgfqpoint{1.070466in}{0.749702in}}%
\pgfpathlineto{\pgfqpoint{1.071318in}{0.749686in}}%
\pgfpathlineto{\pgfqpoint{1.072171in}{0.749670in}}%
\pgfpathlineto{\pgfqpoint{1.073024in}{0.749654in}}%
\pgfpathlineto{\pgfqpoint{1.073877in}{0.749638in}}%
\pgfpathlineto{\pgfqpoint{1.074730in}{0.749621in}}%
\pgfpathlineto{\pgfqpoint{1.075583in}{0.749605in}}%
\pgfpathlineto{\pgfqpoint{1.076436in}{0.749589in}}%
\pgfpathlineto{\pgfqpoint{1.077289in}{0.749573in}}%
\pgfpathlineto{\pgfqpoint{1.078141in}{0.749557in}}%
\pgfpathlineto{\pgfqpoint{1.078994in}{0.749541in}}%
\pgfpathlineto{\pgfqpoint{1.079847in}{0.749525in}}%
\pgfpathlineto{\pgfqpoint{1.080700in}{0.749509in}}%
\pgfpathlineto{\pgfqpoint{1.081553in}{0.749493in}}%
\pgfpathlineto{\pgfqpoint{1.082406in}{0.749477in}}%
\pgfpathlineto{\pgfqpoint{1.083259in}{0.749461in}}%
\pgfpathlineto{\pgfqpoint{1.084112in}{0.749445in}}%
\pgfpathlineto{\pgfqpoint{1.084964in}{0.749429in}}%
\pgfpathlineto{\pgfqpoint{1.085817in}{0.749430in}}%
\pgfpathlineto{\pgfqpoint{1.086670in}{0.749561in}}%
\pgfpathlineto{\pgfqpoint{1.087523in}{0.749557in}}%
\pgfpathlineto{\pgfqpoint{1.088376in}{0.749488in}}%
\pgfpathlineto{\pgfqpoint{1.089229in}{0.749419in}}%
\pgfpathlineto{\pgfqpoint{1.090082in}{0.749351in}}%
\pgfpathlineto{\pgfqpoint{1.090935in}{0.749425in}}%
\pgfpathlineto{\pgfqpoint{1.091788in}{0.749707in}}%
\pgfpathlineto{\pgfqpoint{1.092640in}{0.749678in}}%
\pgfpathlineto{\pgfqpoint{1.093493in}{0.746591in}}%
\pgfpathlineto{\pgfqpoint{1.094346in}{0.737489in}}%
\pgfpathlineto{\pgfqpoint{1.095199in}{0.747834in}}%
\pgfpathlineto{\pgfqpoint{1.096052in}{0.749481in}}%
\pgfpathlineto{\pgfqpoint{1.096905in}{0.749474in}}%
\pgfpathlineto{\pgfqpoint{1.097758in}{0.749466in}}%
\pgfpathlineto{\pgfqpoint{1.098611in}{0.749459in}}%
\pgfpathlineto{\pgfqpoint{1.099463in}{0.749451in}}%
\pgfpathlineto{\pgfqpoint{1.100316in}{0.749443in}}%
\pgfpathlineto{\pgfqpoint{1.101169in}{0.749436in}}%
\pgfpathlineto{\pgfqpoint{1.102022in}{0.749428in}}%
\pgfpathlineto{\pgfqpoint{1.102875in}{0.749421in}}%
\pgfpathlineto{\pgfqpoint{1.103728in}{0.749413in}}%
\pgfpathlineto{\pgfqpoint{1.104581in}{0.749406in}}%
\pgfpathlineto{\pgfqpoint{1.105434in}{0.749398in}}%
\pgfpathlineto{\pgfqpoint{1.106286in}{0.748260in}}%
\pgfpathlineto{\pgfqpoint{1.107139in}{0.741705in}}%
\pgfpathlineto{\pgfqpoint{1.107992in}{0.736702in}}%
\pgfpathlineto{\pgfqpoint{1.108845in}{0.748297in}}%
\pgfpathlineto{\pgfqpoint{1.109698in}{0.749409in}}%
\pgfpathlineto{\pgfqpoint{1.110551in}{0.749330in}}%
\pgfpathlineto{\pgfqpoint{1.111404in}{0.749251in}}%
\pgfpathlineto{\pgfqpoint{1.112257in}{0.749172in}}%
\pgfpathlineto{\pgfqpoint{1.113109in}{0.749109in}}%
\pgfpathlineto{\pgfqpoint{1.113962in}{0.749096in}}%
\pgfpathlineto{\pgfqpoint{1.114815in}{0.749094in}}%
\pgfpathlineto{\pgfqpoint{1.115668in}{0.749092in}}%
\pgfpathlineto{\pgfqpoint{1.116521in}{0.749091in}}%
\pgfpathlineto{\pgfqpoint{1.117374in}{0.749089in}}%
\pgfpathlineto{\pgfqpoint{1.118227in}{0.749087in}}%
\pgfpathlineto{\pgfqpoint{1.119080in}{0.749085in}}%
\pgfpathlineto{\pgfqpoint{1.119932in}{0.749083in}}%
\pgfpathlineto{\pgfqpoint{1.120785in}{0.749081in}}%
\pgfpathlineto{\pgfqpoint{1.121638in}{0.749079in}}%
\pgfpathlineto{\pgfqpoint{1.122491in}{0.749077in}}%
\pgfpathlineto{\pgfqpoint{1.123344in}{0.749075in}}%
\pgfpathlineto{\pgfqpoint{1.124197in}{0.749073in}}%
\pgfpathlineto{\pgfqpoint{1.125050in}{0.749071in}}%
\pgfpathlineto{\pgfqpoint{1.125903in}{0.749069in}}%
\pgfpathlineto{\pgfqpoint{1.126756in}{0.749067in}}%
\pgfpathlineto{\pgfqpoint{1.127608in}{0.749066in}}%
\pgfpathlineto{\pgfqpoint{1.128461in}{0.748990in}}%
\pgfpathlineto{\pgfqpoint{1.129314in}{0.749029in}}%
\pgfpathlineto{\pgfqpoint{1.130167in}{0.749457in}}%
\pgfpathlineto{\pgfqpoint{1.131020in}{0.749793in}}%
\pgfpathlineto{\pgfqpoint{1.131873in}{0.749846in}}%
\pgfpathlineto{\pgfqpoint{1.132726in}{0.749840in}}%
\pgfpathlineto{\pgfqpoint{1.133579in}{0.749834in}}%
\pgfpathlineto{\pgfqpoint{1.134431in}{0.749828in}}%
\pgfpathlineto{\pgfqpoint{1.135284in}{0.749822in}}%
\pgfpathlineto{\pgfqpoint{1.136137in}{0.749816in}}%
\pgfpathlineto{\pgfqpoint{1.136990in}{0.749810in}}%
\pgfpathlineto{\pgfqpoint{1.137843in}{0.749804in}}%
\pgfpathlineto{\pgfqpoint{1.138696in}{0.749798in}}%
\pgfpathlineto{\pgfqpoint{1.139549in}{0.749792in}}%
\pgfpathlineto{\pgfqpoint{1.140402in}{0.749787in}}%
\pgfpathlineto{\pgfqpoint{1.141254in}{0.749781in}}%
\pgfpathlineto{\pgfqpoint{1.142107in}{0.749775in}}%
\pgfpathlineto{\pgfqpoint{1.142960in}{0.749769in}}%
\pgfpathlineto{\pgfqpoint{1.143813in}{0.749763in}}%
\pgfpathlineto{\pgfqpoint{1.144666in}{0.749757in}}%
\pgfpathlineto{\pgfqpoint{1.145519in}{0.749751in}}%
\pgfpathlineto{\pgfqpoint{1.146372in}{0.749745in}}%
\pgfpathlineto{\pgfqpoint{1.147225in}{0.749739in}}%
\pgfpathlineto{\pgfqpoint{1.148077in}{0.749730in}}%
\pgfpathlineto{\pgfqpoint{1.148930in}{0.749714in}}%
\pgfpathlineto{\pgfqpoint{1.149783in}{0.749698in}}%
\pgfpathlineto{\pgfqpoint{1.150636in}{0.749681in}}%
\pgfpathlineto{\pgfqpoint{1.151489in}{0.749665in}}%
\pgfpathlineto{\pgfqpoint{1.152342in}{0.749648in}}%
\pgfpathlineto{\pgfqpoint{1.153195in}{0.749631in}}%
\pgfpathlineto{\pgfqpoint{1.154048in}{0.749615in}}%
\pgfpathlineto{\pgfqpoint{1.154901in}{0.749598in}}%
\pgfpathlineto{\pgfqpoint{1.155753in}{0.749582in}}%
\pgfpathlineto{\pgfqpoint{1.156606in}{0.749565in}}%
\pgfpathlineto{\pgfqpoint{1.157459in}{0.749548in}}%
\pgfpathlineto{\pgfqpoint{1.158312in}{0.749532in}}%
\pgfpathlineto{\pgfqpoint{1.159165in}{0.749515in}}%
\pgfpathlineto{\pgfqpoint{1.160018in}{0.749499in}}%
\pgfpathlineto{\pgfqpoint{1.160871in}{0.749482in}}%
\pgfpathlineto{\pgfqpoint{1.161724in}{0.749465in}}%
\pgfpathlineto{\pgfqpoint{1.162576in}{0.749449in}}%
\pgfpathlineto{\pgfqpoint{1.163429in}{0.749432in}}%
\pgfpathlineto{\pgfqpoint{1.164282in}{0.749416in}}%
\pgfpathlineto{\pgfqpoint{1.165135in}{0.749399in}}%
\pgfpathlineto{\pgfqpoint{1.165988in}{0.749382in}}%
\pgfpathlineto{\pgfqpoint{1.166841in}{0.749366in}}%
\pgfpathlineto{\pgfqpoint{1.167694in}{0.749349in}}%
\pgfpathlineto{\pgfqpoint{1.168547in}{0.749333in}}%
\pgfpathlineto{\pgfqpoint{1.169399in}{0.749316in}}%
\pgfpathlineto{\pgfqpoint{1.170252in}{0.749299in}}%
\pgfpathlineto{\pgfqpoint{1.171105in}{0.749283in}}%
\pgfpathlineto{\pgfqpoint{1.171958in}{0.749266in}}%
\pgfpathlineto{\pgfqpoint{1.172811in}{0.749250in}}%
\pgfpathlineto{\pgfqpoint{1.173664in}{0.749233in}}%
\pgfpathlineto{\pgfqpoint{1.174517in}{0.749216in}}%
\pgfpathlineto{\pgfqpoint{1.175370in}{0.749200in}}%
\pgfpathlineto{\pgfqpoint{1.176222in}{0.749183in}}%
\pgfpathlineto{\pgfqpoint{1.177075in}{0.749167in}}%
\pgfpathlineto{\pgfqpoint{1.177928in}{0.749150in}}%
\pgfpathlineto{\pgfqpoint{1.178781in}{0.749133in}}%
\pgfpathlineto{\pgfqpoint{1.179634in}{0.749117in}}%
\pgfpathlineto{\pgfqpoint{1.180487in}{0.749100in}}%
\pgfpathlineto{\pgfqpoint{1.181340in}{0.749084in}}%
\pgfpathlineto{\pgfqpoint{1.182193in}{0.749067in}}%
\pgfpathlineto{\pgfqpoint{1.183046in}{0.749050in}}%
\pgfpathlineto{\pgfqpoint{1.183898in}{0.749034in}}%
\pgfpathlineto{\pgfqpoint{1.184751in}{0.749017in}}%
\pgfpathlineto{\pgfqpoint{1.185604in}{0.749001in}}%
\pgfpathlineto{\pgfqpoint{1.186457in}{0.748984in}}%
\pgfpathlineto{\pgfqpoint{1.187310in}{0.748967in}}%
\pgfpathlineto{\pgfqpoint{1.188163in}{0.748951in}}%
\pgfpathlineto{\pgfqpoint{1.189016in}{0.748934in}}%
\pgfpathlineto{\pgfqpoint{1.189869in}{0.748918in}}%
\pgfpathlineto{\pgfqpoint{1.190721in}{0.748901in}}%
\pgfpathlineto{\pgfqpoint{1.191574in}{0.748884in}}%
\pgfpathlineto{\pgfqpoint{1.192427in}{0.748868in}}%
\pgfpathlineto{\pgfqpoint{1.193280in}{0.748851in}}%
\pgfpathlineto{\pgfqpoint{1.194133in}{0.748835in}}%
\pgfpathlineto{\pgfqpoint{1.194986in}{0.748818in}}%
\pgfpathlineto{\pgfqpoint{1.195839in}{0.748801in}}%
\pgfpathlineto{\pgfqpoint{1.196692in}{0.748785in}}%
\pgfpathlineto{\pgfqpoint{1.197544in}{0.748768in}}%
\pgfpathlineto{\pgfqpoint{1.198397in}{0.748752in}}%
\pgfpathlineto{\pgfqpoint{1.199250in}{0.748735in}}%
\pgfpathlineto{\pgfqpoint{1.200103in}{0.748718in}}%
\pgfpathlineto{\pgfqpoint{1.200956in}{0.748702in}}%
\pgfpathlineto{\pgfqpoint{1.201809in}{0.748685in}}%
\pgfpathlineto{\pgfqpoint{1.202662in}{0.748669in}}%
\pgfpathlineto{\pgfqpoint{1.203515in}{0.748652in}}%
\pgfpathlineto{\pgfqpoint{1.204367in}{0.748635in}}%
\pgfpathlineto{\pgfqpoint{1.205220in}{0.748619in}}%
\pgfpathlineto{\pgfqpoint{1.206073in}{0.748602in}}%
\pgfpathlineto{\pgfqpoint{1.206926in}{0.748586in}}%
\pgfpathlineto{\pgfqpoint{1.207779in}{0.748569in}}%
\pgfpathlineto{\pgfqpoint{1.208632in}{0.748552in}}%
\pgfpathlineto{\pgfqpoint{1.209485in}{0.748536in}}%
\pgfpathlineto{\pgfqpoint{1.210338in}{0.748519in}}%
\pgfpathlineto{\pgfqpoint{1.211191in}{0.748502in}}%
\pgfpathlineto{\pgfqpoint{1.212043in}{0.748486in}}%
\pgfpathlineto{\pgfqpoint{1.212896in}{0.748469in}}%
\pgfpathlineto{\pgfqpoint{1.213749in}{0.748453in}}%
\pgfpathlineto{\pgfqpoint{1.214602in}{0.748436in}}%
\pgfpathlineto{\pgfqpoint{1.215455in}{0.748419in}}%
\pgfpathlineto{\pgfqpoint{1.216308in}{0.748403in}}%
\pgfpathlineto{\pgfqpoint{1.217161in}{0.748386in}}%
\pgfpathlineto{\pgfqpoint{1.218014in}{0.748370in}}%
\pgfpathlineto{\pgfqpoint{1.218866in}{0.748353in}}%
\pgfpathlineto{\pgfqpoint{1.219719in}{0.748336in}}%
\pgfpathlineto{\pgfqpoint{1.220572in}{0.748320in}}%
\pgfpathlineto{\pgfqpoint{1.221425in}{0.748303in}}%
\pgfpathlineto{\pgfqpoint{1.222278in}{0.748287in}}%
\pgfpathlineto{\pgfqpoint{1.223131in}{0.748270in}}%
\pgfpathlineto{\pgfqpoint{1.223984in}{0.748253in}}%
\pgfpathlineto{\pgfqpoint{1.224837in}{0.748237in}}%
\pgfpathlineto{\pgfqpoint{1.225689in}{0.748220in}}%
\pgfpathlineto{\pgfqpoint{1.226542in}{0.748204in}}%
\pgfpathlineto{\pgfqpoint{1.227395in}{0.748187in}}%
\pgfpathlineto{\pgfqpoint{1.228248in}{0.748170in}}%
\pgfpathlineto{\pgfqpoint{1.229101in}{0.748154in}}%
\pgfpathlineto{\pgfqpoint{1.229954in}{0.748137in}}%
\pgfpathlineto{\pgfqpoint{1.230807in}{0.748121in}}%
\pgfpathlineto{\pgfqpoint{1.231660in}{0.748104in}}%
\pgfpathlineto{\pgfqpoint{1.232512in}{0.748087in}}%
\pgfpathlineto{\pgfqpoint{1.233365in}{0.748071in}}%
\pgfpathlineto{\pgfqpoint{1.234218in}{0.748054in}}%
\pgfpathlineto{\pgfqpoint{1.235071in}{0.748038in}}%
\pgfpathlineto{\pgfqpoint{1.235924in}{0.748021in}}%
\pgfpathlineto{\pgfqpoint{1.236777in}{0.748004in}}%
\pgfpathlineto{\pgfqpoint{1.237630in}{0.747988in}}%
\pgfpathlineto{\pgfqpoint{1.238483in}{0.747971in}}%
\pgfpathlineto{\pgfqpoint{1.239335in}{0.747955in}}%
\pgfpathlineto{\pgfqpoint{1.240188in}{0.747938in}}%
\pgfpathlineto{\pgfqpoint{1.241041in}{0.747921in}}%
\pgfpathlineto{\pgfqpoint{1.241894in}{0.747905in}}%
\pgfpathlineto{\pgfqpoint{1.242747in}{0.747888in}}%
\pgfpathlineto{\pgfqpoint{1.243600in}{0.747872in}}%
\pgfpathlineto{\pgfqpoint{1.244453in}{0.747855in}}%
\pgfpathlineto{\pgfqpoint{1.245306in}{0.747838in}}%
\pgfpathlineto{\pgfqpoint{1.246159in}{0.747822in}}%
\pgfpathlineto{\pgfqpoint{1.247011in}{0.747805in}}%
\pgfpathlineto{\pgfqpoint{1.247864in}{0.747789in}}%
\pgfpathlineto{\pgfqpoint{1.248717in}{0.747772in}}%
\pgfpathlineto{\pgfqpoint{1.249570in}{0.747755in}}%
\pgfpathlineto{\pgfqpoint{1.250423in}{0.747739in}}%
\pgfpathlineto{\pgfqpoint{1.251276in}{0.747722in}}%
\pgfpathlineto{\pgfqpoint{1.252129in}{0.747706in}}%
\pgfpathlineto{\pgfqpoint{1.252982in}{0.747726in}}%
\pgfpathlineto{\pgfqpoint{1.253834in}{0.747765in}}%
\pgfpathlineto{\pgfqpoint{1.254687in}{0.747805in}}%
\pgfpathlineto{\pgfqpoint{1.255540in}{0.747844in}}%
\pgfpathlineto{\pgfqpoint{1.256393in}{0.747884in}}%
\pgfpathlineto{\pgfqpoint{1.257246in}{0.747923in}}%
\pgfpathlineto{\pgfqpoint{1.258099in}{0.747963in}}%
\pgfpathlineto{\pgfqpoint{1.258952in}{0.748002in}}%
\pgfpathlineto{\pgfqpoint{1.259805in}{0.748042in}}%
\pgfpathlineto{\pgfqpoint{1.260657in}{0.748081in}}%
\pgfpathlineto{\pgfqpoint{1.261510in}{0.748121in}}%
\pgfpathlineto{\pgfqpoint{1.262363in}{0.748160in}}%
\pgfpathlineto{\pgfqpoint{1.263216in}{0.748200in}}%
\pgfpathlineto{\pgfqpoint{1.264069in}{0.748239in}}%
\pgfpathlineto{\pgfqpoint{1.264922in}{0.748279in}}%
\pgfpathlineto{\pgfqpoint{1.265775in}{0.748319in}}%
\pgfpathlineto{\pgfqpoint{1.266628in}{0.748358in}}%
\pgfpathlineto{\pgfqpoint{1.267480in}{0.748398in}}%
\pgfpathlineto{\pgfqpoint{1.268333in}{0.748437in}}%
\pgfpathlineto{\pgfqpoint{1.269186in}{0.748477in}}%
\pgfpathlineto{\pgfqpoint{1.270039in}{0.748516in}}%
\pgfpathlineto{\pgfqpoint{1.270892in}{0.748555in}}%
\pgfpathlineto{\pgfqpoint{1.271745in}{0.748515in}}%
\pgfpathlineto{\pgfqpoint{1.272598in}{0.747709in}}%
\pgfpathlineto{\pgfqpoint{1.273451in}{0.747533in}}%
\pgfpathlineto{\pgfqpoint{1.274304in}{0.747553in}}%
\pgfpathlineto{\pgfqpoint{1.275156in}{0.747564in}}%
\pgfpathlineto{\pgfqpoint{1.276009in}{0.747553in}}%
\pgfpathlineto{\pgfqpoint{1.276862in}{0.747540in}}%
\pgfpathlineto{\pgfqpoint{1.277715in}{0.747528in}}%
\pgfpathlineto{\pgfqpoint{1.278568in}{0.747516in}}%
\pgfpathlineto{\pgfqpoint{1.279421in}{0.747503in}}%
\pgfpathlineto{\pgfqpoint{1.280274in}{0.747491in}}%
\pgfpathlineto{\pgfqpoint{1.281127in}{0.747479in}}%
\pgfpathlineto{\pgfqpoint{1.281979in}{0.747467in}}%
\pgfpathlineto{\pgfqpoint{1.282832in}{0.747454in}}%
\pgfpathlineto{\pgfqpoint{1.283685in}{0.747442in}}%
\pgfpathlineto{\pgfqpoint{1.284538in}{0.747430in}}%
\pgfpathlineto{\pgfqpoint{1.285391in}{0.747417in}}%
\pgfpathlineto{\pgfqpoint{1.286244in}{0.747405in}}%
\pgfpathlineto{\pgfqpoint{1.287097in}{0.747393in}}%
\pgfpathlineto{\pgfqpoint{1.287950in}{0.747380in}}%
\pgfpathlineto{\pgfqpoint{1.288802in}{0.747406in}}%
\pgfpathlineto{\pgfqpoint{1.289655in}{0.748019in}}%
\pgfpathlineto{\pgfqpoint{1.290508in}{0.748314in}}%
\pgfpathlineto{\pgfqpoint{1.291361in}{0.748523in}}%
\pgfpathlineto{\pgfqpoint{1.292214in}{0.748480in}}%
\pgfpathlineto{\pgfqpoint{1.293067in}{0.748438in}}%
\pgfpathlineto{\pgfqpoint{1.293920in}{0.748395in}}%
\pgfpathlineto{\pgfqpoint{1.294773in}{0.748352in}}%
\pgfpathlineto{\pgfqpoint{1.295625in}{0.748309in}}%
\pgfpathlineto{\pgfqpoint{1.296478in}{0.748268in}}%
\pgfpathlineto{\pgfqpoint{1.297331in}{0.748252in}}%
\pgfpathlineto{\pgfqpoint{1.298184in}{0.748244in}}%
\pgfpathlineto{\pgfqpoint{1.299037in}{0.748236in}}%
\pgfpathlineto{\pgfqpoint{1.299890in}{0.748228in}}%
\pgfpathlineto{\pgfqpoint{1.300743in}{0.748220in}}%
\pgfpathlineto{\pgfqpoint{1.301596in}{0.748212in}}%
\pgfpathlineto{\pgfqpoint{1.302449in}{0.748205in}}%
\pgfpathlineto{\pgfqpoint{1.303301in}{0.748197in}}%
\pgfpathlineto{\pgfqpoint{1.304154in}{0.748189in}}%
\pgfpathlineto{\pgfqpoint{1.305007in}{0.748181in}}%
\pgfpathlineto{\pgfqpoint{1.305860in}{0.748173in}}%
\pgfpathlineto{\pgfqpoint{1.306713in}{0.748166in}}%
\pgfpathlineto{\pgfqpoint{1.307566in}{0.748158in}}%
\pgfpathlineto{\pgfqpoint{1.308419in}{0.748150in}}%
\pgfpathlineto{\pgfqpoint{1.309272in}{0.748142in}}%
\pgfpathlineto{\pgfqpoint{1.310124in}{0.748134in}}%
\pgfpathlineto{\pgfqpoint{1.310977in}{0.748126in}}%
\pgfpathlineto{\pgfqpoint{1.311830in}{0.748150in}}%
\pgfpathlineto{\pgfqpoint{1.312683in}{0.748243in}}%
\pgfpathlineto{\pgfqpoint{1.313536in}{0.748211in}}%
\pgfpathlineto{\pgfqpoint{1.314389in}{0.748290in}}%
\pgfpathlineto{\pgfqpoint{1.315242in}{0.748240in}}%
\pgfpathlineto{\pgfqpoint{1.316095in}{0.748300in}}%
\pgfpathlineto{\pgfqpoint{1.316947in}{0.748303in}}%
\pgfpathlineto{\pgfqpoint{1.317800in}{0.748260in}}%
\pgfpathlineto{\pgfqpoint{1.318653in}{0.748225in}}%
\pgfpathlineto{\pgfqpoint{1.319506in}{0.748221in}}%
\pgfpathlineto{\pgfqpoint{1.320359in}{0.748213in}}%
\pgfpathlineto{\pgfqpoint{1.321212in}{0.748205in}}%
\pgfpathlineto{\pgfqpoint{1.322065in}{0.748196in}}%
\pgfpathlineto{\pgfqpoint{1.322918in}{0.748188in}}%
\pgfpathlineto{\pgfqpoint{1.323770in}{0.748180in}}%
\pgfpathlineto{\pgfqpoint{1.324623in}{0.748172in}}%
\pgfpathlineto{\pgfqpoint{1.325476in}{0.748164in}}%
\pgfpathlineto{\pgfqpoint{1.326329in}{0.748155in}}%
\pgfpathlineto{\pgfqpoint{1.327182in}{0.748147in}}%
\pgfpathlineto{\pgfqpoint{1.328035in}{0.748139in}}%
\pgfpathlineto{\pgfqpoint{1.328888in}{0.748131in}}%
\pgfpathlineto{\pgfqpoint{1.329741in}{0.748122in}}%
\pgfpathlineto{\pgfqpoint{1.330593in}{0.748114in}}%
\pgfpathlineto{\pgfqpoint{1.331446in}{0.748106in}}%
\pgfpathlineto{\pgfqpoint{1.332299in}{0.748098in}}%
\pgfpathlineto{\pgfqpoint{1.333152in}{0.748090in}}%
\pgfpathlineto{\pgfqpoint{1.334005in}{0.748081in}}%
\pgfpathlineto{\pgfqpoint{1.334858in}{0.748072in}}%
\pgfpathlineto{\pgfqpoint{1.335711in}{0.748041in}}%
\pgfpathlineto{\pgfqpoint{1.336564in}{0.747999in}}%
\pgfpathlineto{\pgfqpoint{1.337417in}{0.747958in}}%
\pgfpathlineto{\pgfqpoint{1.338269in}{0.747916in}}%
\pgfpathlineto{\pgfqpoint{1.339122in}{0.747875in}}%
\pgfpathlineto{\pgfqpoint{1.339975in}{0.747834in}}%
\pgfpathlineto{\pgfqpoint{1.340828in}{0.747792in}}%
\pgfpathlineto{\pgfqpoint{1.341681in}{0.747751in}}%
\pgfpathlineto{\pgfqpoint{1.342534in}{0.747709in}}%
\pgfpathlineto{\pgfqpoint{1.343387in}{0.747668in}}%
\pgfpathlineto{\pgfqpoint{1.344240in}{0.747626in}}%
\pgfpathlineto{\pgfqpoint{1.345092in}{0.747585in}}%
\pgfpathlineto{\pgfqpoint{1.345945in}{0.747544in}}%
\pgfpathlineto{\pgfqpoint{1.346798in}{0.747502in}}%
\pgfpathlineto{\pgfqpoint{1.347651in}{0.747461in}}%
\pgfpathlineto{\pgfqpoint{1.348504in}{0.747419in}}%
\pgfpathlineto{\pgfqpoint{1.349357in}{0.747378in}}%
\pgfpathlineto{\pgfqpoint{1.350210in}{0.747336in}}%
\pgfpathlineto{\pgfqpoint{1.351063in}{0.747295in}}%
\pgfpathlineto{\pgfqpoint{1.351915in}{0.747253in}}%
\pgfpathlineto{\pgfqpoint{1.352768in}{0.747212in}}%
\pgfpathlineto{\pgfqpoint{1.353621in}{0.747171in}}%
\pgfpathlineto{\pgfqpoint{1.354474in}{0.747129in}}%
\pgfpathlineto{\pgfqpoint{1.355327in}{0.747088in}}%
\pgfpathlineto{\pgfqpoint{1.356180in}{0.747046in}}%
\pgfpathlineto{\pgfqpoint{1.357033in}{0.747005in}}%
\pgfpathlineto{\pgfqpoint{1.357886in}{0.746963in}}%
\pgfpathlineto{\pgfqpoint{1.358738in}{0.746922in}}%
\pgfpathlineto{\pgfqpoint{1.359591in}{0.746881in}}%
\pgfpathlineto{\pgfqpoint{1.360444in}{0.746839in}}%
\pgfpathlineto{\pgfqpoint{1.361297in}{0.746798in}}%
\pgfpathlineto{\pgfqpoint{1.362150in}{0.746756in}}%
\pgfpathlineto{\pgfqpoint{1.363003in}{0.746715in}}%
\pgfpathlineto{\pgfqpoint{1.363856in}{0.746673in}}%
\pgfpathlineto{\pgfqpoint{1.364709in}{0.746632in}}%
\pgfpathlineto{\pgfqpoint{1.365562in}{0.746591in}}%
\pgfpathlineto{\pgfqpoint{1.366414in}{0.746549in}}%
\pgfpathlineto{\pgfqpoint{1.367267in}{0.746508in}}%
\pgfpathlineto{\pgfqpoint{1.368120in}{0.746466in}}%
\pgfpathlineto{\pgfqpoint{1.368973in}{0.746425in}}%
\pgfpathlineto{\pgfqpoint{1.369826in}{0.746383in}}%
\pgfpathlineto{\pgfqpoint{1.370679in}{0.746342in}}%
\pgfpathlineto{\pgfqpoint{1.371532in}{0.746300in}}%
\pgfpathlineto{\pgfqpoint{1.372385in}{0.746259in}}%
\pgfpathlineto{\pgfqpoint{1.373237in}{0.746218in}}%
\pgfpathlineto{\pgfqpoint{1.374090in}{0.745998in}}%
\pgfpathlineto{\pgfqpoint{1.374943in}{0.744894in}}%
\pgfpathlineto{\pgfqpoint{1.375796in}{0.744882in}}%
\pgfpathlineto{\pgfqpoint{1.376649in}{0.744870in}}%
\pgfpathlineto{\pgfqpoint{1.377502in}{0.744857in}}%
\pgfpathlineto{\pgfqpoint{1.378355in}{0.744845in}}%
\pgfpathlineto{\pgfqpoint{1.379208in}{0.744833in}}%
\pgfpathlineto{\pgfqpoint{1.380060in}{0.744821in}}%
\pgfpathlineto{\pgfqpoint{1.380913in}{0.744809in}}%
\pgfpathlineto{\pgfqpoint{1.381766in}{0.744796in}}%
\pgfpathlineto{\pgfqpoint{1.382619in}{0.744784in}}%
\pgfpathlineto{\pgfqpoint{1.383472in}{0.744772in}}%
\pgfpathlineto{\pgfqpoint{1.384325in}{0.744760in}}%
\pgfpathlineto{\pgfqpoint{1.385178in}{0.744747in}}%
\pgfpathlineto{\pgfqpoint{1.386031in}{0.744735in}}%
\pgfpathlineto{\pgfqpoint{1.386883in}{0.744723in}}%
\pgfpathlineto{\pgfqpoint{1.387736in}{0.744711in}}%
\pgfpathlineto{\pgfqpoint{1.388589in}{0.744698in}}%
\pgfpathlineto{\pgfqpoint{1.389442in}{0.744686in}}%
\pgfpathlineto{\pgfqpoint{1.390295in}{0.744674in}}%
\pgfpathlineto{\pgfqpoint{1.391148in}{0.744662in}}%
\pgfpathlineto{\pgfqpoint{1.392001in}{0.744649in}}%
\pgfpathlineto{\pgfqpoint{1.392854in}{0.744637in}}%
\pgfpathlineto{\pgfqpoint{1.393707in}{0.744625in}}%
\pgfpathlineto{\pgfqpoint{1.394559in}{0.744613in}}%
\pgfpathlineto{\pgfqpoint{1.395412in}{0.744601in}}%
\pgfpathlineto{\pgfqpoint{1.396265in}{0.744588in}}%
\pgfpathlineto{\pgfqpoint{1.397118in}{0.744576in}}%
\pgfpathlineto{\pgfqpoint{1.397971in}{0.744564in}}%
\pgfpathlineto{\pgfqpoint{1.398824in}{0.744552in}}%
\pgfpathlineto{\pgfqpoint{1.399677in}{0.744539in}}%
\pgfpathlineto{\pgfqpoint{1.400530in}{0.744527in}}%
\pgfpathlineto{\pgfqpoint{1.401382in}{0.744515in}}%
\pgfpathlineto{\pgfqpoint{1.402235in}{0.744503in}}%
\pgfpathlineto{\pgfqpoint{1.403088in}{0.744490in}}%
\pgfpathlineto{\pgfqpoint{1.403941in}{0.744478in}}%
\pgfpathlineto{\pgfqpoint{1.404794in}{0.744466in}}%
\pgfpathlineto{\pgfqpoint{1.405647in}{0.744454in}}%
\pgfpathlineto{\pgfqpoint{1.406500in}{0.744441in}}%
\pgfpathlineto{\pgfqpoint{1.407353in}{0.744429in}}%
\pgfpathlineto{\pgfqpoint{1.408205in}{0.744417in}}%
\pgfpathlineto{\pgfqpoint{1.409058in}{0.744405in}}%
\pgfpathlineto{\pgfqpoint{1.409911in}{0.744393in}}%
\pgfpathlineto{\pgfqpoint{1.410764in}{0.744380in}}%
\pgfpathlineto{\pgfqpoint{1.411617in}{0.744368in}}%
\pgfpathlineto{\pgfqpoint{1.412470in}{0.744356in}}%
\pgfpathlineto{\pgfqpoint{1.413323in}{0.744344in}}%
\pgfpathlineto{\pgfqpoint{1.414176in}{0.744331in}}%
\pgfpathlineto{\pgfqpoint{1.415028in}{0.744319in}}%
\pgfpathlineto{\pgfqpoint{1.415881in}{0.744285in}}%
\pgfpathlineto{\pgfqpoint{1.416734in}{0.744269in}}%
\pgfpathlineto{\pgfqpoint{1.417587in}{0.744257in}}%
\pgfpathlineto{\pgfqpoint{1.418440in}{0.744245in}}%
\pgfpathlineto{\pgfqpoint{1.419293in}{0.744232in}}%
\pgfpathlineto{\pgfqpoint{1.420146in}{0.744220in}}%
\pgfpathlineto{\pgfqpoint{1.420999in}{0.744208in}}%
\pgfpathlineto{\pgfqpoint{1.421852in}{0.744195in}}%
\pgfpathlineto{\pgfqpoint{1.422704in}{0.744183in}}%
\pgfpathlineto{\pgfqpoint{1.423557in}{0.744171in}}%
\pgfpathlineto{\pgfqpoint{1.424410in}{0.744159in}}%
\pgfpathlineto{\pgfqpoint{1.425263in}{0.744146in}}%
\pgfpathlineto{\pgfqpoint{1.426116in}{0.744134in}}%
\pgfpathlineto{\pgfqpoint{1.426969in}{0.744122in}}%
\pgfpathlineto{\pgfqpoint{1.427822in}{0.744110in}}%
\pgfpathlineto{\pgfqpoint{1.428675in}{0.744097in}}%
\pgfpathlineto{\pgfqpoint{1.429527in}{0.744085in}}%
\pgfpathlineto{\pgfqpoint{1.430380in}{0.744073in}}%
\pgfpathlineto{\pgfqpoint{1.431233in}{0.744060in}}%
\pgfpathlineto{\pgfqpoint{1.432086in}{0.744048in}}%
\pgfpathlineto{\pgfqpoint{1.432939in}{0.744036in}}%
\pgfpathlineto{\pgfqpoint{1.433792in}{0.744024in}}%
\pgfpathlineto{\pgfqpoint{1.434645in}{0.744019in}}%
\pgfpathlineto{\pgfqpoint{1.435498in}{0.744018in}}%
\pgfpathlineto{\pgfqpoint{1.436350in}{0.744018in}}%
\pgfpathlineto{\pgfqpoint{1.437203in}{0.744017in}}%
\pgfpathlineto{\pgfqpoint{1.438056in}{0.744016in}}%
\pgfpathlineto{\pgfqpoint{1.438909in}{0.744015in}}%
\pgfpathlineto{\pgfqpoint{1.439762in}{0.744015in}}%
\pgfpathlineto{\pgfqpoint{1.440615in}{0.744014in}}%
\pgfpathlineto{\pgfqpoint{1.441468in}{0.744013in}}%
\pgfpathlineto{\pgfqpoint{1.442321in}{0.744012in}}%
\pgfpathlineto{\pgfqpoint{1.443173in}{0.744011in}}%
\pgfpathlineto{\pgfqpoint{1.444026in}{0.744011in}}%
\pgfpathlineto{\pgfqpoint{1.444879in}{0.744010in}}%
\pgfpathlineto{\pgfqpoint{1.445732in}{0.744009in}}%
\pgfpathlineto{\pgfqpoint{1.446585in}{0.744008in}}%
\pgfpathlineto{\pgfqpoint{1.447438in}{0.744008in}}%
\pgfpathlineto{\pgfqpoint{1.448291in}{0.744007in}}%
\pgfpathlineto{\pgfqpoint{1.449144in}{0.744006in}}%
\pgfpathlineto{\pgfqpoint{1.449996in}{0.744005in}}%
\pgfpathlineto{\pgfqpoint{1.450849in}{0.744005in}}%
\pgfpathlineto{\pgfqpoint{1.451702in}{0.744004in}}%
\pgfpathlineto{\pgfqpoint{1.452555in}{0.744003in}}%
\pgfpathlineto{\pgfqpoint{1.453408in}{0.744002in}}%
\pgfpathlineto{\pgfqpoint{1.454261in}{0.744002in}}%
\pgfpathlineto{\pgfqpoint{1.455114in}{0.744001in}}%
\pgfpathlineto{\pgfqpoint{1.455967in}{0.744000in}}%
\pgfpathlineto{\pgfqpoint{1.456820in}{0.743999in}}%
\pgfpathlineto{\pgfqpoint{1.457672in}{0.743998in}}%
\pgfpathlineto{\pgfqpoint{1.458525in}{0.743998in}}%
\pgfpathlineto{\pgfqpoint{1.459378in}{0.743997in}}%
\pgfpathlineto{\pgfqpoint{1.460231in}{0.743996in}}%
\pgfpathlineto{\pgfqpoint{1.461084in}{0.743995in}}%
\pgfpathlineto{\pgfqpoint{1.461937in}{0.743984in}}%
\pgfpathlineto{\pgfqpoint{1.462790in}{0.743897in}}%
\pgfpathlineto{\pgfqpoint{1.463643in}{0.743795in}}%
\pgfpathlineto{\pgfqpoint{1.464495in}{0.743693in}}%
\pgfpathlineto{\pgfqpoint{1.465348in}{0.743590in}}%
\pgfpathlineto{\pgfqpoint{1.466201in}{0.743488in}}%
\pgfpathlineto{\pgfqpoint{1.467054in}{0.743386in}}%
\pgfpathlineto{\pgfqpoint{1.467907in}{0.743284in}}%
\pgfpathlineto{\pgfqpoint{1.468760in}{0.743182in}}%
\pgfpathlineto{\pgfqpoint{1.469613in}{0.743080in}}%
\pgfpathlineto{\pgfqpoint{1.470466in}{0.742978in}}%
\pgfpathlineto{\pgfqpoint{1.471318in}{0.742875in}}%
\pgfpathlineto{\pgfqpoint{1.472171in}{0.742773in}}%
\pgfpathlineto{\pgfqpoint{1.473024in}{0.742671in}}%
\pgfpathlineto{\pgfqpoint{1.473877in}{0.742569in}}%
\pgfpathlineto{\pgfqpoint{1.474730in}{0.742467in}}%
\pgfpathlineto{\pgfqpoint{1.475583in}{0.742365in}}%
\pgfpathlineto{\pgfqpoint{1.476436in}{0.742263in}}%
\pgfpathlineto{\pgfqpoint{1.477289in}{0.742161in}}%
\pgfpathlineto{\pgfqpoint{1.478141in}{0.742058in}}%
\pgfpathlineto{\pgfqpoint{1.478994in}{0.741956in}}%
\pgfpathlineto{\pgfqpoint{1.479847in}{0.741854in}}%
\pgfpathlineto{\pgfqpoint{1.480700in}{0.741752in}}%
\pgfpathlineto{\pgfqpoint{1.481553in}{0.741650in}}%
\pgfpathlineto{\pgfqpoint{1.482406in}{0.741548in}}%
\pgfpathlineto{\pgfqpoint{1.483259in}{0.741446in}}%
\pgfpathlineto{\pgfqpoint{1.484112in}{0.741344in}}%
\pgfpathlineto{\pgfqpoint{1.484965in}{0.741241in}}%
\pgfpathlineto{\pgfqpoint{1.485817in}{0.741139in}}%
\pgfpathlineto{\pgfqpoint{1.486670in}{0.741037in}}%
\pgfpathlineto{\pgfqpoint{1.487523in}{0.740935in}}%
\pgfpathlineto{\pgfqpoint{1.488376in}{0.740833in}}%
\pgfpathlineto{\pgfqpoint{1.489229in}{0.740731in}}%
\pgfpathlineto{\pgfqpoint{1.490082in}{0.740629in}}%
\pgfpathlineto{\pgfqpoint{1.490935in}{0.740527in}}%
\pgfpathlineto{\pgfqpoint{1.491788in}{0.740424in}}%
\pgfpathlineto{\pgfqpoint{1.492640in}{0.740322in}}%
\pgfpathlineto{\pgfqpoint{1.493493in}{0.740220in}}%
\pgfpathlineto{\pgfqpoint{1.494346in}{0.740118in}}%
\pgfpathlineto{\pgfqpoint{1.495199in}{0.740016in}}%
\pgfpathlineto{\pgfqpoint{1.496052in}{0.739914in}}%
\pgfpathlineto{\pgfqpoint{1.496905in}{0.739812in}}%
\pgfpathlineto{\pgfqpoint{1.497758in}{0.739710in}}%
\pgfpathlineto{\pgfqpoint{1.498611in}{0.739607in}}%
\pgfpathlineto{\pgfqpoint{1.499463in}{0.739505in}}%
\pgfpathlineto{\pgfqpoint{1.500316in}{0.739403in}}%
\pgfpathlineto{\pgfqpoint{1.501169in}{0.739301in}}%
\pgfpathlineto{\pgfqpoint{1.502022in}{0.739199in}}%
\pgfpathlineto{\pgfqpoint{1.502875in}{0.739097in}}%
\pgfpathlineto{\pgfqpoint{1.503728in}{0.738995in}}%
\pgfpathlineto{\pgfqpoint{1.504581in}{0.738892in}}%
\pgfpathlineto{\pgfqpoint{1.505434in}{0.738790in}}%
\pgfpathlineto{\pgfqpoint{1.506286in}{0.738688in}}%
\pgfpathlineto{\pgfqpoint{1.507139in}{0.738586in}}%
\pgfpathlineto{\pgfqpoint{1.507992in}{0.738484in}}%
\pgfpathlineto{\pgfqpoint{1.508845in}{0.738382in}}%
\pgfpathlineto{\pgfqpoint{1.509698in}{0.738280in}}%
\pgfpathlineto{\pgfqpoint{1.510551in}{0.738178in}}%
\pgfpathlineto{\pgfqpoint{1.511404in}{0.738075in}}%
\pgfpathlineto{\pgfqpoint{1.512257in}{0.737973in}}%
\pgfpathlineto{\pgfqpoint{1.513110in}{0.737871in}}%
\pgfpathlineto{\pgfqpoint{1.513962in}{0.737769in}}%
\pgfpathlineto{\pgfqpoint{1.514815in}{0.737667in}}%
\pgfpathlineto{\pgfqpoint{1.515668in}{0.737565in}}%
\pgfpathlineto{\pgfqpoint{1.516521in}{0.737463in}}%
\pgfpathlineto{\pgfqpoint{1.517374in}{0.734440in}}%
\pgfpathlineto{\pgfqpoint{1.518227in}{0.726618in}}%
\pgfpathlineto{\pgfqpoint{1.519080in}{0.726021in}}%
\pgfpathlineto{\pgfqpoint{1.519933in}{0.725835in}}%
\pgfpathlineto{\pgfqpoint{1.520785in}{0.725803in}}%
\pgfpathlineto{\pgfqpoint{1.521638in}{0.725771in}}%
\pgfpathlineto{\pgfqpoint{1.522491in}{0.725865in}}%
\pgfpathlineto{\pgfqpoint{1.523344in}{0.726337in}}%
\pgfpathlineto{\pgfqpoint{1.524197in}{0.726834in}}%
\pgfpathlineto{\pgfqpoint{1.525050in}{0.727331in}}%
\pgfpathlineto{\pgfqpoint{1.525903in}{0.727828in}}%
\pgfpathlineto{\pgfqpoint{1.526756in}{0.728325in}}%
\pgfpathlineto{\pgfqpoint{1.527608in}{0.728822in}}%
\pgfpathlineto{\pgfqpoint{1.528461in}{0.729319in}}%
\pgfpathlineto{\pgfqpoint{1.529314in}{0.729816in}}%
\pgfpathlineto{\pgfqpoint{1.530167in}{0.730313in}}%
\pgfpathlineto{\pgfqpoint{1.531020in}{0.730809in}}%
\pgfpathlineto{\pgfqpoint{1.531873in}{0.731306in}}%
\pgfpathlineto{\pgfqpoint{1.532726in}{0.731803in}}%
\pgfpathlineto{\pgfqpoint{1.533579in}{0.732300in}}%
\pgfpathlineto{\pgfqpoint{1.534431in}{0.732797in}}%
\pgfpathlineto{\pgfqpoint{1.535284in}{0.733294in}}%
\pgfpathlineto{\pgfqpoint{1.536137in}{0.733791in}}%
\pgfpathlineto{\pgfqpoint{1.536990in}{0.734288in}}%
\pgfpathlineto{\pgfqpoint{1.537843in}{0.734668in}}%
\pgfpathlineto{\pgfqpoint{1.538696in}{0.734860in}}%
\pgfpathlineto{\pgfqpoint{1.539549in}{0.735051in}}%
\pgfpathlineto{\pgfqpoint{1.540402in}{0.735241in}}%
\pgfpathlineto{\pgfqpoint{1.541255in}{0.735432in}}%
\pgfpathlineto{\pgfqpoint{1.542107in}{0.735622in}}%
\pgfpathlineto{\pgfqpoint{1.542960in}{0.735813in}}%
\pgfpathlineto{\pgfqpoint{1.543813in}{0.736003in}}%
\pgfpathlineto{\pgfqpoint{1.544666in}{0.736194in}}%
\pgfpathlineto{\pgfqpoint{1.545519in}{0.736384in}}%
\pgfpathlineto{\pgfqpoint{1.546372in}{0.736575in}}%
\pgfpathlineto{\pgfqpoint{1.547225in}{0.736765in}}%
\pgfpathlineto{\pgfqpoint{1.548078in}{0.736956in}}%
\pgfpathlineto{\pgfqpoint{1.548930in}{0.737146in}}%
\pgfpathlineto{\pgfqpoint{1.549783in}{0.737337in}}%
\pgfpathlineto{\pgfqpoint{1.550636in}{0.737527in}}%
\pgfpathlineto{\pgfqpoint{1.551489in}{0.737717in}}%
\pgfpathlineto{\pgfqpoint{1.552342in}{0.737908in}}%
\pgfpathlineto{\pgfqpoint{1.553195in}{0.738098in}}%
\pgfpathlineto{\pgfqpoint{1.554048in}{0.738289in}}%
\pgfpathlineto{\pgfqpoint{1.554901in}{0.738479in}}%
\pgfpathlineto{\pgfqpoint{1.555753in}{0.738670in}}%
\pgfpathlineto{\pgfqpoint{1.556606in}{0.738860in}}%
\pgfpathlineto{\pgfqpoint{1.557459in}{0.738963in}}%
\pgfpathlineto{\pgfqpoint{1.558312in}{0.738941in}}%
\pgfpathlineto{\pgfqpoint{1.559165in}{0.738917in}}%
\pgfpathlineto{\pgfqpoint{1.560018in}{0.738894in}}%
\pgfpathlineto{\pgfqpoint{1.560871in}{0.738871in}}%
\pgfpathlineto{\pgfqpoint{1.561724in}{0.738848in}}%
\pgfpathlineto{\pgfqpoint{1.562576in}{0.738825in}}%
\pgfpathlineto{\pgfqpoint{1.563429in}{0.738802in}}%
\pgfpathlineto{\pgfqpoint{1.564282in}{0.738779in}}%
\pgfpathlineto{\pgfqpoint{1.565135in}{0.738756in}}%
\pgfpathlineto{\pgfqpoint{1.565988in}{0.738733in}}%
\pgfpathlineto{\pgfqpoint{1.566841in}{0.738710in}}%
\pgfpathlineto{\pgfqpoint{1.567694in}{0.738687in}}%
\pgfpathlineto{\pgfqpoint{1.568547in}{0.738663in}}%
\pgfpathlineto{\pgfqpoint{1.569399in}{0.738640in}}%
\pgfpathlineto{\pgfqpoint{1.570252in}{0.738617in}}%
\pgfpathlineto{\pgfqpoint{1.571105in}{0.738594in}}%
\pgfpathlineto{\pgfqpoint{1.571958in}{0.738571in}}%
\pgfpathlineto{\pgfqpoint{1.572811in}{0.738548in}}%
\pgfpathlineto{\pgfqpoint{1.573664in}{0.738525in}}%
\pgfpathlineto{\pgfqpoint{1.574517in}{0.738502in}}%
\pgfpathlineto{\pgfqpoint{1.575370in}{0.738479in}}%
\pgfpathlineto{\pgfqpoint{1.576223in}{0.738456in}}%
\pgfpathlineto{\pgfqpoint{1.577075in}{0.738433in}}%
\pgfpathlineto{\pgfqpoint{1.577928in}{0.738409in}}%
\pgfpathlineto{\pgfqpoint{1.578781in}{0.738386in}}%
\pgfpathlineto{\pgfqpoint{1.579634in}{0.738363in}}%
\pgfpathlineto{\pgfqpoint{1.580487in}{0.738340in}}%
\pgfpathlineto{\pgfqpoint{1.581340in}{0.738334in}}%
\pgfpathlineto{\pgfqpoint{1.582193in}{0.738345in}}%
\pgfpathlineto{\pgfqpoint{1.583046in}{0.738355in}}%
\pgfpathlineto{\pgfqpoint{1.583898in}{0.738361in}}%
\pgfpathlineto{\pgfqpoint{1.584751in}{0.738368in}}%
\pgfpathlineto{\pgfqpoint{1.585604in}{0.738374in}}%
\pgfpathlineto{\pgfqpoint{1.586457in}{0.738380in}}%
\pgfpathlineto{\pgfqpoint{1.587310in}{0.738387in}}%
\pgfpathlineto{\pgfqpoint{1.588163in}{0.738393in}}%
\pgfpathlineto{\pgfqpoint{1.589016in}{0.738399in}}%
\pgfpathlineto{\pgfqpoint{1.589869in}{0.738406in}}%
\pgfpathlineto{\pgfqpoint{1.590721in}{0.738412in}}%
\pgfpathlineto{\pgfqpoint{1.591574in}{0.738418in}}%
\pgfpathlineto{\pgfqpoint{1.592427in}{0.738425in}}%
\pgfpathlineto{\pgfqpoint{1.593280in}{0.738431in}}%
\pgfpathlineto{\pgfqpoint{1.594133in}{0.738437in}}%
\pgfpathlineto{\pgfqpoint{1.594986in}{0.738444in}}%
\pgfpathlineto{\pgfqpoint{1.595839in}{0.738450in}}%
\pgfpathlineto{\pgfqpoint{1.596692in}{0.738456in}}%
\pgfpathlineto{\pgfqpoint{1.597544in}{0.738463in}}%
\pgfpathlineto{\pgfqpoint{1.598397in}{0.738469in}}%
\pgfpathlineto{\pgfqpoint{1.599250in}{0.738475in}}%
\pgfpathlineto{\pgfqpoint{1.600103in}{0.738480in}}%
\pgfpathlineto{\pgfqpoint{1.600956in}{0.738463in}}%
\pgfpathlineto{\pgfqpoint{1.601809in}{0.738452in}}%
\pgfpathlineto{\pgfqpoint{1.602662in}{0.738481in}}%
\pgfpathlineto{\pgfqpoint{1.603515in}{0.738512in}}%
\pgfpathlineto{\pgfqpoint{1.604368in}{0.738543in}}%
\pgfpathlineto{\pgfqpoint{1.605220in}{0.738560in}}%
\pgfpathlineto{\pgfqpoint{1.606073in}{0.738554in}}%
\pgfpathlineto{\pgfqpoint{1.606926in}{0.738547in}}%
\pgfpathlineto{\pgfqpoint{1.607779in}{0.738540in}}%
\pgfpathlineto{\pgfqpoint{1.608632in}{0.738533in}}%
\pgfpathlineto{\pgfqpoint{1.609485in}{0.738526in}}%
\pgfpathlineto{\pgfqpoint{1.610338in}{0.738519in}}%
\pgfpathlineto{\pgfqpoint{1.611191in}{0.738512in}}%
\pgfpathlineto{\pgfqpoint{1.612043in}{0.738505in}}%
\pgfpathlineto{\pgfqpoint{1.612896in}{0.738499in}}%
\pgfpathlineto{\pgfqpoint{1.613749in}{0.738492in}}%
\pgfpathlineto{\pgfqpoint{1.614602in}{0.738485in}}%
\pgfpathlineto{\pgfqpoint{1.615455in}{0.738478in}}%
\pgfpathlineto{\pgfqpoint{1.616308in}{0.738471in}}%
\pgfpathlineto{\pgfqpoint{1.617161in}{0.738464in}}%
\pgfpathlineto{\pgfqpoint{1.618014in}{0.738457in}}%
\pgfpathlineto{\pgfqpoint{1.618866in}{0.738450in}}%
\pgfpathlineto{\pgfqpoint{1.619719in}{0.738443in}}%
\pgfpathlineto{\pgfqpoint{1.620572in}{0.738437in}}%
\pgfpathlineto{\pgfqpoint{1.621425in}{0.738430in}}%
\pgfpathlineto{\pgfqpoint{1.622278in}{0.738423in}}%
\pgfpathlineto{\pgfqpoint{1.623131in}{0.738416in}}%
\pgfpathlineto{\pgfqpoint{1.623984in}{0.738409in}}%
\pgfpathlineto{\pgfqpoint{1.624837in}{0.738402in}}%
\pgfpathlineto{\pgfqpoint{1.625689in}{0.738395in}}%
\pgfpathlineto{\pgfqpoint{1.626542in}{0.738388in}}%
\pgfpathlineto{\pgfqpoint{1.627395in}{0.738381in}}%
\pgfpathlineto{\pgfqpoint{1.628248in}{0.738375in}}%
\pgfpathlineto{\pgfqpoint{1.629101in}{0.738368in}}%
\pgfpathlineto{\pgfqpoint{1.629954in}{0.738361in}}%
\pgfpathlineto{\pgfqpoint{1.630807in}{0.738354in}}%
\pgfpathlineto{\pgfqpoint{1.631660in}{0.738347in}}%
\pgfpathlineto{\pgfqpoint{1.632513in}{0.738340in}}%
\pgfpathlineto{\pgfqpoint{1.633365in}{0.738333in}}%
\pgfpathlineto{\pgfqpoint{1.634218in}{0.738326in}}%
\pgfpathlineto{\pgfqpoint{1.635071in}{0.738320in}}%
\pgfpathlineto{\pgfqpoint{1.635924in}{0.738313in}}%
\pgfpathlineto{\pgfqpoint{1.636777in}{0.738306in}}%
\pgfpathlineto{\pgfqpoint{1.637630in}{0.738299in}}%
\pgfpathlineto{\pgfqpoint{1.638483in}{0.738292in}}%
\pgfpathlineto{\pgfqpoint{1.639336in}{0.738285in}}%
\pgfpathlineto{\pgfqpoint{1.640188in}{0.738278in}}%
\pgfpathlineto{\pgfqpoint{1.641041in}{0.738271in}}%
\pgfpathlineto{\pgfqpoint{1.641894in}{0.738264in}}%
\pgfpathlineto{\pgfqpoint{1.642747in}{0.738258in}}%
\pgfpathlineto{\pgfqpoint{1.643600in}{0.738251in}}%
\pgfpathlineto{\pgfqpoint{1.644453in}{0.738244in}}%
\pgfpathlineto{\pgfqpoint{1.645306in}{0.738237in}}%
\pgfpathlineto{\pgfqpoint{1.646159in}{0.738230in}}%
\pgfpathlineto{\pgfqpoint{1.647011in}{0.738223in}}%
\pgfpathlineto{\pgfqpoint{1.647864in}{0.738216in}}%
\pgfpathlineto{\pgfqpoint{1.648717in}{0.738209in}}%
\pgfpathlineto{\pgfqpoint{1.649570in}{0.738202in}}%
\pgfpathlineto{\pgfqpoint{1.650423in}{0.738196in}}%
\pgfpathlineto{\pgfqpoint{1.651276in}{0.738189in}}%
\pgfpathlineto{\pgfqpoint{1.652129in}{0.738182in}}%
\pgfpathlineto{\pgfqpoint{1.652982in}{0.738175in}}%
\pgfpathlineto{\pgfqpoint{1.653834in}{0.738168in}}%
\pgfpathlineto{\pgfqpoint{1.654687in}{0.738161in}}%
\pgfpathlineto{\pgfqpoint{1.655540in}{0.738154in}}%
\pgfpathlineto{\pgfqpoint{1.656393in}{0.738147in}}%
\pgfpathlineto{\pgfqpoint{1.657246in}{0.738141in}}%
\pgfpathlineto{\pgfqpoint{1.658099in}{0.738134in}}%
\pgfpathlineto{\pgfqpoint{1.658952in}{0.738127in}}%
\pgfpathlineto{\pgfqpoint{1.659805in}{0.739262in}}%
\pgfpathlineto{\pgfqpoint{1.660657in}{0.742441in}}%
\pgfpathlineto{\pgfqpoint{1.661510in}{0.745661in}}%
\pgfpathlineto{\pgfqpoint{1.662363in}{0.748880in}}%
\pgfpathlineto{\pgfqpoint{1.663216in}{0.752099in}}%
\pgfpathlineto{\pgfqpoint{1.664069in}{0.755318in}}%
\pgfpathlineto{\pgfqpoint{1.664922in}{0.758538in}}%
\pgfpathlineto{\pgfqpoint{1.665775in}{0.761757in}}%
\pgfpathlineto{\pgfqpoint{1.666628in}{0.764976in}}%
\pgfpathlineto{\pgfqpoint{1.667481in}{0.768196in}}%
\pgfpathlineto{\pgfqpoint{1.668333in}{0.771415in}}%
\pgfpathlineto{\pgfqpoint{1.669186in}{0.774634in}}%
\pgfpathlineto{\pgfqpoint{1.670039in}{0.777853in}}%
\pgfpathlineto{\pgfqpoint{1.670892in}{0.781073in}}%
\pgfpathlineto{\pgfqpoint{1.671745in}{0.784292in}}%
\pgfpathlineto{\pgfqpoint{1.672598in}{0.787511in}}%
\pgfpathlineto{\pgfqpoint{1.673451in}{0.790730in}}%
\pgfpathlineto{\pgfqpoint{1.674304in}{0.793950in}}%
\pgfpathlineto{\pgfqpoint{1.675156in}{0.797169in}}%
\pgfpathlineto{\pgfqpoint{1.676009in}{0.800388in}}%
\pgfpathlineto{\pgfqpoint{1.676862in}{0.803607in}}%
\pgfpathlineto{\pgfqpoint{1.677715in}{0.806827in}}%
\pgfpathlineto{\pgfqpoint{1.678568in}{0.810046in}}%
\pgfpathlineto{\pgfqpoint{1.679421in}{0.813265in}}%
\pgfpathlineto{\pgfqpoint{1.680274in}{0.816484in}}%
\pgfpathlineto{\pgfqpoint{1.681127in}{0.819704in}}%
\pgfpathlineto{\pgfqpoint{1.681979in}{0.822923in}}%
\pgfpathlineto{\pgfqpoint{1.682832in}{0.803690in}}%
\pgfpathlineto{\pgfqpoint{1.683685in}{0.737831in}}%
\pgfpathlineto{\pgfqpoint{1.684538in}{0.737856in}}%
\pgfpathlineto{\pgfqpoint{1.685391in}{0.737881in}}%
\pgfpathlineto{\pgfqpoint{1.686244in}{0.798615in}}%
\pgfpathlineto{\pgfqpoint{1.687097in}{0.826767in}}%
\pgfpathlineto{\pgfqpoint{1.687950in}{0.826784in}}%
\pgfpathlineto{\pgfqpoint{1.688802in}{0.826801in}}%
\pgfpathlineto{\pgfqpoint{1.689655in}{0.826819in}}%
\pgfpathlineto{\pgfqpoint{1.690508in}{0.826836in}}%
\pgfpathlineto{\pgfqpoint{1.691361in}{0.826853in}}%
\pgfpathlineto{\pgfqpoint{1.692214in}{0.826868in}}%
\pgfpathlineto{\pgfqpoint{1.693067in}{0.826877in}}%
\pgfpathlineto{\pgfqpoint{1.693920in}{0.826887in}}%
\pgfpathlineto{\pgfqpoint{1.694773in}{0.826896in}}%
\pgfpathlineto{\pgfqpoint{1.695626in}{0.826905in}}%
\pgfpathlineto{\pgfqpoint{1.696478in}{0.826915in}}%
\pgfpathlineto{\pgfqpoint{1.697331in}{0.826924in}}%
\pgfpathlineto{\pgfqpoint{1.698184in}{0.826933in}}%
\pgfpathlineto{\pgfqpoint{1.699037in}{0.826943in}}%
\pgfpathlineto{\pgfqpoint{1.699890in}{0.826952in}}%
\pgfpathlineto{\pgfqpoint{1.700743in}{0.826961in}}%
\pgfpathlineto{\pgfqpoint{1.701596in}{0.826971in}}%
\pgfpathlineto{\pgfqpoint{1.702449in}{0.826980in}}%
\pgfpathlineto{\pgfqpoint{1.703301in}{0.826989in}}%
\pgfpathlineto{\pgfqpoint{1.704154in}{0.826999in}}%
\pgfpathlineto{\pgfqpoint{1.705007in}{0.827002in}}%
\pgfpathlineto{\pgfqpoint{1.705860in}{0.826989in}}%
\pgfpathlineto{\pgfqpoint{1.706713in}{0.826975in}}%
\pgfpathlineto{\pgfqpoint{1.707566in}{0.826960in}}%
\pgfpathlineto{\pgfqpoint{1.708419in}{0.826946in}}%
\pgfpathlineto{\pgfqpoint{1.709272in}{0.826931in}}%
\pgfpathlineto{\pgfqpoint{1.710124in}{0.826917in}}%
\pgfpathlineto{\pgfqpoint{1.710977in}{0.826903in}}%
\pgfpathlineto{\pgfqpoint{1.711830in}{0.826679in}}%
\pgfpathlineto{\pgfqpoint{1.712683in}{0.820866in}}%
\pgfpathlineto{\pgfqpoint{1.713536in}{0.812566in}}%
\pgfpathlineto{\pgfqpoint{1.714389in}{0.804265in}}%
\pgfpathlineto{\pgfqpoint{1.715242in}{0.795964in}}%
\pgfpathlineto{\pgfqpoint{1.716095in}{0.787664in}}%
\pgfpathlineto{\pgfqpoint{1.716947in}{0.779363in}}%
\pgfpathlineto{\pgfqpoint{1.717800in}{0.771062in}}%
\pgfpathlineto{\pgfqpoint{1.718653in}{0.762762in}}%
\pgfpathlineto{\pgfqpoint{1.719506in}{0.754461in}}%
\pgfpathlineto{\pgfqpoint{1.720359in}{0.746160in}}%
\pgfpathlineto{\pgfqpoint{1.721212in}{0.740479in}}%
\pgfpathlineto{\pgfqpoint{1.722065in}{0.751060in}}%
\pgfpathlineto{\pgfqpoint{1.722918in}{0.764441in}}%
\pgfpathlineto{\pgfqpoint{1.723771in}{0.777821in}}%
\pgfpathlineto{\pgfqpoint{1.724623in}{0.791202in}}%
\pgfpathlineto{\pgfqpoint{1.725476in}{0.804583in}}%
\pgfpathlineto{\pgfqpoint{1.726329in}{0.817964in}}%
\pgfpathlineto{\pgfqpoint{1.727182in}{0.826610in}}%
\pgfpathlineto{\pgfqpoint{1.728035in}{0.826778in}}%
\pgfpathlineto{\pgfqpoint{1.728888in}{0.826768in}}%
\pgfpathlineto{\pgfqpoint{1.729741in}{0.826758in}}%
\pgfpathlineto{\pgfqpoint{1.730594in}{0.826747in}}%
\pgfpathlineto{\pgfqpoint{1.731446in}{0.826737in}}%
\pgfpathlineto{\pgfqpoint{1.732299in}{0.826727in}}%
\pgfpathlineto{\pgfqpoint{1.733152in}{0.826717in}}%
\pgfpathlineto{\pgfqpoint{1.734005in}{0.826707in}}%
\pgfpathlineto{\pgfqpoint{1.734858in}{0.826697in}}%
\pgfpathlineto{\pgfqpoint{1.735711in}{0.826687in}}%
\pgfpathlineto{\pgfqpoint{1.736564in}{0.826677in}}%
\pgfpathlineto{\pgfqpoint{1.737417in}{0.826667in}}%
\pgfpathlineto{\pgfqpoint{1.738269in}{0.826657in}}%
\pgfpathlineto{\pgfqpoint{1.739122in}{0.826647in}}%
\pgfpathlineto{\pgfqpoint{1.739975in}{0.826637in}}%
\pgfpathlineto{\pgfqpoint{1.740828in}{0.826626in}}%
\pgfpathlineto{\pgfqpoint{1.741681in}{0.826616in}}%
\pgfpathlineto{\pgfqpoint{1.742534in}{0.826607in}}%
\pgfpathlineto{\pgfqpoint{1.743387in}{0.826601in}}%
\pgfpathlineto{\pgfqpoint{1.744240in}{0.826595in}}%
\pgfpathlineto{\pgfqpoint{1.745092in}{0.826590in}}%
\pgfpathlineto{\pgfqpoint{1.745945in}{0.826584in}}%
\pgfpathlineto{\pgfqpoint{1.746798in}{0.826480in}}%
\pgfpathlineto{\pgfqpoint{1.747651in}{0.826061in}}%
\pgfpathlineto{\pgfqpoint{1.748504in}{0.825771in}}%
\pgfpathlineto{\pgfqpoint{1.749357in}{0.826256in}}%
\pgfpathlineto{\pgfqpoint{1.750210in}{0.826571in}}%
\pgfpathlineto{\pgfqpoint{1.751063in}{0.826541in}}%
\pgfpathlineto{\pgfqpoint{1.751916in}{0.826522in}}%
\pgfpathlineto{\pgfqpoint{1.752768in}{0.826540in}}%
\pgfpathlineto{\pgfqpoint{1.753621in}{0.826529in}}%
\pgfpathlineto{\pgfqpoint{1.754474in}{0.826518in}}%
\pgfpathlineto{\pgfqpoint{1.755327in}{0.826508in}}%
\pgfpathlineto{\pgfqpoint{1.756180in}{0.826497in}}%
\pgfpathlineto{\pgfqpoint{1.757033in}{0.826486in}}%
\pgfpathlineto{\pgfqpoint{1.757886in}{0.826475in}}%
\pgfpathlineto{\pgfqpoint{1.758739in}{0.826464in}}%
\pgfpathlineto{\pgfqpoint{1.759591in}{0.826454in}}%
\pgfpathlineto{\pgfqpoint{1.760444in}{0.826443in}}%
\pgfpathlineto{\pgfqpoint{1.761297in}{0.826432in}}%
\pgfpathlineto{\pgfqpoint{1.762150in}{0.826421in}}%
\pgfpathlineto{\pgfqpoint{1.763003in}{0.826411in}}%
\pgfpathlineto{\pgfqpoint{1.763856in}{0.826400in}}%
\pgfpathlineto{\pgfqpoint{1.764709in}{0.826453in}}%
\pgfpathlineto{\pgfqpoint{1.765562in}{0.826494in}}%
\pgfpathlineto{\pgfqpoint{1.766414in}{0.826485in}}%
\pgfpathlineto{\pgfqpoint{1.767267in}{0.826494in}}%
\pgfpathlineto{\pgfqpoint{1.768120in}{0.807629in}}%
\pgfpathlineto{\pgfqpoint{1.768973in}{0.751539in}}%
\pgfpathlineto{\pgfqpoint{1.769826in}{0.737131in}}%
\pgfpathlineto{\pgfqpoint{1.770679in}{0.737122in}}%
\pgfpathlineto{\pgfqpoint{1.771532in}{0.737113in}}%
\pgfpathlineto{\pgfqpoint{1.772385in}{0.737103in}}%
\pgfpathlineto{\pgfqpoint{1.773237in}{0.737094in}}%
\pgfpathlineto{\pgfqpoint{1.774090in}{0.737084in}}%
\pgfpathlineto{\pgfqpoint{1.774943in}{0.737075in}}%
\pgfpathlineto{\pgfqpoint{1.775796in}{0.737065in}}%
\pgfpathlineto{\pgfqpoint{1.776649in}{0.737056in}}%
\pgfpathlineto{\pgfqpoint{1.777502in}{0.737047in}}%
\pgfpathlineto{\pgfqpoint{1.778355in}{0.737037in}}%
\pgfpathlineto{\pgfqpoint{1.779208in}{0.737028in}}%
\pgfpathlineto{\pgfqpoint{1.780060in}{0.737018in}}%
\pgfpathlineto{\pgfqpoint{1.780913in}{0.737009in}}%
\pgfpathlineto{\pgfqpoint{1.781766in}{0.736999in}}%
\pgfpathlineto{\pgfqpoint{1.782619in}{0.736990in}}%
\pgfpathlineto{\pgfqpoint{1.783472in}{0.736981in}}%
\pgfpathlineto{\pgfqpoint{1.784325in}{0.736971in}}%
\pgfpathlineto{\pgfqpoint{1.785178in}{0.736962in}}%
\pgfpathlineto{\pgfqpoint{1.786031in}{0.736952in}}%
\pgfpathlineto{\pgfqpoint{1.786884in}{0.736944in}}%
\pgfpathlineto{\pgfqpoint{1.787736in}{0.736985in}}%
\pgfpathlineto{\pgfqpoint{1.788589in}{0.739186in}}%
\pgfpathlineto{\pgfqpoint{1.789442in}{0.818099in}}%
\pgfpathlineto{\pgfqpoint{1.790295in}{0.826175in}}%
\pgfpathlineto{\pgfqpoint{1.791148in}{0.826168in}}%
\pgfpathlineto{\pgfqpoint{1.792001in}{0.826155in}}%
\pgfpathlineto{\pgfqpoint{1.792854in}{0.826141in}}%
\pgfpathlineto{\pgfqpoint{1.793707in}{0.826127in}}%
\pgfpathlineto{\pgfqpoint{1.794559in}{0.826113in}}%
\pgfpathlineto{\pgfqpoint{1.795412in}{0.826099in}}%
\pgfpathlineto{\pgfqpoint{1.796265in}{0.826086in}}%
\pgfpathlineto{\pgfqpoint{1.797118in}{0.826072in}}%
\pgfpathlineto{\pgfqpoint{1.797971in}{0.826058in}}%
\pgfpathlineto{\pgfqpoint{1.798824in}{0.826044in}}%
\pgfpathlineto{\pgfqpoint{1.799677in}{0.826030in}}%
\pgfpathlineto{\pgfqpoint{1.800530in}{0.826016in}}%
\pgfpathlineto{\pgfqpoint{1.801382in}{0.826002in}}%
\pgfpathlineto{\pgfqpoint{1.802235in}{0.825988in}}%
\pgfpathlineto{\pgfqpoint{1.803088in}{0.825974in}}%
\pgfpathlineto{\pgfqpoint{1.803941in}{0.825960in}}%
\pgfpathlineto{\pgfqpoint{1.804794in}{0.825946in}}%
\pgfpathlineto{\pgfqpoint{1.805647in}{0.825932in}}%
\pgfpathlineto{\pgfqpoint{1.806500in}{0.825918in}}%
\pgfpathlineto{\pgfqpoint{1.807353in}{0.825905in}}%
\pgfpathlineto{\pgfqpoint{1.808205in}{0.825891in}}%
\pgfpathlineto{\pgfqpoint{1.809058in}{0.825877in}}%
\pgfpathlineto{\pgfqpoint{1.809911in}{0.825863in}}%
\pgfpathlineto{\pgfqpoint{1.810764in}{0.825849in}}%
\pgfpathlineto{\pgfqpoint{1.811617in}{0.825835in}}%
\pgfpathlineto{\pgfqpoint{1.812470in}{0.825821in}}%
\pgfpathlineto{\pgfqpoint{1.813323in}{0.825807in}}%
\pgfpathlineto{\pgfqpoint{1.814176in}{0.825793in}}%
\pgfpathlineto{\pgfqpoint{1.815029in}{0.824864in}}%
\pgfpathlineto{\pgfqpoint{1.815881in}{0.814374in}}%
\pgfpathlineto{\pgfqpoint{1.816734in}{0.801341in}}%
\pgfpathlineto{\pgfqpoint{1.817587in}{0.788308in}}%
\pgfpathlineto{\pgfqpoint{1.818440in}{0.775275in}}%
\pgfpathlineto{\pgfqpoint{1.819293in}{0.762242in}}%
\pgfpathlineto{\pgfqpoint{1.820146in}{0.749209in}}%
\pgfpathlineto{\pgfqpoint{1.820999in}{0.741619in}}%
\pgfpathlineto{\pgfqpoint{1.821852in}{0.760114in}}%
\pgfpathlineto{\pgfqpoint{1.822704in}{0.782016in}}%
\pgfpathlineto{\pgfqpoint{1.823557in}{0.803917in}}%
\pgfpathlineto{\pgfqpoint{1.824410in}{0.814746in}}%
\pgfpathlineto{\pgfqpoint{1.825263in}{0.830541in}}%
\pgfpathlineto{\pgfqpoint{1.826116in}{0.830408in}}%
\pgfpathlineto{\pgfqpoint{1.826969in}{0.829837in}}%
\pgfpathlineto{\pgfqpoint{1.827822in}{0.829837in}}%
\pgfpathlineto{\pgfqpoint{1.828675in}{0.829804in}}%
\pgfpathlineto{\pgfqpoint{1.829527in}{0.830384in}}%
\pgfpathlineto{\pgfqpoint{1.830380in}{0.830424in}}%
\pgfpathlineto{\pgfqpoint{1.831233in}{0.830014in}}%
\pgfpathlineto{\pgfqpoint{1.832086in}{0.829515in}}%
\pgfpathlineto{\pgfqpoint{1.832939in}{0.829734in}}%
\pgfpathlineto{\pgfqpoint{1.833792in}{0.816327in}}%
\pgfpathlineto{\pgfqpoint{1.834645in}{0.830544in}}%
\pgfpathlineto{\pgfqpoint{1.835498in}{0.830430in}}%
\pgfpathlineto{\pgfqpoint{1.836350in}{0.830287in}}%
\pgfpathlineto{\pgfqpoint{1.837203in}{0.830144in}}%
\pgfpathlineto{\pgfqpoint{1.838056in}{0.830001in}}%
\pgfpathlineto{\pgfqpoint{1.838909in}{0.829858in}}%
\pgfpathlineto{\pgfqpoint{1.839762in}{0.829715in}}%
\pgfpathlineto{\pgfqpoint{1.840615in}{0.829572in}}%
\pgfpathlineto{\pgfqpoint{1.841468in}{0.829429in}}%
\pgfpathlineto{\pgfqpoint{1.842321in}{0.829286in}}%
\pgfpathlineto{\pgfqpoint{1.843174in}{0.829183in}}%
\pgfpathlineto{\pgfqpoint{1.844026in}{0.829997in}}%
\pgfpathlineto{\pgfqpoint{1.844879in}{0.830620in}}%
\pgfpathlineto{\pgfqpoint{1.845732in}{0.830572in}}%
\pgfpathlineto{\pgfqpoint{1.846585in}{0.829481in}}%
\pgfpathlineto{\pgfqpoint{1.847438in}{0.830083in}}%
\pgfpathlineto{\pgfqpoint{1.848291in}{0.831183in}}%
\pgfpathlineto{\pgfqpoint{1.849144in}{0.830716in}}%
\pgfpathlineto{\pgfqpoint{1.849997in}{0.831153in}}%
\pgfpathlineto{\pgfqpoint{1.850849in}{0.831180in}}%
\pgfpathlineto{\pgfqpoint{1.851702in}{0.831166in}}%
\pgfpathlineto{\pgfqpoint{1.852555in}{0.831153in}}%
\pgfpathlineto{\pgfqpoint{1.853408in}{0.831140in}}%
\pgfpathlineto{\pgfqpoint{1.854261in}{0.831127in}}%
\pgfpathlineto{\pgfqpoint{1.855114in}{0.831147in}}%
\pgfpathlineto{\pgfqpoint{1.855967in}{0.830068in}}%
\pgfpathlineto{\pgfqpoint{1.856820in}{0.831051in}}%
\pgfpathlineto{\pgfqpoint{1.857672in}{0.830746in}}%
\pgfpathlineto{\pgfqpoint{1.858525in}{0.830583in}}%
\pgfpathlineto{\pgfqpoint{1.859378in}{0.830250in}}%
\pgfpathlineto{\pgfqpoint{1.860231in}{0.829914in}}%
\pgfpathlineto{\pgfqpoint{1.861084in}{0.829578in}}%
\pgfpathlineto{\pgfqpoint{1.861937in}{0.829260in}}%
\pgfpathlineto{\pgfqpoint{1.862790in}{0.829202in}}%
\pgfpathlineto{\pgfqpoint{1.863643in}{0.830207in}}%
\pgfpathlineto{\pgfqpoint{1.864495in}{0.830704in}}%
\pgfpathlineto{\pgfqpoint{1.865348in}{0.830794in}}%
\pgfpathlineto{\pgfqpoint{1.866201in}{0.830248in}}%
\pgfpathlineto{\pgfqpoint{1.867054in}{0.830956in}}%
\pgfpathlineto{\pgfqpoint{1.867907in}{0.832868in}}%
\pgfpathlineto{\pgfqpoint{1.868760in}{0.832615in}}%
\pgfpathlineto{\pgfqpoint{1.869613in}{0.831413in}}%
\pgfpathlineto{\pgfqpoint{1.870466in}{0.832634in}}%
\pgfpathlineto{\pgfqpoint{1.871318in}{0.833181in}}%
\pgfpathlineto{\pgfqpoint{1.872171in}{0.833215in}}%
\pgfpathlineto{\pgfqpoint{1.873024in}{0.833248in}}%
\pgfpathlineto{\pgfqpoint{1.873877in}{0.833282in}}%
\pgfpathlineto{\pgfqpoint{1.874730in}{0.833316in}}%
\pgfpathlineto{\pgfqpoint{1.875583in}{0.833336in}}%
\pgfpathlineto{\pgfqpoint{1.876436in}{0.833268in}}%
\pgfpathlineto{\pgfqpoint{1.877289in}{0.833183in}}%
\pgfpathlineto{\pgfqpoint{1.878142in}{0.833098in}}%
\pgfpathlineto{\pgfqpoint{1.878994in}{0.833013in}}%
\pgfpathlineto{\pgfqpoint{1.879847in}{0.832929in}}%
\pgfpathlineto{\pgfqpoint{1.880700in}{0.832844in}}%
\pgfpathlineto{\pgfqpoint{1.881553in}{0.832759in}}%
\pgfpathlineto{\pgfqpoint{1.882406in}{0.832674in}}%
\pgfpathlineto{\pgfqpoint{1.883259in}{0.832633in}}%
\pgfpathlineto{\pgfqpoint{1.884112in}{0.832781in}}%
\pgfpathlineto{\pgfqpoint{1.884965in}{0.832870in}}%
\pgfpathlineto{\pgfqpoint{1.885817in}{0.832904in}}%
\pgfpathlineto{\pgfqpoint{1.886670in}{0.833026in}}%
\pgfpathlineto{\pgfqpoint{1.887523in}{0.832803in}}%
\pgfpathlineto{\pgfqpoint{1.888376in}{0.832608in}}%
\pgfpathlineto{\pgfqpoint{1.889229in}{0.832705in}}%
\pgfpathlineto{\pgfqpoint{1.890082in}{0.832880in}}%
\pgfpathlineto{\pgfqpoint{1.890935in}{0.833057in}}%
\pgfpathlineto{\pgfqpoint{1.891788in}{0.833150in}}%
\pgfpathlineto{\pgfqpoint{1.892640in}{0.833145in}}%
\pgfpathlineto{\pgfqpoint{1.893493in}{0.833140in}}%
\pgfpathlineto{\pgfqpoint{1.894346in}{0.833135in}}%
\pgfpathlineto{\pgfqpoint{1.895199in}{0.833130in}}%
\pgfpathlineto{\pgfqpoint{1.896052in}{0.833125in}}%
\pgfpathlineto{\pgfqpoint{1.896905in}{0.833120in}}%
\pgfpathlineto{\pgfqpoint{1.897758in}{0.833115in}}%
\pgfpathlineto{\pgfqpoint{1.898611in}{0.833111in}}%
\pgfpathlineto{\pgfqpoint{1.899463in}{0.833106in}}%
\pgfpathlineto{\pgfqpoint{1.900316in}{0.833101in}}%
\pgfpathlineto{\pgfqpoint{1.901169in}{0.833096in}}%
\pgfpathlineto{\pgfqpoint{1.902022in}{0.833091in}}%
\pgfpathlineto{\pgfqpoint{1.902875in}{0.833086in}}%
\pgfpathlineto{\pgfqpoint{1.903728in}{0.833081in}}%
\pgfpathlineto{\pgfqpoint{1.904581in}{0.833076in}}%
\pgfpathlineto{\pgfqpoint{1.905434in}{0.833071in}}%
\pgfpathlineto{\pgfqpoint{1.906287in}{0.833066in}}%
\pgfpathlineto{\pgfqpoint{1.907139in}{0.833061in}}%
\pgfpathlineto{\pgfqpoint{1.907992in}{0.833079in}}%
\pgfpathlineto{\pgfqpoint{1.908845in}{0.833063in}}%
\pgfpathlineto{\pgfqpoint{1.909698in}{0.833032in}}%
\pgfpathlineto{\pgfqpoint{1.910551in}{0.833000in}}%
\pgfpathlineto{\pgfqpoint{1.911404in}{0.832969in}}%
\pgfpathlineto{\pgfqpoint{1.912257in}{0.832938in}}%
\pgfpathlineto{\pgfqpoint{1.913110in}{0.832906in}}%
\pgfpathlineto{\pgfqpoint{1.913962in}{0.832875in}}%
\pgfpathlineto{\pgfqpoint{1.914815in}{0.832844in}}%
\pgfpathlineto{\pgfqpoint{1.915668in}{0.832812in}}%
\pgfpathlineto{\pgfqpoint{1.916521in}{0.832781in}}%
\pgfpathlineto{\pgfqpoint{1.917374in}{0.832750in}}%
\pgfpathlineto{\pgfqpoint{1.918227in}{0.832718in}}%
\pgfpathlineto{\pgfqpoint{1.919080in}{0.832687in}}%
\pgfpathlineto{\pgfqpoint{1.919933in}{0.832656in}}%
\pgfpathlineto{\pgfqpoint{1.920785in}{0.832625in}}%
\pgfpathlineto{\pgfqpoint{1.921638in}{0.832593in}}%
\pgfpathlineto{\pgfqpoint{1.922491in}{0.832562in}}%
\pgfpathlineto{\pgfqpoint{1.923344in}{0.832531in}}%
\pgfpathlineto{\pgfqpoint{1.924197in}{0.832499in}}%
\pgfpathlineto{\pgfqpoint{1.925050in}{0.832468in}}%
\pgfpathlineto{\pgfqpoint{1.925903in}{0.832437in}}%
\pgfpathlineto{\pgfqpoint{1.926756in}{0.832405in}}%
\pgfpathlineto{\pgfqpoint{1.927608in}{0.832374in}}%
\pgfpathlineto{\pgfqpoint{1.928461in}{0.832343in}}%
\pgfpathlineto{\pgfqpoint{1.929314in}{0.832312in}}%
\pgfpathlineto{\pgfqpoint{1.930167in}{0.832280in}}%
\pgfpathlineto{\pgfqpoint{1.931020in}{0.832249in}}%
\pgfpathlineto{\pgfqpoint{1.931873in}{0.832218in}}%
\pgfpathlineto{\pgfqpoint{1.932726in}{0.832186in}}%
\pgfpathlineto{\pgfqpoint{1.933579in}{0.832155in}}%
\pgfpathlineto{\pgfqpoint{1.934432in}{0.832124in}}%
\pgfpathlineto{\pgfqpoint{1.935284in}{0.832092in}}%
\pgfpathlineto{\pgfqpoint{1.936137in}{0.832061in}}%
\pgfpathlineto{\pgfqpoint{1.936990in}{0.832030in}}%
\pgfpathlineto{\pgfqpoint{1.937843in}{0.831998in}}%
\pgfpathlineto{\pgfqpoint{1.938696in}{0.831967in}}%
\pgfpathlineto{\pgfqpoint{1.939549in}{0.831936in}}%
\pgfpathlineto{\pgfqpoint{1.940402in}{0.831905in}}%
\pgfpathlineto{\pgfqpoint{1.941255in}{0.831873in}}%
\pgfpathlineto{\pgfqpoint{1.942107in}{0.831842in}}%
\pgfpathlineto{\pgfqpoint{1.942960in}{0.831811in}}%
\pgfpathlineto{\pgfqpoint{1.943813in}{0.831779in}}%
\pgfpathlineto{\pgfqpoint{1.944666in}{0.831748in}}%
\pgfpathlineto{\pgfqpoint{1.945519in}{0.831717in}}%
\pgfpathlineto{\pgfqpoint{1.946372in}{0.831685in}}%
\pgfpathlineto{\pgfqpoint{1.947225in}{0.831655in}}%
\pgfpathlineto{\pgfqpoint{1.948078in}{0.831856in}}%
\pgfpathlineto{\pgfqpoint{1.948930in}{0.832277in}}%
\pgfpathlineto{\pgfqpoint{1.949783in}{0.832261in}}%
\pgfpathlineto{\pgfqpoint{1.950636in}{0.832244in}}%
\pgfpathlineto{\pgfqpoint{1.951489in}{0.832229in}}%
\pgfpathlineto{\pgfqpoint{1.952342in}{0.832216in}}%
\pgfpathlineto{\pgfqpoint{1.953195in}{0.832203in}}%
\pgfpathlineto{\pgfqpoint{1.954048in}{0.832190in}}%
\pgfpathlineto{\pgfqpoint{1.954901in}{0.832154in}}%
\pgfpathlineto{\pgfqpoint{1.955753in}{0.832076in}}%
\pgfpathlineto{\pgfqpoint{1.956606in}{0.831998in}}%
\pgfpathlineto{\pgfqpoint{1.957459in}{0.831918in}}%
\pgfpathlineto{\pgfqpoint{1.958312in}{0.831839in}}%
\pgfpathlineto{\pgfqpoint{1.959165in}{0.831760in}}%
\pgfpathlineto{\pgfqpoint{1.960018in}{0.831681in}}%
\pgfpathlineto{\pgfqpoint{1.960871in}{0.831602in}}%
\pgfpathlineto{\pgfqpoint{1.961724in}{0.831523in}}%
\pgfpathlineto{\pgfqpoint{1.962577in}{0.831444in}}%
\pgfpathlineto{\pgfqpoint{1.963429in}{0.831365in}}%
\pgfpathlineto{\pgfqpoint{1.964282in}{0.831286in}}%
\pgfpathlineto{\pgfqpoint{1.965135in}{0.831207in}}%
\pgfpathlineto{\pgfqpoint{1.965988in}{0.831128in}}%
\pgfpathlineto{\pgfqpoint{1.966841in}{0.831160in}}%
\pgfpathlineto{\pgfqpoint{1.967694in}{0.831229in}}%
\pgfpathlineto{\pgfqpoint{1.968547in}{0.831187in}}%
\pgfpathlineto{\pgfqpoint{1.969400in}{0.831143in}}%
\pgfpathlineto{\pgfqpoint{1.970252in}{0.831079in}}%
\pgfpathlineto{\pgfqpoint{1.971105in}{0.831005in}}%
\pgfpathlineto{\pgfqpoint{1.971958in}{0.830927in}}%
\pgfpathlineto{\pgfqpoint{1.972811in}{0.830850in}}%
\pgfpathlineto{\pgfqpoint{1.973664in}{0.830773in}}%
\pgfpathlineto{\pgfqpoint{1.974517in}{0.830696in}}%
\pgfpathlineto{\pgfqpoint{1.975370in}{0.830619in}}%
\pgfpathlineto{\pgfqpoint{1.976223in}{0.830542in}}%
\pgfpathlineto{\pgfqpoint{1.977075in}{0.830465in}}%
\pgfpathlineto{\pgfqpoint{1.977928in}{0.830387in}}%
\pgfpathlineto{\pgfqpoint{1.978781in}{0.830310in}}%
\pgfpathlineto{\pgfqpoint{1.979634in}{0.830233in}}%
\pgfpathlineto{\pgfqpoint{1.980487in}{0.830156in}}%
\pgfpathlineto{\pgfqpoint{1.981340in}{0.830079in}}%
\pgfpathlineto{\pgfqpoint{1.982193in}{0.830002in}}%
\pgfpathlineto{\pgfqpoint{1.983046in}{0.829925in}}%
\pgfpathlineto{\pgfqpoint{1.983898in}{0.829847in}}%
\pgfpathlineto{\pgfqpoint{1.984751in}{0.829770in}}%
\pgfpathlineto{\pgfqpoint{1.985604in}{0.829693in}}%
\pgfpathlineto{\pgfqpoint{1.986457in}{0.829616in}}%
\pgfpathlineto{\pgfqpoint{1.987310in}{0.829539in}}%
\pgfpathlineto{\pgfqpoint{1.988163in}{0.829462in}}%
\pgfpathlineto{\pgfqpoint{1.989016in}{0.829384in}}%
\pgfpathlineto{\pgfqpoint{1.989869in}{0.829307in}}%
\pgfpathlineto{\pgfqpoint{1.990721in}{0.829230in}}%
\pgfpathlineto{\pgfqpoint{1.991574in}{0.829152in}}%
\pgfpathlineto{\pgfqpoint{1.992427in}{0.829075in}}%
\pgfpathlineto{\pgfqpoint{1.993280in}{0.828997in}}%
\pgfpathlineto{\pgfqpoint{1.994133in}{0.828920in}}%
\pgfpathlineto{\pgfqpoint{1.994986in}{0.828842in}}%
\pgfpathlineto{\pgfqpoint{1.995839in}{0.828765in}}%
\pgfpathlineto{\pgfqpoint{1.996692in}{0.828687in}}%
\pgfpathlineto{\pgfqpoint{1.997545in}{0.828610in}}%
\pgfpathlineto{\pgfqpoint{1.998397in}{0.828532in}}%
\pgfpathlineto{\pgfqpoint{1.999250in}{0.828455in}}%
\pgfpathlineto{\pgfqpoint{2.000103in}{0.828377in}}%
\pgfpathlineto{\pgfqpoint{2.000956in}{0.828300in}}%
\pgfpathlineto{\pgfqpoint{2.001809in}{0.828222in}}%
\pgfpathlineto{\pgfqpoint{2.002662in}{0.828145in}}%
\pgfpathlineto{\pgfqpoint{2.003515in}{0.828067in}}%
\pgfpathlineto{\pgfqpoint{2.004368in}{0.827990in}}%
\pgfpathlineto{\pgfqpoint{2.005220in}{0.827913in}}%
\pgfpathlineto{\pgfqpoint{2.006073in}{0.827835in}}%
\pgfpathlineto{\pgfqpoint{2.006926in}{0.827758in}}%
\pgfpathlineto{\pgfqpoint{2.007779in}{0.827680in}}%
\pgfpathlineto{\pgfqpoint{2.008632in}{0.827603in}}%
\pgfpathlineto{\pgfqpoint{2.009485in}{0.827525in}}%
\pgfpathlineto{\pgfqpoint{2.010338in}{0.827470in}}%
\pgfpathlineto{\pgfqpoint{2.011191in}{0.827464in}}%
\pgfpathlineto{\pgfqpoint{2.012043in}{0.827456in}}%
\pgfpathlineto{\pgfqpoint{2.012896in}{0.827448in}}%
\pgfpathlineto{\pgfqpoint{2.013749in}{0.827439in}}%
\pgfpathlineto{\pgfqpoint{2.014602in}{0.827431in}}%
\pgfpathlineto{\pgfqpoint{2.015455in}{0.827423in}}%
\pgfpathlineto{\pgfqpoint{2.016308in}{0.827414in}}%
\pgfpathlineto{\pgfqpoint{2.017161in}{0.827406in}}%
\pgfpathlineto{\pgfqpoint{2.018014in}{0.827397in}}%
\pgfpathlineto{\pgfqpoint{2.018866in}{0.827389in}}%
\pgfpathlineto{\pgfqpoint{2.019719in}{0.827380in}}%
\pgfpathlineto{\pgfqpoint{2.020572in}{0.827372in}}%
\pgfpathlineto{\pgfqpoint{2.021425in}{0.827364in}}%
\pgfpathlineto{\pgfqpoint{2.022278in}{0.827355in}}%
\pgfpathlineto{\pgfqpoint{2.023131in}{0.827347in}}%
\pgfpathlineto{\pgfqpoint{2.023984in}{0.827338in}}%
\pgfpathlineto{\pgfqpoint{2.024837in}{0.827330in}}%
\pgfpathlineto{\pgfqpoint{2.025690in}{0.827321in}}%
\pgfpathlineto{\pgfqpoint{2.026542in}{0.827313in}}%
\pgfpathlineto{\pgfqpoint{2.027395in}{0.827305in}}%
\pgfpathlineto{\pgfqpoint{2.028248in}{0.827296in}}%
\pgfpathlineto{\pgfqpoint{2.029101in}{0.827288in}}%
\pgfpathlineto{\pgfqpoint{2.029954in}{0.827279in}}%
\pgfpathlineto{\pgfqpoint{2.030807in}{0.827271in}}%
\pgfpathlineto{\pgfqpoint{2.031660in}{0.827262in}}%
\pgfpathlineto{\pgfqpoint{2.032513in}{0.828230in}}%
\pgfpathlineto{\pgfqpoint{2.033365in}{0.828375in}}%
\pgfpathlineto{\pgfqpoint{2.034218in}{0.828370in}}%
\pgfpathlineto{\pgfqpoint{2.035071in}{0.828365in}}%
\pgfpathlineto{\pgfqpoint{2.035924in}{0.828361in}}%
\pgfpathlineto{\pgfqpoint{2.036777in}{0.828356in}}%
\pgfpathlineto{\pgfqpoint{2.037630in}{0.828352in}}%
\pgfpathlineto{\pgfqpoint{2.038483in}{0.828347in}}%
\pgfpathlineto{\pgfqpoint{2.039336in}{0.828342in}}%
\pgfpathlineto{\pgfqpoint{2.040188in}{0.828337in}}%
\pgfpathlineto{\pgfqpoint{2.041041in}{0.828331in}}%
\pgfpathlineto{\pgfqpoint{2.041894in}{0.828326in}}%
\pgfpathlineto{\pgfqpoint{2.042747in}{0.828321in}}%
\pgfpathlineto{\pgfqpoint{2.043600in}{0.828316in}}%
\pgfpathlineto{\pgfqpoint{2.044453in}{0.828311in}}%
\pgfpathlineto{\pgfqpoint{2.045306in}{0.828306in}}%
\pgfpathlineto{\pgfqpoint{2.046159in}{0.828301in}}%
\pgfpathlineto{\pgfqpoint{2.047011in}{0.828296in}}%
\pgfpathlineto{\pgfqpoint{2.047864in}{0.828291in}}%
\pgfpathlineto{\pgfqpoint{2.048717in}{0.828286in}}%
\pgfpathlineto{\pgfqpoint{2.049570in}{0.828281in}}%
\pgfpathlineto{\pgfqpoint{2.050423in}{0.828276in}}%
\pgfpathlineto{\pgfqpoint{2.051276in}{0.828271in}}%
\pgfpathlineto{\pgfqpoint{2.052129in}{0.828266in}}%
\pgfpathlineto{\pgfqpoint{2.052982in}{0.828261in}}%
\pgfpathlineto{\pgfqpoint{2.053835in}{0.828256in}}%
\pgfpathlineto{\pgfqpoint{2.054687in}{0.828250in}}%
\pgfpathlineto{\pgfqpoint{2.055540in}{0.828245in}}%
\pgfpathlineto{\pgfqpoint{2.056393in}{0.828240in}}%
\pgfpathlineto{\pgfqpoint{2.057246in}{0.828235in}}%
\pgfpathlineto{\pgfqpoint{2.058099in}{0.828230in}}%
\pgfpathlineto{\pgfqpoint{2.058952in}{0.828225in}}%
\pgfpathlineto{\pgfqpoint{2.059805in}{0.828220in}}%
\pgfpathlineto{\pgfqpoint{2.060658in}{0.828215in}}%
\pgfpathlineto{\pgfqpoint{2.061510in}{0.828210in}}%
\pgfpathlineto{\pgfqpoint{2.062363in}{0.828205in}}%
\pgfpathlineto{\pgfqpoint{2.063216in}{0.828200in}}%
\pgfpathlineto{\pgfqpoint{2.064069in}{0.828195in}}%
\pgfpathlineto{\pgfqpoint{2.064922in}{0.828190in}}%
\pgfpathlineto{\pgfqpoint{2.065775in}{0.828185in}}%
\pgfpathlineto{\pgfqpoint{2.066628in}{0.828180in}}%
\pgfpathlineto{\pgfqpoint{2.067481in}{0.828175in}}%
\pgfpathlineto{\pgfqpoint{2.068333in}{0.828169in}}%
\pgfpathlineto{\pgfqpoint{2.069186in}{0.828164in}}%
\pgfpathlineto{\pgfqpoint{2.070039in}{0.828159in}}%
\pgfpathlineto{\pgfqpoint{2.070892in}{0.828154in}}%
\pgfpathlineto{\pgfqpoint{2.071745in}{0.828149in}}%
\pgfpathlineto{\pgfqpoint{2.072598in}{0.828144in}}%
\pgfpathlineto{\pgfqpoint{2.073451in}{0.828139in}}%
\pgfpathlineto{\pgfqpoint{2.074304in}{0.828134in}}%
\pgfpathlineto{\pgfqpoint{2.075156in}{0.828129in}}%
\pgfpathlineto{\pgfqpoint{2.076009in}{0.828124in}}%
\pgfpathlineto{\pgfqpoint{2.076862in}{0.828119in}}%
\pgfpathlineto{\pgfqpoint{2.077715in}{0.828114in}}%
\pgfpathlineto{\pgfqpoint{2.078568in}{0.828109in}}%
\pgfpathlineto{\pgfqpoint{2.079421in}{0.828104in}}%
\pgfpathlineto{\pgfqpoint{2.080274in}{0.828099in}}%
\pgfpathlineto{\pgfqpoint{2.081127in}{0.828093in}}%
\pgfpathlineto{\pgfqpoint{2.081980in}{0.828088in}}%
\pgfpathlineto{\pgfqpoint{2.082832in}{0.828083in}}%
\pgfpathlineto{\pgfqpoint{2.083685in}{0.828078in}}%
\pgfpathlineto{\pgfqpoint{2.084538in}{0.828073in}}%
\pgfpathlineto{\pgfqpoint{2.085391in}{0.828068in}}%
\pgfpathlineto{\pgfqpoint{2.086244in}{0.828063in}}%
\pgfpathlineto{\pgfqpoint{2.087097in}{0.828058in}}%
\pgfpathlineto{\pgfqpoint{2.087950in}{0.828053in}}%
\pgfpathlineto{\pgfqpoint{2.088803in}{0.828048in}}%
\pgfpathlineto{\pgfqpoint{2.089655in}{0.828043in}}%
\pgfpathlineto{\pgfqpoint{2.090508in}{0.828037in}}%
\pgfpathlineto{\pgfqpoint{2.091361in}{0.828025in}}%
\pgfpathlineto{\pgfqpoint{2.092214in}{0.828013in}}%
\pgfpathlineto{\pgfqpoint{2.093067in}{0.828001in}}%
\pgfpathlineto{\pgfqpoint{2.093920in}{0.827989in}}%
\pgfpathlineto{\pgfqpoint{2.094773in}{0.827977in}}%
\pgfpathlineto{\pgfqpoint{2.095626in}{0.827965in}}%
\pgfpathlineto{\pgfqpoint{2.096478in}{0.827953in}}%
\pgfpathlineto{\pgfqpoint{2.097331in}{0.827941in}}%
\pgfpathlineto{\pgfqpoint{2.098184in}{0.827929in}}%
\pgfpathlineto{\pgfqpoint{2.099037in}{0.827917in}}%
\pgfpathlineto{\pgfqpoint{2.099890in}{0.827905in}}%
\pgfpathlineto{\pgfqpoint{2.100743in}{0.827893in}}%
\pgfpathlineto{\pgfqpoint{2.101596in}{0.827881in}}%
\pgfpathlineto{\pgfqpoint{2.102449in}{0.827869in}}%
\pgfpathlineto{\pgfqpoint{2.103301in}{0.827857in}}%
\pgfpathlineto{\pgfqpoint{2.104154in}{0.827845in}}%
\pgfpathlineto{\pgfqpoint{2.105007in}{0.827833in}}%
\pgfpathlineto{\pgfqpoint{2.105860in}{0.827821in}}%
\pgfpathlineto{\pgfqpoint{2.106713in}{0.827809in}}%
\pgfpathlineto{\pgfqpoint{2.107566in}{0.827797in}}%
\pgfpathlineto{\pgfqpoint{2.108419in}{0.827785in}}%
\pgfpathlineto{\pgfqpoint{2.109272in}{0.827773in}}%
\pgfpathlineto{\pgfqpoint{2.110124in}{0.827761in}}%
\pgfpathlineto{\pgfqpoint{2.110977in}{0.827749in}}%
\pgfpathlineto{\pgfqpoint{2.111830in}{0.827737in}}%
\pgfpathlineto{\pgfqpoint{2.112683in}{0.827725in}}%
\pgfpathlineto{\pgfqpoint{2.113536in}{0.827713in}}%
\pgfpathlineto{\pgfqpoint{2.114389in}{0.827701in}}%
\pgfpathlineto{\pgfqpoint{2.115242in}{0.827689in}}%
\pgfpathlineto{\pgfqpoint{2.116095in}{0.827677in}}%
\pgfpathlineto{\pgfqpoint{2.116948in}{0.827665in}}%
\pgfpathlineto{\pgfqpoint{2.117800in}{0.827653in}}%
\pgfpathlineto{\pgfqpoint{2.118653in}{0.827641in}}%
\pgfpathlineto{\pgfqpoint{2.119506in}{0.827629in}}%
\pgfpathlineto{\pgfqpoint{2.120359in}{0.827617in}}%
\pgfpathlineto{\pgfqpoint{2.121212in}{0.827605in}}%
\pgfpathlineto{\pgfqpoint{2.122065in}{0.827593in}}%
\pgfpathlineto{\pgfqpoint{2.122918in}{0.827581in}}%
\pgfpathlineto{\pgfqpoint{2.123771in}{0.827569in}}%
\pgfpathlineto{\pgfqpoint{2.124623in}{0.827557in}}%
\pgfpathlineto{\pgfqpoint{2.125476in}{0.827545in}}%
\pgfpathlineto{\pgfqpoint{2.126329in}{0.827533in}}%
\pgfpathlineto{\pgfqpoint{2.127182in}{0.827521in}}%
\pgfpathlineto{\pgfqpoint{2.128035in}{0.827509in}}%
\pgfpathlineto{\pgfqpoint{2.128888in}{0.827497in}}%
\pgfpathlineto{\pgfqpoint{2.129741in}{0.827485in}}%
\pgfpathlineto{\pgfqpoint{2.130594in}{0.827473in}}%
\pgfpathlineto{\pgfqpoint{2.131446in}{0.827461in}}%
\pgfpathlineto{\pgfqpoint{2.132299in}{0.827449in}}%
\pgfpathlineto{\pgfqpoint{2.133152in}{0.827437in}}%
\pgfpathlineto{\pgfqpoint{2.134005in}{0.827425in}}%
\pgfpathlineto{\pgfqpoint{2.134858in}{0.827413in}}%
\pgfpathlineto{\pgfqpoint{2.135711in}{0.827401in}}%
\pgfpathlineto{\pgfqpoint{2.136564in}{0.827389in}}%
\pgfpathlineto{\pgfqpoint{2.137417in}{0.827377in}}%
\pgfpathlineto{\pgfqpoint{2.138269in}{0.827365in}}%
\pgfpathlineto{\pgfqpoint{2.139122in}{0.827353in}}%
\pgfpathlineto{\pgfqpoint{2.139975in}{0.827341in}}%
\pgfpathlineto{\pgfqpoint{2.140828in}{0.827329in}}%
\pgfpathlineto{\pgfqpoint{2.141681in}{0.827317in}}%
\pgfpathlineto{\pgfqpoint{2.142534in}{0.827305in}}%
\pgfpathlineto{\pgfqpoint{2.143387in}{0.827293in}}%
\pgfpathlineto{\pgfqpoint{2.144240in}{0.827281in}}%
\pgfpathlineto{\pgfqpoint{2.145093in}{0.827269in}}%
\pgfpathlineto{\pgfqpoint{2.145945in}{0.827256in}}%
\pgfpathlineto{\pgfqpoint{2.146798in}{0.827244in}}%
\pgfpathlineto{\pgfqpoint{2.147651in}{0.827232in}}%
\pgfpathlineto{\pgfqpoint{2.148504in}{0.827220in}}%
\pgfpathlineto{\pgfqpoint{2.149357in}{0.827208in}}%
\pgfpathlineto{\pgfqpoint{2.150210in}{0.827196in}}%
\pgfpathlineto{\pgfqpoint{2.151063in}{0.827184in}}%
\pgfpathlineto{\pgfqpoint{2.151916in}{0.827172in}}%
\pgfpathlineto{\pgfqpoint{2.152768in}{0.827160in}}%
\pgfpathlineto{\pgfqpoint{2.153621in}{0.827148in}}%
\pgfpathlineto{\pgfqpoint{2.154474in}{0.827136in}}%
\pgfpathlineto{\pgfqpoint{2.155327in}{0.827124in}}%
\pgfpathlineto{\pgfqpoint{2.156180in}{0.827112in}}%
\pgfpathlineto{\pgfqpoint{2.157033in}{0.827100in}}%
\pgfpathlineto{\pgfqpoint{2.157886in}{0.827088in}}%
\pgfpathlineto{\pgfqpoint{2.158739in}{0.827076in}}%
\pgfpathlineto{\pgfqpoint{2.159591in}{0.827064in}}%
\pgfpathlineto{\pgfqpoint{2.160444in}{0.827052in}}%
\pgfpathlineto{\pgfqpoint{2.161297in}{0.827040in}}%
\pgfpathlineto{\pgfqpoint{2.162150in}{0.827028in}}%
\pgfpathlineto{\pgfqpoint{2.163003in}{0.827016in}}%
\pgfpathlineto{\pgfqpoint{2.163856in}{0.827004in}}%
\pgfpathlineto{\pgfqpoint{2.164709in}{0.826992in}}%
\pgfpathlineto{\pgfqpoint{2.165562in}{0.826980in}}%
\pgfpathlineto{\pgfqpoint{2.166414in}{0.826968in}}%
\pgfpathlineto{\pgfqpoint{2.167267in}{0.826956in}}%
\pgfpathlineto{\pgfqpoint{2.168120in}{0.826944in}}%
\pgfpathlineto{\pgfqpoint{2.168973in}{0.826932in}}%
\pgfpathlineto{\pgfqpoint{2.169826in}{0.826920in}}%
\pgfpathlineto{\pgfqpoint{2.170679in}{0.826908in}}%
\pgfpathlineto{\pgfqpoint{2.171532in}{0.826896in}}%
\pgfpathlineto{\pgfqpoint{2.172385in}{0.826884in}}%
\pgfpathlineto{\pgfqpoint{2.173238in}{0.826872in}}%
\pgfpathlineto{\pgfqpoint{2.174090in}{0.826860in}}%
\pgfpathlineto{\pgfqpoint{2.174943in}{0.826848in}}%
\pgfpathlineto{\pgfqpoint{2.175796in}{0.826836in}}%
\pgfpathlineto{\pgfqpoint{2.176649in}{0.826824in}}%
\pgfpathlineto{\pgfqpoint{2.177502in}{0.826812in}}%
\pgfpathlineto{\pgfqpoint{2.178355in}{0.826800in}}%
\pgfpathlineto{\pgfqpoint{2.179208in}{0.826788in}}%
\pgfpathlineto{\pgfqpoint{2.180061in}{0.826776in}}%
\pgfpathlineto{\pgfqpoint{2.180913in}{0.826764in}}%
\pgfpathlineto{\pgfqpoint{2.181766in}{0.826752in}}%
\pgfpathlineto{\pgfqpoint{2.182619in}{0.826740in}}%
\pgfpathlineto{\pgfqpoint{2.183472in}{0.826728in}}%
\pgfpathlineto{\pgfqpoint{2.184325in}{0.826716in}}%
\pgfpathlineto{\pgfqpoint{2.185178in}{0.826704in}}%
\pgfpathlineto{\pgfqpoint{2.186031in}{0.826692in}}%
\pgfpathlineto{\pgfqpoint{2.186884in}{0.826680in}}%
\pgfpathlineto{\pgfqpoint{2.187736in}{0.826668in}}%
\pgfpathlineto{\pgfqpoint{2.188589in}{0.826656in}}%
\pgfpathlineto{\pgfqpoint{2.189442in}{0.826644in}}%
\pgfpathlineto{\pgfqpoint{2.190295in}{0.826632in}}%
\pgfpathlineto{\pgfqpoint{2.191148in}{0.826620in}}%
\pgfpathlineto{\pgfqpoint{2.192001in}{0.826608in}}%
\pgfpathlineto{\pgfqpoint{2.192854in}{0.826596in}}%
\pgfpathlineto{\pgfqpoint{2.193707in}{0.826584in}}%
\pgfpathlineto{\pgfqpoint{2.194559in}{0.826572in}}%
\pgfpathlineto{\pgfqpoint{2.195412in}{0.826560in}}%
\pgfpathlineto{\pgfqpoint{2.196265in}{0.826548in}}%
\pgfpathlineto{\pgfqpoint{2.197118in}{0.826536in}}%
\pgfpathlineto{\pgfqpoint{2.197971in}{0.826524in}}%
\pgfpathlineto{\pgfqpoint{2.198824in}{0.826512in}}%
\pgfpathlineto{\pgfqpoint{2.199677in}{0.826500in}}%
\pgfpathlineto{\pgfqpoint{2.200530in}{0.826488in}}%
\pgfpathlineto{\pgfqpoint{2.201382in}{0.826476in}}%
\pgfpathlineto{\pgfqpoint{2.202235in}{0.826464in}}%
\pgfpathlineto{\pgfqpoint{2.203088in}{0.826452in}}%
\pgfpathlineto{\pgfqpoint{2.203941in}{0.826440in}}%
\pgfpathlineto{\pgfqpoint{2.204794in}{0.826428in}}%
\pgfpathlineto{\pgfqpoint{2.205647in}{0.826416in}}%
\pgfpathlineto{\pgfqpoint{2.206500in}{0.826404in}}%
\pgfpathlineto{\pgfqpoint{2.207353in}{0.826392in}}%
\pgfpathlineto{\pgfqpoint{2.208206in}{0.826380in}}%
\pgfpathlineto{\pgfqpoint{2.209058in}{0.826367in}}%
\pgfpathlineto{\pgfqpoint{2.209911in}{0.826355in}}%
\pgfpathlineto{\pgfqpoint{2.210764in}{0.826343in}}%
\pgfpathlineto{\pgfqpoint{2.211617in}{0.826331in}}%
\pgfpathlineto{\pgfqpoint{2.212470in}{0.826319in}}%
\pgfpathlineto{\pgfqpoint{2.213323in}{0.826307in}}%
\pgfpathlineto{\pgfqpoint{2.214176in}{0.826295in}}%
\pgfpathlineto{\pgfqpoint{2.215029in}{0.826283in}}%
\pgfpathlineto{\pgfqpoint{2.215881in}{0.826271in}}%
\pgfpathlineto{\pgfqpoint{2.216734in}{0.826259in}}%
\pgfpathlineto{\pgfqpoint{2.217587in}{0.826247in}}%
\pgfpathlineto{\pgfqpoint{2.218440in}{0.826235in}}%
\pgfpathlineto{\pgfqpoint{2.219293in}{0.826223in}}%
\pgfpathlineto{\pgfqpoint{2.220146in}{0.826211in}}%
\pgfpathlineto{\pgfqpoint{2.220999in}{0.826199in}}%
\pgfpathlineto{\pgfqpoint{2.221852in}{0.826187in}}%
\pgfpathlineto{\pgfqpoint{2.222704in}{0.826175in}}%
\pgfpathlineto{\pgfqpoint{2.223557in}{0.826163in}}%
\pgfpathlineto{\pgfqpoint{2.224410in}{0.826161in}}%
\pgfpathlineto{\pgfqpoint{2.225263in}{0.826167in}}%
\pgfpathlineto{\pgfqpoint{2.226116in}{0.826174in}}%
\pgfpathlineto{\pgfqpoint{2.226969in}{0.826180in}}%
\pgfpathlineto{\pgfqpoint{2.227822in}{0.826186in}}%
\pgfpathlineto{\pgfqpoint{2.228675in}{0.826193in}}%
\pgfpathlineto{\pgfqpoint{2.229527in}{0.826199in}}%
\pgfpathlineto{\pgfqpoint{2.230380in}{0.826206in}}%
\pgfpathlineto{\pgfqpoint{2.231233in}{0.826212in}}%
\pgfpathlineto{\pgfqpoint{2.232086in}{0.826218in}}%
\pgfpathlineto{\pgfqpoint{2.232939in}{0.826225in}}%
\pgfpathlineto{\pgfqpoint{2.233792in}{0.826231in}}%
\pgfpathlineto{\pgfqpoint{2.234645in}{0.826238in}}%
\pgfpathlineto{\pgfqpoint{2.235498in}{0.826244in}}%
\pgfpathlineto{\pgfqpoint{2.236351in}{0.826250in}}%
\pgfpathlineto{\pgfqpoint{2.237203in}{0.826257in}}%
\pgfpathlineto{\pgfqpoint{2.238056in}{0.826263in}}%
\pgfpathlineto{\pgfqpoint{2.238909in}{0.826270in}}%
\pgfpathlineto{\pgfqpoint{2.239762in}{0.826276in}}%
\pgfpathlineto{\pgfqpoint{2.240615in}{0.826282in}}%
\pgfpathlineto{\pgfqpoint{2.241468in}{0.826289in}}%
\pgfpathlineto{\pgfqpoint{2.242321in}{0.826295in}}%
\pgfpathlineto{\pgfqpoint{2.243174in}{0.826302in}}%
\pgfpathlineto{\pgfqpoint{2.244026in}{0.826308in}}%
\pgfpathlineto{\pgfqpoint{2.244879in}{0.826310in}}%
\pgfpathlineto{\pgfqpoint{2.245732in}{0.826038in}}%
\pgfpathlineto{\pgfqpoint{2.246585in}{0.825603in}}%
\pgfpathlineto{\pgfqpoint{2.247438in}{0.825168in}}%
\pgfpathlineto{\pgfqpoint{2.248291in}{0.824734in}}%
\pgfpathlineto{\pgfqpoint{2.249144in}{0.824299in}}%
\pgfpathlineto{\pgfqpoint{2.249997in}{0.823864in}}%
\pgfpathlineto{\pgfqpoint{2.250849in}{0.823430in}}%
\pgfpathlineto{\pgfqpoint{2.251702in}{0.822995in}}%
\pgfpathlineto{\pgfqpoint{2.252555in}{0.822560in}}%
\pgfpathlineto{\pgfqpoint{2.253408in}{0.822126in}}%
\pgfpathlineto{\pgfqpoint{2.254261in}{0.821691in}}%
\pgfpathlineto{\pgfqpoint{2.255114in}{0.821256in}}%
\pgfpathlineto{\pgfqpoint{2.255967in}{0.820822in}}%
\pgfpathlineto{\pgfqpoint{2.256820in}{0.820953in}}%
\pgfpathlineto{\pgfqpoint{2.257672in}{0.829468in}}%
\pgfpathlineto{\pgfqpoint{2.258525in}{0.833187in}}%
\pgfpathlineto{\pgfqpoint{2.259378in}{0.832374in}}%
\pgfpathlineto{\pgfqpoint{2.260231in}{0.831562in}}%
\pgfpathlineto{\pgfqpoint{2.261084in}{0.830750in}}%
\pgfpathlineto{\pgfqpoint{2.261937in}{0.829937in}}%
\pgfpathlineto{\pgfqpoint{2.262790in}{0.829125in}}%
\pgfpathlineto{\pgfqpoint{2.263643in}{0.828313in}}%
\pgfpathlineto{\pgfqpoint{2.264496in}{0.827501in}}%
\pgfpathlineto{\pgfqpoint{2.265348in}{0.826840in}}%
\pgfpathlineto{\pgfqpoint{2.266201in}{0.826786in}}%
\pgfpathlineto{\pgfqpoint{2.267054in}{0.826795in}}%
\pgfpathlineto{\pgfqpoint{2.267907in}{0.826803in}}%
\pgfpathlineto{\pgfqpoint{2.268760in}{0.826812in}}%
\pgfpathlineto{\pgfqpoint{2.269613in}{0.826821in}}%
\pgfpathlineto{\pgfqpoint{2.270466in}{0.826830in}}%
\pgfpathlineto{\pgfqpoint{2.271319in}{0.826838in}}%
\pgfpathlineto{\pgfqpoint{2.272171in}{0.826847in}}%
\pgfpathlineto{\pgfqpoint{2.273024in}{0.826856in}}%
\pgfpathlineto{\pgfqpoint{2.273877in}{0.826865in}}%
\pgfpathlineto{\pgfqpoint{2.274730in}{0.826873in}}%
\pgfpathlineto{\pgfqpoint{2.275583in}{0.826882in}}%
\pgfpathlineto{\pgfqpoint{2.276436in}{0.826891in}}%
\pgfpathlineto{\pgfqpoint{2.277289in}{0.826900in}}%
\pgfpathlineto{\pgfqpoint{2.278142in}{0.826908in}}%
\pgfpathlineto{\pgfqpoint{2.278994in}{0.826917in}}%
\pgfpathlineto{\pgfqpoint{2.279847in}{0.826926in}}%
\pgfpathlineto{\pgfqpoint{2.280700in}{0.827167in}}%
\pgfpathlineto{\pgfqpoint{2.281553in}{0.827681in}}%
\pgfpathlineto{\pgfqpoint{2.282406in}{0.828195in}}%
\pgfpathlineto{\pgfqpoint{2.283259in}{0.828710in}}%
\pgfpathlineto{\pgfqpoint{2.284112in}{0.829224in}}%
\pgfpathlineto{\pgfqpoint{2.284965in}{0.829739in}}%
\pgfpathlineto{\pgfqpoint{2.285817in}{0.830254in}}%
\pgfpathlineto{\pgfqpoint{2.286670in}{0.830768in}}%
\pgfpathlineto{\pgfqpoint{2.287523in}{0.831283in}}%
\pgfpathlineto{\pgfqpoint{2.288376in}{0.831797in}}%
\pgfpathlineto{\pgfqpoint{2.289229in}{0.832312in}}%
\pgfpathlineto{\pgfqpoint{2.290082in}{0.832826in}}%
\pgfpathlineto{\pgfqpoint{2.290935in}{0.833341in}}%
\pgfpathlineto{\pgfqpoint{2.291788in}{0.833855in}}%
\pgfpathlineto{\pgfqpoint{2.292641in}{0.834370in}}%
\pgfpathlineto{\pgfqpoint{2.293493in}{0.834884in}}%
\pgfpathlineto{\pgfqpoint{2.294346in}{0.835399in}}%
\pgfpathlineto{\pgfqpoint{2.295199in}{0.835913in}}%
\pgfpathlineto{\pgfqpoint{2.296052in}{0.836428in}}%
\pgfpathlineto{\pgfqpoint{2.296905in}{0.836943in}}%
\pgfpathlineto{\pgfqpoint{2.297758in}{0.837457in}}%
\pgfpathlineto{\pgfqpoint{2.298611in}{0.837972in}}%
\pgfpathlineto{\pgfqpoint{2.299464in}{0.838486in}}%
\pgfpathlineto{\pgfqpoint{2.300316in}{0.838648in}}%
\pgfpathlineto{\pgfqpoint{2.301169in}{0.834044in}}%
\pgfpathlineto{\pgfqpoint{2.302022in}{0.834215in}}%
\pgfpathlineto{\pgfqpoint{2.302875in}{0.834385in}}%
\pgfpathlineto{\pgfqpoint{2.303728in}{0.834556in}}%
\pgfpathlineto{\pgfqpoint{2.304581in}{0.834726in}}%
\pgfpathlineto{\pgfqpoint{2.305434in}{0.834897in}}%
\pgfpathlineto{\pgfqpoint{2.306287in}{0.835030in}}%
\pgfpathlineto{\pgfqpoint{2.307139in}{0.834699in}}%
\pgfpathlineto{\pgfqpoint{2.307992in}{0.834230in}}%
\pgfpathlineto{\pgfqpoint{2.308845in}{0.833760in}}%
\pgfpathlineto{\pgfqpoint{2.309698in}{0.833291in}}%
\pgfpathlineto{\pgfqpoint{2.310551in}{0.832821in}}%
\pgfpathlineto{\pgfqpoint{2.311404in}{0.832352in}}%
\pgfpathlineto{\pgfqpoint{2.312257in}{0.831882in}}%
\pgfpathlineto{\pgfqpoint{2.313110in}{0.831413in}}%
\pgfpathlineto{\pgfqpoint{2.313962in}{0.830943in}}%
\pgfpathlineto{\pgfqpoint{2.314815in}{0.830474in}}%
\pgfpathlineto{\pgfqpoint{2.315668in}{0.830004in}}%
\pgfpathlineto{\pgfqpoint{2.316521in}{0.829534in}}%
\pgfpathlineto{\pgfqpoint{2.317374in}{0.829065in}}%
\pgfpathlineto{\pgfqpoint{2.318227in}{0.828595in}}%
\pgfpathlineto{\pgfqpoint{2.319080in}{0.828126in}}%
\pgfpathlineto{\pgfqpoint{2.319933in}{0.827656in}}%
\pgfpathlineto{\pgfqpoint{2.320785in}{0.827292in}}%
\pgfpathlineto{\pgfqpoint{2.321638in}{0.827282in}}%
\pgfpathlineto{\pgfqpoint{2.322491in}{0.827300in}}%
\pgfpathlineto{\pgfqpoint{2.323344in}{0.827317in}}%
\pgfpathlineto{\pgfqpoint{2.324197in}{0.827335in}}%
\pgfpathlineto{\pgfqpoint{2.325050in}{0.827353in}}%
\pgfpathlineto{\pgfqpoint{2.325903in}{0.827371in}}%
\pgfpathlineto{\pgfqpoint{2.326756in}{0.827389in}}%
\pgfpathlineto{\pgfqpoint{2.327609in}{0.827406in}}%
\pgfpathlineto{\pgfqpoint{2.328461in}{0.827424in}}%
\pgfpathlineto{\pgfqpoint{2.329314in}{0.827442in}}%
\pgfpathlineto{\pgfqpoint{2.330167in}{0.827460in}}%
\pgfpathlineto{\pgfqpoint{2.331020in}{0.827478in}}%
\pgfpathlineto{\pgfqpoint{2.331873in}{0.827495in}}%
\pgfpathlineto{\pgfqpoint{2.332726in}{0.827513in}}%
\pgfpathlineto{\pgfqpoint{2.333579in}{0.827531in}}%
\pgfpathlineto{\pgfqpoint{2.334432in}{0.827549in}}%
\pgfpathlineto{\pgfqpoint{2.335284in}{0.827567in}}%
\pgfpathlineto{\pgfqpoint{2.336137in}{0.827585in}}%
\pgfpathlineto{\pgfqpoint{2.336990in}{0.827602in}}%
\pgfpathlineto{\pgfqpoint{2.337843in}{0.827620in}}%
\pgfpathlineto{\pgfqpoint{2.338696in}{0.827638in}}%
\pgfpathlineto{\pgfqpoint{2.339549in}{0.827656in}}%
\pgfpathlineto{\pgfqpoint{2.340402in}{0.827674in}}%
\pgfpathlineto{\pgfqpoint{2.341255in}{0.827691in}}%
\pgfpathlineto{\pgfqpoint{2.342107in}{0.827709in}}%
\pgfpathlineto{\pgfqpoint{2.342960in}{0.827727in}}%
\pgfpathlineto{\pgfqpoint{2.343813in}{0.827745in}}%
\pgfpathlineto{\pgfqpoint{2.344666in}{0.827763in}}%
\pgfpathlineto{\pgfqpoint{2.345519in}{0.827780in}}%
\pgfpathlineto{\pgfqpoint{2.346372in}{0.827798in}}%
\pgfpathlineto{\pgfqpoint{2.347225in}{0.827875in}}%
\pgfpathlineto{\pgfqpoint{2.348078in}{0.828071in}}%
\pgfpathlineto{\pgfqpoint{2.348930in}{0.828270in}}%
\pgfpathlineto{\pgfqpoint{2.349783in}{0.828469in}}%
\pgfpathlineto{\pgfqpoint{2.350636in}{0.828668in}}%
\pgfpathlineto{\pgfqpoint{2.351489in}{0.828867in}}%
\pgfpathlineto{\pgfqpoint{2.352342in}{0.829066in}}%
\pgfpathlineto{\pgfqpoint{2.353195in}{0.829265in}}%
\pgfpathlineto{\pgfqpoint{2.354048in}{0.829464in}}%
\pgfpathlineto{\pgfqpoint{2.354901in}{0.829663in}}%
\pgfpathlineto{\pgfqpoint{2.355754in}{0.829862in}}%
\pgfpathlineto{\pgfqpoint{2.356606in}{0.830061in}}%
\pgfpathlineto{\pgfqpoint{2.357459in}{0.830260in}}%
\pgfpathlineto{\pgfqpoint{2.358312in}{0.830459in}}%
\pgfpathlineto{\pgfqpoint{2.359165in}{0.830658in}}%
\pgfpathlineto{\pgfqpoint{2.360018in}{0.830857in}}%
\pgfpathlineto{\pgfqpoint{2.360871in}{0.831056in}}%
\pgfpathlineto{\pgfqpoint{2.361724in}{0.831255in}}%
\pgfpathlineto{\pgfqpoint{2.362577in}{0.831454in}}%
\pgfpathlineto{\pgfqpoint{2.363429in}{0.831653in}}%
\pgfpathlineto{\pgfqpoint{2.364282in}{0.831852in}}%
\pgfpathlineto{\pgfqpoint{2.365135in}{0.832051in}}%
\pgfpathlineto{\pgfqpoint{2.365988in}{0.832250in}}%
\pgfpathlineto{\pgfqpoint{2.366841in}{0.832449in}}%
\pgfpathlineto{\pgfqpoint{2.367694in}{0.832648in}}%
\pgfpathlineto{\pgfqpoint{2.368547in}{0.832847in}}%
\pgfpathlineto{\pgfqpoint{2.369400in}{0.833046in}}%
\pgfpathlineto{\pgfqpoint{2.370252in}{0.833245in}}%
\pgfpathlineto{\pgfqpoint{2.371105in}{0.833444in}}%
\pgfpathlineto{\pgfqpoint{2.371958in}{0.833643in}}%
\pgfpathlineto{\pgfqpoint{2.372811in}{0.833842in}}%
\pgfpathlineto{\pgfqpoint{2.373664in}{0.834041in}}%
\pgfpathlineto{\pgfqpoint{2.374517in}{0.834240in}}%
\pgfpathlineto{\pgfqpoint{2.375370in}{0.834439in}}%
\pgfpathlineto{\pgfqpoint{2.376223in}{0.834638in}}%
\pgfpathlineto{\pgfqpoint{2.377075in}{0.834826in}}%
\pgfpathlineto{\pgfqpoint{2.377928in}{0.834885in}}%
\pgfpathlineto{\pgfqpoint{2.378781in}{0.834905in}}%
\pgfpathlineto{\pgfqpoint{2.379634in}{0.834906in}}%
\pgfpathlineto{\pgfqpoint{2.380487in}{0.835074in}}%
\pgfpathlineto{\pgfqpoint{2.381340in}{0.835316in}}%
\pgfpathlineto{\pgfqpoint{2.382193in}{0.835336in}}%
\pgfpathlineto{\pgfqpoint{2.383046in}{0.835355in}}%
\pgfpathlineto{\pgfqpoint{2.383899in}{0.835374in}}%
\pgfpathlineto{\pgfqpoint{2.384751in}{0.835394in}}%
\pgfpathlineto{\pgfqpoint{2.385604in}{0.835413in}}%
\pgfpathlineto{\pgfqpoint{2.386457in}{0.835432in}}%
\pgfpathlineto{\pgfqpoint{2.387310in}{0.835452in}}%
\pgfpathlineto{\pgfqpoint{2.388163in}{0.835471in}}%
\pgfpathlineto{\pgfqpoint{2.389016in}{0.835490in}}%
\pgfpathlineto{\pgfqpoint{2.389869in}{0.835510in}}%
\pgfpathlineto{\pgfqpoint{2.390722in}{0.835529in}}%
\pgfpathlineto{\pgfqpoint{2.391574in}{0.835548in}}%
\pgfpathlineto{\pgfqpoint{2.392427in}{0.835568in}}%
\pgfpathlineto{\pgfqpoint{2.393280in}{0.835587in}}%
\pgfpathlineto{\pgfqpoint{2.394133in}{0.835606in}}%
\pgfpathlineto{\pgfqpoint{2.394986in}{0.835626in}}%
\pgfpathlineto{\pgfqpoint{2.395839in}{0.836475in}}%
\pgfpathlineto{\pgfqpoint{2.396692in}{0.838128in}}%
\pgfpathlineto{\pgfqpoint{2.397545in}{0.839781in}}%
\pgfpathlineto{\pgfqpoint{2.398397in}{0.841434in}}%
\pgfpathlineto{\pgfqpoint{2.399250in}{0.842139in}}%
\pgfpathlineto{\pgfqpoint{2.400103in}{0.837243in}}%
\pgfpathlineto{\pgfqpoint{2.400956in}{0.837586in}}%
\pgfpathlineto{\pgfqpoint{2.401809in}{0.834738in}}%
\pgfpathlineto{\pgfqpoint{2.402662in}{0.835611in}}%
\pgfpathlineto{\pgfqpoint{2.403515in}{0.839421in}}%
\pgfpathlineto{\pgfqpoint{2.404368in}{0.839173in}}%
\pgfpathlineto{\pgfqpoint{2.405220in}{0.832699in}}%
\pgfpathlineto{\pgfqpoint{2.406073in}{0.841572in}}%
\pgfpathlineto{\pgfqpoint{2.406926in}{0.837756in}}%
\pgfpathlineto{\pgfqpoint{2.407779in}{0.833650in}}%
\pgfpathlineto{\pgfqpoint{2.408632in}{0.829559in}}%
\pgfpathlineto{\pgfqpoint{2.409485in}{0.828685in}}%
\pgfpathlineto{\pgfqpoint{2.410338in}{0.830226in}}%
\pgfpathlineto{\pgfqpoint{2.411191in}{0.831768in}}%
\pgfpathlineto{\pgfqpoint{2.412044in}{0.833310in}}%
\pgfpathlineto{\pgfqpoint{2.412896in}{0.834851in}}%
\pgfpathlineto{\pgfqpoint{2.413749in}{0.836393in}}%
\pgfpathlineto{\pgfqpoint{2.414602in}{0.837935in}}%
\pgfpathlineto{\pgfqpoint{2.415455in}{0.839476in}}%
\pgfpathlineto{\pgfqpoint{2.416308in}{0.841018in}}%
\pgfpathlineto{\pgfqpoint{2.417161in}{0.842560in}}%
\pgfpathlineto{\pgfqpoint{2.418014in}{0.840411in}}%
\pgfpathlineto{\pgfqpoint{2.418867in}{0.829446in}}%
\pgfpathlineto{\pgfqpoint{2.419719in}{0.833175in}}%
\pgfpathlineto{\pgfqpoint{2.420572in}{0.836824in}}%
\pgfpathlineto{\pgfqpoint{2.421425in}{0.837437in}}%
\pgfpathlineto{\pgfqpoint{2.422278in}{0.837449in}}%
\pgfpathlineto{\pgfqpoint{2.423131in}{0.837442in}}%
\pgfpathlineto{\pgfqpoint{2.423984in}{0.837435in}}%
\pgfpathlineto{\pgfqpoint{2.424837in}{0.837428in}}%
\pgfpathlineto{\pgfqpoint{2.425690in}{0.837422in}}%
\pgfpathlineto{\pgfqpoint{2.426542in}{0.837418in}}%
\pgfpathlineto{\pgfqpoint{2.427395in}{0.837413in}}%
\pgfpathlineto{\pgfqpoint{2.428248in}{0.837409in}}%
\pgfpathlineto{\pgfqpoint{2.429101in}{0.837404in}}%
\pgfpathlineto{\pgfqpoint{2.429954in}{0.837400in}}%
\pgfpathlineto{\pgfqpoint{2.430807in}{0.837395in}}%
\pgfpathlineto{\pgfqpoint{2.431660in}{0.837391in}}%
\pgfpathlineto{\pgfqpoint{2.432513in}{0.837386in}}%
\pgfpathlineto{\pgfqpoint{2.433365in}{0.837382in}}%
\pgfpathlineto{\pgfqpoint{2.434218in}{0.837377in}}%
\pgfpathlineto{\pgfqpoint{2.435071in}{0.837372in}}%
\pgfpathlineto{\pgfqpoint{2.435924in}{0.837368in}}%
\pgfpathlineto{\pgfqpoint{2.436777in}{0.837363in}}%
\pgfpathlineto{\pgfqpoint{2.437630in}{0.837384in}}%
\pgfpathlineto{\pgfqpoint{2.438483in}{0.837584in}}%
\pgfpathlineto{\pgfqpoint{2.439336in}{0.837818in}}%
\pgfpathlineto{\pgfqpoint{2.440188in}{0.838053in}}%
\pgfpathlineto{\pgfqpoint{2.441041in}{0.838288in}}%
\pgfpathlineto{\pgfqpoint{2.441894in}{0.838523in}}%
\pgfpathlineto{\pgfqpoint{2.442747in}{0.838758in}}%
\pgfpathlineto{\pgfqpoint{2.443600in}{0.838992in}}%
\pgfpathlineto{\pgfqpoint{2.444453in}{0.839227in}}%
\pgfpathlineto{\pgfqpoint{2.445306in}{0.839462in}}%
\pgfpathlineto{\pgfqpoint{2.446159in}{0.839697in}}%
\pgfpathlineto{\pgfqpoint{2.447012in}{0.839932in}}%
\pgfpathlineto{\pgfqpoint{2.447864in}{0.840166in}}%
\pgfpathlineto{\pgfqpoint{2.448717in}{0.840401in}}%
\pgfpathlineto{\pgfqpoint{2.449570in}{0.840636in}}%
\pgfpathlineto{\pgfqpoint{2.450423in}{0.840871in}}%
\pgfpathlineto{\pgfqpoint{2.451276in}{0.841106in}}%
\pgfpathlineto{\pgfqpoint{2.452129in}{0.841340in}}%
\pgfpathlineto{\pgfqpoint{2.452982in}{0.841575in}}%
\pgfpathlineto{\pgfqpoint{2.453835in}{0.841810in}}%
\pgfpathlineto{\pgfqpoint{2.454687in}{0.842045in}}%
\pgfpathlineto{\pgfqpoint{2.455540in}{0.842280in}}%
\pgfpathlineto{\pgfqpoint{2.456393in}{0.842514in}}%
\pgfpathlineto{\pgfqpoint{2.457246in}{0.842749in}}%
\pgfpathlineto{\pgfqpoint{2.458099in}{0.842984in}}%
\pgfpathlineto{\pgfqpoint{2.458952in}{0.843208in}}%
\pgfpathlineto{\pgfqpoint{2.459805in}{0.843276in}}%
\pgfpathlineto{\pgfqpoint{2.460658in}{0.845580in}}%
\pgfpathlineto{\pgfqpoint{2.461510in}{0.845245in}}%
\pgfpathlineto{\pgfqpoint{2.462363in}{0.844466in}}%
\pgfpathlineto{\pgfqpoint{2.463216in}{0.843687in}}%
\pgfpathlineto{\pgfqpoint{2.464069in}{0.842908in}}%
\pgfpathlineto{\pgfqpoint{2.464922in}{0.842128in}}%
\pgfpathlineto{\pgfqpoint{2.465775in}{0.841349in}}%
\pgfpathlineto{\pgfqpoint{2.466628in}{0.840570in}}%
\pgfpathlineto{\pgfqpoint{2.467481in}{0.839791in}}%
\pgfpathlineto{\pgfqpoint{2.468333in}{0.839012in}}%
\pgfpathlineto{\pgfqpoint{2.469186in}{0.838233in}}%
\pgfpathlineto{\pgfqpoint{2.470039in}{0.837477in}}%
\pgfpathlineto{\pgfqpoint{2.470892in}{0.837344in}}%
\pgfpathlineto{\pgfqpoint{2.471745in}{0.837488in}}%
\pgfpathlineto{\pgfqpoint{2.472598in}{0.837632in}}%
\pgfpathlineto{\pgfqpoint{2.473451in}{0.837776in}}%
\pgfpathlineto{\pgfqpoint{2.474304in}{0.837920in}}%
\pgfpathlineto{\pgfqpoint{2.475157in}{0.838063in}}%
\pgfpathlineto{\pgfqpoint{2.476009in}{0.838207in}}%
\pgfpathlineto{\pgfqpoint{2.476862in}{0.838351in}}%
\pgfpathlineto{\pgfqpoint{2.477715in}{0.838495in}}%
\pgfpathlineto{\pgfqpoint{2.478568in}{0.838639in}}%
\pgfpathlineto{\pgfqpoint{2.479421in}{0.838783in}}%
\pgfpathlineto{\pgfqpoint{2.480274in}{0.838927in}}%
\pgfpathlineto{\pgfqpoint{2.481127in}{0.839071in}}%
\pgfpathlineto{\pgfqpoint{2.481980in}{0.839215in}}%
\pgfpathlineto{\pgfqpoint{2.482832in}{0.839359in}}%
\pgfpathlineto{\pgfqpoint{2.483685in}{0.839503in}}%
\pgfpathlineto{\pgfqpoint{2.484538in}{0.839647in}}%
\pgfpathlineto{\pgfqpoint{2.485391in}{0.839791in}}%
\pgfpathlineto{\pgfqpoint{2.486244in}{0.839935in}}%
\pgfpathlineto{\pgfqpoint{2.487097in}{0.840079in}}%
\pgfpathlineto{\pgfqpoint{2.487950in}{0.840223in}}%
\pgfpathlineto{\pgfqpoint{2.488803in}{0.840367in}}%
\pgfpathlineto{\pgfqpoint{2.489655in}{0.840511in}}%
\pgfpathlineto{\pgfqpoint{2.490508in}{0.840655in}}%
\pgfpathlineto{\pgfqpoint{2.491361in}{0.840799in}}%
\pgfpathlineto{\pgfqpoint{2.492214in}{0.840943in}}%
\pgfpathlineto{\pgfqpoint{2.493067in}{0.841087in}}%
\pgfpathlineto{\pgfqpoint{2.493920in}{0.841231in}}%
\pgfpathlineto{\pgfqpoint{2.494773in}{0.841375in}}%
\pgfpathlineto{\pgfqpoint{2.495626in}{0.841519in}}%
\pgfpathlineto{\pgfqpoint{2.496478in}{0.841662in}}%
\pgfpathlineto{\pgfqpoint{2.497331in}{0.841806in}}%
\pgfpathlineto{\pgfqpoint{2.498184in}{0.841950in}}%
\pgfpathlineto{\pgfqpoint{2.499037in}{0.842094in}}%
\pgfpathlineto{\pgfqpoint{2.499890in}{0.842238in}}%
\pgfpathlineto{\pgfqpoint{2.500743in}{0.842382in}}%
\pgfpathlineto{\pgfqpoint{2.501596in}{0.842526in}}%
\pgfpathlineto{\pgfqpoint{2.502449in}{0.842670in}}%
\pgfpathlineto{\pgfqpoint{2.503302in}{0.842814in}}%
\pgfpathlineto{\pgfqpoint{2.504154in}{0.842958in}}%
\pgfpathlineto{\pgfqpoint{2.505007in}{0.843102in}}%
\pgfpathlineto{\pgfqpoint{2.505860in}{0.843246in}}%
\pgfpathlineto{\pgfqpoint{2.506713in}{0.843390in}}%
\pgfpathlineto{\pgfqpoint{2.507566in}{0.843534in}}%
\pgfpathlineto{\pgfqpoint{2.508419in}{0.843678in}}%
\pgfpathlineto{\pgfqpoint{2.509272in}{0.843822in}}%
\pgfpathlineto{\pgfqpoint{2.510125in}{0.843966in}}%
\pgfpathlineto{\pgfqpoint{2.510977in}{0.844110in}}%
\pgfpathlineto{\pgfqpoint{2.511830in}{0.844254in}}%
\pgfpathlineto{\pgfqpoint{2.512683in}{0.844398in}}%
\pgfpathlineto{\pgfqpoint{2.513536in}{0.844542in}}%
\pgfpathlineto{\pgfqpoint{2.514389in}{0.844686in}}%
\pgfpathlineto{\pgfqpoint{2.515242in}{0.844830in}}%
\pgfpathlineto{\pgfqpoint{2.516095in}{0.844974in}}%
\pgfpathlineto{\pgfqpoint{2.516948in}{0.845117in}}%
\pgfpathlineto{\pgfqpoint{2.517800in}{0.845261in}}%
\pgfpathlineto{\pgfqpoint{2.518653in}{0.845405in}}%
\pgfpathlineto{\pgfqpoint{2.519506in}{0.845549in}}%
\pgfpathlineto{\pgfqpoint{2.520359in}{0.845693in}}%
\pgfpathlineto{\pgfqpoint{2.521212in}{0.845837in}}%
\pgfpathlineto{\pgfqpoint{2.522065in}{0.845981in}}%
\pgfpathlineto{\pgfqpoint{2.522918in}{0.846125in}}%
\pgfpathlineto{\pgfqpoint{2.523771in}{0.846269in}}%
\pgfpathlineto{\pgfqpoint{2.524623in}{0.846413in}}%
\pgfpathlineto{\pgfqpoint{2.525476in}{0.846543in}}%
\pgfpathlineto{\pgfqpoint{2.526329in}{0.846615in}}%
\pgfpathlineto{\pgfqpoint{2.527182in}{0.846682in}}%
\pgfpathlineto{\pgfqpoint{2.528035in}{0.846748in}}%
\pgfpathlineto{\pgfqpoint{2.528888in}{0.846815in}}%
\pgfpathlineto{\pgfqpoint{2.529741in}{0.846881in}}%
\pgfpathlineto{\pgfqpoint{2.530594in}{0.846947in}}%
\pgfpathlineto{\pgfqpoint{2.531446in}{0.846543in}}%
\pgfpathlineto{\pgfqpoint{2.532299in}{0.845511in}}%
\pgfpathlineto{\pgfqpoint{2.533152in}{0.844477in}}%
\pgfpathlineto{\pgfqpoint{2.534005in}{0.843442in}}%
\pgfpathlineto{\pgfqpoint{2.534858in}{0.842408in}}%
\pgfpathlineto{\pgfqpoint{2.535711in}{0.841373in}}%
\pgfpathlineto{\pgfqpoint{2.536564in}{0.840339in}}%
\pgfpathlineto{\pgfqpoint{2.537417in}{0.839304in}}%
\pgfpathlineto{\pgfqpoint{2.538270in}{0.838269in}}%
\pgfpathlineto{\pgfqpoint{2.539122in}{0.837837in}}%
\pgfpathlineto{\pgfqpoint{2.539975in}{0.847107in}}%
\pgfpathlineto{\pgfqpoint{2.540828in}{0.846630in}}%
\pgfpathlineto{\pgfqpoint{2.541681in}{0.847084in}}%
\pgfpathlineto{\pgfqpoint{2.542534in}{0.847492in}}%
\pgfpathlineto{\pgfqpoint{2.543387in}{0.847490in}}%
\pgfpathlineto{\pgfqpoint{2.544240in}{0.847484in}}%
\pgfpathlineto{\pgfqpoint{2.545093in}{0.847478in}}%
\pgfpathlineto{\pgfqpoint{2.545945in}{0.847471in}}%
\pgfpathlineto{\pgfqpoint{2.546798in}{0.847465in}}%
\pgfpathlineto{\pgfqpoint{2.547651in}{0.847459in}}%
\pgfpathlineto{\pgfqpoint{2.548504in}{0.847452in}}%
\pgfpathlineto{\pgfqpoint{2.549357in}{0.847446in}}%
\pgfpathlineto{\pgfqpoint{2.550210in}{0.847440in}}%
\pgfpathlineto{\pgfqpoint{2.551063in}{0.847433in}}%
\pgfpathlineto{\pgfqpoint{2.551916in}{0.847427in}}%
\pgfpathlineto{\pgfqpoint{2.552768in}{0.847420in}}%
\pgfpathlineto{\pgfqpoint{2.553621in}{0.847414in}}%
\pgfpathlineto{\pgfqpoint{2.554474in}{0.847408in}}%
\pgfpathlineto{\pgfqpoint{2.555327in}{0.847401in}}%
\pgfpathlineto{\pgfqpoint{2.556180in}{0.847395in}}%
\pgfpathlineto{\pgfqpoint{2.557033in}{0.847389in}}%
\pgfpathlineto{\pgfqpoint{2.557886in}{0.847382in}}%
\pgfpathlineto{\pgfqpoint{2.558739in}{0.847376in}}%
\pgfpathlineto{\pgfqpoint{2.559591in}{0.847370in}}%
\pgfpathlineto{\pgfqpoint{2.560444in}{0.847363in}}%
\pgfpathlineto{\pgfqpoint{2.561297in}{0.847357in}}%
\pgfpathlineto{\pgfqpoint{2.562150in}{0.847351in}}%
\pgfpathlineto{\pgfqpoint{2.563003in}{0.847344in}}%
\pgfpathlineto{\pgfqpoint{2.563856in}{0.847338in}}%
\pgfpathlineto{\pgfqpoint{2.564709in}{0.847434in}}%
\pgfpathlineto{\pgfqpoint{2.565562in}{0.847440in}}%
\pgfpathlineto{\pgfqpoint{2.566415in}{0.847435in}}%
\pgfpathlineto{\pgfqpoint{2.567267in}{0.847438in}}%
\pgfpathlineto{\pgfqpoint{2.568120in}{0.847446in}}%
\pgfpathlineto{\pgfqpoint{2.568973in}{0.847453in}}%
\pgfpathlineto{\pgfqpoint{2.569826in}{0.847460in}}%
\pgfpathlineto{\pgfqpoint{2.570679in}{0.847467in}}%
\pgfpathlineto{\pgfqpoint{2.571532in}{0.847475in}}%
\pgfpathlineto{\pgfqpoint{2.572385in}{0.847482in}}%
\pgfpathlineto{\pgfqpoint{2.573238in}{0.847489in}}%
\pgfpathlineto{\pgfqpoint{2.574090in}{0.847497in}}%
\pgfpathlineto{\pgfqpoint{2.574943in}{0.847504in}}%
\pgfpathlineto{\pgfqpoint{2.575796in}{0.847511in}}%
\pgfpathlineto{\pgfqpoint{2.576649in}{0.847518in}}%
\pgfpathlineto{\pgfqpoint{2.577502in}{0.847526in}}%
\pgfpathlineto{\pgfqpoint{2.578355in}{0.847533in}}%
\pgfpathlineto{\pgfqpoint{2.579208in}{0.847540in}}%
\pgfpathlineto{\pgfqpoint{2.580061in}{0.847548in}}%
\pgfpathlineto{\pgfqpoint{2.580913in}{0.847532in}}%
\pgfpathlineto{\pgfqpoint{2.581766in}{0.847384in}}%
\pgfpathlineto{\pgfqpoint{2.582619in}{0.847398in}}%
\pgfpathlineto{\pgfqpoint{2.583472in}{0.847418in}}%
\pgfpathlineto{\pgfqpoint{2.584325in}{0.847432in}}%
\pgfpathlineto{\pgfqpoint{2.585178in}{0.847669in}}%
\pgfpathlineto{\pgfqpoint{2.586031in}{0.848392in}}%
\pgfpathlineto{\pgfqpoint{2.586884in}{0.849128in}}%
\pgfpathlineto{\pgfqpoint{2.587736in}{0.849587in}}%
\pgfpathlineto{\pgfqpoint{2.588589in}{0.849604in}}%
\pgfpathlineto{\pgfqpoint{2.589442in}{0.849616in}}%
\pgfpathlineto{\pgfqpoint{2.590295in}{0.849628in}}%
\pgfpathlineto{\pgfqpoint{2.591148in}{0.849639in}}%
\pgfpathlineto{\pgfqpoint{2.592001in}{0.849651in}}%
\pgfpathlineto{\pgfqpoint{2.592854in}{0.849663in}}%
\pgfpathlineto{\pgfqpoint{2.593707in}{0.849654in}}%
\pgfpathlineto{\pgfqpoint{2.594560in}{0.849579in}}%
\pgfpathlineto{\pgfqpoint{2.595412in}{0.849499in}}%
\pgfpathlineto{\pgfqpoint{2.596265in}{0.849419in}}%
\pgfpathlineto{\pgfqpoint{2.597118in}{0.849338in}}%
\pgfpathlineto{\pgfqpoint{2.597971in}{0.849258in}}%
\pgfpathlineto{\pgfqpoint{2.598824in}{0.849178in}}%
\pgfpathlineto{\pgfqpoint{2.599677in}{0.849097in}}%
\pgfpathlineto{\pgfqpoint{2.600530in}{0.849017in}}%
\pgfpathlineto{\pgfqpoint{2.601383in}{0.848937in}}%
\pgfpathlineto{\pgfqpoint{2.602235in}{0.848909in}}%
\pgfpathlineto{\pgfqpoint{2.603088in}{0.849039in}}%
\pgfpathlineto{\pgfqpoint{2.603941in}{0.849179in}}%
\pgfpathlineto{\pgfqpoint{2.604794in}{0.849319in}}%
\pgfpathlineto{\pgfqpoint{2.605647in}{0.849460in}}%
\pgfpathlineto{\pgfqpoint{2.606500in}{0.849600in}}%
\pgfpathlineto{\pgfqpoint{2.607353in}{0.849740in}}%
\pgfpathlineto{\pgfqpoint{2.608206in}{0.849843in}}%
\pgfpathlineto{\pgfqpoint{2.609058in}{0.849880in}}%
\pgfpathlineto{\pgfqpoint{2.609911in}{0.849915in}}%
\pgfpathlineto{\pgfqpoint{2.610764in}{0.849950in}}%
\pgfpathlineto{\pgfqpoint{2.611617in}{0.849986in}}%
\pgfpathlineto{\pgfqpoint{2.612470in}{0.850021in}}%
\pgfpathlineto{\pgfqpoint{2.613323in}{0.850055in}}%
\pgfpathlineto{\pgfqpoint{2.614176in}{0.850052in}}%
\pgfpathlineto{\pgfqpoint{2.615029in}{0.850034in}}%
\pgfpathlineto{\pgfqpoint{2.615881in}{0.850017in}}%
\pgfpathlineto{\pgfqpoint{2.616734in}{0.849999in}}%
\pgfpathlineto{\pgfqpoint{2.617587in}{0.849981in}}%
\pgfpathlineto{\pgfqpoint{2.618440in}{0.849963in}}%
\pgfpathlineto{\pgfqpoint{2.619293in}{0.849945in}}%
\pgfpathlineto{\pgfqpoint{2.620146in}{0.849928in}}%
\pgfpathlineto{\pgfqpoint{2.620999in}{0.849910in}}%
\pgfpathlineto{\pgfqpoint{2.621852in}{0.849892in}}%
\pgfpathlineto{\pgfqpoint{2.622705in}{0.849874in}}%
\pgfpathlineto{\pgfqpoint{2.623557in}{0.849856in}}%
\pgfpathlineto{\pgfqpoint{2.624410in}{0.849839in}}%
\pgfpathlineto{\pgfqpoint{2.625263in}{0.849822in}}%
\pgfpathlineto{\pgfqpoint{2.626116in}{0.849814in}}%
\pgfpathlineto{\pgfqpoint{2.626969in}{0.849810in}}%
\pgfpathlineto{\pgfqpoint{2.627822in}{0.849805in}}%
\pgfpathlineto{\pgfqpoint{2.628675in}{0.849801in}}%
\pgfpathlineto{\pgfqpoint{2.629528in}{0.849796in}}%
\pgfpathlineto{\pgfqpoint{2.630380in}{0.849792in}}%
\pgfpathlineto{\pgfqpoint{2.631233in}{0.849787in}}%
\pgfpathlineto{\pgfqpoint{2.632086in}{0.849783in}}%
\pgfpathlineto{\pgfqpoint{2.632939in}{0.849784in}}%
\pgfpathlineto{\pgfqpoint{2.633792in}{0.849784in}}%
\pgfpathlineto{\pgfqpoint{2.634645in}{0.849779in}}%
\pgfpathlineto{\pgfqpoint{2.635498in}{0.849774in}}%
\pgfpathlineto{\pgfqpoint{2.636351in}{0.849770in}}%
\pgfpathlineto{\pgfqpoint{2.637203in}{0.849765in}}%
\pgfpathlineto{\pgfqpoint{2.638056in}{0.849760in}}%
\pgfpathlineto{\pgfqpoint{2.638909in}{0.849755in}}%
\pgfpathlineto{\pgfqpoint{2.639762in}{0.849751in}}%
\pgfpathlineto{\pgfqpoint{2.640615in}{0.849746in}}%
\pgfpathlineto{\pgfqpoint{2.641468in}{0.849741in}}%
\pgfpathlineto{\pgfqpoint{2.642321in}{0.849736in}}%
\pgfpathlineto{\pgfqpoint{2.643174in}{0.849732in}}%
\pgfpathlineto{\pgfqpoint{2.644026in}{0.849727in}}%
\pgfpathlineto{\pgfqpoint{2.644879in}{0.849722in}}%
\pgfpathlineto{\pgfqpoint{2.645732in}{0.849717in}}%
\pgfpathlineto{\pgfqpoint{2.646585in}{0.849713in}}%
\pgfpathlineto{\pgfqpoint{2.647438in}{0.849708in}}%
\pgfpathlineto{\pgfqpoint{2.648291in}{0.849703in}}%
\pgfpathlineto{\pgfqpoint{2.649144in}{0.849698in}}%
\pgfpathlineto{\pgfqpoint{2.649997in}{0.849694in}}%
\pgfpathlineto{\pgfqpoint{2.650849in}{0.849689in}}%
\pgfpathlineto{\pgfqpoint{2.651702in}{0.849684in}}%
\pgfpathlineto{\pgfqpoint{2.652555in}{0.849679in}}%
\pgfpathlineto{\pgfqpoint{2.653408in}{0.849675in}}%
\pgfpathlineto{\pgfqpoint{2.654261in}{0.849670in}}%
\pgfpathlineto{\pgfqpoint{2.655114in}{0.849665in}}%
\pgfpathlineto{\pgfqpoint{2.655967in}{0.849660in}}%
\pgfpathlineto{\pgfqpoint{2.656820in}{0.849656in}}%
\pgfpathlineto{\pgfqpoint{2.657673in}{0.849651in}}%
\pgfpathlineto{\pgfqpoint{2.658525in}{0.849646in}}%
\pgfpathlineto{\pgfqpoint{2.659378in}{0.849641in}}%
\pgfpathlineto{\pgfqpoint{2.660231in}{0.849637in}}%
\pgfpathlineto{\pgfqpoint{2.661084in}{0.849632in}}%
\pgfpathlineto{\pgfqpoint{2.661937in}{0.849627in}}%
\pgfpathlineto{\pgfqpoint{2.662790in}{0.849622in}}%
\pgfpathlineto{\pgfqpoint{2.663643in}{0.849618in}}%
\pgfpathlineto{\pgfqpoint{2.664496in}{0.849613in}}%
\pgfpathlineto{\pgfqpoint{2.665348in}{0.849608in}}%
\pgfpathlineto{\pgfqpoint{2.666201in}{0.849017in}}%
\pgfpathlineto{\pgfqpoint{2.667054in}{0.847870in}}%
\pgfpathlineto{\pgfqpoint{2.667907in}{0.847863in}}%
\pgfpathlineto{\pgfqpoint{2.668760in}{0.847958in}}%
\pgfpathlineto{\pgfqpoint{2.669613in}{0.848052in}}%
\pgfpathlineto{\pgfqpoint{2.670466in}{0.848146in}}%
\pgfpathlineto{\pgfqpoint{2.671319in}{0.848241in}}%
\pgfpathlineto{\pgfqpoint{2.672171in}{0.848335in}}%
\pgfpathlineto{\pgfqpoint{2.673024in}{0.848430in}}%
\pgfpathlineto{\pgfqpoint{2.673877in}{0.848524in}}%
\pgfpathlineto{\pgfqpoint{2.674730in}{0.848619in}}%
\pgfpathlineto{\pgfqpoint{2.675583in}{0.848713in}}%
\pgfpathlineto{\pgfqpoint{2.676436in}{0.848807in}}%
\pgfpathlineto{\pgfqpoint{2.677289in}{0.848902in}}%
\pgfpathlineto{\pgfqpoint{2.678142in}{0.848996in}}%
\pgfpathlineto{\pgfqpoint{2.678994in}{0.849091in}}%
\pgfpathlineto{\pgfqpoint{2.679847in}{0.849185in}}%
\pgfpathlineto{\pgfqpoint{2.680700in}{0.849279in}}%
\pgfpathlineto{\pgfqpoint{2.681553in}{0.849374in}}%
\pgfpathlineto{\pgfqpoint{2.682406in}{0.849468in}}%
\pgfpathlineto{\pgfqpoint{2.683259in}{0.849563in}}%
\pgfpathlineto{\pgfqpoint{2.684112in}{0.849613in}}%
\pgfpathlineto{\pgfqpoint{2.684965in}{0.849622in}}%
\pgfpathlineto{\pgfqpoint{2.685818in}{0.849631in}}%
\pgfpathlineto{\pgfqpoint{2.686670in}{0.849639in}}%
\pgfpathlineto{\pgfqpoint{2.687523in}{0.849648in}}%
\pgfpathlineto{\pgfqpoint{2.688376in}{0.849656in}}%
\pgfpathlineto{\pgfqpoint{2.689229in}{0.849665in}}%
\pgfpathlineto{\pgfqpoint{2.690082in}{0.849673in}}%
\pgfpathlineto{\pgfqpoint{2.690935in}{0.849682in}}%
\pgfpathlineto{\pgfqpoint{2.691788in}{0.849690in}}%
\pgfpathlineto{\pgfqpoint{2.692641in}{0.849699in}}%
\pgfpathlineto{\pgfqpoint{2.693493in}{0.849707in}}%
\pgfpathlineto{\pgfqpoint{2.694346in}{0.849716in}}%
\pgfpathlineto{\pgfqpoint{2.695199in}{0.849725in}}%
\pgfpathlineto{\pgfqpoint{2.696052in}{0.849731in}}%
\pgfpathlineto{\pgfqpoint{2.696905in}{0.849735in}}%
\pgfpathlineto{\pgfqpoint{2.697758in}{0.849739in}}%
\pgfpathlineto{\pgfqpoint{2.698611in}{0.849743in}}%
\pgfpathlineto{\pgfqpoint{2.699464in}{0.849747in}}%
\pgfpathlineto{\pgfqpoint{2.700316in}{0.849751in}}%
\pgfpathlineto{\pgfqpoint{2.701169in}{0.849755in}}%
\pgfpathlineto{\pgfqpoint{2.702022in}{0.849759in}}%
\pgfpathlineto{\pgfqpoint{2.702875in}{0.849672in}}%
\pgfpathlineto{\pgfqpoint{2.703728in}{0.848825in}}%
\pgfpathlineto{\pgfqpoint{2.704581in}{0.848063in}}%
\pgfpathlineto{\pgfqpoint{2.705434in}{0.848114in}}%
\pgfpathlineto{\pgfqpoint{2.706287in}{0.848165in}}%
\pgfpathlineto{\pgfqpoint{2.707139in}{0.848419in}}%
\pgfpathlineto{\pgfqpoint{2.707992in}{0.848176in}}%
\pgfpathlineto{\pgfqpoint{2.708845in}{0.848183in}}%
\pgfpathlineto{\pgfqpoint{2.709698in}{0.848180in}}%
\pgfpathlineto{\pgfqpoint{2.710551in}{0.848176in}}%
\pgfpathlineto{\pgfqpoint{2.711404in}{0.848173in}}%
\pgfpathlineto{\pgfqpoint{2.712257in}{0.848169in}}%
\pgfpathlineto{\pgfqpoint{2.713110in}{0.848166in}}%
\pgfpathlineto{\pgfqpoint{2.713963in}{0.848162in}}%
\pgfpathlineto{\pgfqpoint{2.714815in}{0.848159in}}%
\pgfpathlineto{\pgfqpoint{2.715668in}{0.848155in}}%
\pgfpathlineto{\pgfqpoint{2.716521in}{0.848151in}}%
\pgfpathlineto{\pgfqpoint{2.717374in}{0.848148in}}%
\pgfpathlineto{\pgfqpoint{2.718227in}{0.848144in}}%
\pgfpathlineto{\pgfqpoint{2.719080in}{0.848141in}}%
\pgfpathlineto{\pgfqpoint{2.719933in}{0.848137in}}%
\pgfpathlineto{\pgfqpoint{2.720786in}{0.848134in}}%
\pgfpathlineto{\pgfqpoint{2.721638in}{0.848130in}}%
\pgfpathlineto{\pgfqpoint{2.722491in}{0.848127in}}%
\pgfpathlineto{\pgfqpoint{2.723344in}{0.848123in}}%
\pgfpathlineto{\pgfqpoint{2.724197in}{0.848119in}}%
\pgfpathlineto{\pgfqpoint{2.725050in}{0.848121in}}%
\pgfpathlineto{\pgfqpoint{2.725903in}{0.848124in}}%
\pgfpathlineto{\pgfqpoint{2.726756in}{0.848127in}}%
\pgfpathlineto{\pgfqpoint{2.727609in}{0.848130in}}%
\pgfpathlineto{\pgfqpoint{2.728461in}{0.848133in}}%
\pgfpathlineto{\pgfqpoint{2.729314in}{0.848136in}}%
\pgfpathlineto{\pgfqpoint{2.730167in}{0.848139in}}%
\pgfpathlineto{\pgfqpoint{2.731020in}{0.848141in}}%
\pgfpathlineto{\pgfqpoint{2.731873in}{0.848144in}}%
\pgfpathlineto{\pgfqpoint{2.732726in}{0.848147in}}%
\pgfpathlineto{\pgfqpoint{2.733579in}{0.848150in}}%
\pgfpathlineto{\pgfqpoint{2.734432in}{0.848153in}}%
\pgfpathlineto{\pgfqpoint{2.735284in}{0.848156in}}%
\pgfpathlineto{\pgfqpoint{2.736137in}{0.848159in}}%
\pgfpathlineto{\pgfqpoint{2.736990in}{0.848162in}}%
\pgfpathlineto{\pgfqpoint{2.737843in}{0.848165in}}%
\pgfpathlineto{\pgfqpoint{2.738696in}{0.848168in}}%
\pgfpathlineto{\pgfqpoint{2.739549in}{0.848171in}}%
\pgfpathlineto{\pgfqpoint{2.740402in}{0.848174in}}%
\pgfpathlineto{\pgfqpoint{2.741255in}{0.848177in}}%
\pgfpathlineto{\pgfqpoint{2.742107in}{0.848180in}}%
\pgfpathlineto{\pgfqpoint{2.742960in}{0.848183in}}%
\pgfpathlineto{\pgfqpoint{2.743813in}{0.848186in}}%
\pgfpathlineto{\pgfqpoint{2.744666in}{0.848189in}}%
\pgfpathlineto{\pgfqpoint{2.745519in}{0.848191in}}%
\pgfpathlineto{\pgfqpoint{2.746372in}{0.848194in}}%
\pgfpathlineto{\pgfqpoint{2.747225in}{0.848197in}}%
\pgfpathlineto{\pgfqpoint{2.748078in}{0.848200in}}%
\pgfpathlineto{\pgfqpoint{2.748931in}{0.848203in}}%
\pgfpathlineto{\pgfqpoint{2.749783in}{0.848206in}}%
\pgfpathlineto{\pgfqpoint{2.750636in}{0.848210in}}%
\pgfpathlineto{\pgfqpoint{2.751489in}{0.848214in}}%
\pgfpathlineto{\pgfqpoint{2.752342in}{0.848218in}}%
\pgfpathlineto{\pgfqpoint{2.753195in}{0.848223in}}%
\pgfpathlineto{\pgfqpoint{2.754048in}{0.848227in}}%
\pgfpathlineto{\pgfqpoint{2.754901in}{0.848231in}}%
\pgfpathlineto{\pgfqpoint{2.755754in}{0.848235in}}%
\pgfpathlineto{\pgfqpoint{2.756606in}{0.848240in}}%
\pgfpathlineto{\pgfqpoint{2.757459in}{0.848244in}}%
\pgfpathlineto{\pgfqpoint{2.758312in}{0.848248in}}%
\pgfpathlineto{\pgfqpoint{2.759165in}{0.848252in}}%
\pgfpathlineto{\pgfqpoint{2.760018in}{0.848257in}}%
\pgfpathlineto{\pgfqpoint{2.760871in}{0.848261in}}%
\pgfpathlineto{\pgfqpoint{2.761724in}{0.848265in}}%
\pgfpathlineto{\pgfqpoint{2.762577in}{0.848269in}}%
\pgfpathlineto{\pgfqpoint{2.763429in}{0.848274in}}%
\pgfpathlineto{\pgfqpoint{2.764282in}{0.848278in}}%
\pgfpathlineto{\pgfqpoint{2.765135in}{0.848282in}}%
\pgfpathlineto{\pgfqpoint{2.765988in}{0.848287in}}%
\pgfpathlineto{\pgfqpoint{2.766841in}{0.848291in}}%
\pgfpathlineto{\pgfqpoint{2.767694in}{0.848295in}}%
\pgfpathlineto{\pgfqpoint{2.768547in}{0.848299in}}%
\pgfpathlineto{\pgfqpoint{2.769400in}{0.848304in}}%
\pgfpathlineto{\pgfqpoint{2.770252in}{0.848308in}}%
\pgfpathlineto{\pgfqpoint{2.771105in}{0.848312in}}%
\pgfpathlineto{\pgfqpoint{2.771958in}{0.848316in}}%
\pgfpathlineto{\pgfqpoint{2.772811in}{0.848321in}}%
\pgfpathlineto{\pgfqpoint{2.773664in}{0.848325in}}%
\pgfpathlineto{\pgfqpoint{2.774517in}{0.848329in}}%
\pgfpathlineto{\pgfqpoint{2.775370in}{0.848333in}}%
\pgfpathlineto{\pgfqpoint{2.776223in}{0.848338in}}%
\pgfpathlineto{\pgfqpoint{2.777076in}{0.848342in}}%
\pgfpathlineto{\pgfqpoint{2.777928in}{0.848346in}}%
\pgfpathlineto{\pgfqpoint{2.778781in}{0.848350in}}%
\pgfpathlineto{\pgfqpoint{2.779634in}{0.848355in}}%
\pgfpathlineto{\pgfqpoint{2.780487in}{0.848359in}}%
\pgfpathlineto{\pgfqpoint{2.781340in}{0.848363in}}%
\pgfpathlineto{\pgfqpoint{2.782193in}{0.848367in}}%
\pgfpathlineto{\pgfqpoint{2.783046in}{0.848370in}}%
\pgfpathlineto{\pgfqpoint{2.783899in}{0.848368in}}%
\pgfpathlineto{\pgfqpoint{2.784751in}{0.848365in}}%
\pgfpathlineto{\pgfqpoint{2.785604in}{0.848362in}}%
\pgfpathlineto{\pgfqpoint{2.786457in}{0.848359in}}%
\pgfpathlineto{\pgfqpoint{2.787310in}{0.848356in}}%
\pgfpathlineto{\pgfqpoint{2.788163in}{0.848353in}}%
\pgfpathlineto{\pgfqpoint{2.789016in}{0.848351in}}%
\pgfpathlineto{\pgfqpoint{2.789869in}{0.848348in}}%
\pgfpathlineto{\pgfqpoint{2.790722in}{0.848345in}}%
\pgfpathlineto{\pgfqpoint{2.791574in}{0.848342in}}%
\pgfpathlineto{\pgfqpoint{2.792427in}{0.848339in}}%
\pgfpathlineto{\pgfqpoint{2.793280in}{0.848337in}}%
\pgfpathlineto{\pgfqpoint{2.794133in}{0.848334in}}%
\pgfpathlineto{\pgfqpoint{2.794986in}{0.848331in}}%
\pgfpathlineto{\pgfqpoint{2.795839in}{0.848328in}}%
\pgfpathlineto{\pgfqpoint{2.796692in}{0.848325in}}%
\pgfpathlineto{\pgfqpoint{2.797545in}{0.848323in}}%
\pgfpathlineto{\pgfqpoint{2.798397in}{0.848320in}}%
\pgfpathlineto{\pgfqpoint{2.799250in}{0.848317in}}%
\pgfpathlineto{\pgfqpoint{2.800103in}{0.848314in}}%
\pgfpathlineto{\pgfqpoint{2.800956in}{0.848311in}}%
\pgfpathlineto{\pgfqpoint{2.801809in}{0.848307in}}%
\pgfpathlineto{\pgfqpoint{2.802662in}{0.848304in}}%
\pgfpathlineto{\pgfqpoint{2.803515in}{0.848300in}}%
\pgfpathlineto{\pgfqpoint{2.804368in}{0.848296in}}%
\pgfpathlineto{\pgfqpoint{2.805221in}{0.848292in}}%
\pgfpathlineto{\pgfqpoint{2.806073in}{0.848289in}}%
\pgfpathlineto{\pgfqpoint{2.806926in}{0.848285in}}%
\pgfpathlineto{\pgfqpoint{2.807779in}{0.848281in}}%
\pgfpathlineto{\pgfqpoint{2.808632in}{0.848278in}}%
\pgfpathlineto{\pgfqpoint{2.809485in}{0.848274in}}%
\pgfpathlineto{\pgfqpoint{2.810338in}{0.848270in}}%
\pgfpathlineto{\pgfqpoint{2.811191in}{0.848267in}}%
\pgfpathlineto{\pgfqpoint{2.812044in}{0.848263in}}%
\pgfpathlineto{\pgfqpoint{2.812896in}{0.848259in}}%
\pgfpathlineto{\pgfqpoint{2.813749in}{0.848255in}}%
\pgfpathlineto{\pgfqpoint{2.814602in}{0.848252in}}%
\pgfpathlineto{\pgfqpoint{2.815455in}{0.848248in}}%
\pgfpathlineto{\pgfqpoint{2.816308in}{0.848244in}}%
\pgfpathlineto{\pgfqpoint{2.817161in}{0.848241in}}%
\pgfpathlineto{\pgfqpoint{2.818014in}{0.848237in}}%
\pgfpathlineto{\pgfqpoint{2.818867in}{0.848233in}}%
\pgfpathlineto{\pgfqpoint{2.819719in}{0.848230in}}%
\pgfpathlineto{\pgfqpoint{2.820572in}{0.848226in}}%
\pgfpathlineto{\pgfqpoint{2.821425in}{0.848222in}}%
\pgfpathlineto{\pgfqpoint{2.822278in}{0.848219in}}%
\pgfpathlineto{\pgfqpoint{2.823131in}{0.848215in}}%
\pgfpathlineto{\pgfqpoint{2.823984in}{0.848211in}}%
\pgfpathlineto{\pgfqpoint{2.824837in}{0.848207in}}%
\pgfpathlineto{\pgfqpoint{2.825690in}{0.848204in}}%
\pgfpathlineto{\pgfqpoint{2.826542in}{0.848200in}}%
\pgfpathlineto{\pgfqpoint{2.827395in}{0.848196in}}%
\pgfpathlineto{\pgfqpoint{2.828248in}{0.848193in}}%
\pgfpathlineto{\pgfqpoint{2.829101in}{0.848189in}}%
\pgfpathlineto{\pgfqpoint{2.829954in}{0.848185in}}%
\pgfpathlineto{\pgfqpoint{2.830807in}{0.848182in}}%
\pgfpathlineto{\pgfqpoint{2.831660in}{0.848178in}}%
\pgfpathlineto{\pgfqpoint{2.832513in}{0.848174in}}%
\pgfpathlineto{\pgfqpoint{2.833366in}{0.848170in}}%
\pgfpathlineto{\pgfqpoint{2.834218in}{0.848153in}}%
\pgfpathlineto{\pgfqpoint{2.835071in}{0.848118in}}%
\pgfpathlineto{\pgfqpoint{2.835924in}{0.848083in}}%
\pgfpathlineto{\pgfqpoint{2.836777in}{0.848049in}}%
\pgfpathlineto{\pgfqpoint{2.837630in}{0.848014in}}%
\pgfpathlineto{\pgfqpoint{2.838483in}{0.847979in}}%
\pgfpathlineto{\pgfqpoint{2.839336in}{0.847945in}}%
\pgfpathlineto{\pgfqpoint{2.840189in}{0.847910in}}%
\pgfpathlineto{\pgfqpoint{2.841041in}{0.847875in}}%
\pgfpathlineto{\pgfqpoint{2.841894in}{0.847841in}}%
\pgfpathlineto{\pgfqpoint{2.842747in}{0.847806in}}%
\pgfpathlineto{\pgfqpoint{2.843600in}{0.847771in}}%
\pgfpathlineto{\pgfqpoint{2.844453in}{0.847737in}}%
\pgfpathlineto{\pgfqpoint{2.845306in}{0.847702in}}%
\pgfpathlineto{\pgfqpoint{2.846159in}{0.847667in}}%
\pgfpathlineto{\pgfqpoint{2.847012in}{0.847685in}}%
\pgfpathlineto{\pgfqpoint{2.847864in}{0.847989in}}%
\pgfpathlineto{\pgfqpoint{2.848717in}{0.848063in}}%
\pgfpathlineto{\pgfqpoint{2.849570in}{0.848022in}}%
\pgfpathlineto{\pgfqpoint{2.850423in}{0.848064in}}%
\pgfpathlineto{\pgfqpoint{2.851276in}{0.848072in}}%
\pgfpathlineto{\pgfqpoint{2.852129in}{0.848070in}}%
\pgfpathlineto{\pgfqpoint{2.852982in}{0.848069in}}%
\pgfpathlineto{\pgfqpoint{2.853835in}{0.848068in}}%
\pgfpathlineto{\pgfqpoint{2.854687in}{0.848066in}}%
\pgfpathlineto{\pgfqpoint{2.855540in}{0.848065in}}%
\pgfpathlineto{\pgfqpoint{2.856393in}{0.848078in}}%
\pgfpathlineto{\pgfqpoint{2.857246in}{0.848192in}}%
\pgfpathlineto{\pgfqpoint{2.858099in}{0.848324in}}%
\pgfpathlineto{\pgfqpoint{2.858952in}{0.848455in}}%
\pgfpathlineto{\pgfqpoint{2.859805in}{0.848582in}}%
\pgfpathlineto{\pgfqpoint{2.860658in}{0.848705in}}%
\pgfpathlineto{\pgfqpoint{2.861510in}{0.848828in}}%
\pgfpathlineto{\pgfqpoint{2.862363in}{0.848952in}}%
\pgfpathlineto{\pgfqpoint{2.863216in}{0.849075in}}%
\pgfpathlineto{\pgfqpoint{2.864069in}{0.849198in}}%
\pgfpathlineto{\pgfqpoint{2.864922in}{0.849321in}}%
\pgfpathlineto{\pgfqpoint{2.865775in}{0.849445in}}%
\pgfpathlineto{\pgfqpoint{2.866628in}{0.849568in}}%
\pgfpathlineto{\pgfqpoint{2.867481in}{0.849691in}}%
\pgfpathlineto{\pgfqpoint{2.868334in}{0.849814in}}%
\pgfpathlineto{\pgfqpoint{2.869186in}{0.849938in}}%
\pgfpathlineto{\pgfqpoint{2.870039in}{0.850061in}}%
\pgfpathlineto{\pgfqpoint{2.870892in}{0.848478in}}%
\pgfpathlineto{\pgfqpoint{2.871745in}{0.847949in}}%
\pgfpathlineto{\pgfqpoint{2.872598in}{0.848034in}}%
\pgfpathlineto{\pgfqpoint{2.873451in}{0.848119in}}%
\pgfpathlineto{\pgfqpoint{2.874304in}{0.848204in}}%
\pgfpathlineto{\pgfqpoint{2.875157in}{0.848289in}}%
\pgfpathlineto{\pgfqpoint{2.876009in}{0.848375in}}%
\pgfpathlineto{\pgfqpoint{2.876862in}{0.848460in}}%
\pgfpathlineto{\pgfqpoint{2.877715in}{0.848545in}}%
\pgfpathlineto{\pgfqpoint{2.878568in}{0.848630in}}%
\pgfpathlineto{\pgfqpoint{2.879421in}{0.848715in}}%
\pgfpathlineto{\pgfqpoint{2.880274in}{0.848800in}}%
\pgfpathlineto{\pgfqpoint{2.881127in}{0.848885in}}%
\pgfpathlineto{\pgfqpoint{2.881980in}{0.848970in}}%
\pgfpathlineto{\pgfqpoint{2.882832in}{0.849055in}}%
\pgfpathlineto{\pgfqpoint{2.883685in}{0.849141in}}%
\pgfpathlineto{\pgfqpoint{2.884538in}{0.849226in}}%
\pgfpathlineto{\pgfqpoint{2.885391in}{0.849311in}}%
\pgfpathlineto{\pgfqpoint{2.886244in}{0.849396in}}%
\pgfpathlineto{\pgfqpoint{2.887097in}{0.849481in}}%
\pgfpathlineto{\pgfqpoint{2.887950in}{0.849566in}}%
\pgfpathlineto{\pgfqpoint{2.888803in}{0.849651in}}%
\pgfpathlineto{\pgfqpoint{2.889655in}{0.849736in}}%
\pgfpathlineto{\pgfqpoint{2.890508in}{0.849821in}}%
\pgfpathlineto{\pgfqpoint{2.891361in}{0.849907in}}%
\pgfpathlineto{\pgfqpoint{2.892214in}{0.849992in}}%
\pgfpathlineto{\pgfqpoint{2.893067in}{0.850077in}}%
\pgfpathlineto{\pgfqpoint{2.893920in}{0.850162in}}%
\pgfpathlineto{\pgfqpoint{2.894773in}{0.850247in}}%
\pgfpathlineto{\pgfqpoint{2.895626in}{0.850332in}}%
\pgfpathlineto{\pgfqpoint{2.896479in}{0.850417in}}%
\pgfpathlineto{\pgfqpoint{2.897331in}{0.850502in}}%
\pgfpathlineto{\pgfqpoint{2.898184in}{0.850587in}}%
\pgfpathlineto{\pgfqpoint{2.899037in}{0.850673in}}%
\pgfpathlineto{\pgfqpoint{2.899890in}{0.850758in}}%
\pgfpathlineto{\pgfqpoint{2.900743in}{0.850843in}}%
\pgfpathlineto{\pgfqpoint{2.901596in}{0.848766in}}%
\pgfpathlineto{\pgfqpoint{2.902449in}{0.847574in}}%
\pgfpathlineto{\pgfqpoint{2.903302in}{0.847567in}}%
\pgfpathlineto{\pgfqpoint{2.904154in}{0.847560in}}%
\pgfpathlineto{\pgfqpoint{2.905007in}{0.847553in}}%
\pgfpathlineto{\pgfqpoint{2.905860in}{0.847546in}}%
\pgfpathlineto{\pgfqpoint{2.906713in}{0.847539in}}%
\pgfpathlineto{\pgfqpoint{2.907566in}{0.847533in}}%
\pgfpathlineto{\pgfqpoint{2.908419in}{0.847526in}}%
\pgfpathlineto{\pgfqpoint{2.909272in}{0.847519in}}%
\pgfpathlineto{\pgfqpoint{2.910125in}{0.847512in}}%
\pgfpathlineto{\pgfqpoint{2.910977in}{0.847505in}}%
\pgfpathlineto{\pgfqpoint{2.911830in}{0.847498in}}%
\pgfpathlineto{\pgfqpoint{2.912683in}{0.847491in}}%
\pgfpathlineto{\pgfqpoint{2.913536in}{0.847484in}}%
\pgfpathlineto{\pgfqpoint{2.914389in}{0.847477in}}%
\pgfpathlineto{\pgfqpoint{2.915242in}{0.847474in}}%
\pgfpathlineto{\pgfqpoint{2.916095in}{0.847504in}}%
\pgfpathlineto{\pgfqpoint{2.916948in}{0.847544in}}%
\pgfpathlineto{\pgfqpoint{2.917800in}{0.847583in}}%
\pgfpathlineto{\pgfqpoint{2.918653in}{0.847622in}}%
\pgfpathlineto{\pgfqpoint{2.919506in}{0.847662in}}%
\pgfpathlineto{\pgfqpoint{2.920359in}{0.847701in}}%
\pgfpathlineto{\pgfqpoint{2.921212in}{0.847740in}}%
\pgfpathlineto{\pgfqpoint{2.922065in}{0.847779in}}%
\pgfpathlineto{\pgfqpoint{2.922918in}{0.847819in}}%
\pgfpathlineto{\pgfqpoint{2.923771in}{0.847858in}}%
\pgfpathlineto{\pgfqpoint{2.924624in}{0.847897in}}%
\pgfpathlineto{\pgfqpoint{2.925476in}{0.847936in}}%
\pgfpathlineto{\pgfqpoint{2.926329in}{0.847976in}}%
\pgfpathlineto{\pgfqpoint{2.927182in}{0.848015in}}%
\pgfpathlineto{\pgfqpoint{2.928035in}{0.848054in}}%
\pgfpathlineto{\pgfqpoint{2.928888in}{0.848094in}}%
\pgfpathlineto{\pgfqpoint{2.929741in}{0.848133in}}%
\pgfpathlineto{\pgfqpoint{2.930594in}{0.848172in}}%
\pgfpathlineto{\pgfqpoint{2.931447in}{0.848211in}}%
\pgfpathlineto{\pgfqpoint{2.932299in}{0.848251in}}%
\pgfpathlineto{\pgfqpoint{2.933152in}{0.848290in}}%
\pgfpathlineto{\pgfqpoint{2.934005in}{0.848329in}}%
\pgfpathlineto{\pgfqpoint{2.934858in}{0.848368in}}%
\pgfpathlineto{\pgfqpoint{2.935711in}{0.848408in}}%
\pgfpathlineto{\pgfqpoint{2.936564in}{0.848447in}}%
\pgfpathlineto{\pgfqpoint{2.937417in}{0.848486in}}%
\pgfpathlineto{\pgfqpoint{2.938270in}{0.848526in}}%
\pgfpathlineto{\pgfqpoint{2.939122in}{0.848565in}}%
\pgfpathlineto{\pgfqpoint{2.939975in}{0.848604in}}%
\pgfpathlineto{\pgfqpoint{2.940828in}{0.848643in}}%
\pgfpathlineto{\pgfqpoint{2.941681in}{0.848683in}}%
\pgfpathlineto{\pgfqpoint{2.942534in}{0.848722in}}%
\pgfpathlineto{\pgfqpoint{2.943387in}{0.848761in}}%
\pgfpathlineto{\pgfqpoint{2.944240in}{0.848800in}}%
\pgfpathlineto{\pgfqpoint{2.945093in}{0.848840in}}%
\pgfpathlineto{\pgfqpoint{2.945945in}{0.848879in}}%
\pgfpathlineto{\pgfqpoint{2.946798in}{0.848918in}}%
\pgfpathlineto{\pgfqpoint{2.947651in}{0.848958in}}%
\pgfpathlineto{\pgfqpoint{2.948504in}{0.848997in}}%
\pgfpathlineto{\pgfqpoint{2.949357in}{0.849036in}}%
\pgfpathlineto{\pgfqpoint{2.950210in}{0.849075in}}%
\pgfpathlineto{\pgfqpoint{2.951063in}{0.849115in}}%
\pgfpathlineto{\pgfqpoint{2.951916in}{0.849154in}}%
\pgfpathlineto{\pgfqpoint{2.952769in}{0.849193in}}%
\pgfpathlineto{\pgfqpoint{2.953621in}{0.849232in}}%
\pgfpathlineto{\pgfqpoint{2.954474in}{0.849272in}}%
\pgfpathlineto{\pgfqpoint{2.955327in}{0.849311in}}%
\pgfpathlineto{\pgfqpoint{2.956180in}{0.849350in}}%
\pgfpathlineto{\pgfqpoint{2.957033in}{0.849389in}}%
\pgfpathlineto{\pgfqpoint{2.957886in}{0.849429in}}%
\pgfpathlineto{\pgfqpoint{2.958739in}{0.849468in}}%
\pgfpathlineto{\pgfqpoint{2.959592in}{0.849507in}}%
\pgfpathlineto{\pgfqpoint{2.960444in}{0.849547in}}%
\pgfpathlineto{\pgfqpoint{2.961297in}{0.849586in}}%
\pgfpathlineto{\pgfqpoint{2.962150in}{0.849625in}}%
\pgfpathlineto{\pgfqpoint{2.963003in}{0.849664in}}%
\pgfpathlineto{\pgfqpoint{2.963856in}{0.849704in}}%
\pgfpathlineto{\pgfqpoint{2.964709in}{0.849743in}}%
\pgfpathlineto{\pgfqpoint{2.965562in}{0.849782in}}%
\pgfpathlineto{\pgfqpoint{2.966415in}{0.849821in}}%
\pgfpathlineto{\pgfqpoint{2.967267in}{0.849861in}}%
\pgfpathlineto{\pgfqpoint{2.968120in}{0.849900in}}%
\pgfpathlineto{\pgfqpoint{2.968973in}{0.849939in}}%
\pgfpathlineto{\pgfqpoint{2.969826in}{0.849837in}}%
\pgfpathlineto{\pgfqpoint{2.970679in}{0.848041in}}%
\pgfpathlineto{\pgfqpoint{2.971532in}{0.847689in}}%
\pgfpathlineto{\pgfqpoint{2.972385in}{0.850079in}}%
\pgfpathlineto{\pgfqpoint{2.973238in}{0.850497in}}%
\pgfpathlineto{\pgfqpoint{2.974090in}{0.850512in}}%
\pgfpathlineto{\pgfqpoint{2.974943in}{0.850528in}}%
\pgfpathlineto{\pgfqpoint{2.975796in}{0.850531in}}%
\pgfpathlineto{\pgfqpoint{2.976649in}{0.850534in}}%
\pgfpathlineto{\pgfqpoint{2.977502in}{0.850537in}}%
\pgfpathlineto{\pgfqpoint{2.978355in}{0.850538in}}%
\pgfpathlineto{\pgfqpoint{2.979208in}{0.850535in}}%
\pgfpathlineto{\pgfqpoint{2.980061in}{0.850532in}}%
\pgfpathlineto{\pgfqpoint{2.980913in}{0.850528in}}%
\pgfpathlineto{\pgfqpoint{2.981766in}{0.850525in}}%
\pgfpathlineto{\pgfqpoint{2.982619in}{0.850522in}}%
\pgfpathlineto{\pgfqpoint{2.983472in}{0.850518in}}%
\pgfpathlineto{\pgfqpoint{2.984325in}{0.850515in}}%
\pgfpathlineto{\pgfqpoint{2.985178in}{0.850512in}}%
\pgfpathlineto{\pgfqpoint{2.986031in}{0.850508in}}%
\pgfpathlineto{\pgfqpoint{2.986884in}{0.850505in}}%
\pgfpathlineto{\pgfqpoint{2.987737in}{0.850502in}}%
\pgfpathlineto{\pgfqpoint{2.988589in}{0.850498in}}%
\pgfpathlineto{\pgfqpoint{2.989442in}{0.850495in}}%
\pgfpathlineto{\pgfqpoint{2.990295in}{0.850492in}}%
\pgfpathlineto{\pgfqpoint{2.991148in}{0.848810in}}%
\pgfpathlineto{\pgfqpoint{2.992001in}{0.846984in}}%
\pgfpathlineto{\pgfqpoint{2.992854in}{0.848899in}}%
\pgfpathlineto{\pgfqpoint{2.993707in}{0.850225in}}%
\pgfpathlineto{\pgfqpoint{2.994560in}{0.850080in}}%
\pgfpathlineto{\pgfqpoint{2.995412in}{0.851052in}}%
\pgfpathlineto{\pgfqpoint{2.996265in}{0.851239in}}%
\pgfpathlineto{\pgfqpoint{2.997118in}{0.851198in}}%
\pgfpathlineto{\pgfqpoint{2.997971in}{0.851160in}}%
\pgfpathlineto{\pgfqpoint{2.998824in}{0.851131in}}%
\pgfpathlineto{\pgfqpoint{2.999677in}{0.851101in}}%
\pgfpathlineto{\pgfqpoint{3.000530in}{0.851072in}}%
\pgfpathlineto{\pgfqpoint{3.001383in}{0.851043in}}%
\pgfpathlineto{\pgfqpoint{3.002235in}{0.851013in}}%
\pgfpathlineto{\pgfqpoint{3.003088in}{0.850984in}}%
\pgfpathlineto{\pgfqpoint{3.003941in}{0.850984in}}%
\pgfpathlineto{\pgfqpoint{3.004794in}{0.851010in}}%
\pgfpathlineto{\pgfqpoint{3.005647in}{0.851036in}}%
\pgfpathlineto{\pgfqpoint{3.006500in}{0.851062in}}%
\pgfpathlineto{\pgfqpoint{3.007353in}{0.851088in}}%
\pgfpathlineto{\pgfqpoint{3.008206in}{0.851113in}}%
\pgfpathlineto{\pgfqpoint{3.009058in}{0.851139in}}%
\pgfpathlineto{\pgfqpoint{3.009911in}{0.851165in}}%
\pgfpathlineto{\pgfqpoint{3.010764in}{0.851191in}}%
\pgfpathlineto{\pgfqpoint{3.011617in}{0.851217in}}%
\pgfpathlineto{\pgfqpoint{3.012470in}{0.851243in}}%
\pgfpathlineto{\pgfqpoint{3.013323in}{0.851268in}}%
\pgfpathlineto{\pgfqpoint{3.014176in}{0.851294in}}%
\pgfpathlineto{\pgfqpoint{3.015029in}{0.851320in}}%
\pgfpathlineto{\pgfqpoint{3.015882in}{0.851346in}}%
\pgfpathlineto{\pgfqpoint{3.016734in}{0.851372in}}%
\pgfpathlineto{\pgfqpoint{3.017587in}{0.851398in}}%
\pgfpathlineto{\pgfqpoint{3.018440in}{0.851406in}}%
\pgfpathlineto{\pgfqpoint{3.019293in}{0.851376in}}%
\pgfpathlineto{\pgfqpoint{3.020146in}{0.851345in}}%
\pgfpathlineto{\pgfqpoint{3.020999in}{0.851313in}}%
\pgfpathlineto{\pgfqpoint{3.021852in}{0.851282in}}%
\pgfpathlineto{\pgfqpoint{3.022705in}{0.851250in}}%
\pgfpathlineto{\pgfqpoint{3.023557in}{0.851219in}}%
\pgfpathlineto{\pgfqpoint{3.024410in}{0.851187in}}%
\pgfpathlineto{\pgfqpoint{3.025263in}{0.851156in}}%
\pgfpathlineto{\pgfqpoint{3.026116in}{0.851124in}}%
\pgfpathlineto{\pgfqpoint{3.026969in}{0.851092in}}%
\pgfpathlineto{\pgfqpoint{3.027822in}{0.851061in}}%
\pgfpathlineto{\pgfqpoint{3.028675in}{0.851029in}}%
\pgfpathlineto{\pgfqpoint{3.029528in}{0.850998in}}%
\pgfpathlineto{\pgfqpoint{3.030380in}{0.850966in}}%
\pgfpathlineto{\pgfqpoint{3.031233in}{0.850935in}}%
\pgfpathlineto{\pgfqpoint{3.032086in}{0.850903in}}%
\pgfpathlineto{\pgfqpoint{3.032939in}{0.850872in}}%
\pgfpathlineto{\pgfqpoint{3.033792in}{0.850840in}}%
\pgfpathlineto{\pgfqpoint{3.034645in}{0.850809in}}%
\pgfpathlineto{\pgfqpoint{3.035498in}{0.850777in}}%
\pgfpathlineto{\pgfqpoint{3.036351in}{0.850746in}}%
\pgfpathlineto{\pgfqpoint{3.037203in}{0.850714in}}%
\pgfpathlineto{\pgfqpoint{3.038056in}{0.850683in}}%
\pgfpathlineto{\pgfqpoint{3.038909in}{0.850651in}}%
\pgfpathlineto{\pgfqpoint{3.039762in}{0.850619in}}%
\pgfpathlineto{\pgfqpoint{3.040615in}{0.850588in}}%
\pgfpathlineto{\pgfqpoint{3.041468in}{0.850556in}}%
\pgfpathlineto{\pgfqpoint{3.042321in}{0.850525in}}%
\pgfpathlineto{\pgfqpoint{3.043174in}{0.850493in}}%
\pgfpathlineto{\pgfqpoint{3.044027in}{0.850462in}}%
\pgfpathlineto{\pgfqpoint{3.044879in}{0.850430in}}%
\pgfpathlineto{\pgfqpoint{3.045732in}{0.850399in}}%
\pgfpathlineto{\pgfqpoint{3.046585in}{0.850367in}}%
\pgfpathlineto{\pgfqpoint{3.047438in}{0.850336in}}%
\pgfpathlineto{\pgfqpoint{3.048291in}{0.850304in}}%
\pgfpathlineto{\pgfqpoint{3.049144in}{0.850273in}}%
\pgfpathlineto{\pgfqpoint{3.049997in}{0.850241in}}%
\pgfpathlineto{\pgfqpoint{3.050850in}{0.850210in}}%
\pgfpathlineto{\pgfqpoint{3.051702in}{0.850178in}}%
\pgfpathlineto{\pgfqpoint{3.052555in}{0.850146in}}%
\pgfpathlineto{\pgfqpoint{3.053408in}{0.850115in}}%
\pgfpathlineto{\pgfqpoint{3.054261in}{0.850083in}}%
\pgfpathlineto{\pgfqpoint{3.055114in}{0.850052in}}%
\pgfpathlineto{\pgfqpoint{3.055967in}{0.850020in}}%
\pgfpathlineto{\pgfqpoint{3.056820in}{0.849989in}}%
\pgfpathlineto{\pgfqpoint{3.057673in}{0.849957in}}%
\pgfpathlineto{\pgfqpoint{3.058525in}{0.849927in}}%
\pgfpathlineto{\pgfqpoint{3.059378in}{0.849906in}}%
\pgfpathlineto{\pgfqpoint{3.060231in}{0.849888in}}%
\pgfpathlineto{\pgfqpoint{3.061084in}{0.849869in}}%
\pgfpathlineto{\pgfqpoint{3.061937in}{0.849851in}}%
\pgfpathlineto{\pgfqpoint{3.062790in}{0.849833in}}%
\pgfpathlineto{\pgfqpoint{3.063643in}{0.849814in}}%
\pgfpathlineto{\pgfqpoint{3.064496in}{0.849796in}}%
\pgfpathlineto{\pgfqpoint{3.065348in}{0.849778in}}%
\pgfpathlineto{\pgfqpoint{3.066201in}{0.849759in}}%
\pgfpathlineto{\pgfqpoint{3.067054in}{0.849741in}}%
\pgfpathlineto{\pgfqpoint{3.067907in}{0.849723in}}%
\pgfpathlineto{\pgfqpoint{3.068760in}{0.849704in}}%
\pgfpathlineto{\pgfqpoint{3.069613in}{0.849686in}}%
\pgfpathlineto{\pgfqpoint{3.070466in}{0.849668in}}%
\pgfpathlineto{\pgfqpoint{3.071319in}{0.849649in}}%
\pgfpathlineto{\pgfqpoint{3.072171in}{0.849631in}}%
\pgfpathlineto{\pgfqpoint{3.073024in}{0.849612in}}%
\pgfpathlineto{\pgfqpoint{3.073877in}{0.849594in}}%
\pgfpathlineto{\pgfqpoint{3.074730in}{0.849576in}}%
\pgfpathlineto{\pgfqpoint{3.075583in}{0.849557in}}%
\pgfpathlineto{\pgfqpoint{3.076436in}{0.849539in}}%
\pgfpathlineto{\pgfqpoint{3.077289in}{0.849521in}}%
\pgfpathlineto{\pgfqpoint{3.078142in}{0.849502in}}%
\pgfpathlineto{\pgfqpoint{3.078995in}{0.849484in}}%
\pgfpathlineto{\pgfqpoint{3.079847in}{0.849466in}}%
\pgfpathlineto{\pgfqpoint{3.080700in}{0.849447in}}%
\pgfpathlineto{\pgfqpoint{3.081553in}{0.849429in}}%
\pgfpathlineto{\pgfqpoint{3.082406in}{0.849411in}}%
\pgfpathlineto{\pgfqpoint{3.083259in}{0.849392in}}%
\pgfpathlineto{\pgfqpoint{3.084112in}{0.849374in}}%
\pgfpathlineto{\pgfqpoint{3.084965in}{0.849355in}}%
\pgfpathlineto{\pgfqpoint{3.085818in}{0.849337in}}%
\pgfpathlineto{\pgfqpoint{3.086670in}{0.849319in}}%
\pgfpathlineto{\pgfqpoint{3.087523in}{0.849300in}}%
\pgfpathlineto{\pgfqpoint{3.088376in}{0.849282in}}%
\pgfpathlineto{\pgfqpoint{3.089229in}{0.849264in}}%
\pgfpathlineto{\pgfqpoint{3.090082in}{0.849245in}}%
\pgfpathlineto{\pgfqpoint{3.090935in}{0.849227in}}%
\pgfpathlineto{\pgfqpoint{3.091788in}{0.849208in}}%
\pgfpathlineto{\pgfqpoint{3.092641in}{0.849188in}}%
\pgfpathlineto{\pgfqpoint{3.093493in}{0.849168in}}%
\pgfpathlineto{\pgfqpoint{3.094346in}{0.849148in}}%
\pgfpathlineto{\pgfqpoint{3.095199in}{0.849128in}}%
\pgfpathlineto{\pgfqpoint{3.096052in}{0.849108in}}%
\pgfpathlineto{\pgfqpoint{3.096905in}{0.849088in}}%
\pgfpathlineto{\pgfqpoint{3.097758in}{0.849068in}}%
\pgfpathlineto{\pgfqpoint{3.098611in}{0.849048in}}%
\pgfpathlineto{\pgfqpoint{3.099464in}{0.849028in}}%
\pgfpathlineto{\pgfqpoint{3.100316in}{0.849008in}}%
\pgfpathlineto{\pgfqpoint{3.101169in}{0.848988in}}%
\pgfpathlineto{\pgfqpoint{3.102022in}{0.848968in}}%
\pgfpathlineto{\pgfqpoint{3.102875in}{0.848948in}}%
\pgfpathlineto{\pgfqpoint{3.103728in}{0.848928in}}%
\pgfpathlineto{\pgfqpoint{3.104581in}{0.848907in}}%
\pgfpathlineto{\pgfqpoint{3.105434in}{0.848887in}}%
\pgfpathlineto{\pgfqpoint{3.106287in}{0.848867in}}%
\pgfpathlineto{\pgfqpoint{3.107140in}{0.848847in}}%
\pgfpathlineto{\pgfqpoint{3.107992in}{0.848827in}}%
\pgfpathlineto{\pgfqpoint{3.108845in}{0.848807in}}%
\pgfpathlineto{\pgfqpoint{3.109698in}{0.848787in}}%
\pgfpathlineto{\pgfqpoint{3.110551in}{0.848767in}}%
\pgfpathlineto{\pgfqpoint{3.111404in}{0.848747in}}%
\pgfpathlineto{\pgfqpoint{3.112257in}{0.848727in}}%
\pgfpathlineto{\pgfqpoint{3.113110in}{0.848707in}}%
\pgfpathlineto{\pgfqpoint{3.113963in}{0.848687in}}%
\pgfpathlineto{\pgfqpoint{3.114815in}{0.848673in}}%
\pgfpathlineto{\pgfqpoint{3.115668in}{0.848659in}}%
\pgfpathlineto{\pgfqpoint{3.116521in}{0.848646in}}%
\pgfpathlineto{\pgfqpoint{3.117374in}{0.848633in}}%
\pgfpathlineto{\pgfqpoint{3.118227in}{0.848619in}}%
\pgfpathlineto{\pgfqpoint{3.119080in}{0.848606in}}%
\pgfpathlineto{\pgfqpoint{3.119933in}{0.848593in}}%
\pgfpathlineto{\pgfqpoint{3.120786in}{0.848579in}}%
\pgfpathlineto{\pgfqpoint{3.121638in}{0.848566in}}%
\pgfpathlineto{\pgfqpoint{3.122491in}{0.848553in}}%
\pgfpathlineto{\pgfqpoint{3.123344in}{0.848540in}}%
\pgfpathlineto{\pgfqpoint{3.124197in}{0.848526in}}%
\pgfpathlineto{\pgfqpoint{3.125050in}{0.848513in}}%
\pgfpathlineto{\pgfqpoint{3.125903in}{0.848500in}}%
\pgfpathlineto{\pgfqpoint{3.126756in}{0.848486in}}%
\pgfpathlineto{\pgfqpoint{3.127609in}{0.848473in}}%
\pgfpathlineto{\pgfqpoint{3.128461in}{0.848460in}}%
\pgfpathlineto{\pgfqpoint{3.129314in}{0.848446in}}%
\pgfpathlineto{\pgfqpoint{3.130167in}{0.848433in}}%
\pgfpathlineto{\pgfqpoint{3.131020in}{0.848420in}}%
\pgfpathlineto{\pgfqpoint{3.131873in}{0.848407in}}%
\pgfpathlineto{\pgfqpoint{3.132726in}{0.848393in}}%
\pgfpathlineto{\pgfqpoint{3.133579in}{0.848380in}}%
\pgfpathlineto{\pgfqpoint{3.134432in}{0.848377in}}%
\pgfpathlineto{\pgfqpoint{3.135285in}{0.848464in}}%
\pgfpathlineto{\pgfqpoint{3.136137in}{0.848556in}}%
\pgfpathlineto{\pgfqpoint{3.136990in}{0.848552in}}%
\pgfpathlineto{\pgfqpoint{3.137843in}{0.848531in}}%
\pgfpathlineto{\pgfqpoint{3.138696in}{0.848511in}}%
\pgfpathlineto{\pgfqpoint{3.139549in}{0.848491in}}%
\pgfpathlineto{\pgfqpoint{3.140402in}{0.848470in}}%
\pgfpathlineto{\pgfqpoint{3.141255in}{0.848450in}}%
\pgfpathlineto{\pgfqpoint{3.142108in}{0.848430in}}%
\pgfpathlineto{\pgfqpoint{3.142960in}{0.848409in}}%
\pgfpathlineto{\pgfqpoint{3.143813in}{0.848389in}}%
\pgfpathlineto{\pgfqpoint{3.144666in}{0.848368in}}%
\pgfpathlineto{\pgfqpoint{3.145519in}{0.848348in}}%
\pgfpathlineto{\pgfqpoint{3.146372in}{0.848328in}}%
\pgfpathlineto{\pgfqpoint{3.147225in}{0.848307in}}%
\pgfpathlineto{\pgfqpoint{3.148078in}{0.848287in}}%
\pgfpathlineto{\pgfqpoint{3.148931in}{0.848267in}}%
\pgfpathlineto{\pgfqpoint{3.149783in}{0.848246in}}%
\pgfpathlineto{\pgfqpoint{3.150636in}{0.848226in}}%
\pgfpathlineto{\pgfqpoint{3.151489in}{0.848206in}}%
\pgfpathlineto{\pgfqpoint{3.152342in}{0.848185in}}%
\pgfpathlineto{\pgfqpoint{3.153195in}{0.848165in}}%
\pgfpathlineto{\pgfqpoint{3.154048in}{0.848145in}}%
\pgfpathlineto{\pgfqpoint{3.154901in}{0.848124in}}%
\pgfpathlineto{\pgfqpoint{3.155754in}{0.848104in}}%
\pgfpathlineto{\pgfqpoint{3.156606in}{0.848084in}}%
\pgfpathlineto{\pgfqpoint{3.157459in}{0.848063in}}%
\pgfpathlineto{\pgfqpoint{3.158312in}{0.848043in}}%
\pgfpathlineto{\pgfqpoint{3.159165in}{0.848023in}}%
\pgfpathlineto{\pgfqpoint{3.160018in}{0.848002in}}%
\pgfpathlineto{\pgfqpoint{3.160871in}{0.847982in}}%
\pgfpathlineto{\pgfqpoint{3.161724in}{0.847961in}}%
\pgfpathlineto{\pgfqpoint{3.162577in}{0.847941in}}%
\pgfpathlineto{\pgfqpoint{3.163430in}{0.847921in}}%
\pgfpathlineto{\pgfqpoint{3.164282in}{0.847900in}}%
\pgfpathlineto{\pgfqpoint{3.165135in}{0.847880in}}%
\pgfpathlineto{\pgfqpoint{3.165988in}{0.847860in}}%
\pgfpathlineto{\pgfqpoint{3.166841in}{0.847839in}}%
\pgfpathlineto{\pgfqpoint{3.167694in}{0.847819in}}%
\pgfpathlineto{\pgfqpoint{3.168547in}{0.847799in}}%
\pgfpathlineto{\pgfqpoint{3.169400in}{0.847778in}}%
\pgfpathlineto{\pgfqpoint{3.170253in}{0.847758in}}%
\pgfpathlineto{\pgfqpoint{3.171105in}{0.847738in}}%
\pgfpathlineto{\pgfqpoint{3.171958in}{0.847717in}}%
\pgfpathlineto{\pgfqpoint{3.172811in}{0.847697in}}%
\pgfpathlineto{\pgfqpoint{3.173664in}{0.847677in}}%
\pgfpathlineto{\pgfqpoint{3.174517in}{0.847656in}}%
\pgfpathlineto{\pgfqpoint{3.175370in}{0.847636in}}%
\pgfpathlineto{\pgfqpoint{3.176223in}{0.847615in}}%
\pgfpathlineto{\pgfqpoint{3.177076in}{0.847595in}}%
\pgfpathlineto{\pgfqpoint{3.177928in}{0.847575in}}%
\pgfpathlineto{\pgfqpoint{3.178781in}{0.847554in}}%
\pgfpathlineto{\pgfqpoint{3.179634in}{0.847534in}}%
\pgfpathlineto{\pgfqpoint{3.180487in}{0.847514in}}%
\pgfpathlineto{\pgfqpoint{3.181340in}{0.847493in}}%
\pgfpathlineto{\pgfqpoint{3.182193in}{0.847473in}}%
\pgfpathlineto{\pgfqpoint{3.183046in}{0.847453in}}%
\pgfpathlineto{\pgfqpoint{3.183899in}{0.847432in}}%
\pgfpathlineto{\pgfqpoint{3.184751in}{0.847412in}}%
\pgfpathlineto{\pgfqpoint{3.185604in}{0.847392in}}%
\pgfpathlineto{\pgfqpoint{3.186457in}{0.847371in}}%
\pgfpathlineto{\pgfqpoint{3.187310in}{0.847351in}}%
\pgfpathlineto{\pgfqpoint{3.188163in}{0.847331in}}%
\pgfpathlineto{\pgfqpoint{3.189016in}{0.847310in}}%
\pgfpathlineto{\pgfqpoint{3.189869in}{0.847290in}}%
\pgfpathlineto{\pgfqpoint{3.190722in}{0.847270in}}%
\pgfpathlineto{\pgfqpoint{3.191574in}{0.847249in}}%
\pgfpathlineto{\pgfqpoint{3.192427in}{0.847229in}}%
\pgfpathlineto{\pgfqpoint{3.193280in}{0.847208in}}%
\pgfpathlineto{\pgfqpoint{3.194133in}{0.847188in}}%
\pgfpathlineto{\pgfqpoint{3.194986in}{0.847168in}}%
\pgfpathlineto{\pgfqpoint{3.195839in}{0.847147in}}%
\pgfpathlineto{\pgfqpoint{3.196692in}{0.847127in}}%
\pgfpathlineto{\pgfqpoint{3.197545in}{0.847107in}}%
\pgfpathlineto{\pgfqpoint{3.198398in}{0.847086in}}%
\pgfpathlineto{\pgfqpoint{3.199250in}{0.847066in}}%
\pgfpathlineto{\pgfqpoint{3.200103in}{0.847046in}}%
\pgfpathlineto{\pgfqpoint{3.200956in}{0.847025in}}%
\pgfpathlineto{\pgfqpoint{3.201809in}{0.847005in}}%
\pgfpathlineto{\pgfqpoint{3.202662in}{0.846985in}}%
\pgfpathlineto{\pgfqpoint{3.203515in}{0.846964in}}%
\pgfpathlineto{\pgfqpoint{3.204368in}{0.846944in}}%
\pgfpathlineto{\pgfqpoint{3.205221in}{0.846924in}}%
\pgfpathlineto{\pgfqpoint{3.206073in}{0.846903in}}%
\pgfpathlineto{\pgfqpoint{3.206926in}{0.846883in}}%
\pgfpathlineto{\pgfqpoint{3.207779in}{0.846856in}}%
\pgfpathlineto{\pgfqpoint{3.208632in}{0.846792in}}%
\pgfpathlineto{\pgfqpoint{3.209485in}{0.846722in}}%
\pgfpathlineto{\pgfqpoint{3.210338in}{0.846653in}}%
\pgfpathlineto{\pgfqpoint{3.211191in}{0.846584in}}%
\pgfpathlineto{\pgfqpoint{3.212044in}{0.846514in}}%
\pgfpathlineto{\pgfqpoint{3.212896in}{0.846445in}}%
\pgfpathlineto{\pgfqpoint{3.213749in}{0.846375in}}%
\pgfpathlineto{\pgfqpoint{3.214602in}{0.846306in}}%
\pgfpathlineto{\pgfqpoint{3.215455in}{0.846236in}}%
\pgfpathlineto{\pgfqpoint{3.216308in}{0.846167in}}%
\pgfpathlineto{\pgfqpoint{3.217161in}{0.846098in}}%
\pgfpathlineto{\pgfqpoint{3.218014in}{0.846028in}}%
\pgfpathlineto{\pgfqpoint{3.218867in}{0.845959in}}%
\pgfpathlineto{\pgfqpoint{3.219719in}{0.845889in}}%
\pgfpathlineto{\pgfqpoint{3.220572in}{0.845820in}}%
\pgfpathlineto{\pgfqpoint{3.221425in}{0.845751in}}%
\pgfpathlineto{\pgfqpoint{3.222278in}{0.845681in}}%
\pgfpathlineto{\pgfqpoint{3.223131in}{0.845612in}}%
\pgfpathlineto{\pgfqpoint{3.223984in}{0.845542in}}%
\pgfpathlineto{\pgfqpoint{3.224837in}{0.845473in}}%
\pgfpathlineto{\pgfqpoint{3.225690in}{0.845403in}}%
\pgfpathlineto{\pgfqpoint{3.226543in}{0.845334in}}%
\pgfpathlineto{\pgfqpoint{3.227395in}{0.845265in}}%
\pgfpathlineto{\pgfqpoint{3.228248in}{0.845195in}}%
\pgfpathlineto{\pgfqpoint{3.229101in}{0.845126in}}%
\pgfpathlineto{\pgfqpoint{3.229954in}{0.845056in}}%
\pgfpathlineto{\pgfqpoint{3.230807in}{0.844987in}}%
\pgfpathlineto{\pgfqpoint{3.231660in}{0.844917in}}%
\pgfpathlineto{\pgfqpoint{3.232513in}{0.844848in}}%
\pgfpathlineto{\pgfqpoint{3.233366in}{0.844779in}}%
\pgfpathlineto{\pgfqpoint{3.234218in}{0.844709in}}%
\pgfpathlineto{\pgfqpoint{3.235071in}{0.844640in}}%
\pgfpathlineto{\pgfqpoint{3.235924in}{0.844570in}}%
\pgfpathlineto{\pgfqpoint{3.236777in}{0.844501in}}%
\pgfpathlineto{\pgfqpoint{3.237630in}{0.844432in}}%
\pgfpathlineto{\pgfqpoint{3.238483in}{0.843974in}}%
\pgfpathlineto{\pgfqpoint{3.239336in}{0.842880in}}%
\pgfpathlineto{\pgfqpoint{3.240189in}{0.842464in}}%
\pgfpathlineto{\pgfqpoint{3.241041in}{0.842331in}}%
\pgfpathlineto{\pgfqpoint{3.241894in}{0.842676in}}%
\pgfpathlineto{\pgfqpoint{3.242747in}{0.843062in}}%
\pgfpathlineto{\pgfqpoint{3.243600in}{0.843449in}}%
\pgfpathlineto{\pgfqpoint{3.244453in}{0.843835in}}%
\pgfpathlineto{\pgfqpoint{3.245306in}{0.844222in}}%
\pgfpathlineto{\pgfqpoint{3.246159in}{0.844608in}}%
\pgfpathlineto{\pgfqpoint{3.247012in}{0.844995in}}%
\pgfpathlineto{\pgfqpoint{3.247864in}{0.845381in}}%
\pgfpathlineto{\pgfqpoint{3.248717in}{0.845518in}}%
\pgfpathlineto{\pgfqpoint{3.249570in}{0.845154in}}%
\pgfpathlineto{\pgfqpoint{3.250423in}{0.844777in}}%
\pgfpathlineto{\pgfqpoint{3.251276in}{0.844399in}}%
\pgfpathlineto{\pgfqpoint{3.252129in}{0.844021in}}%
\pgfpathlineto{\pgfqpoint{3.252982in}{0.843643in}}%
\pgfpathlineto{\pgfqpoint{3.253835in}{0.843265in}}%
\pgfpathlineto{\pgfqpoint{3.254688in}{0.842887in}}%
\pgfpathlineto{\pgfqpoint{3.255540in}{0.842510in}}%
\pgfpathlineto{\pgfqpoint{3.256393in}{0.842132in}}%
\pgfpathlineto{\pgfqpoint{3.257246in}{0.841739in}}%
\pgfpathlineto{\pgfqpoint{3.258099in}{0.841326in}}%
\pgfpathlineto{\pgfqpoint{3.258952in}{0.841155in}}%
\pgfpathlineto{\pgfqpoint{3.259805in}{0.841119in}}%
\pgfpathlineto{\pgfqpoint{3.260658in}{0.841030in}}%
\pgfpathlineto{\pgfqpoint{3.261511in}{0.840934in}}%
\pgfpathlineto{\pgfqpoint{3.262363in}{0.840838in}}%
\pgfpathlineto{\pgfqpoint{3.263216in}{0.840742in}}%
\pgfpathlineto{\pgfqpoint{3.264069in}{0.840645in}}%
\pgfpathlineto{\pgfqpoint{3.264922in}{0.840549in}}%
\pgfpathlineto{\pgfqpoint{3.265775in}{0.840453in}}%
\pgfpathlineto{\pgfqpoint{3.266628in}{0.840357in}}%
\pgfpathlineto{\pgfqpoint{3.267481in}{0.840261in}}%
\pgfpathlineto{\pgfqpoint{3.268334in}{0.840164in}}%
\pgfpathlineto{\pgfqpoint{3.269186in}{0.840068in}}%
\pgfpathlineto{\pgfqpoint{3.270039in}{0.839972in}}%
\pgfpathlineto{\pgfqpoint{3.270892in}{0.839876in}}%
\pgfpathlineto{\pgfqpoint{3.271745in}{0.839780in}}%
\pgfpathlineto{\pgfqpoint{3.272598in}{0.839683in}}%
\pgfpathlineto{\pgfqpoint{3.273451in}{0.839587in}}%
\pgfpathlineto{\pgfqpoint{3.274304in}{0.839491in}}%
\pgfpathlineto{\pgfqpoint{3.275157in}{0.839395in}}%
\pgfpathlineto{\pgfqpoint{3.276009in}{0.839299in}}%
\pgfpathlineto{\pgfqpoint{3.276862in}{0.839435in}}%
\pgfpathlineto{\pgfqpoint{3.277715in}{0.841815in}}%
\pgfpathlineto{\pgfqpoint{3.278568in}{0.839224in}}%
\pgfpathlineto{\pgfqpoint{3.279421in}{0.839044in}}%
\pgfpathlineto{\pgfqpoint{3.280274in}{0.839147in}}%
\pgfpathlineto{\pgfqpoint{3.281127in}{0.839251in}}%
\pgfpathlineto{\pgfqpoint{3.281980in}{0.839355in}}%
\pgfpathlineto{\pgfqpoint{3.282833in}{0.839459in}}%
\pgfpathlineto{\pgfqpoint{3.283685in}{0.839563in}}%
\pgfpathlineto{\pgfqpoint{3.284538in}{0.839666in}}%
\pgfpathlineto{\pgfqpoint{3.285391in}{0.839770in}}%
\pgfpathlineto{\pgfqpoint{3.286244in}{0.839874in}}%
\pgfpathlineto{\pgfqpoint{3.287097in}{0.839978in}}%
\pgfpathlineto{\pgfqpoint{3.287950in}{0.840081in}}%
\pgfpathlineto{\pgfqpoint{3.288803in}{0.840185in}}%
\pgfpathlineto{\pgfqpoint{3.289656in}{0.840289in}}%
\pgfpathlineto{\pgfqpoint{3.290508in}{0.840393in}}%
\pgfpathlineto{\pgfqpoint{3.291361in}{0.840496in}}%
\pgfpathlineto{\pgfqpoint{3.292214in}{0.840600in}}%
\pgfpathlineto{\pgfqpoint{3.293067in}{0.840704in}}%
\pgfpathlineto{\pgfqpoint{3.293920in}{0.840808in}}%
\pgfpathlineto{\pgfqpoint{3.294773in}{0.840912in}}%
\pgfpathlineto{\pgfqpoint{3.295626in}{0.841015in}}%
\pgfpathlineto{\pgfqpoint{3.296479in}{0.841112in}}%
\pgfpathlineto{\pgfqpoint{3.297331in}{0.841072in}}%
\pgfpathlineto{\pgfqpoint{3.298184in}{0.840977in}}%
\pgfpathlineto{\pgfqpoint{3.299037in}{0.840882in}}%
\pgfpathlineto{\pgfqpoint{3.299890in}{0.840787in}}%
\pgfpathlineto{\pgfqpoint{3.300743in}{0.840692in}}%
\pgfpathlineto{\pgfqpoint{3.301596in}{0.840597in}}%
\pgfpathlineto{\pgfqpoint{3.302449in}{0.840502in}}%
\pgfpathlineto{\pgfqpoint{3.303302in}{0.840407in}}%
\pgfpathlineto{\pgfqpoint{3.304154in}{0.840312in}}%
\pgfpathlineto{\pgfqpoint{3.305007in}{0.840217in}}%
\pgfpathlineto{\pgfqpoint{3.305860in}{0.840122in}}%
\pgfpathlineto{\pgfqpoint{3.306713in}{0.840027in}}%
\pgfpathlineto{\pgfqpoint{3.307566in}{0.839932in}}%
\pgfpathlineto{\pgfqpoint{3.308419in}{0.839837in}}%
\pgfpathlineto{\pgfqpoint{3.309272in}{0.839742in}}%
\pgfpathlineto{\pgfqpoint{3.310125in}{0.839647in}}%
\pgfpathlineto{\pgfqpoint{3.310977in}{0.839552in}}%
\pgfpathlineto{\pgfqpoint{3.311830in}{0.839456in}}%
\pgfpathlineto{\pgfqpoint{3.312683in}{0.839361in}}%
\pgfpathlineto{\pgfqpoint{3.313536in}{0.839266in}}%
\pgfpathlineto{\pgfqpoint{3.314389in}{0.839171in}}%
\pgfpathlineto{\pgfqpoint{3.315242in}{0.839076in}}%
\pgfpathlineto{\pgfqpoint{3.316095in}{0.838981in}}%
\pgfpathlineto{\pgfqpoint{3.316948in}{0.838886in}}%
\pgfpathlineto{\pgfqpoint{3.317801in}{0.838791in}}%
\pgfpathlineto{\pgfqpoint{3.318653in}{0.838693in}}%
\pgfpathlineto{\pgfqpoint{3.319506in}{0.836555in}}%
\pgfpathlineto{\pgfqpoint{3.320359in}{0.834335in}}%
\pgfpathlineto{\pgfqpoint{3.321212in}{0.834297in}}%
\pgfpathlineto{\pgfqpoint{3.322065in}{0.834269in}}%
\pgfpathlineto{\pgfqpoint{3.322918in}{0.834241in}}%
\pgfpathlineto{\pgfqpoint{3.323771in}{0.834213in}}%
\pgfpathlineto{\pgfqpoint{3.324624in}{0.834185in}}%
\pgfpathlineto{\pgfqpoint{3.325476in}{0.834157in}}%
\pgfpathlineto{\pgfqpoint{3.326329in}{0.834129in}}%
\pgfpathlineto{\pgfqpoint{3.327182in}{0.834101in}}%
\pgfpathlineto{\pgfqpoint{3.328035in}{0.834074in}}%
\pgfpathlineto{\pgfqpoint{3.328888in}{0.834046in}}%
\pgfpathlineto{\pgfqpoint{3.329741in}{0.834018in}}%
\pgfpathlineto{\pgfqpoint{3.330594in}{0.833990in}}%
\pgfpathlineto{\pgfqpoint{3.331447in}{0.833962in}}%
\pgfpathlineto{\pgfqpoint{3.332299in}{0.833934in}}%
\pgfpathlineto{\pgfqpoint{3.333152in}{0.833906in}}%
\pgfpathlineto{\pgfqpoint{3.334005in}{0.833879in}}%
\pgfpathlineto{\pgfqpoint{3.334858in}{0.833851in}}%
\pgfpathlineto{\pgfqpoint{3.335711in}{0.833823in}}%
\pgfpathlineto{\pgfqpoint{3.336564in}{0.833795in}}%
\pgfpathlineto{\pgfqpoint{3.337417in}{0.833767in}}%
\pgfpathlineto{\pgfqpoint{3.338270in}{0.833739in}}%
\pgfpathlineto{\pgfqpoint{3.339122in}{0.833711in}}%
\pgfpathlineto{\pgfqpoint{3.339975in}{0.833683in}}%
\pgfpathlineto{\pgfqpoint{3.340828in}{0.833656in}}%
\pgfpathlineto{\pgfqpoint{3.341681in}{0.833628in}}%
\pgfpathlineto{\pgfqpoint{3.342534in}{0.833600in}}%
\pgfpathlineto{\pgfqpoint{3.343387in}{0.833572in}}%
\pgfpathlineto{\pgfqpoint{3.344240in}{0.833544in}}%
\pgfpathlineto{\pgfqpoint{3.345093in}{0.833516in}}%
\pgfpathlineto{\pgfqpoint{3.345946in}{0.833488in}}%
\pgfpathlineto{\pgfqpoint{3.346798in}{0.833460in}}%
\pgfpathlineto{\pgfqpoint{3.347651in}{0.833433in}}%
\pgfpathlineto{\pgfqpoint{3.348504in}{0.833405in}}%
\pgfpathlineto{\pgfqpoint{3.349357in}{0.833377in}}%
\pgfpathlineto{\pgfqpoint{3.350210in}{0.833349in}}%
\pgfpathlineto{\pgfqpoint{3.351063in}{0.833321in}}%
\pgfpathlineto{\pgfqpoint{3.351916in}{0.833293in}}%
\pgfpathlineto{\pgfqpoint{3.352769in}{0.833265in}}%
\pgfpathlineto{\pgfqpoint{3.353621in}{0.833237in}}%
\pgfpathlineto{\pgfqpoint{3.354474in}{0.833210in}}%
\pgfpathlineto{\pgfqpoint{3.355327in}{0.833182in}}%
\pgfpathlineto{\pgfqpoint{3.356180in}{0.833154in}}%
\pgfpathlineto{\pgfqpoint{3.357033in}{0.833126in}}%
\pgfpathlineto{\pgfqpoint{3.357886in}{0.833098in}}%
\pgfpathlineto{\pgfqpoint{3.358739in}{0.833070in}}%
\pgfpathlineto{\pgfqpoint{3.359592in}{0.833042in}}%
\pgfpathlineto{\pgfqpoint{3.360444in}{0.833015in}}%
\pgfpathlineto{\pgfqpoint{3.361297in}{0.832987in}}%
\pgfpathlineto{\pgfqpoint{3.362150in}{0.832959in}}%
\pgfpathlineto{\pgfqpoint{3.363003in}{0.832931in}}%
\pgfpathlineto{\pgfqpoint{3.363856in}{0.832903in}}%
\pgfpathlineto{\pgfqpoint{3.364709in}{0.832875in}}%
\pgfpathlineto{\pgfqpoint{3.365562in}{0.832847in}}%
\pgfpathlineto{\pgfqpoint{3.366415in}{0.832819in}}%
\pgfpathlineto{\pgfqpoint{3.367267in}{0.832792in}}%
\pgfpathlineto{\pgfqpoint{3.368120in}{0.832764in}}%
\pgfpathlineto{\pgfqpoint{3.368973in}{0.832736in}}%
\pgfpathlineto{\pgfqpoint{3.369826in}{0.832708in}}%
\pgfpathlineto{\pgfqpoint{3.370679in}{0.832680in}}%
\pgfpathlineto{\pgfqpoint{3.371532in}{0.832652in}}%
\pgfpathlineto{\pgfqpoint{3.372385in}{0.832624in}}%
\pgfpathlineto{\pgfqpoint{3.373238in}{0.832596in}}%
\pgfpathlineto{\pgfqpoint{3.374091in}{0.832569in}}%
\pgfpathlineto{\pgfqpoint{3.374943in}{0.832541in}}%
\pgfpathlineto{\pgfqpoint{3.375796in}{0.832513in}}%
\pgfpathlineto{\pgfqpoint{3.376649in}{0.832485in}}%
\pgfpathlineto{\pgfqpoint{3.377502in}{0.832457in}}%
\pgfpathlineto{\pgfqpoint{3.378355in}{0.832429in}}%
\pgfpathlineto{\pgfqpoint{3.379208in}{0.832401in}}%
\pgfpathlineto{\pgfqpoint{3.380061in}{0.832355in}}%
\pgfpathlineto{\pgfqpoint{3.380914in}{0.832248in}}%
\pgfpathlineto{\pgfqpoint{3.381766in}{0.832203in}}%
\pgfpathlineto{\pgfqpoint{3.382619in}{0.832177in}}%
\pgfpathlineto{\pgfqpoint{3.383472in}{0.832151in}}%
\pgfpathlineto{\pgfqpoint{3.384325in}{0.832126in}}%
\pgfpathlineto{\pgfqpoint{3.385178in}{0.832100in}}%
\pgfpathlineto{\pgfqpoint{3.386031in}{0.832075in}}%
\pgfpathlineto{\pgfqpoint{3.386884in}{0.832049in}}%
\pgfpathlineto{\pgfqpoint{3.387737in}{0.832023in}}%
\pgfpathlineto{\pgfqpoint{3.388589in}{0.831998in}}%
\pgfpathlineto{\pgfqpoint{3.389442in}{0.831972in}}%
\pgfpathlineto{\pgfqpoint{3.390295in}{0.831946in}}%
\pgfpathlineto{\pgfqpoint{3.391148in}{0.831921in}}%
\pgfpathlineto{\pgfqpoint{3.392001in}{0.831895in}}%
\pgfpathlineto{\pgfqpoint{3.392854in}{0.831870in}}%
\pgfpathlineto{\pgfqpoint{3.393707in}{0.831844in}}%
\pgfpathlineto{\pgfqpoint{3.394560in}{0.831818in}}%
\pgfpathlineto{\pgfqpoint{3.395412in}{0.831793in}}%
\pgfpathlineto{\pgfqpoint{3.396265in}{0.831767in}}%
\pgfpathlineto{\pgfqpoint{3.397118in}{0.831741in}}%
\pgfpathlineto{\pgfqpoint{3.397971in}{0.831716in}}%
\pgfpathlineto{\pgfqpoint{3.398824in}{0.831690in}}%
\pgfpathlineto{\pgfqpoint{3.399677in}{0.831665in}}%
\pgfpathlineto{\pgfqpoint{3.400530in}{0.831639in}}%
\pgfpathlineto{\pgfqpoint{3.401383in}{0.831613in}}%
\pgfpathlineto{\pgfqpoint{3.402235in}{0.831588in}}%
\pgfpathlineto{\pgfqpoint{3.403088in}{0.831562in}}%
\pgfpathlineto{\pgfqpoint{3.403941in}{0.831537in}}%
\pgfpathlineto{\pgfqpoint{3.404794in}{0.831524in}}%
\pgfpathlineto{\pgfqpoint{3.405647in}{0.831523in}}%
\pgfpathlineto{\pgfqpoint{3.406500in}{0.831522in}}%
\pgfpathlineto{\pgfqpoint{3.407353in}{0.831521in}}%
\pgfpathlineto{\pgfqpoint{3.408206in}{0.831519in}}%
\pgfpathlineto{\pgfqpoint{3.409059in}{0.831518in}}%
\pgfpathlineto{\pgfqpoint{3.409911in}{0.831517in}}%
\pgfpathlineto{\pgfqpoint{3.410764in}{0.831516in}}%
\pgfpathlineto{\pgfqpoint{3.411617in}{0.831515in}}%
\pgfpathlineto{\pgfqpoint{3.412470in}{0.831514in}}%
\pgfpathlineto{\pgfqpoint{3.413323in}{0.831490in}}%
\pgfpathlineto{\pgfqpoint{3.414176in}{0.831432in}}%
\pgfpathlineto{\pgfqpoint{3.415029in}{0.831373in}}%
\pgfpathlineto{\pgfqpoint{3.415882in}{0.831315in}}%
\pgfpathlineto{\pgfqpoint{3.416734in}{0.831257in}}%
\pgfpathlineto{\pgfqpoint{3.417587in}{0.831198in}}%
\pgfpathlineto{\pgfqpoint{3.418440in}{0.831140in}}%
\pgfpathlineto{\pgfqpoint{3.419293in}{0.831090in}}%
\pgfpathlineto{\pgfqpoint{3.420146in}{0.831128in}}%
\pgfpathlineto{\pgfqpoint{3.420999in}{0.831191in}}%
\pgfpathlineto{\pgfqpoint{3.421852in}{0.831254in}}%
\pgfpathlineto{\pgfqpoint{3.422705in}{0.831317in}}%
\pgfpathlineto{\pgfqpoint{3.423557in}{0.831380in}}%
\pgfpathlineto{\pgfqpoint{3.424410in}{0.831431in}}%
\pgfpathlineto{\pgfqpoint{3.425263in}{0.831373in}}%
\pgfpathlineto{\pgfqpoint{3.426116in}{0.831257in}}%
\pgfpathlineto{\pgfqpoint{3.426969in}{0.831137in}}%
\pgfpathlineto{\pgfqpoint{3.427822in}{0.831017in}}%
\pgfpathlineto{\pgfqpoint{3.428675in}{0.830897in}}%
\pgfpathlineto{\pgfqpoint{3.429528in}{0.830776in}}%
\pgfpathlineto{\pgfqpoint{3.430380in}{0.830656in}}%
\pgfpathlineto{\pgfqpoint{3.431233in}{0.830536in}}%
\pgfpathlineto{\pgfqpoint{3.432086in}{0.830416in}}%
\pgfpathlineto{\pgfqpoint{3.432939in}{0.830295in}}%
\pgfpathlineto{\pgfqpoint{3.433792in}{0.830175in}}%
\pgfpathlineto{\pgfqpoint{3.434645in}{0.830055in}}%
\pgfpathlineto{\pgfqpoint{3.435498in}{0.829935in}}%
\pgfpathlineto{\pgfqpoint{3.436351in}{0.829814in}}%
\pgfpathlineto{\pgfqpoint{3.437204in}{0.829694in}}%
\pgfpathlineto{\pgfqpoint{3.438056in}{0.829574in}}%
\pgfpathlineto{\pgfqpoint{3.438909in}{0.829454in}}%
\pgfpathlineto{\pgfqpoint{3.439762in}{0.829333in}}%
\pgfpathlineto{\pgfqpoint{3.440615in}{0.829213in}}%
\pgfpathlineto{\pgfqpoint{3.441468in}{0.829093in}}%
\pgfpathlineto{\pgfqpoint{3.442321in}{0.828973in}}%
\pgfpathlineto{\pgfqpoint{3.443174in}{0.828852in}}%
\pgfpathlineto{\pgfqpoint{3.444027in}{0.828732in}}%
\pgfpathlineto{\pgfqpoint{3.444879in}{0.828612in}}%
\pgfpathlineto{\pgfqpoint{3.445732in}{0.828492in}}%
\pgfpathlineto{\pgfqpoint{3.446585in}{0.828490in}}%
\pgfpathlineto{\pgfqpoint{3.447438in}{0.828605in}}%
\pgfpathlineto{\pgfqpoint{3.448291in}{0.828525in}}%
\pgfpathlineto{\pgfqpoint{3.449144in}{0.828411in}}%
\pgfpathlineto{\pgfqpoint{3.449997in}{0.828297in}}%
\pgfpathlineto{\pgfqpoint{3.450850in}{0.828183in}}%
\pgfpathlineto{\pgfqpoint{3.451702in}{0.828078in}}%
\pgfpathlineto{\pgfqpoint{3.452555in}{0.828057in}}%
\pgfpathlineto{\pgfqpoint{3.453408in}{0.828058in}}%
\pgfpathlineto{\pgfqpoint{3.454261in}{0.828059in}}%
\pgfpathlineto{\pgfqpoint{3.455114in}{0.828060in}}%
\pgfpathlineto{\pgfqpoint{3.455967in}{0.828060in}}%
\pgfpathlineto{\pgfqpoint{3.456820in}{0.828061in}}%
\pgfpathlineto{\pgfqpoint{3.457673in}{0.828062in}}%
\pgfpathlineto{\pgfqpoint{3.458525in}{0.828062in}}%
\pgfpathlineto{\pgfqpoint{3.459378in}{0.828063in}}%
\pgfpathlineto{\pgfqpoint{3.460231in}{0.828064in}}%
\pgfpathlineto{\pgfqpoint{3.461084in}{0.828064in}}%
\pgfpathlineto{\pgfqpoint{3.461937in}{0.828065in}}%
\pgfpathlineto{\pgfqpoint{3.462790in}{0.828056in}}%
\pgfpathlineto{\pgfqpoint{3.463643in}{0.827748in}}%
\pgfpathlineto{\pgfqpoint{3.464496in}{0.827300in}}%
\pgfpathlineto{\pgfqpoint{3.465349in}{0.826851in}}%
\pgfpathlineto{\pgfqpoint{3.466201in}{0.826403in}}%
\pgfpathlineto{\pgfqpoint{3.467054in}{0.826039in}}%
\pgfpathlineto{\pgfqpoint{3.467907in}{0.826555in}}%
\pgfpathlineto{\pgfqpoint{3.468760in}{0.826454in}}%
\pgfpathlineto{\pgfqpoint{3.469613in}{0.826099in}}%
\pgfpathlineto{\pgfqpoint{3.470466in}{0.826255in}}%
\pgfpathlineto{\pgfqpoint{3.471319in}{0.826411in}}%
\pgfpathlineto{\pgfqpoint{3.472172in}{0.826568in}}%
\pgfpathlineto{\pgfqpoint{3.473024in}{0.826724in}}%
\pgfpathlineto{\pgfqpoint{3.473877in}{0.826871in}}%
\pgfpathlineto{\pgfqpoint{3.474730in}{0.826855in}}%
\pgfpathlineto{\pgfqpoint{3.475583in}{0.826781in}}%
\pgfpathlineto{\pgfqpoint{3.476436in}{0.826707in}}%
\pgfpathlineto{\pgfqpoint{3.477289in}{0.826634in}}%
\pgfpathlineto{\pgfqpoint{3.478142in}{0.826560in}}%
\pgfpathlineto{\pgfqpoint{3.478995in}{0.826487in}}%
\pgfpathlineto{\pgfqpoint{3.479847in}{0.826413in}}%
\pgfpathlineto{\pgfqpoint{3.480700in}{0.826339in}}%
\pgfpathlineto{\pgfqpoint{3.481553in}{0.826266in}}%
\pgfpathlineto{\pgfqpoint{3.482406in}{0.826192in}}%
\pgfpathlineto{\pgfqpoint{3.483259in}{0.826118in}}%
\pgfpathlineto{\pgfqpoint{3.484112in}{0.826045in}}%
\pgfpathlineto{\pgfqpoint{3.484965in}{0.825971in}}%
\pgfpathlineto{\pgfqpoint{3.485818in}{0.825897in}}%
\pgfpathlineto{\pgfqpoint{3.486670in}{0.825829in}}%
\pgfpathlineto{\pgfqpoint{3.487523in}{0.825783in}}%
\pgfpathlineto{\pgfqpoint{3.488376in}{0.825743in}}%
\pgfpathlineto{\pgfqpoint{3.489229in}{0.825754in}}%
\pgfpathlineto{\pgfqpoint{3.490082in}{0.825754in}}%
\pgfpathlineto{\pgfqpoint{3.490935in}{0.825749in}}%
\pgfpathlineto{\pgfqpoint{3.491788in}{0.825743in}}%
\pgfpathlineto{\pgfqpoint{3.492641in}{0.825738in}}%
\pgfpathlineto{\pgfqpoint{3.493494in}{0.825733in}}%
\pgfpathlineto{\pgfqpoint{3.494346in}{0.825728in}}%
\pgfpathlineto{\pgfqpoint{3.495199in}{0.825723in}}%
\pgfpathlineto{\pgfqpoint{3.496052in}{0.825718in}}%
\pgfpathlineto{\pgfqpoint{3.496905in}{0.825713in}}%
\pgfpathlineto{\pgfqpoint{3.497758in}{0.825708in}}%
\pgfpathlineto{\pgfqpoint{3.498611in}{0.825703in}}%
\pgfpathlineto{\pgfqpoint{3.499464in}{0.825698in}}%
\pgfpathlineto{\pgfqpoint{3.500317in}{0.825693in}}%
\pgfpathlineto{\pgfqpoint{3.501169in}{0.825688in}}%
\pgfpathlineto{\pgfqpoint{3.502022in}{0.825683in}}%
\pgfpathlineto{\pgfqpoint{3.502875in}{0.825678in}}%
\pgfpathlineto{\pgfqpoint{3.503728in}{0.825673in}}%
\pgfpathlineto{\pgfqpoint{3.504581in}{0.825668in}}%
\pgfpathlineto{\pgfqpoint{3.505434in}{0.825663in}}%
\pgfpathlineto{\pgfqpoint{3.506287in}{0.825657in}}%
\pgfpathlineto{\pgfqpoint{3.507140in}{0.825652in}}%
\pgfpathlineto{\pgfqpoint{3.507992in}{0.825647in}}%
\pgfpathlineto{\pgfqpoint{3.508845in}{0.825642in}}%
\pgfpathlineto{\pgfqpoint{3.509698in}{0.825637in}}%
\pgfpathlineto{\pgfqpoint{3.510551in}{0.825632in}}%
\pgfpathlineto{\pgfqpoint{3.511404in}{0.825627in}}%
\pgfpathlineto{\pgfqpoint{3.512257in}{0.825622in}}%
\pgfpathlineto{\pgfqpoint{3.513110in}{0.825617in}}%
\pgfpathlineto{\pgfqpoint{3.513963in}{0.825612in}}%
\pgfpathlineto{\pgfqpoint{3.514815in}{0.825607in}}%
\pgfpathlineto{\pgfqpoint{3.515668in}{0.825602in}}%
\pgfpathlineto{\pgfqpoint{3.516521in}{0.825597in}}%
\pgfpathlineto{\pgfqpoint{3.517374in}{0.825592in}}%
\pgfpathlineto{\pgfqpoint{3.518227in}{0.825587in}}%
\pgfpathlineto{\pgfqpoint{3.519080in}{0.825582in}}%
\pgfpathlineto{\pgfqpoint{3.519933in}{0.825577in}}%
\pgfpathlineto{\pgfqpoint{3.520786in}{0.825571in}}%
\pgfpathlineto{\pgfqpoint{3.521638in}{0.825566in}}%
\pgfpathlineto{\pgfqpoint{3.522491in}{0.825561in}}%
\pgfpathlineto{\pgfqpoint{3.523344in}{0.825556in}}%
\pgfpathlineto{\pgfqpoint{3.524197in}{0.825551in}}%
\pgfpathlineto{\pgfqpoint{3.525050in}{0.825546in}}%
\pgfpathlineto{\pgfqpoint{3.525903in}{0.825541in}}%
\pgfpathlineto{\pgfqpoint{3.526756in}{0.825449in}}%
\pgfpathlineto{\pgfqpoint{3.527609in}{0.824425in}}%
\pgfpathlineto{\pgfqpoint{3.528462in}{0.824178in}}%
\pgfpathlineto{\pgfqpoint{3.529314in}{0.824160in}}%
\pgfpathlineto{\pgfqpoint{3.530167in}{0.824142in}}%
\pgfpathlineto{\pgfqpoint{3.531020in}{0.824124in}}%
\pgfpathlineto{\pgfqpoint{3.531873in}{0.824106in}}%
\pgfpathlineto{\pgfqpoint{3.532726in}{0.824088in}}%
\pgfpathlineto{\pgfqpoint{3.533579in}{0.824070in}}%
\pgfpathlineto{\pgfqpoint{3.534432in}{0.824062in}}%
\pgfpathlineto{\pgfqpoint{3.535285in}{0.824114in}}%
\pgfpathlineto{\pgfqpoint{3.536137in}{0.824176in}}%
\pgfpathlineto{\pgfqpoint{3.536990in}{0.824238in}}%
\pgfpathlineto{\pgfqpoint{3.537843in}{0.824300in}}%
\pgfpathlineto{\pgfqpoint{3.538696in}{0.824363in}}%
\pgfpathlineto{\pgfqpoint{3.539549in}{0.824425in}}%
\pgfpathlineto{\pgfqpoint{3.540402in}{0.824487in}}%
\pgfpathlineto{\pgfqpoint{3.541255in}{0.824549in}}%
\pgfpathlineto{\pgfqpoint{3.542108in}{0.824611in}}%
\pgfpathlineto{\pgfqpoint{3.542960in}{0.824673in}}%
\pgfpathlineto{\pgfqpoint{3.543813in}{0.824735in}}%
\pgfpathlineto{\pgfqpoint{3.544666in}{0.824796in}}%
\pgfpathlineto{\pgfqpoint{3.545519in}{0.824159in}}%
\pgfpathlineto{\pgfqpoint{3.546372in}{0.823620in}}%
\pgfpathlineto{\pgfqpoint{3.547225in}{0.824855in}}%
\pgfpathlineto{\pgfqpoint{3.548078in}{0.826173in}}%
\pgfpathlineto{\pgfqpoint{3.548931in}{0.826119in}}%
\pgfpathlineto{\pgfqpoint{3.549783in}{0.825891in}}%
\pgfpathlineto{\pgfqpoint{3.550636in}{0.825731in}}%
\pgfpathlineto{\pgfqpoint{3.551489in}{0.825596in}}%
\pgfpathlineto{\pgfqpoint{3.552342in}{0.825460in}}%
\pgfpathlineto{\pgfqpoint{3.553195in}{0.825322in}}%
\pgfpathlineto{\pgfqpoint{3.554048in}{0.825183in}}%
\pgfpathlineto{\pgfqpoint{3.554901in}{0.825044in}}%
\pgfpathlineto{\pgfqpoint{3.555754in}{0.824905in}}%
\pgfpathlineto{\pgfqpoint{3.556607in}{0.824767in}}%
\pgfpathlineto{\pgfqpoint{3.557459in}{0.824628in}}%
\pgfpathlineto{\pgfqpoint{3.558312in}{0.824489in}}%
\pgfpathlineto{\pgfqpoint{3.559165in}{0.824350in}}%
\pgfpathlineto{\pgfqpoint{3.560018in}{0.824212in}}%
\pgfpathlineto{\pgfqpoint{3.560871in}{0.824073in}}%
\pgfpathlineto{\pgfqpoint{3.561724in}{0.824016in}}%
\pgfpathlineto{\pgfqpoint{3.562577in}{0.824177in}}%
\pgfpathlineto{\pgfqpoint{3.563430in}{0.823325in}}%
\pgfpathlineto{\pgfqpoint{3.564282in}{0.824690in}}%
\pgfpathlineto{\pgfqpoint{3.565135in}{0.826066in}}%
\pgfpathlineto{\pgfqpoint{3.565988in}{0.826353in}}%
\pgfpathlineto{\pgfqpoint{3.566841in}{0.825589in}}%
\pgfpathlineto{\pgfqpoint{3.567694in}{0.822472in}}%
\pgfpathlineto{\pgfqpoint{3.568547in}{0.822546in}}%
\pgfpathlineto{\pgfqpoint{3.569400in}{0.822701in}}%
\pgfpathlineto{\pgfqpoint{3.570253in}{0.822884in}}%
\pgfpathlineto{\pgfqpoint{3.571105in}{0.823143in}}%
\pgfpathlineto{\pgfqpoint{3.571958in}{0.823408in}}%
\pgfpathlineto{\pgfqpoint{3.572811in}{0.823673in}}%
\pgfpathlineto{\pgfqpoint{3.573664in}{0.823938in}}%
\pgfpathlineto{\pgfqpoint{3.574517in}{0.824203in}}%
\pgfpathlineto{\pgfqpoint{3.575370in}{0.824468in}}%
\pgfpathlineto{\pgfqpoint{3.576223in}{0.824715in}}%
\pgfpathlineto{\pgfqpoint{3.577076in}{0.824917in}}%
\pgfpathlineto{\pgfqpoint{3.577928in}{0.825117in}}%
\pgfpathlineto{\pgfqpoint{3.578781in}{0.825317in}}%
\pgfpathlineto{\pgfqpoint{3.579634in}{0.825517in}}%
\pgfpathlineto{\pgfqpoint{3.580487in}{0.825717in}}%
\pgfpathlineto{\pgfqpoint{3.581340in}{0.825917in}}%
\pgfpathlineto{\pgfqpoint{3.582193in}{0.826116in}}%
\pgfpathlineto{\pgfqpoint{3.583046in}{0.826066in}}%
\pgfpathlineto{\pgfqpoint{3.583899in}{0.822795in}}%
\pgfpathlineto{\pgfqpoint{3.584752in}{0.824329in}}%
\pgfpathlineto{\pgfqpoint{3.585604in}{0.823030in}}%
\pgfpathlineto{\pgfqpoint{3.586457in}{0.822328in}}%
\pgfpathlineto{\pgfqpoint{3.587310in}{0.825785in}}%
\pgfpathlineto{\pgfqpoint{3.588163in}{0.825674in}}%
\pgfpathlineto{\pgfqpoint{3.589016in}{0.825048in}}%
\pgfpathlineto{\pgfqpoint{3.589869in}{0.826221in}}%
\pgfpathlineto{\pgfqpoint{3.590722in}{0.824858in}}%
\pgfpathlineto{\pgfqpoint{3.591575in}{0.824553in}}%
\pgfpathlineto{\pgfqpoint{3.592427in}{0.824712in}}%
\pgfpathlineto{\pgfqpoint{3.593280in}{0.824870in}}%
\pgfpathlineto{\pgfqpoint{3.594133in}{0.825028in}}%
\pgfpathlineto{\pgfqpoint{3.594986in}{0.825187in}}%
\pgfpathlineto{\pgfqpoint{3.595839in}{0.825345in}}%
\pgfpathlineto{\pgfqpoint{3.596692in}{0.825504in}}%
\pgfpathlineto{\pgfqpoint{3.597545in}{0.825662in}}%
\pgfpathlineto{\pgfqpoint{3.598398in}{0.825821in}}%
\pgfpathlineto{\pgfqpoint{3.599250in}{0.825979in}}%
\pgfpathlineto{\pgfqpoint{3.600103in}{0.826138in}}%
\pgfpathlineto{\pgfqpoint{3.600956in}{0.826296in}}%
\pgfpathlineto{\pgfqpoint{3.601809in}{0.826454in}}%
\pgfpathlineto{\pgfqpoint{3.602662in}{0.826613in}}%
\pgfpathlineto{\pgfqpoint{3.603515in}{0.826771in}}%
\pgfpathlineto{\pgfqpoint{3.604368in}{0.826919in}}%
\pgfpathlineto{\pgfqpoint{3.605221in}{0.826992in}}%
\pgfpathlineto{\pgfqpoint{3.606073in}{0.827050in}}%
\pgfpathlineto{\pgfqpoint{3.606926in}{0.826950in}}%
\pgfpathlineto{\pgfqpoint{3.607779in}{0.824465in}}%
\pgfpathlineto{\pgfqpoint{3.608632in}{0.825231in}}%
\pgfpathlineto{\pgfqpoint{3.609485in}{0.825556in}}%
\pgfpathlineto{\pgfqpoint{3.610338in}{0.827033in}}%
\pgfpathlineto{\pgfqpoint{3.611191in}{0.824792in}}%
\pgfpathlineto{\pgfqpoint{3.612044in}{0.824039in}}%
\pgfpathlineto{\pgfqpoint{3.612896in}{0.824042in}}%
\pgfpathlineto{\pgfqpoint{3.613749in}{0.824045in}}%
\pgfpathlineto{\pgfqpoint{3.614602in}{0.824048in}}%
\pgfpathlineto{\pgfqpoint{3.615455in}{0.824052in}}%
\pgfpathlineto{\pgfqpoint{3.616308in}{0.824055in}}%
\pgfpathlineto{\pgfqpoint{3.617161in}{0.824058in}}%
\pgfpathlineto{\pgfqpoint{3.618014in}{0.824061in}}%
\pgfpathlineto{\pgfqpoint{3.618867in}{0.824065in}}%
\pgfpathlineto{\pgfqpoint{3.619720in}{0.824068in}}%
\pgfpathlineto{\pgfqpoint{3.620572in}{0.824071in}}%
\pgfpathlineto{\pgfqpoint{3.621425in}{0.824074in}}%
\pgfpathlineto{\pgfqpoint{3.622278in}{0.824078in}}%
\pgfpathlineto{\pgfqpoint{3.623131in}{0.824081in}}%
\pgfpathlineto{\pgfqpoint{3.623984in}{0.824084in}}%
\pgfpathlineto{\pgfqpoint{3.624837in}{0.824087in}}%
\pgfpathlineto{\pgfqpoint{3.625690in}{0.824091in}}%
\pgfpathlineto{\pgfqpoint{3.626543in}{0.824094in}}%
\pgfpathlineto{\pgfqpoint{3.627395in}{0.824097in}}%
\pgfpathlineto{\pgfqpoint{3.628248in}{0.824100in}}%
\pgfpathlineto{\pgfqpoint{3.629101in}{0.824104in}}%
\pgfpathlineto{\pgfqpoint{3.629954in}{0.824107in}}%
\pgfpathlineto{\pgfqpoint{3.630807in}{0.824114in}}%
\pgfpathlineto{\pgfqpoint{3.631660in}{0.824164in}}%
\pgfpathlineto{\pgfqpoint{3.632513in}{0.824223in}}%
\pgfpathlineto{\pgfqpoint{3.633366in}{0.824283in}}%
\pgfpathlineto{\pgfqpoint{3.634218in}{0.824343in}}%
\pgfpathlineto{\pgfqpoint{3.635071in}{0.824403in}}%
\pgfpathlineto{\pgfqpoint{3.635924in}{0.824463in}}%
\pgfpathlineto{\pgfqpoint{3.636777in}{0.824522in}}%
\pgfpathlineto{\pgfqpoint{3.637630in}{0.824582in}}%
\pgfpathlineto{\pgfqpoint{3.638483in}{0.824642in}}%
\pgfpathlineto{\pgfqpoint{3.639336in}{0.824702in}}%
\pgfpathlineto{\pgfqpoint{3.640189in}{0.824762in}}%
\pgfpathlineto{\pgfqpoint{3.641041in}{0.824821in}}%
\pgfpathlineto{\pgfqpoint{3.641894in}{0.824881in}}%
\pgfpathlineto{\pgfqpoint{3.642747in}{0.824941in}}%
\pgfpathlineto{\pgfqpoint{3.643600in}{0.825001in}}%
\pgfpathlineto{\pgfqpoint{3.644453in}{0.825060in}}%
\pgfpathlineto{\pgfqpoint{3.645306in}{0.825120in}}%
\pgfpathlineto{\pgfqpoint{3.646159in}{0.825180in}}%
\pgfpathlineto{\pgfqpoint{3.647012in}{0.825240in}}%
\pgfpathlineto{\pgfqpoint{3.647865in}{0.825300in}}%
\pgfpathlineto{\pgfqpoint{3.648717in}{0.825359in}}%
\pgfpathlineto{\pgfqpoint{3.649570in}{0.825419in}}%
\pgfpathlineto{\pgfqpoint{3.650423in}{0.825479in}}%
\pgfpathlineto{\pgfqpoint{3.651276in}{0.825539in}}%
\pgfpathlineto{\pgfqpoint{3.652129in}{0.825599in}}%
\pgfpathlineto{\pgfqpoint{3.652982in}{0.825658in}}%
\pgfpathlineto{\pgfqpoint{3.653835in}{0.825718in}}%
\pgfpathlineto{\pgfqpoint{3.654688in}{0.825778in}}%
\pgfpathlineto{\pgfqpoint{3.655540in}{0.825838in}}%
\pgfpathlineto{\pgfqpoint{3.656393in}{0.825898in}}%
\pgfpathlineto{\pgfqpoint{3.657246in}{0.825957in}}%
\pgfpathlineto{\pgfqpoint{3.658099in}{0.826017in}}%
\pgfpathlineto{\pgfqpoint{3.658952in}{0.826077in}}%
\pgfpathlineto{\pgfqpoint{3.659805in}{0.826137in}}%
\pgfpathlineto{\pgfqpoint{3.660658in}{0.826197in}}%
\pgfpathlineto{\pgfqpoint{3.661511in}{0.826256in}}%
\pgfpathlineto{\pgfqpoint{3.662363in}{0.826316in}}%
\pgfpathlineto{\pgfqpoint{3.663216in}{0.826376in}}%
\pgfpathlineto{\pgfqpoint{3.664069in}{0.826436in}}%
\pgfpathlineto{\pgfqpoint{3.664922in}{0.826495in}}%
\pgfpathlineto{\pgfqpoint{3.665775in}{0.826555in}}%
\pgfpathlineto{\pgfqpoint{3.666628in}{0.826615in}}%
\pgfpathlineto{\pgfqpoint{3.667481in}{0.826675in}}%
\pgfpathlineto{\pgfqpoint{3.668334in}{0.826735in}}%
\pgfpathlineto{\pgfqpoint{3.669186in}{0.826794in}}%
\pgfpathlineto{\pgfqpoint{3.670039in}{0.826848in}}%
\pgfpathlineto{\pgfqpoint{3.670892in}{0.826863in}}%
\pgfpathlineto{\pgfqpoint{3.671745in}{0.826871in}}%
\pgfpathlineto{\pgfqpoint{3.672598in}{0.826880in}}%
\pgfpathlineto{\pgfqpoint{3.673451in}{0.826889in}}%
\pgfpathlineto{\pgfqpoint{3.674304in}{0.826898in}}%
\pgfpathlineto{\pgfqpoint{3.675157in}{0.826901in}}%
\pgfpathlineto{\pgfqpoint{3.676010in}{0.826890in}}%
\pgfpathlineto{\pgfqpoint{3.676862in}{0.826879in}}%
\pgfpathlineto{\pgfqpoint{3.677715in}{0.826868in}}%
\pgfpathlineto{\pgfqpoint{3.678568in}{0.826857in}}%
\pgfpathlineto{\pgfqpoint{3.679421in}{0.826846in}}%
\pgfpathlineto{\pgfqpoint{3.680274in}{0.826835in}}%
\pgfpathlineto{\pgfqpoint{3.681127in}{0.826824in}}%
\pgfpathlineto{\pgfqpoint{3.681980in}{0.826813in}}%
\pgfpathlineto{\pgfqpoint{3.682833in}{0.826802in}}%
\pgfpathlineto{\pgfqpoint{3.683685in}{0.826792in}}%
\pgfpathlineto{\pgfqpoint{3.684538in}{0.826781in}}%
\pgfpathlineto{\pgfqpoint{3.685391in}{0.826768in}}%
\pgfpathlineto{\pgfqpoint{3.686244in}{0.826338in}}%
\pgfpathlineto{\pgfqpoint{3.687097in}{0.825573in}}%
\pgfpathlineto{\pgfqpoint{3.687950in}{0.824808in}}%
\pgfpathlineto{\pgfqpoint{3.688803in}{0.824043in}}%
\pgfpathlineto{\pgfqpoint{3.689656in}{0.824654in}}%
\pgfpathlineto{\pgfqpoint{3.690508in}{0.825796in}}%
\pgfpathlineto{\pgfqpoint{3.691361in}{0.826949in}}%
\pgfpathlineto{\pgfqpoint{3.692214in}{0.829460in}}%
\pgfpathlineto{\pgfqpoint{3.693067in}{0.830592in}}%
\pgfpathlineto{\pgfqpoint{3.693920in}{0.831143in}}%
\pgfpathlineto{\pgfqpoint{3.694773in}{0.831131in}}%
\pgfpathlineto{\pgfqpoint{3.695626in}{0.831120in}}%
\pgfpathlineto{\pgfqpoint{3.696479in}{0.831109in}}%
\pgfpathlineto{\pgfqpoint{3.697331in}{0.831098in}}%
\pgfpathlineto{\pgfqpoint{3.698184in}{0.831086in}}%
\pgfpathlineto{\pgfqpoint{3.699037in}{0.831075in}}%
\pgfpathlineto{\pgfqpoint{3.699890in}{0.831064in}}%
\pgfpathlineto{\pgfqpoint{3.700743in}{0.831053in}}%
\pgfpathlineto{\pgfqpoint{3.701596in}{0.831041in}}%
\pgfpathlineto{\pgfqpoint{3.702449in}{0.831030in}}%
\pgfpathlineto{\pgfqpoint{3.703302in}{0.831019in}}%
\pgfpathlineto{\pgfqpoint{3.704155in}{0.831008in}}%
\pgfpathlineto{\pgfqpoint{3.705007in}{0.830996in}}%
\pgfpathlineto{\pgfqpoint{3.705860in}{0.830875in}}%
\pgfpathlineto{\pgfqpoint{3.706713in}{0.829070in}}%
\pgfpathlineto{\pgfqpoint{3.707566in}{0.828552in}}%
\pgfpathlineto{\pgfqpoint{3.708419in}{0.828596in}}%
\pgfpathlineto{\pgfqpoint{3.709272in}{0.828569in}}%
\pgfpathlineto{\pgfqpoint{3.710125in}{0.828528in}}%
\pgfpathlineto{\pgfqpoint{3.710978in}{0.828517in}}%
\pgfpathlineto{\pgfqpoint{3.711830in}{0.828557in}}%
\pgfpathlineto{\pgfqpoint{3.712683in}{0.829125in}}%
\pgfpathlineto{\pgfqpoint{3.713536in}{0.830155in}}%
\pgfpathlineto{\pgfqpoint{3.714389in}{0.830766in}}%
\pgfpathlineto{\pgfqpoint{3.715242in}{0.830883in}}%
\pgfpathlineto{\pgfqpoint{3.716095in}{0.831000in}}%
\pgfpathlineto{\pgfqpoint{3.716948in}{0.831117in}}%
\pgfpathlineto{\pgfqpoint{3.717801in}{0.831031in}}%
\pgfpathlineto{\pgfqpoint{3.718653in}{0.830761in}}%
\pgfpathlineto{\pgfqpoint{3.719506in}{0.830491in}}%
\pgfpathlineto{\pgfqpoint{3.720359in}{0.830221in}}%
\pgfpathlineto{\pgfqpoint{3.721212in}{0.829951in}}%
\pgfpathlineto{\pgfqpoint{3.722065in}{0.829681in}}%
\pgfpathlineto{\pgfqpoint{3.722918in}{0.829410in}}%
\pgfpathlineto{\pgfqpoint{3.723771in}{0.829140in}}%
\pgfpathlineto{\pgfqpoint{3.724624in}{0.828870in}}%
\pgfpathlineto{\pgfqpoint{3.725476in}{0.828600in}}%
\pgfpathlineto{\pgfqpoint{3.726329in}{0.828383in}}%
\pgfpathlineto{\pgfqpoint{3.727182in}{0.828354in}}%
\pgfpathlineto{\pgfqpoint{3.728035in}{0.828341in}}%
\pgfpathlineto{\pgfqpoint{3.728888in}{0.828328in}}%
\pgfpathlineto{\pgfqpoint{3.729741in}{0.828315in}}%
\pgfpathlineto{\pgfqpoint{3.730594in}{0.829409in}}%
\pgfpathlineto{\pgfqpoint{3.731447in}{0.830183in}}%
\pgfpathlineto{\pgfqpoint{3.732299in}{0.828611in}}%
\pgfpathlineto{\pgfqpoint{3.733152in}{0.828379in}}%
\pgfpathlineto{\pgfqpoint{3.734005in}{0.828719in}}%
\pgfpathlineto{\pgfqpoint{3.734858in}{0.829074in}}%
\pgfpathlineto{\pgfqpoint{3.735711in}{0.829429in}}%
\pgfpathlineto{\pgfqpoint{3.736564in}{0.829784in}}%
\pgfpathlineto{\pgfqpoint{3.737417in}{0.830139in}}%
\pgfpathlineto{\pgfqpoint{3.738270in}{0.830494in}}%
\pgfpathlineto{\pgfqpoint{3.739123in}{0.830931in}}%
\pgfpathlineto{\pgfqpoint{3.739975in}{0.831317in}}%
\pgfpathlineto{\pgfqpoint{3.740828in}{0.831350in}}%
\pgfpathlineto{\pgfqpoint{3.741681in}{0.831369in}}%
\pgfpathlineto{\pgfqpoint{3.742534in}{0.831388in}}%
\pgfpathlineto{\pgfqpoint{3.743387in}{0.831407in}}%
\pgfpathlineto{\pgfqpoint{3.744240in}{0.831427in}}%
\pgfpathlineto{\pgfqpoint{3.745093in}{0.831446in}}%
\pgfpathlineto{\pgfqpoint{3.745946in}{0.831465in}}%
\pgfpathlineto{\pgfqpoint{3.746798in}{0.831485in}}%
\pgfpathlineto{\pgfqpoint{3.747651in}{0.831504in}}%
\pgfpathlineto{\pgfqpoint{3.748504in}{0.831523in}}%
\pgfpathlineto{\pgfqpoint{3.749357in}{0.831041in}}%
\pgfpathlineto{\pgfqpoint{3.750210in}{0.830622in}}%
\pgfpathlineto{\pgfqpoint{3.751063in}{0.830605in}}%
\pgfpathlineto{\pgfqpoint{3.751916in}{0.830587in}}%
\pgfpathlineto{\pgfqpoint{3.752769in}{0.830570in}}%
\pgfpathlineto{\pgfqpoint{3.753621in}{0.830552in}}%
\pgfpathlineto{\pgfqpoint{3.754474in}{0.830507in}}%
\pgfpathlineto{\pgfqpoint{3.755327in}{0.830357in}}%
\pgfpathlineto{\pgfqpoint{3.756180in}{0.830198in}}%
\pgfpathlineto{\pgfqpoint{3.757033in}{0.830039in}}%
\pgfpathlineto{\pgfqpoint{3.757886in}{0.829880in}}%
\pgfpathlineto{\pgfqpoint{3.758739in}{0.829721in}}%
\pgfpathlineto{\pgfqpoint{3.759592in}{0.829562in}}%
\pgfpathlineto{\pgfqpoint{3.760444in}{0.829433in}}%
\pgfpathlineto{\pgfqpoint{3.761297in}{0.829469in}}%
\pgfpathlineto{\pgfqpoint{3.762150in}{0.829529in}}%
\pgfpathlineto{\pgfqpoint{3.763003in}{0.829590in}}%
\pgfpathlineto{\pgfqpoint{3.763856in}{0.829650in}}%
\pgfpathlineto{\pgfqpoint{3.764709in}{0.829711in}}%
\pgfpathlineto{\pgfqpoint{3.765562in}{0.829772in}}%
\pgfpathlineto{\pgfqpoint{3.766415in}{0.829832in}}%
\pgfpathlineto{\pgfqpoint{3.767268in}{0.829893in}}%
\pgfpathlineto{\pgfqpoint{3.768120in}{0.829954in}}%
\pgfpathlineto{\pgfqpoint{3.768973in}{0.830014in}}%
\pgfpathlineto{\pgfqpoint{3.769826in}{0.830075in}}%
\pgfpathlineto{\pgfqpoint{3.770679in}{0.830136in}}%
\pgfpathlineto{\pgfqpoint{3.771532in}{0.830196in}}%
\pgfpathlineto{\pgfqpoint{3.772385in}{0.830257in}}%
\pgfpathlineto{\pgfqpoint{3.773238in}{0.830318in}}%
\pgfpathlineto{\pgfqpoint{3.774091in}{0.830378in}}%
\pgfpathlineto{\pgfqpoint{3.774943in}{0.830439in}}%
\pgfpathlineto{\pgfqpoint{3.775796in}{0.830500in}}%
\pgfpathlineto{\pgfqpoint{3.776649in}{0.830560in}}%
\pgfpathlineto{\pgfqpoint{3.777502in}{0.830621in}}%
\pgfpathlineto{\pgfqpoint{3.778355in}{0.830656in}}%
\pgfpathlineto{\pgfqpoint{3.779208in}{0.830650in}}%
\pgfpathlineto{\pgfqpoint{3.780061in}{0.830643in}}%
\pgfpathlineto{\pgfqpoint{3.780914in}{0.830637in}}%
\pgfpathlineto{\pgfqpoint{3.781766in}{0.830631in}}%
\pgfpathlineto{\pgfqpoint{3.782619in}{0.830624in}}%
\pgfpathlineto{\pgfqpoint{3.783472in}{0.830618in}}%
\pgfpathlineto{\pgfqpoint{3.784325in}{0.830611in}}%
\pgfpathlineto{\pgfqpoint{3.785178in}{0.830605in}}%
\pgfpathlineto{\pgfqpoint{3.786031in}{0.830598in}}%
\pgfpathlineto{\pgfqpoint{3.786884in}{0.830592in}}%
\pgfpathlineto{\pgfqpoint{3.787737in}{0.830585in}}%
\pgfpathlineto{\pgfqpoint{3.788589in}{0.830579in}}%
\pgfpathlineto{\pgfqpoint{3.789442in}{0.830572in}}%
\pgfpathlineto{\pgfqpoint{3.790295in}{0.830566in}}%
\pgfpathlineto{\pgfqpoint{3.791148in}{0.830560in}}%
\pgfpathlineto{\pgfqpoint{3.792001in}{0.830553in}}%
\pgfpathlineto{\pgfqpoint{3.792854in}{0.830547in}}%
\pgfpathlineto{\pgfqpoint{3.793707in}{0.830540in}}%
\pgfpathlineto{\pgfqpoint{3.794560in}{0.830534in}}%
\pgfpathlineto{\pgfqpoint{3.795413in}{0.830527in}}%
\pgfpathlineto{\pgfqpoint{3.796265in}{0.830521in}}%
\pgfpathlineto{\pgfqpoint{3.797118in}{0.830514in}}%
\pgfpathlineto{\pgfqpoint{3.797971in}{0.830508in}}%
\pgfpathlineto{\pgfqpoint{3.798824in}{0.830501in}}%
\pgfpathlineto{\pgfqpoint{3.799677in}{0.830495in}}%
\pgfpathlineto{\pgfqpoint{3.800530in}{0.830488in}}%
\pgfpathlineto{\pgfqpoint{3.801383in}{0.830482in}}%
\pgfpathlineto{\pgfqpoint{3.802236in}{0.830476in}}%
\pgfpathlineto{\pgfqpoint{3.803088in}{0.830469in}}%
\pgfpathlineto{\pgfqpoint{3.803941in}{0.830463in}}%
\pgfpathlineto{\pgfqpoint{3.804794in}{0.830456in}}%
\pgfpathlineto{\pgfqpoint{3.805647in}{0.830450in}}%
\pgfpathlineto{\pgfqpoint{3.806500in}{0.830443in}}%
\pgfpathlineto{\pgfqpoint{3.807353in}{0.830437in}}%
\pgfpathlineto{\pgfqpoint{3.808206in}{0.830430in}}%
\pgfpathlineto{\pgfqpoint{3.809059in}{0.830424in}}%
\pgfpathlineto{\pgfqpoint{3.809911in}{0.830417in}}%
\pgfpathlineto{\pgfqpoint{3.810764in}{0.830411in}}%
\pgfpathlineto{\pgfqpoint{3.811617in}{0.830433in}}%
\pgfpathlineto{\pgfqpoint{3.812470in}{0.830967in}}%
\pgfpathlineto{\pgfqpoint{3.813323in}{0.830996in}}%
\pgfpathlineto{\pgfqpoint{3.814176in}{0.831008in}}%
\pgfpathlineto{\pgfqpoint{3.815029in}{0.831020in}}%
\pgfpathlineto{\pgfqpoint{3.815882in}{0.831031in}}%
\pgfpathlineto{\pgfqpoint{3.816734in}{0.831043in}}%
\pgfpathlineto{\pgfqpoint{3.817587in}{0.831055in}}%
\pgfpathlineto{\pgfqpoint{3.818440in}{0.831067in}}%
\pgfpathlineto{\pgfqpoint{3.819293in}{0.831078in}}%
\pgfpathlineto{\pgfqpoint{3.820146in}{0.831090in}}%
\pgfpathlineto{\pgfqpoint{3.820999in}{0.831102in}}%
\pgfpathlineto{\pgfqpoint{3.821852in}{0.831113in}}%
\pgfpathlineto{\pgfqpoint{3.822705in}{0.831125in}}%
\pgfpathlineto{\pgfqpoint{3.823558in}{0.831137in}}%
\pgfpathlineto{\pgfqpoint{3.824410in}{0.831149in}}%
\pgfpathlineto{\pgfqpoint{3.825263in}{0.831160in}}%
\pgfpathlineto{\pgfqpoint{3.826116in}{0.831172in}}%
\pgfpathlineto{\pgfqpoint{3.826969in}{0.831184in}}%
\pgfpathlineto{\pgfqpoint{3.827822in}{0.831196in}}%
\pgfpathlineto{\pgfqpoint{3.828675in}{0.831207in}}%
\pgfpathlineto{\pgfqpoint{3.829528in}{0.831219in}}%
\pgfpathlineto{\pgfqpoint{3.830381in}{0.831231in}}%
\pgfpathlineto{\pgfqpoint{3.831233in}{0.831243in}}%
\pgfpathlineto{\pgfqpoint{3.832086in}{0.831250in}}%
\pgfpathlineto{\pgfqpoint{3.832939in}{0.830813in}}%
\pgfpathlineto{\pgfqpoint{3.833792in}{0.830863in}}%
\pgfpathlineto{\pgfqpoint{3.834645in}{0.831385in}}%
\pgfpathlineto{\pgfqpoint{3.835498in}{0.831083in}}%
\pgfpathlineto{\pgfqpoint{3.836351in}{0.830666in}}%
\pgfpathlineto{\pgfqpoint{3.837204in}{0.830521in}}%
\pgfpathlineto{\pgfqpoint{3.838056in}{0.830605in}}%
\pgfpathlineto{\pgfqpoint{3.838909in}{0.830690in}}%
\pgfpathlineto{\pgfqpoint{3.839762in}{0.830775in}}%
\pgfpathlineto{\pgfqpoint{3.840615in}{0.830860in}}%
\pgfpathlineto{\pgfqpoint{3.841468in}{0.830944in}}%
\pgfpathlineto{\pgfqpoint{3.842321in}{0.831029in}}%
\pgfpathlineto{\pgfqpoint{3.843174in}{0.831114in}}%
\pgfpathlineto{\pgfqpoint{3.844027in}{0.831198in}}%
\pgfpathlineto{\pgfqpoint{3.844879in}{0.831283in}}%
\pgfpathlineto{\pgfqpoint{3.845732in}{0.831368in}}%
\pgfpathlineto{\pgfqpoint{3.846585in}{0.831452in}}%
\pgfpathlineto{\pgfqpoint{3.847438in}{0.831537in}}%
\pgfpathlineto{\pgfqpoint{3.848291in}{0.831622in}}%
\pgfpathlineto{\pgfqpoint{3.849144in}{0.831706in}}%
\pgfpathlineto{\pgfqpoint{3.849997in}{0.831791in}}%
\pgfpathlineto{\pgfqpoint{3.850850in}{0.831876in}}%
\pgfpathlineto{\pgfqpoint{3.851702in}{0.831933in}}%
\pgfpathlineto{\pgfqpoint{3.852555in}{0.831872in}}%
\pgfpathlineto{\pgfqpoint{3.853408in}{0.832204in}}%
\pgfpathlineto{\pgfqpoint{3.854261in}{0.831865in}}%
\pgfpathlineto{\pgfqpoint{3.855114in}{0.831136in}}%
\pgfpathlineto{\pgfqpoint{3.855967in}{0.830497in}}%
\pgfpathlineto{\pgfqpoint{3.856820in}{0.830498in}}%
\pgfpathlineto{\pgfqpoint{3.857673in}{0.830624in}}%
\pgfpathlineto{\pgfqpoint{3.858526in}{0.830751in}}%
\pgfpathlineto{\pgfqpoint{3.859378in}{0.830877in}}%
\pgfpathlineto{\pgfqpoint{3.860231in}{0.831003in}}%
\pgfpathlineto{\pgfqpoint{3.861084in}{0.831130in}}%
\pgfpathlineto{\pgfqpoint{3.861937in}{0.831256in}}%
\pgfpathlineto{\pgfqpoint{3.862790in}{0.831382in}}%
\pgfpathlineto{\pgfqpoint{3.863643in}{0.831508in}}%
\pgfpathlineto{\pgfqpoint{3.864496in}{0.831635in}}%
\pgfpathlineto{\pgfqpoint{3.865349in}{0.831761in}}%
\pgfpathlineto{\pgfqpoint{3.866201in}{0.831887in}}%
\pgfpathlineto{\pgfqpoint{3.867054in}{0.832013in}}%
\pgfpathlineto{\pgfqpoint{3.867907in}{0.832140in}}%
\pgfpathlineto{\pgfqpoint{3.868760in}{0.832266in}}%
\pgfpathlineto{\pgfqpoint{3.869613in}{0.832093in}}%
\pgfpathlineto{\pgfqpoint{3.870466in}{0.831106in}}%
\pgfpathlineto{\pgfqpoint{3.871319in}{0.832570in}}%
\pgfpathlineto{\pgfqpoint{3.872172in}{0.832435in}}%
\pgfpathlineto{\pgfqpoint{3.873024in}{0.831139in}}%
\pgfpathlineto{\pgfqpoint{3.873877in}{0.831782in}}%
\pgfpathlineto{\pgfqpoint{3.874730in}{0.832598in}}%
\pgfpathlineto{\pgfqpoint{3.875583in}{0.832595in}}%
\pgfpathlineto{\pgfqpoint{3.876436in}{0.832591in}}%
\pgfpathlineto{\pgfqpoint{3.877289in}{0.832587in}}%
\pgfpathlineto{\pgfqpoint{3.878142in}{0.832584in}}%
\pgfpathlineto{\pgfqpoint{3.878995in}{0.832580in}}%
\pgfpathlineto{\pgfqpoint{3.879847in}{0.832576in}}%
\pgfpathlineto{\pgfqpoint{3.880700in}{0.832573in}}%
\pgfpathlineto{\pgfqpoint{3.881553in}{0.832569in}}%
\pgfpathlineto{\pgfqpoint{3.882406in}{0.832565in}}%
\pgfpathlineto{\pgfqpoint{3.883259in}{0.832562in}}%
\pgfpathlineto{\pgfqpoint{3.884112in}{0.832558in}}%
\pgfpathlineto{\pgfqpoint{3.884965in}{0.832554in}}%
\pgfpathlineto{\pgfqpoint{3.885818in}{0.832551in}}%
\pgfpathlineto{\pgfqpoint{3.886671in}{0.832547in}}%
\pgfpathlineto{\pgfqpoint{3.887523in}{0.832543in}}%
\pgfpathlineto{\pgfqpoint{3.888376in}{0.832540in}}%
\pgfpathlineto{\pgfqpoint{3.889229in}{0.832536in}}%
\pgfpathlineto{\pgfqpoint{3.890082in}{0.832532in}}%
\pgfpathlineto{\pgfqpoint{3.890935in}{0.832529in}}%
\pgfpathlineto{\pgfqpoint{3.891788in}{0.832525in}}%
\pgfpathlineto{\pgfqpoint{3.892641in}{0.832522in}}%
\pgfpathlineto{\pgfqpoint{3.893494in}{0.832518in}}%
\pgfpathlineto{\pgfqpoint{3.894346in}{0.832514in}}%
\pgfpathlineto{\pgfqpoint{3.895199in}{0.832511in}}%
\pgfpathlineto{\pgfqpoint{3.896052in}{0.832507in}}%
\pgfpathlineto{\pgfqpoint{3.896905in}{0.832503in}}%
\pgfpathlineto{\pgfqpoint{3.897758in}{0.832500in}}%
\pgfpathlineto{\pgfqpoint{3.898611in}{0.832496in}}%
\pgfpathlineto{\pgfqpoint{3.899464in}{0.832492in}}%
\pgfpathlineto{\pgfqpoint{3.900317in}{0.832489in}}%
\pgfpathlineto{\pgfqpoint{3.901169in}{0.832485in}}%
\pgfpathlineto{\pgfqpoint{3.902022in}{0.832481in}}%
\pgfpathlineto{\pgfqpoint{3.902875in}{0.832478in}}%
\pgfpathlineto{\pgfqpoint{3.903728in}{0.832474in}}%
\pgfpathlineto{\pgfqpoint{3.904581in}{0.832470in}}%
\pgfpathlineto{\pgfqpoint{3.905434in}{0.832467in}}%
\pgfpathlineto{\pgfqpoint{3.906287in}{0.832463in}}%
\pgfpathlineto{\pgfqpoint{3.907140in}{0.832460in}}%
\pgfpathlineto{\pgfqpoint{3.907992in}{0.832456in}}%
\pgfpathlineto{\pgfqpoint{3.908845in}{0.832452in}}%
\pgfpathlineto{\pgfqpoint{3.909698in}{0.832449in}}%
\pgfpathlineto{\pgfqpoint{3.910551in}{0.832445in}}%
\pgfpathlineto{\pgfqpoint{3.911404in}{0.832441in}}%
\pgfpathlineto{\pgfqpoint{3.912257in}{0.832438in}}%
\pgfpathlineto{\pgfqpoint{3.913110in}{0.832434in}}%
\pgfpathlineto{\pgfqpoint{3.913963in}{0.832430in}}%
\pgfpathlineto{\pgfqpoint{3.914816in}{0.832427in}}%
\pgfpathlineto{\pgfqpoint{3.915668in}{0.832423in}}%
\pgfpathlineto{\pgfqpoint{3.916521in}{0.832419in}}%
\pgfpathlineto{\pgfqpoint{3.917374in}{0.832416in}}%
\pgfpathlineto{\pgfqpoint{3.918227in}{0.832412in}}%
\pgfpathlineto{\pgfqpoint{3.919080in}{0.832408in}}%
\pgfpathlineto{\pgfqpoint{3.919933in}{0.832405in}}%
\pgfpathlineto{\pgfqpoint{3.920786in}{0.832401in}}%
\pgfpathlineto{\pgfqpoint{3.921639in}{0.832398in}}%
\pgfpathlineto{\pgfqpoint{3.922491in}{0.832394in}}%
\pgfpathlineto{\pgfqpoint{3.923344in}{0.832390in}}%
\pgfpathlineto{\pgfqpoint{3.924197in}{0.832387in}}%
\pgfpathlineto{\pgfqpoint{3.925050in}{0.832383in}}%
\pgfpathlineto{\pgfqpoint{3.925903in}{0.832379in}}%
\pgfpathlineto{\pgfqpoint{3.926756in}{0.832376in}}%
\pgfpathlineto{\pgfqpoint{3.927609in}{0.832372in}}%
\pgfpathlineto{\pgfqpoint{3.928462in}{0.832368in}}%
\pgfpathlineto{\pgfqpoint{3.929314in}{0.832365in}}%
\pgfpathlineto{\pgfqpoint{3.930167in}{0.832361in}}%
\pgfpathlineto{\pgfqpoint{3.931020in}{0.832357in}}%
\pgfpathlineto{\pgfqpoint{3.931873in}{0.832354in}}%
\pgfpathlineto{\pgfqpoint{3.932726in}{0.832350in}}%
\pgfpathlineto{\pgfqpoint{3.933579in}{0.832346in}}%
\pgfpathlineto{\pgfqpoint{3.934432in}{0.832343in}}%
\pgfpathlineto{\pgfqpoint{3.935285in}{0.832339in}}%
\pgfpathlineto{\pgfqpoint{3.936137in}{0.832336in}}%
\pgfpathlineto{\pgfqpoint{3.936990in}{0.832332in}}%
\pgfpathlineto{\pgfqpoint{3.937843in}{0.832328in}}%
\pgfpathlineto{\pgfqpoint{3.938696in}{0.832325in}}%
\pgfpathlineto{\pgfqpoint{3.939549in}{0.832321in}}%
\pgfpathlineto{\pgfqpoint{3.940402in}{0.832317in}}%
\pgfpathlineto{\pgfqpoint{3.941255in}{0.832314in}}%
\pgfpathlineto{\pgfqpoint{3.942108in}{0.832310in}}%
\pgfpathlineto{\pgfqpoint{3.942960in}{0.832306in}}%
\pgfpathlineto{\pgfqpoint{3.943813in}{0.832303in}}%
\pgfpathlineto{\pgfqpoint{3.944666in}{0.832299in}}%
\pgfpathlineto{\pgfqpoint{3.945519in}{0.832295in}}%
\pgfpathlineto{\pgfqpoint{3.946372in}{0.832292in}}%
\pgfpathlineto{\pgfqpoint{3.947225in}{0.832288in}}%
\pgfpathlineto{\pgfqpoint{3.948078in}{0.832284in}}%
\pgfpathlineto{\pgfqpoint{3.948931in}{0.832281in}}%
\pgfpathlineto{\pgfqpoint{3.949784in}{0.832277in}}%
\pgfpathlineto{\pgfqpoint{3.950636in}{0.832274in}}%
\pgfpathlineto{\pgfqpoint{3.951489in}{0.832270in}}%
\pgfpathlineto{\pgfqpoint{3.952342in}{0.832266in}}%
\pgfpathlineto{\pgfqpoint{3.953195in}{0.832263in}}%
\pgfpathlineto{\pgfqpoint{3.954048in}{0.832259in}}%
\pgfpathlineto{\pgfqpoint{3.954901in}{0.832255in}}%
\pgfpathlineto{\pgfqpoint{3.955754in}{0.832252in}}%
\pgfpathlineto{\pgfqpoint{3.956607in}{0.832248in}}%
\pgfpathlineto{\pgfqpoint{3.957459in}{0.832244in}}%
\pgfpathlineto{\pgfqpoint{3.958312in}{0.832241in}}%
\pgfpathlineto{\pgfqpoint{3.959165in}{0.832237in}}%
\pgfpathlineto{\pgfqpoint{3.960018in}{0.832233in}}%
\pgfpathlineto{\pgfqpoint{3.960871in}{0.832230in}}%
\pgfpathlineto{\pgfqpoint{3.961724in}{0.832226in}}%
\pgfpathlineto{\pgfqpoint{3.962577in}{0.832222in}}%
\pgfpathlineto{\pgfqpoint{3.963430in}{0.832219in}}%
\pgfpathlineto{\pgfqpoint{3.964282in}{0.832215in}}%
\pgfpathlineto{\pgfqpoint{3.965135in}{0.832212in}}%
\pgfpathlineto{\pgfqpoint{3.965988in}{0.832208in}}%
\pgfpathlineto{\pgfqpoint{3.966841in}{0.832204in}}%
\pgfpathlineto{\pgfqpoint{3.967694in}{0.832201in}}%
\pgfpathlineto{\pgfqpoint{3.968547in}{0.832197in}}%
\pgfpathlineto{\pgfqpoint{3.969400in}{0.832193in}}%
\pgfpathlineto{\pgfqpoint{3.970253in}{0.832190in}}%
\pgfpathlineto{\pgfqpoint{3.971105in}{0.832186in}}%
\pgfpathlineto{\pgfqpoint{3.971958in}{0.832182in}}%
\pgfpathlineto{\pgfqpoint{3.972811in}{0.832179in}}%
\pgfpathlineto{\pgfqpoint{3.973664in}{0.832175in}}%
\pgfpathlineto{\pgfqpoint{3.974517in}{0.832171in}}%
\pgfpathlineto{\pgfqpoint{3.975370in}{0.832168in}}%
\pgfpathlineto{\pgfqpoint{3.976223in}{0.832164in}}%
\pgfpathlineto{\pgfqpoint{3.977076in}{0.832160in}}%
\pgfpathlineto{\pgfqpoint{3.977929in}{0.832157in}}%
\pgfpathlineto{\pgfqpoint{3.978781in}{0.832153in}}%
\pgfpathlineto{\pgfqpoint{3.979634in}{0.832150in}}%
\pgfpathlineto{\pgfqpoint{3.980487in}{0.832146in}}%
\pgfpathlineto{\pgfqpoint{3.981340in}{0.832142in}}%
\pgfpathlineto{\pgfqpoint{3.982193in}{0.832139in}}%
\pgfpathlineto{\pgfqpoint{3.983046in}{0.832135in}}%
\pgfpathlineto{\pgfqpoint{3.983899in}{0.832131in}}%
\pgfpathlineto{\pgfqpoint{3.984752in}{0.832128in}}%
\pgfpathlineto{\pgfqpoint{3.985604in}{0.832124in}}%
\pgfpathlineto{\pgfqpoint{3.986457in}{0.832120in}}%
\pgfpathlineto{\pgfqpoint{3.987310in}{0.832117in}}%
\pgfpathlineto{\pgfqpoint{3.988163in}{0.832113in}}%
\pgfpathlineto{\pgfqpoint{3.989016in}{0.832109in}}%
\pgfpathlineto{\pgfqpoint{3.989869in}{0.832106in}}%
\pgfpathlineto{\pgfqpoint{3.990722in}{0.832102in}}%
\pgfpathlineto{\pgfqpoint{3.991575in}{0.832098in}}%
\pgfpathlineto{\pgfqpoint{3.992427in}{0.832095in}}%
\pgfpathlineto{\pgfqpoint{3.993280in}{0.832091in}}%
\pgfpathlineto{\pgfqpoint{3.994133in}{0.832088in}}%
\pgfpathlineto{\pgfqpoint{3.994986in}{0.832084in}}%
\pgfpathlineto{\pgfqpoint{3.995839in}{0.832080in}}%
\pgfpathlineto{\pgfqpoint{3.996692in}{0.832077in}}%
\pgfpathlineto{\pgfqpoint{3.997545in}{0.832073in}}%
\pgfpathlineto{\pgfqpoint{3.998398in}{0.832069in}}%
\pgfpathlineto{\pgfqpoint{3.999250in}{0.832066in}}%
\pgfpathlineto{\pgfqpoint{4.000103in}{0.832062in}}%
\pgfpathlineto{\pgfqpoint{4.000956in}{0.832058in}}%
\pgfpathlineto{\pgfqpoint{4.001809in}{0.832055in}}%
\pgfpathlineto{\pgfqpoint{4.002662in}{0.832051in}}%
\pgfpathlineto{\pgfqpoint{4.003515in}{0.832047in}}%
\pgfpathlineto{\pgfqpoint{4.004368in}{0.832044in}}%
\pgfpathlineto{\pgfqpoint{4.005221in}{0.832040in}}%
\pgfpathlineto{\pgfqpoint{4.006074in}{0.832036in}}%
\pgfpathlineto{\pgfqpoint{4.006926in}{0.832033in}}%
\pgfpathlineto{\pgfqpoint{4.007779in}{0.832029in}}%
\pgfpathlineto{\pgfqpoint{4.008632in}{0.832026in}}%
\pgfpathlineto{\pgfqpoint{4.009485in}{0.832022in}}%
\pgfpathlineto{\pgfqpoint{4.010338in}{0.832018in}}%
\pgfpathlineto{\pgfqpoint{4.011191in}{0.832015in}}%
\pgfpathlineto{\pgfqpoint{4.012044in}{0.832011in}}%
\pgfpathlineto{\pgfqpoint{4.012897in}{0.832007in}}%
\pgfpathlineto{\pgfqpoint{4.013749in}{0.832004in}}%
\pgfpathlineto{\pgfqpoint{4.014602in}{0.832000in}}%
\pgfpathlineto{\pgfqpoint{4.015455in}{0.831996in}}%
\pgfpathlineto{\pgfqpoint{4.016308in}{0.831993in}}%
\pgfpathlineto{\pgfqpoint{4.017161in}{0.831989in}}%
\pgfpathlineto{\pgfqpoint{4.018014in}{0.831985in}}%
\pgfpathlineto{\pgfqpoint{4.018867in}{0.831982in}}%
\pgfpathlineto{\pgfqpoint{4.019720in}{0.831978in}}%
\pgfpathlineto{\pgfqpoint{4.020572in}{0.831974in}}%
\pgfpathlineto{\pgfqpoint{4.021425in}{0.831971in}}%
\pgfpathlineto{\pgfqpoint{4.022278in}{0.831967in}}%
\pgfpathlineto{\pgfqpoint{4.023131in}{0.831964in}}%
\pgfpathlineto{\pgfqpoint{4.023984in}{0.831960in}}%
\pgfpathlineto{\pgfqpoint{4.024837in}{0.831956in}}%
\pgfpathlineto{\pgfqpoint{4.025690in}{0.831953in}}%
\pgfpathlineto{\pgfqpoint{4.026543in}{0.831949in}}%
\pgfpathlineto{\pgfqpoint{4.027395in}{0.831945in}}%
\pgfpathlineto{\pgfqpoint{4.028248in}{0.831942in}}%
\pgfpathlineto{\pgfqpoint{4.029101in}{0.831938in}}%
\pgfpathlineto{\pgfqpoint{4.029954in}{0.831934in}}%
\pgfpathlineto{\pgfqpoint{4.030807in}{0.831931in}}%
\pgfpathlineto{\pgfqpoint{4.031660in}{0.831927in}}%
\pgfpathlineto{\pgfqpoint{4.032513in}{0.831923in}}%
\pgfpathlineto{\pgfqpoint{4.033366in}{0.831920in}}%
\pgfpathlineto{\pgfqpoint{4.034219in}{0.831916in}}%
\pgfpathlineto{\pgfqpoint{4.035071in}{0.831912in}}%
\pgfpathlineto{\pgfqpoint{4.035924in}{0.831909in}}%
\pgfpathlineto{\pgfqpoint{4.036777in}{0.831905in}}%
\pgfpathlineto{\pgfqpoint{4.037630in}{0.831902in}}%
\pgfpathlineto{\pgfqpoint{4.038483in}{0.831898in}}%
\pgfpathlineto{\pgfqpoint{4.039336in}{0.831894in}}%
\pgfpathlineto{\pgfqpoint{4.040189in}{0.831891in}}%
\pgfpathlineto{\pgfqpoint{4.041042in}{0.831887in}}%
\pgfpathlineto{\pgfqpoint{4.041894in}{0.831883in}}%
\pgfpathlineto{\pgfqpoint{4.042747in}{0.831880in}}%
\pgfpathlineto{\pgfqpoint{4.043600in}{0.831876in}}%
\pgfpathlineto{\pgfqpoint{4.044453in}{0.831872in}}%
\pgfpathlineto{\pgfqpoint{4.045306in}{0.831869in}}%
\pgfpathlineto{\pgfqpoint{4.046159in}{0.831865in}}%
\pgfpathlineto{\pgfqpoint{4.047012in}{0.831861in}}%
\pgfpathlineto{\pgfqpoint{4.047865in}{0.831858in}}%
\pgfpathlineto{\pgfqpoint{4.048717in}{0.831854in}}%
\pgfpathlineto{\pgfqpoint{4.049570in}{0.831850in}}%
\pgfpathlineto{\pgfqpoint{4.050423in}{0.831847in}}%
\pgfpathlineto{\pgfqpoint{4.051276in}{0.831843in}}%
\pgfpathlineto{\pgfqpoint{4.052129in}{0.831840in}}%
\pgfpathlineto{\pgfqpoint{4.052982in}{0.831836in}}%
\pgfpathlineto{\pgfqpoint{4.053835in}{0.831832in}}%
\pgfpathlineto{\pgfqpoint{4.054688in}{0.831829in}}%
\pgfpathlineto{\pgfqpoint{4.055540in}{0.831825in}}%
\pgfpathlineto{\pgfqpoint{4.056393in}{0.831821in}}%
\pgfpathlineto{\pgfqpoint{4.057246in}{0.831818in}}%
\pgfpathlineto{\pgfqpoint{4.058099in}{0.831814in}}%
\pgfpathlineto{\pgfqpoint{4.058952in}{0.831810in}}%
\pgfpathlineto{\pgfqpoint{4.059805in}{0.831807in}}%
\pgfpathlineto{\pgfqpoint{4.060658in}{0.831803in}}%
\pgfpathlineto{\pgfqpoint{4.061511in}{0.831799in}}%
\pgfpathlineto{\pgfqpoint{4.062363in}{0.831796in}}%
\pgfpathlineto{\pgfqpoint{4.063216in}{0.831792in}}%
\pgfpathlineto{\pgfqpoint{4.064069in}{0.831788in}}%
\pgfpathlineto{\pgfqpoint{4.064922in}{0.831785in}}%
\pgfpathlineto{\pgfqpoint{4.065775in}{0.831781in}}%
\pgfpathlineto{\pgfqpoint{4.066628in}{0.831778in}}%
\pgfpathlineto{\pgfqpoint{4.067481in}{0.831774in}}%
\pgfpathlineto{\pgfqpoint{4.068334in}{0.831770in}}%
\pgfpathlineto{\pgfqpoint{4.069187in}{0.831767in}}%
\pgfpathlineto{\pgfqpoint{4.070039in}{0.831763in}}%
\pgfpathlineto{\pgfqpoint{4.070892in}{0.831759in}}%
\pgfpathlineto{\pgfqpoint{4.071745in}{0.831756in}}%
\pgfpathlineto{\pgfqpoint{4.072598in}{0.831752in}}%
\pgfpathlineto{\pgfqpoint{4.073451in}{0.831748in}}%
\pgfpathlineto{\pgfqpoint{4.074304in}{0.831745in}}%
\pgfpathlineto{\pgfqpoint{4.075157in}{0.831741in}}%
\pgfpathlineto{\pgfqpoint{4.076010in}{0.831737in}}%
\pgfpathlineto{\pgfqpoint{4.076862in}{0.831734in}}%
\pgfpathlineto{\pgfqpoint{4.077715in}{0.831730in}}%
\pgfpathlineto{\pgfqpoint{4.078568in}{0.831726in}}%
\pgfpathlineto{\pgfqpoint{4.079421in}{0.831723in}}%
\pgfpathlineto{\pgfqpoint{4.080274in}{0.831719in}}%
\pgfpathlineto{\pgfqpoint{4.081127in}{0.831716in}}%
\pgfpathlineto{\pgfqpoint{4.081980in}{0.831712in}}%
\pgfpathlineto{\pgfqpoint{4.082833in}{0.831708in}}%
\pgfpathlineto{\pgfqpoint{4.083685in}{0.831705in}}%
\pgfpathlineto{\pgfqpoint{4.084538in}{0.831701in}}%
\pgfpathlineto{\pgfqpoint{4.085391in}{0.831697in}}%
\pgfpathlineto{\pgfqpoint{4.086244in}{0.831694in}}%
\pgfpathlineto{\pgfqpoint{4.087097in}{0.831690in}}%
\pgfpathlineto{\pgfqpoint{4.087950in}{0.831686in}}%
\pgfpathlineto{\pgfqpoint{4.088803in}{0.831683in}}%
\pgfpathlineto{\pgfqpoint{4.089656in}{0.831679in}}%
\pgfpathlineto{\pgfqpoint{4.090508in}{0.831675in}}%
\pgfpathlineto{\pgfqpoint{4.091361in}{0.831672in}}%
\pgfpathlineto{\pgfqpoint{4.092214in}{0.831668in}}%
\pgfpathlineto{\pgfqpoint{4.093067in}{0.831664in}}%
\pgfpathlineto{\pgfqpoint{4.093920in}{0.831661in}}%
\pgfpathlineto{\pgfqpoint{4.094773in}{0.831657in}}%
\pgfpathlineto{\pgfqpoint{4.095626in}{0.831654in}}%
\pgfpathlineto{\pgfqpoint{4.096479in}{0.831663in}}%
\pgfpathlineto{\pgfqpoint{4.097332in}{0.831677in}}%
\pgfpathlineto{\pgfqpoint{4.098184in}{0.831690in}}%
\pgfpathlineto{\pgfqpoint{4.099037in}{0.831704in}}%
\pgfpathlineto{\pgfqpoint{4.099890in}{0.831718in}}%
\pgfpathlineto{\pgfqpoint{4.100743in}{0.831732in}}%
\pgfpathlineto{\pgfqpoint{4.101596in}{0.831746in}}%
\pgfpathlineto{\pgfqpoint{4.102449in}{0.831760in}}%
\pgfpathlineto{\pgfqpoint{4.103302in}{0.831774in}}%
\pgfpathlineto{\pgfqpoint{4.104155in}{0.831788in}}%
\pgfpathlineto{\pgfqpoint{4.105007in}{0.831802in}}%
\pgfpathlineto{\pgfqpoint{4.105860in}{0.831816in}}%
\pgfpathlineto{\pgfqpoint{4.106713in}{0.831830in}}%
\pgfpathlineto{\pgfqpoint{4.107566in}{0.831844in}}%
\pgfpathlineto{\pgfqpoint{4.108419in}{0.831858in}}%
\pgfpathlineto{\pgfqpoint{4.109272in}{0.831871in}}%
\pgfpathlineto{\pgfqpoint{4.110125in}{0.831885in}}%
\pgfpathlineto{\pgfqpoint{4.110978in}{0.831899in}}%
\pgfpathlineto{\pgfqpoint{4.111830in}{0.831913in}}%
\pgfpathlineto{\pgfqpoint{4.112683in}{0.831927in}}%
\pgfpathlineto{\pgfqpoint{4.113536in}{0.831941in}}%
\pgfpathlineto{\pgfqpoint{4.114389in}{0.831955in}}%
\pgfpathlineto{\pgfqpoint{4.115242in}{0.831969in}}%
\pgfpathlineto{\pgfqpoint{4.116095in}{0.834020in}}%
\pgfpathlineto{\pgfqpoint{4.116948in}{0.834674in}}%
\pgfpathlineto{\pgfqpoint{4.117801in}{0.834643in}}%
\pgfpathlineto{\pgfqpoint{4.118653in}{0.834613in}}%
\pgfpathlineto{\pgfqpoint{4.119506in}{0.834582in}}%
\pgfpathlineto{\pgfqpoint{4.120359in}{0.834552in}}%
\pgfpathlineto{\pgfqpoint{4.121212in}{0.834521in}}%
\pgfpathlineto{\pgfqpoint{4.122065in}{0.834491in}}%
\pgfpathlineto{\pgfqpoint{4.122918in}{0.834460in}}%
\pgfpathlineto{\pgfqpoint{4.123771in}{0.834430in}}%
\pgfpathlineto{\pgfqpoint{4.124624in}{0.834399in}}%
\pgfpathlineto{\pgfqpoint{4.125477in}{0.834369in}}%
\pgfpathlineto{\pgfqpoint{4.126329in}{0.834338in}}%
\pgfpathlineto{\pgfqpoint{4.127182in}{0.834308in}}%
\pgfpathlineto{\pgfqpoint{4.128035in}{0.834277in}}%
\pgfpathlineto{\pgfqpoint{4.128888in}{0.834247in}}%
\pgfpathlineto{\pgfqpoint{4.129741in}{0.834216in}}%
\pgfpathlineto{\pgfqpoint{4.130594in}{0.834186in}}%
\pgfpathlineto{\pgfqpoint{4.131447in}{0.834155in}}%
\pgfpathlineto{\pgfqpoint{4.132300in}{0.834125in}}%
\pgfpathlineto{\pgfqpoint{4.133152in}{0.834094in}}%
\pgfpathlineto{\pgfqpoint{4.134005in}{0.834064in}}%
\pgfpathlineto{\pgfqpoint{4.134858in}{0.834034in}}%
\pgfpathlineto{\pgfqpoint{4.135711in}{0.834003in}}%
\pgfpathlineto{\pgfqpoint{4.136564in}{0.833973in}}%
\pgfpathlineto{\pgfqpoint{4.137417in}{0.833942in}}%
\pgfpathlineto{\pgfqpoint{4.138270in}{0.833912in}}%
\pgfpathlineto{\pgfqpoint{4.139123in}{0.833881in}}%
\pgfpathlineto{\pgfqpoint{4.139975in}{0.833851in}}%
\pgfpathlineto{\pgfqpoint{4.140828in}{0.833820in}}%
\pgfpathlineto{\pgfqpoint{4.141681in}{0.833790in}}%
\pgfpathlineto{\pgfqpoint{4.142534in}{0.833759in}}%
\pgfpathlineto{\pgfqpoint{4.143387in}{0.833729in}}%
\pgfpathlineto{\pgfqpoint{4.144240in}{0.833698in}}%
\pgfpathlineto{\pgfqpoint{4.145093in}{0.833668in}}%
\pgfpathlineto{\pgfqpoint{4.145946in}{0.833637in}}%
\pgfpathlineto{\pgfqpoint{4.146798in}{0.833607in}}%
\pgfpathlineto{\pgfqpoint{4.147651in}{0.833576in}}%
\pgfpathlineto{\pgfqpoint{4.148504in}{0.833546in}}%
\pgfpathlineto{\pgfqpoint{4.149357in}{0.833515in}}%
\pgfpathlineto{\pgfqpoint{4.150210in}{0.833485in}}%
\pgfpathlineto{\pgfqpoint{4.151063in}{0.833454in}}%
\pgfpathlineto{\pgfqpoint{4.151916in}{0.833424in}}%
\pgfpathlineto{\pgfqpoint{4.152769in}{0.833393in}}%
\pgfpathlineto{\pgfqpoint{4.153622in}{0.833363in}}%
\pgfpathlineto{\pgfqpoint{4.154474in}{0.833333in}}%
\pgfpathlineto{\pgfqpoint{4.155327in}{0.833302in}}%
\pgfpathlineto{\pgfqpoint{4.156180in}{0.833272in}}%
\pgfpathlineto{\pgfqpoint{4.157033in}{0.833241in}}%
\pgfpathlineto{\pgfqpoint{4.157886in}{0.833211in}}%
\pgfpathlineto{\pgfqpoint{4.158739in}{0.833180in}}%
\pgfpathlineto{\pgfqpoint{4.159592in}{0.833150in}}%
\pgfpathlineto{\pgfqpoint{4.160445in}{0.833119in}}%
\pgfpathlineto{\pgfqpoint{4.161297in}{0.833089in}}%
\pgfpathlineto{\pgfqpoint{4.162150in}{0.833058in}}%
\pgfpathlineto{\pgfqpoint{4.163003in}{0.833028in}}%
\pgfpathlineto{\pgfqpoint{4.163856in}{0.832997in}}%
\pgfpathlineto{\pgfqpoint{4.164709in}{0.832967in}}%
\pgfpathlineto{\pgfqpoint{4.165562in}{0.832936in}}%
\pgfpathlineto{\pgfqpoint{4.166415in}{0.832906in}}%
\pgfpathlineto{\pgfqpoint{4.167268in}{0.832875in}}%
\pgfpathlineto{\pgfqpoint{4.168120in}{0.832845in}}%
\pgfpathlineto{\pgfqpoint{4.168973in}{0.832814in}}%
\pgfpathlineto{\pgfqpoint{4.169826in}{0.832784in}}%
\pgfpathlineto{\pgfqpoint{4.170679in}{0.832753in}}%
\pgfpathlineto{\pgfqpoint{4.171532in}{0.832723in}}%
\pgfpathlineto{\pgfqpoint{4.172385in}{0.832692in}}%
\pgfpathlineto{\pgfqpoint{4.173238in}{0.832662in}}%
\pgfpathlineto{\pgfqpoint{4.174091in}{0.832632in}}%
\pgfpathlineto{\pgfqpoint{4.174943in}{0.832601in}}%
\pgfpathlineto{\pgfqpoint{4.175796in}{0.832571in}}%
\pgfpathlineto{\pgfqpoint{4.176649in}{0.832540in}}%
\pgfpathlineto{\pgfqpoint{4.177502in}{0.832510in}}%
\pgfpathlineto{\pgfqpoint{4.178355in}{0.832479in}}%
\pgfpathlineto{\pgfqpoint{4.179208in}{0.832449in}}%
\pgfpathlineto{\pgfqpoint{4.180061in}{0.832418in}}%
\pgfpathlineto{\pgfqpoint{4.180914in}{0.832388in}}%
\pgfpathlineto{\pgfqpoint{4.181766in}{0.832357in}}%
\pgfpathlineto{\pgfqpoint{4.182619in}{0.832327in}}%
\pgfpathlineto{\pgfqpoint{4.183472in}{0.832296in}}%
\pgfpathlineto{\pgfqpoint{4.184325in}{0.832266in}}%
\pgfpathlineto{\pgfqpoint{4.185178in}{0.832235in}}%
\pgfpathlineto{\pgfqpoint{4.186031in}{0.832205in}}%
\pgfpathlineto{\pgfqpoint{4.186884in}{0.832174in}}%
\pgfpathlineto{\pgfqpoint{4.187737in}{0.832144in}}%
\pgfpathlineto{\pgfqpoint{4.188590in}{0.832113in}}%
\pgfpathlineto{\pgfqpoint{4.189442in}{0.832083in}}%
\pgfpathlineto{\pgfqpoint{4.190295in}{0.832052in}}%
\pgfpathlineto{\pgfqpoint{4.191148in}{0.832022in}}%
\pgfpathlineto{\pgfqpoint{4.192001in}{0.831992in}}%
\pgfpathlineto{\pgfqpoint{4.192854in}{0.831961in}}%
\pgfpathlineto{\pgfqpoint{4.193707in}{0.831931in}}%
\pgfpathlineto{\pgfqpoint{4.194560in}{0.831900in}}%
\pgfpathlineto{\pgfqpoint{4.195413in}{0.831870in}}%
\pgfpathlineto{\pgfqpoint{4.196265in}{0.831839in}}%
\pgfpathlineto{\pgfqpoint{4.197118in}{0.831809in}}%
\pgfpathlineto{\pgfqpoint{4.197971in}{0.831778in}}%
\pgfpathlineto{\pgfqpoint{4.198824in}{0.831748in}}%
\pgfpathlineto{\pgfqpoint{4.199677in}{0.831717in}}%
\pgfpathlineto{\pgfqpoint{4.200530in}{0.831687in}}%
\pgfpathlineto{\pgfqpoint{4.201383in}{0.831656in}}%
\pgfpathlineto{\pgfqpoint{4.202236in}{0.831626in}}%
\pgfpathlineto{\pgfqpoint{4.203088in}{0.831595in}}%
\pgfpathlineto{\pgfqpoint{4.203941in}{0.831565in}}%
\pgfpathlineto{\pgfqpoint{4.204794in}{0.831534in}}%
\pgfpathlineto{\pgfqpoint{4.205647in}{0.831504in}}%
\pgfpathlineto{\pgfqpoint{4.206500in}{0.831473in}}%
\pgfpathlineto{\pgfqpoint{4.207353in}{0.831443in}}%
\pgfpathlineto{\pgfqpoint{4.208206in}{0.831412in}}%
\pgfpathlineto{\pgfqpoint{4.209059in}{0.831382in}}%
\pgfpathlineto{\pgfqpoint{4.209911in}{0.831351in}}%
\pgfpathlineto{\pgfqpoint{4.210764in}{0.831321in}}%
\pgfpathlineto{\pgfqpoint{4.211617in}{0.831291in}}%
\pgfpathlineto{\pgfqpoint{4.212470in}{0.831260in}}%
\pgfpathlineto{\pgfqpoint{4.213323in}{0.831230in}}%
\pgfpathlineto{\pgfqpoint{4.214176in}{0.831199in}}%
\pgfpathlineto{\pgfqpoint{4.215029in}{0.831169in}}%
\pgfpathlineto{\pgfqpoint{4.215882in}{0.831138in}}%
\pgfpathlineto{\pgfqpoint{4.216735in}{0.831108in}}%
\pgfpathlineto{\pgfqpoint{4.217587in}{0.831077in}}%
\pgfpathlineto{\pgfqpoint{4.218440in}{0.831047in}}%
\pgfpathlineto{\pgfqpoint{4.219293in}{0.831016in}}%
\pgfpathlineto{\pgfqpoint{4.220146in}{0.830986in}}%
\pgfpathlineto{\pgfqpoint{4.220999in}{0.830955in}}%
\pgfpathlineto{\pgfqpoint{4.221852in}{0.830925in}}%
\pgfpathlineto{\pgfqpoint{4.222705in}{0.830894in}}%
\pgfpathlineto{\pgfqpoint{4.223558in}{0.830864in}}%
\pgfpathlineto{\pgfqpoint{4.224410in}{0.830833in}}%
\pgfpathlineto{\pgfqpoint{4.225263in}{0.830803in}}%
\pgfpathlineto{\pgfqpoint{4.226116in}{0.830772in}}%
\pgfpathlineto{\pgfqpoint{4.226969in}{0.830742in}}%
\pgfpathlineto{\pgfqpoint{4.227822in}{0.830711in}}%
\pgfpathlineto{\pgfqpoint{4.228675in}{0.830681in}}%
\pgfpathlineto{\pgfqpoint{4.229528in}{0.830650in}}%
\pgfpathlineto{\pgfqpoint{4.230381in}{0.830620in}}%
\pgfpathlineto{\pgfqpoint{4.231233in}{0.830590in}}%
\pgfpathlineto{\pgfqpoint{4.232086in}{0.830559in}}%
\pgfpathlineto{\pgfqpoint{4.232939in}{0.830529in}}%
\pgfpathlineto{\pgfqpoint{4.233792in}{0.830498in}}%
\pgfpathlineto{\pgfqpoint{4.234645in}{0.830468in}}%
\pgfpathlineto{\pgfqpoint{4.235498in}{0.830437in}}%
\pgfpathlineto{\pgfqpoint{4.236351in}{0.830407in}}%
\pgfpathlineto{\pgfqpoint{4.237204in}{0.830376in}}%
\pgfpathlineto{\pgfqpoint{4.238056in}{0.830346in}}%
\pgfpathlineto{\pgfqpoint{4.238909in}{0.830315in}}%
\pgfpathlineto{\pgfqpoint{4.239762in}{0.830285in}}%
\pgfpathlineto{\pgfqpoint{4.240615in}{0.830254in}}%
\pgfpathlineto{\pgfqpoint{4.241468in}{0.830224in}}%
\pgfpathlineto{\pgfqpoint{4.242321in}{0.830193in}}%
\pgfpathlineto{\pgfqpoint{4.243174in}{0.830163in}}%
\pgfpathlineto{\pgfqpoint{4.244027in}{0.830132in}}%
\pgfpathlineto{\pgfqpoint{4.244880in}{0.830102in}}%
\pgfpathlineto{\pgfqpoint{4.245732in}{0.830071in}}%
\pgfpathlineto{\pgfqpoint{4.246585in}{0.830041in}}%
\pgfpathlineto{\pgfqpoint{4.247438in}{0.829990in}}%
\pgfpathlineto{\pgfqpoint{4.248291in}{0.829888in}}%
\pgfpathlineto{\pgfqpoint{4.249144in}{0.829783in}}%
\pgfpathlineto{\pgfqpoint{4.249997in}{0.829679in}}%
\pgfpathlineto{\pgfqpoint{4.250850in}{0.829574in}}%
\pgfpathlineto{\pgfqpoint{4.251703in}{0.829470in}}%
\pgfpathlineto{\pgfqpoint{4.252555in}{0.829365in}}%
\pgfpathlineto{\pgfqpoint{4.253408in}{0.829260in}}%
\pgfpathlineto{\pgfqpoint{4.254261in}{0.829156in}}%
\pgfpathlineto{\pgfqpoint{4.255114in}{0.829051in}}%
\pgfpathlineto{\pgfqpoint{4.255967in}{0.828947in}}%
\pgfpathlineto{\pgfqpoint{4.256820in}{0.828842in}}%
\pgfpathlineto{\pgfqpoint{4.257673in}{0.828738in}}%
\pgfpathlineto{\pgfqpoint{4.258526in}{0.828633in}}%
\pgfpathlineto{\pgfqpoint{4.259378in}{0.828528in}}%
\pgfpathlineto{\pgfqpoint{4.260231in}{0.828424in}}%
\pgfpathlineto{\pgfqpoint{4.261084in}{0.828319in}}%
\pgfpathlineto{\pgfqpoint{4.261937in}{0.828215in}}%
\pgfpathlineto{\pgfqpoint{4.262790in}{0.828110in}}%
\pgfpathlineto{\pgfqpoint{4.263643in}{0.828006in}}%
\pgfpathlineto{\pgfqpoint{4.264496in}{0.827901in}}%
\pgfpathlineto{\pgfqpoint{4.265349in}{0.827796in}}%
\pgfpathlineto{\pgfqpoint{4.266201in}{0.827692in}}%
\pgfpathlineto{\pgfqpoint{4.267054in}{0.827587in}}%
\pgfpathlineto{\pgfqpoint{4.267907in}{0.827483in}}%
\pgfpathlineto{\pgfqpoint{4.268760in}{0.827378in}}%
\pgfpathlineto{\pgfqpoint{4.269613in}{0.827273in}}%
\pgfpathlineto{\pgfqpoint{4.270466in}{0.827169in}}%
\pgfpathlineto{\pgfqpoint{4.271319in}{0.827094in}}%
\pgfpathlineto{\pgfqpoint{4.272172in}{0.833427in}}%
\pgfpathlineto{\pgfqpoint{4.273024in}{0.834699in}}%
\pgfpathlineto{\pgfqpoint{4.273877in}{0.834735in}}%
\pgfpathlineto{\pgfqpoint{4.274730in}{0.834770in}}%
\pgfpathlineto{\pgfqpoint{4.275583in}{0.834806in}}%
\pgfpathlineto{\pgfqpoint{4.276436in}{0.834841in}}%
\pgfpathlineto{\pgfqpoint{4.277289in}{0.834877in}}%
\pgfpathlineto{\pgfqpoint{4.278142in}{0.834912in}}%
\pgfpathlineto{\pgfqpoint{4.278995in}{0.834948in}}%
\pgfpathlineto{\pgfqpoint{4.279848in}{0.835073in}}%
\pgfpathlineto{\pgfqpoint{4.280700in}{0.835278in}}%
\pgfpathlineto{\pgfqpoint{4.281553in}{0.835483in}}%
\pgfpathlineto{\pgfqpoint{4.282406in}{0.835321in}}%
\pgfpathlineto{\pgfqpoint{4.283259in}{0.832130in}}%
\pgfpathlineto{\pgfqpoint{4.284112in}{0.829983in}}%
\pgfpathlineto{\pgfqpoint{4.284965in}{0.834011in}}%
\pgfpathlineto{\pgfqpoint{4.285818in}{0.836340in}}%
\pgfpathlineto{\pgfqpoint{4.286671in}{0.835643in}}%
\pgfpathlineto{\pgfqpoint{4.287523in}{0.835416in}}%
\pgfpathlineto{\pgfqpoint{4.288376in}{0.835364in}}%
\pgfpathlineto{\pgfqpoint{4.289229in}{0.835312in}}%
\pgfpathlineto{\pgfqpoint{4.290082in}{0.835261in}}%
\pgfpathlineto{\pgfqpoint{4.290935in}{0.835209in}}%
\pgfpathlineto{\pgfqpoint{4.291788in}{0.835158in}}%
\pgfpathlineto{\pgfqpoint{4.292641in}{0.835179in}}%
\pgfpathlineto{\pgfqpoint{4.293494in}{0.835323in}}%
\pgfpathlineto{\pgfqpoint{4.294346in}{0.835470in}}%
\pgfpathlineto{\pgfqpoint{4.295199in}{0.835616in}}%
\pgfpathlineto{\pgfqpoint{4.296052in}{0.835762in}}%
\pgfpathlineto{\pgfqpoint{4.296905in}{0.835908in}}%
\pgfpathlineto{\pgfqpoint{4.297758in}{0.836055in}}%
\pgfpathlineto{\pgfqpoint{4.298611in}{0.836201in}}%
\pgfpathlineto{\pgfqpoint{4.299464in}{0.836347in}}%
\pgfpathlineto{\pgfqpoint{4.300317in}{0.836493in}}%
\pgfpathlineto{\pgfqpoint{4.301169in}{0.836640in}}%
\pgfpathlineto{\pgfqpoint{4.302022in}{0.836779in}}%
\pgfpathlineto{\pgfqpoint{4.302875in}{0.836812in}}%
\pgfpathlineto{\pgfqpoint{4.303728in}{0.836809in}}%
\pgfpathlineto{\pgfqpoint{4.304581in}{0.836330in}}%
\pgfpathlineto{\pgfqpoint{4.305434in}{0.837738in}}%
\pgfpathlineto{\pgfqpoint{4.306287in}{0.839473in}}%
\pgfpathlineto{\pgfqpoint{4.307140in}{0.839954in}}%
\pgfpathlineto{\pgfqpoint{4.307993in}{0.839955in}}%
\pgfpathlineto{\pgfqpoint{4.308845in}{0.839956in}}%
\pgfpathlineto{\pgfqpoint{4.309698in}{0.839957in}}%
\pgfpathlineto{\pgfqpoint{4.310551in}{0.839958in}}%
\pgfpathlineto{\pgfqpoint{4.311404in}{0.839959in}}%
\pgfpathlineto{\pgfqpoint{4.312257in}{0.839960in}}%
\pgfpathlineto{\pgfqpoint{4.313110in}{0.839961in}}%
\pgfpathlineto{\pgfqpoint{4.313963in}{0.839962in}}%
\pgfpathlineto{\pgfqpoint{4.314816in}{0.839963in}}%
\pgfpathlineto{\pgfqpoint{4.315668in}{0.839964in}}%
\pgfpathlineto{\pgfqpoint{4.316521in}{0.839965in}}%
\pgfpathlineto{\pgfqpoint{4.317374in}{0.839967in}}%
\pgfpathlineto{\pgfqpoint{4.318227in}{0.839968in}}%
\pgfpathlineto{\pgfqpoint{4.319080in}{0.839785in}}%
\pgfpathlineto{\pgfqpoint{4.319933in}{0.838271in}}%
\pgfpathlineto{\pgfqpoint{4.320786in}{0.838266in}}%
\pgfpathlineto{\pgfqpoint{4.321639in}{0.838691in}}%
\pgfpathlineto{\pgfqpoint{4.322491in}{0.840248in}}%
\pgfpathlineto{\pgfqpoint{4.323344in}{0.840325in}}%
\pgfpathlineto{\pgfqpoint{4.324197in}{0.840361in}}%
\pgfpathlineto{\pgfqpoint{4.325050in}{0.840397in}}%
\pgfpathlineto{\pgfqpoint{4.325903in}{0.840433in}}%
\pgfpathlineto{\pgfqpoint{4.326756in}{0.840469in}}%
\pgfpathlineto{\pgfqpoint{4.327609in}{0.840505in}}%
\pgfpathlineto{\pgfqpoint{4.328462in}{0.840538in}}%
\pgfpathlineto{\pgfqpoint{4.329314in}{0.840556in}}%
\pgfpathlineto{\pgfqpoint{4.330167in}{0.840564in}}%
\pgfpathlineto{\pgfqpoint{4.331020in}{0.840570in}}%
\pgfpathlineto{\pgfqpoint{4.331873in}{0.840576in}}%
\pgfpathlineto{\pgfqpoint{4.332726in}{0.840582in}}%
\pgfpathlineto{\pgfqpoint{4.333579in}{0.840588in}}%
\pgfpathlineto{\pgfqpoint{4.334432in}{0.840587in}}%
\pgfpathlineto{\pgfqpoint{4.335285in}{0.840456in}}%
\pgfpathlineto{\pgfqpoint{4.336138in}{0.840272in}}%
\pgfpathlineto{\pgfqpoint{4.336990in}{0.840089in}}%
\pgfpathlineto{\pgfqpoint{4.337843in}{0.839905in}}%
\pgfpathlineto{\pgfqpoint{4.338696in}{0.839721in}}%
\pgfpathlineto{\pgfqpoint{4.339549in}{0.839538in}}%
\pgfpathlineto{\pgfqpoint{4.340402in}{0.839354in}}%
\pgfpathlineto{\pgfqpoint{4.341255in}{0.839170in}}%
\pgfpathlineto{\pgfqpoint{4.342108in}{0.838987in}}%
\pgfpathlineto{\pgfqpoint{4.342961in}{0.838803in}}%
\pgfpathlineto{\pgfqpoint{4.343813in}{0.838620in}}%
\pgfpathlineto{\pgfqpoint{4.344666in}{0.838436in}}%
\pgfpathlineto{\pgfqpoint{4.345519in}{0.838252in}}%
\pgfpathlineto{\pgfqpoint{4.346372in}{0.838180in}}%
\pgfpathlineto{\pgfqpoint{4.347225in}{0.838668in}}%
\pgfpathlineto{\pgfqpoint{4.348078in}{0.839252in}}%
\pgfpathlineto{\pgfqpoint{4.348931in}{0.839837in}}%
\pgfpathlineto{\pgfqpoint{4.349784in}{0.840421in}}%
\pgfpathlineto{\pgfqpoint{4.350636in}{0.840939in}}%
\pgfpathlineto{\pgfqpoint{4.351489in}{0.841016in}}%
\pgfpathlineto{\pgfqpoint{4.352342in}{0.841011in}}%
\pgfpathlineto{\pgfqpoint{4.353195in}{0.841007in}}%
\pgfpathlineto{\pgfqpoint{4.354048in}{0.841002in}}%
\pgfpathlineto{\pgfqpoint{4.354901in}{0.840998in}}%
\pgfpathlineto{\pgfqpoint{4.355754in}{0.840993in}}%
\pgfpathlineto{\pgfqpoint{4.356607in}{0.840989in}}%
\pgfpathlineto{\pgfqpoint{4.357459in}{0.840985in}}%
\pgfpathlineto{\pgfqpoint{4.358312in}{0.840980in}}%
\pgfpathlineto{\pgfqpoint{4.359165in}{0.840976in}}%
\pgfpathlineto{\pgfqpoint{4.360018in}{0.840971in}}%
\pgfpathlineto{\pgfqpoint{4.360871in}{0.840967in}}%
\pgfpathlineto{\pgfqpoint{4.361724in}{0.840962in}}%
\pgfpathlineto{\pgfqpoint{4.362577in}{0.840958in}}%
\pgfpathlineto{\pgfqpoint{4.363430in}{0.840954in}}%
\pgfpathlineto{\pgfqpoint{4.364283in}{0.840949in}}%
\pgfpathlineto{\pgfqpoint{4.365135in}{0.840945in}}%
\pgfpathlineto{\pgfqpoint{4.365988in}{0.840940in}}%
\pgfpathlineto{\pgfqpoint{4.366841in}{0.840936in}}%
\pgfpathlineto{\pgfqpoint{4.367694in}{0.840931in}}%
\pgfpathlineto{\pgfqpoint{4.368547in}{0.840927in}}%
\pgfpathlineto{\pgfqpoint{4.369400in}{0.840922in}}%
\pgfpathlineto{\pgfqpoint{4.370253in}{0.840918in}}%
\pgfpathlineto{\pgfqpoint{4.371106in}{0.840914in}}%
\pgfpathlineto{\pgfqpoint{4.371958in}{0.840909in}}%
\pgfpathlineto{\pgfqpoint{4.372811in}{0.840905in}}%
\pgfpathlineto{\pgfqpoint{4.373664in}{0.840900in}}%
\pgfpathlineto{\pgfqpoint{4.374517in}{0.840896in}}%
\pgfpathlineto{\pgfqpoint{4.375370in}{0.840891in}}%
\pgfpathlineto{\pgfqpoint{4.376223in}{0.840887in}}%
\pgfpathlineto{\pgfqpoint{4.377076in}{0.840883in}}%
\pgfpathlineto{\pgfqpoint{4.377929in}{0.840878in}}%
\pgfpathlineto{\pgfqpoint{4.378781in}{0.840874in}}%
\pgfpathlineto{\pgfqpoint{4.379634in}{0.840869in}}%
\pgfpathlineto{\pgfqpoint{4.380487in}{0.840865in}}%
\pgfpathlineto{\pgfqpoint{4.381340in}{0.840860in}}%
\pgfpathlineto{\pgfqpoint{4.382193in}{0.840856in}}%
\pgfpathlineto{\pgfqpoint{4.383046in}{0.840852in}}%
\pgfpathlineto{\pgfqpoint{4.383899in}{0.840829in}}%
\pgfpathlineto{\pgfqpoint{4.384752in}{0.840755in}}%
\pgfpathlineto{\pgfqpoint{4.385604in}{0.840682in}}%
\pgfpathlineto{\pgfqpoint{4.386457in}{0.840610in}}%
\pgfpathlineto{\pgfqpoint{4.387310in}{0.840537in}}%
\pgfpathlineto{\pgfqpoint{4.388163in}{0.840465in}}%
\pgfpathlineto{\pgfqpoint{4.389016in}{0.840323in}}%
\pgfpathlineto{\pgfqpoint{4.389869in}{0.840098in}}%
\pgfpathlineto{\pgfqpoint{4.390722in}{0.839873in}}%
\pgfpathlineto{\pgfqpoint{4.391575in}{0.839649in}}%
\pgfpathlineto{\pgfqpoint{4.392427in}{0.839424in}}%
\pgfpathlineto{\pgfqpoint{4.393280in}{0.839199in}}%
\pgfpathlineto{\pgfqpoint{4.394133in}{0.838974in}}%
\pgfpathlineto{\pgfqpoint{4.394986in}{0.838749in}}%
\pgfpathlineto{\pgfqpoint{4.395839in}{0.838524in}}%
\pgfpathlineto{\pgfqpoint{4.396692in}{0.838299in}}%
\pgfpathlineto{\pgfqpoint{4.397545in}{0.838074in}}%
\pgfpathlineto{\pgfqpoint{4.398398in}{0.837849in}}%
\pgfpathlineto{\pgfqpoint{4.399251in}{0.837625in}}%
\pgfpathlineto{\pgfqpoint{4.400103in}{0.837400in}}%
\pgfpathlineto{\pgfqpoint{4.400956in}{0.837175in}}%
\pgfpathlineto{\pgfqpoint{4.401809in}{0.836950in}}%
\pgfpathlineto{\pgfqpoint{4.402662in}{0.836725in}}%
\pgfpathlineto{\pgfqpoint{4.403515in}{0.836500in}}%
\pgfpathlineto{\pgfqpoint{4.404368in}{0.836260in}}%
\pgfpathlineto{\pgfqpoint{4.405221in}{0.835972in}}%
\pgfpathlineto{\pgfqpoint{4.406074in}{0.835785in}}%
\pgfpathlineto{\pgfqpoint{4.406926in}{0.837366in}}%
\pgfpathlineto{\pgfqpoint{4.407779in}{0.838618in}}%
\pgfpathlineto{\pgfqpoint{4.408632in}{0.838464in}}%
\pgfpathlineto{\pgfqpoint{4.409485in}{0.838308in}}%
\pgfpathlineto{\pgfqpoint{4.410338in}{0.838151in}}%
\pgfpathlineto{\pgfqpoint{4.411191in}{0.837994in}}%
\pgfpathlineto{\pgfqpoint{4.412044in}{0.837837in}}%
\pgfpathlineto{\pgfqpoint{4.412897in}{0.837681in}}%
\pgfpathlineto{\pgfqpoint{4.413749in}{0.837524in}}%
\pgfpathlineto{\pgfqpoint{4.414602in}{0.837367in}}%
\pgfpathlineto{\pgfqpoint{4.415455in}{0.837210in}}%
\pgfpathlineto{\pgfqpoint{4.416308in}{0.837053in}}%
\pgfpathlineto{\pgfqpoint{4.417161in}{0.836896in}}%
\pgfpathlineto{\pgfqpoint{4.418014in}{0.836740in}}%
\pgfpathlineto{\pgfqpoint{4.418867in}{0.836583in}}%
\pgfpathlineto{\pgfqpoint{4.419720in}{0.836426in}}%
\pgfpathlineto{\pgfqpoint{4.420572in}{0.836269in}}%
\pgfpathlineto{\pgfqpoint{4.421425in}{0.836112in}}%
\pgfpathlineto{\pgfqpoint{4.422278in}{0.835956in}}%
\pgfpathlineto{\pgfqpoint{4.423131in}{0.835799in}}%
\pgfpathlineto{\pgfqpoint{4.423984in}{0.835642in}}%
\pgfpathlineto{\pgfqpoint{4.424837in}{0.835485in}}%
\pgfpathlineto{\pgfqpoint{4.425690in}{0.835328in}}%
\pgfpathlineto{\pgfqpoint{4.426543in}{0.835172in}}%
\pgfpathlineto{\pgfqpoint{4.427396in}{0.835015in}}%
\pgfpathlineto{\pgfqpoint{4.428248in}{0.834858in}}%
\pgfpathlineto{\pgfqpoint{4.429101in}{0.834701in}}%
\pgfpathlineto{\pgfqpoint{4.429954in}{0.834544in}}%
\pgfpathlineto{\pgfqpoint{4.430807in}{0.834388in}}%
\pgfpathlineto{\pgfqpoint{4.431660in}{0.834235in}}%
\pgfpathlineto{\pgfqpoint{4.432513in}{0.834082in}}%
\pgfpathlineto{\pgfqpoint{4.433366in}{0.833929in}}%
\pgfpathlineto{\pgfqpoint{4.434219in}{0.833776in}}%
\pgfpathlineto{\pgfqpoint{4.435071in}{0.833623in}}%
\pgfpathlineto{\pgfqpoint{4.435924in}{0.833470in}}%
\pgfpathlineto{\pgfqpoint{4.436777in}{0.833317in}}%
\pgfpathlineto{\pgfqpoint{4.437630in}{0.833164in}}%
\pgfpathlineto{\pgfqpoint{4.438483in}{0.833011in}}%
\pgfpathlineto{\pgfqpoint{4.439336in}{0.832858in}}%
\pgfpathlineto{\pgfqpoint{4.440189in}{0.832705in}}%
\pgfpathlineto{\pgfqpoint{4.441042in}{0.832553in}}%
\pgfpathlineto{\pgfqpoint{4.441894in}{0.832400in}}%
\pgfpathlineto{\pgfqpoint{4.442747in}{0.832247in}}%
\pgfpathlineto{\pgfqpoint{4.443600in}{0.832094in}}%
\pgfpathlineto{\pgfqpoint{4.444453in}{0.831941in}}%
\pgfpathlineto{\pgfqpoint{4.445306in}{0.831788in}}%
\pgfpathlineto{\pgfqpoint{4.446159in}{0.831635in}}%
\pgfpathlineto{\pgfqpoint{4.447012in}{0.831482in}}%
\pgfpathlineto{\pgfqpoint{4.447865in}{0.831328in}}%
\pgfpathlineto{\pgfqpoint{4.448717in}{0.831176in}}%
\pgfpathlineto{\pgfqpoint{4.449570in}{0.831092in}}%
\pgfpathlineto{\pgfqpoint{4.450423in}{0.831024in}}%
\pgfpathlineto{\pgfqpoint{4.451276in}{0.830957in}}%
\pgfpathlineto{\pgfqpoint{4.452129in}{0.830890in}}%
\pgfpathlineto{\pgfqpoint{4.452982in}{0.830822in}}%
\pgfpathlineto{\pgfqpoint{4.453835in}{0.830755in}}%
\pgfpathlineto{\pgfqpoint{4.454688in}{0.830688in}}%
\pgfpathlineto{\pgfqpoint{4.455541in}{0.830621in}}%
\pgfpathlineto{\pgfqpoint{4.456393in}{0.830553in}}%
\pgfpathlineto{\pgfqpoint{4.457246in}{0.830486in}}%
\pgfpathlineto{\pgfqpoint{4.458099in}{0.830419in}}%
\pgfpathlineto{\pgfqpoint{4.458952in}{0.830351in}}%
\pgfpathlineto{\pgfqpoint{4.459805in}{0.830284in}}%
\pgfpathlineto{\pgfqpoint{4.460658in}{0.830217in}}%
\pgfpathlineto{\pgfqpoint{4.461511in}{0.830150in}}%
\pgfpathlineto{\pgfqpoint{4.462364in}{0.830082in}}%
\pgfpathlineto{\pgfqpoint{4.463216in}{0.830015in}}%
\pgfpathlineto{\pgfqpoint{4.464069in}{0.829948in}}%
\pgfpathlineto{\pgfqpoint{4.464922in}{0.829880in}}%
\pgfpathlineto{\pgfqpoint{4.465775in}{0.829813in}}%
\pgfpathlineto{\pgfqpoint{4.466628in}{0.829746in}}%
\pgfpathlineto{\pgfqpoint{4.467481in}{0.829681in}}%
\pgfpathlineto{\pgfqpoint{4.468334in}{0.829632in}}%
\pgfpathlineto{\pgfqpoint{4.469187in}{0.829590in}}%
\pgfpathlineto{\pgfqpoint{4.470039in}{0.829549in}}%
\pgfpathlineto{\pgfqpoint{4.470892in}{0.829507in}}%
\pgfpathlineto{\pgfqpoint{4.471745in}{0.829465in}}%
\pgfpathlineto{\pgfqpoint{4.472598in}{0.829424in}}%
\pgfpathlineto{\pgfqpoint{4.473451in}{0.829382in}}%
\pgfpathlineto{\pgfqpoint{4.474304in}{0.829340in}}%
\pgfpathlineto{\pgfqpoint{4.475157in}{0.829299in}}%
\pgfpathlineto{\pgfqpoint{4.476010in}{0.829257in}}%
\pgfpathlineto{\pgfqpoint{4.476862in}{0.829215in}}%
\pgfpathlineto{\pgfqpoint{4.477715in}{0.829174in}}%
\pgfpathlineto{\pgfqpoint{4.478568in}{0.829132in}}%
\pgfpathlineto{\pgfqpoint{4.479421in}{0.829090in}}%
\pgfpathlineto{\pgfqpoint{4.480274in}{0.829049in}}%
\pgfpathlineto{\pgfqpoint{4.481127in}{0.829007in}}%
\pgfpathlineto{\pgfqpoint{4.481980in}{0.828965in}}%
\pgfpathlineto{\pgfqpoint{4.482833in}{0.828924in}}%
\pgfpathlineto{\pgfqpoint{4.483685in}{0.828882in}}%
\pgfpathlineto{\pgfqpoint{4.484538in}{0.828840in}}%
\pgfpathlineto{\pgfqpoint{4.485391in}{0.828798in}}%
\pgfpathlineto{\pgfqpoint{4.486244in}{0.828757in}}%
\pgfpathlineto{\pgfqpoint{4.487097in}{0.828715in}}%
\pgfpathlineto{\pgfqpoint{4.487950in}{0.828673in}}%
\pgfpathlineto{\pgfqpoint{4.488803in}{0.828632in}}%
\pgfpathlineto{\pgfqpoint{4.489656in}{0.828615in}}%
\pgfpathlineto{\pgfqpoint{4.490509in}{0.828619in}}%
\pgfpathlineto{\pgfqpoint{4.491361in}{0.828622in}}%
\pgfpathlineto{\pgfqpoint{4.492214in}{0.828626in}}%
\pgfpathlineto{\pgfqpoint{4.493067in}{0.828630in}}%
\pgfpathlineto{\pgfqpoint{4.493920in}{0.828633in}}%
\pgfpathlineto{\pgfqpoint{4.494773in}{0.828637in}}%
\pgfpathlineto{\pgfqpoint{4.495626in}{0.828640in}}%
\pgfpathlineto{\pgfqpoint{4.496479in}{0.828644in}}%
\pgfpathlineto{\pgfqpoint{4.497332in}{0.828647in}}%
\pgfpathlineto{\pgfqpoint{4.498184in}{0.828651in}}%
\pgfpathlineto{\pgfqpoint{4.499037in}{0.828655in}}%
\pgfpathlineto{\pgfqpoint{4.499890in}{0.828658in}}%
\pgfpathlineto{\pgfqpoint{4.500743in}{0.828662in}}%
\pgfpathlineto{\pgfqpoint{4.501596in}{0.828665in}}%
\pgfpathlineto{\pgfqpoint{4.502449in}{0.828669in}}%
\pgfpathlineto{\pgfqpoint{4.503302in}{0.828672in}}%
\pgfpathlineto{\pgfqpoint{4.504155in}{0.828676in}}%
\pgfpathlineto{\pgfqpoint{4.505007in}{0.828679in}}%
\pgfpathlineto{\pgfqpoint{4.505860in}{0.828683in}}%
\pgfpathlineto{\pgfqpoint{4.506713in}{0.828687in}}%
\pgfpathlineto{\pgfqpoint{4.507566in}{0.828690in}}%
\pgfpathlineto{\pgfqpoint{4.508419in}{0.828694in}}%
\pgfpathlineto{\pgfqpoint{4.509272in}{0.828697in}}%
\pgfpathlineto{\pgfqpoint{4.510125in}{0.828701in}}%
\pgfpathlineto{\pgfqpoint{4.510978in}{0.828704in}}%
\pgfpathlineto{\pgfqpoint{4.511830in}{0.828708in}}%
\pgfpathlineto{\pgfqpoint{4.512683in}{0.828712in}}%
\pgfpathlineto{\pgfqpoint{4.513536in}{0.828715in}}%
\pgfpathlineto{\pgfqpoint{4.514389in}{0.828719in}}%
\pgfpathlineto{\pgfqpoint{4.515242in}{0.828722in}}%
\pgfpathlineto{\pgfqpoint{4.516095in}{0.828726in}}%
\pgfpathlineto{\pgfqpoint{4.516948in}{0.828729in}}%
\pgfpathlineto{\pgfqpoint{4.517801in}{0.828733in}}%
\pgfpathlineto{\pgfqpoint{4.518654in}{0.828736in}}%
\pgfpathlineto{\pgfqpoint{4.519506in}{0.828740in}}%
\pgfpathlineto{\pgfqpoint{4.520359in}{0.828744in}}%
\pgfpathlineto{\pgfqpoint{4.521212in}{0.828747in}}%
\pgfpathlineto{\pgfqpoint{4.522065in}{0.828751in}}%
\pgfpathlineto{\pgfqpoint{4.522918in}{0.828754in}}%
\pgfpathlineto{\pgfqpoint{4.523771in}{0.828758in}}%
\pgfpathlineto{\pgfqpoint{4.524624in}{0.828761in}}%
\pgfpathlineto{\pgfqpoint{4.525477in}{0.828765in}}%
\pgfpathlineto{\pgfqpoint{4.526329in}{0.828769in}}%
\pgfpathlineto{\pgfqpoint{4.527182in}{0.828772in}}%
\pgfpathlineto{\pgfqpoint{4.528035in}{0.828776in}}%
\pgfpathlineto{\pgfqpoint{4.528888in}{0.828778in}}%
\pgfpathlineto{\pgfqpoint{4.529741in}{0.828768in}}%
\pgfpathlineto{\pgfqpoint{4.530594in}{0.828756in}}%
\pgfpathlineto{\pgfqpoint{4.531447in}{0.828743in}}%
\pgfpathlineto{\pgfqpoint{4.532300in}{0.828730in}}%
\pgfpathlineto{\pgfqpoint{4.533152in}{0.828718in}}%
\pgfpathlineto{\pgfqpoint{4.534005in}{0.828705in}}%
\pgfpathlineto{\pgfqpoint{4.534858in}{0.828692in}}%
\pgfpathlineto{\pgfqpoint{4.535711in}{0.828679in}}%
\pgfpathlineto{\pgfqpoint{4.536564in}{0.828667in}}%
\pgfpathlineto{\pgfqpoint{4.537417in}{0.828654in}}%
\pgfpathlineto{\pgfqpoint{4.538270in}{0.828641in}}%
\pgfpathlineto{\pgfqpoint{4.539123in}{0.828629in}}%
\pgfpathlineto{\pgfqpoint{4.539975in}{0.828616in}}%
\pgfpathlineto{\pgfqpoint{4.540828in}{0.828603in}}%
\pgfpathlineto{\pgfqpoint{4.541681in}{0.828591in}}%
\pgfpathlineto{\pgfqpoint{4.542534in}{0.828578in}}%
\pgfpathlineto{\pgfqpoint{4.543387in}{0.828565in}}%
\pgfpathlineto{\pgfqpoint{4.544240in}{0.828552in}}%
\pgfpathlineto{\pgfqpoint{4.545093in}{0.828544in}}%
\pgfpathlineto{\pgfqpoint{4.545946in}{0.828592in}}%
\pgfpathlineto{\pgfqpoint{4.546799in}{0.828615in}}%
\pgfpathlineto{\pgfqpoint{4.547651in}{0.828425in}}%
\pgfpathlineto{\pgfqpoint{4.548504in}{0.828206in}}%
\pgfpathlineto{\pgfqpoint{4.549357in}{0.827986in}}%
\pgfpathlineto{\pgfqpoint{4.550210in}{0.827767in}}%
\pgfpathlineto{\pgfqpoint{4.551063in}{0.827547in}}%
\pgfpathlineto{\pgfqpoint{4.551916in}{0.827328in}}%
\pgfpathlineto{\pgfqpoint{4.552769in}{0.827109in}}%
\pgfpathlineto{\pgfqpoint{4.553622in}{0.827006in}}%
\pgfpathlineto{\pgfqpoint{4.554474in}{0.827050in}}%
\pgfpathlineto{\pgfqpoint{4.555327in}{0.827094in}}%
\pgfpathlineto{\pgfqpoint{4.556180in}{0.827138in}}%
\pgfpathlineto{\pgfqpoint{4.557033in}{0.827182in}}%
\pgfpathlineto{\pgfqpoint{4.557886in}{0.827226in}}%
\pgfpathlineto{\pgfqpoint{4.558739in}{0.827266in}}%
\pgfpathlineto{\pgfqpoint{4.559592in}{0.827265in}}%
\pgfpathlineto{\pgfqpoint{4.560445in}{0.827252in}}%
\pgfpathlineto{\pgfqpoint{4.561297in}{0.827240in}}%
\pgfpathlineto{\pgfqpoint{4.562150in}{0.827227in}}%
\pgfpathlineto{\pgfqpoint{4.563003in}{0.827215in}}%
\pgfpathlineto{\pgfqpoint{4.563856in}{0.827202in}}%
\pgfpathlineto{\pgfqpoint{4.564709in}{0.827190in}}%
\pgfpathlineto{\pgfqpoint{4.565562in}{0.827177in}}%
\pgfpathlineto{\pgfqpoint{4.566415in}{0.827165in}}%
\pgfpathlineto{\pgfqpoint{4.567268in}{0.827153in}}%
\pgfpathlineto{\pgfqpoint{4.568120in}{0.827140in}}%
\pgfpathlineto{\pgfqpoint{4.568973in}{0.827128in}}%
\pgfpathlineto{\pgfqpoint{4.569826in}{0.827115in}}%
\pgfpathlineto{\pgfqpoint{4.570679in}{0.827103in}}%
\pgfpathlineto{\pgfqpoint{4.571532in}{0.827097in}}%
\pgfpathlineto{\pgfqpoint{4.572385in}{0.827109in}}%
\pgfpathlineto{\pgfqpoint{4.573238in}{0.827122in}}%
\pgfpathlineto{\pgfqpoint{4.574091in}{0.827135in}}%
\pgfpathlineto{\pgfqpoint{4.574944in}{0.827148in}}%
\pgfpathlineto{\pgfqpoint{4.575796in}{0.827161in}}%
\pgfpathlineto{\pgfqpoint{4.576649in}{0.827174in}}%
\pgfpathlineto{\pgfqpoint{4.577502in}{0.827188in}}%
\pgfpathlineto{\pgfqpoint{4.578355in}{0.827201in}}%
\pgfpathlineto{\pgfqpoint{4.579208in}{0.827182in}}%
\pgfpathlineto{\pgfqpoint{4.580061in}{0.827039in}}%
\pgfpathlineto{\pgfqpoint{4.580914in}{0.826883in}}%
\pgfpathlineto{\pgfqpoint{4.581767in}{0.826727in}}%
\pgfpathlineto{\pgfqpoint{4.582619in}{0.826571in}}%
\pgfpathlineto{\pgfqpoint{4.583472in}{0.826415in}}%
\pgfpathlineto{\pgfqpoint{4.584325in}{0.826259in}}%
\pgfpathlineto{\pgfqpoint{4.585178in}{0.826103in}}%
\pgfpathlineto{\pgfqpoint{4.586031in}{0.825947in}}%
\pgfpathlineto{\pgfqpoint{4.586884in}{0.825791in}}%
\pgfpathlineto{\pgfqpoint{4.587737in}{0.825635in}}%
\pgfpathlineto{\pgfqpoint{4.588590in}{0.827143in}}%
\pgfpathlineto{\pgfqpoint{4.589442in}{0.827935in}}%
\pgfpathlineto{\pgfqpoint{4.590295in}{0.828750in}}%
\pgfpathlineto{\pgfqpoint{4.591148in}{0.828729in}}%
\pgfpathlineto{\pgfqpoint{4.592001in}{0.828707in}}%
\pgfpathlineto{\pgfqpoint{4.592854in}{0.828593in}}%
\pgfpathlineto{\pgfqpoint{4.593707in}{0.828404in}}%
\pgfpathlineto{\pgfqpoint{4.594560in}{0.828215in}}%
\pgfpathlineto{\pgfqpoint{4.595413in}{0.828026in}}%
\pgfpathlineto{\pgfqpoint{4.596265in}{0.827837in}}%
\pgfpathlineto{\pgfqpoint{4.597118in}{0.827648in}}%
\pgfpathlineto{\pgfqpoint{4.597971in}{0.827459in}}%
\pgfpathlineto{\pgfqpoint{4.598824in}{0.827270in}}%
\pgfpathlineto{\pgfqpoint{4.599677in}{0.827081in}}%
\pgfpathlineto{\pgfqpoint{4.600530in}{0.826904in}}%
\pgfpathlineto{\pgfqpoint{4.601383in}{0.826851in}}%
\pgfpathlineto{\pgfqpoint{4.602236in}{0.826829in}}%
\pgfpathlineto{\pgfqpoint{4.603088in}{0.826807in}}%
\pgfpathlineto{\pgfqpoint{4.603941in}{0.826785in}}%
\pgfpathlineto{\pgfqpoint{4.604794in}{0.826763in}}%
\pgfpathlineto{\pgfqpoint{4.605647in}{0.826741in}}%
\pgfpathlineto{\pgfqpoint{4.606500in}{0.826719in}}%
\pgfpathlineto{\pgfqpoint{4.607353in}{0.826697in}}%
\pgfpathlineto{\pgfqpoint{4.608206in}{0.826675in}}%
\pgfpathlineto{\pgfqpoint{4.609059in}{0.826653in}}%
\pgfpathlineto{\pgfqpoint{4.609912in}{0.826631in}}%
\pgfpathlineto{\pgfqpoint{4.610764in}{0.826609in}}%
\pgfpathlineto{\pgfqpoint{4.611617in}{0.826588in}}%
\pgfpathlineto{\pgfqpoint{4.612470in}{0.826618in}}%
\pgfpathlineto{\pgfqpoint{4.613323in}{0.826677in}}%
\pgfpathlineto{\pgfqpoint{4.614176in}{0.826736in}}%
\pgfpathlineto{\pgfqpoint{4.615029in}{0.826794in}}%
\pgfpathlineto{\pgfqpoint{4.615882in}{0.826853in}}%
\pgfpathlineto{\pgfqpoint{4.616735in}{0.826912in}}%
\pgfpathlineto{\pgfqpoint{4.617587in}{0.826970in}}%
\pgfpathlineto{\pgfqpoint{4.618440in}{0.827029in}}%
\pgfpathlineto{\pgfqpoint{4.619293in}{0.827088in}}%
\pgfpathlineto{\pgfqpoint{4.620146in}{0.827146in}}%
\pgfpathlineto{\pgfqpoint{4.620999in}{0.827205in}}%
\pgfpathlineto{\pgfqpoint{4.621852in}{0.827264in}}%
\pgfpathlineto{\pgfqpoint{4.622705in}{0.827322in}}%
\pgfpathlineto{\pgfqpoint{4.623558in}{0.827381in}}%
\pgfpathlineto{\pgfqpoint{4.624410in}{0.827440in}}%
\pgfpathlineto{\pgfqpoint{4.625263in}{0.827498in}}%
\pgfpathlineto{\pgfqpoint{4.626116in}{0.827557in}}%
\pgfpathlineto{\pgfqpoint{4.626969in}{0.827615in}}%
\pgfpathlineto{\pgfqpoint{4.627822in}{0.827674in}}%
\pgfpathlineto{\pgfqpoint{4.628675in}{0.827733in}}%
\pgfpathlineto{\pgfqpoint{4.629528in}{0.827791in}}%
\pgfpathlineto{\pgfqpoint{4.630381in}{0.827850in}}%
\pgfpathlineto{\pgfqpoint{4.631233in}{0.827909in}}%
\pgfpathlineto{\pgfqpoint{4.632086in}{0.827967in}}%
\pgfpathlineto{\pgfqpoint{4.632939in}{0.828020in}}%
\pgfpathlineto{\pgfqpoint{4.633792in}{0.827945in}}%
\pgfpathlineto{\pgfqpoint{4.634645in}{0.827819in}}%
\pgfpathlineto{\pgfqpoint{4.635498in}{0.827693in}}%
\pgfpathlineto{\pgfqpoint{4.636351in}{0.827568in}}%
\pgfpathlineto{\pgfqpoint{4.637204in}{0.827442in}}%
\pgfpathlineto{\pgfqpoint{4.638057in}{0.827316in}}%
\pgfpathlineto{\pgfqpoint{4.638909in}{0.827191in}}%
\pgfpathlineto{\pgfqpoint{4.639762in}{0.827065in}}%
\pgfpathlineto{\pgfqpoint{4.640615in}{0.826939in}}%
\pgfpathlineto{\pgfqpoint{4.641468in}{0.826813in}}%
\pgfpathlineto{\pgfqpoint{4.642321in}{0.826688in}}%
\pgfpathlineto{\pgfqpoint{4.643174in}{0.826562in}}%
\pgfpathlineto{\pgfqpoint{4.644027in}{0.826436in}}%
\pgfpathlineto{\pgfqpoint{4.644880in}{0.826310in}}%
\pgfpathlineto{\pgfqpoint{4.645732in}{0.826185in}}%
\pgfpathlineto{\pgfqpoint{4.646585in}{0.826059in}}%
\pgfpathlineto{\pgfqpoint{4.647438in}{0.825933in}}%
\pgfpathlineto{\pgfqpoint{4.648291in}{0.825807in}}%
\pgfpathlineto{\pgfqpoint{4.649144in}{0.825722in}}%
\pgfpathlineto{\pgfqpoint{4.649997in}{0.825762in}}%
\pgfpathlineto{\pgfqpoint{4.650850in}{0.825811in}}%
\pgfpathlineto{\pgfqpoint{4.651703in}{0.825860in}}%
\pgfpathlineto{\pgfqpoint{4.652555in}{0.825910in}}%
\pgfpathlineto{\pgfqpoint{4.653408in}{0.825959in}}%
\pgfpathlineto{\pgfqpoint{4.654261in}{0.826008in}}%
\pgfpathlineto{\pgfqpoint{4.655114in}{0.826058in}}%
\pgfpathlineto{\pgfqpoint{4.655967in}{0.826107in}}%
\pgfpathlineto{\pgfqpoint{4.656820in}{0.826156in}}%
\pgfpathlineto{\pgfqpoint{4.657673in}{0.826206in}}%
\pgfpathlineto{\pgfqpoint{4.658526in}{0.826255in}}%
\pgfpathlineto{\pgfqpoint{4.659378in}{0.826304in}}%
\pgfpathlineto{\pgfqpoint{4.660231in}{0.826354in}}%
\pgfpathlineto{\pgfqpoint{4.661084in}{0.826403in}}%
\pgfpathlineto{\pgfqpoint{4.661937in}{0.826452in}}%
\pgfpathlineto{\pgfqpoint{4.662790in}{0.826502in}}%
\pgfpathlineto{\pgfqpoint{4.663643in}{0.826551in}}%
\pgfpathlineto{\pgfqpoint{4.664496in}{0.826601in}}%
\pgfpathlineto{\pgfqpoint{4.665349in}{0.826650in}}%
\pgfpathlineto{\pgfqpoint{4.666202in}{0.826699in}}%
\pgfpathlineto{\pgfqpoint{4.667054in}{0.826749in}}%
\pgfpathlineto{\pgfqpoint{4.667907in}{0.826798in}}%
\pgfpathlineto{\pgfqpoint{4.668760in}{0.826847in}}%
\pgfpathlineto{\pgfqpoint{4.669613in}{0.826897in}}%
\pgfpathlineto{\pgfqpoint{4.670466in}{0.826946in}}%
\pgfpathlineto{\pgfqpoint{4.671319in}{0.826995in}}%
\pgfpathlineto{\pgfqpoint{4.672172in}{0.827045in}}%
\pgfpathlineto{\pgfqpoint{4.673025in}{0.827094in}}%
\pgfpathlineto{\pgfqpoint{4.673877in}{0.827143in}}%
\pgfpathlineto{\pgfqpoint{4.674730in}{0.827177in}}%
\pgfpathlineto{\pgfqpoint{4.675583in}{0.827146in}}%
\pgfpathlineto{\pgfqpoint{4.676436in}{0.827127in}}%
\pgfpathlineto{\pgfqpoint{4.677289in}{0.827164in}}%
\pgfpathlineto{\pgfqpoint{4.678142in}{0.827204in}}%
\pgfpathlineto{\pgfqpoint{4.678995in}{0.827197in}}%
\pgfpathlineto{\pgfqpoint{4.679848in}{0.827175in}}%
\pgfpathlineto{\pgfqpoint{4.680700in}{0.827154in}}%
\pgfpathlineto{\pgfqpoint{4.681553in}{0.827133in}}%
\pgfpathlineto{\pgfqpoint{4.682406in}{0.827111in}}%
\pgfpathlineto{\pgfqpoint{4.683259in}{0.827090in}}%
\pgfpathlineto{\pgfqpoint{4.684112in}{0.827069in}}%
\pgfpathlineto{\pgfqpoint{4.684965in}{0.827047in}}%
\pgfpathlineto{\pgfqpoint{4.685818in}{0.827026in}}%
\pgfpathlineto{\pgfqpoint{4.686671in}{0.827005in}}%
\pgfpathlineto{\pgfqpoint{4.687523in}{0.826980in}}%
\pgfpathlineto{\pgfqpoint{4.688376in}{0.826935in}}%
\pgfpathlineto{\pgfqpoint{4.689229in}{0.826885in}}%
\pgfpathlineto{\pgfqpoint{4.690082in}{0.826849in}}%
\pgfpathlineto{\pgfqpoint{4.690935in}{0.826848in}}%
\pgfpathlineto{\pgfqpoint{4.691788in}{0.826850in}}%
\pgfpathlineto{\pgfqpoint{4.692641in}{0.826852in}}%
\pgfpathlineto{\pgfqpoint{4.693494in}{0.826854in}}%
\pgfpathlineto{\pgfqpoint{4.694347in}{0.826827in}}%
\pgfpathlineto{\pgfqpoint{4.695199in}{0.826371in}}%
\pgfpathlineto{\pgfqpoint{4.696052in}{0.828860in}}%
\pgfpathlineto{\pgfqpoint{4.696905in}{0.828844in}}%
\pgfpathlineto{\pgfqpoint{4.697758in}{0.828827in}}%
\pgfpathlineto{\pgfqpoint{4.698611in}{0.828811in}}%
\pgfpathlineto{\pgfqpoint{4.699464in}{0.828795in}}%
\pgfpathlineto{\pgfqpoint{4.700317in}{0.828778in}}%
\pgfpathlineto{\pgfqpoint{4.701170in}{0.828762in}}%
\pgfpathlineto{\pgfqpoint{4.702022in}{0.828746in}}%
\pgfpathlineto{\pgfqpoint{4.702875in}{0.828730in}}%
\pgfpathlineto{\pgfqpoint{4.703728in}{0.828714in}}%
\pgfpathlineto{\pgfqpoint{4.704581in}{0.828698in}}%
\pgfpathlineto{\pgfqpoint{4.705434in}{0.828682in}}%
\pgfpathlineto{\pgfqpoint{4.706287in}{0.828666in}}%
\pgfpathlineto{\pgfqpoint{4.707140in}{0.828650in}}%
\pgfpathlineto{\pgfqpoint{4.707993in}{0.828633in}}%
\pgfpathlineto{\pgfqpoint{4.708845in}{0.828617in}}%
\pgfpathlineto{\pgfqpoint{4.709698in}{0.828601in}}%
\pgfpathlineto{\pgfqpoint{4.710551in}{0.828585in}}%
\pgfpathlineto{\pgfqpoint{4.711404in}{0.828446in}}%
\pgfpathlineto{\pgfqpoint{4.712257in}{0.828591in}}%
\pgfpathlineto{\pgfqpoint{4.713110in}{0.828573in}}%
\pgfpathlineto{\pgfqpoint{4.713963in}{0.828556in}}%
\pgfpathlineto{\pgfqpoint{4.714816in}{0.828070in}}%
\pgfpathlineto{\pgfqpoint{4.715668in}{0.826648in}}%
\pgfpathlineto{\pgfqpoint{4.716521in}{0.827147in}}%
\pgfpathlineto{\pgfqpoint{4.717374in}{0.827872in}}%
\pgfpathlineto{\pgfqpoint{4.718227in}{0.828436in}}%
\pgfpathlineto{\pgfqpoint{4.719080in}{0.828462in}}%
\pgfpathlineto{\pgfqpoint{4.719933in}{0.828444in}}%
\pgfpathlineto{\pgfqpoint{4.720786in}{0.828426in}}%
\pgfpathlineto{\pgfqpoint{4.721639in}{0.828408in}}%
\pgfpathlineto{\pgfqpoint{4.722491in}{0.828391in}}%
\pgfpathlineto{\pgfqpoint{4.723344in}{0.828373in}}%
\pgfpathlineto{\pgfqpoint{4.724197in}{0.828355in}}%
\pgfpathlineto{\pgfqpoint{4.725050in}{0.828337in}}%
\pgfpathlineto{\pgfqpoint{4.725903in}{0.828319in}}%
\pgfpathlineto{\pgfqpoint{4.726756in}{0.828301in}}%
\pgfpathlineto{\pgfqpoint{4.727609in}{0.828284in}}%
\pgfpathlineto{\pgfqpoint{4.728462in}{0.828266in}}%
\pgfpathlineto{\pgfqpoint{4.729315in}{0.828248in}}%
\pgfpathlineto{\pgfqpoint{4.730167in}{0.828230in}}%
\pgfpathlineto{\pgfqpoint{4.731020in}{0.828212in}}%
\pgfpathlineto{\pgfqpoint{4.731873in}{0.828194in}}%
\pgfpathlineto{\pgfqpoint{4.732726in}{0.828177in}}%
\pgfpathlineto{\pgfqpoint{4.733579in}{0.828159in}}%
\pgfpathlineto{\pgfqpoint{4.734432in}{0.828143in}}%
\pgfpathlineto{\pgfqpoint{4.735285in}{0.828139in}}%
\pgfpathlineto{\pgfqpoint{4.736138in}{0.828131in}}%
\pgfpathlineto{\pgfqpoint{4.736990in}{0.828122in}}%
\pgfpathlineto{\pgfqpoint{4.737843in}{0.828114in}}%
\pgfpathlineto{\pgfqpoint{4.738696in}{0.828105in}}%
\pgfpathlineto{\pgfqpoint{4.739549in}{0.828097in}}%
\pgfpathlineto{\pgfqpoint{4.740402in}{0.828088in}}%
\pgfpathlineto{\pgfqpoint{4.741255in}{0.828079in}}%
\pgfpathlineto{\pgfqpoint{4.742108in}{0.828071in}}%
\pgfpathlineto{\pgfqpoint{4.742961in}{0.828062in}}%
\pgfpathlineto{\pgfqpoint{4.743813in}{0.828054in}}%
\pgfpathlineto{\pgfqpoint{4.744666in}{0.828045in}}%
\pgfpathlineto{\pgfqpoint{4.745519in}{0.828037in}}%
\pgfpathlineto{\pgfqpoint{4.746372in}{0.828028in}}%
\pgfpathlineto{\pgfqpoint{4.747225in}{0.828020in}}%
\pgfpathlineto{\pgfqpoint{4.748078in}{0.828011in}}%
\pgfpathlineto{\pgfqpoint{4.748931in}{0.828002in}}%
\pgfpathlineto{\pgfqpoint{4.749784in}{0.827994in}}%
\pgfpathlineto{\pgfqpoint{4.750636in}{0.827985in}}%
\pgfpathlineto{\pgfqpoint{4.751489in}{0.827977in}}%
\pgfpathlineto{\pgfqpoint{4.752342in}{0.827968in}}%
\pgfpathlineto{\pgfqpoint{4.753195in}{0.827960in}}%
\pgfpathlineto{\pgfqpoint{4.754048in}{0.827951in}}%
\pgfpathlineto{\pgfqpoint{4.754901in}{0.827943in}}%
\pgfpathlineto{\pgfqpoint{4.755754in}{0.827934in}}%
\pgfpathlineto{\pgfqpoint{4.756607in}{0.827926in}}%
\pgfpathlineto{\pgfqpoint{4.757460in}{0.827917in}}%
\pgfpathlineto{\pgfqpoint{4.758312in}{0.827908in}}%
\pgfpathlineto{\pgfqpoint{4.759165in}{0.827900in}}%
\pgfpathlineto{\pgfqpoint{4.760018in}{0.827891in}}%
\pgfpathlineto{\pgfqpoint{4.760871in}{0.827883in}}%
\pgfpathlineto{\pgfqpoint{4.761724in}{0.827874in}}%
\pgfpathlineto{\pgfqpoint{4.762577in}{0.827866in}}%
\pgfpathlineto{\pgfqpoint{4.763430in}{0.827857in}}%
\pgfpathlineto{\pgfqpoint{4.764283in}{0.827849in}}%
\pgfpathlineto{\pgfqpoint{4.765135in}{0.827840in}}%
\pgfpathlineto{\pgfqpoint{4.765988in}{0.827831in}}%
\pgfpathlineto{\pgfqpoint{4.766841in}{0.827823in}}%
\pgfpathlineto{\pgfqpoint{4.767694in}{0.827814in}}%
\pgfpathlineto{\pgfqpoint{4.768547in}{0.827806in}}%
\pgfpathlineto{\pgfqpoint{4.769400in}{0.827797in}}%
\pgfpathlineto{\pgfqpoint{4.770253in}{0.827789in}}%
\pgfpathlineto{\pgfqpoint{4.771106in}{0.827780in}}%
\pgfpathlineto{\pgfqpoint{4.771958in}{0.827772in}}%
\pgfpathlineto{\pgfqpoint{4.772811in}{0.827763in}}%
\pgfpathlineto{\pgfqpoint{4.773664in}{0.827754in}}%
\pgfpathlineto{\pgfqpoint{4.774517in}{0.827746in}}%
\pgfpathlineto{\pgfqpoint{4.775370in}{0.827737in}}%
\pgfpathlineto{\pgfqpoint{4.776223in}{0.827729in}}%
\pgfpathlineto{\pgfqpoint{4.777076in}{0.827720in}}%
\pgfpathlineto{\pgfqpoint{4.777929in}{0.827712in}}%
\pgfpathlineto{\pgfqpoint{4.778781in}{0.827703in}}%
\pgfpathlineto{\pgfqpoint{4.779634in}{0.827695in}}%
\pgfpathlineto{\pgfqpoint{4.780487in}{0.827686in}}%
\pgfpathlineto{\pgfqpoint{4.781340in}{0.827677in}}%
\pgfpathlineto{\pgfqpoint{4.782193in}{0.827669in}}%
\pgfpathlineto{\pgfqpoint{4.783046in}{0.827660in}}%
\pgfpathlineto{\pgfqpoint{4.783899in}{0.827652in}}%
\pgfpathlineto{\pgfqpoint{4.784752in}{0.827643in}}%
\pgfpathlineto{\pgfqpoint{4.785605in}{0.827635in}}%
\pgfpathlineto{\pgfqpoint{4.786457in}{0.827626in}}%
\pgfpathlineto{\pgfqpoint{4.787310in}{0.827618in}}%
\pgfpathlineto{\pgfqpoint{4.788163in}{0.827609in}}%
\pgfpathlineto{\pgfqpoint{4.789016in}{0.827600in}}%
\pgfpathlineto{\pgfqpoint{4.789869in}{0.827592in}}%
\pgfpathlineto{\pgfqpoint{4.790722in}{0.827583in}}%
\pgfpathlineto{\pgfqpoint{4.791575in}{0.827575in}}%
\pgfpathlineto{\pgfqpoint{4.792428in}{0.827566in}}%
\pgfpathlineto{\pgfqpoint{4.793280in}{0.827558in}}%
\pgfpathlineto{\pgfqpoint{4.794133in}{0.827549in}}%
\pgfpathlineto{\pgfqpoint{4.794986in}{0.827541in}}%
\pgfpathlineto{\pgfqpoint{4.795839in}{0.827532in}}%
\pgfpathlineto{\pgfqpoint{4.796692in}{0.827523in}}%
\pgfpathlineto{\pgfqpoint{4.797545in}{0.827514in}}%
\pgfpathlineto{\pgfqpoint{4.798398in}{0.827496in}}%
\pgfpathlineto{\pgfqpoint{4.799251in}{0.827478in}}%
\pgfpathlineto{\pgfqpoint{4.800103in}{0.827459in}}%
\pgfpathlineto{\pgfqpoint{4.800956in}{0.827440in}}%
\pgfpathlineto{\pgfqpoint{4.801809in}{0.827421in}}%
\pgfpathlineto{\pgfqpoint{4.802662in}{0.827402in}}%
\pgfpathlineto{\pgfqpoint{4.803515in}{0.827383in}}%
\pgfpathlineto{\pgfqpoint{4.804368in}{0.827365in}}%
\pgfpathlineto{\pgfqpoint{4.805221in}{0.827346in}}%
\pgfpathlineto{\pgfqpoint{4.806074in}{0.827327in}}%
\pgfpathlineto{\pgfqpoint{4.806926in}{0.827308in}}%
\pgfpathlineto{\pgfqpoint{4.807779in}{0.827289in}}%
\pgfpathlineto{\pgfqpoint{4.808632in}{0.827271in}}%
\pgfpathlineto{\pgfqpoint{4.809485in}{0.827252in}}%
\pgfpathlineto{\pgfqpoint{4.810338in}{0.827233in}}%
\pgfpathlineto{\pgfqpoint{4.811191in}{0.827214in}}%
\pgfpathlineto{\pgfqpoint{4.812044in}{0.827195in}}%
\pgfpathlineto{\pgfqpoint{4.812897in}{0.827177in}}%
\pgfpathlineto{\pgfqpoint{4.813749in}{0.827158in}}%
\pgfpathlineto{\pgfqpoint{4.814602in}{0.827139in}}%
\pgfpathlineto{\pgfqpoint{4.815455in}{0.827125in}}%
\pgfpathlineto{\pgfqpoint{4.816308in}{0.827118in}}%
\pgfpathlineto{\pgfqpoint{4.817161in}{0.827112in}}%
\pgfpathlineto{\pgfqpoint{4.818014in}{0.827106in}}%
\pgfpathlineto{\pgfqpoint{4.818867in}{0.827100in}}%
\pgfpathlineto{\pgfqpoint{4.819720in}{0.827094in}}%
\pgfpathlineto{\pgfqpoint{4.820573in}{0.827087in}}%
\pgfpathlineto{\pgfqpoint{4.821425in}{0.827081in}}%
\pgfpathlineto{\pgfqpoint{4.822278in}{0.827075in}}%
\pgfpathlineto{\pgfqpoint{4.823131in}{0.827069in}}%
\pgfpathlineto{\pgfqpoint{4.823984in}{0.827063in}}%
\pgfpathlineto{\pgfqpoint{4.824837in}{0.827057in}}%
\pgfpathlineto{\pgfqpoint{4.825690in}{0.827051in}}%
\pgfpathlineto{\pgfqpoint{4.826543in}{0.827044in}}%
\pgfpathlineto{\pgfqpoint{4.827396in}{0.827038in}}%
\pgfpathlineto{\pgfqpoint{4.828248in}{0.827032in}}%
\pgfpathlineto{\pgfqpoint{4.829101in}{0.827026in}}%
\pgfpathlineto{\pgfqpoint{4.829954in}{0.827020in}}%
\pgfpathlineto{\pgfqpoint{4.830807in}{0.827014in}}%
\pgfpathlineto{\pgfqpoint{4.831660in}{0.827007in}}%
\pgfpathlineto{\pgfqpoint{4.832513in}{0.826801in}}%
\pgfpathlineto{\pgfqpoint{4.833366in}{0.826307in}}%
\pgfpathlineto{\pgfqpoint{4.834219in}{0.826150in}}%
\pgfpathlineto{\pgfqpoint{4.835071in}{0.826132in}}%
\pgfpathlineto{\pgfqpoint{4.835924in}{0.826115in}}%
\pgfpathlineto{\pgfqpoint{4.836777in}{0.826097in}}%
\pgfpathlineto{\pgfqpoint{4.837630in}{0.826080in}}%
\pgfpathlineto{\pgfqpoint{4.838483in}{0.826062in}}%
\pgfpathlineto{\pgfqpoint{4.839336in}{0.826045in}}%
\pgfpathlineto{\pgfqpoint{4.840189in}{0.826027in}}%
\pgfpathlineto{\pgfqpoint{4.841042in}{0.826010in}}%
\pgfpathlineto{\pgfqpoint{4.841894in}{0.825992in}}%
\pgfpathlineto{\pgfqpoint{4.842747in}{0.825974in}}%
\pgfpathlineto{\pgfqpoint{4.843600in}{0.825952in}}%
\pgfpathlineto{\pgfqpoint{4.844453in}{0.825929in}}%
\pgfpathlineto{\pgfqpoint{4.845306in}{0.825923in}}%
\pgfpathlineto{\pgfqpoint{4.846159in}{0.825925in}}%
\pgfpathlineto{\pgfqpoint{4.847012in}{0.825926in}}%
\pgfpathlineto{\pgfqpoint{4.847865in}{0.825928in}}%
\pgfpathlineto{\pgfqpoint{4.848718in}{0.825929in}}%
\pgfpathlineto{\pgfqpoint{4.849570in}{0.825931in}}%
\pgfpathlineto{\pgfqpoint{4.850423in}{0.825932in}}%
\pgfpathlineto{\pgfqpoint{4.851276in}{0.825934in}}%
\pgfpathlineto{\pgfqpoint{4.852129in}{0.825935in}}%
\pgfpathlineto{\pgfqpoint{4.852982in}{0.825937in}}%
\pgfpathlineto{\pgfqpoint{4.853835in}{0.825938in}}%
\pgfpathlineto{\pgfqpoint{4.854688in}{0.825940in}}%
\pgfpathlineto{\pgfqpoint{4.855541in}{0.825941in}}%
\pgfpathlineto{\pgfqpoint{4.856393in}{0.825943in}}%
\pgfpathlineto{\pgfqpoint{4.857246in}{0.825944in}}%
\pgfpathlineto{\pgfqpoint{4.858099in}{0.825946in}}%
\pgfpathlineto{\pgfqpoint{4.858952in}{0.825947in}}%
\pgfpathlineto{\pgfqpoint{4.859805in}{0.825949in}}%
\pgfpathlineto{\pgfqpoint{4.860658in}{0.825948in}}%
\pgfpathlineto{\pgfqpoint{4.861511in}{0.825938in}}%
\pgfpathlineto{\pgfqpoint{4.862364in}{0.825926in}}%
\pgfpathlineto{\pgfqpoint{4.863216in}{0.825915in}}%
\pgfpathlineto{\pgfqpoint{4.864069in}{0.825903in}}%
\pgfpathlineto{\pgfqpoint{4.864922in}{0.825895in}}%
\pgfpathlineto{\pgfqpoint{4.865775in}{0.825920in}}%
\pgfpathlineto{\pgfqpoint{4.866628in}{0.825952in}}%
\pgfpathlineto{\pgfqpoint{4.867481in}{0.825985in}}%
\pgfpathlineto{\pgfqpoint{4.868334in}{0.826017in}}%
\pgfpathlineto{\pgfqpoint{4.869187in}{0.826050in}}%
\pgfpathlineto{\pgfqpoint{4.870039in}{0.826082in}}%
\pgfpathlineto{\pgfqpoint{4.870892in}{0.826115in}}%
\pgfpathlineto{\pgfqpoint{4.871745in}{0.826147in}}%
\pgfpathlineto{\pgfqpoint{4.872598in}{0.825201in}}%
\pgfpathlineto{\pgfqpoint{4.873451in}{0.825162in}}%
\pgfpathlineto{\pgfqpoint{4.874304in}{0.825695in}}%
\pgfpathlineto{\pgfqpoint{4.875157in}{0.826074in}}%
\pgfpathlineto{\pgfqpoint{4.876010in}{0.826484in}}%
\pgfpathlineto{\pgfqpoint{4.876863in}{0.826572in}}%
\pgfpathlineto{\pgfqpoint{4.877715in}{0.826600in}}%
\pgfpathlineto{\pgfqpoint{4.878568in}{0.826629in}}%
\pgfpathlineto{\pgfqpoint{4.879421in}{0.826657in}}%
\pgfpathlineto{\pgfqpoint{4.880274in}{0.826799in}}%
\pgfpathlineto{\pgfqpoint{4.881127in}{0.826808in}}%
\pgfpathlineto{\pgfqpoint{4.881980in}{0.826773in}}%
\pgfpathlineto{\pgfqpoint{4.882833in}{0.826746in}}%
\pgfpathlineto{\pgfqpoint{4.883686in}{0.826725in}}%
\pgfpathlineto{\pgfqpoint{4.884538in}{0.826707in}}%
\pgfpathlineto{\pgfqpoint{4.885391in}{0.826688in}}%
\pgfpathlineto{\pgfqpoint{4.886244in}{0.826664in}}%
\pgfpathlineto{\pgfqpoint{4.887097in}{0.826579in}}%
\pgfpathlineto{\pgfqpoint{4.887950in}{0.826476in}}%
\pgfpathlineto{\pgfqpoint{4.888803in}{0.826373in}}%
\pgfpathlineto{\pgfqpoint{4.889656in}{0.826270in}}%
\pgfpathlineto{\pgfqpoint{4.890509in}{0.826167in}}%
\pgfpathlineto{\pgfqpoint{4.891361in}{0.826064in}}%
\pgfpathlineto{\pgfqpoint{4.892214in}{0.825961in}}%
\pgfpathlineto{\pgfqpoint{4.893067in}{0.825858in}}%
\pgfpathlineto{\pgfqpoint{4.893920in}{0.825755in}}%
\pgfpathlineto{\pgfqpoint{4.894773in}{0.825652in}}%
\pgfpathlineto{\pgfqpoint{4.895626in}{0.825523in}}%
\pgfpathlineto{\pgfqpoint{4.896479in}{0.825526in}}%
\pgfpathlineto{\pgfqpoint{4.897332in}{0.825666in}}%
\pgfpathlineto{\pgfqpoint{4.898184in}{0.825806in}}%
\pgfpathlineto{\pgfqpoint{4.899037in}{0.825945in}}%
\pgfpathlineto{\pgfqpoint{4.899890in}{0.826085in}}%
\pgfpathlineto{\pgfqpoint{4.900743in}{0.826245in}}%
\pgfpathlineto{\pgfqpoint{4.901596in}{0.826557in}}%
\pgfpathlineto{\pgfqpoint{4.902449in}{0.826521in}}%
\pgfpathlineto{\pgfqpoint{4.903302in}{0.826483in}}%
\pgfpathlineto{\pgfqpoint{4.904155in}{0.826445in}}%
\pgfpathlineto{\pgfqpoint{4.905008in}{0.826407in}}%
\pgfpathlineto{\pgfqpoint{4.905860in}{0.826369in}}%
\pgfpathlineto{\pgfqpoint{4.906713in}{0.826331in}}%
\pgfpathlineto{\pgfqpoint{4.907566in}{0.826293in}}%
\pgfpathlineto{\pgfqpoint{4.908419in}{0.826255in}}%
\pgfpathlineto{\pgfqpoint{4.909272in}{0.826217in}}%
\pgfpathlineto{\pgfqpoint{4.910125in}{0.826179in}}%
\pgfpathlineto{\pgfqpoint{4.910978in}{0.826141in}}%
\pgfpathlineto{\pgfqpoint{4.911831in}{0.826103in}}%
\pgfpathlineto{\pgfqpoint{4.912683in}{0.826065in}}%
\pgfpathlineto{\pgfqpoint{4.913536in}{0.826027in}}%
\pgfpathlineto{\pgfqpoint{4.914389in}{0.825989in}}%
\pgfpathlineto{\pgfqpoint{4.915242in}{0.825951in}}%
\pgfpathlineto{\pgfqpoint{4.916095in}{0.825913in}}%
\pgfpathlineto{\pgfqpoint{4.916948in}{0.825875in}}%
\pgfpathlineto{\pgfqpoint{4.917801in}{0.825837in}}%
\pgfpathlineto{\pgfqpoint{4.918654in}{0.825868in}}%
\pgfpathlineto{\pgfqpoint{4.919506in}{0.826071in}}%
\pgfpathlineto{\pgfqpoint{4.920359in}{0.826020in}}%
\pgfpathlineto{\pgfqpoint{4.921212in}{0.825969in}}%
\pgfpathlineto{\pgfqpoint{4.922065in}{0.825918in}}%
\pgfpathlineto{\pgfqpoint{4.922918in}{0.825868in}}%
\pgfpathlineto{\pgfqpoint{4.923771in}{0.825817in}}%
\pgfpathlineto{\pgfqpoint{4.924624in}{0.825766in}}%
\pgfpathlineto{\pgfqpoint{4.925477in}{0.825715in}}%
\pgfpathlineto{\pgfqpoint{4.926329in}{0.825664in}}%
\pgfpathlineto{\pgfqpoint{4.927182in}{0.825614in}}%
\pgfpathlineto{\pgfqpoint{4.928035in}{0.825563in}}%
\pgfpathlineto{\pgfqpoint{4.928888in}{0.825512in}}%
\pgfpathlineto{\pgfqpoint{4.929741in}{0.825461in}}%
\pgfpathlineto{\pgfqpoint{4.930594in}{0.825411in}}%
\pgfpathlineto{\pgfqpoint{4.931447in}{0.825360in}}%
\pgfpathlineto{\pgfqpoint{4.932300in}{0.825309in}}%
\pgfpathlineto{\pgfqpoint{4.933152in}{0.825258in}}%
\pgfpathlineto{\pgfqpoint{4.934005in}{0.825208in}}%
\pgfpathlineto{\pgfqpoint{4.934858in}{0.825157in}}%
\pgfpathlineto{\pgfqpoint{4.935711in}{0.825106in}}%
\pgfpathlineto{\pgfqpoint{4.936564in}{0.825055in}}%
\pgfpathlineto{\pgfqpoint{4.937417in}{0.825004in}}%
\pgfpathlineto{\pgfqpoint{4.938270in}{0.824954in}}%
\pgfpathlineto{\pgfqpoint{4.939123in}{0.824903in}}%
\pgfpathlineto{\pgfqpoint{4.939976in}{0.824852in}}%
\pgfpathlineto{\pgfqpoint{4.940828in}{0.824801in}}%
\pgfpathlineto{\pgfqpoint{4.941681in}{0.824751in}}%
\pgfpathlineto{\pgfqpoint{4.942534in}{0.824700in}}%
\pgfpathlineto{\pgfqpoint{4.943387in}{0.824649in}}%
\pgfpathlineto{\pgfqpoint{4.944240in}{0.824598in}}%
\pgfpathlineto{\pgfqpoint{4.945093in}{0.824547in}}%
\pgfpathlineto{\pgfqpoint{4.945946in}{0.824497in}}%
\pgfpathlineto{\pgfqpoint{4.946799in}{0.824446in}}%
\pgfpathlineto{\pgfqpoint{4.947651in}{0.824395in}}%
\pgfpathlineto{\pgfqpoint{4.948504in}{0.824344in}}%
\pgfpathlineto{\pgfqpoint{4.949357in}{0.824294in}}%
\pgfpathlineto{\pgfqpoint{4.950210in}{0.824243in}}%
\pgfpathlineto{\pgfqpoint{4.951063in}{0.824192in}}%
\pgfpathlineto{\pgfqpoint{4.951916in}{0.824141in}}%
\pgfpathlineto{\pgfqpoint{4.952769in}{0.824091in}}%
\pgfpathlineto{\pgfqpoint{4.953622in}{0.824040in}}%
\pgfpathlineto{\pgfqpoint{4.954474in}{0.823989in}}%
\pgfpathlineto{\pgfqpoint{4.955327in}{0.823938in}}%
\pgfpathlineto{\pgfqpoint{4.956180in}{0.823832in}}%
\pgfpathlineto{\pgfqpoint{4.957033in}{0.823782in}}%
\pgfpathlineto{\pgfqpoint{4.957886in}{0.823787in}}%
\pgfpathlineto{\pgfqpoint{4.958739in}{0.823788in}}%
\pgfpathlineto{\pgfqpoint{4.959592in}{0.823690in}}%
\pgfpathlineto{\pgfqpoint{4.960445in}{0.823716in}}%
\pgfpathlineto{\pgfqpoint{4.961297in}{0.823745in}}%
\pgfpathlineto{\pgfqpoint{4.962150in}{0.823773in}}%
\pgfpathlineto{\pgfqpoint{4.963003in}{0.823802in}}%
\pgfpathlineto{\pgfqpoint{4.963856in}{0.823830in}}%
\pgfpathlineto{\pgfqpoint{4.964709in}{0.823859in}}%
\pgfpathlineto{\pgfqpoint{4.965562in}{0.823887in}}%
\pgfpathlineto{\pgfqpoint{4.966415in}{0.823916in}}%
\pgfpathlineto{\pgfqpoint{4.967268in}{0.823944in}}%
\pgfpathlineto{\pgfqpoint{4.968121in}{0.823973in}}%
\pgfpathlineto{\pgfqpoint{4.968973in}{0.824001in}}%
\pgfpathlineto{\pgfqpoint{4.969826in}{0.824030in}}%
\pgfpathlineto{\pgfqpoint{4.970679in}{0.824058in}}%
\pgfpathlineto{\pgfqpoint{4.971532in}{0.824086in}}%
\pgfpathlineto{\pgfqpoint{4.972385in}{0.824115in}}%
\pgfpathlineto{\pgfqpoint{4.973238in}{0.824143in}}%
\pgfpathlineto{\pgfqpoint{4.974091in}{0.824172in}}%
\pgfpathlineto{\pgfqpoint{4.974944in}{0.824200in}}%
\pgfpathlineto{\pgfqpoint{4.975796in}{0.824229in}}%
\pgfpathlineto{\pgfqpoint{4.976649in}{0.824257in}}%
\pgfpathlineto{\pgfqpoint{4.977502in}{0.824286in}}%
\pgfpathlineto{\pgfqpoint{4.978355in}{0.824314in}}%
\pgfpathlineto{\pgfqpoint{4.979208in}{0.824343in}}%
\pgfpathlineto{\pgfqpoint{4.980061in}{0.824371in}}%
\pgfpathlineto{\pgfqpoint{4.980914in}{0.824400in}}%
\pgfpathlineto{\pgfqpoint{4.981767in}{0.824427in}}%
\pgfpathlineto{\pgfqpoint{4.982619in}{0.824420in}}%
\pgfpathlineto{\pgfqpoint{4.983472in}{0.824399in}}%
\pgfpathlineto{\pgfqpoint{4.984325in}{0.824378in}}%
\pgfpathlineto{\pgfqpoint{4.985178in}{0.824358in}}%
\pgfpathlineto{\pgfqpoint{4.986031in}{0.824337in}}%
\pgfpathlineto{\pgfqpoint{4.986884in}{0.824317in}}%
\pgfpathlineto{\pgfqpoint{4.987737in}{0.824296in}}%
\pgfpathlineto{\pgfqpoint{4.988590in}{0.824276in}}%
\pgfpathlineto{\pgfqpoint{4.989442in}{0.824255in}}%
\pgfpathlineto{\pgfqpoint{4.990295in}{0.824235in}}%
\pgfpathlineto{\pgfqpoint{4.991148in}{0.824214in}}%
\pgfpathlineto{\pgfqpoint{4.992001in}{0.824193in}}%
\pgfpathlineto{\pgfqpoint{4.992854in}{0.824173in}}%
\pgfpathlineto{\pgfqpoint{4.993707in}{0.824152in}}%
\pgfpathlineto{\pgfqpoint{4.994560in}{0.824132in}}%
\pgfpathlineto{\pgfqpoint{4.995413in}{0.824111in}}%
\pgfpathlineto{\pgfqpoint{4.996266in}{0.824088in}}%
\pgfpathlineto{\pgfqpoint{4.997118in}{0.824064in}}%
\pgfpathlineto{\pgfqpoint{4.997971in}{0.824040in}}%
\pgfpathlineto{\pgfqpoint{4.998824in}{0.824017in}}%
\pgfpathlineto{\pgfqpoint{4.999677in}{0.823993in}}%
\pgfpathlineto{\pgfqpoint{5.000530in}{0.823983in}}%
\pgfpathlineto{\pgfqpoint{5.001383in}{0.821232in}}%
\pgfpathlineto{\pgfqpoint{5.002236in}{0.818127in}}%
\pgfpathlineto{\pgfqpoint{5.003089in}{0.817897in}}%
\pgfpathlineto{\pgfqpoint{5.003941in}{0.817667in}}%
\pgfpathlineto{\pgfqpoint{5.004794in}{0.817437in}}%
\pgfpathlineto{\pgfqpoint{5.005647in}{0.817207in}}%
\pgfpathlineto{\pgfqpoint{5.006500in}{0.816978in}}%
\pgfpathlineto{\pgfqpoint{5.007353in}{0.816748in}}%
\pgfpathlineto{\pgfqpoint{5.008206in}{0.816518in}}%
\pgfpathlineto{\pgfqpoint{5.009059in}{0.816288in}}%
\pgfpathlineto{\pgfqpoint{5.009912in}{0.816058in}}%
\pgfpathlineto{\pgfqpoint{5.010764in}{0.815828in}}%
\pgfpathlineto{\pgfqpoint{5.011617in}{0.815598in}}%
\pgfpathlineto{\pgfqpoint{5.012470in}{0.815368in}}%
\pgfpathlineto{\pgfqpoint{5.013323in}{0.815139in}}%
\pgfpathlineto{\pgfqpoint{5.014176in}{0.814909in}}%
\pgfpathlineto{\pgfqpoint{5.015029in}{0.814679in}}%
\pgfpathlineto{\pgfqpoint{5.015882in}{0.814449in}}%
\pgfpathlineto{\pgfqpoint{5.016735in}{0.814219in}}%
\pgfpathlineto{\pgfqpoint{5.017587in}{0.813989in}}%
\pgfpathlineto{\pgfqpoint{5.018440in}{0.813599in}}%
\pgfpathlineto{\pgfqpoint{5.019293in}{0.811221in}}%
\pgfpathlineto{\pgfqpoint{5.020146in}{0.807531in}}%
\pgfpathlineto{\pgfqpoint{5.020999in}{0.801846in}}%
\pgfpathlineto{\pgfqpoint{5.021852in}{0.800543in}}%
\pgfpathlineto{\pgfqpoint{5.022705in}{0.800531in}}%
\pgfpathlineto{\pgfqpoint{5.023558in}{0.800518in}}%
\pgfpathlineto{\pgfqpoint{5.024411in}{0.800505in}}%
\pgfpathlineto{\pgfqpoint{5.025263in}{0.800493in}}%
\pgfpathlineto{\pgfqpoint{5.026116in}{0.800480in}}%
\pgfpathlineto{\pgfqpoint{5.026969in}{0.800467in}}%
\pgfpathlineto{\pgfqpoint{5.027822in}{0.800455in}}%
\pgfpathlineto{\pgfqpoint{5.028675in}{0.800442in}}%
\pgfpathlineto{\pgfqpoint{5.029528in}{0.800430in}}%
\pgfpathlineto{\pgfqpoint{5.030381in}{0.800417in}}%
\pgfpathlineto{\pgfqpoint{5.031234in}{0.800404in}}%
\pgfpathlineto{\pgfqpoint{5.032086in}{0.800392in}}%
\pgfpathlineto{\pgfqpoint{5.032939in}{0.800379in}}%
\pgfpathlineto{\pgfqpoint{5.033792in}{0.800367in}}%
\pgfpathlineto{\pgfqpoint{5.034645in}{0.800354in}}%
\pgfpathlineto{\pgfqpoint{5.035498in}{0.800341in}}%
\pgfpathlineto{\pgfqpoint{5.036351in}{0.800329in}}%
\pgfpathlineto{\pgfqpoint{5.037204in}{0.800316in}}%
\pgfpathlineto{\pgfqpoint{5.038057in}{0.800303in}}%
\pgfpathlineto{\pgfqpoint{5.038909in}{0.800291in}}%
\pgfpathlineto{\pgfqpoint{5.039762in}{0.800278in}}%
\pgfpathlineto{\pgfqpoint{5.040615in}{0.800266in}}%
\pgfpathlineto{\pgfqpoint{5.041468in}{0.800253in}}%
\pgfpathlineto{\pgfqpoint{5.042321in}{0.800240in}}%
\pgfpathlineto{\pgfqpoint{5.043174in}{0.800228in}}%
\pgfpathlineto{\pgfqpoint{5.044027in}{0.800215in}}%
\pgfpathlineto{\pgfqpoint{5.044880in}{0.800202in}}%
\pgfpathlineto{\pgfqpoint{5.045732in}{0.800190in}}%
\pgfpathlineto{\pgfqpoint{5.046585in}{0.800177in}}%
\pgfpathlineto{\pgfqpoint{5.047438in}{0.800165in}}%
\pgfpathlineto{\pgfqpoint{5.048291in}{0.800152in}}%
\pgfpathlineto{\pgfqpoint{5.049144in}{0.800139in}}%
\pgfpathlineto{\pgfqpoint{5.049997in}{0.800127in}}%
\pgfpathlineto{\pgfqpoint{5.050850in}{0.800114in}}%
\pgfpathlineto{\pgfqpoint{5.051703in}{0.800101in}}%
\pgfpathlineto{\pgfqpoint{5.052555in}{0.800089in}}%
\pgfpathlineto{\pgfqpoint{5.053408in}{0.800076in}}%
\pgfpathlineto{\pgfqpoint{5.054261in}{0.800064in}}%
\pgfpathlineto{\pgfqpoint{5.055114in}{0.800051in}}%
\pgfpathlineto{\pgfqpoint{5.055967in}{0.800038in}}%
\pgfpathlineto{\pgfqpoint{5.056820in}{0.800026in}}%
\pgfpathlineto{\pgfqpoint{5.057673in}{0.800013in}}%
\pgfpathlineto{\pgfqpoint{5.058526in}{0.800001in}}%
\pgfpathlineto{\pgfqpoint{5.059379in}{0.799988in}}%
\pgfpathlineto{\pgfqpoint{5.060231in}{0.799975in}}%
\pgfpathlineto{\pgfqpoint{5.061084in}{0.799963in}}%
\pgfpathlineto{\pgfqpoint{5.061937in}{0.799950in}}%
\pgfpathlineto{\pgfqpoint{5.062790in}{0.799937in}}%
\pgfpathlineto{\pgfqpoint{5.063643in}{0.799913in}}%
\pgfpathlineto{\pgfqpoint{5.064496in}{0.799878in}}%
\pgfpathlineto{\pgfqpoint{5.065349in}{0.799843in}}%
\pgfpathlineto{\pgfqpoint{5.066202in}{0.799808in}}%
\pgfpathlineto{\pgfqpoint{5.067054in}{0.799761in}}%
\pgfpathlineto{\pgfqpoint{5.067907in}{0.799662in}}%
\pgfpathlineto{\pgfqpoint{5.068760in}{0.799557in}}%
\pgfpathlineto{\pgfqpoint{5.069613in}{0.799452in}}%
\pgfpathlineto{\pgfqpoint{5.070466in}{0.799346in}}%
\pgfpathlineto{\pgfqpoint{5.071319in}{0.799241in}}%
\pgfpathlineto{\pgfqpoint{5.072172in}{0.799136in}}%
\pgfpathlineto{\pgfqpoint{5.073025in}{0.799030in}}%
\pgfpathlineto{\pgfqpoint{5.073877in}{0.798925in}}%
\pgfpathlineto{\pgfqpoint{5.074730in}{0.798820in}}%
\pgfpathlineto{\pgfqpoint{5.075583in}{0.798715in}}%
\pgfpathlineto{\pgfqpoint{5.076436in}{0.798609in}}%
\pgfpathlineto{\pgfqpoint{5.077289in}{0.798504in}}%
\pgfpathlineto{\pgfqpoint{5.078142in}{0.798399in}}%
\pgfpathlineto{\pgfqpoint{5.078995in}{0.798293in}}%
\pgfpathlineto{\pgfqpoint{5.079848in}{0.798188in}}%
\pgfpathlineto{\pgfqpoint{5.080700in}{0.798083in}}%
\pgfpathlineto{\pgfqpoint{5.081553in}{0.797978in}}%
\pgfpathlineto{\pgfqpoint{5.082406in}{0.797872in}}%
\pgfpathlineto{\pgfqpoint{5.083259in}{0.797767in}}%
\pgfpathlineto{\pgfqpoint{5.084112in}{0.797662in}}%
\pgfpathlineto{\pgfqpoint{5.084965in}{0.797556in}}%
\pgfpathlineto{\pgfqpoint{5.085818in}{0.797451in}}%
\pgfpathlineto{\pgfqpoint{5.086671in}{0.797346in}}%
\pgfpathlineto{\pgfqpoint{5.087524in}{0.797241in}}%
\pgfpathlineto{\pgfqpoint{5.088376in}{0.797135in}}%
\pgfpathlineto{\pgfqpoint{5.089229in}{0.797030in}}%
\pgfpathlineto{\pgfqpoint{5.090082in}{0.796925in}}%
\pgfpathlineto{\pgfqpoint{5.090935in}{0.796819in}}%
\pgfpathlineto{\pgfqpoint{5.091788in}{0.796714in}}%
\pgfpathlineto{\pgfqpoint{5.092641in}{0.796609in}}%
\pgfpathlineto{\pgfqpoint{5.093494in}{0.796504in}}%
\pgfpathlineto{\pgfqpoint{5.094347in}{0.796398in}}%
\pgfpathlineto{\pgfqpoint{5.095199in}{0.796293in}}%
\pgfpathlineto{\pgfqpoint{5.096052in}{0.796188in}}%
\pgfpathlineto{\pgfqpoint{5.096905in}{0.796082in}}%
\pgfpathlineto{\pgfqpoint{5.097758in}{0.795977in}}%
\pgfpathlineto{\pgfqpoint{5.098611in}{0.795872in}}%
\pgfpathlineto{\pgfqpoint{5.099464in}{0.795767in}}%
\pgfpathlineto{\pgfqpoint{5.100317in}{0.795661in}}%
\pgfpathlineto{\pgfqpoint{5.101170in}{0.795556in}}%
\pgfpathlineto{\pgfqpoint{5.102022in}{0.795451in}}%
\pgfpathlineto{\pgfqpoint{5.102875in}{0.795389in}}%
\pgfpathlineto{\pgfqpoint{5.103728in}{0.795456in}}%
\pgfpathlineto{\pgfqpoint{5.104581in}{0.795552in}}%
\pgfpathlineto{\pgfqpoint{5.105434in}{0.795648in}}%
\pgfpathlineto{\pgfqpoint{5.106287in}{0.795744in}}%
\pgfpathlineto{\pgfqpoint{5.107140in}{0.795840in}}%
\pgfpathlineto{\pgfqpoint{5.107993in}{0.795936in}}%
\pgfpathlineto{\pgfqpoint{5.108845in}{0.796032in}}%
\pgfpathlineto{\pgfqpoint{5.109698in}{0.796128in}}%
\pgfpathlineto{\pgfqpoint{5.110551in}{0.796224in}}%
\pgfpathlineto{\pgfqpoint{5.111404in}{0.796320in}}%
\pgfpathlineto{\pgfqpoint{5.112257in}{0.796416in}}%
\pgfpathlineto{\pgfqpoint{5.113110in}{0.796512in}}%
\pgfpathlineto{\pgfqpoint{5.113963in}{0.796608in}}%
\pgfpathlineto{\pgfqpoint{5.114816in}{0.796704in}}%
\pgfpathlineto{\pgfqpoint{5.115669in}{0.796800in}}%
\pgfpathlineto{\pgfqpoint{5.116521in}{0.796896in}}%
\pgfpathlineto{\pgfqpoint{5.117374in}{0.796992in}}%
\pgfpathlineto{\pgfqpoint{5.118227in}{0.797088in}}%
\pgfpathlineto{\pgfqpoint{5.119080in}{0.797184in}}%
\pgfpathlineto{\pgfqpoint{5.119933in}{0.797280in}}%
\pgfpathlineto{\pgfqpoint{5.120786in}{0.797376in}}%
\pgfpathlineto{\pgfqpoint{5.121639in}{0.797472in}}%
\pgfpathlineto{\pgfqpoint{5.122492in}{0.797483in}}%
\pgfpathlineto{\pgfqpoint{5.123344in}{0.797358in}}%
\pgfpathlineto{\pgfqpoint{5.124197in}{0.797211in}}%
\pgfpathlineto{\pgfqpoint{5.125050in}{0.797064in}}%
\pgfpathlineto{\pgfqpoint{5.125903in}{0.796917in}}%
\pgfpathlineto{\pgfqpoint{5.126756in}{0.796771in}}%
\pgfpathlineto{\pgfqpoint{5.127609in}{0.796624in}}%
\pgfpathlineto{\pgfqpoint{5.128462in}{0.796477in}}%
\pgfpathlineto{\pgfqpoint{5.129315in}{0.796330in}}%
\pgfpathlineto{\pgfqpoint{5.130167in}{0.796183in}}%
\pgfpathlineto{\pgfqpoint{5.131020in}{0.796036in}}%
\pgfpathlineto{\pgfqpoint{5.131873in}{0.795890in}}%
\pgfpathlineto{\pgfqpoint{5.132726in}{0.795743in}}%
\pgfpathlineto{\pgfqpoint{5.133579in}{0.795596in}}%
\pgfpathlineto{\pgfqpoint{5.134432in}{0.795449in}}%
\pgfpathlineto{\pgfqpoint{5.135285in}{0.795302in}}%
\pgfpathlineto{\pgfqpoint{5.136138in}{0.795155in}}%
\pgfpathlineto{\pgfqpoint{5.136990in}{0.795008in}}%
\pgfpathlineto{\pgfqpoint{5.137843in}{0.794862in}}%
\pgfpathlineto{\pgfqpoint{5.138696in}{0.794715in}}%
\pgfpathlineto{\pgfqpoint{5.139549in}{0.794568in}}%
\pgfpathlineto{\pgfqpoint{5.140402in}{0.794421in}}%
\pgfpathlineto{\pgfqpoint{5.141255in}{0.794274in}}%
\pgfpathlineto{\pgfqpoint{5.142108in}{0.794127in}}%
\pgfpathlineto{\pgfqpoint{5.142961in}{0.793981in}}%
\pgfpathlineto{\pgfqpoint{5.143813in}{0.793820in}}%
\pgfpathlineto{\pgfqpoint{5.144666in}{0.794509in}}%
\pgfpathlineto{\pgfqpoint{5.145519in}{0.796482in}}%
\pgfpathlineto{\pgfqpoint{5.146372in}{0.796352in}}%
\pgfpathlineto{\pgfqpoint{5.147225in}{0.796222in}}%
\pgfpathlineto{\pgfqpoint{5.148078in}{0.796092in}}%
\pgfpathlineto{\pgfqpoint{5.148931in}{0.795962in}}%
\pgfpathlineto{\pgfqpoint{5.149784in}{0.795832in}}%
\pgfpathlineto{\pgfqpoint{5.150637in}{0.795702in}}%
\pgfpathlineto{\pgfqpoint{5.151489in}{0.795572in}}%
\pgfpathlineto{\pgfqpoint{5.152342in}{0.795442in}}%
\pgfpathlineto{\pgfqpoint{5.153195in}{0.795312in}}%
\pgfpathlineto{\pgfqpoint{5.154048in}{0.795182in}}%
\pgfpathlineto{\pgfqpoint{5.154901in}{0.795052in}}%
\pgfpathlineto{\pgfqpoint{5.155754in}{0.794922in}}%
\pgfpathlineto{\pgfqpoint{5.156607in}{0.794793in}}%
\pgfpathlineto{\pgfqpoint{5.157460in}{0.794663in}}%
\pgfpathlineto{\pgfqpoint{5.158312in}{0.794533in}}%
\pgfpathlineto{\pgfqpoint{5.159165in}{0.794403in}}%
\pgfpathlineto{\pgfqpoint{5.160018in}{0.794273in}}%
\pgfpathlineto{\pgfqpoint{5.160871in}{0.794107in}}%
\pgfpathlineto{\pgfqpoint{5.161724in}{0.793848in}}%
\pgfpathlineto{\pgfqpoint{5.162577in}{0.793730in}}%
\pgfpathlineto{\pgfqpoint{5.163430in}{0.793664in}}%
\pgfpathlineto{\pgfqpoint{5.164283in}{0.793446in}}%
\pgfpathlineto{\pgfqpoint{5.165135in}{0.793111in}}%
\pgfpathlineto{\pgfqpoint{5.165988in}{0.792708in}}%
\pgfpathlineto{\pgfqpoint{5.166841in}{0.792304in}}%
\pgfpathlineto{\pgfqpoint{5.167694in}{0.791901in}}%
\pgfpathlineto{\pgfqpoint{5.168547in}{0.791497in}}%
\pgfpathlineto{\pgfqpoint{5.169400in}{0.791093in}}%
\pgfpathlineto{\pgfqpoint{5.170253in}{0.790690in}}%
\pgfpathlineto{\pgfqpoint{5.171106in}{0.790286in}}%
\pgfpathlineto{\pgfqpoint{5.171958in}{0.789883in}}%
\pgfpathlineto{\pgfqpoint{5.172811in}{0.789479in}}%
\pgfpathlineto{\pgfqpoint{5.173664in}{0.789076in}}%
\pgfpathlineto{\pgfqpoint{5.174517in}{0.788672in}}%
\pgfpathlineto{\pgfqpoint{5.175370in}{0.788269in}}%
\pgfpathlineto{\pgfqpoint{5.176223in}{0.787865in}}%
\pgfpathlineto{\pgfqpoint{5.177076in}{0.787461in}}%
\pgfpathlineto{\pgfqpoint{5.177929in}{0.787058in}}%
\pgfpathlineto{\pgfqpoint{5.178782in}{0.786654in}}%
\pgfpathlineto{\pgfqpoint{5.179634in}{0.786251in}}%
\pgfpathlineto{\pgfqpoint{5.180487in}{0.785847in}}%
\pgfpathlineto{\pgfqpoint{5.181340in}{0.784897in}}%
\pgfpathlineto{\pgfqpoint{5.182193in}{0.776530in}}%
\pgfpathlineto{\pgfqpoint{5.183046in}{0.775546in}}%
\pgfpathlineto{\pgfqpoint{5.183899in}{0.774563in}}%
\pgfpathlineto{\pgfqpoint{5.184752in}{0.773579in}}%
\pgfpathlineto{\pgfqpoint{5.185605in}{0.772596in}}%
\pgfpathlineto{\pgfqpoint{5.186457in}{0.771612in}}%
\pgfpathlineto{\pgfqpoint{5.187310in}{0.770629in}}%
\pgfpathlineto{\pgfqpoint{5.188163in}{0.769645in}}%
\pgfpathlineto{\pgfqpoint{5.189016in}{0.768924in}}%
\pgfpathlineto{\pgfqpoint{5.189869in}{0.768863in}}%
\pgfpathlineto{\pgfqpoint{5.190722in}{0.768833in}}%
\pgfpathlineto{\pgfqpoint{5.191575in}{0.768804in}}%
\pgfpathlineto{\pgfqpoint{5.192428in}{0.768774in}}%
\pgfpathlineto{\pgfqpoint{5.193280in}{0.768745in}}%
\pgfpathlineto{\pgfqpoint{5.194133in}{0.768715in}}%
\pgfpathlineto{\pgfqpoint{5.194986in}{0.768686in}}%
\pgfpathlineto{\pgfqpoint{5.195839in}{0.768656in}}%
\pgfpathlineto{\pgfqpoint{5.196692in}{0.768627in}}%
\pgfpathlineto{\pgfqpoint{5.197545in}{0.768598in}}%
\pgfpathlineto{\pgfqpoint{5.198398in}{0.768568in}}%
\pgfpathlineto{\pgfqpoint{5.199251in}{0.768539in}}%
\pgfpathlineto{\pgfqpoint{5.200103in}{0.768509in}}%
\pgfpathlineto{\pgfqpoint{5.200956in}{0.768480in}}%
\pgfpathlineto{\pgfqpoint{5.201809in}{0.768450in}}%
\pgfpathlineto{\pgfqpoint{5.202662in}{0.768421in}}%
\pgfpathlineto{\pgfqpoint{5.203515in}{0.768391in}}%
\pgfpathlineto{\pgfqpoint{5.204368in}{0.768362in}}%
\pgfpathlineto{\pgfqpoint{5.205221in}{0.768332in}}%
\pgfpathlineto{\pgfqpoint{5.206074in}{0.768303in}}%
\pgfpathlineto{\pgfqpoint{5.206927in}{0.768273in}}%
\pgfpathlineto{\pgfqpoint{5.207779in}{0.768244in}}%
\pgfpathlineto{\pgfqpoint{5.208632in}{0.768214in}}%
\pgfpathlineto{\pgfqpoint{5.209485in}{0.768185in}}%
\pgfpathlineto{\pgfqpoint{5.210338in}{0.768155in}}%
\pgfpathlineto{\pgfqpoint{5.211191in}{0.768126in}}%
\pgfpathlineto{\pgfqpoint{5.212044in}{0.768096in}}%
\pgfpathlineto{\pgfqpoint{5.212897in}{0.768067in}}%
\pgfpathlineto{\pgfqpoint{5.213750in}{0.768037in}}%
\pgfpathlineto{\pgfqpoint{5.214602in}{0.768008in}}%
\pgfpathlineto{\pgfqpoint{5.215455in}{0.767978in}}%
\pgfpathlineto{\pgfqpoint{5.216308in}{0.767949in}}%
\pgfpathlineto{\pgfqpoint{5.217161in}{0.767919in}}%
\pgfpathlineto{\pgfqpoint{5.218014in}{0.767890in}}%
\pgfpathlineto{\pgfqpoint{5.218867in}{0.767860in}}%
\pgfpathlineto{\pgfqpoint{5.219720in}{0.767831in}}%
\pgfpathlineto{\pgfqpoint{5.220573in}{0.767801in}}%
\pgfpathlineto{\pgfqpoint{5.221425in}{0.767772in}}%
\pgfpathlineto{\pgfqpoint{5.222278in}{0.767742in}}%
\pgfpathlineto{\pgfqpoint{5.223131in}{0.767713in}}%
\pgfpathlineto{\pgfqpoint{5.223984in}{0.767683in}}%
\pgfpathlineto{\pgfqpoint{5.224837in}{0.767654in}}%
\pgfpathlineto{\pgfqpoint{5.225690in}{0.767625in}}%
\pgfpathlineto{\pgfqpoint{5.226543in}{0.767595in}}%
\pgfpathlineto{\pgfqpoint{5.227396in}{0.767566in}}%
\pgfpathlineto{\pgfqpoint{5.228248in}{0.767536in}}%
\pgfpathlineto{\pgfqpoint{5.229101in}{0.767507in}}%
\pgfpathlineto{\pgfqpoint{5.229954in}{0.767477in}}%
\pgfpathlineto{\pgfqpoint{5.230807in}{0.767448in}}%
\pgfpathlineto{\pgfqpoint{5.231660in}{0.767418in}}%
\pgfpathlineto{\pgfqpoint{5.232513in}{0.767389in}}%
\pgfpathlineto{\pgfqpoint{5.233366in}{0.767359in}}%
\pgfpathlineto{\pgfqpoint{5.234219in}{0.767330in}}%
\pgfpathlineto{\pgfqpoint{5.235072in}{0.767300in}}%
\pgfpathlineto{\pgfqpoint{5.235924in}{0.767271in}}%
\pgfpathlineto{\pgfqpoint{5.236777in}{0.767241in}}%
\pgfpathlineto{\pgfqpoint{5.237630in}{0.767212in}}%
\pgfpathlineto{\pgfqpoint{5.238483in}{0.767182in}}%
\pgfpathlineto{\pgfqpoint{5.239336in}{0.767153in}}%
\pgfpathlineto{\pgfqpoint{5.240189in}{0.767123in}}%
\pgfpathlineto{\pgfqpoint{5.241042in}{0.767094in}}%
\pgfpathlineto{\pgfqpoint{5.241895in}{0.767064in}}%
\pgfpathlineto{\pgfqpoint{5.242747in}{0.767035in}}%
\pgfpathlineto{\pgfqpoint{5.243600in}{0.767005in}}%
\pgfpathlineto{\pgfqpoint{5.244453in}{0.766976in}}%
\pgfpathlineto{\pgfqpoint{5.245306in}{0.766946in}}%
\pgfpathlineto{\pgfqpoint{5.246159in}{0.766916in}}%
\pgfpathlineto{\pgfqpoint{5.247012in}{0.766758in}}%
\pgfpathlineto{\pgfqpoint{5.247865in}{0.766686in}}%
\pgfpathlineto{\pgfqpoint{5.248718in}{0.766654in}}%
\pgfpathlineto{\pgfqpoint{5.249570in}{0.766622in}}%
\pgfpathlineto{\pgfqpoint{5.250423in}{0.766589in}}%
\pgfpathlineto{\pgfqpoint{5.251276in}{0.766557in}}%
\pgfpathlineto{\pgfqpoint{5.252129in}{0.766525in}}%
\pgfpathlineto{\pgfqpoint{5.252982in}{0.766493in}}%
\pgfpathlineto{\pgfqpoint{5.253835in}{0.766461in}}%
\pgfpathlineto{\pgfqpoint{5.254688in}{0.766428in}}%
\pgfpathlineto{\pgfqpoint{5.255541in}{0.766396in}}%
\pgfpathlineto{\pgfqpoint{5.256393in}{0.766364in}}%
\pgfpathlineto{\pgfqpoint{5.257246in}{0.766332in}}%
\pgfpathlineto{\pgfqpoint{5.258099in}{0.766299in}}%
\pgfpathlineto{\pgfqpoint{5.258952in}{0.766267in}}%
\pgfpathlineto{\pgfqpoint{5.259805in}{0.766235in}}%
\pgfpathlineto{\pgfqpoint{5.260658in}{0.766203in}}%
\pgfpathlineto{\pgfqpoint{5.261511in}{0.766170in}}%
\pgfpathlineto{\pgfqpoint{5.262364in}{0.766123in}}%
\pgfpathlineto{\pgfqpoint{5.263216in}{0.766035in}}%
\pgfpathlineto{\pgfqpoint{5.264069in}{0.766125in}}%
\pgfpathlineto{\pgfqpoint{5.264922in}{0.766104in}}%
\pgfpathlineto{\pgfqpoint{5.265775in}{0.766083in}}%
\pgfpathlineto{\pgfqpoint{5.266628in}{0.766063in}}%
\pgfpathlineto{\pgfqpoint{5.267481in}{0.766032in}}%
\pgfpathlineto{\pgfqpoint{5.268334in}{0.766002in}}%
\pgfpathlineto{\pgfqpoint{5.269187in}{0.765974in}}%
\pgfpathlineto{\pgfqpoint{5.270040in}{0.765945in}}%
\pgfpathlineto{\pgfqpoint{5.270892in}{0.765917in}}%
\pgfpathlineto{\pgfqpoint{5.271745in}{0.765889in}}%
\pgfpathlineto{\pgfqpoint{5.272598in}{0.765861in}}%
\pgfpathlineto{\pgfqpoint{5.273451in}{0.765832in}}%
\pgfpathlineto{\pgfqpoint{5.274304in}{0.765804in}}%
\pgfpathlineto{\pgfqpoint{5.275157in}{0.765776in}}%
\pgfpathlineto{\pgfqpoint{5.276010in}{0.765748in}}%
\pgfpathlineto{\pgfqpoint{5.276863in}{0.765720in}}%
\pgfpathlineto{\pgfqpoint{5.277715in}{0.765691in}}%
\pgfpathlineto{\pgfqpoint{5.278568in}{0.765663in}}%
\pgfpathlineto{\pgfqpoint{5.279421in}{0.765635in}}%
\pgfpathlineto{\pgfqpoint{5.280274in}{0.765607in}}%
\pgfpathlineto{\pgfqpoint{5.281127in}{0.765579in}}%
\pgfpathlineto{\pgfqpoint{5.281980in}{0.765550in}}%
\pgfpathlineto{\pgfqpoint{5.282833in}{0.765522in}}%
\pgfpathlineto{\pgfqpoint{5.283686in}{0.765494in}}%
\pgfpathlineto{\pgfqpoint{5.284538in}{0.765466in}}%
\pgfpathlineto{\pgfqpoint{5.285391in}{0.765437in}}%
\pgfpathlineto{\pgfqpoint{5.286244in}{0.765409in}}%
\pgfpathlineto{\pgfqpoint{5.287097in}{0.765381in}}%
\pgfpathlineto{\pgfqpoint{5.287950in}{0.765353in}}%
\pgfpathlineto{\pgfqpoint{5.288803in}{0.765323in}}%
\pgfpathlineto{\pgfqpoint{5.289656in}{0.765292in}}%
\pgfpathlineto{\pgfqpoint{5.290509in}{0.765261in}}%
\pgfpathlineto{\pgfqpoint{5.291361in}{0.765230in}}%
\pgfpathlineto{\pgfqpoint{5.292214in}{0.765198in}}%
\pgfpathlineto{\pgfqpoint{5.293067in}{0.765167in}}%
\pgfpathlineto{\pgfqpoint{5.293920in}{0.765136in}}%
\pgfpathlineto{\pgfqpoint{5.294773in}{0.765105in}}%
\pgfpathlineto{\pgfqpoint{5.295626in}{0.765074in}}%
\pgfpathlineto{\pgfqpoint{5.296479in}{0.765043in}}%
\pgfpathlineto{\pgfqpoint{5.297332in}{0.765012in}}%
\pgfpathlineto{\pgfqpoint{5.298185in}{0.764981in}}%
\pgfpathlineto{\pgfqpoint{5.299037in}{0.764950in}}%
\pgfpathlineto{\pgfqpoint{5.299890in}{0.764919in}}%
\pgfpathlineto{\pgfqpoint{5.300743in}{0.764888in}}%
\pgfpathlineto{\pgfqpoint{5.301596in}{0.764857in}}%
\pgfpathlineto{\pgfqpoint{5.302449in}{0.764825in}}%
\pgfpathlineto{\pgfqpoint{5.303302in}{0.764794in}}%
\pgfpathlineto{\pgfqpoint{5.304155in}{0.764763in}}%
\pgfpathlineto{\pgfqpoint{5.305008in}{0.764732in}}%
\pgfpathlineto{\pgfqpoint{5.305860in}{0.764710in}}%
\pgfpathlineto{\pgfqpoint{5.306713in}{0.764695in}}%
\pgfpathlineto{\pgfqpoint{5.307566in}{0.764681in}}%
\pgfpathlineto{\pgfqpoint{5.308419in}{0.764667in}}%
\pgfpathlineto{\pgfqpoint{5.309272in}{0.764653in}}%
\pgfpathlineto{\pgfqpoint{5.310125in}{0.764639in}}%
\pgfpathlineto{\pgfqpoint{5.310978in}{0.764625in}}%
\pgfpathlineto{\pgfqpoint{5.311831in}{0.764610in}}%
\pgfpathlineto{\pgfqpoint{5.312683in}{0.764596in}}%
\pgfpathlineto{\pgfqpoint{5.313536in}{0.764582in}}%
\pgfpathlineto{\pgfqpoint{5.314389in}{0.764568in}}%
\pgfpathlineto{\pgfqpoint{5.315242in}{0.764554in}}%
\pgfpathlineto{\pgfqpoint{5.316095in}{0.764539in}}%
\pgfpathlineto{\pgfqpoint{5.316948in}{0.764525in}}%
\pgfpathlineto{\pgfqpoint{5.317801in}{0.764511in}}%
\pgfpathlineto{\pgfqpoint{5.318654in}{0.764497in}}%
\pgfpathlineto{\pgfqpoint{5.319506in}{0.764483in}}%
\pgfpathlineto{\pgfqpoint{5.320359in}{0.764469in}}%
\pgfpathlineto{\pgfqpoint{5.321212in}{0.764454in}}%
\pgfpathlineto{\pgfqpoint{5.322065in}{0.764440in}}%
\pgfpathlineto{\pgfqpoint{5.322918in}{0.764426in}}%
\pgfpathlineto{\pgfqpoint{5.323771in}{0.764412in}}%
\pgfpathlineto{\pgfqpoint{5.324624in}{0.764398in}}%
\pgfpathlineto{\pgfqpoint{5.325477in}{0.764383in}}%
\pgfpathlineto{\pgfqpoint{5.326330in}{0.764369in}}%
\pgfpathlineto{\pgfqpoint{5.327182in}{0.764361in}}%
\pgfpathlineto{\pgfqpoint{5.328035in}{0.764356in}}%
\pgfpathlineto{\pgfqpoint{5.328888in}{0.764340in}}%
\pgfpathlineto{\pgfqpoint{5.329741in}{0.764319in}}%
\pgfpathlineto{\pgfqpoint{5.330594in}{0.764298in}}%
\pgfpathlineto{\pgfqpoint{5.331447in}{0.764277in}}%
\pgfpathlineto{\pgfqpoint{5.332300in}{0.764255in}}%
\pgfpathlineto{\pgfqpoint{5.333153in}{0.764234in}}%
\pgfpathlineto{\pgfqpoint{5.334005in}{0.764213in}}%
\pgfpathlineto{\pgfqpoint{5.334858in}{0.764192in}}%
\pgfpathlineto{\pgfqpoint{5.335711in}{0.764171in}}%
\pgfpathlineto{\pgfqpoint{5.336564in}{0.764150in}}%
\pgfpathlineto{\pgfqpoint{5.337417in}{0.764129in}}%
\pgfpathlineto{\pgfqpoint{5.338270in}{0.764108in}}%
\pgfpathlineto{\pgfqpoint{5.339123in}{0.764086in}}%
\pgfpathlineto{\pgfqpoint{5.339976in}{0.764065in}}%
\pgfpathlineto{\pgfqpoint{5.340828in}{0.764044in}}%
\pgfpathlineto{\pgfqpoint{5.341681in}{0.764023in}}%
\pgfpathlineto{\pgfqpoint{5.342534in}{0.764002in}}%
\pgfpathlineto{\pgfqpoint{5.343387in}{0.763981in}}%
\pgfpathlineto{\pgfqpoint{5.344240in}{0.763960in}}%
\pgfpathlineto{\pgfqpoint{5.345093in}{0.763939in}}%
\pgfpathlineto{\pgfqpoint{5.345946in}{0.763918in}}%
\pgfpathlineto{\pgfqpoint{5.346799in}{0.763896in}}%
\pgfpathlineto{\pgfqpoint{5.347651in}{0.763875in}}%
\pgfpathlineto{\pgfqpoint{5.348504in}{0.763854in}}%
\pgfpathlineto{\pgfqpoint{5.349357in}{0.763833in}}%
\pgfpathlineto{\pgfqpoint{5.350210in}{0.763812in}}%
\pgfpathlineto{\pgfqpoint{5.351063in}{0.763791in}}%
\pgfpathlineto{\pgfqpoint{5.351916in}{0.763770in}}%
\pgfpathlineto{\pgfqpoint{5.352769in}{0.763749in}}%
\pgfpathlineto{\pgfqpoint{5.353622in}{0.763727in}}%
\pgfpathlineto{\pgfqpoint{5.354474in}{0.763706in}}%
\pgfpathlineto{\pgfqpoint{5.355327in}{0.763685in}}%
\pgfpathlineto{\pgfqpoint{5.356180in}{0.763664in}}%
\pgfpathlineto{\pgfqpoint{5.357033in}{0.763643in}}%
\pgfpathlineto{\pgfqpoint{5.357886in}{0.763622in}}%
\pgfpathlineto{\pgfqpoint{5.358739in}{0.763601in}}%
\pgfpathlineto{\pgfqpoint{5.359592in}{0.763580in}}%
\pgfpathlineto{\pgfqpoint{5.360445in}{0.763558in}}%
\pgfpathlineto{\pgfqpoint{5.361298in}{0.763537in}}%
\pgfpathlineto{\pgfqpoint{5.362150in}{0.763516in}}%
\pgfpathlineto{\pgfqpoint{5.363003in}{0.763495in}}%
\pgfpathlineto{\pgfqpoint{5.363856in}{0.763474in}}%
\pgfpathlineto{\pgfqpoint{5.364709in}{0.763453in}}%
\pgfpathlineto{\pgfqpoint{5.365562in}{0.763432in}}%
\pgfpathlineto{\pgfqpoint{5.366415in}{0.763411in}}%
\pgfpathlineto{\pgfqpoint{5.367268in}{0.763389in}}%
\pgfpathlineto{\pgfqpoint{5.368121in}{0.763368in}}%
\pgfpathlineto{\pgfqpoint{5.368973in}{0.763347in}}%
\pgfpathlineto{\pgfqpoint{5.369826in}{0.763326in}}%
\pgfpathlineto{\pgfqpoint{5.370679in}{0.763305in}}%
\pgfpathlineto{\pgfqpoint{5.371532in}{0.763284in}}%
\pgfpathlineto{\pgfqpoint{5.372385in}{0.763263in}}%
\pgfpathlineto{\pgfqpoint{5.373238in}{0.763242in}}%
\pgfpathlineto{\pgfqpoint{5.374091in}{0.763221in}}%
\pgfpathlineto{\pgfqpoint{5.374944in}{0.763199in}}%
\pgfpathlineto{\pgfqpoint{5.375796in}{0.763178in}}%
\pgfpathlineto{\pgfqpoint{5.376649in}{0.763157in}}%
\pgfpathlineto{\pgfqpoint{5.377502in}{0.763136in}}%
\pgfpathlineto{\pgfqpoint{5.378355in}{0.763115in}}%
\pgfpathlineto{\pgfqpoint{5.379208in}{0.763094in}}%
\pgfpathlineto{\pgfqpoint{5.380061in}{0.763073in}}%
\pgfpathlineto{\pgfqpoint{5.380914in}{0.763052in}}%
\pgfpathlineto{\pgfqpoint{5.381767in}{0.763030in}}%
\pgfpathlineto{\pgfqpoint{5.382619in}{0.763009in}}%
\pgfpathlineto{\pgfqpoint{5.383472in}{0.762988in}}%
\pgfpathlineto{\pgfqpoint{5.384325in}{0.762967in}}%
\pgfpathlineto{\pgfqpoint{5.385178in}{0.762946in}}%
\pgfpathlineto{\pgfqpoint{5.386031in}{0.762925in}}%
\pgfpathlineto{\pgfqpoint{5.386884in}{0.762896in}}%
\pgfpathlineto{\pgfqpoint{5.387737in}{0.762790in}}%
\pgfpathlineto{\pgfqpoint{5.388590in}{0.762701in}}%
\pgfpathlineto{\pgfqpoint{5.389443in}{0.762607in}}%
\pgfpathlineto{\pgfqpoint{5.390295in}{0.762564in}}%
\pgfpathlineto{\pgfqpoint{5.391148in}{0.762550in}}%
\pgfpathlineto{\pgfqpoint{5.392001in}{0.762537in}}%
\pgfpathlineto{\pgfqpoint{5.392854in}{0.762523in}}%
\pgfpathlineto{\pgfqpoint{5.393707in}{0.762510in}}%
\pgfpathlineto{\pgfqpoint{5.394560in}{0.762496in}}%
\pgfpathlineto{\pgfqpoint{5.395413in}{0.762483in}}%
\pgfpathlineto{\pgfqpoint{5.396266in}{0.762469in}}%
\pgfpathlineto{\pgfqpoint{5.397118in}{0.762455in}}%
\pgfpathlineto{\pgfqpoint{5.397971in}{0.762442in}}%
\pgfpathlineto{\pgfqpoint{5.398824in}{0.762428in}}%
\pgfpathlineto{\pgfqpoint{5.399677in}{0.762415in}}%
\pgfpathlineto{\pgfqpoint{5.400530in}{0.762401in}}%
\pgfpathlineto{\pgfqpoint{5.401383in}{0.762388in}}%
\pgfpathlineto{\pgfqpoint{5.402236in}{0.762374in}}%
\pgfpathlineto{\pgfqpoint{5.403089in}{0.762360in}}%
\pgfpathlineto{\pgfqpoint{5.403941in}{0.762347in}}%
\pgfpathlineto{\pgfqpoint{5.404794in}{0.762333in}}%
\pgfpathlineto{\pgfqpoint{5.405647in}{0.762320in}}%
\pgfpathlineto{\pgfqpoint{5.406500in}{0.762306in}}%
\pgfpathlineto{\pgfqpoint{5.407353in}{0.762293in}}%
\pgfpathlineto{\pgfqpoint{5.408206in}{0.762279in}}%
\pgfpathlineto{\pgfqpoint{5.409059in}{0.762265in}}%
\pgfpathlineto{\pgfqpoint{5.409912in}{0.762252in}}%
\pgfpathlineto{\pgfqpoint{5.410764in}{0.762238in}}%
\pgfpathlineto{\pgfqpoint{5.411617in}{0.762225in}}%
\pgfpathlineto{\pgfqpoint{5.412470in}{0.762211in}}%
\pgfpathlineto{\pgfqpoint{5.413323in}{0.762198in}}%
\pgfpathlineto{\pgfqpoint{5.414176in}{0.762184in}}%
\pgfpathlineto{\pgfqpoint{5.415029in}{0.762170in}}%
\pgfpathlineto{\pgfqpoint{5.415882in}{0.762157in}}%
\pgfpathlineto{\pgfqpoint{5.416735in}{0.762143in}}%
\pgfpathlineto{\pgfqpoint{5.417588in}{0.762130in}}%
\pgfpathlineto{\pgfqpoint{5.418440in}{0.762116in}}%
\pgfpathlineto{\pgfqpoint{5.419293in}{0.762103in}}%
\pgfpathlineto{\pgfqpoint{5.420146in}{0.762089in}}%
\pgfpathlineto{\pgfqpoint{5.420999in}{0.762075in}}%
\pgfpathlineto{\pgfqpoint{5.421852in}{0.762062in}}%
\pgfpathlineto{\pgfqpoint{5.422705in}{0.762048in}}%
\pgfpathlineto{\pgfqpoint{5.423558in}{0.762035in}}%
\pgfpathlineto{\pgfqpoint{5.424411in}{0.762021in}}%
\pgfpathlineto{\pgfqpoint{5.425263in}{0.762008in}}%
\pgfpathlineto{\pgfqpoint{5.426116in}{0.761994in}}%
\pgfpathlineto{\pgfqpoint{5.426969in}{0.761980in}}%
\pgfpathlineto{\pgfqpoint{5.427822in}{0.761885in}}%
\pgfpathlineto{\pgfqpoint{5.428675in}{0.761692in}}%
\pgfpathlineto{\pgfqpoint{5.429528in}{0.761500in}}%
\pgfpathlineto{\pgfqpoint{5.430381in}{0.761307in}}%
\pgfpathlineto{\pgfqpoint{5.431234in}{0.761045in}}%
\pgfpathlineto{\pgfqpoint{5.432086in}{0.760638in}}%
\pgfpathlineto{\pgfqpoint{5.432939in}{0.761067in}}%
\pgfpathlineto{\pgfqpoint{5.433792in}{0.761895in}}%
\pgfpathlineto{\pgfqpoint{5.434645in}{0.762691in}}%
\pgfpathlineto{\pgfqpoint{5.435498in}{0.762823in}}%
\pgfpathlineto{\pgfqpoint{5.436351in}{0.762691in}}%
\pgfpathlineto{\pgfqpoint{5.437204in}{0.762559in}}%
\pgfpathlineto{\pgfqpoint{5.438057in}{0.762426in}}%
\pgfpathlineto{\pgfqpoint{5.438909in}{0.762294in}}%
\pgfpathlineto{\pgfqpoint{5.439762in}{0.762161in}}%
\pgfpathlineto{\pgfqpoint{5.440615in}{0.762029in}}%
\pgfpathlineto{\pgfqpoint{5.441468in}{0.761897in}}%
\pgfpathlineto{\pgfqpoint{5.442321in}{0.761764in}}%
\pgfpathlineto{\pgfqpoint{5.443174in}{0.761632in}}%
\pgfpathlineto{\pgfqpoint{5.444027in}{0.761500in}}%
\pgfpathlineto{\pgfqpoint{5.444880in}{0.761367in}}%
\pgfpathlineto{\pgfqpoint{5.445733in}{0.761235in}}%
\pgfpathlineto{\pgfqpoint{5.446585in}{0.760987in}}%
\pgfpathlineto{\pgfqpoint{5.447438in}{0.760614in}}%
\pgfpathlineto{\pgfqpoint{5.448291in}{0.760511in}}%
\pgfpathlineto{\pgfqpoint{5.449144in}{0.760570in}}%
\pgfpathlineto{\pgfqpoint{5.449997in}{0.760516in}}%
\pgfpathlineto{\pgfqpoint{5.450850in}{0.760330in}}%
\pgfpathlineto{\pgfqpoint{5.451703in}{0.760397in}}%
\pgfpathlineto{\pgfqpoint{5.452556in}{0.760218in}}%
\pgfpathlineto{\pgfqpoint{5.453408in}{0.760050in}}%
\pgfpathlineto{\pgfqpoint{5.454261in}{0.759949in}}%
\pgfpathlineto{\pgfqpoint{5.455114in}{0.759857in}}%
\pgfpathlineto{\pgfqpoint{5.455967in}{0.759764in}}%
\pgfpathlineto{\pgfqpoint{5.456820in}{0.759672in}}%
\pgfpathlineto{\pgfqpoint{5.457673in}{0.759579in}}%
\pgfpathlineto{\pgfqpoint{5.458526in}{0.759487in}}%
\pgfpathlineto{\pgfqpoint{5.459379in}{0.759394in}}%
\pgfpathlineto{\pgfqpoint{5.460231in}{0.759302in}}%
\pgfpathlineto{\pgfqpoint{5.461084in}{0.759209in}}%
\pgfpathlineto{\pgfqpoint{5.461937in}{0.759117in}}%
\pgfpathlineto{\pgfqpoint{5.462790in}{0.759024in}}%
\pgfpathlineto{\pgfqpoint{5.463643in}{0.758932in}}%
\pgfpathlineto{\pgfqpoint{5.464496in}{0.758839in}}%
\pgfpathlineto{\pgfqpoint{5.465349in}{0.758747in}}%
\pgfpathlineto{\pgfqpoint{5.466202in}{0.758655in}}%
\pgfpathlineto{\pgfqpoint{5.467054in}{0.758562in}}%
\pgfpathlineto{\pgfqpoint{5.467907in}{0.758464in}}%
\pgfpathlineto{\pgfqpoint{5.468760in}{0.758315in}}%
\pgfpathlineto{\pgfqpoint{5.469613in}{0.758171in}}%
\pgfpathlineto{\pgfqpoint{5.470466in}{0.758084in}}%
\pgfpathlineto{\pgfqpoint{5.471319in}{0.758001in}}%
\pgfpathlineto{\pgfqpoint{5.472172in}{0.757922in}}%
\pgfpathlineto{\pgfqpoint{5.473025in}{0.757864in}}%
\pgfpathlineto{\pgfqpoint{5.473877in}{0.757542in}}%
\pgfpathlineto{\pgfqpoint{5.474730in}{0.757102in}}%
\pgfpathlineto{\pgfqpoint{5.475583in}{0.758978in}}%
\pgfpathlineto{\pgfqpoint{5.476436in}{0.760095in}}%
\pgfpathlineto{\pgfqpoint{5.477289in}{0.760045in}}%
\pgfpathlineto{\pgfqpoint{5.478142in}{0.759984in}}%
\pgfpathlineto{\pgfqpoint{5.478995in}{0.759923in}}%
\pgfpathlineto{\pgfqpoint{5.479848in}{0.759862in}}%
\pgfpathlineto{\pgfqpoint{5.480701in}{0.759801in}}%
\pgfpathlineto{\pgfqpoint{5.481553in}{0.759739in}}%
\pgfpathlineto{\pgfqpoint{5.482406in}{0.759678in}}%
\pgfpathlineto{\pgfqpoint{5.483259in}{0.759617in}}%
\pgfpathlineto{\pgfqpoint{5.484112in}{0.759556in}}%
\pgfpathlineto{\pgfqpoint{5.484965in}{0.759494in}}%
\pgfpathlineto{\pgfqpoint{5.485818in}{0.759433in}}%
\pgfpathlineto{\pgfqpoint{5.486671in}{0.759372in}}%
\pgfpathlineto{\pgfqpoint{5.487524in}{0.759311in}}%
\pgfpathlineto{\pgfqpoint{5.488376in}{0.759249in}}%
\pgfpathlineto{\pgfqpoint{5.489229in}{0.759188in}}%
\pgfpathlineto{\pgfqpoint{5.490082in}{0.759127in}}%
\pgfpathlineto{\pgfqpoint{5.490935in}{0.757825in}}%
\pgfpathlineto{\pgfqpoint{5.491788in}{0.757292in}}%
\pgfpathlineto{\pgfqpoint{5.492641in}{0.757229in}}%
\pgfpathlineto{\pgfqpoint{5.493494in}{0.757167in}}%
\pgfpathlineto{\pgfqpoint{5.494347in}{0.757104in}}%
\pgfpathlineto{\pgfqpoint{5.495199in}{0.757041in}}%
\pgfpathlineto{\pgfqpoint{5.496052in}{0.756979in}}%
\pgfpathlineto{\pgfqpoint{5.496905in}{0.756916in}}%
\pgfpathlineto{\pgfqpoint{5.497758in}{0.756853in}}%
\pgfpathlineto{\pgfqpoint{5.498611in}{0.756791in}}%
\pgfpathlineto{\pgfqpoint{5.499464in}{0.756728in}}%
\pgfpathlineto{\pgfqpoint{5.500317in}{0.756665in}}%
\pgfpathlineto{\pgfqpoint{5.501170in}{0.756603in}}%
\pgfpathlineto{\pgfqpoint{5.502022in}{0.756540in}}%
\pgfpathlineto{\pgfqpoint{5.502875in}{0.756477in}}%
\pgfpathlineto{\pgfqpoint{5.503728in}{0.756415in}}%
\pgfpathlineto{\pgfqpoint{5.504581in}{0.756352in}}%
\pgfpathlineto{\pgfqpoint{5.505434in}{0.756289in}}%
\pgfpathlineto{\pgfqpoint{5.506287in}{0.756227in}}%
\pgfpathlineto{\pgfqpoint{5.507140in}{0.756164in}}%
\pgfpathlineto{\pgfqpoint{5.507993in}{0.756101in}}%
\pgfpathlineto{\pgfqpoint{5.508846in}{0.756039in}}%
\pgfpathlineto{\pgfqpoint{5.509698in}{0.755976in}}%
\pgfpathlineto{\pgfqpoint{5.510551in}{0.755913in}}%
\pgfpathlineto{\pgfqpoint{5.511404in}{0.755851in}}%
\pgfpathlineto{\pgfqpoint{5.512257in}{0.755788in}}%
\pgfpathlineto{\pgfqpoint{5.513110in}{0.755725in}}%
\pgfpathlineto{\pgfqpoint{5.513963in}{0.755663in}}%
\pgfpathlineto{\pgfqpoint{5.514816in}{0.755600in}}%
\pgfpathlineto{\pgfqpoint{5.515669in}{0.755537in}}%
\pgfpathlineto{\pgfqpoint{5.516521in}{0.755475in}}%
\pgfpathlineto{\pgfqpoint{5.517374in}{0.755412in}}%
\pgfpathlineto{\pgfqpoint{5.518227in}{0.755349in}}%
\pgfpathlineto{\pgfqpoint{5.519080in}{0.755287in}}%
\pgfpathlineto{\pgfqpoint{5.519933in}{0.755224in}}%
\pgfpathlineto{\pgfqpoint{5.520786in}{0.755161in}}%
\pgfpathlineto{\pgfqpoint{5.521639in}{0.755099in}}%
\pgfpathlineto{\pgfqpoint{5.522492in}{0.755036in}}%
\pgfpathlineto{\pgfqpoint{5.523344in}{0.754973in}}%
\pgfpathlineto{\pgfqpoint{5.524197in}{0.754911in}}%
\pgfpathlineto{\pgfqpoint{5.525050in}{0.754848in}}%
\pgfpathlineto{\pgfqpoint{5.525903in}{0.754785in}}%
\pgfpathlineto{\pgfqpoint{5.526756in}{0.754722in}}%
\pgfpathlineto{\pgfqpoint{5.527609in}{0.754660in}}%
\pgfpathlineto{\pgfqpoint{5.528462in}{0.754597in}}%
\pgfpathlineto{\pgfqpoint{5.529315in}{0.754534in}}%
\pgfpathlineto{\pgfqpoint{5.530167in}{0.754472in}}%
\pgfpathlineto{\pgfqpoint{5.531020in}{0.754409in}}%
\pgfpathlineto{\pgfqpoint{5.531873in}{0.754346in}}%
\pgfpathlineto{\pgfqpoint{5.532726in}{0.754284in}}%
\pgfpathlineto{\pgfqpoint{5.533579in}{0.754221in}}%
\pgfpathlineto{\pgfqpoint{5.534432in}{0.754158in}}%
\pgfpathlineto{\pgfqpoint{5.535285in}{0.754096in}}%
\pgfpathlineto{\pgfqpoint{5.536138in}{0.754033in}}%
\pgfpathlineto{\pgfqpoint{5.536991in}{0.753970in}}%
\pgfpathlineto{\pgfqpoint{5.537843in}{0.753908in}}%
\pgfpathlineto{\pgfqpoint{5.538696in}{0.753845in}}%
\pgfpathlineto{\pgfqpoint{5.539549in}{0.753782in}}%
\pgfpathlineto{\pgfqpoint{5.540402in}{0.753720in}}%
\pgfpathlineto{\pgfqpoint{5.541255in}{0.753667in}}%
\pgfpathlineto{\pgfqpoint{5.542108in}{0.753621in}}%
\pgfpathlineto{\pgfqpoint{5.542961in}{0.753576in}}%
\pgfpathlineto{\pgfqpoint{5.543814in}{0.753531in}}%
\pgfpathlineto{\pgfqpoint{5.544666in}{0.753485in}}%
\pgfpathlineto{\pgfqpoint{5.545519in}{0.753440in}}%
\pgfpathlineto{\pgfqpoint{5.546372in}{0.753395in}}%
\pgfpathlineto{\pgfqpoint{5.547225in}{0.753349in}}%
\pgfpathlineto{\pgfqpoint{5.548078in}{0.753304in}}%
\pgfpathlineto{\pgfqpoint{5.548931in}{0.753259in}}%
\pgfpathlineto{\pgfqpoint{5.549784in}{0.753213in}}%
\pgfpathlineto{\pgfqpoint{5.550637in}{0.753168in}}%
\pgfpathlineto{\pgfqpoint{5.551489in}{0.753123in}}%
\pgfpathlineto{\pgfqpoint{5.552342in}{0.753080in}}%
\pgfpathlineto{\pgfqpoint{5.553195in}{0.753040in}}%
\pgfpathlineto{\pgfqpoint{5.554048in}{0.752997in}}%
\pgfpathlineto{\pgfqpoint{5.554901in}{0.752952in}}%
\pgfpathlineto{\pgfqpoint{5.555754in}{0.752906in}}%
\pgfpathlineto{\pgfqpoint{5.556607in}{0.752861in}}%
\pgfpathlineto{\pgfqpoint{5.557460in}{0.752816in}}%
\pgfpathlineto{\pgfqpoint{5.558312in}{0.752770in}}%
\pgfpathlineto{\pgfqpoint{5.559165in}{0.752725in}}%
\pgfpathlineto{\pgfqpoint{5.560018in}{0.752680in}}%
\pgfpathlineto{\pgfqpoint{5.560871in}{0.752634in}}%
\pgfpathlineto{\pgfqpoint{5.561724in}{0.752589in}}%
\pgfpathlineto{\pgfqpoint{5.562577in}{0.752544in}}%
\pgfpathlineto{\pgfqpoint{5.563430in}{0.752498in}}%
\pgfpathlineto{\pgfqpoint{5.564283in}{0.752453in}}%
\pgfpathlineto{\pgfqpoint{5.565136in}{0.752408in}}%
\pgfpathlineto{\pgfqpoint{5.565988in}{0.752362in}}%
\pgfpathlineto{\pgfqpoint{5.566841in}{0.752317in}}%
\pgfpathlineto{\pgfqpoint{5.567694in}{0.752271in}}%
\pgfpathlineto{\pgfqpoint{5.568547in}{0.752226in}}%
\pgfpathlineto{\pgfqpoint{5.569400in}{0.752181in}}%
\pgfpathlineto{\pgfqpoint{5.570253in}{0.752135in}}%
\pgfpathlineto{\pgfqpoint{5.571106in}{0.752090in}}%
\pgfpathlineto{\pgfqpoint{5.571959in}{0.752045in}}%
\pgfpathlineto{\pgfqpoint{5.572811in}{0.751999in}}%
\pgfpathlineto{\pgfqpoint{5.573664in}{0.751954in}}%
\pgfpathlineto{\pgfqpoint{5.574517in}{0.751909in}}%
\pgfpathlineto{\pgfqpoint{5.575370in}{0.751863in}}%
\pgfpathlineto{\pgfqpoint{5.576223in}{0.751818in}}%
\pgfpathlineto{\pgfqpoint{5.577076in}{0.751773in}}%
\pgfpathlineto{\pgfqpoint{5.577929in}{0.751727in}}%
\pgfpathlineto{\pgfqpoint{5.578782in}{0.751682in}}%
\pgfpathlineto{\pgfqpoint{5.579634in}{0.751637in}}%
\pgfpathlineto{\pgfqpoint{5.580487in}{0.751591in}}%
\pgfpathlineto{\pgfqpoint{5.581340in}{0.751546in}}%
\pgfpathlineto{\pgfqpoint{5.582193in}{0.751501in}}%
\pgfpathlineto{\pgfqpoint{5.583046in}{0.751455in}}%
\pgfpathlineto{\pgfqpoint{5.583899in}{0.751410in}}%
\pgfpathlineto{\pgfqpoint{5.584752in}{0.751365in}}%
\pgfpathlineto{\pgfqpoint{5.585605in}{0.751319in}}%
\pgfpathlineto{\pgfqpoint{5.586457in}{0.751274in}}%
\pgfpathlineto{\pgfqpoint{5.587310in}{0.751229in}}%
\pgfpathlineto{\pgfqpoint{5.588163in}{0.751183in}}%
\pgfpathlineto{\pgfqpoint{5.589016in}{0.751138in}}%
\pgfpathlineto{\pgfqpoint{5.589869in}{0.751093in}}%
\pgfpathlineto{\pgfqpoint{5.590722in}{0.751047in}}%
\pgfpathlineto{\pgfqpoint{5.591575in}{0.751002in}}%
\pgfpathlineto{\pgfqpoint{5.592428in}{0.750957in}}%
\pgfpathlineto{\pgfqpoint{5.593280in}{0.750911in}}%
\pgfpathlineto{\pgfqpoint{5.594133in}{0.750866in}}%
\pgfpathlineto{\pgfqpoint{5.594986in}{0.750821in}}%
\pgfpathlineto{\pgfqpoint{5.595839in}{0.750775in}}%
\pgfpathlineto{\pgfqpoint{5.596692in}{0.750730in}}%
\pgfpathlineto{\pgfqpoint{5.597545in}{0.750685in}}%
\pgfpathlineto{\pgfqpoint{5.598398in}{0.750639in}}%
\pgfpathlineto{\pgfqpoint{5.599251in}{0.750594in}}%
\pgfpathlineto{\pgfqpoint{5.600104in}{0.750510in}}%
\pgfpathlineto{\pgfqpoint{5.600956in}{0.750242in}}%
\pgfpathlineto{\pgfqpoint{5.601809in}{0.749949in}}%
\pgfpathlineto{\pgfqpoint{5.602662in}{0.749656in}}%
\pgfpathlineto{\pgfqpoint{5.603515in}{0.749363in}}%
\pgfpathlineto{\pgfqpoint{5.604368in}{0.749070in}}%
\pgfpathlineto{\pgfqpoint{5.605221in}{0.748778in}}%
\pgfpathlineto{\pgfqpoint{5.606074in}{0.748485in}}%
\pgfpathlineto{\pgfqpoint{5.606927in}{0.748192in}}%
\pgfpathlineto{\pgfqpoint{5.607779in}{0.747899in}}%
\pgfpathlineto{\pgfqpoint{5.608632in}{0.747606in}}%
\pgfpathlineto{\pgfqpoint{5.609485in}{0.747314in}}%
\pgfpathlineto{\pgfqpoint{5.610338in}{0.747021in}}%
\pgfpathlineto{\pgfqpoint{5.611191in}{0.746728in}}%
\pgfpathlineto{\pgfqpoint{5.612044in}{0.746435in}}%
\pgfpathlineto{\pgfqpoint{5.612897in}{0.746142in}}%
\pgfpathlineto{\pgfqpoint{5.613750in}{0.745850in}}%
\pgfpathlineto{\pgfqpoint{5.614602in}{0.745573in}}%
\pgfpathlineto{\pgfqpoint{5.615455in}{0.745500in}}%
\pgfpathlineto{\pgfqpoint{5.616308in}{0.745488in}}%
\pgfpathlineto{\pgfqpoint{5.617161in}{0.745476in}}%
\pgfpathlineto{\pgfqpoint{5.618014in}{0.745464in}}%
\pgfpathlineto{\pgfqpoint{5.618867in}{0.745452in}}%
\pgfpathlineto{\pgfqpoint{5.619720in}{0.745440in}}%
\pgfpathlineto{\pgfqpoint{5.620573in}{0.745427in}}%
\pgfpathlineto{\pgfqpoint{5.621425in}{0.745415in}}%
\pgfpathlineto{\pgfqpoint{5.622278in}{0.745403in}}%
\pgfpathlineto{\pgfqpoint{5.623131in}{0.745391in}}%
\pgfpathlineto{\pgfqpoint{5.623984in}{0.745379in}}%
\pgfpathlineto{\pgfqpoint{5.624837in}{0.745367in}}%
\pgfpathlineto{\pgfqpoint{5.625690in}{0.745355in}}%
\pgfpathlineto{\pgfqpoint{5.626543in}{0.745342in}}%
\pgfpathlineto{\pgfqpoint{5.627396in}{0.745330in}}%
\pgfpathlineto{\pgfqpoint{5.628249in}{0.745318in}}%
\pgfpathlineto{\pgfqpoint{5.629101in}{0.745306in}}%
\pgfpathlineto{\pgfqpoint{5.629954in}{0.745294in}}%
\pgfpathlineto{\pgfqpoint{5.630807in}{0.745282in}}%
\pgfpathlineto{\pgfqpoint{5.631660in}{0.745270in}}%
\pgfpathlineto{\pgfqpoint{5.632513in}{0.745257in}}%
\pgfpathlineto{\pgfqpoint{5.633366in}{0.745245in}}%
\pgfpathlineto{\pgfqpoint{5.634219in}{0.745233in}}%
\pgfpathlineto{\pgfqpoint{5.635072in}{0.745221in}}%
\pgfpathlineto{\pgfqpoint{5.635924in}{0.745209in}}%
\pgfpathlineto{\pgfqpoint{5.636777in}{0.745219in}}%
\pgfpathlineto{\pgfqpoint{5.637630in}{0.745359in}}%
\pgfpathlineto{\pgfqpoint{5.638483in}{0.745354in}}%
\pgfpathlineto{\pgfqpoint{5.639336in}{0.745350in}}%
\pgfpathlineto{\pgfqpoint{5.640189in}{0.745345in}}%
\pgfpathlineto{\pgfqpoint{5.641042in}{0.745341in}}%
\pgfpathlineto{\pgfqpoint{5.641895in}{0.745336in}}%
\pgfpathlineto{\pgfqpoint{5.642747in}{0.745332in}}%
\pgfpathlineto{\pgfqpoint{5.643600in}{0.745327in}}%
\pgfpathlineto{\pgfqpoint{5.644453in}{0.745322in}}%
\pgfpathlineto{\pgfqpoint{5.645306in}{0.745318in}}%
\pgfpathlineto{\pgfqpoint{5.646159in}{0.745313in}}%
\pgfpathlineto{\pgfqpoint{5.647012in}{0.745309in}}%
\pgfpathlineto{\pgfqpoint{5.647865in}{0.745304in}}%
\pgfpathlineto{\pgfqpoint{5.648718in}{0.745300in}}%
\pgfpathlineto{\pgfqpoint{5.649570in}{0.745295in}}%
\pgfpathlineto{\pgfqpoint{5.650423in}{0.745290in}}%
\pgfpathlineto{\pgfqpoint{5.651276in}{0.745286in}}%
\pgfpathlineto{\pgfqpoint{5.652129in}{0.745281in}}%
\pgfpathlineto{\pgfqpoint{5.652982in}{0.745277in}}%
\pgfpathlineto{\pgfqpoint{5.653835in}{0.745272in}}%
\pgfpathlineto{\pgfqpoint{5.654688in}{0.745268in}}%
\pgfpathlineto{\pgfqpoint{5.655541in}{0.745263in}}%
\pgfpathlineto{\pgfqpoint{5.656394in}{0.745258in}}%
\pgfpathlineto{\pgfqpoint{5.657246in}{0.745254in}}%
\pgfpathlineto{\pgfqpoint{5.658099in}{0.745249in}}%
\pgfpathlineto{\pgfqpoint{5.658952in}{0.745245in}}%
\pgfpathlineto{\pgfqpoint{5.659805in}{0.745240in}}%
\pgfpathlineto{\pgfqpoint{5.660658in}{0.745235in}}%
\pgfpathlineto{\pgfqpoint{5.661511in}{0.745231in}}%
\pgfpathlineto{\pgfqpoint{5.662364in}{0.745226in}}%
\pgfpathlineto{\pgfqpoint{5.663217in}{0.745222in}}%
\pgfpathlineto{\pgfqpoint{5.664069in}{0.745217in}}%
\pgfpathlineto{\pgfqpoint{5.664922in}{0.745213in}}%
\pgfpathlineto{\pgfqpoint{5.665775in}{0.745208in}}%
\pgfpathlineto{\pgfqpoint{5.666628in}{0.745203in}}%
\pgfpathlineto{\pgfqpoint{5.667481in}{0.745199in}}%
\pgfpathlineto{\pgfqpoint{5.668334in}{0.745194in}}%
\pgfpathlineto{\pgfqpoint{5.669187in}{0.745190in}}%
\pgfpathlineto{\pgfqpoint{5.670040in}{0.745185in}}%
\pgfpathlineto{\pgfqpoint{5.670892in}{0.745181in}}%
\pgfpathlineto{\pgfqpoint{5.671745in}{0.745176in}}%
\pgfpathlineto{\pgfqpoint{5.672598in}{0.745171in}}%
\pgfpathlineto{\pgfqpoint{5.673451in}{0.745167in}}%
\pgfpathlineto{\pgfqpoint{5.674304in}{0.745162in}}%
\pgfpathlineto{\pgfqpoint{5.675157in}{0.745158in}}%
\pgfpathlineto{\pgfqpoint{5.676010in}{0.745153in}}%
\pgfpathlineto{\pgfqpoint{5.676863in}{0.745149in}}%
\pgfpathlineto{\pgfqpoint{5.677715in}{0.745144in}}%
\pgfpathlineto{\pgfqpoint{5.678568in}{0.745139in}}%
\pgfpathlineto{\pgfqpoint{5.679421in}{0.745135in}}%
\pgfpathlineto{\pgfqpoint{5.680274in}{0.745130in}}%
\pgfpathlineto{\pgfqpoint{5.681127in}{0.745126in}}%
\pgfpathlineto{\pgfqpoint{5.681980in}{0.745121in}}%
\pgfpathlineto{\pgfqpoint{5.682833in}{0.745117in}}%
\pgfpathlineto{\pgfqpoint{5.683686in}{0.745112in}}%
\pgfpathlineto{\pgfqpoint{5.684538in}{0.745107in}}%
\pgfpathlineto{\pgfqpoint{5.685391in}{0.745103in}}%
\pgfpathlineto{\pgfqpoint{5.686244in}{0.745098in}}%
\pgfpathlineto{\pgfqpoint{5.687097in}{0.745094in}}%
\pgfpathlineto{\pgfqpoint{5.687950in}{0.745089in}}%
\pgfpathlineto{\pgfqpoint{5.688803in}{0.745084in}}%
\pgfpathlineto{\pgfqpoint{5.689656in}{0.745080in}}%
\pgfpathlineto{\pgfqpoint{5.690509in}{0.745075in}}%
\pgfpathlineto{\pgfqpoint{5.691362in}{0.745111in}}%
\pgfpathlineto{\pgfqpoint{5.692214in}{0.745195in}}%
\pgfpathlineto{\pgfqpoint{5.693067in}{0.745279in}}%
\pgfpathlineto{\pgfqpoint{5.693920in}{0.745364in}}%
\pgfpathlineto{\pgfqpoint{5.694773in}{0.745448in}}%
\pgfpathlineto{\pgfqpoint{5.695626in}{0.745607in}}%
\pgfpathlineto{\pgfqpoint{5.696479in}{0.745830in}}%
\pgfpathlineto{\pgfqpoint{5.697332in}{0.746047in}}%
\pgfpathlineto{\pgfqpoint{5.698185in}{0.745761in}}%
\pgfpathlineto{\pgfqpoint{5.699037in}{0.745397in}}%
\pgfpathlineto{\pgfqpoint{5.699890in}{0.746230in}}%
\pgfpathlineto{\pgfqpoint{5.700743in}{0.746163in}}%
\pgfpathlineto{\pgfqpoint{5.701596in}{0.746097in}}%
\pgfpathlineto{\pgfqpoint{5.702449in}{0.746030in}}%
\pgfpathlineto{\pgfqpoint{5.703302in}{0.745963in}}%
\pgfpathlineto{\pgfqpoint{5.704155in}{0.745896in}}%
\pgfpathlineto{\pgfqpoint{5.705008in}{0.745830in}}%
\pgfpathlineto{\pgfqpoint{5.705860in}{0.745763in}}%
\pgfpathlineto{\pgfqpoint{5.706713in}{0.745696in}}%
\pgfpathlineto{\pgfqpoint{5.707566in}{0.745630in}}%
\pgfpathlineto{\pgfqpoint{5.708419in}{0.745563in}}%
\pgfpathlineto{\pgfqpoint{5.709272in}{0.745496in}}%
\pgfpathlineto{\pgfqpoint{5.710125in}{0.745429in}}%
\pgfpathlineto{\pgfqpoint{5.710978in}{0.745363in}}%
\pgfpathlineto{\pgfqpoint{5.711831in}{0.745296in}}%
\pgfpathlineto{\pgfqpoint{5.712683in}{0.745229in}}%
\pgfpathlineto{\pgfqpoint{5.713536in}{0.745162in}}%
\pgfpathlineto{\pgfqpoint{5.714389in}{0.745096in}}%
\pgfpathlineto{\pgfqpoint{5.715242in}{0.745127in}}%
\pgfpathlineto{\pgfqpoint{5.716095in}{0.745366in}}%
\pgfpathlineto{\pgfqpoint{5.716948in}{0.745611in}}%
\pgfpathlineto{\pgfqpoint{5.717801in}{0.745857in}}%
\pgfpathlineto{\pgfqpoint{5.718654in}{0.745959in}}%
\pgfpathlineto{\pgfqpoint{5.719507in}{0.745859in}}%
\pgfpathlineto{\pgfqpoint{5.720359in}{0.745757in}}%
\pgfpathlineto{\pgfqpoint{5.721212in}{0.745656in}}%
\pgfpathlineto{\pgfqpoint{5.722065in}{0.745554in}}%
\pgfpathlineto{\pgfqpoint{5.722918in}{0.745452in}}%
\pgfpathlineto{\pgfqpoint{5.723771in}{0.745350in}}%
\pgfpathlineto{\pgfqpoint{5.724624in}{0.745249in}}%
\pgfpathlineto{\pgfqpoint{5.725477in}{0.745147in}}%
\pgfpathlineto{\pgfqpoint{5.726330in}{0.745045in}}%
\pgfpathlineto{\pgfqpoint{5.727182in}{0.744943in}}%
\pgfpathlineto{\pgfqpoint{5.728035in}{0.744842in}}%
\pgfpathlineto{\pgfqpoint{5.728888in}{0.744740in}}%
\pgfpathlineto{\pgfqpoint{5.729741in}{0.744658in}}%
\pgfpathlineto{\pgfqpoint{5.730594in}{0.744737in}}%
\pgfpathlineto{\pgfqpoint{5.731447in}{0.744853in}}%
\pgfpathlineto{\pgfqpoint{5.732300in}{0.744968in}}%
\pgfpathlineto{\pgfqpoint{5.733153in}{0.745072in}}%
\pgfpathlineto{\pgfqpoint{5.734005in}{0.745022in}}%
\pgfpathlineto{\pgfqpoint{5.734858in}{0.744923in}}%
\pgfpathlineto{\pgfqpoint{5.735711in}{0.744824in}}%
\pgfpathlineto{\pgfqpoint{5.736564in}{0.744725in}}%
\pgfpathlineto{\pgfqpoint{5.737417in}{0.744626in}}%
\pgfpathlineto{\pgfqpoint{5.738270in}{0.744801in}}%
\pgfpathlineto{\pgfqpoint{5.739123in}{0.746135in}}%
\pgfpathlineto{\pgfqpoint{5.739976in}{0.748006in}}%
\pgfpathlineto{\pgfqpoint{5.740828in}{0.748725in}}%
\pgfpathlineto{\pgfqpoint{5.741681in}{0.748712in}}%
\pgfpathlineto{\pgfqpoint{5.742534in}{0.748700in}}%
\pgfpathlineto{\pgfqpoint{5.743387in}{0.748703in}}%
\pgfpathlineto{\pgfqpoint{5.744240in}{0.748697in}}%
\pgfpathlineto{\pgfqpoint{5.745093in}{0.748685in}}%
\pgfpathlineto{\pgfqpoint{5.745946in}{0.748673in}}%
\pgfpathlineto{\pgfqpoint{5.746799in}{0.748661in}}%
\pgfpathlineto{\pgfqpoint{5.747652in}{0.748648in}}%
\pgfpathlineto{\pgfqpoint{5.748504in}{0.748636in}}%
\pgfpathlineto{\pgfqpoint{5.749357in}{0.748624in}}%
\pgfpathlineto{\pgfqpoint{5.750210in}{0.748612in}}%
\pgfpathlineto{\pgfqpoint{5.751063in}{0.748600in}}%
\pgfpathlineto{\pgfqpoint{5.751916in}{0.748587in}}%
\pgfpathlineto{\pgfqpoint{5.752769in}{0.748575in}}%
\pgfpathlineto{\pgfqpoint{5.753622in}{0.748550in}}%
\pgfpathlineto{\pgfqpoint{5.754475in}{0.748460in}}%
\pgfpathlineto{\pgfqpoint{5.755327in}{0.748361in}}%
\pgfpathlineto{\pgfqpoint{5.756180in}{0.748266in}}%
\pgfpathlineto{\pgfqpoint{5.757033in}{0.748255in}}%
\pgfpathlineto{\pgfqpoint{5.757886in}{0.748277in}}%
\pgfpathlineto{\pgfqpoint{5.758739in}{0.748299in}}%
\pgfpathlineto{\pgfqpoint{5.759592in}{0.748321in}}%
\pgfpathlineto{\pgfqpoint{5.760445in}{0.748344in}}%
\pgfpathlineto{\pgfqpoint{5.761298in}{0.748366in}}%
\pgfpathlineto{\pgfqpoint{5.762150in}{0.748388in}}%
\pgfpathlineto{\pgfqpoint{5.763003in}{0.748410in}}%
\pgfpathlineto{\pgfqpoint{5.763856in}{0.748432in}}%
\pgfpathlineto{\pgfqpoint{5.764709in}{0.748454in}}%
\pgfpathlineto{\pgfqpoint{5.765562in}{0.748476in}}%
\pgfpathlineto{\pgfqpoint{5.766415in}{0.748498in}}%
\pgfpathlineto{\pgfqpoint{5.767268in}{0.748520in}}%
\pgfpathlineto{\pgfqpoint{5.768121in}{0.748542in}}%
\pgfpathlineto{\pgfqpoint{5.768973in}{0.748565in}}%
\pgfpathlineto{\pgfqpoint{5.769826in}{0.748587in}}%
\pgfpathlineto{\pgfqpoint{5.770679in}{0.748609in}}%
\pgfpathlineto{\pgfqpoint{5.771532in}{0.748631in}}%
\pgfpathlineto{\pgfqpoint{5.772385in}{0.748653in}}%
\pgfpathlineto{\pgfqpoint{5.773238in}{0.748675in}}%
\pgfpathlineto{\pgfqpoint{5.774091in}{0.748697in}}%
\pgfpathlineto{\pgfqpoint{5.774944in}{0.748719in}}%
\pgfpathlineto{\pgfqpoint{5.775797in}{0.748741in}}%
\pgfpathlineto{\pgfqpoint{5.776649in}{0.748764in}}%
\pgfpathlineto{\pgfqpoint{5.777502in}{0.748786in}}%
\pgfpathlineto{\pgfqpoint{5.778355in}{0.748808in}}%
\pgfpathlineto{\pgfqpoint{5.779208in}{0.748830in}}%
\pgfpathlineto{\pgfqpoint{5.780061in}{0.748852in}}%
\pgfpathlineto{\pgfqpoint{5.780914in}{0.748874in}}%
\pgfpathlineto{\pgfqpoint{5.781767in}{0.748896in}}%
\pgfpathlineto{\pgfqpoint{5.782620in}{0.748918in}}%
\pgfpathlineto{\pgfqpoint{5.783472in}{0.748940in}}%
\pgfpathlineto{\pgfqpoint{5.784325in}{0.748962in}}%
\pgfpathlineto{\pgfqpoint{5.785178in}{0.748985in}}%
\pgfpathlineto{\pgfqpoint{5.786031in}{0.749007in}}%
\pgfpathlineto{\pgfqpoint{5.786884in}{0.749029in}}%
\pgfpathlineto{\pgfqpoint{5.787737in}{0.749051in}}%
\pgfpathlineto{\pgfqpoint{5.788590in}{0.749073in}}%
\pgfpathlineto{\pgfqpoint{5.789443in}{0.749095in}}%
\pgfpathlineto{\pgfqpoint{5.790295in}{0.749117in}}%
\pgfpathlineto{\pgfqpoint{5.791148in}{0.749139in}}%
\pgfpathlineto{\pgfqpoint{5.792001in}{0.749161in}}%
\pgfpathlineto{\pgfqpoint{5.792854in}{0.749183in}}%
\pgfpathlineto{\pgfqpoint{5.793707in}{0.749206in}}%
\pgfpathlineto{\pgfqpoint{5.794560in}{0.749228in}}%
\pgfpathlineto{\pgfqpoint{5.795413in}{0.749250in}}%
\pgfpathlineto{\pgfqpoint{5.796266in}{0.749272in}}%
\pgfpathlineto{\pgfqpoint{5.797118in}{0.749294in}}%
\pgfpathlineto{\pgfqpoint{5.797971in}{0.749316in}}%
\pgfpathlineto{\pgfqpoint{5.798824in}{0.749338in}}%
\pgfpathlineto{\pgfqpoint{5.799677in}{0.749360in}}%
\pgfpathlineto{\pgfqpoint{5.800530in}{0.749382in}}%
\pgfpathlineto{\pgfqpoint{5.801383in}{0.749405in}}%
\pgfpathlineto{\pgfqpoint{5.802236in}{0.749427in}}%
\pgfpathlineto{\pgfqpoint{5.803089in}{0.749449in}}%
\pgfpathlineto{\pgfqpoint{5.803941in}{0.749471in}}%
\pgfpathlineto{\pgfqpoint{5.804794in}{0.749493in}}%
\pgfpathlineto{\pgfqpoint{5.805647in}{0.749515in}}%
\pgfpathlineto{\pgfqpoint{5.806500in}{0.749537in}}%
\pgfpathlineto{\pgfqpoint{5.807353in}{0.749559in}}%
\pgfpathlineto{\pgfqpoint{5.808206in}{0.749581in}}%
\pgfpathlineto{\pgfqpoint{5.809059in}{0.749603in}}%
\pgfpathlineto{\pgfqpoint{5.809912in}{0.749626in}}%
\pgfpathlineto{\pgfqpoint{5.810765in}{0.749648in}}%
\pgfpathlineto{\pgfqpoint{5.811617in}{0.749670in}}%
\pgfpathlineto{\pgfqpoint{5.812470in}{0.749692in}}%
\pgfpathlineto{\pgfqpoint{5.813323in}{0.749714in}}%
\pgfpathlineto{\pgfqpoint{5.814176in}{0.749736in}}%
\pgfpathlineto{\pgfqpoint{5.815029in}{0.749758in}}%
\pgfpathlineto{\pgfqpoint{5.815882in}{0.749780in}}%
\pgfpathlineto{\pgfqpoint{5.816735in}{0.749802in}}%
\pgfpathlineto{\pgfqpoint{5.817588in}{0.749824in}}%
\pgfpathlineto{\pgfqpoint{5.818440in}{0.749847in}}%
\pgfpathlineto{\pgfqpoint{5.819293in}{0.749869in}}%
\pgfpathlineto{\pgfqpoint{5.820146in}{0.749891in}}%
\pgfpathlineto{\pgfqpoint{5.820999in}{0.749913in}}%
\pgfpathlineto{\pgfqpoint{5.821852in}{0.749935in}}%
\pgfpathlineto{\pgfqpoint{5.822705in}{0.749957in}}%
\pgfpathlineto{\pgfqpoint{5.823558in}{0.749979in}}%
\pgfpathlineto{\pgfqpoint{5.824411in}{0.750001in}}%
\pgfpathlineto{\pgfqpoint{5.825263in}{0.749940in}}%
\pgfpathlineto{\pgfqpoint{5.826116in}{0.749729in}}%
\pgfpathlineto{\pgfqpoint{5.826969in}{0.749515in}}%
\pgfpathlineto{\pgfqpoint{5.827822in}{0.749302in}}%
\pgfpathlineto{\pgfqpoint{5.828675in}{0.749088in}}%
\pgfpathlineto{\pgfqpoint{5.829528in}{0.748874in}}%
\pgfpathlineto{\pgfqpoint{5.830381in}{0.748661in}}%
\pgfpathlineto{\pgfqpoint{5.831234in}{0.748447in}}%
\pgfpathlineto{\pgfqpoint{5.832086in}{0.748233in}}%
\pgfpathlineto{\pgfqpoint{5.832939in}{0.748020in}}%
\pgfpathlineto{\pgfqpoint{5.833792in}{0.747806in}}%
\pgfpathlineto{\pgfqpoint{5.834645in}{0.747592in}}%
\pgfpathlineto{\pgfqpoint{5.835498in}{0.747379in}}%
\pgfpathlineto{\pgfqpoint{5.836351in}{0.747165in}}%
\pgfpathlineto{\pgfqpoint{5.837204in}{0.747173in}}%
\pgfpathlineto{\pgfqpoint{5.838057in}{0.747368in}}%
\pgfpathlineto{\pgfqpoint{5.838910in}{0.747563in}}%
\pgfpathlineto{\pgfqpoint{5.839762in}{0.747758in}}%
\pgfpathlineto{\pgfqpoint{5.840615in}{0.747953in}}%
\pgfpathlineto{\pgfqpoint{5.841468in}{0.748148in}}%
\pgfpathlineto{\pgfqpoint{5.842321in}{0.748343in}}%
\pgfpathlineto{\pgfqpoint{5.843174in}{0.748469in}}%
\pgfpathlineto{\pgfqpoint{5.844027in}{0.748471in}}%
\pgfpathlineto{\pgfqpoint{5.844880in}{0.748471in}}%
\pgfpathlineto{\pgfqpoint{5.845733in}{0.748471in}}%
\pgfpathlineto{\pgfqpoint{5.846585in}{0.748471in}}%
\pgfpathlineto{\pgfqpoint{5.847438in}{0.748470in}}%
\pgfpathlineto{\pgfqpoint{5.848291in}{0.748470in}}%
\pgfpathlineto{\pgfqpoint{5.849144in}{0.748470in}}%
\pgfpathlineto{\pgfqpoint{5.849997in}{0.748469in}}%
\pgfpathlineto{\pgfqpoint{5.850850in}{0.748469in}}%
\pgfpathlineto{\pgfqpoint{5.851703in}{0.748469in}}%
\pgfpathlineto{\pgfqpoint{5.852556in}{0.748468in}}%
\pgfpathlineto{\pgfqpoint{5.853408in}{0.748468in}}%
\pgfpathlineto{\pgfqpoint{5.854261in}{0.748468in}}%
\pgfpathlineto{\pgfqpoint{5.855114in}{0.748468in}}%
\pgfpathlineto{\pgfqpoint{5.855967in}{0.748467in}}%
\pgfpathlineto{\pgfqpoint{5.856820in}{0.748467in}}%
\pgfpathlineto{\pgfqpoint{5.857673in}{0.748467in}}%
\pgfpathlineto{\pgfqpoint{5.858526in}{0.748466in}}%
\pgfpathlineto{\pgfqpoint{5.859379in}{0.748730in}}%
\pgfpathlineto{\pgfqpoint{5.860231in}{0.748822in}}%
\pgfpathlineto{\pgfqpoint{5.861084in}{0.749214in}}%
\pgfpathlineto{\pgfqpoint{5.861937in}{0.749715in}}%
\pgfpathlineto{\pgfqpoint{5.862790in}{0.749756in}}%
\pgfpathlineto{\pgfqpoint{5.863643in}{0.749615in}}%
\pgfpathlineto{\pgfqpoint{5.864496in}{0.749474in}}%
\pgfpathlineto{\pgfqpoint{5.865349in}{0.749333in}}%
\pgfpathlineto{\pgfqpoint{5.866202in}{0.749192in}}%
\pgfpathlineto{\pgfqpoint{5.867055in}{0.749051in}}%
\pgfpathlineto{\pgfqpoint{5.867907in}{0.748910in}}%
\pgfpathlineto{\pgfqpoint{5.868760in}{0.748768in}}%
\pgfpathlineto{\pgfqpoint{5.869613in}{0.748627in}}%
\pgfpathlineto{\pgfqpoint{5.870466in}{0.748486in}}%
\pgfpathlineto{\pgfqpoint{5.871319in}{0.748345in}}%
\pgfpathlineto{\pgfqpoint{5.872172in}{0.748204in}}%
\pgfpathlineto{\pgfqpoint{5.873025in}{0.748063in}}%
\pgfpathlineto{\pgfqpoint{5.873878in}{0.747922in}}%
\pgfpathlineto{\pgfqpoint{5.874730in}{0.747780in}}%
\pgfpathlineto{\pgfqpoint{5.875583in}{0.747639in}}%
\pgfpathlineto{\pgfqpoint{5.876436in}{0.747498in}}%
\pgfpathlineto{\pgfqpoint{5.877289in}{0.747357in}}%
\pgfpathlineto{\pgfqpoint{5.878142in}{0.747216in}}%
\pgfpathlineto{\pgfqpoint{5.878995in}{0.747022in}}%
\pgfpathlineto{\pgfqpoint{5.879848in}{0.746171in}}%
\pgfpathlineto{\pgfqpoint{5.880701in}{0.745126in}}%
\pgfpathlineto{\pgfqpoint{5.881553in}{0.744596in}}%
\pgfpathlineto{\pgfqpoint{5.882406in}{0.743571in}}%
\pgfpathlineto{\pgfqpoint{5.883259in}{0.741388in}}%
\pgfpathlineto{\pgfqpoint{5.884112in}{0.741361in}}%
\pgfpathlineto{\pgfqpoint{5.884965in}{0.741353in}}%
\pgfpathlineto{\pgfqpoint{5.885818in}{0.741279in}}%
\pgfpathlineto{\pgfqpoint{5.886671in}{0.741205in}}%
\pgfpathlineto{\pgfqpoint{5.887524in}{0.741131in}}%
\pgfpathlineto{\pgfqpoint{5.888376in}{0.741057in}}%
\pgfpathlineto{\pgfqpoint{5.889229in}{0.740983in}}%
\pgfpathlineto{\pgfqpoint{5.890082in}{0.740909in}}%
\pgfpathlineto{\pgfqpoint{5.890935in}{0.740835in}}%
\pgfpathlineto{\pgfqpoint{5.891788in}{0.740761in}}%
\pgfpathlineto{\pgfqpoint{5.892641in}{0.740687in}}%
\pgfpathlineto{\pgfqpoint{5.893494in}{0.740613in}}%
\pgfpathlineto{\pgfqpoint{5.894347in}{0.740539in}}%
\pgfpathlineto{\pgfqpoint{5.895200in}{0.740465in}}%
\pgfpathlineto{\pgfqpoint{5.896052in}{0.740391in}}%
\pgfpathlineto{\pgfqpoint{5.896905in}{0.740317in}}%
\pgfpathlineto{\pgfqpoint{5.897758in}{0.740243in}}%
\pgfpathlineto{\pgfqpoint{5.898611in}{0.740170in}}%
\pgfpathlineto{\pgfqpoint{5.899464in}{0.740137in}}%
\pgfpathlineto{\pgfqpoint{5.900317in}{0.740127in}}%
\pgfpathlineto{\pgfqpoint{5.901170in}{0.740117in}}%
\pgfpathlineto{\pgfqpoint{5.902023in}{0.740108in}}%
\pgfpathlineto{\pgfqpoint{5.902875in}{0.740098in}}%
\pgfpathlineto{\pgfqpoint{5.903728in}{0.740088in}}%
\pgfpathlineto{\pgfqpoint{5.904581in}{0.740078in}}%
\pgfpathlineto{\pgfqpoint{5.905434in}{0.740068in}}%
\pgfpathlineto{\pgfqpoint{5.906287in}{0.740058in}}%
\pgfpathlineto{\pgfqpoint{5.907140in}{0.740048in}}%
\pgfpathlineto{\pgfqpoint{5.907993in}{0.740039in}}%
\pgfpathlineto{\pgfqpoint{5.908846in}{0.740029in}}%
\pgfpathlineto{\pgfqpoint{5.909698in}{0.740019in}}%
\pgfpathlineto{\pgfqpoint{5.910551in}{0.740009in}}%
\pgfpathlineto{\pgfqpoint{5.911404in}{0.739999in}}%
\pgfpathlineto{\pgfqpoint{5.912257in}{0.739989in}}%
\pgfpathlineto{\pgfqpoint{5.913110in}{0.739980in}}%
\pgfpathlineto{\pgfqpoint{5.913963in}{0.739970in}}%
\pgfpathlineto{\pgfqpoint{5.914816in}{0.739960in}}%
\pgfpathlineto{\pgfqpoint{5.915669in}{0.739950in}}%
\pgfpathlineto{\pgfqpoint{5.916521in}{0.739940in}}%
\pgfpathlineto{\pgfqpoint{5.917374in}{0.739930in}}%
\pgfpathlineto{\pgfqpoint{5.918227in}{0.739920in}}%
\pgfpathlineto{\pgfqpoint{5.919080in}{0.739911in}}%
\pgfpathlineto{\pgfqpoint{5.919933in}{0.739901in}}%
\pgfpathlineto{\pgfqpoint{5.920786in}{0.739891in}}%
\pgfpathlineto{\pgfqpoint{5.921639in}{0.739881in}}%
\pgfpathlineto{\pgfqpoint{5.922492in}{0.739871in}}%
\pgfpathlineto{\pgfqpoint{5.923344in}{0.739861in}}%
\pgfpathlineto{\pgfqpoint{5.924197in}{0.739852in}}%
\pgfpathlineto{\pgfqpoint{5.925050in}{0.739842in}}%
\pgfpathlineto{\pgfqpoint{5.925903in}{0.739832in}}%
\pgfpathlineto{\pgfqpoint{5.926756in}{0.739822in}}%
\pgfpathlineto{\pgfqpoint{5.927609in}{0.739812in}}%
\pgfpathlineto{\pgfqpoint{5.928462in}{0.739802in}}%
\pgfpathlineto{\pgfqpoint{5.929315in}{0.739792in}}%
\pgfpathlineto{\pgfqpoint{5.930168in}{0.739783in}}%
\pgfpathlineto{\pgfqpoint{5.931020in}{0.739773in}}%
\pgfpathlineto{\pgfqpoint{5.931873in}{0.739763in}}%
\pgfpathlineto{\pgfqpoint{5.932726in}{0.739753in}}%
\pgfpathlineto{\pgfqpoint{5.933579in}{0.739743in}}%
\pgfpathlineto{\pgfqpoint{5.934432in}{0.739733in}}%
\pgfpathlineto{\pgfqpoint{5.935285in}{0.739724in}}%
\pgfpathlineto{\pgfqpoint{5.936138in}{0.739714in}}%
\pgfpathlineto{\pgfqpoint{5.936991in}{0.739704in}}%
\pgfpathlineto{\pgfqpoint{5.937843in}{0.739694in}}%
\pgfpathlineto{\pgfqpoint{5.938696in}{0.739684in}}%
\pgfpathlineto{\pgfqpoint{5.939549in}{0.739674in}}%
\pgfpathlineto{\pgfqpoint{5.940402in}{0.739665in}}%
\pgfpathlineto{\pgfqpoint{5.941255in}{0.739655in}}%
\pgfpathlineto{\pgfqpoint{5.942108in}{0.739645in}}%
\pgfpathlineto{\pgfqpoint{5.942961in}{0.739635in}}%
\pgfpathlineto{\pgfqpoint{5.943814in}{0.739625in}}%
\pgfpathlineto{\pgfqpoint{5.944666in}{0.739615in}}%
\pgfpathlineto{\pgfqpoint{5.945519in}{0.739605in}}%
\pgfpathlineto{\pgfqpoint{5.946372in}{0.739596in}}%
\pgfpathlineto{\pgfqpoint{5.947225in}{0.739586in}}%
\pgfpathlineto{\pgfqpoint{5.948078in}{0.739576in}}%
\pgfpathlineto{\pgfqpoint{5.948931in}{0.739566in}}%
\pgfpathlineto{\pgfqpoint{5.949784in}{0.739556in}}%
\pgfpathlineto{\pgfqpoint{5.950637in}{0.739546in}}%
\pgfpathlineto{\pgfqpoint{5.951489in}{0.739537in}}%
\pgfpathlineto{\pgfqpoint{5.952342in}{0.739527in}}%
\pgfpathlineto{\pgfqpoint{5.953195in}{0.739517in}}%
\pgfpathlineto{\pgfqpoint{5.954048in}{0.739507in}}%
\pgfpathlineto{\pgfqpoint{5.954901in}{0.739497in}}%
\pgfpathlineto{\pgfqpoint{5.955754in}{0.739487in}}%
\pgfpathlineto{\pgfqpoint{5.956607in}{0.739477in}}%
\pgfpathlineto{\pgfqpoint{5.957460in}{0.739468in}}%
\pgfpathlineto{\pgfqpoint{5.958313in}{0.739458in}}%
\pgfpathlineto{\pgfqpoint{5.959165in}{0.739448in}}%
\pgfpathlineto{\pgfqpoint{5.960018in}{0.739438in}}%
\pgfpathlineto{\pgfqpoint{5.960871in}{0.739428in}}%
\pgfpathlineto{\pgfqpoint{5.961724in}{0.739418in}}%
\pgfpathlineto{\pgfqpoint{5.962577in}{0.739409in}}%
\pgfpathlineto{\pgfqpoint{5.963430in}{0.739399in}}%
\pgfpathlineto{\pgfqpoint{5.964283in}{0.739408in}}%
\pgfpathlineto{\pgfqpoint{5.965136in}{0.739423in}}%
\pgfpathlineto{\pgfqpoint{5.965988in}{0.739438in}}%
\pgfpathlineto{\pgfqpoint{5.966841in}{0.739454in}}%
\pgfpathlineto{\pgfqpoint{5.967694in}{0.739469in}}%
\pgfpathlineto{\pgfqpoint{5.968547in}{0.739484in}}%
\pgfpathlineto{\pgfqpoint{5.969400in}{0.739500in}}%
\pgfpathlineto{\pgfqpoint{5.970253in}{0.739515in}}%
\pgfpathlineto{\pgfqpoint{5.971106in}{0.739530in}}%
\pgfpathlineto{\pgfqpoint{5.971959in}{0.739546in}}%
\pgfpathlineto{\pgfqpoint{5.972811in}{0.739561in}}%
\pgfpathlineto{\pgfqpoint{5.973664in}{0.739576in}}%
\pgfpathlineto{\pgfqpoint{5.974517in}{0.739592in}}%
\pgfpathlineto{\pgfqpoint{5.975370in}{0.739607in}}%
\pgfpathlineto{\pgfqpoint{5.976223in}{0.739622in}}%
\pgfpathlineto{\pgfqpoint{5.977076in}{0.739638in}}%
\pgfpathlineto{\pgfqpoint{5.977929in}{0.739653in}}%
\pgfpathlineto{\pgfqpoint{5.978782in}{0.739676in}}%
\pgfpathlineto{\pgfqpoint{5.979634in}{0.739764in}}%
\pgfpathlineto{\pgfqpoint{5.980487in}{0.739867in}}%
\pgfpathlineto{\pgfqpoint{5.981340in}{0.739969in}}%
\pgfpathlineto{\pgfqpoint{5.982193in}{0.740071in}}%
\pgfpathlineto{\pgfqpoint{5.983046in}{0.740174in}}%
\pgfpathlineto{\pgfqpoint{5.983899in}{0.740276in}}%
\pgfpathlineto{\pgfqpoint{5.984752in}{0.740241in}}%
\pgfpathlineto{\pgfqpoint{5.985605in}{0.739593in}}%
\pgfpathlineto{\pgfqpoint{5.986458in}{0.739631in}}%
\pgfpathlineto{\pgfqpoint{5.987310in}{0.739721in}}%
\pgfpathlineto{\pgfqpoint{5.988163in}{0.739810in}}%
\pgfpathlineto{\pgfqpoint{5.989016in}{0.739900in}}%
\pgfpathlineto{\pgfqpoint{5.989869in}{0.739990in}}%
\pgfpathlineto{\pgfqpoint{5.990722in}{0.740054in}}%
\pgfpathlineto{\pgfqpoint{5.991575in}{0.740047in}}%
\pgfpathlineto{\pgfqpoint{5.992428in}{0.740036in}}%
\pgfpathlineto{\pgfqpoint{5.993281in}{0.740025in}}%
\pgfpathlineto{\pgfqpoint{5.994133in}{0.740014in}}%
\pgfpathlineto{\pgfqpoint{5.994986in}{0.740003in}}%
\pgfpathlineto{\pgfqpoint{5.995839in}{0.739992in}}%
\pgfpathlineto{\pgfqpoint{5.996692in}{0.739981in}}%
\pgfpathlineto{\pgfqpoint{5.997545in}{0.739970in}}%
\pgfpathlineto{\pgfqpoint{5.998398in}{0.739959in}}%
\pgfpathlineto{\pgfqpoint{5.999251in}{0.739922in}}%
\pgfpathlineto{\pgfqpoint{6.000104in}{0.739667in}}%
\pgfpathlineto{\pgfqpoint{6.000956in}{0.739511in}}%
\pgfpathlineto{\pgfqpoint{6.001809in}{0.739598in}}%
\pgfpathlineto{\pgfqpoint{6.002662in}{0.739596in}}%
\pgfpathlineto{\pgfqpoint{6.003515in}{0.739581in}}%
\pgfpathlineto{\pgfqpoint{6.004368in}{0.739656in}}%
\pgfpathlineto{\pgfqpoint{6.004368in}{0.739656in}}%
\pgfpathlineto{\pgfqpoint{6.004368in}{0.739656in}}%
\pgfpathlineto{\pgfqpoint{6.003515in}{0.739656in}}%
\pgfpathlineto{\pgfqpoint{6.002662in}{0.739656in}}%
\pgfpathlineto{\pgfqpoint{6.001809in}{0.739656in}}%
\pgfpathlineto{\pgfqpoint{6.000956in}{0.739656in}}%
\pgfpathlineto{\pgfqpoint{6.000104in}{0.739656in}}%
\pgfpathlineto{\pgfqpoint{5.999251in}{0.739656in}}%
\pgfpathlineto{\pgfqpoint{5.998398in}{0.739656in}}%
\pgfpathlineto{\pgfqpoint{5.997545in}{0.739656in}}%
\pgfpathlineto{\pgfqpoint{5.996692in}{0.739656in}}%
\pgfpathlineto{\pgfqpoint{5.995839in}{0.739656in}}%
\pgfpathlineto{\pgfqpoint{5.994986in}{0.739656in}}%
\pgfpathlineto{\pgfqpoint{5.994133in}{0.739656in}}%
\pgfpathlineto{\pgfqpoint{5.993281in}{0.739656in}}%
\pgfpathlineto{\pgfqpoint{5.992428in}{0.739656in}}%
\pgfpathlineto{\pgfqpoint{5.991575in}{0.739656in}}%
\pgfpathlineto{\pgfqpoint{5.990722in}{0.739656in}}%
\pgfpathlineto{\pgfqpoint{5.989869in}{0.739656in}}%
\pgfpathlineto{\pgfqpoint{5.989016in}{0.739656in}}%
\pgfpathlineto{\pgfqpoint{5.988163in}{0.739656in}}%
\pgfpathlineto{\pgfqpoint{5.987310in}{0.739656in}}%
\pgfpathlineto{\pgfqpoint{5.986458in}{0.739656in}}%
\pgfpathlineto{\pgfqpoint{5.985605in}{0.739656in}}%
\pgfpathlineto{\pgfqpoint{5.984752in}{0.739656in}}%
\pgfpathlineto{\pgfqpoint{5.983899in}{0.739656in}}%
\pgfpathlineto{\pgfqpoint{5.983046in}{0.739656in}}%
\pgfpathlineto{\pgfqpoint{5.982193in}{0.739656in}}%
\pgfpathlineto{\pgfqpoint{5.981340in}{0.739656in}}%
\pgfpathlineto{\pgfqpoint{5.980487in}{0.739656in}}%
\pgfpathlineto{\pgfqpoint{5.979634in}{0.739656in}}%
\pgfpathlineto{\pgfqpoint{5.978782in}{0.739656in}}%
\pgfpathlineto{\pgfqpoint{5.977929in}{0.739656in}}%
\pgfpathlineto{\pgfqpoint{5.977076in}{0.739656in}}%
\pgfpathlineto{\pgfqpoint{5.976223in}{0.739656in}}%
\pgfpathlineto{\pgfqpoint{5.975370in}{0.739656in}}%
\pgfpathlineto{\pgfqpoint{5.974517in}{0.739656in}}%
\pgfpathlineto{\pgfqpoint{5.973664in}{0.739656in}}%
\pgfpathlineto{\pgfqpoint{5.972811in}{0.739656in}}%
\pgfpathlineto{\pgfqpoint{5.971959in}{0.739656in}}%
\pgfpathlineto{\pgfqpoint{5.971106in}{0.739656in}}%
\pgfpathlineto{\pgfqpoint{5.970253in}{0.739656in}}%
\pgfpathlineto{\pgfqpoint{5.969400in}{0.739656in}}%
\pgfpathlineto{\pgfqpoint{5.968547in}{0.739656in}}%
\pgfpathlineto{\pgfqpoint{5.967694in}{0.739656in}}%
\pgfpathlineto{\pgfqpoint{5.966841in}{0.739656in}}%
\pgfpathlineto{\pgfqpoint{5.965988in}{0.739656in}}%
\pgfpathlineto{\pgfqpoint{5.965136in}{0.739656in}}%
\pgfpathlineto{\pgfqpoint{5.964283in}{0.739656in}}%
\pgfpathlineto{\pgfqpoint{5.963430in}{0.739656in}}%
\pgfpathlineto{\pgfqpoint{5.962577in}{0.739656in}}%
\pgfpathlineto{\pgfqpoint{5.961724in}{0.739656in}}%
\pgfpathlineto{\pgfqpoint{5.960871in}{0.739656in}}%
\pgfpathlineto{\pgfqpoint{5.960018in}{0.739656in}}%
\pgfpathlineto{\pgfqpoint{5.959165in}{0.739656in}}%
\pgfpathlineto{\pgfqpoint{5.958313in}{0.739656in}}%
\pgfpathlineto{\pgfqpoint{5.957460in}{0.739656in}}%
\pgfpathlineto{\pgfqpoint{5.956607in}{0.739656in}}%
\pgfpathlineto{\pgfqpoint{5.955754in}{0.739656in}}%
\pgfpathlineto{\pgfqpoint{5.954901in}{0.739656in}}%
\pgfpathlineto{\pgfqpoint{5.954048in}{0.739656in}}%
\pgfpathlineto{\pgfqpoint{5.953195in}{0.739656in}}%
\pgfpathlineto{\pgfqpoint{5.952342in}{0.739656in}}%
\pgfpathlineto{\pgfqpoint{5.951489in}{0.739656in}}%
\pgfpathlineto{\pgfqpoint{5.950637in}{0.739656in}}%
\pgfpathlineto{\pgfqpoint{5.949784in}{0.739656in}}%
\pgfpathlineto{\pgfqpoint{5.948931in}{0.739656in}}%
\pgfpathlineto{\pgfqpoint{5.948078in}{0.739656in}}%
\pgfpathlineto{\pgfqpoint{5.947225in}{0.739656in}}%
\pgfpathlineto{\pgfqpoint{5.946372in}{0.739656in}}%
\pgfpathlineto{\pgfqpoint{5.945519in}{0.739656in}}%
\pgfpathlineto{\pgfqpoint{5.944666in}{0.739656in}}%
\pgfpathlineto{\pgfqpoint{5.943814in}{0.739656in}}%
\pgfpathlineto{\pgfqpoint{5.942961in}{0.739656in}}%
\pgfpathlineto{\pgfqpoint{5.942108in}{0.739656in}}%
\pgfpathlineto{\pgfqpoint{5.941255in}{0.739656in}}%
\pgfpathlineto{\pgfqpoint{5.940402in}{0.739656in}}%
\pgfpathlineto{\pgfqpoint{5.939549in}{0.739656in}}%
\pgfpathlineto{\pgfqpoint{5.938696in}{0.739656in}}%
\pgfpathlineto{\pgfqpoint{5.937843in}{0.739656in}}%
\pgfpathlineto{\pgfqpoint{5.936991in}{0.739656in}}%
\pgfpathlineto{\pgfqpoint{5.936138in}{0.739656in}}%
\pgfpathlineto{\pgfqpoint{5.935285in}{0.739656in}}%
\pgfpathlineto{\pgfqpoint{5.934432in}{0.739656in}}%
\pgfpathlineto{\pgfqpoint{5.933579in}{0.739656in}}%
\pgfpathlineto{\pgfqpoint{5.932726in}{0.739656in}}%
\pgfpathlineto{\pgfqpoint{5.931873in}{0.739656in}}%
\pgfpathlineto{\pgfqpoint{5.931020in}{0.739656in}}%
\pgfpathlineto{\pgfqpoint{5.930168in}{0.739656in}}%
\pgfpathlineto{\pgfqpoint{5.929315in}{0.739656in}}%
\pgfpathlineto{\pgfqpoint{5.928462in}{0.739656in}}%
\pgfpathlineto{\pgfqpoint{5.927609in}{0.739656in}}%
\pgfpathlineto{\pgfqpoint{5.926756in}{0.739656in}}%
\pgfpathlineto{\pgfqpoint{5.925903in}{0.739656in}}%
\pgfpathlineto{\pgfqpoint{5.925050in}{0.739656in}}%
\pgfpathlineto{\pgfqpoint{5.924197in}{0.739656in}}%
\pgfpathlineto{\pgfqpoint{5.923344in}{0.739656in}}%
\pgfpathlineto{\pgfqpoint{5.922492in}{0.739656in}}%
\pgfpathlineto{\pgfqpoint{5.921639in}{0.739656in}}%
\pgfpathlineto{\pgfqpoint{5.920786in}{0.739656in}}%
\pgfpathlineto{\pgfqpoint{5.919933in}{0.739656in}}%
\pgfpathlineto{\pgfqpoint{5.919080in}{0.739656in}}%
\pgfpathlineto{\pgfqpoint{5.918227in}{0.739656in}}%
\pgfpathlineto{\pgfqpoint{5.917374in}{0.739656in}}%
\pgfpathlineto{\pgfqpoint{5.916521in}{0.739656in}}%
\pgfpathlineto{\pgfqpoint{5.915669in}{0.739656in}}%
\pgfpathlineto{\pgfqpoint{5.914816in}{0.739656in}}%
\pgfpathlineto{\pgfqpoint{5.913963in}{0.739656in}}%
\pgfpathlineto{\pgfqpoint{5.913110in}{0.739656in}}%
\pgfpathlineto{\pgfqpoint{5.912257in}{0.739656in}}%
\pgfpathlineto{\pgfqpoint{5.911404in}{0.739656in}}%
\pgfpathlineto{\pgfqpoint{5.910551in}{0.739656in}}%
\pgfpathlineto{\pgfqpoint{5.909698in}{0.739656in}}%
\pgfpathlineto{\pgfqpoint{5.908846in}{0.739656in}}%
\pgfpathlineto{\pgfqpoint{5.907993in}{0.739656in}}%
\pgfpathlineto{\pgfqpoint{5.907140in}{0.739656in}}%
\pgfpathlineto{\pgfqpoint{5.906287in}{0.739656in}}%
\pgfpathlineto{\pgfqpoint{5.905434in}{0.739656in}}%
\pgfpathlineto{\pgfqpoint{5.904581in}{0.739656in}}%
\pgfpathlineto{\pgfqpoint{5.903728in}{0.739656in}}%
\pgfpathlineto{\pgfqpoint{5.902875in}{0.739656in}}%
\pgfpathlineto{\pgfqpoint{5.902023in}{0.739656in}}%
\pgfpathlineto{\pgfqpoint{5.901170in}{0.739656in}}%
\pgfpathlineto{\pgfqpoint{5.900317in}{0.739656in}}%
\pgfpathlineto{\pgfqpoint{5.899464in}{0.739656in}}%
\pgfpathlineto{\pgfqpoint{5.898611in}{0.739656in}}%
\pgfpathlineto{\pgfqpoint{5.897758in}{0.739656in}}%
\pgfpathlineto{\pgfqpoint{5.896905in}{0.739656in}}%
\pgfpathlineto{\pgfqpoint{5.896052in}{0.739656in}}%
\pgfpathlineto{\pgfqpoint{5.895200in}{0.739656in}}%
\pgfpathlineto{\pgfqpoint{5.894347in}{0.739656in}}%
\pgfpathlineto{\pgfqpoint{5.893494in}{0.739656in}}%
\pgfpathlineto{\pgfqpoint{5.892641in}{0.739656in}}%
\pgfpathlineto{\pgfqpoint{5.891788in}{0.739656in}}%
\pgfpathlineto{\pgfqpoint{5.890935in}{0.739656in}}%
\pgfpathlineto{\pgfqpoint{5.890082in}{0.739656in}}%
\pgfpathlineto{\pgfqpoint{5.889229in}{0.739656in}}%
\pgfpathlineto{\pgfqpoint{5.888376in}{0.739656in}}%
\pgfpathlineto{\pgfqpoint{5.887524in}{0.739656in}}%
\pgfpathlineto{\pgfqpoint{5.886671in}{0.739656in}}%
\pgfpathlineto{\pgfqpoint{5.885818in}{0.739656in}}%
\pgfpathlineto{\pgfqpoint{5.884965in}{0.739656in}}%
\pgfpathlineto{\pgfqpoint{5.884112in}{0.739656in}}%
\pgfpathlineto{\pgfqpoint{5.883259in}{0.739656in}}%
\pgfpathlineto{\pgfqpoint{5.882406in}{0.739656in}}%
\pgfpathlineto{\pgfqpoint{5.881553in}{0.739656in}}%
\pgfpathlineto{\pgfqpoint{5.880701in}{0.739656in}}%
\pgfpathlineto{\pgfqpoint{5.879848in}{0.739656in}}%
\pgfpathlineto{\pgfqpoint{5.878995in}{0.739656in}}%
\pgfpathlineto{\pgfqpoint{5.878142in}{0.739656in}}%
\pgfpathlineto{\pgfqpoint{5.877289in}{0.739656in}}%
\pgfpathlineto{\pgfqpoint{5.876436in}{0.739656in}}%
\pgfpathlineto{\pgfqpoint{5.875583in}{0.739656in}}%
\pgfpathlineto{\pgfqpoint{5.874730in}{0.739656in}}%
\pgfpathlineto{\pgfqpoint{5.873878in}{0.739656in}}%
\pgfpathlineto{\pgfqpoint{5.873025in}{0.739656in}}%
\pgfpathlineto{\pgfqpoint{5.872172in}{0.739656in}}%
\pgfpathlineto{\pgfqpoint{5.871319in}{0.739656in}}%
\pgfpathlineto{\pgfqpoint{5.870466in}{0.739656in}}%
\pgfpathlineto{\pgfqpoint{5.869613in}{0.739656in}}%
\pgfpathlineto{\pgfqpoint{5.868760in}{0.739656in}}%
\pgfpathlineto{\pgfqpoint{5.867907in}{0.739656in}}%
\pgfpathlineto{\pgfqpoint{5.867055in}{0.739656in}}%
\pgfpathlineto{\pgfqpoint{5.866202in}{0.739656in}}%
\pgfpathlineto{\pgfqpoint{5.865349in}{0.739656in}}%
\pgfpathlineto{\pgfqpoint{5.864496in}{0.739656in}}%
\pgfpathlineto{\pgfqpoint{5.863643in}{0.739656in}}%
\pgfpathlineto{\pgfqpoint{5.862790in}{0.739656in}}%
\pgfpathlineto{\pgfqpoint{5.861937in}{0.739656in}}%
\pgfpathlineto{\pgfqpoint{5.861084in}{0.739656in}}%
\pgfpathlineto{\pgfqpoint{5.860231in}{0.739656in}}%
\pgfpathlineto{\pgfqpoint{5.859379in}{0.739656in}}%
\pgfpathlineto{\pgfqpoint{5.858526in}{0.739656in}}%
\pgfpathlineto{\pgfqpoint{5.857673in}{0.739656in}}%
\pgfpathlineto{\pgfqpoint{5.856820in}{0.739656in}}%
\pgfpathlineto{\pgfqpoint{5.855967in}{0.739656in}}%
\pgfpathlineto{\pgfqpoint{5.855114in}{0.739656in}}%
\pgfpathlineto{\pgfqpoint{5.854261in}{0.739656in}}%
\pgfpathlineto{\pgfqpoint{5.853408in}{0.739656in}}%
\pgfpathlineto{\pgfqpoint{5.852556in}{0.739656in}}%
\pgfpathlineto{\pgfqpoint{5.851703in}{0.739656in}}%
\pgfpathlineto{\pgfqpoint{5.850850in}{0.739656in}}%
\pgfpathlineto{\pgfqpoint{5.849997in}{0.739656in}}%
\pgfpathlineto{\pgfqpoint{5.849144in}{0.739656in}}%
\pgfpathlineto{\pgfqpoint{5.848291in}{0.739656in}}%
\pgfpathlineto{\pgfqpoint{5.847438in}{0.739656in}}%
\pgfpathlineto{\pgfqpoint{5.846585in}{0.739656in}}%
\pgfpathlineto{\pgfqpoint{5.845733in}{0.739656in}}%
\pgfpathlineto{\pgfqpoint{5.844880in}{0.739656in}}%
\pgfpathlineto{\pgfqpoint{5.844027in}{0.739656in}}%
\pgfpathlineto{\pgfqpoint{5.843174in}{0.739656in}}%
\pgfpathlineto{\pgfqpoint{5.842321in}{0.739656in}}%
\pgfpathlineto{\pgfqpoint{5.841468in}{0.739656in}}%
\pgfpathlineto{\pgfqpoint{5.840615in}{0.739656in}}%
\pgfpathlineto{\pgfqpoint{5.839762in}{0.739656in}}%
\pgfpathlineto{\pgfqpoint{5.838910in}{0.739656in}}%
\pgfpathlineto{\pgfqpoint{5.838057in}{0.739656in}}%
\pgfpathlineto{\pgfqpoint{5.837204in}{0.739656in}}%
\pgfpathlineto{\pgfqpoint{5.836351in}{0.739656in}}%
\pgfpathlineto{\pgfqpoint{5.835498in}{0.739656in}}%
\pgfpathlineto{\pgfqpoint{5.834645in}{0.739656in}}%
\pgfpathlineto{\pgfqpoint{5.833792in}{0.739656in}}%
\pgfpathlineto{\pgfqpoint{5.832939in}{0.739656in}}%
\pgfpathlineto{\pgfqpoint{5.832086in}{0.739656in}}%
\pgfpathlineto{\pgfqpoint{5.831234in}{0.739656in}}%
\pgfpathlineto{\pgfqpoint{5.830381in}{0.739656in}}%
\pgfpathlineto{\pgfqpoint{5.829528in}{0.739656in}}%
\pgfpathlineto{\pgfqpoint{5.828675in}{0.739656in}}%
\pgfpathlineto{\pgfqpoint{5.827822in}{0.739656in}}%
\pgfpathlineto{\pgfqpoint{5.826969in}{0.739656in}}%
\pgfpathlineto{\pgfqpoint{5.826116in}{0.739656in}}%
\pgfpathlineto{\pgfqpoint{5.825263in}{0.739656in}}%
\pgfpathlineto{\pgfqpoint{5.824411in}{0.739656in}}%
\pgfpathlineto{\pgfqpoint{5.823558in}{0.739656in}}%
\pgfpathlineto{\pgfqpoint{5.822705in}{0.739656in}}%
\pgfpathlineto{\pgfqpoint{5.821852in}{0.739656in}}%
\pgfpathlineto{\pgfqpoint{5.820999in}{0.739656in}}%
\pgfpathlineto{\pgfqpoint{5.820146in}{0.739656in}}%
\pgfpathlineto{\pgfqpoint{5.819293in}{0.739656in}}%
\pgfpathlineto{\pgfqpoint{5.818440in}{0.739656in}}%
\pgfpathlineto{\pgfqpoint{5.817588in}{0.739656in}}%
\pgfpathlineto{\pgfqpoint{5.816735in}{0.739656in}}%
\pgfpathlineto{\pgfqpoint{5.815882in}{0.739656in}}%
\pgfpathlineto{\pgfqpoint{5.815029in}{0.739656in}}%
\pgfpathlineto{\pgfqpoint{5.814176in}{0.739656in}}%
\pgfpathlineto{\pgfqpoint{5.813323in}{0.739656in}}%
\pgfpathlineto{\pgfqpoint{5.812470in}{0.739656in}}%
\pgfpathlineto{\pgfqpoint{5.811617in}{0.739656in}}%
\pgfpathlineto{\pgfqpoint{5.810765in}{0.739656in}}%
\pgfpathlineto{\pgfqpoint{5.809912in}{0.739656in}}%
\pgfpathlineto{\pgfqpoint{5.809059in}{0.739656in}}%
\pgfpathlineto{\pgfqpoint{5.808206in}{0.739656in}}%
\pgfpathlineto{\pgfqpoint{5.807353in}{0.739656in}}%
\pgfpathlineto{\pgfqpoint{5.806500in}{0.739656in}}%
\pgfpathlineto{\pgfqpoint{5.805647in}{0.739656in}}%
\pgfpathlineto{\pgfqpoint{5.804794in}{0.739656in}}%
\pgfpathlineto{\pgfqpoint{5.803941in}{0.739656in}}%
\pgfpathlineto{\pgfqpoint{5.803089in}{0.739656in}}%
\pgfpathlineto{\pgfqpoint{5.802236in}{0.739656in}}%
\pgfpathlineto{\pgfqpoint{5.801383in}{0.739656in}}%
\pgfpathlineto{\pgfqpoint{5.800530in}{0.739656in}}%
\pgfpathlineto{\pgfqpoint{5.799677in}{0.739656in}}%
\pgfpathlineto{\pgfqpoint{5.798824in}{0.739656in}}%
\pgfpathlineto{\pgfqpoint{5.797971in}{0.739656in}}%
\pgfpathlineto{\pgfqpoint{5.797118in}{0.739656in}}%
\pgfpathlineto{\pgfqpoint{5.796266in}{0.739656in}}%
\pgfpathlineto{\pgfqpoint{5.795413in}{0.739656in}}%
\pgfpathlineto{\pgfqpoint{5.794560in}{0.739656in}}%
\pgfpathlineto{\pgfqpoint{5.793707in}{0.739656in}}%
\pgfpathlineto{\pgfqpoint{5.792854in}{0.739656in}}%
\pgfpathlineto{\pgfqpoint{5.792001in}{0.739656in}}%
\pgfpathlineto{\pgfqpoint{5.791148in}{0.739656in}}%
\pgfpathlineto{\pgfqpoint{5.790295in}{0.739656in}}%
\pgfpathlineto{\pgfqpoint{5.789443in}{0.739656in}}%
\pgfpathlineto{\pgfqpoint{5.788590in}{0.739656in}}%
\pgfpathlineto{\pgfqpoint{5.787737in}{0.739656in}}%
\pgfpathlineto{\pgfqpoint{5.786884in}{0.739656in}}%
\pgfpathlineto{\pgfqpoint{5.786031in}{0.739656in}}%
\pgfpathlineto{\pgfqpoint{5.785178in}{0.739656in}}%
\pgfpathlineto{\pgfqpoint{5.784325in}{0.739656in}}%
\pgfpathlineto{\pgfqpoint{5.783472in}{0.739656in}}%
\pgfpathlineto{\pgfqpoint{5.782620in}{0.739656in}}%
\pgfpathlineto{\pgfqpoint{5.781767in}{0.739656in}}%
\pgfpathlineto{\pgfqpoint{5.780914in}{0.739656in}}%
\pgfpathlineto{\pgfqpoint{5.780061in}{0.739656in}}%
\pgfpathlineto{\pgfqpoint{5.779208in}{0.739656in}}%
\pgfpathlineto{\pgfqpoint{5.778355in}{0.739656in}}%
\pgfpathlineto{\pgfqpoint{5.777502in}{0.739656in}}%
\pgfpathlineto{\pgfqpoint{5.776649in}{0.739656in}}%
\pgfpathlineto{\pgfqpoint{5.775797in}{0.739656in}}%
\pgfpathlineto{\pgfqpoint{5.774944in}{0.739656in}}%
\pgfpathlineto{\pgfqpoint{5.774091in}{0.739656in}}%
\pgfpathlineto{\pgfqpoint{5.773238in}{0.739656in}}%
\pgfpathlineto{\pgfqpoint{5.772385in}{0.739656in}}%
\pgfpathlineto{\pgfqpoint{5.771532in}{0.739656in}}%
\pgfpathlineto{\pgfqpoint{5.770679in}{0.739656in}}%
\pgfpathlineto{\pgfqpoint{5.769826in}{0.739656in}}%
\pgfpathlineto{\pgfqpoint{5.768973in}{0.739656in}}%
\pgfpathlineto{\pgfqpoint{5.768121in}{0.739656in}}%
\pgfpathlineto{\pgfqpoint{5.767268in}{0.739656in}}%
\pgfpathlineto{\pgfqpoint{5.766415in}{0.739656in}}%
\pgfpathlineto{\pgfqpoint{5.765562in}{0.739656in}}%
\pgfpathlineto{\pgfqpoint{5.764709in}{0.739656in}}%
\pgfpathlineto{\pgfqpoint{5.763856in}{0.739656in}}%
\pgfpathlineto{\pgfqpoint{5.763003in}{0.739656in}}%
\pgfpathlineto{\pgfqpoint{5.762150in}{0.739656in}}%
\pgfpathlineto{\pgfqpoint{5.761298in}{0.739656in}}%
\pgfpathlineto{\pgfqpoint{5.760445in}{0.739656in}}%
\pgfpathlineto{\pgfqpoint{5.759592in}{0.739656in}}%
\pgfpathlineto{\pgfqpoint{5.758739in}{0.739656in}}%
\pgfpathlineto{\pgfqpoint{5.757886in}{0.739656in}}%
\pgfpathlineto{\pgfqpoint{5.757033in}{0.739656in}}%
\pgfpathlineto{\pgfqpoint{5.756180in}{0.739656in}}%
\pgfpathlineto{\pgfqpoint{5.755327in}{0.739656in}}%
\pgfpathlineto{\pgfqpoint{5.754475in}{0.739656in}}%
\pgfpathlineto{\pgfqpoint{5.753622in}{0.739656in}}%
\pgfpathlineto{\pgfqpoint{5.752769in}{0.739656in}}%
\pgfpathlineto{\pgfqpoint{5.751916in}{0.739656in}}%
\pgfpathlineto{\pgfqpoint{5.751063in}{0.739656in}}%
\pgfpathlineto{\pgfqpoint{5.750210in}{0.739656in}}%
\pgfpathlineto{\pgfqpoint{5.749357in}{0.739656in}}%
\pgfpathlineto{\pgfqpoint{5.748504in}{0.739656in}}%
\pgfpathlineto{\pgfqpoint{5.747652in}{0.739656in}}%
\pgfpathlineto{\pgfqpoint{5.746799in}{0.739656in}}%
\pgfpathlineto{\pgfqpoint{5.745946in}{0.739656in}}%
\pgfpathlineto{\pgfqpoint{5.745093in}{0.739656in}}%
\pgfpathlineto{\pgfqpoint{5.744240in}{0.739656in}}%
\pgfpathlineto{\pgfqpoint{5.743387in}{0.739656in}}%
\pgfpathlineto{\pgfqpoint{5.742534in}{0.739656in}}%
\pgfpathlineto{\pgfqpoint{5.741681in}{0.739656in}}%
\pgfpathlineto{\pgfqpoint{5.740828in}{0.739656in}}%
\pgfpathlineto{\pgfqpoint{5.739976in}{0.739656in}}%
\pgfpathlineto{\pgfqpoint{5.739123in}{0.739656in}}%
\pgfpathlineto{\pgfqpoint{5.738270in}{0.739656in}}%
\pgfpathlineto{\pgfqpoint{5.737417in}{0.739656in}}%
\pgfpathlineto{\pgfqpoint{5.736564in}{0.739656in}}%
\pgfpathlineto{\pgfqpoint{5.735711in}{0.739656in}}%
\pgfpathlineto{\pgfqpoint{5.734858in}{0.739656in}}%
\pgfpathlineto{\pgfqpoint{5.734005in}{0.739656in}}%
\pgfpathlineto{\pgfqpoint{5.733153in}{0.739656in}}%
\pgfpathlineto{\pgfqpoint{5.732300in}{0.739656in}}%
\pgfpathlineto{\pgfqpoint{5.731447in}{0.739656in}}%
\pgfpathlineto{\pgfqpoint{5.730594in}{0.739656in}}%
\pgfpathlineto{\pgfqpoint{5.729741in}{0.739656in}}%
\pgfpathlineto{\pgfqpoint{5.728888in}{0.739656in}}%
\pgfpathlineto{\pgfqpoint{5.728035in}{0.739656in}}%
\pgfpathlineto{\pgfqpoint{5.727182in}{0.739656in}}%
\pgfpathlineto{\pgfqpoint{5.726330in}{0.739656in}}%
\pgfpathlineto{\pgfqpoint{5.725477in}{0.739656in}}%
\pgfpathlineto{\pgfqpoint{5.724624in}{0.739656in}}%
\pgfpathlineto{\pgfqpoint{5.723771in}{0.739656in}}%
\pgfpathlineto{\pgfqpoint{5.722918in}{0.739656in}}%
\pgfpathlineto{\pgfqpoint{5.722065in}{0.739656in}}%
\pgfpathlineto{\pgfqpoint{5.721212in}{0.739656in}}%
\pgfpathlineto{\pgfqpoint{5.720359in}{0.739656in}}%
\pgfpathlineto{\pgfqpoint{5.719507in}{0.739656in}}%
\pgfpathlineto{\pgfqpoint{5.718654in}{0.739656in}}%
\pgfpathlineto{\pgfqpoint{5.717801in}{0.739656in}}%
\pgfpathlineto{\pgfqpoint{5.716948in}{0.739656in}}%
\pgfpathlineto{\pgfqpoint{5.716095in}{0.739656in}}%
\pgfpathlineto{\pgfqpoint{5.715242in}{0.739656in}}%
\pgfpathlineto{\pgfqpoint{5.714389in}{0.739656in}}%
\pgfpathlineto{\pgfqpoint{5.713536in}{0.739656in}}%
\pgfpathlineto{\pgfqpoint{5.712683in}{0.739656in}}%
\pgfpathlineto{\pgfqpoint{5.711831in}{0.739656in}}%
\pgfpathlineto{\pgfqpoint{5.710978in}{0.739656in}}%
\pgfpathlineto{\pgfqpoint{5.710125in}{0.739656in}}%
\pgfpathlineto{\pgfqpoint{5.709272in}{0.739656in}}%
\pgfpathlineto{\pgfqpoint{5.708419in}{0.739656in}}%
\pgfpathlineto{\pgfqpoint{5.707566in}{0.739656in}}%
\pgfpathlineto{\pgfqpoint{5.706713in}{0.739656in}}%
\pgfpathlineto{\pgfqpoint{5.705860in}{0.739656in}}%
\pgfpathlineto{\pgfqpoint{5.705008in}{0.739656in}}%
\pgfpathlineto{\pgfqpoint{5.704155in}{0.739656in}}%
\pgfpathlineto{\pgfqpoint{5.703302in}{0.739656in}}%
\pgfpathlineto{\pgfqpoint{5.702449in}{0.739656in}}%
\pgfpathlineto{\pgfqpoint{5.701596in}{0.739656in}}%
\pgfpathlineto{\pgfqpoint{5.700743in}{0.739656in}}%
\pgfpathlineto{\pgfqpoint{5.699890in}{0.739656in}}%
\pgfpathlineto{\pgfqpoint{5.699037in}{0.739656in}}%
\pgfpathlineto{\pgfqpoint{5.698185in}{0.739656in}}%
\pgfpathlineto{\pgfqpoint{5.697332in}{0.739656in}}%
\pgfpathlineto{\pgfqpoint{5.696479in}{0.739656in}}%
\pgfpathlineto{\pgfqpoint{5.695626in}{0.739656in}}%
\pgfpathlineto{\pgfqpoint{5.694773in}{0.739656in}}%
\pgfpathlineto{\pgfqpoint{5.693920in}{0.739656in}}%
\pgfpathlineto{\pgfqpoint{5.693067in}{0.739656in}}%
\pgfpathlineto{\pgfqpoint{5.692214in}{0.739656in}}%
\pgfpathlineto{\pgfqpoint{5.691362in}{0.739656in}}%
\pgfpathlineto{\pgfqpoint{5.690509in}{0.739656in}}%
\pgfpathlineto{\pgfqpoint{5.689656in}{0.739656in}}%
\pgfpathlineto{\pgfqpoint{5.688803in}{0.739656in}}%
\pgfpathlineto{\pgfqpoint{5.687950in}{0.739656in}}%
\pgfpathlineto{\pgfqpoint{5.687097in}{0.739656in}}%
\pgfpathlineto{\pgfqpoint{5.686244in}{0.739656in}}%
\pgfpathlineto{\pgfqpoint{5.685391in}{0.739656in}}%
\pgfpathlineto{\pgfqpoint{5.684538in}{0.739656in}}%
\pgfpathlineto{\pgfqpoint{5.683686in}{0.739656in}}%
\pgfpathlineto{\pgfqpoint{5.682833in}{0.739656in}}%
\pgfpathlineto{\pgfqpoint{5.681980in}{0.739656in}}%
\pgfpathlineto{\pgfqpoint{5.681127in}{0.739656in}}%
\pgfpathlineto{\pgfqpoint{5.680274in}{0.739656in}}%
\pgfpathlineto{\pgfqpoint{5.679421in}{0.739656in}}%
\pgfpathlineto{\pgfqpoint{5.678568in}{0.739656in}}%
\pgfpathlineto{\pgfqpoint{5.677715in}{0.739656in}}%
\pgfpathlineto{\pgfqpoint{5.676863in}{0.739656in}}%
\pgfpathlineto{\pgfqpoint{5.676010in}{0.739656in}}%
\pgfpathlineto{\pgfqpoint{5.675157in}{0.739656in}}%
\pgfpathlineto{\pgfqpoint{5.674304in}{0.739656in}}%
\pgfpathlineto{\pgfqpoint{5.673451in}{0.739656in}}%
\pgfpathlineto{\pgfqpoint{5.672598in}{0.739656in}}%
\pgfpathlineto{\pgfqpoint{5.671745in}{0.739656in}}%
\pgfpathlineto{\pgfqpoint{5.670892in}{0.739656in}}%
\pgfpathlineto{\pgfqpoint{5.670040in}{0.739656in}}%
\pgfpathlineto{\pgfqpoint{5.669187in}{0.739656in}}%
\pgfpathlineto{\pgfqpoint{5.668334in}{0.739656in}}%
\pgfpathlineto{\pgfqpoint{5.667481in}{0.739656in}}%
\pgfpathlineto{\pgfqpoint{5.666628in}{0.739656in}}%
\pgfpathlineto{\pgfqpoint{5.665775in}{0.739656in}}%
\pgfpathlineto{\pgfqpoint{5.664922in}{0.739656in}}%
\pgfpathlineto{\pgfqpoint{5.664069in}{0.739656in}}%
\pgfpathlineto{\pgfqpoint{5.663217in}{0.739656in}}%
\pgfpathlineto{\pgfqpoint{5.662364in}{0.739656in}}%
\pgfpathlineto{\pgfqpoint{5.661511in}{0.739656in}}%
\pgfpathlineto{\pgfqpoint{5.660658in}{0.739656in}}%
\pgfpathlineto{\pgfqpoint{5.659805in}{0.739656in}}%
\pgfpathlineto{\pgfqpoint{5.658952in}{0.739656in}}%
\pgfpathlineto{\pgfqpoint{5.658099in}{0.739656in}}%
\pgfpathlineto{\pgfqpoint{5.657246in}{0.739656in}}%
\pgfpathlineto{\pgfqpoint{5.656394in}{0.739656in}}%
\pgfpathlineto{\pgfqpoint{5.655541in}{0.739656in}}%
\pgfpathlineto{\pgfqpoint{5.654688in}{0.739656in}}%
\pgfpathlineto{\pgfqpoint{5.653835in}{0.739656in}}%
\pgfpathlineto{\pgfqpoint{5.652982in}{0.739656in}}%
\pgfpathlineto{\pgfqpoint{5.652129in}{0.739656in}}%
\pgfpathlineto{\pgfqpoint{5.651276in}{0.739656in}}%
\pgfpathlineto{\pgfqpoint{5.650423in}{0.739656in}}%
\pgfpathlineto{\pgfqpoint{5.649570in}{0.739656in}}%
\pgfpathlineto{\pgfqpoint{5.648718in}{0.739656in}}%
\pgfpathlineto{\pgfqpoint{5.647865in}{0.739656in}}%
\pgfpathlineto{\pgfqpoint{5.647012in}{0.739656in}}%
\pgfpathlineto{\pgfqpoint{5.646159in}{0.739656in}}%
\pgfpathlineto{\pgfqpoint{5.645306in}{0.739656in}}%
\pgfpathlineto{\pgfqpoint{5.644453in}{0.739656in}}%
\pgfpathlineto{\pgfqpoint{5.643600in}{0.739656in}}%
\pgfpathlineto{\pgfqpoint{5.642747in}{0.739656in}}%
\pgfpathlineto{\pgfqpoint{5.641895in}{0.739656in}}%
\pgfpathlineto{\pgfqpoint{5.641042in}{0.739656in}}%
\pgfpathlineto{\pgfqpoint{5.640189in}{0.739656in}}%
\pgfpathlineto{\pgfqpoint{5.639336in}{0.739656in}}%
\pgfpathlineto{\pgfqpoint{5.638483in}{0.739656in}}%
\pgfpathlineto{\pgfqpoint{5.637630in}{0.739656in}}%
\pgfpathlineto{\pgfqpoint{5.636777in}{0.739656in}}%
\pgfpathlineto{\pgfqpoint{5.635924in}{0.739656in}}%
\pgfpathlineto{\pgfqpoint{5.635072in}{0.739656in}}%
\pgfpathlineto{\pgfqpoint{5.634219in}{0.739656in}}%
\pgfpathlineto{\pgfqpoint{5.633366in}{0.739656in}}%
\pgfpathlineto{\pgfqpoint{5.632513in}{0.739656in}}%
\pgfpathlineto{\pgfqpoint{5.631660in}{0.739656in}}%
\pgfpathlineto{\pgfqpoint{5.630807in}{0.739656in}}%
\pgfpathlineto{\pgfqpoint{5.629954in}{0.739656in}}%
\pgfpathlineto{\pgfqpoint{5.629101in}{0.739656in}}%
\pgfpathlineto{\pgfqpoint{5.628249in}{0.739656in}}%
\pgfpathlineto{\pgfqpoint{5.627396in}{0.739656in}}%
\pgfpathlineto{\pgfqpoint{5.626543in}{0.739656in}}%
\pgfpathlineto{\pgfqpoint{5.625690in}{0.739656in}}%
\pgfpathlineto{\pgfqpoint{5.624837in}{0.739656in}}%
\pgfpathlineto{\pgfqpoint{5.623984in}{0.739656in}}%
\pgfpathlineto{\pgfqpoint{5.623131in}{0.739656in}}%
\pgfpathlineto{\pgfqpoint{5.622278in}{0.739656in}}%
\pgfpathlineto{\pgfqpoint{5.621425in}{0.739656in}}%
\pgfpathlineto{\pgfqpoint{5.620573in}{0.739656in}}%
\pgfpathlineto{\pgfqpoint{5.619720in}{0.739656in}}%
\pgfpathlineto{\pgfqpoint{5.618867in}{0.739656in}}%
\pgfpathlineto{\pgfqpoint{5.618014in}{0.739656in}}%
\pgfpathlineto{\pgfqpoint{5.617161in}{0.739656in}}%
\pgfpathlineto{\pgfqpoint{5.616308in}{0.739656in}}%
\pgfpathlineto{\pgfqpoint{5.615455in}{0.739656in}}%
\pgfpathlineto{\pgfqpoint{5.614602in}{0.739656in}}%
\pgfpathlineto{\pgfqpoint{5.613750in}{0.739656in}}%
\pgfpathlineto{\pgfqpoint{5.612897in}{0.739656in}}%
\pgfpathlineto{\pgfqpoint{5.612044in}{0.739656in}}%
\pgfpathlineto{\pgfqpoint{5.611191in}{0.739656in}}%
\pgfpathlineto{\pgfqpoint{5.610338in}{0.739656in}}%
\pgfpathlineto{\pgfqpoint{5.609485in}{0.739656in}}%
\pgfpathlineto{\pgfqpoint{5.608632in}{0.739656in}}%
\pgfpathlineto{\pgfqpoint{5.607779in}{0.739656in}}%
\pgfpathlineto{\pgfqpoint{5.606927in}{0.739656in}}%
\pgfpathlineto{\pgfqpoint{5.606074in}{0.739656in}}%
\pgfpathlineto{\pgfqpoint{5.605221in}{0.739656in}}%
\pgfpathlineto{\pgfqpoint{5.604368in}{0.739656in}}%
\pgfpathlineto{\pgfqpoint{5.603515in}{0.739656in}}%
\pgfpathlineto{\pgfqpoint{5.602662in}{0.739656in}}%
\pgfpathlineto{\pgfqpoint{5.601809in}{0.739656in}}%
\pgfpathlineto{\pgfqpoint{5.600956in}{0.739656in}}%
\pgfpathlineto{\pgfqpoint{5.600104in}{0.739656in}}%
\pgfpathlineto{\pgfqpoint{5.599251in}{0.739656in}}%
\pgfpathlineto{\pgfqpoint{5.598398in}{0.739656in}}%
\pgfpathlineto{\pgfqpoint{5.597545in}{0.739656in}}%
\pgfpathlineto{\pgfqpoint{5.596692in}{0.739656in}}%
\pgfpathlineto{\pgfqpoint{5.595839in}{0.739656in}}%
\pgfpathlineto{\pgfqpoint{5.594986in}{0.739656in}}%
\pgfpathlineto{\pgfqpoint{5.594133in}{0.739656in}}%
\pgfpathlineto{\pgfqpoint{5.593280in}{0.739656in}}%
\pgfpathlineto{\pgfqpoint{5.592428in}{0.739656in}}%
\pgfpathlineto{\pgfqpoint{5.591575in}{0.739656in}}%
\pgfpathlineto{\pgfqpoint{5.590722in}{0.739656in}}%
\pgfpathlineto{\pgfqpoint{5.589869in}{0.739656in}}%
\pgfpathlineto{\pgfqpoint{5.589016in}{0.739656in}}%
\pgfpathlineto{\pgfqpoint{5.588163in}{0.739656in}}%
\pgfpathlineto{\pgfqpoint{5.587310in}{0.739656in}}%
\pgfpathlineto{\pgfqpoint{5.586457in}{0.739656in}}%
\pgfpathlineto{\pgfqpoint{5.585605in}{0.739656in}}%
\pgfpathlineto{\pgfqpoint{5.584752in}{0.739656in}}%
\pgfpathlineto{\pgfqpoint{5.583899in}{0.739656in}}%
\pgfpathlineto{\pgfqpoint{5.583046in}{0.739656in}}%
\pgfpathlineto{\pgfqpoint{5.582193in}{0.739656in}}%
\pgfpathlineto{\pgfqpoint{5.581340in}{0.739656in}}%
\pgfpathlineto{\pgfqpoint{5.580487in}{0.739656in}}%
\pgfpathlineto{\pgfqpoint{5.579634in}{0.739656in}}%
\pgfpathlineto{\pgfqpoint{5.578782in}{0.739656in}}%
\pgfpathlineto{\pgfqpoint{5.577929in}{0.739656in}}%
\pgfpathlineto{\pgfqpoint{5.577076in}{0.739656in}}%
\pgfpathlineto{\pgfqpoint{5.576223in}{0.739656in}}%
\pgfpathlineto{\pgfqpoint{5.575370in}{0.739656in}}%
\pgfpathlineto{\pgfqpoint{5.574517in}{0.739656in}}%
\pgfpathlineto{\pgfqpoint{5.573664in}{0.739656in}}%
\pgfpathlineto{\pgfqpoint{5.572811in}{0.739656in}}%
\pgfpathlineto{\pgfqpoint{5.571959in}{0.739656in}}%
\pgfpathlineto{\pgfqpoint{5.571106in}{0.739656in}}%
\pgfpathlineto{\pgfqpoint{5.570253in}{0.739656in}}%
\pgfpathlineto{\pgfqpoint{5.569400in}{0.739656in}}%
\pgfpathlineto{\pgfqpoint{5.568547in}{0.739656in}}%
\pgfpathlineto{\pgfqpoint{5.567694in}{0.739656in}}%
\pgfpathlineto{\pgfqpoint{5.566841in}{0.739656in}}%
\pgfpathlineto{\pgfqpoint{5.565988in}{0.739656in}}%
\pgfpathlineto{\pgfqpoint{5.565136in}{0.739656in}}%
\pgfpathlineto{\pgfqpoint{5.564283in}{0.739656in}}%
\pgfpathlineto{\pgfqpoint{5.563430in}{0.739656in}}%
\pgfpathlineto{\pgfqpoint{5.562577in}{0.739656in}}%
\pgfpathlineto{\pgfqpoint{5.561724in}{0.739656in}}%
\pgfpathlineto{\pgfqpoint{5.560871in}{0.739656in}}%
\pgfpathlineto{\pgfqpoint{5.560018in}{0.739656in}}%
\pgfpathlineto{\pgfqpoint{5.559165in}{0.739656in}}%
\pgfpathlineto{\pgfqpoint{5.558312in}{0.739656in}}%
\pgfpathlineto{\pgfqpoint{5.557460in}{0.739656in}}%
\pgfpathlineto{\pgfqpoint{5.556607in}{0.739656in}}%
\pgfpathlineto{\pgfqpoint{5.555754in}{0.739656in}}%
\pgfpathlineto{\pgfqpoint{5.554901in}{0.739656in}}%
\pgfpathlineto{\pgfqpoint{5.554048in}{0.739656in}}%
\pgfpathlineto{\pgfqpoint{5.553195in}{0.739656in}}%
\pgfpathlineto{\pgfqpoint{5.552342in}{0.739656in}}%
\pgfpathlineto{\pgfqpoint{5.551489in}{0.739656in}}%
\pgfpathlineto{\pgfqpoint{5.550637in}{0.739656in}}%
\pgfpathlineto{\pgfqpoint{5.549784in}{0.739656in}}%
\pgfpathlineto{\pgfqpoint{5.548931in}{0.739656in}}%
\pgfpathlineto{\pgfqpoint{5.548078in}{0.739656in}}%
\pgfpathlineto{\pgfqpoint{5.547225in}{0.739656in}}%
\pgfpathlineto{\pgfqpoint{5.546372in}{0.739656in}}%
\pgfpathlineto{\pgfqpoint{5.545519in}{0.739656in}}%
\pgfpathlineto{\pgfqpoint{5.544666in}{0.739656in}}%
\pgfpathlineto{\pgfqpoint{5.543814in}{0.739656in}}%
\pgfpathlineto{\pgfqpoint{5.542961in}{0.739656in}}%
\pgfpathlineto{\pgfqpoint{5.542108in}{0.739656in}}%
\pgfpathlineto{\pgfqpoint{5.541255in}{0.739656in}}%
\pgfpathlineto{\pgfqpoint{5.540402in}{0.739656in}}%
\pgfpathlineto{\pgfqpoint{5.539549in}{0.739656in}}%
\pgfpathlineto{\pgfqpoint{5.538696in}{0.739656in}}%
\pgfpathlineto{\pgfqpoint{5.537843in}{0.739656in}}%
\pgfpathlineto{\pgfqpoint{5.536991in}{0.739656in}}%
\pgfpathlineto{\pgfqpoint{5.536138in}{0.739656in}}%
\pgfpathlineto{\pgfqpoint{5.535285in}{0.739656in}}%
\pgfpathlineto{\pgfqpoint{5.534432in}{0.739656in}}%
\pgfpathlineto{\pgfqpoint{5.533579in}{0.739656in}}%
\pgfpathlineto{\pgfqpoint{5.532726in}{0.739656in}}%
\pgfpathlineto{\pgfqpoint{5.531873in}{0.739656in}}%
\pgfpathlineto{\pgfqpoint{5.531020in}{0.739656in}}%
\pgfpathlineto{\pgfqpoint{5.530167in}{0.739656in}}%
\pgfpathlineto{\pgfqpoint{5.529315in}{0.739656in}}%
\pgfpathlineto{\pgfqpoint{5.528462in}{0.739656in}}%
\pgfpathlineto{\pgfqpoint{5.527609in}{0.739656in}}%
\pgfpathlineto{\pgfqpoint{5.526756in}{0.739656in}}%
\pgfpathlineto{\pgfqpoint{5.525903in}{0.739656in}}%
\pgfpathlineto{\pgfqpoint{5.525050in}{0.739656in}}%
\pgfpathlineto{\pgfqpoint{5.524197in}{0.739656in}}%
\pgfpathlineto{\pgfqpoint{5.523344in}{0.739656in}}%
\pgfpathlineto{\pgfqpoint{5.522492in}{0.739656in}}%
\pgfpathlineto{\pgfqpoint{5.521639in}{0.739656in}}%
\pgfpathlineto{\pgfqpoint{5.520786in}{0.739656in}}%
\pgfpathlineto{\pgfqpoint{5.519933in}{0.739656in}}%
\pgfpathlineto{\pgfqpoint{5.519080in}{0.739656in}}%
\pgfpathlineto{\pgfqpoint{5.518227in}{0.739656in}}%
\pgfpathlineto{\pgfqpoint{5.517374in}{0.739656in}}%
\pgfpathlineto{\pgfqpoint{5.516521in}{0.739656in}}%
\pgfpathlineto{\pgfqpoint{5.515669in}{0.739656in}}%
\pgfpathlineto{\pgfqpoint{5.514816in}{0.739656in}}%
\pgfpathlineto{\pgfqpoint{5.513963in}{0.739656in}}%
\pgfpathlineto{\pgfqpoint{5.513110in}{0.739656in}}%
\pgfpathlineto{\pgfqpoint{5.512257in}{0.739656in}}%
\pgfpathlineto{\pgfqpoint{5.511404in}{0.739656in}}%
\pgfpathlineto{\pgfqpoint{5.510551in}{0.739656in}}%
\pgfpathlineto{\pgfqpoint{5.509698in}{0.739656in}}%
\pgfpathlineto{\pgfqpoint{5.508846in}{0.739656in}}%
\pgfpathlineto{\pgfqpoint{5.507993in}{0.739656in}}%
\pgfpathlineto{\pgfqpoint{5.507140in}{0.739656in}}%
\pgfpathlineto{\pgfqpoint{5.506287in}{0.739656in}}%
\pgfpathlineto{\pgfqpoint{5.505434in}{0.739656in}}%
\pgfpathlineto{\pgfqpoint{5.504581in}{0.739656in}}%
\pgfpathlineto{\pgfqpoint{5.503728in}{0.739656in}}%
\pgfpathlineto{\pgfqpoint{5.502875in}{0.739656in}}%
\pgfpathlineto{\pgfqpoint{5.502022in}{0.739656in}}%
\pgfpathlineto{\pgfqpoint{5.501170in}{0.739656in}}%
\pgfpathlineto{\pgfqpoint{5.500317in}{0.739656in}}%
\pgfpathlineto{\pgfqpoint{5.499464in}{0.739656in}}%
\pgfpathlineto{\pgfqpoint{5.498611in}{0.739656in}}%
\pgfpathlineto{\pgfqpoint{5.497758in}{0.739656in}}%
\pgfpathlineto{\pgfqpoint{5.496905in}{0.739656in}}%
\pgfpathlineto{\pgfqpoint{5.496052in}{0.739656in}}%
\pgfpathlineto{\pgfqpoint{5.495199in}{0.739656in}}%
\pgfpathlineto{\pgfqpoint{5.494347in}{0.739656in}}%
\pgfpathlineto{\pgfqpoint{5.493494in}{0.739656in}}%
\pgfpathlineto{\pgfqpoint{5.492641in}{0.739656in}}%
\pgfpathlineto{\pgfqpoint{5.491788in}{0.739656in}}%
\pgfpathlineto{\pgfqpoint{5.490935in}{0.739656in}}%
\pgfpathlineto{\pgfqpoint{5.490082in}{0.739656in}}%
\pgfpathlineto{\pgfqpoint{5.489229in}{0.739656in}}%
\pgfpathlineto{\pgfqpoint{5.488376in}{0.739656in}}%
\pgfpathlineto{\pgfqpoint{5.487524in}{0.739656in}}%
\pgfpathlineto{\pgfqpoint{5.486671in}{0.739656in}}%
\pgfpathlineto{\pgfqpoint{5.485818in}{0.739656in}}%
\pgfpathlineto{\pgfqpoint{5.484965in}{0.739656in}}%
\pgfpathlineto{\pgfqpoint{5.484112in}{0.739656in}}%
\pgfpathlineto{\pgfqpoint{5.483259in}{0.739656in}}%
\pgfpathlineto{\pgfqpoint{5.482406in}{0.739656in}}%
\pgfpathlineto{\pgfqpoint{5.481553in}{0.739656in}}%
\pgfpathlineto{\pgfqpoint{5.480701in}{0.739656in}}%
\pgfpathlineto{\pgfqpoint{5.479848in}{0.739656in}}%
\pgfpathlineto{\pgfqpoint{5.478995in}{0.739656in}}%
\pgfpathlineto{\pgfqpoint{5.478142in}{0.739656in}}%
\pgfpathlineto{\pgfqpoint{5.477289in}{0.739656in}}%
\pgfpathlineto{\pgfqpoint{5.476436in}{0.739656in}}%
\pgfpathlineto{\pgfqpoint{5.475583in}{0.739656in}}%
\pgfpathlineto{\pgfqpoint{5.474730in}{0.739656in}}%
\pgfpathlineto{\pgfqpoint{5.473877in}{0.739656in}}%
\pgfpathlineto{\pgfqpoint{5.473025in}{0.739656in}}%
\pgfpathlineto{\pgfqpoint{5.472172in}{0.739656in}}%
\pgfpathlineto{\pgfqpoint{5.471319in}{0.739656in}}%
\pgfpathlineto{\pgfqpoint{5.470466in}{0.739656in}}%
\pgfpathlineto{\pgfqpoint{5.469613in}{0.739656in}}%
\pgfpathlineto{\pgfqpoint{5.468760in}{0.739656in}}%
\pgfpathlineto{\pgfqpoint{5.467907in}{0.739656in}}%
\pgfpathlineto{\pgfqpoint{5.467054in}{0.739656in}}%
\pgfpathlineto{\pgfqpoint{5.466202in}{0.739656in}}%
\pgfpathlineto{\pgfqpoint{5.465349in}{0.739656in}}%
\pgfpathlineto{\pgfqpoint{5.464496in}{0.739656in}}%
\pgfpathlineto{\pgfqpoint{5.463643in}{0.739656in}}%
\pgfpathlineto{\pgfqpoint{5.462790in}{0.739656in}}%
\pgfpathlineto{\pgfqpoint{5.461937in}{0.739656in}}%
\pgfpathlineto{\pgfqpoint{5.461084in}{0.739656in}}%
\pgfpathlineto{\pgfqpoint{5.460231in}{0.739656in}}%
\pgfpathlineto{\pgfqpoint{5.459379in}{0.739656in}}%
\pgfpathlineto{\pgfqpoint{5.458526in}{0.739656in}}%
\pgfpathlineto{\pgfqpoint{5.457673in}{0.739656in}}%
\pgfpathlineto{\pgfqpoint{5.456820in}{0.739656in}}%
\pgfpathlineto{\pgfqpoint{5.455967in}{0.739656in}}%
\pgfpathlineto{\pgfqpoint{5.455114in}{0.739656in}}%
\pgfpathlineto{\pgfqpoint{5.454261in}{0.739656in}}%
\pgfpathlineto{\pgfqpoint{5.453408in}{0.739656in}}%
\pgfpathlineto{\pgfqpoint{5.452556in}{0.739656in}}%
\pgfpathlineto{\pgfqpoint{5.451703in}{0.739656in}}%
\pgfpathlineto{\pgfqpoint{5.450850in}{0.739656in}}%
\pgfpathlineto{\pgfqpoint{5.449997in}{0.739656in}}%
\pgfpathlineto{\pgfqpoint{5.449144in}{0.739656in}}%
\pgfpathlineto{\pgfqpoint{5.448291in}{0.739656in}}%
\pgfpathlineto{\pgfqpoint{5.447438in}{0.739656in}}%
\pgfpathlineto{\pgfqpoint{5.446585in}{0.739656in}}%
\pgfpathlineto{\pgfqpoint{5.445733in}{0.739656in}}%
\pgfpathlineto{\pgfqpoint{5.444880in}{0.739656in}}%
\pgfpathlineto{\pgfqpoint{5.444027in}{0.739656in}}%
\pgfpathlineto{\pgfqpoint{5.443174in}{0.739656in}}%
\pgfpathlineto{\pgfqpoint{5.442321in}{0.739656in}}%
\pgfpathlineto{\pgfqpoint{5.441468in}{0.739656in}}%
\pgfpathlineto{\pgfqpoint{5.440615in}{0.739656in}}%
\pgfpathlineto{\pgfqpoint{5.439762in}{0.739656in}}%
\pgfpathlineto{\pgfqpoint{5.438909in}{0.739656in}}%
\pgfpathlineto{\pgfqpoint{5.438057in}{0.739656in}}%
\pgfpathlineto{\pgfqpoint{5.437204in}{0.739656in}}%
\pgfpathlineto{\pgfqpoint{5.436351in}{0.739656in}}%
\pgfpathlineto{\pgfqpoint{5.435498in}{0.739656in}}%
\pgfpathlineto{\pgfqpoint{5.434645in}{0.739656in}}%
\pgfpathlineto{\pgfqpoint{5.433792in}{0.739656in}}%
\pgfpathlineto{\pgfqpoint{5.432939in}{0.739656in}}%
\pgfpathlineto{\pgfqpoint{5.432086in}{0.739656in}}%
\pgfpathlineto{\pgfqpoint{5.431234in}{0.739656in}}%
\pgfpathlineto{\pgfqpoint{5.430381in}{0.739656in}}%
\pgfpathlineto{\pgfqpoint{5.429528in}{0.739656in}}%
\pgfpathlineto{\pgfqpoint{5.428675in}{0.739656in}}%
\pgfpathlineto{\pgfqpoint{5.427822in}{0.739656in}}%
\pgfpathlineto{\pgfqpoint{5.426969in}{0.739656in}}%
\pgfpathlineto{\pgfqpoint{5.426116in}{0.739656in}}%
\pgfpathlineto{\pgfqpoint{5.425263in}{0.739656in}}%
\pgfpathlineto{\pgfqpoint{5.424411in}{0.739656in}}%
\pgfpathlineto{\pgfqpoint{5.423558in}{0.739656in}}%
\pgfpathlineto{\pgfqpoint{5.422705in}{0.739656in}}%
\pgfpathlineto{\pgfqpoint{5.421852in}{0.739656in}}%
\pgfpathlineto{\pgfqpoint{5.420999in}{0.739656in}}%
\pgfpathlineto{\pgfqpoint{5.420146in}{0.739656in}}%
\pgfpathlineto{\pgfqpoint{5.419293in}{0.739656in}}%
\pgfpathlineto{\pgfqpoint{5.418440in}{0.739656in}}%
\pgfpathlineto{\pgfqpoint{5.417588in}{0.739656in}}%
\pgfpathlineto{\pgfqpoint{5.416735in}{0.739656in}}%
\pgfpathlineto{\pgfqpoint{5.415882in}{0.739656in}}%
\pgfpathlineto{\pgfqpoint{5.415029in}{0.739656in}}%
\pgfpathlineto{\pgfqpoint{5.414176in}{0.739656in}}%
\pgfpathlineto{\pgfqpoint{5.413323in}{0.739656in}}%
\pgfpathlineto{\pgfqpoint{5.412470in}{0.739656in}}%
\pgfpathlineto{\pgfqpoint{5.411617in}{0.739656in}}%
\pgfpathlineto{\pgfqpoint{5.410764in}{0.739656in}}%
\pgfpathlineto{\pgfqpoint{5.409912in}{0.739656in}}%
\pgfpathlineto{\pgfqpoint{5.409059in}{0.739656in}}%
\pgfpathlineto{\pgfqpoint{5.408206in}{0.739656in}}%
\pgfpathlineto{\pgfqpoint{5.407353in}{0.739656in}}%
\pgfpathlineto{\pgfqpoint{5.406500in}{0.739656in}}%
\pgfpathlineto{\pgfqpoint{5.405647in}{0.739656in}}%
\pgfpathlineto{\pgfqpoint{5.404794in}{0.739656in}}%
\pgfpathlineto{\pgfqpoint{5.403941in}{0.739656in}}%
\pgfpathlineto{\pgfqpoint{5.403089in}{0.739656in}}%
\pgfpathlineto{\pgfqpoint{5.402236in}{0.739656in}}%
\pgfpathlineto{\pgfqpoint{5.401383in}{0.739656in}}%
\pgfpathlineto{\pgfqpoint{5.400530in}{0.739656in}}%
\pgfpathlineto{\pgfqpoint{5.399677in}{0.739656in}}%
\pgfpathlineto{\pgfqpoint{5.398824in}{0.739656in}}%
\pgfpathlineto{\pgfqpoint{5.397971in}{0.739656in}}%
\pgfpathlineto{\pgfqpoint{5.397118in}{0.739656in}}%
\pgfpathlineto{\pgfqpoint{5.396266in}{0.739656in}}%
\pgfpathlineto{\pgfqpoint{5.395413in}{0.739656in}}%
\pgfpathlineto{\pgfqpoint{5.394560in}{0.739656in}}%
\pgfpathlineto{\pgfqpoint{5.393707in}{0.739656in}}%
\pgfpathlineto{\pgfqpoint{5.392854in}{0.739656in}}%
\pgfpathlineto{\pgfqpoint{5.392001in}{0.739656in}}%
\pgfpathlineto{\pgfqpoint{5.391148in}{0.739656in}}%
\pgfpathlineto{\pgfqpoint{5.390295in}{0.739656in}}%
\pgfpathlineto{\pgfqpoint{5.389443in}{0.739656in}}%
\pgfpathlineto{\pgfqpoint{5.388590in}{0.739656in}}%
\pgfpathlineto{\pgfqpoint{5.387737in}{0.739656in}}%
\pgfpathlineto{\pgfqpoint{5.386884in}{0.739656in}}%
\pgfpathlineto{\pgfqpoint{5.386031in}{0.739656in}}%
\pgfpathlineto{\pgfqpoint{5.385178in}{0.739656in}}%
\pgfpathlineto{\pgfqpoint{5.384325in}{0.739656in}}%
\pgfpathlineto{\pgfqpoint{5.383472in}{0.739656in}}%
\pgfpathlineto{\pgfqpoint{5.382619in}{0.739656in}}%
\pgfpathlineto{\pgfqpoint{5.381767in}{0.739656in}}%
\pgfpathlineto{\pgfqpoint{5.380914in}{0.739656in}}%
\pgfpathlineto{\pgfqpoint{5.380061in}{0.739656in}}%
\pgfpathlineto{\pgfqpoint{5.379208in}{0.739656in}}%
\pgfpathlineto{\pgfqpoint{5.378355in}{0.739656in}}%
\pgfpathlineto{\pgfqpoint{5.377502in}{0.739656in}}%
\pgfpathlineto{\pgfqpoint{5.376649in}{0.739656in}}%
\pgfpathlineto{\pgfqpoint{5.375796in}{0.739656in}}%
\pgfpathlineto{\pgfqpoint{5.374944in}{0.739656in}}%
\pgfpathlineto{\pgfqpoint{5.374091in}{0.739656in}}%
\pgfpathlineto{\pgfqpoint{5.373238in}{0.739656in}}%
\pgfpathlineto{\pgfqpoint{5.372385in}{0.739656in}}%
\pgfpathlineto{\pgfqpoint{5.371532in}{0.739656in}}%
\pgfpathlineto{\pgfqpoint{5.370679in}{0.739656in}}%
\pgfpathlineto{\pgfqpoint{5.369826in}{0.739656in}}%
\pgfpathlineto{\pgfqpoint{5.368973in}{0.739656in}}%
\pgfpathlineto{\pgfqpoint{5.368121in}{0.739656in}}%
\pgfpathlineto{\pgfqpoint{5.367268in}{0.739656in}}%
\pgfpathlineto{\pgfqpoint{5.366415in}{0.739656in}}%
\pgfpathlineto{\pgfqpoint{5.365562in}{0.739656in}}%
\pgfpathlineto{\pgfqpoint{5.364709in}{0.739656in}}%
\pgfpathlineto{\pgfqpoint{5.363856in}{0.739656in}}%
\pgfpathlineto{\pgfqpoint{5.363003in}{0.739656in}}%
\pgfpathlineto{\pgfqpoint{5.362150in}{0.739656in}}%
\pgfpathlineto{\pgfqpoint{5.361298in}{0.739656in}}%
\pgfpathlineto{\pgfqpoint{5.360445in}{0.739656in}}%
\pgfpathlineto{\pgfqpoint{5.359592in}{0.739656in}}%
\pgfpathlineto{\pgfqpoint{5.358739in}{0.739656in}}%
\pgfpathlineto{\pgfqpoint{5.357886in}{0.739656in}}%
\pgfpathlineto{\pgfqpoint{5.357033in}{0.739656in}}%
\pgfpathlineto{\pgfqpoint{5.356180in}{0.739656in}}%
\pgfpathlineto{\pgfqpoint{5.355327in}{0.739656in}}%
\pgfpathlineto{\pgfqpoint{5.354474in}{0.739656in}}%
\pgfpathlineto{\pgfqpoint{5.353622in}{0.739656in}}%
\pgfpathlineto{\pgfqpoint{5.352769in}{0.739656in}}%
\pgfpathlineto{\pgfqpoint{5.351916in}{0.739656in}}%
\pgfpathlineto{\pgfqpoint{5.351063in}{0.739656in}}%
\pgfpathlineto{\pgfqpoint{5.350210in}{0.739656in}}%
\pgfpathlineto{\pgfqpoint{5.349357in}{0.739656in}}%
\pgfpathlineto{\pgfqpoint{5.348504in}{0.739656in}}%
\pgfpathlineto{\pgfqpoint{5.347651in}{0.739656in}}%
\pgfpathlineto{\pgfqpoint{5.346799in}{0.739656in}}%
\pgfpathlineto{\pgfqpoint{5.345946in}{0.739656in}}%
\pgfpathlineto{\pgfqpoint{5.345093in}{0.739656in}}%
\pgfpathlineto{\pgfqpoint{5.344240in}{0.739656in}}%
\pgfpathlineto{\pgfqpoint{5.343387in}{0.739656in}}%
\pgfpathlineto{\pgfqpoint{5.342534in}{0.739656in}}%
\pgfpathlineto{\pgfqpoint{5.341681in}{0.739656in}}%
\pgfpathlineto{\pgfqpoint{5.340828in}{0.739656in}}%
\pgfpathlineto{\pgfqpoint{5.339976in}{0.739656in}}%
\pgfpathlineto{\pgfqpoint{5.339123in}{0.739656in}}%
\pgfpathlineto{\pgfqpoint{5.338270in}{0.739656in}}%
\pgfpathlineto{\pgfqpoint{5.337417in}{0.739656in}}%
\pgfpathlineto{\pgfqpoint{5.336564in}{0.739656in}}%
\pgfpathlineto{\pgfqpoint{5.335711in}{0.739656in}}%
\pgfpathlineto{\pgfqpoint{5.334858in}{0.739656in}}%
\pgfpathlineto{\pgfqpoint{5.334005in}{0.739656in}}%
\pgfpathlineto{\pgfqpoint{5.333153in}{0.739656in}}%
\pgfpathlineto{\pgfqpoint{5.332300in}{0.739656in}}%
\pgfpathlineto{\pgfqpoint{5.331447in}{0.739656in}}%
\pgfpathlineto{\pgfqpoint{5.330594in}{0.739656in}}%
\pgfpathlineto{\pgfqpoint{5.329741in}{0.739656in}}%
\pgfpathlineto{\pgfqpoint{5.328888in}{0.739656in}}%
\pgfpathlineto{\pgfqpoint{5.328035in}{0.739656in}}%
\pgfpathlineto{\pgfqpoint{5.327182in}{0.739656in}}%
\pgfpathlineto{\pgfqpoint{5.326330in}{0.739656in}}%
\pgfpathlineto{\pgfqpoint{5.325477in}{0.739656in}}%
\pgfpathlineto{\pgfqpoint{5.324624in}{0.739656in}}%
\pgfpathlineto{\pgfqpoint{5.323771in}{0.739656in}}%
\pgfpathlineto{\pgfqpoint{5.322918in}{0.739656in}}%
\pgfpathlineto{\pgfqpoint{5.322065in}{0.739656in}}%
\pgfpathlineto{\pgfqpoint{5.321212in}{0.739656in}}%
\pgfpathlineto{\pgfqpoint{5.320359in}{0.739656in}}%
\pgfpathlineto{\pgfqpoint{5.319506in}{0.739656in}}%
\pgfpathlineto{\pgfqpoint{5.318654in}{0.739656in}}%
\pgfpathlineto{\pgfqpoint{5.317801in}{0.739656in}}%
\pgfpathlineto{\pgfqpoint{5.316948in}{0.739656in}}%
\pgfpathlineto{\pgfqpoint{5.316095in}{0.739656in}}%
\pgfpathlineto{\pgfqpoint{5.315242in}{0.739656in}}%
\pgfpathlineto{\pgfqpoint{5.314389in}{0.739656in}}%
\pgfpathlineto{\pgfqpoint{5.313536in}{0.739656in}}%
\pgfpathlineto{\pgfqpoint{5.312683in}{0.739656in}}%
\pgfpathlineto{\pgfqpoint{5.311831in}{0.739656in}}%
\pgfpathlineto{\pgfqpoint{5.310978in}{0.739656in}}%
\pgfpathlineto{\pgfqpoint{5.310125in}{0.739656in}}%
\pgfpathlineto{\pgfqpoint{5.309272in}{0.739656in}}%
\pgfpathlineto{\pgfqpoint{5.308419in}{0.739656in}}%
\pgfpathlineto{\pgfqpoint{5.307566in}{0.739656in}}%
\pgfpathlineto{\pgfqpoint{5.306713in}{0.739656in}}%
\pgfpathlineto{\pgfqpoint{5.305860in}{0.739656in}}%
\pgfpathlineto{\pgfqpoint{5.305008in}{0.739656in}}%
\pgfpathlineto{\pgfqpoint{5.304155in}{0.739656in}}%
\pgfpathlineto{\pgfqpoint{5.303302in}{0.739656in}}%
\pgfpathlineto{\pgfqpoint{5.302449in}{0.739656in}}%
\pgfpathlineto{\pgfqpoint{5.301596in}{0.739656in}}%
\pgfpathlineto{\pgfqpoint{5.300743in}{0.739656in}}%
\pgfpathlineto{\pgfqpoint{5.299890in}{0.739656in}}%
\pgfpathlineto{\pgfqpoint{5.299037in}{0.739656in}}%
\pgfpathlineto{\pgfqpoint{5.298185in}{0.739656in}}%
\pgfpathlineto{\pgfqpoint{5.297332in}{0.739656in}}%
\pgfpathlineto{\pgfqpoint{5.296479in}{0.739656in}}%
\pgfpathlineto{\pgfqpoint{5.295626in}{0.739656in}}%
\pgfpathlineto{\pgfqpoint{5.294773in}{0.739656in}}%
\pgfpathlineto{\pgfqpoint{5.293920in}{0.739656in}}%
\pgfpathlineto{\pgfqpoint{5.293067in}{0.739656in}}%
\pgfpathlineto{\pgfqpoint{5.292214in}{0.739656in}}%
\pgfpathlineto{\pgfqpoint{5.291361in}{0.739656in}}%
\pgfpathlineto{\pgfqpoint{5.290509in}{0.739656in}}%
\pgfpathlineto{\pgfqpoint{5.289656in}{0.739656in}}%
\pgfpathlineto{\pgfqpoint{5.288803in}{0.739656in}}%
\pgfpathlineto{\pgfqpoint{5.287950in}{0.739656in}}%
\pgfpathlineto{\pgfqpoint{5.287097in}{0.739656in}}%
\pgfpathlineto{\pgfqpoint{5.286244in}{0.739656in}}%
\pgfpathlineto{\pgfqpoint{5.285391in}{0.739656in}}%
\pgfpathlineto{\pgfqpoint{5.284538in}{0.739656in}}%
\pgfpathlineto{\pgfqpoint{5.283686in}{0.739656in}}%
\pgfpathlineto{\pgfqpoint{5.282833in}{0.739656in}}%
\pgfpathlineto{\pgfqpoint{5.281980in}{0.739656in}}%
\pgfpathlineto{\pgfqpoint{5.281127in}{0.739656in}}%
\pgfpathlineto{\pgfqpoint{5.280274in}{0.739656in}}%
\pgfpathlineto{\pgfqpoint{5.279421in}{0.739656in}}%
\pgfpathlineto{\pgfqpoint{5.278568in}{0.739656in}}%
\pgfpathlineto{\pgfqpoint{5.277715in}{0.739656in}}%
\pgfpathlineto{\pgfqpoint{5.276863in}{0.739656in}}%
\pgfpathlineto{\pgfqpoint{5.276010in}{0.739656in}}%
\pgfpathlineto{\pgfqpoint{5.275157in}{0.739656in}}%
\pgfpathlineto{\pgfqpoint{5.274304in}{0.739656in}}%
\pgfpathlineto{\pgfqpoint{5.273451in}{0.739656in}}%
\pgfpathlineto{\pgfqpoint{5.272598in}{0.739656in}}%
\pgfpathlineto{\pgfqpoint{5.271745in}{0.739656in}}%
\pgfpathlineto{\pgfqpoint{5.270892in}{0.739656in}}%
\pgfpathlineto{\pgfqpoint{5.270040in}{0.739656in}}%
\pgfpathlineto{\pgfqpoint{5.269187in}{0.739656in}}%
\pgfpathlineto{\pgfqpoint{5.268334in}{0.739656in}}%
\pgfpathlineto{\pgfqpoint{5.267481in}{0.739656in}}%
\pgfpathlineto{\pgfqpoint{5.266628in}{0.739656in}}%
\pgfpathlineto{\pgfqpoint{5.265775in}{0.739656in}}%
\pgfpathlineto{\pgfqpoint{5.264922in}{0.739656in}}%
\pgfpathlineto{\pgfqpoint{5.264069in}{0.739656in}}%
\pgfpathlineto{\pgfqpoint{5.263216in}{0.739656in}}%
\pgfpathlineto{\pgfqpoint{5.262364in}{0.739656in}}%
\pgfpathlineto{\pgfqpoint{5.261511in}{0.739656in}}%
\pgfpathlineto{\pgfqpoint{5.260658in}{0.739656in}}%
\pgfpathlineto{\pgfqpoint{5.259805in}{0.739656in}}%
\pgfpathlineto{\pgfqpoint{5.258952in}{0.739656in}}%
\pgfpathlineto{\pgfqpoint{5.258099in}{0.739656in}}%
\pgfpathlineto{\pgfqpoint{5.257246in}{0.739656in}}%
\pgfpathlineto{\pgfqpoint{5.256393in}{0.739656in}}%
\pgfpathlineto{\pgfqpoint{5.255541in}{0.739656in}}%
\pgfpathlineto{\pgfqpoint{5.254688in}{0.739656in}}%
\pgfpathlineto{\pgfqpoint{5.253835in}{0.739656in}}%
\pgfpathlineto{\pgfqpoint{5.252982in}{0.739656in}}%
\pgfpathlineto{\pgfqpoint{5.252129in}{0.739656in}}%
\pgfpathlineto{\pgfqpoint{5.251276in}{0.739656in}}%
\pgfpathlineto{\pgfqpoint{5.250423in}{0.739656in}}%
\pgfpathlineto{\pgfqpoint{5.249570in}{0.739656in}}%
\pgfpathlineto{\pgfqpoint{5.248718in}{0.739656in}}%
\pgfpathlineto{\pgfqpoint{5.247865in}{0.739656in}}%
\pgfpathlineto{\pgfqpoint{5.247012in}{0.739656in}}%
\pgfpathlineto{\pgfqpoint{5.246159in}{0.739656in}}%
\pgfpathlineto{\pgfqpoint{5.245306in}{0.739656in}}%
\pgfpathlineto{\pgfqpoint{5.244453in}{0.739656in}}%
\pgfpathlineto{\pgfqpoint{5.243600in}{0.739656in}}%
\pgfpathlineto{\pgfqpoint{5.242747in}{0.739656in}}%
\pgfpathlineto{\pgfqpoint{5.241895in}{0.739656in}}%
\pgfpathlineto{\pgfqpoint{5.241042in}{0.739656in}}%
\pgfpathlineto{\pgfqpoint{5.240189in}{0.739656in}}%
\pgfpathlineto{\pgfqpoint{5.239336in}{0.739656in}}%
\pgfpathlineto{\pgfqpoint{5.238483in}{0.739656in}}%
\pgfpathlineto{\pgfqpoint{5.237630in}{0.739656in}}%
\pgfpathlineto{\pgfqpoint{5.236777in}{0.739656in}}%
\pgfpathlineto{\pgfqpoint{5.235924in}{0.739656in}}%
\pgfpathlineto{\pgfqpoint{5.235072in}{0.739656in}}%
\pgfpathlineto{\pgfqpoint{5.234219in}{0.739656in}}%
\pgfpathlineto{\pgfqpoint{5.233366in}{0.739656in}}%
\pgfpathlineto{\pgfqpoint{5.232513in}{0.739656in}}%
\pgfpathlineto{\pgfqpoint{5.231660in}{0.739656in}}%
\pgfpathlineto{\pgfqpoint{5.230807in}{0.739656in}}%
\pgfpathlineto{\pgfqpoint{5.229954in}{0.739656in}}%
\pgfpathlineto{\pgfqpoint{5.229101in}{0.739656in}}%
\pgfpathlineto{\pgfqpoint{5.228248in}{0.739656in}}%
\pgfpathlineto{\pgfqpoint{5.227396in}{0.739656in}}%
\pgfpathlineto{\pgfqpoint{5.226543in}{0.739656in}}%
\pgfpathlineto{\pgfqpoint{5.225690in}{0.739656in}}%
\pgfpathlineto{\pgfqpoint{5.224837in}{0.739656in}}%
\pgfpathlineto{\pgfqpoint{5.223984in}{0.739656in}}%
\pgfpathlineto{\pgfqpoint{5.223131in}{0.739656in}}%
\pgfpathlineto{\pgfqpoint{5.222278in}{0.739656in}}%
\pgfpathlineto{\pgfqpoint{5.221425in}{0.739656in}}%
\pgfpathlineto{\pgfqpoint{5.220573in}{0.739656in}}%
\pgfpathlineto{\pgfqpoint{5.219720in}{0.739656in}}%
\pgfpathlineto{\pgfqpoint{5.218867in}{0.739656in}}%
\pgfpathlineto{\pgfqpoint{5.218014in}{0.739656in}}%
\pgfpathlineto{\pgfqpoint{5.217161in}{0.739656in}}%
\pgfpathlineto{\pgfqpoint{5.216308in}{0.739656in}}%
\pgfpathlineto{\pgfqpoint{5.215455in}{0.739656in}}%
\pgfpathlineto{\pgfqpoint{5.214602in}{0.739656in}}%
\pgfpathlineto{\pgfqpoint{5.213750in}{0.739656in}}%
\pgfpathlineto{\pgfqpoint{5.212897in}{0.739656in}}%
\pgfpathlineto{\pgfqpoint{5.212044in}{0.739656in}}%
\pgfpathlineto{\pgfqpoint{5.211191in}{0.739656in}}%
\pgfpathlineto{\pgfqpoint{5.210338in}{0.739656in}}%
\pgfpathlineto{\pgfqpoint{5.209485in}{0.739656in}}%
\pgfpathlineto{\pgfqpoint{5.208632in}{0.739656in}}%
\pgfpathlineto{\pgfqpoint{5.207779in}{0.739656in}}%
\pgfpathlineto{\pgfqpoint{5.206927in}{0.739656in}}%
\pgfpathlineto{\pgfqpoint{5.206074in}{0.739656in}}%
\pgfpathlineto{\pgfqpoint{5.205221in}{0.739656in}}%
\pgfpathlineto{\pgfqpoint{5.204368in}{0.739656in}}%
\pgfpathlineto{\pgfqpoint{5.203515in}{0.739656in}}%
\pgfpathlineto{\pgfqpoint{5.202662in}{0.739656in}}%
\pgfpathlineto{\pgfqpoint{5.201809in}{0.739656in}}%
\pgfpathlineto{\pgfqpoint{5.200956in}{0.739656in}}%
\pgfpathlineto{\pgfqpoint{5.200103in}{0.739656in}}%
\pgfpathlineto{\pgfqpoint{5.199251in}{0.739656in}}%
\pgfpathlineto{\pgfqpoint{5.198398in}{0.739656in}}%
\pgfpathlineto{\pgfqpoint{5.197545in}{0.739656in}}%
\pgfpathlineto{\pgfqpoint{5.196692in}{0.739656in}}%
\pgfpathlineto{\pgfqpoint{5.195839in}{0.739656in}}%
\pgfpathlineto{\pgfqpoint{5.194986in}{0.739656in}}%
\pgfpathlineto{\pgfqpoint{5.194133in}{0.739656in}}%
\pgfpathlineto{\pgfqpoint{5.193280in}{0.739656in}}%
\pgfpathlineto{\pgfqpoint{5.192428in}{0.739656in}}%
\pgfpathlineto{\pgfqpoint{5.191575in}{0.739656in}}%
\pgfpathlineto{\pgfqpoint{5.190722in}{0.739656in}}%
\pgfpathlineto{\pgfqpoint{5.189869in}{0.739656in}}%
\pgfpathlineto{\pgfqpoint{5.189016in}{0.739656in}}%
\pgfpathlineto{\pgfqpoint{5.188163in}{0.739656in}}%
\pgfpathlineto{\pgfqpoint{5.187310in}{0.739656in}}%
\pgfpathlineto{\pgfqpoint{5.186457in}{0.739656in}}%
\pgfpathlineto{\pgfqpoint{5.185605in}{0.739656in}}%
\pgfpathlineto{\pgfqpoint{5.184752in}{0.739656in}}%
\pgfpathlineto{\pgfqpoint{5.183899in}{0.739656in}}%
\pgfpathlineto{\pgfqpoint{5.183046in}{0.739656in}}%
\pgfpathlineto{\pgfqpoint{5.182193in}{0.739656in}}%
\pgfpathlineto{\pgfqpoint{5.181340in}{0.739656in}}%
\pgfpathlineto{\pgfqpoint{5.180487in}{0.739656in}}%
\pgfpathlineto{\pgfqpoint{5.179634in}{0.739656in}}%
\pgfpathlineto{\pgfqpoint{5.178782in}{0.739656in}}%
\pgfpathlineto{\pgfqpoint{5.177929in}{0.739656in}}%
\pgfpathlineto{\pgfqpoint{5.177076in}{0.739656in}}%
\pgfpathlineto{\pgfqpoint{5.176223in}{0.739656in}}%
\pgfpathlineto{\pgfqpoint{5.175370in}{0.739656in}}%
\pgfpathlineto{\pgfqpoint{5.174517in}{0.739656in}}%
\pgfpathlineto{\pgfqpoint{5.173664in}{0.739656in}}%
\pgfpathlineto{\pgfqpoint{5.172811in}{0.739656in}}%
\pgfpathlineto{\pgfqpoint{5.171958in}{0.739656in}}%
\pgfpathlineto{\pgfqpoint{5.171106in}{0.739656in}}%
\pgfpathlineto{\pgfqpoint{5.170253in}{0.739656in}}%
\pgfpathlineto{\pgfqpoint{5.169400in}{0.739656in}}%
\pgfpathlineto{\pgfqpoint{5.168547in}{0.739656in}}%
\pgfpathlineto{\pgfqpoint{5.167694in}{0.739656in}}%
\pgfpathlineto{\pgfqpoint{5.166841in}{0.739656in}}%
\pgfpathlineto{\pgfqpoint{5.165988in}{0.739656in}}%
\pgfpathlineto{\pgfqpoint{5.165135in}{0.739656in}}%
\pgfpathlineto{\pgfqpoint{5.164283in}{0.739656in}}%
\pgfpathlineto{\pgfqpoint{5.163430in}{0.739656in}}%
\pgfpathlineto{\pgfqpoint{5.162577in}{0.739656in}}%
\pgfpathlineto{\pgfqpoint{5.161724in}{0.739656in}}%
\pgfpathlineto{\pgfqpoint{5.160871in}{0.739656in}}%
\pgfpathlineto{\pgfqpoint{5.160018in}{0.739656in}}%
\pgfpathlineto{\pgfqpoint{5.159165in}{0.739656in}}%
\pgfpathlineto{\pgfqpoint{5.158312in}{0.739656in}}%
\pgfpathlineto{\pgfqpoint{5.157460in}{0.739656in}}%
\pgfpathlineto{\pgfqpoint{5.156607in}{0.739656in}}%
\pgfpathlineto{\pgfqpoint{5.155754in}{0.739656in}}%
\pgfpathlineto{\pgfqpoint{5.154901in}{0.739656in}}%
\pgfpathlineto{\pgfqpoint{5.154048in}{0.739656in}}%
\pgfpathlineto{\pgfqpoint{5.153195in}{0.739656in}}%
\pgfpathlineto{\pgfqpoint{5.152342in}{0.739656in}}%
\pgfpathlineto{\pgfqpoint{5.151489in}{0.739656in}}%
\pgfpathlineto{\pgfqpoint{5.150637in}{0.739656in}}%
\pgfpathlineto{\pgfqpoint{5.149784in}{0.739656in}}%
\pgfpathlineto{\pgfqpoint{5.148931in}{0.739656in}}%
\pgfpathlineto{\pgfqpoint{5.148078in}{0.739656in}}%
\pgfpathlineto{\pgfqpoint{5.147225in}{0.739656in}}%
\pgfpathlineto{\pgfqpoint{5.146372in}{0.739656in}}%
\pgfpathlineto{\pgfqpoint{5.145519in}{0.739656in}}%
\pgfpathlineto{\pgfqpoint{5.144666in}{0.739656in}}%
\pgfpathlineto{\pgfqpoint{5.143813in}{0.739656in}}%
\pgfpathlineto{\pgfqpoint{5.142961in}{0.739656in}}%
\pgfpathlineto{\pgfqpoint{5.142108in}{0.739656in}}%
\pgfpathlineto{\pgfqpoint{5.141255in}{0.739656in}}%
\pgfpathlineto{\pgfqpoint{5.140402in}{0.739656in}}%
\pgfpathlineto{\pgfqpoint{5.139549in}{0.739656in}}%
\pgfpathlineto{\pgfqpoint{5.138696in}{0.739656in}}%
\pgfpathlineto{\pgfqpoint{5.137843in}{0.739656in}}%
\pgfpathlineto{\pgfqpoint{5.136990in}{0.739656in}}%
\pgfpathlineto{\pgfqpoint{5.136138in}{0.739656in}}%
\pgfpathlineto{\pgfqpoint{5.135285in}{0.739656in}}%
\pgfpathlineto{\pgfqpoint{5.134432in}{0.739656in}}%
\pgfpathlineto{\pgfqpoint{5.133579in}{0.739656in}}%
\pgfpathlineto{\pgfqpoint{5.132726in}{0.739656in}}%
\pgfpathlineto{\pgfqpoint{5.131873in}{0.739656in}}%
\pgfpathlineto{\pgfqpoint{5.131020in}{0.739656in}}%
\pgfpathlineto{\pgfqpoint{5.130167in}{0.739656in}}%
\pgfpathlineto{\pgfqpoint{5.129315in}{0.739656in}}%
\pgfpathlineto{\pgfqpoint{5.128462in}{0.739656in}}%
\pgfpathlineto{\pgfqpoint{5.127609in}{0.739656in}}%
\pgfpathlineto{\pgfqpoint{5.126756in}{0.739656in}}%
\pgfpathlineto{\pgfqpoint{5.125903in}{0.739656in}}%
\pgfpathlineto{\pgfqpoint{5.125050in}{0.739656in}}%
\pgfpathlineto{\pgfqpoint{5.124197in}{0.739656in}}%
\pgfpathlineto{\pgfqpoint{5.123344in}{0.739656in}}%
\pgfpathlineto{\pgfqpoint{5.122492in}{0.739656in}}%
\pgfpathlineto{\pgfqpoint{5.121639in}{0.739656in}}%
\pgfpathlineto{\pgfqpoint{5.120786in}{0.739656in}}%
\pgfpathlineto{\pgfqpoint{5.119933in}{0.739656in}}%
\pgfpathlineto{\pgfqpoint{5.119080in}{0.739656in}}%
\pgfpathlineto{\pgfqpoint{5.118227in}{0.739656in}}%
\pgfpathlineto{\pgfqpoint{5.117374in}{0.739656in}}%
\pgfpathlineto{\pgfqpoint{5.116521in}{0.739656in}}%
\pgfpathlineto{\pgfqpoint{5.115669in}{0.739656in}}%
\pgfpathlineto{\pgfqpoint{5.114816in}{0.739656in}}%
\pgfpathlineto{\pgfqpoint{5.113963in}{0.739656in}}%
\pgfpathlineto{\pgfqpoint{5.113110in}{0.739656in}}%
\pgfpathlineto{\pgfqpoint{5.112257in}{0.739656in}}%
\pgfpathlineto{\pgfqpoint{5.111404in}{0.739656in}}%
\pgfpathlineto{\pgfqpoint{5.110551in}{0.739656in}}%
\pgfpathlineto{\pgfqpoint{5.109698in}{0.739656in}}%
\pgfpathlineto{\pgfqpoint{5.108845in}{0.739656in}}%
\pgfpathlineto{\pgfqpoint{5.107993in}{0.739656in}}%
\pgfpathlineto{\pgfqpoint{5.107140in}{0.739656in}}%
\pgfpathlineto{\pgfqpoint{5.106287in}{0.739656in}}%
\pgfpathlineto{\pgfqpoint{5.105434in}{0.739656in}}%
\pgfpathlineto{\pgfqpoint{5.104581in}{0.739656in}}%
\pgfpathlineto{\pgfqpoint{5.103728in}{0.739656in}}%
\pgfpathlineto{\pgfqpoint{5.102875in}{0.739656in}}%
\pgfpathlineto{\pgfqpoint{5.102022in}{0.739656in}}%
\pgfpathlineto{\pgfqpoint{5.101170in}{0.739656in}}%
\pgfpathlineto{\pgfqpoint{5.100317in}{0.739656in}}%
\pgfpathlineto{\pgfqpoint{5.099464in}{0.739656in}}%
\pgfpathlineto{\pgfqpoint{5.098611in}{0.739656in}}%
\pgfpathlineto{\pgfqpoint{5.097758in}{0.739656in}}%
\pgfpathlineto{\pgfqpoint{5.096905in}{0.739656in}}%
\pgfpathlineto{\pgfqpoint{5.096052in}{0.739656in}}%
\pgfpathlineto{\pgfqpoint{5.095199in}{0.739656in}}%
\pgfpathlineto{\pgfqpoint{5.094347in}{0.739656in}}%
\pgfpathlineto{\pgfqpoint{5.093494in}{0.739656in}}%
\pgfpathlineto{\pgfqpoint{5.092641in}{0.739656in}}%
\pgfpathlineto{\pgfqpoint{5.091788in}{0.739656in}}%
\pgfpathlineto{\pgfqpoint{5.090935in}{0.739656in}}%
\pgfpathlineto{\pgfqpoint{5.090082in}{0.739656in}}%
\pgfpathlineto{\pgfqpoint{5.089229in}{0.739656in}}%
\pgfpathlineto{\pgfqpoint{5.088376in}{0.739656in}}%
\pgfpathlineto{\pgfqpoint{5.087524in}{0.739656in}}%
\pgfpathlineto{\pgfqpoint{5.086671in}{0.739656in}}%
\pgfpathlineto{\pgfqpoint{5.085818in}{0.739656in}}%
\pgfpathlineto{\pgfqpoint{5.084965in}{0.739656in}}%
\pgfpathlineto{\pgfqpoint{5.084112in}{0.739656in}}%
\pgfpathlineto{\pgfqpoint{5.083259in}{0.739656in}}%
\pgfpathlineto{\pgfqpoint{5.082406in}{0.739656in}}%
\pgfpathlineto{\pgfqpoint{5.081553in}{0.739656in}}%
\pgfpathlineto{\pgfqpoint{5.080700in}{0.739656in}}%
\pgfpathlineto{\pgfqpoint{5.079848in}{0.739656in}}%
\pgfpathlineto{\pgfqpoint{5.078995in}{0.739656in}}%
\pgfpathlineto{\pgfqpoint{5.078142in}{0.739656in}}%
\pgfpathlineto{\pgfqpoint{5.077289in}{0.739656in}}%
\pgfpathlineto{\pgfqpoint{5.076436in}{0.739656in}}%
\pgfpathlineto{\pgfqpoint{5.075583in}{0.739656in}}%
\pgfpathlineto{\pgfqpoint{5.074730in}{0.739656in}}%
\pgfpathlineto{\pgfqpoint{5.073877in}{0.739656in}}%
\pgfpathlineto{\pgfqpoint{5.073025in}{0.739656in}}%
\pgfpathlineto{\pgfqpoint{5.072172in}{0.739656in}}%
\pgfpathlineto{\pgfqpoint{5.071319in}{0.739656in}}%
\pgfpathlineto{\pgfqpoint{5.070466in}{0.739656in}}%
\pgfpathlineto{\pgfqpoint{5.069613in}{0.739656in}}%
\pgfpathlineto{\pgfqpoint{5.068760in}{0.739656in}}%
\pgfpathlineto{\pgfqpoint{5.067907in}{0.739656in}}%
\pgfpathlineto{\pgfqpoint{5.067054in}{0.739656in}}%
\pgfpathlineto{\pgfqpoint{5.066202in}{0.739656in}}%
\pgfpathlineto{\pgfqpoint{5.065349in}{0.739656in}}%
\pgfpathlineto{\pgfqpoint{5.064496in}{0.739656in}}%
\pgfpathlineto{\pgfqpoint{5.063643in}{0.739656in}}%
\pgfpathlineto{\pgfqpoint{5.062790in}{0.739656in}}%
\pgfpathlineto{\pgfqpoint{5.061937in}{0.739656in}}%
\pgfpathlineto{\pgfqpoint{5.061084in}{0.739656in}}%
\pgfpathlineto{\pgfqpoint{5.060231in}{0.739656in}}%
\pgfpathlineto{\pgfqpoint{5.059379in}{0.739656in}}%
\pgfpathlineto{\pgfqpoint{5.058526in}{0.739656in}}%
\pgfpathlineto{\pgfqpoint{5.057673in}{0.739656in}}%
\pgfpathlineto{\pgfqpoint{5.056820in}{0.739656in}}%
\pgfpathlineto{\pgfqpoint{5.055967in}{0.739656in}}%
\pgfpathlineto{\pgfqpoint{5.055114in}{0.739656in}}%
\pgfpathlineto{\pgfqpoint{5.054261in}{0.739656in}}%
\pgfpathlineto{\pgfqpoint{5.053408in}{0.739656in}}%
\pgfpathlineto{\pgfqpoint{5.052555in}{0.739656in}}%
\pgfpathlineto{\pgfqpoint{5.051703in}{0.739656in}}%
\pgfpathlineto{\pgfqpoint{5.050850in}{0.739656in}}%
\pgfpathlineto{\pgfqpoint{5.049997in}{0.739656in}}%
\pgfpathlineto{\pgfqpoint{5.049144in}{0.739656in}}%
\pgfpathlineto{\pgfqpoint{5.048291in}{0.739656in}}%
\pgfpathlineto{\pgfqpoint{5.047438in}{0.739656in}}%
\pgfpathlineto{\pgfqpoint{5.046585in}{0.739656in}}%
\pgfpathlineto{\pgfqpoint{5.045732in}{0.739656in}}%
\pgfpathlineto{\pgfqpoint{5.044880in}{0.739656in}}%
\pgfpathlineto{\pgfqpoint{5.044027in}{0.739656in}}%
\pgfpathlineto{\pgfqpoint{5.043174in}{0.739656in}}%
\pgfpathlineto{\pgfqpoint{5.042321in}{0.739656in}}%
\pgfpathlineto{\pgfqpoint{5.041468in}{0.739656in}}%
\pgfpathlineto{\pgfqpoint{5.040615in}{0.739656in}}%
\pgfpathlineto{\pgfqpoint{5.039762in}{0.739656in}}%
\pgfpathlineto{\pgfqpoint{5.038909in}{0.739656in}}%
\pgfpathlineto{\pgfqpoint{5.038057in}{0.739656in}}%
\pgfpathlineto{\pgfqpoint{5.037204in}{0.739656in}}%
\pgfpathlineto{\pgfqpoint{5.036351in}{0.739656in}}%
\pgfpathlineto{\pgfqpoint{5.035498in}{0.739656in}}%
\pgfpathlineto{\pgfqpoint{5.034645in}{0.739656in}}%
\pgfpathlineto{\pgfqpoint{5.033792in}{0.739656in}}%
\pgfpathlineto{\pgfqpoint{5.032939in}{0.739656in}}%
\pgfpathlineto{\pgfqpoint{5.032086in}{0.739656in}}%
\pgfpathlineto{\pgfqpoint{5.031234in}{0.739656in}}%
\pgfpathlineto{\pgfqpoint{5.030381in}{0.739656in}}%
\pgfpathlineto{\pgfqpoint{5.029528in}{0.739656in}}%
\pgfpathlineto{\pgfqpoint{5.028675in}{0.739656in}}%
\pgfpathlineto{\pgfqpoint{5.027822in}{0.739656in}}%
\pgfpathlineto{\pgfqpoint{5.026969in}{0.739656in}}%
\pgfpathlineto{\pgfqpoint{5.026116in}{0.739656in}}%
\pgfpathlineto{\pgfqpoint{5.025263in}{0.739656in}}%
\pgfpathlineto{\pgfqpoint{5.024411in}{0.739656in}}%
\pgfpathlineto{\pgfqpoint{5.023558in}{0.739656in}}%
\pgfpathlineto{\pgfqpoint{5.022705in}{0.739656in}}%
\pgfpathlineto{\pgfqpoint{5.021852in}{0.739656in}}%
\pgfpathlineto{\pgfqpoint{5.020999in}{0.739656in}}%
\pgfpathlineto{\pgfqpoint{5.020146in}{0.739656in}}%
\pgfpathlineto{\pgfqpoint{5.019293in}{0.739656in}}%
\pgfpathlineto{\pgfqpoint{5.018440in}{0.739656in}}%
\pgfpathlineto{\pgfqpoint{5.017587in}{0.739656in}}%
\pgfpathlineto{\pgfqpoint{5.016735in}{0.739656in}}%
\pgfpathlineto{\pgfqpoint{5.015882in}{0.739656in}}%
\pgfpathlineto{\pgfqpoint{5.015029in}{0.739656in}}%
\pgfpathlineto{\pgfqpoint{5.014176in}{0.739656in}}%
\pgfpathlineto{\pgfqpoint{5.013323in}{0.739656in}}%
\pgfpathlineto{\pgfqpoint{5.012470in}{0.739656in}}%
\pgfpathlineto{\pgfqpoint{5.011617in}{0.739656in}}%
\pgfpathlineto{\pgfqpoint{5.010764in}{0.739656in}}%
\pgfpathlineto{\pgfqpoint{5.009912in}{0.739656in}}%
\pgfpathlineto{\pgfqpoint{5.009059in}{0.739656in}}%
\pgfpathlineto{\pgfqpoint{5.008206in}{0.739656in}}%
\pgfpathlineto{\pgfqpoint{5.007353in}{0.739656in}}%
\pgfpathlineto{\pgfqpoint{5.006500in}{0.739656in}}%
\pgfpathlineto{\pgfqpoint{5.005647in}{0.739656in}}%
\pgfpathlineto{\pgfqpoint{5.004794in}{0.739656in}}%
\pgfpathlineto{\pgfqpoint{5.003941in}{0.739656in}}%
\pgfpathlineto{\pgfqpoint{5.003089in}{0.739656in}}%
\pgfpathlineto{\pgfqpoint{5.002236in}{0.739656in}}%
\pgfpathlineto{\pgfqpoint{5.001383in}{0.739656in}}%
\pgfpathlineto{\pgfqpoint{5.000530in}{0.739656in}}%
\pgfpathlineto{\pgfqpoint{4.999677in}{0.739656in}}%
\pgfpathlineto{\pgfqpoint{4.998824in}{0.739656in}}%
\pgfpathlineto{\pgfqpoint{4.997971in}{0.739656in}}%
\pgfpathlineto{\pgfqpoint{4.997118in}{0.739656in}}%
\pgfpathlineto{\pgfqpoint{4.996266in}{0.739656in}}%
\pgfpathlineto{\pgfqpoint{4.995413in}{0.739656in}}%
\pgfpathlineto{\pgfqpoint{4.994560in}{0.739656in}}%
\pgfpathlineto{\pgfqpoint{4.993707in}{0.739656in}}%
\pgfpathlineto{\pgfqpoint{4.992854in}{0.739656in}}%
\pgfpathlineto{\pgfqpoint{4.992001in}{0.739656in}}%
\pgfpathlineto{\pgfqpoint{4.991148in}{0.739656in}}%
\pgfpathlineto{\pgfqpoint{4.990295in}{0.739656in}}%
\pgfpathlineto{\pgfqpoint{4.989442in}{0.739656in}}%
\pgfpathlineto{\pgfqpoint{4.988590in}{0.739656in}}%
\pgfpathlineto{\pgfqpoint{4.987737in}{0.739656in}}%
\pgfpathlineto{\pgfqpoint{4.986884in}{0.739656in}}%
\pgfpathlineto{\pgfqpoint{4.986031in}{0.739656in}}%
\pgfpathlineto{\pgfqpoint{4.985178in}{0.739656in}}%
\pgfpathlineto{\pgfqpoint{4.984325in}{0.739656in}}%
\pgfpathlineto{\pgfqpoint{4.983472in}{0.739656in}}%
\pgfpathlineto{\pgfqpoint{4.982619in}{0.739656in}}%
\pgfpathlineto{\pgfqpoint{4.981767in}{0.739656in}}%
\pgfpathlineto{\pgfqpoint{4.980914in}{0.739656in}}%
\pgfpathlineto{\pgfqpoint{4.980061in}{0.739656in}}%
\pgfpathlineto{\pgfqpoint{4.979208in}{0.739656in}}%
\pgfpathlineto{\pgfqpoint{4.978355in}{0.739656in}}%
\pgfpathlineto{\pgfqpoint{4.977502in}{0.739656in}}%
\pgfpathlineto{\pgfqpoint{4.976649in}{0.739656in}}%
\pgfpathlineto{\pgfqpoint{4.975796in}{0.739656in}}%
\pgfpathlineto{\pgfqpoint{4.974944in}{0.739656in}}%
\pgfpathlineto{\pgfqpoint{4.974091in}{0.739656in}}%
\pgfpathlineto{\pgfqpoint{4.973238in}{0.739656in}}%
\pgfpathlineto{\pgfqpoint{4.972385in}{0.739656in}}%
\pgfpathlineto{\pgfqpoint{4.971532in}{0.739656in}}%
\pgfpathlineto{\pgfqpoint{4.970679in}{0.739656in}}%
\pgfpathlineto{\pgfqpoint{4.969826in}{0.739656in}}%
\pgfpathlineto{\pgfqpoint{4.968973in}{0.739656in}}%
\pgfpathlineto{\pgfqpoint{4.968121in}{0.739656in}}%
\pgfpathlineto{\pgfqpoint{4.967268in}{0.739656in}}%
\pgfpathlineto{\pgfqpoint{4.966415in}{0.739656in}}%
\pgfpathlineto{\pgfqpoint{4.965562in}{0.739656in}}%
\pgfpathlineto{\pgfqpoint{4.964709in}{0.739656in}}%
\pgfpathlineto{\pgfqpoint{4.963856in}{0.739656in}}%
\pgfpathlineto{\pgfqpoint{4.963003in}{0.739656in}}%
\pgfpathlineto{\pgfqpoint{4.962150in}{0.739656in}}%
\pgfpathlineto{\pgfqpoint{4.961297in}{0.739656in}}%
\pgfpathlineto{\pgfqpoint{4.960445in}{0.739656in}}%
\pgfpathlineto{\pgfqpoint{4.959592in}{0.739656in}}%
\pgfpathlineto{\pgfqpoint{4.958739in}{0.739656in}}%
\pgfpathlineto{\pgfqpoint{4.957886in}{0.739656in}}%
\pgfpathlineto{\pgfqpoint{4.957033in}{0.739656in}}%
\pgfpathlineto{\pgfqpoint{4.956180in}{0.739656in}}%
\pgfpathlineto{\pgfqpoint{4.955327in}{0.739656in}}%
\pgfpathlineto{\pgfqpoint{4.954474in}{0.739656in}}%
\pgfpathlineto{\pgfqpoint{4.953622in}{0.739656in}}%
\pgfpathlineto{\pgfqpoint{4.952769in}{0.739656in}}%
\pgfpathlineto{\pgfqpoint{4.951916in}{0.739656in}}%
\pgfpathlineto{\pgfqpoint{4.951063in}{0.739656in}}%
\pgfpathlineto{\pgfqpoint{4.950210in}{0.739656in}}%
\pgfpathlineto{\pgfqpoint{4.949357in}{0.739656in}}%
\pgfpathlineto{\pgfqpoint{4.948504in}{0.739656in}}%
\pgfpathlineto{\pgfqpoint{4.947651in}{0.739656in}}%
\pgfpathlineto{\pgfqpoint{4.946799in}{0.739656in}}%
\pgfpathlineto{\pgfqpoint{4.945946in}{0.739656in}}%
\pgfpathlineto{\pgfqpoint{4.945093in}{0.739656in}}%
\pgfpathlineto{\pgfqpoint{4.944240in}{0.739656in}}%
\pgfpathlineto{\pgfqpoint{4.943387in}{0.739656in}}%
\pgfpathlineto{\pgfqpoint{4.942534in}{0.739656in}}%
\pgfpathlineto{\pgfqpoint{4.941681in}{0.739656in}}%
\pgfpathlineto{\pgfqpoint{4.940828in}{0.739656in}}%
\pgfpathlineto{\pgfqpoint{4.939976in}{0.739656in}}%
\pgfpathlineto{\pgfqpoint{4.939123in}{0.739656in}}%
\pgfpathlineto{\pgfqpoint{4.938270in}{0.739656in}}%
\pgfpathlineto{\pgfqpoint{4.937417in}{0.739656in}}%
\pgfpathlineto{\pgfqpoint{4.936564in}{0.739656in}}%
\pgfpathlineto{\pgfqpoint{4.935711in}{0.739656in}}%
\pgfpathlineto{\pgfqpoint{4.934858in}{0.739656in}}%
\pgfpathlineto{\pgfqpoint{4.934005in}{0.739656in}}%
\pgfpathlineto{\pgfqpoint{4.933152in}{0.739656in}}%
\pgfpathlineto{\pgfqpoint{4.932300in}{0.739656in}}%
\pgfpathlineto{\pgfqpoint{4.931447in}{0.739656in}}%
\pgfpathlineto{\pgfqpoint{4.930594in}{0.739656in}}%
\pgfpathlineto{\pgfqpoint{4.929741in}{0.739656in}}%
\pgfpathlineto{\pgfqpoint{4.928888in}{0.739656in}}%
\pgfpathlineto{\pgfqpoint{4.928035in}{0.739656in}}%
\pgfpathlineto{\pgfqpoint{4.927182in}{0.739656in}}%
\pgfpathlineto{\pgfqpoint{4.926329in}{0.739656in}}%
\pgfpathlineto{\pgfqpoint{4.925477in}{0.739656in}}%
\pgfpathlineto{\pgfqpoint{4.924624in}{0.739656in}}%
\pgfpathlineto{\pgfqpoint{4.923771in}{0.739656in}}%
\pgfpathlineto{\pgfqpoint{4.922918in}{0.739656in}}%
\pgfpathlineto{\pgfqpoint{4.922065in}{0.739656in}}%
\pgfpathlineto{\pgfqpoint{4.921212in}{0.739656in}}%
\pgfpathlineto{\pgfqpoint{4.920359in}{0.739656in}}%
\pgfpathlineto{\pgfqpoint{4.919506in}{0.739656in}}%
\pgfpathlineto{\pgfqpoint{4.918654in}{0.739656in}}%
\pgfpathlineto{\pgfqpoint{4.917801in}{0.739656in}}%
\pgfpathlineto{\pgfqpoint{4.916948in}{0.739656in}}%
\pgfpathlineto{\pgfqpoint{4.916095in}{0.739656in}}%
\pgfpathlineto{\pgfqpoint{4.915242in}{0.739656in}}%
\pgfpathlineto{\pgfqpoint{4.914389in}{0.739656in}}%
\pgfpathlineto{\pgfqpoint{4.913536in}{0.739656in}}%
\pgfpathlineto{\pgfqpoint{4.912683in}{0.739656in}}%
\pgfpathlineto{\pgfqpoint{4.911831in}{0.739656in}}%
\pgfpathlineto{\pgfqpoint{4.910978in}{0.739656in}}%
\pgfpathlineto{\pgfqpoint{4.910125in}{0.739656in}}%
\pgfpathlineto{\pgfqpoint{4.909272in}{0.739656in}}%
\pgfpathlineto{\pgfqpoint{4.908419in}{0.739656in}}%
\pgfpathlineto{\pgfqpoint{4.907566in}{0.739656in}}%
\pgfpathlineto{\pgfqpoint{4.906713in}{0.739656in}}%
\pgfpathlineto{\pgfqpoint{4.905860in}{0.739656in}}%
\pgfpathlineto{\pgfqpoint{4.905008in}{0.739656in}}%
\pgfpathlineto{\pgfqpoint{4.904155in}{0.739656in}}%
\pgfpathlineto{\pgfqpoint{4.903302in}{0.739656in}}%
\pgfpathlineto{\pgfqpoint{4.902449in}{0.739656in}}%
\pgfpathlineto{\pgfqpoint{4.901596in}{0.739656in}}%
\pgfpathlineto{\pgfqpoint{4.900743in}{0.739656in}}%
\pgfpathlineto{\pgfqpoint{4.899890in}{0.739656in}}%
\pgfpathlineto{\pgfqpoint{4.899037in}{0.739656in}}%
\pgfpathlineto{\pgfqpoint{4.898184in}{0.739656in}}%
\pgfpathlineto{\pgfqpoint{4.897332in}{0.739656in}}%
\pgfpathlineto{\pgfqpoint{4.896479in}{0.739656in}}%
\pgfpathlineto{\pgfqpoint{4.895626in}{0.739656in}}%
\pgfpathlineto{\pgfqpoint{4.894773in}{0.739656in}}%
\pgfpathlineto{\pgfqpoint{4.893920in}{0.739656in}}%
\pgfpathlineto{\pgfqpoint{4.893067in}{0.739656in}}%
\pgfpathlineto{\pgfqpoint{4.892214in}{0.739656in}}%
\pgfpathlineto{\pgfqpoint{4.891361in}{0.739656in}}%
\pgfpathlineto{\pgfqpoint{4.890509in}{0.739656in}}%
\pgfpathlineto{\pgfqpoint{4.889656in}{0.739656in}}%
\pgfpathlineto{\pgfqpoint{4.888803in}{0.739656in}}%
\pgfpathlineto{\pgfqpoint{4.887950in}{0.739656in}}%
\pgfpathlineto{\pgfqpoint{4.887097in}{0.739656in}}%
\pgfpathlineto{\pgfqpoint{4.886244in}{0.739656in}}%
\pgfpathlineto{\pgfqpoint{4.885391in}{0.739656in}}%
\pgfpathlineto{\pgfqpoint{4.884538in}{0.739656in}}%
\pgfpathlineto{\pgfqpoint{4.883686in}{0.739656in}}%
\pgfpathlineto{\pgfqpoint{4.882833in}{0.739656in}}%
\pgfpathlineto{\pgfqpoint{4.881980in}{0.739656in}}%
\pgfpathlineto{\pgfqpoint{4.881127in}{0.739656in}}%
\pgfpathlineto{\pgfqpoint{4.880274in}{0.739656in}}%
\pgfpathlineto{\pgfqpoint{4.879421in}{0.739656in}}%
\pgfpathlineto{\pgfqpoint{4.878568in}{0.739656in}}%
\pgfpathlineto{\pgfqpoint{4.877715in}{0.739656in}}%
\pgfpathlineto{\pgfqpoint{4.876863in}{0.739656in}}%
\pgfpathlineto{\pgfqpoint{4.876010in}{0.739656in}}%
\pgfpathlineto{\pgfqpoint{4.875157in}{0.739656in}}%
\pgfpathlineto{\pgfqpoint{4.874304in}{0.739656in}}%
\pgfpathlineto{\pgfqpoint{4.873451in}{0.739656in}}%
\pgfpathlineto{\pgfqpoint{4.872598in}{0.739656in}}%
\pgfpathlineto{\pgfqpoint{4.871745in}{0.739656in}}%
\pgfpathlineto{\pgfqpoint{4.870892in}{0.739656in}}%
\pgfpathlineto{\pgfqpoint{4.870039in}{0.739656in}}%
\pgfpathlineto{\pgfqpoint{4.869187in}{0.739656in}}%
\pgfpathlineto{\pgfqpoint{4.868334in}{0.739656in}}%
\pgfpathlineto{\pgfqpoint{4.867481in}{0.739656in}}%
\pgfpathlineto{\pgfqpoint{4.866628in}{0.739656in}}%
\pgfpathlineto{\pgfqpoint{4.865775in}{0.739656in}}%
\pgfpathlineto{\pgfqpoint{4.864922in}{0.739656in}}%
\pgfpathlineto{\pgfqpoint{4.864069in}{0.739656in}}%
\pgfpathlineto{\pgfqpoint{4.863216in}{0.739656in}}%
\pgfpathlineto{\pgfqpoint{4.862364in}{0.739656in}}%
\pgfpathlineto{\pgfqpoint{4.861511in}{0.739656in}}%
\pgfpathlineto{\pgfqpoint{4.860658in}{0.739656in}}%
\pgfpathlineto{\pgfqpoint{4.859805in}{0.739656in}}%
\pgfpathlineto{\pgfqpoint{4.858952in}{0.739656in}}%
\pgfpathlineto{\pgfqpoint{4.858099in}{0.739656in}}%
\pgfpathlineto{\pgfqpoint{4.857246in}{0.739656in}}%
\pgfpathlineto{\pgfqpoint{4.856393in}{0.739656in}}%
\pgfpathlineto{\pgfqpoint{4.855541in}{0.739656in}}%
\pgfpathlineto{\pgfqpoint{4.854688in}{0.739656in}}%
\pgfpathlineto{\pgfqpoint{4.853835in}{0.739656in}}%
\pgfpathlineto{\pgfqpoint{4.852982in}{0.739656in}}%
\pgfpathlineto{\pgfqpoint{4.852129in}{0.739656in}}%
\pgfpathlineto{\pgfqpoint{4.851276in}{0.739656in}}%
\pgfpathlineto{\pgfqpoint{4.850423in}{0.739656in}}%
\pgfpathlineto{\pgfqpoint{4.849570in}{0.739656in}}%
\pgfpathlineto{\pgfqpoint{4.848718in}{0.739656in}}%
\pgfpathlineto{\pgfqpoint{4.847865in}{0.739656in}}%
\pgfpathlineto{\pgfqpoint{4.847012in}{0.739656in}}%
\pgfpathlineto{\pgfqpoint{4.846159in}{0.739656in}}%
\pgfpathlineto{\pgfqpoint{4.845306in}{0.739656in}}%
\pgfpathlineto{\pgfqpoint{4.844453in}{0.739656in}}%
\pgfpathlineto{\pgfqpoint{4.843600in}{0.739656in}}%
\pgfpathlineto{\pgfqpoint{4.842747in}{0.739656in}}%
\pgfpathlineto{\pgfqpoint{4.841894in}{0.739656in}}%
\pgfpathlineto{\pgfqpoint{4.841042in}{0.739656in}}%
\pgfpathlineto{\pgfqpoint{4.840189in}{0.739656in}}%
\pgfpathlineto{\pgfqpoint{4.839336in}{0.739656in}}%
\pgfpathlineto{\pgfqpoint{4.838483in}{0.739656in}}%
\pgfpathlineto{\pgfqpoint{4.837630in}{0.739656in}}%
\pgfpathlineto{\pgfqpoint{4.836777in}{0.739656in}}%
\pgfpathlineto{\pgfqpoint{4.835924in}{0.739656in}}%
\pgfpathlineto{\pgfqpoint{4.835071in}{0.739656in}}%
\pgfpathlineto{\pgfqpoint{4.834219in}{0.739656in}}%
\pgfpathlineto{\pgfqpoint{4.833366in}{0.739656in}}%
\pgfpathlineto{\pgfqpoint{4.832513in}{0.739656in}}%
\pgfpathlineto{\pgfqpoint{4.831660in}{0.739656in}}%
\pgfpathlineto{\pgfqpoint{4.830807in}{0.739656in}}%
\pgfpathlineto{\pgfqpoint{4.829954in}{0.739656in}}%
\pgfpathlineto{\pgfqpoint{4.829101in}{0.739656in}}%
\pgfpathlineto{\pgfqpoint{4.828248in}{0.739656in}}%
\pgfpathlineto{\pgfqpoint{4.827396in}{0.739656in}}%
\pgfpathlineto{\pgfqpoint{4.826543in}{0.739656in}}%
\pgfpathlineto{\pgfqpoint{4.825690in}{0.739656in}}%
\pgfpathlineto{\pgfqpoint{4.824837in}{0.739656in}}%
\pgfpathlineto{\pgfqpoint{4.823984in}{0.739656in}}%
\pgfpathlineto{\pgfqpoint{4.823131in}{0.739656in}}%
\pgfpathlineto{\pgfqpoint{4.822278in}{0.739656in}}%
\pgfpathlineto{\pgfqpoint{4.821425in}{0.739656in}}%
\pgfpathlineto{\pgfqpoint{4.820573in}{0.739656in}}%
\pgfpathlineto{\pgfqpoint{4.819720in}{0.739656in}}%
\pgfpathlineto{\pgfqpoint{4.818867in}{0.739656in}}%
\pgfpathlineto{\pgfqpoint{4.818014in}{0.739656in}}%
\pgfpathlineto{\pgfqpoint{4.817161in}{0.739656in}}%
\pgfpathlineto{\pgfqpoint{4.816308in}{0.739656in}}%
\pgfpathlineto{\pgfqpoint{4.815455in}{0.739656in}}%
\pgfpathlineto{\pgfqpoint{4.814602in}{0.739656in}}%
\pgfpathlineto{\pgfqpoint{4.813749in}{0.739656in}}%
\pgfpathlineto{\pgfqpoint{4.812897in}{0.739656in}}%
\pgfpathlineto{\pgfqpoint{4.812044in}{0.739656in}}%
\pgfpathlineto{\pgfqpoint{4.811191in}{0.739656in}}%
\pgfpathlineto{\pgfqpoint{4.810338in}{0.739656in}}%
\pgfpathlineto{\pgfqpoint{4.809485in}{0.739656in}}%
\pgfpathlineto{\pgfqpoint{4.808632in}{0.739656in}}%
\pgfpathlineto{\pgfqpoint{4.807779in}{0.739656in}}%
\pgfpathlineto{\pgfqpoint{4.806926in}{0.739656in}}%
\pgfpathlineto{\pgfqpoint{4.806074in}{0.739656in}}%
\pgfpathlineto{\pgfqpoint{4.805221in}{0.739656in}}%
\pgfpathlineto{\pgfqpoint{4.804368in}{0.739656in}}%
\pgfpathlineto{\pgfqpoint{4.803515in}{0.739656in}}%
\pgfpathlineto{\pgfqpoint{4.802662in}{0.739656in}}%
\pgfpathlineto{\pgfqpoint{4.801809in}{0.739656in}}%
\pgfpathlineto{\pgfqpoint{4.800956in}{0.739656in}}%
\pgfpathlineto{\pgfqpoint{4.800103in}{0.739656in}}%
\pgfpathlineto{\pgfqpoint{4.799251in}{0.739656in}}%
\pgfpathlineto{\pgfqpoint{4.798398in}{0.739656in}}%
\pgfpathlineto{\pgfqpoint{4.797545in}{0.739656in}}%
\pgfpathlineto{\pgfqpoint{4.796692in}{0.739656in}}%
\pgfpathlineto{\pgfqpoint{4.795839in}{0.739656in}}%
\pgfpathlineto{\pgfqpoint{4.794986in}{0.739656in}}%
\pgfpathlineto{\pgfqpoint{4.794133in}{0.739656in}}%
\pgfpathlineto{\pgfqpoint{4.793280in}{0.739656in}}%
\pgfpathlineto{\pgfqpoint{4.792428in}{0.739656in}}%
\pgfpathlineto{\pgfqpoint{4.791575in}{0.739656in}}%
\pgfpathlineto{\pgfqpoint{4.790722in}{0.739656in}}%
\pgfpathlineto{\pgfqpoint{4.789869in}{0.739656in}}%
\pgfpathlineto{\pgfqpoint{4.789016in}{0.739656in}}%
\pgfpathlineto{\pgfqpoint{4.788163in}{0.739656in}}%
\pgfpathlineto{\pgfqpoint{4.787310in}{0.739656in}}%
\pgfpathlineto{\pgfqpoint{4.786457in}{0.739656in}}%
\pgfpathlineto{\pgfqpoint{4.785605in}{0.739656in}}%
\pgfpathlineto{\pgfqpoint{4.784752in}{0.739656in}}%
\pgfpathlineto{\pgfqpoint{4.783899in}{0.739656in}}%
\pgfpathlineto{\pgfqpoint{4.783046in}{0.739656in}}%
\pgfpathlineto{\pgfqpoint{4.782193in}{0.739656in}}%
\pgfpathlineto{\pgfqpoint{4.781340in}{0.739656in}}%
\pgfpathlineto{\pgfqpoint{4.780487in}{0.739656in}}%
\pgfpathlineto{\pgfqpoint{4.779634in}{0.739656in}}%
\pgfpathlineto{\pgfqpoint{4.778781in}{0.739656in}}%
\pgfpathlineto{\pgfqpoint{4.777929in}{0.739656in}}%
\pgfpathlineto{\pgfqpoint{4.777076in}{0.739656in}}%
\pgfpathlineto{\pgfqpoint{4.776223in}{0.739656in}}%
\pgfpathlineto{\pgfqpoint{4.775370in}{0.739656in}}%
\pgfpathlineto{\pgfqpoint{4.774517in}{0.739656in}}%
\pgfpathlineto{\pgfqpoint{4.773664in}{0.739656in}}%
\pgfpathlineto{\pgfqpoint{4.772811in}{0.739656in}}%
\pgfpathlineto{\pgfqpoint{4.771958in}{0.739656in}}%
\pgfpathlineto{\pgfqpoint{4.771106in}{0.739656in}}%
\pgfpathlineto{\pgfqpoint{4.770253in}{0.739656in}}%
\pgfpathlineto{\pgfqpoint{4.769400in}{0.739656in}}%
\pgfpathlineto{\pgfqpoint{4.768547in}{0.739656in}}%
\pgfpathlineto{\pgfqpoint{4.767694in}{0.739656in}}%
\pgfpathlineto{\pgfqpoint{4.766841in}{0.739656in}}%
\pgfpathlineto{\pgfqpoint{4.765988in}{0.739656in}}%
\pgfpathlineto{\pgfqpoint{4.765135in}{0.739656in}}%
\pgfpathlineto{\pgfqpoint{4.764283in}{0.739656in}}%
\pgfpathlineto{\pgfqpoint{4.763430in}{0.739656in}}%
\pgfpathlineto{\pgfqpoint{4.762577in}{0.739656in}}%
\pgfpathlineto{\pgfqpoint{4.761724in}{0.739656in}}%
\pgfpathlineto{\pgfqpoint{4.760871in}{0.739656in}}%
\pgfpathlineto{\pgfqpoint{4.760018in}{0.739656in}}%
\pgfpathlineto{\pgfqpoint{4.759165in}{0.739656in}}%
\pgfpathlineto{\pgfqpoint{4.758312in}{0.739656in}}%
\pgfpathlineto{\pgfqpoint{4.757460in}{0.739656in}}%
\pgfpathlineto{\pgfqpoint{4.756607in}{0.739656in}}%
\pgfpathlineto{\pgfqpoint{4.755754in}{0.739656in}}%
\pgfpathlineto{\pgfqpoint{4.754901in}{0.739656in}}%
\pgfpathlineto{\pgfqpoint{4.754048in}{0.739656in}}%
\pgfpathlineto{\pgfqpoint{4.753195in}{0.739656in}}%
\pgfpathlineto{\pgfqpoint{4.752342in}{0.739656in}}%
\pgfpathlineto{\pgfqpoint{4.751489in}{0.739656in}}%
\pgfpathlineto{\pgfqpoint{4.750636in}{0.739656in}}%
\pgfpathlineto{\pgfqpoint{4.749784in}{0.739656in}}%
\pgfpathlineto{\pgfqpoint{4.748931in}{0.739656in}}%
\pgfpathlineto{\pgfqpoint{4.748078in}{0.739656in}}%
\pgfpathlineto{\pgfqpoint{4.747225in}{0.739656in}}%
\pgfpathlineto{\pgfqpoint{4.746372in}{0.739656in}}%
\pgfpathlineto{\pgfqpoint{4.745519in}{0.739656in}}%
\pgfpathlineto{\pgfqpoint{4.744666in}{0.739656in}}%
\pgfpathlineto{\pgfqpoint{4.743813in}{0.739656in}}%
\pgfpathlineto{\pgfqpoint{4.742961in}{0.739656in}}%
\pgfpathlineto{\pgfqpoint{4.742108in}{0.739656in}}%
\pgfpathlineto{\pgfqpoint{4.741255in}{0.739656in}}%
\pgfpathlineto{\pgfqpoint{4.740402in}{0.739656in}}%
\pgfpathlineto{\pgfqpoint{4.739549in}{0.739656in}}%
\pgfpathlineto{\pgfqpoint{4.738696in}{0.739656in}}%
\pgfpathlineto{\pgfqpoint{4.737843in}{0.739656in}}%
\pgfpathlineto{\pgfqpoint{4.736990in}{0.739656in}}%
\pgfpathlineto{\pgfqpoint{4.736138in}{0.739656in}}%
\pgfpathlineto{\pgfqpoint{4.735285in}{0.739656in}}%
\pgfpathlineto{\pgfqpoint{4.734432in}{0.739656in}}%
\pgfpathlineto{\pgfqpoint{4.733579in}{0.739656in}}%
\pgfpathlineto{\pgfqpoint{4.732726in}{0.739656in}}%
\pgfpathlineto{\pgfqpoint{4.731873in}{0.739656in}}%
\pgfpathlineto{\pgfqpoint{4.731020in}{0.739656in}}%
\pgfpathlineto{\pgfqpoint{4.730167in}{0.739656in}}%
\pgfpathlineto{\pgfqpoint{4.729315in}{0.739656in}}%
\pgfpathlineto{\pgfqpoint{4.728462in}{0.739656in}}%
\pgfpathlineto{\pgfqpoint{4.727609in}{0.739656in}}%
\pgfpathlineto{\pgfqpoint{4.726756in}{0.739656in}}%
\pgfpathlineto{\pgfqpoint{4.725903in}{0.739656in}}%
\pgfpathlineto{\pgfqpoint{4.725050in}{0.739656in}}%
\pgfpathlineto{\pgfqpoint{4.724197in}{0.739656in}}%
\pgfpathlineto{\pgfqpoint{4.723344in}{0.739656in}}%
\pgfpathlineto{\pgfqpoint{4.722491in}{0.739656in}}%
\pgfpathlineto{\pgfqpoint{4.721639in}{0.739656in}}%
\pgfpathlineto{\pgfqpoint{4.720786in}{0.739656in}}%
\pgfpathlineto{\pgfqpoint{4.719933in}{0.739656in}}%
\pgfpathlineto{\pgfqpoint{4.719080in}{0.739656in}}%
\pgfpathlineto{\pgfqpoint{4.718227in}{0.739656in}}%
\pgfpathlineto{\pgfqpoint{4.717374in}{0.739656in}}%
\pgfpathlineto{\pgfqpoint{4.716521in}{0.739656in}}%
\pgfpathlineto{\pgfqpoint{4.715668in}{0.739656in}}%
\pgfpathlineto{\pgfqpoint{4.714816in}{0.739656in}}%
\pgfpathlineto{\pgfqpoint{4.713963in}{0.739656in}}%
\pgfpathlineto{\pgfqpoint{4.713110in}{0.739656in}}%
\pgfpathlineto{\pgfqpoint{4.712257in}{0.739656in}}%
\pgfpathlineto{\pgfqpoint{4.711404in}{0.739656in}}%
\pgfpathlineto{\pgfqpoint{4.710551in}{0.739656in}}%
\pgfpathlineto{\pgfqpoint{4.709698in}{0.739656in}}%
\pgfpathlineto{\pgfqpoint{4.708845in}{0.739656in}}%
\pgfpathlineto{\pgfqpoint{4.707993in}{0.739656in}}%
\pgfpathlineto{\pgfqpoint{4.707140in}{0.739656in}}%
\pgfpathlineto{\pgfqpoint{4.706287in}{0.739656in}}%
\pgfpathlineto{\pgfqpoint{4.705434in}{0.739656in}}%
\pgfpathlineto{\pgfqpoint{4.704581in}{0.739656in}}%
\pgfpathlineto{\pgfqpoint{4.703728in}{0.739656in}}%
\pgfpathlineto{\pgfqpoint{4.702875in}{0.739656in}}%
\pgfpathlineto{\pgfqpoint{4.702022in}{0.739656in}}%
\pgfpathlineto{\pgfqpoint{4.701170in}{0.739656in}}%
\pgfpathlineto{\pgfqpoint{4.700317in}{0.739656in}}%
\pgfpathlineto{\pgfqpoint{4.699464in}{0.739656in}}%
\pgfpathlineto{\pgfqpoint{4.698611in}{0.739656in}}%
\pgfpathlineto{\pgfqpoint{4.697758in}{0.739656in}}%
\pgfpathlineto{\pgfqpoint{4.696905in}{0.739656in}}%
\pgfpathlineto{\pgfqpoint{4.696052in}{0.739656in}}%
\pgfpathlineto{\pgfqpoint{4.695199in}{0.739656in}}%
\pgfpathlineto{\pgfqpoint{4.694347in}{0.739656in}}%
\pgfpathlineto{\pgfqpoint{4.693494in}{0.739656in}}%
\pgfpathlineto{\pgfqpoint{4.692641in}{0.739656in}}%
\pgfpathlineto{\pgfqpoint{4.691788in}{0.739656in}}%
\pgfpathlineto{\pgfqpoint{4.690935in}{0.739656in}}%
\pgfpathlineto{\pgfqpoint{4.690082in}{0.739656in}}%
\pgfpathlineto{\pgfqpoint{4.689229in}{0.739656in}}%
\pgfpathlineto{\pgfqpoint{4.688376in}{0.739656in}}%
\pgfpathlineto{\pgfqpoint{4.687523in}{0.739656in}}%
\pgfpathlineto{\pgfqpoint{4.686671in}{0.739656in}}%
\pgfpathlineto{\pgfqpoint{4.685818in}{0.739656in}}%
\pgfpathlineto{\pgfqpoint{4.684965in}{0.739656in}}%
\pgfpathlineto{\pgfqpoint{4.684112in}{0.739656in}}%
\pgfpathlineto{\pgfqpoint{4.683259in}{0.739656in}}%
\pgfpathlineto{\pgfqpoint{4.682406in}{0.739656in}}%
\pgfpathlineto{\pgfqpoint{4.681553in}{0.739656in}}%
\pgfpathlineto{\pgfqpoint{4.680700in}{0.739656in}}%
\pgfpathlineto{\pgfqpoint{4.679848in}{0.739656in}}%
\pgfpathlineto{\pgfqpoint{4.678995in}{0.739656in}}%
\pgfpathlineto{\pgfqpoint{4.678142in}{0.739656in}}%
\pgfpathlineto{\pgfqpoint{4.677289in}{0.739656in}}%
\pgfpathlineto{\pgfqpoint{4.676436in}{0.739656in}}%
\pgfpathlineto{\pgfqpoint{4.675583in}{0.739656in}}%
\pgfpathlineto{\pgfqpoint{4.674730in}{0.739656in}}%
\pgfpathlineto{\pgfqpoint{4.673877in}{0.739656in}}%
\pgfpathlineto{\pgfqpoint{4.673025in}{0.739656in}}%
\pgfpathlineto{\pgfqpoint{4.672172in}{0.739656in}}%
\pgfpathlineto{\pgfqpoint{4.671319in}{0.739656in}}%
\pgfpathlineto{\pgfqpoint{4.670466in}{0.739656in}}%
\pgfpathlineto{\pgfqpoint{4.669613in}{0.739656in}}%
\pgfpathlineto{\pgfqpoint{4.668760in}{0.739656in}}%
\pgfpathlineto{\pgfqpoint{4.667907in}{0.739656in}}%
\pgfpathlineto{\pgfqpoint{4.667054in}{0.739656in}}%
\pgfpathlineto{\pgfqpoint{4.666202in}{0.739656in}}%
\pgfpathlineto{\pgfqpoint{4.665349in}{0.739656in}}%
\pgfpathlineto{\pgfqpoint{4.664496in}{0.739656in}}%
\pgfpathlineto{\pgfqpoint{4.663643in}{0.739656in}}%
\pgfpathlineto{\pgfqpoint{4.662790in}{0.739656in}}%
\pgfpathlineto{\pgfqpoint{4.661937in}{0.739656in}}%
\pgfpathlineto{\pgfqpoint{4.661084in}{0.739656in}}%
\pgfpathlineto{\pgfqpoint{4.660231in}{0.739656in}}%
\pgfpathlineto{\pgfqpoint{4.659378in}{0.739656in}}%
\pgfpathlineto{\pgfqpoint{4.658526in}{0.739656in}}%
\pgfpathlineto{\pgfqpoint{4.657673in}{0.739656in}}%
\pgfpathlineto{\pgfqpoint{4.656820in}{0.739656in}}%
\pgfpathlineto{\pgfqpoint{4.655967in}{0.739656in}}%
\pgfpathlineto{\pgfqpoint{4.655114in}{0.739656in}}%
\pgfpathlineto{\pgfqpoint{4.654261in}{0.739656in}}%
\pgfpathlineto{\pgfqpoint{4.653408in}{0.739656in}}%
\pgfpathlineto{\pgfqpoint{4.652555in}{0.739656in}}%
\pgfpathlineto{\pgfqpoint{4.651703in}{0.739656in}}%
\pgfpathlineto{\pgfqpoint{4.650850in}{0.739656in}}%
\pgfpathlineto{\pgfqpoint{4.649997in}{0.739656in}}%
\pgfpathlineto{\pgfqpoint{4.649144in}{0.739656in}}%
\pgfpathlineto{\pgfqpoint{4.648291in}{0.739656in}}%
\pgfpathlineto{\pgfqpoint{4.647438in}{0.739656in}}%
\pgfpathlineto{\pgfqpoint{4.646585in}{0.739656in}}%
\pgfpathlineto{\pgfqpoint{4.645732in}{0.739656in}}%
\pgfpathlineto{\pgfqpoint{4.644880in}{0.739656in}}%
\pgfpathlineto{\pgfqpoint{4.644027in}{0.739656in}}%
\pgfpathlineto{\pgfqpoint{4.643174in}{0.739656in}}%
\pgfpathlineto{\pgfqpoint{4.642321in}{0.739656in}}%
\pgfpathlineto{\pgfqpoint{4.641468in}{0.739656in}}%
\pgfpathlineto{\pgfqpoint{4.640615in}{0.739656in}}%
\pgfpathlineto{\pgfqpoint{4.639762in}{0.739656in}}%
\pgfpathlineto{\pgfqpoint{4.638909in}{0.739656in}}%
\pgfpathlineto{\pgfqpoint{4.638057in}{0.739656in}}%
\pgfpathlineto{\pgfqpoint{4.637204in}{0.739656in}}%
\pgfpathlineto{\pgfqpoint{4.636351in}{0.739656in}}%
\pgfpathlineto{\pgfqpoint{4.635498in}{0.739656in}}%
\pgfpathlineto{\pgfqpoint{4.634645in}{0.739656in}}%
\pgfpathlineto{\pgfqpoint{4.633792in}{0.739656in}}%
\pgfpathlineto{\pgfqpoint{4.632939in}{0.739656in}}%
\pgfpathlineto{\pgfqpoint{4.632086in}{0.739656in}}%
\pgfpathlineto{\pgfqpoint{4.631233in}{0.739656in}}%
\pgfpathlineto{\pgfqpoint{4.630381in}{0.739656in}}%
\pgfpathlineto{\pgfqpoint{4.629528in}{0.739656in}}%
\pgfpathlineto{\pgfqpoint{4.628675in}{0.739656in}}%
\pgfpathlineto{\pgfqpoint{4.627822in}{0.739656in}}%
\pgfpathlineto{\pgfqpoint{4.626969in}{0.739656in}}%
\pgfpathlineto{\pgfqpoint{4.626116in}{0.739656in}}%
\pgfpathlineto{\pgfqpoint{4.625263in}{0.739656in}}%
\pgfpathlineto{\pgfqpoint{4.624410in}{0.739656in}}%
\pgfpathlineto{\pgfqpoint{4.623558in}{0.739656in}}%
\pgfpathlineto{\pgfqpoint{4.622705in}{0.739656in}}%
\pgfpathlineto{\pgfqpoint{4.621852in}{0.739656in}}%
\pgfpathlineto{\pgfqpoint{4.620999in}{0.739656in}}%
\pgfpathlineto{\pgfqpoint{4.620146in}{0.739656in}}%
\pgfpathlineto{\pgfqpoint{4.619293in}{0.739656in}}%
\pgfpathlineto{\pgfqpoint{4.618440in}{0.739656in}}%
\pgfpathlineto{\pgfqpoint{4.617587in}{0.739656in}}%
\pgfpathlineto{\pgfqpoint{4.616735in}{0.739656in}}%
\pgfpathlineto{\pgfqpoint{4.615882in}{0.739656in}}%
\pgfpathlineto{\pgfqpoint{4.615029in}{0.739656in}}%
\pgfpathlineto{\pgfqpoint{4.614176in}{0.739656in}}%
\pgfpathlineto{\pgfqpoint{4.613323in}{0.739656in}}%
\pgfpathlineto{\pgfqpoint{4.612470in}{0.739656in}}%
\pgfpathlineto{\pgfqpoint{4.611617in}{0.739656in}}%
\pgfpathlineto{\pgfqpoint{4.610764in}{0.739656in}}%
\pgfpathlineto{\pgfqpoint{4.609912in}{0.739656in}}%
\pgfpathlineto{\pgfqpoint{4.609059in}{0.739656in}}%
\pgfpathlineto{\pgfqpoint{4.608206in}{0.739656in}}%
\pgfpathlineto{\pgfqpoint{4.607353in}{0.739656in}}%
\pgfpathlineto{\pgfqpoint{4.606500in}{0.739656in}}%
\pgfpathlineto{\pgfqpoint{4.605647in}{0.739656in}}%
\pgfpathlineto{\pgfqpoint{4.604794in}{0.739656in}}%
\pgfpathlineto{\pgfqpoint{4.603941in}{0.739656in}}%
\pgfpathlineto{\pgfqpoint{4.603088in}{0.739656in}}%
\pgfpathlineto{\pgfqpoint{4.602236in}{0.739656in}}%
\pgfpathlineto{\pgfqpoint{4.601383in}{0.739656in}}%
\pgfpathlineto{\pgfqpoint{4.600530in}{0.739656in}}%
\pgfpathlineto{\pgfqpoint{4.599677in}{0.739656in}}%
\pgfpathlineto{\pgfqpoint{4.598824in}{0.739656in}}%
\pgfpathlineto{\pgfqpoint{4.597971in}{0.739656in}}%
\pgfpathlineto{\pgfqpoint{4.597118in}{0.739656in}}%
\pgfpathlineto{\pgfqpoint{4.596265in}{0.739656in}}%
\pgfpathlineto{\pgfqpoint{4.595413in}{0.739656in}}%
\pgfpathlineto{\pgfqpoint{4.594560in}{0.739656in}}%
\pgfpathlineto{\pgfqpoint{4.593707in}{0.739656in}}%
\pgfpathlineto{\pgfqpoint{4.592854in}{0.739656in}}%
\pgfpathlineto{\pgfqpoint{4.592001in}{0.739656in}}%
\pgfpathlineto{\pgfqpoint{4.591148in}{0.739656in}}%
\pgfpathlineto{\pgfqpoint{4.590295in}{0.739656in}}%
\pgfpathlineto{\pgfqpoint{4.589442in}{0.739656in}}%
\pgfpathlineto{\pgfqpoint{4.588590in}{0.739656in}}%
\pgfpathlineto{\pgfqpoint{4.587737in}{0.739656in}}%
\pgfpathlineto{\pgfqpoint{4.586884in}{0.739656in}}%
\pgfpathlineto{\pgfqpoint{4.586031in}{0.739656in}}%
\pgfpathlineto{\pgfqpoint{4.585178in}{0.739656in}}%
\pgfpathlineto{\pgfqpoint{4.584325in}{0.739656in}}%
\pgfpathlineto{\pgfqpoint{4.583472in}{0.739656in}}%
\pgfpathlineto{\pgfqpoint{4.582619in}{0.739656in}}%
\pgfpathlineto{\pgfqpoint{4.581767in}{0.739656in}}%
\pgfpathlineto{\pgfqpoint{4.580914in}{0.739656in}}%
\pgfpathlineto{\pgfqpoint{4.580061in}{0.739656in}}%
\pgfpathlineto{\pgfqpoint{4.579208in}{0.739656in}}%
\pgfpathlineto{\pgfqpoint{4.578355in}{0.739656in}}%
\pgfpathlineto{\pgfqpoint{4.577502in}{0.739656in}}%
\pgfpathlineto{\pgfqpoint{4.576649in}{0.739656in}}%
\pgfpathlineto{\pgfqpoint{4.575796in}{0.739656in}}%
\pgfpathlineto{\pgfqpoint{4.574944in}{0.739656in}}%
\pgfpathlineto{\pgfqpoint{4.574091in}{0.739656in}}%
\pgfpathlineto{\pgfqpoint{4.573238in}{0.739656in}}%
\pgfpathlineto{\pgfqpoint{4.572385in}{0.739656in}}%
\pgfpathlineto{\pgfqpoint{4.571532in}{0.739656in}}%
\pgfpathlineto{\pgfqpoint{4.570679in}{0.739656in}}%
\pgfpathlineto{\pgfqpoint{4.569826in}{0.739656in}}%
\pgfpathlineto{\pgfqpoint{4.568973in}{0.739656in}}%
\pgfpathlineto{\pgfqpoint{4.568120in}{0.739656in}}%
\pgfpathlineto{\pgfqpoint{4.567268in}{0.739656in}}%
\pgfpathlineto{\pgfqpoint{4.566415in}{0.739656in}}%
\pgfpathlineto{\pgfqpoint{4.565562in}{0.739656in}}%
\pgfpathlineto{\pgfqpoint{4.564709in}{0.739656in}}%
\pgfpathlineto{\pgfqpoint{4.563856in}{0.739656in}}%
\pgfpathlineto{\pgfqpoint{4.563003in}{0.739656in}}%
\pgfpathlineto{\pgfqpoint{4.562150in}{0.739656in}}%
\pgfpathlineto{\pgfqpoint{4.561297in}{0.739656in}}%
\pgfpathlineto{\pgfqpoint{4.560445in}{0.739656in}}%
\pgfpathlineto{\pgfqpoint{4.559592in}{0.739656in}}%
\pgfpathlineto{\pgfqpoint{4.558739in}{0.739656in}}%
\pgfpathlineto{\pgfqpoint{4.557886in}{0.739656in}}%
\pgfpathlineto{\pgfqpoint{4.557033in}{0.739656in}}%
\pgfpathlineto{\pgfqpoint{4.556180in}{0.739656in}}%
\pgfpathlineto{\pgfqpoint{4.555327in}{0.739656in}}%
\pgfpathlineto{\pgfqpoint{4.554474in}{0.739656in}}%
\pgfpathlineto{\pgfqpoint{4.553622in}{0.739656in}}%
\pgfpathlineto{\pgfqpoint{4.552769in}{0.739656in}}%
\pgfpathlineto{\pgfqpoint{4.551916in}{0.739656in}}%
\pgfpathlineto{\pgfqpoint{4.551063in}{0.739656in}}%
\pgfpathlineto{\pgfqpoint{4.550210in}{0.739656in}}%
\pgfpathlineto{\pgfqpoint{4.549357in}{0.739656in}}%
\pgfpathlineto{\pgfqpoint{4.548504in}{0.739656in}}%
\pgfpathlineto{\pgfqpoint{4.547651in}{0.739656in}}%
\pgfpathlineto{\pgfqpoint{4.546799in}{0.739656in}}%
\pgfpathlineto{\pgfqpoint{4.545946in}{0.739656in}}%
\pgfpathlineto{\pgfqpoint{4.545093in}{0.739656in}}%
\pgfpathlineto{\pgfqpoint{4.544240in}{0.739656in}}%
\pgfpathlineto{\pgfqpoint{4.543387in}{0.739656in}}%
\pgfpathlineto{\pgfqpoint{4.542534in}{0.739656in}}%
\pgfpathlineto{\pgfqpoint{4.541681in}{0.739656in}}%
\pgfpathlineto{\pgfqpoint{4.540828in}{0.739656in}}%
\pgfpathlineto{\pgfqpoint{4.539975in}{0.739656in}}%
\pgfpathlineto{\pgfqpoint{4.539123in}{0.739656in}}%
\pgfpathlineto{\pgfqpoint{4.538270in}{0.739656in}}%
\pgfpathlineto{\pgfqpoint{4.537417in}{0.739656in}}%
\pgfpathlineto{\pgfqpoint{4.536564in}{0.739656in}}%
\pgfpathlineto{\pgfqpoint{4.535711in}{0.739656in}}%
\pgfpathlineto{\pgfqpoint{4.534858in}{0.739656in}}%
\pgfpathlineto{\pgfqpoint{4.534005in}{0.739656in}}%
\pgfpathlineto{\pgfqpoint{4.533152in}{0.739656in}}%
\pgfpathlineto{\pgfqpoint{4.532300in}{0.739656in}}%
\pgfpathlineto{\pgfqpoint{4.531447in}{0.739656in}}%
\pgfpathlineto{\pgfqpoint{4.530594in}{0.739656in}}%
\pgfpathlineto{\pgfqpoint{4.529741in}{0.739656in}}%
\pgfpathlineto{\pgfqpoint{4.528888in}{0.739656in}}%
\pgfpathlineto{\pgfqpoint{4.528035in}{0.739656in}}%
\pgfpathlineto{\pgfqpoint{4.527182in}{0.739656in}}%
\pgfpathlineto{\pgfqpoint{4.526329in}{0.739656in}}%
\pgfpathlineto{\pgfqpoint{4.525477in}{0.739656in}}%
\pgfpathlineto{\pgfqpoint{4.524624in}{0.739656in}}%
\pgfpathlineto{\pgfqpoint{4.523771in}{0.739656in}}%
\pgfpathlineto{\pgfqpoint{4.522918in}{0.739656in}}%
\pgfpathlineto{\pgfqpoint{4.522065in}{0.739656in}}%
\pgfpathlineto{\pgfqpoint{4.521212in}{0.739656in}}%
\pgfpathlineto{\pgfqpoint{4.520359in}{0.739656in}}%
\pgfpathlineto{\pgfqpoint{4.519506in}{0.739656in}}%
\pgfpathlineto{\pgfqpoint{4.518654in}{0.739656in}}%
\pgfpathlineto{\pgfqpoint{4.517801in}{0.739656in}}%
\pgfpathlineto{\pgfqpoint{4.516948in}{0.739656in}}%
\pgfpathlineto{\pgfqpoint{4.516095in}{0.739656in}}%
\pgfpathlineto{\pgfqpoint{4.515242in}{0.739656in}}%
\pgfpathlineto{\pgfqpoint{4.514389in}{0.739656in}}%
\pgfpathlineto{\pgfqpoint{4.513536in}{0.739656in}}%
\pgfpathlineto{\pgfqpoint{4.512683in}{0.739656in}}%
\pgfpathlineto{\pgfqpoint{4.511830in}{0.739656in}}%
\pgfpathlineto{\pgfqpoint{4.510978in}{0.739656in}}%
\pgfpathlineto{\pgfqpoint{4.510125in}{0.739656in}}%
\pgfpathlineto{\pgfqpoint{4.509272in}{0.739656in}}%
\pgfpathlineto{\pgfqpoint{4.508419in}{0.739656in}}%
\pgfpathlineto{\pgfqpoint{4.507566in}{0.739656in}}%
\pgfpathlineto{\pgfqpoint{4.506713in}{0.739656in}}%
\pgfpathlineto{\pgfqpoint{4.505860in}{0.739656in}}%
\pgfpathlineto{\pgfqpoint{4.505007in}{0.739656in}}%
\pgfpathlineto{\pgfqpoint{4.504155in}{0.739656in}}%
\pgfpathlineto{\pgfqpoint{4.503302in}{0.739656in}}%
\pgfpathlineto{\pgfqpoint{4.502449in}{0.739656in}}%
\pgfpathlineto{\pgfqpoint{4.501596in}{0.739656in}}%
\pgfpathlineto{\pgfqpoint{4.500743in}{0.739656in}}%
\pgfpathlineto{\pgfqpoint{4.499890in}{0.739656in}}%
\pgfpathlineto{\pgfqpoint{4.499037in}{0.739656in}}%
\pgfpathlineto{\pgfqpoint{4.498184in}{0.739656in}}%
\pgfpathlineto{\pgfqpoint{4.497332in}{0.739656in}}%
\pgfpathlineto{\pgfqpoint{4.496479in}{0.739656in}}%
\pgfpathlineto{\pgfqpoint{4.495626in}{0.739656in}}%
\pgfpathlineto{\pgfqpoint{4.494773in}{0.739656in}}%
\pgfpathlineto{\pgfqpoint{4.493920in}{0.739656in}}%
\pgfpathlineto{\pgfqpoint{4.493067in}{0.739656in}}%
\pgfpathlineto{\pgfqpoint{4.492214in}{0.739656in}}%
\pgfpathlineto{\pgfqpoint{4.491361in}{0.739656in}}%
\pgfpathlineto{\pgfqpoint{4.490509in}{0.739656in}}%
\pgfpathlineto{\pgfqpoint{4.489656in}{0.739656in}}%
\pgfpathlineto{\pgfqpoint{4.488803in}{0.739656in}}%
\pgfpathlineto{\pgfqpoint{4.487950in}{0.739656in}}%
\pgfpathlineto{\pgfqpoint{4.487097in}{0.739656in}}%
\pgfpathlineto{\pgfqpoint{4.486244in}{0.739656in}}%
\pgfpathlineto{\pgfqpoint{4.485391in}{0.739656in}}%
\pgfpathlineto{\pgfqpoint{4.484538in}{0.739656in}}%
\pgfpathlineto{\pgfqpoint{4.483685in}{0.739656in}}%
\pgfpathlineto{\pgfqpoint{4.482833in}{0.739656in}}%
\pgfpathlineto{\pgfqpoint{4.481980in}{0.739656in}}%
\pgfpathlineto{\pgfqpoint{4.481127in}{0.739656in}}%
\pgfpathlineto{\pgfqpoint{4.480274in}{0.739656in}}%
\pgfpathlineto{\pgfqpoint{4.479421in}{0.739656in}}%
\pgfpathlineto{\pgfqpoint{4.478568in}{0.739656in}}%
\pgfpathlineto{\pgfqpoint{4.477715in}{0.739656in}}%
\pgfpathlineto{\pgfqpoint{4.476862in}{0.739656in}}%
\pgfpathlineto{\pgfqpoint{4.476010in}{0.739656in}}%
\pgfpathlineto{\pgfqpoint{4.475157in}{0.739656in}}%
\pgfpathlineto{\pgfqpoint{4.474304in}{0.739656in}}%
\pgfpathlineto{\pgfqpoint{4.473451in}{0.739656in}}%
\pgfpathlineto{\pgfqpoint{4.472598in}{0.739656in}}%
\pgfpathlineto{\pgfqpoint{4.471745in}{0.739656in}}%
\pgfpathlineto{\pgfqpoint{4.470892in}{0.739656in}}%
\pgfpathlineto{\pgfqpoint{4.470039in}{0.739656in}}%
\pgfpathlineto{\pgfqpoint{4.469187in}{0.739656in}}%
\pgfpathlineto{\pgfqpoint{4.468334in}{0.739656in}}%
\pgfpathlineto{\pgfqpoint{4.467481in}{0.739656in}}%
\pgfpathlineto{\pgfqpoint{4.466628in}{0.739656in}}%
\pgfpathlineto{\pgfqpoint{4.465775in}{0.739656in}}%
\pgfpathlineto{\pgfqpoint{4.464922in}{0.739656in}}%
\pgfpathlineto{\pgfqpoint{4.464069in}{0.739656in}}%
\pgfpathlineto{\pgfqpoint{4.463216in}{0.739656in}}%
\pgfpathlineto{\pgfqpoint{4.462364in}{0.739656in}}%
\pgfpathlineto{\pgfqpoint{4.461511in}{0.739656in}}%
\pgfpathlineto{\pgfqpoint{4.460658in}{0.739656in}}%
\pgfpathlineto{\pgfqpoint{4.459805in}{0.739656in}}%
\pgfpathlineto{\pgfqpoint{4.458952in}{0.739656in}}%
\pgfpathlineto{\pgfqpoint{4.458099in}{0.739656in}}%
\pgfpathlineto{\pgfqpoint{4.457246in}{0.739656in}}%
\pgfpathlineto{\pgfqpoint{4.456393in}{0.739656in}}%
\pgfpathlineto{\pgfqpoint{4.455541in}{0.739656in}}%
\pgfpathlineto{\pgfqpoint{4.454688in}{0.739656in}}%
\pgfpathlineto{\pgfqpoint{4.453835in}{0.739656in}}%
\pgfpathlineto{\pgfqpoint{4.452982in}{0.739656in}}%
\pgfpathlineto{\pgfqpoint{4.452129in}{0.739656in}}%
\pgfpathlineto{\pgfqpoint{4.451276in}{0.739656in}}%
\pgfpathlineto{\pgfqpoint{4.450423in}{0.739656in}}%
\pgfpathlineto{\pgfqpoint{4.449570in}{0.739656in}}%
\pgfpathlineto{\pgfqpoint{4.448717in}{0.739656in}}%
\pgfpathlineto{\pgfqpoint{4.447865in}{0.739656in}}%
\pgfpathlineto{\pgfqpoint{4.447012in}{0.739656in}}%
\pgfpathlineto{\pgfqpoint{4.446159in}{0.739656in}}%
\pgfpathlineto{\pgfqpoint{4.445306in}{0.739656in}}%
\pgfpathlineto{\pgfqpoint{4.444453in}{0.739656in}}%
\pgfpathlineto{\pgfqpoint{4.443600in}{0.739656in}}%
\pgfpathlineto{\pgfqpoint{4.442747in}{0.739656in}}%
\pgfpathlineto{\pgfqpoint{4.441894in}{0.739656in}}%
\pgfpathlineto{\pgfqpoint{4.441042in}{0.739656in}}%
\pgfpathlineto{\pgfqpoint{4.440189in}{0.739656in}}%
\pgfpathlineto{\pgfqpoint{4.439336in}{0.739656in}}%
\pgfpathlineto{\pgfqpoint{4.438483in}{0.739656in}}%
\pgfpathlineto{\pgfqpoint{4.437630in}{0.739656in}}%
\pgfpathlineto{\pgfqpoint{4.436777in}{0.739656in}}%
\pgfpathlineto{\pgfqpoint{4.435924in}{0.739656in}}%
\pgfpathlineto{\pgfqpoint{4.435071in}{0.739656in}}%
\pgfpathlineto{\pgfqpoint{4.434219in}{0.739656in}}%
\pgfpathlineto{\pgfqpoint{4.433366in}{0.739656in}}%
\pgfpathlineto{\pgfqpoint{4.432513in}{0.739656in}}%
\pgfpathlineto{\pgfqpoint{4.431660in}{0.739656in}}%
\pgfpathlineto{\pgfqpoint{4.430807in}{0.739656in}}%
\pgfpathlineto{\pgfqpoint{4.429954in}{0.739656in}}%
\pgfpathlineto{\pgfqpoint{4.429101in}{0.739656in}}%
\pgfpathlineto{\pgfqpoint{4.428248in}{0.739656in}}%
\pgfpathlineto{\pgfqpoint{4.427396in}{0.739656in}}%
\pgfpathlineto{\pgfqpoint{4.426543in}{0.739656in}}%
\pgfpathlineto{\pgfqpoint{4.425690in}{0.739656in}}%
\pgfpathlineto{\pgfqpoint{4.424837in}{0.739656in}}%
\pgfpathlineto{\pgfqpoint{4.423984in}{0.739656in}}%
\pgfpathlineto{\pgfqpoint{4.423131in}{0.739656in}}%
\pgfpathlineto{\pgfqpoint{4.422278in}{0.739656in}}%
\pgfpathlineto{\pgfqpoint{4.421425in}{0.739656in}}%
\pgfpathlineto{\pgfqpoint{4.420572in}{0.739656in}}%
\pgfpathlineto{\pgfqpoint{4.419720in}{0.739656in}}%
\pgfpathlineto{\pgfqpoint{4.418867in}{0.739656in}}%
\pgfpathlineto{\pgfqpoint{4.418014in}{0.739656in}}%
\pgfpathlineto{\pgfqpoint{4.417161in}{0.739656in}}%
\pgfpathlineto{\pgfqpoint{4.416308in}{0.739656in}}%
\pgfpathlineto{\pgfqpoint{4.415455in}{0.739656in}}%
\pgfpathlineto{\pgfqpoint{4.414602in}{0.739656in}}%
\pgfpathlineto{\pgfqpoint{4.413749in}{0.739656in}}%
\pgfpathlineto{\pgfqpoint{4.412897in}{0.739656in}}%
\pgfpathlineto{\pgfqpoint{4.412044in}{0.739656in}}%
\pgfpathlineto{\pgfqpoint{4.411191in}{0.739656in}}%
\pgfpathlineto{\pgfqpoint{4.410338in}{0.739656in}}%
\pgfpathlineto{\pgfqpoint{4.409485in}{0.739656in}}%
\pgfpathlineto{\pgfqpoint{4.408632in}{0.739656in}}%
\pgfpathlineto{\pgfqpoint{4.407779in}{0.739656in}}%
\pgfpathlineto{\pgfqpoint{4.406926in}{0.739656in}}%
\pgfpathlineto{\pgfqpoint{4.406074in}{0.739656in}}%
\pgfpathlineto{\pgfqpoint{4.405221in}{0.739656in}}%
\pgfpathlineto{\pgfqpoint{4.404368in}{0.739656in}}%
\pgfpathlineto{\pgfqpoint{4.403515in}{0.739656in}}%
\pgfpathlineto{\pgfqpoint{4.402662in}{0.739656in}}%
\pgfpathlineto{\pgfqpoint{4.401809in}{0.739656in}}%
\pgfpathlineto{\pgfqpoint{4.400956in}{0.739656in}}%
\pgfpathlineto{\pgfqpoint{4.400103in}{0.739656in}}%
\pgfpathlineto{\pgfqpoint{4.399251in}{0.739656in}}%
\pgfpathlineto{\pgfqpoint{4.398398in}{0.739656in}}%
\pgfpathlineto{\pgfqpoint{4.397545in}{0.739656in}}%
\pgfpathlineto{\pgfqpoint{4.396692in}{0.739656in}}%
\pgfpathlineto{\pgfqpoint{4.395839in}{0.739656in}}%
\pgfpathlineto{\pgfqpoint{4.394986in}{0.739656in}}%
\pgfpathlineto{\pgfqpoint{4.394133in}{0.739656in}}%
\pgfpathlineto{\pgfqpoint{4.393280in}{0.739656in}}%
\pgfpathlineto{\pgfqpoint{4.392427in}{0.739656in}}%
\pgfpathlineto{\pgfqpoint{4.391575in}{0.739656in}}%
\pgfpathlineto{\pgfqpoint{4.390722in}{0.739656in}}%
\pgfpathlineto{\pgfqpoint{4.389869in}{0.739656in}}%
\pgfpathlineto{\pgfqpoint{4.389016in}{0.739656in}}%
\pgfpathlineto{\pgfqpoint{4.388163in}{0.739656in}}%
\pgfpathlineto{\pgfqpoint{4.387310in}{0.739656in}}%
\pgfpathlineto{\pgfqpoint{4.386457in}{0.739656in}}%
\pgfpathlineto{\pgfqpoint{4.385604in}{0.739656in}}%
\pgfpathlineto{\pgfqpoint{4.384752in}{0.739656in}}%
\pgfpathlineto{\pgfqpoint{4.383899in}{0.739656in}}%
\pgfpathlineto{\pgfqpoint{4.383046in}{0.739656in}}%
\pgfpathlineto{\pgfqpoint{4.382193in}{0.739656in}}%
\pgfpathlineto{\pgfqpoint{4.381340in}{0.739656in}}%
\pgfpathlineto{\pgfqpoint{4.380487in}{0.739656in}}%
\pgfpathlineto{\pgfqpoint{4.379634in}{0.739656in}}%
\pgfpathlineto{\pgfqpoint{4.378781in}{0.739656in}}%
\pgfpathlineto{\pgfqpoint{4.377929in}{0.739656in}}%
\pgfpathlineto{\pgfqpoint{4.377076in}{0.739656in}}%
\pgfpathlineto{\pgfqpoint{4.376223in}{0.739656in}}%
\pgfpathlineto{\pgfqpoint{4.375370in}{0.739656in}}%
\pgfpathlineto{\pgfqpoint{4.374517in}{0.739656in}}%
\pgfpathlineto{\pgfqpoint{4.373664in}{0.739656in}}%
\pgfpathlineto{\pgfqpoint{4.372811in}{0.739656in}}%
\pgfpathlineto{\pgfqpoint{4.371958in}{0.739656in}}%
\pgfpathlineto{\pgfqpoint{4.371106in}{0.739656in}}%
\pgfpathlineto{\pgfqpoint{4.370253in}{0.739656in}}%
\pgfpathlineto{\pgfqpoint{4.369400in}{0.739656in}}%
\pgfpathlineto{\pgfqpoint{4.368547in}{0.739656in}}%
\pgfpathlineto{\pgfqpoint{4.367694in}{0.739656in}}%
\pgfpathlineto{\pgfqpoint{4.366841in}{0.739656in}}%
\pgfpathlineto{\pgfqpoint{4.365988in}{0.739656in}}%
\pgfpathlineto{\pgfqpoint{4.365135in}{0.739656in}}%
\pgfpathlineto{\pgfqpoint{4.364283in}{0.739656in}}%
\pgfpathlineto{\pgfqpoint{4.363430in}{0.739656in}}%
\pgfpathlineto{\pgfqpoint{4.362577in}{0.739656in}}%
\pgfpathlineto{\pgfqpoint{4.361724in}{0.739656in}}%
\pgfpathlineto{\pgfqpoint{4.360871in}{0.739656in}}%
\pgfpathlineto{\pgfqpoint{4.360018in}{0.739656in}}%
\pgfpathlineto{\pgfqpoint{4.359165in}{0.739656in}}%
\pgfpathlineto{\pgfqpoint{4.358312in}{0.739656in}}%
\pgfpathlineto{\pgfqpoint{4.357459in}{0.739656in}}%
\pgfpathlineto{\pgfqpoint{4.356607in}{0.739656in}}%
\pgfpathlineto{\pgfqpoint{4.355754in}{0.739656in}}%
\pgfpathlineto{\pgfqpoint{4.354901in}{0.739656in}}%
\pgfpathlineto{\pgfqpoint{4.354048in}{0.739656in}}%
\pgfpathlineto{\pgfqpoint{4.353195in}{0.739656in}}%
\pgfpathlineto{\pgfqpoint{4.352342in}{0.739656in}}%
\pgfpathlineto{\pgfqpoint{4.351489in}{0.739656in}}%
\pgfpathlineto{\pgfqpoint{4.350636in}{0.739656in}}%
\pgfpathlineto{\pgfqpoint{4.349784in}{0.739656in}}%
\pgfpathlineto{\pgfqpoint{4.348931in}{0.739656in}}%
\pgfpathlineto{\pgfqpoint{4.348078in}{0.739656in}}%
\pgfpathlineto{\pgfqpoint{4.347225in}{0.739656in}}%
\pgfpathlineto{\pgfqpoint{4.346372in}{0.739656in}}%
\pgfpathlineto{\pgfqpoint{4.345519in}{0.739656in}}%
\pgfpathlineto{\pgfqpoint{4.344666in}{0.739656in}}%
\pgfpathlineto{\pgfqpoint{4.343813in}{0.739656in}}%
\pgfpathlineto{\pgfqpoint{4.342961in}{0.739656in}}%
\pgfpathlineto{\pgfqpoint{4.342108in}{0.739656in}}%
\pgfpathlineto{\pgfqpoint{4.341255in}{0.739656in}}%
\pgfpathlineto{\pgfqpoint{4.340402in}{0.739656in}}%
\pgfpathlineto{\pgfqpoint{4.339549in}{0.739656in}}%
\pgfpathlineto{\pgfqpoint{4.338696in}{0.739656in}}%
\pgfpathlineto{\pgfqpoint{4.337843in}{0.739656in}}%
\pgfpathlineto{\pgfqpoint{4.336990in}{0.739656in}}%
\pgfpathlineto{\pgfqpoint{4.336138in}{0.739656in}}%
\pgfpathlineto{\pgfqpoint{4.335285in}{0.739656in}}%
\pgfpathlineto{\pgfqpoint{4.334432in}{0.739656in}}%
\pgfpathlineto{\pgfqpoint{4.333579in}{0.739656in}}%
\pgfpathlineto{\pgfqpoint{4.332726in}{0.739656in}}%
\pgfpathlineto{\pgfqpoint{4.331873in}{0.739656in}}%
\pgfpathlineto{\pgfqpoint{4.331020in}{0.739656in}}%
\pgfpathlineto{\pgfqpoint{4.330167in}{0.739656in}}%
\pgfpathlineto{\pgfqpoint{4.329314in}{0.739656in}}%
\pgfpathlineto{\pgfqpoint{4.328462in}{0.739656in}}%
\pgfpathlineto{\pgfqpoint{4.327609in}{0.739656in}}%
\pgfpathlineto{\pgfqpoint{4.326756in}{0.739656in}}%
\pgfpathlineto{\pgfqpoint{4.325903in}{0.739656in}}%
\pgfpathlineto{\pgfqpoint{4.325050in}{0.739656in}}%
\pgfpathlineto{\pgfqpoint{4.324197in}{0.739656in}}%
\pgfpathlineto{\pgfqpoint{4.323344in}{0.739656in}}%
\pgfpathlineto{\pgfqpoint{4.322491in}{0.739656in}}%
\pgfpathlineto{\pgfqpoint{4.321639in}{0.739656in}}%
\pgfpathlineto{\pgfqpoint{4.320786in}{0.739656in}}%
\pgfpathlineto{\pgfqpoint{4.319933in}{0.739656in}}%
\pgfpathlineto{\pgfqpoint{4.319080in}{0.739656in}}%
\pgfpathlineto{\pgfqpoint{4.318227in}{0.739656in}}%
\pgfpathlineto{\pgfqpoint{4.317374in}{0.739656in}}%
\pgfpathlineto{\pgfqpoint{4.316521in}{0.739656in}}%
\pgfpathlineto{\pgfqpoint{4.315668in}{0.739656in}}%
\pgfpathlineto{\pgfqpoint{4.314816in}{0.739656in}}%
\pgfpathlineto{\pgfqpoint{4.313963in}{0.739656in}}%
\pgfpathlineto{\pgfqpoint{4.313110in}{0.739656in}}%
\pgfpathlineto{\pgfqpoint{4.312257in}{0.739656in}}%
\pgfpathlineto{\pgfqpoint{4.311404in}{0.739656in}}%
\pgfpathlineto{\pgfqpoint{4.310551in}{0.739656in}}%
\pgfpathlineto{\pgfqpoint{4.309698in}{0.739656in}}%
\pgfpathlineto{\pgfqpoint{4.308845in}{0.739656in}}%
\pgfpathlineto{\pgfqpoint{4.307993in}{0.739656in}}%
\pgfpathlineto{\pgfqpoint{4.307140in}{0.739656in}}%
\pgfpathlineto{\pgfqpoint{4.306287in}{0.739656in}}%
\pgfpathlineto{\pgfqpoint{4.305434in}{0.739656in}}%
\pgfpathlineto{\pgfqpoint{4.304581in}{0.739656in}}%
\pgfpathlineto{\pgfqpoint{4.303728in}{0.739656in}}%
\pgfpathlineto{\pgfqpoint{4.302875in}{0.739656in}}%
\pgfpathlineto{\pgfqpoint{4.302022in}{0.739656in}}%
\pgfpathlineto{\pgfqpoint{4.301169in}{0.739656in}}%
\pgfpathlineto{\pgfqpoint{4.300317in}{0.739656in}}%
\pgfpathlineto{\pgfqpoint{4.299464in}{0.739656in}}%
\pgfpathlineto{\pgfqpoint{4.298611in}{0.739656in}}%
\pgfpathlineto{\pgfqpoint{4.297758in}{0.739656in}}%
\pgfpathlineto{\pgfqpoint{4.296905in}{0.739656in}}%
\pgfpathlineto{\pgfqpoint{4.296052in}{0.739656in}}%
\pgfpathlineto{\pgfqpoint{4.295199in}{0.739656in}}%
\pgfpathlineto{\pgfqpoint{4.294346in}{0.739656in}}%
\pgfpathlineto{\pgfqpoint{4.293494in}{0.739656in}}%
\pgfpathlineto{\pgfqpoint{4.292641in}{0.739656in}}%
\pgfpathlineto{\pgfqpoint{4.291788in}{0.739656in}}%
\pgfpathlineto{\pgfqpoint{4.290935in}{0.739656in}}%
\pgfpathlineto{\pgfqpoint{4.290082in}{0.739656in}}%
\pgfpathlineto{\pgfqpoint{4.289229in}{0.739656in}}%
\pgfpathlineto{\pgfqpoint{4.288376in}{0.739656in}}%
\pgfpathlineto{\pgfqpoint{4.287523in}{0.739656in}}%
\pgfpathlineto{\pgfqpoint{4.286671in}{0.739656in}}%
\pgfpathlineto{\pgfqpoint{4.285818in}{0.739656in}}%
\pgfpathlineto{\pgfqpoint{4.284965in}{0.739656in}}%
\pgfpathlineto{\pgfqpoint{4.284112in}{0.739656in}}%
\pgfpathlineto{\pgfqpoint{4.283259in}{0.739656in}}%
\pgfpathlineto{\pgfqpoint{4.282406in}{0.739656in}}%
\pgfpathlineto{\pgfqpoint{4.281553in}{0.739656in}}%
\pgfpathlineto{\pgfqpoint{4.280700in}{0.739656in}}%
\pgfpathlineto{\pgfqpoint{4.279848in}{0.739656in}}%
\pgfpathlineto{\pgfqpoint{4.278995in}{0.739656in}}%
\pgfpathlineto{\pgfqpoint{4.278142in}{0.739656in}}%
\pgfpathlineto{\pgfqpoint{4.277289in}{0.739656in}}%
\pgfpathlineto{\pgfqpoint{4.276436in}{0.739656in}}%
\pgfpathlineto{\pgfqpoint{4.275583in}{0.739656in}}%
\pgfpathlineto{\pgfqpoint{4.274730in}{0.739656in}}%
\pgfpathlineto{\pgfqpoint{4.273877in}{0.739656in}}%
\pgfpathlineto{\pgfqpoint{4.273024in}{0.739656in}}%
\pgfpathlineto{\pgfqpoint{4.272172in}{0.739656in}}%
\pgfpathlineto{\pgfqpoint{4.271319in}{0.739656in}}%
\pgfpathlineto{\pgfqpoint{4.270466in}{0.739656in}}%
\pgfpathlineto{\pgfqpoint{4.269613in}{0.739656in}}%
\pgfpathlineto{\pgfqpoint{4.268760in}{0.739656in}}%
\pgfpathlineto{\pgfqpoint{4.267907in}{0.739656in}}%
\pgfpathlineto{\pgfqpoint{4.267054in}{0.739656in}}%
\pgfpathlineto{\pgfqpoint{4.266201in}{0.739656in}}%
\pgfpathlineto{\pgfqpoint{4.265349in}{0.739656in}}%
\pgfpathlineto{\pgfqpoint{4.264496in}{0.739656in}}%
\pgfpathlineto{\pgfqpoint{4.263643in}{0.739656in}}%
\pgfpathlineto{\pgfqpoint{4.262790in}{0.739656in}}%
\pgfpathlineto{\pgfqpoint{4.261937in}{0.739656in}}%
\pgfpathlineto{\pgfqpoint{4.261084in}{0.739656in}}%
\pgfpathlineto{\pgfqpoint{4.260231in}{0.739656in}}%
\pgfpathlineto{\pgfqpoint{4.259378in}{0.739656in}}%
\pgfpathlineto{\pgfqpoint{4.258526in}{0.739656in}}%
\pgfpathlineto{\pgfqpoint{4.257673in}{0.739656in}}%
\pgfpathlineto{\pgfqpoint{4.256820in}{0.739656in}}%
\pgfpathlineto{\pgfqpoint{4.255967in}{0.739656in}}%
\pgfpathlineto{\pgfqpoint{4.255114in}{0.739656in}}%
\pgfpathlineto{\pgfqpoint{4.254261in}{0.739656in}}%
\pgfpathlineto{\pgfqpoint{4.253408in}{0.739656in}}%
\pgfpathlineto{\pgfqpoint{4.252555in}{0.739656in}}%
\pgfpathlineto{\pgfqpoint{4.251703in}{0.739656in}}%
\pgfpathlineto{\pgfqpoint{4.250850in}{0.739656in}}%
\pgfpathlineto{\pgfqpoint{4.249997in}{0.739656in}}%
\pgfpathlineto{\pgfqpoint{4.249144in}{0.739656in}}%
\pgfpathlineto{\pgfqpoint{4.248291in}{0.739656in}}%
\pgfpathlineto{\pgfqpoint{4.247438in}{0.739656in}}%
\pgfpathlineto{\pgfqpoint{4.246585in}{0.739656in}}%
\pgfpathlineto{\pgfqpoint{4.245732in}{0.739656in}}%
\pgfpathlineto{\pgfqpoint{4.244880in}{0.739656in}}%
\pgfpathlineto{\pgfqpoint{4.244027in}{0.739656in}}%
\pgfpathlineto{\pgfqpoint{4.243174in}{0.739656in}}%
\pgfpathlineto{\pgfqpoint{4.242321in}{0.739656in}}%
\pgfpathlineto{\pgfqpoint{4.241468in}{0.739656in}}%
\pgfpathlineto{\pgfqpoint{4.240615in}{0.739656in}}%
\pgfpathlineto{\pgfqpoint{4.239762in}{0.739656in}}%
\pgfpathlineto{\pgfqpoint{4.238909in}{0.739656in}}%
\pgfpathlineto{\pgfqpoint{4.238056in}{0.739656in}}%
\pgfpathlineto{\pgfqpoint{4.237204in}{0.739656in}}%
\pgfpathlineto{\pgfqpoint{4.236351in}{0.739656in}}%
\pgfpathlineto{\pgfqpoint{4.235498in}{0.739656in}}%
\pgfpathlineto{\pgfqpoint{4.234645in}{0.739656in}}%
\pgfpathlineto{\pgfqpoint{4.233792in}{0.739656in}}%
\pgfpathlineto{\pgfqpoint{4.232939in}{0.739656in}}%
\pgfpathlineto{\pgfqpoint{4.232086in}{0.739656in}}%
\pgfpathlineto{\pgfqpoint{4.231233in}{0.739656in}}%
\pgfpathlineto{\pgfqpoint{4.230381in}{0.739656in}}%
\pgfpathlineto{\pgfqpoint{4.229528in}{0.739656in}}%
\pgfpathlineto{\pgfqpoint{4.228675in}{0.739656in}}%
\pgfpathlineto{\pgfqpoint{4.227822in}{0.739656in}}%
\pgfpathlineto{\pgfqpoint{4.226969in}{0.739656in}}%
\pgfpathlineto{\pgfqpoint{4.226116in}{0.739656in}}%
\pgfpathlineto{\pgfqpoint{4.225263in}{0.739656in}}%
\pgfpathlineto{\pgfqpoint{4.224410in}{0.739656in}}%
\pgfpathlineto{\pgfqpoint{4.223558in}{0.739656in}}%
\pgfpathlineto{\pgfqpoint{4.222705in}{0.739656in}}%
\pgfpathlineto{\pgfqpoint{4.221852in}{0.739656in}}%
\pgfpathlineto{\pgfqpoint{4.220999in}{0.739656in}}%
\pgfpathlineto{\pgfqpoint{4.220146in}{0.739656in}}%
\pgfpathlineto{\pgfqpoint{4.219293in}{0.739656in}}%
\pgfpathlineto{\pgfqpoint{4.218440in}{0.739656in}}%
\pgfpathlineto{\pgfqpoint{4.217587in}{0.739656in}}%
\pgfpathlineto{\pgfqpoint{4.216735in}{0.739656in}}%
\pgfpathlineto{\pgfqpoint{4.215882in}{0.739656in}}%
\pgfpathlineto{\pgfqpoint{4.215029in}{0.739656in}}%
\pgfpathlineto{\pgfqpoint{4.214176in}{0.739656in}}%
\pgfpathlineto{\pgfqpoint{4.213323in}{0.739656in}}%
\pgfpathlineto{\pgfqpoint{4.212470in}{0.739656in}}%
\pgfpathlineto{\pgfqpoint{4.211617in}{0.739656in}}%
\pgfpathlineto{\pgfqpoint{4.210764in}{0.739656in}}%
\pgfpathlineto{\pgfqpoint{4.209911in}{0.739656in}}%
\pgfpathlineto{\pgfqpoint{4.209059in}{0.739656in}}%
\pgfpathlineto{\pgfqpoint{4.208206in}{0.739656in}}%
\pgfpathlineto{\pgfqpoint{4.207353in}{0.739656in}}%
\pgfpathlineto{\pgfqpoint{4.206500in}{0.739656in}}%
\pgfpathlineto{\pgfqpoint{4.205647in}{0.739656in}}%
\pgfpathlineto{\pgfqpoint{4.204794in}{0.739656in}}%
\pgfpathlineto{\pgfqpoint{4.203941in}{0.739656in}}%
\pgfpathlineto{\pgfqpoint{4.203088in}{0.739656in}}%
\pgfpathlineto{\pgfqpoint{4.202236in}{0.739656in}}%
\pgfpathlineto{\pgfqpoint{4.201383in}{0.739656in}}%
\pgfpathlineto{\pgfqpoint{4.200530in}{0.739656in}}%
\pgfpathlineto{\pgfqpoint{4.199677in}{0.739656in}}%
\pgfpathlineto{\pgfqpoint{4.198824in}{0.739656in}}%
\pgfpathlineto{\pgfqpoint{4.197971in}{0.739656in}}%
\pgfpathlineto{\pgfqpoint{4.197118in}{0.739656in}}%
\pgfpathlineto{\pgfqpoint{4.196265in}{0.739656in}}%
\pgfpathlineto{\pgfqpoint{4.195413in}{0.739656in}}%
\pgfpathlineto{\pgfqpoint{4.194560in}{0.739656in}}%
\pgfpathlineto{\pgfqpoint{4.193707in}{0.739656in}}%
\pgfpathlineto{\pgfqpoint{4.192854in}{0.739656in}}%
\pgfpathlineto{\pgfqpoint{4.192001in}{0.739656in}}%
\pgfpathlineto{\pgfqpoint{4.191148in}{0.739656in}}%
\pgfpathlineto{\pgfqpoint{4.190295in}{0.739656in}}%
\pgfpathlineto{\pgfqpoint{4.189442in}{0.739656in}}%
\pgfpathlineto{\pgfqpoint{4.188590in}{0.739656in}}%
\pgfpathlineto{\pgfqpoint{4.187737in}{0.739656in}}%
\pgfpathlineto{\pgfqpoint{4.186884in}{0.739656in}}%
\pgfpathlineto{\pgfqpoint{4.186031in}{0.739656in}}%
\pgfpathlineto{\pgfqpoint{4.185178in}{0.739656in}}%
\pgfpathlineto{\pgfqpoint{4.184325in}{0.739656in}}%
\pgfpathlineto{\pgfqpoint{4.183472in}{0.739656in}}%
\pgfpathlineto{\pgfqpoint{4.182619in}{0.739656in}}%
\pgfpathlineto{\pgfqpoint{4.181766in}{0.739656in}}%
\pgfpathlineto{\pgfqpoint{4.180914in}{0.739656in}}%
\pgfpathlineto{\pgfqpoint{4.180061in}{0.739656in}}%
\pgfpathlineto{\pgfqpoint{4.179208in}{0.739656in}}%
\pgfpathlineto{\pgfqpoint{4.178355in}{0.739656in}}%
\pgfpathlineto{\pgfqpoint{4.177502in}{0.739656in}}%
\pgfpathlineto{\pgfqpoint{4.176649in}{0.739656in}}%
\pgfpathlineto{\pgfqpoint{4.175796in}{0.739656in}}%
\pgfpathlineto{\pgfqpoint{4.174943in}{0.739656in}}%
\pgfpathlineto{\pgfqpoint{4.174091in}{0.739656in}}%
\pgfpathlineto{\pgfqpoint{4.173238in}{0.739656in}}%
\pgfpathlineto{\pgfqpoint{4.172385in}{0.739656in}}%
\pgfpathlineto{\pgfqpoint{4.171532in}{0.739656in}}%
\pgfpathlineto{\pgfqpoint{4.170679in}{0.739656in}}%
\pgfpathlineto{\pgfqpoint{4.169826in}{0.739656in}}%
\pgfpathlineto{\pgfqpoint{4.168973in}{0.739656in}}%
\pgfpathlineto{\pgfqpoint{4.168120in}{0.739656in}}%
\pgfpathlineto{\pgfqpoint{4.167268in}{0.739656in}}%
\pgfpathlineto{\pgfqpoint{4.166415in}{0.739656in}}%
\pgfpathlineto{\pgfqpoint{4.165562in}{0.739656in}}%
\pgfpathlineto{\pgfqpoint{4.164709in}{0.739656in}}%
\pgfpathlineto{\pgfqpoint{4.163856in}{0.739656in}}%
\pgfpathlineto{\pgfqpoint{4.163003in}{0.739656in}}%
\pgfpathlineto{\pgfqpoint{4.162150in}{0.739656in}}%
\pgfpathlineto{\pgfqpoint{4.161297in}{0.739656in}}%
\pgfpathlineto{\pgfqpoint{4.160445in}{0.739656in}}%
\pgfpathlineto{\pgfqpoint{4.159592in}{0.739656in}}%
\pgfpathlineto{\pgfqpoint{4.158739in}{0.739656in}}%
\pgfpathlineto{\pgfqpoint{4.157886in}{0.739656in}}%
\pgfpathlineto{\pgfqpoint{4.157033in}{0.739656in}}%
\pgfpathlineto{\pgfqpoint{4.156180in}{0.739656in}}%
\pgfpathlineto{\pgfqpoint{4.155327in}{0.739656in}}%
\pgfpathlineto{\pgfqpoint{4.154474in}{0.739656in}}%
\pgfpathlineto{\pgfqpoint{4.153622in}{0.739656in}}%
\pgfpathlineto{\pgfqpoint{4.152769in}{0.739656in}}%
\pgfpathlineto{\pgfqpoint{4.151916in}{0.739656in}}%
\pgfpathlineto{\pgfqpoint{4.151063in}{0.739656in}}%
\pgfpathlineto{\pgfqpoint{4.150210in}{0.739656in}}%
\pgfpathlineto{\pgfqpoint{4.149357in}{0.739656in}}%
\pgfpathlineto{\pgfqpoint{4.148504in}{0.739656in}}%
\pgfpathlineto{\pgfqpoint{4.147651in}{0.739656in}}%
\pgfpathlineto{\pgfqpoint{4.146798in}{0.739656in}}%
\pgfpathlineto{\pgfqpoint{4.145946in}{0.739656in}}%
\pgfpathlineto{\pgfqpoint{4.145093in}{0.739656in}}%
\pgfpathlineto{\pgfqpoint{4.144240in}{0.739656in}}%
\pgfpathlineto{\pgfqpoint{4.143387in}{0.739656in}}%
\pgfpathlineto{\pgfqpoint{4.142534in}{0.739656in}}%
\pgfpathlineto{\pgfqpoint{4.141681in}{0.739656in}}%
\pgfpathlineto{\pgfqpoint{4.140828in}{0.739656in}}%
\pgfpathlineto{\pgfqpoint{4.139975in}{0.739656in}}%
\pgfpathlineto{\pgfqpoint{4.139123in}{0.739656in}}%
\pgfpathlineto{\pgfqpoint{4.138270in}{0.739656in}}%
\pgfpathlineto{\pgfqpoint{4.137417in}{0.739656in}}%
\pgfpathlineto{\pgfqpoint{4.136564in}{0.739656in}}%
\pgfpathlineto{\pgfqpoint{4.135711in}{0.739656in}}%
\pgfpathlineto{\pgfqpoint{4.134858in}{0.739656in}}%
\pgfpathlineto{\pgfqpoint{4.134005in}{0.739656in}}%
\pgfpathlineto{\pgfqpoint{4.133152in}{0.739656in}}%
\pgfpathlineto{\pgfqpoint{4.132300in}{0.739656in}}%
\pgfpathlineto{\pgfqpoint{4.131447in}{0.739656in}}%
\pgfpathlineto{\pgfqpoint{4.130594in}{0.739656in}}%
\pgfpathlineto{\pgfqpoint{4.129741in}{0.739656in}}%
\pgfpathlineto{\pgfqpoint{4.128888in}{0.739656in}}%
\pgfpathlineto{\pgfqpoint{4.128035in}{0.739656in}}%
\pgfpathlineto{\pgfqpoint{4.127182in}{0.739656in}}%
\pgfpathlineto{\pgfqpoint{4.126329in}{0.739656in}}%
\pgfpathlineto{\pgfqpoint{4.125477in}{0.739656in}}%
\pgfpathlineto{\pgfqpoint{4.124624in}{0.739656in}}%
\pgfpathlineto{\pgfqpoint{4.123771in}{0.739656in}}%
\pgfpathlineto{\pgfqpoint{4.122918in}{0.739656in}}%
\pgfpathlineto{\pgfqpoint{4.122065in}{0.739656in}}%
\pgfpathlineto{\pgfqpoint{4.121212in}{0.739656in}}%
\pgfpathlineto{\pgfqpoint{4.120359in}{0.739656in}}%
\pgfpathlineto{\pgfqpoint{4.119506in}{0.739656in}}%
\pgfpathlineto{\pgfqpoint{4.118653in}{0.739656in}}%
\pgfpathlineto{\pgfqpoint{4.117801in}{0.739656in}}%
\pgfpathlineto{\pgfqpoint{4.116948in}{0.739656in}}%
\pgfpathlineto{\pgfqpoint{4.116095in}{0.739656in}}%
\pgfpathlineto{\pgfqpoint{4.115242in}{0.739656in}}%
\pgfpathlineto{\pgfqpoint{4.114389in}{0.739656in}}%
\pgfpathlineto{\pgfqpoint{4.113536in}{0.739656in}}%
\pgfpathlineto{\pgfqpoint{4.112683in}{0.739656in}}%
\pgfpathlineto{\pgfqpoint{4.111830in}{0.739656in}}%
\pgfpathlineto{\pgfqpoint{4.110978in}{0.739656in}}%
\pgfpathlineto{\pgfqpoint{4.110125in}{0.739656in}}%
\pgfpathlineto{\pgfqpoint{4.109272in}{0.739656in}}%
\pgfpathlineto{\pgfqpoint{4.108419in}{0.739656in}}%
\pgfpathlineto{\pgfqpoint{4.107566in}{0.739656in}}%
\pgfpathlineto{\pgfqpoint{4.106713in}{0.739656in}}%
\pgfpathlineto{\pgfqpoint{4.105860in}{0.739656in}}%
\pgfpathlineto{\pgfqpoint{4.105007in}{0.739656in}}%
\pgfpathlineto{\pgfqpoint{4.104155in}{0.739656in}}%
\pgfpathlineto{\pgfqpoint{4.103302in}{0.739656in}}%
\pgfpathlineto{\pgfqpoint{4.102449in}{0.739656in}}%
\pgfpathlineto{\pgfqpoint{4.101596in}{0.739656in}}%
\pgfpathlineto{\pgfqpoint{4.100743in}{0.739656in}}%
\pgfpathlineto{\pgfqpoint{4.099890in}{0.739656in}}%
\pgfpathlineto{\pgfqpoint{4.099037in}{0.739656in}}%
\pgfpathlineto{\pgfqpoint{4.098184in}{0.739656in}}%
\pgfpathlineto{\pgfqpoint{4.097332in}{0.739656in}}%
\pgfpathlineto{\pgfqpoint{4.096479in}{0.739656in}}%
\pgfpathlineto{\pgfqpoint{4.095626in}{0.739656in}}%
\pgfpathlineto{\pgfqpoint{4.094773in}{0.739656in}}%
\pgfpathlineto{\pgfqpoint{4.093920in}{0.739656in}}%
\pgfpathlineto{\pgfqpoint{4.093067in}{0.739656in}}%
\pgfpathlineto{\pgfqpoint{4.092214in}{0.739656in}}%
\pgfpathlineto{\pgfqpoint{4.091361in}{0.739656in}}%
\pgfpathlineto{\pgfqpoint{4.090508in}{0.739656in}}%
\pgfpathlineto{\pgfqpoint{4.089656in}{0.739656in}}%
\pgfpathlineto{\pgfqpoint{4.088803in}{0.739656in}}%
\pgfpathlineto{\pgfqpoint{4.087950in}{0.739656in}}%
\pgfpathlineto{\pgfqpoint{4.087097in}{0.739656in}}%
\pgfpathlineto{\pgfqpoint{4.086244in}{0.739656in}}%
\pgfpathlineto{\pgfqpoint{4.085391in}{0.739656in}}%
\pgfpathlineto{\pgfqpoint{4.084538in}{0.739656in}}%
\pgfpathlineto{\pgfqpoint{4.083685in}{0.739656in}}%
\pgfpathlineto{\pgfqpoint{4.082833in}{0.739656in}}%
\pgfpathlineto{\pgfqpoint{4.081980in}{0.739656in}}%
\pgfpathlineto{\pgfqpoint{4.081127in}{0.739656in}}%
\pgfpathlineto{\pgfqpoint{4.080274in}{0.739656in}}%
\pgfpathlineto{\pgfqpoint{4.079421in}{0.739656in}}%
\pgfpathlineto{\pgfqpoint{4.078568in}{0.739656in}}%
\pgfpathlineto{\pgfqpoint{4.077715in}{0.739656in}}%
\pgfpathlineto{\pgfqpoint{4.076862in}{0.739656in}}%
\pgfpathlineto{\pgfqpoint{4.076010in}{0.739656in}}%
\pgfpathlineto{\pgfqpoint{4.075157in}{0.739656in}}%
\pgfpathlineto{\pgfqpoint{4.074304in}{0.739656in}}%
\pgfpathlineto{\pgfqpoint{4.073451in}{0.739656in}}%
\pgfpathlineto{\pgfqpoint{4.072598in}{0.739656in}}%
\pgfpathlineto{\pgfqpoint{4.071745in}{0.739656in}}%
\pgfpathlineto{\pgfqpoint{4.070892in}{0.739656in}}%
\pgfpathlineto{\pgfqpoint{4.070039in}{0.739656in}}%
\pgfpathlineto{\pgfqpoint{4.069187in}{0.739656in}}%
\pgfpathlineto{\pgfqpoint{4.068334in}{0.739656in}}%
\pgfpathlineto{\pgfqpoint{4.067481in}{0.739656in}}%
\pgfpathlineto{\pgfqpoint{4.066628in}{0.739656in}}%
\pgfpathlineto{\pgfqpoint{4.065775in}{0.739656in}}%
\pgfpathlineto{\pgfqpoint{4.064922in}{0.739656in}}%
\pgfpathlineto{\pgfqpoint{4.064069in}{0.739656in}}%
\pgfpathlineto{\pgfqpoint{4.063216in}{0.739656in}}%
\pgfpathlineto{\pgfqpoint{4.062363in}{0.739656in}}%
\pgfpathlineto{\pgfqpoint{4.061511in}{0.739656in}}%
\pgfpathlineto{\pgfqpoint{4.060658in}{0.739656in}}%
\pgfpathlineto{\pgfqpoint{4.059805in}{0.739656in}}%
\pgfpathlineto{\pgfqpoint{4.058952in}{0.739656in}}%
\pgfpathlineto{\pgfqpoint{4.058099in}{0.739656in}}%
\pgfpathlineto{\pgfqpoint{4.057246in}{0.739656in}}%
\pgfpathlineto{\pgfqpoint{4.056393in}{0.739656in}}%
\pgfpathlineto{\pgfqpoint{4.055540in}{0.739656in}}%
\pgfpathlineto{\pgfqpoint{4.054688in}{0.739656in}}%
\pgfpathlineto{\pgfqpoint{4.053835in}{0.739656in}}%
\pgfpathlineto{\pgfqpoint{4.052982in}{0.739656in}}%
\pgfpathlineto{\pgfqpoint{4.052129in}{0.739656in}}%
\pgfpathlineto{\pgfqpoint{4.051276in}{0.739656in}}%
\pgfpathlineto{\pgfqpoint{4.050423in}{0.739656in}}%
\pgfpathlineto{\pgfqpoint{4.049570in}{0.739656in}}%
\pgfpathlineto{\pgfqpoint{4.048717in}{0.739656in}}%
\pgfpathlineto{\pgfqpoint{4.047865in}{0.739656in}}%
\pgfpathlineto{\pgfqpoint{4.047012in}{0.739656in}}%
\pgfpathlineto{\pgfqpoint{4.046159in}{0.739656in}}%
\pgfpathlineto{\pgfqpoint{4.045306in}{0.739656in}}%
\pgfpathlineto{\pgfqpoint{4.044453in}{0.739656in}}%
\pgfpathlineto{\pgfqpoint{4.043600in}{0.739656in}}%
\pgfpathlineto{\pgfqpoint{4.042747in}{0.739656in}}%
\pgfpathlineto{\pgfqpoint{4.041894in}{0.739656in}}%
\pgfpathlineto{\pgfqpoint{4.041042in}{0.739656in}}%
\pgfpathlineto{\pgfqpoint{4.040189in}{0.739656in}}%
\pgfpathlineto{\pgfqpoint{4.039336in}{0.739656in}}%
\pgfpathlineto{\pgfqpoint{4.038483in}{0.739656in}}%
\pgfpathlineto{\pgfqpoint{4.037630in}{0.739656in}}%
\pgfpathlineto{\pgfqpoint{4.036777in}{0.739656in}}%
\pgfpathlineto{\pgfqpoint{4.035924in}{0.739656in}}%
\pgfpathlineto{\pgfqpoint{4.035071in}{0.739656in}}%
\pgfpathlineto{\pgfqpoint{4.034219in}{0.739656in}}%
\pgfpathlineto{\pgfqpoint{4.033366in}{0.739656in}}%
\pgfpathlineto{\pgfqpoint{4.032513in}{0.739656in}}%
\pgfpathlineto{\pgfqpoint{4.031660in}{0.739656in}}%
\pgfpathlineto{\pgfqpoint{4.030807in}{0.739656in}}%
\pgfpathlineto{\pgfqpoint{4.029954in}{0.739656in}}%
\pgfpathlineto{\pgfqpoint{4.029101in}{0.739656in}}%
\pgfpathlineto{\pgfqpoint{4.028248in}{0.739656in}}%
\pgfpathlineto{\pgfqpoint{4.027395in}{0.739656in}}%
\pgfpathlineto{\pgfqpoint{4.026543in}{0.739656in}}%
\pgfpathlineto{\pgfqpoint{4.025690in}{0.739656in}}%
\pgfpathlineto{\pgfqpoint{4.024837in}{0.739656in}}%
\pgfpathlineto{\pgfqpoint{4.023984in}{0.739656in}}%
\pgfpathlineto{\pgfqpoint{4.023131in}{0.739656in}}%
\pgfpathlineto{\pgfqpoint{4.022278in}{0.739656in}}%
\pgfpathlineto{\pgfqpoint{4.021425in}{0.739656in}}%
\pgfpathlineto{\pgfqpoint{4.020572in}{0.739656in}}%
\pgfpathlineto{\pgfqpoint{4.019720in}{0.739656in}}%
\pgfpathlineto{\pgfqpoint{4.018867in}{0.739656in}}%
\pgfpathlineto{\pgfqpoint{4.018014in}{0.739656in}}%
\pgfpathlineto{\pgfqpoint{4.017161in}{0.739656in}}%
\pgfpathlineto{\pgfqpoint{4.016308in}{0.739656in}}%
\pgfpathlineto{\pgfqpoint{4.015455in}{0.739656in}}%
\pgfpathlineto{\pgfqpoint{4.014602in}{0.739656in}}%
\pgfpathlineto{\pgfqpoint{4.013749in}{0.739656in}}%
\pgfpathlineto{\pgfqpoint{4.012897in}{0.739656in}}%
\pgfpathlineto{\pgfqpoint{4.012044in}{0.739656in}}%
\pgfpathlineto{\pgfqpoint{4.011191in}{0.739656in}}%
\pgfpathlineto{\pgfqpoint{4.010338in}{0.739656in}}%
\pgfpathlineto{\pgfqpoint{4.009485in}{0.739656in}}%
\pgfpathlineto{\pgfqpoint{4.008632in}{0.739656in}}%
\pgfpathlineto{\pgfqpoint{4.007779in}{0.739656in}}%
\pgfpathlineto{\pgfqpoint{4.006926in}{0.739656in}}%
\pgfpathlineto{\pgfqpoint{4.006074in}{0.739656in}}%
\pgfpathlineto{\pgfqpoint{4.005221in}{0.739656in}}%
\pgfpathlineto{\pgfqpoint{4.004368in}{0.739656in}}%
\pgfpathlineto{\pgfqpoint{4.003515in}{0.739656in}}%
\pgfpathlineto{\pgfqpoint{4.002662in}{0.739656in}}%
\pgfpathlineto{\pgfqpoint{4.001809in}{0.739656in}}%
\pgfpathlineto{\pgfqpoint{4.000956in}{0.739656in}}%
\pgfpathlineto{\pgfqpoint{4.000103in}{0.739656in}}%
\pgfpathlineto{\pgfqpoint{3.999250in}{0.739656in}}%
\pgfpathlineto{\pgfqpoint{3.998398in}{0.739656in}}%
\pgfpathlineto{\pgfqpoint{3.997545in}{0.739656in}}%
\pgfpathlineto{\pgfqpoint{3.996692in}{0.739656in}}%
\pgfpathlineto{\pgfqpoint{3.995839in}{0.739656in}}%
\pgfpathlineto{\pgfqpoint{3.994986in}{0.739656in}}%
\pgfpathlineto{\pgfqpoint{3.994133in}{0.739656in}}%
\pgfpathlineto{\pgfqpoint{3.993280in}{0.739656in}}%
\pgfpathlineto{\pgfqpoint{3.992427in}{0.739656in}}%
\pgfpathlineto{\pgfqpoint{3.991575in}{0.739656in}}%
\pgfpathlineto{\pgfqpoint{3.990722in}{0.739656in}}%
\pgfpathlineto{\pgfqpoint{3.989869in}{0.739656in}}%
\pgfpathlineto{\pgfqpoint{3.989016in}{0.739656in}}%
\pgfpathlineto{\pgfqpoint{3.988163in}{0.739656in}}%
\pgfpathlineto{\pgfqpoint{3.987310in}{0.739656in}}%
\pgfpathlineto{\pgfqpoint{3.986457in}{0.739656in}}%
\pgfpathlineto{\pgfqpoint{3.985604in}{0.739656in}}%
\pgfpathlineto{\pgfqpoint{3.984752in}{0.739656in}}%
\pgfpathlineto{\pgfqpoint{3.983899in}{0.739656in}}%
\pgfpathlineto{\pgfqpoint{3.983046in}{0.739656in}}%
\pgfpathlineto{\pgfqpoint{3.982193in}{0.739656in}}%
\pgfpathlineto{\pgfqpoint{3.981340in}{0.739656in}}%
\pgfpathlineto{\pgfqpoint{3.980487in}{0.739656in}}%
\pgfpathlineto{\pgfqpoint{3.979634in}{0.739656in}}%
\pgfpathlineto{\pgfqpoint{3.978781in}{0.739656in}}%
\pgfpathlineto{\pgfqpoint{3.977929in}{0.739656in}}%
\pgfpathlineto{\pgfqpoint{3.977076in}{0.739656in}}%
\pgfpathlineto{\pgfqpoint{3.976223in}{0.739656in}}%
\pgfpathlineto{\pgfqpoint{3.975370in}{0.739656in}}%
\pgfpathlineto{\pgfqpoint{3.974517in}{0.739656in}}%
\pgfpathlineto{\pgfqpoint{3.973664in}{0.739656in}}%
\pgfpathlineto{\pgfqpoint{3.972811in}{0.739656in}}%
\pgfpathlineto{\pgfqpoint{3.971958in}{0.739656in}}%
\pgfpathlineto{\pgfqpoint{3.971105in}{0.739656in}}%
\pgfpathlineto{\pgfqpoint{3.970253in}{0.739656in}}%
\pgfpathlineto{\pgfqpoint{3.969400in}{0.739656in}}%
\pgfpathlineto{\pgfqpoint{3.968547in}{0.739656in}}%
\pgfpathlineto{\pgfqpoint{3.967694in}{0.739656in}}%
\pgfpathlineto{\pgfqpoint{3.966841in}{0.739656in}}%
\pgfpathlineto{\pgfqpoint{3.965988in}{0.739656in}}%
\pgfpathlineto{\pgfqpoint{3.965135in}{0.739656in}}%
\pgfpathlineto{\pgfqpoint{3.964282in}{0.739656in}}%
\pgfpathlineto{\pgfqpoint{3.963430in}{0.739656in}}%
\pgfpathlineto{\pgfqpoint{3.962577in}{0.739656in}}%
\pgfpathlineto{\pgfqpoint{3.961724in}{0.739656in}}%
\pgfpathlineto{\pgfqpoint{3.960871in}{0.739656in}}%
\pgfpathlineto{\pgfqpoint{3.960018in}{0.739656in}}%
\pgfpathlineto{\pgfqpoint{3.959165in}{0.739656in}}%
\pgfpathlineto{\pgfqpoint{3.958312in}{0.739656in}}%
\pgfpathlineto{\pgfqpoint{3.957459in}{0.739656in}}%
\pgfpathlineto{\pgfqpoint{3.956607in}{0.739656in}}%
\pgfpathlineto{\pgfqpoint{3.955754in}{0.739656in}}%
\pgfpathlineto{\pgfqpoint{3.954901in}{0.739656in}}%
\pgfpathlineto{\pgfqpoint{3.954048in}{0.739656in}}%
\pgfpathlineto{\pgfqpoint{3.953195in}{0.739656in}}%
\pgfpathlineto{\pgfqpoint{3.952342in}{0.739656in}}%
\pgfpathlineto{\pgfqpoint{3.951489in}{0.739656in}}%
\pgfpathlineto{\pgfqpoint{3.950636in}{0.739656in}}%
\pgfpathlineto{\pgfqpoint{3.949784in}{0.739656in}}%
\pgfpathlineto{\pgfqpoint{3.948931in}{0.739656in}}%
\pgfpathlineto{\pgfqpoint{3.948078in}{0.739656in}}%
\pgfpathlineto{\pgfqpoint{3.947225in}{0.739656in}}%
\pgfpathlineto{\pgfqpoint{3.946372in}{0.739656in}}%
\pgfpathlineto{\pgfqpoint{3.945519in}{0.739656in}}%
\pgfpathlineto{\pgfqpoint{3.944666in}{0.739656in}}%
\pgfpathlineto{\pgfqpoint{3.943813in}{0.739656in}}%
\pgfpathlineto{\pgfqpoint{3.942960in}{0.739656in}}%
\pgfpathlineto{\pgfqpoint{3.942108in}{0.739656in}}%
\pgfpathlineto{\pgfqpoint{3.941255in}{0.739656in}}%
\pgfpathlineto{\pgfqpoint{3.940402in}{0.739656in}}%
\pgfpathlineto{\pgfqpoint{3.939549in}{0.739656in}}%
\pgfpathlineto{\pgfqpoint{3.938696in}{0.739656in}}%
\pgfpathlineto{\pgfqpoint{3.937843in}{0.739656in}}%
\pgfpathlineto{\pgfqpoint{3.936990in}{0.739656in}}%
\pgfpathlineto{\pgfqpoint{3.936137in}{0.739656in}}%
\pgfpathlineto{\pgfqpoint{3.935285in}{0.739656in}}%
\pgfpathlineto{\pgfqpoint{3.934432in}{0.739656in}}%
\pgfpathlineto{\pgfqpoint{3.933579in}{0.739656in}}%
\pgfpathlineto{\pgfqpoint{3.932726in}{0.739656in}}%
\pgfpathlineto{\pgfqpoint{3.931873in}{0.739656in}}%
\pgfpathlineto{\pgfqpoint{3.931020in}{0.739656in}}%
\pgfpathlineto{\pgfqpoint{3.930167in}{0.739656in}}%
\pgfpathlineto{\pgfqpoint{3.929314in}{0.739656in}}%
\pgfpathlineto{\pgfqpoint{3.928462in}{0.739656in}}%
\pgfpathlineto{\pgfqpoint{3.927609in}{0.739656in}}%
\pgfpathlineto{\pgfqpoint{3.926756in}{0.739656in}}%
\pgfpathlineto{\pgfqpoint{3.925903in}{0.739656in}}%
\pgfpathlineto{\pgfqpoint{3.925050in}{0.739656in}}%
\pgfpathlineto{\pgfqpoint{3.924197in}{0.739656in}}%
\pgfpathlineto{\pgfqpoint{3.923344in}{0.739656in}}%
\pgfpathlineto{\pgfqpoint{3.922491in}{0.739656in}}%
\pgfpathlineto{\pgfqpoint{3.921639in}{0.739656in}}%
\pgfpathlineto{\pgfqpoint{3.920786in}{0.739656in}}%
\pgfpathlineto{\pgfqpoint{3.919933in}{0.739656in}}%
\pgfpathlineto{\pgfqpoint{3.919080in}{0.739656in}}%
\pgfpathlineto{\pgfqpoint{3.918227in}{0.739656in}}%
\pgfpathlineto{\pgfqpoint{3.917374in}{0.739656in}}%
\pgfpathlineto{\pgfqpoint{3.916521in}{0.739656in}}%
\pgfpathlineto{\pgfqpoint{3.915668in}{0.739656in}}%
\pgfpathlineto{\pgfqpoint{3.914816in}{0.739656in}}%
\pgfpathlineto{\pgfqpoint{3.913963in}{0.739656in}}%
\pgfpathlineto{\pgfqpoint{3.913110in}{0.739656in}}%
\pgfpathlineto{\pgfqpoint{3.912257in}{0.739656in}}%
\pgfpathlineto{\pgfqpoint{3.911404in}{0.739656in}}%
\pgfpathlineto{\pgfqpoint{3.910551in}{0.739656in}}%
\pgfpathlineto{\pgfqpoint{3.909698in}{0.739656in}}%
\pgfpathlineto{\pgfqpoint{3.908845in}{0.739656in}}%
\pgfpathlineto{\pgfqpoint{3.907992in}{0.739656in}}%
\pgfpathlineto{\pgfqpoint{3.907140in}{0.739656in}}%
\pgfpathlineto{\pgfqpoint{3.906287in}{0.739656in}}%
\pgfpathlineto{\pgfqpoint{3.905434in}{0.739656in}}%
\pgfpathlineto{\pgfqpoint{3.904581in}{0.739656in}}%
\pgfpathlineto{\pgfqpoint{3.903728in}{0.739656in}}%
\pgfpathlineto{\pgfqpoint{3.902875in}{0.739656in}}%
\pgfpathlineto{\pgfqpoint{3.902022in}{0.739656in}}%
\pgfpathlineto{\pgfqpoint{3.901169in}{0.739656in}}%
\pgfpathlineto{\pgfqpoint{3.900317in}{0.739656in}}%
\pgfpathlineto{\pgfqpoint{3.899464in}{0.739656in}}%
\pgfpathlineto{\pgfqpoint{3.898611in}{0.739656in}}%
\pgfpathlineto{\pgfqpoint{3.897758in}{0.739656in}}%
\pgfpathlineto{\pgfqpoint{3.896905in}{0.739656in}}%
\pgfpathlineto{\pgfqpoint{3.896052in}{0.739656in}}%
\pgfpathlineto{\pgfqpoint{3.895199in}{0.739656in}}%
\pgfpathlineto{\pgfqpoint{3.894346in}{0.739656in}}%
\pgfpathlineto{\pgfqpoint{3.893494in}{0.739656in}}%
\pgfpathlineto{\pgfqpoint{3.892641in}{0.739656in}}%
\pgfpathlineto{\pgfqpoint{3.891788in}{0.739656in}}%
\pgfpathlineto{\pgfqpoint{3.890935in}{0.739656in}}%
\pgfpathlineto{\pgfqpoint{3.890082in}{0.739656in}}%
\pgfpathlineto{\pgfqpoint{3.889229in}{0.739656in}}%
\pgfpathlineto{\pgfqpoint{3.888376in}{0.739656in}}%
\pgfpathlineto{\pgfqpoint{3.887523in}{0.739656in}}%
\pgfpathlineto{\pgfqpoint{3.886671in}{0.739656in}}%
\pgfpathlineto{\pgfqpoint{3.885818in}{0.739656in}}%
\pgfpathlineto{\pgfqpoint{3.884965in}{0.739656in}}%
\pgfpathlineto{\pgfqpoint{3.884112in}{0.739656in}}%
\pgfpathlineto{\pgfqpoint{3.883259in}{0.739656in}}%
\pgfpathlineto{\pgfqpoint{3.882406in}{0.739656in}}%
\pgfpathlineto{\pgfqpoint{3.881553in}{0.739656in}}%
\pgfpathlineto{\pgfqpoint{3.880700in}{0.739656in}}%
\pgfpathlineto{\pgfqpoint{3.879847in}{0.739656in}}%
\pgfpathlineto{\pgfqpoint{3.878995in}{0.739656in}}%
\pgfpathlineto{\pgfqpoint{3.878142in}{0.739656in}}%
\pgfpathlineto{\pgfqpoint{3.877289in}{0.739656in}}%
\pgfpathlineto{\pgfqpoint{3.876436in}{0.739656in}}%
\pgfpathlineto{\pgfqpoint{3.875583in}{0.739656in}}%
\pgfpathlineto{\pgfqpoint{3.874730in}{0.739656in}}%
\pgfpathlineto{\pgfqpoint{3.873877in}{0.739656in}}%
\pgfpathlineto{\pgfqpoint{3.873024in}{0.739656in}}%
\pgfpathlineto{\pgfqpoint{3.872172in}{0.739656in}}%
\pgfpathlineto{\pgfqpoint{3.871319in}{0.739656in}}%
\pgfpathlineto{\pgfqpoint{3.870466in}{0.739656in}}%
\pgfpathlineto{\pgfqpoint{3.869613in}{0.739656in}}%
\pgfpathlineto{\pgfqpoint{3.868760in}{0.739656in}}%
\pgfpathlineto{\pgfqpoint{3.867907in}{0.739656in}}%
\pgfpathlineto{\pgfqpoint{3.867054in}{0.739656in}}%
\pgfpathlineto{\pgfqpoint{3.866201in}{0.739656in}}%
\pgfpathlineto{\pgfqpoint{3.865349in}{0.739656in}}%
\pgfpathlineto{\pgfqpoint{3.864496in}{0.739656in}}%
\pgfpathlineto{\pgfqpoint{3.863643in}{0.739656in}}%
\pgfpathlineto{\pgfqpoint{3.862790in}{0.739656in}}%
\pgfpathlineto{\pgfqpoint{3.861937in}{0.739656in}}%
\pgfpathlineto{\pgfqpoint{3.861084in}{0.739656in}}%
\pgfpathlineto{\pgfqpoint{3.860231in}{0.739656in}}%
\pgfpathlineto{\pgfqpoint{3.859378in}{0.739656in}}%
\pgfpathlineto{\pgfqpoint{3.858526in}{0.739656in}}%
\pgfpathlineto{\pgfqpoint{3.857673in}{0.739656in}}%
\pgfpathlineto{\pgfqpoint{3.856820in}{0.739656in}}%
\pgfpathlineto{\pgfqpoint{3.855967in}{0.739656in}}%
\pgfpathlineto{\pgfqpoint{3.855114in}{0.739656in}}%
\pgfpathlineto{\pgfqpoint{3.854261in}{0.739656in}}%
\pgfpathlineto{\pgfqpoint{3.853408in}{0.739656in}}%
\pgfpathlineto{\pgfqpoint{3.852555in}{0.739656in}}%
\pgfpathlineto{\pgfqpoint{3.851702in}{0.739656in}}%
\pgfpathlineto{\pgfqpoint{3.850850in}{0.739656in}}%
\pgfpathlineto{\pgfqpoint{3.849997in}{0.739656in}}%
\pgfpathlineto{\pgfqpoint{3.849144in}{0.739656in}}%
\pgfpathlineto{\pgfqpoint{3.848291in}{0.739656in}}%
\pgfpathlineto{\pgfqpoint{3.847438in}{0.739656in}}%
\pgfpathlineto{\pgfqpoint{3.846585in}{0.739656in}}%
\pgfpathlineto{\pgfqpoint{3.845732in}{0.739656in}}%
\pgfpathlineto{\pgfqpoint{3.844879in}{0.739656in}}%
\pgfpathlineto{\pgfqpoint{3.844027in}{0.739656in}}%
\pgfpathlineto{\pgfqpoint{3.843174in}{0.739656in}}%
\pgfpathlineto{\pgfqpoint{3.842321in}{0.739656in}}%
\pgfpathlineto{\pgfqpoint{3.841468in}{0.739656in}}%
\pgfpathlineto{\pgfqpoint{3.840615in}{0.739656in}}%
\pgfpathlineto{\pgfqpoint{3.839762in}{0.739656in}}%
\pgfpathlineto{\pgfqpoint{3.838909in}{0.739656in}}%
\pgfpathlineto{\pgfqpoint{3.838056in}{0.739656in}}%
\pgfpathlineto{\pgfqpoint{3.837204in}{0.739656in}}%
\pgfpathlineto{\pgfqpoint{3.836351in}{0.739656in}}%
\pgfpathlineto{\pgfqpoint{3.835498in}{0.739656in}}%
\pgfpathlineto{\pgfqpoint{3.834645in}{0.739656in}}%
\pgfpathlineto{\pgfqpoint{3.833792in}{0.739656in}}%
\pgfpathlineto{\pgfqpoint{3.832939in}{0.739656in}}%
\pgfpathlineto{\pgfqpoint{3.832086in}{0.739656in}}%
\pgfpathlineto{\pgfqpoint{3.831233in}{0.739656in}}%
\pgfpathlineto{\pgfqpoint{3.830381in}{0.739656in}}%
\pgfpathlineto{\pgfqpoint{3.829528in}{0.739656in}}%
\pgfpathlineto{\pgfqpoint{3.828675in}{0.739656in}}%
\pgfpathlineto{\pgfqpoint{3.827822in}{0.739656in}}%
\pgfpathlineto{\pgfqpoint{3.826969in}{0.739656in}}%
\pgfpathlineto{\pgfqpoint{3.826116in}{0.739656in}}%
\pgfpathlineto{\pgfqpoint{3.825263in}{0.739656in}}%
\pgfpathlineto{\pgfqpoint{3.824410in}{0.739656in}}%
\pgfpathlineto{\pgfqpoint{3.823558in}{0.739656in}}%
\pgfpathlineto{\pgfqpoint{3.822705in}{0.739656in}}%
\pgfpathlineto{\pgfqpoint{3.821852in}{0.739656in}}%
\pgfpathlineto{\pgfqpoint{3.820999in}{0.739656in}}%
\pgfpathlineto{\pgfqpoint{3.820146in}{0.739656in}}%
\pgfpathlineto{\pgfqpoint{3.819293in}{0.739656in}}%
\pgfpathlineto{\pgfqpoint{3.818440in}{0.739656in}}%
\pgfpathlineto{\pgfqpoint{3.817587in}{0.739656in}}%
\pgfpathlineto{\pgfqpoint{3.816734in}{0.739656in}}%
\pgfpathlineto{\pgfqpoint{3.815882in}{0.739656in}}%
\pgfpathlineto{\pgfqpoint{3.815029in}{0.739656in}}%
\pgfpathlineto{\pgfqpoint{3.814176in}{0.739656in}}%
\pgfpathlineto{\pgfqpoint{3.813323in}{0.739656in}}%
\pgfpathlineto{\pgfqpoint{3.812470in}{0.739656in}}%
\pgfpathlineto{\pgfqpoint{3.811617in}{0.739656in}}%
\pgfpathlineto{\pgfqpoint{3.810764in}{0.739656in}}%
\pgfpathlineto{\pgfqpoint{3.809911in}{0.739656in}}%
\pgfpathlineto{\pgfqpoint{3.809059in}{0.739656in}}%
\pgfpathlineto{\pgfqpoint{3.808206in}{0.739656in}}%
\pgfpathlineto{\pgfqpoint{3.807353in}{0.739656in}}%
\pgfpathlineto{\pgfqpoint{3.806500in}{0.739656in}}%
\pgfpathlineto{\pgfqpoint{3.805647in}{0.739656in}}%
\pgfpathlineto{\pgfqpoint{3.804794in}{0.739656in}}%
\pgfpathlineto{\pgfqpoint{3.803941in}{0.739656in}}%
\pgfpathlineto{\pgfqpoint{3.803088in}{0.739656in}}%
\pgfpathlineto{\pgfqpoint{3.802236in}{0.739656in}}%
\pgfpathlineto{\pgfqpoint{3.801383in}{0.739656in}}%
\pgfpathlineto{\pgfqpoint{3.800530in}{0.739656in}}%
\pgfpathlineto{\pgfqpoint{3.799677in}{0.739656in}}%
\pgfpathlineto{\pgfqpoint{3.798824in}{0.739656in}}%
\pgfpathlineto{\pgfqpoint{3.797971in}{0.739656in}}%
\pgfpathlineto{\pgfqpoint{3.797118in}{0.739656in}}%
\pgfpathlineto{\pgfqpoint{3.796265in}{0.739656in}}%
\pgfpathlineto{\pgfqpoint{3.795413in}{0.739656in}}%
\pgfpathlineto{\pgfqpoint{3.794560in}{0.739656in}}%
\pgfpathlineto{\pgfqpoint{3.793707in}{0.739656in}}%
\pgfpathlineto{\pgfqpoint{3.792854in}{0.739656in}}%
\pgfpathlineto{\pgfqpoint{3.792001in}{0.739656in}}%
\pgfpathlineto{\pgfqpoint{3.791148in}{0.739656in}}%
\pgfpathlineto{\pgfqpoint{3.790295in}{0.739656in}}%
\pgfpathlineto{\pgfqpoint{3.789442in}{0.739656in}}%
\pgfpathlineto{\pgfqpoint{3.788589in}{0.739656in}}%
\pgfpathlineto{\pgfqpoint{3.787737in}{0.739656in}}%
\pgfpathlineto{\pgfqpoint{3.786884in}{0.739656in}}%
\pgfpathlineto{\pgfqpoint{3.786031in}{0.739656in}}%
\pgfpathlineto{\pgfqpoint{3.785178in}{0.739656in}}%
\pgfpathlineto{\pgfqpoint{3.784325in}{0.739656in}}%
\pgfpathlineto{\pgfqpoint{3.783472in}{0.739656in}}%
\pgfpathlineto{\pgfqpoint{3.782619in}{0.739656in}}%
\pgfpathlineto{\pgfqpoint{3.781766in}{0.739656in}}%
\pgfpathlineto{\pgfqpoint{3.780914in}{0.739656in}}%
\pgfpathlineto{\pgfqpoint{3.780061in}{0.739656in}}%
\pgfpathlineto{\pgfqpoint{3.779208in}{0.739656in}}%
\pgfpathlineto{\pgfqpoint{3.778355in}{0.739656in}}%
\pgfpathlineto{\pgfqpoint{3.777502in}{0.739656in}}%
\pgfpathlineto{\pgfqpoint{3.776649in}{0.739656in}}%
\pgfpathlineto{\pgfqpoint{3.775796in}{0.739656in}}%
\pgfpathlineto{\pgfqpoint{3.774943in}{0.739656in}}%
\pgfpathlineto{\pgfqpoint{3.774091in}{0.739656in}}%
\pgfpathlineto{\pgfqpoint{3.773238in}{0.739656in}}%
\pgfpathlineto{\pgfqpoint{3.772385in}{0.739656in}}%
\pgfpathlineto{\pgfqpoint{3.771532in}{0.739656in}}%
\pgfpathlineto{\pgfqpoint{3.770679in}{0.739656in}}%
\pgfpathlineto{\pgfqpoint{3.769826in}{0.739656in}}%
\pgfpathlineto{\pgfqpoint{3.768973in}{0.739656in}}%
\pgfpathlineto{\pgfqpoint{3.768120in}{0.739656in}}%
\pgfpathlineto{\pgfqpoint{3.767268in}{0.739656in}}%
\pgfpathlineto{\pgfqpoint{3.766415in}{0.739656in}}%
\pgfpathlineto{\pgfqpoint{3.765562in}{0.739656in}}%
\pgfpathlineto{\pgfqpoint{3.764709in}{0.739656in}}%
\pgfpathlineto{\pgfqpoint{3.763856in}{0.739656in}}%
\pgfpathlineto{\pgfqpoint{3.763003in}{0.739656in}}%
\pgfpathlineto{\pgfqpoint{3.762150in}{0.739656in}}%
\pgfpathlineto{\pgfqpoint{3.761297in}{0.739656in}}%
\pgfpathlineto{\pgfqpoint{3.760444in}{0.739656in}}%
\pgfpathlineto{\pgfqpoint{3.759592in}{0.739656in}}%
\pgfpathlineto{\pgfqpoint{3.758739in}{0.739656in}}%
\pgfpathlineto{\pgfqpoint{3.757886in}{0.739656in}}%
\pgfpathlineto{\pgfqpoint{3.757033in}{0.739656in}}%
\pgfpathlineto{\pgfqpoint{3.756180in}{0.739656in}}%
\pgfpathlineto{\pgfqpoint{3.755327in}{0.739656in}}%
\pgfpathlineto{\pgfqpoint{3.754474in}{0.739656in}}%
\pgfpathlineto{\pgfqpoint{3.753621in}{0.739656in}}%
\pgfpathlineto{\pgfqpoint{3.752769in}{0.739656in}}%
\pgfpathlineto{\pgfqpoint{3.751916in}{0.739656in}}%
\pgfpathlineto{\pgfqpoint{3.751063in}{0.739656in}}%
\pgfpathlineto{\pgfqpoint{3.750210in}{0.739656in}}%
\pgfpathlineto{\pgfqpoint{3.749357in}{0.739656in}}%
\pgfpathlineto{\pgfqpoint{3.748504in}{0.739656in}}%
\pgfpathlineto{\pgfqpoint{3.747651in}{0.739656in}}%
\pgfpathlineto{\pgfqpoint{3.746798in}{0.739656in}}%
\pgfpathlineto{\pgfqpoint{3.745946in}{0.739656in}}%
\pgfpathlineto{\pgfqpoint{3.745093in}{0.739656in}}%
\pgfpathlineto{\pgfqpoint{3.744240in}{0.739656in}}%
\pgfpathlineto{\pgfqpoint{3.743387in}{0.739656in}}%
\pgfpathlineto{\pgfqpoint{3.742534in}{0.739656in}}%
\pgfpathlineto{\pgfqpoint{3.741681in}{0.739656in}}%
\pgfpathlineto{\pgfqpoint{3.740828in}{0.739656in}}%
\pgfpathlineto{\pgfqpoint{3.739975in}{0.739656in}}%
\pgfpathlineto{\pgfqpoint{3.739123in}{0.739656in}}%
\pgfpathlineto{\pgfqpoint{3.738270in}{0.739656in}}%
\pgfpathlineto{\pgfqpoint{3.737417in}{0.739656in}}%
\pgfpathlineto{\pgfqpoint{3.736564in}{0.739656in}}%
\pgfpathlineto{\pgfqpoint{3.735711in}{0.739656in}}%
\pgfpathlineto{\pgfqpoint{3.734858in}{0.739656in}}%
\pgfpathlineto{\pgfqpoint{3.734005in}{0.739656in}}%
\pgfpathlineto{\pgfqpoint{3.733152in}{0.739656in}}%
\pgfpathlineto{\pgfqpoint{3.732299in}{0.739656in}}%
\pgfpathlineto{\pgfqpoint{3.731447in}{0.739656in}}%
\pgfpathlineto{\pgfqpoint{3.730594in}{0.739656in}}%
\pgfpathlineto{\pgfqpoint{3.729741in}{0.739656in}}%
\pgfpathlineto{\pgfqpoint{3.728888in}{0.739656in}}%
\pgfpathlineto{\pgfqpoint{3.728035in}{0.739656in}}%
\pgfpathlineto{\pgfqpoint{3.727182in}{0.739656in}}%
\pgfpathlineto{\pgfqpoint{3.726329in}{0.739656in}}%
\pgfpathlineto{\pgfqpoint{3.725476in}{0.739656in}}%
\pgfpathlineto{\pgfqpoint{3.724624in}{0.739656in}}%
\pgfpathlineto{\pgfqpoint{3.723771in}{0.739656in}}%
\pgfpathlineto{\pgfqpoint{3.722918in}{0.739656in}}%
\pgfpathlineto{\pgfqpoint{3.722065in}{0.739656in}}%
\pgfpathlineto{\pgfqpoint{3.721212in}{0.739656in}}%
\pgfpathlineto{\pgfqpoint{3.720359in}{0.739656in}}%
\pgfpathlineto{\pgfqpoint{3.719506in}{0.739656in}}%
\pgfpathlineto{\pgfqpoint{3.718653in}{0.739656in}}%
\pgfpathlineto{\pgfqpoint{3.717801in}{0.739656in}}%
\pgfpathlineto{\pgfqpoint{3.716948in}{0.739656in}}%
\pgfpathlineto{\pgfqpoint{3.716095in}{0.739656in}}%
\pgfpathlineto{\pgfqpoint{3.715242in}{0.739656in}}%
\pgfpathlineto{\pgfqpoint{3.714389in}{0.739656in}}%
\pgfpathlineto{\pgfqpoint{3.713536in}{0.739656in}}%
\pgfpathlineto{\pgfqpoint{3.712683in}{0.739656in}}%
\pgfpathlineto{\pgfqpoint{3.711830in}{0.739656in}}%
\pgfpathlineto{\pgfqpoint{3.710978in}{0.739656in}}%
\pgfpathlineto{\pgfqpoint{3.710125in}{0.739656in}}%
\pgfpathlineto{\pgfqpoint{3.709272in}{0.739656in}}%
\pgfpathlineto{\pgfqpoint{3.708419in}{0.739656in}}%
\pgfpathlineto{\pgfqpoint{3.707566in}{0.739656in}}%
\pgfpathlineto{\pgfqpoint{3.706713in}{0.739656in}}%
\pgfpathlineto{\pgfqpoint{3.705860in}{0.739656in}}%
\pgfpathlineto{\pgfqpoint{3.705007in}{0.739656in}}%
\pgfpathlineto{\pgfqpoint{3.704155in}{0.739656in}}%
\pgfpathlineto{\pgfqpoint{3.703302in}{0.739656in}}%
\pgfpathlineto{\pgfqpoint{3.702449in}{0.739656in}}%
\pgfpathlineto{\pgfqpoint{3.701596in}{0.739656in}}%
\pgfpathlineto{\pgfqpoint{3.700743in}{0.739656in}}%
\pgfpathlineto{\pgfqpoint{3.699890in}{0.739656in}}%
\pgfpathlineto{\pgfqpoint{3.699037in}{0.739656in}}%
\pgfpathlineto{\pgfqpoint{3.698184in}{0.739656in}}%
\pgfpathlineto{\pgfqpoint{3.697331in}{0.739656in}}%
\pgfpathlineto{\pgfqpoint{3.696479in}{0.739656in}}%
\pgfpathlineto{\pgfqpoint{3.695626in}{0.739656in}}%
\pgfpathlineto{\pgfqpoint{3.694773in}{0.739656in}}%
\pgfpathlineto{\pgfqpoint{3.693920in}{0.739656in}}%
\pgfpathlineto{\pgfqpoint{3.693067in}{0.739656in}}%
\pgfpathlineto{\pgfqpoint{3.692214in}{0.739656in}}%
\pgfpathlineto{\pgfqpoint{3.691361in}{0.739656in}}%
\pgfpathlineto{\pgfqpoint{3.690508in}{0.739656in}}%
\pgfpathlineto{\pgfqpoint{3.689656in}{0.739656in}}%
\pgfpathlineto{\pgfqpoint{3.688803in}{0.739656in}}%
\pgfpathlineto{\pgfqpoint{3.687950in}{0.739656in}}%
\pgfpathlineto{\pgfqpoint{3.687097in}{0.739656in}}%
\pgfpathlineto{\pgfqpoint{3.686244in}{0.739656in}}%
\pgfpathlineto{\pgfqpoint{3.685391in}{0.739656in}}%
\pgfpathlineto{\pgfqpoint{3.684538in}{0.739656in}}%
\pgfpathlineto{\pgfqpoint{3.683685in}{0.739656in}}%
\pgfpathlineto{\pgfqpoint{3.682833in}{0.739656in}}%
\pgfpathlineto{\pgfqpoint{3.681980in}{0.739656in}}%
\pgfpathlineto{\pgfqpoint{3.681127in}{0.739656in}}%
\pgfpathlineto{\pgfqpoint{3.680274in}{0.739656in}}%
\pgfpathlineto{\pgfqpoint{3.679421in}{0.739656in}}%
\pgfpathlineto{\pgfqpoint{3.678568in}{0.739656in}}%
\pgfpathlineto{\pgfqpoint{3.677715in}{0.739656in}}%
\pgfpathlineto{\pgfqpoint{3.676862in}{0.739656in}}%
\pgfpathlineto{\pgfqpoint{3.676010in}{0.739656in}}%
\pgfpathlineto{\pgfqpoint{3.675157in}{0.739656in}}%
\pgfpathlineto{\pgfqpoint{3.674304in}{0.739656in}}%
\pgfpathlineto{\pgfqpoint{3.673451in}{0.739656in}}%
\pgfpathlineto{\pgfqpoint{3.672598in}{0.739656in}}%
\pgfpathlineto{\pgfqpoint{3.671745in}{0.739656in}}%
\pgfpathlineto{\pgfqpoint{3.670892in}{0.739656in}}%
\pgfpathlineto{\pgfqpoint{3.670039in}{0.739656in}}%
\pgfpathlineto{\pgfqpoint{3.669186in}{0.739656in}}%
\pgfpathlineto{\pgfqpoint{3.668334in}{0.739656in}}%
\pgfpathlineto{\pgfqpoint{3.667481in}{0.739656in}}%
\pgfpathlineto{\pgfqpoint{3.666628in}{0.739656in}}%
\pgfpathlineto{\pgfqpoint{3.665775in}{0.739656in}}%
\pgfpathlineto{\pgfqpoint{3.664922in}{0.739656in}}%
\pgfpathlineto{\pgfqpoint{3.664069in}{0.739656in}}%
\pgfpathlineto{\pgfqpoint{3.663216in}{0.739656in}}%
\pgfpathlineto{\pgfqpoint{3.662363in}{0.739656in}}%
\pgfpathlineto{\pgfqpoint{3.661511in}{0.739656in}}%
\pgfpathlineto{\pgfqpoint{3.660658in}{0.739656in}}%
\pgfpathlineto{\pgfqpoint{3.659805in}{0.739656in}}%
\pgfpathlineto{\pgfqpoint{3.658952in}{0.739656in}}%
\pgfpathlineto{\pgfqpoint{3.658099in}{0.739656in}}%
\pgfpathlineto{\pgfqpoint{3.657246in}{0.739656in}}%
\pgfpathlineto{\pgfqpoint{3.656393in}{0.739656in}}%
\pgfpathlineto{\pgfqpoint{3.655540in}{0.739656in}}%
\pgfpathlineto{\pgfqpoint{3.654688in}{0.739656in}}%
\pgfpathlineto{\pgfqpoint{3.653835in}{0.739656in}}%
\pgfpathlineto{\pgfqpoint{3.652982in}{0.739656in}}%
\pgfpathlineto{\pgfqpoint{3.652129in}{0.739656in}}%
\pgfpathlineto{\pgfqpoint{3.651276in}{0.739656in}}%
\pgfpathlineto{\pgfqpoint{3.650423in}{0.739656in}}%
\pgfpathlineto{\pgfqpoint{3.649570in}{0.739656in}}%
\pgfpathlineto{\pgfqpoint{3.648717in}{0.739656in}}%
\pgfpathlineto{\pgfqpoint{3.647865in}{0.739656in}}%
\pgfpathlineto{\pgfqpoint{3.647012in}{0.739656in}}%
\pgfpathlineto{\pgfqpoint{3.646159in}{0.739656in}}%
\pgfpathlineto{\pgfqpoint{3.645306in}{0.739656in}}%
\pgfpathlineto{\pgfqpoint{3.644453in}{0.739656in}}%
\pgfpathlineto{\pgfqpoint{3.643600in}{0.739656in}}%
\pgfpathlineto{\pgfqpoint{3.642747in}{0.739656in}}%
\pgfpathlineto{\pgfqpoint{3.641894in}{0.739656in}}%
\pgfpathlineto{\pgfqpoint{3.641041in}{0.739656in}}%
\pgfpathlineto{\pgfqpoint{3.640189in}{0.739656in}}%
\pgfpathlineto{\pgfqpoint{3.639336in}{0.739656in}}%
\pgfpathlineto{\pgfqpoint{3.638483in}{0.739656in}}%
\pgfpathlineto{\pgfqpoint{3.637630in}{0.739656in}}%
\pgfpathlineto{\pgfqpoint{3.636777in}{0.739656in}}%
\pgfpathlineto{\pgfqpoint{3.635924in}{0.739656in}}%
\pgfpathlineto{\pgfqpoint{3.635071in}{0.739656in}}%
\pgfpathlineto{\pgfqpoint{3.634218in}{0.739656in}}%
\pgfpathlineto{\pgfqpoint{3.633366in}{0.739656in}}%
\pgfpathlineto{\pgfqpoint{3.632513in}{0.739656in}}%
\pgfpathlineto{\pgfqpoint{3.631660in}{0.739656in}}%
\pgfpathlineto{\pgfqpoint{3.630807in}{0.739656in}}%
\pgfpathlineto{\pgfqpoint{3.629954in}{0.739656in}}%
\pgfpathlineto{\pgfqpoint{3.629101in}{0.739656in}}%
\pgfpathlineto{\pgfqpoint{3.628248in}{0.739656in}}%
\pgfpathlineto{\pgfqpoint{3.627395in}{0.739656in}}%
\pgfpathlineto{\pgfqpoint{3.626543in}{0.739656in}}%
\pgfpathlineto{\pgfqpoint{3.625690in}{0.739656in}}%
\pgfpathlineto{\pgfqpoint{3.624837in}{0.739656in}}%
\pgfpathlineto{\pgfqpoint{3.623984in}{0.739656in}}%
\pgfpathlineto{\pgfqpoint{3.623131in}{0.739656in}}%
\pgfpathlineto{\pgfqpoint{3.622278in}{0.739656in}}%
\pgfpathlineto{\pgfqpoint{3.621425in}{0.739656in}}%
\pgfpathlineto{\pgfqpoint{3.620572in}{0.739656in}}%
\pgfpathlineto{\pgfqpoint{3.619720in}{0.739656in}}%
\pgfpathlineto{\pgfqpoint{3.618867in}{0.739656in}}%
\pgfpathlineto{\pgfqpoint{3.618014in}{0.739656in}}%
\pgfpathlineto{\pgfqpoint{3.617161in}{0.739656in}}%
\pgfpathlineto{\pgfqpoint{3.616308in}{0.739656in}}%
\pgfpathlineto{\pgfqpoint{3.615455in}{0.739656in}}%
\pgfpathlineto{\pgfqpoint{3.614602in}{0.739656in}}%
\pgfpathlineto{\pgfqpoint{3.613749in}{0.739656in}}%
\pgfpathlineto{\pgfqpoint{3.612896in}{0.739656in}}%
\pgfpathlineto{\pgfqpoint{3.612044in}{0.739656in}}%
\pgfpathlineto{\pgfqpoint{3.611191in}{0.739656in}}%
\pgfpathlineto{\pgfqpoint{3.610338in}{0.739656in}}%
\pgfpathlineto{\pgfqpoint{3.609485in}{0.739656in}}%
\pgfpathlineto{\pgfqpoint{3.608632in}{0.739656in}}%
\pgfpathlineto{\pgfqpoint{3.607779in}{0.739656in}}%
\pgfpathlineto{\pgfqpoint{3.606926in}{0.739656in}}%
\pgfpathlineto{\pgfqpoint{3.606073in}{0.739656in}}%
\pgfpathlineto{\pgfqpoint{3.605221in}{0.739656in}}%
\pgfpathlineto{\pgfqpoint{3.604368in}{0.739656in}}%
\pgfpathlineto{\pgfqpoint{3.603515in}{0.739656in}}%
\pgfpathlineto{\pgfqpoint{3.602662in}{0.739656in}}%
\pgfpathlineto{\pgfqpoint{3.601809in}{0.739656in}}%
\pgfpathlineto{\pgfqpoint{3.600956in}{0.739656in}}%
\pgfpathlineto{\pgfqpoint{3.600103in}{0.739656in}}%
\pgfpathlineto{\pgfqpoint{3.599250in}{0.739656in}}%
\pgfpathlineto{\pgfqpoint{3.598398in}{0.739656in}}%
\pgfpathlineto{\pgfqpoint{3.597545in}{0.739656in}}%
\pgfpathlineto{\pgfqpoint{3.596692in}{0.739656in}}%
\pgfpathlineto{\pgfqpoint{3.595839in}{0.739656in}}%
\pgfpathlineto{\pgfqpoint{3.594986in}{0.739656in}}%
\pgfpathlineto{\pgfqpoint{3.594133in}{0.739656in}}%
\pgfpathlineto{\pgfqpoint{3.593280in}{0.739656in}}%
\pgfpathlineto{\pgfqpoint{3.592427in}{0.739656in}}%
\pgfpathlineto{\pgfqpoint{3.591575in}{0.739656in}}%
\pgfpathlineto{\pgfqpoint{3.590722in}{0.739656in}}%
\pgfpathlineto{\pgfqpoint{3.589869in}{0.739656in}}%
\pgfpathlineto{\pgfqpoint{3.589016in}{0.739656in}}%
\pgfpathlineto{\pgfqpoint{3.588163in}{0.739656in}}%
\pgfpathlineto{\pgfqpoint{3.587310in}{0.739656in}}%
\pgfpathlineto{\pgfqpoint{3.586457in}{0.739656in}}%
\pgfpathlineto{\pgfqpoint{3.585604in}{0.739656in}}%
\pgfpathlineto{\pgfqpoint{3.584752in}{0.739656in}}%
\pgfpathlineto{\pgfqpoint{3.583899in}{0.739656in}}%
\pgfpathlineto{\pgfqpoint{3.583046in}{0.739656in}}%
\pgfpathlineto{\pgfqpoint{3.582193in}{0.739656in}}%
\pgfpathlineto{\pgfqpoint{3.581340in}{0.739656in}}%
\pgfpathlineto{\pgfqpoint{3.580487in}{0.739656in}}%
\pgfpathlineto{\pgfqpoint{3.579634in}{0.739656in}}%
\pgfpathlineto{\pgfqpoint{3.578781in}{0.739656in}}%
\pgfpathlineto{\pgfqpoint{3.577928in}{0.739656in}}%
\pgfpathlineto{\pgfqpoint{3.577076in}{0.739656in}}%
\pgfpathlineto{\pgfqpoint{3.576223in}{0.739656in}}%
\pgfpathlineto{\pgfqpoint{3.575370in}{0.739656in}}%
\pgfpathlineto{\pgfqpoint{3.574517in}{0.739656in}}%
\pgfpathlineto{\pgfqpoint{3.573664in}{0.739656in}}%
\pgfpathlineto{\pgfqpoint{3.572811in}{0.739656in}}%
\pgfpathlineto{\pgfqpoint{3.571958in}{0.739656in}}%
\pgfpathlineto{\pgfqpoint{3.571105in}{0.739656in}}%
\pgfpathlineto{\pgfqpoint{3.570253in}{0.739656in}}%
\pgfpathlineto{\pgfqpoint{3.569400in}{0.739656in}}%
\pgfpathlineto{\pgfqpoint{3.568547in}{0.739656in}}%
\pgfpathlineto{\pgfqpoint{3.567694in}{0.739656in}}%
\pgfpathlineto{\pgfqpoint{3.566841in}{0.739656in}}%
\pgfpathlineto{\pgfqpoint{3.565988in}{0.739656in}}%
\pgfpathlineto{\pgfqpoint{3.565135in}{0.739656in}}%
\pgfpathlineto{\pgfqpoint{3.564282in}{0.739656in}}%
\pgfpathlineto{\pgfqpoint{3.563430in}{0.739656in}}%
\pgfpathlineto{\pgfqpoint{3.562577in}{0.739656in}}%
\pgfpathlineto{\pgfqpoint{3.561724in}{0.739656in}}%
\pgfpathlineto{\pgfqpoint{3.560871in}{0.739656in}}%
\pgfpathlineto{\pgfqpoint{3.560018in}{0.739656in}}%
\pgfpathlineto{\pgfqpoint{3.559165in}{0.739656in}}%
\pgfpathlineto{\pgfqpoint{3.558312in}{0.739656in}}%
\pgfpathlineto{\pgfqpoint{3.557459in}{0.739656in}}%
\pgfpathlineto{\pgfqpoint{3.556607in}{0.739656in}}%
\pgfpathlineto{\pgfqpoint{3.555754in}{0.739656in}}%
\pgfpathlineto{\pgfqpoint{3.554901in}{0.739656in}}%
\pgfpathlineto{\pgfqpoint{3.554048in}{0.739656in}}%
\pgfpathlineto{\pgfqpoint{3.553195in}{0.739656in}}%
\pgfpathlineto{\pgfqpoint{3.552342in}{0.739656in}}%
\pgfpathlineto{\pgfqpoint{3.551489in}{0.739656in}}%
\pgfpathlineto{\pgfqpoint{3.550636in}{0.739656in}}%
\pgfpathlineto{\pgfqpoint{3.549783in}{0.739656in}}%
\pgfpathlineto{\pgfqpoint{3.548931in}{0.739656in}}%
\pgfpathlineto{\pgfqpoint{3.548078in}{0.739656in}}%
\pgfpathlineto{\pgfqpoint{3.547225in}{0.739656in}}%
\pgfpathlineto{\pgfqpoint{3.546372in}{0.739656in}}%
\pgfpathlineto{\pgfqpoint{3.545519in}{0.739656in}}%
\pgfpathlineto{\pgfqpoint{3.544666in}{0.739656in}}%
\pgfpathlineto{\pgfqpoint{3.543813in}{0.739656in}}%
\pgfpathlineto{\pgfqpoint{3.542960in}{0.739656in}}%
\pgfpathlineto{\pgfqpoint{3.542108in}{0.739656in}}%
\pgfpathlineto{\pgfqpoint{3.541255in}{0.739656in}}%
\pgfpathlineto{\pgfqpoint{3.540402in}{0.739656in}}%
\pgfpathlineto{\pgfqpoint{3.539549in}{0.739656in}}%
\pgfpathlineto{\pgfqpoint{3.538696in}{0.739656in}}%
\pgfpathlineto{\pgfqpoint{3.537843in}{0.739656in}}%
\pgfpathlineto{\pgfqpoint{3.536990in}{0.739656in}}%
\pgfpathlineto{\pgfqpoint{3.536137in}{0.739656in}}%
\pgfpathlineto{\pgfqpoint{3.535285in}{0.739656in}}%
\pgfpathlineto{\pgfqpoint{3.534432in}{0.739656in}}%
\pgfpathlineto{\pgfqpoint{3.533579in}{0.739656in}}%
\pgfpathlineto{\pgfqpoint{3.532726in}{0.739656in}}%
\pgfpathlineto{\pgfqpoint{3.531873in}{0.739656in}}%
\pgfpathlineto{\pgfqpoint{3.531020in}{0.739656in}}%
\pgfpathlineto{\pgfqpoint{3.530167in}{0.739656in}}%
\pgfpathlineto{\pgfqpoint{3.529314in}{0.739656in}}%
\pgfpathlineto{\pgfqpoint{3.528462in}{0.739656in}}%
\pgfpathlineto{\pgfqpoint{3.527609in}{0.739656in}}%
\pgfpathlineto{\pgfqpoint{3.526756in}{0.739656in}}%
\pgfpathlineto{\pgfqpoint{3.525903in}{0.739656in}}%
\pgfpathlineto{\pgfqpoint{3.525050in}{0.739656in}}%
\pgfpathlineto{\pgfqpoint{3.524197in}{0.739656in}}%
\pgfpathlineto{\pgfqpoint{3.523344in}{0.739656in}}%
\pgfpathlineto{\pgfqpoint{3.522491in}{0.739656in}}%
\pgfpathlineto{\pgfqpoint{3.521638in}{0.739656in}}%
\pgfpathlineto{\pgfqpoint{3.520786in}{0.739656in}}%
\pgfpathlineto{\pgfqpoint{3.519933in}{0.739656in}}%
\pgfpathlineto{\pgfqpoint{3.519080in}{0.739656in}}%
\pgfpathlineto{\pgfqpoint{3.518227in}{0.739656in}}%
\pgfpathlineto{\pgfqpoint{3.517374in}{0.739656in}}%
\pgfpathlineto{\pgfqpoint{3.516521in}{0.739656in}}%
\pgfpathlineto{\pgfqpoint{3.515668in}{0.739656in}}%
\pgfpathlineto{\pgfqpoint{3.514815in}{0.739656in}}%
\pgfpathlineto{\pgfqpoint{3.513963in}{0.739656in}}%
\pgfpathlineto{\pgfqpoint{3.513110in}{0.739656in}}%
\pgfpathlineto{\pgfqpoint{3.512257in}{0.739656in}}%
\pgfpathlineto{\pgfqpoint{3.511404in}{0.739656in}}%
\pgfpathlineto{\pgfqpoint{3.510551in}{0.739656in}}%
\pgfpathlineto{\pgfqpoint{3.509698in}{0.739656in}}%
\pgfpathlineto{\pgfqpoint{3.508845in}{0.739656in}}%
\pgfpathlineto{\pgfqpoint{3.507992in}{0.739656in}}%
\pgfpathlineto{\pgfqpoint{3.507140in}{0.739656in}}%
\pgfpathlineto{\pgfqpoint{3.506287in}{0.739656in}}%
\pgfpathlineto{\pgfqpoint{3.505434in}{0.739656in}}%
\pgfpathlineto{\pgfqpoint{3.504581in}{0.739656in}}%
\pgfpathlineto{\pgfqpoint{3.503728in}{0.739656in}}%
\pgfpathlineto{\pgfqpoint{3.502875in}{0.739656in}}%
\pgfpathlineto{\pgfqpoint{3.502022in}{0.739656in}}%
\pgfpathlineto{\pgfqpoint{3.501169in}{0.739656in}}%
\pgfpathlineto{\pgfqpoint{3.500317in}{0.739656in}}%
\pgfpathlineto{\pgfqpoint{3.499464in}{0.739656in}}%
\pgfpathlineto{\pgfqpoint{3.498611in}{0.739656in}}%
\pgfpathlineto{\pgfqpoint{3.497758in}{0.739656in}}%
\pgfpathlineto{\pgfqpoint{3.496905in}{0.739656in}}%
\pgfpathlineto{\pgfqpoint{3.496052in}{0.739656in}}%
\pgfpathlineto{\pgfqpoint{3.495199in}{0.739656in}}%
\pgfpathlineto{\pgfqpoint{3.494346in}{0.739656in}}%
\pgfpathlineto{\pgfqpoint{3.493494in}{0.739656in}}%
\pgfpathlineto{\pgfqpoint{3.492641in}{0.739656in}}%
\pgfpathlineto{\pgfqpoint{3.491788in}{0.739656in}}%
\pgfpathlineto{\pgfqpoint{3.490935in}{0.739656in}}%
\pgfpathlineto{\pgfqpoint{3.490082in}{0.739656in}}%
\pgfpathlineto{\pgfqpoint{3.489229in}{0.739656in}}%
\pgfpathlineto{\pgfqpoint{3.488376in}{0.739656in}}%
\pgfpathlineto{\pgfqpoint{3.487523in}{0.739656in}}%
\pgfpathlineto{\pgfqpoint{3.486670in}{0.739656in}}%
\pgfpathlineto{\pgfqpoint{3.485818in}{0.739656in}}%
\pgfpathlineto{\pgfqpoint{3.484965in}{0.739656in}}%
\pgfpathlineto{\pgfqpoint{3.484112in}{0.739656in}}%
\pgfpathlineto{\pgfqpoint{3.483259in}{0.739656in}}%
\pgfpathlineto{\pgfqpoint{3.482406in}{0.739656in}}%
\pgfpathlineto{\pgfqpoint{3.481553in}{0.739656in}}%
\pgfpathlineto{\pgfqpoint{3.480700in}{0.739656in}}%
\pgfpathlineto{\pgfqpoint{3.479847in}{0.739656in}}%
\pgfpathlineto{\pgfqpoint{3.478995in}{0.739656in}}%
\pgfpathlineto{\pgfqpoint{3.478142in}{0.739656in}}%
\pgfpathlineto{\pgfqpoint{3.477289in}{0.739656in}}%
\pgfpathlineto{\pgfqpoint{3.476436in}{0.739656in}}%
\pgfpathlineto{\pgfqpoint{3.475583in}{0.739656in}}%
\pgfpathlineto{\pgfqpoint{3.474730in}{0.739656in}}%
\pgfpathlineto{\pgfqpoint{3.473877in}{0.739656in}}%
\pgfpathlineto{\pgfqpoint{3.473024in}{0.739656in}}%
\pgfpathlineto{\pgfqpoint{3.472172in}{0.739656in}}%
\pgfpathlineto{\pgfqpoint{3.471319in}{0.739656in}}%
\pgfpathlineto{\pgfqpoint{3.470466in}{0.739656in}}%
\pgfpathlineto{\pgfqpoint{3.469613in}{0.739656in}}%
\pgfpathlineto{\pgfqpoint{3.468760in}{0.739656in}}%
\pgfpathlineto{\pgfqpoint{3.467907in}{0.739656in}}%
\pgfpathlineto{\pgfqpoint{3.467054in}{0.739656in}}%
\pgfpathlineto{\pgfqpoint{3.466201in}{0.739656in}}%
\pgfpathlineto{\pgfqpoint{3.465349in}{0.739656in}}%
\pgfpathlineto{\pgfqpoint{3.464496in}{0.739656in}}%
\pgfpathlineto{\pgfqpoint{3.463643in}{0.739656in}}%
\pgfpathlineto{\pgfqpoint{3.462790in}{0.739656in}}%
\pgfpathlineto{\pgfqpoint{3.461937in}{0.739656in}}%
\pgfpathlineto{\pgfqpoint{3.461084in}{0.739656in}}%
\pgfpathlineto{\pgfqpoint{3.460231in}{0.739656in}}%
\pgfpathlineto{\pgfqpoint{3.459378in}{0.739656in}}%
\pgfpathlineto{\pgfqpoint{3.458525in}{0.739656in}}%
\pgfpathlineto{\pgfqpoint{3.457673in}{0.739656in}}%
\pgfpathlineto{\pgfqpoint{3.456820in}{0.739656in}}%
\pgfpathlineto{\pgfqpoint{3.455967in}{0.739656in}}%
\pgfpathlineto{\pgfqpoint{3.455114in}{0.739656in}}%
\pgfpathlineto{\pgfqpoint{3.454261in}{0.739656in}}%
\pgfpathlineto{\pgfqpoint{3.453408in}{0.739656in}}%
\pgfpathlineto{\pgfqpoint{3.452555in}{0.739656in}}%
\pgfpathlineto{\pgfqpoint{3.451702in}{0.739656in}}%
\pgfpathlineto{\pgfqpoint{3.450850in}{0.739656in}}%
\pgfpathlineto{\pgfqpoint{3.449997in}{0.739656in}}%
\pgfpathlineto{\pgfqpoint{3.449144in}{0.739656in}}%
\pgfpathlineto{\pgfqpoint{3.448291in}{0.739656in}}%
\pgfpathlineto{\pgfqpoint{3.447438in}{0.739656in}}%
\pgfpathlineto{\pgfqpoint{3.446585in}{0.739656in}}%
\pgfpathlineto{\pgfqpoint{3.445732in}{0.739656in}}%
\pgfpathlineto{\pgfqpoint{3.444879in}{0.739656in}}%
\pgfpathlineto{\pgfqpoint{3.444027in}{0.739656in}}%
\pgfpathlineto{\pgfqpoint{3.443174in}{0.739656in}}%
\pgfpathlineto{\pgfqpoint{3.442321in}{0.739656in}}%
\pgfpathlineto{\pgfqpoint{3.441468in}{0.739656in}}%
\pgfpathlineto{\pgfqpoint{3.440615in}{0.739656in}}%
\pgfpathlineto{\pgfqpoint{3.439762in}{0.739656in}}%
\pgfpathlineto{\pgfqpoint{3.438909in}{0.739656in}}%
\pgfpathlineto{\pgfqpoint{3.438056in}{0.739656in}}%
\pgfpathlineto{\pgfqpoint{3.437204in}{0.739656in}}%
\pgfpathlineto{\pgfqpoint{3.436351in}{0.739656in}}%
\pgfpathlineto{\pgfqpoint{3.435498in}{0.739656in}}%
\pgfpathlineto{\pgfqpoint{3.434645in}{0.739656in}}%
\pgfpathlineto{\pgfqpoint{3.433792in}{0.739656in}}%
\pgfpathlineto{\pgfqpoint{3.432939in}{0.739656in}}%
\pgfpathlineto{\pgfqpoint{3.432086in}{0.739656in}}%
\pgfpathlineto{\pgfqpoint{3.431233in}{0.739656in}}%
\pgfpathlineto{\pgfqpoint{3.430380in}{0.739656in}}%
\pgfpathlineto{\pgfqpoint{3.429528in}{0.739656in}}%
\pgfpathlineto{\pgfqpoint{3.428675in}{0.739656in}}%
\pgfpathlineto{\pgfqpoint{3.427822in}{0.739656in}}%
\pgfpathlineto{\pgfqpoint{3.426969in}{0.739656in}}%
\pgfpathlineto{\pgfqpoint{3.426116in}{0.739656in}}%
\pgfpathlineto{\pgfqpoint{3.425263in}{0.739656in}}%
\pgfpathlineto{\pgfqpoint{3.424410in}{0.739656in}}%
\pgfpathlineto{\pgfqpoint{3.423557in}{0.739656in}}%
\pgfpathlineto{\pgfqpoint{3.422705in}{0.739656in}}%
\pgfpathlineto{\pgfqpoint{3.421852in}{0.739656in}}%
\pgfpathlineto{\pgfqpoint{3.420999in}{0.739656in}}%
\pgfpathlineto{\pgfqpoint{3.420146in}{0.739656in}}%
\pgfpathlineto{\pgfqpoint{3.419293in}{0.739656in}}%
\pgfpathlineto{\pgfqpoint{3.418440in}{0.739656in}}%
\pgfpathlineto{\pgfqpoint{3.417587in}{0.739656in}}%
\pgfpathlineto{\pgfqpoint{3.416734in}{0.739656in}}%
\pgfpathlineto{\pgfqpoint{3.415882in}{0.739656in}}%
\pgfpathlineto{\pgfqpoint{3.415029in}{0.739656in}}%
\pgfpathlineto{\pgfqpoint{3.414176in}{0.739656in}}%
\pgfpathlineto{\pgfqpoint{3.413323in}{0.739656in}}%
\pgfpathlineto{\pgfqpoint{3.412470in}{0.739656in}}%
\pgfpathlineto{\pgfqpoint{3.411617in}{0.739656in}}%
\pgfpathlineto{\pgfqpoint{3.410764in}{0.739656in}}%
\pgfpathlineto{\pgfqpoint{3.409911in}{0.739656in}}%
\pgfpathlineto{\pgfqpoint{3.409059in}{0.739656in}}%
\pgfpathlineto{\pgfqpoint{3.408206in}{0.739656in}}%
\pgfpathlineto{\pgfqpoint{3.407353in}{0.739656in}}%
\pgfpathlineto{\pgfqpoint{3.406500in}{0.739656in}}%
\pgfpathlineto{\pgfqpoint{3.405647in}{0.739656in}}%
\pgfpathlineto{\pgfqpoint{3.404794in}{0.739656in}}%
\pgfpathlineto{\pgfqpoint{3.403941in}{0.739656in}}%
\pgfpathlineto{\pgfqpoint{3.403088in}{0.739656in}}%
\pgfpathlineto{\pgfqpoint{3.402235in}{0.739656in}}%
\pgfpathlineto{\pgfqpoint{3.401383in}{0.739656in}}%
\pgfpathlineto{\pgfqpoint{3.400530in}{0.739656in}}%
\pgfpathlineto{\pgfqpoint{3.399677in}{0.739656in}}%
\pgfpathlineto{\pgfqpoint{3.398824in}{0.739656in}}%
\pgfpathlineto{\pgfqpoint{3.397971in}{0.739656in}}%
\pgfpathlineto{\pgfqpoint{3.397118in}{0.739656in}}%
\pgfpathlineto{\pgfqpoint{3.396265in}{0.739656in}}%
\pgfpathlineto{\pgfqpoint{3.395412in}{0.739656in}}%
\pgfpathlineto{\pgfqpoint{3.394560in}{0.739656in}}%
\pgfpathlineto{\pgfqpoint{3.393707in}{0.739656in}}%
\pgfpathlineto{\pgfqpoint{3.392854in}{0.739656in}}%
\pgfpathlineto{\pgfqpoint{3.392001in}{0.739656in}}%
\pgfpathlineto{\pgfqpoint{3.391148in}{0.739656in}}%
\pgfpathlineto{\pgfqpoint{3.390295in}{0.739656in}}%
\pgfpathlineto{\pgfqpoint{3.389442in}{0.739656in}}%
\pgfpathlineto{\pgfqpoint{3.388589in}{0.739656in}}%
\pgfpathlineto{\pgfqpoint{3.387737in}{0.739656in}}%
\pgfpathlineto{\pgfqpoint{3.386884in}{0.739656in}}%
\pgfpathlineto{\pgfqpoint{3.386031in}{0.739656in}}%
\pgfpathlineto{\pgfqpoint{3.385178in}{0.739656in}}%
\pgfpathlineto{\pgfqpoint{3.384325in}{0.739656in}}%
\pgfpathlineto{\pgfqpoint{3.383472in}{0.739656in}}%
\pgfpathlineto{\pgfqpoint{3.382619in}{0.739656in}}%
\pgfpathlineto{\pgfqpoint{3.381766in}{0.739656in}}%
\pgfpathlineto{\pgfqpoint{3.380914in}{0.739656in}}%
\pgfpathlineto{\pgfqpoint{3.380061in}{0.739656in}}%
\pgfpathlineto{\pgfqpoint{3.379208in}{0.739656in}}%
\pgfpathlineto{\pgfqpoint{3.378355in}{0.739656in}}%
\pgfpathlineto{\pgfqpoint{3.377502in}{0.739656in}}%
\pgfpathlineto{\pgfqpoint{3.376649in}{0.739656in}}%
\pgfpathlineto{\pgfqpoint{3.375796in}{0.739656in}}%
\pgfpathlineto{\pgfqpoint{3.374943in}{0.739656in}}%
\pgfpathlineto{\pgfqpoint{3.374091in}{0.739656in}}%
\pgfpathlineto{\pgfqpoint{3.373238in}{0.739656in}}%
\pgfpathlineto{\pgfqpoint{3.372385in}{0.739656in}}%
\pgfpathlineto{\pgfqpoint{3.371532in}{0.739656in}}%
\pgfpathlineto{\pgfqpoint{3.370679in}{0.739656in}}%
\pgfpathlineto{\pgfqpoint{3.369826in}{0.739656in}}%
\pgfpathlineto{\pgfqpoint{3.368973in}{0.739656in}}%
\pgfpathlineto{\pgfqpoint{3.368120in}{0.739656in}}%
\pgfpathlineto{\pgfqpoint{3.367267in}{0.739656in}}%
\pgfpathlineto{\pgfqpoint{3.366415in}{0.739656in}}%
\pgfpathlineto{\pgfqpoint{3.365562in}{0.739656in}}%
\pgfpathlineto{\pgfqpoint{3.364709in}{0.739656in}}%
\pgfpathlineto{\pgfqpoint{3.363856in}{0.739656in}}%
\pgfpathlineto{\pgfqpoint{3.363003in}{0.739656in}}%
\pgfpathlineto{\pgfqpoint{3.362150in}{0.739656in}}%
\pgfpathlineto{\pgfqpoint{3.361297in}{0.739656in}}%
\pgfpathlineto{\pgfqpoint{3.360444in}{0.739656in}}%
\pgfpathlineto{\pgfqpoint{3.359592in}{0.739656in}}%
\pgfpathlineto{\pgfqpoint{3.358739in}{0.739656in}}%
\pgfpathlineto{\pgfqpoint{3.357886in}{0.739656in}}%
\pgfpathlineto{\pgfqpoint{3.357033in}{0.739656in}}%
\pgfpathlineto{\pgfqpoint{3.356180in}{0.739656in}}%
\pgfpathlineto{\pgfqpoint{3.355327in}{0.739656in}}%
\pgfpathlineto{\pgfqpoint{3.354474in}{0.739656in}}%
\pgfpathlineto{\pgfqpoint{3.353621in}{0.739656in}}%
\pgfpathlineto{\pgfqpoint{3.352769in}{0.739656in}}%
\pgfpathlineto{\pgfqpoint{3.351916in}{0.739656in}}%
\pgfpathlineto{\pgfqpoint{3.351063in}{0.739656in}}%
\pgfpathlineto{\pgfqpoint{3.350210in}{0.739656in}}%
\pgfpathlineto{\pgfqpoint{3.349357in}{0.739656in}}%
\pgfpathlineto{\pgfqpoint{3.348504in}{0.739656in}}%
\pgfpathlineto{\pgfqpoint{3.347651in}{0.739656in}}%
\pgfpathlineto{\pgfqpoint{3.346798in}{0.739656in}}%
\pgfpathlineto{\pgfqpoint{3.345946in}{0.739656in}}%
\pgfpathlineto{\pgfqpoint{3.345093in}{0.739656in}}%
\pgfpathlineto{\pgfqpoint{3.344240in}{0.739656in}}%
\pgfpathlineto{\pgfqpoint{3.343387in}{0.739656in}}%
\pgfpathlineto{\pgfqpoint{3.342534in}{0.739656in}}%
\pgfpathlineto{\pgfqpoint{3.341681in}{0.739656in}}%
\pgfpathlineto{\pgfqpoint{3.340828in}{0.739656in}}%
\pgfpathlineto{\pgfqpoint{3.339975in}{0.739656in}}%
\pgfpathlineto{\pgfqpoint{3.339122in}{0.739656in}}%
\pgfpathlineto{\pgfqpoint{3.338270in}{0.739656in}}%
\pgfpathlineto{\pgfqpoint{3.337417in}{0.739656in}}%
\pgfpathlineto{\pgfqpoint{3.336564in}{0.739656in}}%
\pgfpathlineto{\pgfqpoint{3.335711in}{0.739656in}}%
\pgfpathlineto{\pgfqpoint{3.334858in}{0.739656in}}%
\pgfpathlineto{\pgfqpoint{3.334005in}{0.739656in}}%
\pgfpathlineto{\pgfqpoint{3.333152in}{0.739656in}}%
\pgfpathlineto{\pgfqpoint{3.332299in}{0.739656in}}%
\pgfpathlineto{\pgfqpoint{3.331447in}{0.739656in}}%
\pgfpathlineto{\pgfqpoint{3.330594in}{0.739656in}}%
\pgfpathlineto{\pgfqpoint{3.329741in}{0.739656in}}%
\pgfpathlineto{\pgfqpoint{3.328888in}{0.739656in}}%
\pgfpathlineto{\pgfqpoint{3.328035in}{0.739656in}}%
\pgfpathlineto{\pgfqpoint{3.327182in}{0.739656in}}%
\pgfpathlineto{\pgfqpoint{3.326329in}{0.739656in}}%
\pgfpathlineto{\pgfqpoint{3.325476in}{0.739656in}}%
\pgfpathlineto{\pgfqpoint{3.324624in}{0.739656in}}%
\pgfpathlineto{\pgfqpoint{3.323771in}{0.739656in}}%
\pgfpathlineto{\pgfqpoint{3.322918in}{0.739656in}}%
\pgfpathlineto{\pgfqpoint{3.322065in}{0.739656in}}%
\pgfpathlineto{\pgfqpoint{3.321212in}{0.739656in}}%
\pgfpathlineto{\pgfqpoint{3.320359in}{0.739656in}}%
\pgfpathlineto{\pgfqpoint{3.319506in}{0.739656in}}%
\pgfpathlineto{\pgfqpoint{3.318653in}{0.739656in}}%
\pgfpathlineto{\pgfqpoint{3.317801in}{0.739656in}}%
\pgfpathlineto{\pgfqpoint{3.316948in}{0.739656in}}%
\pgfpathlineto{\pgfqpoint{3.316095in}{0.739656in}}%
\pgfpathlineto{\pgfqpoint{3.315242in}{0.739656in}}%
\pgfpathlineto{\pgfqpoint{3.314389in}{0.739656in}}%
\pgfpathlineto{\pgfqpoint{3.313536in}{0.739656in}}%
\pgfpathlineto{\pgfqpoint{3.312683in}{0.739656in}}%
\pgfpathlineto{\pgfqpoint{3.311830in}{0.739656in}}%
\pgfpathlineto{\pgfqpoint{3.310977in}{0.739656in}}%
\pgfpathlineto{\pgfqpoint{3.310125in}{0.739656in}}%
\pgfpathlineto{\pgfqpoint{3.309272in}{0.739656in}}%
\pgfpathlineto{\pgfqpoint{3.308419in}{0.739656in}}%
\pgfpathlineto{\pgfqpoint{3.307566in}{0.739656in}}%
\pgfpathlineto{\pgfqpoint{3.306713in}{0.739656in}}%
\pgfpathlineto{\pgfqpoint{3.305860in}{0.739656in}}%
\pgfpathlineto{\pgfqpoint{3.305007in}{0.739656in}}%
\pgfpathlineto{\pgfqpoint{3.304154in}{0.739656in}}%
\pgfpathlineto{\pgfqpoint{3.303302in}{0.739656in}}%
\pgfpathlineto{\pgfqpoint{3.302449in}{0.739656in}}%
\pgfpathlineto{\pgfqpoint{3.301596in}{0.739656in}}%
\pgfpathlineto{\pgfqpoint{3.300743in}{0.739656in}}%
\pgfpathlineto{\pgfqpoint{3.299890in}{0.739656in}}%
\pgfpathlineto{\pgfqpoint{3.299037in}{0.739656in}}%
\pgfpathlineto{\pgfqpoint{3.298184in}{0.739656in}}%
\pgfpathlineto{\pgfqpoint{3.297331in}{0.739656in}}%
\pgfpathlineto{\pgfqpoint{3.296479in}{0.739656in}}%
\pgfpathlineto{\pgfqpoint{3.295626in}{0.739656in}}%
\pgfpathlineto{\pgfqpoint{3.294773in}{0.739656in}}%
\pgfpathlineto{\pgfqpoint{3.293920in}{0.739656in}}%
\pgfpathlineto{\pgfqpoint{3.293067in}{0.739656in}}%
\pgfpathlineto{\pgfqpoint{3.292214in}{0.739656in}}%
\pgfpathlineto{\pgfqpoint{3.291361in}{0.739656in}}%
\pgfpathlineto{\pgfqpoint{3.290508in}{0.739656in}}%
\pgfpathlineto{\pgfqpoint{3.289656in}{0.739656in}}%
\pgfpathlineto{\pgfqpoint{3.288803in}{0.739656in}}%
\pgfpathlineto{\pgfqpoint{3.287950in}{0.739656in}}%
\pgfpathlineto{\pgfqpoint{3.287097in}{0.739656in}}%
\pgfpathlineto{\pgfqpoint{3.286244in}{0.739656in}}%
\pgfpathlineto{\pgfqpoint{3.285391in}{0.739656in}}%
\pgfpathlineto{\pgfqpoint{3.284538in}{0.739656in}}%
\pgfpathlineto{\pgfqpoint{3.283685in}{0.739656in}}%
\pgfpathlineto{\pgfqpoint{3.282833in}{0.739656in}}%
\pgfpathlineto{\pgfqpoint{3.281980in}{0.739656in}}%
\pgfpathlineto{\pgfqpoint{3.281127in}{0.739656in}}%
\pgfpathlineto{\pgfqpoint{3.280274in}{0.739656in}}%
\pgfpathlineto{\pgfqpoint{3.279421in}{0.739656in}}%
\pgfpathlineto{\pgfqpoint{3.278568in}{0.739656in}}%
\pgfpathlineto{\pgfqpoint{3.277715in}{0.739656in}}%
\pgfpathlineto{\pgfqpoint{3.276862in}{0.739656in}}%
\pgfpathlineto{\pgfqpoint{3.276009in}{0.739656in}}%
\pgfpathlineto{\pgfqpoint{3.275157in}{0.739656in}}%
\pgfpathlineto{\pgfqpoint{3.274304in}{0.739656in}}%
\pgfpathlineto{\pgfqpoint{3.273451in}{0.739656in}}%
\pgfpathlineto{\pgfqpoint{3.272598in}{0.739656in}}%
\pgfpathlineto{\pgfqpoint{3.271745in}{0.739656in}}%
\pgfpathlineto{\pgfqpoint{3.270892in}{0.739656in}}%
\pgfpathlineto{\pgfqpoint{3.270039in}{0.739656in}}%
\pgfpathlineto{\pgfqpoint{3.269186in}{0.739656in}}%
\pgfpathlineto{\pgfqpoint{3.268334in}{0.739656in}}%
\pgfpathlineto{\pgfqpoint{3.267481in}{0.739656in}}%
\pgfpathlineto{\pgfqpoint{3.266628in}{0.739656in}}%
\pgfpathlineto{\pgfqpoint{3.265775in}{0.739656in}}%
\pgfpathlineto{\pgfqpoint{3.264922in}{0.739656in}}%
\pgfpathlineto{\pgfqpoint{3.264069in}{0.739656in}}%
\pgfpathlineto{\pgfqpoint{3.263216in}{0.739656in}}%
\pgfpathlineto{\pgfqpoint{3.262363in}{0.739656in}}%
\pgfpathlineto{\pgfqpoint{3.261511in}{0.739656in}}%
\pgfpathlineto{\pgfqpoint{3.260658in}{0.739656in}}%
\pgfpathlineto{\pgfqpoint{3.259805in}{0.739656in}}%
\pgfpathlineto{\pgfqpoint{3.258952in}{0.739656in}}%
\pgfpathlineto{\pgfqpoint{3.258099in}{0.739656in}}%
\pgfpathlineto{\pgfqpoint{3.257246in}{0.739656in}}%
\pgfpathlineto{\pgfqpoint{3.256393in}{0.739656in}}%
\pgfpathlineto{\pgfqpoint{3.255540in}{0.739656in}}%
\pgfpathlineto{\pgfqpoint{3.254688in}{0.739656in}}%
\pgfpathlineto{\pgfqpoint{3.253835in}{0.739656in}}%
\pgfpathlineto{\pgfqpoint{3.252982in}{0.739656in}}%
\pgfpathlineto{\pgfqpoint{3.252129in}{0.739656in}}%
\pgfpathlineto{\pgfqpoint{3.251276in}{0.739656in}}%
\pgfpathlineto{\pgfqpoint{3.250423in}{0.739656in}}%
\pgfpathlineto{\pgfqpoint{3.249570in}{0.739656in}}%
\pgfpathlineto{\pgfqpoint{3.248717in}{0.739656in}}%
\pgfpathlineto{\pgfqpoint{3.247864in}{0.739656in}}%
\pgfpathlineto{\pgfqpoint{3.247012in}{0.739656in}}%
\pgfpathlineto{\pgfqpoint{3.246159in}{0.739656in}}%
\pgfpathlineto{\pgfqpoint{3.245306in}{0.739656in}}%
\pgfpathlineto{\pgfqpoint{3.244453in}{0.739656in}}%
\pgfpathlineto{\pgfqpoint{3.243600in}{0.739656in}}%
\pgfpathlineto{\pgfqpoint{3.242747in}{0.739656in}}%
\pgfpathlineto{\pgfqpoint{3.241894in}{0.739656in}}%
\pgfpathlineto{\pgfqpoint{3.241041in}{0.739656in}}%
\pgfpathlineto{\pgfqpoint{3.240189in}{0.739656in}}%
\pgfpathlineto{\pgfqpoint{3.239336in}{0.739656in}}%
\pgfpathlineto{\pgfqpoint{3.238483in}{0.739656in}}%
\pgfpathlineto{\pgfqpoint{3.237630in}{0.739656in}}%
\pgfpathlineto{\pgfqpoint{3.236777in}{0.739656in}}%
\pgfpathlineto{\pgfqpoint{3.235924in}{0.739656in}}%
\pgfpathlineto{\pgfqpoint{3.235071in}{0.739656in}}%
\pgfpathlineto{\pgfqpoint{3.234218in}{0.739656in}}%
\pgfpathlineto{\pgfqpoint{3.233366in}{0.739656in}}%
\pgfpathlineto{\pgfqpoint{3.232513in}{0.739656in}}%
\pgfpathlineto{\pgfqpoint{3.231660in}{0.739656in}}%
\pgfpathlineto{\pgfqpoint{3.230807in}{0.739656in}}%
\pgfpathlineto{\pgfqpoint{3.229954in}{0.739656in}}%
\pgfpathlineto{\pgfqpoint{3.229101in}{0.739656in}}%
\pgfpathlineto{\pgfqpoint{3.228248in}{0.739656in}}%
\pgfpathlineto{\pgfqpoint{3.227395in}{0.739656in}}%
\pgfpathlineto{\pgfqpoint{3.226543in}{0.739656in}}%
\pgfpathlineto{\pgfqpoint{3.225690in}{0.739656in}}%
\pgfpathlineto{\pgfqpoint{3.224837in}{0.739656in}}%
\pgfpathlineto{\pgfqpoint{3.223984in}{0.739656in}}%
\pgfpathlineto{\pgfqpoint{3.223131in}{0.739656in}}%
\pgfpathlineto{\pgfqpoint{3.222278in}{0.739656in}}%
\pgfpathlineto{\pgfqpoint{3.221425in}{0.739656in}}%
\pgfpathlineto{\pgfqpoint{3.220572in}{0.739656in}}%
\pgfpathlineto{\pgfqpoint{3.219719in}{0.739656in}}%
\pgfpathlineto{\pgfqpoint{3.218867in}{0.739656in}}%
\pgfpathlineto{\pgfqpoint{3.218014in}{0.739656in}}%
\pgfpathlineto{\pgfqpoint{3.217161in}{0.739656in}}%
\pgfpathlineto{\pgfqpoint{3.216308in}{0.739656in}}%
\pgfpathlineto{\pgfqpoint{3.215455in}{0.739656in}}%
\pgfpathlineto{\pgfqpoint{3.214602in}{0.739656in}}%
\pgfpathlineto{\pgfqpoint{3.213749in}{0.739656in}}%
\pgfpathlineto{\pgfqpoint{3.212896in}{0.739656in}}%
\pgfpathlineto{\pgfqpoint{3.212044in}{0.739656in}}%
\pgfpathlineto{\pgfqpoint{3.211191in}{0.739656in}}%
\pgfpathlineto{\pgfqpoint{3.210338in}{0.739656in}}%
\pgfpathlineto{\pgfqpoint{3.209485in}{0.739656in}}%
\pgfpathlineto{\pgfqpoint{3.208632in}{0.739656in}}%
\pgfpathlineto{\pgfqpoint{3.207779in}{0.739656in}}%
\pgfpathlineto{\pgfqpoint{3.206926in}{0.739656in}}%
\pgfpathlineto{\pgfqpoint{3.206073in}{0.739656in}}%
\pgfpathlineto{\pgfqpoint{3.205221in}{0.739656in}}%
\pgfpathlineto{\pgfqpoint{3.204368in}{0.739656in}}%
\pgfpathlineto{\pgfqpoint{3.203515in}{0.739656in}}%
\pgfpathlineto{\pgfqpoint{3.202662in}{0.739656in}}%
\pgfpathlineto{\pgfqpoint{3.201809in}{0.739656in}}%
\pgfpathlineto{\pgfqpoint{3.200956in}{0.739656in}}%
\pgfpathlineto{\pgfqpoint{3.200103in}{0.739656in}}%
\pgfpathlineto{\pgfqpoint{3.199250in}{0.739656in}}%
\pgfpathlineto{\pgfqpoint{3.198398in}{0.739656in}}%
\pgfpathlineto{\pgfqpoint{3.197545in}{0.739656in}}%
\pgfpathlineto{\pgfqpoint{3.196692in}{0.739656in}}%
\pgfpathlineto{\pgfqpoint{3.195839in}{0.739656in}}%
\pgfpathlineto{\pgfqpoint{3.194986in}{0.739656in}}%
\pgfpathlineto{\pgfqpoint{3.194133in}{0.739656in}}%
\pgfpathlineto{\pgfqpoint{3.193280in}{0.739656in}}%
\pgfpathlineto{\pgfqpoint{3.192427in}{0.739656in}}%
\pgfpathlineto{\pgfqpoint{3.191574in}{0.739656in}}%
\pgfpathlineto{\pgfqpoint{3.190722in}{0.739656in}}%
\pgfpathlineto{\pgfqpoint{3.189869in}{0.739656in}}%
\pgfpathlineto{\pgfqpoint{3.189016in}{0.739656in}}%
\pgfpathlineto{\pgfqpoint{3.188163in}{0.739656in}}%
\pgfpathlineto{\pgfqpoint{3.187310in}{0.739656in}}%
\pgfpathlineto{\pgfqpoint{3.186457in}{0.739656in}}%
\pgfpathlineto{\pgfqpoint{3.185604in}{0.739656in}}%
\pgfpathlineto{\pgfqpoint{3.184751in}{0.739656in}}%
\pgfpathlineto{\pgfqpoint{3.183899in}{0.739656in}}%
\pgfpathlineto{\pgfqpoint{3.183046in}{0.739656in}}%
\pgfpathlineto{\pgfqpoint{3.182193in}{0.739656in}}%
\pgfpathlineto{\pgfqpoint{3.181340in}{0.739656in}}%
\pgfpathlineto{\pgfqpoint{3.180487in}{0.739656in}}%
\pgfpathlineto{\pgfqpoint{3.179634in}{0.739656in}}%
\pgfpathlineto{\pgfqpoint{3.178781in}{0.739656in}}%
\pgfpathlineto{\pgfqpoint{3.177928in}{0.739656in}}%
\pgfpathlineto{\pgfqpoint{3.177076in}{0.739656in}}%
\pgfpathlineto{\pgfqpoint{3.176223in}{0.739656in}}%
\pgfpathlineto{\pgfqpoint{3.175370in}{0.739656in}}%
\pgfpathlineto{\pgfqpoint{3.174517in}{0.739656in}}%
\pgfpathlineto{\pgfqpoint{3.173664in}{0.739656in}}%
\pgfpathlineto{\pgfqpoint{3.172811in}{0.739656in}}%
\pgfpathlineto{\pgfqpoint{3.171958in}{0.739656in}}%
\pgfpathlineto{\pgfqpoint{3.171105in}{0.739656in}}%
\pgfpathlineto{\pgfqpoint{3.170253in}{0.739656in}}%
\pgfpathlineto{\pgfqpoint{3.169400in}{0.739656in}}%
\pgfpathlineto{\pgfqpoint{3.168547in}{0.739656in}}%
\pgfpathlineto{\pgfqpoint{3.167694in}{0.739656in}}%
\pgfpathlineto{\pgfqpoint{3.166841in}{0.739656in}}%
\pgfpathlineto{\pgfqpoint{3.165988in}{0.739656in}}%
\pgfpathlineto{\pgfqpoint{3.165135in}{0.739656in}}%
\pgfpathlineto{\pgfqpoint{3.164282in}{0.739656in}}%
\pgfpathlineto{\pgfqpoint{3.163430in}{0.739656in}}%
\pgfpathlineto{\pgfqpoint{3.162577in}{0.739656in}}%
\pgfpathlineto{\pgfqpoint{3.161724in}{0.739656in}}%
\pgfpathlineto{\pgfqpoint{3.160871in}{0.739656in}}%
\pgfpathlineto{\pgfqpoint{3.160018in}{0.739656in}}%
\pgfpathlineto{\pgfqpoint{3.159165in}{0.739656in}}%
\pgfpathlineto{\pgfqpoint{3.158312in}{0.739656in}}%
\pgfpathlineto{\pgfqpoint{3.157459in}{0.739656in}}%
\pgfpathlineto{\pgfqpoint{3.156606in}{0.739656in}}%
\pgfpathlineto{\pgfqpoint{3.155754in}{0.739656in}}%
\pgfpathlineto{\pgfqpoint{3.154901in}{0.739656in}}%
\pgfpathlineto{\pgfqpoint{3.154048in}{0.739656in}}%
\pgfpathlineto{\pgfqpoint{3.153195in}{0.739656in}}%
\pgfpathlineto{\pgfqpoint{3.152342in}{0.739656in}}%
\pgfpathlineto{\pgfqpoint{3.151489in}{0.739656in}}%
\pgfpathlineto{\pgfqpoint{3.150636in}{0.739656in}}%
\pgfpathlineto{\pgfqpoint{3.149783in}{0.739656in}}%
\pgfpathlineto{\pgfqpoint{3.148931in}{0.739656in}}%
\pgfpathlineto{\pgfqpoint{3.148078in}{0.739656in}}%
\pgfpathlineto{\pgfqpoint{3.147225in}{0.739656in}}%
\pgfpathlineto{\pgfqpoint{3.146372in}{0.739656in}}%
\pgfpathlineto{\pgfqpoint{3.145519in}{0.739656in}}%
\pgfpathlineto{\pgfqpoint{3.144666in}{0.739656in}}%
\pgfpathlineto{\pgfqpoint{3.143813in}{0.739656in}}%
\pgfpathlineto{\pgfqpoint{3.142960in}{0.739656in}}%
\pgfpathlineto{\pgfqpoint{3.142108in}{0.739656in}}%
\pgfpathlineto{\pgfqpoint{3.141255in}{0.739656in}}%
\pgfpathlineto{\pgfqpoint{3.140402in}{0.739656in}}%
\pgfpathlineto{\pgfqpoint{3.139549in}{0.739656in}}%
\pgfpathlineto{\pgfqpoint{3.138696in}{0.739656in}}%
\pgfpathlineto{\pgfqpoint{3.137843in}{0.739656in}}%
\pgfpathlineto{\pgfqpoint{3.136990in}{0.739656in}}%
\pgfpathlineto{\pgfqpoint{3.136137in}{0.739656in}}%
\pgfpathlineto{\pgfqpoint{3.135285in}{0.739656in}}%
\pgfpathlineto{\pgfqpoint{3.134432in}{0.739656in}}%
\pgfpathlineto{\pgfqpoint{3.133579in}{0.739656in}}%
\pgfpathlineto{\pgfqpoint{3.132726in}{0.739656in}}%
\pgfpathlineto{\pgfqpoint{3.131873in}{0.739656in}}%
\pgfpathlineto{\pgfqpoint{3.131020in}{0.739656in}}%
\pgfpathlineto{\pgfqpoint{3.130167in}{0.739656in}}%
\pgfpathlineto{\pgfqpoint{3.129314in}{0.739656in}}%
\pgfpathlineto{\pgfqpoint{3.128461in}{0.739656in}}%
\pgfpathlineto{\pgfqpoint{3.127609in}{0.739656in}}%
\pgfpathlineto{\pgfqpoint{3.126756in}{0.739656in}}%
\pgfpathlineto{\pgfqpoint{3.125903in}{0.739656in}}%
\pgfpathlineto{\pgfqpoint{3.125050in}{0.739656in}}%
\pgfpathlineto{\pgfqpoint{3.124197in}{0.739656in}}%
\pgfpathlineto{\pgfqpoint{3.123344in}{0.739656in}}%
\pgfpathlineto{\pgfqpoint{3.122491in}{0.739656in}}%
\pgfpathlineto{\pgfqpoint{3.121638in}{0.739656in}}%
\pgfpathlineto{\pgfqpoint{3.120786in}{0.739656in}}%
\pgfpathlineto{\pgfqpoint{3.119933in}{0.739656in}}%
\pgfpathlineto{\pgfqpoint{3.119080in}{0.739656in}}%
\pgfpathlineto{\pgfqpoint{3.118227in}{0.739656in}}%
\pgfpathlineto{\pgfqpoint{3.117374in}{0.739656in}}%
\pgfpathlineto{\pgfqpoint{3.116521in}{0.739656in}}%
\pgfpathlineto{\pgfqpoint{3.115668in}{0.739656in}}%
\pgfpathlineto{\pgfqpoint{3.114815in}{0.739656in}}%
\pgfpathlineto{\pgfqpoint{3.113963in}{0.739656in}}%
\pgfpathlineto{\pgfqpoint{3.113110in}{0.739656in}}%
\pgfpathlineto{\pgfqpoint{3.112257in}{0.739656in}}%
\pgfpathlineto{\pgfqpoint{3.111404in}{0.739656in}}%
\pgfpathlineto{\pgfqpoint{3.110551in}{0.739656in}}%
\pgfpathlineto{\pgfqpoint{3.109698in}{0.739656in}}%
\pgfpathlineto{\pgfqpoint{3.108845in}{0.739656in}}%
\pgfpathlineto{\pgfqpoint{3.107992in}{0.739656in}}%
\pgfpathlineto{\pgfqpoint{3.107140in}{0.739656in}}%
\pgfpathlineto{\pgfqpoint{3.106287in}{0.739656in}}%
\pgfpathlineto{\pgfqpoint{3.105434in}{0.739656in}}%
\pgfpathlineto{\pgfqpoint{3.104581in}{0.739656in}}%
\pgfpathlineto{\pgfqpoint{3.103728in}{0.739656in}}%
\pgfpathlineto{\pgfqpoint{3.102875in}{0.739656in}}%
\pgfpathlineto{\pgfqpoint{3.102022in}{0.739656in}}%
\pgfpathlineto{\pgfqpoint{3.101169in}{0.739656in}}%
\pgfpathlineto{\pgfqpoint{3.100316in}{0.739656in}}%
\pgfpathlineto{\pgfqpoint{3.099464in}{0.739656in}}%
\pgfpathlineto{\pgfqpoint{3.098611in}{0.739656in}}%
\pgfpathlineto{\pgfqpoint{3.097758in}{0.739656in}}%
\pgfpathlineto{\pgfqpoint{3.096905in}{0.739656in}}%
\pgfpathlineto{\pgfqpoint{3.096052in}{0.739656in}}%
\pgfpathlineto{\pgfqpoint{3.095199in}{0.739656in}}%
\pgfpathlineto{\pgfqpoint{3.094346in}{0.739656in}}%
\pgfpathlineto{\pgfqpoint{3.093493in}{0.739656in}}%
\pgfpathlineto{\pgfqpoint{3.092641in}{0.739656in}}%
\pgfpathlineto{\pgfqpoint{3.091788in}{0.739656in}}%
\pgfpathlineto{\pgfqpoint{3.090935in}{0.739656in}}%
\pgfpathlineto{\pgfqpoint{3.090082in}{0.739656in}}%
\pgfpathlineto{\pgfqpoint{3.089229in}{0.739656in}}%
\pgfpathlineto{\pgfqpoint{3.088376in}{0.739656in}}%
\pgfpathlineto{\pgfqpoint{3.087523in}{0.739656in}}%
\pgfpathlineto{\pgfqpoint{3.086670in}{0.739656in}}%
\pgfpathlineto{\pgfqpoint{3.085818in}{0.739656in}}%
\pgfpathlineto{\pgfqpoint{3.084965in}{0.739656in}}%
\pgfpathlineto{\pgfqpoint{3.084112in}{0.739656in}}%
\pgfpathlineto{\pgfqpoint{3.083259in}{0.739656in}}%
\pgfpathlineto{\pgfqpoint{3.082406in}{0.739656in}}%
\pgfpathlineto{\pgfqpoint{3.081553in}{0.739656in}}%
\pgfpathlineto{\pgfqpoint{3.080700in}{0.739656in}}%
\pgfpathlineto{\pgfqpoint{3.079847in}{0.739656in}}%
\pgfpathlineto{\pgfqpoint{3.078995in}{0.739656in}}%
\pgfpathlineto{\pgfqpoint{3.078142in}{0.739656in}}%
\pgfpathlineto{\pgfqpoint{3.077289in}{0.739656in}}%
\pgfpathlineto{\pgfqpoint{3.076436in}{0.739656in}}%
\pgfpathlineto{\pgfqpoint{3.075583in}{0.739656in}}%
\pgfpathlineto{\pgfqpoint{3.074730in}{0.739656in}}%
\pgfpathlineto{\pgfqpoint{3.073877in}{0.739656in}}%
\pgfpathlineto{\pgfqpoint{3.073024in}{0.739656in}}%
\pgfpathlineto{\pgfqpoint{3.072171in}{0.739656in}}%
\pgfpathlineto{\pgfqpoint{3.071319in}{0.739656in}}%
\pgfpathlineto{\pgfqpoint{3.070466in}{0.739656in}}%
\pgfpathlineto{\pgfqpoint{3.069613in}{0.739656in}}%
\pgfpathlineto{\pgfqpoint{3.068760in}{0.739656in}}%
\pgfpathlineto{\pgfqpoint{3.067907in}{0.739656in}}%
\pgfpathlineto{\pgfqpoint{3.067054in}{0.739656in}}%
\pgfpathlineto{\pgfqpoint{3.066201in}{0.739656in}}%
\pgfpathlineto{\pgfqpoint{3.065348in}{0.739656in}}%
\pgfpathlineto{\pgfqpoint{3.064496in}{0.739656in}}%
\pgfpathlineto{\pgfqpoint{3.063643in}{0.739656in}}%
\pgfpathlineto{\pgfqpoint{3.062790in}{0.739656in}}%
\pgfpathlineto{\pgfqpoint{3.061937in}{0.739656in}}%
\pgfpathlineto{\pgfqpoint{3.061084in}{0.739656in}}%
\pgfpathlineto{\pgfqpoint{3.060231in}{0.739656in}}%
\pgfpathlineto{\pgfqpoint{3.059378in}{0.739656in}}%
\pgfpathlineto{\pgfqpoint{3.058525in}{0.739656in}}%
\pgfpathlineto{\pgfqpoint{3.057673in}{0.739656in}}%
\pgfpathlineto{\pgfqpoint{3.056820in}{0.739656in}}%
\pgfpathlineto{\pgfqpoint{3.055967in}{0.739656in}}%
\pgfpathlineto{\pgfqpoint{3.055114in}{0.739656in}}%
\pgfpathlineto{\pgfqpoint{3.054261in}{0.739656in}}%
\pgfpathlineto{\pgfqpoint{3.053408in}{0.739656in}}%
\pgfpathlineto{\pgfqpoint{3.052555in}{0.739656in}}%
\pgfpathlineto{\pgfqpoint{3.051702in}{0.739656in}}%
\pgfpathlineto{\pgfqpoint{3.050850in}{0.739656in}}%
\pgfpathlineto{\pgfqpoint{3.049997in}{0.739656in}}%
\pgfpathlineto{\pgfqpoint{3.049144in}{0.739656in}}%
\pgfpathlineto{\pgfqpoint{3.048291in}{0.739656in}}%
\pgfpathlineto{\pgfqpoint{3.047438in}{0.739656in}}%
\pgfpathlineto{\pgfqpoint{3.046585in}{0.739656in}}%
\pgfpathlineto{\pgfqpoint{3.045732in}{0.739656in}}%
\pgfpathlineto{\pgfqpoint{3.044879in}{0.739656in}}%
\pgfpathlineto{\pgfqpoint{3.044027in}{0.739656in}}%
\pgfpathlineto{\pgfqpoint{3.043174in}{0.739656in}}%
\pgfpathlineto{\pgfqpoint{3.042321in}{0.739656in}}%
\pgfpathlineto{\pgfqpoint{3.041468in}{0.739656in}}%
\pgfpathlineto{\pgfqpoint{3.040615in}{0.739656in}}%
\pgfpathlineto{\pgfqpoint{3.039762in}{0.739656in}}%
\pgfpathlineto{\pgfqpoint{3.038909in}{0.739656in}}%
\pgfpathlineto{\pgfqpoint{3.038056in}{0.739656in}}%
\pgfpathlineto{\pgfqpoint{3.037203in}{0.739656in}}%
\pgfpathlineto{\pgfqpoint{3.036351in}{0.739656in}}%
\pgfpathlineto{\pgfqpoint{3.035498in}{0.739656in}}%
\pgfpathlineto{\pgfqpoint{3.034645in}{0.739656in}}%
\pgfpathlineto{\pgfqpoint{3.033792in}{0.739656in}}%
\pgfpathlineto{\pgfqpoint{3.032939in}{0.739656in}}%
\pgfpathlineto{\pgfqpoint{3.032086in}{0.739656in}}%
\pgfpathlineto{\pgfqpoint{3.031233in}{0.739656in}}%
\pgfpathlineto{\pgfqpoint{3.030380in}{0.739656in}}%
\pgfpathlineto{\pgfqpoint{3.029528in}{0.739656in}}%
\pgfpathlineto{\pgfqpoint{3.028675in}{0.739656in}}%
\pgfpathlineto{\pgfqpoint{3.027822in}{0.739656in}}%
\pgfpathlineto{\pgfqpoint{3.026969in}{0.739656in}}%
\pgfpathlineto{\pgfqpoint{3.026116in}{0.739656in}}%
\pgfpathlineto{\pgfqpoint{3.025263in}{0.739656in}}%
\pgfpathlineto{\pgfqpoint{3.024410in}{0.739656in}}%
\pgfpathlineto{\pgfqpoint{3.023557in}{0.739656in}}%
\pgfpathlineto{\pgfqpoint{3.022705in}{0.739656in}}%
\pgfpathlineto{\pgfqpoint{3.021852in}{0.739656in}}%
\pgfpathlineto{\pgfqpoint{3.020999in}{0.739656in}}%
\pgfpathlineto{\pgfqpoint{3.020146in}{0.739656in}}%
\pgfpathlineto{\pgfqpoint{3.019293in}{0.739656in}}%
\pgfpathlineto{\pgfqpoint{3.018440in}{0.739656in}}%
\pgfpathlineto{\pgfqpoint{3.017587in}{0.739656in}}%
\pgfpathlineto{\pgfqpoint{3.016734in}{0.739656in}}%
\pgfpathlineto{\pgfqpoint{3.015882in}{0.739656in}}%
\pgfpathlineto{\pgfqpoint{3.015029in}{0.739656in}}%
\pgfpathlineto{\pgfqpoint{3.014176in}{0.739656in}}%
\pgfpathlineto{\pgfqpoint{3.013323in}{0.739656in}}%
\pgfpathlineto{\pgfqpoint{3.012470in}{0.739656in}}%
\pgfpathlineto{\pgfqpoint{3.011617in}{0.739656in}}%
\pgfpathlineto{\pgfqpoint{3.010764in}{0.739656in}}%
\pgfpathlineto{\pgfqpoint{3.009911in}{0.739656in}}%
\pgfpathlineto{\pgfqpoint{3.009058in}{0.739656in}}%
\pgfpathlineto{\pgfqpoint{3.008206in}{0.739656in}}%
\pgfpathlineto{\pgfqpoint{3.007353in}{0.739656in}}%
\pgfpathlineto{\pgfqpoint{3.006500in}{0.739656in}}%
\pgfpathlineto{\pgfqpoint{3.005647in}{0.739656in}}%
\pgfpathlineto{\pgfqpoint{3.004794in}{0.739656in}}%
\pgfpathlineto{\pgfqpoint{3.003941in}{0.739656in}}%
\pgfpathlineto{\pgfqpoint{3.003088in}{0.739656in}}%
\pgfpathlineto{\pgfqpoint{3.002235in}{0.739656in}}%
\pgfpathlineto{\pgfqpoint{3.001383in}{0.739656in}}%
\pgfpathlineto{\pgfqpoint{3.000530in}{0.739656in}}%
\pgfpathlineto{\pgfqpoint{2.999677in}{0.739656in}}%
\pgfpathlineto{\pgfqpoint{2.998824in}{0.739656in}}%
\pgfpathlineto{\pgfqpoint{2.997971in}{0.739656in}}%
\pgfpathlineto{\pgfqpoint{2.997118in}{0.739656in}}%
\pgfpathlineto{\pgfqpoint{2.996265in}{0.739656in}}%
\pgfpathlineto{\pgfqpoint{2.995412in}{0.739656in}}%
\pgfpathlineto{\pgfqpoint{2.994560in}{0.739656in}}%
\pgfpathlineto{\pgfqpoint{2.993707in}{0.739656in}}%
\pgfpathlineto{\pgfqpoint{2.992854in}{0.739656in}}%
\pgfpathlineto{\pgfqpoint{2.992001in}{0.739656in}}%
\pgfpathlineto{\pgfqpoint{2.991148in}{0.739656in}}%
\pgfpathlineto{\pgfqpoint{2.990295in}{0.739656in}}%
\pgfpathlineto{\pgfqpoint{2.989442in}{0.739656in}}%
\pgfpathlineto{\pgfqpoint{2.988589in}{0.739656in}}%
\pgfpathlineto{\pgfqpoint{2.987737in}{0.739656in}}%
\pgfpathlineto{\pgfqpoint{2.986884in}{0.739656in}}%
\pgfpathlineto{\pgfqpoint{2.986031in}{0.739656in}}%
\pgfpathlineto{\pgfqpoint{2.985178in}{0.739656in}}%
\pgfpathlineto{\pgfqpoint{2.984325in}{0.739656in}}%
\pgfpathlineto{\pgfqpoint{2.983472in}{0.739656in}}%
\pgfpathlineto{\pgfqpoint{2.982619in}{0.739656in}}%
\pgfpathlineto{\pgfqpoint{2.981766in}{0.739656in}}%
\pgfpathlineto{\pgfqpoint{2.980913in}{0.739656in}}%
\pgfpathlineto{\pgfqpoint{2.980061in}{0.739656in}}%
\pgfpathlineto{\pgfqpoint{2.979208in}{0.739656in}}%
\pgfpathlineto{\pgfqpoint{2.978355in}{0.739656in}}%
\pgfpathlineto{\pgfqpoint{2.977502in}{0.739656in}}%
\pgfpathlineto{\pgfqpoint{2.976649in}{0.739656in}}%
\pgfpathlineto{\pgfqpoint{2.975796in}{0.739656in}}%
\pgfpathlineto{\pgfqpoint{2.974943in}{0.739656in}}%
\pgfpathlineto{\pgfqpoint{2.974090in}{0.739656in}}%
\pgfpathlineto{\pgfqpoint{2.973238in}{0.739656in}}%
\pgfpathlineto{\pgfqpoint{2.972385in}{0.739656in}}%
\pgfpathlineto{\pgfqpoint{2.971532in}{0.739656in}}%
\pgfpathlineto{\pgfqpoint{2.970679in}{0.739656in}}%
\pgfpathlineto{\pgfqpoint{2.969826in}{0.739656in}}%
\pgfpathlineto{\pgfqpoint{2.968973in}{0.739656in}}%
\pgfpathlineto{\pgfqpoint{2.968120in}{0.739656in}}%
\pgfpathlineto{\pgfqpoint{2.967267in}{0.739656in}}%
\pgfpathlineto{\pgfqpoint{2.966415in}{0.739656in}}%
\pgfpathlineto{\pgfqpoint{2.965562in}{0.739656in}}%
\pgfpathlineto{\pgfqpoint{2.964709in}{0.739656in}}%
\pgfpathlineto{\pgfqpoint{2.963856in}{0.739656in}}%
\pgfpathlineto{\pgfqpoint{2.963003in}{0.739656in}}%
\pgfpathlineto{\pgfqpoint{2.962150in}{0.739656in}}%
\pgfpathlineto{\pgfqpoint{2.961297in}{0.739656in}}%
\pgfpathlineto{\pgfqpoint{2.960444in}{0.739656in}}%
\pgfpathlineto{\pgfqpoint{2.959592in}{0.739656in}}%
\pgfpathlineto{\pgfqpoint{2.958739in}{0.739656in}}%
\pgfpathlineto{\pgfqpoint{2.957886in}{0.739656in}}%
\pgfpathlineto{\pgfqpoint{2.957033in}{0.739656in}}%
\pgfpathlineto{\pgfqpoint{2.956180in}{0.739656in}}%
\pgfpathlineto{\pgfqpoint{2.955327in}{0.739656in}}%
\pgfpathlineto{\pgfqpoint{2.954474in}{0.739656in}}%
\pgfpathlineto{\pgfqpoint{2.953621in}{0.739656in}}%
\pgfpathlineto{\pgfqpoint{2.952769in}{0.739656in}}%
\pgfpathlineto{\pgfqpoint{2.951916in}{0.739656in}}%
\pgfpathlineto{\pgfqpoint{2.951063in}{0.739656in}}%
\pgfpathlineto{\pgfqpoint{2.950210in}{0.739656in}}%
\pgfpathlineto{\pgfqpoint{2.949357in}{0.739656in}}%
\pgfpathlineto{\pgfqpoint{2.948504in}{0.739656in}}%
\pgfpathlineto{\pgfqpoint{2.947651in}{0.739656in}}%
\pgfpathlineto{\pgfqpoint{2.946798in}{0.739656in}}%
\pgfpathlineto{\pgfqpoint{2.945945in}{0.739656in}}%
\pgfpathlineto{\pgfqpoint{2.945093in}{0.739656in}}%
\pgfpathlineto{\pgfqpoint{2.944240in}{0.739656in}}%
\pgfpathlineto{\pgfqpoint{2.943387in}{0.739656in}}%
\pgfpathlineto{\pgfqpoint{2.942534in}{0.739656in}}%
\pgfpathlineto{\pgfqpoint{2.941681in}{0.739656in}}%
\pgfpathlineto{\pgfqpoint{2.940828in}{0.739656in}}%
\pgfpathlineto{\pgfqpoint{2.939975in}{0.739656in}}%
\pgfpathlineto{\pgfqpoint{2.939122in}{0.739656in}}%
\pgfpathlineto{\pgfqpoint{2.938270in}{0.739656in}}%
\pgfpathlineto{\pgfqpoint{2.937417in}{0.739656in}}%
\pgfpathlineto{\pgfqpoint{2.936564in}{0.739656in}}%
\pgfpathlineto{\pgfqpoint{2.935711in}{0.739656in}}%
\pgfpathlineto{\pgfqpoint{2.934858in}{0.739656in}}%
\pgfpathlineto{\pgfqpoint{2.934005in}{0.739656in}}%
\pgfpathlineto{\pgfqpoint{2.933152in}{0.739656in}}%
\pgfpathlineto{\pgfqpoint{2.932299in}{0.739656in}}%
\pgfpathlineto{\pgfqpoint{2.931447in}{0.739656in}}%
\pgfpathlineto{\pgfqpoint{2.930594in}{0.739656in}}%
\pgfpathlineto{\pgfqpoint{2.929741in}{0.739656in}}%
\pgfpathlineto{\pgfqpoint{2.928888in}{0.739656in}}%
\pgfpathlineto{\pgfqpoint{2.928035in}{0.739656in}}%
\pgfpathlineto{\pgfqpoint{2.927182in}{0.739656in}}%
\pgfpathlineto{\pgfqpoint{2.926329in}{0.739656in}}%
\pgfpathlineto{\pgfqpoint{2.925476in}{0.739656in}}%
\pgfpathlineto{\pgfqpoint{2.924624in}{0.739656in}}%
\pgfpathlineto{\pgfqpoint{2.923771in}{0.739656in}}%
\pgfpathlineto{\pgfqpoint{2.922918in}{0.739656in}}%
\pgfpathlineto{\pgfqpoint{2.922065in}{0.739656in}}%
\pgfpathlineto{\pgfqpoint{2.921212in}{0.739656in}}%
\pgfpathlineto{\pgfqpoint{2.920359in}{0.739656in}}%
\pgfpathlineto{\pgfqpoint{2.919506in}{0.739656in}}%
\pgfpathlineto{\pgfqpoint{2.918653in}{0.739656in}}%
\pgfpathlineto{\pgfqpoint{2.917800in}{0.739656in}}%
\pgfpathlineto{\pgfqpoint{2.916948in}{0.739656in}}%
\pgfpathlineto{\pgfqpoint{2.916095in}{0.739656in}}%
\pgfpathlineto{\pgfqpoint{2.915242in}{0.739656in}}%
\pgfpathlineto{\pgfqpoint{2.914389in}{0.739656in}}%
\pgfpathlineto{\pgfqpoint{2.913536in}{0.739656in}}%
\pgfpathlineto{\pgfqpoint{2.912683in}{0.739656in}}%
\pgfpathlineto{\pgfqpoint{2.911830in}{0.739656in}}%
\pgfpathlineto{\pgfqpoint{2.910977in}{0.739656in}}%
\pgfpathlineto{\pgfqpoint{2.910125in}{0.739656in}}%
\pgfpathlineto{\pgfqpoint{2.909272in}{0.739656in}}%
\pgfpathlineto{\pgfqpoint{2.908419in}{0.739656in}}%
\pgfpathlineto{\pgfqpoint{2.907566in}{0.739656in}}%
\pgfpathlineto{\pgfqpoint{2.906713in}{0.739656in}}%
\pgfpathlineto{\pgfqpoint{2.905860in}{0.739656in}}%
\pgfpathlineto{\pgfqpoint{2.905007in}{0.739656in}}%
\pgfpathlineto{\pgfqpoint{2.904154in}{0.739656in}}%
\pgfpathlineto{\pgfqpoint{2.903302in}{0.739656in}}%
\pgfpathlineto{\pgfqpoint{2.902449in}{0.739656in}}%
\pgfpathlineto{\pgfqpoint{2.901596in}{0.739656in}}%
\pgfpathlineto{\pgfqpoint{2.900743in}{0.739656in}}%
\pgfpathlineto{\pgfqpoint{2.899890in}{0.739656in}}%
\pgfpathlineto{\pgfqpoint{2.899037in}{0.739656in}}%
\pgfpathlineto{\pgfqpoint{2.898184in}{0.739656in}}%
\pgfpathlineto{\pgfqpoint{2.897331in}{0.739656in}}%
\pgfpathlineto{\pgfqpoint{2.896479in}{0.739656in}}%
\pgfpathlineto{\pgfqpoint{2.895626in}{0.739656in}}%
\pgfpathlineto{\pgfqpoint{2.894773in}{0.739656in}}%
\pgfpathlineto{\pgfqpoint{2.893920in}{0.739656in}}%
\pgfpathlineto{\pgfqpoint{2.893067in}{0.739656in}}%
\pgfpathlineto{\pgfqpoint{2.892214in}{0.739656in}}%
\pgfpathlineto{\pgfqpoint{2.891361in}{0.739656in}}%
\pgfpathlineto{\pgfqpoint{2.890508in}{0.739656in}}%
\pgfpathlineto{\pgfqpoint{2.889655in}{0.739656in}}%
\pgfpathlineto{\pgfqpoint{2.888803in}{0.739656in}}%
\pgfpathlineto{\pgfqpoint{2.887950in}{0.739656in}}%
\pgfpathlineto{\pgfqpoint{2.887097in}{0.739656in}}%
\pgfpathlineto{\pgfqpoint{2.886244in}{0.739656in}}%
\pgfpathlineto{\pgfqpoint{2.885391in}{0.739656in}}%
\pgfpathlineto{\pgfqpoint{2.884538in}{0.739656in}}%
\pgfpathlineto{\pgfqpoint{2.883685in}{0.739656in}}%
\pgfpathlineto{\pgfqpoint{2.882832in}{0.739656in}}%
\pgfpathlineto{\pgfqpoint{2.881980in}{0.739656in}}%
\pgfpathlineto{\pgfqpoint{2.881127in}{0.739656in}}%
\pgfpathlineto{\pgfqpoint{2.880274in}{0.739656in}}%
\pgfpathlineto{\pgfqpoint{2.879421in}{0.739656in}}%
\pgfpathlineto{\pgfqpoint{2.878568in}{0.739656in}}%
\pgfpathlineto{\pgfqpoint{2.877715in}{0.739656in}}%
\pgfpathlineto{\pgfqpoint{2.876862in}{0.739656in}}%
\pgfpathlineto{\pgfqpoint{2.876009in}{0.739656in}}%
\pgfpathlineto{\pgfqpoint{2.875157in}{0.739656in}}%
\pgfpathlineto{\pgfqpoint{2.874304in}{0.739656in}}%
\pgfpathlineto{\pgfqpoint{2.873451in}{0.739656in}}%
\pgfpathlineto{\pgfqpoint{2.872598in}{0.739656in}}%
\pgfpathlineto{\pgfqpoint{2.871745in}{0.739656in}}%
\pgfpathlineto{\pgfqpoint{2.870892in}{0.739656in}}%
\pgfpathlineto{\pgfqpoint{2.870039in}{0.739656in}}%
\pgfpathlineto{\pgfqpoint{2.869186in}{0.739656in}}%
\pgfpathlineto{\pgfqpoint{2.868334in}{0.739656in}}%
\pgfpathlineto{\pgfqpoint{2.867481in}{0.739656in}}%
\pgfpathlineto{\pgfqpoint{2.866628in}{0.739656in}}%
\pgfpathlineto{\pgfqpoint{2.865775in}{0.739656in}}%
\pgfpathlineto{\pgfqpoint{2.864922in}{0.739656in}}%
\pgfpathlineto{\pgfqpoint{2.864069in}{0.739656in}}%
\pgfpathlineto{\pgfqpoint{2.863216in}{0.739656in}}%
\pgfpathlineto{\pgfqpoint{2.862363in}{0.739656in}}%
\pgfpathlineto{\pgfqpoint{2.861510in}{0.739656in}}%
\pgfpathlineto{\pgfqpoint{2.860658in}{0.739656in}}%
\pgfpathlineto{\pgfqpoint{2.859805in}{0.739656in}}%
\pgfpathlineto{\pgfqpoint{2.858952in}{0.739656in}}%
\pgfpathlineto{\pgfqpoint{2.858099in}{0.739656in}}%
\pgfpathlineto{\pgfqpoint{2.857246in}{0.739656in}}%
\pgfpathlineto{\pgfqpoint{2.856393in}{0.739656in}}%
\pgfpathlineto{\pgfqpoint{2.855540in}{0.739656in}}%
\pgfpathlineto{\pgfqpoint{2.854687in}{0.739656in}}%
\pgfpathlineto{\pgfqpoint{2.853835in}{0.739656in}}%
\pgfpathlineto{\pgfqpoint{2.852982in}{0.739656in}}%
\pgfpathlineto{\pgfqpoint{2.852129in}{0.739656in}}%
\pgfpathlineto{\pgfqpoint{2.851276in}{0.739656in}}%
\pgfpathlineto{\pgfqpoint{2.850423in}{0.739656in}}%
\pgfpathlineto{\pgfqpoint{2.849570in}{0.739656in}}%
\pgfpathlineto{\pgfqpoint{2.848717in}{0.739656in}}%
\pgfpathlineto{\pgfqpoint{2.847864in}{0.739656in}}%
\pgfpathlineto{\pgfqpoint{2.847012in}{0.739656in}}%
\pgfpathlineto{\pgfqpoint{2.846159in}{0.739656in}}%
\pgfpathlineto{\pgfqpoint{2.845306in}{0.739656in}}%
\pgfpathlineto{\pgfqpoint{2.844453in}{0.739656in}}%
\pgfpathlineto{\pgfqpoint{2.843600in}{0.739656in}}%
\pgfpathlineto{\pgfqpoint{2.842747in}{0.739656in}}%
\pgfpathlineto{\pgfqpoint{2.841894in}{0.739656in}}%
\pgfpathlineto{\pgfqpoint{2.841041in}{0.739656in}}%
\pgfpathlineto{\pgfqpoint{2.840189in}{0.739656in}}%
\pgfpathlineto{\pgfqpoint{2.839336in}{0.739656in}}%
\pgfpathlineto{\pgfqpoint{2.838483in}{0.739656in}}%
\pgfpathlineto{\pgfqpoint{2.837630in}{0.739656in}}%
\pgfpathlineto{\pgfqpoint{2.836777in}{0.739656in}}%
\pgfpathlineto{\pgfqpoint{2.835924in}{0.739656in}}%
\pgfpathlineto{\pgfqpoint{2.835071in}{0.739656in}}%
\pgfpathlineto{\pgfqpoint{2.834218in}{0.739656in}}%
\pgfpathlineto{\pgfqpoint{2.833366in}{0.739656in}}%
\pgfpathlineto{\pgfqpoint{2.832513in}{0.739656in}}%
\pgfpathlineto{\pgfqpoint{2.831660in}{0.739656in}}%
\pgfpathlineto{\pgfqpoint{2.830807in}{0.739656in}}%
\pgfpathlineto{\pgfqpoint{2.829954in}{0.739656in}}%
\pgfpathlineto{\pgfqpoint{2.829101in}{0.739656in}}%
\pgfpathlineto{\pgfqpoint{2.828248in}{0.739656in}}%
\pgfpathlineto{\pgfqpoint{2.827395in}{0.739656in}}%
\pgfpathlineto{\pgfqpoint{2.826542in}{0.739656in}}%
\pgfpathlineto{\pgfqpoint{2.825690in}{0.739656in}}%
\pgfpathlineto{\pgfqpoint{2.824837in}{0.739656in}}%
\pgfpathlineto{\pgfqpoint{2.823984in}{0.739656in}}%
\pgfpathlineto{\pgfqpoint{2.823131in}{0.739656in}}%
\pgfpathlineto{\pgfqpoint{2.822278in}{0.739656in}}%
\pgfpathlineto{\pgfqpoint{2.821425in}{0.739656in}}%
\pgfpathlineto{\pgfqpoint{2.820572in}{0.739656in}}%
\pgfpathlineto{\pgfqpoint{2.819719in}{0.739656in}}%
\pgfpathlineto{\pgfqpoint{2.818867in}{0.739656in}}%
\pgfpathlineto{\pgfqpoint{2.818014in}{0.739656in}}%
\pgfpathlineto{\pgfqpoint{2.817161in}{0.739656in}}%
\pgfpathlineto{\pgfqpoint{2.816308in}{0.739656in}}%
\pgfpathlineto{\pgfqpoint{2.815455in}{0.739656in}}%
\pgfpathlineto{\pgfqpoint{2.814602in}{0.739656in}}%
\pgfpathlineto{\pgfqpoint{2.813749in}{0.739656in}}%
\pgfpathlineto{\pgfqpoint{2.812896in}{0.739656in}}%
\pgfpathlineto{\pgfqpoint{2.812044in}{0.739656in}}%
\pgfpathlineto{\pgfqpoint{2.811191in}{0.739656in}}%
\pgfpathlineto{\pgfqpoint{2.810338in}{0.739656in}}%
\pgfpathlineto{\pgfqpoint{2.809485in}{0.739656in}}%
\pgfpathlineto{\pgfqpoint{2.808632in}{0.739656in}}%
\pgfpathlineto{\pgfqpoint{2.807779in}{0.739656in}}%
\pgfpathlineto{\pgfqpoint{2.806926in}{0.739656in}}%
\pgfpathlineto{\pgfqpoint{2.806073in}{0.739656in}}%
\pgfpathlineto{\pgfqpoint{2.805221in}{0.739656in}}%
\pgfpathlineto{\pgfqpoint{2.804368in}{0.739656in}}%
\pgfpathlineto{\pgfqpoint{2.803515in}{0.739656in}}%
\pgfpathlineto{\pgfqpoint{2.802662in}{0.739656in}}%
\pgfpathlineto{\pgfqpoint{2.801809in}{0.739656in}}%
\pgfpathlineto{\pgfqpoint{2.800956in}{0.739656in}}%
\pgfpathlineto{\pgfqpoint{2.800103in}{0.739656in}}%
\pgfpathlineto{\pgfqpoint{2.799250in}{0.739656in}}%
\pgfpathlineto{\pgfqpoint{2.798397in}{0.739656in}}%
\pgfpathlineto{\pgfqpoint{2.797545in}{0.739656in}}%
\pgfpathlineto{\pgfqpoint{2.796692in}{0.739656in}}%
\pgfpathlineto{\pgfqpoint{2.795839in}{0.739656in}}%
\pgfpathlineto{\pgfqpoint{2.794986in}{0.739656in}}%
\pgfpathlineto{\pgfqpoint{2.794133in}{0.739656in}}%
\pgfpathlineto{\pgfqpoint{2.793280in}{0.739656in}}%
\pgfpathlineto{\pgfqpoint{2.792427in}{0.739656in}}%
\pgfpathlineto{\pgfqpoint{2.791574in}{0.739656in}}%
\pgfpathlineto{\pgfqpoint{2.790722in}{0.739656in}}%
\pgfpathlineto{\pgfqpoint{2.789869in}{0.739656in}}%
\pgfpathlineto{\pgfqpoint{2.789016in}{0.739656in}}%
\pgfpathlineto{\pgfqpoint{2.788163in}{0.739656in}}%
\pgfpathlineto{\pgfqpoint{2.787310in}{0.739656in}}%
\pgfpathlineto{\pgfqpoint{2.786457in}{0.739656in}}%
\pgfpathlineto{\pgfqpoint{2.785604in}{0.739656in}}%
\pgfpathlineto{\pgfqpoint{2.784751in}{0.739656in}}%
\pgfpathlineto{\pgfqpoint{2.783899in}{0.739656in}}%
\pgfpathlineto{\pgfqpoint{2.783046in}{0.739656in}}%
\pgfpathlineto{\pgfqpoint{2.782193in}{0.739656in}}%
\pgfpathlineto{\pgfqpoint{2.781340in}{0.739656in}}%
\pgfpathlineto{\pgfqpoint{2.780487in}{0.739656in}}%
\pgfpathlineto{\pgfqpoint{2.779634in}{0.739656in}}%
\pgfpathlineto{\pgfqpoint{2.778781in}{0.739656in}}%
\pgfpathlineto{\pgfqpoint{2.777928in}{0.739656in}}%
\pgfpathlineto{\pgfqpoint{2.777076in}{0.739656in}}%
\pgfpathlineto{\pgfqpoint{2.776223in}{0.739656in}}%
\pgfpathlineto{\pgfqpoint{2.775370in}{0.739656in}}%
\pgfpathlineto{\pgfqpoint{2.774517in}{0.739656in}}%
\pgfpathlineto{\pgfqpoint{2.773664in}{0.739656in}}%
\pgfpathlineto{\pgfqpoint{2.772811in}{0.739656in}}%
\pgfpathlineto{\pgfqpoint{2.771958in}{0.739656in}}%
\pgfpathlineto{\pgfqpoint{2.771105in}{0.739656in}}%
\pgfpathlineto{\pgfqpoint{2.770252in}{0.739656in}}%
\pgfpathlineto{\pgfqpoint{2.769400in}{0.739656in}}%
\pgfpathlineto{\pgfqpoint{2.768547in}{0.739656in}}%
\pgfpathlineto{\pgfqpoint{2.767694in}{0.739656in}}%
\pgfpathlineto{\pgfqpoint{2.766841in}{0.739656in}}%
\pgfpathlineto{\pgfqpoint{2.765988in}{0.739656in}}%
\pgfpathlineto{\pgfqpoint{2.765135in}{0.739656in}}%
\pgfpathlineto{\pgfqpoint{2.764282in}{0.739656in}}%
\pgfpathlineto{\pgfqpoint{2.763429in}{0.739656in}}%
\pgfpathlineto{\pgfqpoint{2.762577in}{0.739656in}}%
\pgfpathlineto{\pgfqpoint{2.761724in}{0.739656in}}%
\pgfpathlineto{\pgfqpoint{2.760871in}{0.739656in}}%
\pgfpathlineto{\pgfqpoint{2.760018in}{0.739656in}}%
\pgfpathlineto{\pgfqpoint{2.759165in}{0.739656in}}%
\pgfpathlineto{\pgfqpoint{2.758312in}{0.739656in}}%
\pgfpathlineto{\pgfqpoint{2.757459in}{0.739656in}}%
\pgfpathlineto{\pgfqpoint{2.756606in}{0.739656in}}%
\pgfpathlineto{\pgfqpoint{2.755754in}{0.739656in}}%
\pgfpathlineto{\pgfqpoint{2.754901in}{0.739656in}}%
\pgfpathlineto{\pgfqpoint{2.754048in}{0.739656in}}%
\pgfpathlineto{\pgfqpoint{2.753195in}{0.739656in}}%
\pgfpathlineto{\pgfqpoint{2.752342in}{0.739656in}}%
\pgfpathlineto{\pgfqpoint{2.751489in}{0.739656in}}%
\pgfpathlineto{\pgfqpoint{2.750636in}{0.739656in}}%
\pgfpathlineto{\pgfqpoint{2.749783in}{0.739656in}}%
\pgfpathlineto{\pgfqpoint{2.748931in}{0.739656in}}%
\pgfpathlineto{\pgfqpoint{2.748078in}{0.739656in}}%
\pgfpathlineto{\pgfqpoint{2.747225in}{0.739656in}}%
\pgfpathlineto{\pgfqpoint{2.746372in}{0.739656in}}%
\pgfpathlineto{\pgfqpoint{2.745519in}{0.739656in}}%
\pgfpathlineto{\pgfqpoint{2.744666in}{0.739656in}}%
\pgfpathlineto{\pgfqpoint{2.743813in}{0.739656in}}%
\pgfpathlineto{\pgfqpoint{2.742960in}{0.739656in}}%
\pgfpathlineto{\pgfqpoint{2.742107in}{0.739656in}}%
\pgfpathlineto{\pgfqpoint{2.741255in}{0.739656in}}%
\pgfpathlineto{\pgfqpoint{2.740402in}{0.739656in}}%
\pgfpathlineto{\pgfqpoint{2.739549in}{0.739656in}}%
\pgfpathlineto{\pgfqpoint{2.738696in}{0.739656in}}%
\pgfpathlineto{\pgfqpoint{2.737843in}{0.739656in}}%
\pgfpathlineto{\pgfqpoint{2.736990in}{0.739656in}}%
\pgfpathlineto{\pgfqpoint{2.736137in}{0.739656in}}%
\pgfpathlineto{\pgfqpoint{2.735284in}{0.739656in}}%
\pgfpathlineto{\pgfqpoint{2.734432in}{0.739656in}}%
\pgfpathlineto{\pgfqpoint{2.733579in}{0.739656in}}%
\pgfpathlineto{\pgfqpoint{2.732726in}{0.739656in}}%
\pgfpathlineto{\pgfqpoint{2.731873in}{0.739656in}}%
\pgfpathlineto{\pgfqpoint{2.731020in}{0.739656in}}%
\pgfpathlineto{\pgfqpoint{2.730167in}{0.739656in}}%
\pgfpathlineto{\pgfqpoint{2.729314in}{0.739656in}}%
\pgfpathlineto{\pgfqpoint{2.728461in}{0.739656in}}%
\pgfpathlineto{\pgfqpoint{2.727609in}{0.739656in}}%
\pgfpathlineto{\pgfqpoint{2.726756in}{0.739656in}}%
\pgfpathlineto{\pgfqpoint{2.725903in}{0.739656in}}%
\pgfpathlineto{\pgfqpoint{2.725050in}{0.739656in}}%
\pgfpathlineto{\pgfqpoint{2.724197in}{0.739656in}}%
\pgfpathlineto{\pgfqpoint{2.723344in}{0.739656in}}%
\pgfpathlineto{\pgfqpoint{2.722491in}{0.739656in}}%
\pgfpathlineto{\pgfqpoint{2.721638in}{0.739656in}}%
\pgfpathlineto{\pgfqpoint{2.720786in}{0.739656in}}%
\pgfpathlineto{\pgfqpoint{2.719933in}{0.739656in}}%
\pgfpathlineto{\pgfqpoint{2.719080in}{0.739656in}}%
\pgfpathlineto{\pgfqpoint{2.718227in}{0.739656in}}%
\pgfpathlineto{\pgfqpoint{2.717374in}{0.739656in}}%
\pgfpathlineto{\pgfqpoint{2.716521in}{0.739656in}}%
\pgfpathlineto{\pgfqpoint{2.715668in}{0.739656in}}%
\pgfpathlineto{\pgfqpoint{2.714815in}{0.739656in}}%
\pgfpathlineto{\pgfqpoint{2.713963in}{0.739656in}}%
\pgfpathlineto{\pgfqpoint{2.713110in}{0.739656in}}%
\pgfpathlineto{\pgfqpoint{2.712257in}{0.739656in}}%
\pgfpathlineto{\pgfqpoint{2.711404in}{0.739656in}}%
\pgfpathlineto{\pgfqpoint{2.710551in}{0.739656in}}%
\pgfpathlineto{\pgfqpoint{2.709698in}{0.739656in}}%
\pgfpathlineto{\pgfqpoint{2.708845in}{0.739656in}}%
\pgfpathlineto{\pgfqpoint{2.707992in}{0.739656in}}%
\pgfpathlineto{\pgfqpoint{2.707139in}{0.739656in}}%
\pgfpathlineto{\pgfqpoint{2.706287in}{0.739656in}}%
\pgfpathlineto{\pgfqpoint{2.705434in}{0.739656in}}%
\pgfpathlineto{\pgfqpoint{2.704581in}{0.739656in}}%
\pgfpathlineto{\pgfqpoint{2.703728in}{0.739656in}}%
\pgfpathlineto{\pgfqpoint{2.702875in}{0.739656in}}%
\pgfpathlineto{\pgfqpoint{2.702022in}{0.739656in}}%
\pgfpathlineto{\pgfqpoint{2.701169in}{0.739656in}}%
\pgfpathlineto{\pgfqpoint{2.700316in}{0.739656in}}%
\pgfpathlineto{\pgfqpoint{2.699464in}{0.739656in}}%
\pgfpathlineto{\pgfqpoint{2.698611in}{0.739656in}}%
\pgfpathlineto{\pgfqpoint{2.697758in}{0.739656in}}%
\pgfpathlineto{\pgfqpoint{2.696905in}{0.739656in}}%
\pgfpathlineto{\pgfqpoint{2.696052in}{0.739656in}}%
\pgfpathlineto{\pgfqpoint{2.695199in}{0.739656in}}%
\pgfpathlineto{\pgfqpoint{2.694346in}{0.739656in}}%
\pgfpathlineto{\pgfqpoint{2.693493in}{0.739656in}}%
\pgfpathlineto{\pgfqpoint{2.692641in}{0.739656in}}%
\pgfpathlineto{\pgfqpoint{2.691788in}{0.739656in}}%
\pgfpathlineto{\pgfqpoint{2.690935in}{0.739656in}}%
\pgfpathlineto{\pgfqpoint{2.690082in}{0.739656in}}%
\pgfpathlineto{\pgfqpoint{2.689229in}{0.739656in}}%
\pgfpathlineto{\pgfqpoint{2.688376in}{0.739656in}}%
\pgfpathlineto{\pgfqpoint{2.687523in}{0.739656in}}%
\pgfpathlineto{\pgfqpoint{2.686670in}{0.739656in}}%
\pgfpathlineto{\pgfqpoint{2.685818in}{0.739656in}}%
\pgfpathlineto{\pgfqpoint{2.684965in}{0.739656in}}%
\pgfpathlineto{\pgfqpoint{2.684112in}{0.739656in}}%
\pgfpathlineto{\pgfqpoint{2.683259in}{0.739656in}}%
\pgfpathlineto{\pgfqpoint{2.682406in}{0.739656in}}%
\pgfpathlineto{\pgfqpoint{2.681553in}{0.739656in}}%
\pgfpathlineto{\pgfqpoint{2.680700in}{0.739656in}}%
\pgfpathlineto{\pgfqpoint{2.679847in}{0.739656in}}%
\pgfpathlineto{\pgfqpoint{2.678994in}{0.739656in}}%
\pgfpathlineto{\pgfqpoint{2.678142in}{0.739656in}}%
\pgfpathlineto{\pgfqpoint{2.677289in}{0.739656in}}%
\pgfpathlineto{\pgfqpoint{2.676436in}{0.739656in}}%
\pgfpathlineto{\pgfqpoint{2.675583in}{0.739656in}}%
\pgfpathlineto{\pgfqpoint{2.674730in}{0.739656in}}%
\pgfpathlineto{\pgfqpoint{2.673877in}{0.739656in}}%
\pgfpathlineto{\pgfqpoint{2.673024in}{0.739656in}}%
\pgfpathlineto{\pgfqpoint{2.672171in}{0.739656in}}%
\pgfpathlineto{\pgfqpoint{2.671319in}{0.739656in}}%
\pgfpathlineto{\pgfqpoint{2.670466in}{0.739656in}}%
\pgfpathlineto{\pgfqpoint{2.669613in}{0.739656in}}%
\pgfpathlineto{\pgfqpoint{2.668760in}{0.739656in}}%
\pgfpathlineto{\pgfqpoint{2.667907in}{0.739656in}}%
\pgfpathlineto{\pgfqpoint{2.667054in}{0.739656in}}%
\pgfpathlineto{\pgfqpoint{2.666201in}{0.739656in}}%
\pgfpathlineto{\pgfqpoint{2.665348in}{0.739656in}}%
\pgfpathlineto{\pgfqpoint{2.664496in}{0.739656in}}%
\pgfpathlineto{\pgfqpoint{2.663643in}{0.739656in}}%
\pgfpathlineto{\pgfqpoint{2.662790in}{0.739656in}}%
\pgfpathlineto{\pgfqpoint{2.661937in}{0.739656in}}%
\pgfpathlineto{\pgfqpoint{2.661084in}{0.739656in}}%
\pgfpathlineto{\pgfqpoint{2.660231in}{0.739656in}}%
\pgfpathlineto{\pgfqpoint{2.659378in}{0.739656in}}%
\pgfpathlineto{\pgfqpoint{2.658525in}{0.739656in}}%
\pgfpathlineto{\pgfqpoint{2.657673in}{0.739656in}}%
\pgfpathlineto{\pgfqpoint{2.656820in}{0.739656in}}%
\pgfpathlineto{\pgfqpoint{2.655967in}{0.739656in}}%
\pgfpathlineto{\pgfqpoint{2.655114in}{0.739656in}}%
\pgfpathlineto{\pgfqpoint{2.654261in}{0.739656in}}%
\pgfpathlineto{\pgfqpoint{2.653408in}{0.739656in}}%
\pgfpathlineto{\pgfqpoint{2.652555in}{0.739656in}}%
\pgfpathlineto{\pgfqpoint{2.651702in}{0.739656in}}%
\pgfpathlineto{\pgfqpoint{2.650849in}{0.739656in}}%
\pgfpathlineto{\pgfqpoint{2.649997in}{0.739656in}}%
\pgfpathlineto{\pgfqpoint{2.649144in}{0.739656in}}%
\pgfpathlineto{\pgfqpoint{2.648291in}{0.739656in}}%
\pgfpathlineto{\pgfqpoint{2.647438in}{0.739656in}}%
\pgfpathlineto{\pgfqpoint{2.646585in}{0.739656in}}%
\pgfpathlineto{\pgfqpoint{2.645732in}{0.739656in}}%
\pgfpathlineto{\pgfqpoint{2.644879in}{0.739656in}}%
\pgfpathlineto{\pgfqpoint{2.644026in}{0.739656in}}%
\pgfpathlineto{\pgfqpoint{2.643174in}{0.739656in}}%
\pgfpathlineto{\pgfqpoint{2.642321in}{0.739656in}}%
\pgfpathlineto{\pgfqpoint{2.641468in}{0.739656in}}%
\pgfpathlineto{\pgfqpoint{2.640615in}{0.739656in}}%
\pgfpathlineto{\pgfqpoint{2.639762in}{0.739656in}}%
\pgfpathlineto{\pgfqpoint{2.638909in}{0.739656in}}%
\pgfpathlineto{\pgfqpoint{2.638056in}{0.739656in}}%
\pgfpathlineto{\pgfqpoint{2.637203in}{0.739656in}}%
\pgfpathlineto{\pgfqpoint{2.636351in}{0.739656in}}%
\pgfpathlineto{\pgfqpoint{2.635498in}{0.739656in}}%
\pgfpathlineto{\pgfqpoint{2.634645in}{0.739656in}}%
\pgfpathlineto{\pgfqpoint{2.633792in}{0.739656in}}%
\pgfpathlineto{\pgfqpoint{2.632939in}{0.739656in}}%
\pgfpathlineto{\pgfqpoint{2.632086in}{0.739656in}}%
\pgfpathlineto{\pgfqpoint{2.631233in}{0.739656in}}%
\pgfpathlineto{\pgfqpoint{2.630380in}{0.739656in}}%
\pgfpathlineto{\pgfqpoint{2.629528in}{0.739656in}}%
\pgfpathlineto{\pgfqpoint{2.628675in}{0.739656in}}%
\pgfpathlineto{\pgfqpoint{2.627822in}{0.739656in}}%
\pgfpathlineto{\pgfqpoint{2.626969in}{0.739656in}}%
\pgfpathlineto{\pgfqpoint{2.626116in}{0.739656in}}%
\pgfpathlineto{\pgfqpoint{2.625263in}{0.739656in}}%
\pgfpathlineto{\pgfqpoint{2.624410in}{0.739656in}}%
\pgfpathlineto{\pgfqpoint{2.623557in}{0.739656in}}%
\pgfpathlineto{\pgfqpoint{2.622705in}{0.739656in}}%
\pgfpathlineto{\pgfqpoint{2.621852in}{0.739656in}}%
\pgfpathlineto{\pgfqpoint{2.620999in}{0.739656in}}%
\pgfpathlineto{\pgfqpoint{2.620146in}{0.739656in}}%
\pgfpathlineto{\pgfqpoint{2.619293in}{0.739656in}}%
\pgfpathlineto{\pgfqpoint{2.618440in}{0.739656in}}%
\pgfpathlineto{\pgfqpoint{2.617587in}{0.739656in}}%
\pgfpathlineto{\pgfqpoint{2.616734in}{0.739656in}}%
\pgfpathlineto{\pgfqpoint{2.615881in}{0.739656in}}%
\pgfpathlineto{\pgfqpoint{2.615029in}{0.739656in}}%
\pgfpathlineto{\pgfqpoint{2.614176in}{0.739656in}}%
\pgfpathlineto{\pgfqpoint{2.613323in}{0.739656in}}%
\pgfpathlineto{\pgfqpoint{2.612470in}{0.739656in}}%
\pgfpathlineto{\pgfqpoint{2.611617in}{0.739656in}}%
\pgfpathlineto{\pgfqpoint{2.610764in}{0.739656in}}%
\pgfpathlineto{\pgfqpoint{2.609911in}{0.739656in}}%
\pgfpathlineto{\pgfqpoint{2.609058in}{0.739656in}}%
\pgfpathlineto{\pgfqpoint{2.608206in}{0.739656in}}%
\pgfpathlineto{\pgfqpoint{2.607353in}{0.739656in}}%
\pgfpathlineto{\pgfqpoint{2.606500in}{0.739656in}}%
\pgfpathlineto{\pgfqpoint{2.605647in}{0.739656in}}%
\pgfpathlineto{\pgfqpoint{2.604794in}{0.739656in}}%
\pgfpathlineto{\pgfqpoint{2.603941in}{0.739656in}}%
\pgfpathlineto{\pgfqpoint{2.603088in}{0.739656in}}%
\pgfpathlineto{\pgfqpoint{2.602235in}{0.739656in}}%
\pgfpathlineto{\pgfqpoint{2.601383in}{0.739656in}}%
\pgfpathlineto{\pgfqpoint{2.600530in}{0.739656in}}%
\pgfpathlineto{\pgfqpoint{2.599677in}{0.739656in}}%
\pgfpathlineto{\pgfqpoint{2.598824in}{0.739656in}}%
\pgfpathlineto{\pgfqpoint{2.597971in}{0.739656in}}%
\pgfpathlineto{\pgfqpoint{2.597118in}{0.739656in}}%
\pgfpathlineto{\pgfqpoint{2.596265in}{0.739656in}}%
\pgfpathlineto{\pgfqpoint{2.595412in}{0.739656in}}%
\pgfpathlineto{\pgfqpoint{2.594560in}{0.739656in}}%
\pgfpathlineto{\pgfqpoint{2.593707in}{0.739656in}}%
\pgfpathlineto{\pgfqpoint{2.592854in}{0.739656in}}%
\pgfpathlineto{\pgfqpoint{2.592001in}{0.739656in}}%
\pgfpathlineto{\pgfqpoint{2.591148in}{0.739656in}}%
\pgfpathlineto{\pgfqpoint{2.590295in}{0.739656in}}%
\pgfpathlineto{\pgfqpoint{2.589442in}{0.739656in}}%
\pgfpathlineto{\pgfqpoint{2.588589in}{0.739656in}}%
\pgfpathlineto{\pgfqpoint{2.587736in}{0.739656in}}%
\pgfpathlineto{\pgfqpoint{2.586884in}{0.739656in}}%
\pgfpathlineto{\pgfqpoint{2.586031in}{0.739656in}}%
\pgfpathlineto{\pgfqpoint{2.585178in}{0.739656in}}%
\pgfpathlineto{\pgfqpoint{2.584325in}{0.739656in}}%
\pgfpathlineto{\pgfqpoint{2.583472in}{0.739656in}}%
\pgfpathlineto{\pgfqpoint{2.582619in}{0.739656in}}%
\pgfpathlineto{\pgfqpoint{2.581766in}{0.739656in}}%
\pgfpathlineto{\pgfqpoint{2.580913in}{0.739656in}}%
\pgfpathlineto{\pgfqpoint{2.580061in}{0.739656in}}%
\pgfpathlineto{\pgfqpoint{2.579208in}{0.739656in}}%
\pgfpathlineto{\pgfqpoint{2.578355in}{0.739656in}}%
\pgfpathlineto{\pgfqpoint{2.577502in}{0.739656in}}%
\pgfpathlineto{\pgfqpoint{2.576649in}{0.739656in}}%
\pgfpathlineto{\pgfqpoint{2.575796in}{0.739656in}}%
\pgfpathlineto{\pgfqpoint{2.574943in}{0.739656in}}%
\pgfpathlineto{\pgfqpoint{2.574090in}{0.739656in}}%
\pgfpathlineto{\pgfqpoint{2.573238in}{0.739656in}}%
\pgfpathlineto{\pgfqpoint{2.572385in}{0.739656in}}%
\pgfpathlineto{\pgfqpoint{2.571532in}{0.739656in}}%
\pgfpathlineto{\pgfqpoint{2.570679in}{0.739656in}}%
\pgfpathlineto{\pgfqpoint{2.569826in}{0.739656in}}%
\pgfpathlineto{\pgfqpoint{2.568973in}{0.739656in}}%
\pgfpathlineto{\pgfqpoint{2.568120in}{0.739656in}}%
\pgfpathlineto{\pgfqpoint{2.567267in}{0.739656in}}%
\pgfpathlineto{\pgfqpoint{2.566415in}{0.739656in}}%
\pgfpathlineto{\pgfqpoint{2.565562in}{0.739656in}}%
\pgfpathlineto{\pgfqpoint{2.564709in}{0.739656in}}%
\pgfpathlineto{\pgfqpoint{2.563856in}{0.739656in}}%
\pgfpathlineto{\pgfqpoint{2.563003in}{0.739656in}}%
\pgfpathlineto{\pgfqpoint{2.562150in}{0.739656in}}%
\pgfpathlineto{\pgfqpoint{2.561297in}{0.739656in}}%
\pgfpathlineto{\pgfqpoint{2.560444in}{0.739656in}}%
\pgfpathlineto{\pgfqpoint{2.559591in}{0.739656in}}%
\pgfpathlineto{\pgfqpoint{2.558739in}{0.739656in}}%
\pgfpathlineto{\pgfqpoint{2.557886in}{0.739656in}}%
\pgfpathlineto{\pgfqpoint{2.557033in}{0.739656in}}%
\pgfpathlineto{\pgfqpoint{2.556180in}{0.739656in}}%
\pgfpathlineto{\pgfqpoint{2.555327in}{0.739656in}}%
\pgfpathlineto{\pgfqpoint{2.554474in}{0.739656in}}%
\pgfpathlineto{\pgfqpoint{2.553621in}{0.739656in}}%
\pgfpathlineto{\pgfqpoint{2.552768in}{0.739656in}}%
\pgfpathlineto{\pgfqpoint{2.551916in}{0.739656in}}%
\pgfpathlineto{\pgfqpoint{2.551063in}{0.739656in}}%
\pgfpathlineto{\pgfqpoint{2.550210in}{0.739656in}}%
\pgfpathlineto{\pgfqpoint{2.549357in}{0.739656in}}%
\pgfpathlineto{\pgfqpoint{2.548504in}{0.739656in}}%
\pgfpathlineto{\pgfqpoint{2.547651in}{0.739656in}}%
\pgfpathlineto{\pgfqpoint{2.546798in}{0.739656in}}%
\pgfpathlineto{\pgfqpoint{2.545945in}{0.739656in}}%
\pgfpathlineto{\pgfqpoint{2.545093in}{0.739656in}}%
\pgfpathlineto{\pgfqpoint{2.544240in}{0.739656in}}%
\pgfpathlineto{\pgfqpoint{2.543387in}{0.739656in}}%
\pgfpathlineto{\pgfqpoint{2.542534in}{0.739656in}}%
\pgfpathlineto{\pgfqpoint{2.541681in}{0.739656in}}%
\pgfpathlineto{\pgfqpoint{2.540828in}{0.739656in}}%
\pgfpathlineto{\pgfqpoint{2.539975in}{0.739656in}}%
\pgfpathlineto{\pgfqpoint{2.539122in}{0.739656in}}%
\pgfpathlineto{\pgfqpoint{2.538270in}{0.739656in}}%
\pgfpathlineto{\pgfqpoint{2.537417in}{0.739656in}}%
\pgfpathlineto{\pgfqpoint{2.536564in}{0.739656in}}%
\pgfpathlineto{\pgfqpoint{2.535711in}{0.739656in}}%
\pgfpathlineto{\pgfqpoint{2.534858in}{0.739656in}}%
\pgfpathlineto{\pgfqpoint{2.534005in}{0.739656in}}%
\pgfpathlineto{\pgfqpoint{2.533152in}{0.739656in}}%
\pgfpathlineto{\pgfqpoint{2.532299in}{0.739656in}}%
\pgfpathlineto{\pgfqpoint{2.531446in}{0.739656in}}%
\pgfpathlineto{\pgfqpoint{2.530594in}{0.739656in}}%
\pgfpathlineto{\pgfqpoint{2.529741in}{0.739656in}}%
\pgfpathlineto{\pgfqpoint{2.528888in}{0.739656in}}%
\pgfpathlineto{\pgfqpoint{2.528035in}{0.739656in}}%
\pgfpathlineto{\pgfqpoint{2.527182in}{0.739656in}}%
\pgfpathlineto{\pgfqpoint{2.526329in}{0.739656in}}%
\pgfpathlineto{\pgfqpoint{2.525476in}{0.739656in}}%
\pgfpathlineto{\pgfqpoint{2.524623in}{0.739656in}}%
\pgfpathlineto{\pgfqpoint{2.523771in}{0.739656in}}%
\pgfpathlineto{\pgfqpoint{2.522918in}{0.739656in}}%
\pgfpathlineto{\pgfqpoint{2.522065in}{0.739656in}}%
\pgfpathlineto{\pgfqpoint{2.521212in}{0.739656in}}%
\pgfpathlineto{\pgfqpoint{2.520359in}{0.739656in}}%
\pgfpathlineto{\pgfqpoint{2.519506in}{0.739656in}}%
\pgfpathlineto{\pgfqpoint{2.518653in}{0.739656in}}%
\pgfpathlineto{\pgfqpoint{2.517800in}{0.739656in}}%
\pgfpathlineto{\pgfqpoint{2.516948in}{0.739656in}}%
\pgfpathlineto{\pgfqpoint{2.516095in}{0.739656in}}%
\pgfpathlineto{\pgfqpoint{2.515242in}{0.739656in}}%
\pgfpathlineto{\pgfqpoint{2.514389in}{0.739656in}}%
\pgfpathlineto{\pgfqpoint{2.513536in}{0.739656in}}%
\pgfpathlineto{\pgfqpoint{2.512683in}{0.739656in}}%
\pgfpathlineto{\pgfqpoint{2.511830in}{0.739656in}}%
\pgfpathlineto{\pgfqpoint{2.510977in}{0.739656in}}%
\pgfpathlineto{\pgfqpoint{2.510125in}{0.739656in}}%
\pgfpathlineto{\pgfqpoint{2.509272in}{0.739656in}}%
\pgfpathlineto{\pgfqpoint{2.508419in}{0.739656in}}%
\pgfpathlineto{\pgfqpoint{2.507566in}{0.739656in}}%
\pgfpathlineto{\pgfqpoint{2.506713in}{0.739656in}}%
\pgfpathlineto{\pgfqpoint{2.505860in}{0.739656in}}%
\pgfpathlineto{\pgfqpoint{2.505007in}{0.739656in}}%
\pgfpathlineto{\pgfqpoint{2.504154in}{0.739656in}}%
\pgfpathlineto{\pgfqpoint{2.503302in}{0.739656in}}%
\pgfpathlineto{\pgfqpoint{2.502449in}{0.739656in}}%
\pgfpathlineto{\pgfqpoint{2.501596in}{0.739656in}}%
\pgfpathlineto{\pgfqpoint{2.500743in}{0.739656in}}%
\pgfpathlineto{\pgfqpoint{2.499890in}{0.739656in}}%
\pgfpathlineto{\pgfqpoint{2.499037in}{0.739656in}}%
\pgfpathlineto{\pgfqpoint{2.498184in}{0.739656in}}%
\pgfpathlineto{\pgfqpoint{2.497331in}{0.739656in}}%
\pgfpathlineto{\pgfqpoint{2.496478in}{0.739656in}}%
\pgfpathlineto{\pgfqpoint{2.495626in}{0.739656in}}%
\pgfpathlineto{\pgfqpoint{2.494773in}{0.739656in}}%
\pgfpathlineto{\pgfqpoint{2.493920in}{0.739656in}}%
\pgfpathlineto{\pgfqpoint{2.493067in}{0.739656in}}%
\pgfpathlineto{\pgfqpoint{2.492214in}{0.739656in}}%
\pgfpathlineto{\pgfqpoint{2.491361in}{0.739656in}}%
\pgfpathlineto{\pgfqpoint{2.490508in}{0.739656in}}%
\pgfpathlineto{\pgfqpoint{2.489655in}{0.739656in}}%
\pgfpathlineto{\pgfqpoint{2.488803in}{0.739656in}}%
\pgfpathlineto{\pgfqpoint{2.487950in}{0.739656in}}%
\pgfpathlineto{\pgfqpoint{2.487097in}{0.739656in}}%
\pgfpathlineto{\pgfqpoint{2.486244in}{0.739656in}}%
\pgfpathlineto{\pgfqpoint{2.485391in}{0.739656in}}%
\pgfpathlineto{\pgfqpoint{2.484538in}{0.739656in}}%
\pgfpathlineto{\pgfqpoint{2.483685in}{0.739656in}}%
\pgfpathlineto{\pgfqpoint{2.482832in}{0.739656in}}%
\pgfpathlineto{\pgfqpoint{2.481980in}{0.739656in}}%
\pgfpathlineto{\pgfqpoint{2.481127in}{0.739656in}}%
\pgfpathlineto{\pgfqpoint{2.480274in}{0.739656in}}%
\pgfpathlineto{\pgfqpoint{2.479421in}{0.739656in}}%
\pgfpathlineto{\pgfqpoint{2.478568in}{0.739656in}}%
\pgfpathlineto{\pgfqpoint{2.477715in}{0.739656in}}%
\pgfpathlineto{\pgfqpoint{2.476862in}{0.739656in}}%
\pgfpathlineto{\pgfqpoint{2.476009in}{0.739656in}}%
\pgfpathlineto{\pgfqpoint{2.475157in}{0.739656in}}%
\pgfpathlineto{\pgfqpoint{2.474304in}{0.739656in}}%
\pgfpathlineto{\pgfqpoint{2.473451in}{0.739656in}}%
\pgfpathlineto{\pgfqpoint{2.472598in}{0.739656in}}%
\pgfpathlineto{\pgfqpoint{2.471745in}{0.739656in}}%
\pgfpathlineto{\pgfqpoint{2.470892in}{0.739656in}}%
\pgfpathlineto{\pgfqpoint{2.470039in}{0.739656in}}%
\pgfpathlineto{\pgfqpoint{2.469186in}{0.739656in}}%
\pgfpathlineto{\pgfqpoint{2.468333in}{0.739656in}}%
\pgfpathlineto{\pgfqpoint{2.467481in}{0.739656in}}%
\pgfpathlineto{\pgfqpoint{2.466628in}{0.739656in}}%
\pgfpathlineto{\pgfqpoint{2.465775in}{0.739656in}}%
\pgfpathlineto{\pgfqpoint{2.464922in}{0.739656in}}%
\pgfpathlineto{\pgfqpoint{2.464069in}{0.739656in}}%
\pgfpathlineto{\pgfqpoint{2.463216in}{0.739656in}}%
\pgfpathlineto{\pgfqpoint{2.462363in}{0.739656in}}%
\pgfpathlineto{\pgfqpoint{2.461510in}{0.739656in}}%
\pgfpathlineto{\pgfqpoint{2.460658in}{0.739656in}}%
\pgfpathlineto{\pgfqpoint{2.459805in}{0.739656in}}%
\pgfpathlineto{\pgfqpoint{2.458952in}{0.739656in}}%
\pgfpathlineto{\pgfqpoint{2.458099in}{0.739656in}}%
\pgfpathlineto{\pgfqpoint{2.457246in}{0.739656in}}%
\pgfpathlineto{\pgfqpoint{2.456393in}{0.739656in}}%
\pgfpathlineto{\pgfqpoint{2.455540in}{0.739656in}}%
\pgfpathlineto{\pgfqpoint{2.454687in}{0.739656in}}%
\pgfpathlineto{\pgfqpoint{2.453835in}{0.739656in}}%
\pgfpathlineto{\pgfqpoint{2.452982in}{0.739656in}}%
\pgfpathlineto{\pgfqpoint{2.452129in}{0.739656in}}%
\pgfpathlineto{\pgfqpoint{2.451276in}{0.739656in}}%
\pgfpathlineto{\pgfqpoint{2.450423in}{0.739656in}}%
\pgfpathlineto{\pgfqpoint{2.449570in}{0.739656in}}%
\pgfpathlineto{\pgfqpoint{2.448717in}{0.739656in}}%
\pgfpathlineto{\pgfqpoint{2.447864in}{0.739656in}}%
\pgfpathlineto{\pgfqpoint{2.447012in}{0.739656in}}%
\pgfpathlineto{\pgfqpoint{2.446159in}{0.739656in}}%
\pgfpathlineto{\pgfqpoint{2.445306in}{0.739656in}}%
\pgfpathlineto{\pgfqpoint{2.444453in}{0.739656in}}%
\pgfpathlineto{\pgfqpoint{2.443600in}{0.739656in}}%
\pgfpathlineto{\pgfqpoint{2.442747in}{0.739656in}}%
\pgfpathlineto{\pgfqpoint{2.441894in}{0.739656in}}%
\pgfpathlineto{\pgfqpoint{2.441041in}{0.739656in}}%
\pgfpathlineto{\pgfqpoint{2.440188in}{0.739656in}}%
\pgfpathlineto{\pgfqpoint{2.439336in}{0.739656in}}%
\pgfpathlineto{\pgfqpoint{2.438483in}{0.739656in}}%
\pgfpathlineto{\pgfqpoint{2.437630in}{0.739656in}}%
\pgfpathlineto{\pgfqpoint{2.436777in}{0.739656in}}%
\pgfpathlineto{\pgfqpoint{2.435924in}{0.739656in}}%
\pgfpathlineto{\pgfqpoint{2.435071in}{0.739656in}}%
\pgfpathlineto{\pgfqpoint{2.434218in}{0.739656in}}%
\pgfpathlineto{\pgfqpoint{2.433365in}{0.739656in}}%
\pgfpathlineto{\pgfqpoint{2.432513in}{0.739656in}}%
\pgfpathlineto{\pgfqpoint{2.431660in}{0.739656in}}%
\pgfpathlineto{\pgfqpoint{2.430807in}{0.739656in}}%
\pgfpathlineto{\pgfqpoint{2.429954in}{0.739656in}}%
\pgfpathlineto{\pgfqpoint{2.429101in}{0.739656in}}%
\pgfpathlineto{\pgfqpoint{2.428248in}{0.739656in}}%
\pgfpathlineto{\pgfqpoint{2.427395in}{0.739656in}}%
\pgfpathlineto{\pgfqpoint{2.426542in}{0.739656in}}%
\pgfpathlineto{\pgfqpoint{2.425690in}{0.739656in}}%
\pgfpathlineto{\pgfqpoint{2.424837in}{0.739656in}}%
\pgfpathlineto{\pgfqpoint{2.423984in}{0.739656in}}%
\pgfpathlineto{\pgfqpoint{2.423131in}{0.739656in}}%
\pgfpathlineto{\pgfqpoint{2.422278in}{0.739656in}}%
\pgfpathlineto{\pgfqpoint{2.421425in}{0.739656in}}%
\pgfpathlineto{\pgfqpoint{2.420572in}{0.739656in}}%
\pgfpathlineto{\pgfqpoint{2.419719in}{0.739656in}}%
\pgfpathlineto{\pgfqpoint{2.418867in}{0.739656in}}%
\pgfpathlineto{\pgfqpoint{2.418014in}{0.739656in}}%
\pgfpathlineto{\pgfqpoint{2.417161in}{0.739656in}}%
\pgfpathlineto{\pgfqpoint{2.416308in}{0.739656in}}%
\pgfpathlineto{\pgfqpoint{2.415455in}{0.739656in}}%
\pgfpathlineto{\pgfqpoint{2.414602in}{0.739656in}}%
\pgfpathlineto{\pgfqpoint{2.413749in}{0.739656in}}%
\pgfpathlineto{\pgfqpoint{2.412896in}{0.739656in}}%
\pgfpathlineto{\pgfqpoint{2.412044in}{0.739656in}}%
\pgfpathlineto{\pgfqpoint{2.411191in}{0.739656in}}%
\pgfpathlineto{\pgfqpoint{2.410338in}{0.739656in}}%
\pgfpathlineto{\pgfqpoint{2.409485in}{0.739656in}}%
\pgfpathlineto{\pgfqpoint{2.408632in}{0.739656in}}%
\pgfpathlineto{\pgfqpoint{2.407779in}{0.739656in}}%
\pgfpathlineto{\pgfqpoint{2.406926in}{0.739656in}}%
\pgfpathlineto{\pgfqpoint{2.406073in}{0.739656in}}%
\pgfpathlineto{\pgfqpoint{2.405220in}{0.739656in}}%
\pgfpathlineto{\pgfqpoint{2.404368in}{0.739656in}}%
\pgfpathlineto{\pgfqpoint{2.403515in}{0.739656in}}%
\pgfpathlineto{\pgfqpoint{2.402662in}{0.739656in}}%
\pgfpathlineto{\pgfqpoint{2.401809in}{0.739656in}}%
\pgfpathlineto{\pgfqpoint{2.400956in}{0.739656in}}%
\pgfpathlineto{\pgfqpoint{2.400103in}{0.739656in}}%
\pgfpathlineto{\pgfqpoint{2.399250in}{0.739656in}}%
\pgfpathlineto{\pgfqpoint{2.398397in}{0.739656in}}%
\pgfpathlineto{\pgfqpoint{2.397545in}{0.739656in}}%
\pgfpathlineto{\pgfqpoint{2.396692in}{0.739656in}}%
\pgfpathlineto{\pgfqpoint{2.395839in}{0.739656in}}%
\pgfpathlineto{\pgfqpoint{2.394986in}{0.739656in}}%
\pgfpathlineto{\pgfqpoint{2.394133in}{0.739656in}}%
\pgfpathlineto{\pgfqpoint{2.393280in}{0.739656in}}%
\pgfpathlineto{\pgfqpoint{2.392427in}{0.739656in}}%
\pgfpathlineto{\pgfqpoint{2.391574in}{0.739656in}}%
\pgfpathlineto{\pgfqpoint{2.390722in}{0.739656in}}%
\pgfpathlineto{\pgfqpoint{2.389869in}{0.739656in}}%
\pgfpathlineto{\pgfqpoint{2.389016in}{0.739656in}}%
\pgfpathlineto{\pgfqpoint{2.388163in}{0.739656in}}%
\pgfpathlineto{\pgfqpoint{2.387310in}{0.739656in}}%
\pgfpathlineto{\pgfqpoint{2.386457in}{0.739656in}}%
\pgfpathlineto{\pgfqpoint{2.385604in}{0.739656in}}%
\pgfpathlineto{\pgfqpoint{2.384751in}{0.739656in}}%
\pgfpathlineto{\pgfqpoint{2.383899in}{0.739656in}}%
\pgfpathlineto{\pgfqpoint{2.383046in}{0.739656in}}%
\pgfpathlineto{\pgfqpoint{2.382193in}{0.739656in}}%
\pgfpathlineto{\pgfqpoint{2.381340in}{0.739656in}}%
\pgfpathlineto{\pgfqpoint{2.380487in}{0.739656in}}%
\pgfpathlineto{\pgfqpoint{2.379634in}{0.739656in}}%
\pgfpathlineto{\pgfqpoint{2.378781in}{0.739656in}}%
\pgfpathlineto{\pgfqpoint{2.377928in}{0.739656in}}%
\pgfpathlineto{\pgfqpoint{2.377075in}{0.739656in}}%
\pgfpathlineto{\pgfqpoint{2.376223in}{0.739656in}}%
\pgfpathlineto{\pgfqpoint{2.375370in}{0.739656in}}%
\pgfpathlineto{\pgfqpoint{2.374517in}{0.739656in}}%
\pgfpathlineto{\pgfqpoint{2.373664in}{0.739656in}}%
\pgfpathlineto{\pgfqpoint{2.372811in}{0.739656in}}%
\pgfpathlineto{\pgfqpoint{2.371958in}{0.739656in}}%
\pgfpathlineto{\pgfqpoint{2.371105in}{0.739656in}}%
\pgfpathlineto{\pgfqpoint{2.370252in}{0.739656in}}%
\pgfpathlineto{\pgfqpoint{2.369400in}{0.739656in}}%
\pgfpathlineto{\pgfqpoint{2.368547in}{0.739656in}}%
\pgfpathlineto{\pgfqpoint{2.367694in}{0.739656in}}%
\pgfpathlineto{\pgfqpoint{2.366841in}{0.739656in}}%
\pgfpathlineto{\pgfqpoint{2.365988in}{0.739656in}}%
\pgfpathlineto{\pgfqpoint{2.365135in}{0.739656in}}%
\pgfpathlineto{\pgfqpoint{2.364282in}{0.739656in}}%
\pgfpathlineto{\pgfqpoint{2.363429in}{0.739656in}}%
\pgfpathlineto{\pgfqpoint{2.362577in}{0.739656in}}%
\pgfpathlineto{\pgfqpoint{2.361724in}{0.739656in}}%
\pgfpathlineto{\pgfqpoint{2.360871in}{0.739656in}}%
\pgfpathlineto{\pgfqpoint{2.360018in}{0.739656in}}%
\pgfpathlineto{\pgfqpoint{2.359165in}{0.739656in}}%
\pgfpathlineto{\pgfqpoint{2.358312in}{0.739656in}}%
\pgfpathlineto{\pgfqpoint{2.357459in}{0.739656in}}%
\pgfpathlineto{\pgfqpoint{2.356606in}{0.739656in}}%
\pgfpathlineto{\pgfqpoint{2.355754in}{0.739656in}}%
\pgfpathlineto{\pgfqpoint{2.354901in}{0.739656in}}%
\pgfpathlineto{\pgfqpoint{2.354048in}{0.739656in}}%
\pgfpathlineto{\pgfqpoint{2.353195in}{0.739656in}}%
\pgfpathlineto{\pgfqpoint{2.352342in}{0.739656in}}%
\pgfpathlineto{\pgfqpoint{2.351489in}{0.739656in}}%
\pgfpathlineto{\pgfqpoint{2.350636in}{0.739656in}}%
\pgfpathlineto{\pgfqpoint{2.349783in}{0.739656in}}%
\pgfpathlineto{\pgfqpoint{2.348930in}{0.739656in}}%
\pgfpathlineto{\pgfqpoint{2.348078in}{0.739656in}}%
\pgfpathlineto{\pgfqpoint{2.347225in}{0.739656in}}%
\pgfpathlineto{\pgfqpoint{2.346372in}{0.739656in}}%
\pgfpathlineto{\pgfqpoint{2.345519in}{0.739656in}}%
\pgfpathlineto{\pgfqpoint{2.344666in}{0.739656in}}%
\pgfpathlineto{\pgfqpoint{2.343813in}{0.739656in}}%
\pgfpathlineto{\pgfqpoint{2.342960in}{0.739656in}}%
\pgfpathlineto{\pgfqpoint{2.342107in}{0.739656in}}%
\pgfpathlineto{\pgfqpoint{2.341255in}{0.739656in}}%
\pgfpathlineto{\pgfqpoint{2.340402in}{0.739656in}}%
\pgfpathlineto{\pgfqpoint{2.339549in}{0.739656in}}%
\pgfpathlineto{\pgfqpoint{2.338696in}{0.739656in}}%
\pgfpathlineto{\pgfqpoint{2.337843in}{0.739656in}}%
\pgfpathlineto{\pgfqpoint{2.336990in}{0.739656in}}%
\pgfpathlineto{\pgfqpoint{2.336137in}{0.739656in}}%
\pgfpathlineto{\pgfqpoint{2.335284in}{0.739656in}}%
\pgfpathlineto{\pgfqpoint{2.334432in}{0.739656in}}%
\pgfpathlineto{\pgfqpoint{2.333579in}{0.739656in}}%
\pgfpathlineto{\pgfqpoint{2.332726in}{0.739656in}}%
\pgfpathlineto{\pgfqpoint{2.331873in}{0.739656in}}%
\pgfpathlineto{\pgfqpoint{2.331020in}{0.739656in}}%
\pgfpathlineto{\pgfqpoint{2.330167in}{0.739656in}}%
\pgfpathlineto{\pgfqpoint{2.329314in}{0.739656in}}%
\pgfpathlineto{\pgfqpoint{2.328461in}{0.739656in}}%
\pgfpathlineto{\pgfqpoint{2.327609in}{0.739656in}}%
\pgfpathlineto{\pgfqpoint{2.326756in}{0.739656in}}%
\pgfpathlineto{\pgfqpoint{2.325903in}{0.739656in}}%
\pgfpathlineto{\pgfqpoint{2.325050in}{0.739656in}}%
\pgfpathlineto{\pgfqpoint{2.324197in}{0.739656in}}%
\pgfpathlineto{\pgfqpoint{2.323344in}{0.739656in}}%
\pgfpathlineto{\pgfqpoint{2.322491in}{0.739656in}}%
\pgfpathlineto{\pgfqpoint{2.321638in}{0.739656in}}%
\pgfpathlineto{\pgfqpoint{2.320785in}{0.739656in}}%
\pgfpathlineto{\pgfqpoint{2.319933in}{0.739656in}}%
\pgfpathlineto{\pgfqpoint{2.319080in}{0.739656in}}%
\pgfpathlineto{\pgfqpoint{2.318227in}{0.739656in}}%
\pgfpathlineto{\pgfqpoint{2.317374in}{0.739656in}}%
\pgfpathlineto{\pgfqpoint{2.316521in}{0.739656in}}%
\pgfpathlineto{\pgfqpoint{2.315668in}{0.739656in}}%
\pgfpathlineto{\pgfqpoint{2.314815in}{0.739656in}}%
\pgfpathlineto{\pgfqpoint{2.313962in}{0.739656in}}%
\pgfpathlineto{\pgfqpoint{2.313110in}{0.739656in}}%
\pgfpathlineto{\pgfqpoint{2.312257in}{0.739656in}}%
\pgfpathlineto{\pgfqpoint{2.311404in}{0.739656in}}%
\pgfpathlineto{\pgfqpoint{2.310551in}{0.739656in}}%
\pgfpathlineto{\pgfqpoint{2.309698in}{0.739656in}}%
\pgfpathlineto{\pgfqpoint{2.308845in}{0.739656in}}%
\pgfpathlineto{\pgfqpoint{2.307992in}{0.739656in}}%
\pgfpathlineto{\pgfqpoint{2.307139in}{0.739656in}}%
\pgfpathlineto{\pgfqpoint{2.306287in}{0.739656in}}%
\pgfpathlineto{\pgfqpoint{2.305434in}{0.739656in}}%
\pgfpathlineto{\pgfqpoint{2.304581in}{0.739656in}}%
\pgfpathlineto{\pgfqpoint{2.303728in}{0.739656in}}%
\pgfpathlineto{\pgfqpoint{2.302875in}{0.739656in}}%
\pgfpathlineto{\pgfqpoint{2.302022in}{0.739656in}}%
\pgfpathlineto{\pgfqpoint{2.301169in}{0.739656in}}%
\pgfpathlineto{\pgfqpoint{2.300316in}{0.739656in}}%
\pgfpathlineto{\pgfqpoint{2.299464in}{0.739656in}}%
\pgfpathlineto{\pgfqpoint{2.298611in}{0.739656in}}%
\pgfpathlineto{\pgfqpoint{2.297758in}{0.739656in}}%
\pgfpathlineto{\pgfqpoint{2.296905in}{0.739656in}}%
\pgfpathlineto{\pgfqpoint{2.296052in}{0.739656in}}%
\pgfpathlineto{\pgfqpoint{2.295199in}{0.739656in}}%
\pgfpathlineto{\pgfqpoint{2.294346in}{0.739656in}}%
\pgfpathlineto{\pgfqpoint{2.293493in}{0.739656in}}%
\pgfpathlineto{\pgfqpoint{2.292641in}{0.739656in}}%
\pgfpathlineto{\pgfqpoint{2.291788in}{0.739656in}}%
\pgfpathlineto{\pgfqpoint{2.290935in}{0.739656in}}%
\pgfpathlineto{\pgfqpoint{2.290082in}{0.739656in}}%
\pgfpathlineto{\pgfqpoint{2.289229in}{0.739656in}}%
\pgfpathlineto{\pgfqpoint{2.288376in}{0.739656in}}%
\pgfpathlineto{\pgfqpoint{2.287523in}{0.739656in}}%
\pgfpathlineto{\pgfqpoint{2.286670in}{0.739656in}}%
\pgfpathlineto{\pgfqpoint{2.285817in}{0.739656in}}%
\pgfpathlineto{\pgfqpoint{2.284965in}{0.739656in}}%
\pgfpathlineto{\pgfqpoint{2.284112in}{0.739656in}}%
\pgfpathlineto{\pgfqpoint{2.283259in}{0.739656in}}%
\pgfpathlineto{\pgfqpoint{2.282406in}{0.739656in}}%
\pgfpathlineto{\pgfqpoint{2.281553in}{0.739656in}}%
\pgfpathlineto{\pgfqpoint{2.280700in}{0.739656in}}%
\pgfpathlineto{\pgfqpoint{2.279847in}{0.739656in}}%
\pgfpathlineto{\pgfqpoint{2.278994in}{0.739656in}}%
\pgfpathlineto{\pgfqpoint{2.278142in}{0.739656in}}%
\pgfpathlineto{\pgfqpoint{2.277289in}{0.739656in}}%
\pgfpathlineto{\pgfqpoint{2.276436in}{0.739656in}}%
\pgfpathlineto{\pgfqpoint{2.275583in}{0.739656in}}%
\pgfpathlineto{\pgfqpoint{2.274730in}{0.739656in}}%
\pgfpathlineto{\pgfqpoint{2.273877in}{0.739656in}}%
\pgfpathlineto{\pgfqpoint{2.273024in}{0.739656in}}%
\pgfpathlineto{\pgfqpoint{2.272171in}{0.739656in}}%
\pgfpathlineto{\pgfqpoint{2.271319in}{0.739656in}}%
\pgfpathlineto{\pgfqpoint{2.270466in}{0.739656in}}%
\pgfpathlineto{\pgfqpoint{2.269613in}{0.739656in}}%
\pgfpathlineto{\pgfqpoint{2.268760in}{0.739656in}}%
\pgfpathlineto{\pgfqpoint{2.267907in}{0.739656in}}%
\pgfpathlineto{\pgfqpoint{2.267054in}{0.739656in}}%
\pgfpathlineto{\pgfqpoint{2.266201in}{0.739656in}}%
\pgfpathlineto{\pgfqpoint{2.265348in}{0.739656in}}%
\pgfpathlineto{\pgfqpoint{2.264496in}{0.739656in}}%
\pgfpathlineto{\pgfqpoint{2.263643in}{0.739656in}}%
\pgfpathlineto{\pgfqpoint{2.262790in}{0.739656in}}%
\pgfpathlineto{\pgfqpoint{2.261937in}{0.739656in}}%
\pgfpathlineto{\pgfqpoint{2.261084in}{0.739656in}}%
\pgfpathlineto{\pgfqpoint{2.260231in}{0.739656in}}%
\pgfpathlineto{\pgfqpoint{2.259378in}{0.739656in}}%
\pgfpathlineto{\pgfqpoint{2.258525in}{0.739656in}}%
\pgfpathlineto{\pgfqpoint{2.257672in}{0.739656in}}%
\pgfpathlineto{\pgfqpoint{2.256820in}{0.739656in}}%
\pgfpathlineto{\pgfqpoint{2.255967in}{0.739656in}}%
\pgfpathlineto{\pgfqpoint{2.255114in}{0.739656in}}%
\pgfpathlineto{\pgfqpoint{2.254261in}{0.739656in}}%
\pgfpathlineto{\pgfqpoint{2.253408in}{0.739656in}}%
\pgfpathlineto{\pgfqpoint{2.252555in}{0.739656in}}%
\pgfpathlineto{\pgfqpoint{2.251702in}{0.739656in}}%
\pgfpathlineto{\pgfqpoint{2.250849in}{0.739656in}}%
\pgfpathlineto{\pgfqpoint{2.249997in}{0.739656in}}%
\pgfpathlineto{\pgfqpoint{2.249144in}{0.739656in}}%
\pgfpathlineto{\pgfqpoint{2.248291in}{0.739656in}}%
\pgfpathlineto{\pgfqpoint{2.247438in}{0.739656in}}%
\pgfpathlineto{\pgfqpoint{2.246585in}{0.739656in}}%
\pgfpathlineto{\pgfqpoint{2.245732in}{0.739656in}}%
\pgfpathlineto{\pgfqpoint{2.244879in}{0.739656in}}%
\pgfpathlineto{\pgfqpoint{2.244026in}{0.739656in}}%
\pgfpathlineto{\pgfqpoint{2.243174in}{0.739656in}}%
\pgfpathlineto{\pgfqpoint{2.242321in}{0.739656in}}%
\pgfpathlineto{\pgfqpoint{2.241468in}{0.739656in}}%
\pgfpathlineto{\pgfqpoint{2.240615in}{0.739656in}}%
\pgfpathlineto{\pgfqpoint{2.239762in}{0.739656in}}%
\pgfpathlineto{\pgfqpoint{2.238909in}{0.739656in}}%
\pgfpathlineto{\pgfqpoint{2.238056in}{0.739656in}}%
\pgfpathlineto{\pgfqpoint{2.237203in}{0.739656in}}%
\pgfpathlineto{\pgfqpoint{2.236351in}{0.739656in}}%
\pgfpathlineto{\pgfqpoint{2.235498in}{0.739656in}}%
\pgfpathlineto{\pgfqpoint{2.234645in}{0.739656in}}%
\pgfpathlineto{\pgfqpoint{2.233792in}{0.739656in}}%
\pgfpathlineto{\pgfqpoint{2.232939in}{0.739656in}}%
\pgfpathlineto{\pgfqpoint{2.232086in}{0.739656in}}%
\pgfpathlineto{\pgfqpoint{2.231233in}{0.739656in}}%
\pgfpathlineto{\pgfqpoint{2.230380in}{0.739656in}}%
\pgfpathlineto{\pgfqpoint{2.229527in}{0.739656in}}%
\pgfpathlineto{\pgfqpoint{2.228675in}{0.739656in}}%
\pgfpathlineto{\pgfqpoint{2.227822in}{0.739656in}}%
\pgfpathlineto{\pgfqpoint{2.226969in}{0.739656in}}%
\pgfpathlineto{\pgfqpoint{2.226116in}{0.739656in}}%
\pgfpathlineto{\pgfqpoint{2.225263in}{0.739656in}}%
\pgfpathlineto{\pgfqpoint{2.224410in}{0.739656in}}%
\pgfpathlineto{\pgfqpoint{2.223557in}{0.739656in}}%
\pgfpathlineto{\pgfqpoint{2.222704in}{0.739656in}}%
\pgfpathlineto{\pgfqpoint{2.221852in}{0.739656in}}%
\pgfpathlineto{\pgfqpoint{2.220999in}{0.739656in}}%
\pgfpathlineto{\pgfqpoint{2.220146in}{0.739656in}}%
\pgfpathlineto{\pgfqpoint{2.219293in}{0.739656in}}%
\pgfpathlineto{\pgfqpoint{2.218440in}{0.739656in}}%
\pgfpathlineto{\pgfqpoint{2.217587in}{0.739656in}}%
\pgfpathlineto{\pgfqpoint{2.216734in}{0.739656in}}%
\pgfpathlineto{\pgfqpoint{2.215881in}{0.739656in}}%
\pgfpathlineto{\pgfqpoint{2.215029in}{0.739656in}}%
\pgfpathlineto{\pgfqpoint{2.214176in}{0.739656in}}%
\pgfpathlineto{\pgfqpoint{2.213323in}{0.739656in}}%
\pgfpathlineto{\pgfqpoint{2.212470in}{0.739656in}}%
\pgfpathlineto{\pgfqpoint{2.211617in}{0.739656in}}%
\pgfpathlineto{\pgfqpoint{2.210764in}{0.739656in}}%
\pgfpathlineto{\pgfqpoint{2.209911in}{0.739656in}}%
\pgfpathlineto{\pgfqpoint{2.209058in}{0.739656in}}%
\pgfpathlineto{\pgfqpoint{2.208206in}{0.739656in}}%
\pgfpathlineto{\pgfqpoint{2.207353in}{0.739656in}}%
\pgfpathlineto{\pgfqpoint{2.206500in}{0.739656in}}%
\pgfpathlineto{\pgfqpoint{2.205647in}{0.739656in}}%
\pgfpathlineto{\pgfqpoint{2.204794in}{0.739656in}}%
\pgfpathlineto{\pgfqpoint{2.203941in}{0.739656in}}%
\pgfpathlineto{\pgfqpoint{2.203088in}{0.739656in}}%
\pgfpathlineto{\pgfqpoint{2.202235in}{0.739656in}}%
\pgfpathlineto{\pgfqpoint{2.201382in}{0.739656in}}%
\pgfpathlineto{\pgfqpoint{2.200530in}{0.739656in}}%
\pgfpathlineto{\pgfqpoint{2.199677in}{0.739656in}}%
\pgfpathlineto{\pgfqpoint{2.198824in}{0.739656in}}%
\pgfpathlineto{\pgfqpoint{2.197971in}{0.739656in}}%
\pgfpathlineto{\pgfqpoint{2.197118in}{0.739656in}}%
\pgfpathlineto{\pgfqpoint{2.196265in}{0.739656in}}%
\pgfpathlineto{\pgfqpoint{2.195412in}{0.739656in}}%
\pgfpathlineto{\pgfqpoint{2.194559in}{0.739656in}}%
\pgfpathlineto{\pgfqpoint{2.193707in}{0.739656in}}%
\pgfpathlineto{\pgfqpoint{2.192854in}{0.739656in}}%
\pgfpathlineto{\pgfqpoint{2.192001in}{0.739656in}}%
\pgfpathlineto{\pgfqpoint{2.191148in}{0.739656in}}%
\pgfpathlineto{\pgfqpoint{2.190295in}{0.739656in}}%
\pgfpathlineto{\pgfqpoint{2.189442in}{0.739656in}}%
\pgfpathlineto{\pgfqpoint{2.188589in}{0.739656in}}%
\pgfpathlineto{\pgfqpoint{2.187736in}{0.739656in}}%
\pgfpathlineto{\pgfqpoint{2.186884in}{0.739656in}}%
\pgfpathlineto{\pgfqpoint{2.186031in}{0.739656in}}%
\pgfpathlineto{\pgfqpoint{2.185178in}{0.739656in}}%
\pgfpathlineto{\pgfqpoint{2.184325in}{0.739656in}}%
\pgfpathlineto{\pgfqpoint{2.183472in}{0.739656in}}%
\pgfpathlineto{\pgfqpoint{2.182619in}{0.739656in}}%
\pgfpathlineto{\pgfqpoint{2.181766in}{0.739656in}}%
\pgfpathlineto{\pgfqpoint{2.180913in}{0.739656in}}%
\pgfpathlineto{\pgfqpoint{2.180061in}{0.739656in}}%
\pgfpathlineto{\pgfqpoint{2.179208in}{0.739656in}}%
\pgfpathlineto{\pgfqpoint{2.178355in}{0.739656in}}%
\pgfpathlineto{\pgfqpoint{2.177502in}{0.739656in}}%
\pgfpathlineto{\pgfqpoint{2.176649in}{0.739656in}}%
\pgfpathlineto{\pgfqpoint{2.175796in}{0.739656in}}%
\pgfpathlineto{\pgfqpoint{2.174943in}{0.739656in}}%
\pgfpathlineto{\pgfqpoint{2.174090in}{0.739656in}}%
\pgfpathlineto{\pgfqpoint{2.173238in}{0.739656in}}%
\pgfpathlineto{\pgfqpoint{2.172385in}{0.739656in}}%
\pgfpathlineto{\pgfqpoint{2.171532in}{0.739656in}}%
\pgfpathlineto{\pgfqpoint{2.170679in}{0.739656in}}%
\pgfpathlineto{\pgfqpoint{2.169826in}{0.739656in}}%
\pgfpathlineto{\pgfqpoint{2.168973in}{0.739656in}}%
\pgfpathlineto{\pgfqpoint{2.168120in}{0.739656in}}%
\pgfpathlineto{\pgfqpoint{2.167267in}{0.739656in}}%
\pgfpathlineto{\pgfqpoint{2.166414in}{0.739656in}}%
\pgfpathlineto{\pgfqpoint{2.165562in}{0.739656in}}%
\pgfpathlineto{\pgfqpoint{2.164709in}{0.739656in}}%
\pgfpathlineto{\pgfqpoint{2.163856in}{0.739656in}}%
\pgfpathlineto{\pgfqpoint{2.163003in}{0.739656in}}%
\pgfpathlineto{\pgfqpoint{2.162150in}{0.739656in}}%
\pgfpathlineto{\pgfqpoint{2.161297in}{0.739656in}}%
\pgfpathlineto{\pgfqpoint{2.160444in}{0.739656in}}%
\pgfpathlineto{\pgfqpoint{2.159591in}{0.739656in}}%
\pgfpathlineto{\pgfqpoint{2.158739in}{0.739656in}}%
\pgfpathlineto{\pgfqpoint{2.157886in}{0.739656in}}%
\pgfpathlineto{\pgfqpoint{2.157033in}{0.739656in}}%
\pgfpathlineto{\pgfqpoint{2.156180in}{0.739656in}}%
\pgfpathlineto{\pgfqpoint{2.155327in}{0.739656in}}%
\pgfpathlineto{\pgfqpoint{2.154474in}{0.739656in}}%
\pgfpathlineto{\pgfqpoint{2.153621in}{0.739656in}}%
\pgfpathlineto{\pgfqpoint{2.152768in}{0.739656in}}%
\pgfpathlineto{\pgfqpoint{2.151916in}{0.739656in}}%
\pgfpathlineto{\pgfqpoint{2.151063in}{0.739656in}}%
\pgfpathlineto{\pgfqpoint{2.150210in}{0.739656in}}%
\pgfpathlineto{\pgfqpoint{2.149357in}{0.739656in}}%
\pgfpathlineto{\pgfqpoint{2.148504in}{0.739656in}}%
\pgfpathlineto{\pgfqpoint{2.147651in}{0.739656in}}%
\pgfpathlineto{\pgfqpoint{2.146798in}{0.739656in}}%
\pgfpathlineto{\pgfqpoint{2.145945in}{0.739656in}}%
\pgfpathlineto{\pgfqpoint{2.145093in}{0.739656in}}%
\pgfpathlineto{\pgfqpoint{2.144240in}{0.739656in}}%
\pgfpathlineto{\pgfqpoint{2.143387in}{0.739656in}}%
\pgfpathlineto{\pgfqpoint{2.142534in}{0.739656in}}%
\pgfpathlineto{\pgfqpoint{2.141681in}{0.739656in}}%
\pgfpathlineto{\pgfqpoint{2.140828in}{0.739656in}}%
\pgfpathlineto{\pgfqpoint{2.139975in}{0.739656in}}%
\pgfpathlineto{\pgfqpoint{2.139122in}{0.739656in}}%
\pgfpathlineto{\pgfqpoint{2.138269in}{0.739656in}}%
\pgfpathlineto{\pgfqpoint{2.137417in}{0.739656in}}%
\pgfpathlineto{\pgfqpoint{2.136564in}{0.739656in}}%
\pgfpathlineto{\pgfqpoint{2.135711in}{0.739656in}}%
\pgfpathlineto{\pgfqpoint{2.134858in}{0.739656in}}%
\pgfpathlineto{\pgfqpoint{2.134005in}{0.739656in}}%
\pgfpathlineto{\pgfqpoint{2.133152in}{0.739656in}}%
\pgfpathlineto{\pgfqpoint{2.132299in}{0.739656in}}%
\pgfpathlineto{\pgfqpoint{2.131446in}{0.739656in}}%
\pgfpathlineto{\pgfqpoint{2.130594in}{0.739656in}}%
\pgfpathlineto{\pgfqpoint{2.129741in}{0.739656in}}%
\pgfpathlineto{\pgfqpoint{2.128888in}{0.739656in}}%
\pgfpathlineto{\pgfqpoint{2.128035in}{0.739656in}}%
\pgfpathlineto{\pgfqpoint{2.127182in}{0.739656in}}%
\pgfpathlineto{\pgfqpoint{2.126329in}{0.739656in}}%
\pgfpathlineto{\pgfqpoint{2.125476in}{0.739656in}}%
\pgfpathlineto{\pgfqpoint{2.124623in}{0.739656in}}%
\pgfpathlineto{\pgfqpoint{2.123771in}{0.739656in}}%
\pgfpathlineto{\pgfqpoint{2.122918in}{0.739656in}}%
\pgfpathlineto{\pgfqpoint{2.122065in}{0.739656in}}%
\pgfpathlineto{\pgfqpoint{2.121212in}{0.739656in}}%
\pgfpathlineto{\pgfqpoint{2.120359in}{0.739656in}}%
\pgfpathlineto{\pgfqpoint{2.119506in}{0.739656in}}%
\pgfpathlineto{\pgfqpoint{2.118653in}{0.739656in}}%
\pgfpathlineto{\pgfqpoint{2.117800in}{0.739656in}}%
\pgfpathlineto{\pgfqpoint{2.116948in}{0.739656in}}%
\pgfpathlineto{\pgfqpoint{2.116095in}{0.739656in}}%
\pgfpathlineto{\pgfqpoint{2.115242in}{0.739656in}}%
\pgfpathlineto{\pgfqpoint{2.114389in}{0.739656in}}%
\pgfpathlineto{\pgfqpoint{2.113536in}{0.739656in}}%
\pgfpathlineto{\pgfqpoint{2.112683in}{0.739656in}}%
\pgfpathlineto{\pgfqpoint{2.111830in}{0.739656in}}%
\pgfpathlineto{\pgfqpoint{2.110977in}{0.739656in}}%
\pgfpathlineto{\pgfqpoint{2.110124in}{0.739656in}}%
\pgfpathlineto{\pgfqpoint{2.109272in}{0.739656in}}%
\pgfpathlineto{\pgfqpoint{2.108419in}{0.739656in}}%
\pgfpathlineto{\pgfqpoint{2.107566in}{0.739656in}}%
\pgfpathlineto{\pgfqpoint{2.106713in}{0.739656in}}%
\pgfpathlineto{\pgfqpoint{2.105860in}{0.739656in}}%
\pgfpathlineto{\pgfqpoint{2.105007in}{0.739656in}}%
\pgfpathlineto{\pgfqpoint{2.104154in}{0.739656in}}%
\pgfpathlineto{\pgfqpoint{2.103301in}{0.739656in}}%
\pgfpathlineto{\pgfqpoint{2.102449in}{0.739656in}}%
\pgfpathlineto{\pgfqpoint{2.101596in}{0.739656in}}%
\pgfpathlineto{\pgfqpoint{2.100743in}{0.739656in}}%
\pgfpathlineto{\pgfqpoint{2.099890in}{0.739656in}}%
\pgfpathlineto{\pgfqpoint{2.099037in}{0.739656in}}%
\pgfpathlineto{\pgfqpoint{2.098184in}{0.739656in}}%
\pgfpathlineto{\pgfqpoint{2.097331in}{0.739656in}}%
\pgfpathlineto{\pgfqpoint{2.096478in}{0.739656in}}%
\pgfpathlineto{\pgfqpoint{2.095626in}{0.739656in}}%
\pgfpathlineto{\pgfqpoint{2.094773in}{0.739656in}}%
\pgfpathlineto{\pgfqpoint{2.093920in}{0.739656in}}%
\pgfpathlineto{\pgfqpoint{2.093067in}{0.739656in}}%
\pgfpathlineto{\pgfqpoint{2.092214in}{0.739656in}}%
\pgfpathlineto{\pgfqpoint{2.091361in}{0.739656in}}%
\pgfpathlineto{\pgfqpoint{2.090508in}{0.739656in}}%
\pgfpathlineto{\pgfqpoint{2.089655in}{0.739656in}}%
\pgfpathlineto{\pgfqpoint{2.088803in}{0.739656in}}%
\pgfpathlineto{\pgfqpoint{2.087950in}{0.739656in}}%
\pgfpathlineto{\pgfqpoint{2.087097in}{0.739656in}}%
\pgfpathlineto{\pgfqpoint{2.086244in}{0.739656in}}%
\pgfpathlineto{\pgfqpoint{2.085391in}{0.739656in}}%
\pgfpathlineto{\pgfqpoint{2.084538in}{0.739656in}}%
\pgfpathlineto{\pgfqpoint{2.083685in}{0.739656in}}%
\pgfpathlineto{\pgfqpoint{2.082832in}{0.739656in}}%
\pgfpathlineto{\pgfqpoint{2.081980in}{0.739656in}}%
\pgfpathlineto{\pgfqpoint{2.081127in}{0.739656in}}%
\pgfpathlineto{\pgfqpoint{2.080274in}{0.739656in}}%
\pgfpathlineto{\pgfqpoint{2.079421in}{0.739656in}}%
\pgfpathlineto{\pgfqpoint{2.078568in}{0.739656in}}%
\pgfpathlineto{\pgfqpoint{2.077715in}{0.739656in}}%
\pgfpathlineto{\pgfqpoint{2.076862in}{0.739656in}}%
\pgfpathlineto{\pgfqpoint{2.076009in}{0.739656in}}%
\pgfpathlineto{\pgfqpoint{2.075156in}{0.739656in}}%
\pgfpathlineto{\pgfqpoint{2.074304in}{0.739656in}}%
\pgfpathlineto{\pgfqpoint{2.073451in}{0.739656in}}%
\pgfpathlineto{\pgfqpoint{2.072598in}{0.739656in}}%
\pgfpathlineto{\pgfqpoint{2.071745in}{0.739656in}}%
\pgfpathlineto{\pgfqpoint{2.070892in}{0.739656in}}%
\pgfpathlineto{\pgfqpoint{2.070039in}{0.739656in}}%
\pgfpathlineto{\pgfqpoint{2.069186in}{0.739656in}}%
\pgfpathlineto{\pgfqpoint{2.068333in}{0.739656in}}%
\pgfpathlineto{\pgfqpoint{2.067481in}{0.739656in}}%
\pgfpathlineto{\pgfqpoint{2.066628in}{0.739656in}}%
\pgfpathlineto{\pgfqpoint{2.065775in}{0.739656in}}%
\pgfpathlineto{\pgfqpoint{2.064922in}{0.739656in}}%
\pgfpathlineto{\pgfqpoint{2.064069in}{0.739656in}}%
\pgfpathlineto{\pgfqpoint{2.063216in}{0.739656in}}%
\pgfpathlineto{\pgfqpoint{2.062363in}{0.739656in}}%
\pgfpathlineto{\pgfqpoint{2.061510in}{0.739656in}}%
\pgfpathlineto{\pgfqpoint{2.060658in}{0.739656in}}%
\pgfpathlineto{\pgfqpoint{2.059805in}{0.739656in}}%
\pgfpathlineto{\pgfqpoint{2.058952in}{0.739656in}}%
\pgfpathlineto{\pgfqpoint{2.058099in}{0.739656in}}%
\pgfpathlineto{\pgfqpoint{2.057246in}{0.739656in}}%
\pgfpathlineto{\pgfqpoint{2.056393in}{0.739656in}}%
\pgfpathlineto{\pgfqpoint{2.055540in}{0.739656in}}%
\pgfpathlineto{\pgfqpoint{2.054687in}{0.739656in}}%
\pgfpathlineto{\pgfqpoint{2.053835in}{0.739656in}}%
\pgfpathlineto{\pgfqpoint{2.052982in}{0.739656in}}%
\pgfpathlineto{\pgfqpoint{2.052129in}{0.739656in}}%
\pgfpathlineto{\pgfqpoint{2.051276in}{0.739656in}}%
\pgfpathlineto{\pgfqpoint{2.050423in}{0.739656in}}%
\pgfpathlineto{\pgfqpoint{2.049570in}{0.739656in}}%
\pgfpathlineto{\pgfqpoint{2.048717in}{0.739656in}}%
\pgfpathlineto{\pgfqpoint{2.047864in}{0.739656in}}%
\pgfpathlineto{\pgfqpoint{2.047011in}{0.739656in}}%
\pgfpathlineto{\pgfqpoint{2.046159in}{0.739656in}}%
\pgfpathlineto{\pgfqpoint{2.045306in}{0.739656in}}%
\pgfpathlineto{\pgfqpoint{2.044453in}{0.739656in}}%
\pgfpathlineto{\pgfqpoint{2.043600in}{0.739656in}}%
\pgfpathlineto{\pgfqpoint{2.042747in}{0.739656in}}%
\pgfpathlineto{\pgfqpoint{2.041894in}{0.739656in}}%
\pgfpathlineto{\pgfqpoint{2.041041in}{0.739656in}}%
\pgfpathlineto{\pgfqpoint{2.040188in}{0.739656in}}%
\pgfpathlineto{\pgfqpoint{2.039336in}{0.739656in}}%
\pgfpathlineto{\pgfqpoint{2.038483in}{0.739656in}}%
\pgfpathlineto{\pgfqpoint{2.037630in}{0.739656in}}%
\pgfpathlineto{\pgfqpoint{2.036777in}{0.739656in}}%
\pgfpathlineto{\pgfqpoint{2.035924in}{0.739656in}}%
\pgfpathlineto{\pgfqpoint{2.035071in}{0.739656in}}%
\pgfpathlineto{\pgfqpoint{2.034218in}{0.739656in}}%
\pgfpathlineto{\pgfqpoint{2.033365in}{0.739656in}}%
\pgfpathlineto{\pgfqpoint{2.032513in}{0.739656in}}%
\pgfpathlineto{\pgfqpoint{2.031660in}{0.739656in}}%
\pgfpathlineto{\pgfqpoint{2.030807in}{0.739656in}}%
\pgfpathlineto{\pgfqpoint{2.029954in}{0.739656in}}%
\pgfpathlineto{\pgfqpoint{2.029101in}{0.739656in}}%
\pgfpathlineto{\pgfqpoint{2.028248in}{0.739656in}}%
\pgfpathlineto{\pgfqpoint{2.027395in}{0.739656in}}%
\pgfpathlineto{\pgfqpoint{2.026542in}{0.739656in}}%
\pgfpathlineto{\pgfqpoint{2.025690in}{0.739656in}}%
\pgfpathlineto{\pgfqpoint{2.024837in}{0.739656in}}%
\pgfpathlineto{\pgfqpoint{2.023984in}{0.739656in}}%
\pgfpathlineto{\pgfqpoint{2.023131in}{0.739656in}}%
\pgfpathlineto{\pgfqpoint{2.022278in}{0.739656in}}%
\pgfpathlineto{\pgfqpoint{2.021425in}{0.739656in}}%
\pgfpathlineto{\pgfqpoint{2.020572in}{0.739656in}}%
\pgfpathlineto{\pgfqpoint{2.019719in}{0.739656in}}%
\pgfpathlineto{\pgfqpoint{2.018866in}{0.739656in}}%
\pgfpathlineto{\pgfqpoint{2.018014in}{0.739656in}}%
\pgfpathlineto{\pgfqpoint{2.017161in}{0.739656in}}%
\pgfpathlineto{\pgfqpoint{2.016308in}{0.739656in}}%
\pgfpathlineto{\pgfqpoint{2.015455in}{0.739656in}}%
\pgfpathlineto{\pgfqpoint{2.014602in}{0.739656in}}%
\pgfpathlineto{\pgfqpoint{2.013749in}{0.739656in}}%
\pgfpathlineto{\pgfqpoint{2.012896in}{0.739656in}}%
\pgfpathlineto{\pgfqpoint{2.012043in}{0.739656in}}%
\pgfpathlineto{\pgfqpoint{2.011191in}{0.739656in}}%
\pgfpathlineto{\pgfqpoint{2.010338in}{0.739656in}}%
\pgfpathlineto{\pgfqpoint{2.009485in}{0.739656in}}%
\pgfpathlineto{\pgfqpoint{2.008632in}{0.739656in}}%
\pgfpathlineto{\pgfqpoint{2.007779in}{0.739656in}}%
\pgfpathlineto{\pgfqpoint{2.006926in}{0.739656in}}%
\pgfpathlineto{\pgfqpoint{2.006073in}{0.739656in}}%
\pgfpathlineto{\pgfqpoint{2.005220in}{0.739656in}}%
\pgfpathlineto{\pgfqpoint{2.004368in}{0.739656in}}%
\pgfpathlineto{\pgfqpoint{2.003515in}{0.739656in}}%
\pgfpathlineto{\pgfqpoint{2.002662in}{0.739656in}}%
\pgfpathlineto{\pgfqpoint{2.001809in}{0.739656in}}%
\pgfpathlineto{\pgfqpoint{2.000956in}{0.739656in}}%
\pgfpathlineto{\pgfqpoint{2.000103in}{0.739656in}}%
\pgfpathlineto{\pgfqpoint{1.999250in}{0.739656in}}%
\pgfpathlineto{\pgfqpoint{1.998397in}{0.739656in}}%
\pgfpathlineto{\pgfqpoint{1.997545in}{0.739656in}}%
\pgfpathlineto{\pgfqpoint{1.996692in}{0.739656in}}%
\pgfpathlineto{\pgfqpoint{1.995839in}{0.739656in}}%
\pgfpathlineto{\pgfqpoint{1.994986in}{0.739656in}}%
\pgfpathlineto{\pgfqpoint{1.994133in}{0.739656in}}%
\pgfpathlineto{\pgfqpoint{1.993280in}{0.739656in}}%
\pgfpathlineto{\pgfqpoint{1.992427in}{0.739656in}}%
\pgfpathlineto{\pgfqpoint{1.991574in}{0.739656in}}%
\pgfpathlineto{\pgfqpoint{1.990721in}{0.739656in}}%
\pgfpathlineto{\pgfqpoint{1.989869in}{0.739656in}}%
\pgfpathlineto{\pgfqpoint{1.989016in}{0.739656in}}%
\pgfpathlineto{\pgfqpoint{1.988163in}{0.739656in}}%
\pgfpathlineto{\pgfqpoint{1.987310in}{0.739656in}}%
\pgfpathlineto{\pgfqpoint{1.986457in}{0.739656in}}%
\pgfpathlineto{\pgfqpoint{1.985604in}{0.739656in}}%
\pgfpathlineto{\pgfqpoint{1.984751in}{0.739656in}}%
\pgfpathlineto{\pgfqpoint{1.983898in}{0.739656in}}%
\pgfpathlineto{\pgfqpoint{1.983046in}{0.739656in}}%
\pgfpathlineto{\pgfqpoint{1.982193in}{0.739656in}}%
\pgfpathlineto{\pgfqpoint{1.981340in}{0.739656in}}%
\pgfpathlineto{\pgfqpoint{1.980487in}{0.739656in}}%
\pgfpathlineto{\pgfqpoint{1.979634in}{0.739656in}}%
\pgfpathlineto{\pgfqpoint{1.978781in}{0.739656in}}%
\pgfpathlineto{\pgfqpoint{1.977928in}{0.739656in}}%
\pgfpathlineto{\pgfqpoint{1.977075in}{0.739656in}}%
\pgfpathlineto{\pgfqpoint{1.976223in}{0.739656in}}%
\pgfpathlineto{\pgfqpoint{1.975370in}{0.739656in}}%
\pgfpathlineto{\pgfqpoint{1.974517in}{0.739656in}}%
\pgfpathlineto{\pgfqpoint{1.973664in}{0.739656in}}%
\pgfpathlineto{\pgfqpoint{1.972811in}{0.739656in}}%
\pgfpathlineto{\pgfqpoint{1.971958in}{0.739656in}}%
\pgfpathlineto{\pgfqpoint{1.971105in}{0.739656in}}%
\pgfpathlineto{\pgfqpoint{1.970252in}{0.739656in}}%
\pgfpathlineto{\pgfqpoint{1.969400in}{0.739656in}}%
\pgfpathlineto{\pgfqpoint{1.968547in}{0.739656in}}%
\pgfpathlineto{\pgfqpoint{1.967694in}{0.739656in}}%
\pgfpathlineto{\pgfqpoint{1.966841in}{0.739656in}}%
\pgfpathlineto{\pgfqpoint{1.965988in}{0.739656in}}%
\pgfpathlineto{\pgfqpoint{1.965135in}{0.739656in}}%
\pgfpathlineto{\pgfqpoint{1.964282in}{0.739656in}}%
\pgfpathlineto{\pgfqpoint{1.963429in}{0.739656in}}%
\pgfpathlineto{\pgfqpoint{1.962577in}{0.739656in}}%
\pgfpathlineto{\pgfqpoint{1.961724in}{0.739656in}}%
\pgfpathlineto{\pgfqpoint{1.960871in}{0.739656in}}%
\pgfpathlineto{\pgfqpoint{1.960018in}{0.739656in}}%
\pgfpathlineto{\pgfqpoint{1.959165in}{0.739656in}}%
\pgfpathlineto{\pgfqpoint{1.958312in}{0.739656in}}%
\pgfpathlineto{\pgfqpoint{1.957459in}{0.739656in}}%
\pgfpathlineto{\pgfqpoint{1.956606in}{0.739656in}}%
\pgfpathlineto{\pgfqpoint{1.955753in}{0.739656in}}%
\pgfpathlineto{\pgfqpoint{1.954901in}{0.739656in}}%
\pgfpathlineto{\pgfqpoint{1.954048in}{0.739656in}}%
\pgfpathlineto{\pgfqpoint{1.953195in}{0.739656in}}%
\pgfpathlineto{\pgfqpoint{1.952342in}{0.739656in}}%
\pgfpathlineto{\pgfqpoint{1.951489in}{0.739656in}}%
\pgfpathlineto{\pgfqpoint{1.950636in}{0.739656in}}%
\pgfpathlineto{\pgfqpoint{1.949783in}{0.739656in}}%
\pgfpathlineto{\pgfqpoint{1.948930in}{0.739656in}}%
\pgfpathlineto{\pgfqpoint{1.948078in}{0.739656in}}%
\pgfpathlineto{\pgfqpoint{1.947225in}{0.739656in}}%
\pgfpathlineto{\pgfqpoint{1.946372in}{0.739656in}}%
\pgfpathlineto{\pgfqpoint{1.945519in}{0.739656in}}%
\pgfpathlineto{\pgfqpoint{1.944666in}{0.739656in}}%
\pgfpathlineto{\pgfqpoint{1.943813in}{0.739656in}}%
\pgfpathlineto{\pgfqpoint{1.942960in}{0.739656in}}%
\pgfpathlineto{\pgfqpoint{1.942107in}{0.739656in}}%
\pgfpathlineto{\pgfqpoint{1.941255in}{0.739656in}}%
\pgfpathlineto{\pgfqpoint{1.940402in}{0.739656in}}%
\pgfpathlineto{\pgfqpoint{1.939549in}{0.739656in}}%
\pgfpathlineto{\pgfqpoint{1.938696in}{0.739656in}}%
\pgfpathlineto{\pgfqpoint{1.937843in}{0.739656in}}%
\pgfpathlineto{\pgfqpoint{1.936990in}{0.739656in}}%
\pgfpathlineto{\pgfqpoint{1.936137in}{0.739656in}}%
\pgfpathlineto{\pgfqpoint{1.935284in}{0.739656in}}%
\pgfpathlineto{\pgfqpoint{1.934432in}{0.739656in}}%
\pgfpathlineto{\pgfqpoint{1.933579in}{0.739656in}}%
\pgfpathlineto{\pgfqpoint{1.932726in}{0.739656in}}%
\pgfpathlineto{\pgfqpoint{1.931873in}{0.739656in}}%
\pgfpathlineto{\pgfqpoint{1.931020in}{0.739656in}}%
\pgfpathlineto{\pgfqpoint{1.930167in}{0.739656in}}%
\pgfpathlineto{\pgfqpoint{1.929314in}{0.739656in}}%
\pgfpathlineto{\pgfqpoint{1.928461in}{0.739656in}}%
\pgfpathlineto{\pgfqpoint{1.927608in}{0.739656in}}%
\pgfpathlineto{\pgfqpoint{1.926756in}{0.739656in}}%
\pgfpathlineto{\pgfqpoint{1.925903in}{0.739656in}}%
\pgfpathlineto{\pgfqpoint{1.925050in}{0.739656in}}%
\pgfpathlineto{\pgfqpoint{1.924197in}{0.739656in}}%
\pgfpathlineto{\pgfqpoint{1.923344in}{0.739656in}}%
\pgfpathlineto{\pgfqpoint{1.922491in}{0.739656in}}%
\pgfpathlineto{\pgfqpoint{1.921638in}{0.739656in}}%
\pgfpathlineto{\pgfqpoint{1.920785in}{0.739656in}}%
\pgfpathlineto{\pgfqpoint{1.919933in}{0.739656in}}%
\pgfpathlineto{\pgfqpoint{1.919080in}{0.739656in}}%
\pgfpathlineto{\pgfqpoint{1.918227in}{0.739656in}}%
\pgfpathlineto{\pgfqpoint{1.917374in}{0.739656in}}%
\pgfpathlineto{\pgfqpoint{1.916521in}{0.739656in}}%
\pgfpathlineto{\pgfqpoint{1.915668in}{0.739656in}}%
\pgfpathlineto{\pgfqpoint{1.914815in}{0.739656in}}%
\pgfpathlineto{\pgfqpoint{1.913962in}{0.739656in}}%
\pgfpathlineto{\pgfqpoint{1.913110in}{0.739656in}}%
\pgfpathlineto{\pgfqpoint{1.912257in}{0.739656in}}%
\pgfpathlineto{\pgfqpoint{1.911404in}{0.739656in}}%
\pgfpathlineto{\pgfqpoint{1.910551in}{0.739656in}}%
\pgfpathlineto{\pgfqpoint{1.909698in}{0.739656in}}%
\pgfpathlineto{\pgfqpoint{1.908845in}{0.739656in}}%
\pgfpathlineto{\pgfqpoint{1.907992in}{0.739656in}}%
\pgfpathlineto{\pgfqpoint{1.907139in}{0.739656in}}%
\pgfpathlineto{\pgfqpoint{1.906287in}{0.739656in}}%
\pgfpathlineto{\pgfqpoint{1.905434in}{0.739656in}}%
\pgfpathlineto{\pgfqpoint{1.904581in}{0.739656in}}%
\pgfpathlineto{\pgfqpoint{1.903728in}{0.739656in}}%
\pgfpathlineto{\pgfqpoint{1.902875in}{0.739656in}}%
\pgfpathlineto{\pgfqpoint{1.902022in}{0.739656in}}%
\pgfpathlineto{\pgfqpoint{1.901169in}{0.739656in}}%
\pgfpathlineto{\pgfqpoint{1.900316in}{0.739656in}}%
\pgfpathlineto{\pgfqpoint{1.899463in}{0.739656in}}%
\pgfpathlineto{\pgfqpoint{1.898611in}{0.739656in}}%
\pgfpathlineto{\pgfqpoint{1.897758in}{0.739656in}}%
\pgfpathlineto{\pgfqpoint{1.896905in}{0.739656in}}%
\pgfpathlineto{\pgfqpoint{1.896052in}{0.739656in}}%
\pgfpathlineto{\pgfqpoint{1.895199in}{0.739656in}}%
\pgfpathlineto{\pgfqpoint{1.894346in}{0.739656in}}%
\pgfpathlineto{\pgfqpoint{1.893493in}{0.739656in}}%
\pgfpathlineto{\pgfqpoint{1.892640in}{0.739656in}}%
\pgfpathlineto{\pgfqpoint{1.891788in}{0.739656in}}%
\pgfpathlineto{\pgfqpoint{1.890935in}{0.739656in}}%
\pgfpathlineto{\pgfqpoint{1.890082in}{0.739656in}}%
\pgfpathlineto{\pgfqpoint{1.889229in}{0.739656in}}%
\pgfpathlineto{\pgfqpoint{1.888376in}{0.739656in}}%
\pgfpathlineto{\pgfqpoint{1.887523in}{0.739656in}}%
\pgfpathlineto{\pgfqpoint{1.886670in}{0.739656in}}%
\pgfpathlineto{\pgfqpoint{1.885817in}{0.739656in}}%
\pgfpathlineto{\pgfqpoint{1.884965in}{0.739656in}}%
\pgfpathlineto{\pgfqpoint{1.884112in}{0.739656in}}%
\pgfpathlineto{\pgfqpoint{1.883259in}{0.739656in}}%
\pgfpathlineto{\pgfqpoint{1.882406in}{0.739656in}}%
\pgfpathlineto{\pgfqpoint{1.881553in}{0.739656in}}%
\pgfpathlineto{\pgfqpoint{1.880700in}{0.739656in}}%
\pgfpathlineto{\pgfqpoint{1.879847in}{0.739656in}}%
\pgfpathlineto{\pgfqpoint{1.878994in}{0.739656in}}%
\pgfpathlineto{\pgfqpoint{1.878142in}{0.739656in}}%
\pgfpathlineto{\pgfqpoint{1.877289in}{0.739656in}}%
\pgfpathlineto{\pgfqpoint{1.876436in}{0.739656in}}%
\pgfpathlineto{\pgfqpoint{1.875583in}{0.739656in}}%
\pgfpathlineto{\pgfqpoint{1.874730in}{0.739656in}}%
\pgfpathlineto{\pgfqpoint{1.873877in}{0.739656in}}%
\pgfpathlineto{\pgfqpoint{1.873024in}{0.739656in}}%
\pgfpathlineto{\pgfqpoint{1.872171in}{0.739656in}}%
\pgfpathlineto{\pgfqpoint{1.871318in}{0.739656in}}%
\pgfpathlineto{\pgfqpoint{1.870466in}{0.739656in}}%
\pgfpathlineto{\pgfqpoint{1.869613in}{0.739656in}}%
\pgfpathlineto{\pgfqpoint{1.868760in}{0.739656in}}%
\pgfpathlineto{\pgfqpoint{1.867907in}{0.739656in}}%
\pgfpathlineto{\pgfqpoint{1.867054in}{0.739656in}}%
\pgfpathlineto{\pgfqpoint{1.866201in}{0.739656in}}%
\pgfpathlineto{\pgfqpoint{1.865348in}{0.739656in}}%
\pgfpathlineto{\pgfqpoint{1.864495in}{0.739656in}}%
\pgfpathlineto{\pgfqpoint{1.863643in}{0.739656in}}%
\pgfpathlineto{\pgfqpoint{1.862790in}{0.739656in}}%
\pgfpathlineto{\pgfqpoint{1.861937in}{0.739656in}}%
\pgfpathlineto{\pgfqpoint{1.861084in}{0.739656in}}%
\pgfpathlineto{\pgfqpoint{1.860231in}{0.739656in}}%
\pgfpathlineto{\pgfqpoint{1.859378in}{0.739656in}}%
\pgfpathlineto{\pgfqpoint{1.858525in}{0.739656in}}%
\pgfpathlineto{\pgfqpoint{1.857672in}{0.739656in}}%
\pgfpathlineto{\pgfqpoint{1.856820in}{0.739656in}}%
\pgfpathlineto{\pgfqpoint{1.855967in}{0.739656in}}%
\pgfpathlineto{\pgfqpoint{1.855114in}{0.739656in}}%
\pgfpathlineto{\pgfqpoint{1.854261in}{0.739656in}}%
\pgfpathlineto{\pgfqpoint{1.853408in}{0.739656in}}%
\pgfpathlineto{\pgfqpoint{1.852555in}{0.739656in}}%
\pgfpathlineto{\pgfqpoint{1.851702in}{0.739656in}}%
\pgfpathlineto{\pgfqpoint{1.850849in}{0.739656in}}%
\pgfpathlineto{\pgfqpoint{1.849997in}{0.739656in}}%
\pgfpathlineto{\pgfqpoint{1.849144in}{0.739656in}}%
\pgfpathlineto{\pgfqpoint{1.848291in}{0.739656in}}%
\pgfpathlineto{\pgfqpoint{1.847438in}{0.739656in}}%
\pgfpathlineto{\pgfqpoint{1.846585in}{0.739656in}}%
\pgfpathlineto{\pgfqpoint{1.845732in}{0.739656in}}%
\pgfpathlineto{\pgfqpoint{1.844879in}{0.739656in}}%
\pgfpathlineto{\pgfqpoint{1.844026in}{0.739656in}}%
\pgfpathlineto{\pgfqpoint{1.843174in}{0.739656in}}%
\pgfpathlineto{\pgfqpoint{1.842321in}{0.739656in}}%
\pgfpathlineto{\pgfqpoint{1.841468in}{0.739656in}}%
\pgfpathlineto{\pgfqpoint{1.840615in}{0.739656in}}%
\pgfpathlineto{\pgfqpoint{1.839762in}{0.739656in}}%
\pgfpathlineto{\pgfqpoint{1.838909in}{0.739656in}}%
\pgfpathlineto{\pgfqpoint{1.838056in}{0.739656in}}%
\pgfpathlineto{\pgfqpoint{1.837203in}{0.739656in}}%
\pgfpathlineto{\pgfqpoint{1.836350in}{0.739656in}}%
\pgfpathlineto{\pgfqpoint{1.835498in}{0.739656in}}%
\pgfpathlineto{\pgfqpoint{1.834645in}{0.739656in}}%
\pgfpathlineto{\pgfqpoint{1.833792in}{0.739656in}}%
\pgfpathlineto{\pgfqpoint{1.832939in}{0.739656in}}%
\pgfpathlineto{\pgfqpoint{1.832086in}{0.739656in}}%
\pgfpathlineto{\pgfqpoint{1.831233in}{0.739656in}}%
\pgfpathlineto{\pgfqpoint{1.830380in}{0.739656in}}%
\pgfpathlineto{\pgfqpoint{1.829527in}{0.739656in}}%
\pgfpathlineto{\pgfqpoint{1.828675in}{0.739656in}}%
\pgfpathlineto{\pgfqpoint{1.827822in}{0.739656in}}%
\pgfpathlineto{\pgfqpoint{1.826969in}{0.739656in}}%
\pgfpathlineto{\pgfqpoint{1.826116in}{0.739656in}}%
\pgfpathlineto{\pgfqpoint{1.825263in}{0.739656in}}%
\pgfpathlineto{\pgfqpoint{1.824410in}{0.739656in}}%
\pgfpathlineto{\pgfqpoint{1.823557in}{0.739656in}}%
\pgfpathlineto{\pgfqpoint{1.822704in}{0.739656in}}%
\pgfpathlineto{\pgfqpoint{1.821852in}{0.739656in}}%
\pgfpathlineto{\pgfqpoint{1.820999in}{0.739656in}}%
\pgfpathlineto{\pgfqpoint{1.820146in}{0.739656in}}%
\pgfpathlineto{\pgfqpoint{1.819293in}{0.739656in}}%
\pgfpathlineto{\pgfqpoint{1.818440in}{0.739656in}}%
\pgfpathlineto{\pgfqpoint{1.817587in}{0.739656in}}%
\pgfpathlineto{\pgfqpoint{1.816734in}{0.739656in}}%
\pgfpathlineto{\pgfqpoint{1.815881in}{0.739656in}}%
\pgfpathlineto{\pgfqpoint{1.815029in}{0.739656in}}%
\pgfpathlineto{\pgfqpoint{1.814176in}{0.739656in}}%
\pgfpathlineto{\pgfqpoint{1.813323in}{0.739656in}}%
\pgfpathlineto{\pgfqpoint{1.812470in}{0.739656in}}%
\pgfpathlineto{\pgfqpoint{1.811617in}{0.739656in}}%
\pgfpathlineto{\pgfqpoint{1.810764in}{0.739656in}}%
\pgfpathlineto{\pgfqpoint{1.809911in}{0.739656in}}%
\pgfpathlineto{\pgfqpoint{1.809058in}{0.739656in}}%
\pgfpathlineto{\pgfqpoint{1.808205in}{0.739656in}}%
\pgfpathlineto{\pgfqpoint{1.807353in}{0.739656in}}%
\pgfpathlineto{\pgfqpoint{1.806500in}{0.739656in}}%
\pgfpathlineto{\pgfqpoint{1.805647in}{0.739656in}}%
\pgfpathlineto{\pgfqpoint{1.804794in}{0.739656in}}%
\pgfpathlineto{\pgfqpoint{1.803941in}{0.739656in}}%
\pgfpathlineto{\pgfqpoint{1.803088in}{0.739656in}}%
\pgfpathlineto{\pgfqpoint{1.802235in}{0.739656in}}%
\pgfpathlineto{\pgfqpoint{1.801382in}{0.739656in}}%
\pgfpathlineto{\pgfqpoint{1.800530in}{0.739656in}}%
\pgfpathlineto{\pgfqpoint{1.799677in}{0.739656in}}%
\pgfpathlineto{\pgfqpoint{1.798824in}{0.739656in}}%
\pgfpathlineto{\pgfqpoint{1.797971in}{0.739656in}}%
\pgfpathlineto{\pgfqpoint{1.797118in}{0.739656in}}%
\pgfpathlineto{\pgfqpoint{1.796265in}{0.739656in}}%
\pgfpathlineto{\pgfqpoint{1.795412in}{0.739656in}}%
\pgfpathlineto{\pgfqpoint{1.794559in}{0.739656in}}%
\pgfpathlineto{\pgfqpoint{1.793707in}{0.739656in}}%
\pgfpathlineto{\pgfqpoint{1.792854in}{0.739656in}}%
\pgfpathlineto{\pgfqpoint{1.792001in}{0.739656in}}%
\pgfpathlineto{\pgfqpoint{1.791148in}{0.739656in}}%
\pgfpathlineto{\pgfqpoint{1.790295in}{0.739656in}}%
\pgfpathlineto{\pgfqpoint{1.789442in}{0.739656in}}%
\pgfpathlineto{\pgfqpoint{1.788589in}{0.739656in}}%
\pgfpathlineto{\pgfqpoint{1.787736in}{0.739656in}}%
\pgfpathlineto{\pgfqpoint{1.786884in}{0.739656in}}%
\pgfpathlineto{\pgfqpoint{1.786031in}{0.739656in}}%
\pgfpathlineto{\pgfqpoint{1.785178in}{0.739656in}}%
\pgfpathlineto{\pgfqpoint{1.784325in}{0.739656in}}%
\pgfpathlineto{\pgfqpoint{1.783472in}{0.739656in}}%
\pgfpathlineto{\pgfqpoint{1.782619in}{0.739656in}}%
\pgfpathlineto{\pgfqpoint{1.781766in}{0.739656in}}%
\pgfpathlineto{\pgfqpoint{1.780913in}{0.739656in}}%
\pgfpathlineto{\pgfqpoint{1.780060in}{0.739656in}}%
\pgfpathlineto{\pgfqpoint{1.779208in}{0.739656in}}%
\pgfpathlineto{\pgfqpoint{1.778355in}{0.739656in}}%
\pgfpathlineto{\pgfqpoint{1.777502in}{0.739656in}}%
\pgfpathlineto{\pgfqpoint{1.776649in}{0.739656in}}%
\pgfpathlineto{\pgfqpoint{1.775796in}{0.739656in}}%
\pgfpathlineto{\pgfqpoint{1.774943in}{0.739656in}}%
\pgfpathlineto{\pgfqpoint{1.774090in}{0.739656in}}%
\pgfpathlineto{\pgfqpoint{1.773237in}{0.739656in}}%
\pgfpathlineto{\pgfqpoint{1.772385in}{0.739656in}}%
\pgfpathlineto{\pgfqpoint{1.771532in}{0.739656in}}%
\pgfpathlineto{\pgfqpoint{1.770679in}{0.739656in}}%
\pgfpathlineto{\pgfqpoint{1.769826in}{0.739656in}}%
\pgfpathlineto{\pgfqpoint{1.768973in}{0.739656in}}%
\pgfpathlineto{\pgfqpoint{1.768120in}{0.739656in}}%
\pgfpathlineto{\pgfqpoint{1.767267in}{0.739656in}}%
\pgfpathlineto{\pgfqpoint{1.766414in}{0.739656in}}%
\pgfpathlineto{\pgfqpoint{1.765562in}{0.739656in}}%
\pgfpathlineto{\pgfqpoint{1.764709in}{0.739656in}}%
\pgfpathlineto{\pgfqpoint{1.763856in}{0.739656in}}%
\pgfpathlineto{\pgfqpoint{1.763003in}{0.739656in}}%
\pgfpathlineto{\pgfqpoint{1.762150in}{0.739656in}}%
\pgfpathlineto{\pgfqpoint{1.761297in}{0.739656in}}%
\pgfpathlineto{\pgfqpoint{1.760444in}{0.739656in}}%
\pgfpathlineto{\pgfqpoint{1.759591in}{0.739656in}}%
\pgfpathlineto{\pgfqpoint{1.758739in}{0.739656in}}%
\pgfpathlineto{\pgfqpoint{1.757886in}{0.739656in}}%
\pgfpathlineto{\pgfqpoint{1.757033in}{0.739656in}}%
\pgfpathlineto{\pgfqpoint{1.756180in}{0.739656in}}%
\pgfpathlineto{\pgfqpoint{1.755327in}{0.739656in}}%
\pgfpathlineto{\pgfqpoint{1.754474in}{0.739656in}}%
\pgfpathlineto{\pgfqpoint{1.753621in}{0.739656in}}%
\pgfpathlineto{\pgfqpoint{1.752768in}{0.739656in}}%
\pgfpathlineto{\pgfqpoint{1.751916in}{0.739656in}}%
\pgfpathlineto{\pgfqpoint{1.751063in}{0.739656in}}%
\pgfpathlineto{\pgfqpoint{1.750210in}{0.739656in}}%
\pgfpathlineto{\pgfqpoint{1.749357in}{0.739656in}}%
\pgfpathlineto{\pgfqpoint{1.748504in}{0.739656in}}%
\pgfpathlineto{\pgfqpoint{1.747651in}{0.739656in}}%
\pgfpathlineto{\pgfqpoint{1.746798in}{0.739656in}}%
\pgfpathlineto{\pgfqpoint{1.745945in}{0.739656in}}%
\pgfpathlineto{\pgfqpoint{1.745092in}{0.739656in}}%
\pgfpathlineto{\pgfqpoint{1.744240in}{0.739656in}}%
\pgfpathlineto{\pgfqpoint{1.743387in}{0.739656in}}%
\pgfpathlineto{\pgfqpoint{1.742534in}{0.739656in}}%
\pgfpathlineto{\pgfqpoint{1.741681in}{0.739656in}}%
\pgfpathlineto{\pgfqpoint{1.740828in}{0.739656in}}%
\pgfpathlineto{\pgfqpoint{1.739975in}{0.739656in}}%
\pgfpathlineto{\pgfqpoint{1.739122in}{0.739656in}}%
\pgfpathlineto{\pgfqpoint{1.738269in}{0.739656in}}%
\pgfpathlineto{\pgfqpoint{1.737417in}{0.739656in}}%
\pgfpathlineto{\pgfqpoint{1.736564in}{0.739656in}}%
\pgfpathlineto{\pgfqpoint{1.735711in}{0.739656in}}%
\pgfpathlineto{\pgfqpoint{1.734858in}{0.739656in}}%
\pgfpathlineto{\pgfqpoint{1.734005in}{0.739656in}}%
\pgfpathlineto{\pgfqpoint{1.733152in}{0.739656in}}%
\pgfpathlineto{\pgfqpoint{1.732299in}{0.739656in}}%
\pgfpathlineto{\pgfqpoint{1.731446in}{0.739656in}}%
\pgfpathlineto{\pgfqpoint{1.730594in}{0.739656in}}%
\pgfpathlineto{\pgfqpoint{1.729741in}{0.739656in}}%
\pgfpathlineto{\pgfqpoint{1.728888in}{0.739656in}}%
\pgfpathlineto{\pgfqpoint{1.728035in}{0.739656in}}%
\pgfpathlineto{\pgfqpoint{1.727182in}{0.739656in}}%
\pgfpathlineto{\pgfqpoint{1.726329in}{0.739656in}}%
\pgfpathlineto{\pgfqpoint{1.725476in}{0.739656in}}%
\pgfpathlineto{\pgfqpoint{1.724623in}{0.739656in}}%
\pgfpathlineto{\pgfqpoint{1.723771in}{0.739656in}}%
\pgfpathlineto{\pgfqpoint{1.722918in}{0.739656in}}%
\pgfpathlineto{\pgfqpoint{1.722065in}{0.739656in}}%
\pgfpathlineto{\pgfqpoint{1.721212in}{0.739656in}}%
\pgfpathlineto{\pgfqpoint{1.720359in}{0.739656in}}%
\pgfpathlineto{\pgfqpoint{1.719506in}{0.739656in}}%
\pgfpathlineto{\pgfqpoint{1.718653in}{0.739656in}}%
\pgfpathlineto{\pgfqpoint{1.717800in}{0.739656in}}%
\pgfpathlineto{\pgfqpoint{1.716947in}{0.739656in}}%
\pgfpathlineto{\pgfqpoint{1.716095in}{0.739656in}}%
\pgfpathlineto{\pgfqpoint{1.715242in}{0.739656in}}%
\pgfpathlineto{\pgfqpoint{1.714389in}{0.739656in}}%
\pgfpathlineto{\pgfqpoint{1.713536in}{0.739656in}}%
\pgfpathlineto{\pgfqpoint{1.712683in}{0.739656in}}%
\pgfpathlineto{\pgfqpoint{1.711830in}{0.739656in}}%
\pgfpathlineto{\pgfqpoint{1.710977in}{0.739656in}}%
\pgfpathlineto{\pgfqpoint{1.710124in}{0.739656in}}%
\pgfpathlineto{\pgfqpoint{1.709272in}{0.739656in}}%
\pgfpathlineto{\pgfqpoint{1.708419in}{0.739656in}}%
\pgfpathlineto{\pgfqpoint{1.707566in}{0.739656in}}%
\pgfpathlineto{\pgfqpoint{1.706713in}{0.739656in}}%
\pgfpathlineto{\pgfqpoint{1.705860in}{0.739656in}}%
\pgfpathlineto{\pgfqpoint{1.705007in}{0.739656in}}%
\pgfpathlineto{\pgfqpoint{1.704154in}{0.739656in}}%
\pgfpathlineto{\pgfqpoint{1.703301in}{0.739656in}}%
\pgfpathlineto{\pgfqpoint{1.702449in}{0.739656in}}%
\pgfpathlineto{\pgfqpoint{1.701596in}{0.739656in}}%
\pgfpathlineto{\pgfqpoint{1.700743in}{0.739656in}}%
\pgfpathlineto{\pgfqpoint{1.699890in}{0.739656in}}%
\pgfpathlineto{\pgfqpoint{1.699037in}{0.739656in}}%
\pgfpathlineto{\pgfqpoint{1.698184in}{0.739656in}}%
\pgfpathlineto{\pgfqpoint{1.697331in}{0.739656in}}%
\pgfpathlineto{\pgfqpoint{1.696478in}{0.739656in}}%
\pgfpathlineto{\pgfqpoint{1.695626in}{0.739656in}}%
\pgfpathlineto{\pgfqpoint{1.694773in}{0.739656in}}%
\pgfpathlineto{\pgfqpoint{1.693920in}{0.739656in}}%
\pgfpathlineto{\pgfqpoint{1.693067in}{0.739656in}}%
\pgfpathlineto{\pgfqpoint{1.692214in}{0.739656in}}%
\pgfpathlineto{\pgfqpoint{1.691361in}{0.739656in}}%
\pgfpathlineto{\pgfqpoint{1.690508in}{0.739656in}}%
\pgfpathlineto{\pgfqpoint{1.689655in}{0.739656in}}%
\pgfpathlineto{\pgfqpoint{1.688802in}{0.739656in}}%
\pgfpathlineto{\pgfqpoint{1.687950in}{0.739656in}}%
\pgfpathlineto{\pgfqpoint{1.687097in}{0.739656in}}%
\pgfpathlineto{\pgfqpoint{1.686244in}{0.739656in}}%
\pgfpathlineto{\pgfqpoint{1.685391in}{0.739656in}}%
\pgfpathlineto{\pgfqpoint{1.684538in}{0.739656in}}%
\pgfpathlineto{\pgfqpoint{1.683685in}{0.739656in}}%
\pgfpathlineto{\pgfqpoint{1.682832in}{0.739656in}}%
\pgfpathlineto{\pgfqpoint{1.681979in}{0.739656in}}%
\pgfpathlineto{\pgfqpoint{1.681127in}{0.739656in}}%
\pgfpathlineto{\pgfqpoint{1.680274in}{0.739656in}}%
\pgfpathlineto{\pgfqpoint{1.679421in}{0.739656in}}%
\pgfpathlineto{\pgfqpoint{1.678568in}{0.739656in}}%
\pgfpathlineto{\pgfqpoint{1.677715in}{0.739656in}}%
\pgfpathlineto{\pgfqpoint{1.676862in}{0.739656in}}%
\pgfpathlineto{\pgfqpoint{1.676009in}{0.739656in}}%
\pgfpathlineto{\pgfqpoint{1.675156in}{0.739656in}}%
\pgfpathlineto{\pgfqpoint{1.674304in}{0.739656in}}%
\pgfpathlineto{\pgfqpoint{1.673451in}{0.739656in}}%
\pgfpathlineto{\pgfqpoint{1.672598in}{0.739656in}}%
\pgfpathlineto{\pgfqpoint{1.671745in}{0.739656in}}%
\pgfpathlineto{\pgfqpoint{1.670892in}{0.739656in}}%
\pgfpathlineto{\pgfqpoint{1.670039in}{0.739656in}}%
\pgfpathlineto{\pgfqpoint{1.669186in}{0.739656in}}%
\pgfpathlineto{\pgfqpoint{1.668333in}{0.739656in}}%
\pgfpathlineto{\pgfqpoint{1.667481in}{0.739656in}}%
\pgfpathlineto{\pgfqpoint{1.666628in}{0.739656in}}%
\pgfpathlineto{\pgfqpoint{1.665775in}{0.739656in}}%
\pgfpathlineto{\pgfqpoint{1.664922in}{0.739656in}}%
\pgfpathlineto{\pgfqpoint{1.664069in}{0.739656in}}%
\pgfpathlineto{\pgfqpoint{1.663216in}{0.739656in}}%
\pgfpathlineto{\pgfqpoint{1.662363in}{0.739656in}}%
\pgfpathlineto{\pgfqpoint{1.661510in}{0.739656in}}%
\pgfpathlineto{\pgfqpoint{1.660657in}{0.739656in}}%
\pgfpathlineto{\pgfqpoint{1.659805in}{0.739656in}}%
\pgfpathlineto{\pgfqpoint{1.658952in}{0.739656in}}%
\pgfpathlineto{\pgfqpoint{1.658099in}{0.739656in}}%
\pgfpathlineto{\pgfqpoint{1.657246in}{0.739656in}}%
\pgfpathlineto{\pgfqpoint{1.656393in}{0.739656in}}%
\pgfpathlineto{\pgfqpoint{1.655540in}{0.739656in}}%
\pgfpathlineto{\pgfqpoint{1.654687in}{0.739656in}}%
\pgfpathlineto{\pgfqpoint{1.653834in}{0.739656in}}%
\pgfpathlineto{\pgfqpoint{1.652982in}{0.739656in}}%
\pgfpathlineto{\pgfqpoint{1.652129in}{0.739656in}}%
\pgfpathlineto{\pgfqpoint{1.651276in}{0.739656in}}%
\pgfpathlineto{\pgfqpoint{1.650423in}{0.739656in}}%
\pgfpathlineto{\pgfqpoint{1.649570in}{0.739656in}}%
\pgfpathlineto{\pgfqpoint{1.648717in}{0.739656in}}%
\pgfpathlineto{\pgfqpoint{1.647864in}{0.739656in}}%
\pgfpathlineto{\pgfqpoint{1.647011in}{0.739656in}}%
\pgfpathlineto{\pgfqpoint{1.646159in}{0.739656in}}%
\pgfpathlineto{\pgfqpoint{1.645306in}{0.739656in}}%
\pgfpathlineto{\pgfqpoint{1.644453in}{0.739656in}}%
\pgfpathlineto{\pgfqpoint{1.643600in}{0.739656in}}%
\pgfpathlineto{\pgfqpoint{1.642747in}{0.739656in}}%
\pgfpathlineto{\pgfqpoint{1.641894in}{0.739656in}}%
\pgfpathlineto{\pgfqpoint{1.641041in}{0.739656in}}%
\pgfpathlineto{\pgfqpoint{1.640188in}{0.739656in}}%
\pgfpathlineto{\pgfqpoint{1.639336in}{0.739656in}}%
\pgfpathlineto{\pgfqpoint{1.638483in}{0.739656in}}%
\pgfpathlineto{\pgfqpoint{1.637630in}{0.739656in}}%
\pgfpathlineto{\pgfqpoint{1.636777in}{0.739656in}}%
\pgfpathlineto{\pgfqpoint{1.635924in}{0.739656in}}%
\pgfpathlineto{\pgfqpoint{1.635071in}{0.739656in}}%
\pgfpathlineto{\pgfqpoint{1.634218in}{0.739656in}}%
\pgfpathlineto{\pgfqpoint{1.633365in}{0.739656in}}%
\pgfpathlineto{\pgfqpoint{1.632513in}{0.739656in}}%
\pgfpathlineto{\pgfqpoint{1.631660in}{0.739656in}}%
\pgfpathlineto{\pgfqpoint{1.630807in}{0.739656in}}%
\pgfpathlineto{\pgfqpoint{1.629954in}{0.739656in}}%
\pgfpathlineto{\pgfqpoint{1.629101in}{0.739656in}}%
\pgfpathlineto{\pgfqpoint{1.628248in}{0.739656in}}%
\pgfpathlineto{\pgfqpoint{1.627395in}{0.739656in}}%
\pgfpathlineto{\pgfqpoint{1.626542in}{0.739656in}}%
\pgfpathlineto{\pgfqpoint{1.625689in}{0.739656in}}%
\pgfpathlineto{\pgfqpoint{1.624837in}{0.739656in}}%
\pgfpathlineto{\pgfqpoint{1.623984in}{0.739656in}}%
\pgfpathlineto{\pgfqpoint{1.623131in}{0.739656in}}%
\pgfpathlineto{\pgfqpoint{1.622278in}{0.739656in}}%
\pgfpathlineto{\pgfqpoint{1.621425in}{0.739656in}}%
\pgfpathlineto{\pgfqpoint{1.620572in}{0.739656in}}%
\pgfpathlineto{\pgfqpoint{1.619719in}{0.739656in}}%
\pgfpathlineto{\pgfqpoint{1.618866in}{0.739656in}}%
\pgfpathlineto{\pgfqpoint{1.618014in}{0.739656in}}%
\pgfpathlineto{\pgfqpoint{1.617161in}{0.739656in}}%
\pgfpathlineto{\pgfqpoint{1.616308in}{0.739656in}}%
\pgfpathlineto{\pgfqpoint{1.615455in}{0.739656in}}%
\pgfpathlineto{\pgfqpoint{1.614602in}{0.739656in}}%
\pgfpathlineto{\pgfqpoint{1.613749in}{0.739656in}}%
\pgfpathlineto{\pgfqpoint{1.612896in}{0.739656in}}%
\pgfpathlineto{\pgfqpoint{1.612043in}{0.739656in}}%
\pgfpathlineto{\pgfqpoint{1.611191in}{0.739656in}}%
\pgfpathlineto{\pgfqpoint{1.610338in}{0.739656in}}%
\pgfpathlineto{\pgfqpoint{1.609485in}{0.739656in}}%
\pgfpathlineto{\pgfqpoint{1.608632in}{0.739656in}}%
\pgfpathlineto{\pgfqpoint{1.607779in}{0.739656in}}%
\pgfpathlineto{\pgfqpoint{1.606926in}{0.739656in}}%
\pgfpathlineto{\pgfqpoint{1.606073in}{0.739656in}}%
\pgfpathlineto{\pgfqpoint{1.605220in}{0.739656in}}%
\pgfpathlineto{\pgfqpoint{1.604368in}{0.739656in}}%
\pgfpathlineto{\pgfqpoint{1.603515in}{0.739656in}}%
\pgfpathlineto{\pgfqpoint{1.602662in}{0.739656in}}%
\pgfpathlineto{\pgfqpoint{1.601809in}{0.739656in}}%
\pgfpathlineto{\pgfqpoint{1.600956in}{0.739656in}}%
\pgfpathlineto{\pgfqpoint{1.600103in}{0.739656in}}%
\pgfpathlineto{\pgfqpoint{1.599250in}{0.739656in}}%
\pgfpathlineto{\pgfqpoint{1.598397in}{0.739656in}}%
\pgfpathlineto{\pgfqpoint{1.597544in}{0.739656in}}%
\pgfpathlineto{\pgfqpoint{1.596692in}{0.739656in}}%
\pgfpathlineto{\pgfqpoint{1.595839in}{0.739656in}}%
\pgfpathlineto{\pgfqpoint{1.594986in}{0.739656in}}%
\pgfpathlineto{\pgfqpoint{1.594133in}{0.739656in}}%
\pgfpathlineto{\pgfqpoint{1.593280in}{0.739656in}}%
\pgfpathlineto{\pgfqpoint{1.592427in}{0.739656in}}%
\pgfpathlineto{\pgfqpoint{1.591574in}{0.739656in}}%
\pgfpathlineto{\pgfqpoint{1.590721in}{0.739656in}}%
\pgfpathlineto{\pgfqpoint{1.589869in}{0.739656in}}%
\pgfpathlineto{\pgfqpoint{1.589016in}{0.739656in}}%
\pgfpathlineto{\pgfqpoint{1.588163in}{0.739656in}}%
\pgfpathlineto{\pgfqpoint{1.587310in}{0.739656in}}%
\pgfpathlineto{\pgfqpoint{1.586457in}{0.739656in}}%
\pgfpathlineto{\pgfqpoint{1.585604in}{0.739656in}}%
\pgfpathlineto{\pgfqpoint{1.584751in}{0.739656in}}%
\pgfpathlineto{\pgfqpoint{1.583898in}{0.739656in}}%
\pgfpathlineto{\pgfqpoint{1.583046in}{0.739656in}}%
\pgfpathlineto{\pgfqpoint{1.582193in}{0.739656in}}%
\pgfpathlineto{\pgfqpoint{1.581340in}{0.739656in}}%
\pgfpathlineto{\pgfqpoint{1.580487in}{0.739656in}}%
\pgfpathlineto{\pgfqpoint{1.579634in}{0.739656in}}%
\pgfpathlineto{\pgfqpoint{1.578781in}{0.739656in}}%
\pgfpathlineto{\pgfqpoint{1.577928in}{0.739656in}}%
\pgfpathlineto{\pgfqpoint{1.577075in}{0.739656in}}%
\pgfpathlineto{\pgfqpoint{1.576223in}{0.739656in}}%
\pgfpathlineto{\pgfqpoint{1.575370in}{0.739656in}}%
\pgfpathlineto{\pgfqpoint{1.574517in}{0.739656in}}%
\pgfpathlineto{\pgfqpoint{1.573664in}{0.739656in}}%
\pgfpathlineto{\pgfqpoint{1.572811in}{0.739656in}}%
\pgfpathlineto{\pgfqpoint{1.571958in}{0.739656in}}%
\pgfpathlineto{\pgfqpoint{1.571105in}{0.739656in}}%
\pgfpathlineto{\pgfqpoint{1.570252in}{0.739656in}}%
\pgfpathlineto{\pgfqpoint{1.569399in}{0.739656in}}%
\pgfpathlineto{\pgfqpoint{1.568547in}{0.739656in}}%
\pgfpathlineto{\pgfqpoint{1.567694in}{0.739656in}}%
\pgfpathlineto{\pgfqpoint{1.566841in}{0.739656in}}%
\pgfpathlineto{\pgfqpoint{1.565988in}{0.739656in}}%
\pgfpathlineto{\pgfqpoint{1.565135in}{0.739656in}}%
\pgfpathlineto{\pgfqpoint{1.564282in}{0.739656in}}%
\pgfpathlineto{\pgfqpoint{1.563429in}{0.739656in}}%
\pgfpathlineto{\pgfqpoint{1.562576in}{0.739656in}}%
\pgfpathlineto{\pgfqpoint{1.561724in}{0.739656in}}%
\pgfpathlineto{\pgfqpoint{1.560871in}{0.739656in}}%
\pgfpathlineto{\pgfqpoint{1.560018in}{0.739656in}}%
\pgfpathlineto{\pgfqpoint{1.559165in}{0.739656in}}%
\pgfpathlineto{\pgfqpoint{1.558312in}{0.739656in}}%
\pgfpathlineto{\pgfqpoint{1.557459in}{0.739656in}}%
\pgfpathlineto{\pgfqpoint{1.556606in}{0.739656in}}%
\pgfpathlineto{\pgfqpoint{1.555753in}{0.739656in}}%
\pgfpathlineto{\pgfqpoint{1.554901in}{0.739656in}}%
\pgfpathlineto{\pgfqpoint{1.554048in}{0.739656in}}%
\pgfpathlineto{\pgfqpoint{1.553195in}{0.739656in}}%
\pgfpathlineto{\pgfqpoint{1.552342in}{0.739656in}}%
\pgfpathlineto{\pgfqpoint{1.551489in}{0.739656in}}%
\pgfpathlineto{\pgfqpoint{1.550636in}{0.739656in}}%
\pgfpathlineto{\pgfqpoint{1.549783in}{0.739656in}}%
\pgfpathlineto{\pgfqpoint{1.548930in}{0.739656in}}%
\pgfpathlineto{\pgfqpoint{1.548078in}{0.739656in}}%
\pgfpathlineto{\pgfqpoint{1.547225in}{0.739656in}}%
\pgfpathlineto{\pgfqpoint{1.546372in}{0.739656in}}%
\pgfpathlineto{\pgfqpoint{1.545519in}{0.739656in}}%
\pgfpathlineto{\pgfqpoint{1.544666in}{0.739656in}}%
\pgfpathlineto{\pgfqpoint{1.543813in}{0.739656in}}%
\pgfpathlineto{\pgfqpoint{1.542960in}{0.739656in}}%
\pgfpathlineto{\pgfqpoint{1.542107in}{0.739656in}}%
\pgfpathlineto{\pgfqpoint{1.541255in}{0.739656in}}%
\pgfpathlineto{\pgfqpoint{1.540402in}{0.739656in}}%
\pgfpathlineto{\pgfqpoint{1.539549in}{0.739656in}}%
\pgfpathlineto{\pgfqpoint{1.538696in}{0.739656in}}%
\pgfpathlineto{\pgfqpoint{1.537843in}{0.739656in}}%
\pgfpathlineto{\pgfqpoint{1.536990in}{0.739656in}}%
\pgfpathlineto{\pgfqpoint{1.536137in}{0.739656in}}%
\pgfpathlineto{\pgfqpoint{1.535284in}{0.739656in}}%
\pgfpathlineto{\pgfqpoint{1.534431in}{0.739656in}}%
\pgfpathlineto{\pgfqpoint{1.533579in}{0.739656in}}%
\pgfpathlineto{\pgfqpoint{1.532726in}{0.739656in}}%
\pgfpathlineto{\pgfqpoint{1.531873in}{0.739656in}}%
\pgfpathlineto{\pgfqpoint{1.531020in}{0.739656in}}%
\pgfpathlineto{\pgfqpoint{1.530167in}{0.739656in}}%
\pgfpathlineto{\pgfqpoint{1.529314in}{0.739656in}}%
\pgfpathlineto{\pgfqpoint{1.528461in}{0.739656in}}%
\pgfpathlineto{\pgfqpoint{1.527608in}{0.739656in}}%
\pgfpathlineto{\pgfqpoint{1.526756in}{0.739656in}}%
\pgfpathlineto{\pgfqpoint{1.525903in}{0.739656in}}%
\pgfpathlineto{\pgfqpoint{1.525050in}{0.739656in}}%
\pgfpathlineto{\pgfqpoint{1.524197in}{0.739656in}}%
\pgfpathlineto{\pgfqpoint{1.523344in}{0.739656in}}%
\pgfpathlineto{\pgfqpoint{1.522491in}{0.739656in}}%
\pgfpathlineto{\pgfqpoint{1.521638in}{0.739656in}}%
\pgfpathlineto{\pgfqpoint{1.520785in}{0.739656in}}%
\pgfpathlineto{\pgfqpoint{1.519933in}{0.739656in}}%
\pgfpathlineto{\pgfqpoint{1.519080in}{0.739656in}}%
\pgfpathlineto{\pgfqpoint{1.518227in}{0.739656in}}%
\pgfpathlineto{\pgfqpoint{1.517374in}{0.739656in}}%
\pgfpathlineto{\pgfqpoint{1.516521in}{0.739656in}}%
\pgfpathlineto{\pgfqpoint{1.515668in}{0.739656in}}%
\pgfpathlineto{\pgfqpoint{1.514815in}{0.739656in}}%
\pgfpathlineto{\pgfqpoint{1.513962in}{0.739656in}}%
\pgfpathlineto{\pgfqpoint{1.513110in}{0.739656in}}%
\pgfpathlineto{\pgfqpoint{1.512257in}{0.739656in}}%
\pgfpathlineto{\pgfqpoint{1.511404in}{0.739656in}}%
\pgfpathlineto{\pgfqpoint{1.510551in}{0.739656in}}%
\pgfpathlineto{\pgfqpoint{1.509698in}{0.739656in}}%
\pgfpathlineto{\pgfqpoint{1.508845in}{0.739656in}}%
\pgfpathlineto{\pgfqpoint{1.507992in}{0.739656in}}%
\pgfpathlineto{\pgfqpoint{1.507139in}{0.739656in}}%
\pgfpathlineto{\pgfqpoint{1.506286in}{0.739656in}}%
\pgfpathlineto{\pgfqpoint{1.505434in}{0.739656in}}%
\pgfpathlineto{\pgfqpoint{1.504581in}{0.739656in}}%
\pgfpathlineto{\pgfqpoint{1.503728in}{0.739656in}}%
\pgfpathlineto{\pgfqpoint{1.502875in}{0.739656in}}%
\pgfpathlineto{\pgfqpoint{1.502022in}{0.739656in}}%
\pgfpathlineto{\pgfqpoint{1.501169in}{0.739656in}}%
\pgfpathlineto{\pgfqpoint{1.500316in}{0.739656in}}%
\pgfpathlineto{\pgfqpoint{1.499463in}{0.739656in}}%
\pgfpathlineto{\pgfqpoint{1.498611in}{0.739656in}}%
\pgfpathlineto{\pgfqpoint{1.497758in}{0.739656in}}%
\pgfpathlineto{\pgfqpoint{1.496905in}{0.739656in}}%
\pgfpathlineto{\pgfqpoint{1.496052in}{0.739656in}}%
\pgfpathlineto{\pgfqpoint{1.495199in}{0.739656in}}%
\pgfpathlineto{\pgfqpoint{1.494346in}{0.739656in}}%
\pgfpathlineto{\pgfqpoint{1.493493in}{0.739656in}}%
\pgfpathlineto{\pgfqpoint{1.492640in}{0.739656in}}%
\pgfpathlineto{\pgfqpoint{1.491788in}{0.739656in}}%
\pgfpathlineto{\pgfqpoint{1.490935in}{0.739656in}}%
\pgfpathlineto{\pgfqpoint{1.490082in}{0.739656in}}%
\pgfpathlineto{\pgfqpoint{1.489229in}{0.739656in}}%
\pgfpathlineto{\pgfqpoint{1.488376in}{0.739656in}}%
\pgfpathlineto{\pgfqpoint{1.487523in}{0.739656in}}%
\pgfpathlineto{\pgfqpoint{1.486670in}{0.739656in}}%
\pgfpathlineto{\pgfqpoint{1.485817in}{0.739656in}}%
\pgfpathlineto{\pgfqpoint{1.484965in}{0.739656in}}%
\pgfpathlineto{\pgfqpoint{1.484112in}{0.739656in}}%
\pgfpathlineto{\pgfqpoint{1.483259in}{0.739656in}}%
\pgfpathlineto{\pgfqpoint{1.482406in}{0.739656in}}%
\pgfpathlineto{\pgfqpoint{1.481553in}{0.739656in}}%
\pgfpathlineto{\pgfqpoint{1.480700in}{0.739656in}}%
\pgfpathlineto{\pgfqpoint{1.479847in}{0.739656in}}%
\pgfpathlineto{\pgfqpoint{1.478994in}{0.739656in}}%
\pgfpathlineto{\pgfqpoint{1.478141in}{0.739656in}}%
\pgfpathlineto{\pgfqpoint{1.477289in}{0.739656in}}%
\pgfpathlineto{\pgfqpoint{1.476436in}{0.739656in}}%
\pgfpathlineto{\pgfqpoint{1.475583in}{0.739656in}}%
\pgfpathlineto{\pgfqpoint{1.474730in}{0.739656in}}%
\pgfpathlineto{\pgfqpoint{1.473877in}{0.739656in}}%
\pgfpathlineto{\pgfqpoint{1.473024in}{0.739656in}}%
\pgfpathlineto{\pgfqpoint{1.472171in}{0.739656in}}%
\pgfpathlineto{\pgfqpoint{1.471318in}{0.739656in}}%
\pgfpathlineto{\pgfqpoint{1.470466in}{0.739656in}}%
\pgfpathlineto{\pgfqpoint{1.469613in}{0.739656in}}%
\pgfpathlineto{\pgfqpoint{1.468760in}{0.739656in}}%
\pgfpathlineto{\pgfqpoint{1.467907in}{0.739656in}}%
\pgfpathlineto{\pgfqpoint{1.467054in}{0.739656in}}%
\pgfpathlineto{\pgfqpoint{1.466201in}{0.739656in}}%
\pgfpathlineto{\pgfqpoint{1.465348in}{0.739656in}}%
\pgfpathlineto{\pgfqpoint{1.464495in}{0.739656in}}%
\pgfpathlineto{\pgfqpoint{1.463643in}{0.739656in}}%
\pgfpathlineto{\pgfqpoint{1.462790in}{0.739656in}}%
\pgfpathlineto{\pgfqpoint{1.461937in}{0.739656in}}%
\pgfpathlineto{\pgfqpoint{1.461084in}{0.739656in}}%
\pgfpathlineto{\pgfqpoint{1.460231in}{0.739656in}}%
\pgfpathlineto{\pgfqpoint{1.459378in}{0.739656in}}%
\pgfpathlineto{\pgfqpoint{1.458525in}{0.739656in}}%
\pgfpathlineto{\pgfqpoint{1.457672in}{0.739656in}}%
\pgfpathlineto{\pgfqpoint{1.456820in}{0.739656in}}%
\pgfpathlineto{\pgfqpoint{1.455967in}{0.739656in}}%
\pgfpathlineto{\pgfqpoint{1.455114in}{0.739656in}}%
\pgfpathlineto{\pgfqpoint{1.454261in}{0.739656in}}%
\pgfpathlineto{\pgfqpoint{1.453408in}{0.739656in}}%
\pgfpathlineto{\pgfqpoint{1.452555in}{0.739656in}}%
\pgfpathlineto{\pgfqpoint{1.451702in}{0.739656in}}%
\pgfpathlineto{\pgfqpoint{1.450849in}{0.739656in}}%
\pgfpathlineto{\pgfqpoint{1.449996in}{0.739656in}}%
\pgfpathlineto{\pgfqpoint{1.449144in}{0.739656in}}%
\pgfpathlineto{\pgfqpoint{1.448291in}{0.739656in}}%
\pgfpathlineto{\pgfqpoint{1.447438in}{0.739656in}}%
\pgfpathlineto{\pgfqpoint{1.446585in}{0.739656in}}%
\pgfpathlineto{\pgfqpoint{1.445732in}{0.739656in}}%
\pgfpathlineto{\pgfqpoint{1.444879in}{0.739656in}}%
\pgfpathlineto{\pgfqpoint{1.444026in}{0.739656in}}%
\pgfpathlineto{\pgfqpoint{1.443173in}{0.739656in}}%
\pgfpathlineto{\pgfqpoint{1.442321in}{0.739656in}}%
\pgfpathlineto{\pgfqpoint{1.441468in}{0.739656in}}%
\pgfpathlineto{\pgfqpoint{1.440615in}{0.739656in}}%
\pgfpathlineto{\pgfqpoint{1.439762in}{0.739656in}}%
\pgfpathlineto{\pgfqpoint{1.438909in}{0.739656in}}%
\pgfpathlineto{\pgfqpoint{1.438056in}{0.739656in}}%
\pgfpathlineto{\pgfqpoint{1.437203in}{0.739656in}}%
\pgfpathlineto{\pgfqpoint{1.436350in}{0.739656in}}%
\pgfpathlineto{\pgfqpoint{1.435498in}{0.739656in}}%
\pgfpathlineto{\pgfqpoint{1.434645in}{0.739656in}}%
\pgfpathlineto{\pgfqpoint{1.433792in}{0.739656in}}%
\pgfpathlineto{\pgfqpoint{1.432939in}{0.739656in}}%
\pgfpathlineto{\pgfqpoint{1.432086in}{0.739656in}}%
\pgfpathlineto{\pgfqpoint{1.431233in}{0.739656in}}%
\pgfpathlineto{\pgfqpoint{1.430380in}{0.739656in}}%
\pgfpathlineto{\pgfqpoint{1.429527in}{0.739656in}}%
\pgfpathlineto{\pgfqpoint{1.428675in}{0.739656in}}%
\pgfpathlineto{\pgfqpoint{1.427822in}{0.739656in}}%
\pgfpathlineto{\pgfqpoint{1.426969in}{0.739656in}}%
\pgfpathlineto{\pgfqpoint{1.426116in}{0.739656in}}%
\pgfpathlineto{\pgfqpoint{1.425263in}{0.739656in}}%
\pgfpathlineto{\pgfqpoint{1.424410in}{0.739656in}}%
\pgfpathlineto{\pgfqpoint{1.423557in}{0.739656in}}%
\pgfpathlineto{\pgfqpoint{1.422704in}{0.739656in}}%
\pgfpathlineto{\pgfqpoint{1.421852in}{0.739656in}}%
\pgfpathlineto{\pgfqpoint{1.420999in}{0.739656in}}%
\pgfpathlineto{\pgfqpoint{1.420146in}{0.739656in}}%
\pgfpathlineto{\pgfqpoint{1.419293in}{0.739656in}}%
\pgfpathlineto{\pgfqpoint{1.418440in}{0.739656in}}%
\pgfpathlineto{\pgfqpoint{1.417587in}{0.739656in}}%
\pgfpathlineto{\pgfqpoint{1.416734in}{0.739656in}}%
\pgfpathlineto{\pgfqpoint{1.415881in}{0.739656in}}%
\pgfpathlineto{\pgfqpoint{1.415028in}{0.739656in}}%
\pgfpathlineto{\pgfqpoint{1.414176in}{0.739656in}}%
\pgfpathlineto{\pgfqpoint{1.413323in}{0.739656in}}%
\pgfpathlineto{\pgfqpoint{1.412470in}{0.739656in}}%
\pgfpathlineto{\pgfqpoint{1.411617in}{0.739656in}}%
\pgfpathlineto{\pgfqpoint{1.410764in}{0.739656in}}%
\pgfpathlineto{\pgfqpoint{1.409911in}{0.739656in}}%
\pgfpathlineto{\pgfqpoint{1.409058in}{0.739656in}}%
\pgfpathlineto{\pgfqpoint{1.408205in}{0.739656in}}%
\pgfpathlineto{\pgfqpoint{1.407353in}{0.739656in}}%
\pgfpathlineto{\pgfqpoint{1.406500in}{0.739656in}}%
\pgfpathlineto{\pgfqpoint{1.405647in}{0.739656in}}%
\pgfpathlineto{\pgfqpoint{1.404794in}{0.739656in}}%
\pgfpathlineto{\pgfqpoint{1.403941in}{0.739656in}}%
\pgfpathlineto{\pgfqpoint{1.403088in}{0.739656in}}%
\pgfpathlineto{\pgfqpoint{1.402235in}{0.739656in}}%
\pgfpathlineto{\pgfqpoint{1.401382in}{0.739656in}}%
\pgfpathlineto{\pgfqpoint{1.400530in}{0.739656in}}%
\pgfpathlineto{\pgfqpoint{1.399677in}{0.739656in}}%
\pgfpathlineto{\pgfqpoint{1.398824in}{0.739656in}}%
\pgfpathlineto{\pgfqpoint{1.397971in}{0.739656in}}%
\pgfpathlineto{\pgfqpoint{1.397118in}{0.739656in}}%
\pgfpathlineto{\pgfqpoint{1.396265in}{0.739656in}}%
\pgfpathlineto{\pgfqpoint{1.395412in}{0.739656in}}%
\pgfpathlineto{\pgfqpoint{1.394559in}{0.739656in}}%
\pgfpathlineto{\pgfqpoint{1.393707in}{0.739656in}}%
\pgfpathlineto{\pgfqpoint{1.392854in}{0.739656in}}%
\pgfpathlineto{\pgfqpoint{1.392001in}{0.739656in}}%
\pgfpathlineto{\pgfqpoint{1.391148in}{0.739656in}}%
\pgfpathlineto{\pgfqpoint{1.390295in}{0.739656in}}%
\pgfpathlineto{\pgfqpoint{1.389442in}{0.739656in}}%
\pgfpathlineto{\pgfqpoint{1.388589in}{0.739656in}}%
\pgfpathlineto{\pgfqpoint{1.387736in}{0.739656in}}%
\pgfpathlineto{\pgfqpoint{1.386883in}{0.739656in}}%
\pgfpathlineto{\pgfqpoint{1.386031in}{0.739656in}}%
\pgfpathlineto{\pgfqpoint{1.385178in}{0.739656in}}%
\pgfpathlineto{\pgfqpoint{1.384325in}{0.739656in}}%
\pgfpathlineto{\pgfqpoint{1.383472in}{0.739656in}}%
\pgfpathlineto{\pgfqpoint{1.382619in}{0.739656in}}%
\pgfpathlineto{\pgfqpoint{1.381766in}{0.739656in}}%
\pgfpathlineto{\pgfqpoint{1.380913in}{0.739656in}}%
\pgfpathlineto{\pgfqpoint{1.380060in}{0.739656in}}%
\pgfpathlineto{\pgfqpoint{1.379208in}{0.739656in}}%
\pgfpathlineto{\pgfqpoint{1.378355in}{0.739656in}}%
\pgfpathlineto{\pgfqpoint{1.377502in}{0.739656in}}%
\pgfpathlineto{\pgfqpoint{1.376649in}{0.739656in}}%
\pgfpathlineto{\pgfqpoint{1.375796in}{0.739656in}}%
\pgfpathlineto{\pgfqpoint{1.374943in}{0.739656in}}%
\pgfpathlineto{\pgfqpoint{1.374090in}{0.739656in}}%
\pgfpathlineto{\pgfqpoint{1.373237in}{0.739656in}}%
\pgfpathlineto{\pgfqpoint{1.372385in}{0.739656in}}%
\pgfpathlineto{\pgfqpoint{1.371532in}{0.739656in}}%
\pgfpathlineto{\pgfqpoint{1.370679in}{0.739656in}}%
\pgfpathlineto{\pgfqpoint{1.369826in}{0.739656in}}%
\pgfpathlineto{\pgfqpoint{1.368973in}{0.739656in}}%
\pgfpathlineto{\pgfqpoint{1.368120in}{0.739656in}}%
\pgfpathlineto{\pgfqpoint{1.367267in}{0.739656in}}%
\pgfpathlineto{\pgfqpoint{1.366414in}{0.739656in}}%
\pgfpathlineto{\pgfqpoint{1.365562in}{0.739656in}}%
\pgfpathlineto{\pgfqpoint{1.364709in}{0.739656in}}%
\pgfpathlineto{\pgfqpoint{1.363856in}{0.739656in}}%
\pgfpathlineto{\pgfqpoint{1.363003in}{0.739656in}}%
\pgfpathlineto{\pgfqpoint{1.362150in}{0.739656in}}%
\pgfpathlineto{\pgfqpoint{1.361297in}{0.739656in}}%
\pgfpathlineto{\pgfqpoint{1.360444in}{0.739656in}}%
\pgfpathlineto{\pgfqpoint{1.359591in}{0.739656in}}%
\pgfpathlineto{\pgfqpoint{1.358738in}{0.739656in}}%
\pgfpathlineto{\pgfqpoint{1.357886in}{0.739656in}}%
\pgfpathlineto{\pgfqpoint{1.357033in}{0.739656in}}%
\pgfpathlineto{\pgfqpoint{1.356180in}{0.739656in}}%
\pgfpathlineto{\pgfqpoint{1.355327in}{0.739656in}}%
\pgfpathlineto{\pgfqpoint{1.354474in}{0.739656in}}%
\pgfpathlineto{\pgfqpoint{1.353621in}{0.739656in}}%
\pgfpathlineto{\pgfqpoint{1.352768in}{0.739656in}}%
\pgfpathlineto{\pgfqpoint{1.351915in}{0.739656in}}%
\pgfpathlineto{\pgfqpoint{1.351063in}{0.739656in}}%
\pgfpathlineto{\pgfqpoint{1.350210in}{0.739656in}}%
\pgfpathlineto{\pgfqpoint{1.349357in}{0.739656in}}%
\pgfpathlineto{\pgfqpoint{1.348504in}{0.739656in}}%
\pgfpathlineto{\pgfqpoint{1.347651in}{0.739656in}}%
\pgfpathlineto{\pgfqpoint{1.346798in}{0.739656in}}%
\pgfpathlineto{\pgfqpoint{1.345945in}{0.739656in}}%
\pgfpathlineto{\pgfqpoint{1.345092in}{0.739656in}}%
\pgfpathlineto{\pgfqpoint{1.344240in}{0.739656in}}%
\pgfpathlineto{\pgfqpoint{1.343387in}{0.739656in}}%
\pgfpathlineto{\pgfqpoint{1.342534in}{0.739656in}}%
\pgfpathlineto{\pgfqpoint{1.341681in}{0.739656in}}%
\pgfpathlineto{\pgfqpoint{1.340828in}{0.739656in}}%
\pgfpathlineto{\pgfqpoint{1.339975in}{0.739656in}}%
\pgfpathlineto{\pgfqpoint{1.339122in}{0.739656in}}%
\pgfpathlineto{\pgfqpoint{1.338269in}{0.739656in}}%
\pgfpathlineto{\pgfqpoint{1.337417in}{0.739656in}}%
\pgfpathlineto{\pgfqpoint{1.336564in}{0.739656in}}%
\pgfpathlineto{\pgfqpoint{1.335711in}{0.739656in}}%
\pgfpathlineto{\pgfqpoint{1.334858in}{0.739656in}}%
\pgfpathlineto{\pgfqpoint{1.334005in}{0.739656in}}%
\pgfpathlineto{\pgfqpoint{1.333152in}{0.739656in}}%
\pgfpathlineto{\pgfqpoint{1.332299in}{0.739656in}}%
\pgfpathlineto{\pgfqpoint{1.331446in}{0.739656in}}%
\pgfpathlineto{\pgfqpoint{1.330593in}{0.739656in}}%
\pgfpathlineto{\pgfqpoint{1.329741in}{0.739656in}}%
\pgfpathlineto{\pgfqpoint{1.328888in}{0.739656in}}%
\pgfpathlineto{\pgfqpoint{1.328035in}{0.739656in}}%
\pgfpathlineto{\pgfqpoint{1.327182in}{0.739656in}}%
\pgfpathlineto{\pgfqpoint{1.326329in}{0.739656in}}%
\pgfpathlineto{\pgfqpoint{1.325476in}{0.739656in}}%
\pgfpathlineto{\pgfqpoint{1.324623in}{0.739656in}}%
\pgfpathlineto{\pgfqpoint{1.323770in}{0.739656in}}%
\pgfpathlineto{\pgfqpoint{1.322918in}{0.739656in}}%
\pgfpathlineto{\pgfqpoint{1.322065in}{0.739656in}}%
\pgfpathlineto{\pgfqpoint{1.321212in}{0.739656in}}%
\pgfpathlineto{\pgfqpoint{1.320359in}{0.739656in}}%
\pgfpathlineto{\pgfqpoint{1.319506in}{0.739656in}}%
\pgfpathlineto{\pgfqpoint{1.318653in}{0.739656in}}%
\pgfpathlineto{\pgfqpoint{1.317800in}{0.739656in}}%
\pgfpathlineto{\pgfqpoint{1.316947in}{0.739656in}}%
\pgfpathlineto{\pgfqpoint{1.316095in}{0.739656in}}%
\pgfpathlineto{\pgfqpoint{1.315242in}{0.739656in}}%
\pgfpathlineto{\pgfqpoint{1.314389in}{0.739656in}}%
\pgfpathlineto{\pgfqpoint{1.313536in}{0.739656in}}%
\pgfpathlineto{\pgfqpoint{1.312683in}{0.739656in}}%
\pgfpathlineto{\pgfqpoint{1.311830in}{0.739656in}}%
\pgfpathlineto{\pgfqpoint{1.310977in}{0.739656in}}%
\pgfpathlineto{\pgfqpoint{1.310124in}{0.739656in}}%
\pgfpathlineto{\pgfqpoint{1.309272in}{0.739656in}}%
\pgfpathlineto{\pgfqpoint{1.308419in}{0.739656in}}%
\pgfpathlineto{\pgfqpoint{1.307566in}{0.739656in}}%
\pgfpathlineto{\pgfqpoint{1.306713in}{0.739656in}}%
\pgfpathlineto{\pgfqpoint{1.305860in}{0.739656in}}%
\pgfpathlineto{\pgfqpoint{1.305007in}{0.739656in}}%
\pgfpathlineto{\pgfqpoint{1.304154in}{0.739656in}}%
\pgfpathlineto{\pgfqpoint{1.303301in}{0.739656in}}%
\pgfpathlineto{\pgfqpoint{1.302449in}{0.739656in}}%
\pgfpathlineto{\pgfqpoint{1.301596in}{0.739656in}}%
\pgfpathlineto{\pgfqpoint{1.300743in}{0.739656in}}%
\pgfpathlineto{\pgfqpoint{1.299890in}{0.739656in}}%
\pgfpathlineto{\pgfqpoint{1.299037in}{0.739656in}}%
\pgfpathlineto{\pgfqpoint{1.298184in}{0.739656in}}%
\pgfpathlineto{\pgfqpoint{1.297331in}{0.739656in}}%
\pgfpathlineto{\pgfqpoint{1.296478in}{0.739656in}}%
\pgfpathlineto{\pgfqpoint{1.295625in}{0.739656in}}%
\pgfpathlineto{\pgfqpoint{1.294773in}{0.739656in}}%
\pgfpathlineto{\pgfqpoint{1.293920in}{0.739656in}}%
\pgfpathlineto{\pgfqpoint{1.293067in}{0.739656in}}%
\pgfpathlineto{\pgfqpoint{1.292214in}{0.739656in}}%
\pgfpathlineto{\pgfqpoint{1.291361in}{0.739656in}}%
\pgfpathlineto{\pgfqpoint{1.290508in}{0.739656in}}%
\pgfpathlineto{\pgfqpoint{1.289655in}{0.739656in}}%
\pgfpathlineto{\pgfqpoint{1.288802in}{0.739656in}}%
\pgfpathlineto{\pgfqpoint{1.287950in}{0.739656in}}%
\pgfpathlineto{\pgfqpoint{1.287097in}{0.739656in}}%
\pgfpathlineto{\pgfqpoint{1.286244in}{0.739656in}}%
\pgfpathlineto{\pgfqpoint{1.285391in}{0.739656in}}%
\pgfpathlineto{\pgfqpoint{1.284538in}{0.739656in}}%
\pgfpathlineto{\pgfqpoint{1.283685in}{0.739656in}}%
\pgfpathlineto{\pgfqpoint{1.282832in}{0.739656in}}%
\pgfpathlineto{\pgfqpoint{1.281979in}{0.739656in}}%
\pgfpathlineto{\pgfqpoint{1.281127in}{0.739656in}}%
\pgfpathlineto{\pgfqpoint{1.280274in}{0.739656in}}%
\pgfpathlineto{\pgfqpoint{1.279421in}{0.739656in}}%
\pgfpathlineto{\pgfqpoint{1.278568in}{0.739656in}}%
\pgfpathlineto{\pgfqpoint{1.277715in}{0.739656in}}%
\pgfpathlineto{\pgfqpoint{1.276862in}{0.739656in}}%
\pgfpathlineto{\pgfqpoint{1.276009in}{0.739656in}}%
\pgfpathlineto{\pgfqpoint{1.275156in}{0.739656in}}%
\pgfpathlineto{\pgfqpoint{1.274304in}{0.739656in}}%
\pgfpathlineto{\pgfqpoint{1.273451in}{0.739656in}}%
\pgfpathlineto{\pgfqpoint{1.272598in}{0.739656in}}%
\pgfpathlineto{\pgfqpoint{1.271745in}{0.739656in}}%
\pgfpathlineto{\pgfqpoint{1.270892in}{0.739656in}}%
\pgfpathlineto{\pgfqpoint{1.270039in}{0.739656in}}%
\pgfpathlineto{\pgfqpoint{1.269186in}{0.739656in}}%
\pgfpathlineto{\pgfqpoint{1.268333in}{0.739656in}}%
\pgfpathlineto{\pgfqpoint{1.267480in}{0.739656in}}%
\pgfpathlineto{\pgfqpoint{1.266628in}{0.739656in}}%
\pgfpathlineto{\pgfqpoint{1.265775in}{0.739656in}}%
\pgfpathlineto{\pgfqpoint{1.264922in}{0.739656in}}%
\pgfpathlineto{\pgfqpoint{1.264069in}{0.739656in}}%
\pgfpathlineto{\pgfqpoint{1.263216in}{0.739656in}}%
\pgfpathlineto{\pgfqpoint{1.262363in}{0.739656in}}%
\pgfpathlineto{\pgfqpoint{1.261510in}{0.739656in}}%
\pgfpathlineto{\pgfqpoint{1.260657in}{0.739656in}}%
\pgfpathlineto{\pgfqpoint{1.259805in}{0.739656in}}%
\pgfpathlineto{\pgfqpoint{1.258952in}{0.739656in}}%
\pgfpathlineto{\pgfqpoint{1.258099in}{0.739656in}}%
\pgfpathlineto{\pgfqpoint{1.257246in}{0.739656in}}%
\pgfpathlineto{\pgfqpoint{1.256393in}{0.739656in}}%
\pgfpathlineto{\pgfqpoint{1.255540in}{0.739656in}}%
\pgfpathlineto{\pgfqpoint{1.254687in}{0.739656in}}%
\pgfpathlineto{\pgfqpoint{1.253834in}{0.739656in}}%
\pgfpathlineto{\pgfqpoint{1.252982in}{0.739656in}}%
\pgfpathlineto{\pgfqpoint{1.252129in}{0.739656in}}%
\pgfpathlineto{\pgfqpoint{1.251276in}{0.739656in}}%
\pgfpathlineto{\pgfqpoint{1.250423in}{0.739656in}}%
\pgfpathlineto{\pgfqpoint{1.249570in}{0.739656in}}%
\pgfpathlineto{\pgfqpoint{1.248717in}{0.739656in}}%
\pgfpathlineto{\pgfqpoint{1.247864in}{0.739656in}}%
\pgfpathlineto{\pgfqpoint{1.247011in}{0.739656in}}%
\pgfpathlineto{\pgfqpoint{1.246159in}{0.739656in}}%
\pgfpathlineto{\pgfqpoint{1.245306in}{0.739656in}}%
\pgfpathlineto{\pgfqpoint{1.244453in}{0.739656in}}%
\pgfpathlineto{\pgfqpoint{1.243600in}{0.739656in}}%
\pgfpathlineto{\pgfqpoint{1.242747in}{0.739656in}}%
\pgfpathlineto{\pgfqpoint{1.241894in}{0.739656in}}%
\pgfpathlineto{\pgfqpoint{1.241041in}{0.739656in}}%
\pgfpathlineto{\pgfqpoint{1.240188in}{0.739656in}}%
\pgfpathlineto{\pgfqpoint{1.239335in}{0.739656in}}%
\pgfpathlineto{\pgfqpoint{1.238483in}{0.739656in}}%
\pgfpathlineto{\pgfqpoint{1.237630in}{0.739656in}}%
\pgfpathlineto{\pgfqpoint{1.236777in}{0.739656in}}%
\pgfpathlineto{\pgfqpoint{1.235924in}{0.739656in}}%
\pgfpathlineto{\pgfqpoint{1.235071in}{0.739656in}}%
\pgfpathlineto{\pgfqpoint{1.234218in}{0.739656in}}%
\pgfpathlineto{\pgfqpoint{1.233365in}{0.739656in}}%
\pgfpathlineto{\pgfqpoint{1.232512in}{0.739656in}}%
\pgfpathlineto{\pgfqpoint{1.231660in}{0.739656in}}%
\pgfpathlineto{\pgfqpoint{1.230807in}{0.739656in}}%
\pgfpathlineto{\pgfqpoint{1.229954in}{0.739656in}}%
\pgfpathlineto{\pgfqpoint{1.229101in}{0.739656in}}%
\pgfpathlineto{\pgfqpoint{1.228248in}{0.739656in}}%
\pgfpathlineto{\pgfqpoint{1.227395in}{0.739656in}}%
\pgfpathlineto{\pgfqpoint{1.226542in}{0.739656in}}%
\pgfpathlineto{\pgfqpoint{1.225689in}{0.739656in}}%
\pgfpathlineto{\pgfqpoint{1.224837in}{0.739656in}}%
\pgfpathlineto{\pgfqpoint{1.223984in}{0.739656in}}%
\pgfpathlineto{\pgfqpoint{1.223131in}{0.739656in}}%
\pgfpathlineto{\pgfqpoint{1.222278in}{0.739656in}}%
\pgfpathlineto{\pgfqpoint{1.221425in}{0.739656in}}%
\pgfpathlineto{\pgfqpoint{1.220572in}{0.739656in}}%
\pgfpathlineto{\pgfqpoint{1.219719in}{0.739656in}}%
\pgfpathlineto{\pgfqpoint{1.218866in}{0.739656in}}%
\pgfpathlineto{\pgfqpoint{1.218014in}{0.739656in}}%
\pgfpathlineto{\pgfqpoint{1.217161in}{0.739656in}}%
\pgfpathlineto{\pgfqpoint{1.216308in}{0.739656in}}%
\pgfpathlineto{\pgfqpoint{1.215455in}{0.739656in}}%
\pgfpathlineto{\pgfqpoint{1.214602in}{0.739656in}}%
\pgfpathlineto{\pgfqpoint{1.213749in}{0.739656in}}%
\pgfpathlineto{\pgfqpoint{1.212896in}{0.739656in}}%
\pgfpathlineto{\pgfqpoint{1.212043in}{0.739656in}}%
\pgfpathlineto{\pgfqpoint{1.211191in}{0.739656in}}%
\pgfpathlineto{\pgfqpoint{1.210338in}{0.739656in}}%
\pgfpathlineto{\pgfqpoint{1.209485in}{0.739656in}}%
\pgfpathlineto{\pgfqpoint{1.208632in}{0.739656in}}%
\pgfpathlineto{\pgfqpoint{1.207779in}{0.739656in}}%
\pgfpathlineto{\pgfqpoint{1.206926in}{0.739656in}}%
\pgfpathlineto{\pgfqpoint{1.206073in}{0.739656in}}%
\pgfpathlineto{\pgfqpoint{1.205220in}{0.739656in}}%
\pgfpathlineto{\pgfqpoint{1.204367in}{0.739656in}}%
\pgfpathlineto{\pgfqpoint{1.203515in}{0.739656in}}%
\pgfpathlineto{\pgfqpoint{1.202662in}{0.739656in}}%
\pgfpathlineto{\pgfqpoint{1.201809in}{0.739656in}}%
\pgfpathlineto{\pgfqpoint{1.200956in}{0.739656in}}%
\pgfpathlineto{\pgfqpoint{1.200103in}{0.739656in}}%
\pgfpathlineto{\pgfqpoint{1.199250in}{0.739656in}}%
\pgfpathlineto{\pgfqpoint{1.198397in}{0.739656in}}%
\pgfpathlineto{\pgfqpoint{1.197544in}{0.739656in}}%
\pgfpathlineto{\pgfqpoint{1.196692in}{0.739656in}}%
\pgfpathlineto{\pgfqpoint{1.195839in}{0.739656in}}%
\pgfpathlineto{\pgfqpoint{1.194986in}{0.739656in}}%
\pgfpathlineto{\pgfqpoint{1.194133in}{0.739656in}}%
\pgfpathlineto{\pgfqpoint{1.193280in}{0.739656in}}%
\pgfpathlineto{\pgfqpoint{1.192427in}{0.739656in}}%
\pgfpathlineto{\pgfqpoint{1.191574in}{0.739656in}}%
\pgfpathlineto{\pgfqpoint{1.190721in}{0.739656in}}%
\pgfpathlineto{\pgfqpoint{1.189869in}{0.739656in}}%
\pgfpathlineto{\pgfqpoint{1.189016in}{0.739656in}}%
\pgfpathlineto{\pgfqpoint{1.188163in}{0.739656in}}%
\pgfpathlineto{\pgfqpoint{1.187310in}{0.739656in}}%
\pgfpathlineto{\pgfqpoint{1.186457in}{0.739656in}}%
\pgfpathlineto{\pgfqpoint{1.185604in}{0.739656in}}%
\pgfpathlineto{\pgfqpoint{1.184751in}{0.739656in}}%
\pgfpathlineto{\pgfqpoint{1.183898in}{0.739656in}}%
\pgfpathlineto{\pgfqpoint{1.183046in}{0.739656in}}%
\pgfpathlineto{\pgfqpoint{1.182193in}{0.739656in}}%
\pgfpathlineto{\pgfqpoint{1.181340in}{0.739656in}}%
\pgfpathlineto{\pgfqpoint{1.180487in}{0.739656in}}%
\pgfpathlineto{\pgfqpoint{1.179634in}{0.739656in}}%
\pgfpathlineto{\pgfqpoint{1.178781in}{0.739656in}}%
\pgfpathlineto{\pgfqpoint{1.177928in}{0.739656in}}%
\pgfpathlineto{\pgfqpoint{1.177075in}{0.739656in}}%
\pgfpathlineto{\pgfqpoint{1.176222in}{0.739656in}}%
\pgfpathlineto{\pgfqpoint{1.175370in}{0.739656in}}%
\pgfpathlineto{\pgfqpoint{1.174517in}{0.739656in}}%
\pgfpathlineto{\pgfqpoint{1.173664in}{0.739656in}}%
\pgfpathlineto{\pgfqpoint{1.172811in}{0.739656in}}%
\pgfpathlineto{\pgfqpoint{1.171958in}{0.739656in}}%
\pgfpathlineto{\pgfqpoint{1.171105in}{0.739656in}}%
\pgfpathlineto{\pgfqpoint{1.170252in}{0.739656in}}%
\pgfpathlineto{\pgfqpoint{1.169399in}{0.739656in}}%
\pgfpathlineto{\pgfqpoint{1.168547in}{0.739656in}}%
\pgfpathlineto{\pgfqpoint{1.167694in}{0.739656in}}%
\pgfpathlineto{\pgfqpoint{1.166841in}{0.739656in}}%
\pgfpathlineto{\pgfqpoint{1.165988in}{0.739656in}}%
\pgfpathlineto{\pgfqpoint{1.165135in}{0.739656in}}%
\pgfpathlineto{\pgfqpoint{1.164282in}{0.739656in}}%
\pgfpathlineto{\pgfqpoint{1.163429in}{0.739656in}}%
\pgfpathlineto{\pgfqpoint{1.162576in}{0.739656in}}%
\pgfpathlineto{\pgfqpoint{1.161724in}{0.739656in}}%
\pgfpathlineto{\pgfqpoint{1.160871in}{0.739656in}}%
\pgfpathlineto{\pgfqpoint{1.160018in}{0.739656in}}%
\pgfpathlineto{\pgfqpoint{1.159165in}{0.739656in}}%
\pgfpathlineto{\pgfqpoint{1.158312in}{0.739656in}}%
\pgfpathlineto{\pgfqpoint{1.157459in}{0.739656in}}%
\pgfpathlineto{\pgfqpoint{1.156606in}{0.739656in}}%
\pgfpathlineto{\pgfqpoint{1.155753in}{0.739656in}}%
\pgfpathlineto{\pgfqpoint{1.154901in}{0.739656in}}%
\pgfpathlineto{\pgfqpoint{1.154048in}{0.739656in}}%
\pgfpathlineto{\pgfqpoint{1.153195in}{0.739656in}}%
\pgfpathlineto{\pgfqpoint{1.152342in}{0.739656in}}%
\pgfpathlineto{\pgfqpoint{1.151489in}{0.739656in}}%
\pgfpathlineto{\pgfqpoint{1.150636in}{0.739656in}}%
\pgfpathlineto{\pgfqpoint{1.149783in}{0.739656in}}%
\pgfpathlineto{\pgfqpoint{1.148930in}{0.739656in}}%
\pgfpathlineto{\pgfqpoint{1.148077in}{0.739656in}}%
\pgfpathlineto{\pgfqpoint{1.147225in}{0.739656in}}%
\pgfpathlineto{\pgfqpoint{1.146372in}{0.739656in}}%
\pgfpathlineto{\pgfqpoint{1.145519in}{0.739656in}}%
\pgfpathlineto{\pgfqpoint{1.144666in}{0.739656in}}%
\pgfpathlineto{\pgfqpoint{1.143813in}{0.739656in}}%
\pgfpathlineto{\pgfqpoint{1.142960in}{0.739656in}}%
\pgfpathlineto{\pgfqpoint{1.142107in}{0.739656in}}%
\pgfpathlineto{\pgfqpoint{1.141254in}{0.739656in}}%
\pgfpathlineto{\pgfqpoint{1.140402in}{0.739656in}}%
\pgfpathlineto{\pgfqpoint{1.139549in}{0.739656in}}%
\pgfpathlineto{\pgfqpoint{1.138696in}{0.739656in}}%
\pgfpathlineto{\pgfqpoint{1.137843in}{0.739656in}}%
\pgfpathlineto{\pgfqpoint{1.136990in}{0.739656in}}%
\pgfpathlineto{\pgfqpoint{1.136137in}{0.739656in}}%
\pgfpathlineto{\pgfqpoint{1.135284in}{0.739656in}}%
\pgfpathlineto{\pgfqpoint{1.134431in}{0.739656in}}%
\pgfpathlineto{\pgfqpoint{1.133579in}{0.739656in}}%
\pgfpathlineto{\pgfqpoint{1.132726in}{0.739656in}}%
\pgfpathlineto{\pgfqpoint{1.131873in}{0.739656in}}%
\pgfpathlineto{\pgfqpoint{1.131020in}{0.739656in}}%
\pgfpathlineto{\pgfqpoint{1.130167in}{0.739656in}}%
\pgfpathlineto{\pgfqpoint{1.129314in}{0.739656in}}%
\pgfpathlineto{\pgfqpoint{1.128461in}{0.739656in}}%
\pgfpathlineto{\pgfqpoint{1.127608in}{0.739656in}}%
\pgfpathlineto{\pgfqpoint{1.126756in}{0.739656in}}%
\pgfpathlineto{\pgfqpoint{1.125903in}{0.739656in}}%
\pgfpathlineto{\pgfqpoint{1.125050in}{0.739656in}}%
\pgfpathlineto{\pgfqpoint{1.124197in}{0.739656in}}%
\pgfpathlineto{\pgfqpoint{1.123344in}{0.739656in}}%
\pgfpathlineto{\pgfqpoint{1.122491in}{0.739656in}}%
\pgfpathlineto{\pgfqpoint{1.121638in}{0.739656in}}%
\pgfpathlineto{\pgfqpoint{1.120785in}{0.739656in}}%
\pgfpathlineto{\pgfqpoint{1.119932in}{0.739656in}}%
\pgfpathlineto{\pgfqpoint{1.119080in}{0.739656in}}%
\pgfpathlineto{\pgfqpoint{1.118227in}{0.739656in}}%
\pgfpathlineto{\pgfqpoint{1.117374in}{0.739656in}}%
\pgfpathlineto{\pgfqpoint{1.116521in}{0.739656in}}%
\pgfpathlineto{\pgfqpoint{1.115668in}{0.739656in}}%
\pgfpathlineto{\pgfqpoint{1.114815in}{0.739656in}}%
\pgfpathlineto{\pgfqpoint{1.113962in}{0.739656in}}%
\pgfpathlineto{\pgfqpoint{1.113109in}{0.739656in}}%
\pgfpathlineto{\pgfqpoint{1.112257in}{0.739656in}}%
\pgfpathlineto{\pgfqpoint{1.111404in}{0.739656in}}%
\pgfpathlineto{\pgfqpoint{1.110551in}{0.739656in}}%
\pgfpathlineto{\pgfqpoint{1.109698in}{0.739656in}}%
\pgfpathlineto{\pgfqpoint{1.108845in}{0.739656in}}%
\pgfpathlineto{\pgfqpoint{1.107992in}{0.739656in}}%
\pgfpathlineto{\pgfqpoint{1.107139in}{0.739656in}}%
\pgfpathlineto{\pgfqpoint{1.106286in}{0.739656in}}%
\pgfpathlineto{\pgfqpoint{1.105434in}{0.739656in}}%
\pgfpathlineto{\pgfqpoint{1.104581in}{0.739656in}}%
\pgfpathlineto{\pgfqpoint{1.103728in}{0.739656in}}%
\pgfpathlineto{\pgfqpoint{1.102875in}{0.739656in}}%
\pgfpathlineto{\pgfqpoint{1.102022in}{0.739656in}}%
\pgfpathlineto{\pgfqpoint{1.101169in}{0.739656in}}%
\pgfpathlineto{\pgfqpoint{1.100316in}{0.739656in}}%
\pgfpathlineto{\pgfqpoint{1.099463in}{0.739656in}}%
\pgfpathlineto{\pgfqpoint{1.098611in}{0.739656in}}%
\pgfpathlineto{\pgfqpoint{1.097758in}{0.739656in}}%
\pgfpathlineto{\pgfqpoint{1.096905in}{0.739656in}}%
\pgfpathlineto{\pgfqpoint{1.096052in}{0.739656in}}%
\pgfpathlineto{\pgfqpoint{1.095199in}{0.739656in}}%
\pgfpathlineto{\pgfqpoint{1.094346in}{0.739656in}}%
\pgfpathlineto{\pgfqpoint{1.093493in}{0.739656in}}%
\pgfpathlineto{\pgfqpoint{1.092640in}{0.739656in}}%
\pgfpathlineto{\pgfqpoint{1.091788in}{0.739656in}}%
\pgfpathlineto{\pgfqpoint{1.090935in}{0.739656in}}%
\pgfpathlineto{\pgfqpoint{1.090082in}{0.739656in}}%
\pgfpathlineto{\pgfqpoint{1.089229in}{0.739656in}}%
\pgfpathlineto{\pgfqpoint{1.088376in}{0.739656in}}%
\pgfpathlineto{\pgfqpoint{1.087523in}{0.739656in}}%
\pgfpathlineto{\pgfqpoint{1.086670in}{0.739656in}}%
\pgfpathlineto{\pgfqpoint{1.085817in}{0.739656in}}%
\pgfpathlineto{\pgfqpoint{1.084964in}{0.739656in}}%
\pgfpathlineto{\pgfqpoint{1.084112in}{0.739656in}}%
\pgfpathlineto{\pgfqpoint{1.083259in}{0.739656in}}%
\pgfpathlineto{\pgfqpoint{1.082406in}{0.739656in}}%
\pgfpathlineto{\pgfqpoint{1.081553in}{0.739656in}}%
\pgfpathlineto{\pgfqpoint{1.080700in}{0.739656in}}%
\pgfpathlineto{\pgfqpoint{1.079847in}{0.739656in}}%
\pgfpathlineto{\pgfqpoint{1.078994in}{0.739656in}}%
\pgfpathlineto{\pgfqpoint{1.078141in}{0.739656in}}%
\pgfpathlineto{\pgfqpoint{1.077289in}{0.739656in}}%
\pgfpathlineto{\pgfqpoint{1.076436in}{0.739656in}}%
\pgfpathlineto{\pgfqpoint{1.075583in}{0.739656in}}%
\pgfpathlineto{\pgfqpoint{1.074730in}{0.739656in}}%
\pgfpathlineto{\pgfqpoint{1.073877in}{0.739656in}}%
\pgfpathlineto{\pgfqpoint{1.073024in}{0.739656in}}%
\pgfpathlineto{\pgfqpoint{1.072171in}{0.739656in}}%
\pgfpathlineto{\pgfqpoint{1.071318in}{0.739656in}}%
\pgfpathlineto{\pgfqpoint{1.070466in}{0.739656in}}%
\pgfpathlineto{\pgfqpoint{1.069613in}{0.739656in}}%
\pgfpathlineto{\pgfqpoint{1.068760in}{0.739656in}}%
\pgfpathlineto{\pgfqpoint{1.067907in}{0.739656in}}%
\pgfpathlineto{\pgfqpoint{1.067054in}{0.739656in}}%
\pgfpathlineto{\pgfqpoint{1.066201in}{0.739656in}}%
\pgfpathlineto{\pgfqpoint{1.065348in}{0.739656in}}%
\pgfpathlineto{\pgfqpoint{1.064495in}{0.739656in}}%
\pgfpathlineto{\pgfqpoint{1.063643in}{0.739656in}}%
\pgfpathlineto{\pgfqpoint{1.062790in}{0.739656in}}%
\pgfpathlineto{\pgfqpoint{1.061937in}{0.739656in}}%
\pgfpathlineto{\pgfqpoint{1.061084in}{0.739656in}}%
\pgfpathlineto{\pgfqpoint{1.060231in}{0.739656in}}%
\pgfpathlineto{\pgfqpoint{1.059378in}{0.739656in}}%
\pgfpathlineto{\pgfqpoint{1.058525in}{0.739656in}}%
\pgfpathlineto{\pgfqpoint{1.057672in}{0.739656in}}%
\pgfpathlineto{\pgfqpoint{1.056819in}{0.739656in}}%
\pgfpathlineto{\pgfqpoint{1.055967in}{0.739656in}}%
\pgfpathlineto{\pgfqpoint{1.055114in}{0.739656in}}%
\pgfpathlineto{\pgfqpoint{1.054261in}{0.739656in}}%
\pgfpathlineto{\pgfqpoint{1.053408in}{0.739656in}}%
\pgfpathlineto{\pgfqpoint{1.052555in}{0.739656in}}%
\pgfpathlineto{\pgfqpoint{1.051702in}{0.739656in}}%
\pgfpathlineto{\pgfqpoint{1.050849in}{0.739656in}}%
\pgfpathlineto{\pgfqpoint{1.049996in}{0.739656in}}%
\pgfpathlineto{\pgfqpoint{1.049144in}{0.739656in}}%
\pgfpathlineto{\pgfqpoint{1.048291in}{0.739656in}}%
\pgfpathlineto{\pgfqpoint{1.047438in}{0.739656in}}%
\pgfpathlineto{\pgfqpoint{1.046585in}{0.739656in}}%
\pgfpathlineto{\pgfqpoint{1.045732in}{0.739656in}}%
\pgfpathlineto{\pgfqpoint{1.044879in}{0.739656in}}%
\pgfpathlineto{\pgfqpoint{1.044026in}{0.739656in}}%
\pgfpathlineto{\pgfqpoint{1.043173in}{0.739656in}}%
\pgfpathlineto{\pgfqpoint{1.042321in}{0.739656in}}%
\pgfpathlineto{\pgfqpoint{1.041468in}{0.739656in}}%
\pgfpathlineto{\pgfqpoint{1.040615in}{0.739656in}}%
\pgfpathlineto{\pgfqpoint{1.039762in}{0.739656in}}%
\pgfpathlineto{\pgfqpoint{1.038909in}{0.739656in}}%
\pgfpathlineto{\pgfqpoint{1.038056in}{0.739656in}}%
\pgfpathlineto{\pgfqpoint{1.037203in}{0.739656in}}%
\pgfpathlineto{\pgfqpoint{1.036350in}{0.739656in}}%
\pgfpathlineto{\pgfqpoint{1.035498in}{0.739656in}}%
\pgfpathlineto{\pgfqpoint{1.034645in}{0.739656in}}%
\pgfpathlineto{\pgfqpoint{1.033792in}{0.739656in}}%
\pgfpathlineto{\pgfqpoint{1.032939in}{0.739656in}}%
\pgfpathlineto{\pgfqpoint{1.032086in}{0.739656in}}%
\pgfpathlineto{\pgfqpoint{1.031233in}{0.739656in}}%
\pgfpathlineto{\pgfqpoint{1.030380in}{0.739656in}}%
\pgfpathlineto{\pgfqpoint{1.029527in}{0.739656in}}%
\pgfpathlineto{\pgfqpoint{1.028674in}{0.739656in}}%
\pgfpathlineto{\pgfqpoint{1.027822in}{0.739656in}}%
\pgfpathlineto{\pgfqpoint{1.026969in}{0.739656in}}%
\pgfpathlineto{\pgfqpoint{1.026116in}{0.739656in}}%
\pgfpathlineto{\pgfqpoint{1.025263in}{0.739656in}}%
\pgfpathlineto{\pgfqpoint{1.024410in}{0.739656in}}%
\pgfpathlineto{\pgfqpoint{1.023557in}{0.739656in}}%
\pgfpathlineto{\pgfqpoint{1.022704in}{0.739656in}}%
\pgfpathlineto{\pgfqpoint{1.021851in}{0.739656in}}%
\pgfpathlineto{\pgfqpoint{1.020999in}{0.739656in}}%
\pgfpathlineto{\pgfqpoint{1.020146in}{0.739656in}}%
\pgfpathlineto{\pgfqpoint{1.019293in}{0.739656in}}%
\pgfpathlineto{\pgfqpoint{1.018440in}{0.739656in}}%
\pgfpathlineto{\pgfqpoint{1.017587in}{0.739656in}}%
\pgfpathlineto{\pgfqpoint{1.016734in}{0.739656in}}%
\pgfpathlineto{\pgfqpoint{1.015881in}{0.739656in}}%
\pgfpathlineto{\pgfqpoint{1.015028in}{0.739656in}}%
\pgfpathlineto{\pgfqpoint{1.014176in}{0.739656in}}%
\pgfpathlineto{\pgfqpoint{1.013323in}{0.739656in}}%
\pgfpathlineto{\pgfqpoint{1.012470in}{0.739656in}}%
\pgfpathlineto{\pgfqpoint{1.011617in}{0.739656in}}%
\pgfpathlineto{\pgfqpoint{1.010764in}{0.739656in}}%
\pgfpathlineto{\pgfqpoint{1.009911in}{0.739656in}}%
\pgfpathlineto{\pgfqpoint{1.009058in}{0.739656in}}%
\pgfpathlineto{\pgfqpoint{1.008205in}{0.739656in}}%
\pgfpathlineto{\pgfqpoint{1.007353in}{0.739656in}}%
\pgfpathlineto{\pgfqpoint{1.006500in}{0.739656in}}%
\pgfpathlineto{\pgfqpoint{1.005647in}{0.739656in}}%
\pgfpathlineto{\pgfqpoint{1.004794in}{0.739656in}}%
\pgfpathlineto{\pgfqpoint{1.003941in}{0.739656in}}%
\pgfpathlineto{\pgfqpoint{1.003088in}{0.739656in}}%
\pgfpathlineto{\pgfqpoint{1.002235in}{0.739656in}}%
\pgfpathlineto{\pgfqpoint{1.001382in}{0.739656in}}%
\pgfpathlineto{\pgfqpoint{1.000529in}{0.739656in}}%
\pgfpathlineto{\pgfqpoint{0.999677in}{0.739656in}}%
\pgfpathlineto{\pgfqpoint{0.998824in}{0.739656in}}%
\pgfpathlineto{\pgfqpoint{0.997971in}{0.739656in}}%
\pgfpathlineto{\pgfqpoint{0.997118in}{0.739656in}}%
\pgfpathlineto{\pgfqpoint{0.996265in}{0.739656in}}%
\pgfpathlineto{\pgfqpoint{0.995412in}{0.739656in}}%
\pgfpathlineto{\pgfqpoint{0.994559in}{0.739656in}}%
\pgfpathlineto{\pgfqpoint{0.993706in}{0.739656in}}%
\pgfpathlineto{\pgfqpoint{0.992854in}{0.739656in}}%
\pgfpathlineto{\pgfqpoint{0.992001in}{0.739656in}}%
\pgfpathlineto{\pgfqpoint{0.991148in}{0.739656in}}%
\pgfpathlineto{\pgfqpoint{0.990295in}{0.739656in}}%
\pgfpathlineto{\pgfqpoint{0.989442in}{0.739656in}}%
\pgfpathlineto{\pgfqpoint{0.988589in}{0.739656in}}%
\pgfpathlineto{\pgfqpoint{0.987736in}{0.739656in}}%
\pgfpathlineto{\pgfqpoint{0.986883in}{0.739656in}}%
\pgfpathlineto{\pgfqpoint{0.986031in}{0.739656in}}%
\pgfpathlineto{\pgfqpoint{0.985178in}{0.739656in}}%
\pgfpathlineto{\pgfqpoint{0.984325in}{0.739656in}}%
\pgfpathlineto{\pgfqpoint{0.983472in}{0.739656in}}%
\pgfpathlineto{\pgfqpoint{0.982619in}{0.739656in}}%
\pgfpathlineto{\pgfqpoint{0.981766in}{0.739656in}}%
\pgfpathlineto{\pgfqpoint{0.980913in}{0.739656in}}%
\pgfpathlineto{\pgfqpoint{0.980060in}{0.739656in}}%
\pgfpathlineto{\pgfqpoint{0.979208in}{0.739656in}}%
\pgfpathlineto{\pgfqpoint{0.978355in}{0.739656in}}%
\pgfpathlineto{\pgfqpoint{0.977502in}{0.739656in}}%
\pgfpathlineto{\pgfqpoint{0.976649in}{0.739656in}}%
\pgfpathlineto{\pgfqpoint{0.975796in}{0.739656in}}%
\pgfpathlineto{\pgfqpoint{0.974943in}{0.739656in}}%
\pgfpathlineto{\pgfqpoint{0.974090in}{0.739656in}}%
\pgfpathlineto{\pgfqpoint{0.973237in}{0.739656in}}%
\pgfpathlineto{\pgfqpoint{0.972385in}{0.739656in}}%
\pgfpathlineto{\pgfqpoint{0.971532in}{0.739656in}}%
\pgfpathlineto{\pgfqpoint{0.970679in}{0.739656in}}%
\pgfpathlineto{\pgfqpoint{0.969826in}{0.739656in}}%
\pgfpathlineto{\pgfqpoint{0.968973in}{0.739656in}}%
\pgfpathlineto{\pgfqpoint{0.968120in}{0.739656in}}%
\pgfpathlineto{\pgfqpoint{0.967267in}{0.739656in}}%
\pgfpathlineto{\pgfqpoint{0.966414in}{0.739656in}}%
\pgfpathlineto{\pgfqpoint{0.965561in}{0.739656in}}%
\pgfpathlineto{\pgfqpoint{0.964709in}{0.739656in}}%
\pgfpathlineto{\pgfqpoint{0.963856in}{0.739656in}}%
\pgfpathlineto{\pgfqpoint{0.963003in}{0.739656in}}%
\pgfpathlineto{\pgfqpoint{0.962150in}{0.739656in}}%
\pgfpathlineto{\pgfqpoint{0.961297in}{0.739656in}}%
\pgfpathlineto{\pgfqpoint{0.960444in}{0.739656in}}%
\pgfpathlineto{\pgfqpoint{0.959591in}{0.739656in}}%
\pgfpathlineto{\pgfqpoint{0.958738in}{0.739656in}}%
\pgfpathlineto{\pgfqpoint{0.957886in}{0.739656in}}%
\pgfpathlineto{\pgfqpoint{0.957033in}{0.739656in}}%
\pgfpathlineto{\pgfqpoint{0.956180in}{0.739656in}}%
\pgfpathlineto{\pgfqpoint{0.955327in}{0.739656in}}%
\pgfpathlineto{\pgfqpoint{0.954474in}{0.739656in}}%
\pgfpathlineto{\pgfqpoint{0.953621in}{0.739656in}}%
\pgfpathlineto{\pgfqpoint{0.952768in}{0.739656in}}%
\pgfpathlineto{\pgfqpoint{0.951915in}{0.739656in}}%
\pgfpathlineto{\pgfqpoint{0.951063in}{0.739656in}}%
\pgfpathlineto{\pgfqpoint{0.950210in}{0.739656in}}%
\pgfpathlineto{\pgfqpoint{0.949357in}{0.739656in}}%
\pgfpathlineto{\pgfqpoint{0.948504in}{0.739656in}}%
\pgfpathlineto{\pgfqpoint{0.947651in}{0.739656in}}%
\pgfpathlineto{\pgfqpoint{0.946798in}{0.739656in}}%
\pgfpathlineto{\pgfqpoint{0.945945in}{0.739656in}}%
\pgfpathlineto{\pgfqpoint{0.945092in}{0.739656in}}%
\pgfpathlineto{\pgfqpoint{0.944240in}{0.739656in}}%
\pgfpathlineto{\pgfqpoint{0.943387in}{0.739656in}}%
\pgfpathlineto{\pgfqpoint{0.942534in}{0.739656in}}%
\pgfpathlineto{\pgfqpoint{0.941681in}{0.739656in}}%
\pgfpathlineto{\pgfqpoint{0.940828in}{0.739656in}}%
\pgfpathlineto{\pgfqpoint{0.939975in}{0.739656in}}%
\pgfpathlineto{\pgfqpoint{0.939122in}{0.739656in}}%
\pgfpathlineto{\pgfqpoint{0.938269in}{0.739656in}}%
\pgfpathlineto{\pgfqpoint{0.937416in}{0.739656in}}%
\pgfpathlineto{\pgfqpoint{0.936564in}{0.739656in}}%
\pgfpathlineto{\pgfqpoint{0.935711in}{0.739656in}}%
\pgfpathlineto{\pgfqpoint{0.934858in}{0.739656in}}%
\pgfpathlineto{\pgfqpoint{0.934005in}{0.739656in}}%
\pgfpathlineto{\pgfqpoint{0.933152in}{0.739656in}}%
\pgfpathlineto{\pgfqpoint{0.932299in}{0.739656in}}%
\pgfpathlineto{\pgfqpoint{0.931446in}{0.739656in}}%
\pgfpathlineto{\pgfqpoint{0.930593in}{0.739656in}}%
\pgfpathlineto{\pgfqpoint{0.929741in}{0.739656in}}%
\pgfpathlineto{\pgfqpoint{0.928888in}{0.739656in}}%
\pgfpathlineto{\pgfqpoint{0.928035in}{0.739656in}}%
\pgfpathlineto{\pgfqpoint{0.927182in}{0.739656in}}%
\pgfpathlineto{\pgfqpoint{0.926329in}{0.739656in}}%
\pgfpathlineto{\pgfqpoint{0.925476in}{0.739656in}}%
\pgfpathlineto{\pgfqpoint{0.924623in}{0.739656in}}%
\pgfpathlineto{\pgfqpoint{0.923770in}{0.739656in}}%
\pgfpathlineto{\pgfqpoint{0.922918in}{0.739656in}}%
\pgfpathlineto{\pgfqpoint{0.922065in}{0.739656in}}%
\pgfpathlineto{\pgfqpoint{0.921212in}{0.739656in}}%
\pgfpathlineto{\pgfqpoint{0.920359in}{0.739656in}}%
\pgfpathlineto{\pgfqpoint{0.919506in}{0.739656in}}%
\pgfpathlineto{\pgfqpoint{0.918653in}{0.739656in}}%
\pgfpathlineto{\pgfqpoint{0.917800in}{0.739656in}}%
\pgfpathlineto{\pgfqpoint{0.916947in}{0.739656in}}%
\pgfpathlineto{\pgfqpoint{0.916095in}{0.739656in}}%
\pgfpathlineto{\pgfqpoint{0.915242in}{0.739656in}}%
\pgfpathlineto{\pgfqpoint{0.914389in}{0.739656in}}%
\pgfpathlineto{\pgfqpoint{0.913536in}{0.739656in}}%
\pgfpathlineto{\pgfqpoint{0.912683in}{0.739656in}}%
\pgfpathlineto{\pgfqpoint{0.911830in}{0.739656in}}%
\pgfpathlineto{\pgfqpoint{0.910977in}{0.739656in}}%
\pgfpathlineto{\pgfqpoint{0.910124in}{0.739656in}}%
\pgfpathlineto{\pgfqpoint{0.909271in}{0.739656in}}%
\pgfpathlineto{\pgfqpoint{0.908419in}{0.739656in}}%
\pgfpathlineto{\pgfqpoint{0.907566in}{0.739656in}}%
\pgfpathlineto{\pgfqpoint{0.906713in}{0.739656in}}%
\pgfpathlineto{\pgfqpoint{0.905860in}{0.739656in}}%
\pgfpathlineto{\pgfqpoint{0.905007in}{0.739656in}}%
\pgfpathlineto{\pgfqpoint{0.904154in}{0.739656in}}%
\pgfpathlineto{\pgfqpoint{0.903301in}{0.739656in}}%
\pgfpathlineto{\pgfqpoint{0.902448in}{0.739656in}}%
\pgfpathlineto{\pgfqpoint{0.901596in}{0.739656in}}%
\pgfpathlineto{\pgfqpoint{0.900743in}{0.739656in}}%
\pgfpathlineto{\pgfqpoint{0.899890in}{0.739656in}}%
\pgfpathlineto{\pgfqpoint{0.899037in}{0.739656in}}%
\pgfpathlineto{\pgfqpoint{0.898184in}{0.739656in}}%
\pgfpathlineto{\pgfqpoint{0.897331in}{0.739656in}}%
\pgfpathlineto{\pgfqpoint{0.896478in}{0.739656in}}%
\pgfpathlineto{\pgfqpoint{0.895625in}{0.739656in}}%
\pgfpathlineto{\pgfqpoint{0.894773in}{0.739656in}}%
\pgfpathlineto{\pgfqpoint{0.893920in}{0.739656in}}%
\pgfpathlineto{\pgfqpoint{0.893067in}{0.739656in}}%
\pgfpathlineto{\pgfqpoint{0.892214in}{0.739656in}}%
\pgfpathlineto{\pgfqpoint{0.891361in}{0.739656in}}%
\pgfpathlineto{\pgfqpoint{0.890508in}{0.739656in}}%
\pgfpathlineto{\pgfqpoint{0.889655in}{0.739656in}}%
\pgfpathlineto{\pgfqpoint{0.888802in}{0.739656in}}%
\pgfpathlineto{\pgfqpoint{0.887950in}{0.739656in}}%
\pgfpathlineto{\pgfqpoint{0.887097in}{0.739656in}}%
\pgfpathlineto{\pgfqpoint{0.886244in}{0.739656in}}%
\pgfpathlineto{\pgfqpoint{0.885391in}{0.739656in}}%
\pgfpathlineto{\pgfqpoint{0.884538in}{0.739656in}}%
\pgfpathlineto{\pgfqpoint{0.883685in}{0.739656in}}%
\pgfpathlineto{\pgfqpoint{0.882832in}{0.739656in}}%
\pgfpathlineto{\pgfqpoint{0.881979in}{0.739656in}}%
\pgfpathlineto{\pgfqpoint{0.881127in}{0.739656in}}%
\pgfpathlineto{\pgfqpoint{0.880274in}{0.739656in}}%
\pgfpathlineto{\pgfqpoint{0.879421in}{0.739656in}}%
\pgfpathlineto{\pgfqpoint{0.878568in}{0.739656in}}%
\pgfpathlineto{\pgfqpoint{0.877715in}{0.739656in}}%
\pgfpathlineto{\pgfqpoint{0.876862in}{0.739656in}}%
\pgfpathlineto{\pgfqpoint{0.876009in}{0.739656in}}%
\pgfpathlineto{\pgfqpoint{0.875156in}{0.739656in}}%
\pgfpathlineto{\pgfqpoint{0.874303in}{0.739656in}}%
\pgfpathlineto{\pgfqpoint{0.873451in}{0.739656in}}%
\pgfpathlineto{\pgfqpoint{0.872598in}{0.739656in}}%
\pgfpathlineto{\pgfqpoint{0.871745in}{0.739656in}}%
\pgfpathlineto{\pgfqpoint{0.870892in}{0.739656in}}%
\pgfpathlineto{\pgfqpoint{0.870039in}{0.739656in}}%
\pgfpathlineto{\pgfqpoint{0.869186in}{0.739656in}}%
\pgfpathlineto{\pgfqpoint{0.868333in}{0.739656in}}%
\pgfpathlineto{\pgfqpoint{0.867480in}{0.739656in}}%
\pgfpathlineto{\pgfqpoint{0.866628in}{0.739656in}}%
\pgfpathlineto{\pgfqpoint{0.865775in}{0.739656in}}%
\pgfpathlineto{\pgfqpoint{0.864922in}{0.739656in}}%
\pgfpathlineto{\pgfqpoint{0.864069in}{0.739656in}}%
\pgfpathlineto{\pgfqpoint{0.863216in}{0.739656in}}%
\pgfpathlineto{\pgfqpoint{0.862363in}{0.739656in}}%
\pgfpathlineto{\pgfqpoint{0.861510in}{0.739656in}}%
\pgfpathlineto{\pgfqpoint{0.860657in}{0.739656in}}%
\pgfpathlineto{\pgfqpoint{0.859805in}{0.739656in}}%
\pgfpathlineto{\pgfqpoint{0.858952in}{0.739656in}}%
\pgfpathlineto{\pgfqpoint{0.858099in}{0.739656in}}%
\pgfpathlineto{\pgfqpoint{0.857246in}{0.739656in}}%
\pgfpathlineto{\pgfqpoint{0.856393in}{0.739656in}}%
\pgfpathlineto{\pgfqpoint{0.855540in}{0.739656in}}%
\pgfpathlineto{\pgfqpoint{0.854687in}{0.739656in}}%
\pgfpathlineto{\pgfqpoint{0.853834in}{0.739656in}}%
\pgfpathlineto{\pgfqpoint{0.852982in}{0.739656in}}%
\pgfpathlineto{\pgfqpoint{0.852129in}{0.739656in}}%
\pgfpathlineto{\pgfqpoint{0.851276in}{0.739656in}}%
\pgfpathlineto{\pgfqpoint{0.850423in}{0.739656in}}%
\pgfpathlineto{\pgfqpoint{0.849570in}{0.739656in}}%
\pgfpathlineto{\pgfqpoint{0.848717in}{0.739656in}}%
\pgfpathlineto{\pgfqpoint{0.847864in}{0.739656in}}%
\pgfpathlineto{\pgfqpoint{0.847011in}{0.739656in}}%
\pgfpathlineto{\pgfqpoint{0.846158in}{0.739656in}}%
\pgfpathlineto{\pgfqpoint{0.845306in}{0.739656in}}%
\pgfpathlineto{\pgfqpoint{0.844453in}{0.739656in}}%
\pgfpathlineto{\pgfqpoint{0.843600in}{0.739656in}}%
\pgfpathlineto{\pgfqpoint{0.842747in}{0.739656in}}%
\pgfpathlineto{\pgfqpoint{0.841894in}{0.739656in}}%
\pgfpathlineto{\pgfqpoint{0.841041in}{0.739656in}}%
\pgfpathlineto{\pgfqpoint{0.840188in}{0.739656in}}%
\pgfpathlineto{\pgfqpoint{0.839335in}{0.739656in}}%
\pgfpathlineto{\pgfqpoint{0.838483in}{0.739656in}}%
\pgfpathlineto{\pgfqpoint{0.837630in}{0.739656in}}%
\pgfpathlineto{\pgfqpoint{0.836777in}{0.739656in}}%
\pgfpathlineto{\pgfqpoint{0.835924in}{0.739656in}}%
\pgfpathlineto{\pgfqpoint{0.835071in}{0.739656in}}%
\pgfpathlineto{\pgfqpoint{0.834218in}{0.739656in}}%
\pgfpathlineto{\pgfqpoint{0.833365in}{0.739656in}}%
\pgfpathlineto{\pgfqpoint{0.832512in}{0.739656in}}%
\pgfpathlineto{\pgfqpoint{0.831660in}{0.739656in}}%
\pgfpathlineto{\pgfqpoint{0.830807in}{0.739656in}}%
\pgfpathlineto{\pgfqpoint{0.829954in}{0.739656in}}%
\pgfpathlineto{\pgfqpoint{0.829101in}{0.739656in}}%
\pgfpathlineto{\pgfqpoint{0.828248in}{0.739656in}}%
\pgfpathlineto{\pgfqpoint{0.827395in}{0.739656in}}%
\pgfpathlineto{\pgfqpoint{0.826542in}{0.739656in}}%
\pgfpathlineto{\pgfqpoint{0.825689in}{0.739656in}}%
\pgfpathlineto{\pgfqpoint{0.824837in}{0.739656in}}%
\pgfpathlineto{\pgfqpoint{0.823984in}{0.739656in}}%
\pgfpathlineto{\pgfqpoint{0.823131in}{0.739656in}}%
\pgfpathlineto{\pgfqpoint{0.822278in}{0.739656in}}%
\pgfpathlineto{\pgfqpoint{0.821425in}{0.739656in}}%
\pgfpathlineto{\pgfqpoint{0.820572in}{0.739656in}}%
\pgfpathlineto{\pgfqpoint{0.819719in}{0.739656in}}%
\pgfpathlineto{\pgfqpoint{0.818866in}{0.739656in}}%
\pgfpathlineto{\pgfqpoint{0.818013in}{0.739656in}}%
\pgfpathlineto{\pgfqpoint{0.817161in}{0.739656in}}%
\pgfpathlineto{\pgfqpoint{0.816308in}{0.739656in}}%
\pgfpathlineto{\pgfqpoint{0.815455in}{0.739656in}}%
\pgfpathlineto{\pgfqpoint{0.814602in}{0.739656in}}%
\pgfpathlineto{\pgfqpoint{0.813749in}{0.739656in}}%
\pgfpathlineto{\pgfqpoint{0.812896in}{0.739656in}}%
\pgfpathlineto{\pgfqpoint{0.812043in}{0.739656in}}%
\pgfpathlineto{\pgfqpoint{0.811190in}{0.739656in}}%
\pgfpathlineto{\pgfqpoint{0.810338in}{0.739656in}}%
\pgfpathlineto{\pgfqpoint{0.809485in}{0.739656in}}%
\pgfpathlineto{\pgfqpoint{0.808632in}{0.739656in}}%
\pgfpathlineto{\pgfqpoint{0.807779in}{0.739656in}}%
\pgfpathlineto{\pgfqpoint{0.806926in}{0.739656in}}%
\pgfpathlineto{\pgfqpoint{0.806073in}{0.739656in}}%
\pgfpathlineto{\pgfqpoint{0.805220in}{0.739656in}}%
\pgfpathlineto{\pgfqpoint{0.804367in}{0.739656in}}%
\pgfpathlineto{\pgfqpoint{0.803515in}{0.739656in}}%
\pgfpathlineto{\pgfqpoint{0.802662in}{0.739656in}}%
\pgfpathlineto{\pgfqpoint{0.801809in}{0.739656in}}%
\pgfpathlineto{\pgfqpoint{0.800956in}{0.739656in}}%
\pgfpathlineto{\pgfqpoint{0.800103in}{0.739656in}}%
\pgfpathlineto{\pgfqpoint{0.799250in}{0.739656in}}%
\pgfpathlineto{\pgfqpoint{0.798397in}{0.739656in}}%
\pgfpathlineto{\pgfqpoint{0.797544in}{0.739656in}}%
\pgfpathlineto{\pgfqpoint{0.796692in}{0.739656in}}%
\pgfpathlineto{\pgfqpoint{0.795839in}{0.739656in}}%
\pgfpathlineto{\pgfqpoint{0.794986in}{0.739656in}}%
\pgfpathlineto{\pgfqpoint{0.794133in}{0.739656in}}%
\pgfpathlineto{\pgfqpoint{0.793280in}{0.739656in}}%
\pgfpathlineto{\pgfqpoint{0.792427in}{0.739656in}}%
\pgfpathlineto{\pgfqpoint{0.791574in}{0.739656in}}%
\pgfpathlineto{\pgfqpoint{0.790721in}{0.739656in}}%
\pgfpathlineto{\pgfqpoint{0.789868in}{0.739656in}}%
\pgfpathlineto{\pgfqpoint{0.789016in}{0.739656in}}%
\pgfpathlineto{\pgfqpoint{0.788163in}{0.739656in}}%
\pgfpathlineto{\pgfqpoint{0.787310in}{0.739656in}}%
\pgfpathlineto{\pgfqpoint{0.786457in}{0.739656in}}%
\pgfpathlineto{\pgfqpoint{0.785604in}{0.739656in}}%
\pgfpathlineto{\pgfqpoint{0.784751in}{0.739656in}}%
\pgfpathlineto{\pgfqpoint{0.783898in}{0.739656in}}%
\pgfpathlineto{\pgfqpoint{0.783045in}{0.739656in}}%
\pgfpathlineto{\pgfqpoint{0.782193in}{0.739656in}}%
\pgfpathlineto{\pgfqpoint{0.781340in}{0.739656in}}%
\pgfpathlineto{\pgfqpoint{0.780487in}{0.739656in}}%
\pgfpathlineto{\pgfqpoint{0.779634in}{0.739656in}}%
\pgfpathlineto{\pgfqpoint{0.778781in}{0.739656in}}%
\pgfpathlineto{\pgfqpoint{0.777928in}{0.739656in}}%
\pgfpathlineto{\pgfqpoint{0.777075in}{0.739656in}}%
\pgfpathlineto{\pgfqpoint{0.776222in}{0.739656in}}%
\pgfpathlineto{\pgfqpoint{0.775370in}{0.739656in}}%
\pgfpathlineto{\pgfqpoint{0.774517in}{0.739656in}}%
\pgfpathlineto{\pgfqpoint{0.773664in}{0.739656in}}%
\pgfpathlineto{\pgfqpoint{0.772811in}{0.739656in}}%
\pgfpathlineto{\pgfqpoint{0.771958in}{0.739656in}}%
\pgfpathlineto{\pgfqpoint{0.771105in}{0.739656in}}%
\pgfpathlineto{\pgfqpoint{0.770252in}{0.739656in}}%
\pgfpathlineto{\pgfqpoint{0.769399in}{0.739656in}}%
\pgfpathlineto{\pgfqpoint{0.768547in}{0.739656in}}%
\pgfpathlineto{\pgfqpoint{0.767694in}{0.739656in}}%
\pgfpathlineto{\pgfqpoint{0.766841in}{0.739656in}}%
\pgfpathlineto{\pgfqpoint{0.765988in}{0.739656in}}%
\pgfpathlineto{\pgfqpoint{0.765135in}{0.739656in}}%
\pgfpathlineto{\pgfqpoint{0.764282in}{0.739656in}}%
\pgfpathlineto{\pgfqpoint{0.763429in}{0.739656in}}%
\pgfpathlineto{\pgfqpoint{0.762576in}{0.739656in}}%
\pgfpathlineto{\pgfqpoint{0.761724in}{0.739656in}}%
\pgfpathlineto{\pgfqpoint{0.760871in}{0.739656in}}%
\pgfpathlineto{\pgfqpoint{0.760018in}{0.739656in}}%
\pgfpathlineto{\pgfqpoint{0.759165in}{0.739656in}}%
\pgfpathlineto{\pgfqpoint{0.758312in}{0.739656in}}%
\pgfpathlineto{\pgfqpoint{0.757459in}{0.739656in}}%
\pgfpathlineto{\pgfqpoint{0.756606in}{0.739656in}}%
\pgfpathlineto{\pgfqpoint{0.755753in}{0.739656in}}%
\pgfpathlineto{\pgfqpoint{0.754900in}{0.739656in}}%
\pgfpathlineto{\pgfqpoint{0.754048in}{0.739656in}}%
\pgfpathlineto{\pgfqpoint{0.753195in}{0.739656in}}%
\pgfpathlineto{\pgfqpoint{0.752342in}{0.739656in}}%
\pgfpathlineto{\pgfqpoint{0.751489in}{0.739656in}}%
\pgfpathlineto{\pgfqpoint{0.750636in}{0.739656in}}%
\pgfpathlineto{\pgfqpoint{0.749783in}{0.739656in}}%
\pgfpathlineto{\pgfqpoint{0.748930in}{0.739656in}}%
\pgfpathlineto{\pgfqpoint{0.748077in}{0.739656in}}%
\pgfpathlineto{\pgfqpoint{0.747225in}{0.739656in}}%
\pgfpathlineto{\pgfqpoint{0.746372in}{0.739656in}}%
\pgfpathlineto{\pgfqpoint{0.745519in}{0.739656in}}%
\pgfpathlineto{\pgfqpoint{0.744666in}{0.739656in}}%
\pgfpathlineto{\pgfqpoint{0.744666in}{0.739656in}}%
\pgfpathclose%
\pgfusepath{fill}%
\end{pgfscope}%
\begin{pgfscope}%
\pgfsetbuttcap%
\pgfsetroundjoin%
\definecolor{currentfill}{rgb}{0.000000,0.000000,0.000000}%
\pgfsetfillcolor{currentfill}%
\pgfsetlinewidth{0.501875pt}%
\definecolor{currentstroke}{rgb}{0.000000,0.000000,0.000000}%
\pgfsetstrokecolor{currentstroke}%
\pgfsetdash{}{0pt}%
\pgfsys@defobject{currentmarker}{\pgfqpoint{0.000000in}{0.000000in}}{\pgfqpoint{0.000000in}{0.041667in}}{%
\pgfpathmoveto{\pgfqpoint{0.000000in}{0.000000in}}%
\pgfpathlineto{\pgfqpoint{0.000000in}{0.041667in}}%
\pgfusepath{stroke,fill}%
}%
\begin{pgfscope}%
\pgfsys@transformshift{0.854687in}{0.586309in}%
\pgfsys@useobject{currentmarker}{}%
\end{pgfscope}%
\end{pgfscope}%
\begin{pgfscope}%
\pgfsetbuttcap%
\pgfsetroundjoin%
\definecolor{currentfill}{rgb}{0.000000,0.000000,0.000000}%
\pgfsetfillcolor{currentfill}%
\pgfsetlinewidth{0.501875pt}%
\definecolor{currentstroke}{rgb}{0.000000,0.000000,0.000000}%
\pgfsetstrokecolor{currentstroke}%
\pgfsetdash{}{0pt}%
\pgfsys@defobject{currentmarker}{\pgfqpoint{0.000000in}{-0.041667in}}{\pgfqpoint{0.000000in}{0.000000in}}{%
\pgfpathmoveto{\pgfqpoint{0.000000in}{0.000000in}}%
\pgfpathlineto{\pgfqpoint{0.000000in}{-0.041667in}}%
\pgfusepath{stroke,fill}%
}%
\begin{pgfscope}%
\pgfsys@transformshift{0.854687in}{0.893003in}%
\pgfsys@useobject{currentmarker}{}%
\end{pgfscope}%
\end{pgfscope}%
\begin{pgfscope}%
\definecolor{textcolor}{rgb}{0.000000,0.000000,0.000000}%
\pgfsetstrokecolor{textcolor}%
\pgfsetfillcolor{textcolor}%
\pgftext[x=0.397160in, y=0.220556in, left, base,rotate=30.000000]{\color{textcolor}\rmfamily\fontsize{7.000000}{8.400000}\selectfont 2021-07-28}%
\end{pgfscope}%
\begin{pgfscope}%
\pgfsetbuttcap%
\pgfsetroundjoin%
\definecolor{currentfill}{rgb}{0.000000,0.000000,0.000000}%
\pgfsetfillcolor{currentfill}%
\pgfsetlinewidth{0.501875pt}%
\definecolor{currentstroke}{rgb}{0.000000,0.000000,0.000000}%
\pgfsetstrokecolor{currentstroke}%
\pgfsetdash{}{0pt}%
\pgfsys@defobject{currentmarker}{\pgfqpoint{0.000000in}{0.000000in}}{\pgfqpoint{0.000000in}{0.041667in}}{%
\pgfpathmoveto{\pgfqpoint{0.000000in}{0.000000in}}%
\pgfpathlineto{\pgfqpoint{0.000000in}{0.041667in}}%
\pgfusepath{stroke,fill}%
}%
\begin{pgfscope}%
\pgfsys@transformshift{1.468760in}{0.586309in}%
\pgfsys@useobject{currentmarker}{}%
\end{pgfscope}%
\end{pgfscope}%
\begin{pgfscope}%
\pgfsetbuttcap%
\pgfsetroundjoin%
\definecolor{currentfill}{rgb}{0.000000,0.000000,0.000000}%
\pgfsetfillcolor{currentfill}%
\pgfsetlinewidth{0.501875pt}%
\definecolor{currentstroke}{rgb}{0.000000,0.000000,0.000000}%
\pgfsetstrokecolor{currentstroke}%
\pgfsetdash{}{0pt}%
\pgfsys@defobject{currentmarker}{\pgfqpoint{0.000000in}{-0.041667in}}{\pgfqpoint{0.000000in}{0.000000in}}{%
\pgfpathmoveto{\pgfqpoint{0.000000in}{0.000000in}}%
\pgfpathlineto{\pgfqpoint{0.000000in}{-0.041667in}}%
\pgfusepath{stroke,fill}%
}%
\begin{pgfscope}%
\pgfsys@transformshift{1.468760in}{0.893003in}%
\pgfsys@useobject{currentmarker}{}%
\end{pgfscope}%
\end{pgfscope}%
\begin{pgfscope}%
\definecolor{textcolor}{rgb}{0.000000,0.000000,0.000000}%
\pgfsetstrokecolor{textcolor}%
\pgfsetfillcolor{textcolor}%
\pgftext[x=1.011233in, y=0.220556in, left, base,rotate=30.000000]{\color{textcolor}\rmfamily\fontsize{7.000000}{8.400000}\selectfont 2021-08-27}%
\end{pgfscope}%
\begin{pgfscope}%
\pgfsetbuttcap%
\pgfsetroundjoin%
\definecolor{currentfill}{rgb}{0.000000,0.000000,0.000000}%
\pgfsetfillcolor{currentfill}%
\pgfsetlinewidth{0.501875pt}%
\definecolor{currentstroke}{rgb}{0.000000,0.000000,0.000000}%
\pgfsetstrokecolor{currentstroke}%
\pgfsetdash{}{0pt}%
\pgfsys@defobject{currentmarker}{\pgfqpoint{0.000000in}{0.000000in}}{\pgfqpoint{0.000000in}{0.041667in}}{%
\pgfpathmoveto{\pgfqpoint{0.000000in}{0.000000in}}%
\pgfpathlineto{\pgfqpoint{0.000000in}{0.041667in}}%
\pgfusepath{stroke,fill}%
}%
\begin{pgfscope}%
\pgfsys@transformshift{2.082832in}{0.586309in}%
\pgfsys@useobject{currentmarker}{}%
\end{pgfscope}%
\end{pgfscope}%
\begin{pgfscope}%
\pgfsetbuttcap%
\pgfsetroundjoin%
\definecolor{currentfill}{rgb}{0.000000,0.000000,0.000000}%
\pgfsetfillcolor{currentfill}%
\pgfsetlinewidth{0.501875pt}%
\definecolor{currentstroke}{rgb}{0.000000,0.000000,0.000000}%
\pgfsetstrokecolor{currentstroke}%
\pgfsetdash{}{0pt}%
\pgfsys@defobject{currentmarker}{\pgfqpoint{0.000000in}{-0.041667in}}{\pgfqpoint{0.000000in}{0.000000in}}{%
\pgfpathmoveto{\pgfqpoint{0.000000in}{0.000000in}}%
\pgfpathlineto{\pgfqpoint{0.000000in}{-0.041667in}}%
\pgfusepath{stroke,fill}%
}%
\begin{pgfscope}%
\pgfsys@transformshift{2.082832in}{0.893003in}%
\pgfsys@useobject{currentmarker}{}%
\end{pgfscope}%
\end{pgfscope}%
\begin{pgfscope}%
\definecolor{textcolor}{rgb}{0.000000,0.000000,0.000000}%
\pgfsetstrokecolor{textcolor}%
\pgfsetfillcolor{textcolor}%
\pgftext[x=1.625305in, y=0.220556in, left, base,rotate=30.000000]{\color{textcolor}\rmfamily\fontsize{7.000000}{8.400000}\selectfont 2021-09-26}%
\end{pgfscope}%
\begin{pgfscope}%
\pgfsetbuttcap%
\pgfsetroundjoin%
\definecolor{currentfill}{rgb}{0.000000,0.000000,0.000000}%
\pgfsetfillcolor{currentfill}%
\pgfsetlinewidth{0.501875pt}%
\definecolor{currentstroke}{rgb}{0.000000,0.000000,0.000000}%
\pgfsetstrokecolor{currentstroke}%
\pgfsetdash{}{0pt}%
\pgfsys@defobject{currentmarker}{\pgfqpoint{0.000000in}{0.000000in}}{\pgfqpoint{0.000000in}{0.041667in}}{%
\pgfpathmoveto{\pgfqpoint{0.000000in}{0.000000in}}%
\pgfpathlineto{\pgfqpoint{0.000000in}{0.041667in}}%
\pgfusepath{stroke,fill}%
}%
\begin{pgfscope}%
\pgfsys@transformshift{2.696905in}{0.586309in}%
\pgfsys@useobject{currentmarker}{}%
\end{pgfscope}%
\end{pgfscope}%
\begin{pgfscope}%
\pgfsetbuttcap%
\pgfsetroundjoin%
\definecolor{currentfill}{rgb}{0.000000,0.000000,0.000000}%
\pgfsetfillcolor{currentfill}%
\pgfsetlinewidth{0.501875pt}%
\definecolor{currentstroke}{rgb}{0.000000,0.000000,0.000000}%
\pgfsetstrokecolor{currentstroke}%
\pgfsetdash{}{0pt}%
\pgfsys@defobject{currentmarker}{\pgfqpoint{0.000000in}{-0.041667in}}{\pgfqpoint{0.000000in}{0.000000in}}{%
\pgfpathmoveto{\pgfqpoint{0.000000in}{0.000000in}}%
\pgfpathlineto{\pgfqpoint{0.000000in}{-0.041667in}}%
\pgfusepath{stroke,fill}%
}%
\begin{pgfscope}%
\pgfsys@transformshift{2.696905in}{0.893003in}%
\pgfsys@useobject{currentmarker}{}%
\end{pgfscope}%
\end{pgfscope}%
\begin{pgfscope}%
\definecolor{textcolor}{rgb}{0.000000,0.000000,0.000000}%
\pgfsetstrokecolor{textcolor}%
\pgfsetfillcolor{textcolor}%
\pgftext[x=2.239378in, y=0.220556in, left, base,rotate=30.000000]{\color{textcolor}\rmfamily\fontsize{7.000000}{8.400000}\selectfont 2021-10-26}%
\end{pgfscope}%
\begin{pgfscope}%
\pgfsetbuttcap%
\pgfsetroundjoin%
\definecolor{currentfill}{rgb}{0.000000,0.000000,0.000000}%
\pgfsetfillcolor{currentfill}%
\pgfsetlinewidth{0.501875pt}%
\definecolor{currentstroke}{rgb}{0.000000,0.000000,0.000000}%
\pgfsetstrokecolor{currentstroke}%
\pgfsetdash{}{0pt}%
\pgfsys@defobject{currentmarker}{\pgfqpoint{0.000000in}{0.000000in}}{\pgfqpoint{0.000000in}{0.041667in}}{%
\pgfpathmoveto{\pgfqpoint{0.000000in}{0.000000in}}%
\pgfpathlineto{\pgfqpoint{0.000000in}{0.041667in}}%
\pgfusepath{stroke,fill}%
}%
\begin{pgfscope}%
\pgfsys@transformshift{3.310977in}{0.586309in}%
\pgfsys@useobject{currentmarker}{}%
\end{pgfscope}%
\end{pgfscope}%
\begin{pgfscope}%
\pgfsetbuttcap%
\pgfsetroundjoin%
\definecolor{currentfill}{rgb}{0.000000,0.000000,0.000000}%
\pgfsetfillcolor{currentfill}%
\pgfsetlinewidth{0.501875pt}%
\definecolor{currentstroke}{rgb}{0.000000,0.000000,0.000000}%
\pgfsetstrokecolor{currentstroke}%
\pgfsetdash{}{0pt}%
\pgfsys@defobject{currentmarker}{\pgfqpoint{0.000000in}{-0.041667in}}{\pgfqpoint{0.000000in}{0.000000in}}{%
\pgfpathmoveto{\pgfqpoint{0.000000in}{0.000000in}}%
\pgfpathlineto{\pgfqpoint{0.000000in}{-0.041667in}}%
\pgfusepath{stroke,fill}%
}%
\begin{pgfscope}%
\pgfsys@transformshift{3.310977in}{0.893003in}%
\pgfsys@useobject{currentmarker}{}%
\end{pgfscope}%
\end{pgfscope}%
\begin{pgfscope}%
\definecolor{textcolor}{rgb}{0.000000,0.000000,0.000000}%
\pgfsetstrokecolor{textcolor}%
\pgfsetfillcolor{textcolor}%
\pgftext[x=2.853451in, y=0.220556in, left, base,rotate=30.000000]{\color{textcolor}\rmfamily\fontsize{7.000000}{8.400000}\selectfont 2021-11-25}%
\end{pgfscope}%
\begin{pgfscope}%
\pgfsetbuttcap%
\pgfsetroundjoin%
\definecolor{currentfill}{rgb}{0.000000,0.000000,0.000000}%
\pgfsetfillcolor{currentfill}%
\pgfsetlinewidth{0.501875pt}%
\definecolor{currentstroke}{rgb}{0.000000,0.000000,0.000000}%
\pgfsetstrokecolor{currentstroke}%
\pgfsetdash{}{0pt}%
\pgfsys@defobject{currentmarker}{\pgfqpoint{0.000000in}{0.000000in}}{\pgfqpoint{0.000000in}{0.041667in}}{%
\pgfpathmoveto{\pgfqpoint{0.000000in}{0.000000in}}%
\pgfpathlineto{\pgfqpoint{0.000000in}{0.041667in}}%
\pgfusepath{stroke,fill}%
}%
\begin{pgfscope}%
\pgfsys@transformshift{3.925050in}{0.586309in}%
\pgfsys@useobject{currentmarker}{}%
\end{pgfscope}%
\end{pgfscope}%
\begin{pgfscope}%
\pgfsetbuttcap%
\pgfsetroundjoin%
\definecolor{currentfill}{rgb}{0.000000,0.000000,0.000000}%
\pgfsetfillcolor{currentfill}%
\pgfsetlinewidth{0.501875pt}%
\definecolor{currentstroke}{rgb}{0.000000,0.000000,0.000000}%
\pgfsetstrokecolor{currentstroke}%
\pgfsetdash{}{0pt}%
\pgfsys@defobject{currentmarker}{\pgfqpoint{0.000000in}{-0.041667in}}{\pgfqpoint{0.000000in}{0.000000in}}{%
\pgfpathmoveto{\pgfqpoint{0.000000in}{0.000000in}}%
\pgfpathlineto{\pgfqpoint{0.000000in}{-0.041667in}}%
\pgfusepath{stroke,fill}%
}%
\begin{pgfscope}%
\pgfsys@transformshift{3.925050in}{0.893003in}%
\pgfsys@useobject{currentmarker}{}%
\end{pgfscope}%
\end{pgfscope}%
\begin{pgfscope}%
\definecolor{textcolor}{rgb}{0.000000,0.000000,0.000000}%
\pgfsetstrokecolor{textcolor}%
\pgfsetfillcolor{textcolor}%
\pgftext[x=3.467523in, y=0.220556in, left, base,rotate=30.000000]{\color{textcolor}\rmfamily\fontsize{7.000000}{8.400000}\selectfont 2021-12-25}%
\end{pgfscope}%
\begin{pgfscope}%
\pgfsetbuttcap%
\pgfsetroundjoin%
\definecolor{currentfill}{rgb}{0.000000,0.000000,0.000000}%
\pgfsetfillcolor{currentfill}%
\pgfsetlinewidth{0.501875pt}%
\definecolor{currentstroke}{rgb}{0.000000,0.000000,0.000000}%
\pgfsetstrokecolor{currentstroke}%
\pgfsetdash{}{0pt}%
\pgfsys@defobject{currentmarker}{\pgfqpoint{0.000000in}{0.000000in}}{\pgfqpoint{0.000000in}{0.041667in}}{%
\pgfpathmoveto{\pgfqpoint{0.000000in}{0.000000in}}%
\pgfpathlineto{\pgfqpoint{0.000000in}{0.041667in}}%
\pgfusepath{stroke,fill}%
}%
\begin{pgfscope}%
\pgfsys@transformshift{4.539123in}{0.586309in}%
\pgfsys@useobject{currentmarker}{}%
\end{pgfscope}%
\end{pgfscope}%
\begin{pgfscope}%
\pgfsetbuttcap%
\pgfsetroundjoin%
\definecolor{currentfill}{rgb}{0.000000,0.000000,0.000000}%
\pgfsetfillcolor{currentfill}%
\pgfsetlinewidth{0.501875pt}%
\definecolor{currentstroke}{rgb}{0.000000,0.000000,0.000000}%
\pgfsetstrokecolor{currentstroke}%
\pgfsetdash{}{0pt}%
\pgfsys@defobject{currentmarker}{\pgfqpoint{0.000000in}{-0.041667in}}{\pgfqpoint{0.000000in}{0.000000in}}{%
\pgfpathmoveto{\pgfqpoint{0.000000in}{0.000000in}}%
\pgfpathlineto{\pgfqpoint{0.000000in}{-0.041667in}}%
\pgfusepath{stroke,fill}%
}%
\begin{pgfscope}%
\pgfsys@transformshift{4.539123in}{0.893003in}%
\pgfsys@useobject{currentmarker}{}%
\end{pgfscope}%
\end{pgfscope}%
\begin{pgfscope}%
\definecolor{textcolor}{rgb}{0.000000,0.000000,0.000000}%
\pgfsetstrokecolor{textcolor}%
\pgfsetfillcolor{textcolor}%
\pgftext[x=4.081596in, y=0.220556in, left, base,rotate=30.000000]{\color{textcolor}\rmfamily\fontsize{7.000000}{8.400000}\selectfont 2022-01-24}%
\end{pgfscope}%
\begin{pgfscope}%
\pgfsetbuttcap%
\pgfsetroundjoin%
\definecolor{currentfill}{rgb}{0.000000,0.000000,0.000000}%
\pgfsetfillcolor{currentfill}%
\pgfsetlinewidth{0.501875pt}%
\definecolor{currentstroke}{rgb}{0.000000,0.000000,0.000000}%
\pgfsetstrokecolor{currentstroke}%
\pgfsetdash{}{0pt}%
\pgfsys@defobject{currentmarker}{\pgfqpoint{0.000000in}{0.000000in}}{\pgfqpoint{0.000000in}{0.041667in}}{%
\pgfpathmoveto{\pgfqpoint{0.000000in}{0.000000in}}%
\pgfpathlineto{\pgfqpoint{0.000000in}{0.041667in}}%
\pgfusepath{stroke,fill}%
}%
\begin{pgfscope}%
\pgfsys@transformshift{5.153195in}{0.586309in}%
\pgfsys@useobject{currentmarker}{}%
\end{pgfscope}%
\end{pgfscope}%
\begin{pgfscope}%
\pgfsetbuttcap%
\pgfsetroundjoin%
\definecolor{currentfill}{rgb}{0.000000,0.000000,0.000000}%
\pgfsetfillcolor{currentfill}%
\pgfsetlinewidth{0.501875pt}%
\definecolor{currentstroke}{rgb}{0.000000,0.000000,0.000000}%
\pgfsetstrokecolor{currentstroke}%
\pgfsetdash{}{0pt}%
\pgfsys@defobject{currentmarker}{\pgfqpoint{0.000000in}{-0.041667in}}{\pgfqpoint{0.000000in}{0.000000in}}{%
\pgfpathmoveto{\pgfqpoint{0.000000in}{0.000000in}}%
\pgfpathlineto{\pgfqpoint{0.000000in}{-0.041667in}}%
\pgfusepath{stroke,fill}%
}%
\begin{pgfscope}%
\pgfsys@transformshift{5.153195in}{0.893003in}%
\pgfsys@useobject{currentmarker}{}%
\end{pgfscope}%
\end{pgfscope}%
\begin{pgfscope}%
\definecolor{textcolor}{rgb}{0.000000,0.000000,0.000000}%
\pgfsetstrokecolor{textcolor}%
\pgfsetfillcolor{textcolor}%
\pgftext[x=4.695668in, y=0.220556in, left, base,rotate=30.000000]{\color{textcolor}\rmfamily\fontsize{7.000000}{8.400000}\selectfont 2022-02-23}%
\end{pgfscope}%
\begin{pgfscope}%
\pgfsetbuttcap%
\pgfsetroundjoin%
\definecolor{currentfill}{rgb}{0.000000,0.000000,0.000000}%
\pgfsetfillcolor{currentfill}%
\pgfsetlinewidth{0.501875pt}%
\definecolor{currentstroke}{rgb}{0.000000,0.000000,0.000000}%
\pgfsetstrokecolor{currentstroke}%
\pgfsetdash{}{0pt}%
\pgfsys@defobject{currentmarker}{\pgfqpoint{0.000000in}{0.000000in}}{\pgfqpoint{0.000000in}{0.041667in}}{%
\pgfpathmoveto{\pgfqpoint{0.000000in}{0.000000in}}%
\pgfpathlineto{\pgfqpoint{0.000000in}{0.041667in}}%
\pgfusepath{stroke,fill}%
}%
\begin{pgfscope}%
\pgfsys@transformshift{5.767268in}{0.586309in}%
\pgfsys@useobject{currentmarker}{}%
\end{pgfscope}%
\end{pgfscope}%
\begin{pgfscope}%
\pgfsetbuttcap%
\pgfsetroundjoin%
\definecolor{currentfill}{rgb}{0.000000,0.000000,0.000000}%
\pgfsetfillcolor{currentfill}%
\pgfsetlinewidth{0.501875pt}%
\definecolor{currentstroke}{rgb}{0.000000,0.000000,0.000000}%
\pgfsetstrokecolor{currentstroke}%
\pgfsetdash{}{0pt}%
\pgfsys@defobject{currentmarker}{\pgfqpoint{0.000000in}{-0.041667in}}{\pgfqpoint{0.000000in}{0.000000in}}{%
\pgfpathmoveto{\pgfqpoint{0.000000in}{0.000000in}}%
\pgfpathlineto{\pgfqpoint{0.000000in}{-0.041667in}}%
\pgfusepath{stroke,fill}%
}%
\begin{pgfscope}%
\pgfsys@transformshift{5.767268in}{0.893003in}%
\pgfsys@useobject{currentmarker}{}%
\end{pgfscope}%
\end{pgfscope}%
\begin{pgfscope}%
\definecolor{textcolor}{rgb}{0.000000,0.000000,0.000000}%
\pgfsetstrokecolor{textcolor}%
\pgfsetfillcolor{textcolor}%
\pgftext[x=5.309741in, y=0.220556in, left, base,rotate=30.000000]{\color{textcolor}\rmfamily\fontsize{7.000000}{8.400000}\selectfont 2022-03-25}%
\end{pgfscope}%
\begin{pgfscope}%
\pgfsetbuttcap%
\pgfsetroundjoin%
\definecolor{currentfill}{rgb}{0.000000,0.000000,0.000000}%
\pgfsetfillcolor{currentfill}%
\pgfsetlinewidth{0.501875pt}%
\definecolor{currentstroke}{rgb}{0.000000,0.000000,0.000000}%
\pgfsetstrokecolor{currentstroke}%
\pgfsetdash{}{0pt}%
\pgfsys@defobject{currentmarker}{\pgfqpoint{0.000000in}{0.000000in}}{\pgfqpoint{0.000000in}{0.020833in}}{%
\pgfpathmoveto{\pgfqpoint{0.000000in}{0.000000in}}%
\pgfpathlineto{\pgfqpoint{0.000000in}{0.020833in}}%
\pgfusepath{stroke,fill}%
}%
\begin{pgfscope}%
\pgfsys@transformshift{0.547651in}{0.586309in}%
\pgfsys@useobject{currentmarker}{}%
\end{pgfscope}%
\end{pgfscope}%
\begin{pgfscope}%
\pgfsetbuttcap%
\pgfsetroundjoin%
\definecolor{currentfill}{rgb}{0.000000,0.000000,0.000000}%
\pgfsetfillcolor{currentfill}%
\pgfsetlinewidth{0.501875pt}%
\definecolor{currentstroke}{rgb}{0.000000,0.000000,0.000000}%
\pgfsetstrokecolor{currentstroke}%
\pgfsetdash{}{0pt}%
\pgfsys@defobject{currentmarker}{\pgfqpoint{0.000000in}{-0.020833in}}{\pgfqpoint{0.000000in}{0.000000in}}{%
\pgfpathmoveto{\pgfqpoint{0.000000in}{0.000000in}}%
\pgfpathlineto{\pgfqpoint{0.000000in}{-0.020833in}}%
\pgfusepath{stroke,fill}%
}%
\begin{pgfscope}%
\pgfsys@transformshift{0.547651in}{0.893003in}%
\pgfsys@useobject{currentmarker}{}%
\end{pgfscope}%
\end{pgfscope}%
\begin{pgfscope}%
\pgfsetbuttcap%
\pgfsetroundjoin%
\definecolor{currentfill}{rgb}{0.000000,0.000000,0.000000}%
\pgfsetfillcolor{currentfill}%
\pgfsetlinewidth{0.501875pt}%
\definecolor{currentstroke}{rgb}{0.000000,0.000000,0.000000}%
\pgfsetstrokecolor{currentstroke}%
\pgfsetdash{}{0pt}%
\pgfsys@defobject{currentmarker}{\pgfqpoint{0.000000in}{0.000000in}}{\pgfqpoint{0.000000in}{0.020833in}}{%
\pgfpathmoveto{\pgfqpoint{0.000000in}{0.000000in}}%
\pgfpathlineto{\pgfqpoint{0.000000in}{0.020833in}}%
\pgfusepath{stroke,fill}%
}%
\begin{pgfscope}%
\pgfsys@transformshift{0.649996in}{0.586309in}%
\pgfsys@useobject{currentmarker}{}%
\end{pgfscope}%
\end{pgfscope}%
\begin{pgfscope}%
\pgfsetbuttcap%
\pgfsetroundjoin%
\definecolor{currentfill}{rgb}{0.000000,0.000000,0.000000}%
\pgfsetfillcolor{currentfill}%
\pgfsetlinewidth{0.501875pt}%
\definecolor{currentstroke}{rgb}{0.000000,0.000000,0.000000}%
\pgfsetstrokecolor{currentstroke}%
\pgfsetdash{}{0pt}%
\pgfsys@defobject{currentmarker}{\pgfqpoint{0.000000in}{-0.020833in}}{\pgfqpoint{0.000000in}{0.000000in}}{%
\pgfpathmoveto{\pgfqpoint{0.000000in}{0.000000in}}%
\pgfpathlineto{\pgfqpoint{0.000000in}{-0.020833in}}%
\pgfusepath{stroke,fill}%
}%
\begin{pgfscope}%
\pgfsys@transformshift{0.649996in}{0.893003in}%
\pgfsys@useobject{currentmarker}{}%
\end{pgfscope}%
\end{pgfscope}%
\begin{pgfscope}%
\pgfsetbuttcap%
\pgfsetroundjoin%
\definecolor{currentfill}{rgb}{0.000000,0.000000,0.000000}%
\pgfsetfillcolor{currentfill}%
\pgfsetlinewidth{0.501875pt}%
\definecolor{currentstroke}{rgb}{0.000000,0.000000,0.000000}%
\pgfsetstrokecolor{currentstroke}%
\pgfsetdash{}{0pt}%
\pgfsys@defobject{currentmarker}{\pgfqpoint{0.000000in}{0.000000in}}{\pgfqpoint{0.000000in}{0.020833in}}{%
\pgfpathmoveto{\pgfqpoint{0.000000in}{0.000000in}}%
\pgfpathlineto{\pgfqpoint{0.000000in}{0.020833in}}%
\pgfusepath{stroke,fill}%
}%
\begin{pgfscope}%
\pgfsys@transformshift{0.752342in}{0.586309in}%
\pgfsys@useobject{currentmarker}{}%
\end{pgfscope}%
\end{pgfscope}%
\begin{pgfscope}%
\pgfsetbuttcap%
\pgfsetroundjoin%
\definecolor{currentfill}{rgb}{0.000000,0.000000,0.000000}%
\pgfsetfillcolor{currentfill}%
\pgfsetlinewidth{0.501875pt}%
\definecolor{currentstroke}{rgb}{0.000000,0.000000,0.000000}%
\pgfsetstrokecolor{currentstroke}%
\pgfsetdash{}{0pt}%
\pgfsys@defobject{currentmarker}{\pgfqpoint{0.000000in}{-0.020833in}}{\pgfqpoint{0.000000in}{0.000000in}}{%
\pgfpathmoveto{\pgfqpoint{0.000000in}{0.000000in}}%
\pgfpathlineto{\pgfqpoint{0.000000in}{-0.020833in}}%
\pgfusepath{stroke,fill}%
}%
\begin{pgfscope}%
\pgfsys@transformshift{0.752342in}{0.893003in}%
\pgfsys@useobject{currentmarker}{}%
\end{pgfscope}%
\end{pgfscope}%
\begin{pgfscope}%
\pgfsetbuttcap%
\pgfsetroundjoin%
\definecolor{currentfill}{rgb}{0.000000,0.000000,0.000000}%
\pgfsetfillcolor{currentfill}%
\pgfsetlinewidth{0.501875pt}%
\definecolor{currentstroke}{rgb}{0.000000,0.000000,0.000000}%
\pgfsetstrokecolor{currentstroke}%
\pgfsetdash{}{0pt}%
\pgfsys@defobject{currentmarker}{\pgfqpoint{0.000000in}{0.000000in}}{\pgfqpoint{0.000000in}{0.020833in}}{%
\pgfpathmoveto{\pgfqpoint{0.000000in}{0.000000in}}%
\pgfpathlineto{\pgfqpoint{0.000000in}{0.020833in}}%
\pgfusepath{stroke,fill}%
}%
\begin{pgfscope}%
\pgfsys@transformshift{0.957033in}{0.586309in}%
\pgfsys@useobject{currentmarker}{}%
\end{pgfscope}%
\end{pgfscope}%
\begin{pgfscope}%
\pgfsetbuttcap%
\pgfsetroundjoin%
\definecolor{currentfill}{rgb}{0.000000,0.000000,0.000000}%
\pgfsetfillcolor{currentfill}%
\pgfsetlinewidth{0.501875pt}%
\definecolor{currentstroke}{rgb}{0.000000,0.000000,0.000000}%
\pgfsetstrokecolor{currentstroke}%
\pgfsetdash{}{0pt}%
\pgfsys@defobject{currentmarker}{\pgfqpoint{0.000000in}{-0.020833in}}{\pgfqpoint{0.000000in}{0.000000in}}{%
\pgfpathmoveto{\pgfqpoint{0.000000in}{0.000000in}}%
\pgfpathlineto{\pgfqpoint{0.000000in}{-0.020833in}}%
\pgfusepath{stroke,fill}%
}%
\begin{pgfscope}%
\pgfsys@transformshift{0.957033in}{0.893003in}%
\pgfsys@useobject{currentmarker}{}%
\end{pgfscope}%
\end{pgfscope}%
\begin{pgfscope}%
\pgfsetbuttcap%
\pgfsetroundjoin%
\definecolor{currentfill}{rgb}{0.000000,0.000000,0.000000}%
\pgfsetfillcolor{currentfill}%
\pgfsetlinewidth{0.501875pt}%
\definecolor{currentstroke}{rgb}{0.000000,0.000000,0.000000}%
\pgfsetstrokecolor{currentstroke}%
\pgfsetdash{}{0pt}%
\pgfsys@defobject{currentmarker}{\pgfqpoint{0.000000in}{0.000000in}}{\pgfqpoint{0.000000in}{0.020833in}}{%
\pgfpathmoveto{\pgfqpoint{0.000000in}{0.000000in}}%
\pgfpathlineto{\pgfqpoint{0.000000in}{0.020833in}}%
\pgfusepath{stroke,fill}%
}%
\begin{pgfscope}%
\pgfsys@transformshift{1.059378in}{0.586309in}%
\pgfsys@useobject{currentmarker}{}%
\end{pgfscope}%
\end{pgfscope}%
\begin{pgfscope}%
\pgfsetbuttcap%
\pgfsetroundjoin%
\definecolor{currentfill}{rgb}{0.000000,0.000000,0.000000}%
\pgfsetfillcolor{currentfill}%
\pgfsetlinewidth{0.501875pt}%
\definecolor{currentstroke}{rgb}{0.000000,0.000000,0.000000}%
\pgfsetstrokecolor{currentstroke}%
\pgfsetdash{}{0pt}%
\pgfsys@defobject{currentmarker}{\pgfqpoint{0.000000in}{-0.020833in}}{\pgfqpoint{0.000000in}{0.000000in}}{%
\pgfpathmoveto{\pgfqpoint{0.000000in}{0.000000in}}%
\pgfpathlineto{\pgfqpoint{0.000000in}{-0.020833in}}%
\pgfusepath{stroke,fill}%
}%
\begin{pgfscope}%
\pgfsys@transformshift{1.059378in}{0.893003in}%
\pgfsys@useobject{currentmarker}{}%
\end{pgfscope}%
\end{pgfscope}%
\begin{pgfscope}%
\pgfsetbuttcap%
\pgfsetroundjoin%
\definecolor{currentfill}{rgb}{0.000000,0.000000,0.000000}%
\pgfsetfillcolor{currentfill}%
\pgfsetlinewidth{0.501875pt}%
\definecolor{currentstroke}{rgb}{0.000000,0.000000,0.000000}%
\pgfsetstrokecolor{currentstroke}%
\pgfsetdash{}{0pt}%
\pgfsys@defobject{currentmarker}{\pgfqpoint{0.000000in}{0.000000in}}{\pgfqpoint{0.000000in}{0.020833in}}{%
\pgfpathmoveto{\pgfqpoint{0.000000in}{0.000000in}}%
\pgfpathlineto{\pgfqpoint{0.000000in}{0.020833in}}%
\pgfusepath{stroke,fill}%
}%
\begin{pgfscope}%
\pgfsys@transformshift{1.161724in}{0.586309in}%
\pgfsys@useobject{currentmarker}{}%
\end{pgfscope}%
\end{pgfscope}%
\begin{pgfscope}%
\pgfsetbuttcap%
\pgfsetroundjoin%
\definecolor{currentfill}{rgb}{0.000000,0.000000,0.000000}%
\pgfsetfillcolor{currentfill}%
\pgfsetlinewidth{0.501875pt}%
\definecolor{currentstroke}{rgb}{0.000000,0.000000,0.000000}%
\pgfsetstrokecolor{currentstroke}%
\pgfsetdash{}{0pt}%
\pgfsys@defobject{currentmarker}{\pgfqpoint{0.000000in}{-0.020833in}}{\pgfqpoint{0.000000in}{0.000000in}}{%
\pgfpathmoveto{\pgfqpoint{0.000000in}{0.000000in}}%
\pgfpathlineto{\pgfqpoint{0.000000in}{-0.020833in}}%
\pgfusepath{stroke,fill}%
}%
\begin{pgfscope}%
\pgfsys@transformshift{1.161724in}{0.893003in}%
\pgfsys@useobject{currentmarker}{}%
\end{pgfscope}%
\end{pgfscope}%
\begin{pgfscope}%
\pgfsetbuttcap%
\pgfsetroundjoin%
\definecolor{currentfill}{rgb}{0.000000,0.000000,0.000000}%
\pgfsetfillcolor{currentfill}%
\pgfsetlinewidth{0.501875pt}%
\definecolor{currentstroke}{rgb}{0.000000,0.000000,0.000000}%
\pgfsetstrokecolor{currentstroke}%
\pgfsetdash{}{0pt}%
\pgfsys@defobject{currentmarker}{\pgfqpoint{0.000000in}{0.000000in}}{\pgfqpoint{0.000000in}{0.020833in}}{%
\pgfpathmoveto{\pgfqpoint{0.000000in}{0.000000in}}%
\pgfpathlineto{\pgfqpoint{0.000000in}{0.020833in}}%
\pgfusepath{stroke,fill}%
}%
\begin{pgfscope}%
\pgfsys@transformshift{1.264069in}{0.586309in}%
\pgfsys@useobject{currentmarker}{}%
\end{pgfscope}%
\end{pgfscope}%
\begin{pgfscope}%
\pgfsetbuttcap%
\pgfsetroundjoin%
\definecolor{currentfill}{rgb}{0.000000,0.000000,0.000000}%
\pgfsetfillcolor{currentfill}%
\pgfsetlinewidth{0.501875pt}%
\definecolor{currentstroke}{rgb}{0.000000,0.000000,0.000000}%
\pgfsetstrokecolor{currentstroke}%
\pgfsetdash{}{0pt}%
\pgfsys@defobject{currentmarker}{\pgfqpoint{0.000000in}{-0.020833in}}{\pgfqpoint{0.000000in}{0.000000in}}{%
\pgfpathmoveto{\pgfqpoint{0.000000in}{0.000000in}}%
\pgfpathlineto{\pgfqpoint{0.000000in}{-0.020833in}}%
\pgfusepath{stroke,fill}%
}%
\begin{pgfscope}%
\pgfsys@transformshift{1.264069in}{0.893003in}%
\pgfsys@useobject{currentmarker}{}%
\end{pgfscope}%
\end{pgfscope}%
\begin{pgfscope}%
\pgfsetbuttcap%
\pgfsetroundjoin%
\definecolor{currentfill}{rgb}{0.000000,0.000000,0.000000}%
\pgfsetfillcolor{currentfill}%
\pgfsetlinewidth{0.501875pt}%
\definecolor{currentstroke}{rgb}{0.000000,0.000000,0.000000}%
\pgfsetstrokecolor{currentstroke}%
\pgfsetdash{}{0pt}%
\pgfsys@defobject{currentmarker}{\pgfqpoint{0.000000in}{0.000000in}}{\pgfqpoint{0.000000in}{0.020833in}}{%
\pgfpathmoveto{\pgfqpoint{0.000000in}{0.000000in}}%
\pgfpathlineto{\pgfqpoint{0.000000in}{0.020833in}}%
\pgfusepath{stroke,fill}%
}%
\begin{pgfscope}%
\pgfsys@transformshift{1.366414in}{0.586309in}%
\pgfsys@useobject{currentmarker}{}%
\end{pgfscope}%
\end{pgfscope}%
\begin{pgfscope}%
\pgfsetbuttcap%
\pgfsetroundjoin%
\definecolor{currentfill}{rgb}{0.000000,0.000000,0.000000}%
\pgfsetfillcolor{currentfill}%
\pgfsetlinewidth{0.501875pt}%
\definecolor{currentstroke}{rgb}{0.000000,0.000000,0.000000}%
\pgfsetstrokecolor{currentstroke}%
\pgfsetdash{}{0pt}%
\pgfsys@defobject{currentmarker}{\pgfqpoint{0.000000in}{-0.020833in}}{\pgfqpoint{0.000000in}{0.000000in}}{%
\pgfpathmoveto{\pgfqpoint{0.000000in}{0.000000in}}%
\pgfpathlineto{\pgfqpoint{0.000000in}{-0.020833in}}%
\pgfusepath{stroke,fill}%
}%
\begin{pgfscope}%
\pgfsys@transformshift{1.366414in}{0.893003in}%
\pgfsys@useobject{currentmarker}{}%
\end{pgfscope}%
\end{pgfscope}%
\begin{pgfscope}%
\pgfsetbuttcap%
\pgfsetroundjoin%
\definecolor{currentfill}{rgb}{0.000000,0.000000,0.000000}%
\pgfsetfillcolor{currentfill}%
\pgfsetlinewidth{0.501875pt}%
\definecolor{currentstroke}{rgb}{0.000000,0.000000,0.000000}%
\pgfsetstrokecolor{currentstroke}%
\pgfsetdash{}{0pt}%
\pgfsys@defobject{currentmarker}{\pgfqpoint{0.000000in}{0.000000in}}{\pgfqpoint{0.000000in}{0.020833in}}{%
\pgfpathmoveto{\pgfqpoint{0.000000in}{0.000000in}}%
\pgfpathlineto{\pgfqpoint{0.000000in}{0.020833in}}%
\pgfusepath{stroke,fill}%
}%
\begin{pgfscope}%
\pgfsys@transformshift{1.571105in}{0.586309in}%
\pgfsys@useobject{currentmarker}{}%
\end{pgfscope}%
\end{pgfscope}%
\begin{pgfscope}%
\pgfsetbuttcap%
\pgfsetroundjoin%
\definecolor{currentfill}{rgb}{0.000000,0.000000,0.000000}%
\pgfsetfillcolor{currentfill}%
\pgfsetlinewidth{0.501875pt}%
\definecolor{currentstroke}{rgb}{0.000000,0.000000,0.000000}%
\pgfsetstrokecolor{currentstroke}%
\pgfsetdash{}{0pt}%
\pgfsys@defobject{currentmarker}{\pgfqpoint{0.000000in}{-0.020833in}}{\pgfqpoint{0.000000in}{0.000000in}}{%
\pgfpathmoveto{\pgfqpoint{0.000000in}{0.000000in}}%
\pgfpathlineto{\pgfqpoint{0.000000in}{-0.020833in}}%
\pgfusepath{stroke,fill}%
}%
\begin{pgfscope}%
\pgfsys@transformshift{1.571105in}{0.893003in}%
\pgfsys@useobject{currentmarker}{}%
\end{pgfscope}%
\end{pgfscope}%
\begin{pgfscope}%
\pgfsetbuttcap%
\pgfsetroundjoin%
\definecolor{currentfill}{rgb}{0.000000,0.000000,0.000000}%
\pgfsetfillcolor{currentfill}%
\pgfsetlinewidth{0.501875pt}%
\definecolor{currentstroke}{rgb}{0.000000,0.000000,0.000000}%
\pgfsetstrokecolor{currentstroke}%
\pgfsetdash{}{0pt}%
\pgfsys@defobject{currentmarker}{\pgfqpoint{0.000000in}{0.000000in}}{\pgfqpoint{0.000000in}{0.020833in}}{%
\pgfpathmoveto{\pgfqpoint{0.000000in}{0.000000in}}%
\pgfpathlineto{\pgfqpoint{0.000000in}{0.020833in}}%
\pgfusepath{stroke,fill}%
}%
\begin{pgfscope}%
\pgfsys@transformshift{1.673451in}{0.586309in}%
\pgfsys@useobject{currentmarker}{}%
\end{pgfscope}%
\end{pgfscope}%
\begin{pgfscope}%
\pgfsetbuttcap%
\pgfsetroundjoin%
\definecolor{currentfill}{rgb}{0.000000,0.000000,0.000000}%
\pgfsetfillcolor{currentfill}%
\pgfsetlinewidth{0.501875pt}%
\definecolor{currentstroke}{rgb}{0.000000,0.000000,0.000000}%
\pgfsetstrokecolor{currentstroke}%
\pgfsetdash{}{0pt}%
\pgfsys@defobject{currentmarker}{\pgfqpoint{0.000000in}{-0.020833in}}{\pgfqpoint{0.000000in}{0.000000in}}{%
\pgfpathmoveto{\pgfqpoint{0.000000in}{0.000000in}}%
\pgfpathlineto{\pgfqpoint{0.000000in}{-0.020833in}}%
\pgfusepath{stroke,fill}%
}%
\begin{pgfscope}%
\pgfsys@transformshift{1.673451in}{0.893003in}%
\pgfsys@useobject{currentmarker}{}%
\end{pgfscope}%
\end{pgfscope}%
\begin{pgfscope}%
\pgfsetbuttcap%
\pgfsetroundjoin%
\definecolor{currentfill}{rgb}{0.000000,0.000000,0.000000}%
\pgfsetfillcolor{currentfill}%
\pgfsetlinewidth{0.501875pt}%
\definecolor{currentstroke}{rgb}{0.000000,0.000000,0.000000}%
\pgfsetstrokecolor{currentstroke}%
\pgfsetdash{}{0pt}%
\pgfsys@defobject{currentmarker}{\pgfqpoint{0.000000in}{0.000000in}}{\pgfqpoint{0.000000in}{0.020833in}}{%
\pgfpathmoveto{\pgfqpoint{0.000000in}{0.000000in}}%
\pgfpathlineto{\pgfqpoint{0.000000in}{0.020833in}}%
\pgfusepath{stroke,fill}%
}%
\begin{pgfscope}%
\pgfsys@transformshift{1.775796in}{0.586309in}%
\pgfsys@useobject{currentmarker}{}%
\end{pgfscope}%
\end{pgfscope}%
\begin{pgfscope}%
\pgfsetbuttcap%
\pgfsetroundjoin%
\definecolor{currentfill}{rgb}{0.000000,0.000000,0.000000}%
\pgfsetfillcolor{currentfill}%
\pgfsetlinewidth{0.501875pt}%
\definecolor{currentstroke}{rgb}{0.000000,0.000000,0.000000}%
\pgfsetstrokecolor{currentstroke}%
\pgfsetdash{}{0pt}%
\pgfsys@defobject{currentmarker}{\pgfqpoint{0.000000in}{-0.020833in}}{\pgfqpoint{0.000000in}{0.000000in}}{%
\pgfpathmoveto{\pgfqpoint{0.000000in}{0.000000in}}%
\pgfpathlineto{\pgfqpoint{0.000000in}{-0.020833in}}%
\pgfusepath{stroke,fill}%
}%
\begin{pgfscope}%
\pgfsys@transformshift{1.775796in}{0.893003in}%
\pgfsys@useobject{currentmarker}{}%
\end{pgfscope}%
\end{pgfscope}%
\begin{pgfscope}%
\pgfsetbuttcap%
\pgfsetroundjoin%
\definecolor{currentfill}{rgb}{0.000000,0.000000,0.000000}%
\pgfsetfillcolor{currentfill}%
\pgfsetlinewidth{0.501875pt}%
\definecolor{currentstroke}{rgb}{0.000000,0.000000,0.000000}%
\pgfsetstrokecolor{currentstroke}%
\pgfsetdash{}{0pt}%
\pgfsys@defobject{currentmarker}{\pgfqpoint{0.000000in}{0.000000in}}{\pgfqpoint{0.000000in}{0.020833in}}{%
\pgfpathmoveto{\pgfqpoint{0.000000in}{0.000000in}}%
\pgfpathlineto{\pgfqpoint{0.000000in}{0.020833in}}%
\pgfusepath{stroke,fill}%
}%
\begin{pgfscope}%
\pgfsys@transformshift{1.878142in}{0.586309in}%
\pgfsys@useobject{currentmarker}{}%
\end{pgfscope}%
\end{pgfscope}%
\begin{pgfscope}%
\pgfsetbuttcap%
\pgfsetroundjoin%
\definecolor{currentfill}{rgb}{0.000000,0.000000,0.000000}%
\pgfsetfillcolor{currentfill}%
\pgfsetlinewidth{0.501875pt}%
\definecolor{currentstroke}{rgb}{0.000000,0.000000,0.000000}%
\pgfsetstrokecolor{currentstroke}%
\pgfsetdash{}{0pt}%
\pgfsys@defobject{currentmarker}{\pgfqpoint{0.000000in}{-0.020833in}}{\pgfqpoint{0.000000in}{0.000000in}}{%
\pgfpathmoveto{\pgfqpoint{0.000000in}{0.000000in}}%
\pgfpathlineto{\pgfqpoint{0.000000in}{-0.020833in}}%
\pgfusepath{stroke,fill}%
}%
\begin{pgfscope}%
\pgfsys@transformshift{1.878142in}{0.893003in}%
\pgfsys@useobject{currentmarker}{}%
\end{pgfscope}%
\end{pgfscope}%
\begin{pgfscope}%
\pgfsetbuttcap%
\pgfsetroundjoin%
\definecolor{currentfill}{rgb}{0.000000,0.000000,0.000000}%
\pgfsetfillcolor{currentfill}%
\pgfsetlinewidth{0.501875pt}%
\definecolor{currentstroke}{rgb}{0.000000,0.000000,0.000000}%
\pgfsetstrokecolor{currentstroke}%
\pgfsetdash{}{0pt}%
\pgfsys@defobject{currentmarker}{\pgfqpoint{0.000000in}{0.000000in}}{\pgfqpoint{0.000000in}{0.020833in}}{%
\pgfpathmoveto{\pgfqpoint{0.000000in}{0.000000in}}%
\pgfpathlineto{\pgfqpoint{0.000000in}{0.020833in}}%
\pgfusepath{stroke,fill}%
}%
\begin{pgfscope}%
\pgfsys@transformshift{1.980487in}{0.586309in}%
\pgfsys@useobject{currentmarker}{}%
\end{pgfscope}%
\end{pgfscope}%
\begin{pgfscope}%
\pgfsetbuttcap%
\pgfsetroundjoin%
\definecolor{currentfill}{rgb}{0.000000,0.000000,0.000000}%
\pgfsetfillcolor{currentfill}%
\pgfsetlinewidth{0.501875pt}%
\definecolor{currentstroke}{rgb}{0.000000,0.000000,0.000000}%
\pgfsetstrokecolor{currentstroke}%
\pgfsetdash{}{0pt}%
\pgfsys@defobject{currentmarker}{\pgfqpoint{0.000000in}{-0.020833in}}{\pgfqpoint{0.000000in}{0.000000in}}{%
\pgfpathmoveto{\pgfqpoint{0.000000in}{0.000000in}}%
\pgfpathlineto{\pgfqpoint{0.000000in}{-0.020833in}}%
\pgfusepath{stroke,fill}%
}%
\begin{pgfscope}%
\pgfsys@transformshift{1.980487in}{0.893003in}%
\pgfsys@useobject{currentmarker}{}%
\end{pgfscope}%
\end{pgfscope}%
\begin{pgfscope}%
\pgfsetbuttcap%
\pgfsetroundjoin%
\definecolor{currentfill}{rgb}{0.000000,0.000000,0.000000}%
\pgfsetfillcolor{currentfill}%
\pgfsetlinewidth{0.501875pt}%
\definecolor{currentstroke}{rgb}{0.000000,0.000000,0.000000}%
\pgfsetstrokecolor{currentstroke}%
\pgfsetdash{}{0pt}%
\pgfsys@defobject{currentmarker}{\pgfqpoint{0.000000in}{0.000000in}}{\pgfqpoint{0.000000in}{0.020833in}}{%
\pgfpathmoveto{\pgfqpoint{0.000000in}{0.000000in}}%
\pgfpathlineto{\pgfqpoint{0.000000in}{0.020833in}}%
\pgfusepath{stroke,fill}%
}%
\begin{pgfscope}%
\pgfsys@transformshift{2.185178in}{0.586309in}%
\pgfsys@useobject{currentmarker}{}%
\end{pgfscope}%
\end{pgfscope}%
\begin{pgfscope}%
\pgfsetbuttcap%
\pgfsetroundjoin%
\definecolor{currentfill}{rgb}{0.000000,0.000000,0.000000}%
\pgfsetfillcolor{currentfill}%
\pgfsetlinewidth{0.501875pt}%
\definecolor{currentstroke}{rgb}{0.000000,0.000000,0.000000}%
\pgfsetstrokecolor{currentstroke}%
\pgfsetdash{}{0pt}%
\pgfsys@defobject{currentmarker}{\pgfqpoint{0.000000in}{-0.020833in}}{\pgfqpoint{0.000000in}{0.000000in}}{%
\pgfpathmoveto{\pgfqpoint{0.000000in}{0.000000in}}%
\pgfpathlineto{\pgfqpoint{0.000000in}{-0.020833in}}%
\pgfusepath{stroke,fill}%
}%
\begin{pgfscope}%
\pgfsys@transformshift{2.185178in}{0.893003in}%
\pgfsys@useobject{currentmarker}{}%
\end{pgfscope}%
\end{pgfscope}%
\begin{pgfscope}%
\pgfsetbuttcap%
\pgfsetroundjoin%
\definecolor{currentfill}{rgb}{0.000000,0.000000,0.000000}%
\pgfsetfillcolor{currentfill}%
\pgfsetlinewidth{0.501875pt}%
\definecolor{currentstroke}{rgb}{0.000000,0.000000,0.000000}%
\pgfsetstrokecolor{currentstroke}%
\pgfsetdash{}{0pt}%
\pgfsys@defobject{currentmarker}{\pgfqpoint{0.000000in}{0.000000in}}{\pgfqpoint{0.000000in}{0.020833in}}{%
\pgfpathmoveto{\pgfqpoint{0.000000in}{0.000000in}}%
\pgfpathlineto{\pgfqpoint{0.000000in}{0.020833in}}%
\pgfusepath{stroke,fill}%
}%
\begin{pgfscope}%
\pgfsys@transformshift{2.287523in}{0.586309in}%
\pgfsys@useobject{currentmarker}{}%
\end{pgfscope}%
\end{pgfscope}%
\begin{pgfscope}%
\pgfsetbuttcap%
\pgfsetroundjoin%
\definecolor{currentfill}{rgb}{0.000000,0.000000,0.000000}%
\pgfsetfillcolor{currentfill}%
\pgfsetlinewidth{0.501875pt}%
\definecolor{currentstroke}{rgb}{0.000000,0.000000,0.000000}%
\pgfsetstrokecolor{currentstroke}%
\pgfsetdash{}{0pt}%
\pgfsys@defobject{currentmarker}{\pgfqpoint{0.000000in}{-0.020833in}}{\pgfqpoint{0.000000in}{0.000000in}}{%
\pgfpathmoveto{\pgfqpoint{0.000000in}{0.000000in}}%
\pgfpathlineto{\pgfqpoint{0.000000in}{-0.020833in}}%
\pgfusepath{stroke,fill}%
}%
\begin{pgfscope}%
\pgfsys@transformshift{2.287523in}{0.893003in}%
\pgfsys@useobject{currentmarker}{}%
\end{pgfscope}%
\end{pgfscope}%
\begin{pgfscope}%
\pgfsetbuttcap%
\pgfsetroundjoin%
\definecolor{currentfill}{rgb}{0.000000,0.000000,0.000000}%
\pgfsetfillcolor{currentfill}%
\pgfsetlinewidth{0.501875pt}%
\definecolor{currentstroke}{rgb}{0.000000,0.000000,0.000000}%
\pgfsetstrokecolor{currentstroke}%
\pgfsetdash{}{0pt}%
\pgfsys@defobject{currentmarker}{\pgfqpoint{0.000000in}{0.000000in}}{\pgfqpoint{0.000000in}{0.020833in}}{%
\pgfpathmoveto{\pgfqpoint{0.000000in}{0.000000in}}%
\pgfpathlineto{\pgfqpoint{0.000000in}{0.020833in}}%
\pgfusepath{stroke,fill}%
}%
\begin{pgfscope}%
\pgfsys@transformshift{2.389869in}{0.586309in}%
\pgfsys@useobject{currentmarker}{}%
\end{pgfscope}%
\end{pgfscope}%
\begin{pgfscope}%
\pgfsetbuttcap%
\pgfsetroundjoin%
\definecolor{currentfill}{rgb}{0.000000,0.000000,0.000000}%
\pgfsetfillcolor{currentfill}%
\pgfsetlinewidth{0.501875pt}%
\definecolor{currentstroke}{rgb}{0.000000,0.000000,0.000000}%
\pgfsetstrokecolor{currentstroke}%
\pgfsetdash{}{0pt}%
\pgfsys@defobject{currentmarker}{\pgfqpoint{0.000000in}{-0.020833in}}{\pgfqpoint{0.000000in}{0.000000in}}{%
\pgfpathmoveto{\pgfqpoint{0.000000in}{0.000000in}}%
\pgfpathlineto{\pgfqpoint{0.000000in}{-0.020833in}}%
\pgfusepath{stroke,fill}%
}%
\begin{pgfscope}%
\pgfsys@transformshift{2.389869in}{0.893003in}%
\pgfsys@useobject{currentmarker}{}%
\end{pgfscope}%
\end{pgfscope}%
\begin{pgfscope}%
\pgfsetbuttcap%
\pgfsetroundjoin%
\definecolor{currentfill}{rgb}{0.000000,0.000000,0.000000}%
\pgfsetfillcolor{currentfill}%
\pgfsetlinewidth{0.501875pt}%
\definecolor{currentstroke}{rgb}{0.000000,0.000000,0.000000}%
\pgfsetstrokecolor{currentstroke}%
\pgfsetdash{}{0pt}%
\pgfsys@defobject{currentmarker}{\pgfqpoint{0.000000in}{0.000000in}}{\pgfqpoint{0.000000in}{0.020833in}}{%
\pgfpathmoveto{\pgfqpoint{0.000000in}{0.000000in}}%
\pgfpathlineto{\pgfqpoint{0.000000in}{0.020833in}}%
\pgfusepath{stroke,fill}%
}%
\begin{pgfscope}%
\pgfsys@transformshift{2.492214in}{0.586309in}%
\pgfsys@useobject{currentmarker}{}%
\end{pgfscope}%
\end{pgfscope}%
\begin{pgfscope}%
\pgfsetbuttcap%
\pgfsetroundjoin%
\definecolor{currentfill}{rgb}{0.000000,0.000000,0.000000}%
\pgfsetfillcolor{currentfill}%
\pgfsetlinewidth{0.501875pt}%
\definecolor{currentstroke}{rgb}{0.000000,0.000000,0.000000}%
\pgfsetstrokecolor{currentstroke}%
\pgfsetdash{}{0pt}%
\pgfsys@defobject{currentmarker}{\pgfqpoint{0.000000in}{-0.020833in}}{\pgfqpoint{0.000000in}{0.000000in}}{%
\pgfpathmoveto{\pgfqpoint{0.000000in}{0.000000in}}%
\pgfpathlineto{\pgfqpoint{0.000000in}{-0.020833in}}%
\pgfusepath{stroke,fill}%
}%
\begin{pgfscope}%
\pgfsys@transformshift{2.492214in}{0.893003in}%
\pgfsys@useobject{currentmarker}{}%
\end{pgfscope}%
\end{pgfscope}%
\begin{pgfscope}%
\pgfsetbuttcap%
\pgfsetroundjoin%
\definecolor{currentfill}{rgb}{0.000000,0.000000,0.000000}%
\pgfsetfillcolor{currentfill}%
\pgfsetlinewidth{0.501875pt}%
\definecolor{currentstroke}{rgb}{0.000000,0.000000,0.000000}%
\pgfsetstrokecolor{currentstroke}%
\pgfsetdash{}{0pt}%
\pgfsys@defobject{currentmarker}{\pgfqpoint{0.000000in}{0.000000in}}{\pgfqpoint{0.000000in}{0.020833in}}{%
\pgfpathmoveto{\pgfqpoint{0.000000in}{0.000000in}}%
\pgfpathlineto{\pgfqpoint{0.000000in}{0.020833in}}%
\pgfusepath{stroke,fill}%
}%
\begin{pgfscope}%
\pgfsys@transformshift{2.594560in}{0.586309in}%
\pgfsys@useobject{currentmarker}{}%
\end{pgfscope}%
\end{pgfscope}%
\begin{pgfscope}%
\pgfsetbuttcap%
\pgfsetroundjoin%
\definecolor{currentfill}{rgb}{0.000000,0.000000,0.000000}%
\pgfsetfillcolor{currentfill}%
\pgfsetlinewidth{0.501875pt}%
\definecolor{currentstroke}{rgb}{0.000000,0.000000,0.000000}%
\pgfsetstrokecolor{currentstroke}%
\pgfsetdash{}{0pt}%
\pgfsys@defobject{currentmarker}{\pgfqpoint{0.000000in}{-0.020833in}}{\pgfqpoint{0.000000in}{0.000000in}}{%
\pgfpathmoveto{\pgfqpoint{0.000000in}{0.000000in}}%
\pgfpathlineto{\pgfqpoint{0.000000in}{-0.020833in}}%
\pgfusepath{stroke,fill}%
}%
\begin{pgfscope}%
\pgfsys@transformshift{2.594560in}{0.893003in}%
\pgfsys@useobject{currentmarker}{}%
\end{pgfscope}%
\end{pgfscope}%
\begin{pgfscope}%
\pgfsetbuttcap%
\pgfsetroundjoin%
\definecolor{currentfill}{rgb}{0.000000,0.000000,0.000000}%
\pgfsetfillcolor{currentfill}%
\pgfsetlinewidth{0.501875pt}%
\definecolor{currentstroke}{rgb}{0.000000,0.000000,0.000000}%
\pgfsetstrokecolor{currentstroke}%
\pgfsetdash{}{0pt}%
\pgfsys@defobject{currentmarker}{\pgfqpoint{0.000000in}{0.000000in}}{\pgfqpoint{0.000000in}{0.020833in}}{%
\pgfpathmoveto{\pgfqpoint{0.000000in}{0.000000in}}%
\pgfpathlineto{\pgfqpoint{0.000000in}{0.020833in}}%
\pgfusepath{stroke,fill}%
}%
\begin{pgfscope}%
\pgfsys@transformshift{2.799250in}{0.586309in}%
\pgfsys@useobject{currentmarker}{}%
\end{pgfscope}%
\end{pgfscope}%
\begin{pgfscope}%
\pgfsetbuttcap%
\pgfsetroundjoin%
\definecolor{currentfill}{rgb}{0.000000,0.000000,0.000000}%
\pgfsetfillcolor{currentfill}%
\pgfsetlinewidth{0.501875pt}%
\definecolor{currentstroke}{rgb}{0.000000,0.000000,0.000000}%
\pgfsetstrokecolor{currentstroke}%
\pgfsetdash{}{0pt}%
\pgfsys@defobject{currentmarker}{\pgfqpoint{0.000000in}{-0.020833in}}{\pgfqpoint{0.000000in}{0.000000in}}{%
\pgfpathmoveto{\pgfqpoint{0.000000in}{0.000000in}}%
\pgfpathlineto{\pgfqpoint{0.000000in}{-0.020833in}}%
\pgfusepath{stroke,fill}%
}%
\begin{pgfscope}%
\pgfsys@transformshift{2.799250in}{0.893003in}%
\pgfsys@useobject{currentmarker}{}%
\end{pgfscope}%
\end{pgfscope}%
\begin{pgfscope}%
\pgfsetbuttcap%
\pgfsetroundjoin%
\definecolor{currentfill}{rgb}{0.000000,0.000000,0.000000}%
\pgfsetfillcolor{currentfill}%
\pgfsetlinewidth{0.501875pt}%
\definecolor{currentstroke}{rgb}{0.000000,0.000000,0.000000}%
\pgfsetstrokecolor{currentstroke}%
\pgfsetdash{}{0pt}%
\pgfsys@defobject{currentmarker}{\pgfqpoint{0.000000in}{0.000000in}}{\pgfqpoint{0.000000in}{0.020833in}}{%
\pgfpathmoveto{\pgfqpoint{0.000000in}{0.000000in}}%
\pgfpathlineto{\pgfqpoint{0.000000in}{0.020833in}}%
\pgfusepath{stroke,fill}%
}%
\begin{pgfscope}%
\pgfsys@transformshift{2.901596in}{0.586309in}%
\pgfsys@useobject{currentmarker}{}%
\end{pgfscope}%
\end{pgfscope}%
\begin{pgfscope}%
\pgfsetbuttcap%
\pgfsetroundjoin%
\definecolor{currentfill}{rgb}{0.000000,0.000000,0.000000}%
\pgfsetfillcolor{currentfill}%
\pgfsetlinewidth{0.501875pt}%
\definecolor{currentstroke}{rgb}{0.000000,0.000000,0.000000}%
\pgfsetstrokecolor{currentstroke}%
\pgfsetdash{}{0pt}%
\pgfsys@defobject{currentmarker}{\pgfqpoint{0.000000in}{-0.020833in}}{\pgfqpoint{0.000000in}{0.000000in}}{%
\pgfpathmoveto{\pgfqpoint{0.000000in}{0.000000in}}%
\pgfpathlineto{\pgfqpoint{0.000000in}{-0.020833in}}%
\pgfusepath{stroke,fill}%
}%
\begin{pgfscope}%
\pgfsys@transformshift{2.901596in}{0.893003in}%
\pgfsys@useobject{currentmarker}{}%
\end{pgfscope}%
\end{pgfscope}%
\begin{pgfscope}%
\pgfsetbuttcap%
\pgfsetroundjoin%
\definecolor{currentfill}{rgb}{0.000000,0.000000,0.000000}%
\pgfsetfillcolor{currentfill}%
\pgfsetlinewidth{0.501875pt}%
\definecolor{currentstroke}{rgb}{0.000000,0.000000,0.000000}%
\pgfsetstrokecolor{currentstroke}%
\pgfsetdash{}{0pt}%
\pgfsys@defobject{currentmarker}{\pgfqpoint{0.000000in}{0.000000in}}{\pgfqpoint{0.000000in}{0.020833in}}{%
\pgfpathmoveto{\pgfqpoint{0.000000in}{0.000000in}}%
\pgfpathlineto{\pgfqpoint{0.000000in}{0.020833in}}%
\pgfusepath{stroke,fill}%
}%
\begin{pgfscope}%
\pgfsys@transformshift{3.003941in}{0.586309in}%
\pgfsys@useobject{currentmarker}{}%
\end{pgfscope}%
\end{pgfscope}%
\begin{pgfscope}%
\pgfsetbuttcap%
\pgfsetroundjoin%
\definecolor{currentfill}{rgb}{0.000000,0.000000,0.000000}%
\pgfsetfillcolor{currentfill}%
\pgfsetlinewidth{0.501875pt}%
\definecolor{currentstroke}{rgb}{0.000000,0.000000,0.000000}%
\pgfsetstrokecolor{currentstroke}%
\pgfsetdash{}{0pt}%
\pgfsys@defobject{currentmarker}{\pgfqpoint{0.000000in}{-0.020833in}}{\pgfqpoint{0.000000in}{0.000000in}}{%
\pgfpathmoveto{\pgfqpoint{0.000000in}{0.000000in}}%
\pgfpathlineto{\pgfqpoint{0.000000in}{-0.020833in}}%
\pgfusepath{stroke,fill}%
}%
\begin{pgfscope}%
\pgfsys@transformshift{3.003941in}{0.893003in}%
\pgfsys@useobject{currentmarker}{}%
\end{pgfscope}%
\end{pgfscope}%
\begin{pgfscope}%
\pgfsetbuttcap%
\pgfsetroundjoin%
\definecolor{currentfill}{rgb}{0.000000,0.000000,0.000000}%
\pgfsetfillcolor{currentfill}%
\pgfsetlinewidth{0.501875pt}%
\definecolor{currentstroke}{rgb}{0.000000,0.000000,0.000000}%
\pgfsetstrokecolor{currentstroke}%
\pgfsetdash{}{0pt}%
\pgfsys@defobject{currentmarker}{\pgfqpoint{0.000000in}{0.000000in}}{\pgfqpoint{0.000000in}{0.020833in}}{%
\pgfpathmoveto{\pgfqpoint{0.000000in}{0.000000in}}%
\pgfpathlineto{\pgfqpoint{0.000000in}{0.020833in}}%
\pgfusepath{stroke,fill}%
}%
\begin{pgfscope}%
\pgfsys@transformshift{3.106287in}{0.586309in}%
\pgfsys@useobject{currentmarker}{}%
\end{pgfscope}%
\end{pgfscope}%
\begin{pgfscope}%
\pgfsetbuttcap%
\pgfsetroundjoin%
\definecolor{currentfill}{rgb}{0.000000,0.000000,0.000000}%
\pgfsetfillcolor{currentfill}%
\pgfsetlinewidth{0.501875pt}%
\definecolor{currentstroke}{rgb}{0.000000,0.000000,0.000000}%
\pgfsetstrokecolor{currentstroke}%
\pgfsetdash{}{0pt}%
\pgfsys@defobject{currentmarker}{\pgfqpoint{0.000000in}{-0.020833in}}{\pgfqpoint{0.000000in}{0.000000in}}{%
\pgfpathmoveto{\pgfqpoint{0.000000in}{0.000000in}}%
\pgfpathlineto{\pgfqpoint{0.000000in}{-0.020833in}}%
\pgfusepath{stroke,fill}%
}%
\begin{pgfscope}%
\pgfsys@transformshift{3.106287in}{0.893003in}%
\pgfsys@useobject{currentmarker}{}%
\end{pgfscope}%
\end{pgfscope}%
\begin{pgfscope}%
\pgfsetbuttcap%
\pgfsetroundjoin%
\definecolor{currentfill}{rgb}{0.000000,0.000000,0.000000}%
\pgfsetfillcolor{currentfill}%
\pgfsetlinewidth{0.501875pt}%
\definecolor{currentstroke}{rgb}{0.000000,0.000000,0.000000}%
\pgfsetstrokecolor{currentstroke}%
\pgfsetdash{}{0pt}%
\pgfsys@defobject{currentmarker}{\pgfqpoint{0.000000in}{0.000000in}}{\pgfqpoint{0.000000in}{0.020833in}}{%
\pgfpathmoveto{\pgfqpoint{0.000000in}{0.000000in}}%
\pgfpathlineto{\pgfqpoint{0.000000in}{0.020833in}}%
\pgfusepath{stroke,fill}%
}%
\begin{pgfscope}%
\pgfsys@transformshift{3.208632in}{0.586309in}%
\pgfsys@useobject{currentmarker}{}%
\end{pgfscope}%
\end{pgfscope}%
\begin{pgfscope}%
\pgfsetbuttcap%
\pgfsetroundjoin%
\definecolor{currentfill}{rgb}{0.000000,0.000000,0.000000}%
\pgfsetfillcolor{currentfill}%
\pgfsetlinewidth{0.501875pt}%
\definecolor{currentstroke}{rgb}{0.000000,0.000000,0.000000}%
\pgfsetstrokecolor{currentstroke}%
\pgfsetdash{}{0pt}%
\pgfsys@defobject{currentmarker}{\pgfqpoint{0.000000in}{-0.020833in}}{\pgfqpoint{0.000000in}{0.000000in}}{%
\pgfpathmoveto{\pgfqpoint{0.000000in}{0.000000in}}%
\pgfpathlineto{\pgfqpoint{0.000000in}{-0.020833in}}%
\pgfusepath{stroke,fill}%
}%
\begin{pgfscope}%
\pgfsys@transformshift{3.208632in}{0.893003in}%
\pgfsys@useobject{currentmarker}{}%
\end{pgfscope}%
\end{pgfscope}%
\begin{pgfscope}%
\pgfsetbuttcap%
\pgfsetroundjoin%
\definecolor{currentfill}{rgb}{0.000000,0.000000,0.000000}%
\pgfsetfillcolor{currentfill}%
\pgfsetlinewidth{0.501875pt}%
\definecolor{currentstroke}{rgb}{0.000000,0.000000,0.000000}%
\pgfsetstrokecolor{currentstroke}%
\pgfsetdash{}{0pt}%
\pgfsys@defobject{currentmarker}{\pgfqpoint{0.000000in}{0.000000in}}{\pgfqpoint{0.000000in}{0.020833in}}{%
\pgfpathmoveto{\pgfqpoint{0.000000in}{0.000000in}}%
\pgfpathlineto{\pgfqpoint{0.000000in}{0.020833in}}%
\pgfusepath{stroke,fill}%
}%
\begin{pgfscope}%
\pgfsys@transformshift{3.413323in}{0.586309in}%
\pgfsys@useobject{currentmarker}{}%
\end{pgfscope}%
\end{pgfscope}%
\begin{pgfscope}%
\pgfsetbuttcap%
\pgfsetroundjoin%
\definecolor{currentfill}{rgb}{0.000000,0.000000,0.000000}%
\pgfsetfillcolor{currentfill}%
\pgfsetlinewidth{0.501875pt}%
\definecolor{currentstroke}{rgb}{0.000000,0.000000,0.000000}%
\pgfsetstrokecolor{currentstroke}%
\pgfsetdash{}{0pt}%
\pgfsys@defobject{currentmarker}{\pgfqpoint{0.000000in}{-0.020833in}}{\pgfqpoint{0.000000in}{0.000000in}}{%
\pgfpathmoveto{\pgfqpoint{0.000000in}{0.000000in}}%
\pgfpathlineto{\pgfqpoint{0.000000in}{-0.020833in}}%
\pgfusepath{stroke,fill}%
}%
\begin{pgfscope}%
\pgfsys@transformshift{3.413323in}{0.893003in}%
\pgfsys@useobject{currentmarker}{}%
\end{pgfscope}%
\end{pgfscope}%
\begin{pgfscope}%
\pgfsetbuttcap%
\pgfsetroundjoin%
\definecolor{currentfill}{rgb}{0.000000,0.000000,0.000000}%
\pgfsetfillcolor{currentfill}%
\pgfsetlinewidth{0.501875pt}%
\definecolor{currentstroke}{rgb}{0.000000,0.000000,0.000000}%
\pgfsetstrokecolor{currentstroke}%
\pgfsetdash{}{0pt}%
\pgfsys@defobject{currentmarker}{\pgfqpoint{0.000000in}{0.000000in}}{\pgfqpoint{0.000000in}{0.020833in}}{%
\pgfpathmoveto{\pgfqpoint{0.000000in}{0.000000in}}%
\pgfpathlineto{\pgfqpoint{0.000000in}{0.020833in}}%
\pgfusepath{stroke,fill}%
}%
\begin{pgfscope}%
\pgfsys@transformshift{3.515668in}{0.586309in}%
\pgfsys@useobject{currentmarker}{}%
\end{pgfscope}%
\end{pgfscope}%
\begin{pgfscope}%
\pgfsetbuttcap%
\pgfsetroundjoin%
\definecolor{currentfill}{rgb}{0.000000,0.000000,0.000000}%
\pgfsetfillcolor{currentfill}%
\pgfsetlinewidth{0.501875pt}%
\definecolor{currentstroke}{rgb}{0.000000,0.000000,0.000000}%
\pgfsetstrokecolor{currentstroke}%
\pgfsetdash{}{0pt}%
\pgfsys@defobject{currentmarker}{\pgfqpoint{0.000000in}{-0.020833in}}{\pgfqpoint{0.000000in}{0.000000in}}{%
\pgfpathmoveto{\pgfqpoint{0.000000in}{0.000000in}}%
\pgfpathlineto{\pgfqpoint{0.000000in}{-0.020833in}}%
\pgfusepath{stroke,fill}%
}%
\begin{pgfscope}%
\pgfsys@transformshift{3.515668in}{0.893003in}%
\pgfsys@useobject{currentmarker}{}%
\end{pgfscope}%
\end{pgfscope}%
\begin{pgfscope}%
\pgfsetbuttcap%
\pgfsetroundjoin%
\definecolor{currentfill}{rgb}{0.000000,0.000000,0.000000}%
\pgfsetfillcolor{currentfill}%
\pgfsetlinewidth{0.501875pt}%
\definecolor{currentstroke}{rgb}{0.000000,0.000000,0.000000}%
\pgfsetstrokecolor{currentstroke}%
\pgfsetdash{}{0pt}%
\pgfsys@defobject{currentmarker}{\pgfqpoint{0.000000in}{0.000000in}}{\pgfqpoint{0.000000in}{0.020833in}}{%
\pgfpathmoveto{\pgfqpoint{0.000000in}{0.000000in}}%
\pgfpathlineto{\pgfqpoint{0.000000in}{0.020833in}}%
\pgfusepath{stroke,fill}%
}%
\begin{pgfscope}%
\pgfsys@transformshift{3.618014in}{0.586309in}%
\pgfsys@useobject{currentmarker}{}%
\end{pgfscope}%
\end{pgfscope}%
\begin{pgfscope}%
\pgfsetbuttcap%
\pgfsetroundjoin%
\definecolor{currentfill}{rgb}{0.000000,0.000000,0.000000}%
\pgfsetfillcolor{currentfill}%
\pgfsetlinewidth{0.501875pt}%
\definecolor{currentstroke}{rgb}{0.000000,0.000000,0.000000}%
\pgfsetstrokecolor{currentstroke}%
\pgfsetdash{}{0pt}%
\pgfsys@defobject{currentmarker}{\pgfqpoint{0.000000in}{-0.020833in}}{\pgfqpoint{0.000000in}{0.000000in}}{%
\pgfpathmoveto{\pgfqpoint{0.000000in}{0.000000in}}%
\pgfpathlineto{\pgfqpoint{0.000000in}{-0.020833in}}%
\pgfusepath{stroke,fill}%
}%
\begin{pgfscope}%
\pgfsys@transformshift{3.618014in}{0.893003in}%
\pgfsys@useobject{currentmarker}{}%
\end{pgfscope}%
\end{pgfscope}%
\begin{pgfscope}%
\pgfsetbuttcap%
\pgfsetroundjoin%
\definecolor{currentfill}{rgb}{0.000000,0.000000,0.000000}%
\pgfsetfillcolor{currentfill}%
\pgfsetlinewidth{0.501875pt}%
\definecolor{currentstroke}{rgb}{0.000000,0.000000,0.000000}%
\pgfsetstrokecolor{currentstroke}%
\pgfsetdash{}{0pt}%
\pgfsys@defobject{currentmarker}{\pgfqpoint{0.000000in}{0.000000in}}{\pgfqpoint{0.000000in}{0.020833in}}{%
\pgfpathmoveto{\pgfqpoint{0.000000in}{0.000000in}}%
\pgfpathlineto{\pgfqpoint{0.000000in}{0.020833in}}%
\pgfusepath{stroke,fill}%
}%
\begin{pgfscope}%
\pgfsys@transformshift{3.720359in}{0.586309in}%
\pgfsys@useobject{currentmarker}{}%
\end{pgfscope}%
\end{pgfscope}%
\begin{pgfscope}%
\pgfsetbuttcap%
\pgfsetroundjoin%
\definecolor{currentfill}{rgb}{0.000000,0.000000,0.000000}%
\pgfsetfillcolor{currentfill}%
\pgfsetlinewidth{0.501875pt}%
\definecolor{currentstroke}{rgb}{0.000000,0.000000,0.000000}%
\pgfsetstrokecolor{currentstroke}%
\pgfsetdash{}{0pt}%
\pgfsys@defobject{currentmarker}{\pgfqpoint{0.000000in}{-0.020833in}}{\pgfqpoint{0.000000in}{0.000000in}}{%
\pgfpathmoveto{\pgfqpoint{0.000000in}{0.000000in}}%
\pgfpathlineto{\pgfqpoint{0.000000in}{-0.020833in}}%
\pgfusepath{stroke,fill}%
}%
\begin{pgfscope}%
\pgfsys@transformshift{3.720359in}{0.893003in}%
\pgfsys@useobject{currentmarker}{}%
\end{pgfscope}%
\end{pgfscope}%
\begin{pgfscope}%
\pgfsetbuttcap%
\pgfsetroundjoin%
\definecolor{currentfill}{rgb}{0.000000,0.000000,0.000000}%
\pgfsetfillcolor{currentfill}%
\pgfsetlinewidth{0.501875pt}%
\definecolor{currentstroke}{rgb}{0.000000,0.000000,0.000000}%
\pgfsetstrokecolor{currentstroke}%
\pgfsetdash{}{0pt}%
\pgfsys@defobject{currentmarker}{\pgfqpoint{0.000000in}{0.000000in}}{\pgfqpoint{0.000000in}{0.020833in}}{%
\pgfpathmoveto{\pgfqpoint{0.000000in}{0.000000in}}%
\pgfpathlineto{\pgfqpoint{0.000000in}{0.020833in}}%
\pgfusepath{stroke,fill}%
}%
\begin{pgfscope}%
\pgfsys@transformshift{3.822705in}{0.586309in}%
\pgfsys@useobject{currentmarker}{}%
\end{pgfscope}%
\end{pgfscope}%
\begin{pgfscope}%
\pgfsetbuttcap%
\pgfsetroundjoin%
\definecolor{currentfill}{rgb}{0.000000,0.000000,0.000000}%
\pgfsetfillcolor{currentfill}%
\pgfsetlinewidth{0.501875pt}%
\definecolor{currentstroke}{rgb}{0.000000,0.000000,0.000000}%
\pgfsetstrokecolor{currentstroke}%
\pgfsetdash{}{0pt}%
\pgfsys@defobject{currentmarker}{\pgfqpoint{0.000000in}{-0.020833in}}{\pgfqpoint{0.000000in}{0.000000in}}{%
\pgfpathmoveto{\pgfqpoint{0.000000in}{0.000000in}}%
\pgfpathlineto{\pgfqpoint{0.000000in}{-0.020833in}}%
\pgfusepath{stroke,fill}%
}%
\begin{pgfscope}%
\pgfsys@transformshift{3.822705in}{0.893003in}%
\pgfsys@useobject{currentmarker}{}%
\end{pgfscope}%
\end{pgfscope}%
\begin{pgfscope}%
\pgfsetbuttcap%
\pgfsetroundjoin%
\definecolor{currentfill}{rgb}{0.000000,0.000000,0.000000}%
\pgfsetfillcolor{currentfill}%
\pgfsetlinewidth{0.501875pt}%
\definecolor{currentstroke}{rgb}{0.000000,0.000000,0.000000}%
\pgfsetstrokecolor{currentstroke}%
\pgfsetdash{}{0pt}%
\pgfsys@defobject{currentmarker}{\pgfqpoint{0.000000in}{0.000000in}}{\pgfqpoint{0.000000in}{0.020833in}}{%
\pgfpathmoveto{\pgfqpoint{0.000000in}{0.000000in}}%
\pgfpathlineto{\pgfqpoint{0.000000in}{0.020833in}}%
\pgfusepath{stroke,fill}%
}%
\begin{pgfscope}%
\pgfsys@transformshift{4.027395in}{0.586309in}%
\pgfsys@useobject{currentmarker}{}%
\end{pgfscope}%
\end{pgfscope}%
\begin{pgfscope}%
\pgfsetbuttcap%
\pgfsetroundjoin%
\definecolor{currentfill}{rgb}{0.000000,0.000000,0.000000}%
\pgfsetfillcolor{currentfill}%
\pgfsetlinewidth{0.501875pt}%
\definecolor{currentstroke}{rgb}{0.000000,0.000000,0.000000}%
\pgfsetstrokecolor{currentstroke}%
\pgfsetdash{}{0pt}%
\pgfsys@defobject{currentmarker}{\pgfqpoint{0.000000in}{-0.020833in}}{\pgfqpoint{0.000000in}{0.000000in}}{%
\pgfpathmoveto{\pgfqpoint{0.000000in}{0.000000in}}%
\pgfpathlineto{\pgfqpoint{0.000000in}{-0.020833in}}%
\pgfusepath{stroke,fill}%
}%
\begin{pgfscope}%
\pgfsys@transformshift{4.027395in}{0.893003in}%
\pgfsys@useobject{currentmarker}{}%
\end{pgfscope}%
\end{pgfscope}%
\begin{pgfscope}%
\pgfsetbuttcap%
\pgfsetroundjoin%
\definecolor{currentfill}{rgb}{0.000000,0.000000,0.000000}%
\pgfsetfillcolor{currentfill}%
\pgfsetlinewidth{0.501875pt}%
\definecolor{currentstroke}{rgb}{0.000000,0.000000,0.000000}%
\pgfsetstrokecolor{currentstroke}%
\pgfsetdash{}{0pt}%
\pgfsys@defobject{currentmarker}{\pgfqpoint{0.000000in}{0.000000in}}{\pgfqpoint{0.000000in}{0.020833in}}{%
\pgfpathmoveto{\pgfqpoint{0.000000in}{0.000000in}}%
\pgfpathlineto{\pgfqpoint{0.000000in}{0.020833in}}%
\pgfusepath{stroke,fill}%
}%
\begin{pgfscope}%
\pgfsys@transformshift{4.129741in}{0.586309in}%
\pgfsys@useobject{currentmarker}{}%
\end{pgfscope}%
\end{pgfscope}%
\begin{pgfscope}%
\pgfsetbuttcap%
\pgfsetroundjoin%
\definecolor{currentfill}{rgb}{0.000000,0.000000,0.000000}%
\pgfsetfillcolor{currentfill}%
\pgfsetlinewidth{0.501875pt}%
\definecolor{currentstroke}{rgb}{0.000000,0.000000,0.000000}%
\pgfsetstrokecolor{currentstroke}%
\pgfsetdash{}{0pt}%
\pgfsys@defobject{currentmarker}{\pgfqpoint{0.000000in}{-0.020833in}}{\pgfqpoint{0.000000in}{0.000000in}}{%
\pgfpathmoveto{\pgfqpoint{0.000000in}{0.000000in}}%
\pgfpathlineto{\pgfqpoint{0.000000in}{-0.020833in}}%
\pgfusepath{stroke,fill}%
}%
\begin{pgfscope}%
\pgfsys@transformshift{4.129741in}{0.893003in}%
\pgfsys@useobject{currentmarker}{}%
\end{pgfscope}%
\end{pgfscope}%
\begin{pgfscope}%
\pgfsetbuttcap%
\pgfsetroundjoin%
\definecolor{currentfill}{rgb}{0.000000,0.000000,0.000000}%
\pgfsetfillcolor{currentfill}%
\pgfsetlinewidth{0.501875pt}%
\definecolor{currentstroke}{rgb}{0.000000,0.000000,0.000000}%
\pgfsetstrokecolor{currentstroke}%
\pgfsetdash{}{0pt}%
\pgfsys@defobject{currentmarker}{\pgfqpoint{0.000000in}{0.000000in}}{\pgfqpoint{0.000000in}{0.020833in}}{%
\pgfpathmoveto{\pgfqpoint{0.000000in}{0.000000in}}%
\pgfpathlineto{\pgfqpoint{0.000000in}{0.020833in}}%
\pgfusepath{stroke,fill}%
}%
\begin{pgfscope}%
\pgfsys@transformshift{4.232086in}{0.586309in}%
\pgfsys@useobject{currentmarker}{}%
\end{pgfscope}%
\end{pgfscope}%
\begin{pgfscope}%
\pgfsetbuttcap%
\pgfsetroundjoin%
\definecolor{currentfill}{rgb}{0.000000,0.000000,0.000000}%
\pgfsetfillcolor{currentfill}%
\pgfsetlinewidth{0.501875pt}%
\definecolor{currentstroke}{rgb}{0.000000,0.000000,0.000000}%
\pgfsetstrokecolor{currentstroke}%
\pgfsetdash{}{0pt}%
\pgfsys@defobject{currentmarker}{\pgfqpoint{0.000000in}{-0.020833in}}{\pgfqpoint{0.000000in}{0.000000in}}{%
\pgfpathmoveto{\pgfqpoint{0.000000in}{0.000000in}}%
\pgfpathlineto{\pgfqpoint{0.000000in}{-0.020833in}}%
\pgfusepath{stroke,fill}%
}%
\begin{pgfscope}%
\pgfsys@transformshift{4.232086in}{0.893003in}%
\pgfsys@useobject{currentmarker}{}%
\end{pgfscope}%
\end{pgfscope}%
\begin{pgfscope}%
\pgfsetbuttcap%
\pgfsetroundjoin%
\definecolor{currentfill}{rgb}{0.000000,0.000000,0.000000}%
\pgfsetfillcolor{currentfill}%
\pgfsetlinewidth{0.501875pt}%
\definecolor{currentstroke}{rgb}{0.000000,0.000000,0.000000}%
\pgfsetstrokecolor{currentstroke}%
\pgfsetdash{}{0pt}%
\pgfsys@defobject{currentmarker}{\pgfqpoint{0.000000in}{0.000000in}}{\pgfqpoint{0.000000in}{0.020833in}}{%
\pgfpathmoveto{\pgfqpoint{0.000000in}{0.000000in}}%
\pgfpathlineto{\pgfqpoint{0.000000in}{0.020833in}}%
\pgfusepath{stroke,fill}%
}%
\begin{pgfscope}%
\pgfsys@transformshift{4.334432in}{0.586309in}%
\pgfsys@useobject{currentmarker}{}%
\end{pgfscope}%
\end{pgfscope}%
\begin{pgfscope}%
\pgfsetbuttcap%
\pgfsetroundjoin%
\definecolor{currentfill}{rgb}{0.000000,0.000000,0.000000}%
\pgfsetfillcolor{currentfill}%
\pgfsetlinewidth{0.501875pt}%
\definecolor{currentstroke}{rgb}{0.000000,0.000000,0.000000}%
\pgfsetstrokecolor{currentstroke}%
\pgfsetdash{}{0pt}%
\pgfsys@defobject{currentmarker}{\pgfqpoint{0.000000in}{-0.020833in}}{\pgfqpoint{0.000000in}{0.000000in}}{%
\pgfpathmoveto{\pgfqpoint{0.000000in}{0.000000in}}%
\pgfpathlineto{\pgfqpoint{0.000000in}{-0.020833in}}%
\pgfusepath{stroke,fill}%
}%
\begin{pgfscope}%
\pgfsys@transformshift{4.334432in}{0.893003in}%
\pgfsys@useobject{currentmarker}{}%
\end{pgfscope}%
\end{pgfscope}%
\begin{pgfscope}%
\pgfsetbuttcap%
\pgfsetroundjoin%
\definecolor{currentfill}{rgb}{0.000000,0.000000,0.000000}%
\pgfsetfillcolor{currentfill}%
\pgfsetlinewidth{0.501875pt}%
\definecolor{currentstroke}{rgb}{0.000000,0.000000,0.000000}%
\pgfsetstrokecolor{currentstroke}%
\pgfsetdash{}{0pt}%
\pgfsys@defobject{currentmarker}{\pgfqpoint{0.000000in}{0.000000in}}{\pgfqpoint{0.000000in}{0.020833in}}{%
\pgfpathmoveto{\pgfqpoint{0.000000in}{0.000000in}}%
\pgfpathlineto{\pgfqpoint{0.000000in}{0.020833in}}%
\pgfusepath{stroke,fill}%
}%
\begin{pgfscope}%
\pgfsys@transformshift{4.436777in}{0.586309in}%
\pgfsys@useobject{currentmarker}{}%
\end{pgfscope}%
\end{pgfscope}%
\begin{pgfscope}%
\pgfsetbuttcap%
\pgfsetroundjoin%
\definecolor{currentfill}{rgb}{0.000000,0.000000,0.000000}%
\pgfsetfillcolor{currentfill}%
\pgfsetlinewidth{0.501875pt}%
\definecolor{currentstroke}{rgb}{0.000000,0.000000,0.000000}%
\pgfsetstrokecolor{currentstroke}%
\pgfsetdash{}{0pt}%
\pgfsys@defobject{currentmarker}{\pgfqpoint{0.000000in}{-0.020833in}}{\pgfqpoint{0.000000in}{0.000000in}}{%
\pgfpathmoveto{\pgfqpoint{0.000000in}{0.000000in}}%
\pgfpathlineto{\pgfqpoint{0.000000in}{-0.020833in}}%
\pgfusepath{stroke,fill}%
}%
\begin{pgfscope}%
\pgfsys@transformshift{4.436777in}{0.893003in}%
\pgfsys@useobject{currentmarker}{}%
\end{pgfscope}%
\end{pgfscope}%
\begin{pgfscope}%
\pgfsetbuttcap%
\pgfsetroundjoin%
\definecolor{currentfill}{rgb}{0.000000,0.000000,0.000000}%
\pgfsetfillcolor{currentfill}%
\pgfsetlinewidth{0.501875pt}%
\definecolor{currentstroke}{rgb}{0.000000,0.000000,0.000000}%
\pgfsetstrokecolor{currentstroke}%
\pgfsetdash{}{0pt}%
\pgfsys@defobject{currentmarker}{\pgfqpoint{0.000000in}{0.000000in}}{\pgfqpoint{0.000000in}{0.020833in}}{%
\pgfpathmoveto{\pgfqpoint{0.000000in}{0.000000in}}%
\pgfpathlineto{\pgfqpoint{0.000000in}{0.020833in}}%
\pgfusepath{stroke,fill}%
}%
\begin{pgfscope}%
\pgfsys@transformshift{4.641468in}{0.586309in}%
\pgfsys@useobject{currentmarker}{}%
\end{pgfscope}%
\end{pgfscope}%
\begin{pgfscope}%
\pgfsetbuttcap%
\pgfsetroundjoin%
\definecolor{currentfill}{rgb}{0.000000,0.000000,0.000000}%
\pgfsetfillcolor{currentfill}%
\pgfsetlinewidth{0.501875pt}%
\definecolor{currentstroke}{rgb}{0.000000,0.000000,0.000000}%
\pgfsetstrokecolor{currentstroke}%
\pgfsetdash{}{0pt}%
\pgfsys@defobject{currentmarker}{\pgfqpoint{0.000000in}{-0.020833in}}{\pgfqpoint{0.000000in}{0.000000in}}{%
\pgfpathmoveto{\pgfqpoint{0.000000in}{0.000000in}}%
\pgfpathlineto{\pgfqpoint{0.000000in}{-0.020833in}}%
\pgfusepath{stroke,fill}%
}%
\begin{pgfscope}%
\pgfsys@transformshift{4.641468in}{0.893003in}%
\pgfsys@useobject{currentmarker}{}%
\end{pgfscope}%
\end{pgfscope}%
\begin{pgfscope}%
\pgfsetbuttcap%
\pgfsetroundjoin%
\definecolor{currentfill}{rgb}{0.000000,0.000000,0.000000}%
\pgfsetfillcolor{currentfill}%
\pgfsetlinewidth{0.501875pt}%
\definecolor{currentstroke}{rgb}{0.000000,0.000000,0.000000}%
\pgfsetstrokecolor{currentstroke}%
\pgfsetdash{}{0pt}%
\pgfsys@defobject{currentmarker}{\pgfqpoint{0.000000in}{0.000000in}}{\pgfqpoint{0.000000in}{0.020833in}}{%
\pgfpathmoveto{\pgfqpoint{0.000000in}{0.000000in}}%
\pgfpathlineto{\pgfqpoint{0.000000in}{0.020833in}}%
\pgfusepath{stroke,fill}%
}%
\begin{pgfscope}%
\pgfsys@transformshift{4.743813in}{0.586309in}%
\pgfsys@useobject{currentmarker}{}%
\end{pgfscope}%
\end{pgfscope}%
\begin{pgfscope}%
\pgfsetbuttcap%
\pgfsetroundjoin%
\definecolor{currentfill}{rgb}{0.000000,0.000000,0.000000}%
\pgfsetfillcolor{currentfill}%
\pgfsetlinewidth{0.501875pt}%
\definecolor{currentstroke}{rgb}{0.000000,0.000000,0.000000}%
\pgfsetstrokecolor{currentstroke}%
\pgfsetdash{}{0pt}%
\pgfsys@defobject{currentmarker}{\pgfqpoint{0.000000in}{-0.020833in}}{\pgfqpoint{0.000000in}{0.000000in}}{%
\pgfpathmoveto{\pgfqpoint{0.000000in}{0.000000in}}%
\pgfpathlineto{\pgfqpoint{0.000000in}{-0.020833in}}%
\pgfusepath{stroke,fill}%
}%
\begin{pgfscope}%
\pgfsys@transformshift{4.743813in}{0.893003in}%
\pgfsys@useobject{currentmarker}{}%
\end{pgfscope}%
\end{pgfscope}%
\begin{pgfscope}%
\pgfsetbuttcap%
\pgfsetroundjoin%
\definecolor{currentfill}{rgb}{0.000000,0.000000,0.000000}%
\pgfsetfillcolor{currentfill}%
\pgfsetlinewidth{0.501875pt}%
\definecolor{currentstroke}{rgb}{0.000000,0.000000,0.000000}%
\pgfsetstrokecolor{currentstroke}%
\pgfsetdash{}{0pt}%
\pgfsys@defobject{currentmarker}{\pgfqpoint{0.000000in}{0.000000in}}{\pgfqpoint{0.000000in}{0.020833in}}{%
\pgfpathmoveto{\pgfqpoint{0.000000in}{0.000000in}}%
\pgfpathlineto{\pgfqpoint{0.000000in}{0.020833in}}%
\pgfusepath{stroke,fill}%
}%
\begin{pgfscope}%
\pgfsys@transformshift{4.846159in}{0.586309in}%
\pgfsys@useobject{currentmarker}{}%
\end{pgfscope}%
\end{pgfscope}%
\begin{pgfscope}%
\pgfsetbuttcap%
\pgfsetroundjoin%
\definecolor{currentfill}{rgb}{0.000000,0.000000,0.000000}%
\pgfsetfillcolor{currentfill}%
\pgfsetlinewidth{0.501875pt}%
\definecolor{currentstroke}{rgb}{0.000000,0.000000,0.000000}%
\pgfsetstrokecolor{currentstroke}%
\pgfsetdash{}{0pt}%
\pgfsys@defobject{currentmarker}{\pgfqpoint{0.000000in}{-0.020833in}}{\pgfqpoint{0.000000in}{0.000000in}}{%
\pgfpathmoveto{\pgfqpoint{0.000000in}{0.000000in}}%
\pgfpathlineto{\pgfqpoint{0.000000in}{-0.020833in}}%
\pgfusepath{stroke,fill}%
}%
\begin{pgfscope}%
\pgfsys@transformshift{4.846159in}{0.893003in}%
\pgfsys@useobject{currentmarker}{}%
\end{pgfscope}%
\end{pgfscope}%
\begin{pgfscope}%
\pgfsetbuttcap%
\pgfsetroundjoin%
\definecolor{currentfill}{rgb}{0.000000,0.000000,0.000000}%
\pgfsetfillcolor{currentfill}%
\pgfsetlinewidth{0.501875pt}%
\definecolor{currentstroke}{rgb}{0.000000,0.000000,0.000000}%
\pgfsetstrokecolor{currentstroke}%
\pgfsetdash{}{0pt}%
\pgfsys@defobject{currentmarker}{\pgfqpoint{0.000000in}{0.000000in}}{\pgfqpoint{0.000000in}{0.020833in}}{%
\pgfpathmoveto{\pgfqpoint{0.000000in}{0.000000in}}%
\pgfpathlineto{\pgfqpoint{0.000000in}{0.020833in}}%
\pgfusepath{stroke,fill}%
}%
\begin{pgfscope}%
\pgfsys@transformshift{4.948504in}{0.586309in}%
\pgfsys@useobject{currentmarker}{}%
\end{pgfscope}%
\end{pgfscope}%
\begin{pgfscope}%
\pgfsetbuttcap%
\pgfsetroundjoin%
\definecolor{currentfill}{rgb}{0.000000,0.000000,0.000000}%
\pgfsetfillcolor{currentfill}%
\pgfsetlinewidth{0.501875pt}%
\definecolor{currentstroke}{rgb}{0.000000,0.000000,0.000000}%
\pgfsetstrokecolor{currentstroke}%
\pgfsetdash{}{0pt}%
\pgfsys@defobject{currentmarker}{\pgfqpoint{0.000000in}{-0.020833in}}{\pgfqpoint{0.000000in}{0.000000in}}{%
\pgfpathmoveto{\pgfqpoint{0.000000in}{0.000000in}}%
\pgfpathlineto{\pgfqpoint{0.000000in}{-0.020833in}}%
\pgfusepath{stroke,fill}%
}%
\begin{pgfscope}%
\pgfsys@transformshift{4.948504in}{0.893003in}%
\pgfsys@useobject{currentmarker}{}%
\end{pgfscope}%
\end{pgfscope}%
\begin{pgfscope}%
\pgfsetbuttcap%
\pgfsetroundjoin%
\definecolor{currentfill}{rgb}{0.000000,0.000000,0.000000}%
\pgfsetfillcolor{currentfill}%
\pgfsetlinewidth{0.501875pt}%
\definecolor{currentstroke}{rgb}{0.000000,0.000000,0.000000}%
\pgfsetstrokecolor{currentstroke}%
\pgfsetdash{}{0pt}%
\pgfsys@defobject{currentmarker}{\pgfqpoint{0.000000in}{0.000000in}}{\pgfqpoint{0.000000in}{0.020833in}}{%
\pgfpathmoveto{\pgfqpoint{0.000000in}{0.000000in}}%
\pgfpathlineto{\pgfqpoint{0.000000in}{0.020833in}}%
\pgfusepath{stroke,fill}%
}%
\begin{pgfscope}%
\pgfsys@transformshift{5.050850in}{0.586309in}%
\pgfsys@useobject{currentmarker}{}%
\end{pgfscope}%
\end{pgfscope}%
\begin{pgfscope}%
\pgfsetbuttcap%
\pgfsetroundjoin%
\definecolor{currentfill}{rgb}{0.000000,0.000000,0.000000}%
\pgfsetfillcolor{currentfill}%
\pgfsetlinewidth{0.501875pt}%
\definecolor{currentstroke}{rgb}{0.000000,0.000000,0.000000}%
\pgfsetstrokecolor{currentstroke}%
\pgfsetdash{}{0pt}%
\pgfsys@defobject{currentmarker}{\pgfqpoint{0.000000in}{-0.020833in}}{\pgfqpoint{0.000000in}{0.000000in}}{%
\pgfpathmoveto{\pgfqpoint{0.000000in}{0.000000in}}%
\pgfpathlineto{\pgfqpoint{0.000000in}{-0.020833in}}%
\pgfusepath{stroke,fill}%
}%
\begin{pgfscope}%
\pgfsys@transformshift{5.050850in}{0.893003in}%
\pgfsys@useobject{currentmarker}{}%
\end{pgfscope}%
\end{pgfscope}%
\begin{pgfscope}%
\pgfsetbuttcap%
\pgfsetroundjoin%
\definecolor{currentfill}{rgb}{0.000000,0.000000,0.000000}%
\pgfsetfillcolor{currentfill}%
\pgfsetlinewidth{0.501875pt}%
\definecolor{currentstroke}{rgb}{0.000000,0.000000,0.000000}%
\pgfsetstrokecolor{currentstroke}%
\pgfsetdash{}{0pt}%
\pgfsys@defobject{currentmarker}{\pgfqpoint{0.000000in}{0.000000in}}{\pgfqpoint{0.000000in}{0.020833in}}{%
\pgfpathmoveto{\pgfqpoint{0.000000in}{0.000000in}}%
\pgfpathlineto{\pgfqpoint{0.000000in}{0.020833in}}%
\pgfusepath{stroke,fill}%
}%
\begin{pgfscope}%
\pgfsys@transformshift{5.255541in}{0.586309in}%
\pgfsys@useobject{currentmarker}{}%
\end{pgfscope}%
\end{pgfscope}%
\begin{pgfscope}%
\pgfsetbuttcap%
\pgfsetroundjoin%
\definecolor{currentfill}{rgb}{0.000000,0.000000,0.000000}%
\pgfsetfillcolor{currentfill}%
\pgfsetlinewidth{0.501875pt}%
\definecolor{currentstroke}{rgb}{0.000000,0.000000,0.000000}%
\pgfsetstrokecolor{currentstroke}%
\pgfsetdash{}{0pt}%
\pgfsys@defobject{currentmarker}{\pgfqpoint{0.000000in}{-0.020833in}}{\pgfqpoint{0.000000in}{0.000000in}}{%
\pgfpathmoveto{\pgfqpoint{0.000000in}{0.000000in}}%
\pgfpathlineto{\pgfqpoint{0.000000in}{-0.020833in}}%
\pgfusepath{stroke,fill}%
}%
\begin{pgfscope}%
\pgfsys@transformshift{5.255541in}{0.893003in}%
\pgfsys@useobject{currentmarker}{}%
\end{pgfscope}%
\end{pgfscope}%
\begin{pgfscope}%
\pgfsetbuttcap%
\pgfsetroundjoin%
\definecolor{currentfill}{rgb}{0.000000,0.000000,0.000000}%
\pgfsetfillcolor{currentfill}%
\pgfsetlinewidth{0.501875pt}%
\definecolor{currentstroke}{rgb}{0.000000,0.000000,0.000000}%
\pgfsetstrokecolor{currentstroke}%
\pgfsetdash{}{0pt}%
\pgfsys@defobject{currentmarker}{\pgfqpoint{0.000000in}{0.000000in}}{\pgfqpoint{0.000000in}{0.020833in}}{%
\pgfpathmoveto{\pgfqpoint{0.000000in}{0.000000in}}%
\pgfpathlineto{\pgfqpoint{0.000000in}{0.020833in}}%
\pgfusepath{stroke,fill}%
}%
\begin{pgfscope}%
\pgfsys@transformshift{5.357886in}{0.586309in}%
\pgfsys@useobject{currentmarker}{}%
\end{pgfscope}%
\end{pgfscope}%
\begin{pgfscope}%
\pgfsetbuttcap%
\pgfsetroundjoin%
\definecolor{currentfill}{rgb}{0.000000,0.000000,0.000000}%
\pgfsetfillcolor{currentfill}%
\pgfsetlinewidth{0.501875pt}%
\definecolor{currentstroke}{rgb}{0.000000,0.000000,0.000000}%
\pgfsetstrokecolor{currentstroke}%
\pgfsetdash{}{0pt}%
\pgfsys@defobject{currentmarker}{\pgfqpoint{0.000000in}{-0.020833in}}{\pgfqpoint{0.000000in}{0.000000in}}{%
\pgfpathmoveto{\pgfqpoint{0.000000in}{0.000000in}}%
\pgfpathlineto{\pgfqpoint{0.000000in}{-0.020833in}}%
\pgfusepath{stroke,fill}%
}%
\begin{pgfscope}%
\pgfsys@transformshift{5.357886in}{0.893003in}%
\pgfsys@useobject{currentmarker}{}%
\end{pgfscope}%
\end{pgfscope}%
\begin{pgfscope}%
\pgfsetbuttcap%
\pgfsetroundjoin%
\definecolor{currentfill}{rgb}{0.000000,0.000000,0.000000}%
\pgfsetfillcolor{currentfill}%
\pgfsetlinewidth{0.501875pt}%
\definecolor{currentstroke}{rgb}{0.000000,0.000000,0.000000}%
\pgfsetstrokecolor{currentstroke}%
\pgfsetdash{}{0pt}%
\pgfsys@defobject{currentmarker}{\pgfqpoint{0.000000in}{0.000000in}}{\pgfqpoint{0.000000in}{0.020833in}}{%
\pgfpathmoveto{\pgfqpoint{0.000000in}{0.000000in}}%
\pgfpathlineto{\pgfqpoint{0.000000in}{0.020833in}}%
\pgfusepath{stroke,fill}%
}%
\begin{pgfscope}%
\pgfsys@transformshift{5.460231in}{0.586309in}%
\pgfsys@useobject{currentmarker}{}%
\end{pgfscope}%
\end{pgfscope}%
\begin{pgfscope}%
\pgfsetbuttcap%
\pgfsetroundjoin%
\definecolor{currentfill}{rgb}{0.000000,0.000000,0.000000}%
\pgfsetfillcolor{currentfill}%
\pgfsetlinewidth{0.501875pt}%
\definecolor{currentstroke}{rgb}{0.000000,0.000000,0.000000}%
\pgfsetstrokecolor{currentstroke}%
\pgfsetdash{}{0pt}%
\pgfsys@defobject{currentmarker}{\pgfqpoint{0.000000in}{-0.020833in}}{\pgfqpoint{0.000000in}{0.000000in}}{%
\pgfpathmoveto{\pgfqpoint{0.000000in}{0.000000in}}%
\pgfpathlineto{\pgfqpoint{0.000000in}{-0.020833in}}%
\pgfusepath{stroke,fill}%
}%
\begin{pgfscope}%
\pgfsys@transformshift{5.460231in}{0.893003in}%
\pgfsys@useobject{currentmarker}{}%
\end{pgfscope}%
\end{pgfscope}%
\begin{pgfscope}%
\pgfsetbuttcap%
\pgfsetroundjoin%
\definecolor{currentfill}{rgb}{0.000000,0.000000,0.000000}%
\pgfsetfillcolor{currentfill}%
\pgfsetlinewidth{0.501875pt}%
\definecolor{currentstroke}{rgb}{0.000000,0.000000,0.000000}%
\pgfsetstrokecolor{currentstroke}%
\pgfsetdash{}{0pt}%
\pgfsys@defobject{currentmarker}{\pgfqpoint{0.000000in}{0.000000in}}{\pgfqpoint{0.000000in}{0.020833in}}{%
\pgfpathmoveto{\pgfqpoint{0.000000in}{0.000000in}}%
\pgfpathlineto{\pgfqpoint{0.000000in}{0.020833in}}%
\pgfusepath{stroke,fill}%
}%
\begin{pgfscope}%
\pgfsys@transformshift{5.562577in}{0.586309in}%
\pgfsys@useobject{currentmarker}{}%
\end{pgfscope}%
\end{pgfscope}%
\begin{pgfscope}%
\pgfsetbuttcap%
\pgfsetroundjoin%
\definecolor{currentfill}{rgb}{0.000000,0.000000,0.000000}%
\pgfsetfillcolor{currentfill}%
\pgfsetlinewidth{0.501875pt}%
\definecolor{currentstroke}{rgb}{0.000000,0.000000,0.000000}%
\pgfsetstrokecolor{currentstroke}%
\pgfsetdash{}{0pt}%
\pgfsys@defobject{currentmarker}{\pgfqpoint{0.000000in}{-0.020833in}}{\pgfqpoint{0.000000in}{0.000000in}}{%
\pgfpathmoveto{\pgfqpoint{0.000000in}{0.000000in}}%
\pgfpathlineto{\pgfqpoint{0.000000in}{-0.020833in}}%
\pgfusepath{stroke,fill}%
}%
\begin{pgfscope}%
\pgfsys@transformshift{5.562577in}{0.893003in}%
\pgfsys@useobject{currentmarker}{}%
\end{pgfscope}%
\end{pgfscope}%
\begin{pgfscope}%
\pgfsetbuttcap%
\pgfsetroundjoin%
\definecolor{currentfill}{rgb}{0.000000,0.000000,0.000000}%
\pgfsetfillcolor{currentfill}%
\pgfsetlinewidth{0.501875pt}%
\definecolor{currentstroke}{rgb}{0.000000,0.000000,0.000000}%
\pgfsetstrokecolor{currentstroke}%
\pgfsetdash{}{0pt}%
\pgfsys@defobject{currentmarker}{\pgfqpoint{0.000000in}{0.000000in}}{\pgfqpoint{0.000000in}{0.020833in}}{%
\pgfpathmoveto{\pgfqpoint{0.000000in}{0.000000in}}%
\pgfpathlineto{\pgfqpoint{0.000000in}{0.020833in}}%
\pgfusepath{stroke,fill}%
}%
\begin{pgfscope}%
\pgfsys@transformshift{5.664922in}{0.586309in}%
\pgfsys@useobject{currentmarker}{}%
\end{pgfscope}%
\end{pgfscope}%
\begin{pgfscope}%
\pgfsetbuttcap%
\pgfsetroundjoin%
\definecolor{currentfill}{rgb}{0.000000,0.000000,0.000000}%
\pgfsetfillcolor{currentfill}%
\pgfsetlinewidth{0.501875pt}%
\definecolor{currentstroke}{rgb}{0.000000,0.000000,0.000000}%
\pgfsetstrokecolor{currentstroke}%
\pgfsetdash{}{0pt}%
\pgfsys@defobject{currentmarker}{\pgfqpoint{0.000000in}{-0.020833in}}{\pgfqpoint{0.000000in}{0.000000in}}{%
\pgfpathmoveto{\pgfqpoint{0.000000in}{0.000000in}}%
\pgfpathlineto{\pgfqpoint{0.000000in}{-0.020833in}}%
\pgfusepath{stroke,fill}%
}%
\begin{pgfscope}%
\pgfsys@transformshift{5.664922in}{0.893003in}%
\pgfsys@useobject{currentmarker}{}%
\end{pgfscope}%
\end{pgfscope}%
\begin{pgfscope}%
\pgfsetbuttcap%
\pgfsetroundjoin%
\definecolor{currentfill}{rgb}{0.000000,0.000000,0.000000}%
\pgfsetfillcolor{currentfill}%
\pgfsetlinewidth{0.501875pt}%
\definecolor{currentstroke}{rgb}{0.000000,0.000000,0.000000}%
\pgfsetstrokecolor{currentstroke}%
\pgfsetdash{}{0pt}%
\pgfsys@defobject{currentmarker}{\pgfqpoint{0.000000in}{0.000000in}}{\pgfqpoint{0.000000in}{0.020833in}}{%
\pgfpathmoveto{\pgfqpoint{0.000000in}{0.000000in}}%
\pgfpathlineto{\pgfqpoint{0.000000in}{0.020833in}}%
\pgfusepath{stroke,fill}%
}%
\begin{pgfscope}%
\pgfsys@transformshift{5.869613in}{0.586309in}%
\pgfsys@useobject{currentmarker}{}%
\end{pgfscope}%
\end{pgfscope}%
\begin{pgfscope}%
\pgfsetbuttcap%
\pgfsetroundjoin%
\definecolor{currentfill}{rgb}{0.000000,0.000000,0.000000}%
\pgfsetfillcolor{currentfill}%
\pgfsetlinewidth{0.501875pt}%
\definecolor{currentstroke}{rgb}{0.000000,0.000000,0.000000}%
\pgfsetstrokecolor{currentstroke}%
\pgfsetdash{}{0pt}%
\pgfsys@defobject{currentmarker}{\pgfqpoint{0.000000in}{-0.020833in}}{\pgfqpoint{0.000000in}{0.000000in}}{%
\pgfpathmoveto{\pgfqpoint{0.000000in}{0.000000in}}%
\pgfpathlineto{\pgfqpoint{0.000000in}{-0.020833in}}%
\pgfusepath{stroke,fill}%
}%
\begin{pgfscope}%
\pgfsys@transformshift{5.869613in}{0.893003in}%
\pgfsys@useobject{currentmarker}{}%
\end{pgfscope}%
\end{pgfscope}%
\begin{pgfscope}%
\pgfsetbuttcap%
\pgfsetroundjoin%
\definecolor{currentfill}{rgb}{0.000000,0.000000,0.000000}%
\pgfsetfillcolor{currentfill}%
\pgfsetlinewidth{0.501875pt}%
\definecolor{currentstroke}{rgb}{0.000000,0.000000,0.000000}%
\pgfsetstrokecolor{currentstroke}%
\pgfsetdash{}{0pt}%
\pgfsys@defobject{currentmarker}{\pgfqpoint{0.000000in}{0.000000in}}{\pgfqpoint{0.000000in}{0.020833in}}{%
\pgfpathmoveto{\pgfqpoint{0.000000in}{0.000000in}}%
\pgfpathlineto{\pgfqpoint{0.000000in}{0.020833in}}%
\pgfusepath{stroke,fill}%
}%
\begin{pgfscope}%
\pgfsys@transformshift{5.971959in}{0.586309in}%
\pgfsys@useobject{currentmarker}{}%
\end{pgfscope}%
\end{pgfscope}%
\begin{pgfscope}%
\pgfsetbuttcap%
\pgfsetroundjoin%
\definecolor{currentfill}{rgb}{0.000000,0.000000,0.000000}%
\pgfsetfillcolor{currentfill}%
\pgfsetlinewidth{0.501875pt}%
\definecolor{currentstroke}{rgb}{0.000000,0.000000,0.000000}%
\pgfsetstrokecolor{currentstroke}%
\pgfsetdash{}{0pt}%
\pgfsys@defobject{currentmarker}{\pgfqpoint{0.000000in}{-0.020833in}}{\pgfqpoint{0.000000in}{0.000000in}}{%
\pgfpathmoveto{\pgfqpoint{0.000000in}{0.000000in}}%
\pgfpathlineto{\pgfqpoint{0.000000in}{-0.020833in}}%
\pgfusepath{stroke,fill}%
}%
\begin{pgfscope}%
\pgfsys@transformshift{5.971959in}{0.893003in}%
\pgfsys@useobject{currentmarker}{}%
\end{pgfscope}%
\end{pgfscope}%
\begin{pgfscope}%
\pgfsetbuttcap%
\pgfsetroundjoin%
\definecolor{currentfill}{rgb}{0.000000,0.000000,0.000000}%
\pgfsetfillcolor{currentfill}%
\pgfsetlinewidth{0.501875pt}%
\definecolor{currentstroke}{rgb}{0.000000,0.000000,0.000000}%
\pgfsetstrokecolor{currentstroke}%
\pgfsetdash{}{0pt}%
\pgfsys@defobject{currentmarker}{\pgfqpoint{0.000000in}{0.000000in}}{\pgfqpoint{0.000000in}{0.020833in}}{%
\pgfpathmoveto{\pgfqpoint{0.000000in}{0.000000in}}%
\pgfpathlineto{\pgfqpoint{0.000000in}{0.020833in}}%
\pgfusepath{stroke,fill}%
}%
\begin{pgfscope}%
\pgfsys@transformshift{6.074304in}{0.586309in}%
\pgfsys@useobject{currentmarker}{}%
\end{pgfscope}%
\end{pgfscope}%
\begin{pgfscope}%
\pgfsetbuttcap%
\pgfsetroundjoin%
\definecolor{currentfill}{rgb}{0.000000,0.000000,0.000000}%
\pgfsetfillcolor{currentfill}%
\pgfsetlinewidth{0.501875pt}%
\definecolor{currentstroke}{rgb}{0.000000,0.000000,0.000000}%
\pgfsetstrokecolor{currentstroke}%
\pgfsetdash{}{0pt}%
\pgfsys@defobject{currentmarker}{\pgfqpoint{0.000000in}{-0.020833in}}{\pgfqpoint{0.000000in}{0.000000in}}{%
\pgfpathmoveto{\pgfqpoint{0.000000in}{0.000000in}}%
\pgfpathlineto{\pgfqpoint{0.000000in}{-0.020833in}}%
\pgfusepath{stroke,fill}%
}%
\begin{pgfscope}%
\pgfsys@transformshift{6.074304in}{0.893003in}%
\pgfsys@useobject{currentmarker}{}%
\end{pgfscope}%
\end{pgfscope}%
\begin{pgfscope}%
\pgfsetbuttcap%
\pgfsetroundjoin%
\definecolor{currentfill}{rgb}{0.000000,0.000000,0.000000}%
\pgfsetfillcolor{currentfill}%
\pgfsetlinewidth{0.501875pt}%
\definecolor{currentstroke}{rgb}{0.000000,0.000000,0.000000}%
\pgfsetstrokecolor{currentstroke}%
\pgfsetdash{}{0pt}%
\pgfsys@defobject{currentmarker}{\pgfqpoint{0.000000in}{0.000000in}}{\pgfqpoint{0.000000in}{0.020833in}}{%
\pgfpathmoveto{\pgfqpoint{0.000000in}{0.000000in}}%
\pgfpathlineto{\pgfqpoint{0.000000in}{0.020833in}}%
\pgfusepath{stroke,fill}%
}%
\begin{pgfscope}%
\pgfsys@transformshift{6.176649in}{0.586309in}%
\pgfsys@useobject{currentmarker}{}%
\end{pgfscope}%
\end{pgfscope}%
\begin{pgfscope}%
\pgfsetbuttcap%
\pgfsetroundjoin%
\definecolor{currentfill}{rgb}{0.000000,0.000000,0.000000}%
\pgfsetfillcolor{currentfill}%
\pgfsetlinewidth{0.501875pt}%
\definecolor{currentstroke}{rgb}{0.000000,0.000000,0.000000}%
\pgfsetstrokecolor{currentstroke}%
\pgfsetdash{}{0pt}%
\pgfsys@defobject{currentmarker}{\pgfqpoint{0.000000in}{-0.020833in}}{\pgfqpoint{0.000000in}{0.000000in}}{%
\pgfpathmoveto{\pgfqpoint{0.000000in}{0.000000in}}%
\pgfpathlineto{\pgfqpoint{0.000000in}{-0.020833in}}%
\pgfusepath{stroke,fill}%
}%
\begin{pgfscope}%
\pgfsys@transformshift{6.176649in}{0.893003in}%
\pgfsys@useobject{currentmarker}{}%
\end{pgfscope}%
\end{pgfscope}%
\begin{pgfscope}%
\definecolor{textcolor}{rgb}{0.000000,0.000000,0.000000}%
\pgfsetstrokecolor{textcolor}%
\pgfsetfillcolor{textcolor}%
\pgftext[x=3.374517in,y=0.148667in,,top]{\color{textcolor}\rmfamily\fontsize{8.000000}{9.600000}\selectfont Zeit}%
\end{pgfscope}%
\begin{pgfscope}%
\pgfsetbuttcap%
\pgfsetroundjoin%
\definecolor{currentfill}{rgb}{0.000000,0.000000,0.000000}%
\pgfsetfillcolor{currentfill}%
\pgfsetlinewidth{0.501875pt}%
\definecolor{currentstroke}{rgb}{0.000000,0.000000,0.000000}%
\pgfsetstrokecolor{currentstroke}%
\pgfsetdash{}{0pt}%
\pgfsys@defobject{currentmarker}{\pgfqpoint{0.000000in}{0.000000in}}{\pgfqpoint{0.041667in}{0.000000in}}{%
\pgfpathmoveto{\pgfqpoint{0.000000in}{0.000000in}}%
\pgfpathlineto{\pgfqpoint{0.041667in}{0.000000in}}%
\pgfusepath{stroke,fill}%
}%
\begin{pgfscope}%
\pgfsys@transformshift{0.481681in}{0.635663in}%
\pgfsys@useobject{currentmarker}{}%
\end{pgfscope}%
\end{pgfscope}%
\begin{pgfscope}%
\pgfsetbuttcap%
\pgfsetroundjoin%
\definecolor{currentfill}{rgb}{0.000000,0.000000,0.000000}%
\pgfsetfillcolor{currentfill}%
\pgfsetlinewidth{0.501875pt}%
\definecolor{currentstroke}{rgb}{0.000000,0.000000,0.000000}%
\pgfsetstrokecolor{currentstroke}%
\pgfsetdash{}{0pt}%
\pgfsys@defobject{currentmarker}{\pgfqpoint{-0.041667in}{0.000000in}}{\pgfqpoint{-0.000000in}{0.000000in}}{%
\pgfpathmoveto{\pgfqpoint{-0.000000in}{0.000000in}}%
\pgfpathlineto{\pgfqpoint{-0.041667in}{0.000000in}}%
\pgfusepath{stroke,fill}%
}%
\begin{pgfscope}%
\pgfsys@transformshift{6.267353in}{0.635663in}%
\pgfsys@useobject{currentmarker}{}%
\end{pgfscope}%
\end{pgfscope}%
\begin{pgfscope}%
\definecolor{textcolor}{rgb}{0.000000,0.000000,0.000000}%
\pgfsetstrokecolor{textcolor}%
\pgfsetfillcolor{textcolor}%
\pgftext[x=0.204222in, y=0.601927in, left, base]{\color{textcolor}\rmfamily\fontsize{7.000000}{8.400000}\selectfont \ensuremath{-}0.5}%
\end{pgfscope}%
\begin{pgfscope}%
\pgfsetbuttcap%
\pgfsetroundjoin%
\definecolor{currentfill}{rgb}{0.000000,0.000000,0.000000}%
\pgfsetfillcolor{currentfill}%
\pgfsetlinewidth{0.501875pt}%
\definecolor{currentstroke}{rgb}{0.000000,0.000000,0.000000}%
\pgfsetstrokecolor{currentstroke}%
\pgfsetdash{}{0pt}%
\pgfsys@defobject{currentmarker}{\pgfqpoint{0.000000in}{0.000000in}}{\pgfqpoint{0.041667in}{0.000000in}}{%
\pgfpathmoveto{\pgfqpoint{0.000000in}{0.000000in}}%
\pgfpathlineto{\pgfqpoint{0.041667in}{0.000000in}}%
\pgfusepath{stroke,fill}%
}%
\begin{pgfscope}%
\pgfsys@transformshift{0.481681in}{0.739656in}%
\pgfsys@useobject{currentmarker}{}%
\end{pgfscope}%
\end{pgfscope}%
\begin{pgfscope}%
\pgfsetbuttcap%
\pgfsetroundjoin%
\definecolor{currentfill}{rgb}{0.000000,0.000000,0.000000}%
\pgfsetfillcolor{currentfill}%
\pgfsetlinewidth{0.501875pt}%
\definecolor{currentstroke}{rgb}{0.000000,0.000000,0.000000}%
\pgfsetstrokecolor{currentstroke}%
\pgfsetdash{}{0pt}%
\pgfsys@defobject{currentmarker}{\pgfqpoint{-0.041667in}{0.000000in}}{\pgfqpoint{-0.000000in}{0.000000in}}{%
\pgfpathmoveto{\pgfqpoint{-0.000000in}{0.000000in}}%
\pgfpathlineto{\pgfqpoint{-0.041667in}{0.000000in}}%
\pgfusepath{stroke,fill}%
}%
\begin{pgfscope}%
\pgfsys@transformshift{6.267353in}{0.739656in}%
\pgfsys@useobject{currentmarker}{}%
\end{pgfscope}%
\end{pgfscope}%
\begin{pgfscope}%
\definecolor{textcolor}{rgb}{0.000000,0.000000,0.000000}%
\pgfsetstrokecolor{textcolor}%
\pgfsetfillcolor{textcolor}%
\pgftext[x=0.291028in, y=0.705920in, left, base]{\color{textcolor}\rmfamily\fontsize{7.000000}{8.400000}\selectfont 0.0}%
\end{pgfscope}%
\begin{pgfscope}%
\pgfsetbuttcap%
\pgfsetroundjoin%
\definecolor{currentfill}{rgb}{0.000000,0.000000,0.000000}%
\pgfsetfillcolor{currentfill}%
\pgfsetlinewidth{0.501875pt}%
\definecolor{currentstroke}{rgb}{0.000000,0.000000,0.000000}%
\pgfsetstrokecolor{currentstroke}%
\pgfsetdash{}{0pt}%
\pgfsys@defobject{currentmarker}{\pgfqpoint{0.000000in}{0.000000in}}{\pgfqpoint{0.041667in}{0.000000in}}{%
\pgfpathmoveto{\pgfqpoint{0.000000in}{0.000000in}}%
\pgfpathlineto{\pgfqpoint{0.041667in}{0.000000in}}%
\pgfusepath{stroke,fill}%
}%
\begin{pgfscope}%
\pgfsys@transformshift{0.481681in}{0.843649in}%
\pgfsys@useobject{currentmarker}{}%
\end{pgfscope}%
\end{pgfscope}%
\begin{pgfscope}%
\pgfsetbuttcap%
\pgfsetroundjoin%
\definecolor{currentfill}{rgb}{0.000000,0.000000,0.000000}%
\pgfsetfillcolor{currentfill}%
\pgfsetlinewidth{0.501875pt}%
\definecolor{currentstroke}{rgb}{0.000000,0.000000,0.000000}%
\pgfsetstrokecolor{currentstroke}%
\pgfsetdash{}{0pt}%
\pgfsys@defobject{currentmarker}{\pgfqpoint{-0.041667in}{0.000000in}}{\pgfqpoint{-0.000000in}{0.000000in}}{%
\pgfpathmoveto{\pgfqpoint{-0.000000in}{0.000000in}}%
\pgfpathlineto{\pgfqpoint{-0.041667in}{0.000000in}}%
\pgfusepath{stroke,fill}%
}%
\begin{pgfscope}%
\pgfsys@transformshift{6.267353in}{0.843649in}%
\pgfsys@useobject{currentmarker}{}%
\end{pgfscope}%
\end{pgfscope}%
\begin{pgfscope}%
\definecolor{textcolor}{rgb}{0.000000,0.000000,0.000000}%
\pgfsetstrokecolor{textcolor}%
\pgfsetfillcolor{textcolor}%
\pgftext[x=0.291028in, y=0.809913in, left, base]{\color{textcolor}\rmfamily\fontsize{7.000000}{8.400000}\selectfont 0.5}%
\end{pgfscope}%
\begin{pgfscope}%
\pgfsetbuttcap%
\pgfsetroundjoin%
\definecolor{currentfill}{rgb}{0.000000,0.000000,0.000000}%
\pgfsetfillcolor{currentfill}%
\pgfsetlinewidth{0.501875pt}%
\definecolor{currentstroke}{rgb}{0.000000,0.000000,0.000000}%
\pgfsetstrokecolor{currentstroke}%
\pgfsetdash{}{0pt}%
\pgfsys@defobject{currentmarker}{\pgfqpoint{0.000000in}{0.000000in}}{\pgfqpoint{0.020833in}{0.000000in}}{%
\pgfpathmoveto{\pgfqpoint{0.000000in}{0.000000in}}%
\pgfpathlineto{\pgfqpoint{0.020833in}{0.000000in}}%
\pgfusepath{stroke,fill}%
}%
\begin{pgfscope}%
\pgfsys@transformshift{0.481681in}{0.594066in}%
\pgfsys@useobject{currentmarker}{}%
\end{pgfscope}%
\end{pgfscope}%
\begin{pgfscope}%
\pgfsetbuttcap%
\pgfsetroundjoin%
\definecolor{currentfill}{rgb}{0.000000,0.000000,0.000000}%
\pgfsetfillcolor{currentfill}%
\pgfsetlinewidth{0.501875pt}%
\definecolor{currentstroke}{rgb}{0.000000,0.000000,0.000000}%
\pgfsetstrokecolor{currentstroke}%
\pgfsetdash{}{0pt}%
\pgfsys@defobject{currentmarker}{\pgfqpoint{-0.020833in}{0.000000in}}{\pgfqpoint{-0.000000in}{0.000000in}}{%
\pgfpathmoveto{\pgfqpoint{-0.000000in}{0.000000in}}%
\pgfpathlineto{\pgfqpoint{-0.020833in}{0.000000in}}%
\pgfusepath{stroke,fill}%
}%
\begin{pgfscope}%
\pgfsys@transformshift{6.267353in}{0.594066in}%
\pgfsys@useobject{currentmarker}{}%
\end{pgfscope}%
\end{pgfscope}%
\begin{pgfscope}%
\pgfsetbuttcap%
\pgfsetroundjoin%
\definecolor{currentfill}{rgb}{0.000000,0.000000,0.000000}%
\pgfsetfillcolor{currentfill}%
\pgfsetlinewidth{0.501875pt}%
\definecolor{currentstroke}{rgb}{0.000000,0.000000,0.000000}%
\pgfsetstrokecolor{currentstroke}%
\pgfsetdash{}{0pt}%
\pgfsys@defobject{currentmarker}{\pgfqpoint{0.000000in}{0.000000in}}{\pgfqpoint{0.020833in}{0.000000in}}{%
\pgfpathmoveto{\pgfqpoint{0.000000in}{0.000000in}}%
\pgfpathlineto{\pgfqpoint{0.020833in}{0.000000in}}%
\pgfusepath{stroke,fill}%
}%
\begin{pgfscope}%
\pgfsys@transformshift{0.481681in}{0.614865in}%
\pgfsys@useobject{currentmarker}{}%
\end{pgfscope}%
\end{pgfscope}%
\begin{pgfscope}%
\pgfsetbuttcap%
\pgfsetroundjoin%
\definecolor{currentfill}{rgb}{0.000000,0.000000,0.000000}%
\pgfsetfillcolor{currentfill}%
\pgfsetlinewidth{0.501875pt}%
\definecolor{currentstroke}{rgb}{0.000000,0.000000,0.000000}%
\pgfsetstrokecolor{currentstroke}%
\pgfsetdash{}{0pt}%
\pgfsys@defobject{currentmarker}{\pgfqpoint{-0.020833in}{0.000000in}}{\pgfqpoint{-0.000000in}{0.000000in}}{%
\pgfpathmoveto{\pgfqpoint{-0.000000in}{0.000000in}}%
\pgfpathlineto{\pgfqpoint{-0.020833in}{0.000000in}}%
\pgfusepath{stroke,fill}%
}%
\begin{pgfscope}%
\pgfsys@transformshift{6.267353in}{0.614865in}%
\pgfsys@useobject{currentmarker}{}%
\end{pgfscope}%
\end{pgfscope}%
\begin{pgfscope}%
\pgfsetbuttcap%
\pgfsetroundjoin%
\definecolor{currentfill}{rgb}{0.000000,0.000000,0.000000}%
\pgfsetfillcolor{currentfill}%
\pgfsetlinewidth{0.501875pt}%
\definecolor{currentstroke}{rgb}{0.000000,0.000000,0.000000}%
\pgfsetstrokecolor{currentstroke}%
\pgfsetdash{}{0pt}%
\pgfsys@defobject{currentmarker}{\pgfqpoint{0.000000in}{0.000000in}}{\pgfqpoint{0.020833in}{0.000000in}}{%
\pgfpathmoveto{\pgfqpoint{0.000000in}{0.000000in}}%
\pgfpathlineto{\pgfqpoint{0.020833in}{0.000000in}}%
\pgfusepath{stroke,fill}%
}%
\begin{pgfscope}%
\pgfsys@transformshift{0.481681in}{0.656462in}%
\pgfsys@useobject{currentmarker}{}%
\end{pgfscope}%
\end{pgfscope}%
\begin{pgfscope}%
\pgfsetbuttcap%
\pgfsetroundjoin%
\definecolor{currentfill}{rgb}{0.000000,0.000000,0.000000}%
\pgfsetfillcolor{currentfill}%
\pgfsetlinewidth{0.501875pt}%
\definecolor{currentstroke}{rgb}{0.000000,0.000000,0.000000}%
\pgfsetstrokecolor{currentstroke}%
\pgfsetdash{}{0pt}%
\pgfsys@defobject{currentmarker}{\pgfqpoint{-0.020833in}{0.000000in}}{\pgfqpoint{-0.000000in}{0.000000in}}{%
\pgfpathmoveto{\pgfqpoint{-0.000000in}{0.000000in}}%
\pgfpathlineto{\pgfqpoint{-0.020833in}{0.000000in}}%
\pgfusepath{stroke,fill}%
}%
\begin{pgfscope}%
\pgfsys@transformshift{6.267353in}{0.656462in}%
\pgfsys@useobject{currentmarker}{}%
\end{pgfscope}%
\end{pgfscope}%
\begin{pgfscope}%
\pgfsetbuttcap%
\pgfsetroundjoin%
\definecolor{currentfill}{rgb}{0.000000,0.000000,0.000000}%
\pgfsetfillcolor{currentfill}%
\pgfsetlinewidth{0.501875pt}%
\definecolor{currentstroke}{rgb}{0.000000,0.000000,0.000000}%
\pgfsetstrokecolor{currentstroke}%
\pgfsetdash{}{0pt}%
\pgfsys@defobject{currentmarker}{\pgfqpoint{0.000000in}{0.000000in}}{\pgfqpoint{0.020833in}{0.000000in}}{%
\pgfpathmoveto{\pgfqpoint{0.000000in}{0.000000in}}%
\pgfpathlineto{\pgfqpoint{0.020833in}{0.000000in}}%
\pgfusepath{stroke,fill}%
}%
\begin{pgfscope}%
\pgfsys@transformshift{0.481681in}{0.677261in}%
\pgfsys@useobject{currentmarker}{}%
\end{pgfscope}%
\end{pgfscope}%
\begin{pgfscope}%
\pgfsetbuttcap%
\pgfsetroundjoin%
\definecolor{currentfill}{rgb}{0.000000,0.000000,0.000000}%
\pgfsetfillcolor{currentfill}%
\pgfsetlinewidth{0.501875pt}%
\definecolor{currentstroke}{rgb}{0.000000,0.000000,0.000000}%
\pgfsetstrokecolor{currentstroke}%
\pgfsetdash{}{0pt}%
\pgfsys@defobject{currentmarker}{\pgfqpoint{-0.020833in}{0.000000in}}{\pgfqpoint{-0.000000in}{0.000000in}}{%
\pgfpathmoveto{\pgfqpoint{-0.000000in}{0.000000in}}%
\pgfpathlineto{\pgfqpoint{-0.020833in}{0.000000in}}%
\pgfusepath{stroke,fill}%
}%
\begin{pgfscope}%
\pgfsys@transformshift{6.267353in}{0.677261in}%
\pgfsys@useobject{currentmarker}{}%
\end{pgfscope}%
\end{pgfscope}%
\begin{pgfscope}%
\pgfsetbuttcap%
\pgfsetroundjoin%
\definecolor{currentfill}{rgb}{0.000000,0.000000,0.000000}%
\pgfsetfillcolor{currentfill}%
\pgfsetlinewidth{0.501875pt}%
\definecolor{currentstroke}{rgb}{0.000000,0.000000,0.000000}%
\pgfsetstrokecolor{currentstroke}%
\pgfsetdash{}{0pt}%
\pgfsys@defobject{currentmarker}{\pgfqpoint{0.000000in}{0.000000in}}{\pgfqpoint{0.020833in}{0.000000in}}{%
\pgfpathmoveto{\pgfqpoint{0.000000in}{0.000000in}}%
\pgfpathlineto{\pgfqpoint{0.020833in}{0.000000in}}%
\pgfusepath{stroke,fill}%
}%
\begin{pgfscope}%
\pgfsys@transformshift{0.481681in}{0.698059in}%
\pgfsys@useobject{currentmarker}{}%
\end{pgfscope}%
\end{pgfscope}%
\begin{pgfscope}%
\pgfsetbuttcap%
\pgfsetroundjoin%
\definecolor{currentfill}{rgb}{0.000000,0.000000,0.000000}%
\pgfsetfillcolor{currentfill}%
\pgfsetlinewidth{0.501875pt}%
\definecolor{currentstroke}{rgb}{0.000000,0.000000,0.000000}%
\pgfsetstrokecolor{currentstroke}%
\pgfsetdash{}{0pt}%
\pgfsys@defobject{currentmarker}{\pgfqpoint{-0.020833in}{0.000000in}}{\pgfqpoint{-0.000000in}{0.000000in}}{%
\pgfpathmoveto{\pgfqpoint{-0.000000in}{0.000000in}}%
\pgfpathlineto{\pgfqpoint{-0.020833in}{0.000000in}}%
\pgfusepath{stroke,fill}%
}%
\begin{pgfscope}%
\pgfsys@transformshift{6.267353in}{0.698059in}%
\pgfsys@useobject{currentmarker}{}%
\end{pgfscope}%
\end{pgfscope}%
\begin{pgfscope}%
\pgfsetbuttcap%
\pgfsetroundjoin%
\definecolor{currentfill}{rgb}{0.000000,0.000000,0.000000}%
\pgfsetfillcolor{currentfill}%
\pgfsetlinewidth{0.501875pt}%
\definecolor{currentstroke}{rgb}{0.000000,0.000000,0.000000}%
\pgfsetstrokecolor{currentstroke}%
\pgfsetdash{}{0pt}%
\pgfsys@defobject{currentmarker}{\pgfqpoint{0.000000in}{0.000000in}}{\pgfqpoint{0.020833in}{0.000000in}}{%
\pgfpathmoveto{\pgfqpoint{0.000000in}{0.000000in}}%
\pgfpathlineto{\pgfqpoint{0.020833in}{0.000000in}}%
\pgfusepath{stroke,fill}%
}%
\begin{pgfscope}%
\pgfsys@transformshift{0.481681in}{0.718858in}%
\pgfsys@useobject{currentmarker}{}%
\end{pgfscope}%
\end{pgfscope}%
\begin{pgfscope}%
\pgfsetbuttcap%
\pgfsetroundjoin%
\definecolor{currentfill}{rgb}{0.000000,0.000000,0.000000}%
\pgfsetfillcolor{currentfill}%
\pgfsetlinewidth{0.501875pt}%
\definecolor{currentstroke}{rgb}{0.000000,0.000000,0.000000}%
\pgfsetstrokecolor{currentstroke}%
\pgfsetdash{}{0pt}%
\pgfsys@defobject{currentmarker}{\pgfqpoint{-0.020833in}{0.000000in}}{\pgfqpoint{-0.000000in}{0.000000in}}{%
\pgfpathmoveto{\pgfqpoint{-0.000000in}{0.000000in}}%
\pgfpathlineto{\pgfqpoint{-0.020833in}{0.000000in}}%
\pgfusepath{stroke,fill}%
}%
\begin{pgfscope}%
\pgfsys@transformshift{6.267353in}{0.718858in}%
\pgfsys@useobject{currentmarker}{}%
\end{pgfscope}%
\end{pgfscope}%
\begin{pgfscope}%
\pgfsetbuttcap%
\pgfsetroundjoin%
\definecolor{currentfill}{rgb}{0.000000,0.000000,0.000000}%
\pgfsetfillcolor{currentfill}%
\pgfsetlinewidth{0.501875pt}%
\definecolor{currentstroke}{rgb}{0.000000,0.000000,0.000000}%
\pgfsetstrokecolor{currentstroke}%
\pgfsetdash{}{0pt}%
\pgfsys@defobject{currentmarker}{\pgfqpoint{0.000000in}{0.000000in}}{\pgfqpoint{0.020833in}{0.000000in}}{%
\pgfpathmoveto{\pgfqpoint{0.000000in}{0.000000in}}%
\pgfpathlineto{\pgfqpoint{0.020833in}{0.000000in}}%
\pgfusepath{stroke,fill}%
}%
\begin{pgfscope}%
\pgfsys@transformshift{0.481681in}{0.760455in}%
\pgfsys@useobject{currentmarker}{}%
\end{pgfscope}%
\end{pgfscope}%
\begin{pgfscope}%
\pgfsetbuttcap%
\pgfsetroundjoin%
\definecolor{currentfill}{rgb}{0.000000,0.000000,0.000000}%
\pgfsetfillcolor{currentfill}%
\pgfsetlinewidth{0.501875pt}%
\definecolor{currentstroke}{rgb}{0.000000,0.000000,0.000000}%
\pgfsetstrokecolor{currentstroke}%
\pgfsetdash{}{0pt}%
\pgfsys@defobject{currentmarker}{\pgfqpoint{-0.020833in}{0.000000in}}{\pgfqpoint{-0.000000in}{0.000000in}}{%
\pgfpathmoveto{\pgfqpoint{-0.000000in}{0.000000in}}%
\pgfpathlineto{\pgfqpoint{-0.020833in}{0.000000in}}%
\pgfusepath{stroke,fill}%
}%
\begin{pgfscope}%
\pgfsys@transformshift{6.267353in}{0.760455in}%
\pgfsys@useobject{currentmarker}{}%
\end{pgfscope}%
\end{pgfscope}%
\begin{pgfscope}%
\pgfsetbuttcap%
\pgfsetroundjoin%
\definecolor{currentfill}{rgb}{0.000000,0.000000,0.000000}%
\pgfsetfillcolor{currentfill}%
\pgfsetlinewidth{0.501875pt}%
\definecolor{currentstroke}{rgb}{0.000000,0.000000,0.000000}%
\pgfsetstrokecolor{currentstroke}%
\pgfsetdash{}{0pt}%
\pgfsys@defobject{currentmarker}{\pgfqpoint{0.000000in}{0.000000in}}{\pgfqpoint{0.020833in}{0.000000in}}{%
\pgfpathmoveto{\pgfqpoint{0.000000in}{0.000000in}}%
\pgfpathlineto{\pgfqpoint{0.020833in}{0.000000in}}%
\pgfusepath{stroke,fill}%
}%
\begin{pgfscope}%
\pgfsys@transformshift{0.481681in}{0.781253in}%
\pgfsys@useobject{currentmarker}{}%
\end{pgfscope}%
\end{pgfscope}%
\begin{pgfscope}%
\pgfsetbuttcap%
\pgfsetroundjoin%
\definecolor{currentfill}{rgb}{0.000000,0.000000,0.000000}%
\pgfsetfillcolor{currentfill}%
\pgfsetlinewidth{0.501875pt}%
\definecolor{currentstroke}{rgb}{0.000000,0.000000,0.000000}%
\pgfsetstrokecolor{currentstroke}%
\pgfsetdash{}{0pt}%
\pgfsys@defobject{currentmarker}{\pgfqpoint{-0.020833in}{0.000000in}}{\pgfqpoint{-0.000000in}{0.000000in}}{%
\pgfpathmoveto{\pgfqpoint{-0.000000in}{0.000000in}}%
\pgfpathlineto{\pgfqpoint{-0.020833in}{0.000000in}}%
\pgfusepath{stroke,fill}%
}%
\begin{pgfscope}%
\pgfsys@transformshift{6.267353in}{0.781253in}%
\pgfsys@useobject{currentmarker}{}%
\end{pgfscope}%
\end{pgfscope}%
\begin{pgfscope}%
\pgfsetbuttcap%
\pgfsetroundjoin%
\definecolor{currentfill}{rgb}{0.000000,0.000000,0.000000}%
\pgfsetfillcolor{currentfill}%
\pgfsetlinewidth{0.501875pt}%
\definecolor{currentstroke}{rgb}{0.000000,0.000000,0.000000}%
\pgfsetstrokecolor{currentstroke}%
\pgfsetdash{}{0pt}%
\pgfsys@defobject{currentmarker}{\pgfqpoint{0.000000in}{0.000000in}}{\pgfqpoint{0.020833in}{0.000000in}}{%
\pgfpathmoveto{\pgfqpoint{0.000000in}{0.000000in}}%
\pgfpathlineto{\pgfqpoint{0.020833in}{0.000000in}}%
\pgfusepath{stroke,fill}%
}%
\begin{pgfscope}%
\pgfsys@transformshift{0.481681in}{0.802052in}%
\pgfsys@useobject{currentmarker}{}%
\end{pgfscope}%
\end{pgfscope}%
\begin{pgfscope}%
\pgfsetbuttcap%
\pgfsetroundjoin%
\definecolor{currentfill}{rgb}{0.000000,0.000000,0.000000}%
\pgfsetfillcolor{currentfill}%
\pgfsetlinewidth{0.501875pt}%
\definecolor{currentstroke}{rgb}{0.000000,0.000000,0.000000}%
\pgfsetstrokecolor{currentstroke}%
\pgfsetdash{}{0pt}%
\pgfsys@defobject{currentmarker}{\pgfqpoint{-0.020833in}{0.000000in}}{\pgfqpoint{-0.000000in}{0.000000in}}{%
\pgfpathmoveto{\pgfqpoint{-0.000000in}{0.000000in}}%
\pgfpathlineto{\pgfqpoint{-0.020833in}{0.000000in}}%
\pgfusepath{stroke,fill}%
}%
\begin{pgfscope}%
\pgfsys@transformshift{6.267353in}{0.802052in}%
\pgfsys@useobject{currentmarker}{}%
\end{pgfscope}%
\end{pgfscope}%
\begin{pgfscope}%
\pgfsetbuttcap%
\pgfsetroundjoin%
\definecolor{currentfill}{rgb}{0.000000,0.000000,0.000000}%
\pgfsetfillcolor{currentfill}%
\pgfsetlinewidth{0.501875pt}%
\definecolor{currentstroke}{rgb}{0.000000,0.000000,0.000000}%
\pgfsetstrokecolor{currentstroke}%
\pgfsetdash{}{0pt}%
\pgfsys@defobject{currentmarker}{\pgfqpoint{0.000000in}{0.000000in}}{\pgfqpoint{0.020833in}{0.000000in}}{%
\pgfpathmoveto{\pgfqpoint{0.000000in}{0.000000in}}%
\pgfpathlineto{\pgfqpoint{0.020833in}{0.000000in}}%
\pgfusepath{stroke,fill}%
}%
\begin{pgfscope}%
\pgfsys@transformshift{0.481681in}{0.822850in}%
\pgfsys@useobject{currentmarker}{}%
\end{pgfscope}%
\end{pgfscope}%
\begin{pgfscope}%
\pgfsetbuttcap%
\pgfsetroundjoin%
\definecolor{currentfill}{rgb}{0.000000,0.000000,0.000000}%
\pgfsetfillcolor{currentfill}%
\pgfsetlinewidth{0.501875pt}%
\definecolor{currentstroke}{rgb}{0.000000,0.000000,0.000000}%
\pgfsetstrokecolor{currentstroke}%
\pgfsetdash{}{0pt}%
\pgfsys@defobject{currentmarker}{\pgfqpoint{-0.020833in}{0.000000in}}{\pgfqpoint{-0.000000in}{0.000000in}}{%
\pgfpathmoveto{\pgfqpoint{-0.000000in}{0.000000in}}%
\pgfpathlineto{\pgfqpoint{-0.020833in}{0.000000in}}%
\pgfusepath{stroke,fill}%
}%
\begin{pgfscope}%
\pgfsys@transformshift{6.267353in}{0.822850in}%
\pgfsys@useobject{currentmarker}{}%
\end{pgfscope}%
\end{pgfscope}%
\begin{pgfscope}%
\pgfsetbuttcap%
\pgfsetroundjoin%
\definecolor{currentfill}{rgb}{0.000000,0.000000,0.000000}%
\pgfsetfillcolor{currentfill}%
\pgfsetlinewidth{0.501875pt}%
\definecolor{currentstroke}{rgb}{0.000000,0.000000,0.000000}%
\pgfsetstrokecolor{currentstroke}%
\pgfsetdash{}{0pt}%
\pgfsys@defobject{currentmarker}{\pgfqpoint{0.000000in}{0.000000in}}{\pgfqpoint{0.020833in}{0.000000in}}{%
\pgfpathmoveto{\pgfqpoint{0.000000in}{0.000000in}}%
\pgfpathlineto{\pgfqpoint{0.020833in}{0.000000in}}%
\pgfusepath{stroke,fill}%
}%
\begin{pgfscope}%
\pgfsys@transformshift{0.481681in}{0.864447in}%
\pgfsys@useobject{currentmarker}{}%
\end{pgfscope}%
\end{pgfscope}%
\begin{pgfscope}%
\pgfsetbuttcap%
\pgfsetroundjoin%
\definecolor{currentfill}{rgb}{0.000000,0.000000,0.000000}%
\pgfsetfillcolor{currentfill}%
\pgfsetlinewidth{0.501875pt}%
\definecolor{currentstroke}{rgb}{0.000000,0.000000,0.000000}%
\pgfsetstrokecolor{currentstroke}%
\pgfsetdash{}{0pt}%
\pgfsys@defobject{currentmarker}{\pgfqpoint{-0.020833in}{0.000000in}}{\pgfqpoint{-0.000000in}{0.000000in}}{%
\pgfpathmoveto{\pgfqpoint{-0.000000in}{0.000000in}}%
\pgfpathlineto{\pgfqpoint{-0.020833in}{0.000000in}}%
\pgfusepath{stroke,fill}%
}%
\begin{pgfscope}%
\pgfsys@transformshift{6.267353in}{0.864447in}%
\pgfsys@useobject{currentmarker}{}%
\end{pgfscope}%
\end{pgfscope}%
\begin{pgfscope}%
\pgfsetbuttcap%
\pgfsetroundjoin%
\definecolor{currentfill}{rgb}{0.000000,0.000000,0.000000}%
\pgfsetfillcolor{currentfill}%
\pgfsetlinewidth{0.501875pt}%
\definecolor{currentstroke}{rgb}{0.000000,0.000000,0.000000}%
\pgfsetstrokecolor{currentstroke}%
\pgfsetdash{}{0pt}%
\pgfsys@defobject{currentmarker}{\pgfqpoint{0.000000in}{0.000000in}}{\pgfqpoint{0.020833in}{0.000000in}}{%
\pgfpathmoveto{\pgfqpoint{0.000000in}{0.000000in}}%
\pgfpathlineto{\pgfqpoint{0.020833in}{0.000000in}}%
\pgfusepath{stroke,fill}%
}%
\begin{pgfscope}%
\pgfsys@transformshift{0.481681in}{0.885246in}%
\pgfsys@useobject{currentmarker}{}%
\end{pgfscope}%
\end{pgfscope}%
\begin{pgfscope}%
\pgfsetbuttcap%
\pgfsetroundjoin%
\definecolor{currentfill}{rgb}{0.000000,0.000000,0.000000}%
\pgfsetfillcolor{currentfill}%
\pgfsetlinewidth{0.501875pt}%
\definecolor{currentstroke}{rgb}{0.000000,0.000000,0.000000}%
\pgfsetstrokecolor{currentstroke}%
\pgfsetdash{}{0pt}%
\pgfsys@defobject{currentmarker}{\pgfqpoint{-0.020833in}{0.000000in}}{\pgfqpoint{-0.000000in}{0.000000in}}{%
\pgfpathmoveto{\pgfqpoint{-0.000000in}{0.000000in}}%
\pgfpathlineto{\pgfqpoint{-0.020833in}{0.000000in}}%
\pgfusepath{stroke,fill}%
}%
\begin{pgfscope}%
\pgfsys@transformshift{6.267353in}{0.885246in}%
\pgfsys@useobject{currentmarker}{}%
\end{pgfscope}%
\end{pgfscope}%
\begin{pgfscope}%
\definecolor{textcolor}{rgb}{0.000000,0.000000,0.000000}%
\pgfsetstrokecolor{textcolor}%
\pgfsetfillcolor{textcolor}%
\pgftext[x=0.148667in,y=0.739656in,,bottom,rotate=90.000000]{\color{textcolor}\rmfamily\fontsize{8.000000}{9.600000}\selectfont Unterschied}%
\end{pgfscope}%
\begin{pgfscope}%
\pgfsetrectcap%
\pgfsetmiterjoin%
\pgfsetlinewidth{0.501875pt}%
\definecolor{currentstroke}{rgb}{0.000000,0.000000,0.000000}%
\pgfsetstrokecolor{currentstroke}%
\pgfsetdash{}{0pt}%
\pgfpathmoveto{\pgfqpoint{0.481681in}{0.586309in}}%
\pgfpathlineto{\pgfqpoint{0.481681in}{0.893003in}}%
\pgfusepath{stroke}%
\end{pgfscope}%
\begin{pgfscope}%
\pgfsetrectcap%
\pgfsetmiterjoin%
\pgfsetlinewidth{0.501875pt}%
\definecolor{currentstroke}{rgb}{0.000000,0.000000,0.000000}%
\pgfsetstrokecolor{currentstroke}%
\pgfsetdash{}{0pt}%
\pgfpathmoveto{\pgfqpoint{6.267353in}{0.586309in}}%
\pgfpathlineto{\pgfqpoint{6.267353in}{0.893003in}}%
\pgfusepath{stroke}%
\end{pgfscope}%
\begin{pgfscope}%
\pgfsetrectcap%
\pgfsetmiterjoin%
\pgfsetlinewidth{0.501875pt}%
\definecolor{currentstroke}{rgb}{0.000000,0.000000,0.000000}%
\pgfsetstrokecolor{currentstroke}%
\pgfsetdash{}{0pt}%
\pgfpathmoveto{\pgfqpoint{0.481681in}{0.586309in}}%
\pgfpathlineto{\pgfqpoint{6.267353in}{0.586309in}}%
\pgfusepath{stroke}%
\end{pgfscope}%
\begin{pgfscope}%
\pgfsetrectcap%
\pgfsetmiterjoin%
\pgfsetlinewidth{0.501875pt}%
\definecolor{currentstroke}{rgb}{0.000000,0.000000,0.000000}%
\pgfsetstrokecolor{currentstroke}%
\pgfsetdash{}{0pt}%
\pgfpathmoveto{\pgfqpoint{0.481681in}{0.893003in}}%
\pgfpathlineto{\pgfqpoint{6.267353in}{0.893003in}}%
\pgfusepath{stroke}%
\end{pgfscope}%
\end{pgfpicture}%
\makeatother%
\endgroup%

  \end{center}
  \caption{Verhältnis von Aufwand und Komplexität im Lacustris Projekt.}
\end{figure}


Die Verläufe der verschiedenen Messreihen sollen nun beschrieben und
Anhand der Hinweise aus dem Abschlussinterivew des Projektes erläutert
werden. Die verschiedenen Ereignisse werden dabei in die Kategorien
People, Process und Technologies gegliedert.

In der Kategorie Process fallen vier Ereignisse auf. Alle diese
Ereignisse beziehen sich auf die Projektphasen des Projektes. Vom Anfang
des Projektes am 22. Juli 2021 bis Mitte August 2021 fällt eine
vergleichsweise geringe Produktivität auf. Hier wurde als Begründung
herbeigeführt, dass sich die Dynamik innerhalb des Teams erst bilden
musste, und das Team somit noch nicht so produktiv sein konnte (\#1). Von Mitte August 2021 bis Mitte
November 2021 ist die Produktivität des Teams deutlich höher und das
Team befindet sich in einer Phase der kontinuierlichen Entwicklung (\#2
und \#4). Ab November 2021
geht das Projekt in eine Phase der Stabilisierung über und die Codebasis
wird eher konsolidiert als erweitert. Hier steigen die Metriken also
weniger stark (\#5).

In der Kategorie People fallen drei Ereignisse auf. Zunächst war in der
letzten Woche des Novembers das komplette ägyptische Team im Urlaub (\#7). Dem folgte im Januar 2022 das Englische Team
(\#8). Als drittes Ereignis wurde Mitte Februar
2022 eine Entwicklerin von dem Projekt entlassen (\#6). Diese drei Ereignisse wirkten sich kaum auf
den Projektverlauf aus, da sich das Projekt hier schon in einer Phase
der Stabilisierung befand.

Neben diesen Projektereignissen fallen vom 4. bis zum 13. September 2021
starke Schwankungen in allen Metriken auf. Nach diesen Schwankungen
verblieben die Metriken für den Rest der Projektdauer auf einem, um 20\%
höheren Stand als vorher. Dieser starke Anstieg lie\ss  sich auf das
Hinzufügen von vorgeneriertem Code zurückführen. So wurden durch ein
externes Programm Datenmodelle generiert, welche an dieser Stelle dem
Code hinzu gefügt wurden\footnote{Lacustris Zeile 19}.

In einem nächsten Schritt werden die Korrelationsmetriken berechnet:

TODO Grafik

Die Spearman Rangkorrelationskoeffizienten sind über alle Metriken
hinweg sehr stark mit Werten über 0,99. Die
Rangkorrelationskoeffizienten nach Kendall liegen mit Werten um 0,95 nur
leicht unter den Werten von Spearman. Auffallend ist bei beiden
Koeffizienten, dass die die Werte der Korrelationskoeffizienten über die
Metriken hinweg konstant bleiben. Die
Pearson-Produkt-Moment-Korrelationskoeffizienten liegen im Schnitt
deutlich unter den anderen Koeffizienten. Nur bei dem Aufwand nach
Halstead konnte hier ein höherer Wert erzielt werden. Das könnte darauf
zurückzuführen sein, dass der Störfaktor des vorgenerierten Codes in dem
Halstead Aufwand nicht so stark widergespiegelt wird. Dass bei den
anderen Koeffizienten der Halstead Aufwand nicht heraussticht deutet
darauf hin, dass die Koeffizienten nach Kendall und Spearman von dem
Störfaktor nicht so stark beeinflusst wurden.

\subsection{Kritik}\label{lm-kritik}

Klar zu kritisieren ist an der Analyse dieses Projektes, dass durch den
vorgenerierten Code Mitte September 2021 eine starke Verfälschung der
Messergebnisse stattgefunden hat. Die künstliche Erhöhung der Metriken
könnte die Korrelation zwischen den Aufwandsabschätzungen und den
Komplexitätsmetriken schwächen

\subsection{Fazit}\label{lm-fazit}

Insgesamt lassen sich in diesem Projekt starke Korrelationen zwischen
den Aufwandsabschätzungen und den Komplexitätsmetriken nachweisen. Eine
niedrigere Korrelation nach Pearson könnte auf den Störfaktor des
vorgenerierten Codes zurückzuführen sein. Es lässt sich also
abschlie\ss end sagen, dass dieser Fall für eine generelle Korrelation von
Aufwandsabschätzungen und Codekomplexitätsmetriken spricht.

\section{Engelmannii}\label{engelmannii}

Das Engelmannii Projekt\footnote{Engelmannii Zeile 5 TODO} wird von der
DXC für einen deutschen Energieversorger entwickelt. Es soll den
Anwendern als zentrale Verwaltungsschnittstelle für alle Dienste des
Energieversorgers dienen\footnote{Zeile 6}. Realisiert wird das Projekt
in einer agilen Arbeitsweise nach Scrum sowohl von offshore, als auch
onshore Teams. Das Offshore Team ist in Manila lokalisiert und das
Onshore Team über Deutschland verteilt. Die Teams bestehen aus
Entwicklern aller Erfahrungslevels, wobei alle seit mindestens zwei
Jahren an dem Projekt arbeiten. Aufgebaut ist das Projekt aus einem
Frontend und einem Backend. Das Frontend wird primär auf JavaScript
aufgebaut mit Technologien wie Angular und Typescript. Das Backend wird
mit .NET realisiert. Insgesamt ist die Codebasis des Projektes über drei
Repositories verteilt. Ein Repository beinhaltet sämtlichen Frontend
Code, eines den Code für die \ac{API} Middleware in C\# und ein letztes
Repository umfasst eine sog. Progressive Web App.

\subsection{Datenerhebung}\label{engelmannii-Datenerhebung}

Auch in dem Engelmannii-Projekt sollen einerseits die
Aufwandsabschätzungen und andererseits die Quelltext Repositories als
Primärquellen verwendet werden.

Die Aufwandsabschätzungen werden in dem Ernergieportal-Projekt in der
Software LifeCycle Management Software Microsoft DevOps verwaltet. Das
Format der Userstorys entspricht dabei zwar nicht in Gänze dem Format
der anderen Projekte, die relevanten Teilinformation können aber auch
hier problemlos abgelesen werden.

In einem Interview mit einem Stakeholder des Projektes\footcite[Vgl. ][]{stakeholdernInterviewMitStakeholdern2022}
konnten die relevanten Teilbereiche der User Storys identifiziert
werden. Die Aufwandsabschätzungen seien in einem Feld mit dem Namen
„Front End Dev Story Points`` zu finden. Das Abschlussdatum jedes
Features sei dem Feld „Closed Date`` zu entnehmen. Zusätzlich konnte
eine Abfrage zur Identifizierung aller relevanten User Storys erstellt
werden. Dabei schränkt die Abfrage die Auswahl der User Storys auf alle
Elemente mit einem Wert für das Feld „Front End Dev Story Points`` und
einem Status von „Closed`` ein. Von insgesamt 2329 User Storys wurden
588 als relevant ausgewählt.

Die so ausgewählten Userstorys können als \ac{CSV}-Datei exportiert und in
die Analysesoftware eingelesen werden.

Zur Ermittlung der Codekomplexitätsmetriken ist, wie in Kapitel \ref{verbindung-von-daten-und-hypothesen}
beschrieben eine Analyse des Quelltextes der Anwendung angestrebt. Wie
auch in den anderen Projekten, soll ein Änderungsverlauf der
Komplexitätsmetriken aus der Versionshistorie des Quelltextes erstellt
werden. Diese Versionshistorie ist für alle Quelltext Repositories der
Software im Format eines Git Repositories verfügbar. Wie eingangs
beschrieben besteht das Projekt aus drei Repositories. In dieser Analyse
wird jedoch lediglich die Entwicklung am Frontend der Applikation
betrachtet. Also ist auch nur das Repository des Frontends von
Interesse. Mit der in \ref{implementierung-einer-untersuchungssoftware}. beschriebenen Software kann ein Zeitverlauf
der Codekomplexitätsmetriken aus der Versionshistorie des Sourcecodes
erstellt werden. Die so ermittelten Daten werden nun ausgewertet.

\subsection{Auswertung}\label{engelmannii-Auswertung}

Zur Auswertung werden auch in diesem Projekt alle Daten in ein
Verhältnis zueinander gesetzt. Das Verhältnis der Storypoint
Aufwandsabschätzungen zu den Komplexitätsma\ss zahlen findet sich in
Abbildung TODO

\begin{figure}\label{engelmanii-graph}
  \begin{center}
      \input{includes/5 analysis of the cases/figures/engelmanii-plot.pgf}
  \end{center}
  \caption{Verhältnis von Aufwand und Komplexität im Engelmanii Projekt.}
\end{figure}

Bei einer genaueren Betrachtung der Messwerte fällt auf, dass die
zyklomatische Komplexität und die Einrückungskomplexität sehr nahe den
Story Point Abschätzungen verlaufen. Die Anzahl der logischen Codezeilen
und der Aufwand nach Halstead weisen jedoch von Mai 2019 bis August 2020
eine signifikante Abweichung von den Story Point Metriken von mehr als
300\% auf. Für diese Abweichung muss ein Fehler in der Datenerhebung
verantwortlich gemacht werden. Dabei wurden gewisse Berichtsdateien in
die Analyse eingeschlossen, die aber nicht zu dem eigentlichen Quelltext
der Anwendung dazugehören\footcite[Vgl. ][]{stakeholdernInterviewMitStakeholdern2022}. Im August 2020
fallen die Anzahl der logischen Codezeilen und der Halstead Aufwand
stark ab. Nach diesem Abfall befinden sie sich wieder auf einem
ähnlichen Niveau wie die anderen Metriken. Hier wurde in einem Interview
als Erklärung gefunden, dass im August 2020 die zusätzlichen
Berichtsdateien aus dem Quelltextrepository entfernt wurden. Des
weiteren ist ab März 2021 ein Abflachen des Anstieges der Komplexität
bei einem gleichzeitigen weiteren Anstieg der Aufwandsabschätzungen zu
beobachten. Ab März 2021 wurde an einer Grundlegenden Umstrukturierung
der Anwendung gearbeitet. Diese wurde jedoch in einem separaten
Entwicklungszweig durchgeführt und konnte deswegen in der Analyse der
Codemetriken nur begrenzt erfasst werden. Der Aufwand dieser
Umstrukturierung ist aber gleichzeitig in den Aufwandsabschätzungen zu
sehen. Das erklärt die Abweichung der beiden Werte in diesem Zeitraum.
Für die frequenten kleineren Sprünge in der Einrückungskomplexität ist
ein ähnlicher Messfehler wie in dem Alstonii Projekt verantwortlich.

TODO Grafik

Wie zu erwartet fallen die Korrelationskoeffizienten für die Anazhl der
logischen Codezeilen äu\ss erst gerin aus. Mit Werten unter 0,25 ist hier
keine Korrelation nachweisbar. Das ist auf die Störung der Messung durch
Berichtsdaten zurückzuführen. Hier wurde die Messung über ein Viertel
der Projektdauer um 300\% erhöht, was den geringen
Korrelationskoeffizienten erklärt. Ähnlich wurde auch für den Aufwand
nach Halstead ein sehr geringer Korrelationkoeffizient von 0,5
ermittelt. Für die anderen beiden Komplexitätsmetriken (Zyklomatische
Komplexität und Einrückungskomplexität) konnten sehr starke
Korrelationen nachgewiesen werden. Das unterstützt wiederrum die
Vermutung, dass die geringen Korrelationskoeffizienten der anderen
Metriken mit dem Störfaktor der Berichtsdaten zu begründen sind.

\subsection{Fazit}\label{engelmanii-fazit}

Zusammenfassend konnten in diesem Fall nur bei der zyklomatischen
Komplexität und der Einrückungskomplexität eine ausreichend starke
Korrelation der Codekomplexitätsmetriken mit den Aufwandsabschätzungen
festgestellt werden. Aufgrund eines starken Störfaktors konnte bei den
anderen beiden Metriken keine Korrelation festgestellt werden. Insgesamt
sind zwar auch in diesem Projekt Korrelationen zwischen
Codekomplexitätsmetriken und Aufwandsabschätzungen feststellbar jedoch
fehlt es dem Fall für eine eindeutige Unterstützung der Hypothese an
Aussagekraft.

\subsection{Kritik}\label{engelmanii-kritik}

Klar zu kritisieren ist in diesem Fall, dass die Metriken des Halstead
Aufwandes und der logischen Codezeilen durch einen Störfaktor sehr stark
beeinträchtigt wurden.

\section{GitLab}\label{GitLab}

Als fünftes und letztes Projekt wurde die DevOps Software GitLab
untersucht. Die GitLab Software ist ein Open-Source Tool zum Entwickeln,
Absichern und Betreiben von Softwareanwendungen\footcite[Vgl. ][]{GitLab14Delivers2021}.
Besonders zu bemerken ist an dieser Stelle, dass die Firma eine
besonders transparente Strategie zum Veröffentlichen von Informationen
verfolgt\footcite[Vgl. ][]{GitLabValues}. Nach dieser Strategie werden alle
firmeninternen Informationen zunächst publiziert, sofern für sie keine
Ausnahmeregelung besteht. Eine Liste an Ausnahmen von dieser Richtlinie
ist ebenfalls publiziert. So werden zum Beispiel Informationen zu der
finanziellen Lage der Firma sowie Informationen zu Kunden nicht
publiziert\footcite[Vgl. ][]{GitLabCommunication}.

Entwickelt wird die GitLab Applikation von über 1500 Mitarbeitern,
welche ohne physische Büros aus 66 Ländern verteilt arbeiten\footcite[Vgl. ][]{GitLabGitLab}.
Die gesamte Anwendung umfasst über 650 Tausend Codezeilen und wird hauptsächlich in den Sprachen Ruby (66\%) und
JavaScript (20\%)\footcite[Vgl. ][]{GitLabOrgGitLab} geschrieben. Sämtlicher code
der Anwendung wird in einem zentral Git Repository verwaltet.

\subsection{Datenerhebung}\label{gitlab-Datenerhebung}

Aufgrund der Transparenten Kommunikation der GitLab gestaltet sich eine
Datenerhebung in diesem Projekt vergleichsweise einfach. Wie auch in den
anderen Projekten werden in diesem Projekt zwei Datenquellen benötigt.
Zum einen sollen die Codekomplexitätsmetriken aus der
Quelltextverwaltung der Software berechnet werden. Weiter sollen die
Aufwandsabschätzungen aus der Projektmanagementsoftware des Projektes
geladen werden. Beide Quellen sind inmFalle der GitLab Anwendung
öffentlich zugänglich.

Das Quelltext Repository kann unter der Adresse
\url{https://gitlab.com/gitlab-org/gitlab} abgerufen werden. Für die
Analyse wurde von dieser Adresse eine lokale Kopie des Repositories
geladen und in die Analysesoftware eingespeist. Über die Projektdauer
von fünf Jahren konnten 281.414 Entwicklungsstände identifiziert werden.

Die Aufwandsabschätzungen werden in dem GitLab Projekt im Rahmen von
Issues festgehalten. Issues entsprechen in ihrer Struktur in diesem
Kontext den User Storys. Der Aufwand zum Realisieren eines Issues wird
als das Gewicht des Issues angegeben. Es kann dabei Synonym zu den Story
Point Abschätzungen gesehen werden\footcite[Vgl. ][]{ScaledAgileGitLab}. Alle
Issues des GitLab Projektes sind auf einer Website öffentlich verfügbar
und lassen sich als \ac{CSV}-Datei exportieren\footcite[Vgl. ][]{IssuesGitLabOrg}. Auf
diese Weise konnten 68.366 Datensätze erhoben werden. Die \ac{CSV}-Datei mit
diesen Datensätzen kann unverarbeitet in die Analysesoftware geladen
werden.

Insgesamt konnten alle, für die Analyse benötigten Daten erfolgreich aus
öffentlich verfügbaren Quellen erhoben werden und ohne eine Modifikation
in die Analysesoftware geladen werden.

\subsection{Auswertung}\label{gitlab-Auswertung}

Die gesammelten Daten sollen nun ausgewertet werden. Der Umfang der
Auswertung ist in diesem Projekt ca. 15-mal so gro\ss  wie in allen anderen
Projekten. Aus diesem Grund muss die Auswertungsmethodik abgeändert
werden. Die Berechnung der Entwicklungsstände der Software wird in einem
Cluster aus vier Rechnungseinheiten in einem Rechenzentrum durchgeführt.
Zusätzlich wurden lediglich die Module Lizard und MultiMetric des Code
Parsers aktiviert. Auch mit diesen Änderungen dauerte die Analyse des
Projektes mehrere Wochen. Aufgrund des begrenzten zeitlichen Umfangs
dieser Arbeit konnte nur eine Analyse der ersten dreieinhalb Jahre des
Projektes erfolgen. Das Ergebnis dieser Analyse ist in Abbildung ..
gezeigt.

\begin{figure}\label{gitlab-graph}
  \begin{center}
      \input{includes/5 analysis of the cases/figures/gitlab-plot.pgf}
  \end{center}
  \caption{Verhältnis von Aufwand und Komplexität im GitLab Projekt.}
\end{figure}

Zunächst ist visuell eine starke Korrelation aller Komplexitätsmetriken
mit den Aufwandsabschätzungen erkenntlich. Gleichzeitig fallen jedoch
auch kurzfristige Schwankungen in den Messdaten von ca. 10\% auf. Diese
Schwankungen sind über die Projektdauer hinweg in regelmä\ss igen Abständen
zu beobachten. Bei einer genaueren Analyse wurde festgestellt, dass
diese Schwankungen wahrscheinliche auf eine Asynchronität in der
Berechnungsmethodik der vier Berechnungsknoten zurückzuführen sind.
Dementsprechend handelt es sich hierbei also mit einer hohen
Wahrscheinlichkeit um einen Messfehler. Trotz dieser Schwankungen ist
ein hoher Korrelationsgrad zu erwarten. Dieser wird nun berechnet

TODO Grafik

Entsprechend der Erwartungen wurden in diesem Projekt starke,
konsistente Korrelationen zwischen allen Metriken und den
Aufwandsabschätzungen festgestellt. Würde man den Messfehler beheben,
ist anzunehmen, dass die Korrelationen noch stärker ausfallen.

\subsection{Kritik}\label{gitlab-kritik}

An der Analyse dieses Projektes fallen mehrere Kritikpunkte auf. Zum
einen unterliegt die Analyse potenziell einem Messfehler, welcher zu
Schwankungen in den Messdaten führt. Eine Behebung dieses Messfehlers
war aufgrund des begrenzten zeitlichen Rahmens dieser Arbeit nicht
möglich. Zusätzlich konnte in dem Projekt kein Abschlussinterview zu
Validierung der Daten durchgeführt werden. Die Validität des
Datenerhebungsverfahren konnte nur aus den öffentlich verfügbaren
Informationen der Firma begründet werden.

\subsection{Fazit}\label{gitlab-Fazit}

Zusammenfassend wurde in diesem Projekt eine starke Korrelation der
Aufwandsabschätzungen mit den Codekomplexitätsmetriken festgestellt.
Diese Korrelation könnte potenziell noch stärker ausfallen, wenn die
potenziellen Messfehler in den Daten behoben werden könnten. Insgesamt
lässt sich also sagen, dass dieser Fall die Hypothese einer Korrelation
zwischen Aufwandsabschätzungen und Codekomplexitätsmetriken stützt.