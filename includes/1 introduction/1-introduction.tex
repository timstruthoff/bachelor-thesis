\chapter{Einleitung}\label{einleitung}

Diese Bachelorarbeit wird als Teil eines Praktikums in der Digital
Innovation Lab (DIL) Abteilung des IT-Beratungs- und
Dienstleistungsunternehmens DXC Technology geschrieben. Die DXC
Technology befasst sich unter anderem mit der Herstellung von
Individualsoftware für eine Vielzahl von Kunden\footcite[Vgl. ][]{InterviewMitVertretern2022}.
Die DIL Abteilung im Speziellen ist dabei für die Erstellung von ersten,
minimal lauffähigen Versionen (Minimum Viable Products, MVPs) dieser
Softwareprodukte verantwortlich. Diese MVPs werden dann von anderen
Abteilungen der DXC zu größeren Softwareprodukten weiterentwickelt.

Ein Teil des Dienstleistungsangebotes der DXC liegt in der Wartung,
Betreuung und Weiterentwicklung eben dieser Software. Angebote zur
Wartung und Weiterentwicklung von Software stellen auch eine wesentliche
Umsatzquelle der DXC dar. Dieser Bereich ist also aus
betriebswirtschaftlicher Sicht von besonderem Interesse. Auch aus der
Perspektive der Kunden der DXC ist der Bereich der Wartung und
Weiterentwicklung der Software von Interesse. So entfällt ein
signifikanter Anteil der Kosten eines Softwareproduktes auf deren
Wartung und Weiterentwicklung\footcite[Vgl. ][S. 301]{jonesAppliedSoftwareMeasurement2008}. Im Interesse des
Kunden sollten diese Kosten gesenkt werden\footcite[Vgl. ][]{InterviewMitMitarbeiterin2022}.

Eine Vielzahl an Studien kamen zu dem Konsens, dass die Komplexität
eines Softwareproduktes einen wesentlichen Einflussfaktor für den
Aufwand von Wartung, Betrieb und Weiterentwicklung der Software
darstellt. Jones 2008 konnte beweisen, dass die Komplexität und der
Umfang von Software stark und direkt mit dem Wartungsaufwand\footcite[Vgl. ][S. 64, 335 und 627]{jonesAppliedSoftwareMeasurement2008} und der durchschnittlichen Fehleranzahl
korrelieren\footcite[Vgl. ][S. 64 und 503]{jonesAppliedSoftwareMeasurement2008}. Also sei es sinnvoll,
die Komplexität der Software permanent zu beobachten\footcite[Vgl. ][S. 503]{jonesAppliedSoftwareMeasurement2008}.

\section{Zielsetzung}\label{zielsetzung}

Motiviert durch die betriebswirtschaftliche Relevanz der
Softwarekomplexität liegt die Zielsetzung dieser Arbeit in der Auswahl
und Validierung von Methoden zur automatisierten Bestimmung der
Komplexität von Software. Die Auswahl der Berechnungsmethoden erfolgt
anhand einer umfangreichen Literaturanalyse. Das Arbeitsergebnis ist in
diesem ersten Schritt eine Aufstellung von Berechnungsmethoden mit
jeweils einer Erklärung. So soll es den Leser*innen möglich sein, sich
ein umfangreiches Bild der aktuellen Praxis in der
Softwarekomplexitätsbestimmung zu verschaffen.

Für die Verifizierung der zuvor aufgestellten Metriken wird ein
Vergleich angestellt. Dabei werden für fünf Projekte der DXC
Technologies, sowie für ein externes Projekt Komplexitätsabschätzungen
von Experten mit den berechneten Metriken verglichen. Als Ergebnis
dieses Arbeitsschrittes ist eine Bestimmung des Grades der Korrelation
vorgesehen.

Mit diesem Arbeitsergebnis soll dann in einem letzten Schritt ein
Ausblick auf die weitere Verwendung der Ergebnisse gegeben werden.
Insbesondere wird die zukünftige Umsetzung einer sog.
Umgebungsparameteranalyse in Aussicht gestellt. Diese wurde bereits von
einem unternehmensinternen Experten skizziert. Dabei soll das Ergebnis
dieser Arbeit mit verschiedenen Einflussfaktoren der Projektumgebung
verglichen werden. Aus diesem Vergleich sollen logische Schlüsse auf
mögliche Einflussfaktoren gebildet werden.

\section{Verwandte Arbeiten}\label{verwandte-arbeiten}

Das Feld der Komplexitätsbestimmung von Software besteht bereits ähnlich
lange wie die Softwareentwicklung selbst. Erste
Komplexitätsmaßzahlen, wie z.B. die zyklomatische Komplexität von
McCabe\footcite[Vgl. ][]{mccabeComplexityMeasure1976} wurden in den sechziger und
siebziger Jahren des zqwanzigsten Jahrhunderts entwickelt \footnote{\cite[Vgl. ][S. 25]{zuseSoftwareComplexityMeasures1991} und \cite[][]{rubeyQuantitativeMeasurementProgram1968}}. Also ist
anzunehmen, dass auch zu der Verifizierung dieser Arbeiten bereits eine
Vielzahl an theoretischen Abhandlungen existieren. Durch eine
Literaturrecherche konnten einige Arbeiten zu der Verifizierung von
Softwarekomplexitätsmetriken identifiziert werden. Diese werden im
Folgenden zusammenfassend beschrieben.

In \cite[][]{kemererEmpiricalValidationSoftware1987} werden verschiedene
Komplexitätsmaßen auf 15 Projekte einer Firma angewendet. Dabei kommen
die Codezeilen basierten Maßen SLIM und COCOMO, sowie die
nicht-Codezeilen-basierten Maßen ESTIMACS und Function Points zum
Einsatz\footcite[Vgl. ][S. 2]{kemererEmpiricalValidationSoftware1987}. Die Autoren kamen zu dem Ergebnis, dass
die Metriken nur aussagekräftig sind, wenn sie auf die Projekte
individuell kalibriert werden. Mit der Kalibrierung konnten
Genauigkeitsraten von 88\% erzielt werden\footcite[Vgl. ][S. 12]{kemererEmpiricalValidationSoftware1987}.

In rumreichExaminingSoftwareDesign2019 werden die Komplexitätsmetriken
Anzahl Codezeilen, die zyklomatische Komplexität, die Halstead
Komplexitätsmaßzahl und der Maintainability Index auf Projekte von
Studenten angewendet, um so zu erfahren, ob eine Änderung der
Lehrmethode die Komplexität der Projekte beeinflusst\footcite[Vgl. ][S. 1]{rumreichExaminingSoftwareDesign2019}

In aleneziEmpiricalAnalysisComplexity2015 wird der generelle Verlauf von
Softwarekomplexitätsmetriken untersucht. Dabei kann das sechste
Lehmansche Gesetz bewiesen warden, nach dem die Komplexität einer
Software über Zeit steigt\footcite[Vgl. ][S. 262]{aleneziEmpiricalAnalysisComplexity2015}.

\section{Forschungsbeitrag}\label{forschungsbeitrag}

Der Forschungsbeitrag dieser Arbeit liegt in der Beschreibung der
Korrelation von mathematisch berechneten Komplexitätsmaßen mit
Aufwandsabschätzungen von Experten. Dieser Arbeitsbeitrag stellt in
dreierlei Hinsicht einen Zusatznutzen dar:

Zum einen können Softwarekomplexitätsmetriken für den spezifischen
Anwendungsbereich des DIL ausgewählt und validiert werden. Das
ermöglicht es der Abteilung, diese Maßzahlen in Zukunft für die
kontinuierliche Analyse bestehender Projekte sowie für die initiale
Beurteilung neuer Projekte zu verwenden.

Zweitens werden die Metriken nicht nur im Kontext des DIL, sondern auch
für die Allgemeinheit validiert. Zu den hier behandelten Projekten
existieren noch keine öffentlichen Softwarekomplexitätsanalysen. Mit
dieser Arbeit können die Metriken also besser beurteilt werden.

Drittens wurde im Laufe dieser Arbeit festgestellt, dass eine Abweichung
der Metriken von den Aufwandsabschätzungen ein Indikator für
Prozessereignisse der Softwareentwicklung seien könnte. Diese Arbeit
könnte also die Basis für ein System zur kontinuierlichen Verbesserung
agiler Softwareentwicklungspraktiken darstellen.

\section{Methodisches Vorgehen}\label{methodisches-vorgehen}

Als Abschlussarbeit eines Wirtschaftsinformatik (WI) Studiums wird sich
diese Arbeit auch an den Methodiken der WI orientieren. Die WI ist in
ihrer Erkenntnisgewinnung methodenpluralistisch aufgestellt\footcite[Vgl. ][S. 1]{wildeMethodenspektrumWirtschaftsinformatikUberblick2006}. Ihr instrumentales Portfolio beinhaltet
sowohl Methodiken aus den Real-, den Formal- sowie den
Ingenieurswissenschaften\footcite[Vgl. ][S. 1]{wildeMethodenspektrumWirtschaftsinformatikUberblick2006}.

Unter einer Methode wird generell eine spezielle Vorgehensweise
verstanden. Sie zeichnet sich durch ein Regelsystem von
Untersuchungsinstrumenten aus. Ist diese Methode als wissenschaftlich zu
klassifizieren, müssen diese Regeln auch intersubjektiv nachvollziehbar
sein\footcite[Vgl. ][S. 1f]{wildeMethodenspektrumWirtschaftsinformatikUberblick2006}.

In der WI lassen sich zwei erkenntnistheoretische Ansätze herausstellen:
Das konstruktionswissenschaftliche Paradigma strebt nach dem Schaffen
und Evaluieren von Informationssystemen in Form von Modellen, Methoden
und Softwaresystemen\footcite[Vgl. ][S. 2]{wildeMethodenspektrumWirtschaftsinformatikUberblick2006}. Es weist z.B.
Vorgehen der Informationssystemsgestaltung, wie das Prototyping
auf\footnote{\cite[Vgl. ][S. 3]{wildeMethodenspektrumWirtschaftsinformatikUberblick2006} und \cite{simonSciencesArtificial2019}}. Dem gegenüber steht das
behavioristische und verhaltenswissenschaftliche Paradigma. Nach diesem
Paradigma wird in der WI das Verhalten und die Auswirkungen von
bestehenden Informationssystemen untersucht\footcite[Vgl. ][S. 3]{wildeMethodenspektrumWirtschaftsinformatikUberblick2006}.

In dieser Arbeit soll nach dem behavioristischen Paradigma eine
Erkenntnis gewonnen werden\footcite[Vgl. ][S. 3]{wildeMethodenspektrumWirtschaftsinformatikUberblick2006}. Der
bestehende Sachzusammenhang der Korrelation von mathematischen
Komplexitätsmetriken mit subjektiven Aufwandsabschätzungen in agilen
Projekten soll untersucht werden. Aus dieser Untersuchung soll induktiv
auf einen Gesamtzusammenhang bzw. eine Gesetzesmäßigkeit geschlossen
werden.

Als Hypothese zu dieser Gesetzesmäßigkeit wird in dieser Arbeit eine
eingeschränkte Korrelation zwischen den Aufwandsabschätzungen und den
Codekomplexitätsmetriken vermutet. Es ist grundsätzlich anzunehmen, dass
eine Erhöhung der mathematischen Komplexität auch in einer Erhöhung des
geschätzten Aufwandes widergespiegelt wird. Jedoch sind auch
Störfaktoren dieser Korrelation abzusehen.

Die Validierung einer Metrik durch die Untersuchung ihrer Korrelation
mit einer anderen Größe ist nach Zuse, Fenton und Bowl ein verbreiteter
Ansatz\footcite[Vgl. ][S. 561f]{zuseSoftwareComplexityMeasures1991}. Mit der Untersuchung dieser
Korrelation können die Komplexitätsmetriken also im Kontext des DIL
validiert werden. Eine genaue Korrelation kann und soll in dieser Arbeit
aufgrund der zu erwartenden Störfaktoren und des geringen Umfangs der
Untersuchung aber nicht berechnet werden\footcite[Vgl. ][S. 449]{jonesAppliedSoftwareMeasurement2008}.
Vielmehr soll eine unvollständige Theorie in Form einer Näherungsangabe
als ceteris-paribus Hypothese aufgestellt werden\footcite[Vgl. ][S. 3]{wildeMethodenspektrumWirtschaftsinformatikUberblick2006}.

Trotz ihres vergleichsweisen jungen Alters bietet die WI eine üppige
Bandbreite an Methoden der Erkenntnisgewinnung\footcite[Vgl. ][S. 113]{heinrichForschungsmethodikIntegrationsdisziplinBeitrag2005}. Gerade die Methodik der Fallstudienforschung erlebt einen
stetigen Anstieg in Popularität\footcite[Vgl. ][S. 22]{yinCaseStudyResearch2014}. Insbesondere für eine beschreibende
Forschung, wie sie in dieser Arbeit angestrebt wird, sei die
Fallstudienmethodik geeignet\footcite[Vgl. ][S. 607]{dubeRigorInformationSystems2003}. In einer Fallstudie wird ein Phänomen in seinem natürlichen
Kontext beschrieben. Es wird eine geringe Anzahl von Fällen intensiv
sowohl mit qualitativen als auch quantitativen Analysemethoden
untersucht\footcite[Vgl. ][S. 7]{gothlichFallstudienAlsForschungsmethode2003}.
Es werden verschiedene Datentypen gesammelt und diese in Verbindung
zueinander und zu der Hypothese gebracht. Zum Herstellen dieser
logischen Verbindungen steht eine Reihe von Werkzeugen zur Verfügung.
Insbesondere die Zeitreihenanalyse findet in dieser Arbeit Anwendung
\footcite[Vgl. ][S. 6]{gothlichFallstudienAlsForschungsmethode2003}. Der Aufbau
der Fallstudie\footcite[Vgl. ][S.
8ff]{gothlichFallstudienAlsForschungsmethode2003} wird in Kapitel 4 weiter beschrieben.

\section{Aufbau der Arbeit}\label{aufbau-der-arbeit}

Der Aufbau dieser Arbeit orientiert sich an dem Aufbau ähnlicher
Arbeiten\footcite[Vgl. ][S. 260]{aleneziEmpiricalAnalysisComplexity2015} und soll so eine zielgerichtete und
übersichtliche Erfassung der Maßzahlen ermöglichen.

Zunächst wird das Thema der Softwarekomplexität aus einem generellen
Blickwinkel betrachtet (Kapitel \ref{softwarekomplexituxe4tsmauxdfe}). Dabei wird der Begriff
Softwarekomplexität zunächst definiert und dann erläutert (Kapitel \ref{definitionen}). In einem zweiten Schritt werden verschiedene Methoden zur
Messung von Softwarekomplexität begründet ausgewählt und erklärt
(Kapitel \ref{eine-auswahl-von-softwarekomplexitatsmasszahlen}).

Als Gegenstück zu den Softwarekomplexitätsmetriken werden in Kapitel \ref{aufwandsabschuxe4tzungen-agiler-projekte}
die Komplexitätsabschätzungen der Experten erläutert. Dabei wird
insbesondere auf den Kontext der Komplexitätsabschätzungen in der agilen
Entwicklung der Projekte Bezug genommen.

Nach der Definition der beiden Untersuchungsgrößen wird in Kapitel \ref{forschungsaufbau-der-fallstudie} der
Aufbau der Fallstudienforschung beleuchtet. Es wird ein formales
Forschungsprotokoll definiert. Insbesondere wird ein Analysealgorithmus
zur automatisierten Untersuchung der Projekte vorgestellt.

Ab Kapitel \ref{untersuchung-der-softwarekomplexituxe4tsmetriken} findet die praktische Untersuchung der Komplexitätsmetriken
statt. Es werden die einzelnen Projekte als Fälle vorgestellt. Jeweils
wird die Datenerhebung beschrieben und mit der zuvor erläuterten
Forschungsmethodik ausgewertet. In jedem Projekt wird ein Fazit zu der
Korrelation der Metriken gezogen. In einem letzten Schritt werden
gesammelte Zusatzdaten aus den Projekten hinzugezogen, um das
Forschungsergebnis zu erklären und auch mögliche Störfaktoren
aufgezeigt.

Eine gesammelte Analyse aller Ergebnisse findet in Kapitel \ref{zusammenfassende-analyse-der-untersuchungsergebnisse} statt. Hier
wird versucht, aus den Ergebnissen der einzelnen Projekte einen Schluss
auf eine generelle Korrelation zu ziehen.

Zuletzt wird in Kapitel \ref{conclusion} ein Fazit der Arbeit gegeben. Dabei wird
insbesondere auf mögliche Kritik an den Forschungsergebnissen
eingegangen (Kapitel \ref{kritische-evaluierung}) und ein Ausblick auf weitere Entwicklungen
gegeben (Kapitel \ref{Ausblick}).