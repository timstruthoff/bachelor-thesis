\chapter{Conclusion}\label{conclusion}

Die vorliegende Bachelorarbeit befasst sich mit der Fragestellung, ob
eine Korrelation zwischen Story Point Aufwandsabschätzungen und
Softwarekomplexitätsmetriken besteht. Das Ziel dieser Arbeit ist eine
Näherungsangabe als Antwort zu dieser Frage zu formulieren. Als
Forschungsmethodik wird eine explorative Fallstudien über sechs
Softwareprojekte durchgeführt.

Zur Beantwortung der Fragestellung wird zunächst die Theorie der
Softwarekomplexitätsmetriken erläutert. Die Softwarekomplexitätsmetriken
werden als Teil der statischen Code-Analyse in der
Softwarequalitätssicherung in den Software Lebenszyklus eingeordnet.
Dann werden sie als Verfahren zur Quantifizierung der Vielschichtigkeit
der Struktur eines Computerpro-gramms definiert. Es wird auf die
umfangreiche Kritik an Softwarekomplexitätsmetriken eingegangen. So
seien diese Metriken schwach in ihrer Aussagekraft und könnten die
subjektiv wahrgenommene Komplexität einer Software nicht quantifizieren.
Im nächsten Unterkapitel der theoretischen Abhandlung der
Softwarekomplexitätsmetriken wird eine Auswahl von Metriken getätigt.
Die Metriken Anzahl von logischen Codezeilen, Zyklomatische Komplexität
und Halstead Aufwand werden aufgrund ihrer hohen Anzahl an Zitierungen
in Fachliteratur ausgewählt. Die Metrik Einrückungskomplexität wird
aufgrund ihrer einzigartigen Berechnungsmethode betrachtet. Für alle
vier Metriken wird dann ihr Ursprung und ihre Berechnungsmethodik
erläutert. Au\ss erdem wird an dieser Stelle noch einmal auf Kritik an den
individuellen Metriken eingegangen. Als zweite Vergleichsgrö\ss e werden
auch die Story Point Aufwandsabschätzungen erklärt. So wird im Kontext
der agilen Softwareentwicklung nach Scrum der Aufwand zur Realisierung
einer User Story, bzw. eines Features vor der Entwicklung abgeschätzt.

Nach der theoretischen Abhandlung der Softwarekomplexitätsmetriken und
der Aufwandsabschätzungen folgt nun eine Beschreibung des
Forschungsaufbaus. Es wird eine explorative Fallstudie an sechs
Softwareprojekten durchgeführt. In jedem Projekt werden zwei
Datenquellen untersucht. Als erste Quelle werden die
Aufwandsabschätzungen aus der Projektmanagementsoftware extrahiert. Die
zweite Quelle sind die Quelltextverwaltungen der Projekte. Aus der
Versionshistorie des Quelltextes können die Softwarekomplexitätsangaben
für die einzelnen Entwicklungsstände der Software berechnet werden. Aus
beiden Quellen wird jeweils eine Zeitreihe der Messwerte aufgestellt.
Beide Zeitreihen werden verbunden und es wird ein
Korrelationskoeffizient zwischen ihnen berechnet. Dieser
Korrelationskoeffizient bezeichnet die Korrelation zwischen den
Softwarekomplexitätsmetriken und den Aufwandsabschätzungen. Er wird als
Ergebnis der Fälle betrachtet. Die Berechnung des
Korrelationskoeffizienten aus den Quellen erfolgt mit einer
Analysesoftware. Die Implementierung dieser Analysesoftware wird
ebenfalls beschrieben.

In einem nächsten Schritt findet die Untersuchung der sechs
Softwareprojekte als Fälle der Fallstudie statt. Im ersten, dritten,
fünften und sechsten Fall kann eine starke Korrelation aller
Komplexitätsmetriken mit den Aufwandsabschätzungen gezeigt werden. Im
zweiten Fall und dritten Fall hingegen sind nur starke Korrelationen
zweier Metriken nachweisbar.

Insgesamt konnte durch die Fallstudie das Forschungsziel einer
Näherungsangabe zu der Korrelation von Softwarekomplexitätsmetriken und
Aufwandsabschätzungen erreicht werden. In allen Fällen korreliert die
zyklomatische Komplexität stark mit den Aufwandsabschätzungen. Also ist
eine allgemeine Korrelation wahrscheinlich. Für den Aufwand nach
Halstead und die Einrückungskomplexität konnte in 5 von 6, also 80\% der
Fällen eine Korrelation nachgewiesen werden. Bei der Anzahl logischer
Codezeilen wurde nur in vier von sechs Fällen, also zu 70\% eine
Korrelation nachgewiesen.

Insgesamt lautet das Ergebnis, dass eine starke Korrelation
(\textgreater{} 0,8) der Softwarekomplexitätsmetriken mit den
Aufwandsabschätzungen in jedem Fall wahrscheinlich, in keinem Fall aber
garantiert ist.

\section{Kritische Evaluierung}\label{kritische-evaluierung}

Die Ergebnisse dieser Arbeit werden nun kritisch evaluiert. Insbesondere
wird dabei auf die Gütekriterien zur Bewertung von Fallstudien nach
Gothlich 2003 genommen\footcite[Vgl. ][S. 13]{gothlichFallstudienAlsForschungsmethode2003}.

An der \emph{Objektivität}\footcite[Vgl. ][S. 13]{gothlichFallstudienAlsForschungsmethode2003} der Fallstudie lässt sich kritisieren, dass die Quelldaten der
Analyse nicht als Teil der Arbeit veröffentlicht wurden. Diese
beinhalten teils vetrauliche Daten und können deswegen nicht
veröffentlicht werden. Die Analysemethoden kann jedoch anhand der
öffentlichen Datenquellen des GitLab falls überprüft werden.

Die \emph{validität der Analysekonstrukte} wurde zwar nicht in einem
Gutachtenstil bewiesen, in einigen der Projekte jedoch durch
Projektbeteiligte validiert.

Auch an der \emph{internen Validität} der Arbeit gibt es Kritikpunkte.
So wurde in der Interpretation der Daten keinerlei alternative
Interpretationsmöglichkeiten berücksichtigt.

Zu der \emph{externen Validität} der Fälle lässt sich sagen, dass zwar
eine Replikationslogik angewendet wurde, das Analysevorgehen jedoch
nicht durch Feedbackschleifen mit den Ergebnissen der Fälle verfeinert
wurde.

Für die \emph{Reliabilität} spricht zunächst das sämtliche Analysen
vollständig automatisiert durchgeführt wurden, was Rechenfehler
grö\ss tenteils ausschlie\ss t. Jedoch werden dadurch eventuell auch
eigenheiten in den Messdaten ignoriert. So sind in mehreren der Fälle
Messfehler aufgefallen, die auf einer Fehlinterpretation der Daten durch
die Analysesoftware beruhen.

Zuletzt ist auch die \emph{Utilitarität} der Studie gegeben. Abgesehen
von einem regen Interesse innerhalb der Firma steht dem Aufwand zur
Realisierung der Studie auch der generelle Nutzen der Validierung der
Komplexitätsmetriken gegenüber.

Zusätzlich zu den Gütekriterien von Gothlich werden auch weitere
Kritikpunkte

\section{Ausblick}\label{Ausblick}

Im Folgenden soll aufbauend auf den Ergebnissen dieser Arbeit ein
Ausblick auf mögliche weitere Entwicklungen gegeben werden.

\subsection{Zusätzliche Validierung der Ergebnisse}\label{Zusatzliche-Validierung-der-Ergebnisse}

Zunächst können die hier erlangten Ergebnisse weiter validiert werden.
Dazu sollten einerseits die Messfehler in der Untersuchung der Projekte
behoben werden. Hierzu könnte die Analysesoftware verbessert werden.
Auch kann die Validierung der Ergebnisse durch mehr und umfangreichere
Interviews mit Projektbeteiligten verbessert werden. Zuletzt würde eine
Ausweitung der Fallstudie auf mehr Projekte die Validität der
Gesamtaussage weiter stützen.

\subsection{Umgebungsparameteranalyse}\label{Umgebungsparameteranalyse}

Neben der Weiterentwicklung der Korrelationsanalyse ist im Rahmen dieser
Arbeit noch ein weiterer interessanter Zusammenhang aufgefallen. An den
Stellen, an denen die Codekomplexitätsmetriken besonders stark von den
Aufwandsabschätzungen abwichen lie\ss en sich in vielen Fällen besondere
Projektereignisse nachweisen. Also liegt die Vermutung nahe, dass
bestimmte Ereignisse im Verlauf eines Softwareprojektes eine Abweichung
der Kompelxitätsmetriken von den Aufwandsabschätzungen herbeiführen.
Dieser Zusammenhang könnte in einer weiteren Analyse untersucht werden.

Zur Untersuchung der Analyseergebnisse wurde von einem Experten der DXC
Technologies bereits eine sogenannte Umgebungsparameteranalyse
skizziert. Diese wird als mögliche Weiterentwicklung dieser Arbeit im
Folgenden beschrieben.

Im Rahmen der Umgebungsparameteranalyse werden als erster Schritt Fokus
Punkte für die weitere Analyse ausgesucht. Das sollen Zeitintervalle
seien, an denen die Codekomplexitätsmetriken über einen bestimmten
Schwellenwert hinweg von den Aufwandsabschätzungen abweichen. Für das
Alstonii Projekt ist die Auswahl dieser Zeitintervalle beispielhaft in
Abbildung TODO skizziert.

TODO Grafik

Durch langjährige Erfahrung und Befragung von Projektbeteiligten konnten
Umgebungsvariablen identifiziert werden, welche die Korrelation
beeinflussen. Diese werden in drei Überkategorien gegliedert:
Menschliche Faktoren (People), Prozessfaktoren (Process) und
Technologiefaktoren (Technology). Zu jeder Hauptkategorie konnten Vier
Messbereiche identifiziert werden. Diese werden mit unterschiedlichen
quantitativen Indikatoren objektiv durch Befragung erfasst.

TODO Grafik

In einem dritten Schritt werden diese zwei Informationsstränge in Verbindung
zueinander gesetzt. So sollen potenzielle Kausalitäten erkannt werden.
Ein Umgebungsparameter könnte z.B. zu einer Abweichung der
Komplexitätsmetriken von den Aufwandsabschätzungen führen. Der so
generierte Kausalitätsgraph ist beispielhaft in Abbildung TODO skizziert.

TODO Grafik

In einem letzten Schritt soll versucht werden, aus den Kausalitäten
Verbesserungsempfehlungen für das Management des Projektes abzuleiten.
So könnte z.B. eine Änderung in der Arbeitsmethode (Prozessfaktor) eine
Produktivitätserhöhung herbeiführen.

Insgesamt könnte mit der hier in Aussicht gestellten Arbeit ein System
zur kontinuierlichen Verbesserung der Entwicklungstechnik in agilen
Projekten geschaffen werden.