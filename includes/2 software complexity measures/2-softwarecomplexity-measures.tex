\chapter{Softwarekomplexitätsmaße}\label{softwarekomplexituxe4tsmauxdfe}

Die Analyse der Softwarekomplexität ist ein Teil der statischen
Sourcecode-Analyse. In der Code-Analyse wird im Nachhinein (a
posterio\footnote{Hoffmann, D. W. 2013, S. 261}) anhand von statischen
Analysen einer Stichprobe zu einem bestimmten Zeitpunkt\footnote{(Ebert
  1996: 86)} des Sourcecodes festgestellt, ob die Software vorher
definierten Qualitätsanforderungen entspricht\footnote{Ebenda, S. 261}.
Diese Analyse kann sowohl automatisiert als auch manuell erfolgen und
wird im Gegensatz zu Verfahren der konstruktiven Qualitätssicherung auf
bereits vorhandene Software angewendet\footnote{(Hoffmann 2013:261)}.

Als Teil der statischen Code-Analyse ist die Messung der
Softwarekomplexität ein Teil der Qualitätssicherung der
Software\footnote{Quelle fehlt} \footnote{Hoffmann 2013:261}. Dabei
sollen oft unsichtbare Eigenschaften der Software sichtbar und
quantitativ messbar gemacht werden\footnote{Hoffmann 2013:261}
\footnote{Zuse, 1991, p. S. 561)}. Der Qualität von Software können
verschiedene Teilbereiche untergeordnet werden. Je nach Definition
gehören hierzu unter anderem Funktionalität, Laufzeit, Zuverlässigkeit,
Benutzbarkeit, Wartbarkeit, Transparenz, Übertragbarkeit und
Testbarkeit\footnote{Hoffmann, D. W. 2013, S. 22f} \footnote{Liggesmeyer
  2009:245}. Gerade auf die Qualitätsmerkmale Wartbarkeit und
Testbarkeit hat die Komplexität der Software einen direkten Einfluss. So
steigt der Aufwand des Wartens und Testens mit dem Umfang und der
Komplexität der Software\footnote{Quelle fehlt}.

Im Lebenszyklus einer Software kann die Komplexitätsuntersuchung als
Teil der Qualitätssicherung sowohl an das Ende der Entwicklungsphase
gestellt werden als auch kontinuierlich parallel zu der
Weiterentwicklung stattfinden\footnote{Quelle fehlt}.

\section{Definitionen}\label{definitionen}

Eine Betrachtung von Softwarekomplexitätsmaßen setzt zunächst eine
genaue Definition dieser voraus. Softwarekomplexitätsmetriken sind
Metriken zur Messung der Komplexität einer Software. Diese drei Teile
der Definition von Softwarekomplexitätsmetriken werden im Folgenden
definiert.

Laut dem IEEE Standard Glossar der Softwareentwicklung besteht
\emph{Software} aus den Computerprogrammen, Prozeduren und
gegebenenfalls der Dokumentation und den Daten, die den Betrieb eines
Computersystems betreffen\footnote{IEEEStandardGlossary S. 66}

Die Definition von \emph{Komplexität} wird allgemein als schwierig
erachtet\footnote{Quelle fehlt} \footnote{Jones 2008:335, Jones 2008:627}.
Aber auch hier schlägt das IEEE Standard Glossar eine allgemein
anerkannte Definition vor: „{[}Complexity is{]} the degree to which a
system or component has a design or implementation that is difficult to
understand and verify``\footnote{O. V. o. J., S. 18}. Nach dieser
Definition ist Komplexität ein Maß dafür, wie schwierig eine Software zu
verstehen und zu validieren ist. Fenton und Jones bauen auf dieser
Definition auf und schlagen vier Arten von Komplexität vor: 1. Die
Komplexität des Problems, welches die Software zu lösen versucht, 2. Die
Komplexität der Algorithmen der Software, 3. Die Komplexität der
Struktur der Software und 4. Die kognitive Komplexität der
Software\footnote{(fentonSoftwareMetricsRigorous2003 S. 258, Jones
  2008:449).}. Nach dieser Kategorisierung bestimmen die, in dieser
Arbeit behandelten Metriken die Komplexität der Struktur der Software
(3). Weiter lässt sich zwischen inter- und intra-modularen
Komplexitätsmaßen unterscheiden. Hier werden ausschließlich
intra-modulare Komplexitätsmaßen behandelt. Diese Messen die Komplexität
einzelner Programmteile\footnote{Zuse, H. 1991, S. 7ff}.

Im Kontext der Softwaremetrie werden Metriken als ein Messystem bzw. ein
Verfahren zum Quantifizieren von Eigenschaften von Software
definiert\footnote{dumkeTheorieUndPraxis1994 S. 35ff} \footnote{ebertSoftwareMetrikenPraxisEinfuhrung1996
  S. 4ff} \footnote{augstenWasSindSoftwaremetriken} \footnote{IEEEStandardSoftware
  S. 2f}. Im Deutschen werden sie oft als Synonym für Maß bzw. Maßzahl
genutzt. Im Englischen findet zwischen den Begriffen „metric`` und
„measure`` eine genauere Unterscheidung statt: Eine Metrik sei eine
Funktion, die als Eingabe Daten eines Gegenstandes verwendet und hieraus
eine Zahl zur Quantifizierung dieser Eigenschaft(en) liefert\footnote{IEEEStandardSoftware
  S. 3}. Ein Maß (measure) sei dahingegen die Konkrete Anwendung dieser
Metrik\footnote{IEEEStandardSoftware S. 2}. Aufgrund dieser
Unterscheidung erscheint der Begriff Metrik für die, in dieser Arbeit
behandelten Verfahren als passender.

Zusammengefasst werden Softwarekomplexitätsmetriken in dieser Arbeit als
Verfahren zur Quantifizierung der Vielschichtigkeit der Struktur eines
Computerprogramms definiert.

\section{Kritik}\label{kritik}

Die Praktik der Softwarekomplexitätsmetrie wird allgemein stark
kritisiert.

Zum einen besteht Kritik an dem quantitativen Erfassen subjektiver
Softwareeigenschaften im Generellen. So können Metriken nicht
\emph{direkt} messen, was für Menschen als subjektive Komplexität
wahrgenommen wird, auch wenn das Verwenden von mehreren Maßzahlen hilft,
eine robustere Perspektive zu schaffen\footnote{Rumreich and Kecskemety
  2019:2}.

Weiter werden auch die Komplexitätsmetriken im Speziellen angezweifelt:
Viele Metriken vereinen mehrere, oft konfliktäre Messziele\footnote{fentonSoftwareMetricsRigorous2003
  S. 322}. Dadurch wird ihre Aussagekraft verwässert. Nicht nur an dem
mathematischen Aufbau der Metriken, sondern auch an den
Implementierungen dieser gibt es Kritik. So liefern verschiedene
Implementierungen derselben Metrik oft verschiedene
Ergebnisse\footnote{Rumreich and Kecskemety 2019:2}. Zusätzlich wurden
die klassischen Metriken, wie z.B. die zyklomatische Komplexität für
imperative Programmiertechniken entwickelt. Das lässt sich darauf
zurückführen, dass zu ihrem Entwicklungszeitpunkt die imperative
Programmierung noch am weitesten verbreitet war\footnote{Hoffmann
  2013:277}. Heutzutage sind andere Programmiertechniken, wie z.B. die
objekt-orientierte Programmierung verbreiteter. Diese neuen
Programmiertechniken wurden jedoch in den klassischen Metriken nicht
berücksichtigt\footnote{Hoffmann 2013:277}. Ein Beispiel dieser
Problematik sind die Vererbungsmechanismen in objekt-orientierten
Programmiersprachen. Eine Vererbung erhöht z.B. in der klassischen
Metrik der zyklomatischen Komplexität nicht die Komplexität, was an
dieser Stelle die Korrelation von Code-Größe und der Komplexität des
Codes aushebelt\footnote{Hoffmann 2013:277}.

Ferner ist auch der Zusammenhang zwischen Softwaremetriken und externen
Softwareeigenschaften, wie z.B. der Fehleranfälligkeit
umstritten\footnote{Jones 2008:627}. Hier konnten sowohl Studien
gefunden werden, die für eine Korrelation sprechen, also auch Studien,
die dagegensprechen. In Ebert 1996 konnte z.B. ein Zusammenhang zwischen
der gemessenen Komplexität einer Aufgabe und den dabei entstandenen
Fehlern nachgewiesen werden\footnote{Ebert 1996:65}. Im Gegensatz dazu
konnten in Revilla 2007 keine Zusammenhänge zwischen internen
Softwaremetriken und externen Eigenschaften von Software festgestellt
werden\footnote{Revilla 2007:203,208}.

Als mögliches Fazit aus diesen Studien wird in dieser Arbeit
vorschlagen, dass die Anwendbarkeit der Maßzahlen vom jeweiligen Kontext
des Softwareprojektes und von dem erwarteten Ergebnis abhängt. Dieses
Fazit unterstreicht noch einmal die Relevanz dieser Arbeit. So kann mit
dem Ergebnis dieser Arbeit evaluiert werden, ob und welche
Softwaremaßzahlen im Kontext des DIL Ratingen anwendbar sind.

\section{Eine Auswahl von Softwarekomplexitätsmaßzahlen}\label{eine-auswahl-von-softwarekomplexitatsmasszahlen}

Trotz der umfangreichen Kritik an der Vermessung von Software wurden in
den letzten Jahrzehnten eine Vielzahl von Maßzahlen entwickelt.

In dieser Arbeit werden insgesamt vier Metriken für die Komplexität von
Software betrachtet. Zunächst bietet die Anzahl an logischen Codezeilen
einen ersten Eindruck der Komplexität. Sie ist aber noch eher ein
Umfangsmaß als ein Komplexitätsmaß. Weiter werden die zyklomatische
Komplexität von Thomas McCabe, sowie die Softwaremaßzahlen von Halstead
behandelt. Als letzte Maßzahl wird die Einrückungskomplexität der
Autoren Hindle, Godfrey, und Holt betrachtet.

Die Auswahl der ersten drei Komplexitätsmaßen beruft sich auf ihre
generelle Popularität in der Softwaremetrie. Im Rahmen einer
umfangreichen Literaturrecherche konnten für diese Metriken die meisten
Referenzen gefunden werden. Die Maßzahl der logischen Codezeilen wird
von Zuse91 als ein wichtiges Maß zum Bestimmen des Umfangs und der
Komplexität einer Softwareanwendung eingestuft\footnote{Zuse 91 S. 145}.
Diese Position wird auch von satoExperiencesTrackingAgile2006 und
aleneziEmpiricalAnalysisComplexity2015 bekräftigt. Die zyklomatische
Komplexität von McCabe wird ebenfalls von Zuse als eine der bekanntesten
Maßzahlen eingestuft \footnote{Zuse 91 S. 145}. Weiter bekräftigt
fentonSoftwareMetricsRigorous2003, dass diese Maßzahl zum Messen der
Softwarekomplexität geeignet sei\footnote{fentonSoftwareMetricsRigorous2003
  S. 31}. Auch die vier Autoren Revilla\footnote{Revilla 2007:203},
Jones\footnote{Jones 2008:S. 335, 627, 449}, Sato\footnote{satoExperiencesTrackingAgile2006}
und Alenezi\footnote{aleneziEmpiricalAnalysisComplexity2015} verweisen
auf die zyklomatische Komplexität als Maßzahl für die Komplexität einer
Applikation. Jones sagt dabei zusätzlich aus, dass diese
Komplexitätsmaßzahl ein besonders breites Anwendungsspektrum bedienen
könne\footnote{Jones 2008:S. 335, 627, 449}. Auch die Softwaremaßzahlen
von Halstead werden in der Literatur häufig referenziert. Zuse und
Revilla bekräftigen die Verbreitung dieser Maßzahlen\footnote{Zuse 91 S.
  145, Revilla 2007:203}. Zusätzlich sagt Fenton aus, dass die Maßzahlen
geeignet zum Messen der Softwarekomplexität seien\footnote{fentonSoftwareMetricsRigorous2003
  S. 31}.

Zusätzlich wird die Einrückungskomplexität betrachtet, da sie einen, im
Gegensatz zu den anderen Maßzahlen methodisch sehr abweichenden Ansatz
verfolgt\footnote{Quelle fehlt}. So betrachtet sie im Gegensatz zu den
anderen Maßzahlen nicht die Aufteilung auf Codezeilen oder den Inhalt
des Codes, sondern die Einrückung der einzelnen Codezeilen.

Neben diesen vier Maßzahlen bestehen auch noch eine Vielzahl an weiteren
Messungsmethoden, die in dieser Arbeit aber keine weitere Betrachtung
finden sollen. Zum Beispiel wurden die gewichtete Anzahl an Methoden pro
Klasse (WMC)\footnote{satoExperiencesTrackingAgile2006}, die Anzahl an
Kommentaren\footnote{Revilla 2007:203}, die Größe der Klasse\footnote{satoExperiencesTrackingAgile2006}
und die Anzahl an Testcodezeilen\footnote{satoExperiencesTrackingAgile2006}
nicht weiter betrachtet.

Zwischen drei der ausgewählten Maßzahlen wurden bereits Korrelationen
nachgewiesen. Zunächst besteht eine starke Korrelation zwischen der
Anzahl an logischen Codezeilen und dem Aufwand nach Halstead\footnote{Jones
  2008: 627}. Eine noch stärkere, linear-stabile Korrelation lässt sich
zwischen der Anzahl logischer Code Zeilen und der zyklomatischen
Komplexitätsmaßzahl nachweisen. Diese Korrelation konnte über
Programmierer*innen, Programme, Sprachen und Programmierparadigma hinweg
bewiesen werden\footnote{Jones 2008:627, Jay et al 2009:137}.

Die vier Komplexitätsmetriken werden nun der Reihe nach genauer
erläutert.

\subsection{Logische Codezeilen}\label{logische-codezeilen}

Als erste hier vorgestellte Metrik weist die Anzahl der logischen
Codezeilen (SLOC) den geringsten Berechnungsaufwand auf. Sie wird
teilweise auch als LOC bzw. Lines of Code bezeichnet\footnote{Hoffmann
  2013:263}. Dabei werden die Zeilen an Sourcecode in der Software
berechnet\footnote{Rumreich and Kecskemety 2019:2}. Die Metrik lässt
sich somit ohne die Unterstützung komplexer Algorithmen berechnen und
lässt sich auf nahezu alle Programmiersprachen anwenden\footnote{Hoffmann
  2013:263}. Eine Ausnahme bilden hier grafische
Programmiersprachen\footnote{Quelle fehlt}.

In ihrer Aussagekraft wird die Anzahl von Codezeilen allgemein als
begrenzt eingestuft\footnote{Hoffmann 2013:264, Rumreich and Kecskemety
  2019:2}. Unter anderem der Programmierstil und die Programmiersprache
haben einen Einfluss auf die Anzahl der Codezeilen (Hoffmann 2013:264,
Hoffmann 2013:264). Gleichzeitig erfüllt sie jedoch die Assoziativität,
Kommutativität und Monotonität\footnote{zuseSoftwareComplexityMeasures1991
  S. 142} als wesentliche Qualitätskriterien für Maßzahlen.

Eine Weiterentwicklung der LOC-Metrik ist u.a. die NCSS-Metrik (Non
Commented Source Statements). Sie misst genauso wie die LOC-Metrik die
Anzahl an Codezeilen, ignoriert dabei aber alle Kommentarzeilen
(Hoffmann 2013:264). Zusätzlich bestehen auch empirisch ermittelte
Sprachfaktoren, die es ermöglichen, die Ergebnisse von LOC und NCSS
Metriken vor ihrer Weiterverarbeitung zu gewichten (Hoffmann 2013:264).

\subsection{Zyklomatische Komplexität}\label{Zyklomatische-Komplexitat}

Die zyklomatische Komplexität wurde von Tom McCabe entwickelt und zuerst
in seinem Aufsatz „A Complexity Metric``\footnote{(McCabe 76)}
veröffentlicht\footnote{(Sneed et al 2010:185).}. Die zyklomatische
Komplexität gehört zu den Kontrollflussmetriken\footnote{(Ebert
  1996:88).}. In der Praxis wird sie von vielen Firmen
entwicklungsbegleitend kontinuierlich erhoben, um vor problematischen
Programmkomponenten frühzeitig zu warnen. Dabei fließt sie oft auch in
die Abnahmekriterien für Software ein\footnote{(Hoffmann 2013:275).}.

Die zyklomatische Komplexität ist definiert als eine Maßzahl für die
Kontrollflusskomplexität eines Programmes\footnote{(Rumreich and
  Kecskemety 2019:2, Jones 2008:335).}. Sie misst die Anzahl von linear
unabhängigen Pfaden auf dem Kontrollflussgraphen eines Programmes und
liefert so ein normalisiertes Komplexitätsmaß\footnote{(Rumreich and
  Kecskemety 2019:2 (Quelle 15 dort)).}.

Zur Berechnung der zyklomatischen Komplexität wird der Programmcode
zunächst in einen Graphen aus Knoten und Kanten umgewandelt\footnote{(Sneed
  et al 2010:185, Jones 2008:335).}. Dazu wird der Quelltext in einen
Syntaxbaum umgewandelt. Diese Baumstruktur spiegelt die Struktur des
Codes wider. Aus dem Syntaxbaum wird ein Kontrollflussgraph der
Anwendung konstruiert. Der Kontrollflussgraph besteht aus den
Anweisungen und Verzweigungen des Programms und stellt alle möglichen
Ablaufwege durch das Programm dar\footnote{rackeKontrollflussdiagramme2017}.
Der Umwandlungsprozess wird in Abbildung .. beschrieben

GRAFIK

Aus den Eigenschaften dieses Kontrollflussgraphen kann nun die
zyklomatische Komplexität mit Mitteln der Graphentheorie\footnote{(Hoffmann
  2013:273).} berechnet werden. Als relevante Größen kommen dabei die
Anzahl der Kanten des Graphen (E), die Anzahl der Knoten (N), sowie die
Anzahl unabhängiger Teilgraphen (p) zum Einsatz. Die Anzahl unabhängiger
Teilgraphen entspricht der Anzahl unabhängiger Module bzw. Funktionen in
dem Programm. Wird ein Programm mit mehreren Modulen untersucht, werden
die Kanten und Knoten der Kontrollflussgraphen der Module
aufsummiert\footnote{(Hoffmann 2013:275f)}.

Die Formel für die zyklomatische Komplexität lautet wie folgt:

V(Z) = \textbar E\textbar-\textbar N\textbar{} + 2p

Die zyklomatische Komplexität (V(Z)) entspricht also der Anzahl an
Kanten (E) minus der Anzahl an Knoten (N) plus der doppelten Anzahl der
zusammenhängenden Einzelgraphen (p) (Hoffmann 2013:273, Jones 2008:335,
Hoffmann 2013:273, Ebert 1996:88, Sneed et al 2010:185). Also ist sie
stets um eins größer als die Anzahl an Verzweigungen in einem Programm
(Hoffmann 2013:274).

In der konkreten Anwendung wird die zyklomatische Komplexität oft als
Indikator für die Wartbarkeit des Code und für die Produktivität in der
Weiterentwicklung des Codes verwendet\footnote{(Jones 2008:336)}.
Zusätzlich liefert sie die obere Grenze der Anzahl an Tests, die für
eine vollständige Testabdeckung des Quelltextes benötigt werden.

Die zyklomatische Komplexität wurde in der Fachliteratur vielfach
zitiert und vor allem auch vielfach kritisiert.

Zunächst wird allgemein kritisiert, dass die Maßzahl nur die Komplexität
der Ablauflogik des Codes und nicht die Komplexität des gesamten Codes
misst\footnote{(Sneed et al 2010:186).}. Interne Eigenschaften der
Knoten\footnote{(Hoffmann 2013:273).} \footnote{(Rumreich and Kecskemety
  2019:2)}, die Komplexität im Datenfluss\footnote{(Rumreich and
  Kecskemety 2019:2)} sowie die Verschachtelung von Modulen werden
ignoriert\footnote{(Zuse S. 89ff).}. Also lässt sich sagen, dass die
Maßzahl für prozeduralen Code eine gewisse Bedeutung hat, in
objektorientiertem Code jedoch nur die Komplexität der einzelnen
Methodenkörper misst\footnote{(Sneed et al 2010:186).}.

Insgesamt wird geschlussfolgert, dass die zyklomatische Komplexität nur
mit Vorsicht anzuwenden sei und in Relation zu anderen Maßzahlen gesetzt
werden müsse\footnote{(fentonSoftwareMetricsRigorous2003 S. 52).}. Genau
aus diesem Grund wird sie in dieser Arbeit in den Kontext der agilen
Aufwandsabschätzungen und in Relation zu anderen Maßzahlen gesetzt.

\subsection{Halstead Metriken}\label{Halstead-Metriken}

Als dritte Komplexitätsmetrik werden die Halstead Metriken herangezogen.
Bei diesen Metriken wendet der Autor Maurice Howard Halstead\footnote{halsteadElementsSoftwareScience1979}
Elemente der Kommunikationstheorie auf die Programmierung an\footnote{(Sneed
  et al 2010:183)}.

Die Berechnung der Halstead Metriken wird im folgenden an einem Beispiel
erklärt\footnote{(Sneed et al 2010:184)}. Das Beispiel behandelt
folgenden Code:

main() (Programmiersprache C)

\{

int countOne, countTwo, countThree, average;

scanf(``\%d \%d \%d'', \&countOne, \&countTwo, \&countThree);

average = (countOne + countTwo + countThree) / 3;

printf(``average = \%d'', average);

\}

Für die Berechnung seiner Metriken zählt Halstead zunächst die
Grundelemente der Programmiersprache, nämlich die Operatoren und
Operanden\footnote{(Sneed et al 2010:183), (Rumreich and Kecskemety
  2019:2)}. Operatoren führen Operationen auf Operanden durch. Beispiele
für Operatoren sind arithmetische Operatoren (z.B. Addition)
Vergleichsoperatoren und Zuweisungsoperatoren. Operanden sind in der
Regel Datenelemente wie Zahlen oder Text.

In diesem Beispiel sind die unterschiedlichen Operatoren: main, (),
\{\}, int, scanf, \&, =, +, /, printf, ,, ;

Die unterschiedlichen Operanden sind: countOne, countTwo, countThree,
average, "\%d \%d \%d", 3, "avg = \%d"

Zu den Operatoren und Operanden werden folgende Zahlen erfasst:

\begin{itemize}
\item
  n1: Die Anzahl an unterschiedlichen Operatoren (12)
\item
  n2: Die Anzahl an unterschiedlichen Operanden (7)
\item
  N1: Die Anzahl insgesamt vorkommender Operatoren (27)
\item
  N2: Die Anzahl insgesamt vorkommender Operanden (15)
\end{itemize}

Aus diesen Zahlen können nun die Halstead Metriken berechnet werden:

Der Wortschatz eines Programms ergibt sich aus der Summe der
unterschiedlichen Operatoren und Operanden:

\emph{Wortschatz (n) = Operatoren (n1) + Operanden (n2) 19 = 12 + 7}

Die Länge des Programmes ergibt sich aus der Summe der insgesamt
vorkommenden Operanden und Operatoren :

\emph{Länge (N) = Operatorenvorkommnisse (N1) + Operandenvorkommnisse
(N2) 42 = 27 + 15}

Weiter Berechnet Halstead die Größe des Programmcodes durch einen
Logarithmus:

\emph{Größe (Volume / V) = Länge (N) * log2 (Wortschatz (n) ) 53,71 = 42
* log(19)}

Für die Sprachkomplexität setzt er jeweils die Operatoren ins Verhältnis
zu den Operatorenvorkommnissen und die Operanden ins Verhältnis zu den
Operandenvorkommnissen. Beide Werte werden miteinander multipliziert:

\emph{Komplexität = (Operatoren / Operatorenvorkommnisse) * (Operanden /
Operandenvorkommnisse) 0,207 =ca (12 / 27) * (7 / 15)}

Zuletzt berechnet er den Aufwand als Quotient aus Programmgröße und
Komplexität. Je komplexer oder größer ein Programm ist, desto länger
dauere, es dieses Programm zu schreiben:

\emph{Aufwand (E) = Größe / Komplexität 259,47 = 53,71 / 0,207}

Die von Halstead vorgeschlagenen Metriken wurden vielfach kritisiert. Er
hat die Berechnungen nie empirisch untermauert und in späteren
empirischen Studien wurden sie sogar widerlegt\footnote{(Sneed et al
  2010:185)}. Insbesondere sei zu kritisieren, dass die Codegröße,
Komplexität und der Aufwand nach Halstead nicht das Kriterium der
Monotonität erfüllen. Sie können also nicht auf einer Verhältnisskala
verwendetet werden und nur die Berechnung von Medianwerten sei
sinnvoll\footnote{Zuse, H. 1991, S. 142}. Eine zusätzliche Einschränkung
sei, dass die Halstead Metriken die Berechnungskomplexität in den
Vordergrund rücken und keine Aussage über den Kontrollfluss der
Anwendung treffen\footnote{(Rumreich and Kecskemety 2019:2, Quelle 16
  dort}. Insgesamt seien sie also kritisch anzusehen\footnote{(Sneed et
  al 2010:185)}.


\subsection{Einrückungskomplexität}\label{Einruckungskomplexitat}

Als letztes Komplexitätsmaß wird hier die Einrückungskomplexität
vorgestellt. Sie wurde von Abram Hindle, Michael W. Godfrey und Richard
C. Holt an der University of Waterloo entwickelt und zuerst mit ihrer
Arbeit „Reading Beside the Lines: Indentation as a Proxy for Complexity
Metrics `` auf der sechzehnten internationalen IEEE Konferenz zum
Verständnis von Computerprogrammen vorgestellt\footnote{hindleReadingLinesIndentation2008
  S. 1}. Die Komplexitätsmaßzahl ist im Vergleich zu den anderen hier
behandelten Metriken verhältnismäßig neu, wurde aber bereits in Arbeiten
von Lalouche et. al\footnote{gilWhenSoftwareComplexity2016 S. 10}
\footnote{gilCorrelationSizeMetric2017 S. 7}und Landman et. al
\footnote{landmanEmpiricalAnalysisRelationship2016 S. 6} referenziert.

Die Einrückungskomplexität soll nun erklärt werden. In den meisten
Programmiersprachen werden bestimmte Zeilen zur besseren Lesbarkeit des
Codes eingerückt. Die Einrückungskomplexität macht sich genau diesen
Umstand zu Nutzen und verwendet die Einrückungen der Codezeilen als
Indikator für Komplexität\footnote{hindleReadingLinesUsing2009 S. 1}.

In prozeduralen Code, wie z.B. in C, kann die Einrückung von Codezeilen
Kontrollstrukturen, wie Verzweigungen und Schleifen anzeigen. In
objektorientierten Sprachen, wie C++, Java und JavaScript kann die
Einrückung eine Verkapselung in Form von Klassen, Subklassen oder
Methoden anzeigen. In funktionalen Sprachen wie OCaml und Lisp zeigt die
Einrückung einen neuen Handlungskontext, neue Funktionen oder einen
neuen Ausdruck\footnote{}. In jedem dieser drei Typen von
Programmiersprachen sind in der Regel Stellen im Code eingerückt, die
die Komplexität des Codes erhöhen. Also lasse sich schlussfolgern, dass
die Menge der Einrückungen ein Maß der Komplexität eines Programmes sei.

Zur Verifizierung dieser These setzten die Autoren die
Einrückungskomplexität in ein Verhältnis zu den älteren Metriken wie der
zyklomatischen Komplexität und den Halstead Metriken. Bei der
zyklomatischen Komplexität erhöhen z.B. Verzweigungen die Komplexität.
Diese werden in den meisten Programmiersprachen durch eine Einrückung
der Codezeilen gekennzeichnet. Also stünde die Vermutung einer
Korrelation der Metriken nahe\footnote{hindleReadingLinesUsing2009 S. 2}.
Diese Korrelation konnte von den Autoren in einer Studie von 278
populären Open Source Projekten erfolgreich nachgewiesen
werden\footnote{Quelle}. Damit sei die Validität der
Einrückungskomplexität als Komplexitätsmaß von Software nachgewiesen.

Für die Berechnung der Einrückungskomplexität werden die Einrückungen
der einzelnen Codezeilen im Quelltext betrachtet. Die Anzahl an
physischen Einrückungen in einer Zeile bezeichnet zunächst die Anzahl an
Leerzeichen bzw. Tabs, welche sich am Anfang einer Zeile befinden. Diese
werden für jede Zeile berechnet. Anhand dieser Angaben wird auf der
Ebene einer Datei berechnet, wie viele Leerzeichen bzw. Tabs einer
logischen Einrückungsebene entsprechen\footnote{hindleReadingLinesUsing2009
  S. 5}. In den meisten Projekten entsprechen vier Leerzeichen bzw. ein
Tab einem Einrückungslevel\footnote{hindleReadingLinesUsing2009 S. 9}.
Dieser Zusammenhang wird von den Autoren als das Einrückungsmodell einer
Datei bezeichnet. Mit dem Einrückungsmodell kann nun die Anzahl an
logischen Einrückungen für jede Datei berechnet werden. Die Anzahl
dieser Einrückungen gibt für jede Zeile die Einrückungsebene an. Die
Summe der logischen Einrückungen wird als die Einrückungskomplexität
bezeichnet.

Ein wesentlicher Vorteil der Einrückungskomplexität ist ihre
Sprachenunabhängigkeit. Im Gegensatz zu der zyklomatischen Komplexität
und den Halstead Metriken setzt sie kein Verständnis der Grammatik einer
Programmiersprache voraus\footnote{hindleReadingLinesUsing2009 S. 1}
\footnote{hindleReadingLinesUsing2009 S. 2}.

Als zusätzlichen Vorteil führen die Autoren auf, dass die
Einrückungskomplexität weniger Berechnungsschritte benötige als andere
Komplexitätsmaßen. Demnach sei sie günstiger in der Durchführung als
manche andere Metriken\footnote{hindleReadingLinesUsing2009 S. 20f}.
Vergleichsoperatoren und Zuweisungsoperatoren. Operanden sind in der
Regel Datenelemente wie Zahlen oder Text.

In diesem Beispiel sind die unterschiedlichen Operatoren: main, (),
\{\}, int, scanf, \&, =, +, /, printf, ,, ;

Die unterschiedlichen Operanden sind: countOne, countTwo, countThree,
average, "\%d \%d \%d", 3, "avg = \%d"

Zu den Operatoren und Operanden werden folgende Zahlen erfasst:

\begin{itemize}
\item
  n1: Die Anzahl an unterschiedlichen Operatoren (12)
\item
  n2: Die Anzahl an unterschiedlichen Operanden (7)
\item
  N1: Die Anzahl insgesamt vorkommender Operatoren (27)
\item
  N2: Die Anzahl insgesamt vorkommender Operanden (15)
\end{itemize}

Aus diesen Zahlen können nun die Halstead Metriken berechnet werden:

Der Wortschatz eines Programms ergibt sich aus der Summe der
unterschiedlichen Operatoren und Operanden:

\emph{Wortschatz (n) = Operatoren (n1) + Operanden (n2) 19 = 12 + 7}

Die Länge des Programmes ergibt sich aus der Summe der insgesamt
vorkommenden Operanden und Operatoren :

\emph{Länge (N) = Operatorenvorkommnisse (N1) + Operandenvorkommnisse
(N2) 42 = 27 + 15}

Weiter Berechnet Halstead die Größe des Programmcodes durch einen
Logarithmus:

\emph{Größe (Volume / V) = Länge (N) * log2 (Wortschatz (n) ) 53,71 = 42
* log(19)}

Für die Sprachkomplexität setzt er jeweils die Operatoren ins Verhältnis
zu den Operatorenvorkommnissen und die Operanden ins Verhältnis zu den
Operandenvorkommnissen. Beide Werte werden miteinander multipliziert:

\emph{Komplexität = (Operatoren / Operatorenvorkommnisse) * (Operanden /
Operandenvorkommnisse) 0,207 =ca (12 / 27) * (7 / 15)}

Zuletzt berechnet er den Aufwand als Quotient aus Programmgröße und
Komplexität. Je komplexer oder größer ein Programm ist, desto länger
dauere, es dieses Programm zu schreiben:

\emph{Aufwand (E) = Größe / Komplexität 259,47 = 53,71 / 0,207}

Die von Halstead vorgeschlagenen Metriken wurden vielfach kritisiert. Er
hat die Berechnungen nie empirisch untermauert und in späteren
empirischen Studien wurden sie sogar widerlegt\footnote{(Sneed et al
  2010:185)}. Insbesondere sei zu kritisieren, dass die Codegröße,
Komplexität und der Aufwand nach Halstead nicht das Kriterium der
Monotonität erfüllen. Sie können also nicht auf einer Verhältnisskala
verwendetet werden und nur die Berechnung von Medianwerten sei
sinnvoll\footnote{Zuse, H. 1991, S. 142}. Eine zusätzliche Einschränkung
sei, dass die Halstead Metriken die Berechnungskomplexität in den
Vordergrund rücken und keine Aussage über den Kontrollfluss der
Anwendung treffen\footnote{(Rumreich and Kecskemety 2019:2, Quelle 16
  dort}. Insgesamt seien sie also kritisch anzusehen\footnote{(Sneed et
  al 2010:185)}.


\subsection{Einrückungskomplexität}\label{Einruckungskomplexitat}

Als letztes Komplexitätsmaß wird hier die Einrückungskomplexität
vorgestellt. Sie wurde von Abram Hindle, Michael W. Godfrey und Richard
C. Holt an der University of Waterloo entwickelt und zuerst mit ihrer
Arbeit „Reading Beside the Lines: Indentation as a Proxy for Complexity
Metrics `` auf der sechzehnten internationalen IEEE Konferenz zum
Verständnis von Computerprogrammen vorgestellt\footnote{hindleReadingLinesIndentation2008
  S. 1}. Die Komplexitätsmaßzahl ist im Vergleich zu den anderen hier
behandelten Metriken verhältnismäßig neu, wurde aber bereits in Arbeiten
von Lalouche et. al\footnote{gilWhenSoftwareComplexity2016 S. 10}
\footnote{gilCorrelationSizeMetric2017 S. 7}und Landman et. al
\footnote{landmanEmpiricalAnalysisRelationship2016 S. 6} referenziert.

Die Einrückungskomplexität soll nun erklärt werden. In den meisten
Programmiersprachen werden bestimmte Zeilen zur besseren Lesbarkeit des
Codes eingerückt. Die Einrückungskomplexität macht sich genau diesen
Umstand zu Nutzen und verwendet die Einrückungen der Codezeilen als
Indikator für Komplexität\footnote{hindleReadingLinesUsing2009 S. 1}.

In prozeduralen Code, wie z.B. in C, kann die Einrückung von Codezeilen
Kontrollstrukturen, wie Verzweigungen und Schleifen anzeigen. In
objektorientierten Sprachen, wie C++, Java und JavaScript kann die
Einrückung eine Verkapselung in Form von Klassen, Subklassen oder
Methoden anzeigen. In funktionalen Sprachen wie OCaml und Lisp zeigt die
Einrückung einen neuen Handlungskontext, neue Funktionen oder einen
neuen Ausdruck\footnote{}. In jedem dieser drei Typen von
Programmiersprachen sind in der Regel Stellen im Code eingerückt, die
die Komplexität des Codes erhöhen. Also lasse sich schlussfolgern, dass
die Menge der Einrückungen ein Maß der Komplexität eines Programmes sei.

Zur Verifizierung dieser These setzten die Autoren die
Einrückungskomplexität in ein Verhältnis zu den älteren Metriken wie der
zyklomatischen Komplexität und den Halstead Metriken. Bei der
zyklomatischen Komplexität erhöhen z.B. Verzweigungen die Komplexität.
Diese werden in den meisten Programmiersprachen durch eine Einrückung
der Codezeilen gekennzeichnet. Also stünde die Vermutung einer
Korrelation der Metriken nahe\footnote{hindleReadingLinesUsing2009 S. 2}.
Diese Korrelation konnte von den Autoren in einer Studie von 278
populären Open Source Projekten erfolgreich nachgewiesen
werden\footnote{Quelle}. Damit sei die Validität der
Einrückungskomplexität als Komplexitätsmaß von Software nachgewiesen.

Für die Berechnung der Einrückungskomplexität werden die Einrückungen
der einzelnen Codezeilen im Quelltext betrachtet. Die Anzahl an
physischen Einrückungen in einer Zeile bezeichnet zunächst die Anzahl an
Leerzeichen bzw. Tabs, welche sich am Anfang einer Zeile befinden. Diese
werden für jede Zeile berechnet. Anhand dieser Angaben wird auf der
Ebene einer Datei berechnet, wie viele Leerzeichen bzw. Tabs einer
logischen Einrückungsebene entsprechen\footnote{hindleReadingLinesUsing2009
  S. 5}. In den meisten Projekten entsprechen vier Leerzeichen bzw. ein
Tab einem Einrückungslevel\footnote{hindleReadingLinesUsing2009 S. 9}.
Dieser Zusammenhang wird von den Autoren als das Einrückungsmodell einer
Datei bezeichnet. Mit dem Einrückungsmodell kann nun die Anzahl an
logischen Einrückungen für jede Datei berechnet werden. Die Anzahl
dieser Einrückungen gibt für jede Zeile die Einrückungsebene an. Die
Summe der logischen Einrückungen wird als die Einrückungskomplexität
bezeichnet.

Ein wesentlicher Vorteil der Einrückungskomplexität ist ihre
Sprachenunabhängigkeit. Im Gegensatz zu der zyklomatischen Komplexität
und den Halstead Metriken setzt sie kein Verständnis der Grammatik einer
Programmiersprache voraus\footnote{hindleReadingLinesUsing2009 S. 1}
\footnote{hindleReadingLinesUsing2009 S. 2}.

Als zusätzlichen Vorteil führen die Autoren auf, dass die
Einrückungskomplexität weniger Berechnungsschritte benötige als andere
Komplexitätsmaßen. Demnach sei sie günstiger in der Durchführung als
manche andere Metriken\footnote{hindleReadingLinesUsing2009 S. 20f}.